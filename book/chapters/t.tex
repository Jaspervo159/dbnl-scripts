\chapter[14 auteurs, 1617 haiku's]{veertien auteurs, zestienhonderdzeventien haiku's}

\section{Herman Teirlinck}

\subsection{Uit: Het gevecht met de engel}

\haiku{Het woud galmt weldra.}{van de bonk der aksten en}{de schreeuw van het hout}\\

\haiku{Achteraan rijst het.}{somber beukenmassief van}{de Berg ter Oigne}\\

\haiku{... weet hij - Mijnheer Gomeer.}{heft rustig zijn rechterhand}{op en hij glimlacht}\\

\haiku{Hij interesseert.}{zich voornamelijk voor de}{twee oudste jongens}\\

\haiku{Het gaat met deze.}{delikate teelt niet vlot}{in den beginne}\\

\haiku{Hij kan hem nog eens,.}{duchtig dooreenschudden met}{zijn oude knuisten}\\

\haiku{In 1856 is Terve.}{mede-eigenaar van zes}{en dertig serren}\\

\haiku{Maar het is goed ook,.}{want zij doen het woud zwellen}{en drijven het sap}\\

\haiku{{\textquoteright} Er dreigt iets los te,.}{barsten de tijd slechts van een}{dubbele ademstoot}\\

\haiku{Hij zal dan even uit.}{de weg van de Burcht en langs}{het dorp heen omgaan}\\

\haiku{En ook zijn gelaat,,.}{dat blozend is ligt nog in}{donzige lijnen}\\

\haiku{Het rijtuig baant zich.}{een moeizame weg en staat}{v\'o\'or het hoofdportaal}\\

\haiku{Wat er ook in hun,.}{geest kan omgaan zij reppen}{er over met geen woord}\\

\haiku{Hij komt buigen en.}{vouwt profijtig zijn gele}{handen over de borst}\\

\haiku{Ik ben Gomerus de,.}{derde gehoorzame stem}{Maria toegewijd}\\

\haiku{Echt voortrekkersbloed.}{nochtans stoort zich slechts matig}{aan zulke grillen}\\

\haiku{Zijn lip is door een,.}{fijn snorretje beschaduwd}{de moeite niet waard}\\

\haiku{{\textquoteright} Het mag niet te ver,,.}{gaan meent Iffratje en hij}{blaast zijn rook scheef uit}\\

\haiku{Nu pas zag ik hoe.}{vernuftig Mak het voor de}{jacht had ingericht}\\

\haiku{{\textquoteright} Iffratje sliert met:}{zijn beide handen over zijn}{gelaat en kreunt stil}\\

\haiku{En Achiel, wanneer men,.}{het bloed niet opjaagt wordt zo}{tam als een poedel}\\

\haiku{De winkel is vol,.}{licht want het is een danig}{propere winkel}\\

\haiku{Het wordt door Balten,.}{die anderhalf jaar ouder}{is dan Bruin geleid}\\

\haiku{Iffratje staat op.}{de drempel van de sakristij}{naast Toontje Rozier}\\

\haiku{En dat daarbij de?}{ganse bevolking in het}{gedrang kan komen}\\

\haiku{Maar achteraf is.}{hij er de mannelijkheid}{van gaan ervaren}\\

\haiku{Klaus heeft dan het enig,:}{middel aangewend het enig}{dat mogelijk was}\\

\haiku{Zijn bruin gelaat, in,.}{zijn beheerste vlakheid is nooit}{zo krachtig geweest}\\

\haiku{Maar nog zachter is,.}{zijn vinger die traag de ogen}{van Vrouw Odile sluit}\\

\haiku{Alleen zijn houding,,.}{blijft tegenover zijn vader}{nors en vijandig}\\

\haiku{Hij kan onder geen.}{omstandigheid dergelijk}{vooruitzicht dulden}\\

\haiku{Brozen hoort niets dan.}{het geruis van zijn voeten}{door het droge loof}\\

\haiku{Dit is van al de,.}{bossen op Zoni\"en de}{wildste de dichtste}\\

\haiku{En daarboven spreidt.}{zich de blauwigheid van}{de winterhemel}\\

\haiku{Alle drie zijn zij.}{bezeten door een hete}{vreugd zonder geluid}\\

\haiku{Twee dennekegels,.}{zijn haar ogen een knoestige}{wortelstomp haar neus}\\

\haiku{Bruin verstaat wel dat.}{nu het beste is dat zij}{gauw thuis geraken}\\

\haiku{Die God laat zich zien,,,,.}{en horen en ruiken en}{smaken en tasten}\\

\haiku{Het loof dat goed is,.}{om te eten en het loof dat}{de koortsen aansteekt}\\

\haiku{Want het beste deel.}{van Gods wijde wereld heeft}{een hemelse smaak}\\

\haiku{Dat is zo altijd,.}{geweest van in de tijd der}{holen te Rhode}\\

\haiku{Veerle moet zo lang.}{op de Burcht blijven tot ze}{geheel hersteld is}\\

\haiku{Maar de grol stijgt reeds.}{in de baard van Klaus en hij}{noemt haar een schijtkont}\\

\haiku{En wat er met de,:}{jongens omgaat weet hij zo}{goed als een ieder}\\

\haiku{'t is de hitte.}{van het leven die in hun}{bloed is geslagen}\\

\haiku{hoe zouden zij hem,?}{niet kennen die weerhelmt in}{hun bevende vlees}\\

\haiku{Hij ziet haar benen,.}{die van een kille blankheid}{zijn en hij haat haar}\\

\haiku{Want dat mag hij wel,.}{weten hij stinkt naar lafheid}{en huichelarij}\\

\haiku{Daarop verlaat hij,. '}{de huiskamer te wege}{naar de stallingen}\\

\haiku{Tegelijkertijd.}{nadert aan zijn rechter een}{paard te viervoete}\\

\haiku{Zij moeten aan de,.}{overkant van de diepte de}{andere flank op}\\

\haiku{Over haar glanzende.}{huid hangt de late dag een}{rozige klaarte}\\

\haiku{Zij neemt de goede,.}{afstand veert uit en beukt met}{haar kop in zijn maag}\\

\haiku{Hij zou nu moeten,.}{bidden want deze avond is}{uitermate goed}\\

\haiku{Het is nabij het,.}{krieken van de morgen dat}{de aanval geschiedt}\\

\haiku{Uitroeiing is voor.}{het minste verraad nog een}{veel te milde straf}\\

\haiku{De Dulle schudt van,;}{neen en Emke heeft schoon naar}{haar ogen te zoeken}\\

\haiku{Zij hopen dat hun.}{moeder de richting goed heeft}{gade geslagen}\\

\haiku{Wat ook echter de,.}{bevelen mogen zijn hij}{zal ze uitvoeren}\\

\haiku{Zonder naar Balten,.}{om te zien wacht hij op wat}{Balten zal zeggen}\\

\haiku{Nee, 't was een van,.}{die rabauwen waardoor het}{gebit zo sleeuw wordt}\\

\haiku{{\textquoteright} Lieven heeft uit zijn.}{armen de benen van Bruin}{voelen wegglijden}\\

\haiku{Het gelaat van zijn.}{moeder is daar plots in de}{klaarte gerezen}\\

\haiku{Emke staat aan de.}{rand van het open graf waar Bruin}{is omgekanteld}\\

\haiku{Isa\"u kwam als eerste,.}{zoon ter wereld gevolgd door}{zijn broeder Jakob}\\

\haiku{Er was een zolder,.}{onder het rieten dak van}{het huis in het woud}\\

\haiku{Ik ben nooit bang juist.}{omdat ik die angst steeds heb}{moeten bevechten}\\

\haiku{En ik heb altijd.}{gevoeld dat mijn ras  uit}{het woud is ontstaan}\\

\haiku{Zo niet hadt ge niet.}{zo zorgvuldig vermeden}{te spreken over haar}\\

\haiku{Ik zag niets anders,,.}{onder mij dan het groene}{bodemloze meer}\\

\haiku{Het werk waarmede,.}{de Burcht mij had bedacht was}{als geknipt voor mij}\\

\haiku{Ik zag een grijze,.}{haarbos fraai opgekamd naar}{een voorbije mode}\\

\haiku{{\textquoteleft}Mijnheer Brozen zal,.}{ons denkelijk een handje}{toereiken Abdon}\\

\haiku{Aanvankelijk was.}{ik met deze kentering}{aardig in mijn schik}\\

\haiku{De vriendelijkheid.}{van zijn open glimlach kwam mij}{plots verontrusten}\\

\haiku{En door de ruimte.}{wiegelde hier en daar een}{blaadje nederwaarts}\\

\haiku{En ten laatste zag.}{men de webslierten aan de}{dakpannen hangen}\\

\haiku{Ik wist toen nog niet,.}{wat het was en dat ze aan}{het paren waren}\\

\haiku{Het trof mij dat er,.}{geen tapijtje lag niet het}{kleinste karpetje}\\

\haiku{En zij ging een breed,.}{gordijn openschuiven dat zo}{wit was als de muur}\\

\haiku{Vader had u uit.}{de Burcht in zijn armen naar}{uw kamer gebracht}\\

\haiku{Ik had mijn houvast,.}{mijn centrale positie}{weggeredeneerd}\\

\haiku{Mijn droom naar God was,.}{een steriele zucht uit mijn}{onmacht geboren}\\

\haiku{{\textquoteright} Nu ja, ik had de,}{vraag gesteld ik moest ze wel}{staande houden al}\\

\haiku{{\textquoteleft}Kijk, Seigneur Abdon!}{neemt een zonnebad op het}{torenterrasje}\\

\haiku{Daar stond de toren, ...}{met het laag portaaltje en}{de ovale schaduw}\\

\haiku{Ik was ziek, dat moest,.}{Roedi vooral weten ziek}{en ongelukkig}\\

\haiku{En de koffie, die,.}{sterk was en heet geurde door}{mijn ganse wezen}\\

\haiku{Maar hij zou een zo.}{gruwelijke misgreep op}{het leven wreken}\\

\haiku{Ik geraakte door.}{Roedi's optreden niet het}{minst uit mijn humeur}\\

\haiku{Toen wendde hij zich:}{naar de openstaande deur van}{het salon en riep}\\

\haiku{{\textquoteleft}Laat mij niet alleen,,, '!}{laat mij niet achter Brozen}{ins hemels naam}\\

\haiku{Enfin, ik was thans,,,.}{gewaarschuwd en hij Gomeer III}{had zijn plicht gedaan}\\

\haiku{Dan werd ze moe en.}{verzocht een dag met rust te}{worden gelaten}\\

\haiku{Ze was nu plots als.}{van de duivel bezeten}{om weer thuis te zijn}\\

\haiku{{\textquoteright} Wel neen, ik snakte.}{naar een kind uit loutere}{liefde voor Veerle}\\

\haiku{Gelijk een hond zo,.}{gedwee en zo laf liep ik}{naar mijn Meesterse}\\

\haiku{Ik deed het, en ik.}{schaamde me over de pap die}{in mijn aders vloeide}\\

\haiku{Honderd maal had ik.}{het gehate manwijf in}{gedachten gedood}\\

\haiku{Ze zat de meeste.}{tijd in een zetel voor het}{raam van de living}\\

\haiku{Ik zag dat haar ogen.}{uitermate groot en rond}{waren geworden}\\

\haiku{Hij legt daarbij zijn,,.}{handen overeen op zijn borst}{leent het oor en wacht}\\

\haiku{{\textquoteleft}Kom binnen,{\textquoteright} zegt de, {\textquoteleft}.}{Burgemeesterwe gaan een}{glaasje port drinken}\\

\haiku{Wel, ge belooft al,.}{wat ze willen en ge zoudt}{er een eed op doen}\\

\haiku{Over wensen uit die.}{hoek doet men wijselijk niet}{te redeneren}\\

\haiku{En hij brengt een toast,.}{uit in zo schoon Vlaams dat ge}{er niets van verstaat}\\

\haiku{Want hij heeft er al,.}{een hele boel gezien en}{van alle soorten}\\

\haiku{Dat is het geheim.}{van zijn alomvermaarde}{zachtheid van gemoed}\\

\haiku{Ik sluit aan, tot het,.}{uiterste van mijn krachten}{om leven en dood}\\

\haiku{Want, hij mag het wel,.}{weten iedereen in de}{Karot houdt van hem}\\

\haiku{En verder keurt zij.}{de opgewondenheid van}{Ida tenemaal af}\\

\haiku{Het is niet vlakbij,,.}{merkt hij op en de regen}{heeft de nacht vervroegd}\\

\haiku{Zij weet niet waar hij,.}{gebleven is en zij moet}{het ook niet weten}\\

\haiku{Zij zal gelaten,.}{luisteren nu naar Mak al}{geeft zij er niets om}\\

\haiku{De ganse Burcht, met,.}{zijn meesters en zijn macht kan}{haar nooit meer schelen}\\

\haiku{Hij heeft ineens de.}{goede kant van zijn slechte}{positie ontwaard}\\

\haiku{Integendeel brengt,.}{zij die op haar eigen hart}{dat dreigt te barsten}\\

\haiku{Hij sluipt opwaarts en.}{nadert met de lucht die heet}{uit zijn lippen stoot}\\

\haiku{Maar het is alles,.}{volkomen onbelangrijk}{alles doodgewoon}\\

\haiku{Het zit hem vers in}{de plooien aan het lijf en}{hij vertrouwt blijkbaar}\\

\haiku{{\textquoteright} Toontje herleest en.}{herleest het briefje dat aan}{zijn vingeren beeft}\\

\haiku{Wacht nu een beetje,.}{mompelt de koster op de}{korte weg naar huis}\\

\haiku{Zij wiegelt in het.}{licht van de winkel en zij}{nadert vol gratie}\\

\haiku{{\textquoteright} roept Achiel, terwijl hij.}{Ida tegemoet loopt en haar}{van haar fiets bevrijdt}\\

\haiku{Zij is zo bleek, dat.}{Achiel het raadzaam acht haar naar}{huis te brengen}\\

\haiku{Hij kust haar wild over,,.}{het aangezicht over de ogen}{hij zoekt haar lippen}\\

\haiku{{\textquoteright} Niemand verwacht dat.}{Toontje Rozier daarop iets}{te antwoorden heeft}\\

\haiku{Zij besluiten langs.}{de Vogelenzang weer naar}{het dorp te keren}\\

\haiku{Hij staat v\'o\'or haar, hoog,.}{in de kleur en zijn oog schiet}{wrokkige blikken}\\

\haiku{Hij is een dikke.}{vriend van Rozier die aan het}{harmonium zit}\\

\haiku{En vermits hij nu,.}{het geld teruggeeft zal er}{geen haan over kraaien}\\

\haiku{En zij ervaart een.}{afstand die verder dan de}{oneindigheid reikt}\\

\haiku{En zijn stiernek stoot.}{onmeedogend zijn voorhoofd}{tegen het plankier}\\

\haiku{De winnaar scharrelt.}{zijn kleren bijeen en maakt}{zich uit de gaten}\\

\haiku{Er valt nu plots op.}{de Berg-ter-Oigne}{een doodse stilte}\\

\haiku{En zo blijft zij een,.}{tijdje totdat de wereld}{ophoudt te bestaan}\\

\haiku{Daarboven dunkt hem.}{dat hij mee aan het zweven}{gaat met de wieken}\\

\haiku{Er is een aarde,.}{en de Schepper heeft ons uit}{aardse klei gewekt}\\

\haiku{Zij besluit op een.}{middag een bezoek aan de}{Terve's te brengen}\\

\haiku{Het is een gemak,.}{voor hem en het verhoogt zijn}{priesterlijk gezag}\\

\haiku{De dag eindigt in.}{een zwoelte die uit de ovens}{van de hemel valt}\\

\haiku{Tante Celesta.}{en Ida bezoeken in der}{haast het achterhuis}\\

\haiku{Misschien begrijpt hij.}{niet wat zij zo overvloedig}{onder woorden brengt}\\

\haiku{Meteen worden zijn,,.}{ogen droog en hij houdt niet af}{haar aan te staren}\\

\haiku{De ruimte gonst van.}{de prille vliegbeestjes die}{blijven aanzweven}\\

\haiku{Zou hij van binnen,?}{gekwetst zijn want aan het hoofd}{is niets te merken}\\

\haiku{Aneveert met herden!}{moet dat leven ende laet}{Bourgongnen waeyen}\\

\haiku{'t Was alsof de.}{hand Gods de hoogmoed in het}{stof had geslagen}\\

\haiku{Zij doet zich gelden,,.}{zonder twijfel zonder spijt}{zonder genade}\\

\haiku{Zij sluit het boek en.}{verstaat dat zij wordt verzocht}{de deur te openen}\\

\haiku{Hij heeft het sterkste,.}{kroost van het woudland gehad}{jongens als beren}\\

\haiku{Hij maakt aanstalten.}{om er onmiddellijk van}{onder te trekken}\\

\haiku{de pastorij van.}{deze stervensnood op de}{hoogte heeft gebracht}\\

\haiku{Onder de grauwe.}{deken kan men niet eens meer}{een lichaam raden}\\

\haiku{{\textquoteright} En hij slaat voor zijn.}{eigen geruststelling een}{suppletair kruisje}\\

\haiku{Mak Jeroen hoeft zich.}{niet af te vragen van waar}{zij gekomen zijn}\\

\haiku{Veel liever echter:}{zouden zij gebaren als}{de gemene man}\\

\haiku{Laat het de woorden,.}{slaken dat het minder dan}{uzelf verschuldigd is}\\

\haiku{De kleine Pia zal.}{gauw haar waardige plaats op}{de Burcht innemen}\\

\haiku{{\textquoteright} Zij merkt hoe door de.}{wijde baard van de Reus een}{korte rilling vaart}\\

\haiku{Het kan een ree zijn,.}{natuurlijk maar Mak aarzelt}{nooit zich te dekken}\\

\haiku{Ten ware dat de ...}{reden bij de gestrenge}{paters van Achel ligt}\\

\haiku{Iffratje vangt met.}{de lering aan elf maanden}{v\'o\'or de plechtigheid}\\

\haiku{Het is zijn gebod.}{dat wij onze moed aan de}{zijne beproeven}\\

\haiku{Zullen wij nochtans?}{ons brood uit de hand van God}{blijven bedelen}\\

\haiku{Zij geniet van die,.}{stilte waarbinst zij Klaus weet}{ontredderd worden}\\

\haiku{Men zou ook zeggen.}{dat het zich met het noodlot}{vereenzelvigd heeft}\\

\haiku{Het ingrijpen van.}{Iffratje heeft daarbij de}{doorslag gegeven}\\

\haiku{Zij zal leven naast.}{de stamvader die de wet}{in zijn handen houdt}\\

\haiku{{\textquoteleft}Het is goed,{\textquoteright} herhaalt;}{Nicodeem terwijl hij de}{checks verder invult}\\

\haiku{Met een dergelijk.}{resultaat zal iedereen}{opgetogen zijn}\\

\haiku{Het ligt in zijn stijl.}{er zo fijntjes mogelijk}{achter te zitten}\\

\haiku{{\textquoteright} Zij danst de trappen,.}{af en het hout kraakt niet eens}{onder haar zolen}\\

\haiku{Zij bekomt echter.}{niet seffens van de drift die}{in haar nadavert}\\

\haiku{Hij gaat eigenlijk.}{met de minimale vorm}{nog het minst akkoord}\\

\haiku{Een hele wereld.}{is geruisloos rondom hem}{in het niet gestort}\\

\haiku{De stilte van het, '.}{woud wacht op de stoot diet}{al bevrijden moet}\\

\haiku{Het speeksel schiet hem,.}{onder de tong en hij geeft}{zijn paard de sporen}\\

\haiku{Hij moet verder, de,.}{dennen voorbij de snoer van}{bergen bereiken}\\

\haiku{Het is het brood dat.}{Zo\"e naar haar mond te wege}{was op te steken}\\

\haiku{Hij strijkt rustig over,.}{zijn baard zoals hij na een}{eetmaal pleegt te doen}\\

\haiku{En nu moet Zo\"e eens.}{goed luisteren naar wat hij}{haar te vragen heeft}\\

\haiku{De Nachtegaal draait,.}{gesmeerd onder de impuls}{van Mamme zijn vrouw}\\

\haiku{{\textquoteright} Zij glijdt langzaam aan.}{Ida's knie\"en neer en neigt haar}{voorhoofd op Ida's borst}\\

\haiku{Dan verschijnt Plone,.}{met het schenkbord waarop de}{roemers fonkelen}\\

\haiku{De jonge Burchtheer,.}{is daar hij vraagt of hij mag}{komen luisteren}\\

\haiku{En discretie de.}{doelmatigste voorwaarde}{van zijn uitvoering}\\

\haiku{Pia haast zich om de.}{Burchtheer bij zijn inspanning}{behulpzaam te zijn}\\

\haiku{En wat heeft dan een?}{kristene niet al door de}{vingeren te zien}\\

\haiku{De torens van vuur.}{aan de windhoeken van de}{bergtop doven uit}\\

\subsection{Uit: Mijnheer J.B. Serjanszoon. Orator didacticus}

\haiku{Eigenlijk weten,,.}{wij niet v\'o\'or 1785 wie mijnheer}{J.B. Serjanszoon was}\\

\haiku{Die ouwe heer is,,.}{haar oom monsieur de la}{H\^etraie van Brussel}\\

\haiku{hoe blozen op haar.}{tere wangen haar onschuld}{en haar schuchterheid}\\

\haiku{- Lieve dames en,,.}{heren riep hij daar moeten}{wij eens bij klinken}\\

\haiku{De Griffier stond recht.}{en riep dat hij dadelijk}{zijn knecht nodig had}\\

\haiku{De officier der;}{dragonders bood zijn arm aan}{juffrouw Dieulafoy}\\

\haiku{Maar ik beveel aan.}{al uwe zorgen mijn boeken}{en mijn papieren}\\

\haiku{Ik vraag geen pauw op,.}{een gouden schaal omdat ik}{dit niet nodig heb}\\

\haiku{Zal ik mij wild en?}{ongenadig overgeven}{aan al mijn driften}\\

\haiku{Serjanszoon vroeg hem}{dan waar hij heentrok en met}{welk bijzonder doel}\\

\haiku{Hij stond waarlijk recht,...}{op zijn magere benen}{viel noch wankelde}\\

\haiku{zodat ik verplicht...}{ben geweest mij alweer wat}{te gaan opknappen}\\

\haiku{Het leven, mevrouw,.}{is beter naarmate men}{het dieper beleeft}\\

\haiku{Hij ging, schuivend langs.}{de dikke tapijten en}{leunend op zijn staf}\\

\haiku{Mijnheer Serjanszoon.}{schoof zijn hand om de le\^en van}{juffrouw Cornelie}\\

\haiku{Ik hink zelf, mijn vriend,,.}{ofschoon het zeker is dat}{ik nooit zal sjieken}\\

\haiku{Mijnheer Serjanszoon.}{werd waarlijk aanmatigend}{en rondde zijn borst}\\

\haiku{Hij perst de druiven,.}{van zijn wijngaarden uit over}{zijn mond en pinkoogt}\\

\haiku{vloekt de vermoeide,,,!}{Winter gij zijt laf jonker}{en schiet van verre}\\

\haiku{Ik zie de botten.}{springen op de hazelaars}{en de vlierstruiken}\\

\haiku{Hij boog en wuifde,.}{met zijn hand over de tafel}{al glimlachend flauw}\\

\haiku{Hij doopte zonder.}{reden zijn duim tot op de}{kneukel in zijn wijn}\\

\haiku{Aldus staat vast, dat,,.}{ik om iedereens bestwil}{voorzichtig moet zijn}\\

\haiku{Lichte meerminnen.}{spuwden een wit waterschuim}{in zijn aangezicht}\\

\haiku{- Inderdaad, mevrouw,,:}{antwoordde hij dankbaar zo}{ziet gij in mijn ziel}\\

\haiku{Maar, lieve mevrouw,.}{ik verwonder met al dit}{geschrijf mijzelven}\\

\haiku{mocht ik nooit meer zien,,!}{geduldige vriendin wat}{toen mijn ogen zagen}\\

\haiku{Hij weende niet over,.}{zijn ziekte niet meer over zijn}{vermaledijding}\\

\haiku{Hij knikte, reikte,.}{zijn hand naar het bultje dat}{lachend naderkwam}\\

\haiku{- Ja! - En gij zult, vroeg,'?}{angstig mijnheer Serjanszoon}{de Louis weernemen}\\

\subsection{Uit: De nieuwe Uilenspiegel}

\haiku{Hij voelde wel dat.}{er iets onverkwikkelijks}{in zijn toestand was}\\

\haiku{Toen meende hij te,:}{zeggen zijn linkerhand naar}{het Broodhuis gericht}\\

\haiku{Twee dagen later,.}{moest Bettel Broederlam gaan}{liggen met het pleures}\\

\haiku{Hij vulde zijn oogen.}{met het spektakel van clowns}{en danseressen}\\

\haiku{Op den bek van den.}{hoogsten steenen spuwer kwam een}{bloot kindje zitten}\\

\haiku{{\textquoteright} - {\textquoteleft}Nu komt er een man,.}{met vuile handen om de}{kaartjes te knippen}\\

\haiku{Dan ging hij weer bij.}{Nelleken zitten en vlocht}{den beloofden hoed}\\

\haiku{Hij bleek goed van pas,.}{maar zij moest toch haar hoofdje}{stijf rechtop houden}\\

\haiku{{\textquoteright} En hij djakte zoo...}{geweldig tot Sarelke met}{zijn zweep zwijgen moest}\\

\haiku{Ze vond dat hij wel,, -.}{rood was en fel en moedig}{maar lang geen Judas}\\

\haiku{De kassei wilde,.}{niet klinken onder hunne}{beslijkte hielen}\\

\haiku{Thijl stortte neer op.}{zijne knie\"en en begon}{te bidden luidop}\\

\haiku{De zon ging onder.}{in een apotheose van}{goud en karmozijn}\\

\haiku{{\textquoteright} Een halve maat te.}{vroeg en geheel alleen in}{de plots ijle lucht}\\

\haiku{Langzaam wuifde ze,.}{ermede en keerde zich}{om en danste licht}\\

\haiku{{\textquoteright} vroeg Jakeliene,.}{en ze verscheen te midden}{van het wingerdgroen}\\

\haiku{Zijn oogen lagen vast,.}{op de witte kassei waar}{niets meer te zien was}\\

\haiku{Zijn kale schedel,.}{glansde gedempt gelijk een}{eenzame avondkim}\\

\haiku{Thijl hing, boven de,.}{beukkruin aan de uiterste}{taknaald te wiegen}\\

\haiku{De kleinste droeg een.}{aapje op zijn schouder en}{blies op een schalmei}\\

\haiku{Hij struikelde over.}{de gracht en stapte beslist}{naar het Eremijtbosch}\\

\haiku{{\textquoteright} lachte hij luid, {\textquoteleft}geen, '....}{hoogwater ofk begin}{u te kriebelen}\\

\haiku{Wegens zijn statig.}{voorkomen heette men den}{vader Pijken-Aas}\\

\haiku{De haag smoorde een,.}{vloek en Pijke-Zeven}{deinsde achterwaarts}\\

\haiku{Hij had een jong lijf.}{dat de frissche vormen van}{den groei vertoonde}\\

\haiku{Zijne satijnen.}{voeten rustten op een}{windhond van albast}\\

\haiku{Het begon al laat,:}{te worden maar hij wist haar}{toch zitten en riep}\\

\haiku{{\textquoteright} Tegelijkertijd,.}{ontdekte hij Markies die}{langs de gracht wegsloop}\\

\haiku{En hij mompelde.}{er iets bij van zatlappen}{en van nachtraven}\\

\haiku{Boer Soeverein en,.}{juffrouw Ursule hadden}{hem gehoord tweemaal}\\

\haiku{{\textquoteright} Thijl Uilenspiegel.}{reikte de handen naar een}{beeld dat te hoog hing}\\

\haiku{Hij viel op zijne.}{knie\"en en krabbelde in}{het groengrijze net}\\

\haiku{Het bed gaapte hem.}{heet toe en dekte hem met}{intieme reuken}\\

\haiku{Een prevelen van.}{haren mond zuchtte gretig}{aan zijne lippen}\\

\haiku{In den laten avond.}{had hij met Nelleken een}{belangrijk gesprek}\\

\haiku{Thijl wendde zich even,.}{naar den kant waar het wijde}{water donkerde}\\

\haiku{- {\textquoteleft}Zoo is 't goed,{\textquoteright} zei, {\textquoteleft}.}{hijen ik blijf bij u. Spreek}{mij maar niet tegen}\\

\haiku{Ze paste nochtans.}{zoo aardig in dees vroom en}{peinzende midden}\\

\haiku{Na den eten, leidde,.}{hij mij over den tuin tot bij}{den grooten mesting}\\

\haiku{Thijl vroeg: - {\textquoteleft}Zeg, Pijke,?}{mag ik u vragen wat gij}{te Brugge komt doen}\\

\haiku{Dan nam  hij het.}{kind en stak het op in het}{licht van het venster}\\

\haiku{Elken dag komt ge,.}{en elken dag is het ons}{een nieuwe blijdschap}\\

\haiku{Hij was tevreden.}{over het akkoord dat hij met}{haar gesloten had}\\

\haiku{to car je taime}{et tu doit pardonner tout}{les jalouseries}\\

\haiku{Hij kon een zucht niet.}{neerdwingen die als een prop}{in zijne keel zwol}\\

\haiku{{\textquoteright} Thijl's lach daverde,:}{door de keuken maar hij zag}{geen steek v\'oor oogen meer}\\

\haiku{Die kijken naar een,;}{vogelken wit Dat in den}{wijden hemel zit}\\

\haiku{De stad lag geheel.}{onder een poeiering van}{teer-gulden licht}\\

\haiku{Ze was al een heil,:}{eind ver als ze zich verschrikt}{omwendde en riep}\\

\haiku{Daar schoot ineens de,!...{\textquoteright}}{merrie van Belle-Trees naar}{voren als een pijl}\\

\haiku{Ik zag haar onder.}{water vreeselijk werken}{met armen en beenen}\\

\haiku{Nelleken, bleek van,.}{schrik reikte naar hem hare}{smeekende handen}\\

\haiku{Op den Brusschelschen.}{steenweg had hij een bonten}{rolwagen gezien}\\

\haiku{Zijn neus vleugelde.}{bevend bij elken geur dien}{de zomerwind bracht}\\

\haiku{Thijl had Pierlapeu.}{vastgekregen en danste}{ermee in het rond}\\

\haiku{Maar het andere}{oog puilde heerlijk uit en}{sloeg naar links en rechts}\\

\haiku{{\textquoteright} deed Zoster, {\textquoteleft}ik heb.}{geen reden om niet naar de}{foor te gaan met haar}\\

\haiku{{\textquoteright} riep Miss Violet {\textquoteleft}?}{met verontwaardigingwaar}{zijn uwe gedachten}\\

\haiku{Ze gingen in een.}{kolk van klaarte en roken}{de motorhitte}\\

\haiku{Uit al uwe macht moogt,.}{ge knokkelen maar mijd u}{voor den weeromstuit}\\

\haiku{Ge zijt gekomen.}{om hem te toetsen aan de}{taaiheid van zijn lijf}\\

\haiku{- {\textquoteleft}Maar gij dan, rosse{\textquoteright},, {\textquoteleft}?}{riep Mandienezijt gij zelf}{niet aan te spreken}\\

\haiku{Hij riep, de handen: - {\textquoteleft},!}{ten hemelBeiaardier houd}{op met rammelen}\\

\haiku{De arme jongen.}{was uitgeput van honger}{en vermoeienis}\\

\haiku{Voor de eerste maal,.}{in mijn leven zou ik dooden}{en ik beefde niet}\\

\haiku{En ik aanvaardde.}{een pijnlijke reis in de}{richting van Brussel}\\

\haiku{Vandaag was zijn plan.}{om mij aan den vijand over}{te leveren}\\

\haiku{De schreeuw smoorde in,.}{zijn strot dien Thijl met de koord}{had toegenepen}\\

\haiku{{\textquoteright}, kloeg hij, {\textquoteleft}'t Verraad!}{zit aan uw haard en wacht op}{u. Keer om en vlucht}\\

\haiku{{\textquoteright} deed Uilenspiegel {\textquoteleft},.}{als alles tegenwerkt dan}{kan het nog verkeeren}\\

\haiku{- {\textquoteleft}Geef mij het pakje,.}{dat de juffrouw u in de}{handen heeft gestopt}\\

\haiku{Hij zei: - {\textquoteleft}Oolijke,!...}{bult ik zal u wel anders}{te knippen krijgen}\\

\haiku{De Koning heeft ons.}{gefeliciteerd en het}{heeft ons aangedaan}\\

\haiku{Maar ik had bloed aan,.}{mijn mond want de mond van den}{sergeant bloedde}\\

\haiku{Ge zoudt over hun gansch.}{lijf geen enkel luisje meer}{gevonden hebben}\\

\haiku{We waren eens op.}{de baan en uw lach klonk als}{een kelk van kristal}\\

\haiku{{\textquoteright} als een die mij in,,.}{zijn glorie na een verre}{reis weder ontmoet}\\

\haiku{Gloeiende torens,.}{gingen op hingen hoog in}{de lucht te lichten}\\

\haiku{Twee Zwarte Bloemen.}{duiken op uit den grond als}{uit een valdeurken}\\

\haiku{Een arm steekt roerloos,.}{op naar mij uit een grijzen}{stapel van lijven}\\

\haiku{Het is toch waarlijk!}{niet mogelijk dat  ik}{iets bedreven heb}\\

\haiku{Nu zie ik Sarelke.}{en el'Dj\^ozef en Staaf op}{een hoopken liggen}\\

\haiku{{\textquoteright} Maar Piet Hein fronste.}{zijne wenkbrauwen en zijn}{blik werd hard als staal}\\

\haiku{Kerels met witte.}{kluppels regelden zwijgend}{en ernstig den gang}\\

\haiku{Thijl voelde rond hem,,.}{de werkzame ziedende}{stampende ruimte}\\

\haiku{Toen gingen hare.}{magere handen streelen}{over een kinderkop}\\

\haiku{John werd gewaar.}{dat hij een onverwachten}{uitslag had bereikt}\\

\haiku{Thijl bedankte en.}{liep vroolijk naar den kijker}{van zijnen blauwvoet}\\

\haiku{Zijn gansche wezen.}{sprong gelijk een pijl uit de}{boeien van zijn angst}\\

\haiku{Pierlapeu stond met.}{gapenden mond en wilde}{de sneeuw opvangen}\\

\section{F.C. Terborgh}

\subsection{Uit: De condottiere en andere verhalen, gevolgd door Le petit chateau}

\haiku{Langzaam lopen we.}{terug naar het andere}{uiterste vertrek}\\

\haiku{dat niemand buiten{\textquoteright}.}{het aannemen van zulk een}{voortleven kan}\\

\haiku{de grauwe morgen.}{breekt aan en dringt langzaam door}{de bovenlichten}\\

\haiku{Een krakende kar,,.}{reed voorbij een hond bleef voor}{hem staan snuffelend}\\

\haiku{Soms verblindde hem:}{terzijde een zoeklicht vlak}{boven het water}\\

\haiku{Men leunt roerloos in.}{een hoek en wordt gewiegd en}{kijkt door het open raam}\\

\haiku{Weer valt  een blad,,,.}{dof steels als ware het bang}{betrapt te worden}\\

\haiku{gletscher had gezien,;}{onwerkelijk als uit een}{andere wereld}\\

\haiku{Met de middagbus.}{ben ik naar Pensacola}{teruggereden}\\

\haiku{Een spiegel en troost.}{voor zijn eigen hopeloos}{onnut bestaan}\\

\haiku{alles hing slechts af,;}{van de mate van moeheid}{van uitgeputheid}\\

\haiku{Windstoten droegen;}{het slaapdronken luiden aan}{van schapenklokjes}\\

\haiku{Sedert vijf avonden;}{herleef ik die vaart langs de}{Bahamaeilanden}\\

\haiku{een zwart zwijn, zich in,.}{een modderplas wentelend}{versperde den weg}\\

\haiku{hij leek reeds ver op.}{het pad der verlossende}{contemplatie}\\

\haiku{eksters klapten er;}{en wielewalen vlogen}{roepend door het veld}\\

\haiku{Ik kroop voorzichtig.}{naar de vlak bij het water}{gelegen schuiten}\\

\haiku{Hij bewoog zich niet.}{en scheen mijn aanwezigheid}{niet te bemerken}\\

\haiku{pas nu ten volle,.}{geopenbaard die vroeger}{blind moet zijn geweest}\\

\haiku{Schaduwen en licht.}{en traliewerk brachten me}{terug naar Coimbra}\\

\haiku{En wie is tegen,?}{den sleur bestand de leegte}{van den grauwen dag}\\

\haiku{Het doel van den tocht,.}{was in zicht de eerste der}{goudsteden bereikt}\\

\haiku{In de groeiende:}{verwarring greep eindelijk}{de student het woord}\\

\haiku{Hij trok zijn degen,.}{en stormde vooruit recht in}{de lege steppe}\\

\haiku{of heft het hart zich?}{slechts tot hun kille verten}{kort voor den overgang}\\

\haiku{Maar voelt het lichaam;}{niet nog de koesterende}{warmte der aarde}\\

\haiku{misschien reeds hun graf. '.}{s Middags moesten de paarden}{worden afgemaakt}\\

\haiku{Van uitsteeksel tot,;}{uitsteeksel zou men dalen}{tot aan het water}\\

\section{P. Tesselhoff jr.}

\subsection{Uit: Het succes van den rechercheur}

\haiku{{\textquoteleft}En hoelang is u,,{\textquoteright}.}{weduwe mevrouw zoo liet}{Ad\`eles stem zich hooren}\\

\haiku{Zooals altijd werd om.}{half negen thee gedronken}{in de tuinkamer}\\

\haiku{Dominee had iets ',;}{opt hart maar hij wist niet}{hoe te beginnen}\\

\haiku{Eindelijk trok hij:}{de stoute schoenen aan en}{vroeg op den man af}\\

\haiku{Den volgenden dag.}{had domin\'ee Dijker voor}{zijn vertrek bepaald}\\

\haiku{Al dien tijd werd ik,.}{niets gewaar hoe nauwkeurig}{ik ook oplette}\\

\haiku{Geen kwartier later.}{sloeg het hek achter in den}{tuin wederom dicht}\\

\haiku{Gij zijt beschuldigd.}{en gestraft voor een misdaad}{door mij bedreven}\\

\haiku{Een blanco cheque.}{ontvreemde ik tot dit doel}{uit uw schrijftafel}\\

\haiku{Nu evenwel geef ik,.}{het op het leven is mij}{tot last geworden}\\

\section{Ger Thijs}

\subsection{Uit: De huilende man}

\haiku{Maar ze gebruikte.}{die woorden alsof ze nooit}{anders gedaan had}\\

\haiku{En ik verwachtte.}{dat ze meteen snerpend zou}{reageren}\\

\haiku{Ik schoof de tassen,.}{de bus in en nam afscheid}{van de dikke vrouw}\\

\haiku{Ik stak de straat over,,.}{opende het tuinhekje en}{liep naar de voordeur}\\

\haiku{Hoe kunnen we ons!}{dingen herinneren als}{ik je niet herken}\\

\haiku{Oom Theo had ook een!}{baard en je weet wat er van}{hem geworden is}\\

\haiku{Het was alsof het.}{lichaam daarboven groeide}{met elke ademtocht}\\

\haiku{Ze veranderde.}{niet van houding toen ik haar}{lichaam aanraakte}\\

\haiku{Anderzijds moet ze:}{aan dat afschuwelijke}{zinnetje denken}\\

\haiku{Ze kruipt in  bed.}{en ligt de hele nacht naar}{de muur te staren}\\

\haiku{Hij koopt meubels van.}{het geld dat hij gespaard heeft}{voor een motorfiets}\\

\haiku{Hij heeft haar grommend.}{op de matras  gedrukt}{en haar genomen}\\

\haiku{Ik speelde met de,.}{zaklantaarn zoals ik het}{vroeger had gedaan}\\

\haiku{Je zei steeds maar dat,.}{ik in niets in helemaal}{niets op vader leek}\\

\haiku{Waarschijnlijk was ik,.}{de enige die wakker was}{nu mijn zuster sliep}\\

\haiku{Zoals Matti die.}{opeens een klein mannetje}{van hem gemaakt had}\\

\haiku{Ik schraapte luid mijn,.}{keel om mijn aanwezigheid}{kenbaar te maken}\\

\haiku{De man die aan de,.}{andere kant naast me zat}{boog zich naar me over}\\

\haiku{Hoe maak ik mijn vrouw?}{duidelijk dat ik liever}{in de kelder zit}\\

\haiku{Ergens binnen in.}{jou zit een klein vogeltje}{dat vaders naam piept}\\

\haiku{Maar ik besloot niet.}{op te staan om ze tot de}{orde te roepen}\\

\haiku{D{\'\i}t bedoelde ik,{\textquoteright},.}{zei hij en ik voelde zijn}{gewicht op het bed}\\

\haiku{{\textquoteleft}O, dat was anders{\textquoteright},,.}{zei ze en het leek wel of}{haar stem teder werd}\\

\haiku{Ik keek verbaasd naar.}{de grauwe koppen die uit}{de ramen hingen}\\

\haiku{Maar juist toen ik me,}{om wilde draaien stond de}{eerste vrouw naast me.}\\

\haiku{Maar ik vermoedde.}{dat hij al had begrepen}{dat ik niet wilde}\\

\haiku{{\textquoteright} Hij schreeuwde opnieuw,.}{en luisterde daarna met}{ingehouden adem}\\

\haiku{Het was alsof ze,.}{naar het fotoalbum keek}{met geschrokken blik}\\

\haiku{Ik wist zo gauw niets,.}{te bedenken ik lachte}{wat bevreemd naar haar}\\

\haiku{{\textquoteright} Ze sprong op en trok.}{het met een paar stevige}{rukken van de muur}\\

\section{Theo Thijssen}

\subsection{Uit: Barend Wels}

\haiku{Pool, voorop, trachtte;}{ze tikken te geven op}{hun gladde koppen}\\

\haiku{vlak naast hem lag 'n, '.}{schaar en een eindje verder}{stondn klos garen}\\

\haiku{{\textquoteleft}Zal wel gauw komme,{\textquoteright}, {\textquoteleft}.}{toch antwoordde Barendmaar}{ik heb niet veel tijd}\\

\haiku{We hebbe van die.}{jonge nou toch nooit anders}{als plezier gehad}\\

\haiku{{\textquoteleft}En toch kajje.}{om dezen tijd nog welles}{w\'armer weer hebben}\\

\haiku{en {\textquoteleft}Ja, h\'e\'el geschikt,{\textquoteright}, {\textquoteleft}...}{vond Barenden dan met zoo'n}{stelletje boeken}\\

\haiku{scharrelde daar even, '.}{en kwam terug metn paar}{andere cahiers}\\

\haiku{Op de donkere ',;}{gracht hadden zet over Pool}{die wel sjeezen zou}\\

\haiku{Wat 'n stomme vent,,.}{eigenlijk Pool om nooit es}{wat uit te voeren}\\

\haiku{Nou moeten jullie '.}{maares een kwartiertje tot}{bedaren komen}\\

\haiku{Het \'e\'ene eiland;}{benedenaan was net een}{omgekeerde pijp}\\

\haiku{{\textquoteright} De jongen hield z'n,.}{armen beschuttend boven}{z'n hoofd en kreunde}\\

\haiku{{\textquoteleft}Dat afloopen van,,{\textquoteright}.}{sommige klassen h\`e zei}{hij zuchtend tot Wels}\\

\haiku{{\textquoteright} De moeder zette.}{voor beiden brood klaar en ze}{begonnen te eten}\\

\haiku{Ze zaten zeker,.}{in de kamer en dronken}{hun kopje koffie}\\

\haiku{{\textquoteright} Je kon toch merken,,,;}{dat het broers waren dacht de}{moeder vergenoegd}\\

\haiku{voelde zich als een,.}{machtig heerscher die \'o\'ok wel}{eens royaal kan zijn}\\

\haiku{Anderen keken,,,.}{in twijfel den meester aan}{of het zoo maar mocht}\\

\haiku{noemde af en toe,.}{een naam en dan trok er weer}{een kind aan het werk}\\

\haiku{Jij kwam net binnen,.}{zeg toen ik bezig was met}{hoogere dressuur}\\

\haiku{Maar ik heb ze wel.}{vanmiddag een beetje d'r}{onder gehouen}\\

\haiku{Wels ging in de post, {\textquoteleft},!}{van z'n lokaaldeur staan en}{riepH\'e h\'e-daar}\\

\haiku{Met een slag haalde,,.}{Wels de deur toe en bleef staan}{z'n klas aankijkend}\\

\haiku{de leien weg, net,.}{of ze al een half uurtje}{gedresseerd waren}\\

\haiku{Maar toch, als je af,.}{en toe eens iets gezegd werd}{schoot je gauwer op}\\

\haiku{Ze zaten beiden.}{in Barends kamertje en}{rookten een sigaar}\\

\haiku{Tegenwoordig praat,.}{ik er niet over met ze en}{ga m'n gangetje}\\

\haiku{{\textquoteleft}Moeder, scheid u toch,{\textquoteright}, {\textquoteleft}.}{uit riep Barend vroolijkwe}{zullen nog sjeezen}\\

\haiku{{\textquoteright} riep Nico blij, en -.}{rikke tak ging z'n potlood}{de proef te maken}\\

\haiku{Anders zie 'k het,.}{eerste uur toch niet anders}{dan cijfers da's waar}\\

\haiku{Komieke idee\"en, '....}{toch soms lui die buitent}{onderwijs stonden}\\

\haiku{voor in de zaal, op,.}{het tooneel stonden groote tafels}{met groene kleeden}\\

\haiku{Barend grinnikte,,:}{weer en zei tegen Nico}{die mee-glimlachte}\\

\haiku{{\textquoteright} ~ De schoolmeesters.}{begonnen langzamerhand}{weer te verdwijnen}\\

\haiku{als pauwtjes stapten ',;}{n paar heeren rond tusschen}{de tafeltjes door}\\

\haiku{en meneer Jansen.}{bemoeide zich bijzonder}{druk met de ramen}\\

\haiku{{\textquoteright} Een jongen, die niet,:}{dadelijk doorliep kreeg een}{draai om z'n ooren}\\

\haiku{hij zag Beckers z'n, '.}{ernstige gezicht en vond}{m bespottelijk}\\

\haiku{wel glimlachend, want;}{anders zou-ie oproerig}{geleken hebben}\\

\haiku{en o, \`als dan weer,:}{de beroerdigheid kwam dat}{raadselachtige}\\

\haiku{stom, dat hij \'o\'ok niet;}{een paar jongens gehouden}{had om te helpen}\\

\haiku{{\textquoteright} Zwaan keek 'em even aan;}{met een vreemden  blik in}{z'n donkere oogen}\\

\haiku{En Wels wenschten,.}{ze erg veel succes hij moest}{zich maar goedhouden}\\

\haiku{Wels gaf iedereen,:}{w\'e\'er een hand en natuurlijk}{zei juffrouw Veen toen}\\

\haiku{ze praatten koeltjes,;}{over het examen maar kregen}{er geen ruzie over}\\

\haiku{{\textquoteright} 't Leek wel, of de, ';}{bijzitter begreep dat ze}{t over hem hadden}\\

\haiku{{\textquoteleft}Ja, maar {\`\i}k ga gauw,{\textquoteright},.}{naar huis zei De Haas al half}{w\`egloopend van ze}\\

\haiku{{\textquoteleft}Ik kan niet helpen,{\textquoteright}, {\textquoteleft}.}{klaagde zemaar ik vind het}{toch een naren tijd}\\

\haiku{Hij zou d'r \'o\'ok door - -.}{en niet op het kantje maar}{met vlag en wimpel}\\

\haiku{Dat er nou iemand....}{door boffen net zoo ver kwam}{als hij door werken}\\

\haiku{{\textquoteright} {\textquoteleft}Och nee,{\textquoteright} speelde Wels, {\textquoteleft},.}{den onverschilligewie}{d\`an leeft wie d\`an zorgt}\\

\haiku{Hij voelde zich thuis, '.}{in het scheeve zaaltje van}{t eerste uur af}\\

\haiku{En hij hoorde niet,.}{meer wat ze nog zeiden maar}{vloog al vooruit}\\

\haiku{As-ie nou nog een, '.}{acte haalde zout de}{trouwacte wel zijn}\\

\haiku{{\textquoteleft}Ja,{\textquoteright} sprak Wels ineens {\textquoteleft}.}{ik heb de laatste dagen}{doorloopend het land}\\

\haiku{merkwaardig zeg, van, '.}{ons twee\"en dat we dat zoo}{t zelfde hebben}\\

\haiku{{\textquoteright} {\textquoteleft}Die verdient driemaal ',{\textquoteright}.}{zooveel geld asn ander}{viel Nico weer in}\\

\haiku{Je bent zwaar op de,{\textquoteright},.}{hand vanavond zei Nico en}{geeuwde overdreven}\\

\haiku{{\textquoteright} {\textquoteleft}O, dat kan best, hoor,{\textquoteright}, {\textquoteleft},,}{was Barend cynischhet is}{ook persoonlijk h\`e}\\

\haiku{Hij besloot nog even,,.}{te blijven stak een sigaar}{op en ging lezen}\\

\haiku{{\textquoteleft}Vooruit maar,{\textquoteright} riep Henk,.}{net of ze iets doen gingen}{als een veldtocht}\\

\haiku{En ernstig keek-ie, '.}{nu Wels z'n schrift in omes}{precies te kijken}\\

\haiku{dat hij niet wachten,;}{kon tot het Wels beliefde}{om aan te pakken}\\

\haiku{Hij zat bij Pool op,,.}{de kamer heele avonden}{te redetwisten}\\

\subsection{Uit: Een bonte bundel}

\haiku{Je was een beetje...,,:}{h\`e Disa je had zo je}{lievelingswoordjes}\\

\haiku{Maar, ze zal d'r nog, '.}{wel wonen anders had ze}{t wel geschreven}\\

\haiku{Een kleine jonge,,.}{arbeidersvrouw bezem in}{de hand deed hem open}\\

\haiku{{\textquoteleft}Aber,{\textquoteright} begon hij, en, {\textquoteleft}}{hij bleef voor securiteit}{maar Duits sprekenaber}\\

\haiku{Grootvader stond op.}{en liep met het spiegeltje}{naar de voorkamer}\\

\haiku{Och, och, wat is dat,....}{jaren lang jaren lang een}{gemartel geweest}\\

\haiku{{\textquoteright} er op te pakken,.}{en mijn sigarenbaas had}{z'n twee gulden hoor}\\

\haiku{Nou, daar mot je dan ',,.}{maares om vragen as-je}{d'r an toe bent h\`e}\\

\haiku{Op de kamer stond,,:}{behalve de twee bedden}{een soupertje klaar}\\

\haiku{Maar de typiste,.}{kwam niet onder indruk en}{zei kalmpjes van neen}\\

\haiku{en toen we z\'o de,.}{kwestie stelden voerden hij}{ons naar de ingang}\\

\haiku{dat is bij voorbeeld,.}{ook de grote les die we}{in Gen\`eve leerden}\\

\haiku{Ik heb trouwens nog;}{nooit zo iets ouds gezien als}{mijn ene overbuurvrouw}\\

\haiku{En die Cercle de,?}{la Renaissance waar we}{vanmiddag gaan eten}\\

\haiku{Dumas hield vol, dat.}{we eigenlijk helemaal}{geen tijd meer hadden}\\

\haiku{proberen eens, hoe;}{dubbelslaan over het bruine}{hekje hun bevalt}\\

\haiku{Ik schrok me dood, en.}{de hele coup\'e verging}{van medelijden}\\

\haiku{Nooit weer proberen,....}{zo'n al-wegrijdende}{tram in te halen}\\

\haiku{ik weggezonken.}{in een grote deftige}{voorzitterszetel}\\

\haiku{zo'n echte trouwe,;}{gewone schoolmeester gek}{op de kinderen}\\

\haiku{Voor \'e\'en nacht zat ik,.}{aan deze uitspatting vast}{dat was duidelijk}\\

\haiku{{\textquoteleft}Dank-u ja,{\textquoteright} zegt-ie, {\textquoteleft}'.}{rustigk was met hiernaast}{de klas in de war}\\

\haiku{Als ik dat nu weer,.}{eens vraag dan moet het antwoord}{wat vlugger komen}\\

\haiku{{\textquoteleft}Kom jij 'es even bij,, '.}{bord ventje en schrijf jij dat}{zinnetjees op}\\

\haiku{Is anders niet van,;}{de stomsten maar hier kan-ie}{met z'n hoofd niet bij}\\

\haiku{Dat heb je zo niet,?}{geleerd dat k\`an je zo niet}{geleerd hebben toch}\\

\haiku{Nee, de enkele,,?}{drie niet de dertig weet je}{d\`at nou ook nog niet}\\

\haiku{Hij neemt het krijt en '.}{trekt nijdig een streep onder}{t laatste getal}\\

\haiku{{\textquoteright} en verdween weer, nog:}{v\'o\'or hij de bakkerin gul}{had horen zeggen}\\

\haiku{En hij liep met een ' ';}{boog naart Zuiden naart}{Lago Maggiore}\\

\haiku{Ik doe voortaan nooit.}{meer een lange reis zonder}{huispantoffeltjes}\\

\haiku{Ja, 't is lente,,....}{zoele lente je kunt je}{jas wel uitlaten}\\

\haiku{hij beheerst het zo,;}{volkomen dat-ie er}{in weet te kletsen}\\

\haiku{Daar zit je dan met.}{een internationaal}{mysterie naast je}\\

\haiku{In Basel vind ik,.}{een d\'o\'orgaande wagon naar}{Holland finaal leeg}\\

\haiku{de portier blijkt ons.}{al bepaalde kamers te}{hebben toebedacht}\\

\haiku{Piet mikte met heel - '.}{z'n ziel de haas buitelde}{vant randje af}\\

\haiku{{\textquoteleft}Van dat betere,.}{mikken van jou heb ik nog}{niet veel gemerkt zeg}\\

\haiku{We keken achter,;}{al de schilderijen en}{achter de spiegel}\\

\haiku{we zochten op de,.}{onmogelijkste plaatsen}{maar de pijl was weg}\\

\haiku{Overal een grote;}{karaf drinkwater met twee}{kristallen glazen}\\

\haiku{Doorzoek ten derde,.}{male m'n tas en daarna}{weer al m'n zakken}\\

\haiku{{\textquoteright} heb ik gezegd, en.}{ik sta al met de sleutel}{m'n deur te openen}\\

\haiku{Verduveld leuke ',,'.}{lui warent. Hm zei ik}{la we maar gaan eten}\\

\haiku{Het woord {\textquoteleft}industrie{\textquoteright}...}{krijgt door zo'n bezichtiging}{inhoud voor je}\\

\haiku{Soms ook velden, met,;}{lage struikjes  een goeie}{halve meter hoog}\\

\haiku{zoiets als onze,.}{Westlandse kassen maar dan}{in de open lucht}\\

\haiku{En m'n overbuurvrouw;}{wordt elke vijf minuten}{een jaartje ouder}\\

\haiku{Wij stegen uit en,?}{dromden om hem heen was er}{nu eigenlijk was}\\

\haiku{{\textquoteright} De kelner geeft een,.}{soort uitleg in het Frans die}{we niet begrijpen}\\

\haiku{Onze kelner raakt,,....}{in geestdrift en stelt voor een}{likeurtje er bij}\\

\haiku{En laat ik je nu,.}{vertellen wat ik nog m\'e\'er}{heb gehad vannacht}\\

\haiku{'t Is er geen een, ',.}{t is werkelijk een zwart}{vlekje anders niet}\\

\haiku{in dat dorp huur ik, ';}{eenvoudig een auto en}{ik rij naart kamp}\\

\haiku{Ik kon alleen nog,;}{niet onderscheiden wie van}{de vijf het waren}\\

\haiku{Ze keken elkaar,:}{met akelig doffe ogen aan}{ze vroegen elkaar}\\

\haiku{Op de bodem, het,.}{was precies de jampot lag}{nog wat gekleurd vocht}\\

\haiku{{\textquoteleft}Z\'o precies kan je{\textquoteright},.}{het in een kamp niet nemen}{zei hij verzoenend}\\

\haiku{Maar Jaap die nu ook,.}{met z'n fiets voor de tent stond}{keek bedenkelijk}\\

\haiku{'k Heb nog nooit zulk!}{een zakdoek uit m'n zakken}{te voorschijn gehaald}\\

\haiku{Jaap vergat zich, en;}{hapte dadelijk een mop}{chocolade af}\\

\subsection{Uit: Egeltje}

\haiku{Nou allemaal je{\textquoteright},, {\textquoteleft}{\textquoteright}.}{smoelen dicht zei de majoor}{zal je es wat zien}\\

\haiku{De egel begon, met:}{eigenlijk den majoor een}{beetje te pesten}\\

\haiku{'k Zal 'm v\'o\'or dat....}{ik wegga de nodige}{instructies geven}\\

\haiku{Toch was het krabben;}{van die nageltjes wel een}{beetje vervelend}\\

\haiku{Ik stapte m'n bed, '.}{uit en zette de deur naar}{t voorvertrek open}\\

\haiku{{\textquoteleft}Daar begrijp ik nou,! '}{letterlijk niks van waar dat}{beest gebleven is}\\

\haiku{Wonderbaarlijke.}{schijnbare verdwijningen}{had-ie meegemaakt}\\

\haiku{maar wij hadden de,.}{smoor in om die wacht om half}{twee die zat ons dwars}\\

\haiku{{\textquoteright} O, man, ik dacht da ',;}{k me beroerd zou lachen}{maar ik hield me in}\\

\haiku{Laat hem 'es lopen, ', '?}{ik moetem toch zien lopen}{voor ikn bod doe}\\

\section{Antoon Thiry}

\subsection{Uit: Ach, de kleine stad...}

\haiku{Zoo 'ne rijke mensch,?}{zou ginder wel algemeen}{bekend zijn niet waar}\\

\haiku{Neen, 't vermoeden,.}{van voor den aap gehouden}{te zijn was verkeerd}\\

\haiku{En 'k vroeg hem toen.}{of hij daar geenen zekeren}{Mister Tuits kende}\\

\haiku{'t Is nog wel zoo,,,?}{ver niet maar dat komt wel he}{Mitteke-lief}\\

\haiku{{\textquoteright} vroeg de notaris.}{en aanstonds had hij met zijn}{vouwbeen het stuk open}\\

\haiku{Om een oogenblik.}{te peinzen of die kozijn}{stapel-zot was}\\

\haiku{{\textquoteright} zuchtte Stafke en.}{zijn handen beefden als hij}{den brief teruggaf}\\

\haiku{En zie, Stafke, eer,.}{ik aan de deur was waren}{ze al bezweken}\\

\haiku{Na de klucht van dien,.}{kozijn en deze van Fee}{Verbuken de uwe}\\

\haiku{{\textquoteright} En met een leelijk,.}{woord naar den Lode stapten}{ze nijdig verder}\\

\haiku{Hij had ons moeten,!}{verwittigen en daar gaan}{we geenen steek van af}\\

\haiku{Mijn gedachten staan,!}{nu op iets anders op iets}{heelemaal anders}\\

\haiku{Een maand ging voorbij,,,}{een tweede een derde en}{men dacht niets anders}\\

\haiku{{\textquoteleft}Hang mijnen overjas',,.}{en mijnen hoed goe weg he}{dat niemand hem ziet}\\

\haiku{{\textquoteright} herhaalde hij steeds,.}{binst hij zich langzaam van zijn}{zot-pak ontdeed}\\

\haiku{Ze zouden hem ferm.}{in zijn bottigheid hebben}{laten staan blinken}\\

\haiku{En de weinigen ', '?}{diet zagen wilden ze}{t wel begrijpen}\\

\haiku{Ge weet het zoo goed,,,.}{als ik dat is alles voor}{ons voor u voor mij}\\

\haiku{De koning, wat zeg,,,, '.}{ik de keizer neen meer nog}{de god vant bier}\\

\haiku{Maar zoo 's avonds zie, ', ',!}{alst stil is zoone pot}{of drie vier daarvan}\\

\haiku{{\textquoteleft}Als ik u kon doen, ',,}{krimpenk deed het direct}{Gust-jongen}\\

\haiku{Maar ja, lang duurde,,:}{het niet geen twee maand of weer}{kwam er muziek in}\\

\haiku{{\textquoteright} bromde Gust terwijl.}{hij hun twee slaphangende}{vingeren reikte}\\

\haiku{En Piet-hierzie, die.}{beweert dat het de zwaarste}{vent uit de stad is}\\

\haiku{Ge moogt mij gelooven, '!}{er loopen er int land}{niet veel rond lijk gij}\\

\haiku{En wij die hem al,,!}{op weg zagen tusschen twee}{Cellebroers naar Gheel}\\

\haiku{Wat ruiten kapot,.}{en ook wat pannen doch dat}{was van het minste}\\

\haiku{Met een wip was de.}{andere de keuken uit}{en achter den toog}\\

\haiku{Een stomme brief van.}{de dikken uit Brussel was}{er de oorzaak van}\\

\haiku{Heel de stad en ook.}{menschen van buiten de stad}{kwamen er naar zien}\\

\subsection{Uit: De droomer}

\haiku{Hoe schoon en heilig!}{lag de tuin daar nu in het}{stille witte licht}\\

\haiku{Waarom was Vader?}{altijd zoo streng en was hem}{niets toegelaten}\\

\haiku{Ze gingen buiten.}{en de deur sloeg met een bons}{dicht achter de twee}\\

\haiku{Als de krant kraakte.}{kromp hij ervoor ineen als}{een hond voor den stok}\\

\haiku{Michiel liep daar lijk.}{dronken en hij struikelde}{alle vijf voeten}\\

\haiku{Michiel voelde dat.}{hij ging wegduizelen als}{hij zich niet sterk hield}\\

\haiku{Langs de fabriek om,,.}{die verlaten was op dit}{uur ging hij naar huis}\\

\haiku{Ja, hij zag het nu,... {\textquoteleft},{\textquoteright}.}{Vader wist er niets vanLees}{Michiel zei Vader}\\

\haiku{Al spelend wandelt.}{hij tot aan het vensterken}{en lonkt den tuin in}\\

\haiku{Michiel was blij iets.}{gevonden te hebben om}{naar te luisteren}\\

\haiku{En rillend liep hij,.}{door den regen die hem nat}{sloeg tot op zijn hemd}\\

\haiku{Ja, wat moesten wij er,.}{mee doen nu dat er geen plaats}{meer is op het Hof}\\

\haiku{Hij lei haastig de}{viool op het kastje v\'o\'or}{den witplaasteren}\\

\haiku{Want morgen is het.}{Vrijdag en dan wordt de kant}{binnengedragen}\\

\haiku{Wie had toen kunnen?...}{peinzen dat al die muziek}{voor haar bestemd was}\\

\haiku{Het doet haar beven.}{van geluk en brengt warme}{tranen in heur oogen}\\

\haiku{Ze legt de handen.}{op het hert en kijkt hem lang}{en gelukkig aan}\\

\haiku{Doch zijn hert bedroog.}{hem telkens en alles bleek}{aleven nutteloos}\\

\haiku{Thans, aan den noendisch,.}{gunde ze zich amper een}{vogelenbeetje}\\

\haiku{Michiel voelde het,...}{toen begon er iets van zijn}{droom te verzwinden}\\

\haiku{Een stap klonk op straat,.}{een deur bonsde en ievers}{schreeuwde er een kind}\\

\haiku{Hij wou recht staan, doch,.}{Eveline die tegenover}{hem zat was hem voor}\\

\haiku{En wie had het hem?}{toevertrouwd waar men Agnes}{naar toe gevoerd had}\\

\subsection{Uit: Gasten in het huis ten halven}

\haiku{Hij zweeg, kromp angstig.}{ineen en stak smeekend de}{handen naar hem uit}\\

\haiku{Geloof me, 'k ga...{\textquoteright}.}{wel vanzelf Piet liet toen zijn}{lawijd maar zakken}\\

\haiku{Als familie soms,}{begon over dat kleedsel dat}{het toch niet noodig was}\\

\haiku{Zelfs voor de stoep en.}{de ruiten nam hij een vrouw}{uit de geburen}\\

\haiku{Heel de stad kon niets,.}{anders denken als ze van}{zijn kuren hoorden}\\

\haiku{uiteendoen waren.}{ze den sigarenmaker}{zelfs zoo gaan heeten}\\

\haiku{Van een inzinking.}{was er aan den Generaal}{niet d\'at te merken}\\

\haiku{'t Kon niet anders,.}{of hij had schulden en hij}{maakte er nog bij}\\

\haiku{{\textquoteleft}Ze zijn er hier nog,!}{niet voor opgewassen voor}{chique dingen}\\

\haiku{hielp hij de dames '}{die om handwerkjes kwamen}{met raad en daad bij}\\

\haiku{een lieke in den.}{mond en locht wippend van stap}{als een dans bijna}\\

\haiku{{\textquoteleft}'k Geloof dat er!}{in dat schoone huisje nog}{wel wat plaats over is}\\

\haiku{een model van een,.}{weg effen en fijn om er}{op te rolschaatsen}\\

\haiku{'t Zullen er wel,,.}{zeldzame zijn uit vreemde}{landen vermoed ik}\\

\haiku{Hij kon er niets aan,.}{doen maar toen schoot hij toch in}{een lach die helmde}\\

\haiku{Och, zoo geerne werd,.}{hij onderwijzer niet om}{te gelooven bijkans}\\

\haiku{Waar dat alzoo met,.}{Pinneke naartoe moest was}{moeilijk te voorzien}\\

\haiku{En dat nogal wel!...}{nadat ze hem framasson}{hadden gemaakt}\\

\haiku{De Muzen{\textquoteright}, opdat.}{ze zich met hun eigen oogen}{konden overtuigen}\\

\haiku{Dat kan niet anders, ' '!}{t zit er in gebakken}{ent moet er uit}\\

\haiku{'t Waren echter.}{geen gewone artikels}{die hij daarin schreef}\\

\haiku{En 't waren me '!}{de mannekens die hij aan}{t woord liet komen}\\

\haiku{Een stuk dat te Lier!}{speelt en dat zouden ze niet}{mogen opvoeren}\\

\haiku{Want Mijnheer Michel,. '}{was niet uitgezongen na}{die drie vier eerste}\\

\haiku{Heur oogen pikten, ze.}{had pijn in de knie\"en en}{steken in den rug}\\

\haiku{Mijnheer Pastoor heeft' '.}{mij gezegd dak daarvoor}{bij u moest komen}\\

\haiku{En als die eerste.}{drie stoelkens af waren kreeg}{hij er andere}\\

\haiku{Doch van heur eigen,.}{werk hield ze hem angstvallig}{af zooveel het kon}\\

\haiku{Dat wordt nu alle,!}{jaren schooner en schooner  ulie}{Kindeke-Jezus}\\

\haiku{En 't was wel erg,.}{voor Mijnheer Lampaert maar}{niemand vond het goed}\\

\subsection{Uit: In ''t hofken van Oliveten' en VII andere verhalen van simpele menschen}

\haiku{{\textquoteright} Begijntje Bellijn.}{bleef binnenshuis en liet heur}{eigen niet meer zien}\\

\haiku{Voor elken stap die,,}{hij deed trok hij zijn grooten}{grofgeschoeiden voet}\\

\haiku{Zelfs werd hij toen lid.}{van de Congregatie van}{den H. Aloysius}\\

\haiku{Maar 'k bedwong me.}{en mijn lach verging in drie}{zuchtjes door mijn neus}\\

\haiku{En hoe komt het, denkt,?}{ge dat ik alles geloof}{wat in dit boek staat}\\

\haiku{Corneel voelde zich.}{zeer klein in deze groote kerk}{en hij boog het hoofd}\\

\haiku{Maar bij 't eerste,.}{woord over die zaak heeft Leo hem}{buiten gesmeten}\\

\haiku{'k Laat onzen hond ',!}{los als hij nog opt erf}{durft komen zeit Leo}\\

\haiku{* * * ~ En de zoete.}{maand van Mei heerschte wit}{en groen over de streek}\\

\haiku{Corneel zat in de.}{lommer van een kriekenboom}{aan den rand der gracht}\\

\haiku{Van wat er in zijn.}{huis gebeurde werd menig}{wonder ding verteld}\\

\haiku{Mijnheer Cosijn het.}{gevreesde huis achter de}{kerk zag binnen gaan}\\

\haiku{Hij zag hem zitten,,.}{onduidelijk in den smoor}{maar hij zag hem toch}\\

\haiku{De stilte streek neer,.}{uit de hooge ijle lucht en}{de nacht was op komst}\\

\haiku{Morgen avond zal ik{\textquoteright},.}{er voort over peinzen zei hij}{tegen zijn eigen}\\

\haiku{Hij had het op een}{tombola gewonnen en}{hij hield z\'o\'o veel}\\

\haiku{Hij stond recht, stak zijn,.}{bril weg zette zijn hoedje}{op en sprak weerom}\\

\haiku{Waarom was hij geenen:}{herder geweest tot wien de}{engelen zeiden}\\

\haiku{Toen was het dat op.}{verwijderde plaatsen de}{klokken nood klepten}\\

\haiku{Een stoelken kwam door '.}{t deurgat gedreven en}{stiet tegen zijn kop}\\

\haiku{En dat verhoedde,.}{de hemel want zijn tijd moet}{aleerst nog komen}\\

\haiku{Meestal is het.}{in het stadje zoo stil lijk}{op een Begijnhof}\\

\haiku{De vrouw lag op 't, '.}{sterven naart scheen uit het}{gesprek bij den toog}\\

\haiku{Dien nacht passeerde...}{onder mijn venster de bel}{eener bediening}\\

\haiku{Natuurlijk lachten,.}{de menschen hem en zijn werk}{erbij vierkant uit}\\

\subsection{Uit: Izegrim}

\haiku{{\textquoteright} {\textquoteleft}'k Spreek niet van 't,{\textquoteright}.}{Menneke-op-demaan}{sprak Trintje driftig}\\

\haiku{Wat er daar binnen, '.}{in dien jongen omging geen}{mensch diet verstond}\\

\haiku{En dan trok hij een:}{gezicht zoo venijnig en}{duvelsch da'k soms dacht}\\

\haiku{dat was 's nachts top.}{in den donkeren door den}{buiten gaan zwerven}\\

\haiku{Hij nam mij zoo in,}{beslag dat ik den dag van}{zijn jaargetijde}\\

\haiku{{\textquoteright} En zich tot den baas,:}{wendend achter zijnen toog}{kommandeerde hij}\\

\haiku{Wa' gaat hij daar toch'!}{doen en wa gaan we nu nog}{te hooren krijgen}\\

\haiku{En maakt er nu maar, '...}{rap komaf vank wil er}{dezen avond nog in}\\

\haiku{En ik, ik staan nu, ''...}{hier enk weet waarachtig}{ni wa beginnen}\\

\haiku{{\textquoteleft}Zie, Jos, als 't voor', ' '!}{u ni wask liet u staan}{enk trok naar huis}\\

\haiku{Voorloopig zei.}{hij nog niets te huis van wat}{er aan den gang was}\\

\haiku{Hij sloeg wit uit en.}{begon te beven terwijl}{hij het briefje las}\\

\haiku{Wat dien dolleman}{in zijn bol geslagen was}{om hier te komen}\\

\haiku{De Izegrim zei weer,,,.}{niets geen ja geen neen knikte}{zelf niet eens en ging}\\

\haiku{Ge gaat toch zoo geen?...}{patatten gaan schillen en}{roo-koolensnijen}\\

\haiku{Maar ineens met een,.}{stoot van zijn rooien kop deed}{hij een stap voorwaarts}\\

\haiku{{\textquoteright} ~ En 't was de.}{Vader niet alleen die dien}{dag zoo oordeelde}\\

\haiku{da's 't begin van ',.}{t ende hij trekt er vast}{en zekers weer uit}\\

\haiku{Door de hel gaan ik.}{er mee en geen haarke zal}{er aan miskomen}\\

\subsection{Uit: Meester Vindevogel}

\haiku{dat de menschen er....}{niet al te veel van merkten}{van het gebroddel}\\

\haiku{hier allemaal al.}{ronddraaide en wie er nog}{de trappen opkwam}\\

\haiku{De schrik zat er al}{zoovele dagen in en}{die klapte hij er}\\

\haiku{de stond dat Marus.}{recht rees en tegen zijn glas}{om stilte tikte}\\

\haiku{Da' zulde gij ook',!}{wel wete Meester gij die}{hier groot zijt gebracht}\\

\haiku{Waaromme toch had hij!....}{dat zelfde aan Lieneke}{ni mogen geven}\\

\haiku{Want pas was hij dan '}{ieverans buiten oft}{zelf-ver wijt}\\

\subsection{Uit: Mijnheer pastoor en zijn vogelenparochie}

\haiku{Maar om bezig te,.}{zijn met den grond om den grond}{te  zien geven}\\

\haiku{Zonder uw preeken!}{waar heel de stad zoo geerne}{komt naar luisteren}\\

\haiku{{\textquoteright} Zoowaar Mijnheer Doktoor:}{ookal was op zijnen}{poot komen spelen}\\

\haiku{Alsof hij al jaar,}{en dag hier was alsof hij}{nooit van zijn leven}\\

\haiku{ze zoo van zelfs open,,.}{in hem zetten zijn hert in}{schoonen witten brand}\\

\haiku{{\textquoteright} dat hem boven het.}{geboor en jachtig geklop}{toegeroepen werd}\\

\subsection{Uit: Het schoone jaar van Carolus}

\haiku{{\textquoteright} en hij keek van zijn.}{brevier naar den brief en zoo}{maar overentweer}\\

\haiku{Van rond Lichtmis had,.}{een breede warme wind de}{verten opengevaagd}\\

\haiku{{\textquoteleft}Eenen moet er in ons,!}{familie toch de dagen}{open houden Petrus}\\

\haiku{Carolus stond er.}{voor getroffen als voor een}{veropenbaring}\\

\haiku{Mijnheer Klabots had.}{reden tot klagen en hij}{sprak daarom zeer veel}\\

\haiku{De gebeeldhouwde ' '.}{poort ent hekken vant}{stadhuis waren toe}\\

\haiku{Voor de menschen in '.}{de kleine stad wast een}{gewenscht Goeieweekweer}\\

\haiku{{\textquoteleft}Maar nu komt de goeie! '!}{tijd toch nog weeromk Ben}{er danig blij mee}\\

\haiku{Treza viel algauw.}{in slaap en er kwam stilte}{in het huizeken}\\

\haiku{Ze sloeg even de oogen.}{eens op en glimlachte toen}{door heur tranen heen}\\

\haiku{Dank om de velden.}{en de boomen die spreken}{van Uwe heerlijkheid}\\

\haiku{Hij vond er geen, zei.}{toen maar wie hij was en vroeg}{brutaal naar zijn wijf}\\

\haiku{Ze is zij met ne,.}{schipper opgestoke en}{de kinders zijn mee}\\

\haiku{Amperkes was de,}{Joppes erover en liep al}{rapper dan zijn beenen}\\

\haiku{k Luisterde ne ':, ','.}{keer enk zeg jat is}{hem g hebt gelijk}\\

\haiku{{\textquoteright} {\textquoteleft}Lucifeir{\textquoteright}, zeg 'k, {\textquoteleft} '{\textquoteright}.}{draag dien beddebak weerom}{ofk doen ik het}\\

\haiku{- Twee dagen later '}{opt uur van den noen als}{de straten vol zon}\\

\haiku{ik gon man wijf en.}{man kindere opsoeke}{van hier tot Hollant}\\

\haiku{Partonneer het ma.}{en duzend kiere bedankt}{veur al a goetheit}\\

\haiku{Maar hij ontweek het.}{antwoord en begon seffens}{over den kapitein}\\

\haiku{De schaterende.}{Pastoor hield de handen op}{zijn schuddenden buik}\\

\haiku{Ze stonden allen.}{recht en staken geroerd hun}{glas naar den leegen stoel}\\

\haiku{{\textquoteright} zei hij, {\textquoteleft}op mijn twee!}{voeten bij den  eenen of}{bij den anderen}\\

\haiku{{\textquoteleft}Wij zijn al te groot,!}{om nog Margrietjeskaarsen}{te branden Juffrouw}\\

\haiku{Het koralen kruis '.}{int putteken van haar}{hals beefde lichtjes}\\

\haiku{Kozijn Duyvewaert.}{wees Carolus monkelend}{naar het ontbloote doek}\\

\haiku{Daar is den duvel,{\textquoteright}.}{mee gemoeid besloot hij}{ten langen laatste}\\

\haiku{stok onder den arm,,.}{deftig de pui af alsof}{er niets gebeurd was}\\

\haiku{Maar, met den besten, '.}{wil van de wereldt was}{al even veel gekort}\\

\haiku{Maar hij hield toch een.}{poos de oogen toe om het beeld}{niet te vergeten}\\

\haiku{Hij schoof ontroerd zijn.}{zetel achteruit en moest}{diep naar asem snakken}\\

\haiku{Mijnheer Duyvewaert.}{werd kwaad lijk een huis en deed}{maar niets dan zuchten}\\

\haiku{'t Water en de.}{lucht en de huizen errond}{zagen er wit van}\\

\haiku{t Ging Carolus.}{als een steek door zijn hart als}{hij dat gewaar werd}\\

\haiku{{\textquoteleft}Ze peinze da we,!}{de dood meebrenge da wij}{de dood bij hebbe}\\

\haiku{{\textquoteleft}Over een dag of tien ',,.}{ist begonnen Menheer}{in de Kerkhofpoort}\\

\haiku{{\textquoteright} zei Petrus met een '.}{natte stem terwijl hijt}{hekken openmaakte}\\

\haiku{Als 't gedaan was,:}{dansten ze er allemaal}{eens rond al zingend}\\

\haiku{de venten waren,}{nu weerom recht gekropen}{drumden allen lijk}\\

\haiku{{\textquoteright} Heur hart brak ervan,}{maar ze hield zich sterk tot hij}{terug op de been}\\

\haiku{Vervoerd sloeg ze toen.}{heur armen rond zijn nek en}{trok zich aan hem op}\\

\haiku{Machtig voelde ze.}{de wijding van het huis over}{heur ziel neerkomen}\\

\haiku{Carolus hief haar,.}{seffens op en droeg haar lijk}{een moeder heur kind}\\

\haiku{De onrust en de '.}{angst sloegen haar zinnent}{onderste boven}\\

\haiku{was het dat Mijnbeer.}{Duyvewaert voor den eersten}{keer het bed verliet}\\

\haiku{Met een snok, als liep,.}{ze tegen een muur bleef ze}{in het deurgat staan}\\

\haiku{Anna-Liza,.}{voelde zich ineens zinken}{rapper en rapper}\\

\haiku{zonder dat ze er,.}{iets tegen doen kan breekt heur}{hart nu opeens open}\\

\haiku{Een groote voldoening.}{komt over hem neer en hij kan}{bijna niet spreken}\\

\subsection{Uit: Voghelen in der muyte}

\haiku{Van den boer hadden ' '.}{zet gehaald en moesten ze}{t blijven halen}\\

\haiku{Veilig geborgen,,?}{te zijn met de zakken vol}{eer het dak inviel}\\

\haiku{Belogen werden!}{ze en bestolen en ze}{zagen er niets van}\\

\haiku{Even stond hij daar met,.}{de armen uiteen en mond}{en  ogen dwaas open}\\

\haiku{En als ge 't wat,.}{beter bekeekt was het niet}{te verwonderen}\\

\haiku{Tante Trees sprak er '.}{met de familie over die}{t heel heel goed vond}\\

\haiku{'t Weten dat Piet,.}{er ook nog was desnoods had}{hem sterk gehouden}\\

\haiku{Een  geluk was}{het voor hem dat Fin zich zo}{stilaan al eens meer}\\

\haiku{{\textquoteright} 't Was zo, hij was,.}{zot op het jongske en dat}{hoe langer hoe meer}\\

\haiku{Niet te geloven}{was het wat hij allemaal}{kocht en welk plezier}\\

\haiku{Zonder uw preken!}{waar heel de stad zo geerne}{komt naar luisteren}\\

\haiku{het kosten zou eer!}{Bienus er een hand zou doen}{naar uitsteken}\\

\haiku{{\textquoteright} dat hem boven het.}{geboor en jachtig geklop}{toegeroepen werd}\\

\haiku{die was verdeeld 'lijk,.}{een damberd  in kleine}{vierkanten perkjes}\\

\haiku{dat was er niet meer, '.}{te vinden in geen twintig}{uren int ronde}\\

\haiku{... 't Brood met hesp en.}{mosterd ging maar moeilijk naar}{binnen bij Baziel}\\

\haiku{Hij was kapot eer'!}{hij t halverwege van}{zijn bezoeken was}\\

\haiku{Morgen terug te ',.}{komen alst klare dag}{was dat deed hij niet}\\

\haiku{Hij sloeg met den piek,:}{van zijn stok eens hard tegen}{de open deur riep luid}\\

\haiku{t Wierd nog schoner '.}{dan hijt van zijn leven}{had durven dromen}\\

\haiku{Hij zette met een.}{klop zijn leeg glas neer op het}{trapke en stond recht}\\

\haiku{Den morgen daarop.}{reeds kwam er een bode met}{een huifkar van daar}\\

\haiku{E\'en kruiske alleen.}{hield zijn hart vast en dat was}{dat van Benooke}\\

\haiku{En gij die vroeger!}{nog geen vogelke in een}{kevieke wilde}\\

\haiku{Ge zijt toch zekers' ',?}{ni bang int dorp zo top}{onder de mensen}\\

\haiku{Van Lewieke zag.}{hij zelfs het tippeken van}{zijnen neus niet meer}\\

\haiku{Zijn hoofd lag op zijn.}{open gazetten-boeken}{en hij glimlachte}\\

\subsection{Uit: De zevenslager}

\haiku{Dat loopen, en dan...!}{die lach op dat oud gezicht}{God in den hemel}\\

\haiku{Waar hij dat alles,.}{haalde wisten de tantes}{en de nonkels wel}\\

\haiku{En niet alleen dat,.}{maar Flipke moest goed in de}{kleeren steken ook nog}\\

\haiku{De doopeling, die zich ',.}{zoo goed alst kon weerde}{vond dit toch wat sterk}\\

\haiku{{\textquoteright} dacht Flipke en zijn.}{hert begon te kloppen toen}{hij aan de beurt kwam}\\

\haiku{Zoo was het dezen.}{morgen geweest en zoo zou}{het nu ook wel zijn}\\

\haiku{Hij luisterde en,.}{sloeg zijn vette rooie handen}{meen van verbazing}\\

\haiku{{\textquoteright} En Flipke was fier.}{geweest en gelukkig lijk}{nog nooit te voren}\\

\haiku{{\textquoteright} zei de meester den,.}{volgenden keer toen het weer}{declamatie was}\\

\haiku{Hij sprak zacht en heel.}{op de letter en zijn stem}{beefde een beetje}\\

\haiku{God de Heer zag dat}{de boosheid der menschen groot}{was op de aarde}\\

\haiku{Dien kan Ik toch niet,{\textquoteright} {\textquoteleft} '.}{mee vernietigen zei O.L.H.}{zoonen braven mensch}\\

\haiku{Hij stapte er op,,,.}{en avant ze dreven den}{hemel uit naar hier}\\

\haiku{En als het daar was,.}{zwommen ze naar de hooger}{gelegen huizen}\\

\haiku{{\textquoteleft}Ik geloof dat die,!}{meester al even zot is als}{gij Zevenslager}\\

\haiku{Er waren er nog,.}{die vloekten maar geeneen zoo}{hard en luid als hij}\\

\haiku{{\textquoteright} dan klonk het telkens}{zoo luid of het vlak achter}{hem uit een mond kwam}\\

\haiku{'t Ergste was dat.}{het zooveel roeten op de}{eerde had gebracht}\\

\haiku{{\textquoteleft}Calmez-vous,,,!}{Monsieur Mutsaers de grace}{calmez-vous}\\

\haiku{Zoo gaarne had hij.}{moeder eens aangesproken}{over al die dingen}\\

\haiku{En werkt nu maar goed,,.}{Flipke dat ge primus wordt}{van de hoogste klas}\\

\haiku{Hij zat er naast den.}{baas en mocht van meet af aan}{ontwerpen maken}\\

\haiku{Flip verschoot nog wel. {\textquoteleft}?}{het meest van allen toen hij}{dat hoordeWatte}\\

\haiku{'t Ging hem af, hoe,.}{langer hoe beter en zijn}{gasten leerden goed}\\

\haiku{Flipke kreeg er nog.}{een geweldigeren schok}{van dan den eerste}\\

\haiku{En dan plots rood als,.}{vuur waarop hij spottend te}{grinniken begon}\\

\haiku{que tout allait bien.}{et qu'on entendrait bient\^ot plus}{de ses nouvelles}\\

\haiku{Als 't nu nog een,.}{beetje geduurd had was ik}{moeten gaan loopen}\\

\haiku{stond de oude heer.}{daar een wijle als van de}{hand Gods geslagen}\\

\haiku{Want op vader zijn ',:}{vraag voor wanneer hijt zich}{dacht bracht Flip's antwoord}\\

\haiku{W' hebben nog geld,!}{voor drie jaar dat moet er dan}{ook maar af kunnen}\\

\haiku{Maar veel sterker als, '!}{leer hechter van kleur ent}{neemt niets geen vocht op}\\

\section{Johan Rudolph Thorbecke}

\subsection{Uit: Thorbecke op de romantische tour}

\haiku{Wij zullen zien, hoe.}{ik hier en daar met hem en}{zijn systeem klaar word}\\

\haiku{Daarom vertrouw ik,,,,;}{gij mijn ouders blijft nog lang}{nog zeer lang bij ons}\\

\haiku{Ik verdien deze.}{rijkdom der tederste zorg}{en toeneiging niet}\\

\haiku{Van de overige;}{merkwaardigheden heb ik}{nog weinig gezien}\\

\haiku{Het bezoeken der,,;}{opera is gelijk alles}{in Berlijn kostbaar}\\

\haiku{Wellicht heb ik eerst,.}{heden gevoeld hoe lief ik}{mijn vaderland heb}\\

\haiku{Denkt, wijl ik zo schrijf,,.}{daarom niet dat ik somber}{of droefgeestig ben}\\

\haiku{Zo ging Thorbecke.}{in oktober 1820 op reis}{naar G\"ottingen}\\

\haiku{Thorbecke heeft zich.}{ook op zijn wijze met de}{romantiek verzoend}\\

\haiku{Hij was echter geen;}{predikant geworden maar}{in zaken gegaan}\\

\haiku{een studievriend van,.}{jrt en Koolhaas die ook naar}{Berlijn was gereisd}\\

\section{Aegidius W. Timmerman}

\subsection{Uit: Tim's herinneringen}

\haiku{Dit alles werd door,.}{Van Kinsbergen bekostigd}{ook de gebouwen}\\

\haiku{Er werd om half vijf.}{ingespannen en alles}{onder luid gejuich}\\

\haiku{Omdat alles zoo.}{langzaam ging bleef men ook maar}{liever in zijn dorp}\\

\haiku{hadden zij het met.}{hun karige lonen ook}{veel beter dan nu}\\

\haiku{Toen zuchtte hij diep,,...}{glimlachte sloot de oogen en}{werd bewusteloos}\\

\haiku{{\textquoteleft}De liefde voor de,,.}{natuur die jij hebt was ook}{je vader eigen}\\

\haiku{Gewoonlijk werd dit.}{verbod na eenige weken}{weer ingetrokken}\\

\haiku{Zijn godsdienstigheid.}{is m\'e\'er dan oppervlakkig}{en dikwijls blague}\\

\haiku{Nou krijg ik genoeg,.}{te vreten al is het ook}{rijst met torretjes}\\

\haiku{Ik heb er een uur.}{op je staan wachten en ben}{je nageloopen}\\

\haiku{Op mijn vijfde jaar,,!}{las ik vlug en wat meer zegt}{had er plezier in}\\

\haiku{Want ik dorst er mijn}{vader niet over te klagen}{omdat ik vreesde}\\

\haiku{{\textquotedblright} En als de koning.}{in zijn humeur is dan geeft}{hij ze een sigaar}\\

\haiku{Zelfs Hofdijk niet, bij.}{wien toch ook   het vuur van}{het ijzer spatte}\\

\haiku{bont en blauw is 'n,!}{e-pleonasme omdat}{in bont al blauw zit}\\

\haiku{Het gevolg was, dat.}{hij dikwijls pas tegen half}{tien in zijn klas kwam}\\

\haiku{Hij was dat alleen,;}{dan wanneer hij zich in zijn}{eer getast voelde}\\

\haiku{{\textquoteright} Dan zat hij maar stil,}{bij de kachel en liet ons}{wat anders werken}\\

\haiku{Stel je voor dat ik.}{d\'at voor mijn brave vijfde}{niet zou over hebben}\\

\haiku{De eerste smeert men, '.}{over zijn boteram de tweede}{smeertm over de zee}\\

\haiku{Ik herinner mij:}{nog zeer goed zijn woedende}{kijken bij mijn vraag}\\

\haiku{Dit is zoo vaak het.}{lot van leeraren die eenmaal}{wanorde hebben}\\

\haiku{de minsten uit angst,.}{de meesten uit liefde en}{uit respect voor hem}\\

\haiku{Hoe is dat alles!}{tegenwoordig veranderd}{en   verbeterd}\\

\haiku{Hektors Liebe stirbt...{\textquoteright},.}{im Lethe nicht O ik weet}{het allemaal wel}\\

\haiku{Al heel gauw kwam het {\textquoteleft},!}{verzoekAber setzen Sie sich}{doch Herr von Santen}\\

\haiku{Een dezer dames.}{heeft mij een hoogst penibel}{oogenblik bezorgd}\\

\haiku{En ga dan zugleich,!}{beim Schuhmacher je Sjoenen}{zind niet heel frisch meer}\\

\haiku{Geheel alleen, en.}{keek heel vergenoegd toen de}{deur voor hem openvloog}\\

\haiku{Al het huisvuil werd.}{op de straat gesmeten en}{door hen verslonden}\\

\haiku{Aandoenlijk is de.}{toevoeging van den datum}{van zijn overlijden}\\

\haiku{{\textquoteleft}Nou dank ik je voor:}{je aandacht en er voor om}{verder te pennen}\\

\haiku{Dit is een van de,.}{weinige plaatsen waarin}{Perk over zijn werk spreekt}\\

\haiku{Te zamen schrijden!}{zij over de Regenboogbrug}{ter onsterflijkheid}\\

\haiku{Uit dat vervloekte.}{paperassengesnotter}{en -gesnater}\\

\haiku{Dan zou de eenige.}{gegronde grief tegen hem}{ondervangen zijn}\\

\haiku{Maastricht, Valkenburg,,,.}{Meerssen Simpelveld Rolduc}{en ten slotte Aken}\\

\haiku{Fons vertelde mij, '.}{dat dit int geheel niets}{buitengewoons was}\\

\haiku{{\textquoteleft}Dit is de reden.}{waarom Ibels en Lautrec zoo}{intens in trek zijn}\\

\haiku{Dan was hij even scherp.}{en geestig als hij in zijn}{brieven placht te zijn}\\

\haiku{Toch zag ik Herman.}{nog wel enkele malen}{op een criquetveld}\\

\haiku{In hem was niet de.}{dichter op wiens klankbord mijn}{snaren meetrilden}\\

\haiku{Van Eeden, die in,.}{de buurt woonde werd op een}{afstand gehouden}\\

\haiku{{\textquoteright} ~ Het is meer dan,!}{vijftig jaar geleden dat}{ik hem het eerst zag}\\

\haiku{Herinnert gij u, -!}{nog hoe mijn zoontje thans een}{deftige mijnheer}\\

\haiku{{\textquoteright} volgens den Rector,.}{die op mijn oefeningen}{het oog moest houden}\\

\haiku{Nausika\"a begon,:}{met haar blanke arm met de}{bal te spelen maar}\\

\haiku{je zit nou in de,,{\textquoteright}:}{lucht te kijken en denkt was}{ik maar op straat maar}\\

\haiku{{\textquoteleft}Pardon, meneer, ik {\textquotedblleft}{\textquotedblright}.}{zat te denken wat u met}{moreel bedoelde}\\

\haiku{3 cahiers heb ik,,.}{al af in een week gemaakt}{nu zijn er nog 9}\\

\haiku{Hij sprak ook altijd.}{met veel waardeering over Jou en}{Jacques en Herman}\\

\haiku{Ik zou mij heel erg.}{moeten vergissen als het}{niet Louise heette}\\

\haiku{{\textquoteleft}Meestal kan ik.}{niet meer dan 50 versregels}{per dag vertalen}\\

\haiku{Ik heb toen aan den ():}{man die er bij was geweest}{Van Vriesland gevraagd}\\

\haiku{N\'og zie 'k u, hoe,}{gij met uw forsch gelaat}{op eens verschenen}\\

\haiku{Hij heeft nooit in iets,,.}{geloofd of in iemand ja}{niet eens in zich zelf}\\

\haiku{{\textquoteleft}Dank je wel voor je.}{vriendelijke woorden in}{de NG over de Od}\\

\haiku{Dit is een vraag die.}{men wel voortdurend tegen}{Quack zelf kan richten}\\

\section{Felix Timmermans}

\subsection{Uit: Adriaan Brouwer}

\haiku{Mijn vader had nest.}{gemaakt in een arm straatje}{te Audenaarde}\\

\haiku{Ik heb mij moeten.}{inhouden terwille van}{mijn zondagsche kleeren}\\

\haiku{Het zweet liep van mijn.}{voorhoofd als ik het gerij}{hoorde naderen}\\

\haiku{Ik zal u misschien,,.}{nooit meer zien Isabel door de}{schuld van dien vetzak}\\

\haiku{Zoo wonderzalig.}{goed is dat. Voor vier stuivers}{koopt men den Hemel}\\

\haiku{En daar werd hier en.}{daar nog dapper gevochten}{tegen den Spanjool}\\

\haiku{Een vlammende klop,.}{onder mijn schouder en ik}{rolde de beek in}\\

\haiku{Als 't mijn beurt was,.}{om te zingen werd ik op}{luid bravo onthaald}\\

\haiku{Twee of drie dagen,,.}{nadien voor den noen werd er}{op de deur geklopt}\\

\haiku{Ik wou mij seffens,.}{troosten en wreken met aan}{Isabel te denken}\\

\haiku{Maar die gedachte.}{aan Primula Mia zat als}{een spin op mijn hart}\\

\haiku{Een uur gaans en ge,.}{staat boven op de duinen}{onder u de zee}\\

\haiku{In mijn hart dank ik,.}{den ouden Breugel ik dank}{Rubens en Rembrandt}\\

\haiku{Om d'een schilderij.}{achter d'ander op haren}{dotskop stuk te slaan}\\

\haiku{Hij eerbiedigde,.}{het zoo dat hij er nooit met}{een penseel aan titste}\\

\haiku{Gelijk een bie naar,!}{den honing riekt zoo wil ik}{naar de verf rieken}\\

\haiku{Dan kauwde hij als,.}{op kurk als op iets dat niet}{te verkauwen was}\\

\haiku{en blijven en niets ',.}{int buske leggen was}{om dood te vallen}\\

\haiku{Dan, met het boekje,.}{in mijn hand speelde ik den}{verliefden Floris}\\

\haiku{Van Someren wist:}{reeds dat ik het viezeke}{had uitgehangen}\\

\haiku{Als ze het speelden.}{geloofden ze voor zoolang}{dat ze het waren}\\

\haiku{Nu wijn en oesters,,!}{zalm en reebok die ge door}{een pijp kon zuigen}\\

\haiku{{\textendash} Verdomd, dat ik dat,,....}{wijf toch heb getrouwd of ik}{trok mee mee met u}\\

\haiku{Achter alle leed.}{glinstert er iets dat u van}{schoonheid zuchten doet}\\

\haiku{Op een avond was Joos '.}{nog niet thuis gekomen van}{t Zuiderkasteel}\\

\haiku{En al zingend kunt,.}{ge altijd iemand beter}{bezien dan anders}\\

\haiku{Toen riep Joos, blij om:}{al de eer die ik over zijn}{geitenhaar uitgoot}\\

\haiku{Tusschendoor hoorde.}{ik ook wat zij samen thuis}{over mij vertelden}\\

\haiku{Dat is als vloeken,.}{in een kerk of speeken in}{een wijwatervat}\\

\haiku{maar ik zeg u op,,....}{voorhand ik geloof er niets}{van heelemaal niets}\\

\haiku{Maar geen woord er over,.}{hij kon zwijgen lijk over die}{familiezaken}\\

\haiku{Ik kon er uren, soms.}{halve dagen en heelder}{avonden doorbrengen}\\

\haiku{{\textendash} Met uw werk kunt gij,.}{u zoo rijk maken als de}{zee diep is zei hij}\\

\haiku{Jordaens maakte het,:}{meest lawaai al kon hij dien}{bakker niet uitstaan}\\

\haiku{Komt vrienden luister.}{naar mijn lied wat er nu voor}{wonders is geschied}\\

\haiku{Door koopen en leenen.}{kan iedereen den fijnsten}{vogel uithangen}\\

\haiku{Zij geleek bijna.}{nog een kind en hij was al}{diep in de vijftig}\\

\haiku{Ik versta er geen,.}{kruimel van zoo'n geluk heeft}{nog niemand gehad}\\

\haiku{dat jaar heb ik in,.}{groote klaarte des harten hard}{veel en goed gewerkt}\\

\haiku{Straks kwam dat verken,.}{zat en klam naar huis en zat}{zij met den afschuw}\\

\haiku{Om door de liefde.}{weer altijd en immer meer}{menschen te kweeken}\\

\haiku{Wijs als altijd, liet.}{hij mij zonder verdere}{woorden weer alleen}\\

\haiku{Hij zou de trap niet.}{op gekund hebben zoo had}{hij zich bedronken}\\

\subsection{Uit: Anna-Marie}

\haiku{Als hij haar in 't;}{sterfhuis ontmoette kreeg hij}{een bots van liefde}\\

\haiku{Ik heb er met mijn;}{zuster Rachel zaliger}{altijd naar gewacht}\\

\haiku{Zoudt gij een nieuwe?}{Romeo en Julia op uw}{geweten nemen}\\

\haiku{Onder de poort, v\'o\'or ',.}{t commiezenhuizeke}{stond een hoopke volk}\\

\haiku{{\textquoteright} zei Pirroen ontroerd.}{en daar door verviel al zijn}{wrok voor Livinus}\\

\haiku{Hij hield haar mollig,}{handeken vast streek kalm een}{klisje krullend haar}\\

\haiku{Met half-Mei {\textquoteleft}}{zou zij met hare voedster}{te Brussel zijn in}\\

\haiku{hij hulde zich in,,:}{den rook knorde en ineens}{vroeg hij bevelend}\\

\haiku{de tijd scheen haar wel,.}{wat lang maar des te zoeter}{zou het geluk zijn}\\

\haiku{zij was verheven:}{boven alle woorden en}{hij zei met een zucht}\\

\haiku{eerst als ze voorbij.}{zijn huis waren konden ze}{doen wat ze wou\"en}\\

\haiku{Hij sloot er zijn oogen,.}{voor zuchtte en verachtte}{zijn triestig leven}\\

\haiku{Hij heeft veel verdriet,.}{gehad en nog en daardoor}{is hij een dichter}\\

\haiku{Maar de zoete stem,;}{bleef in haar oor hommelen}{vaag en broksgewijs}\\

\haiku{Een toevlucht waarnaar,.}{ze uit zag maar waaraan ze}{zich niet geven kon}\\

\haiku{Zijn bleek recht gelaat;}{stond als een bleeke bloem wazig}{in de schemering}\\

\haiku{Ik zag haar voor het.}{eerst toen ik te paard weerkwam}{van de vossenjacht}\\

\haiku{Alleen, alleen met,.}{mijn verlangen en ach de}{nachten zijn zoo lang}\\

\haiku{Toch moet gij eenmaal,.}{wederkomen waar gij ook}{dwaalt op vreemden grond}\\

\haiku{{\textquoteright} Na het lied kwam weer.}{de donkere stilte en}{hier en daar een zucht}\\

\haiku{Het was een flinke;}{man die gewoon was in hooge}{kringen te verkeeren}\\

\haiku{En terwijl ze dan}{naar het kleed zocht dat ze zou}{aandoen betrapte}\\

\haiku{{\textquoteright} Aanstonds begon ze,.}{zich te kleeden gejaagd en}{opgewonden}\\

\haiku{Maar hij zweeg lijk een.}{visch en kropte zijn woede}{en zijn verdriet op}\\

\haiku{misschien zou hij door '?}{t vertellen verlicht en}{verhelderd worden}\\

\haiku{Als g' het niet doet... '....}{ist een teeken dat ge}{mij niet gaarne ziet}\\

\haiku{naar {\textquoteleft}den Eenhoren{\textquoteright}.}{moest komen en deze haar}{laatste brief zou zijn}\\

\haiku{waar niets te zien was.}{dan een wakende haan met}{zijn schuwe kiekens}\\

\haiku{{\textquoteleft}Als men er eenen keer,}{van geproefd heeft kan men er}{niet meer afblijven}\\

\haiku{{\textquoteright} Bij elke {\textquoteleft}knol{\textquoteright} ging,.}{er een vuist weg tot er die}{van Pirroen overschoot}\\

\haiku{{\textquoteleft}Mij gebroken voor,...{\textquoteright} {\textquoteleft}???}{u mijnen kop in den schoot}{gelegdHoe wie wat}\\

\haiku{en voelde met een.}{zekere spijt dat met haar}{weer alles goed kwam}\\

\haiku{Hij verzwijgt zijn plan.}{en zou het aan iedereen}{willen vertellen}\\

\haiku{En buiten in den,.}{nacht hoort hij het zoetekens}{malzig regenen}\\

\haiku{In zulken roes schrijf,.}{ik dan naar u en dat duurt}{lang mijn geliefde}\\

\haiku{Wat is mij pijn en!}{ellende als mijn hart maar}{licht van vrede is}\\

\haiku{Moet ik dat met bloed?}{en vuur schrijven om U dat}{doen te begrijpen}\\

\haiku{{\textquoteright} {\textquoteleft}Als er een van ons,.}{gedrie\"en niet sterft worden}{wij slechte menschen}\\

\haiku{Het beekje gaf soms.}{een zilveren geluidje}{in de stilte}\\

\haiku{Onderwegen bidt;}{ze verstrooid dat hij het niet}{zou gezien hebben}\\

\haiku{Maar 'k doe het niet, ',.}{neenk doe het niet al stond}{alles op zijn kop}\\

\haiku{Ge weet zelf wat ik......{\textquoteright}}{zeggen wil toen ik uwe hand}{in de zijne zag}\\

\haiku{{\textquoteleft}Weet ge 't nog dat?}{ik u vroeger eens gevraagd}{heb om te trouwen}\\

\haiku{{\textquoteright} Zijn voorhoofd was nu.}{heelemaal als bestikt met}{dikke zweetperels}\\

\haiku{Hij ging naar haar toe,,.}{de armen open gereed om}{haar te omhelzen}\\

\haiku{Met dezen dronk dus,,!}{beste vrienden zeg ik u}{vaarwel en adieu}\\

\haiku{Dat was nu heel veel,}{jaren geleden en hij}{vond het later}\\

\haiku{De doktoor trok zijn.}{linkermondhoek in rimpels}{en schuddebolde}\\

\haiku{Dat kon hij zich zelf,?}{niet voorstellen en dat vroeg}{hij immers ook niet}\\

\haiku{Hij ging bij Pirroen, '.}{niet meer eten kwams avonds niet}{meer in den Dolfijn}\\

\subsection{Uit: Boerenpsalm}

\haiku{Een boer moet een boer,.}{blijven anders verstopt de}{gang van de wereld}\\

\haiku{Zij sprong op voor haar,.}{kleed en haar boterhammen}{vallen op den grond}\\

\haiku{Ik had noch rust, noch, '}{duur en als ik het gedaan}{kon krijgen trok ik}\\

\haiku{Daar zitten haar broers,,,.}{wel vijf en haar pere een}{vent lijk een pilaar}\\

\haiku{Mijnheer pastoor zegt. '}{dat de sterren zoo groot als}{wereldbollen zijn}\\

\haiku{Het leven is geen,,.}{lach zei hij maar uw varken}{is een binnenbeer}\\

\haiku{Ik hield mijn hart in}{mijn handen en ik vergat}{de overstrooming}\\

\haiku{Ze houden u arm.}{en metselen u in een}{toren van kommer}\\

\haiku{Voor een millioen,}{wilt ge er geen  enkel}{kwijt ge zoudt er voor}\\

\haiku{emmers zweet, blaren,.}{op uw handen korstknie\"en}{en later een bult}\\

\haiku{Toen gaf ze mij een,.}{porseleinen koffiepot}{nog een trouwcadeau}\\

\haiku{- Hoort ge Franelle,.}{de Wortel wil mij  voor}{een dief doen doorgaan}\\

\haiku{Van alle kanten.}{loert het leven om u een}{pee te steken}\\

\haiku{En daar hoor ik hen (!)....}{vertellendat ik daar juist}{moest op uitkomen}\\

\haiku{Zijn schoone ziel, die,.}{in zijn woorden brandde heeft}{mijn hart opengedaan}\\

\haiku{Wortel, omdat gij.}{mij meer betrouwt buiten den}{biechtstoel dan er in}\\

\haiku{Ik probeer met de,.}{maan altijd goed te staan ge}{moet haar leeren kennen}\\

\haiku{- Zijt ge nu zot, riep,!}{ze tegen ons Fien van daar}{zoo naar staan te zien}\\

\haiku{Ons Fien kunt ge maar.}{niet wijs maken dat het de}{schuld van de maan is}\\

\haiku{We hebben alles,.}{geprobeerd beewegen en}{medicamenten}\\

\haiku{Als 't koren dan '.}{eindelijk in zijn schooven staat}{ist dorpskermis}\\

\haiku{Weer thuis, gaat onze,.}{Fransoo de hoeven af met}{zijn bedelzaksken}\\

\haiku{We doen ons verken,,}{dood en ge weet niet hoe hij}{het weet maar mijnheer}\\

\haiku{Vooreerst hebt ge ze.}{alles gegeven wat in}{uw vermogen was}\\

\haiku{Hun geluk is het,.}{uwe hun verdriet snijdt dieper}{bij u dan bij hen}\\

\haiku{Als er een pan van ',,.}{t dak valt wees gerust ze}{valt op mijnen kop}\\

\haiku{En toch bleef ik soms.}{tuschen de spleet van de deur}{naar haar staan loeren}\\

\haiku{Ik kon het niet meer:}{houden en ik riep als voor}{een groote zaal vol volk}\\

\haiku{Haar vent lag daar als,.}{een versleten keerborstel}{zat en lam te bed}\\

\haiku{'t Is uw eigen '.}{bloed ent roept nog harder}{als het tegenslaat}\\

\haiku{Is dat uwe smart, o,?}{Heer die zoo zwaar is dat wij}{moeten meehelpen}\\

\haiku{Want ik denk op de.}{mijne meer dan op die van}{U. Vergeef het mij}\\

\haiku{Het is zoo schoon, en ',.}{t is zoo stil dicht bij en}{heel in de verte}\\

\haiku{Als het regent en.}{de vruchten blinken en ge}{uitlekt als een hond}\\

\haiku{Ik vertel haar heel.}{mijn wedervaren tot aan}{het zien van het lijk}\\

\haiku{Maar ook daarom wordt.}{zij van Onzen Lieven Heer}{zoo gaarne gezien}\\

\haiku{Ja, het was toen de,.}{verschrikkelijkste zomer}{dien ik gekend heb}\\

\haiku{In de Nethe kon.}{men bij hooge tij wat stinkend}{slijkwater halen}\\

\haiku{Een liter water.}{had bijna zooveel waarde}{als een liter melk}\\

\haiku{Maar 't was of de '.}{lucht ent oor van God ook}{uitgedroogd waren}\\

\haiku{Hewel ik ben niet,!}{bang ik wil eens laten zien}{dat ik een man ben}\\

\haiku{Ik voel rond mij een,.}{vreemde macht die het op ons}{leven gemunt heeft}\\

\haiku{En toen heb ik ook:}{mijnen kop gebogen en}{gelaten gezegd}\\

\haiku{Ze mocht niets zeggen.}{van den doktoor en niemand}{mocht haar iets vragen}\\

\haiku{Wat baatte het dat,?}{zij zweeg om een of twee uur}{langer te leven}\\

\haiku{Bij een onweer had,.}{ze altijd geren dat ik}{thuis was dicht bij haar}\\

\haiku{Och, hoe kon ik zoo'n.}{goed mensch eens vergeten voor}{die meid met den stier}\\

\haiku{Dan voelt men dat men.}{oud wordt en het leven als}{een smoor voorbijgaat}\\

\haiku{Nu wist ik het, dien.}{Jesus zou ik op het graf}{van ons Fien zetten}\\

\haiku{Zulk werk is goed voor,.}{mannen als onzen Fransoo}{den minderbroeder}\\

\haiku{- Ik geloof het ook,, -.}{zeg ik dan en ik ben er}{bijna zeker van}\\

\haiku{Ja, dat was veel schooner.}{en als men het fijn naging}{ook zoo moeilijk niet}\\

\haiku{Ik was zelf verbaasd,.}{over mijn verzinsel alsof}{het niet van mij kwam}\\

\haiku{Ze kwam voorbij om,.}{een rok te halen dien ze}{gerepareerd had}\\

\haiku{Gelukkiglijk zijn.}{dat de laatste kuren van}{Mijnheer de Winter}\\

\haiku{Frisine geeft zich.}{meer moeite voor den wasch en}{voor het huishouden}\\

\haiku{Ik lig soms op  ,,.}{de loer in huis of buiten}{ik achtervolg haar}\\

\haiku{en die andere.}{zorg van achterdocht was er}{niet eens mee verlicht}\\

\haiku{Die stilte en dit!}{zwijgen er rond maken mij}{nog wanhopiger}\\

\haiku{Ge kunt voelen hoe.}{laf en plat die woorden bij}{mij er uit kwamen}\\

\haiku{Maar 's anderdaags}{zat mijn hof vol slekken en}{bij den Ossenkop}\\

\haiku{- Als die eerst bij mij,.}{waren zult gij ze er wel}{opgezet hebben}\\

\haiku{maar ze zou 't nu,.}{niet doen daar was ze immers}{te geslepen voor}\\

\haiku{Zoo lag ze daar vier,.}{dagen te blaken zonder}{\'e\'en woord te zeggen}\\

\haiku{Wat is er dan aan,?}{mij aan dat de vrouwen mij}{zoo gaarne hebben}\\

\haiku{Ik doe mijn oogen toe,.}{als ik op Angelik denk}{om haar niet te zien}\\

\haiku{Ik zat in den stal,.}{toe te zien van tusschen de}{pooten van ons koei}\\

\haiku{of er iemand meer....}{is of minder en of er}{iets in den weg staat}\\

\haiku{- Ge kunt den pastoor,,....}{zeggen zuchtte ze dat ik}{toestem op uw vraag}\\

\haiku{Ik moet het toch  ,?}{doen waarom dan leelijke}{gezichten trekken}\\

\haiku{Z'is blind, en 't is.}{misschien daardoor dat ze zoo}{goed en vroolijk is}\\

\subsection{Uit: Driekoningentryptiek}

\haiku{- Savez-vous?}{pourquoi nous avons tout donn\'e}{\`a ces pauvres gens}\\

\haiku{En of ge dat aan,.}{arme menschen doet of aan}{God dat is eender}\\

\haiku{Elle arrivait.}{tout droit sans se soucier des}{haies ni des chemins}\\

\haiku{Il prit l'\'etoile.}{et suivit l'enfant nimb\'e}{d'arc-en-ciel}\\

\haiku{Doch hij zag nog eens,.}{om naar zijn schapen die zoo}{meewarig blaatten}\\

\haiku{{\textquoteright} Suskewiet deed het,,.}{hek open en allen volgden}{dicht bijeen gedrumd}\\

\haiku{A force de les,.}{saluer chaque jour il les}{connaissait toutes}\\

\haiku{lui qui autrefois,.}{vivait au jour le jour plein}{de contentement}\\

\haiku{ni trop br\^ulant ni trop,.}{lourd il ne le d\'edaignait}{pas et l'emportait}\\

\haiku{Schrobberbeeck was nu,,.}{een heel andere vent van}{binnen in zijn hart}\\

\haiku{Hij had nu geen schrik,.}{meer hij verlangde nog naar}{zoo'n hooge momenten}\\

\subsection{Uit: De familie Hernat}

\haiku{{\textquoteright} en Mevrouw sloot de.}{oogen om even in een verren}{droom te verzinken}\\

\haiku{En hij hoorde de,!}{klanken het heerlijkste lied}{van zijn leven}\\

\haiku{En ik heb daar nog!}{een hemelbier alleen voor}{de beste vrienden}\\

\haiku{Het was een felle,,.}{roekelooze liefde die ze}{voor Stefan voelde}\\

\haiku{Hij dierf niet in de.}{klare oogen van Henriette}{te voorschijn treden}\\

\haiku{Toen slikte Mevrouw,,.}{Lorier-Frisijn zich offerend}{haar verdriet weer in}\\

\haiku{Als de Baron iets,.}{vertelde was er geen speld}{tusschen te krijgen}\\

\haiku{Het ligt soms aan zoo,.}{weinig tenminste zoo van}{buiten af gezien}\\

\haiku{Wat een comedie....?}{is er nu eigenlijk bij}{de Loriers gespeeld}\\

\haiku{In dat water had,.}{ze eens gewenscht Henriette}{te zien verdrinken}\\

\haiku{En bij drink-.}{en smulpartij schoot hij steeds}{den hoogvogel af}\\

\haiku{Dan huiverden de.}{haren op zijn handen recht}{van ontsteltenis}\\

\haiku{Doch hij riep het niet,.}{tegen God niet tegen het}{Lot of het Leven}\\

\haiku{{\textquoteright} {\textquoteleft}Dan zal ik er mijn,,.}{broeder lichtaanbrenger ten}{zeerste om danken}\\

\haiku{Haar vader had in.}{het begin opgespeeld en}{haar afgerammeld}\\

\haiku{Ze wou alleen maar.}{zijn lief zijn en daar was ze}{al gelukkig mee}\\

\haiku{Anders loopt het scheef.}{en ze sollen met u als}{een kat met de muis}\\

\haiku{Het zien van al die.}{dingen was voor hem reeds een}{rijkdom op zichzelf}\\

\haiku{En terug aan de,.}{feesttafel zat Annette}{tegenover Simon}\\

\haiku{Men wierd gewaar, dat.}{zij op het kasteel veel ging}{te zeggen krijgen}\\

\haiku{Wie had dat ooit voor,?}{mogelijk gedacht dat ik}{met u zou trouwen}\\

\haiku{Onze Turk is nog,.}{zoo onnoozel niet als wij ons}{voorgesteld hebben}\\

\haiku{Ruytenbroeckx was een.}{man van kennis in het fruit}{en stoefte er mee}\\

\haiku{Maar de verbazing,....}{verlamde hem toen zij zijn}{hand aan haar borst bracht}\\

\haiku{Zelfs Ruytenbroeckx had.}{er toen nog niet het minste}{vermoeden van}\\

\haiku{Maar Lucie sneed en.}{kerfde heftig zijn flauwe}{beloften kapot}\\

\haiku{Hij wou die {\textquoteleft}zaak{\textquoteright} van.}{Anna-Lise tot in}{de puntjes weten}\\

\haiku{Hij pinkte schelmsch:}{naar haar opgeroepen beeld}{en zei tot zichzelf}\\

\haiku{Hij kleedt zich snel, werpt.}{voor de gelegenheid zijn}{witte burnoes om}\\

\haiku{{\textquoteright} vroeg ze aan Simon,.}{die in het midden van de}{leeszaal beeldstijf stond}\\

\haiku{De knechten hebben}{hem in den vroegen morgen}{hooren wegrijden}\\

\haiku{Anna-Lise,,?}{hebt ge er geen spijt van dat}{hij weggegaan is}\\

\haiku{Anna-Lise.}{kwam bij hem na een boodschap}{te Nivesdonck}\\

\haiku{Lucie zat op wraak!}{te zinnen en die wraak zou}{niet voor de poes zijn}\\

\haiku{En het ergste dat.}{ze overal den lof hoorde}{van die indringster}\\

\haiku{{\textquoteleft}Wij mogen dat de.}{nagedachtenis van zijn}{vader niet aandoen}\\

\haiku{Ik heb u daar juist.}{toch al gevraagd of ze u}{niets gezegd hebben}\\

\haiku{Van uw dochterke?}{Annette en den jongen}{uit De Regenboog}\\

\haiku{Ik ben daar juist aan,.}{Sidonie gaan zeggen dat}{ik niet kan komen}\\

\haiku{{\textquoteleft}Stommerik, seffens.}{komt hij nog met kar en paard}{binnengereden}\\

\haiku{De \'e\'ene beloert.}{den andere om hem in}{de klem te krijgen}\\

\haiku{En het kwam omdat {\textquoteleft}{\textquoteright}.}{de lange Vereecken geen}{spraak in zijn oogen had}\\

\haiku{En na lang zwijgen,,,:}{zei de rentmeester eenigszins}{smalend tot Simon}\\

\haiku{Ga twee dagen op,,}{reis tegen dat gij weerom}{zijt zit zij terug}\\

\haiku{Nu versta, begrijp....}{en vergeef ik ook de fout}{van uw dochterken}\\

\haiku{Gun die hyena's!}{toch het genoegen niet dat}{zij over mij juichen}\\

\haiku{Het bloed steeg hem met,.}{gulpen terug naar het hoofd}{rood om te bersten}\\

\haiku{Hij keerde zich om,,.}{verbaasd verbolgen om den}{durf van Philipien}\\

\haiku{Ze kwamen terug....}{uit den tabakswinkel met}{vier kistjes Havanas}\\

\haiku{{\textquoteright} en na {\textquoteleft}aux fonds des{\textquoteright}.}{bois een rits van zijn duim over}{de mandoliensnaren}\\

\haiku{Doch zij zou alle,....}{moeite doen desnoods om het}{toch te overbruggen}\\

\haiku{Zij vertelde over,,.}{zijn kunde over zijn talent}{zijn galanterie}\\

\haiku{Zouden er dan soms,?}{twee Stella mia's zijn twee}{Anna-Lise's}\\

\haiku{{\textquoteright} Bij de woorden van.}{Isabella was het koud zweet}{hem uitgebroken}\\

\haiku{Na lang zoeken vond '.}{hij hem aant wandelen}{op den Nethedijk}\\

\haiku{{\textquoteright} ~ Als hij in den,:}{avond naar huis slenterde zei}{hij kwaad tot zichzelf}\\

\haiku{Hij verstoutte zich,.}{zelfs een goeden dag aan de}{boeren te zeggen}\\

\haiku{Er was ineens naar,....}{haar een begeerte die zijn}{rust omverspoelde}\\

\haiku{{\textquoteright} En daarna ging de.}{lange Vereecken weer in}{zijn eigen kantoor}\\

\haiku{Doch ik kreeg de eer.}{en het geluk niet meer U}{nog te ontmoeten}\\

\haiku{Ge weet Verhaegen.}{doet zoo iets zeer handig en}{is daarbij niet duur}\\

\haiku{Rentmeester Adriaen, die,.}{riep de bulten saam al om}{te tribunalen}\\

\haiku{Van verder dan hier,{\textquoteright}.}{is het schot niet gekomen}{beweerde Simon}\\

\haiku{Er gingen eenige.}{schokken doorheen het lichaam}{van Adriaen Ruytenbroeckx}\\

\haiku{Hij ging voor de deur,:}{staan met de armen uiteen}{en riep wanhopig}\\

\haiku{En Charobin van.}{zijn kant reed nu en dan eens}{naar St. Rochushof}\\

\haiku{Geef dat al maar hier,,.}{cher Ange ik houd dit al}{bereid op mijn hart}\\

\haiku{Ik zal, dunkt me, zoo....}{gerust zijn onder den grond}{met die dingen aan}\\

\haiku{{\textquoteright} Octavie had ook,.}{liefde gekend vroeger op}{een ander kasteel}\\

\haiku{Hij zal het zelf zijn,,....}{die met de vrouwen speelt lijk}{de kat met de muis}\\

\haiku{Gij hebt gelijk, mijn.}{wil en mijn kunst zijn er de}{prooi van geworden}\\

\haiku{Hij rook seffens de.}{bedoeling van barones}{Emma de Vara}\\

\haiku{{\textquoteright} en hij zit er een....}{heelen avond in gedachten}{achter te zoeken}\\

\haiku{Er wierd iets wakker,,.}{in hem een nieuw bestaan een}{nieuwe horizont}\\

\haiku{De schrik is hem op,.}{het hart geslagen zegt men}{in Nivesdonck}\\

\haiku{{\textquoteright} vroeg hij en met elk.}{woord voelde hij zich rooder}{en rooder worden}\\

\haiku{Doch kom morgen niet.}{af met het geval van die}{vlek en dien jongen}\\

\haiku{dan ontbrak er iets.}{aan den gang der natuur en}{de overlevering}\\

\haiku{Het was dan of die.}{honderden wezens voor den}{tweeden keer stierven}\\

\haiku{Karnol ging van deur.}{tot deur aan den Heikant om}{te laten bidden}\\

\haiku{{\textquoteleft}Heer, blijf bij ons, de, '....}{avond nadert de groote schaduw}{strekt zich overt land}\\

\haiku{hij moest op zijnen,.}{alleen zijn met zijn verwijt}{en verdriet alleen}\\

\haiku{Het kind gaat aan de.}{bank staan met het hoofd in zijn}{handen geborgen}\\

\haiku{ze geweest was in....}{al die voorbije jaren van}{groote zorg en kommer}\\

\haiku{Het meisje kreeg daar.}{koffie met boterhammen}{en speculatie}\\

\haiku{{\textquoteright} vroeg Verhoeven, die.}{toch nog een halve flesch m\'e\'er}{had moeten drinken}\\

\haiku{En Karel-Jan ging:}{zijn hoofd opheffen om te}{roepen naar dien lach}\\

\haiku{Met de intresten,;}{kwam men toch niet meer toe om}{er van te leven}\\

\haiku{En hij straalde ook,.}{omdat het met baas Pittoors niet}{goed ging in de zaak}\\

\haiku{Ook de engelsche,.}{gouvernante bleef zonder}{betaald te worden}\\

\haiku{Dat kasteel was van,}{nonkel Arnold en niemand}{anders dan nonkel}\\

\haiku{{\textquoteleft}Ja, God leidt ons,{\textquoteright} zei,.}{Karel-Jan overtuigd van de}{leuze van hun huis}\\

\haiku{Voor wat dienen die?}{dingen anders dan om ons}{geluk te geven}\\

\haiku{{\textquoteright} {\textquoteleft}Geen gebabbel,{\textquoteright} zei.}{meneer Verhoeven zoo wat}{van uit de hoogte}\\

\haiku{Roselie, die een,.}{weduwe was zou daar zeer}{mee in haar schik zijn}\\

\haiku{Ze placht nu ook veel.}{te bidden voor haar moeders}{zielezaligheid}\\

\haiku{Zoo zal hij eeuwig,:}{blijven leven en kan een}{legende worden}\\

\haiku{{\textquoteleft}Ik slaap nooit meer of.}{die kerel moet eerst v\'o\'or mijn}{voeten dood liggen}\\

\haiku{Zijn moeder vond hem.}{zoo schoon in zijn berouw en}{zelfbeschuldiging}\\

\haiku{Verder wandelde.}{hij veel door de velden en}{las over de sterren}\\

\haiku{Hij zag een nieuwen,.}{grooten plicht een nieuwe taak}{voor zijn oud leven}\\

\subsection{Uit: Ik zag Cecilia komen}

\haiku{Twee ranken, die slechts.}{kunnen opbloeien als ze}{malkander steunen}\\

\haiku{Een maneschil schijnt,.}{in het gladde water als}{een verdronken kroon}\\

\haiku{Een groote vogel wiekt.}{zwaar op en blijft mij schuins van}{uit een boom bezien}\\

\haiku{Ik scheur mij angstig.}{uit die mijmering los en}{vlucht naar beneden}\\

\haiku{Een ding weet ik, ik,.}{ga mij overtuigen ik ga}{mijn liefde redden}\\

\haiku{De slinger van de.}{hangklok flitst telkens in een}{zonnepijltje op}\\

\haiku{De kelderdeur staat,.}{open zware holleblokken}{komen de trap op}\\

\haiku{ik gaf haar schoentjes.}{van maneschijn en sluiers}{van regenbogen}\\

\haiku{Ik voel mij zoo klein,.}{en zoo nietig en onze}{liefde is zoo groot}\\

\haiku{zij kan zondag niet,.}{komen hare moeder is}{zwaar ziek geworden}\\

\haiku{Mijne dagen zijn.}{een mengeling van kwelling}{en verheuging}\\

\haiku{Om mijn opstijgend.}{geluk wetens en willens}{te vernietigen}\\

\haiku{- Maar ik even veel van,,.}{u meer dan van Roelinde}{meer dan van allen}\\

\haiku{Ik zal gelukkig.}{zijn en niemand zal weten}{waarom en waardoor}\\

\haiku{De hofgracht onder.}{hem is als een  effen}{laken van het kroes}\\

\haiku{Zoo kan ik in elk.}{geval de naderende}{dingen uitstellen}\\

\haiku{Niet zoo zeer voor mij,,;}{want ik zal haar niet meer zien}{maar voor haar zelve}\\

\haiku{Doch hij is nog geen.}{honderd met er gegaan of}{angstig keert hij weer}\\

\haiku{- Dat is van geluk,,...}{stamel ik ik ben zoo blij}{u nog eens te zien}\\

\haiku{Want ieder krijgt een.}{nummer van het dienstmeisje}{als hij in huis komt}\\

\haiku{Ik sta van achter.}{de boomen naar het huis en}{naar het kruis te zien}\\

\haiku{En het is meteen.}{alsof ik een lied om het}{dak hoor ruischen}\\

\subsection{Uit: Karel en Elegast}

\haiku{Hij praatte met zijn;}{edellingen en Bisschoppen}{die rond hem waren}\\

\haiku{Hij was ijler dan;}{den adem van een miertje en}{dunner dan de lucht}\\

\haiku{Gaat uit stelen al '.}{ist u nog zoo bitter}{en onaangenaam}\\

\haiku{en loste zich op.}{in een manestraal die}{door het venster stond}\\

\haiku{De Koning meende,;}{buiten te gaan maar zijn angst}{mocht nog niet over zijn}\\

\haiku{hij steeg vlug te paard.}{en rende naar beneden}{naar den platten grond}\\

\haiku{{\textquoteleft}Dat is iemand die.}{zijn weg is kwijtgeraakt en}{hier verloren loopt}\\

\haiku{Ik heb liever dat.}{wij vechten dan dat ik door}{dwang iets zeggen zou}\\

\haiku{De vier ademen van.}{mensch en dier zoefden als de}{wind om een hoog huis}\\

\haiku{{\textquoteright} En daarom sloeg de, '.}{Koning niet weerom en was}{t gevecht gedaan}\\

\haiku{BEIDEN stonden stil,.}{maar menigvuldig waren}{hunne gedachten}\\

\haiku{Ik zal er u den.}{uitleg van geven als gij}{mij eerst uwen naam noemt}\\

\haiku{Is hij van zulke?}{macht dat gij niet anders dan}{bij nacht kunt rijden}\\

\haiku{ik zal u mijn naam.}{zeggen indien ik er u}{mee van dienst kan zijn}\\

\haiku{De schat is zoo groot,}{dat als wij er vijfhonderd}{pond van wegnamen}\\

\haiku{Het zou mij dus maar,.}{slecht vergaan als ik er u}{de weg moest wijzen}\\

\haiku{Het moest toch voor zijn,.}{goed zijn want anders zou God}{het hem niet zenden}\\

\haiku{En met het kruid in}{den mond luisterde hij of}{er van die dieren}\\

\haiku{Dan vreeze ik dat,}{er mij onheil overkomt dan}{twijfel ik er niet}\\

\haiku{wat zou de Koning?}{hier komen doen hier in de}{nacht en zoo verre}\\

\haiku{Zoudt gij dan de taal '?}{van een haan gelooven ent}{blaffen van een hond}\\

\haiku{Zie de mane is,.}{haast niet meer zichtbaar zij zakt}{achter de bosschen}\\

\haiku{tot aan hun sporen,.}{zal het bloed vloeien en bij}{Eggeric het eerst}\\

\haiku{Verders werden er,.}{in de gangen het hof en}{de zalen geplaatst}\\

\haiku{Doch de Koning wreef:}{over zijn zilveren baard en}{sprak tot Elegast}\\

\haiku{Ze ging recht zitten.}{en stak haar aanschijn buiten}{de legerstede}\\

\subsection{Uit: Het kindeken Jezus in Vlaanderen}

\haiku{En Maria zag over.}{het land en voelde tranen}{in de ogen komen}\\

\haiku{Zij zag de wereld.}{door haar geluk en alles}{juichte in haar}\\

\haiku{Waarom sprongen de?}{zilveren vissen telkens}{boven het water}\\

\haiku{{\textquoteleft}'t Is niets, 't is, '.}{niets ik heb het verdiend en}{t zal wel over gaan}\\

\haiku{{\textquoteleft}En,{\textquoteright} voegde ze er, {\textquoteleft};}{nadien bijgij zult Jozef}{gelukkig maken}\\

\haiku{Jozef zag hem, en.}{wilde henengaan om den}{pastoor te mijden}\\

\haiku{en daarom zei ik,}{bij me zelven daar vraag ik}{Jozef zelf eens naar}\\

\haiku{{\textquoteleft}Ja, iets beter dan, ';}{zij zelvek geloof dat}{ik haar zielken zag}\\

\haiku{{\textquoteleft}Die heeft nog iet van,{\textquoteright}.}{zijn voorouder David op}{de tong peinsde hij}\\

\haiku{Aan den horizont,.}{lagen witte wolken als}{sneeuwbergen gevlijd}\\

\haiku{ze draaide, alles.}{zou dan wel eens met Gods hulp}{ten rechte komen}\\

\haiku{Een grijze sluier.}{weefde zich rond de gulden}{blijdschap van haar hart}\\

\haiku{{\textquoteright} {\textquoteleft}Ach, Jozef, 'k weet,{\textquoteright}.}{het niet en haar handekens}{zochten de zijne}\\

\haiku{{\textquoteright} Jozef vertelde.}{hem zijn droeve reis en de}{toestand van Maria}\\

\haiku{Het in puin liggend.}{kasteel bezijds de dorpskom}{getuigde daarvan}\\

\haiku{{\textquoteleft}Filmene,{\textquoteright} gebood, {\textquoteleft}.}{hij de meidgeef hun elk een}{dikke boterham}\\

\haiku{{\textquoteleft}Ik geloof dat in.}{Bethle\"em grote dingen}{zullen gebeuren}\\

\haiku{Hij heeft zijn stemme,.}{gegeven de aerde}{is beroert geweest}\\

\haiku{En ach ik ben zo!}{klein en nietig en als mens}{van gener weerde}\\

\haiku{Ik zal dat van mijn!}{leven niet vergeten al}{word ik honderd jaar}\\

\haiku{Hij stak den haring.}{eens naar omhoog en trok er}{zonder meer van door}\\

\haiku{maar niettemin bleef.}{dit zatte woeste Herodeshoofd}{voor haar ogen hangen}\\

\haiku{'t Volk kwam uit de.}{huizen gelopen en klom}{op de heuvelen}\\

\haiku{Hij ging toen op een}{omgevelden olm staan en}{hief het wicht zo hoog}\\

\haiku{Hij gruwelde van.}{zich zelven en deed zijn ogen}{een stondeken toe}\\

\haiku{{\textquoteright} brulde Herodes, en.}{Sausisken kefte mee}{naar de ministers}\\

\haiku{Eenieder is er.}{vast van overtuigd dat het kind}{gedood moet worden}\\

\haiku{, maar een vraag van den.}{kapitein leidde me van}{het voornemen af}\\

\haiku{Groter als heel de,,!}{wereld en niets dan water}{altijd maar water}\\

\haiku{Ze waren daarstraks!}{zo blij geweest als ze de}{zee gewaarwierden}\\

\haiku{t Is spijtig dat,}{het avond is maar gij zoudt zien}{hoe uitgemergeld}\\

\haiku{{\textquoteright} Maar zij waren reeds.}{den bollewinkel in met}{den wind van achter}\\

\haiku{De uren sleepten zich.}{lui en rustvol door den dag}{die duren  bleef}\\

\haiku{Het gaf hem hogen,...}{moed en jeugd en lust tot veel}{en lang te werken}\\

\haiku{{\textquoteright} {\textquoteleft}'k Zal mijn hoofddoek,{\textquoteright}.}{dieper over mijn ogen trekken}{wedervoer Maria}\\

\haiku{daar zijn de hutten,;}{rond de kerk en ginder de}{brug over de Nethe}\\

\haiku{Hij wist niet meer wat,}{zeggen en terwijl hij met}{een roden zakdoek}\\

\subsection{Uit: Pieter Breughel, zoo heb ik u uit uwe werken geroken}

\haiku{Dat zonk allemaal, ;}{zuiver in zijn hart waar hij}{het zoet bewaarde}\\

\haiku{Pieter had er iets ':}{bijgeschreven en las het}{voor aant meiske}\\

\haiku{En Pieter zich rap '.}{gebukt en aant lachen}{lijk een waterval}\\

\haiku{Om een puntje aan,!}{den honger te slijpen laat}{ons een kaartje spelen}\\

\haiku{Hij ligt tegen dien.}{gespleten knotwilg aan den}{mestput te ronken}\\

\haiku{passeerden rap, rap,:}{voorbij snel gegroet door den}{gezamenlijken}\\

\haiku{Maar de Pad, die door,:}{een deurspleet loerde riep toen}{van achter de deur}\\

\haiku{Dat is de geest van'',.}{Trees uit het Belofte Land}{bibberde er eene}\\

\haiku{In groepkes trokken',,,.}{z er uit dicht bijeen het}{mes bloot dat beefde}\\

\haiku{een rond rood neuske,.}{en kikvorschenoogen puilden uit}{hunne spekbalken}\\

\haiku{s Zondags kreeg hij;}{er gelukkiglijk toch nog}{een zwaaiken spek bij}\\

\haiku{waternimfen en.}{hoornen van overvloed rond het}{kleurig landswapen}\\

\haiku{En ge moogt gij nog:}{zoo dik en zoo rijk zijn van}{hier tot aan de zon}\\

\haiku{De pater bad voor,:}{Pieters zielezaligheid}{en Pieter floot}\\

\haiku{''O malzig teeder '.''}{boelekent Is zoet met}{u te leven}\\

\haiku{Maar de pater bad, ''',.}{en Pieter floott Is zoet}{met u te leven}\\

\haiku{Ei ja, 't is waar,'', ''....}{voegde  hij er blij bij}{die had er geen aan}\\

\haiku{Toen hij zijn oogen weer,.}{open deed zag hij ginder een}{meisken aankomen}\\

\haiku{En nu ga ik de,.}{schilderijen zien die in}{de kerken hangen}\\

\haiku{Hij bezag haar als,}{een wonder hij zag hoe ze}{huiverde en heur}\\

\haiku{Later als ze dood,,,''.}{is och arme heb ik nog}{tijd genoeg dacht hij}\\

\haiku{Maar bij de Dikken ''''.}{was het spek in de pan en}{bier\'a volont\'e}\\

\haiku{En de geheime!}{Broederschap van Den Naakten}{Dolk zal u helpen}\\

\haiku{Van dezen morgen,!}{nog niets gegeten maar de}{liefde voor alles}\\

\haiku{''Als Veronica,....}{niet arm geweest was had ik}{haar niet gevonden}\\

\haiku{'t Verwonderde,.}{Pieter dat hij niet weende}{hij had er spijt van}\\

\haiku{Hij zou boer blijven, '.}{maar daarom dan eerst gezorgd}{nen boer te worden}\\

\haiku{Het juichende bloed.}{ratelde draaiputten in}{zijn lijf van geestdrift}\\

\haiku{Het was zoo stil als,.}{op pantoffels alsof er}{daar niemand leefde}\\

\haiku{Eens de brug over, moet ',.}{hij doornen koker die den}{vestingwal doorsnijdt}\\

\haiku{Volk en kinderen,....}{loopen mee nieuwsgierig naar}{het ongeluk}\\

\haiku{'' En den eene riep tot. ''}{den andere al wreeder}{en wreeder dingen}\\

\haiku{En was er daar ook,?}{geene Simon van Cyrenen}{die mee het kruis droeg}\\

\haiku{Maar toen zag hij juist. ''}{twee paters Franciscanen}{ginder voorbijgaan}\\

\haiku{Ten einde van de,,.}{gang achter een brandende}{kaars was De Nood Gods}\\

\haiku{Ik ben gisteren!}{pas uit mijn klooster van Gent}{hier aangekomen}\\

\haiku{'t Was 'ne lange,,, '.}{vinnige jonge pater}{metnen lach opzij}\\

\haiku{Hij teekende den;}{man die door de soldeniers}{meegenomen wordt}\\

\haiku{Baskwadder, aan 't;}{preeken om de menschen te}{kunnen verraden}\\

\haiku{''Als ik niet rap weg,.''}{ben langs de deur zwieren ze}{mij door de vensters}\\

\haiku{maar in zijn dapper,.}{oogen lag iets achterdochtigs}{smeekends en onvast}\\

\haiku{de goddelijke,.}{schoonheid van den mensch zien de}{schoonheid van alles}\\

\haiku{Jan Nagel bekeek,.}{het strak met zijnen blauwen}{verdrietigen blik}\\

\haiku{Ik vind hem zelf een,.}{van de beste schilders maar}{dat ziet gij nog niet}\\

\haiku{Wij vangen God met, ';}{ons verf lijknen heilige}{met zijn gebeden}\\

\haiku{dat ik hier ben met ',.}{nen jongen vriend die met hem}{wil kennis maken}\\

\haiku{ik kan dat ook met}{tooverkracht maar dat doe ik bij}{mijn eigen bloed niet}\\

\haiku{om de menschen te}{verschalken want eenzamen}{worden gauw verdacht}\\

\haiku{Als vaderke weet, ',.}{dat ik weer aant spelen}{ben dan dondert hij}\\

\haiku{In twee dagen is,!}{hij niet geweest en dan geen}{woord laten zeggen}\\

\haiku{en die knorhanen,;}{door de roode ziel van den}{koraalplant doorsopt}\\

\haiku{hij veegde er met,.}{zijnen duim over krabde er}{met zijn nagels aan}\\

\haiku{Want verbeelding had,.}{Jan Nagel niet en durfde}{hij ook niet hebben}\\

\haiku{De meid, die het bracht,,,.}{had roode dikke handen}{die naar melk roken}\\

\haiku{'t Zal rap Winter,,'';}{worden Menheer zei ze met}{een bevende stem}\\

\haiku{Toen hij over de brug,.}{in de stad ging verdwijnen}{zag hij nog eens om}\\

\haiku{Ik zal misschien wel.}{wekelijks hier werk komen}{halen en brengen}\\

\haiku{''Ge moogt zoo hevig,;}{uw gedachten op mij niet}{zetten Filleke}\\

\haiku{En hij was blij, dat.}{Filleke nu niet meer over}{liefde kon spreken}\\

\haiku{Hij hoopte, dat ze;}{nu haar gedachten van hem}{wel zou wegtrekken}\\

\haiku{nu had hij er spijt,;}{van haar niet altijd geerne}{te hebben gezien}\\

\haiku{Achter de ruitjes.}{van het schotelhuis zag hij}{Anneke loeren}\\

\haiku{maar zijn beenen waren.}{als lam van de lafheid die}{hij had uitgehaald}\\

\haiku{''Ik kan er niet van,'', ''.}{slapen snikte zeals ge}{zoo kwaad op mij zijt}\\

\haiku{''Als ge toch niet met,,''.}{mij trouwt kort het niet zei ze}{frank en uitdagend}\\

\haiku{Al kan ikzelf geen,;}{kunst maken ik kan er toch}{niet afgeraken}\\

\haiku{de weeren, die de;}{bergen kleeden en met licht}{en kleur versieren}\\

\haiku{, en de Brusselaars.}{zaten daar nog versch alsof}{ze moesten beginnen}\\

\haiku{Kom spoedig van de,!''}{taveerne want uw liefde}{is beter dan wijn}\\

\haiku{onderwerpen in '.}{overvloed en modellen zoo}{maar voort scheppen}\\

\haiku{Hij asemde de goeie,.}{lentereuken op na dit}{dwaas hard regentje}\\

\haiku{''Hoe spijtig, dat ik!''.}{toen mijnen teekenboek niet}{bij had riep Pieter}\\

\haiku{''Maar nu heb ik hem,!}{bij en hij moet zwaar worden}{van wat er in komt}\\

\haiku{Om zeven uren       ,.}{moest ze gewekt worden want}{ze wou vroeg thuis zijn}\\

\haiku{''O, Heer, vergeef mij.''}{de zonden die ik vandaag}{ga bedrijven}\\

\haiku{de goede wierden, ':}{uitgelachen ent volk}{deed de slechte na}\\

\haiku{Hij was blij als een '.}{kind dats morgens zijnen}{Sinterklaas verwacht}\\

\haiku{Het dorp was d' helft.}{verkleind door den oorlog en}{de plunderingen}\\

\haiku{maar zoo 't schijnt, is.}{die jongen ook leelijk aan}{zijn eind gekomen}\\

\haiku{Maar als ik er nog,:}{aan denk dan zie ik het nog}{v\'o\'or mij geschilderd}\\

\haiku{'' Zoo kwamen er soms,.}{prenten uit jaren nadat}{ze gemaakt waren}\\

\haiku{armen van zijn groote,,!}{langverwachte beminde}{de schilderkunst}\\

\haiku{Zoo leefde Pieter,.}{gelukkig knapperend van}{geestdrift in zijn kunst}\\

\haiku{op anderen tijd.}{vergaf hij het haar als een}{onnozel plezier}\\

\haiku{In uw hart steken '!''}{meer pijlen dan zwaarden in}{t hart van O.L. Vrouw}\\

\haiku{''Maar \'e\'en groot zwaard zal,!''}{uw hart doorboren en dat}{zwaard steekt er nu in}\\

\haiku{Maar Anneke heeft.}{haar novice-jaren}{bij u schoon gedaan}\\

\haiku{speel door, opnieuw, en, '....}{als ik van een ander hield}{dan wast nu uit}\\

\haiku{Maar 't was, alsof,.}{ze gewaar wierd dat ze haar}{wouen verschalken}\\

\haiku{En daar Pieter haar,.}{geen achterdocht wou geven}{kon hij niets redden}\\

\haiku{Misschien eenmaal weg,,:}{dat Hans er wel wat zou op}{vinden of anders}\\

\haiku{de zonnekloppers,:}{maar de kloekste kerels als}{er te werken valt}\\

\haiku{Daar, in 'nen hoek, wordt,}{gekaart en gedobbeld en}{v\'o\'or den brandenden}\\

\haiku{Of Eva blond of  , ','!}{zwart haar hadt kan mij niet}{schelen z had haar}\\

\haiku{Ja!'' zegt den hoogen rug,,.}{achterdochtig en bang maar}{gereed om te slaan}\\

\haiku{Haal een gans naar de,''.}{Braderij en een konijn}{beval Marieke}\\

\haiku{Tot Pieter over het.}{heimelijk groeiend verzet}{begon te spreken}\\

\haiku{De zon duwde al '.}{haar gloed als omt nijdigst}{op de wereld}\\

\haiku{Hij is kruis en munt,,.}{kop en letter allebei}{ineens tegelijk}\\

\haiku{''Omdat de wereld.}{is zoo ongetrouw daarom}{ga ik in den rouw}\\

\subsection{Uit: Pijp en toebak}

\haiku{Dat wisten ze ook,,.}{zoo van binnen maar kenden}{het niet van buiten}\\

\haiku{En hij smoorde met,.}{zijn oogen toe van de deugd uit}{de lange steenen pijp}\\

\haiku{Ze zat dikwijls te.}{peinzen en hoorde niet dat}{de moor overkookte}\\

\haiku{En mag hij dan in ',?}{t geheel niet meer werken}{menheer den doktoor}\\

\haiku{En hij had maar spijt.}{dat de burgemeester niet}{naar zijn kousen zag}\\

\haiku{{\textquoteleft}Als ik zoo zeker,?}{ben waarom wandel ik dan}{zoo onrustig hier}\\

\haiku{{\textquoteleft}Ons Net zal schreeuwen.}{van blijdschap als ik het haar}{seffens kom zeggen}\\

\haiku{Hij ging de stoep af,,.}{en hij ging naar de vensters}{als aangetrokken}\\

\haiku{Het konijn Susken.}{Andries had zijn konijnen}{in zijn huis zitten}\\

\haiku{Dat was goed tegen,.}{de dieven en hij had er}{dan meer plezier van}\\

\haiku{Ze vonden hem 's.}{morgens krinselend onder}{den Kruislievenheer}\\

\haiku{s Nachts droomde ik '.}{er van ent maakte mij}{heelemaal anders}\\

\haiku{{\textquoteleft}Fons, houd Emmaken,.}{maar goed bij u z'is zoo goed}{voor de kinderen}\\

\haiku{{\textquoteleft}Als g'het niet gaarne,, '.}{hebt geef mij dan een teeken}{t is eender hoe}\\

\haiku{Hij stapte buiten.}{in den dikken sneeuw en de}{venijnige kou}\\

\haiku{Hij schoot in 't dorp.}{nog eens met de gauwte De}{Roode Leeuw binnen}\\

\haiku{{\textquoteleft}Laat ons maar vroeg gaan,.}{slapen en lig morgen wat}{langer in uw bed}\\

\haiku{Tist, Tist, heb ik het}{altijd niet gezegd dat er}{spoken in dien boom}\\

\haiku{Neen zoo iets had ze,!}{van haren Tist nooit gedacht}{wat nen sterken Tist}\\

\haiku{nu zoo ne klaren,!}{nachtegaal kunnen vangen}{wat zou die deugd doen}\\

\haiku{Die Kristus doorboord, '.}{met zijn dolkt deed even zijn}{kloekheid wankelen}\\

\haiku{Ze vermagerde,.}{er van lijk een spiering maar}{bleef even modieus}\\

\haiku{{\textquoteright} De fotograaf plaatste.}{het toestel v\'o\'or het gelaat}{van Marjanneke}\\

\haiku{{\textquoteright} 't Was te hooren.}{dat ze haar vertelsel al}{dikwijls had gedaan}\\

\haiku{s avonds om negen,,.}{uur brengt Gust mijn verloofde}{deze bloemekens}\\

\haiku{Hij heeft er mij nooit,.}{iets van gezegd maar bij zijn}{dood bleek het alzoo}\\

\haiku{Want zij zijn het hart,.}{van mijn man en mijn eenigste}{goed op de wereld}\\

\haiku{En 's morgens wierd.}{zij wakker en zij lachte}{naar het mandeken}\\

\haiku{De lente was daar,.}{met al zijn genade van}{kleuren geur en licht}\\

\haiku{Eindelijk stond heel.}{het gevulde korfken te}{blinken in de zon}\\

\haiku{Dan hier en daar een ':}{zucht ent was de vorsch}{die geestdriftig riep}\\

\haiku{Daarmee vloog hij weg.}{naar een rustiger plaats in}{het balsemend woud}\\

\haiku{Fier en blij klopte,:}{hij aan den knotwilg waarin}{de vleermuis woonde}\\

\haiku{- Haasje zijt gij dat,?}{die daar zoo behagelijk}{te knabbelen zit}\\

\haiku{- Ewel Eekhorentje,,}{als ik klimmen kon zooals gij}{en niet zoo bang was}\\

\haiku{zooals ik, dan zou ik.}{op den muur kruipen en het}{zoo te zien krijgen}\\

\subsection{Uit: Schemeringen van de dood}

\haiku{Felix Timmermans}{Schemeringen van de dood}{Colofon}\\

\haiku{wij wisten niet eens -.}{van wat dat wij er nooit zijn}{in wedergekeerd}\\

\haiku{Vroeger stond er links,.}{ook een maar de bliksem had}{het er afgehaald}\\

\haiku{Het was alsof ik.}{daar ver van afstond en dit}{zag als in een droom}\\

\haiku{Mij leek het dat er.}{onder het vlees een dunne}{klaarte zichtbaar werd}\\

\haiku{Zij opende groot de.}{ogen en zag met een goede}{blik de kamer rond}\\

\haiku{Het leek mij dat ze}{luideloos vertelden wat}{hij in zijn hoofd wist}\\

\haiku{{\textquotedblright} En vaders handen...}{kwamen bijeen en vouwden}{zich samen vol dank}\\

\haiku{het was alsof de!}{muren een adem hadden die}{door mijn ziele ging}\\

\haiku{Een voorzichtige,....}{belletik klonk in de gang}{nog een en nog een}\\

\haiku{mij staan in een groot,.}{verblindend licht dat mij de}{ziel verzengelde}\\

\haiku{mijn groet, en Mina,...}{liet op mijn ogen rusten een}{koude lange blik}\\

\haiku{Het is daar dat de.}{artiesten van alle slag}{komen luisteren}\\

\haiku{O Herman, ik ben,,!}{zo bang van die kelder doe}{hem toe doe hem toe}\\

\haiku{Maar na een wijle,:}{fluisterde ze als bevreesd}{het uit te spreken}\\

\haiku{Er was beneden;}{steeds een stilte als in een}{huis zonder mensen}\\

\haiku{{\textquoteleft}Dan kunnen wij met.}{hoop de lente afwachten}{die aanstaande is}\\

\haiku{'t Lag in de blik, '.}{van haar ogen ent rilde}{door gans haar wezen}\\

\haiku{Maar hoe groot was mijn!}{verwondering toen ik de}{kelder open zag staan}\\

\haiku{De ogen werden groot,.}{en wild met veel wit rond de}{donkere kinnen}\\

\haiku{{\textquoteleft}Kom, Mina,{\textquoteright} zei ik, {\textquoteleft},,;}{bedarendkom wees rustig}{leg me uw hart bloot}\\

\haiku{Treurwilgen leekten.}{hun moedeloze twijgen}{over zwarte zerken}\\

\haiku{Lag de grafmaker,?}{daar misschien zelf dood zonder}{dat iemand dat wist}\\

\haiku{Het verwonderde.}{mij dat er geen doden meer}{werden aangezegd}\\

\haiku{Ik  kleedde mij.}{voorzichtig en zachtekens}{sloop ik naar de deur}\\

\haiku{Het scheen of heel de.}{wereld in een ijzeren}{stilte lag gekneld}\\

\haiku{De vrees haakte in!}{mij en de onnozelheid}{spotte in mijn oor}\\

\haiku{Hoe meer ik van het,.}{graf afweek hoe dichter ik}{de dood naderde}\\

\haiku{Bos noch toren deed;}{verhopen dat er ginder}{nog mensen leefden}\\

\haiku{Omdat het mij zo,.}{benauwd werd begon ik naar}{gerucht te zoeken}\\

\haiku{Het lied zwol en boog,.}{en golfde weg en weer als}{het lied van de zee}\\

\haiku{De voet alleen der.}{schone vaas stond nog in het}{midden der tafel}\\

\haiku{{\textquoteleft}Ja, ik weet het,{\textquoteright} zei.}{ze gelukkig en ze trok}{hem tegen zich aan}\\

\haiku{Zij zagen beangst,.}{om vrezend dat het moeder}{of de zusters was}\\

\haiku{Hij stak gauw de koord.}{weg en zij trok het sjaaltje}{dieper over het hoofd}\\

\haiku{Hendrik bleef liggen.}{als een vod en het water}{spoelde uit zijn mond}\\

\haiku{En dan, met een {\textquoteleft}nu{\textquoteright},.}{de andere begon hij}{opnieuw te zoeken}\\

\haiku{Meteen overviel hem.}{een grote angst  en zijn}{hart wrong zich samen}\\

\haiku{Hij zou de avond in,.}{de kamer afwachten het}{zien donker worden}\\

\haiku{Weer moest hij wachten,.}{in de witte celkamer}{nu was het donker}\\

\subsection{Uit: De zeer schone uren van Juffrouw Symforosa, begijntjen}

\haiku{{\textquoteleft}Ik leef van de zon,,,?....}{wat zou ik en mijn bloemen}{gaan doen zonder zon}\\

\haiku{Dat maakte haar vrij.}{en ze kon terug onder}{de menschen komen}\\

\haiku{'t Beste ware,.}{nog dat hij niet kwam och haar}{hart is zoo angstig}\\

\haiku{Zij durft niet omzien,.}{en blijft staan en voelt het bloed}{in haar beenen zinken}\\

\haiku{{\textquoteright} {\textquoteleft}Ja, Symforosa{\textquoteright},, {\textquoteleft}!}{zegt hij verblijden ik zal}{veel voor u bidden}\\

\haiku{Maar ze is hare.}{belofte aan Martienus}{nog niet vergeten}\\

\haiku{En valsch heete ons, '.}{iedere waarheid waart}{lachen bij ontbrak}\\

\haiku{Z\'o\'o is Pallieter,.}{dat in 1916 verscheen en reeds}{herhaald is herdrukt}\\

\section{Oscar Timmers}

\subsection{Uit: Landklimaat}

\haiku{Je zegt mimisch och,.}{en ach als een pierrot die}{de minuten telt}\\

\haiku{Mijn bok schuurt zich langs.}{de tafelpoot en gooit een}{glas wijn over mijn broek}\\

\haiku{Jij bent zo bang, jij,}{schreeuwt zo hard dat niemand het}{de moeite waard vindt}\\

\haiku{{\textquoteright} {\textquoteleft}Hijg niet zo, je ziet,,{\textquoteright}, {\textquoteleft}}{zwart je slaat zwart uit als een}{geraamte zegt Ana}\\

\haiku{{\textquoteright} De wanhoop brandt zeer,.}{hevig de verwijten slaan}{in wolken as neer}\\

\haiku{Maar daarna ben ik,,.}{degene die wacht om en}{om een uur misschien}\\

\haiku{{\textquoteright} {\textquoteleft}H\'e...{\textquoteright} zegt ze, en met.}{verwoede ijver zet ze}{haar pogingen voort}\\

\haiku{Ze draait haar lichaam.}{naar me toe en neemt mijn hoofd}{vlak voor het hare}\\

\haiku{Het is Jessica.}{aan te zien dat ze op weg}{is naar een avontuur}\\

\haiku{We zien de huid van.}{het water wanneer ze over}{de Alpen heen springt}\\

\haiku{De deuren klapten.}{open en weer dicht en lieten}{de brug leeg achter}\\

\haiku{De dode Henker,.}{werd dood verklaard zijn daden}{werden dood verklaard}\\

\haiku{Je ogen openen je,}{schoot en rijzen ontsteken}{in het voorbijgaan}\\

\haiku{ze stromen tussen.}{mij en de matglazen deur}{door naar Jessica}\\

\haiku{En het zijn niet eens.}{de vrijbuiters die er zich}{meester van maken}\\

\haiku{{\textquoteright} (Je gooit een woord in) {\textquoteleft}.}{een echoputEn dan is de}{eenzaamheid klankrijk}\\

\haiku{De zee begint als,,.}{dal ze schuift omhoog en valt}{weer neer rijst en daalt}\\

\haiku{Je stroomt voort durend.}{als woestijn en als zee en}{als lucht door me heen}\\

\haiku{Het addertje bijt,...}{in zijn eigen staart het vormt}{een lus om je hals}\\

\haiku{{\textquoteright} fluistert Jessica.}{wanneer we het geluid voor}{het eerst vernemen}\\

\haiku{Ik verlaat haar en.}{ga op mijn knie\"en opzij}{van het bed zitten}\\

\haiku{Ik blijf naast het bed.}{zitten en leg mijn hand op}{Jessica's schouder}\\

\haiku{{\textquoteleft}Het is de dikke,,,,.}{man hij slaapt hiernaast hij snorkt}{o o wat snorkt hij}\\

\haiku{Maar  veel in de,.}{zon lopen flink zwemmen en}{lief voor elkaar zijn}\\

\haiku{Ik kon mijzelf in,.}{het oog zien ik kon mijzelf}{onder ogen komen}\\

\haiku{wiens vakantie is.}{afgelopen en weer naar}{kostschool wordt gestuurd}\\

\haiku{Goed, goed, (en ik wrijf),.}{me in de handen doe maar}{met me wat je wilt}\\

\haiku{En ik lach, maar {\'\i}k,.}{lach goed want ik zie dat ik}{je overwonnen heb}\\

\haiku{Ik voorspel je, en (),:}{ik beloof jefluisterend}{want dat wil je toch}\\

\section{Jan Gerhard Toonder}

\subsection{Uit: Kasteel in Ierland}

\haiku{Ik heb zo lang niets -!}{van je gehoord ik zou je}{bijna vergeten}\\

\haiku{Het elegante Rhein (,;}{Hotelalles glanzend nieuw}{en effici\"ent}\\

\haiku{Want, waarachtig, denk;}{eens aan de Kommandantur}{daar in Aardenbosch}\\

\haiku{En je vrouw, hoe heet,;}{ze ook weer die heeft ook wel}{wat meegenomen}\\

\haiku{Ik heb altijd wel.}{geweten dat daar heel wat}{mee gedokterd wordt}\\

\haiku{Je drinkt een borrel.}{met die kerels en dan is}{alles in orde}\\

\haiku{de zoete Jezus.}{heeft tot nu toe de kleine}{kindertjes beschermd}\\

\haiku{misschien kunnen we...}{intussen vast eens naar die}{badkamers kijken}\\

\haiku{Dr. Franz Krause had,.}{geen lucifers bij zich maar}{hij miste ze niet}\\

\haiku{{\textquoteright} {\textquoteleft}Patrick vroeg of je,.}{in de hal kwam hij heeft een}{verrassing voor je}\\

\haiku{Ze zweeg abrupt, en;}{nu hoorde hij het kraken}{van de traptreden}\\

\haiku{Laat die verdomde -!}{scharnier maar zitten we gaan}{de plee aanleggen}\\

\haiku{Maar als zijn voeten.}{dan pijn gingen doen werd het}{allemaal onzin}\\

\haiku{{\textquoteright} Dr. Franz Krause, de,.}{wettige grondeigenaar}{gaf niet eens antwoord}\\

\haiku{Neem me niet kwalijk,,.}{ik praat maar en praat maar dat}{komt van mijn beroep}\\

\haiku{{\textquoteright} {\textquoteleft}Maar hij begrijpt me,.}{niet terwijl ik toch alleen}{maar op mijn recht sta}\\

\haiku{Hij vertelt mijn vrouw;}{verhaaltjes over de vogels}{van de ru{\"\i}ne}\\

\haiku{Hij kon niet denken,.}{geen zin wilde zich in zijn}{gedachten vormen}\\

\haiku{Dit is veel en veel.}{meer dan ik in Duitsland zou}{moeten betalen}\\

\haiku{{\textquoteright} Krause vroeg zich een;}{ogenblik af waarom dat zo'n}{verschil zou maken}\\

\haiku{Hoe kon een normaal,,?}{praktisch hard werkend mens hier}{ooit iets bereiken}\\

\haiku{Als door een wonder;}{bleek de elektriciteit te}{functioneren}\\

\haiku{in bog or wood, her!}{lovely smile makes}{him feel glad and good}\\

\haiku{Met wat vruchtbomen,.}{langs de kant die bloeien zo}{mooi in de lente}\\

\haiku{{\textquoteleft}Dus,{\textquoteright} zei hij kortaf,, {\textquoteleft}.}{beledigdhij kwam niet over}{mijn weiland praten}\\

\haiku{Hij ging, natuurlijk - -;}{dit moest zijn tijdbom wel zijn}{maar met ergernis}\\

\haiku{Een jodenjongen -;}{van lang geleden als zij}{dat nu aardig vond}\\

\haiku{Hij vloog overeind met.}{zo'n schok dat er iets van het}{nachtkastje afviel}\\

\haiku{{\textquoteleft}Ga dan zitten en,;}{vertel me terwijl ik je}{een kop koffie schenk}\\

\haiku{Hij zette zich dus,,,:}{schrap schraapte zijn keel keek naar}{de lucht en begon}\\

\haiku{Jawel, daarin moest.}{hij controleposten en}{de grens passeren}\\

\haiku{{\textquoteright} En dat kwam van de,.}{ingang door het krassen van}{de vogels heen}\\

\haiku{{\textquoteleft}Laat ons er niet meer,{\textquoteright}.}{over spreken en dat hadden}{ze niet meer gedaan}\\

\section{Fernand Toussaint van Boelaere}

\subsection{Uit: Landelijk minnespel}

\haiku{Tegelijk drong ook;}{de nachtelijke stallucht}{zwoeler op hem aan}\\

\haiku{De hengst bekeek hem, ';}{met vlammend oog rond alsn}{glad-bruine schijf}\\

\haiku{{\textquoteleft}Hij mag hij doen wat, ' '.}{hij wilk ben toch opt}{voorhand gewroken}\\

\haiku{Maar Bello, 't is,,.}{zeker staat en wacht nog op}{den drempel kaarsrecht}\\

\haiku{En nu al met eens,, ':}{zie vernam hij hoet nog}{trilde in de lucht}\\

\haiku{{\textquoteleft}- Zeg nu ja, Zalia,{\textquoteright},.}{bad hij met de lippen en}{de smeekende oogen}\\

\haiku{Evenwel, zijn gelaat.}{bleef gewend naar de plek waar}{hij de brugge wist}\\

\haiku{{\textquoteleft}Ge moet gij zeker,{\textquoteright}....}{ook toch \'eten klonk het in de}{rust van de kamer}\\

\haiku{De andere knechts,....}{had ze maar laten gaan hij}{was immers nog uit}\\

\haiku{- Hoe ook weg met zijn,,,}{gedachten de Langen al}{gaande zag wel dat}\\

\haiku{Het geluid van heur, '....}{stem klonk aldoor zachtert}{was haast fezelen}\\

\haiku{Toen liet zij langzaam,;}{die hand omhoogwaarts keeren als}{een deur die opendraait}\\

\haiku{{\textquoteright} De toon der stem steeg,,?}{al hooger scherper Ging er}{toch laweit bij zijn}\\

\haiku{steeds dieper drong dan '.}{ookt snijdend ijzer door}{den schoot der aarde}\\

\haiku{het blijde dier, vrij,;}{thans van teugel en getuig}{snoof luidruchtig}\\

\haiku{Plots echter brak 't.}{ijzerig geklep van een}{deurklink de stilte}\\

\haiku{Het was precies of.}{Bello hem niet z\'o\'o vast en}{geerig verwachtte}\\

\haiku{Moeizaam vernam dan.}{zijn oor hoe heel de stal moest}{staan in rep en roer}\\

\haiku{- van een dier dat plots,.}{recht opsprong een wijle dan}{zoo steigerend bleef}\\

\haiku{Z\'o\'o deed de hengst het,.}{ook dezen morgen  toen}{hij hem afstrafte}\\

\haiku{Wat ze niet voor een {\textquoteleft}{\textquoteright}, {\textquoteleft}.}{deugd aanzagen noemden ze}{kort-wegondeugd}\\

\section{Arnaud de Trega}

\subsection{Uit: Ramp en misdaad. 1623-1638}

\haiku{Wij schreven, wars van,.}{mode of mooidoenerij}{zonder pretentie}\\

\haiku{twee Heeren \'e\'en Heer,,.}{die is onze God en het}{vrije Volk van Maestricht}\\

\haiku{Lachend sprong hij te,.}{paard en het was mij als zag}{ik hem voor het laatst}\\

\haiku{Frederik Hendrik,.}{is een bekwaam kapitein}{die alles voorziet}\\

\haiku{Ofschoon streng in de,.}{leer was hij buitengewoon}{breed van opvatting}\\

\haiku{Vaarwel, wat gij U.}{hadt gedroomd van grooter en}{beter en hooger}\\

\haiku{zij stonden bij hun,.}{paarden aan den grooten weg}{achter het kasteel}\\

\haiku{in galop door de,,.}{poort en links af langs de kerk}{waar hun paarden staan}\\

\haiku{vijf en zeventig,.}{voet van de Zuidzij in den}{kelder op 6 voet}\\

\haiku{- Zij zijn opgehitst,.}{door de geestelijken zei}{een der kapiteins}\\

\haiku{De gemetselde;}{gang had eene manslengte en}{was wel 4 voet breed}\\

\haiku{Een legioen van}{ratten spoedde zich voor hen}{uit en menigmaal}\\

\haiku{zie, ze kruipen weg,.}{in dezelfde richting daar}{moet de put wezen}\\

\haiku{Inderdaad, stonden:}{zij weldra bij den put en}{Carabin bromde}\\

\haiku{Weldra stonden zij:}{voor den sergeant en de}{Leegtenborg zeide}\\

\haiku{- Wij hebben aan hun;}{lijken niet de laatste eer}{kunnen bewijzen}\\

\haiku{- De kapitein kan,.}{elk oogenblik komen zoodat}{U even wachten kunt}\\

\haiku{zij stopten mij een;}{prop in den mond en sleurden}{mij in den kelder}\\

\haiku{- Wat meent gij, dat ik,,.}{doen kan mejuffrouw Agnes}{vroeg hij eindelijk}\\

\haiku{- Carabin, zei de,,.}{kapitein zie wel toe wat}{achter ons gebeurt}\\

\haiku{Indien wij deze,.}{vrouw kunnen achterhalen}{is het kind gered}\\

\haiku{- De overwinning was,,?}{ons zei Kapelaan Schreuders}{doch tot welken prijs}\\

\haiku{Wanneer gij in die,,......}{richting denkt heer Hofmeyer}{bent U er glad naast}\\

\haiku{Zoo bad hij om kracht.}{tot den Hemel en deed St.}{Servaes geweld aan}\\

\haiku{wat kan ik doen? - Heer,.}{Proost nog heden nacht moet het}{plan beraamd worden}\\

\haiku{- Broeders, zei meester,.}{Wynans thans zijt gij een en}{onverdeeld met ons}\\

\haiku{Schipper de Gye heeft.}{verlof tot lossen en kan}{vrij heen en weer gaan}\\

\haiku{De beide zwagers.}{en krijgsmakkers omhelsden}{elkaar in stilte}\\

\haiku{Wenscht een uwer zulks,,?}{te doen of verlangt gij u}{nog te beraden}\\

\haiku{Ik zal leven en...}{sterven in- en met mijn}{herinneringen}\\

\haiku{Zoudt gij weigeren,.}{te spreken dan wachten U}{pijnbank en schavot}\\

\haiku{geve de hemel,.}{of de hel dat zijn beulshand}{hem niet beroere}\\

\haiku{- Wij zien hem, hooren,.}{hem en rieken hem braden}{spotte een soldaat}\\

\haiku{In de Gothische,,.}{zaal een ruim vertrek werd de}{vierschaar gespannen}\\

\haiku{- Jezabelle de,,?}{Rax vrouwe Sleussel waartoe}{zijt gij gekomen}\\

\haiku{Hij weigerde den,,.}{blinddoek legde zijn hoofd op}{het blok de bijl viel}\\

\haiku{In de kluis brandde.}{nog licht en ik hoorde het}{geluid van stemmen}\\

\haiku{- Ik breng den laatsten.}{wil en een laatste verzoek}{van een stervende}\\

\haiku{gij hebt het verlangd.}{en gij zult dezen nacht uw}{leven herleven}\\

\haiku{Op korten afstand,,.}{volgde Jan van Sichem een}{pak op den bochel}\\

\haiku{Wij cre\"eerden uw,,,.}{lichaam maar Hij wiens dienaar}{Gij waart schiep de ziel}\\

\haiku{Eetwaren mogen.}{niet hooger verkocht worden}{dan vastgesteld is}\\

\haiku{Met Gods hulp komt de.}{verlossing zeker Vrijdag}{of Zaterdag}\\

\section{Pieter Jelles Troelstra}

\subsection{Uit: Gedenkschriften. Deel I. Wording}

\haiku{Daar komt zijn oude.}{goudsmidsbaas binnen en gaat}{recht op hem af}\\

\haiku{Stroef en ernstig was,.}{hij geheel anders dan zijn}{jongere broeders}\\

\haiku{Waar zij met geest en.}{moed beladen Steeds stierven}{voor de zaak van God}\\

\haiku{Ziedaar                         mede.}{een aanleiding tot deze}{korrespondentie}\\

\haiku{Weldra blijkt het noodig,;}{de belijdenis elders}{af                         te leggen}\\

\haiku{Dit was, geloof ik,,.}{het eenige standpunt dat zij}{kon                         innemen}\\

\haiku{Het duurde niet lang,, {\textquoteleft}{\textquoteright}.}{of mijn grootste genoegen}{wasvoor te lezen}\\

\haiku{Halbertsma tot min.}{of meer offici\"eele}{schrijftaal ontwikkeld}\\

\haiku{Het belangrijkste.}{vraagstuk was natuurlijk het}{kinderen krijgen}\\

\haiku{Omstreeks dien tijd lag}{ik ziek aan de mazelen}{en daarmede}\\

\haiku{Hiermede begon.}{voor mij een                     nieuw tijdperk}{van mijn leven}\\

\haiku{Eerst maakten wij de, {\textquoteleft}{\textquoteright}.}{hokken der konijnen}{dan dat vanbaas schoon}\\

\haiku{Aan dezen man voel.}{ik mij door                     innige}{dankbaarheid verknocht}\\

\haiku{Terwijl ik daar stond,,,.}{trillend naakt ging de deur van}{de                     keuken open}\\

\haiku{Mijn levensdrang en.}{zelfvertrouwen uitten zich}{o.a. in                     spotzucht}\\

\haiku{, zijn oog glinsteren.}{en het zweet op zijn voorhoofd}{zien parelen}\\

\haiku{als een machtsuiting,.}{waarvoor ik al heel weinig}{lust had te bukken}\\

\haiku{v\'o\'or dien tijd                     reeds.}{vond hij een plaats aan een der}{Twentsche fabrieken}\\

\haiku{Van wanken, draait, en;}{wordt gedreven Om't een}{en eenig middelpunt}\\

\haiku{het lag voor de hand,;}{dat ik liefst zou                     studeeren}{in de Letteren}\\

\haiku{Ik nam                     daarvoor.}{les bij een der leeraren van}{het Gymnasium}\\

\haiku{De wereld breidt zich,,;}{voor mij uit De lachende}{lokkende verte}\\

\haiku{Ik smacht naar den blik,.}{van twee oogen Die stralen als}{zonnen voor mij}\\

\haiku{Ik tast om mij heen,;}{naar twee handen Wier druk mij}{doortintelt met kracht}\\

\haiku{{\textquoteright} {\textquoteleft}Kan mijnheer mij                     ,?}{ook zeggen wanneer mijnheer}{mij ontvangen kan}\\

\haiku{Deze ging ver uit;}{buiten de grenzen eener}{gewone rechtszaak}\\

\haiku{{\textquoteleft}Uit uw vroegeren, {\textquotedblleft}{\textquotedblright}.}{brief begrepen wij reeds dat}{men unegerde}\\

\haiku{Van twee\"en                     een,}{\`of onderwerp u aan die}{barbaarschheden}\\

\haiku{Neen, Piet, ik moet je,.}{eerlijk bekennen dat}{je mij erg afvalt}\\

\haiku{{\textquoteleft}Ons ongeluk is,;}{dat wij teveel met elkaar}{gemeen hebben}\\

\haiku{50Zie blz. 80, 81,,.}{en 83 uitgave Fischer}{Verlag Berlin}\\

\subsection{Uit: Gedenkschriften. Deel II. Groei}

\haiku{Ik heb voor dit werk.}{geen letter zelf op papier}{kunnen                         zetten}\\

\haiku{Zelf geeft hij dit in, ():}{zijn gedenkschriften toe waar}{hij                     zegtblz. 252}\\

\haiku{{\textquoteleft}Ik was namelijk.}{niet in de wieg gelegd voor}{politiek leider}\\

\haiku{De toegeworpen.}{handschoen moest door mij worden}{opgenomen}\\

\haiku{Zoo kunnen schijnbaar.}{tegenstrijdige daden}{worden                     verklaard}\\

\haiku{Voorloopige.}{hechtenis was aan de}{orde van den dag}\\

\haiku{De zaak-Nawijn,,.}{ook te Heerenveen maakte}{vrij wat                     gerucht}\\

\haiku{Men                     spreekt voor een.}{goede zaak en wordt beschouwd}{als misdadiger}\\

\haiku{de tijd was gunstig.}{en nieuwe afdeelingen}{van den Soc.-Dem}\\

\haiku{In mijn krant                     schreef.}{ik over de stemming in de}{vergaderingen}\\

\haiku{daar was een klein                     ,.}{meisje door haar ouders om}{de krant uitgestuurd}\\

\haiku{A. DIJKMAN TE UTRECHT}{J. DIEMEL TE UTRECHT}{A.J.E. RAVESTEIN}\\

\haiku{Onze toestand                     :}{werd steeds hachelijker en}{toen ging het er om}\\

\haiku{Misschien is de tijd,.}{wel spoedig daar dat ik de}{stad moet verlaten}\\

\haiku{Zoodra deze,,:}{mij in het vizier kreeg kwam}{hij op mij toe zei}\\

\haiku{In den loop van                      {\textquoteleft}{\textquoteright};}{onze polemiek schreef ik}{in deBaanbreker}\\

\haiku{Aan het verslag in {\textquoteleft}{\textquoteright}:}{deBaanbreker                    ontleen}{ik het volgende}\\

\haiku{De uitslag was, dat.}{ik 2276 stemmen kreeg en dr.}{Bos                     slechts 1606}\\

\haiku{Zijn opvattingen,.}{ontsprongen aan een zacht maar}{vurig  gemoed}\\

\haiku{Wij riepen alle;}{medestanders op zich}{naast ons te scharen}\\

\haiku{het algemeen                     {\textquoteright}.}{kiesrecht of de strijd voor het}{algemeen kiesrecht}\\

\haiku{Tak, die toen nog geen, {\textquoteleft}{\textquoteright}:}{lid van de Partij was}{schreef in deKroniek}\\

\haiku{{\textquoteleft}Laat ze nog maar wat,.}{geduld hebben                             eerst moet}{die Troelstra van de baan}\\

\haiku{Eenige dagen reeds,:}{ben ik bezig me in die}{kunst te oefenen}\\

\haiku{In                     dezen tijd;}{heb ik veel voorgelezen}{aan mijn kinderen}\\

\haiku{Doodelijk                     uitgeput,.}{begroette hij mij weer met}{dien stralenden blik}\\

\haiku{Spreken kon hij niet,.}{meer zonder woorden nam hij}{van ons                     afscheid}\\

\haiku{{\textquoteleft}Ik zal wel spreken{\textquoteright},.}{hetgeen hij ook                     op den}{juisten toon volbracht}\\

\haiku{de theorie zoekt;}{naar scheiding en ontleding}{van                     begrippen}\\

\haiku{De Nederlandsche....}{vakbeweging gelijkt thans}{een groot slagveld}\\

\haiku{Wij zijn                     er trotsch;}{op uw strijd en nederlaag}{te hebben gedeeld}\\

\haiku{Daaraan gaf op het,:}{kongres ook                         van der Goes}{uiting die zeide}\\

\haiku{Toen ik verleden....}{jaar                     optrad tegen de}{zwendelarijen}\\

\haiku{Naar aanleiding van {\textquoteleft}}{de schoolkwestie had ik in}{een artikel in}\\

\haiku{Pi\"eteit jegens.}{zijn vader heeft zijn gansche}{leven beheerscht}\\

\haiku{Daarom schreef ik een,:}{brief aan van der Goes waarin}{ik o.a.                     zeide}\\

\subsection{Uit: Gedenkschriften. Deel III. Branding}

\haiku{De betooging te.}{Amsterdam van 1906 telde}{15.000                     bezoekers}\\

\haiku{het werd duidelijk,.}{dat de                     stroom niet lang meer}{te stuiten zou zijn}\\

\haiku{censuskiesrecht zou.}{evengoed mogelijk zijn als}{algemeen kiesrecht}\\

\haiku{Tot mijn spijt                     was;}{Tak door mijn woorden zeer in}{zijn wiek geschoten}\\

\haiku{Ik                     eindigde:}{met het voorstellen van de}{volgende motie}\\

\haiku{haar militairen;}{romantikus weer binnen}{de rails te brengen}\\

\haiku{Dit alles gaf                     .}{aan die tochten voor hen een}{boeiend karakter}\\

\haiku{het viel mij op, hoe.}{dit klassieke werk de}{kinderen boeide}\\

\haiku{\'e\'en voor \'e\'en uit het,,.}{bootje en hij bleef verstomd}{alleen                     zitten}\\

\haiku{Dat het hem niet licht,,.}{viel mij dit te zeggen zult}{ge wel begrijpen}\\

\haiku{Ook voor onze zaak,.}{hoop ik steeds te doen wat mijn}{krachten toelaten}\\

\haiku{Mijn positie in;}{de Kamer bewoog zich}{in stijgende lijn}\\

\haiku{In de eerste plaats.}{verwart men                             klassenstrijd}{met klassenoorlog}\\

\haiku{De verwarring in.}{de Partij nam steeds grooter}{afmetingen aan}\\

\haiku{De mededeeling.}{werd met luid en driewerf}{hoera ontvangen}\\

\haiku{Driemaal                     drong ik,.}{daarop tevergeefs aan toen}{liet ik het verder}\\

\haiku{Ik zal niet verder.}{ingaan op het verwarde}{verloop dezer zaak}\\

\haiku{Zij is niet tragisch,....}{en niet dramatisch maar toch}{een                     stap voorwaarts}\\

\haiku{Dan wordt van nu af.}{aan geregeerd                     onder}{zijn patronage}\\

\haiku{Den volgenden dag.}{verscheen een vierde stille}{bondgenoot ten tooneele}\\

\haiku{De Heer Colijn                      {\textquoteleft}{\textquoteright}:}{schreef in een artikel in}{deStemmen des Tijds}\\

\haiku{Ook deze lezing.}{werd in  de geheele}{pers overgenomen}\\

\haiku{op het                     kerkplein.}{traden o.a. Anseele en}{ik als sprekers op}\\

\haiku{{\textquoteright} De 19de Mei sprak {\textquoteleft}}{ik in een vergadering}{te Amsterdam over}\\

\haiku{Deze povere.}{konklusie bewees}{onze nederlaag}\\

\haiku{niet zoo gauw tot de.}{wereldoorlog zou hebben}{laten                     komen}\\

\haiku{Men moet dat hebben;}{meegemaakt om het goed te}{kunnen begrijpen}\\

\haiku{onze laatste hoop,.}{niet in den                     oorlog te}{worden meegesleept}\\

\haiku{Ten overvloede                     ,:}{schreef ik nog een artikel}{waarin ik zeide}\\

\subsection{Uit: Gedenkschriften. Deel IV. Storm}

\haiku{organisatie.}{van het                     bedrijf onder}{leiding van de staat}\\

\haiku{Huysmans was na een;}{avontuurlijke tocht eveneens}{gearriveerd}\\

\haiku{Van u moet de kracht,.}{uitgaan die ons werk                     tot}{een goed einde brengt}\\

\haiku{Maar dan moeten zij,.}{het maar ronduit zeggen dat}{zij niet willen}\\

\haiku{een redeneering,.}{die door Albert Thomas in}{Frankrijk werd herhaald}\\

\haiku{Wij moesten dan vooral.}{oppassen de stations}{te bezetten}\\

\haiku{wij moesten zorg dragen,;}{daarbij                     de leiding in}{handen te houden}\\

\haiku{er moest snel iets                     ,.}{gebeuren anders zouden}{anderen het doen}\\

\haiku{Ankersmit heeft het {\textquoteleft}{\textquoteright}:}{in zijn                     bespreking van}{Groei zoo juist gezegd}\\

\haiku{inwilliging                     ;}{van de eischen van de Bond}{van Dienstplichtigen}\\

\haiku{{\textquoteright} Met dit                     voorstel.}{ging de meerderheid van het}{P.B. echter niet mee}\\

\haiku{De konklusie,,:}{van Cramer wiens meening}{ik heb gevraagd luidt}\\

\haiku{Op deze vraag kan;}{slechts een subjektief antwoord}{gegeven worden}\\

\haiku{organisatie,.}{parlementaire aktie}{en propaganda}\\

\haiku{een groot deel van mijn.}{artikelen dikteerde}{ik op mijn bed}\\

\haiku{Zeer pijnlijk was te.}{Luzern de strijd tusschen de}{twee Duitsche groepen}\\

\haiku{Ik heb mij tegen;}{alle voorstellen tot}{afscheiding verzet}\\

\haiku{Pinksteren 1923 kwam.}{het hereenigingskongres}{te Hamburg bijeen}\\

\haiku{na de opheffing.}{dier tegenstellingen zal}{de staat afsterven}\\

\haiku{kon nu maar op die.}{manier de heele                     zaak}{in orde komen}\\
