\chapter[16 auteurs, 1682 haiku's]{zestien auteurs, zestienhonderdtweeëntachtig haiku's}

\section{Pieter Ecrevisse}

\subsection{Uit: Vier verhalen uit het land van Zwentibold}

\haiku{tilde hem op uit,;}{den kring welke zich om den}{kerel gevormd had}\\

\haiku{Ik weet zeker, dat.}{gij vader en moeder zoudt}{gelukkig maken}\\

\haiku{maar dat hij ook eene.}{verantwoordelijkheid op}{zijne schouders droeg}\\

\haiku{want zwijgen, waar men,.}{spreken moet is een bewijs}{van lafhartigheid}\\

\haiku{dit jaar zie ik van,.}{mijn voorrecht af en zal maar}{de laatste komen}\\

\haiku{de voldoening moet,!}{openbaar zijn dat zweer ik u}{hier op mijne eer}\\

\haiku{Indien gij eventwel,,}{wist hoe rechtzinnig ik't}{met u meene gij}\\

\haiku{Hoe lang schenen mij,:}{de zes dagen die ons van}{dit bezoek scheidden}\\

\haiku{en nog hadden de.}{paarden te traagzaam gestapt}{voor ons ongeduld}\\

\haiku{Zoolang wij hier in,';}{bezetting liggen reken}{ik mij nog t huis}\\

\haiku{te meer, daar ik aan,;}{niemand mocht bekend maken}{wat in mij omging}\\

\haiku{Ik haalde u hier;}{niet alleen de colonne}{mobile in huis}\\

\haiku{Op deze wijze,.}{heb ik meer dan drie honderd}{mijlen afgelegd}\\

\haiku{Eenige minuten.}{verliepen onder deze}{heilige stilte}\\

\haiku{Langzamerhand heb;}{ik hem allerlei gerief}{medegenomen}\\

\haiku{Ook vermeed ik, uit,;}{dien hoofde alle verkeer}{met gezelschappen}\\

\haiku{mij zoo lastig, dat;}{ik beslote een einde}{daaraan te stellen}\\

\haiku{Let wel op, dat er,,.}{elke tien tot twaalf stappen}{zulk kruis geteekend staat}\\

\haiku{Dat ik in hunne,...}{handen viele daaraan is}{weinig gelegen}\\

\haiku{en Barbara sloot;}{den broeder met meer geestdrift}{in hare armen}\\

\haiku{Er ging nog een uur..,,!}{voorbij een uur hetgeen haar}{een eeuwigheid scheen}\\

\haiku{Op de vraag van het,:}{meisje begint Lemmens het}{volgende verhaal}\\

\haiku{Wat mij persoonlijk,.}{betrof vreesde ik deze}{agenten zeer weinig}\\

\haiku{De overwinningen.}{begonnen hoe langer zoo}{schaarscher te worden}\\

\haiku{{\textquoteright} De commandant sloeg:}{een register open en las}{wat er geboekt stond}\\

\haiku{want het eene schijnt mij.}{zoo pijnlijk om bekennen}{als het andere}\\

\haiku{doch ik zal mij niet!}{medeplichtig maken aan}{eene barbaarsche daad}\\

\haiku{Gij, franschmans, hebt;}{geen medelijden gehad}{met ons Vlamingen}\\

\haiku{{\textquoteright} Op deze wijze.}{maakte hij de aandacht der}{gendarmen gaande}\\

\haiku{Het is een eerlooze,!}{duitscher die tegen zijne}{stamgenoten dient}\\

\haiku{Zoo niet, dit zweer ik,!}{dat gij zonder ooren en}{neus van hier zult gaan}\\

\haiku{doch de smaak kwam hem,.}{zoo walgend voor dat hij het}{vocht moest uitspuwen}\\

\haiku{men kan niet weten, -, -;}{zegde hij of men den wolf}{niet in den muil loopt}\\

\haiku{Weinige woorden;}{werden gewisseld tusschen}{deze vier menschen}\\

\haiku{{\textquoteleft}- Mijn plichtbesef en;}{mijne dankbaarheid jegens}{de Voorzienigheid}\\

\haiku{stamelen, terwijl.}{hij den rug der rechter hand}{op zijn voorhoofd bracht}\\

\haiku{{\textquoteright} {\textquoteleft}- Neen, mijn kind, ik ben,:}{er verre af u deze}{hulp te verwijten}\\

\haiku{Deze dankbaarheid.}{ga ik oogenblikkelijk}{op de proef stellen}\\

\haiku{{\textquoteleft}- Uit hoofde van een,.}{kwalijk geplaatst eergevoel}{heeft hij gezwegen}\\

\haiku{bemint gij mijnen,?}{zoon en wilt gij hem uwe hand}{en uw hart schenken}\\

\haiku{Men bracht er schier den.}{geheelen nacht over in de}{hevigste spanning}\\

\haiku{verliet het huis met:}{even zooveel spoed als hij was}{binnengekomen}\\

\haiku{weldra stroomden de.}{overvloedigste tranen langs}{zijn bleek aangezicht}\\

\haiku{Zij vond het meisje ',:}{alleent huis en begon}{het gesprek aldus}\\

\haiku{Indien Isabella,!}{een arm meisje ware hij}{zou haar niet willen}\\

\haiku{Deze twee menschen.}{waren onafscheidbare}{vrienden geworden}\\

\haiku{De golven sloten...}{zich eenen oogenblik boven}{het hoofd des ruiters}\\

\haiku{Nooit zou het ruchtbaar,.}{geworden zijn zonder een}{zonderling toeval}\\

\haiku{Wij hadden deze:}{bewegingen zonder eenig}{overleg toegezien}\\

\haiku{Ik zelf begrijp niet,,,,;}{wat de heer graaf mijn vader}{wil of beraamd heeft}\\

\haiku{Wat belette hem?}{ten minste eenig teeken van}{bestaan te geven}\\

\haiku{De heer graaf en zijn;}{zoon moesten redenen hebben}{om bedroefd te zijn}\\

\haiku{omtrent al deze;}{omstandigheden was Jan's}{verhaal haar ontsnapt}\\

\haiku{De priester en ik...}{hadden den tijd niet om een}{antwoord te geven}\\

\haiku{Hier op mijn hart zult;}{gij den sleutel vinden van}{de deur zijns verblijfs}\\

\haiku{staarde verwilderd,.}{rond en toen hij ons beide}{zag scheen hij verschrikt}\\

\haiku{want de kristlijke.}{gelatenheid keerde op}{zijn gelaat terug}\\

\haiku{Het duurde eventwel,.}{niet lang of ik begon de}{oorzaak te gissen}\\

\haiku{O hemel, dacht ik,;}{thans gaat het woedende dier}{op den jager los}\\

\haiku{want zij zegde met,:}{bezorgdheid terwijl hare}{stem lichtlijk trilde}\\

\haiku{Ik beloofde en,;}{hoopte mijnen vader te}{zullen bewegen}\\

\haiku{Den dag voor Kersmis,,.}{des jaars 1683 had ik goeden}{buit ter jacht gemaakt}\\

\haiku{Met eene hijgende.}{borst sloop ik tot aan de deur}{van vaders kamer}\\

\haiku{ik beet op mijne.}{lippen en zocht naar eenen tekst}{om te beginnen}\\

\haiku{Ik stond op het punt,:}{de kamer te verlaten}{toen hij mij toeriep}\\

\haiku{, Walter, dat gij niet.}{ongenegen zijt voor den}{huwelijken staat}\\

\haiku{Zeg, bid ik u, dat!}{gij mij enklijk hebt willen}{op de proef stellen}\\

\haiku{Want hij behoort noch,.}{aan mij noch aan zich zelven}{maar aan zijn stamhuis}\\

\haiku{Hij straft dengenen,,;}{die zijne ouders niet eert}{reeds in dit leven}\\

\haiku{Daarna keerde ik.}{met haastige schreden naar}{het kasteel terug}\\

\haiku{Van honger wil ik,;}{niet omkomen dat ware}{eene soort van zelfsmoord}\\

\haiku{in Julia's oogen,;}{wilde ik geenszins voor een}{meineedige doorgaan}\\

\haiku{Tot hiertoe had de;}{zieke nog altoos met eenig}{gemak gesproken}\\

\haiku{Adriaan, Jan en ik.}{besproeiden het plekje grond}{met onze tranen}\\

\haiku{uwe hutten zullen!}{twee luiken ter straat en twee}{ter zijde hebben}\\

\haiku{Limbrichts grond, water,,,.}{straten bewooners in een woord}{alles behoort mij}\\

\haiku{want zij geleken.}{zich als twee droppels water}{uit het zelfde vat}\\

\haiku{want hij is bereid:}{om ook zijnen besten vriend}{te verloochenen}\\

\haiku{Maar maken wij de:}{lezers nader bekend met}{de beide zieken}\\

\haiku{De lieutenant;}{besloeg de eerste plaats op}{het ziekeleger}\\

\haiku{nochtans scheen 't hem;}{toe dat hij nog weinig of}{niets gewonnen had}\\

\haiku{{\textquoteleft}Wij hadden sedert,;}{30 jaren een tuinman met}{name Pelart}\\

\haiku{Waart gij niet vroeger,?}{Pierre P\'edart de zoon van}{mijnen hovenier}\\

\haiku{- Druk de woorden wel;}{in uw geheugen welke}{ik u ga zeggen}\\

\haiku{door den ouden adel!}{aanzien als een mengsel van}{alle boosheden}\\

\haiku{Ein herz von edelmuth,}{bewohnt Ist durch sich selbst}{am herrlichsten}\\

\haiku{Misschien viel de stok '!}{opt hoofd des bisschops van}{Trier uit gewoonte}\\

\haiku{De onschuldigste;}{daden van Johan werden ten}{ergsten uitgelegd}\\

\haiku{Aanstonds deed hij de;}{overige hovelingen}{voor zich verschijnen}\\

\haiku{geheel zijne macht:}{en wil schenen in de oogen}{alleen te heerschen}\\

\section{Frederik van Eeden}

\subsection{Uit: Gedachten}

\haiku{Parodieeren en.}{lachen zijn gevaarlijke}{wijzen van kritiek}\\

\haiku{Een organisme,.}{stort ineen met schokken maar}{groeit niet met schokken}\\

\haiku{Tot ons geluk zijn.}{onze instincten sterker}{dan onze rede}\\

\haiku{Wat voor een klein mensch,.}{ondeugd is kan zeker nooit}{deugd zijn voor een groot}\\

\subsection{Uit: De kleine Johannes. Deel 2}

\haiku{Zeker hebt ge niet, '?}{gedacht dat ik mijn woord zou}{houden ist wel}\\

\haiku{Gaat het niet, dan spijt,.}{me dat voor u maar ge moet}{er niet om liegen}\\

\haiku{- {\textquoteleft}wat wordt ge nat, bind,.}{mijn jasje om uw hoofd ik}{kan het wel missen}\\

\haiku{{\textquoteright} - - {\textquoteleft}En heel spoedig zult,.}{ge mij weer niet zien en toch}{ben ik er even goed}\\

\haiku{En zijn verheven,.}{metgezel was er ook een}{een gewone man}\\

\haiku{Hij dacht aan zijn dooden,.}{vader en dat hij nu naar}{een kermis-spel ging}\\

\haiku{Hij heeft veel verdriet,,.}{gehad Marjon zijn vader}{is pas gestorven}\\

\haiku{Hij heeft z'n verstand,,}{zoo goed als jelui met z'n}{vieren bij mekaar}\\

\haiku{In vlechten was nu '.}{t lichtblonde haar rondom}{haar hoofd gebonden}\\

\haiku{Het mooie en blijde.}{alleen is goed en dat wat}{wij moeten zoeken}\\

\haiku{Ga nu maar gauw, of.}{de blindheid van je kind komt}{op je geweten}\\

\haiku{Er werden woorden.}{in hem geboren die hij}{zorgvuldig vasthield}\\

\haiku{Want de wijze heeft.}{zijn ouders lief en wil hun}{kwaad wel goed maken}\\

\haiku{Vader en moeder.}{leven nog en gedijen}{door onze moeite}\\

\haiku{{\textquoteleft}Denk niet Johannes,.}{dat ik je telkens zeggeu}{zal wat je doen moet}\\

\haiku{Jamaar, tante, als,?}{ik vroom ben kom ik ook in}{den hemel niet waar}\\

\haiku{Rondom was het zeer,}{stil de blauwe en witte}{sterbloemen tusschen}\\

\haiku{hij moest leven, dat,,.}{hadden ze allen man en}{vrouw grijsaard en kind}\\

\haiku{Daar was schemerschaduw,.}{en een geheimzinnige}{plechtige stilte}\\

\haiku{- En dat na al het.}{goede wat ik dacht voor je}{gedaan te hebben}\\

\haiku{{\textquoteleft}Daatje wil je maar naar, '.}{de keuken gaan ik zalt}{verder wel afdoen}\\

\haiku{{\textquoteright} De gemeente keek.}{onthutst van den spreker naar}{dominee Kraalboom}\\

\haiku{{\textquoteleft}Ik verzoek u, de.}{orde in dit kerkgebouw}{niet te verstoren}\\

\haiku{Wie onzer zou de?}{genade niet begeeren en}{de zaligheid niet}\\

\haiku{Het leven van een,.}{oud mensch is zoo dor als er}{niets jongs bij opgroeit}\\

\haiku{Eindelijk bracht de.}{vlooien-temmer Johannes}{bij Marjon's wagen}\\

\haiku{En wijzer worden.}{was iets waartoe hij de kans}{niet graag verzuimde}\\

\haiku{{\textquoteright} zei Johannes en.}{klemde zijn handen samen}{in groote ontroering}\\

\haiku{Jij mot maar voor de,.}{woorden zorgen dan zorg ik}{wel voor de muziek}\\

\haiku{je zult wel motte,,.}{of je wilt of niet om niet}{te verhongeren}\\

\haiku{Het stonk er zoo, naar.}{uien en gebakken vet}{en naar veel erger}\\

\haiku{{\textquoteright} De meisjes klapten ' {\textquoteleft}!}{in de handjes en riepen}{omt hardstbravo}\\

\haiku{{\textendash} {\textquoteleft}Wij kunnen wel met,.}{zekerheid zeggen dat dit}{niet toevallig is}\\

\haiku{Misschien komt hij wel.}{tot den eersten met studie}{en goede cultuur}\\

\haiku{{\textquoteleft}Ik ben nog niet goed,.}{maar ik wil graag mijn best doen}{om het te worden}\\

\haiku{Nog eer hij te huis,.}{kwam was de taak hem reeds te}{machtig geworden}\\

\haiku{Wistik vooraan, want.}{die was gewend in duister}{en kende den weg}\\

\haiku{Zij riepen over de:}{gansche aarde en in den}{donkeren hemel}\\

\haiku{Ze hadden allen,.}{dier-vormen maar grooter}{en beter voltooid}\\

\haiku{Pan's doodbaar stond aan,.}{den oever der zee al het}{levende rondom}\\

\haiku{het blauw-en-witte,.}{kwam afdalen als een}{sneeuw-lawine}\\

\subsection{Uit: De kleine Johannes. Deel 3}

\haiku{Het land zag somber.}{en troosteloos als een land}{door lava verwoest}\\

\haiku{je kunt eens bij me,.}{op de vleugelen onzer}{dichter-vriendschap}\\

\haiku{Of geloof je 't? ',!}{nog niett Is heusch geen}{mopje van me hoor}\\

\haiku{Een wat ouder kind '.}{stond bij zijn knie en keek naar}{t dampende eten}\\

\haiku{boven je staat, v\`er,.}{en toch doet ze aldoor of}{ze de minste is}\\

\haiku{{\textquoteright} Deze, die voorover,.}{gebogen had gezeten}{richtte het hoofd op}\\

\haiku{- Als Jo nu fijner,....}{gezelschap noodig heeft dan moet}{hij zelf maar kiezen}\\

\haiku{Maar op 't laatste ',}{oogenblik schreef hijt af}{zonder te zeggen}\\

\haiku{Tot aan zijn neus kroop.}{hij onder de lakens en}{het zweet brak hem uit}\\

\haiku{Spoorrails en een trein.}{in de verte en opeens}{niet verder kunnen}\\

\haiku{Verwarde onzin,,.}{zonder ophouden met een}{zachte prevelstem}\\

\haiku{Toen ik bij haar kwam.}{zat ze te schreien en wou}{niets van me weten}\\

\haiku{Maar ze hield steeds vol,....}{dat zij leefde en terug}{zou komen en ook}\\

\haiku{Ja, ja, dat ware,.}{wel ontzettend al was het}{niet aan hen te zien}\\

\haiku{Dus hoorde wellicht.}{al die pracht aan de arme}{gekke Hel\'ene}\\

\haiku{Dat was echter van '.}{t poeder en hoorde zoo}{voor de deftigheid}\\

\haiku{Maar hij begreep niet.}{wat ze allen aan elkaar}{te zeggen hadden}\\

\haiku{Veel gereisd - papa -.}{kostschoolhouder van alles}{wat opgevangen}\\

\haiku{Hij deed zijn best en {\textquoteleft}.}{zong uit den treure vanO}{moeder de zeeman}\\

\haiku{Hij vroeg of hij naar,.}{huis mocht gaan daar hij moe was}{en hier niet hoorde}\\

\haiku{De deur ging open, de.}{verpleegster kwam er uit en}{liet de deur open staan}\\

\haiku{Alleen de sterren.}{schitterden strak en klaar aan}{den zwarten hemel}\\

\haiku{Zijn twee kinderen,}{waren nog even bekoorlijk}{maar zij maakten hem}\\

\haiku{{\textquoteright} - {\textquoteleft}Maar dan moet u toch {\textquotedblleft}{\textquotedblright}.}{noodzakelijk mijn boek over}{de Magie lezen}\\

\haiku{Toen zij dit zei, wierp.}{zij een verwijtenden blik}{naar den professor}\\

\haiku{Geen woord, noch blik, noch.}{teeken verraadde dat zij}{Johannes kende}\\

\haiku{{\textquoteright} ~ Maar Johannes'.}{verlangen naar Markus werd}{dagelijks sterker}\\

\haiku{- {\textquoteleft}Wat miste je dan,,?}{dat je bij hen niet vond maar}{wel ergens anders}\\

\haiku{{\textquoteright} - - {\textquoteleft}Nu dan, die Satan,.}{loert altijd op ons als een}{wolf op de schapen}\\

\haiku{Hij vond het echter:}{beter dat onopgemerkt}{te laten en zei}\\

\haiku{{\textquoteright} Van Lieverlee dacht,.}{na terwijl hij Johannes}{strak bleef aankijken}\\

\haiku{Een klerkje aan den.}{overkant werd opmerkzaam en}{hield op met zijn werk}\\

\haiku{Ook van Lieverlee,.}{keek belangstellend eenigszins}{onder den indruk}\\

\haiku{{\textquoteright} - Hierbij knikte het.}{mannetje fier en zette}{zijn mutsje vaster}\\

\haiku{{\textquoteright} - {\textquoteleft}Juist,{\textquoteright} zei Wistik, {\textquoteleft}weet?}{je nog wat Markus zei van}{de herinnering}\\

\haiku{Het water spoelde.}{en klotste aldoor om de}{uitgeholde steenen}\\

\haiku{Hij voelde zich nu,.}{een held na het aandurven}{van den Octopus}\\

\haiku{Het, dat ook bij den,.}{vijver zat toen het arme}{meisje zich verdronk}\\

\haiku{{\textquoteright} Johannes kon niet.}{laten te huiveren toen}{hij Bangeling zag}\\

\haiku{{\textquoteright} riep de stem van het,.}{mannetje nu als heel uit}{de verte omhoog}\\

\haiku{- {\textquoteleft}Dat dacht je niet, wel,?}{dat wij hier zulk een goede}{verlichting hadden}\\

\haiku{Door een laag en nauw '.}{gangetje gingen zij naar}{t volgend nummer}\\

\haiku{Bij duizenden rooft.}{hij de mooiste exemplaren}{van mijn collectie}\\

\haiku{Toen kwam een zeer lang,:}{en al smaller toeloopend}{zaaltje waarop stond}\\

\haiku{Het lange zaaltje.}{met de bedjes liep al door}{en werd steeds enger}\\

\haiku{{\textquoteright} Johannes zweeg en.}{de andere twee spraken}{een tijdlang samen}\\

\haiku{Toen hij weg was, was.}{eenige oogenblikken een}{gedwongen stilte}\\

\haiku{Hij kwam half overeind '.}{en staarde naar de zee en}{toen weer naart duin}\\

\haiku{Maar Johannes liet.}{haar niet in en zeide dat}{hij alleen wou zijn}\\

\haiku{Marjon had hij niet.}{gezien en hij wist niet of}{zij vertrokken was}\\

\haiku{Hij tikte den man,.}{op den schouder maar deze}{verroerde zich niet}\\

\haiku{- {\textquoteleft}Maar gij zijt nog geen,?}{mensch wilt gij een priester des}{Allerhoogsten zijn}\\

\haiku{{\textquoteright} {\textquoteleft}Bedelen hebt gij,.}{ze geleerd en de roede}{kussen die hen sloeg}\\

\haiku{{\textquoteright} {\textquoteleft}Goed voor u is het,?}{dat Hij niet anders doet want}{waar was uw redding}\\

\haiku{Ook eten en drinken,.}{is niet slecht maar alleen voor}{wie het noodig hebben}\\

\haiku{Mij hat-tie,.}{z'n bord in me snuit gegooid}{kijk h{\"\i}er wat een snee}\\

\haiku{{\textquoteleft}Perdon! - mevrouw en.}{de kleine worden niet mee}{derin geviteerd}\\

\haiku{Dat werd aanleiding.}{om hem voorloopig hier}{af te zonderen}\\

\haiku{- {\textquoteleft}Wilt u ons toestaan,,.}{mijnheer uw schedelmaten}{even op te nemen}\\

\haiku{En zoudt ge anders?}{beslissen als ik niet ben}{wat ge brutaal noemt}\\

\haiku{En Johannes ging.}{toen snel en dapper weg eer}{de tranen kwamen}\\

\haiku{Hij zag Marjon naast,.}{hem met wijd-gesperde}{oogen van ontzetting}\\

\haiku{{\textquoteright} - {\textquoteleft}Maar ik heb gezien,,.}{wat je uitstond Markus dien}{ellendigen avond}\\

\haiku{Doch tegen den avond,,.}{toen Marjon kwam was het met}{Keesje gedaan}\\

\haiku{En nogmaals hief hij,,:}{aan zachter maar vlijmend en}{pijnlijk doordringend}\\

\haiku{Maar Markus werd, straf,.}{gebonden door een zijdeur}{naar buiten gevoerd}\\

\haiku{Marjon's ijzeren,.}{bedje dat gansch schudde als}{ze zich even bewoog}\\

\haiku{Het klonk als een psalm,.}{maar zoo schoon en ernstig als}{hij nimmer hoorde}\\

\haiku{Met die vlam willen.}{de menschen hun brandende}{liefde aanduiden}\\

\haiku{{\textquoteright} - {\textquoteleft}En Aeschylus,{\textquoteright} zei de, {\textquoteleft}.}{vaderbij Marathon werd}{zijn hand afgehakt}\\

\haiku{Het instrument is,.}{gebarsten en zal in kort}{geen toon meer geven}\\

\haiku{Het is me of ik.}{nog weken lang dag en nacht}{zou moeten vragen}\\

\haiku{- we moesten toch maar eens, '?}{probeeren wat het mes hier doen}{kan ist niewaar}\\

\haiku{We zullen eens zien,?}{of je die hand niet nog weer}{gebruiken kunt he}\\

\haiku{{\textquoteright} Markus stak zijn hand,,: - {\textquoteleft}!}{uit die zij beiden kusten}{en sprakTot weerziens}\\

\subsection{Uit: De nachtbruid}

\haiku{Het leeft nog maar als.}{onbeteekenend deel van}{een grooter leven}\\

\haiku{Luister dan, lieve,,.}{lezer met wat geduld en}{geef u wat moeite}\\

\haiku{Ik wilde w\'a\'ar voor,.}{mijn geld want ik geloofde}{aan rechtvaardigheid}\\

\haiku{een dagblad lectuur, -:}{een afgeluisterd gesprek}{soms lichamelijk}\\

\haiku{{\textquoteright} - {\textquoteleft}Dat is duidelijk,! -.}{vader Maar toch houd ik nog}{iets te vragen over}\\

\haiku{- {\textquoteleft}E\'en ding is me nu,,.}{duidelijk mijn jongen dat}{je gauw trouwen moet}\\

\haiku{Ik wist dat zij nog.}{leefde en ik wist ook den}{naam van ons landgoed}\\

\haiku{Ik zie je nog zoo.}{duidelijk alsof ik je}{gisteren verliet}\\

\haiku{Want dat ik liever, -!}{met God omhoog wou dan met}{Satan omlaag nu}\\

\haiku{Bekommer je niet{\textquoteright} - - {\textquoteleft}!}{om wat er van me terecht}{komtArme jongen}\\

\haiku{Op 't punt van 't,.}{water van de zee en van}{het zeil-vermaak}\\

\haiku{Maar dat zijn eenmaal.}{zoo onze Italiaansche}{uitbundigheden}\\

\haiku{Het ware woord, de,,}{juiste redeneering het}{sluitend taalverband}\\

\haiku{Maar je hoeft die nu.}{tegenover de directie}{niet meer te spelen}\\

\haiku{de oplossing van.}{het geheim onzes levens}{ligt in den droom}\\

\haiku{{\textquoteleft}Dag Vico mio!,{\textquoteright} En, '.}{het was zijn stem meer nog dan}{t zijn gezicht was}\\

\haiku{Omdat ik toen eerst.}{zelf geseind had om naar haar}{te informeeren}\\

\haiku{Ik kwam langs boomen,.}{en groen en nam alles scherp}{en duidelijk waar}\\

\haiku{Ik verwaarloosde,}{mijn dagelijksch werk daarom}{niet integendeel}\\

\haiku{Links onder me was,.}{een geweldige afgrond}{ook een bergverschiet}\\

\haiku{We onderscheidden,.}{de menschen op den steenen pier}{die in zee uitstak}\\

\haiku{{\textquoteright} riep hij in 't Fransch,, {\textquoteleft}.}{toen het meisje voorbij was}{die je gister zag}\\

\haiku{Zoo ras als ik mij.}{een dag vrij kon maken ging}{ik weer uit zeilen}\\

\haiku{{\textquoteright} zei Elsje.  {\textquoteleft}Dat.}{geeft weer een heele week stof}{tot conversatie}\\

\haiku{{\textquoteright} - {\textquoteleft}Dat ik getrouwd ben?}{en een goede vrouw en}{vier kinderen heb}\\

\haiku{Maar ik weet niet of.}{het mij gelukken zal je}{dat te doen inzien}\\

\haiku{Ben je het oneens?}{met een der algemeene}{dingen die ik zei}\\

\haiku{met 't minste wat,}{je me geven wilt nu ik}{zoo oneindig meer}\\

\haiku{Niet Jan Baars, maar zijn,.}{zuster waar ik als kind aan}{huis genomen ben}\\

\haiku{Een laagheid van de,.}{soort waaruit ik mij juist met}{trots bevrijd voelde}\\

\haiku{Hij is geen vriend van,.}{tranen en geeft den zwaarmoed}{niet gaarne vat}\\

\haiku{{\textquoteright} en ik begreep dat.}{hij bedoelde dat ik mijn}{titel niet meer had}\\

\haiku{{\textquoteright} Toen sloeg Elsje haar:}{beide armen om mij heen}{en riep vreugdevol}\\

\haiku{Als wij kinderen.}{zijn houden we vader en}{moeder voor volmaakt}\\

\haiku{Maar ze deden niets,.}{spoedig weer in hun eigen}{belangen verdiept}\\

\haiku{Maar in mij woonde.}{een somber v\'o\'orgevoel met}{strakke zekerheid}\\

\haiku{als ons kindje wat, -?}{grooter is dat we weer in}{Holland gaan wonen}\\

\haiku{Ik ging naar buiten.}{en zag de blauwe lucht en}{een heerlijk landschap}\\

\subsection{Uit: Sirius en Siderius}

\haiku{Taede sloot het raam, '.}{stak een kaars op en ging met}{het licht naart bed}\\

\haiku{Moeder moest wel zeer.}{moe zijn om bij dit gerucht}{maar d\'o\'or te slapen}\\

\haiku{Neen, daar stond ze nog,, ',.}{rechtop midden int pad}{opziend naar zijn raam}\\

\haiku{Verschillend waren.}{ze van grootte en allen}{vreemd en bont gekleed}\\

\haiku{{\textquoteright} klonk achter hem het.}{hijgend-angstig fluisteren}{van zijn geleidster}\\

\haiku{Weet je wel tot wie '? -{\textquoteright} - {\textquoteleft} '?}{jet hebtZou ik anders}{doen als ikt wist}\\

\haiku{'t Is nu alles -....}{voor zijn verantwoording en}{van die anderen}\\

\haiku{Maar dat zag niemand.}{anders en neem ik niet op}{mijn verantwoording}\\

\haiku{{\textquoteleft}Ik heb een zwager,.}{die is leidekker en die}{heet Jezus Christus}\\

\haiku{Aan den wegkant was.}{een trog waarin een frissche}{waterstraal stroomde}\\

\haiku{'t Is goed dat hij,.}{gekomen is want er wordt}{met smart op gewacht}\\

\haiku{{\textquoteright} - {\textquoteleft}God zal hem net zoo,{\textquoteright}.}{lang laten leeven als hij}{noodig is zei Enna}\\

\haiku{De mooie liverei,,}{was erg vuil geworden zelfs}{het roode gladde}\\

\haiku{Boem! - klonk het van de, - -!}{nu nabije stad een dreunend}{kanonschot en Boem}\\

\haiku{{\textquoteright} - {\textquoteleft}Gun het ze dan, dat,{\textquoteright}.}{ze nu eens wijzer doen dan}{ze zijn zei Enna}\\

\haiku{Een officier te.}{paard kwam aanrijden en vroeg}{wat dat beduidde}\\

\haiku{En toen het gelukt, {\textquoteleft}!}{was juichten de passagiers}{en riepenhoera}\\

\haiku{Angstig luisterde.}{hij of Enna kreunde of}{Sirius schreidde}\\

\haiku{Toen bezon hij zich,, {\textquoteleft}!}{dat hij niet alleen wilde}{zijn en riepEnna}\\

\haiku{Het is alles veel.}{mooier en heerlijker dan}{het ooverdag zou zijn}\\

\haiku{De bedoeling van,,:}{zijn gebaar verstonden de}{ouders beiden als}\\

\haiku{{\textquoteleft}ik houd wel van een,.}{grapje maar ik laat me niet}{voor den mal houden}\\

\haiku{Daarom is hij de.}{eenige mensch op  aarde}{waar ze bang voor is}\\

\haiku{Zoo is dit schip een,.}{groot lijf bestuurd door een Ziel}{die niet het schip is}\\

\haiku{tot aan het toplicht,.}{dat als een kleine ster juist}{boven haar uitscheen}\\

\haiku{Maar ik geloof toch.}{niet dat de Oceaan deeze}{maal oover-wonnen is}\\

\haiku{Waarom verdient gij?}{het meer dan die allen die}{verongelukt zijn}\\

\haiku{- {\textquoteleft}U moest u schamen,,.}{mijnheer al deze menschen}{zoo te ontstemmen}\\

\haiku{{\textquoteright} {\textquoteleft}Ja, als de mist wil,{\textquoteright}.}{opklaren zei de jonge}{stuurman voorzichtig}\\

\haiku{{\textquoteright} Taede keek hem strak,.}{aan zonder te doen blijken}{dat hij hem verstond}\\

\haiku{Ik ben niet meer noodig,,.}{ik heb afgedaan maar jij}{moet nog beginnen}\\

\haiku{Je hebt zeker nog '.}{niet geslapen sinds wij van}{t schip afvoeren}\\

\haiku{Op zijn tijd zal het.}{donkere vlekje aan de}{kim wel verrijzen}\\

\haiku{Hij gaat den Herder.}{zoeken dien de aarde uit}{zicht verlooren heeft}\\

\haiku{- {\textquoteleft}Neen,{\textquoteright} zei de oude, {\textquoteleft},.}{ik ben niet blind maar de zon}{is niet meer zoo sterk}\\

\haiku{Maar mijn vrouw, die de,:}{wijste is van  alle}{menschen die zei mij}\\

\haiku{- {\textquoteleft}Het andere kind,.}{had ik \'o\'ok meer gezien maar}{ik weet niet meer w\'a\'ar}\\

\haiku{Maar voor mij, voor mijn.}{vrouw en mijn kind zijn het geen}{gewoone menschen}\\

\haiku{En ik begrijp nu,.}{dat het altijd om dat kind}{te doen is geweest}\\

\haiku{Ik geloof zelfs dat,.}{hij haar alleen getrouwd heeft}{omdat ik koomen moest}\\

\haiku{Zij waarschuuwde ons.}{ook als er booze plannen teegen}{ons beraamd werden}\\

\haiku{Ik begreep nu dat.}{mijn moeder samen met mij}{verbranden wilde}\\

\haiku{{\textquoteleft}Belet hem niet mij.}{van kant te maken als hem}{dat verligting geeft}\\

\haiku{Vier uuren duurde de,.}{spanning maar zij scheenen kort}{in Taede's gevoel}\\

\haiku{Maar daaruit besloot.}{zij ook dat hij nog onder}{de leevenden was}\\

\subsection{Uit: Van de koele meren des doods}

\haiku{De vorm van het brood,.}{dat elken morgen op de}{ontbijttafel lag}\\

\haiku{En toch was zij meest.}{van allen afkeerig van}{naar stad teruggaan}\\

\haiku{Er was iets boos en.}{verkeerds ingeslopen en}{zij begreep het niet}\\

\haiku{Zij liep door den gang,.}{over het witte hart bleef even}{staan en glimlachte}\\

\haiku{Zij dacht aan alles,.}{met genot en kon het niet}{genoeg herdenken}\\

\haiku{En Hedwig knikte dan ' {\textquoteleft}!}{vrindelijkt eerste en}{zeiDag grijsjasjes}\\

\haiku{De kinderen van,.}{het dorp waren onthaald zooals}{dat gewoonte was}\\

\haiku{Op school had zij les.}{van een jongen meester die}{veel aan kiespijn leed}\\

\haiku{De ernstige schijn.}{der dingen werd tegen den}{avond strak en somber}\\

\haiku{Wat leek het oude,.}{leven zalig waaraan die}{klank herinnerde}\\

\haiku{Hij zat plat neer met.}{gevouwen knie\"en en leek}{veel jonger dan Hedwig}\\

\haiku{ik wou dat ik met,.}{alle jongens met alle}{menschen zoo doen kon}\\

\haiku{Zwart en kaal ook de.}{heester-skeletten rondom}{langs den ouden muur}\\

\haiku{Het was in dezen.}{levenstijd dat zij muziek}{begon te verstaan}\\

\haiku{{\textquoteleft}Mijn zuster is zoo,{\textquoteright},.}{ziek zei hij niet wetend of}{dit onoprecht was}\\

\haiku{Ze kneep haar vingers {\textquoteleft}!}{in elkaar en prevelde}{Was ik toch maar dood}\\

\haiku{O neen, het was een.}{veel ouder en sterker en}{dierbaarder man}\\

\haiku{{\textquoteright} {\textquoteleft}Maar waarom doe je,?}{dan zoo anders den eenen dag}{of den andere}\\

\haiku{Maar had hem toen, in,}{dat uur iemand herinnerd}{of medegedeeld}\\

\haiku{nu bleef 't weg, of ',.}{t kwam door gelijkenis}{als schaduw of echo}\\

\haiku{En dat, nu zij toch!}{in liefde leefde en een}{heilstaat voor zich zag}\\

\haiku{Maar doodgaan scheen haar,.}{altijd nog veel beter nog}{veel begeerlijker}\\

\haiku{Zij wilde daarvan.}{zekerheid en beproefde}{zich op te richten}\\

\haiku{Eindelijk reed een,.}{wagentje aan met een boer}{en een veldwachter}\\

\haiku{Toen poogde zij te,.}{doen zooals de vromen doen en}{aan God te denken}\\

\haiku{{\textquoteright} En zij dacht hoe zij.}{zelve z\'o\'o liggen zou en}{gevonden worden}\\

\haiku{Alleen Johan had het.}{er mooi gevonden en hield}{van huis en menschen}\\

\haiku{, zij zag niets waaraan.}{zich vast te klemmen om niet}{weder te zinken}\\

\haiku{De oudste broeder,.}{was over zee Aernout in}{een andere stad}\\

\haiku{Toen besloot Gerard,,.}{door zorg en leed gedreven}{zich te vermannen}\\

\haiku{Al wat Gerard van,,.}{Ritsaart wist was hem tegen}{en leek ongunstig}\\

\haiku{Hij ging naar Hedwig en.}{bood geen weerstand meer aan de}{zoete betoovering}\\

\haiku{Deze ontdekking}{echter verruimde hem en}{stemde hem zachter}\\

\haiku{Maar word goed verliefd,.}{en bind dan want het rechte}{komt maar \'e\'ens goed}\\

\haiku{De maaltijden hield,.}{hij bij Joob daarvan was hij}{niet af te brengen}\\

\haiku{Dit is Monica,{\textquoteright}, {\textquoteleft},.}{zei Joobvulgo Jansje mijn}{beschermengel}\\

\haiku{Maar toen Ritsert haar.}{daarop aanzag begreep zij}{wat zij gezegd had}\\

\haiku{{\textquoteright} - - {\textquoteleft}Ja, maar, het \'e\'ene,.}{uur vind ik d{\'\i}t beter het}{andere uur d\'at}\\

\haiku{{\textquoteright} - - {\textquoteleft}Maar kan ik dan niet '?}{eeuwig heil verspelen door}{n verkeerde keus}\\

\haiku{Dit beangste hem, '?}{zeer zou Hedwig nu gaan baden}{en int duister}\\

\haiku{Snel tilde hij haar ',.}{uitt water het slappe}{hoofd op zijn schouder}\\

\haiku{{\textquoteright} Gerard zweeg, het hoofd,.}{afwendend zijn woede en}{afkeer verbijtend}\\

\haiku{En hoe kon in 't?}{hart der doodzonde de weg}{tot God zich openen}\\

\haiku{Maar dit kon zij niet,.}{ontwarren het was te zeer}{ineengevlochten}\\

\haiku{Omdat hij voor de.}{daglooners niet veel slechter zorgt}{dan voor zijn paarden}\\

\haiku{Het huisje lag hoog,.}{en alleen op open klip aan}{een inham der zee}\\

\haiku{Alles leek mooier,.}{en dierbaarder thuis zelfs de}{zee en de wolken}\\

\haiku{Zij was zoo overtuigd.}{en welsprekend dat Janet}{zich om liet stemmen}\\

\haiku{Daarna ging zij er '.}{mee int rijtuig en nam}{den trein naar Londen}\\

\haiku{{\textquoteright} - Toen wilde hij uit.}{minzaamheid Hedwigs koffertje}{op het rek zetten}\\

\haiku{{\textquoteleft}Geef me je beurs om,{\textquoteright}.}{den koetsier te betalen}{zei de vagebond}\\

\haiku{Niemand had kunnen.}{uitvorschen wie zij was of}{waar zij vandaan kwam}\\

\haiku{Zij at te middag,.}{bij den docter en zij was}{wat opgewekter}\\

\haiku{Er was een planken,,.}{vloer een ijzeren bed en}{twee stoelen meer niet}\\

\haiku{\'e\'en was, al voelde.}{zij tevens dat zij hem nooit}{meer mocht terugzien}\\

\haiku{Ik werd vreeselijk.}{weemoedig en moest aldoor}{aan Gerard denken}\\

\haiku{Het was zoo heerlijk.}{te hooren en hij nam mij}{mee naar een concert}\\

\haiku{Maar ik vond een klein.}{oud kerkje in een oude}{stille achterbuurt}\\

\haiku{Hedwig keek even \'op, met,,.}{snellen schuwen blik benieuwd}{wie z\'o\'o spreken kon}\\

\haiku{Grooter kinderen,,.}{zijn altijd nog zelfzuchtig}{hebberig twistziek}\\

\haiku{Het hoogste wezen,?}{is de hoogste vreugde wat}{kan het anders zijn}\\

\haiku{Maar het eigene.}{en persoonlijke aan ons}{is het meest kwetsbaar}\\

\haiku{{\textquoteright} - {\textquoteleft}Ook wel omdat ik,.}{mij schaamde omdat ik mij}{zelve redden wou}\\

\haiku{Elke dag probeer.}{ik nu den raad van zuster}{Paula te volgen}\\

\haiku{Nu zal God zich wel.}{weer van mij afkeeren en ik}{zal weer gek worden}\\

\haiku{Ik heb twee dagen,.}{op mijn kamer doorgebracht}{in angst en jammer}\\

\haiku{{\textquoteright} - {\textquoteleft}Is men in Holland?}{niet streng in de eischen voor}{liefde-zusters}\\

\haiku{{\textquoteright} - {\textquoteleft}Maar, mijn zuster, als,.}{het niet zoo ware dan zou}{het een wonder zijn}\\

\haiku{Het eigene sterft,.}{alleen door het leven niet}{door den lijfs-dood}\\

\haiku{{\textquoteright} - {\textquoteleft}Zouden wij dan na?}{den dood weer een nieuw leven}{moeten beginnen}\\

\haiku{Zij had dien al zoo, '.}{lang overdacht zij had hem maar}{voort opschrijven}\\

\haiku{Zij stond om half zes.}{op en dronk dan koffie met}{Harmsen en de vrouw}\\

\haiku{Ik schrijf nu al een,.}{jaar aan mijn boek en ben nog}{niet eens recht op gang}\\

\haiku{Van dat dagboek wordt.}{gezegd dat dit de oorsprong}{is van de roman}\\

\haiku{Pas jaren later,,.}{op 30 maart 1918 schreef zij hem}{in een brief erover}\\

\subsection{Uit: Van de koele meren des doods}

\haiku{De vorm van het brood.}{dat elken morgen op de}{ontbijttafel lag}\\

\haiku{En toch was zij meest.}{van allen afkeerig van}{naar stad teruggaan}\\

\haiku{Er was iets boos en.}{verkeerds ingeslopen en}{zij begreep het niet}\\

\haiku{Zij liep door den gang,.}{over het witte hart bleef even}{staan en glimlachte}\\

\haiku{Zij dacht aan alles,.}{met genot en kon het niet}{genoeg herdenken}\\

\haiku{En Hedwig knikte dan ' {\textquoteleft}!}{vrindelijkt eerste en}{zeidag grijsjasjes}\\

\haiku{De kinderen van,.}{het dorp waren onthaald zooals}{dat gewoonte was}\\

\haiku{De ernstige schijn.}{der dingen werd tegen den}{avond strak en somber}\\

\haiku{Wat leek het oude.}{leven zalig waaraan die}{klank herinnerde}\\

\haiku{Hij zat plat neer met.}{gevouwen knie\"en en leek}{veel jonger dan Hedwig}\\

\haiku{ik wou dat ik met,.}{alle jongens met alle}{menschen zoo doen kon}\\

\haiku{Zwart en kaal ook de.}{heester-skeletten rondom}{langs den ouden muur}\\

\haiku{Zij kwam niet, maar dat '.}{zij niet komen zou wist zij}{niet eert avond werd}\\

\haiku{Het was in dezen.}{levenstijd dat zij muziek}{begon te verstaan}\\

\haiku{{\textquoteleft}Mijn zuster is zoo,{\textquoteright},.}{ziek zei hij niet wetend of}{dit onoprecht was}\\

\haiku{Ze kneep haar vingers {\textquoteleft}!}{in elkaar en prevelde}{Was ik toch maar dood}\\

\haiku{O neen, het was een.}{veel ouder en sterker en}{dierbaarder man}\\

\haiku{{\textquoteright} {\textquoteleft}Maar waarom doe je,?}{dan zoo anders den eenen dag}{of den andere}\\

\haiku{Maar had hem toen, in,}{dat uur iemand herinnerd}{of medegedeeld}\\

\haiku{nu bleef 't weg, of ',.}{t kwam door gelijkenis}{als schaduw of echo}\\

\haiku{En dat, nu zij toch!}{in liefde leefde en een}{heilstaat voor zich zag}\\

\haiku{Maar doodgaan scheen haar,.}{altijd nog veel beter nog}{veel begeerlijker}\\

\haiku{Zij wilde daarvan.}{zekerheid en beproefde}{zich op te richten}\\

\haiku{Eindelijk reed een,.}{wagentje aan met een boer}{en een veldwachter}\\

\haiku{Toen poogde zij te,.}{doen zooals de vromen doen en}{aan God te denken}\\

\haiku{{\textquoteright} En zij dacht hoe zij.}{zelve z\'o\'o liggen zou en}{gevonden worden}\\

\haiku{Alleen Johan had het.}{er mooi gevonden en hield}{van huis en menschen}\\

\haiku{De oudste broeder,.}{was over zee Aernout in}{een andere stad}\\

\haiku{Toen besloot Gerard,,.}{door zorg en leed gedreven}{zich te vermannen}\\

\haiku{Al wat Gerard van,,.}{Ritsaart wist was hem tegen}{en leek ongunstig}\\

\haiku{Hij ging naar Hedwig en.}{bood geen weerstand meer aan de}{zoete betoovering}\\

\haiku{Deze ontdekking}{echter verruimde hem en}{stemde hem zachter}\\

\haiku{Maar word goed verliefd,.}{en bind dan want het rechte}{komt maar \'e\'ens goed}\\

\haiku{De maaltijden hield,.}{hij bij Joob daarvan was hij}{niet af te brengen}\\

\haiku{Maar toen Ritsert haar.}{daarop aanzag begreep zij}{wat zij gezegd had}\\

\haiku{{\textquoteright} - - {\textquoteleft}Maar kan ik dan niet '?}{eeuwig heil verspelen door}{n verkeerde keus}\\

\haiku{Dit beangste hem, '?}{zeer zou Hedwig nu gaan baden}{en int duister}\\

\haiku{Snel tilde hij haar ',.}{uitt water het slappe}{hoofd op zijn schouder}\\

\haiku{En hoe kon in 't?}{hart der doodzonde de weg}{tot God zich openen}\\

\haiku{Maar dit kon zij niet,.}{ontwarren het was te zeer}{ineengevlochten}\\

\haiku{Omdat hij voor de.}{daglooners niet veel slechter zorgt}{dan voor zijn paarden}\\

\haiku{Alles leek mooier,.}{en dierbaarder thuis zelfs de}{zee en de wolken}\\

\haiku{Zij was zoo overtuigd.}{en welsprekend dat Janet}{zich om liet stemmen}\\

\haiku{Daarna ging zij er '.}{mee int rijtuig en nam}{den trein naar Londen}\\

\haiku{{\textquoteright} - Toen wilde hij uit.}{minzaamheid Hedwig's koffertje}{op het rek zetten}\\

\haiku{Niemand had kunnen.}{uitvorschen wie zij was of}{waar zij vandaan kwam}\\

\haiku{Zij at te middag,.}{bij den docter en zij was}{wat opgewekter}\\

\haiku{Er was een planken,,.}{vloer een ijzeren bed en}{twee stoelen meer niet}\\

\haiku{\'e\'en was, al voelde.}{zij tevens dat zij hem nooit}{meer mocht terugzien}\\

\haiku{Dan ging zij op de,.}{drukke straat meenend weerstand}{te kunnen bieden}\\

\haiku{Ik werd vreeselijk.}{weemoedig en moest aldoor}{aan Gerard denken}\\

\haiku{Het was zoo heerlijk.}{te hooren en hij nam mij}{mee naar een concert}\\

\haiku{Maar ik vond een klein.}{oud kerkje in een oude}{stille achterbuurt}\\

\haiku{Hedwig keek even \`op, met,,.}{snellen schuwen blik benieuwd}{wie z\'o\'o spreken kon}\\

\haiku{Grooter kinderen,,.}{zijn altijd nog zelfzuchtig}{hebberig twistziek}\\

\haiku{Het hoogste wezen,?}{is de hoogste vreugde wat}{kan het anders zijn}\\

\haiku{Maar het eigene.}{en persoonlijke aan ons}{is het meest kwetsbaar}\\

\haiku{{\textquoteright} - {\textquoteleft}Ook wel omdat ik,.}{mij schaamde omdat ik mij}{zelve redden wou}\\

\haiku{Nu zal God zich wel.}{weer van mij afkeeren en ik}{zal weer gek worden}\\

\haiku{Ik heb twee dagen,.}{op mijn kamer doorgebracht}{in angst en jammer}\\

\haiku{{\textquoteright} - {\textquoteleft}Is men in Holland?}{niet streng in de eischen voor}{liefde-zusters}\\

\haiku{{\textquoteright} - {\textquoteleft}Maar, mijn zuster, als,.}{het niet zoo ware dan zou}{het een wonder zijn}\\

\haiku{{\textquoteright} - {\textquoteleft}Zouden wij dan na?}{den dood weer een nieuw leven}{moeten beginnen}\\

\haiku{Zij had dien al zoo, '}{lang overdacht zij had hem maar}{voort opschrijven}\\

\haiku{Zij stond om half zes.}{op en dronk dan koffie met}{Harmsen en de vrouw}\\

\section{Georges Eekhoud}

\subsection{Uit: Kees Doorik of een bloedig half-vasten}

\haiku{Kees verzamelde.}{het alaam in een hoek van de}{schuur en sloot de deur}\\

\haiku{Met een zichtbare,}{ingenomenheid vernam}{hij welk een klein eter}\\

\haiku{Nelis Cramp kwam hem.}{voor als de vrijgevigste}{van \`al de bazen}\\

\haiku{Op zijn vijftiende.}{jaar was Kees Doorik reeds een}{kranige jongen}\\

\haiku{Ze liet ieder maar,.}{razen en verloor daarom}{noch eetlust noch lach}\\

\haiku{Daar ze hun pijpen,.}{stopten uit een varkensblaas}{gaf Paulien wat vuur}\\

\haiku{Bijna op het bed,.}{uitgestrekt  draaide zij}{den rug toe naar Kees}\\

\haiku{'k Geloof toch, dat '!}{n knecht zoowel bloed in z'n lijf}{heeft als zijn meester}\\

\haiku{Kees was weg om het,,.}{veld van IJlwaal aan de}{Schelde te beren}\\

\haiku{'t Is markt morgen,...?}{en ik kwam uw boodschappen}{halen Geen belet}\\

\haiku{Wij hebben nog maar...}{de rogge en de toemaat}{binnen te halen}\\

\haiku{Mie, wa' zijt-de ',.}{gijn gelukkige het}{gelukskind zelf}\\

\haiku{hernam hij, terwijl.}{ze koffie inschonk en een}{boterham smeerde}\\

\haiku{Daar zijn nog genoeg...}{vondelingen en bastaards}{na hem te vinden}\\

\haiku{maar gij zijt toch niet,...}{daaronder te rekenen}{veronderstel ik}\\

\haiku{{\textquoteleft}Er is een middel:}{om dien zeldzamen knecht in}{uw dienst te krijgen}\\

\haiku{De aanhouding van ',;}{nen Duitschen smokkelaar door}{de tolbeambten}\\

\haiku{Jurgen Faas riep den, ':}{man met het varken toenen}{daglooner van Stabroeck}\\

\haiku{Bij dit spelleken '.}{barstte gansch de zaal haast van}{t danig lachen}\\

\haiku{maar hij beloofde,.}{zich vast er niet \'al te kort}{spel mee te maken}\\

\haiku{Ze vertrok met haar}{broer  Tist en den van haar}{niet weg te slagen}\\

\haiku{Kees antwoordde met ',,:}{nen vloek en vertrok na een}{onzedig gebaar}\\

\haiku{'t Is Jurgen, die... '}{zijn lierenaar tegen mij}{heeft uitgetrokken}\\

\haiku{Tranen welden op,.}{in den vervloekte zijn oogen}{en hij boog het hoofd}\\

\section{J.K. van Eerbeek}

\subsection{Uit: Lichting '18}

\haiku{Hij staat daar als een,.}{kind dat een prachtig geschenk}{heeft ontvangen}\\

\haiku{Een effen gezicht,.}{dat over de verbazing van}{den H.B.S.-er heenziet}\\

\haiku{Als een reus, zoo gaat.}{die aandacht staan boven de}{doffe schoolbanken}\\

\haiku{Het is of hij er.}{maar amper het zwarte bord}{mee durft te raken}\\

\haiku{hij glimlacht, grijnst, duikt,.....}{diep weg achter zijn eigen}{masker verslagen}\\

\haiku{De klas werkt weer, lang.}{niet ieder heeft begrepen}{wat er gebeurd is}\\

\haiku{Op tafel staat nog,.}{het glas waar de heer Speckmans}{uit gedronken heeft}\\

\haiku{een verschil, dat men.}{niet verklaren kan uit het}{groeiproces alleen}\\

\haiku{Twee tegenstanders,.}{die ieder door hun eigen}{angst verslagen zijn}\\

\haiku{dat is links-om, zoover,{\textquoteright}....}{ik weet had Toon Homan toen}{listig geantwoord}\\

\haiku{Van Veen,{\textquoteright} zei hij, {\textquoteleft}kun?}{je me niet eens aan een mud}{aardappels helpen}\\

\haiku{Een blauwe linnen;}{marktjas heeft hij zich om de}{schouders geschoven}\\

\haiku{Ik heb dat eenmaal....}{eens bij een man gezien op}{de markt te Nijkerk}\\

\haiku{en mijn tante ligt,.}{ziek versterkend eten is voor}{haar niet te krijgen}\\

\haiku{Het is al bijna,,....}{geen waarheid meer de waarheid}{die hij gezien heeft}\\

\haiku{Er begon zich een;}{vreemde beklemming op de}{menschen te leggen}\\

\haiku{snel overlegt hij, wat,.}{hem te doen staat wanneer een}{agent komt controleeren}\\

\haiku{En hij kende ook,,.}{een zekere trots van daar}{te staan waar hij stond}\\

\haiku{Van Toon's gedachten.}{uit stroomt het licht onder de}{grauwe pannen uit}\\

\haiku{De jonge Homan.}{probeert het eene gezegde}{na het andere}\\

\haiku{Hij houdt haar hand vast,;}{en strijkt met de beenen spoel van}{Coty over haar hand}\\

\haiku{bij 't begin...., zoover,.}{als ze bekend zijn staan de}{jaartallen erbij}\\

\haiku{Ik wed, dat je daar.}{de namen van deze muur}{op geschreven hebt}\\

\haiku{Daar kon hij goed mee,;}{uit want hij had een baard zoo}{hard als een spijker}\\

\haiku{dan zeilt hij van de;}{eene dolle situatie}{in de andere}\\

\haiku{Ik was al een paar;}{keer die week met m'n wagen}{de straat op geweest}\\

\haiku{Het komt me vreemd voor,.}{dat je met een kind in dit}{weer uit rijden gaat}\\

\haiku{om een kapotte.}{autoband zetten ze een}{volksfeest in mekaar}\\

\haiku{Kapitein Colff, de,;}{compagnies-commandant}{komt uit de barak}\\

\haiku{Er is niemand die....}{er bezwaar tegen heeft zijn}{lichaam te trainen}\\

\haiku{Enfin, Homan, als,....}{een oud model tank schommelt}{voorbij den ouwe}\\

\haiku{men beoefent het;}{schuinsrechts achterwaarts plaatsen}{van het rechterbeen}\\

\haiku{En de verhalen,.}{die over Wehmeyer in omloop}{zijn worden verteld}\\

\haiku{Hij voelt dat hij er,,;}{niet buiten kan buiten die}{zorg die risico}\\

\haiku{hij is niet zeker,,.}{of hij niet capituleeren}{zal op het eind}\\

\haiku{{\textquoteleft}Ik neem het je niet,....}{kwalijk dat je me gister}{uitgelachen hebt}\\

\haiku{{\textquoteleft}Ik begrijp niet,{\textquoteright} zegt,;}{hij als hij een tijdlang die}{kaart heeft bekeken}\\

\haiku{Ze kunnen me hier,.}{niet laten zitten met dat}{ijzer in me maag}\\

\haiku{hij let nu op zijn,,.}{oogen en daar ziet hij iets in}{dat hem niet aanstaat}\\

\haiku{Daarom gaat hij naar.}{den dokter en rapporteert}{zijn bevindingen}\\

\haiku{Er wordt besloten.}{dat men het nog drie dagen}{met hem zal probeeren}\\

\haiku{Hij licht een tip op,,.}{van de sluier maar er is}{niemand die nog ziet}\\

\haiku{Hij was er zich op,}{dat oogenbiik pijnlijk van}{bewust dat hij me\^e}\\

\haiku{In de kazerne.}{wordt met behoorlijk respect}{over voedsel gepraat}\\

\haiku{Homan begrijpt nu.}{waarom alle koks er zoo}{welgedaan uitzien}\\

\haiku{de leider is het,,.}{die het doel bepaalt en de}{verantwoording draagt}\\

\haiku{Er zijn er, die als;}{wild geworden inlanders}{met krissen zwaaien}\\

\haiku{De twee, van wie een,.}{naar de Distel zal moeten}{kijken elkaar aan}\\

\haiku{Het komt me voor, of,;}{ik deze kamer eerder}{heb gezien dacht hij}\\

\haiku{Ze bloosde, liet haar,,.}{viool glijden en wist niet}{wat te antwoorden}\\

\haiku{Het geval wilde,.}{namelijk dat hij geplaatst}{werd naar een kustwacht}\\

\haiku{{\textquoteleft}Eerste peleton,{\textquoteright}.}{allen present rapporteert}{de klasse-oudste}\\

\haiku{hij spreekt nog door, maar.}{zijn aandacht is al ginds op}{het kazerneplein}\\

\haiku{die een mengsel is.}{van de uitwaseming van}{menschen en leergoed}\\

\haiku{voor het feit dat ze.}{het niet eerder deden mag}{je God dankbaar zijn}\\

\haiku{Ik bedoel hier niet,;}{alleen de orthodoxen wat}{de godsdienst betreft}\\

\haiku{weet je, ik heb een,.}{gevoel of ik al jaren}{steenen gebakken heb}\\

\haiku{En werkelijk, men.}{kon ook niet anders dan er}{haar dankbaar voor zijn}\\

\haiku{men ziet de zoom zacht;}{in de schaduw achter de}{muur wegdoezelen}\\

\haiku{Voor het geluk van....}{tallooze gezinnen vechten}{de diplomaten}\\

\haiku{Wij hebben geen schijn,....}{van de macht die noodig is om}{dat te beletten}\\

\haiku{Het was de avond van,.}{de eerste lentedag die}{het jaar gebracht had}\\

\haiku{Hij ziet zich zelf als.}{het ware zitten op die}{plek op de wereld}\\

\haiku{Over de doktoren.}{praten ze als jongens over}{hun onderwijzer}\\

\haiku{Op de bestofte,,.}{zolder die hij eerst over moest}{kwam al niemand meer}\\

\haiku{Wat beteekent deze,?}{kamer die telkens in zijn}{droomen terugkeert}\\

\haiku{De bewoners van;}{het hospitaal vormen een}{kleine maatschappij}\\

\haiku{het personeel, de.}{bezoekers behooren bij}{een andere groep}\\

\haiku{Nous venons te voir,,,.}{Zuus le dimanche tu sais.{\textquoteright}11}{Maar hij komt terug}\\

\haiku{Wie aangewezen,,.}{is weet meteen wat hij van}{zichzelf denken moet}\\

\haiku{Nu eens kraakt in de,.}{nacht het bed van den een dan}{dat van den ander}\\

\haiku{Zoo nu en dan kijkt, {\textquoteleft}{\textquoteright}.}{hij tusschen de stammen door}{deverstuiving in}\\

\haiku{Hij kijkt alsof hij.}{het beeld van de heide met}{zich mee wil nemen}\\

\haiku{achter de blinde,....}{muur moest een boschage staan die}{dempte het geluid}\\

\haiku{naar van Beek, die met.}{eindeloos geduld telkens}{weer zijn spuwglas grijpt}\\

\haiku{maar voor die er van,.}{gedronken heeft is de helft}{over de rand gestort}\\

\haiku{{\textquoteright} Wanneer men met hem,,.}{spreekt bedaart het schokken dat}{door zijn lichaam trekt}\\

\haiku{Er is niemand, die.}{den vreemdeling een verwijt}{maakt van zijn gedrag}\\

\haiku{En de aralia die,.}{in de conversatiezaal}{staat is gaan werken}\\

\haiku{Je denkt wel eens, dat;}{je te maken hebt met de}{menschen om je heen}\\

\haiku{Dan is het immers,,,....}{of die schedel die je brein}{omspant zoo ruim is}\\

\haiku{Even zakelijk heeft,.}{hij dit vastgesteld alsof}{het hem niet aanging}\\

\haiku{Alleen als hij op,.}{zijn rechterzij ligt kan hij}{soms amper ademen}\\

\haiku{Donker was het in,....}{de kamer en alleen in}{die nis zag hij iets}\\

\haiku{Zou het mogelijk,,?}{zijn dat een wond geneest en}{geen lidteeken nalaat}\\

\haiku{Het kan een feest zijn,,,.}{gewoon ergens te staan te}{ademen daar te zijn}\\

\haiku{een schot is het, dat.}{men z\'o\'o tusschen de huizen}{vandaan nemen kan}\\

\haiku{Hij kwam hier - en er.}{was een gruwelijk hiaat}{in zijn gedachten}\\

\haiku{Eigenlijk, zoo denkt,,.}{hij wanneer hij alleen is}{ben ik twee menschen}\\

\haiku{Zoo een laat de wel,....}{van zijn persoonlijkheid niet}{nooit heelemaal los}\\

\haiku{, maar er is zulk een....}{rare looden onrust over zijn}{wezen gekomen}\\

\haiku{{\textquoteleft}Je houdt eenvoudig,.}{het geld onder je en je}{schrijft het Maandag in}\\

\haiku{En.... er bestaat een,.}{aapachtige scepsis die}{men niet kwalijk neemt}\\

\haiku{Een van z{\`\i}jn lichting,,.}{een die naast hem zat in het}{gymnastieklokaal}\\

\haiku{En Homan herkent,.}{de glimlach waarmee hun hun}{plaats gewezen is}\\

\haiku{telkens knikt ze met,:}{het hoofd bij iedere keer}{dat ze een naam noemt}\\

\haiku{En dan praat Anje,.}{weer zoo lang tot ze opnieuw}{moet afscheid nemen}\\

\haiku{hij verwacht het een,.}{of ander van dit bezoek}{en weet zelf niet wat}\\

\haiku{een antwoord op de,.}{vraag die dat stilstaan daar voor}{de kleine vrouw is}\\

\haiku{Tot grooter afscheid,.}{reikt ze hem nu de hand dan}{waartoe ze hem groet}\\

\haiku{En het lichtste licht,,....}{dat er op de wereld is}{gaat er over lichten}\\

\haiku{Hij was vervuld van;}{een diep medelijden met}{de oude Anje}\\

\haiku{Moet het leven, dat,?}{hem straks weer ontvangen zal}{zoo geleefd worden}\\

\haiku{Alle liedjes die,;}{hem aangesproken hebben}{gaat hij begrijpen}\\

\haiku{Een die zijn geduld;}{tegen de verlate tram}{staat uit te meten}\\

\haiku{Op die manier leeft.}{iemand met die afspraak als}{met een daimonion}\\

\haiku{En het wezen van:}{die afspraak is vastgelegd}{in deze woorden}\\

\haiku{Een oogenblik luwt,.}{de kramp die het dier in de}{ingewanden snijdt}\\

\haiku{er leeft wel zeker;}{nog een herinnering in}{deze pupillen}\\

\haiku{Als een onttakeld,,.}{schip dat dronken op de stroom}{drijft zoo ligt het daar}\\

\haiku{{\textquoteleft}Onze Toon heeft z'n,{\textquoteright}.}{diploma onze Toon voor}{en onze Toon na}\\

\haiku{En het grappige,....}{was dat zijn oom zelf evenzoo}{in het feit niets zag}\\

\haiku{de warme roode.}{gloed achter de lichtbakken}{lokt hem als een kind}\\

\haiku{Deze stond bij het,....}{buffet en keek als hij door}{het raam naar buiten}\\

\haiku{dat na{\"\i}ever is,....}{naarmate wij grooter \'echec}{in ons leven zien}\\

\haiku{{\textquoteright} {\textquoteleft}U is schromelijk,{\textquoteright};}{onbillijk tegenover de}{kerk roept nu iemand}\\

\haiku{Kil was de lucht, de.}{gele bladen zaten aan}{de eikenstammen}\\

\haiku{Hij ziet den kleinen,.}{man op jacht in dit veld en}{in zijn leven ook}\\

\haiku{Niets zagen deze.}{oogen van een ander dan wat}{hem vernederde}\\

\haiku{Als hij een spoor had,,,....}{geweten waardoor zijn ziel}{gaaf had kunnen gaan}\\

\haiku{Want op zijn manier,.}{is hij een van die welke}{moeilijk vergeten}\\

\haiku{fosforiseert de,.}{grauwe kalk op de witte}{muur die gaat lichten}\\

\haiku{Een tweede lamp gleed.}{over de ruiten der kleine}{voorstadswinkels mee}\\

\haiku{Meer ruimte had hij;}{eigenlijk niet noodig gehad}{om in te wonen}\\

\haiku{Een donderslag als.}{het knallen van een zweep en}{de angst in het hart}\\

\haiku{Telkens vingen de....}{grauwe muren der huizen}{een moment de glans}\\

\haiku{Hij verlaagt zich er;}{toe zijn tegenstander als}{gangmaker te zien}\\

\haiku{Een gedachte heeft;}{zich vastgehaakt aan wat daar}{op het doek gebeurt}\\

\haiku{hij ziet dwars door de,;}{atmosfeer die zich om zijn}{schedel heen verdicht}\\

\haiku{Ook zij zouden zich.}{legeren aan de voet van}{deze torens}\\

\subsection{Uit: Strooschippers}

\haiku{Het was 'n schip zooals.... '}{er die dagen alleen op}{de Gulle voeren}\\

\haiku{Van een gedeelte;}{van de Holterdiek had ie}{et zand afgebrocht}\\

\haiku{as et knappen wil,, '....}{laat et knappen dan leggen}{we d'rn knoop bij}\\

\haiku{De volgende dag ',.}{hadden zet in de wind}{en ze moesten jagen}\\

\haiku{en z'n vrouw ook) - Man, ' '.}{ikeb van  de Gulle}{nog geen druppelezien}\\

\haiku{De grauwe vloer, die,....}{de dunne planken van het}{schip draagt glijdt en schuift}\\

\haiku{De flarden van dat.}{bidden keven daar onder}{de dichte hemel}\\

\haiku{Want hij wist, hoe stijf;}{die menschen op de luiken}{van hun schip stonden}\\

\haiku{Rustig en ernstig.}{voeren ze de haven van}{het stadje binnen}\\

\haiku{Daar lagen ze nu,.}{met twintig vijf en twintig}{andere schippers}\\

\haiku{En s'avonds zaten,.}{ze onder de linden die}{langs de kaai stonden}\\

\haiku{Of mijnheer hen niet, '.}{wat wegwijs kon makenoe}{ze daar mee aan moesten}\\

\haiku{- Mijnheer wilde wel,.}{zorgen dat dat in orde}{kwam met dat liggeld}\\

\haiku{En m'n pake komt, ' ' '....}{daar en praat wat mitr en}{eeftr wat egeven}\\

\haiku{Ze zit steeds voor 'et, '.}{bed wanneer ze alleen is}{en geen menscher ziet}\\

\haiku{Ze had noch voor de,;}{zegen noch voor de vloek van}{haar defect een naam}\\

\haiku{Maar de spoken, waar,;}{hij in de somp mee gespeeld}{had plaagden hem nu}\\

\haiku{En het maakt zooveel,.}{wind dat de kleeren wapperen}{hun om het lichaam}\\

\haiku{'t Is goed, da'k 'ier,,....}{sta dacht hij anders vlogen}{ze menander an}\\

\haiku{er voer geen tweede,.}{zoo mooi van de werf dat kan}{men eerlijk zeggen}\\

\haiku{- Wanneer laden we,,.}{moest hij den jongen vragen}{omdat die niet sprak}\\

\haiku{- Je moest me even naar,, -.}{de wal brengen Jouk moest zijn}{vader herhalen}\\

\haiku{hij haatte het schip.}{zooals hij het zijn vader deed}{op dat oogenblik}\\

\haiku{hij zag er niet meer,.}{van dan de stofwolk die zij}{achter zich opwierp}\\

\haiku{ik ben nu toch aan,.}{de winnende hand zei hij}{maar tegen zich zelf}\\

\haiku{Toen had de goede;}{Bernard hem met zijn bruine}{oogen lang aangezien}\\

\haiku{Doch het duurde lang,.}{die avond eer de harten tot}{elkander kwamen}\\

\haiku{Toen werd 'et me toch.}{een beetje bar en ik schoof}{m'n stoel achteruit}\\

\haiku{Die beweerde van,;}{hem dat hij rustte in de}{schoot van Delila}\\

\haiku{De dokter kwam, en.}{meende de ouders gerust}{te kunnen stellen}\\

\haiku{Op die zelfde plaat {\textquoteleft}{\textquoteright}.}{heeftDe Vrouw Geertje nog twee}{jaar gevaren}\\

\haiku{de herinnering.}{aan helden en veldslagen}{hindert daar geen mensch}\\

\haiku{De schoonheid van die;}{landstreek is niet vermetel}{en ze is niet schuw}\\

\haiku{Hij had lust, hem in.}{de schouder te vatten en}{overboord te zetten}\\

\haiku{- Och, kreeg Katrien tot, - '.}{bescheid we kennenem}{langer as vandaag}\\

\haiku{- Verbazing, afkeer,.}{ongeloof stonden op de}{gezichten geteekend}\\

\haiku{Ik 'eb er samen '.}{mit de ouwe heer zelf de}{schuit door eenepeuterd}\\

\haiku{Eerder wel 'ad ie ',;}{geen beenen genoegad om de}{stad in te komen}\\

\haiku{Hij keek overboord, en.}{zag de kleine schilfers in}{het water schimmen}\\

\haiku{As die 'n visschien,.}{vangen kan prakkizeert ie}{over z'n werk niet meer}\\

\haiku{As je denken, dat, '.}{ie geld in brengt komt ie mit}{n paar visschies thuis}\\

\haiku{De onrust was wel '.}{heelendal de baas daar in}{dat huis aant Atje}\\

\haiku{Een krampachtige.}{stilte had het rumoer in}{de ban geslagen}\\

\haiku{Louwrens opende de,.... -,.}{mond wilde zich weren Mooie}{thuuskomst begon hij}\\

\haiku{En de vrouw, door dit,.}{zwijgen misleid roddelde}{voort met heete oogen}\\

\haiku{Hij had een gat in.}{het ijs geslagen om aan}{water te komen}\\

\haiku{Dat was geen kwaaie, maar.}{die had met alle schippers}{en vrouwlui schik}\\

\haiku{Jouk had die heele.}{middag een onzekerheid}{over zich  gehad}\\

\haiku{Sporen door het ijs,.}{zaagden ze waar ze met de}{schepen door konden}\\

\haiku{En hij ging met een;}{paar menschen kleine dennen}{uit het bosch halen}\\

\haiku{- We komen door dit,.}{ongeluk te laat bij de}{turf mopperde Jouk}\\

\haiku{En vlugger dan ooit, {\textquoteleft}{\textquoteright}.}{een schuit gelost is kwam toen}{De Vrouw Geertje leeg}\\

\haiku{Je kon  zoo niet, ';}{zeggen wat voor apartigs je}{aanem bespeurden}\\

\haiku{De kleine kajuit,.}{leek een hok van duiven die}{op uitvliegen staan}\\

\haiku{- Ik zou niet weten '.}{waarom de jongen minder}{is asn ander}\\

\haiku{Maar toen hij op de,.}{plecht stond kwam hem een zwart stuk}{leer onder de voet}\\

\haiku{- Maak nou maar dat je,.}{je schepen door de brug krijgt}{zegt m'n grootvader}\\

\haiku{we'ier 'n glas bier van,,.}{zes cent zei Sjoerd stil houdend}{voor een klein caf\'e}\\

\haiku{De olie-lucht van het.}{tentzeil en de kille damp}{van de avond buiten}\\

\haiku{En begon tegen,.}{hem te spreken alsof hij}{een broer van hem was}\\

\haiku{Met verwonderde.}{gezichten drong men te hoop}{om de twistenden}\\

\haiku{- Oordeelt niet, opdat.... -,,}{gij niet geoordeeld wordt Ja}{ja zei Weerselo}\\

\haiku{een zwakke gloor van.}{de gensters over de golven}{sloop er over de grond}\\

\haiku{Een kleine mensch, die.}{door het donker verraden}{en vereenzaamd was}\\

\haiku{Zijn voet zweefde al,.}{boven de plank toen hij zich}{scheen te bedenken}\\

\haiku{de vader en de.}{zoon stonden als vijanden}{tegenover elkaar}\\

\haiku{- Misschien weet ik 'n, ',.}{middel om je teelpen}{zei hij toen langzaam}\\

\haiku{nu, hij is maar een,?}{schipper wat heeft hij in dit}{kantoor willen doen}\\

\haiku{Nog een minuut, en,;}{hij kan weer in de straat staan}{flitst net door hem heen}\\

\haiku{het licht, de loopers,,....}{de linnen zak en hij stort}{naar de deur terug}\\

\haiku{het ligt achter de,.}{deur die hij zooeven niet open}{heeft kunnen krijgen}\\

\haiku{Hij is naar het schip {\textquoteleft}{\textquoteright},.}{deVrouw Geertje geloopen}{om Abel te spreken}\\

\haiku{En.... bewoog er zich?}{een licht achter het venster}{van het bovenraam}\\

\haiku{Sluipen zoo menschen,?}{die een eerlijke zaak te}{verrichten hebben}\\

\haiku{Het kleine kille,,;}{vlekje dat in de iris glimt}{gaat grooter worden}\\

\haiku{Wat een vreemd, angstig.}{geklepper van vleugels was}{dat in het vertrek}\\

\haiku{- Hoe komt het, dat Abel?}{op dit oogenblik deze}{woorden invallen}\\

\haiku{Zoo, als Jouk daar zijn,:}{vader op de rug ziet vliegt}{hem een onrust aan}\\

\haiku{- Dus op die manier ' ' '?}{ad je anet geld voorn}{schip willen komen}\\

\haiku{- Nee, maar ik zou hypteek,.}{op de praam kunnen nemen}{antwoordde toen Abel}\\

\haiku{Enfin, we raken,.}{de kolk uit en op eens is}{m'n knecht verdwenen}\\

\haiku{Er was nu niets geen,.}{rood meer in de streep die over}{de horizon liep}\\

\haiku{Er stond een emmer,, '....}{water een fornuispot en}{eenalve pot brei}\\

\haiku{- Wel, - zei Louwrens, - en,?}{je stond er bij of je de}{ruzie niet aanging}\\

\section{Noud van den Eerenbeemt}

\subsection{Uit: De berenkuil}

\haiku{Als ik jou was, zou ' '.}{ik maarnsn oogje op}{m'n dochter houden}\\

\haiku{Was dat een reden?}{om Toke te verbieden}{met hem uit te gaan}\\

\haiku{{\textquoteright} Hij keek naar Greets bed,.}{dat tegenover dat van Ria}{onder het raam stond}\\

\haiku{Ze griste het uit.}{zijn handen en wikkelde}{het in het papier}\\

\haiku{{\textquoteright} {\textquoteleft}Nou, daar moeten we,{\textquoteright}.}{eerst nog eens over praten zei}{Pierre geschrokken}\\

\haiku{Pierre, trek gauw een...{\textquoteright} {\textquoteleft},{\textquoteright}.}{jas aan en ga kijken of}{jeOnzin zei hij}\\

\haiku{{\textquoteleft}Toen ik vanmiddag,.}{bij Olivier was heb ik daar}{Helga gesproken}\\

\haiku{Het appelboompje.}{in een hoek van het kleine}{grasveld stond in knop}\\

\haiku{Hij glimlachte, nam.}{een paar slokken en wendde}{zich weer tot Pierre}\\

\haiku{Op school...{\textquoteright} {\textquoteleft}Natuurlijk!}{wordt de school er weer met de}{haren bij gesleept}\\

\haiku{{\textquoteleft}Hoor 'ns mens, er is ' '.}{n tijd van komen en er}{isn tijd van gaan}\\

\haiku{{\textquoteleft}... ziet er voor ons ook,...{\textquoteright} {\textquoteleft}......{\textquoteright} {\textquoteleft}...}{niet zo best uit vader veel}{minder verdienen}\\

\haiku{Toen ze op de top,.}{waren bleven zij staan en}{keken om zich heen}\\

\haiku{{\textquoteleft}Luistert u ook al,?}{naar de kletspraatjes die je}{hier in de buurt hoort}\\

\haiku{De dokter nam zijn.}{sigaar uit zijn mond en blies}{langzaam de rook uit}\\

\haiku{{\textquoteleft}Ik ken nog een paar.}{jongens als Frans en het is}{overal hetzelfde}\\

\haiku{{\textquoteright} {\textquoteleft}En ik dacht, dat je...{\textquoteright} {\textquoteleft}...{\textquoteright}}{met die jongen van Olivier}{in de stad zouAch}\\

\haiku{Toch zou het leuk zijn!}{geweest om morgen in de}{zaak te vertellen}\\

\haiku{n Duits type met.}{zo'n brede mond en van die}{opgemaakte ogen}\\

\haiku{Geen papieren, geen... ',.}{vergunningent Land van}{de toekomst Pierre}\\

\haiku{n Instorting...{\textquoteright} {\textquoteleft}We,.}{dachten dat het ergens in}{de Oude Man was}\\

\haiku{{\textquoteleft}Het was precies aan,.}{de andere kant op de}{vijfhonderdveertig}\\

\haiku{Hij reed de hoek om.}{en zag Pie voor de winkel}{van zijn ouders staan}\\

\haiku{Hij trok langzaam zijn.}{leren jekker uit en hing}{hem over een  stoel}\\

\haiku{Je zorgt er voor, dat...{\textquoteright}}{je binnen een maand aan het}{werk bent of anders}\\

\haiku{je iedere schicht.}{zes meter verder in het}{kolenveld werken}\\

\haiku{Dacht je, dat het mij,?!}{lag dag in dag uit onder}{de grond te zitten}\\

\haiku{Pierre voelde, dat.}{er van beide kanten niets}{meer te zeggen viel}\\

\haiku{Gevoelens, die uit,.}{een ver onbegrepen land}{diep in hem stamden}\\

\haiku{Het is verdomme...{\textquoteright}}{elf uur en jij ligt nog langs}{de straat te slieren}\\

\haiku{Je vader heeft me,.}{verteld dat jij alles van}{de provo's afweet}\\

\haiku{In het bijzonder,.}{de toekomst van uw bedrijf}{mag ik wel zeggen}\\

\haiku{Lange tijd was het.}{de enige winkel in de}{omgeving geweest}\\

\haiku{Haar handen lagen.}{in haar schoot en de vingers}{omknelden elkaar}\\

\haiku{{\textquoteright} zei hij, {\textquoteleft}maar je kunt.}{je het beste maar in het}{gebeurde schikken}\\

\haiku{{\textquoteleft}Laten we eerst de...}{uitslag van de kikkerproef}{maar eens afwachten}\\

\haiku{{\textquoteleft}U begrijpt wel, dat.}{die snelheid met de grootste}{zorg is uitgekiend}\\

\haiku{{\textquoteright} Toen hij weer in de,.}{kamer kwam stond er een kop}{koffie voor hem klaar}\\

\haiku{Allerlei dromen,,...}{die zij voor de toekomst had}{gaan nu in rook op}\\

\haiku{Haar vingers streken,.}{even langs zijn gezicht tastten}{toen naar zijn handen}\\

\haiku{Kun je hem nu niet,?}{eens zo ver krijgen dat hij}{naar de kapper gaat}\\

\haiku{En waarom moet hij?}{eeuwig en altijd in een}{spijkerbroek lopen}\\

\haiku{{\textquoteleft}We zijn naar Maaseik.}{gereden en hebben daar}{wat rond gelopen}\\

\haiku{{\textquoteleft}Het was net als al,{\textquoteright}.}{die andere Belgische}{caf\'es zei Frans}\\

\haiku{Hij stond rechtop, zijn.}{handen in zijn zakken en}{knikte bedachtzaam}\\

\haiku{{\textquoteleft}Word niet kwaad op 'r... ' -!}{t Is niet allemaal haar}{schuld vergeet dat niet}\\

\haiku{{\textquoteleft}Als hij daar niet was,!}{geweest zouden we nog niet}{weten waar ze zit}\\

\haiku{Tot het kind er was,!}{kon ze beter maar net doen}{alsof ze gek was}\\

\haiku{We zouden naar de...{\textquoteright}}{film kunnen gaan of misschien}{ergens iets drinken}\\

\haiku{{\textquoteleft}Ik moet zeggen, dat!}{je me wel de stuipen op}{het lijf hebt gejaagd}\\

\haiku{M'n vader en m'n.}{grootvader hebben hier in}{de winkel gestaan}\\

\haiku{Ze zullen je een.}{voor een in de steek laten}{en naar ons komen}\\

\haiku{Daar komt bij, dat je}{vrouw en jij ook niet meer van}{de jongsten zijn en}\\

\haiku{Aan onze kant staan.}{specialisten klaar om}{aan het werk te gaan}\\

\haiku{Hij besefte zelf,:}{wel dat zoiets niet haalbaar}{was en zei haastig}\\

\haiku{Maar het is nog maar '.}{n heel klein kiertje en het}{zal niet lang duren}\\

\haiku{Ze besefte, dat '.}{hetn strohalm was waar ze}{zich aan vastklampte}\\

\haiku{Greet is direct na...{\textquoteright} {\textquoteleft}?}{het eten weggegaan naar de}{InstuifEn Toke}\\

\haiku{Misschien voelde die '.}{zich eenzaam en kwam hijn}{kop koffie halen}\\

\haiku{Hij loopt echt niet in,!}{zeven sloten tegelijk}{mevrouw Vasterman}\\

\haiku{{\textquoteright} In het schenken van.}{troost was Charles Cleophas}{nooit erg goed geweest}\\

\haiku{Ze was naar dokter.}{Tijsen geweest en had wat}{zitten napraten}\\

\haiku{Weer was er de spijt,...}{dat ze het niet ongedaan}{had kunnen maken}\\

\haiku{Ze schrok er zelf van,.}{durfde in het begin niet}{verder te denken}\\

\haiku{Tot een huwelijk;}{tussen Toke en Pie zou}{het wel nooit komen}\\

\haiku{Ze liet zich van haar,.}{kruk glijden liep naar het raam}{en bleef ervoor staan}\\

\haiku{Het had in lang niet.}{geregend en het zand was}{droog en fijn als stof}\\

\haiku{{\textquoteleft}Bij 'n hond of 'n,.}{kat weet je altijd nog wel}{wat-ie bedoelt}\\

\haiku{De zon weerkaatste.}{schitterend in een open plek}{vlakbij de oever}\\

\haiku{Dokter Tijsen zat.}{op de rand van het bed zacht}{met Giel te praten}\\

\haiku{{\textquoteright} zei hij en het was,.}{duidelijk dat  hij het}{tegen zich zelf had}\\

\haiku{En toen had op een.}{avond meneer De Wever voor}{de deur  gestaan}\\

\haiku{{\textquoteright} {\textquoteleft}Hij vindt, dat hij 't,{\textquoteright}.}{ergens anders beter heeft}{dan thuis zei Pierre}\\

\haiku{{\textquoteright} Mia hoorde zelf hoe.}{schutterig die woorden over}{haar lippen kwamen}\\

\haiku{Waarom heb je me,?}{nooit gezegd dat je er met}{Pie over hebt gepraat}\\

\haiku{Eerst wilde ik het, '.}{hem niet zeggen maar later}{heb ikt verteld}\\

\haiku{{\textquoteright} Mia probeerde haar.}{eigen ongerustheid weg}{te redeneren}\\

\haiku{Die mocht eens denken,!}{dat Pierre al van alles}{op de hoogte was}\\

\haiku{Ze keek hem vragend,.}{aan herinnerde zich het}{gesprek met Toke}\\

\haiku{{\textquoteright} De dokter wachtte,.}{tot zij zelf was gaan zitten}{nam toen ook een stoel}\\

\haiku{Dan ben ik dus niet!}{het enige lid van de Mia}{Vasterman Fanclub}\\

\haiku{{\textquoteright} Charles Cleophas,.}{keek naar de doeken die aan}{de muren hingen}\\

\haiku{{\textquoteleft}'n Paar staan al op, '}{onze lijst zie ik maar dat}{zoekt de juffrouw op}\\

\haiku{Het was duidelijk,.}{dat het Berk absoluut niet}{interesseerde}\\

\haiku{Hij droeg een witte,.}{spijkerbroek en een oude}{haveloze trui}\\

\haiku{Twee dagen zou 't, ' '...}{goed gaan maar dan ist weer}{t ouwe liedje}\\

\haiku{Vertel ze, dat je '.}{me gesproken hebt en dat}{t best met me gaat}\\

\haiku{Ik ben gelukkig...{\textquoteright} {\textquoteleft} -!}{nog niet te oud omTrouwen}{daar komt niks van in}\\

\haiku{{\textquoteleft}Jullie gaan wanneer, ',}{je uit school komt meteenn}{nachtje logeren}\\

\haiku{{\textquotedblleft}'t Hele leger.}{van Napoleon is zo}{op de been gebracht}\\

\haiku{De televisie.}{bleef uit omdat Toke er}{misschien last van had}\\

\haiku{Als we de dokter,.}{niet gauw waarschuwen hoeft hij}{niet meer te komen}\\

\haiku{Opeens doken er.}{recht voor hem twee verblindend}{witte lichten op}\\

\haiku{Toen haar vader de,.}{kamer uitging gleed er een}{traan over haar wangen}\\

\haiku{Gisteravond toen hij...{\textquoteright} {\textquoteleft}!}{naar ons toe wilde komen}{Godallemachtig}\\

\section{Homme Eernstma}

\subsection{Uit: Liefdedood}

\haiku{Zij had juist dat, wat,.}{hij miste maar in wezen}{toch wel in zich had}\\

\haiku{Gedachteloos liep.}{hij als vanzelfsprekend het}{pad naar de tuin af}\\

\haiku{{\textquoteright} {\textquoteleft}Mijn moeder heeft een,{\textquoteright}.}{staart was het afwezige}{antwoord van Wibe}\\

\haiku{Daarnaast lagen twee.}{grote vijgenbladen bij}{wijze van bordjes}\\

\haiku{Tegelijk greep hij.}{een handvol frambozen uit}{de bloempot naast hem}\\

\haiku{Hij gaf de eerste,.}{portie aan Eabeltsje om}{haar te kalmeren}\\

\haiku{Zij greep zijn piemel.}{en probeerde die bij haar}{binnen te werken}\\

\haiku{Bij haar zesde maand.}{rende ze van onder de}{vijgenboom vandaan}\\

\haiku{Als Wibe wilde,.}{kon hij een rijtuig huren}{en er heen rijden}\\

\haiku{Plechtig legde hij.}{de boeken op tafel naast}{het bord van Wibe}\\

\haiku{Onze Wibe wordt,.}{een kleine Darwin en trouwt}{met zijn achternicht}\\

\haiku{Ten slotte zette:}{hij ze met een wijze raad}{op het goede spoor}\\

\haiku{Haastig liep ze naar.}{de keuken terug om het}{nieuws te vertellen}\\

\haiku{probeerde ze de,.}{muziek uit te blazen als}{een brandende kaars}\\

\haiku{Tegen zijn gezicht.}{voelde en rook hij Elises}{bezwete schaamhaar}\\

\haiku{Hij mengde het met.}{een teugje van zijn jonge}{rode bourgogne}\\

\haiku{Meestal hebben.}{vrouwen van sluiswachters het}{vrij wat naar hun zin}\\

\haiku{Het was op de avond.}{van de ineenstorting van}{de New Yorkse beurs}\\

\haiku{Het is weer precies.}{hetzelfde als met die tram}{en de melkfabriek}\\

\haiku{Het was hem op dat.}{ogenblik ontgaan dat hij nog}{commissaris was}\\

\haiku{Heilige Maria,,{\textellipsis}{\textquoteright}.}{Moeder van God bid voor ons}{Verder kwam ze niet}\\

\haiku{De zuster kwam naar.}{beneden om te zeggen}{dat die weer dicht moesten}\\

\haiku{Aan de vaart viel de.}{ophaalbrug bij de sluis dicht}{met een doffe dreun}\\

\haiku{Op die manier stijgt.}{de roman in po\"ezie}{boven zichzelf uit}\\

\section{Justus van Effen}

\subsection{Uit: Brief van een bejaard man en Reis naar Zweden}

\haiku{Maar hoezeer ze mijn,.}{trots ook pijnigden temmen}{deden ze haar niet}\\

\haiku{Ik vorderde zo}{geweldig in het leren}{dat ik weldra v\'o\'or}\\

\haiku{Zij beantwoordt ze;}{in een taal die ver boven}{haar leeftijd uitgaat}\\

\haiku{naar om mij neer te.}{leggen bij de machtspreuken}{van mijn leermeesters}\\

\haiku{Ik leverde er.}{meer op twee bladzijden dan}{zij in een heel boek}\\

\haiku{Ik dacht dat de hand.}{van mijn godinnetje me}{genoeg had gezegd}\\

\haiku{Ik meende zelfs te.}{zien dat ze mij nog verder}{in haar gunst brachten}\\

\haiku{Mijn eerste bezoek,.}{viel me niet tegen hoewel}{het erg lang duurde}\\

\haiku{Zij hoeft hem alleen.}{maar te laten merken dat}{zij smaak in hem vindt}\\

\haiku{Zij begon niet te.}{schreeuwen of als een viswijf}{van zich af te slaan}\\

\haiku{Zij verdween met de.}{zwakke oorzaken die haar}{hadden voortgebracht}\\

\haiku{Maar het kostte haar.}{veel meer moeite om zich van}{mij los te maken}\\

\haiku{Ze vallen meteen.}{op als men niet aan deze}{bouwtrant gewend is}\\

\haiku{Zolang alles in:}{het Italiaans gebeurt kan}{het er nog mee door}\\

\haiku{Zij schenen blij met.}{zijn ongeluk en trots op}{zijn vernedering}\\

\haiku{Hun lijf was mager,,.}{maar gespierd hun gang krachtig}{hun houding kaarsrecht}\\

\haiku{Ik ben er trouwens.}{ook geen smeden of barbiers}{tegengekomen}\\

\haiku{We kregen dikwijls.}{kinderen van elf \`a twaalf}{jaar als postiljon}\\

\haiku{Er zijn hier kroegen,.}{noch herbergen behalve}{dan in de steden}\\

\haiku{Heel vaak gaat het hier.}{om ondeugden onder het}{mom van hero{\"\i}ek}\\

\haiku{Het schip droeg alle.}{sporen van de pracht van de}{Engelse natie}\\

\haiku{Daarna trokken zij.}{elkaar de manchetten en}{dassen van het lijf}\\

\haiku{Toen de koningin,.}{zag dat hij niets at vroeg ze}{hem naar de reden}\\

\haiku{Men dacht daarom dat,.}{hij ziek was wat weer nieuwe}{vragen uitlokte}\\

\haiku{Zelfs zijn geringste.}{bedienden ontbrak het niet}{aan goud en zilver}\\

\haiku{{\textquoteleft}Ik hoef van niemand,,{\textquoteright}.}{mijn plicht te leren mijnheer}{antwoordde de Deen}\\

\haiku{R.G. voegt aan het eind.}{van zijn vertaling ook enig}{commentaar toe}\\

\haiku{Horatius, Brief (),.}{aan de PisonenDe arte}{poetica 412}\\

\section{Marcellus Emants}

\subsection{Uit: Inwijding}

\haiku{De enige raad van, ':}{oom diet hem moeilik viel}{op te volgen was}\\

\haiku{Als ze je vragen ',.}{inn bestuur te komen}{dan neem je dat aan}\\

\haiku{{\textquoteright} {\textquoteleft}Mag ik je dan zeer,,.}{danken Gertrude voor je}{keurige dienee}\\

\haiku{Als-t-ie nou in,.}{de toekomst ook maar op je}{rekene mag h\`e}\\

\haiku{Als de mensen eens,!}{wisten hoe pedant hij zich}{van avond wel voelde}\\

\haiku{letterlik schuw te....}{worde voor de omgang met}{fatsoenlike lui}\\

\haiku{'t Was, of de ogen.}{van die agent hem dwongen in}{de richting naar huis}\\

\haiku{Overal grote en;}{kleine schilderijen in}{vergulde lijsten}\\

\haiku{in de week was 'k.... ' '....}{in Utrechtn enkele keer}{ooks Zaterdags}\\

\haiku{Nieuwsgierigheid naar;}{haar verleden welde even}{in Theodoor op}\\

\haiku{{\textquoteright} Hij beloofde 't,.}{kuste haar op de handrug}{en wilde heengaan}\\

\haiku{Het antwoord van de.}{Griffier van de Rechtbank had}{hij maar half vertrouwd}\\

\haiku{Die hebbe gelijk{\textquoteright} {\textquoteleft} '{\textquoteright}.}{prevelden zijn lippendag}{wordtt toch niet meer}\\

\haiku{Jij bleef in Utrecht.... en........, '....}{en de meisjes alleen en}{ikn ouwe vrouw}\\

\haiku{van goeie familie....}{en die nog wat te wachte}{heeft van z'n tante}\\

\haiku{een kleur, die haar slecht.}{kleedde en die vloekte met}{de omgeving}\\

\haiku{Duurt 't je al te,,.}{lang snij dan maar uit terwijl}{die hier boven is}\\

\haiku{En wie weet, of de!}{lummel op dit ogenblik niet}{door haar werd gezoend}\\

\haiku{Wie weet, of je niet?}{veel meer van me verwacht dan}{ik betale kan}\\

\haiku{'k zal 't voortaan.}{ook wel doen en jij zal d'r}{geen last van hebbe}\\

\haiku{Hij meende, dat zijn.}{ernst werd miskend en wilde}{haar dit doen inzien}\\

\haiku{{\textquoteright} 't Was hem dan ook, '.}{te moede oft in hem}{schaterde van pret}\\

\haiku{Met vrouwelike:}{intu{\"\i}tievieteit riep}{zij eindelik uit}\\

\haiku{Toch wil ik wel 'ns;}{met je klinken en mag je}{me gelukwense}\\

\haiku{zolang ik zelf me,,.}{niet van onze borrel dat}{is de wijn onthou}\\

\haiku{Over mijn iedeaal '!}{vann man heb ik geen plan}{me uit te late}\\

\haiku{Toen het dienee was,}{afgelopen vroeg mevrouw}{van Onderwaarden}\\

\haiku{{\textquoteright} {\textquoteleft}Waarom heb ie me,?}{niet vooruit gezeid dat je}{misschien komme zou}\\

\haiku{houwe, die je elk....}{ogenblik in je ellende}{kan late stikke}\\

\haiku{'n Vrouw as ik mot ' '.}{n steen in d'r lijf hebben}{in plaats vann hart}\\

\haiku{Soms kon dit hem niets,.}{schelen vond hij die leukheid}{zelfs verkieselik}\\

\haiku{Verliefde lui - hij ' -.}{wistt immers luisteren}{nooit naar goede raad}\\

\haiku{{\textquoteright} Theodoor hield wel.}{eens graag een babbeltje met}{zijn jongste zuster}\\

\haiku{{\textquoteleft}Als 'k verstandig, '.}{kon zijn liet ik me int}{geheel niet neme}\\

\haiku{'t Is wat fijns!... Gaan?}{wij ooit met jonges om als}{met kamerade}\\

\haiku{tot an d'r dood en.}{dan alleen op kamers te}{moge gaan zitte}\\

\haiku{Ik zal me schame,.}{zoveel je verlangt als jij}{me dan ook maar helpt}\\

\haiku{kennelik had ze.}{weinig vertrouwen in dat}{voorspelde genot}\\

\haiku{{\textquoteright} {\textquoteleft}Zijn wij nog altijd?}{te slecht om ies van de goeie}{werke te hore}\\

\haiku{ik raai 't al. U,.}{mot uw geheime beter}{wegsluite moesje}\\

\haiku{want die gode en,,.}{die helde dat benne toch}{mooie ouwe dinge}\\

\haiku{Het trof Theodoor,,.}{dat ze keurig zuiver de}{melodie weergaf}\\

\haiku{{\textquoteright} Zulk soort klachten vond.}{Theodoor hoe langer hoe}{onaangenamer}\\

\haiku{{\textquoteright} Sinds een paar weken {\textquoteleft}{\textquoteright}.}{had zij de benamingschat}{voor hem ingesteld}\\

\haiku{vindt dit verstand dan.}{nu voor Willie elke man}{beter dan geen man}\\

\haiku{{\textquoteright} Drie dagen later:}{galmde Theodoor bij het}{aantafel-gaan}\\

\haiku{Tot de dokter of.}{de luitenant richtte hij}{hoogst zelden het woord}\\

\haiku{Het vond bij mevrouw.}{van Onderwaarden een zeer}{ongunstig onthaal}\\

\haiku{en zoende haar ogen,,,.}{haar mond het kuiltje in de}{ronde blanke hals}\\

\haiku{en daarom heb ie,.}{al die dage nie na me}{omgekeke h\`e}\\

\haiku{Zeg 't, h\`e, zeg 't. '....}{Dan gak weer na die man}{toe en weer en weer}\\

\haiku{{\textquoteright} Nog dreigend, maar nu:}{smekend tegelijkertijd}{gaat zij doffer voort}\\

\haiku{maar Theodoor hield.}{haar hand terug en schoof de}{kaarten op een hoop}\\

\haiku{als je ooit weer met,,.}{die nonsens aankomt dan word}{ik boos ernstig boos}\\

\haiku{Zo hoort 't.{\textquoteright} Daar kon.}{Theodoor onmogelik}{ernstig bij blijven}\\

\haiku{As 'k uit ete ga, ' '.}{zalk m'n ouwe blauwe}{nog wels andoen}\\

\haiku{Andere vrouwe.}{nisse altoos veel mooier}{gekleed dan Tonia}\\

\haiku{Op eens werd 't hem,;}{niet alleen klaar dat hij zich}{vergaloppeerd had}\\

\haiku{Ik ben d'r Theo heel,.}{dankbaar voor dat hij me d'r}{niet in laat lope}\\

\haiku{En als-t-ie zich,....}{eerst vrij heeft gemaakt v\'o\'or dat}{hij jou kwam vrage}\\

\haiku{Ik zal heel graag wat,;}{van u aannemen meneer}{van Onderwaarden}\\

\haiku{wat er zo van uw,........}{mama gezegd wordt is niet}{geheel zonder grond}\\

\haiku{Was ieder mens z\'o,?}{veranderlik z\'o weinig}{zeker van zich zelf}\\

\haiku{{\textquoteright} Stemmen galmden aan,;}{uit het kreupelhout waardoor}{hij gekomen was}\\

\haiku{As je zo opstuift, '.}{maak ie me maar bang en kan}{k toch niks zegge}\\

\haiku{Al wat ik verlang,.}{is behandeld te worden}{als fatsoenlik man}\\

\haiku{{\textquoteright} Even hield zij haar hand;}{met uitgespreide vingers}{zich weer voor de ogen}\\

\haiku{maar Theodoor wist.}{nog niet wat redeneren}{is met je gevoel}\\

\haiku{met de nadruk van:}{een verbitterde viel hij}{haar in de rede}\\

\haiku{Bedarend streek hij}{over het krullende haar en}{zonder dat hij wist}\\

\haiku{an 'n andere.}{late zien en lache ze}{me dan samen uit}\\

\haiku{{\textquoteright} Onder die wilde;}{woordenwarreling luwde}{Theodoors woede}\\

\haiku{{\textquoteright} Even glom de hoop in,.}{hem op dat hij haar nu zou}{kunnen overtuigen}\\

\haiku{kom weerom en je.}{zult geen halven dag op me}{hoeven te wachten}\\

\haiku{{\textquoteleft}'k Heb je maar nie;}{met m'n ongerustheid}{lastig gevalle}\\

\haiku{anders wenst ie je.}{naar de drommel en daar heeft}{ie gelijk in ook}\\

\haiku{op 'n bolstaande....',,:}{slip en d ouwe man die}{al bij God was roept}\\

\haiku{En 't kwam hem voor,.}{dat er voor dit huwelik}{veel te zeggen viel}\\

\haiku{O, die mode, wat!}{een mensonterend dwangbuis}{was dat in haar ogen}\\

\haiku{je luistert, welke,....}{amuzemente je bezoekt}{welke taal je spreekt}\\

\haiku{al is dat nieuwe....}{ook nog zo gemakkelik}{tot stand te brenge}\\

\haiku{hoe 't zou moete, '....;}{wezen en wat hij zou doen}{als-t-ie maarns}\\

\haiku{Is 'n argeloos?}{mens dan alleen bestemd om}{geplukt te worde}\\

\haiku{De aanstellerij '.}{zit bij de van Ouderhoorns}{immers int bloed}\\

\haiku{{\textquoteright} Nu lei zijn moeder:}{haar zwaardooraderde hand}{op zijn arm en zei}\\

\haiku{al herinnerde....}{hij zich de ellende met}{Tonia uitgestaan}\\

\haiku{n paar mooie dasse,:}{weg te gooie toen je zonder}{enige rede zei}\\

\haiku{Ze speelt met u als '....}{de kat met de muis enn}{betrekking zoeke}\\

\haiku{{\textquoteright} 't Was Theodoor,.}{of eensklaps zijn bloed in al}{zijn aderen stolde}\\

\haiku{'t Zit d'r aldoor....}{maar in d'r buik en harde}{koors hei ze gehad}\\

\haiku{die lange, lange, '....}{tijd toe je nooits na me}{ben komme kijke}\\

\haiku{{\textquoteright} Tegen de laatste.}{woorden was Theodoor weer}{niet opgewassen}\\

\haiku{We mogen ons niet,.}{zwak tonen niet maar door dik}{en dun toegeven}\\

\haiku{De opienies van,,.}{de mensen heb je nodig}{m'n vrind hoog nodig}\\

\haiku{- Weet je ook niet, dat....}{van Harmelen geen griffier}{is geworden en}\\

\haiku{Toch spande hij al.}{zijn krachten in om zich tot}{kalmte te dwingen}\\

\haiku{n dief, omdat ik ',....}{niet alsn monnik leef en}{wie weet of-t-ie}\\

\haiku{wat ik aan d' \'e\'en,';}{geef dat mot ik ook an d}{andere geve}\\

\haiku{'t Was al heel mooi, '.}{dat zijt uithield in dat}{ellendige gat}\\

\haiku{{\textquoteright} En met voorgewend:}{hooghartige kalmte liet}{hij er op volgen}\\

\haiku{Je mot 'm in toom,,.}{houwe zie je en nou en}{dan verbetere}\\

\haiku{maar dat je iemand, '....}{neme kan die je zelfn}{kwiebus hebt genoemd}\\

\haiku{Je hoeft de mense.}{toch overal d'r neus niet in}{te late steke}\\

\haiku{Twintig mienute....}{lang heeft ie van morge met}{me zitte prate}\\

\haiku{Onder het dooreten....}{nadenkend zag hij Dora}{voor zich oprijzen}\\

\haiku{maar ik maakte 't.}{de mense moeilik ies van}{me an te neme}\\

\haiku{Wat ik nou heb, zal.}{ik me hele leve lang}{wel motte houwe}\\

\haiku{'k Wou ze juist op;}{de muziekschool doen en dat}{kost nog al veel geld}\\

\haiku{maar u vraagt er naar,,.}{alsof u meent dat ik niets}{anders te doen heb}\\

\haiku{Al zijn grieven zou;}{hij kunnen uitspreken en}{toch prakties blijven}\\

\haiku{{\textquoteleft}Krijg ik antwoord of........}{ben je van plan stommetje}{te blijve spele}\\

\haiku{{\textquoteright} 't Was, of een zwart.}{glanzende straal Tonia's}{ogen ontbliksemde}\\

\haiku{Daarom huichelde,:}{hij nieuwe opwinding nog}{eenmaal opstuivend}\\

\haiku{Ze hebben mekaar.}{de verschrikkelikste}{dinge verwete}\\

\haiku{Plots voelde hij als:}{vroeger behoefte om zich}{stil af te vragen}\\

\haiku{Klein tegen Dorskamp,:}{en van zijn confr\`eres links}{die respondeerden}\\

\haiku{Nu ging hij na, op.}{welke wijze een werkman}{aan de fabriek komt}\\

\haiku{hij wist, dat zijn stem,.}{onzeker zou klinken en}{mat misschien wel schor}\\

\haiku{Misbruik-make.}{van hun poziesie doen die}{bazen allemaal}\\

\haiku{De slijtende tijd.}{zou zijn daad uitwissen in}{ieders gedachten}\\

\haiku{daardoor bewijs je,.}{ten minste voor je zelf dat}{je verstandig bent}\\

\haiku{Maar  oom had de.}{fles gegrepen en wilde}{hem wijn inschenken}\\

\haiku{ouwe koeien ben;}{ik niet voornemens  uit}{de sloot te halen}\\

\haiku{in huis was 't maar, '',....}{trap op trap af en ask}{d'r wa van zee dan}\\

\haiku{{\textquoteright} Een afschuw van zich.}{zelf barstte naar buiten in}{een walgend geluid}\\

\haiku{ze tokkelde zo '....}{graagn operawijsje op}{die ouwe toetse}\\

\haiku{dan loopt de derde,,.}{die geen partij achter zich}{heeft met de kluif weg}\\

\subsection{Uit: Jong Holland}

\haiku{Reeds vele jaren.}{zijn sedert de conceptie}{er van verlopen}\\

\haiku{- Dit is het tweede,.}{gelui het derde zal niet}{lang meer uitblijven}\\

\haiku{Aan jou alleen durf.}{ik de opvoeding van dat}{kind toevertrouwen}\\

\haiku{zij is gestorven,,.}{door mijn ruwheid mijn koelheid}{mijn nuchter verstand}\\

\haiku{- Wie weet of hij niet!}{juist van pas komt in deze}{wonderlijke tijd}\\

\haiku{Ons past het op het,.}{goede te letten en hij}{had veel zeer veel goeds}\\

\haiku{van het bord rees hij.}{weder naar het aangezicht}{van zijn oom omhoog}\\

\haiku{Ons past het op het,,.}{goede te letten en hij}{had veel zeer veel goeds}\\

\haiku{Evenals Pietekoo,}{het woord steeds aan Eveline}{liet gunde zij haar}\\

\haiku{De vrienden van den.}{huize dachten er niet aan}{haar te naderen}\\

\haiku{De bediende had,.}{stoelen aangeschoven het}{gezelschap nam plaats}\\

\haiku{{\textquoteright} Toch gevoelde zij.}{er meer bij dan de zusters}{konden beseffen}\\

\haiku{{\textquoteleft}Ja{\textquoteright} en wierpen dan,,.}{al breiend en bordurend}{blikken naar buiten}\\

\haiku{dat ik uw broer voor ',,{\textquoteright}.}{t laatst ontmoet heb dames}{ging Van Dijck voort}\\

\haiku{Evelines eerste;}{bemerking had dit getal}{van drie gegolden}\\

\haiku{Met een stem, welke,:}{van aandoening trilde riep}{hij plotseling uit}\\

\haiku{Een tik op de deur,.}{stoorde zijn overpeinzingen}{Gijsbrecht stond voor hem}\\

\haiku{{\textquoteright} Met de knop in de,:}{hand keerde hij zich evenwel}{weder om en vroeg}\\

\haiku{{\textquoteleft}Ik zal het met je,,;}{proberen misschien een half}{jaar misschien een jaar}\\

\haiku{{\textquoteleft}Anderen zouden,...}{de vruchten plukken wij voor}{niets hebben gewerkt}\\

\haiku{{\textquoteright} Sleek liep vluchtig de,:}{wijzigingen door hief toen}{het hoofd op en vroeg}\\

\haiku{{\textquoteright} Momstra echter had er,:}{blijkbaar niets geen lust meer in}{en galmde geeuwend}\\

\haiku{{\textquoteright} Van Dijck bromde.}{iets onverstaanbaars zonder}{de ogen op te slaan}\\

\haiku{{\textquoteleft}In 's hemels naam!}{geen letters eten na zulk een}{uitmuntend diner}\\

\haiku{Een zachtzinnige.}{aard verloochent zich wel eens}{na een goed diner}\\

\haiku{Gijsbrecht zorgde er.}{evenwel voor dat de honing}{niet al te zoet werd}\\

\haiku{Was het niet schoon door,?}{een dichter door een genie}{bemind te worden}\\

\haiku{-- Was hij dan in haar -?}{ogen even als in die van zijn}{broeder maar een kind}\\

\haiku{Ook had hij er in}{den beginne niets tegen}{gehad aan enigen}\\

\haiku{{\textquoteleft}Zodra er vier of,;}{vijf leden tegenwoordig}{zijn drinken wij thee}\\

\haiku{{\textquoteleft}Kan je waarachtig{\textquoteright}.}{nog spreken vroeg Gijsbrecht op}{minachtende toon}\\

\haiku{De eerste spellen,.}{wonnen zij zonder moeite}{en dit gaf Frits moed}\\

\haiku{Allen, Reelijn niet,.}{uitgezonderd keken met}{belangstelling op}\\

\haiku{Ah, jeune homme,!}{quelle triste vieillesse}{vous vous pr\'eparez}\\

\haiku{hert. Zolang ik geen.}{besluit genomen heb gaat}{het mij altijd zo}\\

\haiku{Hoe dikwijls moet ik?}{je zeggen dat ik mij niet}{met vrouwen ophoud}\\

\haiku{Daarna keerde hij.}{naar het plein terug en ging}{zijn woning binnen}\\

\haiku{{\textquoteleft}Zo vroeg al op, jij,?}{die anders om tien ure nog}{in de veren ligt}\\

\haiku{De kuur beperkte.}{het thans tot een glas melk met}{enige beschuiten}\\

\haiku{Uitgifte van 100,, -:}{000 Aandelen elk groot f}{10  directeur}\\

\haiku{{\textquoteright} {\textquoteleft}Frederik is er.}{zo onschuldig aan als een}{pasgeboren duif}\\

\haiku{{\textquoteright} Een donkere gloed,.}{liep over Mathildes wangen}{maar verdween spoedig}\\

\haiku{Haar verlegenheid,:}{was evenwel nog merkbaar toen}{zij ten antwoord gaf}\\

\haiku{maar, weet u, als hij,...}{de vorige avond veel wijn}{had gedronken dan}\\

\haiku{{\textquoteright} riep Emile op zijn,.}{gewone gemeenzame}{toon de bankier toe}\\

\haiku{Je weet, ik stelde,.}{vroeger grote prijs op zijn}{scherp geoefend oog}\\

\haiku{{\textquoteleft}Heb nu de goedheid{\textquoteright}}{hier plaats te nemen tussen}{mijn oudste en mij}\\

\haiku{Voor mij is en blijft,.}{het een heilige zaak de}{hoeksteen der moraal}\\

\haiku{De veertien jaren,.}{hadden haar wel iets gebracht}{wel veel ontnomen}\\

\haiku{{\textquoteright} {\textquoteleft}En hoe zijn de twee,?}{pupillen die Henri tot}{zich genomen heeft}\\

\haiku{Hij had eerst nu de;}{volle kracht bereikt van het}{mannelijk leven}\\

\haiku{zij hield echter zijn,.}{blik uit en hij moest ditmaal}{zelf de ogen neerslaan}\\

\haiku{Ik kan haar toch niet!}{met die oudgediende aan}{\'e\'en tafel plaatsen}\\

\haiku{een fijne fles zal,?}{beter smaken na de reis}{wat dunkt Johan er van}\\

\haiku{{\textquoteright} {\textquoteleft}Dan moet je maar eens.}{voor geruime tijd bij ons}{je intrek nemen}\\

\haiku{het voorrecht je tot,.}{leidsman te mogen dienen}{voor mij zelve op}\\

\haiku{{\textquoteright} {\textquoteleft}Wanneer ik er voor,?}{insta dat niemand de zaak}{zelfs vermoeden zal}\\

\haiku{Mevrouw Van Weerdt had.}{haar een onaangename}{indruk gegeven}\\

\haiku{Ondanks mijn brief en.}{telegram is de japon}{niet aangekomen}\\

\haiku{de woorden, welke,.}{zij opving kwamen van de}{overkant der tafel}\\

\haiku{{\textquoteleft}Frits, papa heeft op.}{de goede uitslag van je}{examen geklonken}\\

\haiku{- had u dan zulk een?}{machtige roeping toen u}{Nederland verliet}\\

\haiku{Gijsbrecht liet zich door,:}{die uitroep niet uit het veld}{slaan maar ging kalm voort}\\

\haiku{{\textquoteright} {\textquoteleft}Frits gaat nu niet meer{\textquoteright}, {\textquoteleft}'.}{uit merkte Elisabeth op}{t is al half tien}\\

\haiku{Smijt me in een gracht,.}{ik zal geen poging doen er}{weer uit te kruipen}\\

\haiku{Jij denkt er misschien.}{anders over omdat je nog}{altijd van haar houdt}\\

\haiku{Neen, laten wij nog ';}{wat rondlopen en ins}{hemelsnaam praten}\\

\haiku{Hij durfde echter,.}{niet weigeren klonk dus met}{iedereen en dronk}\\

\haiku{Reelijn vroeg welke.}{gasten de oude heer aan}{zijn dis had gehad}\\

\haiku{Doch eensklaps dacht hij.}{aan Frederika en zijn}{miskende liefde}\\

\haiku{{\textquoteright} Zwijgend ging Scheffer,.}{naar de deur en voerde haar}{de kamer binnen}\\

\haiku{Op een goede dag:}{namelijk had deze tot}{haar moeder gezegd}\\

\haiku{Het zwaard zinkt hem uit,.}{de hand machteloos zijgt hij}{op een stoel neder}\\

\haiku{Dan vraag ik Suze,.}{zondag avond en ben dus voor}{de regen binnen}\\

\haiku{{\textquoteright} Die aanwijzing was.}{voor Scheffer een lichtstraal in}{pikdonkere nacht}\\

\haiku{{\textquoteright} {\textquoteleft}Kijk eens Henri, wat,.}{jij als bankier doen kunt past}{ons daarom nog niet}\\

\haiku{{\textquoteright} De klemtoon, op het,.}{woord duurzaam gelegd ging voor}{Henri verloren}\\

\haiku{Buigend vertrekken,.}{zij doch ook nu vermindert}{het handgeklap niet}\\

\haiku{Opgewonden was;}{de gelukkige auteur}{naar boven gesneld}\\

\haiku{'t Is onwaar dat.}{wij uit eigen kracht iets goeds}{kunnen voortbrengen}\\

\haiku{wie durft beweren...{\textquoteright} {\textquoteleft},,.}{dat hij zijn evenmensKomaan}{oom geen praatjes meer}\\

\haiku{Integendeel vraag.}{ik mij af of Frits je wel}{waardig zal blijken}\\

\haiku{Slechts een nieuw doelwit,.}{een nieuwe eerzucht konden}{mij weer opheffen}\\

\haiku{{\textquoteright} Dit zou echter voor,.}{heden onmogelijk zijn}{beweerde Clara}\\

\haiku{Natuurlijk moet hij.}{afstuderen voordat het}{publiek kan worden}\\

\haiku{Was zij zich bewust?}{het toppunt van haar geluk}{bereikt te hebben}\\

\haiku{Eindelijk rees hij,,:}{op drukte haar onstuimig}{aan zijn borst en vroeg}\\

\haiku{je is zo spoedig.}{mogelijk te trouwen en}{dan eerst te werken}\\

\haiku{Geen terechtwijzing.}{van wie ook heeft mij ooit als}{deze getroffen}\\

\haiku{Niets belette hem.}{immers ogenblikkelijk naar}{boven te snellen}\\

\haiku{hij wist zelf niet meer,...}{hoe hij beiden voor het laatst}{gezien had maar dan}\\

\haiku{{\textquoteleft}Kom toch hier, en stel{\textquoteright}.}{u zo dwaas niet aan riep hij}{hem gemelijk toe}\\

\haiku{alle vertogen.}{van Gijsbrecht dienaangaande}{bleven vruchteloos}\\

\haiku{{\textquoteleft}Dag vrouw, dag moeder,!}{ziedaar mij gelukkig weer}{bij u aangeland}\\

\haiku{{\textquoteright} Terwijl Adolf aldus.}{sprak had Clara haar hand op}{zijn schouder gelegd}\\

\haiku{{\textquoteright} Dit zeggend scheurde.}{hij het telegram langzaam}{in kleine stukken}\\

\haiku{Nawoord Jong Holland ().}{1881 is de eerste roman}{van Marcellus Emants}\\

\haiku{Het was geenszins de.}{eerste maal dat Gijsbrecht ze}{ten beste  gaf}\\

\subsection{Uit: Lichte kost}

\haiku{Hij wond zich op, werd,.}{zenuwachtig en stootte}{meestal te hard}\\

\haiku{Monte Carlo is,!}{verloren en graaf Trogach}{strijkt met zijn schatten}\\

\haiku{Aan tafel was de;}{Italiaan nog steeds een en}{al bewondering}\\

\haiku{Ik maak mij sterk u.}{in twee minuten op de}{hoogte te brengen}\\

\haiku{Weer hield De Morrien,;}{vijftien Louis weer schoof Masset}{er dertig vooruit}\\

\haiku{- Mag ik de veertig?}{Louis aan de heer Masset voor}{u uitbetalen}\\

\haiku{Won hij nu en dan,;}{een zet dan verloor hij weer}{de twee volgende}\\

\haiku{Tienduizend franken,!}{verloren en bovendien}{veertig Louis geleend}\\

\haiku{{\textquoteright} {\textquoteleft}Juist daarom,{\textquoteright} voegde, {\textquoteleft}.}{Prudent er bijspeelde ik}{hoger en hoger}\\

\haiku{Zonder de minste:}{voorbereiding barstte zijn}{bekentenis los}\\

\haiku{Ik geloof niet, dat.}{De Morrien ooit in Monte}{Carlo heeft gespeeld}\\

\haiku{{\textquoteright} De commissaris.}{rees op en beantwoordde}{deze uitroep niet}\\

\haiku{in elk geval mocht.}{zij spoedig enig bericht van}{hem tegemoet zien}\\

\haiku{{\textquoteright} Gelukkig verstond.}{de commissaris geen woord}{van al dat gegrom}\\

\haiku{jongeheer, en met,.}{recht want ik stelde mij lang}{niet heerachtig aan}\\

\haiku{Geen van de kale.}{hoofden paste en geen van}{de pruiken voldeed}\\

\haiku{Cr\'ebillard hechtte,.}{aan typen en ik wilde}{er niet van horen}\\

\haiku{wij behielpen er,.}{ons echter mee zo goed en}{zo kwaad als het ging}\\

\haiku{De dames houden,;}{aan maar worden naar boven}{teruggezonden}\\

\haiku{{\textquoteleft}Monsjieu Cr\'ebillard... heeft.......}{al sedert een half uur zijn}{winkel verlaten}\\

\haiku{Je hebt zeker een.}{droppel van je mastik er}{op laten vallen}\\

\haiku{{\textquoteright} De lust tot praten;}{had onze kapper er niet}{bij ingeschoten}\\

\haiku{{\textquoteright} {\textquoteleft}Tot weerziens, mijnheer,.}{Cr\'ebillard tot weerziens in}{de nieuwe woning}\\

\haiku{{\textquoteright} {\textquoteleft}Welnu, heren, die,.}{dochter is een canaille}{een waar canaille}\\

\haiku{Nu denk je, dat ik '.}{mijt eerst weer wendde tot}{juffrouw Van Limburg}\\

\haiku{juffrouw Van Limburg.}{had zich nooit tot verpleegster}{mogen opwerken}\\

\haiku{Voor iedereen had,'}{hij een aardig woordje zelfs}{op juffrouw Timmers}\\

\haiku{juffrouw Van Limburg.}{begon mij bijzonder te}{interesseren}\\

\haiku{As ie thoch rijk is, '?}{wat mot so'n man dan inn}{sanatorium}\\

\haiku{mevrouw Van der Lof.}{sloeg het dampende vocht in}{een teug naar binnen}\\

\haiku{Maar als ze denkt, dat, '!}{ze zo van me afkomt dan}{heeft zet toch mis}\\

\haiku{Zij was er zeker;}{van deze melodie meer}{te hebben gehoord}\\

\haiku{Dit wilde ik in;}{dichterlijke beelden voor}{aller ogen stellen}\\

\haiku{De hele les is,!}{er aan heen gegaan en zij}{was opgetogen}\\

\haiku{Bovendien... jij doet.}{in de muziekkunst en ik}{doe in de bouwkunst}\\

\haiku{Doch plotseling boog:}{Frau Troistorff zich nog verder}{voorover en riep uit}\\

\haiku{zij had niet gedacht,.}{dat hij zo flink voor de dag}{zou durven komen}\\

\haiku{hij verbeeldde zich,:}{dat hij werkelijk voor de}{verleiding bezweek}\\

\haiku{Een oude dame;}{bette hem de slapen met}{eau de Cologne}\\

\subsection{Uit: Liefdeleven}

\haiku{Dat doet toch ieder,.}{die het bericht krijgt van z'n}{vriends engagement}\\

\haiku{ik dacht wel, dat je.}{de jaren van verliefdheid}{achter de rug hadt}\\

\haiku{dat is jouw idee en,;}{ik kan niet bewijzen dat}{je ongelijk hebt}\\

\haiku{Geloof jij, dat 'en '?}{groot man gelukkiger is}{danen obskuur baasje}\\

\haiku{daar heb je zo wat,.}{alles wat medici met}{hun kunst vermogen}\\

\haiku{ook de luitjes op.}{een dorp weten wat er in}{de wereld omgaat}\\

\haiku{Daar achter wist hij.}{de vaart en aan die vaart naast}{de koepel een bank}\\

\haiku{Misschien had hij toch.}{beter gedaan niet op dit}{bal te verschijnen}\\

\haiku{Fluks hief ze zich weer......}{recht omhoog kwam dichterbij}{een bloem in de hand.}\\

\haiku{Ja... als u vindt, dat......}{mijn mijn genegenheid niks}{te beduiden heeft}\\

\haiku{{\textquoteright} Weer aarzelde ze.}{en die aarzeling stelde}{hem opnieuw teleur}\\

\haiku{Als ik trouwde, zou.}{ik natuurlik dat beroep}{op moeten geven}\\

\haiku{maar ook, omdat hij.}{zich moeilik behelpen kon}{met een eng verblijf}\\

\haiku{Ik geloof, dat je.}{zo kalm bent en anderen}{zo kalm kunt maken}\\

\haiku{Christiaan vond, dat:}{ze flink optrad en voegde}{er alleen nog bij}\\

\haiku{maar overigens had:}{zij zich de naaste toekomst}{als volgt voorgesteld}\\

\haiku{maar die verkilden,...}{zodra een beeld oprees uit}{de nare dagen}\\

\haiku{{\textquoteleft}Ch\^atelaine, nou.}{ga ik je installeren}{in je nieuwe slot}\\

\haiku{Soms zeg je maar wat,.}{anders alleen om tegen}{te kunnen spreken}\\

\haiku{Dat je 't in zo'n!}{kamer ook maar \'e\'en nacht hebt}{kunnen uithouwen}\\

\haiku{Wat moeten ze er,?}{beneden van denken als}{we niet opdagen}\\

\haiku{wou je mij die ook...,?}{aanrekenen als iets goeds}{iets moois iets ideaals}\\

\haiku{Maar eerst ben je me......}{onverschillig geweest en}{nu nu haat ik je}\\

\haiku{Dan zou je weer met!}{je ouwe Trijn en met d'r}{dochter gaan heulen}\\

\haiku{Tot ze dicht genoeg,.}{bij hem stond dat haar hand de}{zijne kon grijpen}\\

\haiku{Misschien zal hier of ';}{daaren muur weggebroken}{dienen te worden}\\

\haiku{Enkel, dat ze geen {\textquoteleft}{\textquoteright},.}{sc\`enes maakte dat ze bleef}{zoals ze nu was}\\

\haiku{doorklinken, soms haar!}{zingen te horen galmen}{door voorhuis of gang}\\

\haiku{et Is ook wat die '.}{mensenen beetje naar de}{mond te babbelen}\\

\haiku{Dit antwoord schrijnde.}{nu weer door zijn zalige}{jubel-stemming}\\

\haiku{Zo zenuwachtig,,.}{dacht Christiaan heb ik ze}{vroeger nooit gezien}\\

\haiku{Maar dat belet niet, ' '.}{datet een of ander m'n}{aandacht wels trekt}\\

\haiku{Trijn heeft iets gezegd,,...}{waaruit ik heb opgemaakt}{dat wij gegeven}\\

\haiku{als we...{\textquoteright} {\textquoteleft}Zie je nu,?}{wel dat je de waarheid voor}{me verbergen wilt}\\

\haiku{Doch op alles zei:}{ze neen en altijd volgde}{dezelfde fraze}\\

\haiku{Ik heb nooit gedacht, '.}{dat ik vanen man z\'oveel}{zou kunnen houwen}\\

\haiku{Voor iets ernstigs, iets,,...}{moois en goeds iets dat alleen}{ik zou kunnen doen}\\

\haiku{want sinds ze weet, dat,.}{u komt is de toestand al}{merkbaar verbeterd}\\

\haiku{Mina heeft van de '.}{eerste dag afen hekel}{aan dat mens gehad}\\

\haiku{Weet u, dat ik haar?}{aangeboden heb Trijn te}{pensioneren}\\

\haiku{maar 't is misschien, '.}{toch beter dat ik je de}{ogen daars voor open}\\

\haiku{Je hebt mij de schuld.}{van alles gegeven en}{dat doe je altijd}\\

\haiku{Voorzeker had hij.}{niet de bedoeling haar slecht}{te behandelen}\\

\haiku{de randen van haar.}{ogen en een zilverig waas}{overtoog er het blauw}\\

\haiku{hoe waar 't is, dat.}{menig mens in zich zelf z'n}{ergste vijand heeft}\\

\haiku{Maar \'e\'en ding wil ik...}{je wel voorspelle al doe}{je alles voor d'r}\\

\haiku{{\textquoteright} Nu verweet ze hem '?}{z'n opstuiven en wie stoof}{altijdet eerst op}\\

\haiku{{\textquoteleft}Mijn God, 'en mens kan '!}{toch eindelik weles van}{idee veranderen}\\

\haiku{Maak ik 'es 'en plan,!}{dan heb jij altijd honderd}{duizend bezwaren}\\

\haiku{En hij voelde het:}{beetje nieuwsgierige vrees}{weer van hem wijken}\\

\haiku{Dat doet nou maar net,!}{of d'r geen mens anders op}{de wereld bestond}\\

\haiku{Totdat Mina plots,.}{stilhield lucht opsnoof en}{hoorbaar weer uitblies}\\

\haiku{je hebt 'et gezegd,...}{en je hebt er bij gevoegd}{dat je ook mij haat}\\

\haiku{het gouden haar, en,.}{in een nieuw kostuum dat haar}{biezonder goed stond}\\

\haiku{Trouwens,{\textquoteright} voegde hij, {\textquoteleft}.}{er bijmet de tijd wordt dat}{van zelf wel anders}\\

\haiku{grinniklachjes,;}{grapten waar niet ieder de}{reden van begreep}\\

\haiku{Over de tafel heen.}{vroeg hij Diepe naar een van}{zijn pasi\"enten}\\

\haiku{En eindelik en.}{ten laatste had zij de hoop}{ook opgegeven}\\

\haiku{maar niet zag hij haar,.}{die alleen hem zijn rust had}{kunnen hergeven}\\

\haiku{Dadelik was hij,.}{overeind haar gevolgd en nu}{liep hij naast haar door}\\

\haiku{Maar door haar schouders,:}{voer een licht schokje en zacht}{doch beslist sprak ze}\\

\haiku{Me dunkt, dat ie zich!}{niet ontzien heeft z'n afkeer}{van mij te tonen}\\

\haiku{{\textquoteleft}Als ik me maar door!}{iedereen vernederen}{en vertrappen laat}\\

\haiku{Twee dagen later.}{belde Christiaan om half}{vijf bij Diepe aan}\\

\haiku{de indruk, dat ze.}{tamelik prikkelbaar en}{opgewonden was}\\

\haiku{bleef hij bedaard, dan.}{schold ze op zijn tergende}{onverschilligheid}\\

\haiku{Ja... als er nou 's ', '...?}{en kleintje kwam zout je}{welkom zijn of niet}\\

\haiku{dat geef ik toe, Dus... '......, '.}{komtet nou wat mij betreft}{zalt welkom zijn}\\

\haiku{Soms is 't zelfs of '.......}{zeen tegenzin heeft in}{mij of in de zaak}\\

\haiku{Maar toch... toch speet 'et...}{hem nu dubbel zijn vriend te}{hebben geraadpleegd}\\

\haiku{En kwam hij... nu, dan '.}{wast nog niet te laat om}{te zien wat te doen}\\

\haiku{{\textquoteleft}Ach... wat ik daarbij,....}{voel kan jij niet kan niemand}{met me mee voelen}\\

\haiku{Van morgen wist je,.}{zeker nog niet dat ie je}{zou komen halen}\\

\haiku{Bij de andere...... ';}{vrouw een weduwe wast}{ook wel armoedig}\\

\haiku{Christiaan ving daar;}{nu en dan in zijn atelier}{wel een galm van op}\\

\haiku{{\textquoteright} {\textquoteleft}Dat ik niet mee wil,...;}{komen is de zaak niet heb}{ik ook niet gezegd}\\

\haiku{{\textquoteleft}Je zult wel moe zijn,,{\textquoteright}:}{ga maar vroeg naar bed dan was}{telkens haar antwoord}\\

\haiku{maar tans al... en nog......}{wel alleen naar Heijdestein}{terug te keren}\\

\haiku{Het lokaal, dat hij,.}{eindelik vond en nam was}{weinig naar zijn zin}\\

\haiku{Toen hij zich naar de,.}{deur wendde om weer heen te}{gaan stond zij voor hem}\\

\haiku{met z\'o'n dokter... dat... '.}{begrijp je toch is er niet}{et minste gevaar}\\

\haiku{{\textquoteright} {\textquoteleft}Beloof me dan, dat,...}{je geen vrouw zult nemen van}{wie je niet zeker}\\

\haiku{Hoe Mina dan zou..., '.}{worden hij kon hij dorstet}{zich niet voorstellen}\\

\haiku{Ik zou 't zo naar,.}{zo vreselik vinden als}{je me die afsloeg}\\

\haiku{Christiaan vond er,.}{dus niets vreemds in dat Mina}{Geertje te hulp riep}\\

\haiku{Een uur te voren.}{was Jantje ontwaakt en had}{hij weer even geschreeuwd}\\

\haiku{moest worden gedaan,.}{wat er nog verder in huis}{viel te verrichten}\\

\haiku{en zij... zou zij nog}{iets voor me voelen als ik}{niet de vader was}\\

\haiku{Hij zelf kon enkel,,.}{door angst te tonen haar nog}{angstiger maken}\\

\haiku{Zonder afscheid te.}{nemen vertrok Christiaan}{naar zijn atelier}\\

\haiku{Zonneschijn lag in;}{de kamer als een fletse}{vlek op het tapijt}\\

\haiku{Zodra Jantje aan,....}{werd geraakt begon hij te}{kreunen te krijten}\\

\haiku{Christiaan hoorde,.}{de fletse klanken verstond}{de zin er van niet}\\

\haiku{Dringend herhaalde,.}{ze die vraag als hij geen zeer}{stellig antwoord gaf}\\

\haiku{Was dat ook niet 'et, ',?}{bestet enige dat hem}{nog restte te doen}\\

\haiku{Voor geen geld zou hij.}{de herinnering aan die}{droom willen missen}\\

\haiku{Op die schijning ging,,,;}{hij in gedachten verdiept}{af doorschreed de deur}\\

\haiku{Maar... hier hing nog 'en... '.......}{fletse geuren geur van van}{Iris-poeier juist}\\

\haiku{En een tinteling.}{van hoop had nog eenmaal zijn}{somber denken doorflitst}\\

\haiku{Geen  blik was naar,;}{hem opgerezen geen woord}{tot hem uitgegaan}\\

\haiku{maar zijn alle, die,?}{er een grotere dosis}{van bezitten ziek}\\

\haiku{De afwijkende;}{mens speelt dus even goed een rol}{als ieder ander}\\

\haiku{het wegwerpen van.}{het kind met het badwater.11}{Marcellus Emants}\\

\haiku{Normaal doet zich voor -,!}{de psyche ziehier de lijn}{die de praktijk trekt}\\

\haiku{11Met opzet heb {\textquoteleft}{\textquoteright}.}{ik in dit betoog het woord}{schoonheid vermeden}\\

\subsection{Uit: Waan}

\haiku{'t Is zelfs wel... 't,?}{is in alle geval z\'o}{niet ordinair h\`e}\\

\haiku{Of er meer zulke?}{zonderlinge mensen als}{zij zouden bestaan}\\

\haiku{We kunne tante.}{toch de hele avond niet aan}{d'r lot overlate}\\

\haiku{Weer leunt zij over het.}{balkonhek heen en naast haar}{buigt Hendrik neder}\\

\haiku{Ik heb nooit meer te.}{hore gekrege dan zo'n}{paar losse mate}\\

\haiku{maar aan voorspele ' '.}{heb ikn hekel en van}{avond doe ikt niet}\\

\haiku{O, als ik dat niet...,;}{had geloof me dan was ik}{er al lang niet meer}\\

\haiku{Grootogig en met.}{open mond kijkt zij hem bijna}{verontwaardigd aan}\\

\haiku{En ik kan 't ook,.}{niet vele dat je me zo}{zit aan te stare}\\

\haiku{- Nou, kind, dat jij niks,.}{gebroke hebt is ook meer}{geluk dan wijsheid}\\

\haiku{Dra blijft ze ook staan,,....}{de hand op het hart gedrukt}{hijgend sprakeloos}\\

\haiku{- Nee, kind, dat moet je.}{juist niet en daar ben je zelf}{ook wel van overtuigd}\\

\haiku{Maggie zal zich niet;}{behoeven af te sloven}{in haar huishouden}\\

\haiku{Zonder een woord meer.}{te zeggen bereiken ze}{het pension}\\

\haiku{... wil u 't me nou ',?...}{s inpepere dat ik}{onlief ben geweest}\\

\haiku{- Denkt u dan, dat we......?}{niet voor mekaar passe of}{of iets dergelijks}\\

\haiku{- Ja... wat zal 'k je? ';}{daar nou van zegget Kan}{best aan mij ligge}\\

\haiku{maar... mijn hemel, als...!}{ik dat vergelijk met mijn}{engagementstijd}\\

\haiku{Bloemen en blade,.}{leven en pluk je ze af}{dan maak je ze dood}\\

\haiku{toch vond ik 't de,.}{eerste keer nog enger nog}{veel ontzettender}\\

\haiku{de lange, lege,,;}{weg zo rul bezond lijkt hem}{troosteloos eenzaam}\\

\haiku{maar er jets voor doen,,:}{er jets voor laten er voor}{in de plaats geven}\\

\haiku{'t Is genoeg, dat,;}{hij haar wil aanraken of}{ze duwt hem terug}\\

\haiku{Als ze terugkeert,,}{herhaalt de oude vrouw haar}{vraag gist ze vorst ze}\\

\haiku{Nu alles voorbij,.}{is kan ik dus helemaal}{vrij tot je spreken}\\

\haiku{Hoe dat komt, begrijp,.}{ik zelf niet want vroeger hield}{ik toch ook van je}\\

\haiku{zo kan je 't weer '...}{geen twee minute zonder}{t kind uithouwe}\\

\haiku{Hij ziet, dat ze van.}{alle kanten bekeken}{en befluisterd wordt}\\

\haiku{Rietzen wordt een fles.}{Heidsieck Monopol met drie}{glazen neergezet}\\

\haiku{hij heeft 't al lang, '.}{geweten al kont hem}{vroeger niets schelen}\\

\haiku{daar benne d'r heel...:}{wat die stemme toch zeker}{op u. Die zegge}\\

\haiku{Daarna begint ze.}{ineens weer opgewekt te}{spreken over het feest}\\

\haiku{toen heb je Rietzen '.}{geen rekenschap gevraagd of}{n klap gegeve}\\

\haiku{En nu ze geen vrees...}{meer koestert voor de scherpe}{klauwen van de beer}\\

\haiku{In het leven van.}{de man is de liefde maar}{een episode}\\

\section{Graad Engels}

\subsection{Uit: Det dank'tich d'n duvel}

\haiku{Ze konden echter.}{geen plaats passeren waar een}{kruis hing of stond}\\

\haiku{Ze lag daar echter.}{met een verband om haar hand}{te kermen van pijn}\\

\haiku{Ze was uiterlijk.}{een heel normale vrouw van}{rond de zeventig}\\

\haiku{Hij zei dat tegen.}{zijn vrouw en gebood haar de}{blaasbalg te trekken}\\

\haiku{Men adviseerde, {\textquoteleft}{\textquoteright}.}{dan ook te proberen van}{henbloed te trekken}\\

\haiku{Die ging schoorvoetend.}{mee en ook hij hoorde de}{wanmolen draaien}\\

\haiku{Het begon met de.}{plotselinge sterfte van}{een koe en een paard}\\

\haiku{Boer Jan raakte door.}{de miserie zijn geld en}{zijn gezondheid kwijt}\\

\haiku{Binnen een week viel.}{zijn van huis meegekregen}{paard dood in de kar}\\

\haiku{Deze noemde men {\textquoteleft}'{\textquoteright},.}{t Menke van Gengk uit de}{Belgische Kempen}\\

\haiku{Let dan goed op voor.}{het te laat is en schiet hem}{zijn gat vol hagel}\\

\haiku{Het werd zo laat dat.}{ze tegen middernacht nog}{onderweg waren}\\

\haiku{{\textquoteright} Een boer uit Helden {\textquoteleft}'{\textquoteright}.}{was op koehandel int}{Rooth onder Maasbree}\\

\haiku{Boven op de steel,.}{hing een klomp die telkens mee}{op en neer wipte}\\

\haiku{Men hoorde ze dan,,.}{kloppen maar niemand mocht noch}{durfde gaan kijken}\\

\haiku{Hij vertelde mij.}{dat daar in de wal vijftien}{kabouters woonden}\\

\haiku{Op de verjaardag.}{van een der leiders wilden}{ze eens wat aparts doen}\\

\haiku{De schaapherders uit.}{die tijd bleven graag uit de}{buurt van het Soemeer}\\

\haiku{Hij was daarover zo.}{kwaad dat hij een week  aan}{een stuk door vloekte}\\

\haiku{De duivel had hem.}{op den duur onder vloeken}{moeten loslaten}\\

\haiku{de hele dag zoet.}{zijn en er waarschijnlijk niet}{eens mee klaar komen}\\

\haiku{Zo leven hier nog.}{talrijke verhalen van}{dezelfde soort}\\

\haiku{Dat scheelde maar een.}{haar of er hadden hier twee}{doden gelegen}\\

\haiku{Het volgende lied.}{werd voor mij gezongen door}{een vrouw uit Helden}\\

\haiku{Ik wil met al mijn.}{oude leden Gaan leven}{naar mijn zin en lust}\\

\haiku{Op een avond waren.}{we met kameraden op}{buurtbezoek geweest}\\

\haiku{Nog erg slaapdronken.}{legde ik me languit op}{de kar te rusten}\\

\haiku{De moeder kwam op.}{bezoek om naar het graf van}{haar zoon te zoeken}\\

\haiku{{\textquoteleft}Als jullie menen,.}{dat er iets bijzonders is}{ga dan eens kijken}\\

\haiku{Voorzichtig kwamen,.}{de omwonenden uit hun}{huizen meest vrouwen}\\

\haiku{Het waren reuzen,.}{van kerels wel twee en een}{halve meter lang}\\

\haiku{Na wat gerust te.}{hebben besloten ze de}{Rijn te gaan graven}\\

\haiku{Die bevond zich op.}{het eerste verdiep aan de}{westkant in een gang}\\

\haiku{Ze vlogen wild uit,.}{elkaar maar een hele klis}{tuimelde omlaag}\\

\haiku{Een vertelling over.}{Onze Lieve Heer zelf van}{een man uit Maasbree}\\

\haiku{{\textquoteleft}Hoe kunnen jullie?}{nou geloven dat er wat}{bijzonders gebeurt}\\

\section{Desiderius Erasmus}

\subsection{Uit: De lof der zotheid}

\haiku{Vooreerst, wat kan er?}{zoeter of kostbaarder zijn}{dan het leven zelf}\\

\haiku{Hoofdstuk XIV   De.}{Zotheid verlengt de jeugd en}{weert den ouderdom}\\

\haiku{alsof niet juist het,.}{sterven hierin bestaat dat}{men iets anders wordt}\\

\haiku{Maar door die dwaling.}{weg te nemen bederft men}{het geheele stuk}\\

\haiku{Drinkt of ga heen, maar,.}{verlangt dat het tooneelstuk geen}{tooneelstuk meer zal zijn}\\

\haiku{Welk wijsgeer heeft ooit,?}{een staat uitgedacht die den}{haren nabijkomt}\\

\haiku{Nog zeldzamer is',.}{Mercurius geschenk de}{welsprekendheid}\\

\haiku{Zou ik soms Diana,?}{benijden omdat men haar}{menschenbloed offert215}\\

\haiku{Voorts bestaat een groot.}{gedeelte van hun geluk}{in hun bijnamen}\\

\haiku{{\textquoteleft}Hij vangt slapend visch{\textquoteright},:}{in zijn net werd toegepast}{en ook een ander}\\

\haiku{neem toch de misdaad,;}{uws knechts weg want ik heb zeer}{zottelijk gedaan441}\\

\haiku{44Jupiter wordt,.}{bedoeld die gezoogd zou zijn}{door de geit Amalthea}\\

\haiku{VII) wordt Bacchus op.}{de meest belachelijke}{wijze voorgesteld}\\

\haiku{114Dit fabeltje.}{wordt bij sommige oude}{dichters gevonden}\\

\haiku{122Iets dergelijks wordt,,.}{van Numa den tweeden koning}{van Rome verteld}\\

\haiku{142Grieksch wijsgeer, 396- (.}{314 v. Chr. 143Cato de}{jongerezie hoofdst}\\

\haiku{195{\textquoteleft}De eene muilezel,{\textquoteright} {\textquoteleft}}{krabt den anderen spreekwoord}{uit de oudheid =}\\

\section{Charles-Alexandre Chatrian en Emile Erckmann}

\subsection{Uit: De loteling van 1813}

\haiku{{\textquoteright} {\textquoteleft}Moeder, wij deden.}{hun wel kwaad en dat komen}{zij ons vergelden}\\

\haiku{{\textquoteright} Deze sterke vrouw.}{had haar toestand begrepen}{en den plicht haars zoons}\\

\haiku{Daarom wendde hij:}{het gesprek op een ander}{onderwerp en zei}\\

\haiku{de toestand van den.}{huisvader boezemde hem}{ongerustheid in}\\

\haiku{Abraham wilde zijn;}{zoon zelfs opofferen om}{te gehoorzamen}\\

\haiku{Het leven behoort,.}{u niet toe God alleen kan}{het u ontnemen}\\

\haiku{{\textquotedblleft}Noch aan u zelf, noch.}{aan anderen zult gij de}{schennende hand slaan}\\

\haiku{Maar wie heeft op de?}{kinderen toch meer recht dan}{vaders en moeders}\\

\haiku{Men rakelde den;}{haard op en allen schaarden}{zich om neef Laurens}\\

\haiku{Ten derden male,.}{doorkruist de tamboer de stad}{doch zonder gevolg}\\

\haiku{{\textquoteright} {\textquoteleft}Waarom,{\textquoteright} zei deze, {\textquoteleft}?}{staat het er overal niet zoo}{gunstig voor als hier}\\

\haiku{{\textquoteleft}Vendee\"ers, zijt gij?}{uw wettige koningen}{vergeten of niet}\\

\haiku{In de kazerne.}{vonden zij slechts twee dooden en}{vele gewonden}\\

\haiku{Daarna gelastte.}{zij een der omstanders een}{priester te halen}\\

\haiku{Gun hem den tijd om.}{zijn misdaad te beweenen}{en uit te boeten}\\

\haiku{Toch is het nog niet,.}{lang geleden dat ik den}{arbeid vaarwel zei}\\

\haiku{Hebt gij geen rust, en?}{geniet gij geen gezondheid}{op uw ouden dag}\\

\haiku{Aan mijn oudsten zoon,.}{laat ik mijn hut na hij zal}{mijn plaats innemen}\\

\haiku{Op dit gezicht sprong}{Berenger van zijn paard en}{door schrik bevangen}\\

\haiku{- Ik zal dit kasteel,.}{nooit verlaten antwoordde}{Berenger met kracht}\\

\haiku{- Mijn zoon, alvorens,.}{monnik te worden was ik}{een krijgsman zooals gij}\\

\haiku{Ik zie duidelijk,.}{den heiligen standaard die}{op dat schip wappert}\\

\haiku{- Berenger d'Elvaz,,?}{stamelde hij eindelijk}{is het mogelijk}\\

\section{Emile Erens}

\subsection{Uit: Korte verhalen}

\haiku{In een oofttuintje:}{achter het leemen boerenhuis}{zaten zij samen}\\

\haiku{Antje plaatste de lamp.}{op tafel en nu werd er}{koffie gedronken}\\

\haiku{Ook de heeren van.}{de zandgroeve speelden kaart}{dicht bij het buffet}\\

\haiku{Hij kroop een eindje:}{terug en opstaand ging hij}{rechtop op haar toe}\\

\haiku{Want ook Tinus was,.}{de vijand van Frits omdat}{deze geen boer was}\\

\haiku{Nevelwit hing de.}{lichtstilte der nacht over het}{heuvelige land}\\

\haiku{Bij den biechtstoel van:}{den kaplaan zat een jonge}{meid in groote onrust}\\

\haiku{{\textquoteright} {\textquoteleft}Och neen, nu niet, ik,{\textquoteright}.}{wil toch nog niet trouwen sprak}{zij met schuwen blik}\\

\haiku{{\textquoteleft}Ween niet zoo, Mieke,,{\textquoteright}.}{als we trouwen is immers}{alles goed zei Johan}\\

\haiku{Toen de winter kwam,.}{vond Liesje haar dood in den}{donkeren stal}\\

\haiku{zij schrok heel bleek van,.}{het plotselinge woord en}{voelde het op eens}\\

\haiku{Even buiten het dorp.}{liep de weg door een bosch met}{groote beukeboomen}\\

\haiku{En hij hoorde haar.....}{stille woorden zeggen in}{het donkere woud}\\

\haiku{Korte verhalen,!}{Noten 1{\textquoteleft}O Isra\"el}{hoe groot is Gods huis}\\

\section{Fons Erens}

\subsection{Uit: Bazen knuppels knechten. IJohannes Hendrijckos (1791-1866). Een dagloner in Wijnandsrade}

\haiku{In Maastricht viel mij.}{dat voor het eerst op in de}{dertiger jaren}\\

\haiku{De tantes zorgden.}{voor koffie en waren druk}{aan het babbelen}\\

\haiku{Nu ik oud ben, smaakt:}{die jeugd-boterham weer}{het allerbeste}\\

\haiku{'s Avonds kregen wij, {\textquoteleft}{\textquoteright}.}{dikwijls karnemelksepap}{die wijketsj noemden}\\

\haiku{Of de mensen zich?}{bij dat sobere bestaan}{gelukkig voelden}\\

\haiku{In ieder geval.}{was het dat zeker voor mij}{in het verleden}\\

\haiku{Nu wordt die heuvel;}{overhuifd met een koepel van}{hoge boomkruinen}\\

\haiku{De mensen moesten zelf.}{maar zien hoe zich tegen hen}{te verdedigen}\\

\haiku{Met vijf vliegende.}{vendels maken de Fransen}{jacht op deserteurs}\\

\haiku{Jean Henry Er(r).}{ens uit Wijnandsrade met}{trekkingsnummer 117}\\

\haiku{De mensen wisten,.}{het niet meer maar het kon hen}{ook weinig schelen}\\

\haiku{Al was ik wat bang, {\textquoteleft}{\textquoteright}.}{voor Han toch bracht ik graag de}{melk naargene Bek}\\

\haiku{De kachel zorgde.}{voor warm eten en voor warmte}{in de woonkeuken}\\

\haiku{Naast dat {\textquoteleft}sjaap{\textquoteright} hing een.}{achter glas geschilderde}{heiligenfiguur}\\

\haiku{In 1920 werd een pas.}{geslacht varken uit onze}{kelder gestolen}\\

\haiku{ik had meer te doen.}{met de jongen dan met het}{gestolen varken}\\

\haiku{Ik moet er als een.}{pasja in dat grote bed}{gelegen hebben}\\

\haiku{Soms kwamen meisjes.}{met hun spinnewiel samen}{op een boerderij}\\

\haiku{Niet minder dan 10\%.}{van alle gezinshoofden}{was zonder beroep}\\

\haiku{De {\textquoteleft}goeie ouwe tijd{\textquoteright},.}{heeft niet bestaan ook niet voor}{IJoannes Hendrijckos}\\

\haiku{Dit neemt niet weg, dat.}{ook daar de situatie}{erbarmelijk was}\\

\haiku{De kindersterfte.}{bleef ook in de Franse tijd}{nog ontstellend hoog}\\

\haiku{Hij is dan al 47!}{jaar lang schoolhoofd en zal het}{nog tot 1905 blijven}\\

\haiku{De mensen stonden.}{niet bepaald te springen om}{hun heer te dienen}\\

\haiku{Op mijn vraag naar de:}{geloofsbeleving in zijn}{jeugd vertelde hij}\\

\haiku{In het begin van.}{de middeleeuwen was de}{toga algemeen}\\

\haiku{{\textquoteleft}Ki\"esk\"opke{\textquoteright} zegt,,.}{de meester koster Coenen}{wel eens tegen hem}\\

\haiku{Per ongeluk schuurt - -.}{zijn stoppelbaard langs die o}{zo zachte wangen}\\

\haiku{Vader zit bij de,.}{schouw en dopt bonen moeder}{zorgt voor het avondeten}\\

\haiku{Even later gaat bij {\textquoteleft}{\textquoteright}.}{hen de buitendeur naar de}{neire krakend open}\\

\haiku{Trillend staat hij op,.}{zijn zware handen steunen}{op de tafelrand}\\

\haiku{Tot driemaal toe werd,.}{er geklopt maar telkens was}{niemand aan de deur}\\

\haiku{Aan die trekos kon,.}{iedereen zien dat vader}{een echte boer was}\\

\haiku{Ze bleven er twee,.}{dagen totdat Joean wat}{bijgekomen was}\\

\haiku{Heeft hij die bidweg?}{nu aan nonk Giel of aan zijn}{vader te danken}\\

\haiku{Het zand op het graf.}{was er toch en het water}{uit de put ook}\\

\haiku{Achter de grote.}{hoeve van Arensgenhout komt}{de winterzon op}\\

\haiku{Een kantkraag van sneeuw - -.}{spel van de wind krult over de}{berm langs het voetpad}\\

\haiku{Het is Bieldermans.}{van te Nietese met zijn}{zwager uit Hulsberg}\\

\haiku{De dienst is plechtig,,,.}{met veel misdienaars kaarsen}{wierook en gezang}\\

\haiku{De meeste boeren.}{zitten gebogen met de}{koppen bij elkaar}\\

\haiku{Twee mannen staan daar,,:}{nog nalachend tegen}{een muur te plassen}\\

\haiku{Hij gaat niet meer naar,.}{school hij heeft gebiecht en de}{kommunie gedaan}\\

\haiku{de drie koeien van {\textquoteleft}{\textquoteright}.}{Kleintjesomsjpanne in}{de stoppelklaver}\\

\haiku{Zij zal h\`em ook wel,.....}{horen alsof zij elkaar}{signalen geven}\\

\haiku{Hij gaat weer liggen,,.}{zijn gezicht naar de hemel}{de ogen gesloten}\\

\haiku{Joep is de jongste,.}{broer van hun overbuur Driek kan}{goed met hem overweg}\\

\haiku{Als een bronstig dier,....}{als een doldrieste duivel}{gooit hij zich op haar}\\

\haiku{Hij voert de paarden,.}{en streelt de kop van Bella}{zijn lievelingspaard}\\

\haiku{{\textquoteright} In \'e\'en adem vertelt.}{Driek hem alles wat die dag}{is voorgevallen}\\

\haiku{Joep is zijn beste,.}{kameraad hij begrijpt hem}{beter dan wie ook}\\

\haiku{En straks zullen hier....}{weer kinderen van Marie}{en Driek rondlopen}\\

\haiku{Na de mikken en.}{de vlaaien gaat een ham de}{nog warme oven in}\\

\haiku{Nonk Giel zet een lied:}{in op de melodie van}{de Marseillaise}\\

\haiku{Lange planken zijn:}{vastgespijkerd op hoge}{en lage paaltjes}\\

\haiku{Als gendarmes in,:}{de buurt komen begint een}{vrouw hard te roepen}\\

\haiku{Voor zichzelf had hij.}{al lang precies uitgemaakt}{wat hem te doen stond}\\

\haiku{Daar weet men alleen, '. '}{dat hijs morgens nog de}{paarden gevoerd heeft}\\

\haiku{Leike heeft Joean.}{bezworen er met geen woord}{over te beginnen}\\

\haiku{Ja maar, zeggen 'n,.}{paar grote boeren Leike}{je begrijpt dat niet}\\

\haiku{Diezelfde avond nog.}{rijdt Lommele Joean met}{zijn handkar naar Swier}\\

\haiku{Om tien uur in de.}{avond van 11 januari}{is hij gestorven}\\

\haiku{Niemand weet 't. Sinds.}{Napoleon is er al}{zoveel veranderd}\\

\haiku{{\textquoteright} Vrouw Dresen, die helpt,.}{bij geboorte en dood heeft}{hem al afgelegd}\\

\haiku{{\textquoteright} Rustig gaat hij voor,.}{hen uit ongeveer twintig}{mensen volgen hem}\\

\haiku{Als een woelrat kruipt.}{hij daarin weg onder het}{vochtige lover}\\

\haiku{Kort gezegd zijn het.}{vooral drie dingen die ik}{van hem geleerd heb}\\

\haiku{{\textquoteright} Vragend kijkt hij naar,:}{de pastoor die min of meer}{instemmend aanvult}\\

\haiku{En de pastoor.... 'n,!}{deserteur helpen dat is}{levensgevaarlijk}\\

\haiku{Ieder kind kreeg er.}{een en hield het angstvallig}{in beide handen}\\

\haiku{Een heel eind van het.}{karretje vandaan komen}{zij op het kerkpad}\\

\haiku{In de loop van de,.}{ochtend trekt de mist op er}{zit vorst in de lucht}\\

\haiku{De Fransen zouden.}{al allemaal over de Maas}{weggetrokken zijn}\\

\haiku{Hei, Kleintjes mit de,!}{korte beintjes zing ins get}{om werm te waere}\\

\haiku{Soms stonden ze zo.}{maar als reuzenfakkels in}{het veld te branden}\\

\haiku{Zij moet de garven.}{naar de wagen steken en}{Joep moet optasten}\\

\haiku{Hij staat alweer op.}{de wagen en zij heft een}{garf op de gaffel}\\

\haiku{Maar.... aan zijn arm loopt.}{Leineke en om hem heen}{jubelt de lente}\\

\haiku{ja, hij zal zorgen.}{dat hun kinderen leren}{lezen en schrijven}\\

\haiku{Maar nu is het een,.}{wildeman die ruig en vlug}{de mis afraffelt}\\

\haiku{Twee keer per dag moesten.}{de boeren er hun paarden}{en koeien drenken}\\

\haiku{Het gewas stond er,.}{schriel bij er kwam maar weinig}{korrel in de aren}\\

\haiku{De Vildersjtr\"op zit.}{met een wit weggetrokken}{gezicht bij de schouw}\\

\haiku{Op hoeveel van haar!}{huwelijksdagen heeft zij}{het niet gebeden}\\

\haiku{Hij foetert tegen,.}{Anna en Adam die het ook}{niet kunnen helpen}\\

\haiku{Bleef Driek aandringen.}{dan klauwde zij van zich af}{als een wilde kat}\\

\haiku{Kleintjes zelf zou hem.}{zeker aanraden om naar}{de Bongerd te gaan}\\

\haiku{De citer kraakt, een.}{snaar springt los en blijft langzaam}{heen en weer veren}\\

\haiku{Ku\"eb zegt, dat hij.}{even zal gaan vragen en loopt}{terug naar Anna}\\

\haiku{Op 30 april 1844 ging.}{hij haar dood aangeven bij}{de burgemeester}\\

\haiku{Daartussen trekt de.}{horizon een strakke lijn}{tegen de hemel}\\

\haiku{Een machtig kasteel.}{en een kerk liggen binnen}{brede slotgrachten}\\

\haiku{Zijn gouden borstkruis.}{en zijn ring glinsteren op}{in de schemering}\\

\haiku{Vertwijfeld grijpen.}{zij elkaar bij de hand en}{roepen om Sjtefan}\\

\haiku{Dat eigenbelang.}{uiteindelijk het belang}{van anderen is}\\

\haiku{Geschiedenis van{\textquoteright}.}{de beide Limburgen door}{Dr. Jappe Alberts}\\

\haiku{{\textquoteleft}Verloskunde en{\textquoteright}.}{kindersterfte in Limburg}{door Dr. J.H. Starmans}\\

\section{Frans Erens}

\subsection{Uit: Vertelling en mijmering}

\haiku{Hij sloeg aan ieder,.}{keer in den nacht wanneer er}{iemand voorbijging}\\

\haiku{Nu en dan kwam de.}{sikkel der maan achter de}{wolken te voorschijn}\\

\haiku{'Ja,' zei de vrouw, 'ik,,'.}{moet zeggen dat hij nooit zoo}{doet als dezen avond}\\

\haiku{Ik heb mij over niets.}{te beklagen gehad en}{bleef daar twee jaren}\\

\haiku{* * * ~ Zoo vertelde,.}{Nicolaas den nacht door tot}{vroeg in den morgen}\\

\haiku{breng mijnheer naar zijn,.}{kamer want hij zal zelf den}{weg moeilijk vinden}\\

\haiku{Toen ik mijn voeten,.}{onder de lakens strekte}{schrok ik ongemeen}\\

\haiku{Ik durfde  daar.}{zeker niet naar vragen den}{volgenden morgen}\\

\haiku{De zekerheid van.}{mijn eigen gedachten zou}{er onder lijden}\\

\haiku{Het was een grijze.}{haarlok ter lengte van meer}{dan een kwart meter}\\

\haiku{Kynon, zoon van Clydno,.}{verlangde van Kai wat de}{Koning had gezegd}\\

\haiku{De nacht duurde zeer.}{lang en de stilte werd door}{niets daar gebroken}\\

\haiku{Zij zullen zingen,.}{zooals gij het in uw eigen}{land nooit hebt gehoord}\\

\haiku{Geen wind rimpelde.}{het zilveren water aan}{den voet der muren}\\

\haiku{'Ik ben blij, dat gij,.}{geene andere oorzaak hebt}{mij weg te zenden}\\

\haiku{Misschien deed hij het.}{om niet ongehoorzaam te}{zijn aan den koning}\\

\haiku{Haar blauwen voorschoot}{had zij opgenomen en}{aanhoudend veegde}\\

\haiku{Van den vader kon,.}{men nooit veel zeggen dat was}{wel een brave man}\\

\haiku{Een paar zwaluwen,,.}{die op de luiken zaten}{te kweelen vlogen}\\

\haiku{De knecht kwam uit zijn:}{bed en naar binnen gaande}{vroeg hij aan Anna}\\

\haiku{Ongemerkt was zij.}{ingeslapen en zij had}{erg akelig gedroomd}\\

\haiku{Zet maar de koffie,.}{klaar ik sta op en ga er}{van morgen nog heen}\\

\haiku{Het is goed, dat de.'}{menschen zoo min mogelijk}{komen te weten}\\

\haiku{hij dompelde in,:}{de kom besprenkelde hij}{muren en bodem}\\

\haiku{'Wij moeten even gaan',, '.}{zitten zei de pastoorwant}{dat was hard werken}\\

\haiku{'Het verwondert me,.}{dat nicht Olfers nog niet bij}{ons heeft gelogeerd}\\

\haiku{Ja, ze zouden nu,}{hun hart eens ophalen aan}{een warm bordje soep}\\

\haiku{Zijn das is breed en,.}{zwart maar het zwart is door den}{ouderdom verkleurd}\\

\haiku{Hij is zorgvuldig.}{geschoren en draagt alleen}{een kleinen knevel}\\

\haiku{Men denke echter,}{niet dat hij het er op aan}{zou willen leggen}\\

\haiku{* * * ~ De heer Strowski.}{is een nog zeer jong docent}{van de Sorbonne}\\

\haiku{spreekt met hem niet uit '',.}{de hoogte maar als tot een}{vriend en gelijke}\\

\haiku{Zoo zag ik langs de.}{daken een spion door vier}{mannen achtervolgd}\\

\haiku{* * * ~ Uit steden en.}{dorpen trekken jammerend}{de vluchtelingen}\\

\haiku{* * * ~ Dat Goethe nooit,.}{een vurig patriot is}{geweest is een troost}\\

\haiku{Nu en dan hoort men.}{eens spreken van Joffre}{of van Hindenburg}\\

\haiku{De namen von Kluck,.}{Hausen enz. zijn verdwenen}{in de duisternis}\\

\haiku{* * * ~ De Pruisische.}{geest is een verschijnsel van}{den modernen tijd}\\

\haiku{Of het werkelijk,.}{overwonnen zal worden blijkt}{niet met zekerheid}\\

\haiku{Meent gij, dat hij zich?}{ooit van verdelgingszucht zal}{kunnen onthouden}\\

\haiku{Zij ziet dan geheel;}{wit als van karton met een}{blanke schittering}\\

\haiku{Zij vroegen het als.}{in een sleur en roeiden gauw}{weer naar land terug}\\

\haiku{Spoedig waren we.}{aan land en wij moesten eerst naar}{het customhouse}\\

\haiku{Daarom heen zaten.}{op banken gesluierde}{vrouwen te wachten}\\

\haiku{De lange baard in,,.}{twee gedeeld doch nog niet grijs}{doet hem herkennen}\\

\haiku{Doch de vrouwen bij,,,.}{de Spaansche Basken hebben}{dat naar ik meen wel}\\

\haiku{In hun land vonden.}{dan ook de Carlisten hun}{meesten aanhang}\\

\haiku{Maar zij glijden niet,.}{af zij kennen de waarde}{van iederen stap}\\

\haiku{Het zich bewegen,.}{is de strijd tegen den dood}{tegen den stilstand}\\

\subsection{Uit: Vervlogen jaren}

\haiku{In de betrachting.}{van dien plicht week Frans Erens voor}{niets op de wereld}\\

\haiku{Hij aanvaardde geen.}{zekerheid op gezag der}{domme herhaling}\\

\haiku{Dit was een gracht, door,.}{menschenhanden gegraven}{die diep en breed was}\\

\haiku{{\textquoteleft}Im Namen Gottes,,,.}{Vatters und des Sohns und}{des H. Geistes Amen}\\

\haiku{Het grondtype van.}{deze bouworde is de}{Romeinsche villa}\\

\haiku{{\textquoteleft}Geef mij dien eens hier{\textquoteright},, {\textquoteleft}.}{zei de pastooren rij nu}{die kar uit den mest}\\

\haiku{Soms liet ik het in.}{het midden langs het touw naar}{beneden vallen}\\

\haiku{slechts viel er nu en.}{dan een schietgebedje in}{het Fransch tusschendoor}\\

\haiku{Mijn grootmoeder zei.}{dat half voor zichzelf en dacht}{er verder niet over}\\

\haiku{Soms zag ik hem met.}{een groot houten plateel op}{de knie\"en zitten}\\

\haiku{Dit generaalschap.}{was voor Mathies de grootste}{vreugd van zijn leven}\\

\haiku{Hij was met alles.}{tevreden en maakte nooit}{eenige aanmerking}\\

\haiku{{\textquoteright} {\textquoteleft}O, ja{\textquoteright}, zei ik met, {\textquoteleft}.}{veel zelfbewustzijndaar heb}{ik veel aan gedaan}\\

\haiku{Wanneer ik eenige,.}{weken in het Noorden was}{rook ik het niet meer}\\

\haiku{De geur van een stad;}{of een land is toch wel iets}{zeer eigenaardigs}\\

\haiku{Een Leidsch professor;}{was voor mij een wezen van}{een hoogeren rang}\\

\haiku{{\textquoteleft}Conduisez-moi{\textquoteright}.}{au Quartier Latin \`a un}{hotel quelconque}\\

\haiku{Dubreuil was een zeer.}{begaafd artiest en een goed}{muziekcriticus}\\

\haiku{Hij schilderde in,.}{dien tijd een portret van mij}{dat ik nog bezit}\\

\haiku{Hij was uit Tours naar.}{Parijs gekomen om de}{Rechten te studeeren}\\

\haiku{Dat was het woord, dat.}{men gewoonlijk gebruikte}{voor supr\^emen lof}\\

\haiku{In ieder geval,.}{zal er een uitwerking zijn}{die hem gunstig is}\\

\haiku{la main mon cher ami,.}{Maurice Barr\`es   9}{Rue Victor Cousin}\\

\haiku{Die eerste tijd van,.}{Le Chat Noir was de beste}{ontegenzeglijk}\\

\haiku{Mor\'eas' ambitie.}{in Parijs was een groot Fransch}{dichter te worden}\\

\haiku{Den laatsten keer, dat,.}{ik hem heb gesproken was}{in La Vachette}\\

\haiku{Wij, Hollanders, wij.}{vonden dien toon op het laatst}{wel wat vermoeiend}\\

\haiku{Hij gedroeg zich als.}{een sto{\"\i}cijn en zag kalm}{den dood aankomen}\\

\haiku{hij zoo, evenals het, {\textquoteleft}{\textquoteright}.}{achterhoofd dat voor hemune}{face \'eteinte was}\\

\haiku{Den derden broer, die,;}{een gezien beeldhouwer was}{heb ik niet gekend}\\

\haiku{Rollinat was dus,,.}{wat men noemt arriv\'e met}{vijfendertig jaar}\\

\haiku{Aan de beweging.}{van zijn verzen ontbrak de}{spontane{\"\i}teit}\\

\haiku{Aan hoeveel smarten!}{en verwikkelingen was}{ik dan ontkomen}\\

\haiku{Toen Kloos er voor het,.}{eerst kwam zat ik toevallig}{naast hem in den kring}\\

\haiku{Van Deyssel nam het.}{bezoek zeer ernstig op en}{was zenuwachtig}\\

\haiku{Deze losheid was.}{juist voor ons een bekoring}{en hield ons bijeen}\\

\haiku{De nakomer heeft;}{er behoefte aan om te}{synthetiseeren}\\

\haiku{het boertige is,.}{niet zijn zaak maar wel het fijn}{humoristische}\\

\haiku{maar zooals veel feiten.}{uit die verre dagen is}{dat mij ontvallen}\\

\haiku{wellicht vergeet ik.}{bij deze opsomming nog}{den een of ander}\\

\haiku{Het is waar, dat wij.}{Feith en Bilderdijk op het}{oogenblik waardeeren}\\

\haiku{Eenigen tijd daarna,.}{hoorde ik dat dit Karel}{Alberdingk Thijm was}\\

\haiku{Het groote gebouw was.}{stampvol en Paap was toen een}{belangrijk persoon}\\

\haiku{Niemand was ooit op.}{het idee gekomen om die}{bron te controleeren}\\

\haiku{Veth was een man, wiens;}{conversatie bijzonder}{interesseerde}\\

\haiku{Goes of anderen,:}{werd er wel eens gescheld en}{als ik aan Kloos vroeg}\\

\haiku{Op een avond had zijn,.}{groote hond Kloos aangevallen}{maar niet gebeten}\\

\haiku{zij hield zich geheel}{onbeweeglijk en langs haar}{wang zag ik langzaam}\\

\haiku{Van het begin van}{zijn verblijf te Amsterdam}{ging hij met ons om}\\

\haiku{Soms zei hij dat hij.}{ging eten met dien of dien en}{vroeg of ik mee ging}\\

\haiku{hij behoefde er,.}{niet zuinig mee te zijn want}{hij had er genoeg}\\

\haiku{Als een werkman stond,.}{hij daar voor het doek deze}{kleine Hercules}\\

\haiku{Josef Israels had.}{in zijn gebaren iets meer}{magistraals dan Isa\"ac}\\

\haiku{en alleen aandacht.}{had voor hetgeen in Parijs}{over hem werd gezegd}\\

\haiku{ook aan mij bekend.}{waren en dat ik zijn vrouw}{dikwijls had ontmoet}\\

\haiku{Hij bleef passief en.}{liet zich die houding van Kloos}{goedig aanleunen}\\

\haiku{Verlaine was een,,.}{rakker een stijfhoofdige}{een onwillige}\\

\haiku{Tegenover het huis.}{en verderop lagen meer}{buitenverblijven}\\

\haiku{ik vergeten, ik.}{meen dat het op den linker}{Seine-oever was}\\

\haiku{Ik word gekend door.}{den Vader en dat is mij}{ten slotte genoeg}\\

\haiku{{\textquoteleft}Ik sta aan den rand...{\textquoteright},;}{is door den auteur kort v\'o\'or}{zijn dood gedicteerd}\\

\haiku{Ik herinner mij:}{nog heel goed hoe door een der}{onzen werd gezegd}\\

\haiku{Hier moet men letten:}{op de teekenen en men}{zou kunnen zeggen}\\

\haiku{4Gringoire was een.}{Fransch dichter van het einde}{der middeleeuwen}\\

\haiku{Hij heeft zich meen ik,.}{teruggetrokken omdat}{hij niet werd betaald}\\

\subsection{Uit: Vervlogen jaren}

\haiku{daarom had ik in.}{alle kamers papier en}{potlood klaar liggen}\\

\haiku{het allerlaatste.}{gedeelte heeft hij kort voor}{zijn dood gedicteerd}\\

\haiku{Mijn moeder had mij}{geleerd het afgepelde}{eitje plat te slaan}\\

\haiku{Dit was een gracht, door,.}{mensenhanden gegraven}{die diep en breed was}\\

\haiku{{\textquoteleft}Im Namen Gottes,,,.}{Vatters und des Sohns und}{des H. Geistes Amen}\\

\haiku{Het grondtype van.}{deze bouworde is de}{Romeinse villa}\\

\haiku{{\textquoteleft}Geef mij die eens hier{\textquoteright}, {\textquoteleft}.}{zei de pastooren rij nu}{die kar uit de mest}\\

\haiku{Soms liet ik het in.}{het midden langs het touw naar}{beneden vallen}\\

\haiku{Hijzelf was dronken,.}{geweest maar de dokter was}{volkomen nuchter}\\

\haiku{{\textquoteleft}Mijnheer Menten, wat,!}{moeten wij beginnen wij}{krijgen hongersnood}\\

\haiku{slechts viel er nu en.}{dan een schietgebedje in}{het Frans tussendoor}\\

\haiku{Mijn grootmoeder zei.}{dat half voor zichzelf en dacht}{er verder niet over}\\

\haiku{Soms zag ik hem met.}{een groot houten plateel op}{de knie\"en zitten}\\

\haiku{Dit generaalschap.}{was voor Mathies de grootste}{vreugd van zijn leven}\\

\haiku{Hij was met alles.}{tevreden en maakte nooit}{enige aanmerking}\\

\haiku{Hij kwam dikwijls bij.}{ons logeren en ging dan}{met mij wandelen}\\

\haiku{{\textquoteright} Als het donker was,.}{was de wagen van binnen}{dikwijls niet verlicht}\\

\haiku{Op de bodem werd '.}{s winters stro gelegd voor}{de koude voeten}\\

\haiku{Het was toen mode.}{zich de kop van de Franse}{keizer te maken}\\

\haiku{{\textquoteright} Een bravo-geroep.}{steeg op uit de bezoekers}{van het restaurant}\\

\haiku{Op het koor naast het.}{altaar zat een deftig heer}{met een jong meisje}\\

\haiku{Hij was een lange,.}{bleke jongen met blond haar}{en een plat gezicht}\\

\haiku{Hij vertelde met}{wie zij getrouwd waren en}{hoeveel kinderen}\\

\haiku{Hij sprak in het Frans,}{en toen hij aan het begin}{zei dat hij \'etranger}\\

\haiku{Wanneer ik enige,.}{weken in het Noorden was}{rook ik het niet meer}\\

\haiku{De geur van een stad;}{of een land is toch wel iets}{zeer eigenaardigs}\\

\haiku{Een Leids professor;}{was voor mij een wezen van}{een hogere rang}\\

\haiku{Men had de zaal door.}{een barri\`ere verdeeld}{in twee gedeelten}\\

\haiku{vier mensen van mijn.}{reisgezelschap in een open}{rijtuig gezeten}\\

\haiku{{\textquoteleft}Conduisez-moi.}{au quartier Latin \`a un}{hotel quelconque}\\

\haiku{Dubreuilh was een zeer.}{begaafd artiest en een goed}{muziekcriticus}\\

\haiku{Zij waren beiden.}{ongeveer \'e\'en- of}{twee\"entwintig jaar}\\

\haiku{Hij schilderde in,}{die tijd een portret van mij}{dat ik nog bezit.125}\\

\haiku{{\textquoteright} Dat was het woord, dat.}{men gewoonlijk gebruikte}{voor supr\^eme lof}\\

\haiku{In ieder geval,.}{zal er een uitwerking zijn}{die hem gunstig is}\\

\haiku{Ik bleef dan niet lang,.}{wilde hem  niet van zijn}{arbeid afhouden}\\

\haiku{C'est am\`erement peu,.}{pay\'e mais enfin c'est aussi}{peu de travail}\\

\haiku{Die eerste tijd van,.}{Le Chat Noir was de beste}{ontegenzeglijk}\\

\haiku{Mor\'eas' ambitie.}{in Parijs was een groot Frans}{dichter te worden}\\

\haiku{Plus de r\^eves d'}{azur au fond des bosquets verts O\`u}{le rossignolet}\\

\haiku{Hij heeft ook enige.}{werken uit de zestiende}{eeuw uitgegeven}\\

\haiku{Mevrouw Verlaine.}{zag er in die tijd breed en}{welgedaan uit}\\

\haiku{De laatste keer, dat,.}{ik hem heb gesproken was}{in de Vachette}\\

\haiku{Wij, Hollanders, wij.}{vonden die toon op het laatst}{wel wat vermoeiend}\\

\haiku{Hij gedroeg zich als.}{een sto{\"\i}cijn en zag kalm}{de dood aankomen}\\

\haiku{Alle genot is.}{enkelvoudig en zo ook}{het esthetische}\\

\haiku{De derde broer,229 die,;}{een gezien beeldhouwer was}{heb ik niet gekend}\\

\haiku{Rollinat was dus,,.}{wat men noemt arriv\'e met}{vijfendertig jaar}\\

\haiku{Aan de beweging.}{van zijn verzen ontbrak de}{spontane{\"\i}teit}\\

\haiku{Een figuur in het,.}{quartier Latin zoals ze}{zelden voorkwamen}\\

\haiku{Hij vertelde die,.}{zaak zeer uitvoerig onder}{ademloze stilte}\\

\haiku{Aan hoeveel smarten!}{en verwikkelingen was}{ik dan ontkomen}\\

\haiku{Toen Kloos er voor het,.}{eerst kwam zat ik toevallig}{naast hem in de kring}\\

\haiku{Van Deyssel nam het.}{bezoek zeer ernstig op en}{was zenuwachtig}\\

\haiku{Deze losheid was.}{juist voor ons een bekoring}{en hield ons bijeen}\\

\haiku{De nakomer heeft;}{er behoefte aan om te}{synthetiseren}\\

\haiku{het boertige is,.}{niet zijn zaak maar wel het fijn}{humoristische}\\

\haiku{Ik zie nog haar fraaie,,.}{bleke hand waarmee zij mij}{de regels aanwees}\\

\haiku{Frank van der Goes was.}{misschien wel de geestigste}{van de vriendenkring}\\

\haiku{wellicht vergeet ik.}{bij deze opsomming nog}{de een of ander}\\

\haiku{Alberdingk Thijm was,.}{gaf hij altijd een stuk van}{hemzelf ten beste}\\

\haiku{Het is waar, dat wij.}{Feith en Bilderdijk op het}{ogenblik waarderen}\\

\haiku{Enige tijd daarna,.}{hoorde ik dat dit Karel}{Alberdingk Thijm was}\\

\haiku{Daar zaten alleen;}{die drie jongemannen in}{een grote stilte}\\

\haiku{{\textquoteleft}Ik zet het je, ze.}{op een geschikte manier}{terug te leggen}\\

\haiku{Hij zei, dat hij er.}{niet werken kon en verliet}{Nieder-Ingelheim}\\

\haiku{Verder wist hij ook.}{niets over de kwaliteit van}{het grote gedicht}\\

\haiku{Een der paden die,;}{zijn geest heeft doorwandeld was}{ook een der mijne}\\

\haiku{Zo vertelde hij ',. '}{s avonds deze aankomst toen}{Groux er niet bij was}\\

\haiku{{\textquotedblright}{\textquoteright} Veth was een man, wiens;}{conversatie bijzonder}{interesseerde}\\

\haiku{Ook voelde zij zich.}{ongelukkig omdat zij}{geen kinderen had}\\

\haiku{zij hield zich geheel}{onbeweeglijk en langs haar}{wang zag ik langzaam}\\

\haiku{Van het begin van}{zijn verblijf te Amsterdam}{ging hij met ons om}\\

\haiku{Om twaalf uur gaf Ising.}{aan Betsy een revolver}{om af te schieten}\\

\haiku{Een houten beeld van,,.}{Vondel uit de achttiende}{eeuw stond in de gang}\\

\haiku{Hij woonde op de;}{Nieuwezijds Voorburgwal op}{twee achterkamers}\\

\haiku{Geloof aan God en.}{godsdienst heeft hij bij zijn dood}{teruggekregen}\\

\haiku{en alleen aandacht.}{had voor hetgeen in Parijs}{over hem werd gezegd}\\

\haiku{Hij bleef passief en.}{liet zich die houding van Kloos}{goedig aanleunen}\\

\haiku{Verlaine was een,,.}{rakker een stijfhoofdige}{een onwillige}\\

\haiku{Tegenover het huis.}{en verderop lagen meer}{buitenverblijven}\\

\haiku{ik vergeten, ik}{meen dat het op de linker}{Seine-oever was.556}\\

\haiku{Wanneer deze vorst,.}{een vrouw ware geweest kon}{hij niet anders zijn}\\

\haiku{Zoals ik reeds zei,;}{zag ik in Roosdorp een goed}{schrijverstalent}\\

\haiku{Het is genoeg, dat,.}{hij \'e\'en goed \'e\'en zeer goed boek}{heeft nagelaten}\\

\haiku{binnengingen en,:}{ik het lege bilart zag}{zei ik plotseling}\\

\haiku{Soms zei hij dat hij.}{ging eten met die of die en}{vroeg of ik meeging}\\

\haiku{Daarna gingen wij,.}{naar L'Isle Saint Louis naar de}{Quai de  Bourbon}\\

\haiku{hij behoefde er,.}{niet zuinig mee te zijn want}{hij had er genoeg}\\

\haiku{Hij maakte Isaac.}{nog een compliment over diens}{vloeiend Spaans-spreken}\\

\haiku{Als ik dat dronk, zei,.}{ze zou ik de volgende}{dag weer beter zijn}\\

\haiku{In geheel Spanje.}{is waarschijnlijk geen mooier}{kerkgebouw dan dit}\\

\haiku{En ook van buiten.}{dunkt mij deze kathedraal}{enigszins overladen}\\

\haiku{Als een werkman stond,.}{hij daar voor het doek deze}{kleine Hercules}\\

\haiku{Hij is volgens mij.}{niet overtroffen door die na}{hem zijn gekomen}\\

\haiku{Ik herinner mij:}{nog heel goed hoe door een der}{onzen werd gezegd}\\

\haiku{Hier moet men letten:}{op de tekenen en men}{zou kunnen zeggen}\\

\haiku{Ik word gekend door.}{de Vader en dat is mij}{ten slotte genoeg}\\

\haiku{Binnen een dag of.}{wat stel ik mij voor er u}{een cadeau te doen}\\

\haiku{110, 121, 125, 127-136,,,,,,,,,,-,.}{143 156 159 175 189 192 204}{212 291 303306 377}\\

\haiku{378, 383, 384, 386, 390-,,,,,,,:}{394 396 402 403 409 439 445}{Barr\`es Philippe}\\

\haiku{431 Belderok, A.J.:,,,, (-):}{244 245 423 424 Beljame}{Alexandre18431906}\\

\haiku{117, 250, 252-255, 260,,,,, (-):}{276 341 425 426 Diepenbrock}{Melchior17981853}\\

\haiku{117-119, 122, 132, 381,,,,, (-):}{382 387 389 393 Duinkerken}{Anton van19031968}\\

\haiku{30, 33, 34, 53-56,,, [] (-):}{73 74 76 Erens-Borghans}{grootmoeder17841867}\\

\haiku{57, 84, 180, 209, 210,,,,,,, (-):}{216 223 282 352 370 410 Vos}{Jan C. de18551931}\\

\haiku{Een van de laatste (-)}{portretfoto's van Isaac}{Israels18651934}\\

\haiku{Eins mach Zehn, / Und, /, /.}{Zwei lasz gehn Und Drei macht}{gleich So bist du reich}\\

\haiku{62In het handschrift, van,:}{8 september 1922 voegde}{Erens hierachter toe}\\

\haiku{Quoi qu'il en soit, c'est,.}{fait et il ne me reste}{qu'\`a dire c'est bien}\\

\haiku{90Gaston Dubreuilh.}{was in 1882 een jongeman}{van vijfentwintig}\\

\haiku{Vous voyez cela,.}{entre la rue Racine}{et la rue de l'Od\'eon}\\

\haiku{Je l'apprends l\`a, et.}{pour le vexer je m'occupe}{de la petite}\\

\haiku{Het blijkt vervaardigd,.}{te zijn door M. Alophe}{rue Royale 25}\\

\haiku{Hij heeft zich meen ik,.}{teruggetrokken omdat}{hij niet werd betaald}\\

\haiku{{\textquoteleft}en als u u nu,.}{helemaal ontkleedt dan laat}{mij dat nog ijskoud}\\

\haiku{301Jan C. de Vos (1855-),.}{1931 als acteur ook een groot}{karakterspeler}\\

\haiku{is haast de druppel,, '?}{daar Waar hij me\^e valt of gunt}{gem nog een poos}\\

\haiku{Het boek van Kuid en{\textquoteright},;}{God verscheen in De Nieuwe}{Gids oktober 1888}\\

\haiku{van Looy vertelde,:}{op 10 april 1890 in een brief}{aan Willem Witsen}\\

\haiku{Hij lag erg klein, net,.}{een beestje zoals Isaac}{Israels dat noemde}\\

\haiku{Vermoedelijk stond.}{een en ander hem niet meer}{helder voor de geest}\\

\haiku{{\textquoteleft}Fran\c{c}ois / doe me het / /.}{genoegen en laat onder}{geen pretext Juffr}\\

\haiku{520In het onder 517.}{genoemde boek valt zijn naam}{geen enkele maal}\\

\haiku{556Odilon Redon (-).}{18401916 woonde toen in de}{rue St. Romain 20}\\

\haiku{558Onder de schuilnaam:}{J. Staphorst schreef Jan Veth over}{Odilon Redon in}\\

\haiku{Er was een slapte.}{ingetreden en die hield}{nog enige tijd aan}\\

\section{Henrica van Erp}

\subsection{Uit: Kroniek van Vrouwenklooster in De Bilt}

\haiku{wij schonken die schout.}{met die buer die daer bij}{waren een vat bier}\\

\haiku{hij was in groote noot.}{op zee en sterf in Spangen}{op den 27 dach}\\

\haiku{'s Middags omstreeks.}{\'e\'en uur verbrandde het huis}{van onze priester}\\

\haiku{Een huis met een man.}{en vrouw en vijf kinderen}{erin dreef daar weg}\\

\haiku{De Schalkwijkse.}{wetering slibde voor een}{groot deel met zand dicht}\\

\haiku{Het ingezaaide.}{koren kwam onder water}{te staan en bedierf}\\

\haiku{de borgers van Utrecht]}{houden haar biscop buyten de}{stad ~ Anno 1527}\\

\haiku{Edoch wy brochten al}{ons beste goed en beesten}{tot Amersfoort ende}\\

\haiku{waren sy binnen,.}{Vyanen tot Bueren}{by hare ouders}\\

\haiku{Ende daer word die}{cloosteren voorgehouden}{dat sy verdingen}\\

\haiku{Ende hij lag 't.}{huys op St. Janskerkhof in}{des domdekens huys}\\

\haiku{tocht van Marten van]}{Rossem in Brabant ~ Int}{jaer Ons Heeren 1542}\\

\haiku{Die camenieren.}{hebben 5 hoorns guldens voor}{die bruyt klederen}\\

\haiku{Al die andere[].}{kneghs en maagden elx voor}{t een braspenning}\\

\haiku{Dinslo kwamen al;}{in 1277 en 1279 in handen}{van Vrouwenklooster}\\

\haiku{In 1506 vertrok hij.}{naar Spanje om zijn rechten}{te laten gelden}\\

\haiku{dit is dat gelt dat}{wij ghebuert hebben tot}{dat recht van saeken}\\

\haiku{Hij ging al vroeg over;}{tot de partij van hertog}{Karel van Gelre}\\

\haiku{86Bovenkleed voor.}{mannen met wijde slippen}{onder de gordel}\\

\haiku{Henrica van Erp.}{verantwoordt op deze plaats}{de 18 stuivers niet}\\

\haiku{143Mogelijk gaat;}{het hier om een zekere}{hopman Belrebus}\\

\haiku{Gijsbertss[oen] scout op[]}{Die Bilt aen meyster Peter}{Goesen woenen}\\

\haiku{Item die metzelaers[].}{scheenck ons convent}{een halliff aem wijns}\\

\haiku{van Bommel of van (),.}{Boemel{\textdagger}1549 waarschijnlijk te}{Bommel geboren}\\

\section{P.N. van Eyck}

\subsection{Uit: Opgang}

\haiku{Ik sloot mijn oogen en.}{zag niets dan de rozige}{helheid der leden}\\

\haiku{wanneer ik hem eens.}{zal ontvangen hebben als}{een roerlooze rijkdom}\\

\haiku{Het is goed in mij,.}{daar ik zijn kracht als een lust}{voel in mijn zwakheid}\\

\haiku{Men is een van hen:}{die haar bruising dwingt tot een}{zoete slavernij}\\

\haiku{De natuur is te}{groot dan dat zij rijker kon}{worden als de mensch}\\

\haiku{Zij geschiedt langzaam,,.}{onder verschrikkelijke}{krampen maar geschiedt}\\

\haiku{Daarom ook k\'unnen,.}{wij niet meer te gronde gaan}{als die voorgangers}\\

\haiku{een glans, hing het te.}{ademen in de glans waar het}{langzaam in verging}\\
