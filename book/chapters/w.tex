\chapter[13 auteurs, 938 haiku's]{dertien auteurs, negenhonderdachtendertig haiku's}

\section{Jacqueline van der Waals}

\subsection{Uit: Noortje Velt}

\haiku{In het zachte blauw,....}{des hemels dreven wit de}{wolken ongekleurd}\\

\haiku{{\textquoteright} En moeder knikte:}{dan van neen en Nora zei}{verontschuldigend}\\

\haiku{Beschaamd vluchtte ze - -.}{heen holde ze de trappen}{af naar beneden}\\

\haiku{Wat zijn de bloemen,,?}{de kleuren wat is het licht}{voor een blinde}\\

\haiku{Straf me niet in het,.}{medaillonnetje dat ik}{niet verliezen mag}\\

\haiku{{\textquoteright} Maar met die woorden...}{had ze toch weer het bestaan}{van het kind erkend}\\

\haiku{Waarom had juf niet,?}{geschreven als ze haar iets}{te vertellen had}\\

\haiku{Hoe dankbaar verrast,!}{hoe vol bewondering zou}{Jan dan voor haar zijn}\\

\haiku{Ze had zich immers?}{juist zoo gelukkig in haar}{eenzaamheid gevoeld}\\

\haiku{En was er dan ook,?}{eens niemand die  zich om}{je bekommerde}\\

\haiku{{\textquoteleft}En de geheele.}{familie is er toen ook}{erg boos over geweest}\\

\haiku{u vergist u, ik,.}{ben geen meisje dat op school}{uitgelachen wordt}\\

\haiku{Vader en Jan en....!}{Henri en juf en Emy en}{nu ook de poppen}\\

\haiku{Nora zweeg dus en.}{speelde en zoo spelende}{ging de tijd voorbij}\\

\haiku{Nous voulons vivre?}{dans l'id\'ee des autres d'une}{vie imaginaire}\\

\haiku{telkens bedachten.}{ze iets nieuws om haar aandacht}{tot zich te trekken}\\

\haiku{Maar zou Nora het?}{niet vervelend vinden naar}{haar te luisteren}\\

\haiku{die was verleden.}{week allebei haar knie\"en}{kapot gevallen}\\

\haiku{Zag juffrouw Nora,?}{niet dat de kousen met een}{M gemerkt waren}\\

\haiku{Ze begreep volstrekt,.}{niet wat de meisjes nu weer}{te lachen hadden}\\

\haiku{{\textquoteright} Nora's mond lachte.}{ondeugend en in haar oogen}{flikkerde de pret}\\

\haiku{Maar natuurlijk jij -.}{Marie en ik vallen nog}{niet in de termen}\\

\haiku{Triomfeerend hield,.}{ze den brief in de hoogte}{haar oogen schitterden}\\

\haiku{Voelde ze misschien,?}{toch dat ze eigenlijk iets}{goed te maken had}\\

\haiku{Hij danste weinig,?}{misschien verwachtte hij zijn}{Asschepoester nog}\\

\haiku{{\textquoteright} {\textquoteleft}En mij onwereldsch,{\textquoteright}, {\textquoteleft}.}{lachte Ellywant ik vind}{een bal goddelijk}\\

\haiku{- We hebben geen recht,.}{te spreken over hetgeen we}{zoo weinig kennen}\\

\haiku{Nora keek om zich,.}{heen naar de grauwe trieste}{velden en zuchtte}\\

\haiku{Ook dit, dat ze nu,,.}{naar het Leger des Heils ging}{had geen doel geen zin}\\

\haiku{Zie, nu kon er geen,,.}{aarzeling geen twijfel geen}{bangheid meer bestaan}\\

\haiku{de stem, waarmee ze,.}{spreken ging klonk wonderbaar}{rustig en innig}\\

\haiku{Gij zoudt spreken als,,:}{ik spreek getuigen als ik}{getuig zeggende}\\

\haiku{{\textquoteleft}Dier'bre Heiland,, ',!}{mijn Verlosserk Ben de}{uwe goddelijk Lam}\\

\haiku{{\textquoteright} zei hij hartelijk.}{en Nora's hart sprong op van}{dankbare blijdschap}\\

\haiku{{\textquoteright} {\textquoteleft}Neen,{\textquoteright} lachte hij, d\`at.}{moest ze alleen doen als ze}{zelf er lust in had}\\

\haiku{Ook Jaap, die voor zijn,.}{examen zat verscheen weinig}{op het tenniscourt}\\

\haiku{Wat beteekent goed of?}{kwaad anders dan goed of kwaad}{in de gevolgen}\\

\haiku{Mevrouw Merlin was.}{waarlijk ingenomen met}{het engagement}\\

\haiku{Cum laude{\textquoteright} was hij.}{gepromoveerd op een zeer}{uitstekend proefschrift}\\

\haiku{Maar daar lag nu het,.}{maanlicht en sliep en gedroeg}{zich als eigenaar}\\

\haiku{Hoe konden ze haar?}{zoo laf verloochenen voor}{dat slapende licht}\\

\haiku{{\textquoteright} sprak ze zacht, {\textquoteleft}indien,....}{iemand me liefhad hij zou}{me z\'o\'o kunnen zien}\\

\haiku{En wees nu niet meer,,}{boos vergeef het me indien}{ik onredelijk}\\

\haiku{Jaap,{\textquoteright} zei ze haastig,, {\textquoteleft},!}{en haar stem klonk heel warm en}{hartelijko Jaap}\\

\haiku{Dacht ze heusch, dat?}{er iets was voorgevallen}{tusschen Jaap en haar}\\

\haiku{- {\textquoteleft}Nietwaar, moedertje,?}{maar u wilt er voor ons maar}{niet voor uitkomen}\\

\haiku{Nora verborg het.}{gelaat in de handen en}{snikte en snikte}\\

\haiku{Ik wil alles, ik,,...{\textquoteright}}{durf alles als ik maar weet}{dat het uw wil is}\\

\haiku{{\textquoteleft}Moederliefde is,{\textquoteright}, {\textquoteleft}.}{iets bijzonders zei Marie}{scherpdie is eeuwig}\\

\haiku{wat wijsheid is bij,,!}{haar wordt onzin zoodra}{{\`\i}k het uiten durf}\\

\haiku{Ze glimlachte even.}{om die onwillekeurig}{zelfbekentenis}\\

\haiku{Maar eenvoudige,{\textquoteright}.}{menschen doen anders voegde}{ze er ernstig hij}\\

\haiku{{\textquoteright} {\textquoteleft}Dat is het niet, wat,{\textquoteright},.}{ik noodig heb zei Marie nu}{een beetje kalmer}\\

\haiku{Het is het eenige.}{wat wij worden kunnen met}{onze opvoeding}\\

\haiku{{\textquoteright} {\textquoteleft}Natuurlijk, dat maak,.}{je jezelf wijs wanneer je}{bang bent voor den strijd}\\

\haiku{Je moogt immers net,,.}{doen wat je wilt wat je zelf}{voelt noodig te hebben}\\

\section{Willem Adriaan Wagener}

\subsection{Uit: Sjanghai}

\haiku{Van het jaar nog zal.}{hij uit de gereedschapskist}{worden verwijderd}\\

\haiku{Het zijn de laatste,.}{woorden die de dame in}{haar moedertaal leest}\\

\haiku{Om de 61/2 maat heft.}{zich langs beide zijden van}{het schip een boeggolf}\\

\haiku{Als twee steenen tegen,.}{elkaar worden geslagen}{ontstaat er een vonk}\\

\haiku{Hij stierf verrassend,}{snel met een deeltje van}{Ibsen in de hand.}\\

\haiku{Hij hoopt echter, dat.}{het niet in zijn nadeel zal}{worden uitgelegd}\\

\haiku{De lift knielt achter.}{de dichtkleppende deuren}{en zakt door den grond}\\

\haiku{met zoo'n keurig klein - -;}{schortje van astrakanbont}{kuische vrouwen}\\

\haiku{Hij buigt zich langs den,.}{dokter die Nora opnieuw}{op bed heeft gelegd}\\

\haiku{Laat Japan nu maar.}{wat  mariniertjes op}{de kade plaatsen}\\

\haiku{De motorboot heeft.}{den kop op het Japansche}{eskader gericht}\\

\haiku{Er is veel, waarop.}{zoo direct geen antwoord kan}{worden gegeven}\\

\haiku{Trouwens, u weet toch... -?}{hoe de toestand is En de}{wapenindustrie}\\

\haiku{De journalisten:}{glimlachen en voegen aan}{hun notities toe}\\

\haiku{Godversche, Ypersche, - -.}{lakensche grijp-krengen}{voor een jen pest puist}\\

\haiku{Hem zal de nacht niet,,,.}{deren de nacht die dreigend}{nadert de bloednacht}\\

\haiku{De trein vertrouwt zich.}{aarzelend aan het warnet}{van rangeer-rails toe}\\

\haiku{Zuster Estelle,.}{legt hem de flesch aan maar hij}{heeft het al gedaan}\\

\haiku{Het is een weinig,.}{logische methode die}{hij daarbij toepast}\\

\haiku{Het grijze baardje.}{vangt den eenigen lichtstraal in}{het grijze vertrek}\\

\haiku{Het grijze baardje.}{vangt den eenigen lichtstraal in}{het grijze vertrek}\\

\haiku{En de Japansche,.}{Regeering deelt mee dat daarvan}{geen sprake zal zijn}\\

\haiku{{\textquoteleft}The United States'{\textquoteright}.}{protest was couched in the}{most indignant terms}\\

\haiku{Te vroeg geboren.}{raakt een kind ontijdig aan}{de straat verslingerd}\\

\haiku{De moeder raapt het,.}{kind op wikkelt de streng om}{haar arm en snelt heen}\\

\haiku{Twee dagen is er,,,,.}{aan gewerkt geschaafd geschuurd}{gevijld gepolijst}\\

\haiku{Helaas... de feiten,, (.}{zooals ik reeds zeide hebben}{haar achterhaaldspr}\\

\haiku{dat Japan zich niet,;}{de wet laat stellen zelfs niet}{door den Volkenbond}\\

\haiku{Het staat een nieuwe.}{productie en verdeeling}{der goederen voor}\\

\haiku{Geef mij voor mijn part,.}{communistisch brood maar geef}{mij ook mijn spelen}\\

\haiku{Nora is in een.}{stemming om den heelen nacht}{over Freud te praten}\\

\haiku{Japan zal wapens.}{noodig hebben en China zal}{wapens noodig hebben}\\

\haiku{Over plaatsing van de.}{leening behoeft men zich nog niet}{bezorgd te maken}\\

\haiku{Elke bewuste.}{houding van Engeland zou}{krankzinnigheid zijn}\\

\haiku{Maar bloed en olie zijn.}{hinderlijke begrippen}{voor vermoeide oogen}\\

\haiku{Het gebaar is zelfs.}{iets te kort voor zoo een groote}{ruimte als deze}\\

\haiku{De Japansche leening.}{van 1907 daalde 2{\textonehalf} punt tot}{72 en de 5{\textonehalf} pCt}\\

\haiku{- We m\`oeten elkaar,.}{nog even zien anders ben ik}{doodongelukkig}\\

\haiku{In een verkeersknoop.}{zuigen de remmen alle}{vaart uit den wagen}\\

\haiku{Het Amerikaansche.}{transportschip Chaumont werkt zich}{los van de kade}\\

\haiku{Om den mond van den.}{Jangtse ligt een gordel van}{pantserstaal gesnoerd}\\

\haiku{Kunnen wij onzen?}{kinderen niet een beetje}{vrede meegeven}\\

\haiku{De verklaring van.}{Thomas maakte in den Raad}{een diepen indruk}\\

\haiku{In afwachting van.}{versterkingen staken zij}{het bombardement}\\

\haiku{Hij grondt zijn beroep.}{op art. 15 van het statuut}{van den Volkenbond}\\

\haiku{De dood zaait grijze.}{asch van schemering in de}{conferentiezaal}\\

\haiku{Om twaalf uur zal het.}{noodlot van China breken}{als een rijpe vrucht}\\

\haiku{Als de laatste slag,.}{verklonken zal zijn ontvonkt}{in Sjanghai de hel}\\

\section{Maurits Wagenvoort}

\subsection{Uit: De droomers. Deel 1}

\haiku{En t\`och niet draagt het.}{menschelijk pogen den naam}{van Machteloosheid}\\

\haiku{de schrijver verlangt,.}{slechts te zien dat zij meer en}{beter doen dan hij}\\

\haiku{hij moest zich-zelf,.}{een eigen leven bouwen}{van den bodem af}\\

\haiku{de boomen stil van.}{berusting in het niet af}{te wijzen noodlot}\\

\haiku{Slechts nu en dan had,:}{hij aan het verleden aan}{het begin gedacht}\\

\haiku{zegt zij vrouw, zij spreekt,;}{van de barende moeder}{waarborg der toekomst}\\

\haiku{Hugo bezocht die,;}{armen soms maar hij kon zoo}{zelden wat missen}\\

\haiku{hij klopte aan de,:}{deur der familie G\'erard}{een kinderstem riep}\\

\haiku{{\textquoteleft}wel, meneer Devos,,.}{dat is lief van u dat u}{ons komt bezoeken}\\

\haiku{Ik ben v\'o\'or alles,:}{Italiaan de roem van mijn}{Huis ligt in Itali\"e}\\

\haiku{Geen villa ook had,.}{prachtiger waterwerken}{mooier fonteinen}\\

\haiku{In mijn land zijn veel,,;}{menschen heethoofden die zich}{anarchist noemen}\\

\haiku{het onbeperkte:}{eigendom van den bodem}{en wat daarop leeft}\\

\haiku{tot de volken is,,.}{zij ingevoerd van elders}{nooit doorgedrongen}\\

\haiku{De Christelijke:}{Kerk is sinds lang verdeeld in}{twee  hoofdgroepen}\\

\haiku{Daarom was 't maar:}{het beste het leven te}{nemen zooals het kwam}\\

\haiku{Zij was weggevlucht,.}{toen zij den winkelier had}{hooren naderen}\\

\haiku{Vandamme leerde,;}{hem lezen en schrijven en}{later zelfs latijn}\\

\haiku{{\textquoteleft}Er behoort een goed;}{geloof toe om dat alles}{er in te vinden}\\

\haiku{{\textquoteright} vroeg Hugo, die niet.}{goed in de symboliek van}{zijn kennis thuis was}\\

\haiku{Buitendien, de tien -?}{geboden weten we of}{die wel van Hem zijn}\\

\haiku{Medelijden met,.}{de menschen niemand kan me}{daarin overtreffen}\\

\subsection{Uit: De droomers. Deel 2}

\haiku{{\textquoteright} Pierrot zag hem een,.}{oogenblik verschrikt aan toen}{schudde hij het hoofd}\\

\haiku{Hij schrikte er van,,:}{zoo licht als de jongen}{was en zoo mager}\\

\haiku{het anarchisme.}{een misdadige droom van}{onmogelijkheid}\\

\haiku{De woekeraar, die;}{de uiterste winst zoekt in}{den geringsten arbeid}\\

\haiku{{\textquoteright} Hij wachtte den groet,.}{der kinderen niet af maar}{ging dadelijk heen}\\

\haiku{een klein eindje maar,;}{en  Hugo zette zich}{weer bij zijn bed neer}\\

\haiku{{\textquoteright} {\textquoteleft}Dus u gelooft, dat?}{revoluti\"en voortaan}{onmogelijk zijn}\\

\haiku{Behoeft een naakte,?}{minder een pak kleeren dan een}{hongerige brood}\\

\haiku{elkaars gelijke,,.}{elkaars aanvulling elkaars}{steun bij den arbeid}\\

\haiku{Dat is het Einde,,.}{d\'at is het Beloofde Land}{d\`at is de Hemel}\\

\haiku{De zieke keek hem,;}{een oogenblik zwijgend aan}{en drukte zijn hand}\\

\haiku{{\textquoteright} Hij ging uit om het,}{leven nog eens goed aan te}{zien te zoeken w\'a\'ar}\\

\haiku{Niets kwam hem in de,,.}{gedachte geen enkel plan}{dat uitvoerbaar was}\\

\haiku{Deze vraag bracht hem,;}{in verwarring het bloed steeg}{hem in zijn gezicht}\\

\haiku{Tevreden sloot hij,.}{het mes en legde het op}{zijne papieren}\\

\haiku{{\textquoteright} {\textquoteleft}Maar zeg 't mij dan,,{\textquoteright}.}{meneer Hugo zei Mlle}{Malise verschrikt}\\

\haiku{{\textquoteleft}Ik vraag u dat, zei,.}{ze om me-zelf over u}{gerust te stellen}\\

\haiku{{\textquoteright} De profeet ging voort,.}{maar in eens voelde Hugo}{zijn hart stil schokken}\\

\haiku{De hertog ging den,,....}{portier voorbij kwam op het}{trottoir naderde}\\

\haiku{De joviale.}{rechter had moeite om het}{niet uit te proesten}\\

\haiku{hij keek op door zijn,:}{venster zijn blik stuitte af}{op de trali\"en}\\

\haiku{{\textquoteright} {\textquoteleft}Gij zult sterven, ja,;}{maar uw voorbeeld is  niet}{vergeefsch geweest}\\

\haiku{, wordt uw hart niet tot....}{bloedens toe getroffen bij}{de aanschouwing van}\\

\section{Gerard Walschap}

\subsection{Uit: Adela{\"\i}de}

\haiku{Wondere weemoed, ...}{die mijn leven over-lommert}{zoet zijt gij en wreed}\\

\haiku{Enfin het leek er.}{wel op of hij schande bracht}{over de familie}\\

\haiku{Kijk, de veefokker,,?}{Reynders indien hij haar zoo}{zag zou het kwaad zijn}\\

\haiku{Opeens stond hij recht,.}{tikte op het tafelblad}{en sprak tot allen}\\

\haiku{Zelfs begon hij zich.}{te verteederen in zijn}{zondarigen klerk}\\

\haiku{ik ben als ik zie.}{dat het u goed gaat en dat}{ge gelukkig zijt}\\

\haiku{Och, zei Ernest, men.}{is toch maar eerst getrouwd als}{er een  kind is}\\

\haiku{Een tweede angst nam,.}{bezit van haar de angst haar}{man te verwaarloozen}\\

\haiku{Dan vond zij dat hij.}{zeker tot bij het kindje}{had kunnen komen}\\

\haiku{Toen hadden  zij.}{hier de villa gekocht die}{zij nu bewoonden}\\

\haiku{Ik heb zoo'n hoofpijn,,,:}{zoo'n hoofdpijn klaagde zij zacht}{en dan altijd weer}\\

\haiku{Dan greep zij wild het.}{kind uit de wieg en zat het}{uren lang te bezien}\\

\haiku{het beest zelf halen,,,.}{want ik ben vandaag niet goed}{man overgeefachtig}\\

\haiku{Er woonde daar een,.}{arm duivelken op het dorp}{Daelemanneken}\\

\haiku{Ze zonden hem dan.}{maar naar een kapelleken}{even buiten het dorp}\\

\haiku{Ik weet heel juist wat.}{gebeuren moet en ik neem}{het allemaal aan}\\

\haiku{{\textquoteleft}En uw acht, dat zijn,{\textquoteright}.}{wilden zegt Adela{\"\i}de}{bits uit haren hoek}\\

\haiku{Juleken pas op,.}{hoor die pop teruggeven}{of anders naar bed}\\

\haiku{{\textquoteright} Vijf minuten lang.}{houdt hij het woord die kerel}{met temperament}\\

\haiku{Als hij veilig zit,.}{durft hij weer glimlachen maar}{alleen met zijn oogen}\\

\haiku{En hoe daarin twee.}{tranen groeiden omdat hij}{zoo lief was voor haar}\\

\haiku{Adela{\"\i}de had.}{er den heelen namiddag}{plezier mee gehad}\\

\haiku{Oscar pakte het.}{vlug op en Ernest wou het}{hem maar afnemen}\\

\haiku{{\textquoteright} Op een vooravond kwam.}{zij in den winkel met een}{pakje brochuren}\\

\haiku{Als Ernest weg was.}{stelde zij zich voor  hoe}{de ontrouw begon}\\

\haiku{Doe het nog eens, gij,,.}{twee smeerlappen zegt zij laat}{het mij eens goed zien}\\

\haiku{Nog even lichtte over.}{haar arme ziel de uitkomst}{van een goede biecht}\\

\haiku{En zij had nu met {\textquoteleft}{\textquoteright}.}{dat geval zooveel aan het}{huwelijk gedacht}\\

\haiku{Adela{\"\i}de sloeg.}{met haar volle vuist een ruit}{stuk en ging loopen}\\

\haiku{En nu weg, zoo ver,,,.}{als ge kunt altijd maar gaan}{gaan gaan en bidden}\\

\haiku{Dat kon zij en dat,?}{zou ze dachten ze misschien}{dat ze zoo gek was}\\

\subsection{Uit: Bejegening van Christus}

\haiku{Gerard Walschap}{Bejegening van Christus}{Colofon}\\

\haiku{Gaat hij den weg op,?}{van zijn vrouw dat hij wartaal}{begint te spreken}\\

\haiku{Maar het is Asveer.}{of de grond langzaam onder}{zijn voeten wegschuift}\\

\haiku{De stem van Zachaar,,:}{den reus sloeg op uit de groep}{en luidde oproer}\\

\haiku{{\textquoteleft}Grijpt mij dien muiter.}{en werpt hem in den kelder}{met water en brood}\\

\haiku{Ter wille van den.}{Messias lieten zij zich}{door hem opsluiten}\\

\haiku{Die hem kenden van.}{aangezicht hielden hem voor}{een menschenhater}\\

\haiku{met al de wijsheid,.}{van schrifturen stond hij bot}{den mond vol tanden}\\

\haiku{De hoogepriester.}{komt overeind en roept zelf met}{luider stem den naam}\\

\haiku{Gelooven is op Gods,.}{gezag aannemen wat men}{niet bewijzen kan}\\

\haiku{Nu mogen zij dan,.}{ook maar zeelieden zijn maar}{zooiets is aartsdom}\\

\haiku{Welken vijand nu,,?}{als ze eenmaal verdeeld zijn}{het eerst aanpakken}\\

\haiku{De Messias is,.}{onze redder hij heeft het}{recht en de waarheid}\\

\haiku{Hij weet niet, waar de,.}{Messias is noch wanneer}{hij zal verschijnen}\\

\haiku{Hij weet, dat hij niet,.}{sterven zal alvorens hem}{te hebben gezien}\\

\haiku{Het kereltje hoest,,.}{tusschen haakjes dat het niet}{aan te hooren is}\\

\haiku{Johannes, zeggen,.}{zij is door Salome de}{strot afgebeten}\\

\haiku{Gansch zijn jeugd heeft hij,.}{de Redders nageloopen}{de ontgoocheling}\\

\haiku{niemand kent deze,.}{krachten niemand kan hare}{grenzen afpalen}\\

\haiku{Laat het zelfs waar zijn,?}{moet hij dat zeggen aan het}{gewone  volk}\\

\haiku{Nu had de heer van,.}{den wijngaard nog eenen eenigen}{zoon dien hij liefhad}\\

\haiku{God verschillende?}{wijzen om hem te dienen}{welgevallig zijn}\\

\haiku{De Meester leerde.}{ons een Jehova die zijn}{schepselen liefheeft}\\

\haiku{Wij weten dat de.}{aarde eene schijf is en dat}{de zon er rond draait}\\

\subsection{Uit: Carla}

\haiku{Het paste toch dat.}{ze niet meer bij Caluwaers bleef}{maar moeder volgde}\\

\haiku{Die weggestuurd in:}{het pensionaat komen}{worden de beste}\\

\haiku{Niet  met Gustaaf,.}{want die zegt dat hij zich met}{dat meisje niet moeit}\\

\haiku{Men moet zijn vader,.}{eeren ook als hij misschien}{een klein gebrek heeft}\\

\haiku{Het verlangen naar.}{de wereld werd opgezweept}{door de eenzaamheid}\\

\haiku{Men zwijgt er over bij,.}{wereldsche menschen die het}{toch niet begrijpen}\\

\haiku{Dan gaan zij langs het.}{deftige witte huis van}{Dr. Yvo Verhaegen}\\

\haiku{Zij heeft er niet meer,.}{aan gedacht er met hem geen}{woord over gewisseld}\\

\haiku{Hij kan niet weten.}{wat tijdens de terugreis}{in den trein gebeurt}\\

\haiku{Ze hadden ieder.}{een haartje uitgetrokken}{om ze te meten}\\

\haiku{Leo moest eens nazien.}{of er iets bij was dat niet}{mocht verbrand worden}\\

\haiku{Hij loopt klein, vulgair,,.}{bekrompen bespottelijk}{onder haar oogen}\\

\haiku{Ook met deze vrouw.}{kan Carla nu niet over haar}{verwachting spreken}\\

\haiku{Leo, op uw moeder,.}{en op ons kindje ik ben}{goed voor haar geweest}\\

\haiku{Och God ja, nog iets,,.}{plechtigs enfin ge weet het}{wel h\'e Carlienke}\\

\haiku{zij valt hem huilend,,.}{om den hals Henri zij is}{zoo ongelukkig}\\

\haiku{Dan zit er volgens.}{Henri niets anders op dan}{eens te vechten}\\

\haiku{Carla alleen ziet.}{dat er iets zonderlings is}{in zijn vroolijkheid}\\

\haiku{Maar hij heeft er dan,:}{iets anders op gevonden}{dat van den polis}\\

\haiku{Maar als ik het schot.}{gehoord had werd ik razend}{en vergat alles}\\

\haiku{ik wist dat alles.}{maar een vergissing was en}{dat ge zoudt komen}\\

\haiku{Schreiend loopt zij naar,.}{boven knielt voor het kind en}{kermt vergiffenis}\\

\haiku{Zoodat Henri haar op.}{de stoep in het voorbijgaan}{kon gelukwenschen}\\

\haiku{Het verschil met den.}{rijkdom van Henri of Paul}{zou verminderd zijn}\\

\haiku{En als zij zag dat,}{hij bizonder slecht gestemd}{was verdubbelde}\\

\haiku{Maar onder den drang.}{van bronst nadert hij met of}{zonder verkoudheid}\\

\haiku{Zijn verkiezingsstrijd.}{begon hij op den dag van}{Dolf's begrafenis}\\

\haiku{Leo lette meer op.}{het bedrijf van Paul dan op}{zijn kandidatuur}\\

\haiku{Die triomfeert bij,.}{de verkiezingen eerste}{vergissing van Leo}\\

\haiku{Het geloof is eene.}{genade en men verliest}{het door de zonde}\\

\haiku{Het moest zooeens waar zijn,,.}{denken ze en dan hebben}{wij ze beet mijn kind}\\

\haiku{J'ai connu quelqu'un, ...{\textquoteright}.}{un ing\'enieur Carla}{voelt zich machteloos}\\

\haiku{Hij houdt haar zijnen,.}{glimlach voor een schild dat zij}{niet kan doorsteken}\\

\haiku{Zij zit dagen lang}{opgescheept met dat Mieke}{Demey en Mieke}\\

\haiku{Die zit hij zwijgend}{leeg te drinken en als hij}{goed dronken is houdt}\\

\haiku{Veronderstel dat.}{de commerce van Herman}{Stevens van mij is}\\

\haiku{Maar deze was met,,.}{een boerken getrouwd en had}{meende ze welstand}\\

\haiku{En toen het groot en,.}{bronstig geworden was had}{Leo haar dat gezegd}\\

\haiku{Willen of niet, zij.}{moet met hem alleen blijven}{bij de twee dieren}\\

\haiku{Hij durft er zelf niet,.}{naar vragen maar de makkers}{roepen het voor hem}\\

\haiku{{\textquoteleft}Heeft ze u gestuurd,,,.}{om mij te verdrinken hijgt}{hij goed gij of ik}\\

\haiku{Voelt ge nu nog niet?}{dat ik u kan kraken en}{in twee\"en breken}\\

\haiku{in elk geval zal.}{voor haar het einde eerder}{komen dan voor hem}\\

\haiku{Ze heeft veel te groote,.}{zwarte oogen en bohemersch}{zwart haar wij zijn blond}\\

\haiku{{\textquoteleft}Papa, ik heb het,,.}{niet gedaan Mamake ik}{heb het niet gedaan}\\

\haiku{Niets kan zoo wreed zijn,.}{als jong geluk het gekscheert}{onder doodsklokken}\\

\haiku{Daaraan ziet Carla.}{dan dat Mieke waarlijk van}{God gezonden is}\\

\haiku{Hij geeft de doode,.}{een kruisteeken het zieke kind}{wat fruit en speelgoed}\\

\subsection{Uit: Eric}

\haiku{Nonkel Oscar nam.}{zijn handje en mama zou}{wel lang wegblijven}\\

\haiku{Ten slotte dierf zij,.}{niet meer in het huis komen}{tenzij met Ernest}\\

\haiku{Na elken zin wacht:}{zij op antwoord van al de}{kinderen in koor}\\

\haiku{'s Anderdaags ging.}{hij met Cyriel van dokter}{Tierens naar de school}\\

\haiku{Bezeten gaat hij.}{te keer met zijn vingeren}{vol razende kramp}\\

\haiku{Eerst dat arm vrouwke,.}{en nu hijzelf want daar moet}{hij van ten onder}\\

\haiku{Daarom ging Lizy:}{op een stoel zitten weenen}{en Eric troostte haar}\\

\haiku{{\textquoteleft}Ik zal u ook eens,, '.}{onderzoeken madam leg}{u maar int bed}\\

\haiku{Oscar kon dat niet,.}{aanhooren maar bonpapa}{luisterde stralend}\\

\haiku{Om zijn goed humeur.}{luid te luchten begon hij}{soms op te spelen}\\

\haiku{Ik heb u goeden.}{dag gezegd en verzoek u}{mij ook te groeten}\\

\haiku{Wij trekken samen,,.}{op Ernest en ik dacht hij}{en zijn rug boog door}\\

\haiku{Telkens hij opkeek.}{zag hij de oogen uit het bed}{scherp op hem gericht}\\

\haiku{Hoe rap kijkt hij weg,,.}{dacht  Ernest vroeger keek}{hij mij zoo recht aan}\\

\haiku{De directeur moest:}{de deur van de spreekkamer}{met kracht openduwen}\\

\haiku{{\textquoteleft}ja, ik heb het er.}{bij bonpapa eindelijk}{toch doorgekregen}\\

\haiku{Het ging nu zonder.}{speech \`a la Verhaeghen en}{zonder piano}\\

\haiku{Den boer bespreekt het.}{opstel dat hij een beetje}{te romantisch vindt}\\

\haiku{Ze was zeker ziek,,.}{vragen ze of ze zal iets}{gekregen hebben}\\

\haiku{want hij troost het kind,.}{met een doode moeder en}{een blinden vader}\\

\haiku{Dus tot zijn groot spijt.}{enz. Maar Eric doorstaat al de}{pijnen der hel}\\

\haiku{Ja, dat is papa,.}{met een verband om het hoofd}{juist gelijk mama}\\

\haiku{het hoorde, want voor.}{hen en de professoren}{was Eric een voorbeeld}\\

\haiku{Ze trokken tegen.}{als hij de nestels aantrok}{en een nestel brak}\\

\haiku{Daarmee waren ze,.}{voor een dag gered hij deed}{een ander paar aan}\\

\haiku{Zoo was ze dan hier.}{gebleven en hoe waren}{ze nu niet gestraft}\\

\haiku{In mijn dorp staat een.}{kasteel waarvan ze zeggen}{dat het zwart goed is}\\

\haiku{Maar op Dries kunt ge.}{een stad bouwen en van Eric}{zijt ge nooit zeker}\\

\haiku{Iedereen kan goed,.}{met hem om maar niemand die}{hem eigenlijk kent}\\

\haiku{Eric zat den ganschen.}{avond roerloos in zijn zetel}{voor zich uit te zien}\\

\haiku{Terwijl hij haar na,.}{de vergadering zocht stond}{zij opeens voor hem}\\

\haiku{En dan in wachtzaal.}{tweede klas voor den trein van}{8 uren en zooveel}\\

\haiku{Er zijn er die er.}{te veel hebben en dat loopt}{nogal dikwijls mis}\\

\haiku{Maar er gaat geen dag,.}{voorbij zonder kus geen week}{zonder den doodelijken}\\

\haiku{Allengerhand wordt.}{hij kalmer en zij dankt het}{aan haren invloed}\\

\haiku{hij bewaarde zijn,:}{geheim maar Cyriel sprak het}{met leedvermaak uit}\\

\haiku{{\textquoteright} Eric stond stil, zag hem.}{sprakeloos met wilde oogen}{aan en ging verder}\\

\haiku{Als hij weg is denkt.}{Eric dat hij absoluut een}{revolver noodig heeft}\\

\haiku{Ze vraagt verschrikt of,,.}{hij dan al iets weet maar neen}{wat zou hij weten}\\

\haiku{ge houdt van Eric en,,?}{hij van u. Zeg hij is toch}{niet geladen he}\\

\haiku{zij zoo hatelijk,.}{en ze kan mis zijn maar ze}{voelde de afgunst}\\

\haiku{En ondertusschen.}{probeert Lizy toch maar bij}{Eric te geraken}\\

\haiku{Hij wil gedoome.}{wel eens zien of hij van hem}{geene goeie zal maken}\\

\haiku{Geef het liever geen,, '.}{patatten zeit hem alst}{maar spinazie eet}\\

\haiku{{\textquoteleft}Wacht,{\textquoteright} roept Eric, en vliegt,.}{de trappen af zoo vlug als}{toen grootvader stierf}\\

\haiku{Hij neemt Eric bepaald, {\textquoteleft}{\textquoteright}.}{onder handen die moet zich}{nu eens gaanzetten}\\

\haiku{Hij herinnert zich.}{hoe Oscar te Leuven over}{een revolver sprak}\\

\haiku{Maar zelf schrijft hij naar.}{Zuster Ismelda dat het}{haar zaken niet zijn}\\

\haiku{Waarop zouden ze,.}{wachten voor hem is het geen}{leven zonder thuis}\\

\subsection{Uit: Sibylle}

\haiku{Omdat hij alleen.}{was en het practisch werk nauw}{toezicht eischte}\\

\haiku{Daardoor merkten zij.}{de afwezigheid van de}{nieuwe moeder op}\\

\haiku{Maar niets was hem zoo.}{belangrijk als de uitslag}{van Sibylle}\\

\haiku{Het dorp  was er,.}{weldra over uitgepraat maar}{Michel bleef wrokken}\\

\haiku{Kwikstaartje sprong hem.}{om den hals en zeide dat}{zij verloofd waren}\\

\haiku{Hij wordt Goddank niet.}{begraven en al wat leeft}{schrijft en zendt kiekjes}\\

\haiku{{\textquoteleft}En kardinaal Van,,?}{Rossum Celest hebt ge dien}{ook wel eens gezien}\\

\haiku{{\textquoteleft}Schrijf en publiceer,,.}{het zei die ze moesten het maar}{niet gedaan hebben}\\

\haiku{{\textquoteright} - {\textquoteleft}Dat ik denk dat ze}{niet ver mis zullen zijn als}{ze ongeveer doen}\\

\haiku{{\textquoteright} - {\textquoteleft}Durft ge al wat ge,?}{mij zegt en al gezegd hebt}{overal herhalen}\\

\haiku{Twee wegen zwenken.}{in sierlijke bocht rechts en}{links van het gebouw}\\

\haiku{Geen schandaal, dat bracht.}{haar werk in gevaar en zij}{had het Cest beloofd}\\

\haiku{Getroffen drukte, '}{hij haar de hand als op een}{zwijgend verbond maar}\\

\haiku{Hij zou er morgen,,.}{maar mee  doorgaan in de}{pastorij overal}\\

\haiku{Die hoe heet ze, die,,.}{half-rosse enfin die}{vriendin is ook weg}\\

\haiku{we moeten een reis,.}{doen maar waren niet in staat}{te denken waarheen}\\

\haiku{{\textquoteright} Sibylle gaf.}{hun juist een uur tijd om het}{huis te verlaten}\\

\subsection{Uit: Tor}

\haiku{Hij zegt dat als Nel,.}{het niet doet Tor Nel nog eens}{zal kapot maken}\\

\haiku{Toen hij met zijn Ad\`ele,.}{Tas trouwde was hij al rijk}{maar zij nog rijker}\\

\haiku{Dan krijgt hun gebed,.}{een mystieke tragische}{opgetogenheid}\\

\haiku{Hij heeft Mieke niet:}{lang moeten smeeken om het}{te mogen worden}\\

\haiku{Heeft een veldwachter?}{meer toekomst dan iemand die}{hun familie dient}\\

\haiku{Mieke denkt niet aan.}{ja omdat Tor niet te best}{met hem overeenkomt}\\

\haiku{Zij beweert in het,.}{donker bang te zijn hij moet}{haar naar huis brengen}\\

\haiku{Waartoe het dient zien,.}{wij voor ons oogen het maakt Vera}{nog zotter van Tor}\\

\haiku{Tor is zeker maar,,?}{een veldwachter maar ik was}{ik geen metsersknaap}\\

\haiku{Als zij niets van hem,.}{hooren gaan Tor en Vera met}{den auto eens zien}\\

\haiku{Reine Priestman heeft,.}{nogal stem zij wil beter}{leeren zingen bij Vera}\\

\haiku{Ge zult dat inzien.}{als ge nog wat meer uwen smaak}{zult gevormd hebben}\\

\haiku{Celis meestert bij,,.}{de Dherts  maar Colfs is geen}{Dhert hij vraagt Priestman}\\

\haiku{De doktoor spreekt er.}{over gelijk wij over het weer}{en de politiek}\\

\haiku{Zijn zinnen staan er.}{niet meer naar om een beroemd}{dichter te worden}\\

\haiku{Anders nog een jaar.}{en hij is Napoleon}{en God de Vader}\\

\haiku{Hij kon er waarlijk,.}{niet meer mee spreken ze niet}{meer in de oogen zien}\\

\haiku{En toch miszie ik}{iets aan u. Bij alles bracht}{ze onzen lieven}\\

\haiku{En zie, Tor heeft de,?}{oogen toegedrukt maar staan ze}{nu weer niet half open}\\

\haiku{Ze durven zonder,.}{hem niet boven gaan maar hij}{zelf durft ook niet meer}\\

\haiku{Die bibliotheek.}{van Tor de commissaris}{was buitengewoon}\\

\haiku{Hij kan dat niet meer,.}{halen hij doet afstand ten}{voordeele van Priestman}\\

\haiku{En wat Fran\c{c}ois ook,.}{heeft Fran\c{c}ois is goed voor zijn}{ondergeschikten}\\

\haiku{Hij meet thuis haren,,.}{bloeddruk veel te weinig en}{maakt haar een  flesch}\\

\haiku{Hij schrijft er schoon uwen,.}{naam in en hulde van den}{schrijver Muys Victor}\\

\haiku{Zijn moeder heeft hem,.}{doorzien Mieke niet en hij}{heeft ze bedrogen}\\

\haiku{Eens wijst hij haar het,.}{fleschken en vraagt of er iets}{van komt ja of neen}\\

\haiku{De daders moeten,.}{aangehouden worden de}{gekwetsten verzorgd}\\

\haiku{Zoo genomen zijn,.}{zenuwen zijn aangetast}{ik geef dat nog toe}\\

\haiku{Ze zeggen vallen,.}{en opstaan maar bij hem is}{het maar stronkelen}\\

\haiku{Hij gaat er met inkt.}{over opdat ze goed zien dat}{het geenen doubl\'e is}\\

\haiku{Het moest plaats hebben.}{op ne Zondag omdat het}{meisje dan uitging}\\

\haiku{En voorts, allee, hoe,.}{gaat het op den buiten de}{menschen babbelen}\\

\section{Andr\'e Weber}

\subsection{Uit: E\'en jaar maximumstraf}

\haiku{Hij is niet ouder,.}{dan dertig jaar en is heel}{slank bijna mager}\\

\haiku{Terwijl hij langzaam,.}{het gerechtsgebouw verlaat}{neemt hij een besluit}\\

\haiku{We benne kletsnat.}{en willen vannacht graag een}{beetje droog pitten}\\

\haiku{{\textquoteright} {\textquoteleft}Dat stelletje ook,{\textquoteright}, {\textquoteleft}.}{niet zegt Krook beslisten wij}{waren hier het eerst}\\

\haiku{Kort en goed, ik doe,.}{mijn best hem het leven tot}{een hel te maken}\\

\haiku{{\textquoteleft}Ik heb er maar drie,{\textquoteright}.}{neer kunnen schieten antwoordt}{Hart onverschillig}\\

\haiku{Mijn vrouw heeft hem pas.}{een paar maal gezien en kent}{hem dus niet zoo goed}\\

\haiku{{\textquoteleft}'n Oogenblik{\textquoteright}, zegt,.}{Miep verlaat de kamer en}{doet de voordeur open}\\

\haiku{Mocht U eens hulp of,.}{goede raad noodig hebben komt}{U dan naar mij toe}\\

\haiku{Maar ze hebben vier.}{kerels in het ziekenhuis}{moeten verbinden}\\

\haiku{{\textquoteright} Verwonderd neemt de:}{inspecteur het briefje in}{ontvangst en leest}\\

\haiku{We hebben in elk.}{geval een aanwijzing en}{ik ga er op af}\\

\haiku{{\textquoteright} Inmiddels heeft de.}{auto Zutphen gepasseerd}{en nadert Gorssel}\\

\haiku{Aan het einde der,.}{gang komen ze bij een deur}{die op een kier staat}\\

\haiku{En als ik tegen,,.}{U zeg dat U volmacht heeft}{dan heeft U volmacht}\\

\haiku{Hij vroeg mij, hem te,.}{vergeven dat hij mijn oom}{had doodgereden}\\

\haiku{Maar het is hem  ,.}{nog steeds een raadsel waarheen}{het hem zal leiden}\\

\haiku{De man schijnt het land,{\textquoteright}.}{in te hebben voegt hij er}{grinnikend aan toe}\\

\haiku{Rechtuit de trap op{\textquoteright} {\textquoteleft}.}{en dan de eerste deur aan}{Uw linker hand.Goed}\\

\haiku{Hart ziet het gevaar:}{nog net bijtijds aankomen}{en zegt plotseling}\\

\haiku{{\textquoteleft}U komt me zeker,?}{vertellen dat er bij U}{is ingebroken}\\

\haiku{Hij rekent af, neemt.}{zijn tasch op en begeeft zich}{hoopvol op weg}\\

\haiku{{\textquoteleft}Ik ben agent Mulder.}{en kom met een boodschap van}{hoofdinspecteur Hart}\\

\haiku{Op een gegeven {\textquotedblleft}{\textquotedblright}.}{oogenblik noemde ze zich}{een weerloos meisje}\\

\haiku{En het begint er,.}{op te lijken dat zijn hop}{in vervulling gaat}\\

\haiku{{\textquoteright} Vluchtig bekijkt hij,.}{den inhoud maar ook hier vindt}{hij niets bijzonders}\\

\haiku{De kop thee, die voor,.}{haar op tafel staat is al}{lang koud geworden}\\

\haiku{{\textquoteleft}Mocht U eens hulp of,.}{goede raad noodig hebben komt}{U dan naar mij toe}\\

\haiku{{\textquoteright} Miep vertelt hem nu,.}{uitvoerig hoe de zaak zich}{heeft toegedragen}\\

\haiku{Ik had geen voorschot,.}{noodig maar hij wilde het toch}{zonder meer geven}\\

\haiku{Ik heb een paar ton.}{te beleggen en moet eens}{kalm met U praten}\\

\haiku{{\textquoteleft}Het is pas kwart voor,.}{zeven dus zijn we in elk}{geval vroeg genoeg}\\

\haiku{Ik was baron van,.}{Teuningen van de rijke}{tak moet je weten}\\

\haiku{De inbraak, die ze,.}{van plan zijn zal heel ergens}{anders plaats hebben}\\

\haiku{{\textquoteleft}Het licht schijnt precies.}{een halve minuut te vroeg}{te zijn opgegaan}\\

\haiku{{\textquoteright} {\textquoteleft}Heel graag, tenminste,.}{als je nog van die oude}{cognac in huis hebt}\\

\haiku{Mijnheer Martens, het,.}{lijkt me nu wel verantwoord}{U vrij te laten}\\

\haiku{Of denk je, dat ie,?}{de brutaliteit heeft naar}{zijn woning te gaan}\\

\haiku{Hij heeft een voorsprong,.}{van bijna drie uur en dat}{is een heeleboel}\\

\haiku{binnen en Mander,.}{drukt hun op het hart uiterst}{voorzichtig te zijn}\\

\haiku{{\textquoteleft}Maar wees een beetje,.}{voorzichtig en schiet niet in}{mijn rug scherpschutter}\\

\haiku{Eindelijk heeft hij.}{het knopje gevonden en}{het licht aangedraaid}\\

\haiku{Als je 't toch doet.}{zal ik je met mij eigen}{handen vermoorden}\\

\haiku{Ik heb het vrouwtje.}{opgezocht en haar een paar}{theedoeken verkocht}\\

\haiku{Over een half uur  .}{is het licht en dan moeten}{we verdwenen zijn}\\

\haiku{Er zal niets anders,.}{opzitten dan het morgen}{opnieuw te probeeren}\\

\haiku{Ten minste, als je '.}{s middags om een uur of}{twaalf al wakker bent}\\

\haiku{Ik heb het toen voor.}{hem gemaakt en hij heeft me}{vorstelijk betaald}\\

\haiku{{\textquotedblleft}Maar U wist toch pas,?}{een paar dagen dat U die}{prijs had getrokken}\\

\section{Constant van Wessem}

\subsection{Uit: Celly. Lessen in charleston}

\haiku{In dit eene opzicht.}{heeft het verhaal zelfs een vrij}{ouderwetsch verloop}\\

\haiku{Maar waarom zegt zij,?}{niets staart zij het plaatje aan}{als ziet zij het niet}\\

\haiku{Je kunt je gerust,.}{komen overtuigen dat ik}{het w\`el gedaan heb}\\

\haiku{Soms kon zij het fel.}{voelen als zij het portret}{van Moeder bezag}\\

\haiku{Celly ziet neer op.}{haar eigen voetpunten en}{die van Miklos}\\

\haiku{dit is dansen, en.}{zij wordt er zelf volslagen}{moedeloos onder}\\

\haiku{Daar lag hij nu en.}{probeerde zijn gedachten}{te verzamelen}\\

\haiku{- Toen salueerde.}{hij aan den rand van zijn hoed}{en wendde zich om}\\

\haiku{klokslag vier uur stond {\textquoteleft}{\textquoteright}.}{Victoria met haar auto}{voorMaison Ilse}\\

\haiku{Jeroen wist best, dat.}{hij zijn mythe van besten}{danser verspeelde}\\

\haiku{Het werd nu een sprong,.}{pardoes in een witgloeiend}{sprookjesachtig licht}\\

\haiku{Haar eene hand wreef zij.}{over de andere alsof}{zij ze afdroogde}\\

\haiku{Boven het water}{scheen de gloeiende hitte}{hooger te hangen}\\

\haiku{Maar reeds was Dandy.}{lachend en jongensachtig}{op haar toegesneld}\\

\haiku{Celly glimlacht met,.}{een zacht rood dat over haar oogen}{en haar voorhoofd trekt}\\

\haiku{trouw zit hij bij haar.}{op haar kamer als zij van}{kantoor terug komt}\\

\haiku{{\textquoteleft}Lieveling{\textquoteright} hebben.}{Dandy's lippen zich reeds om}{haar mond gesloten}\\

\haiku{Dandy voelt hoe haar.}{lichaam terug wijkt en toch}{van verlangen trilt}\\

\haiku{Maar de volgende.}{maal is alles anders en}{de vrouw niet meer zwak}\\

\haiku{Dandy - de overgang -:}{was hem zelf een raadsel moest}{toen opeens denken}\\

\haiku{{\textquoteleft}Veronderstel eens,,.}{dat je een man ontmoette}{die je vertrouwdet}\\

\haiku{Maar hij voelt zich zelf,.}{zoo landerig dat hij een}{nieuwe flesch bestelt}\\

\haiku{Nauwelijks luistert.}{hij naar het doezelige}{praten van Harry}\\

\subsection{Uit: De Clowns en de fantasten (onder ps. Frederik Chasalle)}

\haiku{De necromant De}{necromant Over zijn glazen}{stolpen en bollen}\\

\subsection{Uit: Galop chromatique}

\haiku{Een smeulend vuur, dat.}{slechts op den windstoot wachtte}{om te ontvlammen}\\

\haiku{Een attractie van.}{den zwarten salon was het}{somnambulisme}\\

\haiku{enkelen bleven,...}{te lang op den grond op zoek}{naar elkaars monden}\\

\haiku{Dat was nu eenmaal {\textquoteleft}{\textquoteright}.}{de aardige gewoonte}{dersleutelromans}\\

\haiku{Het leven was niet.}{meer interessant zonder}{een Grote Liefde}\\

\haiku{En toch zou zij na:}{de eerste ontmoeting al}{zichzelf bekennen}\\

\haiku{twijfel in zich wil,:}{doden zelfs zijn jeugdliefde}{verloochenend}\\

\haiku{Dat zal je meteen.}{onsterfelijk maken als}{je komt te sneven}\\

\haiku{Zelfs Schlesinger moet.}{glimlachen bij Liszt's}{bravourtirade}\\

\haiku{de kogels hebben.}{het luchtruim of de takken}{der bomen doorboord}\\

\haiku{Het is Liszt, die,.}{Parijs is ontvlucht op den}{loop voor zijn hartstocht}\\

\haiku{Het is enkel zaak...}{dezen waan tot aan het graf}{te laten duren}\\

\haiku{Stil en voornaam als,.}{hij zelf is tovert hij in}{zijn spel het zonlicht}\\

\haiku{Het knutselen was.}{altijd een liefhebberij}{van haar gebleven}\\

\haiku{De uitstapjes en.}{de gesprekken golden voor}{de ontspanningsuren}\\

\haiku{{\textquoteright} Ja, Liszt wist het,,.}{nog als zoveel dat hij thans}{vergeten wenste}\\

\haiku{Nu slaat de ander.}{de ogen op en hun blikken}{ontmoeten elkaar}\\

\haiku{{\textquoteleft}Het is au fond maar,.}{een eenvoudige waarheid}{die ik je vertel}\\

\subsection{Uit: Gustaaf}

\haiku{Het zwijgen kent men.}{van hem als de uiting van}{een groote opwinding}\\

\haiku{De aderen op het.}{voorhoofd van den jongen zijn}{gezwollen van drift}\\

\haiku{Maar er is nog de,.}{glanzende verstilling die}{begeerteloos maakt}\\

\haiku{Het is t\`e banaal.}{om met je muziekmeester}{te gaan flirten}\\

\haiku{Het hout is gloeiend,.}{van de brandende zon maar}{het rusten doet goed}\\

\haiku{De menschen zullen}{ook niet begrijpen waarom}{het laatste deel zoo}\\

\haiku{- O, hoe gaarne zou;}{ik nog tot den lichten dans}{der sferen opgaan}\\

\haiku{En toch blijft er een,:}{heimelijke stem in hem}{wakker waarschuwend}\\

\haiku{Als Gustaaf zich van,.}{dit ziekbed verheft voelt hij}{zich lichter jonger}\\

\haiku{Den bliksem hebben,.}{zij als zweep den donder als}{wagenraderen}\\

\subsection{Uit: De ijzeren maarschalk. Het leven van Daendels, 'soldat de fortune'}

\haiku{Krayenhoff loopt met.}{bedenkelijk gebogen}{hoofd te luisteren}\\

\haiku{Steun van buiten werd.}{alleen gezocht met het oog}{op steun naar binnen}\\

\haiku{Als op den dreun van}{een kinderversje leeren de}{Hattemsche burgers}\\

\haiku{Hoogst beleedigd beklaagt.}{zij zich bij haar broeder den}{koning van Pruisen}\\

\haiku{Daendels, met fakkels,.}{bijgelicht ziet op een kaart}{zijn positie na}\\

\haiku{De {\textquoteleft}burgers{\textquoteright} kunnen.}{het gewicht van hun nieuwe}{positie niet aan}\\

\haiku{In de hoek van het;}{rood-wit-en-blauw staat de}{leeuw in het tuintje}\\

\haiku{Best bier hebben de....}{Bataven en vooral hun}{jenever is goed}\\

\haiku{Eenigen tijd later.}{zit er een nieuwe Fransche}{gezant in den Haag}\\

\haiku{{\textquoteright} Van Langen houdt de.}{handen met een wanhopig}{gebaar wijd uiteen}\\

\haiku{Buiten staan zijn drie.}{compagnie\"en grenadiers}{aangetreden}\\

\haiku{Maar op de smalle.}{zandstrook kunnen zij slechts een}{voor een uit komen}\\

\haiku{Patriot of geen, {\textquoteleft}{\textquoteright}.}{patriot op deflinke}{kerels komt het aan}\\

\haiku{De koning zelf is,;}{een melancholiek uitziend}{wat gebrekkig man}\\

\haiku{Storm over Indi\"e I.}{Daendels komt op Nieuwjaarsdag}{1808 in Indi\"e aan}\\

\haiku{Zoo bluft de rijkdom.}{van de eene tegen die van}{de andere op}\\

\haiku{wat zich overdag voor.}{de warmte schuilhoudt wordt het}{een vroolijk vertier}\\

\haiku{Als de stemming er,:}{is komen ook als vanzelf}{de vechtpartijen}\\

\haiku{Buitenzorg is zijn!}{eigendom en hij mag er}{mee doen wat hij wil}\\

\haiku{De 29sten Juni gaat {\textquoteleft}{\textquoteright}.}{hij te Soerabaya scheep op}{de korvetSapho}\\

\haiku{Maar helaas ik had,.}{slechts dappere soldaten}{maar geen zeelieden}\\

\haiku{Bij een onderhoud}{met Decr\`es heeft hij dezen}{onder zijn dictee}\\

\haiku{dat zijn vermogen.}{300 mille bedraagt kan hij}{dat niet veel vinden}\\

\haiku{Hij ziet niet graag zijn.}{collega's met de eer van}{den veldtocht strijken}\\

\haiku{Om het behoud van,.}{het leger om het behoud}{van Napoleon}\\

\haiku{een uitval, met het,.}{doel vee te rooven uit het kamp}{der Russen mislukt}\\

\haiku{Zelfs Kossecki geeft.}{de onhoudbare toestand}{van de vesting toe}\\

\haiku{Het ongeluk van.}{ons Vaderland vereenigt}{alle Hollanders}\\

\haiku{Wat zal Daendels doen,?}{dien men in zijn land niet heeft}{willen gebruiken}\\

\haiku{Als Daendels nog eens?}{bij zijn vroegeren Keizer}{emplooi ging vragen}\\

\subsection{Uit: Margreet vervult de wet}

\haiku{Heb je toen niet bij,?}{jezelve besloten dat}{het uit moest wezen}\\

\haiku{Zij bleef naast hem gaan,,:}{zwijgend nu hij voelde het}{meer dan hij het zag}\\

\haiku{{\textquoteleft}En de volgende,.}{week ga ik naar X ik kom}{daar op een kantoor}\\

\haiku{{\textquoteright} vroeg zij nogmaals aan.}{zichzelf en in haar toon klonk}{een lichte zelfspot}\\

\haiku{Alles moest klaar voor.}{haar wezen en ook dit woord}{zei haar te weinig}\\

\haiku{in hoeverre kon?}{hier van opzettelijke}{moord sprake wezen}\\

\haiku{Maar laat ik straks over,.}{Ferdinand vertellen als}{ik aan hem toe ben}\\

\haiku{het besef, dat God}{overal rondom je is en}{met Argus-oogen}\\

\haiku{Ik ging immers niet,,.}{alleen ik ging met Greet en}{Her die ik kende}\\

\haiku{Om die twijfel van,.}{mij weg te doen nam ik zijn}{hand in de mijne}\\

\haiku{Wat beteekende het?}{nog voor mij als Ferdinand}{mij er om verliet}\\

\haiku{Het tumult van haar.}{gedachten en gevoelens}{werd steeds heftiger}\\

\haiku{Weinig, bijna niets,,.}{een drama dat zij alleen}{maar kon vermoeden}\\

\haiku{Maar in dat mooie boek,:}{van Faust staat ook dat er een}{stem van omhoog riep}\\

\haiku{Haar oogen bevreemdden,}{haar het was of hun effen}{grijze blik slechts v\'o\'or}\\

\haiku{Niet met het gevoel,,.}{met het gevoel alleen wil}{ze de dingen doen}\\

\haiku{En ook hij zal voor,!}{An Winters moeten pleiten}{of hij wil of niet}\\

\haiku{Ik heb het immers,?}{niet gewild dat weet zij toch}{net zoo goed als ik}\\

\haiku{Hij zou beginnen.}{met te wijzen op deze}{zedelooze tijden}\\

\haiku{Zij weet, dat zij in,}{een zaak optreedt waarin zij}{uiterlijk bezien}\\

\haiku{Achter blijven, in,.}{de stoflucht der verlaten}{kamers de acten}\\

\haiku{Het moest mogelijk,,.}{zijn haar hart zei het haar dat}{het mogelijk was}\\

\haiku{Het is haar haast een,,.}{vreugde een bevrijding het}{z\'o\'o voor zich te zien}\\

\haiku{{\textquoteleft}Wie van u zonder,.}{zonden is die werpe de}{eerste steen op haar}\\

\haiku{Het had hem geheel,.}{vervuld als iets plezierigs}{dat hij voor zich zag}\\

\haiku{{\textquoteright} riep Moeder, {\textquoteleft}Agnes,,.}{laat de honden uit ga een}{grachtje met ze om}\\

\haiku{Verdachte heeft op.}{achtjarige leeftijd haar}{ouders verloren}\\

\haiku{het besef, dat God}{overal rondom je is en}{met Argus-oogen}\\

\haiku{Ik ging immers niet,,.}{alleen ik ging met Greet en}{Her die ik kende}\\

\haiku{Wat haar tot haar daad.}{moest brengen was op zichzelf}{al niet meer normaal}\\

\haiku{{\textquotedblleft}Wie van u zonder,{\textquotedblright}.}{zonde is die werpe de}{eerste steen op haar}\\

\haiku{{\textquoteright} ~ Na een kwartier.}{keerden de rechters uit de}{raadkamer terug}\\

\haiku{Ongetwijfeld, wat.}{met menschenkrachten mogelijk}{was had zij beproefd}\\

\haiku{In mijn hart heb ik,}{ook nooit durven aannemen}{dat je bereiken}\\

\subsection{Uit: De vuistslag}

\haiku{Zoo'n lichaam valt te:}{zwaar neer en kan slechts roerloos}{blijven na de smak}\\

\haiku{Mijn houding zal ik,.}{zelf wel bepalen daarvoor}{heb ik haar niet noodig}\\

\haiku{Het sloeg haar op haar,.}{zenuwen dit leven met}{John in \'e\'en huis}\\

\haiku{Een gezin had hen{\textquoteright},.}{beiden beter gebonden}{besliste Anka}\\

\haiku{Heel waarschijnlijk is}{die vrouw even laf als jij en}{was haar eenige zorg}\\

\haiku{Neen, ik kan me niet,.}{voorstellen dat ik me zou}{hebben laten slaan}\\

\haiku{Een romantische,;}{overbodigheid meende zij}{van John's daad}\\

\haiku{De tijd, dat zij zich - -:}{afvroeg en zij glimlachte}{als was het komiek}\\

\haiku{Het is reeds lang drie,.}{uur voorbij zoo aanstonds zal}{het vier uur wezen}\\

\haiku{weet hij toch wel, dat.}{hij zich driftig maakt om een}{reclamewijzer}\\

\haiku{Het lichaam is wat,.}{uitgerust de zenuwen}{zijn wat ontspannen}\\

\haiku{Als het noodlot heeft.}{door John's leven een}{paard gegaloppeerd}\\

\haiku{Beiden omhult een,.}{stofwolk die als dunne rook}{optrekt over de weg}\\

\haiku{Ik ben haar niet goed,,.}{genoeg geweest een ander}{was haar beter goed}\\

\haiku{Hij heft de oogen op,.}{de eene wenkbrauw iets hooger}{dan de andere}\\

\haiku{En vlak bij, als een,:}{ontploffing in zijn oor hoort}{hij Paula gillen}\\

\haiku{Ten slotte is het.}{toch niet veel anders dan een}{ceremonieel}\\

\section{Jacob Campo Weyerman}

\subsection{Uit: Den Laplandschen tovertrommel}

\haiku{Ook al valt de naam,:}{Campo enige malen de}{schrijver blijft naamloos}\\

\haiku{wijsheid is zowel.}{de basis als de bron van}{een goede pen11}\\

\haiku{Vergader akers, pluk, ';}{kornoeljes die int Sticht}{Zo heerlyk gloeien}\\

\haiku{dat torst een kroon, Ghy,}{die den straffen raad verzelt}{die harde bollen}\\

\haiku{3Meer informatie:}{over dit werk is te vinden}{in T. van der Meer}\\

\haiku{19Zie nummer 103.}{in de bibliografie}{van Marleen de Vries}\\

\section{Erich Wichman}

\subsection{Uit: Het witte gevaar}

\haiku{oude is goed voor -}{de koude en jonge is}{goed voor de longen}\\

\haiku{Bieren zijn goed voor,,}{de nieren Jenever is}{goed voor de lever}\\

\haiku{Een Koninkrijk, neen,,!}{liever een Republiek voor}{een grooten fopspeen}\\

\haiku{4Het {\textquoteleft}melkkapitaal{\textquoteright}.}{heeft zich ook reeds van deze}{leus meester gemaakt}\\

\section{Willem Jacob Domis, Simon Eikelenberg, Jan Jacobsz Stoop en Jacob Dircsz Wijnkoper}

\subsection{Uit: Kroniek van Wijnkoper}

\haiku{en aan 't eynde,:}{derzelven een verklaring}{in dezer voegen}\\

\haiku{so vroeg hij om de,}{slagorde van den vijand}{en daar op gebood}\\

\haiku{Leeuwaarden], geboren}{van Alcmaer worde doot}{geslagen vanden}\\

\haiku{Coning Philip is.}{geboren anno 1527 in}{maij den 12 dag}\\

\haiku{] Op dit jaer is die '.}{verstoelinge geschiet op}{t lant van Overdie}\\

\haiku{] Den Oosterdijk met.}{4 watermolens binnen}{Geestmerambacht gemaakt}\\

\haiku{] Anno 1562 en 1563 '.}{ist concilium van}{Trent gehouden}\\

\haiku{Anno ut supra.}{begost men de toorn van de}{waag af te breken}\\

\haiku{Daer lach een groote vier,}{aen daer hebben sij haer bij}{gewarmt seggende}\\

\section{Eddy Wijnkoop}

\subsection{Uit: Tim Taccle}

\haiku{Vertelt U eens, wat,.}{U bekend is als U zoo}{vriendelijk zijn wilt}\\

\haiku{Zij had niet op de.}{tijd gelet en was eerst laat}{naar huis gekomen}\\

\haiku{{\textquoteright} De detective.}{bedwong zijn verwondering}{over deze woorden}\\

\haiku{Bijzonderheden.}{omtrent zijn signalement}{kon hij niet brengen}\\

\haiku{{\textquoteright} {\textquoteleft}Ja, mijnheer, ik heb,{\textquoteright},:}{U een verzoek te doen zei}{onze vriend en toen}\\

\haiku{Zij was nu echter.}{waarachtig een beetje bang}{geworden van hem}\\

\haiku{Morgen zou hij mr.,.}{Taccle opzoeken om eens}{met hem te spreken}\\

\haiku{Er wordt geklopt en.}{op zijn uitnoodiging komt een}{verpleegster binnen}\\

\haiku{wij hebben ons geld.}{gehad en zijn verders niets}{meer te verwachten}\\

\haiku{maar het komt mij goed,.}{uit dat U Uw belofte}{niet hebt gehouden}\\

\haiku{Want het prettige,}{van de zaak was dat geen der}{geattaqueerden}\\

\haiku{Werkelijk, ook dat.}{is in het belang van de}{zaak noodzakelijk}\\

\haiku{Als eerste daad zocht.}{hij de verblijfskaart van den}{ouden Strenger op}\\

\haiku{Hopkins verlegde dus.}{zijn interesse naar de}{rest van het gebouw}\\

\haiku{Zijn schot barstte los,.}{gevolgd door een knal uit de}{andere pistool}\\

\haiku{hij zou met behulp.}{van beitel en hamer de}{bergplaats openbreken}\\

\haiku{{\textquoteright} {\textquoteleft}Nu, maakt U zich nu,.}{dan maar geen zorgen meer Uw}{geheim is veilig}\\

\haiku{Maar gaat U verder,;}{neemt U mij niet kwalijk dat}{ik U onderbrak}\\

\section{Augusta de Wit}

\subsection{Uit: De avonturen van den muzikant}

\haiku{Een stilte in de,,.}{lucht een reuk uit den grond een}{zachtheid in het licht}\\

\haiku{Allard zette de.}{lippen aan de opening en}{blies uit alle macht}\\

\haiku{De Boegi kwamen,.}{in schepen hoog van boeg als}{heuvels op de zee}\\

\haiku{Zoo dikwijls vroeg hij.}{er om dat zij het uit het}{hoofd had leeren spelen}\\

\haiku{Het was nog in de,.}{eerste vroegte het koele}{begin van den dag}\\

\haiku{Daarboven begon,,.}{een lichte melodie speelsch}{lachend enkel vreugd}\\

\haiku{Zijn zoon, Benkol, bracht.}{hem water uit het ravijn}{en hout voor het vuur}\\

\haiku{De vader reikte.}{naar dat witte wangetje}{om het te streelen}\\

\haiku{Als in een spiegel.}{zag hij in het gezicht der}{moeder naar zijn kind}\\

\haiku{Maar hoe armer wij,.}{zijn hoe meer wij de vreugd der}{muziek noodig hebben}\\

\haiku{Maar als de klokken,.}{begonnen te spelen dan}{luisterde alles}\\

\haiku{Op het Eiland, waar,.}{ik geboren ben klinkt dan}{zoo de gamelan}\\

\haiku{Maar de blik uit die,.}{doordringende oogen was z\'oo}{goed daar kwam moed van}\\

\haiku{Daar waren Haydn en,.}{Mozart de goede geesten}{van Beethovens jeugd}\\

\haiku{twee anderen de,;}{een een banjo de ander}{een mandoline}\\

\haiku{Zij gingen al de,.}{straten door de arme zoo}{goed als de rijke}\\

\haiku{maar als hij toch maar!}{tot na de Pinksteren had}{gewacht met komen}\\

\haiku{op de gelige.}{bladen bleven woorden en}{noten duidelijk}\\

\haiku{Hij hoorde er de.}{dagelijksche geluiden}{van de hofstede}\\

\haiku{maar door die strenge.}{statigheid heen breekt telkens}{een gloed van hartstocht}\\

\haiku{Het is of de stroom.}{rood wordt in den afschijn van}{een brandende stad}\\

\haiku{Zij bad Allard te;}{beginnen met de studie}{voor ingenieur}\\

\haiku{Maar hij liet het in.}{zijn zak zitten en deed braaf}{zijn best op een klomp}\\

\haiku{Ik moet nu eenmaal.}{naar de Schoone Danseres}{in den Waterval}\\

\haiku{Hij liep voort langs de,.}{rivier die gaf nog licht langs}{het zwartige riet}\\

\haiku{Juist doofde het licht;}{in het boogvenster en de}{menschen gingen heen}\\

\haiku{De man wischte zich:}{het zweet van het gezicht en}{haalde adem en riep}\\

\haiku{Hij sloeg Joris,.}{op den schouder en schudde}{hem haast de hand af}\\

\haiku{zulk een waterval,?}{als deze woont daarin de}{Schoone Danseres}\\

\haiku{wat had hem bezield,!}{dat hij juichte over wat nu}{hem woedend maakte}\\

\haiku{maar toen pas goed, wat.}{je over hebt voor de muziek}{en voor de vriendschap}\\

\haiku{Nu gingen aan de.}{bergbeek nieuwe wijdten open}{voor zijn gedachte}\\

\haiku{enkel klaarte en,.}{liefelijkheid een beekje}{door den zonneschijn}\\

\haiku{Hij speelde wat een:}{lievelingsstuk was geweest}{van zijn grootvader}\\

\haiku{Alois Tieffenbrucker,,.}{fecit en een datum van}{veertig jaar her 1860}\\

\haiku{Daarom moesten ook zijn;}{f-gaten anders zijn}{dan die van Stainer}\\

\haiku{Maar weer peinsde hij,,,.}{te te vergeefs waar wanneer}{hij dien had gehoord}\\

\haiku{gaarne, zei hij, liet,.}{hij hem de viool houden}{zoolang hij wilde}\\

\haiku{en de jaren van.}{nood wanneer de vezel los}{en slap was gegroeid}\\

\haiku{Is, dan, zoo broos het,?}{schoone dat het breekt aan de}{eigen volmaking}\\

\haiku{{\textquoteright} Betooverd zag hij op,.}{de ranke rosblonde de}{leidster van den dans}\\

\haiku{{\textquoteright} {\textquoteleft}Bijen, wij doen U,.}{te weten dat de oude}{Vrouw gestorven is}\\

\haiku{Een is er over den,;}{uil die zit te kijken met}{zijn starende oogen}\\

\haiku{tirannie vervoert.}{mij tot een hartstocht dien ik}{niet kan beheerschen}\\

\haiku{{\textquoteleft}Het eerste is geld.}{en het tweede is geld en}{het derde is geld}\\

\haiku{den korten zin van.}{deze lange rede heb}{ik nog in petto}\\

\haiku{De naam van een neef,,.}{die haar mans compagnon was}{klonk telkens daarin}\\

\haiku{Zij lachte zelve,.}{en dat was als het klokken}{van een bronnetje}\\

\haiku{Allard hielp Lucie;}{er in en roeide naar het}{midden van het meer}\\

\haiku{Aloisl, die met een,:}{tak lijsterbessen zwaaide}{kreet uit volle keel}\\

\haiku{een warme wind droeg,.}{geur aan van hars varenkruid}{en klein gebloemte}\\

\haiku{En wie was hij zelf?}{om zelfs maar uit de verte}{tot haar op te zien}\\

\haiku{{\textquoteright} De herinnering:}{aan den toon van haar stem bij}{dat haast onhoorbaar}\\

\haiku{Zij viel mij in het -,!}{oog door haar kleur fel groen groen}{als een kikvorsch}\\

\haiku{- dan zal zij zingen!}{dat de engelen in den}{hemel glimlachen}\\

\haiku{De varkensslachters!}{betalen uit beurzen als}{hun varkens zoo vet}\\

\haiku{Mannen, vrouwen en}{kinders uit de bergdorpen}{grepen naar zijn hand.}\\

\haiku{Later had Allard;}{op zijn vriends werktafel het}{manuscript gezien}\\

\haiku{Zij vermaanden hem.}{te sparen wat zijn eigen}{muziek behoefde}\\

\haiku{Het gerucht verstierf,,.}{de haast verlangzaamde het}{geweld verstilde}\\

\haiku{Ook velen kwamen:}{die bij dage aan feesten}{niet konden denken}\\

\haiku{Onder het lezen.}{moest Allard telkens denken}{aan Tieffenbrucker}\\

\haiku{{\textquoteleft}Wie weet of hij het,{\textquoteright}.}{er niet weer in voegde nu}{zei Tieffenbrucker}\\

\haiku{, en tot tjilpende.}{vogeltjes van de blijheid}{sprak die eeuwig is}\\

\haiku{Het rosse goud der.}{lokken gloorde boven een}{lelieblanken nek}\\

\haiku{Het instrument van;}{de eerste viool was zijn}{werk en zijn geschenk}\\

\haiku{Zij knielde alleen.}{om de verheffing uit te}{zingen van haar geest}\\

\haiku{nu reeds maakten zij:}{telkens aanmerkingen op}{zijn repetities}\\

\haiku{en verweerden zij {\textquoteleft}}{zich zoo woedend omdat een}{binnenste stemJa}\\

\haiku{{\textquotedblright} tot dit dochtertje.}{van Cunera tevergeefs}{gezegd zal blijken}\\

\haiku{{\textquoteleft}Maar toen heeft Kempfen,.}{gedacht hoe minder woorden}{daarover hoe beter}\\

\haiku{Hoe vond hij ineens?}{de woorden weder van de}{taal der Eilanders}\\

\subsection{Uit: De drie vrouwen in het Heilige Woud}

\haiku{{\textquoteright} en ging heen uit het,;}{prachtige paleis door vrouw}{noch dienaar gevolgd}\\

\haiku{Telkens echter ging;}{zij met bloemenoffers naar}{het Heilige Woud}\\

\haiku{Ook toen het tijd werd.}{om de sawahs te ploegen}{kwam hij niet terug}\\

\haiku{Haar armen deden.}{pijn van verlangen naar zulk}{een glad klein lijfje}\\

\haiku{Er was stof op haar,}{ruig haar achteloos hing haar}{sarong dien zij tot}\\

\haiku{en hem  ook scheen}{het dat niet anders dan een}{zieke naar den geest}\\

\haiku{Mboq-Inten nam.}{haar hand en streelde die langs}{haar eigen gezicht}\\

\haiku{van  doornigen.}{rottan die met gehaakte}{zweepen naar hen striemt}\\

\haiku{Flikkerend in de.}{middagzon opgevlogen}{schreeuwen zij van vreugd}\\

\haiku{Groen en goud stond het;}{kustgebergte in schoonen}{halfboog te blinken}\\

\haiku{Nog eens beproefde,.}{hij de spanning van zijn zeil}{het spel van zijn roer}\\

\subsection{Uit: Het dure moederschap}

\haiku{Als een groene zee,,.}{lag zij daar blinkende te}{deinen in de zon}\\

\haiku{hij dacht dat het wel.}{aardig zou wezen kermis}{te houden met haar}\\

\haiku{Hij heeft zijn nest op,{\textquoteright}, {\textquoteleft}.}{Hartestein zei Tijmener}{zijn vijf jongen in}\\

\haiku{Zij had het licht op.}{en de halfdeur open voor de}{andere klopte}\\

\haiku{Dien Zondag in de.}{kerk leek het haar of Tijmen}{telkens naar haar keek}\\

\haiku{geen zonneschijn nog.}{maar de kleurlooze klaarte die}{er aan voorafgaat}\\

\haiku{op dat oogenblik,.}{eerst begreep hij dat zij van}{niets geweten had}\\

\haiku{Halfweg thuis, meende, {\textquoteleft},!}{zij hem nog te hooren met}{zijn snikkendMoetje moetje}\\

\haiku{Toen zij gehoord had,,:}{waar Fokje was ging zij hem}{dadelijk halen}\\

\haiku{Eens zelfs wou hij het.}{bed uit en bleef een poosje}{zitten bij het vuur}\\

\haiku{volstrekt noodig niet, of,?}{het dan goed was voor het kind}{of hij het aanried}\\

\haiku{De grond deinde voor,.}{haar voeten of zij z\'o\'o in}{het leege zou treden}\\

\haiku{Marretje dacht al,:}{dat hij sliep toen hij ineens}{rechtop ging zitten}\\

\subsection{Uit: De godin die wacht}

\haiku{Je zult zien dat je,!}{nog niet eens klaar bent met je}{werk voor het om is}\\

\haiku{Hij smeet het scheldwoord.}{op goed geluk naar \'een van}{die twee gezichten}\\

\haiku{{\textquoteleft}Er was een kuil in,{\textquoteright}.}{den weg zei de voerman in}{zijn neurend maleisch}\\

\haiku{na een kwartier leek.}{het hem of hij ze beiden}{al lang gekend had}\\

\haiku{Daarop klonk een stap.}{over het kiezelpad en van}{Heemsbergen verscheen}\\

\haiku{De rechter zag hem,.}{aan verwonderd als over een}{nog nooit gehoord iets}\\

\haiku{Hij komt zeker veel,?}{aan huis bij professoren}{en zoo is het niet}\\

\haiku{Je mag al dankbaar!}{zijn als je achterstand niet}{\`al te erg oploopt}\\

\haiku{Een plotselinge;}{ruk slingerde hem tegen}{van Heemsbergen aan}\\

\haiku{de Bakker zit daar,{\textquoteright}.}{binnen voegde hij er met}{een knipoogje bij}\\

\haiku{{\textquoteleft}Dat is hier vandaan,{\textquoteright}.}{te hooren antwoordde de}{controleur droogjes}\\

\haiku{{\textquoteleft}Ja, de kerel is,.}{niet te vinden geweest juist}{zooals je voorspeld hadt}\\

\haiku{Ook een beste man,,.}{een beste man als je hem}{maar eerst leert kennen}\\

\haiku{Ik ken hem nu al,.}{een paar jaar en ik kan het}{best met hem vinden}\\

\haiku{Van Heemsbergen, die,.}{dacht goed Maleisch te kennen}{verstond er niets van}\\

\haiku{En dan weer op zijn,,!}{plaats hoor je daar achter op}{de bovenste plank}\\

\haiku{en vroeg naar dingen.}{waarvan hij op zijn best wist}{dat zij bestonden}\\

\haiku{{\textquoteright} {\textquoteleft}De Bakker heeft me,..... -,?}{den naam wel gezegd maar wacht}{eens Bruton kan dat}\\

\haiku{W\`at hij eigenlijk,.}{van hem verwachtte had hij}{niet kunnen zeggen}\\

\haiku{Een ijle geur van.}{kruiden verlevendigde}{de zuivere lucht}\\

\haiku{{\textquoteleft}Hij heeft graag dat ik,,....}{luister naar wat er gezegd}{wordt soms over en weer}\\

\haiku{{\textquoteleft}Het is mijn werk,{\textquoteright} zei,.}{hij na een oogenblik en}{zijn toon was weer koel}\\

\haiku{Nu stond hij stil voor:}{Hendriks en zijne vrouw en}{vroeg hartstochtelijk}\\

\haiku{In erger tweedracht.}{met zich zelven dan toen hij}{ging kwam hij weerom}\\

\haiku{Hij zei het hardop,,.}{in zijn verbazing en vond}{verder geen woorden}\\

\haiku{En ondertusschen!}{zit ik hier en niemand hoort}{of ziet wat van me}\\

\haiku{Dat is goed om over.}{twintig jaar een standaardwerk}{te kunnen schrijven}\\

\haiku{In en rondom het:}{machinegebouw gonsde}{het van de drukte}\\

\haiku{{\textquoteleft}En als je ze dan -!,!}{van dichtbij O die \'eene}{wat een prinsesje}\\

\haiku{Een dommelige:}{herinnering schoot wakker}{in van Heemsbergen}\\

\haiku{Hij zag een woord op.}{de lippen van den schilder}{en hield het tegen}\\

\haiku{Je verwijt me dat....}{ik niet genoeg belang stel}{in den Inlander}\\

\haiku{Ik heb mijn knuppel,.}{naar ze gesmeten maar ik}{raakte er niet een}\\

\haiku{Op zijn beurt keek hij.}{eens naar wat den ander zoo}{opgetogen hield}\\

\haiku{Als ik niet wist dat....}{mevrouw de Bakker naar me}{zou laten vragen}\\

\haiku{De een wist zich niets;}{meer te herinneren bij}{de ondervraging}\\

\haiku{Eerst had hij den voogd:}{van den overledene met}{hem vereenzelvigd}\\

\haiku{{\textquoteright} {\textquoteleft}Die man achter de;}{schermen is nu een idee van}{Bossing en van u}\\

\haiku{de woorden zich in,:}{den mond vormen waarop hij}{wist dat zij wachtte}\\

\haiku{Als een zilveren.}{plas lag het maanlicht binnen}{oevers van schaduw}\\

\haiku{Hij hoorde alleen,,:}{in zich diezelfde nog maar}{\'eens vernomen stem}\\

\haiku{Beide handen op.}{haar schouders leggend hield hij}{haar even van zich af}\\

\haiku{{\textquoteleft}U hebt zeker al -.}{wel geraden wie ik ben}{de vriendin van Ada}\\

\haiku{Maar hij voelde hun.}{fijne stralen prikkelen}{door zijn duisternis}\\

\haiku{genoeg zou zijn voor.}{een sitsen kabaja of}{een netten hoofddoek}\\

\haiku{- Dat beeld waar ze zoo,.}{veel offeren laten we}{daarnaar gaan kijken}\\

\haiku{de greep die hij om,.}{haar polsen geslagen had}{spande zijn vingers}\\

\haiku{Met je welnemen, {\textquotedblleft}{\textquotedblright},!}{zoo'n volmondigja zoo ja}{en amen is dat niet}\\

\haiku{De groote meerderheid!}{doet dat. Op dat punt zijn de}{meesten Inlanders}\\

\haiku{als ik leef naar mijn.}{hoogste dan als ik leef naar}{mijn laagste kunnen}\\

\haiku{{\textquoteleft}Nu we er toch over,, -!}{spreken laat nu ook alles}{gezegd zijn alles}\\

\haiku{Ik sprak alleen maar;}{van hem om je geheugen}{op weg te helpen}\\

\haiku{Mevrouw Meerhuys zat,;}{aan de tafel in het}{schijnsel van de lamp}\\

\subsection{Uit: Gods goochelaartjes}

\haiku{Het was oorlog op -.}{Ambon toen het was altijd}{oorlog op Ambon}\\

\haiku{Er zijn er niet zoo.}{velen die lust hebben hem}{daar te gaan vangen}\\

\haiku{het verhaal van zijn;}{drieen-twintigsten tocht den}{Mont-Ventoux op}\\

\haiku{het moest doodsangst zijn.}{die hun de wanhopige}{kracht had gegeven}\\

\haiku{Hij had boeken, riep.}{hij uit die een Museum}{hem zou benijden}\\

\haiku{Hoeveel mijn vader;}{van de zaak wist hebben wij}{zoons nooit vernomen}\\

\haiku{Voor wien werkten hij?}{en mijn moeder anders dan}{voor ons kinderen}\\

\haiku{Dan, zachtjes, klom ik;}{het raam uit en in de kruin}{van den magnolia}\\

\haiku{Fabre, ik wist,.}{het had er twintig jaar te}{vergeefs naar gezocht}\\

\haiku{Hij zag mij aan of.}{hij dacht dat ik plotseling}{gek was geworden}\\

\haiku{{\textquoteright} En hij ging de trap.}{achter mij op en sloot mijn}{deur van buiten af}\\

\haiku{Ik verveelde mij!}{schromelijk tusschen al die}{doode insecten}\\

\haiku{En hier op Ambon,,.}{zag ik kon een mensch leven}{van zoo goed als niets}\\

\haiku{maar sterk gevormd, als,,;}{gebeiteld hier zwak belijnd}{als uitgewischt daar}\\

\haiku{En als er iemand,,}{uit het dorp komt hoort hij hem}{aan maar zelf zegt hij}\\

\haiku{vernieling straks, was.}{het vruchtbaarheid nu over het}{afgeoogste land}\\

\haiku{Jan moest dadelijk.}{Meneer Schepers narijden}{en hem thuis brengen}\\

\haiku{Hij had een klaproos}{achter zijn oor gestoken}{die wijdopen vlamde}\\

\haiku{Maar die vreugde zou.}{voorspel zijn van de Vreugde}{in der Eeuwigheid}\\

\haiku{In mijn kamertje:}{boven den stal kon ik een}{piano zetten}\\

\haiku{O hoe mooi heb ik!}{menschen zien worden die vaal}{en dof waren eerst}\\

\haiku{Hij sloeg den deksel,,.}{op en begon te spelen}{mijn vlinderdeuntje}\\

\haiku{hij wist dat wat voor,;}{slechtheid wordt veroordeeld \'ook}{leed is het ergste}\\

\haiku{De gieren van die,.}{nachtmerrie die mij zoo lang}{had gekweld schreeuwden}\\

\haiku{Zij alleen weten,.}{raad die niet macht en bezit}{maar liefde willen}\\

\haiku{ik heb je altijd;}{geprezen om je schoonheid}{en je vroolijkheid}\\

\haiku{Men kan het zich haast.}{niet voorstellen wat de man}{toen geleden heeft}\\

\haiku{hij zich had verheugd,.}{als Vader in mij verblijd}{ook om zichzelfs wil}\\

\haiku{Opeens stak Herman,.}{mij de hand toe ik legde}{er de mijne in}\\

\haiku{De kleine man, van,.}{doorstane smarten zoo wreed}{geteekend jubelde}\\

\subsection{Uit: Orpheus in de dessa}

\haiku{Zoo bedwelmd was hij,.}{dat hij zelfs niet bewoog toen}{ik vlak voor hem stond}\\

\haiku{[III] Tot het donker;}{werd wachtte Bake op hem}{den volgenden avond}\\

\haiku{Hij voelde zich  .}{voortgestuwd op den stroom die}{de werelden draagt}\\

\haiku{Met een bekommerd;}{gezicht stond Bake bij de}{nieuwe machine}\\

\haiku{En dat gaf hem een,}{gevoel van vroolijken moed}{als voor een slag dien}\\

\haiku{Hij stapte van 't,.}{paard en ging in den lommer}{zitten uitrusten}\\

\haiku{Maar als het nu eens,?}{een heele bende was die}{systematisch steelt}\\

\haiku{Of ze verkoopen.}{hun oogst voor een prikje een}{paar maanden vooruit}\\

\haiku{aan den kant van den,,!}{weg blijven op het gras dat}{ze ons niet hooren}\\

\haiku{Behoedzaam droeg hij '.}{het opt koele zachte}{leger in de kar}\\

\haiku{{\textquoteright} De houding van het.}{vergroeide lichaampje leek}{hem ondragelijk}\\

\subsection{Uit: Verborgen bronnen}

\haiku{De Heggelersdijk.}{was de buurt van de stroopers}{en de smokkelaars}\\

\haiku{Wel ieder papwurm ' '.}{hier int dorp zout je}{kunnen vertellen}\\

\haiku{Ik ga naar 't dorp,!}{en haal brandewijn dat is}{veel beter voor hem}\\

\haiku{- Daar ging de deur open,.}{en het gemutste hoofd der}{meid keek naar binnen}\\

\haiku{Zijne moeder had:}{het koffie-water over}{het vuur gehangen}\\

\haiku{{\textquoteleft}Laat ze maar liever,{\textquoteright}.}{op der eigen huid passen}{antwoordde Nellis}\\

\haiku{De poelier in de.}{stad had er in geen tijden}{zulke mooie gehad}\\

\haiku{Hij droogde zich het,,:}{klamme voorhoofd af en na}{een wijle zeer zacht}\\

\haiku{Dacht de Domin\'e!}{dat ik een kameraad zijn}{haas zou afstelen}\\

\haiku{Nellis nam het glas,.}{van de mooie Jaan aan zonder}{haar toe te knikken}\\

\haiku{Er was niets te zien.}{tusschen de lage stammen}{der kerseboomen}\\

\haiku{Gerrit stapte naar,.}{zijn werk dien morgen als ging}{hij naar zijn geluk}\\

\haiku{En ie zal wel niet,.}{hard deuge ook want ie is}{in den Oost geweest}\\

\haiku{Met rinkelende,.}{schreden stapte hij heen zijn}{knevel opstrijkend}\\

\haiku{{\textquoteleft}Maar je hebt 'em dan ',,}{tochezien verdikkeme toen}{hij daar zoo vlak langs}\\

\haiku{{\textquoteleft}A-j wat gewaar,,.}{wordt dan roep je maar dan kom}{ik je wel helpen}\\

\haiku{Welzeker, zweer jij,,!}{maar zweer jij maar en zie dan}{eens wie je gelooft}\\

\haiku{Het zal je heugen,!}{dat je veldwachter Koenen}{hebt voorgelogen}\\

\haiku{{\textquoteleft}U stond toch bij den,,?}{hooiberg nietwaar waar de brand}{aangekomen is}\\

\haiku{Gerrit staarde op.}{het papier dat hij in zijn}{klamme vingers hield}\\

\haiku{Daar buiten is de.}{glorie van Veneti\"e en}{den Junihemel}\\

\haiku{Uit het doorlouterd.}{zwart van het kelkhart welde}{rijk azuur te voorschijn}\\

\haiku{Hij was er, om te,.}{zorgen dat dat geschiedde}{daarom bestond hij}\\

\haiku{{\textquoteright} Driemaal in de week.}{ging hij nu des avonds naar den}{ouden schoolmeester}\\

\haiku{{\textquoteright}... Eindelijk legde.}{hij zich op zijn leger van}{mos en varenkruid}\\

\haiku{De man staat langzaam,.}{op zonder om te kijken}{of te antwoorden}\\

\haiku{Hij droeg een nieuwe,.}{sarong en zijn schoonvaders}{kris in den gordel}\\

\haiku{Dus, dag aan dag, zag.}{Mian zijns vijands zoon schooner}{en sterker worden}\\

\haiku{Nu vielen alle,:}{vrouwen tegelijk in door}{elkaar heen roepend}\\

\haiku{Tusschen hen gleden -.}{de handen der vrouwen heen}{en weer heen en weer}\\

\haiku{En als een zwoele.}{zwarte vloed nam mij op en}{omving mij het woud}\\

\haiku{De dessaman had,}{onderweg van een hollen}{bamboestengel dien}\\

\haiku{Hij ging te midden.}{daarvan als des konings zoon}{door des konings hof}\\

\haiku{Van mijne plaats kon.}{ik haar heen en weder zien}{gaan bij het rijsvuur}\\

\haiku{Voor een oogenblik.}{hield zich het gerucht van den}{tropischen nacht in}\\

\haiku{Ik heb den dienaar,!}{van den wedono gezien}{dragende uw kind}\\

\haiku{{\textquoteleft}Wat bekommer je?}{je zoo zeer om dat kind van}{een geringen man}\\

\haiku{Zijn gezicht verwrong.}{dat het niet meer het gezicht}{van een mensch geleek}\\

\subsection{Uit: De wake bij de brug}

\haiku{{\textquoteleft}Waarom heeft hij dit,?}{gedaan hij die toch om hulp}{riep tegen den dood}\\

\haiku{De stam der Boegi,,.}{het zwervende zeevolk is}{vrienden met den wind}\\

\haiku{Van den steven naar,!}{den boeg loopend kom met een}{driftige vaart Heer}\\

\haiku{Korven vol werden.}{naar hen afgelaten en}{gulpende emmers}\\

\haiku{Daar beukten storm en.}{stortzee de jonk te pletter}{en een van hen stierf}\\

\haiku{Rond en rondom de.}{stoomboot heen ging muziekend}{de statige dans}\\

\haiku{te hard is die op:}{hun aan zacht vloeienden klank}{gewende lippen}\\

\haiku{dochters van den adel.}{kwamen daar en kinderen}{uit dessahuizen}\\

\haiku{De zenith was een,.}{zwarte wel waar wolken de}{golven in waren}\\

\haiku{De muizen hadden.}{hun kracht  opgegeten}{die nog te veld stond}\\

\haiku{Een wijze was het.}{die nu begon als niemand}{nog vernomen had}\\

\haiku{Hij dacht, wat is toch,?}{dat witte daar dat witte}{in den zwarten grond}\\

\haiku{druipenden boomstronk.}{dien het gedrocht losgewroet}{had uit de diepte}\\

\haiku{Die zij vingen en.}{aan wal trokken legden zij}{vast aan de boomen}\\

\haiku{Wij zagen het op,.}{ons afkomen wij die op}{den steiger stonden}\\

\haiku{Hij stond een poos stil.}{en zag weer naar den oever}{en keerde weerom}\\

\section{Aagje Deken en Betje Wolff}

\subsection{Uit: Historie van mejuffrouw Cornelia Wildschut}

\haiku{Mama ook niet, maar;}{zij doet veel om andere}{lui te plezieren}\\

\haiku{mijn beurs lijdt meer  *.}{door haar dan mijn ziels-}{of lichaamskrachten}\\

\haiku{Lichtmis als ik ben;}{kan ik geen vrouw dulden die}{haar sekse verzaakt}\\

\haiku{Of zij meer dom dan,;}{wel onkundig is kan ik}{nog niet bepalen}\\

\haiku{hij schijnt veel meer in.}{zijn huis gelogeerd te zijn}{dan er te wonen}\\

\haiku{zijt om u met uw?}{afwezige familie}{niet te bemoeien}\\

\haiku{*~        4 Cornelia!}{Wildschut aan Betje Stamhorst}{Mijn lieve Betje}\\

\haiku{Ik weet niets, en ben,,;}{zo als gij wel denken kunt}{verlegen om stof}\\

\haiku{Keetje heeft er geen,.}{nadeel van zij weten niet}{van wie die brief komt}\\

\haiku{nu dat weet gij, maar,.}{slapen is gezond en ook}{ik heb niets te doen}\\

\haiku{in 't eerst liep ik,;}{machtig hoog met haar maar het}{was al gauw gedaan}\\

\haiku{Kort gezegd, zo ik,;}{trouw dan zal het zeker uit}{zelfverveling zijn}\\

\haiku{waarover kan u niet,}{dan onverschillig zijn want}{gij zijt zo weinig}\\

\haiku{8 Willem Stamhorst!}{aan Paulus Wildschut Mijnheer}{zeer waarde broeder}\\

\haiku{als hij wat ouder.}{is zal dat grote vuur wel}{wat verminderen}\\

\haiku{Ik hou wel veel van,.}{nicht doch onze humeuren}{verschillen te zeer}\\

\haiku{Ging ik al eens mee,;}{dan was het om te doen als}{alle anderen}\\

\haiku{A propos Hein, het?}{is immers uw voornemen}{om haar te trouwen}\\

\haiku{{\textquoteright} Wat mijn vrijerij,.}{betreft gij kunt wel merken}{dat die niet vordert}\\

\haiku{dan vreemdelingen;}{die enig en alleen om hun}{negotie reizen}\\

\haiku{en dewijl gij met.}{een vriend uit de stad waart zag}{ik u even weinig}\\

\haiku{{\textquoteright} - {\textquoteleft}Ik zal deze avond,}{mijn kamerdeur openlaten}{en dan kunt gij zo}\\

\haiku{Gij weet dat ik u,;}{liefheb en ik weet dat ik}{daar reden toe heb}\\

\haiku{het welk niet in haar.}{moeders bijzijn kon gezegd}{of gedaan worden}\\

\haiku{zo al niet volstrekt,.}{onmogelijk echter hoogst}{onwaarschijnlijk is}\\

\haiku{De heer Wildschut poogt,:}{dit goed te maken en het}{is alle ogenblik}\\

\haiku{Ik kon daar niet veel -.}{op antwoorden ik zag het}{gezelschap eens over}\\

\haiku{in 't kort, ieder,.}{zat of bij geval of met}{oogmerk daar hij zat}\\

\haiku{Het spijt mij bijna,:}{dat ik het met haar eens was}{doch zij had gelijk}\\

\haiku{{\textquoteright} Men presenteerde,;}{thee en toen sloeg mevrouw voor}{om pand te spelen}\\

\haiku{ik hoop dat gij het.}{onnodig zult maken veel}{daarover te schrijven}\\

\haiku{{\textquoteleft}Maar,{\textquoteright} voegde zij er, {\textquoteleft},!}{bijhet is zijn zusters schuld}{mijn lieve mevrouw}\\

\haiku{en daar moet gij u,,.}{niet aan storen want het is}{zo en niet anders}\\

\haiku{En zouden onze?}{jonge juffrouwen u zo}{gauw zetten mogen}\\

\haiku{Want zulk een oorvijg.}{voor mijn eigenliefde zou}{mij razend maken}\\

\haiku{de heer Wildschut zal;}{van kwaadheid of chagrijn of}{van beide sterven}\\

\haiku{Maar gij hebt ook haar;}{vader wel duizend mijlen}{van mij verwijderd}\\

\haiku{Het zijn zo zeer niet;}{talenten en verstand die}{ons daar aanprijzen}\\

\haiku{Ja mijn goede heer,!}{Van Arkel ik heb thans een}{kruis in de wereld}\\

\haiku{{\textquoteright} Hij durfde, denk ik,,.}{mijn antwoord niet afwachten}{maar vloog de zaal uit}\\

\haiku{en Wildschut, al wil,.}{hij het niet weten ziet er}{ook ongedaan uit}\\

\haiku{hoe dikwijls heb ik!}{u dit niet door lessen en}{voorbeelden getoond}\\

\haiku{{\textquoteright} Had gij mij dat nu,;}{vooraf gevraagd ik zou u}{zulks gezegd hebben}\\

\haiku{Ik bemin haar, en.}{ik weet dat zij voor mij niet}{onverschillig is}\\

\haiku{met zo iemand, die,.}{niet ouder is is nog wel}{iets te beginnen}\\

\haiku{en dewijl de heer,}{De Groot hier juist aan huis was}{heb ik hem doen zien}\\

\haiku{men kan niet weten,?}{en waarom zou een braaf mens}{in verdriet komen}\\

\haiku{Uw zegepraal zal,.}{niet lang duren uw vader}{zal het besterven}\\

\haiku{doodsbenauwd, bijna,,,...}{naakt alles losgescheurd bleek}{en stuiptrekkingen}\\

\haiku{Kort daarop komt Frans - {\textquoteleft}!}{buiten adem gelopen en}{zeer ontsteldMevrouw}\\

\haiku{u mocht vragen - doch -,:}{dat zal niet gebeuren waar}{wij gaan zeg dan maar}\\

\haiku{{\textquoteleft}Mijnheer Lenting{\textquoteright} zei, {\textquoteleft};}{zijis sedert acht dagen}{in  commissie}\\

\haiku{{\textquoteright} Frans weigerde dit,.}{doch hij moest het doen om haar}{te vergenoegen}\\

\haiku{{\textquoteright} Met een sidderend;}{verlangen rukte ik haar}{de brief uit de hand}\\

\haiku{Zoek mij niet, onze.}{maatregelen hebben dat}{vruchteloos gemaakt}\\

\haiku{Spreekt zij niet van mij,?}{als van haar moeders man niet}{als van haar vader}\\

\haiku{Als men geen plezier,;}{van zijn kinderen heeft raakt}{het hart er ook af}\\

\haiku{De jonge heer wacht,;}{voor de deur doch het begon}{hem te vervelen}\\

\haiku{En onze lieve.}{Heer kan een mens altoos de}{bekering geven}\\

\haiku{Zij was mij (al had):}{ik zulks veilig kunnen doen}{geen geweld waardig}\\

\haiku{{\textquoteright} zei ik, {\textquoteleft}dan zult gij.}{best doen om weder naar uw}{vaders huis te gaan}\\

\haiku{Denkt gij dat zij hen?}{zal behandelen zoals}{zij mij behandelt}\\

\haiku{Ja zo waar, des daags;}{na de dag van uw vertrek}{kwam hier Frans Ligthart}\\

\haiku{, ik mij geen rust gaf.}{voor ik mijn kind gevonden}{en behouden had}\\

\haiku{Ik zag u reeds van,;}{verre en meen dat gij in}{verlegenheid zijt}\\

\haiku{Ziende dat ieder;}{zijn gewonnen geld met zich}{nam deed ik ook zo}\\

\haiku{doch ik durfde het,.}{niet te wagen uit vrees of}{men het merken zou}\\

\haiku{Ik zei mijn vader -,;}{te haten mijn hart wringt mij}{terwijl ik dit schrijf}\\

\haiku{gij zult des tegen.}{deze uw grootste vijand}{op uw hoede zijn}\\

\haiku{Deze brief, mijn kind,;}{is alles wat uw vader}{u kan nalaten}\\

\haiku{Bij hem komende.}{vond ik hem tegen een boom}{leunende zitten}\\

\haiku{{\textquoteright} barstte ik uit, {\textquoteleft}hoe!}{gaarne zou ik u als mijn}{vriend bemind hebben}\\

\haiku{{\textquoteright} Hier vouwde hij zijn,}{handen samen en zijn hoofd}{opheffende bad}\\

\haiku{{\textquoteleft}Mijn besluit{\textquoteright}, zei hij, {\textquoteleft},.}{is genomen ik ontwaak}{uit mijn bedwelming}\\

\haiku{wij zien er ook uit,.}{hoe w\'el men zich uitdrukt als}{het hart ons dicteert}\\

\haiku{Kunt gij uw arme,?}{vernederde vriendin wel}{alles vergeven}\\

\haiku{Ik ben de oorzaak ',!}{vant verlies van uw man}{mijn lieve vader}\\

\haiku{Zij schijnt zeer gerust,,.}{maar ligt doorgaans stil hoewel}{haar ogen ons volgen}\\

\haiku{Ik ben altoos uw,.}{eerbiedigende vriendin}{Anna Hofman}\\

\haiku{Zij zag haar tante,,:}{aan met een aandoenlijke}{minzaamheid en zei}\\

\haiku{Ik hoop immers dat,,?}{gij mijn kind plichtmatig denkt}{omtrent uw moeder}\\

\haiku{waarin men de fijnste;}{roersels en verborgenste}{springveren doorziet}\\

\haiku{Karakter- en.}{levensbeeld van Betje Wolff}{en Aagje Deken}\\

\haiku{bededag biddag,;}{dag van georganiseerd}{algemeen gebed}\\

\haiku{of hij het niet te.}{vast heeft of hij niet geheel}{bij zijn verstand is}\\

\section{J. Woltjer}

\subsection{Uit: Verzamelde redevoeringen en verhandelingen}

\haiku{maar een zelfstandig.}{bestaan als wetenschap had}{de philologie niet}\\

\haiku{een \ensuremath{\lambda}o\ensuremath{\gamma}o\ensuremath{\pi}o\ensuremath{\i}\'{o}\ensuremath{\varsigma}, een maker,.}{van een logos is bij hem}{een geschiedschrijver}\\

\haiku{maar de waarneming;}{en de voorstelling waren}{volkomen zuiver}\\

\haiku{en zooals Adam alle,.}{levende ziel noemen zou}{dat zou haar naam zijn}\\

\haiku{God zullen kennen,.}{gelijk zij zelven nu door}{Hem gekend zijn98}\\

\haiku{God heeft uit \'e\'enen;}{bloede het gansche geslacht}{der menschen gemaakt}\\

\haiku{want dit is wel het,.}{eerste dat hun de woorden}{Gods zijn toebetrouwd}\\

\haiku{in de eerste, in,:}{de tweede in de derde}{plaats en altoos weer}\\

\haiku{'t Zij mij vergund.}{een paar punten slechts voor uwe}{aandacht te brengen}\\

\haiku{'t Kan toch het doel?}{niet zijn een tekst zoogenoemd}{leesbaar te maken}\\

\haiku{'de liefde handelt,;}{niet lichtvaardiglijk zij is}{niet opgeblazen}\\

\haiku{dit deel van hunnen;}{arbeid komt hun voor een vast}{resultaat te zijn}\\

\haiku{Zoo nu schrijft geen mensch ({\textquotedblleft}{\textquotedblright}).}{van gezonde zinnenquis}{sanus ita scribat}\\

\haiku{Daarin vinden de;}{professoren P. en N.}{tegenstrijdigheid}\\

\haiku{Ons denken kennen;}{wij voor zoover het zich aan ons}{bewustzijn openbaart}\\

\haiku{Het levende woord,,,,.}{dat is zooals ik zeide de}{zin de gedachte}\\

\haiku{bidden, gebed, of,,,.}{om de tegenstelling het}{bidden het gebed}\\

\haiku{Dit moeilijke punt.}{is voor de uitlegging van}{het grootste gewicht}\\

\haiku{Zijn werk werd door R..}{Laqueur onlangs uitvoerig}{gerecenseerd255}\\

\haiku{Maar toch is de kunst,.}{dienstbaar zij is hulpmiddel}{voor de dialectiek}\\

\haiku{die van het denken,.}{en die van het dichten van}{wetenschap en kunst}\\

\haiku{het is mogelijk;}{dat beide in \'e\'enen mensch}{gevonden worden}\\

\haiku{de blik des geestes.}{wordt klaarder en scherper dan}{hij te voren was281}\\

\haiku{Oefening kan en,:}{moet die gave ontplooien}{leiden en sterken}\\

\haiku{zij moet zijn in de.}{uitwerking wat zij krachtens}{het beginsel is}\\

\haiku{van eenen invloed van;}{andere volken is ons}{zeer weinig bekend}\\

\haiku{Ontzaglijk breed is.}{in deze tijden de stroom}{der literatuur}\\

\haiku{De geesten zijn als,;}{eene voortgedrevene zee}{die niet kan rusten}\\

\haiku{En zoo breidt zich juist}{voor het denken de kring der}{dingen hoe langer}\\

\haiku{als zoodanig, als,.}{idee\"en zijn zij en hebben}{zij realiteit}\\

\haiku{De problemen die}{de theorie der kennis}{ons voorlegt vinden}\\

\haiku{het onzes inziens.}{abnormale kunnen wij}{er uit elimineeren}\\

\haiku{Vergunt mij nu nog.}{eene tweede opmerking hier}{aan toe te voegen}\\

\haiku{Ik ga dus bij het:}{zoeken van het antwoord op}{de gestelde vraag}\\

\haiku{Het tweede deel, de,;}{binnensfeer is zoo te zeggen}{ontoegankelijk}\\

\haiku{Het laatste is door,,.}{zijnen grondslag de H. Schrift}{voldoende bepaald}\\

\haiku{Buitendien past de;}{onderscheiding slecht voor de}{interpolaties}\\

\haiku{V. p. 463 Kirchm..,:}{322Spencer drukt zich onjuist}{uit wanneer hij zegt}\\

\haiku{'Steht denn nicht unser?}{erkennendes Bewusstsein}{mitten in der Welt}\\
