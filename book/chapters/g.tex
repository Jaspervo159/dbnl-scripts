\chapter[16 auteurs, 1627 haiku's]{zestien auteurs, zestienhonderdzevenentwintig haiku's}

\section{Bish Ganga}

\subsection{Uit: Lalbahadoer. Een fatale liefde}

\haiku{zij door toedoen van.}{haar familie niet zijn vrouw}{zou kunnen worden}\\

\haiku{Ze deed de deur van.}{de knip en duwde hem open}{met een licht gekraak}\\

\haiku{Want ze had hem wel,.}{binnengelaten maar nog}{geen woord gesproken}\\

\haiku{Alle relaties.}{waren verstoord en hoe moest}{het nu verder}\\

\haiku{De moeder werd in.}{haar voorhoofd geraakt en viel}{ter plekke dood neer}\\

\haiku{Politieagenten.}{bewaken de woning van}{Lalbahadoer}\\

\haiku{Het kostte hem veel.}{moeite de koeien in het}{gareel te houden}\\

\haiku{Ze waren bang om.}{doodgeschoten te worden}{door Lalbahadoer}\\

\haiku{In dat geval was.}{hij beter bewapend op}{patrouille gegaan}\\

\haiku{begaat, waarbij zijn.}{geliefde en haar moeder}{het leven laten}\\

\haiku{Het natrekken van.}{de tips liet naar hun gevoel}{te wensen over}\\

\haiku{Vanaf dat moment.}{werd Lalbahadoer zichtbaar}{nog onrustiger}\\

\haiku{Elke indringer.}{van het domein zou weldra}{opgemerkt worden}\\

\haiku{{\textquoteleft}Jammer dat het niet{\textquoteright}.}{tot een sprookjeshuwelijk}{is gekomen}\\

\haiku{Dit alles vond plaats.}{onder leiding en toezicht}{van de politie}\\

\haiku{Dat was voor Janssen.}{het signaal dat hij zich niet}{vergist kon hebben}\\

\haiku{Het was - zo op het -.}{eerste gezicht met respect}{in het graf gelegd}\\

\haiku{Met de valstrik is.}{een geweer met gespannen}{trekker verbonden}\\

\section{Rudolf Geel}

\subsection{Uit: Een afgezant uit niemandsland}

\haiku{Rudolf Geel Een}{afgezant uit niemandsland}{Colofon}\\

\haiku{Paul hing midscheeps over,}{de reling niet meer in staat}{te voelen hoe ziek}\\

\haiku{Zij danste voor zijn,.}{ogen hij kon haar niet recht in}{het vizier krijgen}\\

\haiku{Je doet me denken.}{aan een handelsreiziger}{die ik gekend heb}\\

\haiku{De jongen zakte.}{langzaam en statig terug}{op de achterbank}\\

\haiku{- Dat vind ik erg flink,.}{antwoordde Ellen weer met}{haar gewone stem}\\

\haiku{Terwijl de vingers,.}{hun lenig werk vervolgden}{wachtte Paul beleefd}\\

\haiku{- Ze staan allemaal,.}{tegen mij aan te zij ken}{fluisterde Asquit}\\

\haiku{Al te lang heeft de.}{ongare verveling u}{beziggehouden}\\

\haiku{Kunnen jullie door?}{de stad lopen zonder de}{beest uit te hangen}\\

\haiku{Een zwoele windvlaag.}{komt de kamer binnen en}{slaat op je darmen}\\

\haiku{Hij werd wakker, zo.}{abrupt dat hij nog even in}{zijn droom bleef steken}\\

\haiku{Toen plofte deze,.}{uit elkaar of eigenlijk}{was het dat niet eens}\\

\haiku{Buiten zijn kamer.}{liep iemand in pyjama}{hem voor de voeten}\\

\haiku{Links (als je van de).}{haven kwam waren hoge}{struiken en bomen}\\

\haiku{Ik moet kijken wie.}{van jullie het meest geschikt}{is voor commissies}\\

\haiku{Toen hij dit gezegd.}{had leek het alsof de lucht}{vol olifanten kwam}\\

\haiku{Ook onzin kon uit.}{zijn mond komen en niemand}{nam hem dat kwalijk}\\

\haiku{De ouvreuses in.}{het Luchtjeshuis stonden niet}{als prettig bekend}\\

\haiku{Dit dient om jullie.}{kennis te laten maken}{met de bevolking}\\

\haiku{Toen de man met de.}{mondharmonika klaar was}{werd hij toegejuicht}\\

\haiku{Maar het opletten.}{werd steeds moeilijker en er}{ging niets beginnen}\\

\haiku{- Wijs mij een vrouw, zei,. -.}{hij bijna plechtig met een}{hoge stem Ellen}\\

\haiku{Mijn oom droeg een groen.}{jachtkostuum om niet zo naast}{haar op te vallen}\\

\haiku{Daarna gaf hij ook.}{haar wat geld en tapte voor}{zichzelf een glas bier}\\

\haiku{Zo oud als ze wordt,,,;}{die lieve Spernip kippig}{een beetje aftands}\\

\haiku{Dit was verveling,.}{over mensen horen die in}{jouw hoofd leeg blijven}\\

\haiku{informeerde de.}{dichter met een overdreven}{vraagtoon in zijn stem}\\

\haiku{Het verwonderde.}{hem dat het onweer nog niet}{was losgebroken}\\

\haiku{- Als jij het eens wel,,.}{dacht zei Asquit steun zoekend}{tegen de deurlijst}\\

\haiku{En dan blijven ze,,.}{maar hier als opvrolijking}{desnoods met geweld}\\

\haiku{Een windvlaag sloeg in,.}{zijn gezicht tegelijk met}{wat regendruppels}\\

\haiku{De dakgoot liep over,.}{zodat een guts regen zijn}{kraag binnen plensde}\\

\haiku{Zijn hoofd kwam met een.}{klap op de grond en leek zelfs}{even te stuiteren}\\

\haiku{Zij nam zijn hand en.}{nam hem mee door het natte}{gras naar de rotsrand}\\

\haiku{Zijn voeten sopten.}{door de halmen die al lang}{niet gemaaid waren}\\

\haiku{Hij legde zijn hoofd.}{tegen haar schouder en liet}{zijn ogen dichtvallen}\\

\haiku{Hij zat op zijn bed.}{terwijl hij nauwelijks wist}{dat hij wakker was}\\

\haiku{Mensen mogen niet.}{teleurgesteld worden in}{hun aanbiedingen}\\

\haiku{Zoals Asquit had.}{geweten dat hij Senkar}{niets moest wij smaken}\\

\haiku{Ze zagen er niet.}{uit als soldaten die naar}{iemand op zoek zijn}\\

\haiku{- Dat zou ik niet zo,.}{prettig vinden hier in huis}{Jimmy zei Alissa}\\

\haiku{Ik geloof dat men.}{zich met zijn eigen woorden}{het beste uitdrukt}\\

\haiku{- Vanmorgen nog heb,.}{ik Gossep gesproken zei}{hij geheimzinnig}\\

\haiku{- We praten er niet,,.}{over zei hij voor hij het ding}{op zijn neus zette}\\

\haiku{Wij zullen trachten.}{u zo spoedig mogelijk}{te evacueren}\\

\haiku{Godverdomme, ik! -,.}{stik hier Luister zei Senkar}{een beetje vermoeid}\\

\haiku{- Kom nou verdomme,,.}{zei hij met zijn linkervoet}{op de grond stampend}\\

\haiku{Maar als jullie je.}{bek opendoen flikker ik je}{liever de straat op}\\

\haiku{Mary als je wist.}{wat de directeur op de}{wereld te koop weet}\\

\haiku{Hun voeten hadden.}{stof achtergelaten op}{de glanzende grond}\\

\haiku{De directeur ging,.}{achter een bureau zitten}{zijn rug naar de zee}\\

\haiku{Wij zouden het niet.}{tolereren wanneer hier}{een vijand vertrok}\\

\haiku{Zij renden naar zee.}{en verscholen zich bij het}{strand in struikgewas}\\

\haiku{Zij klommen tegen,.}{de heuvel op af en toe}{steentjes losschoppend}\\

\haiku{Elke meter die.}{hij nu aflegde zou hem}{minder nat maken}\\

\haiku{De bomen lagen,,.}{evenals bij de uitspanning}{te midden van gras}\\

\haiku{De drank maakte hem,,.}{warm overal scheen nu de zon}{behalve buiten}\\

\haiku{- Je hebt zes weken,.}{in mijn spoor gelopen zei}{Henri veel zachter}\\

\haiku{De deur tussen de.}{fabricagehal en de}{loods werd geopend}\\

\haiku{- Je hebt gelijk, zei,.}{hij kwaadaardig waarna hij}{op de grond spuugde}\\

\haiku{Het leek of alle.}{wolken zich in deze kring}{hadden verzameld}\\

\haiku{Het was duidelijk.}{dat hij de modder op zijn}{knie\"en niet voelde}\\

\haiku{Hij had de fles die.}{nog niemand had opgevraagd}{aan de mond en dronk}\\

\haiku{uit ieder van de.}{vletten kwamen twee witte}{mannen zonder helm}\\

\haiku{Op het ogenblik dat.}{Asquit wegvoer kon Paul zich}{juist aan boord hijsen}\\

\haiku{Zij liepen langzaam,.}{naar Asquit toe die op het}{dek was gaan zitten}\\

\haiku{Het water klotste.}{in kalme deining tegen}{de wand van het schip}\\

\haiku{- Dagen als deze,,.}{zei Gossep moeizaam terwijl}{ze Henri streelde}\\

\subsection{Uit: Al is de waarheid nog zo snel}

\haiku{Je hebt een beeld op.}{hem overgebracht waaraan hij}{nog wat houvast heeft}\\

\haiku{Ook vroeger deden.}{mensen op deze manier}{aan zelfbescherming}\\

\haiku{Hera, de vrouw van,.}{Zeus overigens beloofde}{hem macht en rijkdom}\\

\haiku{omdat mensen die {\textquoteleft}{\textquoteright}.}{doeg zeggen iets persoonlijks}{op het oog hebben}\\

\haiku{Mijn irritatie.}{wordt dus op verschillende}{manieren gewekt}\\

\haiku{Boontje vindt dit zo - -.}{grappig dat hij letterlijk}{barst van het lachen}\\

\haiku{het gaat hier niet om.}{een beloning maar om een}{welverdiende straf}\\

\haiku{En in Antwerpen {\textquoteleft}{\textquoteright}.}{heeft men het over iemandin}{het o'ken trekken}\\

\haiku{De zegswijze {\textquoteleft}een{\textquoteright}:}{nieuwsgierig Aagje zijn heeft}{vaak als toevoeging}\\

\haiku{Kostbare karaf,.}{onherstelbaar beschadigd}{amsterdam 27 jan}\\

\haiku{ze zijn dom, koppig,,,:}{gedoemd tot slavernij en}{wat het ergste is}\\

\haiku{Bovendien heb ik.}{geleerd dingen die ik niet}{begrijp te vragen}\\

\haiku{Ik sla een tijdschrift:}{open en meteen stuit ik op}{de brandende vraag}\\

\haiku{Aan haar voeten heerst,.}{een jong gevoel door die nieuw}{elastische kousen}\\

\haiku{En daar zat ik dan,.}{met die stijve trut en dat}{glas goedkope wijn}\\

\haiku{Maar geestelijke.}{pijn concentreert zich niet op}{een bepaalde plaats}\\

\subsection{Uit: De ambitie}

\haiku{Opa, vechtend tegen,.}{de slaap want hij was tien jaar}{ouder dan zijn vrouw}\\

\haiku{Die laatste zin had.}{hij voldoende op zijn zoon}{kunnen oefenen}\\

\haiku{Haar vader kwam zo.}{weinig mogelijk in het}{huis van zijn ouders}\\

\haiku{Ze waren bang dat.}{ik ophield met schoonmaken}{van hun zwijnestal}\\

\haiku{Ze hield altijd dat.}{rotsvaste vertrouwen in}{de goede afloop}\\

\haiku{Hij was er altijd,.}{vaag over geweest vond dat het}{een grote kans was}\\

\haiku{Maar toen ze het parkje,.}{binnenkwam wist zij al dat}{zij zich vergist had}\\

\haiku{Zij dacht aan Frits, die.}{met gesloten gordijnen}{bij de open haard lag}\\

\haiku{Daarna zou hij haar.}{in zijn armen nemen en}{met haar naar bed gaan}\\

\haiku{{\textquoteleft}Er kunnen nu toch,{\textquoteright}.}{geen boodschappen meer worden}{gedaan zei Nila}\\

\haiku{De discussie die.}{nu zou beginnen had zij}{al zo vaak gevoerd}\\

\haiku{Opnieuw was het een.}{onaangenaam idee alleen}{te moeten slapen}\\

\haiku{Ik wil niet langer.}{dan twee weken iets met hem}{te maken hebben}\\

\haiku{{\textquoteright} {\textquoteleft}Die heeft de MULO.}{niet eens zien staan op weg naar}{het gymnasium}\\

\haiku{Maar ik lag daar met.}{die  kinderen en zag}{ze de hele dag}\\

\haiku{Nooit had zij ook maar.}{enige preutsheid gekend ten}{opzichte van Frits}\\

\haiku{{\textquoteleft}Jullie kunnen het.}{beste met z'n twee\"en in}{het zwembad springen}\\

\haiku{{\textquoteright} Nila voelde een.}{spanning in haar buik die zij}{niet kon verdrijven}\\

\haiku{Hij sloeg zijn arm om.}{haar schouders en drukte haar}{tegen zich aan}\\

\haiku{Voor iemand die pas,.}{een dag in Spanje was zag}{hij bijzonder bruin}\\

\haiku{Wie garandeerde.}{haar dat Ronnie alleen op}{ontspanning uit was}\\

\haiku{Zijn werkelijke.}{bestaan speelde zich op een}{ander niveau af}\\

\haiku{Neuri\"end begon.}{zij aan het karwei waaraan}{zij een hekel had}\\

\haiku{Het kon niet anders.}{of dit was een signaal van}{hun generatie}\\

\haiku{Wie mis jij op dit,?}{ogenblik het meeste je man}{of je kinderen}\\

\haiku{Bij jou, wat zal ik,.}{zeggen klonk het helemaal}{niet automatisch}\\

\haiku{Ze trok de lipjes.}{uit de ijskoude blikjes}{en keerde zich om}\\

\haiku{{\textquoteleft}Het was bijna nog,{\textquoteright}.}{heel leuk geworden zei Henk}{op de achtergrond}\\

\haiku{Hij aarzelde, het.}{lege bierblikje tussen}{duim en wijsvinger}\\

\haiku{Je hebt er net zo.}{goed recht op te weten wat}{er gebeurt als ik}\\

\haiku{Zij bewoog snel heen.}{en weer tussen weerloosheid}{en irritatie}\\

\haiku{Alles kon nu met.}{haar gebeuren zonder dat}{zij eronder leed}\\

\haiku{Ben jij zo iemand.}{die het best vindt dat ze haar}{man een lui noemen}\\

\haiku{Probeerde iets te.}{zeggen in het besef dat}{het niet lukken zou}\\

\haiku{je alweer ergens.}{geweest op het moment dat}{je arriveerde}\\

\haiku{Dat was haar eigen.}{vraag die nooit door een ander}{aan haar was gesteld}\\

\haiku{Hij lag tegen haar.}{aan en warmde haar als zij}{het koud had in bed}\\

\haiku{Een noviteit van,.}{toen inmiddels afgeboekt}{van de rekening}\\

\haiku{Gaf haar nauwelijks.}{de kans die wens ook van haar}{kant uit te spreken}\\

\haiku{Hun gezamenlijk.}{leven bezat al te veel}{vanzelfsprekendheid}\\

\haiku{En toen stonden ze.}{tegelijk stil en keerden}{zich in haar richting}\\

\haiku{Daarachter was het,.}{land misschien zat hij tussen}{de bomen gehurkt}\\

\haiku{Een ogenblik later.}{wilde zij niet denken over}{verwarring of rust}\\

\haiku{Ze wilde dat hij,.}{naar haar toekwam wegging en}{weer terugkeerde}\\

\haiku{Toen Frank drie maanden,.}{oud was wilde zij op een}{morgen niet opstaan}\\

\haiku{Zij zat naast haar kind.}{en zag hem wakker worden}{en haar herkennen}\\

\haiku{Op een dag kwam haar.}{vader uit de inrichting}{waar oma verpleegd werd}\\

\haiku{Als ze een van haar,.}{buien had bracht ze niet veel}{meer dan gepiep voort}\\

\haiku{{\textquoteright} {\textquoteleft}Misschien omdat het,{\textquoteright}.}{verder altijd zo stil was}{antwoordde Nila}\\

\haiku{Aan de andere.}{kant irriteerde haar dit}{gebrek aan aandacht}\\

\haiku{Zij wilde haar hoofd.}{in Ronnie's schoot duwen en}{zo blijven liggen}\\

\haiku{Die herinnering.}{oproepen zonder de pijn}{ervan te voelen}\\

\haiku{De eerste keer dat,{\textquoteright}.}{je me erover vraagt na je}{vakantie zei Frits}\\

\haiku{Als je moet zeggen,}{wat je ervan vindt weet je}{niet welke woorden}\\

\haiku{Zij wist dat als Frits,.}{naar haar toekwam zij hem niet}{zou tegenhouden}\\

\haiku{Maar het was beter.}{voor haar dat ze hen rustig}{in Groningen liet}\\

\haiku{Je kon geen bezoek.}{verwachten of je maakte}{al ruzie met opa}\\

\haiku{Onsterfelijkheid.}{najagen door joden in}{elkaar te trappen}\\

\haiku{Je verdedigt hem,}{omdat je de rechtsorde}{wilt laten bestaan}\\

\haiku{Zij dacht erover na.}{hoe het zou zijn als Frits in}{elkaar zou klappen}\\

\haiku{Zij keerde terug.}{naar de slaapkamer en trok}{haar klerenkast open}\\

\haiku{Zij zag wel dat haar.}{omgeving pogingen deed}{haar te isoleren}\\

\haiku{Vroeg zich af of zij.}{niet beter de eerste trein}{terug kon nemen}\\

\haiku{Oog in oog met zijn.}{beul zou hij de pijn opnieuw}{voelen aanzwellen}\\

\haiku{Ze ging naar de stad.}{om haar eigen leven weer}{ter hand te nemen}\\

\haiku{wel een verzorgster,.}{maar bij nader inzien een}{gehalveerd gezin}\\

\haiku{Zij kleedde zich uit.}{met de bedoeling onder}{de douche te gaan}\\

\haiku{Terugkeren naar.}{het donzen gebied tussen}{slapen en waken}\\

\haiku{Zij was er zeker.}{van dat de fluitketel al}{op het gas stond}\\

\haiku{{\textquoteright} {\textquoteleft}Het is een kwestie,{\textquoteright}.}{van verantwoordelijkheid}{zei haar schoonmoeder}\\

\haiku{Zij moest niet te snel.}{toegeven aan eisen die}{de ander stelde}\\

\haiku{De ober keek haar aan,.}{schatte haar op de vrouw die}{vergeefs had verwacht}\\

\haiku{Ik sta hier maar in,.}{mijzelf te schreeuwen zonder}{dat iemand het merkt}\\

\haiku{En zo haalde zij.}{vakantiereizen in haar}{geheugen terug}\\

\haiku{Als je daarin klom,.}{en over de rand keek was je}{terug in de tijd}\\

\haiku{Ze kon zich alleen.}{niet herinneren hoe ze}{er gekomen was}\\

\haiku{Haar schoonmoeder kwam.}{de kamer binnen en liep}{naar de boekenkast}\\

\haiku{Niet naar school, niet erg,.}{ziek te weinig koorts om de}{dokter te roepen}\\

\haiku{Voor het eerst had hij.}{volkomen genoeg van de}{dingen die hij deed}\\

\haiku{Ze probeerde een.}{opgewekte toon in haar}{woorden te leggen}\\

\haiku{En daarvan weet ik.}{tenminste zeker dat het}{werkelijkheid is}\\

\haiku{Aan de andere.}{kant van de lijn begon een}{bel te rinkelen}\\

\haiku{Ze herinnerde,.}{zich geen woorden lieve noch}{afstandelijke}\\

\haiku{Haar dagboek was ze.}{begonnen met gedachten}{over haar grootmoeder}\\

\haiku{Die lui hebben door.}{de oorlog een soort mystiek}{over zich gekregen}\\

\haiku{Ik denk dat Van Raay.}{Melgers hoogst persoonlijk in}{zijn cel zou wurgen}\\

\haiku{Na afloop begon,.}{zij te huilen ontspannen}{en tevreden}\\

\haiku{Ze voelde zich niet,.}{overdreven zeker maar toch}{een stuk rustiger}\\

\haiku{Ze vroeg het hem en.}{hij antwoordde dat hij niet}{aan haar twijfelde}\\

\haiku{Ze stelde zich voor.}{hoe ze hier naakt met Ronnie}{door de keuken liep}\\

\haiku{En nu, dacht ze, heb}{ik een vriendin die op de}{wc zit te huilen}\\

\haiku{En hij zei dat Frits.}{misschien zijn best niet deed bij}{de verdediging}\\

\haiku{Ik vind dat je er,{\textquoteright}.}{te sjieke opvattingen}{op nahoudt zei Henk}\\

\haiku{We worden alleen.}{maar ongelukkig als we}{onszelf beheersen}\\

\haiku{Misschien heb jij wel.}{ontzettend veel zin om met}{Ronnie te neuken}\\

\haiku{Ze stak haar hand uit{\textquoteright}.}{en legde die een ogenblik}{op Frits onderarm}\\

\haiku{{\textquoteleft}Soms droom ik ervan.}{dat ik met een lief meisje}{in een hutje zit}\\

\haiku{Ik denk dat het kind.}{van het dek af zou waaien}{als ik haar meenam}\\

\haiku{Frits: ik heb het idee.}{dat mijn leven er anders}{door geworden is}\\

\haiku{Maar misschien zijn mijn.}{eigen eisen aan mensen}{wel te redelijk}\\

\haiku{Want ze voelde zich.}{ook prettig en vertrouwd in}{Ronnie's nabijheid}\\

\haiku{Ze liet toe dat hij.}{zijn hand op haar borst legde}{en haar betastte}\\

\haiku{{\textquoteleft}Ik kan begrijpen,}{dat je met een andere}{kerel iets begint}\\

\haiku{bij het vorderen.}{van de dag voelde zij de}{spanning toenemen}\\

\haiku{{\textquoteleft}Je maakt het leven,{\textquoteright}.}{zo beperkt met die nadruk}{op seks zei Ronnie}\\

\haiku{Alledrie hadden.}{ze hun eigen opvatting}{over het gebeuren}\\

\haiku{Maar ook dat had zich.}{misschien te gemakkelijk}{in haar vastgezet}\\

\haiku{Tussen Haarlem en.}{Den Haag hadden zij elkaar}{langdurig gekust}\\

\haiku{Sta ik eigenlijk?}{buiten de beslissing wat}{ik zal opschrijven}\\

\haiku{Geen lekke banden,.}{onweersbuien die ons koud}{en treurig maakten}\\

\haiku{Het ergste is dat,.}{je je niet voor kunt stellen}{wat het is dood zijn}\\

\haiku{En dan kan ik het,.}{nog even uitstellen omdat}{jullie bij me zijn}\\

\haiku{Mijn grootmoeder was,.}{een vrolijke en gastvrije}{vrouw voor de oorlog}\\

\haiku{{\textquoteright} Ronnie stopte het,.}{spiegeltje weg stond op en}{kwam naar Nila toe}\\

\haiku{Op de wastafel.}{lagen twee plastic builtjes}{gevuld met badschuim}\\

\haiku{Ze begon, op een,.}{uitgestelde manier haar}{jeugd te beleven}\\

\haiku{wat doen wij elkaar,.}{aan wat voor verwachtingen}{wek ik bij Ronnie}\\

\haiku{Het gaat er niet om,.}{dat hij dat doet maar dat hij}{mij vergeten is}\\

\haiku{Als ik een beetje.}{ziek ben en me voorstel dat}{het nooit meer overgaat}\\

\haiku{Sommigen van hen.}{hadden het kijken van de}{mensen veranderd}\\

\haiku{Maar toen de vriendin,:}{als een geslagen hond bleef}{staan zei de dame}\\

\haiku{ik praat er niet meer.}{over maar ik heb ook wel eens}{zin in wat anders}\\

\haiku{{\textquoteleft}En om eerlijk te.}{zijn weet ik niet waarom ik}{daar nu over begin}\\

\haiku{Misschien zat ze op.}{het Gare du Nord en kon}{geen trein meer krijgen}\\

\haiku{Ronnie's kennis van.}{de wereld was gelardeerd}{met kleurenfoto's}\\

\haiku{De opwindende.}{eenzaamheid wanneer zij uit}{het zolderraam keek}\\

\haiku{Maar de aankomst van.}{haar vader zou nog uren op}{zich laten wachten}\\

\haiku{Alles wat zij had.}{opgeschreven was nog maar}{een voorbereiding}\\

\haiku{En zij zou de tuin,,.}{betreden met de hoge}{donkere bomen}\\

\haiku{Hij stond op en liep,.}{naar een kastje waarvan hij}{het deurtje opende}\\

\haiku{{\textquoteright} {\textquoteleft}De slimste zul je,{\textquoteright},.}{nooit worden zei oma met iets}{spottends in haar stem}\\

\haiku{H\`e moeder je doet,{\textquoteright}.}{helemaal niet gezellig}{bouwde oma hem na}\\

\haiku{En toen ik je in,.}{mijn armen hield was ik zo}{verschrikkelijk blij}\\

\haiku{Maar ik vraag me af.}{of je wel weet waarom je}{niet meer terugwilt}\\

\haiku{En ik wist dat het,.}{niet alleen om jou was maar}{ook om je moeder}\\

\subsection{Uit: Bitter \& zoet}

\haiku{{\textquoteleft}Wel aardig, maar je.}{kon wel meteen zien dat ie}{alles beter wist}\\

\haiku{Henrietta draaide.}{zich om en legde haar kin}{op de rugleuning}\\

\haiku{Daar zat mijn moeder.}{op een keurige manier}{taart te lepelen}\\

\haiku{Ze praatte dan zacht.}{op hem in zonder dat ik}{verstond wat ze zei}\\

\haiku{Ik draaide mij een,.}{kwartslag naar Henrietta die}{strak voor zich uitkeek}\\

\haiku{{\textquoteright} {\textquoteleft}Ik zal laten zien{\textquoteright},.}{dat er hier tenminste \'e\'en}{fatsoen heeft zei pa}\\

\haiku{Weet je dat we hier?}{al kwamen toen we nog niet}{eens getrouwd waren}\\

\haiku{Toen ik terugkwam.}{zat mijn vader in een stoel}{een boek te lezen}\\

\haiku{Ik voelde me zo,.}{opgewonden jongen en}{ik was ook doodsbang}\\

\haiku{En dan zullen we.}{naar haar toegaan en zien dat}{het een ander is}\\

\haiku{Dat met je moeder{\textquoteright},.}{dat was ook vlak voor Pasen}{zei hij bedachtzaam}\\

\haiku{Toen iedereen me,,:}{toesprak op die receptie}{nou ja toen dacht ik}\\

\haiku{Hij wilde weten.}{hoe ze had gereageerd}{op zijn verhuizing}\\

\haiku{Zo reden wij met.}{ons drie\"en uren lang door de}{uitgestorven stad}\\

\haiku{Ze had altijd zo.}{de pest aan kots ruimen en}{dan ging ze maar weg}\\

\haiku{Dacht hij werkelijk?}{dat door zijn vertrek alles}{anders werd voor hem}\\

\haiku{Mijn vader stuurde,.}{de eerste fles terug met}{een knipoog naar mij}\\

\haiku{Het publiceren.}{van een artikel werd een}{obsessie voor hem}\\

\haiku{Maar eenmaal terug.}{in het mulle zand ploegden}{wij voort als vanouds}\\

\haiku{Hij zei dat als hij,.}{zijn hoofd schuin rechtop hield het}{draaien wel meeviel}\\

\haiku{En ik raakte in,.}{verwarring hoewel ze}{niet mijn hulp inriep}\\

\haiku{In werkelijkheid.}{weet zij zich misschien geen raad}{van de zenuwen}\\

\haiku{Zij tokkelde op,.}{een gitaar die een ander}{voor haar bespeelde}\\

\haiku{Zij had een grote.}{waardering voor mensen die}{iets bereikt hadden}\\

\haiku{De journalist bracht.}{haar om een uur of twee in}{zijn auto naar huis}\\

\haiku{{\textquoteleft}Ze haalde veertien.}{dagen diep adem om daarna}{de sprong te wagen}\\

\haiku{Heeft het publiek niet?}{ademloos de hele avond op}{hem zitten wachten}\\

\haiku{Die kieren zijn in.}{dit geval aangebracht in}{de werkelijkheid}\\

\haiku{Als hij me niet goed{\textquoteright},, {\textquoteleft}.}{vond zei zedan had hij me}{niet meer genomen}\\

\haiku{De filmpers had haar.}{drie keer zo luid geprezen}{als de regisseur}\\

\haiku{Ik verzekerde.}{haar dat zijzelf inderdaad}{heel goed meekon}\\

\haiku{Onder die laatsten,.}{bevond Mona zich en wel}{op de eerste rij}\\

\haiku{Toch maakte zij zich.}{al zorgen over het moment}{van uiteenspatten}\\

\haiku{Er bestonden nog.}{zoveel andere levens}{dan dat van haar}\\

\haiku{Tussen alles wat,.}{verging stonden bomen die}{ieder jaar bloeiden}\\

\haiku{Voor zijn derde film:}{kwam Preston met een geheel}{ander scenario}\\

\haiku{'s Avonds stond zij in,.}{de krant temidden van een}{aantal collega's}\\

\haiku{Een paar uur later.}{klopte ik op de deur van}{haar hotelkamer}\\

\haiku{uit de nevel komt.}{de schim van iemand uit een}{andere wereld}\\

\haiku{of ik van haar houd{\textquoteright},.}{zegt Mona Robbins in haar}{eerste artikel}\\

\subsection{Uit: Bloedmadonna}

\haiku{Zij wilde niet bij.}{de vrouwenpagina en}{nog minder bij sport}\\

\haiku{{\textquoteright} Hanna prikte een.}{stronkje bloemkool aan haar vork}{en hield het omhoog}\\

\haiku{Tegenover Smeets had.}{zij gehandeld volgens een}{vooropgezet plan}\\

\haiku{Vanaf 's morgens.}{negen uur had zij op het}{archief gezeten}\\

\haiku{Als goed marxist had.}{hij in gedachten zijn hand}{eraf gebeten}\\

\haiku{De zon zakte op;}{een gegeven moment weg}{achter de heuvels}\\

\haiku{een jaar geleden.}{had hij gedroomd dat zij naakt}{in zijn kamer stond}\\

\haiku{Haar vader wist niet,.}{hoe hij ze moest weigeren}{of tegenhouden}\\

\haiku{Hoe kwam de natuur?}{erbij zoveel ellende}{te verdubbelen}\\

\haiku{{\textquoteleft}Het zijn kwetsbare.}{en tere zielen en die}{heb ik nooit gehaat}\\

\haiku{{\textquoteright} {\textquoteleft}Het is zondag,{\textquoteright} zei,.}{Hanna en opnieuw begon}{zij te schateren}\\

\haiku{Van de huilende.}{madonna raakte hij niet}{onder de indruk}\\

\haiku{Eigenlijk vond hij,.}{het een smerig gezicht dat}{uitgelopen oog}\\

\haiku{liet, mede gezien.}{het feit dat zijn uitleg nog}{steeds over Agnes ging}\\

\haiku{{\textquoteright} {\textquoteleft}Dat die kerels naar.}{haar kijken met zo'n blik waar}{een vrouw eng van wordt}\\

\haiku{Bloedde zij voor hem,?}{om hem aan zijn gedachten}{te herinneren}\\

\haiku{Bij het openen van.}{de deur ving Hanna een glimp}{op van het beeldje}\\

\haiku{{\textquoteright} {\textquoteleft}Betekent dit dat?}{u de mogelijkheid van}{een wonder openhoudt}\\

\haiku{Ze bestonden uit.}{kalk en daaroverheen zat een}{laagje porselein}\\

\haiku{Nog even en zij had;}{alleen nog herdenkingen}{om over te schrijven}\\

\haiku{in haar hoofd begon,;}{iets te kraken de wereld}{was vol wonderen}\\

\haiku{Dit zinnetje sloeg.}{totaal niet op het stuk dat}{Hanna moest schrijven}\\

\haiku{Van een beeldje kun.}{je niet verwachten dat het}{maandverband gebruikt}\\

\haiku{En voor je het wist {\textquoteleft}{\textquoteright}.}{vervingen zeonderbuik}{nog door iets ergers}\\

\haiku{Deze gedachten,.}{die haar invielen en haar}{dagen verpestten}\\

\haiku{Op dit uur van de;}{avond was de snelweg niet meer}{bezaaid met auto's}\\

\haiku{Nuchelmans was niet.}{over de hoofdweg naar Uffel}{komen aanrijden}\\

\haiku{Waarom verwees hij?}{zo gemakkelijk naar God}{als het moeilijk was}\\

\haiku{Hij had het niet slecht,.}{gedaan was geliefd bij de}{parochianen}\\

\haiku{Zij stapte uit het.}{karretje en voelde de}{hand van haar moeder}\\

\haiku{Het bericht van de.}{moord had zich vanzelfsprekend}{razendsnel verspreid}\\

\haiku{De arts stelde het.}{tijdstip van overlijden op}{omstreeks middernacht}\\

\haiku{{\textquoteright} Streng keek hij Thieu aan,.}{maar deze voelde slechts een}{lachbui opkomen}\\

\haiku{Een ander idee was;}{de constructie van beeldjes}{met een reservoir}\\

\haiku{konden de mensen.}{de madonna hun eigen}{bloed laten huilen}\\

\haiku{Voorzichtig had hij.}{de vorige dag wat van}{het bloed afgekrabd}\\

\haiku{{\textquoteleft}Dus u gelooft ook?}{in de waarachtigheid van}{onze madonna}\\

\haiku{Hij trok een witte.}{zakdoek uit zijn broekzak en}{bette zijn voorhoofd}\\

\haiku{Ik wil dat u de.}{vastberadenheid laat zien}{van de moederkerk}\\

\haiku{Franske stond erbij,,.}{wees op haar riep luidkeels dat}{hij ging vliegeren}\\

\haiku{Luister eens,{\textquoteright} zei hij,.}{bijna fluisterend zich naar}{Hanna toebuigend}\\

\haiku{Op dezelfde toon.}{had zij kunnen zeggen dat}{ze er niets aan vond}\\

\haiku{Dan noem ik jou de.}{mooie jonge journaliste}{van het ochtendblad}\\

\haiku{Zij keerde juist naar}{mensen terug omdat zij}{wilde uitvinden}\\

\haiku{Een ander tochtje.}{had hen gebracht naar de kust}{in Noord-Frankrijk}\\

\haiku{Gemakkelijker.}{maakten zijn verdiensten het}{er voor hem niet op}\\

\haiku{Hij deed zijn best zijn.}{stem zo ironisch mogelijk}{te laten klinken}\\

\haiku{Zij bereikte het.}{doel na enige navraag en}{een kleine omweg}\\

\haiku{Van het eerste had;}{de burgemeestersvrouw zelf}{meer dan voldoende}\\

\haiku{Zij hep ernaartoe.}{en spiegelde zich in het}{heldere water}\\

\haiku{{\textquoteleft}E\'en,{\textquoteright} antwoordde zij,.}{zonder iets over de engel}{te durven zeggen}\\

\haiku{{\textquoteleft}Ik heb zojuist een,{\textquoteright}.}{theorie ontwikkeld zei}{Dantzig bedachtzaam}\\

\haiku{Erg ongesteld is,,.}{ze niet dacht Hanna met een}{lichte ergernis}\\

\haiku{{\textquoteleft}Wij zorgen er wel.}{voor dat ons madonneke}{een beetje rust krijgt}\\

\haiku{Een enkele keer.}{geeft een van die maaksels een}{zekere vreugde}\\

\haiku{Eerst bladerde zij.}{die even onverschillig door}{als de andere}\\

\haiku{In dat geval zou;}{ook het bloed van zijn dochter}{weer op gang komen}\\

\haiku{Franske maakte zijn.}{broek open en piste in de}{bloemen naast de kerk}\\

\haiku{Zij was in paniek,.}{geraakt zonder dat zij iets}{kon ondernemen}\\

\haiku{Voor de belangen.}{van de hemel wilden zij}{wel harder rennen}\\

\haiku{Of begon het te,?}{vervagen als een te kort}{gefixeerde foto}\\

\haiku{Als Zij Thieu in zijn.}{blote kont op een vrouw van}{lichte zeden zag}\\

\haiku{Met stijve knie\"en.}{stond hij op en begon op}{en neer te springen}\\

\haiku{Misschien is het zelfs.}{goed dat wij eerst als mannen}{onder elkaar zijn}\\

\haiku{{\textquoteleft}Ik zie voldoende.}{kansen om iets voor je zoon}{te betekenen}\\

\haiku{Hij omarmde de.}{jongen en zag voor zich hoe}{hij hem zou zoenen}\\

\haiku{Voor haar stond een glas,?}{oranje limonade was}{het prik of ranja}\\

\haiku{En in die wereld.}{bestond slechts een schromelijk}{gebrek aan kennis}\\

\haiku{Als het aan hen ligt.}{zullen al je plannetjes}{in duigen vallen}\\

\haiku{{\textquoteleft}Wat trekt volgens jou,{\textquoteright}.}{meer en vooral langer de}{aandacht vroeg Dantzig}\\

\haiku{Dit was nooit gebeurd,.}{die alles verterende}{woede bestond niet}\\

\haiku{die handen drukte,.}{hij tegen zijn oren tot ze}{barstten van de pijn}\\

\haiku{{\textquoteleft}Kalm, kalm toch,{\textquoteright} riep Rog,.}{hij legde een hand op het}{hoofd van het meisje}\\

\haiku{Zij glimlachten naar.}{elkaar en Wolffers zei iets}{dat zij niet verstond}\\

\haiku{Naar de opera van,.}{Veneti\"e nadat die in}{brand was gestoken}\\

\haiku{In een impuls deed.}{zij het gordijntje open en}{stapte naar binnen}\\

\haiku{{\textquoteleft}Ik werd al bang dat.}{ze je op een vuilnisbelt}{zouden afzetten}\\

\haiku{Zoek me en je zult.}{me tegenkomen als de}{tijd er rijp voor is}\\

\haiku{Zij zocht de kast na.}{en belichtte de muren}{met een zaklantaarn}\\

\subsection{Uit: Een dichteres uit Los Angeles}

\haiku{Daar stond de jonge.}{schilder in de aankomsthal}{van  een vliegveld}\\

\haiku{Henrietta draaide.}{zich om en legde haar kin}{op de rugleuning}\\

\haiku{Daar zat mijn moeder.}{op een keurige manier}{taart te lepelen}\\

\haiku{Ze praatte dan zacht.}{op hem in zonder dat ik}{verstond wat ze zei}\\

\haiku{Ik draaide mij een,.}{kwartslag naar Henrietta die}{strak voor zich uitkeek}\\

\haiku{{\textquoteright} {\textquoteleft}Ik zal laten zien,{\textquoteright}.}{dat er hier tenminste \'e\'en}{fatsoen heeft zei pa}\\

\haiku{Weet je dat we hier?}{al kwamen toen we nog niet}{eens getrouwd waren}\\

\haiku{Toen ik terugkwam.}{zat mijn vader in een stoel}{een boek te lezen}\\

\haiku{Ik voelde me zo,,.}{opgewonden jongen en}{ik was ook doodsbang}\\

\haiku{En dan zullen we.}{naar haar toe gaan en zien dat}{het een ander is}\\

\haiku{Toen iedereen me,,:}{toesprak op die receptie}{nou ja toen dacht ik}\\

\haiku{Hij wilde weten.}{hoe ze had gereageerd}{op zijn verhuizing}\\

\haiku{Zo reden wij met.}{ons drie\"en urenlang door de}{uitgestorven stad}\\

\haiku{Zij had altijd zo.}{de pest aan kots ruimen en}{dan ging ze maar weg}\\

\haiku{Dacht hij werkelijk?}{dat door zijn vertrek alles}{anders werd voor hem}\\

\haiku{Mijn vader stuurde,.}{de eerste fles terug met}{een knipoog naar mij}\\

\haiku{Het publiceren.}{van een artikel werd een}{obsessie voor hem}\\

\haiku{Maar eenmaal terug.}{in het mulle zand ploegden}{wij voort als vanouds}\\

\haiku{Hij zei dat als hij,.}{zijn hoofd schuin rechtop hield het}{draaien wel meeviel}\\

\haiku{sprekender dan de,.}{werkelijkheid hoewel dat}{bijna niet meer kon}\\

\haiku{Wuivende halmen.}{en boeren met de rode}{zakdoek om de hals}\\

\haiku{{\textquoteright} Nadat ik van de,.}{eerste schrik was bekomen}{voelde ik geen angst}\\

\haiku{Ik begreep dat ik.}{eenzaam was geweest toen ik}{als student aankwam}\\

\haiku{Maar de teksten die.}{de stilte van mij hoorde}{waren heel anders}\\

\haiku{Ik ging een dagje.}{vissen met een oom die in}{de Wormer woonde}\\

\haiku{{\textquoteleft}Wat zou een beschaafd?}{mens niet van zo'n schitterend}{pand kunnen maken}\\

\haiku{Toen hij na een uur,.}{of twee terugkwam verviel}{hij in stilzwijgen}\\

\haiku{Stel dat het echt een,,?}{incident was Hugo waar}{moest ik dan naar toe}\\

\haiku{En het kon je niet.}{schelen waarom Nicolas}{zo vaak zijn mond hield}\\

\haiku{Je krijgt nog eens tbc,}{van al dat rondhangen in}{bibliotheken}\\

\haiku{Opeens verscheen een.}{interview van hem met een}{bekende dichter}\\

\haiku{Waarom had hij nooit}{doorgekregen dat zij zich}{alleen maar inhield}\\

\haiku{Ons dochtertje had.}{nooit gedacht dat vakantie}{zo vrolijk kon zijn}\\

\haiku{Ik probeerde haar,.}{voor de geest te halen maar}{dat lukte maar half}\\

\haiku{{\textquoteleft}Als je erg nodig,{\textquoteright}, {\textquoteleft}.}{moet zei het dienstmeisjedan}{mag je hierbij mij}\\

\haiku{Tegelijkertijd,.}{wilde ik ook de tuin zien}{met zijn regendak}\\

\haiku{De tafels die er.}{stonden waren beladen}{met glazen en drank}\\

\haiku{Ik probeerde iets,.}{te zeggen maar nog steeds was}{mijn stem op de loop}\\

\haiku{De tijd heeft aan de.}{herinnering het een en}{ander veranderd}\\

\haiku{Ik legde de kaart.}{weg en ging bij een glas pils}{zitten nadenken}\\

\haiku{ik ooit kinderen,.}{zou hebben maar daar zag het}{nu even niet naar uit}\\

\haiku{Hij schrok af en toe.}{op en ging een half uur lang}{zitten luisteren}\\

\haiku{Tijdens het schrijven.}{kwamen andere dingen}{in mij naar boven}\\

\haiku{Maar het is de vraag.}{of ik me daarvan op dat}{moment bewust was}\\

\haiku{Zo belandden wij,.}{op een gegeven ogenblik}{bij het hek dat openstond}\\

\haiku{Is zoiets trouwens?}{mogelijk als het om je}{eigen leven gaat}\\

\haiku{Een paar weken per.}{maand verdiende zij wat geld}{met kleine baantjes}\\

\haiku{Als zij een vriend had.}{ging zij er van uit dat hij}{betaalde voor haar}\\

\haiku{Dan had ik tijdens.}{mijn huwelijk met Anna}{heel wat meer geschmierd}\\

\haiku{Julia, Duisterhofs,.}{vrouw bezorgt het bedrijf zijn}{bittere kantjes}\\

\haiku{Het viel mij op dat.}{zij zich goed verzorgde en}{met smaak gekleed ging}\\

\haiku{Maar  ook zelf was.}{ik allesbehalve een}{feestelijk type}\\

\haiku{Wat deed zij haar best.}{om iets te maken uit hun}{modieuze droom}\\

\haiku{Alles wat ik nooit,.}{geweten had kwam nu tot}{mij in vermoedens}\\

\haiku{Toen we elkaar pas,}{kenden wilde zij soms niet}{de deur uit verzon}\\

\haiku{Af en toe was het.}{fantastisch om haar eigen}{weg te kunnen gaan}\\

\haiku{{\textquoteright} zei Anna, nadat.}{Titia haar stoel verwisseld}{had voor het toilet}\\

\haiku{Ze waren wakker.}{geworden door het geluid}{van Anna's auto}\\

\haiku{Zij streelde Anna.}{over haar borsten en begon}{opnieuw te huilen}\\

\haiku{Ik zou er door sneeuw.}{willen lopen en het vuur}{laaiend opstoken}\\

\haiku{{\textquoteright} wilde zij weten,.}{toen hij haar voorstelde hem}{te vergezellen}\\

\haiku{{\textquoteleft}Het is toch vreemd dat.}{ik overal kan voorlezen}{en niet in Holland}\\

\haiku{So beautiful,{\textquoteright} sprak,.}{zij met de klemtoon op de}{laatste lettergreep}\\

\haiku{Aan Erwins gezicht.}{viel af te lezen dat hij}{zich niet amuseerde}\\

\haiku{Toch zou hij vroeger;}{geweigerd hebben het te}{fotograferen}\\

\subsection{Uit: Dierbaar venijn}

\haiku{Bij het volgende.}{bezoek aan zijn vader is}{er iets veranderd}\\

\haiku{Enkele van die.}{C\'ezanne-pastiches}{ontroeren mij zelfs}\\

\haiku{Bracht haar voor een paar.}{dagen naar een hotel aan}{de rand van een bos}\\

\haiku{Het is allemaal,.}{erg onbelangrijk wil hij}{daarmee uitdrukken}\\

\haiku{Uw moeder ligt er,{\textquoteright},.}{vredig bij zei hij in het}{mortuarium}\\

\haiku{Mijn moeder stond in,.}{de keukendeur haar handen}{glad van het zeepsop}\\

\haiku{Dat zij elkaar ook,.}{konden afbranden kwam niet}{in mijn moeder op}\\

\haiku{Zij kende zichzelf.}{een plaats toe in die wereld}{zonder eenzaamheid}\\

\haiku{{\textquoteleft}Stel je niet aan, Paul,{\textquoteright},.}{zei ze op haar gewone}{opgewonden toon}\\

\haiku{misschien verzetten;}{de aanstaande doden zich}{nog in hun dromen}\\

\haiku{De aanleiding tot.}{het nemen ervan verstrooid}{over de aarde}\\

\haiku{Tenzij hij iemand.}{tegenkwam met wie hij goed}{kon samenwerken}\\

\haiku{Daarna begon zij.}{te begrijpen wat hij haar}{wilde vertellen}\\

\haiku{Over eigenmachtig.}{daarbij ingrijpen hadden}{zij nooit gesproken}\\

\haiku{Bij het bed staan en.}{de arts vragen of hij de}{naald naar binnen drukt}\\

\haiku{Bud Powell speelde.}{in een kelder in de rue}{de la Huchette}\\

\haiku{Ze schopten je met.}{niet minder liefde van de}{wereld dan vandaag}\\

\haiku{{\textquoteright} schreeuwde zij, en mijn,.}{vader erachteraan op}{een sukkeldrafje}\\

\haiku{Niet lang geleden.}{had hij zijn zeventigste}{verjaardag gevierd}\\

\haiku{Een wankel evenwicht,,.}{liefje in het grensgebied}{van de waardigheid}\\

\haiku{{\textquoteright} Niet lang geleden.}{heeft hij zijn zeventigste}{verjaardag gevierd}\\

\haiku{{\textquoteright} Zijn ogen richtten zich.}{op de bessestruik opzij}{van het oprijpad}\\

\haiku{zo ver het oog reikt,,.}{mijn oog dat kan waarnemen}{zo lang er licht is}\\

\haiku{Maar wie zit op een?}{nagelaten bijdrage}{van hem te wachten}\\

\haiku{{\textquoteright} {\textquoteleft}Als je het weten,{\textquoteright}, {\textquoteleft}:}{wilt zei hijwant daar zit je}{toch naar te vissen}\\

\haiku{Wie zijn zij die 's:}{morgens uit bed stappen en}{opgewekt denken}\\

\haiku{Zij komen uit de,.}{stille witte wereld waar}{geen mensen wonen}\\

\haiku{Niet anders wil ik,.}{naar mijn leven kijken naar}{dat van mijn ouders}\\

\subsection{Uit: Dode bladeren}

\haiku{Dan kom ik aan de.}{ingang van het theater}{Th\'er\`ese tegen}\\

\haiku{zo  iemand ben.}{ik op dit ogenblik voor mijn}{eigen kinderen}\\

\haiku{{\textquoteleft}En in de hemel.}{krijgen we ongetwijfeld}{andere namen}\\

\haiku{Mijn generatie.}{benaderde de ouders}{met omzichtigheid}\\

\haiku{{\textquoteright} In die tijd was de.}{wolkenkrabber het hoogste}{gebouw van de stad}\\

\haiku{{\textquoteright} Als het zo ging, zou.}{ik van zakendoen ook wel}{niet veel begrijpen}\\

\haiku{Ik keek weer voor mij.}{en zag uit mijn ooghoeken}{dat zij glimlachte}\\

\haiku{En ik bereidde.}{mij voor op een avondje vol}{misverstanden}\\

\haiku{Je klampte je in.}{je slaap op een gegeven}{ogenblik aan mij vast}\\

\haiku{Als we dit straatje.}{uit zijn komen we op het}{Place des Vosges}\\

\haiku{Hij hield veel van je,{\textquoteright}.}{moeder meende zij ook te}{moeten opmerken}\\

\haiku{Als ik dingen zeg,.}{die je vervelend vindt moet}{je me waarschuwen}\\

\haiku{Van het ene op het.}{andere moment voel ik}{me als herboren}\\

\haiku{Beverig knoopte.}{ik de bovenste knoopjes}{van haar blouse los}\\

\subsection{Uit: Een gedoodverfde winnaar}

\haiku{Ik had me nooit  .}{bekommerd om de dingen}{die voorbij waren}\\

\haiku{Vergeten ook of.}{hij ooit thuis was toen ik nog}{niet zelf kon lezen}\\

\haiku{Of was ik het in?}{de drukte van mijn eigen}{leven vergeten}\\

\haiku{{\textquoteleft}Ik zou je kunnen,{\textquoteright}.}{helpen met dat archief zei}{Hugo even later}\\

\haiku{{\textquoteright} {\textquoteleft}Het is goedkoper,{\textquoteright}.}{om te lenen zei Hugo}{bij gelegenheid}\\

\haiku{Het plaatselijke.}{elftal betrad het veld en}{een gejuich ging op}\\

\haiku{De platen lagen.}{op een keurig stapeltje}{naast de grammofoon}\\

\haiku{In het gegeven.}{geval kwam zij niet over voor}{de begrafenis}\\

\haiku{Hem vragen hoe hij.}{reageerde op dit soort}{beledigingen}\\

\haiku{Wanneer ik nu zijn.}{stem hoorde zou ik dat ook}{moeilijk verdragen}\\

\haiku{Toen hij werkte was.}{hij houtbewerker op een}{machinefabriek}\\

\haiku{Hij was vijf jaar lang,.}{het spoor bijster bij gebrek}{aan initiatief}\\

\haiku{Kitsch bootste hij even.}{gemakkelijk na als een}{landschap van Ruysdael}\\

\haiku{Voor het eerst in zijn.}{aan mij bekende leven}{slikte hij iets weg}\\

\haiku{Op zijn bekende.}{wijze kwam hij al snel tot}{de kern van de zaak}\\

\haiku{{\textquoteleft}Ik weet niet hoe ik,{\textquoteright}, {\textquoteleft}}{het moet uitdrukken schreef mijn}{vaderen dat komt}\\

\haiku{Uiteindelijk was.}{collega Furstner hem maar}{twee jaar voor geweest}\\

\haiku{[VIII] Voordat ik.}{naar Bretagne reisde had ik}{twee ontmoetingen}\\

\haiku{{\textquoteleft}Als hij maar eens echt,{\textquoteright}.}{met me wou praten deelde}{Olga woedend mee}\\

\haiku{{\textquoteleft}Ik denk erover hem,{\textquoteright}.}{jaloers te maken met een}{ander zei Olga}\\

\haiku{Hij schudde zijn hoofd.}{in het vaste voornemen}{niet kwaad te worden}\\

\haiku{Vorrink evenwel had.}{het niet over patatkramen}{maar over danszalen}\\

\haiku{Maar het verval van.}{de achtergrond waarop zij}{steunden ging verder}\\

\haiku{{\textquoteright} {\textquoteleft}Moet je nou zien wat,{\textquoteright}.}{je gedaan hebt riep Alma}{tegen haar zoontje}\\

\haiku{{\textquoteleft}Hugo heeft zo naar,{\textquoteright}.}{je uitgezien zei ze de}{avond na mijn aankomst}\\

\haiku{Ze hadden elkaar.}{al zoveel jaren ergens}{mee bezig gezien}\\

\haiku{{\textquoteleft}Je irriteert me,{\textquoteright}.}{had Hugo de avond voor mijn}{aankomst opgemerkt}\\

\haiku{Ik nam Hugo mee.}{naar de keuken en zocht in}{de ijskast naar bier}\\

\haiku{Toen ze naast mij stond.}{legde ze haar hand op mijn}{haar en streelde het}\\

\haiku{Hij legde zijn hoofd.}{tegen haar bovenbeen en}{wilde een koekje}\\

\haiku{Ik had mijn vader.}{nooit  eerder in de weer}{gezien met een kind}\\

\haiku{Of was hij toen uit?}{angst niet toegekomen aan}{een afrekening}\\

\haiku{Hij hoefde niets meer.}{en besloot dus dat hij zich}{bezighield met mij}\\

\haiku{{\textquoteright} ~ Ik bestelde.}{nog een Calvados en keek}{op mijn horloge}\\

\haiku{Ik weet dat ik mij.}{in die droom tegen deze}{beelden verzette}\\

\haiku{Hij had waarschijnlijk.}{diezelfde wandeling met}{mijn vader gemaakt}\\

\haiku{Toen ik hun stemmen,.}{niet meer hoorde liep ik met}{Alma langs het strand}\\

\haiku{Vlak voor mijn vader.}{overleed had Hugo met mij}{over Olga gepraat}\\

\haiku{Die altijd zei dat.}{ze geen enkel probleem had}{met hun verhouding}\\

\haiku{Maar onmiddellijk.}{wordt ons duidelijk dat dit}{niet de manier is}\\

\haiku{Er was een tijd dat,{\textquoteright}.}{ik mijn moeder miste zei}{hij na enige tijd}\\

\haiku{{\textquoteleft}Ik zal je maar niet,}{zeggen dat je voor je op}{de weg moet kijken}\\

\haiku{{\textquoteleft}Als ik eraan denk!}{met wat voor belangrijke}{dingen jullie god}\\

\haiku{Het wild was van hem.}{gaan houden en niet meer weg}{te slaan van zijn spoor}\\

\haiku{Hij tilde zijn zoon.}{van de grond en drukte zijn}{hoofd tegen diens wang}\\

\haiku{Ik dacht eraan hoe.}{mijn vader op Fritsje}{had gereageerd}\\

\haiku{Hij wilde dat hij}{terug kon keren in de}{tijd om nog eenmaal}\\

\haiku{Ook wilde hij nog.}{een paar dingen beleven}{die hij gemist had}\\

\haiku{Omdat hij niet wist.}{wat hij dan tegen haar zou}{moeten zeggen}\\

\haiku{Ik stelde mij voor.}{dat Hugo al geruime}{tijd bij Olga was}\\

\haiku{Het irriteerde.}{hem dat Olga geen aandacht}{aan hem besteedde}\\

\haiku{Zij legde haar hand.}{tegen Asquits wang terwijl}{ze naar Hugo keek}\\

\haiku{Olga rukte zich.}{van hem los en gaf hem een}{klap tegen zijn wang}\\

\haiku{Daarna pakte ze.}{haar koffer op en liep de}{gang in naar de lift}\\

\haiku{Hugo  pakte.}{haar in haar nek en draaide}{haar hoofd naar zich toe}\\

\haiku{Wij liepen om het,.}{huis maakten een tochtje en}{deden boodschappen}\\

\haiku{Later op de avond.}{kwamen Alma en Hugo}{aangeschoten thuis}\\

\haiku{Toch  was het niet.}{alleen mode geweest maar}{tevens romantiek}\\

\haiku{Plotseling trad daar,.}{een ambtenaar op zijn pad}{als een soort Jezus}\\

\haiku{Maar hij kon er niet.}{onderuit te vertellen}{dat hij er wel was}\\

\subsection{Uit: Genoegens van weleer}

\haiku{sprekender dan de,.}{werkelijkheid hoewel dat}{bijna niet meer kon}\\

\haiku{Opeens begreep ik,}{dat zij mij een lafaard vond}{dat ze gehoopt had}\\

\haiku{Ze had altijd  .}{genoegen moeten nemen}{met de kruimeltjes}\\

\haiku{Zij was niet gesteld.}{op het uitwisselen van}{intimiteiten}\\

\haiku{Wuivende halmen.}{en boeren met de rode}{zakdoek om de hals}\\

\haiku{Toen bedacht ik mij {\textquoteleft}{\textquoteright}.}{dat het woordzonde niet het}{juiste was geweest}\\

\haiku{Ze dacht zeker dat.}{ik haar na afloop van de}{pret zou laten gaan}\\

\haiku{Ik begreep dat ik.}{eenzaam was geweest toen ik}{als student aankwam}\\

\haiku{Maar de teksten die.}{de stilte van mij hoorde}{waren heel anders}\\

\haiku{Ik ging een dagje.}{vissen met een oom die in}{de Wormer woonde}\\

\haiku{{\textquoteleft}Wat zou een beschaafd?}{mens niet van zo'n schitterend}{pand kunnen maken}\\

\haiku{Toen hij na een uur,.}{of twee terugkwam verviel}{hij in stilzwijgen}\\

\haiku{Stel dat het echt een,,?}{incident was Hugo waar}{moest ik dan naar toe}\\

\haiku{En het kon je niet.}{schelen waarom Nicolas}{zo vaak zijn mond hield}\\

\haiku{Je krijgt nog eens tbc,}{van al dat rondhangen in}{bibliotheken}\\

\haiku{Opeens verscheen een.}{interview van hem met een}{bekende dichter}\\

\haiku{Ik informeerde.}{of ze wel eens meeging met}{een vreemde kerel}\\

\haiku{Waarom had hij nooit}{doorgekregen dat zij zich}{alleen maar inhield}\\

\haiku{Ons dochtertje had.}{nooit gedacht dat vakantie}{zo vrolijk kon zijn}\\

\haiku{Ik probeerde haar,.}{voor de geest te halen maar}{dat lukte maar half}\\

\haiku{{\textquoteleft}Als je erg nodig,{\textquoteright}, {\textquoteleft}.}{moet zei het dienstmeisjedan}{mag je hier bij mij}\\

\haiku{Tegelijkertijd,.}{wilde ik ook de tuin zien}{met zijn regendak}\\

\haiku{De tafels die er.}{stonden waren beladen}{met glazen en drank}\\

\haiku{Ik probeerde iets,.}{te zeggen maar nog steeds was}{mijn stem op de loop}\\

\haiku{De tijd heeft aan de.}{herinnering het een en}{ander veranderd}\\

\haiku{Dat maakt natuurlijk.}{niets uit voor de waarde van}{de reconstructie}\\

\haiku{Ik legde de kaart.}{weg en ging bij een glas pils}{zitten nadenken}\\

\haiku{ik ooit kinderen,.}{zou hebben maar daar zag het}{nu even niet naar uit}\\

\haiku{Hij schrok af en toe.}{op en ging een half uur lang}{zitten luisteren}\\

\haiku{Tijdens het schrijven.}{kwamen andere dingen}{in mij naar boven}\\

\haiku{Maar het is de vraag.}{of ik me daarvan op dat}{moment bewust was}\\

\haiku{Is zoiets trouwens?}{mogelijk als het om je}{eigen leven gaat}\\

\haiku{Een paar weken per.}{maand verdiende zij wat geld}{met kleine baantjes}\\

\haiku{Zij trok haar jas aan,.}{en reed naar haar moeder die}{vlak bij haar woonde}\\

\haiku{Julia, Duisterhofs,.}{vrouw bezorgt het bedrijf zijn}{bittere kantjes}\\

\haiku{Het viel mij op dat.}{zij zich goed verzorgde en}{met smaak gekleed ging}\\

\haiku{Alles wat ik nooit,.}{geweten had kwam nu tot}{mij in vermoedens}\\

\haiku{Toen we elkaar pas,}{kenden wilde zij soms niet}{de deur uit verzon}\\

\haiku{Af en toe was het.}{fantastisch om haar eigen}{weg te kunnen gaan}\\

\haiku{{\textquoteright} zei Anna, nadat.}{Titia haar stoel verwisseld}{had voor het toilet}\\

\haiku{Ze waren wakker.}{geworden door het geluid}{van Anna's auto}\\

\haiku{Zij streelde Anna.}{over haar borsten en begon}{opnieuw te huilen}\\

\haiku{Ik zou er door sneeuw.}{willen lopen en het vuur}{laaiend opstoken}\\

\haiku{Anna zei dat ze.}{moe was en begaf zich naar}{de logeerkamer}\\

\subsection{Uit: De magere heilige}

\haiku{Zwijgend stonden ze,.}{elkaar aan te kijken zijn}{metgezel en hij}\\

\haiku{E\'en was lang en blond,,.}{de ander droeg een kort spits}{toelopend baardje}\\

\haiku{Nu is het zo dat.}{je zelfs kleine jongens niet}{met rust kunt laten}\\

\haiku{Zij kwamen niemand.}{tegen hoewel duizenden}{aanwezig moesten zijn}\\

\haiku{Maar  het meisje.}{had alles gezien en ze}{had het eten bij zich}\\

\haiku{Hij wil alleen dat.}{je naar het water kijkt en}{je kleren uittrekt}\\

\haiku{Als je mijn naam hoort}{en je ziet mij terug en}{je herkent mij niet}\\

\haiku{Niet bang zijn, ik heb -.}{je gered en ook jij De}{troep stond achter hem}\\

\haiku{Naakt en rillend in,.}{een kouder licht stond hij haar}{zo aan te gapen}\\

\haiku{{\textquoteright} lachte ze, maar hij,.}{zag niet dat ze lachte en}{zij lachte ook niet}\\

\haiku{De aanvoerder is,.}{toch altijd de lui die heeft}{zich kalm te houden}\\

\haiku{Onherkenbaar,{\textquoteright} zei.}{de jongen en keek alsof}{hem dat bedroefde}\\

\haiku{Hij keek naar haar door.}{de nog sterker wordende}{regenvlagen heen}\\

\haiku{Zij legde haar hoofd.}{even tegen zijn borst om niet}{te hoeven spreken}\\

\haiku{Maar dan volkomen, {\textquoteleft},{\textquoteright}.}{geheel en al.Ga mee naar}{de heuvel zei hij}\\

\haiku{Maar zij bleef in de,.}{abdij liep daar rond zonder}{om zich heen te zien}\\

\haiku{Dichtbij die uitgang.}{week zij plotseling naar rechts}{en volgde de muur}\\

\haiku{Maar daar, bij een luik,.}{in een half rondgebogen}{venstergat stond zij}\\

\haiku{Hij liep langzaam, met.}{zijn handen in zijn zakken}{van de abdij weg}\\

\haiku{Hij wilde het niet,,.}{geloven zo kon het niet}{zijn zo wanhopig}\\

\haiku{Paul gaf hem een hand,.}{wees hem een stoel en bood hem}{een sigaret aan}\\

\haiku{Niemand stak een hand.}{uit om hen tenminste naar}{boven te trekken}\\

\haiku{Vertel mij dat nog,{\textquoteright},.}{eens zei Rufus opeens zich}{naar haar toedraaiend}\\

\haiku{{\textquoteleft}Je bent een rothoer,.}{die geen man kan krijgen een}{vuile indringster}\\

\haiku{Alleen waar hij stond,.}{was een huis het wachthuis met}{daarnaast de kelder}\\

\haiku{Ook zij werd door de,.}{mannen gevonden vlak na}{hun verkrachtingen}\\

\haiku{Een algehele.}{gedaanteverwisseling}{had gezegevierd}\\

\haiku{Zij sliep vredig in.}{en de kerkklok luidde toen}{zij begraven werd}\\

\haiku{Zo zal ik je niet,.}{meer vinden zo zal je de}{afkomst nooit horen}\\

\haiku{Ellen die niet wil.}{weten dat jij nog kent wat}{zij heeft verloren}\\

\haiku{Het vulde zijn mond,.}{zijn tong bewoog er doorheen}{om het te vormen}\\

\haiku{Er kwamen mannen.}{die je meevoerden en niet}{durfden aanraken}\\

\haiku{Ik was net zo, en.}{daarom ben ik geworden}{zoals ik nu ben}\\

\haiku{Een meisje van op.}{zijn hoogst zestien jaar kwam uit}{de zee naar haar toe}\\

\haiku{Zij huilde en lag,.}{schokkend op haar buik met haar}{gezicht in het zand}\\

\haiku{Ik was bij hem en,.}{hij zorgde te goed  voor}{me en ik was jong}\\

\haiku{Zij is niets voor jou, -.}{en als je haar geloofd had}{Dat is de reden}\\

\haiku{{\textquoteright} Hij voelde hoe iets.}{in zijn keel schoot waardoor hij}{moeilijk kon praten}\\

\haiku{De boom helde en,.}{worstelde helde verder}{en brak langzaam af}\\

\haiku{- Want bedrogen in.}{haar poging nam de moeder}{haar dochter terug}\\

\haiku{Maar in het geheim.}{van hun verdwijning schreef zich}{de werkelijkheid}\\

\haiku{In die schijnbare}{seconde rust de hoop en}{groeit het raadsel dat}\\

\haiku{Soms voelde hij een.}{grote behoefte tussen}{mensen te zijn}\\

\haiku{De wijzer op de.}{snelheidsmeter boven het}{stuur kwam snel omhoog}\\

\haiku{Zijn metgezellen.}{letten erop dat hij niet}{in het gedrang kwam}\\

\haiku{Enkel de eenvoud:}{en enkel dat wat zij nog}{konden verkrijgen}\\

\haiku{Weggaan om in de.}{auto rustig te wachten}{op hun terugkomst}\\

\haiku{Je moet voor jezelf.}{zorgen en wij lieten het}{aan anderen over}\\

\haiku{Maar buiten het raam,.}{ontwaarde hij duisternis}{niets dan duisternis}\\

\haiku{Een grijsaard van het,.}{denken moe een idioot in}{zijn gevangenis}\\

\haiku{Het prikkeldraad was.}{tussen lange ruwhouten}{palen gespannen}\\

\haiku{Beiden keken zij,.}{recht voor zich uit wachtend op}{het eerste teken}\\

\haiku{Hij kon niet langer,.}{wankelde er naar toe en}{trok de dekens weg}\\

\haiku{Zij sloeg haar armen.}{om zijn hals en drukte zich}{vast tegen hem aan}\\

\haiku{Een ander die zich.}{in mijn plaats had gesteld of}{van wie zij dat dacht}\\

\haiku{{\textquoteleft}Je zal haar overal,.}{denken te vinden waar je}{zoekt zal je haar zien}\\

\subsection{Uit: Het moet allemaal nog even wennen}

\haiku{Toen Karel bij me.}{van de bank sprong knapten er}{meteen twee veren}\\

\haiku{Ik zag haar opeens.}{weer voor mij toen zij jong was}{en ik een kleuter}\\

\haiku{Zojuist samen lid.}{geworden voor f 3,50 van}{de McDonald's Club}\\

\haiku{{\textquoteright} riep iemand met een,.}{vrouwelijk accent wier stem}{ik niet herkende}\\

\haiku{De arts vulde met.}{vaste hand een spuit met sterk}{verdovend middel}\\

\haiku{Met mijn vader ging.}{ik naar de paardenraces}{op de boulevard}\\

\haiku{Daar is een kapper}{aan de gang geweest die zich}{ten doel heeft gesteld}\\

\haiku{De buren hebben.}{zojuist Rudolf en een hond}{langs de straat zien gaan}\\

\haiku{De onderste steen.}{zal bovenkomen bij het}{uitzoeken hiervan}\\

\haiku{Nemesis was bij.}{haar vergeleken een klein}{wurm in de luiers}\\

\haiku{Zij werd overreden,.}{terwijl zij in haar eentje}{door de natuur trok}\\

\haiku{Toen ik jong was deed.}{ik in de badkamer een}{jazztrompettist na}\\

\haiku{{\textquoteleft}En een prettige,{\textquoteright}.}{vakantie verder dan maar}{zei de saxofonist}\\

\haiku{het komende jaar.}{vooral goed te doen aan de}{gehandicapten}\\

\subsection{Uit: Ongenaakbaar}

\haiku{hij leverde het.}{absolute bewijs dat}{god nooit kon bestaan}\\

\haiku{Bovendien had hij.}{een zoon die dat allemaal}{wel voor hem uitzocht}\\

\haiku{{\textquoteright} {\textquoteleft}Lieveling, ik was.}{niet in de stemming voor de}{haute cuisine}\\

\haiku{Hij was er trouwens.}{niet aan gewend dat iets hem}{niet meteen lukte}\\

\haiku{Die zich uitkleedden.}{terwijl zij hun truttige}{kleren aanhielden}\\

\haiku{Maar bij die poging.}{vergat hij dat hij zichzelf}{niet kon ontlopen}\\

\haiku{Het speeksel droop langs.}{zijn brilleglas omlaag en}{drupte op zijn neus}\\

\haiku{Maar om die ander.}{te herkennen had hij geen}{helder zicht nodig}\\

\haiku{In die dagen vond,.}{Alice een vriend die haar met}{raad en daad bijstond}\\

\haiku{Philip zocht naar een.}{onderwerp waarover hij echt}{iets wilde zeggen}\\

\haiku{vond hij de zaak ook.}{weer niet ernstig genoeg voor}{al te veel tamtam}\\

\haiku{Hij besefte dat.}{die laatste uitspraak vooral}{iets over hemzelf zei}\\

\haiku{In zijn eigen huis.}{had hij de zaken perfect}{onder controle}\\

\haiku{Op die dagen ging}{Pietje Groenendijk gebukt}{onder een lading}\\

\haiku{Wat moest hij daarom?}{met een aanklacht tegen hem}{bij de politie}\\

\haiku{De dingen die hij.}{kon bedenken leken uit}{hem weg te zakken}\\

\haiku{Waarschijnlijk hadden.}{ze de avond met een wilde}{paring uitgeluid}\\

\haiku{En die vervolgens?}{opstond en in zijn wanhoop}{naar de bioscoop ging}\\

\haiku{{\textquoteright} {\textquoteleft}Omdat jij nog niet!}{eens een stukje deeg aan een}{haak durft te steken}\\

\haiku{Voorzichtig liet hij.}{het horloge uit zijn mouw}{op zijn schoot glijden}\\

\haiku{Die foto's wisten.}{de zorgelijke kanten}{van hun bestaan af}\\

\haiku{Dat de man van wie.}{zij had gehouden ver van}{haar af was geraakt}\\

\haiku{Hij schopte Vlasmans,.}{hoofd in elkaar terwijl hij}{hardop huilde}\\

\haiku{Het was opvallend.}{hoe prettig hij zich op dat}{moment al voelde}\\

\haiku{{\textquoteleft}Als we zeven keer.}{rond die school lopen zakt hij}{misschien in elkaar}\\

\haiku{{\textquoteright} Hij aarzelde, bleef.}{staan en keek omhoog langs de}{strenge voorgevel}\\

\haiku{{\textquoteleft}Als je een beetje.}{opschiet rijden we daarna}{vlug door naar Dinant}\\

\haiku{{\textquoteleft}Wilt u alstublieft.}{de rector bellen of hij}{mij ontvangen kan}\\

\haiku{{\textquoteright} {\textquoteleft}Daar stond niets over in,.}{zijn aantekeningen noch}{in het artikel}\\

\haiku{komaan, ik ga de.}{school van Vlasman eens met een}{bezoek vereren}\\

\haiku{{\textquoteleft}Ik vermoed dat u,}{in mij een medestander}{tegen Vlasman ziet}\\

\haiku{Zijn vader was geen.}{liefhebber geweest van een}{gevaarlijk leven}\\

\haiku{Hij wilde nog een,.}{keer met haar naar bed maar ze}{zei dat ze moe was}\\

\haiku{Ik ben bang dat je.}{van mij een verlengstuk van}{jezelf wilt maken}\\

\haiku{hij vreesde dat een.}{antwoord scherp en fluitend uit}{zijn strot zou komen}\\

\haiku{Zij kwam voor hem op.}{de grond zitten en legde}{een hand op zijn knie}\\

\haiku{Het proberen is,,.}{bedrog verwaandheid jezelf}{voor de gek houden}\\

\haiku{Vooral wanneer zijn.}{lichaam hem liet weten dat}{het haar nodig had}\\

\haiku{Ik heb er nog niet,{\textquoteright},.}{over nagedacht antwoordde}{hij niet naar waarheid}\\

\haiku{{\textquoteright} {\textquoteleft}Bij mij gaat het er,{\textquoteright}.}{niet helemaal om wat ik}{zelf wil zei Van Roon}\\

\haiku{Je vond het zeker?}{wel een buitenkansje wat}{je overkomen is}\\

\haiku{{\textquoteleft}Ik neem aan dat je.}{ervan genoten hebt het}{middelpunt te zijn}\\

\haiku{al was het alleen.}{maar omdat hij pas morgen}{hoefde optreden}\\

\haiku{Het liefst had hij de.}{verkoper een klap in zijn}{gezicht gegeven}\\

\haiku{Ik ben een van de.}{eersten in de wereld die}{iets van hem uitvoert}\\

\haiku{Ik had  ervoor.}{moeten zorgen dat ik me}{ook zo kon voelen}\\

\haiku{Zijn ouders kwamen.}{in die dagen om bij een}{auto-ongeluk}\\

\haiku{Eerst woonde hij met.}{Karen een paar maanden in}{een hotel in Aix}\\

\haiku{De bedrijfsleider.}{wilde weten van welke}{aard het bericht was}\\

\haiku{Ik was er zeker.}{van dat jij de hoorn niet van}{de haak zou nemen}\\

\haiku{De tederheid die.}{zij elkaar bewezen had}{iets vanzelfsprekends}\\

\haiku{Al in zijn laatste.}{jaar als leraar waren die}{dingen veranderd}\\

\haiku{{\textquoteright} {\textquoteleft}Stel dat ik bij jou.}{slaap en ik vertrek zonder}{je iets te zeggen}\\

\haiku{Als hij zo begon,.}{te redeneren kon hij}{nog een eind komen}\\

\haiku{Hij keek uitdagend,.}{naar zijn gastvrouw de kin een}{beetje geheven}\\

\haiku{En dat hij waar je?}{bijzit beslist over al of}{niet publiceren}\\

\haiku{{\textquoteright} {\textquoteleft}Zou je alles zelf?}{in handen willen houden}{wat je werk aangaat}\\

\haiku{Hij draaide zich naar,.}{Karen die tegen een der}{vleugels stond geleund}\\

\haiku{{\textquoteleft}Wat ons te doen staat.}{is het tegengaan van de}{vanzelfsprekendheid}\\

\haiku{Weer stond zij roerloos.}{op de plaats die zij eerder}{had ingenomen}\\

\haiku{Wanneer u zichzelf}{als een idioot beschouwt zou}{ik het niet wagen}\\

\haiku{Ik heb ze gezien,.}{in Parijs in de jaren}{twintig onder meer}\\

\haiku{Er bestaan dingen,.}{die van alle tijden zijn}{meneer Greveling}\\

\haiku{Verliest u dat niet,.}{uit het oog wanneer u voor}{uzelf op de vlucht slaat}\\

\haiku{Maar hij was te moe.}{om er op dit ogenblik diep}{in door te dringen}\\

\haiku{Bij de deur keerde.}{Alkowski zich nog een keer}{om en wenkte hem}\\

\haiku{{\textquoteright} {\textquoteleft}Op Henri na ben.}{jij de eerste aan wie hij}{dit wil laten zien}\\

\haiku{De zon stuiterde.}{een paar keer en rolde toen}{over de heuvels weg}\\

\haiku{Het voornaamste dat.}{mij aan hem bindt is dat ik}{hem moet beschermen}\\

\haiku{Zij verzekerde.}{hem dat hij zich geen zorgen}{hoefde te maken}\\

\haiku{Vervolgens rende.}{hij bijna in de richting}{van de aankomsthal}\\

\haiku{Hij probeerde zich.}{Alkowski voor te stellen}{op zijn  sterfbed}\\

\subsection{Uit: De paradijsganger}

\haiku{{\textquoteleft}Mag ik vanavond even?}{langskomen om te zien of}{alles naar wens is}\\

\haiku{{\textquoteright} Adela{\"\i}de loopt.}{met hem mee in de richting}{van het station}\\

\haiku{{\textquoteright} roept Adela{\"\i}de,.}{opeens met een verheugde}{lach op het gezicht}\\

\haiku{{\textquoteright} Johannes beseft.}{dat hij zich geen fouten meer}{kan veroorloven}\\

\haiku{Het gelach breidt zich.}{uit tot in de voorste en}{achterste coup\'es}\\

\haiku{De straal waarmee hij.}{doorgaans plast is niet meer zo}{krachtig als vroeger}\\

\haiku{Johannes knikt de.}{jongeman vriendelijk toe}{en steekt zijn hand uit}\\

\haiku{Tenslotte zou hij,.}{dichter worden zij het pas}{na het eindexamen}\\

\haiku{{\textquoteleft}Ik zou eigenlijk.}{best eens met vakantie naar}{Miami willen}\\

\haiku{{\textquoteleft}Je kunt niet surfen,{\textquoteright}.}{op de Amazone stelt zijn}{zoon spijtig vast}\\

\haiku{Een hem tegemoet.}{rijdende dame klemt haar}{handen om het stuur}\\

\haiku{Bij het nemen van.}{een lichte bocht begint zijn}{fiets te zwabberen}\\

\haiku{Alleen een bel. {\textquoteleft}En,{\textquoteright}.}{vraagt Debbie als hij later}{op de avond thuiskomt}\\

\haiku{Daarna stapt hij naakt,.}{de gang in waar een kast staat}{met zijn ondergoed}\\

\haiku{Eigenlijk zou hij.}{het short over zijn spijkerbroek}{moeten aantrekken}\\

\haiku{Dat zou prachtig zijn,.}{een flitsend schot dat als een}{streep de lijn passeert}\\

\haiku{{\textquoteleft}Wat een mafkees zeg,,{\textquoteright}.}{die ouwe hoort hij Bertus}{op de trap zeggen}\\

\haiku{{\textquoteleft}Waarbij de ster van.}{Bethlehem zal verbleken}{tot een vuurvliegje}\\

\haiku{Debbie zegt dat de,.}{cake van binnen zacht is}{met sinaasappelsmaak}\\

\haiku{{\textquoteleft}Ieder jaar wordt de.}{korting die het leven biedt}{een beetje minder}\\

\haiku{Hij gaat op tafel.}{klimmen en meedelen dat}{hij heeft gelogen}\\

\haiku{{\textquoteright} informeert hij, na.}{een korte observatie}{van de pati\"ent}\\

\haiku{{\textquoteleft}Als jullie dan niet,}{naar me willen luisteren}{vind ik het ook best.}\\

\haiku{Misschien valt hij in,.}{slaap of raakt anderszins in}{het ongerede}\\

\haiku{Johannes stapt in,.}{steekt het sleuteltje in het}{contact en toetert}\\

\haiku{Maar hij weet al te,.}{goed dat Debbie daar niet ligt}{noch iemand anders}\\

\haiku{Maar om de een of.}{andere reden heeft hij}{weinig zin in vis}\\

\haiku{En dan moet je nog.}{maar hopen dat die vrouwen}{ook de straat opgaan}\\

\haiku{\ensuremath{\star} ~ Johannes,,.}{zit met zijn vriend de dokter}{in het dorpscaf\'e}\\

\haiku{{\textquoteleft}Ik wil u een klein,{\textquoteright}.}{probleem voorleggen zegt hij}{tegen de dokter}\\

\haiku{Hoewel hij dan weer,.}{op een boekhouder lijkt zo}{keurig afgepast}\\

\haiku{In de tussentijd.}{kan de hond moeiteloos tot}{de aanval overgaan}\\

\haiku{dat ik hier niet loop.}{als een soort geleidemens}{voor bange hondjes}\\

\haiku{{\textquoteright} Het kwispelen van.}{de hond is niet langer een}{uiting van instinct}\\

\haiku{Als Johannes dicht,.}{bij haar is bukt hij zich om}{een schoen te strikken}\\

\haiku{Hij glimlacht en na.}{enige vertraging glimlacht}{zij naar hem terug}\\

\haiku{{\textquoteleft}Ik heb uw kaasjes,{\textquoteright},.}{gegeten zegt Johannes}{als een grote knul}\\

\haiku{{\textquoteright} Vroeger, toen hij zelf, {\textquoteleft}{\textquoteright}.}{studeerde waswerkstudent}{een bekend begrip}\\

\haiku{{\textquoteleft}Ik had de laatste.}{maanden erg veel mensen om}{mij heen Nadine}\\

\haiku{Verwacht zij dat hij?}{zal zeggen dat hij altijd}{naar haar verlangd heeft}\\

\haiku{Haar vervolgens ook.}{weer niet als een dolleman}{in de nek bijten}\\

\haiku{Het is alsof hij.}{zich aan iets vastklampt dat hij}{geen naam kan geven}\\

\haiku{Maar kijk, daar is het,.}{terug onverhoeds en met}{alle tederheid}\\

\haiku{Volzinnen wellen.}{in hem op als luchtbellen}{in kokend water}\\

\haiku{Waarna ik over de.}{relatie met mijn vriend zou}{moeten nadenken}\\

\haiku{Niet ver van hen, waar,.}{de stroom zich vernauwt tot een}{straaltje wacht een hond}\\

\haiku{{\textquoteleft}Maar ik moet toch eens,{\textquoteright}.}{zien waar je bent opgegroeid}{probeert Johannes}\\

\haiku{Hij weet zo gauw niets,.}{te verzinnen daarom laat}{hij de map schieten}\\

\haiku{Met zijn kloppende}{en naar aandacht en liefde}{hunkerende hand.}\\

\haiku{Zo Johannes, je?}{begrijpt zeker wel wat ik}{je kom aanzeggen}\\

\haiku{Aan de bomen hangt.}{een uitgelezen keur aan}{verboden vruchten}\\

\haiku{Johannes begint.}{te toeteren en met zijn}{lampen te seinen}\\

\haiku{Zij is veranderd.}{in een besnorde dikke}{man met een blauw hemd}\\

\haiku{{\textquoteright} Nathalie pakt een.}{badhandoek en wikkelt hem}{er helemaal in}\\

\haiku{Toch is het raar, het.}{onverwachte geschenk stemt}{hem niet  vrolijk}\\

\haiku{Ik wist niet dat je,.}{er zulke praktijken op}{nahield Johannes}\\

\haiku{{\textquoteleft}En dat ik na de.}{oorlog kon meewerken aan}{de wederopbouw}\\

\haiku{{\textquoteright} {\textquoteleft}Als kind zat ik het,.}{liefst in de box met een mooi}{boek over konijntjes}\\

\haiku{Ze plukt de zwarte.}{kater van de bank en stopt}{hem onder haar trui}\\

\haiku{Als hij het raam in,}{zijn kamer open heeft gezet}{komt Wanda binnen}\\

\haiku{Ik vond 75.000 meteen,{\textquoteright}.}{al aan de hoge kant moet}{hij toegeven}\\

\haiku{Maar op de een of.}{andere manier heeft hij}{dit niet meer nodig}\\

\haiku{In de verte, zo,.}{ver dat hij hem niet meer kan}{inhalen gaat Bas}\\

\subsection{Uit: De terugkeer van Buffalo Bill}

\haiku{Wisowsky dacht aan zijn,.}{kinderjaren aan vriendjes}{en aan schatgraven}\\

\haiku{Het leek of zijn brein.}{zich verzette tegen een}{uitspraak over de kaart}\\

\haiku{Bierflessen zorgden.}{voor de opvulling van open}{plekken op de grond}\\

\haiku{Hij liep regelrecht.}{op de antiquair af en}{sloeg hem om zijn oren}\\

\haiku{De kleinzoon van de:}{vervalser Costello}{vouwde een krant open}\\

\haiku{Hij vroeg zich af in}{hoeverre Marylou een}{idee had wat voor werk}\\

\haiku{Lorenzo opent zijn.}{ogen die niet meer dan een vlek}{kunnen waarnemen}\\

\haiku{ik word hoe meer ik.}{alleen nog kan lachen om}{Laurel en Hardy}\\

\haiku{Eerdaags zag hij die.}{meid misschien terug in een}{iets grotere rol}\\

\haiku{Geheimzinnige.}{tekens vielen er voor hem}{niet te ontdekken}\\

\haiku{we hebben twee uur.}{zitten kijken en alles}{is voor niets geweest}\\

\haiku{- Hij zou niet op mij,.}{lijken maar op u. Als u}{in hem geloofde}\\

\haiku{Op z'n hoogst lijkt het.}{of de steenmassa's nog wat}{zijn toegenomen}\\

\haiku{Nu vergaap ik mij.}{aan de uitgestrektheid van}{deze ru{\"\i}nes}\\

\haiku{Hij zat in bus 118.}{en sprak deze keer met een}{Amerikaans accent}\\

\haiku{Dat laatste is in.}{\'e\'en zin de inhoud van een}{klein miljoen preken}\\

\haiku{Het is duidelijk.}{dat iemand daarmee over het}{water kon skie\"en}\\

\haiku{Ik zeg niet dat ik,.}{er nu naar verlang maar ik}{ben er wel aan toe}\\

\haiku{Pompeji is nog,.}{beklemmender daar sloeg de}{dood plotseling toe}\\

\haiku{Dit alles had bij:}{nader inzien iets weg van}{een reclamespot}\\

\haiku{Amerika, dat is,?}{typisch een land voor zusjes}{vind je ook niet}\\

\haiku{En je onderwijst.}{studenten hoe ze in de}{grond moeten graven}\\

\haiku{Sauzen druipen uit.}{kant-en-klaarflessen over}{de rozestruiken}\\

\haiku{Jouw publiek beleeft.}{in ieder geval nog wat}{plezier aan de schoft}\\

\haiku{En dat geval met.}{die wielrenner heeft je dat}{duidelijk gemaakt}\\

\haiku{Hun prestatie is.}{al vergeten wanneer hij}{nog op de pers ligt}\\

\haiku{Die poppen staan er,.}{misschien nog ik ben daar in}{geen jaren geweest}\\

\haiku{Meestal hadden.}{ze een van die twee wel op}{hun kamer hangen}\\

\haiku{Als hij zijn kop in,.}{een t.v.-studio laat zien}{wordt hij gelynchd}\\

\haiku{Hoe ouder ik word,.}{Ronald hoe minder ik zou}{moeten aandringen}\\

\haiku{Een boek waarin je.}{geschreven moet hebbenom}{erbij te horen}\\

\haiku{Het zal een boek zijn.}{over de studenten en het}{omslag maak ik wit}\\

\haiku{Dit is Amsterdam{\textquoteright},.}{zei de reporter op de}{andere zender}\\

\haiku{Nu konden alleen.}{de mensen die zijknat wilden}{worden naar buiten}\\

\haiku{Enige tijd later.}{belde de burgemeester}{voor de tweede keer}\\

\haiku{- Vroeger was ik een,,.}{nobody een nitwit riep}{de medewerker}\\

\haiku{- Is heden op de,.}{Dam ongesteld geworden}{antwoordde Bertie}\\

\haiku{Ik wil vrede op.}{aarde en in de mensen}{een welbehagen}\\

\haiku{Daarin ligt de zin:}{van al die zogenaamde}{gelijkvormigheid}\\

\haiku{Zo waren ze weer.}{met het verleden in een}{onderonsje}\\

\subsection{Uit: Verleidingen}

\haiku{{\textquoteleft}Ik herinner me,{\textquoteright}.}{die ene kelner zei opa op}{zijn zeventigste}\\

\haiku{hij figuren die.}{hem doen achterblijven en}{voor zich uit staren}\\

\haiku{Jij hebt meer aan een,{\textquoteright}.}{stok zonder punt voegde mijn}{moeder eraan toe}\\

\haiku{Mijn vader hief zijn.}{staf en wees ermee in de}{richting van het raam}\\

\haiku{Oma vond het maar niks,{\textquoteright}.}{dat ze Frans praatten zei mijn}{vader op een keer}\\

\haiku{De caf\'es stonden.}{vol koffiemachines en}{fruitautomaten}\\

\haiku{Onmiddellijk dacht.}{ik aan mijn vader en zijn}{gepunte bergstok}\\

\haiku{Aan de andere.}{kant van de draaideur scheen een}{waterige zon}\\

\haiku{Het verbaasde hem.}{dat zij onmiddellijk in}{het boek verdiept was}\\

\haiku{Terwijl hij naar haar,.}{staarde keek zij even van het}{boek op en zag hem}\\

\haiku{Opnieuw voelde hij.}{de zachte aandrang van het}{mensje achter hem}\\

\haiku{Hij wist niet precies,.}{wie het was maar die dode}{daar hoorde bij hem}\\

\haiku{De directie was.}{van mening dat het hem aan}{niets mocht ontbreken}\\

\haiku{Misschien begaf hij.}{zich vaker onopgemerkt}{onder de mensen}\\

\haiku{Verzoeken om een,.}{optreden voor een vader}{die op sterven ligt}\\

\haiku{Alsof Dauzenberg.}{bij een sterfbed grappen zou}{kunnen verkopen}\\

\haiku{{\textquoteleft}Maar dan door iemand.}{die deskundig is op het}{gebied van dromen}\\

\haiku{Hij trok zich in zijn.}{studeerkamer terug en}{ik bleef bij Suze}\\

\haiku{Zij las mijn grappen,.}{lachte erom en vulde}{mij af en toe aan}\\

\haiku{Eerst leek het of hij.}{een herhaling bracht van zijn}{oude successen}\\

\haiku{Zij kreeg zoveel geld.}{in handen en toch was zij}{nog altijd een kind}\\

\haiku{Door de luidspreker.}{hoorden wij iedere avond}{dezelfde grappen}\\

\haiku{Ogenschijnlijk is dat.}{de slechtste plaats die ik had}{kunnen bedenken}\\

\haiku{Is dit niet het punt?}{waar overdrijving het wint van}{de redelijkheid}\\

\haiku{En ik voorvoelde.}{dat ik mij deze keer niet}{bekocht zou voelen}\\

\haiku{Deze keer is hij.}{hier voor de eerste keer niet}{alleen gekomen}\\

\haiku{Misschien zijn wij bang,}{om ons te binden vinden}{wij het allebei}\\

\haiku{Klachten bleven uit,.}{omdat niemand nog hier zijn}{vakantie doorbracht}\\

\haiku{Er was iets in haar.}{manier van lopen dat hem}{biologeerde}\\

\haiku{Het besef dat zij.}{niets met hem te maken had}{vergrootte zijn pijn}\\

\haiku{{\textquoteright} Zij schepte water.}{in haar hand en spatte het}{tegen mijn lichaam}\\

\haiku{Toen mijn vader het,.}{bootje kocht zag het er zo}{fris geschilderd uit}\\

\haiku{Dat je kunt denken.}{dat je tijd voldoende hebt}{om te verslapen}\\

\haiku{De vorige avond.}{had ik de inhoud voor het}{laatst gecontroleerd}\\

\haiku{Meneer Simon zat.}{op zijn knie\"en voor me en}{wreef over mijn wangen}\\

\haiku{En zoals zij zich,!}{moest wassen omgeven door}{andere vrouwen}\\

\haiku{toen ik zag dat dit,.}{niet lukte besloot ik de}{kamer uit te gaan}\\

\haiku{nu was ik het die.}{het ene miniflesje na het}{andere leegdronk}\\

\haiku{Vanavond mag je er,{\textquoteright}.}{net zo lang over doen als je}{wilt zei mijn vader}\\

\haiku{Ik keek door de ruit.}{naar de boulevard en zag}{daarin mijn moeder}\\

\subsection{Uit: Verliefdheid is een raar gevoel}

\haiku{Die Madock moet dus.}{weer een ander werk van de}{schrijver Willem zijn}\\

\haiku{Toch komt er bij dit:}{alles nog \'e\'en ding dat wij}{niet kunnen voorzien}\\

\haiku{Het jongetje dat,.}{deze tekst eens opschreef heet}{toevallig Jasper}\\

\haiku{Na die eerste zin.}{komt er iets dat de lezer}{aan het lachen maakt}\\

\haiku{Als je die koning,.}{dan toch Joop noemt streef je een}{bepaald effect na}\\

\haiku{Dus hij gebruikt ook,,.}{zijn vrouw en zijn moeder en}{misschien de slager}\\

\haiku{Iedere schrijver.}{heeft in zijn werk goede en}{slechte koningen}\\

\haiku{Intussen zag ik.}{heel wat mannen elke keer}{weer naar haar kijken}\\

\haiku{{\textquoteright} Bij besprekingen:}{in kranten en weekbladen}{zien wij hetzelfde}\\

\haiku{er zijn lezers die.}{doodgewoon niet leuk vinden}{wat zij schrijven}\\

\haiku{En misschien is het}{aardig om eerst eens over die}{vraag na te denken}\\

\haiku{Zat die processie,.}{er eenmaal op dan was het}{feesten geblazen}\\

\haiku{Die omtrekkende.}{bewegingen verhogen}{de natuurlijkheid}\\

\haiku{Die kwam al snel en.}{kon deze dief nog net op}{tijd arresteren}\\

\haiku{De man deed nog een.}{stap in de richting van de}{boom waar Peter stond}\\

\haiku{een boek mag nooit zo.}{zijn dat je erbij in slaap}{valt van verveling}\\

\haiku{Schrijvers van series.}{als Arendsoog beheersen hun}{vak helemaal niet}\\

\haiku{Dat leidt om zo te.}{zeggen een beetje af van}{de slordige stijl}\\

\haiku{) is een omheining,.}{of muur later ook gebruikt}{voor hek of slagboom}\\

\haiku{Een ouwe kluiver{\textquoteright},.}{is dus een oude snoeper}{een oude geilaard}\\

\haiku{Alleen de laatste,,.}{sc\`ene waarin Cyrano}{sterft is heel verstild}\\

\haiku{Mijn ontroering bij.}{de laatste sc\`ene zei wat}{dat betreft genoeg}\\

\haiku{In de eerste plaats.}{is daar de vraag of een tekst}{leesbaar is gedrukt}\\

\haiku{Trouwens, toen ik het,.}{boek zelf weer eens inkeek vond}{ik er niet veel aan}\\

\haiku{{\textquoteleft}Het was een leuk boek.}{dat over een jongetje ging}{die naar New York ging}\\

\haiku{Want zijn vader en.}{moeder kwamen in New York}{trug van vakantie}\\

\haiku{{\textquoteright} Of bij de dame,:}{met de slangen die immuun}{is geraakt voor gif}\\

\haiku{{\textquoteright} De man had dan aan}{die woorden genoeg gehad}{om te begrijpen}\\

\haiku{Dan had hij via die.}{dialoog de emotie van de}{spreker overgebracht}\\

\subsection{Uit: Een vrouw als een gedicht}

\haiku{Maar het gaat hier niet.}{om racefietsen maar om}{die nauwkeurigheid}\\

\haiku{in die kringen zag.}{men de beeldenstorm toch meer}{als een dieptepunt}\\

\haiku{{\textquoteleft}Als 't niet op de,{\textquoteright}:}{dominee drupt dan drupt het}{wel op de koster}\\

\haiku{Bij de koekoek als.}{duivel blijft het overigens}{ook een linke zaak}\\

\haiku{Persoonlijk wil ik.}{maar met heel weinig mensen}{iets tot stand brengen}\\

\haiku{Wij bieden deze,.}{aan via een formule een}{beleefdheidsfrase}\\

\haiku{{\textquoteright} Dit kun je zeggen.}{tegen iemand die altijd}{maar jammert en zeurt}\\

\haiku{{\textquoteright}        Opstand van de.}{Kamper uien De mensen}{in Kampen zijn kwaad}\\

\haiku{De volzin van Luns.}{die we hier ontleden is}{ook heel welsprekend}\\

\haiku{{\textquoteleft}tortre le cou \`a{\textquoteright} (:}{une bouteillede hals van}{een fles afwringen}\\

\haiku{Je had er geen flauw.}{benul van dat je bijna}{had moeten dansen}\\

\haiku{Het antwoord daarop.}{ligt voor de hand.             Een vrouw}{als een gedicht 1}\\

\subsection{Uit: De weerspannige naaktschrijver}

\haiku{Rudolf Geel De}{weerspannige naaktschrijver}{Colofon}\\

\haiku{in je hoofd draag je.}{dat fraaie boek op een zijden}{kussen met je mee}\\

\haiku{duivelse driften.}{werpen hun venijnige}{schaduwen vooruit}\\

\haiku{Een heimelijke,.)}{sigaret wij rookten geen}{van allen openlijk}\\

\haiku{, verdomde strakke -.}{rotonderbroek ik kom hier}{nooit op tijd vandaan}\\

\haiku{Misschien zullen wij (.}{aan het eind van de tochtIk}{moet nodig pissen}\\

\haiku{Ik leg mijn arm om.}{Ansje's schouder en probeer}{haar aan te kijken}\\

\haiku{Het oude wijfje,;}{klemt zich aan mij vast haar mond}{tegen mijn knie\"en}\\

\haiku{Een pijltje zeilt door.}{de lucht maar valt nog v\'o\'or de}{schutting op de grond}\\

\haiku{Een enkele dringt.}{met een dof geluid in het}{hout van de schutting}\\

\haiku{In de sluis zien wij.}{de bliksem als wij achter}{elkaar voortrennen}\\

\haiku{Ik heb mijzelf tot,}{gevangene gemaakt ik}{moet hier wegkomen}\\

\haiku{Het regent overal,.}{het is voortdurend herfst op}{de oude kopie}\\

\haiku{De verdovende.}{luchtdruk doet de vijanden}{tegen de grond slaan}\\

\haiku{En steeds keren wij,,.}{terug draaien om hen heen}{wij kennen geen angst}\\

\haiku{Ik moet gaan zitten,.}{wegzinken in een chaos}{of mij omdraaien}\\

\haiku{Meteen springt hij op.}{mij af en duwt mijn hoofd met}{kracht tegen de grond}\\

\haiku{En ik - wil zeggen,.}{dat wij samen over de hei}{moeten wandelen}\\

\haiku{Wat zal ik doen, ik.}{zet mijn pen op papier en}{begin te schrijven}\\

\haiku{Maar ik bedenk, dat,:}{daar geen kans op is laat}{ik niet nadenken}\\

\haiku{Ik klim het stenen.}{trapje op naar het platform}{voor de keukendeur}\\

\haiku{- Maar hij wil ons niet.}{vertellen waarom hij in}{mijn tuin wandelde}\\

\haiku{Deze eigenschap.}{begrijpen wij in hoge}{mate van elkaar}\\

\haiku{Op Zondagsschool het.}{nummer met de dikke en}{de dunne boekjes}\\

\haiku{En wanneer valt de?}{verwezenlijking van de}{wens schrijver te zijn}\\

\haiku{- Neem die rommel maar,,,,.}{mee naar huis zegt ze Johan Johan}{je bent een warhoofd}\\

\haiku{- Dan drinken we straks,.}{een slokje om het weg te}{spoelen zegt Elize}\\

\haiku{- Alle begin is,,.}{moeilijk zeg ik hees terwijl}{de hoop mij ontzinkt}\\

\haiku{Anneke is de,.}{eerste om de beurt gaan de}{meisjes naar de wc}\\

\haiku{Opnieuw beginnen,}{wij te lachen wij slaan de}{deken van ons af}\\

\haiku{Weet U nog, Pieter?}{die uitgleed met een cognacfles}{in zijn achterzak}\\

\haiku{Als je natuurlijk,.}{met een eigen meisje gaat}{ben ik \'e\'en teveel}\\

\haiku{Als het vervelend,.}{is kunnen we nog naar de}{tweede voorstelling}\\

\haiku{Dat is niet zo heel,.}{bijzonder want ik schreef er}{iedere dag \'e\'en}\\

\haiku{Je moet me echter.}{nooit vragen of ik het voor}{je wil opzoeken}\\

\haiku{- Nee hoor, zegt Elize,.}{maar mijn verloofde moest naar}{een begrafenis}\\

\haiku{Ik pak het glas op,.}{ledig het in \'e\'en teug en}{let scherp op de deur}\\

\haiku{Zij zorgen dat de.}{mensheid hem kennen leert via}{kranten en folders}\\

\haiku{- Luister je nu, vraagt,.}{de jongen verlegen het}{blad verfrommelend}\\

\haiku{De dichter buigt zich.}{over mij heen en verzoekt mij}{te blijven zitten}\\

\haiku{- Merkwaardig toeval,.}{mompelt de rektor die over}{haar schouder meeleest}\\

\haiku{Het is vertekend.}{en op de nieuwe vorm ben}{je niet ingesteld}\\

\haiku{Kevelkin kent mijn,.}{verrichtingen van horen}{zeggen uit haar krant}\\

\haiku{Hij loopt naar zijn plaats ().}{voor de toonbanknadat hij}{zich geordend heeft}\\

\haiku{Niet weg te slaan van.}{de stierenvechters in de}{glimmende pakjes}\\

\haiku{ze is geen maagd meer,,?}{waar bemoei je je mee hoe}{kan ik dat weten}\\

\haiku{Voorlopig vertel,,;}{ik van de hak op de tak}{zoals ik nooit doe}\\

\haiku{En Albert draait zich,,.}{ach verdomme het kan me}{geen flikker schelen}\\

\haiku{Ik heb slaap, de week.}{loopt ten einde en ik ben}{nog steeds oververmoeid}\\

\haiku{hier is Matti, wij.}{mogen hem geen moment aan}{zichzelf overlaten}\\

\haiku{Als ik terugkom,,!}{in Peru dan staan ze daar}{weer ze zijn overal}\\

\haiku{Ik ga alleen dood,,.}{net als jullie maar het is}{niet fijn met een mes}\\

\haiku{Het is misschien het,.}{laatste dat ik hoor maar ik}{zal dit meenemen}\\

\section{Reinier van Genderen Stort}

\subsection{Uit: Hinne Rode}

\haiku{Soms brandde de zon,.}{op haar wangen ondanks het}{groene bladerdak}\\

\haiku{Zij was nu vijftien.}{jaar en geleek haar vader}{meer dan haar moeder}\\

\haiku{Maar ook den harden,,}{vader had Hinne lief met}{een liefde dieper}\\

\haiku{de slag, waarmede,.}{zij dichtviel weergalmde in}{de stilte der gracht}\\

\haiku{- Ik weet nog niet of,...}{ik dan zal kunnen ik zal}{je nog berichten}\\

\haiku{- Ik voel er alles,...}{voor weer eenige dagen naar}{Gelderland te gaan}\\

\subsection{Uit: Kleine Inez}

\haiku{Zijn staatkundige;}{meerderheid werd door zijn felste}{vijanden erkend}\\

\haiku{Maar na eenigen tijd;}{hervatte de jonge man}{zijn zwervend leven}\\

\haiku{Een groote haviksneus,,.}{aan het einde lichtelijk}{afwijkend dreigde}\\

\haiku{Hij hoorde gejaagd,.}{loopen onderdrukt praten}{en slaan van deuren}\\

\haiku{vormde zich weer als,.}{op de vroolijke school die}{hij verlaten had}\\

\haiku{Zijn vader kwam over,.}{onderzocht hem langen tijd}{met den huisdokter}\\

\haiku{Dan werd een keulsche.}{aak door een kleine stoomboot}{stroomopwaarts gesleept}\\

\haiku{zoo bleef het licht ook.}{van den hellen zomerdag}{gedempt en rustig}\\

\haiku{zij zat bij hem en.}{liet spelenderwijs heur haar}{glanzen in de zon}\\

\haiku{- Ja oom, erkende,,...}{zij verward en kleurend ik}{zal het niet meer doen}\\

\haiku{dan vergat hij te.}{luisteren naar hetgeen de}{touwslagers zeiden}\\

\haiku{Een winterstorm had;}{een noorschen driemaster op}{het strand geworpen}\\

\haiku{Dan schoof langzaam en.}{sissend een locomotief}{achter hem voorbij}\\

\haiku{Hij stond op na een,'.}{poos slenterde terug tot}{kleine Inez huis}\\

\haiku{het scheen, als restte.}{haar geen kracht meer voor opstand}{en verbittering}\\

\haiku{Het sneeuwde en vaal.}{stroomde de rivier tusschen}{de witte oevers}\\

\haiku{Na dit gesprek werd.}{hun omgang weer gelijk hij}{vroeger was geweest}\\

\haiku{Peter las dezen,,.}{brief staande bij het raam een}{weinig gebogen}\\

\haiku{Een stoomschip, heel klein,.}{trok een matte rookstreep}{aan den einder}\\

\section{Caesar Gezelle}

\subsection{Uit: Uit het leven der dieren}

\haiku{liggen kijken door}{een grooten kijker naar den}{hemel en sterren}\\

\haiku{evengauw waren ze '}{we\^er open en ze wendden in}{hunne holten bij}\\

\haiku{Menig stonden daar ':}{de boomen int gelid}{op schuinsche reken}\\

\haiku{'Nen schoonen stillen:}{avond was er een van de twee}{buitengebleven}\\

\haiku{Sam keek ze na en,.}{wachtte wel indachtig wat}{er gebeuren moest}\\

\haiku{hij volgde haar op,:}{den voet al hijgen overal}{in en overal uit}\\

\haiku{met den hongerdood}{in zijn ingewand zat hij}{erop te bijten}\\

\haiku{ze trok haar we\^er in;}{en bleef zitten boven den}{trap voor een verbei}\\

\haiku{deed de goorpuid we\^er, '}{met eene zware basstem en}{ze dedent hem}\\

\haiku{Geschorst was ineens,;}{de male en geen puiden}{waren meer te zien}\\

\haiku{Zoo gauw ze u in,.}{de ooge krijgen  gapen}{ze lijk een afgrond}\\

\haiku{Maar we moeten van,}{den nood eene deugd maken en}{alles nemen zoo}\\

\haiku{Al krinkelen met,}{kop en eers verdween hij en}{in twee drie slokken}\\

\haiku{Ten langen, ten heel.}{langen laatste was hij het}{toch beu geworden}\\

\haiku{de grootste was hij,,;}{zoo daar geen meerdere en}{kwam de felste zanger}\\

\section{Josephine Giese}

\subsection{Uit: Van een droom}

\haiku{{\textquoteright} {\textquoteleft}Indien u voor mij,.}{gezongen hebt zal ik u}{zeggen wie u ben}\\

\haiku{Als in een droom, ging,:}{zij naar de piano en}{hoorde achter zich}\\

\haiku{daar gaat het leven,.}{om als gegrepen  in}{een reuzig vliegwiel}\\

\haiku{Eentonig als het:}{luiden eener doodsklok gaat}{door haar ziel de klacht}\\

\haiku{Wie is de man, die?}{de volkomen hoogheid vat}{van zulk een geven}\\

\haiku{Zij scheidde met groote,,.}{smart want zij wist dat na dit}{heengaan geen keeren was}\\

\haiku{Hoog aan den hemel,,,.}{stond aan wolkige lucht in}{rosse krans de maan}\\

\haiku{Het scheen te dalen,,.}{van de trap die wegvluchtte}{hoog in het donker}\\

\haiku{Maar telkens keerde.}{zij tot het lezen weer en}{tot het overdenken}\\

\haiku{Zij had in beide,,.}{stadi\"en als het ware}{hem voorbijgedroomd}\\

\haiku{Slechts liefde had haar - -.}{kunnen binden doch die was}{dood dus was zij vrij}\\

\haiku{Uit de brieven van.}{Ren\'e begon te spreken}{een haat aan het tooneel}\\

\haiku{Zou die wereld nu,:}{ook de oogen niet eens opengaan}{zou zij niet vragen}\\

\haiku{Toen hief zij als een,.}{bloem naar het licht haar gelaat}{naar het zijne op}\\

\haiku{Hij klemde haar in,,.}{zijn armen zoende haar op}{wangen mond en oogen}\\

\haiku{dit was de stem der,.}{maatschappij die sprak met de}{stem harer moeder}\\

\haiku{Zij nam zijn gezicht,;}{tusschen haar beide handen}{en keek hem aan lang}\\

\haiku{Later ging Richard.}{tot zijn vrienden en bleef het}{beneden doodstil}\\

\haiku{'t Was als de lang.}{verkropte behoefte van}{een gevangen geest}\\

\haiku{Reeds lang had Richard.}{stil gezonnen op iets dat}{uitkomst brengen zou}\\

\haiku{Want Richard had haar -, - {\textquoteleft}.}{lief ja dat begreep zij nu}{opzijne wijze}\\

\haiku{Ada nam haar hoed af,.}{legde het donkere hoofd}{stil aan zijn schouder}\\

\haiku{Doch op den bodem.}{van dien beker lag voor haar}{de alsemdruppel}\\

\haiku{wat was zij dan in,?}{wezen anders al werd zij}{naar de wet zijn vrouw}\\

\haiku{het stelde haar voor.}{een martelenden twijfel}{omtrent de zijne}\\

\haiku{{\textquoteleft}O, Richard, Richard,?}{waarom doe je jezelf en}{mij dit alles aan}\\

\haiku{een geruisch als.}{sloegen over de hoofden der}{menschen groote vleugels}\\

\haiku{Wanneer zij weten -,}{wil dan moet hij spreken heeft}{hij gesproken zoo}\\

\haiku{Vielleicht daas Unheil.}{dich erwartet Wird aller}{Welt es offen kund}\\

\haiku{Haar liefde moet hem,.}{veel vergelden haar liefde}{moet hem alles zijn}\\

\haiku{Hoe kon zij op den,?}{duur hem boeien als hij zich}{tot haar nederboog}\\

\haiku{Liefde en heimwee,.}{lag daar in haar stem maar niet}{beheerscht door kunst}\\

\haiku{Niet meer zichzelve,,.}{en niet de andere die}{zij wezen wilde}\\

\haiku{Eu nog altijd had,.}{zij hem niet gezien naar wien}{haar ziele smachtte}\\

\haiku{dat zou haar toch een}{feestelijk aanzien geven}{om te huldigen}\\

\haiku{Wat deerde het haar,.}{of hij haar al niet liefhad}{zooals zij had gewenscht}\\

\haiku{Hoe dikwijls had zij....}{zich in haar verbeelding aan}{zijn hals geworpen}\\

\haiku{Daar was de kloof, de.}{duistere spelonk waar zij}{haar licht ging blusschen}\\

\haiku{O, kon zij aan een.}{ieder zeggen hoe zalig}{het was te sterven}\\

\haiku{Langzaam - voet bij voet -,.}{veldwinnend op den zwaren}{den wijkenden nacht}\\

\haiku{En ook dat ging zich,,...}{vervullen wel niet precies}{zooals zij had gedacht}\\

\haiku{aan het eind, bij de,;}{breedoploopende trap}{stond een schuifraam open}\\

\haiku{die op die kleine....}{aardestip uw arm klein}{menschenkind nog ziet}\\

\haiku{Wanneer Elisabeth,}{dien blik ontmoette dan was}{het haar als zengde}\\

\haiku{{\textquoteright} vroeg dringend Ren\'e,.}{die voor niets oogen had dan voor}{dat beeldschoon wezen}\\

\haiku{Die kinderen hier,,!}{hoe lief waren zij voor haar}{hoe aanhankelijk}\\

\haiku{En toch ook nu was,.}{er in haar hart iets dat naar}{verwachten zweemde}\\

\haiku{wenn ich zu w\"unschen,}{wagte hoffen w\"urd ich auch}{zugleich wenn ich nicht}\\

\haiku{Maar daar hoorde zij,;}{haastige voeten die van}{beneden kwamen}\\

\haiku{Nog eenmaal wendde;}{zij haar lief gelaat omlaag}{en lachte hem toe}\\

\haiku{- Waren zij dan nu,?}{niet van elkander zij van}{hem en hij van haar}\\

\haiku{Zij zette haar hoed,.}{op hing een mantel om en}{begaf zich op straat}\\

\haiku{Sprakeloos leidde,.}{Lord Annavon Elisabeth}{den weg naar het Bosch}\\

\haiku{Wij zouden in een.}{schilderachtig oord een oud}{kasteel bewonen}\\

\haiku{Weer boog dat blonde.}{hoofd daar aan de overzij zich}{naar Lord Annavon}\\

\haiku{Ja, zij zou kunnen!}{leven op het geweten}{dragend zulk een dood}\\

\haiku{Zij zat gelijk een,.}{steenen beeld luisterend naar wat}{omging in het huis}\\

\haiku{... word wakker... en wist,.}{niet of het kind gevallen}{was of behouden}\\

\section{Maurice Gilliams}

\subsection{Uit: Elias of het gevecht met de nachtegalen}

\haiku{Maurice Gilliams,.}{Elias of het gevecht met}{de nachtegalen}\\

\haiku{Het is de eerste;}{keer niet dat ik haar van de}{blauwe hand vertel}\\

\haiku{En nu het eenmaal,.}{verontrust was wilde het}{niet meer stilliggen}\\

\haiku{Hij schijnt sterker en.}{mannelijker geworden}{tijdens deze tocht}\\

\haiku{Door het landgoed trekt,.}{onze kleine zwijgende}{stoet in de nacht}\\

\haiku{alleen de gouden.}{geest van het licht woont er nog}{rustig binnen in}\\

\haiku{de kin wrijft over de.}{bloedelooze kreukels van haar}{gevouwen handen}\\

\haiku{Zij draagt een kleed met.}{uiterst enge mouwen en}{witte manchetten}\\

\haiku{Blijkbaar zonder zich,.}{te haasten en toch in een}{omzien is hij weg}\\

\haiku{Hij staart mij aan, en.}{als ik me niet vergis is}{zijn gelaat beschreid}\\

\haiku{Elias, waarom hebt?}{ge uw vingertoppen met}{een naald gefolterd}\\

\haiku{Wanneer ik er een,:}{poos later over nadenk ben}{ik gerustgesteld}\\

\haiku{ik spring recht om de.}{vluchtelinge in de hall}{te achtervolgen}\\

\haiku{het is alsof hij.}{onverrichter zake van}{een eiland thuiskomt}\\

\haiku{Eerst wij jk zachtjes;}{naar binnen fluiten om hem}{welkom te wenschen}\\

\haiku{Ik durf niet zien waar,.}{ze ligt want het ergste zal}{met haar gebeurd zijn}\\

\haiku{mijn beenen worden moe,.}{als moesten ze onzichtbare}{draden voortslepen}\\

\haiku{De schali\"en aan;}{het torentje glimmen als}{met water bevloeid}\\

\haiku{Mijne moeder wordt,}{op eens weer levend en zij}{wandelt weg tusschen}\\

\haiku{tijdens het onweer.}{heeft zij aan het koperen}{kooitje staan prutsen}\\

\haiku{ik kom binnen als,.}{een klein stout kind aan de hand}{van mijne moeder}\\

\haiku{En waarom zie ik,?}{de gezichten niet wat zij}{van mij verwachten}\\

\haiku{Het is nu bijna}{middernacht en ik moet me}{nog wasschen v\'o\'or ik}\\

\haiku{morgen moet ik een.}{zuiver hemd en een frissche}{blouse aantrekken}\\

\haiku{laat ik me op de.}{knie\"en vallen en bespied}{zijn bedoelingen}\\

\haiku{ik voel een warmte,.}{naar het hoofd stijgen die mijn}{hersens geen pijn doet}\\

\haiku{En alsof ik hem:}{zijn voornemen uit het hoofd}{had willen praten}\\

\haiku{zij steekt haar hand in;}{een kous die zij tot aan haar}{elleboog optrekt}\\

\haiku{hij stapt de breede,.}{trap  op naast de portier}{die zijn valies draagt}\\

\haiku{Om mij dadelijk;}{te genezen heeft ze de}{pillen meegebracht}\\

\haiku{Zonder mijn mouw op.}{te stroopen steek ik mijn arm}{diep in het water}\\

\section{Anna van Gogh-Kaulbach}

\subsection{Uit: De hooge toren}

\haiku{{\textquoteright} {\textquoteleft}Nou,{\textquoteright} verdedigde, {\textquoteleft}.}{Henk zijn planze zien nog es}{wat van Sinterklaas}\\

\haiku{dadelijk week 't ':}{weer voor de gedachte aan}{t werk dat wachtte}\\

\haiku{In de bedste\^e werd,,.}{met hooge vogelgeluidjes}{het zusje wakker}\\

\haiku{{\textquoteright} spoorde Henk aan, als.}{was in die groote winkelstraat}{warmte te vinden}\\

\haiku{{\textquoteright} wees Wimpie en Henk,:}{meetrekkend vleide en drong}{zijn schraal stemmetje}\\

\haiku{Toch, in hun verval,:}{bewaarden zij restes van}{vroegere schoonheid}\\

\haiku{Of als vader weer,.}{werk had en moeder thuis kon}{blijven zooals vroeger}\\

\haiku{{\textquoteright} barstte Henk uit en,}{Wimpie loslatend goot hij}{met wild gebaren}\\

\haiku{{\textquoteright} Binnen smeet hij het,.}{boek op de plank veegde met}{zijn mouw over zijn oogen}\\

\haiku{Maar daarom moest zij, '.}{voorzichtig zijn dat zet}{niet merkten van Henk}\\

\haiku{Zoo kon Kees ook doen,.}{als ze vroeger eens naar de}{comedie gingen}\\

\haiku{{\textquoteright} En toen Henk haastig,,,:}{oprees nam ze zijn arm trok}{hem mee fluisterend}\\

\haiku{Met een ruk duwde,.}{hij de zware jassen op}{zij drong naar voren}\\

\haiku{hij voelde zich wee.}{en misselijk en in zijn}{oogen stonden tranen}\\

\haiku{{\textquoteright} gaf Henk terug, trok.}{meteen Wimpie mee in den}{kruidenierswinkel}\\

\haiku{{\textquoteright} zei hij vlug, met het.}{gevoel dat hij iets anders}{had moeten zeggen}\\

\haiku{{\textquoteright} smaalde hij en stak,.}{zijn tong uit naar Henk die snel}{zich had omgekeerd}\\

\haiku{Janus, plotseling,,.}{bevrijd riep een scheldwoord waar}{niemand op lette}\\

\haiku{{\textquoteright} De hevigheid van;}{Henks drift kalmde onder haar}{doen van oudere}\\

\haiku{Alleen Wimpie bleef.}{met Janus spelen en kwam}{soms bij hem in huis}\\

\haiku{Z'n vrouw zat er maar,}{mee die kon zich dood sloven}{en dat arme jong}\\

\haiku{Je vader is ook ',.}{n flinke vent dat hebben}{we altijd gezeid}\\

\haiku{Maar nu, onbewust,.}{voelde hij het vrije huilen}{als iets weldadigs}\\

\haiku{Gelukkig, dat de,.}{zon vandaag schijnt dacht Henk met}{nieuwe verheuging}\\

\haiku{En dat allemaal '.}{omdat ik die smeerlapn}{opzaniker gaf}\\

\haiku{En...{\textquoteright} hij aarzelde,:}{maar toen in eens gooide hij}{toch de vraag eruit}\\

\haiku{{\textquoteright} Ze zalen nog aan,;}{tafel toen er bedeesd aan}{de deur werd geklopt}\\

\haiku{{\textquoteright} En zichzelf in de,:}{rede vallend de oogen op}{de roode tulpen}\\

\haiku{die was tevrejen,.}{als hij wat moois had gemaakt}{dat eeuwen staan zou}\\

\haiku{hij wou niet weg juist,.}{nu in zijn nieuwe blijdschap}{om vaders thuiszijn}\\

\haiku{hij is niet gewoon,.}{om der vroeg in te leggen}{net zoo min as ik}\\

\haiku{{\textquoteright} Maar de vrouw, nu gansch,:}{zich gaan latend stootte uit}{door haar huilen heen}\\

\haiku{even - om Jen jongen -.}{gerust te stellen knipte}{hij met de leden}\\

\haiku{Over Vermeers gelaat.}{sloeg een heete driftblos en}{zijn oogen werden hard}\\

\haiku{Jans zonk terug in,:}{haar stoel en uitbarstend in}{huilen kermde zij}\\

\haiku{Moe was al naar bed,,}{maar ze had uit de alkoof}{geklaagd dat ze zoo}\\

\haiku{{\textquoteright} Vermeer keerde even,.}{zijn hoofd naar de slapende}{knikte tegen Henk}\\

\haiku{Nogal wonder, as!}{je de heele dag langs de}{straat mot schooieren}\\

\haiku{Nu lachten ze weer,.}{met hun drie\"en om haar dacht}{zij verdrietig}\\

\haiku{{\textquoteright} Hier dacht Vermeer weer.}{aan onder het loopen in}{de drukke marktstraat}\\

\haiku{Je mot heel vast in '.}{je schoenen staan omt vol}{te kennen houwen}\\

\haiku{{\textquoteright} Zij week iets voor hem,.}{op zij sloot de deur toen hij}{in de kamer was}\\

\haiku{Toen, terwijl zij het,:}{servet opvouwde zei ze}{ietwat aarzelend}\\

\haiku{t verdragen 'n,?}{schooier te blijven nou je}{goed werk ken krijgen}\\

\haiku{'t leek Vermeer dat.}{Jans nog bij geen bevalling}{z\'o\'o geleden had}\\

\haiku{Nu al maakte de {\textquoteleft}{\textquoteright} '.}{gedachte aan zijnneent}{moeilijker voor haar}\\

\haiku{Vermeer trachtte het.}{te onderscheiden in het}{weifelende licht}\\

\haiku{Jans deed de oogen open,.}{knipte ze even dicht als om}{Henk toe te knikken}\\

\haiku{{\textquoteright} Met zijn vuile hand,.}{streelde hij haar voorhoofd liep}{toen naar zijn vader}\\

\haiku{Tusschen hen was geen;}{woord meer gesproken over het}{voorgevallene}\\

\haiku{Zij schokte even met.}{de schouders en wendde haar}{gezicht naar bem toe}\\

\haiku{Op de hoeken van}{de straten groepten menschen}{samen en luchtten}\\

\haiku{Dus nu was hij er,,;}{de oorlog die ze al zoo}{lang voorspeld hadden}\\

\haiku{de kruijenier in de.}{Nieuwsteeg zou wat gort voor me}{bewaren en meel}\\

\haiku{{\textquoteright} Zijn oogen vingen een,:}{blik van Henk en sterker nog}{het hoofd geheven}\\

\haiku{Vermeer stond met Bosch;}{en Vermaas in een  hoek}{van het podium}\\

\haiku{Maar nou... verdom ik ',!}{t wat met die rotzooi te}{maken te hebben}\\

\haiku{In een flits herzag;}{Henk wat hij wist van moeders}{moeitevol leven}\\

\haiku{Bij 't hek van de,:}{textiel-fabriek waar zij}{werkte zei hij nog}\\

\haiku{de jongen deed haar,.}{nu telkens denken aan Kees}{zooals die vroeger was}\\

\haiku{Als een gerucht van.}{heel ver klonk achter hen het}{rumoer van de markstraat}\\

\haiku{{\textquoteleft}honderd doojen,{\textquoteright} {\textquoteleft}De,{\textquoteright}.}{oorlog kost er miljoenen}{gaf Vermeer terug}\\

\haiku{{\textquoteright} Vermeer legde even,.}{zijn hand op die van Henk die}{op de tafel lag}\\

\haiku{Ja, dammen ken je!}{en domineeren tot je der}{misselijk van bent}\\

\haiku{{\textquoteleft}En de andere.}{jonge kerels laten ze}{mekaar doodschieten}\\

\haiku{De luchtplaats lag als;}{een diepe kom tusschen groen}{begroeide wallen}\\

\haiku{Even blikte hij om.}{zich heen in de lachende}{jonge gezichten}\\

\haiku{Henk hoestte, droog en,;}{pijnlijk zooals hij was blijven}{doen na zijn ziekte}\\

\haiku{{\textquoteright} {\textquoteleft}En je berust er.}{in en weet tenslotte niet}{eens meer wat je mist}\\

\haiku{{\textquoteright} {\textquoteleft}Waarom berustte?}{hij der dan ook niet in om}{soldaat te worden}\\

\haiku{Hij gooide zich om,.}{op zijn stroozak vocht tegen een}{nieuwe hoestbui}\\

\haiku{Een schok van vreugde:}{voer door hem heen bij Vermeers}{plotselinge vraag}\\

\haiku{Hij streek zich over zijn,,.}{knevel nam de courant op}{begon te spreken}\\

\haiku{Maar 'n mensch wist soms '.}{niet meer waar hijt zoeken}{moest tegenwoordig}\\

\haiku{dat er een eind kwam.}{aan de beklemming van het}{gevangen-zijn}\\

\haiku{de mogelijkheid,;}{is niet uitgesloten dat}{hij verdronken is}\\

\haiku{Zwijgend kauwden zij,:}{het brood karig besmeerd met}{margarine}\\

\haiku{Vermeer rilde even.}{toen hij een slok nam uit de}{tinnen koffiekruik}\\

\haiku{hij v\'o\'or het Dagblad,;}{het bulletin waar Janssen}{over had gesproken}\\

\haiku{{\textquoteright} Wim knikte, zijn oogen.}{in kinderlijke peinzing}{voor zich uit starend}\\

\haiku{toen schoot Jans vooruit,;}{naar den ingang waar zij den}{schildwacht ontwaarde}\\

\haiku{{\textquoteright} de jongen had zijn.}{opstandigen geest blijkbaar}{niet van een vreemde}\\

\haiku{maar dan is er de.}{troost dat zij gevallen zijn}{op het veld van eer}\\

\haiku{Tegen Jans zei hij;}{niet hoe hijzelf alle hoop}{had opgegeven}\\

\haiku{Wim, begrijpend dat,;}{Henk dood was drong zich huilend}{tegen moeder aan}\\

\subsection{Uit: Jeugd}

\haiku{daarvoor ga je laat!}{na bed en zit je morgen}{op school te suffen}\\

\haiku{Toen de trein stil stond,,;}{stapte ze vlug uit liep met}{de anderen mee}\\

\haiku{{\textquoteright} {\textquoteleft}Ik ben anders zoo.}{gewend dat die lieve oogen}{me even ankijken}\\

\haiku{Ze liep in eens een,;}{zijstraatje in waar een paar}{meisjes aankwamen}\\

\haiku{ze liep de gang door,,;}{de huiskamer binnen waar}{ook al licht brandde}\\

\haiku{{\textquoteleft}Dat kan zijn, maar 't...}{past toch niet voor een meisje}{zooiets te vragen}\\

\haiku{{\textquoteright} Guust ging weer aan 't, ':}{rekenen maart ging nog}{moeielijker dan straks}\\

\haiku{je moet het zien met,.}{eigen oogen het voelen als}{een deel van je zelf}\\

\haiku{het papier was in, '...}{zijn zak maar hij durfdet}{haar haast niet geven}\\

\haiku{{\textquoteleft}Maar je moet denken, '.}{ik beschrijf niet enkel wat}{k mooi vind of goed}\\

\haiku{ze zou zoo graag haar,.}{moeder trotseeren maar ze}{durfde toch niet goed}\\

\haiku{{\textquoteright} {\textquoteleft}Dan doe je 't maar,.}{zonder zin je moet naaiwerk}{ook doen op z'n tijd}\\

\haiku{Nou ja, dat zijn van,.}{die bijzonderheden die}{haast nooit voorkomen}\\

\haiku{{\textquoteright} {\textquoteleft}Zou ze weer zooveel?}{te vertellen hebben over}{die neef uit Indi\"e}\\

\haiku{{\textquoteright} {\textquoteleft}Boven geloof 'k,,.}{ze was pas klaar met der werk}{daar komt ze net aan}\\

\haiku{Ze voelde altijd.}{weinig belangstelling voor}{zulke schandaaltjes}\\

\haiku{{\textquoteright} {\textquoteleft}Ja, ze zouen je,{\textquoteright}.}{maar exploiteeren viel de}{postdirecteur in}\\

\haiku{{\textquoteright} {\textquoteleft}Nee, Germinal is ',{\textquoteright}.}{t beste zei \'e\'en van de}{jongelui ernstig}\\

\haiku{hij wou iets doen wat,,}{ze zou bewonderen een}{kunstwerk scheppen z\'o\'o}\\

\haiku{Heil, heil, ik voel haar.....}{handen en den week en boog}{Van haren arm}\\

\haiku{Ze vond haar moeder, '.}{in de huiskamer waart}{al schemerig was}\\

\haiku{het volgende was,,;}{hooiland daar stond het gras hoog}{wachtend op de zeis}\\

\haiku{De tilbury was,:}{een zandweg ingedraaid waar}{het paard stappen moest}\\

\haiku{Guust trok haar arm door;}{de zijne en zoo liepen}{ze een eindje voort}\\

\haiku{hou je nou z\'o\'o maar,,?}{van me zonder dat ik wat}{ben wat gedaan heb}\\

\haiku{Haar moeder bleef haar,,:}{even aankijken ze nam haar}{hand zei bedarend}\\

\haiku{Z'n vader wil 't, '.}{immers ook die ziett maar}{w\`at verstandig in}\\

\haiku{En 't is voor z'n,.}{eigen geluk later zal}{ie je dankbaar zijn}\\

\haiku{vreemden bleven er;}{nu zoo ver van af met hun}{kijken en praten}\\

\haiku{Hij bleef in zijn boek, '.}{kijken om te laten zien}{datt hem ernst was}\\

\haiku{de groote massa dreef,.}{verder en waar hij langs streek}{bleef iets vaals achter}\\

\haiku{Of ja, ze moesten haar,.}{nemen anders verdiende}{ze immers geen geld}\\

\haiku{{\textquoteleft}Zeg vent,{\textquoteright} zei ze in,, {\textquoteleft},;}{eens naar hem opkijkendik}{vin dat je bleek ziet}\\

\haiku{Den laatsten tijd had:}{hij met Leida wel eens over}{godsdienst gesproken}\\

\haiku{{\textquoteleft}Vervloekt,{\textquoteright} mompelde,.}{hij woedend sloeg met zijn stok}{heftig op de steenen}\\

\haiku{{\textquoteleft}Ik hoop maar, dat Guust,{\textquoteright};}{eenvoudig zal blijven zei}{mevrouw Heerling weer}\\

\haiku{{\textquoteleft}hij krijgt den laatsten.}{tijd zoo iets verwaands over zich}{van jong studentje}\\

\haiku{{\textquoteleft}Ja, we zullen ons,.}{geluk veroveren een}{mooi leven hebben}\\

\haiku{Ze schrikte even, in.}{eens vooruitziende den strijd}{in vollen omvang}\\

\haiku{'t Was een zalig,.}{stil loopen een mooie rust na}{den mooien dag}\\

\haiku{hij wenschte weer,}{dit te leeren kennen nu niet}{alleen om te zien}\\

\haiku{{\textquoteright} 't Ontglipte haar,.}{op een toon van spijt die ze}{niet bedwingen kon}\\

\haiku{Maar evenmin kan ik:}{afstand doen van wat ik zie}{als m'n levenstaak}\\

\haiku{Toen Guust binnenkwam,,:}{keek hij op verbaasd en op}{zijn knorrigen toon}\\

\haiku{Hij zat luisterend,.}{naar den storm stilwachtend met}{zijn oogen op de deur}\\

\haiku{we hebben eenmaal,.}{bepaald dat alles tusschen}{jullie uit moet zijn}\\

\haiku{{\textquoteleft}och jongen, je haalt;}{jezelf en haar  zooveel}{verdriet op je hals}\\

\haiku{En de redactie,:}{van de courant rekende}{er op natuurlijk}\\

\haiku{'t Is wel aardig, ', '.}{n beetje droog bij alt}{nat dat je beschrijft}\\

\haiku{zijn hand vloog over 't,,.}{papier hij voelde geen kou}{meer geen moeheid ook}\\

\haiku{In den warmen trein ':}{gaft rhytmisch zacht schokken}{hem iets soezerigs}\\

\haiku{Nu zag ze hem, nu,;}{leefde haar gezicht op nu}{lachten haar oogen blij}\\

\haiku{wie weet hoe u uw,{\textquoteright}.}{hart nog op kunt halen aan}{vorst zei Willemien}\\

\haiku{Van Staaren reikte hem,.}{zijn twee handen drukte de}{zijne onstuimig}\\

\haiku{thuis trok hij zijn kleeren,,.}{uit liet ze liggen op den}{grond viel op zijn bed}\\

\haiku{We hebben misschien,.}{te veel geloopen je kwam}{om uit te rusten}\\

\haiku{Ik ben wat moe van ', '.}{t werken maart zal nu}{wel gauw beter zijn}\\

\haiku{{\textquoteright} {\textquoteleft}'N ei, nee juffrouw, ',.}{k heb der geen \'e\'en meer en}{dat gaot ook niet}\\

\haiku{ze namen hem mee,, '}{naar vergaderingen en}{gaven hem lectuur}\\

\haiku{hij was al jaren,.}{lang in de beweging gaf}{er al zijn tijd aan}\\

\haiku{ze had er dan toch '.}{bovendien nogt geluk}{bij van haar liefde}\\

\haiku{Ik heb heusch niet.}{genoeg om twee huishouwens}{te onderhouwen}\\

\haiku{Ze deed haar best, kalm,,,.}{gewoon te spreken stond op}{trok haar japon recht}\\

\haiku{Heerling trok haar arm,,:}{door de zijne begon toen}{langzaam voortloopend}\\

\haiku{Hij hoeft 't toch niet '?}{met alle personen uit}{t boek eens te zijn}\\

\haiku{{\textquoteleft}Kom,{\textquoteright} zei Willemien, {\textquoteleft} '!}{bedarendlees u maar eerst}{t heele boek uit}\\

\haiku{Guust voelde 't bloed,.}{opstijgen in zijn gezicht}{maar hij bleef toch kalm}\\

\haiku{'k Moet zeggen, dat '!}{je vadern prettige}{ouwe dag bezorgt}\\

\haiku{Ik vond er veel moois, '.}{in maark kan niet alles}{met je meevoelen}\\

\haiku{, wees heelemaal van,,.}{me leef mijn leven mee dan}{zal je leeren verstaan}\\

\haiku{schulden heb 'k niet,, ', '.}{toe laten wet probeeren}{v\'o\'ort te laat is}\\

\haiku{En aan dit boek kan',,,.}{k niets veranderen}{niets niets geen letter}\\

\haiku{En ook omdat strijd.}{eenmaal de natuurlijke}{en eenige weg is}\\

\haiku{{\textquoteleft}Schat,{\textquoteright} fluisterde hij, {\textquoteleft} '....?}{vlak bij haar oorje kuntt}{toch wel begrijpen}\\

\haiku{Ze stak haar hand uit,.}{als om afscheid te nemen}{hij drukte die vast}\\

\subsection{Uit: Het rijke leven}

\haiku{En vertel nu eens,.}{hoe alle kennissen in}{Boschvoort het maken}\\

\haiku{Je schijnt me niet te,,:}{willen begrijpen maar ik}{weet wat ik bedoel}\\

\haiku{ze liet er zich  ,.}{door omvangen zonk er in}{neer zonder denken}\\

\haiku{dat is nu al het,,.}{derde goede aanzoek dat}{je afslaat voor niets}\\

\haiku{Willy had geen hoed,.}{op liet den zoelen wind met}{haar haren spelen}\\

\haiku{Ze waren nu bij.}{de eerste huizen van het}{stadje gekomen}\\

\haiku{{\textquoteleft}Heeft u geen lust met?}{de dames eens naar het werk}{te komen kijken}\\

\haiku{Over het water hing,;}{een lichte nevel als een}{vochtige sluier}\\

\haiku{{\textquoteleft}Ik heb je al eens,.}{gezegd welke hooge eischen}{ik aan vriendschap stel}\\

\haiku{{\textquoteright} {\textquoteleft}Neen, zeg dat niet,{\textquoteright} viel, {\textquoteleft}.}{Willy driftig uithij heeft}{leelijk gehandeld}\\

\haiku{Mama begrijpt me,;}{niet en jammert alleen over}{het verloren geld}\\

\haiku{den dood van haar man,:}{maar deze had haar niet hard}{gemaakt of morrend}\\

\haiku{Ik moet u iemands,{\textquoteright}.}{groet brengen vervolgde hij}{na een oogenblik}\\

\haiku{'t Bloed vloog Willy ',.}{naart gezicht zelfs haar hals}{en voorhoofd kleurend}\\

\haiku{haar hart bonsde, in.}{haar gezicht waren vreemde}{zenuwtrekkingen}\\

\haiku{dit was dus alles.}{wat zij hem te zeggen had}{van haar verleden}\\

\haiku{Zou je werkelijk;}{niet kunnen gelooven in de}{liefde van dien man}\\

\haiku{het zich geven van;}{de vrouw aan den man scheen haar}{plotseling laag toe}\\

\haiku{ze zou George,;}{niet meer zien haar geluk zou}{voor altijd weg zijn}\\

\haiku{De kamer zag er,;}{zoo vreemd uit geheimzinnig}{in het schemerlicht}\\

\haiku{{\textquoteleft}Toe mama,{\textquoteright} begon, {\textquoteleft};}{ze weer na een oogenblik}{vraag me nu niet meer}\\

\haiku{{\textquoteleft}je hebt mij immers,.}{zelf gevraagd net te doen of}{er niets gebeurd was}\\

\haiku{Nu echter begreep, ',;}{hij dat d\`att was waardoor}{Willy zoo vreemd deed}\\

\haiku{toen hij weer sprak van;}{de biljartkamer zei ze}{als in gedachte}\\

\haiku{Het zou haar nu ook,;}{niets kunnen schelen al zag}{ze een kind doodslaan}\\

\haiku{{\textquoteright} {\textquoteleft}Dat weet ik juist niet, '.}{en u moet me helpen om}{t te bedenken}\\

\haiku{De ijzerfabriek;}{van den heer Dryfel lag een}{eind buiten Arnhem}\\

\haiku{toen was ze nog jong, ';}{na{\"\i}ef geloovend int}{mooie van het leven}\\

\haiku{{\textquoteright} {\textquoteleft}Weet je wat, kind, je,.}{moest oom en tante zeggen}{dat klinkt prettiger}\\

\haiku{Ze keek peinzend naar;}{zijn open jongensgezicht met}{de ernstige oogen}\\

\haiku{voor hen allen is,;}{meer arbeid te vinden dan}{ze af kunnen doen}\\

\haiku{Toch ben ik dikwijls,;}{bang voor Gerard nog minder}{dan later voor Louis}\\

\haiku{beloofde ze dien;}{winter een paar maanden thuis}{te zullen komen}\\

\haiku{Misschien is 't wel,,;}{goed als je daar dikwijls aan}{denkt zei ze langzaam}\\

\haiku{Die gedachte riep:}{hem een gezegde van zijn}{vader te binnen}\\

\haiku{{\textquoteleft}Ik zag gisteren,,}{op de tram aan je dat je}{zoo veranderd was}\\

\haiku{Wardorf was wel geen,;}{rijke partij maar toch heel}{goed aannemelijk}\\

\haiku{Gerard schreef haar een,,}{langen brief waarin hij wel}{tienmaal zei hoe blij}\\

\haiku{Maar ze moest toch nog,.}{eens aan hem denken hij kon}{haar niet meer missen}\\

\haiku{'t Begon haar te,.}{kwellen maar ze wist er niet}{van te beginnen}\\

\haiku{'t Scheen ook zoo dwaas,.}{dat ik beter wilde zijn}{dan de anderen}\\

\section{Paula Gomes}

\subsection{Uit: Sudah, laat maar}

\haiku{Wij worden vermoord.}{en Sonja verdwijnt in een}{huis voor geisha's}\\

\haiku{Tegen de ochtend.}{zagen we de stoet in de}{verte naderen}\\

\haiku{Je kon immers nooit,.}{weten er werden telkens}{mensen weggehaald}\\

\haiku{Ik had de kleine,.}{witte bloemetjes in het}{maanlicht zien prijken}\\

\haiku{Djongossen, nee, dat,.}{woord was taboe geworden}{pelayans die klaarstonden}\\

\haiku{Erna gaf mij een,.}{paarse steen mee die ik in}{mijn hand moest houden}\\

\haiku{Ik keek hoe iemand.}{eruitzag die een ander}{ging mishandelen}\\

\haiku{Mijn celgenoten.}{wisten het voor ik zelf op}{de binnenplaats kwam}\\

\haiku{De gevangene,.}{stond erbij de handen op}{de rug gebonden}\\

\haiku{Sylva en Erna.}{hielpen met het oprollen}{van de matrassen}\\

\haiku{We werden naar het.}{kamp gebracht dat eigenlijk}{al was opgedoekt}\\

\haiku{Een paar keer in de.}{week bestond het uit rijst met}{gebakken larons}\\

\haiku{We wachtten met z'n,,.}{allen in die ene kamer}{stiller dan de dood}\\

\haiku{Ik moest eerst het hek,.}{door langs de schildwachten met}{de bajonetten}\\

\haiku{Toen hij eindelijk,.}{afremde stonden we voor}{een gesloten huis}\\

\haiku{Alleen dat we in.}{Surabaja zaten toen}{de oorlog begon}\\

\haiku{Hierbij hoorden nu,.}{ook de doden de doden}{in het dodenrijk}\\

\haiku{Na de atoombom had.}{hij nog niets van zijn vrouw en}{kinderen gehoord}\\

\haiku{In een b\`etjak.}{zocht ik de plekjes op die}{mij bekend moesten zijn}\\

\haiku{Maar toen we bij het,.}{huis kwamen zag ik dat ze}{er niet meer woonden}\\

\haiku{je zag alleen de,.}{banden zonder dat je de}{titels kon lezen}\\

\haiku{De djongos zou me,.}{voorlopig naar zijn zuster}{brengen naar Sitih}\\

\haiku{Vannacht toen ik hem;}{naar buiten liet is hij niet}{teruggekomen}\\

\haiku{Niet alleen dat hij,.}{geen schoenen droeg in zijn hemd}{zaten slijtgaten}\\

\haiku{Indonesische.}{militairen liepen de}{huizen in en uit}\\

\haiku{Ik was verbaasd, maar.}{volgde hem naar de tuin tot}{bij het fietsenrek}\\

\haiku{Mijn grootvader was,{\textquoteright}.}{een Hollander vertelde}{een hotelportier}\\

\haiku{Legertrucks kwamen.}{de Hollanders overal in}{de stad ophalen}\\

\haiku{Ze zagen me niet,.}{staan verscholen tussen de}{struiken in de tuin}\\

\haiku{Die gaf het weer door.}{naar de volgende en zo}{de hele weg langs}\\

\haiku{Ze kreunde in haar.}{slaap na de injectie die}{ze gekregen had}\\

\haiku{Of dat ze zelfs maar.}{uit beleefdheid zeiden dat}{we welkom waren}\\

\haiku{Zo zat ik weer met.}{een heleboel mensen bij}{elkaar in een kamp}\\

\haiku{De mannen strekten.}{hun armen naar het licht en}{wiegden heen en weer}\\

\haiku{Misschien is voor dit.}{inzicht een soortgelijke}{ervaring nodig}\\

\section{Leon Gommers}

\subsection{Uit: Het uurwerk van Floor}

\haiku{Als ik dan zomaar:}{van het bed omhoogkom zie}{ik mezelf twee keer}\\

\haiku{De Dodenweg loopt.}{tussen die zilverig licht}{spattende kuilen}\\

\haiku{Ik vergeet, nog steeds,}{een beetje onrustig ik}{vergeet niet hoe laag}\\

\haiku{Ik knik dan zo lang.}{omdat hij daarop net zo}{lang terugtwinkelt}\\

\haiku{Ken je die van de?}{televisiekijkers op}{het Drielandenpunt}\\

\haiku{Soms zegt Floor dingen,.}{zonder dat ik hem iets vraag}{en dat is iets nieuws}\\

\haiku{De stiekjes rond een.}{kleurpotlood waar je soepel}{mee kunt trommelen}\\

\haiku{ze noemen het in.}{Duitsland natuurlijk niet voor}{niks een Rummelplatz}\\

\haiku{De rijtjeshuizen.}{verdwijnen in het eerste}{groen van de bomen}\\

\haiku{Dat schroef je bij dat,...}{paar gewone wijzers van}{zes uur Borstelkop}\\

\haiku{Ik probeer met een.}{vinger hoe koel een messing}{traproede kan zijn}\\

\haiku{Zijn duim gaat omhoog}{en deze bal komt niet zo}{hard en  vliegt aan}\\

\haiku{mijn hoofd terwijl ik.}{een vallende warrel van}{botten en vlees ben}\\

\haiku{het is en blijft met:}{jou hetzelfde als in de}{rest van de wereld}\\

\haiku{Broer heeft het stuur als.}{een vrijgezel vast en rukt}{aan het stuur en slipt}\\

\haiku{Dan stokt oma's gang.}{en de straat wordt drukker en}{zenuwachtiger}\\

\haiku{matrone vind ik,}{nog wel leuk maar mijn vader}{weet allang niet meer}\\

\haiku{boos naar Floor op de.}{grond en tussen de bomen}{van ons klokkenbos}\\

\haiku{Evenmin hou ik van.}{spelletjes waar veels te veel}{beweging in zit}\\

\haiku{De begerige.}{vingers van twee dringende}{neefjes gaan omhoog}\\

\haiku{Het zijn speelkaarten.}{en ze zijn er ook in het}{nachtblauw natuurlijk}\\

\haiku{Ik kijk dan door het}{zwarte raam naar de sterren}{buiten en dan hoor}\\

\haiku{, en Otto is een,.}{oude schoonzoon ongerust}{naar Anton kijken}\\

\haiku{En dan wachten tot.}{het uurwerk in een enkel}{ogenblik tot stand komt}\\

\haiku{Floors opengeknipte.}{scheermes hangt op hoogte van}{het blauwe schaamplekje}\\

\haiku{En als we snijden,.}{zijn wij trots en voelen niets}{van onze schaamte}\\

\haiku{Floor heeft het mes plat.}{op het vlees van de wang en}{tegen de oogkas}\\

\section{H.A. Gomperts}

\subsection{Uit: Intenties I, Kritieken en over kritiek}

\haiku{Eerst moet ik je in.}{de rede vallen om je}{gelijk te geven}\\

\haiku{5 Wrijving tussen.}{kunstenaars en publiek is}{geen nieuw verschijnsel}\\

\haiku{'Ik ben niet Cinna,.'}{de samenzweerder ik ben}{Cinna de dichter}\\

\haiku{De schrijver moet dus.}{inderdaad soms niet op zijn}{woord geloofd worden}\\

\haiku{Hetzelfde geldt voor,.}{geloof levensbeschouwing}{en filosofie}\\

\haiku{Later verschijnt de:}{vrouw bij Hemingway nog slechts}{in twee gedaanten}\\

\haiku{De vis is te groot.}{voor zijn bootje en daarom}{bindt hij hem langszij}\\

\haiku{Hij is discreter.}{en terughoudender dan}{de echte schrijver}\\

\haiku{Zij schrikt echter van,.}{de wildheid van zijn wanhoop}{kust hem en loopt weg}\\

\haiku{Zij bevatten geen.}{puzzel die uit het leven}{ge{\"\i}soleerd is}\\

\haiku{Poesjkin vertoont:}{allerlei kenmerken van}{de romanticus}\\

\haiku{Ik geloof dat men.}{hem onrecht doet door hem zo}{te klasseren}\\

\haiku{Na de dood van zijn ''.}{moeder schrijft Toergenjew het}{verhaalMoemoe}\\

\haiku{'Op een dag zullen.}{wij achter ons huis zitten}{om thee te drinken}\\

\haiku{Hij probeert het toch.}{en de inspanning bezorgt}{hem een attaque}\\

\haiku{Het probleem van de.}{psychologische waarheid}{bestaat voor hem niet}\\

\haiku{Hij is eerder een.}{losbandige figuur dan}{een wellusteling}\\

\haiku{Nu is hij dus weer.}{op die laatste verklaring}{teruggekomen}\\

\haiku{De heren van dit.}{landgoed doen niet veel anders}{dan eten en drinken}\\

\haiku{waaruit volgt dat men.}{nooit orthodox moet zijn in}{kwesties van smaak}\\

\section{Sam Goudsmit}

\subsection{Uit: Zoekenden}

\haiku{ze zagen, wie er,;}{op visite kwam en ze}{begrepen waarom}\\

\haiku{Och ja, Moeder, det,.}{vleesch kump wel ga\`ar zonder}{o\`e goa oe g\`ang maar}\\

\haiku{as ze de koppen, ',....}{bij mekare steken dan}{giett over mien d\`ah}\\

\haiku{zeg ze, och mensche, ' ' '.}{ik adder graag mien bord}{inr snoeteegooid}\\

\haiku{bah, 'k zol mien toch, '?}{mien oogen uut mien heufd schamen}{hef ze d\`etezegd \`e}\\

\haiku{{\textquoteright} {\textquoteleft}Ze mut zekers booles ',{\textquoteright}.}{alen veur Sjabbes probeerde}{Meijer te spotten}\\

\haiku{nou ebben we al....}{twintigduizend gulden schuld}{onder de boeren}\\

\haiku{{\textquoteright} riep Naatje, op den, {\textquoteleft} '!}{grond voor hem uitspugendwat}{n minne kerel}\\

\haiku{{\textquoteright} Sam zette 't luik '.}{voort winkelraam en liet}{de gordijnen neer}\\

\haiku{Maar enkel v\'o\'or zich....}{wou-ie zien in z'n boek en}{denken aan Sjabbes}\\

\haiku{baron de Nekoome, ',....}{of hoe heet die potsneus}{isn geschiewes}\\

\haiku{{\textquoteright} {\textquoteleft}Goa oe gang maar, ogod,,{\textquoteright}.}{h\`ee kreunde Vader van de}{ledenverwringing}\\

\haiku{Waarom kon-ie geen'?}{moppen krijgen en die hond}{van  n Moos wel}\\

\haiku{dat tuig wou zeker.}{voor tien lappies van honderd}{nog een graaf hebben}\\

\haiku{was 'n aardige... ' '....}{roestige sjiddesch ewestn}{sjiddisch metn luchien}\\

\haiku{maar moest die h\`ond 'm?....}{dan n\`ou ook vragen om die}{vervloekte centen}\\

\haiku{{\textquoteleft}hij vertrapte 't, ',{\textquoteright}.}{h\`aar kastanjes uitt vuur}{te halen zei-ie}\\

\haiku{nou,{\textquoteright} zei-ie, {\textquoteleft}ie 'em....}{ook al genog meeemaakt in}{d\`eze zeuventig}\\

\haiku{waar-ie niet an 't,?}{bod dan op det perceel an}{de Laragediek}\\

\haiku{wij ons kapitaal,}{beter in onze z\`ake}{gebruukenn doar ebbe}\\

\haiku{t kan ze t\`och niks,....}{verdommen waj koopen of}{niet koopen waj \`em}\\

\haiku{Rozette, dichter,:}{naar Naatje geschoven zag}{naar haar en Jette}\\

\haiku{met 'n schort voor kon ' ';}{zer met plezier w\`at graag}{n beetje helpen}\\

\haiku{Zij scheen ongewoon:}{beminnelijk vandaag aan}{Naatje en Jette}\\

\haiku{Ja, ze moest naar huis,, '.}{want die ze n\`ou weer had die}{snoeptet huis leeg}\\

\haiku{Goddank, as je geen,,,.}{meid noodig had zooals zij juffrouw}{Beem die Jette had}\\

\haiku{bah, wat had ze toch ';}{eigenlijk een hekel an}{t heele zoodje}\\

\haiku{David en Kobus.}{Koopmans en de amechtige}{Davids met z'n vrouw}\\

\haiku{{\textquoteleft}Ik zol zoo'n soten '.}{int zwart werachtig niet}{binnenloaten}\\

\haiku{{\textquoteright}, vroeg hem Moos in de', {\textquoteleft} '?}{Vries Brabantsch accentvinde}{t soms ni\`et mooi}\\

\haiku{in geen drie weken....}{had-ie een stukje vleesch}{an de haak gehad}\\

\haiku{de jonge Rebbe,,.}{het kerkbestuur met Mr. van}{Lier voorbijrenden}\\

\haiku{en as me dan an, '....}{iemand k\`ent dan weet me toch}{wat mer  ge\`eft}\\

\haiku{Gij hebt gezien dat,.}{er vreugde dat er blijdschap}{is in dit leven}\\

\haiku{als gij eenmaal zult;}{gekomen zijn aan de grens}{van deze woestijn}\\

\haiku{{\textquoteright} vroeg Moos naar De Beer,;}{en z'n vrouw die alleen nog}{gebleven waren}\\

\haiku{Joop, nou tegenover,.}{de sterk gemaakte zaak moest}{er totaal onder}\\

\haiku{ik kan mien doar van,:}{zelf niet mee bemujen}{maar d\`et weet ik wel}\\

\haiku{hij zal zien k\"ossien,;}{wel opscharrelen wees do\`ar}{maar verzekerd van}\\

\haiku{{\textquoteright} besloot Moos, {\textquoteleft}ja, ze,}{bin nog al zoo lekker de}{knechten um lief veur}\\

\haiku{wat heb 'k \'um oe, '!}{heen edreid net zoo lange tut}{ask vrij ware}\\

\haiku{'k Kan wel zien, da'j ',.}{t verliezen anders zo'j}{niet um thee denken}\\

\haiku{a'j speulen willen,,.}{dan speul-ie maar ik kan mien}{ier best verm\`aken}\\

\haiku{Is d{\`\i}t nu de stad,?}{die heet een zoo heerlijke}{schoonheidsvolbouwing}\\

\haiku{Die, schuw, was tusschen.}{de doorloopen naar Joede}{Rosenstein gegaan}\\

\haiku{minachtend had hij,;}{naar hen neergezien snauwend}{hen toegesproken}\\

\haiku{Lion met zijn pet,.}{af Joede altijd onder}{z'n groenigen hoed}\\

\haiku{{\textquoteright} {\textquoteleft}Nou, dan goa we maar,{\textquoteright}, {\textquoteleft}.}{zei Lionik blieve toch}{maar hier sloapenn}\\

\haiku{urn mien niet bij 't?}{bedde van mien Vader te}{willen loaten}\\

\haiku{k heb zorrege,....}{genog an mien kop ik kan}{oe niet hellepen}\\

\haiku{ik geleuve well....}{dat oe warrek hier nou is}{of-eloopenn}\\

\haiku{{\textquoteright} {\textquoteleft}Graag,{\textquoteright} zei Joop kort en, {\textquoteleft}.}{stond opMorgen za'k mien geld}{wel komen halen}\\

\haiku{Hij merkte het niet,;}{hoe hij overal \`om zich de}{teleurstelling vond}\\

\haiku{Zij gingen de straat,.}{op lieten h\`aar thuis om den}{boel te verzorgen}\\

\haiku{ewest urn de boel in....}{orde te maken en to\'e}{was jullie Sam egoan}\\

\haiku{was Joop, dezelfde,,....}{Joop soms niet met dreug brood noa}{schoole egoan ja of ne\'e}\\

\haiku{Ja, 't ging goed zoo,, '.}{hij kon gauw goan bedelen}{ast zoo voortging}\\

\haiku{{\textquoteright} {\textquoteleft}Goeienoavond,{\textquoteright} zei Joop:}{binnensmonds en dadelijk}{zachter tot Grietje}\\

\haiku{{\textquoteright} Hartog zat even te ',.}{denken overt voorstel dat}{ie Joop dacht te doen}\\

\haiku{Morren mun ze mien,....}{beloven dat ze niks meer}{zullen verlangen}\\

\haiku{{\textquoteright} {\textquoteleft}De's ofgespreuken,{\textquoteright}, {\textquoteleft} '....}{besliste Hartogen n\`ou}{goak noa huus toe}\\

\haiku{hellep mien tenminsten ' '....}{ann paar centen umn}{koegien te koopen}\\

\haiku{De boeren kregen.}{hun geld dan niet en gaven}{geen tweede stuk vee}\\

\haiku{{\textquoteleft}Ja, det kon 'k wel.......}{dadelijk denken meneer}{toe'j van lo\`od sprakken}\\

\haiku{{\textquoteright} {\textquoteleft}Moet jij zingen,{\textquoteright} riep,:}{Naatje en zachter tot den}{violist meteen}\\

\haiku{Kalm moest-ie voort, straks,....}{het huis met hem door en hem}{br\`engen naar het lood}\\

\haiku{dat er toch niks w\`as, '.}{datt onzin was om mee}{naar binnen te gaan}\\

\haiku{t Is in orde,{\textquoteright}, {\textquoteleft}.}{zei de rechercheurik leg}{beslag op dat lood}\\

\section{Johan Graafland}

\subsection{Uit: Van toen en thans}

\haiku{{\textquoteleft}Manne{\textquoteright}, begon hij, {\textquoteleft},.}{weerjullie bevalle me}{jullie kletse niet}\\

\haiku{Onmiddellijk liep '.}{n soldaat op hem toe om}{zijn pas te vragen}\\

\haiku{Schoppenheer rolde,,.}{neer morsdood kogel dwars door}{de halsslagaders}\\

\haiku{Toen hebben wij Haar:}{spreuk op Haar energiek gelaat}{belichaamd gezien}\\

\haiku{gij werdt genummerd,,,,.}{gekeurd inge\"ent gebaad}{geknipt en gericht}\\

\haiku{Die Dood, welke wij,,.}{Nederlandsche soldaten}{niet genoeg kennen}\\

\haiku{deze tijd van den.}{Dood heeft ons niet gebeterd}{en niet gelouterd}\\

\haiku{Deden zij dit, dan.}{werden zij op de krijtrots}{tot koning gekroond}\\

\haiku{{\textquoteleft}mon lieutenant,.}{voici le commencement}{de ma trag\'edie}\\

\haiku{hij stelde daarbij ' {\textquoteleft}{\textquoteright}.}{een dronk in opn spoedig}{pax hominibus}\\

\haiku{Tegen 'n boomstam.}{lag de oppasser van den}{vlieger te kermen}\\

\haiku{Na haar lied liep ze, ';}{naar den boom wiens takkenn}{kribbe overhuifden}\\

\haiku{Toen nam ze oma Reg,:}{bij haar rokken en draaide}{haar om zeggende}\\

\haiku{Diep en dwars drongen;}{er de karresporen in}{den vettigen grond}\\

\haiku{Met de uiterste.}{moeite handhaafde hij nog}{z'n autoriteit}\\

\haiku{Jij vecht elken dag ' ' '.}{voor me.k Moett tochns}{aan iemand zeggen}\\

\haiku{Hij kladt cahiers vol.}{met engelen-kopjes}{en koeien-pooten}\\

\haiku{- {\textquoteleft}'k Verbied je van,:}{Wijk onkrijgstuchtelijk over}{m'n chefs te praten}\\

\haiku{Toen mengde zich de ':}{luitenant-adjudant van}{Wijk int gesprek}\\

\haiku{Wat hij vanaf zijn,!}{twintigste jaar gevreesd had}{werd nu bewaarheid}\\

\haiku{Na eenigen tijd zei {\textquotedblleft}{\textquotedblright} {\textquotedblleft}{\textquotedblright}.}{de luitzijn hand wordt stijf en}{toenzijn hand is dood}\\

\haiku{{\textquoteright} commandeeren en de '....}{echo vant diepe graf gaf}{de schoten terug}\\

\haiku{E\'ens in die vier....}{maanden bezochten hem de}{majoor en Queen in}\\

\section{Jan Greshoff}

\subsection{Uit: Currente calamo}

\haiku{beste makker, daar,;}{geloof ik geen woord van dat}{is pure nonsens}\\

\haiku{dat alles is mooi,;}{daar valt voor de burgerij}{niet aan te tornen}\\

\haiku{En ik geloof dat.}{niemand hierop een precies}{antwoord kan geven}\\

\haiku{Zoo vergaat het mij,.}{wanneer ik zoo nu en dan}{mijn Kloos ter hand neem}\\

\haiku{Wij hebben, voor ons,.}{zelf een legendarischen}{held van hem gemaakt}\\

\subsection{Uit: In alle ernst}

\haiku{De ware minnaar.}{leeft met zijn liefde alleen}{in een leege wereld}\\

\haiku{Men behoort niet op}{reis te gaan zonder eenige}{teksten van zijn hand.}\\

\haiku{Het zijn geringe.}{gebeurtenissen en het}{vermelden niet waard}\\

\haiku{De bohemien voert.}{iederen dag opnieuw een}{strijd op twee fronten}\\

\haiku{Wie het meesterschap,:}{zoekt doet afstand van wat het}{ware genot is}\\

\haiku{Speenhoff heeft immer.}{een eigen stijl van schrijven}{en spreken gehad}\\

\haiku{- Voor mij - zegt hij - is,.}{een boek volgepropt met kunst}{altijd een slecht werk}\\

\haiku{Ik schrijf uitsluitend,.}{voor mezelf omdat ik niet}{weet wat eerzucht is}\\

\haiku{vroeg ik verbaasd en.}{nu toch wel in mijn diepste}{gevoelens geschokt}\\

\haiku{wat de moeite waard,,}{is gezegd te worden is}{de moeite waard goed}\\

\haiku{Van Schendel weet  :}{dat zelf en hij heeft dit ook}{willen openbaren}\\

\haiku{Met dit boek voltooit.}{Arthur van Schendel de trits van}{zijn noodlotromans}\\

\haiku{Wie er vijandig;}{tegenover staat ontdekt er}{niets van belang in}\\

\subsection{Uit: Rebuten}

\haiku{Maar dat doet er niet,.}{toe het komt er op aan om}{gedrukt te worden}\\

\haiku{En nog ben ik er.}{niet zeker van dat zij in}{hun ongelijk staan}\\

\haiku{Het eenige wat ons.}{in de krant interesseert}{is het Gemengd Nieuws}\\

\haiku{Omdat die maar \'e\'en,:}{levensdoel maar \'e\'en reden}{van bestaan hebben}\\

\haiku{En zoodra wij}{ophouden met alles wat}{wij niet begrijpen}\\

\haiku{Ik houd ook wel eens,,.}{en mijn goede vrienden niet}{minder van lachen}\\

\haiku{dat is nu alles,.}{wat wij verlangen alles}{wat wij noodig hebben}\\

\haiku{Hier onderbreek ik,.}{even mijn brief omdat ambtsplicht}{mij naar buiten drijft}\\

\haiku{Er volgen er nog,.}{vele even positief en}{even kinderachtig}\\

\haiku{{\textquoteright} {\textquoteleft}Het rythme van de.}{nabije toekomst trilt reeds in}{de verleden tijd}\\

\haiku{all\'e\'en, uitsluitend).}{en all\'e\'en het koffijhuis}{in ons kan wekken}\\

\haiku{Het dwaasklinkende {\textquoteleft}{\textquoteright},;}{spreekwoordliefde is blind is}{au fond niet zoo dwaas}\\

\haiku{Gij hebt er geen flauw.}{vermoeden van hoezeer mij}{dat alles koud laat}\\

\haiku{Het doet mij pijn het.}{tegenover u openlijk te}{moeten erkennen}\\

\haiku{dat het goed is z\'o\'o;}{te handelen en slecht om}{het anders te doen}\\

\haiku{Wilt gij nu, waarde,?}{strijdmakker met den heer X.}{gaan discussieeren}\\

\haiku{Ik zal mij dus wel.}{wachten om er een oordeel}{over uit te spreken}\\

\haiku{\'o\'ok hier weer precies.}{dezelfde verhaspeling}{der verhoudingen}\\

\haiku{Wanneer ik uitglij,:}{over de vuiligheid van den}{heer X zeg ik ook}\\

\haiku{Ik ook, Theodoor,.}{wil in m{\`\i}jn wereld leven}{en blijven leven}\\

\haiku{Maar nu kom ik op.}{mijn betoog terug en aan}{Marnix Gysen toe}\\

\haiku{Als hij niets anders,.}{op zijn geweten had zou}{het nog wel schikken}\\

\haiku{{\textquoteright} Mij trof hier in het {\textquoteleft}{\textquoteright}.}{bijzonder het boeiende}{gebruik vantelkens}\\

\haiku{Uit zijn dichtproeven {\textquoteleft}.}{spreekt niet zelden groote deernis}{met de misdeelden}\\

\haiku{Van Willem Elsschot,;}{zwijgen we omdat men daar}{gewoonlijk over zwijgt}\\

\haiku{Maar het ergste zijn.}{professors omschrijvingen}{en aanwijzingen}\\

\haiku{Is het u wel eens?}{overkomen mij werkelijk}{boos aan te treffen}\\

\haiku{En ik zelf heb \'e\'erst;}{Gallinaria tegenover}{Alassio veroverd}\\

\section{Alfons Grond}

\subsection{Uit: Schetsen van Heihoven}

\haiku{g'n loch is blieve.}{hange En doer d'r baemd is}{opgevange}\\

\haiku{Voelde hij niet hoe?....}{wrede klauwen geslagen}{werden in zijn arm}\\

\haiku{De vreemde heer hing....}{over de onderdeur van de}{schuur en betoogde}\\

\haiku{uch volmach, va mich ' '............}{zoltr geene las kriege al}{howtr em dat e}\\

\haiku{es vader toch wal,{\textquoteright}.}{im sjtand zie om zoene}{peumes te regeere}\\

\haiku{{\textquoteleft}zaat sjamt ier uuch nit,{\textquoteright}.}{los jevelles och jet uvver}{vuer angere}\\

\haiku{Om dat grote doel,.}{te bereiken was er bier}{en veel bier nodig}\\

\haiku{Pitter-Grades.}{wilde  eten en met rust}{gelaten worden}\\

\haiku{Ee tuurke wie een,;}{paeperbus Wees ooch noch}{sjeef noa boave}\\

\haiku{Nee ich ka mit d'r,.}{beste wil Dien oetzich neet}{erg loave}\\

\haiku{D'r Piet dach dat 'n ':}{t nogal wis en begoos}{aaf te ratele}\\

\haiku{Zonger mich veul doa,}{bie te dinke bloos ich dat}{dink op en zoot doe}\\

\haiku{mit eene roe kop en '}{der doem opt flutje te}{kieke en wos neet}\\

\haiku{Mae 't mot toch 't ' '.}{int ent angert uever}{m gezag waere}\\

\haiku{De bladeren van.}{de bomen en de heggen}{hingen slap en stil}\\

\haiku{Marie stond achter.}{het buffet met de kin op}{haar vuisten geleund}\\

\haiku{Da zien ich dek der,.}{boer mit sjup of hak Al op}{d'r akker sjtoa}\\

\haiku{puende-n 't, ',......}{Treesje althans ich wolm}{puene mae i}\\

\haiku{Op ins doa huer......}{iech jet hinger miech roespere}{en sjnoespere}\\

\haiku{Opins doa huer......}{iech jet hinger miech roespere}{en sjnoespere}\\

\section{Robert H. van Gulik}

\subsection{Uit: Vier vingers}

\haiku{Dat werkte, want hij.}{zag tersluiks dat de aap hem}{geboeid gade sloeg}\\

\haiku{Gan dronk zijn kop leeg.}{en ging eens verzitten op}{zijn houten krukje}\\

\haiku{Met die anderen!}{bedoel je dan zeker al}{die vieze vliegen}\\

\haiku{{\textquoteleft}In zo'n sjieke zaak,{\textquoteright}.}{komen geen landlopers zei}{Tao Gan tot zichzelf}\\

\haiku{Ik wou haar net te,}{woord staan toen de baas d'r an}{kwam waggelen as}\\

\haiku{Gan schoof de strengen.}{koperstukken met een vies}{gezicht in Leng's schoot}\\

\haiku{Amper was ik voor,;}{m'n poort uitgestapt of ik}{werd ineens erg naar}\\

\haiku{{\textquoteleft}Ik wist toch niet dat,?}{die meneer bij het gerecht}{hoort Edelachtbare}\\

\haiku{nou maar 's of je.}{d'r vijf andere precies}{zo naast ken legge}\\

\haiku{Hij bromde een paar.}{woorden van dank en liep de}{Rode Karper uit}\\

\haiku{Tao Gan  zag dat.}{van de linker pink alleen}{een stompje over was}\\

\haiku{Seng Kioe bekeek:}{de hoge stenen muur eens}{en zei met ontzag}\\

\haiku{Hij gaf de reus een.}{klap tegen zijn kuiten met}{het plat van zijn zwaard}\\

\haiku{het lijk niet te lang,.}{boven de grond te laten}{met dit warme weer}\\

\haiku{Seng Kioe vatte.}{het stilzwijgen van Rechter}{Tie als twijfel op}\\

\haiku{as me zus die oom!}{Twan an het lijntje houdt dan}{benne we binne}\\

\haiku{Zeg op, wanneer heeft?}{Leng de lommerdhouder jou}{in dienst genomen}\\

\haiku{Toen hij haar eens goed,:}{had opgenomen begon}{de rechter rustig}\\

\haiku{Hij was een goeierd,.}{ook al heb-ie me in}{de steek gelate}\\

\haiku{Dan trekke we het,!}{Kanaal op en af met vracht}{da's een fijn bestaan}\\

\haiku{Heb je dan nog niet?}{ingezien dat die moordzaak}{nu is opgelost}\\

\haiku{{\textquoteleft}Ja, ik beken dat,.}{ik Twan Mou-tsai vermoord}{heb Edelachtbare}\\

\haiku{{\textquoteright} Hij keek de rechter:}{smekend aan en vervolgde}{met trillende stem}\\

\haiku{Toen heeft hij met de.}{huismeester gegeten en}{is naar bed gegaan}\\

\haiku{Hij is altijd zo...}{goedig en behulpzaam en}{zo lief voor dieren}\\

\section{Maurits Gysseling en W. Pijnenburg}

\subsection{Uit: Corpus van Middelnederlandse teksten. Reeks II. Literaire handschriften. II-6. Sinte Lutgart, Sinte Kerstine, Nederrijns moraalboek}

\haiku{Anja de Man, Het:}{leven van sinte Lutgard}{door broeder Geraert}\\

\haiku{Het volgende vers.}{begint dan meestal met}{een kapitaaltje}\\

\haiku{Deze heeft soms de,.}{vorm van een schuin soms van een}{gebogen streepje}\\

\haiku{die hi249 g[eliken250] [}{sinen oghen woude jc}{seid mindan251 ic se}\\

\haiku{] [ente]n301 melk[e] [][] [........}{der minschlecheitdie dar uloy}{en vte cristus mont}\\

\haiku{walschelant wonen}{node maer si hadde eer}{te herkenrode}\\

\haiku{ghsciet691 692Doen tfolc}{dit oppenbarlec kinde}{so liet men rusten}\\

\haiku{ende sach dat hi}{in sorchleken leuene lach}{doen hi van herten}\\

\haiku{Doen si hem bat sus,}{jnnechlec sere 748 voer sijn}{siele so antwerd}\\

\haiku{di haer hadde haer.}{pine ghec\r{u}rt jnt veghvier}{so langh te sine}\\

\haiku{ende seide / haer.}{wiedwijs dat hi een nonne}{bedrogen hadde}\\

\haiku{alle sondaghe.}{al sint austijn maent dat doen}{sal elc goet kerstijn}\\

\haiku{962 Et es heilech}{salech end goet dat men den}{sieken dan gracia}\\

\haiku{metten nonnen sanc}{in den coer 998 daer sach een}{nonne di scegen}\\

\haiku{hi vuedet ende}{heuet lief ende smeket}{ende huedet Hien}\\

\haiku{dees waert O dochter}{van iherusalem stant oppe}{van dinen bedde}\\

\haiku{haer god een ander.}{manyre der lichamlec}{martyrien scire}\\

\haiku{die heilech waren}{beide ende goet die daer}{na dat si sturtte}\\

\haiku{hoe dat was wijlneer}{sijn name 1212 ende van}{wat verdientten hi}\\

\haiku{] seide dat si soud []}{werdden sciereverloest}{genedechlec vten}\\

\haiku{beddeken daer men []....}{af seghtdat scone bloyt}{ende blomen dreght}\\

\haiku{dit varen ende..}{ginghen spreken stappans van}{andren dinghen}\\

\haiku{Dies oec geloeft moet.}{werden de vrie di edele}{maght sinte kerstine}\\

\haiku{Op di vre dat men,}{voer mi sprac in de messe}{dierste agnus dei so}\\

\haiku{hem lief ocht waest hem..}{leet 1997 ende nayet aen}{haer clederkine}\\

\haiku{, hebben omgeghaen}{daer god sijn gracie hadde}{met gedaen ende}\\

\haiku{natuerlec sijn}{bloede 2170 Boude maeghde}{sijn selden bleuen}\\

\haiku{Jn dalder leste}{iaer dat si van ertrike}{sciet de maghet2245 vri}\\

\haiku{om en laet di mi}{nyet wedercomen ter}{erden daer ic af}\\

\haiku{135,34 136,3 136,4 136,4 136,6}{136,7 136,11 136,18 136,21 136,29 136,30 136,35}{136,37 137,1 137,3 137,5 137,13}\\

\haiku{) gewaerlec 69,8 () () (}{133,352 gewagen 9,141}{gewaghde 146,361}\\

\haiku{ysaac gemac ghemac [}{brac gebrac sprac trac uertrac}{vertrac sac stac.]ec}\\

\haiku{waarschijnlijk is dit.}{gebeurd door aankoop in de}{17de-18de eeuw}\\

\haiku{/   Urinscap is als.}{en minsche g\r{u}den wille}{he/uet tegens enen man}\\

\haiku{man / vint selden    /.}{dat sconheit inde suuerheit}{te gader bliuen}\\

\haiku{/ als lange als man /.}{wel helpen mag so heuet}{man   vrinde gn\r{u}g}\\

\haiku{/   D\r{u} al dat du /.}{d\r{u}s gelik of t\r{u}t v\r{u}r al}{den luden dedes}\\

\haiku{Jnde namelik /.}{dit gescrigt   is sulik}{dat varwe beh\r{u}ft}\\

\haiku{so / valt dicke dat /.}{sine slain in   dogen}{mit den vl\r{u}gele}\\

\haiku{Jnde als manne /.}{vindet s\r{u}nder dat lit.}{so lait manne gain}\\

\haiku{376,18 376,19 376,29 376,30 376,32}{376,36 376,38 376,40 376,43 377,2 377,2 377,11}{377,30 377,33 377,39 378,4 378,12}\\

\haiku{388,19 388,23 388,42 389,3 389,4}{389,11 389,28 389,33 389,38 390,10 390,33 390,35}{391,9 392,1 392,2 392,9 392,12}\\

\haiku{401,40 402,12 403,3 403,4 403,6}{403,11 403,11 403,13 403,15 403,18 403,34 404,6}{404,7 404,9 404,18 404,20 404,25}\\

\haiku{363,21 363,25 363,26 363,36 367,28}{368,1 368,17 369,16 370,30 370,37 371,36 372,12}{372,18 373,2 373,3 373,3 373,14}\\

\haiku{356,33 356,33 356,33 356,34 356,35}{356,35 356,36 356,37 356,40 357,1 357,4 357,4}{357,4 357,5 357,8 357,8 357,9}\\

\haiku{372,26 372,26 372,27 372,29 372,30}{372,34 372,36 372,39 372,41 372,42 372,43 373,4}{373,4 373,9 373,12 373,12 373,15}\\

\haiku{386,29 386,29 386,29 386,29 386,33}{386,33 386,33 386,41 387,1 387,6 387,6 387,7}{387,8 387,12 387,16 387,17 387,20}\\

\haiku{405,43 407,5 419,22 419,24 421,11 () () ()}{12 drag 389,71 drage 398,18}{1 dragen 356,38 370,36}\\

\haiku{366,8 366,15 366,15 372,19 (6) () (}{edelre 372,201 egen 359,35 359,35}{375,5 391,42 393,10 393,116}\\

\haiku{387,22 387,31 387,33 387,35 387,36}{387,38 387,38 387,39 388,1 388,1 388,11 388,14}{388,18 389,3 389,3 389,5 389,13}\\

\haiku{392,24 392,25 392,25 392,27 392,27}{392,33 392,33 392,35 392,35 392,37 392,39 392,42}{392,43 393,4 393,8 393,14 393,15}\\

\haiku{396,29 396,29 396,30 396,32 396,33}{396,36 396,37 396,40 397,4 397,6 397,6 397,7}{397,8 397,9 397,9 397,9 397,13}\\

\haiku{410,35 410,43 411,9 411,13 411,13 ()}{411,18 411,42 412,11 412,14 412,16 412,18 412,19}{412,2117 er 355,7 357,16}\\

\haiku{377,27 384,28 389,22 393,37 399,8 () (}{15 erliken 356,22 356,27 371,27}{372,14 387,25 394,8 397,347}\\

\haiku{392,24 392,25 392,27 392,40 394,30}{394,31 395,2 395,26 395,29 396,27 398,12 399,8}{399,33 400,8 402,3 404,23 408,14}\\

\haiku{384,43 390,35 391,7 410,5 (5) ()}{getempertheide 357,81}{getempertheit 357,12}\\

\haiku{366,34 367,1 368,25 370,20 371,39}{372,41 374,4 374,26 374,31 375,8 375,9 375,39}{376,17 376,18 377,8 377,11 377,17}\\

\haiku{372,27 373,42 382,7 382,8 382,11}{383,17 383,19 383,37 385,1 385,39 386,23 389,4}{389,39 392,11 393,21 397,38 398,17}\\

\haiku{414,15 414,34 414,36 414,39 414,41}{414,44 415,5 415,8 415,11 415,12 415,18 415,20}{415,21 415,29 415,40 416,1 416,9}\\

\haiku{373,37 373,41 373,41 373,43 374,2}{374,6 374,7 374,8 374,11 374,13 374,13 374,13}{374,24 374,40 374,40 375,12 375,20}\\

\haiku{375,29 375,35 375,44 376,8 376,30}{376,34 376,34 376,35 376,36 376,36 376,38 377,10}{377,16 378,15 378,15 378,26 378,31}\\

\haiku{420,33 421,23 421,33 421,36 422,7 () ()}{422,9 422,9 422,10 422,11327 idel 379,35}{389,3 399,93 idelen}\\

\haiku{370,26 370,27 370,28 370,29 370,31}{370,31 370,35 370,40 371,6 371,12 371,14 371,15}{371,19 371,30 371,30 371,32 371,33}\\

\haiku{414,24 (3) i\r{u}ng 367,39 367,39 () ()}{410,17 420,124 iunge 399,14 418,22}{2 iungen 385,37 388,7}\\

\haiku{378,20 378,21 378,24 378,26 378,31}{378,33 378,36 378,38 378,38 379,5 379,6 379,8}{379,8 379,9 379,11 379,17 379,30}\\

\haiku{408,27 408,31 408,35 408,36 408,38}{408,39 408,41 408,43 409,1 409,2 409,4 409,5}{409,6 409,9 409,12 409,13 409,16}\\

\haiku{) node 360,14 364,32 368,40 () ()}{392,20 399,21 408,396 nof 369,29 376,29}{376,29 407,294 nog 356,28}\\

\haiku{384,26 385,4 385,18  388,18}{388,21 391,12 391,17 400,29 401,32 405,8 408,10}{409,30 410,3 411,33 412,5 414,8}\\

\haiku{380,31 383,10 383,11 388,12 388,18}{388,32 389,8 389,12 390,7 391,37 391,37 394,23}{398,24 398,28 400,1 400,6 401,20}\\

\haiku{387,6 387,6 389,8 391,32 394,17}{396,19 397,3 401,3 401,15 403,2 403,18 404,15}{404,26 405,5 405,10 405,36 406,2}\\

\haiku{378,30 379,31 379,34 379,36 379,37}{380,10 382,2 382,15 382,31 383,15 384,8 385,1}{385,17 385,26 385,27 387,1 387,26}\\

\haiku{377,32 377,33 377,34 378,3 378,17}{378,21 378,24 378,25 378,27 379,4 379,30 380,9}{380,9 380,12 380,36 381,6 382,6}\\

\haiku{406,29 417,37 (7) simmen ()}{406,12 406,16 407,3 407,30 409,31 416,10 416,11}{416,29 417,169 sin 355,6}\\

\haiku{394,20 394,22 394,23 394,26 394,26}{394,33 394,33 394,35 395,12 395,13 395,14 395,20}{395,22 395,24 395,43 396,5 396,6}\\

\haiku{371,21 374,16 377,37 378,6 378,23}{379,36 380,25 384,17 385,42 396,7 397,7 400,38}{401,22 401,24 405,34 406,39 408,28}\\

\haiku{) tr\r{u}we 420,31 (1) tr\r{u}welik () () ()}{421,41 tr\r{u}welike 422,61}{tr\r{u}wen 415,5 421,52 tscip}\\

\haiku{362,32 362,33 (3) tsingen () () () ()}{405,11 tsloit 419,111 tspar 411,36}{1 tstert 411,37 411,382}\\

\haiku{408,26 408,39 410,9 410,10 413,4}{413,5 413,6 413,7 413,8 413,9 413,9 413,10}{413,11 413,12 413,13 413,13 413,13}\\

\haiku{366,33 370,30 372,18 377,31 385,35 ()}{385,36 392,6 407,5 413,36 417,2611 vir}{357,6 357,13 357,32 359,8 363,12}\\

\haiku{394,20 394,25 395,2 395,23 396,10}{396,11 396,19 396,33 397,23 399,8 399,20 400,8}{400,38 400,39 401,1 401,21 401,28}\\

\haiku{407,12 407,26 409,5 410,18 410,20}{410,33 411,3 411,6 411,13 411,14 411,19 411,38}{412,2 412,18 412,19 413,17 413,26}\\

\haiku{370,35 370,36 370,39 371,1 371,3}{371,7 371,9 371,11 371,15 371,21 371,23 371,35}{371,40 372,17 372,20 372,22 372,25}\\

\haiku{) wintelt 414,28 415,14 421,29 () () ()}{3 wirdet 418,141 wirken}{357,17 382,172 wirkene}\\

\haiku{373,42 391,1 399,6 399,22 404,31 ( ()}{405,32 408,208) wiuer 405,41 415,37}{2 wiuere 406,26}\\

\haiku{420,33 422,8 (30) wort 374,25 () ()}{379,29 379,30 382,6 382,9 403,41 418,47}{wragt 378,151 wrake}\\

\haiku{736165 737170 73837 ro 739175?}{740di op rasuur 741180 742i}{verbeterd uit e}\\
