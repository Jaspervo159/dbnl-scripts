\chapter[22 auteurs, 5931 haiku's]{tweeëntwintig auteurs, vijfduizendnegenhonderdeenendertig haiku's}

\section{M.H. van Campen}

\subsection{Uit: Bikoerim}

\haiku{Zij kreeg nu in haar:}{vermoeienis-doezig}{hoofd gedachten van}\\

\haiku{{\textquoteleft}Nou bent u de vrouw,.}{met de twee hoofden u kunt}{op de kermis gaan}\\

\haiku{hij zat nog goed, maar '...}{daar had jet weer met die}{plooien in de rok}\\

\haiku{En ze trad op haar, ', '.}{moeder toe zoender sloeg}{r arm om haar heen}\\

\haiku{Nu zit-ie goed,{\textquoteright}.}{zei ze zich weer oprichtend}{en nog even kijkend}\\

\haiku{En ze wendde zich,.}{om streek in wachting met de}{hand over de tafel}\\

\haiku{Tante, terwijl zij,:}{opstond en naar de tafel}{trad zei plagerig}\\

\haiku{{\textquoteleft}Och mevro\`uw, hoe kan,...}{u nou zoo'n gekheid prate}{ja we zijn op reis}\\

\haiku{{\textquoteright} En zij bleef haar 'n.}{seconde in dreigende}{afwachting aanzien}\\

\haiku{{\textquoteleft}O, da's goed, U zorgt,...}{er wel voor dat we vanavond}{kunnen afreizen}\\

\haiku{{\textquoteleft}Betsy was een heel,...}{klein meisje ze was erg stout}{en zoo'n dr{\`\i}ftkopj\`e}\\

\haiku{heel voorzichtig, dat... '... '}{je niet wakker zou worde}{enk hiel me stil}\\

\haiku{In mijn zwijgen kent...}{Gij mijn vragen en ik blijf}{aan U verbonden}\\

\haiku{God kan toch om de...}{smart van een mensch zijn eigen}{werk niet verstoren}\\

\haiku{Wat is de vreugde ', '...}{van alt mooie de liefheid}{van alt teere}\\

\haiku{n\^o, zoolang zal me..., '.}{dalles dure die brosj die}{neemt nogn sof in}\\

\haiku{{\textquoteright} {\textquoteleft}O,{\textquoteright} zei de oude,.}{vrouw en ze keek star voor zich}{uit in gedachten}\\

\haiku{hij... met z'n linksheid..., '...}{in zoo'n huis waart bepaald}{erg vroolijk toeging}\\

\haiku{{\textquoteleft}S't, s't,{\textquoteright} fluisterde,.}{mevrouw angstig den vinger}{op den mond leggend}\\

\haiku{precies...{\textquoteright} En ze moest,...}{even ophouden ze kon niet}{verder van lachen}\\

\haiku{gij zult \`alle uwe:}{levensdagen den tocht uit}{Egypte herdenken}\\

\haiku{Toch ben 'k trotsch op,...}{Z{\`\i}jn grootheid en ik b\`en niet}{arm ik b\`en niet arm}\\

\haiku{paaienden toon van,}{overdreven-geuite}{blijheid waarme\^e men}\\

\haiku{{\textquoteleft}God heeft toch alles,...}{ten beste gekeerd denkt u}{d'r maar niet meer an}\\

\haiku{Toen hief de vader:}{het smeeklied voor den herbouw}{van den tempel aan}\\

\haiku{dan heb ik weer te.}{veel gedronke en dan ben}{ik weer uit geweest}\\

\haiku{{\textquoteleft}Maar hoe moet dat nu,{\textquoteright}, {\textquoteleft}...?}{eigenlijk vroeg Roelofsen}{twee partijen of}\\

\haiku{staat dat niet ergens,?}{de wolf en het lam zullen}{te zamen grazen}\\

\haiku{{\textquoteleft}Neen 't is verkeerd,, '!}{te vroeg wij zouden gekraakt}{worden alsn noot}\\

\haiku{Stomdronken had-ie,...}{zich over het tapijt gerold}{had overgegeven}\\

\haiku{{\textquoteleft}Wilt uwes dan effen, ',}{anneme d'r isn brief}{voor u uwes ma zegt}\\

\haiku{{\textquoteright} {\textquoteleft}Och, dat geeft niets... 't.}{zal daar wel bedaren als}{de lucht wat optrekt}\\

\haiku{{\textquoteleft}O, ma, da's nou niet,.}{aardig u weet toch heel goed}{hoe ze van u houdt}\\

\haiku{{\textquoteright} {\textquoteleft}God, nee jonge, dat ', ',.}{zegk niet krijgm maar uit}{de kast daar staat-ie}\\

\haiku{{\textquoteright} {\textquoteleft}Nee, nee, ik met m'n!}{onprecieze handen in}{jou precieze kast}\\

\haiku{{\textquoteleft}Weet je nou dat ik, '!}{je h\'a\'at je niet kan uitstaan}{opt oogenblik}\\

\haiku{'t was zoo'n heerlijk}{onderwerp om d'r over te}{philosopheeren}\\

\haiku{Toch gingen zij er,, ' '.}{door op een drafjet was}{maarn klein eindje}\\

\haiku{Iets vreemds moest zich in,...}{hem hebben genesteld zich}{hebben uitgedijd}\\

\haiku{{\textquoteright} {\textquoteleft}Nee, nee, Harmse, blijf,,.}{maar ik kom boven ik heb}{iemand meegebracht}\\

\haiku{dat was 't begin......}{scheen den weg te wijzen naar}{dat onbekende}\\

\haiku{{\textquoteleft}Nee, meheer, dat zal,.}{niet gaan kan de j\'uffer niet}{na binne late}\\

\haiku{zou late blikseme,,.}{de heele keet as-ie om}{elf uur geen geld had}\\

\haiku{{\textquoteleft}Neen... bonsoir neef{\textquoteright} en.}{verdween achter de tochtdeur}{van den corridor}\\

\haiku{Even bleef hij staan, als,,:}{versuft trippelde dan de}{stoep af mompelend}\\

\haiku{Toen je van jongen,, '}{man begon te worden dacht}{k wat verdienste}\\

\haiku{Iemand, die z\'o\'o iets,...}{zonder noodzaak dee mo\'est toch}{wel krankzinnig zijn}\\

\haiku{{\textquoteleft}Nee... zoo laat 'k je...'...}{niet gaan kom hier dat  k}{je gezicht kan zien}\\

\haiku{{\textquoteleft}Maar 'k d\'ank u, 'k, '...}{d\'ank uk kan u daar nooit}{genoeg voor danken}\\

\haiku{Jaw\'el, hij wou h\'em '...}{wel eris opn winkel zien}{zitten an d\'at goed}\\

\haiku{hij sprak, dicht bij zich,,.}{te hebben ging hij naast Jaap}{loopen greep diens arm}\\

\haiku{En hij dacht nu aan,:}{de vergadering die hij}{straks zou bijwonen}\\

\haiku{Die vergadering... '}{van avond kon-ie nou ook wel}{missen als kiespijn}\\

\haiku{Jaap teuterde nog,.}{even niet kunnend besluiten}{naar binnen te gaan}\\

\haiku{Ze zou 't 'm daar}{wel an z'n verstand zien te}{brengen as Sem}\\

\haiku{En dan moet je niet '...}{zoo alle woorden opn}{goudschaaltje legge}\\

\haiku{En nou kon-ie 't...}{ook juist beter dan morge}{as-ie d'r voor stond}\\

\haiku{Nou speet 't 'm wel,,... '}{hij had ze zoo graag verteld}{dat-ie werk had}\\

\haiku{Nou... Weet je wat, laat,,}{die kist dan maar staan k'm hier}{ken je me helpe}\\

\haiku{Siesoo, geef nou m'r, '.}{de verf dan salle wem}{effe schildere}\\

\haiku{Je bint nog soo jong,, ' '...}{sal je segge maark mot}{t je toch s\`egge}\\

\haiku{En dan vraag je maar, '.}{meteen of je uit ken snije}{t is toch op slag}\\

\haiku{Och nee... dat k\'on-ie, ',...}{niet doen zij was \'o\'okn oud}{mensch zoo goed als hij}\\

\haiku{net of 'n m\`ensch je - '...}{wat gedaan h\^et toen had-ie}{n soort wrok in zich}\\

\haiku{Hij had de arme '... '...}{omhoog geslage inn}{stuip inn wildheid}\\

\haiku{Och 't verveelt me,.}{daarover te spreken daar weet}{je toch nog niks van}\\

\haiku{Dat lag voor hem als,.}{iets langs iets blinkends waaraan}{geen eind was te zien}\\

\haiku{{\textquoteleft}Wel nee jonge, dat '... '...{\textquoteright} {\textquoteleft}}{wasn Portugeesche jood}{n baronHad-ie}\\

\haiku{net of 't arme.}{schaap daar wat an  doen kon}{en dan ging ie weg}\\

\haiku{as je ergens zoo,}{lang gewees bint en je word}{dan zoo behandeld}\\

\haiku{{\textquoteright} {\textquoteleft}Nou, dat hindert toch...,{\textquoteright} '.}{niks juffrouw toe poogde hij}{r over te halen}\\

\haiku{Hield ze 'm nog voor ', '.}{n kind daar straks had ze nog}{zoo metm gepraat}\\

\haiku{och heere-j\'e in ' '.}{t werkelijke leven}{gaatt anders toe}\\

\haiku{en nou worre we......}{an de dijk geset sie nou}{maar hoe je d'r komt}\\

\section{Jan Campert en Ben van Eysselsteijn}

\subsection{Uit: Het Chineesche mysterie}

\haiku{{\textquoteleft}Geef mij 'n oorlam,{\textquoteright},;}{commandeerde het Jantje}{dat een Jaantje was}\\

\haiku{Hij vormde met de.}{beide oude heeren een}{merkwaardig contrast}\\

\haiku{{\textquoteleft}Ga werken, man of,,!}{meld je bij het armbestuur}{ingerukt marsch}\\

\haiku{{\textquoteright} {\textquoteleft}Maar u houdt maar vol,,{\textquoteright}.}{dat ik zorgen heb trachtte}{Veraart te schertsen}\\

\haiku{Hij zette den stoel.}{overeind en liep door naar de}{geopende deur}\\

\haiku{{\textquoteright} {\textquoteleft}Precies wat ik dacht,{\textquoteright}, {\textquoteleft}:}{zei dr. van Buren peinzend}{precies wat ik dacht}\\

\haiku{hij is niet op de,.}{Witte niet thuis en noch bij}{een van u beiden}\\

\haiku{Het is beter om,{\textquoteright}.}{alles te zeggen ging kalm}{van Buren verder}\\

\haiku{{\textquoteright} Oversteeg lichtte zijn,.}{hoed en wilde verdwijnen}{rechts de Javastraat in}\\

\haiku{{\textquoteright} {\textquoteleft}Da's allemaal mooi,{\textquoteright}, {\textquoteleft}:}{bromde de overstemaar het}{is zooals je zelf zegt}\\

\haiku{{\textquoteright} {\textquoteleft}Was er niets waar je,,,?}{aandacht op viel op de gang}{op de trap op straat}\\

\haiku{Je verbeeld je toch?}{soms niet dat je een halve}{Sherlock Holmes bent}\\

\haiku{Zooeven heb ik me,.}{ook al erover verbaasd dat}{je daar op inging}\\

\haiku{Even plechtig als hij,.}{gekomen was verdween de}{correcte dienaar}\\

\haiku{de vlammende haard;}{met breede clubfauteuils en}{het rooktafeltje}\\

\haiku{Geef mij het kalme.}{Voorhout en den Vijverberg}{met z'n meeuwen maar}\\

\haiku{{\textquoteright} De bediende - een -.}{Chinees gluurde even naar de}{auto en verdween}\\

\haiku{Hoeng Tsi Lang ging hen.}{voor de trap op en liet hen}{in een ruim vertrek}\\

\haiku{De garage kwam.}{op een ruime binnenplaats}{uit met tal van boxes}\\

\haiku{Hij keek op toen hij.}{hen hoorde aankomen en}{zag hen vragend aan}\\

\haiku{Misschien willen de?}{heeren even binnen komen}{en telefoneeren}\\

\haiku{We gaan nu naar de.}{politie en verzoeken}{haar medewerking}\\

\haiku{{\textquoteright} knalde een kort en.}{droog schot uit het dakraam van}{het boardinghouse}\\

\haiku{Met een paar passen.}{was hij de kamer door en}{achter het buffet}\\

\haiku{{\textquoteleft}Ik maak mijn excuus{\textquoteright},.}{voor dit binnendringen kwam}{Veraart hoffelijk}\\

\haiku{Dr. van Buren hield.}{op met trommelen en wreef}{nerveus zijn handen}\\

\haiku{De heeren willen?}{hun getuigenis wel even}{onderteekenen}\\

\haiku{Dr. van Buren stak,.}{een sigaar op maar Overste}{Mensing bedankte}\\

\haiku{{\textquoteleft}Nee, beste kerel, '.}{schei noues uit met al die}{fraaie beleefdheden}\\

\haiku{Overste Mensing liep.}{nerveus heen en weer als een}{gekooide tijger}\\

\haiku{meneer,{\textquoteright} antwoordde.}{Max en verwijderde zich}{naar de leestafel}\\

\haiku{In dien tusschentijd,}{bel ik den commissaris}{van politie op}\\

\haiku{{\textquoteright} De commissaris.}{had diepe rimpels in zijn}{voorhoofd gekregen}\\

\haiku{{\textquoteright} {\textquoteleft}Zeker,{\textquoteright} antwoordde, {\textquoteleft}?}{van Buren gehaastkunnen}{wij u van dienst zijn}\\

\haiku{{\textquoteright} De commissaris.}{deed het verhaal rustig en}{zonder eenig vertoon}\\

\haiku{{\textquoteright} {\textquoteleft}Dat is het voor ons,,{\textquoteright}.}{allemaal Dr. van Buren}{klonk koel het antwoord}\\

\haiku{Kunnen we met z'n?}{allen niet \'e\'en zoo'n Chinees}{bij zijn staart pakken}\\

\haiku{v\'o\'or dien tijd was het....}{niet anders geweest dan een}{eenvoudige moord}\\

\haiku{Informeer naar die, ', '.}{auto gam achterna}{zorg dat jem krijgt}\\

\haiku{Vooral omdat zij.}{niet in conditie zijn om}{op ijs te landen}\\

\haiku{Hij zweeg, boog zich tot.}{zijn Haagschen collega over}{en fluisterde iets}\\

\haiku{commissaris gaat,.}{u      Wie een beweging}{maakt schiet ik neer}\\

\haiku{Dit was bittere,....}{ernst \'e\'en beweging en het}{was afgeloopen}\\

\haiku{Het heeft ook geen zin.}{middelen te beramen}{om te ontkomen}\\

\haiku{{\textquoteleft}Met \'e\'en revolver....}{had ik u allen niet in}{bedwang gehouden}\\

\haiku{lange dubbele,.}{rijen waar zij over rechte}{hoofdstraten vlogen}\\

\haiku{zij knipoogden als.}{uilen in het felle licht}{van de schijnwerpers}\\

\haiku{Je hebt eieren.}{voor je geld gekozen en}{geen lawaai gemaakt}\\

\haiku{Je zult wel gelooven,.}{dat we dat niet doen voor een}{vaatje Schiedammer}\\

\haiku{Even om den hoek stond.}{een donkere gesloten}{auto aan den kant}\\

\haiku{Als hij niet oppast,;}{slaat hij met zijn sloep en z'n}{koffer te pletter}\\

\haiku{Hij stak zijn hand uit,,.}{naar de jonge mooie vrouw die}{een lichten kreet gaf}\\

\haiku{Hij snelde op de.}{jonge vrouw toe en ving haar}{in zijn armen op}\\

\haiku{Zacht legde hij haar.}{neer en steunde knielend haar}{hoofd in zijn armen}\\

\haiku{Hij liep kaarsrecht zijn,}{kamer op en neer veegde}{zijn brilleglazen}\\

\haiku{{\textquoteright} Veraart lachte en:}{ging naast Paula H\"ulshoff op}{den divan zitten}\\

\haiku{Den laatsten keer dat,:}{ik weer in den Haag kwam trof}{mij een verrassing}\\

\haiku{zijn eerste vraag was.}{naar vader's bagage en}{naar zijn handkoffer}\\

\haiku{{\textquoteright} dacht ik en ik bad.}{vurig dat hij nog tijdig}{arriveeren zou}\\

\haiku{Ik bad in stilte.}{nog vuriger dat nu Dr.}{Li toch komen mocht}\\

\haiku{Een heer wachtte op.}{mijn kamer en verzocht mij}{dringend mee te gaan}\\

\haiku{Zij kon mij dus niet.}{op de hoogte brengen van}{de documenten}\\

\haiku{Een Chinees met een.}{auto heeft onzen vriend Mr.}{Veraart opgelicht}\\

\section{Jan Campert}

\subsection{Uit: Die in het donker...}

\haiku{Er is wel eens een,.}{tijd geweest in zijn leven}{dat hij dat w\`el deed}\\

\haiku{E\'en keer kreeg hij een,.}{baantje in een sportwinkel}{voor drie  maanden}\\

\haiku{- Tot ziens, heeft ook Joost.}{Verheijde gezegd en is}{zijn dag begonnen}\\

\haiku{Met mosterd en dan.}{haalt Truus eigenhandig het}{velletje er af}\\

\haiku{Ze heeft krullende,.}{bruine haren en draagt een}{helder-witte schort}\\

\haiku{Hij gaat zitten in,.}{een der fauteuils en glimlacht}{voor het \'e\'erst sinds lang}\\

\haiku{Maar het is nog vroeg.}{in het borrel-uur en er}{zijn nog geen heeren}\\

\haiku{Reuze fideel, we '.}{drinken samenn borrel}{en nog een borrel}\\

\haiku{Joost Verheijde en.}{Toontje M\"uller hebben De}{Kakatoe verlaten}\\

\haiku{Een tram raast voorbij... -,,...}{Daar moet ik mee oppassen}{denkt Joost met die trams}\\

\haiku{Een oogenblik schuilt,.}{hij in een portiek steekt er}{een sigaret op}\\

\haiku{Hij leunt tegen den,.}{kouden muur even strijkt zijn hand}{langs de kille steenen}\\

\haiku{Ieder op eigen.}{gelegenheid naar Oome}{Daan was de afspraak}\\

\haiku{De vrouw verlaat de,.}{kamer slaat de deur met een}{slag achter zich dicht}\\

\haiku{Joost gaat zitten aan.}{de met een rood pluche kleed}{overdekte tafel}\\

\haiku{- Vooruit, steek weg, voegt,.}{Toontje M\"uller hem toe dan}{gaan we naar voren}\\

\haiku{Het is niet alleen.}{de wetenschap dat Zwarte}{Lizzy op hem wacht}\\

\haiku{- Och, zegt hij vaag, ik... -?}{weet niet Je ken toch bij mij}{blijve tot morge}\\

\haiku{- Da's m'n vaste, legt,.}{ze hem uit die komt altijd}{\`e\`en avond in de week}\\

\haiku{- Verrek jij, antwoordt,.}{Marie maar gooit evengoed het}{pilsje achterover}\\

\haiku{Maar as ik jou was!}{zou ik dat niessie~		 in}{de gaten houden}\\

\haiku{Eens in het jaar gaan,}{ik d'r altijd opzoeke}{ze weet niet beter}\\

\haiku{Dan staat ze op, hangt.}{haar mantel en hoed weg in}{het zijkabinet}\\

\haiku{Ik zou best wille',.}{eindigt de vrouw en haar oogen}{zien hem vragend aan}\\

\haiku{Hij bladert in de.}{papieren met notities}{die voor hem liggen}\\

\haiku{Hij zou best Mabel.}{nog kunnen afhalen als}{het werk gedaan was}\\

\haiku{Hij verneemt dat de.}{voorstelling om kwart over elf}{is afgeloopen}\\

\haiku{- Ik zie dat u toch,.}{niet zoo verkeerd ingelicht}{bent commissaris}\\

\haiku{Maar des middags draalt.}{het licht langer en het ijs}{in de grachten smelt}\\

\haiku{En De Koorddanser.}{die hem kent van vroeger geeft}{dan ook geen antwoord}\\

\haiku{Je luistert naar Buys,.}{Colebrander zonder te}{hooren wat hij smoest}\\

\haiku{De Koorddanser maakt?}{een onverschillig gebaar}{met zijn hand. Morgen}\\

\haiku{Als Toontje het nu...}{niet te laat maakt hier zal hij}{haar straks nog treffen}\\

\haiku{Oh, niet om het een,.}{of ander want Sjan\`etje is}{met Bill tevreden}\\

\haiku{Hij luistert naar een;}{lezing die over Daventry}{wordt uitgezonden}\\

\haiku{Als Sjan\`etje zich bij.}{een clubje voegde daalde}{de stemming zichtbaar}\\

\haiku{Daar heeft ze het bij.}{de hand en morgen komen}{ze toch om de huur}\\

\haiku{- 't Is m'n brood, zegt,.}{De Stille kortaf als zij}{er soms op zinspeelt}\\

\haiku{als het iets of  , -...}{iemand was die je te lijf}{kon gaan verlo\`or je}\\

\haiku{Je weet niet waar het,}{vandaan komt en je weet niet}{eens ho\`e het komt maar}\\

\haiku{Nee, denkt de vrouw, hij,.}{heb eigenlijk gelijk ook}{da's nergens voor noodig}\\

\haiku{- Voor de politie.}{wist U zich toch heel wat meer}{te herinneren}\\

\haiku{Niets op haar gezicht,.}{niets in haar oogen verraadt wat}{er in haar omgaat}\\

\haiku{Langzaam, voorzichtig,.}{beetje bij beetje wordt het}{net dichtgetrokken}\\

\haiku{W{\`\i}j vinden het dooden;}{van een mede-mensch een}{der grootste zonden}\\

\haiku{Huizen en water,...}{lichte nevel en zon op}{grauwe daken}\\

\haiku{Ook al moet je maar.}{zien hoe je er morgen en}{overmorgen weer komt}\\

\haiku{As je vroeger 'n '.}{klant had kwam die nog weles}{royaal over de brug}\\

\haiku{Je wordt op een avond:}{aangesproke en dan denk}{je bij je eigen}\\

\haiku{En dan die boeren.}{van tegenwoordig brengen}{alles naar de bank}\\

\haiku{Over 'n paar uur zal.}{hij terugkomen alsof}{er niets gebeurd is}\\

\haiku{Het valt hem mee, je.}{loopt niet elken dag tegen}{negen meier aan}\\

\haiku{- Da\`ag, zegt de vrouw en,.}{wrijft met haar hoofd tegen zijn}{schouder je bent vroeg}\\

\haiku{Dien middag achter.}{in Juni was de borrel}{zeer geanimeerd}\\

\haiku{Ze zijn alle drie.}{in een zeer plezierige}{stemming  geraakt}\\

\haiku{W\`el, in De Kakatoe!}{is het dien avond achter in}{Juni gezellig}\\

\haiku{- Lekkere jonge',...}{ben jij zegt de vrouw en dringt}{zich tegen hem aan}\\

\haiku{Maar Joost Verheijde -,!}{had hem w\`el in De Kakatoe}{gezien en goed goed}\\

\haiku{De dagen gaan, het.}{water ruischt en lang zijn}{de zachte nachten}\\

\haiku{Om de vreugde die.}{hij beleeft aan deze vrouw}{en aan haar liefde}\\

\haiku{En ze ziet zichzelf.}{al op het bal masqu\'e als}{Zeeuwsch boerinnetje}\\

\haiku{De vrouw die leefde.}{zonder problemen en die}{zoo gelukkig was}\\

\haiku{Er is iets, denkt De,... -?}{Stille er is iets met haar}{Is er iets gebeurd}\\

\haiku{Hij ziet alleen het}{doodelijk-witte gelaat daar voor}{hem op het kussen}\\

\haiku{Ze zal met een paar.}{andere vrouwen Zwarte}{Lizzy afleggen}\\

\haiku{Dat heeft de boerin.}{De Stille verteld en De}{Stille vond het goed}\\

\haiku{- Die, zegt De Stille,'.}{dan onverschillig die zal}{wel wijzer weze}\\

\haiku{Een jongen als De,.}{Stille en dan zonder meid}{dat is niks gedaan}\\

\haiku{Hij aarzelde even.}{toen Greet hem voorstelde bij}{haar in te trekken}\\

\haiku{Kwart over twaalf kan de,.}{man weer binnen zijn kan De}{Stille verdwijnen}\\

\haiku{Tusschen het water.}{en hem ligt vijftig meter}{sneeuw-overdekt terrein}\\

\haiku{maar hij ziet alleen,}{het doodelijk-witte gelaat daar}{voor hem in de sneeuw}\\

\subsection{Uit: Slordig beheer}

\haiku{Helen reisde 's.}{morgens vroeg met denzelfden}{trein als ik terug}\\

\haiku{Voorzichtig opende,.}{ik de deur stond in een vrij}{ruim en licht vertrek}\\

\haiku{Als Anka dien toon.}{laat hooren is er geen kruid}{tegen gewassen}\\

\haiku{Maar dat kan ook wel... -?}{aan mij liggen Moeten we}{niet eens wat gaan eten}\\

\haiku{We zijn al binnen,.}{enkele oogenblikken}{verblind door het licht}\\

\haiku{Ook kon men immers.}{nooit weten of de gezant}{je niet opmerkte}\\

\haiku{Ook de verhouding.}{met haar moeder zou dan niet}{geleden hebben}\\

\haiku{Hoe het zij, er kwam.}{een amoureuze inslag in}{onze relatie}\\

\haiku{Op den duur werd het,.}{vervelend maar je went er}{tenslotte wel aan}\\

\haiku{Ik zag dat ook nu.}{niets of niemand hem van zijn}{plan zou afbrengen}\\

\haiku{{\textquoteright} Voor het overige}{was hij redelijker dan}{ik hem ooit  had}\\

\haiku{Waarom heb ik ooit?}{de moeite genomen om}{dit op te schrijven}\\

\haiku{Eindelijk val ik,,...}{op mijn bed neer stort in een}{looden droomloozen slaap}\\

\subsection{Uit: Wier}

\haiku{Het kind zit op de.}{vloer te spelen met een paar}{blokken en een stoof}\\

\haiku{Even nog wacht Tanne.}{totdat zij zekerheid heeft}{dat het kind slaapt}\\

\haiku{Traag wendt zij het hoofd:}{terzijde en half over de}{schouder heen vraagt zij}\\

\haiku{Voor het eerst die nacht,,.}{lacht Tanne Ingelse een}{kleine korte lacht}\\

\haiku{Aarzelend welhaast.}{beroeren haar vingers het}{zachte kinderhaar}\\

\haiku{Arjaan onderbreekt,.}{even zijn werk haalt minachtend}{snuivend zijn neus op}\\

\haiku{Er ligt een tjalk met,,.}{stenen die gelost worden}{bij het Grote Hoofd}\\

\haiku{Daarbij is hij lid.}{van de kerkeraad en van}{de gemeenteraad}\\

\haiku{Een  klein domein.}{dat hij verdedigen zou}{tot het uiterste}\\

\haiku{De enige, die er,.}{wat aan doen kan is Cysouw}{en die verdomt het}\\

\haiku{En de grond zou, met,.}{God's hulp goed voor hem zijn en}{de pacht opbrengen}\\

\haiku{En als zij elkaar.}{al eens ontmoetten dan had}{Wanne altijd haast}\\

\haiku{Maar dan zal Jaap hem.}{zeker zien en de honden}{op hem aanhitsen}\\

\haiku{Maar Gekke Floris,...}{knikt heftig van neen klemt de}{lippen op elkaar}\\

\haiku{Hij ziet een zwarte.}{vlek afsteken tegen het}{lichte duinzand}\\

\haiku{Tanne keek soms uit.}{die ogen van d'r alsof er}{niets gelegen kwam}\\

\haiku{Gabe zet zijn fiets.}{in een stalling en slentert}{wat door de straten}\\

\haiku{Hij weet eigenlijk.}{niet goed wat hij nu met die}{uren beginnen moet}\\

\haiku{Dan zat hij trots naast}{vader en klakte met zijn}{tong net als vader}\\

\haiku{De kastelein schijnt.}{zo'n  beetje te dutten}{achter de toonbank}\\

\haiku{En hij stelt er ook,.}{niet zo heel veel belang}{in eerlijk gezegd}\\

\haiku{Lou van Zakke en.}{Loes zijn in een fluisterend}{gesprek gewikkeld}\\

\haiku{De kleuren van helm,,,.}{grassen duindoorn distels en}{zand vloeien ineen}\\

\haiku{Zij neemt het kind van,.}{hem over dat hij behoedzaam}{in haar armen legt}\\

\haiku{Gekke Floris zit.}{op de bank voor het huis van}{Tanne Ingelse}\\

\haiku{Die heeft altijd d'r.}{eigen wil gehad en d'r}{eigen zin gedaan}\\

\haiku{Lou broeit hoe hij over.}{wat hij vanmiddag ontdekt}{heeft kan beginnen}\\

\haiku{Zonder dat zij het.}{zelf merken zijn hun woorden}{luider geworden}\\

\haiku{Al vertellen zijn.}{woorden dan ook precies hoe}{alles gebeurde}\\

\haiku{Vroeger zou hij niet.}{zo licht met iemand als Lou}{op stap zijn gegaan}\\

\haiku{En dan gaene me' '.}{morrege noges mee die}{van Verdeene uut}\\

\haiku{Tussen een uit stad '.}{en een vant dorp is het}{nog nooit goed gegaan}\\

\haiku{Al moet je de winst,.}{dan ook delen dit heeft toch}{ook zijn voordelen}\\

\haiku{Als het zo te pas}{komt en burgemeester Van}{Ryssel kennend zal}\\

\haiku{De boer zet zijn bril.}{op en slaat de bijbel open}{bij de leeswijzer}\\

\haiku{Zij zal een pronte,.}{boerin zijn op Nooit Gedacht}{denkt hij bij zichzelf}\\

\haiku{Belangrijker is.}{of de mens wandelt in de}{vreze des Heren}\\

\haiku{Ze hebben op het.}{dorp altijd medelijden}{gehad met Sanne}\\

\haiku{Die is eigenlijk,,.}{zo zeggen ze op het dorp}{van hetzelfde slag}\\

\haiku{Zie je, zeiden die,.}{op het dorp dan dat kind wordt}{noe al eenzelvig}\\

\haiku{In breed en machtig.}{zwieren tegen de hemel}{Lein kijkt de zot aan}\\

\haiku{- Jezis-Maria,,'...}{fluistert ze in haar angst daer}{komme rampen van}\\

\haiku{- Den '\'ond, zegt ze zacht,, '...}{hardnekkig en vol van haat}{die verdoemdenond}\\

\haiku{Soms denkt Kee wel eens.}{dat zij dokter maar zijn zin}{had moeten geven}\\

\haiku{Er brandt nog licht in,.}{de gelagkamer maar de}{deur is al op slot}\\

\haiku{naer je wijf en je, '?}{huus waerom zit je dan}{ier in het Waepen}\\

\haiku{Ge hebt er meer zorg,.}{en verdriet van dan aarigheid}{denkt hij bij zichzelf}\\

\haiku{Als hij een avond in.}{stad wil blijven dan  blijft}{hij een avond in stad}\\

\haiku{Want tenslotte, d'r.}{zijn er niet velen op het}{dorp met zo'n toestel}\\

\haiku{En Lou is er de:}{man niet naar om tegen wie}{maar wil te zeggen}\\

\haiku{Dat had ook  zo.}{zijn praktiese kant als ge}{met z'n twee\"en zijt}\\

\haiku{Want waar bleef hij zo.}{gauw met een vrachtje van het}{een of het ander}\\

\haiku{Daar had hij dan wel,?}{niet aan gedacht maar waarom}{zou hij niet meegaan}\\

\haiku{Aan de andere.}{kant der duinen heeft men er}{minder hinder van}\\

\haiku{Een korte strekke,.}{verderop ligt de pan die}{hij op het oog heeft}\\

\haiku{Vandaag zal Wanne.}{Cysouw met die oudste van}{Hubrechtse trouwen}\\

\haiku{En op het dorp zijn,.}{er maar weinigen die het}{geen goed stel achten}\\

\haiku{Voor Lena, de meid,.}{van de Olmenhoeve zijn}{het grote dagen}\\

\haiku{Tegen Kees, de knecht,.}{toen ze samen  op de}{deel bezig waren}\\

\haiku{Zij vertelt het niet,.}{alleen aan de boer maar aan}{al de anderen}\\

\haiku{Er gingen hele.}{dagen voorbij dat haar beeld}{niet in hem opkwam}\\

\haiku{Ergens in een der.}{hoeken staat een fust bier om}{de dorst te verslaan}\\

\haiku{hij knikt maar met zijn,.}{rode verhitte kop van}{ja en schudt van neen}\\

\haiku{Als er alleen de,.}{boeren waren dan zou het}{zo'n vaart niet lopen}\\

\haiku{Hij steunt met zijn hand,.}{op een stoel zijn ogen staan hard}{en vol dreigement}\\

\haiku{Die van Aagtekerke.}{en het andere knechtvolk}{dringen naar voren}\\

\haiku{Achter hen, midden,.}{op straat probeert Lein Lap zich}{staande te houden}\\

\haiku{Arjaan kan van die.}{slimmigheden hebben als}{een volwassene}\\

\haiku{Andere malen,.}{is het ver in de nacht dat}{hij aan komt zetten}\\

\haiku{Die is, hoe lang zij,.}{ook op het dorp woont haar jeugd}{nog niet vergeten}\\

\haiku{En zo\"even nog}{toen hij zijn hof verliet wist}{hij niet precies h\'oe}\\

\haiku{Zo dadelijk kan.}{zij nu haar bedrijvigheid}{niet terugvinden}\\

\haiku{Na jaren kwam soms.}{een vreemdeling de weg naar}{het kerkhof vragen}\\

\haiku{Hier en daar in het.}{dorp neemt een vrouw de blinden}{van de vensters weg}\\

\haiku{Zij zit rechtop op.}{de met leer overtrokken bank}{die langs de muur staat}\\

\haiku{Daar behoeft ge niet,.}{aan te raken want wijken}{doen zij geen duimbreed}\\

\haiku{Zij likt aan duim en.}{wijsvinger en telt het geld}{voor de mannen uit}\\

\haiku{- Hei-hei, valt Gabe,}{haar in de rede en ik}{doch dat je toch na\^er}\\

\haiku{Waarachtig, z\'o heeft.}{Marie nog maar liever dat}{die twee opkrassen}\\

\haiku{- Tj\`essis, zegt Marie,'.}{met veel luidruchtigheid wat}{binne jullie stil}\\

\haiku{advocaat uit haar:}{glas en zegt zo argeloos}{als een vrouw dat kan}\\

\haiku{Dat komt wel later,.}{tegen de tijd dat ge een}{mens eerst missen gaat}\\

\haiku{Tegen zes uur in.}{de morgen sterft de boer van}{de Olmenhoeve}\\

\haiku{- ... van zwa\^er eiken en... -?}{met zuiveren beslag Met}{zuiveren beslag}\\

\haiku{Maar op een dag als.}{vandaag ligt het niet alleen}{aan de jaren}\\

\haiku{Er zal temet nog,.}{meer komen vallen daar kunt}{ge gerust op zijn}\\

\haiku{Die laat nog na zijn.}{dood goed merken dat hij een}{man van gewicht is}\\

\haiku{Die slaapt zijn laatste,.}{slaap onder een hemel grauw}{en dik van sneeuw}\\

\haiku{- Laat 'es zien, Vader,,...}{zegt Marie liefjes wat ge}{aan te bieden hebt}\\

\haiku{Daarmee is ze sinds.}{enige maanden bij Gabe}{niet aan een goed adres}\\

\haiku{Voordat de ander,,.}{de list doorziet springt hij naar}{voren grijpt hem beet}\\

\haiku{Zij gaat ermee naar,.}{de pomp en reinigt het geeft}{het hem dan terug}\\

\haiku{Maar moeizaam tast hij,,.}{de weg verder wankelt even}{herstelt zijn evenwicht}\\

\haiku{Het geluid van de.}{boei voor de kust loeit dof en}{gonzend in hun oren}\\

\haiku{- 't Zal morgen 'n'',...}{goeie dag wizze om ons land}{te spitte Ga\^obe}\\

\section{Bernard Canter}

\subsection{Uit: Kalverstraat}

\haiku{Eduard moest dokter.}{worden omdat het deftig}{was dokter te zijn}\\

\haiku{Twee duizend gulden.}{bruidschat had ze van den vader}{mede gekregen}\\

\haiku{Dat genot moet ik.}{nu waardeeren als een geluk}{zegt rebbe Zadik}\\

\haiku{Toen was ze op een.}{dag zoek geraakt en sedert}{stond Mantua alleen}\\

\haiku{Mantua was alleen.}{met zijn schilderskistje naar}{Vincennes gereisd}\\

\haiku{Hij schilderde met.}{een zeker overleg en met}{een zekeren dwang}\\

\haiku{Daar in 't midden,,,,.}{tusschen twee kolommen stond}{strak rechtop een vrouw}\\

\haiku{{\textquoteright} {\textquoteleft}Nou 'k zou maar met ', '.}{m oppassent Lijkt me}{toch al zoo'n rare}\\

\haiku{Wat anders kon ze,.}{niet zingen omdat ze dat}{nog niet geleerd had}\\

\haiku{De moeder had haar '.}{dadelijk een slag int}{gelaat gegeven}\\

\haiku{Halma keek even naar:}{de wonde in den nek en}{dadelijk lachend}\\

\haiku{wij zijn allemaal ..,}{menschen en Dani\"el m\`ot}{men veel vergeven}\\

\haiku{Men krijgt crediet en - '...}{men geeft crediett een reikt}{het ander de hand}\\

\haiku{{\textquoteright} Er bleven telkens.}{menschen buiten voor de groote}{glazen ruiten staan}\\

\haiku{Zijn huis was immers...{\textquoteright} {\textquoteleft}...{\textquoteright} {\textquoteleft} .....}{vaders huisDag vaderDag}{David ben je daar}\\

\haiku{Weet u wat met een......}{g\`assene weggaat minstens}{driehonderd gulden}\\

\haiku{Het deed hem z\'e\'er in ',.}{t hart dat hij den ouden}{man moest afschepen}\\

\haiku{{\textquoteright} {\textquoteleft}Nou... vooruit dan maar...{\textquoteright}, {\textquoteleft},}{Souget klom haastig de trap}{opEen bestdoener}\\

\haiku{Ze wierp de bundel,}{bloemen in den linkerarm}{stond stil op den weg}\\

\haiku{Als hij zag, dat de,.}{bloemen goed gingen had hij}{w\'eer een ander plan}\\

\haiku{{\textquoteright} Maar in haar hart had.}{ze pret in dat boenen en}{zeepen en schrobben}\\

\haiku{En later, toen je...}{bij ons kwam en toen je het}{marmer schilderde}\\

\haiku{daareven heb ik de,...}{Moedermaagd voor mij gezien}{die jou beschermde}\\

\haiku{Hij ademde vol en,.}{krachtig met een gevoel van}{stage zaligheid}\\

\haiku{Zij greep hem om zijn .....! ..!...}{arm vast keek voor zich uit met}{een verrukt o o}\\

\haiku{{\textquoteleft}Omdat het anders,.}{is alsof er iets in je}{gezicht gebluscht wordt}\\

\haiku{{\textquoteleft}Waarom ben je van, '?}{middag niet gebleven toen}{ik  jet zei}\\

\haiku{Hij begon er over,.}{te denkenvan betrekking}{te veranderen}\\

\haiku{Hij wou niet meer met...{\textquoteright}}{hun wissewassen lastig}{gevallen worden}\\

\haiku{Suzanna liep hem,.}{angstig tegemoet vroeg of}{hij ziek geweest was}\\

\haiku{Hoe zou het dan met...{\textquoteright} {\textquoteleft}!}{den eerlijken handel gaan}{Eerlijke handel}\\

\haiku{Hij stapte op de,.}{twee toe rukte Mantua met}{kracht van Treesje weg}\\

\haiku{O, juist toen dat vuil,...}{tegen zijn hoed aankwam had}{zij zijn oogen gezien}\\

\haiku{Z\'o\'o moesten de oogen van,...}{Jezus geweest zijn toen hij}{voor Pilatus stond}\\

\haiku{Als hij nu eens niet.}{dat accept van Dietrich en}{Cohn betaald had}\\

\haiku{D\`at mocht hij, David,!}{de Leeuw zich wel veroorlooven}{voor al zijn zorgen}\\

\haiku{{\textquoteleft}Vader, ik kom nog...{\textquoteright} {\textquoteleft}...}{eens spreken over de bruiloft}{Spreek er niet meer over}\\

\haiku{Toen kon het niet en...{\textquoteright} {\textquoteleft}?}{nu kan het welEn hoeveel}{kan je dan geven}\\

\haiku{Alle arbeid schoof,.}{hij van zijn hals hij moest voor}{de bruiloft zorgen}\\

\haiku{t Is mooi geweest, ', '...}{t Is mooi geweestt Is}{bliksems mooi geweest}\\

\haiku{Op het bevel van,,...,,...}{\'e\'en tw\'e\'e drie hokus pokus}{pons mijnheer Goudsmit}\\

\haiku{Maar als 't vandaag, '...}{niet was gebeurd wast mij}{morgen overkomen}\\

\haiku{Hebben ze ook met?}{honger thuis gezeten met}{vrouw en kinderen}\\

\haiku{Ze zaten stil in ',.}{t wiebelende rijtuig}{alle drie vermoeid}\\

\haiku{Sommigen zeiden,.}{dat Nauman h\'e\'el rijk was en}{eigen huizen had}\\

\haiku{Hij was taai... hield zich.}{langer boven water dan}{Nauman gedacht had}\\

\haiku{k Heb wat maagzout...{\textquoteright}.}{ingenomen David had}{hem eens aangezien}\\

\haiku{Harde uren had hij, '.}{doorgebrachts nachts wakend}{in zijn kantoortje}\\

\haiku{Zij zou weldra voor.}{een beslissend oogenblik}{in haar leven staan}\\

\haiku{Hij had zich in zijn.}{hart week en tot schreiens toe}{bewogen gevoeld}\\

\haiku{{\textquoteright} Maar hij kon voor 's.}{avonds niet weg uit zijn zaak en}{zoo ontging zij hem}\\

\haiku{Franscli sprak ze en.}{Duitsch als water en zij zong}{als een chanteuse}\\

\haiku{Ondergeschikt moest,,.}{zij worden zij de dochter}{van David de Leeuw}\\

\haiku{Je zei, dat het hier,,!}{goedkoop was maar dat is veel}{te duur veel te duur}\\

\haiku{Straks zullen wij de.}{mooie beelden uit het Grieksche}{bloeitijdperk gaan zien}\\

\haiku{Dat anderen, dat...?}{vreemden zoo maar jou lichaam}{zouden kunnen zien}\\

\haiku{Voelde hij niet, dat?}{dat een beleediging was voor}{hemzelf en voor haar}\\

\haiku{Dan kan ik uw werk, {\textquoteright}.}{w\`el trachten plaatsen had \'e\'en}{kunstkooper gezegd}\\

\haiku{Je mag mij op je,...}{knie\"en danken dat ik je}{d\`at laat verdienen}\\

\haiku{Als 't heelemaal,.}{uit de lucht gegrepen was}{had je er niets aan}\\

\haiku{Hij hoorde dicht bij....}{zich spreken in een vreemde}{taal luisterde toe}\\

\haiku{Een lange, slanke,,...}{man met een grijzende baard}{sprak met een meisje}\\

\haiku{Jongen, jongen, dat,,...{\textquoteright} {\textquoteleft}...}{duurt lang hoor langDen langsten}{tijd heeft het geduurd}\\

\haiku{hem doodsbleek en met.}{behuilde oqgen in den}{winkel te gemoet}\\

\haiku{Mantua Fresco was.}{in de stad geweest met dat}{sjieksie van Vlissingen}\\

\haiku{Als jij je niet wat,,...}{tammer houdt opvreetster gooi}{ik jou de straat op}\\

\haiku{Hij hoorde haar aan,,...}{keek naar haar mooie gestalte}{haar vollen boezem}\\

\haiku{Hij overlegde, was,.}{besluiteloos keek nogmaals}{naar heur gestalte}\\

\haiku{Nu, zeg nou nog 'ris,.}{dat er geen rechtvaardigheid}{in de wereld is}\\

\haiku{Als men 'm nog kwaad,...}{an had gezien had-ie mij}{de zaak uitgezet}\\

\haiku{Zoolang die twee daar,,.}{boven jammerden had hij}{zijn huis gemeden}\\

\haiku{En alle week komt,}{de dokter driemaal zoodat ze}{bij mij niet zoo gauw}\\

\haiku{Toen ze hem zagen,,.}{schenen beiden te schrikken}{weken voor hem uit}\\

\haiku{een paar handen vol.}{van en toen duwde-ie er}{een paar in me mond}\\

\haiku{Altijd die vuile.}{bloedboel rakkere en die}{stinkrommel schrobben}\\

\haiku{op de lange duur.}{zat je de heelen dag je}{maar te vervelen}\\

\haiku{Wil je wel gelooven, '.}{datk nog net zeventien}{centen in huis heb}\\

\haiku{En drink 'n slokkie...{\textquoteright} {\textquoteleft}............}{Een blous een blauwe blous met}{opslagen fijn hoor}\\

\haiku{En wanneer ze ook, '.}{herkend werden dan nog kon}{t hem niets bommen}\\

\haiku{As je een talhout,...}{op m'n brood leit eet ik het}{op voor zoete koek}\\

\haiku{Voor {\textquoteleft}Isra\"el en{\textquoteright}.}{Oranje had-ie hem zoowaar een}{tientje gegeven}\\

\haiku{{\textquoteright} {\textquoteleft}Is 't u bekend, '.}{waar de heer Ricardi op}{t oogenblik is}\\

\haiku{Hij spiegelde zich,.}{in de ruit van een winkel}{die hij voorbijliep}\\

\haiku{en mevrouw De Leeuw,.}{God geve u dat geluk}{bij al uw dochters}\\

\haiku{Wacht... ze zullen zien...}{dat er nog rechtvaardigheid}{in de wereld is}\\

\haiku{{\textquoteright} riep hij, door den toon,.}{den bezoeker uitnoodigend}{binnen te komen}\\

\haiku{Een vuil stukkie hout,.}{waar men negen jaar op krast}{om w\`at te worden}\\

\haiku{Ik heb hier dikwijls,.}{door de straat geloopen dat}{ik gewanhoopt heb}\\

\haiku{Blz. 105 regel 1 {\textquoteleft}{\textquoteright}.}{te lezen achterverteld}{en het woord hij}\\

\subsection{Uit: Twee weken bedelaar}

\haiku{Ik trek een oude,.}{grijze wollen trui aan aan}{den hals uitgescheurd}\\

\haiku{Daarover gaat een vest,,.}{voor een veel dikker persoon}{dan ik ben gemaakt}\\

\haiku{En meteen rinkelt,.}{ze de deur dicht dat de tocht}{koel langs mij heen waait}\\

\haiku{Je bent duur vanavond...{\textquoteright} {\textquoteleft}',,{\textquoteright}.}{t Is f\"ur morgen f fr\"uh}{smeekte ik heesch}\\

\haiku{V\'o\'or 't bed, links in.}{den hoek stond een stoel en ik}{ging mij ontkleeden}\\

\haiku{In het bed van de.}{klagende vrouwestem kwam}{nu ook beweging}\\

\haiku{Hij had zich pas op '.}{t plaatsje gewasschen en}{zijn haar was nog nat}\\

\haiku{En net op die stoel,...}{die ik de vorige week}{heb laten matten}\\

\haiku{Ik nam den cent aan, '.}{stopte hem int kleine}{zakje van mijn jas}\\

\haiku{Aan den weg stond een,.}{bedelaar met een houten}{been een \`echt gebrek}\\

\haiku{Wat dorre bruine,.}{bladeren dreven niet weg}{zoo windstil was het}\\

\haiku{'k Nam mijn kruk en,.}{mijn postpapier strompelde}{weer naar hen terug}\\

\haiku{Vijf f, f, f, fel,...,}{voor een st stuiver met}{v vijf \`e \`e \`e}\\

\haiku{Opende een kleine, '.}{portemonnai\'e gaf het kind}{wat int handje}\\

\haiku{En 'k mot mijn vrouw.}{van de week nog twee kwartjes}{sturen voor de huur}\\

\haiku{Ja, terwijl ik hem,.}{aanzag herinnerde ik}{mij zijn beschrijving}\\

\haiku{Hij nam de doos aan.}{een band traag om den arm en}{ging sluik de deur uit}\\

\haiku{De gelagkamer,.}{was wazig van den rook die}{uit de keuken kwam}\\

\haiku{{\textquoteleft}Ja, 't is mij soms.}{net of die heele boel van}{binnen samentrekt}\\

\haiku{Nou, wij hebben er,.}{met z'n drie\"en een maal aan}{gehad een fijn maal}\\

\haiku{Als-ie 's avonds op,.}{de kamer ligt verpest-ie}{de heele kamer}\\

\haiku{Doch zij verweten.}{elkaar de meest intieme}{bijzonderheden}\\

\haiku{En ik zeg je, as ',.}{jem nog is anraakt trap}{ik je uit mekaar}\\

\haiku{Dan ga je rechtdoor.}{en dan rechts af en dan weer}{recht door langs de tram}\\

\haiku{De juffrouw bleef even,...}{talmen kierde de deur een}{beetje verder open}\\

\haiku{{\textquoteleft}Een cent goedkooper,{\textquoteright} '.}{zeik nogmaals zeer ernstig}{en nadrukkelijk}\\

\haiku{{\textquoteleft}Mijn vader geeft 'r,{\textquoteright}.}{een beetje boter op d'r}{brood lachte het kind}\\

\haiku{{\textquoteright} {\textquoteleft}Ja moeder, genoeg,{\textquoteright} ',.}{mompeldet knaapje nog}{altijd kruiperig}\\

\haiku{{\textquoteright} Jantje keek mij met.}{zijn fletse maar eerlijke}{oogen verwonderd aan}\\

\haiku{Ik ging heel dicht naar,...}{de golven tot ze even mijn}{schoenen bespoelden}\\

\haiku{Maar zij verkwijnen,,.}{langzaam als een plant die te}{weinig water krijgt}\\

\haiku{Kom man, ze poese '.}{ze zelfs morgens om een}{fooi uit te spare}\\

\haiku{Ze het polsies man '.}{asn pijpesteel en ze hoest}{as een simpanzee}\\

\haiku{Nou, en ik telkens,.}{vijf kwartjes voor me loopie}{dat rekende \`an}\\

\haiku{hoor,{\textquoteright} en toen vloekt-ie.}{me uit voor stommerd en de}{smerigste vloeken}\\

\haiku{Je hebt 'm ze toch...{\textquoteright} {\textquoteleft},.}{uit zijn handen genomen}{Nou dat begrijp je}\\

\haiku{Ieder gaat maar naar...{\textquoteright} {\textquoteleft},?}{den IndiesmanNou en wat}{hindert het ze dan}\\

\haiku{Maar deze zocht op,.}{zijn beurt een slachtoffer dat}{Piet natuurlijk werd}\\

\haiku{Maar ik ken 'r dan,,.}{honderden die God danken}{as de kou wegblijft}\\

\haiku{an d'r lijf...{\textquoteright} {\textquoteleft}Nou, die '...{\textquoteright} {\textquoteleft}!}{kenne naart Toevlucht gaan}{Of naar meneertje}\\

\haiku{Daar ben ik voor in ',}{doodsgevaar geweest en daar}{heb ikt gal door}\\

\haiku{Na de soep werden.}{borden met aardappelen}{en vet rondgedeeld}\\

\haiku{Zoodanig en op.}{deze wijze spraken de}{lieden tot twaalf uur}\\

\haiku{{\textquoteleft}Geef me nou toch de,. ' '.}{emmer baask Zalt straks}{zelf wel schoon maken}\\

\haiku{Je hebt je voor den....}{dnder hier in mijn huis niet}{zoo in te zeepen}\\

\haiku{Wie in mijn huis of,.}{in mijn bedden d\`at vindt heeft}{het zelf meegebracht}\\

\haiku{{\textquoteleft}Dank-je Gasman en '.}{hier is een gulden voort}{schoene-poetse}\\

\haiku{As jij d'r onder, '...{\textquoteright} {\textquoteleft}}{leit wil ikr nog wel als}{weduwvrouw hebben}\\

\haiku{{\textquoteright} En hij laat mij een, '.}{portefeuille zien z\'o\'o hoog}{vant bankpapier}\\

\haiku{De menschen kenne,.}{je niet bijhouwe die je}{wat wille geve}\\

\haiku{Vier duiten in 't '.}{zakje en vier duite op}{straat ann arm mensch}\\

\haiku{hier heeft u uw geld,,}{terug as je meent dat ik}{loopen kan Loopen}\\

\haiku{Het vrouwtje was den.}{hoek al om en een eindje}{verder op de brug}\\

\haiku{As u gelooft, dat,.}{ik goed loopen ken moet u}{uw geld maar houden}\\

\haiku{{\textquoteright} Laat ons niet alle,.}{reglementeering alle}{orde verwerpen}\\

\haiku{Ja, Den Haag was mooi,,.}{was warm was innig op dien}{heerlijken herfstdag}\\

\haiku{'k Mot nou ook maar, '.}{hopen datk weer een goeie}{boer onderweg tref}\\

\haiku{'k Had mijn verstand,.}{motte gebruiken toen ik}{bij den ouwe was}\\

\haiku{As de huisbaas komt, '.}{en z'n cente ligge klaar}{dan ist toch goed}\\

\haiku{Hoeveel hadden die,?}{menschen ontvangen hoeveel}{verdiend op \'e\'en dag}\\

\haiku{Vind morgen weer drie,...}{gekken die je een kwartje}{geven voor zoo'n pop}\\

\haiku{Daarom trok ik een,:}{pijnlijk gezicht zette het}{kopje neer en zei}\\

\haiku{{\textquoteright} Zij had het kwartje.}{aangenomen en mij een}{bed aangewezen}\\

\haiku{Ik sloeg de dekens,.}{op stak een lucifer aan}{en bekeek het bed}\\

\haiku{'t Kind pakte de,.}{pop niet aan maar scheeloogde}{schuw naar zijn moeder}\\

\haiku{{\textquoteright} Het knaapje nam de.}{pop aan en ging met pop en}{kous naar zijn moeder}\\

\haiku{{\textquoteleft}Waarom huil je weer,...{\textquoteright} {\textquoteleft},....}{gedrochtIkke ikke heb}{niet gegegeten}\\

\haiku{Zij droeg een fluweel,.}{jaquette een zwart lustren rok}{op breede heupen}\\

\haiku{Hij nam het \'etui aan,,.}{haalde het horloge er}{uit deed de kast open}\\

\haiku{Zij waren moe, schor,,.}{riepen machinaal maar zij}{riepen nog altijd}\\

\haiku{k Had vandaag nog.}{niks gehad dan een stuk oud}{brood van een dienstmeid}\\

\haiku{'k Mot twaalf stuivers,.}{thuis brengen anders laten}{ze me er niet in}\\

\haiku{De knecht is ziek en.}{daarom moet ik vanavond voor}{alles zelf zorgen}\\

\haiku{{\textquoteright} En hij wees op een,:}{agent in uniform die ook heel}{gemoedelijk zei}\\

\haiku{heilig gevoel, dat.}{der persoonlijke vrijheid}{van denken aanrandt}\\

\haiku{{\textquoteleft}Daar mannetje, als,.}{je er dat in hebt zul je}{geen trek meer hebben}\\

\haiku{Ik ben vanmorgen...{\textquoteright} {\textquoteleft}!}{door een agent uit mijn hotel}{gehaaldZijn hotel}\\

\haiku{De cither-speler.}{was er niet meer en ik heb}{hem niet meer gezien}\\

\haiku{Een der mannen stond.}{plotseling op en hield de}{lamp met de hand stil}\\

\haiku{De kachel brandt, zooals.}{je ziet en wij hebben geen}{cent van iemand noodig}\\

\haiku{Deze nacht, besloot,.}{ik zou mijn laatste van mijn}{bedelaarsreis zijn}\\

\haiku{Ik ging nu voor 't,.}{raam staan liet een pop dansen}{voor de kinderen}\\

\haiku{Je hebt meer kans mij,......,...}{te zien dan die dnderhnd}{die je vanmorgen}\\

\haiku{Ik verliet het huis,.}{ging dineeren en tegen half}{elf kwam ik terug}\\

\section{Willem Capel}

\subsection{Uit: 'Gl\"uck auf,' kompeltje! Roman uit het mijnwerkersleven in Nederland}

\haiku{Dat ging zoowat altijd,,,;}{door uur na uur dag en nacht}{winter en zomer}\\

\haiku{Toen Kompeltje aan {\textquoteleft}{\textquoteright},.}{de mijnMaurits dacht schrok hij}{op uit zijn gepeins}\\

\haiku{Daar zaten Vader ';}{en Moeder ook zoos avonds}{plannen te smeden}\\

\haiku{Toch wist hij heel goed,,.}{hoe hij het zeggen moest maar}{hij durfde het niet}\\

\haiku{Hij hoorde zijn vrouw;}{in het achterkamertje}{tegen Sjef bezig}\\

\haiku{In het kleedlokaal,.}{trok Vader al zijn kleeren uit}{zelfs zijn ondergoed}\\

\haiku{De lift zakte weer.}{wat en Vader hoorde het}{debat boven zich}\\

\haiku{Geen enkel accoord is,:}{gelijk maar overal wordt op}{de minuut gewerkt}\\

\haiku{Langzaam stond Vader,;}{op een lichte kreet kon hij}{niet onderdrukken}\\

\haiku{De lucht ziet nu mooi.}{blauw en de koude is niet}{zoo hevig  meer}\\

\haiku{{\textquoteleft}Sla je goed af,{\textquoteright} maant, {\textquoteleft}.}{Vaderdat die rotzooi niet}{mee naar binnen komt}\\

\haiku{het kon best zijn, dat.}{Stephan over een week of wat al}{weer aan het werk mocht}\\

\haiku{hij zag haar rose,;}{directoirtje waarover zij}{gauw haar rokje trok}\\

\haiku{{\textquoteleft}Ik kan nu niet naar,,}{Lutterade met die sneeuw ik}{ga maar na Nieuwjaar}\\

\haiku{de rest lag als een.}{zwarte pyramide in}{het witte landschap}\\

\haiku{Razend snel ging het.}{nu door de straatjes van de}{kolonie naar huis}\\

\haiku{Hij gleed weer met Thea,.}{de helling af voelde zich}{weer heel dicht bij haar}\\

\haiku{Dan kun je van je,.}{kinderen afhangen of}{naar het armbestuur}\\

\haiku{Van de 37 duizend,.}{mijnwerkers zijn er maar 13.500}{georganiseerd}\\

\haiku{De reprimande,.}{die hij in ontvangst heeft te}{nemen is niet malsch}\\

\haiku{De steengang kan nu.}{weer een eindje verder door}{getrokken worden}\\

\haiku{Het puin zal worden.}{opgeruimd en Vader zal}{de bouwen zetten}\\

\haiku{Vader helpt even mee,.}{dan kan hij eerder aan het}{stutten beginnen}\\

\haiku{{\textquoteright} luidde de eerste:}{vraag en het antwoord las de}{pastoor langzaam voor}\\

\haiku{Deze knikte slechts:}{en Kompeltje nam Sjeng eens}{van terzijde op}\\

\haiku{Het was een glad, rond,.}{stuk steen z\'o\'o mooi glad alsof}{het geslepen was}\\

\haiku{Wat is goed beschouwd?}{het werkelijke leven}{van een mijnwerker}\\

\haiku{Vandaag of morgen.}{loopt ie tegen een meid aan}{en trekt het huis uit}\\

\haiku{{\textquoteleft}Ik heb spijt meester,.}{dat ik niet eerder naar U}{toegekomen ben}\\

\haiku{Kompeltje loog, dat,.}{ie zoo weer terug was hij}{moest even naar de grens}\\

\haiku{'t Was bladstil en.}{de rook uit de hooge schoorsteenen}{kroop loodrecht omhoog}\\

\haiku{Achter dat hek was,.}{nu Duitschland maar de grens}{viel Thea erg tegen}\\

\haiku{{\textquoteleft}Een mooi hoofd en glad,.}{geschoren kin Daar steekt des}{mannes schoonheid in}\\

\haiku{Zwaans had Kompeltje,.}{een stille wenk gegeven}{en deze begon}\\

\haiku{Een meidenbaantje,,.}{is het een vak voor iemand}{met een bult barst nou}\\

\haiku{Aan die jas, zat nog,.}{een soort van kap die je over}{je kop kon trekken}\\

\haiku{dat had je er nou,.}{van als je een leerling uit}{dezelfde plaats had}\\

\haiku{Die muizen komen;}{met de takkenbossen en}{het hout de mijn in}\\

\haiku{Hij komt nu bij een, {\textquoteleft}}{verlaten veldgedeelte}{een zoogenaamde}\\

\haiku{En nu, als eerste,.}{Souren die in een ander}{vak gaat Kompeltje}\\

\haiku{Maar waarom mag ik,?}{de mijn dan niet in als U}{het zoo'n mooi vak vindt}\\

\haiku{Het bosch is dan in,.}{eens heel anders een nieuw dak}{en een nieuw vloerkleed}\\

\haiku{{\textquoteleft}Zeg Souren, kun jij?}{je nou nooit losmaken van}{die vervloekte mijn}\\

\haiku{De woorden van Geurts;}{den sleeper kwamen hem weer}{in de gedachten}\\

\haiku{Ik moest maar goed leeren,,.}{zei hij misschien dat het nog}{wel eens te pas kwam}\\

\haiku{Van de mijnwerkers,?}{heb je al een andere}{indruk is het niet}\\

\haiku{{\textquoteright} {\textquoteleft}Maar beste jongen,,!}{je verveelt me absoluut}{niet integendeel}\\

\haiku{Ge kunt toch wel stiekum,,}{een kruis slaan dat niemand het}{ziet als je bang bent}\\

\haiku{Zoo'n wagentje werd.}{dan gekiept en de kolen}{vielen op een zeef}\\

\haiku{Wel wist hij, dat hij,.}{veel last van zijn zenuwen}{had de laatste tijd}\\

\haiku{Maar de jongen was.}{dolgelukkig en daar was}{Vader ook blij om}\\

\haiku{En dan had je nog,:}{die idiote Janus die}{als eenig antwoord wist}\\

\haiku{Bij de uitvaart zelf.}{had half Terwinselen in}{de kerk gezeten}\\

\haiku{Die kerel had nu.}{werkelijk van alles bij}{de mijn meegemaakt}\\

\haiku{En ja, daar stapte,.}{Thea uit er was echter geen}{Kompeltje te zien}\\

\haiku{Het was nu tien over,,.}{tien nog een kwartier dan kon}{Vader binnen zijn}\\

\haiku{een sterke lier stond,.}{opgesteld die de trein naar}{boven zou trekken}\\

\haiku{Daar ging de sleep al,.}{als door een onzichtbare}{kracht opgetrokken}\\

\haiku{{\textquoteleft}Der Kasper{\textquoteright} blijft van brood,).}{en koffie af maar eet nu}{muizen voor zijn straf}\\

\haiku{Hij zou...... ja, hij zou,.}{van alles ze moesten het maar}{aan hem overlaten}\\

\haiku{Wagen na wagen.}{kwam aanrollen en de stoet}{werd hier opgesteld}\\

\haiku{Maar ze aarzelt en,:}{herhaalt alsof ze zich eerst}{nog bezinnen moet}\\

\haiku{Alleen, er was geen,.}{steenkool de jongens werkten}{alleen maar in steen}\\

\haiku{Die Stephan was totaal.}{krankzinnig en werd nog steeds}{in Venray verpleegd}\\

\haiku{{\textquoteleft}Alles drukt op den{\textquoteright},;}{werkman het was een oud maar}{een waar gezegde}\\

\haiku{Verstrooid gaf hij zijn.}{contr\^ole-penning af en}{daalde nu zelf}\\

\haiku{Hij leunde tegen.}{de wand en had wel kunnen}{janken van de pijn}\\

\haiku{Het viel hem tegen,,.}{hij verlangde naar boven}{naar de frissche lucht}\\

\haiku{Dat is een ijzer,{\textquoteright}.}{van zoo lang en zoo dik zoowat}{wees een der mannen}\\

\haiku{Hij rukte zich uit,,.}{zijn verdooving los niet naar}{huis nog niet naar huis}\\

\haiku{Maar het was toch een,.}{schrik voor Moeder geweest toen}{hij zoo voor haar stond}\\

\haiku{Zijn lange Duitsche}{pijp staat nog steeds in de hoek}{bij de linnenkast}\\

\section{S. Carmiggelt}

\subsection{Uit: Allemaal onzin}

\haiku{{\textquoteright} vroeg mijn zoontje, die.}{de woorden allemaal niet}{zo precies weet}\\

\haiku{Hij bemerkte mijn.}{honger naar contact en}{lachte zindelijk}\\

\haiku{Ik                     lig in bed.}{en droom dat een steenbok me}{in het borstbeen bijt}\\

\haiku{dat ik hem nu,                      {\textquoteleft} -!}{onder de uitroepWat gij}{beledigt mijn st\'am}\\

\haiku{We namen elkaar,.}{op als twee worstelaars voor}{de wedstrijd begint}\\

\haiku{Opeens ging de deur.}{open en trad de oude heer}{Kortlever binnen}\\

\haiku{{\textquoteright} riep Kozels verschrikt,,.}{als vreesde hij dat ik mij}{zou                     verdrinken}\\

\haiku{We konden ons dus,.}{tenminste ophangen}{als het tegenviel}\\

\haiku{We zaten er                     , ', '.}{ongemakkelijk maars}{lands wijss lands eer}\\

\haiku{{\textquoteleft}Z\'o, knulletje,{\textquoteright} zeg?}{ik tegen hem en                     raai}{eens wat hij antwoordt}\\

\haiku{Het raadsel van de.}{naamloze voorbijganger}{kwam ter tafel}\\

\haiku{In ieder geval,{\textquoteright}.}{is uw oom een gourmet zei}{Annie inschenkend}\\

\haiku{{\textquoteright} riep de buurman, uit, {\textquoteleft},.}{het raam wijzenddaar met die}{blauwe jas                     aan}\\

\haiku{Koos kijkt nu bepaald,:}{op de staart getrapt maar het}{stemmetje klaagt}\\

\haiku{Ik kan er                     niets, -...}{aan doen maar hij st\'a\'at er weer}{voluit huilend niu}\\

\haiku{{\textquoteleft}Ik kom later nog,,.}{wel eens kerel als je vrouw}{er is of                     zo}\\

\haiku{Ik strompelde naar:}{die helse bel en hoorde}{een meneer roepen}\\

\haiku{Dan wordt het duister.}{over zijn                     minuscule}{problematiek}\\

\haiku{{\textquoteright} roept de bestuurder,,{\textquoteleft}}{die de voornamen bepaald}{uit de mouw schudt}\\

\haiku{Ik verloor opeens.}{mijn linkerschoen en moest er}{even naar                     zoeken}\\

\haiku{Ergens bij de deur}{zag ik vaag hoe hij rukte}{aan de                     ketting}\\

\haiku{wat hij zei sloeg, naast,.}{deze haaibaai als een tang}{op een                     varken}\\

\subsection{Uit: Allemaal onzin}

\haiku{{\textquoteright} vroeg mijn zoontje, die.}{de woorden allemaal niet}{zo precies weet}\\

\haiku{Hij bemerkte mijn.}{honger naar contact en}{lachte zindelijk}\\

\haiku{{\textquoteright} vroeg mijn zoontje, toen.}{we in de schemergrijze}{zijstraat                     liepen}\\

\haiku{Ik                     Hg in bed.}{en droom dat een steenbok me}{in het borstbeen bijt}\\

\haiku{dat ik hem nu,                      {\textquoteleft} -!}{onder de uitroepWat gij}{beledigt mijn st\'am}\\

\haiku{{\textquoteleft}Die jongen                     moet{\textquoteright} {\textquoteleft},{\textquoteright}:}{opstaan ofH\'e vlegel en}{mijn tante zei paars}\\

\haiku{We namen elkaar,.}{op als twee worstelaars voor}{de wedstrijd begint}\\

\haiku{De lesuren werden.}{aan het uitwisselen van}{verhalen besteed}\\

\haiku{Had-ie maar                     niet,.}{moeten duwen toen ik zijn}{stoep bezemde}\\

\haiku{Opeens ging de deur.}{open en trad de oude heer}{Kortlever binnen}\\

\haiku{{\textquoteright} riep Kozels verschrikt,,.}{als vreesde hij dat ik mij}{zou                     verhangen}\\

\haiku{We konden ons dus,.}{tenminste ophangen}{als het tegenviel}\\

\haiku{{\textquoteleft}Z\'o knulletje{\textquoteright}, zeg?}{ik tegen hem en raai}{eens wat hij antwoordt}\\

\haiku{{\textquoteleft}Verdraaid{\textquoteright}, riep hij, {\textquoteleft}dat ',.}{zitm in de krachtfabriek}{als ik                     goed zie}\\

\haiku{Het raadsel van de.}{naamloze voorbijganger}{kwam ter tafel}\\

\haiku{In ieder geval{\textquoteright},.}{is uw oom een gourmet zei}{Annie inschenkend}\\

\haiku{{\textquoteright} riep de buurman, uit, {\textquoteleft},.}{het raam wijzenddaar met die}{blauwe jas                     aan}\\

\haiku{Koos kijkt nu bepaald,:}{op de staart getrapt maar het}{stemmetje klaagt}\\

\haiku{Ik kan er                     niets, -...}{aan doen maar hij st\'a\'at er weer}{voluit huilend nu}\\

\haiku{{\textquoteleft}Ik kom later nog,,.}{wel eens kerel als je vrouw}{er is of                     zo}\\

\haiku{ze duwden er                     ,.}{tegen aan de andere}{kant huilend van angst}\\

\haiku{Ik strompelde naar:}{die helse bel en hoorde}{een meneer roepen}\\

\haiku{Dan wordt het duister.}{over zijn                     minuscule}{problematiek}\\

\haiku{we hadden net zo.}{goed een huwelijksdatum}{kunnen vaststellen}\\

\haiku{Ik verloor opeens.}{mijn linkerschoen                         en moest}{er even naar zoeken}\\

\haiku{wat hij zei sloeg, naast,.}{deze haaibaai als een tang}{op een                     varken}\\

\haiku{En ge moet goed                     ,.}{begrijpen dat ik er geen}{geld in steken wil}\\

\subsection{Uit: Drie in een}

\haiku{Vreemde kostgangers.}{I Het antiekwinkeltje}{keek uit op een gracht}\\

\haiku{Maar als ik nou 't,.}{lirium heb dan beweegt}{dat                         mannetje}\\

\haiku{{\textquoteright} {\textquoteleft}Hij is gesneuveld,{\textquoteright}.}{bij de slag in de Java}{Zee antwoordde hij}\\

\haiku{Op een dag was zij.}{tenminste verdwenen en}{keerde nimmer weer}\\

\haiku{Gisteren, op het,.}{station van die wijze}{stad zag ik haar werk}\\

\haiku{De keukendeur week:{\textquoteleft}.}{en daar stond die goede Piet}{en zei                    Hallo}\\

\haiku{{\textquoteleft}Ja, daar moest                     je.}{op school altijd aan vragen}{als je iets kwijt was}\\

\haiku{Maar misschien vraagt ze.}{het nog steeds stilletjes aan}{Antonius}\\

\haiku{Toen hij de                     kroeg,:}{binnentrad zei een oude}{man aan de tapkast}\\

\haiku{Hij hield van moppen.}{vertellen en mensen voor}{de gek houden}\\

\haiku{{\textquoteright} Tante Fiet deed dat,.}{onmiddellijk                     gewoon}{d\'o\'orconverserend}\\

\haiku{Ik condoleer u,}{met het grote verlies dat}{u heeft getroffen}\\

\haiku{{\textquoteright} Onder het praten.}{was ze teruggelopen}{naar de huiskamer}\\

\haiku{Ze wonen weer in,.}{hetzelfde huis waar ze in}{1940 uitgevlucht zijn}\\

\haiku{Dat is een kwestie,{\textquoteright}.}{van appreciatie zei}{het rinkelmens koel}\\

\haiku{{\textquoteright} Zij legde het stuk.}{op tafel en schroefde de}{dop van haar vulpen}\\

\haiku{antieke borden -.}{waren                         het bij mij dan}{lachten ze je uit}\\

\haiku{Maar op een middag,,.}{om drie                     minuten voor}{twee stond ik voor school}\\

\haiku{Er stond al een man-, -.}{kort breed en grijzend en}{hij rookte een pijp}\\

\haiku{{\textquotedblleft}Een, twee, drie in                     ,,.}{Godsnaam zo vlak onder de}{kust dat doe ik niet}\\

\haiku{Toen Hij was niet eens,.}{zo ingrijpend veranderd}{in al die jaren}\\

\haiku{Hij liet zijn ogen wat,.}{groter worden en schudde}{het hoofd meewarig}\\

\haiku{Hij stak met de                     :}{peuk een nieuwe sigaret}{op en sprak hardop}\\

\haiku{Ik denk altijd, ik.}{hoop dat ze het later}{voor mij net zo doen}\\

\haiku{Nou weet je, wij die,.}{met handel op                     straat staan}{wij worden getrapt}\\

\haiku{De trap naar boven.}{kon oom alleen bereiken}{via de winkel}\\

\haiku{Ze waren lang en.}{in een handschrift dat zich}{haastig voortrepte}\\

\haiku{Er stond geen aanhef.}{boven en ze vielen met}{de deur in huis}\\

\haiku{{\textquoteleft}Omdat-ie met,.}{iedereen ruzie maakte}{had-ie maar \'e\'en vrind}\\

\haiku{Geen moeite, hoor. 't,.}{Lukt prima want ik heb}{hier een goed stekkie}\\

\haiku{{\textquoteright} {\textquoteleft}Eh... wacht even, lieve,.}{schat ik moet de radio wat}{zachter zetten}\\

\haiku{'t Was of Jeroen.}{Bosch zich                     even had bediend}{van een kinderhand}\\

\haiku{{\textquoteleft}Wie gedoemd is te,.}{verdrinken verdrinkt in}{een lepel water}\\

\haiku{op hem hebben we,.}{ons verkeken tenzij hij}{rijp is voor ontslag}\\

\haiku{ik zat namelijk.}{in een helikopter en}{vloog                         er overheen}\\

\haiku{Hij maakt 'n                         {\textquoteleft}nou{\textquoteright}:}{ja gebaar en zegt in zijn}{zorgvuldig Engels}\\

\haiku{{\textquoteright} Naast zijn bureau in.}{het redactievertrek hing}{een groot stuk karton}\\

\haiku{De helft, maar dan een,.}{klein stukkie                         meer een half}{vingertje zowat}\\

\haiku{Over een jongetje.}{dat aan het zwerven ging met}{een bedelaar}\\

\haiku{II Tegen zessen,,.}{stapte ik op het Damrak}{in de tram naar huis}\\

\haiku{{\textquoteright} Ook het weglaten.}{van de lidwoorden hoorde}{helemaal bij hem}\\

\haiku{Toen hij                     alles,.}{had doorgeslikt ademde hij}{zwaar door de neus}\\

\haiku{Je krijgt er wel het,.}{lendewater                     van maar}{je blijft bij de tijd}\\

\haiku{Die vrouw probeerde, '.}{hem af te                     weren maar}{ze konm niet baas}\\

\haiku{Als ze jou op straat,.}{door je porum steken loopt}{iedereen                     door}\\

\haiku{Zo om 'n uur of.}{tien rommel ik een beetje}{aan                     m'n ontbijt}\\

\haiku{Dat wil zeggen - ik,.}{heb iets met een vrouwtje}{van eenendertig}\\

\haiku{Dat het ook minder.}{stijlvol kan ervoeren wij}{onlangs in Parijs}\\

\haiku{Bovendien is haar.}{leven ook op een ander}{niveau veranderd}\\

\haiku{{\textquoteleft}Ik ben heel jong ter.}{wereld                         gekomen in}{een heel oude tijd}\\

\haiku{Maar mij sprak het wel,.}{aan vooral omdat ik zo}{iets nooit zou durven}\\

\haiku{Veel  mensen                     .}{hebben op die leeftijd een}{leeg soort somberheid}\\

\haiku{Maar het hielp wel, want.}{hij dronk eindelijk                         uit}{en verliet de bar}\\

\haiku{Ziekten, die je zou,.}{kunnen krijgen spelen er}{een grote rol in}\\

\haiku{{\textquoteright} De naam luidde bij {\textquoteleft},{\textquoteright}.}{mij geen bel.De beroemde}{humorist riep ze}\\

\haiku{{\textquoteleft}Ach, ach, wat heb ik.}{vroeger om hem gelachen}{als hij optrad}\\

\haiku{Ik geloof dat het.}{beter uit zou komen op}{die andere muur}\\

\haiku{Maar hij raapte al:}{zijn krachten bijeen en}{wist uit te brengen}\\

\haiku{Hij kijkt uit op een.}{ambitieus grondwerk van}{de gemeente}\\

\haiku{Ze voeden                     zich.}{met alles wat in staat van}{ontbinding verkeert}\\

\haiku{{\textquoteleft}Elke dag warme{\textquoteright}.}{vis opeens een andere}{gevoelsinhoud}\\

\haiku{Helemaal op d'r,.}{eentje waarschijnlijk werd}{ze gemainteneerd}\\

\haiku{{\textquoteright} Hij boog zijn hoofd nu.}{dicht in mijn richting en de}{dranklucht werd sterker}\\

\haiku{Zijn regenjas week.}{in de hals en zijn                     hoed}{rustte op zijn oren}\\

\haiku{Ze gaf me een arm.}{en we liepen                     samen}{door de motregen}\\

\haiku{We stappen in zijn:}{antiquarisch voertuig}{en Mohammed zegt}\\

\haiku{{\textquoteleft}Lampie, je moet uit{\textquoteright},}{maar dat helpt niet                     want u}{weet hoe halsstarrig}\\

\haiku{{\textquoteright} Maar als ik haar op:}{wil tillen protesteert}{ze heftig en roept}\\

\haiku{Pas later heb ik.}{beseft dat hij                     te veel}{bier gedronken had}\\

\haiku{Men is heden ten.}{dage monkelend en}{onderhuids geleerd}\\

\haiku{De dame was in.}{haar                     ganse leven nog}{nooit wedersproken}\\

\haiku{{\textquoteleft}Wel, uit hun manier.}{van reageren op het}{voetgangerslicht}\\

\haiku{{\textquoteright} Tenminste bijna,.}{want hij was bezig de a}{te                         voltooien}\\

\haiku{Ik zei zo zacht dat:{\textquoteleft}.}{ik het zelf amper horen}{kon                    Moordenaars}\\

\haiku{Kerels die ouder,.}{waren dan hij trouwden mooie}{jonge                     meisjes}\\

\haiku{Toch                     is dat niet.}{de enige reden waarom}{ik hier buiten zit}\\

\haiku{Ik had het graag voor,.}{hem gedaan want                     hij is}{een aardige man}\\

\haiku{{\textquoteleft}En geloven die?}{acht                     lieve kindertjes}{in Jezus Christus}\\

\haiku{De chauffeur houdt met.}{zijn nog vrije hand de                         deur}{voor de jongen open}\\

\haiku{In mijn jeugd was Nol.}{veel ouder dan ik en nu}{maar zeven jaar}\\

\haiku{Ze had veel met hem.}{uitgestaan en ze was het}{niet                     vergeten}\\

\haiku{Dat is die vrouw bij.}{wie ze                     op een kamer}{woont komen zeggen}\\

\haiku{{\textquoteleft}Paraplu in de,{\textquoteright},.}{bak beveelt ze wijzend naar}{een vaas naast de deur}\\

\haiku{Het woord {\textquoteleft}vreemdeling{\textquoteright}.}{staat met koeieletters op}{mij                         geschilderd}\\

\haiku{{\textquoteright} Nadat ik het hoofd,:}{schudde keek ze zeer ernstig}{naar de kat en zei}\\

\haiku{En eigenlijk kan '.}{t me niet                     verdommen}{wat Joost Jan uitvoert}\\

\haiku{'t Is een corvee, ',.}{met vlammende flensjes aan}{t slot je weet wel}\\

\haiku{{\textquoteleft}Dat flauwe grapje.}{over meneer Bok had ik niet}{moeten                     maken}\\

\haiku{Gelachen heb ik,{\textquoteright} {\textquoteleft},.}{w\'el zegt de man naast me.Niet}{hier in Dordrecht hoor}\\

\haiku{Hij hoorde niet bij,.}{de reiniging maar bij zijn}{eigen creatie}\\

\haiku{Ik zag haar haarscherp,.}{voor me zoals ze er in}{mijn                     jeugd uitzag}\\

\haiku{Uit zijn binnenzak.}{haalde hij een plat flesje}{en nam een                     slok}\\

\haiku{Daar ik aannam dat,:}{het geen Abdij-siroop}{bevatte zei ik}\\

\haiku{Maar weet je wanneer?}{ik zin kreeg om die rails in}{mekaar te zetten}\\

\haiku{Er wordt op gewacht.}{door een jong stelletje dat}{al ondertrouwd is}\\

\haiku{En ik plakte de.}{antwoordenvelop van de}{Raad van Arbeid dicht}\\

\subsection{Uit: Dwaasheden}

\haiku{{\textquoteleft}Maar ik kan toch net!}{zo goed ergens anders}{telefoneren}\\

\haiku{Hij leek op een ets,.}{van Paul Klee zo scheef en}{gebarsten was-ie}\\

\haiku{Haar rechtschapenheid.}{had haar inmiddels niet voor}{ongeluk behoed}\\

\haiku{Het uit elkaar                     .}{nemen van voertuigen is}{een hartstocht van hem}\\

\haiku{Hij bewoonde een.}{portiekhuis met vochtplekken}{en een luchtje}\\

\haiku{De knaap zag mij even.}{aan en boog zich toen weer}{over zijn gefrutsel}\\

\haiku{Kleine tragedie;}{In de grote zaal speelt een}{amateur-blaaskorps}\\

\haiku{Ik legde de haak.}{op het toestel en bleef nog}{even staan kijken}\\

\haiku{Een jongetje, dat,.}{juist voorbijging stortte van}{zijn autopedje}\\

\haiku{Ik acteer de                     .}{houdingloze echtgenoot}{en raak op de gang}\\

\haiku{Als ik terugkom,.}{met de zakjes krijgen we}{hetzelfde nog eens}\\

\haiku{De man merkte dat.}{er helemaal geen triomf}{in haar stem                     was}\\

\haiku{Was het de regen,?}{die mij gistermiddag bij}{hem deed aankloppen}\\

\haiku{Toen ik binnentrad,.}{zat hij te lezen                         op}{zijn pantoffeltjes}\\

\haiku{Zoveel bedroeg ook.}{mijn                     kapitaal en ik}{was nog niet thuis}\\

\haiku{Toen ik klein was, kwam.}{de heer Nieuwkerk vaak bij mijn}{ouders over de vloer}\\

\haiku{Er kwam water aan,.}{te pas hij  bibberde}{tegen                     het glas}\\

\haiku{Mijn vrouw logeerde.}{namelijk bij haar moeder}{in de provincie}\\

\haiku{Geen nood, u komt hem.}{nog wel eens tegen op een}{donker weggetje}\\

\haiku{Eindelijk waren:}{we het er dan met elkaar}{over eens geworden}\\

\haiku{Ze droegen blauwe.}{kielen en vroegen of}{het hier moest wezen}\\

\haiku{{\textquoteleft}IJs,{\textquoteright} waarop hij                     .}{wantrouwig informeerde}{wat zoiets kostte}\\

\haiku{Die des anderen:}{daags waarschijnlijk tegen een}{kennis heeft gezegd}\\

\haiku{Zeg maar gerust dat}{we geen huur meer betalen}{als die kerel}\\

\subsection{Uit: Dwaasheden}

\haiku{{\textquoteleft}Maar ik kan toch net!}{zo goed ergens anders}{telefoneren}\\

\haiku{Hij leek op een ets,.}{van Paul Klee zo scheef en}{gebarsten was-ie}\\

\haiku{Haar rechtschapenheid.}{had haar inmiddels niet voor}{ongeluk behoed}\\

\haiku{Het uit elkaar                     .}{nemen van voertuigen is}{een hartstocht van hem}\\

\haiku{Hij bewoonde een.}{portiekhuis met vochtplekken}{en een luchtje}\\

\haiku{De knaap zag mij even.}{aan en boog zich toen weer}{over zijn gefrutsel}\\

\haiku{Kleine tragedie;}{In de grote zaal speelt een}{amateur-blaaskorps}\\

\haiku{Ik legde de haak.}{op het toestel en bleef nog}{even staan kijken}\\

\haiku{Een jongetje, dat,.}{juist voorbijging stortte van}{zijn autopedje}\\

\haiku{Ik acteer de                     .}{houdingloze echtgenoot}{en raak op de gang}\\

\haiku{Als ik terugkom,.}{met de zakjes krijgen we}{hetzelfde nog eens}\\

\haiku{De man merkte dat.}{er helemaal geen triomf}{in haar stem                     was}\\

\haiku{Was het de regen,?}{die mij gistermiddag bij}{hem deed aankloppen}\\

\haiku{Toen ik binnentrad,.}{zat hij te lezen                         op}{zijn pantoffeltjes}\\

\haiku{Zoveel bedroeg ook.}{mijn                     kapitaal en ik}{was nog niet thuis}\\

\haiku{Toen ik klein was, kwam.}{de heer Nieuwkerk vaak bij mijn}{ouders over de vloer}\\

\haiku{Er kwam water aan,.}{te pas hij  bibberde}{tegen                     het glas}\\

\haiku{Mijn vrouw logeerde.}{namelijk bij haar moeder}{in de provincie}\\

\haiku{Geen nood, u komt hem.}{nog wel eens tegen op een}{donker weggetje}\\

\haiku{Eindelijk waren:}{we het er dan met elkaar}{over eens geworden}\\

\haiku{Ze droegen blauwe.}{kielen en vroegen of}{het hier moest wezen}\\

\haiku{{\textquoteleft}IJs,{\textquoteright} waarop hij                     .}{wantrouwig informeerde}{wat zoiets kostte}\\

\haiku{Die des anderen:}{daags waarschijnlijk tegen een}{kennis heeft gezegd}\\

\haiku{Zeg maar gerust dat}{we geen huur meer betalen}{als die kerel}\\

\subsection{Uit: Een Hollander in Parijs}

\haiku{{\textquoteleft}Catch{\textquoteright} heet                     dit uit.}{Amerika overgewaaide}{divertissement}\\

\haiku{leest de                     vroegste,.}{editie                                              van de krant die}{net is uitgevent}\\

\haiku{Kijk eens wat een Kitsch{\textquoteright}.}{Op de bank tegenover me}{zaten drie priesters}\\

\haiku{Maar als Ernest een,:}{vader                     met zijn zoon ziet}{wandelen denkt hij}\\

\haiku{{\textquoteleft}Ze behoort aan een,{\textquoteright}.}{andere wereld luidt}{zijn motivering}\\

\haiku{Dan draait                     hij zich.}{gedesinteresseerd om}{en slentert weer weg}\\

\haiku{{\textquoteright} Omdat niemand geld,.}{biedt ontrolt het mannetje}{nu ook de tweede}\\

\subsection{Uit: Een Hollander in Parijs}

\haiku{{\textquoteleft}Catch{\textquoteright} heet                     dit uit.}{Amerika overgewaaide}{divertissement}\\

\haiku{Met de nuttiging.}{van het ooft gingen zowat}{tien minuten heen}\\

\haiku{Maar als Ernest een,:}{vader                     met zijn zoon ziet}{wandelen denkt hij}\\

\haiku{Dan draait                     hij zich.}{gedesinteresseerd om}{en slentert weer weg}\\

\haiku{Dat bevalt me. Men.}{amuseert zich en is toch in}{goed                     gezelschap}\\

\haiku{{\textquoteleft}Als er nou maar niet.}{zo'n lelijk oud vrouwtje in}{komt te                         wonen}\\

\haiku{Zolang de                         mens,.}{een beest is dient het beest de}{mens tot voedsel}\\

\haiku{Of u en ik dat -.}{ooit zullen leren ik}{twijfel eraan}\\

\haiku{Om het te breken.}{houd ik hem onhandig het}{doosje weer                     voor}\\

\haiku{Dit                         rijtuig is{\textquoteright}, {\textquoteleft}?}{niet gemakkelijk \'ofZiet}{gij die regenboog}\\

\haiku{Voor de deur van het{\textquoteleft}!}{reisbureau schalt ons dankbaar}{Hiep hiep hoera}\\

\subsection{Uit: Ik lieg de waarheid. De beste Kronkels}

\haiku{De beste Kronkels}{Editie Sylvia Witteman}{Colofon}\\

\haiku{De man merkte dat.}{er helemaal geen triomf}{in haar stem                     was}\\

\haiku{we hadden net zo.}{goed een huwelijksdatum}{kunnen vaststellen}\\

\haiku{En ze was meteen,.}{geschrokken omdat                     haar}{dochter al op was}\\

\haiku{{\textquoteleft}O, uitstekend, dank{\textquoteright}?}{u. Want waarom zou het zo'n}{man niet                     goed gaan}\\

\haiku{{\textquoteright} De leraar stapte,.}{met een klein zwart wichtje aan}{de arm op hen af}\\

\haiku{Hij zag de leraar:}{met zijn jas aan langskomen}{en hoorde hem gen}\\

\haiku{Maar net zoveel thee,,,,.}{als je wou                     drie vier vijf}{koppen dat gaf niet}\\

\haiku{De deur ging open en.}{de zwarte kelner stond}{weer op de drempel}\\

\haiku{Ze snelde weer naar,.}{de deur kennelijk op weg}{naar duizend plichten}\\

\haiku{{\textquoteright}        Grut Om zeven '}{uurs avonds luiden ze voor}{het vakantiehuis}\\

\haiku{Want dat is toch geen -?}{manier van doen iemand zijn}{tekst afnemen}\\

\haiku{{\textquoteright} {\textquoteleft}Maar als ik eruit,{\textquoteright}.}{ga moet ik op het koude}{zeil wierp ik tegen}\\

\haiku{Bijna                     plechtig.}{stond ik op en ging er op}{blote voeten heen}\\

\haiku{Plotseling sprong er:}{een bazige jongen}{van de fiets en riep}\\

\haiku{'t Is net als 't,.}{schippersvak je moet erin}{geboren worden}\\

\haiku{{\textquoteright} {\textquoteleft}Dat is waar, wij ben',{\textquoteright}.}{anders dan de mensen hier}{gaf de man toe}\\

\haiku{Voor ze opstapte,.}{probeerde ze de lamp maar}{die gaf geen                     licht}\\

\haiku{Zolang hij dus niets,.}{zegt bestaat de kans dat hij}{de                     waarheid kent}\\

\haiku{{\textquoteleft}O ja, meneer Van -,,.}{Driel                    gaat u zitten neemt}{u plaats alstublieft}\\

\haiku{De angst was over, maar.}{de wilde blijdschap keerde}{toch niet terug}\\

\haiku{Ik heb                     eens in...{\textquoteright} {\textquoteleft},,}{een boek gelezenChurchill}{die bijna verzoop}\\

\haiku{Dit embleem duidde.}{onze muzikale}{bedoelingen aan}\\

\haiku{En die andere,,.}{die dikke dat is een}{halve loodgieter}\\

\haiku{Nou ja, je moet                     ,.}{er niet over praten hoor maar}{die laat ik doodgaan}\\

\haiku{Het groen werd moe en.}{begon toen aan de punten}{te                     vergelen}\\

\haiku{{\textquoteright} Toen er gezoend en,.}{geknuffeld was streek men neer}{in de voorkamer}\\

\haiku{Thuis,{\textquoteright} antwoordde ik,?}{want leerde moeder niet dat}{jokken zonde is}\\

\haiku{Misschien is het wel,....}{op dat beetje talent dacht}{hij vol zelfbeklag}\\

\haiku{Ze groetten stijf toen.}{hij binnenkwam en bij het}{raam ging                     zitten}\\

\haiku{De dichter stond op,.}{en liep het caf\'e door dat}{lachje om zijn mond}\\

\haiku{{\textquoteleft}Ik mag toch zeker? '}{wel vragen of het al}{afgelopen is}\\

\haiku{Harry - ik droom nog.}{wel eens van hem als ik wat}{zwaar getafeld heb}\\

\haiku{{\textquoteleft}Die jongen die ze,,,.}{nou heeft och hij is goed}{maar sukkelachtig}\\

\haiku{Hij hield van moppen.}{vertellen en mensen voor}{de gek houden}\\

\haiku{Ze gingen zwijgend,,.}{zitten twee kleine oude}{verknochte mensen}\\

\haiku{En ofschoon hij nog,:}{steeds dacht aan de dood zei}{hij precies op tijd}\\

\haiku{H\`e, wat doe je nou,,{\textquoteright}.}{je handen aan je goeie broek}{afvegen zei Fie}\\

\haiku{Dan                     naar het blauw.}{worden en het naar buiten}{treden van de tong}\\

\haiku{Ik weet niet precies,.}{waarom het is maar                     ze}{vinden het niet goed}\\

\haiku{Ja, dat kan. 't Doet, '.}{geen pijn wantt gaat meteen}{uit in je mond}\\

\haiku{De man ging op het.}{bed liggen en liet de brief}{op de grond glijden}\\

\haiku{Er was                     geen geld.}{voor een echte acteur en}{toen mocht ik het doen}\\

\haiku{Het laatste wat hij,.}{zich herinnerde was uit}{zijn kindertijd}\\

\haiku{Hij probeert daar op.}{zijn schuchtere wijze iets}{tegen                     te doen}\\

\haiku{Hij neuriet liedjes.}{uit zijn                     jeugd en vindt dat}{hij goed bij stem is}\\

\haiku{Vader Ouwe Jan.}{kende ik al lang voordat}{hij in het huis zat}\\

\haiku{Vroeger zag ik hem,.}{vrijwel                     dagelijks in}{een kleine buurtkroeg}\\

\haiku{Ze zijn 't hem net.}{een halfuur geleden}{komen vertellen}\\

\haiku{{\textquoteleft}Ik heb het eraf,{\textquoteright}, {\textquoteleft}.}{geveegd zei hijomdat je}{niet in je zaak was}\\

\haiku{{\textquoteleft}Ik dacht dat je 't,{\textquoteright}.}{eraf geveegd had zei hij}{met geknepen stem}\\

\haiku{Hij is iets in de.}{haven en dat kun je wel}{aan hem zien ook}\\

\haiku{Ze heeft nooit iets in.}{haar                     blik van zo'n vrouw die}{haar man komt halen}\\

\haiku{En vandaag is Joop.}{verkwikt en tevreden aan}{het werk gegaan}\\

\haiku{Wie een penny voor,.}{zijn gedachten gaf zou zich}{bekocht voelen}\\

\haiku{Het jongetje had.}{zijn mond leeg en keek enigszins}{kritisch naar de man}\\

\haiku{{\textquoteright} De man, die juist een,:}{ochtendblad ontvouwde schrok}{een beetje en sprak}\\

\haiku{Hij liep weer naar de,:}{kast keerde met het boek bij}{mij terug en zei}\\

\haiku{Op die manier was.}{het dus                         mogelijk de}{mens uit te rekken}\\

\haiku{of ze, als ik het,.}{vertelde zou begrijpen}{wat ik hier                         most}\\

\haiku{Het wijfje bleef nog,.}{door de kier loeren wat de}{zaak compliceerde}\\

\haiku{{\textquoteleft}Piet, ken je me nog,.}{we hebben samen onder}{dienst                         gelegen}\\

\haiku{Ik draaide mij om.}{en daalde in zeer snel}{tempo de trap af}\\

\haiku{Haar doezelige ' - -.}{genotsogen zagent hem}{niet zonder haat doen}\\

\haiku{{\textquoteright}        Duiven In de.}{buurt van mijn Londens hotel}{bevindt zich een parkje}\\

\haiku{Zijn rechterpoot was,.}{normaal maar met de linker}{trok hij                     hevig}\\

\haiku{Maar dat brak ze dan,.}{lelijk op want nu was}{al dat brood voor hem}\\

\haiku{{\textquotedblleft}Als je het doet, dan.}{zal                     het altijd een kloof}{tussen ons blijven}\\

\haiku{{\textquoteleft}Wilt u tegen uw?}{man                     zeggen dat ik niet}{met hem kan trouwen}\\

\haiku{{\textquoteleft}Legt u het daar maar,,{\textquoteright},.}{neer juffrouw zei hij wijzend}{op een tafeltje}\\

\haiku{Ik had het gevoel.}{bij haar voor een klein examen}{te zijn                     geslaagd}\\

\haiku{Een eindje verder.}{ging ik aan een tafeltje}{bij het raam zitten}\\

\haiku{Er stond nu van die.}{miezerige nieuwbouw}{uit een krappe beurs}\\

\haiku{{\textquoteright} En dat was in die.}{dagen een synoniem voor}{zwarte reactie}\\

\haiku{Ze hebben 't te,...{\textquoteright} {\textquoteleft},{\textquoteright}.}{goed bij moeder zekerGunst}{zei de dominee}\\

\haiku{{\textquoteleft}Ik wou zo'n lampje,{\textquoteright}.}{van zestien gulden uit de}{etalage zei ik}\\

\haiku{Nou, die hoeven ze.}{ook zijn ogen alleen nog maar}{toe te                     drukken}\\

\haiku{Allemaal van                         ,.}{die haastige smurrie uit}{de idee\"enwinkel}\\

\haiku{Hij had geen talent,.}{voor geluk zoals Max Nord}{het eens uitdrukte}\\

\haiku{Ik heb laatst een                         .}{artikel gelezen over}{alcoholisten}\\

\haiku{Want bijna al die.}{winkeliertjes waren}{arm en neringziek}\\

\haiku{Ik had altijd een,.}{stukkie ijzer bij me}{op zaterdagavond}\\

\haiku{{\textquoteright} Fransje kwam terug.}{met de zak patat en ging}{op een kruk zitten}\\

\haiku{Hij was er nog niet.}{toen ik Jan de soldaat moest}{spelen in Indi\"e}\\

\haiku{En jij zat                     ook.}{nog in de verrekijker}{van je ouwe heer}\\

\haiku{Iets van Chopin -.}{of van zijn buurman ook}{geen vrolijke broek}\\

\haiku{Nu glimlachte ze.}{tegen me en kneep de ogen}{even helemaal dicht}\\

\haiku{De stem zweeg dan, maar,.}{de geur                     verplaatste zich en}{dat maakte veel goed}\\

\haiku{De oude was het,.}{vergeten maar knikte om}{eraf te wezen}\\

\haiku{{\textquoteright}        Bezoek Precies.}{om acht uur in de ochtend}{werd ik klaarwakker}\\

\subsection{Uit: Kroeglopen 2}

\haiku{Vader Ouwe Jan.}{kende ik al lang voordat}{hij in het huis zat}\\

\haiku{Vroeger zag ik hem,.}{vrijwel                     dagelijks in}{een kleine buurtkroeg}\\

\haiku{Ze zijn 't hem net.}{een half uur                     geleden}{komen vertellen}\\

\haiku{We gaan vanmiddag,.}{even naar Henk en Marie want}{Jopie is jarig}\\

\haiku{Ze staat schichtig op,:}{gaat bij de beschonken man}{zitten en vraagt}\\

\haiku{Toen ik de ronde,.}{net volgeschonken had ging}{de telefoon weer}\\

\haiku{Terwijl ik de fles,.}{weer hief ging de telefoon}{ten derden male}\\

\haiku{De man achter de.}{bierkraan haatte mij op het}{eerste                     geziect}\\

\haiku{Hij vertelde mij,,.}{in ons stamcaf\'e het eerst}{over Rastelli}\\

\haiku{Hij nam een ferme.}{teug en begon toen luid in}{zich zelf te praten}\\

\haiku{De een zei iets, langs.}{zijn neus weg en de ander}{bulderde daar om}\\

\haiku{{\textquoteleft}Ik dacht dat je 't,{\textquoteright}.}{eraf geveegd had zei hij}{met geknepen stem}\\

\haiku{{\textquoteright}        De vakantie.}{van Joop De vakantie van}{Joop zit er weer op}\\

\haiku{Hij is iets in de.}{haven en dat kun je wel}{aan hem zien ook}\\

\haiku{Ze heeft nooit iets in.}{haar                     blik van zo'n vrouw die}{haar man komt halen}\\

\haiku{En vandaag is Joop.}{verkwikt en tevreden aan}{het werk gegaan}\\

\haiku{Als ik er om vier.}{uur vijf binnen treed ben ik}{er de enige klant}\\

\haiku{Met een schuwe blik:}{naar                     de stoelen en de}{tafeltjes vroeg hij}\\

\haiku{{\textquoteleft}Zie je, ik ben niet.}{gewend om in zulke}{zaken te komen}\\

\haiku{Alleen - als kennis.}{is hij bepaald een                     wat}{onrustig bezit}\\

\haiku{Van                     buiten ziet.}{zo'n etablissementje er}{nog wel aardig uit}\\

\haiku{Wie een penny voor,.}{zijn gedachten gaf zou zich}{bekocht voelen}\\

\haiku{Hij nam een slok die.}{zijn bierglas ledigde en}{verlangde een nieuw}\\

\haiku{{\textquoteright} De man keerde door.}{het volle caf\'e naar het}{tafeltje terug}\\

\haiku{En we komen in.}{Amsterdam                     en dat geld}{brandde in me zak}\\

\subsection{Uit: Met de neus in de boeken}

\haiku{{\textquoteright} Pas later bleek dat.}{hij nooit eerder op een fiets}{gezeten                     had}\\

\haiku{{\textquoteleft}Hij mag dan haar op,,.}{z'n borst hebben maar zusters}{dat heeft Lassie \'o\'ok}\\

\haiku{{\textquoteleft}Als ik haar de                     ,?}{wol geef kan ze er dan voor}{mij \'o\'ok een maken}\\

\haiku{Ik was altijd blij.}{als ik de                     deur achter}{zo'n vent kon dichtdoen}\\

\haiku{Hij verhief zich voor,.}{het open raam sidderend over}{zijn hele lichaam}\\

\haiku{De natuur bracht hem.}{altijd zijn kindertijd}{in herinnering}\\

\haiku{En                     nou gaan we.}{uw mooie wasmachine uit}{de auto halen}\\

\haiku{De jongedame.}{in kwestie is inderdaad}{een Nederlandse}\\

\haiku{{\textquoteright} zei ik tegen mijn, {\textquoteleft}.}{vrouwmaar ik                     maak bezwaar}{tegen zijn aanhef}\\

\haiku{Want wat was er aan,?}{de hand met het boek dat mijn}{vrouw uit de kast trok}\\

\haiku{Er was \'e\'en zo'n klein,,.}{slecht ventje bij en                     die}{sloeg toch z\'o hard h\`e}\\

\haiku{Anders zouden ze,}{denken dat ik helemaal}{hoogmoedswaanzin heb}\\

\haiku{En misschien een                     .}{bijdrage tot het probleem}{van je depressies}\\

\haiku{Vandaar mijn vrees,                     .}{toen ik je zag staan voor het}{Citytheater}\\

\haiku{Particulier{\textquoteright} is -.}{het wel vervelend voor je}{toegegeven}\\

\haiku{D. had een afspraak,.}{met onze oude vriend}{de schrijver Jef Last}\\

\haiku{Hij klaagde over zijn.}{vreselijk leven in het}{bejaardenhuis}\\

\haiku{Toen begaf hij zich -.}{naar de toiletten al}{voor de derde keer}\\

\haiku{En schrijf vooral niet.}{terug als je het uit je}{tenen halen moet}\\

\haiku{{\textquoteleft}God, als ik alleen,.}{kom                     te staan laat me dan}{eenzaam kunnen zijn}\\

\haiku{Wat je schrijft over angst.}{en schuwheid voor mensen}{herken ik toch wel}\\

\haiku{Ik ben, net als jij,,.}{momenteel slecht op stoot maar}{ik knok ertegen}\\

\haiku{En zo ja - was die,?}{persoon de man over wie}{wij het laatst hadden}\\

\haiku{Ik herlas {\textquoteleft}De taal{\textquoteright}.}{der liefde en vond                     het}{opnieuw geweldig}\\

\haiku{Maar ik heb gewoon.}{zitten lezen in een}{eenvoudig bestek}\\

\haiku{de metselklinker {\textquoteleft}{\textquoteright}.}{is slechtsenigszins getrokken}{of                     beregend}\\

\haiku{O ja, men geeft hem.}{de hand. Maar telt vervolgens}{zijn vingers                     na}\\

\haiku{Breng het gesprek op,:}{Shakespeare en zeg}{dan langs uw neus weg}\\

\haiku{Toen de kelner op,:}{moeilijke platvoeten was}{weggestapt zei hij}\\

\haiku{Ze logeerden er.}{of kleefden er alleen aan}{de sublieme bar}\\

\haiku{Een zijner boeken.}{noemde hij Een diamant zo}{groot als de                     Ritz}\\

\haiku{Maar juist                     daarom.}{wilde Proust bij voorkeur door}{hem bediend worden}\\

\haiku{In de rue Blanche,,}{waar hij woonde ontmoette}{hij op een ochtend}\\

\haiku{van een grote                         .}{dichter heeft ze er vrede}{mee zo te heten}\\

\haiku{Ik zat bij hem en.}{na een tijdje vroeg hij wat}{mijn plannen waren}\\

\haiku{De vrouwen op straat.}{waren weggedoken in}{wijde                     jassen}\\

\haiku{De arrestatie?}{van Frech hebben we zeker}{aan jou te danken}\\

\haiku{Het is als met het -.}{geluid van Caruso jammer}{dat hij er mee praat}\\

\haiku{Maar hij heeft, hoop ik,.}{aan mijn toon gehoord dat ik}{het meende}\\

\haiku{je wordt                     er toch,.}{niet moe van je kunt er toch}{bij blijven zitten}\\

\haiku{Daarom ging Thomas.}{de trap niet af en Gerhardt}{de                         trap niet op}\\

\haiku{Die inspireerde.}{Mann tot het                         scheppen van}{zijn nieuwe figuur}\\

\haiku{Mijn vriend was er nog.}{niet en ik moest wachten in}{zijn werkkamer}\\

\haiku{In de spreektaal kunt{\textquoteleft}{\textquoteright}.}{u met het woordje                     mooi}{vele kanten op}\\

\haiku{{\textquoteright} De eerste jongen.}{floot niet zonder waardering}{door zijn voortanden}\\

\haiku{Nee meneer, tegen.}{snelheid en herrie is}{geen kruid gewassen}\\

\subsection{Uit: Onzin}

\haiku{{\textquoteright} vroeg mijn zoontje, die.}{de woorden allemaal niet}{zo precies weet}\\

\haiku{Ik                     lig in bed.}{en droom dat een steenbok me}{in het borstbeen bijt}\\

\haiku{dat ik hem nu,                      {\textquoteleft} -!}{onder de uitroepWat gij}{beledigt mijn st\'am}\\

\haiku{{\textquoteleft}Die                     jongen moet{\textquoteright}, {\textquoteleft}{\textquoteright},:}{opstaan ofHee vlegel en}{mijn tante zei paars}\\

\haiku{We namen elkaar,.}{op als twee worstelaars voor}{de wedstrijd begint}\\

\haiku{Opeens ging de deur.}{open en trad de oude heer}{Kortlever binnen}\\

\haiku{{\textquoteright} Hij bootste het na,.}{even omkijkend of niemand}{hem                     bespiedde}\\

\haiku{{\textquoteright} riep Kozels verschrikt,.}{als vreesde hij dat ik mij}{zou                     verdrinken}\\

\haiku{We zaten erg                     , ', '.}{ongemakkelijk maars}{lands wijss lands eer}\\

\haiku{Verdraaid,{\textquoteright} riep hij, {\textquoteleft}dat ',.}{zitm in de krachtfabriek}{als ik                     goed zie}\\

\haiku{De langste gaf, net,.}{toen ik passeerde                     de}{ander een schop}\\

\haiku{Het raadsel van de.}{naamloze voorbijganger}{kwam ter tafel}\\

\haiku{In ieder geval,{\textquoteright}.}{is uw oom een gourmet zei}{Annie inschenkend}\\

\haiku{Zij was half gedekt,,.}{er stond een bord op waarvan}{gegeten was}\\

\haiku{Bij Ina bevroren.}{de                     waterlanders op}{haar fijne kopje}\\

\haiku{{\textquoteright} riep de buurman, uit, {\textquoteleft},.}{het raam wijzenddaar met die}{blauwe jas                     aan}\\

\haiku{Koos kijkt nu bepaald,:}{op de staart getrapt maar het}{stemmetje klaagt}\\

\haiku{Ik kan er                     niets, -...}{aan doen maar hij st\'a\'at er weer}{voluit huilend nu}\\

\haiku{Ik strompelde naar:}{die helse bel en hoorde}{een meneer roepen}\\

\haiku{Dan wordt het duister.}{over zijn                     minuscule}{problematiek}\\

\haiku{{\textquoteright} roept de bestuurder,,{\textquoteleft}}{die de voornamen bepaald}{uit de mouw schudt}\\

\haiku{we hadden net zo.}{goed een huwelijksdatum}{kunnen vaststellen}\\

\haiku{Ik verloor opeens.}{mijn linkerschoen                         en moest}{er even naar zoeken}\\

\haiku{wat hij zei sloeg, naast,.}{deze haaibaai als een tang}{op een                     varken}\\

\haiku{En ge moet                     goed,.}{begrijpen dat ik er geen}{geld in steken wil}\\

\subsection{Uit: Onzin}

\haiku{{\textquoteright} vroeg mijn zoontje, die.}{de woorden allemaal niet}{zo precies weet}\\

\haiku{Ik                     lig in bed.}{en droom dat een steenbok me}{in het borstbeen bijt}\\

\haiku{dat ik hem nu,                      {\textquoteleft} -!}{onder de uitroepWat gij}{beledigt mijn st\'am}\\

\haiku{{\textquoteleft}Die                     jongen moet{\textquoteright}, {\textquoteleft}{\textquoteright},:}{opstaan ofHee vlegel en}{mijn tante zei paars}\\

\haiku{We namen elkaar,.}{op als twee worstelaars voor}{de wedstrijd begint}\\

\haiku{Opeens ging de deur.}{open en trad de oude heer}{Kortlever binnen}\\

\haiku{{\textquoteright} Hij bootste het na,.}{even omkijkend of niemand}{hem                     bespiedde}\\

\haiku{{\textquoteright} riep Kozels verschrikt,.}{als vreesde hij dat ik mij}{zou                     verdrinken}\\

\haiku{We zaten erg                     , ', '.}{ongemakkelijk maars}{lands wijss lands eer}\\

\haiku{Verdraaid,{\textquoteright} riep hij, {\textquoteleft}dat ',.}{zitm in de krachtfabnek}{als ik                     goed zie}\\

\haiku{De langste gaf, net,.}{toen ik passeerde                     de}{ander een schop}\\

\haiku{Het raadsel van de.}{naamloze voorbijganger}{kwam ter tafel}\\

\haiku{In ieder geval,{\textquoteright}.}{is uw oom een gourmet zei}{Annie inschenkend}\\

\haiku{Zij was half gedekt,,.}{er stond een bord op waarvan}{gegeten was}\\

\haiku{Bij Ina bevroren.}{de                     waterlanders op}{haar fijne kopje}\\

\haiku{{\textquoteright} riep de buurman, uit, {\textquoteleft},.}{het raam wijzenddaar met die}{blauwe jas                     aan}\\

\haiku{Koos kijkt nu bepaald,:}{op de staart getrapt maar het}{stemmetje klaagt}\\

\haiku{Ik kan er                     niets, -...}{aan doen maar hij st\'a\'at er weer}{voluit huilend nu}\\

\haiku{Ik strompelde naar:}{die helse bel en hoorde}{een meneer roepen}\\

\haiku{Dan wordt het duister.}{over zijn                     minuscule}{problematiek}\\

\haiku{{\textquoteright} roept de bestuurder,,{\textquoteleft}}{die de voornamen bepaald}{uit de mouw schudt}\\

\haiku{we hadden net zo.}{goed een huwelijksdatum}{kunnen vaststellen}\\

\haiku{Ik verloor opeens.}{mijn linkerschoen                         en moest}{er even naar zoeken}\\

\haiku{wat hij zei sloeg, naast,.}{deze haaibaai als een tang}{op een                     varken}\\

\haiku{En ge moet                     goed,.}{begrijpen dat ik er geen}{geld in steken wil}\\

\subsection{Uit: Een stoet van dwergen}

\haiku{Eindelijk waren:}{we het er dan met elkaar}{over eens geworden}\\

\haiku{Als hij  met een,.}{zaag terugkomt heb ik}{z\'o mijn broek weer aan}\\

\haiku{De man merkte dat.}{er helemaal geen triomf}{in haar stem                     was}\\

\haiku{{\textquoteright} vroeg mijn zoontje, die.}{de woorden allemaal}{niet zo precies weet}\\

\haiku{Hij bemerkte                         .}{mijn honger naar contact en}{lachte zindelijk}\\

\haiku{{\textquoteright} vroeg mijn zoontje, toen.}{we in de schemergrijze}{zijstraat                         liepen}\\

\haiku{Het laatst zag ik de.}{heer Cohen in de oorlog}{voor het station}\\

\haiku{Nou, aan zijn manier,.}{van lopen kon je zien dat}{hij mij gelijk gaf}\\

\haiku{En zij besloot hem,.}{die avond eens extra te}{raken in de gang}\\

\haiku{{\textquoteleft}Maar ik zal even mijn,.}{zaklantaarn halen dan kan}{ik beter                         zien}\\

\haiku{{\textquoteright} vroeg de chauffeur, die.}{bestoft doch ongebroken}{op de vloer                         zat}\\

\haiku{Met de nuttiging.}{van het ooft gingen zowat}{tien minuten heen}\\

\haiku{Maar als Ernest een,:}{vader                     met zijn zoon ziet}{wandelen denkt hij}\\

\haiku{{\textquoteright} Omdat niemand geld,.}{biedt ontrolt het mannetje}{nu ook de tweede}\\

\haiku{{\textquoteleft}O, uitstekend, dank{\textquoteright}?}{u. Want waarom zou het zo'n}{man                         niet goed gaan}\\

\haiku{Een half uur later:}{zat hij met haar op de}{trap en was al aan}\\

\haiku{{\textquoteright} De leraar stapte,.}{met een klein zwart wichtje aan}{de arm op hen af}\\

\haiku{hij wist het niet, maar.}{hij was in elk geval}{hogelijk bekoord}\\

\haiku{Kwam je 's ochtends,.}{beneden dan was de}{tafel al gedekt}\\

\haiku{Maar net zoveel thee,,,,.}{als je wou                     drie vier vijf}{koppen dat gaf niet}\\

\haiku{{\textquoteleft}Drink ze zelf maar op,,{\textquoteright}.}{artis zegt de man met een}{grimmig soort humor}\\

\haiku{{\textquoteleft}O ja, meneer Van -,,.}{Driel                    gaat u zitten neemt}{u plaats alstublieft}\\

\haiku{We hebben iets nieuws,.}{ingevoerd                     waardoor het}{werk wat sneller gaat}\\

\haiku{Want dat is toch geen -?}{manier van doen iemand zijn}{tekst afnemen}\\

\haiku{'t Is net als 't,.}{schippersvak je moet erin}{geboren worden}\\

\haiku{Ik heb wel eens in...{\textquoteright} {\textquoteleft},,}{een boek gelezenChurchill}{die bijna verzoop}\\

\haiku{Ze                     dronken wel,.}{hun glaasje maar mankeren}{deden ze nooit wat}\\

\haiku{Zacht en kalm was de.}{kroeg onder zijn stevige}{handen overleden}\\

\haiku{Steeds kleiner wordt het.}{schemerige rijk van de}{vaste jongens}\\

\haiku{Morgenochtend zijn,.}{lekker de slagers weer open}{dacht hij na{\"\i}ef}\\

\haiku{Ik hou van een meeuw,{\textquoteright}, {\textquoteleft}.}{vervolgde hij koppigen}{van Richard Tauber}\\

\haiku{Je komt er als mens.}{binnen en je gaat meteen}{op                         de knie\"en}\\

\haiku{{\textquoteright} Frits deed het en een.}{poosje later belde hij}{bij zijn moeder aan}\\

\haiku{Twintig Toen ik bij,.}{de haringkar kwam stond daar}{al een man te eten}\\

\haiku{Harry - ik droom nog.}{wel eens van hem als ik wat}{zwaar getafeld heb}\\

\haiku{{\textquoteleft}Die jongen die ze,,,.}{nou heeft                     och hij is goed}{maar sukkelachtig}\\

\haiku{{\textquoteright}        Maanlicht Om drie.}{uur in de nacht werd de man}{met een schok wakker}\\

\haiku{Op zijn kamertje.}{kleedde hij zich uit en ging}{in bed                     liggen}\\

\haiku{{\textquoteleft}Ach, wat heeft het voor?}{zin om het allemaal zo}{donker in te zien}\\

\haiku{{\textquoteleft}Je denkt toch niet dat?}{er louter parels van je}{lippen rollen}\\

\haiku{Anders verstrekt                     .}{Fie dat genoegen alleen}{op mijn verjaardag}\\

\haiku{Dat                         wil zeggen -,.}{ik was geen dwarse jongen}{maar ik wou vrij zijn}\\

\haiku{wachtte                         ik niet,.}{af en als ik wou deed ik}{een paar dagen niks}\\

\haiku{En alleen omdat,.}{hij het niet gedaan had moest}{jij de                         laan uit}\\

\haiku{Haar lach is echt en.}{ze kan niet tegen alles}{wat                     zielig is}\\

\subsection{Uit: Tussen mal en dwaas \& Klein beginnen}

\haiku{* ~ Mijn moeder is.}{altijd woedend als zij in}{een verhaal voorkomt}\\

\haiku{Uw vriend stelt vast                         .}{belang in uw verhaal over}{die krabbende vent}\\

\haiku{Nou moet er toch                     ,,.}{h\'e\'el wat gebeuren eer ik}{hier uit kom dacht ik}\\

\haiku{de vrede van een.}{naderende dut begon}{te                         vertonen}\\

\haiku{Het was een hele.}{stapel en er waren}{fraaie doorkijkjes bij}\\

\haiku{W\'at De Gaulle ook -.}{zeggen mag de wijn is weer}{goedkoop in Frankrijk}\\

\haiku{En dat maakt maar                     .}{droge-naaldetsen of}{het geen  geld kost}\\

\haiku{Wie een beroerde,,}{roman baart stelt alleen maar}{teleur maar                     schrijft}\\

\haiku{Zo'n Rubens kon                     ,.}{je de ruimte geven met}{zijn bolle dames}\\

\haiku{Maar dat presentje.}{heb ik gewoon weer in de}{plas gesmeten}\\

\haiku{Zij nam de arm                         .}{van haar galant en liep met}{hem het portiek uit}\\

\haiku{Ach, er zijn toch wel.}{fronten waarop Hitler heeft}{gezegevierd}\\

\haiku{Zij opende haar tas,:}{pakte er een zakje uit}{en zei                         opeens}\\

\haiku{Tot huis toe heeft dat.}{zuurtje als een kei op}{mijn maag gelegen}\\

\haiku{Waardig stapte ik.}{met mijn bordje                         terug}{naar de huiskamer}\\

\haiku{Hij vroeg dan ook geen,:}{twintig kaartjes                         voor het}{Stadion maar zei}\\

\haiku{{\textquoteleft}Nee, er h\'o\'eft immers,!}{nooit gekocht te worden}{dat liedje ken ik}\\

\haiku{Het vervelende.}{was echter dat die twee}{uit mijn beeld liepen}\\

\haiku{een troep b\^ete                         :}{schoolkinderen staat voor het}{tijgerhok en roept}\\

\haiku{Het zou niet haar                     .}{enige aanraking met de}{journalistiek zijn}\\

\haiku{Laat ons tezamen.}{in alle rust enige}{genres bekijken}\\

\haiku{Ze dachten dat ze.}{de brieven voor Holland in}{z\'e\'e moesten gooien}\\

\haiku{Die kerstavond was het.}{zwijgen nog drukkender in}{de keukenkamer}\\

\haiku{Toen hervatte de,:}{zwarte aarzelend doch niet}{onzakelijk}\\

\haiku{{\textquoteright} {\textquoteleft}H\`e, verdorie - dat,!}{w\'e\'et ik toch wel ik ben toch}{zeker niet kippig}\\

\haiku{{\textquoteright} begon hij                     op,:}{verlokkende toon maar ik}{wenkte af en vroeg}\\

\haiku{De                     uitspanning,:}{was reeds ver achter ons toen}{ik wanhopig vroeg}\\

\haiku{{\textquoteleft}Maar er moet wel een,.}{lappie om anders zou er}{vuil                     in komen}\\

\haiku{Moeizaam klom ik uit,.}{de veren                         want ik was}{heel dik geworden}\\

\haiku{{\textquoteright} Nu was het toch wel.}{zeker dat ik het een flink}{eind geschopt had}\\

\haiku{Maak het echter niet,... {\textquoteleft}}{\'al te suggestief want dat}{breekt je later op}\\

\haiku{Goed, als een vrouw in,.}{het parlement wil dan vind}{ik dat                     prachtig}\\

\haiku{Ik begreep er niets,,.}{van maar mijn moeder trok mij}{mee steeds huilend}\\

\haiku{{\textquoteright} En ze gaf me een,.}{klap op mijn hand zodat de}{punt in een plas viel}\\

\haiku{{\textquoteleft}Meester, we hebben,.}{er allemaal aan betaald}{behalve Frits}\\

\haiku{Het laatst zag ik de.}{heer Cohen in de oorlog}{voor het station}\\

\haiku{{\textquoteright} Alweer een grapje -?}{zou hij soms geheim abonnee}{van Kiekeboe zijn}\\

\haiku{Hij dacht even na, want.}{ondoordachte bescheiden}{geeft hij niet graag af}\\

\haiku{In de propvolle.}{trein wilde iedereen het}{meteen vertrappen}\\

\haiku{En n\'et toen ik in,.}{dat laatje rommelde kwamen}{die mensen binnen}\\

\haiku{Ik voor mij geloof.}{dat hij nu van dat zingen}{verlost is}\\

\haiku{{\textquoteleft}Wij gaan naar tante,{\textquoteright},.}{Aaltje zei het ene meisje}{toen wij weer reden}\\

\haiku{Hij zette het                     :}{toestel na  enig mikken}{aan zijn oor en sprak}\\

\haiku{De hele middag:}{op school hadie kennelijk}{zitten spinnen}\\

\haiku{Nou, aan zijn manier.}{van lopen kon je zien dat}{hij mij gelijk gaf}\\

\section{Jacob Cats}

\subsection{Uit: Huwelijk}

\haiku{Zo wordt nu 't oog,;}{hem los gedaan Dies ziet het}{zijn gevangen aan}\\

\haiku{Dit speeltjen heeft,.}{een grote sleep Men houdt daar}{eeuwig wat men greep}\\

\haiku{En wat de vrek in.}{d'aarde groef Dat is dan voor}{een malle oschroef}\\

\haiku{Veracht dan niet, o,.}{weerde vriend Wat u en mij}{ten goede dient}\\

\haiku{*~        T' samenspraak}{tussen Anna en Phyllis}{Gij die met vrucht}\\

\haiku{Phyllis Voor mij, ik,.}{wil hier open gaan Gelijk bij}{vrienden dient gedaan}\\

\haiku{Het gaat zo eender,.}{met de min Krakeeltjens}{brengen vriendschap in}\\

\haiku{Bij deze kwam een,.}{jong-gezel Een leerling}{in het minnespel}\\

\haiku{*~         De tweede valt,;}{hem in het haar En stelt een}{wonder vreemd gebaar}\\

\haiku{De last van 't huis,.}{de wil des mans En alle}{jaar een kind bijkans}\\

\haiku{Ziet, kind nu olijd ik,.}{dat ge gaat En dank u voor}{uw goeden raad}\\

\haiku{- Het meisje fluks en.}{onvermoeid         Kwam naar de}{kade toe geroeid}\\

\haiku{De vrijster wil naar, ';}{dezen kant Maar slaat het oog}{opt ander land}\\

\haiku{Schoon of een maagd een '.}{rugge biedtt En hindert}{aan het roeien niet}\\

\haiku{Ik prijze ja het,.}{echte-bond Maar niet als}{op den rechten stond}\\

\haiku{*~         Zegt wat ge wilt,,;}{slechts om de dracht Is menig}{mense hoog geacht}\\

\haiku{*~         Ik wil niet zo,}{versaagden geest Die ook zijn}{eigen schaduw vreest}\\

\haiku{*~         Hierachter woont,,}{een zeldzaam wijf En doet de}{jonkheid groot gerijf}\\

\haiku{*~         Door hem is al,.}{het stuk beleid Hem zij de}{lof in eeuwigheid}\\

\haiku{Een wijf, een krone,,.}{van den man Dat ospillen}{en dat sparen kan}\\

\haiku{als zich enig mens laat,;}{tot de lusten drijven De}{korte vreugd verdwijnt}\\

\haiku{wat ze brengen moet,.}{Omdat ze vreemde zucht in}{haren boezem voedt}\\

\haiku{een licht, een helder,.}{licht Dat roept u tot de zorg}{van uwen nieuwen plicht}\\

\haiku{Een wettig overheer.}{Moet ja de voorste zijn tot}{alle goede leer}\\

\haiku{Gij, die in echte,,,.}{paart Gaat heult met uwen man ook}{tegen uwen aard}\\

\haiku{hij, die het wonder,.}{ziet En prijst nog evenwel den}{groten Schepper niet}\\

\haiku{Wie dezen regel,,.}{houdt Die blijft dan even maagd ook}{als hij is getrouwd}\\

\haiku{gij, stelt uw dingen,.}{vast Eer u een hete koorts}{met smarten overlast}\\

\haiku{Vrouwen en mannen.}{moeten met een gelijke}{in leeftijd trouwen}\\

\haiku{klappers kletsmajoors;}{21 vuile zwangere 22}{sluiten opsluiten}\\

\haiku{met stijve kaken;}{onverschrokken 41 tot haar}{dagen bescheiden}\\

\section{August van Cauwelaert}

\subsection{Uit: Het licht achter den heuvel}

\haiku{Een enkele maal;}{kwam een vlaag van gezang den}{heuvel opgewaaid}\\

\haiku{Waarom was jonkheer?}{Leonce van de Burcht toen}{niet bijgesprongen}\\

\haiku{Dat duurde echter,.}{slechts eenige dagen hoogstens}{enkele weken}\\

\haiku{Hij zag den pastoor.}{staan praten tegen de vrouw}{met de kindertjes}\\

\haiku{{\textquoteleft}Ik zou 't op den,{\textquoteright}.}{duur nog te warm krijgen zei}{de geestelijke}\\

\haiku{{\textquoteright} De geestelijke '.}{vreesde dat zet jaar niet}{ten einde zou gaan}\\

\haiku{Willem klopte haar.}{vriendelijk op den schouder}{en dat deed haar goed}\\

\haiku{hij ontmoette was,;}{de veldwachter op gang met}{de lastenbrieven}\\

\haiku{Daarop schonk ze nog {\textquoteleft},{\textquoteright}.}{eens de glazen vol.Klara}{zei Willem gedempt}\\

\haiku{nog wankelend in,,.}{zijn wil aarzelend maar meer}{en meer verwonnen}\\

\haiku{Daar was geen een die,.}{nog verroerde maar ze}{keken hun oogen uit}\\

\haiku{De kinderen die,.}{onder het zeil te loeren}{lagen kletsten mee}\\

\haiku{Hij was blij dat hij.}{zijn militaire kleedij had}{mogen wegbergen}\\

\haiku{Willem zag vader,,.}{van het braakland komen naar}{hem toe recht en vlug}\\

\haiku{Het zong in Baltus.}{hart als een lied van kracht en}{voorspoed en geluk}\\

\haiku{Willem schrok toen hij.}{kort getrappel van voeten}{hoorde in den gang}\\

\haiku{{\textquoteleft}Het moet u wel vreemd,}{hebben geschenen zoo weer}{veilig thuis te zijn}\\

\haiku{{\textquoteright} {\textquoteleft}Nu is 't goed,{\textquoteright} zei,.}{Lucette toen Dorry maar}{doorging met zoenen}\\

\haiku{Ze was weer af en}{toe piano gaan spelen}{en ze las zooveel}\\

\haiku{Ze stapte nu 's,;}{zondags in losse lichte}{kleedjes en blouses}\\

\haiku{Tegen den kerkmuur.}{was Jan de grafmaker een}{put aan het delven}\\

\haiku{een lage vlucht van.}{duiven die uit het veld naar}{het dorp toeroeiden}\\

\haiku{{\textquoteright} vroeg Kardoentje, die.}{ongemerkt den veldwegel}{was afgekomen}\\

\haiku{Zijn bloed sloeg naar zijn.}{slapen en zijn hart ging plots}{aan het hameren}\\

\haiku{Niet zoo hard loopen,,.}{waarschuwde een vrouwestem}{die hij herkende}\\

\haiku{Toen kwam Lucette.}{zelf te voorschijn en wou dat}{ze even rusten zou}\\

\haiku{{\textquoteright} Theo lei 't nestje.}{in Mariette's hand als in}{een roze schelpje}\\

\haiku{waar ze te broeden.}{zaten en waar er kleintjes}{in het nest lagen}\\

\haiku{Want den volgenden;}{morgen kwamen er vreemde}{heeren op het dorp}\\

\haiku{{\textquoteright} 't Was vier uur v\'o\'or.}{de heeren klaar waren met}{hun onderzoek}\\

\haiku{Ze lachte een paar,.}{malen ironisch en keek in}{het glanzend water}\\

\haiku{tot ze heelemaal...}{voorbij was en spreken op}{zoo'n afstand evenmin}\\

\haiku{Dan heeft een boer geen.}{tijd om te rusten en geen}{tijd om te denken}\\

\haiku{Die zal nu duren.}{tot de nacht komt en de slaap}{over de  menschen}\\

\haiku{Moeder was nu thuis.}{gekomen en riep dat de}{koffie gereed stond}\\

\haiku{Maar daarop lag de.}{stilte weer onverbroken}{over land en hoeven}\\

\haiku{maar hij vermande,:}{zich lei zijne armen over}{haar schouders en zei}\\

\haiku{dat zeg ik niet, maar,.}{wat in mijne ooren valt}{dat knoop ik erin}\\

\haiku{{\textquoteright} {\textquoteleft}Laat het dan nu voor,{\textquoteright}.}{de eerste maal geschieden}{drong de jonkheer aan}\\

\haiku{{\textquoteleft}maar ik heb hem al,.}{maanden niet gezien tenzij}{van den predikstoel}\\

\haiku{Hij werd moe en ging '.}{even zitten in de schaduw}{vant prieeltje}\\

\haiku{{\textquoteleft}Mijn liefste meisje,{\textquoteright}, {\textquoteleft}.}{zei Kardoentjedat hoeft ge}{me niet te zeggen}\\

\haiku{{\textquoteright} riep hij terug, en.}{verdween achter het huis van}{den klompenmaker}\\

\haiku{Het werd een kermis.}{zooals zij er zes jaar lang geen}{meer gezien hadden}\\

\haiku{Harder door,{\textquoteright} riep Theo, '.}{den orgeldraaier toe in}{t voorbij walsen}\\

\haiku{maar Vital raakte.}{weer aarde en schoorde zijn}{beenen sterk als een muur}\\

\haiku{Maar Theo draaide zich;}{bliksemsnel om en tilde}{Vital op den rug}\\

\haiku{Theo had snel onder '.}{zijn linkerarmt hoofd van}{Vital gegrepen}\\

\haiku{Het leed en de zorg.}{kan een mensch tien jaar ouder}{maken op \'e\'en dag}\\

\haiku{De boeren waren.}{nu druk aan het ploegen en}{eggen en zaaien}\\

\haiku{De burgemeester;}{haalde er nauwelijks drie}{kandidaten door}\\

\haiku{*** ~ Maar Willem moest.}{den volgenden morgen al}{vroeg in de stad zijn}\\

\haiku{Toen Lucette thuis.}{kwam vond ze een briefje van}{Albert Durenne}\\

\haiku{Meneer pastoor moest.}{dus nog een tijdje wachten}{en hij wachtte nog}\\

\haiku{Een gelegenheid,.}{misschien een tijdverdrijf of}{een experiment}\\

\haiku{Ik groei er uit, dacht,.}{hij dan bij zichzelf ik voel}{dat ik er uit groei}\\

\haiku{En Klara zelf leek.}{lang zoo frisch en jong niet meer}{als in den zomer}\\

\haiku{Haar gelaat rees naar.}{het zijne op als een bloem}{en haar oogen glansden}\\

\haiku{Zijn hoofd helde naar.}{haar kopje toe en zijn arm}{boog om haar middel}\\

\haiku{De knecht die mest aan,.}{het onderploegen was hield}{zijn span stil en keek}\\

\haiku{want hij voelde zijn.}{hart onrustig en vaardig}{voor alle avontuur}\\

\haiku{Ze hangt boven de,;}{huizen in den hoogeren}{blauweren hemel}\\

\haiku{{\textquoteright} Willem keek verschrikt,.}{om want hij hoorde stappen}{zoo heel dicht bij hen}\\

\haiku{Zijn handen gleden.}{met een wonderen schroom en}{wijding over haar hoofd}\\

\haiku{Maar nu hoopte hij.}{dat hij haar niet meer onder}{de oogen komen zou}\\

\haiku{dat kon dan later.}{voor andere doeleinden}{worden aangewend}\\

\haiku{Fabrieken vergen.}{meer handen dan er in zoo'n}{streek beschikbaar zijn}\\

\haiku{{\textquoteleft}op voorwaarde dat...{\textquoteright}}{ge me niet een tweede maal}{met ontrouw betaalt}\\

\haiku{{\textquoteleft}Wie zich vergooit aan,,{\textquoteright}.}{een vrouw vergooit zijn toekomst}{bromde Baltus weer}\\

\haiku{Wat er verder met,.}{zijn leven gebeuren moest}{zou de toekomst leeren}\\

\haiku{Hij kwam verder op.}{den weg den pastoor tegen}{en hij hield hem staan}\\

\haiku{Ik zal u helpen.}{in den nood En vooral in}{het uur der dood}\\

\haiku{In de donkerte '.}{zag zet vuur van een pijp}{die werd aangepaft}\\

\haiku{Dat is tenslotte,{\textquoteright}.}{nog het beste wat we doen}{kunnen zei Mina}\\

\haiku{het is voor een vrouw,}{die begeert te voelen dat}{de jonge man dien}\\

\haiku{Ik zou zelf naar u,}{komen maar Dorry is nog}{wat onwel en nu}\\

\haiku{Een tijd bleef hij zoo,.}{nog zitten verstard in zijn}{leed en zijn wroeging}\\

\haiku{Willem leunde met.}{den rug tegen het raam en}{staarde naar den grond}\\

\haiku{Wat reeds voorbij was,.}{leek hem nu zoo jong ondiep}{en onervaren}\\

\haiku{Eindelijk hoorde.}{ze een stoel verschuiven en}{geklop van voeten}\\

\haiku{Zijne oogen lagen,.}{dieper in hunne holen}{maar glansden rustig}\\

\haiku{Moeder haalde nog.}{een paar kussens bij en hielp}{hem recht in zijn bed}\\

\haiku{Het gezang golfde.}{tegen de hoevemuren}{aan en over het dak}\\

\haiku{Moeder vroeg of ze '.}{t venster weer sluiten wou}{en hij vond het goed}\\

\haiku{Baltus fluisterde.}{hunne namen naarmate}{ze voorbijgingen}\\

\haiku{Ik was pas op de.}{universiteit en zij was}{nauwelijks achttien}\\

\haiku{{\textquoteright} {\textquoteleft}De pastoor zei toch,{\textquoteright}.}{dat ge er gelukkig zoudt}{mee zijn zei moeder}\\

\haiku{Hij stapte door den.}{mulligen landweg naar den}{grooten steenweg toe}\\

\subsection{Uit: Vertellen in toga}

\haiku{De slachter had de;}{wonde weer dicht gebrand met}{een gloeiend ijzer}\\

\haiku{Zoo stonden die twee,.}{elkaar daar uit te schelden}{ik weet niet hoe lang}\\

\haiku{En ze zou er heel.}{haar leven een lidteeken van}{dragen op haar been}\\

\haiku{Dan begon hij met,.}{zijn achterwerk te stooten}{lijk met een stormram}\\

\haiku{En den volgenden.}{morgen trok de Vilder naar}{zijnen advokaat}\\

\haiku{met verzoek dien dag.}{en dat uur te verschijnen}{in de raadskamer}\\

\haiku{Nu wou de rechter.}{nog weten wanneer de weg}{in orde zou zijn}\\

\haiku{En dan weet ge niet.}{meer of het van verdriet is}{of van zattigheid}\\

\haiku{Maar hij deed het niet,.}{en dien dag was hij overal}{bezig over zijn vrouw}\\

\haiku{Ze kwamen vragen.}{of hij dezen keer de boet}{niet kon kwijtschelden}\\

\haiku{Wanneer gaat ge toch,...}{ophouden vroeg de rechter}{ge drinkt u dood of}\\

\haiku{die komt er altijd.}{tusschen en dan vallen ze}{allemaal op mij}\\

\haiku{En het karretje,.}{reed weg de stad door en de}{vestingen voorbij}\\

\haiku{Meneer de rechter,,...}{zei ze ik zweer bij God en}{al zijn heiligen}\\

\haiku{- Da... da... wee... - Maar wat?}{hebt ge dan zelf verteld aan}{Verhulst Justine}\\

\haiku{Van den koster, zei,,?}{ik zoo ik weet van niks wat}{is daar mee gebeurd}\\

\haiku{Want ik ben nog wel.}{twee keeren blijven staan om Jef}{kwijt te geraken}\\

\haiku{ne satyr, Phanie,.}{dat is iemand die achter}{de kinderen loopt}\\

\haiku{Ja, die achter de,,.}{kinderen loopt zei Rinne}{dat is ne satyr}\\

\haiku{maar ge moet ons de,.}{volle waarheid zeggen juist}{zooals het is gebeurd}\\

\haiku{Maar de rechter hield:}{ze weer in en zei nog eens}{traag en met nadruk}\\

\haiku{Maar Buken liep de}{rest van den dag verloren}{in en rond het huis}\\

\haiku{Dat was een tweede.}{kermisdag en Buken liet}{nonkel niet meer los}\\

\haiku{Maar den volgenden '?}{middag was hij dadelijk}{aant reklameeren}\\

\haiku{maar iedermaal dat,:}{hij op een stoel klom moest hij}{weer eens zingen van}\\

\haiku{- 't Is toch kurieus,,...}{kloeg nonkel Baptist dat hij}{thuis zoo bot kan zijn}\\

\haiku{vroeg een vrouw, die met.}{een zwart korfken op haren}{schoot naast Buken zat}\\

\haiku{Buken was nog maar.}{pas den dorpel over of hij}{pakte nonkel aan}\\

\haiku{en met welk geld had,?}{hij dat perceel land betaald}{daar achter den smid}\\

\haiku{kocht een nieuwe klak;}{en een dubbelen col en}{een nieuwe kravat}\\

\haiku{en dat het koren,.}{schoon staat of dat de plaag aan}{de patatten zit}\\

\haiku{Maar toen Buken naar,;}{het onderzoek ging keurden}{ze hem ineens af}\\

\haiku{Buken werd triestig '.}{en krikkel ent werk werd}{allemaal te zwaar}\\

\haiku{Een droge hoest en,.}{daarop een kou en toen was}{het direkt gedaan}\\

\haiku{Het was duidelijk;}{genoeg dat het meisje op}{een verkeerd spoor was}\\

\haiku{En zei er nog iets.}{bij dat zuster Bernarda}{niet goed verstaan had}\\

\haiku{Mijn schatteke lief,,...}{kom dezen avond onder dit}{raam ik zal er zijn}\\

\haiku{Als er geen godsvrucht,.}{in zit is er niets aan te}{vangen met een kind}\\

\haiku{En de meisjes die,;}{binnenkwamen begonnen}{zooals het regel was}\\

\haiku{Een draai en de deur.}{ging open en Robbetje was}{weg de vrijheid in}\\

\haiku{Het was iets van dit,.}{alles het was misschien dit}{alles tegelijk}\\

\haiku{Toen de kleintjes van,.}{Robbetje weg waren kwam}{Ir\`ene naast haar staan}\\

\section{Charivarius}

\subsection{Uit: Nieuwe groene Charivaria. Deel 2}

\haiku{Zoo kan het vervoer.}{billijker en beduidend}{vlugger geschieden}\\

\haiku{{\textquoteright} (H.D.) {\textquoteleft}Dergelijke.}{landen zijn niet al te zeer}{in ernst te nemen}\\

\haiku{dit is vervelend,,.}{en Charivarius wou}{gaarne dat het N.v.d}\\

\haiku{20 - Wanneer gij u,.}{zwaarmoedig gevoelt wijt het}{niet aan anderen}\\

\haiku{Het schijnt ons niet meer,,.}{dan fair dat zij weten wat}{hun te wachten staat}\\

\haiku{, is te gast geweest.}{bij den chef van den Duitschen}{Generalen Staf}\\

\haiku{{\textquoteleft}In 't bijzonder.}{vestig ik de aandacht op}{den naam van de soep}\\

\haiku{Wellicht komt er uit.}{dezen oorlog toch iets goeds}{voor Nederland voort}\\

\haiku{{\textquoteright} Wij zijn benieuwd te,.}{hooren wie het ten slotte}{gekregen heeft}\\

\haiku{Wat wij nu zouden,,.}{willen weten is of er}{getrapt is of niet}\\

\haiku{Edelachtbare, De.}{Spaansche Vlieg is aanstootelijk}{voor de eerbaarheid}\\

\haiku{{\textquoteright}  (Lisser C.) ~ ;}{De Lisser C. is ons niet}{belangrijk genoeg}\\

\haiku{- Of 't hierop zal,.}{uitloopen kan voorshands nog}{niet beslist worden}\\

\haiku{Wij meenen onze.}{abonn\'e's met dat in kennis}{te moeten stellen}\\

\haiku{Meer menschen dan de,{\textquoteright}.}{kerk kon bevatten konden}{geen plaats vinden}\\

\section{J.B. Charles}

\subsection{Uit: Ontmoeting in den vreemde}

\haiku{Kraus Leermens, die ik!}{ergens in Nederland in}{een gesticht waande}\\

\haiku{Was het weer feest in?}{de stad en was  jij er}{deze keer \'o\'ok bij}\\

\haiku{, en het antwoord dat,,.}{vlot kwam bewees dat ik niets}{had moeten vragen}\\

\haiku{{\textquoteright} Hier onderbrak hij.}{zijn verbeelding om weer in}{het bier te kijken}\\

\haiku{Hij zou mij leren,,.}{kleinere jongens af te}{rossen schreeuwde hij}\\

\haiku{In de toekomst zal,.}{zij niet meer Moeder heten}{maar Minnares zijn}\\

\haiku{De bodem heeft wat.}{er op leefde altijd aan}{zich onderworpen}\\

\haiku{Ik ben niet met haar,.}{meegegaan maar heb haar met}{mij meegenomen}\\

\haiku{Maar nu die wezens,,?}{hier waarin zit het hem hun}{sociale sfeer}\\

\haiku{Ik ging natuurlijk,.}{niet maar bleef uren lang door de}{oude stad dwalen}\\

\subsection{Uit: Van het kleine koude front}

\haiku{waar in 1968 nog over,.}{X of Y gesproken is}{noem ik nu namen}\\

\haiku{Van daar uit had men.}{een beter gezicht op de}{ddr dan van Bonn uit}\\

\haiku{Alleen de boerin.}{blijft van Norfolk komen en}{blijft engels zijn}\\

\haiku{Het interesseert.}{hem helemaal niet waar ik}{dan w\`el vandaan kom}\\

\haiku{De duitsers, ook de,.}{kommunistiese konden}{niet saboteren}\\

\haiku{Ik zou van deze.}{man zelf horen wat hij zelf}{van de zaak denkt}\\

\haiku{Wij drinken een glas.}{bier in het kafee op de}{eerste verdieping}\\

\haiku{Men werkt daar dag en,.}{nacht in drie groepen van 90}{man telkens 8 uren}\\

\haiku{Zij vertellen mij.}{dat hij naar west gevlucht is}{en daar en daar zit}\\

\haiku{{\textquoteleft}die artikel-6'ers.}{denken dat ze beter zijn}{dan de anderen}\\

\haiku{In de eerste plaats.}{de Hortus Botanicus}{aan het Rapenburg}\\

\haiku{Misschien hoort het bij.}{een tijd van absolute}{tegenstellingen}\\

\haiku{Dimitroff spreekt, als,,.}{Van der Lubbe geen goed duits}{dat leek een voordeel}\\

\haiku{Ein Recht ist es der}{kommunistischen Partei in}{Deutschland illegal}\\

\haiku{Zijn vaders wens dat:}{hij officier zou worden}{heeft hij genegeerd}\\

\haiku{Dan is er een groot.}{stenen bord met veel namen}{daarin gebeiteld}\\

\haiku{Ik reis door dit land -?}{en geniet ervan terwijl}{ik broed op waarop}\\

\haiku{Een paar jaar later,,,:}{eind november 1961 bericht}{mijn krant uit Wenen}\\

\haiku{Hongarije v\'o\'or het?}{in deze rampzalige}{toestand geraakte}\\

\haiku{Een enkele heeft.}{ook wel es een beetje aan}{expansie gedaan}\\

\haiku{Nog steeds ben ik er.}{niet achter waarom zij dit}{staatshoofd zo haatte}\\

\haiku{Op 8 november;}{1931 wordt de eerste nieuwe}{kamer gekozen}\\

\haiku{En dan nog deze:}{kleine gedachte om mee}{naar huis te nemen}\\

\haiku{zij geven daarbij}{de nazi's van het land zelf}{de gelegenheid}\\

\haiku{kortom het is de.}{totale oorlog van het}{fascistiese beest}\\

\haiku{Hoe zwaarder men woog,.}{des te meer men zich blijkbaar}{kon veroorloven}\\

\haiku{Misschien zijn zij dood,,.}{waarschijnlijk zijn zij arm maar}{zij hadden gelijk}\\

\haiku{Ik had daar met mijn.}{buro aan meegedaan en}{hij keurde dat af}\\

\haiku{Anderen hebben.}{het er na de oorlog niet}{zo goed afgebracht}\\

\haiku{Een amsterdamse.}{konfektiefabrikant had}{gekollaboreerd}\\

\haiku{Ziehier de strekking.}{van mijn opmerkingen bij}{Martin korda dp}\\

\haiku{men zal dus misschien.}{willen begrijpen dat ik}{nu wel iets m\'oest doen}\\

\haiku{werkzaam waren op.}{hetzelfde uur dat Powers}{boven Rusland vloog}\\

\haiku{Van \'e\'en naoorlogs:}{verschijnsel was hij vooral}{onder de indruk}\\

\haiku{Het geweld valt het.}{geweld aan en de oorlog}{verdedigt zichzelf}\\

\haiku{Ik reken die hem,.}{toe ook al beweert hij dat}{het zijn schuld niet is}\\

\haiku{Daar stond echter een.}{onoverkomelijke}{barri\`ere voor}\\

\haiku{wat hij nog m\'e\'er was,.}{leren wij uit het gedicht}{van Engelman niet}\\

\haiku{Ja, maar ik h\'oef het.}{vod waar het in afgedrukt}{staat niet te lezen}\\

\haiku{Eros en Thanatos,,:}{liefde en haat voortbrenging}{en vernietiging}\\

\haiku{Het spijt mij, maar ik.}{breng je weer van Willy naar}{Siebe  terug}\\

\haiku{Dat is in onze.}{kring al dikwijls aan alle}{kanten bekeken}\\

\haiku{Maar  wat kan hij?}{doen om zijn inzichten tot}{gelding te brengen}\\

\haiku{Daarvan is helaas.}{in Oost-Duitsland op het}{ogenblik geen sprake}\\

\haiku{Kein Frieden ohne,.}{Freiheit aber keine Freiheit}{ohne Wahrheit}\\

\haiku{Het betrekkelijk.}{stellen van alle waarden}{laat er geen een over}\\

\haiku{het is uit angst voor.}{of kwaadwilligheid jegens}{bepaalde waarden}\\

\haiku{dan is het die in.}{het goede en het slechte}{konservatisme}\\

\haiku{Dat is ook moeilijk,.}{want er is namelijk niets}{bizonders gebeurd}\\

\haiku{Als wij w\`el van hem:}{vernemen en het is een}{ander geluid dan}\\

\haiku{De oorlog tegen.}{Egypte in 1956 was \'e\'en van}{die schietpartijen}\\

\haiku{Voor de joden is.}{deze nieuwe simpatie}{bijna even pijnlijk}\\

\haiku{De joden zijn nooit {\textquoteleft}{\textquoteright}.}{slechter geweest danwij en}{zijn nu niet beter}\\

\haiku{Het is een zwaar woord {\textquoteleft}{\textquoteright},.}{oorlogsmisdadiger maar}{het is niet te zwaar}\\

\haiku{Toen ik bij hem kwam,.}{was hij al steenkoud zo groot}{was het bloedverlies}\\

\haiku{Hij moest dansen met,.}{een kip in zijn handen zijn}{gebedskleden aan}\\

\haiku{Maar wij willen een;}{model van de bataafse}{kristenen tonen}\\

\haiku{Deze inleider,,:}{drs. G. Puchinger verklaart}{dat dit boek ons toont}\\

\haiku{Nu is het leven.}{dan ook eens een klein beetje}{naar voor de Colijns}\\

\haiku{Den Haag en vele,.}{plaatsen zijn vreselijk om}{aan te zien zegt men}\\

\haiku{Als het christelijk.}{is dan is het evengoed joods}{of mohammedaans}\\

\haiku{Men schijnt dit niet te.}{willen zeggen en evenmin}{hoe lang het nog duurt}\\

\haiku{Na de bijbelse.}{geschiedenis krijgen}{we aardrijkskunde}\\

\haiku{Het zou ook onjuist,.}{van hem geweest zijn zich pas}{weer op die XX}\\

\haiku{Hier zijn wij in het.}{gezelschap van Lunshof en}{wijlen Piet Bakker}\\

\haiku{Zo is het met dat,.}{grote humanisme het}{socialisme}\\

\haiku{Diezelfde stroom voedt,,.}{als men dat zo mag zeggen}{de loop van het boek}\\

\haiku{Het was dus een boek.}{dat niet klaar kon komen en}{toch een keer af moest}\\

\section{Ernest Claes}

\subsection{Uit: Clementine}

\haiku{, en hij wilde zich.}{zoo gauw mogelijk van het}{zaakje afmaken}\\

\haiku{- Clementine hield,.}{van niets en niemand zij dacht}{alleen aan zich zelf}\\

\haiku{{\textquoteleft}Clementine, ge,....}{moet niet ongerust zijn voor}{uw ouden dag wicht}\\

\haiku{De maanden gingen,.}{en daar kwamen er altijd}{weer andere}\\

\haiku{Het docht haar opeens.}{dat hij naar haar keek met een}{vreemd licht in de oogen}\\

\haiku{{\textquoteleft}Clementine, is?}{dat misschien voor mij dat ge}{die laat koud worden}\\

\haiku{maar dat hadt ge op.}{een andere manier ook}{kunnen regeleeren}\\

\haiku{Als er een brief kwam.}{deed ze dien zelf open en las}{hem mijnheerke voor}\\

\haiku{De pastoor is op.}{het kasteelke twee keeren}{komen aanbellen}\\

\haiku{Die koffie was zeer,.}{duti en de boterhams}{weinig in getal}\\

\haiku{{\textquoteright} De gierigheid van.}{Clementine vond daarin}{een groote voldoening}\\

\haiku{Hij heeft het deurtje.}{even open gezet om het vuur}{wat te temperen}\\

\haiku{tak komt er een dof,,:}{bijgeluidje bijna als}{een na-snikje zoo}\\

\haiku{Hier binnen leeft er,.}{iets onwezenlijks daar zijn}{geesten in dat huis}\\

\haiku{Clementine blikt,: - {\textquoteleft}'....}{op naar den wekker en zegt}{t Is negen uur}\\

\haiku{In het dorp blaften,.}{een paar honden en op den}{toren sloeg het uur}\\

\haiku{Als dat gedaan was, -....}{kwam ze voor de kachel staan}{aldoor kreunend \`eheu}\\

\subsection{Uit: De heiligen van Sichem}

\haiku{Ge leest er in uw.}{kerkboek gelijk ge thuis de}{gazet zoudt lezen}\\

\haiku{Z\'o\'o stom zijn we te.}{Sichem niet dat we daar ook}{geen oogen voor hebben}\\

\haiku{Daar moeten twee of.}{drie geslachten van menschen}{overheen zijn gegaan}\\

\haiku{Zooals hij daar staat is.}{Sint Rochus anders een heel}{kurieus manneke}\\

\haiku{Ze kijkt recht omhoog,.}{en ze lacht in heur eigen}{met wat ze daar ziet}\\

\haiku{Zijn gezicht is veel,.}{te weemoe{\"\i}g voor Sichem}{en te schoon ook}\\

\haiku{den uitgang gereed,,.}{wat verder de harmonie}{en ze komen af}\\

\haiku{En het is nu zoo '.}{stil gelijk oft heele}{dorp \'e\'en kerk was}\\

\subsection{Uit: Herodes}

\haiku{En deze, terwijl,.}{hij ze aankeek wist wat er}{in hun geest omging}\\

\haiku{Hij had geen tolk noodig,.}{want hij zelf sprak de meeste}{Oostersche talen}\\

\haiku{Misschien heeft het stuk...}{wel een wondere kracht in}{de Oosterlanden}\\

\haiku{waar hij anderen,.}{niet vertrouwde begreep hij}{er ditmaal niets van}\\

\subsection{Uit: Jeroom en Benzamien}

\haiku{ik geloof dat ik.}{niet verder op het geval}{zou zijn ingegaan}\\

\haiku{{\textquoteleft}Als ge daar niet in,.}{zijt verstaat ge dat niet zoo}{van den eersten keer}\\

\haiku{Ten slotte weze.}{nog herhaald dat zij beiden}{katholiek waren}\\

\haiku{Enkele keeren had.}{hij aangebeld en gevraagd}{de woning te zien}\\

\haiku{Liever vet smelten!...}{en biefstukken snijden tot}{hij er bij dood viel}\\

\haiku{{\textquoteright} {\textquoteleft}Altijd minder dan.}{een eigen appartement}{met een huishoudster}\\

\haiku{Die gingen ook zoo.}{moeilijk aan en uit over hun}{weeke slagersvingers}\\

\haiku{De R\'ev\'erende:}{M\`ere groette haar met een}{buiging van het hoofd}\\

\haiku{Ze zetten hun hoed.}{maar terug op toen ze aan}{het straathek kwamen}\\

\haiku{Het was hun of er,.}{een nieuw leven begon op}{een hoogeren trap}\\

\haiku{Ik dacht dat we daar...}{iederen dag naar de kerk}{zouden moeten gaan}\\

\haiku{In de Bermijnstraat.}{vroeg hen op een keer iemand}{in het Vlaamsch den weg}\\

\haiku{Benzamien was op,:}{het punt te antwoorden toen}{Jeroom ineens vroeg}\\

\haiku{eerste, tweede en,.}{derde klas naar het kostgeld}{dat men betaalde}\\

\haiku{Mijnheer B. Van Snick,..., '.}{Beenhouwer of Boucher als}{het int Fransch was}\\

\haiku{Van het drama dat?}{zich voor haar persoon voltrok}{in hun gezond hart}\\

\haiku{Hij hoorde de deur,...}{opengaan hij trok den handdoek}{van zijn gezicht weg}\\

\haiku{De R\'ev\'erende.}{M\`ere Sup\'erieure}{stond in de kamer}\\

\haiku{Haar week wit gezicht,.}{bleef daarbij even koud vroom en}{ontoegankelijk}\\

\haiku{Postcheckrekening) -.}{Nr 49911 bijgestaan door twee}{gewone priesters}\\

\haiku{Benzamien was niet.}{geheel zeker van zijn stuk}{met zijn vertaling}\\

\haiku{De barones de.}{Haricourt zei dat hij un}{humoriste was}\\

\haiku{Die roode sjawl op.}{de bank was van madame}{Van Holleb\`eque}\\

\haiku{Eens zat ze met haar.}{broertje op het grasplein voor}{het huis te spelen}\\

\haiku{Jeroom waagde nu,:}{toch een kleine opmerking}{bijna fluisterend}\\

\haiku{Van de familie.}{waren er hoop en al twee}{menschen gekomen}\\

\haiku{Een Spanjaard heeft dat, '.}{geschreven maart is even}{waar voor Anderlecht}\\

\haiku{De teerlingbak is. ...}{ons hart waarin het noodlot}{zijn dobbelsteenen gooit}\\

\haiku{De oogen van Jeroom,,.}{Meulepas waren ijskoud}{hard ongenadig}\\

\haiku{Het dikke blauwe.}{wangvleesch trok neerwaarts over}{zijn jukbeenderen}\\

\subsection{Uit: Kobeke}

\haiku{Bruu Kalot put een,.}{akerke water en laat zijn}{twee honden drinken}\\

\haiku{3 Die vader van,,.}{Kobeke och wat is dat}{een aardige vent}\\

\haiku{'t Wordt zoo erg dat.}{Broos het niet meer uithouden}{kan en hardop lacht}\\

\haiku{Pardoes draait op zijn,,.}{rug de pooten omhoog en}{krinkelt van plezier}\\

\haiku{Ze moeten alle,.}{twee lachen want dat was al}{lang afgesproken}\\

\haiku{Broos is al achter,.}{den eikenkant op weg naar}{huis als hij nadenkt}\\

\haiku{Ze loeren van tijd,.}{tot tijd eens om om den weg}{niet te vergeten}\\

\haiku{Boven de bosschen.}{komt het groote witte gezicht}{van de maan loeren}\\

\haiku{Pardoes dribbelt met,.}{zijn kop naar den grond zijn staart}{tusschen zijn pooten}\\

\haiku{De menschen loeren,,:}{van achter het gordijn en}{lachen en zeggen}\\

\haiku{{\textquoteleft}Gedoopt is gedoopt{\textquoteright},, {\textquoteleft}....}{zegt hijen ik doe geen twee}{missen voor een geld}\\

\haiku{Ze knoopt heur jakske '.}{los en geeft Kobeke en}{Nelleken mem}\\

\haiku{Bellemoeike wordt.}{wakker geschud en ze weet}{niet meer waar ze is}\\

\haiku{Maar Melle Komfoor.}{was ook wat locht in den kop}{van al die drupkens}\\

\haiku{Hij zal 't schaap zijn{\textquoteright},.}{nagelband nog doen springen}{zegt Bellemoeike}\\

\haiku{Hij trekt zijn ooren.}{achteruit en likt nu en}{dan eens over zijn snuit}\\

\haiku{Lulle-Mie kijkt.}{van uit heur stalleke door}{de spleet van de deur}\\

\haiku{Dat is alles wat '. '}{er over zoon onnoozel schaap}{te vertellen is}\\

\haiku{{\textquoteright} {\textquoteleft}Zeker, m'ne jong,.}{en dan kookt Tekla Penne}{daar hutsepot van}\\

\haiku{{\textquoteright} 't Is te zien dat.}{Kobeke en Nelleke}{het benauwd krijgen}\\

\haiku{Ze treiteren den}{koster omdat hij al twee}{keeren buiten gaan zien}\\

\haiku{Mieke Lies krest van ',.}{t giechelen precies of}{ze gekitteld wordt}\\

\haiku{Kajoet zit neven,.}{het vuur te slapen zijn oogen}{vast toegeknepen}\\

\haiku{Tegen den gevel,,.}{van de hut op een hoopke}{slommer ligt Kajoet}\\

\haiku{Dat konijn had hij,,.}{met een zuur gezicht zonder}{smaak opgevreten}\\

\haiku{Van dichterbij klinkt:}{nu de minneklagende}{lokroep van Purre}\\

\haiku{Lulle-Mie heeft.}{op iederen horen een}{dikke raap steken}\\

\haiku{{\textquoteleft}Nee Nelleke, maar.}{me docht precies dat er weer}{iemand op me riep}\\

\haiku{{\textquoteright} Broos is met hak en '.}{bijl aant struiken uitdoen}{tegen de zandbaan}\\

\haiku{Zijn broek zit er vol,.}{van en het kriebelt hem nog}{over zijn blooten rug}\\

\haiku{Ze lezen in hun.}{boek iets over paddestoelen}{en dolle kervel}\\

\haiku{Hij ziet de sneeuw op.}{de haag en op de zwik van}{de putkuip liggen}\\

\haiku{Hier en daar lukt het.}{toch en krijgen ze een cent}{of een koekske}\\

\haiku{Het raam van den hof,.}{staat open en de vogelkens}{zingen zijn oogen toe}\\

\haiku{Bellemoeike weet.}{van niks en leest maar voort aan}{heur paternoster}\\

\haiku{De jongens wringen.}{hun achterwerk efkens over}{den stoel weg en weer}\\

\haiku{ik zou dien Golo.}{de oogen uit zijn leelijken}{kop gekrabd hebben}\\

\haiku{Hij lacht niet zoo veel,.}{meer en ze gaan allang niet}{meer gelijk zwemmen}\\

\haiku{Ze gaan nog dikwijls '.}{gelijk hout of denappels}{rapen int bosch}\\

\haiku{En op een schoonen.}{dag in den vroegen voortijd}{komt Dorusoome binnen}\\

\haiku{Hij had niet gepeinsd.}{dat het zoo erg zou geweest}{zijn bij Nelleke}\\

\haiku{Hij meent dat hij een.}{gezicht moet zetten als een}{halve heilige}\\

\haiku{Broos kan Nelleke,,.}{ginder tegen dien boom niet}{uit zijn kop zetten}\\

\haiku{Ze zeggen dat die,.}{iederen dag vleesch eten}{en vrijdags stokvisch}\\

\haiku{Dat is een nieuwe,....}{Vader Abt die komt recht van}{den Paus van Rome}\\

\haiku{Had iemand van de?}{Broeders hem ooit gezien met}{een stuk in zijn kraag}\\

\haiku{Maar zelfs onder die.}{mis kondt ge hooren dat ze}{niet akkoord waren}\\

\haiku{Ge prakkezeert over,.}{alle dingen en ge weet}{niet van waar het komt}\\

\haiku{'t Was de tijd niet.}{meer dat de merels floten}{in den vollen dag}\\

\haiku{Maar ineens hoorde.}{hij ze zoo fel dat al de}{rest niet meer bestond}\\

\haiku{Die klanken botsten.}{achter tegen zijn kop lijk}{of er iemand sloeg}\\

\haiku{Al van er naar te.}{kijken deden Broederke}{Kobus zijn oogen zeer}\\

\haiku{En hij is er nu.}{zeker van dat hij geen kwaad}{doet met weg te gaan}\\

\haiku{Een vrouwmensch  stond, '.}{in haar hemd voor den open haard}{bezig aant vuur}\\

\haiku{Hij hielp nu mede,}{de kar laden en als ze}{de vracht had pakte}\\

\haiku{Over het klooster van.}{Zeveslote was het}{nu heimelijk stil}\\

\haiku{Wat had hij nu met?}{zijn heiligheid en zijn goed}{voorbeeld uitgehaald}\\

\haiku{{\textquoteright} Kobeke met zijn.}{pakske in zijn hand stond wat}{bedutst voor het wijf}\\

\haiku{{\textquoteright} Kobeke verstond,.}{dat niet en hij witte voort}{tot tegen den avond}\\

\haiku{Het jong ding zat al.}{bekanst heelegaar op den}{schoot van den koetsier}\\

\haiku{Een vrouwmensch, het jong,.}{ding docht hem antwoordde ook}{met een dubbel tong}\\

\haiku{Hij deed het met een.}{gezicht om iemand ineens}{nuchter te maken}\\

\haiku{Maar hij zag er niet.}{zooveel kwaad in omdat het}{een pastoorsmeid was}\\

\haiku{Ze hooren uwen hond ' '.}{wel bassen maart is of}{zet niet hooren}\\

\haiku{Hij luisterde met.}{al zijn attentie naar den}{man op het verhoog}\\

\haiku{Kobeke stond recht,,.}{kruchte van de pijn en ging}{door dit gat loeren}\\

\haiku{Een er van was de}{vieze werkman die den avond}{te voren neven}\\

\haiku{Hoort nu en ziet, hier '.}{begint nu echt het ende}{vant verhaal}\\

\haiku{Kobeke stond een.}{moment in de straat naar links}{en rechts te kijken}\\

\haiku{Straat uit en straat in,,.}{altijd rechtdoor naar den kant}{waar de zon opkomt}\\

\haiku{'t Is waar, het is,.}{zondag vandaag hij had dat}{bekanst vergeten}\\

\haiku{Rond den noen kwamen.}{ze in een parochie waar}{hij in den trein moest}\\

\haiku{{\textquoteright} Onder de wortels.}{van dien zelfden den liep een}{konijn door zijn pijp}\\

\subsection{Uit: Namen 1914}

\haiku{het is te koud en,.}{we zijn te plots opgejaagd}{uit onze nachtrust}\\

\haiku{{\textquoteleft}Zwijg, sergeant, on.}{ne fait pas d'omelettes}{sans casser des oeufs}\\

\haiku{Het is hoogstens tien,.}{minuten ver maar het is}{een akelige tocht}\\

\haiku{Ik weet niet wat het,.}{is maar daar is iets dat ik}{nog moet bijvoegen}\\

\haiku{{\textquoteright} Beneden in het.}{dal van Gelbress\'ee begint}{een huis te branden}\\

\haiku{Ik begrijp maar niet.}{waarom die batterij daar}{verlaten blijft staan}\\

\haiku{Ditmaal zijn het geen,.}{kleine groepjes maar een dicht}{gesloten bende}\\

\haiku{Hij kijkt even op, en:}{fluistert mij zachtjes toe als}{tot een oude vriend}\\

\haiku{'t Is een jonge,.}{kerel met fijne handen}{en een wit gezicht}\\

\haiku{Over zijn achterhoofd.}{en zijn hals loopt het bloed neer}{op zijn  ransel}\\

\haiku{Bruno komt eveneens.}{de gewonden verzorgen}{achter het huisje}\\

\haiku{Daar liggen nu meer,.}{gewonden in zijn loopgraaf}{en ook meer doden}\\

\haiku{Hij doet nog eens de,,:}{ogen wijd open staart mij vlak in}{het gezicht fluistert}\\

\haiku{Ik adem met wellust.}{de frisse lucht in en word}{bijna misselijk}\\

\haiku{Ik word duizelig,.}{langzaam aan onbewust van}{de plaats waar ik ben}\\

\haiku{Hij ziet er deerlijk,,.}{uit met zijn kortgeknipte}{haren zijn rond hoofd}\\

\haiku{Dan vraagt hij of er.}{niemand een kaart van de streek}{in zijn bezit heeft}\\

\haiku{{\textquoteleft}Trek je  toch niets.}{aan van wat die melkbaarden}{je daar vertellen}\\

\haiku{De ene wagen houdt,.}{stil na de andere vlak}{voor de ingangshal}\\

\haiku{Hij roept nog veel meer,.}{telkens de herhaling van}{dezelfde woorden}\\

\haiku{Wij zijn even goede...}{patriotten als gij en}{ik ben ook beschaamd}\\

\haiku{Het is een avond van,.}{bijster bange verschrikking}{en naar als een moord}\\

\haiku{{\textquoteright} {\textquoteleft}In het klooster, daar,,.}{achter die muur meen ik in}{het dorp zeker niet}\\

\haiku{{\textquoteleft}Moeten die arme?}{mensen nu de hele nacht}{in de kerk blijven}\\

\haiku{Ik leun tegen een.}{stoel en moet zin voor zin zijn}{toespraak vertolken}\\

\haiku{Het dienstpersoneel...}{moet eveneens de nacht in de}{kapel doorbrengen}\\

\haiku{Vele gewonden,,.}{vooral verbranden zijn in}{de nacht gestorven}\\

\haiku{hij schudt het hoofd heen:}{en weer en zegt nu en dan}{op hoog-ijle toon}\\

\haiku{{\textquoteright} komt hij mij nog eens,:}{knipogend nafluisteren en}{ik zeg in mezelf}\\

\haiku{Een enkele keer.}{hoor ik door het openstaande}{raam iemand zingen}\\

\haiku{Enkele dagen.}{later vertrekken de twee}{Belgische dokters}\\

\subsection{Uit: Oorlogsnovellen}

\haiku{Het is van ruw en,. :}{ongeschaafd hout maar het is}{mijn eenig eerekruis}\\

\haiku{-: Ik zat tegen den,.}{gevelmuur en Saelens lag}{languit op den grond}\\

\haiku{Hun stemmen waren. :}{zacht en kalm als de stilte}{van den zomernacht}\\

\haiku{{\textquoteleft}Il n'y a pas de,.}{consommations \`a dix}{centimes m'sieu}\\

\haiku{De kommandant kreeg '. :}{ern gegeneerden blos}{van op de wangen}\\

\haiku{A la guerre,.}{comme \`a la guerre}{zegt de Franschman}\\

\haiku{Hun handen waren;}{verweerd en hun gezichten}{gebruind door de zon}\\

\haiku{'t Moedertje deed.}{hare schoenen uit en liep}{op hare kousen}\\

\haiku{{\textquoteleft}Marcus, mijn jongen,!}{toch en hebben ze u zoo}{maar doodgeschoten}\\

\haiku{ons als van verre. :}{gezien door tegen den grond}{liggende wachters}\\

\haiku{Onze oogen deden '.}{pijn vant strakke staren}{in de duisternis}\\

\haiku{De Franschen hebben.}{een verbazend gemak om}{boeken te schrijven}\\

\haiku{Ik bleef en ben steeds {\textquoteleft}{\textquoteright}.}{in Mammy's oogenle cousin}{\`a M'sieu Raoul}\\

\haiku{Maar wanneer zij hem.}{aanspreken schrikt Vadertje}{Musset schichtig weg}\\

\haiku{Dan weet hij niet, o,.}{neen dan begrijpt hij niet wat}{het allemaal is}\\

\haiku{Ik vraag hem wat het,. :}{is hij trekt even de schouders}{op en gaat verder}\\

\section{Willem de Clercq}

\subsection{Uit: Woelige weken: november-december 1813}

\haiku{Men zegt dat de prins.}{van Eckm\"uhl heeft moeten}{capituleren}\\

\haiku{Men zegt dat er te.}{Utrecht een korps van 16 000 man}{gevormd moet worden}\\

\haiku{Onder vrienden legt.}{men Franse pakkers deze}{woorden in de mond}\\

\haiku{Niets, zie daar het feit.}{dat wij moeten toevoegen}{aan onze nieuwtjes}\\

\haiku{Vandaag was alles,.}{vervuld van neerslachtigheid}{van ontmoediging}\\

\haiku{Op de Leliesluis,,.}{houdt mij iemand in een kraag}{gewikkeld staande}\\

\haiku{Dit is geen lachend.}{beeld door het spelend vernuft}{eens dichters gevormd}\\

\haiku{dat het getal der.}{vrijwilligers in Den Haag}{zeer groot geweest is}\\

\haiku{Men geeft in Den Haag, -.}{aan elke vrijwilliger}{f 100 handgeld}\\

\haiku{Men zei dat morgen.}{onze ware regering}{zou worden gemaakt}\\

\haiku{Hij zat op een klein,.}{paard had een blauw buis aan en}{droeg een grote lans}\\

\haiku{Wij zullen nu weer,.}{een volk en de prins onze}{soeverein worden}\\

\haiku{Gisteren is hier.}{een bende Russisch voetvolk}{binnen getrokken}\\

\haiku{Klijn, Lierzangen van.}{Helmers Klijn moest in Felix een}{spreekbeurt vervullen}\\

\haiku{Kemper zei hierover.}{te zullen denken en reed}{naar Leiden terug}\\

\haiku{Bij de heer Collot.}{d'Escury ontvangen zij}{dezelfde boodschap}\\

\haiku{Kemper verklaarde.}{dat niets hem zo speet als het}{geval van Truguet}\\

\haiku{Harpe, J.F. de la (-) -,.}{17391803 Frans criticus en}{dichter hoogleraar}\\

\haiku{Scaevola = die.}{de linker in plaats van de}{rechter hand gebruikt}\\

\section{Alfons de Cock}

\subsection{Uit: Herdenkings-album 1850-1921}

\haiku{Alfons de Cock-}{Herdenkings-album 18501921}{Colofon}\\

\haiku{hij begon te 5 ' '.}{uurs avonds en eindigde}{te 2 uurs nachts}\\

\haiku{In zijn vertrouwde.}{omgeving was hij vooral}{te bewonderen}\\

\haiku{- Mevrouw Courtmans, geb. ( -).}{Johanna Desideria}{Berchmans1811 1890}\\

\haiku{De Molenaar, de.}{wever en de kleermaker}{uit de volksmeening}\\

\haiku{Gids15, jg. 1905, blz. - -.}{155 169 en 245 284.1906Het Liedje}{van Heer Halewijn}\\

\haiku{207 - 208.1911De Macht der.}{Kinderlijke Onschuld in}{de Sagenwereld}\\

\haiku{En 't en is maar '.}{een te kort Ent is Pe}{Joenens rostekop}\\

\haiku{{\textquoteleft}Keizer Karel mee.}{al zijn knechten Ging tegen}{de Franschen vechten}\\

\haiku{twee zijn Westvl, en,,,.}{komen voor in Rond de Heerd}{I 34 en III 32}\\

\haiku{{\textquoteleft}Heere Koning, die ',!}{ins Gravenhage zijt}{vervloekt zij uw naam}\\

\haiku{Van heden af{\textquoteright} zei, {\textquoteleft};}{toen de Heerontneem ik de}{boomen alle spraak}\\

\haiku{de goedhartige}{meid eener gravin had juist}{een hongerigen}\\

\haiku{{\textquoteright} - {\textquoteleft}Neen,{\textquoteright} antwoordde het, {\textquoteleft}.}{meisjevoor mijn meester wil}{ik alles wagen}\\

\haiku{Eer zij vertrekt, streelt.}{zij met haar vingers door den}{knaap zijn kroezelkop}\\

\haiku{Deze draalt niet, om.}{den getroffen persoon te}{komen onttooveren}\\

\haiku{tous mes cheveaux sont,{\textquoteright}.}{tomb\'es et voil\`a pourquoi je}{porte un bonnet}\\

\haiku{le guerrier.}{\`a la t\^ete d'or avait de}{nouveau disparu}\\

\haiku{Peu apr\`es une f\^ete:}{solennelle devait avoir}{lieu \`a la cour}\\

\haiku{je ferai passer}{tous les invit\'es devant}{moi et chacune}\\

\haiku{vaste \'etendue d'eau,).}{o\`u le pers\'ecuteur se}{noie avec son chameau}\\

\haiku{16) SEBILLOT, C.,, (:}{de la Haute-Bretagne}{III no 13brosse}\\

\haiku{un loup, un ours, un ().}{renardici il y a 3}{courses distinctes}\\

\haiku{27) KRAUSS, Sagen und,, (:}{Maerchen der S\"udslaven}{I no 89brosse}\\

\haiku{31Die Evangelien,,}{vanden Spinrocke 3e dag}{3e cap. 32Volgens}\\

\haiku{relat{\'\i}ves \`a...,.}{l'anatopiie \`a la}{m\'edecine 304}\\

\haiku{106Le h\'eros l\`eve,.}{l'\'ep\'ee il ne la jette}{pas derri\`ere}\\

\subsection{Uit: Vlaamsche sagen uit den volksmond}

\haiku{{\textquoteleft}Ja maar, meneer, dat,,{\textquoteright}.}{is geen vertelsel zulle}{dat is waar gebeurd}\\

\haiku{Een jongeling van.}{Kuntsem3 maakte het hof aan}{een beeldschoon meisje}\\

\haiku{- {\textquoteleft}Weet gij,{\textquoteright} vroeg hem op, {\textquoteleft}?}{zekeren dag een vrienddat}{gij met een heks vrijt}\\

\haiku{Sagenboek, I, 7-),.}{11 nl. de terugkeer van}{vrijer en vrijster}\\

\haiku{de bewoners van {\textquoteleft}{\textquoteright}.}{Iddergem worden dan ook}{Toovereers genoemd}\\

\haiku{WOLF, D. M\"archen,,,, {\textsection}.}{u. Sagen nr 159 WUTTKE D.}{Volksaberglaube 415}\\

\haiku{De zuster van den.}{jongeling ging een wijzen}{pater raadplegen}\\

\haiku{Bevend van schrik, wist;}{hij niet wat doen en liet de}{katten maar begaan}\\

\haiku{{\textquoteleft}Morgen zal ik een{\textquoteright}.}{kapmes meebrengen en eene}{haar beenen afsmijten}\\

\haiku{Het duurde niet lang.}{of hij zag de zwarte kat}{weer eens afkomen}\\

\haiku{Op eens springt er een,.}{schoone gevlekte kat uit}{een kreupelboschje}\\

\haiku{De weduwe legt,.}{de zaak uit zooals zij gestaan}{en gelegen is}\\

\haiku{{\textquoteright} - {\textquoteleft}Kom maar eens mee naar,,}{den stal en gij zult zien welk}{schoon wit paardeken}\\

\haiku{Een vrouw van Oelter.}{was naar Ninove drie}{brooden komen koopen}\\

\haiku{Er was toove ij ',,.}{int spel dat was zeker}{zei men eenparig}\\

\haiku{Wanneer de boer in.}{den morgen den betooverde}{v\'o\^or de staldeur vond}\\

\haiku{het voert zelfs al het,.}{geld mede dat er mede}{in aanraking komt}\\

\haiku{Hij bemerkte een.}{vrouwmensch op zijn stuk land en}{liet haar betijen}\\

\haiku{Steekt dit onder den,.}{dorpel van de deur dan kan}{zij niet meer binnen}\\

\haiku{{\textquoteright} De ouders van 't;}{kind waren in den hoogsten}{hemel van blijdschap}\\

\haiku{Intusschen werd de;}{zieke onderzocht door een}{tweeden geneesheer}\\

\haiku{Intusschen was de.}{zoon de Jezu{\"\i}eten van}{Aalst gaan raadplegen}\\

\haiku{- {\textquoteleft}Ge moet,{\textquoteright} zei ze, {\textquoteleft}een,.}{keermis laten doen om het}{paard te doen weerkeeren}\\

\haiku{het was de hond, en.}{den ganschen nacht moest hij er}{mede rondloopen}\\

\haiku{Dit diertje werd zoo,.}{groot dat hij er op den duur}{wel kon op rijden}\\

\haiku{Hij sprong met zijn twee.}{voorpooten op haar schouders}{en lachte luidop}\\

\haiku{Men bewaakte het,;}{paard en toch was het haar des}{morgens gevlochten}\\

\haiku{Maar de zoon gedroeg.}{zich uiterst slecht en deed niets}{anders dan jagen}\\

\haiku{Zij hebben maar te,.}{willen en onmiddellijk}{wordt hun wensch vervuld}\\

\haiku{En allen zetten,.}{zich aan tafel en aten en}{dronken smakelijk}\\

\haiku{{\textquoteright} 's Avonds was Jan in.}{de kerk en ging zich op den}{preekstoel verstoppen}\\

\haiku{De veldwachter stierf.}{evenwel zonder rechterlijk}{vervolgd te worden}\\

\haiku{M... kwam verleden,,.}{jaar op een Zaterdagavond}{van zijn werk terug}\\

\haiku{{\textquoteright} ~ (Wichelen.) ~ .}{Lezing b doet meer aan de}{mare denken}\\

\haiku{In Munkzwalm wijst een,.}{stalkaars de plaats aan waar een}{schat verborgen ligt}\\

\haiku{Steekt men den vinger,;}{naar een stalkaars uit dan komt}{zij er opzitten}\\

\haiku{toen bevond hij zich,.}{in Haasdonk anderhalf uur}{van zijn woning of}\\

\haiku{In Volkskunde, II, -,,-;}{89 vlg. Vgl. Wodana 23 vlg.}{verder blz. 154156}\\

\haiku{Men vertelt, dat er.}{in een hol te Lebbeke}{kabouters woonden}\\

\haiku{Hieruit blijkt tevens.}{hoe sagen afslijten en}{zich vervormen}\\

\haiku{{\textquoteright} - {\textquoteleft}J a, zeker,{\textquoteright} zei, {\textquoteleft}}{de weveren ik durf het}{u wel zeggen ook}\\

\haiku{Over het thema, {\textquoteleft}de{\textquoteright},.}{Duivel als kaartspeler zie}{men DE COCK-TEIRL}\\

\haiku{Ik heb het door mijn.}{grootmoeder wel honderd maal}{hooren vertellen}\\

\haiku{Vgl. ons volg. nr en.}{zie de aldaar gegeven}{referencies}\\

\haiku{er waren drie groote,,.}{merken als ingebrand op}{de deur te zien}\\

\haiku{Standaert, van Haaltert,.}{ging op een Zondag morgen}{met een vriend naar huis}\\

\haiku{daar bevond zich het,.}{huis van Schallekens en hier}{vluchtte hij binnen}\\

\haiku{uw eerste werk van,.}{dezen morgen zult gij heel}{den dag voortzetten}\\

\haiku{- {\textquoteleft}Uw eerste werk van.}{dezen morgen zult gij heel}{den dag voortzetten}\\

\haiku{) ~ Wordt ook verteld,.}{te Aalst te Wichelep en te}{Adegem Vgl. Wodana}\\

\haiku{Vgl onze nrs 235,,,.}{237 244 en 260 alsmede}{A. de COCK en IS}\\

\haiku{Vgl. onze nrs 224,,,,;}{en 226 SCHOUTENS Maria's}{Henegouw 54 130}\\

\haiku{Te Nevele, dicht {\textquoteleft}{\textquoteright},,.}{bij hetdorp ligt een weide}{de Oostbroek genaamd}\\

\haiku{Een tweede maal werd,.}{de deur geopend doch met}{hetzelfde gevolg}\\

\haiku{- {\textquoteleft}Houd u aan dit strooitje,;}{vast ik zal u tegen den}{donder beschermen}\\

\haiku{Sint-Leu ging binnen,.}{en vroeg aan den smid of hij}{bij hem werken mocht}\\

\haiku{(Zegelsem.) ~ Vgl.,,;}{P. DE MONT en A. de COCK}{Vlaamsche Vert. blz. 364}\\

\section{Frans Coenen}

\subsection{Uit: Burgermenschen}

\haiku{Zij trok hem aan 't,.}{handje mee den kruiwagen}{voor zich uitstuwend}\\

\haiku{Je leefde hier toch '...}{zoo ondert uitschot van}{de samenleving}\\

\haiku{En de trekken van.}{zijn schamelen kop stonden}{zorgelijk gegroefd}\\

\haiku{Hij zou d'r 's even,!}{een standje maken as ze}{terug-kwam}\\

\haiku{Maar ondertusschen '.}{bleef die Leen maar weg ent}{was al over vieren}\\

\haiku{as-ti dat hoort '...}{is t'r heelemaal geen huis}{metm te houen}\\

\haiku{En ze kleedde 'm '... '!}{toch zoo graag int witt}{Stond hem zoo lekker}\\

\haiku{En z\'o\'o aardig was.}{dat vrouwtje waarempel ook}{niet altijd voor d'r}\\

\haiku{Kom, gekheid,... ze zou '...}{t met een balletje straks}{wel weer goed maken}\\

\haiku{Maar ook die kleine.}{plek van rustiging wilde}{allengs verzinken}\\

\haiku{Het was al lang over,.}{en weer afgesproken maar}{telkens uitgesteld}\\

\haiku{Afijn... als de jongen...}{dan nog maar iets m\'e\'er werd dan}{zij geweest waren}\\

\haiku{Lena hoorde Truus'.}{zeurige verhalen over}{haar huishouen toe}\\

\haiku{Dat eeuwig grootdoen ',...}{vanm daarin was-i net}{zoo erg als z'n broer}\\

\haiku{Als ze 't niet om, '!...}{Truus liet en om de herrie}{dan zou zem toch}\\

\haiku{Het bloed schoot haar in,.}{de wangen terwijl ze hem}{van terzij aanzag}\\

\haiku{blaakte fel-wit, met,.}{het dood-strak keukenraam}{en de stille deur}\\

\haiku{Ze is knap, meer niet... ',...}{ent is een gezicht dat}{ook erg gauw verveelt}\\

\haiku{Binnen den rand van,.}{den stoel zag zij hem komen}{langzaam voortstappend}\\

\haiku{wat stak 'r nou voor ' '?}{kwaad in dat zoo'n jongens}{int water ging}\\

\haiku{Zoo haastten ze dan,.}{voort op niets lettend dan op}{hun eigen loopen}\\

\haiku{Tante Griet was al:}{v\'o\'or hen thuisgekomen en}{haar eerste vraag was}\\

\haiku{Je h\'et gelijk, Griet...,{\textquotedblright};}{je het schoon gelijk stemde}{ze benepen toe}\\

\haiku{Maar zij begreep 't, '.}{wel dat de jonge ant}{water zou wezen}\\

\haiku{Laat je los, laat je,... '...!}{dat kind los riep ze maar En}{t dee w\`at zeer no\`u}\\

\haiku{de hand d'r op zou,....{\textquoteright} {\textquoteleft}... '}{willen geven dat-i}{Och gofferdomme}\\

\haiku{Neem u er maar geen...,...}{notitie van heusch dat}{heb ik veel liever}\\

\haiku{{\textquoteright} haastte zich mama,,.}{verschrikt en verontwaardigd}{te protesteeren}\\

\haiku{... wat kan u dat nou,... '}{schelen wat er precies is}{voorgevallen als}\\

\subsection{Uit: In duisternis}

\haiku{En hij trachtte niet,.}{meer te denken met ogen dicht}{weer in te dutten}\\

\haiku{Hij was nu klaar en ':}{bezag zich nog eens scherp in}{t hangspiegeltje}\\

\haiku{Halfbewust onder ',...}{t denken had hij de kast}{horen openkraken}\\

\haiku{Een beetje verdoofd.}{trad hij voort over de straat vol}{glimmende plassen}\\

\haiku{Maar hij begon pas,.}{dit leven was ook voor iets}{beters opgevoed}\\

\haiku{Daar werd eerstens in.}{een sigarenwinkel een}{bediende gevraagd}\\

\haiku{Waar moest hij gaan over,?}{een paar dagen als hij zijn}{kamer ontruimd had}\\

\haiku{Vrijheid was toch wel,...!}{iets heerlijks eigenlijk iets}{absoluut nodigs}\\

\haiku{{\textquoteright} hijgde hij tegen, {\textquoteleft} '.}{haar inu deet expres}{om mijn te pesten}\\

\haiku{sal u de boeken ', '?}{weer int hangertje doen}{of mot ikt doen}\\

\haiku{Hij had zonder smaak,.}{gegeten schoon hij wel een}{kwartje verteerd had}\\

\haiku{Nee, dan toch niet op, '...}{die kamer en niet z\'o als}{hijt wilde doen}\\

\haiku{De jongeman zag.}{een poos door het glas heen en}{kreeg er eetlust van}\\

\haiku{in de groeiende,.}{verlegenheid in benauwd}{lichaams-voelen}\\

\haiku{maar ze hielden zich,...}{liever onnozel die er}{van profiteerden}\\

\haiku{... \'e\'en het er s'n oor,.}{beseerd en \'e\'en een snee over}{se wang geloof ik}\\

\haiku{Hij wilde er niet,.}{heen en wist tegelijk dat}{hij er komen zou}\\

\haiku{Allengs verzonk hij,,.}{al lopende weer in de}{vorige dofheid}\\

\haiku{Hij wachtte, de deur,.}{halfopen met een hand drukkend}{zijn jagende borst}\\

\subsection{Uit: Onpersoonlijke herinneringen}

\haiku{Men presenteerde.}{ons te koop veele vogels uit}{Egijpte en Afrika}\\

\haiku{Maar dan is er 's.}{Zondags een onverwachte}{kink in den kabel}\\

\haiku{Talloze menschen.}{en Corporaties buiten}{de Militaire}\\

\haiku{Bovendien heeft zij,.}{niets wezenlijks te doen zoo}{min als haar ouders}\\

\haiku{Zoo is Louise voor.}{den Vader eigenlijk een}{teleurstelling}\\

\haiku{onverschillige,,.}{verbazing en verachting}{als voor een zonde}\\

\haiku{dit liet gebeuren,,.}{zonder aanmoediging maar}{ook niet afstootend}\\

\haiku{En op dat boord staat,.}{het hoofd het gezicht diep en}{gewichtig peinzend}\\

\haiku{Er viel niets aan Le,.}{Roy meer te helpen zoo het}{al ooit had gekund}\\

\haiku{Rouaan, Chartres,,,.}{Amiens Saint Quentin om er de}{kerken te bezien}\\

\haiku{Iets, dat hem van zich?}{zelf af leidde en zijn}{leven een doel gaf}\\

\haiku{Maar zij geloofde,.}{eigenlijk niet dat hij zoo}{zelfbewust leefde}\\

\haiku{Er was niets meer te.}{verwachten dan een wellicht}{bezwaarlijk einde}\\

\haiku{Zoo wilde zij hem.}{dan nu nog recht doen en zijn}{naam vereeuwigen}\\

\haiku{Dat gebeurde toch,.}{al  vaak als het middel}{te vroeg uitwerkte}\\

\subsection{Uit: Reizen, een uitweiding en inwijding}

\haiku{En dan is er al,,:}{spoedig op eenigen afstand}{een stadje in zicht}\\

\haiku{Een nieuw mensch gaat op}{nieuwe wegen temidden}{van nieuwe dingen}\\

\subsection{Uit: Studies}

\haiku{Rillend stapte hij,.}{voort zoveel mogelijk de}{plassen vermijdend}\\

\haiku{hoe laat 't was, want.}{hij kon zich heel geen denkbeeld}{van de tijd maken}\\

\haiku{Enige ogenblikken, '.}{lag hij zo stil maar spoedig}{kreeg hijt te warm}\\

\haiku{Zij stond snel op en,.}{kwam tot hem die de armen}{naar haar uitstrekte}\\

\haiku{Kalm lag Paul daar nu,,.}{met de grote ogen wijd open}{zonder te ijlen}\\

\haiku{Meneer Dalman zelf,:}{was er maar kort in geweest}{jaren geleden}\\

\haiku{en nu vulde dat.}{zijn hele hoofd en trok zijn}{denken naar binnen}\\

\haiku{Als 't vandaag niet,.}{beter ging moest er morgen}{een dokter komen}\\

\haiku{Toen aten zij in de,.}{kamer die hij vroeger het}{zaaltje had genoemd}\\

\haiku{Mevrouw schreide nog,,.}{altijd wanhopig haar kracht}{uitgeput voelend}\\

\haiku{Dat zeker wel, want,...}{anders zou hij toch niet in}{bed zijn bij daglicht}\\

\haiku{Schuw rondblikkend, zag,...}{hij in de kamer waar de}{schemering waarde}\\

\haiku{om hulp roepen, 't,,...}{bed uitspringen de kamer}{uit naar beneden}\\

\haiku{een aanhoudende.}{wisseling van verwachting}{en teleurstelling}\\

\haiku{Ja, Vermeer had hen,...}{al gezien hij zou straks bij}{hen komen zitten}\\

\haiku{Maar hier, ook hier, drong ',,.}{t hem iets te doen dat hun}{genot zou geven}\\

\haiku{zij gaan, soms luider,,.}{op heftiger alsof het}{paar v\'o\'or hen twistte}\\

\subsection{Uit: Vluchtige verschijningen}

\haiku{ja... dat kwam er van,,}{bijna zes jaren getrouwd}{en vier kinderen}\\

\haiku{As je in de bos,.}{was dan hadt je tenminste}{de dokter voor nies}\\

\haiku{t Was waarlijk een,.}{otium cum dignitate}{dien hij hier genoot}\\

\haiku{een oogenblik zwaar....?}{wordt van doffe droefheid bij}{het herdenken}\\

\haiku{En dit duurde drie,.}{weken zonder dat hij er}{veel verder mee kwam}\\

\haiku{Hel-geelend licht:}{het uit de druischende}{machinekamer}\\

\haiku{Ja, 's winters was '.}{t varen lang geen plezier}{en vaak hard genoeg}\\

\haiku{Het was dadelijk,,:}{als ze een paar uur over hun}{tijd kwamen de vraag}\\

\haiku{zee en bergen en,...}{stad het lag alles roerloos}{en zeer verlaten}\\

\haiku{Hier en daar lag een.}{landhuis rustig temidden}{van nog jong plantsoen}\\

\haiku{Wij kijken uit het,.}{coup\'e-raampje terwijl de}{trein nog roerloos staat}\\

\haiku{daar wonen zullen.}{misschien en al die punten}{van nabij kennen}\\

\haiku{Over het plein is de.}{vochtige frischheid van den}{opengaanden ochtend}\\

\haiku{Maar er gaat lichte.}{onrust onder de groep bij}{den stalingang}\\

\haiku{Men zit er veel te.}{dicht op en ziet altijd de}{menschen in hun rug}\\

\haiku{twee mannen en vier,.}{jongens altemaal goede}{Duitsche gezichten}\\

\haiku{Zij waren voor het.}{meerendeel zeer mistroostig}{en zenuwachtig}\\

\haiku{Het waren deze,.}{die den aanstootelijken naam}{van Smouzen droegen}\\

\subsection{Uit: Zondagsrust}

\haiku{terwijl Verhoef met,...}{de leege theekop wegging en de}{alkoofdeur openbleef}\\

\haiku{Zij gromde terug,....}{dat Verhoef zich t'r niet mee}{hoefde bemoeien}\\

\haiku{10 guldes voor 't,.......}{huishoue in de week dat was}{toch ordentelijk}\\

\haiku{... helderde haar stem,.}{met dat geaffecteerde}{klankje haar eigen}\\

\haiku{We motte spare...,...}{zei-di altijd maar voor}{de kwaje dage}\\

\haiku{Ze most 'n echte,......}{dame zijn en natuurlijk}{altijd handschoene}\\

\haiku{De verpleger en '.}{de meid same haddenm}{nie meer kenne houe}\\

\haiku{Maar affijn, d'r was '...}{toch in alle geval nog}{s variatie}\\

\haiku{Nou, dat kon zij niet,....}{bekostige dan moste}{ze d'r maar vrijhoue}\\

\haiku{Een andere vent ',...}{keek nogs na z'n vrouw om}{hoe ze d'r uitzag}\\

\haiku{- ik... maar hij keerde.}{zich schokschouderend af en}{nam z'n krant weer op}\\

\haiku{Onder het drukke, '....}{smakken vroeg ze Dirk of hij}{r ook nog van wou}\\

\haiku{- Zus... aarzelde de,,......?...}{moeder die haar vergeten}{had ja wil ze wel}\\

\haiku{Haar zelf klonk het vreemd,.}{terwijl Verhoef en het kind}{er van opschrikten}\\

\haiku{Toen zich heelemaal ' '.}{afwendend vant raam om}{naart kind te zien}\\

\haiku{Het kind stond al klaar,.}{bij de deur vraagoogend van}{vader naar moeder}\\

\haiku{- Ze komt zeker 's...}{kijke of d'r wat te snaaie}{valt met de Zondag}\\

\haiku{wat is 't al weer,?}{lang da'k niet bij me lieve}{kindere was niet}\\

\haiku{riep Verhoef haar na,:}{dan uitschaterend in een}{zenuwige lach}\\

\haiku{En dat ze d'r niks - -!...}{van gemerkt hadde geen van}{beie Kato ook niet}\\

\haiku{In elk geval had,.}{hij haar liever zoo dan dat}{ze een kop toonde}\\

\haiku{Toen Marietje de,.}{deur had toegetrokken bleef}{het boven even stil}\\

\haiku{En hij had juist geen,!...}{poot verzet altijd maar op}{een stoel gehange}\\

\haiku{Ja... daar most-i,!}{bij haar maar mee ankomme}{met zukke dinge}\\

\haiku{As zij maar mooie kleere,...}{en opschik koope kon dan ging}{er de rest niet an}\\

\haiku{- Neen 't waren wel {\textquoteleft}{\textquoteright}...,..!}{allemaalvrindjes en wat}{voor godbeware}\\

\haiku{Zij voelden zich heel.}{thuis in hun koffiehuizen}{in hun eigen stad}\\

\haiku{Ja, je weet niet wat...}{je beginne moet om zoo'n}{dag om te krijgen}\\

\haiku{Maar hij meende 't... ...}{zoo goed en hij geloofde}{haar zoo onschuldig}\\

\haiku{t was zoo'n lange,.}{jongen zoo interessant}{bleek en met zwart haar}\\

\haiku{Ben was wel een goeie.......}{jongen zoo hartelijk en}{hij meende het zoo}\\

\haiku{Maar hij kwam altijd, '...}{en als hij er dan was was}{t toch ook weer goed}\\

\haiku{Maar nu werd zij zich.}{haar gedachten bewust en}{haar blijheid was heen}\\

\haiku{Maar hij was niet chic,......}{voor geen cent en hij kon haar}{ook niet veel geven}\\

\subsection{Uit: Een zwakke}

\haiku{Maar soms hadden ze.}{nog niet gekookt en dan moest}{hij t\`och melk nemen}\\

\haiku{Hij stond op, legde.}{een dubbeltje neer v\'o\'or haar}{groette en ging heen}\\

\haiku{Bij het eene raam, in,:}{de voltaire was n\`og eene}{vage gestalte}\\

\haiku{Hij was eerst in de.}{vierde klas burgerschool en}{voor weinig bruikbaar}\\

\haiku{Het was intusschen,;}{stil geworden binnen een}{stilte vol woede}\\

\haiku{Knagend woelde en '.}{wroette in hemt oude}{leegte-gevoel}\\

\haiku{Als hij tenminste,.....}{maar kans had later chef te}{worden dan of neen}\\

\haiku{Johan was even geschrikt '.}{van dat leven buiten hem}{int kamertje}\\

\haiku{Maar terwijl ze de,.}{knop omdraaide sloeg al de}{glazen keukendeur}\\

\haiku{Het duurde lang eer.}{de koppen zichtbaar werden}{boven den traprand}\\

\haiku{Ik vin 't heel lief,...}{van u dat u nog aan me}{heb willen denken}\\

\haiku{- Dag lieve mevrouw, ',.}{nout ga u goed heel}{veel pleizier op reis}\\

\haiku{Een zwakkeling en.....}{een ijdeltuit en voor geen}{cent zakenkennis}\\

\haiku{- Maar ik geloof toch,,:}{niet dat ik ooit zoo kijf zoo}{leelijk schreeuw als jij}\\

\haiku{Hij schudde met een.}{ongeduldigen ruk haar}{arm van zijn schouder}\\

\haiku{Dat kwam zoo te pas ....}{en zou hem zoo'n gevoel van}{meerderheid geven}\\

\haiku{Een breede zee van.}{nutteloosheid lag er nu}{tusschen hem en hen}\\

\haiku{God, we zijn al om '.}{half twaalf begonnen ent}{is nou over half twee}\\

\haiku{dat-i wel wat...}{minder lawaai kon maken}{als-ti uitgaat}\\

\haiku{Mevrouw stond op en.}{begon langs de ramen heen}{en weer te loopen}\\

\haiku{- Ach, dan heb u ze...}{zeker onder een ander}{stapeltje gelegd}\\

\haiku{en dan krijgen wij,....}{nog de schuld dat we niet op}{u gepast hebben}\\

\haiku{Het was goed en groot....}{hier op aarde te wachten}{en stil te zijn}\\

\haiku{Mevrouw voelde zich,.}{ellendig benauwd in de}{kamer in haar stoel}\\

\haiku{Heel langzaam zonk de.}{groote wijzer naar omlaag van}{cijfer tot cijfer}\\

\haiku{Maandag dan, dan kon,....}{Woensdag de boel in de kast}{zijn of Donderdag}\\

\haiku{Als ze dan maar de.}{boel niet te veel vernielden}{in dat mangelhuis}\\

\haiku{En dan was 't er,.}{zeker ook nog duur ook en}{dat kon niet lijen}\\

\haiku{Hij moest toch maar 's,....}{wat tegen haar zeggen een}{praatje beginnen}\\

\haiku{Hij wilde dat voor,,.}{hem geheel om er mee te}{doen wat hem lustte}\\

\haiku{suste mevrouw Dirks,.}{die achter liep en moeite}{had mee te komen}\\

\haiku{Waarom heb u pa,?}{dan nooit gevraagd of \`u de}{boeken mocht houen}\\

\haiku{Een klein, laag geluid,:}{van wielknarpen naderde}{links van den grindweg}\\

\haiku{Overal hetzelfde.}{koude bleeke licht schemerend}{tusschen de dennen}\\

\haiku{Johan had dat standje.}{en dat air van stout kind heel}{aardig gevonden}\\

\haiku{En met de handen.}{vooruit en gretige oogen}{liep hij op haar toe}\\

\haiku{Wat zou ze niet van?}{hem denken en zou ze zich}{niet beleedigd voelen}\\

\haiku{Langs de klammige.}{boomstammen streepte de vocht}{een donkere lijn}\\

\haiku{- Seg 's, wat het je,...,!}{me no\`u uitgevoerd jongie}{dat ga\`at zoo niet hoor}\\

\haiku{ik wou u nog wel ..}{even spreken over wat u me}{gistermiddag zei}\\

\haiku{dat ging zoo gauw en, .......}{was zoo onverwacht dat ik}{geen tijd had enz. enz}\\

\haiku{Na den even-kijk,:}{ter zijde had de ander}{weer voor zich gezien}\\

\haiku{Hij wist niet of de,.}{steken verscherpten maar zij}{vermoeiden hem reeds}\\

\haiku{Hij meende dat de,....}{steken al minder werden}{veel dragelijker}\\

\haiku{ik heb met m'n oogen ',....}{gezien datt er was en}{ik zeg dat je liegt}\\

\haiku{ze bedankte er...}{toch voor van die meid ook nog}{een gunst te vragen}\\

\haiku{Als 't eens een list '?}{was van zoo'n meid om haart}{huis u{\`\i}t te krijgen}\\

\haiku{- Och, Betje, zou je?}{me een pleizier willen doen}{en niet meer zingen}\\

\haiku{Dat kwam niet te pas,... '}{eigenlijk haar moeder maar}{te laten stikken}\\

\section{Ed Coenraads}

\subsection{Uit: Eiland van geluk}

\haiku{Ze zeggen dat de, -.}{maatschappij niet deugt ik heb}{dat nooit gevonden}\\

\haiku{{\textquoteleft}Het is beter niets.}{vooruit te horen en z\`elf}{maar te beleven}\\

\haiku{{\textquoteright} {\textquoteleft}Tschudi, tja... wel 'n.}{vent waar wat bij zat en waar}{je mee kon praten}\\

\haiku{Op Palmzondag lag ';}{ern paar honderd meter}{daarboven nog sneeuw}\\

\haiku{Met besten groet ook, -.}{aan juffrouw Errina en}{de anderen Uwdw}\\

\haiku{Dat had hij toch in.}{dienst als menagemeester}{ook altijd gedaan}\\

\haiku{En als Luzato...{\textquoteright} {\textquoteleft} ',?}{een brief aan des Voeux schreefBen}{jijt Errina}\\

\haiku{Och kom, was  ze,!}{in Lugano in een van}{de grote hotels}\\

\haiku{{\textquoteright} {\textquoteleft}Maar weet je wel dat?}{het anderhalf uur roeien}{is naar de overkant}\\

\haiku{Dan komt ze woedend.}{uit het witte huis en maakt}{een hele sc\`ene}\\

\haiku{Het was alsof de... {\textquoteleft}?}{zonnestralen zongenKom}{je naast mij zitten}\\

\haiku{Dan gaf je twintig, -.}{of dertig rappen wat je}{maar te missen had}\\

\haiku{Maar ik voel mij niet ', -?}{geroepen haar leed metr}{te gaan delen jij}\\

\haiku{{\textquoteright} {\textquoteleft}Jawel,{\textquoteright} pijnlikte, {\textquoteleft}?}{Julesmaar kun je dan niet}{tot morgen wachten}\\

\haiku{- De autobus van.}{elven schokte om de hoek}{bij de spitse rots}\\

\haiku{Doch de ander viel.}{hem bruusk in de rede dat}{dat toch niet aanging}\\

\haiku{Vooral Weckerlin.}{hield van het brede uitzicht}{van dat plateau af}\\

\haiku{Ja, ja, haastte Leo,,.}{zich dan weer daar was veel van}{waar heel veel van waar}\\

\haiku{Want zo als het hier ', '}{vandaag int klein was zo}{ging het int groot}\\

\haiku{Soms vielen ze van,.}{de gladde muur af op je}{hand of in je nek}\\

\haiku{hoe kletterde die '.}{regen weer op het platte}{dak vant atelier}\\

\haiku{Even, heel even haar in,:}{de lachende eerlike}{ogen te kunnen zien}\\

\haiku{Weckerlin zei een,.}{mop tegen Errina waar}{allen om lachten}\\

\haiku{Mi-ma mundus, -...,,.}{vult decipi mi-ma}{kom Free nog \'e\'en keer}\\

\haiku{{\textquoteleft}Ik weet niet of dat,.}{Albrecht's sterke kant wel is}{zulke houtsneetjes}\\

\haiku{Hij sloot de ogen om.}{de gedachten van zo even}{verder te dromen}\\

\haiku{Doch deze laatsten.}{hadden al genoeg van hem}{gezien en gehoord}\\

\haiku{Jaunes-citrons,...{\textquoteright}:}{jaunes et bleu\^atres Giulietta}{schaterde het uit}\\

\haiku{Dat was heel gew\'o\'on,, '.}{doodgewoont ging altijd}{zo in de wereld}\\

\haiku{Achter zich hoorde,;}{zij Jules die nu ook weer}{mee dorst te komen}\\

\haiku{Ik kijk om en toen...{\textquoteright} {\textquoteleft}!}{kwamen ze al op ons af}{En hij d'r van door}\\

\haiku{hoe bleef een raadsel,.}{want Luzato had beloofd}{te zullen zwijgen}\\

\haiku{Hij bukte zich juist,.}{over zijn tomaten de rug}{naar haar toegekeerd}\\

\haiku{Haar denken zocht zijn,:}{spoor terug kwam weer aan de}{pas verlaten plek}\\

\haiku{In de schachten van.}{de werkelikheid daalden}{haar gedachten af}\\

\haiku{Line sloot de ogen,.}{kuste hem en lei dan haar}{hoofd op zijn schouder}\\

\haiku{Het was een brief van,,.}{des Voeux zakelik-koud}{maar uitvoerig scherp}\\

\haiku{{\textquoteright} Albrecht haalde de,.}{schouders op alsof het hem}{onverschillig was}\\

\haiku{Om een eerlike,;}{zaak vooral om iets dat te}{verdedigen is}\\

\haiku{Doch de ontmoeting.}{met Albrecht viel haar lichter}{dan zij had gevreesd}\\

\haiku{Albrecht wendde zich,.}{af van het lichte venster}{keek Weckerlin aan}\\

\haiku{Hij was in een brand,.}{gesneld in de waan een mens}{te kunnen redden}\\

\haiku{er was geen mens, die,,.}{om hulp bad er waren geen}{vlammen geen gevaar}\\

\haiku{En die dreef hem uit,,.}{wierp hem op zij aleer de}{dag was opgegaan}\\

\subsection{Uit: Fakkeldragers}

\haiku{Het satiriese - -;}{gedicht op pag. 129 is meen}{ik van Erich M\"uhsam}\\

\haiku{Twee honderd in de, - '.}{maand waarachtign vijfde}{van zijn salaris}\\

\haiku{{\textquoteright} Ja gil maar toe, de.}{bediening was hier net als}{overal in Indi\"e}\\

\haiku{Dan bracht met traag en;}{stil beweeg de djongos de}{lange luie rietstoel}\\

\haiku{De direkteur van.}{de drukkerij had kortaf}{met neen geantwoord}\\

\haiku{Hij maakt er altijd,.}{een makan besar van en}{wil dan niet naar huis}\\

\haiku{Door geld,  door dat,:}{laatste middel dat stomme}{uiterste middel}\\

\haiku{Daar scharrelde de.}{Europeaan met vrouwen}{van zijn eigen ras}\\

\haiku{Nou, dan heb-ie;}{je aardrijkskunde ook slecht}{onthouwe meneer}\\

\haiku{in de verte het:}{licht van een vuurtoren aan}{de Afrikaanse kust}\\

\haiku{nee meneerr, - scheidt u, -.}{nu uit meneerr ik wil heus}{niet hebben m'neerr}\\

\haiku{Over Barbusse of...}{over de literatuur en}{de oorlog of over}\\

\haiku{En toen, na Durban,:}{en Kaapstad bij het kalmer}{worden van de zee}\\

\haiku{Zonder naar haar te,:}{kijken nam Marti haar toch}{van terzijde op}\\

\haiku{De reis is goddank,{\textquoteright}.}{achter de rug antwoordde}{hij ietwat nuchter}\\

\haiku{Er kwamen veel heel,:}{mensen en daarbij zulke}{interessante}\\

\haiku{mevrouw Schindler,.}{kwam uit kanton Glarus net}{als Heinrich's moeder}\\

\haiku{Hij ging niet meer naar,:}{Indi\"e terug hij wilde}{ook niet hier blijven}\\

\haiku{Waarom, waarom moest.}{het met Moeder altoos z\'o}{gaan en niet anders}\\

\haiku{daarh\'een trekken, waar.}{eindelik kleur kwam breken}{door den grauwen nacht}\\

\haiku{Het hele plan was.}{te zot om er twee woorden}{aan te verspillen}\\

\haiku{Het was al bij half.}{tien en een berg van werk lag}{op hem te wachten}\\

\haiku{Vraag aan \'een van de.}{andere Zwitsers of ze}{meer van hem weten}\\

\haiku{Officieren en.}{hoofdofficieren waren}{er v\'e\'el te weinig}\\

\haiku{Onaangenaam dat:}{Aloys Rantzau steeds weer op dat}{denkbeeld terugkwam}\\

\haiku{En zij zal de kop '.}{opsteken zo als zet}{nog niet heeft gedurfd}\\

\haiku{Dan legde zij haar.}{hand op Marti's arm en vroeg}{hem een sigaret}\\

\haiku{{\textquoteleft}Ik ben wel \'een van,{\textquoteright}.}{de weinigen vervolgde}{hij wat ernstiger}\\

\haiku{of voorzichtig met,.}{halve woorden aanduiden}{bang hem te kwetsen}\\

\haiku{Een adder dook op.}{in het warrig groeiende}{gras van zijn denken}\\

\haiku{Dan zal ik meteen.}{aan Forster zeggen dat u}{morgenochtend komt}\\

\haiku{Of kan hij vannacht,?}{nog op u rekenen als}{hij u nodig heeft}\\

\haiku{Ook die bende had.}{zich nu en dan nog in de}{Astoria gewaagd}\\

\haiku{Den eersten dag had,.}{Marti gemerkt dat die zich}{gepasseerd voelde}\\

\haiku{Thea Schumacher... zou ',:}{zijt begrijpen als hij}{haar straks vertelde}\\

\haiku{Elk woord geleek een,,.}{dorre oude munt glansloos}{en toch van waarde}\\

\haiku{iederen dag werd.}{het een grijs geschubde slang}{voor een dode deur}\\

\haiku{Mijn liefste is blank,.}{en rood hij draagt de banier}{boven tienduizend}\\

\haiku{Hij dacht nog eens na,:}{over haar bedoeling toen hij}{weer haar stem hoorde}\\

\haiku{In het begin was.}{de regering niet eens zo}{erg op mij gebrand}\\

\haiku{Anders Zou ik dat.}{ook allemaal niet tegen}{u gezegd hebben}\\

\haiku{en een gelaten.}{begrip van eigen kunnen}{en eigen onmacht}\\

\haiku{het kanon tegen.}{het nachtelik duister van}{Sinzheim's buitenwijk}\\

\haiku{hoe genoten zij,,...}{hoe groeiden zij in h\'a\'at als}{de tegenstander}\\

\haiku{Zij voelde in zijn:}{vlugge houding en in heel}{zijn jonge wezen}\\

\haiku{Zij kende Marti,.}{weinig hij was nog geen zes}{maanden in Sinzheim}\\

\haiku{O, die dorheid der,!}{praktijk dat harteloze}{praktiese leven}\\

\haiku{De beklemming, de.}{nijpende voetangel van}{het gegeven woord}\\

\haiku{Hij stond op van het,.}{bed dat rust beloofde en}{geen rust wou brengen}\\

\haiku{En toch... leek 't hem?}{z\'o omdat het alweer een}{jaar geleden was}\\

\haiku{Al wie oprecht en,.}{warm voor Sinzheim voelde moest}{daartoe meewerken}\\

\haiku{Hij zou koers houden,,.}{in de richting van Thea van}{Schwarz van Werner Stolz}\\

\haiku{- dat het verstand, het,:}{kille koele verstand straks}{ook zou oordelen}\\

\haiku{En verder, verder,...}{had hij willen verklaren}{moeten verklaren}\\

\haiku{De Meimaand was in.}{zachte zomerse dagen}{langzaam doorgebloeid}\\

\haiku{Maar schrik lag over de.}{felgeslagen daken en}{huizen van Sinzheim}\\

\haiku{{\textquoteleft}Ik kan er niets aan,{\textquoteright},.}{doen klonk het kil toen Marti}{had uitgesproken}\\

\haiku{De telefoon ging,.}{over luidruchtig getril door}{Hellmuth's werkkamer}\\

\haiku{Door het open raam klonk,.}{het kanonnengedonder}{onafgebroken}\\

\haiku{{\textquoteleft}Als ik je raden...,?}{mag maar je houdt niet erg van}{m'n raad geloof ik}\\

\haiku{{\textquoteright} ~ Tegen vier uur.}{begaf Marti zich w\'eer naar}{de universiteit}\\

\haiku{Zij keek hem aan door,,;}{haar wimpers als donkere}{violen fluweel}\\

\haiku{Uit een troepje op:}{de korridor der eerste}{verdieping klonk het}\\

\haiku{Dan klonk plotseling,:}{de hoge krankzinnige}{stem van Meierhofer}\\

\haiku{Zij drongen d\`an om,.}{Minsterberg dan om Schwarz heen}{en eisten hun geld}\\

\haiku{Een paar maal wist Schwarz,.}{het te winnen door vloeken}{snauwen en dreigen}\\

\haiku{ze roesden in de.}{vaalheid der halfduistere}{gangen over en weer}\\

\haiku{E\'en van de vuilste!}{verraders van de hele}{gele regering}\\

\haiku{Toen werd het drukkend.}{stil en geheel donker in}{zijn gevangenis}\\

\haiku{Langzaam at en dronk,;}{hij en genoot bijna van}{de stilte rondom}\\

\haiku{{\textquoteright} hoeveel maal had hij.}{dat de laatste dagen al}{niet moeten horen}\\

\haiku{Haatte hij wellicht,?}{niet genoeg kon hij dan niet}{fel genoeg haten}\\

\haiku{Naar en kleinzielig,.}{zo lang te mijmeren over}{eigen leed en strijd}\\

\haiku{Alles was immers;}{beter dan de val van de}{raden-republiek}\\

\haiku{Graaf Bermondt was aan.}{het hoofd van zijn troepen de}{stad binnen gerukt}\\

\haiku{Verwaarloosd gelijk;}{een huisdier dat men vergeet}{voedsel te geven}\\

\haiku{Eerst dienzelfden avond,,:}{terug in zijn cel flitste}{Marti te binnen}\\

\haiku{Nog steeds lijdend aan,.}{zwaarmoedigheid stemde hem}{dit mistroostig}\\

\haiku{Hij was tegelijk,.}{bruikbaar mens en betrouwbaar}{oprecht kameraad}\\

\haiku{zijn tussenkomst op?}{den middag van den moord in}{de universiteit}\\

\haiku{Toch berouwde hij,.}{niet dat hij zo tegen Schwarz}{was opgetreden}\\

\haiku{Vijf minuten v\'o\'or,:}{Zijn komst krabbelde Marti}{nog vlug onderaan}\\

\haiku{Van dat sterven gaat,.}{onmetelike kracht uit}{een grondeloos licht}\\

\haiku{maar ik zelf weet, dat,.}{ik ten slotte misbaar ben}{zo als iedereen}\\

\haiku{ieder die  hart,.}{heeft en verstand hoopt dat zo}{iets anders afloopt}\\

\haiku{de man die op het.}{laatste ogenblik de grote}{zaak in den steek liet}\\

\haiku{aan de volmaking,.}{de trapsgewijze opgang}{van het  mensdom}\\

\haiku{Hij zag opeens h\'aar,;}{verblijf de cel van Gertrud}{Faucherre voor zich}\\

\haiku{Maar hij geloofde.}{dat de werkelikheid met}{dit droombeeld spotte}\\

\haiku{De beloning die;}{Lilienfeld hiervoor ontving}{was niet zo gering}\\

\haiku{Hij richtte de spits:}{van zijn denken nu weer op}{het punt van uitgang}\\

\haiku{Je bent toch te zwak.}{om op je te nemen wat}{wij op ons nemen}\\

\haiku{Alles hebben de '.}{ploerten geprobeerd omt}{te onderdrukken}\\

\haiku{Als antwoord werd eerst,, -.}{een S. getikt toen een c.}{en dan h meer niet}\\

\haiku{dat ellendige!}{kloppen tegen de buizen}{wilde ophouden}\\

\haiku{en toen Marti niet:}{antwoordde liet hij er met}{een vloek op volgen}\\

\haiku{Ziet ge voor u het}{kleine portret dat stond op}{mijn schrijftafel dien}\\

\haiku{Denn dieses macht das,;}{Sterben fremd und schwer dass es}{nicht unser Tod ist}\\

\haiku{o Herr, gib jedem,,}{seinen Tod das Sterben das}{aus jenem Leben}\\

\haiku{en dat het - altans -.}{in zijn grondslagen niet mocht}{worden aangetast}\\

\haiku{op beleidvolle.}{wijze werd de produktie}{weer ingekrompen}\\

\haiku{welke dan terstond.}{onder anderen naam weer}{werden opgericht}\\

\haiku{Het was reeds gelukt,}{op Meierhofer de hand te}{leggen gevaarlik}\\

\section{Hendrik Conscience}

\subsection{Uit: Volledige werken 1. Batavia}

\haiku{voor een oud zeebonk,,.}{als ik is het niet goed aan}{land vrouw Pietersen}\\

\haiku{Maar, Hopman, geloof,!}{het mij ook slaat een Neerlandsch}{hart in den boezem}\\

\haiku{{\textquoteleft}Wie van u beiden,?}{het eerst den andere zal}{vergeten vraagt gij}\\

\haiku{{\textquoteleft}Walter, ik weet een.}{middel om niet lang van u}{gescheiden te zijn}\\

\haiku{{\textquoteright} {\textquoteleft}Kom, verstoor u niet,{\textquoteright}.}{om zoo weinig antwoordde}{zijne echtgenoote}\\

\haiku{{\textquoteleft}O, moeder, ik bid,,!}{u ik smeek u vertrek met}{mij naar het Oosten}\\

\haiku{Tegen den dikken.}{stam van eenen Billingbing6 lag}{de meid te slapen}\\

\haiku{Vier  jaren staat,,;}{Congo bij de zee bij de}{wijde stomme zee}\\

\haiku{{\textquoteleft}Wees niet droef, goede,{\textquoteright},.}{meester zeide Congo hem}{de hand nemende}\\

\haiku{De neger poogde;}{hem met versnelde stappen}{te doen vooruitgaan}\\

\haiku{Men moet overtuigd zijn,,.}{dat het Walter is om het}{te kunnen gelooven}\\

\haiku{Om uwer waardig te,.}{worden mag ik geenen mijner}{plichten verzuimen}\\

\haiku{Wij spraken bijna.....{\textquoteright} {\textquoteleft},,.....{\textquoteright}}{dagelijks van uNu ik}{ga tot straks Aleidis}\\

\haiku{{\textquoteleft}Rosalia, kom, kom,!}{dat ik mijnen heer vader}{ga verwittigen}\\

\haiku{Misschien is het nog,.}{iets anders dat hem wel te}{moede deed worden}\\

\haiku{nu zal de Hopman,?}{M. Walter ook vriendelijk}{onthalen niet waar}\\

\haiku{ik wil weten hoe.}{gij voert sedert uw vertrek}{uit  Amsterdam}\\

\haiku{Congo kan niets voor,,;}{u niets dan God bidden zooals}{gij hem hebt geleerd}\\

\haiku{maar dit toch zal hij,,.}{doen alle morgens elken}{avond den ganschen dag}\\

\haiku{{\textquoteright} {\textquoteleft}Ach, die brave vrouw,!}{Pietersen ik zal ze dus}{nooit meer wederzien}\\

\haiku{en is Aleidis nog,,}{jong meisje dan belet u}{niets mij te komen}\\

\haiku{maar zooals gij zegt, zij.}{zijn ten onzen gelukke}{zeer slecht gewapend}\\

\haiku{{\textquoteleft}Vertel ons eens in,.}{korte woorden hoe de zaak}{is afgeloopen}\\

\haiku{- Dadelijk maakten;}{wij den overloop klaar en het}{grof geschut vaardig}\\

\haiku{{\textquoteleft}Hopman, die heiden.}{met zijnen tulband moet van}{daar gelicht worden}\\

\haiku{Het is een kerel,,,.}{die het verre zal brengen}{wees daar zeker van}\\

\haiku{{\textquoteright} riep een der Factors,.}{die op dit oogenblik even}{door het venster keek}\\

\haiku{Dan antwoordde hij:}{den heere Stedevoogd in}{de Maleische taal}\\

\haiku{Dan hief hij weder:}{het hoofd op en zeide even}{kalm tot den gezant}\\

\haiku{maar Heemskerk, die als,}{een arend op hem losschiet heeft}{hem welhaast bereikt}\\

\haiku{{\textquoteleft}Mannen, broeders, laat;}{u de borst zwellen door het}{gevoel van den plicht}\\

\haiku{Ziet hier, welken last.}{wij van den heer Stedevoogd}{hebben ontvangen}\\

\haiku{Wel mogen wij over;}{het gebeurde als over eene}{zegepraal roemen}\\

\haiku{De bootsgezellen;}{zouden nog tot den avond de}{wallen bewaken}\\

\haiku{men kan er zich niet,.}{goed van bedienen zelfs niet}{tot verdediging}\\

\haiku{hij zal toonen, dat.....}{het hart hem goed is onder}{zijne zwarte huid}\\

\haiku{maar Van den Broeck stiet.}{hem terug en gebood hem}{te huis te blijven}\\

\haiku{{\textquoteright} riep Van den Broeck met, {\textquoteleft}!}{blijdschap uitdaar geven de}{Engelschen het op}\\

\haiku{{\textquoteleft}Makkers,{\textquoteright} riep Walter, {\textquoteleft},!}{de vijand van voren de}{vijand van achter}\\

\haiku{Wat deze mannen,;}{tot elkander zeiden was}{moielijk te verstaan}\\

\haiku{Soo drinckt dan hier}{Naer u plaisier Een pijp of}{vier Bij wijn of bier!31}\\

\haiku{{\textquoteright} Hier matigde de:}{Sjahbandar zijne stem en sprak}{op stilleren toon}\\

\haiku{nood en begeerten,.}{zijn de prikkels die den mensch}{sterk en groot maken}\\

\haiku{Weerhoud uwe tranen.}{en laat mij Congo eenige}{vragen toesturen}\\

\haiku{{\textquoteright} Dan bemerkte de,.}{Orang-kay's dat de Hopman}{hen had bedrogen}\\

\haiku{{\textquoteright} De Hopman hield den.}{blik ten gronde en schudde}{ontkennend het hoofd}\\

\haiku{Niet waar, Van den Broeck,,?}{gij zult u herinneren}{dat gij vader zijt}\\

\haiku{En zulk leven zou;}{ik pogen te behouden}{door trouweloosheid}\\

\haiku{De Portugees scheen.}{bedroefd en schudde het hoofd}{met medelijden}\\

\haiku{Eindelijk toch bleef.}{hunne slagorde voor des}{vorsten zitplaats staan}\\

\haiku{Elk voorval in hun,!}{lot elke verandering}{kon eene redding zijn}\\

\haiku{Ik ben hier onder.}{mijne landgenooten}{als een vreemdeling}\\

\haiku{hun gelaat werd geel,,;}{hunne lippen loodvervig}{hunne oogen verglaasd}\\

\haiku{ons allen bleef slechts,;}{over als trouwe soldaten}{te gehoorzamen}\\

\haiku{Wist gij nochtans, hoe!}{het gevoel der schaamte mij}{den boezem verscheurt}\\

\haiku{Zijn toestand sloeg de;}{hardvochtigste soldaten}{met medelijden}\\

\haiku{hij zeide weinig;}{en hield meesttijds den blik als}{beschaamd ten gronde}\\

\haiku{Oh,  konden wij,!}{hem levend verlossen wat}{schoone zegepraal}\\

\subsection{Uit: Volledige werken 2. De minnezanger. Een slachtoffer der moederliefde. Eene stem uit het graf}

\haiku{Een oude man met (.).}{zilverwitte haren en}{grijzen baardbladz 17}\\

\haiku{Waren de laatste,,:}{woorden die zij heden tot}{u sprak niet deze}\\

\haiku{hij hoorde zijnen.....}{naam als een noodkreet door veld}{en wouden galmen}\\

\haiku{{\textquoteright} {\textquoteleft}Met blijdschap, met groote,;}{blijdschap indien die zanger}{eene ware kunst toont}\\

\haiku{maar Ser Adelbert, met,:}{den gebiedenden vinger}{tot de dienaars riep}\\

\haiku{Het mildste van al,.}{wat er mild is op aarde}{Is de vrouwelach}\\

\haiku{Het sterkste van al,.}{wat er sterk is op aarde}{Is de vrouwelach}\\

\haiku{Het zoetste van al,.}{wat er zoet is op aarde}{Is de vrouwelach}\\

\haiku{ik wil nevens u,.}{zitten en u zeggen}{wat ik heb gedroomd}\\

\haiku{maar gij kondet niet.}{op Rotsburg blijven wonen}{en moest vertrekken}\\

\haiku{Bij het einde van:}{het ontbijt nam Ser Gunther}{het woord en zeide}\\

\haiku{{\textquoteleft}Mijn vader is wel...........}{een vrij man maar hij wint zijn}{brood met koophandel}\\

\haiku{Des avonds zal er een.}{vroolijk feestmaal op Rotsburg}{gehouden worden}\\

\haiku{het overige van.}{den tijd brengen wij door in}{vriendelijken kout}\\

\haiku{Vertrouwen wij op,,!}{Gods goedheid en gaan wij Hem}{dankend ter ruste}\\

\haiku{De ridders deden;}{de andere paarden meer}{naar de poort leiden}\\

\haiku{- Het wangedrocht viel.}{ter zijde en stierf in eene}{laatste stuiptrekking}\\

\haiku{Met overdreven wil,:}{met wanhoop worstelde ik}{tegen mij zelve}\\

\haiku{Gij zegt, dat gij ziek,?}{zult worden indien meester}{Wilfried ons verlaat}\\

\haiku{mijne dochter zal;}{hare zinnelooze zwakheid}{blijven beweenen}\\

\haiku{{\textquoteleft}O, mijn goede heer,!}{van Haviksberg wat ben ik}{blijde u te zien}\\

\haiku{want het rustte met!}{het aangezicht ter aarde}{en roerde zich niet}\\

\haiku{{\textquoteleft}Heer ridder,{\textquoteright} zeide, {\textquoteleft}?}{de kluizenaarwaarom zijt}{gij dus hopeloos}\\

\haiku{{\textquoteright} {\textquoteleft}Kom, Ser Wilfried, kom,}{telkens als de mismoed of}{de angst u aangrijpt}\\

\haiku{Wat hij in de kluis,;}{te vreezen had dit stond hem}{niet klaar voor den geest}\\

\haiku{de bleekheid op zijn.}{gelaat werd door den gloed der}{woede vervangen}\\

\haiku{Zij gedragen zich,.}{ten minste alsof zij het}{nooit hadden bemerkt}\\

\haiku{{\textquoteright} verklaren, en mijn.....}{arme zoon zou alle hoop}{moeten verzaken}\\

\haiku{Mijn vader trad met:}{M. Steurs in de kamer en}{zeide hem haastig}\\

\haiku{Dit voorval liet op.}{het gemoed mijner vrouw een}{diepen indruk na}\\

\haiku{Tegen den morgen,,.}{in de eerste dagklaarheid}{wekte mij Maria}\\

\haiku{Hij had veel praktijk;}{en was in aanzien als een}{schrander geneesheer}\\

\haiku{Het was het eenige,;}{middel dat nog kansen op}{genezing aanbood}\\

\haiku{{\textquoteright} {\textquoteleft}Kunt gij het kind niet?}{eenige oogenblikken van}{ons verwijderen}\\

\haiku{{\textquoteright} De grijsaard stond op.}{en ging in de weide bij}{het kleine meisje}\\

\haiku{Eindelijk werd ik,}{weder bedaard kuste nog}{herhaalde malen}\\

\haiku{Ik wensch en wil, dat.}{gij haar verpleegt en liefhebt}{als mijn eigen kind}\\

\haiku{Ja, daar zag ik zijn,}{rijtuig en met haast keerde}{ik naar het landgoed}\\

\haiku{{\textquoteleft}hier is mijnheer de,.}{dokter die u gaarne een}{handje zou geven}\\

\haiku{Hij bepeinsde zich:}{eene wijl en zeide dan met}{geestdrift in de oogen}\\

\haiku{In den hopeloozen.....}{toestand uwer echtgenoote is}{daar niets aan gewaagd}\\

\haiku{Geve God, dat gij!}{deze vermetelheid niet}{te betreuren hebt}\\

\haiku{Ik greep haar bij de,:}{hand trok haar in de kamer}{en riep met geestdrift}\\

\haiku{maar, o hemel, zij!}{moet de huwelijksakte}{onderteekenen}\\

\haiku{luister met kalmte.}{en wijsheid op hetgeen ik}{u ga openbaren}\\

\haiku{elke minuut is!}{eene eeuw van smart voor mijne}{arme verloofde}\\

\haiku{De meesten zouden;}{schrijven en hielden daartoe}{de pen in de hand}\\

\haiku{- Hebt gij bij de brug.....}{onderwege Meerhout eene}{samenkomst gehad}\\

\haiku{Ik ben gelukkig,,}{Klaas dat men uwe zaak bij mij}{erger gemaakt heeft}\\

\haiku{Ha, daar komt zij uit,!}{eene achterdeur met eene flesch}{wijn in de hand}\\

\haiku{Ofschoon gansch niet op,;}{zijn gemak stapte hij met}{koortsige haast voort}\\

\haiku{Hij stond op en liep.}{in \'e\'enen adem tot achter}{des schoolmeesters huis}\\

\haiku{dan weder stond des:}{brouwers lijk voor zijn bed en}{riep hem smeekend toe}\\

\haiku{Het geheele dorp,.}{stond er van overhoop zeide}{de onderwijzer}\\

\subsection{Uit: Volledige werken 3. Everard t'Serclaes}

\haiku{{\textquoteleft}Zie, zie, ginder bij,.}{Sint-Michielsheuvel komt}{Jan de goudslager}\\

\haiku{Dilbeek en al de!}{omliggende hofsteden}{staan in vollen brand}\\

\haiku{Ons gezantschap heeft;}{de heer graaf inderdaad met}{toorn en spot onthaald}\\

\haiku{De bestorming, de?}{plundering van Brussel een}{ingebeeld gevaar}\\

\haiku{- Eens in vrijheid, heb,;}{ik mij gespoed over Vorst de}{stad te bereiken}\\

\haiku{- een gulden leeuw op, -.}{zwart veld door den lieer van}{Assche gedragen}\\

\haiku{Het volk zweeg dus in;}{het openbaar en scheen lijdzaam}{zijn lot te dragen}\\

\haiku{het is tijd om den.}{dreigenden opstand zelfs in}{bloed te versmachten}\\

\haiku{Ik bezit nog meer.}{dan de helft van wat gij mij}{laatst hebt gegeven}\\

\haiku{Hij zweeg om zijnen.}{vader den tijd te laten}{den brief te lezen}\\

\haiku{{\textquoteleft}Welke innige,,!}{zuivere liefde zij u}{toedraagt Everard}\\

\haiku{Blijf voorzichtig met,,,.}{hem word voorzichtiger nog}{ik bid u mijn zoon}\\

\haiku{{\textquoteright} Everard greep de:}{handen des hopmans en riep}{opgetogen uit}\\

\haiku{{\textquoteright} {\textquoteleft}Gij bedriegt u niet, -,.....}{Goffredo en hebt gij mij}{niets meer te zeggen}\\

\haiku{{\textquoteleft}Ik zal mij houden,.}{alsof het toeval mij naar}{het bosch had geleid}\\

\haiku{{\textquoteright} {\textquoteleft}Ja, heer Willem,{\textquoteright} was, {\textquoteleft}.....{\textquoteright} {\textquoteleft}?}{het antwoordhet is zulk zacht}{wederWaar zijn ze}\\

\haiku{Hij gelijkt naar den}{jongen heer die vroeger zoo}{dikwijls aan onzen}\\

\haiku{het is, als zag ik.}{ze reeds in mijne volle}{handen glinsteren}\\

\haiku{Hoe de zucht tot geld!}{den menschelijken geest toch}{machtig kan maken}\\

\haiku{{\textquoteleft}Er zijn werklieden -;}{in mijns vaders slaapzaal ik}{had het vergeten}\\

\haiku{morgen toch, zoohaast mijn,.}{vader beneden is zal}{ik het schrijn openen}\\

\haiku{Wat kost het u, den?}{jongen t'Serclaes op uw}{feest te verzoeken}\\

\haiku{Hier kondigde een.}{hofmeester met luider stem}{hunne namen af}\\

\haiku{Deze gedachten.}{deden velen in hunnen}{afkeer wankelen}\\

\haiku{Hare schijnbare.}{koelheid had hem troost in den}{boezem gegoten}\\

\haiku{Zijn vader, die hein,:}{voorbijging sloeg hem op den}{schouder en vroeg hem}\\

\haiku{Gij moet veinzen en,.}{u houden alsof wij u}{niets hadden gezegd}\\

\haiku{{\textquoteright} {\textquoteleft}Ik ben Everard,{\textquoteright}.}{t'Serclaes antwoordde de}{jongeling spijtig}\\

\haiku{zeg den hopman, dat.}{ik voor den middag hem een}{bezoek zal brengen}\\

\haiku{Goffredo heeft wel.}{werkelijk naar Everard}{t'Serclaes gezocht}\\

\haiku{Die verpander was,?}{niemand anders dan hopman}{Goffredo niet waar}\\

\haiku{{\textquoteright} {\textquoteleft}Ik wensch te spreken,,{\textquoteright}.}{heer voorzitter zeide een}{zwaarlijvig brouwer}\\

\haiku{want hij opende de,:}{deur der zaal terwijl hij met}{luider stemme riep}\\

\haiku{{\textquoteright} {\textquoteleft}Wij zijn vijanden,,{\textquoteright}.}{sedert lang ik weet het wel}{zeide de Amman}\\

\haiku{{\textquoteleft}Het zou dus iemand,?}{uwer uitgenoodigden zijn die}{het juweel ontstal}\\

\haiku{Ha, ha, ziedaar dus,}{het geheim dat gij mij te}{veropenbaren}\\

\haiku{want nu zeide hij,:}{op smartelijken toon ja}{met nederigheid}\\

\haiku{Was de Amman geen,?}{boos en gewetenloos mensch}{tot alles bekwaam}\\

\haiku{Zwijgen, zwijgen moet,!}{gij of mijne gramschap rust}{op u voor eeuwig}\\

\haiku{Wat liefderijke!}{moeite heeft hij aangewend}{om mij te troosten}\\

\haiku{Is het de vrees van,?}{des Ammans boosheid die u}{zoo wreed doet lijden}\\

\haiku{Dezen avond was het;}{weder vergadering der}{eedgenooten}\\

\haiku{{\textquoteright} Tranen borsten hem.}{over de wangen en het hoofd}{viel hem op de borst}\\

\haiku{{\textquoteleft}Ja, ja, het is de,{\textquoteright}, {\textquoteleft}.}{heer t'Serclaes zeide de}{knechtik ken hem wel}\\

\haiku{Zoohaast ik maar weg kan,,.}{loop ik en ga hem melden}{wat er is geschied}\\

\haiku{{\textquoteright} Inderdaad, tot groote}{verbaasdheid der dienstboden}{volgde Everard}\\

\haiku{{\textquoteright} vroeg hij, terwijl hij.}{verrast luisterde en van}{zijnen stoel opstond}\\

\haiku{Uw vader, heer, is.}{mij sedert vele jaren}{een vurig vijand}\\

\haiku{Toen hij in de zaal,:}{trad kwam zijn vader hem te}{gemoet en zeide}\\

\haiku{{\textquoteright} {\textquoteleft}Ba, tot zooverre toch!}{zijn wij de slaven dezer}{grove lieden niet}\\

\haiku{{\textquoteright} {\textquoteleft}Dan gedragen wij.}{ons volgens de bevelen}{van meester Lankhals}\\

\haiku{{\textquoteright} Nog eene wijl zette:}{hij zijne overdenkingen}{voort en zeide dan}\\

\haiku{{\textquoteright} En hij sprong op met;}{gloeiende oogen en meende}{zijn zwaard te grijpen}\\

\haiku{{\textquoteright} Maar de overste der:}{wacht legde hem de hand op}{den mond en zeide}\\

\haiku{Maar t'Serclaes, voor,.}{zulken troost gansch gevoelloos}{antwoordde hem niet}\\

\subsection{Uit: Volledige werken 4. Het wassen beeld}

\haiku{maar de baljuw heeft,;}{Rosa zoo lief dat hij haar}{niets kan weigeren}\\

\haiku{zij slaapt zoo rustig,,!}{zij bloost zij heeft gelachen}{in eenen zoeten droom}\\

\haiku{Hoe verbleekten en,!}{beefden zij toen zij hunne}{woning ontwaarden}\\

\haiku{{\textquoteleft}Eilaas, Mevrouw, ik,{\textquoteright};}{ben half dood van schrik was het}{hopelooze antwoord}\\

\haiku{Zuchtend en kermend;}{trad de baljuw met zijnen}{schoonzoon in de zaal}\\

\haiku{daar lieten zij zich.}{op eene zitbank vallen en}{weenden in stilte}\\

\haiku{{\textquoteleft}Luitenant, gij hebt,,?}{gehoord waarover deze heer}{komt klagen niet waar}\\

\haiku{Ja, ja, een kind van,.....}{omtrent twee jaar een meisje}{in het wit gekleed}\\

\haiku{De roofzuchtige,.}{lieden gaan en komen maar}{keeren telkens terug}\\

\haiku{{\textquoteright} klaagde hij, terwijl,,.}{hij de vuisten krampachtig}{wringende bleef staan}\\

\haiku{Daarop vertelde.}{hij zijn wedervaren met}{volle oprechtheid}\\

\haiku{Wel had Frederic.}{zijne opzoekingen niet}{beslissend gestaakt}\\

\haiku{zie eens ginder, het,!}{kleine meisje dat zoo licht}{op de koorde danst}\\

\haiku{Indien zij eens in?}{handen van zulke lieden}{gevallen ware}\\

\haiku{{\textquoteleft}Maar gij moet u sterk,.}{houden en u niet laten}{ontroeren Dina}\\

\haiku{Gelukkig was de.}{dorpsdokter te huis en kwam}{hij onmiddellijk}\\

\haiku{De zinnelooze scheen;}{eensklaps al hare krachten}{terug te vinden}\\

\haiku{Sluit nu de deur der,.}{zaal achter mij en doe zooals}{ik u heb gezegd}\\

\haiku{Daarbij voegden zich.}{de gevolgen van slechten}{oogst en duren tijd}\\

\haiku{maar hij gelukte.}{daarin niet beter dan met}{Sebastiaan Groof}\\

\haiku{Onze gemeente!}{schijnt in een hol van dieven}{en roovers veranderd}\\

\haiku{Het belang, dat het?}{zichtbaar bedrukte meisje}{hem inboezemde}\\

\haiku{Antwoord slechts als men,{\textquoteright}.}{u iets vraagt viel de baljuw}{hem in de rede}\\

\haiku{{\textquoteright} {\textquoteleft}Zeker, Mijnheer, ik,.}{heb er van hooren spreken}{gelijk iedereen}\\

\haiku{Zoo niet, bevindt men,!}{u schuldig de galg zonder}{hoop op genade}\\

\haiku{{\textquoteleft}Preter, blijft gij met;}{drie uwer gezellen deze}{lieden bewaken}\\

\haiku{Heila, Heila, zoudt?}{gij mij kunnen verlaten}{in mijne droefheid}\\

\haiku{{\textquoteright} {\textquoteleft}Hoe, Samson, gij wilt?}{u zelven beschuldigen}{om ons te redden}\\

\haiku{Naar gewoonte zal.}{het ongetwijfeld op eene}{dwaasheid uitloopen}\\

\haiku{{\textquoteright} {\textquoteleft}Hawida Kavandar,?}{gij insgelijks blijft bij uwe}{eerste verklaring}\\

\haiku{Heila hield eene hand.}{haars broeders en besprengde}{deze met tranen}\\

\haiku{Krachtens een bevel.}{van den baljuw hebt gij mij}{aanstonds te volgen}\\

\haiku{, bleef ze eene lange:}{wijl met aandacht bekijken}{en mompelde dan}\\

\haiku{Wie heeft er niet ten,?}{minste \'e\'ens bemind in}{zijn leven niet waar}\\

\haiku{ik zwijg als het graf,{\textquoteright},.}{gromde de waarzegster zich}{naar de deur keerende}\\

\haiku{Dan geef ik bevel,;}{u naar de gevangenis}{terug te leiden}\\

\haiku{Rosa honderdmaal.}{gezien en ze dikwijls in}{de armen gedrukt}\\

\haiku{Ja, ik ben bereid;}{de kleine Rosa in uwe}{armen te brengen}\\

\haiku{{\textquoteright} {\textquoteleft}Weiger, heer, het staat,.}{u vrij maar uw kind blijft voor}{altijd verloren}\\

\haiku{De kaartenkijkster.}{zou dezen nacht in zijne}{woning overbrengen}\\

\haiku{O, mocht het waarheid,!}{zijn dat gij mijn arm kind mij}{gaat terugschenken}\\

\haiku{De lieden, die u,.}{hebben opgevoed zult gij}{niet meer terugzien}\\

\haiku{De meid had haar niet;}{alleen gewasschen en met}{liefde opgeschikt}\\

\haiku{{\textquoteright} {\textquoteleft}Katrien, indien de!}{goede God eens ons geluk}{wilde volbrengen}\\

\haiku{Het kon zes uren des,.}{morgens zijn en het was reeds}{lang klaar dag toen Mev}\\

\haiku{Misschien had zij het,,;}{bewusteloos zelf eenen zoen}{teruggegeven}\\

\subsection{Uit: Volledige werken 5. Bella Stock}

\haiku{{\textquoteleft}ik hoor nog zijne,.}{zware stem die zingt op de}{maat van het paardje}\\

\haiku{en dus droomende,,}{ben ik gaan overwegen dat}{gij ongelukkig}\\

\haiku{Er lag wel iets van;}{de logge vormen des beers}{in zijne leden}\\

\haiku{Djosep zal dan maar.....}{in zee met nuchteren mond}{en ledige maag}\\

\haiku{Maar ik ben vijf en.}{twintig jaar te vroeg op de}{wereld gekomen}\\

\haiku{Zes schellingen in,!}{dezen ongelukkigen}{tijd het is een schat}\\

\haiku{maar die lomperd lacht.}{mij uit en drijft zijnen ezel}{met meer geweld voort}\\

\haiku{{\textquoteleft}Wel, Ko, man lief, gij!}{zoudt een Christenmensch den dood}{op het lijf jagen}\\

\haiku{Ko volgde haar met:}{eene zure uitdrukking van}{spijt op het gelaat}\\

\haiku{{\textquoteright} riep de maagd verstoord, {\textquoteleft},?}{gij durft droomen dat ik uwe}{vrouw geworden ben}\\

\haiku{maar, God zij er om,,.}{geloofd droomen is bedrog}{zooals het spreekwoord zegt}\\

\haiku{- Eensklaps liet zij haar:}{voorschoot vallen en zeide}{tot den strandlooper}\\

\haiku{Niet waar, vader lief,,!}{wat God kan behagen is}{altijd wel gedaan}\\

\haiku{En indien ik zulks,,?}{geloof waarom zoudt gij dan}{wanhopen mijn kind}\\

\haiku{{\textquoteright} murmelde Bella,.}{met den helderen lach der}{blijdschap in de oogen}\\

\haiku{{\textquoteright} De vrouwen zagen,.}{op naar het kind dat met de}{hand naar de zee wees}\\

\haiku{noch de oude Stock,.}{noch zijne dochter wilden}{van trouwen hooren}\\

\haiku{Het is wel schoon van,.}{Bella dat zij  haren}{vader dus bemint}\\

\haiku{{\textquoteright} Allen klommen op,.}{tot het kleine kamertje}{achter M. Darings}\\

\haiku{{\textquoteright} zuchtte het meisje, {\textquoteleft}!}{met eenen stralenden glimlach}{wat ben ik blijde}\\

\haiku{voor zijnen armen,,!}{vader voor zijne moeder}{voor zijne zuster}\\

\haiku{{\textquoteleft}Ach, ja, ik bid u,!}{hij heeft de koorts nog gehad}{dezen namiddag}\\

\haiku{M. Darings deed eenen.}{stap naar de deur der hut en}{trok ze met zorg toe}\\

\haiku{Beklagen wij niet,.....{\textquoteright} {\textquoteleft}?}{de slachtoffers maar wel de}{moordenaarsMaar hij}\\

\haiku{Ik gevoel mij den,{\textquoteright}}{moed niet tot het vervullen}{dier wreede boodschap}\\

\haiku{Ik zal morgen, op,.}{den noen komen zien wat er}{kan worden gedaan}\\

\haiku{{\textquoteleft}Kom, mijn kind, laat ons,{\textquoteright};}{binnengaan en ontsteek de}{lampe zeide hij}\\

\haiku{{\textquoteleft}Laat mij nog wat met,,{\textquoteright}.}{u blijven waken Bella}{zeide de blinde}\\

\haiku{Tante Claar zal eerst,.}{te middernacht komen om}{u af te lossen}\\

\haiku{gij schat het veel te,,.}{hoog het weinige dat wij}{voor u kunnen doen}\\

\haiku{Indien ik tante,,}{Claar niet had ik zou nog niet}{eens weten voor wien}\\

\haiku{D\'a\'ar, tusschen die twee,,.}{duinen de donkergroene}{vlek dat is de zee}\\

\haiku{{\textquoteleft}Ja, ja, mijnheer,{\textquoteright} riep, {\textquoteleft}:}{Djosep half verlegenik}{durf het wel zeggen}\\

\haiku{Zijn broeder Louis ging,.}{te Duinkerken scheep op de}{labberdaanvangst}\\

\haiku{M. de Milval, gansch,.}{opgekleed daalde van de}{trap in de kamer}\\

\haiku{de wereld heeft dan.}{tusschen arme visschers zijn}{leven te slijten}\\

\haiku{Uwe afwezigheid,,;}{mijnheer zal mij voor langen}{tijd treurig maken}\\

\haiku{Zij scheen zeer bedrukt.}{en liet het hoofd mismoedig}{op de borst hangen}\\

\haiku{Gij  zoudt mij eene!}{geraaktheid doen krijgen met}{dat eeuwig talmen}\\

\haiku{hij eenen bos zeedoorn:}{aan en rukte zijne taaie}{stammen uit den grond}\\

\haiku{en d\'a\'ar eenen zijner,:}{bekenden op den schouder}{kloppende vroeg hij}\\

\haiku{{\textquoteleft}Jan, vriend, wat is er,?}{voorgevallen dat men hier}{zoo te hoopen staat}\\

\haiku{Zoudt gij bij geval?}{aan de domme uitvindsels}{der priesters gelooven}\\

\haiku{{\textquoteright} {\textquoteleft}Daarin bedriegt gij,,{\textquoteright}.}{u voorwaar viel zijn gezel}{hem in de rede}\\

\haiku{{\textquoteright} zeide de oude,.}{krijgsman op treurigen doch}{vriendelijken toon}\\

\haiku{maar ik weet nog zoo,;}{goed wat moeite gij deedt om}{mij te leeren lezen}\\

\haiku{U te zeggen, wat,.}{ik dien nacht heb geleden}{is onmogelijk}\\

\haiku{ik ben het niet, die.}{lust heb om met uwen barschen}{kozijn te vechten}\\

\haiku{{\textquoteright} En, zich tot zijnen,:}{broeder en zijne nichte}{wendende sprak hij}\\

\haiku{Het scheen hem, dat hij.}{het gerucht van stappen voor}{de deur ontwaarde}\\

\haiku{ik had mij belast.}{met onze beste dekens}{en met een kussen}\\

\haiku{Hij weigerde op,.}{een zachter bed te rusten}{uit edelmoed alleen}\\

\haiku{hij zal vertrekken,,;}{v\'o\'or den morgenstond verre}{verre zal hij gaan}\\

\haiku{Ik heb den Heer voor!}{uwe blijde terugkomst zoo}{vurig gezegend}\\

\haiku{Hij zou dus weten,.}{dat de \'emigr\'e langs de zee}{wilde ontvluchten}\\

\haiku{Onderwerp u dus,,.}{zonder tegenstand of zeg}{dat gij liever sterft}\\

\haiku{Bella schoof uit en;}{sloeg geweldig tegen den}{rand van het vaartuig}\\

\haiku{Ik bewonder u,,!}{ik eerbiedig u als waart}{gij eene heilige}\\

\haiku{{\textquoteleft}Bekommer u niet,.}{om de wonde die gij mij}{hebt  toegebracht}\\

\haiku{Ach, ik schreef het met,.}{zinnelooze haast onder de}{oogen mijner wachten}\\

\haiku{Gij hebt waarschijnlijk?}{nooit van liefde met mijne}{nichte gesproken}\\

\haiku{als hij eene reden,.}{tot groot verdriet heeft dan is}{alles hem pijnlijk}\\

\haiku{Het was zichtbaar, dat;}{een gedachtenvloed hem door}{de hersens stroomde}\\

\haiku{- want in der waarheid,.}{gij hebt geene reden om het}{ergste te gelooven}\\

\haiku{{\textquoteright} {\textquoteleft}Eilaas, heb ik u?}{dan reden gegeven om}{mij te verstooten}\\

\haiku{God zou mijne ziel.}{over zulke laffe baatzucht}{rekening vragen}\\

\haiku{{\textquoteright} Zij onderbraken.}{hunnen edelmoedigen twist}{en traden in huis}\\

\haiku{Kozijn Djosep heeft;}{mij vele goede dingen}{gezegd onderweg}\\

\haiku{{\textquoteright} zuchtte Djosep, zich.}{tranen van ontroering uit}{de oogen vegende}\\

\haiku{Nu, Bella, gij ziet,.}{wel dat gij geene reden hebt}{om dus te schrikken}\\

\haiku{{\textquoteleft}Ah, woorden, die M.!}{de Milval met eigene}{hand heeft geschreven}\\

\haiku{{\textquoteright} Louis Stock opende den,:}{brief hield eene wijl de oogen er}{op gericht en vroeg}\\

\haiku{en nochtans indien,,}{zijn broeder ja indien zijn}{eigen zoon tusschen}\\

\haiku{geef mij het middel,.}{om de plaats te herkennen}{waar hij zich bevindt}\\

\haiku{{\textquoteleft}Lieve nichte, gij?}{beschuldigt mij wellicht van}{gevoelloosheid}\\

\haiku{{\textquoteleft}Uit medelijden,.}{met mijne nichte verkort}{het treurig afscheid}\\

\haiku{Deze bloedstorting,;}{is nutteloos voor de zaak}{die gij verdedigt}\\

\haiku{Ginder, ginder, niet,!}{verre van het fort tusschen}{die hoopen aarde}\\

\haiku{want een hagel van.}{kogels en ballen kliefde}{de lucht rondom hem}\\

\haiku{en het huisje, dat,.}{zij heeft bewoond staat op drie}{boogschoten van hier}\\

\subsection{Uit: Volledige werken 6. Simon Turchi}

\haiku{evenwel, het was nog,:}{met zoete gelatenheid}{dat zij antwoordde}\\

\haiku{{\textquoteright} Een knecht opende de,,,:}{deur en terwijl hij iemand}{inleidde riep hij}\\

\haiku{{\textquoteright} {\textquoteleft}Maar met uw oorlof,,{\textquoteright}, {\textquoteleft}}{signore riep de oude}{ridder half vergramd}\\

\haiku{{\textquoteright} {\textquoteleft}Inderdaad, gij hebt,{\textquoteright},.}{wel zeer wel gedaan sprak de}{heer Van de Werve}\\

\haiku{hij zeide mij, dat!}{hij het noodige geld nog niet}{heeft kunnen vinden}\\

\haiku{zit daar neder op;}{die banke en laat mij aan}{uwe zijde zitten}\\

\haiku{Neen, daarom ben ik.}{op mijnen ouden dag niet}{over zee gekomen}\\

\haiku{Hem eensklaps op den,:}{schouder slaande riep Julio}{met eenen schaterlach}\\

\haiku{wat ik maak, moet slechts,.}{dienen om eene luim onzes}{meesters te voldoen}\\

\haiku{indien het spreken,.}{kon het zou u wondere}{dingen vertellen}\\

\haiku{gij zult dezen avond.....}{Geronimo afwachten}{en hem doorsteken}\\

\haiku{Volvoer uwen last met,.....}{schranderheid en ik  zal}{u goed beloonen}\\

\haiku{Gij zult een weinig}{v\'o\'or tien uren u ten huize}{van Geronimo}\\

\haiku{{\textquoteright} {\textquoteleft}Het is drie dagen,.}{geleden dat gij ze nog}{in de handen naamt}\\

\haiku{{\textquoteright} {\textquoteleft}Dit is volgens den.}{stand der personen en het}{gewicht der zaken}\\

\haiku{Er waren daar ook.}{eenige Carolusguldens}{te winnen nochtans}\\

\haiku{{\textquoteright} {\textquoteleft}Hij draagt gansch bruine;}{kleederen en eene witte}{veder op den hoed}\\

\haiku{Dat deze laatste,;}{nog had kunnen vluchten was}{hem onverstaanbaar}\\

\haiku{{\textquoteleft}De serenata;}{kan niet gegeven worden}{zonder luitspelers}\\

\haiku{De jonge Maria.}{was nu waarlijk schoon boven}{alle verbeelding}\\

\haiku{Tot Simon Turchi:}{sprak zij in het voorbijgaan}{met blijden glimlach}\\

\haiku{Wat ik doe, is slechts.}{om u uit een dreigenden}{toestand te redden}\\

\haiku{En zal de vreemde?}{koopman mij de somme ter}{hand stellen in geld}\\

\haiku{Er kon niets beters,.....}{uitgedacht worden om zijn}{leven te sparen}\\

\haiku{{\textquoteright} Eene onwillige.}{beweging van spijtigheid}{ontsnapte den knecht}\\

\haiku{{\textquoteright} {\textquoteleft}Zult gij nog hier zijn,,?}{signore als ik uit den}{kelder terugkeer}\\

\haiku{Hij nam de lamp en,.}{verliet den kelder zonder}{de deur te sluiten}\\

\haiku{Ik zal de deure.....}{sluiten en morgen mijn werk}{komen voltooien}\\

\haiku{Hij wist wel, dat zulks;}{niet kon geschieden dan na}{verloop van veel tijds}\\

\haiku{{\textquoteright} {\textquoteleft}Maar kon hij dan zoo?}{lichtelijk de hand mijner}{dochter verzaken}\\

\haiku{Geronimo nog,}{levend vinden hoe zouden}{wij al te zamen}\\

\haiku{ik zal zeer laat in.}{den avond Bernardo zenden}{om u te helpen}\\

\haiku{{\textquoteright} {\textquoteleft}Neen, ik zal hem op,.}{zijn leven gebieden u}{te gehoorzamen}\\

\haiku{Julio zette zich;}{op eenen stoel en legde het}{hoofd in de handen}\\

\haiku{Vergeet van nu af;}{aan uwe andere namen}{en wees voorzichtig}\\

\haiku{dan weder liet hij.....}{ze van de eene hand in de}{andere glijden}\\

\haiku{ik zal mij kleeden,,;}{als een edelman in satijn}{fluweel en zijde}\\

\haiku{een zenuwschok trof,.}{al zijne leden en hij}{verbleekte van schrik}\\

\haiku{Gij ten minste zult;}{daarboven de belooning}{der onschuld vinden}\\

\haiku{Morgen zal de Wet.....}{den speelhof en ook dezen}{kelder doorzoeken}\\

\haiku{moest zulk ijselijk?}{ongeluk uwe dagen op}{aarde verkorten}\\

\haiku{Ik gevoel mij de,.....}{kracht niet meer om de wreede}{taak te volvoeren}\\

\haiku{Ik zal uitgaan, om.}{te zien of ik ergens nog}{eenen winkel open vind}\\

\haiku{{\textquoteright} De schout keerde zich.}{om en stapte ter zijde}{van Simon Turchi}\\

\haiku{om mijnen besten.....?}{vriend te te mishandelen}{of te vermoorden}\\

\haiku{Wat mij betreft, ik,.}{bid u als vriend duid mij de}{zaak niet ten kwade}\\

\haiku{Merkbaar was het, dat;}{hij zijnen knecht niet zonder}{inzicht bespiedde}\\

\haiku{{\textquoteleft}Het is zonde, dat.}{zulke wijn nutteloos ten}{gronde moet vlieten}\\

\haiku{koop daar een goed paard;}{en vertrek in aller haast}{over Aerschot en Diest}\\

\haiku{akeligen schreeuw, als.}{hadde een geheime slag}{hem het hart doorboord}\\

\haiku{Ach, barmhartige,!}{God laat mij toch genade}{vinden voor uw oog}\\

\haiku{ik voelde een kort;}{oogenblik het bloed als eenen}{stroom mij ontvlieten}\\

\haiku{{\textquoteleft}Reeds acht dagen ligt,.}{het schip gereed dat hen naar}{Itali\"e moet voeren}\\

\haiku{{\textquoteright} Geronimo's hand;}{beefde in de hand zijner}{schoone echtgenoote}\\

\subsection{Uit: Volledige werken 7. De oom van Felix Roobeek}

\haiku{hij ware in staat.}{om u onaangename}{dingen te zeggen}\\

\haiku{hem zoo hard tegen,.}{het lijf dat hij er bijna}{van omtuimelde}\\

\haiku{Ik voelde allengs;}{weder mijne geneigdheid}{tot hem terugkeeren}\\

\haiku{Het is wonder, mijn.....}{vader verlaat van gansche}{dagen dien stoel niet}\\

\haiku{{\textquoteright} {\textquoteleft}Maar, goede heer, ik.}{heb nooit iets van zijne hand}{gehoord of gezien}\\

\haiku{{\textquoteright} {\textquoteleft}Wel, lieve hemel,!}{dan heeft de schurk weder aan}{alles gelogen}\\

\haiku{Maar hij bad mij, nog,:}{eenige oogenblikken te}{blijven en zeide}\\

\haiku{Ach, ik smeek u, poog!}{hem dit verzenmaken uit}{het hoofd te kouten}\\

\haiku{{\textquoteright} De brouwer stond op.}{en deed eenige stappen om}{tot mij te komen}\\

\haiku{Wij zijn rijk, schatrijk,.}{en ik lijd meer gebrek dan}{eene bedelaarster}\\

\haiku{{\textquoteleft}Gelief mij nu te,{\textquoteright},.}{volgen zeide hij zich den}{mond afvegende}\\

\haiku{{\textquoteright} En daarop ging hij,;}{weg zonder mij eenen glimlach}{te hebben gegund}\\

\haiku{Een kwart uurs later.}{draafden wij met dezelfde}{haast de stad Gent uit}\\

\haiku{Jan-oom had, dacht mij,!}{van den duivel gesproken}{en gevloekt misschien}\\

\haiku{grommend en het hoofd,.}{spijtig schuddende liep hij}{de spreekkamer uit}\\

\haiku{Achter hem stapte.}{ik door twee of drie gangen}{en beklom eene trap}\\

\haiku{gij zult onder hen.}{brave jongens en goede}{vrienden aantreffen}\\

\haiku{Die pocher van eenen.}{Waal kome nog eens hier om}{u te bespotten}\\

\haiku{Ik besloot daaruit,,,.}{dat hij hoe spotziek ook een}{goed hart moest hebben}\\

\haiku{Ik moest hem alles,.}{vertellen wat ik mij maar}{kon herinneren}\\

\haiku{De tijd, door mijnen,.....}{oom bepaald kon niet verre}{meer verwijderd zijn}\\

\haiku{uw oom is rijk, en.}{stoffelijke middelen}{ontbreken u niet}\\

\haiku{Moest ik niet voortaan?}{mij gansch toewijden aan het}{geluk van Jan-oom}\\

\haiku{{\textquoteright} mompelde hij, het.}{hoofd met eenen zonderlingen}{glimlach schuddende}\\

\haiku{Gij zult er binnen,.}{eenige dagen van weten}{te spreken Mijnheer}\\

\haiku{Daar is lekker bier,{\textquoteright}.}{uit de brouwerij van Frans}{Cools zeide hij}\\

\haiku{Mijn oom won  zijn,?}{fortuin door den handel den}{eerlijken handel}\\

\haiku{Mijne zuster is.}{gebrekkelijk en kan geene}{trappen beklimmen}\\

\haiku{Zijn   Elkander.}{de leelijkste dingen naar}{het hoofd geworpen}\\

\haiku{maar dit belet hem,.}{niet te eten en te drinken}{als een gezond mensch}\\

\haiku{Dat is de eenige,,.}{tijd dat wij vrij zijn om te}{doen wat wij willen}\\

\haiku{Nicht Margriet duwde;}{aan mijnen arm en wilde}{mij vooruit doen gaan}\\

\haiku{niets hooren dan die?}{eeuwige lamentatie}{van Jeremias}\\

\haiku{{\textquoteleft}Kozijn, wij gaan naar,,{\textquoteright}.}{boven om het avondmaal te}{nemen zeide zij}\\

\haiku{Snijd den staart van het,,!}{beestje en begin te eten}{rommeledommel}\\

\haiku{De kussens waren.}{niet opgeschud en ik moest}{ze beter schikken}\\

\haiku{Wat mij betreft, ik,.}{zal u daartoe behulpzaam}{zijn zooveel ik kan}\\

\haiku{Nog in mij zelven,.}{mompelende stond ik op}{en verliet den tuin}\\

\haiku{maar zij kwam telkens.}{terug met een barsch en}{afwijzend antwoord}\\

\haiku{Ja, donnerwetter,!}{gij zult later weten hoe}{Jan Roobeek zich wreekt}\\

\haiku{Het gezicht mijner;}{tranen voerde de woede}{van Jan-oom ten top}\\

\haiku{en die dag werd een.}{der bitterste in mijn reeds}{zoo bitter leven}\\

\haiku{de maandrozen, bij het,;}{huis had zij hem op zijnen}{naamdag geschonken}\\

\haiku{zij waren voor het,.}{genoegen dat ik hunnen}{kinderen aandeed}\\

\haiku{143.)overdreven zijn, want.}{eenen engel zelven kon men}{niet hooger prijzen}\\

\haiku{Daarover iemand te,;}{ondervragen dit zou ik}{niet gewaagd hebben}\\

\haiku{Zij had bruin haar en,.}{groote zwarte oogen vol gevoel}{en vol uitdrukking}\\

\haiku{{\textquoteright} vroeg mij Margriet, toen.}{ik haar in den gang onzer}{woning ontmoette}\\

\haiku{{\textquoteright} Zij bekeek mij eene.}{wijl met zonderlingen blik}{en schudde het hoofd}\\

\haiku{- dat ik wenschte,.}{u deel in onze blijdschap}{te kunnen geven}\\

\haiku{Dan zal mijn arme.}{vader zijne oogen niet meer}{moeten bederven}\\

\haiku{{\textquoteright} {\textquoteleft}Hemel, mijnheer, gij?}{zoudt wreed genoeg zijn om ons}{dien hoon aan te doen}\\

\haiku{Hebt gij Beatrix,?}{niet bekend dat gij smoorlijk}{op haar verliefd zijt}\\

\haiku{en om toch iets te,,.}{zeggen vroeg ik Helena}{of zij verdriet had}\\

\haiku{maar hoe kwam het toch,,,?}{dat zij dit zeggende eenen}{diepen zucht slaakte}\\

\haiku{Hij drukte mij de,:}{hand en zeide terwijl hij}{zich verwijderde}\\

\haiku{gij zijt zachtaardig,,,,,.....}{teergevoelig kiesch beleefd}{geleerd verstandig}\\

\haiku{Eene gansche week bleef,.}{ik te huis zonder eenen voet}{op straat te zetten}\\

\haiku{want anders zou een.}{van ons beiden uit dezen}{weg niet meer opstaan}\\

\haiku{Eindelijk keerde.}{zij terug en riep mij van}{beneden de trap}\\

\haiku{De arme jongen,.}{weet niet meer wat hij zegt en}{is waarlijk half dol}\\

\haiku{Volgens hem weigert.}{zij zijne hand om met u}{te kunnen trouwen}\\

\haiku{En wie heeft het recht?}{om u rekenschap van uwe}{daden te vragen}\\

\haiku{Hij komt altijd in;}{ons huis om haar leelijke}{woorden te zeggen}\\

\haiku{{\textquoteright} {\textquoteleft}Weet gij, Mijnheer,{\textquoteright} vroeg, {\textquoteleft}?}{hijwaarvan men Helena}{durft beschuldigen}\\

\haiku{{\textquoteleft}Alzoo, gij wist niet,?}{dat Helena onder de}{Linde zou komen}\\

\haiku{{\textquoteright} {\textquoteleft}Ja, er is moed toe,{\textquoteright}.}{noodig ging zij onbewogen}{en met nadruk voort}\\

\haiku{In Loochristy en te.}{Gent had ik wel een twaalftal}{kozijns en nichten}\\

\haiku{Het is, omdat gij}{een goede jongen zijt en}{ik u eene wrare}\\

\haiku{evenwel, de koele}{hardheid der laatste woorden}{van Jan-oom leenden}\\

\haiku{Ik kon mijnen angst.}{en mijne ontsteltenis}{niet meer bedwingen}\\

\haiku{Wij blijven daarna,,?}{toch immer vrij te doen wat}{wij willen niet waar}\\

\haiku{maar ik toch zal hem,.....}{zegenen en God bidden}{dat Hij hem loone}\\

\haiku{twee grepen mij bij.}{de armen en rukten mij}{eenige stappen voort}\\

\haiku{Zoo dreigend en van!}{zoo nabij had hij den dood}{hem zien aangrijnzen}\\

\haiku{{\textquoteleft}Beminde Neef, ~ {\textquoteleft}.}{Jan-oom heeft gisteren eenen}{aanval gekregen}\\

\haiku{Mijn oom was ontwaakt,.}{en ik had haast om mij voor}{hem aan te bieden}\\

\haiku{de hemel zou ons,.}{het geluk gunnen hem nog}{lang te behouden}\\

\haiku{Dit vierde deel kon.}{beloopen tot elfduizend}{vijfhonderd kronen}\\

\subsection{Uit: Volledige werken 8. Mengelingen. Het geluk van rijk te zijn. De podagrist}

\haiku{En niettemin, dat.}{verwelkt gelaat getuigt van}{vergane schoonheid}\\

\haiku{{\textquoteleft}Ik zal u wijzen,,{\textquoteright},.}{mijnheer antwoordde de man}{tot mij komende}\\

\haiku{Dan kon Sus blijven,!}{slapen totdat de pachter}{van zelf ontwaakte}\\

\haiku{onder zijne oogen;}{hing de loodvervige toon}{der onkuischheid}\\

\haiku{Daar zat de Lafheid,;}{die slapend het hoofd boog in}{den schoot des meesters}\\

\haiku{Eene uitdrukking van;}{razernij grijnsde op het}{gelaat des meesters}\\

\haiku{Ik zal met Geert er,.}{over spreken en zien hoe hij}{zelf de zaak verstaat}\\

\haiku{Ach, gevoeldet gij,,,!}{Spits wat ik lijd wat helsche}{pijnen ik doorsta}\\

\haiku{ons met minder vrees.}{dan wij de krekels op de}{heide vertrappen}\\

\haiku{gij zult leven als.....}{een landheer tusschen al de}{honden van Oostmal}\\

\haiku{Met bevende hand;}{deed de herder de riemen}{van den buidel los}\\

\haiku{Het is een geschrei,.}{en een leven van droefheid}{dat men hoort noch ziet}\\

\haiku{Betteken is een;}{lief kind met blonde haren}{en roode kaakjes}\\

\haiku{krullebol Susken;}{veegt zijn paard uit en legt de}{schalie op den grond}\\

\haiku{Maar, och arme, de!}{onnoozele schaapkens waren}{dood en bevrozen}\\

\haiku{Wat baat het mij, fraai?}{gekleed en liefelijk van}{aangezicht te zijn}\\

\haiku{die smullen aan een,;}{stuk spek en aan een schotel}{pap dat ze blinken}\\

\haiku{maar hunne wangen,!}{zijn doorgaans zoo bleek hunne}{leden zoo mager}\\

\haiku{Zoo bedriegen ze,,}{altijd de arme menschen}{die moeten erven}\\

\haiku{ten minste volgens,;}{het zeggen der lieden want}{ik weet er niets van}\\

\haiku{Zoo gij nu nog niet,,.}{aan het zot worden zijt Trees}{dan geef ik het op}\\

\haiku{verwonderd zag zij,.}{op dewijl zij geen licht op}{de trap bemerkte}\\

\haiku{De man naderde,:}{haar stak zijne hand uit en}{sprak op drogen toon}\\

\haiku{Maar, Trees, gij zult toch,,?}{redelijk zijn niet waar en}{ons geld wat sparen}\\

\haiku{Schouwvegers fijn van,,.....}{den A.B. Aardige kwasten}{Vroolijke gasten}\\

\haiku{Schouwvegers fijn van,,.....}{den A.B. Aardige gasten}{Vroolijke kwasten}\\

\haiku{Het scheen, dat deze;}{laatste overweging hem troost}{in den boezem goot}\\

\haiku{Nu was bazin Smet;}{sedert een kwart uurs uit de}{stad teruggekeerd}\\

\haiku{Die bleeke garnaat, met?}{al hare strikskens en al}{hare krullekens}\\

\haiku{Zulk goed kind, die voor,;}{haren Pauw zou sterven}{als het noodig ware}\\

\haiku{En ik zal zelfs nog,,}{bidden dat God u eene vrouw}{geve die u zoo}\\

\haiku{{\textquoteright} Van ongeduld op,:}{den grond stampende viel de}{jongen spijtig uit}\\

\haiku{het was niets anders.}{dan eene ontsteltenis}{van de zenuwen}\\

\haiku{Denk eens, Pauw, zij is....}{daar straks uitgegaan om eene}{dienstmeid te zoeken}\\

\haiku{ik geloof, dat ze!}{altemaal eenen slag van den}{molen weg hebben}\\

\haiku{{\textquoteleft}Moeder, mag ik u,?}{eens iets verzoeken eer gij}{uwen mantel aflegt}\\

\haiku{Moeten hooren, dat,!}{mijn vader gestolen heeft}{dat hij een dief is}\\

\haiku{Gij zult het erfdeel?}{van mijnen vader aan de}{Wet overleveren}\\

\haiku{{\textquoteleft}Maar Ons Heerken lief,,?}{toch weet gij nu hoe Jan Grap}{den slag heeft gedaan}\\

\haiku{{\textquoteright} {\textquoteleft}Wat kan Pauw daaraan,?}{doen dat zijnen vader een}{ongeluk overkomt}\\

\haiku{{\textquoteright} {\textquoteleft}Dit moet ik het best,{\textquoteright}.}{weten antwoordde de vrouw}{zonder aarzelen}\\

\haiku{Voor de laatste maal,,.}{ik bid u om uw eigen}{welzijn spreek waarheid}\\

\haiku{Dit liedeken van!}{mijnheeren en mevrouwen}{is uitgezongen}\\

\haiku{Dien ganschen dag, tot,.}{in den avond was er krakeel}{en droefheid in huis}\\

\haiku{{\textquoteright} galmde bazin Smet, {\textquoteleft},!}{met de angstbleekheid op het}{gelaatoh mijn geld}\\

\haiku{Het verlies van het,;}{geld moet u pijnlijk zijn}{ik gevoel het wel}\\

\haiku{het hoofd met fierheid.}{opheffen en ieder vrij}{onder de oogen zien}\\

\haiku{{\textquoteright}   Van allerlei,... (.}{vieze sprongen ter kamer}{ingehuppeldbladz}\\

\haiku{Schouwvegers fijn van,,!}{den A.B. Aardige kwasten}{Vroolijke gasten}\\

\haiku{{\textquoteright} Pauw greep Kaatje bij de;}{hand en wilde met haar de}{kamer ronddansen}\\

\haiku{{\textquoteleft}O, ja, ja, gij zult,{\textquoteright}.}{mijne goede moeder zijn}{zuchtte het meisje}\\

\haiku{En dan neuriede,:}{de dokter binnensmonds op}{dezelfde wijs}\\

\haiku{Ai, ai, het is als!}{woelde men met gloeiende}{ijzers in mijn lijf}\\

\haiku{Reeds is de kwaal tot;}{mijn ingewand en mijne}{maag opgeklommen}\\

\haiku{{\textquoteright} {\textquoteleft}Ja, mijnheer,{\textquoteright} was het, {\textquoteleft},;}{antwoordmaar hier is een brief}{die zeer haastig schijnt}\\

\haiku{{\textquoteright} Met den vinger aan,:}{het voorhoofd riep hij na een}{oogenblik stilte}\\

\haiku{O,  God, dan denk,.}{ik telkens dat mijn laatste}{uur is verschenen}\\

\haiku{mijne beslissing.}{ten uwen opzichte hangt af}{van uwen goeden wil}\\

\haiku{Er valt hier niet te,.}{aarzelen ik ga u een}{drankje bereiden}\\

\subsection{Uit: Volledige werken 9. De schat van Felix Roobeek}

\haiku{En, tot belooning,!}{had hij haar zelfs haar wettig}{erfdeel ontnomen}\\

\haiku{{\textquoteright} Zonder nog haren,,}{raad te bestrijden drukten}{wij de meening uit}\\

\haiku{{\textquoteright} En zonder onze,.}{toestemming af te wachten}{opende zij de deur}\\

\haiku{ik toonde den stal,.}{waar men de beide paarden}{zou laten rusten}\\

\haiku{Hoe menigeen had!.....}{reeds zijne verkleefdheid met}{het leven geboet}\\

\haiku{{\textquoteright} zuchtte de moeder,.}{met de handen vooruit om}{haar kind te grijpen}\\

\haiku{hoeveel last zij ons;}{dus in het midden van den}{nacht veroorzaakte}\\

\haiku{Zoudt gij weigeren,?}{het hier te houden totdat}{het genezen is}\\

\haiku{Ziet hier wat ik, in,:}{zulk geval van uwe goedheid}{voor ons kind verwacht}\\

\haiku{32.)reepels van het lijf en.}{ontnam hem insgelijks eene}{soort van grove vest}\\

\haiku{maar Margriet, die mijn,.}{inzicht merkte liep toe en}{ontrukte het mij}\\

\haiku{Ik betaalde mijn.}{glas bier en ging dubbende}{voort langs den steenweg}\\

\haiku{Zeker, dit verlies.}{moest den dapperen man zeer}{smartelijk vallen}\\

\haiku{De reismaal lag in;}{de kas zooals de \'emigr\'e ze}{er in had geplaatst}\\

\haiku{Ik ben insgelijks,.}{ervaren in allerlei}{naaiwerk gij weet het}\\

\haiku{gij zoekt een ander.}{middel van bestaan en ik}{treed in een klooster}\\

\haiku{{\textquoteleft}Niet de helft der winst,.}{zal mij toebehooren maar}{slechts het derde deel}\\

\haiku{gij beiden zult de.}{twee derden der huiskosten}{te dragen hebben}\\

\haiku{want de schout had zelf,.}{mij gezegd dat hij mij geen}{werk kon bezorgen}\\

\haiku{{\textquoteleft}Ik ben naar uw huis,{\textquoteright}, {\textquoteleft}.}{geweest zeide hijom u}{een voorstel te doen}\\

\haiku{Omdat gij het zijt,.}{zal ik u vijftien gulden}{in de maand geven}\\

\haiku{Het gezicht dezer,.}{aankondiging verraste}{noch bedroefde mij}\\

\haiku{{\textquoteright} {\textquoteleft}Het is eene erge,,{\textquoteright};}{tijding welke gij mij geeft}{zeide ik treurig}\\

\haiku{Is het niet om den?}{ons toevertrouwden schat te}{kunnen bewaren}\\

\haiku{Het werd ingezet;}{voor den spotprijs van duizend}{gulden wisselgeld}\\

\haiku{Om zulk eene zending,.}{te vervullen zijt gij niet}{stoutmoedig genoeg}\\

\haiku{het nagelaten.}{teeken van den eersten druk}{mijner vingeren}\\

\haiku{IX Margriet had slechts.}{een paar honderd gulden op}{hare reis verteerd}\\

\haiku{Hij kende de list,,.}{gromde hij en zou wel wijn}{weten te vinden}\\

\haiku{En dit was alleen.}{de voornaamste bron mijner}{bekommerdheid niet}\\

\haiku{Louis Gapet boet voor,;}{misdaden waaraan hij geen}{het minste deel had}\\

\haiku{{\textquoteright} Eenigen tijd daarna.}{verliet ik de hofstede}{en keerde naar huis}\\

\haiku{een zijner mannen.}{liep naar boven en haalde}{mijnen sleutelbos}\\

\haiku{Onze nicht had zich;}{met de pistool in de hand}{op de trap vertoond}\\

\haiku{Wij bleven nog lang;}{onder den invloed van den}{doorgestanen schrik}\\

\haiku{{\textquoteleft}Veronderstel, Felix,.}{dat wij De Blauwe Vos voor}{80,000 franken koopen}\\

\haiku{Zie eens, Felix, hoe veel!}{lieden er voor de poort der}{olieslagerij staan}\\

\haiku{Op de Markt zagen.}{wij vele lieden voor de}{olieslagerij staan}\\

\haiku{ik zag in stommen,.}{angst ten gronde want ik wist}{niet wat te zeggen}\\

\haiku{ter hand. Des avonds schreef,,.}{ik met verbrijzeld hart dit}{geld op ons debet}\\

\haiku{ik wras schatrijk en!}{behield eenen post van vijftig}{gulden in de maand}\\

\haiku{{\textquoteright} En met eenen lach, die,.}{mij op voorhand deed vreezen}{liep zij de deur uit}\\

\haiku{Het staat ons vrij, de,.}{dwalingen te vermijden}{die hij heeft begaan}\\

\haiku{{\textquoteright}        XVI Drie weken;}{later woonden wij in het}{fraaie huis aan de Markt}\\

\haiku{De prijzen waren.}{niet veranderd en neigden}{veeleer tot daling}\\

\haiku{maar ik bespeurde,;}{wel dat zij de ontrust van}{haar gemoed verborg}\\

\haiku{De gelukkigste.}{vooruitzichten maakten mij}{licht van geest en hart}\\

\haiku{Zij nam eenen stoel en,:}{zeide mij zeer bedaard doch}{met scherpen nadruk}\\

\haiku{maar sedert drie of.....}{vier jaar kan hij zijne beenen}{niet meer gebruiken}\\

\haiku{integendeel, ik.}{zal u dankbaar blijven voor}{deze eerste gunst}\\

\haiku{De jongeling had:}{een weinig van de inborst}{mijner nicht Margriet}\\

\haiku{Maar gij wilt veel meer;}{teruggeven dan wat men}{u ter hand stelde}\\

\haiku{{\textquoteright} {\textquoteleft}Uwe verkleefdheid voor,}{mij doet u de zaak in mijn}{voordeel verwringen}\\

\haiku{ik was gelukkig.}{geweest en had tweeduizend}{franken gewonnen}\\

\haiku{Maar zij, zonder zich,;}{te laten troosten zakte}{terug op den stoel}\\

\haiku{want verlaat hij ons,!}{dorp hij neemt mijn geluk en}{mijn leven mede}\\

\haiku{maar ik alleen ben,.}{de oorzaak dat hij hier niet}{langer durft blijven}\\

\haiku{Ach, hoe gelukkig!}{had ik mij geacht met u}{te kunnen blijven}\\

\haiku{Ik schudde eene wijl:}{in stille overweging het}{hoofd en zeide dan}\\

\haiku{ga, verwijder u,,.....}{van hier gij die afkeer en}{schande met u voert}\\

\haiku{haar man nam zijne,.}{muts af doch bleef in zijnen}{leunstoel gezeten}\\

\haiku{Na het wisselen,.}{van eenen groet vroeg ik hen of}{hun zoon te huis was}\\

\haiku{Ondanks al, wat ik,.}{hem kon zeggen bleef hij mijn}{aanbod verstooten}\\

\haiku{Nu bezat zij niet,;}{alleen een kind schoon en lief}{als een hemelwicht}\\

\haiku{{\textquoteright} {\textquoteleft}En gij zijt zeker,?}{dat Victor de zoon van een}{der Bohemers is}\\

\haiku{Was hij niet van edel?}{bloed en waarschijnlijk van eenen}{doorluchtigen stam}\\

\haiku{Zij had het hoofd op.}{de borst laten vallen en}{weende in stilte}\\

\haiku{{\textquoteright} {\textquoteleft}O, mijn God,{\textquoteright} riep hij, {\textquoteleft}!}{de blijdschap heeft mij belet}{daaraan te denken}\\

\haiku{broeders en zusters,;}{uws vaders en uwer moeder}{neven en nichten}\\

\haiku{{\textquoteleft}Hoe, gij biedt de hand?}{van uw engelachtig kind}{aan uwen armen klerk}\\

\haiku{{\textquoteright} {\textquoteleft}Weet gij,{\textquoteright} vroeg ik, {\textquoteleft}dat?}{dit kind aan zijnen hals een}{zeker teeken droeg}\\

\haiku{{\textquoteright} {\textquoteleft}Ik zelve, Mijnheer,,:}{heb het met een zwart snoer aan}{zijnen hals geknoopt}\\

\haiku{maar Fr\'ed\'eric, door,.}{zijn ongeduld weggerukt}{hoorde haar niet meer}\\

\subsection{Uit: Volledige werken 10. De plaag der dorpen. Eene welopgevoede dochter}

\haiku{Hij intusschen loopt,,;}{zingt tiert en vloekt tot schande}{van het heele dorp}\\

\haiku{{\textquoteright} {\textquoteleft}Het is, omdat hij.}{de pacht van verleden jaar}{nog niet heeft betaald}\\

\haiku{maar het einde is,......}{de bedelzak de misdaad}{of of nog erger}\\

\haiku{ik voelde in mij.}{eene uitnemende kracht en}{wonderlijken moed}\\

\haiku{{\textquoteright} {\textquoteleft}Maar nu, vader, nu.}{zal hij zijne toestemming}{met blijdschap geven}\\

\haiku{Ik moet er eerst nog,.}{over slapen en weten hoe}{haar vader het meent}\\

\haiku{Oh, ik wist niet, dat;}{een mensch zooveel liefde voor}{een beest kan hebben}\\

\haiku{Alle drie stonden!}{in tranen te smelten over}{den dood eener koe}\\

\haiku{{\textquoteright} {\textquoteleft}Om den achterstal,{\textquoteright}.}{zijner pacht te betalen}{zeide het meisje}\\

\haiku{en nu, nu zou ik,?}{leven tusschen vrienden}{die mij liefhebben}\\

\haiku{Jan Staers vatte de;}{boterham en nam er eenen}{beet van in den mond}\\

\haiku{hij is een brave,.}{jongen die arbeidt van den}{morgen tot den avond}\\

\haiku{{\textquoteright} {\textquoteleft}Zie, gebuur,{\textquoteright} sprak de, {\textquoteleft};}{oudehet is nutteloos}{met mij te veinzen}\\

\haiku{doch bij het einde.}{wondden hem de bloedige}{verwijten zeer diep}\\

\haiku{Weinig tijds daarna.}{opende zich de deur der}{steenen hoeve opnieuw}\\

\haiku{Waart gij in zijne,.}{plaats gij zoudt misschien nog meer}{vergramd zijn dan hij}\\

\haiku{{\textquoteright} {\textquoteleft}Zie, Lucas, zoo gij,.}{geen geduld wilt hebben ik}{kan er niet aan doen}\\

\haiku{het is vandaag slecht,.}{weder morgen zal de zon}{misschien wel schijnen}\\

\haiku{doch zijn vader was:}{hem vooruit en sprak met een}{bevelend gebaar}\\

\haiku{Zie, nu drijft hij de,.}{lieden uiteen omdat zij}{te nader komen}\\

\haiku{Kunt gij het zien en,?}{daar koud blijven staan gelijk}{een steen zonder ziel}\\

\haiku{ik heb hem dezen,}{morgen in den hollen weg}{ontmoet hij had eenen}\\

\haiku{Maar wij hebben geen.}{recht om de dochter van den}{vader te scheiden}\\

\haiku{{\textquoteleft}Kom, laat ons geenen tijd,,:}{verliezen Beth neem wat er}{noodig is tot schuren}\\

\haiku{maar ik zie wel, dat.}{de ellende zelve u}{niet heeft veranderd}\\

\haiku{de arbeid, dien men,;}{met goeden wil aanvaard heeft}{nog niemand gedood}\\

\haiku{ik zal u toonen,.}{dat gij geene redenen hebt}{om te wanhopen}\\

\haiku{Uw voorspoed was mij,.....}{een eeuwig verwijt dat ik}{niet kon verkroppen}\\

\haiku{{\textquoteright} {\textquoteleft}Maar onderwerpt ge,?}{u aan de proef met goeden}{wil en in vriendschap}\\

\haiku{Wij zullen samen.}{nog goede dagen op de}{wereld beleven}\\

\haiku{Ik zal loopen en,.}{hem gaan zeggen dat hij wat}{stiller moet rijden}\\

\haiku{Hij poogde hun te,;}{doen begrijpen dat alles}{nog ten beste ging}\\

\haiku{Zij greep zijne hand,,:}{en hem smeekende in de}{oogen ziende vroeg zij}\\

\haiku{Drinken, drinken en,!}{daar nedervallen zonder}{rede zonder ziel}\\

\haiku{De Schallebijter,.....}{zou u wegjagen zoo gij}{nog eens eenen druppel}\\

\haiku{Jan Staers, jongen, wat,!}{zijt gij leelijk met die groote}{glasachtige oogen}\\

\haiku{{\textquoteleft}Vaarwel,{\textquoteright} mompelde,.}{de zandboer zich weder tot}{de deure richtend}\\

\haiku{Men moet ze weten,.}{aan te pakken of het is}{er verkeerd mede}\\

\haiku{Een telloor, waar een,;}{stuk afgebroken is kan}{nog haren dienst doen}\\

\haiku{Wacht maar eens, totdat.}{gij aan het kapittel van}{de kinderen komt}\\

\haiku{maar ik trapte hem,:}{al gauw op zijnen teen en}{dan eerst zeide hij}\\

\haiku{Ik had hem tegen,?}{te half vier hier verzocht en}{het is al vier uur}\\

\haiku{Dat zal hem leeren, mij,.}{ook te plagen gelijk hij}{Zondag heeft gedaan}\\

\haiku{{\textquoteleft}Ja, maar ik was nog,.}{geen vijf stappen verder of}{daar lag er nog een}\\

\haiku{De ware reden.}{van mijnen brief is echter}{het voorgaande niet}\\

\haiku{Het meisje is niet,.}{schoon maar ze speelt uitmuntend}{op de piano}\\

\haiku{{\textquoteleft}Comment, Messieurs,.....?}{vous avez pu croir eque cette}{pauvre Virginie}\\

\haiku{{\textquoteright} De goede lieden.}{toonden zich tevreden over}{mijne belofte}\\

\haiku{hoe iedereen, van,:}{den morgen tot den avond haar}{met vleitaal overlaadt}\\

\haiku{Ik heb daareven een,.}{brief ontvangen die in het}{Fransch is geschreven}\\

\haiku{Van daar gingen wij.}{naar de Kroon en bleven er}{tot laat in den avond}\\

\haiku{Hare moeder was.}{boven om haar te troosten}{en te verzorgen}\\

\haiku{{\textquoteleft}Mijnheer en Mevrouw,.}{ik verlang een ernstig woord}{u toe te richten}\\

\haiku{{\textquoteleft}Kan geld opwegen?}{tegen het leven en het}{geluk van mijn kind}\\

\haiku{Dat zij M. Gustaf,;}{ten hoogste genegen is}{dit wil ik gelooven}\\

\haiku{{\textquoteright} stamelde hij, van.}{ontsteltenis op zijne}{beenen wankelende}\\

\haiku{Was mijn kameraad?}{langs eene andere baan mij}{vooruitgeloopen}\\

\haiku{Het verschrikte mij,;}{weder zoo geheel alleen}{te moeten blijven}\\

\haiku{{\textquoteleft}Ach, Mijnheer Bernard,,.}{ik weet wel waarover gij mij}{wilt onderhouden}\\

\haiku{Ik aarzelde om.}{uwen armen vriend dien wreeden}{slag toe te brengen}\\

\haiku{{\textquoteleft}O, Mijnheer Spronck,;}{wacht nog eenige dagen}{om te beslissen}\\

\haiku{{\textquoteright} {\textquoteleft}Blijf bedaard, Gustaf,{\textquoteright}, {\textquoteleft}.....}{zeide ikde zaken staan}{ginder niet gunstig}\\

\haiku{Spronck  schrijven.}{en hare toestemming tot}{zijn bezoek vragen}\\

\haiku{Mijne belofte;}{aan den kapitein moest ik}{evenwel vervullen}\\

\haiku{Hij is daar, achter,.}{gindsche hofstede uit ons}{gezicht verdwenen}\\

\haiku{Ik luisterde niet.}{meer en trad met jagend hart}{in Gustafs kamer}\\

\haiku{{\textquoteleft}Bernard,{\textquoteright} zeide hij, {\textquoteleft}.}{ik heb een ernstig verzoek}{u toe te richten}\\

\haiku{maar ik houd er aan,.}{dat gij nooit vergetet wat}{ik u ga vragen}\\

\haiku{M. Wijkevorst had,,.}{haar voor een paar maanden}{eenen brief geschreven}\\

\subsection{Uit: Volledige werken 11. De Boerenkrijg}

\haiku{{\textquoteleft}Zijt ge niet beschaamd,,?}{groot mensch dat ge daar ligt te}{huilen als een kind}\\

\haiku{Aan de andere.}{zijde van het doek stond eene}{vrouw met eene viool}\\

\haiku{De beul zal komen,.....}{af te kappen Of anders}{ben ik niet voldaan}\\

\haiku{{\textquoteright} {\textquoteleft}Ziet daar, burgers en,.}{boeren hoe Robespierre}{zelf staat te beven}\\

\haiku{Het volk, als razend, '.}{bond Charlot En sleurde haar}{naart duister kot}\\

\haiku{men sprak er juichend.}{en met luider stemme over}{eene goede tijding}\\

\haiku{Gansch Waasland is op....}{dit oogenblik overdekt met}{Fransche soldaten}\\

\haiku{Meer dan zeshonderd;}{priesters zijn reeds naar verre}{eilanden vervoerd}\\

\haiku{en, wil God hem de,.}{martelkroon gunnen hij zal}{ze niet ontvlieden}\\

\haiku{doch een licht deurken,,.}{voor den trap dat gesloten}{was we\^erhield hem}\\

\haiku{eenigen schoten op;}{den vliedenden priester en}{op zijnen gezel}\\

\haiku{Een blij gejuich, een:}{angstige vreugdekreet stond}{op in den tempel}\\

\haiku{In de verte zag;}{men nog eenige vrouwen en}{kinderen vluchten}\\

\haiku{hij had een papier.}{en pennen voor zich liggen}{en scheen te schrijven}\\

\haiku{Die domme baas heeft,;}{ons wijsgemaakt dat hij van}{onze komst niets wist}\\

\haiku{In twee, drie uren zou,;}{hij al de dieren slachten}{die in het dorp zijn}\\

\haiku{Ik zal Jan van den,.}{notaris verzoeken dat}{hij ze ga halen}\\

\haiku{{\textquoteright} Allen zagen hem,.}{met nieuwsgierigheid aan doch}{niemand antwoordde}\\

\haiku{{\textquoteright} Simon-Brutus.}{en zijne makkers schoten}{in eenen langen lach}\\

\haiku{Een paard kan naar de,.}{herberg niet gaan het is een}{menschelijk gebrek}\\

\haiku{{\textquoteright} Simon-Brutus.}{vatte de pen en schreef het}{gevraagde bevel}\\

\haiku{{\textquoteright} Maar de korporaal,:}{greep hem bij den kraag schudde}{hem hevig en riep}\\

\haiku{{\textquoteright} De vrouw antwoordde.}{niet en bleef met het hoofd in}{de handen zitten}\\

\haiku{gij zoudt mij grootelijks,.}{verplichten zoo gij mij dit}{wildet toelaten}\\

\haiku{boven haar hoofd, op,,.}{de trap brandde de lamp die}{zij er had geplaatst}\\

\haiku{Hij was uitnemend.}{bleek en scheen in al zijne}{leden te beven}\\

\haiku{{\textquoteright} Maar de sergeant:}{wees met verwondering naar}{buiten en zeide}\\

\haiku{het scheen hem zelfs, dat.}{hij eenig bijna onvatbaar}{gerucht had gehoord}\\

\haiku{Ik gebied u in,!}{naam der Fransche Republiek}{verklaar wat gij weet}\\

\haiku{Welnu, Citoyen,,,?}{Torfs nog eens zult gij zeggen}{wat gij weet of niet}\\

\haiku{een der soldaten,,.}{in het been getroffen viel}{neder in het zand}\\

\haiku{{\textquoteright} Simon-Brutus:}{stampte ongeduldig op}{den grond en zeide}\\

\haiku{Breng insgelijks den,.}{Citoyen binnen die mij}{verlangt te spreken}\\

\haiku{En toch, dit alles!}{brengen wij u in naam der}{Fransche Republiek}\\

\haiku{zijn leven is ten,.}{einde hij zal sterven in}{de gevangenis}\\

\haiku{gun hem genade!}{om nevens zijn kerkje in}{vrede te sterven}\\

\haiku{in de rechterhand;}{hield hij  met krampachtig}{geweld een geweer}\\

\haiku{{\textquoteright} antwoordde de knecht;}{met luider stemme en naar}{het dorp wijzende}\\

\haiku{{\textquoteright} Intusschen waren;}{de andere personen}{tot Jan genaderd}\\

\haiku{Ik beef, alsof uw.}{mond mij de schrikkelijkste}{tijding melden moest}\\

\haiku{{\textquoteright} {\textquoteleft}Ah,{\textquoteright} riep Bruno uit, {\textquoteleft},!}{ik denk aan mijnen vader}{aan mijne moeder}\\

\haiku{En nu, geen ontzien,,,, -,!.....}{meer geene rust geene vrees geene hoop}{zelfs wraak wraak alleen}\\

\haiku{daaruit bleek, dat de.}{knecht zich in zijne gissing}{niet had bedrogen}\\

\haiku{Niemand verroerde,:}{iedereen weerhield de kracht}{zijner ademhaling}\\

\haiku{{\textquoteright} Bevend liet hij de.}{maagd ten gronde zakken en}{viel bij haar neder}\\

\haiku{zijn haar was verward,.}{de oogen gloeiden hem in het}{hoofd van vermoeidheid}\\

\haiku{geen enkele straal:}{der hoop daalde in zijnen}{benauwden boezem}\\

\haiku{zij doet geweld om.}{uit het verbrijzeld lichaam}{zich los te rukken}\\

\haiku{{\textquoteleft}Vrienden, luistert met.}{koelen bloede op hetgeen}{ik u melden ga}\\

\haiku{Karel uit de Leeuw:}{wrong zijn geweer krampachtig}{in de vuist en riep}\\

\haiku{{\textquoteleft}Die brief bewijst, dat;}{ik een afgezondene}{van uwe vrienden ben}\\

\haiku{zij dan de benden,.}{die de Franschen ons op het}{lijf zenden willen}\\

\haiku{En  zoudt gij hem?}{dan verlaten om u aan}{mijn lot te hechten}\\

\haiku{ik ook beroep mij,.}{op uwe edelmoedigheid op}{uwe menschenliefde}\\

\haiku{{\textquoteright} {\textquoteleft}O, Simon,{\textquoteright} zuchtte, {\textquoteleft}.}{de maagd op grievenden toon}{ik durf niet spreken}\\

\haiku{Eene doodsche stilte;}{heerschte tusschen deze}{ongelukkigen}\\

\haiku{Eene doodsche stilte.}{heerschte tusschen deze}{ongelukkigen}\\

\haiku{Mij dunkt, die windels;}{om zijn hoofd drukken te veel}{op zijne wonde}\\

\haiku{{\textquoteleft}Ik vermeen, dat wij.}{binnen de twee uren zullen}{kunnen terug zijn}\\

\haiku{{\textquoteright} {\textquoteleft}Ach, Kaat lief,{\textquoteright} smeekte, {\textquoteleft},;}{Genovevaom Gods wil}{laat ons spoed maken}\\

\haiku{op die voorwaarde.}{alleen heeft zij toegestemd}{om mij te volgen}\\

\haiku{{\textquoteright} {\textquoteleft}Uw ontwerp is goed,,{\textquoteright}.}{en gelukkig Veva lief}{antwoordde de knecht}\\

\haiku{Wees zeker, Veva,.}{de ongelukkige man}{zal er van sterven}\\

\haiku{Waar men haar naartoe,.}{geleid had die schuilplaats kon}{niemand ontdekken}\\

\haiku{Nu staat hij gewis,.}{op den heuvel om uit te}{zien of Jan niet komt}\\

\haiku{{\textquoteright} De brouwer scheen te.}{ontwaken en zag de maagd}{ondervragend aan}\\

\haiku{Dat ik niet vroeger,.}{tot hier geraakt ben mag u}{niet verwonderen}\\

\haiku{Een der ruiters kwam.}{in vollen draf in de baan}{teruggereden}\\

\haiku{niet minder was de.}{vernieling onder eenige}{andere vendels}\\

\haiku{{\textquoteright} Een zelfde roep steeg;}{op uit het gansche leger}{der patriotten}\\

\haiku{Slechts een honderdtal.}{mannen had hij binnen de}{vesting gelaten}\\

\haiku{De Generaal deed;}{de trommels slaan en gaf het}{verlangde teeken}\\

\haiku{Twee uren tijds werden.}{hem vergund om zich tot den}{dood te bereiden}\\

\haiku{Alle volkeren,,.}{de wilden zelfs hopen op}{een beter leven}\\

\haiku{{\textquoteleft}Ach, vader, wat ben,!}{ik blijde dat ik u nog}{eens mag aanschouwen}\\

\haiku{Aanvaarden wij het.....{\textquoteright} {\textquoteleft},,,.}{lot zooals het isNeen Simon}{wanhoop niet mijn zoon}\\

\haiku{{\textquoteleft}Ik heb den dood van;}{Bruno als een langgewenscht}{geluk nagejaagd}\\

\haiku{Gij, kapitein van,;}{Waldeghem zult met uw}{vendel vooruitgaan}\\

\haiku{Intusschen vlogen;}{de kogels met vernieuwde}{kracht boven zijn hoofd}\\

\haiku{Allen verspreidden,.}{zich door het gehucht om eene}{rustplaats te vinden}\\

\haiku{sommigen hebben;}{hoofd of arm met bebloede}{doeken omwonden}\\

\haiku{Eindelijk, dezen,;}{merken wel dat geene tegen}{weer meer kan baten}\\

\haiku{bij het gezicht van.....}{dien nieuwen vijand bevangt}{hen een doodelijke schrik}\\

\haiku{Zij zien eene opening,;}{in den kring door de komst der}{ruiters veroorzaakt}\\

\haiku{De bezwijkende?}{deugd moet zich schamen voor het}{zegepralend kwaad}\\

\subsection{Uit: Volledige werken 14. De burgemeester van Luik}

\haiku{allen, klein en groot,{\textquoteright} {\textquoteleft}?}{staan wij hier rondom u.En}{de vijf anderen}\\

\haiku{{\textquoteright} {\textquoteleft}Het zijn verraders,{\textquoteright}.}{of bespieders zeide een}{der jonge mannen}\\

\haiku{maar dien linkschen, dien.}{schijnheiligen blik heb ik}{meer dan eens gezien}\\

\haiku{{\textquoteleft}Twijfel er niet aan,{\textquoteright}, {\textquoteleft}.}{was het antwoordik ben nog}{wat ik vroeger was}\\

\haiku{{\textquoteleft}Ik geloof waarlijk,.}{dat wij de goede richting}{hebben verloren}\\

\haiku{{\textquoteright} Het meisje hief het:}{hoofd op en antwoordde met}{eenen zoeten glimlach}\\

\haiku{Zij schoof terzijde,;}{op de bank als om hem eene}{plaats aan te wijzen}\\

\haiku{De droeve tijden,,.}{die wij beleven kunnen}{niet eeuwig duren}\\

\haiku{Zoo wreed zijn voor eenen,!}{kameraad voor eenen armen}{dienstbode als gij}\\

\haiku{{\textquoteleft}Het is de stem van,{\textquoteright}.}{Dani\"el Laruelle}{zeide Elisakeith}\\

\haiku{van den eersten dag?}{af heb ik uw inzicht ten}{volle begrepen}\\

\haiku{Ik dacht, dat er een;}{einde aan die comedie}{moest gesteld worden}\\

\haiku{{\textquoteright} {\textquoteleft}Elisabeth, gij weet,.}{het de macht der zwakken is}{de voorzichtigheid}\\

\haiku{Is het niet de stem,?}{van mijnheer de Saizan die}{ik in de gang hoor}\\

\haiku{En wat mag ik den?}{burgemeester in naam des}{konings aanbieden}\\

\haiku{Poog te weten, wat,.}{hem zou kunnen verleiden}{en beloof het hem}\\

\haiku{{\textquoteright} {\textquoteleft}Maar die wisselaar,?}{zal dus weten dat ik geld}{ontvang van Frankrijk}\\

\haiku{Haddet gij u aan?}{zulke onderscheiding wel}{durven verwachten}\\

\haiku{Ik heb hem gevraagd,,;}{of hij niet verlangde u}{te groeten mevrouw}\\

\haiku{maar hij wilde u.}{niet storen en zal tegen}{den avond terugkeeren}\\

\haiku{{\textquoteleft}Ik herken zijne!}{wijze van kloppen en loop}{om hem te openen}\\

\haiku{Mejuffer Clara.}{bovenal boezemde mij}{medelijden in}\\

\haiku{Laat mij evenwel u}{nog eenige woorden zeggen}{om de vernieuwing}\\

\haiku{En nu, mijn waarde,.}{vriend vraag ik u op mijne}{beurt om verschooning}\\

\haiku{Ha, Laruelle,,,!}{ha de Mouzon gij hebt nog}{niet gedaan met mij}\\

\haiku{Inderdaad, ik heb,.}{het goed geoordeeld mij niet}{te doen aanmelden}\\

\haiku{, want ik heb daar eenen.....}{bijzonderen bode tot}{mijne beschikking}\\

\haiku{Toen de graaf zijne,:}{uitlegging eindigde vroeg}{de burgemeester}\\

\haiku{{\textquoteleft}Neen, heer resident,.}{gij zult mij toelaten dien}{naam te verzwijgen}\\

\haiku{Wij roepen dezen,.}{tot ons alsook de leden}{der schuttersgilden}\\

\haiku{De knecht Jaspar:}{trad in de zaal en zeide}{tot zijnen meester}\\

\haiku{mijn geweten zou.}{het mij verwijten tot op}{den boord van mijn graf}\\

\haiku{{\textquoteright} {\textquoteleft}Ik ben maar een smid,{\textquoteright}.}{antwoordde de kapitein}{met eenen diepen zucht}\\

\haiku{{\textquoteright} {\textquoteleft}Welnu, het lot heeft,{\textquoteright}.}{recht over hen gedaan zeide}{de burgemeester}\\

\haiku{anderen, gekwetst,.}{hadden hoofd of armen met}{doeken omwonden}\\

\haiku{maar hij bedwong zich,:}{greep de beide handen zijns}{makkers en zeide}\\

\haiku{hij keerde zelfs zich,;}{geheel om als wilde hij}{het huis verlaten}\\

\haiku{{\textquoteright} Een hevig schaamrood.}{beklom haar voorhoofd en zij}{staarde ten gronde}\\

\haiku{Ik verberg het u,,,}{niet Dani\"el hij hoopt dat}{hij door zijnen raad}\\

\haiku{Hij is uw vriend, gij.}{hebt hem de gewichtigste}{diensten bewezen}\\

\haiku{En gij gelooft, dat?}{de graaf van Warfuz\'ee zou}{kunnen toestemmen}\\

\haiku{Trad ik dus binnen,.}{zonder aanmelding het is}{de schuld van Gobert}\\

\haiku{Ik zeg u dit slechts,}{omdat men somwijlen eenen}{bekenden Chiroux}\\

\haiku{Het is enkel een,.}{uitstel dat door zich zelf wel}{betamelijk is}\\

\haiku{Welnu,{\textquoteright} zeide De, {\textquoteleft}.}{Vrieseopenbaar mij het}{schrikkelijk geheim}\\

\haiku{Uw duistere brief.}{heeft ons alle te Brussel}{met angst geslagen}\\

\haiku{{\textquoteleft}Alzoo is voor het.}{oogenblik het doel onzer}{bijeenkomst bereikt}\\

\haiku{Alles drijft mij aan.}{om u een onmiddellijk}{vaarwel te zeggen}\\

\haiku{Hadde hij zijn doel,;}{getroffen alles ware}{verloren geweest}\\

\haiku{{\textquoteright} Deze overweging.}{ontrukte hem eenen kreet van}{hoogmoed en blijdschap}\\

\haiku{{\textquoteleft}Ja, ja, ik ben de!}{stadhouder des keizers in}{het prinsdom van Luik}\\

\haiku{Laruelle heeft,.}{zijn dood verdiend en ik ik}{spreek zijn vonnis uit}\\

\haiku{De kanunniken.}{hebben den verrader bij}{het volk aangeklaagd}\\

\haiku{Zie, zie, gij zijt niets,!}{dan bloed rookend bloed van het}{hoofd tot de voeten}\\

\haiku{Hij maakte met haast:}{de armen van Jaspar}{los en zeide hem}\\

\subsection{Uit: Volledige werken 15. De geldduivel}

\haiku{honderden stemmen}{klinken uit struik en heester}{en zingen den Heer}\\

\haiku{Eene wijl hield hij den:}{blik met klimmenden angst in}{de ruimte gericht}\\

\haiku{{\textquoteright} Over het gelaat van.}{M. Kemenaer was eene wolk}{van spijt neergezakt}\\

\haiku{{\textquoteright} {\textquoteleft}Het is schoon, omdat,{\textquoteright}.}{het u verblijdt murmelde}{de maagd met koelheid}\\

\haiku{aan de deur van het.....}{geld moet komen kloppen en}{eene aalmoes vragen}\\

\haiku{{\textquoteright} {\textquoteleft}Veel goedheid, te veel,,{\textquoteright}.}{goedheid mijnheer Kemenaer}{antwoordde Koenraad}\\

\haiku{Ik zal terugkeeren,;}{al ware het twee- of}{driemaal op eenen dag}\\

\haiku{Zij stak haren arm,:}{in den zijnen en zijde}{op streelenden toon}\\

\haiku{het is maar, opdat.}{ik bereid zou zijn tegen}{dat Berthold komt}\\

\haiku{Het geld heerscht over,;}{bijzondere menschen niet}{over de menschheid}\\

\haiku{Zijn het de namen,?}{van mannen die boogden op}{de macht van het geld}\\

\haiku{Monck liet zijne;}{armen eindelijk op den}{lessenaar vallen}\\

\haiku{Hij stapte zelfs naar,;}{de deure toe alsof hij}{iemand verwachtte}\\

\haiku{want, wees zeker, de.}{vrek zou iets kunnen krijgen}{en ons ontsnappen}\\

\haiku{Berthold heeft eenen;}{put v\'o\'or zijne eigene}{voeten gegraven}\\

\haiku{duivel bedriegen,?}{die zich in groot gevaar brengt}{om u te dienen}\\

\haiku{{\textquoteleft}Het zij dan zoo, mits!}{er geen ander middel is}{om mij te redden}\\

\haiku{{\textquoteright} {\textquoteleft}Nog de vrouw, aan wie,.}{ik gisteren zeide dat}{gij niet te huis waart}\\

\haiku{Zeg mij hoeveel gij,{\textquoteright}.}{noodig hebt morde Robyn met}{spijtig ongeduld}\\

\haiku{Maar Monck hoestte,;}{om de aandacht zijns meesters}{tot zich te trekken}\\

\haiku{- De grijsaard scheen voor:}{den raad van zijnen klerk te}{zwichten en zeide}\\

\haiku{mijnen zoon van de,.....}{gevangenis misschien van}{den dood verlossen}\\

\haiku{Waarschijnlijk waren.}{het duizend franken in het}{water gesmeten}\\

\haiku{Nog eenige dagen,.....}{indien het niet beter met}{mijne borst wil gaan}\\

\haiku{Gij hadt ongelijk,,.}{Monck hem zulk gevaarlijk}{spel aan te raden}\\

\haiku{hij zal nu niet veel.}{neiging hebben om verzen}{te hooren lezen}\\

\haiku{{\textquoteright} Een krachtige klank;}{der bel onderbrak hem in}{zijne uitroeping}\\

\haiku{Indien de kunst het?}{eenig middel ware om dit}{doel te bereiken}\\

\haiku{misschien was het de,:}{droefheid zijns dienaars die hem}{tot bewustheid riep}\\

\haiku{niet sterven, nog niet,{\textquoteright}.}{sterven stamelde Robyn}{met halve stemme}\\

\haiku{De voorzitter van.}{het gerechtshof mag alleen}{het zegel breken}\\

\haiku{God behoede u!}{voor de uitvoering van uw}{noodlottig besluit}\\

\haiku{Dus vertrouwend, treed.}{ik binnen en bied mijnen}{oom het boekdeel aan}\\

\haiku{maar van al deze.}{beschouwingen was mijne}{natuur afkeerig}\\

\haiku{Ik geraakte op,;}{eenen zolder ik leed honger}{en vernedering}\\

\haiku{{\textquoteright} riep de jongeling, {\textquoteleft};}{verbaasd en bevendmaar het}{is onmogelijk}\\

\haiku{laat een nieuw blad met.}{andere verzen drukken}{en er invoegen}\\

\haiku{hij treurt nu misschien,.}{reeds over het verdriet dat hij}{mij heeft aangedaan}\\

\haiku{Met het hoofd tegen:}{de borst des muzikants liet}{vallen.mij zelven}\\

\haiku{de zwoeging harer.}{borst vervult de kamer met}{eentonig geluid}\\

\haiku{Als al degenen,,!}{die de menschen bedriegen}{moesten terugkomen}\\

\haiku{want, ik weet niet, ik,!}{ben zoo flauw aan mijn hart en}{het is hier zoo koud}\\

\haiku{maar toch ben ik door.}{mijne gedachten te vroeg}{uit het bed gejaagd}\\

\haiku{maar het was toch niet,!}{voor zijne schoone oogen den}{knorrigen buffel}\\

\haiku{{\textquoteright} {\textquoteleft}Zeg eens, Margriet, is,?}{het waar dat hij boven het}{millioen rijk was}\\

\haiku{Het is mogelijk,;}{dat mij een onvoorzichtig}{woord is ontvallen}\\

\haiku{{\textquoteright} {\textquoteleft}Mijnheer heeft gelijk,,{\textquoteright}.}{volstrekt gelijk viel Monck}{hem in de rede}\\

\haiku{Allerlei droeve;}{overwegingen hadden mij}{belet te slapen}\\

\haiku{Zij meent, dat daarin;}{het middel bestaat om mij}{desnoods te dwingen}\\

\haiku{{\textquoteright} Het hoofd oprichtend,:}{antwoordde Berthold op}{wanhopigen toon}\\

\haiku{{\textquoteright} riep de jongeling,.}{al gaande de hand zijns vriends}{dankbaar drukkende}\\

\haiku{want het geld mijns  .....}{ooms zou mij het hart vervuld}{hebben met wroeging}\\

\haiku{Hij zegt, dat mijn oom.}{op zijne aanbeveling}{het hem heeft verzocht}\\

\haiku{In het huis, waar ik,:}{eene kamer bewoon staat een}{fraai kwartier ledig}\\

\haiku{het is een geluk,.....}{dat mij met dankbaarheid ten}{hemel doet blikken}\\

\haiku{eene ontroering, die,.}{hem zelven verraste had}{hem aangegrepen}\\

\haiku{Neen, neen, daartoe is.}{zijne vaderlijke}{liefde te innig}\\

\haiku{Heeft Berthold dan?}{nog de stoutheid gehad om}{ten uwent te komen}\\

\haiku{{\textquoteright} {\textquoteleft}Welaan, heer Monck,!}{dit gaat op uwe gezondheid}{en op uw geluk}\\

\haiku{Nu, in dit geval,.}{ziet gij mij voor de laatste}{maal heer Kemenaer}\\

\haiku{Mijn goede Monck,,,!}{onder ons gezegd gij zijt}{leelijk zeer leelijk}\\

\haiku{Zij stierve van schrik.}{bij de gedachte alleen}{van zulk huwelijk}\\

\haiku{zijn gelaat bewoog,.}{krampachtig zijne stem was}{dor en ratelend}\\

\haiku{{\textquoteleft}Alzoo, ik zou de,?}{echtgenoote worden van eenen}{man dien ik niet ken}\\

\haiku{Mijne hand en mijn?}{hart zouden de prijs worden}{eener somme gelds}\\

\haiku{ik verwachtte mij.}{aan deze genegenheid}{van uwentwege niet}\\

\haiku{voedsel zoeken voor,.....}{de koorts die mijne ziel en}{mijn lichaam verslindt}\\

\haiku{{\textquoteright} De muzikant greep:}{Bertholds beide handen}{en sprak met nadruk}\\

\haiku{Arme vriend, ik zal:}{u niet vragen welke kwaal}{u doet verkwijnen}\\

\haiku{Zoo deed ook de zoon,:}{van mijnen pachter toen hij}{op het trouwen stond}\\

\haiku{en ik uit goedheid,,;}{uit vriendschap ik laat u hier}{meesteresse zijn}\\

\haiku{Waarom komt Laura's,?}{naam altijd op uwe lippen}{als gij alleen zijt}\\

\haiku{Maar gij begrijpt wel,,;}{Margriet dat dit huwelijk}{moet worden belet}\\

\haiku{Blijf voortaan gerust.}{en luister niet meer naar den}{praat der geburen}\\

\haiku{Zij maakt zich ziek en.}{verkwijnt om haar huwelijk}{te doen verdagen}\\

\haiku{dieper waren de;}{rimpels des kommers in zijn}{voorhoofd gegraven}\\

\haiku{Hij liet de handen.}{nedervallen en legde}{ze aan zijn voorhoofd}\\

\haiku{Laura bemerkte,;}{haren vader eerst toen hij}{haar zeer nabij was}\\

\haiku{beloof mij, dat gij.....}{het beeld des doods van voor uwe}{oogen zult verjagen}\\

\haiku{Gij verlangt immers,?}{niet meer dat het huwelijk}{worde uitgesteld}\\

\haiku{En dat het de bloote,;}{waarheid bevat daaraan is}{niet te twijfelen}\\

\haiku{Gedurende eene.}{lange wijl wandelden zij}{allen zwijgend voort}\\

\haiku{gij zult de schoonste,.}{bruid zijn die men in lange}{jaren heeft gezien}\\

\haiku{Drie slechte stoelen;}{en eene tafel vormden er}{het gansche huisraad}\\

\haiku{{\textquoteright} De zieke hief de:}{oogen ten hemel en zuchtte}{op grievenden toon}\\

\haiku{{\textquoteleft}Welke waardigheid?}{blijft er mij in mij zelven}{te eerbiedigen}\\

\haiku{de bleekheid was op;}{zijn gelaat door den blos der}{hitte vervangen}\\

\haiku{daarom klopte het.}{hart der beide vrienden bij}{het openen des briefs}\\

\haiku{er zou toch voor mij.}{geen enkele rustige}{dag meer kunnen zijn}\\

\haiku{Morgen zal Laura.}{voor Gods altaar de hand van}{Monck aanvaarden}\\

\haiku{Nu, goede vrouw, zet,{\textquoteright}.}{u neer en wees gerust sprak}{hij met minzaamheid}\\

\haiku{{\textquoteright} viel de muzikant.}{met aangejaagd ongeduld}{haar in de rede}\\

\haiku{- en hij heeft mij wat,.....}{anders wijsgemaakt om dit}{te doen vergeten}\\

\haiku{en haar vader had.}{al lang in haar huwelijk}{met hem toegestemd}\\

\haiku{Al de anderen;}{luisterden met open mond en}{opgespalkte oogen}\\

\haiku{Laura stond in hun.}{midden met hangend hoofd en}{halfgesloten oogen}\\

\haiku{zoo vol spot, zoo vol,.}{droefheid dat de vrouw verbaasd}{achteruitdeinsde}\\

\haiku{verberg als ik voor,;}{iedereen wat pijnen uwen}{boezem doorwoelen}\\

\haiku{{\textquoteleft}Mijn God, mijn God,{\textquoteright} kreet,, {\textquoteleft}}{Kemenaer zich de handen}{voor de oogen slaande}\\

\haiku{Daar, neem uw handteeken,{\textquoteright}, {\textquoteleft}.}{zeide hijhet is Laura}{die ik hebben moet}\\

\haiku{{\textquoteright} Berthold legde.}{zich de handen voor de oogen}{en bleef sprakeloos}\\

\haiku{De rechter heeft het;}{noodlottig huwelijk niet}{kunnen beletten}\\

\haiku{uwe liefde alleen,.}{kan haar behoeden voor het}{graf dat op haar gaapt}\\

\subsection{Uit: Volledige werken 16. Eene gekkenwereld. De twee vrienden. Rikke-tikke-tak}

\haiku{niet alleen scheen hij;}{wel dertig voet loodrechte}{hoogte te hebben}\\

\haiku{Het was echter geene;}{eigenlijke verschriktheid}{die mij ontstelde}\\

\haiku{Wij hebben tijd om.}{wat van Antwerpen en de}{vrienden te kouten}\\

\haiku{Luister, zij zingt het,:}{eeuwige het eenige lied}{dat men hier kent}\\

\haiku{Daar opende men, recht,:}{over mij eene deur welke ik}{niet had opgemerkt}\\

\haiku{Houdt  uwe tanden,!}{gesloten of ik streel uwen}{rug met mijne knots}\\

\haiku{{\textquoteright} uit de alkoof, en.}{de kat springt huilend tusschen}{de gordijnen door}\\

\haiku{{\textquoteleft}Met dit haspelen.}{en dit schreeuwen geraken}{wij tot geen besluit}\\

\haiku{Lieve man, het is,{\textquoteright}.}{moeilijk met u te kouten}{begon vrouw Noppe}\\

\haiku{Hij is landmeter.}{en zal zich dit ambacht met}{meer vlijt aantrekken}\\

\haiku{{\textquoteright} Baas Noppe slaakte.}{eenen zucht en wreef zich met de}{hand over het voorhoofd}\\

\haiku{Neen, vrouw, dit doe ik,,,,!}{niet zeg ik u noch vandaag}{noch morgen noch ooit}\\

\haiku{Lisa, gij weet dat.}{gij met eenen korf eieren}{naar den winkel moet}\\

\haiku{{\textquoteright} {\textquoteleft}Gij zegt het om te,,{\textquoteright}.}{lachen majoor wedersprak}{zijn jonge gezel}\\

\haiku{zij zouden dus nu.}{maar heengaan en tegen den}{middag wederkeeren}\\

\haiku{Hebt gij ergens eene,.}{pijnlijke wonde zij kan}{slechts aan het hart zijn}\\

\haiku{Wij zijn insgelijks.}{jong en weten ook al iets}{van zulke dingen}\\

\haiku{{\textquoteleft}Lisa heeft hare.....{\textquoteright} {\textquoteleft},,!}{zinnen op TheodoorNeen}{neen verdenk haar niet}\\

\haiku{Geen verschil was er.}{tusschen de stof der schaal en}{die der letteren}\\

\haiku{{\textquoteright} riep moeder Noppe,.}{die eenen lichtstraal in haren}{geest voelde dringen}\\

\haiku{Toen de fourier hem,:}{genaderd was fluisterde}{hij hem in het oor}\\

\haiku{{\textquoteleft}Ik doe u gaarne,;}{uitgeleide gij twijfelt}{daar zeker niet aan}\\

\haiku{Let maar op, dat uw.}{ziekelijk medelijden}{u zelf niet zot maakt}\\

\haiku{Gij hebt tot nu toe,;}{maar vier onzer zotten}{gezien Mijnheeren}\\

\haiku{Carabos te zien,}{zou ik hun aanraden tot}{morgen te wachten}\\

\haiku{Leelijk - zooals wij het -;}{woord verstaan zijn ze boven}{alle beschrijving}\\

\haiku{op het gelaat van;}{den sergeant-majoor}{zweefde een glimlach}\\

\haiku{niets hebbende dan!}{den trouwen dienst van onzen}{goeden dwerg Topaas}\\

\haiku{{\textquoteright} zuchtte de fourier, {\textquoteleft},.}{opstaandemij dunkt ik zou}{er ziek van worden}\\

\haiku{want het verblijf te,.}{Gheel was voor hem niet goed dit}{gevoelde hij wel}\\

\haiku{Dien namiddag dreef.}{er een hevig onweder}{over de gemeente}\\

\haiku{De overdrevenheid.}{zijner ontroeringen had}{hem gansch genezen}\\

\haiku{Nu deed het gezicht;}{der zinneloozen bijna geenen}{indruk meer op hem}\\

\haiku{Welaan, gij zijt een;}{goede kameraad en een}{bescheiden  vriend}\\

\haiku{geen lichtstraal kan er,!}{nog binnen geene hoop meer voor}{mij dan in den dood}\\

\haiku{Lucia zit alleen,.}{beneden de prinses is}{op hare kamer}\\

\haiku{Vallen op het veld,,!}{van eer en zoo den worm dooden}{die mijn hart verscheurt}\\

\haiku{{\textquoteright} {\textquoteleft}Iedereen weet, dat.}{gij een gevoelig hart hebt}{en menschlievend zijt}\\

\haiku{{\textquoteleft}Ik heet Willem Hoofs,.}{en woon te Elsene}{bij het Keienveld}\\

\haiku{{\textquoteleft}Toen hij twee of drie,.}{uren later mij terugvond}{scheen hij zeer blijde}\\

\haiku{en toch, ik wilde!}{haar voorbij en deed eenen stap}{meer naar den afgrond}\\

\haiku{{\textquoteright} {\textquoteleft}O, Mijnheer,{\textquoteright} kreet de.}{ontstelde jongeling met}{tranen in de oogen}\\

\haiku{hij kon zijne vrouw.}{en zich zelven niet zoo van}{alles ontblooten}\\

\haiku{{\textquoteright} De jongeling sprong.}{op en scheen van verrassing}{en hoop te beven}\\

\haiku{{\textquoteright} kreet zij, tot in het.}{midden der kamer hem te}{gemoet komende}\\

\haiku{Maar zijn vriend, hem de,:}{hand grijpende zeide met}{geestdrift in de stem}\\

\haiku{{\textquoteright} De veekoopman liet;}{weder een oogenblik in}{stilte voorbijgaan}\\

\haiku{{\textquoteright} De dokter keerde.}{zich om en vervorderde}{langzaam zijnen weg}\\

\haiku{u gebruikt hebben?}{om de verlossing van mijn}{kind te bewerken}\\

\haiku{Mijne moeder zal.}{nu zoo gansch alleen dag en}{nacht aan mij denken}\\

\haiku{Hij heeft eene oude -.}{moeder eene deugdzame en}{edelhartige vrouw}\\

\haiku{{\textquoteleft}O, laat ze met ons!}{op het schoone landgoed te}{Boendale wonen}\\

\haiku{Hoe schilderachtig,:}{dit huis ook zij het biedt toch}{niets bijzonders aan}\\

\haiku{Geloof mij, moeder,!}{of ik bezweer het met eenen}{schrikkelijken eed}\\

\haiku{Deze vroolijke;}{muziek scheen den kolonel}{zeer te ontroeren}\\

\haiku{gras, heide, water,,.}{boomen alles groet mij in}{eene roerende taal}\\

\haiku{{\textquoteright} riep zij, {\textquoteleft}o, wees niet,?}{bedroefd ik zal immers nog}{wel wederkomen}\\

\haiku{{\textquoteright} De jonge boer sloeg.}{den blik ten gronde en bleef}{eenen tijd roerloos staan}\\

\haiku{{\textquoteright} {\textquoteleft}Vergeef mij, vader,{\textquoteright}.}{sprak de jongeling met waar}{berouw in de oogen}\\

\haiku{{\textquoteleft}Ik weet, dat gij niets;}{verlangt dan wat mij goed en}{voordeelig zou zijn}\\

\haiku{want sedert lang zijn.}{mijne droomen niets meer dan}{gal en bitterheid}\\

\haiku{Althans, zij had het.}{nooit aan zich zelve of aan}{anderen bekend}\\

\haiku{{\textquoteright} Trien trok een breiwerk:}{uit haren zak en zeide}{met even stille stem}\\

\haiku{de rijke menschen, -:}{geven hun geld en ik ik}{geef ook wat ik heb}\\

\haiku{de eene zegt dit, de,.}{andere zegt dat en op}{den duur weet men niets}\\

\haiku{{\textquoteright} {\textquoteleft}Maar, Meken, hoe hebt?}{gij hem kunnen verzorgen}{en onderhouden}\\

\haiku{of tenzij dat gij?}{ergens eene kous onder de}{pannen hebt steken9}\\

\haiku{Want een bruidegom.}{eischt meer tot zijn geluk}{dan koude vriendschap}\\

\haiku{het hart verdroogt, als.}{men het niet in een ander}{hart uitstorten kan}\\

\haiku{{\textquoteright} Zichtbaar beefde de,.}{maagd terwijl zij sprakeloos}{het hoofd voorover boog}\\

\haiku{Monica lag met;}{het hoofd op de tafel en}{moest bitter weenen}\\

\subsection{Uit: Volledige werken 17. De arme edelman. Eene 0 te veel}

\haiku{maar ik zou, door eene,?}{verkooping al mijne hoop}{gaan verbrijzelen}\\

\haiku{Nu toch verlicht een.....}{laatste straal der hoop onze}{duistere toekomst}\\

\haiku{Onze eenzaamheid:}{zal onze armoede niet}{langer verbergen}\\

\haiku{de edelman sprong recht,;}{zoo haast de bediende de}{zaal had verlaten}\\

\haiku{Daar staande, scheen hij;}{nog aan een ijselijken}{strijd overgeleverd}\\

\haiku{De edelman ging de;}{deur voorbij en wandelde}{de straat ten einde}\\

\haiku{En, inderdaad, het,.}{was een geheim zelfs voor den}{pachter der hoeve}\\

\haiku{Mijnheer De Necker.}{en zijn neef zullen hier het}{middagmaal nemen}\\

\haiku{want niettemin blonk.}{een liefdevolle glimlach}{op zijn aangezicht}\\

\haiku{{\textquoteright} De edelman zonk eene.}{wijl in de bespiegeling}{van zijns broeders lot}\\

\haiku{aldus twee flesschen.}{voor mijnheer De Necker en}{\'e\'ene voor zijn neef}\\

\haiku{Morgen zal het oog;}{der menschen  mistrouwend}{op u zich richten}\\

\haiku{- De koopman voelde}{zich door een waar gevoel van}{vriendschap tot mijnheer}\\

\haiku{hem kwam het vreemd voor,;}{dat men zich om zijn vertrek}{te verblijden scheen}\\

\haiku{doch zijne woorden.}{schenen het gewenschte doel}{niet te bereiken}\\

\haiku{Uwe oneindige;}{goedheid alleen geeft mij den}{noodigen moed daartoe}\\

\haiku{Wij zijn armer dan,.}{de pachter die de hoeve}{bij de poort bewoond}\\

\haiku{De eenzaamheid van.}{den Grinselhof bezielen}{door onze liefde}\\

\haiku{Eene vrouw moet haren.}{echtgenoot onderdanig}{volgen waar hij gaat}\\

\haiku{{\textquoteright} zuchtte de edelman,.}{zich de vuisten nevens het}{lichaam wringende}\\

\haiku{En, vermits gij mij,}{dwingt tot spreken vooraleer}{ik uw voornemen}\\

\haiku{dan, daartoe hebt gij,;}{eene  slechte keus gedaan}{heer Van Vlierbeke}\\

\haiku{Lenora bemerkte,.}{even  spoedig dat diepe}{smart hem ontstelde}\\

\haiku{Lenora had twee of;}{drie stappen gedaan om zich}{te verwijderen}\\

\haiku{Nu kome wat wil,.}{ik zal moedig zijn tegen}{verdriet en treurnis}\\

\haiku{den wreeden worm uit,!}{zijn hart rukken hem redden}{door mijne liefde}\\

\haiku{{\textquoteleft}Lenora, Lenora, mijn,,?}{kind zijt gij een bovenaardsch}{wezen een Engel}\\

\haiku{Zij herhaalde in;}{stilte zijne teederste}{bekentenissen}\\

\haiku{Te tien uren was de,,;}{zaal waar men beginnen zou}{met menschen vervuld}\\

\haiku{Een oogenblik slechts;}{duurde die hoonende houding}{der aanwezigen}\\

\haiku{Zij dragen beiden.}{een pakje in de hand en}{scheinen reisvaardig}\\

\haiku{Met trage stappen.}{gingen vader en dochter}{tot bij de hoeve}\\

\haiku{ik zou geheel mijn.}{leven het mij verwijten}{en er om treuren}\\

\haiku{En wiste ik het,{\textquoteright}.}{de voorzichtigheid zou het}{mij doen verzwijgen}\\

\haiku{{\textquoteright} Eene wijle stond hij,.}{beweegloos met de hand aan}{het voorhoofd gedrukt}\\

\haiku{Eene jonge dienstmeid.}{staat op den dorpel en lacht}{en praat met de knechts}\\

\haiku{de pachteresse.}{stond met het hoofd gebogen}{en was diep ontroerd}\\

\haiku{Misschien berust nog;}{in uw hart een rijke schat}{van moed en van hoop}\\

\haiku{Met het hoofd over haar,;}{werk gebogen schijnt zij ten}{gronde te blikken}\\

\haiku{Ah, hoe schatert gij,!}{van blijdschap hoe machtig slaat}{gij uwe vlerken uit}\\

\haiku{Gij zelve schijnt door;}{deze ongelukkige}{tijding getroffen}\\

\haiku{Welnu, welnu,{\textquoteright} vroeg, {\textquoteleft},?}{hijwat is het dan dat u}{zoo gelukkig maakt}\\

\haiku{hij zeide, terwijl:}{tranen van ontroering uit}{zijne oogen sprongen}\\

\haiku{maar hij stond tegen:}{zijne ontsteltenis op}{en zeide troostend}\\

\haiku{{\textquoteright} {\textquoteleft}Onder ons gezegd,,.}{ik heb nooit gedacht dat ze}{nog kon genezen}\\

\haiku{Somwijlen bekruipt;}{mij eene bekoring om Isidoor}{den hals te breken}\\

\haiku{{\textquoteright} {\textquoteleft}Neen, de knecht, dien wij,.}{zullen hebben zal de melk}{naar de stad voeren}\\

\haiku{Dan keerde hij zich,,!}{om en   Vaarwel goede}{edele vriendinne}\\

\haiku{en kom  mij dan,.}{zeggen of uwe moeder wel}{blijde is geweest}\\

\haiku{door zulke gekke.}{droomen brengt men de meisjes}{op eenen slechten weg}\\

\haiku{En gij spreekt juist, vrouw,.}{alsof Simon Storms niet meer}{op de wereld was}\\

\haiku{Ach,  kon zulk lot,:}{ons ook eens te beurt vallen}{zoudt gij niet zeggen}\\

\haiku{{\textquoteright} {\textquoteleft}Wel, goede man,{\textquoteright} kreet, {\textquoteleft}.}{de apothekergij zijt nog}{van den ouden tijd}\\

\haiku{Het antwoord liet zich,.}{wachten als was de koeboer}{in twijfel geraakt}\\

\haiku{Geef, lieve Heer, ons,!}{kost en kle\^er Het hemelrijk}{en dan niet meer}\\

\haiku{Hij was reeds tot bij,:}{de deur toen vrouw Storms hem met}{aandringen toeriep}\\

\haiku{{\textquoteright} {\textquoteleft}Gauw dan, moeder, ik,{\textquoteright},.}{heb geenen tijd morde hij tot}{haar terugkeerende}\\

\haiku{{\textquoteleft}Krijsch niet, moeder lief,.}{ik zal medegaan en u}{nimmer verlaten}\\

\haiku{{\textquoteright} {\textquoteleft}Simon, wij hebben:}{eene verkeerde gedachte}{over de Godshuizen}\\

\haiku{{\textquoteright} {\textquoteleft}Neen, ik ga niet weg,{\textquoteright}, {\textquoteleft}.}{morde hij beradenik}{moet u iets vragen}\\

\haiku{Hij zette zich op,:}{eenen stoel nam de hand zijner}{moeder en zeide}\\

\haiku{{\textquoteright} Vrouw Storms haalde de.}{sleutels uit haren zak en}{reikte ze hem toe}\\

\haiku{want zij vreesde, dat;}{deze aangejaagdheid}{hem ziek zou maken}\\

\haiku{Zoo ontsnapte hij,,.}{acht dagen later aan eenen}{gevaarlijken slag}\\

\haiku{Simon liet de pen (.}{uit zijne hand vallen en}{opende eene deurbladz}\\

\haiku{Ach, wat kon eene 0!}{te veel toch schrikkelijke}{gevolgen hebben}\\

\haiku{Simon onderging;}{eveneens den invloed van het}{moedbarend metaal}\\

\haiku{{\textquoteright} Baas Verhoeven en.}{zijne vrouw aanschouwden hem}{immer even verbaasd}\\

\haiku{Simon,{\textquoteright} vroeg zij, {\textquoteleft}is,,?}{de koets die daarbuiten voor}{de deur staat van u}\\

\haiku{{\textquoteleft}Ja, een dwaashoofd en,{\textquoteright}.}{een slecht mensch ben ik zeide}{de koeboer zuchtend}\\

\subsection{Uit: Volledige werken 18. De kwaal des tijds}

\haiku{Die oude izegrim;}{zit ons op den nek van den}{morgen tot den avond}\\

\haiku{Gij meent, dat ik voor?}{zoo weinig van schrik door den}{mestput zou loopen}\\

\haiku{Dan bracht hij zijnen,:}{stoel nader vatte de hand}{des grijsaards en sprak}\\

\haiku{over aan het lot, dat.}{hij zich zelven wetens en}{willens voorbereidt}\\

\haiku{betuig, bid ik u,;}{mevrouw Van Everdael mijne}{erkentelijkheid}\\

\haiku{Maar neen, het hart van!}{Dani\"el is een schat van}{goedheid en liefde}\\

\haiku{maar dat hij zijnen!}{ouden voedstervader niet}{meer zou beminnen}\\

\haiku{Gij hebt beiden eene!}{zonderlinge wijze van}{gelukkig te zijn}\\

\haiku{tranen der liefde.}{en der vriendschap stroomden in}{stilte rondom hem}\\

\haiku{met uw oorlof, de.}{schoone kleederen doen eene}{boerin ook geen kwaad}\\

\haiku{{\textquoteright} Dit zeggende, sprong.}{zij met de handen omhoog}{in de baan vooruit}\\

\haiku{Deze persoon droeg.}{eenen langen blauwen jas met}{vergulde knoopen}\\

\haiku{Moet gij mij daarom,?}{bekijken alsof gij mij}{wildet verslinden}\\

\haiku{Hij trekt een gezicht,.}{alsof de wereld tegen}{zijnen dank draaide}\\

\haiku{{\textquoteleft}Is het z\'o\'o, dat men?}{in Parijs zijne meesters}{leert eerbiedigen}\\

\haiku{Eindelijk daalde.}{het vertrouwen weder in}{des grijsaards boezem}\\

\haiku{{\textquoteleft}Ah sa, Dani\"el,!}{gij begint mij schrikkelijk}{te  vervelen}\\

\haiku{het eene staat onder;}{bevel van mijne rede}{en van mijnen wil}\\

\haiku{Deze twee zielen;}{strijden in mijn binnenste}{om de overwinning}\\

\haiku{{\textquoteleft}Zie de wereld zooals,.}{zij is en vraag haar niet wat}{zij niet geven kan}\\

\haiku{Ik heb haar bemind,.}{toen mijn hart even eenvoudig}{was als het hare}\\

\haiku{de stilte, de rust.}{alleen kan zijn gemoed tot}{bedaren brengen}\\

\haiku{{\textquoteleft}Ach, lieve tante,{\textquoteright}, {\textquoteleft}!}{zeide zij met verdoofde}{stemhij is zoo ziek}\\

\haiku{{\textquoteleft}Maar er is niets hoonends,.}{voor u in deze meening}{goede Willibald}\\

\haiku{IV De nacht moest de;}{arme Dani\"el niet veel}{rust gegund hebben}\\

\haiku{Wanneer gij eens ligt,:}{neergeknakt kan niets weder}{uwen stengel rechten}\\

\haiku{Eilaas, verloren,,!}{verloren voor altijd de}{kracht tot beminnen}\\

\haiku{Hoe komt het, dat gij?}{zijne plaats voor dit getouw}{ingenomen hebt}\\

\haiku{Ik heb ze beiden,;}{verzorgd nacht en dag alleen}{bij hun bed gestaan}\\

\haiku{{\textquoteleft}En gij, Rosalie,,?}{gij vervult uwe heilige}{belofte niet waar}\\

\haiku{{\textquoteright} {\textquoteleft}O, gij goede vrouw,{\textquoteright}, {\textquoteleft}!}{zuchtte de jonkheerwat moet}{gij gelukkig zijn}\\

\haiku{Oh, zij is slechts eene,.....}{boerinne een nederig}{wezen op aarde}\\

\haiku{Het is een handel,.}{waarin ik vroeger niet}{onervaren was}\\

\haiku{Nevens hem op eene;}{tafel lagen groote boeken}{opeengestapeld}\\

\haiku{Ik zal ze op uwe,.}{vraag gaan halen indien gij}{het mocht verlangen}\\

\haiku{dit alles is goed,;}{en wel en de boeken zijn}{met zorg geschreven}\\

\haiku{ik mijnen eigen.}{zoon gereed om zich in het}{verderf te storten}\\

\haiku{Het is in Frankrijk,,;}{te Parijs alleen dat ik}{kan en wil leven}\\

\haiku{{\textquoteleft}Verwacht hem heden,,{\textquoteright}.}{niet meer Celesta zeide}{de oude dame}\\

\haiku{En nochtans, aan die.}{noodlottige wreedheid kon}{hij niet ontsnappen}\\

\haiku{Tot nu toe heb ik.}{mij over die voorzienigheid}{niet te beklagen}\\

\haiku{Dom genoeg om ten!}{minste in de waarheid uwer}{vriendschap te gelooven}\\

\haiku{Welnu, zeg, dat gij,{\textquoteright} {\textquoteleft}}{niet van verveling op den}{Wulfhof wilt sterven}\\

\haiku{De grijsaard stapte;}{ter zaal in en ging langzaam}{rondom de tafel}\\

\haiku{In de duisternis;}{trapte hij op brokken en}{scherven van flesschen}\\

\haiku{maar, God zij er om,,!}{gezegend Dani\"el gij}{zijt zooverre nog niet}\\

\haiku{{\textquoteright} morde Gumbert, {\textquoteleft}dit.}{moet een slimme vogel of}{een dommerik zijn}\\

\haiku{{\textquoteright} {\textquoteleft}Bah, bah, waarom iets,?}{verbergen dat gansch gewoon}{en natuurlijk is}\\

\haiku{{\textquoteleft}Kom in de keuken,{\textquoteright},{\textquoteright},.}{zeide zei en leg mij eens}{uit wat dit beteekent}\\

\haiku{M. Willibald zal;}{dezen morgen zeker ons}{komen bezoeken}\\

\haiku{Iedereen zal over;}{dit onuitlegbaar vertrek}{zich verwonderen}\\

\haiku{gij begrijpt immers,?}{dat het afscheid hem te zeer}{zou ontsteld hebben}\\

\haiku{en, wanneer ik mijn,:}{hart te rade ga dan durf}{ik er bijvoegen}\\

\haiku{De notaris trok:}{den rentmeester een weinig}{ter zijde en sprak}\\

\haiku{{\textquoteleft}Ik zie ginder eene,.}{koets die in volle vaart naar}{hier komt gereden}\\

\haiku{Ga daarna in de.}{keuken en eet metterhaast}{insgelijks een stuk}\\

\haiku{in den stal hoorde;}{ik hem zingen van geluk}{en tevredenheid}\\

\haiku{In die gedachte.}{vertrok ik tegen den avond}{weder naar Brussel}\\

\haiku{{\textquoteright} {\textquoteleft}Wacht eens wat,{\textquoteright} zeide,.}{Katrien hare gezellin}{wederhoudende}\\

\haiku{{\textquoteright} Judocus verschoot,.}{en nog rooder werden hem}{wangen en voorhoofd}\\

\haiku{{\textquoteright} {\textquoteleft}En dat de Wulfhof?}{onzen jonkheer Dani\"el}{niet meer toebehoort}\\

\haiku{Och, Judocus, hij,!}{zal vernemen dat gij te}{veel gesproken hebt}\\

\haiku{Zij heeft wel gelijk,.}{te denken dat ik hare}{hulp zal weigeren}\\

\haiku{{\textquoteright} Hij trok zijn uurwerk,,,:}{uit en met het oog er op}{gevestigd sprak hij}\\

\haiku{Gumbert, mijn trouwe,,?}{gezel mijn boezemvriend mijn}{verkleefde broeder}\\

\haiku{271.)om, ging wankelend.}{tot eenen stoel en liet zich er}{op nedervallen}\\

\haiku{Alles, wat gij in,.}{uwe milde jeugd hebt gedroomd}{gaat waarheid worden}\\

\haiku{de verschrikte maagd,.}{was bleek en tranen rolden}{van hare wangen}\\

\subsection{Uit: Volledige werken 19. Geld en adel}

\haiku{Er liggen wel drie.}{kruiwagenvrachten er van}{op zijnen zolder}\\

\haiku{Sedert eenigen tijd,;}{hebben wij het nog al druk}{gehad inderdaad}\\

\haiku{Laat ons al deze.}{bedroevende gedachten}{ter zijde stellen}\\

\haiku{{\textquoteleft}Het zijn de jonge,{\textquoteright}.}{heeren uit den Gulden Arend}{zeide Jan Wouters}\\

\haiku{{\textquoteright} {\textquoteleft}De arme jongen,{\textquoteright}.}{kent geene beenen meer voegde de}{weduwe er bij}\\

\haiku{Sta mij maar zoo dom,!}{niet aan te kijken en reik}{mij mijne schoenen}\\

\haiku{{\textquoteleft}Is het zoo, ik dank,,{\textquoteright}.}{u uiterharte goede}{man mompelde hij}\\

\haiku{Gaat en vraagt het in.}{den Gulden Arend aan den baas}{en zijne dochters}\\

\haiku{{\textquoteright} {\textquoteleft}Ziet gij wel, moeder,{\textquoteright}, {\textquoteleft},?}{riep Linadat hij het nog}{niet heeft vergeten}\\

\haiku{- Begeef u op het.}{kantoor en deel den overste}{deze zaak mede}\\

\haiku{van zaken, goede -,.}{heer van Overburg mijn vriend zou}{ik durven zeggen}\\

\haiku{Kom morgen terug,.}{dan zal ik u mijn besluit}{te kennen geven}\\

\haiku{De edelman zag hem,.}{verbaasd aan als hadde hij}{hem niet begrepen}\\

\haiku{Indien gij in dit,{\textquoteright}, {\textquoteleft}}{huwelijk toestemt hernam}{Steenvlietleen ik u}\\

\haiku{Mijne dochter zal.}{waarschijnlijk uw voorstel met}{blijdschap vernemen}\\

\haiku{Mijnheer,{\textquoteright} kondigde, {\textquoteleft}.}{hij aanuw zoon is daareven}{te huis gekomen}\\

\haiku{{\textquoteleft}Het is de eerste,,;}{maal niet dat ik ze u zou}{aanwijzen vader}\\

\haiku{Ik moet bekennen,:}{dat ik werkelijk verre}{beneden hen sta}\\

\haiku{maar ik durf er niet,.}{voor instaan dat ik mijn woord}{zal kunnen houden}\\

\haiku{Is het Clemence,?}{van Overburg die men mij tot}{bruid wil voorstellen}\\

\haiku{Opspringende, vroeg:}{zij met eene uitdrukking van}{angst op het gelaat}\\

\haiku{In Brussel ging ik.}{onzen rijken vriend van den}{Kruisboom bezoeken}\\

\haiku{Die burgers mogen.}{nijverheid uitoefenen}{en handel drijven}\\

\haiku{Integendeel, gij,.}{zult mij helpen oprecht en}{zonder aarzeling}\\

\haiku{Deze samenspraak,.}{nam eene zeer ongunstige}{wending meende hij}\\

\haiku{Moeder, moeder, gij?}{hebt u ten mijnen koste}{willen vermaken}\\

\haiku{{\textquoteleft}Ik ben aan de Bank.}{tweehonderdvijftig duizend}{franken verschuldigd}\\

\haiku{{\textquoteright} {\textquoteleft}Het denkbeeld van zijn,.}{eerste bezoek verschrikt mij}{inderdaad vader}\\

\haiku{ik zal mij beleefd,.....}{en minzaam voor hem toonen}{zooveel mogelijk}\\

\haiku{{\textquoteright} {\textquoteleft}Nu, goede lieden,{\textquoteright},, {\textquoteleft}}{zeide de jongeling zich}{naar de deur keerende}\\

\haiku{{\textquoteright} En Herman Steenvliet.}{stapte door den voorhof en}{in den aardeweg}\\

\haiku{Jan Wouters wilde.}{Herman den boomgaard en den}{groenselhof toonen}\\

\haiku{Maar daar voelde hij,.}{eensklaps dat iemand hem op}{den schouder klopte}\\

\haiku{{\textquoteright} {\textquoteleft}O, mijnheer, heb toch!}{medelijden met mij en}{mijne kinderen}\\

\haiku{Niemand sprak nog een.}{woord en allen schenen min}{of meer verlegen}\\

\haiku{{\textquoteleft}De rijtuigen van,.}{M. den graaf van M. den}{markies en van Mev}\\

\haiku{{\textquoteleft}Het rijtuig van M.!}{den baron van Moorsbeke}{is ingespannen}\\

\haiku{{\textquoteright} De jongeling, die}{wel gevoelde dat het nu}{geen oogenblik was}\\

\haiku{gezondheid, die ons,{\textquoteright}.}{angstig maakte voegde de}{weduwe er bij}\\

\haiku{Nu begin ik eerst.}{goed te begrijpen waarvan}{men ons beschuldigt}\\

\haiku{Herman Steenliet zal.}{na dezen dag geenen voet meer}{in ons huis zetten}\\

\haiku{Wees zeker, kind, van,,;}{al wat men in het dorp zegt}{geloof ik niets niets}\\

\haiku{{\textquoteleft}Lina, belooft gij,,?}{de waarheid mij te zeggen}{geheel de waarheid}\\

\haiku{In \'e\'en woord, zij rooven.}{onze eer en bevlekken}{onzen goeden naam}\\

\haiku{Nauwelijks durf ik,.}{u openbaren wat men van}{haar zegt en gelooft}\\

\haiku{maar ik ben te vast.}{overtuigd dat uwe nieuwe gril}{geenen stand zal houden}\\

\haiku{{\textquoteright} En deze woorden,.}{murmelende ging Herman}{uit het kabinet}\\

\haiku{{\textquoteright} {\textquoteleft}En blijkt er uit de,?}{woorden zijns briefs dat hij even}{gunstig gestemd blijft}\\

\haiku{Zijne vroegere.}{kameraden ontmoeten}{hem  nergens meer}\\

\haiku{Ik geloof zulks niet,;}{en zou in alle geval}{het niet goedkeuren}\\

\haiku{Wij zullen dan in.....}{zijne tegenwoordigheid}{alles regelen}\\

\haiku{{\textquoteleft}Eilaas, ja, mijnheer,{\textquoteright}, {\textquoteleft}.}{was het antwoordik ben er}{nog diep van ontsteld}\\

\haiku{Zoohaast ik eene vaste,.}{verblijfplaats heb gevonden}{zal ik u schrijven}\\

\haiku{Het antwoord, dat uit,;}{hare gedachte oprees}{moest niet gunstig zijn}\\

\haiku{{\textquoteleft}Ik mag hem niet meer,.....}{wederzien en ik verlang}{niet hem nog te zien}\\

\haiku{Bedien  haar maar,,.}{allereerst bazin opdat}{ze spoedig wegga}\\

\haiku{men heeft mij dezen;}{morgen nog met slijken steenen}{uit het dorp gejaagd}\\

\haiku{{\textquoteleft}Kom eens hier bij mij,,{\textquoteright}, {\textquoteleft}.}{Clemence zeide hijik}{moet u iets vragen}\\

\haiku{hij maakte zich los,:}{uit hare armen terwijl}{hij somber gromde}\\

\haiku{Ach, ik ga afstand,;}{doen van mijne geboorte}{van mijnen adelstand}\\

\haiku{Houd op, Clemence,,.}{mijn toorn is wettig ik ben}{onverbiddelijk}\\

\haiku{Zij liep de kamer,.}{uit en zag den markies in}{den gang verschijnen}\\

\haiku{{\textquoteright} riep deze, zich nog, {\textquoteleft},!}{naar de zaal omkeerendeneen}{ik ken u niet meer}\\

\haiku{{\textquoteright} {\textquoteleft}Het schijnt echter, mijn,.}{jongen dat deze verre}{reis u niet toelacht}\\

\haiku{Niet waar, markies, het,?}{is om die reden alleen}{dat gij ons misprijst}\\

\haiku{Weldra werd hare.}{aandacht afgekeerd door eenig}{gerucht in den stal}\\

\haiku{Uwe eenvoudige,, -?}{goedheid de zuiverheid uws}{harten wat weet ik}\\

\subsection{Uit: Volledige werken 20. Het ijzeren graf}

\haiku{Het nederige.}{kerkje verheft zich boven}{den akker des doods}\\

\haiku{bloemen wiegelen;}{in den zoelen zuiderwind}{boven de graven}\\

\haiku{het is er alle;}{dagen Zondag en men speelt}{en zingt er altijd}\\

\haiku{{\textquoteright} {\textquoteleft}Och, onnoozel Mieken,,.}{het is de kluizenaar die}{ze daar altijd plant}\\

\haiku{En vertel maar niet!}{al te veel fabelen van}{het ijzeren graf}\\

\haiku{Eene stem, die mijnen,.}{naam noemde achter mij deed}{mij het hoofd omkeeren}\\

\haiku{het stomme kind was,.....}{niemand anders dan ik zelf}{die nu tot u spreek}\\

\haiku{hare oogjes blauw;}{en diep als de hemel op}{eenen helderen dag}\\

\haiku{al mijne leden,.}{wrongen zich krampachtig mijn}{aangezicht werd blauw}\\

\haiku{En als gij spreken,.}{kunt dan zal ik u nog veel}{schooner dingen geven}\\

\haiku{{\textquoteright} Het goede kind had;}{voorwaar geen ander inzicht}{dan mij te troosten}\\

\haiku{Gij begrijpt, dat wij;}{dit jaar op het kasteel niet}{zullen verblijven}\\

\haiku{Ik legde mij de.}{handen voor de oogen om haar}{niet te zien heengaan}\\

\haiku{Arme Leo,{\textquoteright} zeide, {\textquoteleft}.}{het goedhartig kindgij moogt}{daarom niet krijschen}\\

\haiku{Ik liep in \'e\'enen;}{adem door het dorp en in de}{dreve des kasteels}\\

\haiku{alles sterk en fraai,.}{gemaakt van fijn Engelsch staal}{zooals mijn vader zegt}\\

\haiku{Dan daalt in mij de,,;}{overtuiging dat ik verdrink}{dat ik ga sterven}\\

\haiku{Er was vooruitzicht;}{en edelmoedigheid in die}{kinderlijke scherts}\\

\haiku{Na eene lange wijl:}{trad M. Pavelyn terug}{in de school en sprak}\\

\haiku{, en zelfs in mijnen.}{slaap ontvielen mij dikwijls}{bittere tranen}\\

\haiku{en daarenboven,,,.}{hare stem hoewel fijn en}{zuiver was zeer zwak}\\

\haiku{ik zou ze nooit meer;}{zien zooals ze onverpoosd mij}{voor den geest zweefde}\\

\haiku{Ik was opgestaan.}{en had uit eerbied eenen stap}{ter zijde gedaan}\\

\haiku{{\textquoteleft}Kom aan, zeg ons toch,?}{wat is uwe gedachte over}{Leo's eerste werk}\\

\haiku{Een dezer heeren.}{was een man van fijn gevoel}{en diepe kennis}\\

\haiku{Ik was verheugd, dat;}{ik eene reden vond om mij}{neder te zetten}\\

\haiku{Ik zeide, dat ik;}{onpasselijk was en nu}{geenen lust tot eten had}\\

\haiku{indien ik iets noodig,.}{had zou ik kloppen om haar}{te verwittigen}\\

\haiku{de tranen rolden.}{in stilte als parelen}{over hare wangen}\\

\haiku{{\textquoteright} En hare stem nog,:}{meer bedwingende suisde}{zij schier onhoorbaar}\\

\haiku{{\textquoteright} Mij zwol de borst van,:}{moed mij popelde het hart}{van levensblijheid}\\

\haiku{Hij schudde mij nog.}{de hand met kracht en daalde}{dan de trappen af}\\

\haiku{de tranen zijn eene,.}{klacht een gebed om hulp of}{om medelijden}\\

\haiku{Hij hief mij met eene:}{korte beweging van den}{grond op en zeide}\\

\haiku{Tegen den avond had.}{ik het engelenhoofd schier}{geheel afgewerkt}\\

\haiku{Mijn werk was dus te.}{tenger van vormen en te}{mager van lijnen}\\

\haiku{maar hij liet mij den.}{tijd niet om uit te drukken}{wat ik gevoelde}\\

\haiku{{\textquoteright} {\textquoteleft}En indien ik u,?}{zeide dat ik de maker}{van dit beeld niet ben}\\

\haiku{maar ik wil mij de.}{verdiensten van anderen}{niet aanmatigen}\\

\haiku{ik deed geweld om.}{mijne smart te bedwingen}{en hief het hoofd op}\\

\haiku{{\textquoteright} {\textquoteleft}Neen, neen, wees gerust,,{\textquoteright}.}{Leo viel zij met eenen glimlach}{mij in de rede}\\

\haiku{Er moesten door Rosa's,;}{geest gepeinzen vlotten die}{zij niet uitdrukte}\\

\haiku{Er is iets, dat u,?}{bedroeft en gij weigert mij}{mijn deel van uwe smart}\\

\subsection{Uit: Volledige werken 21. De baanwachter. Gerechtigheid van Hertog Karel}

\haiku{Ongetwijfeld was;}{de baanwachter een getrouwd}{man met kinderen}\\

\haiku{Bij de blinde vrouw,:}{teruggekeerd antwoordde}{hij op hare vraag}\\

\haiku{het plantsoen onzer;}{koolen en onzer salade}{komt uit zijnen hof}\\

\haiku{Nu bemerkte de.}{jongen zijne grootmoeder}{onder het pri\"eel}\\

\haiku{Er liggen er nog,.}{wel acht of tien op mijne}{telloor ik voel het}\\

\haiku{Mie-Wanna hield zich.}{bezig met de kousen van}{Sander te stoppen}\\

\haiku{De  hagelsteenen.....}{sloegen zijne handen en}{wangen ten bloede}\\

\haiku{Bedwing uwe tranen,.}{en stapt recht vooruit nevens}{het spoor Mie-Wanna}\\

\haiku{, blijf toch rustig en,{\textquoteright}.}{doe de kinderen zoo niet}{schrikken zeide hij}\\

\haiku{Jan Verhelst zat op.}{zijne bank met de handen}{v\'o\'or het aangezicht}\\

\haiku{want nu en dan keek;}{hij op en luisterde met}{overspannen aandacht}\\

\haiku{Iedereen acht mij,.}{schuldig iedereen haat en}{vermaledijdt mij}\\

\haiku{Geef mij tijding van,!}{hen en ik zegen uwen naam}{tot op mijn doodbed}\\

\haiku{{\textquoteright} {\textquoteleft}Ach, Meken lief, ik,!}{zou het zoo gaarne gelooven}{het is toch zoo schoon}\\

\haiku{zelfs niet toelaten,!}{dat wij vernemen of hij}{gezond of ziek is}\\

\haiku{Zijn de menschen wreed,}{en onrechtvaardig jegens}{uwen armen vader}\\

\haiku{{\textquoteright} {\textquoteleft}Ja, mijnheer, ik ben.}{in de stad geboren en}{heb er lang gewoond}\\

\haiku{{\textquoteleft}Vader, hier is de,,.}{heer Masmans onze vriend die}{u komt bezoeken}\\

\haiku{Wat gaat gij doen met?}{uwe arme kinderen en}{uwe blinde moeder}\\

\haiku{Hij vatte weder:}{hare hand en zeide met}{aangrijpende ernst}\\

\haiku{maar Barbeltje moet.}{gij naar mijne zuster te}{Vilvoorden dragen}\\

\haiku{Ik heb slecht jegens,;}{u gehandeld omdat ik}{u schuldig waande}\\

\haiku{Integendeel, ik.}{had medelijden met uw}{ijselijk verdriet}\\

\haiku{Gij hebt er telkens.}{vermaak in gehad en er}{mede gelachen}\\

\haiku{Mijne moeder, mijn,;}{broeder en ik wij toonen}{ons onderdanig}\\

\haiku{De boter is schier,;}{in prijs verdubbeld en niet}{alleen de boter}\\

\haiku{Zou er eindelijk?}{recht te verhopen zijn voor}{den minderen man}\\

\haiku{hem scheen, dat hij den.....}{grond onder zijne voeten}{voelde daveren}\\

\haiku{maar eensklaps heft hij:}{de handen in de hoogte}{en roept kermend uit}\\

\haiku{Zeg mij eens, gij die,}{jong zijt en geenen last hebt te}{dragen waarom rijdt}\\

\haiku{Gij hebt een hof, eenen,,.}{burg in deze omstreken}{heeft men mij gezegd}\\

\haiku{{\textquoteleft}Ik woon te Brugge,,.}{in het hof van Uutkerke}{bij mijne moeder}\\

\haiku{Dan zal ik eens te;}{Hersberge aan de poort van}{uwen burg gaan kloppen}\\

\haiku{en mijn verlangen,,.....}{is hoort gij wel dat de prijs}{gewonnen worde}\\

\haiku{Geeft acht, heeren, ik!}{zal zien wie onder u de}{beste ruiters zijn}\\

\haiku{{\textquoteleft}Daar, Liedekerke,,{\textquoteright}.}{berg dit in uwe gordeltasch}{zeide de Hertog}\\

\haiku{Ik wil intusschen.}{hier wat wandelen en de}{frischheid genieten}\\

\haiku{Gij, Liedekerke,.}{ga uit het bosch en poog de}{jacht in te halen}\\

\haiku{Meent gij dan waarlijk,?}{dat ik hier gekomen ben}{om u kwaad te doen}\\

\haiku{De smart doet u het,,{\textquoteright}.}{kwaad overdrijven goede man}{zeide de Hertog}\\

\haiku{Ik heb mij voor den,,.}{schout aangeboden heer en}{hem mijn ramp geklaagd}\\

\haiku{zijne schildknapen,;}{niet spreken want hij zag er}{zeer verbolgen uit}\\

\haiku{{\textquoteleft}Zoo, gij weet dus wat?}{mij gisteren op de jacht}{is wedervaren}\\

\haiku{Er behoefden niet.}{min dan vier dienaars om het}{haar te ontnemen}\\

\haiku{Men zal toch wel zien,.}{dat zij anders u geheel}{onverschillig is}\\

\haiku{Wel zeker hebt gij.}{eenige dienaars op wie gij}{vast moogt vertrouwen}\\

\haiku{{\textquoteleft}Welnu, Martijn,{\textquoteright} vroeg, {\textquoteleft}?}{de vorstzal meester Antoon}{niet haast gaan komen}\\

\haiku{{\textquoteright} {\textquoteleft}Het is wel, geef mij;}{mijnen tabbaard en zet mij}{mijne kaproen op}\\

\haiku{{\textquoteleft}Is alle gevoel?}{van rechtvaardigheid dan in}{uwen geest gestorven}\\

\haiku{de jonkheer zal mijn.}{tweede vonnis aanvaarden}{en dus niet sterven}\\

\haiku{In alle geval.}{zal hij eindigen met zich}{te onderwerpen}\\

\haiku{{\textquoteright} antwoordde zij, op,.}{den hofmeester wijzende}{die bij de deur stond}\\

\haiku{De jonkheer Van der.}{Hameide heeft dus tweemaal}{het leven verbeurd}\\

\haiku{Gij zoudt u evenwel.}{in uwe verwachting kunnen}{bedrogen vinden}\\

\haiku{De schout van Brugge.}{verscheen welhaast in zijne}{tegenwoordigheid}\\

\haiku{Welk vonnis meent gij,?}{dat de Schepenbank tegen}{hem zal uitspreken}\\

\haiku{de opoffering,,.}{van geld hoe aanzienlijk ook}{schrikte hen niet af}\\

\haiku{{\textquoteleft}O, mijn God, indien!}{de Schepenbank hem tot den}{dood veroordeelde}\\

\haiku{Om Gods wil, heer schout,!}{folter eene arme moeder}{zoo wreedelijk niet}\\

\haiku{Er is een middel,,.....}{een onfeilbaar middel om}{Walter te redden}\\

\haiku{eene andere keus.}{dan de onderwerping is}{u niet gelaten}\\

\haiku{{\textquoteright} {\textquoteleft}Wij koesteren geene,,{\textquoteright}.}{wraakzucht eerwaarde vader}{antwoordde Thomas}\\

\haiku{Zij naderde met:}{hare dochter en zeide}{op smeekenden toon}\\

\haiku{{\textquoteleft}Vader, vader, ach,!}{heb medelijden met die}{arme edelvrouwen}\\

\haiku{{\textquoteleft}Ja, eerwaarde,{\textquoteright} was, {\textquoteleft}.}{het antwoordmaar  niemand}{mag hem naderen}\\

\haiku{{\textquoteright} {\textquoteleft}Het doet mij leed, voor,.}{uw gebed te moeten doof}{blijven eerwaarde}\\

\haiku{Nu scheen het hun toe,.}{dat zij geen hinderpaal meer}{konden ontmoeten}\\

\haiku{{\textquoteleft}Staat op, Mevrouw, en,,{\textquoteright}.}{gij goede lieden zeide}{de vorst zeer minzaam}\\

\haiku{De voorwaarde heeft,.}{onze geduchte vorst u}{zelf gesteld mijn zoon}\\

\subsection{Uit: Volledige werken 22. De loteling. Bavo en Lieveken}

\haiku{In het eene woonde;}{eene arme weduwe met}{hare  dochter}\\

\haiku{Maar op eens kwam men.}{van de leemen hutjes den tol}{des bloeds afeischen}\\

\haiku{{\textquoteright} Een onmerkbare;}{glimlach zweefde tusschen de}{tranen der maagd}\\

\haiku{Op die heilige,}{plaats waar elken dag iemand}{van hen den goeden}\\

\haiku{O, ik zal alle}{dagen een gebed lezen}{voor uwen heiligen}\\

\haiku{zij was bezig met -:}{het papier af te likken}{en riep half verstoord}\\

\haiku{Ik neem de pen in.}{de hand om te vernemen}{naar den staat van UL}\\

\haiku{Daarom, beminde,,.}{ouders als ge kunt zendt mij}{toch een beetje geld}\\

\haiku{maar ik heb hem, woord,.}{voor woord gezegd wat hij er}{in zetten moest}\\

\haiku{doch, daar men Fransch of,,;}{Waalsch sprak verstond Trien niet wat}{men over haar zeide}\\

\haiku{zijne glimmende;}{knevels waren met zwart was}{omhoog gestreken}\\

\haiku{{\textquoteleft}Och, goede vriend, gij,;}{moest bij mij blijven zitten}{als het u belieft}\\

\haiku{Zij voelen aan uwe,.}{kleederen om te weten}{van wat streek gij zijt}\\

\haiku{{\textquoteright}   Het meisje lag (.}{te weenen tegen de borst}{des jongelingsbladz}\\

\haiku{Laat ons liever in,;}{het donker hoeksken op de}{bank gaan zitten Jan}\\

\haiku{Ik heb er wel twee;}{uren voor dood van gelegen}{in eenen eikenkant}\\

\haiku{{\textquoteright} Ach, ik weet het wel,,{\textquoteright}, {\textquoteleft}?}{mevrouw zuchtte Trienwaar heb}{ik het toch verdiend}\\

\haiku{Welnu, beloof mij,;}{dat gij voor mij niets meer zult}{zijn dan eene zuster}\\

\haiku{op haar hoofd stond eene.}{zilveren kroon van zeven}{blinkende starren}\\

\haiku{- Laat ons nu maar wat,;}{beter voortstappen voordat}{de zon omhoog ga}\\

\haiku{Ik geloof, dat wij.}{verloren geloopen zijn}{met dit vertellen}\\

\haiku{Trien kwam bij hem staan.}{en schudde het water van}{hare kleederen}\\

\haiku{het zalig uitzicht,.}{was verdwenen de hoop van}{geluk was vergaan}\\

\haiku{{\textquoteright} De jongeling deed:}{geweld om te ademen en}{antwoordde zuchtend}\\

\haiku{en, de hand  der,:}{maagd vattende antwoordde}{hij met groote koelheid}\\

\haiku{Hij bond den kranke:}{onmiddellijk het lichtscherm}{voor de oogen en vroeg}\\

\haiku{Zeg aan de meid, dat.}{zij insgelijks eten brenge}{voor deze dochter}\\

\haiku{de kleine kruiden;}{sluiten hunne bloemkelken}{en bladeren toe}\\

\haiku{Hij scheen treurig en.}{bleef eene wijl met gebogen}{hoofde overwegen}\\

\haiku{Licht, waardigheid en!}{plichtbesef in het hart der}{moeders van het volk}\\

\haiku{Maar de vrouw, na eene,:}{lange poging mompelde}{met moedeloosheid}\\

\haiku{Zij wist het niet, en.}{evenwel dankte zij God uit}{den grond des harten}\\

\haiku{Op dit oogenblik,.}{werd de deur geopend en}{een man trad binnen}\\

\haiku{ik heb hem beloofd,,.}{dat ik zou komen indien}{het mogelijk was}\\

\haiku{{\textquoteright} herhaalden de twee,.}{zusterkens lachend in de}{handen kletsende}\\

\haiku{{\textquoteleft}Ik weet niet, ik kan.}{in dat woeste leven geen}{vermaak meer vinden}\\

\haiku{misschien zelfs zou hij.}{niet meer of zeer zelden met}{haar kunnen spelen}\\

\haiku{Inderdaad, vader.}{Wildenslag was een vijand}{van het onderwijs}\\

\haiku{Meer dan eens reeds had;}{hij met zijne vrouw over zijn}{inzicht gesproken}\\

\haiku{ik wil wel naar het,}{kantwerkhuis gaan ik zal er}{mijn best doen zooveel}\\

\haiku{Is het geluk van?}{haar kind de hoogste vreugde}{eener moeder niet}\\

\haiku{{\textquoteright} {\textquoteleft}Ja, maar zou zij dan?}{hare arme ouders wel}{blijven beminnen}\\

\haiku{mijn geluk, mijne.}{welvaart ben ik verschuldigd}{aan mijne moeder}\\

\haiku{In het eerst hadden;}{zij elkander getroost met}{de hoop op goed nieuws}\\

\haiku{Ik heb al moeite;}{gedaan om de kleinste naar}{de school te krijgen}\\

\haiku{zij was arm in de,.}{wereld en moest een ambacht}{leeren dit wist zij wel}\\

\haiku{Ja, ja, zeker, ik.}{neem uw edelmoedig voorstel}{aan uit gansch mijn hart}\\

\haiku{maar zoohaast er weder,.}{veel werk in Gent was zouden}{zij terugkomen}\\

\haiku{Dan werd het hem, als.}{hadde men met geweld iets}{uit zijn hart gerukt}\\

\haiku{somtijds stond hij op.}{en ging naar de  deur bij}{het minste gerucht}\\

\haiku{{\textquoteright} {\textquoteleft}Inderdaad, Adriaan,,.}{het is natuurlijk hij zal}{u gelukwenschen}\\

\haiku{{\textquoteright} {\textquoteleft}Ach, het is mijne,,{\textquoteright}.}{vrouw mijnheer antwoordde de}{ontroerde werkman}\\

\haiku{M. Raemdonck zou,.}{u goed willen doen indien}{het mogelijk is}\\

\haiku{{\textquoteright} ~ Bavo liet den:}{brief ter tafel vallen en}{begon te weenen}\\

\haiku{Maanden lang wachtte;}{hij op een tweede antwoord}{van Godelieve}\\

\haiku{Volgens zijn zeggen,;}{wonnen de Wildenslags veel}{geld ja veel te veel}\\

\haiku{Daar trad onverwachts,:}{een man in de kamer naar}{buiten roepende}\\

\haiku{Hij verborg hem zelfs,.}{niet dat hij het deed met een}{bijzonder inzicht}\\

\haiku{Dit is tamelijk.}{veel voor een jongeling van}{twee\"entwintig jaar}\\

\haiku{het zou mij dienen.}{om mijn vermaak een beetje}{af te wisselen}\\

\haiku{Het was ook wel wat;}{de meisjes eerst en daarna}{de ouders deden}\\

\haiku{maar vrouw Damhout zag,.}{wel dat iets anders zijnen}{geest benevelde}\\

\haiku{vreest gij nu, dat de?}{Damhouts ons gebed zouden}{kunnen verstooten}\\

\haiku{O, mijne schoone,!}{kindsheid hoe tergend ontstaat}{gij voor mijne oogen}\\

\haiku{en indien zij mijn,?}{gebed verwerpen omdat}{gij niet met mij zijt}\\

\haiku{dan zal ik alle,.}{schaamte alle gevoel in}{mijn hart versmachten}\\

\haiku{Gij hebt haar naar eene,;}{fabriek gezonden om geld}{uit haar te trekken}\\

\haiku{Mij ook beschaamde,;}{de poging welke ik bij}{u moest beproeven}\\

\haiku{hij liet zich op eenen:}{stoel zakken en zeide met}{schijnbare kalmte}\\

\haiku{Vergun mij als eene,.}{genade voor heden dit}{huis te verlaten}\\

\haiku{De jongeling stak,;}{de handen uit als om haar}{te wederhouden}\\

\subsection{Uit: Volledige werken 23. De jonge dokter}

\haiku{In deze kamer.}{woonde dus een student in}{de medicijnen}\\

\haiku{er was beweging.}{en gerucht van menschen in}{de straat gekomen}\\

\haiku{Gij erkent dus, dat?}{uwe gezondheid gevaar loopt}{van te bezwijken}\\

\haiku{hij lachte met de;}{anderen en sprak soms ook}{al een vroolijk woord}\\

\haiku{{\textquoteright} hem van de lippen,.}{viel wierp hij zich juichend aan}{den hals des grijsaards}\\

\haiku{{\textquoteleft}Daar, grootvader, drink,.}{dit glas het zal u deugd doen}{en u verkwikken}\\

\haiku{{\textquoteleft}kon ik gelooven wat,!}{uwe woorden mij voorspellen}{ik wierd zinneloos}\\

\haiku{Daar, grootvader, lees,{\textquoteright},.}{gij zelf sprak zij hem het schrift}{overhandigende}\\

\haiku{De voorzitter van;}{de Jury overlaadde mij}{met loftuitingen}\\

\haiku{De grootvader scheen.}{op dit oogenblik zich iets}{te herinneren}\\

\haiku{gedurende eene:}{lange wijl wist Adolf niet aan}{wien te antwoorden}\\

\haiku{Adelina alleen;}{hield zich meer ingetogen}{dan naar gewoonte}\\

\haiku{{\textquoteleft}Het is een man te,.}{paard die waarschijnlijk ook al}{om den dokter komt}\\

\haiku{{\textquoteright} Zij herkende dan.}{eerst de wachtende vrouw en}{stapte tot haar}\\

\haiku{{\textquoteright} Met droeven oogslag.}{wees de vrouw op het misvormd}{gelaat van haar kind}\\

\haiku{Zonder iemand den:}{tijd tot spreken te laten}{zeide de dokter}\\

\haiku{{\textquoteleft}Sa, vrienden, ik heb.}{slechts eenige oogenblikken}{om u te helpen}\\

\haiku{Ik kan nergens  ,!}{den voet zetten of men spreekt}{mij van Adolf Valkiers}\\

\haiku{Francisca heeft mij,.}{genoeg gezegd om mij het}{te doen vermoeden}\\

\haiku{Aan den knecht, die op,:}{het klinken der bel het hek}{kwam openen vroeg hij}\\

\haiku{Gij zult wel bemerkt,,?}{hebben zeker dat het niet}{zeer wel met mij gaat}\\

\haiku{Alzoo, Mijnheer, gij?}{zult den wintertijd in de}{stad gaan doorbrengen}\\

\haiku{Voorbij den molen,.}{gaande kwam hij weldra op}{eenen grooten steenweg}\\

\haiku{integendeel, er.}{speelde een stille glimlach}{op hare lippen}\\

\haiku{Ziet gij niet, moeder,?}{dat hij alle dagen meer}{en meer vermagert}\\

\haiku{Ik zal het u straks.}{beter en met meer vrijheid}{van geest bewijzen}\\

\haiku{{\textquoteleft}Adelina, kind, ik.}{weet niet welke macht uw woord}{op mij uitoefent}\\

\haiku{O, beroof mij niet,!}{van dien eenigen troost van die}{bron mijner sterkte}\\

\haiku{anderen zeiden,;}{dat het hem overkomen was}{door sterk te niezen}\\

\haiku{maar de zieke bleef.}{gedurig in eenen staat van}{halve bezwijmdheid}\\

\haiku{Ik eisch, dat er!}{oogenblikkelijk iemand}{worde geroepen}\\

\haiku{Wilt gij volstrekt uwe,}{nieuwe uitvinding op den}{pastoor beproeven}\\

\haiku{{\textquoteright} Bij het hooren van.}{Adolfs naam had Heuvels zich de}{leden gewrongen}\\

\haiku{{\textquoteright} {\textquoteleft}Inderdaad, maar het?}{middel om die vijfhonderd}{franken te vinden}\\

\haiku{indien het voor zijn,.}{geluk is moeten wij ons}{er over verheugen}\\

\haiku{Altijd, altijd zie,;}{ik hem onverpoosd waart zijn}{beeld voor mijne oogen}\\

\haiku{straks zal ik in den:}{hof wandelen en mij een}{beetje vermoeien}\\

\haiku{{\textquoteleft}Wel, wel,{\textquoteright} riep zij, {\textquoteleft}wie!}{hadde ooit zich aan dit nieuws}{kunnen verwachten}\\

\haiku{De handen tot een,:}{gebed samenvoegende}{smeekte Adelina}\\

\haiku{Gij zult mij helpen;}{in den strijd tegen mijne}{herinneringen}\\

\haiku{Waarom zou ik mij?}{laten wederhouden door}{eene zinnelooze hoop}\\

\haiku{Is de zucht om in,?}{eene groote stad te wonen nog}{nooit in u ontstaan}\\

\haiku{Daar ging ik nog het!}{bijzonder doel van mijn}{bezoek vergeten}\\

\haiku{De wind waaide uit,;}{het noordwesten en het was}{zeer koud en vochtig}\\

\haiku{en gij, meester, gij!}{verstoot meedoogenloos de}{bede van uw kind}\\

\haiku{{\textquoteright} De zieke zag de,.}{meid met verstoorden blik aan}{doch antwoordde niet}\\

\haiku{Langen tijd bleef de.}{volledigste stilte in}{de kamer heerschen}\\

\haiku{Gij poogt ons te doen,;}{gelooven dat zijne ziekte}{geen gevaar aanbiedt}\\

\haiku{Waar de hemel zou,,}{beslist hebben wat kan daar}{toch een geneesheer}\\

\haiku{maar de jongeling.}{behoefde voorwaar deze}{aanmoediging niet}\\

\haiku{{\textquoteright} kreet de meid, met de.}{handen opgeheven van}{de trappen springend}\\

\haiku{Het hart wilde van.}{zalig ongeduld mij in}{den boezem breken}\\

\haiku{{\textquoteleft}Hij toonde zooveel,.}{betrouwen niet als gij en}{scheen zelfs bekommerd}\\

\haiku{dat wij dezelfde?}{geneesmiddelen moeten}{blijven aanwenden}\\

\haiku{Het was voorwaar eene,}{openbaring des hemels die}{mij deed besluiten}\\

\haiku{De grootvader hield.}{de oogen biddend ten hemel}{opgeheven}\\

\subsection{Uit: Volledige werken 24. De burgers van Darlingen}

\haiku{Hunne goederen;}{bestonden in pachthoeven}{en landerijen}\\

\haiku{Een lakensch vestje,.}{grijpende begon zij er}{aan voort te naaien}\\

\haiku{{\textquoteright} {\textquoteleft}Eilaas, hoe is het,!}{toch mogelijk dat gij dus}{van uwe zuster spreekt}\\

\haiku{Kwas, over de zotte!}{kleederdracht mijner zuster}{Hermina spreken}\\

\haiku{De zaak is niet, te;}{weten of mijne zuster}{kwaad doet of kwaad meent}\\

\haiku{zooals het behoort, gij.}{zoudt redenen hebben om}{het te betreuren}\\

\haiku{De weenende vrouw,;}{sprong hem achterna om hem}{te wederhouden}\\

\haiku{Ik zal niet lang meer,,{\textquoteright}.}{hier blijven wonen mevrouw}{hernam de dienstmeid}\\

\haiku{{\textquoteleft}Daar, Sophie, daar.}{is een beetje geld op uwe}{huur van deze maand}\\

\haiku{Daarmede kan men.}{eene vrouw in het nieuw kleeden}{van hoofd tot voeten}\\

\haiku{Nauwelijks kon zij.}{eenige erkentelijke}{woorden stamelen}\\

\haiku{de gedachte van!}{het huwelijk vervult mij}{met eenen doodelijken schrik}\\

\haiku{{\textquoteleft}Hermina, mijne,,.}{arme Hermina schep moed}{ween niet zoo bitter}\\

\haiku{waarom M, Blondeel,}{zoo aandrong om Hermina}{te doen vertrekken}\\

\haiku{Is hunne eenige,}{studie niet hunnen armen}{pachters den laatsten}\\

\haiku{Theresia is eene.}{gezworene vijandin}{van het huwelijk}\\

\haiku{onderweg, mij dunkt,.}{ik hadde tranen gestort}{te midden der straat}\\

\haiku{maar wanneer men trouwt,.}{moet men toch een bestaan in}{de wereld hebben}\\

\haiku{hij was een ware,;}{Engelsch man doch hij had er}{niets bij verloren}\\

\haiku{dat gij, mijnheer, dat.}{uwe goede zuster mijnen}{wenschen gunstig waart}\\

\haiku{{\textquoteleft}Komt, komt, dit is toch;}{geene reden om het eten te}{laten koud worden}\\

\haiku{hij schijnt te vreezen,.....}{dat gij hem redenen tot}{gramschap zult geven}\\

\haiku{Ik voorzie, dat uwe.}{goestingen dezelfde niet}{zijn als de mijne}\\

\haiku{Uwe zuster schijnt mij.}{eene zoete inborst en een}{goed hart te hebben}\\

\haiku{{\textquoteright} Op dit oogenblik.}{hergalmde de straat van een}{plotseling gerucht}\\

\haiku{beiden gevoelen,.}{dat zij de reden zijn van}{elkanders verdriet}\\

\haiku{verkoop uw kind aan,.}{een grof mensch versleten naar}{ziel en naar lichaam}\\

\haiku{Het werd in den tuin,.}{stil en eenzaam als ware}{er niets geschied}\\

\haiku{{\textquoteright} {\textquoteleft}Ik weet niet tot wat,{\textquoteright}.}{te besluiten antwoordde}{Blondeel weifelend}\\

\haiku{De arme jongen;}{ligt nu bedolven in eene}{sombere wanhoop}\\

\haiku{De hemel gunne.}{mij krachten om tegen het}{verdriet op te staan}\\

\haiku{De vrederechter.}{en het tribunaal moeten}{er tusschenkomen}\\

\haiku{Dit zal mij toch niet.}{beletten hem aan te zien}{als ons beider kind}\\

\haiku{Hij zweeg eene wijl en.}{deed geweld om zich zei ven}{meester te worden}\\

\haiku{Dan zeide hij met:}{eenen scherpen grimlach en met}{beklemde gramschap}\\

\haiku{Ziehier, Romys, wat,,.}{ik om het kort te maken}{u te zeggen heb}\\

\haiku{Gij zijt volstrekt  ,.}{meester over uw kind evenals}{wij over ons fortuin}\\

\haiku{maar is uw fortuin?}{dan insgelijks niet het goed}{onzer familie}\\

\haiku{gij weet zoo goed als,.}{ik dat hij in het geheel}{geene familie heeft}\\

\haiku{Niettemin, al die;}{diamanten steken de oogen}{der aanschouwers uit}\\

\haiku{de meisjes tusschen.}{het volk murmelen woorden}{van bewondering}\\

\haiku{De bruidegom zegt,:}{met zoete blijde stem tot}{zijne echtgenoote}\\

\haiku{M. Pottewal en.}{zijne vrouw zitten aan het}{midden der tafel}\\

\haiku{zij houdt het hoofd recht.}{en toont een onbewogen}{en deftig gelaat}\\

\haiku{het is zichtbaar, dat.}{hij zijne ontsteltenis}{poogt te bedwingen}\\

\haiku{{\textquoteright} {\textquoteleft}Maar gij beeldt u dit,{\textquoteright}.}{alles in wedervoer de}{oude dame}\\

\haiku{M. Pottewal zal,?}{met den tocht van zes uren te}{huis komen niet waar}\\

\haiku{Eindelijk ging zij.}{ter kamer uit en stapte}{over eene opene plaats}\\

\haiku{Ik had eene goede;}{zaak ontmoet en tienduizend}{franken gewonnen}\\

\haiku{{\textquoteright} {\textquoteleft}Ik wacht, Theresia,,.}{totdat ik wete waarvan}{gij mij beschuldigt}\\

\haiku{Neen, bedrieger, er.}{staan wel andere dingen}{op uwe rekening}\\

\haiku{{\textquoteleft}Schier eene gansche maand!}{huishouden verzwolgen op}{eenen enkelen dag}\\

\haiku{Verwondert het u,?}{dat ik insgelijks zijne}{dame heb gegroet}\\

\haiku{Weet gij wat het eenig,?}{middel is om hier de rust}{te doen wederkeeren}\\

\haiku{Pottewal zette:}{zich nevens haar en wilde}{haar de hand nemen}\\

\haiku{Dan hief hij traaglijk:}{zijne oogen ten hemel en}{morde wanhopig}\\

\haiku{{\textquoteright} {\textquoteleft}De tweede maal liep,.}{ik hem schier tegen het lijf}{in de Melkstraat}\\

\haiku{Wat zij hem meldde,,;}{scheen hem noch buitengewoon}{noch wonderlijk}\\

\haiku{Hij deed een teeken,,;}{als toonde hij de meid die}{bij de wiege zat}\\

\haiku{Eerst was het een twist,.}{om te weten aan wie het}{wichtje wel geleek}\\

\haiku{M. De Cock zal mij.}{helpen om de wiege naar}{boven te dragen}\\

\haiku{Dit ontwerp schijnt slechts;}{uitgevonden om ons voor}{den zot te houden}\\

\haiku{{\textquoteright} {\textquoteleft}Mijn broeder Blondeel.}{verzekert nochtans dat het}{zal aanvaard worden}\\

\haiku{Schoon als zij is, zou.}{zij misschien een millioen}{gevonden hebben}\\

\haiku{God zij geloofd, ik{\textquoteright} {\textquoteleft},!}{zie u wel te pas.Wat schoon}{wat betooverend kind}\\

\haiku{zij had haren man,.}{gezien met het kind harer}{zuster op de knie}\\

\haiku{de blijdschap en de.}{opgeruimdheid waren weg}{van het gezelschap}\\

\haiku{{\textquoteleft}Inderdaad, ik heb,{\textquoteright}.}{het stoomtuig hooren fluiten}{bemerkte Romys}\\

\haiku{{\textquoteleft}Hadde ik mijn kind,.}{niet op den schoot ik vloog u}{daarvoor aan den hals}\\

\haiku{zij wierp hem de winst;}{en het geluk van Ernest}{in het aangezicht}\\

\haiku{Onderweg streelde,.}{zij den eenigen hond die was}{gespaard geworden}\\

\haiku{Hare oogen schenen,.}{te vlammen nochtans en ik}{was vervaard van haar}\\

\haiku{Een kreet van blijdschap,.}{ontvloog haar toen zij hare}{moeder bemerkte}\\

\haiku{de overmatige.}{blijdschap niet minder dan het}{overmatig verdriet}\\

\haiku{Theresia, ik heb.}{u arm gemaakt en den naam}{van uw kind onteerd}\\

\haiku{Heeft uw man slechte,.}{zaken gedaan het is voor}{zijne rekening}\\

\haiku{Ik liet u den naam;}{van mijnen echtgenoot met}{laster overladen}\\

\haiku{{\textquoteright} {\textquoteleft}Toe, onkel lief, van!}{Janneken en Mieken en}{van den suikerberg}\\

\haiku{{\textquoteright} {\textquoteleft}Het is inderdaad,,{\textquoteright}.}{zoo Hermina antwoordde}{Blondeel met fierheid}\\

\haiku{{\textquoteleft}Gij moet inzien, dat.}{ik geene vingeren heb van}{vijf en twintig jaar}\\

\haiku{Die muzikanten,.}{zij munten zeker niet uit}{door ootmoedigheid}\\

\haiku{gij zult niet zonder,.}{vrienden niet zonder hulp op}{de wereld blijven}\\

\haiku{{\textquoteleft}uwe goedheid kan den,,.}{noodlottigen slag die ons}{bedreigt niet afkeeren}\\

\haiku{Neen, neen, o hemel,!}{geen vlekje op het voorhoofd}{mijner Hermina}\\

\haiku{Het fortuin kan men;}{winnen en verliezen en}{nogmaals herwinnen}\\

\haiku{Dezen avond zult gij,.}{vertrekken naar Holland en}{van daar naar Londen}\\

\haiku{maar ik heb u geld,.}{geleend en gij hebt het mij}{teruggegeven}\\

\haiku{Ik ben schuldig, ten.}{minste aan eene zinnelooze}{onvoorzichtigheid}\\

\haiku{{\textquoteright} En zij ontvouwde.}{voor zijne oogen eenen ganschen}{bundel bankbriefjes}\\

\subsection{Uit: Volledige werken 26. Het Goudland}

\haiku{{\textquoteright} Een zucht ontsnapte,.}{den klerk en hij hief de oogen}{klagend ten hemel}\\

\haiku{Nog vier andere;}{schepen zendt de maatschappij}{naar Californi\"e}\\

\haiku{want een glimlach van.}{bewondering blonk als een}{licht op zijn gelaat}\\

\haiku{Eenige andere.}{booten zag men insgelijks}{op den stroom varen}\\

\haiku{Zou hij lamme beenen,?}{krijgen nu het beslissend}{uur gekomen is}\\

\haiku{gij moest mij maar eens,,.}{zeggen Mijnheeren wat ik}{hier in de hand heb}\\

\haiku{Het is Donatus,:}{toch niet die de eerste van}{honger zou sterven}\\

\haiku{Lage ik nu slechts!}{op onzen hooizolder te}{Natten-Haesdonck}\\

\haiku{Spaansche peper, ik,.}{ken het het dient om ezels op}{den loop te jagen}\\

\haiku{Wij zullen dien tijd.}{waarnemen om op alles}{orde te stellen}\\

\haiku{Tegen den middag.}{werden de reizigers op}{het dek geroepen}\\

\haiku{Niemand daaronder.}{verstond hem of betuigde}{hem eenige vriendschap}\\

\haiku{Daarenboven, voor.}{de minste reden stampt hij}{als een woedende}\\

\haiku{Gij wilt de Jonas?}{het lot bereiden van het}{Portugeesche schip}\\

\haiku{Evenwel, van dien dag.}{af aan had hij maar een slecht}{leven op het schip}\\

\haiku{Hij ziet altijd zuur,,;}{nochtans en het is hem te}{veel dat hij glimlacht}\\

\haiku{{\textquoteleft}Sedert het verlies.}{uwer banknoten ziet gij}{niets meer dan dieven}\\

\haiku{het was zichtbaar, dat.}{een heete wraakdorst hem in}{den boezem brandde}\\

\haiku{zijne tempels zijn,;}{de speelhuizen die gij hebt}{gezien of zult zien}\\

\haiku{Uit het antwoord bleek,;}{dat zulke verzending zeer}{gemakkelijk was}\\

\haiku{misschien zal hetgeen,.}{ik over hem schrijf hem nuttig}{zijn voor de toekomst}\\

\haiku{ik neem tien dollars;}{en zet ze te gelijk op}{\'e\'enen slag der kaart}\\

\haiku{De bediende liet.}{het zich geen tweemaal zeggen}{en vloog naar buiten}\\

\haiku{de revolvers en}{de andere messen heb}{ik gekocht voor u.}\\

\haiku{{\textquoteright} riep Jan, terwijl hij.}{zijne twee gezellen de}{deur uitduwde}\\

\haiku{Ik geloof, dat ik.}{er ten minste vier of vijf}{heb doodgeschoten}\\

\haiku{het is tevens de,.}{streek waar de wildemannen}{zich het meest toonen}\\

\haiku{wij zullen eiken.}{dag over onze aanstaande}{reis kunnen spreken}\\

\haiku{{\textquoteleft}Wie weet, of die sal......?}{sal die struikroovers niet reeds te}{San-Francisco zijn}\\

\haiku{vooruit dan..... en bij,!}{de minste vijandige}{beweging geeft vuur}\\

\haiku{maar richt uw oog naar.....}{alle kanten en houdt de}{geweren gereed}\\

\haiku{Van de schelmen, die,;}{ons hebben aangerand is}{niets meer te vreezen}\\

\haiku{Geen wonder dus, dat,;}{ik u gaarne zie omdat}{gij een muilezel zijt}\\

\haiku{{\textquoteleft}Ja, vriend Roozeman, het,{\textquoteright}.}{zijn verre uit de slimsten}{bemerkte Pardoes}\\

\haiku{{\textquoteleft}Het is geen half uur,.}{geleden dat ik u heb}{hooren aflossen}\\

\haiku{{\textquoteright} En toen men hem op,:}{een twintigtal stappen had}{gevolgd hernam hij}\\

\haiku{Wat de belooning,,;}{betreft die hij ons belooft}{vertrouwt daar niet op}\\

\haiku{de gereedschappen,.}{wie wil mij komt geen enkel}{stuk meer op den rug}\\

\haiku{Wij stelden ons te;}{weer en losten insgelijks}{onze vuurwapens}\\

\haiku{Dezen losten te;}{gelijk hunne geweren}{op de vijanden}\\

\haiku{maar Jan Creps had zich.}{vooruitgeworpen en den}{strop doorgesneden}\\

\haiku{{\textquoteleft}Langs het bed van dien.}{waterval zullen wij in}{de vlakte dalen}\\

\haiku{{\textquoteright} Zij bevonden zich.}{nu omtrent den winkel van}{eenen goudwisselaar}\\

\haiku{Achter dezen toog.}{stond een magere man met}{eenen bril op den neus}\\

\haiku{Gij meent, dat hij  ?}{die eenvoudige zwetsers}{niet heeft bedrogen}\\

\haiku{Zonder dit middel.}{zouden zij het hier niet lang}{kunnen volhouden}\\

\haiku{{\textquoteright} {\textquoteleft}Neen, voor eenen klomp goud,,!}{zoo groot als mijne vuist ga}{ik daar niet binnen}\\

\haiku{Toen hij vernam, dat,}{zij elk eenen grog gedronken}{hadden eischte}\\

\haiku{maar hier is geen draak.}{met zeven koppen noodig om}{venijn te spuwen}\\

\haiku{{\textquoteleft}Spoedig, mannen, geeft;}{mij nog een paar schoppen van}{dit roodachtig zand}\\

\haiku{{\textquoteright} In weinig tijds was.}{de koffie gekookt en de}{koeken gebakken}\\

\haiku{Zulke brokken goud!}{glinsteren er vele in}{den grond van den put}\\

\haiku{William stond voor.}{de deur en de moordenaar}{ongetwijfeld ook}\\

\haiku{Het was een galm, hol,.}{grof en ratelend als een}{lange donderslag}\\

\haiku{{\textquoteleft}Beer of geen beer, ik!}{zal het arme dier z\'o\'o niet}{laten vermoorden}\\

\haiku{{\textquoteleft}Vleesch is vleesch,?}{en indien het goed smaakt en}{niet schadelijk is}\\

\haiku{Ik betrouw mij niet,;}{op vrienden die beervleesch}{in hun lijf hebben}\\

\haiku{al wilden wij ze,.}{ontwijken wij zouden er}{niet in gelukken}\\

\haiku{een paar breede   ,.}{Tusschen zijne armen om}{hem te versmachten}\\

\haiku{Een van ons beiden!}{zal hier in de woestijn zijn}{gebeente laten}\\

\haiku{Kon ik mij het haar!}{tot het laatste pijltje maar}{uit den kop trekken}\\

\haiku{Het eenige, dat men,;}{nog hoorde was eene klacht over}{gebrek aan water}\\

\haiku{Wat hij hun zeide,;}{waren slechts uitvindsels om}{hun moed te geven}\\

\haiku{Dus deze beek is,.}{een teeken dat wij onzen}{placer naderen}\\

\haiku{Hij mompelde van,,:}{macht van eer van grootheid en}{scheen half dwaalzinnig}\\

\haiku{met door het water,.}{te gaan kunnen wij deze}{kloven bereiken}\\

\haiku{zullen wij dan elk?}{met een gewicht van tien pond}{aan den hals loopen}\\

\haiku{{\textquoteright} {\textquoteleft}Gij moet op voorhand.}{goed ademen en dan den mond}{gesloten houden}\\

\haiku{Victor was langer;}{dan de anderen onder}{water gebleven}\\

\haiku{Deze hoop gaf hun.}{moed en scheen hunne krachten}{te verdubbelen}\\

\haiku{Opdat het werk niet,,}{te zeer vertraagd wierd stelde}{hij voor van morgen}\\

\haiku{Hij naderde hem,:}{klopte hem op den schouder}{en zeide schertsend}\\

\haiku{{\textquoteleft}Voor al de schatten;}{der wereld zou Jan Creps hier}{niet meer wederkeeren}\\

\haiku{{\textquoteright} juichte de matroos,.}{zijnen mond naar het oor zijns}{makkers neigende}\\

\haiku{Deze teekenen.}{brachten hem tot aan den voet}{eener steile rots}\\

\haiku{hij knarsetandde,.}{en de oogen schenen hem uit}{het hoofd te komen}\\

\haiku{maar gij hebt te doen,.}{met magen die hunnen reuk}{verloren hebben}\\

\haiku{{\textquoteright} {\textquoteleft}Dit is te zeggen,{\textquoteright}, {\textquoteleft}.}{riep Donatusdat ik eens}{overvloedig zal eten}\\

\haiku{Het overige ei,.}{behoud ik voor mij om te}{weten hoe het smaakt}\\

\haiku{een mensch op zulke?}{onbarmhartige wijze}{uit enkel plezier}\\

\haiku{en misschien, misschien!}{zou hij hem de hand van zijn}{Anneken toestaan}\\

\haiku{Moest ik daarom naar?}{het vermaledijd land van}{Californi\"e gaan}\\

\haiku{de ziel, die leefde,!}{in haren blik zoo zuiver}{en zoo beminnend}\\

\haiku{{\textquoteleft}Madam, er is een,.}{man in den winkel die u}{volstrekt wil spreken}\\

\subsection{Uit: Volledige werken 28. Moeder Job. Een goed hart. Houten Clara}

\haiku{dan, na vervulling,:}{van den heiligen plicht zal}{het vreugde worden}\\

\haiku{8).daarom liet gij.}{de U uit nieuwe door uwe}{vingeren vallen}\\

\haiku{{\textquoteright} {\textquoteleft}Ik ben de beste,{\textquoteright}, {\textquoteleft}!}{schutter riep de brouweren}{iedereen weet het}\\

\haiku{Op de tien schoten!}{maar \'e\'ene roos en tweemaal}{buiten  het wit}\\

\haiku{Zij komt mij toe, en:}{ik zal ze door een ander}{moeten zien winnen}\\

\haiku{maar, zeg wat ge wilt,,.}{waar het geluk is  wil}{het zijn en blijven}\\

\haiku{{\textquoteright} {\textquoteleft}Eene dochter, die zal.....}{trouwen met Gabri\"el van}{onzen notaris}\\

\haiku{meesttijds hield hij het;}{hoofd gebogen en den blik}{ter aarde gericht}\\

\haiku{maar nu scheen een ver.}{gerucht zijne ooren te}{hebben getroffen}\\

\haiku{zijne lippekens,:}{verroerden en mij dacht dat}{het wilde zeggen}\\

\haiku{als men  het kind,.}{warm houdt dan zal de dokter}{het wel genezen}\\

\haiku{{\textquoteleft}Niemand weet het,{\textquoteright} was, {\textquoteleft},.}{het antwoorddan gij alleen}{misschien Rosina}\\

\haiku{reeds drie dagen..... zijn.....}{vader en zijne moeder}{vergaan in tranen}\\

\haiku{doch zij herstelde,:}{zich onmiddellijk schoof eenen}{stoel bij en zeide}\\

\haiku{- en trouw maar met eenen,!}{anderen man gij zult toch}{niet gelukkig zijn}\\

\haiku{Dit papier heb ik;}{den ganschen nacht met mijne}{tranen overgoten}\\

\haiku{de eenzaamheid en.}{de avondkoelte zullen mij}{misschien versterken}\\

\haiku{{\textquoteleft}Ik geloof niet, dat.}{de notaris daar nog zal}{willen van hooren}\\

\haiku{Kaat, de huismeid, was;}{met het hoofd op eene tafel}{in slaap gevallen}\\

\haiku{Verlicht mijnen geest,.}{of ik bezwijk onder het}{gewicht mijner smart}\\

\haiku{maar dit bewijst ten.}{voordeele van zijn eenvoudig}{en beminnend hart}\\

\haiku{Daar ziet gij het nu,.}{wel dat wij voor den rampspoed}{zijn geboren}\\

\haiku{Welnu, waar blijft gij '?}{met het liedeken vant}{zal wel beter gaan}\\

\haiku{Elk oogenblik, dat,.}{verloopt is eene hel van angst}{en lijden voor hem}\\

\haiku{Hugo ligt gebukt,;}{onder het akeligst lot dat}{eenen man kan treffen}\\

\haiku{zijne eer, zijnen,,.}{naam zijne vrijheid alles}{wil men hem ontrooven}\\

\haiku{Maar, notaris, het:}{moet u gemakkelijk zijn}{het geld te vinden}\\

\haiku{Zijne moeder heeft;}{sedert vier dagen nog niets}{gedaan dan weenen}\\

\haiku{{\textquoteright} {\textquoteleft}Welnu,{\textquoteright} hernam de, {\textquoteleft}.}{notarisGabri\"el zelf}{heeft het geschreven}\\

\haiku{Houdt het kind warm, dat.}{geen trek van deur of venster}{het raken kunne}\\

\haiku{Zijn vader zal de,,.}{zaak waarvan gij spreekt al lang}{vergeten hebben}\\

\haiku{laat mij door, ik moet,{\textquoteright}.}{eene haastige boodschap doen}{smeekte moeder Job}\\

\haiku{Nog eenige stappen;}{en zij zou van tusschen het}{koren geraken}\\

\haiku{Welhaast, als hadde,:}{de wanhoop haar overwonnen}{zuchtte zij verschrikt}\\

\haiku{dat Hij in zijne}{goedheid mij krachten late}{tot het volbrengen}\\

\haiku{{\textquoteright} {\textquoteleft}Ik heb geenen tijd, vriend,{\textquoteright},.}{Mols riep zij zonder haren}{stap te vertragen}\\

\haiku{de baron jaagde,,.}{hem om zoo te zeggen als}{eenen schelm van den hof}\\

\haiku{de minste traagheid.}{in de uitvoering zijner}{wenschen maakt hem gram}\\

\haiku{Er was immers niets?}{in zijnen beklaaglijken}{toestand veranderd}\\

\haiku{dit zijn woorden van,,.}{menschen die alles in het}{zwart zien gelijk gij}\\

\haiku{Hij is van hoofd tot;}{voeten in het zwart gekleed}{als een lijkbidder}\\

\haiku{zij verduisterde.}{uwen geest en deed u al wat}{kwaad is overdrijven}\\

\haiku{maar gij moest maar eens,!}{weten wat ik altemaal}{in mijn hart doorsta}\\

\haiku{Walter is op de;}{Pruisische grenzen door de}{gendarmes verrast}\\

\haiku{hun oog is zonder,.}{leven zij hebben honger}{en zoeken voedsel}\\

\haiku{- wat zoudt gij dan wel!}{met uwen enkelen wagen}{kunnen verrichten}\\

\haiku{Anders hadden zij.}{zekerde zware huispacht}{niet kunnen dragen}\\

\haiku{Het is gelijk, gij,!}{zult morgen naar Rijsel op}{den ijzeren weg}\\

\haiku{vindt gij daar nog de,.}{helft van twaalf franken in dan}{zal het alles zijn}\\

\haiku{Misschien is het maar,.....}{koud in eene kamer waar nooit}{vuur heeft  gebrand}\\

\haiku{Met Paschen of.}{weinige dagen daarna}{zouden zij trouwen}\\

\haiku{Hij trad binnen en.}{legde een groot papieren}{pak op de tafel}\\

\haiku{maar niet verschieten,,,}{Christina want het is iets}{dat u eenen fellen}\\

\haiku{Hij vond dezen met.}{eenen omwonden voet voor eene}{tafel gezeten}\\

\haiku{{\textquoteright} {\textquoteleft}Dit verlies brengt mij,{\textquoteright}.}{in eene groote verlegenheid}{zeide de koopman}\\

\haiku{{\textquoteright} {\textquoteleft}Mijnheer, o, Mijnheer,,!}{gij veroordeelt mij tot de}{schande tot den dood}\\

\haiku{mijne hersens zijn,.}{zoo vermoeid dat ik er gansch}{duizelig van ben}\\

\haiku{Ach, daar staat nog de,;}{stoof die mijne verstijfde}{leden heeft ontdooid}\\

\haiku{Zij was het, die mij,.}{ter schole geleidde toen}{ik nog een kind was}\\

\haiku{het is alsof een.}{nieuw leven met het bloed door}{mijne aderen vloeit}\\

\haiku{Hief den ijzeren.}{klopper der poort op en liet}{hem nedervallen}\\

\haiku{hare ziel hing nog.}{aan het mondelijn van het}{aangebeden kind}\\

\haiku{welk ook het geheim,:}{haars harten zij ik toch zal}{het niet verraden}\\

\haiku{Haar echtgenoot vond;}{in hare blijdschap eene bron}{van troost en genot}\\

\haiku{{\textquoteleft}Gravinne, laat ons,.}{in de kamer gaan waar de}{klavecimbel staat}\\

\haiku{zuster Cathelyne;}{uit het Falconsklooster heeft}{haar muziek geleerd}\\

\haiku{de Dwene zette;}{zich insgelijks nevens de}{moeder in eenen stoel}\\

\haiku{Nu,{\textquoteright} sprak de moeder, {\textquoteleft}.}{zing het lied van Met vreugde}{willen wij zingen}\\

\haiku{{\textquoteleft}Ja, 't is goed, nu......}{zing ik toch niet meer en van}{mijn leven niet meer}\\

\haiku{Uwe vereerende.}{vriendschap is mij eene schoone}{belooning genoeg}\\

\haiku{Als zij dan uit het,;}{huis trouwen verstrekt haar het}{gespaarde tot bruidschat}\\

\haiku{{\textquoteright} Met ontroering greep:}{de gravinne de hand der}{moeder en zeide}\\

\haiku{Eensklaps verbleekte,.}{de edelvrouw en zij begon}{van angst te beven}\\

\haiku{De Signora, door,.}{de Dwene gevolgd trad in}{hare slaapkamer}\\

\haiku{want haar uur is nooit,.}{zoo juist dat het som wijlen}{niet veel verschille}\\

\haiku{pijnlijke zuchten.}{waren het eenige antwoord}{op des meiskens vraag}\\

\haiku{evenwel was er nog.}{een schijn van ongeloof in}{haren vreugdelach}\\

\haiku{{\textquoteleft}Ik ben uwe eenige, -!}{uwe echte moeder en ik}{heb geen ander kind}\\

\haiku{{\textquoteright} De Dwene rustte.}{met het hoofd op den stoel en}{weende sprakeloos}\\

\haiku{ik ben in zijn oog;}{een vuig en verachtelijk}{schepsel geworden}\\

\haiku{Ik ben gekomen.}{om de hel des twijfels in}{uwen boezem te dooven}\\

\haiku{indien het waarheid,,.}{is wat gij zeggen gaat zoo}{zegene u God}\\

\haiku{{\textquoteright} {\textquoteleft}Eilaas,{\textquoteright} zuchtte de, {\textquoteleft}?}{graafwaarom mij die droeve}{tijden herinnerd}\\

\haiku{hoe dikwijls zij, voor,}{uwe voeten knielend onder}{eenen vloed van tranen}\\

\haiku{{\textquoteleft}Clara, het is de,.}{graaf d'Almata de man van}{uwe  beschermster}\\

\haiku{{\textquoteright} {\textquoteleft}Mijn vader is in,{\textquoteright}, {\textquoteleft}.....}{den hemel zuchtte Clara}{hij bidt God voor mij}\\

\haiku{{\textquoteright} riep d'Almata met, {\textquoteleft};}{ontroeringdit geheim wilt}{gij niet verraden}\\

\haiku{Ik zou de stem uws,.....}{vaders  miskennen zijn}{gebed verwerpen}\\

\haiku{- Maar gij zult  niets,?}{zeggen van hetgene hier}{geschied is niet waar}\\

\haiku{Reeds drie dagen was.}{het feest onder de meiskens}{in het Maagdenhuis}\\

\haiku{Zie nu maar, dat gij,!}{de waarheid kunt zeggen als}{het mogelijk is}\\

\subsection{Uit: Volledige werken 29. Valentijn. Eene verwarde zaak}

\haiku{Wanneer er ten uwent,.}{eene koe iets overkomt dan staat}{gij het beest wel bij}\\

\haiku{{\textquoteright} {\textquoteleft}Ik begrijp die hooge,{\textquoteright}.}{woorden maar half mompelde}{Monica dubbend}\\

\haiku{Maar ik ben eenig kind,.}{en mijne ouders willen}{daar niet van hooren}\\

\haiku{Zij is oud, en wat,.}{zij eens heeft besloten blijft}{onveranderlijk}\\

\haiku{en geen mensch op het,,,;}{dorp die mij nadert geen die}{mij de hand toereikt}\\

\haiku{want des avonds adem ik,;}{de geuren die van achter}{de haag opwalmen}\\

\haiku{En geen mensch die het,.}{reddend woord spreekt niemand die}{de hand hem toereikt}\\

\haiku{Moeilijk ware het,;}{geweest den ouderdom van}{dit mensch te raden}\\

\haiku{Wel waren zijne;}{kleederen tot op den draad}{versleten misschien}\\

\haiku{Nu is het te laat.}{om tot dit middel mijne}{toevlucht te nemen}\\

\haiku{Dan keerde hij zich.}{om en stapte mijmerend}{naar zijne woning}\\

\haiku{Helena wees met:}{den vinger naar eene kleine}{hoogte en zeide}\\

\haiku{in zulke zware.}{aarde als deze moet zij}{allengs versterven}\\

\haiku{Ik heb er in mijn,.}{leven vele duizenden}{gemaakt Mejuffer}\\

\haiku{{\textquoteright} De onderwijzer;}{aanschouwde het meisje met}{groote verwondering}\\

\haiku{En gij zelf, poogt gij?}{niet altijd op uw best voor}{den man te komen}\\

\haiku{- Intusschen wiesch zij,;}{mijne wonde met eene hand}{zoo zacht als fluweel}\\

\haiku{ja, hij zette zich.}{zelfs op de bank en schouwde}{eene wijl ten gronde}\\

\haiku{de gansche wereld,;}{die zich voor ons beglanst met}{een onbekend licht}\\

\haiku{Oh, wat zullen wij!}{altezamen gelukkig}{zijn op de wereld}\\

\haiku{Heb ik gevraagd om?}{die hel in mijnen boezem}{te voelen branden}\\

\haiku{{\textquoteleft}Ja, ja,{\textquoteright} zeide hij, {\textquoteleft}.}{in zich zelvende goede}{Hendrik heeft gelijk}\\

\haiku{Op een paar duizend.}{franken zal ik niet zien om}{u te beloonen}\\

\haiku{{\textquoteright} mompelde hij, {\textquoteleft}speelt,,?}{gij comedie Valentijn}{of is het gemeend}\\

\haiku{Dan de zegepraal.}{van het arglistige mensch}{zou niet lang duren}\\

\haiku{Anders, hoe zou hij?}{zelf het al lachende ons}{komen vertellen}\\

\haiku{Dit huwelijk moet.}{u ongelukkig maken}{voorgansch uw leven}\\

\haiku{Ik vermoedde niet,;}{dat er u zooveel nijd in}{het hart kon liggen}\\

\haiku{{\textquoteright} Het meisje misgreep;}{zich gewis over den echten}{zin dezer woorden}\\

\haiku{ongelukkig en,.}{bedroefd bovenmate en}{toch rustig en sterk}\\

\haiku{Het vuur der woede;}{heeft een oogen-blik mijn}{voorhoofd doen gloeien}\\

\haiku{maar de heiligheid.}{der plaats en mijn eerbied voor}{haar weerhielden mij}\\

\haiku{Uw hart is goed, gij,.}{hebt verstand en gij zoudt uw}{geld niet verkwisten}\\

\haiku{Het laatste woord van;}{Helena's vader verbrak}{echter zijnen droom}\\

\haiku{{\textquoteright} {\textquoteleft}Ik ben een wees en,{\textquoteright}.}{heb geene familie zeide}{de onderwijzer}\\

\haiku{Alleen zijnde, bleef;}{Valentijn nog eene wijl in}{zijne verbluftheid}\\

\haiku{Slechts bij het begin.}{der tweede bladzijde werd}{zijne stem luider}\\

\haiku{{\textquoteright} De onderwijzer;}{bezweek schier van geluk en}{van ontsteltenis}\\

\haiku{Helena moet de,.}{vrouw van u worden of de}{vrouw van Casimir}\\

\haiku{want het was voor hem.}{onder meer dan \'e\'en opzicht}{een plechtige dag}\\

\haiku{Al kwam de koning,,.}{zelf ik zou niet toelaten}{dat men u stoorde}\\

\haiku{maar daar klonk de stem,:}{van den kosterszoon die hem}{van buiten toeriep}\\

\haiku{Gij waart gereed tot,;}{alles om ons dit doel te}{helpen bereiken}\\

\haiku{{\textquoteright} M. Stoop gaf niet veel;}{acht op de gramme woorden}{van den olieslager}\\

\haiku{maar de olieslager;}{moest toch een zijner woorden}{hebben opgevat}\\

\haiku{Helena uwe hand.}{aanbieden en den bloodaard}{niet met haar spelen}\\

\haiku{want hare wangen.}{toonden nog de sporen van}{vergoten tranen}\\

\haiku{het eenige middel,,.}{daartoe het eenige is uw}{huwelijk met mij}\\

\haiku{zij stapte naar de:}{deur der kamer en morde}{nog in het heengaan}\\

\haiku{Dan zeide hij zeer,:}{stil tot de vrouwen die hem}{bevend aanzagen}\\

\haiku{De olieslager, reeds,:}{een beetje rood van den wijn}{trad binnen en vroeg}\\

\haiku{Mag ik hopen, dat?}{gij mij deze eenige gunst}{niet zult weigeren}\\

\haiku{ik heb  eenige.}{aanteekeningen in ons}{pachtboek geschreven}\\

\haiku{Met kalmte, zonder,.}{aangejaagdheid voor eenigen}{tijd nog ten minste}\\

\haiku{Ik wil u niet in.}{die oude diligence}{laten vertrekken}\\

\haiku{{\textquoteleft}Maar, Helena, het,.}{is niet volstrekt noodig dat ik}{heden vertrekke}\\

\haiku{{\textquoteleft}Wat drommel, Jan, is?}{dit nu rijden als een mensch}{met gezond verstand}\\

\haiku{de weiden waren,.}{zuur of te droog de akkers}{mager of vochtig}\\

\haiku{Hij leeft op zijne.}{renten en woont alleen met}{eene zijner zusters}\\

\haiku{De koets bleef welhaast.}{voor eene afspanning in het}{voorgeborchte staan}\\

\haiku{Bij de tafel zat.}{Helena met het hoofd op}{eenige papieren}\\

\haiku{Moeder Roosens zou.....}{hare dochter niet gaarne}{veel medegeven}\\

\haiku{Dit had den armen.}{knecht meer dan eens kletsen en}{stompen aangebracht}\\

\haiku{{\textquoteright} De jongeling was:}{opgestaan en morde met}{gebalde vuisten}\\

\haiku{{\textquoteleft}Ik misgrijp mij wel,{\textquoteright},.}{zeker zeide de pachter}{het hoofd schuddende}\\

\haiku{{\textquoteleft}Thomas, Thomas, laat,!}{uw hart vermurwen blijf niet}{onverbiddelijk}\\

\haiku{Ach, hoe kunt gij toch?}{zoo koel het doodelijk verdriet van}{ons arm kind aanzien}\\

\haiku{Is de dood niet daar,?}{om den menschelijken wil}{te verijdelen}\\

\haiku{Denk niet, Urbaan, dat.}{een gevoel van eigenbaat}{mij dus doet spreken}\\

\haiku{{\textquoteleft}Eh, eh, Blaas, Trees, komt,,:!}{geloopen gauw gauw ik mag}{trouwen met Cilia}\\

\haiku{En het is dus waar,,?}{Cilia gij gaat trouwen met}{Urbaan Couterman}\\

\haiku{Zij heeft haar kleed te!}{Brussel laten maken en}{zij weet wat het kost}\\

\haiku{{\textquoteright} {\textquoteleft}Zinnelooze woorden,,{\textquoteright};}{ijdele bedreigingen}{antwoordde Cilia}\\

\haiku{neen, bij den duivel,,!}{die mijne ziel beloert gij}{zijt nog niet getrouwd}\\

\haiku{{\textquoteright} Cilia hield de oogen.}{nedergeslagen en scheen}{van vrees te beven}\\

\haiku{het was, als liep daar.}{een mensch of een wild dier door}{het gebladerte}\\

\haiku{maar hoe hard en hoe,.}{dikwijls zij naar den knecht riep}{zij kreeg geen antwoord}\\

\haiku{Hij verborg zijnen.}{eigen angst om zijne vrouw}{te kunnen troosten}\\

\haiku{Zij zag hem zitten,;}{in den donkeren kerker}{op wat vochtig stroo}\\

\haiku{Maar de bewustheid;}{moest even ras weder in haar}{teruggekeerd zijn}\\

\haiku{de pachter had het.}{hoofd gebogen en scheen door}{wanhoop verpletterd}\\

\haiku{{\textquoteleft}Hum, hum, indien gij.}{maar de zaak niet te veel naar}{de eene zijde wringt}\\

\haiku{Zijne oogen waren,;}{rood en het was zichtbaar dat}{hij lang had geweend}\\

\haiku{Niemand vroeg ons, wie.}{van ons beiden den doodelijden}{slag heeft gegeven}\\

\haiku{{\textquoteright} {\textquoteleft}Weet gij dan niet, dat?}{hij verklaart den messteek te}{hebben gegeven}\\

\haiku{Was het niet uw zoon,?}{of waart gij het niet die den}{eersten slag toebracht}\\

\haiku{gij hebt het mij reeds,{\textquoteright},.}{gezegd morde de drossaard}{het hoofd schuddende}\\

\haiku{{\textquoteright} Het meisje vatte.}{eenen bezem en begon de}{kamer te vegen}\\

\haiku{{\textquoteright} {\textquoteleft}Gij, gij zoudt naar het?}{kasteel durven gaan om voor}{Urbaan te spreken}\\

\haiku{maar gelijk alle,.}{goedhartige menschen is}{hij zwak van gemoed}\\

\haiku{{\textquoteright} {\textquoteleft}Neen, in de hand van,.}{schutter Dierkx die ze naar den}{drossaard ging dragen}\\

\haiku{Gedurende eene,:}{wijl liet Karel haar begaan}{doch dan zeide hij}\\

\haiku{Wat hem terughield,.}{was alleen de dubbele}{zelfbeschuldiging}\\

\haiku{Maar, hoe akelig ook,.....}{indien God dit ongeluk}{heeft toegelaten}\\

\haiku{{\textquoteleft}Hoe is de arme!}{pachteresse vermagerd}{op zoo korten tijd}\\

\haiku{Hij wendde zich tot;}{de getuigen en stelde}{hun vele vragen}\\

\haiku{Urbaan, mijn kind, gij,?}{waart het dus niet die Marcus}{de wonde toebracht}\\

\haiku{{\textquoteright} {\textquoteleft}Thomas Couterman,?}{gij hebt de getuigenis}{van uwen knecht gehoord}\\

\haiku{totdat het allengs.....}{op de andere helling}{van het dal wegstierf}\\

\subsection{Uit: Volledige werken 33. Moederliefde. Lambrecht Hensmans}

\haiku{Marian zal mij,.}{den sleutel brengen zoohaast haar}{werk afgedaan is}\\

\haiku{En al wat gij mij,;}{te vragen hebt is zijne}{genade alleen}\\

\haiku{maar welhaast scheen eene;}{klimmende verwondering}{haar aan te grijpen}\\

\haiku{{\textquoteright} {\textquoteleft}Welaan, ik ga u.}{de reden mijner komst te}{Orsdael verklaren}\\

\haiku{{\textquoteright} gilde de boerin,.}{zich verstomd op haren stoel}{latende vallen}\\

\haiku{Maar ik zou Orsdael,!}{voor altijd ontvluchten}{om het nooit te zien}\\

\haiku{Er zijn misschien nog,?}{meer mevrouwen die den naam}{van Bruinsteen dragen}\\

\haiku{En na eene wijl te,:}{hebben gezwegen zeide}{zij op blijden toon}\\

\haiku{de nood dwingt mij tot.}{het zoeken eener plaats bij}{deftige lieden}\\

\haiku{Van Bruinsteen niets te?}{zien heeft in de aanvaarding}{harer dienstboden}\\

\haiku{Zij heeft u misschien,?}{aangegrijnsd gelijk zij het}{mij gewoonlijk doet}\\

\haiku{ik ben toch altijd.}{hier om u voor te staan en}{u te beschermen}\\

\haiku{Ik heb meenen te,;}{zien dat uw vertrouwen hem}{min of meer trotsch maakt}\\

\haiku{Van Bruinsteen met eene.}{grijns van verwondering en}{ontevredenheid}\\

\haiku{Eene lange wijl bleef;}{zij dus met uitge rekten}{hals luisterend staan}\\

\haiku{Terwijl de jonkvrouw,}{dus denkend voor den spiegel}{stond meende zij eenen}\\

\haiku{Zij legde zich de,:}{hand aan het voorhoofd slaakte}{eenen kreet en zuchtte}\\

\haiku{maar zij hoorde den.}{sleutel in het slot draaien}{en richtte zich op}\\

\haiku{maar alsdan waren,;}{er nog dienstboden die met}{mij mochten spreken}\\

\haiku{{\textquoteleft}Neen, zoo kan hij mij,.}{niet beminnen hij is schoon}{en ontzagwekkend}\\

\haiku{Houd u stil, en, kwam,.}{er iemand vergeet niet wat}{ik u heb gezegd}\\

\haiku{Daarin ligt echter.}{de uitlegging van den haat}{der gravin voor haar}\\

\haiku{{\textquoteright} {\textquoteleft}Ik moet wachten en,.}{zien welke middelen zich}{kunnen aanbieden}\\

\haiku{die ledigheid in,.....}{mijn hart als uw aangezicht}{mij niet tegenstraalt}\\

\haiku{Ik ben gekomen:}{om u eene gelukkige}{tijding te brengen}\\

\haiku{Het is reeds een kwart,.}{uurs dat ik geweld doe om}{haar te doen knielen}\\

\haiku{Ik ben geene vrouw om.}{mij door eenen knecht dagelijks}{te laten hoonen}\\

\haiku{Het is beter hem.}{te bedriegen dan hem tot}{vijand te hebben}\\

\haiku{Ik durfde vragen,,.....}{Martha dat God u mijne}{moeder liete zijn}\\

\haiku{{\textquoteright} Met eene ernstige:}{uitdrukking op het gelaat}{sprak de weduwe}\\

\haiku{Gij weet, dat ik op.}{vele uwer vragen niet mag}{of kan antwoorden}\\

\haiku{Ook glinsterde op.}{zijne borst de star der eer}{en der dapperheid}\\

\haiku{Ik wist, dat Hector;}{soldaat geworden was om}{mij te verdienen}\\

\haiku{Hij heeft mij nog meer;}{dan eens van den Leeuw zonder}{manen gesproken}\\

\haiku{mijn Hector bleef tot;}{den avond ongedeerd en vocht}{als een ware leeuw}\\

\haiku{geheimen, waarvan.}{gij nogtans de uitlegging}{eens zult bekomen}\\

\haiku{Ik weet wel, hoe hem.}{eenen afkeer van Helena}{in te boezemen}\\

\haiku{Het is een gezicht,;}{dat mij in mijnen slaap moet}{gemarteld hebben}\\

\haiku{het overige hangt.}{af van uw verstand en van}{uwe behendigheid}\\

\haiku{gij zoudt Helena.}{ongelukkig maken en}{u zelven met haar}\\

\haiku{Het is met moeite,.}{dat zij zich nog gewaardigt}{mij te antwoorden}\\

\haiku{het was zichtbaar, dat;}{zij tegen een smartelijk}{gepeins worstelde}\\

\haiku{{\textquoteleft}Hem laten gelooven,?}{dat ik toestem om zijne}{vrouw te  worden}\\

\haiku{Gansch roerloos bleef zij,;}{staan totdat zij overtuigd was}{van haren misgreep}\\

\haiku{Leen haar de kracht om!}{de stem harer benauwde}{ziel te verstikken}\\

\haiku{nochtans gij kent al.}{de uitgestrektheid mijner}{genegenheid niet}\\

\haiku{Mathijs aanschouwde.}{haar met eene uitdrukking van}{fierheid en geluk}\\

\haiku{{\textquoteright} {\textquoteleft}Waarom verbergt gij?}{mij dan deze reden zoo}{onverbiddelijk}\\

\haiku{maar wees zeker, dat,,.....}{het waar is want Martha zegt}{het en wat zij zegt}\\

\haiku{Deze laatste, toen,:}{hij genaderd was riep met}{grove barschheid}\\

\haiku{Wil zij wraak nemen,.}{over het gebeurde het zij}{dan op mij alleen}\\

\haiku{Wat kunt gij daaraan?}{doen dat die hooze Frederik}{onverwachts verschijnt}\\

\haiku{Zoudt gij wel gelooven,?}{dat zij in het geheel aan}{zich zelve niet denkt}\\

\haiku{mijne moeder heeft,.}{ons beiden vergiffenis}{geschonken zegt gij}\\

\haiku{Gij zult edelmoedig?}{genoeg zijn om mij uwe gunst}{terug te schenken}\\

\haiku{Ik zal zijne komst.}{afspieden en tot hem gaan}{op zijne kamer}\\

\haiku{{\textquoteleft}Ik doe al wat mij,.}{mogelijk is en gij schijnt}{nog niet tevreden}\\

\haiku{Zij zou zich zelve?}{dus in gevaar brengen om}{mij te verderven}\\

\haiku{maar door den dood van.}{dat kind ontsnapte haar het}{fortuin van den graaf}\\

\haiku{Omdat gij een schrift,?}{bezit dat met hare hand}{is onderteekend}\\

\haiku{{\textquoteright} {\textquoteleft}Het ligt daar in dien,{\textquoteright}.}{ijzeren koffer zeide}{de gouvernante}\\

\haiku{{\textquoteright} Toen de jager in,:}{de kamer getreden was}{zeide de boerin}\\

\haiku{Dries-Jan stiet de kolf:}{van zijn geweer zachtjes ten}{gronde en zeide}\\

\haiku{Ik wil, indien het,.}{mogelijk is hem heden}{nog zien en spreken}\\

\haiku{{\textquoteright} {\textquoteleft}Haar vader is dood,{\textquoteright}, {\textquoteleft}.}{antwoordde Marthamaar zij}{heeft nog eene moeder}\\

\haiku{die tijding, mocht zij,.}{waar zijn is gewichtiger}{dan gij kunt denken}\\

\haiku{Ook de weduwe.}{en Frederik weenden van}{blijde ontroering}\\

\haiku{Hij opende de deur.}{van Martha's kamer en sloeg}{den blik op het bed}\\

\haiku{Bleek en bevend stak.}{hij den anderen sleutel}{op de tweede deur}\\

\haiku{Hij zag de jonkvrouw.}{op eenen stoel in het diepe}{der kamer zitten}\\

\haiku{{\textquoteright} De notaris greep:}{haar de hand en zeide met}{sidderende stem}\\

\haiku{Verheug u, mijn kind,.}{gij wordt de gelukkige}{bruid van M. Bergmans}\\

\haiku{- daar beneden aan.}{mijne voeten bruisten de}{golven van den vloed}\\

\haiku{Ligt er misschien over?}{zijn leven een sluier van}{onbekende smart}\\

\haiku{maar eene akelige.}{verschijning nagelde haar}{vast op haren stoel}\\

\haiku{men doeme mijne,;}{onnoozele zusters als het}{kroost van eenen booswicht}\\

\haiku{Willem had lang en.}{stilzwijgend op de stem van}{Klara geluisterd}\\

\haiku{zij storte eenen traan, -, -!}{over het lot mijns vaders en}{verder niets niets meer}\\

\haiku{mijne liefde zal, -!}{eeuwig zijn ik heb het aan}{God zelven beloofd}\\

\haiku{Toen zij weder in,:}{boozen klap zich uitliet snauwde}{een arm wijf haar toe}\\

\haiku{Daar alleen is nog!}{hoop op rechtvaardigheid en}{op herstelling}\\

\haiku{IJselijk moest in;}{zulke geestgesteltenis}{hem het leven zijn}\\

\haiku{mijne toekomst op,;}{aarde is als eene kalme}{doodsche zeevlakte}\\

\haiku{Ik vergeef u dit, -.}{niet altijd dezelfde moet}{gij voor mij blijven}\\

\haiku{ik trap met voeten:}{wat daar nog van mijn vorig}{leven in overbleef}\\

\haiku{- daar, tegen den muur,.....}{stond de arme vrouw lachend}{haar kind te kussen}\\

\haiku{In het eerst wilde;}{de moeder zich het kind niet}{laten ontnemen}\\

\haiku{Zij wekte bevend,:}{haren man en als deze}{opsprong met den roep}\\

\haiku{de som was te groot,.}{en hij had op aarde geene}{andere vrienden}\\

\haiku{- {\textquoteleft}Ik ben een dief en,,;}{een schelm dit is waar ik heb}{mijne straf verdiend}\\

\haiku{want Willem zal gaan,.}{te huis komen en gij moet}{uwe les nog overzien}\\

\haiku{{\textquoteleft}Moeder lief, wil ik,?}{u eens iets zeggen dat u}{blijde maken zal}\\

\haiku{{\textquoteright} De vrouw scheen ditmaal:}{zeer goed te beseffen en}{sprak op blijden toon}\\

\haiku{{\textquoteright} Maar de vrouw vatte:}{hem bij de hand en sprak op}{zonderlingen toon}\\

\haiku{Het zijn niet alleen,.....}{tranen van blijdschap die over}{uwe wangen stroomen}\\

\haiku{Reeds had eene zoete;}{samenspraak de stille taal}{der oogen vervangen}\\

\haiku{De Besteek heeft in.}{Brabant immer plaats op den}{avond v\'o\'or den naamdag}\\

\subsection{Uit: Volledige werken 34. Levenslust}

\haiku{Ha, een gansch jaar van,.}{arbeid zorg en worsteling}{is weder voorbij}\\

\haiku{Was het de juffer,?}{of de norsche grijsaard die}{hem dus bespiedde}\\

\haiku{Nu zien wij slechts een.}{gedeelte der Alpen van}{het Berner Oberland}\\

\haiku{Deze laatste is.}{de witste berg van allen en}{een van de hoogsten}\\

\haiku{Ik gevoel mij hoogst.}{gelukkig aan zijnen wensch}{te mogen voldoen}\\

\haiku{Zij hoorden wel, dat;}{eene groote menigte menschen}{zich daar bevonden}\\

\haiku{{\textquoteright} Opvolgens werden;}{hun verschillige soorten}{van visch voorgediend}\\

\haiku{Uwe verbeelding droomt;}{en dweept onophoudelijk}{en laat u geene rust}\\

\haiku{{\textquoteleft}zij is waarschijnlijk.}{uitgegaan om in de stad}{eene boodschap te doen}\\

\haiku{{\textquoteright} {\textquoteleft}Neen, ditmaal misgrijpt{\textquoteright},,.}{gij u wedersprak Herman}{het hoofd schuddende}\\

\haiku{Mijne trefbare.}{verbeelding is de eenige}{reden daarvan niet}\\

\haiku{Hun voet baadt in de!}{blauwe zee en hunne}{kruin raakt den hemel}\\

\haiku{Herman meende naar;}{het roer te gaan om zijnen}{vriend te vervoegen}\\

\haiku{{\textquoteleft}Laat ons vergeten,.}{dat de bleeke juffer aan boord}{van de stoomboot is}\\

\haiku{onder anderen,...}{den Rothhorn Sigriswyl}{meer dan 7,000 voet hoog}\\

\haiku{Wij zullen den Rus:}{achternarijden en hem}{in het oog houden}\\

\haiku{Dit gezicht bracht eenen;}{plotselijken omkeer in}{zijn gemoed te weeg}\\

\haiku{eene kleur als die van.}{zekere parelen of}{van melkachtig glas}\\

\haiku{Ik verwar mij in;}{mijne neuswijzerij als}{in eene klis garen}\\

\haiku{Wat ik van den Rus,.}{zeg is enkel scherts om een}{beetje te lachen}\\

\haiku{Hij is een tooveraar,,.}{Herman en heeft zich voor ons}{onzichtbaar gemaakt}\\

\haiku{{\textquoteright} {\textquoteleft}Maar wat wilt gij toch?}{op die naakte rotsen gaan}{doen zonder leidsman}\\

\haiku{{\textquoteright} {\textquoteleft}Ik zwem in mijne{\textquoteright},.}{kleederen antwoordde de}{jonge advocaat}\\

\haiku{Het gerucht van den!}{piassenden regen doet}{zoo zalig slapen}\\

\haiku{{\textquoteright} {\textquoteleft}Het is eene zieke,;}{waarover men ons veel spreekt in}{onze studi\"en}\\

\haiku{{\textquoteright} {\textquoteleft}Neen, Max, ik bid u,,.}{stoor het stil genot niet dat}{mijne ziel overstroomt}\\

\haiku{Ik geloof waarlijk.}{dat gij het ditmaal doet om}{met mij te spotten}\\

\haiku{{\textquoteleft}Planken van boven,,;}{planken van onder planken}{van wederzijde}\\

\haiku{Wij zullen morgen.}{tijd genoeg hebben om}{den ijsberg te zien}\\

\haiku{Van beschrijving tot.}{beschrijving rekte zijn brief}{zich uitermate}\\

\haiku{{\textquoteleft}Dat is te zeggen{\textquoteright},, {\textquoteleft}.}{verbeterde Maxindien}{er geen gevaar is}\\

\haiku{{\textquoteright} {\textquoteleft}Neen, heeren, met eenen.}{goeden leidsman is er geen}{het minste gevaar}\\

\haiku{{\textquoteleft}Volgt mij, en gevoelt,.}{gij u vermoeid wij zullen}{onderweg rusten}\\

\haiku{{\textquoteright} Zij begonnen de;}{bestijging met goeden moed}{en spraken niet veel}\\

\haiku{Tot het dragen van;}{zulken zetel behoeven}{vier sterke mannen}\\

\haiku{Ik heb tot nu toe,;}{halvelings geloofd dat dit}{mogelijk kon zijn}\\

\haiku{{\textquoteright} {\textquoteleft}In den winter woon,{\textquoteright},.}{ik beneden in het dorp}{heer antwoordde zij}\\

\haiku{{\textquoteright} {\textquoteleft}Daaraan zijn wij van{\textquoteright},.}{onze kindsheid af gewend}{antwoordde de vrouw}\\

\haiku{geen kruid, geen vogel,.}{geen levend wezen meer in}{deze wildernis}\\

\haiku{{\textquoteleft}Staan wij hier boven?}{een grondeloos water op}{eene dunne ijskorst}\\

\haiku{{\textquoteleft}Breek ik onderweg,!}{zonder leidsman den hals zoo}{zal het uwe schuld zijn}\\

\haiku{Waarschijnlijk had de:}{ontsteltenis van Herman}{eene dubbele bron}\\

\haiku{{\textquoteright} {\textquoteleft}Gij toch, vriendje, gaat,?}{niet meer ter school vermits gij}{gids of leidsman zijt}\\

\haiku{Gij zult gaan hooren.}{dat hij hem misschien sedert}{zes maanden bezit}\\

\haiku{alle betrekking.}{tusschen hem en ons is voor}{altijd verbroken}\\

\haiku{{\textquoteleft}De vader der bruid;}{moet bij het huwelijksfeest}{tegenwoordig zijn}\\

\haiku{{\textquoteleft}Ziet ginder, heeren,{\textquoteright},.}{het hotel op den Faulhorn}{zeide de jongen}\\

\haiku{Ik zou wel willen{\textquoteright},, {\textquoteleft}}{weten zeide Hermanwie}{van ons beiden den}\\

\haiku{{\textquoteright} Max maakte eenen sprong,.}{als hadde hij lust om den}{waard te omhelzen}\\

\haiku{Het moet insgelijks.}{door de menschen op den berg}{gedragen worden}\\

\haiku{Deze ijswereld?}{zal dus dood blijven tot het}{einde der eeuwen}\\

\haiku{Ik zou niet gaarne.}{eene tweede maal door de zon}{gebraden worden}\\

\haiku{Indien wij te zes,.}{uren niet op weg zijn begaan}{wij eene groote domheid}\\

\haiku{{\textquoteleft}Vreest niet, heeren{\textquoteright}, sprak, {\textquoteleft},;}{de leidsmande weg is wel}{moeilijk inderdaad}\\

\haiku{want het is hier te.}{vochtig en te koud om er}{lang stil te blijven}\\

\haiku{Wij hebben geenen tijd.}{om ons hier met kluchten te}{blijven vermaken}\\

\haiku{{\textquoteright} {\textquoteleft}Vergeef mij deze{\textquoteright}, {\textquoteleft}:}{opmerking zeide Max.In}{mij ontstaat twijfel}\\

\haiku{misschien zijt gij zelf,,.}{zonder het te weten door}{den schijn bedrogen}\\

\haiku{Geloof mij, wat er,:}{ook geschiede het arme}{kind is veroordeeld}\\

\haiku{Spot niet met dien wensch,}{lach niet met dat inzicht of}{ik verwittig u}\\

\haiku{Max Rapelings had;}{zelfs eene samenspraak met de}{meisjes begonnen}\\

\haiku{{\textquoteright} De jonge dokter:}{verschrikte op zijne beurt}{en zeide zuchtend}\\

\haiku{Iedereen heeft zoo.}{zijne oogenblikken van}{geestafdwaling}\\

\haiku{Ik heb geenen tijd om.}{mij langer met deze zaak}{bezig te houden}\\

\haiku{Komt, stijgen wij in}{de koets en denken wij dat}{het nog beter is}\\

\haiku{Terwijl zij aan het,:}{eten waren zeide Max tot}{den tafeldiener}\\

\haiku{{\textquoteleft}Vergeef mij het erg.}{verdenken dat ik tegen}{u had opgevat}\\

\haiku{Wij gingen dikwijls,.}{naar Gent op de feesten der}{groote maatschappijen}\\

\haiku{{\textquoteright} {\textquoteleft}Hebt gij gedwaald, heer,{\textquoteright}, {\textquoteleft}}{dan was het slechts door te veel}{liefde zeide Max.}\\

\haiku{het denkbeeld alleen.}{van zulk voornemen zou haar}{doen terugschrikken}\\

\haiku{een twijfelachtig,.}{woord en gij doet mijn ontwerp}{in duigen storten}\\

\haiku{Rus, den gewaanden?}{verdrukker eener zieke}{maagd had toegewijd}\\

\haiku{Het is uwe maag, uwe.}{maag alleen die ontsteld of}{liever verzwakt is}\\

\haiku{Gave God dat ik;}{nooit ergere gevallen}{onder handen kreeg}\\

\haiku{{\textquoteright} {\textquoteleft}Ja, dan valt er niets,{\textquoteright}.}{meer te zeggen schertste de}{jonge advocaat}\\

\haiku{{\textquoteleft}Maar, oom lief, gij hadt}{besloten dat wij morgen}{zouden vertrekken}\\

\haiku{Er staat op een uur.}{en een half van de kruin des}{bergs een groot hotel}\\

\haiku{Ja, ja, tenzij gij,.}{zeer sterk zijt zult gij het kwaad}{hebben tegen mij}\\

\haiku{die zang, zoo vol ziel,:}{en vol gevoel heeft u de}{zenuwen ontsteld}\\

\haiku{{\textquoteright} {\textquoteleft}Ja, het moet wel zijn;}{dat uwe verbeelding weder}{op hol was geraakt}\\

\haiku{{\textquoteright} {\textquoteleft}Ah, dokter, gij hebt!}{mij de hoop op genezing}{in het hart gestort}\\

\haiku{Allen, behalve,;}{Herman slaakten eenen kreet van}{blijde verrassing}\\

\haiku{Laat ons nu onze,.}{reis voortzetten alsof er}{niets geschied ware}\\

\haiku{het was hier wel wat,.}{koud doch men kon wandelen}{en zich verwarmen}\\

\haiku{Gaan denken dat gij!}{mejuffer Halewijn hier}{op den Rigi ziet}\\

\haiku{zijne klagende.}{tonen hergalmden boven}{den Rigi-kulm}\\

\haiku{met zijne witte.}{bakkebaarden en zijne}{bruine paletot}\\

\haiku{Van dan af is zij.}{moedeloos gebleven en}{heeft zich ziek gevoeld}\\

\haiku{Dit troostend geloof,,.}{heeft in uwe afwezigheid}{haar gansch verlaten}\\

\haiku{ik begrijp het niet,}{en evenwel dank ik God voor}{den zoeten troost dien}\\

\haiku{{\textquoteright} Max Rapelings trok;}{eenen stoel bij en zette zich}{nevens Florentia}\\

\haiku{{\textquoteright} {\textquoteleft}Ik zou altijd met.}{u willen blijven en u}{nooit meer verlaten}\\

\haiku{Zij zouden dus te;}{zamen naar Fluelen en}{naar Amsteg reizen}\\

\haiku{{\textquoteright} {\textquoteleft}Op het tijdstip dat?}{eene doodelijke ziekte haar in}{dat gesticht overviel}\\

\haiku{Was het de wil van?}{God een derde wezen in}{dien band te sluiten}\\

\haiku{want zoo vernietigt.}{men zelfs de mogelijkheid}{der opoffering}\\

\haiku{Uwe opoffering,?}{wat ware zij anders dan}{een droeve zelfmoord}\\

\haiku{Dan zullen zij niets.}{zoo stil kunnen zeggen dat}{wij het niet hooren}\\

\haiku{Zie, wat beteekent toch?}{dat zonderling houten kruis}{tegen de rots daar}\\

\haiku{Het fiere hart van.}{Willem Tell weigerde dien}{smaad te onderstaan}\\

\haiku{De jongelieden.}{kregen elk eene kamer op}{een hooger verdiep}\\

\haiku{nog drie boogschoten.}{hooger en wij waren}{op de kruin des bergs}\\

\haiku{Ook luisterde hij;}{met gespannen aandacht op}{des dokters woorden}\\

\haiku{Daar voor mij staat de.}{bleeke schimme mijner zuster}{Frederika}\\

\haiku{maar ik ben bijna.}{zeker dat wij ons beiden}{hebben bedrogen}\\

\haiku{{\textquoteright} Herman reikte haar.}{den handschoen met zichtbare}{onverschilligheid}\\

\haiku{Hij deed nog eenige...}{stappen vooruit en beklom}{eene kleine hoogte}\\

\haiku{Zij zendt mij om te.}{vernemen hoe het met den}{heer Van Borgstal gaat}\\

\haiku{Zijn aangezicht was.}{bleek en getuigde van eene}{diepe verschriktheid}\\

\haiku{{\textquoteright} {\textquoteleft}Neen{\textquoteright}, antwoordde de, {\textquoteleft};}{burgeronze gemeente}{is daartoe te klein}\\

\haiku{{\textquoteright} {\textquoteleft}Ja, het arme kind,.}{ziet er diep bedrukt uit zij}{moet geweend hebben}\\

\haiku{{\textquoteleft}O, heer professor,!}{hoe dankbaar ben ik u voor}{uwe gedienstigheid}\\

\haiku{Gij zult hem den arm?}{houden en het bloed in die}{waschkom ontvangen}\\

\haiku{{\textquoteright} sprak M. Halewijn,.}{in zich zelven verschrikt van}{den stoel opstaande}\\

\haiku{{\textquoteleft}Ah, ja, ja, daar is,!}{hij daar is de engel die}{mij heeft genezen}\\

\haiku{De gelukkigste,!}{van allen zijt gij niet het}{is de arme Max}\\

\subsection{Uit: Volledige werken 35. De koopman van Antwerpen}

\haiku{ik drijf koophandel,.}{ik bemin den vrede en}{daarmede gedaan}\\

\haiku{{\textquoteleft}De koffie schijnt van.}{hare levendigheid te}{hebben verloren}\\

\haiku{wees rechtzinniglijk,.}{bedankt voor den troost dien gij}{mij hebt gegeven}\\

\haiku{hij schudde zelfs eens;}{het hoofd met eene uitdrukking}{van bekommernis}\\

\haiku{{\textquoteleft}Zeker, mijnheer, het,.}{is zijn portret dat gij mij}{hebt afgeschilderd}\\

\haiku{hij was nieuwsgierig,.....}{om te weten hoe ik te}{Brussel zou varen}\\

\haiku{{\textquoteleft}Daarenboven, ik,;}{heb u iets te melden dat}{u vermaak zal doen}\\

\haiku{{\textquoteright} De koopman sloeg den.}{blik ten gronde en overwoog}{eene wijl in stilte}\\

\haiku{Het verandert uwen,.}{toestand in de wereld mijn}{goede Rapha\"el}\\

\haiku{Ik wil spoedig rijk,,.....}{zijn op eenige jaren en}{gelukt mij dit niet}\\

\haiku{Maak de mat van de.}{mande los en begiet de}{bloemen voorzichtig}\\

\haiku{maar eindelijk, bij,.}{het verlaten van een hoog}{schaarbosch zag hij Mev}\\

\haiku{{\textquoteleft}Arme Verboord, hij.}{is altijd vol zorgen en}{bekommernissen}\\

\haiku{Ik zou welhaast gaan,.}{denken dat mijne woorden}{u verdriet aandoen}\\

\haiku{Er was er een, die.}{schier van den ganschen avond mij}{niet heeft verlaten}\\

\haiku{Het is reeds zoolang,.}{dat zij dit feest met vroolijk}{ongeduld afwacht}\\

\haiku{{\textquoteright} {\textquoteleft}En dan, mijnheer, het.}{is dezen avond feest in de}{groote Harmonie}\\

\haiku{Het middagmaal gaat.....{\textquoteright} {\textquoteleft};}{onmiddellijk opgediend}{wordenHet is zoo}\\

\haiku{maar nu zag hij, dat.}{de koopman hem een teeken}{van ongeduld deed}\\

\haiku{{\textquoteleft}Verboord, mijn vriend, gij.}{handelt noch rechtvaardig noch}{redelijk met ons}\\

\haiku{{\textquoteleft}Mevrouw weet wel, dat.}{mijnheer sedert drie maanden}{zeer zwaarmoedig is}\\

\haiku{Het is niet enkel,?}{om ons te troosten dat gij}{zulk vertrouwen toont}\\

\haiku{Zij alleen, en ik!}{zou het kunnen verzwijgen}{voor gansch de wereld}\\

\haiku{De prijsverhooging?}{van de koffie zouden zij}{toch moeten kennen}\\

\haiku{{\textquoteright} {\textquoteleft}Zijn de prijzen dan?}{eensklaps tot eenen gunstigen}{koers geklommen}\\

\haiku{Ik kom u spreken,,.}{van stoffelijke zaken}{van geld van fondsen}\\

\haiku{Mijne zaken zijn.....{\textquoteright} {\textquoteleft}?}{een beetje in de warUwe}{zaken in de war}\\

\haiku{- Welnu, ik bedank.}{u uiterharte voor dit}{bewijs uwer vriendschap}\\

\haiku{Gij kunt het,{\textquoteright} was het, {\textquoteleft}.}{antwoorden ik twijfel niet}{of gij zult het doen}\\

\haiku{Mijn huisgezin zou?}{door banden des bloeds aan het}{uwe worden gehecht}\\

\haiku{zij scheen zeer wel te,.}{begrijpen doch aarzelde}{nog in haar geloof}\\

\haiku{Felicita, mijn,.}{kind binnen eene maand zal men}{u groeten als Mev}\\

\haiku{{\textquoteright} Hij was reeds buiten,.}{de deur der kamer toen hij}{deze woorden sprak}\\

\haiku{Nu keerde hij zich.}{geheel om en richtte zich}{met haast naar de poort}\\

\haiku{{\textquoteright} Dit grievend gepeins,:}{deed hem rechtspringen terwijl}{hij nog herhaalde}\\

\haiku{Vooronderstel, dat.}{dit huwelijk inderdaad}{voltrokken worde}\\

\haiku{Ware ik in uwe,.}{plaats ik zou mij wreken door}{een diep misprijzen}\\

\haiku{Mijn vurigste wensch.}{was met hem voor altijd te}{worden vereenigd}\\

\haiku{Ik was gelukkig,.}{nochtans en niettemin ik}{schrikte en weende}\\

\haiku{Met mijne hand te,.}{aanvaarden verkreeg hij zulk}{kapitaal ruimschoots}\\

\haiku{Heeft het gebed u?}{dan niet versterkt tegen uwe}{ongegronde vrees}\\

\haiku{maar, wees zeker, v\'o\'or.}{dezen dag heb ik er nooit}{iets van geweten}\\

\haiku{maar dewijl men van,.}{buiten op de deur klopte}{bleef hij gezeten}\\

\haiku{Men zou zeggen, dat.}{gij mager geworden zijt}{sedert gisteren}\\

\haiku{Hij vatte de hand,:}{des jongelings trok hem naar}{een stoel en zeide}\\

\haiku{{\textquoteleft}Zij  heeft sterke;}{hoofdpijn en is voor een uur}{of twee gaan rusten}\\

\haiku{{\textquoteleft}Ik moet mij met spoed,.}{gereedmaken om naar M.}{Dorneval te gaan}\\

\haiku{Ach, ik smeek u, doe!}{geene moeite om mijn besluit}{te veranderen}\\

\haiku{Gij zult hem zeggen,.}{dat ik hem onmiddellijk}{verlang te spreken}\\

\haiku{Nu echter kan ik,.}{in veiligheid eene maand ja}{zelfs langer wachten}\\

\haiku{want zijn groet was schier,.}{onhoorbaar en zijn gelaat}{bleef ernstig en koel}\\

\haiku{Laat hooren, welk is?}{dit schrikkelijk geheim dat}{ik niet weten mag}\\

\haiku{{\textquoteright} M. Verboord sprong met;}{de beide handen in het}{haar achteruit}\\

\haiku{Dat ik arm worde,;}{maar dat ik toch eer moge}{doen aan mijn handteeken}\\

\haiku{{\textquoteright} De deur der zaal werd.}{onverwachts geopend na}{eenen haastigen klop}\\

\haiku{M. Dooms, de dokter,;}{van het dorp was redelijk}{in zijne eischen}\\

\haiku{maar zouden wij wel}{kunnen winnen wat er noodig}{is om den dokter}\\

\haiku{Walput genoeg liet,.}{gevoelen dat wij hare}{komst niet verlangden}\\

\haiku{Ik heb haar gezegd,.}{dat vader ziekelijk is}{en niemand wil zien}\\

\haiku{want Felicita:}{zeide met verdoofde stem}{tot hare moeder}\\

\haiku{{\textquoteleft}Felicita, hoe!}{zuiver is uw hart en hoe}{innig uwe liefde}\\

\haiku{Toen hij gevonden,:}{had wat hij zocht murmelde}{hij in zich zelven}\\

\haiku{{\textquoteright} {\textquoteleft}Maar, goede heer Banks,?}{weet gij dan niet meer wat wij}{u verschuldigd zijn}\\

\haiku{Sedert dan heeft de:}{fortuin niet opgehouden}{ons toe te lachen}\\

\haiku{Een mijner makkers,.....}{bleef er dood en ik kreeg eenen}{kogel door den arm}\\

\haiku{{\textquoteright} Walput schudde het.}{hoofd met verwondering en}{ontevredenheid}\\

\haiku{Hoe onverdelgbaar!}{is toch een ingeworteld}{gevoel in den mensch}\\

\haiku{Het was klaarblijkend,;}{voor hem dat de Verboords niet}{geheel arm waren}\\

\haiku{want anders zouden.}{zij dat rijke huisraad niet}{hebben behouden}\\

\haiku{Ik zal u zeggen.}{waarom hij van Amerika}{is wedergekeerd}\\

\haiku{Zal ik het gezicht?}{van den ondankbare wel}{kunnen verdragen}\\

\haiku{Mijn vader is ziek,.}{en zijne zenuwen zijn}{uiterst ontstelbaar}\\

\haiku{{\textquoteright} {\textquoteleft}Neen, neen, verlies den,{\textquoteright}, {\textquoteleft};}{moed niet zeide zijalles}{staat ten gunstigste}\\

\haiku{ik bid u, ik smeek,.}{u heb toch eenige deernis}{met mijn droevig lot}\\

\haiku{Wel had men nog niets;}{van hem gehoord sedert zijn}{vertrek naar Parijs}\\

\haiku{hij zal ontwaken.}{en allengs geheel uit de}{bezwijming opstaan}\\

\haiku{Verboord greep zijne:}{hand en zeide met diepe}{treurnis in de stem}\\

\haiku{Gij hebt gezien hoe.}{noodlottig die naam nog op}{zijne zinnen werkt}\\

\haiku{Zoeken wij vrede.}{en rust in de uitstorting}{onzer dankbaarheid}\\

\haiku{En zou daarom zijn?}{eindelooze edelmoed zonder}{belooning blijven}\\

\haiku{wierden de schulden,.}{vereffend dan moest hij het}{onfeilbaar weten}\\

\haiku{{\textquoteright} riep Felicita,.}{die hare ontsteltenis}{niet kon bedwingen}\\

\haiku{{\textquoteright} De grijsaard bleef eene;}{wijl met onvasten blik in}{de ruimte staren}\\

\subsection{Uit: Volledige werken 36. Schandevrees. Koning Oriand. Blinde Rosa}

\haiku{Ik geloof het niet,,{\textquoteright},;}{mijnheer zeide de dienstmeid}{het hoofd schuddende}\\

\haiku{maar nu twijfelde,;}{hij niet meer of hij moest zich}{misgrepen hebben}\\

\haiku{De arme man zou.}{naar alle waarschijnlijkheid}{niet lang meer leven}\\

\haiku{Of anders hebt gij.}{hem dien doodsteek wetens en}{willens toegebracht}\\

\haiku{Des anderen daags.}{is hij op reis gegaan met}{de arme jonkvrouw}\\

\haiku{Zij waren weg, en?}{gij hebt hen sedert dan niet}{meer wedergezien}\\

\haiku{Gelooft gij, Homans,?}{dat zij zich in nood zouden}{kunnen bevinden}\\

\haiku{Kan iets, kan geld tot,.}{uw geluk nog bijdragen}{zeg een enkel woord}\\

\haiku{{\textquoteright} De graaf de Hammes:}{greep de beide handen des}{grijsaards en zeide}\\

\haiku{Die teedere bloem,?}{men zou ze doen verstikken}{bij gebrek aan lucht}\\

\haiku{{\textquoteright} {\textquoteleft}Gij weet wel, moeder,,,?}{Jakob de jager die bij}{den Reigerspoel woont}\\

\haiku{Ach, in die kamer,,!}{binnen het somber gebouw}{leefde en leed Ida}\\

\haiku{Dan eerst wendde hij.}{den blik weder naar den kant}{van den Ouden Steen}\\

\haiku{{\textquoteleft}gij, zoo schoon, zoo zoet,, -?}{zoo zuiver sterven in de}{Lente des levens}\\

\haiku{Ons leven zou een,,.}{hemel zijn van vrede van}{blijdschap van liefde}\\

\haiku{Nu hij wist, dat Ida,.}{hem beminde gevoelde}{hij zich reuzensterk}\\

\haiku{{\textquoteright} gromde de oude.}{heer met eene nieuwe vlaag van}{gramschap in de oogen}\\

\haiku{ik moest eene hooge plaats....}{bekleeden in het bestuur}{van mijn vaderland}\\

\haiku{ik sluit haar op in.}{hare kamer en ik draag}{den sleutel op mij}\\

\haiku{Tegen den morgen,,;}{in mijnen slaap zag ik mij}{zeiven in het park}\\

\haiku{Dat zulke aanslag,.}{zou kunnen gelukken hoeft}{gij niet te vreezen}\\

\haiku{Mijn zoon Hugo is.}{vurig verliefd geworden}{op uwe kleindochter}\\

\haiku{{\textquoteleft}Ja, mijnheer, achter,.}{in het park niet verre van}{de groene rustbank}\\

\haiku{{\textquoteright} {\textquoteleft}Ginder tusschen het,.}{kreupelhout liggen de twee}{ladders en de plank}\\

\haiku{O, God, Gij, die weet,!}{hoe zuiver mijn inzicht is}{laat mij gelukken}\\

\haiku{Had zij het briefje,?}{gevonden dat hij over den}{muur geworpen had}\\

\haiku{hij ademde hijgend.}{en bewoog nu en dan de}{leden krampachtig}\\

\haiku{gij zult weten wat,.}{het is eene moeder doodelijk in}{haar kind te treffen}\\

\haiku{Hugo wrong zijne;}{leden krampachtig en scheen}{te willen opstaan}\\

\haiku{zij liet het hoofd op.}{de borst vallen en blikte}{zwijgend ten gronde}\\

\haiku{Kom, Maria, verhef.}{uwen moed in de maat der}{wreedheid van het lot}\\

\haiku{Zouden wij nu als?}{lafaards het hoofd bukken en}{den strijd opgeven}\\

\haiku{{\textquoteright} {\textquoteleft}Gij weet wel, mevrouw.....{\textquoteright} {\textquoteleft},?}{Von WeilerZoo mijnheer kent}{mijnen nieuwen naam}\\

\haiku{{\textquoteright} {\textquoteleft}Gij weet, mevrouw, dat.}{ik slechts de onschuldige}{oorzaak daarvan was}\\

\haiku{Mijne treurige,?}{geschiedenis ontrukt u}{eenen traan Hortensia}\\

\haiku{{\textquoteleft}Mijnheer..... Willem, het,.}{betaamt niet dat wij langer}{te zamen blijven}\\

\haiku{{\textquoteright} De graaf deinste een.}{paar stappen achteruit en}{verbleekte zichtbaar}\\

\haiku{In zijn  hart maakt.}{hij u verantwoordelijk}{voor ons bitter lot}\\

\haiku{{\textquoteright} {\textquoteleft}Vijf minuten voor,{\textquoteright},.}{tienen zeide de graaf op}{zijn uurwerk ziende}\\

\haiku{lief,{\textquoteright} murmelde zij, {\textquoteleft},.....{\textquoteright} {\textquoteleft}?}{wat de graaf mij zeide was}{integendeelHoe}\\

\haiku{De graaf De Hammes?}{was vermetel genoeg om}{ons op te zoeken}\\

\haiku{wel lijdt hij nog pijn,.}{van zijnen val maar hij zit}{reeds in eenen leunstoel}\\

\haiku{Gij zult toch v\'o\'or uw,.....}{huwelijk haar niet zeggen}{wie haar vader is}\\

\haiku{{\textquoteleft}Ja, lieve nicht, ik,{\textquoteright},;}{ben slechts uw oom sprak hij haar}{teeders treelende}\\

\haiku{{\textquoteright} riep de veldwachter.}{met eene soort van angstige}{verontwaardiging}\\

\haiku{{\textquoteright} Door deze laatste,:}{woorden vergramd gebood de}{koning zeer barsch}\\

\haiku{ik meende evenwel.....}{eenen glimlach op uwe lippen}{te hebben verrast}\\

\haiku{De minste glimlach;}{van haar voerde hem op tot}{het hoogste geluk}\\

\haiku{{\textquoteright} {\textquoteleft}Maar hoe zal Oriand,?}{vernemen wat de ridders}{en het volk zeggen}\\

\haiku{Men leide nu de;}{beschuldigden in mijne}{tegenwoordigheid}\\

\haiku{In de troonzaal van:}{het paleis waren vele}{lieden vergaderd}\\

\haiku{kreet, machtig genoeg.}{om op zekeren afstand}{te worden gehoord}\\

\haiku{{\textquoteright} murmelde hij, {\textquoteleft}maar;}{alle hoop op leven is}{voor mij verloren}\\

\haiku{dit glansrijke werk.}{van God is getuige van}{hetgeen hier geschiedt}\\

\haiku{Hij laat zich evenmin.}{uitroeien als het immer}{woekerend onkruid}\\

\haiku{Marcus en eenige.}{anderen stonden zelfs de}{tranen in de oogen}\\

\haiku{Gij, Coenraad, die een,.}{wijs man zijt neem eerst het woord}{en verberg mij niets}\\

\haiku{- uwe moeder, smeltend,.....}{in tranen en bezwijkend}{van smart toonde ons}\\

\haiku{Wassche haar schuldig!}{bloed zooveel mogelijk de}{schandvlek van uwe kroon}\\

\haiku{De koning bleef eene;}{wijl zwijgend en hield den blik}{nedergeslagen}\\

\haiku{{\textquoteright} {\textquoteleft}Indien Savary!}{het kind had gered en ons}{wilde verraden}\\

\haiku{{\textquoteright} {\textquoteleft}Neen, neen, uwe vrees is,{\textquoteright}.}{ongegrond wedersprak hem}{Mattabruna}\\

\haiku{Kon mijn dood zijne,}{ziel de verlorene rust}{terugschenken}\\

\haiku{dit zal mijnen zoon.}{verontwaardigen en hem}{ten uwen gunste stemmen}\\

\haiku{Allen zwegen en;}{hielden met angst het oog naar}{hunnen vorst gericht}\\

\haiku{Maar weder brak de.}{blik van Beatrix zijnen}{wil en zijne macht}\\

\haiku{Eindeloos is mijn,{\textquoteright}.}{medelijden antwoordde}{de  koningin}\\

\haiku{De overtuiging, dat,;}{hij zich niet bedroog stelde}{Helias gerust}\\

\haiku{Zoohaast wij dien voorbij,.}{zijn zien wij mijns vaders kluis}{in het groene dal}\\

\haiku{{\textquoteright} {\textquoteleft}Ik ben een ridder,{\textquoteright}.}{der Tafel-rond antwoordde}{de onbekende}\\

\haiku{{\textquoteright} {\textquoteleft}Ongetwijfeld kent,?}{gij den ongelukkige}{die op uwe hulp wacht}\\

\haiku{{\textquoteleft}Helias, ga naar.}{de weide en let op het}{paard van dezen heer}\\

\haiku{{\textquoteright} {\textquoteleft}Zeker, zeker, men.}{roemt hem als den dappersten}{leenman des keizers}\\

\haiku{De koning stapte.}{aan het hoofd van eenen langen}{stoet uit het paleis}\\

\haiku{{\textquoteright} {\textquoteleft}Arme vorstinne,,{\textquoteright}.}{denk aan uwe moeder zeide}{hare gezellin}\\

\haiku{Ik zal u daarvoor!}{gaarne zien en beminnen}{uit geheel mijn hart}\\

\haiku{Tranen glinsterden,:}{hem in de oogen terwijl hij}{overwonnen zeide}\\

\haiku{- maar het uitwerksel.}{van dit gezegde was niet}{zooals hij had verwacht}\\

\haiku{{\textquoteright} riep hij uit, {\textquoteleft}dat Hij.}{ten minste \'e\'enen mijner}{makkers gespaard heeft}\\

\haiku{{\textquoteleft}God zij gezegend,,!}{dat Hij u zoolang leven}{liet baas Joris}\\

\haiku{Ik belast mij met.}{u in Zijnen heiligen}{naam te beloonen}\\

\haiku{{\textquoteleft}Als  ik Rosa,;}{te bedelen leid dan spreekt}{ze altijd van u}\\

\haiku{Blinde Karel is.....}{besteed op eene hoeve langs}{de kanten van Lier}\\

\haiku{Er staat altijd een.}{bijzonder potteken voor}{haar in de assche}\\

\haiku{De blinde duwde.}{hem evenwel zachtjes met de}{handen van zich weg}\\

\haiku{Overblijfsel van een,!}{leger dat verslagen werd}{door het zwaard des tijds}\\

\subsection{Uit: Volledige werken 38. Jacob van Artevelde}

\haiku{die nijverheid en;}{handel opvoerde tot den}{hoogsten top van bloei}\\

\haiku{hij reisde, welke.}{last hem was opgelegd en}{wat hij uitvoerde}\\

\haiku{Op het verzoek des.}{Wijzen Mans vloog hier alles}{ten zijnen dienste}\\

\haiku{Welaan, gezellen{\textquoteright},, {\textquoteleft}:}{hernam Artevelde met}{koelheidzijt gerust}\\

\haiku{iets dichterlijks, iets;}{kwijnends lag er in zijnen}{langzamen oogslag}\\

\haiku{Die mannen, men zou.}{zeggen dat ze dorst hebben}{naar elkanders bloed}\\

\haiku{men heeft ons gezegd}{dat gij door uwe wijsheid en}{uwe groote goederen}\\

\haiku{daar heeft Graaf Gwijde;}{het schandelijk verdrag van}{Melun bekrachtigd}\\

\haiku{Dit alles kon hem:}{toch niet voor valschheid en}{onrecht behoeden}\\

\haiku{hun bestaan getuigt,.}{van Frankrijks snoode heerschzucht}{niet van onzen plicht}\\

\haiku{Ziet hier, hoe ik meen,:}{dat Vlaanderen vrijheid en}{nering zal hebben}\\

\haiku{maar dat men ook een.}{wapen hebben moet om ze}{te verdedigen}\\

\haiku{{\textquoteleft}Gezellen, wat men,;}{u daar gezegd heeft kan in}{den grond heel schoon zijn}\\

\haiku{Wat zullen wij, bij,?}{voorbeeld doen als onze Graaf}{terug wil komen}\\

\haiku{Zullen wij weder?}{kruipen onder de hand die}{ons gegeeseld heeft}\\

\haiku{ik geloofde een.}{bloedig zwaard te zien dat de}{vrijheid bedreigde}\\

\haiku{Indien hij komt als?}{Graaf van Vlaanderen en niet}{als Frankrijks veldheer}\\

\haiku{Hij zit krachtens recht;}{met de Schepenen van der}{Keure in den Raad}\\

\haiku{hooren hoeverre;}{de menigte op haren}{weg gevorderd was}\\

\haiku{{\textquoteright} {\textquoteleft}Ah, Muggelyn, er.}{moet met voorzichtigheid te}{werk gegaan worden}\\

\haiku{misschien denkt hij dat;}{gij de zaken reeds te diep}{begint in te zien}\\

\haiku{Zij weerhield zich en;}{naderde tot Lieven om}{hem te bedaren}\\

\haiku{Dat komt van in die...}{vervloekte kiezing voor uwen}{vader te loopen}\\

\haiku{Hij zou mij zulken,!}{stoel ten geschenke moeten}{geven uw vader}\\

\haiku{maar het zal er niet, '!}{bij blijven al moest Roeland}{int spel komen}\\

\haiku{Misschien wildet gij;}{Hoofdman in St-Michiels}{gekozen worden}\\

\haiku{voor de maat hunner,:}{kleine driften zij mogen}{laken en schelden}\\

\haiku{maar wat deze op,.}{minluiden toon antwoordde}{verstonden zij niet}\\

\haiku{hij zijnen mantel.}{en zijne huike aan en}{ging de kamer uit}\\

\haiku{als ik het Gentsche}{volk in alle straten hoor}{juichen en zingen}\\

\haiku{de bezetting van;}{Biervliet zal het grondgebied}{van Gent verwoesten}\\

\haiku{En vooronderstel:}{dat wij den eersten aanval}{des vijands afkeeren}\\

\haiku{maar hoe zullen wij?}{met de wacht van St-Jan door}{het volk geraken}\\

\haiku{Van Rupelmonde{\textquoteright},, {\textquoteleft}.}{was het antwoordik moet den}{Opperhoofdman zien}\\

\haiku{maar Artevelde;}{voorzag de mogelijkheid}{van zulken toestand}\\

\haiku{Gewis, de ziel van;}{Geeraart Denys moest door}{vreugde ontroerd zijn}\\

\haiku{de vischverkoopers met;}{hunne gestreepte kolders}{en lange lansen}\\

\haiku{De Overdeken liep;}{met zijne mannen tegen}{de ruiterij in}\\

\haiku{Artevelde had;}{reeds eene lichte wonde aan}{het hoofd bekomen}\\

\haiku{U ziende zou men!}{waarlijk zeggen dat gij de}{wereld hebt verroerd}\\

\haiku{Eindelijk besloot.}{men tot de misdaad zijne}{toevlucht te nemen}\\

\haiku{Laffe honden, die!}{tegen de zon blaft omdat}{haar licht u bedwelmt}\\

\haiku{{\textquoteright} {\textquoteleft}Opperhoofdman, zult,?}{gij over de Vrijdagmarkt gaan}{als gij naar huis keert}\\

\haiku{{\textquoteright} {\textquoteleft}Koude zucht naar de.}{waarheid en warme liefde}{voor mijn vaderland}\\

\haiku{mijn eed verplicht mij, -?}{en mijne graafschappen van}{Rethel en van Nevers}\\

\haiku{Men luistert  zoo.}{lichtelijk als men lijdt en}{ontevreden is}\\

\haiku{maar gij wijst mij daar;}{iets aan dat inderdaad niet}{te verzuimen is}\\

\haiku{- Gent, Brugge, Yperen,,,.}{Kortrijk Audenaerde}{Aelst en Geeraertsberge158}\\

\haiku{{\textquoteright} {\textquoteleft}Ik zou het doen en{\textquoteright},.}{geef daarop mijn ridderwoord}{antwoordde Edward}\\

\haiku{men ging ze nogmaals!}{gebruiken om Vlaanderens}{recht te verkrachten}\\

\haiku{Wees grootmoedig en!}{aanvaard het bestand zooals het}{u wordt voorgesteld}\\

\haiku{{\textquoteleft}Dezen morgen zal.}{het bestand ongetwijfeld}{worden bezegeld}\\

\haiku{- Belachelijke,,!}{spotternij die gij aanvaardt}{als een onnoozel kind}\\

\haiku{hoofde voor zijnen.}{vader en hoorde wellicht}{niet wat hij zeide}\\

\haiku{Sterven misschien als...}{eene bloem welker wortel van}{wormen is doorknaagd}\\

\haiku{Alles zal ijdel;}{worden in zijnen geest en}{in zijnen boezem}\\

\haiku{zijn gefolterde...}{geest stoot dit beeld terug en}{smeekt om genade}\\

\haiku{{\textquoteright} riep de Koning der, {\textquoteleft}!}{Ribauden lachendhet is}{onze vriend Lieven}\\

\haiku{wat gij mij nu ten.}{laste legt meende ik zelf}{u te verwijten}\\

\haiku{zij moge in uwen.}{geest geprent blijven tot het}{einde uwer dagen}\\

\haiku{Ver Artevelde{\textquoteright},.}{antwoordde Ghelnoot met eene}{minzame buiging}\\

\haiku{Heeft hij mijn geheim,.}{veropenbaard hij deed het}{zonder boos inzicht}\\

\haiku{maar ik wilde wel;}{weten welke lastertong}{u dit gezegd heeft}\\

\haiku{dat de koning der}{Ribauden niet afhouden}{zou vooraleer hij}\\

\haiku{Gij begrijpt wel dat:}{ik niet van den beginne}{vooruitspringen kan}\\

\haiku{ik verzoek u, mij;}{uwe gansche aandacht te leenen}{en niet te spotten}\\

\haiku{Daar zij van mij niet.}{weten zullen zij alles}{schreeuwen wat men wil}\\

\haiku{Ik weet het wel, en.}{hoop dat het bij mijne komst}{meest zal gedaan zijn}\\

\haiku{Naarmate de tijd,;}{verliep vermeerderde ook}{de toevloed des volks}\\

\haiku{doch men kon daarin}{geen ander doel vermoeden}{dan het inrichten}\\

\haiku{Ik bid God dat Hij!}{in dit plechtig oogenblik}{uwen geest verlichte}\\

\haiku{Het scheen dat zulks noch,;}{van Steenbeke noch zijne}{aanhangers beviel}\\

\haiku{{\textquoteleft}Neen, neen{\textquoteright}, antwoordde,.}{de Opperhoofdman hem de}{hand aangrijpende}\\

\haiku{maar zij zullen hem,!}{zelven eten hij moge hun}{smaken ofte niet}\\

\haiku{gebruikt is om de;}{Schepenen te bedriegen}{en te verschrikken}\\

\haiku{{\textquoteright} ~ {\textquoteleft}Welaan, vertrekt{\textquoteright},;}{gij maar van hier antwoordde}{Comyne met haast}\\

\haiku{Op dit oogenblik;}{is men bezig met over mijn}{lot te beslissen}\\

\haiku{{\textquoteright} Lieven sprong verschrikt.}{van zijnen stoel op en week}{bevend achteruit}\\

\haiku{Hij verliet alsdan.}{de Nederschelde en klom}{de Brabantstraat op}\\

\haiku{{\textquoteright} De jonge Denys:}{las het schrift over en sprak dan}{met verwondering}\\

\haiku{Gij wilt mij zonder,!}{hulp overleveren aan den}{beul ondankbare}\\

\haiku{{\textquoteright} {\textquoteleft}En zal de Jonkvrouw?}{dan niet herkennen welken}{weg zij gevolgd heeft}\\

\haiku{{\textquoteleft}Ah, Muggelyn, zijt}{gij reeds zoo oud geworden}{zonder te weten}\\

\haiku{Gij wilt weder langs,;}{kromme wegen tot uw doel}{geraken meester}\\

\haiku{Gij gevoelt dat de}{gelegenheid gunstig is}{om mij weder eenen}\\

\haiku{{\textquoteright} {\textquoteleft}Indien gij toch zoo,?}{moedig zijt waarom vermoordt}{gij haar dan zelf niet}\\

\haiku{Het is alsof het...}{haar ingegeven werd om}{mij te martelen}\\

\haiku{Geene beweging er,:}{in bespeurende zeide}{hij met blijdschap}\\

\haiku{belang hebben in?}{het gelooven aan de snoodste}{beschuldigingen}\\

\haiku{maar, Jacob, vriend, wij:}{mogen het zeggen in Gods}{tegenwoordigheid}\\

\haiku{laat mij afstand doen.}{van de ambten die het volk}{mij toevertrouwde}\\

\haiku{Oordeel in volle,;}{vrijheid beslis volgens uwe}{eigene inspraak}\\

\haiku{431.)water gieten?}{op de vlam die mijn vijand}{dreigt te verteren}\\

\haiku{Het is wel waar dat;}{al uwe pogingen op niets}{zullen uitloopen}\\

\haiku{gij maakt het verschiet,;}{wel zwart en veel zou daarop}{te zeggen vallen}\\

\haiku{229{\textquoteright} Geeraart Denys;}{viel eensklaps met gramschap uit}{en wilde spreken}\\

\haiku{Bevlekt u niet door.}{de onderwerping aan de}{verwaande Vollers}\\

\haiku{Ik spreek hier in naam.}{der gansche Weverij wier}{eer ik verdedig}\\

\haiku{{\textquoteright} {\textquoteleft}Neen, neen, vader{\textquoteright}, riep, {\textquoteleft}.}{Lieven met droefheidzeg mij}{zulke dingen niet}\\

\haiku{De Opperhoofdman.}{is zoolang gelasterd en}{vervolgd geworden}\\

\haiku{Gij ziet dus, dat het.}{morgen tijds genoeg is om}{uwe boodschap te doen}\\

\haiku{En hoe gij het ook,.}{beschouwet ik ga te bed}{en gij insgelijks}\\

\haiku{Geen laster loopt er,,,}{in Gent geen haat blaakt er geen}{bloed wordt er gestort}\\

\haiku{u toeroepen dat;}{ik het oogenblik mijner}{geboorte vervloek}\\

\haiku{Ongelooflijk en! -.}{toch waar Gij hebt nog vijftig}{moordenaars in huur}\\

\haiku{de eenige hoop op;}{verzoening met den Vorst moest}{worden verlaten}\\

\haiku{maar een andere.}{eisch kon zoo lichtelijk}{niet voldaan worden}\\

\haiku{Sedert de aankomst.}{der Engelsche vloot was de}{toestand veranderd}\\

\haiku{Hebt gij mij nog iets,?}{te zeggen of durft gij niet}{naar Gent wederkeeren}\\

\haiku{Geloof hem dan ook;}{op dit uur en wijs zijne}{smeekingen niet af}\\

\haiku{Deze verhuizing;}{ontsnapte in het eerst aan}{onze opmerking}\\

\haiku{Wat ik ben, werd ik;}{alleen door de keus mijner}{landgenooten}\\

\haiku{mijne vrienden zijn,.}{reeds naar boven gegaan zij}{brengen mij goed nieuws}\\

\haiku{{\textquoteright} Vrouw Artevelde.}{drukte nog met meer kracht de}{hand haars echtgenoots}\\

\haiku{Allen waren met,,.}{bijlen hamers zwaarden of}{daggen gewapend}\\

\haiku{98(3)Dat dit gebruik}{in Gent bestond is mij door}{den heer professor}\\

\haiku{99(1)Item ghaven zy,...}{meester Arnoude van Leene}{den stede surgien}\\

\haiku{se setten ende...}{spannen in dumysers voor}{scapiteins logyst}\\

\haiku{239{\textquoteleft}Verleenende...}{hem ter bewarenesse}{van zynen persoon}\\

\subsection{Uit: Volledige werken 39. Hlodwig en Clothildis}

\haiku{doch men kon zien, dat.}{hij daaronder een kleed van}{sneeuwwit linnen droeg}\\

\haiku{ik ben bereid mij.}{aan uwe uitspraak met ootmoed}{te onderwerpen}\\

\haiku{{\textquoteleft}Ik moet u nog iets;}{vragen over den toestand der}{zaken in Galli\"e}\\

\haiku{Lederen schoenen.}{waren hem met riemen aan}{de voeten gegespt}\\

\haiku{Hij zou eeuwig zijn,,?}{de band dien de zoete Freya}{niet gevlochten heeft}\\

\haiku{Voor de Wijtafel,;}{lagen bijlen messen en}{hamers van keisteen}\\

\haiku{{\textquoteright} Hierop besproeide.....}{hij de verloofden opnieuw}{met het offerbloed}\\

\haiku{zij twijfelden niet,,.}{of hij was het die tot den}{Maalberg naderde}\\

\haiku{Bij 's Heeren disch.}{waren nu de Scalden of}{dichters vergaderd}\\

\haiku{{\textquoteright} Langzaam voortgaande,;}{hield hij den blik vragend in}{de ruimte gericht}\\

\haiku{Allengs nochtans scheen;}{hij weder in eenen zachten}{droom weg te zinken}\\

\haiku{Eensklaps ontwaakte:}{hij uit zijne mijmering}{en zeide met spijt}\\

\haiku{Eensklaps ontwaart mijn;}{oor een hemelsch geluid van}{zang en snarenspel}\\

\haiku{Mij gewaardigt gij,.....{\textquoteright}}{niet te vragen of de nacht}{mij zacht is geweest}\\

\haiku{- de lucht is hier zoo,,!}{zuiver het lommer zoo frisch}{de natuur zoo mild}\\

\haiku{{\textquoteleft}Mijne dochter, ga;}{en wandel in den tuin met}{uwe gezellinnen}\\

\haiku{De toestand is zeer,.}{bedreigend voor mij men kan}{het niet miskennen}\\

\haiku{Zijne Weermannen;}{zouden tegen hem opstaan}{of hem verlaten}\\

\haiku{Om het gevaar te,?}{bezweren dat mijne eer}{en mijn hart bedreigt}\\

\haiku{Haar vader, hare,,.}{ooms hare broeders gansch haar}{geslacht is Ariaansch}\\

\haiku{maar laat mij sterven,!.....}{met de hoop dat ik eens uwe}{bruid zal  worden}\\

\haiku{Een lach van blijdschap.}{en zegepralenden nijd}{blonk op haar gelaat}\\

\haiku{Deze moest voor hen.}{als toegelaten gezant}{geheiligd blijven}\\

\haiku{Hier geschiedde de.}{beruchte wapendans der}{Noordervolkeren}\\

\haiku{Aurelianus trad,.}{in de tente om op den}{heirtog te wachten}\\

\haiku{Siagrius heeft.}{uwe uitdaging aanvaard en}{zal zijn woord houden}\\

\haiku{Luister en erken.....}{de onmogelijkheid uwer}{dwaze uitzichten}\\

\haiku{{\textquoteright} {\textquoteleft}Misschien bedriegt de,{\textquoteright}.}{koning zich niet bermerkte}{de Gallo-Remein}\\

\haiku{- Wat kan hij op mij,?}{die beschermd ben door al de}{Asen van het Glansheim}\\

\haiku{Hlodwig stond in de.}{laagte met de lieden van}{Doorniker-Gouw}\\

\haiku{- zoo spoedig had het!}{lot de bestemming van dit}{paleis veranderd}\\

\haiku{het heiligst is op,.}{aarde en in den hemel}{heeft er in gerust}\\

\haiku{Volg mij ter markte,;}{waar de buit op den middag}{zal verdeeld worden}\\

\haiku{{\textquoteright} {\textquoteleft}Neen, van mij zult gij,{\textquoteright}.}{het vat heden nog krijgen}{antwoordde Hlodwig}\\

\haiku{zij zijn de Goden, -.}{van ons geslacht niet van het}{zwartharige volk}\\

\haiku{en zeker, het zou.}{de inwoners van Galli\"e}{met recht bedroeven}\\

\haiku{Men bracht het bij den,,.}{Opperbloedman die het eenen}{doek voor de oogen bond}\\

\haiku{dan weder liep zij}{tot den Opperheirtog en}{herhaalde hare}\\

\haiku{{\textquoteright} {\textquoteleft}Maar waartoe zich met?}{zulke schoone uitzichten}{bezig gehouden}\\

\haiku{{\textquoteright} {\textquoteleft}Zij zal ook reeds den?}{Opperheirtog der Franken}{vergeten hebben}\\

\haiku{{\textquoteright} De Gallo-Romein.}{luisterde in verbaasdheid}{op Hlodwigs woorden}\\

\haiku{{\textquoteright} De heirtog wendde;}{zich om en meende tot de}{deur te naderen}\\

\haiku{Zij stak de handen:}{biddend tot hem op en riep}{in Latijnsche taal}\\

\haiku{De grond der straten;}{was overdekt met de lijken}{der weerlooze burgers}\\

\haiku{{\textquoteleft}Lutgardis, lieve,,.}{treur zoo niet om den hoon die}{ons is aangedaan}\\

\haiku{{\textquoteright} Een bittere lach',:}{bewoog Lutgardis lippen}{daar zij antwoordde}\\

\haiku{gij meent nog, dat de?}{dochter Hilperiks schuld heeft}{aan uw ongeluk}\\

\haiku{Elken kamper werd;}{een schild gebracht en aan den}{linkerarm geriemd}\\

\haiku{{\textquoteright} {\textquoteleft}Maar zij is Christin,{\textquoteright}.}{bemerkte een der Franken}{op spijtigen toon}\\

\haiku{{\textquoteright} {\textquoteleft}Machtige koning,{\textquoteright}, {\textquoteleft}}{der Burgonden antwoordde}{de Gallo-Romein}\\

\haiku{Het zijn de eenige,;}{bedreigingen waarvoor de}{koning zwichten kan}\\

\haiku{Zijt gij wel zeker,?}{dat de stomme jongeling}{Hilperiks zoon was}\\

\haiku{Neem uwe sleutels, wij.}{zullen de veroordeelde}{gaan verwittigen}\\

\haiku{Hij was de huisgraaf.....}{van koning Hilperik en}{stond nevens den troon}\\

\haiku{Zijn mijne ouders,,?}{en allen die mij dierbaar}{waren niet tot God}\\

\haiku{Uw oom stemt toe in;}{uwe echtverbintenis met}{mijnen heer Hlodwig}\\

\haiku{{\textquoteright} zeide Clothildis,.}{tot Aurelianus daar zij}{reeds ter deure ging}\\

\haiku{Zij klommen met de;}{Burgondische heeren in}{den breeden wagen}\\

\haiku{Het was een laag, maar,.}{zeer uitgestrekt gebouw van}{hout opgetimmerd}\\

\haiku{Ramold zal eischen,;}{dat de verloving in den}{Wijhof geschiedde}\\

\haiku{Ware der Franken!}{leger door de Romeinen}{vernield geworden}\\

\haiku{Glanzende sterren ';}{Ontvallen de lucht Aant}{einde der tijden}\\

\haiku{en heden nog wil.}{hij de zwartharige tot}{echtgenoote hebben}\\

\haiku{Verboden is het,;}{ons de zwartharige met}{bloed te besproeien}\\

\haiku{Kom, goede broeder,:}{stijg weder te paard en blijf}{dicht bij den wagen}\\

\haiku{hij deed gansch Frankrijk,.}{en gansch Belgi\"e doorzoeken}{om haar te vinden}\\

\haiku{{\textquoteright} Eenigen der vrouwen.}{verbleekten en maakten het}{teeken des kruises}\\

\haiku{Van onder hare.}{wangen vloot een tranenstroom}{over den lessenaar}\\

\haiku{Wij moeten onze,:}{wenschen bedwingen ons vriend}{der tijden maken}\\

\haiku{) Haar bij den schouder,:}{vattend bulderde Hlodwig}{op somberen toon}\\

\haiku{Aurelianus ging;}{ter rechterzijde nevens}{een der kinderen}\\

\haiku{{\textquoteright} Een burger legde,:}{den jonkman de hand op den}{mond hem zeggende}\\

\haiku{- Het naakt zwaard zich over,.}{den schouder leggende trad}{hij veldwaarts in}\\

\haiku{- Ik kon in het eerst;}{aan zulke schandelijke}{lafheid niet gelooven}\\

\haiku{{\textquoteright} morde de huisgraaf,.}{daar hij vol angst en ijzing}{zijn hoofd terugtrok}\\

\haiku{ik heb niet gaarne,.}{dat gij Hlodomar minder}{schijnt te beminnen}\\

\haiku{- Eene enkele bleef}{bij de koninginne met}{de zilveren kom}\\

\haiku{Ingomer gelijkt{\textquoteright} {\textquoteleft}?}{integendeel aan u.Mijn}{haar is toch niet bruin}\\

\haiku{- Hlodwig wist niet, wat,.}{hij voorhad en zag hem met}{verwondering aan}\\

\haiku{het is de eerste,;}{maal dat hij met ons ter wacht}{is opgeroepen}\\

\haiku{eenige druppelen,.....}{er van zijn genoeg om eenen}{os te dooden zegt zij}\\

\haiku{Zij bracht een voorwerp,:}{als een eikel in zijne}{hand daar zij zeide}\\

\haiku{tegen den laster.....}{is de verschooning zelve}{eene beschuldiging}\\

\haiku{{\textquoteleft}Luister, hoort gij de?}{koninginne zelve niet}{om bijstand huilen}\\

\haiku{Arme Ingomer,,{\textquoteright}, {\textquoteleft},!}{mijn lieveling kreet zijach}{mij scheurt het harte}\\

\haiku{Hlodwig sloeg den blik;}{op het loodvervig gelaat}{van zijn stervend kind}\\

\haiku{Kom, red mijn kind, ik,,.}{zal u schatten geven u}{danken u bidden}\\

\haiku{{\textquoteleft}O, mijn Ingomer,,,!}{mijn dierbaar kind neen de dood}{zal u niet hebben}\\

\haiku{Vergiffenis voor,.....}{eene moeder die verdwaalt van}{onzeglijke smart}\\

\haiku{het is omringd van,,,;}{licht het heeft vleugelen het}{juicht het zingt uwen lof}\\

\haiku{Zoo zegt men, dat het;}{leger ontevreden is}{en dreigt op te staan}\\

\haiku{eenen harden blik op:}{Aurelianus en sprak met}{koude bitsigheid}\\

\haiku{Siegebald hield zich;}{nevens den koning en sprak}{gemeenzaam met hem}\\

\haiku{Bij elke Gouw of;}{afzonderlijke bende}{bleef slechts \'e\'en edeling}\\

\haiku{en slechts de stoutsten.}{durfden geheel in zijne}{nabijheid blijven}\\

\haiku{Opstaande, zeide:}{de edeling met statige}{koelheid tot den vorst}\\

\haiku{{\textquoteright} En, Aurelianus,:}{met eenen korten wenk tot zich}{roepend beval hij}\\

\haiku{{\textquoteright} De Gallo-Romein.}{sprong met eenen luiden schreeuw uit}{zijnen zetel op}\\

\haiku{O, God, leg dit kruis;}{der bitterste martelie}{mij op de schouders}\\

\haiku{Vooronderstel, dat,.}{ik waarheid niets dan waarheid}{gesproken hebbe}\\

\haiku{mijnen zegen met.....}{het teeken des kruises op}{zijn voorhoofd schrijven}\\

\haiku{{\textquoteright} {\textquoteleft}Wil ik bevelen,,?}{heer dat men eenig warm water}{in het bad storte}\\

\haiku{zie, of de wachten.}{bij het paleis hunnen plicht}{doen en waakzaam zijn}\\

\haiku{Of wel, men zou hem.}{door de straten leiden als}{eenen onnoozelen dwaas}\\

\haiku{De voorbijgangers.....}{zouden met medelijden}{op hem nederzien}\\

\haiku{onwillig vat ik,;}{mijn zwaard en sla toe om den}{eerschender te dooden}\\

\haiku{Uw gemoed heeft mij,.....}{gelasterd uw geest heeft mij}{die eer ontstolen}\\

\haiku{uwe liefde, het woord.}{uwer genegenheid is haar}{noodig om te leven}\\

\haiku{verhaal mij, wat er,{\textquoteright},.}{is geschied sprak zij hem tot}{de banke leidend}\\

\haiku{zijn naam is Warnfried,.}{en hij hoorde te huis in}{Lominger-Gouw}\\

\haiku{{\textquoteright} Eene uitdrukking van'.}{spijtigen toorn ontstelde}{Lutgardis gelaat}\\

\haiku{Ik had ongelijk,,,;}{u te hoonen Siegebald}{lieve dierbare}\\

\haiku{{\textquoteright} Zij scheen eensklaps te:}{verschieten en zeide met}{angst op het gelaat}\\

\haiku{{\textquoteleft}Siegebald, het zijn,?}{geene blijde gedachten die}{u door het hoofd gaan}\\

\haiku{Ik twijfelde, of;}{ik nog wel een mannenhart}{in den boezem had}\\

\haiku{Ik zou haar alleen?}{ten prooi der wraakgierige}{Heidenen laten}\\

\haiku{Gij zult getuigen,,,.....}{heer dat ik mij zelven noch}{mijn paard heb gespaard}\\

\haiku{{\textquoteright} {\textquoteleft}En heeft onze heer,?}{koning u niet gelast mij}{iets meer te zeggen}\\

\haiku{ontferm U over het,.}{akelig lot van het kind dat}{U werd toegewijd}\\

\haiku{Dienvolgens, langs de:}{groote poort tegen het Forum}{kunnen wij niet gaan}\\

\haiku{{\textquoteright} De koningin wees.}{naar eene kleine tafel bij}{hare bedstede}\\

\haiku{gij weet niet, welke.}{benauwdheid uwe moeder in}{den boezem verstikt}\\

\haiku{{\textquoteright} {\textquoteleft}Beef zoo niet, Maria,{\textquoteright};}{antwoordde de koningin}{met gelatenheid}\\

\haiku{{\textquoteright} Maria wierp zich met;}{dwalende treurnis aan den}{hals der koningin}\\

\haiku{Hoort gij, dat mij een,.}{ongeluk overkomt vlucht dan}{op Gods genade}\\

\haiku{{\textquoteleft}Clothildis, goede,;}{een schrikkelijk ongeluk}{is mij overkomen}\\

\haiku{{\textquoteright} {\textquoteleft}D\'a\'ar, d\'a\'ar in het bed,{\textquoteright}, {\textquoteleft},!}{stamelde de koningin}{eene ademing een zucht}\\

\haiku{{\textquoteright} Als een pijl vloog hij.}{ter deure uit en verdween}{in de duisternis}\\

\haiku{geheug u alleen,.}{den schat van grootmoedigheid}{die zijn hart besluit}\\

\haiku{{\textquoteright} {\textquoteleft}Met u, met u wil,{\textquoteright}, {\textquoteleft}.....}{ik zijn zuchtte Clothildis}{u niet verlaten}\\

\haiku{de meesten echter.}{stonden in de verte bij}{de offerdieren}\\

\haiku{Tot nu was de vlucht;}{der raven en het lot der}{Runen ons gunstig}\\

\haiku{{\textquoteleft}Clothildis, waarom?}{deed gij een kruis bij mijne}{tente oprichten}\\

\haiku{Deze overweging ';}{was als een bliksem doors}{konings geest gegaan}\\

\haiku{Hier opent zich een graf,!}{voor ons allen ook voor ons}{ongelukkig kind}\\

\haiku{{\textquoteleft}Ik wil ten hunnen;}{gunste de knie voor mijnen heer}{en koning buigen}\\

\haiku{dit teeken voor den,:}{koning stellende sprak hij}{op plechtigen toon}\\

\haiku{De burgers hadden:}{zich daar in vaste drommen}{te zaam gedrongen}\\

\haiku{Dit is de echte,.}{naam van Clovis den eersten}{koning van Frankrijk}\\

\haiku{doch ten onrechte,,.}{zooals blijkt uit Gregorius}{Turonensis lib}\\

\haiku{God heet de derde ().}{dag der week Dysendag of}{Dynsdagdies Martis}\\

\haiku{Hij bekleedt in de.}{Noordsche Godenleer eenigszins}{de plaats des duivels}\\

\haiku{Dit is het wapen, ().}{door de Romeinen framea}{Fr. fram\'ee genaamd}\\

\haiku{71, waar er staat {\textquoteleft}The{\textquoteright}.}{AEthelings or chiefs of the}{Angles or Saxons}\\

\haiku{38Vienna is,.}{de stad Vienne op de}{Rh\^one in Frankrijk}\\

\haiku{60De kleine stad,.}{Auxonne op de rivier la}{Sa\^one in Frankrijk}\\

\haiku{73Zie over deze,.}{strafpleging Gregorius}{Turonensis Lib}\\

\subsection{Uit: Volledige werken 40. De kerels van Vlaanderen}

\haiku{Wat mij verhindert,!}{wat mij in den weg staat zal}{ik verbrijzelen}\\

\haiku{zij geschiedde op.}{geheel christelijke en}{stichtende wijze}\\

\haiku{{\textquoteright} vroeg de ridder met.}{eene stem die uit eenen kelder}{scheen op te klimmen}\\

\haiku{{\textquoteright} vroeg Burchard, nadat.}{hij den ridders eenen korten}{groet had toegestuurd}\\

\haiku{{\textquoteright} Eene uitdrukking van.}{ongenoegen trok Burchards}{lippen te zamen}\\

\haiku{{\textquoteright} {\textquoteleft}Dat geloove de!}{duivel indien hij zich wil}{laten bedriegen}\\

\haiku{want zij spreken er:}{van en beslissen den twist}{met de Walsche spreuk}\\

\haiku{Onze berichten.}{zijn niet zoo geruststellend}{als gij het voorgeeft}\\

\haiku{maar hij onderstond.}{deemoedig den uitval van}{zijnen ouden oom}\\

\haiku{Is alle gevoel?}{van eer en plicht eensklaps}{in u gestorven}\\

\haiku{{\textquoteright} {\textquoteleft}Toon u ten minste.}{een weinig lieftallig voor}{jonkver Placida}\\

\haiku{Wel vertraagde hij;}{zijnen stap en wel scheen hij}{soms te willen staan}\\

\haiku{maar de vrijheid, ziet,;}{gij is een kostbaardere}{schat dan het leven}\\

\haiku{Dit hoofd is nu de.}{machtige en gevreesde}{koning van Frankrijk}\\

\haiku{want zijne scherpe.}{lippen trokken bevend tot}{eenen grijns te zamen}\\

\haiku{Disdir Vos zag hem.}{achterna met een zuren}{lach op de lippen}\\

\haiku{{\textquoteright} In de kleeding dezer;}{Kerels heerschte vooral}{de blauwe verf30}\\

\haiku{Waarom dan voeren,?}{zij zwaarden als waren zij}{edelgeborenen}\\

\haiku{{\textquoteright} {\textquoteleft}Het is eene zaak van,{\textquoteright},.}{tijd heer hertog antwoordde}{de graaf zeer bedaard}\\

\haiku{{\textquoteright}, zeide de graaf, het, {\textquoteleft};}{hoofd schuddendedie toestand}{zal veranderen}\\

\haiku{{\textquoteleft}Het lust mij heden.}{niet den bedroevenden kant}{der dingen te zien}\\

\haiku{De dikke baard die;}{hem op de borst hing begon}{reeds te vergrijzen}\\

\haiku{Blijf immer dicht bij,,,,}{mij Strena Komt kinderen}{geeft mij elk eene hand.}\\

\haiku{Er moet een einde!}{aan onze schandelijke}{lijdzaamheid komen}\\

\haiku{hij zal hopen haar.}{te beminnen en hij zal}{er in gelukken}\\

\haiku{Tusschen u en mij;}{en uwen broeder zal het lot}{eenen afgrond delven}\\

\haiku{O, behoede de!}{barmhartige God u voor}{zulk akelig lijden}\\

\haiku{{\textquoteright} {\textquoteleft}Maar laat mij spreken{\textquoteright},,.}{morde Witta hare klacht}{onderbrekende}\\

\haiku{maar gij moet bedaard...}{blijven of ik verlaat u}{oogenblikkelijk}\\

\haiku{Maar de stilte die.}{haar omringde riep haar tot}{bewustzijn terug}\\

\haiku{{\textquoteleft}Wat men te Rijssel.}{heeft besloten zal men hier}{niet veranderen}\\

\haiku{{\textquoteright} {\textquoteleft}Maar is het zoo, mijn,.}{neef laat mij nog eene poging}{bij hem beproeven}\\

\haiku{Nu wil ik van dit}{huwelijk niet meer hooren}{en ik bevestig}\\

\haiku{Een onnoozel kind van,;}{veertien jaar met Kerlenbloed}{in de aderen toch}\\

\haiku{Gij ziet wel, heer proost,.}{dat er met lankmoedigheid}{niets is te winnen}\\

\haiku{{\textquoteright} {\textquoteleft}Mijn eigendom is{\textquoteright},.}{goed bewaard zeide Burchard}{met fieren glimlach}\\

\haiku{En hem niet weder,!}{levend kunnen maken zelfs}{niet in stroomen bloed}\\

\haiku{Hier sprong hij op zijn.}{brieschend paard en drukte het}{de spoor door de huid}\\

\haiku{Gaat nu naar huis, wekt.}{onderwege de Kerels}{en zendt ze herwaarts}\\

\haiku{de beste Runnen...}{zijne sterke leden en}{mannelijken moed}\\

\haiku{Hier steeg Burchard met.}{de Keurmans af en deed de}{Kerels stilhouden}\\

\haiku{Hij is in den burcht,!}{de vuige moordenaar van}{mijnen armen Eric}\\

\haiku{Hij snakt om tusschen.}{de Isegrims als een hunner}{te worden aanvaard}\\

\haiku{{\textquoteleft}Mij spijt het zeer eenen,.}{vriend dus toe te spreken maar}{gij dwingt er mij toe}\\

\haiku{Wordt er bloed tusschen,,.}{ons gestort het valle dan}{op u mher Disdir}\\

\haiku{Een strenge oogslag.}{van den proost dwong hem echter}{weder tot stilte}\\

\haiku{als gij nu zijt acht.}{gij mogelijk wat geheel}{onmogelijk is}\\

\haiku{Robrecht, ik herdenk;}{dat ik veroordeeld was tot}{eeuwige treurnis}\\

\haiku{Maar Ghyselbrecht, die,:}{het had gehoord antwoordde}{op tergenden toon}\\

\haiku{{\textquoteright} kreet Dakerlia op den, {\textquoteleft},!}{toon der diepste wanhoopo}{mijn arme vader}\\

\haiku{Wat Jakob de Leeuw,;}{betreft die had wel waarlijk}{den geest gegeven}\\

\haiku{{\textquoteleft}Alzoo, gij meent het?}{nog mogelijk dat Dakerlia}{Wulf uwe vrouw worde}\\

\haiku{Zijne wonde, die,.}{nog altijd was ontstoken}{is nu gesloten}\\

\haiku{mher Wulf kan niets van{\textquoteright},.}{de afkondiging verstaan}{bemerkte de proost}\\

\haiku{Maar Dakerlia, door eenen,.}{doodelijken schrik aangejaagd was}{niet te bedaren}\\

\haiku{{\textquoteright} riep zij uit, terwijl.}{men haar poogde van de deur}{terug te houden}\\

\haiku{Mocht zij hem dan niet?}{een laatst vaarwel wenschen en}{hem de oogen sluiten}\\

\haiku{hunne bestierders,,.}{jaarlijks door hen gekozen}{noemden zij Keurmans}\\

\haiku{Wij mogen geenen tijd.}{verliezen om ons zulken}{leidsman te geven}\\

\haiku{Op eene vertroosting:}{van Witta antwoordde zij}{met gelatenheid}\\

\haiku{Is er iets dat u,}{belet ons morgen vaarwel}{te komen wenschen}\\

\haiku{{\textquoteleft}De verwijdering,:}{van mher Sneloghe bedroeft}{mij ik beken het}\\

\haiku{Hier nam hij echter.}{afscheid van hen en keerde}{terug naar den burg}\\

\haiku{Ik wil niets gemeens}{hebben met lieden die eenen}{afschuwelijken}\\

\haiku{Overal waren de.}{lieden gevlucht en niemand}{bood ons wederstand}\\

\haiku{{\textquoteleft}De schalken wilden,;}{mij beletten tot u te}{naderen heeren}\\

\haiku{Daar zijn Houtkerels;}{die het lijk van graaf Karel}{allen smaad aandoen}\\

\haiku{wij verzoeken u '.}{naars graven kapelle}{te willen komen}\\

\haiku{Welnu, kastelein{\textquoteright},, {\textquoteleft}?}{vroeg hij hemgaat het werk goed}{voort aan de vesten}\\

\haiku{Men is dus verplicht.}{te denken dat het hun aan}{geenen strijdlust ontbreekt}\\

\haiku{Dakerlia, zoo op de?}{borst van Robrecht tranen van}{liefde stortende}\\

\haiku{ik dank u. Ga nu.}{tot den heer kastelein en}{draag hem uwe boodschap}\\

\haiku{Si ne connens niet,.}{ontgangen Si ne dogen}{niet sonder bedwanc}\\

\haiku{zooals ik doelmatig.}{zal oordeelen om hare}{hand te bekomen}\\

\haiku{Mher Van Praet zette.}{zich neder op eenen stoel en}{veegde zijn zweet af}\\

\haiku{God zegene u,,.}{mher Vos die ons ter hulpe}{komt in onzen nood}\\

\haiku{Weet gij wat ik hem?}{met saamgevoegde handen}{heb toegeroepen}\\

\haiku{Hij liet zich op de;}{knie\"en vallen en smeekte}{reeds om genade}\\

\haiku{Daar stonden voor de.}{poort een twintigtal Kerels}{op hen te wachten}\\

\haiku{{\textquoteright} {\textquoteleft}Maar, vriend Sneloghe,{\textquoteright},.}{wanhopen moogt gij toch niet}{murmelde Ludgard}\\

\haiku{Is de moordenaar,?}{Burchard niet van het bloed der}{Erembalds evenals ik}\\

\haiku{hij was belast, tot.}{het afweren van eenen storm}{in gereedheid was}\\

\haiku{Op de Kerels deed.}{deze afkondiging eenen}{min diepen indruk}\\

\haiku{{\textquoteright} {\textquoteleft}Weigert gij dan aan?}{het vriendelijk verzoek van}{den proost te voldoen}\\

\haiku{Uit den mond van hem.}{die de valschheid zelf is}{vloeit niets dan logen}\\

\haiku{Gij zegt dat ik den?}{vijand onze stad Brugge}{heb overgeleverd}\\

\haiku{Robrecht wilde den;}{wal aan den linkerkant der}{Hofpoort beklimmen}\\

\haiku{doch allengs zakte,.}{het hoofd hem op den schouder}{en hij sloot de oogen}\\

\haiku{Dan keerde hij, als,.}{uitzinnig van schrik terug}{tot achter de poort}\\

\haiku{{\textquoteright} En hij zakte, door,.}{de wanhoop verpletterd op}{eene bank neder}\\

\haiku{Na eene lange wijl,;}{nog naderde inderdaad}{een wapenbode}\\

\haiku{Eenige woorden slechts.}{wenschte hij met den proost}{te verwisselen}\\

\haiku{en het is zoo dat.}{ik belast werd als bode}{tot u te komen}\\

\haiku{Indien God over uw,.}{leven had beschikt dan wierd}{uw graf het mijne}\\

\haiku{Ach, wees goed, geef mij!}{het liefdebewijs dat ik}{u smeekend afbid}\\

\haiku{{\textquoteright} galmde Robrecht, de.}{verkrampte vuist tot Burchard}{vooruitstekende}\\

\haiku{Het verlies van den.}{kastelein liet de Kerels}{zonder opperhoofd}\\

\haiku{Een weinig voor elf.}{uur werd hun de reden dier}{beweging verklaard}\\

\haiku{opdat elk hunner.}{gestraft wierde in de maat}{zijner schuldigheid}\\

\haiku{Eene onduldbare:}{vermaledijding bonsde}{op uit Burchards borst}\\

\haiku{Tot nu toe had geen;}{hunner binnen Brugge zich}{durven vertoonen}\\

\haiku{479.)wenteltrap die zoo.}{nauw is dat men haar man voor}{man moet beklimmen}\\

\haiku{{\textquoteleft}Veldheer, heeft men nu?}{eene betere herberg voor}{mij doen bereiden}\\

\haiku{De refter van het.}{klooster was bijna hoog als}{de beuk eener kerk}\\

\haiku{Dan betuigde hij;}{zijne ontevredenheid}{aan meester Arnold}\\

\haiku{{\textquoteright} {\textquoteleft}Wat mij betreft, ik,;}{ben bereid om het werk voort}{te zetten veldheer}\\

\haiku{weigerden zij, de.}{beukram zou onmeedoogend}{zijn werk voltrekken}\\

\haiku{Zandkorrel dien het,!}{lot mede voert evenals de}{wind een vlokje stof}\\

\haiku{, uwe zuster en u:}{zelven daarboven in de}{armen te drukken}\\

\haiku{Wij smeeken op de!}{knie\"en uwe koninklijke}{grootmoedigheid af}\\

\haiku{, en met ongeduld.}{wachtte een ieder op het}{besluit des konings}\\

\haiku{Men zal zelfs zich niet.}{gewaardigen de Kerels}{te onderhooren}\\

\haiku{Indien de koning?}{van Frankrijk genade wil}{schenken aan Robrecht}\\

\haiku{maar Dakerlia liep tot,:}{hem en weerhield hem terwijl}{zij bevend kermde}\\

\haiku{{\textquoteright} {\textquoteleft}Dit is alles wat?}{gij tot uwe verdediging}{hebt in te brengen}\\

\haiku{{\textquoteleft}Ik zal u dankbaar...,,?}{blijven voor deze weldaad}{en wie weet wie weet}\\

\haiku{zij dienden later.}{tot zoogezegde tooverij}{of waarzeggerij}\\

\section{Antoon Coolen}

\subsection{Uit: Bevrijd vaderland}

\haiku{Stonden wij buiten?}{die tragische crisis der}{democratie\"en}\\

\haiku{De natuur heeft er.}{niet mee te maken en zij}{laat zich niet storen}\\

\haiku{Waarom in den trein,,?}{zoo het mogelijk is den}{ledigen coup\'e}\\

\haiku{Hij is de held in,.}{het gezelschap en zijn vrouw}{deelt in zijn glorie}\\

\haiku{Nu is het eenige,.}{uren diepstil en we voelen}{ons ge{\"\i}soleerd}\\

\haiku{Hij kwam bij ons huis,:}{toen ik niet thuis was en liet}{de boodschap achter}\\

\haiku{Maar de verwoesting,.}{hier is  zooveel jonger}{zooveel vreeslijker}\\

\haiku{Sie haben es hier,.}{h\"ubsch angelegt zei hij}{mij aan de voordeur}\\

\haiku{De menschen op de.}{trottoirs en in de caf\'e's}{kijken langs hem heen}\\

\haiku{Vlaanderen staat nu.}{voor de eerste maal op de}{winnende zijde}\\

\haiku{de kogels fluiten.}{door het hout der bosschen en}{over onze huizen}\\

\haiku{Dit uur der Duitsche.}{overwinning is het uur van}{Frankrijks nederlaag}\\

\haiku{De Gebroeders, De,,.}{Hoop De Eendracht Nooit gedacht}{en De Verwachting}\\

\haiku{De Godsgedachte.}{volk moet in deze dagen}{in vervulling gaan}\\

\haiku{De katholieke.}{priester is de dienaar der}{pauselijke kerk}\\

\haiku{Neen, die Domkerken {\textquoteleft}{\textquoteright}.}{en M\"unsters zijn gebouwdzur}{Andacht und K\"undung}\\

\haiku{Deze moeten hun.}{plaats vinden terzijde van}{altaar en kansel}\\

\haiku{Ziet hier, zegt hij, een.}{lijst van noordsche sterren aan}{den Duitschen hemel}\\

\haiku{dank zij Duitschland,!}{weet de wereld weer dat er}{een Vlaanderen is}\\

\haiku{Engeland mist zulk.}{een propaganda en zulk}{een propagandist}\\

\haiku{nuchterheid, vroomheid.}{en vrijheidszin er vreemd en}{vijandig aan is}\\

\haiku{beb kan die man ook}{wel want hij is bij der thuis}{geweest met zijn vrouw}\\

\haiku{- In Den Haag aan de,}{stations controleeren zij}{bij de uitgangen}\\

\haiku{Voor het eerst hoor ik:}{nu uit dezen kindermond}{de verzekering}\\

\haiku{overal waar ge in.}{die hooge wijdheid kijkt zijn ze}{in groepen bijeen}\\

\haiku{de kleine, goede.}{warmte van licht en feest van}{Zondag en Kerstmis}\\

\haiku{(Zoo angstig ben ik,.}{dat ik mijzelf zoo krachtig}{moet geruststellen}\\

\haiku{Als het eenigen tijd.}{stil is geweest worden de}{vensters geopend}\\

\haiku{De oudste heeft een.}{paar gedoode Duitschers zien}{liggen in den tuin}\\

\haiku{Na Kerstmis was het,...}{bij den arme weer armoe}{en bij den rijke}\\

\haiku{Please, it is a,:}{quarter to eight en zij brengt}{hun warm  water}\\

\haiku{Er is verteld, dat.}{zij de tranen in de oogen}{had bij den aanblik}\\

\haiku{En met elken dag,,.}{dien hij langer duurt wint het}{hart aan zekerheid}\\

\haiku{Reeds kon Eisenhower:}{een proclamatie tot het}{Duitsche volk richten}\\

\haiku{De Duitschers hebben:}{nooit dien onderstroom der groote}{krachten in ons volk}\\

\subsection{Uit: Herberg In 't Misverstand}

\haiku{De schildersbaas en}{drogist vond het juist prachtig}{als hij er weinig}\\

\haiku{Maar ook toen hij recht.}{zat zag hij de Rooy omhoog}{en omlaag wiegen}\\

\haiku{Er waren ook te,.}{veel gasten ze hadden een}{groote ruimte van doen}\\

\haiku{Een der meisjes kwam.}{den gemeenteontvanger}{bij haar wegtrekken}\\

\haiku{'s Morgens bij het.}{wakker worden dacht zij er}{niet dadelijk aan}\\

\haiku{In haar bitterheid.}{werd zij onverschilliger}{voor dat \`andere}\\

\haiku{En omdat ik het,.}{niet heb gewild hoeft hij het}{ook nooit te weten}\\

\haiku{Hij onderging de,.}{welgezindheid en rookte}{tevreden een pijp}\\

\haiku{Ze gebruikte de.}{uitdrukking van de vrouwen}{in haar omgeving}\\

\haiku{Over haar schoot tastte,.}{haar hand om haar schort bij den}{rand op te nemen}\\

\haiku{Op een buiigen.}{dag in den nawinter werd}{het kind geboren}\\

\haiku{Marjanne vroeg wat,?}{er aan de hand was had hij}{een zweer in den hals}\\

\haiku{Het zal uw eigen,.}{ondervinding zijn maar hier}{is het dan niet zoo}\\

\haiku{Toen tikte ze met.}{den scherpen kant van de kaart}{tegen de lippen}\\

\haiku{- Je zult zien als het,.}{kind er is heb je er de}{zachtste vrouw aan}\\

\haiku{Gedurende de}{dagen dat zij weg waren}{rekte Jan Jacob}\\

\haiku{Jan Jacob liet zich,.}{tracteeren ze wilden hem}{schadeloos stellen}\\

\haiku{In 't Misverstand,.}{totdat haar man zijn glas bier}{had leeggedronken}\\

\haiku{Verbeeldde hij zich?}{werkelijk dat hij hier iets}{te gelasten had}\\

\haiku{Daar zat - wacht, hij zou -.}{hem eens even vastpakken daar}{zat zijn revolver}\\

\haiku{Dat de kinderen.}{de school dichterbij hadden}{kon hem niet schelen}\\

\haiku{Ze gleden telkens,.}{naar onder weg ze wisten}{met hun beenen geen raad}\\

\haiku{Van dat sublimaat,,?}{die rattentarwe en dat}{koord dat wisten ze}\\

\haiku{Maar een m\'o\'oi dorp was,,.}{het daar ging niets van af je}{ging eraan hechten}\\

\haiku{Daar keek Anna naar,,.}{in haar opkamertje bij}{de Deysselbloemen}\\

\haiku{Thuis gebruikte haar.}{moeder voor het middageten}{een tafellaken}\\

\haiku{- Ik kan mij er niet,.}{mee bemoeien ieder is}{baas over zijn eigen}\\

\haiku{Daar zat zij, in de,.}{kilte maar zij had er een}{besloten toevlucht}\\

\haiku{Zij wist niet precies.}{wat haar in die vrouw aantrok}{en vertrouwen gaf}\\

\haiku{Zij zat op een stoel.}{tegen den muur en hield het}{hoofd ver achterover}\\

\haiku{De moeder nam hen,.}{stil bijeen Thijs Rooyakkers}{zei niets tegen hen}\\

\haiku{Als een man niet deugt,.}{dan zijn vrouwen er zacht en}{vriendelijk tegen}\\

\haiku{Ze sloeg er planken.}{voor en timmerde die vast}{in de kozijnen}\\

\haiku{Want schadepostjes,.}{nam hij zelf die schoof hij niet}{op de klanten af}\\

\haiku{En daar, ge krijgt het.}{beneden den prijs dien het}{me zelf heeft gekost}\\

\haiku{Toen ze drie weken.}{weg was geweest kwam Anna}{een avond weer terug}\\

\haiku{Bij dit werk werden.}{haar gedachten heel kalm en}{heel vriendelijk}\\

\haiku{Als er kinderen.}{bij waren deden  de}{grooten voorzichtig}\\

\haiku{- H\`e, h\`e toch ja, dat,!}{is nogal een mooi zeggen}{dat ge u niet schaamt}\\

\haiku{Maar na een tijdje,.}{hield de benauwdheid op hij}{kwam er weer doorheen}\\

\haiku{Toen Anna en haar '.}{mans avonds thuiskwamen was}{het gaan regenen}\\

\haiku{Even later kwam zij,.}{terug met haar groot kerkboek}{dat rood op snee was}\\

\haiku{Zij haalde er de.}{meegebrachte bidprentjes}{van haar vader uit}\\

\haiku{In de hitte van.}{den zomer wiedde Anna}{mee in de mangels}\\

\haiku{Bij den akkerzoom.}{nam hij met een aanloop den}{sprong over de bermsloot}\\

\haiku{Toen vroegen ze Jan,.}{Jacob hoe het bij hem thuis}{was afgeloopen}\\

\haiku{Als hij dan ziek werd,.}{alleen door bij haar te zijn}{dan moest hij maar gaan}\\

\haiku{Het is gedwongen.}{door het gebruik en omdat}{iedereen het doet}\\

\haiku{Dat was niet om den,.}{jongen alleen Marjanne}{begreep dat heel goed}\\

\haiku{Het heele huis was,.}{vol van het kind het kind was}{in alle dingen}\\

\haiku{En Lodewijk zet.}{voor de gelegenheid zijn}{hoed tot aan zijn rug}\\

\haiku{Na den doop liepen.}{ook Martiens schoonbroers met een}{sigaar in den mond}\\

\haiku{- Als ge dat ook maar,.}{moet aanzien zooals zij smijt met}{geld en met alles}\\

\haiku{Hij hoorde nog die,:}{vraag in de stem van zijn vrouw}{toen ze zijn naam zei}\\

\haiku{- die boer Martien! - Nee,,.}{voor een postzegel was het}{niet zei hij nog eens}\\

\haiku{De Rooy  zag, hoe.}{den boer het zweet in straaltjes}{langs het gezicht liep}\\

\haiku{Zij vroeg niet, of hij.}{het had gekund en of het}{moeilijk was geweest}\\

\haiku{Op het oogenblik.}{van overlijden was hij lid}{van de fanfare}\\

\haiku{Nadien zat zij weer.}{met groote oogen te kijken naar}{dat ijlende kind}\\

\haiku{Hij bleef ook lang op.}{een stoel bij het bed van den}{jongen zitten}\\

\haiku{Zij deed, of dat van,.}{het hijgen kwam omdat zij}{hard had geloopen}\\

\haiku{Marjanne vroeg, een,:}{beetje angstig maar zij deed}{daarbij zoo gewoon}\\

\haiku{Maar de drogist en.}{schildersbaas werd daardoor een}{beetje geprikkeld}\\

\haiku{Maar omdat ze laat,.}{begonnen bleven ze laat}{in den nacht zitten}\\

\haiku{Notaris Duchateau.}{stond aandachtig en ernstig}{naar hem te kijken}\\

\haiku{Er is geene cent van,,,,!}{jou bij geene c\`ent geene c\`ent hoort}{ge nog niet z\'o\'oveel}\\

\haiku{- En daarom zal ik:}{er niet over uitscheiden voor}{ik mijnen zin heb}\\

\haiku{Zij stond op, gaf hem,,:}{het kind dat zijn handjes naar}{hem uitstak en zei}\\

\haiku{- Als we goed boeren,.}{dan koopt ge dien grond er nog}{bij voor ontginning}\\

\haiku{Maar zooveel geeft ge, '.}{toch wel om geld dat get}{onze wilt hebben}\\

\haiku{Wie bijgeboden,.}{had werd opgeroepen hij}{moest zijn naam zeggen}\\

\haiku{Hij wrong zijn sigaar,.}{in den aschbak rond klopte}{telkens asch eraf}\\

\haiku{Dan, alsof  het,:}{een verpletterend vonnis}{was brulde Fleskens}\\

\haiku{Met den verkooper.}{hoefde niet lang beraad te}{worden gehouden}\\

\haiku{Hij kwam binnen bij.}{de kinderen Deysselbloem}{en vroeg naar Martien}\\

\haiku{Hij stak een sigaar.}{op en bood uit zijn koker}{Martien er een aan}\\

\haiku{De fanfare Sint '.}{Cecilia bracht hems avonds}{een serenade}\\

\haiku{Toen het sluitingsuur.}{sloeg trokken de feestende}{muzikanten af}\\

\haiku{De Rooy sloot en ging.}{daarop het gezelschap voor}{met zijn fakkels}\\

\haiku{- Nee, dat is aardig,,.}{zei hij dat je ondanks je}{voet gekomen bent}\\

\haiku{Met een paar dagen.}{ging hij naar de griffie voor}{de eedsaflegging}\\

\haiku{Tegen Jan Jacob,,.}{die liet merken dat hij het}{zag glimlachte hij}\\

\haiku{En die kunsten en.}{malligheden met dat touw}{en die revolvers}\\

\haiku{Hoe waren nu in '?}{s hemelsnaam die twee bij}{elkaar gekomen}\\

\haiku{Want Wilde Maria,.}{kon hen aankijken dat ze}{wonder wat dachten}\\

\haiku{Het oudste was van,.}{school dat kreeg lange beenen en}{schoot de hoogte in}\\

\haiku{De moeder zat bij,.}{het hoofdeind Jan Jacob zat}{bij het voeteneind}\\

\haiku{Overdag, in het licht,.}{hield hij ervan buiten in}{de velden te zijn}\\

\haiku{In alle weien, ',.}{staat het vee gezegend in}{t licht en graast}\\

\haiku{Maar ginds aan den berm,.}{stond zijn knecht te zwaaien dat}{Martien zou komen}\\

\haiku{Hij liep den bunder,.}{af sprong uit de ruigte van}{den berm over de sloot}\\

\subsection{Uit: Hun grond verwaait}

\haiku{In de wei en langs}{den wegkant vreet ze haar gras}{en onder den balg}\\

\haiku{In de verte is.}{de hemel vol van een kleur}{als van roode wijn}\\

\haiku{Daar op de tafel,.}{staat het komke waaruit hij}{heeft gedronken}\\

\haiku{Daarom zee ze ja.}{en ze was verstandig en}{wijs in dat ja}\\

\haiku{Johannes van Goch,.}{na den arbeid treedt zijn huis}{tegemoet zijn vrouw}\\

\haiku{Johannes had van '.}{t zomer zijnen klot van}{zijn veldje gehaald}\\

\haiku{da de hemel wit.}{was van sterren hun tweede}{kind wier geboren}\\

\haiku{Zijn vader en zijn.}{moeder vonden dat schoon en}{lachten van plezier}\\

\haiku{'t Was hem aan zijn,, '.}{hart gegaan waarachtig maar}{t ging niet anders}\\

\haiku{De avond buigt over ons.}{en de lente geurt en streelt}{ons tot op het bloed}\\

\haiku{Verdomme, zee Piet,.}{hij had weer wat vergeten}{en hij gong terug}\\

\haiku{Vroeg of laat had ik,.}{hier toch weg gemoeten hier}{kom ik nooit vooruit}\\

\haiku{- Ik zeg maar zoo, zee, '.}{Johannes dak pleizier}{van mijn jongens heb}\\

\haiku{Achterna maakt hij,.}{zijn doos open zoekt en haalt er}{een envelop uit}\\

\haiku{Zoo praten ze, over '.}{ent weer wa woorden mee}{stiltes ertusschen}\\

\haiku{hij en was hij mee '.}{een vergeeflijke zuchtn}{bietje apart te zijn}\\

\haiku{Da was Eimertje. ',.}{Schoonemanst Wier verteld}{hoe dat gegaan was}\\

\haiku{Toen ie Lammeke,,.}{zag toen liep ie er heen mee}{zijnen manken poot}\\

\haiku{- Ah, zegt Lodewijk,?}{begint de industrie tot}{hier door te dringen}\\

\haiku{Maar Lodewijk gaat -,?}{er op door Toen ze d'r over}{sprak wat zei ze toen}\\

\haiku{Waar zou nou wel juist?}{het plekje zijn waar ze toen}{hebben gezeten}\\

\haiku{De menschen worre.}{ouder en het aanschijn van}{ons liefland vernieuwt}\\

\haiku{Nee, Friedus bracht met.}{deze streek zijn ouders in}{groote moeilijkheden}\\

\haiku{Na 't heengaan van,,.}{Lodewijk indertijd was}{zijn verzet verzwakt}\\

\haiku{Vader komt binnen.}{en komt den buil tabak van}{den schouwrand halen}\\

\haiku{- Kijk 'es, zegt meester,,.}{Frunt maar dat blijft onder ons}{ik heb mijn plannen}\\

\haiku{En kijk es, ik heb,.}{je moeder hier gehad die}{kwam over je praten}\\

\haiku{Als hij thuiskomt kijkt,}{hij er zijn moeder op aan}{wa ze in haren}\\

\haiku{Mijn oogen kijken zacht.}{en sluiten zich onder de}{aandacht van jouw oogen}\\

\haiku{Bij iedere dood, '.}{zijn tranen van hen die over}{n hortje volgen}\\

\haiku{En ze stuurden  ,.}{hem een bidprentje voor hem}{en zijne vrouw}\\

\haiku{Te lente gaat hij '.}{trouwen meet durske van}{Verleijsdonke}\\

\haiku{zegt Friedus, 't ja, ',!}{t ja waarom heb ik mijn}{vrouw niet meegebracht}\\

\subsection{Uit: Jantje den schoenlapper en zijn Weensch kiendje}

\haiku{Alexandrine loopt de,,:}{achterdeur in de keuken}{door het gangske in}\\

\haiku{- Wier moeten essen,,.}{zegt Jan op zijn weensch als hij}{in de keuken komt}\\

\haiku{Ze droogt zorgvuldig.}{af en zet de borden op}{de tafel ineen}\\

\haiku{Ze hoeft heelemaal,.}{niet voorzichtig te zijn de}{borden zijn van blik}\\

\haiku{Boven haar knerpt en.}{knaagt een houtworm met korte}{rhythmische geluidjes}\\

\haiku{- Wat is tafel bij,.}{ellie vraagt Jan en hij slaat}{op het tafelblad}\\

\haiku{Ziedege, als ge,.}{goei water hebt dan zette}{ge goeje koffie}\\

\haiku{Xandrieke wier uit.}{den hof geroepen en de}{brief wier open gemaakt}\\

\haiku{ergernis, waar ze,.}{maar meent da de deugd geweld}{wordt aangedaan}\\

\haiku{Ja, zegt Gondeke,,.}{pak d'r kleerkes in ik breng}{ze bij de zusters}\\

\haiku{De slag kletst op het.}{jonge bruine vleesch van}{het kinderwangske}\\

\haiku{De hemel is weer,.}{diepblauw geworre over de}{vochtige aarde}\\

\haiku{daar zat een merel,}{in de wije wereld waarin}{het avond geworren}\\

\subsection{Uit: Kerstmis in de Kempen}

\haiku{Daar ging het niet over,,.}{zei hij maar recht was recht en}{reden was reden}\\

\haiku{Den oudste te zien,,,.}{zoo'ne groote schoone mensch dat}{geeft oe gedachten}\\

\haiku{Daar zat den ouden.}{Keunen zachtjes bij nee te}{schudden met het hoofd}\\

\haiku{Hij zette den kraag,.}{van zijnen jas op trok de}{pet over zijnen kop}\\

\haiku{In de volte kwam.}{Hanna van Bommel en zocht}{plaats op een stoelke}\\

\haiku{Zij trad in den herd,,.}{van het avondhuis daar wachtte}{Eimerd hare mensch}\\

\haiku{ik zeg het oe in,,.}{vertrouwen ge haalt er geen}{woord meer uit ze zwijgt}\\

\haiku{De man en de vrouw.}{vieten hunnen verket en}{aten uit den schotel}\\

\haiku{Onder den eten keek,.}{zij op van haar handen zij}{zag altijd het paard}\\

\haiku{Maar de plaveien,.}{trapte het kapot dat kon}{het niet voorkomen}\\

\haiku{Toen ging ze bleek en.}{met slepende voeten den}{akker af naar huis}\\

\haiku{Langzaam trok zij een.}{kruis over het rood cijfer van}{den eersten Kerstdag}\\

\haiku{Eimerd zat achter,.}{de plattebuis waarin hij}{een goed vuur stookte}\\

\haiku{{\textquoteleft}ga es op zij{\textquoteright}, zei, {\textquoteleft} ' '.}{zega d\'a\'ares zitten da}{k uit de weeg kan}\\

\haiku{- Ja, maar wie zegt ou,'?}{da gij oe eigen niet net}{zoo goed tekort doet}\\

\haiku{Hij lachte gegriefd.}{en troeste den geplooiden}{gerimpelden mond}\\

\haiku{- De naakten kleeden,.}{voor die dat verzuimt zal het}{er leelijk uitzien}\\

\haiku{- O, Kuunders, zei ze,,,.}{van de Panneschop ja daar}{heb ik af gehoord}\\

\haiku{Thuis dacht Govert Kuunders,.}{eraan waar de goejvrouw}{kon zijn gebleven}\\

\haiku{- 't Is het eenigste,.}{wat ik oe aan kan bieden}{het is oe gegund}\\

\haiku{Druk klopte hij de.}{asch van zijn sigaar in den}{koperen aschbak}\\

\haiku{Of er ook niet wat.}{geld vastgelegd moest worden}{voor zielemissen}\\

\haiku{gij daar en gij daar,.}{en gij daar en doorloopen}{en naar oe plaats gaan}\\

\haiku{Anne-Marie,.}{zei dat het toch met een paar}{dagen Kerstmis was}\\

\haiku{Hij maakte een breed,.}{overhangend dak van stroo dat}{hij bond met poppen}\\

\haiku{Daar waren 'nen hoop,.}{dingen waar zij hem nooit toe}{had kunnen krijgen}\\

\haiku{Veel hadden wij thuis,:}{niet maar met Kerstmis moest het}{er op overschieten}\\

\haiku{Nu daalde er, als,.}{met de sneeuw toen weer zoo iets}{goeds over de wereld}\\

\haiku{En de verre, hoog,.}{canada's die met eenen bocht}{van den weg meegaan}\\

\subsection{Uit: Lentebloesem}

\haiku{- Ik ben jouw moeder,,,,,,.}{ik ik ik ja kijk maar zoo}{lieve wijzeman}\\

\haiku{- In mijn oogen moogt ge,.}{n\'o\'oit anders zien dan liefde}{voor jou dan vreugde}\\

\haiku{Fritsje durfde.}{niet hard te stappen en geen}{geluid te maken}\\

\haiku{Want dat gescheurde.}{en gebarsten plafond was}{een klein wereldje}\\

\haiku{Vanavond gaan we met.}{z'n vieren naar een concert}{in de groote Staarzaal}\\

\haiku{Lieve grootvader,,.}{ik moet eindigen want daar}{komt net oom binnen}\\

\haiku{En wat heb ik een.}{leed over dit v\'o\'or mijn komst zoo}{leedlooze huis gebracht}\\

\haiku{- Oom Herman pakte, {\textquoteleft},{\textquoteright}.}{m'n handenje t'aime je}{t'aime zei hij snel}\\

\haiku{Miep E.J. Vanberghe.}{en   Max Wilde}{Ingenieur}\\

\haiku{Den Bosch, \} 25 Juni -, \} -}{19 Maastricht 25 Juni 19}{Kwezelke I}\\

\haiku{Hij lachte nu nog,:}{meer en antwoordde dat die}{rozen duur waren}\\

\haiku{{\textquoteleft}Hier,{\textquoteright} zei hij, {\textquoteleft}rozen,...{\textquoteright} {\textquoteleft},{\textquoteright}.}{LeentjeVoor de Lieve Vrouw}{kwam zij verlegen}\\

\haiku{Ze leefden stil en:}{rustig voort van den eenen dag}{in den anderen}\\

\haiku{Wa' zoude gij er,,,?}{van zeggen Peer van te gaan}{trouwen wij twee\"en}\\

\haiku{Dat was een angst en,,.}{een pijn een dag lang en op}{haar bed schreide ze}\\

\haiku{Toen begon hij weer,.}{te praten wat moeilijk zijn}{woorden zoekende}\\

\haiku{Ter bemoediging {\textquoteleft},.}{drukte hij toen wat vaster}{haar hand.Nou dag dan}\\

\haiku{Er viel een streepje,.}{zon naar binnen juist op het}{kruis boven de schouw}\\

\haiku{Hij met zijn goed hart -',,;}{t\`och nee immers hij is niet}{slecht hij is niet slecht}\\

\haiku{ik u geluk, op,{\textquoteright}}{dezen gedenkwaardigen}{dag in uw leven}\\

\haiku{Ze lachten luid, ze,.}{praatten zoo druk dooreen ze}{tierden rumoerig}\\

\haiku{Het gouden paar zat,!}{te glunderen en Manders}{had pleizier nee maar}\\

\haiku{Zij zongen, maar dat.}{was veel meer schreeuwen en maar}{tieren van pleizier}\\

\haiku{Zou hij nu staan te,?}{droomen bij dat witte ruw}{geschilderde kruis}\\

\haiku{De vrouw van zijn zoon,,!}{een flink vrouw-mensch maar een}{karnalli voor hem}\\

\haiku{als ik binnen kwam,.}{scheen hij te voren mijn komst}{te hebben vermoed}\\

\haiku{Toen besefte ik,}{plotseling hoe volstrekt en}{ontzettend alleen}\\

\haiku{Misschien, dacht hij, ligt;}{het verscholen onder het}{blaadje van een plant}\\

\haiku{en als zijn oogen het,:}{raadsel niet vinden konden}{zei hij bij zichzelf}\\

\haiku{Het was heerlijk bij '.}{dat zomerfeests morgens}{wakker te worden}\\

\haiku{Als ze slapen ging,,,.}{droeg moeder op haar rug haar}{naar haar slaapkamer}\\

\haiku{Hij lachte opnieuw,,,.}{een warme prettige}{lach knikte haar toe}\\

\subsection{Uit: De man met het Jan Klaassenspel}

\haiku{Verandert deze?}{kleine gebeurtenis het}{uitzicht van dit land}\\

\haiku{De kleine man laat,.}{dit beeld voorbijgaan hij noemt}{nadien een nummer}\\

\haiku{Ze kijken hem met,.}{hun groote doode oogen maar eens}{aan daar lacht Proens bij}\\

\haiku{- Nog nie, zee Nolda,.}{al waarde-gij den eenigsten}{mensch op de wereld}\\

\haiku{Toen nadien Proens weer,:}{bij Corneliske over den}{vloer kwam toen zee Proens}\\

\haiku{Anders, als ik maar,.}{eenen stoel krijg voor vannacht dan}{ben ik tevreden}\\

\haiku{nen toevallijder,.}{hij glijdt langs den deurstijl af}{langzaam op den grond}\\

\haiku{- Ge kant zeggen wa',,.}{ge wilt zegt Nolda hij kan}{tenacht hier blijven}\\

\haiku{Zij blijven beiden,}{met dien man bezig en met}{de vraag wat voor eene}\\

\haiku{Nolda is ook al,.}{zoo'ne gek eene zij gaat wat}{wroeten in den zak}\\

\haiku{Hij had niet veel te,.}{strijden zij stond aan zijnen}{kant in dezen strijd}\\

\haiku{Ze kost ook ruzie.}{maken en hem wegduwen}{met den elleboog}\\

\haiku{Nadien, het groote brood.}{voor de borst geplant snijdt ze}{driftig de sneden}\\

\haiku{Corneliske ging,.}{er tegen in maar hij kreeg}{zijn woorden terug}\\

\haiku{En Proens lachte van,.}{de blijdschap dat hij er niet}{mede getrouwd was}\\

\haiku{- Als ik dan morgen,,.}{oe fiets kan leenen vraagt hij om}{er heen te fietsen}\\

\haiku{Haar klompen klijven.}{en glijden in het slijk van}{haar pad langs den weg}\\

\haiku{Die kan niet lachend,.}{gaan liggen om over zich heen}{te laten schieten}\\

\subsection{Uit: De ontmoeting}

\haiku{we doen het om de.}{verloren waardigheid een}{beetje te redden}\\

\haiku{Maurits, thuis, had de.}{krant voor zich en staarde naar}{dat korte bericht}\\

\haiku{Toen ging hij het pad,.}{op naar den landweg waarlangs}{hij thuis zou komen}\\

\haiku{Hij hield zijn paard in,.}{legde de leidsels neer en}{ging den akker af}\\

\haiku{En eten, hij stond zelf.}{verbaasd dat hij niet meer te}{verzadigen was}\\

\haiku{Ze gingen zwijgend,.}{tot zij uit het gezicht der}{anderen waren}\\

\haiku{- Waaraan kan ik het,?}{te danken hebben dat ik}{vrij gekomen ben}\\

\haiku{De jonge boer deed.}{zijn best om zijn aandoening}{niet te verraden}\\

\haiku{- Zij had ons vroeger, -}{al eens een paar keer voor Don}{Jos\'e gewaarschuwd}\\

\haiku{- Barends zal je op,,}{de hoogte hebben gebracht}{zei Tom tot Maurits}\\

\haiku{Maar we kunnen dien.}{moord ook heelemaal buiten}{beschouwing laten}\\

\haiku{Wij nemen aan, dat.}{je een priester bij je wilt}{hebben voor je sterft}\\

\haiku{Nadat de deur was.}{dichtgedaan viel opnieuw een}{diepe stilte in}\\

\haiku{(naam) \_\_\_\_\_\_\_\_\_\_\_\_\_\_\_\_\_\_\_\_ \_\_\_\_\_\_\_\_\_\_\_\_\_\_\_\_\_\_\_\_ (adres) \_\_\_\_\_\_\_\_\_\_\_\_\_\_\_\_\_\_\_\_ (woonplaats) \_\_\_\_\_\_\_\_\_\_\_\_\_\_\_\_\_\_\_\_ meent, \_\_\_\_\_\_\_\_\_\_\_\_\_\_\_\_\_\_\_\_() {\textquoteleft}{\textquoteright}.}{dat de schrijverster is van}{De Ontmoeting}\\

\subsection{Uit: Peerke den Haas}

\haiku{Het vertrek hangt vol,,.}{rook die naar de lamp trekt waar}{hij dik om heen zweeft}\\

\haiku{Het bed, was is het,.}{een bed met wat todden van}{dekens en baalzak}\\

\haiku{Zij let niet op de,.}{bedoeling die hij daarmee}{kan hebben gehad}\\

\haiku{Peerke gaat weer bij,.}{het raam staan daar staat hij voor}{evenveel te kijken}\\

\haiku{Peerke is van den,.}{weg af gegaan de ruigte}{en het hooge bund in}\\

\haiku{Ze komen in de,,.}{Zwaan bij Jans de vierde vrouw}{van Jan den Trouwer}\\

\haiku{Ze staat daar nou te,.}{blinken en te lachen ze}{geeft nergens niks om}\\

\haiku{Een groote vlag, groen als,.}{de zee en met franje in}{allerlei kleuren}\\

\haiku{- Zijde gij zat, dat?}{gij mee oe goej dinge aan}{hebt liggen vallen}\\

\haiku{Hij gaat onder het.}{raam aan de tafel zitten}{en kijkt langs de hor}\\

\haiku{De tanden blijven,.}{er in steken zoodat Peerke}{met kracht moet trekken}\\

\haiku{Zij heeft zeker geen,.}{besef of ze schuld heeft aan}{de dingen of niet}\\

\haiku{Frans den Heete wordt.}{in het licht van de lamp eens}{extra bekeken}\\

\haiku{en 't plekte van ', ',.}{t bloedt was klaar bloed wat}{er op zijn gat zat}\\

\haiku{Dat was voor Frans den:}{Heete eenvoudig genoeg}{om te vertellen}\\

\haiku{- Nee, natuurlijk, hier.}{doe je geen mond open en kun}{je geen tien tellen}\\

\haiku{Hij salueert met.}{slappe vingers en wil ook}{hier parmantig doen}\\

\haiku{Peerke zijn vrouw werd,.}{gehoord ze hoefde den eed}{niet af te leggen}\\

\haiku{De vernieling, ja,,,,.}{die was zoo de kruiwagen}{het schuurke de ruit}\\

\haiku{Hij zette een fijn,}{gouden brilleke op en}{zette het weer af}\\

\haiku{Hij heeft gezeten,.}{hij heeft die schande over hen}{allebei gebracht}\\

\haiku{Jan kwam er wel, hij,.}{had meer van die huiskes tot}{in de voorpeel toe}\\

\haiku{Er vielen wat groote,.}{vlokken die waren verdwaald}{tusschen den regen}\\

\haiku{Toen hield de regen.}{op en sneeuwde het drukker}{in de stilte}\\

\haiku{Nou moest Ciska nog.}{even den sleutel naar Jan den}{Trouwer gaan brengen}\\

\haiku{Alexandrine loopt de,,:}{achterdeur in de keuken}{door het gangske in}\\

\haiku{Ze droogt zorgvuldig.}{af en zet de borden op}{de tafel ineen}\\

\haiku{'s Avonds, als de lamp,,.}{brandt raadpleegt Jan zijn grootboek}{dat kleine boekske}\\

\haiku{Ziedege, als ge,.}{goei water hebt dan zette}{ge goeje koffie}\\

\haiku{Xandrieke wier uit.}{den hof geroepen en de}{brief wier open gemaakt}\\

\haiku{ergernis, waar ze,.}{maar meent dat de deugd geweld}{wordt aangedaan}\\

\haiku{Ja, zegt Gondeke,,.}{pak d'r kleerkes in ik breng}{ze bij de zusters}\\

\haiku{daar zat een merel,}{in de wije wereld waarin}{het avond geworren}\\

\haiku{Het gezicht van den,,.}{jongen boer ontspande zich}{breed tot een lach}\\

\haiku{Het was zoo onnoozel,.}{dat zij den veger niet uit}{de handen legde}\\

\haiku{- Ja, daar moet ge eens,.}{ernstig over denken wat het}{zal moeten worden}\\

\haiku{Helder waaien de.}{tippen van den neteldoek}{en de baker lacht}\\

\haiku{Alla, met een breed,.}{gebaar nam hij een stoel ook}{Frederik lachte}\\

\haiku{Alsof zij koorts had,.}{zat zij nadien bij zijn bed}{te klappertanden}\\

\haiku{Niets is beter dan,,.}{een huishouden kinderen}{een levendig huis}\\

\haiku{- Weet ge, waarom ons,? -?}{moeder zei dat ik u naar}{huis moest brengen Nee}\\

\haiku{Verandert deze?}{kleine gebeurtenis het}{uitzicht van dit land}\\

\haiku{De kleine man laat,.}{dit beeld voorbijgaan hij noemt}{nadien een nummer}\\

\haiku{en het wiegelt van.}{verbeeldingen een bietje}{gek in hunnen kop}\\

\haiku{Ze kijken hem met,.}{hun groote doode oogen maar eens}{aan daar lacht Proens bij}\\

\haiku{Hij lachte rustig,,.}{alsof hij geloofde dat}{hij onkwetsbaar was}\\

\haiku{- Nog nie, zee Nolda,.}{al waarde-gij den eenigsten}{mensch op de wereld}\\

\haiku{Toen nadien Proens weer,:}{bij Corneliske over den}{vloer kwam toen zee Proens}\\

\haiku{Zij blijven beide,}{met dien man bezig en met}{de vraag wat voor eene}\\

\haiku{Nolda is ook al ',.}{zoone gek eene zij gaat wat}{wroeten in den zak}\\

\haiku{De oude man heft.}{in aandacht het hoofd op en}{ziet naar den zolder}\\

\haiku{Kobuske de Pint.}{en den  koster zijn daar}{praat over gaan maken}\\

\haiku{Hij had niet veel te,.}{strijden zij stond aan zijnen}{kant in dezen strijd}\\

\haiku{Ze kost ook ruzie.}{maken en hem wegduwen}{met den elleboog}\\

\haiku{Nadien, het groote brood,.}{voor de borst geplant snijdt ze}{driftig de sneden}\\

\haiku{Corneliske ging,.}{er tegen in maar hij kreeg}{zijn woorden terug}\\

\haiku{Ge hebt nog nooit eenen,,.}{mensch gehoord waar ge zoo om}{moet lachen zee ze}\\

\haiku{- Als ik dan morgen,,.}{oe fiets kan leenen vraagt hij om}{er heen te fietsen}\\

\haiku{Want ze hadden nou.}{diejen mensch  uit Scheijndel}{de deur uitgegooid}\\

\haiku{Haar klompen klijven.}{en glijden in het slijk van}{haar pad langs den weg}\\

\haiku{Daar zijn dingen, diep,.}{verholen zij hebben een}{verborgen teeken}\\

\haiku{- D'r zijn er zat die,.}{met tweejen zijn en die mij}{benijden zegt Proens}\\

\haiku{Die kan niet lachend,.}{gaan liggen om over zich heen}{te laten schieten}\\

\subsection{Uit: Stijntje}

\haiku{De groet wordt z\'o\'o lang.}{en z\'o\'o veel herhaald tot ik}{teruggegroet heb}\\

\haiku{Als vader daar vrij,?}{mag zijn waarom scheurt hij dan}{niet alles kapot}\\

\haiku{Ik schroef de dop van:}{de vulpen en de dreumes}{aan mijn knie\"en zegt}\\

\haiku{Die hem kleedt, die een.}{eindeloos geduld met hem}{heeft aan het ontbijt}\\

\haiku{- Eerst gaat de vleermuis,.}{vliegen dan gaat ze op de}{paddestoel zitten}\\

\subsection{Uit: Tsjechische suite}

\haiku{Maar als lustverblijf:}{voor de Tsjechische schrijvers}{is het een sprookje}\\

\haiku{Integendeel, ik.}{geloofde zelf even oprecht}{in het spel als hij}\\

\haiku{Naar dotters en lisch,.}{en de boot riekt een beetje}{naar teer en naar hout}\\

\haiku{Na twee maanden had:}{hij reeds zijn stempel van bloed}{op het land gedrukt}\\

\haiku{Hun ontzetting is.}{wellicht nog grooter dan die}{van de eerste tien}\\

\haiku{De Tsjechische pers.}{vermeldde haar met eenige}{hartelooze regels}\\

\haiku{De binnenkant van.}{de handen tegen den muur}{wordt klam en vochtig}\\

\haiku{de beschaduwde,;}{hellingen in hun plechtig}{fluweelig donker}\\

\haiku{Hij vertelt, hoe diep,.}{de put is en hij zal dat}{eens demonstreeren}\\

\haiku{Niets kan geluidloozer,.}{zijn dan de schokjes waarmee}{die lichtjes aangaan}\\

\haiku{Daar staan ijzeren}{tafeltjes en ijzeren}{stoeltjes en het zit}\\

\haiku{De groote filmstudio's,.}{van Barrandov daar op die}{hoogte buiten Praag}\\

\haiku{En ik moest aan die:}{kenschetsende benaming}{denken van Ehrenburg}\\

\haiku{Nee, zei hij lachend,,,}{maar hij liet wel merken dat}{het hem pleizier deed}\\

\subsection{Uit: Uit het kleine rijk}

\haiku{Als vader daar vrij,?}{mag zijn waarom scheurt hij dan}{niet alles kapot}\\

\haiku{Nu is het anders.}{dan het doorwaaid opene van}{een warme ruimte}\\

\haiku{De verrassingen:}{der werkelijkheid maken}{deze vraag klemmend}\\

\haiku{Ik schroef den dop van:}{de vulpen en de dreumes}{aan mijn knie\"en zegt}\\

\haiku{Die hem kleedt, die een.}{eindeloos geduld met hem}{heeft aan het ontbijt}\\

\haiku{- Eerst gaat de vleermuis,.}{vliegen dan gaat ze op den}{paddenstoel zitten}\\

\haiku{Ik beken, dat ik.}{op die vraag eenvoudig geen}{antwoord voor hem weet}\\

\haiku{hij zien, dat dat toch,:}{niet zoo erg geweest kan zijn}{hij toont zijn handjes}\\

\haiku{Ze hebben 't over,:}{als ze gr\'o\'ot zijn dan is het}{leven pas heerlijk}\\

\haiku{Dan brengen ze het,.}{naar den molen en dan wordt}{er meel van gemaakt}\\

\haiku{De eene rimpelvlaag.}{na de andere rent over}{het smalle water}\\

\haiku{Een anderen keer -,}{betrap ik hem en zooals hij}{opstaand uit zijn spel}\\

\haiku{En hij geeft weer voer,,,.}{de vingers die niets hebben}{strooien het neer}\\

\haiku{De oudste kon er,.}{niet bij zijn die zit met een}{zeeren voet in huis}\\

\haiku{Zwarte Piet moet het,.}{met minder doen die krijgt van}{elk een sigaret}\\

\haiku{Als ze naar bed zijn.}{blijft die kinderkamer vol}{van hun verwachting}\\

\haiku{deurtjes, die open en,,:}{dicht kunnen een mannetje}{er in en vooral}\\

\haiku{Dan komt hij midden,.}{in de kamer merkbaar}{met een bedoeling}\\

\haiku{En weer bekijkt hij.}{onderzoekend van alle}{kanten zijn auto}\\

\haiku{niet zooals ik het mij,,,.}{had voorgesteld wel groen zooals}{moest maar niet d\`at groen}\\

\haiku{dag Lieve Heer, dag,,.}{Lieve Vrouw dag engel die}{mij moet bewaren}\\

\haiku{En later nog, thuis,,.}{en bij de boterham en}{als ze naar bed gaan}\\

\haiku{De handjes op den,:}{rug zeggen ze kort en een}{beetje verlegen}\\

\haiku{Ze moeten bij de.}{hand worden genomen en}{naar de wieg geleid}\\

\haiku{In een nieuw huis, een,:}{nieuwe kinderkamer een}{nieuwe slaapkamer}\\

\haiku{- Geef mij maar een v\'e\'el,?}{grootere straf als ik maar}{w\`el mag luisteren}\\

\haiku{Ze luisteren met,,,.}{de ooren met den mond de}{oogen met heel hun ziel}\\

\haiku{Maar zoo gauw zit hij,.}{weer recht en roerloos want het}{verhaal gaat verder}\\

\haiku{Wat deed Andersen?}{als het oogsttijd was daar over}{de Deensche velden}\\

\haiku{Maar nu het portret,:}{van den grootvader dat in}{Hjalmars kamer hangt}\\

\haiku{ge v\'o\'or op zijn paard}{zitten en dan vertelt hij}{u een verhaaltje}\\

\haiku{- Dat zijn berichten,.}{die langs een draad gaan net als}{deze papiertjes}\\

\haiku{Den volgenden dag}{in denzelfden wind uit het}{zuidwesten komen}\\

\haiku{van een bronchitis.}{hersteld mocht de oudste voor}{het eerst weer buiten}\\

\haiku{Daar liggen de twee,,.}{anderen al even rustig}{maar ze slapen niet}\\

\haiku{Ik weet nog van niets,.}{en ik heb een zekerheid}{vaster dan alles}\\

\haiku{Langzamer rijdt de, '.}{motor maart donderend}{geraas mindert niet}\\

\subsection{Uit: Zegen der goedheid}

\haiku{Zij groet Joachim met,.}{den zachten eenderen groet}{van iederen avond}\\

\haiku{daar de zon flonkert}{met hare gerimpelde}{gensters en glanzen}\\

\haiku{Uit de bergen zijn.}{herdersknapen naar zijne}{kudden gekomen}\\

\haiku{En de stralende.}{blik harer oogen was bijna}{niet te verdragen}\\

\haiku{De oogen der zangers,.}{sloten zich daar half bij toe}{versluierden zich}\\

\haiku{Nereas echter.}{brulde de mannen toe met}{dreunende woorden}\\

\haiku{Daar verliest hij zijn,.}{hooghartig zwijgen voor om}{het te vertellen}\\

\haiku{En hij herhaalt en:}{hij doet alsof hij dit een}{beetje wil zingen}\\

\haiku{Maar Dismas, de zoon,.}{hij gevoelt zijn onrust en}{zwerft langs de wegen}\\

\haiku{Er is een stilte.}{waarin de stemmen voor de}{sprake aarzelen}\\

\haiku{O, het zijn zeker,.}{liederen van rooftochten}{misschien van moorden}\\

\haiku{De sterren en de.}{bevolen winden aan het}{raam houden de wacht}\\

\haiku{- Als ik daar in de,.}{eenzaamheid woon dan komt gij}{mij maar opzoeken}\\

\haiku{Het snelle zeil lag.}{trillend in het kabbelend}{water weerspiegeld}\\

\haiku{Zij hebben zeker,.}{een ver doel daar richten zij}{hun wandelstaf heen}\\

\haiku{Er zijn werklieden,.}{gekomen die hebben zijn}{huis grooter gebouwd}\\

\haiku{En de lach week niet,.}{van hun gezicht toen zij het}{evangelie hoorden}\\

\haiku{Daarom onderbrak.}{Goar zijn reis en keerde naar}{zijn woning terug}\\

\haiku{De goede blik van,.}{zijn oogen verandert er niet}{om zijn gezicht niet}\\

\haiku{Reprobus laat zich.}{op hetzelfde oogenblik}{niet meer bevelen}\\

\haiku{De getrokken en.}{gebaande wegen leiden}{naar den koning toe}\\

\haiku{De kleine koning.}{ziet met inspanning van de}{oogen op naar den reus}\\

\haiku{De duivel is in,.}{het zwart dat zijn lichaam glad}{als een huid omsluit}\\

\haiku{Als Reprobus het,.}{kruis is genaderd is de}{duivel verdwenen}\\

\haiku{Velen, die over de,.}{rivier willen verliezen}{er hun leven in}\\

\haiku{De booze vraagt altijd.}{de onschuld en het  bloed}{der onschuldigen}\\

\haiku{De stemmen van rouw,.}{de steenen in het plaveisel}{der straat verstrakten}\\

\haiku{Ik moet eerst op de.}{boerderij de koeien nog}{gemolken hebben}\\

\haiku{- Gij moet zien, hoe ik,,.}{krullen smeed zegt de smid dat}{zijn nog andere}\\

\haiku{Jan Hamers zegt, over,:}{de onderdeur tegen den}{boer die buiten wacht}\\

\haiku{In al het lawaai.}{gaan de steigeringen van}{het paard verzwakken}\\

\haiku{De handen en de.}{mond van een donkere man}{waren daar dicht bij}\\

\haiku{In de gewelven.}{trilt het eerste licht van den}{nieuwen dageraad}\\

\haiku{In haar slaap blijven.}{de oogen voor de sterren en}{de maan geopend}\\

\haiku{Zeven keer trokken.}{de zon en de sterren hun}{banen over haar heen}\\

\haiku{Hij ging bij haar in.}{het zand zitten en begon}{met haar te praten}\\

\haiku{Het belgerinkel,,.}{dat werd gedragen er was}{een adem die dit droeg}\\

\haiku{Zozimus, zei de,,.}{vrouw ik moet u danken dat}{gij gekomen zijt}\\

\haiku{Hij vond haar lichaam,.}{dat levenloos op den grond}{lag uitgestrekt}\\

\haiku{Nadien hief hij met.}{een ruk den kop overeind en}{schudde de manen}\\

\haiku{De zachte witte,.}{visch was zoet en edel van smaak}{om daarbij te eten}\\

\haiku{Theodotus bleef.}{in dit klein huis verscholen}{totdat de avond viel}\\

\haiku{Ook zij dachten hem,,.}{met deze waarschuwing een}{dienst te bewijzen}\\

\haiku{Hij gaf het gezicht.}{aan blinden terug en den}{gang aan kreupelen}\\

\haiku{De haken sloegen.}{zich opnieuw in zijn vleesch}{en scheurden het open}\\

\haiku{Theodotus, zei,.}{hij ik zie u wijken voor}{de pijnigingen}\\

\haiku{Nu zal ik u eens,.}{van mijnen wijn doen proeven}{geef mij eenen beker}\\

\haiku{Toen zag Fronto het.}{in lakens gewikkelde}{onthoofde lichaam}\\

\haiku{Een warm gekleede.}{vrouw hield hare schreden in}{naast hare vriendin}\\

\haiku{De  Heer is met.}{u. Maria versnelt hun vlucht}{en verblijdt hun hart}\\

\haiku{Hij heeft zijn pet zoo,.}{gek en zijn haren dat is}{iets grappigs aan hem}\\

\haiku{Hij warmt zich daaraan,.}{dicht naar het warm licht van het}{kerstkind gebogen}\\

\haiku{Hij vroeg voor den man.}{mildheid en verzachting en}{de groote verlossing}\\

\section{Dirk Coster}

\subsection{Uit: Marginalia}

\haiku{Het natuurlijke.}{leven is een proces van}{zelfvernietiging}\\

\haiku{Zij vallen het kwaad, '.}{in den man aan daar waart}{belachelijk is}\\

\haiku{Zooveel de mensch reeds,.}{is zooveel verstaat hij van}{de Evangeli\"en}\\

\subsection{Uit: Waarheen gaan wij?}

\haiku{Deze jeugd heeft zich.}{niets te herinneren en}{niets te vergeten}\\

\haiku{E\'en tegenspraak blijft.}{voortdurend boven deze}{regelen zweven}\\

\section{Louis Couperus}

\subsection{Uit: Antiek toerisme}

\haiku{- Schuif de gordijnen,,.}{dicht Tarrar beval hij het}{Libysche knaapje}\\

\haiku{Verandering van.}{spijs is het geheim van een}{goede gezondheid}\\

\haiku{En dan een gekraak....}{van zware koorden over groote}{katrollen heen}\\

\haiku{De ziel is h\'een uit,,....}{deze met kostbare zalf}{gebalsemde huls}\\

\haiku{Het Diversorium.}{bestond uit verschillende}{lage gebouwen}\\

\haiku{- Het is zoo mooi als,,:}{het maar kan meester Ghizla}{zei oom Catullus}\\

\haiku{Romein was aldaar:}{op twee rijen geschaard om}{hem af te wachten}\\

\haiku{Oom Catullus was,:}{blijde dat hij geen oesters}{kreeg en geen pauwbraad}\\

\haiku{Alleen, voelde hij.}{zijn leed en smartelijke}{verdrietelijkheid}\\

\haiku{Maar zij bereikten,.}{den uitgang zonder dat er}{bloed was vergoten}\\

\haiku{Als van een fijne,,,....}{vrouw ijl en dun een schim die}{heen en we\^er bewoog}\\

\haiku{Want Tarrar, niet meer,:}{verbonden zag er uit als}{een kleine wilde}\\

\haiku{Hij wilde op het,,....}{tempeldak gehuld in den}{droomensluier droomen}\\

\haiku{Wees gij het, die zegt!}{uw Eroten de Droomen mijn}{Meester te brengen}\\

\haiku{- Wees gij het, die zegt....}{uw Eroten de Droomen mijn}{Meester te brengen}\\

\haiku{De stad ruischte.}{van muziek en gloeide van}{illuminatie}\\

\haiku{Lucius neeg ter,,.}{aarde zonk op de knie\"en}{en kuste den vloer}\\

\haiku{Ge hebt gedroomd de,.}{vele roovers die geleken}{als dubbelgangers}\\

\haiku{De meester snikte,.}{het hoofd omwikkeld in zijn}{gouden droomsluier}\\

\haiku{Serapis had de.}{hemelsluizen geopend}{en het regende}\\

\haiku{Deze stad, kind, is,.}{een ontaarde stad al is}{zij schoon om te zien}\\

\haiku{in het Muzeum,,.}{in het Serapeum hier}{en te Canope}\\

\haiku{Zoo ik het woord niet,.}{te Memfis vind zal ik}{het verder zoeken}\\

\haiku{Zij naderden nu,.}{de hoofdplaats Sa{\"\i}s hoofdplaats van}{geheel Laag-Egypte}\\

\haiku{Ik ben te oud en,,.}{te dik Lucius voor een}{spokige orgie}\\

\haiku{IK BEN, DIE GEWEEST,,}{IS   IS   EN ZIJN ZAL}{EN NIEMAND HEEFT}\\

\haiku{Zij wrong zich als een,.}{witte waternymf die den}{vloed was ontstegen}\\

\haiku{- Het zijn de zeeroovers,,;}{geweest Lucius zeide}{afwendend Thrasyllus}\\

\haiku{- Meester Thrasyllus zal,!}{het niet tegen spreken hoe}{geleerd hij ook is}\\

\haiku{Maar des Profeten.}{donderende lach deed hem}{deinzen achteruit}\\

\haiku{- Ik dacht aan Kos, mijn,.}{vaderland en of ik het}{wel ooit we\^er zo\^u zien}\\

\haiku{henen boort, als hij,...}{honger heeft en die dorpen}{en steden in slokt}\\

\haiku{- Treed dan binnen in,.}{het Huis van de Zon noodde}{de opperpriester}\\

\haiku{De meester ging de,,.}{woestijn in en Tarrar steeds}{verwonderde zich}\\

\haiku{- De stem in mijn ziel,.}{zelve die de orakels in}{mij deden spreken}\\

\haiku{in zijn muil goten,,.}{lachende de priesters een}{kruik hydromel uit}\\

\haiku{Ik smacht naar een paar.}{malsche oesters en een jong}{gebraden pauwtje}\\

\haiku{Maar zij aten ook bloed,.}{en melk en kaas en er was}{geen ander voedsel}\\

\haiku{- Dan zijt gij terug,,....}{gekeerd en waar gij zijt voel}{ik mij het veiligst}\\

\haiku{Van de jacht terug,....}{komende zag ik daar even}{de lijn van de zee}\\

\haiku{Morgen zijn we te,.}{Dire bij de kolommen}{van Sesostris}\\

\haiku{Zij was voor hem ne\^er.}{gestort en zij snikte en}{kuste zijn voeten}\\

\haiku{Op de kaap, hand voor,....}{de oogen zag Kaleb uit en}{verwonderde zich}\\

\haiku{Luisteren er geen?}{slaven aan de deuren en}{is Kaleb verre}\\

\haiku{Bleek verscheen Kaleb,.}{voor Lucius die hem had}{laten ontbieden}\\

\subsection{Uit: De berg van licht}

\haiku{En hij trok op steenen,.}{bank den knaap tot zich den arm}{om zijn schouder heen}\\

\haiku{Bloed van menschen, van,,...}{dieren duizende menschen}{duizende dieren}\\

\haiku{een azuren lap, die...}{viel voor haar ne\^er met druiping}{als van zwaar water}\\

\haiku{morgenbries woei aan.}{de ziel der Syrische en}{Perzische rozen}\\

\haiku{De rinkelbommen,.}{rammelden en rommelden}{de trommen allen}\\

\haiku{De Menigte zag;}{Bassianus achter den}{Steen ommedansen}\\

\haiku{toen hij we\^er voortrad, -.}{naderde hem de tweede}{deerne Livilla}\\

\haiku{mystiek-helder als -;}{een priester des Lichts zoo zo\^u}{hij altijd blijven}\\

\haiku{Heil de eerwaarde,!}{Moeza en heil Mammea en}{Semiamira}\\

\haiku{hij oefende zijn,:}{Latijn en hij sprak sierlijk}{en zuiver die taal}\\

\haiku{De verhevene.}{Julia Moeza vereert mij}{met haar vertrouwen}\\

\haiku{En dan, hij is een... -?}{Christen Maar de Christenen}{zijn toch in aanzien}\\

\haiku{een eenvoudige,.}{hulde die je zeker ook}{wel zult willen doen}\\

\haiku{Zeg, ze zeggen, dat,:}{die geen mannetje is en}{geen meisje maar niets}\\

\haiku{Een tijger sloeg zijn;}{klauw door de tralies in den}{schouder van een vrouw}\\

\haiku{groote, donkere oogen,.}{die de verre Menigte}{zochten te boeien}\\

\haiku{Het volk koos partij,,.}{voor het Oosten voor de Zon}{voor Helegabalus}\\

\haiku{De gunstelingen,:}{sloegen mantel of toga}{af vlijden zich ne\^er}\\

\haiku{Ik weet meer van je,,...}{af dan je denkt mijn brave}{oude pappias}\\

\haiku{Beste pappias,;}{ik ben een leerling van den}{kn\`apsten Magi\"er}\\

\haiku{Hydaspes, die mij,,,...}{helaas niet gevolgd heeft heeft}{zelve mij geleerd}\\

\haiku{Want ik ben wel een,,?}{rank ventje voor zestien vindt}{je niet Maximinus}\\

\haiku{U is mooi als ik... -.}{nooit een knaap zag Maar ik ben}{ook Helegabalus}\\

\haiku{- Alleen om u trouw,!}{te dienen o zoon van mijn}{vroegere keizers}\\

\haiku{aan mijn vleesch en,!}{frisch in mijn bloed maar verwend}{word ik niet van daag}\\

\haiku{- Den eersten keer, dat,!}{hij je ziet spreekt hij tot je}{en duldt hij je kus}\\

\haiku{h\`em, Helegabalus,,}{hongerde naar h\`em voelde}{zich gloeien naar h\`em}\\

\haiku{Daar zij hem telkens,...}{anders zagen herkenden}{zij hem niet altijd}\\

\haiku{Des morgens had het,,;}{Hof hem gezien nijdig en}{norsch om Alexianus}\\

\haiku{nauwlijks konden de;}{acht dikken zich nestelen}{op de sigma klein}\\

\haiku{zij begrijpen niet,,;}{goed zij hebben noch de vrouw}{noch het kind herkend}\\

\haiku{Aanbiddelijk was,.}{hij zeer zeker bekoring}{oefende hij uit}\\

\haiku{Toen ik viel, stortte,.}{je op mij toe vroeg je mij}{of ik gewond was}\\

\haiku{- Je genade is,.}{onmetelijk je liefde}{zal mij alleen zijn}\\

\haiku{- Dat je alleen dreigt,,.}{Antoninus omdat je}{te veel mij lief hebt}\\

\haiku{Ik noem je voortaan,,.}{niet anders o mijn liefde}{dan Antoninus}\\

\haiku{De auriga droeg,;}{een lang zijden feestgewaad}{met gemmen bezaaid}\\

\haiku{en ik zo\^u wel heel...... -?}{veel van hem houden als ik}{niet in hem zag Wat}\\

\haiku{Hij is \`alles, hij!}{is de Zonneziel in heel}{haar veelvuldigheid}\\

\haiku{aan een zwarten baard.}{hangt een levend robijntje}{en tikkelt we\^er ne\^er}\\

\haiku{Is het omdat de,?}{avond kil is en zij zoo heel}{lang hebben gewacht}\\

\haiku{Hij stottert, hij durft,,;}{niet zeggen dat hij Hierocles}{aanbidt zijn Gemaal}\\

\haiku{Man van vorm, vo\`elde,...:}{hij zich al vrouw en had hij}{gehuwd zijn Gemaal}\\

\haiku{- Ik ben de alom!}{uitstralende Dubbelheid}{van Helegabalus}\\

\haiku{Keizerin zal je,.}{zijn en mijn vrouw middel tot}{mijn vervolmaking}\\

\haiku{maar zij had hooger.}{geklonken en zij dreigde}{te gelijker tijd}\\

\haiku{Zoo zuiver mikte,;}{de keizer dat hij er geen}{enkele miste}\\

\haiku{- Maar Matthias en...}{Ganadasa wisten zich}{staande te houden}\\

\haiku{het harde marmer,;}{van de levende statue}{die zijn lichaam was}\\

\haiku{klokkebengelend...,}{datura's en zij vielen}{zij vielen allen}\\

\haiku{Een begin van brand;}{laaide op in een hoek van}{het Triclinium}\\

\haiku{Mammea was uit het,.}{Vrouwenhof aangesneld zij}{ook in nachtgewaad}\\

\haiku{Severa rees op,,...}{opende de bronzen deur en}{zag omzichtig uit}\\

\haiku{- Waarom anders dan'!}{om de bezoedeling van}{Alexianus standbeelden}\\

\haiku{hij wierp zich \`op hem,';}{zijn vierkante knie drukte}{Antoninus borst}\\

\haiku{Misschien gingen wel,...}{handen schuw naar Alexander uit}{om hem te streelen}\\

\haiku{En zijn moeder... zijn......!}{moeder konkelt tegen mij}{tegen mijn gezag}\\

\haiku{- Hij is de Caezar,,!}{je bloedeigen neef en je}{aangenomen zoon}\\

\haiku{IV Dien middag kwam.}{de Senaat in de Oude}{Hoop ter audientie}\\

\haiku{zij roepen buiten:}{de wallen en grachten van}{het zomerpaleis}\\

\haiku{roept Antoninus,,:}{we\^er en waar zij knielen wijst}{hij tien slaven aan}\\

\haiku{Aristomachos,...}{is niet gekomen als toch}{was afgesproken}\\

\haiku{Ze dachten van hem,,...}{te eten van hem te drinken}{van hem te leven}\\

\haiku{Nu duwt zij en dringt...}{als een razende dwars door}{die volksmassa's door}\\

\haiku{lijken van vrouwen......}{in feestgewaad lijken van}{leeuwen en narren}\\

\haiku{haar zoon is een g\`od.........}{we\^ergeboren gebaard}{uit haar liefdelijf}\\

\haiku{je zoon is een knaap,,,,!}{een lieve flinke jonge}{Romein en niet meer}\\

\haiku{hij werpt zich - velen -;}{hooren toe plat over den grond}{aan Moeza's voeten}\\

\haiku{Onder geen keizer.}{is ooit gezien zulk een pracht}{van kavallerie}\\

\haiku{maar Alexander vermoord,,...}{als verrader die stond naar}{des keizers leven}\\

\haiku{Voor den drang van het...}{Volk zijn de opgestelde}{troepen bezweken}\\

\haiku{Had zij Alexander maar,.}{keizer gemaakt en hem te}{Emessa gelaten}\\

\haiku{de deur splintert breed,.}{open en door die opening wringt}{Hierocles zich binnen}\\

\haiku{Gij naamt ze aan, gij,!}{naamt ze aan gij naamt aan al}{die waardigheden}\\

\haiku{ik zal het niet te:}{erg maken en denken aan}{mijn Hollandsch publiek}\\

\haiku{het eerste deel zo\^u:}{bijna als apart boek kunnen}{gegeven worden}\\

\haiku{Het eerste deel is,.}{apart te lezen en wellicht}{voor grooter publiek}\\

\haiku{deel i en ii in,.}{november 1905 deel iii in}{februari 1906}\\

\haiku{{\textquoteleft}1 October zal:}{je het geheele boek in}{copie bezitten}\\

\haiku{Met het derde deel.}{werd begin 1906 waarschijnlijk}{hetzelfde gedaan}\\

\haiku{Gecorrigeerd is {\textquotedblleft}{\textquotedblright}'.}{sc\'ene in de vierde zin}{van Couperus tekst}\\

\subsection{Uit: De boeken der kleine zielen. Deel 1 en 2}

\haiku{Dorine trad in,.}{de zitkamer van haar bro\^er}{Karel van Lowe}\\

\haiku{dat Constance toch;}{even goed als wij allen een}{kind van mama was}\\

\haiku{Er heerscht een groote,.}{sympathie een warm gevoel}{tusschen allemaal}\\

\haiku{Je maakte altijd.}{de boterhammen voor de}{kinderen van Bertha}\\

\haiku{zoo geleidelijk,...}{tot de misdaad als had het}{niet anders gekund}\\

\haiku{Beiden hadden zij.}{wel hun leven in moeten}{zien als \'een groote fout}\\

\haiku{Maar moesten zij niet eerst,?}{weten hoe de familie}{hen ontvangen zo\^u}\\

\haiku{Mama heeft ons w\`el......}{het voorbeeld gegeven om}{h\`artelijk te zijn}\\

\haiku{Maar met Henri, met.}{Van der Welcke k\`on ik}{me niet uitspreken}\\

\haiku{Ik ben alleen in.}{Den Haag gekomen om veel}{van jullie te zien}\\

\haiku{Zij trok hem ook naar,.}{zich toe hield haar zuster en}{haar kind dicht bij zich}\\

\haiku{En hij hielp haar zich.}{wat uitkleeden en schikte}{hare kussens op}\\

\haiku{Bertha was al bijna -.}{geheel grijs grijzer zelfs dan}{mama Van Lowe}\\

\haiku{- Het is, alsof je,!}{het niet goed vindt dat je zoon}{op zijn vader lijkt}\\

\haiku{Ze doen veel goed, ze,...}{leven voor hun kinderen}{ze zien veel menschen}\\

\haiku{Die komen ook nooit.}{op hun diners en jours en}{bals etcetera}\\

\haiku{voor je vader om.}{je te omhelzen en te}{houden op zijn schoot}\\

\haiku{in de stem van het,:}{kind was een teederheid als}{wilde hij zeggen}\\

\haiku{de eiken deuren,.}{der vertrekken die zwijgend}{bleven gesloten}\\

\haiku{het voorhoofd welfde,.}{zich ivorig en hoog uit een}{dunnen krans grauw haar}\\

\haiku{hare schoonouders,.}{haar niet hadden willen zien}{als  een schande}\\

\haiku{- Ik merk, dat je nooit,.}{veel hebt nagedacht net zooals}{de meeste vrouwen}\\

\haiku{de oudste zoon nu:}{uit Indi\"e verwacht met vrouw}{en twee kinderen}\\

\haiku{zij genoot van het,,;}{solide degelijke}{officieele huis}\\

\haiku{Er was het Hof en.}{haar man had haar geleerd van}{grootheid te houden}\\

\haiku{Mama Van Lowe,.}{ging voorbij aan den arm van}{Otto van Naghel}\\

\haiku{Ik had het alleen,.}{heel beroerd gevonden als}{het weg was geraakt}\\

\haiku{de joviale...}{huzaar met de breede borst}{en de brandebourgs}\\

\haiku{Otto met Francis,! -,:}{waarom Zij voelde dat hij}{op de lippen had}\\

\haiku{- Omdat ze het zeer... -?}{eenzaam in Brussel hadden}{Maar de familie}\\

\haiku{- Zeg... die mevrouw Van...?}{der Welcke wat komt die}{hier eigenlijk doen}\\

\haiku{In een zitkamer,,.}{was Francis Otto's vrouw met}{de twee kinderen}\\

\haiku{Geen diners meer en,......}{japonnen en omslag om}{niets maar een geluk}\\

\haiku{Het is gelukkig,...}{dat Constance er niets op}{tegen zal hebben}\\

\haiku{Van der Welcke,,,.}{dan na den eten was blij dat}{het zijne beurt was}\\

\haiku{Waarom doen we het,,...}{dat alles al die drukte}{en al dien omslag}\\

\haiku{Dorine had maar,!}{eens moeten opperen dat}{regen nat maakte}\\

\haiku{alleen ho\^u ik niet...,,...}{van den snit van zijn vest te}{hoog vind ik zijn vest}\\

\haiku{hij had haar wel eens,...}{anders gezien dadelijk}{hooren uitvaren}\\

\haiku{Haar glimlach gaf een,.}{ronding aan haar wangen die}{haar verjeugdigde}\\

\haiku{Hoe ben je toch hier,...?}{komen wonen zeg tusschen}{twee kerkhoven in}\\

\haiku{en waarom is ze}{zoo verdraagzaam tegenover}{tante Adolfine}\\

\haiku{Kijk mevrouw Bruys haar...}{taartje eten met een bijna}{dierlijk genoegen}\\

\haiku{dat fluweel van den... -? -.}{kraag van Saetzema's rok Ja}{Dat is m\'ooi fluweel}\\

\haiku{zelfs Adolfine was,,...}{den laatsten tijd voor zij op}{reis ging heel aardig}\\

\haiku{Ja, zij vroegen het,:}{elka\^ar zonderdat zij het}{elkander zeiden}\\

\haiku{En de andere,...}{jongens ranselden Jaap af}{omdat hij het zei}\\

\haiku{Kom, Addy, bemoei...}{je voortaan maar niet veel met}{die boerenkinkels}\\

\haiku{Het was zijn eerste,,.}{verdriet en het was zoo zwaar}{zoo verstikkend zwaar}\\

\haiku{en omdat het dan,...}{alles voor niets was geweest}{niet voor mijn vader}\\

\haiku{wil je je brouilleeren.}{met de kinderen van een}{zuster van mama}\\

\haiku{En w\`at hij wist, woog,:}{hij in zijn naar zekerheid}{verlangende ziel}\\

\haiku{Maar toen schrikte Van,:}{der Welcke en als met}{een schok dacht hij na}\\

\haiku{- Waarom zouden wij...}{den kleinen jongen niet eens}{te logeeren vragen}\\

\haiku{- Ik dacht - en Van der,,.}{Welcke van het lachen}{schudde op en ne\^er}\\

\haiku{zoo jong, dat hij als,,.}{een bro\^er en zoo zwak dat hij}{als een kind was}\\

\haiku{- Ik kom je vragen.}{of je overmorgen op een}{diner bij me komt}\\

\haiku{- Je komst heeft dingen,...}{opgerakeld die al lang}{vergeten waren}\\

\haiku{Ook weet ik zeker,.}{dat je schoonouders je komst}{hebben afgekeurd}\\

\haiku{En om je huis te,,?...}{arrangeeren daar heb je ook}{niet veel idee van wel}\\

\haiku{Bertha doet die dingen,,?}{meer als vrouw van de wereld}{geloof ik niet waar}\\

\haiku{Ik dacht, dat je meer,.}{in de c\^oterie was van}{Bruis de Telefoon}\\

\haiku{mij misplaatst als een,,.}{indringer als iemand die}{niet is te avoueeren}\\

\haiku{Wie hebben wij, wie,!}{komt er bij ons bij wie zijn}{wij eenigszins in tel}\\

\haiku{In jaren had zij,.}{ze niet gezien in jaren}{niet van ze gehoord}\\

\haiku{er zo\^u niets komen,.}{en er zo\^u dus ook niets staan}{in den Dwarskijker}\\

\haiku{Ook dien dag bleef zij,,.}{thuis daar het stortregende}{en zij zag niemand}\\

\haiku{- Ik zo\^u heel gaarne,,...}{alles willen doen wat je}{mij vraagt Constance}\\

\haiku{Maar een andere,...}{gedachte gaf haar nieuwen}{strijdlust nieuwen moed}\\

\haiku{Mijn visite van... -!}{Dinsdag heeft hij mij kwalijk}{genomen Kwalijk}\\

\haiku{Voor mijn kind, en hem......}{later de carri\`ere}{de carri\`ere}\\

\haiku{Des te liever zal,...}{het mij zijn zoo ik niets met}{je te maken heb}\\

\haiku{Om h\'a\'ar... om zijn vrouw...!}{had hij Van Naghel op zijn}{gezicht willen slaan}\\

\haiku{en zelfs, dat zij haar...}{zoon achterliet had haar niet}{tot rede gebracht}\\

\haiku{al had hij ook aan,...}{haar vader geschreven zij}{zo\^u niet meer komen}\\

\haiku{- Ik heb ze verteld,,.}{dat je op reis was dat ze}{dus m\`oesten wachten}\\

\haiku{- Dat weet ik wel... - Maar......}{je moet niet zoo droefgeestig}{zijn op jou leeftijd}\\

\haiku{Maar dat ben ik al. -,... -,...}{Nu dan is het goed Kijk hoe}{donker is het Bosch}\\

\haiku{Van der Welcke,,.}{lachte blij gestreeld in zijn}{jonge ijdelheid}\\

\haiku{Er waren in mijn....}{leven veel vrouwen en toch}{waren ze er niet}\\

\haiku{- Ik ga lezingen,,.}{houden niet alleen hier maar}{overal in Holland}\\

\haiku{Men riep hem terug,,.}{maar hij kwam niet meer en het}{publiek stroomde weg}\\

\haiku{maar ik leefde van,,.}{mijn loon als een arbeider}{die ik toen ook was}\\

\haiku{vroeg Addy, die als,.}{er een vreemde was nooit me\^e}{ging naar den salon}\\

\haiku{- Kapitalisten,.}{zonder kapitaal lachte}{Van der Welcke}\\

\haiku{dat n\`a die jaren...,,...}{toen ik als Gerrit zeide}{een lief kindje was}\\

\haiku{- Wat meen je daarme\^e... -,.}{Die oude vriend van oom die}{over den Vrede spreekt}\\

\haiku{Soms... soms zo\^u ik hier... -?}{kunnen huilen Maak ik je}{zoo melancholiek}\\

\haiku{Er zijn menschen, die,,.}{nooit voelen en anderen}{die altijd zwijgen}\\

\haiku{Ik ben - ik weet niet -...}{waarom in een stemming om}{te schreien met u}\\

\haiku{Dat Marianne,.}{Van der Welcke naloopt}{zoodat het geen naam heeft}\\

\haiku{zei Adolfine, boos, {\textquoteleft}{\textquoteright}.}{omdat Floortje gesproken}{had vanoude vrouw}\\

\haiku{En Marianne,,.}{voor de variatie wordt}{verliefd op haar oom}\\

\haiku{In een hoek bij de.}{deur van de serre zaten}{de oude tantes}\\

\haiku{Zij voelde, dat Brauws,.}{naar haar keek en zij voelde}{dat Brauws nog boos was}\\

\haiku{- Dat komt, omdat ik,...}{het zoo prettig vind jullie}{bij me te hebben}\\

\haiku{Ik zeg andere,.}{dingen dan anderen en}{ik zeg ze anders}\\

\haiku{dan ga ik de straat,.......}{op ik weet niet waarheen naar}{Leiden naar Henri}\\

\haiku{Ik wil... - Ik wil, dat... -?}{je stil bent en geen sc\`ene}{maakt Waar is Emilie}\\

\haiku{En terwijl zij met,,:}{Henri de kamer verliet}{zeide zij hardop}\\

\haiku{Nu vond hij in de.}{kamer van Marianne}{zijn beide zusters}\\

\haiku{Maar als ik hem lief,......}{heb als ik hem lief heb als}{ik gelukkig ben}\\

\haiku{zij wilde als voor.}{hem en de anderen dooven}{dien te grooten glans}\\

\haiku{Marianne, zeg,... -?}{me dat het niet waar is Dat}{hij me het hof maakt}\\

\haiku{- was zij niet bang, of,.}{treurig want ze voelde in}{haar droom zich veilig}\\

\haiku{en niets blijft... niet de... -,...}{minste herinnering Neen}{dat gaat alles weg}\\

\haiku{Dat mijn leven heel....}{anders geweest moest zijn om}{goed te zijn geweest}\\

\haiku{hij heeft gevochten,...}{met Eduard is handgemeen}{met hem geworden}\\

\haiku{Misschien komt hij in,,...}{huis bij Adolf zijn voogd die heel}{streng voor hem zal zijn}\\

\haiku{Alleen met wat heel,...}{sympathiek was aan haar hield}{zij samenleven}\\

\haiku{Het was mij, of ik......}{in mijn kindersprookjes iets}{vermoedde van hem}\\

\haiku{in de intieme, -:}{schemeringen praatten zij}{dikwijls veel dacht zij}\\

\haiku{Zij vond zich als een.........}{schoolmeisje dat droomde en}{haar lessen leerde}\\

\haiku{hij voelde vol in...}{zich hun beider erfenis}{van jalouzie}\\

\haiku{want zoo dikwijls, mo\^e,...}{bewogen zag hij voor zich}{een ideaal van rust}\\

\haiku{God, God, wat zag ze,.........}{er aardig uit dat kleine}{ding en wij jongens}\\

\haiku{Constance schoof de,.}{tusschendeur open voerde het}{kind in den salon}\\

\haiku{zelfs tegenover Van,......}{Vreeswijck dacht zij zo\^u het}{misschien niet fair zijn}\\

\haiku{- Ja, ging zij voort, met,.}{die peinzend kalme stem een}{beetje gedwongen}\\

\haiku{Hare oogen vlamden,.}{op zij voelde zijn opzet}{haar te beleedigen}\\

\haiku{Arme Vreeswijck... -......,.}{Ja arme kerel zeide}{hij werktuigelijk}\\

\haiku{Er was een stilte,,,.}{terwijl zij zoo stond hij haar}{aanzag doordringend}\\

\haiku{Henri, laat ons het.......}{doen als het kan met iets van}{liefde voor elka\^ar}\\

\haiku{Toen zonk zij in een,.}{stoel terwijl om haar heen de}{kamer duizelde}\\

\haiku{Nou... je hebt lekker...,.}{liggen maffen zei Addy}{ruw makend zijn stem}\\

\haiku{Ik rust nu wel uit, -.}{nu ik het je zoo zeg en}{tegen je aan lig}\\

\haiku{toen voelde die... - Wat... -...;}{Dat die meer hield van jou dan}{van Marianne}\\

\haiku{Hij zo\^u hen immers...!}{toch spoedig verlaten zelf}{zijn leven zoeken}\\

\haiku{Zij, bevende, was,...}{gaan zitten omdat zij zich}{wankelen voelde}\\

\haiku{schel was de hoop, zoo,:}{verblindend dat hij haar eerst}{niet hoorde zeggen}\\

\haiku{de illuzie... - Ja......,.}{de illuzie sprak hij met}{een glimlach van pijn}\\

\haiku{En zij omhelsde,... -,.}{hem als vroeg zij vergeving}{Addy sprak zij zacht}\\

\haiku{De dagen waren,...}{langzaam voortgegaan de een}{na den anderen}\\

\haiku{De kleine zielen - {\textquoteleft}{\textquoteright}.}{geleidelijk in het net}{werd overgeschreven}\\

\haiku{Hij beloofde de}{kopij voor ongeveer half}{januari.49 Veen}\\

\haiku{{\textquoteleft}Ik hoop, dat ge van:}{ons ivoor-en-gouden boek}{pleizier zult hebben}\\

\haiku{\ensuremath{<} zijn h1236,32als hij ons, - -}{ziet fietsen \ensuremath{<} als hij ons}{fietsen ziet h1237,12wil}\\

\haiku{\ensuremath{<} lach, - die soms in,:}{een schater om vd W kon}{eindigen en zei}\\

\haiku{na h1370,10/11waait... alsof hij,...}{de ramen wil openen en}{binnen wil komen}\\

\haiku{h2461,24al \ensuremath{<} was het al}{h2463,20zonder \ensuremath{<} als zonder}{h2463,28bazuinen \ensuremath{<}}\\

\haiku{aan Veen, gedateerd,.}{31 december 1902 in het}{archief-Veen}\\

\haiku{Waarde Heer Veen, p..).}{221 75In h2 staan boven de}{komma drie puntjes}\\

\subsection{Uit: De boeken der kleine zielen. Deel 3 en 4}

\haiku{kinderen moet je,,}{hebben zonder kinderen}{heb je geen leven}\\

\haiku{De kussens van het;}{bed vertoonden maar even den}{indruk van Pauls hoofd}\\

\haiku{Gerrit sloot het raam,.}{de regen ruischte niet}{de kamer meer in}\\

\haiku{omdat ik me met...}{zorg kleed en je een kwartier}{langer laat wachten}\\

\haiku{De wereld is al,.}{smerig genoeg ook al is}{men nog zoo netjes}\\

\haiku{Hij keek, onder zijn,.}{parapluie razend naar}{den regenhemel}\\

\haiku{maar nu was hij den}{draad van zijn redeneering}{kwijt en daarbij moest}\\

\haiku{Ik heb ze om mij,.}{heen verzameld verzameld}{uit alle eeuwen}\\

\haiku{Soms zijn ze prachtig,...}{gekleed en zingen ze met}{heerlijke stemmen}\\

\haiku{Maar den laatsten tijd - -...}{hij schudde weemoedig het}{hoofd den laatsten tijd}\\

\haiku{Onder haar arm werd,,.}{hij hard als ijzer en boos}{hard zag hij haar aan}\\

\haiku{- Ja... - Dat de ploert ze... -...... -?}{niet wakker maakt en trapt Ja}{ja Beloof je dat}\\

\haiku{dat hij verkeerd deed......,.}{Als een vrouw waardig is slaat}{haar geen man mijn kind}\\

\haiku{Het is niet zoo, als....}{de jonge meisjes denken}{als ze verliefd zijn}\\

\haiku{en als hij dat doet.........}{lachen de menschen eerst en}{huilen ze daarna}\\

\haiku{het is ook een groot.......}{geheim voor de familie}{voor de kennissen}\\

\haiku{Ik heb juist gevoeld,...}{dat ik me los van alle}{banden moest maken}\\

\haiku{Zij vroeg zich af, hoe...}{zij het hier vier weken zo\^u}{moeten uithouden}\\

\haiku{Kunnen we niet eens,.}{het karretje huren dan}{zal ik je mennen}\\

\haiku{we kunnen het niet... -!}{iederen morgen nemen}{Iederen morgen}\\

\haiku{- Neen, hoor, zoo ben ik......}{niet om op mijn oom verliefd}{te worden voor niets}\\

\haiku{- De dokter heeft veel... -...,.}{hoop Ja zei Bertha nu alsof}{dat van zelve sprak}\\

\haiku{er is anders niets,......}{dan een beetje te zijn te}{doen voor anderen}\\

\haiku{Maar... soms... is het me... -,... -,...}{heel zwaar Kind mijn kind Ja soms}{is het me heel zwaar}\\

\haiku{Blind ook liep ze door,...}{dien droom van ijdelheid en}{ze dacht dat ze zag}\\

\haiku{Zij stonden op, en,,,,,.}{liepen voort de duinen op}{af op af zwijgend}\\

\haiku{Er gebeuren 's,,...}{nachts schandalen schandalen}{in alle kamers}\\

\haiku{Zij voelde zijn hand;}{op haar arm zwaar liggen als}{de hand van een man}\\

\haiku{zo\^u het leven en,.}{de loopbaan die zij voor zijn}{vader geknakt had}\\

\haiku{Ook niet aan papa,,.}{omdat ik voel dat hij het}{niet zo\^u begrijpen}\\

\haiku{Ernst, behoorende tot -;}{de donkere Van Lowe's}{het bloed van papa}\\

\haiku{iets van goedigheid:}{we\^erhield hem werkelijk}{driftig te worden}\\

\haiku{een aardig huisje,,,...}{een interieur een lief}{vrouwtje kinderen}\\

\haiku{de gedroogde visch,:}{die bij het bakken opzwol}{tot brosse schulpen}\\

\haiku{en daarbij is hij,...}{een melancholiek heer die}{zijn manie\"en heeft}\\

\haiku{Ik heb ook niet veel...,.}{honger jokte Gerrit die}{altijd honger had}\\

\haiku{En zeg me nu eens,,.}{Dorine ga je niet eens}{naar Nunspeet kijken}\\

\haiku{- Nu ja... dat weet ik,...?}{wel maar bij ons zo\^u je toch}{gezelliger zijn}\\

\haiku{Zij deed zich geweld.}{om de tranen in haar oogen}{terug te persen}\\

\haiku{Ach, mama weet wel,,...}{dat ze niet tegen je op}{kan niet waar Papa}\\

\haiku{Probeer u er me\^e,,.}{eigen te maken lieve}{oma dat ik niet mag}\\

\haiku{- Nou maar, ik vind er,,.}{niets aardigs aan en maak jij}{maar dat je weg komt}\\

\haiku{- Wacht jongens, papa,...}{moet eerst naar boven om zijn}{handen te wasschen}\\

\haiku{jij bent zoo lief en -...}{zoo zacht al ben je nog zoo}{een ruwe kerel}\\

\haiku{En - het was misschien -:}{allerstomst van hem}{hij had haar geloofd}\\

\haiku{Daar moeten we nu:}{altijd in bewondering}{voor ne\^erknielen}\\

\haiku{Zij zag vleierig,.}{tegen hem op haar handen}{streelden zijn lichaam}\\

\haiku{Zelfs den laatsten tijd...,... -? -.}{in Parijs Gerrit Wat Dacht}{ik wel eens aan jou}\\

\haiku{Al dat frissche - als, -;}{een vrucht waarin hij hapte}{van vroeger was weg}\\

\haiku{Zoo zo\^u het eenmaal -,...}{worden als hij heel oud heel}{oud was geworden}\\

\haiku{Nu was het zoo nog......}{niet nu daagde het nog uit}{het blonde troepje}\\

\haiku{Je denkt, dat je je...,...}{charme voor de eeuwigheid}{hebt Alles slijt kind}\\

\haiku{- Je hebt zeker nog......... -}{wel een portret een groepje}{van je kinderen}\\

\haiku{En zij zag ginds ver,,,,.}{weg te ver voor haar een vrouw}{oud als zij sterven}\\

\haiku{Maar je bent wel boos... -,,... -...,.}{geweest Stil stil mama Neen}{neen laat me spreken}\\

\haiku{Ik las, niet dat hij......}{ziek was maar dat hij uit zijn}{betrekking moest gaan}\\

\haiku{zo\^u in de toekomst -.}{dat hij uit dien verren dood}{terug zo\^u komen}\\

\haiku{Was het niet je nog,...}{wrokkende jeugd die niet de}{verzoening wilde}\\

\haiku{O, wat zo\^u er toch,!}{dreigen nu de oude vrouw}{ginds gestorven was}\\

\haiku{zo\^u die korrel niet...}{voldoende zijn om d\`at met}{wijsheid te dragen}\\

\haiku{hij, die nooit, na een,...}{gebroken carri\`ere}{had werken gekund}\\

\haiku{Het moedertje was:}{nog altijd overweldigd door}{haar stillen dollach}\\

\haiku{- Het is ook met dat... -,.}{beroerde we\^er Ja dat maakt}{de menschen somber}\\

\haiku{En ik zeg je ze,.}{en voel dat het nutteloos}{is ze te zeggen}\\

\haiku{Had hij nog koorts en,...}{had hij niet goed gedaan op}{te staan uit te gaan}\\

\haiku{Maar hij had niet meer,:}{kunnen in bed blijven hij}{had niet meer gekund}\\

\haiku{- Wil u dat aan de...}{juffrouw geven Stoppen in}{de hand van de vrouw}\\

\haiku{ze brengen het lijk,...}{van Henri me\^e oom en ze}{brengen Emilie me\^e}\\

\haiku{Was het leven dan...:}{niet meer gewoon Waren er}{dan niet als altijd}\\

\haiku{Constance opende,...}{de deur der eetkamer haar}{arm om Emilie heen}\\

\haiku{Zij snikte, als gek,.}{ne\^ergezonken tegen}{Constances knie\"en}\\

\haiku{Want hij zag niet meer:}{door de kamerwanden het}{heelal en het Beest}\\

\haiku{Nu was hij zoo ver,}{aangebeterd dat hij zat}{in een ruimen stoel}\\

\haiku{Onderzoek eens, wat,... -...}{er van is wil je Hoe heet}{zij en waar woont ze}\\

\haiku{- Ja, kerel... - Je wordt,... -,...}{nu we\^er beter h\`e Ja ik}{word nu we\^er beter}\\

\haiku{bleef het niet altijd,...}{kil en koud ook al laaide}{het nog zoo hoog op}\\

\haiku{Wat zocht ze in de,...}{diepe diepte wat smeet ze}{om hem heen het zand}\\

\haiku{Het is beter, dat... -,.}{je ook gaat Addy is nu}{al gegaan mama}\\

\haiku{Ze zijn nog niet naar... -... -.}{bed En Addy Addy moet}{er wel binnen zijn}\\

\haiku{Zeg haar... zeg haar nog......,....}{niets zeg haar zeg haar dat De}{wanhoopssnik schreeuwde}\\

\haiku{Toen staken zij het,.}{licht op en brachten zij de}{oude vrouw naar bed}\\

\haiku{hij, die toen, kleine,...}{jongen gegaan was naar den}{dikken aannemer}\\

\haiku{als het arme kind...}{nu maar niet ziek werd van die}{lange wandeling}\\

\haiku{O, wat was het een,...}{sombere gang de eiken}{deuren we\^erszijden}\\

\haiku{Maar trappeltripjes,:}{naderden snel en nu aan}{de deur klop-klop}\\

\haiku{En ze is de vrouw.}{van Addy en de moeder}{van zijn kinderen}\\

\haiku{Hij groette hier en,,}{daar in het rond maar kuste}{niemand gaf geen hand.}\\

\haiku{wat kan ik er aan... -,.}{doen Je moest eens eerlijk met}{me spreken zei hij}\\

\haiku{Maar als we daarheen,,.}{verhuizen Tilly moeten}{we heel zuinig zijn}\\

\haiku{- Nu, zeide zij met.}{haar matte gepiqueerde}{onverschilligheid}\\

\haiku{Addy fronste zijn,.}{brauwen en dat gaf hem een}{pijnlijk droeven trek}\\

\haiku{Tusschen de zorgen,:}{voor oom Gerrits kinderen}{had hij ze gevoeld}\\

\haiku{voor zichzelven wist,...}{hij niets en vooral wist hij}{het niet voor zijn ziel}\\

\haiku{De wind kwam binnen,.}{en blies met \'een ademtocht het}{licht uit van de lamp}\\

\haiku{- Ik ben altijd stil... -,,.}{Probeer eens Alex vertrouwen}{in me te hebben}\\

\haiku{Nu, laten wij dan...}{nu niet anders spreken dan}{over de Handelsschool}\\

\haiku{- Hij zal wel veel geld... -...,.}{al verdienen Zoo Niet zoo}{heel veel geloof ik}\\

\haiku{Ik zo\^u niet gaarne,...}{hebben dat hij Marietje}{hypnotizeerde}\\

\haiku{- Het is zoo kil aan... -,...}{de voeten Guy geef tante}{een voetenbankje}\\

\haiku{De ontzenuwing,:}{vereffende zich het lunch}{eindigde rustig}\\

\haiku{Adolfine was heel,,.}{ontroerd met roode huiloogen die}{zij telkens wischte}\\

\haiku{Ik heb geen geld om......}{om langen tijd ergens met}{haar buiten te gaan}\\

\haiku{Meneer is altijd,,.}{vreemd niet waar maar hij is niet}{lastig en vrij wel}\\

\haiku{In het grauwe licht.}{van het kleine kamertje}{rees het meisje op}\\

\haiku{Zeg, Addy, dat zijn,?}{alle de kinderen van}{oom Gerrit niet waar}\\

\haiku{O, zij had hem niet,,:}{getrouwd om zijn geld of zijn}{titel dat ook niet}\\

\haiku{haar rozigblanke,,!}{tint haar volle vormen haar}{jong-sterke leden}\\

\haiku{Zij had nog maar even,:}{den tijd zich te kleeden haar}{schaatsen te nemen}\\

\haiku{- Neen, dat is immers... -:}{het beste Het is altijd}{het zelfde geluid}\\

\haiku{Marietje zocht naar,.}{Constance om de sleutels}{van de linnenkast}\\

\haiku{Ik ben het type... -,...}{van een ouden vrijer Je}{moest nog trouwen Paul}\\

\haiku{Het is een mooie das,...}{maar ik heb er niet meer zoo}{een collectie van}\\

\haiku{Gelukkig, jullie,.}{kibbelen niet en jij zelfs}{niet meer met je man}\\

\haiku{- Neen, niet alleen... - U... -.}{heeft zich aan elka\^ar gewend}{Zonder veel woorden}\\

\haiku{Addy... zoo bij u...... -... -? -...}{altijd Mijn arme jongen}{Waarom Ik ben bang}\\

\haiku{- De innerlijke... -,}{dingen niet Wees gelukkig}{dat uw leven zoo}\\

\haiku{- Win haar dan... - Het is... -}{zoo heel moeilijk en waar er}{geen sympathie is}\\

\haiku{Zij ging  in de:}{aangrenzende kamer naar}{hare kinderen}\\

\haiku{En dan, dan hem ook,...}{jaloersch te maken van haar}{als zij was van hem}\\

\haiku{Waarom spreekt u van,... -...}{Den Haag en wat geven wij}{nu om een bal Juist}\\

\haiku{Hoe meen je kind... - Niet,,...}{zoo als tante Constance}{en Emilie en u}\\

\haiku{O hij hield van de,,... -...}{kinderen maar hield hij van}{ha\`ar zijn vrouw Addy}\\

\haiku{- Het is wel jammer,,...}{Tilly dat je het hier zoo}{weinig kunt schikken}\\

\haiku{voor niets... - Probeer te,......}{voelen Tilly dat ik me}{niet afsloof voor niets}\\

\haiku{- had hij schuld, - omdat!}{hij zijn vrouw alleen liefhad}{met de helft van zich}\\

\haiku{Ik zal doen als je,.}{zegt ik zal me in Den Haag}{een praktijk zoeken}\\

\haiku{Een nieuwheid, banaal,,;}{en frisch als de verf van haar}{huis was om haar heen}\\

\haiku{- Je hebt die koele,,,}{ver-affe stem kerel}{die ik zoo goed van}\\

\haiku{men zocht, de dames -,.}{vooral omdat hij er goed}{uitzag baron was}\\

\haiku{s avonds naar hare,,,...}{familie naar kennissen}{theedrinken alleen}\\

\haiku{het was mooi zacht, en.}{de lente weefde groentjes}{tusschen de boomen}\\

\haiku{meneer Erzeele -,}{zij kuste Mathilde gaf}{Erzeele de hand.}\\

\haiku{- Dat weet ik... - Hij kwam.......}{een afspraak maken om te}{tennissen morgen}\\

\haiku{En hoe de jongen...}{dat kind haar geest heeft weten}{te ontwikkelen}\\

\haiku{zoo heeft je leven,,...}{een doel zelfs mijn leven ook}{al doe ik zelf niets}\\

\haiku{Hare stem klonk als,.}{een stem van vroeger zeide}{dingen van vroeger}\\

\haiku{die jongen is te.}{gezond om altijd in die}{boeken te zitten}\\

\haiku{Maar juist dat alles............}{die gesprekken dat afscheid}{daarvoor vreesde hij}\\

\haiku{hij nam den brief voor... -....}{Adeline me\^e Hoe het haar}{te zeggen dacht hij}\\

\haiku{het is of het een......}{eigen zoon van me is die}{me heeft verlaten}\\

\haiku{Ik ho\^u van mijn man...... -,.}{van mijn kinderen en ik}{zo\^u Ja zeide hij}\\

\haiku{- Ja... maar... - Je houdt niet -... -...}{van me. Niet zoo Je zo\^u toch}{gelukkig worden}\\

\haiku{Hier - hij zocht in zijn -,.}{zak hier is een brief van Guy}{uit New-York}\\

\haiku{Het was na een avond.........}{dat hij gespeeld had in het}{circus en Eduard}\\

\haiku{Wat klonk hun gesprek,...}{zoo vertrouwelijk wat klonk}{het treurig bijna}\\

\haiku{Ik verwachtte je,,,,.}{morgen zei bleek Mathilde}{ondanks zichzelve}\\

\haiku{- Hier neem ik afscheid,,.}{zei Erzeele toen zij de}{zijstraat insloegen}\\

\haiku{ook is het beter... -... -...,... -}{Wat Dat je terug keert naar}{Driebergen Addy}\\

\haiku{- Dat je zoo veel van... -,...}{hem zal houden Ik weet het}{niet ik weet het niet}\\

\haiku{o God... nu zo\^u ik.........}{zoo gaarne wenschen om hen}{voor zijn kinderen}\\

\haiku{{\textquoteright} Het laatste deel zou {\textquoteleft} [...]:}{de cyclus  afsluiten}{met een nieuw belang}\\

\haiku{De laatste sc\`ene:}{van het Heilige Weten}{pakte mij erg aan}\\

\haiku{{\textquoteleft}Nu niet brommen dat:}{Het Late Leven wat kort}{is uitgevallen}\\

\haiku{Wil ik U eens een?}{paar bladzijden zenden om}{een proef te nemen}\\

\haiku{voltooid.55 Couperus.}{beloofde de proeven af}{te maken voor april}\\

\haiku{Op verzoek van Veen:}{gaf Thieme een overzicht van}{de stand van zaken}\\

\haiku{Ja56,35zijn van \ensuremath{<} gaan,}{voor57,23een \ensuremath{<} dan een58,24/25moest dat hij}{energie moest hebben}\\

\haiku{Zij had60,3gevonden.}{\ensuremath{<} geraden60,3zij had \ensuremath{<}}{zelve had zij61,5leven}\\

\haiku{de drie meisjes, de:}{jongen Herman100,19dan lachten}{tante en nichtjes}\\

\haiku{\ensuremath{<} niet om235,8/9twee,, -,}{machteloos \ensuremath{<} twee -235,9en}{van \ensuremath{<} van235,19kamer}\\

\haiku{\ensuremath{<} keer, alleen250,28die,}{\ensuremath{<} die zelfs251,11Constance}{\ensuremath{<} zij251,26zal \ensuremath{<} moet}\\

\haiku{Zelfs Alex272,3lang aan \ensuremath{<} aan,,}{272,4bijna in \ensuremath{<} bijna}{ziende om zich heen}\\

\haiku{- Ik ben altijd je,...-.}{vriend vadertje~ Ben je}{het nog altijd}\\

\haiku{De avonden363,23lazen \ensuremath{<},.}{stil lazen364,5een trap \ensuremath{<} de trap364,8hun}{\ensuremath{<} zijn364,34tusschendeur}\\

\haiku{een fout zoo men wil,, [...].}{een vergissing zeker maar}{zoo heel verklaarbaar}\\

\subsection{Uit: Brieven van Louis Couperus aan zijn uitgever}

\haiku{Het rekenen in.}{geld correspondeert met het}{rekenen in tijd}\\

\haiku{Enige malen deelt:}{hij Veen het geheim van zijn}{grote werkkracht mee}\\

\haiku{Ik hoop, dat wij het.}{eens zullen worden over de}{volgende edities}\\

\haiku{Tot mijn spijt kan ik;}{niet v\'oor 10 September in}{Amsterdam komen}\\

\haiku{Zoo mogelijk, zo\^u.}{ik het overige ook zeer}{gaarne ontvangen}\\

\haiku{Oscar Wilde Het.}{Portret van Dorian Gray}{vertaald door Mevr}\\

\haiku{U schreef eerst over 2de:}{druk van Extaze en nu}{van Lent v Vaerzen}\\

\haiku{Het boek zelf verschijnt.}{in de Duitse vertaling}{van Dr. P. Rach\'e}\\

\haiku{Vroegere Verzen,.}{in 1895 te verschijnen als}{Williswinde}\\

\haiku{Semiramis, twee.}{gedichten uit Gouverneurs}{Onze Huisvriend.46 5}\\

\haiku{Heeft de Josselin?}{de Jong met U gesproken}{over Majesteit}\\

\haiku{Het zo\^u mij leed doen.}{zoo U zich hier niet mede}{kon vereenigen}\\

\haiku{Achtend L.C. 62.}{Baarn   ten huize van Mr.}{J.J. van Santen}\\

\haiku{Maar laat mij voortaan,.}{een woordje me\^espreken waar}{het den band aangaat}\\

\haiku{Het boek is prettig.}{los ingebonden en ziet}{er nog al flink uit}\\

\haiku{Wanneer denkt U over []?}{hoek weggescheurd uitgave}{van Extaze}\\

\haiku{revizie is niet,.}{noodig als U er nog eens streng}{het oog over laat gaan}\\

\haiku{Ik weet niet, of ik,;}{ze afmaak hoewel ik er}{veel pleizier in heb}\\

\haiku{natuurlijk met de,!}{belofte dat U de Gids}{niet te vroeg inhaalt}\\

\haiku{verschijnt de rest in,.}{de Gids iets minder dan het}{eerste gedeelte}\\

\haiku{Heeft U dus nog niet,.}{met Minden afgesproken}{denk dan eens over hem}\\

\haiku{Van een tweeden druk.}{Majesteit alleen een paar}{exemplaren aan mij}\\

\haiku{ik meen, letters die,.}{niet goed aansluiten in het}{woord maar dwarrelen}\\

\haiku{L.C. ~ Mag ik U?}{de zorg van deze Poolsche}{dame opdragen}\\

\haiku{Ook de uitgave.}{in boekvorm laat daarna niet}{lang op zich wachten}\\

\haiku{Rotterdam Dokter:}{Vlaanderen Hilversum}{Zend U mij de ex}\\

\haiku{maar liever als ik.}{U mijn adres te Veneti\"e}{heb opgegeven}\\

\haiku{maar ik vind beider:}{uiterlijk niet zoo geslaagd}{als vorige keeren}\\

\haiku{trouwens Hooge Troeven,?}{alleen is toch te klein voor}{een boekje niet waar}\\

\haiku{Van Antonius,.}{zal ik wel drie proeven noodig}{hebben denk ik}\\

\haiku{Steeds gaarne Uw dw..}{L.C. 140 Rome   H\^otel}{du Sud   4.3.96}\\

\haiku{Met vriendelijke.}{groeten Steeds gaarne Uw dw}{L.C. 147 den Haag}\\

\haiku{Wat Else Otten,.}{betreft zal ik mij gaarne}{aan Uw raad houden}\\

\haiku{154 Parijs   18.}{Rue Chateaubriand   5.12.96}{Waarde Heer Veen}\\

\haiku{Het is heel aardig,.}{ook dat artikel in de}{Bergensche revue}\\

\haiku{Met vriendelijke.}{groeten van mijne vrouw}{Steeds gaarne Uw dw}\\

\haiku{Mag ik rekenen, {\textquoteleft}{\textquoteright}?}{dat U mij delauwers en}{disteltakken zend}\\

\haiku{ontving, kan ik U.}{nu zekere gegevens}{voor Psyche geven}\\

\haiku{Als de corrector,.}{het zorgvuldig doet laat ik}{het maar aan hem over}\\

\haiku{Steeds gaarne Uw dw:}{L.C. ~ Mevrouw informeert}{naar de vertaling}\\

\haiku{Wij zullen het heel:}{gezellig vinden U eens}{in Brussel te zien}\\

\haiku{In vriendelijken.}{dank het douceurtje van {\textflorin}}{25- ontvangen}\\

\haiku{Mag ik U dan mijn}{schuld terug betalen met}{1e druk Fidessa}\\

\haiku{mag ik dus hopen:}{de rest te ontvangen aan}{het adres van mijn broer}\\

\haiku{Daarme\^e is mij dus:}{voldaan de uitgave 1ste}{druk van den roman}\\

\haiku{Finantieel geeft,.}{het wel nooit iets maar daar leg}{ik mij maar bij ne\^er}\\

\haiku{B.:180 die las ik al..}{Andere kritieken heb}{ik helemaal niet}\\

\haiku{Het is mogelijk,,.}{dat dat alles zoo is maar}{het kwam mij nieuw voor}\\

\haiku{Ik zend U heden.}{in vriendelijken dank Uw}{testament terug}\\

\haiku{Wij rekenen er:}{vast op U te zien van den}{winter in Nice}\\

\haiku{kunt ge dus de drie?}{eerste vellen Tweede Deel}{nog eens zenden}\\

\haiku{Hoog Ed. Gestr Heer,,,.}{J.R. Couperus Rezident}{Bezoeki Java}\\

\haiku{Ik had gaarne 6,.}{exemplaren gebonden en}{niet gebonden}\\

\haiku{morgen gaan wij over,;}{in ons huis maar wij wachten}{nog onze meubels}\\

\haiku{en zitten dus met.}{een beetje een primitief}{comfort om ons heen}\\

\haiku{ik geloof, dat Van,??}{Hall ook geschrikt is maar is}{het heusch zoo erg}\\

\haiku{Louis Couperus 259.}{Nice   Villa Jules}{24.I.I. ~ Amice}\\

\haiku{Over een paar weken.}{zal het wel heelemaal af}{en in orde zijn}\\

\haiku{ik zeggen dat Van}{Deyssels kritieken ze ook}{verloren hebben:210}\\

\haiku{in boek IV is mijn,,}{kleine held nu 13 jaren}{een man.- Zoo ge}\\

\haiku{Het Late Leven ...,.}{vordert goed en wordt heel mooi}{al zeg ik het zelf}\\

\haiku{Ook de andere,.}{boeken zullen goed worden}{zullen we hopen}\\

\haiku{Juist na mijn brief kwam,;}{het pakket aan maar drukwerk}{gaat altijd vlugger}\\

\haiku{{\textquoteleft}Metamorfoze{\textquoteright}.}{en Fidessa's kopje vind ik}{altijd het mooiste}\\

\haiku{bv. Dan meld ik U.}{later precies hoeveel die}{twee bedragen zijn}\\

\haiku{het is eenvoudig.}{verregaande slordigheid}{met het afdrukken}\\

\haiku{Ik hoop, dat ge van:}{ons ivoor-en-gouden boek}{pleizier zult hebben}\\

\haiku{Wil ik U eens een?}{paar bladzijden zenden om}{een proef te nemen}\\

\haiku{Terwijl in \`al de!}{boeken de Hollandsche lucht}{grauw en angstig dreigt}\\

\haiku{Het heele huis ligt!}{overhoop en alles ruikt naar}{de naftaline}\\

\haiku{303 Wiesbaden   .}{Promenade-H\^otel}{10.VI.II ~ Amice}\\

\haiku{Nu niet brommen dat:}{Het Late Leven wat kort}{is uitgevallen}\\

\haiku{dan vrolijkt ge mij.}{wat op en dat heb ik wel}{noodig.- ~ Steeds t.\`a.v}\\

\haiku{in de toekomst zal.}{Het heilige Weten den}{cyclus voltooien}\\

\haiku{Is in Stefanie?}{misschien zelfs een portret van}{M\'elanie gegeven}\\

\haiku{Intussen is het.}{voor Couperus niet alles}{onbewolkt genot}\\

\haiku{Ik hoop in Rome.}{wat goed we\^er te krijgen en}{geen influenza}\\

\haiku{Het is dus alleen,.}{de finantieele kwestie}{die ik even aantik}\\

\haiku{Ik zal de proeven.}{van het Heilige Weten}{klaar maken voor April}\\

\haiku{Mijn schuld hoop ik je,}{spoedig af te doen maar het}{duurt ontzettend lang}\\

\haiku{Ach, beste kerel,,.}{ieder heeft het zijne een}{pakje te dragen}\\

\haiku{Dionyzos heeft.}{hij in de voorafgaande}{maand Juli voltooid}\\

\haiku{Met Veen overweegt hij.}{nu een complete editie}{van al zijn werken}\\

\haiku{Couperus laat de:}{maar al te juist gebleken}{opmerking volgen}\\

\haiku{Enfin, zoodra,.}{ik het geld heb wo\^u ik met}{je de schuld afdoen}\\

\haiku{Ik zal er zelve.}{van de laatste revizie}{nazien en herzien}\\

\haiku{mijn arme vrouw doet.}{wonderen om de boel aan}{de gang te houden}\\

\haiku{en Maart, iedere.}{maand {\textflorin} 500.- dan zo\^u ik}{je zeer verplicht zijn}\\

\haiku{In haast- ~ t.t.,!}{L.C. ~ Ik hoop dat het goed}{met U allen gaat}\\

\haiku{Mag ik je morgen?}{er een doos geconfijte}{vruchten voor zenden}\\

\haiku{Wij hebben het ook,,:}{nog geloof ik in een oud}{familie-album}\\

\haiku{Wil je een contract?}{opstellen over den roman}{van dezen zomer}\\

\haiku{Villa Jules, St..}{Maurice Nice ~ t.t. Louis}{Couperus        IV}\\

\haiku{Van den zomer heb,:}{ik niet gewerkt maar nu ga}{ik goed aan den gang}\\

\haiku{ik zal het niet te:}{erg maken en denken aan}{mijn Hollandsch publiek}\\

\haiku{Ik schrijf, of liever.}{geef voortaan geen letter meer}{uit in het Hollandsch}\\

\haiku{Ik vertelde U.}{reeds dat het is de roman}{van Helegabalus}\\

\haiku{het eerste deel zo\^u:}{bijna als apart boek kunnen}{gegeven worden}\\

\haiku{zelfs al zo\^u het niets,.}{zijn om we\^er het verlies in}{evenwicht te brengen}\\

\haiku{{\textflorin} 4500.-voor drie,.}{deelen waar je mij {\textflorin} 3000}{geeft voor twee deelen}\\

\haiku{Het eerste deel is,.}{apart te lezen en wellicht}{voor grooter publiek}\\

\haiku{Ik herhaal, dat ik;}{later gaarne bereid ben}{tot een concessie}\\

\haiku{En daarom kom ik,.}{nog eens hooren wat je denkt}{van mijn laatsten brief}\\

\haiku{Wil je het Eerste,:}{Deel voor het najaar dan is}{dat natuurlijk goed}\\

\haiku{1 October zal:}{je het geheele boek in}{copie bezitten}\\

\haiku{Ziet Couperus toch?}{op tegen het verschijnen}{van De Berg van Licht}\\

\haiku{zijn schoonmoeder is.}{bereid De Berg van Licht in}{ontvangst te nemen}\\

\haiku{Zend ook proef van dat,,.}{prospectus hoor want ik moet}{het beslist nazien}\\

\haiku{- Over het algemeen.}{ben ik niet erg tevreden}{over het afdrukken}\\

\haiku{Zend mij s.v.p. zoo veel.}{mogelijk kritieken en}{uitscheldpartijen}\\

\haiku{heel belangrijk vind,.}{ik het werk niet en het vult}{maar mijn koffer}\\

\haiku{Mijn vrouw zit rustig,.}{aan mijn zijde en is niet}{in Holland geweest}\\

\haiku{ik maak iederen;}{dag groote wandelingen en}{het is niets te warm}\\

\haiku{Wat wil je, het is{\textquoteright}.}{het eenige wat ik kan een}{gevaarlijk besluit}\\

\haiku{De waarheid is, dat}{er geld verdiend moet worden.345}{Aan Emma Garzes}\\

\haiku{het beste wensch ik.}{van mijn kant toe aan U en}{Uw huisgezin}\\

\haiku{Er is geen bezwaar ().}{dat de bundelselk in 2}{deelen verschijnen}\\

\haiku{Antwoord mij s.v.p. zoo,.}{spoedig mogelijk daar ik}{beslissen moet}\\

\haiku{Er is geen kwestie:}{van auteursrecht verkoopen}{aan het Vaderland}\\

\haiku{Van hier nodigt hij:}{Emma Garzes uit ook naar}{Rome te komen}\\

\haiku{we zullen dus maar {\textquoteleft}{\textquoteright}.}{niet meer denken overnog een}{bundel voor dit jaar}\\

\haiku{anders blijft het, als,.}{de Berg tien dagen liggen}{aan het Postkantoor}\\

\haiku{Wil je het boek in,.}{1 of 2 deelen geven}{en in welk formaat}\\

\haiku{wat pleizier hebt van}{mijn uitgaven en zal mijn}{best doen de boeken}\\

\haiku{L.C. 487 M\"unchen   .}{Pension Grebenau}{Wittelsbacherplatz}\\

\haiku{Maar ik kan heusch!}{zoo moeilijk reclame voor}{mijzelven maken}\\

\haiku{zo\^u je er reeds over}{kunnen denken wat je van}{mij kunt uitgeven}\\

\haiku{zend revizie, die.}{ik per ommegaande}{zal terugzenden}\\

\haiku{j'en fais mon deuil, als,:}{de Franschman zegt wanneer}{hij niet zeggen wil}\\

\haiku{ik zend je nog het,}{bundeltje waar je recht op}{hebt maar zo\^u je zoo}\\

\haiku{de Italiaanse.}{vriend is op 11 Juni naar}{Smyrna vertrokken}\\

\haiku{zij gaan mij ook niet.}{aan en ik heb heusch al}{genoeg aan mijn hoofd}\\

\haiku{Ik zo\^u je alleen;}{willen vragen er nog wat}{mede te wachten}\\

\haiku{Geef de bundels (met,):}{het eerste dat je reeds hebt}{uit in den vorm van}\\

\haiku{gumtig, voor gunstig,.}{Keyk voor Keyx terwijl overal}{anders goed Keyx staat}\\

\haiku{Zend mij ook proef van.}{Inhoud met de verwijzing}{naar Het Vaderland}\\

\haiku{Ik heb het nu in.}{het Duitsch gelezen en het}{is een prachtig boek}\\

\haiku{een hartelijken.}{handdruk van ons beiden en}{steeds gaarne ~ t.\`a.v}\\

\haiku{Ik hoop [dat] het niet,.}{verloren is want ik heb}{er geen brouillons van}\\

\haiku{L.C. 524 Florence.}{Pension Rochat}{Via dei Fossi 16}\\

\haiku{het is mijn laatste,.}{en ben er dus een beetje}{difficiel me\^e}\\

\haiku{Meld mij even ontvangst,,}{want als zij weg zijn is het}{een heele moeite}\\

\haiku{Vindt U het goed als?}{ik U de vertaling zend}{einde Augustus}\\

\haiku{Is de andere,...}{titel dus niet verkoopbaar}{ga je gang dan maar}\\

\haiku{Hij schrijft dus slechts zijn,:}{feuilletons en gaat door met}{die te bundelen}\\

\haiku{Ik wo\^u je alleen,,}{zeggen beste vriend dat ik}{voortaan toch wel we\^er}\\

\haiku{Couperus had die.}{in 1914 ontvangen voor zijn}{Antiek Toerisme}\\

\haiku{Louis Couperus ~ ??}{Is er papier voor de twee}{andere boekjes}\\

\haiku{Over een paar dagen,.}{zijn ze klaar en dan zend ik}{het heele deeltje}\\

\haiku{het publiek houdt er.}{niet van als de titel niet}{zijn belofte houdt}\\

\haiku{De Verliefde Ezel,.}{uit te geven die in Het}{Vaderland verscheen}\\

\haiku{Is de copie zoo?}{voldoende of wilt ge er}{nog een stukje bij}\\

\haiku{de prachteditie (van:).}{{\textflorin} 25.- het ex is zoo}{goed als uitverkocht}\\

\haiku{Het bundeltje zo\^u,.}{zeer verzorgd moeten worden}{breed en kort formaat}\\

\haiku{Ontvingt ge mijn brief,??}{waar over ik U het een en}{ander voorstelde}\\

\haiku{Ik had ook onlangs}{nog een prettige briefkaart}{van hem ontvangen.555}\\

\haiku{{\textquoteright} - Op donderdag 19:}{Juli schrijft Van Eeden dan}{de laatste regel}\\

\haiku{Couperus heeft la.}{Duse mogelijk maar een}{of twee keer ontmoet}\\

\haiku{Zijn naam stond voluit.}{op de achterzijde van}{zijn portretfoto's}\\

\haiku{Uit Engeland zijn.}{geen brieven van Couperus}{aan Wilde bekend}\\

\haiku{Een Lent van Vaerzen.}{was in 1884 bij J.L. Beyers}{te Utrecht verschenen}\\

\haiku{s Avonds draag ik nu,}{mijn oranje roos maar of ik}{het altijd zal doen}\\

\haiku{138Schertsenderwijs,.}{voor Trooper Peter naar de}{Engelse titel}\\

\haiku{Van een vertaling.}{van Noodlot van haar hand is}{het nooit gekomen}\\

\haiku{197De {\textquoteleft}beruchte{\textquoteright}.}{vrouwelijke hoofdpersoon}{uit De Stille Kracht}\\

\haiku{228Ten Brink is in.}{1901 op 67-jarige}{leeftijd overleden}\\

\haiku{289Vermoedelijk,.}{Mr. Dr. Willem Frans Donker}{Curtius geb. 1882}\\

\haiku{304Dit is, volgens,.}{wens dus van Couperus zelf}{zo ook uitgevoerd}\\

\haiku{Als patriot en.}{nationalist heeft hij}{veel gepubliceerd}\\

\haiku{Na de oorlog werd.}{hij ondersecretaris}{van Schone Kunsten}\\

\haiku{De twee bedongen.}{bundels groeiden tenslotte}{uit tot een vijftal}\\

\haiku{Roelvink nam regie.}{en mise-en-sc\`ene}{voor zijn rekening}\\

\haiku{Niet uit afgunst, maar '.}{omt publiek dat hem nu}{hemelhoog verheft}\\

\subsection{Uit: Nippon}

\haiku{De komedie                     ,.}{die hij anders speelde was}{nu werkelijkheid}\\

\haiku{Gedecideerd, hij,:}{w\`as een Mandarijn van het}{oude r\'egime}\\

\haiku{Dit is nog maar om.}{en bij Nagasaki}{en dit is nog niets}\\

\haiku{Er zijn steeds zieke.}{oogen                     en huidziekten aan}{wie u omringen}\\

\haiku{Gonse gaf                     rijk.}{ge{\"\i}llustreerde deelen}{over Japansche kunst}\\

\haiku{Maar ik geloof, dat.}{zij den geschiedschrijver te}{legendarisch zijn}\\

\haiku{Waardeeren doet hij zijn;}{landtongen en kaapjes en}{kronkelboomen}\\

\haiku{De voet                     zoekt den,.}{grooten steen zelfs na regen}{altijd droog en rein}\\

\haiku{Dat de Japanner,:}{hybridisch is zullen}{wij vaak bemerken}\\

\haiku{Zoowel het een als het:}{andere treffen tot in}{het uiterste}\\

\haiku{Dit laatste te doen;}{scheen het                     correctste en}{meest esthetische}\\

\haiku{Ik weet niet hoe ge,.}{over geld denkt                     maar ik meen}{dat als ik 3000 gld}\\

\haiku{iets ignobels is,.}{geschikt om duizend ziekten}{op                     te roepen}\\

\haiku{Het oude                     goud,.}{is het mooiste het is maar}{even dof of verweerd}\\

\haiku{plotseling draait de,,:}{gids die naast                     den chauffeur}{zit zich om en zegt}\\

\haiku{Zij verzorgt mij, wat.}{krachtdadig en toch teeder}{en toch weldadig}\\

\haiku{- Maar Araya, zeg ik, gel\`o\`of,??}{je nu dat je neef door een}{vos was bezeten}\\

\haiku{ik weet alleen, dat}{ik geen lust                     heb naar den}{Grooten Muur te gaan}\\

\haiku{O, ik ben maar een,:}{vlugge toerist maar ik}{ken mij een recht toe}\\

\haiku{Zeker om mij te,}{straffen slaat de vos klauwen}{uit en                     rijten}\\

\haiku{Het is zeer treffend,.}{hoe zeer velen hunner op}{apen gelijken}\\

\haiku{Nergens in dit land,.}{voel ik den gloed van                     een}{geestelijk Ideaal}\\

\haiku{En wij zouden om '?!}{elf                     uurs morgens in}{Yokohoma zijn}\\

\haiku{Hij is m\'e\'er voor hen,.}{dan de hoogste en schoonste}{berg van                     hun land}\\

\haiku{Want de dagen zijn,;}{lichtelang de zon rijst zeer}{vroeg ter kimme}\\

\haiku{Toen was beschaamd de.}{knaap en gaf aan de engel}{haar rok                     terug}\\

\haiku{De                     tuinman moet.}{van ieder boompje weten}{na hoeveel weken}\\

\haiku{Eens vond hij een dik,.}{zwaar boek                     over marine}{in een boekwinkel}\\

\haiku{- Kawamoto zeg ik,,?}{opstaande willen wij de}{karpers gaan voeden}\\

\haiku{Zijn er vijf, zes te,,.}{zamen dan spoort hij uren lang}{om ze te                     zien}\\

\haiku{Tot het schutsel, dat.}{rondom hen was opengeplooid}{hun aandacht trok}\\

\haiku{Helaas, als men geen,.}{schrijver geboren is kan}{men geen boek schrijven}\\

\haiku{Probeer ze nu eens.}{te gaan zien in den tempel}{van                     Horiuji}\\

\haiku{Haar vader, de zwaar,.}{gebouwde                     generaal}{hield veel van het kind}\\

\haiku{Zij had ook nog haar {\textquoteleft}{\textquoteright}, {\textquoteleft}{\textquoteright},.}{nurse hareama die}{alles voor haar was}\\

\haiku{Navrant is het leed.}{van het zieker en zieker}{wordende vrouwtje}\\

\haiku{de generaal brengt.}{zijn                     zieke dochter naar}{zijn buitenverblijf}\\

\haiku{Zoo ver van ons staand.}{als het Oosten maar                     kan}{staan van het Westen}\\

\haiku{Er is echter ook,.}{een verzameling goudlak}{die bizonder is}\\

\haiku{Het is                     klein, dit,,.}{kerkhof dit tempeltje een}{beetje rommelig}\\

\haiku{Zij zijn fun\`ebre,,.}{boomen grafboomen als}{onze cypressen}\\

\haiku{Zij rijen zich links,.}{en rechts meer dan                     twintig}{kilometer lang}\\

\haiku{zo\^u ik t\`och                     een?}{indruk krijgen van wat ik}{zo\^u hooren en zien}\\

\haiku{Wij zagen nog de.}{Wind-in-de-Pijnen}{en den Herfstregen}\\

\haiku{Ik herinnerde,:}{mij dien Slapenden Vos door}{Tetsuzan                     geschilderd}\\

\haiku{Ik herinnerde:}{mij de schilderij van}{Hiroshig\`e}\\

\haiku{Dit heeft in Japan.}{wel een heel ander type}{dan in China}\\

\haiku{Hij vindt de zonde;}{van de Yoshiwara}{verschrikkelijk}\\

\haiku{het groote steenen pad dwars,;}{door het mos zoodat het mos}{nooit betreden wordt}\\

\haiku{hij lijkt meer op een -:}{ouden                     markies dan op}{een zeebonk beslist}\\

\subsection{Uit: Noodlot}

\haiku{- Ik ben Robert van,... -,?}{Maeren misschien herinnert}{u zich Bertie jij}\\

\haiku{En dan altijd een, '!}{hooge hoed ens avonds altijd}{een rok met een bloem}\\

\haiku{Hij zo\^u zich schudden,...}{uit zijn zieleslaap hij zo\^u}{Bertie wegzenden}\\

\haiku{Bertie vroeg naar den;}{duur van de wandeling en}{wat men er zien zo\^u}\\

\haiku{Jullie kakelen.}{ook maar in plaats van eens naar}{het pad te kijken}\\

\haiku{De weg was breeder,}{geworden zij stegen dus}{gemakkelijker}\\

\haiku{Zij knikte en zij.}{hielpen beiden haar af te}{stijgen van de steenen}\\

\haiku{Hurk fatalistisch,;}{ne\^er als een Arabier en laat}{dag volgen op dag}\\

\haiku{Want hij was zooals hij,!}{was hij w\`as laf en kon zich}{niet veranderen}\\

\haiku{Toen een glas water,.}{en hij legde zich we\^er zich}{dwingend tot kalmte}\\

\haiku{Eigenlijk verborg,,...}{hij Frank dus Eve niets dan dat}{Bertie geen geld had}\\

\haiku{Intusschen, Bertie,.}{moest het dulden dat Frank zeer}{koel tegen hem werd}\\

\haiku{zijn fatalisme,.}{was als een godsdienst die hem}{sterkte en hoop gaf}\\

\haiku{Frank stond reeds op, om,.}{hem naar zijn kabinet te}{volgen Bertie ook}\\

\haiku{riep zij eindelijk,,,.}{uit toch nog vreezende hem}{Bertie te kwetsen}\\

\haiku{Hij is zoo open, zoo,...}{oprecht je weet zoo precies}{wat je aan hem hebt}\\

\haiku{Omdat hij een nieuw?}{leven wilde beginnen}{en nu niet meer kan}\\

\haiku{hij zo\^u daar, als het,.}{niet gerechtvaardigd is boos}{over kunnen worden}\\

\haiku{Maar dan komt het we\^er... -,,.}{terug Heusch Eve praat niet}{zulke gekkepraat}\\

\haiku{Het komt en het gaat,...}{voor een poosje weg en het}{komt en het gaat we\^er}\\

\haiku{Toe, o toe, spreek met,,}{hem enkele woorden maar}{ik bid er je om}\\

\haiku{Zij sidderde meer,:}{en meer en toen kwam het we\^er}{over haar en in haar}\\

\haiku{{\textquoteright} zeggen jullie, en.}{daarom noemen jullie niets}{wat ik wel iets noem}\\

\haiku{Maar... was het waarlijk,?}{de schuld van Frank dat hij Eve}{niet vergeten kon}\\

\haiku{Het is treurig, dat,...}{het zoo geworden is maar}{gooi het van je af}\\

\haiku{Je kan het dus niet,,.}{meenen als je zegt dat je}{er naar toe wilt gaan}\\

\haiku{In het halfduister.}{stiet Frank bij eene beweging}{even Bertie's hand aan}\\

\haiku{Ik heb net wat we.}{noodig hebben om naar Buenos}{Ayres te komen}\\

\haiku{Nou goed, hoor, ik zal,,:}{wel eens zien maar ga nu naar}{bed want ik heb slaap}\\

\haiku{Ho\^u me dicht tegen,,...}{je aan zoo in beide je}{armen in beide}\\

\haiku{Neen, neen... - Geloof je,... -... -?}{dan dat hij er belang bij}{had Ja Maar wat dan}\\

\haiku{Toen ik later over,.}{zijn woorden nadacht heb ik}{er dat in gevoeld}\\

\haiku{Maar Frank greep hem bij,:}{zijn schouders schudde hem en}{heesch brulde hij}\\

\haiku{En wat heb ik in...}{dat geslinger om me in}{evenwicht te houden}\\

\haiku{Het verleden werd:}{meer en meer het verleden}{en moest het blijven}\\

\haiku{ze wisten het toch...}{altijd beter en deden}{toch altijd hun zin}\\

\haiku{Integendeel, Eve:}{vreesde nu de geheele}{week voor dien Zondag}\\

\haiku{vroeg zij, verwonderd,.}{door heure tranen blikkend}{om zijn vreemden toon}\\

\haiku{Hare stem vloeide,}{zoet als balsem zij voelde}{om hem te sterken}\\

\haiku{Je hebt gezien, dat.}{Bertie een schurk was en je}{hebt hem doodgemaakt}\\

\haiku{Maar ik, ik voel, dat}{ik alles in mij mis om}{gelukkig te zijn}\\

\haiku{- Niets te veel, ik heb,.}{geleefd door jou zonder jou}{had ik nooit geleefd}\\

\haiku{Als de corrector,.}{het zorgvuldig doet laat ik}{het maar aan hem over}\\

\subsection{Uit: Oostwaarts}

\haiku{Ik doorleefde op...}{Java een schooljongenstijd}{van vijfjaren}\\

\haiku{vroegste uchtendmist.}{strekt dun mousseline uit}{over lucht en water}\\

\haiku{Stel je voor, dat de {\textquoteleft}{\textquoteright}?}{Prins der Nederlanden weg}{stoomde zonder ons}\\

\haiku{De vrouwen, die wij,,.}{zien zijn dikwijls blond al zijn}{ze Italiaansche}\\

\haiku{de loods besliste,,...}{echter dat het land was een}{meer en geboomte}\\

\haiku{Dan we\^er de engte,...}{van het kanaal in naar het}{Bittere Meer}\\

\haiku{Dat dringt zich aan je:}{op om je alles en nog}{wat te verkoopen}\\

\haiku{vertoonen, die U.}{nog treft aan zulke oude}{huisjes in Indi\"e}\\

\haiku{Het is een rijkdom,.}{een overstelping als alles}{is in het Oosten}\\

\haiku{eene verdieping is,;}{niet ouderwetschIndiesch en}{meer iets nieuwerwetsch}\\

\haiku{Het Europeesch,.}{effort dat hier zoo krachtig}{tot rezultaat kwam}\\

\haiku{De planter heeft zijn.}{eigen dagverdeeling en}{zijn eigen costuum}\\

\haiku{De gastvrouw is reeds,,.}{naar nieuwen trant gekleed in}{keurig wit toilet}\\

\haiku{Sarong en kabaai.}{worden nergens meer door de}{dames gedragen}\\

\haiku{Hij tjankoelt dit fijn,}{hij legt zijn zaadbedden aan}{en onderhoudt den}\\

\haiku{Ginds zijn de loodsen.}{en de administrateur}{komt ons te gemoet}\\

\haiku{De vrouwelijke.}{bloem kleurt zich van wit naar rood}{toe tot purperzwart}\\

\haiku{de afval dient tot.}{stookmateriaal van de}{locomobielen}\\

\haiku{De kar is een met,,:}{den weg een met de natuur}{een met het landschap}\\

\haiku{er me\^e, en beter,,.}{gesoldeerd hoor dat er geen}{drup ontsnappe}\\

\haiku{daarbij, de heeren.}{officieren zijn trouwe}{lezers van de h.p}\\

\haiku{- Dan zal ik voortaan,!}{zorgen dat de weg beter}{onderhouden wordt}\\

\haiku{Een rivier kronkelt.}{te voorschijn en verdwijnt we\^er}{tusschen blokken rots}\\

\haiku{Dit water noemt de, {\textquoteleft}{\textquoteright}:}{Maleier-om-de-kust}{desmakelooze zee}\\

\haiku{De Minnaar en de,.}{Bruid zij beheerschen voet bij}{voet den horizon}\\

\haiku{Nu zie ik al die.}{huizen weer en je lacht om}{die souvenirtjes}\\

\haiku{Maar hij had immers,!}{geen kindersouvenirs die}{hem bedrongen}\\

\haiku{Men houde zijn rijst.}{zelve zoo lang mogelijk}{wit en maagdelijk}\\

\haiku{Hij is te winnen,.}{door een enkel vriendelijk}{woord door een glimlach}\\

\haiku{Hij vindt het prettig.}{zijn meester een titel van}{grootheid te geven}\\

\haiku{Sangkoeriang roept,.}{zijn diengeesten te zamen}{zijne dewata's}\\

\haiku{Ik ken deze mooie.}{landen en deze wegen}{sinds twintig jaren}\\

\haiku{hier v\'o\'or u, tusschen,,}{deze bergen op deze}{aarde zij heffen}\\

\haiku{Vroolijk blijven zij,,,}{en hongerig geloof ik}{zijn zij of doen zij}\\

\haiku{er is nu niets aan,.}{te doen aan het feit dat acht}{koelies mij torsen}\\

\haiku{Af wiegelt het nu,.}{op hun schouders den bergweg}{af het oerwoud door}\\

\haiku{zijne ziel bleef nog.}{immer een feodale}{en middeneeuwsche}\\

\haiku{Het was geen spel van,.}{oorlog en helden het was}{een spel van liefde}\\

\haiku{De prins is reeds van,.}{kind af verloofd met zijn nicht}{prinses Schartadja}\\

\haiku{Een Boeddhistische,,.}{non zuster des konings komt}{den prins vermanen}\\

\haiku{Zij geven kniekus - -.}{en voetkus zeer lang deze}{kussen aan allen}\\

\haiku{Hoe moet de toerist}{het bewonderen als de}{Rezidenten zoo}\\

\haiku{Maar als ze hi\`er,...}{begonnen zijn is het d\`a\`ar}{al we\^er vol onkruid}\\

\haiku{het is zeker een,,.}{geschenk een kleine hulde}{van ik weet niet wie}\\

\haiku{Wat een vetuste!}{overblijfselen van langzaam}{wegteerende macht}\\

\haiku{de {\textquoteleft}translateur{\textquoteright} van,,.}{den Rezident vertolkte}{ze mij helaas niet}\\

\haiku{(Waarom heet deze {\textquoteleft}{\textquoteright} {\textquoteleft}{\textquoteright}?}{heer maar niet op zijn Hollandsch}{tolk ofvertaler}\\

\haiku{dan is het Rijk van.}{Kediri in deze tijden}{tot macht gekomen}\\

\haiku{De riffen in zee:}{geleken op den bouwtrant}{van poort en tempel}\\

\haiku{Het was met niets van.}{de groote Soenda-eilanden}{te vergelijken}\\

\haiku{nu en dan scheen het,;}{mij toe dat hij de stem der}{demonen nadeed}\\

\haiku{Zij gingen de hooge,,.}{gebeeldhouwde trap op in}{edele theorie}\\

\haiku{is even natuurlijk.}{als dat de rietstengel nijgt}{voor de waringin}\\

\haiku{Ik verlaat het schip,,.}{loop langen steiger af tuf}{naar het hoofdbureau}\\

\haiku{Een jonge man van ().}{wellicht nog geen veertigik}{schat moeilijk leeftijd}\\

\haiku{Dat is dus wel iets {\textquoteleft}{\textquoteright}.}{voor het Nederlandscheffort}{om trotsch op te zijn}\\

\haiku{\ensuremath{<} rij der bergreuzen}{staan153,6zingen \ensuremath{<} springen153,10in}{\ensuremath{<} op153,15kleine \ensuremath{<}}\\

\subsection{Uit: De stille kracht}

\haiku{De  marmeren;}{vloer van de voorgalerij}{spiegelde gladwit}\\

\haiku{Zij was bang, dat hij,;}{boos zo\^u blijven en dan had}{ze niets en niemand}\\

\haiku{Wat zij had willen,,...}{zijn als zij niet behoefde}{te zijn die zij was}\\

\haiku{- Die andere... zijn......,.}{\`ook valsch Mevrouw Van Does zag}{haar aan met pleizier}\\

\haiku{- Mevrouw Van Does laat,.}{ons een heele boel moois zien}{zeide zij streelend}\\

\haiku{Zij liet den brillant.}{even flonkeren en de steen}{schoot een blauwen straal}\\

\haiku{Het moet heusch van,,.}{boven zijn gevallen uit}{de goot door het raam}\\

\haiku{Hij was wat zwaar en,}{had aanleg nog zwaarder te}{worden maar toch had}\\

\haiku{Maar hij had ook een,.}{prettigen toon met ze al}{was het werken zwaar}\\

\haiku{Hij kon joviaal,.}{vriendschappelijk zijn al was}{hij de rezident}\\

\haiku{Hij zeide het niet,.}{ronduit maar iets ontsnapte}{hem in den Regent}\\

\haiku{Een paar jongelui,.}{in het wit wandelden en}{namen den hoed af}\\

\haiku{Eensklaps waren de}{huizen gedaan en langs een}{breeden weg strekten}\\

\haiku{Zij had zich gewend,;}{aan het spel der teenen aan de}{mest om de rozen}\\

\haiku{Zij was aan die haar,,:}{kenden of antipathiek}{of zeer sympathisch}\\

\haiku{En om iets goeds tot,,.}{stand te brengen heerschte}{zij met haar clubje}\\

\haiku{Zij stelde rok en,.}{witte das in en zij was}{onverbiddelijk}\\

\haiku{Het mannelijke.}{element mengde zich niet met}{het vrouwelijke}\\

\haiku{ze heeft die blauwe...}{irissen zelve geschilderd}{op Chineesche zij}\\

\haiku{Zij was vol kleine,,;}{geheimzinnige nukjes}{haatjes liefde-tjes}\\

\haiku{De rezident, hoe,.}{koel praktisch ook heeft daarin}{iets van een po\"eet}\\

\haiku{- Laten wij heusch,...}{eerlijk zijn anders is er}{geen aardigheid aan}\\

\haiku{- Ik moet over een half,...}{uur bij den rezident zijn}{maar ik ben te vroeg}\\

\haiku{Er is niets wat ik.}{buiten het geluk in mijn}{huis zoo hoog waardeer}\\

\haiku{De oudste dochter,;}{was gehuwd met een volbloed}{blonden Hollander}\\

\haiku{Zijn tuin was vol van,;}{bloemen die  zich alle}{hieven naar hem toe}\\

\haiku{maar wat het ook was,,;}{zij zag het dadelijk met}{een enkelen blik}\\

\haiku{Zocht in den cirkel,.}{een voet den hare zij trok}{den hare terug}\\

\haiku{Zij was verwonderd,.}{dertig te zijn en dit voor}{het eerst te voelen}\\

\haiku{Aan alle hoeken,.}{van het huis luisterden de}{bedienden talloos}\\

\haiku{de familie liep.}{buiten in den tuin en in}{de tjemara-laan}\\

\haiku{Zij hijgde naar adem,,,.}{half in zwijm steeds de kabaia}{open de haren los}\\

\haiku{Nooit had hij zijn vrouw,,.}{iets geweigerd hoe kostbaar}{het was wat zij vroeg}\\

\haiku{Een ontroering van.}{drukte voer veertien dagen}{door Laboewangi}\\

\haiku{de regenmoesson,,.}{onveranderlijk trad in}{op St. Nicolaas}\\

\haiku{Zij had zich nog nooit,.}{zoo gevoeld maar er was niet}{tegen te strijden}\\

\haiku{met een blikje op...?}{uw deur en een rijksdaalder}{in den hoeveel tijd}\\

\haiku{een blank gordijn van.}{regen daalde in rechte}{plooien van water}\\

\haiku{Zij misten in haar,.}{de vroolijkheid die hen eerst}{had aangetrokken}\\

\haiku{- In den waringin,,.}{van het achtererf hoog in}{de hoogste takken}\\

\haiku{Van Oudijck had,,.}{haar gehoord hij stond op kwam}{van achter het schut}\\

\haiku{Door den plassenden.}{tuin zagen zij een witte}{gedaante komen}\\

\haiku{- Als het iets is... stel,,.}{dat het iets is dat wij niet}{verklaren kunnen}\\

\haiku{Zij gaf gil op gil,.}{geheel krankzinnig van het}{vreemde gebeuren}\\

\haiku{En eens, 's avonds, even,.}{een paar passen me\^egaande}{met hem vroeg zij hem}\\

\haiku{zij maakte alle,}{jonge vrouwen en meisjes}{jaloersch en daar}\\

\haiku{genegenheid wat.}{zij er door haar behaagzucht}{in verloren had}\\

\haiku{Theo wilde zij niet,.}{meer en moederlijk deed zij}{voortaan met  hem}\\

\haiku{Want hij was het kind.}{van zijn moeder meer dan de}{zoon van zijn vader}\\

\haiku{Maar hij was te lui,.}{en te weinig helder om}{kwaad te kunnen doen}\\

\haiku{een hoogere plaats -.}{die altijd de lijn van zijn}{leven ge-weest was}\\

\haiku{onduidelijke,,.}{ridders het eene been vooruit}{in de hand een speer}\\

\haiku{Hij trok haar nu in,.}{zijn armen maar zij duwde}{hem zachtjes terug}\\

\haiku{Op Patjaram, je,,.}{oude moeder je zusters}{alles lik je maar}\\

\haiku{Men belde elka\^ar,.}{op om niets alleen om het}{pleizier te bellen}\\

\haiku{Waarlijk, Eva vond het.}{te Laboewangi toch nog}{veel gezelliger}\\

\haiku{Ziet u eens, hier in,.}{Indi\"e ben ik wat geweest}{daar zo\^u ik niets zijn}\\

\haiku{Het land heeft zich van.}{mij meester gemaakt en ik}{behoor het nu toe}\\

\haiku{Het was alles glad,.}{voor mij uit ten minste in}{mijn carri\`ere}\\

\haiku{Ik hield van mijn vrouw,,:}{ik hield van mijn kinderen}{ik hield van mijn huis}\\

\haiku{Mijn eerste vrouw was,}{een nonna die ik trouwde}{omdat ik verliefd}\\

\haiku{Hij zweeg even, toen ging,,:}{hij voort geheimzinniger}{fluisterender nog}\\

\haiku{Wat zal u blij zijn,.}{uw ouders te zien en mooie}{muziek te hooren}\\

\subsection{Uit: Van oude menschen, de dingen, die voorbijgaan...}

\haiku{hij had Lot, om te,.}{werken een van de beste}{kamers gegeven}\\

\haiku{Hij zeide het kalm,.}{heel onverschillig en nam}{zijn courant we\^er op}\\

\haiku{Laat Steyn nu rustig,,...}{wandelen denk niet meer aan}{Steyn denk aan niets meer}\\

\haiku{Dokter Thielens meent,...}{een voorbode van spoedig}{totaal doof worden}\\

\haiku{Lid van den Raad van,;}{Indi\"e nu reeds jaren lang}{gepensioneerd}\\

\haiku{- Je zo\^u misschien, met,,....}{wat tact het kunnen doen kind}{bij de Steyns wonen}\\

\haiku{Naar mevrouw Dercksz... -... -,,...}{Naar grootmama Ja ja zeg}{nu maar grootmama}\\

\haiku{In hoogruggigen,,,}{stoel als in troon zat zij recht}{gesteund door een stijf}\\

\haiku{Ik geloof, dat, \`als...,.}{ik gestorven ben hij hij}{me zal aanklagen}\\

\haiku{Dat zeg ik niet om,:}{je iets onaangenaams te}{zeggen hoor jongen}\\

\haiku{Notaris Deelhof... -?,:}{zei nog verleden Hoeveel}{zei Ina begeerig}\\

\haiku{Nu bevrijdt hij zich,:}{uit de tulle plooien en}{we\^er wil hij roepen}\\

\haiku{In den donker rilt,...}{hij en klappertandt en zijn}{oogen puilen puilen}\\

\haiku{Hij kruipt in bed, trekt,.}{de klamboe dicht trekt de sprei}{tot over zijn ooren}\\

\haiku{Hij heeft het sedert,...}{altijd gezien en hij werd}{een oude man}\\

\haiku{en... dr\`eigt er iets...?}{achter de boomstammen van}{dat eindelooze pad}\\

\haiku{maar het kwam niet van,:}{de Derckszen als tante}{Stefanie meende}\\

\haiku{Hij poogde zich te,,.}{beheerschen mannelijk te}{zijn flink en moedig}\\

\haiku{En hij streelde haar.}{op zijne beurt en gaf haar}{een innigen zoen}\\

\haiku{- Ja, zie je kind, een,:}{parapluie dat vind ik}{een ellendig ding}\\

\haiku{- Neen kind, een rijtuig.}{vind ik nog vreeslijker dan}{een parapluie}\\

\haiku{- Elly had alles,,;}{kunnen krijgen wat ze wo\^u}{zei de oude heer}\\

\haiku{De tandelooze mond,.}{trilt en mummelt de vingers}{sidderen hevig}\\

\haiku{Maar wat is die ziel,,!}{nu verstompt en wat is zij}{oud wat is zij oud}\\

\haiku{Zo\^u het dan toch zoo,:}{zijn als de menschen wel iets}{hadden gemompeld}\\

\haiku{in Indi\"e... vijftig... -,,.}{jaar geleden God wat een}{tijd huiverde Lot}\\

\haiku{- Dat is nu zestig...,.}{zestig jaren geleden}{zei Pauws droomerig}\\

\haiku{Elly wordt er maar -,!}{wit als een doek van kindje}{wat zie je er uit}\\

\haiku{Maar waarom ze niet,.}{trouwen willen dat is en}{blijft me een raadsel}\\

\haiku{Ik denk soms, dat het,...}{voorbij is dat dat alles}{voorbij is gegaan}\\

\haiku{- Niet zoo boos doen... Wat... -...}{nu van tante Ther\`ese We}{zullen maar niet gaan}\\

\haiku{En dat u niet eet... -,,.}{Ik eet ik eet zei tante}{Ther\`ese zacht en traag}\\

\haiku{Het was vaag, maar er,,.}{was eerzucht in en eerzucht}{uit liefde om h\`em}\\

\haiku{Hoe anders dan de,.}{huilwind van ons Noorden die}{zoo luguber giert}\\

\haiku{Onze natuur slaapt.}{den heelen zomer onder}{den brand van de zon}\\

\haiku{Net goed, had tante,.}{gedacht en toch hield ze wel}{van haar vogeltjes}\\

\haiku{En daar achter, wat...?}{verborg hij daar achter die}{Latijnsche boeken}\\

\haiku{We zouden van daag,,?}{samen er heen gaan niet waar}{tante Stefanie}\\

\haiku{Nog een heel korten... -,...}{tijd en d\`an Ja dan is het}{heelemaal voorbij}\\

\haiku{En tante Floor, oud,,...}{boos mandarijnengezicht}{zag naar haar man Ddaan}\\

\haiku{ik geloof dat niet -,.}{maar er is iets waarom oom}{Daan is gekomen}\\

\haiku{gezellig, vroolijk,;}{praatte Pol niet \`al te druk}{toch om grootpapa}\\

\haiku{n\`u wist hij het deel,...}{zijner moeder in de schuld}{de vreeslijke schuld}\\

\haiku{zijn buik wierp zich van,.}{links naar rechts hing nu over het}{gezonde been heen}\\

\haiku{Ik moest Harold zien,,,...}{jou zien ik moest mama zien}{ik moest Takma zien}\\

\haiku{ga dadelijk naar,.}{dokter Thielens en dan naar}{meneer Steyn de Weert}\\

\haiku{De dokter kan niets,...}{voor hem doen maar de dokter}{moet constateeren}\\

\haiku{- op den rechten stoel,...}{in die roode schemering}{van het raamgordijn}\\

\haiku{Hij ging en in het,,.}{sterfhuis de blinden dicht bleef}{tante Ad\`ele alleen}\\

\haiku{De menschen zouden,:}{er wel over praten misschien}{niet eens zoo h\'e\'el veel}\\

\haiku{En zij ging in de,.}{sterfkamer op de punten}{van haar pantoffels}\\

\haiku{Welke stemmen had,?}{hij gehoord welke stem had}{hij hooren roepen}\\

\haiku{Ik wil alleen op,,:}{de hoogte zijn \`of je erft}{en hoeveel je erft}\\

\haiku{Zij waren alleen,,:}{in het compartiment en}{zij zeide liefkoozend}\\

\haiku{Ook den volgenden,,.}{dag dien der begrafenis}{was tante Ad\`ele kalm}\\

\haiku{En dat hij me nu,,.}{en dan zoo aardig een paar}{honderd gulden gaf}\\

\haiku{- Mama zo\^u van avond,,.}{hier komen om jullie te}{zien zei tante Ad\`ele}\\

\haiku{Ik wo\^u... om al het,.}{geld van de wereld dat ik}{het niet gedaan had}\\

\haiku{lees... in Godsnaam... ter,...:}{wille van mij Steyn om het}{met me te deelen}\\

\haiku{Een afleiding was,,:}{ook dat Ina boven kwam met}{Netta op den arm}\\

\haiku{- Neen, zei Ad\`ele Takma,,,.}{en zij huiverde hier in}{huis sedert zij wist}\\

\haiku{Zij sprak niet veel, zat,...}{naast de moeder die hare}{hand had genomen}\\

\haiku{Maar zie je, kind, het...:}{z\'o\'o doen als jij het gaarne}{hebt dat kan ik niet}\\

\haiku{- Dat is lekker, met,...}{het gure we\^er meneer Lot}{en mevrouw Elly}\\

\haiku{integendeel, ze,,}{is kalm en heel blij dat ze}{mevrouw Ther\`ese we\^er}\\

\haiku{Daar stonden, zaten,,...}{de kinderen zij allen}{menschen van leeftijd}\\

\haiku{Hij plukte een paar,.}{van de druiven en zoog ze}{uit en belde toen}\\

\haiku{Zie je, John en.}{ik zien haar nooit en Mary}{komt gauw uit Indi\"e}\\

\haiku{twee zulke dotten,...}{van kinderen zulke mooie}{lieve kinderen}\\

\haiku{scheer me heel netjes,,...}{niet waar want met die baard ben}{ik afschuwelijk}\\

\haiku{Ik voel me herleefd.}{met mijn zachte vel en met}{mijn zijden hemdje}\\

\haiku{Lot nam zijn vaders -,... -!}{hand. Niet zoo spreken vader}{Verd\`omde vrouwen}\\

\haiku{Er waren mooie en,;}{interessante dingen}{vooral in Itali\"e}\\

\haiku{{\textquoteleft}Van den zomer heb,:}{ik niet gewerkt maar nu ga}{ik goed aan den gang}\\

\haiku{iets deed... \ensuremath{<} wat o\`ok...,,}{deed H14,24toch \ensuremath{<} toch toch GN}{H14,31jezelve \ensuremath{<} wij}\\

\haiku{39De correcte.}{lezing is overgenomen}{uit het kladhandschrift}\\

\section{Johanna Desideria Courtmans-Berchmans}

\subsection{Uit: De gemeente-onderwijzer}

\haiku{Op 25jarigen.}{ouderdom trad zij met hem}{in het huwelijk}\\

\haiku{*** In haren rampspoed.}{verloor Vrouwe Courtmans geen}{oogenblik het hoofd}\\

\haiku{Letterkunde als.}{zuivere kunst beschouwen}{daar dacht niemand aan}\\

\haiku{beiden vonden dat.}{het goed was en tevreden}{waren zonder meer}\\

\haiku{Het volk gaat me\^e of,}{gaat niet me\^e de schilders gaan}{hun gang en later}\\

\haiku{Zou men inderdaad?}{niet mogen spreken van een}{verkeerd uitwerksel}\\

\haiku{De meester nam zijn,;}{bolivard af en streek zijn}{dun grijs hair omhoog}\\

\haiku{die schoone kle\^eren.}{kan hij allemaal op den}{plak gekocht hebben}\\

\haiku{iets waarover hij zich.}{natuurlijk zeer blijmoedig}{in de handen wreef}\\

\haiku{Gij zijt immers bij?}{kruideniers en beenhouwer}{ten winkel geweest}\\

\haiku{{\textquoteright} {\textquoteleft}Gij trekt ook altijd,,}{partij voor dat vreemd volkje}{Mietje Raveschoot}\\

\haiku{{\textquoteright} De meester stond op.}{de teenen om de boeren over}{de schouders te zien}\\

\haiku{uit die zoo zoet, zoo,.}{hartroerend waren dat er}{Irma bij weende}\\

\haiku{neem de banknoot,.}{aan en geef de lessen in}{alle vriendschap voort}\\

\haiku{{\textquoteleft}Maar het nemen van.}{dat jongsken was ook al eene}{slechte spekulatie}\\

\haiku{{\textquoteright} vroeg de meester, die.}{belang in de samenspraak}{begon te stellen}\\

\haiku{Kozijn, zou er wel?}{een doove in W. zijn die}{het geval niet kent}\\

\haiku{nu stegen hare.}{kreten in luide klanken}{met het gebed op}\\

\haiku{{\textquoteleft}Maar waarom bleeft gij?}{aan het mastbosch met Mietje}{Raveschoot achter}\\

\haiku{{\textquoteright} {\textquoteleft}Denkt gij dat ik de?}{konkurrencie van zulk eenen}{man  moet vreezen}\\

\haiku{{\textquoteright} {\textquoteleft}Het zij zoo,{\textquoteright} was het,.}{antwoord en de twee vrienden}{gingen aan het spel}\\

\haiku{Ook vielen zij als.}{lavende hemeldauw op}{zijn brandend gemoed}\\

\haiku{zoo beminden zij,.}{wel zonder voldoende hoop}{maar ook zonder vrees}\\

\haiku{{\textquoteleft}Sa, pater kies er,{\textquoteright};}{een nonneken uit koos zij}{Mietje Raveschoot}\\

\haiku{en zij schonk de twee.}{beste vriendinnen nog eenen}{kus en eenen handdruk}\\

\haiku{maar helderer dan;}{de glans harer oogen straalde}{de blik harer ziel}\\

\haiku{Doch alhoewel de;}{inrigting der kostschool hem}{zeer verontrustte}\\

\haiku{ten voordeele van het,.}{volksonderwijs spreekt mogen}{wij niet herkiezen}\\

\haiku{{\textquoteright} {\textquoteleft}Ik geloof, om de,}{waarheid te zeggen dat gij}{beter dan iemand}\\

\haiku{{\textquoteright} {\textquoteleft}De ongesteldheid,,{\textquoteright}.}{is reeds voorbij Mijnheer Van}{Dale ze{\^\i} Irma}\\

\haiku{Viel mij echter dat,,;}{onverhoopt geluk te beurt}{dan gij kent Irma}\\

\haiku{{\textquoteright} {\textquoteleft}Ja, dit zal mijnheer,{\textquoteright}.}{Blommaert ons niet verhuren}{bemerkte Roosje}\\

\haiku{Ga, hij is in de.}{groote kamer bezig met de}{gazet te lezen}\\

\haiku{Anders zou men er,{\textquoteright},}{zoo druk niet in arbeiden}{antwoordde Tonia}\\

\haiku{Edward Van Dale,.}{en Irma Blommaert beiden}{alhier woonachtig}\\

\haiku{{\textquoteright} {\textquoteleft}In den tegenspoed,:}{stond Edward pal gelijk de}{volwassen woudeik}\\

\haiku{Het zwakke Roosje,,.}{was na eene kortstondige}{ziekte overleden}\\

\haiku{Slechts bij u, bij uwen.}{vader en uwe moeder vind}{ik mij hier te huis}\\

\haiku{de mantel die van,.}{haar hoofd tot op den grond daalt}{schittert als diamant}\\

\haiku{Prudens, die jonger,;}{is dan Hector blijft nog een}{jaar voortstuderen}\\

\haiku{een ongeluk komt,:}{nooit alleen en heden zou}{ik kunnen zeggen}\\

\haiku{{\textquoteleft}Vloet ik niet zorgen?}{om mijne kinderen iets}{achter te laten}\\

\haiku{Hoe jammer is het,{\textquoteright}.}{dat meester Save dood is}{zeiden de meisjes}\\

\section{Frits Criens}

\subsection{Uit: Vergaete aorlog}

\haiku{Ich k\'os waal janke.}{wie ze mich vanmiddig nao}{mien kamer brachte}\\

\haiku{Op eine kier waas.}{ich het kantoe\"er van de}{baas aan het poetse}\\

\haiku{Van die angere.}{wis allein de Boer waem en}{waat ich waas gewaest}\\

\haiku{Nemes geluifdje.}{det dees evakuasie weer zoe\"e}{v\"a\"orbie  zooj zeen}\\

\haiku{De Spass waas in\`ens.}{geine Spass mier en doerdje}{dan ouch mer efkes}\\

\haiku{Wie hae gen\'og haaj, sjtook}{hae de fles tr\"ok en sjpiensdje}{\'onr\"ostig de zaal}\\

\haiku{eine sjlaag van,.}{ziene houte poe\"et en}{ich waas d'r gewaest}\\

\haiku{Wie rijk wil worden,,{\textquoteright}.}{moet bij de mond beginnen}{waas h\"a\"ore vaste kal}\\

\haiku{waar, want die sjloge.}{alles in v\"a\"orraod op}{wie de eikuurkes}\\

\haiku{nog mier bewaeging}{in het gammel sjevaak det}{neet mier te haoje}\\

\haiku{Mer ich kwoom tr\"ok in,.}{ein kapot gesjaote}{sjtad moorzeels allein}\\

\haiku{het hoeshaoje waal}{\"orges t\"osse Li\"ewe en}{Frieslandj begrave}\\

\haiku{Allebei de borste,.}{ginge d'raan m\`et de kliere}{\'onger de erm}\\

\haiku{Achteraaf is.}{det mesjien sjt\'om want ich kwoom d'r}{door en m\'os nao hoes}\\

\haiku{De Magere waas.}{hel op waeg het werk aan mien}{lief aaf te make}\\

\haiku{M\`et waat ich in het,.}{laeve geli\"erdj h\"ob sjeet}{ich noe gein poes op}\\

\haiku{Mer ich h\"ob mich in.}{die k\`erk geine kier op}{mien gemaak geveuldj}\\

\haiku{Groe\"etvader en.}{vader wore allebei}{k\`erkemeister}\\

\haiku{{\textquoteright} vroog de geistelik,.}{wie hae de formuliere}{in m\'os v\"olle}\\

\haiku{Mer die vergote.}{det h\"a\"ore God miense ouch in}{de kop kan kieke}\\

\haiku{Vanaovendj moog ich,.}{in ein echt likbad v\"a\"or het}{ierst van mien laeve}\\

\haiku{Miene m\'ondj is de.}{bewaeging van het baeje}{nog neet verli\"erdj}\\

\haiku{as ze naeve de.}{trein op ginge \'om \'os nao}{binne te jage}\\

\haiku{Zoe\"e drejts se de.}{diekste liene dook nog in}{ein houmou kapot}\\

\haiku{Groe\"etvader waas.}{op waeg doe\"ed te waere}{mer ging neet doe\"ed}\\

\haiku{m\`et de ganse buurt,}{en wie hae de negendje}{daag nog laefdje zoog}\\

\haiku{ze v\"a\"or maagd oet Belsj.}{in B\"oggeme gek\'omme en}{dao later getrouwdj}\\

\haiku{Mer vanaaf dae tied dors.}{ich in b\`ed al gaaroets gein}{oug mier toe te doon}\\

\haiku{Daor\'om b\`en ich ouch,.}{det werk gaon doon \'omdet}{ich van dae miens heel}\\

\haiku{Ich goof te v\"a\"ol \'om,.}{hem \'om det ein laeve lang}{van hem te wille}\\

\haiku{Het waas door al die.}{kiere opgeloupe tot}{ein sjoe\"en bedraag}\\

\haiku{Same h\"obbe we,.}{de evakuasie \"a\"overlaefdj}{de w\`ekker en ich}\\

\haiku{Nao det bad h\"ob ich.}{mich \`ens tegooj bekeke}{in de sjpegel}\\

\haiku{In B\"oggeme sjt\'ong.}{in miene tied de waereldj}{zoe\"eget sjtil}\\

\haiku{Ich zooj neet weite.}{waat in miene j\'onge tied}{neet good is gewaest}\\

\haiku{Tien waas mer fien van.}{sjt\"ok en haaj zeker gein posjtuur}{\'om zich te houwe}\\

\haiku{En haod in de.}{duustere good en kwaod}{volk mer \`ens oetein}\\

\haiku{Daor\'om sjaote.}{die Ingelse drek as ze}{emes  trappeerdje}\\

\haiku{As ze eier te,,}{min haaj sjtook ze d'r ein paar}{bie oet ein anger}\\

\haiku{Op het l\`etst van;}{de aorlog t\`eldje allein}{mer dien sjaemel pens}\\

\haiku{Det gemurmel duit.}{mich dinke aan ein besjeting}{door de Ingelse}\\

\haiku{De piep loog m\`et de.}{aope kantj in de lingdje}{van de kogelbaan}\\

\haiku{Het insigste waat.}{nog ein bietje w\`erktj}{zeen mien gedachte}\\

\haiku{Sjt\'om eigelik, die:}{veugel doon niks angers as}{waat miense ouch doon}\\

\haiku{De Boer trouwdje in,}{het begin van de aorlog}{drek naodet zien}\\

\haiku{Duits gehuerdj,{\textquoteright} zag.}{de Boer mich sjtraks en hae}{jankdje opnoeuw}\\

\haiku{Ich wis neet wie het,.}{m\`et h\"a\"or ging allein det ze}{al bevriedj wore}\\

\haiku{Det vader net krank,,.}{waas wis ich neet det ze nao}{Baoksem m\'oste ouch neet}\\

\haiku{Aan zichzelf en h\"a\"or,}{twi\"e klein kranke wichter haaj}{ze de henj al vol.}\\

\haiku{Ich trok zoe\"e wit weg.}{det de kl\"a\"or allein al mich}{zooj verraoje}\\

\haiku{flot achterein m\`et de.}{luip van het gewaer in het}{kruuts en \'ongerlief}\\

\haiku{De naef probeerdje,.}{\'omhoe\"eg te k\'omme mer}{zakdje inein}\\

\haiku{Wie ich weer keek, zoog.}{ich h\"a\"or de ker same de}{br\"ok \"a\"overduje}\\

\haiku{Josef en Maria.}{geveuldj h\"obbe wie ze nao}{Betlehem ginge}\\

\haiku{{\textquoteleft}Nu mijn man toch niet,.}{meer hier is kan ik net zo}{goed naar Montfort gaan}\\

\haiku{zieptj, riets de vrucht die?}{aafkumtj van h\"a\"or weg en}{voors die aan de huunj}\\

\haiku{Ouch het sjlengske in,.}{mien naas m\'ot d'roet het kepke}{geit van miene m\'ondj}\\

\haiku{Achter mien ouge.}{haet emes ein groe\"ete lamp}{aangesjtaoke}\\

\haiku{Woe\"e is de d\"a\"or, woe\"e,,?}{is diene mantjel woe\"e b\`es}{se zelf woe\"e b\`en ich}\\

\section{Edward G. Croffts}

\subsection{Uit: Wie is S.S. 777?}

\haiku{{\textquoteleft}Aangenaam met u,.}{kennis gemaakt te hebben}{meneer Lewison}\\

\haiku{Degene, die hem,!}{dat koopje geleverd had}{zou ervoor boeten}\\

\haiku{- Er moet dus iemand,.}{langs gegaan zijn zonder dat}{u het gemerkt hebt}\\

\haiku{In dat geval kom;}{ik het bedrag zelf wel even}{uit uw kluis halen}\\

\haiku{{\textquoteleft}Ik ben Inspecteur,{\textquoteright}.}{MacNewvish antwoordde}{de Inspecteur droog}\\

\haiku{ik ben een beetje...,......{\textquoteright} {\textquoteleft},,}{een beetje nerveus ziet u}{enJa dat zie ik}\\

\haiku{We zullen dan eens,.}{zien of we daar geen stokje}{voor kunnen steken}\\

\haiku{Gibbs keek haar strak aan,:}{en het scheen dat zij zijn blik}{niet verdragen kon}\\

\haiku{Zij was de eenige,.........}{die hem er neergelegd kon}{hebben en dan dan}\\

\haiku{{\textquoteright} {\textquoteleft}Om u de waarheid,.}{te zeggen begrijp ik er}{minder van dan ooit}\\

\haiku{Maar als ik u een,:}{wijzen raad mag geven zou}{ik u aanraden}\\

\haiku{{\textquoteleft}Inderdaad,{\textquoteright} lachte,.}{Gibbs die aanstalten maakte}{om te vertrekken}\\

\haiku{Zijn lichaam raakte;}{echter verslapt door overwerk}{en ondervoeding}\\

\haiku{En natuurlijk wist.}{hij ook van de uitvinding}{van zijn employ\'e af}\\

\haiku{De employ\'e zag zijn;}{schoonste levensillusie}{in rook vervliegen}\\

\haiku{Lewison kende,.}{geen tegenwerking kende}{geen vernedering}\\

\haiku{nu hij oud was kwam -.}{hij tot dit bewustzijn ten}{koste van zichzelf}\\

\haiku{{\textquoteright} {\textquoteleft}Laten we er geen,{\textquoteright}.}{woorden meer over vuil maken}{antwoordde de reus}\\

\haiku{Tegelijkertijd.}{zocht hij naar zijn revolver}{in zijn achterzak}\\

\haiku{- Maar hij wist me op.}{het laatste oogenblik toch}{nog te ontsnappen}\\

\haiku{{\textquoteleft}Laat es zien,{\textquoteright} peinsde.}{hij en zware denkrimpels}{fronsten zijn voorhoofd}\\

\haiku{Van het onderzoek.}{hoorde hij gedurende}{deze dagen niets}\\

\haiku{{\textquoteright} {\textquoteleft}Natuurlijk, mijnheer,{\textquoteright}.}{Lewison antwoordde het}{meisje eenvoudig}\\

\haiku{In een ervan vond.}{hij de portefeuille van}{den automagnaat}\\

\haiku{{\textquoteleft}En lijkt het jou ook,?}{niet dat de stijl van dien brief}{erg mannelijk is}\\

\haiku{Ik geloof niet, dat.}{een vrouw op dergelijke}{wijze schrijven zou}\\

\haiku{Jim Gibbs bleef in het.}{priv\'e-kantoor achter}{en keek nog wat rond}\\

\haiku{{\textquoteright} {\textquoteleft}En zat miss Edwards,?}{op dien stoel daar naast dien van}{mijnheer Lewison}\\

\haiku{{\textquoteright} {\textquoteleft}Waar stond gewoonlijk?}{de presse-papier van}{mijnheer Lewison}\\

\haiku{Bovendien blijkt zij.}{plotseling van het kantoor}{verdwenen te zijn}\\

\haiku{{\textquoteleft}Kwaad en kwaad is twee,{\textquoteright}.}{antwoordde de Inspecteur}{op knorrigen toon}\\

\haiku{{\textquoteleft}Liefste Ned, Nog voor,.}{de nacht valt wil ik je een}{paar regels schrijven}\\

\haiku{Ik ben zoo uiterst,.}{beangst dat dit gevaarlijk}{voor je kan worden}\\

\haiku{Verder vond hij op.}{een enveloppe het adres}{van Ned Pambroke}\\

\haiku{{\textquoteleft}O, nee meneer, laat,,.}{de menschen die hier wonen}{er niets van merken}\\

\haiku{{\textquoteright} En de brave vrouw,.}{knipte de lamp aan die de}{trap verlichten moest}\\

\haiku{Dus miss Edwards heeft?}{u geholpen het gebouw}{binnen te dringen}\\

\haiku{{\textquoteleft}Gaat uw gang dan maar,, '.}{mijnheer Gibbs maar maakt u het}{kort alst u blieft}\\

\haiku{Dit hoorde ik van,.}{het personeel dat vlak bij}{hem in de buurt zat}\\
