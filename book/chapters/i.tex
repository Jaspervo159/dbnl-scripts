\chapter[2 auteurs, 88 haiku's]{twee auteurs, achtentachtig haiku's}

\section{Victor Ido}

\subsection{Uit: De paupers}

\haiku{Als ik 't maar heb,...{\textquoteright} {\textquoteleft}, '.}{BoongNonsens dat zwijn ist}{waarempel wel waard}\\

\haiku{{\textquoteright} {\textquoteleft}Nou ja... als ze Pa ', '.}{maarn groot present geven}{dan ist ook goed}\\

\haiku{Dat wist Boong heel goed, ':}{en daarom had ie stierlijk}{t land aan Vincent}\\

\haiku{{\textquoteright}, kwam hij, in nachtbroek,.}{en kabaai gestoken weer}{in den kring zitten}\\

\haiku{Aller aandacht was.}{nu gespannen op hetgeen}{hij vertellen zou}\\

\haiku{{\textquoteleft}Je weet, kind, grootpa, '....}{kan d'r niet meer tegenn}{half glaasje maar z\'o\'o}\\

\haiku{Zij ziet niet rood van, ',.}{trots maar vant vuur in de}{keuken natuurlijk}\\

\haiku{Tjang Sina, zorgzaam,}{door Tietie bediend maalde}{onder zacht gesmek}\\

\haiku{En ongeduldig,.}{werden ze wijl het al zoo}{laat was in den nacht}\\

\haiku{Deze mopperde, ';}{zei datt al zoo laat en}{zijn paard zoo moe was}\\

\haiku{hij had liever dat,.}{hij nu maar betaald werd dan}{kon hij naar huis gaan}\\

\haiku{O, die meisjes en,, ',.}{vrouwen vooral de mooie zijn}{n raadsel vond Krol}\\

\haiku{Zij snakte er naar,,....}{daarmee kennis te maken}{daarin te verkeeren}\\

\haiku{Dat was al jaren.}{zoo geweest en dat zou wel}{nooit veranderen}\\

\haiku{{\textquoteright} Sam voelde een groot*.}{verdriet rijzen in zijn}{gemoed en zweeg}\\

\haiku{{\textquoteright} Malie beurde het,,:}{hoofd dit heftig schuddend op}{en riep beslist uit}\\

\haiku{Ik houd niks van Boong, ' ',.}{t isn leeglooper die}{voor de galg opgroeit}\\

\haiku{De twijfel kwelde.}{hem al zoo lang in zijn werk}{en in zijn droomen}\\

\haiku{{\textquoteright} Vincent poogde kalm,.}{te blijven ofschoon hij ook}{niet meer gerust was}\\

\haiku{Vincent kon niet meer,.}{spreken het bloed stroomde uit}{en langs zijn lippen}\\

\haiku{In dien stillen nacht:}{streed hij een hevigen strijd}{in zijn binnenste}\\

\haiku{De vermoorde stond.}{bekend als een fatsoenlijk}{en oppassend man}\\

\haiku{Maar dan zou ze ook '....}{nog bemind worden doorn}{ander dan dien neef}\\

\haiku{'t Zou dom van haar, '.}{zijn indien zet al niet}{lang geprobeerd had}\\

\haiku{hij merkte hoezeer.}{het jonge meisje zich het}{gebeurde aantrok}\\

\haiku{{\textquoteright} {\textquoteleft}O, zeker, hij moet,,.}{je mooi vinden want je bent}{werkelijk mooi Da{\"\i}}\\

\haiku{*~          Spijtig vond hij ',;}{t niet dat grootpa weer naar}{zijn werk was gegaan}\\

\haiku{Wat bliksem, hij zou.}{toch zelf wel weten wat ie}{doen en laten moest}\\

\haiku{Kon ie dat beest maar;}{ald\'o\'or aan den praat krijgen tot}{Tjang uitgehuild had}\\

\haiku{Wat zou dat toch zijn,?}{dat ie daaraan niet langer}{weerstand kon bieden}\\

\haiku{{\textquoteleft}Je weet niet wat je,,.}{vraagt Da{\"\i} en daarom zal ik}{er niet boos over zijn}\\

\haiku{En we houden veel.}{meer van de vrijheid dan de}{totoks wel denken}\\

\haiku{De wijze, waarop,;}{zij hem toesprak stelde hem}{geheel gerust}\\

\haiku{hij was 'n Indo,*!}{en daarom wilden ze}{hem niet hebben}\\

\haiku{{\textquoteright} Boong's oogen schitterden.}{van ingehouden toorn en}{verontwaardiging}\\

\haiku{Nou ja, toch sama,,{\textquoteright}*.}{djoega zeg plaagde}{Perisa vroolijk}\\

\haiku{{\textquoteleft}We moeten trachten,....}{een verandering in den}{toestand te brengen}\\

\haiku{{\textquoteleft}Maar de bami is, ',{\textquoteright}.}{verrukkelijk dats waar}{vervolgde Reumer}\\

\haiku{Vroeger had-tie ', '....}{toch nooitn oogje op Da{\"\i}}{ik begrijpt niet}\\

\haiku{Nooit te voren had.}{hij haar ge{\"\i}nviteerd om}{samen uit te gaan}\\

\haiku{Dadelijk verdween.}{de droeve uitdrukking van}{haar mooi gezichtje}\\

\haiku{Zoo was hij op z'n:}{best gekleed en voelde zich}{deftig aangedaan}\\

\haiku{Men zou dan zien, dat,.}{zij niet alleen was maar een}{sterk geleide had}\\

\haiku{Snel was die liefde,....}{in haar opgekomen op}{het eerste gezicht}\\

\haiku{Nu, als u er geen, '{\textquoteright},.}{bezwaar tegen hebt doe ik}{t graag sprak Reumer}\\

\haiku{{\textquoteleft}U zegt nu wel, dat ', '.}{t niet waar is maar hoe kunt}{ut bewijzen}\\

\haiku{{\textquoteright} zuchtte kwasi-ernstig,.}{Reumer Nini's glas witten}{wijn opnieuw vullend}\\

\haiku{Almachtige God,,....}{Boong k\`ende die gestalte}{hij k\`ende die stem}\\

\haiku{{\textquoteleft}Ik zal tot 't laatst,,....}{toe om Lien vechten maar als}{je me vermoordt Boong}\\

\haiku{Vincent wilde niet, -.}{hebben dat hij Lien trouwde}{dat was duidelijk}\\

\haiku{Heden en toekomst.}{behoorden van toen af niet}{meer hem alleen toe}\\

\haiku{Oedit = stoffen.}{ceintuur waarmee de sarong}{wordt opgehouden}\\

\haiku{182.Serimpi =.}{hofdanseres der Sultans}{van Midden-Java}\\

\section{Ad van Iterson}

\subsection{Uit: Zuiderlingen}

\haiku{{\textquoteleft}and his mama cried...{\textquoteright}:}{En toen zei de diskjockey}{wat er was gebeurd}\\

\haiku{Je zal ons altijd.}{en overal treffen en dan}{drinken we er \'e\'en}\\

\haiku{de gangen van de -:}{economische faculteit}{om precies te zijn}\\

\haiku{Het lirium Van.}{de winter hebben we oom}{Pierre begraven}\\

\haiku{{\textquoteleft}En nu is hij dood,,{\textquoteright}.}{die arme Pierre zei de}{moeder van Nelly}\\

\haiku{maar... maar had hij nu,!}{maar naar haar geluisterd dan}{was het niet gebeurd}\\

\haiku{{\textquoteleft}Het spijt me, jong, we -,?}{hebben geen oud papier ach}{val om ben jij het}\\

\haiku{Mijn zoons hebben hem,.}{Pele genoemd omdat het}{ook zo'n zwarte is}\\

\haiku{{\textquoteright} zei hij, maar voegde,,:}{er toen ik het hek al had}{opengedaan aan toe}\\

\haiku{{\textquoteright} zei hij, terwijl hij.}{maar met moeite zijn evenwicht}{wist te bewaren}\\

\haiku{Oom Pierre, die dan,.}{al uren in Brand's Bierhuis zat}{zag ik haast nooit meer}\\

\haiku{Hij zeeg helemaal,.}{achterover van het lachen}{totdat zijn hoed viel}\\

\haiku{De volgende dag.}{in alle vroegte zijn ze}{hem komen halen}\\

\haiku{Daar hebben we de:}{klassieke tegenstelling}{uit de zielsleer weer}\\

\haiku{vijf met de claxon en.}{drie met de pneumatische}{deurinstallatie}\\

\haiku{De witte pater.}{beschreef hen in enkele}{brede verfstreken}\\

\haiku{De weg tonen en -?}{de waarheid brengen wordt dat}{nog wel eens gedaan}\\

\haiku{Daar in de buurt op.}{een hoek wist hij een caf\'e}{waar veel werd gekaart}\\

\haiku{{\textquoteleft}Vanmiddag kreeg hij,.}{opeens geen lucht meer hebben}{ze me net verteld}\\

\haiku{Omdat Pierre te,.}{muzikaal bleek werd hij op}{de tuba gezet}\\

\haiku{{\textquoteright} zei Caroline, die.}{nog steeds duizelig van de}{pirouettes was}\\

\haiku{{\textquotedblleft}De allergrootste.}{reden was toch dat hij het}{gewoon in zich had}\\

\haiku{Deze herschikking.}{gaf de groepsdans n\'og meer vuur}{en locomotie}\\

\haiku{{\textquotedblright} En ik spring, tussen,,.}{twee boten door whaaf het meer}{van Bolsena in}\\

\haiku{zou hij soms naar haar?}{benen kijken in plaats van}{naar haar onderrok}\\

\haiku{Als ze een heuvel,,}{opreden keek ze aan \'e\'en}{stuk achterom bang}\\

\haiku{{\textquoteright} {\textquoteleft}Meer op Belgi\"e en,{\textquoteright}.}{Duitsland geori\"enteerd}{knikte Tiny Ummels}\\

\haiku{Hier groeien planten.}{die nergens anders op de}{wereld voorkomen}\\

\haiku{Ze wierp een steelse,.}{blik opzij maar ze zag niets}{bijzonders aan hem}\\

\haiku{{\textquotedblleft}Het onweer zal wel,,{\textquotedblright}:}{overdrijven want het komt van}{Belgi\"e af zei hij}\\

\haiku{Meneer Rudy staat.}{met een glunderend gezicht}{onder de schoolbel}\\

\haiku{En of ze nou echt,,.}{precies zo zijn gebeurd jong}{dat maakt geen tuit uit}\\

\haiku{Ik herinner me.}{dat we een keer in de trein}{naar Brussel zaten}\\

\haiku{Ich k\"usse ihre Hand.{\textquoteright} -!}{ik geef hun een handkus gans}{in het nette hoor}\\

\haiku{En wat een geduw!}{en een getrek van al die}{carnavalsgekken}\\
