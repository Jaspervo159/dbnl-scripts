\chapter[24 auteurs, 3257 haiku's]{vierentwintig auteurs, drieduizendtweehonderdzevenenvijftig haiku's}

\section{H.H.J. Maas}

\subsection{Uit: Het goud van de Peel}

\haiku{- Dat kund' ummers nie... -,, '...}{Och erm dingske mit \'e\'e\"en}{haend draagk ow weg}\\

\haiku{En het eind was, dat.}{de meid over zes weken zou}{moeten vertrekken}\\

\haiku{Niemand nam immers? '}{werkvolk in dienst om het te}{laten leegloopen}\\

\haiku{Toen moest de mest over.}{het land gebracht worden voor}{het winterkoren}\\

\haiku{Ondanks het gezwoeg,.}{voelde zij dat een kou haar}{lijf overhuiverde}\\

\haiku{Dat kwam bij jonge, {\textquoteleft}{\textquoteright}.}{vrouwen wel meer voor vooral}{bij den eersten keer}\\

\haiku{en als de winter, '...}{om was zou ment weer duur}{moeten inkoopen}\\

\haiku{Hij wist er niets van,,.}{keek er verwonderd van op}{niet-geloovend}\\

\haiku{Hij was echter groot,.}{en sterk en daarom was hij}{heel vroeg gaan dienen}\\

\haiku{En de zorg begon,.}{pijnlijk te knagen aan hun}{leven voor het eerst}\\

\haiku{Was de winter toch,...}{maar om de zomertijd zou}{werk genoeg brengen}\\

\haiku{Maar kort daarna wist, {\textquoteleft}'{\textquoteright},.}{ze datt al zoo was toen}{ze den kerkgang deed}\\

\haiku{Ze was toen al weer {\textquoteleft}'{\textquoteright},.}{n heel eind maar ze spande}{zich in wat ze kon}\\

\haiku{Dan zou het ergste,.}{gelejen zijn beurden ze}{zich zelf hopend op}\\

\haiku{'t Was er een van....}{een ander dorp na een week}{ging-ie weer terug}\\

\haiku{dat het er met z'n.}{twaalf en een halven gulden}{leelijk zou uitzien}\\

\haiku{Maar de twee helpers,,!}{die Toon gevraagd had moesten toch}{ook betaald worden}\\

\haiku{Hij zat in 'n klein,.}{huisje met z'n vrouw die men}{zelden te zien kreeg}\\

\haiku{D'r werd gezopen,...}{zooveel als ze maar door de}{keel konden krijgen}\\

\haiku{... maar later werd het...}{verbranden op de meeste}{plaatsen verboden}\\

\haiku{Die uitdrukking had.}{de spreker onthouden van}{den burgemeester}\\

\haiku{Na eenige jaren.}{had de maatschappij al het}{grauwe veen verwerkt}\\

\haiku{de eenvoudige...}{dorpelingen ophitsen}{tegen het gezag}\\

\haiku{Bovendien bracht het.}{hem enkele honderden}{guldens per jaar op}\\

\haiku{Dat orgel-spelen...}{zou-ie voorloopig niet}{laten varen}\\

\haiku{'t Ging allemaal, '.}{zoo goed in zijn tijdt was}{een plezier geweest}\\

\haiku{Daarna sneed zijn stem,:}{knerpend als het scheuren van}{een ijzeren plaat}\\

\haiku{hij liet het nog eer, '...}{in brand steken dant voor}{zoo'n geld te geven}\\

\haiku{- Ja, het w\`as wel wat,..}{veel maar dat wilde hij er}{toch voor betalen}\\

\haiku{Haar broers verlolden.}{op \'e\'en avond meer dan hij in}{een week verdiende}\\

\haiku{Plotseling rukte,.}{het dier aan schichtig door het}{slaan en het geraas}\\

\haiku{Gewoonlijk werden.}{in plaats hiervan andere}{regels gezongen}\\

\haiku{Wat had hij toen veel!}{gezien zonder er iets van}{geleerd te hebben}\\

\haiku{Sinds hij echter den,.}{strijd aanvaard had kon hij dien}{niet meer ontvluchten}\\

\haiku{Als daar een kanaal...}{naar toe werd gegraven of}{een tram aangelegd}\\

\haiku{De meesten hadden,,.}{klompen bruin van aangeplakt}{moer aan de voeten}\\

\haiku{wel wil ik gaarne,...}{toegeven dat we niet te}{veel keuze hebben}\\

\haiku{daarom vind ik het,...}{juist zoo vreemd dat u niet van}{Den Schoolmeester houdt}\\

\haiku{Eindelijk begon,:}{hij geirriteerd doordat hij}{geen ontkomen zag}\\

\haiku{Neen, door den dag kwam. ' '.}{er weinigs Zaterdags}{ens Zondags meer}\\

\haiku{Het mes sneed er aan,,.}{twee kanten helderde hij}{nog op grinnekend}\\

\haiku{Hij had den naam van...}{in allerlei gemeene}{straatjes te sjouwen}\\

\haiku{Hoe konden ze nu?}{toch eigenlijk in zoo iets}{nog plezier hebben}\\

\haiku{Daar kan je jezelf,...}{gaan liggen kietelen as}{je wat hebben wilt}\\

\haiku{- Ja, die was verdomd,...,...}{goed z\`eg en een mooie meid ook}{die de mop voordroeg}\\

\haiku{Ze zetten hem op,.}{de tafel en dan sloeg ie}{een onzin uit bar}\\

\haiku{Aan het station.}{Lizaveen stapte niemand}{in dan De Visscher}\\

\haiku{ik had niet gedacht,...}{dat ik mij in dien mensch zou}{kunnen vergissen}\\

\haiku{En wat ze na haar,.}{dood kon nalaten dat was}{voor de familie}\\

\haiku{Opeens wendde zij,.}{zich naar hem toe of-ie niet}{naar de hoogmis ging}\\

\haiku{Deed bestraffend, dat '...}{de kapelaant toch ook}{al goed zou meenen}\\

\haiku{Waar haalden zij de?}{kunst vandaan om hun leven}{zoo in te richten}\\

\haiku{Hij ging nog wel naar, '.}{de kerk maar dat deed-ie}{alleen voort oog}\\

\haiku{'t Schijnt, dat ik in...}{de wieg gelegd ben om den}{strijd uit te vechten}\\

\haiku{Voor het onderhoud.}{van het gezin schiet er dan}{zooveel niet meer over}\\

\haiku{Nou, dan maar voor drie...?...}{maanden en zou er dan goed}{voor gezorgd worden}\\

\haiku{De gemeente zou,.}{er wel bij varen als zij}{haar veen verkocht had}\\

\haiku{Het bevatte een:}{lange correspondentie}{als hoofdartikel}\\

\haiku{iemand zijn toevlucht,.}{neemt die verblind is door haat}{tegen het goede}\\

\haiku{Beter dan thuis,  .}{want zij hadden geen tijd om}{maar hen te kijken}\\

\haiku{ik dacht, breng het prul,...}{aan meneer pastoor dat die}{het verbranden kan}\\

\haiku{Om het verband goed,.}{te kunnen leggen spelde}{hij het hemd omhoog}\\

\haiku{Na een paar dagen.}{verzonden zij de copie}{naar de drukkerij}\\

\haiku{- Ja, jongen, 'k zal... - '...}{toch weg moetent Leven}{is niet makkelijk}\\

\haiku{Nu vele Groeten.}{van mijn vrouw of Fientle wie}{ik ze noemen wil}\\

\haiku{Hen uit de verdienst,,.}{schoppen alsof zij honden}{waren dat kon-ie}\\

\haiku{Daar moest het publiek,...}{buiten blijven het had er}{niets mee te maken}\\

\haiku{Ja, zooals verteld werd, ',...}{wast h\`em immers ook zoo}{gegaan met zijn baas}\\

\haiku{Maar met de verdienst '.}{wast bij de maatschappij}{toch een boel beter}\\

\haiku{Zij zelf hadden voor.}{hun trouwen in Pruisen een}{ledikant gekocht}\\

\haiku{in 't ongeluk,...}{stooten als-ie een woord te}{v\'e\'el naar hun zin zei}\\

\haiku{En hij was dan toch...}{een bittere vijand van}{den secretaris}\\

\haiku{Jennesen en zijn.}{aanhangers waren woedend}{over zijn kandidaatuur}\\

\haiku{Maar de ander was,...}{toch wel bang dat Van Eijzen}{het niet halen zou}\\

\haiku{Allee, de lui moesten,...}{maar drinken al kostte het}{hem duizend gulden}\\

\haiku{Zijt gij vergeten,?}{dat de steenen van den oven van}{een raadslid kwamen}\\

\haiku{Er kwamen handen.}{tekort om jenever en}{bier aan te voeren}\\

\haiku{Als het ingevuld,.}{was met den naam Van Eijzen}{dan deugde het niet}\\

\haiku{'s Morgens hing een.}{mand zonder bodem bij van}{Eijzen aan de deur}\\

\haiku{De gemeente moest,.}{ook voor de godsdienstige}{belangen zorgen}\\

\haiku{Hij woelde door zijn, {\textquoteleft}{\textquoteright}...}{bed al maar pratend over turf}{en aken enstriekke}\\

\haiku{Fien en Dien snikten.}{hardop en vluchtten uit het}{kleine slaaphokje}\\

\haiku{Ze moest zeker dien...}{vreemden bocht ook nog laten}{zat-vreten}\\

\haiku{Hij had gezegd, dat...}{de vrome menschen niet te}{vertrouwen waren}\\

\haiku{De regeering zoog hen,.}{uit deed totaal niets om den}{landbouw te steunen}\\

\haiku{Bij het onderzoek...}{van hoogerhand was er zelfs}{nog te v\'e\'el in kas}\\

\subsection{Uit: Landelijke eenvoud}

\haiku{Het was dan  toch,.}{ook schand voor de ouwers als}{de dochter zoo deed}\\

\haiku{Ja, het eerste jaar,!}{maar over een tijdje zou het}{wel anders worden}\\

\haiku{En die was tegen,...}{haar uitgevallen dat ze}{haar mond moest houden}\\

\haiku{Dit zou ze haar voor,,,,....}{de voeten gooien neen nog}{beter harder z\'o\'o}\\

\haiku{ge het ummers ow,...}{meid ien de stad daor zien}{ja veul m\`oier megjes}\\

\haiku{Dat was zoo maar een,,.}{kaart geweest beweerde hij}{om te versturen}\\

\haiku{Ze wist, dat moeder,.}{dan bad dat deed ze altijd}{in zoo'n gevallen}\\

\haiku{Iederen dag zou, '....}{ze voortaan naar hem toegaan}{dat zet zagen}\\

\haiku{Hard zette zich haar.}{willen tegen het geweld}{van haar ouders in}\\

\haiku{Nel hoefde d'r nooit,.}{op te rekenen dat hij}{zijn toestemming gaf}\\

\haiku{De trekken van zijn,.}{gezicht verstrakten omdat}{Nel nog tegenhield}\\

\haiku{In vol ontwaken.}{van hun verlangen hielden}{zij elkander vast}\\

\haiku{maar niet meer schreien...,...}{maar naar huis gaan als vader}{en moeder riepen}\\

\haiku{De dokter trachtte,.}{haar te bedaren dat het}{toch z\'o\'o erg niet was}\\

\haiku{Ze zou nu wel gaan,.}{trouwen en dan was alle}{leed weer vergeten}\\

\haiku{dat iets vreeselijks,.}{haar wachtte van haar ouders}{en van de menschen}\\

\haiku{Maar zij begreep het, {\textquoteleft}{\textquoteright}.}{wel dat kwam allemaal van}{hetfusegerei}\\

\haiku{Ze lieten de lui, {\textquoteleft}{\textquoteright}...}{maar raak kletsen die konden}{hunde mars lekken}\\

\haiku{eigenlijk zou men...}{met h\`em geen medelijden}{hoeven te hebben}\\

\haiku{'t Was nu eenmaal,,.}{zoo en waar d'r twee kijven}{hebben d'r twee schuld}\\

\haiku{Klokslag twaalf ging hij,.}{aan tafel zitten of het}{eten klaar was of niet}\\

\haiku{Terwijl hij wegging,.}{snauwden ze elkander de}{scheldwoorden nog toe}\\

\haiku{In alles deed hij,.}{zijn wil zonder haar zelfs ooit}{iets te vertellen}\\

\haiku{*** Driek had thuis nog niet, {\textquoteleft}.}{dadelijk gezegd dat hij}{met Nelmoest trouwen}\\

\haiku{Hij meende zeker,,.}{dat hij den vogel af had}{als hij getrouwd was}\\

\haiku{De teleurstelling.}{zweepte zijn woede nog aan}{tot h\`arder zwoegen}\\

\haiku{Voor een beetje geld.}{kan hij dan toch zonder zorg}{alles afwachten}\\

\haiku{Zijn kameraden.}{durfden eerst niet ronduit zijn}{partij te kiezen}\\

\haiku{Hannes Knik zeurde.}{zijn haat uit tegen zijn}{eigen gezelschap}\\

\haiku{Goemans raasde nog,:}{na zijn vuist telkens op de}{tafel bonkerend}\\

\haiku{Hij had gezien van,.}{stadsche heeren Hoe hij moest}{gaan in fijne kleeren}\\

\haiku{Maar mooie Nel kon hem,,.}{behagen Hij dacht ik zal}{het maar eens wagen}\\

\haiku{Die lag als een brand.}{op zijn ziel en gloeide den}{haat voortdurend aan}\\

\haiku{Eindelijk, opeens,,.}{schoten de schreeuwstemmen uit}{allemaal gelijk}\\

\haiku{Zij zou er geld voor,..}{kwijt willen zijn als Driek die}{Nel nooit had gezien}\\

\haiku{- Jao mer, ik wee\"et,..}{ummers van niks hoe wil ik}{wat kunne zegge}\\

\haiku{Als ze zeiden, dat,.}{ze wat gezien hadden dan}{zou\"en ze liegen}\\

\haiku{Voor h\`a\`ar was daarom ',,.}{alt zware werk spitten}{zaaien en maaien}\\

\haiku{Dikwijls moest ze even,,.}{rusten steunend op de schop}{tot het wat over trok}\\

\haiku{Want het zou er nu.}{toch wel alle dagen om}{te doen kunnen zijn}\\

\haiku{Langzaam reide zich.}{de eene zwarte aardevoor}{aan de andere}\\

\haiku{Daar, ze begreep n\`og, '.}{niet dat zer het leven}{bij had gehouden}\\

\haiku{Anders mankeeren die.}{mevrouwen altijd w\`at in}{de laatste dagen}\\

\haiku{Daar was niks aan te,..}{veranderen zoo was de}{wereld geschapen}\\

\haiku{En wie trouwde er,?}{in dezen slechten tijd nog}{v\'o\'or het moetens werd}\\

\haiku{Die dan, begrijpend,.}{elkander aankijken en}{naar Willem wijzen}\\

\haiku{Alle middelen.}{wendden zij aan om hem dat}{gebrek af te leeren}\\

\haiku{Zijn dunne lippen.}{scherpen zich vast op elkaar}{tot een harden rand}\\

\haiku{'k wil niet, 'k doe '.. -, '.}{t niet Verschoor ik hebt}{goed met je gemeend}\\

\haiku{'t Is jou eigen,,?}{schuld dat je straf hebt zie je}{dat nou zelf ook niet}\\

\haiku{Hoe kan ik dan je, '?..}{belofte gelooven dat je}{t niet meer doen zult}\\

\haiku{Vlijmend wee doorschrijnt, '..}{zijn borst zijn ziel kreunt vant}{jammerende leed}\\

\haiku{Nogmaals klopte ze:}{op de deur en riep met eene}{mistroostige stem}\\

\haiku{{\textquoteleft}'k Wacht op iemand,{\textquoteright},,}{antwoordt Marie maar hij zal}{toch niet meer komen}\\

\haiku{Zijn hart sprong op van.}{blijdschap en hij begon nog}{sneller te loopen}\\

\haiku{ze zou trachten zich.}{niet meer door haar gevoel te}{laten medesleepen}\\

\haiku{Och, kom, ik vind het,.}{nu zelfs aardig om me zoo}{eens te verrassen}\\

\subsection{Uit: Peel omnibus}

\haiku{Ja, als haar moeder,;}{weer beter was zou zij een}{nieuwe dienst zoeken}\\

\haiku{Nee, in het dorp had,.}{ze niks in de stad was het}{veel plezieriger}\\

\haiku{Ze moest maar naar huis,.}{gaan als ze niet wilde doen}{wat zij aanraadden}\\

\haiku{Door de praatjes van. '}{het volk zouden die daartoe}{genoodzaakt worden}\\

\haiku{En het is God zelf,,.}{die het gezegd heeft dat er}{armen moeten zijn}\\

\haiku{Hij had een huisje,.}{op het oog met wat grond}{dat paste hem juist}\\

\haiku{In bijzijn van haar.}{man durfde Floortje hem niet}{meer voor te spreken}\\

\haiku{Toen hadden ze een,.}{zware mars gehad honger}{en dorst geleden}\\

\haiku{Eer de dag om was,.}{was het gepraat het hele}{dorp doorgetrokken}\\

\haiku{De zieke ging hard '.}{achteruit en opeens was}{t afgelopen}\\

\haiku{Het zou een eerste,.}{klas begrafenis zijn zeer}{deftig en plechtig}\\

\haiku{Als hij maar eens kwam,.}{en wat geld meebracht dan zou}{het wel weer goed zijn}\\

\haiku{Jan moest maar zuipen,, '.}{zoveel als hij lusttet}{kwam er niet op aan}\\

\haiku{En dikwijls al had,.}{het hele dorp verteld dat}{hij weer ging trouwen}\\

\haiku{Dat zou wat anders.}{zijn dan zich bij de boeren}{kapot te werken}\\

\haiku{De vrienden zouden.}{hen uitgeleide doen naar}{het station}\\

\haiku{Wat van zijn verdienst,?}{afgeven waarvoor hij zo}{hard moest arbeiden}\\

\haiku{Alleen een klaagbrief.}{van zijn moeder kwam Jan nu}{en dan hinderen}\\

\haiku{Ik wil hem nu niet,,!}{in huis hebben ziet ge en}{daarmee is het uit}\\

\haiku{Allemaal naar de,,!}{piotten potverdikke}{vivat de piot}\\

\haiku{Maar de Rooie van de.}{wethouder kwam met hoge}{stem tussenbeide}\\

\haiku{Wankelde nu naar,.}{deze dan naar die kant door}{het tartend stoten}\\

\haiku{Het ergste is dat.}{de ouwe niet wil hebben}{dat ik ga dienen}\\

\haiku{Anders laten ze,!}{je zitten met je armoe}{die mooie kanaljes}\\

\haiku{{\textquoteright} Vrouw Van der Poorten.}{kwam neven de aanbouw en}{liep juist bij Jan uit}\\

\haiku{Terwijl zij met vlug.}{armbeweeg de stenen aan}{elkander rijden}\\

\haiku{Ze maakte het met,.}{Jan nooit af al moest ze met}{hem weglopen}\\

\haiku{'t Was niet mooi van,.}{Jan dat hij daarover zoveel}{te mopperen had}\\

\haiku{Gauw genoeg zou hij.}{een-te-veel zijn bij}{de Van der Poortens}\\

\haiku{Veranderen ging.}{niet meer of er zou grote}{herrie van komen}\\

\haiku{Ze lieten het geld, '.}{wel rollent kwam er hun}{op een paar niet aan}\\

\haiku{Op een plaats werd zelfs.}{op een avond de lamp van de}{zolder geslagen}\\

\haiku{Zijn kameraden ' ',.}{vant werk meendent niet}{kwaad dat wist hij wel}\\

\haiku{Liepen toen nog eens,.}{over het terrein naar alle}{kanten uitkijkend}\\

\haiku{Zij had de beste '.}{arbeiders dan ook maar voor}{t nemen gehad}\\

\haiku{De burgemeester.}{had op haar vragen niet veel}{willen antwoorden}\\

\haiku{Nou verdomde hij.}{het toch ook om nog langer}{te gaan bedelen}\\

\haiku{Dat Mien nou zelf wel.}{zag dat ze maar met de Rooie}{had moeten trouwen}\\

\haiku{'t Was een streek van.}{die kerel geweest om de}{boel te bedriegen}\\

\haiku{Een andere kant, ',.}{uitkijken doen of iem}{niet zag dat schoelje}\\

\haiku{Ik heb al een paar,{\textquoteright},.}{glazen bier gedronken loog}{ie groot-doende}\\

\haiku{Wie dat ni\'et kon, ',.}{wasn sul daar deden ze}{mee wat ze wouen}\\

\haiku{Maar de Rooie spotte:}{met gewichtig-doening}{in stem en gezicht}\\

\haiku{Jan had ook nooit met, '.}{Mien moeten trouwen dat was}{stom vanm geweest}\\

\haiku{Nou moesten ze maar eens.}{voor de dag komen met de}{lekkere brokjes}\\

\haiku{Vooruit, voor de dag,, '.}{er mee of hij zou eens zien}{watm te doen stond}\\

\haiku{En Van der Poorten,.}{ging stiekem aan de fles dat}{was toch nog erger}\\

\haiku{Waar dat naar toe moest,,.}{wisten ze niet maar dat kon}{z\'o toch niet blijven}\\

\haiku{Die meenden maar, dat.}{de mannen het met een bak}{koffie konden doen}\\

\haiku{Gelach en gepraat,.}{rumoerde druk rond \'al meer}{bezoekers lokkend}\\

\haiku{Dan waren ze als,,.}{duivels die anders toch heel}{goed waren nuchter}\\

\haiku{Eerst tussen Jan en ',.}{Mienr moeder en later}{met Mien zelf ook niet}\\

\haiku{Als men 't zo nam, ',.}{wast geen wonder dat Jan}{ook wel eens kwaad werd}\\

\haiku{Altijd en altijd,.}{datzelfde gezanik dat}{hij niets verdiende}\\

\haiku{En aansluitend bij:}{zijn eigen denken bitste zij}{hem schamperend toe}\\

\haiku{Zij zouden liever,.}{gehad hebben dat hij maar}{was weggebleven}\\

\haiku{Daar zo zat als een,.}{kwajongen die kijven kreeg}{over een domme streek}\\

\haiku{Alles moest weer goed,.}{worden geen oude koeien}{uit de sloot halen}\\

\haiku{'t Was dezelfde,.}{Jan niet meer als toen hij aan}{het klooster werkte}\\

\haiku{Een mens moest nog al.}{wat ondervinden om aan}{zijn eind te raken}\\

\haiku{In een hoek lagen.}{witte en zwarte klompen}{opeengestapeld}\\

\haiku{Langs de muur planken.}{met kruidenierswaren en}{katoenen stoffen}\\

\haiku{En al sloeg hij hem,.}{d'r neer dan kon niemand hem}{daar wat over maken}\\

\haiku{Van der Poorten zou,,.}{wel zorgen dat ie wat in}{huis had dreigde hij}\\

\haiku{Even ontmoetten de.}{blikken van de Rooie en die}{van Marie elkaar}\\

\haiku{Ge hebt vandaag uw,{\textquoteright} {\textquoteleft},,?}{portie wel gehad dunkt me.}{Kom kom wat is dat}\\

\haiku{De mensen zullen,,{\textquoteright}.}{wel denken dat hier ruzie}{is drift Drieka op}\\

\haiku{hij zijn arm los en.}{zonder verder op haar te}{letten kijft hij voort}\\

\haiku{U ophouden met,!}{dat wijf en mijn vrouw gemeen}{schoelje dat ge bent}\\

\haiku{Dan, verzoenend, stelt,.}{hij voor dat ze samen maar}{moeten afdrinken}\\

\haiku{Van der Poorten was ',.}{s middags uit geweest was}{zo zat als een snip}\\

\haiku{Zou nou wel liggen.}{te ronken als een os om}{zich uit te roesen}\\

\haiku{Een brede wonde,.}{gaapt in zijn achterhoofd als}{een zwarte holte}\\

\haiku{Een huivering van,.}{angst doorkilt haar lijf telkens}{als zij wat horen}\\

\haiku{Zo'n ouwe mens nog.}{geen tijd te geven om een}{jas aan te trekken}\\

\haiku{Het was dan toch ook,.}{schande voor de ouders als}{de dochter zo deed}\\

\haiku{{\textquoteright} 't Rood van lichte.}{opwinding schemerde door}{haar gelaatshuid heen}\\

\haiku{Ja, het eerste jaar,!}{maar over een tijdje zou het}{wel anders worden}\\

\haiku{En die was tegen,.}{haar uitgevallen dat ze}{haar mond moest houden}\\

\haiku{Als wij zoveel geld,...{\textquoteright} {\textquoteleft},?}{hadden danJa hoe zijn ze}{d'r aangekomen}\\

\haiku{Dit zou ze haar voor,,,,.}{de voeten gooien neen nog}{beter harder z\'o}\\

\haiku{Dat was zo maar een,,.}{kaart geweest beweerde hij}{om te versturen}\\

\haiku{Ze zouden wel gauw.}{tegenspoed krijgen op stal}{of op de akker}\\

\haiku{Met een angstkreet vloog,.}{ze op en krijste hem toe}{haar los te laten}\\

\haiku{En met angst-kracht,.}{trok ze hem van Nel weg weer}{uit de slaapkamer}\\

\haiku{Ze wist, dat moeder,.}{dan bad dat deed ze altijd}{in zo'n gevallen}\\

\haiku{Iedere dag zou, '.}{ze voortaan naar hem toegaan}{dat zet zagen}\\

\haiku{Nel hoefde d'r nooit,.}{op te rekenen dat hij}{zijn toestemming gaf}\\

\haiku{De trekken van zijn,.}{gezicht verstrakten omdat}{Nel nog tegenhield}\\

\haiku{In vol ontwaken.}{van hun verlangen hielden}{zij elkander vast}\\

\haiku{Hun blikken felden,.}{elkaar tegen in het wild}{bewogen gelaat}\\

\haiku{De dokter trachtte,.}{haar te bedaren dat het}{toch z\'o erg niet was}\\

\haiku{Ze zou nu wel gaan,.}{trouwen en dan was alle}{leed weer vergeten}\\

\haiku{Ja, maar vader en.}{moeder waren zo kwaad op}{de Jorissen}\\

\haiku{Maar zij begreep het,.}{wel dat kwam allemaal van}{de boterfabriek}\\

\haiku{'t Zou gemeen van,.}{hem zijn als hij het niet deed}{en onnozel ook}\\

\haiku{Driek zou ook wel eens,.}{zelf met haar vader spreken}{of Jorissen}\\

\haiku{Eigenlijk zou men.}{met h\'em geen medelijden}{hoeven te hebben}\\

\haiku{'t Was nu eenmaal,,.}{zo en waar d'r twee kijven}{hebben d'r twee schuld}\\

\haiku{Ze wou nog liever,.}{altijd hard werken als het}{maar geen zondag werd}\\

\haiku{Klokslag twaalf ging hij,.}{aan tafel zitten of het}{eten klaar was of niet}\\

\haiku{'n Schande voor zo'n,.}{ouwe kerel zo in de}{herbergen te doen}\\

\haiku{Nadat de ergste,.}{woede gelucht was bleef hij}{nog nazaniken}\\

\haiku{{\textquoteright} Terwijl hij wegging,.}{snauwden ze elkander de}{scheldwoorden nog toe}\\

\haiku{In alles deed hij,.}{zijn wil zonder haar zelfs ooit}{iets te vertellen}\\

\haiku{Driek had thuis nog niet, {\textquoteleft}{\textquoteright}.}{dadelijk gezegd dat hij}{met Nelmoest trouwen}\\

\haiku{Hij meende zeker,,.}{dat hij de vogel af had}{als hij getrouwd was}\\

\haiku{De teleurstelling.}{zweepte zijn woede nog aan}{tot h\'arder zwoegen}\\

\haiku{Maar zijn vader, die,;}{kinkelig zwetste dat hij}{nooit bang was geweest}\\

\haiku{Zijn kameraden.}{durfden eerst niet ronduit zijn}{partij te kiezen}\\

\haiku{{\textquoteright} {\textquoteleft}Ja, nou moet ie het,.}{zelf maar weten nou Nel er}{mee zit te kijken}\\

\haiku{Goemans raasde nog,:}{na zijn vuist telkens op de}{tafel bonkerend}\\

\haiku{Maar mooie Nel kon hem,,.}{behagen Hij dacht ik zal}{het maar eens wagen}\\

\haiku{Die lag als een brand.}{op zijn ziel en gloeide de}{haat voortdurend aan}\\

\haiku{Eindelijk, opeens,,.}{schoten de schreeuwstemmen uit}{allemaal gelijk}\\

\haiku{Zij zou er geld voor,.}{kwijt willen zijn als Driek die}{Nel nooit had gezien}\\

\haiku{Een ogenblik later.}{kwam Hannes met een grote}{troep binnenkabalen}\\

\haiku{Als ze zeiden, dat,.}{ze wat gezien hadden dan}{zouden ze liegen}\\

\haiku{Dan was er nog iets,,:}{een misbruik in sommige}{dorpen ingeroest}\\

\haiku{En het eind was, dat.}{de meid over zes weken zou}{moeten vertrekken}\\

\haiku{Er moest gegeten.}{worden en brood gaven de}{meesters in school niet}\\

\haiku{Niemand nam immers?}{werkvolk in dienst om het te}{laten leeglopen}\\

\haiku{Toen moest de mest over.}{het land gebracht worden voor}{het winterkoren}\\

\haiku{Ondanks het gezwoeg,.}{voelde zij dat een kou haar}{lijf overhuiverde}\\

\haiku{Dat de boerin maar,.}{eens wat mee aanpakte zo}{druk was het niet meer}\\

\haiku{Hij was echter groot,.}{en sterk en daarom was hij}{heel vroeg gaan dienen}\\

\haiku{En de zorg begon,.}{pijnlijk te knagen aan hun}{leven voor het eerst}\\

\haiku{Was de winter toch,.}{maar om de zomertijd zou}{werk genoeg brengen}\\

\haiku{{\textquoteright} Maar kort daarna wist, {\textquoteleft}'{\textquoteright},.}{ze datt al zo was toen}{ze de kerkgang deed}\\

\haiku{Ze was toen al weer {\textquoteleft}'{\textquoteright},.}{n heel eind maar ze spande}{zich in wat ze kon}\\

\haiku{Gedurende de.}{zomer hadden ze het al}{hard genoeg gehad}\\

\haiku{Dan zou het ergste,.}{gelejen zijn beurden ze}{zich zelf hopend op}\\

\haiku{{\textquoteleft}Ja, Peeters, mijn goeie,,.}{man ziet ge dat zult ge nog}{wel niet begrijpen}\\

\haiku{Zij trok zich dat aan, '.}{en driftte hem tegen dat}{zet ook niet op\'at}\\

\haiku{Maar de twee helpers,,!}{die Toon gevraagd had moesten toch}{ook betaald worden}\\

\haiku{Hij zat in een klein,.}{huisje met zijn vrouw die men}{zelden te zien kreeg}\\

\haiku{Er werd gezopen,.}{zoveel als ze maar door de}{keel konden krijgen}\\

\haiku{Ze hielden hem vast!}{en twee trokken hem de broek}{van de benen af}\\

\haiku{Maar later werd het.}{verbranden op de meeste}{plaatsen verboden}\\

\haiku{Die uitdrukking had.}{de spreker onthouden van}{de burgemeester}\\

\haiku{Bovendien bracht het.}{hem enkele honderden}{guldens per jaar op}\\

\haiku{Daarna sneed zijn stem,:}{knerpend als het scheuren van}{een ijzeren plaat}\\

\haiku{Zag hij dan ook, of?}{de arbeiders werkten zo}{hard als ze konden}\\

\haiku{hij liet het nog eer,.}{in brand steken dan het voor}{zo'n geld te geven}\\

\haiku{{\textquoteright} {\textquoteleft}Ja, het w\'as wel wat,.}{veel maar dat wilde hij er}{toch voor betalen}\\

\haiku{{\textquoteleft}wij hebben maar een,.}{prul van behuizing maar er}{zitten guldens in}\\

\haiku{12 De armoede.}{joeg hen op van de ene plaats}{naar de andere}\\

\haiku{Haar broers verlolden.}{op \'e\'en avond meer dan hij in}{een week verdiende}\\

\haiku{Plotseling rukte,.}{het dier aan schichtig door het}{slaan en het geraas}\\

\haiku{Gewoonlijk werden.}{in plaats hiervan andere}{regels gezongen}\\

\haiku{Wat had hij toen veel!}{gezien zonder er iets van}{geleerd te hebben}\\

\haiku{Sinds hij echter de,.}{strijd aanvaard had kon hij die}{niet meer ontvluchten}\\

\haiku{{\textquoteright} {\textquoteleft}Wat ik meen, dat is,.}{geld dat de winkeliers hier}{konden verdienen}\\

\haiku{Als daar een kanaal.}{naar toe werd gegraven of}{een tram aangelegd}\\

\haiku{De meesten hadden,,.}{klompen bruin van aangeplakt}{moer aan de voeten}\\

\haiku{Een moedeloze.}{loomheid somberde om die}{groep arbeiders heen}\\

\haiku{Een huivering van.}{aanstuwende somberte}{trok over zijn lichaam}\\

\haiku{Als ge er een jaar,.}{of wat in gestaan hebt dan}{voelt ge het niet meer}\\

\haiku{Bij harde winter.}{kan het wel eens gebeuren}{dat een ketting knapt}\\

\haiku{{\textquoteright} {\textquoteleft}Ja, ik heb er wel,.}{eens een gehad maar hij is}{kapot gevallen}\\

\haiku{{\textquoteright} {\textquoteleft}Ja, heel eenvoudig,{\textquoteright},.}{stemde De Visscher toe om}{toch iets te zeggen}\\

\haiku{Wel wil ik gaarne,.}{toegeven dat we niet te}{veel keuze hebben}\\

\haiku{Daarom vind ik het,.}{juist zo vreemd dat u niet van}{De Schoolmeester houdt}\\

\haiku{Juffrouw Verhoeven.}{nodigde hem uit toch even}{binnen te komen}\\

\haiku{{\textquoteright} {\textquoteleft}Zeker, maar ik heb '{\textquoteright} {\textquoteleft},,.}{t ook niet tegen u.O}{nee nee dat is waar}\\

\haiku{Het geld regeert de,.}{wereld het geld is de ziel}{van de negotie}\\

\haiku{Neen, door de dag kwam. ' '.}{er weinigs Zaterdags}{ens zondags meer}\\

\haiku{Het mes sneed er aan,,.}{twee kanten helderde hij}{nog op grinnekend}\\

\haiku{Hij had de naam van.}{in allerlei gemene}{straatjes te sjouwen}\\

\haiku{Hoe konden ze nu?}{toch eigenlijk in zo iets}{nog plezier hebben}\\

\haiku{{\textquoteright} Ja, bitterde het, '!}{in zijn somber denken op}{zoals ment ziet}\\

\haiku{Hard werken was 't,.}{er ook maar d\'at was toch niks}{tegen turfgraven}\\

\haiku{Daar kan je jezelf,.}{gaan liggen kietelen als}{je wat hebben wilt}\\

\haiku{Ze zetten hem op,.}{de tafel en dan sloeg ie}{een onzin uit bar}\\

\haiku{Maar als ze in de,.}{hel liggen dan zal het er}{wel anders spannen}\\

\haiku{Nu en dan sloop een,.}{paar weg de lijven tegen}{elkaar gedrongen}\\

\haiku{Een ander schopte,.}{de benen terug die hem}{in de weg kwamen}\\

\haiku{Aan het station.}{Lizaveen stapte niemand}{in dan De Visscher}\\

\haiku{ik had niet gedacht,.}{dat ik mij in die mens zou}{kunnen vergissen}\\

\haiku{En wat ze na haar,.}{dood kon nalaten dat was}{voor de familie}\\

\haiku{Opeens wendde zij,.}{zich naar hem toe of ie niet}{naar de hoogmis ging}\\

\haiku{Deed bestraffend, dat '.}{de kapelaant toch ook}{al goed zou menen}\\

\haiku{Waar haalden zij de?}{kunst vandaan om hun leven}{zo in te richten}\\

\haiku{Die waarderende.}{brief bracht De Visscher in een}{roes van vreugde}\\

\haiku{'t Schijnt, dat ik in.}{de wieg gelegd ben om de}{strijd uit te vechten}\\

\haiku{Voor het onderhoud.}{van het gezin schiet er dan}{zoveel niet meer over}\\

\haiku{Eigenlijk wel geen,,.}{familie zo gesproken}{maar toch meer als vreemd}\\

\haiku{De gemeente zou,.}{er wel bij varen als zij}{haar veen verkocht had}\\

\haiku{{\textquoteleft}Wie de mensen wil,.}{leren kan hen nooit te dom}{veronderstellen}\\

\haiku{Het bevatte een:}{lange correspondentie}{als hoofdartikel}\\

\haiku{Ik dacht, breng het prul,.}{aan meneer pastoor dat die}{het verbranden kan}\\

\haiku{Op een schrijven aan.}{Het Nieuws van Peelland kreeg De}{Visscher geen antwoord}\\

\haiku{Een hees gebrul als.}{van een gepijnigd dier ging}{uit de hoop omhoog}\\

\haiku{Als je de beest wilt,.}{uithangen zie dan maar dat}{je de deur uitkomt}\\

\haiku{Om het verband goed,.}{te kunnen leggen spelde}{hij het hemd omhoog}\\

\haiku{Na een paar dagen.}{verzonden zij de kopij}{naar de drukkerij}\\

\haiku{dat is, dat wij veel.}{te weinig verdienen voor}{zo'n harde arbeid}\\

\haiku{Met gezondhijd moet}{ik U nu schrijfen dat het}{zoo lang geduurt heeft}\\

\haiku{Nu vele Groeten.}{van mijn vrouw of Fientje wie}{ik ze noemen wil}\\

\haiku{Hen uit de verdienst,,.}{schoppen alsof zij honden}{waren dat kon hij}\\

\haiku{En met de bezem,.}{de straat laten vegen waar}{hij had gelopen}\\

\haiku{Maar met de verdienst.}{was het bij de maatschappij}{toch een boel beter}\\

\haiku{Men heeft er toch al.}{genoeg aan onze lieve}{Heer moeten geven}\\

\haiku{Zij zelf hadden voor.}{hun trouwen in Pruisen een}{ledikant gekocht}\\

\haiku{Nee, er zit zo'n schelm.}{in het hele land niet in}{de gevangenis}\\

\haiku{Maar de ander was,.}{toch wel bang dat Van Eijzen}{het niet halen zou}\\

\haiku{Daarna werd nog een:}{circulaire verspreid door}{de gemeente}\\

\haiku{Allee, die lui moesten,.}{maar drinken al kostte het}{hem duizend gulden}\\

\haiku{De mensen waren,,.}{blij als hij niet snauwde dat}{het niet nodig was}\\

\haiku{Een kapelaan uit.}{het dorp ging met het hoofd van}{de openbare school}\\

\haiku{Er kwamen handen.}{tekort om jenever en}{bier aan te voeren}\\

\haiku{Als het ingevuld,.}{was met de naam Van Eijzen}{dan deugde het niet}\\

\haiku{'s Morgens hing een.}{mand zonder bodem bij van}{Eijzen aan de deur}\\

\haiku{De gemeente moest.}{ook voor de godsdienstige}{belangen zorgen}\\

\haiku{En het hoofd van de.}{dorpsschool had verhoging van}{salaris gevraagd}\\

\haiku{Hij woelde door zijn,.}{bed al maar pratend over turf}{en aken en staken}\\

\haiku{Fien en Dien snikten.}{hardop en vluchtten uit het}{kleine slaaphokje}\\

\haiku{ik woon nu ver van}{de post af zuster in de}{week heb ik geen tijd}\\

\haiku{Bij het onderzoek.}{van hogerhand was er zelfs}{nog te v\'e\'el in kas}\\

\subsection{Uit: Verstooteling}

\haiku{Ja, als haar moeder,;}{weer beter was zou zij een}{nieuwen dienst zoeken}\\

\haiku{Nee, in het dorp had,.}{ze niks in de stad was het}{veel pleizieriger}\\

\haiku{Ze moest maar naar huis,,.}{gaan als ze niet wilde doen}{wat zij aanraadden}\\

\haiku{Door de praatjes van. '}{het volk zouden die daartoe}{genoodzaakt worden}\\

\haiku{En het is God zelf,,.}{die het gezegd heeft dat er}{armen moeten zijn}\\

\haiku{En of hij haar geen.}{proces kon maken wegens}{grove beleediging}\\

\haiku{Chiek in huis, en 's...}{middags zich aankleeden en}{gaan wandelen}\\

\haiku{ie klimt zoo\"e mer in....}{de b\"oem en scheurt de t\"ak en}{lacht ow nog uut ok}\\

\haiku{In bijzijn van haar.}{man durfde Floortje hem niet}{meer voor te spreken}\\

\haiku{In Holland was 't,?}{soldaat-zijn niks wat}{had je daar nou aan}\\

\haiku{Eer de dag om was,.}{was het gepraat het heele}{dorp doorgetrokken}\\

\haiku{De zieke ging hard '.}{achteruit en opeens was}{t afgeloopen}\\

\haiku{'s Nachts kon ze  .}{geen oog toe doen van al de}{zorg en het verdriet}\\

\haiku{Als hij maar eens kwam,.}{en wat geld meebracht dan zou}{het wel weer goed zijn}\\

\haiku{wer, we l\`even mer...{\textquoteright},,,...}{eens  Sauf broeder sauf den}{daalder d\`e mos auf}\\

\haiku{{\textquoteright} Tot Jan zijn woede.}{uitheeschte en zich op}{hem wilde werpen}\\

\haiku{Jan moest maar zuipen,, '...}{zooveel als hij lusttet}{kwam er nikt op aan}\\

\haiku{En dikwijls al had,.}{het heele dorp verteld dat}{hij weer ging trouwen}\\

\haiku{Dat was de smart, die.}{in zulke oogenblikken}{in hem opleefde}\\

\haiku{Dat zou wat anders.}{zijn dan zich bij de boeren}{kapot te werken}\\

\haiku{De vrienden zouden.}{hen uitgeleide doen naar}{het station}\\

\haiku{Wat van zijn verdienst,?}{afgeven waarvoor hij zoo}{hard moest errebeie}\\

\haiku{Alleen een klaagbrief.}{van zijn moeder kwam Jan nu}{en dan hinderen}\\

\haiku{en gii most eier,,, ()}{st\`ele h\`e vur eur en dan}{opnieuw knipoogend}\\

\haiku{Wankelde nu naar,.}{dezen dan naar dien kant door}{het tartend stooten}\\

\haiku{Sakkerloot nee, d'r, '...}{maar niet aan denket was}{niet om uit te staan}\\

\haiku{Ze maakte het met,...}{Jan nooit af al moest ze met}{hem wegloopen}\\

\haiku{Zoolang als ze met,.}{hun beien waren hadden}{ze niet veel plaats noodig}\\

\haiku{'t Was niet mooi van,.}{Jan dat hij daarover zooveel}{te mopperen had}\\

\haiku{Gauw genoeg zou hij.}{een-te-veel zijn bij}{de Van der Poortens}\\

\haiku{Veranderen ging.}{niet meer of er zou groote}{herrie van komen}\\

\haiku{Op een plaats werd zelfs.}{op een avond de lamp van den}{zolder geslagen}\\

\haiku{Liepen toen nog eens,.}{over het terrein naar alle}{kante uitkijkend}\\

\haiku{Die er boven op,}{stond haalde het aan en de}{overigen trokken}\\

\haiku{Zij had de beste '.}{arbeiders dan ook maar voor}{t nemen gehad}\\

\haiku{De burgemeester.}{had op haar vragen niet veel}{willen antwoorden}\\

\haiku{Nou verdomde'n ie '.}{t toch ook om nog langer}{te gaan bedelen}\\

\haiku{Dat Mien nou zelf wel, {\textquoteleft}{\textquoteright};}{zag dat ze maar met denRooie}{had moeten trouwen}\\

\haiku{'t Was een streek van.}{dien kerel geweest om den}{boel te bedriegen}\\

\haiku{kon hij maar ergens,...}{wat leenen maar daar hoefde hij}{niet om te komen}\\

\haiku{{\textquoteright} Een schampering van.}{d'r-alles-van-weten}{lag in z'n woorden}\\

\haiku{barst maar... z\'o\'o leep was,,?...}{dat mirakel toch niet dat}{ie d\`at snapte h\`e}\\

\haiku{toen was die kerel...}{begonnen met een groote mond}{van verdammter lump}\\

\haiku{Maar de {\textquoteleft}Rooie{\textquoteright} spotte:}{met gewichtig-doening}{in stem en gezicht}\\

\haiku{Mien en de {\textquoteleft}Rooie{\textquoteright}... als {\textquoteleft}{\textquoteright}, '...}{ze met denRooie getrouwd was}{zout beter zijn}\\

\haiku{nou moesten ze maar eens...}{voor den dag komen met de}{lekkere br\"okskes}\\

\haiku{Vooruit, voor den dag,, '...}{er mee of hij zou eens zien}{watm te doen stond}\\

\haiku{hoe Marie was, dat... '...}{wist iedereen en Mien zou}{welt zelfde zijn}\\

\haiku{Gelach en gepraat,.}{rumoerde druk rond \`al meer}{bezoekers lokkend}\\

\haiku{Men wist niet meer, wat..., ',...}{men gelooven moest jat was}{dan toch te gek h\`e}\\

\haiku{een maand of drie... ja,, '}{d\`a\`ar zorgde hij wel voor maar}{met de verdienst ging}\\

\haiku{Wat een gezwets zou ',.}{t geven in het dorp over}{hem en zijne vrouw}\\

\haiku{Altijd en altijd,.}{datzelfde gezanik dat}{hij niets verdiende}\\

\haiku{Onbewust verviel.}{hij een oogenblik in een}{nadenkend zwijgen}\\

\haiku{En aansluitend bij:}{zijn eigen denken bitste zij}{hem schamperend toe}\\

\haiku{Daar zoo zat als een,...}{kwajongen die kijven kreeg}{over een dommen streek}\\

\haiku{{\textquoteleft}Wat denk'te wel,...}{ge het ok noo\"et wat te}{moppere geh\`ad}\\

\haiku{Jan moest maar dikwijls,...}{naar huis komen moeder was}{er toch nog altijd}\\

\haiku{'t Was dezelfde,.}{Jan niet meer als toen hij aan}{het klooster werkte}\\

\haiku{Een mensch moest nog al...}{wat ondervinden om aan}{zijn eind te raken}\\

\haiku{In een hoek {\^\i}agen.}{witte en zwarte klompen}{opeengestapeld}\\

\haiku{Langs den muur planken.}{met kruidenierswaren en}{katoenen stoffen}\\

\haiku{En al sloeg hij hem,.}{d'r neer dan kon niemand hem}{daar wat over maken}\\

\haiku{Van der Poorten zou,,.}{wel zorgen dat ie wat in}{huis had dreigde hij}\\

\haiku{Even ontmoetten de {\textquoteleft}{\textquoteright}.}{blikken van denRooie en die}{van Marie elkaar}\\

\haiku{{\textquoteleft}Der, mer da's ok het,,...}{leste dan tap ik nie mer}{drink nou mer gaauw uut}\\

\haiku{hij zijn arm los en.}{zonder verder op haar te}{letten kijft hij voort}\\

\haiku{Dan, verzoenend, stelt,.}{hij voor dat ze samen maar}{moeten afdrinken}\\

\haiku{Een geweldige.}{trap tegen de deur doet hem}{half-opspringen}\\

\haiku{Een breede wonde,...}{gaapt in zijn achterhoofd als}{een zwarte holte}\\

\haiku{Een huivering van,.}{angst doorkilt haar lijf telkens}{als zij wat hooren}\\

\haiku{De Zondagsche kleeren.}{rijen op de stoelen in}{alle vertrekken}\\

\section{Marcel Maassen}

\subsection{Uit: Blauwe damp}

\haiku{Want als Jos er niet,.}{was geweest dan was het nooit}{meer iets geworden}\\

\haiku{Ik bleef lekker thuis,.}{een rustig en tevreden}{hamstertje wezen}\\

\haiku{Ik zat achter de.}{computer en vroeg me af}{wat ik moest schrijven}\\

\haiku{Bekijk 't maar, ga,...}{maar lekker schrijven ga maar}{gauw beroemd worden}\\

\haiku{Dat lachje van haar,,:}{die valsverliefde ogen dat}{handje door mijn haar}\\

\haiku{Ja, schatje gaf wel,,.}{kusje ik wel maar schatje}{wist heus wel beter}\\

\haiku{was vroeg je, of ik,.}{lekker met hem geneukt heb}{dat bedoel je toch}\\

\haiku{Nee, nooit heb ik haar.}{kunnen dwingen te buigen}{en te bekennen}\\

\haiku{Tenminste, niet meer.}{vanaf het moment dat ik}{erop ging letten}\\

\haiku{Vriendinnen bleven, '.}{ze al zagen ze elkaar}{opt laatst niet meer}\\

\haiku{{\textquoteright} Moeder kreeg eerst twee.}{andere kinderen en}{daarna kreeg ze mij}\\

\haiku{Met die vier jaren,.}{die daartussen liggen weet}{ik me niet goed raad}\\

\haiku{En vooral: je moest.}{beter zijn dan iedereen}{die je omringde}\\

\haiku{Natuurlijk haatte.}{ik leren maar ik was er}{redelijk goed in}\\

\haiku{Allemaal dingen.}{die je allang wist en die}{nooit veranderden}\\

\haiku{Eerst alleen naar de,.}{thuiswedstrijden later ook}{naar uitwedstrijden}\\

\haiku{{\textquoteleft}Nog even{\textquoteright}, dacht hij, {\textquoteleft}nog,.}{heel even meelachen ze}{houden vanzelf op}\\

\haiku{Twee pilsjes in de.}{handen en nog twee op de}{bar en dan zuipen}\\

\haiku{Trouwens, jijzelf ook,.}{niet je zou niet weten wat}{je moest beginnen}\\

\haiku{Ze bleef gewoon waar,,.}{ze was fijn bij Walter en}{ik kon doodvallen}\\

\haiku{Bijna zeventig,.}{pagina's die niet je niet}{zomaar aan elkaar}\\

\haiku{Dat vond vader een, ({\textquoteleft}}{prima plan precies wat hij}{zelf al had bedacht}\\

\haiku{Zelf had ik ook niets,.}{tegen damesbezoek maar}{het kwam er niet van}\\

\haiku{En als ik er mijn,.}{toekomst voor moet opgeven}{dan moet dat ook maar}\\

\haiku{Aan schrijven had ik.}{niet gedacht toen ik zei dat}{ik zou gaan schrijven}\\

\haiku{Maar ze hoort me weer.}{eens niet want de tv begint}{alsnog te spelen}\\

\haiku{Zoef, we kunnen niet.}{slapen als jij de hele}{tijd staat te lullen}\\

\haiku{Hij knijpt z'n ogen toe,.}{en z'n mond elke spier in}{zijn lijf trekt samen}\\

\haiku{Nog elke dag zijn,.}{zij samen een keer of drie}{en soms wel vaker}\\

\haiku{{\textquoteright} Dan begint Jos te.}{bulderen zoals alleen}{Jos bulderen kan}\\

\haiku{Twee woorden maar, want {\textquoteleft}{\textquoteright}.}{datmevrouw had voor moeder}{niet eens gehoeven}\\

\haiku{Natascha was dik.}{en lelijk en altijd veel}{te zwaar opgemaakt}\\

\haiku{We waren veertien.}{en er was weer eens een kamp}{van Jong Nederland}\\

\haiku{Vervolgens gaf hij.}{mij een knipoog en ik gaf}{er eentje terug}\\

\haiku{Voor haar huis zou ik,}{zeggen dat ik haar leuk vond}{zou ik haar kussen}\\

\haiku{In z'n blote kont.}{zat hij langs de kant van het}{water en keek toe}\\

\haiku{{\textquoteleft}De ogen van Jeffrey.}{zijn niet meer dan twee zwarte}{gaten in zijn hoofd}\\

\haiku{Toen stonden we naakt,.}{tegenover elkaar nog een}{moment onwennig}\\

\haiku{{\textquoteleft}Stel dat je het zou,?}{mogen overdoen hoe zou je}{het dan aanpakken}\\

\haiku{Op dit moment ben.}{ik niet verliefd en ik wil}{me ook niet binden}\\

\haiku{Ze draagt een groene.}{onderbroek en een witte}{bh en verder niets}\\

\haiku{Zij werd kwaad en ik.}{werd kwader en we sliepen}{in zonder nachtzoen}\\

\haiku{We hebben al vier.}{maanden verkering en je}{bent nog niet zwanger}\\

\haiku{{\textquoteright} Ieder had gewoon.}{z'n eigen schoenendoos vol}{pijnen en pijntjes}\\

\haiku{{\textquoteright} {\textquoteleft}Wat is dat nou voor{\textquoteright},, {\textquoteleft}?}{onzin zegt moederben jij}{een volwassen vent}\\

\haiku{Hij glimlachte, hij,.}{grijnsde keek naar de plaat die}{voor zijn voeten lag}\\

\haiku{En het was niet eens}{zozeer om het bedrog dat}{ik hem inniger}\\

\haiku{Op Tweede Kerstdag, '.}{bijvoorbeelds avonds met z'n}{allen in de kroeg}\\

\haiku{Geleen en Sittard,.}{is water en vuur vooral}{met carnaval}\\

\haiku{We gingen er wel,,.}{naartoe gingen wel kijken}{maar we keken niet}\\

\haiku{Ik word razend en:}{zij lacht alsmaar liever en}{spreekt alsmaar zoeter}\\

\haiku{Walter en zij en,.}{nog iemand met z'n drie\"en}{in \'e\'en botsauto}\\

\haiku{Je tas stond open en.}{dus dacht ik dat ik er wel}{in mocht snuffelen}\\

\haiku{Zij gaat weg en ik,.}{blijf hier zij is met Walter}{en ik ben alleen}\\

\haiku{Ze slaat een hand voor,.}{haar mond buigt zich over mij heen}{en tuit haar lippen}\\

\haiku{Zij over kunst, Bertram.}{over filosofie en ik}{over literatuur}\\

\haiku{Het beddegoed is.}{afgetrokken en schoon goed}{ligt klaar op de stoel}\\

\haiku{Toen waren we weer,,.}{alleen gezellig met z'n}{tweetjes muis en ik}\\

\haiku{Zo weet ik dus niet,.}{of ze wakker is of slaapt}{of ze denkt of droomt}\\

\haiku{Ik kruip weer tegen,:}{haar aan sla mijn arm om haar}{middel en fluister}\\

\section{Herman de Man}

\subsection{Uit: Aardebanden}

\haiku{Aan Walter Thiry mijn.}{tweejarigen vriend ~ I.}{Moeder's uitvaart}\\

\haiku{ik wensch, dat het voor{\textquoteright}.}{lange jaren uw laatste}{leed geweest zal zijn}\\

\haiku{Zijn kinderachtig.}{benauwde laatste woorden}{bleef ze onthouden}\\

\haiku{Willem stond achter,.}{de tapkast zij kon dus goed}{even gemist worden}\\

\haiku{Zijn sterkte, zijn niet.}{te breidelen mannekracht}{deed haar duizelen}\\

\haiku{Met twee, drie stappen.}{als van een grooten beer was}{hij de kamer door}\\

\haiku{ze wist dat het t\`och.}{komen moest en dat zij het}{te aanvaarden had}\\

\haiku{die kleurde altijd.}{subiet als een meneer uit}{de stad hem aansprak}\\

\haiku{{\textquoteleft}wilt U mij volgen,.}{de meid zal uw koffers straks}{wel boven brengen}\\

\haiku{{\textquoteleft}En Pa, is het nu,?}{heusch waar dat hij op het}{water loopen kon}\\

\haiku{Aan weerskanten van,;}{de kaai waren slooten de}{een daarvan was breed}\\

\haiku{Dat deed de stijve.}{Juffrouw Th\'erees goed tot in}{haar diepste wezen}\\

\haiku{Dat deed haar angstig - '.}{omzien en vreezen vreezen}{t allerergste}\\

\haiku{Die man was Liesje's {\textquoteleft}{\textquoteright}.}{Vader en zij was Th\'erees}{Versteeg uitDe Zalm}\\

\haiku{daarmee depte ze.}{haar heete voorhoofd en dat}{gaf koelte en rust}\\

\haiku{als ik goed zie met,.}{dien vrijer die er met zijn}{vrouw op logies is}\\

\haiku{Toen nam ze moedig.}{twee wijnglazen en keerde}{ze om bij de voetjes}\\

\haiku{{\textquoteleft}hij zal pijn in zijn.}{buik hebben of anders in}{zijn potteman\'e}\\

\haiku{{\textquoteright} De vreemdeling sloeg.}{zijn oogen op en keek lang en}{vorschend naar Jacques}\\

\haiku{Neen jongen, ga jij {\textquoteleft}{\textquoteright}.}{maar weervan alles doen en}{laat mij met vrede}\\

\haiku{Juffrouw Th\'erees zat;}{stijf als een ijspegel op}{het keukenstoeltje}\\

\haiku{Over zijn gezicht kwam,.}{een vredige trek tijdens}{hij den brief doorlas}\\

\haiku{- {\textquoteleft}En nu, burgers, met,!}{z'n allen driemaal hoera}{voor de Koningin}\\

\haiku{Toen stonden ze op,.}{en liepen terug in de}{richting van de Lek}\\

\haiku{Ze liet hem maar weer,.}{doen en meteen was hij van}{dat gesprek weer af}\\

\haiku{Toen ze heur groene,,:}{stijve bloeze dichtgeknoopt}{had riep hij luchtig}\\

\subsection{Uit: Omnibus}

\haiku{Jochem plant dan zijn:}{vleeshompen vaster op de}{aard en zal zeggen}\\

\haiku{Hij kocht, en 't wijf,.}{mag dat niet weten \'o\'ok een}{loterijbriefje}\\

\haiku{{\textquoteright} {\textquoteleft}Kerel, je maakt me, {\textquotedblleft},.}{nog eens gek jij met jeja}{meneer nee meneer}\\

\haiku{Met vreemde ogen keek,, '.}{die even maar int wezen}{van deze schooier}\\

\haiku{Toen ze het franshuis, '.}{naderden zagen zem}{al in de tuin staan}\\

\haiku{De organist staart.}{verbaasd over z'n boek naar z'n}{twee onte gasten}\\

\haiku{Luuk, dees twee mannen,,.}{zeggen dat het kul is die}{bedelaarstekens}\\

\haiku{Maar wel hebben ze '.}{t geluid al half om de}{aarde heen gestuwd}\\

\haiku{Hij met z'n lompe,.}{korsterige handen en}{z'n onwetendheid}\\

\haiku{Hij is misschien iets,.}{als een stuk oom van een neef}{van een neef z'n broer}\\

\haiku{Wist hij maar een woord,.}{om d'r gedachte op wat}{anders te richten}\\

\haiku{{\textquoteleft}Je kan een boer nooit,.}{te veul centjes afhandig}{maken zeg ik maar}\\

\haiku{Overal was geld, veel,,....}{geld veel te veel geld overal}{behalve bij hem}\\

\haiku{Z\'o maar te grijp, als.}{eerst maar de drager zich niet}{meer verweren kon}\\

\haiku{Ze wisten goed, dat.}{het nu een kwaad uur werd voor}{de bedelaren}\\

\haiku{Maar mannen, ik zeg,.}{maar zo kom d'r in en la\^an}{we er een vatten}\\

\haiku{Toen grepen we beur.}{en metselden de ouwe}{zog helegaar in}\\

\haiku{Hij stond op en sloeg.}{met z'n pezige vuist op}{de keukentafel}\\

\haiku{arme lat, blijf jij,.}{maar hier wat wonen op de}{werf tot het je past}\\

\haiku{Als je maar geld hebt, '.}{dan mag jen keer de brand}{steken in je keel}\\

\haiku{{\textquoteright} {\textquoteleft}Maar stellig, en wel.}{volgens art. 432 ten eerste}{Wetboek van Strafrecht}\\

\haiku{En Jochem loopt met,.}{smoesjes te leuren die t\'och}{niet zallen helpen}\\

\haiku{Wat dee die ander?}{ook krek op dat moment langs}{te kommen lopen}\\

\haiku{niet zeggen... 't is ', '.}{de wet dust is de wet}{wantt is de wet}\\

\haiku{ja, dan, dan... wat mot,?}{uedele dan doen as u}{dat met ons eens zift}\\

\haiku{Ze hebben over de,.}{witkop niet meer gepraat die}{dag al viel dat zwaar}\\

\haiku{weerom zal zijn van.}{z'n vechtpartij tegen de}{lucht en de wolken}\\

\haiku{Je bekomt hier, deur,.}{ons toedoen wat vrijplezier}{en een centje toe}\\

\haiku{{\textquoteright} En de jongens, wild,.}{op een kluit gooien elk een}{cent in de lijmpot}\\

\haiku{Negen gezonde,,.}{blije kinderen woesten en}{stillen alderhand}\\

\haiku{Z'n twee getrouwe,.}{bedelgasten die nooit of}{nooit zullen overslaan}\\

\haiku{En waar, in Lopik,?}{bestaan nog overig zulke}{mooie lichtende ogen}\\

\haiku{Dat zou echtig ook.}{wel de dochter kunnen zijn}{van mijnheer Koekkoek}\\

\haiku{Bastiaar Six zat,,.}{als een woestgemaakt beest naast}{Aartje en Aartje beefde}\\

\haiku{{\textquoteright} riep de voerman woest, {\textquoteleft} ',!}{hum smijten wet lijk veur}{z'n voeten verstaan}\\

\haiku{{\textquoteright} En 't was toen maar,.}{goed dat mijnheer Koekkoek den}{huis uit kwam lopen}\\

\haiku{ze gaven om 't,.}{rouwbeklag allemaal wat}{centen uit de knip}\\

\haiku{Allenig Jochem,.}{hield van z'n portie tien rooie}{centen over niet meer}\\

\haiku{Want het geld van m'n '.}{maat is me even lief alst}{geld van een jonker}\\

\haiku{{\textquoteright} En toen 't duppie, '.}{weer in mijn vingers zat wier}{t me daar zo heet}\\

\haiku{Want we zaten daar.}{alleen maar wat lustig langs}{de dijk te zingen}\\

\haiku{Zo zie je, zaken,,.}{doen mag niet allenig rijk}{zijnde d\'an mag het}\\

\haiku{As ik nog 'ns zou,, '.}{motten beginnen met jou}{dan deek het niet}\\

\haiku{Hij heeft geleerd, geld,.}{te porren uit arm en rijk}{uit leed en pleizier}\\

\haiku{Hij had 'em de kop,{\textquoteright}.}{in motten slaan en jou d'r}{bij barst Jochem uit}\\

\haiku{zij lachten mee om.}{de kwaaie aard in die ouwe}{meid te bezweren}\\

\haiku{Heel de aarde met,.}{alle mensen er op draait}{om de rijksdaalders}\\

\haiku{Daarom bleef hij van;}{toen af maar rond lopen in}{een kleine cirkel}\\

\haiku{wat liefde en wat,,.}{haat geboren worden wat}{groeien en sterven}\\

\haiku{as je gaat rechten,.}{om een plank van drie gulden}{dat kost drie duzend}\\

\haiku{En... wij gaan alvast,.}{de ouwe plank over als ie}{nog lang genog is}\\

\haiku{Best. Maar la\^an ze dan,.}{naar d'r eigen streek gaan die}{smerige Belzen}\\

\haiku{En nou heb ik niks......}{niks Geen koei en geen werkloon}{om wat te prutsen}\\

\haiku{En dat weer omdat,.}{jullie geen geld hebben om}{er slecht mee te doen}\\

\haiku{je moet je in die.}{zaken nooit door je gevoel}{laten meeslepen}\\

\haiku{{\textquoteright} {\textquoteleft}Juist Chef, kinderen.}{vragen hartelijkheid en}{huisgezinsgeluk}\\

\haiku{{\textquoteright} {\textquoteleft}Schaam jij je eigen,,.}{maar wat om dat te zeggen}{waar die twee bij zijn}\\

\haiku{Ik geef als 't mij,,?!}{belieft en dat gaat jou geen}{bliksem aan verstaan}\\

\haiku{Ik docht al, wat een....}{hoop zoveel Pausen bennen}{d'r niet eens gewist}\\

\haiku{{\textquoteright} {\textquoteleft}Jaat, en 't mankeert,.}{er nog maar aan das ze je}{antwoord gaan geven}\\

\haiku{{\textquoteright} zei Jochem toen hij ',.}{t stuivertje eerst goed vast}{had ter verklaring}\\

\haiku{Hij is met z'n knecht,.}{de zevende boom van twee}{en halfpond al kwijt}\\

\haiku{{\textquoteleft}En Bart, as je an, '.}{de kaart komt slaan ik jen}{paar ribben kapot}\\

\haiku{Vooral, omdat ze.}{veel nieuws wisten van tussen}{Schoonhoven en hier}\\

\haiku{{\textquoteright} - Maar Brandewijn met,.}{Suiker gaat er nou veur zes}{jaar achter minstens}\\

\haiku{Een stuk pleizier is, '.}{van z'n avond aft geluk}{van leed uitdragen}\\

\haiku{Als Piet Miltenburg,.}{honderd jaren wordt zal ie}{er n\'og van schenden}\\

\haiku{{\textquoteright} {\textquoteleft}Verdomd, nee, bliksem,;}{Goof daar vat jij eigentlijk}{de pan bij de steel}\\

\haiku{wat is er ievers?}{in die zekere orde}{voor h\'em weggelegd}\\

\haiku{Elk ander mens  .}{zou neergestort zijn bij zo'n}{liefelijk klapje}\\

\haiku{Wij zijn nou eenmaal,,.}{aangewezen om tot het}{end niks te hebben}\\

\haiku{wij vragen niet wijex,.}{dan een homp leverworst te}{maggen bekommen}\\

\haiku{z\'o beroerd kan u.e, ',.}{t nooit hebben of wij zijn}{er naakter aan toe}\\

\haiku{Wij hebben niks, maar....}{heel de wereld is van ons}{om op te lopen}\\

\haiku{{\textquoteright} vond de Notaris,.}{gram en hij moedigde Chef}{aan met z'n stokje}\\

\haiku{al stem ik toe dat,.}{het een gedachte is waard}{om te overwegen}\\

\haiku{Als 't niet naar recht, '.}{is dan motten wet niet}{goed willen praten}\\

\haiku{E\'en ding is zeker,{\textquoteright}, '.}{zei Jochem toen ze buiten}{t gehoor waren}\\

\haiku{Maar als hij nader...}{stuift en veel plaats is er niet}{in hun scheepskeuken}\\

\haiku{dat jonk is nou z\'o.}{uit het nest gekropen en}{z\'o voor vreugd bestemd}\\

\haiku{Hij valt daar buiten,.}{ook al houdt hij stijf z'n pet}{op zijn glibberkop}\\

\haiku{Twintig uren in de,,...}{sloep en dan roeien zonder}{richting te weten}\\

\haiku{{\textquoteleft}Volgende keer krijg,.}{je kerels van me mee om}{aarpels te jassen}\\

\haiku{En ineens hoorde.}{hij de stem terug van Bartje}{Rijkelijkhuizen}\\

\haiku{Naar Rotterdam is.}{hij helemaal eens in zijn}{leven heen geweest}\\

\haiku{Als je te bedde,.}{ligt dan is de wereld stil}{en ook de mensen}\\

\haiku{Toen kwam er wat wilds.}{in zijn kop en hij greep zijn}{knikkerbuiltje vast}\\

\haiku{{\textquoteright} Maar Aai wou bij Brok.}{heel niet zeggen waarvoor hij}{naar die kooplui trok}\\

\haiku{De juffrouw vatte.}{daarop heur man bij de arm}{en fluisterde wat}\\

\haiku{Ze hebben 't hem.}{niet nagewezen en er}{niet op gezinspeeld}\\

\haiku{{\textquoteleft}het vlees is beter;}{dan de benen en niemand}{is te vertrouwen}\\

\haiku{Waarom had nou die?}{Gert Koeimans zo'n hekel aan}{de drie serpenten}\\

\haiku{Maar van toen af moest.}{Gert Koeimoes dag aan dag die}{loop er bij maken}\\

\haiku{Bijgelovig ben,,.}{ik niet maar dat ze heksten}{dat geloof ik w\'el}\\

\haiku{Maar 't ergste was,. '}{wel dat ze als honden op}{de landpacht waren}\\

\haiku{{\textquoteright} Daarop riepen ze,.}{in de nacht dat ze van de}{politie waren}\\

\haiku{Willen ze soms niet,.}{lopen dan kittelen ze}{met de zwiep hun oren}\\

\haiku{la\^an ze dan maar blij,;}{zijn dat ze met de kast mee}{omgevallen zijn}\\

\haiku{Dewijl gij zegt, dat.}{ik door Be\"elzebub de}{duivelen uitwierp}\\

\haiku{{\textquoteleft}En d\'an, 't is toch,.}{maar toeval dat we hier bij}{mekare zitten}\\

\haiku{{\textquoteright} {\textquoteleft}Als je z\'o begint,{\textquoteright}, {\textquoteleft} '.}{vond Jasdan hou jet maar}{netjes onder je}\\

\haiku{Zou er toen ergens,?}{wat bij hem gesprongen zijn}{een bloedaar of zo}\\

\haiku{{\textquoteright} Maar d\'at had Jas niet,.}{moeten zeggen want to\'en was}{het hek van de dam}\\

\haiku{uedele verkoopt,....}{geen lullificatie maar}{lullificatie}\\

\haiku{Niemand heeft uit zijn.}{eigen levensloop en van}{zijn vak wat verteld}\\

\haiku{Want als ze eenmaal,,.}{wat zei juffrouw Naatje dan}{duisterde ze niet}\\

\haiku{{\textquoteleft}Ach,{\textquoteright} zei de meester, {\textquoteleft}?}{heb ik daar werkelijk nog}{Jenny Lindjes}\\

\haiku{En nou zeg ik maar,,...}{met de hand op de Grondwet}{dat als het waar is}\\

\haiku{{\textquoteright} Maar Kees liet weten,.}{dat hij nog zeker een half}{uur te stoken had}\\

\haiku{{\textquoteright} {\textquoteleft}Dat je je geld kreeg,.}{een hand en een brief van de}{geheelonthouding}\\

\haiku{{\textquoteleft}en hier, meneer de,...{\textquoteright}.}{commissaris hebben we}{en toen zag hij mijn}\\

\haiku{{\textquoteright} Daarop kreeg ik van.}{die meneer een gulden en}{een fijne sigaar}\\

\haiku{- En eer Chef er op,.}{bedacht was had zijn maat dat}{alw\'e\'er verkondigd}\\

\haiku{ik zat zelf aan de,;}{cassa onze voorstelling}{moest nog beginnen}\\

\haiku{Sommige wouwen,.}{het kooitje nog niet uit maar}{daar wist ik raad op}\\

\haiku{Zo'n anker van een;}{Maasstroom-bootje is in}{een oortje gelicht}\\

\haiku{En madame was,.}{w\'e\'er in gebreke ze heeft}{het w\'e\'er niet voorzien}\\

\haiku{{\textquoteleft}hij kent dat liedje,.}{niet goed want ineens raakt hij}{uit de melodie}\\

\haiku{had geld tekort in.}{z'n kas  en die wou zich}{voor zijn raap schieten}\\

\haiku{Maar 't was aardig,.}{om te zien hoe hef ze met}{elkaar omgingen}\\

\haiku{{\textquoteright} {\textquoteleft}En Jacq de Zeehond,?}{wat is er naderhand nog}{van hem geworden}\\

\haiku{En nog, als ik in,.}{Amsterdam kom ga ik wel}{eens naar Francientje}\\

\haiku{Tussen de benen,.}{der mensen door hobbelde}{hij voorzichtig voort}\\

\haiku{die vent is aan 't,;}{malen maar geld is geld en}{handel is handel}\\

\haiku{En dan... je kan nooit;}{weten wat zo'n snijer op}{z'n hoofd heeft lopen}\\

\haiku{{\textquoteleft}Ja,{\textquoteright} ging hij voort, {\textquoteleft}want.}{dezer dagen heb je er}{niet \'e\'en gesleten}\\

\subsection{Uit: Een stoombootje in den mist}

\haiku{'t is jammer voor,.}{Dorus maar hij heeft  een zeer}{hoofd onder z'n pet}\\

\haiku{Jas heeft het niet op,.}{die walle-bazen ze}{heulen met den wal}\\

\haiku{IJs is er dat jaar,.}{niet veel geweest maar water}{en sneeuw niet zuinig}\\

\haiku{Een mensch kan maar aan,.}{den gang blijven helpen doet}{het geen donderament}\\

\haiku{Half drie is het ruim,.}{als hij de schotten van de}{loopplank laat sjorren}\\

\haiku{Hun kleer was er van,,,.}{overtogen hun haren het}{anker de trossen}\\

\haiku{{\textquoteright} {\textquoteleft}Kan me geen donder,!}{schelen ik jaag de schuit niet}{op het zand verstaan}\\

\haiku{Jullie weten nou, '.}{hoet een schipper vergaan}{kan in den bliksem}\\

\haiku{t Was mijn jong niet, ',.}{t was Toontje niet dat zee}{ik toch al d\^alijk}\\

\haiku{'t Laken gong er.}{af en we zagen niks dan}{rauw vleesch en bloed}\\

\haiku{dan slijt de nacht \'o\'ok.}{en valt er lichtelijk nog}{wat te verdienen}\\

\haiku{En Kees, de stoker,.}{was ook al van de schuit en}{ook de sloep was weg}\\

\haiku{En als je niet meer,.}{bekomen kan d\`an eerst voel}{je de ontbering}\\

\haiku{Als vandaag een boer,.}{sterft morgen vat een ander}{z'n spaai bij den steel}\\

\haiku{En z'n ouwe vrouw;}{gong hemelen en toen heit}{ie dat veurtgezet}\\

\haiku{Maar daar sturen we,,}{den bond op af m\'a\'ar dat is}{voor later zorg m\'a\'ar}\\

\haiku{Nou moet je weten, ';}{die schijf opt lijf van den}{tamboer is knap groot}\\

\haiku{Hij regende nat ';}{en toen maakten we vant}{zeil wat meer luifel}\\

\haiku{En 'k docht al, nou,.}{gaat het uit zijn maar hij nam}{n\`og voor een kwartje}\\

\haiku{We hadden op 't....}{lest handen te kort \'e\'en klant}{en een tent vol werk}\\

\haiku{Hij raakte er een,...}{keer acht van de tien hij kocht}{w\'e\'er voor een kwartje}\\

\haiku{En direct docht ik,.}{op dien daggelder ik weet}{zuiver niet waarom}\\

\haiku{Vreet heel 't schip maar!}{leeg en zuip al de melk op}{en al het water}\\

\haiku{Zulk een bericht gaat,.}{als op den wind van den een}{op den ander over}\\

\haiku{Maar als hij nader...}{stuift en veel plaats is er niet}{in hun scheepskeuken}\\

\haiku{dat jonk is nou z\'o\'o.}{uit het nest gekropen en}{z\`o\`o voor vreugd bestemd}\\

\haiku{Hij valt daar buiten,.}{ook al houdt hij stijf z'n pet}{op zijn glibberkop}\\

\haiku{Twintig uren in de,,...}{sloep en dan roeien zonder}{richting te weten}\\

\haiku{Toe nou Dorusje lief,...{\textquoteright} {\textquoteleft},{\textquoteright}, {\textquoteleft}.}{kereltje toe nouNee riep}{de rostejij niet}\\

\haiku{Volgenden keer krijg,.}{je kerels van me mee om}{aarpels te jassen}\\

\haiku{En ineens hoorde.}{hij de stem terug van Bartje}{Rijkelijkhuizen}\\

\haiku{jullie waagde en?}{dat de reederij nog geld}{toegeven moest \'o\'ok}\\

\haiku{Naar Rotterdam is.}{hij heelemaal eens in zijn}{leven heen geweest}\\

\haiku{Als je te bedde,.}{ligt dan is de wereld stil}{en ook de menschen}\\

\haiku{Toen kwam er wat wilds.}{in zijn kop en hij greep zijn}{knikkerbuilje vast}\\

\haiku{- Maar Aai wou bij Brok.}{heel niet zeggen waarvoor hij}{naar die kooplui trok}\\

\haiku{De juffrouw vatte.}{daarop heur man bij den arm}{en fluisterde wat}\\

\haiku{{\textquoteright} {\textquoteleft}'t Is,{\textquoteright} zei toen het, {\textquoteleft}',}{hoofd van Jutt is dat jij}{eigens nog te jong}\\

\haiku{Ze hebben 't hem.}{niet nagewezen en er}{niet op gezinspeeld}\\

\haiku{Waarom had nou die?}{Gert Koeimoes zoo'n hekel aan}{de drie serpenten}\\

\haiku{Maar van toen af moest.}{Gert Koeimoes dag aan dag dien}{loop er bij maken}\\

\haiku{ze hadden een hoop.}{van die uitheemsche katten}{met lange haren}\\

\haiku{Bijgeloovig ben,,.}{ik niet maar dat ze heksten}{dat geloof ik w\`el}\\

\haiku{Maar 't ergste was,. '}{wel dat ze als honden op}{de landpacht waren}\\

\haiku{) op een kippenhok.}{gevallen en half kreupel}{van den val gevlucht}\\

\haiku{De dienders dan, want,:}{daar was ik gebleven de}{dienders die zee\"en}\\

\haiku{Daarop riepen ze,.}{in den nacht dat ze van de}{politie waren}\\

\haiku{la\^an ze dan maar blij,;}{zijn dat ze met de kast mee}{omgevallen zijn}\\

\haiku{Dewijl gij zegt, dat.}{ik door Be\"elzebub de}{duivelen uitwerp}\\

\haiku{Zou er toen ergens,?}{wat bij hem gesprongen zijn}{een bloedaar of zoo}\\

\haiku{{\textquoteright} Maar d\`at had Jas niet,.}{moeten zeggen want t\`oen was}{het hek van den dam}\\

\haiku{Uedele verkoopt,....}{g\'e\'en lullificatie maar}{lullificatie}\\

\haiku{Niemand heeft uit zijn.}{eigen levensloop en van}{zijn vak wat verteld}\\

\haiku{Want als ze eenmaal,,.}{wat zei juffrouw Naatje dan}{fluisterde ze niet}\\

\haiku{En nou zeg ik maar,,...}{met de hand op de Grondwet}{dat als het waar is}\\

\haiku{{\textquoteright} Maar Kees liet weten,.}{dat hij nog zeker een half}{uur te stoken had}\\

\haiku{{\textquoteleft}Dat ig zoo'n ouwe ',.}{truc Gewoonwegt kistje}{omkeeren anders niks}\\

\haiku{{\textquoteright} {\textquoteleft}Dat je je geld kreeg,.}{een hand en een brief van de}{geheelonthouding}\\

\haiku{Daarop kreeg ik van.}{dien meneer een gulden en}{een fijne sigaar}\\

\haiku{- En eer Chef er op,.}{bedacht was had zijn maat dat}{alw\'e\'er verkondigd}\\

\haiku{ik zat zelf aan de,;}{cassa onze voorstelling}{moest nog beginnen}\\

\haiku{Sommigen wouwen,.}{het kooitje nog niet uit maar}{daar wist ik raad op}\\

\haiku{Zoo'n anker van een;}{Maasstroom-bootje is in}{een oortje gelicht}\\

\haiku{En madame was,.}{w\'e\'er in gebreke ze heeft}{het w\'e\'er niet voorzien}\\

\haiku{- hij kent dat liedje,.}{niet goed want ineens raakt hij}{uit de melodie}\\

\haiku{Maar 't was aardig,.}{om te zien hoe lief ze met}{elkaar omgingen}\\

\haiku{{\textquoteright} {\textquoteleft}En Jacq de Zeehond,?}{wat is er naderhand nog}{van hem geworden}\\

\haiku{En nog, als ik in,.}{Amsterdam kom ga ik wel}{eens naar Francientje}\\

\haiku{die vent is aan 't,;}{malen maar geld is geld en}{handel is handel}\\

\haiku{En dan... je kan nooit;}{weten wat zoo'n snijer op}{z'n hoofd heeft loopen}\\

\haiku{{\textquoteleft}Ja,{\textquoteright} ging hij voort, {\textquoteleft}want.}{dezer dagen heb je er}{niet \'e\'en gesleten}\\

\haiku{Een vakman uit de {\textquoteleft}{\textquoteright},.}{kleine wereld door Herman}{de Man beschreven}\\

\haiku{We laten 't hem:}{echter liever persoonlijk}{voor U vertellen}\\

\subsection{Uit: Het wassende water}

\haiku{Ze was gierig met,.}{haar liefde want ervaring}{had haar slim gemaakt}\\

\haiku{Daarom wier er ten.}{avond zoo dukkels over moeders}{gepraat in de schuit}\\

\haiku{Zoo wijd hij zien kon,.}{leek het of er toen geen rund}{meer lag in het land}\\

\haiku{{\textquoteright} {\textquoteleft}Zoo, nou dan kan 't,{\textquoteright}.}{n\`et over zijn zegde hij droog}{en liep den huis in}\\

\haiku{Hij aardde d\'a\'arin zijn.}{vader die ook nooit veel van}{woorden was geweest}\\

\haiku{Hij buigt het hoofd en,,}{heft het weer hij dwaalt en zwaait}{onvast en aleer}\\

\haiku{wou en keek verward '.}{doort koekoekraamt over de}{landen tot de Lek}\\

\haiku{Hij heeft zijn barre.}{woorden opgevreten en}{het hoofd gebogen}\\

\haiku{Moeders kende die}{drift en \'o\'ok de luwing en}{ze viel heur rauw jonk}\\

\haiku{Dirk Hoogerzeil hun,.}{gebuur toen kon ze b\`est diens}{plagen verdragen}\\

\haiku{Mogelijk had je.}{me d\`an toendertijd niet}{zoo kwalijk geraa{\"\i}en}\\

\haiku{En Willem... wou jij?}{rechtevoort op vader zijn}{stoel zien te kommen}\\

\haiku{Daarom is 't zoo,.}{jammer dat jij en moeders}{verschillen hebben}\\

\haiku{een koebeest dat wel.}{twee honderd gulden boven}{prijs wordt afgemijnd}\\

\haiku{Weggedrongen zat.}{zijn schamende gestalte}{tegen een schot aan}\\

\haiku{Doch ie, da 'k me?}{w\'e\'er op dat punt zou laten}{bepraten deur haar}\\

\haiku{het boerenbedrijf.}{dat is in mijn leven zoo}{noodig als het asemen}\\

\haiku{Hij keert manhaftig.}{terug tot het mooie werksche}{leven van een boer}\\

\haiku{'t Is goed rond en, '.}{glanzend best doort kalf heen}{en in vollen geef}\\

\haiku{Hij wordt Jaap van Jan, '.}{de Pauw genaamd naart huis}{de Pauw in Teckop}\\

\haiku{{\textquoteleft}En as Willem dan ' '...{\textquoteright} {\textquoteleft}?}{trouwen gaat en moeders doet}{emt land overH\`e}\\

\haiku{Wat een kw\^agast, om '.}{t veur zijn eigen bro\^er zoo}{lang op te vreten}\\

\haiku{Hij is fortuinlijk,,.}{in koop en verkoop op den}{huis en de markten}\\

\haiku{{\textquoteright} 't Was toen maar weer,.}{moeders die er later op}{den dag over begon}\\

\haiku{In weinig uren tijds,.}{was aan beider leven een}{groote draai gegeven}\\

\haiku{{\textquoteleft}ge het mijn vader.}{gedacht en ik zal hier mijn}{vader gedinken}\\

\haiku{Vliet en Dijkveld mocht;}{nooit of nooit onderloopen}{ofwel vervuilen}\\

\haiku{{\textquoteright} En ze begreep, dat,.}{hij \`al de schouwbrieven thans}{langs ging het groote keind}\\

\haiku{Stuur morgen 'an den '.}{dag maar een man omt ding}{bij mijn in te slaan}\\

\haiku{v\'o\'or heden had hij.}{nooit of nooit aan Geertrui de}{Goei aldus gedacht}\\

\haiku{Zal je later met,?}{me me\^e willen naar onze}{woning gunterwijd}\\

\haiku{{\textquoteright} En d\^alijk zocht hij,.}{van heur oogen af te lezen}{hoe ze dat wel vond}\\

\haiku{{\textquoteright} {\textquoteleft}En nou 'n elkeen}{hieromtrent er van afweet}{dat wij trouwen gaan}\\

\haiku{och toch gelukkig, '.}{alt lieve was nog niet}{verdroogd in zijn lijf}\\

\haiku{Maar haar verdoken?}{verlangen vroeg mogelijk}{\`andere vreugde}\\

\haiku{En ik bin d'r niet,.}{zeker van of ik jou veur}{den kop mag stooten}\\

\haiku{Ik zal gaan, ik zal,.}{opschuiven veur den ware}{veur de jeugdigheid}\\

\haiku{in beslagenheid -.}{gaat onze Gieljan Beijen}{zijn vaders weg op}\\

\haiku{En vraag mijn ook maar,.}{gien raad meer je doen t\`och je}{eigen believen}\\

\haiku{Maar toch kwam deze.}{nieuwe verdrietigheid niet}{op haar hard wezen}\\

\haiku{'t Allereerst zijn.}{benoeming tot Hoogheemraad}{van het dijkgebied}\\

\haiku{Zoo'n rustig en t\`och;}{bespraakt man zond een ieder}{graag naar het Dijkhuis}\\

\haiku{E\'en zak chili is,.}{duur een wagon chili maakt}{den prijs al lager}\\

\haiku{Hij wier er niet wild,.}{van maar neep aandachtig zijn}{vingers tot vuisten}\\

\haiku{{\textquoteleft}Ik bin den Dijkgraaf,!}{van den Lekkendijk meneer}{de Burgemeester}\\

\haiku{Het was hem daarna,:}{een blije veraseming toen}{gerapporteerd wier}\\

\haiku{Mogelijk ware.}{alzoo het water te keeren}{uit zijn laag gebied}\\

\haiku{We weten, dat thans,.}{een land volloopt dat daardeur}{veul schaai zal lijen}\\

\haiku{Maar met minder schaai,!}{is nievers het water te}{laten w\`el met meer}\\

\haiku{Na deze woorden.}{brak de ontstelling eerst in}{waren omvang uit}\\

\haiku{Een broer teugen een,.}{broer da's in het openbaar gien}{verheven gevecht}\\

\haiku{as hem zijn eigen,,?!}{nest wil bevuilen dan mot}{hem dat maar doen waar}\\

\haiku{Gij als ik, we zijn,!}{toch mannen van den arbeid}{mannen van het land}\\

\haiku{Maar w\`el onder het.}{boerenvolk was hittigheid}{om die benoeming}\\

\haiku{Hebt eerbied voor den,.}{boerenstand Den gulden stand}{van Nederland}\\

\haiku{12Met behulp van.}{bijvoer meer koeien houden}{dan het land toelaat}\\

\section{Lidy van Marissing}

\subsection{Uit: De omgekeerde wereld}

\haiku{Steeds minder slaat hij.}{als een razende met z'n}{handen op de grond}\\

\haiku{Hij tilt zijn prooi net.}{even van de grond en laat haar}{plotseling weer los}\\

\haiku{Z{\'\i}j een bruine rug -!}{en w{\'\i}j de zenuwen dat}{pikken we niet meer}\\

\haiku{Z{\'\i}j een bruine rug -!}{en w{\'\i}j de zenuwen dat}{pikken we niet meer}\\

\haiku{Wanneer ik hem zeg,.}{dat hij moet kijken kijkt hij}{niet naar behoren}\\

\haiku{Hij neemt de prijskaart.}{van een rode lakjas en}{hangt die op zijn borst}\\

\haiku{zij werd tegen de).}{Arabiese gebruiken in}{aan mannen getoond}\\

\haiku{- dat hij net zo goed,.}{iemand anders had kunnen}{zijn op dat moment}\\

\haiku{Vlak v\'o\'or hij tegen:}{de deur duwde hoorde hij}{een korte  klik}\\

\haiku{- Soms denk ik wel eens.}{aan dingen die te slecht zijn}{om over te praten}\\

\haiku{Deze techniek mag.}{alleen dan gebruikt worden}{als het terecht is}\\

\haiku{Ook deze techniek;}{mag alleen worden gebruikt}{als het terecht is}\\

\haiku{hij zal verstandig{\textquoteright}).}{zijn en volmaakt en alles}{doen wat ik zeg}\\

\haiku{- Ik heb een wereld,.}{van dagdromen waarover ik}{niemand iets vertel}\\

\haiku{{\textquoteright} ~ Dat wat niet te,.}{verdragen is blijft in ons}{opgesloten}\\

\haiku{Men heeft dan meer kans {\textquoteleft}{\textquoteright} {\textquoteleft}{\textquoteright},}{omregt te zien enregt te}{loopen zoodra}\\

\haiku{men hoorde zeggen {\textquoteleft}, {\textquoteleft},!}{Als het u gelieft Mevrouw!{\textquoteright}en}{Na u Eerwaarde}\\

\haiku{Voor het overige.}{beeldt men zich in alleen te}{kunnen leven}\\

\haiku{{\textquoteright} Een minuut later.}{hoorden we de auto de}{straat uit scheuren}\\

\haiku{zijn arbeid blijft hem,.}{vreemd zijn werksituatie}{blijft onbegrepen}\\

\haiku{{\textquoteleft}Ja, met een duur woord.}{mag je het nu inderdaad}{wel koncern noemen}\\

\haiku{Jullie hebben al.}{twee keer zoveel loon als een}{paar jaar geleden}\\

\haiku{Voor het schrijven van.}{deze tekst beschikken wij}{maar over weinig tijd}\\

\haiku{Zoals elke munt.}{of medalje heeft ook elk}{begrip twee kanten}\\

\haiku{Bijvoorbeeld door het.}{kloppen met een steen of stuk}{hout op de tafel}\\

\haiku{Iedereen is het,.}{met een plan eens maar eentje}{gaat er tegen in}\\

\haiku{Hij stapte uit met,.}{een vaag schuldgevoel dat hij}{zelf onderdrukte}\\

\haiku{WIJ PROTESTEREN?}{TEGEN        Hoofdstuk XX Als}{ratten in de val}\\

\haiku{Inmiddels waren.}{de eveneens schaars geklede}{vrouwen gekomen}\\

\haiku{de woorden af die.}{vaak slechts misverstanden en}{illusies scheppen}\\

\haiku{Wij bekleden een.}{vooraanstaande positie}{in de maatschappij}\\

\haiku{1969 - Ruim twintig man.}{zijn gebleven en hebben}{de fabriek bezet}\\

\haiku{Maar voor je er bent,.}{moet je een heel stuk lopen}{wel acht straten door}\\

\haiku{Hier en daar was een.}{huis als bij toeval op het}{land neergevallen}\\

\haiku{In de zekerheid}{dat wet en orde aan hun}{kant staan generen}\\

\haiku{D.w.z. het is al eens,?)}{gebeurd maar wanneer gaat het}{weer gebeuren}\\

\haiku{Vertelt hij wat hij?}{heeft gezien of wat hij denkt}{te hebben gezien}\\

\haiku{Nou, luister 's goed,{\textquoteright}.}{zei hij en ging recht overeind}{op de bank zitten}\\

\section{Pauline Marres}

\subsection{Uit: En toen brak de hel los. De verwoesting van Maastricht door Parma in 1579}

\haiku{De verwoesting van}{Maastricht door Parma in 1579}{Colofon}\\

\haiku{Weet je dat ik je?}{alleen heb meegenomen}{om haar te treffen}\\

\haiku{Kom Mirza, nu gaan '.}{we proberenn beetje}{geluk te vinden}\\

\haiku{{\textquoteright} Stapvoets reed hij tot,.}{bij de dichtstbijzijnde poort}{de Tongerse Poort}\\

\haiku{{\textquoteleft}Als Parma komt, zal,}{hij hier ergens wel een brug}{over de rivier slaan}\\

\haiku{de Proenens zijn aan,?}{hen verwant daar hebt u toch}{wel eens van gehoord}\\

\haiku{Weet u dat keizer?}{Karel zelfs bij hen aan huis}{kwam in Antwerpen}\\

\haiku{Die zwetskop van een,.}{waard vertelt meer dan nodig}{is straks ook van mij}\\

\haiku{{\textquoteright} {\textquoteleft}Jan, misschien is hij '.}{opt ogenblik de enige}{wijze in Maastricht}\\

\haiku{Maar als ik geen geld...{\textquoteright} {\textquoteleft}.}{krijgDe stad kan nauwelijks}{haar schuld betalen}\\

\haiku{Toen ze lager keek,,;}{naar hun sluisje sloeg haar een}{plotselinge schrik}\\

\haiku{Ja, liefje, je lag,.}{in de Jeker we hebben}{je eruit gehaald}\\

\haiku{Veel kinderen in ',.}{t kamp hadden er wel een}{altijd dezelfde}\\

\haiku{'t Lag zomaar op,,:}{een bankje voor een tent er}{was niemand die zei}\\

\haiku{{\textquoteleft}Misschien beschermt 't,{\textquoteright}.}{je nog eens als er kogels}{rondvliegen zei ze}\\

\haiku{En bovendien een,,?}{kale muts zonder kant zelfs}{geen geplooide strook}\\

\haiku{{\textquoteright} zei moeder Mina, {\textquoteleft}.}{kanten en stroken horen}{bij rijke mensen}\\

\haiku{Toen ze in de tuin,.}{waren kwam opeens een hond}{blaffend op hen af}\\

\haiku{Toen reikte de knecht,.}{hem het wambuis aan eveneens}{bruin en zwart gestreept}\\

\haiku{als ik er weer een,.}{moet verkopen zal ik naar}{Luik of Leuven gaan}\\

\haiku{{\textquoteright} zei hij verbluft en.}{tevens beledigd door de}{gemeenzame toon}\\

\haiku{Alleen maar kijken '.}{naar de mooie mensen ent}{lekkere  eten}\\

\haiku{Toen sprong ze brutaal '.}{op Nol's bed om te voelen}{hoe zachtt wel was}\\

\haiku{Van Daelen had wel;}{verwacht dat hij Manzano}{hier zou aantreffen}\\

\haiku{die elk half uur op,.}{zijn trompet blaast ten teken}{dat hij wakker is}\\

\haiku{Er werd hard gewerkt,,;}{aan de wallen en poorten}{torens en grachten}\\

\haiku{'t Oude geschut,.}{werd weer bruikbaar gemaakt ook}{nieuw werd gegoten}\\

\haiku{In ploegen werkten.}{alle gildeleden mee}{met de soldaten}\\

\haiku{{\textquoteright} herhaalde hij en.}{keek de ander een hele}{poos weifelend aan}\\

\haiku{maar een mokkend of.}{ontevreden gezicht kon}{hij niet ontdekken}\\

\haiku{{\textquoteright} Langzaam, en onder.}{veel gesteun haalde hij een}{mand vol houtblokken}\\

\haiku{{\textquoteleft}Jij, je vader, je.}{moeder zullen zelf moeten}{zorgen voor alles}\\

\haiku{Proenen kwam na wat.}{heen- en weergeloop de}{kamer binnen}\\

\haiku{Ze moet dan morgen,,.}{een goede kip brengen Nol}{en wat eieren}\\

\haiku{Onderwijl moet jij, '.}{hier opruimen ookt bed}{in orde maken}\\

\haiku{{\textquoteright} Manzano had een.}{tijdlang in noordelijke}{richting staan kijken}\\

\haiku{Dat was nu heel wat.}{eenvoudiger geworden}{dan de eerste keer}\\

\haiku{Omdat door deze.}{deur zo'n heerlijke etenslucht}{komt als ze opengaat}\\

\haiku{De poorten waren,.}{dichtgegooid de bezetting}{betrok de wallen}\\

\haiku{Dag-in dag-uit zat '.}{ze in haar stoel en prikte}{haar naald int goed}\\

\haiku{{\textquoteleft}Hoe kon 'n mooie vrouw:}{als mevrouw Suetendael zich}{zo toetakelen}\\

\haiku{Oude monniken,.}{liepen ernaast met dikke}{brandende kaarsen}\\

\haiku{Vrouwen, kinderen,;}{grijsaards en zieken volgden}{in dichte rijen}\\

\haiku{Links en rechts verkocht.}{hij kaarsen aan mensen die}{zich mee aansloten}\\

\haiku{Maar als je wat vindt,;}{zorg dan dat niemand op straat}{kan zien wat je draagt}\\

\haiku{Aan de overkant van,, '.}{de Maas in Wijk int huis}{van sinjeur Tapijn}\\

\haiku{En terwijl hij naar,.}{boven klom greep de wanhoop}{hem weer bij de keel}\\

\haiku{De wal was hoog, de.}{gracht daaronder heel diep en}{met water gevuld}\\

\haiku{Als reden werd hem.}{verteld dat de grond er te}{vochtig was voor kruit}\\

\haiku{hij meende zelfs nu.}{en dan een druppel water}{te horen vallen}\\

\haiku{Als hij maar onder, '.}{die vervloekte gracht door was}{zout beter gaan}\\

\haiku{Inderdaad keek de.}{man met een schuin oog naar Nols}{manipulaties}\\

\haiku{Hij miste de moed.}{en de lust om zich naar zijn}{post te begeven}\\

\haiku{Hier en daar stond een.}{verlaten houten hut van}{een warmoezenier}\\

\haiku{'t Restje luie zweet,.}{dat je nog hebt zullen we}{er wel uitpersen}\\

\haiku{Waar zal ik schuilen?}{als de Spaanse benden over}{de stad losbreken}\\

\haiku{Blijkbaar herkende,:}{ze hem want hij hoorde een}{stem zachtjes roepen}\\

\haiku{Toen ze de eerste,;}{tak had bereikt zwiepte ze}{de ladder omver}\\

\haiku{Ze hield een hand in.}{haar zij om haar lachen in}{bedwang te houden}\\

\haiku{{\textquoteright} Bella zag niets, ze;}{keek weer recht voor zich uit als}{een slaapwandelaar}\\

\haiku{Ze zuchtte, omdat.}{ze zo opeens duidelijk}{de plaats herkende}\\

\haiku{Over 't dak kon hij,.}{niet gevlucht zijn de raampjes}{waren gesloten}\\

\haiku{Hij doorboorde 't;}{gezwel zodat de vuile}{inhoud eruit spoot}\\

\haiku{Ze stelden zich in,.}{twee rijen op voor zijn tent}{trokken hun degens}\\

\haiku{Als u mee wilt gaan '...}{ent voor deze keer wilt}{verontschuldigen}\\

\haiku{voor deze keer want,;}{die vreemde heren blijven}{niet zo lang ziet u}\\

\haiku{Hij keerde zich om.}{en zag een vrouw zitten die}{groente schoonmaakte}\\

\section{H. Marsman}

\subsection{Uit: De dood van Ang\`ele Degroux}

\haiku{Dit is mijn klooster,.}{jammer voor jou dat jij er}{moeder-overste bent}\\

\haiku{landschap na landschap;}{vouwde zich in hem open en}{verschemerde weer}\\

\haiku{Hij liep door drukke,.}{en stillere straten maar}{hij merkte het niet}\\

\haiku{Het was op een avond.}{in September geweest in}{het vorige jaar}\\

\haiku{Met dit gedrocht had,!}{zij geleefd met die vrouw had}{dit monster geleefd}\\

\haiku{- Wij kennen elkaar,... -}{nog zoo weinig maar als het}{u interesseert}\\

\haiku{Toen, terugkeerend uit,,:}{zijn gedachten zeide hij}{Charles strak aanziend}\\

\haiku{Maar toen hebben uw.}{oogen mij haar beeld met een schok}{teruggegeven}\\

\haiku{Neen, bewijzen kan,?}{ik het niet maar wat kunnen}{we w\`el bewijzen}\\

\haiku{Rutgers zat, klein en,.}{wanstaltig aan zijn voeten}{en keek in het vuur}\\

\haiku{Toen het roepen aan.}{den concierge om de deur}{voor hem open te doen}\\

\haiku{De avenue was een.}{zwarte spiegel waarin de}{lantarens dansten}\\

\haiku{Haar antwoord klonk zacht,,,.}{al lag er dacht Charles iets}{van afkeuring in}\\

\haiku{Het omgaf zijn  .}{lichaam met een frissche en}{intieme warmte}\\

\haiku{Het was laat in den.}{avond toen Charles begon te}{spreken over Ang\`ele}\\

\haiku{Maar het gevoel van,,.}{verwantschap waarover hij sprak}{ontbrak haar geheel}\\

\haiku{Maar hij kreeg ondanks.}{herhaalde en dringende}{aanvraag geen gehoor}\\

\haiku{Ik geloof dat u.}{ook de overeenkomst tusschen}{haar en uzelf overschat}\\

\haiku{Hij had alleen, na,.}{hun breuk maandenlang als een}{kluizenaar geleefd}\\

\haiku{Maar de dood deed niets,.}{overhaast hij ging langzaam en}{overwogen te werk}\\

\haiku{Maar misschien had hij,....}{haar wel vergeten misschien}{ook was hij getroost}\\

\haiku{Van der Mark was een;}{lange magere man van}{omstreeks veertig jaar}\\

\haiku{De scherpte van zijn:}{onderzoekenden blik had}{echter niets agressiefs}\\

\haiku{- Het is ontzettend,,.}{zei Charles maar hij had het}{gevoel dat hij loog}\\

\haiku{Van der Mark zag dat.}{hij streed met een moeilijkheid}{en bekortte die}\\

\haiku{de eenzaamheid zijn.}{kan en hoe weinig wij voor}{elkaar kunnen doen}\\

\haiku{wij zijn het alleen...}{w\`el geweest vijf jaar lang in}{het hart van Ang\`ele}\\

\subsection{Uit: Vijf versies van 'Vera'}

\haiku{- tenzij je het op.}{een heel                         bizondere}{wijze verantwoordt}\\

\haiku{Ik weet het niet, hoop.}{daarover binnenkort een}{groot stuk te schrijven}\\

\haiku{V\'o\'or het einde van.}{het jaar waren er 1200}{exemplaren verkocht}\\

\haiku{Vera overhandigt den-?}{man met de stofjas haar pas.}{Waar woont u}\\

\haiku{Zal ik mij moeten,;}{verkwanselen voor weinig}{geld voor veel geld}\\

\haiku{Waar loopt                       d\`eze,*?}{op uit        56  ~          waar}{loopt alles op uit}\\

\haiku{en                     hij schatte}{de soepele spanning der}{borsten waartusschen}\\

\haiku{De jongen trok met,;}{ruime krachtige slagen}{de boot door het meer}\\

\haiku{Daarbinnen leefden:}{zij    alsof dit eiland}{de wereld                     was}\\

\haiku{Het is alsof ik.}{een bron ben geworden}{of een fontein}\\

\haiku{Niemand van hen    .}{dorst te breken met een}{ontspoord resultaat}\\

\haiku{bitterheid stond     '.}{in zijn                     mond als hij dacht}{aan November18}\\

\haiku{Zij liep haastig, licht,.}{voorovergebogen dicht langs}{de huizen}\\

\haiku{hun armen stonden.}{op elkaars    schouders en}{maakten een                     brug}\\

\haiku{Berlijn was    een,.}{stad                     vol inventie vol}{wilskracht en energie}\\

\haiku{Hij was slank en                     .}{forsch en onweerstaan-}{baar-krachtig}\\

\haiku{de oude boomen.}{voor het    hotel in}{de Victoriastrasse}\\

\haiku{Dan nam zij zijn hoofd;}{in haar handen en streelde}{hem door zijn haar}\\

\haiku{Is het niet beter}{\'e\'en klein maar stellig geluk}{te bezitten}\\

\haiku{De trein remde en.}{reed knarsend de dreunende}{overkapping binnen}\\

\haiku{plotseling was zij.}{onzichtbaar    geworden}{in het gedrang}\\

\haiku{Plotseling riep zij,.}{den kellner betaalde en}{ging haastig weg}\\

\haiku{Zij riep een taxi, die.}{de straat afzakte en gaf}{haar                     hotel op}\\

\haiku{Och, dat had ik je.}{ook van    te voren wel}{kunnen voorspellen}\\

\haiku{wat heeft zij nu nog,(}{over nu zij haar    geloof}{heeft verloren}\\

\haiku{Het is misschien niet}{waar dat ik niet in het}{klooster gegaan ben}\\

\haiku{ik begrijp Judas,,,.}{die hem verried en Petrus}{die                     hem verried}\\

\haiku{Ik heb als dienstmeid,,.}{gewerkt in Dahlem in}{Hermsdorf in Gr\"unau}\\

\haiku{ik rekende op,;}{zijn wel-levendheid op}{zijn ridderlijkheid}\\

\haiku{Nu mag er niets dood,,;}{zijn en wij wij mogen ook}{nooit meer doodgaan}\\

\haiku{een 11                     droomloos [, [;}{en droomloos en 12 leven}{is en leven is}\\

\haiku{Zij kreunde, haar mond,.}{bleef gesloten ook haar}{oogen waren nu dicht}\\

\haiku{Maar hij had daarna, -:}{toen hij                     voelde dat dit}{gemis of liever}\\

\haiku{- Maar... ik doe toch niet.}{enkel uit heerschzucht het}{werk op het atelier}\\

\haiku{Misschien [ van Dreyer,,.}{d\'aarom althans niet boven}{de Moeder stellen}\\

\haiku{- h\`eldere pijn 15--.}{17                      Hoe zwak was zij en}{hoe klein was haar hart}\\

\haiku{7 - Ja ik ben moe,?}{en nu wil ik alleen}{zijn begrijp je dat}\\

\haiku{Ik [ - Ja, ik ben moe,,?}{en nu wil ik alleen zijn}{begrijp                     je dat}\\

\haiku{17Tekst C. Zie                             ,.}{Verantwoording blz. 19 van}{deze uitgave}\\

\haiku{ander woord                                                                                                                        12-}{13                                Het laken over haar}{schouder ademt heel zacht}\\

\haiku{en hij, na den nacht,.}{met Hedda voelde zich}{verjongd en ontsmet}\\

\haiku{] wijd en hij, na den,.}{nacht met Hedda                                     voelde}{zich als verjongd}\\

\haiku{88 c                                                                7-13 [...].}{zoo sterk contrasteerde}{machtige boeren}\\

\haiku{altijd {\textquoteleft}voor                                     het...{\textquoteright}- [...]...}{eerst                                                                                                                        2324                                Voor het}{eerst aarzelde hij}\\

\haiku{33                                z\'o\'o [ z\'oo                                                                         - [...}{110 c                                                                45                                dan}{het lieflijk verband}\\

\haiku{] simpele                                                                                                                        28-....{\textquoteright}.}{32                                Walter Hedda schuift}{de krant van zich af}\\

\haiku{overbodig                                                                                                                        24 ],}{was vergeleken was}{vergeleken}\\

\haiku{valsch omlijnd                                                                                                                        12,, -:}{binnen zou komen groot}{jong en krachtig OPM}\\

\haiku{- h\`eldere pijn het.}{zou temperen maar                                     toch}{tevens doorvlijmen}\\

\haiku{er ] het kan niet meer,,....}{hij is al veranderd}{er                                                                                                                        29                                mij}\\

\haiku{Toen zag                                                                                                                        33                                 ]:}{zichzelve zichzelf                                                                         138}{c Aan het slot OPM}\\

\section{H. Marsman en Simon Vestdijk}

\subsection{Uit: Heden ik, morgen gij}

\haiku{s Avonds hadden wij.}{elkander niet gezien of}{althans niet herkend}\\

\haiku{Waaraan ontleenen die?}{vlerken dat air behalve}{aan hun onnoozelheid}\\

\haiku{omdat het hunkert.}{naar krasse bewijzen van}{vriendschap en liefde}\\

\haiku{veel lucht (en vergeet:}{niet de monden waar die lucht}{uit is gekomen}\\

\haiku{- Ja, zei hij, vooral.}{in Huelva heb ik heel}{veel geluk gehad}\\

\haiku{Ik verheug me in,.}{je woede dat je dit weer}{van me slikken moet}\\

\haiku{dan kan je meteen....}{die watten voor nuttiger}{dingen gebruiken}\\

\haiku{Acht uur ongeveer.}{werd ik door stemmen uit een}{diepe slaap gewekt}\\

\haiku{Wij staan 's morgens.}{vroeg op en na het ontbijt}{gaat ieder zijn gang}\\

\haiku{Alleen zijn hoofd is,.}{gevonden door een jongen}{die zwom in de beek}\\

\haiku{Daarop gingen wij,.}{met ons vieren het veld in}{op zoek naar het lijk}\\

\haiku{Paul keek dan vragend,.}{naar Wevers in afwachting}{van diens beslissing}\\

\haiku{- Mijne heeren, ik!}{drink op de duizendste meid}{die ik gehad heb}\\

\haiku{Toch ben ik na het.}{verslag van je fuifje niet}{volkomen gerust}\\

\haiku{de geschiedenis,.}{met Annie is van later}{datum zooals je weet}\\

\haiku{blonde meisjes met,.}{een sentimenteel lachje}{dom en opgeprikt}\\

\haiku{Aan zijn manier van,.}{spreken hoorde ik dat zijn}{lippen droog moesten zijn}\\

\haiku{Ik begrijp nu ook;}{wel dat mijn vorige brief}{je geprikkeld heeft}\\

\haiku{De rivier slingert.}{rondom de stad en op \'e\'en}{plaats even er doorheen}\\

\haiku{{\textquoteleft}U hebt volkomen,,;}{gelijk Don Pedro hij werkt}{inderdaad te hard}\\

\haiku{Snellen overschat de.}{dood inderdaad omdat hij}{het leven overschat}\\

\haiku{Maar goed, wat geeft het,.}{Holland heeft zijn tijd en zijn}{beschaving gehad}\\

\haiku{zoo gelukkig als.}{een niet-bekeerde zich}{zelfs niet kan droomen}\\

\haiku{{\textquoteright} Maar Don Pedro, haar,,,:}{vraag negeerend zei nauwelijks}{ironisch tot Wevers}\\

\haiku{{\textquoteleft}Neen, wij slapen nog,{\textquoteright} {\textquoteleft}}{zei Nettie lachend en greep}{hartelijk zijn hand.}\\

\haiku{Ik kan je zeggen,,.}{v. M. dat het mij nu toch}{een opluchting is}\\

\haiku{Wevers' verdere.}{ontwikkeling wacht op mijn}{welversneden pen}\\

\haiku{Zonder 't misschien.}{te willen dring je je in}{mijn intimiteit}\\

\haiku{dag meneer Wevers,,.}{mag ik uw schoenen likken}{of iets dergelijks}\\

\haiku{weet je, dat je me,,?}{die avond verleden jaar had}{kunnen vermoorden}\\

\haiku{{\textquoteright} Maar waarom Nettie,.}{daar nu ineens heen wil is}{mij niet duidelijk}\\

\haiku{Maar bewijst het niet,?}{iets dat ik nu juist deze}{stemmingen doormaak}\\

\haiku{ik zal maar blijven.}{en trachten de zaak nog in}{orde te krijgen}\\

\haiku{- Ik dacht wel dat je,,}{er niets van begrijpen zou}{zei ze iets zachter}\\

\haiku{Het spijt me, maar nu.}{zal ik je illusie toch}{moeten verstoren}\\

\haiku{Begrijp je nu dat?}{ik toch nog geloof dat je}{verliefd op hem bent}\\

\haiku{Intusschen is het,}{heel goed mogelijk dat hij}{mediumiek niet zoo}\\

\haiku{eerst liet ze me vijf;}{minuten in datzelfde}{kamertje wachten}\\

\haiku{Op een avond zaten.}{Wevers en ik bij elkaar}{in een Haagsch caf\'e}\\

\haiku{Ik kon rekenen;}{op een betrekking aan een}{werf in Rotterdam}\\

\haiku{Ze liep traag en met.}{iets van geremdheid zooals men}{dat in droomen heeft}\\

\haiku{dat ik op dit stuk,.}{iets te verdedigen heb}{of te verraden}\\

\haiku{Als ik mij goed in,.}{die mogelijkheid verdiep}{word ik wanhopig}\\

\haiku{{\textquoteleft}Feitelijkheden.}{over hun leven heeft Wevers}{mij weinig verteld}\\

\haiku{hield zij van hem of,,?}{van M. hield deze van haar}{hield Wevers van haar}\\

\haiku{Ik denk nog altijd,,}{aan haar zij leeft in m{\`\i}j ik}{draag haar in mij om}\\

\haiku{Kom spoedig, lieve, -;}{Wevers maar natuurlijk met}{andere woorden}\\

\haiku{Walging voelde ik,,:}{niet alleen ontgoocheling}{en medelijden}\\

\haiku{wat mij op Wevers'.}{kamer te wachten stond voor}{ik in zou grijpen}\\

\haiku{Ik hoop, dat je je.}{vanavond geamuseerd hebt}{onder mijn leiding}\\

\haiku{Maar ik had er toch.}{blijkbaar Praag bij noodig om tot}{schrijven te komen}\\

\section{Marita Mathijsen}

\subsection{Uit: Seks in Limburg. Gevolgd door dezelfde tekst in het Belfelds}

\haiku{Nee,  dat wist ik,.}{niet en ik vond het toen ook}{geen plausibel idee}\\

\haiku{als je de ene plek.}{dichtplakt barst die wel op een}{andere plaats open}\\

\haiku{t gedeurd had, en.}{of ich drek \'ongerop waas}{k\'omme te ligge}\\

\haiku{Netuurlik goof \'os ',}{mamt kink de bors aevel}{altied zoe\"e det weej}\\

\haiku{Op 'n gegaeve}{moment beg\'os ich mich z\"org}{te make euver}\\

\haiku{n plasgaetje en ' '.}{n poepgaetje ouch nagn}{anger gaetje had}\\

\haiku{As ich d'r zeker,}{van waas det nemes binne}{k\'os k\'omme deej ich}\\

\haiku{Weej krege ze flink.}{euver \'os klungels en weej}{woorte oetgehuuerd}\\

\haiku{zoe\"e det meugelik,}{de h\`elf van mien toete te}{zeen ware gewaes}\\

\haiku{Wie ich  d'r mei,.}{te make kreeg waas ich al}{verhoes oet Limburg}\\

\haiku{Aevel ich b\"on d'r:}{ouch hie\"el zeker van det}{d'r ouch bedoeld woort}\\

\section{Justus van Maurik}

\subsection{Uit: Burgerluidjes}

\haiku{Eindelijk op een ';}{morgen zag ik iemand voor}{t venster zitten}\\

\haiku{Een groote stoel stond bij,;}{de tafel en in dien stoel}{zat een oude vrouw}\\

\haiku{ik hoop niets meer, ik, -;}{verwacht niets meer ik zal hier}{wel eenmaal sterven}\\

\haiku{zij had weinig, maar.}{ze wist het weinige wat}{ze had te waardeeren}\\

\haiku{Hendrik moest een vrouw; '.}{zoekenkwou nog zoo graag mijn}{kleinkinderen zien}\\

\haiku{Ja, zij had hem lief;}{uit al de volheid van haar}{onbedorven hart}\\

\haiku{Toen zij alleen was,.}{met haar moeder was zij naast}{haar stoel neergeknield}\\

\haiku{{\textquoteleft}Heb je er nog nooit,?}{over gedacht wat we met hen}{beginnen moeten}\\

\haiku{Ik zal die oudjes, -....{\textquoteright} {\textquoteleft},...?}{in Godsnaam maar mee laten}{varen maarNu maar}\\

\haiku{{\textquoteright} klonk op eens de stem.}{van den kindschen man uit de}{andere kamer}\\

\haiku{vader en moeder,{\textquoteright} -,.}{riep Truitje en zonk in de}{kussens terug}\\

\haiku{er is dan toch nog,,.}{iemand die mij liefheeft die}{ik liefhebben mag}\\

\haiku{langzaam opdraaien,,:}{en dan hou je aan totdat}{meneer Hopkamp zegt}\\

\haiku{Ziehier in korte.}{trekken den inhoud van Dr.}{Penners eersteling}\\

\haiku{Eindelijk komt het, '.}{oogenblik datt uur van}{middernacht zal slaan}\\

\haiku{Dat vindt Penner ook,:}{en daarom zwijgt hij totdat}{de directeur zegt}\\

\haiku{maar als hij 't me,.}{al te bont maakt loopt hij een}{pak slaag van mij op}\\

\haiku{Zoodoende staat hij, ':}{alleen in zijn oordeel en}{dan zegtt publiek}\\

\haiku{Zeer fideel slaat Hart:{\textquoteleft}!}{zijn arm om Andr\'ees taille}{en neuriet  Ach}\\

\haiku{Elke gedachte,,;}{die Willem voedde deelde}{hij met zijn moeder}\\

\haiku{'k Zou me niet goed,.}{kunnen voorstellen dat je}{van me weg zoudt gaan}\\

\haiku{als vastgenageld!}{bleef zij op haar stoel zitten}{en hoorde alles}\\

\haiku{Toen omvatte zij;}{haar kind met inspanning van}{haar laatste krachten}\\

\haiku{{\textquoteright} - Bram zwijgt even, doet eeu:}{paar flinke trekken aan zijn}{pijp en vervolgt dan}\\

\haiku{{\textquoteright} {\textquoteleft}En wij gaan reizen(!){\textquoteright}.}{door de wollek-khik}{gilt de andere}\\

\haiku{omdat je weet, dat,.}{je ouwe niet naar bed gaat}{v\'o\'or jij binnen bent}\\

\haiku{- Buiten klettert de '.}{regen op het zinken dak}{vant wachthuisje}\\

\haiku{Alleen datgene,,.}{wat opvalt of bijzonder}{is wordt opgemerkt}\\

\haiku{{\textquoteleft}'t Is mooi weer voor, -.}{een inbreker zoo'n nacht is}{geld waard voor een boef}\\

\haiku{{\textquoteright} zegt Bram, die de vrouw.}{met beide handen onder}{de armen vasthoudt}\\

\haiku{{\textquoteleft}Die lap zal 't hem,.}{net doen voor een gordijn als}{hij lang genoeg is}\\

\haiku{dat is dan toch een, -.}{heele lastpost voor je zoo'n}{zwak schepsel in huis}\\

\haiku{Dan komt de schimmel, -!}{niet in je krullebol dat}{zou zonde wezen}\\

\haiku{{\textquoteright} en terwijl zij dit.}{vraagt buigt zij zich voorover met}{haar oor naar zijn mond}\\

\haiku{Komt hij thuis, dan slaapt.}{hij uit en kan op avontuur}{morgen weer werken}\\

\haiku{'t Is maar voor de. '}{schandaligheid onder weg}{en voor de buren}\\

\haiku{'t Was alleen maar, ':}{een Baron diet zwaar op}{z'n zenuwen had}\\

\haiku{Ik ben het met u,.}{eens al de nieuwigheid is}{geen verbetering}\\

\haiku{Zij nam het glas, dat,.}{op mijn beddetafeltje}{stond en rook er aan}\\

\haiku{{\textquoteright} zei ik, blij dat er,.}{een oogenblik was dat haar}{woordenvloed ophield}\\

\haiku{{\textquoteright} {\textquoteleft}Ja, spot er maar mee,,,:}{meneer maar het is een goeie}{raad dien ik je geef}\\

\haiku{drie klontjes witte....}{suiker met een droppel of}{zes Haarlemmerolie}\\

\haiku{Uw\'e heeft mij een dienst;}{gedaan en daarom heb ik}{voor u ook wat over}\\

\haiku{zij naderde mijn,:}{bed lei zachtkens haar hand op}{mijn voorhoofd en zei}\\

\haiku{{\textquoteright} riep het goede mensch,.}{met een ontdaan gelaat toen}{zij in het bed keek}\\

\haiku{{\textquoteright} Glimlachend dacht ik}{na over het sic transit en}{richtte mij zoo goed}\\

\haiku{Nu eens trilde ik,;}{op mijn beenen als een door den}{wind bewogen riet}\\

\haiku{Ik zal zoo vrij zijn,.}{om dan ook dit eindje voor}{hem over te laten}\\

\haiku{dat is naar mijne.}{meening een kort begrip van}{de geneeskunde}\\

\haiku{{\textquoteright} {\textquoteleft}Neem je pillen trouw,.}{in lees het boek Job en laat}{de rest aan mij over}\\

\haiku{{\textquoteright} {\textquoteleft}Goeie hemel! 't schijnt;}{wel alsof de heele buurt}{weet dat ik ziek ben}\\

\haiku{{\textquoteleft}Als uw\'e dan partoet,;}{uit wil moet je maar in je}{sjamperloepie gaan}\\

\haiku{Ik voelde, dat ik,:}{wit werd van kwaadheid maar ik}{hield mij kalm en zei}\\

\haiku{{\textquoteright} Een tweede slag op:}{den kant van de tafel deed}{mij verschrikt vragen}\\

\haiku{Anders kon jij net...{\textquoteright} {\textquoteleft},!}{als ieder ander voor mijn}{part naar denMaar oom}\\

\haiku{Hij ging vlak voor mij,:}{staan streek zijn knevels op en}{bulderde mij toe}\\

\haiku{Zij zag in hem een:}{bondgenoot en voegde hem}{daarom halfluid toe}\\

\haiku{{\textquoteright} waagde ik in het, {\textquoteleft}!}{midden te brengendat is}{toch puur bijgeloof}\\

\haiku{hij dweept er mee en {\textquotedblleft}{\textquotedblright}.}{daarom he\`eft hij zijn huis ook}{villa Bay genoemd}\\

\haiku{'k geloof dat de ';}{commandants avonds wel wat}{veel aan de wieg stoot}\\

\subsection{Uit: Op reis en thuis}

\haiku{'k habbe 'm van '.}{t versoepe gered en}{noe is d'r so trouw}\\

\haiku{'t Volk krijgt aan boord!}{twee maal per dag een oorlam}{en daarmee basta}\\

\haiku{Zie je, meneer, dat.}{is zoowat schering en inslag}{van z'n redenasies}\\

\haiku{*** Op 't achterdek {\textquoteleft}{\textquoteright};}{klinkt vroolijk het orkest van}{de gloeiende pook}\\

\haiku{Als hij er maar bij,.}{was kon je zeker zijn dat}{een fuif goed afliep}\\

\haiku{Zuiver en helder.}{klonk het eenvoudige lied}{in den stillen nacht}\\

\haiku{- Dat is een van de,,;}{mooiste melodi\"en die}{ik ken zei hij zacht}\\

\haiku{Toen de aanvoerder,}{gevallen was kwamen al}{de apen naar hem toe}\\

\haiku{Ik maakte gebruik '.}{van de gelegenheid en}{koost hazenpad}\\

\haiku{Een nieuwe, zeker,.}{door hem uitgevonden term}{voor verliefd te zijn}\\

\haiku{de dame begint.}{zachtjes te kuchen en wrijft}{nu en dan haar oogen}\\

\haiku{- Dat geveugel an,.}{m'n lijf kan ik niet velen}{laat dat maar waaien}\\

\haiku{ik heb zoo nu en:}{dan wat van u gelezen}{en daarbij dacht ik}\\

\haiku{Sina's vader, dit,.}{wou ik u straks vertellen}{was niets inhalig}\\

\haiku{Volgens de wet was,:}{hij haar erfgenaam maar hij}{kwam bij me en zei}\\

\haiku{Aan boord waren reeds;}{de meeste passagiers in}{in zalige rust}\\

\haiku{dan moet hij voor mij,....}{voor een schuifje werken dan}{helpen we mekaar}\\

\haiku{J. Bruin, Beeldhouwer,,,.}{3 {\texttimes} schellen toonde hem}{dat hij terecht was}\\

\haiku{- maar, ziet u, d'r is '....}{tegenwoordig al niet veel}{te doen int vak}\\

\haiku{bedaard aan, we zijn, ';}{nog zoo ver niet laat maar eens}{eerstt model zien}\\

\haiku{- Daar zijn we nu, - zei,:}{Capelli zijn hoed en stok}{op een stoel leggend}\\

\haiku{En d\`en, om je de,,.}{waerheid te zeggen z'n}{neus die w\`es \'enders}\\

\haiku{- D'r is van voren.}{wat afgehaald en aan de}{eene zij bijgebracht}\\

\haiku{{\textquoteright} - Zie je, juffrouw Bruin, '.}{dat is nut r\'esum\'e}{van al de opinies}\\

\haiku{Hij keek somber v\'o\'or:}{zich en zei met een weinig}{gemaakte tragiek}\\

\haiku{Mijn familie is.}{nog bijna geheel en al}{ten mijnen laste}\\

\haiku{Onlangs ben ik zes;}{weken op Tourn\'ee geweest}{in de provincie}\\

\haiku{wat een ordinair:.}{gezicht wat kijkt die kerel}{knorrig als hij gaapt}\\

\haiku{'t is me alsof -.....}{ik draai ik dommel in en}{ik droom van mooi we\^er}\\

\haiku{De voerman heeft een,.}{blauwen kiel aan die zwart ziet}{door de nattigheid}\\

\haiku{Daar nadert iemand,, '.}{ik herken hemt is een}{kurgast evenals ik}\\

\haiku{Hun vesten hangen.}{open en de veelkleurige}{bretels zijn zichtbaar}\\

\haiku{{\textquoteright} was de eenig troost, die:}{hij van haar kreeg en vinnig}{voegde zij erbij}\\

\haiku{toch moet hij {\textquoteleft}voor das.}{fraumensch ergens in abendbrot}{zoesammenblasen}\\

\haiku{Enfin! 't doet me ' '.}{toch pleizier datkreis weer}{Hollandsch kan spreken}\\

\haiku{dat's waar ook, hoe?}{heet de burgemeester van}{Amsterdam ook weer}\\

\haiku{k ontmoette hem.}{een jaar of wat geleden}{ook bij de courses}\\

\haiku{Schuins d'r over kocht ik}{altijd mijn sigaren in}{een klein winkeltje}\\

\haiku{, meneer, tegen die.}{dingen kunnen de Fransche}{patissiers niet aan}\\

\subsection{Uit: Oude kennissen}

\haiku{- denk je om de leege....}{flesschen van de bessensap}{te laten terug}\\

\haiku{gossiemijne, neem, ';}{me niet kwalijk ik kont}{heusch niet helpen}\\

\haiku{ik ben jaren bij, '}{de van Palens over den vloer}{gekomenk Was}\\

\haiku{Ja, ja, en dan zoo, ' '.}{een en ander meert is}{toch nooitt eigen}\\

\haiku{en tien minuten.}{later waren ze al bij}{onzen lieven Heer}\\

\haiku{- Wil je die Cantenac,.}{van Burgers niet eens proeven}{voor de aardigheid}\\

\haiku{ga ik me niet gauw,;}{te waag vooral niet als ik}{de firma niet ken}\\

\haiku{Gul {\textendash} Uche! - den - neen, maar! -?}{dat is onmogelijk Vindt}{je hem dan te duur}\\

\haiku{Zeg Meijer, nu kun.}{je meteen kennis maken}{met je concurrent}\\

\haiku{zegt de voorzitter,, -!}{met welgevallen zich zelf}{hoorend Mijne heeren}\\

\haiku{- Onze taal is rijk, -! -! -!}{aan klanken die klinken hum}{klanken vol klank hum}\\

\haiku{- en ik zou hier voor -!}{uw pleizier over de vreugd staan}{schetteren merci}\\

\haiku{zijn er intusschen,?}{ook nog stukken of brieven}{gekomen Bode}\\

\haiku{{\textquoteright} {\textquoteleft}Haal dadelijk den,.}{agent die hier altijd aan den}{hoek van de straat staat}\\

\haiku{ge neemt afscheid van,, -!}{een goede lieve vriendin}{die ver ver weg gaat}\\

\haiku{Wie kent ze niet, wie -?}{heeft ze niet lief en wie is}{er niet benauwd voor}\\

\haiku{wat 'n mooie meid was -, -!}{dat mooi meneeren een engel}{om te stelen}\\

\haiku{je herinnert 't,.}{je wel je hebt er laatst nog}{zoo over geroepen}\\

\haiku{Voor hem, van hem, om,,,.}{hem is alles zijn wil is}{wet zijn ik alles}\\

\haiku{heeft ze die w\`el, dan.}{is ook ook haar stervensuur}{nog een marteling}\\

\haiku{Er is veel geestkracht - ';}{toe noodig om zacht te zijnt}{is het deel der vrouw}\\

\haiku{er zich al gauw mee -.}{wennen ik doe alles met}{Jantje op mijn arm}\\

\haiku{- Wat zij vandaag weer,,;}{heeft begrijp ik niet maar ze}{is uit haar humeur}\\

\haiku{- Ik zou wel uit de,? ......................}{zaken willen want waar werk}{ik eigenlijk voor}\\

\haiku{Dat begint hem toch;}{te h{\`\i}nderen en hij zoekt}{zich te verstrooien}\\

\haiku{'t Is maar voor de,.}{meiden die moeten sluiten}{en op tijd naar bed}\\

\haiku{Honderd maal op een.}{dag hoort men tegenwoordig}{deze verzuchting}\\

\haiku{wel de moeite waard,}{om n\'og eens een schoonpapa}{te vereeuwigen}\\

\haiku{Om te beginnen,,!}{dan vergun mij dat ik mijn}{vriend Pluim even voorstel}\\

\haiku{je moest dat liever - ', '.}{niet doent hindert me ik}{kant niet velen}\\

\haiku{Sophie is een......}{hoogst fatsoenlijke vrouw en}{ik verbied je om}\\

\haiku{Jaloersche mannen,.}{worden er ook gevonden}{maar meer sporadisch}\\

\haiku{'k zou toch wel eens,.}{even willen kijken hoe die}{baker er uitziet}\\

\haiku{Voordat zij 't zelf.}{weet is ze teruggegaan}{en heeft aangeklopt}\\

\haiku{- Honnig! - Nou Mevrouw! ';}{t ziet er dan maar reintjes}{en illegant uit}\\

\haiku{Wij menschen hebben, ';}{er al gauw een kijkkie op}{watt worden zal}\\

\haiku{Hij heeft grooten lust,?}{om maar dadelijk weer heen}{te gaan maar waarheen}\\

\haiku{De kantoorjas, met ',;}{t stof van den dag er op}{wordt weggehangen}\\

\haiku{dan zal 't eten je,,,!}{des te beter smaken Komt}{jongens aan tafel}\\

\haiku{je zult papa  ,;}{knorrig maken Vader denkt}{er echter niet aan}\\

\haiku{Als hij vertrokken,;}{is gebruikt zij haar ontbijt}{en haar twaalf-uurtje}\\

\haiku{ze gaven ieder, -}{het zijne en hielden zelf}{een kleinigheid over}\\

\haiku{Pardon - ik zag zoo,...;}{dadelijk niet dat U is}{alles wat hij uit}\\

\haiku{- Nu ben je heusch,,!}{een lieve man dat je me}{helpt opruimen hoor}\\

\haiku{zoodat de ruiter zijn - -.}{arm de sabel ontbreekt we\^er}{dreigend vooruitstreekt}\\

\haiku{Ik was dol op 'r,;}{en ik heb altijd geloofd}{dat zij mij ook mocht}\\

\haiku{Krampachtig knelde:}{hij zijn handen om zijn hoed}{en heesch zei hij}\\

\haiku{Als ze 't op d'r ' '.}{heupen had kon niemandt}{metr uithouwen}\\

\subsection{Uit: Stille menschen}

\haiku{Hei jij niet grappies,!}{verteld an de ouwe man}{terwijl ie werkte}\\

\haiku{- dat's bloed voor bloed,,!}{dat heb je aan Berbertje}{verdiend valsche hond}\\

\haiku{- Al ben je ook nog,...!}{zoo goedkoop ze deuge niet}{ze deu-eugen niet}\\

\haiku{Berbertje zei dat ' '.}{t z\'o\'o was en daarom zou}{t wel z\'o\'o wezen}\\

\haiku{zorg jij maar dat je,.}{goed en mooi werk maakt dan zorg}{ik voor klandisie}\\

\haiku{{\textquoteright} Te Amsterdam zou.}{zij een ruimer veld vinden}{voor haar werkzaamheid}\\

\haiku{- dat ik ook dood was - ' - '!}{God gaftk heb toch niets}{meer op de wereld}\\

\haiku{mijn vrouw was dan al ',,!}{n buitengewoon mensch zoo'n}{geestige goeie ziel}\\

\haiku{- Ben je gek, dikke,,!}{mogol wat verbeeld jij je}{wel oud gedierte}\\

\haiku{Denk je misschien dat.}{ik voor jou borsteltjes zal}{gaan zitten maken}\\

\haiku{Oome Daan richtte,:}{zich een weinig op en vroeg}{haastig verwijtend}\\

\haiku{ik had er niet meer,.}{naar omgekeken evenmin}{als naar mijn model}\\

\haiku{{\textquotedblleft}juffrouw{\textquotedblright} mag je d'r,{\textquoteright}.}{wel aflaten riep schamper}{de wijnhuishoudster}\\

\haiku{- Dat's niet waar - en -?}{je hoed vol deuken wat is}{er met je gebeurd}\\

\haiku{- - Ze verkochten ze - -.}{d\'a\'ar ten voordeele van van de}{kamerontbinding}\\

\haiku{{\textquoteleft}U is hier zeker,,?}{verkeerd meneer of wenscht}{u mij te spreken}\\

\haiku{nu begon ik te,;}{begrijpen dat ik te doen}{had met een slimmerd}\\

\haiku{- Welnu, dan moeten '.}{de goeient maar met de}{kwaaien ontgelden}\\

\haiku{Een goeie geeseling.}{of een brandmerk is misschien}{ook al voldoende}\\

\haiku{je ziet er toch niet,.}{naar uit dacht ik en nam den}{oude beter op}\\

\haiku{En Brammetje, die ',:}{zich mett geval amuseert}{zingt eensklaps solo}\\

\haiku{op 't portaal niest, '.}{hij een paar maal wantt is}{er koud en tochtig}\\

\haiku{- Daar is de ketel;}{en nu weet ik meteen waar}{die tocht vandaan komt}\\

\haiku{En 't kind schreeuwt zich -,;}{een ongeluk ik begrijp}{er niets van niets helpt}\\

\haiku{hoe komt zoo'n kind nu -,.}{ineens aan influenza}{hoor hij hoest ook weer}\\

\haiku{Zij zag zijn profiel;}{scherp afsteken tegen het}{verlichte venster}\\

\haiku{{\textquoteright} {\textquoteleft}U kent me niet meer,,;}{meneer Bernard maar dat wil}{ik wel aannemen}\\

\haiku{hij verzon voor mij,,:}{allerlei ik lachte d'r}{wel reis om en zei}\\

\haiku{{\textquoteleft}'k Moest kijken of, ';}{de juffrouw al weg wast}{eten staat op tafel}\\

\haiku{{\textquoteleft}Lieber Bernard, du,!}{bist ein so guter Mensch ein}{so pr\"achtiger Kerl}\\

\haiku{{\textquoteleft}Als ik haar niet had,,!}{stond ik hier niet voor u dan}{was ik allang weg}\\

\haiku{hij rookt ze van vijf, - -.}{om een dubbeltje dat is}{toch geen overdaad Aug}\\

\haiku{hij weet zelfs niet, dat!}{hij zonder overjas de deur}{is uitgeloopen}\\

\haiku{'t Bleef toch nog licht, '.}{in de kamer want de maan}{scheen doort venster}\\

\haiku{ik, daar heb 'k al,.}{reis een paar maal voor in de}{nor gezeten dankie}\\

\haiku{k heb Toon niet eens - '....}{meer kunnen inschenken hij}{zit opn droogie}\\

\haiku{Ik nam dadelijk,?}{men duiten en Klaas keerde}{zijn zak om niet waar}\\

\haiku{- En, vulde Klaas weer,;}{aan toen stonden we op straat}{zonder dubbeltjes}\\

\haiku{zoo na aan 't hart, ',;}{k word er ziek van als ik}{ze niet spreken mag}\\

\haiku{toen jij al lang sliep,...}{heb ik nog een pak gemaakt}{van die boeken en}\\

\haiku{Al 't water in,?}{die kist met boeken wat doe}{jelui aan die kom}\\

\haiku{Plof! - De linnenkast.}{glipt uit de leng en barst op}{straat uit elkander}\\

\haiku{Tegen tien uur houdt.}{de kruiersbaas met zijn knechts}{een half uur schafttijd}\\

\haiku{In 't huishouden! -;}{een kleine handreiking zoo}{iets doet hij gaarne}\\

\haiku{Langzamerhand weet;}{hij juist hoeveel spinasie}{of wittekool kost}\\

\haiku{Na 't diner moet,.}{hij zijn dutje doen dat is}{levensbehoefte}\\

\subsection{Uit: Toen ik nog jong was}

\haiku{je zult zoo aanstonds.}{je adem wel noodig hebben op}{de steile trappen}\\

\haiku{Kijk nu maar eens goed,, '}{rond maar hou je pet goed vast}{wantt waait nog al.}\\

\haiku{- D\'a\'ar was eenmaal de,....}{Buitensingel daar is nu}{de Nassaukade}\\

\haiku{ze dronken d'r ook -}{wel ris een roemertje wijn}{of wat anders en}\\

\haiku{U begrijpt, daardoor,}{snapten ze eindelijk dat}{ze gefopt waren}\\

\haiku{En de schepen, waar,;}{ze dat goed mee aanbrachten}{le{\"\i}en in de gracht}\\

\haiku{Je had daar aan de,.}{Ouwe-brug tegen de}{Korenbeurs gebouwd}\\

\haiku{Het Damrak, Oude.}{brug en Korenbeurs met de}{kleine winkeltjes}\\

\haiku{Ze rollen ze over:}{een uitgelegde plank aan}{wal met een vroolijk}\\

\haiku{Te deksel, Teun, wat. ' '.}{maek je ze blauwt Laikt}{t firmement wel}\\

\haiku{zoo'n kuip doodstil staan,, '.}{dan komt er een dik vel op}{netn stuk le\^er}\\

\haiku{Jaco is reeds meer,.}{dan anderhalve eeuw dood}{want hij werd in Ao}\\

\haiku{ze heit 'r nou een ' (), '.}{metn kriekiebochel dat}{s ook nog een tref}\\

\haiku{hij leest alles wat - '}{ie te pakke kan krijge}{metn ouwe krant}\\

\haiku{Zoolang een van hen, '.}{nog geld of drank had hadden}{zet allemaal}\\

\haiku{Blijf met je poote van,, '....}{de deur ze zelle wel ope}{doen ast tijd is}\\

\haiku{- Ik heb nog 'n neef,,?}{in Nieuwjork die ken jij ook}{wel niewaar Manus}\\

\haiku{- 't Is waarachtig,,!}{of ze binne ruike dat}{ze der ankomme}\\

\haiku{Daarom tikt hij den.}{dikken man zachtjes aan en}{herhaalt zijn verzoek}\\

\haiku{- Broeder en zusters,,.}{geliefde vrienden laat ons}{samen zingen}\\

\haiku{Luid verheft hij zijn:}{stem en met een akelige}{neusklank zingt hij}\\

\haiku{Ischt daar iemand,,,}{die mijn nicht bejreipt das}{hij frage ich soll}\\

\haiku{- Profeet, ik kan u,.}{niet erg goed zien de kachel}{staat zoo in den weg}\\

\haiku{- Daar heb je nummer,,.... -!}{drie Trui die zal ook wel over}{de kachel Profeet}\\

\haiku{{\textquoteleft}Sister, come to,!}{the front der heilige Jeist}{sal doerch you schpreeken}\\

\haiku{we zullen eerst die,.... '....}{vuile boel afwasschen laat}{je oogris kijken}\\

\haiku{En ik wil niks met ' - ';}{m te make hebbe ik}{ben schuchter vanm}\\

\haiku{Jan d'r naar toe en,:}{op z'n luister en toen ie}{terugkwam zei ik}\\

\haiku{en als ik ze weer, ',?}{snap neem ik ze mee naart}{bureau versta je}\\

\haiku{'t Heele Hol ruikt -.}{naar gehak maar wij blijve}{d'r nuchtere van}\\

\haiku{Iedereen had in {\textquoteleft}{\textquoteright}.}{datGemeenebest zoo'n soort}{van liefhebberij}\\

\haiku{All\'e\'en is toch ook,....}{maar all\'e\'en in gezelschap}{geniet je dubbel}\\

\haiku{- Als je soms dorst krijgt,,!}{steek je zoo'n partje in je}{mond dat frischt zoo op}\\

\haiku{Zou je denke, dat.}{ik voor juillie plezier hier}{sta te blauwbekke}\\

\haiku{'t Was me, alsof.}{ik een engelenstem uit}{den Hemel hoorde}\\

\haiku{Piet was destijds nog, - '?}{al erg hoog op de beenen geef}{jen zoet slokkie}\\

\haiku{of denk jij soms, dat ',?}{k voor jou pet mijn riem zal}{weggooie mooie jonge}\\

\haiku{maar toen zaten we.}{heel deftig in het eerste}{Amphitheater}\\

\haiku{{\textquoteleft}Wandel met den Heer{\textquoteright} -,?}{ik wandel liever met een}{knappe meid snap je}\\

\haiku{De bezoeker wipt ':}{t restantje uit zijn glas}{naar binnen en vraagt}\\

\haiku{- Ga eerst de cente,...}{hale ze hebbe vandaag}{zeker veel verkocht}\\

\haiku{Jetje Meloen slaat hard:}{en vinnig met haar stok op}{de zinken toonbank}\\

\haiku{... scheldt de meid, en zich:}{knorrig omdraaiend zegt ze}{luid tot haar vriendin}\\

\haiku{Mokkie is een van.}{de typigste figuren}{uit de Duvelshoek}\\

\haiku{Schouderophalend,:}{ziet hij den ander lachend}{aan en zegt kalmpjes}\\

\haiku{Keizersgragt No. 380,, -:}{geeft niet aan de deur w\`el aan}{blindemanne of}\\

\haiku{Doch 't gebeurde,.}{meermalen dat hij er geen}{herberg kon vinden}\\

\haiku{- Dat is een cadeau,.}{van een Italiaander die}{hier loosjies heit gehad}\\

\haiku{'t Is een wenk, een.}{waarschuwing ten nutte van}{mijn collegas}\\

\haiku{daar zat de kruier.}{al op me te wachten met}{een nijdig gezicht}\\

\haiku{- Dat waren zoo de,.}{gewone scheldnamen die}{je naar je hoofd kreeg}\\

\haiku{'t Zat er aan bij;}{die lui en alles was van}{de bovenste plank}\\

\haiku{- Gerrit zouen we?}{er niet een halfie ouwe}{klare op gieten}\\

\haiku{Gerrit, neem dat kind, '!}{die nare steek toch aft}{is God verzoeken}\\

\haiku{Ik werd er compleet,.}{ziek van want ik maakte me}{voortdurend driftig}\\

\haiku{- kleed je aan en zet,;}{je pet op we gaan van avond}{naar de komedie}\\

\haiku{- Karel de Groote met, '!}{een pijp en lezendt was}{waarachtig komisch}\\

\haiku{Nu is Veltman 83.}{jaar en treedt sedert eenige}{jaren niet meer op}\\

\haiku{zijn bewegingen.}{waren volstrekt niet die van}{een hoogbejaarde}\\

\haiku{- Veltman, z\'o\'o moet    ';}{Veltman als Gijsbreght.jijt}{ook trachten te doen}\\

\haiku{We bleven trouw bij.}{mekaar en dit hinderde}{Duport geweldig}\\

\haiku{Het kamertje van.}{een waschmeisje{\textquoteright}.man met}{geen gezicht meer zien}\\

\haiku{'t Was {\textquoteleft}de uitgang{\textquoteright}.}{voor de Amsterdammers en}{voor de buitenlui}\\

\haiku{Dat het publiek bij,. '}{z'n entr\'ee vertering kreeg}{was dus bepaald noodig}\\

\haiku{Hij is feitelijk,.}{dood zoodra hij niet meer}{op de planken staat}\\

\haiku{we binne vijftien,.}{jaar getrouwd en ze heit nooit}{getwijfeld meheer}\\

\haiku{Ik wou uwe vrage,, ',}{zou je niet denke dat ik}{t best doe er mee}\\

\haiku{- Och, schei uit met je.... -!}{moffinnen gezeur Ick doe}{toch alles voor jou}\\

\haiku{zij liep terug en,.}{zette een vervaarlijke}{keel op hulp roepend}\\

\haiku{ze verdwijnen niet,.}{doch verplaatsen zich als men}{hun nesten verstoord}\\

\subsection{Uit: Van allerlei slag}

\haiku{Grootvader vraagt, of;}{u eventjes een knoop aan zijn}{overjas wil zetten}\\

\haiku{Haar gezicht, misschien,,.}{eenmaal schoon is thans mager}{vervallen en beenig}\\

\haiku{Maar, alle gekheid,,;}{op een stokje vertel me}{eens wat er aan scheelt}\\

\haiku{'k heb 't destijds, '.}{willen doen maart is bij}{willen gebleven}\\

\haiku{en neem je 't niet,.}{dan beschouw ik de vriendschap}{tusschen ons als uit}\\

\haiku{Is 't niet een traan,,:}{die hij steelswijze afwischt}{terwijl hij antwoordt}\\

\haiku{{\textquoteright} {\textquoteleft}Dat zeit grootvader,{\textquoteright} -.}{ook en opnieuw begint de}{kleine te hoesten}\\

\haiku{{\textquoteleft}'k Hoop morgen na!}{de koffie een visite}{te komen maken}\\

\haiku{Jongens, Wilders, hij,;}{is zoo wel-off zooals de}{Engelschen zeggen}\\

\haiku{Is 't geen schande,?}{zoo'n ouwen stakkerd aan zijn}{lot over te laten}\\

\haiku{Ze is zooveel als - -! - '?}{president van die nou hm}{hoe heett ook weer}\\

\haiku{ge geeft alleen de.}{bagatel van 20 centen}{of twie dubbelkens}\\

\haiku{Pas nu maar op de,.}{kleintjes dan blijft er wat over}{voor den ouden dag}\\

\haiku{Ze maken geloof,,{\textquoteright}.}{ik pret over ons oudjes voegt}{hij er zachtjes bij}\\

\haiku{{\textquoteright} Mina en Bertha zien.}{eerst elkander en daarna}{Sarah vragend aan}\\

\haiku{onwillekeurig,:}{sloeg zij de oogen neder en}{toen hij zachtkens vroeg}\\

\haiku{'k laat hem in den,.}{brand zitten totdat we hier}{in de kamer zijn}\\

\haiku{Kom hier, laat ik je,.}{nog eens extra bedanken}{en jou ook Mientje}\\

\haiku{{\textquoteright} {\textquoteleft}Wel, uw maar sluit mijns,;}{inziens in dat u niet tot}{kiezen kan komen}\\

\haiku{Ze voelde iets, dat,:}{als een steek door haar hart ging}{toen hij vervolgde}\\

\haiku{{\textquoteleft}Houd een oogje in ',{\textquoteright};}{t zeil sprak ze bij't afscheid}{nemen tot Mina}\\

\haiku{{\textquoteright} {\textquoteleft}Nu, goed dan, Sarah,,.}{ga mee naar boven maar loop}{zachtjes op de trap}\\

\haiku{{\textquoteleft}Dat 's bijna net,,.}{zooals den vorigen keer maar}{toch veel veel mooier}\\

\haiku{maar ik kan 't niet,,:}{helpen dat ik lachen moet}{als ik om haar denk}\\

\haiku{Tante Saar is in;}{je oom Scheffler's huishouden}{totaal opgegaan}\\

\haiku{Gunst nog toe, ik ben.}{heelemaal in de war door}{al die drukte hier}\\

\haiku{De kinderen? 't;}{Zijn immers bijna allen}{volwassen menschen}\\

\haiku{zij, de oudste, had, '.}{de jongsten grootgebracht en}{t was goed gegaan}\\

\haiku{Wou je nu nog de,!}{oude jongejuffrouw gaan}{spelen malle Saar}\\

\haiku{Toen keerde Scheffler:}{zich plotseling om en vroeg}{met ernstige stem}\\

\haiku{ik dank je, Adriaan,.}{dat je me z\'o\'o lang bij je}{hebt willen houden}\\

\haiku{Waarom ga je niet,.}{aan de Station daar is}{meer te verdienen}\\

\haiku{Is den Dam me toch,.}{lief geworden omdat hij}{voor mij alles is}\\

\haiku{Nou hoef je niks niet, '.}{meer te vragen w\'a\'ar zes}{nachts geweest bennen}\\

\haiku{Veel bennen er bij,,!}{die je door een ringetje}{kunt halen keurig}\\

\haiku{Isa\"ak is dol op een,:}{ouwe jas maar die pakjes}{geeft hij je cadeau}\\

\haiku{Dan slaapt Isa\"ak als een,,.}{roos aan \'e\'en stuk door totdat}{hij er weer uit moet}\\

\haiku{'t is alsof het,.}{geen geld kost want ze vegen}{er de straat mede}\\

\haiku{Moosie smeert ze in, en.}{mijn jongste dochter past op}{den kleinen Sampie}\\

\haiku{Er klonk iets in die,;}{stem dat mij sympathetisch}{voor dien man stemde}\\

\haiku{hij keek strak voor zich, ',.}{ent scheen mij toe dat zijn}{onderlip trilde}\\

\haiku{maar - we waren aan.}{den dijk bij de zwemschool en}{ik moest uitstappen}\\

\haiku{op de hoogte van.}{Madera is hij over de}{fokkeschoot gegaan}\\

\haiku{Liesbeth wist niet wat ';}{ze mij doen zou van plezier}{datk weer thuis was}\\

\haiku{In die drie jaren,, '.}{dat je weg waart heb ikt}{dikwijls opgemerkt}\\

\haiku{{\textquoteleft}Ik w\`el, - ze houdt niet,.}{van hem ze houdt van geen een}{van de jongens hier}\\

\haiku{ben je gek, Tijs, - zei, -?}{ik tot mezelf zoo'n jonge}{blom zou jou nemen}\\

\haiku{Hier zweeg de man en.}{wischte zich een traan van de}{verweerde wangen}\\

\haiku{{\textquoteleft}Dat ziet er goed uit,...{\textquoteright}.}{goed uit als hij de dames}{in baltoilet ziet}\\

\haiku{de tanden buigt de,:}{candidaat-notaris}{terwijl hij vervolgt}\\

\haiku{{\textquoteright} Bevallig dankend.}{laat Marie zich door Emmer}{naar haar plaats brengen}\\

\haiku{maar daaraan wen je.}{al net zoo gauw als aan de}{lucht van de klare}\\

\haiku{Zoo'n slijterij is, - ',!}{geen broodwinningt is een}{goudwinning meneer}\\

\haiku{{\textquoteright} {\textquoteleft}Of 't! 'k Heb 't, ' '}{dikwijls genoeg tegen m'n}{vrouw gezegd alsk}\\

\haiku{Salomons wijsheid,.}{ten tweede Jobs geduld en}{ten derde Simsons kracht}\\

\haiku{die kwam driemaal in,:}{de week met een flesch daar een}{\'etiquetje op stond}\\

\haiku{- Eens op 'n keer, dat,,.}{ik ziek was deed mijn maat er}{in wat er op stond}\\

\haiku{daarom wou ik haar.}{op dien middag'n paar centen}{in de hand drukken}\\

\haiku{hij zag er uit als.}{de geletterde dood en}{hij kreunde van pijn}\\

\haiku{Ik merkte al gauw,.}{dat de stumperd stijf van de}{rheumatiek was}\\

\haiku{Op eens maakte 't,:}{kind zich van mij los ging vlak}{voor me staan en zei}\\

\haiku{t past mij niet om!}{iets van de klanten van mijn}{patroon te zeggen}\\

\haiku{Nou zal je reis wat,{\textquoteright}, {\textquoteleft}!}{hooren vervolgt hij maar een}{barschHou je mond}\\

\haiku{Met een scherpen gil '.}{eindigt de klarinet en}{t scherm gaat weer op}\\

\haiku{Wij verlaten de.}{keuken en zien in de zaal}{het laatste bedrijf}\\

\section{Pieter van der Meer de Walcheren}

\subsection{Uit: Mijn dagboek. Dagboek 1. 1907-1911}

\haiku{O onvindbare,,!}{onzichtbare Vader heb}{deernis met mij}\\

\haiku{Zijn wil en Zijn plan.}{worden openbaar en liggen}{bloot als een zee}\\

\haiku{Ik sta aan den rand.}{van het lichte geheim en}{buig mij stil voorover}\\

\haiku{Die dag brak aan als.}{alle andere dagen}{van alle weken}\\

\haiku{die voor vrijheid en;}{menselijkheid met Christus}{hun leven wagen}\\

\haiku{Ik voel mij in hem,,,,.}{mijn medemens vernederd}{bezoedeld verlaagd}\\

\haiku{Deze beeldhouwer.}{is kloek en rustig en vol}{helderen eenvoud}\\

\haiku{Ik begrijp zelf al.}{niet goed wat ik ben komen}{doen in het leven}\\

\haiku{- Laat ik uitscheiden,.}{ik voel dat ik mijn schone}{vreugde storen ga}\\

\haiku{Welk een afschuw had,}{zij van die tamme mensjes}{wier hoogste ideaal}\\

\haiku{Wij moesten elkander,, -?}{vinden wij werden geleid}{door een blind toeval}\\

\haiku{In de enkele}{dorpen aan den groten weg}{gelegen welken}\\

\haiku{Des nachts om \'e\'en uur,,.}{na het diner bij H. een}{jeugdig geleerde}\\

\haiku{wij vormden de kern.}{van een leger waarvan hij}{de aanvoerder was}\\

\haiku{De boulevard was,.}{verlaten slechts af en toe}{reed een rijtuig langs}\\

\haiku{Zij was zijn bode,,;}{zijn sekretaresse zijn}{vertrouwelinge}\\

\haiku{een vijfde zoekt het,;}{oneindige in diepste}{gemeenste zonden}\\

\haiku{Blijde levenskracht,,.}{doorgloeit mij ik voel mij sterk}{ik ben welgemoed}\\

\haiku{bij het zien van den,,.}{sterrenhemel van een bloem}{van gans het leven}\\

\haiku{- voelt uw verbeelding?}{den grondelozen afgrond}{naar alle kanten}\\

\haiku{Het weder blijft zacht;}{en zonnig als bij ons in}{een schone lente}\\

\haiku{Wanneer de grote,.}{zon over hen heen straalt dan is}{de aarde een feest}\\

\haiku{Ik wilde wel dat,,.}{door een wonder zij genas}{opeens volkomen}\\

\haiku{Het leven is een,.}{kalme bezigheid is een}{doodgewone zaak}\\

\haiku{Schuw spoedden zij zich,,.}{langs de huizen of zochten}{een rijtuig een taxi}\\

\haiku{De Melkweg boogt zijn.}{ontzagwekkende baan door}{den besterden nacht}\\

\haiku{Wij zijn op orde.}{en het geregeld leven}{gaat weer beginnen}\\

\haiku{En het schouwspel van.}{het leven is werkelijk}{niet zielverheffend}\\

\haiku{- {\textquoteleft}Hij bederft zelf zijn,{\textquoteright}, - {\textquoteleft}.}{succes zeide mij iemand}{hij mist soepelheid}\\

\haiku{Welk een vrome ziel!}{vol heimwee bezat deze}{grote muzikant}\\

\haiku{Die Durand is het.}{type van den mislukten}{intellektueel}\\

\haiku{En ik verheug mij.}{grotelijks op de lezing}{dezer boeken}\\

\haiku{In plaats van een rok,;}{hing een vuile blauwe lap}{om haar naakt lichaam}\\

\haiku{Hetzelfde drukke,.}{leven van mensen-zien}{werken en uitgaan}\\

\haiku{ontzagwekkend zijt.}{Ge en ik honger naar Uw}{tegenwoordigheid}\\

\haiku{laat door een wel schoon,,.}{maar geen werkelijkheid in}{zich dragend waanbeeld}\\

\haiku{de oneindige.}{wereld van den geest zal u}{geopend worden}\\

\haiku{ik leg haar in de;}{holte van Uw Hand. Zij is}{vuil en vol modder}\\

\haiku{Ik heb de macht van;}{de gewijde vingers van}{den priester gevoeld}\\

\haiku{Wij volgen Jezus,,.}{schrede na schrede op zijn}{smartelijken tocht}\\

\haiku{Hij herinnerde.}{zich mijn bezoeken van den}{vorigen winter}\\

\haiku{En toen ineens heb,}{ik het gevraagd ik had het}{uitgesproken v\'o\'or}\\

\haiku{Later, later zal.}{ik wel eens die donkere}{dagen verhalen}\\

\haiku{De rest liet ik, eerst,;}{toen ik heiden was over aan}{het onverwachte}\\

\haiku{En ik vroeg mij af:}{in die tijdloze uren van}{de stilte van God}\\

\haiku{God kan alles, - nu...}{Hij het klaar speelt van mij een}{priester te maken}\\

\section{J.H. Meerwaldt}

\subsection{Uit: 'Pid\'ari'. Of de strijd van het licht tegen de duisternis in de Bataklanden}

\haiku{Hij staat met den rug,.}{naar het huis toe dat aan Si}{Panggoe toebehoort}\\

\haiku{Den avond van dien dag.}{bezoeken wij het huis van}{Si Panggoe nogmaals}\\

\haiku{Bij de haardsteden.}{heeft de engel des slaaps reeds}{zijn werk begonnen}\\

\haiku{Met \'e\'en sprong is Si}{Panggoe op de been en ijlt}{zijn woning binnen}\\

\haiku{Terwijl hij naar zijn,}{luid schreeuwende strijdmakkers}{omkeek om te zien}\\

\haiku{De geslepen schelm.}{heeft na een paar dagen een}{uitweg gevonden}\\

\haiku{U behooren zij.}{toe en gij alleen hebt over}{hen te beschikken}\\

\haiku{Dat is waarlijk een.}{groot bewijs uwer liefde en}{goedheid jegens mij}\\

\haiku{{\textquoteleft}Als wij het kind eens!}{naar den zendeling te Pansoer}{na pitoe brachten}\\

\haiku{zoo meer tot hem zijn,.}{toevlucht in de velerlei}{nooden die het drukte}\\

\haiku{Nog even ootmoedig.}{als vroeger achtte hij dat}{ambt voor zich te hoog}\\

\haiku{De Heer zette hem.}{daartoe ook nog verder tot}{een rijken zegen}\\

\haiku{De woede van het.}{wilde volk werd hoe langer}{zoo meer opgewekt}\\

\haiku{Van uit de verte.}{rommelde reeds de donder}{grommend door het woud}\\

\haiku{{\textquoteleft}Waarom zoo haastig,,.}{vriend het onwe\^er kunt ge toch}{niet meer ontkomen}\\

\haiku{Eindelijk raakte}{de maat van Panggalam\"ei's}{euveldaden vol.}\\

\haiku{Er heerschte een.}{droeve bedrijvigheid in}{Loboe Hamindjon}\\

\haiku{Zijn geest moest er voor,.}{beloond worden dat hij in}{hem gebleven was}\\

\haiku{Op die wijze werd.}{dus eenvoudig twist gezocht}{tegen Ama Laboe}\\

\haiku{als hij maar naar de.}{gouvernementsschool wilde}{gaan en vlijtig leeren}\\

\haiku{{\textquoteright} - {\textquoteleft}Zeker, mijnheer, ik,{\textquoteright}.}{ben werkelijk geen Maleier}{luidde het antwoord}\\

\haiku{Ten slotte heeft ook.}{een brand nog een gedeelte}{van het dorp verwoest}\\

\section{Johan de Meester}

\subsection{Uit: Geertje (2 delen)}

\haiku{ach kijk, spatten op ', '.}{er mouw bij de pols en een}{groote vlok oper rok}\\

\haiku{- Ga nu weer zitten,,.}{de tijd is kort hoorde ze}{Grootvader zeggen}\\

\haiku{Nou hei je toch wat.... -,,.}{Heel vriendelijk van je hoor}{Lina zei Meester}\\

\haiku{Net of zij niet heel,!}{goed wist dat die eigenlijk}{op haar neer zagen}\\

\haiku{Die had er ook gauw,.}{een vrouw gekregen een mooie}{rijke Duitsche vrouw}\\

\haiku{En almaar, almaar -....}{vloog de trein en ze was er}{zeker nog lang niet}\\

\haiku{Hij ging enkel maar,.}{naar Utrecht maar tot zoover konden}{ze samen reizen}\\

\haiku{Wie zingt niet mee op, '....}{dee-ee-zen dag Vant}{Lindenhout ter eer}\\

\haiku{Het was aan de Schie,,,.}{een korte zijstraat als een}{l\^a zoo hol hoog-recht}\\

\haiku{- Zie je, daar kwam te,. '}{veel konkerensie hier heb}{ik het rijk alleen}\\

\haiku{Altijd treurde hij, ':}{over die dingen Groo'moe had}{noges eens gezeid}\\

\haiku{Tante zei niets meer, '.}{als aarzelend sjokte ze}{t keukentje in}\\

\haiku{vijand was ze dus,,....}{met Tante en ze had nog}{geen dienst ze had niks}\\

\haiku{Maar een duimvlak op,.}{de brief zei dat Tante hem}{had gelezen}\\

\haiku{- als die het wist van,....}{Oom z'en afval \'en hoe de}{zaak verloopen was}\\

\haiku{Groo'va was altoos,.}{dage lang van streek als u}{om geld had gevraagd}\\

\haiku{mit j' eens, da's geen.}{zaak om je grootvader mee}{an boord te komme}\\

\haiku{medeplichtig, door,;}{te zwijgen aan zijn verraad}{tegenover Groo'va}\\

\haiku{Heef u er nog over,?}{gesproke tegen mevrouw}{Gobius Tante}\\

\haiku{die naam k\`omp hier niet!.}{te p\`as galmde Maandag en}{lachte het eerste}\\

\haiku{- Hai je da' gehoord,,?}{Riek wat Gobius mit de}{koster gehad he't}\\

\haiku{- {\textquoteleft}Zoo, zit jij maar weer,'?}{te leze wat hei je nou}{uit de kas gehaald}\\

\haiku{Het was de eenige,.}{keer dat Oom van de nieuwe}{krant had gesproken}\\

\haiku{In de winkell\^a.}{lagen drie dagen lang vier}{en dertig centen}\\

\haiku{Wat een rijkdom, wat,.}{vreemde ramen maar je kon}{niks naar binnen zien}\\

\haiku{- Geer, morrege ga,....}{je met mijn uit dan mag je}{mee na Heins z'en huis}\\

\haiku{Hu, ze werd er haast,:}{duizelig van toen ze pal}{naar beneden keek}\\

\haiku{Toch bleef 't hoofdje,, '.}{branderig hoogrood van kleur}{ent mondje droog}\\

\haiku{Wacht, ze zou de deur,.}{maar toedoen Truus mocht is te}{veel overend komme}\\

\haiku{Nooit kon Meneer wat,!}{doen in huis of de Juffrouw}{had wat te zegge}\\

\haiku{Laa'st, toen Truusje zoo -?}{d\'o\'odziek was kon je wat an}{de Juffrouw merke}\\

\haiku{Wat had het anders,,.}{prettig kunnen zijn van avond}{weer daar op het plat}\\

\haiku{Zooas ie bevoorbeeld '!}{overet huwelijk en de}{liefde kon prate}\\

\haiku{En die moeder is -....}{zijn zuster en hij houdt van}{de Maagd Maria}\\

\haiku{Schuw keek ze naar Oom, ',;}{z'en glast hoeveelste was}{dit al ze wist niet}\\

\haiku{Tante scheen 't nie',....}{naar te vinde anders zou}{ze niet z\'o\'o lang staan}\\

\haiku{Ze lag op de rug,,.}{het hoofd half op zij ze kon}{Tante zoo niet zien}\\

\haiku{En zag, vol angst voor,.}{nieuwe schande Tante in}{de deur verschijnen}\\

\haiku{onder het wasschen.}{zou ze de tranen weten}{in te houden}\\

\haiku{Wat zou d'er zijn, dat,,?}{de feeks d'er was nou al en}{nog wel op Zondag}\\

\haiku{Zij deed het, zonder,:}{zich verder rekenschap te}{geven vrijmoedig}\\

\haiku{Wat had dat-er, '!}{mee te maken dat zet}{prettig had hier thuis}\\

\haiku{- Wat hei'j de kinders',.}{blij gemaak hoorde ze zijn}{vriendelijke stem}\\

\haiku{En kijk daar us, hoe,.}{aardig die twee moeders met}{al dat kleine grut}\\

\haiku{zag hij er weer {\textquoteleft}wel '{\textquoteright},.}{asen heer uit bekende}{Geertje zichzelve}\\

\haiku{Als ze haar liever, '.}{niet meer hadden moesten zet}{maar ronduit zeggen}\\

\haiku{Wij gaan alleen bij.}{Domenee Gobius en}{Domenee De Valk}\\

\haiku{O jee, komp Arie je! -.}{hale Hij had-t-er al}{wel kunne weze}\\

\haiku{die armen, die als,,.}{klauwen waren dat beestesnoet}{dat op haar toedrong}\\

\haiku{Zij hield de oogen nog,.}{gesloten tranen welden}{de oogleden door}\\

\haiku{Doch zoodra hij,.}{kwam te spreken was er voor}{Geertje slechts zijn stem}\\

\haiku{Wat had ze ook voor,;}{slaap gehad zoo kort en dan}{in die benauwdheid}\\

\haiku{V. Toen zij na de,.}{kerk thuis aanbelde trok de}{juffrouw aan het koord}\\

\haiku{het was of zij de,;}{handdruk na-voelde in}{haar borst door haar lijf}\\

\haiku{and're jongens,;}{bleven staan meenende dat}{er wat gebeurde}\\

\haiku{Ze wou wel graag naar,.}{de middagkerk maar dan moest}{ze eerst nog thuis zijn}\\

\haiku{O, die z\'alige,,;}{lucht van et hout toch na de}{nacht et elzenhout}\\

\haiku{Dat was zoolang as -!}{Geertje heugde en n\'o\'oit was}{er wat an gedaan}\\

\haiku{Bij het thuiskomen;}{van de wandeling vond ze}{Wouter Heukelman}\\

\haiku{Als instinktmatig.}{loenschte Geertje even naar}{de kant van Groo'va}\\

\haiku{De larikse, de, ',;}{konifeeret glansde}{zilver op et groen}\\

\haiku{{\textquoteleft}Mijn liefste is blank,.}{en rood hij draagt de banier}{boven tien duizend}\\

\haiku{Vele wateren:}{zouden deze liefde niet}{kunnen uitblusschen}\\

\haiku{Nicht was er al van ',.}{s morgens af Rika zou}{niet zijn gekomen}\\

\haiku{De jongens zouden,.}{straks wel komen zij waren}{nu nog aan het werk}\\

\haiku{- Dee je niet beter,,?}{Geertjen as je nou bij}{je grootmoeder bleef}\\

\haiku{leege kamer er nog.}{ongezelliger uit dan}{eerst in de schemer}\\

\haiku{Jan had het over Gijs,,,.}{hun knecht die nu getrouwd was}{maar niet gelukkig}\\

\haiku{Koos naar school en Moe,,!}{boven in bed o het was}{zoo heerlijk zitten}\\

\haiku{'t Was of er oogen,....}{al grooter werden of die}{hem omvatten moesten}\\

\haiku{ze wist et niet meer,,,!....}{ze had een ijskoud hoofd dat}{leeg was ze wier g\`ek}\\

\haiku{- H\`e ja, en nou met.... '.}{zoo'n regent Kind keek haar}{aan en zij het kind}\\

\haiku{Radeloos trok ze;}{de schouders op om de bloote}{nek te beschermen}\\

\haiku{Gotogot en d'er,,!}{was toch niks ze weerde hem}{af zooveel ze kon}\\

\haiku{Hij mocht de Juffrouw, '.}{niet verlatet zou de}{grootste zonde zijn}\\

\haiku{{\textquoteright} Maar ze kon niet, ze,....}{m\`oest Tante spreken over die}{boodschap van Groo'moe}\\

\haiku{Sophie, met 'en -!}{schoone japon an die wist}{ook wel wat ze dee}\\

\haiku{maar opeens, daar was! ',;}{de werkvrouwt mensch w\`as daar}{al de deur al toe}\\

\haiku{'En drukkie op d'er,....}{lijf of ze de vrouw is van}{de burgemeester}\\

\haiku{Ze was geen kind meer,,.}{ze wist toch wel wat het was}{getrouwd-zijn}\\

\haiku{Tusschen hem en zijn,,;}{vrouw was het uit was er n{\`\i}ets}{meer sedert lang al}\\

\haiku{Zij, alleen-wakker,.}{als om af te luisteren}{dat allen sliepen}\\

\haiku{Zij voelde dat zij,,;}{hevig kleurde want ja ze}{had aan Hem gedacht}\\

\haiku{Hij was nog uit - stond}{ze er in haar kamertje}{lang mee in de hand.}\\

\haiku{Maar de kinderen,,.}{bleven  slapen heel het}{huis sliep of was stil}\\

\haiku{t Zou zijn, als kon,.}{het haar niet schelen of hij}{nog meer geld uitgaf}\\

\haiku{- Heerejee, nou ben ' ''.}{k weer te stil en strakjes}{mochk nie prate}\\

\haiku{En net zoo wat op....}{etzelfde mement was goeie}{Groo'moe gestorven}\\

\haiku{Ze gaf nu nog om,.}{Hem alleen om het geluk}{dat Hij bij haar vond}\\

\haiku{Niets op aarde was.}{haar iets waard meer bij wat zij}{zich nu bewust werd}\\

\haiku{blij keek ze neer op,:}{de ontbijtboel die nu gauw}{het eerst aan kant moest}\\

\haiku{Wel aan haar gezicht,,,,;}{haar kleeding ook aan haar figuur}{heur haar heur handen}\\

\haiku{Waarom 'en weekblad? - ''.}{k Dach dat Heins je d'er een}{had meegegeve}\\

\haiku{Ze wilde 't graag,,.}{uit medelijden doch dorst}{het niet goed vragen}\\

\haiku{Geertje nam nu dus, '.}{maar afscheid buurvrouw zout}{verdere wel doen}\\

\haiku{Zondag had ze kou,,.}{gevat met dat lange}{verschoonen boven}\\

\haiku{Zoo aak'lig als 't,,.}{was zoo laag en zoo somber}{toch had zij het lief}\\

\haiku{Ze zou'en het nu wel, {\textquoteleft}.}{niet meer gel\'o\'oven dat zelaa'st}{wat kou gevat had}\\

\haiku{Zij had haar lange.}{uitgaansavond en om vijf uur}{trok ze de deur dicht}\\

\haiku{maar zou Oom dan niks,?....}{verdienen wat deed hij dan}{toch aldoor op straat}\\

\haiku{- en thuis moest d'er nog;}{z\'o\'oveel gedaan voor de groote}{partij van vanavond}\\

\haiku{{\textquoteleft}'t Hijgend hert, der{\textquoteright} -....}{jacht ontkomen dat ze zelf}{zoo'o hijgend hert was}\\

\haiku{{\textquoteleft}Hou je dan heelem\'a\'al',?}{nie meer van me wil je nou}{volstrekt m'en verderf}\\

\haiku{als het kon in zijn,.}{eigen dienst dat ze misschien}{met Oom kon deelen}\\

\haiku{Maar norsch liep het.}{mensch langs haar heen zonder eenig}{antwoord te geven}\\

\haiku{Krabben wou ze, dat, ',.}{\`andre menscht slet dat ze}{h\'a\'ast m\'e\'er nog haatte}\\

\haiku{En die vent die z'en, ':}{centen op-zijn met drie}{juffers omem heen}\\

\haiku{H\`e, die gevel van ',!}{t Loterijketoor net}{een brief met rouwrand}\\

\haiku{En dat netstille, '?}{van de winkels w\'a\'arom maakte}{t haar zoo angstig}\\

\haiku{O, h\`e, 't gaf haar,.}{een verruiming die joodsche}{slagerij nog open}\\

\haiku{Waar\`om zou ze vand\'a\'ag juist,!}{bevallen z\'o\'oveel maanden}{en maanden te vroeg}\\

\haiku{Toen die heer, die haar,,,....}{z\'o\'o maar aansprak lachend met}{knipoogjes gemeen}\\

\haiku{Toch - soms z\`onk ze weg,,:}{als in ijs kreeg ze een drang}{om weg te h\`ollen}\\

\haiku{- Ge mee, hieraufer,....}{dan drinke me-n-en kep}{keffie en prate}\\

\haiku{Ze dacht, huiv'rend, aan -,'....}{de middag zij daar tot schand}{van een heele buurt}\\

\haiku{hij, in een willen,;}{woedend blijven woedend van}{verontwaardiging}\\

\haiku{mejelij mot je,,.}{met ter hebben mejelij}{niks as mejelij}\\

\haiku{Driftig schokkend met,,,.}{de schouders liep ze om te}{kunnen zwijgen weg}\\

\haiku{En ze knielde v\'o\'or,,....}{de koffer prikte met het}{sleuteltje draaide}\\

\haiku{Oom vertelde van,.}{Cohen dat die zijn neef pas}{weer had belazerd}\\

\haiku{Als ze stil op zoo'n,.}{kamer woonde zou hij wel}{weer durven komen}\\

\haiku{voordat hij met zijn,.}{stoel had geschoven was z{\`\i}j}{uit haar leunstoel op}\\

\haiku{V\'o\'or 'um te staan, z'en,.}{stem te hoore dat z'en hoofd}{weer boog naar haar over}\\

\haiku{As die de brief in,,'.}{hande kreeg tien tegen een}{da jij um nooit zag}\\

\haiku{maar hij moest het doen,,.}{om thuis voor de kinders want}{anders geen leven}\\

\haiku{{\textquoteleft}Zou mijn aangezicht?}{moeten medegaan om u}{gerust te stellen}\\

\haiku{zoo mag je niet doen, '.}{je ratelt ze af als een}{roomschet latijn}\\

\haiku{Vele wateren....}{zouden deze liefde niet}{kunnen uitblusschen}\\

\haiku{Zij zou er toch ook,;}{geen kunnen krijgen want de}{planken lagen leeg}\\

\haiku{Maar Donderdags was,:}{ze weer zooveel beter dat}{hij had bevolen}\\

\haiku{Ze zei zich, dat ze,.}{niet verlangde dat haar angst}{te hevig was}\\

\haiku{De ontnuchtering,.}{die in haar viel maakte haar}{weer plotseling moe}\\

\haiku{Maar Geertje keek sta\^ag,,.}{lusteloos zij zat er bij}{suf-onverschillig}\\

\haiku{Heele weke gaan,.}{d'er voorbij da'k alleen met}{Cohen te doen heb}\\

\haiku{Weer nam zij deel aan,,.}{den dienst van God in het Huis}{van God weer mocht zij}\\

\haiku{thuis gevoeld had ze,.}{zich in beide als nooit een}{oogenblik bij Oom}\\

\haiku{Zenuwachtig, wist,;}{ze zelf niet waar ze naar toe}{zou gaan en waar niet}\\

\haiku{Strompelend kwam hij.}{het trapjen af en bleef}{schuins v\'o\'or Geertje staan}\\

\haiku{Met de oogen toe had,:}{ze aan tafel gezeten}{tot Tante zelf zei}\\

\haiku{Wat het slet dan toch '!}{wel dachtt Geld lag zeker}{te grabbel op straat}\\

\haiku{Zoo pijnigde het,:}{denken haar telkens wanneer}{ze besloten was}\\

\haiku{Als een priemsteek kwam:}{dan de gedachte aan de}{brief in haar koffer}\\

\haiku{Ze moest blijven in?}{Jan zijn stad en wie had ze}{hier anders dan Oom}\\

\haiku{Geertje wist, dat zij,.}{nu het moest zeggen doch zij}{vond de woorden niet}\\

\haiku{Jasses, om zoo ie's, '.}{gemeens te denken en dat}{enkel vanen grap}\\

\haiku{Nu, Maandag zou dan,.}{op de hoek van de straat aan}{de Schie blijven staan}\\

\haiku{Om toch geen gerucht,.}{te maken bleef ze in haar}{gebogen houding}\\

\haiku{de laatste woorden.}{van Groo'moe prevelden een}{gebed voor jou}\\

\haiku{daar nu scharrelen.}{moeten om bij de lage}{kapstok te komen}\\

\haiku{om Geertje nau bai,,....}{u thuys te hebbe \`as et}{kan as Geertje will}\\

\haiku{- Och ik kan hier toch',!}{z\'o\'o nie weg de bedde ben}{nog niet eens aan kant}\\

\haiku{Hij, hij, Groo'va, net, ', '.}{als Tanteen slet was ze}{in hun oogenen slet}\\

\haiku{dat, nu zij niet schreef,.}{aan Jan God hem misschien zou}{neigen tot schrijven}\\

\haiku{Hij ijlde naar zijn,,;}{bedstee trok er een deken}{uit nam het kussen}\\

\haiku{{\textquoteright} 't Was m\'ogelijk,.}{dat ze enkel van die schrik}{was flauw gevallen}\\

\haiku{nee, duisternis, nee,, '}{zij was niet meer ziek zij lag}{hier iner bedste\^e}\\

\haiku{Bij Maandag kon niet,,,?}{ze wou niet bij Oom dus toch}{weg met Groo'va mee}\\

\haiku{Waarom dringt Maandag,?}{toch aldoor m\'e\'e aan dat ze}{Willen nemen zal}\\

\haiku{Bij menschen spreekt en,.}{lacht zij mee houdt de schijn op}{van jonge Geertje}\\

\haiku{- {\textquoteleft}Vindt je dat dan niet,?}{mooi van die jongen dat hij}{z\'o\'oveel van je houdt}\\

\haiku{Eerst d'er ziekte, toen,,, '.}{de miskraam en veel werken}{nu dater zwaar viel}\\

\haiku{Ze stond achter een,,}{hekje in een groep menschen}{die opeens om d'er}\\

\haiku{Zij gaan nu nogmaals.}{samen naar Oom en nog loopt}{hij te zaniken}\\

\haiku{In de menigte:}{nam hij haar arm en toen zij}{die los wou trekken}\\

\haiku{hier aan het eind, hij,.}{weet het wel dan ontmoet hij}{de trem van het Park}\\

\subsection{Uit: De zonde in het deftige dorp}

\haiku{als het, met heel veel}{geduldigen takt van flink}{zijn en kalm je wensch}\\

\haiku{maar het was Stork, of:}{hij opeens een anderen}{kijk op de meid kreeg}\\

\haiku{Emmy vond het toen ',.}{z\'o\'on eer dat zij voor de}{frankeering wou zorgen}\\

\haiku{- Vrinden hebben me,.}{verweten dat ik een vrouw}{met geld heb getrouwd}\\

\haiku{het d\`orp zou hij hem,!}{hebben ontzegd als het nog}{in zijne macht stond}\\

\haiku{vast de voerman uit -.}{de stad dan kreeg men zijn goed}{alweer pas morgen}\\

\haiku{een nooit voorziene,.....}{bloei van geluksvreugd somtijds}{bijna overstelpend}\\

\haiku{En hij aarzelde,,....}{hij was bang terwijl zijn kind}{zijn hulp behoefde}\\

\haiku{Even het geloovig;}{geduld der oude freule}{Sitsen bewonderd}\\

\haiku{Sprekend groeten bleef, {\textquoteleft},{\textquoteright} {\textquoteleft}.}{uitzondering bewaard voor}{DokterDaumenee}\\

\haiku{Het openen van het.}{raam had stagnatie in het}{gesprek veroorzaakt}\\

\haiku{De vader en zijn.}{vriend waren in de warme}{kamer gebleven}\\

\haiku{Joop vond het van Leo,:}{een prachtig idee maar stond er}{wat beteuterd bij}\\

\haiku{Het meisje wipte,,.}{van haar kruk en keek pruilend}{of ze zou huilen}\\

\haiku{Neuri\"end stapte.}{hij door de eenzaamheid van}{zijn kamer terug}\\

\haiku{Hij verzekerde,.}{zich zooveel mogelijk te}{hebben afgeschud}\\

\haiku{ingebroken op,.}{drie leege buitens maar nergens}{een beurs gevonden}\\

\haiku{- Ik dacht juist dat u.}{met dit weer graag van de hei}{zou zijn weggevlucht}\\

\haiku{- Dan moeten we hem '.}{ookes naar Holland halenl}{grappigde Berkie}\\

\haiku{Bronchitis en ik,.}{weet al niet wat enkel door}{onvoorzichtigheid}\\

\haiku{Het gaf hem eenige,.}{voldoening Dina rust te}{kunnen bezorgen}\\

\haiku{In de andere.}{zaal meende Stork onderdrukt}{gelach te hooren}\\

\haiku{- Schaken ken ik ook,,.}{zei de ander zelfbewust}{maar onverschillig}\\

\haiku{Ach, maar daar hoorde,:}{hij Loesje roepen vlakbij}{op de bovengang}\\

\haiku{Hij bleef naar de deur,.}{gewend staan totdat Wedelaar}{haar had gesloten}\\

\haiku{Ik merk aan je, dat.}{je Dina's verzekering}{niet in twijfel trekt}\\

\haiku{- Arme goeie man, had.}{zij gefluisterd en hem op}{het voorhoofd gekust}\\

\haiku{Wedelaar vouwde de.}{briefblaadjes v\'o\'or zich dicht en}{legde ze bijeen}\\

\haiku{- Dina, altoos grootsch,,.}{bleef stug doen ook terwijl ze}{van Neeltje afhing}\\

\haiku{Daar was moeder voor,,.}{doende geweest Dinsdagsavonds}{ook l\`ang nog in bed}\\

\haiku{nooit hoorde je d\'a\'ar ',.}{vanen motje praten als}{bij d'erlui menschen}\\

\haiku{Doortastend nam zij,, -.}{dezen op blies blies nog eens}{het lichtje was uit}\\

\haiku{Doch zij wilde de -,;}{schuld niet aanvaarden niet uit}{liefde voor Wedelaar}\\

\haiku{Op \'e\'en had zij het,'.}{lang gezet van dat z een}{kind van elf was af}\\

\haiku{Het was het eerste,:}{geweest wat ze met mekaar}{hadden meegevoeld}\\

\haiku{De schrikkelijke.}{tijding had de lieve z\'o\'o}{droef doen ontstellen}\\

\haiku{Toen haar brief kwam met,:}{haars vaders adres achterop}{had hij begrepen}\\

\haiku{{\textquoteleft}want zij was eene maagd,,.}{zoodat het in Amnons oogen zwaar}{was haar iets te doen}\\

\haiku{zij wilden Hugo,.}{beslist nog spreken v\'o\'or dit}{plotseling vertrek}\\

\haiku{Maar nu komt Hendrik,.}{uit het dorp met een vreemde}{treurige boodschap}\\

\haiku{Met een wenk, dat hij,.}{zwijgen zou ging Aleid hem in}{de huiskamer voor}\\

\haiku{Nu was ook Jan van.}{Loodijck opgeschrikt uit zijn}{rustige houding}\\

\haiku{- Zegt u me eerlijk,,;}{Dominee uw zelfverwijt}{brengt me tot die vraag}\\

\haiku{De vraag met een knik,,,:}{beantwoordend bond Hovink}{aangemoedigd aan}\\

\haiku{nou, de meid dan had,....}{verteld dat Dina en het}{medisch studentje}\\

\haiku{VROUW Van Rooien zat,,,.}{verslagen scheef op haar stoel}{gekromd moe hijgend}\\

\haiku{Mijn gaf-t-ie niet,.}{op nou liep-t-ie de meid}{d'er broers achterop}\\

\haiku{Toen zij melk voor Wim, ':}{ging halen had zijt uit}{de verte gezien}\\

\haiku{Het ligt geenszins in,.}{mijn bedoeling pressie op}{hem te oefenen}\\

\haiku{Hij dorst, hij kon niet,.}{alles schrijven juist zooals het}{Papa gegaan was}\\

\haiku{Dus stuurde Papa,....}{het rijtuig-terug toen}{moest het nog wachten}\\

\haiku{Berkie gaf een draai,,.}{aan zijn stoel nam het papier}{en deed of hij las}\\

\haiku{Allen stonden, doch.}{op aandringen van broeder}{Van Loodijck was Ds}\\

\haiku{Krookje was een heel,.}{goed meisje maar men kon niets}{aan haar overlaten}\\

\haiku{al die bezoeken,;}{dikwijls zoo ver. Wedelaar was}{immers niet jong meer}\\

\haiku{toen ze blij van de -.}{huisbel schrikte eindelijk}{zou daar Wedelaar zijn}\\

\haiku{*** Vrouw Van Rooien vroeg,.}{wel freskuus dat zij Mevrouw}{dorst lastig vallen}\\

\haiku{, Leo was twee uren thuis,.}{en nog had hij niet met zijn}{Vader gesproken}\\

\haiku{haastte hij zich tot,.}{den bediende nog voordat}{hij gezeten was}\\

\haiku{Toen bleef hij zitten,.}{het glas in de hand. Stork zag}{tranen in zijn oogen}\\

\haiku{En dien had hij nog ',,!}{es ten eten gevraagd denkend}{dat misschien met Em}\\

\haiku{hij moest altoos de,.}{menschen nog vinden waar zijn}{geld niet welkom was}\\

\haiku{Als opblazend van,:}{voldaan gegrinnik legde}{Hovink aan Stork uit}\\

\haiku{Stork en Claartje van.}{Lakervelde zaten in}{zijn studeerkamer}\\

\haiku{Maar wel had hij sterk:}{de bekoring gevoeld van}{dat enkele ding}\\

\haiku{De jongen was er,.}{zuinig op wat Leo plezier}{deed als blijk van smaak}\\

\haiku{Als dat te lastig,;}{is wil jij het misschien van}{jou boekje nemen}\\

\section{J.T. de Meesters}

\subsection{Uit: Memento mori}

\haiku{gelukkig dat hij.}{zich een oogenblik aan zijn}{studie kon wijden}\\

\haiku{{\textquoteleft}Je weet dat de kans,.}{dan groot is dat ik dwaze}{dingen doe of zeg}\\

\haiku{Maar, zooals ik je zei,.}{probeer nu eerst een beetje}{tot rust te komen}\\

\haiku{Toen zij geen antwoord,.}{kreeg waagde zij het nog niet}{naar binnen te gaan}\\

\haiku{De collega van.}{den specialist liet niet}{lang op zich wachten}\\

\haiku{vond hij toch dat hij.}{de meeste dingen beter}{deed dan anderen}\\

\haiku{Naar Willy's meening!}{in ieder geval veel meer}{dan wenschelijk was}\\

\haiku{Ik ben nog onder,.}{behandeling geweest van}{den dokter haar man}\\

\haiku{{\textquotedblleft}een ziekelijke{\textquotedblright}.}{vrouw is het ergste wat een}{man kan overkomen}\\

\haiku{van den toestand van,.}{zijn passagiere besloot}{hij tot het laatste}\\

\haiku{{\textquoteright} zei Jonkmans streng, het.}{ijzer willende smeden}{terwijl het heet was}\\

\haiku{Ineens verlies ik,.}{hem uit het oog spring op m'n}{fiets en zie niets meer}\\

\haiku{Wij zullen, vriendje,.}{dit zaakje eens fijn met z'n}{twee\"en opknappen}\\

\haiku{Denk je dat ik mijn?}{reputatie door jou wil}{laten bekladden}\\

\haiku{Wat zou het - hebben?}{we niet allemaal een of}{andere manie}\\

\haiku{driftig zou kunnen,!}{worden dat hij je een pak}{slaag gaf dan ril ik}\\

\haiku{Een volwassen vrouw....}{die beweert magere-Hein}{gezien te hebben}\\

\haiku{{\textquoteleft}Jij bent den laatsten,!}{tijd bepaald niet heelemaal}{in orde Hermans}\\

\haiku{Als u uw kleeding even,.}{los wilt maken zal ik uw}{hart onderzoeken}\\

\haiku{Op het bureau zat.}{Jonkmans al vol ongeduld}{op hem te wachten}\\

\haiku{{\textquoteright} {\textquoteleft}Jawel, chef, maar hij,.}{knipt niet alleen zijn eigen}{baard maar ook zijn haar}\\

\haiku{Mevrouw Hobbel had.}{meer dan \'e\'en reden om naar}{buiten te kijken}\\

\haiku{Toen ik nu hoorde,.}{bellen vertrouwde ik het}{heelemaal niet meer}\\

\haiku{begrijpelijk, daar.}{zij slechts zelden aan licht en}{lucht werd blootgesteld}\\

\section{Paul Meeuws}

\subsection{Uit: Badhuis in de sneeuw}

\haiku{Maar aan die woorden.}{zat het parfumluchtje van}{een vreemd accent}\\

\haiku{Blom had het over zijn,.}{ontmoeting met Feininger}{eerder op de dag}\\

\haiku{Ik ook, daarom heb.}{ik geprobeerd je tegen}{hem te beschermen}\\

\haiku{Misschien hebben ze,,.}{gelijk denkt Feininger en}{klinkt het toch te zacht}\\

\haiku{Blom turft het aantal.}{gespeelde stukken op een}{notitieblokje}\\

\haiku{Er was geen durf voor.}{nodig om er je vinger}{doorheen te steken}\\

\haiku{Hij zou rechtstandig,,.}{omlaag storten de grond in}{samen met het huis}\\

\haiku{De soldaten voor.}{het Stadhuis keken verbaasd}{in onze richting}\\

\haiku{{\textquoteright} {\textquoteleft}Pas jij maar beter,{\textquoteright}.}{op je vrouw zei mijn moeder}{met trillende stem}\\

\haiku{Ik kon mij bij het {\textquoteleft}{\textquoteright}.}{woordheulen eigenlijk niets}{verkeerds voorstellen}\\

\haiku{Mijn broertje en ik,.}{gingen in de houding staan}{net als iedereen}\\

\haiku{Binnenkort zou het.}{klavichord verhuizen naar}{de voorkamer}\\

\haiku{Al wekenlang was,.}{het volop zomer het hout}{was droger dan ooit}\\

\haiku{Hier en daar dook haar.}{glinsterend oog tussen het}{gebladerte op}\\

\haiku{Die vrolijkten het,.}{heilig huisgezin op met}{zaag schaaf en hamer}\\

\haiku{Ze boog een beetje.}{voorover en leunde met haar}{kin in haar handen}\\

\haiku{Ik stelde me het.}{hebben van borsten tot in}{de finesses voor}\\

\haiku{Van afrastering;}{tot afrastering draafde}{hij als een jong paard}\\

\haiku{Eindelijk rolde.}{de bal tussen zijn blote}{voeten op de grond}\\

\haiku{De galg verhief zich.}{tegen een schitterende}{diepblauwe hemel}\\

\haiku{De mannen op het.}{verlichte terras keken}{op van hun kaartspel}\\

\haiku{moest je het spoor over,,.}{de stad uit richting M. waar}{ook de ijsbaan lag}\\

\haiku{Ze trok haar hand van.}{hem weg en rende pardoes}{de drukke straat op}\\

\haiku{Ongrijpbaar als een.}{vogeltje fladderde ze}{boven zijn bereik}\\

\section{Geerten Meijsing}

\subsection{Uit: Tussen mes en keel}

\haiku{Als aangeklede;}{aap was ik degene die}{belachelijk was}\\

\haiku{Wat mij betrof, voor.}{het allemaal tot een goed}{einde was gebracht}\\

\haiku{Wel wil ik zeggen,.}{waar ik woon al heb ik thuis}{niets aan te bieden}\\

\haiku{We wachtten tot de.}{vrouwelijke collega}{ook was ingestapt}\\

\haiku{Als hij het te kwaad,,.}{heb ben ik er altijd voor}{hem en andersom}\\

\haiku{Veel beter af dan.}{met het leven dat ik haar}{had kunnen bieden}\\

\haiku{Woorden van liefde -...}{zijn in water geschreven}{als {\'\i}emand dat wist}\\

\haiku{In de luwte van.}{het bestaan was alles tot}{stilstand gekomen}\\

\haiku{Zijn ijdelheid wordt.}{licht gekrenkt en dan laat hij}{de ander vallen}\\

\haiku{Vlot in de omgang,}{maar hij moet oppassen niet}{uit zijn nek te gaan}\\

\haiku{Zij was altijd een.}{beetje boos wanneer ze mij}{van advies diende}\\

\haiku{Maar voor die dingen,.}{had ik nu geen tijd want ik}{moest weer aan het werk}\\

\haiku{En steeds opnieuw dat.}{kleine reservoirtje van}{die vulpen vullen}\\

\haiku{wanneer ik me in.}{het midden van een witte}{nacht bij haar voegde}\\

\haiku{Ik was zo bang haar.}{te irriteren dat ik}{haar irriteerde}\\

\haiku{We hadden ze slechts,.}{voor het grijpen zo dichtbij}{waren de sterren}\\

\haiku{Daarin  waren.}{we niet beter af dan het}{redeloze vee}\\

\haiku{Maar waarom was het,?}{dan zo zoet veel lekkerder}{dan voedsel of drank}\\

\haiku{aan Martha, beide,;}{Mireilles Montalcino en}{de Magere Brug}\\

\haiku{{\textquoteright}, dan glimlachte ze:}{geheimzinnig en haalde}{ze haar schouders op}\\

\haiku{Ik wist ook - en van;}{geen enkele wetenschap}{word je vrolijker}\\

\haiku{En dat degene.}{die het meest liefhad altijd}{in het nadeel was}\\

\haiku{Want als zij eenmaal.}{alle sluizen openzette}{was ik nergens meer}\\

\haiku{Je werd beroofd, niet.}{alleen van je vrijheid maar}{ook van je wilskracht}\\

\haiku{Te minder omdat.}{er weinig te verwachten}{was als beloning}\\

\haiku{{\textquoteright} ~ Een van onze.}{grootste problemen was het}{verschil in tempo}\\

\haiku{Dit was het enige,.}{wat ik nog had mijn geloof}{in dit laatste boek}\\

\haiku{Het was er wel, als,.}{een baksteen in de hand maar}{je kon er niets mee}\\

\haiku{De mogelijkheid,.}{kreeg pr\'esence de spoken}{kregen gestalte}\\

\haiku{Een tweede keer zou.}{ik deze komedie niet}{kunnen opvoeren}\\

\haiku{Maar paniek kan de -.}{toestand wel omschrijven al}{is dat niets voor mij}\\

\haiku{Ik herkende het.}{soort vlees niet onmiddellijk}{en vroeg wat het was}\\

\haiku{Ik wil kinderen -,.}{is het niet van jou dan maar}{van iemand anders}\\

\haiku{De meegebrachte.}{studieboeken liet ik even}{voor wat ze waren}\\

\haiku{Vooral als zij haar,.}{formule uitsprak kon ik}{het niet meer houden}\\

\haiku{Ik wilde daar geen.}{kwestie van maken en dat}{deed zij ook niet}\\

\haiku{Ten leste was ik,.}{nauwelijks meer bruikbaar zelfs}{niet als seksobject}\\

\haiku{Maar wat mij het meest,.}{verontrustte was dat ik}{genoot van mijn smart}\\

\haiku{Hij was uit Holland.}{weggegaan om in New York}{beroemd te worden}\\

\haiku{Waarschijnlijk had ze.}{tegenover hem heel wat meer}{over mij te klagen}\\

\haiku{Ze zei wel {\textquoteleft}liefje{\textquoteright},,.}{tegen me afstandelijk}{ironisch en gemeend}\\

\haiku{En in de oranje.}{streep van de ondergaande}{zon zag ik de hoop}\\

\haiku{Ik hoef hem niet uit.}{de kast te trekken om de}{hoes terug te zien}\\

\haiku{Het deed pijn (zoals:}{mijn moeder gezegd had na}{elke bevalling}\\

\haiku{Tenslotte kon ik.}{ook op eigen kracht tegen}{de berg opklimmen}\\

\haiku{En kracht hadden ze,.}{mij bepaald niet gegeven}{deze roespillen}\\

\haiku{Toen ik haar leerde.}{kennen woonde ze tussen}{uitdragersspullen}\\

\haiku{Net als bij klussen.}{begint goed koken met het}{juiste gereedschap}\\

\haiku{Zij was er ook niet,.}{graag wist niet zo goed wat te}{doen als ze thuis was}\\

\haiku{Als mijn boek later,.}{uitkwam dan gepland zou ik}{gewoon wegblijven}\\

\haiku{Die wisten ervan,.}{maar konden niet zeggen hoe}{serieus het was}\\

\haiku{Deed de afwas, en.}{werkte een middag met de}{zeis in de boomgaard}\\

\haiku{Zoals Tolstoj zei ():}{die ik ooit van onder haar}{aanrecht bevrijd had}\\

\haiku{Voor de duur van een.}{kop thee is het oor van de}{ander gewillig}\\

\haiku{Zwarte schoonmakers.}{in witte jassen kwamen}{traag dweilend voorbij}\\

\haiku{Maar koffie hoort bij,.}{de psychiatrie zoals}{we zullen merken}\\

\haiku{Daarna zou hij met.}{zijn staf bespreken of ik}{in aanmerking kwam}\\

\haiku{Ik zei sorry toen.}{onze handen aan het stuur}{elkaar even raakten}\\

\haiku{Ze was gekleed om,.}{te winnen niet op punten}{maar met een knock-out}\\

\haiku{En stelt u zich dan.}{uw vriendin voor in de rol}{van Schopenhauer}\\

\haiku{Het water klotste.}{met de wind mee heen en weer}{onder de bruggen}\\

\haiku{Als ik mijzelf een,:}{graf uitzocht kon de tombe}{niet groot genoeg zijn}\\

\haiku{een groene kapel,.}{de hele natuur moest mij}{tot kathedraal zijn}\\

\haiku{Het begraven der.}{doden in het laagland is}{levenloos en kil}\\

\haiku{Aldus verging het,.}{mij de eerste keer en zo}{is het gebleven}\\

\haiku{Hoe meer Pierlala,}{mij te na komt hoe harder}{en uitbundiger}\\

\haiku{Lange tijd heb ik. '}{het idee gehad dat zij er}{alleen voor mij was}\\

\haiku{Als ik niet verliefd,.}{op of idolaat van iemand}{was schortte er iets}\\

\haiku{Mijn ouders waren,;}{voor langere tijd verreisd}{mijn broer zat in dienst}\\

\haiku{Want de gedachte.}{aan zelfmoord vergezelde}{me nu voortdurend}\\

\haiku{Daarna vertrok mijn.}{vrouw en bleef ik alleen met}{mijn dochter achter}\\

\haiku{De concurrentie.}{met mijn dochter was bijna}{iedereen te zwaar}\\

\haiku{Kost het u moeite,?}{om uit bed te komen zelfs}{later op de dag}\\

\haiku{Ik leed dus aan iets.}{waarvoor andere mensen}{behandeld werden}\\

\haiku{Maar ik wilde mij,.}{niet laten behandelen}{nooit van mijn leven}\\

\haiku{Liever had ze die.}{pas in gebruik genomen}{na de verhuizing}\\

\haiku{Maar ze begreep dat.}{ik mijn behoefte moeilijk}{in het park kon doen}\\

\haiku{Ik moest mijn werk doen.}{en tegemoetkomen aan}{haar verwachtingen}\\

\haiku{In ieder geval -.}{moest ik op het werk zijn dat}{was vaak al genoeg}\\

\haiku{{\textquoteleft}Meneer Provenier,,,.}{wij voeren een gesprek u}{en ik een dialoog}\\

\haiku{En u beschouw ik,.}{niet als analysant die maar}{wat aan kan kletsen}\\

\haiku{In Itali\"e had ik.}{nog een kettingzaag die ik}{wilde ophalen}\\

\haiku{zijn afkeer voor de.}{soort was onverholen en}{welgeformuleerd}\\

\haiku{Voor zo'n dor leven.}{als het zijne had ik mij}{willen behoeden}\\

\haiku{Jij blijft hier tot het,!}{laatst al  was het om mij}{een plezier te doen}\\

\haiku{Ik kneep hard in mijn.}{arm om te voelen of ik}{nog iets kon voelen}\\

\haiku{Met mijn stemming was.}{mijn libido tot onder}{het nulpunt gezakt}\\

\haiku{het leek me alleen.}{beter als we allebei}{even tot rust kwamen}\\

\haiku{Ze hadden mij niet.}{eerder zo trefzeker op}{mijn plaats gewezen}\\

\haiku{koffers uitpakken,,.}{wat opruimen eenvoudig}{voedsel bereiden}\\

\haiku{Een diep geworteld.}{pessimisme moest eraan}{ten grondslag liggen}\\

\haiku{Toen hij de nachtmis,.}{uitkwam dacht hij dat hij het}{ergste gehad had}\\

\haiku{Laatst had hij nog een.}{paar knopen aangezet en}{zijn schoenen gepoetst}\\

\haiku{Daar liep een lange.}{weg tussen de pijnbomen}{door tot op het strand}\\

\haiku{dat hij geen vrije keus -.}{meer had bij weid gedwongen}{een stap te zetten}\\

\haiku{{\textquoteleft}Meneer Provenier,{\textquoteright}, {\textquoteleft}.}{zei hij onrustig snuivend}{doet u niets overhaast}\\

\haiku{Mijn bezigheden.}{en gedachten waren in}{\'e\'en keer stilgelegd}\\

\haiku{Vergis je niet - jij.}{hebt een staat van dienst om u}{tegen te zeggen}\\

\haiku{Alles wat je daar,.}{hebt bereikt mag je niet in}{\'e\'en keer weggooien}\\

\haiku{Ik zag voortdurend,,.}{hoe zij in haar standje knieen}{wijd bovenop zat}\\

\haiku{In deze laatste -.}{rit had zij haar ziel gelegd}{van 2000 cc inhoud}\\

\haiku{Mijn auto's waren.}{een waarborg dat ik nergens}{zou blijven steken}\\

\haiku{Ooit had mijn vader.}{mij verteld dat z{\'\i}jn vader}{nooit gelachen had}\\

\haiku{Elk woord dat op mijn,.}{zwakte duidde maakte mijn}{moeder radeloos}\\

\haiku{Die scheerspullen kun,{\textquoteright}.}{je weer meenemen was het}{eerste wat ze zei}\\

\haiku{De psychiater,.}{liet mij plaatsnemen in zijn}{stoel en bleef zelf staan}\\

\haiku{Natuurlijk hield zij.}{van mij en zou zij me nooit}{in de steek laten}\\

\haiku{{\textquoteright} Bewegingen van.}{benen en armen kon ik}{niet meer beheersen}\\

\haiku{De foto's van mijn.}{dochter en van h\'a\'ar prikte}{ik op het prikbord}\\

\haiku{De rokers kwamen.}{op hun elfendertigst rond}{de snoepkar drommen}\\

\haiku{Ik wilde bellen,.}{een verbinding leggen met}{de buitenwereld}\\

\haiku{{\textquoteright} {\textquoteleft}Als je een verzoek,.}{indient kan ik vragen of}{hij morgen tijd heeft}\\

\haiku{Zij komt onder de.}{douche vandaan en stapt in}{haar ciabatten}\\

\haiku{het linkerbeen recht,.}{uitgestrekt het rechter over}{de leuning gehaakt}\\

\haiku{Zij trok haar dikke.}{lippen weg om alles aan}{het licht te brengen}\\

\haiku{Zou ik het nog eens,.}{na kunnen vertellen dan}{bleef niets onvermeld}\\

\haiku{Zelfs voor de ergste.}{types liep ik plotseling}{over van sympathie}\\

\haiku{of als het hoppen,.}{van een overvoed konijn vlak}{voor de kerstdagen}\\

\haiku{Alles geregeld,,.}{natje en droogje aanspraak}{en mededogen}\\

\haiku{deze apen met een.}{overmaat aan sociale}{intelligentie}\\

\haiku{Wij mensen hadden.}{ons opgericht om naar de}{sterren te kijken}\\

\haiku{{\textquoteright} {\textquoteleft}Mevrouw, ik heb drie.}{kinderen en voorlopig}{twaalf kleinkinderen}\\

\haiku{Bijna had ik hem.}{mijn Italiaanse shampoo}{ook nog toegestopt}\\

\haiku{Die dingen doe je,.}{in de ergo of anders}{in een spreekkamer}\\

\haiku{omdat ik al mijn.}{oude correspondenties}{had afgebroken}\\

\haiku{Niet veel, de mensen.}{weten niet meer wat het is}{een brief te schrijven}\\

\haiku{van het andere.}{uiteinde was je evenwel}{pas als laatste weg}\\

\haiku{{\textquoteright} Gemakkelijk viel '.}{het me niets ochtends om}{twaalf uur warm te eten}\\

\haiku{We hebben toen nog.}{lang moeten zoeken naar}{een kurkentrekker}\\

\haiku{Natuurlijk was ik.}{solidair met mensen die}{ook wilden vallen}\\

\haiku{De enige die wel,,.}{eens voor mij kwam was Cindy}{zonder haar vriendin}\\

\haiku{{\textquoteright} Heel voorzichtig sneed {\textquoteleft}{\textquoteright}.}{zeg-alsjeblieft-Hella}{haar tafelplan aan}\\

\haiku{We mogen toch wel?}{\'e\'en of twee keer per week met}{onze arts spreken}\\

\haiku{Mijn ongeschoren;}{wangen hadden niets zieligs}{of onderworpens}\\

\haiku{Ik nam een appel.}{uit de fruitkrat en wreef die}{glimmend over mijn mouw}\\

\haiku{{\textquoteright} Wat kunnen mensen,.}{flink zijn onderweg naar hun}{eigen ondergang}\\

\haiku{Nog voor ik langs het,.}{kantoortje kwam schokten mijn}{schouders geluidloos}\\

\haiku{Op geen enkele.}{manier kon ik mijzelf tot}{bedaren brengen}\\

\haiku{Daar stond hij met zijn.}{twee valiezen en de doos}{bij de tramhalte}\\

\haiku{Ik dacht zo dat je.}{dan precies weer terugkomt}{op je uitgangspunt}\\

\haiku{{\textquoteright} Twee zones en een.}{overstapje waren genoeg}{om thuis te komen}\\

\haiku{Niet van die dunne}{tissues waarvan altijd een}{doos klaarstond in geval}\\

\haiku{Het leven dat hij.}{leiden moest als hij zich aan}{haar aan zou passen}\\

\haiku{omdat je daardoor.}{de dingen gaat zien zoals}{ze werkelijk zijn}\\

\haiku{Elke dag dat je.}{nog geld had uit te geven}{was meegenomen}\\

\haiku{Hij was weer terug,.}{uit de grote landen ver}{van de mooie dingen}\\

\haiku{Adriaan moest niet gaan.}{denken dat hij voor hem zou}{blijven koerieren}\\

\haiku{Ze zullen hun best.}{doen om te bewijzen dat}{het niet aan hen ligt}\\

\haiku{Om zichzelf wakker.}{te schudden begint hij aan}{haar kraag te trekken}\\

\haiku{hoe kon je weten?}{of iemand maar enigszins in}{de smaak zou vallen}\\

\haiku{Zijn knie\"en waren,,.}{stram zijn kin voelde ruw zijn}{kleren waren nat}\\

\haiku{Ondertussen was.}{het imperatief dat hij}{zijn blaas ledigde}\\

\haiku{Hoe vaker hij dat,.}{deed hoe gedecideerder}{de afwijzingen}\\

\haiku{Hij was al niemand,.}{meer dus zijn spullen konden}{het best de deur uit}\\

\haiku{Of had hij nog dat,?}{van gisteren aan waarin}{hij had geslapen}\\

\haiku{je bent, dat je de,.}{beste was dat het nog nooit}{zo goed is geweest}\\

\haiku{Ik ben net zo goed.}{en net zo slecht als alle}{andere mensen}\\

\haiku{Wanneer twee mensen,.}{bijeen zijn praten ze graag}{over  een derde}\\

\haiku{vier lieve meisjes.}{die zich onder die lampen}{over mij heen bogen}\\

\haiku{Ik schaam mij niet voor,.}{deze littekens en ben}{er ook niet trots op}\\

\haiku{Met mijn vrienden, en,.}{zeker met mijn exen had zij}{nooit contact gezocht}\\

\haiku{was ik te laf om?}{de zelfdoding tot een goed}{einde te brengen}\\

\haiku{Vroeger had het hier.}{aangenaam geroken naar}{boeken en tabak}\\

\haiku{Goedkoop was het ook,.}{niet bepaald maar gelukkig}{was ik verzekerd}\\

\haiku{Ik wist niet wat me,.}{overkwam zo hevig als ik}{door haar werd bemind}\\

\haiku{en deze tussen;}{novelty seeking en non}{novelty seeking}\\

\section{Johannes van Melle}

\subsection{Uit: Bart Nel, de opstandeling}

\haiku{Een meisje van een,.}{jaar of negen kwam binnen}{haar schort vol kuikens}\\

\haiku{{\textquoteleft}Maar dit s\^e ek vir,}{oom as Botha ons in die}{oorlog wil insleep}\\

\haiku{Fransina staarde.}{naar buiten en oom Giel zat}{maar stil te roken}\\

\haiku{wist al dat zij zich.}{de gehele morgen had}{lopen opwinden}\\

\haiku{Hy is niks meer as '.}{n werktuig in die hande}{van Engeland nie}\\

\haiku{Ek wens jy wil by.}{die huis bly en jou nie met}{die dinge moei nie}\\

\haiku{Telkens weer stond een.}{van de ouden op om tot}{kalmte te manen}\\

\haiku{Maar dat gaf je nog.}{niet het recht om tegen het}{gezag op te staan}\\

\haiku{{\textquoteright} {\textquoteleft}Jy wil in die veld.}{gaan en oorlog maak en jou}{boerdery verwaarloos}\\

\haiku{of zijn kaffers wel.}{werkten en of er geen vee}{in de landen kwam}\\

\haiku{Je hoort veel in een.}{enkel woord en je begrijpt}{veel uit een zwijgen}\\

\haiku{Ze zei niet veel meer,.}{dekte met Annekie en}{de meid de tafel}\\

\haiku{Sy kan miskien praat.}{en dinge uitbring wat my}{planne verydel}\\

\haiku{Twee dagen later,,.}{nog voor de zon op was liep}{Bart al in zijn land}\\

\haiku{het ek 'n kaffer ' '.}{opn bycicle metn}{brief na hom gestuur}\\

\haiku{Gisteraand laat kom.}{die kaffer toe terug en}{hier is die antwoord}\\

\haiku{Maar trachten, hem tot.}{andere gedachten te}{brengen deed ze niet}\\

\haiku{Hij kon het nu toch.}{niet meer geheim houden en}{het hoefde ook niet}\\

\haiku{{\textquoteright} {\textquoteleft}Maar as ek iets vir;}{jou kan doen terwijl jij van}{die huis af weg is}\\

\haiku{Ek wil julle nie{\textquoteright},.}{verbloem nie dat die toestand}{ernstig is zei hij}\\

\haiku{Nou sal die Jood mos.}{partij goed byskryf en}{pryse verander}\\

\haiku{{\textquoteleft}Ek wonder of ons{\textquoteright},.}{dan maar nie eers na hom toe}{moet ry nie zei Bart}\\

\haiku{Die motorkarre.}{spioen ons en hulle weet}{presies waar ons is}\\

\haiku{Het kommando hield.}{stil en de paarden draaiden}{hun rug naar den wind}\\

\haiku{Na de dienst zocht hij.}{zijn ligplaats een eindje van}{de anderen af}\\

\haiku{Aan de luchtkimmen}{schoten telkens bliksems en}{een enkele maal}\\

\haiku{Op een  afstand.}{leek het of het den grond met}{een kleed bedekte}\\

\haiku{Hier is twee k\^erels,{\textquoteright},.}{van die kommando Venter}{kommandant zei hij}\\

\haiku{{\textquoteleft}Gauw{\textquoteright}, zei Fransoois en.}{zij joegen naar een ruig stuk}{blauwe haakdorens}\\

\haiku{Ons weet nou dat die.}{regeringsmense weer ons}{spore gekry het}\\

\haiku{{\textquoteleft}Nee wat{\textquoteright}, vertelde, {\textquoteleft}.}{de een mismoedigdaar is}{geen rebellie nie}\\

\haiku{Ons kan die Vader.}{dank dat die rebellie so}{gou doodgeloop het}\\

\haiku{Ek is bly dat Bart{\textquoteright},.}{eindelik rede verstaan}{zei de kommandant}\\

\haiku{{\textquoteright} {\textquoteleft}Elkeen moet self weet, '.}{wat hy doen maar ek vind dit}{n verkeerde ding}\\

\haiku{Sal oom bietjie sit.}{dan sal ek gou bietjie vir}{oom koffie ingooi}\\

\haiku{{\textquoteleft}Kornelia vra of;}{jy nie liewer in die nag}{by ons wil wees nie}\\

\haiku{Hij zag hoe kwaad zij,.}{was en stond op gaf haar koel}{de hand en reed heen}\\

\haiku{{\textquoteleft}die huis beheer en.}{dan nog buitekant sake}{behartig daarby}\\

\haiku{{\textquoteright} Maria's gezicht,.}{werd geheel rood tot haar hals}{en oren waren rood}\\

\haiku{{\textquoteright} {\textquoteleft}As Bart net weer by{\textquoteright},.}{die huis is sal alles wel}{regkom troostte hij}\\

\haiku{Dit moet ellendig.}{wees as jy in so'n tyd van}{mekaar verskil}\\

\haiku{Of sou hulle so?}{kwaad wees vir mekaar dat hy}{nie  wil skryf nie}\\

\haiku{Dat was natuurlik,.}{een vergissing de man had}{een verkeerde voor}\\

\haiku{Nu eerst bemerkte,.}{zij hoe fel zij gehoopt had}{Bart terug te zien}\\

\haiku{Dat zijn brief te laat,.}{gekomen was maakte voor}{hem weinig verschil}\\

\haiku{wie het was zag hij.}{haar achter onder de tent}{zitten en hield stil}\\

\haiku{{\textquoteright} {\textquoteleft}Sal ek nie vir jou?}{wag nie en jou gou na huis}{ry met die motor}\\

\haiku{Zacht, om hem niet te,.}{wekken knapte zij zich in}{haar kamer wat op}\\

\haiku{Zij deed verkeerd, gaf.}{de mensen aanleiding om}{over haar te praten}\\

\haiku{Toen hij bij haar hek.}{afsteeg kwam Fransina naar}{buiten gelopen}\\

\haiku{{\textquoteleft}Dag Fransina{\textquoteright}, zei.}{hij op een toon alsof hij}{haar wilde troosten}\\

\haiku{Die brief oor Basson ',.}{wasn onvergeeflike fout}{maar Bart sal verstaan}\\

\haiku{Sy moet hom nou maar.}{opreg alles skryf en dan sal}{dinge weer regkom}\\

\haiku{{\textquoteright} {\textquoteleft}Nee eintlik nie, maar,.}{tog Japie van Martha was}{gister daar by ons}\\

\haiku{Wat een giftige,.}{bende mensen was dat toch}{die buren van haar}\\

\haiku{Jy moet self maar kos,.}{vir jou opskep ek wil}{perbeer bietjie slaap}\\

\haiku{Het was er heet en}{benauwd en lui lagen ze}{allen uitgestrekt}\\

\haiku{{\textquoteleft}Jammer{\textquoteright}, dacht hij, {\textquoteleft}dan.}{sou ek nou iets gehad}{het om in te lees}\\

\haiku{Het was een man, die.}{in de buurt van Mooiplaats een}{kleine winkel had}\\

\haiku{Ag meneer Nel, jou{\textquoteright},.}{regering maak darem nie}{goed nie zei de Jood}\\

\haiku{het ges\^e dat as.}{die gras in sy sate is}{alles sal oor wees}\\

\haiku{{\textquoteright} vroeg hij op een dag,.}{nadat ze daartoe verlof}{verkregen hadden}\\

\haiku{{\textquoteright} Toen hij de brief van,.}{Malherbe kreeg liet hij oom}{Gawie die lezen}\\

\haiku{Ze kregen hout en.}{gereedschap en zaagden en}{schaafden de tijd om}\\

\haiku{Oom Gawie had een,,.}{jaar gekregen en ging toen}{zijn tijd om was heen}\\

\haiku{{\textquoteleft}Ons rebelle het{\textquoteright},, {\textquoteleft}.}{swaar gekry zei hijmaar ons}{moet probeer vergeef}\\

\haiku{Japie kan tog nie.}{daarvan maak wat hy daarvan}{behoort te maak nie}\\

\haiku{{\textquoteright} {\textquoteleft}Dan wou ek sommer,.}{vandag nog ry so gou as}{Japie terug is}\\

\haiku{Alles wat zij deed.}{ging alsof het geen moeite}{of overleg kostte}\\

\haiku{Hij legde even zijn.}{hand op de hare en zij}{glimlachte hem aan}\\

\haiku{Dat was nu eenmaal.}{voorbij en zij was nu de}{vrouw van Ferdinand}\\

\haiku{Het was evenwel toch.}{bijna half tien toen hij op}{Bassons Rust aankwam}\\

\haiku{Hij reed zijn auto}{de garage in en toen}{hij terugkwam stond}\\

\haiku{{\textquoteleft}Jammer dat ek tog{\textquoteright},.}{maar nie na swaer Jan toe gery}{het nie dacht hij}\\

\haiku{Zij zag het ineens,.}{alsof hij het zelf haar in}{woorden had gezegd}\\

\haiku{Het was nog vroeg en.}{na een nacht van regen woei}{er een koele wind}\\

\haiku{Die beste is om '{\textquoteright},.}{haar nou maarn tydjie aan haarself}{oor te laat dacht hij}\\

\haiku{As sy nog altyd,.}{meer van hom hou as van my}{laat haar dan maar gaan}\\

\haiku{'n Vrou wat by my '.}{is en aann ander dink}{wil ek nie h\^e nie}\\

\haiku{Maar hij vermoedde,.}{toch ook dat zij hem liever}{zag gaan dan komen}\\

\haiku{Dan hoop ek dat hy{\textquoteright},.}{haar terugvat en dat hy}{haar hel gee dacht hij}\\

\haiku{Anders sal jy tog.}{maar langs die pad moet stilhou}{om iets te gebruik}\\

\haiku{Sy s\^e hy doen wat.}{hy kan om haar die lewe}{aangenaam te maak}\\

\haiku{Hij stak zijn pijp aan.}{de kaars aan en geraakte}{diep in gedachten}\\

\haiku{het, wat se oneer?}{steek dan daarin om jou eie}{ouers te gaan vra}\\

\haiku{dit moet ek eerlik,.}{s\^e al kan ek die vent ook}{nie meer verdra nie}\\

\haiku{{\textquoteleft}Van Fransina het.}{ek gehou soos min mans van}{hulle vrouens hou}\\

\haiku{Soms als zij tegen,.}{hem aangeleund zat had zij}{tranen in haar ogen}\\

\haiku{Nou my kind, dat ek.}{nog baie baie jare op jou}{gesondheid mag drink}\\

\haiku{{\textquoteleft}'n Kind is darem ' '{\textquoteright},.}{n groot aantreklikheid in}{n huis zei Maritz}\\

\haiku{Jy kan haar maklik.}{elke week of eenmaal in}{die veertien dae sien}\\

\haiku{Ze ging weer tegen;}{haar tante aan staan en keek}{naar de gezichten}\\

\section{Josepha Mendels}

\subsection{Uit: Als wind en rook}

\haiku{{\textquoteright} {\textquoteleft}Visite of niet,{\textquoteright}, {\textquoteleft}.}{zei hij weervoor ons is het}{vandaag een werkdag}\\

\haiku{Herman zorgt nu voor,.}{zichzelf dat maakt het mij wel}{gemakkelijker}\\

\haiku{Maar ja, hij had me,...}{niet gezoend mij niet in zijn}{armen genomen}\\

\haiku{En wat wil je, dan.}{ben je meteen weer eens in}{Amsterdam geweest}\\

\haiku{{\textquoteright} {\textquoteleft}Je haar,{\textquoteright} antwoordde, {\textquoteleft}.}{hij toenalleen je haar heb}{je van je moeder}\\

\haiku{Ik probeerde op,,.}{te staan maar het lukte niet}{de stoel was te diep}\\

\haiku{Simon vroeg mij op.}{te staan en bracht me naar zijn}{warme bed terug}\\

\haiku{Hij zoende haar zo.}{nu en dan en de nacht bracht}{hem knapendromen}\\

\haiku{{\textquoteleft}Als een mens alles.}{vooruit wist zou hij heel veel}{dingen anders doen}\\

\haiku{Meer zal voorlopig,.}{niet mogelijk zijn want dan}{wordt het huis te klein}\\

\haiku{{\textquoteleft}Nu zal ik dan toch,{\textquoteright}.}{eindelijk mijn vrouw van je}{maken verdween het}\\

\haiku{{\textquoteright} {\textquoteleft}O,{\textquoteright} antwoordde ik, {\textquoteleft}...{\textquoteright} {\textquoteleft},}{het arme schaapLaten we}{er iets op drinken}\\

\haiku{Ik liet hem begaan,.}{zoals ik van de eerste}{keer af gedaan had}\\

\haiku{{\textquoteleft}Want weet je, Hoefstar,.}{ik moet die zelf vanavond in}{de trein meenemen}\\

\haiku{{\textquoteleft}Het is nu drie uur,.}{voor de melkboer belt zal het}{wichtje er wel zijn}\\

\haiku{Voor zover Simon.}{zich met haar bemoeide was}{hij heel lief voor haar}\\

\haiku{Judith leek heel veel,;}{op Simon was donker en}{had zijn bruine ogen}\\

\haiku{Het was de kiem van.}{haat voor Simon die in mijn}{handen opwelde}\\

\haiku{je vond sommige,.}{even mooi als weer andere}{even middelmatig}\\

\haiku{{\textquoteright} antwoordde Elisa, {\textquoteleft}.}{ik ben zo gewend aan mijn}{eigen instrument}\\

\haiku{wanneer Judith nog,.}{zoiets gezegd zou hebben}{maar dit stille kind}\\

\haiku{{\textquoteleft}Hij is al weg,{\textquoteright} zei,.}{ik lachend en toen zijn wij}{gaan musiceren}\\

\haiku{Elisa had veel werk.}{om zijn garderobe in}{orde te brengen}\\

\haiku{in strijd met Simons.}{opvattingen geeft Elisa}{een ander gezicht}\\

\haiku{Na dit verblijf aan.}{zee gingen wij nog een week}{naar mijn ouders toe}\\

\haiku{Het zou de eerste.}{keer zijn dat ze haar na de}{vakantie weer zag}\\

\haiku{De denneboom keek.}{door het halfgeopende}{gordijn naar binnen}\\

\haiku{Als mijn moeder nog,?}{leefde wat zou zij daarvan}{wel gezegd hebben}\\

\haiku{Als mijn moeder nog,?}{leefde wat zou ze wel van}{dit meisje denken}\\

\haiku{{\textquoteleft}U bent zo blootshoofds.}{toch veel leuker en die hoed}{maakt u nog ouder}\\

\haiku{Zij ruimde af, trok.}{hem zijn jasje uit en bond}{hem een vaatdoek voor}\\

\haiku{{\textquoteright} riep een stem, en een.}{arm strekte zich uit naar een}{lampje naast het bed}\\

\haiku{Minouche was de.}{tweede vrouw in zijn leven}{die dit had gedaan}\\

\haiku{Men heeft mij verteld.}{dat het zo buitengewoon}{interessant is}\\

\haiku{En boven op dat.}{blanke wit lagen zelfs wat}{oude broodkruimels}\\

\haiku{Ik verbeeldde me,...}{toen dat ik ongelukkig}{was ongelukkig}\\

\haiku{De hangklok slaat juist,.}{tien uur de kamer is reeds}{keurig opgeruimd}\\

\haiku{De dag ligt open voor,.}{me ik kan eten wanneer ik}{wil en wat ik wil}\\

\haiku{Maar het is al zo,...}{lang geleden dat hij dit}{gedaan heeft zo lang}\\

\haiku{Of denk je dat je,?}{te oud bent om kinderen}{te krijgen Louise}\\

\haiku{{\textquoteright} zegt ze dan, {\textquoteleft}of je.}{vanavond weer de seider bij}{ons komt doorbrengen}\\

\haiku{{\textquoteright} {\textquoteleft}Maar als ze een kind,?}{moet krijgen hoe kan ze dan}{nog dikker worden}\\

\haiku{{\textquoteright} {\textquoteleft}En heb je het niet,,?}{te druk zo alleen met een}{dagmeisje Elisa}\\

\haiku{Nu zegt ze opeens.}{dat dit het enige is waar}{ze plezier in heeft}\\

\haiku{Heeft Simon zijn werk?}{voor het concours ingestuurd}{en maakt hij een kans}\\

\haiku{ik ben er, Richard,.}{Palmers ik zou hem zo graag}{terug willen zien}\\

\haiku{Voor Richard is er,.}{altijd plaats al zou het maar}{in haar schooltas zijn}\\

\haiku{Nu legt ze alle... {\textquoteleft}}{uitgescheurde bladzijden}{op haar schoot en wacht}\\

\haiku{Hij geeft het haar en.}{zij lacht en likt met haar tong}{die boter eraf}\\

\haiku{{\textquoteleft}Ik heb ook tot tien,.}{jaar met poppen en beren}{gespeeld Rebecca}\\

\haiku{Maar je ziet er niets,,.}{van integendeel het kind}{lijkt wel opgelucht}\\

\haiku{Vanzelfsprekend kan...}{niemand mij beletten niet}{aan hem te denken}\\

\haiku{het daar altijd zo,.}{fris ik herinner mij niet}{precies meer naar wat}\\

\haiku{Maar je weet zelf hoe,.}{dat gaat de helft zal er wel}{van gelogen zijn}\\

\haiku{Maar dat geklets gaat,,.}{vanzelf weer over Simon dat}{begrijp je ook wel}\\

\haiku{{\textquoteright} En hij duwde me.}{achter het scherm opdat ik}{mij zou uitkleden}\\

\haiku{Het hield niet op, het.}{leek op het laatst bijna of}{hij ze zelf verzon}\\

\haiku{Hoe kan ik u de?}{stemming uitbeelden die er}{op vrijdagavond heerste}\\

\haiku{Ik huilde bijna.}{dat ik niet groter was en}{mijn stem niet sterker}\\

\haiku{Zou Klazien willen,?}{blijven en zal Simon de}{kinderen houden}\\

\haiku{Want voor mij is die.}{met het vertrek van Elisa}{reeds voorgoed voorbij}\\

\haiku{De groeten aan je,{\textquoteright}, {\textquoteleft}.}{vader zei Louiseik kom}{wel gauw weer eens langs}\\

\haiku{Jij blijft dus maar hier.}{in ons midden en vertrekt}{niet naar het zuiden}\\

\haiku{Eerst wist hij er geen,:}{antwoord op maar toen werd hij}{er zich van bewust}\\

\haiku{Maar zijn goede oog.}{is nu ook aangetast en}{hij ziet steeds slechter}\\

\haiku{Misschien zal hij pas;}{bij mij komen wanneer hij}{geheel blind zal zijn}\\

\haiku{hij komt dan als een,.}{hulpeloze maar dat zal}{hij zelf niet weten}\\

\haiku{{\textquoteright} Nu echter kijkt ze,.}{vol vertrouwen naar Richards}{kleine blauwe ogen}\\

\haiku{{\textquoteleft}Het sneeuwde zo bij,.}{ons het was er koud en hier}{is het heerlijk warm}\\

\haiku{Met sangh of spel haer, '?}{man verquickt Alst nodigh}{huyswerk is beschikt}\\

\section{Victor de Meyere}

\subsection{Uit: Langs den stroom}

\haiku{t Was de eenige;}{maal dat hij op zwier ging en}{hij deed het dan goed}\\

\haiku{Sneller liep hij voort,,.}{altijd maar sneller alsof}{hij achtervolgd werd}\\

\haiku{Ook vong hij muizen.}{die hij hem dan met ware}{zelfvoldoening bracht}\\

\haiku{Hij herinnerde.}{zich nog de eerste muis die}{hij gevangen had}\\

\haiku{De uil zat in 't.}{midden van de kooi op eenen}{poot en pinkoogde}\\

\haiku{hij stak ze in de,,.}{kooi een voor een den arm in}{de lange slobkous}\\

\haiku{'t reutelde maar,.}{even over zijne tong om ze}{niet te doen schrikken}\\

\haiku{'t Duurde echter '.}{niet lang of hij ging weer aan}{t prakkezeeren}\\

\haiku{Toen  werd het weer;}{duister in hem en duister}{v\'o\'or zijne oogen ook}\\

\haiku{De boeien nepen ' ';}{hem diep int vleesch en}{t bloed stond er v\'o\'or}\\

\haiku{De uil werd wakker,.}{zette zich wat verder en}{dommelde weer in}\\

\haiku{Hij lei zich terug.}{op de brits met de handen}{onder  den kop}\\

\haiku{En hij huiverde.}{van de gedachte die den}{lierenaar opriep}\\

\haiku{Ge moogt gerust zijn.}{en spreken als tegen uw}{eigen moeder t'huis}\\

\haiku{Duidelijk hoorden ';}{wijt lawaai van het volk}{op de kasseide}\\

\haiku{'t Was gedrest tot.}{in zijn haar dat met lange}{klissen saamplakte}\\

\haiku{Door 't veel werken.}{van den morgen tot den avond}{verkort hun leven}\\

\haiku{Had hij het zelf niet?}{gedaan toen hij dacht dat de}{tijd gekomen was}\\

\haiku{Z\'o\'o gauw hij maar kon.}{kroop hij onder de dekens}{in de groote alkoof}\\

\haiku{En 's morgens werd,.}{hij wakker met het eerste}{gekraai van den haan}\\

\haiku{de borst blaasbalgde,.}{op en neer om een beetje}{asem op te vangen}\\

\haiku{{\textquoteleft}ge zult het hier goed,{\textquoteright}.}{hebben in uwen ouden dag}{verzekerde zij}\\

\haiku{- En toch is het z\'oo,,....}{peinsde hij voort ik kan die}{pree niet verdienen}\\

\haiku{ik wacht niet langer,,.}{zegde hij terwijl hij zich}{langzaam aankleedde}\\

\haiku{En dan, ik weet het,:}{immers toch mijn geld kan ik}{niet meer verdienen}\\

\haiku{Hij wist dat het voor.}{altijd was en dat hij niet}{meer zou wederkeeren}\\

\haiku{Toke schreef het nieuws.}{in een langen brief aan den}{meester van de werf}\\

\haiku{Er vielen er in ',,.}{t water niet ver van de}{boot met luid geplons}\\

\haiku{men vertelde goed,,.}{en kwaad alles ondereen}{maar kwaad wel het meest}\\

\haiku{Als men hem terug ',.}{naart prison bracht voelde}{hij zich gelukkig}\\

\haiku{Hij wierp zich te bed,, '.}{en sliep in een zwaren roes}{tots morgens toe}\\

\haiku{'t waren altijd.}{gevangenen die men met}{dat werk gelastte}\\

\haiku{'t Zou hem overal,!}{blijven vervolgen overal}{waar hij komen zou}\\

\haiku{dat is niet goed voor.}{u. Trek ergens naar een vreemd}{dorp als schoenmaker}\\

\haiku{Heel den Rupelkant,,.}{liep hij af dorp na dorp maar}{nergens vond hij iets}\\

\haiku{'t Was nu weeral,?}{aan den waterkant maar wat}{kon hij er aan doen}\\

\haiku{Een Charleroische {\textquoteleft}{\textquoteright}.}{bakDen Jongen Jan volgde}{den beurtschipper op}\\

\haiku{Hij sloot de deur en '.}{ging op den driepikkel aan}{t venster zitten}\\

\haiku{- En dan, - ging hij voort, -,.}{die kerel daar heeft niet}{veel op te vragen}\\

\haiku{Hij moest maar weggaan,,....}{zoo gauw mogelijk naar een}{ander dorp ver weg}\\

\haiku{Een pijl uit eenen boog '!}{kon hem met zoo'n geweld niet}{int hart snorren}\\

\haiku{Ge zijt gij Andries,?}{Eyckmans geboren te}{Wintham niet waar}\\

\haiku{t briefken weer te.}{voorschijn en ontplooide het}{voor de tweede maal}\\

\haiku{Ge waart gij bij de,,?}{bende der woldieven over}{een jaar of vier he}\\

\haiku{Hongerig viel hij ',}{aant eten een korst brood van}{den proviand dien}\\

\haiku{Dezen nacht, als 't,.}{dorp in slaap lag zou hij er}{mede vertrekken}\\

\haiku{Dat ge beter moogt,,.}{varen Dries jongen dat ge}{beter moogt varen}\\

\haiku{hoe dieper en schooner,,.}{van kleur het werd als donker}{blauwachtig fluweel}\\

\haiku{Zij hadden ook hun.}{vreugde aan de fantasie}{van de kinderen}\\

\haiku{Als de kalkman is,;}{geboren Gaan we naar Dort}{om kaf en koren}\\

\haiku{En de zatlap - ik -;}{moet het zeggen heeft het niet}{lang meer getrokken}\\

\haiku{Als hij buitenkwam,,...}{zette de ziekte zich op}{hem vast ineens d\'aar}\\

\haiku{Op den stroom, volgde.}{hij de wondere speling}{van licht en schaduw}\\

\haiku{overal in heel het, ' '.}{land was er wat gaandet}{een oft ander}\\

\haiku{Men ontwaarde nog,.}{alleen een wit mat streepken}{tusschen de wimpers}\\

\haiku{Een glimlach zweefde,.}{om zijnen mond alsof hij}{zich verlicht voelde}\\

\haiku{Er lag zooveel leed.}{in dat enkel woord dat de}{moeder niet aandrong}\\

\haiku{Jef was eens lange,.}{maanden kwaad geweest op haar}{door haar eigen schuld}\\

\haiku{Maar eens de eerste,.}{bangheid voorbij bleef het eene}{leute heel den dag}\\

\haiku{Eens, toen een schip van,.}{stapel liep gaf men een fooi}{voor de werklieden}\\

\haiku{De moeder hief 't.}{hoofd op en hoorde zijne}{stappen versterven}\\

\haiku{de koster heeft me.}{gezeid dat ik u maar een}{handje moest helpen}\\

\haiku{daverde van 't.}{geweld waarmede de wind}{over het dak scheerde}\\

\haiku{- En dat het tijdens,.}{zoo'n onweer gebeuren moest}{vervolgde Nette}\\

\haiku{Zij gunde zich geen.}{oogenblikje rust en vond}{altijd een doening}\\

\haiku{Monica, nu zij,.}{weer alleen was luisterde}{angstig naar den storm}\\

\haiku{'t was of hij, in,...}{zwaren stormloop gestadig}{het huis bebeukte}\\

\haiku{Zij viel hem aan den.}{hals en kuste hem in een}{wilde omarming}\\

\haiku{Weer greep hij 't lijk,.}{op met moede handen die}{van pijn tintelden}\\

\haiku{Groote, lange droppen.}{zweepte de wind als hagel}{in zijn aangezicht}\\

\haiku{Hij voelde alleen,.}{de vochtigheid onder de}{voeten in het gras}\\

\haiku{Het leek een hortend.}{gevaarte dat naderde}{in een snellen rit}\\

\haiku{Met luide gillen,.}{snikte hij het uit in een}{stortvloed van tranen}\\

\haiku{Ik weet niet hoe ik,!}{tot hier ben geraakt met dat}{lijk in mijn armen}\\

\haiku{Het water kwam en.}{ik ben gaan vluchten met het}{lijk van mijn vader}\\

\haiku{Z\'oo geraakte ik '.}{t water v\'o\'or en kon ik}{den dijk bereiken}\\

\haiku{Maar eerst twee borrels,...}{klare jenever een voor}{u en een voor mij}\\

\haiku{En toch, als hij recht,.}{voor zich uitkeek scheen Neel niets}{vreemds te bemerken}\\

\haiku{De menschen van het.}{dorp stonden er met heelder}{hoopen saamgeschaard}\\

\subsection{Uit: De Vlaamsche vertelselschat. Deel 1}

\haiku{dat lekker soepken,,.}{kermde het manneken ik}{heb toch zoo'n honger}\\

\haiku{Hoe het ook poogde,.}{het kon zijn soep maar niet tot}{in zijn mond brengen}\\

\haiku{Nu, aanstaanden keer}{is het mijne beurt om te}{blijven en luiden}\\

\haiku{Begeef u zoo gauw;}{mogelijk naar de opening}{van deze spelonk}\\

\haiku{En als ge 't nu,.}{nogmaals probeeren moest sla ik}{u dood als ne pier}\\

\haiku{- Maar ik, met mijn vier,;}{pooten zal dan veel vroeger}{aankomen dan gij}\\

\haiku{- Als ge dan toch naar,.}{de stad meewilt kruipt dan maar}{in mijn achterste}\\

\haiku{'t Halfhaantje was.}{uiterst tevreden en liet}{het zich goed smaken}\\

\haiku{In een ommezien.}{waren al de schapen dood}{en opgevreten}\\

\haiku{- Water, kom uit mijn.}{achterste en zet het hier}{allemaal onder}\\

\haiku{Ik heb hier iets in.}{mijnen zak dat ik daartoe}{wel bezigen kan}\\

\haiku{De herder trok blij.}{met zijn tafeltje weg en}{de prins sprong te paard}\\

\haiku{{\textquoteleft}Knuppel sla den hoed{\textquoteright},.}{in den zak zei hij en het}{gebeurde alzoo}\\

\haiku{En toen de prins hem,.}{uitnoodigde mee te eten liet}{hij zich niet wachten}\\

\haiku{Hij vermoedde wel.}{wie de moordenaar was en}{werd woedend van toorn}\\

\haiku{Maar de jonge prins:}{kreeg het in de mot en riep}{op den kluppelzak}\\

\haiku{En hij ging recht door,,,.}{van dorp tot dorp van stad tot}{stad van land tot land}\\

\haiku{- Dat is ook al niet,.}{slecht geantwoord vervolgde}{de kapitein}\\

\haiku{Ik heb zin in uwen,.}{snuit maar zoo meteen nemen}{wij u nog niet mee}\\

\haiku{En de kapitein,.}{die ook kon zijn oogen van de}{prinses niet houden}\\

\haiku{Z\'o\'o werd hij van den.}{hoogsten berg van de stad naar}{beneden gerold}\\

\haiku{t Geraamte ging.}{de groote zaal uit tot in de}{gang en Jan volgde}\\

\haiku{Zij stonden v\'o\'or een,.}{arduinen draaitrap die naar}{den kelder leidde}\\

\haiku{Te dien einde moest.}{ik alle nachten op het}{kasteel gaan spoken}\\

\haiku{De duivels van de.}{hel hadden mij de grootste}{wreedheid opgelegd}\\

\haiku{Ge ziet het he, hoe?}{ge door een oog van een}{naald gekropen zijt}\\

\haiku{Jan Snoef en Kroes*~         .}{Er was eens een jongen en}{die heette Jan Snoef}\\

\haiku{om u het geheim,,.}{dat hij mij toevertrouwde}{mede te deelen}\\

\haiku{Knechten hebt gij tot,.}{uwe beschikking meer dan er}{u dienen kunnen}\\

\haiku{Hij nam het schip van.}{zijne schouders en zette}{het in het water}\\

\haiku{En daar hoorde hij.}{opeens in de verte een}{schrikkelijk gevecht}\\

\haiku{En zonder wachten.}{deed Hans een nieuwen wensch en}{hij werd ook een duif}\\

\haiku{- Beter iets dan niets,.}{al is dat iets dan ook zoo'n}{krimpieperig ding}\\

\haiku{De visscher moest bij:}{zijn thuiskomst het vischje in}{drie stukken snijden}\\

\haiku{Stijg van uw paard en}{neem uw zwaard in de hand. En}{zoo deed Gouden Lok}\\

\haiku{Zijn ouders zouden.}{kunnen nagaan wat er met}{hem geschieden zou}\\

\haiku{En de tooverheks was.}{toen wel verplicht van de nood}{eene deugd te maken}\\

\haiku{vroeg de ridder aan.}{den type en begon hem}{de mauw te veegen}\\

\haiku{De duivel bleef daar.}{maar staan blinken en zag ze}{niet buiten komen}\\

\haiku{De duivel trok al.}{jankende af en liet ne}{geweldigen vloek}\\

\haiku{'t zoog den man de '.}{hersens uit den kop ent}{bloed uit het hart}\\

\haiku{Een oude heer kwam.}{opendoen en die vroeg hem wat}{er hem beliefde}\\

\haiku{- Jongen, zegde hij,,?}{nogmaals ge weet wat ik u}{gezegd heb niet waar}\\

\haiku{- Weet gij wel, zei het,?}{paard dat het hier voor u een}{verboden plaats is}\\

\haiku{Nu, de meester heeft,.}{een paard dat sneller loopt dan}{ik maar dat is niets}\\

\haiku{een doornenhaag zoo.}{dicht begroeid dat mensch noch dier}{er doordringen kon}\\

\haiku{Hij liep te allen.}{kant om hem te vinden en}{de hand te drukken}\\

\haiku{- Zie maar eens hoe ik,.}{werpen kan en hij wierp den}{vogel in de lucht}\\

\haiku{En Jan Onversaagd.}{sprong zoo hoog hij maar kon en}{de tak deed de rest}\\

\haiku{De tijden zijn zoo}{lastig en onzeker en}{elke sterke man}\\

\haiku{Maar het zwijn liep veel.}{harder dan hij en haalde}{hem maar altijd in}\\

\haiku{gelukte ik er.}{in een van de reuzen bij}{de beenen te grijpen}\\

\haiku{Jan, echter, had al.}{spoedig in de gaten wat}{er op handen was}\\

\haiku{Als zij allemaal:}{binnen waren begon hij}{luidop te droomen}\\

\haiku{En Duimken, terwijl, '.}{al zijn broeders reeds ronkten}{aant luisteren}\\

\haiku{En met een kroontje.}{op zijn eigen hoofd lei hij}{zich daarna te rust}\\

\haiku{Met heel veel moeite.}{gelukte het hem nog uit}{de teil te kruipen}\\

\haiku{Voortaan was zij rijk,.}{en werken deed zij niet meer}{evenmin als Duimken}\\

\haiku{Ge kunt er van avond.}{met de schoolmeesteres eens}{lekker van smullen}\\

\haiku{Krak, krak, krak ging het.}{maar altijd bij elk woord dat}{over haar lippen kwam}\\

\haiku{Deze week is het.}{lot op de dochter van den}{koning gevallen}\\

\haiku{en Pakt en verslindt.}{en Vecht en overwint hielden}{dapper voet bij stek}\\

\haiku{En het arm boerken,.}{kwam naar voren schuchter en}{meer dood dan levend}\\

\haiku{- Maar zeg mij nu eens,,?}{vroeg hij verder wie heeft u}{dat ingeblazen}\\

\haiku{dat is zeker, en;}{staat het niet geschreven dan}{heb ik het gedroomd}\\

\haiku{- Vaarwel dan, zei de,!}{oude man u ook zal ik}{niet meer wederzien}\\

\haiku{En toen, onder 't,:}{opsmullen van zijn prooi zei}{de draak voortdurend}\\

\haiku{Hij sneed den buik van.}{den draak open en haalde den}{lever te voorschijn}\\

\haiku{De verloren zoon*.}{Er was eens een man en}{die had twee zonen}\\

\haiku{ik heb opgepast.}{terwijl mijn broeder alles}{vertierelierde}\\

\haiku{- Wacht, zei de moeder,.}{ik zal dien stok eens roepen}{om den hond te slaan}\\

\haiku{En hij meende al.}{gelijk dat zij een verzoek}{deed aan de H. Maagd}\\

\haiku{Zij ging en ging en,.}{bleef gaan tot zij aan den voet}{van een hoogen berg kwam}\\

\haiku{- Liever dan hier te,!}{blijven staan wil ik in een}{reiger vergaan}\\

\haiku{Tk zou ook gaarne.}{van dat leelijk ding op mijn}{rug afgeraken}\\

\haiku{Zij bedankten het.}{Manneken voor al wat hij}{voor hen had gedaan}\\

\haiku{Ge kunt denken hoe.}{het Manneken Miserie}{er oin lachen moest}\\

\haiku{Ik ben te ziek en.}{te kramakkelijk om in}{den boom te kruipen}\\

\haiku{Zie, sinds de halve!}{uur dat ik hier zit sterven}{er geen menschen meer}\\

\haiku{De draak spartelde,.}{nog een stondeken maar viel}{daarop mors-dood}\\

\haiku{Daar ze eene tooveres,.}{was zou ze de prinses wel}{uit den weg ruimen}\\

\haiku{Nog had de jongen;}{niet het minste nieuws over zijn}{verloren prinses}\\

\haiku{Allen kwamen weer.}{zonder inlichtingen en}{zonder de prinses}\\

\haiku{Aan den haard zat een,.}{oude kluizenaar die zijn}{paternoster las}\\

\haiku{V\'o\'or den morgen zult.}{gij uw stiefmoeder levend}{moeten verbranden}\\

\haiku{Hij ging maar altijd.}{door want hij wilde naar het}{eind van de wereld}\\

\haiku{Op mijn smeeken heeft.}{hij me een laatste uitstel}{van drie maand verleend}\\

\haiku{De pastoor liep zeer.}{naar de pastorij en Jan}{den Dief achter hem}\\

\haiku{dacht hij dat het waar.}{gebeurde en hij op reis}{naar den hemel ging}\\

\haiku{- Laat maar gaan, ging Jan,.}{den Dief voort klagen  op}{voorhand helpt geen zier}\\

\haiku{Een sukkelaar van,,.}{een pelgrim gelijk ik zal}{uw paard niet stelen}\\

\haiku{- Dat is geen refuus,.}{aan u die z\'o\'o braaf en goed}{zijt geweest voor mij}\\

\haiku{Haal maar water en.}{wat gauw of er worden geen}{koeken gebakken}\\

\haiku{Wanneer ge dan nog.}{te biechten komt hebt ge maar}{keitjes te tellen}\\

\haiku{Naarmate 't vlas.}{opgeraakte bracht hij een}{nieuwen voorraad aan}\\

\haiku{De overeenkomst was.}{gesloten en de duivel}{wipte de deur uit}\\

\haiku{Hij vloekte en liep....}{haver-abaver de deur}{uit        XLVIII}\\

\haiku{Kom met mij naar de.}{stad en ik zal er u een}{deel van meegeven}\\

\haiku{- Ta, ta, ta, 't is,,!}{al meer dan goed En dan ik}{heb geen kogels meer}\\

\haiku{En de roover gaf hem '.}{seffenst geld terug en}{won op den loop gaan}\\

\haiku{- Ik weet het wel, baas,, '.}{antwoordde Jan maar ik ben}{niet voort zoet}\\

\haiku{In 1917 kwam mij een.}{bijna gelijkluidende}{lezing in handen}\\

\haiku{- Halfhaantje vindt een.}{geldbeurs en een man ontsteelt}{ze of ontleent ze}\\

\haiku{I. - Aan den voet van.}{de galg ontkomt hij echter}{door zijn fluitjesspel}\\

\haiku{Van Jaaksken met zijn ( -).}{FluitjeA.B.C. slechts het fluitje}{als toovermiddel}\\

\haiku{De motieven van:}{dit sprookjestype kunnen}{als volgt aangeduid}\\

\haiku{A. Lootens, Oude,:.}{Kindervertelsels blz. 39}{Pietji en Jantji}\\

\haiku{J.F. Vincx, Dit zijn,,:}{grappige Vertelsels en}{Sprookjes I blz. 41}\\

\haiku{De drie Gebroeders,.}{Nr 654 Antti Aarne en}{Maurits de Meyer}\\

\haiku{Maurits de Meyer:}{rangschikt het thema van dit}{vertelsel als volgt}\\

\subsection{Uit: De Vlaamsche vertelselschat. Deel 2}

\haiku{- Hier zal ik wel een,,.}{onderkomen vinden dacht}{ze en zij belde}\\

\haiku{De koning deed de.}{klokken luiden op al de}{torens van het land}\\

\haiku{De varkenshoedster.}{kwam in haar werkkleedij met de}{ezelsmuts op het hoofd}\\

\haiku{Zij dooden den gouden,.}{fezant braadden hem aan het}{spit en aten hem op}\\

\haiku{Zoo was hij dan op.}{het eind van het jaar rijker}{dan de zee diep is}\\

\haiku{Een man tikte hem.}{evenwel op de schouders en}{vroeg waar hij heenging}\\

\haiku{In een oogwenk had.}{hij ze allemaal dood voor}{zijn voeten liggen}\\

\haiku{- Maar hier, zegde zij,,}{schenk ik u tot bewijs dat}{gij mij gered hebt}\\

\haiku{ik voelde dat gij,.}{de appels had afgeplukt}{was ik genezen}\\

\haiku{Een vrouw was bezig.}{met den overschot van den oogst}{in te zamelen}\\

\haiku{Maar ik zal u, in,.}{het naaste dorp er eens een}{staaltje van geven}\\

\haiku{Zooals ge weet, werden.}{de paters er uit verjaagd}{in den Franschen tijd}\\

\haiku{Ineens keerde hij.}{zich om en keek verbaasd naar}{alle kanten uit}\\

\haiku{Iedereen keek hem,.}{na want hij leek aan een die}{geen gewest meer weet}\\

\haiku{De knecht van den boer.}{schoot huilend op den loop en}{ging de wet halen}\\

\haiku{De twee andere.}{bulten zagen dat echter}{met leede oogen aan}\\

\haiku{Een man naderde '.}{ent leek waarachtig of}{de bult daar weer was}\\

\haiku{- Wat zal er van u?}{geworden in de wereld}{als gij niet leeren walt}\\

\haiku{Ik ben genezen!}{en gezond gelijk ik nog}{nimmer ben geweest}\\

\haiku{Als gij er u drie. '}{dagen op te broeden zet}{hebt gij jonge ezels}\\

\haiku{'t Moest een rijke.}{van elders zijn dien zij eens}{tot man zou nemen}\\

\haiku{Beiden lieten een.}{verschrikkelijken schreeuw en}{sloegen op de vlucht}\\

\haiku{Ook ben ik overtuigd.}{dat Albaan u deze drie}{veeren zal brengen}\\

\haiku{Albaan groef op de.}{aangewezen plaats en vond}{den gouden sleutel}\\

\haiku{Albaan nam zulks aan,:}{en die vragen luidden bij}{den tweeden koning}\\

\haiku{Hij zag daar, op een,.}{boogscheut afstand het slot van}{Vogel Veen liggen}\\

\haiku{- Neen, vervolgde het,:}{meisje nu blijft er nog de}{vraag van den veerman}\\

\haiku{Zij wierp zich om den.}{hals van Albaan en kuste}{hem wel duizendmaal}\\

\haiku{- Neen, zoo kunt ge niet,.}{blijven loopen ik zal nog}{eens voor u zorgen}\\

\haiku{Met die woorden trok '.}{t manneken terug naar}{zijn sjees en reed voort}\\

\haiku{En de filosoof.}{moest beslissen wat men er}{mee aanvangen zou}\\

\haiku{Een anderen keer.}{moest de koning een bezoek}{te Kessel brengen}\\

\haiku{Dan hoort ge altijd}{waar hij zit en den dag voor}{dat de koning komt}\\

\haiku{Om twaalf uur kwamen,.}{ze allemaal af ieder}{met twee tellooren}\\

\haiku{Eerst ging de pastoor,,.}{dan de burgemeester dan}{de twintig boeren}\\

\haiku{Op eens echter kwam:}{de baas vol alteratie}{binnengeloopen}\\

\haiku{En als dat niet pakt,}{beginnen ze te vloeken}{en te sakkeren}\\

\haiku{Zij dronken een pint:}{en rookten een pijp en dan}{zegde het smidje}\\

\haiku{- Maar ik, ik, ik vloog.}{tot het uiterste puntje}{van de eeuwigheid}\\

\haiku{We kwamen aan een.}{rivier waar we over moesten langs}{een smal bruggetje}\\

\haiku{En als ik opsta,.}{moet ge er ook uit of de}{duivel houdt de kaars}\\

\haiku{Sinds twee dagen had.}{het geen grummelken eten over}{de lippen gehad}\\

\haiku{Gestolen is 't! -,,, '.}{Goed goed pastoor z\'o\'o moet ge}{t nu maar zeggen}\\

\haiku{En 't was eten van.}{goud en drinken van goud dat}{men haar voorzette}\\

\haiku{Maar ginds zie, in 't,,.}{aardsch paradijs staat een boom}{de boom van de spraak}\\

\haiku{Daarom moet ge me.}{laten leven en terug}{in de zee werpen}\\

\haiku{- Mijn vrouw zou graag in.}{een steenen huis wonen met een}{grooten hof er aan}\\

\haiku{Zijn vrouw woonde nu.}{in een prachtig steenen huis met}{een grooten hof}\\

\haiku{De vrouw echter was. '.}{het niett Leek wel of er}{haar nog wat ontbrak}\\

\haiku{'t Manneken kwam,.}{gelukkig thuis maar zijn vrouw}{was niet gelukkig}\\

\haiku{En toen zij paus was,,...}{wilde zij nog meer worden}{ja meer nog dan Paus}\\

\haiku{Na elk schot raapte.}{hij iets op van den grond en}{stak het in zijn zak}\\

\haiku{De ingeslapen.}{looper sprong wakker van het}{daverend gerucht}\\

\haiku{Hij opende al zijn.}{zakken en joeg de muggen}{naar de soldaten}\\

\haiku{Geen een die tijdens.}{zijn levensjaren nog aan}{werken moest denken}\\

\haiku{- Dan zal het haantje,.}{geen hartje gehad hebben}{zei de schoenmaker}\\

\haiku{Beiden peuzelden.}{het haantje op en togen}{in stilte verder}\\

\haiku{De schoenmaker die.}{zulks vernomen had kwam het}{zeggen aan Ons Heer}\\

\haiku{Maar de nacht die kwam.}{geschiedde alles gelijk}{den vorigen nacht}\\

\haiku{Dan bleef hij in zijn, '.}{bed wakker liggen als een}{muisje int meel}\\

\haiku{Eer men 't wist lag.}{zij aan den steenen trap van het}{koninklijk paleis}\\

\haiku{Enkele stonden.}{later kwam ook de prinses}{de kamer binnen}\\

\haiku{'t Leek wel of hij.}{uren ver werd weggedragen}{door een wervelwind}\\

\haiku{En hij ging naar de.}{reuzen en werd door hen als}{een broer ontvangen}\\

\haiku{Na lang zoeken vond.}{hij er een hut waarin een}{kluizenaar woonde}\\

\haiku{Als gij dien hond doodt.}{is al wat in de kamer}{ligt uw eigendom}\\

\haiku{En daar stond, op den,.}{vliegenden minuut een zak}{goud v\'o\'or zijn voeten}\\

\haiku{ze den volgenden.}{morgen op het eerste uur}{te doen optrekken}\\

\haiku{riep de kapitein,.}{toen de witte gedaante}{zich voor hem ophief}\\

\haiku{Welnu, raadt, raadt wat?}{er het schoonste is in mijn}{tuin en mijn kasteel}\\

\haiku{Zijne vrouw had het.}{schot gehoord en kwam eens zien}{wat er gebeurd was}\\

\haiku{De schoenmaker ging.}{aan zijne vrouw vertellen}{wat er gebeurd was}\\

\haiku{De schoenmaker trok.}{het zich niet verder aan en}{Het het lijk liggen}\\

\haiku{Toen hij weer lange,.}{dagen gegaan had kwam hij}{voorbij een moeras}\\

\haiku{Hij schudde met de.}{beurs zoo gelijk hij het de}{mannen had zien doen}\\

\haiku{Hij zal hij daar wel,.}{voor te vinden zijn als wij}{hem vijf frank geven}\\

\haiku{Hij nam de maat en.}{de schaar want hij moest de stof}{ter plaatse snijden}\\

\haiku{Ten slotte sneed hij, '.}{er nog een stuk af dat hij}{doort venster wierp}\\

\haiku{- Sint Antonius.}{geef mij eens twee frank om naar}{de kermis te gaan}\\

\haiku{- Wat is hier gebeurd,,?}{zegde deze leeft die man}{nog of is hij dood}\\

\haiku{Haar {\textquoteleft}venken{\textquoteright}3 zocht met,,,.}{veel gedruisch Trap op trap}{af door heel het huis}\\

\haiku{Slaat uw oogen neer, 't.}{is de heilige geest die}{nu nederdaalt}\\

\haiku{En deze won het.}{en kreeg de honderd kronen}{van den koning}\\

\haiku{De koster had, ten,.}{slotte een gedachte die}{aangenomen werd}\\

\haiku{De duivel liet zich.}{in den zetel vallen en}{Smidje Smee lachte}\\

\haiku{Daarop nam hij den,.}{ring van den vinger zoodat hij}{opnieuw zichtbaar werd}\\

\haiku{Hij is in een ver.}{land en aan de menschen vraagt}{hij den weg naar huis}\\

\haiku{Daar stond die koning,.}{nu terug in zijn eigen}{land maar heel alleen}\\

\haiku{*~         Er was eens een,.}{jongen op het dorp die niet}{van de slimste was}\\

\haiku{- Een mensch is wel geenen,,.}{vorsch zei Jan maar hij springt}{ook al eens gaarne}\\

\haiku{De schoonste vrouw van,.}{heel het land Is Sneeuwwitje}{rein Zoo teer en fijn}\\

\haiku{De schoonste vrouw van,.}{heel het land Is Sneeuwwitje}{rein Zoo teer en fijn}\\

\haiku{De booze vrouw stak hem:}{zelf in de lokken van het}{arm meisje en zei}\\

\haiku{Als de kabouters.}{nu thuis kwamen schoten zij}{dadelijk ter hulp}\\

\haiku{Het aan dit thema:}{verwante sprookje van de}{gebroeders Grimm heet}\\

\haiku{Verteld te Boom, in,,,.}{1890 door vrouw J.D.K. waschvrouw}{geboren te Ranst}\\

\haiku{Vader geboren,.}{te Wuestwezel moeder}{van Duitsche afkomst}\\

\haiku{Rond den Heerd, II, blz.: ',:}{79t Manneken uit de}{Mane V. blz. 262}\\

\haiku{'t Vertelselke ',,:}{vant Manneke uit de}{Mane VII blz. 27}\\

\haiku{Wodana, blz. 47 en,,:}{J.W. Wolf Deutsche M\"archen}{und Sagen blz. 105}\\

\haiku{der Tod zu F\"ussen ().}{des Krankendas Bett oder den}{Kranken umgelegt}\\

\haiku{der Schmied wird weder.}{in den Himmel noch in die}{H\"olle gelassen}\\

\haiku{Een dergelijke.}{lezing werd in Vlaanderen}{nog niet geboekt}\\

\subsection{Uit: De Vlaamsche vertelselschat. Deel 3}

\haiku{'t Scheen wel of er.}{daar een stoet van engelen}{kwam aangezwommen}\\

\haiku{Den vierden dag, toen,.}{hij opstond was hij als dood}{van honger en dorst}\\

\haiku{Wij moeten dien ring.}{bemachtigen en zullen}{dan even machtig zijn}\\

\haiku{Nu ge toch niet naar,.}{de school moet kunt ge er u}{eens goed vermaken}\\

\haiku{Al gaande sloeg hij.}{zijn lokken van links naar rechts}{en van rechts naar links}\\

\haiku{- Kruip eens rap in den,,.}{oven zei ze en zie eens of}{hij heet genoeg is}\\

\haiku{Hij trok recht naar de.}{Hel om zijn voornemen ten}{uitvoer te brengen}\\

\haiku{Eerst moest hij weten.}{wat er daar onder  den}{grond ruizemuisde}\\

\haiku{Buiten gekomen.}{floot hij op de drie paarden}{van de drie ridders}\\

\haiku{De juffrouw dacht niet.}{anders of Jan de Koeter}{zou niet weerkomen}\\

\haiku{- Welnu, zei de heer,.}{tegen zulken baas heb ik}{niets in te brengen}\\

\haiku{De tweede zoon ging.}{steeds naar het oosten en kwam}{aan het Zilverland}\\

\haiku{Eindelijk zag hij,,.}{op een klein zandheuveltje}{een witte roos staan}\\

\haiku{Hij stapte af en.}{ging er een wandeling doen}{in de groote bosschen}\\

\haiku{Vier maanden later.}{kwam hij dan ook in de stad}{van zijn vader aan}\\

\haiku{Het was twaalf uur in.}{den nacht en daarom ging hij}{in den hof slapen}\\

\haiku{Alles werd door den.}{vijand gebombardeerd en}{kapotgeschoten}\\

\haiku{De prins nam al de.}{lapjes van den neusdoek en}{smeet ze overal heen}\\

\haiku{*~         Koning Alexander.}{had van boven op het hoofd}{een grooten hoorn staan}\\

\haiku{Zij besloten die.}{geboorte te vieren met}{een wafelenbak}\\

\haiku{Zoo ging het meisje,,.}{voort zoo ver zoo ver tot het}{eind van de wereld}\\

\haiku{Zij trok daarop naar,.}{de overzij waar Madam de}{Maan haar kasteel stond}\\

\haiku{- Ik moet de zeven.}{kauwkens spreken die hier in}{den IJsberg wonen}\\

\haiku{Het meisje had slechts.}{den tijd zich even achter de}{deur te verbergen}\\

\haiku{Eens gebeurde het.}{dat Toon in de Schelde een}{bad wilde nemen}\\

\haiku{Toon bleef zitten aan, '.}{den kant altijd maar diep in}{t water kijkend}\\

\haiku{De kinderen van.}{de buurt speelden daar juist en}{maakten veel lawaai}\\

\haiku{Jan was de zoon van.}{een weduwe die op een}{klein pachthoef woonde}\\

\haiku{- Te naasten keer zal ',.}{t beter gaan antwoordde}{de jongen daarop}\\

\haiku{Eens moest de moeder.}{van slimme Jan enkele}{kiekens verkoopen}\\

\haiku{En die boer deed zoo, ',.}{maar alst oogsttijd was zat}{hij met leege armen}\\

\haiku{- Kom, Heer, we trekken,,.}{weg zei Sinte Pieter die}{beefde als een riet}\\

\haiku{Een weinig verder.}{kwam daar een man aan met een}{mand rijpe kersen}\\

\haiku{- En wat hebben de,?}{boeren uit den Polder van}{mij gezeid Pieter}\\

\haiku{Sinte Pieter en.}{Sint Jan mochten beiden hun}{rechten doen gelden}\\

\haiku{Hij had al de kracht.}{van zijn lijf noodig om het net}{boven te halen}\\

\haiku{Gij, die niets bezit,.}{zult niets bijbrengen als ik}{morgen te kort kom}\\

\haiku{Daarop liep hij recht.}{naar het paard om het naar zijn}{stalling te brengen}\\

\haiku{En zie, hij vond hem.}{veel verder dan zijn oudste}{broer geschoten had}\\

\haiku{De duif, waarvan de,.}{prinses gesproken had vloog}{voor hem de lucht in}\\

\haiku{eten en drinken naar,!}{ons hartje lust zonder dat}{het ons een cent kost}\\

\haiku{De baas, die zulks in ',,:}{t oog kreeg vond dat aardig}{kwam buiten en vroeg}\\

\haiku{- Ge moet weten, zei,.}{de student dat het kermis}{is op het kasteel}\\

\haiku{Eindelijk bood zich,.}{Siepe aan verkleed in een}{Spaanschen geneesheer}\\

\haiku{De kabouters moesten ',,.}{s avonds als loon een groote teil}{zoete melk hebben}\\

\haiku{- En nochtans, ging hij,,}{weer voort daar waar ik mijn sprong}{nam had ik het nog}\\

\haiku{- Misschien had ik het.}{daar toch nog en is het op}{den grond gevallen}\\

\haiku{Een boer, die daar wat ',.}{verder aant zaaien was}{kwam toegeloopen}\\

\haiku{Geef mij uw riek, ik,.}{zal zelf zoeken ik weet best}{wat ik hebben moet}\\

\haiku{En hij klom op een,...-}{boom met zijn twee molensteenen}{en zijn koevel}\\

\haiku{'t Bleef  echter.}{aan een der laagste takken}{van den boom hangen}\\

\haiku{En de man ging de.}{brandende kool halen en}{lei ze in den haard}\\

\haiku{die had een stok in,.}{zijn mond met een gat erin}{en daar kwam rook uit}\\

\haiku{De jeugd was er wars '.}{vant werk en alleen op}{spel en dans verzot}\\

\haiku{Hij trok de groote baan.}{op en al de koppels hem}{dansend achterna}\\

\haiku{De baren sloegen,.}{toe speleman en dansers}{ineens verzwelgend}\\

\haiku{t sloot zich zoo dicht,.}{achter hem toe dat er geen}{mensch meer door en kon}\\

\haiku{'t Was altijd een.}{en hetzelfde antwoord dat}{er gegeven werd}\\

\haiku{Haast u nu naar huis.}{en wees verstandig in uw}{doen en handelen}\\

\haiku{Ha, ik begrijp het,.}{die herbergier is u nog}{eens te plat geweest}\\

\haiku{\'en uw tafelken,.}{kunt weerkrijgen als ge maar}{uit uw oogen wilt zien}\\

\haiku{Ja, de bol kreeg een.}{eereplaats in de beste}{kamer van hun huis}\\

\haiku{En de boer liet het.}{wit konijn in gewijden}{grond begraven}\\

\haiku{Joosken bond den zak:}{stevig dicht en de herder}{begon te roepen}\\

\haiku{Altijd-aan, dag en,.}{nacht had hij op menschen en}{dingen nagedacht}\\

\haiku{De koningszoon zag, '.}{nu duidelijk dat de heks}{aant sterven was}\\

\haiku{Zoo sprak de heks met. '}{stille stem en nam daarop}{den derden spiegel}\\

\haiku{Hij spoedde zich naar,:}{huis sleurde de geit binnen}{en riep op zijn vrouw}\\

\haiku{Na nog lang op hem,.}{gewacht te hebben viel hij}{eindelijk in slaap}\\

\haiku{Ik, voor mijn paart, ik;}{neem het poeder mee dat doof}{maakt en stom en blind}\\

\haiku{'t Was meteen een '.}{herrie van belang int}{vijandelijk kamp}\\

\haiku{De koning, zooals gij,.}{wel denken kunt was daar zeer}{ongelukkig over}\\

\haiku{Ga u verhuren,.}{bij den reus die ginder op}{den Wolkenberg woont}\\

\haiku{Als ge dat doet, zal.}{het u de schuilplaats van de}{prinses doen kennen}\\

\haiku{Zij zit gevangen.}{in den ondersten kerker}{tegen den vijver}\\

\haiku{Gloeiende kolen,,.}{gloeiende kolen moet ik}{hebben anders niet}\\

\haiku{Met een slag van zijn.}{sabel sloeg Fluppen den bol}{in duizend stukken}\\

\haiku{Hij droeg evenwel goed.}{zorg zijn kluppeltje onder}{den arm te houden}\\

\haiku{Nog vele landen.}{en nog meerdere steden}{moest hij doortrekken}\\

\haiku{Daarom heb ik reeds.}{een boontje voor u en zal}{ik u aanhooren}\\

\haiku{Hij ronkte meteen, '.}{dat men hem op een uur in}{t rond kon hooren}\\

\haiku{*~         Er was eens een.}{arme weduwe en die}{had maar eenen jongen}\\

\haiku{Hij was nog maar pas.}{het dorp uit of hij kwam een}{ouden ezel tegen}\\

\haiku{Jan, al had hij maar,.}{weinig gaf de vijf centen}{en de hond liep mee}\\

\haiku{Ik zat daar onder '!}{t stoelken van ons meken}{te spinnen en krak}\\

\haiku{Nu, als de wereld,.}{gaat vergaan ben ik liever}{op de wijde baan}\\

\haiku{De haan vloog van den.}{boom en trok met Jan en zijn}{kameraden mee}\\

\haiku{Iedereen sprong op,.}{want zij meenden allemaal}{dat er onraad was}\\

\haiku{- 't Is een licht dat.}{men ginder ver in een huis}{aangestoken heeft}\\

\haiku{Ge weet wel wat voor,:}{een spel dat is al wordt het}{nu niet meer gespeeld}\\

\haiku{Meteen hoorde hij.}{dan weer den man achter hem}{dapper doorstappen}\\

\haiku{De kwezel pakte.}{haar dikken kerkboek en wierp}{hem achter de kast}\\

\haiku{Heel hun leven lang.}{hadden die twee menschen naar}{kinderen getracht}\\

\haiku{Zij waren beiden '.}{zoo gelukkig dat zijt}{niet zeggen konden}\\

\haiku{'t Was of er een.}{jongetje van rond de vier}{jaar in hun bed lag}\\

\haiku{De goede arme.}{vrouw besloot het kindje mee}{naar huis te nemen}\\

\haiku{Een pintje voor den.}{reus was als een groote ton voor}{een gewonen mensch}\\

\haiku{Heeft er ooit iemand?}{regelmatiger dan ik}{zijn tol aanbetaald}\\

\haiku{Zij hadden reeds van.}{het Goudland hooren spreken}{en trokken er heen}\\

\haiku{Maar de manschappen.}{met de pieken stonden hun}{heeren dapper bij}\\

\haiku{Als het regende;}{was het reukwater en als}{het sneeuwde suiker}\\

\haiku{Maar toen meteen had,.}{hij dorst gekregen dorst lijk}{honderdduizend man}\\

\haiku{Hij ontwaakte op,.}{den dijk van de rivier daar}{dicht tegen zijn dorp}\\

\haiku{Duimelingsken en;}{de Wolf en de gewone}{sprookjes van Duimken}\\

\haiku{Dit sprookje werd mij.}{door den heer Florimond Van}{Duyse meegedeeld}\\

\haiku{der Knabe gem\"astet die (:}{Hexe in den Ofen geworfen}{Vergelijk nr 1121}\\

\haiku{Het overeenstemmend:}{sprookje bij de Gebroeders}{Grimm draagt als titel}\\

\haiku{- De Duivel en de ().}{HouthakkerSprookjes van den}{dankbaren duivel}\\

\haiku{Van Kluppelken uit,;}{den Zak en Van het Vrouwken}{en heur Kanneken}\\

\haiku{Vlaamsche Moppen van,;}{Victor de Meyere en}{Leo Verkein nr 11}\\

\haiku{H. Befreiung aus ();}{dem SackeKasten durch Tausch}{mit einem Hirten}\\

\haiku{Vertelling van de,,,,;}{Kat den Hond de Zwaan de Koe}{het Peerd en den Haan}\\

\haiku{waar zij voldoende.}{verklarend zijn worden de}{thema's niet vermeld}\\

\haiku{- De boerenzoon komt:}{van den troep en gebaart geen}{Vlaamsch meer te kennen}\\

\haiku{Daarna zijn vijand ':}{in den zak steken en in}{t water werpen}\\

\haiku{Hoe Onze Lieve,,.}{Heer den Duivel machteloos}{maakte III nr 187}\\

\subsection{Uit: De Vlaamsche vertelselschat. Deel 4}

\haiku{En hij zag al de!}{sterren van den hemel v\'o\'or}{zijn oogen flikkeren}\\

\haiku{wauw! deed de hond, en.}{hij beet alom  waar hij}{den wolf kon grijpen}\\

\haiku{De kraai en de puit*:}{De kraai zat aan den kant}{van eenen put en riep}\\

\haiku{Eindelijk kwam de. ' '.}{wintert Sneeuwde ent}{vroor dat het kraakte}\\

\haiku{Na lang zoeken vond.}{hij hem eindelijk in een}{verlaten spelonk}\\

\haiku{- 't Is waar, zei de,,.}{bie ik heb het moeilijk maar}{ik doe niemand kwaad}\\

\haiku{Het woog z\'o\'o zwaar, dat.}{hij al zijn macht noodig had om}{het op te halen}\\

\haiku{De vorschen maken.}{nu voortdurend kennis met}{zijn koningswet}\\

\haiku{Zoo konden ratten.}{en muizen zich bijtijds uit}{de voeten maken}\\

\haiku{zei Sinte Pieter,.}{die altijd goed betrouwen}{had in de menschen}\\

\haiku{Zij dringen in al.}{de woningen en spelen}{er heer en meester}\\

\haiku{De haas was aan 't.}{napeinzen hoe dat men de}{afzetting zou doen}\\

\haiku{Kom met mij mee, ik.}{weet een schoonen kaas zitten}{op gindsche hoeve}\\

\haiku{Nu eerst ging de musch,}{voor een goei te werk schreeuwend}{en huilend gelijk}\\

\haiku{De twee laatste die.}{aan de beurt kwamen waren}{de musch en de eend}\\

\haiku{het had voorzegd, maar.}{de uil heeft de woorden van}{den gier onthouden}\\

\haiku{- Tot uw straf zult gij.}{voortaan geen tien meter ver}{meer kunnen vliegen}\\

\haiku{En de ekster en.}{de tortelduif begonnen}{hun nest te bouwen}\\

\haiku{Mijn nest mag ik niet.}{verlaten of mijn jongen}{sterven in de schaal}\\

\haiku{Onze moeder heeft.}{sneeuwwitte pootjes en de}{uwe zijn zwart als roet}\\

\haiku{Op den gestelden.}{dag stond hij met zijn wonder}{dier op den kruisweg}\\

\haiku{daarom kwam hij maar,.}{met zijn eigen beest vooruit}{om tijd te winnen}\\

\haiku{De duivel voelde,.}{zich verloren vloekte en}{wilde wegloopen}\\

\haiku{- Ziedewel, dacht het,.}{schaap mijn eerste gedachte}{was de beste}\\

\haiku{Daartoe hadden zij.}{allerhande steenen op een}{groote baan saamgebracht}\\

\haiku{En bovendien heeft.}{zij drie duivelsharen op}{haren kop staan}\\

\haiku{'t Duurde echter,}{niet lang of er kwamen er}{al naar beneden}\\

\haiku{Tot koning ben ik, ' '.}{verheven Enk blijft}{mijn leven lang}\\

\haiku{Lang vloog zij en haar.}{aankomst meldde zij met een}{nijdig hoorngeschal}\\

\haiku{En de vos was de,.}{pijp uit maar bleef van verre}{op den loer liggen}\\

\haiku{Het koningsken bleef.}{diep in het hol zitten en}{kikte noch mikte}\\

\haiku{De arend ontving het.}{beestje met fatsoen en dacht}{een oogenblik na}\\

\haiku{Van toen af, werd de.}{vijgeboom een heilige}{boom geheeten}\\

\haiku{Sindsdien worden er. '}{madeliefjes met roode}{vlekken gevonden}\\

\haiku{En de sneeuw ging tot.}{de roode kollebloem en}{vroeg haar roode kleur}\\

\haiku{Erger ging het met!}{diegenen die een tijdlang}{geloopen hadden}\\

\haiku{Verteld te Hamme,,,.}{in l909 door S.B. een meisje}{van Elverzele}\\

\haiku{Ons Volksleven, II,,,.}{blz. 126 P.J. Cornelissen}{en J.B. Vervliet Vl}\\

\haiku{Pourquoi les chats se;}{lavent-ils la figure}{quand ils ont mang\'e}\\

\haiku{Der Fuchs verleitet;}{den Hahn mit geschlossenen}{Augen zu kr\"ahen}\\

\haiku{Pol de Mont en A.,,:}{de Cock Zoo vertellen de}{Vlamingen blz. 19}\\

\haiku{J. Cornelissen,,:}{en J.-B. Vervliet Vlaamsche}{Vertelsels blz. 225}\\

\haiku{Van Triene Giet (slechts);}{twee van de drie motieven}{die wij meedeelen}\\

\haiku{- Waarom de Menschen.}{in alle Richtingen over}{de Wereld loopen}\\

\haiku{waar zij voldoende,.}{verklarend zijn worden de}{thema's niet vermeld}\\

\haiku{De haan vleit de gaai:}{en gelukt er in den buit}{voor zich te houden}\\

\haiku{De wolf in don put ():}{de weerkaatsing van de maan}{gelijkt aan een kaas}\\

\haiku{-Wat de Mug in ',.}{t Kamp van de loopende}{Dieren verneemt 425}\\

\section{Herman Middendorp}

\subsection{Uit: De schaduw van Capoulet}

\haiku{Mijnheer de graaf had,.}{geloof ik gedacht dat u}{vroeger zou komen}\\

\haiku{de overvloedige,.}{regen verminderde maar}{hield niet geheel op}\\

\haiku{Natuurlijk, ik wist,....}{zeker dat ik de deurknop}{had zien bewegen}\\

\haiku{de onheilen, die,.}{zich later voltrokken had}{moeten voorvoelen}\\

\haiku{Ik keek naar buiten,;}{over de golvingen van de}{beboschte bergen}\\

\haiku{Het beloofde een.}{mooie na-zomersche dag}{te zullen worden}\\

\haiku{Nu wil het verhaal,;}{dat zij na haar dood geen rust}{heeft kunnen vinden}\\

\haiku{Het is een eenigszins.}{pijnlijke geschiedenis}{met dien jongeman}\\

\haiku{Monique had een,.}{brief in de hand dien ze haar}{vader overreikte}\\

\haiku{hij wou hem aan u,,.}{brengen en toen zei ik dat}{ik het wel doen zou}\\

\haiku{eerst tegen half acht;}{behoefde ik weer thuis te}{zijn voor het avondeten}\\

\haiku{ik zal den graaf er,.}{van op de hoogte stellen}{wat er gebeurd is}\\

\haiku{We hebben hier wel ',}{middelen om u aant}{spreken te krijgen}\\

\haiku{In dat geval zou.}{de bende dus uit nog m\'e\'er}{personen bestaan}\\

\haiku{{\textquoteright} zei hij zacht, {\textquoteleft}als we,.}{gesnapt worden vermoorden}{ze ons alle twee}\\

\haiku{Het meisje, dat voor,;}{mij uitging keek eerst spiedend}{naar beide kanten}\\

\haiku{Zeg Donia, heb jij?}{vandaag niet weer iets vreemds aan}{De Fontenay bemerkt}\\

\haiku{als er vanavond nog,.}{geen hulp komt zal ik Parijs}{moeten opbellen}\\

\haiku{Dat wou ik nu weer,,;}{probeeren maar het is luk-raak}{als ze hem vinden}\\

\haiku{Dan kunnen we straks,.}{met ons drie\"en zien wat we}{verder zullen doen}\\

\haiku{Op den grond, naast de,.}{tafel lag het lichaam van}{den graaf De Fontenay}\\

\haiku{Ik zweeg er over,  .}{dat ik Vergniaud naast}{het huis gezien had}\\

\haiku{Met de handen op.}{den rug liep de maire de}{kamer op en neer}\\

\haiku{er is geen bezwaar,.}{tegen dat de klok van het}{kasteel wordt geluid}\\

\haiku{{\textquoteright} {\textquoteleft}Ja,{\textquoteright} zei ik, eenigszins, {\textquoteleft}.}{verbaasd over deze vraagwij}{zijn goede vrienden}\\

\haiku{Hij vertelt veel goeds,.}{van u. En ik geloof dat}{hij zich niet vergist}\\

\haiku{Hij maakte hem niet,,.}{open waar we bij waren maar}{ging de kamer uit}\\

\haiku{Het adres was slordig,,.}{geschreven met groote krullen}{maar zonder fouten}\\

\haiku{Mijnheer de graaf keek.}{vluchtig naar het adres en stak}{den brief in zijn zak}\\

\haiku{{\textquoteleft}Maar ik vermoed, dat.}{we dat bewijsstuk ook niet}{noodig zullen hebben}\\

\haiku{{\textquoteright} {\textquoteleft}Nu, heb jij zelf wel,?}{eens wat gezien hier in huis}{dat je bang maakte}\\

\haiku{trouwens, de heele,,.}{manier waarop hij sprak was}{mij antipatiek}\\

\haiku{{\textquoteleft}Hoe laat was het, toen,?}{u het schot hoorde dat op}{den graaf gelost werd}\\

\haiku{{\textquoteleft}U was zeker wel,,}{verbaasd toen mijnheer Donia}{u dat vertelde}\\

\haiku{Ongerust keken.}{professor Chalosse en}{ik elkander aan}\\

\haiku{Op de vraag, wat haar,.}{broer voor den kost deed zei ze}{dat hij handel dreef}\\

\haiku{Bij de deur hield de.}{heer Armandy mij nog een}{oogenblik terug}\\

\haiku{Hij werkt samen met,{\textquoteright}.}{den Service de S\^uret\'e}{vervolgde Firmin}\\

\haiku{Er klonk een kreet in,.}{de stilte onmiddellijk}{gevolgd door een schot}\\

\haiku{Direct werd het mij,.}{duidelijk dat hier geen hulp}{meer te bieden was}\\

\haiku{Wij lieten hem in.}{zijn alteratie achter}{en gingen naar huis}\\

\haiku{{\textquoteright} {\textquoteleft}Wat zou u gedaan,,?}{hebben professor als hij}{het niet geweest was}\\

\haiku{we weten was hij,.}{de eenige die een hekel}{aan mijn vader had}\\

\haiku{De misdadiger,,}{kan dicht bij het huis gestaan}{hebben toen hij schoot}\\

\haiku{De ramen zijn niet,.}{hoog niet meer dan een voet of}{vier boven den grond}\\

\haiku{De professor liet.}{Pernod brengen en dronk er}{twee groote glazen van}\\

\haiku{{\textquoteright} Terloops merk ik hier,.}{even op dat ik dit stopwoord}{van Crampton kende}\\

\haiku{Natuurlijk was hij,;}{bang dat hij dan zelf ook in}{de val zou loopen}\\

\haiku{De detective.}{ging voort met boeken van de}{planken te nemen}\\

\haiku{{\textquoteright} Men kon duidelijk,.}{zien dat de bladzijden er}{uit gescheurd waren}\\

\haiku{De detective;}{en de inspecteur kropen}{in het struikgewas}\\

\haiku{hij alleen kende.}{het huis en het bestaan van}{het famieliboek}\\

\section{P.H. van Moerkerken jr.}

\subsection{Uit: De bevrijders}

\haiku{In de gesprekken;}{der ouderen sprankelde}{vernuftige scherts}\\

\haiku{In de trekken van.}{het blonde meisje kon zij}{echter niet lezen}\\

\haiku{op wiens vol gelaat;}{een sterke overtuiging van}{eigenwaarde lag}\\

\haiku{er waren landen\%,.}{die geen 20 er waren er}{die niets betaalden}\\

\haiku{Zijn plan om hen in;}{Holland op te zoeken werd}{telkens uitgesteld}\\

\haiku{Doch Ter Wisch zag dat.}{zijn kleine groene ogen naar}{Th\'er\`ese loerden}\\

\haiku{Maar de heftigste;}{revolutionnairen}{hadden zich verzet}\\

\haiku{Kort na den middag.}{reed de wagen het voorplein}{van Den Ulenhoek op}\\

\haiku{zij wist de hoge.}{waarde van het verborgen}{leven des harten}\\

\haiku{het opdringend volk,.}{beschimpte hen bauwde hun}{vreemden tongval na}\\

\haiku{Anne-Marie.}{ontving haar broeder met een}{vermoeiden glimlach}\\

\haiku{In het geboomte;}{der hofsteden langs den weg}{zongen  vogels}\\

\haiku{tijd om naar de stad,,.}{naar Haarlem te gaan hadden}{zij geen van beiden}\\

\haiku{een leeuwerik hoog;}{boven de weiden en de}{verre wateren}\\

\haiku{Zij moet een man van, ... ...}{talenten hebben en zij}{is voor mij te oud}\\

\haiku{Zijn innigsten wens.}{vernam hij nu uit den mond}{van een vriend als raad}\\

\haiku{Bovendien, hij was;}{niet presentabel op een}{huwelijksaanzoek}\\

\haiku{Schrijf dit over, met je{\textquoteright}}{elegantste hand. In zoeten}{droom liet Tobias}\\

\haiku{Toen Floris de deur.}{had horen dichtslaan barstte}{hij in lachen uit}\\

\haiku{Op de kermis had.}{zij dien heer in den groenen}{rok om geld gevraagd}\\

\haiku{De twe vuren  ;}{rustten in evenwicht aan de}{einden der aarde}\\

\haiku{Maar zulk een kind, ... het;}{was nieuw voor hem en hij gaf}{er toch zijn geld voor}\\

\haiku{Het verdriet mij zeer.}{dat ik u een deceptie}{moet veroorzaken}\\

\haiku{De Katholieken,{\textquoteright}.}{zouden het niet met u eens}{zijn hernam Aagje}\\

\haiku{Hij stond op en liep.}{onrustig in het grote}{vertrek heen en weer}\\

\haiku{al zijn vrienden bij.}{den vijfden Beurspilaar spraken}{er hun vrees over uit}\\

\haiku{{\textquoteright} Maar Floris boog zich:}{over het tafeltje en gaf}{er een vuistslag op}\\

\haiku{Zo-even had hij;}{de zakjes met dukaten}{uit de kluis gehaald}\\

\haiku{s Avonds riep Santje.}{met zwakke stem haar broeder}{Bart bij de bedste\^e}\\

\haiku{En hij schreef haar over;}{de rampen die vrouw Breevoort}{hadden getroffen}\\

\haiku{Zij antwoordde niet,;}{maar gaf hem brandewijn en}{zelfgebakken brood}\\

\haiku{Doch zij herkenden.}{den weg aan de donkere}{plekken der lijken}\\

\haiku{Zelfs het geschut van.}{den vijand wekte hem niet}{uit zijn mijmering}\\

\haiku{naar wezen was zo.}{anders dan van haar die hem}{nu vergezelde}\\

\haiku{En zo verzoelde;}{een weldadige warmte}{den Decembernacht}\\

\haiku{David zelf verscheen ' ';}{alleens morgens ens}{avonds aan het ziekbed}\\

\haiku{Den Zaturdagavond.}{en den Zondag bracht hij op}{Wijckervelt door}\\

\haiku{Des Maandagmorgens,,.}{den 22sten was hij weer in de}{Kalverstraat terug}\\

\haiku{leefde zij nog, was,? ...}{zij wellicht getrouwd wist zij}{nog van zijn bestaan}\\

\haiku{de oude zaak van.}{zijn vader en grootvader}{mocht niet verlopen}\\

\haiku{Naar dien man had zij,;}{verlangd om zijn leven was}{zij angstig geweest}\\

\haiku{Aagje Fabian.}{en Jacob ter Wisch zagen}{elkander niet meer}\\

\haiku{De schone vrede ...}{van een leven vol liefde}{was toch niet voor hem}\\

\haiku{de rivieren staan,,;}{niet stil de zee golft aldcor}{de wolken jagen}\\

\haiku{Aan den rand van het}{woud gekomen hoorde hij}{op de hoogvlakte}\\

\haiku{hij glimlachte om}{de anderen die in de}{dwaze verblinding}\\

\haiku{het jonggestorven;}{hoofd was met den klassieken}{lauwerkrans getooid}\\

\haiku{De beide vrouwen.}{luisterden met in den schoot}{gevouwen handen}\\

\haiku{zorgvol zag zij naar,;}{haar kleinen grond waarover zij}{alleen nu waakte}\\

\subsection{Uit: De ondergang van het dorp}

\haiku{wier kruinen het verst.}{zichtbaar waren uit heide}{en akker en vloed}\\

\haiku{Eindelijk drong de.}{leer der Hervorming in de}{naastbije steden door}\\

\haiku{Doch in het volk bleef.}{nog lang het ruwe gemoed}{der oorlogstijden}\\

\haiku{Tien jaren hadden,.}{zij daar geleefd eer hun een}{zoon geboren werd}\\

\haiku{met de rechter hield,.}{hij een klein in perkament}{gebonden boekje}\\

\haiku{Zij arbeidden elk.}{aan een historie hunner}{beminde landstreek}\\

\haiku{De arme stomme.}{was opgestaan aan de hand}{van moeder Tuinder}\\

\haiku{ik dacht aan mijn jeugd.}{en aan alles wat ik toen}{hoopvol en mooi vond}\\

\haiku{het nauwgeboren,;}{licht over de oude akkers}{de oude stulpen}\\

\haiku{Hij keerde zich om.}{en kwam de volgende twe}{maanden niet buiten}\\

\haiku{van S. Thomas den:}{Apostel en het mirakel}{zijner relikwie}\\

\haiku{Verveloos was het,.}{hout der kozijnen verweerd}{de kleine ruiten}\\

\haiku{{\textquoteleft}Het was niet de wil,.}{van de boeren maar de wil}{van den Heilige}\\

\haiku{{\textquoteright} Pastoor Hedel zag.}{even naar de grijze urnen}{op het kabinet}\\

\haiku{Zijn grootvader had,.}{hem die verhalen gedaan}{voor een halve eeuw}\\

\haiku{Toen stond hij langzaam.}{op en pakte ordeloos}{zijn gerei bijeen}\\

\haiku{De notabelen:}{van Aarloo en Nierode}{traden toe als lid}\\

\haiku{In dien nacht klopte.}{hij aan de lage deurtjes}{bij de slaapsteden}\\

\haiku{de bewijzen voor;}{de burgemeesterlijke}{bevoegdheid gevraagd}\\

\haiku{hij had er oude.}{komforen neergezet en}{tinnen asbakjes}\\

\haiku{of niet zijn zachte.}{en vaste leiding te zeer}{een dwang was geweest}\\

\haiku{De zoon dacht aan  ,.}{de toekomst aan het leven}{dat nu eerst begon}\\

\haiku{{\textquoteright} {\textquoteleft}En als de ruimte,;}{eens wel eindig was enkel}{maar  onbegrensd}\\

\haiku{Hij trof haar met de.}{doofstomme moeder in den}{boomgaard voor het huis}\\

\haiku{Kan 't bloed in mijn{\textquoteright}.}{als gist'ge wijn Weer tot}{het brein doen springenn}\\

\haiku{Na drie weken was;}{de staking in de grote}{steden verlopen}\\

\haiku{nu hoopte hij dat,....?}{Marretje er bij was zou}{zij hem herkennen}\\

\haiku{was hier niet overal?....}{de schone vrede van dien}{beminden grijsaard}\\

\haiku{En die eerste nacht,,.}{dien ik nu weerzie was de}{schoonste mijns levens}\\

\haiku{Was het geen wanhoop,?}{aan Uw macht geen wantrouwen}{jegens Uw wijsheid}\\

\haiku{En de mare der.}{nieuwe modelboerderij}{ging snel over het land}\\

\haiku{Te midden van mijn.}{lieve boeken heeft de tijd}{mij neergeslagen}\\

\haiku{Ik lig hier nu stil,,....}{en goed het is alles schoon}{en lief om mij heen}\\

\haiku{maar om zes uur reed.}{hij met sierlijke rijzing}{weg van de aarde}\\

\haiku{Hij wilde verder.}{en de herinneringen}{beangstigden hem}\\

\haiku{meer-en-meer werd.}{de streek door renteniers en}{forensen gezocht}\\

\haiku{Frajer en groter.}{werden er de openbare}{gebouwen hersticht}\\

\section{Pol de Mont}

\subsection{Uit: De amman van Antwerpen}

\haiku{Neen, ligh nu stille...,{\textquoteright},;}{gans stille zei hij haastig}{toen ze zich bewoog}\\

\haiku{Wat zou ze nu doen,?}{de volgende dag als het}{licht gekomen was}\\

\haiku{{\textquoteleft}Hebbic uw verlof,,?}{Jonkver Veerle om een woord}{met u te spreken}\\

\haiku{Ze was bedwelmend,}{mooi in haar mollige}{naaktheid met haar huid}\\

\haiku{Hij voelde het, - hij, -.}{meende het te voelen dat}{hij overwonnen had}\\

\haiku{Ze lag gerust te,}{slapen in een heel vreemde}{houding met de voetjes}\\

\section{Henricus van Moorsel}

\subsection{Uit: Kronijk, of Aantekening der merkwaardige voorvallen binnen de gemeente Heeze en eenige omliggende dorpen en enkelde welken algemene belangstelling verdienen}

\haiku{Op Onze Lieve,.}{Vrouw-Geboorte ~ 8}{September 1952}\\

\haiku{1680193 heeft men gezien.}{eene staartster die zich in het}{Noorden vertoonde194}\\

\haiku{deze waren aan}{de staken bevroren en}{kosten 6 gulden}\\

\haiku{1799 can sterken en,.}{langdurigen winter heeft}{geduurd tot 1 April306}\\

\haiku{men dagt dat er geen / /;}{gebouw Pag. 26 meer zoude}{hebben blijven staan}\\

\haiku{29 December des.}{nachts gedonderd en gestormd}{met hagelbuijen}\\

\haiku{het metselwerk was;}{vroeger aanbesteed tot aan}{de vengster durpels480}\\

\haiku{6 Augustus de.}{kappen op de nieuwe kerk}{te Heeze gesteld}\\

\haiku{14 Februarij;}{zijn de eerste pijpen in}{het orgel gesteld}\\

\haiku{der 8 Afd. Inf. van;}{Someren na Heeze en}{Leende gekomen660}\\

\haiku{dat drie maal per dag,;}{hervat wordt des  morgens}{smiddags en des avonds}\\

\haiku{op het ogenblik stond;}{de Peer of Knop des torens}{als in volle vlam}\\

\haiku{Men kon naauwelijks;}{te voet droog aan het Kasteel}{te Heeze komen}\\

\haiku{- Weder veel regen.}{gevallen gelijk ook de}{volgende dagen}\\

\haiku{Kronijkschrijver.}{heeft zich in zake de brand}{van Helmond vergist}\\

\haiku{Hermanni, H. / Jongh, /, /, /,.}{A. de Laats J. Raucamp J.}{Chr. Sprang E. van}\\

\haiku{G. BANNENBERG, Sint,.}{Willibrord in Waalre en}{Valkenswaard p. 24}\\

\haiku{is ook bekend als.}{beneficiant van het}{altaar van de HH}\\

\haiku{121L. VAN AITZEMA,,,-.}{Saken van Staet en Oorlogh}{dl. II p. 451452}\\

\haiku{Uit den toorn zijn wij.}{gestooten Alexius Jullien}{heeft ons gegooten}\\

\haiku{M. THEODARDUS,,-.}{ROELOFS Geschiedenis van}{Grave p. 3839}\\

\haiku{M. THEODARDUS,,-.}{ROELOFS Geschiedenis van}{Grave p. 3839}\\

\haiku{A. FRENKEN, Helmond,,-.}{in het Verleden dl. II}{p. 184225 passim}\\

\haiku{376Het woonhuis van de;}{drossaard van Heeze is nooit}{pastorie geweest}\\

\haiku{Middelprijzen der;}{levensmiddelen oyer}{de maand Julij 1824}\\

\haiku{PROVINCIAAL BLAD,,,-;}{VAN NOORD-BRABAND 1830}{n. 92 p. 7778}\\

\haiku{PROVINCIAAL BLAD,,*,-;}{VAN NOORD-BRABAND 1832}{n. 110 p. 910}\\

\haiku{N. 6 van deze.}{Memorie luidt als volgt Den}{10 Augustus 1831}\\

\haiku{Deze notitie.}{is door kronijkschrijver}{later toegevoegd}\\

\haiku{PROVINCIAAL BLAD,,,-.}{VAN NOORD-BRABAND 1833}{n. 101o p. 34}\\

\haiku{Had het commando.}{overgenomen van Majoor}{W. Senn van Bazel}\\

\haiku{603De 2e Comp. was.}{reeds te Heeze sedert 15}{Januari 1833}\\

\haiku{694Koop II bekend.}{als Asten Sectie E nrs.}{796 tot en met 803}\\

\haiku{697Sectie B n., {\textquoteleft}{\textquoteright} {\textquoteleft}{\textquoteright}.}{114 bekend alsDe Donk of}{Den Ommelschen Bosch}\\

\haiku{PROVINCIAAL BLAD,,,,.}{VAN NOORD-BRABAND 1837}{n. 99 p. 17 53}\\

\haiku{742PROVINCIAAL,,,.}{BLAD VAN NOORD-BRABAND}{1838 n. 90o p. 5}\\

\haiku{Zie de afbeelding.;}{van deze pastorie op}{p. 140 van het Hs}\\

\haiku{772PROVINCIAAL,,,.}{BLAD VAN NOORD-BRABAND}{1839 n. 107 p. 39}\\

\haiku{(PROVINCIAAL BLAD,,,-).}{VAN NOORD-BRABAND 1838}{n. 90 p. 4445}\\

\haiku{807PROVINCIAAL,,,.}{BLAD VAN NOORD-BRABAND}{1842 n. 115 p. 49}\\

\haiku{PROVINCIAAL BLAD,,,-.}{VAN NOORD-BRABAND 1842}{n. 115 p. 2223}\\

\haiku{873De toren der.}{kerk oorspronkelijk bekend}{als Sectie A 862}\\

\haiku{PROVINCIAAL BLAD,,,-.}{VAN NOORD-BRABAND 1844}{n. 80 p. 4445}\\

\haiku{1028Zie voor deze,,.}{benoeming DE GODSDIENSTVRIEND}{dl. XXXVI p. 156}\\

\haiku{Reeds eerder was een.}{poging tot verkoop van de}{Heerlijkheid mislukt}\\

\haiku{beroepen Juli,.}{1689 vertrokken naar Waalre}{11 November 1693}\\

\haiku{PROVINCIAAL BLAD,,, \&;}{VOOR NOORD-BRABAND 1836}{nr. 8 p. 7 12}\\

\section{Lodewijk Mulder}

\subsection{Uit: Humor en satire}

\haiku{den tijd, dien dan nog,;}{overschiet gebruiken ze voor}{nuttigezaken}\\

\haiku{- hanc tuemur hac -,!}{nitimur o vaderland}{o dierbre grond}\\

\haiku{Dat is er nog een, '.}{zooals ik ze int jaar 1500}{heb zien gebruiken}\\

\haiku{want ik werd het hoe.}{langer hoe minder met de}{Hollandsche maagd eens}\\

\haiku{Verbeeldt u mijne,:}{verbazing toen ik daarop}{het adres zag staan}\\

\haiku{{\textquoteleft}Dat is een {\textquoteleft}kwaal,{\textquoteright} zei, {\textquoteleft} ';}{ze toendie heb ik int}{jaar 1814 opgedaan}\\

\haiku{U behoef ik het,,:}{niet te zeggen wie ik ben}{en wat ik bedoel}\\

\haiku{maar iedereen voelt, -.}{zoo niet en dat is jammer}{voor het gewormte}\\

\haiku{Het model van zulk.}{sterk haar zal wel hier of daar}{te vinden wezen}\\

\haiku{Hofdijk, Aeddon I,}{zang ~  Hoewel dat ghij zijt}{schichtigh als een rhee}\\

\haiku{zooals ik, en op een,}{verheven onpartijdig}{standpunt te gaan staan}\\

\haiku{{\textquotedblleft}Ik heb hetzelve,!}{bewerkstelligd gelijk gij}{mij hadt bevolen}\\

\haiku{Ik wist het op dat.}{oogenblik niet en bleef in}{gedachten verdiept}\\

\haiku{{\textquoteright} {\textquoteleft}Niets liever dan dat,{\textquoteright}, {\textquoteleft} -;}{zei hijmaar wat maakt je in}{eens zoo pleizierig}\\

\haiku{Ik wil me op een,,!}{verheven onpartijdig}{standpunt plaatsen Joost}\\

\haiku{Die drenken in den,}{morgendouw De duifjes met}{haar trekkebekken}\\

\haiku{{\textquoteright} {\textquoteleft}En zoudt gij op uwe,?}{beurt denken dat die taal zoo}{moeilijk te leeren is}\\

\haiku{Lees er \'e\'ene maand,.}{in en gij zult geheel over}{dat vreemde heen zijn}\\

\haiku{een wit piqu\'e vest,.}{glac\'e handschoenen en een}{fijne bruine rok}\\

\haiku{{\textquoteleft}Jongen,{\textquoteright} zei hij, na, {\textquoteleft} ';}{eenig aarzelenmaar vindt je}{t toch niet wat gek}\\

\haiku{maar 't is een kunst,,!}{om dat geheim te houden}{want die intriges}\\

\haiku{Nu of nooit,{\textquoteright} dacht ik, {\textquoteleft}.}{een gevoelvol discours over}{het buitenleven}\\

\haiku{als wij eens een keer ',;}{of vier in de maands avonds}{uitgaan is het veel}\\

\haiku{gaat men liever niet,.}{naar den eenen kant dan gaat men}{naar den anderen}\\

\haiku{maar dat is gezond,;}{in de morgenlucht dat wekt}{de appetijt op}\\

\haiku{{\textquoteleft}de majoor zal hem, '.}{ook wel kennen wantt was}{ook een militair}\\

\haiku{{\textquoteright} vroeg de majoor, {\textquoteleft}'t,;}{staat me niet voor dat ik er}{ooit van gehoord heb}\\

\haiku{{\textquoteleft}Ik ben officier,{\textquoteright}, {\textquoteleft};}{zei de officieren ik}{ben uit A. vandaan}\\

\haiku{{\textquoteleft}En daar heb je de,{\textquoteright}, {\textquoteleft}.}{kippen vervolgde Holman}{en dat is de haan}\\

\haiku{{\textquoteright} {\textquoteleft}Wel foei, mevrouw,{\textquoteright} viel, {\textquoteleft}!}{mevrouw Holman haar in de}{redewat een woord}\\

\haiku{{\textquoteleft}Mijnheer Rentink, mag?}{ik u eens verzoeken de}{glazen te vullen}\\

\haiku{{\textquoteleft}'t Is toch een lief,{\textquoteright}.}{talent zei mevrouw Seller}{mij in vertrouwen}\\

\haiku{Ik heb het met een, '.}{jongen naar huis gestuurd want}{ik draagt zelf nooit}\\

\haiku{U hebt u zeker,{\textquoteright}.}{hier of daar aan geblesseerd}{ging de jager voort}\\

\haiku{ik heb nooit aan de, -}{jacht gedaan maar ik heb er}{heel veel van gehoord}\\

\haiku{maar de tweede helft:}{van zijne toespraak maakte}{mij dat duidelijk}\\

\haiku{{\textquoteleft}en al was dat  ,.}{zoo dan ben ik toch niet van}{plan om heen te gaan}\\

\haiku{Adieu, tot Dinsdag,,{\textquoteright}.}{dan we eten om half drie en}{hij ging de deur uit}\\

\haiku{Laten we 't dan, '.}{maar afspreken zooals jet}{me voorgesteld hadt}\\

\haiku{Nu wij opnieuw van.}{onzen toren De klok van}{twaalven hooren slaan}\\

\haiku{{\textquoteright} riep hij eindelijk, {\textquoteleft}?}{uitrepeteer je een rol}{voor een Duitsch drama}\\

\haiku{Ga maar zitten, neem.}{zes vel papier voor u en}{een vollen inktpot}\\

\haiku{Laten we 't nu,.}{nog eens overlezen tot waar}{we gekomen zijn}\\

\haiku{{\textquoteleft}Maar komaan, vooruit,, ',!}{op denzelfden weg en als}{t kan nog doller}\\

\haiku{, en ik gevoelde.}{mij wel honderd pond lichter}{dan voor zijne komst}\\

\haiku{{\textquoteleft}Mijnheer,{\textquoteright} zei hij, {\textquoteleft}ik.}{bewonder uw richtigen}{po\"etischen blik}\\

\haiku{Op eenmaal mijne,?}{rust weer in gevaar brengen}{of hem glad afslaan}\\

\haiku{{\textquoteright} {\textquoteleft}Wilt gij mijn gesp van?}{vijfentwintigjarigen}{dienst in pand hebben}\\

\haiku{voor andren de)!}{stoeitijd Reeds langs den straatweg}{een karretje trok}\\

\haiku{vreeslijk doet hikken?}{En slikken en proesten en}{snurken en snikken}\\

\haiku{Hoeveel wonderen, '!}{zagen ze zwervend int}{prachtige London}\\

\haiku{Ook in het British '.}{Museumt gesprek met}{Athena beluistrend}\\

\haiku{Een ernstig woord over ',?}{sonnetten ~ Mooi ist}{maar een sonnet}\\

\haiku{ure zal de gort en.}{om twee ure het middageten}{worden gebruikt}\\

\haiku{ure zal de gort en.}{om twee ure het middageten}{worden gebruikt}\\

\haiku{Geen voorrechten, ten,!}{minste niet zulke groote in}{het oog vallende}\\

\subsection{Uit: Mengelwerk}

\haiku{als 't donker was,}{werd dat naar voren gedraaid}{met een dun kaarsje}\\

\haiku{Ja mijn kind, ik ben,?}{de baddokter wou je me}{gesproken hebben}\\

\haiku{Dat is zeker niet -.}{goed laat dat liever aan de}{geleerden over}\\

\haiku{En zei hij dus, als,?}{hij verliefd werd zou hij je}{laten trouwen}\\

\haiku{Ik geef de maan van -,.}{al hun kuren om vijf uur}{eten versta je}\\

\haiku{de een vertelt iets,,;}{dat hij weet aan een ander}{die het ook al weet}\\

\haiku{vergeef mij, ik heb.}{u in uw samenspraak met}{den dokter gestoord}\\

\haiku{Dat is een gevolg,;}{van de overgroote teerheid van}{uw gestel Fr\"aulein}\\

\haiku{Fr\"aulein v. S. Neem,,?}{mij niet kwalijk vriend mag ik}{u even iets vragen}\\

\haiku{Kijk, mijnheer, daar heb, '.}{ik er eent mooiste wat}{er te vinden was}\\

\haiku{En zeg er bij, dat,.}{het van iemand komt die je}{niet moogt noemen}\\

\haiku{Dat zou ik ook niet,.}{kunnen want ik weet niet eens}{hoe mijnheer heet}\\

\haiku{ik moet hem zien, en.}{dan kan ik oordeelen hoe}{hij over mij denkt}\\

\haiku{Ik dien nu heen te.}{gaan en hem het eerst hier te}{laten komen}\\

\haiku{Best, mijnheer  (ter),!}{zijde onder het weggaan}{Ja laarzenpoetsen}\\

\haiku{(Hij neemt den zakdoek,).}{in de hand staat op en gaat}{naar de Fr\"aulein}\\

\haiku{U kent allebei,?}{dat clubje Hollanders wel}{zooals ze dat noemen}\\

\haiku{het nieuwe verband,.}{wordt gelegd en van daar gaat}{het naar een ander}\\

\haiku{Daarom roep ik het:}{met den meesten aandrang al}{mijnen lezers toe}\\

\haiku{Mogelijk deel ik,}{daarvan en van hetgeen ik}{nog later hier zien}\\

\haiku{Dan is er leven;}{en drukte op dien anders}{zoo stillen oever}\\

\haiku{van de groote wijnbowl (),;}{ik meen van twee ankers die}{daar geledigd werd}\\

\haiku{Ces toujour \`a vous.}{M. et Madame que l'ons}{doit ces remersiment}\\

\haiku{din\'e, eens goed, de,.}{tweede maal middelmatig}{de derde maal slecht}\\

\haiku{Onder den indruk,}{van al die beschouwingen}{begonnen wij met}\\

\haiku{de noordsche tafel.}{was best en alles was net}{en behagelijk}\\

\haiku{maar ik had buiten:}{de wisselvalligheid van}{het weer gerekend}\\

\haiku{Geen zweem van woning;}{of bebouwing aan de kust}{van het vaste land}\\

\haiku{de vrouwelijke.}{leden van zijn gezin had}{hij nooit gesproken}\\

\haiku{Eindelijk scheen het:}{oogenblik tot beginnen}{gekomen te zijn}\\

\haiku{- Het was duidelijk,.}{dat niemand iets wonderlijks}{in een wonder zag}\\

\haiku{ik eene kleine deur,;}{waardoor ik op straat meende}{te zullen komen}\\

\haiku{de bergen in de;}{rondte liggen zwijgend en}{grijs te sluimeren}\\

\haiku{Wat een Maggio is,.}{wisten we toen evenmin als}{waarschijnlijk 99 pCt}\\

\haiku{hij genezen was,.}{moest hij naar dat feest om zijn}{herstel te vieren}\\

\haiku{Intusschen was het.}{altijd imposant en het}{bezoek dubbel waard}\\

\haiku{twee zware koffers,.}{waarvan de eene niet minder}{dan 68 kilo woog}\\

\section{Multatuli}

\subsection{Uit: Volledige werken. Deel 1. Geloofsbelydenis. Max Havelaar. Brief aan ds. W. Francken Azn. Brief aan den gouverneur-generaal in ruste. Aan de stemgerechtigden in het kiesdistrikt Tiel. Max Havelaar aan Multatuli. Het gebed van den onwetende. Wys my [...]}

\haiku{Zie, daar richt zy zich,... -....}{op en ziet ons dankbaar aan}{Juist riep de vader}\\

\haiku{Ik ben makelaar,,.}{in koffie en woon op de}{Lauriergracht No. 37}\\

\haiku{Men vlucht met het een.}{of ander voorwerp naar het}{einde der aarde}\\

\haiku{weet niet, waarom? - En!}{vraag eens naar den prys van een}{stel biljartballen}\\

\haiku{Het is zo niet in, '.}{de wereld ent is goed}{dat het niet zo is}\\

\haiku{Hy had dan ook wel,.}{iets van een Duitser en van}{een reiziger ook}\\

\haiku{Ja, ja, hy was het,!}{die my uit de handen van}{den Griek had verlost}\\

\haiku{Ik schreide, en bad,.}{om genade want ik zat}{vreselyk in angst}\\

\haiku{Hy zag zeer bleek, en, '.}{toen ik hem vroeg hoe laat het}{was wist hyt niet}\\

\haiku{{\textquoteright} Myn vrouw zei hierop.}{dat Frits dan voortaan niet meer}{mee zou naar den krans}\\

\haiku{Maar Louise schreide,.}{weer en de dames zeiden}{dat het heel mooi was}\\

\haiku{Over hydraulische.}{onderwerpen in verband}{met de rystkultuur}\\

\haiku{Maar hy zegt, dat de.}{invloed zich eerst openbaart in}{het tweede geslacht}\\

\haiku{Natuurlyk had hy,.}{my om geld gevraagd en van}{zyn pak gesproken}\\

\haiku{hoe zou het wezen,,?}{dacht ik als ik hem de plaats}{van Bastiaans gaf}\\

\haiku{Dat myn naam niet op,.}{den titel zou staan omdat}{ik makelaar ben}\\

\haiku{Bovendien, ik houd.}{niet van mensen die altyd}{ontevreden zyn}\\

\haiku{- Je weet immers dat.}{die m'nheer gister alles}{heeft meegenomen}\\

\haiku{Wat hy zeide, was,}{gewoonlyk lang overdacht een}{eigenaardigheid}\\

\haiku{Daar is me juist iets,...?}{voorgekomen dat zou de}{Regent ons verstaan}\\

\haiku{hy is beschaafd, niet? -... -?}{waar O ja En hy heeft een}{grote familie}\\

\haiku{Ik beveel me zeer,!}{aan voor uw medewerking}{m'nheer Verbrugge}\\

\haiku{{\textquoteright} niet uitsprak, voor hy ' {\textquoteleft}{\textquoteright}.}{haartniet zondigen had}{mogelyk gemaakt}\\

\haiku{En, kinderen als,.}{ze waren vermaakten zy}{zich met hun nieuw huis}\\

\haiku{Rypt niet uw padi?}{dikwerf ter voeding van wie}{niet geplant hebben}\\

\haiku{Wat zal er gezegd?}{worden in de dorpen waar}{wy gezag hadden}\\

\haiku{zyn schryver heeft het,...!}{my gezegd en bovendien}{dat bruske vragen}\\

\haiku{- Ook de staten die,,.}{ik vandaag ontving zyn vals}{ging Havelaar voort}\\

\haiku{Het was myn hart dat!}{ge daar hebt opgeslikt als}{een versnapering}\\

\haiku{En wacht geen meisjes ',.}{op alst uit is want dit}{neemt de stichting weg}\\

\haiku{niemand kan hem dus.}{een grondige kennis der}{zaken betwisten}\\

\haiku{Er is een juffrouw...}{flauw gevallen toen hy van}{dat zwarte kind sprak}\\

\haiku{Heimlich erz\"ahlen.}{die Rosen Sich duftende}{M\"archen ins Ohr}\\

\haiku{Marie is daar in -,?}{dien rooien tuin waarom rood}{en niet geel of paars}\\

\haiku{{\textquoteleft}deze kapel is...}{opgericht door den bisschop}{van Munster in 1423}\\

\haiku{Ze bewegen zich,.}{als een hobbelpaard minus}{nog het va et vient}\\

\haiku{Eigenliefde en.}{verveling dringen u iets}{dergelyks te doen}\\

\haiku{- Welnu, dan weet je.}{dat er peperkultuur in}{het Natalse is}\\

\haiku{Bovendien, ik had... -'!}{allerlei gekheden in}{het hoofd Ta oesah}\\

\haiku{Ik zei dat elk mens.}{in zyn medemens een soort}{van konkurrent ziet}\\

\haiku{Den gevangene.}{immers is men onderhoud}{en voedsel schuldig}\\

\haiku{Hebt ge niet geweend,...}{by die moeder vruchteloos}{zoekend naar haar kind}\\

\haiku{, zou die betaling.}{spoedig zyn middelen zyn}{te boven gegaan}\\

\haiku{Gesichtchen nahe,,.}{dr\"uckt Dann lachst du freudig das}{ist auch Gef\"uhl}\\

\haiku{Men verkrygt daardoor '.}{den roep van bekwaamheid en}{yver voors lands dienst}\\

\haiku{Gisteren heb ik:}{Sjaal man gezien met zyn vrouw}{en hun jongetje}\\

\haiku{Hy is bleek als de,,.}{dood zyn ogen puilen uit en}{zyn wangen staan hol}\\

\haiku{Men ziet uit alles.}{dat Stern jong is en weinig}{ondervinding heeft}\\

\haiku{Ook de ouders van.}{zyn vrouw woonden altyd in}{datzelfde distrikt}\\

\haiku{Ik zal uitblyven....}{driemaal twaalf manen deze}{maan rekent niet mee}\\

\haiku{Zie, Adinda, kerf.}{een streep in je rystblok by}{elke nieuwe maan}\\

\haiku{het vastgeknoopt aan '.}{t eindelyk terugzien}{onder den ketapang}\\

\haiku{{\textquoteleft}wie van ons zal het,?}{lichaam verslinden dat daar}{daalt in het water}\\

\haiku{En ook vroeg hy zich,?}{wie er toch wel wonen zou}{in zyns vaders huis}\\

\haiku{- en ze hadden zich.}{neergelaten aan sterke}{rotan-koorden}\\

\haiku{Maar nog altyd is...!}{myn ziel En myn hart bitter}{bedroefd Adinda}\\

\haiku{De man was zeker!}{bezig met een jaarverslag}{over rustige rust}\\

\haiku{Er is altyd een.}{soort van belediging in}{dat niet herkennen}\\

\haiku{En ze hebben niets,,!}{te eten en ze slapen op}{den weg en eten zand}\\

\haiku{Daar zit hy rustig,...}{by vrouw en kind en tekent}{borduurpatroontjes}\\

\haiku{{\textquoteright} Ja, ik hoor 't wel, ',!}{ik hoort wel dat roepen}{om wraak over myn hoofd}\\

\haiku{Het gesprek liep over.}{de verwachte beslissing}{van de Regering}\\

\haiku{Maar ik zal tot hem,.}{gaan en hem aantonen hoe}{hier de zaken staan}\\

\haiku{wederlegging der!}{hoofdstrekking van myn werk}{is onmogelyk}\\

\haiku{voor gezondheid en, '.}{leven maar voorn gering}{deel van hun welstand}\\

\haiku{Fluks wierp hy 't in, '.}{de goot en reinigde het}{metn stalbezem}\\

\haiku{Nederland heeft niet.}{verkozen recht te doen in}{de Havelaarszaak}\\

\haiku{de herinnering '.}{int leven roept aan den}{Javasen oorlog}\\

\haiku{{\textquoteleft}man, bloemlezer, weet?}{je wel wat je beweert ons}{te willen leren}\\

\haiku{Dit is al iets in!}{onzen tyd van jammerlyk}{ordinarisme}\\

\haiku{Me dunkt toch dat ze,,.}{vooral met het oog op z'n}{dood zeer treffend zyn}\\

\haiku{Op strandplaatsen als,.}{Marseille verbasteren}{de rassen zeer snel}\\

\haiku{) Het is zeer gewaagd '.}{dit op grond vann enkel}{woord aan te nemen}\\

\haiku{mannelyk voor, en '.}{even onsmakelyk alst}{bedoeld delikt zelf}\\

\haiku{Men behoorde den.}{moed te hebben zyner}{gewetenloosheid}\\

\haiku{Het blad zit, als 't ',.}{yzer vann houweel loodrecht}{op den houten steel}\\

\haiku{{\textquoteright} Maar nooit bleek er dat.}{er iets gedaan werd om dit}{doel te bereiken}\\

\haiku{De terechtwyzing:}{van den heer Bensen is my}{te meer welkom}\\

\haiku{Doch juist, omdat het;}{u niet maar te doen was om}{een boek te schrijven}\\

\haiku{Gy duldt geen spat op, '...}{uw kleed geen schampschot int}{gelaat ge dwingt my}\\

\haiku{5 - Men begrypt, hoe.}{ik er aan hecht dat deze}{brief bewaard blyve}\\

\haiku{maar in de buurt van.}{Job vond ik werkelyk den}{brief waarvan hy spreekt}\\

\haiku{- men wachtte op een ',;}{ommekeer vant lot op}{Himmels einfallen}\\

\haiku{en al ware het, -.}{dat ik my weder bedroog}{ik kan niet anders}\\

\haiku{Zy het dezen avond,,,...{\textquoteright}}{schreef ik zy het heden nacht}{zy het morgen vroeg}\\

\haiku{Ik ging niettemin,.}{met zachte vriendelyke}{vermaningen voort}\\

\haiku{dat ik dwaalde, toch.}{getoond had in die dwaling}{eerlyk te zyn}\\

\haiku{Wederom is een.}{nieuwe parlementaire}{zitting geopend}\\

\haiku{Neen, niet omdat hy,.}{gedaan heeft wat hy kon maar}{omdat hy iets kon}\\

\haiku{Het geheel bestaan.}{van ons Vaderland staat of}{valt met dat gebouw}\\

\haiku{dan zoude het een.}{onbillykheid wezen aan}{valsheid te denken}\\

\haiku{- D\^as en broekie f\^a,,.}{me bro\^ertje weet uwe f\^a me}{bro\^ertje d\^a doot is}\\

\haiku{Ik zelf vergelyk,.}{het by een kalf met zeven}{staarten zonder kop}\\

\haiku{zeer goed, - dat stuk nu,,.}{was w\'elgeboren en heeft}{geen geluk gehad}\\

\haiku{Er is geen enkel, '}{boek goed geschreven en het}{uwet minst van al.}\\

\haiku{Wie zich toelegt op.}{goed schryven kan nooit v\'e\'el voor}{den dag  brengen}\\

\haiku{de kleine moet in,.}{de lucht en de bonne volgt}{U met het kindje}\\

\haiku{Fanny moet er by,... {\textquoteleft}{\textquoteright} '.}{en de onsterfelykheid}{fancy int eind}\\

\haiku{Wat had hy, met zyn!}{nuchter verstand nog veel goeds}{kunnen uitrichten}\\

\haiku{Uw ouders zullen,.}{schateren over de grappen}{die lieve mensen}\\

\haiku{Wie 't goede doet,}{Opdat een God hem lonen}{zou maakt juist d\'a\'ardoor}\\

\haiku{Het geloof aan God '.}{heeft geen vaster grond dant}{geloof aan spoken}\\

\haiku{De vader zal het -,;}{kind omhoog houden neen dat}{zal de moeder doen}\\

\haiku{Hy spoorde na, en... '!}{snuffelde daar blaasde iets}{int geboomte}\\

\haiku{Welnu, de arbeid,...}{is volbracht wy stonden u}{onze akkers af}\\

\haiku{22 - Ryst stampen in. ' '.}{een houten blokt Woord is}{n onomatopee}\\

\haiku{De afgedrukte.}{tekst is overeenkomstig de}{Bloemlezing van 1865}\\

\haiku{zyde555kan worden -;}{gelyk gesteld met die op}{Java volgens Hs}\\

\haiku{het leek ongewenst,, -;}{die te bestendigen.13313Broek}{Berlyn volgens 1881}\\

\haiku{op de15010iets tot stand te -;}{brengen ten voordele van}{Natal volgens Hs}\\

\haiku{ik bezit daarvan -;}{nog de minuut16636hadden plaats}{gehad volgens 1881}\\

\haiku{van het vermoeden ' -;}{der25914de opvatting vant}{begrip volgens 1881}\\

\haiku{residentie op,:}{West-Java verdeeld in}{drie afdelingen}\\

\haiku{in militaire;}{dienst terug belegerde}{hij Utrecht en Naarden}\\

\haiku{Zweeds schrijfster (1801-1865),.}{die voor de emancipatie}{der vrouw ijverde}\\

\haiku{Op 15 November:}{1851 zond hij dit spel onder}{de nieuwe titel}\\

\haiku{later ambtenaar.}{aan het Ministerie van}{Handel te Parijs}\\

\haiku{Rangkasbetoeng en) (:}{Sadjira en Lebak}{Kidoeldistricten}\\

\haiku{Translated from the.}{original Manuscript by}{Alphonse Nahuys}\\

\haiku{boomsoort, uitstekend,.}{geschikt voor timmer- en}{sierhout z.g. teakhout}\\

\haiku{18.5 gedrukt had, en.}{Multatuli in 1875 uit}{het hoofd invulde}\\

\subsection{Uit: Volledige werken. Deel 2. Minnebrieven. Over vrijen arbeid in Nederlands-Indi\"e. Brief aan Quintillianus. Idee\"en, eerste bundel}

\haiku{En halen kan ik,.}{haar ook niet omdat ik geen}{geld heb voor de reis}\\

\haiku{Helpt ze hem niet, troost, -?}{ze hem niet zou hy sterker}{wezen zonder h\'a\'ar}\\

\haiku{Tot nader order,,,!}{als de jongens op zyn de}{jongens die v\'o\'orgaan}\\

\haiku{En men speculeert '?}{opt gebrek-lyden}{onzer kinderen}\\

\haiku{Papa is moe, moe,,;}{van gort stroop Oostenrykers}{en assurantie}\\

\haiku{een vrouw, die meer was,;}{dan de paus schoon ze maar eten}{had voor drie dagen}\\

\haiku{Le D\'esespoir van,....}{Lamartine is veel}{te lang en te mooi}\\

\haiku{Ik heb gemerkt dat,.}{er veel zaken zyn die men}{niet zegt aan vrouwen}\\

\haiku{En God, of de god...? '!}{begrypt ge dat woordt Is}{verwant met weten}\\

\haiku{Dit is alles zeer, ',.}{duidelyk en wiet niet}{begrypt is verdoemd}\\

\haiku{Gy, die groter zyt,,.}{dan ik steek uw arm uit en}{pluk opdat ik ete}\\

\haiku{- Het middel is zeer,.}{eenvoudig antwoordde de}{vrome heremiet}\\

\haiku{Wat moet ik doen, dat,?}{myn kind niet van my wegga}{als het lopen kan}\\

\haiku{- Wie nu geduldig,,.}{melkt tot het laatste toe brengt}{vette melk tehuis}\\

\haiku{- Eilieve, wie zal?}{beletten dat ze weet wat}{ik haar niet leerde}\\

\haiku{Ook wy zouden niets,.}{geweten hebben als ge}{ons niets hadt gezegd}\\

\haiku{Hassan wenste den,.}{vogel lengte van veren}{en noemde hem Rock}\\

\haiku{Hoe kunt gy zo scherp,,?}{zyn Max gy die toch zo goed}{weet lief te hebben}\\

\haiku{Maar om haar goed te,.}{beschryven zou ik haar by}{my moeten hebben}\\

\haiku{En zonder yverzucht,.}{hoor ik aan wat zy gezegd}{heeft zonder schaamte}\\

\haiku{Dat beestje... welnu,... '...... '}{uit verveling heb ikt}{wel eens geknepen}\\

\haiku{Want er komt veel wet.}{in de historie die my}{nu overkomen is}\\

\haiku{De dochters van myn,.}{man stoppen scheef en stikken}{dat het schande is}\\

\haiku{men dwingt u niet tot;}{goddienen op een manier}{die u niet aanstaat}\\

\haiku{Dit alles leidt nu:}{tot de volgende vreemde}{conclusi\"en}\\

\haiku{In Indi\"e zal de.}{stryd gevoerd worden om de}{wereldheerschappy}\\

\haiku{Ik denk het wel, van,...}{tyd tot tyd omdat ik hier}{zoveel verdriet heb}\\

\haiku{- dan nog kan ik niet.}{uitgaan om te schryven in}{die boekenkamer}\\

\haiku{Dat wegstoppen van.}{aandoeningen is my te}{lastig op den duur}\\

\haiku{- Ik dank u voor 't,,...}{schryven heer Ridder doch myn}{gezin lydt gebrek}\\

\haiku{Waarom laten zy?}{ons vier lange maanden in}{de onzekerheid}\\

\haiku{Ga vrymoedig tot,:}{hem en zeg aan den Keizer}{van Insulinde}\\

\haiku{Liever had hy nog, ',.}{slechter gespeeld of int}{geheel niet dan z\'o}\\

\haiku{Hy is wat moe van ',,!}{t gaan maar als hy hangt Is}{dat terstond voorby}\\

\haiku{Vriend Nathan, help my 't,...}{kind Eens overzetten op dien}{and'ren schouder}\\

\haiku{daar valt het weer, En...-}{ryst nu langzaam weer omhoog}{Dat zie ikzelf}\\

\haiku{kind'ren bauwt hem na,, '...}{En sart hem dat hy n\'og wat}{zegge aant kruis}\\

\haiku{Is dat ironie, is,,...?}{dat sarkasme is dat hoon}{of is dat domheid}\\

\haiku{het afnemen van!}{buffels aan de bevolking}{is het ergste niet}\\

\haiku{Die zo-even nog?}{daarginds natie en Koning}{representeerde}\\

\haiku{{\textquoteleft}Als hy langer hier,.}{was gebleven ware hy}{stellig vergeven?{\textquoteright}32}\\

\haiku{Ik weet niet of gy ',,!}{t genoeg vindt o Kiezers}{maar ik vind het v\'e\'el}\\

\haiku{t Is schande, dat!}{ge my aan zoiets blootstelt}{door uw slordigheid}\\

\haiku{{\textquoteright} Misschien heb ik 't.}{antwoord van dien ouden man}{niet goed begrepen}\\

\haiku{Dat begreep ik toen,,.}{niet omdat ik maar een kind}{was beste Tine}\\

\haiku{Dit althans schreef my,, '}{de kleine maar ik verdenk}{haar nu van foppery}\\

\haiku{5 - Zie 't boek over,, (.}{de Veilingen eerste deel}{pag. 145eerste druk)32}\\

\haiku{30 - Ziehier wat Prof.:}{Veth daarvan zegt in den Gids}{van Augustus 1860}\\

\haiku{Multatuli, niet,.}{alleen in aangenomen}{naam maar inderdaad}\\

\haiku{{\textquoteright} Verbeeld u het lid,:}{ener jury die  tot een}{kollega zeide}\\

\haiku{Maar toen ik 't las,.}{was ik beschaamd en heb de}{P\`ene bedankt}\\

\haiku{Waar de fantaisie.}{kristal meende te zien vond}{ze schimmel en vuil}\\

\haiku{Later blykt dat ze}{schoon is en dat ik pedant}{was door te meenen}\\

\haiku{Uit dien strijd tussen (,.}{vurige begeerte naar}{waarheidexact maths}\\

\haiku{Want, Sire, ik lees,,.}{geen Engels en geen dichters}{ook dat begrypt ge}\\

\haiku{ik vragen wat hy,,}{is door niets te zeggen door}{niets te verrichten}\\

\haiku{*~         Ja, er waren, '!}{veel bylagen by dien brief}{t was een bundel}\\

\haiku{Vier jaren lang was,}{Sjaalman bespot en gesard}{door Droogstoppel v\'o\'or}\\

\haiku{Volgt niet oorlog op,,?}{oorlog roering op roering}{onrust op onrust}\\

\haiku{Is 't zyn schuld dat,?}{uw vaderland zo bar is}{zo onvriendelyk}\\

\haiku{Of als hy, al te, ',?}{lang geperst aant gisten}{was gegaan hyzelf}\\

\haiku{Altyd zulke fraaie,!}{naar studie en wetenschap}{riekende woorden}\\

\haiku{Ziet hier de staten,,.}{van invoer van verkoop van}{netto provenu}\\

\haiku{En als de schryver?}{van zulke ongepaste}{klaagbrieven aanhoudt}\\

\haiku{Take it away...{\textquoteright} To...!}{honester men maakt plaats}{voor eerlyker lui}\\

\haiku{Ik wil trachten u, ',.}{dit te zeggen schoont me}{moeite kost lezer}\\

\haiku{Wie twyfelt er aan,?}{of het goede beter is}{dan het kwade}\\

\haiku{De kleine plichten,.}{die zyn leeftyd hem oplegt}{vervult hy gaarne}\\

\haiku{In \'e\'en woord, tot nog.}{toe arbeidde hy geheel}{en al vrywillig}\\

\haiku{En ik ben blyde:}{toegegeven te hebben}{in een wens die}\\

\haiku{Er zal geen hagel '.}{vallen zolangt gewas}{op het veld staat}\\

\haiku{Ieder inwoner.}{van den Staat zal behoorlyk}{gevoed worden}\\

\haiku{o d\'at zou ik ook, '...}{wel kunnen en denk aant}{ei van Columbus}\\

\haiku{Ik citeer myzelf, '}{gaarne alst dienen kan}{om aan te tonen}\\

\haiku{{\textquoteleft}dat de Javaan het, '.}{zyne ontving al ist}{dan ook wat weinig}\\

\haiku{Ook een tweede stuk.}{van gelyke strekking is}{reeds drie jaren oud}\\

\haiku{Ge smoort met zulke.}{kinderachtigheid uzelf en}{uw eigen belang}\\

\haiku{Multatuli z\'al,.}{niet bezwijken omdat hij}{niet bezwijken k\'an}\\

\haiku{Dat ik moeilyk werk ',.}{met zo'n logge vracht opt}{gemoed is ook waar}\\

\haiku{Een verzameling,,,.}{van hout steen kalk enz. is niet}{altyd een gebouw}\\

\haiku{Een vergadering.}{van mensen is niet altyd}{een gezelschap}\\

\haiku{Eens-voor-al, het:}{woordjen is gebruik ik}{tot verkorting van}\\

\haiku{Er zyn meer muggen,.}{dan wespen meer kappellui}{dan droogstoppels}\\

\haiku{Ik bied 'n vel druks.}{voor een goed voornaamwoord van}{de tweede persoon}\\

\haiku{Als 'n hardloper, '.}{z'n been breekt ist bal par\'e}{by de kruipers}\\

\haiku{Ik weet niet hoe 't,}{komt maar  Ein M\"archen}{aus alten Zeiten}\\

\haiku{Ik viel neer op 'n,,,...}{stoel afgemat uitgeput}{byna moedeloos}\\

\haiku{Ik was zwaar gewond, '.}{en kont hoofd niet keren}{naar de pendule}\\

\haiku{Zy hadden de vraag, '.}{der moeder niet verstaan en}{zagent kind niet}\\

\haiku{Maar als men heenstapt,.}{over dien eersten regel dan}{volgt de rest vanzelf}\\

\haiku{A en B wilden,.}{een rivier overtrekken die}{van Oost naar West liep}\\

\haiku{Dan heb ik in myn.}{Idee\"en wat meer ruimte voor}{andere zaken}\\

\haiku{een voorzichtigheid,;}{die soms aan bezorgdheid voor}{teleurstelling grenst}\\

\haiku{zijn van de totaal,:}{bedorven politieke}{atmosfeer dat is}\\

\haiku{Ik vind dat dit de.}{passage naar den kelder}{belemmeren zou}\\

\haiku{ik 't af, dat men.}{z'n aangezicht gebruikt om}{er op te liggen}\\

\haiku{Na al 't bidden,.}{om den Geest verbaas ik my}{over die verbazing}\\

\haiku{Dat komt er van, als '!}{men een opzichter niet aan}{t woord laat komen}\\

\haiku{Maar overigens - o!}{heerlyke overeenstemming}{in verscheidenheid}\\

\haiku{{\textquoteright} Gy meent misschien dat,...!}{ze naar bed gingen pour tout}{de bon ditmaal mis}\\

\haiku{{\textquoteright} Hy vraagt om zegen ' {\textquoteleft} '{\textquoteright}.}{opt weeshuiswaarin wij}{t zo goed hebben}\\

\haiku{Er zou voor ons geen, '.}{plus bestaan wanneer wet}{minus niet kenden}\\

\haiku{Ik wandelde met.}{kleinen Max. Voor ons uit ging}{een man met zyn kind}\\

\haiku{Welke werking heeft?}{op dien steen de hefboom die}{1/x deel korter is}\\

\haiku{Onsterfelykheid '.}{zonder eeuwigheid isn}{koord met \'e\'en eind}\\

\haiku{By 't opstaan 's,.}{morgens zag ik dien berg en}{wat hy uitblaasde}\\

\haiku{Ik klaag niet \'omdat,.}{het zo is maar juist d\'a\'arover dat}{het zo wezen moet}\\

\haiku{Die som kan niet meer,:}{wezen dan twee ze kan niet}{minder zyn dan twee}\\

\haiku{Die God voegt samen,,,,,,,,...}{ontbindt maakt vermaakt richt wendt}{buigt heft perst en plet}\\

\haiku{Een schoolknaap kan ze,.}{u aantonen en ditmaal}{zonder vals vernuft}\\

\haiku{Dien Constantyn noemt,.}{ge groot en de rest is even}{waar als die grootheid}\\

\haiku{Agatha vraagde niet ':}{naart verband tussen h\'a\'ar}{zonden en Adams val}\\

\haiku{Wie wysheid spreekt in den,,.}{tempel en dwaasheid geeft aan}{zyn vrouw is een dief}\\

\haiku{wat Origenes deed {\textquoteleft}{\textquoteright},.}{om des hemelryks wil zie}{d\'at begryp ik niet}\\

\haiku{Zo spreekt geen meisje...{\textquoteright}.}{Ziedaar schering en inslag}{van de opvoeding}\\

\haiku{*~         Ik wou wel eens ' {\textquoteleft}{\textquoteright},}{nHeer zien die de macht had}{u te beletten}\\

\haiku{Ook moest gy weten.}{wat er te doen was tegen}{de apen-attractie}\\

\haiku{Franschen zouden des.}{anderen daags myn voorbeeld}{nagevolgd hebben}\\

\haiku{eerste hoofdstuk te.}{schryven boven den aanvang}{dezer historie}\\

\haiku{Jazelfs, hy begon.}{dien klank lief te krygen om}{de betekenis}\\

\haiku{- Oui, mon fils, du plus,...}{grand malheur du seul malheur}{qui soit au monde}\\

\haiku{Men leeft met en in,,.}{de heilige Rosalia}{Lucia Monica}\\

\haiku{Zo-even had hy}{den monnik verdriet gedaan}{door de betuiging}\\

\haiku{Nous sommes pay\'es,!}{pour le savoir nous autres}{passagers de pont}\\

\haiku{{\textquoteleft}dat dit epitheton!}{naar omstandigheden moet}{gewyzigd worden}\\

\haiku{d\'at boek kon wel eens.}{de vonk aanbrengen om die}{te doen ontvlammen}\\

\haiku{Er waren altyd ',.}{Grieken diet geloofden}{vooral in Boeoti\"e}\\

\haiku{Ik verwacht nu 'n:}{annonce van dezen of}{genen bierbrouwer}\\

\haiku{Ge wist dat ik 't.}{Volk misselyk had gemaakt}{van uw programmen}\\

\haiku{aan 't bouwen van ',.}{n huis met dikke muren}{en hoge torens}\\

\haiku{Het Nederlandse.}{Volk kan gerust wezen over}{die 115 millioen}\\

\haiku{En, Sire, dat is, ' {\textquoteleft}{\textquoteright}.}{wel waar maart was de vraag}{niet of ikmooi schreef}\\

\haiku{Het was niet blauw of,,:}{geel niet rood of groen of grijs}{maar alle verf door\'e\'en}\\

\haiku{Er zyn velen die,.}{geloven in myn God de}{Noodzakelykheid}\\

\haiku{Want, ook daarvoor zorgt,, '.}{de Natuur als men d\'at doet}{zalt niet vallen}\\

\haiku{'t Moet dus geweest.}{zyn v\'o\'or de ontdekking der}{staathuishoudkunde}\\

\haiku{Ja, van de toekomst,.}{als ons die muffe schoollucht}{zal afgewaaid zyn}\\

\haiku{die Wouter vergeefs '.}{tegent licht hield om er}{meer van te weten}\\

\haiku{- Ik heb gevraagd aan,,.}{m'n moeder zei hy maar ze}{wil me niets geven}\\

\haiku{- 't Is maar, weet je,.}{om je te zeggen dat we}{heel fatsoenlyk zyn}\\

\haiku{Lang na Habakuk,:}{dacht Wouter nog meermalen}{aan haar deemoedig}\\

\haiku{Ze was foei-lelyk,,.}{nogal vuil en bovendien}{wat onrecht van leest}\\

\haiku{Vyf vingers heb ik '.}{aan myn hand Ter eer vant}{lieve vaderland}\\

\haiku{De pruik maakte 'n,.}{vreugdesprong en de krullen}{omhelsden elkaar}\\

\haiku{En ook ken-i al '.}{de werrikwoorden f'nt}{frouwelik cheslacht}\\

\haiku{- Skenkerrissin, Trui,... '.}{en blaas es in de tuit d'r}{sitn blaatje foor}\\

\haiku{want ik heb altyd ',... '}{heel int fatsoenlyke}{gebakerd weet je}\\

\haiku{want-i was in de,,!}{granen en d\'a\'ar kan je na}{me vragen hoor je}\\

\haiku{Maar als men driftig, ' ' '.}{is neemt men welns meert}{een voort ander}\\

\haiku{O Baker, hoor ons,.}{juichen aan Als wij met U}{uit baak'ren gaan}\\

\haiku{Weg van Hem, en vaart,.}{naar de diepe gewelven}{waar geen Baker is}\\

\haiku{{\textquoteleft}De verhouding met.}{het ryk der Pieterse's is}{allercordiaalst}\\

\haiku{En juffrouw Laps hield,.}{veel van oefenen zoals}{wy gezien hebben}\\

\haiku{- Zie je, jongeheer, '.}{dat kan ik ze maar niet aan}{t verstand brengen}\\

\haiku{Die blik slaagde, maar,}{ik kan niet ontveinzen dat}{men verwonderd was}\\

\haiku{Neen, de verbazing ' '.}{vant gezelschap hadn}{heel anderen grond}\\

\haiku{- Maar wat moeten we?}{dan in godsnaam aanvangen}{met dien kwajongen}\\

\haiku{Jazelfs wordt spotten,.}{plicht waar redenering te}{vergeefs wezen zou}\\

\haiku{Maar toen m'n zuster, ' {\textquoteleft}{\textquoteright}.}{trouwde heeft zen grote}{kastmeegekregen}\\

\haiku{Dat deed Fancy niet,.}{dat deed de priester om die}{zes schellingen}\\

\haiku{Dat deed ik reeds te,:}{Lebak toen ik tot de}{bevolking zeide}\\

\haiku{Ik wou zo graag 't.}{kopyrecht daarvan aan m'n}{kinderen laten}\\

\haiku{Ook is 't onjuist.}{dat ik niet meer over hem zou}{te zeggen hebben}\\

\haiku{Doch byna overal ':}{elders sluiten wy zulke}{woorden metn t}\\

\haiku{Nu, als gyzelf op,.}{z'n nederlands denkt hebt ge}{kans van juist raden}\\

\haiku{Elk mot wordt weer op.}{zyn beurt door andere mots}{krachteloos gemaakt}\\

\haiku{Dat talent is my,}{niet gegeven waarin dan}{ook de reden ligt}\\

\haiku{Dit wordt wel gedaan}{in de voor omstreeks een jaar}{by den uitgever}\\

\haiku{De voorbeelden die,.}{ik hiervan kan aanhalen}{zyn anecdotisch}\\

\haiku{niet iedere sloep '.}{geoorloofdn schip van dien}{kant te naderen}\\

\haiku{op 't herstellen.}{der latynse u in haar}{wezenlyken klank}\\

\haiku{De etymologie '.}{vant woord presenning kan}{ik niet opgeven}\\

\haiku{Daarin schitteren.}{trouwens de meeste scheepstermen}{door afwezigheid}\\

\haiku{We zyn denkdieren,,:}{kunnen denken en voelen}{aandrang tot denken}\\

\haiku{het uitroeien der.}{vervloekte gewoonte van}{niet-begrypen}\\

\haiku{Des te gegronder '.}{is alzo de klacht int}{nu volgend nummer}\\

\haiku{Men zie hierin geen - '!}{bewondering van m'n werk}{t lykt er niets naar}\\

\haiku{Die zogenaamde.}{tussenscholen zullen nu}{wel niet meer bestaan}\\

\haiku{Maar dat men hem nu,.}{ook als auteur verheft is}{me enigszins  nieuw}\\

\haiku{Ziehier weder 'n,.}{fout van dezelfde soort als}{in 183 385 en 409}\\

\haiku{Met dit woord spreekt men.}{op beleefde wys een niet}{zeer jonge vrouw aan}\\

\haiku{de ene heeft op het:}{titelblad van het eerste}{deel de vermelding}\\

\haiku{En dit~3183'n K, ' ' -;}{die slechtsn C is metn}{stokje volgens 1879}\\

\haiku{Eerst in het najaar.}{van 1857 gelukte het de}{opstand te dempen}\\

\haiku{In 1753 hield Buffon:}{in de Acad\'emie Fran\c{c}aise}{zijn intreerede}\\

\haiku{derde regel van:}{het tot volkslied geworden}{gedicht van Heine}\\

\haiku{Het oratorium ().}{Die Sch\"opfung1798 is een van}{zijn meesterwerken}\\

\haiku{kozakkenhetman, (-).}{vertrouweling van Peter}{de Grote16441709}\\

\haiku{Aanvankelijk bij;}{de rechterlijke macht in}{Nederlands-Indi\"e}\\

\haiku{Sappho: grootste Griekse, (.}{dichteres geboren op}{Lesbos\ensuremath{\pm} 600 v}\\

\haiku{resp. staatsman (1623-1672) (-).}{en raadpensionaris}{van Holland16251672}\\

\subsection{Uit: Max Havelaar of De koffiveilingen der Nederlandsche Handelmaatschappy}

\haiku{Maar geen wonder dat,!}{later de toon bitterder}{het woord scherper werd}\\

\haiku{ik laat inspannen;}{en rij naar de Fille de}{madame Angot}\\

\haiku{Moet ik u zoeken,?}{in de wolken of in de}{straten eener stad}\\

\haiku{want niet ieder had.}{een hoornen huid waarop de}{indruk afschampte}\\

\haiku{{\textquoteright} En zij gaf eerst den,,.}{wil dan de kracht eindelijk}{de overwinning}\\

\haiku{De minnebrieven;}{zijn een vonkelend vuurwerk}{van vernuft en geest}\\

\haiku{dat de kunstenaar,.}{Goethe hier verkeerd deed maar}{de mensch Goethe juist}\\

\haiku{wat ik verlang... maak...}{dat je in drie maanden de}{eerste bent op school}\\

\haiku{- Heerejesis, zeit zijn,!}{moeder waar haalt de jongen}{de dingen van daan}\\

\haiku{Ik Heb van uw roem,.}{als rechtsman veel gehoord En}{wilde Van Huisde}\\

\haiku{Onder den titel;}{Verspreide stukken zijn er}{eenigen verzameld}\\

\haiku{Nu, zeker, Laura.}{Ernst of de school des levens}{is een juweeltje}\\

\haiku{lord Ci-devant:}{en de tabakshandelaar}{van de Brakke Grond}\\

\haiku{Lezer, ik heb u.}{genoeg gezegd om u op}{den weg te helpen}\\

\haiku{Wie redeneert, dient,.}{de rede en de rede}{zal u vrijmaken}\\

\haiku{Men vlucht met het een.}{of ander voorwerp naar het}{einde der aarde}\\

\haiku{De familie die,.}{van dezen arbeid leven}{kan heeft weinig noodig}\\

\haiku{Het is zoo niet in, '.}{de wereld ent is goed}{dat het niet zoo is}\\

\haiku{Hy had dan ook wel,.}{iets van een Duitscher en van}{een reiziger ook}\\

\haiku{Ja, ja, hy was het,!}{die my uit de handen van}{den Griek had verlost}\\

\haiku{Ik schreide, en bad,.}{om genade want ik zat}{vreeselyk in angst}\\

\haiku{Hy zag zeer bleek, en, '.}{toen ik hem vroeg hoe laat het}{was wist hyt niet}\\

\haiku{Maar Louise schreide,.}{weer en de dames zeiden}{dat het heel mooi was}\\

\haiku{Over hydraulische.}{onderwerpen in verband}{met de rystkultuur}\\

\haiku{Maar hy zegt, dat de.}{invloed zich eerst openbaart in}{het tweede geslacht}\\

\haiku{Ook vond ik brieven,.}{waarvan velen in talen}{die ik niet verstond}\\

\haiku{Natuurlyk had hy,.}{my om geld gevraagd en van}{zyn pak gesproken}\\

\haiku{Bovendien hy zag,...}{er armoedig uit en wist}{niet hoe laat het was}\\

\haiku{hoe zou 't wezen,,?}{dacht ik als ik hem de plaats}{van Bastiaans gaf}\\

\haiku{Ik ben zeker dat.}{hy met tweehonderd gulden}{tevreden zou zyn}\\

\haiku{Wat hy zeide, was -}{gewoonlyk lang overdacht een}{eigenaardigheid}\\

\haiku{Daar is me juist iets,...?}{voorgekomen dat zou de}{Regent ons verstaan}\\

\haiku{Maar toon my iets uit,.}{je weitasch dan denkt hy dat}{we d\'a\'arover spreken}\\

\haiku{D\'a\'ar moet geleden,,!}{zyn veel geleden daar is}{ondervonden}\\

\haiku{hy is beschaafd, niet? -... -?}{waar O ja En hy heeft een}{groote familie}\\

\haiku{Ik beveel me zeer,!}{aan voor uw medewerking}{m'nheer Verbrugge}\\

\haiku{, en daar het spreken, '.}{zoo moeielyk viel brak men}{t gesprek af}\\

\haiku{De toestand is er!}{sedert dien tyd niet beter}{op geworden}\\

\haiku{{\textquoteright} niet uitsprak, voor hy {\textquoteleft}{\textquoteright}.}{haar datniet zondigen had}{mogelyk gemaakt}\\

\haiku{een argument, zooals,.}{men weet waartegen niets valt}{intebrengen}\\

\haiku{Hy wie zyn woord spreekt,.}{opdat zy zich oprichten}{in hun ellende}\\

\haiku{Rypt niet uw padie?}{dikwerf ter voeding van wie}{niet geplant hebben}\\

\haiku{Wat zal er gezegd?}{worden in de dorpen waar}{wy gezag hadden}\\

\haiku{Hyzelf heeft dat geld, '.}{noodig en de kollekteur wil}{t hem voorschieten}\\

\haiku{- Ook de staten die,,.}{ik vandaag ontving zyn valsch}{ging Havelaar voort}\\

\haiku{t Was myn hart dat!}{ge daar hebt opgeslikt als}{een versnapering}\\

\haiku{Er is een juffrouw...}{flauw gevallen toen hy van}{dat zwarte kind sprak}\\

\haiku{Is niet reeds dit boek - -}{dat Stern me zoo zuur maakt een}{bewys hoe goed}\\

\haiku{Heimlich erz\"ahlen.}{die Rosen Sich duftende}{M\"archen ins Ohr}\\

\haiku{{\textquoteleft}deze kapel is...}{opgericht door den bisschop}{van Munster in 1423}\\

\haiku{Ze bewegen zich,.}{als een hobbelpaard minus}{nog het va et vient}\\

\haiku{Eigenliefde en.}{verveling dringen u iets}{dergelyks te doen}\\

\haiku{Ik zeg niet, daar heb,,:}{ik een vrouw gezien die z\'o\'o}{of z\'o\'o schoon was neen}\\

\haiku{waarom ik in uw!}{schatting verheven moest zyn}{boven verkoudheid}\\

\haiku{{\textquoteright} Havelaar scheen te,:}{verstaan wat Tine meende}{want hy antwoordde}\\

\haiku{- Welnu, dan weet je '.}{dat er peperkultuur in}{t Natalsche is}\\

\haiku{Myn verdienste was.}{te grooter omdat zy heel}{weinig antwoordde}\\

\haiku{wat is dit, dat die,?}{man macht heeft boven my en}{steenen houwt uit myn schoot}\\

\haiku{Als de ommelet,,... -}{overigens goed was zou dat}{geen bezwaar zyn maar}\\

\haiku{Maar de generaal.}{wilde my niet naar Natal}{laten vertrekken}\\

\haiku{De gevangene.}{immers is men onderhoud}{en voedsel schuldig}\\

\haiku{Hebt ge niet geweend,...}{by die moeder vruchteloos}{zoekend naar haar kind}\\

\haiku{- Ich weiss es nicht, ~ .}{Die Sterne Zahl hat Niemand}{noch gez\"ahlt}\\

\haiku{Men verkrygt daardoor '.}{den roep van bekwaamheid en}{yver voors lands dienst}\\

\haiku{Gisteren heb ik:}{Sjaalman gezien met zyn vrouw}{en hun jongetje}\\

\haiku{Hy is bleek als de,,.}{dood zyn oogen puilen uit en}{zyn wangen staan hol}\\

\haiku{Men ziet uit alles,.}{dat Stern jong is en weinig}{ondervinding heeft}\\

\haiku{En ook Sa{\"\i}djah's,.}{vader was zeer bedroefd doch}{zyn moeder het meest}\\

\haiku{Ook de ouders van.}{zyn vrouw woonden altyd in}{hetzelfde distrikt}\\

\haiku{Zie, Adinda, kerf.}{een streep in je rystblok by}{elke nieuwe maan}\\

\haiku{het vastgeknoopt aan '.}{t eindelyk terugzien}{onder den ketapan}\\

\haiku{{\textquoteleft}wie van ons zal het?}{lichaam verslinden dat daar}{daalt in het water}\\

\haiku{langs de knie neerviel...}{in heerlyke golving op}{den kleinen voet}\\

\haiku{En ook vroeg hy zich,?}{wie er toch wel wonen zou}{in zyns vaders huis}\\

\haiku{Ik weet het wel, ik,!}{weet het wel dat myn verhaal}{eentoonig is}\\

\haiku{Rangkas-Betoeng,,.}{25 Februari 1856 des}{avends te 11 ure}\\

\haiku{En ze hebben niets,,!}{te eten en ze slapen op}{den weg en eten zand}\\

\haiku{{\textquoteright} Ja, ik hoor het wel,,!}{ik hoor het wel dat roepen}{om wraak over myn hoofd}\\

\haiku{Het gesprek liep over.}{de verwachte beslissing}{van de Regeering}\\

\haiku{ze zyn in eere,!}{van hier gegaan en men schryft}{aan U zulk een brief}\\

\haiku{Maar ik zal tot hem.}{gaan en hem aantoonen hoe}{hier de zaken staan}\\

\haiku{Ik zou te Ngawi:}{hetzelfde moeten doen wat}{ik hier gedaan heb}\\

\haiku{Eindelyk liet hy.}{op-nieuw verzoeken om}{gehoord te worden}\\

\haiku{Fluks wierp hy 't in, '.}{de goot en reinigde het}{metn stalbezem}\\

\haiku{Nederland heeft niet.}{verkozen recht te doen in}{de Havelaarszaak}\\

\haiku{kort overzicht van de,.}{Woutergeschiedenis door}{Holda in den Ned}\\

\haiku{de herinnering.}{in het leven roept aan den}{javaschen oorlog}\\

\haiku{Dit is al iets in!}{onzen tyd van jammerlyk}{ordinarisme}\\

\haiku{Me dunkt toch dat ze,,.}{vooral met het oog op z'n}{dood zeer treffend zyn}\\

\haiku{Op strandplaatsen als.}{Marseille verbasteren}{de rassen zeer snel}\\

\haiku{) Het is zeer gewaagd '.}{dit op grond vann enkel}{woord aantenemen}\\

\haiku{Men behoorde den.}{moed te hebben zyner}{gewetenloosheid}\\

\haiku{Het blad zit, als 't ',.}{yzer vann houweel loodrecht}{op den houten steel}\\

\haiku{{\textquoteright} Maar nooit bleek er dat.}{er iets gedaan werd om dit}{doel te bereiken}\\

\haiku{wat hem te wachten '}{stond vann chef die toch even}{als hy gezworen}\\

\haiku{Multatuli, niet,.}{alleen in aangenomen}{naam maar inderdaad}\\

\haiku{Doch ik dacht er niet,.}{aan en heb geen verdienste}{van m'n discretie}\\

\haiku{staatkundige kleur -.}{nog altyd voor byzonder}{achtenswaardig door}\\

\subsection{Uit: Max Havelaar. Deel 1. Tekst}

\haiku{Men vlucht met het een.}{of ander voorwerp naar het}{einde der aarde}\\

\haiku{De familie die,.}{van dezen arbeid leven}{kan heeft weinig noodig}\\

\haiku{Het is zoo niet in, '.}{de wereld ent is goed}{dat het niet zoo is}\\

\haiku{Hy had dan ook wel,.}{iets van een Duitscher en van}{een reiziger ook}\\

\haiku{Ja, ja, hy was het,!}{die my uit de handen van}{den Griek had verlost}\\

\haiku{Ik schreide, en bad,.}{om genade want ik zat}{vreeselyk in angst}\\

\haiku{Hy zag zeer bleek, en, '.}{toen ik hem vroeg hoe laat het}{was wist hyt niet}\\

\haiku{Maar Louise schreide,.}{weer en de dames zeiden}{dat het heel mooi was}\\

\haiku{Over hydraulische.}{onderwerpen in verband}{met de rystkultuur}\\

\haiku{Maar hy zegt, dat de.}{invloed zich eerst openbaart in}{het tweede geslacht}\\

\haiku{Ook vond ik brieven,.}{waarvan velen in talen}{die ik niet verstond}\\

\haiku{Natuurlyk had hy,.}{my om geld gevraagd en van}{zyn pak gesproken}\\

\haiku{Bovendien, hy zag,...}{er armoedig uit en wist}{niet hoe laat het was}\\

\haiku{hoe zou 't wezen,,?}{dacht ik als ik hem de plaats}{van Bastiaans gaf}\\

\haiku{Ik ben zeker dat.}{hy met tweehonderd gulden}{tevreden zou zyn}\\

\haiku{Wat hy zeide, was -}{gewoonlyk lang overdacht een}{eigenaardigheid}\\

\haiku{Daar is me juist iets,...?}{voorgekomen dat zou de}{Regent ons verstaan}\\

\haiku{Maar toon my iets uit,.}{je weitasch dan denkt hy dat}{we d\'a\'arover spreken}\\

\haiku{D\'a\'ar moet geleden,,!}{zyn veel geleden daar is}{ondervonden}\\

\haiku{hy is beschaafd, niet? -... -?}{waar O ja En hy heeft een}{groote familie}\\

\haiku{Ik beveel me zeer,!}{aan voor uw medewerking}{m'nheer Verbrugge}\\

\haiku{, en daar het spreken, '.}{zoo moeielyk viel brak men}{t gesprek af}\\

\haiku{De toestand is er!}{sedert dien tyd niet beter}{op geworden}\\

\haiku{{\textquoteright} niet uitsprak, voor hy {\textquoteleft}{\textquoteright}.}{haar datniet zondigen had}{mogelyk gemaakt}\\

\haiku{een argument, zooals,.}{men weet waartegen niets valt}{intebrengen}\\

\haiku{zoo geen menschen rond,.}{die van problematische}{beroepen leven}\\

\haiku{Hy wie zyn woord spreekt,.}{opdat zy zich oprichten}{in hun ellende}\\

\haiku{Rypt niet uw padie?}{dikwerf ter voeding van wie}{niet geplant hebben}\\

\haiku{Wat zal er gezegd?}{worden in de dorpen waar}{wy gezag hadden}\\

\haiku{Hyzelf heeft dat geld, '.}{noodig en de kollekteur wil}{t hem voorschieten}\\

\haiku{- Ook de staten die,,.}{ik vandaag ontving zyn valsch}{ging Havelaar voort}\\

\haiku{t Was myn hart dat!}{ge daar hebt opgeslikt als}{een versnapering}\\

\haiku{Er is een juffrouw...}{flauw gevallen toen hy van}{dat zwarte kind sprak}\\

\haiku{Marie is daar in -,?}{dien rooien tuin waarom rood}{en niet geel of paars}\\

\haiku{{\textquoteleft}deze kapel is...}{opgericht door den bisschop}{van Munster in 1423}\\

\haiku{Ze bewegen zich,.}{als een hobbelpaard minus}{nog het va et vient}\\

\haiku{Eigenliefde en.}{verveling dringen u iets}{dergelyks te doen}\\

\haiku{ik zeg niet, daar heb,,:}{ik een vrouw gezien die z\'o\'o}{of z\'o\'o schoon was neen}\\

\haiku{waarom ik in uw!}{schatting verheven moest zyn}{boven verkoudheid}\\

\haiku{- Welnu, dan weet je '.}{dat er peperkultuur in}{t Natalsche is}\\

\haiku{Myn verdienste was.}{te grooter omdat zy heel}{weinig antwoordde}\\

\haiku{En de koning des,.}{lands toog voorby met ruiters}{voor zyn wagen}\\

\haiku{wat is dit, dat die,?}{man macht heeft boven my en}{steenen houwt uit myn schoot}\\

\haiku{Twaalfde hoofdstuk  -,,.}{Beste Max zei Tine ons}{dessert is zoo schraal}\\

\haiku{Ik zei dat elk mensch.}{in zyn medemensch een soort}{van konkurrent ziet}\\

\haiku{Maar de generaal.}{wilde my niet naar Natal}{laten vertrekken}\\

\haiku{te veel reeds gewerkt,.}{dan dat ik me verschuilen}{zou achter myn jeugd}\\

\haiku{Den gevangene.}{immers is men onderhoud}{en voedsel schuldig}\\

\haiku{Hebt ge niet geweend,...}{by die moeder vruchteloos}{zoekend naar haar kind}\\

\haiku{Men verkrygt daardoor '.}{den roep van bekwaamheid en}{yver voors lands dienst}\\

\haiku{Gisteren heb ik:}{Sjaalman gezien met zyn vrouw}{en hun jongetje}\\

\haiku{Hy is bleek als de,,.}{dood zyn oogen puilen uit en}{zyn wangen staan hol}\\

\haiku{Men ziet uit alles,.}{dat Stern jong is en weinig}{ondervinding heeft}\\

\haiku{Ook de ouders van.}{zyn vrouw woonden altyd in}{hetzelfde distrikt}\\

\haiku{Ik zal uitblyven....}{driemaal twaalf manen deze}{maan rekent niet mee}\\

\haiku{Zie, Adinda, kerf.}{een streep in je rystblok by}{elke nieuwe maan}\\

\haiku{het vastgeknoopt aan '.}{t eindelyk terugzien}{onder den ketapan}\\

\haiku{{\textquotedblleft}wie van ons zal het?}{lichaam verslinden dat daar}{daalt in het water}\\

\haiku{langs de knie neerviel...}{in heerlyke golving op}{den kleinen voet}\\

\haiku{En ook vroeg hy zich,?}{wie er toch wel wonen zou}{in zyns vaders huis}\\

\haiku{Ik weet het wel, ik,!}{weet het wel dat myn verhaal}{eentonig is}\\

\haiku{- Ach, zeide zy, er! -,.}{is zooveel slecht volk Zeker}{dat is er overal}\\

\haiku{Ik heb de meeste,}{hoogachting voor u maar ik}{ken den geest dien men}\\

\haiku{vroeg de moeder.175~		 -, -? -!}{Ik zei kleine Max. En wat}{beduidt dat Bedtyd}\\

\haiku{En ze hebben niets,,!}{te eten en ze slapen op}{den weg en eten zand}\\

\haiku{{\textquoteright} Ja, ik hoor het wel,,!}{ik hoor het wel dat roepen}{om wraak over myn hoofd}\\

\haiku{Het gesprek liep over.}{de verwachte beslissing}{van de Regeering}\\

\haiku{Duclari, een zeer,:}{beschaafd mensch berstte in een}{wilden vloek uit}\\

\haiku{ze zyn in eere,!}{van hier gegaan en men schryft}{aan U zulk een brief}\\

\haiku{Maar ik zal tot hem.}{gaan en hem aantoonen hoe}{hier de zaken staan}\\

\haiku{Ik zou te Ngawi:}{hetzelfde moeten doen wat}{ik hier gedaan heb}\\

\haiku{Eindelyk liet hy.}{op-nieuw verzoeken om}{gehoord te worden}\\

\haiku{Fluks wierp hy 't in, '.}{de goot en reinigde het}{metn stalbezem}\\

\haiku{de herinnering '.}{int leven roept aan den}{javaschen oorlog}\\

\haiku{Dit is al iets in!}{onzen tyd van jammerlyk}{ordinarisme}\\

\haiku{Me dunkt toch dat ze,,.}{vooral met het oog op z'n}{dood zeer treffend zyn}\\

\haiku{Op strandplaatsen als.}{Marseille verbasteren}{de rassen zeer snel}\\

\haiku{) Het is zeer gewaagd '.}{dit op grond vann enkel}{woord aantenemen}\\

\haiku{Men behoorde den.}{moed te hebben zyner}{gewetenloosheid}\\

\haiku{Het blad zit, als 't ',.}{yzer vann houweel loodrecht}{op den houten steel}\\

\haiku{{\textquoteright} Maar nooit bleek er dat.}{er iets gedaan werd om dit}{doel te bereiken}\\

\haiku{wat hem te wachten '}{stond vann chef die toch even}{als hy gezworen}\\

\haiku{staatkundige kleur -.}{nog altyd voor byzonder}{achtenswaardig door}\\
