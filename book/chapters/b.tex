\chapter[52 auteurs, 9460 haiku's]{tweeënvijftig auteurs, negenduizendvierhonderdzestig haiku's}

\section{Franz de Backer}

\subsection{Uit: Longinus}

\haiku{O, daar was een God,,!}{die m\`ensch was en stierf in de}{wrangste eenzaamheid}\\

\haiku{Wat onmiddellijk.}{daarna gebeurde is me}{niet zeer duidelijk}\\

\haiku{Ik trok mijn zwaard, maar,,.}{hij bewoog niet keek me even}{aan keek naar het zwaard}\\

\haiku{Wij speculeerden,.}{over de kansen dat wij geen}{gevecht zouden zien}\\

\haiku{Die heel bleeke, stille,,.}{knaap bewusteloos maar zijn}{lippen bewogen}\\

\haiku{- {\textquoteleft}'t Is wel te zien,,.}{sergeant dat gij nog niet}{gekwetst geweest zijt}\\

\haiku{Plots, v\'o\'or ons, bij den,.}{vijand een dof gerucht van}{werkende schoppen}\\

\haiku{Ik kon lang proeven,.}{aan dien troost dien ik diep in}{mij verdoken hield}\\

\haiku{Van huis geen nieuws, mijn, -,.}{streek bezet niets dan oorlog}{oorlog v\'o\'or mijn neus}\\

\haiku{zij hadden meer kracht,,,:}{dan ik meer moed of soms meer}{lafheid maar altijd}\\

\haiku{* * * ~ Die stemming werd.}{een wijl vergeten toen ik}{zelf met verlof ging}\\

\haiku{Dites-moi o\`u,,,...}{n'en quel pays Est Flora la}{belle Romaine}\\

\haiku{- mijn hals was \'e\'en groote.}{wonde waar de kogel was}{buitengebroken}\\

\haiku{de loopgraven die.}{ik verlaten had werden}{niet aangevallen}\\

\haiku{Twee vingers van mijn,.}{linker hand ze lagen op}{mijn bloedigen buik}\\

\haiku{Wat ieder van u,.}{draagt is absoluut noodig om}{te kunnen schieten}\\

\haiku{De jongen kruipt weg,,.}{terug in de vlakte van}{dood zonder \'e\'en woord}\\

\haiku{, en het koppel dat,,}{hem belet had vertelt ook}{en met stelligheid}\\

\section{Lode Baekelmans}

\subsection{Uit: Het geheim van 'De drie snoeken'}

\haiku{We zullen zeker,.... -!}{getwee\"en moeten zitten}{dubben terwijl Zwijg}\\

\haiku{Een paar dagen bleef.}{de winkel gesloten en}{Putzeys onzichtbaar}\\

\haiku{Silberfeld was niet,.}{te vinden maar Bollekens}{liep hij op het lijf}\\

\haiku{- Maar toch slecht.... - Vergeet!}{niet dat zij de zuster van}{een Kolonel is}\\

\haiku{Intusschen had zij.}{zich gekleed en sloot zij haar}{pruik over den schedel}\\

\haiku{Een vrouw van de straat,,.}{mocht zij wel getuigen die}{geld noch eer bezat}\\

\haiku{Haar ziel verging van.}{onrust en haar gedachten}{waren in de war}\\

\haiku{zij had geen eetlust.}{en geen verlangen om op}{visite te gaan}\\

\haiku{- Maar zeg het dan toch,,.}{drong Mademoiselle aan}{ik verga van angst}\\

\haiku{Zoo zijn de menschen,,,.}{dacht Mademoiselle uit}{het oog uit het hart}\\

\haiku{In het troebel beeld.}{herkende zij de vrouw van}{het portret niet meer}\\

\haiku{Rubens was moe na.}{een langen namiddag van}{carnaval plezier}\\

\haiku{- Kom maar mee, verzocht,....}{Rubens we zijn vlak in de}{buurt van mijn atelier}\\

\haiku{'t Waren alle.}{jonge kerels met lang haar}{en vreemde hoeden}\\

\haiku{Wat mag hij van mij,.}{wel elders vertellen vroeg}{zich de schilder af}\\

\haiku{Hoe hoog voelden zij!}{zich verheven boven den}{gewonen bourgeois}\\

\haiku{De burgers keken.}{toegeeflijk de joelige}{artistenjeugd na}\\

\haiku{een meisje dat zong:}{op het Burchtplein onder de}{kale boomen}\\

\haiku{Het gebeurde wel.}{dat Mieke Serafijn met}{Rubens vergeleek}\\

\haiku{De stilte hing als.}{een eindelooze diepte in}{den wijden hemel}\\

\haiku{Nooit sprak Rubens over. '}{de ontrouwe vrouw noch over}{zijn drinkgelagen}\\

\haiku{Met den Stadhuisklerk.}{ging hij de gedachte aan}{Parijs wegspoelen}\\

\haiku{Het orkaan snokte,.}{en joelde dreef de regen}{tegen de vensters}\\

\haiku{De zeep schuimde over.}{zijn gelaat en hij kneep de}{oogen dicht van plezier}\\

\haiku{- Hundenwetter, zei,.}{ze zonder de handen uit}{den schoot te lichten}\\

\haiku{Zij ziet er nog goed,,....}{uit oordeelde de man zij}{is amper veertig}\\

\haiku{Op een dooden tak zat,.}{een verloren roodborstje}{dik in de veeren}\\

\haiku{Lui koesterde zich.}{de oude herdershond in}{de zon voor het huis}\\

\haiku{Wekelijks bracht de.}{bode met zijn vrachtwagen}{de boodschappen mee}\\

\haiku{Ik wist op voorhand.}{dat het tevergeefs was en}{toch hoopte ik nog}\\

\haiku{Ik zei immers dat....}{het een belachelijke}{geschiedenis was}\\

\haiku{hoe zijn papieren.}{reddeloos door het water}{bedorven werden}\\

\haiku{Koko had toch veel.}{smaak en een zekeren chic}{om zich te kleeden}\\

\subsection{Uit: De idealisten}

\haiku{De kapitein stond.}{juist boven de kajuit zijn}{sigaar te rooken}\\

\haiku{In den dag zag ik,.}{niets maar den volgenden avond}{herbegon het spel}\\

\haiku{Soms kwam hij dan toch.}{eens onverwacht kijken tot}{aller verrassing}\\

\haiku{Misschien... misschien... - Ik,!}{zoek naar iets anders iets om}{meer te verdienen}\\

\haiku{het was plezant met... -,.}{Hortense Dat geloof ik}{zei Max pinkoogend}\\

\haiku{Herder, ik haat geen,!}{menschen maar ik kan er ook}{geen liefhebben}\\

\haiku{Ik ben een oud man!}{en amuseer mij maar alleen}{tusschen mijn boeken}\\

\haiku{Tusschen uw vrienden!}{staan er veel die elkander}{niet kunnen luchten}\\

\haiku{Hij kloeg altijd over...}{eenzaamheid en verdorde}{tusschen de boeken}\\

\haiku{De man had een bril.}{met donkere glazen die}{zijn oogen verborgen}\\

\haiku{Toen zij het waagden}{het hoofd te wenden zagen}{zij een vrouw tusschen}\\

\haiku{Doch ditmaal hoorden. '.}{zij geen glasgerinkelt}{Gaf een opluchting}\\

\haiku{Maar als de vrouw weer.}{in de kamer kwam zat hij}{rustig te rooken}\\

\haiku{ik meende in den... -,.}{kelder nog wat te hooren}{Bepaald niets Mevrouw}\\

\haiku{Zij namen hun glas.}{en stonden aarzelend in}{de roode schaduw}\\

\haiku{de zedelijkste,;}{want zij brengt niet mede het}{slachten van dieren}\\

\haiku{altijd werken... - Ja,,... -}{onderwierp zich de vrouw de}{jongens kosten geld}\\

\haiku{Zij wuifde toen de,.}{tram afreed dan ging zij in}{het rijtuig zitten}\\

\haiku{Mieke, die den stoet,.}{had zien voorbijtrekken was}{er door aangedaan}\\

\haiku{Te bed lagen zij.}{wakker onder de spanning}{hunner zenuwen}\\

\haiku{Dat neem ik niet mee,,.}{peinsde zij nu beginnen}{wij een nieuw leven}\\

\haiku{- Ik weet het wel, gaf,.}{de Zaalwachter toe het was}{maar uit gewoonte}\\

\haiku{- Gij zijt nog een van,.}{de ouden fluisterde de}{Portier met warmte}\\

\haiku{Zij snoepte nog een,.}{stukje koek trok het licht uit}{en kroop in haar bed}\\

\haiku{Amelia fluisterde,.}{iets wat hij niet verstond want}{hij was een Duitscher}\\

\haiku{Zij stond op, deed het.}{vuur branden en zette de}{koffieketel op}\\

\haiku{Rond twaalven viel de.}{muziekdoos stil en kraamden}{de bezoekers op}\\

\haiku{ik was de eenige...}{stoker die geen jenever}{dronk en boeken kocht}\\

\haiku{Een hoeve, land en,!...}{weiden een schuur en een stal}{en wel zes peerden}\\

\haiku{En nu moet ik weg,... -!}{want ik heb nog veel te doen}{Neem nog een sigaar}\\

\haiku{Zij dronken opnieuw,.}{op de vriendschap en op de}{schoone ingeving}\\

\haiku{Boven de deur in,.}{kleur van geronnen bloed stak}{de vlaggestok uit}\\

\haiku{ik moet rondtrekken... -?}{en drinken als ik cens heb}{Gaan de zaken slecht}\\

\haiku{Het venster sloot zij,.}{draaide de lamp neer die met}{een lichtflap uitging}\\

\haiku{Zijn haren waren.}{kort geknipt en zijn klak zat}{diep op zijn voorhoofd}\\

\haiku{De rechter zei wat,.}{en hij antwoordde wat maar}{ik luisterde niet}\\

\haiku{Ik verliet de zaal.}{en zag niet verder om naar}{mijn gevangene}\\

\haiku{Juffrouw Augusta stond.}{alleen in haar winkel en}{staarde voor zich uit}\\

\haiku{Juffrouw Augusta nam,.}{een register bladerde}{er in en knikte}\\

\haiku{Mijnheer Deckers heeft...}{hem zekeren dag in zijn}{kantoor geroepen}\\

\haiku{Terwijl zijn vrouw haar,.}{kerkboek in een lade sloot}{overwoog zij luidop}\\

\haiku{- Ze krijgen een buis,, -,?}{fluisterde hij wat denkt gij}{er van Champetter}\\

\haiku{t is geuzenweer -,.}{Ja ge kunt moeilijk spreken}{in uw positie}\\

\haiku{Hij rook het gebraad.}{en de roode kooltjes en}{was zeer vergenoegd}\\

\haiku{Dan ging plots de vlag {\textquoteleft}{\textquoteright} {\textquoteleft}{\textquoteright}.}{omhoog inDe Engel en}{inDe Witte Leeuw}\\

\haiku{Wij zijn allemaal...}{poesjenellen die dwaze}{grimassen maken}\\

\haiku{- Ja, die zijn bij haar... -?}{moeder En van haar vent heeft}{ze niets meer gehoord}\\

\haiku{k laat haar... - 'k Zou,,!}{dat niet doen Stafke ge zult}{haar zoo'n verdriet doen}\\

\haiku{Zoo won Marie haar.}{Stafke weer en ging met hem}{naar het spektakel}\\

\haiku{Terwijl hij de cens,.}{opstreek fluisterde iemand}{hem wat in het oor}\\

\haiku{- Nu moeten wij toch... -,? -,.}{eventjes in de kerk gaan Ach}{waarom Kom Wagner}\\

\haiku{Aan den staart van den.}{stoet liepen van Dijck en}{Wagner naar buiten}\\

\haiku{- Neen, Edele Heer, zei.}{hoffelijk Van Dijck en}{streek door zijn haren}\\

\haiku{- En vooral vergeet.}{niet mijn complimenten aan}{Johanna te doen}\\

\haiku{Wagner was de baan,...}{op maar zij verwachtte hem}{wel wat vroeger thuis}\\

\haiku{Maar de verliezer.}{had het niet goed begrepen}{en wou herkansen}\\

\haiku{- En groenten uit den,;}{hof en ongedoopte melk}{droomde Katinka}\\

\haiku{- We gaan samen, zei,.}{de Wisselagent we kunnen}{er nog rijk worden}\\

\haiku{Een grijze poes kwam}{uit een hoek te voorschijn en}{vlijde zich neder}\\

\haiku{Maar noch de vischjes, '.}{noch de alcohol hielden}{hem int leven}\\

\subsection{Uit: Robinson}

\haiku{Daar kefte een hond.}{en speelde een schipper op}{zijn harmonika}\\

\haiku{De dag, een dag uit,.}{het Aards Paradijs gevuld}{met eten en drinken}\\

\haiku{En wanneer Fientje,...}{haar portret zou missen had}{hij het verkorven}\\

\haiku{In het vijfde jaar.}{na zijn afscheid bracht het}{toeval hem er weer}\\

\haiku{- Een goeie sloeber... - Wel,! -.}{vooruit dan met de muziek}{Adieu Robinson}\\

\haiku{En hij kon zich niet.}{onthouden een Gandhi zeer}{te bewonderen}\\

\haiku{Had Robinson een,,}{enkele keer de pest in}{wat niet vaak voorviel}\\

\haiku{Want Boy was als de.}{mensen die bij Robinson}{op bezoek kwamen}\\

\haiku{s Winters droeg hij.}{zijn muts op zijn ronde door}{het Schipperskwartier}\\

\haiku{Met zijn neef kon hij.}{beter overweg dan met de}{eigen kinderen}\\

\haiku{Eens, toen het hem te,.}{zwaar geworden was ging hij}{naar een dominee}\\

\haiku{Zij waren alleen.}{in de wereld en wisten}{niet wat aanvangen}\\

\haiku{Maar het angstgevoel.}{voor het hiernamaals kon hij}{niet kwijt geraken}\\

\haiku{Aan de binnenkant.}{van elk deksel werd een klein}{spiegeltje gehecht}\\

\haiku{Robinson sloot de.}{deur en ging wandelen in}{de bewogen stad}\\

\haiku{- Mijn affaire is,,?}{om zeep kloeg hij wat moet ik}{nu gaan aanvangen}\\

\haiku{s Anderendaags.}{ging Robinson de bouw van}{de schipbrug volgen}\\

\haiku{- Of dat ge naar een,.}{familiebegrafenis}{moest gaan vond Vrijdag}\\

\haiku{En het einde was,,.}{toch gekomen en zoals}{altijd onverwacht}\\

\haiku{In het seizoen hield,.}{Vrijdag ook op Zondag open}{althans tot \'e\'en uur}\\

\haiku{- Ge moogt wel dankbaar,.}{zijn meende de Notaris}{te moeten zeggen}\\

\haiku{s Morgens was haar:}{kamer reeds vrij en stonden}{de ramen wijd open}\\

\haiku{En verder vooruit.}{met de muziek tot aan het}{einde van de weg}\\

\haiku{De eenden zochten.}{het water op en pluisden}{onder hun veren}\\

\haiku{Een slag van het net,.}{een gelukkige slag en}{zij zat gevangen}\\

\haiku{De straat oversteken, '!}{of een lange bootreist}{was al hetzelfde}\\

\subsection{Uit: Uit grauwe nevels}

\haiku{Verder sjouwde men;}{reusachtige balen wol}{van plompe wagens}\\

\haiku{Charel had er eene:}{met de bloote hand gevat en}{neep ze den adem af}\\

\haiku{Sander stond op den,;}{drempel zag vragend naar de}{gesloten lucht op}\\

\haiku{Plots in 't licht  :}{van eenen gaslantaarn hield de}{weifelaar hem staan}\\

\haiku{de versiering moest!}{dan kant en klaar zijn en dat}{was geen kleinigheid}\\

\haiku{Wij hebben maar een, '!}{jong leven En als wij dood}{zijn ist gedaan}\\

\haiku{de deur werd achter.}{hun rug toegesloten en}{het licht neergedraaid}\\

\haiku{Het water bruiste;}{onder de wielraderen}{van den overzetboot}\\

\haiku{Lang bespiedden ze;}{den nachtwaker die op-}{en neerwandelde}\\

\haiku{Ze keken naar dien.}{vent die daar hoog stond als den}{meester van het hooi}\\

\haiku{ik wedden, dat er... '}{geen vetter beesten op de}{markt te vinden zijn}\\

\haiku{klonk het gillend en,.}{tierend uit den hoop die nu}{het hazenpad koos}\\

\section{Kamiel van Baelen}

\subsection{Uit: De oude symphonie van ons hart}

\haiku{maar de meesten zijn,.}{zieke menschen we moeten}{daar niet mee lachen}\\

\haiku{Krankzinnig is ie,.}{dan ook al staat dat niet in}{het contract vermeld}\\

\haiku{Kijkt, denkt Lou, die gaan,!}{ze nu ergens op een huis}{leggen de sullen}\\

\haiku{Maar de mis\`ere.}{loopt op drie passen achter}{het geldgebrek aan}\\

\haiku{*** ~ - Wacht even, zegt Lou,.}{ik heb nog een en ander}{te verrichten hier}\\

\haiku{Ook Magda vindt hem,,.}{nu een man d\'e man zij ziet}{zoowaar tegen hem op}\\

\haiku{*** ~ In de Golf van,;}{Gascogne overvalt hen de}{storm niet zoo zacht ook}\\

\haiku{, maar hij voelt zich zoo, '.}{ziek als een hond en werpt zich}{languit opt bed}\\

\haiku{Hij heeft echter mooi,.}{rukken en duwen die lijkt}{wel vastgenageld}\\

\haiku{Maar z'n eerste klop.}{op die gevreesde deur is}{al veel te schuchter}\\

\haiku{De acteur Vreebos:}{heeft tranen in de oogen en}{stoot z'n glas bier om}\\

\haiku{Werd de F\'e bijna,,!}{kwaad om hij had er maar \'e\'en}{geteekend labberdaan}\\

\haiku{Hij heeft alleen een,.}{huis van goud dat is te mooi}{voor een dronkelap}\\

\haiku{Maar eerst wil hij nog,.}{ergens heen de rest van z'n}{drie mille moet op}\\

\haiku{Dacht je werkelijk,?}{dat hij de bons gekregen}{had van z'n meisje}\\

\haiku{Het is stilletjes,.}{gaan motregenen maar hij}{zet z'n kraag niet op}\\

\haiku{Zie je wel, en daar,.}{ligt een boek op den grond ook}{een oude reisgids}\\

\haiku{Maar niet lang meer, hij,.}{stelt zich voor naar het klooster}{te gaan voor broeder}\\

\haiku{Een nieuw hart moet Lou,.}{hebben en een eerlijke}{kans in het leven}\\

\haiku{En onwaarschijnlijk,.}{haast zoo onwaarschijnlijk als}{de waarheid vaak is}\\

\haiku{Hij maakt het open en,.}{v\'o\'or ze rechtstaat hangt hij haar}{een parelsnoer om}\\

\haiku{blad onder F.V. V.}{Uit Het Nieuwe Geluid van}{12 Februari}\\

\haiku{Gisteren werd in.}{de Stationstraat een zwart}{hondje overreden}\\

\haiku{Had mij dan liever,,.}{doodgereden siste hij}{bleek van woede}\\

\haiku{Naar we hoorden, mag.}{hij zich in een volledig}{herstel verheugen}\\

\haiku{Tante Jet heeft om.}{gezondheidsredenen}{ontslag ingediend}\\

\haiku{Johan begrijpt er niets,.}{van maar daar wen je wel aan}{bij rijke menschen}\\

\haiku{\'e\'en naar de ruime,.}{toekomst waar ons hunkerend}{hart misschien rust vindt}\\

\haiku{Lou's hart klopt mild van,?}{ontroering maar wat moet hij}{nu zeggen of doen}\\

\haiku{Zij neemt ze van hem.}{over en laat ze keurend door}{de vingers glijden}\\

\haiku{Hij krijgt ze in geen,. -?}{geval meer terug als hij}{dat maar niet denkt Niet}\\

\haiku{- Laat me los, - zegt ze,,,.}{opeens bevelend en Lou}{ja hij doet het zoowaar}\\

\haiku{- Of had hij misschien?}{een manchester-pak aan}{en een slappen hoed}\\

\haiku{- Ach ja, zegt Lou met,.}{een droomenden glimlach z'n}{krullen is ie kwijt}\\

\haiku{- En dat is Gary,?}{Cooper zou je daar niet zoo}{verliefd op worden}\\

\haiku{De sterren aan  ,.}{den hemel loopen mee die}{hebben ook geen haast}\\

\haiku{Als het weer niet te,,.}{kwaad is doet hij wel een toer}{een doellooze zwerftocht}\\

\haiku{hij zit weer bij den,.}{Kluizenaar we zullen maar}{niet op hem wachten}\\

\haiku{*** ~ Als de slaap zich,.}{eindelijk over hem ontfermt}{krijgt hij nog geen rust}\\

\haiku{En pas op voor je,,.}{koningin jongetje hier}{kom ik op mijn paard}\\

\haiku{De ander laat zich,.}{niet bidden dat doen alleen}{kleine artisten}\\

\haiku{Hij blijft getroffen.}{staan en drukt Kloos wat vaster}{onder z'n arm}\\

\haiku{Als die af is, moet.}{Frans daar zoet blijven zitten}{tot de rest klaar komt}\\

\haiku{Niets dan een groote haard,.}{en een hangklok en zoo een}{gezellig tehuis}\\

\haiku{Maar toen hij wakker,.}{werd kwam z'n lot hem des te}{ondraaglijker voor}\\

\haiku{Maar wanneer hij haar,}{net genaderd was keerde}{zij hem den rug toe}\\

\haiku{Ik bid u. - Zeker,,;}{zal ik je helpen kind der}{menschen sprak de toovenaar}\\

\haiku{en toch is dit mijn,,.}{Land niet het Land van Geluk}{waar ik van droomde}\\

\haiku{Zij zag hem uit het.}{doksaaldeurtje komen en}{bleef verwonderd staan}\\

\haiku{Maar Arthur grijpt haar bij.}{de haren vast en trekt haar}{hoofd naar achteren}\\

\haiku{Die zit alleen, wordt:}{stilaan grijs en speelt nog wat}{herinneringen}\\

\haiku{Het is ontzettend,,.}{zoo iets praat er maar niet met}{de kinderen over}\\

\haiku{Arthur scheurt nu ook de.}{foto in vier stukken en}{werpt ze in het vuur}\\

\haiku{De nieuwe is maar,,.}{een bleekneus zoo recht van de}{boeken je weet wel}\\

\haiku{Maar we zullen goed,.}{opletten op wie het wicht}{van Leen gaat lijken}\\

\haiku{Vroeg of laat leer ik,.}{ze wel kennen ik woon hier}{pas drie of vier maand}\\

\haiku{{\textquoteright} de heele stemming.}{weg en ging met veel drukte}{en omhaal zitten}\\

\haiku{Achteraf heb ik,.}{me toch afgevraagd wat er}{wel mocht haperen}\\

\haiku{Misschien had ik haar,.}{eerder moeten opslaan dat}{kan het eenige zijn}\\

\haiku{Ik zal het op het,.}{einde van de week doen als}{blijde verrassing}\\

\haiku{Hij heeft zelfs \'e\'en van.}{z'n gevleugelde woorden}{aan me gespendeerd}\\

\haiku{En in haar blozend....}{gelaat stonden haar oogen zoo}{blauw als heerlijk blauw}\\

\haiku{Ging de bel daar niet? -,.}{Nee snakte ze overtuigd en}{huilde rustig door}\\

\haiku{ik woon hier nu haast,.}{vier of vijf maanden het wordt}{tijd dat ik ze ken}\\

\haiku{de detective,.}{en zeg tegen Emmy dat}{de prijs te laag is}\\

\haiku{Blijf dan wat, zei ik.}{haastig en wees uitnoodigend}{naar een clubzetel}\\

\haiku{Maar voor mij begint.}{nu de narigheid weer met}{een nieuw dienstmeisje}\\

\haiku{hij draagt denzelfden,.}{voornaam als Rousseau de}{would-be-wijsgeer}\\

\haiku{Zelfs loopt ze af en,.}{toe om uitleg bij me aan}{maar niet half genoeg}\\

\haiku{dat is maar goed ook,.}{want ik houd m'n hart niet eens}{onder controle}\\

\haiku{Laat het zinken in,,!}{den oceaan met een blok lood}{voer het naar de hel}\\

\haiku{Zij knikte ernstig,.}{er kwam een ongewone}{zachtheid in haar blik}\\

\haiku{Zij beloofde ten.}{slotte en scheen opgelucht}{toen ik haar uitliet}\\

\haiku{De dokters namen:}{mekaar eens goed op en men}{ging aldus te werk}\\

\haiku{Lou lacht, nerveus en,.}{superieur alsof hij}{dat wel verwachtte}\\

\haiku{Als de man gek is,;}{is het geen waanvoorstelling}{en is hij normaal}\\

\haiku{Je derde dochter.}{zou dat veel behoorlijker}{opgelost hebben}\\

\haiku{Maar de eerste de,.}{beste komt hier niet binnen}{probeer het liever}\\

\haiku{Misschien is de zon,;}{daarboven echt misschien de}{zon hierbeneden}\\

\haiku{Maar toen begon het.}{pas. De vijver kende geen}{rustigen dag meer}\\

\haiku{Intusschen steeg de.}{verwarring op den vijver}{tot haar hoogtepunt}\\

\haiku{Haar volg ik met open,;}{kelkje van Oost naar West om}{warmte en leven}\\

\haiku{Doorbuigen, knie\"en,......}{saamhouden vooroverleunen}{en het kwam nooit uit}\\

\haiku{Want het lichaam moet,,?}{flink gevoed worden niet waar}{dat is van belang}\\

\haiku{weelde van nog wat.}{late zon in een paleis}{van roode wolken}\\

\haiku{Toen ik nog kind was,.}{had ik dikwijls last met de}{lessen van godsdienst}\\

\haiku{Maar even later moet.}{Lou iets vragen en wenkt den}{professor vlakbij}\\

\haiku{*** ~ Een paar dagen.}{later is de professor}{al zeer tevreden}\\

\haiku{- Niet veel zaaks, moet de.}{professor zich hulpeloos}{verontschuldigen}\\

\haiku{Lou heeft nu minder,.}{last van z'n buren dan de}{concierge van hem}\\

\haiku{Lach me niet uit, Lou,...}{Anders het verstand hebben}{ze van hun vader}\\

\haiku{Wat een geluk voor!}{beiden dat hij niet aan haar}{blijven plakken is}\\

\haiku{- Hij weet wel dat ze,.}{nu Lena heet maar gebruikt}{dien naam met opzet}\\

\haiku{Daar een variant.}{op moet ik van avond zeker}{te pas brengen}\\

\haiku{Modern en classiek,.}{noemen ze mijn werk maar dat}{weet u natuurlijk}\\

\haiku{Lou is blij voor den,.}{kerel maar niet voldaan over}{z'n onderzoek}\\

\haiku{En ze voelt niets voor,,.}{hem geen sikkepit ze heeft}{het zelf gezegd}\\

\haiku{- Ze hebben je dus?}{al \'e\'en keer voor chantage}{de nor ingedraaid}\\

\haiku{- En heeft ze niet zoo'n,,......?}{knaap van veertien vijftien jaar}{wat heu zonderling}\\

\haiku{Hoe gemakkelijk?...}{nemen we ons persoonlijk}{geval tot maatstaf}\\

\haiku{Maar haar beschrijven,.}{kan hij niet het  is ook}{zoo lang geleden}\\

\haiku{Vroeger speelde hij,...}{wel eens aan het kerkmuurtje}{met Felicitas}\\

\haiku{Maar voor zichzelf maakt.}{hij uit dat het slot van het}{verhaal anders is}\\

\haiku{En hij nam haar tot.}{zich en ze leefden samen}{lang en gelukkig}\\

\haiku{Heeft Lou Anders een,?}{zoon in deze wereld een}{kleinen Aeneas}\\

\haiku{Dat schijnt hem  als '.}{een kwade microbe in}{t bloed te zitten}\\

\haiku{Lou kent heel weinig.}{menschen die daar een afdoend}{antwoord op weten}\\

\section{Clovis Baert}

\subsection{Uit: Het tweede leven van Wieske Veyt}

\haiku{De mane was een.}{reusachtige topaas met}{diep-gelen grond}\\

\haiku{Mijn leventje sliert.}{weg lijk een stroelke water}{in gloeiend-heet zand}\\

\haiku{Iederen keer als, '}{hij asem haalde wast hem}{lijk of hij sterken}\\

\haiku{bult en blies lijk een.}{verstopte blaasbalg in een}{half uitgegaan vuur}\\

\haiku{'t Was al. Van 't.}{leven in de wereld weet}{ik weinig of niets}\\

\haiku{het zijn liedeke!}{met de grootste overtuiging}{tot aan het einde}\\

\haiku{en hij loech ook en.}{liet zich nu stillekes op}{zijnen rug drijven}\\

\haiku{Een meiske  dat,!}{in den vijver zat dat in}{den vijver leefde}\\

\haiku{Wieske stond daar nog.}{altijd aan zijn kijkgat en}{verroerde geen spier}\\

\haiku{er is nooit een klank,.}{te weinig maar er is ook}{nooit een klank te veel}\\

\haiku{En ze waren nog ';}{maar even opt gras of ze}{lieten zich vallen}\\

\haiku{Ze gebaren, dat,.}{ze er niet van moeten van}{weten dacht Wieske}\\

\haiku{En onvermoeibaar.}{zat het bokske terbinst op}{de fluit te blazen}\\

\haiku{t was hier zake.}{van zich een beetje op den}{stok te verlaten}\\

\haiku{Zij verschoot danig,.}{want dat handeke was lijk}{een puur ijsklompke}\\

\haiku{zuiver goud viel op.}{Wieske en verguldde zijn}{wezen en zijn borst}\\

\haiku{of ze nepen eens.}{in de billen en dan ging}{het er wreed op los}\\

\haiku{'k geloof dat ik,.}{mijnen weg verloren heb}{antwoordde Wieske}\\

\haiku{Ze omhullen ons,.}{met een sluier van blanke}{zuivere zijde}\\

\haiku{De bliksemsnelle.}{vaart pakte zijnen asem en}{sneed hem bijkans af}\\

\haiku{Wieske struikelde.}{altemets en bezeerde}{zijn moede voeten}\\

\haiku{Hij ging straat-op en:}{straat-af en zag hier overal}{\'een en hetzelfde}\\

\haiku{Onder zijn weg zag.}{hij het beeld van de eene helft}{der menschenwereld}\\

\haiku{Hij hield zijn asem in}{van schrik en opeens keerde}{hij zich om en liep}\\

\haiku{zoo rap mogelijk,.}{een helling op tot op het}{hoogste van de stad}\\

\haiku{En ons Wieske werd.}{nu al ineens geplaagd door}{een priemende vraag}\\

\haiku{Zou Magdalena?}{hier nu nog rondwaren tot}{hij haar bereikt had}\\

\haiku{Was het nu zake?}{van hier te blijven of van}{weer te gaan zwerven}\\

\haiku{Het was lijk of de.}{brand voor Magdalena een}{weldaad geweest was}\\

\haiku{De stem was lijk de,;}{klank van een klok gegoten}{uit de zuiverste spijs}\\

\haiku{Het vuile, vieze.}{dier spon daar een dik net en}{zat dan te wachten}\\

\section{John Bake}

\subsection{Uit: Reisbrieven}

\haiku{Spoedig ging het weer,.}{regenen en alles was}{daar slik en plassen}\\

\haiku{Die tour is een der.}{merkwaardigste geweest die}{wij gedaan hebben}\\

\haiku{Eerst langs de rechter,,.}{Rijnoever tot Eglisau}{zeer berg op en neer}\\

\haiku{Wij vonden er ook.}{andere kennissen en}{de familie Ochsner}\\

\haiku{De aankomst is op,.}{het strand voor de herberg waar}{wij eerst dineerden}\\

\haiku{Voor de overtocht van ():}{het meer4 uur lang moesten wij}{twee schuiten hebben}\\

\haiku{Links valt een smalle ().}{stroomlintvormig van zeker}{drieduizend voet hoog}\\

\haiku{Maar het schijnt een hol,.}{te zijn hoewel het vrij groot}{en protestants is}\\

\haiku{Daarom was het mij.}{ook van belang dit nu zelf}{te onderzoeken}\\

\haiku{In 't midden is.}{een wacht te paard om voor de}{orde te zorgen}\\

\haiku{Reichmann beloofde *.}{ons te zullen waarschuwen}{als zij speelde}\\

\haiku{De komedie, in,!}{de parterre kostte ons}{8 stuiver per plaats}\\

\haiku{Veel hartelijke.}{complimenten aan allen}{die mij liefhebben}\\

\haiku{Voghera, waar wij,.}{te twee uur aankwamen is}{een slordige plaats}\\

\haiku{De bomen ieder.}{ruim zo groot als een fikse}{pereboom bij ons}\\

\haiku{Het was reeds donker,.}{toen wij terugkerende}{de stad naderden}\\

\haiku{De dag was reeds vroeg,.}{heet zelfs naar getuigenis}{van de mensen hier}\\

\haiku{Mijn laatste, lieve,,,.}{beste Anna zond ik u}{meen ik uit Turijn}\\

\haiku{s Avonds gingen wij.}{ijs gebruiken en toen met}{een kop thee naar bed}\\

\haiku{Zo veel reizen en.}{vliegen laat niets tot stille}{afzondering over}\\

\haiku{Geel en ik hebben.}{die tocht van 9 \`a 10 uur}{meest te voet gedaan}\\

\haiku{Het is bij half vijf.}{en dus bijna etenstijd aan}{de table d'h\^ote}\\

\haiku{Men is daar volmaakt,.}{goed zeer geschikt om een paar}{dagen te bijven}\\

\haiku{De Italiaanse.}{heldere lucht schijnt voor ons}{verloren te zijn}\\

\haiku{Ik schrijf u dit nog ',.}{te 9 uurs avonds zo van}{tafel komende}\\

\haiku{Hij kwam van Parijs.}{en ging nu de winter in}{Itali\"e doorbrengen}\\

\haiku{Ik neem de zin op).}{te Altdorf te gaan zien de}{finstere Aareschlucht}\\

\haiku{Het was alsof men *.}{de gehele~         aarde}{aan zijn voeten had}\\

\haiku{Te 5 uur aten wij,.}{met 4 Engelsen aan wie}{wij ons niet stoorden}\\

\haiku{De Aar vloeit erlangs.}{en zo langzaam dalende}{rijdt men de stad in}\\

\haiku{Zeker hebt gij mijn.}{complimenten aan Koos en}{Dientje overgebracht}\\

\haiku{Van Bern heb ik u,.}{mijn laatste brief gezonden}{lieve beste vrouw}\\

\haiku{Dit plaatsje ligt aan,.}{het meer van dezelfde naam}{westelijk van Bern}\\

\haiku{Deze lindeboom.}{heeft beneden een stam van}{32 voet diameter}\\

\haiku{In de zeer goede.}{en nette salon gingen}{wij ons neervlijen}\\

\haiku{En dan gaat het op,.}{Bazel waar ik morgenavond}{hoop aan te komen}\\

\haiku{{\textquoteright} In dezelfde geest:}{liet de Geelbiografe}{Hamaker zich uit}\\

\haiku{indien wij allen,?}{dat voorbeeld volgden wat wierd}{er dan van het k.b.}\\

\haiku{Van belang is, dat {\textquoteleft}{\textquoteright}.}{in Geels ogen het hier eenvrij}{verre reis betrof}\\

\haiku{De vloot stond onder (-).}{bevel van Victor-Guy}{Duperr\'e17751846}\\

\haiku{W.E. Mead, The Grand Tour,.}{in the Eighteenth Century}{Boston/New York 1914}\\

\section{Piet Bakker}

\subsection{Uit: Branding}

\haiku{Hij kefte zachtjes,.}{een paar keer de ogen brandend}{naar de zee gericht}\\

\haiku{Als er wat in  ,,.}{dreef kon je pas zien hoe hoog}{die krullers gingen}\\

\haiku{{\textquoteleft}Hoepla Pantertje,!}{dat kraigt die voile zee niet}{van den baas terug}\\

\haiku{Bloedrood stroomde de,.}{wijn in het zand toen Steven}{het vat neergooide}\\

\haiku{Nou ga je weg en....}{vanavond om acht uur kom je}{je antwoord brengen}\\

\haiku{Toen hij de duinpan,.}{afliep stond Steef al aan de}{deur en hij lachte}\\

\haiku{En hai je 'm toen....?!}{niet meteen een klap voor z'n}{harses gegeve}\\

\haiku{Ze moesten niet denken, '!}{dat zem op die manier}{weg konden pesten}\\

\haiku{Reiers hond blafte.}{er hol tegenin en kroop}{achter zijn meester}\\

\haiku{Roerloos, borst en hals,.}{\'e\'en rode vlek mouwen en}{broek donker besmeurd}\\

\haiku{Jai kan je grote,....?!}{bek tege me houwe hai}{je dat begrepe}\\

\haiku{De dorpsheid van de.}{Zandwijkse meisjes miste}{zij ten enenmale}\\

\haiku{Dat je vader een,.}{liefertje is zal je uit}{mijn mond niet horen}\\

\haiku{Eigenaardig, dat.}{ie niet met die wond naar u}{is toegelopen}\\

\haiku{Niemand zag reder,.}{Arends meer doch allen voelden}{zijn aanwezigheid}\\

\haiku{De {\textquotedblleft}IJsland{\textquotedblright} met drie....?}{duizend manden schelvis en}{twaalf honderd gemengd}\\

\haiku{Zijn ogen verrieden,.}{een boze vreugde toen hij}{zijn dochter aankeek}\\

\haiku{Zo'n bink as Steven.}{Paauwels kon je een roer in}{se pote geve}\\

\haiku{'k Zou me nog een}{beetje met de bezem op}{me flikker komme}\\

\haiku{We wazze gestrand.}{en niemand  had er wat}{van gemorreke}\\

\haiku{Iedereen heeft het,?}{recht om de smoor aan iemand}{te hebben niet waar}\\

\haiku{Alleen Ome Hannes.}{wisselde nu en dan een}{paar woorden met hem}\\

\haiku{Alle romantiek.}{van het zeedorp bewoog zich}{om de reddingboot}\\

\haiku{Als er een stranding,.}{was geweest dan preekte de}{domin\'e daarover}\\

\haiku{Dat ze hem in 't,.}{dorp voorbijliepen kon hij}{nu wel verdragen}\\

\haiku{Arends en de baron,.}{zaten niet stil daar kon je}{donder op zeggen}\\

\haiku{Maar zes walmende.}{flambouwen gingen voor de}{driftige stoet uit}\\

\haiku{{\textquoteright} Jongens werden ruw.}{bij de kraag genomen en}{in het zand gekwakt}\\

\haiku{Dan legden zij hun,.}{kleinheid af vergaten zij}{hun jaloezietjes}\\

\haiku{{\textquoteright} Ook nu liep weer het.}{vrouwvolk achter de boot en}{de bemanning aan}\\

\haiku{Daar hadden ze een,!}{degelijk mannetje aan}{gekregen da's vast}\\

\haiku{{\textquoteright} lachte Steef, nu de.}{inspanning van het roeien}{niet meer nodig was}\\

\haiku{Nog v\'o\'or Ome Hannes,.}{een bevel gaf had Steef naar}{zijn riem gegrepen}\\

\haiku{Die Raier was een.}{goeie roeier en dat is ie}{natuurlijk nog wel}\\

\haiku{Steef voelde niet veel,.}{voor zo'n uittocht maar wilde}{geen spelbreker zijn}\\

\haiku{Hij sloeg zijn glas om.}{en ging met de rug naar de}{bezoekers staan}\\

\haiku{Zonder verdriet dacht,.}{hij aan haar terug zelfs met}{een zeker verwijt}\\

\haiku{Wie hem nu wat in,.}{de weg zou leggen kon het}{goed bij hem hebben}\\

\haiku{De koddebeier.}{wilde zo opgewonden}{het dorp nog niet in}\\

\haiku{De schelle stem van.}{Klunder wekte hem uit zijn}{halve verdoving}\\

\haiku{{\textquoteleft}Hee daar, ik moet vier!}{man hebben om dit vrachie}{naar huis te kruien}\\

\haiku{{\textquoteright} Steef Pauwels keek hem,:}{met \'e\'en oog aan pakte een}{nieuwe strik en zei}\\

\haiku{Dat was al goud waard,.}{zoals die klungel daar op}{z'n platte kont zat}\\

\haiku{Daar zullen ze van,.}{afblijven al zal ik me}{d'r dood voor vechten}\\

\haiku{'k Mag blij zijn, als.}{ik die schuit in IJmuie zelf}{kan inspecteren}\\

\haiku{Over een maand of zo,.}{dan leit er weer een nieuwe}{troller in IJmuie}\\

\haiku{Was niet te vinden.}{voor een aardigheidje met}{de assurantie}\\

\haiku{Ze moeten alleen,.}{niet denken dat ik me als}{een kalf laat slachten}\\

\haiku{{\textquoteright} {\textquoteleft}Ze zal hier nog een,{\textquoteright},.}{trekpleister hebben denk ik}{grinnikte Crijnssen}\\

\haiku{Die waren van een,.}{stroper boven wien het duin}{ineengestort was}\\

\haiku{Toen hij de blote,,.}{gezwollen voet zag liet hij}{het geweer zakken}\\

\haiku{zou een van jullie.}{naer het dorp kunne fietse}{om hulp te haele}\\

\haiku{Want as ik 'm had, '!}{laete verzoipe zout wel}{moeilijk zain geweest}\\

\haiku{{\textquoteleft}Je had tegen de,?}{zolder moeten vliegen van}{de pijn weet je dat}\\

\haiku{Dat getuigen bij.}{het trouwen van Arends vond hii}{een beroerd karwei}\\

\haiku{De koddebeier.}{werd door elkaar gehusseld}{en neergesmeten}\\

\haiku{{\textquoteleft}As je \'e\'en poot over,!}{me drempel zet dan slae ik}{je de harses in}\\

\haiku{Zij brachten hier hun,.}{buit die Lauwers dan weer naar}{den grossier doorzond}\\

\haiku{Dat was toch om een.}{dodelijke ziekte op}{je lijf te krijgen}\\

\haiku{Dat moisie heb nog{\textquoteright},.}{een staertje voorspelde een}{ouwe visserman}\\

\haiku{Zijn moeder was goed,.}{voor hem geweest een vader}{had hij nooit gekend}\\

\haiku{Dokter Hagens gaf.}{Steef die dag verlof om het}{bed te verlaten}\\

\haiku{{\textquoteright} Fraukje glimlachte.}{hem vriendelijk toe en nicht}{vond het overdreven}\\

\haiku{{\textquoteright} Toen Steef die nacht op,.}{zijn harde brits lag kon hij}{de slaap niet vatte}\\

\haiku{Die man kon toch geen?}{tevreden ogenblik meer in}{z'n leven hebben}\\

\haiku{Jai mot dat zaekie.}{niet allenig opknappe}{voor ons allemael}\\

\haiku{waren zij midden.}{in een groot jachtgezelschap}{terecht gekomen}\\

\haiku{Thois was het armoe.}{en de knaine brachte een}{raiksdaelder op}\\

\haiku{De zes weken in.}{het gevang waren als een}{boze droom geweest}\\

\haiku{{\textquoteright} {\textquoteleft}Laete we er dan{\textquoteright},.}{bai gaen zitte zei Steef met}{een schamper lachje}\\

\haiku{een man tegenover,,.}{u die tot het ainde toe}{doorvecht as het mot}\\

\haiku{Toen draaide Steef zich.}{langzaam om en wilde het}{vertrek verlaten}\\

\haiku{Tegen den koster,.}{had hij gezegd dat er geen}{hulp voor nodig was}\\

\haiku{Dit was de tweede,,,.}{kist die hij laat in de avond}{in het duin begroef}\\

\haiku{Is dat verdomme?!}{nog toe een manier om me}{te late wachte}\\

\haiku{Daar had een meester,.}{op gezeten die hart voor}{zijn spulletjes had}\\

\haiku{Een tweede Grimsby,.}{een Cuxhafen zou hij van}{IJmuiden maken}\\

\haiku{Steef Paauwels schipper - '!}{op die nieuwe trawler dan}{zou jes wat zien}\\

\haiku{{\textquoteleft}As je een wijf was, ',,.}{zouk haast zeggen dat je}{er slecht uit ziet maatje}\\

\haiku{En laten ze me,!}{nou allemaal uitschelden}{dat het een aard had}\\

\haiku{Op den duur zou dit,.}{werk hem tegenstaan dat wist}{Steef met zekerheid}\\

\haiku{Heel zijn gezicht wees,.}{uit dat hij er nu wat voor}{begon te voelen}\\

\haiku{Daar zouden ze zich '!}{int dorp toch zeker een}{koliek om lachen}\\

\haiku{Wordt dat weer es een,....!}{ouwerwets daggie klaine}{doivel dat je bent}\\

\haiku{Het was waterkoud.}{en Steef rilde toen hij zijn}{trui over de kop trok}\\

\haiku{Als uitgelaten.}{jongens liepen zij naar de}{zeewering terug}\\

\haiku{In haastige drift.}{als wilde de een over de}{ander heenlopen}\\

\haiku{het schip de kop te.}{zien opheffen en in het}{golfdal dompelen}\\

\haiku{{\textquoteright} {\textquoteleft}Blaive jullie maar '{\textquoteright},.}{int zand wroete snoof}{Steef verachtelijk}\\

\haiku{{\textquoteright} Van Weerden liet de:}{bui rustig over zich heen gaan}{en antwoordde stuurs}\\

\haiku{{\textquoteright} {\textquoteleft}En ik kan maar eens{\textquoteright},.}{verzuipe zei Van Weerden}{onder het weggaan}\\

\haiku{Je moest zeker oud.}{zijn om met alles vrede}{te kunnen hebben}\\

\haiku{{\textquoteleft}Het volk, dat nu op,.}{de trawlers zit zal het wel}{minder mooi vinden}\\

\haiku{{\textquoteleft}Ik ben nu eens niet,!}{in een stemming om gekijf}{te verwekken vrind}\\

\haiku{De mannen van de -.}{vuurtoren daar boven de}{jutters beneden}\\

\haiku{Van z'n eerste reis!}{al achter de sleepboot naar}{huis in vliegend weer}\\

\haiku{De brekers spoelden.}{over het dek en namen mee}{wat niet muurvast stond}\\

\haiku{Als een witte muur.}{stond de zee secondenlang}{boven de sleepboot}\\

\haiku{Als een litteken.}{stond de vertrokken mond in}{het witte gezicht}\\

\haiku{Toen de boot omlaag.}{zakte kwam de verlamde}{schipbreukeling mee}\\

\haiku{De kiel bonkte al,.}{op het zand nog \'e\'en breker}{en ze waren er}\\

\haiku{Hij gelastte, dat.}{de boot weer op de wagen}{zou worden gebracht}\\

\haiku{bedaard zijn zwarte,.}{sigaar te roken toen de}{mannen zich meldden}\\

\haiku{Zonder aarzelen.}{liep zij op de baar toe en}{zij stak de hand uit}\\

\haiku{{\textquoteright} Vol bewondering.}{keek meneer Vermaas naar den}{ouden zeeman op}\\

\haiku{De vaste hand van.}{Hannes kon de boot haaks op}{de golven houden}\\

\haiku{De kostelijke.}{blinkende machine kon}{in zee verroesten}\\

\haiku{Het stuurhuis met de.}{kaartenkamer waren al}{lang kapot gebeukt}\\

\section{Jacobus Barnaart jr.}

\subsection{Uit: Dagverhaal van merkwaardige voorvallen}

\haiku{22.Ben ik op het,;}{Latijnsche school gekomen}{oud zijnde 11 Jaaren}\\

\haiku{Egberts Rector een.}{redenvoering gedaan over}{de waare geleerdheid}\\

\haiku{dit huis gepoogd heeft[] *.}{te bewaaren.6        17421742}{February 9}\\

\haiku{zo een helder ligt.}{van sig dat de oog en daar}{bijna van traanden}\\

\haiku{wierden terwijl zij}{besig waaren met wijn te}{sluiken61 in de kasjot62}\\

\haiku{Den 9e May zijn wij;}{met de armee93 gecampeert}{1,5 uur van Doornik}\\

\haiku{die arme zuster}{met de overige wijven}{gehoord hebbende}\\

\haiku{En Barnaart, om haar,;}{hart te streelen In schyn van}{Zephirus te speelen}\\

\haiku{37{\textquoteleft}Mijn ogen waren{\textquoteright}.}{even beschoten d.w.z. Ik was}{licht ingesluimerd}\\

\haiku{Waarschijnlijk in 1685.}{bij de herroeping van het}{Edict van Nantes}\\

\haiku{Op die plek nu nog.}{de straatnamen Prinsen-}{en Statenbolwerk}\\

\haiku{251Het onder water.}{zetten ten behoeve van}{de verdediging}\\

\section{Belcampo}

\subsection{Uit: De zwerftocht van Belcampo}

\haiku{Hij stortte al zijn.}{teleurstelling voor me uit}{en wat zeg je dan}\\

\haiku{We namen afscheid:}{en even later klonk het weer}{door de bossen van}\\

\haiku{Daarbij lieten zij.}{hem de onmogelijkste}{standen innemen}\\

\haiku{Het eten was heerlijk,.}{varkensvlees en rode kool}{met aardappelen}\\

\haiku{Maar als het enigszins,,.}{kan teken ik iedereen}{die mij eten geeft uit}\\

\haiku{Een oude boerin;}{stak het hoofd uit het raam en}{keek eerst erg lelijk}\\

\haiku{en wandelde ik.}{bij stralend weer het tweede}{vreemde land binnen}\\

\haiku{Het was daar warm en.}{ze waren juist bezig met}{oliebollenbakken}\\

\haiku{{\textquoteright} In Frankrijk beseft,;}{men pas dat van eten een kunst}{gemaakt kan worden}\\

\haiku{Haha, ik ben sterk,,.}{ik ben fors gebouwd ik kan}{wel vijf Duitsers aan}\\

\haiku{Hij was er dan ook.}{erg mee in zijn schik en gaf}{mij er vijf franc voor}\\

\haiku{Nergens praat men zo.}{gezellig als achter een}{gedekte tafel}\\

\haiku{Gelukkig ging het,.}{allemaal goed in Gap was}{de rit ten einde}\\

\haiku{Aan de weg was een,.}{punt waar ik wel een half uur}{heb v\`ergezien}\\

\haiku{De heuvels wierpen.}{naar mij toe een steeds dieper}{werdende schaduw}\\

\haiku{de koper had er}{nog nooit in gezeten en}{ik zat er al in}\\

\haiku{Zie je wel, zei de,.}{Fransman aan de Riviera is}{het altijd mooi weer}\\

\haiku{Nu was ik waar ik.}{wezen wilde en had dus}{verder geen haast meer}\\

\haiku{Het paviljoen zag;}{er uit als de werkplaats van}{een openluchtalchimist}\\

\haiku{Ik hield mijn broek aan.}{met de contanten er in}{uit vrees voor diefstal}\\

\haiku{de dames hadden.}{blijkbaar al de dood voor ogen}{en gaven het over}\\

\haiku{Ik was vol blijde,.}{verwachting van het avontuur}{dat in de lucht hing}\\

\haiku{Het beste, wat men,,.}{met schone vrouwen kan doen}{is naar ze te zien}\\

\haiku{Een er van keek mij:}{een ogenblikje aan en zei}{toen zo maar plompweg}\\

\haiku{Gelukkig werden.}{ze door mijn aanbod geboeid}{en geen een ontkwam}\\

\haiku{E\'en, die vond dat hij,.}{op minister Cavour leek}{wilde zelfs twee keer}\\

\haiku{Het hele huis stonk,,.}{ik weet niet waarnaar misschien}{wel naar oude lucht}\\

\haiku{In de spitsuren, van,,}{1 tot half 3 tekende}{ik in de caf\'es}\\

\haiku{wond zich daar erg over,,,.}{op ze willen wel maar ze}{durven niet zei hij}\\

\haiku{Vaak begonnen ze.}{dan te lachen en lieten}{zich toch tekenen}\\

\haiku{Gezicht op Napels;}{met de Vesuvius}{op de achtergrond}\\

\haiku{En die lag daar al,.}{in 1200 toen bij ons alles}{nog moest beginnen}\\

\haiku{Wie gaat er nu voor,?}{zijn plezier naar een eenzaam}{man die verdriet heeft}\\

\haiku{De graaf was verbaasd,.}{en ontdaan het was hem te}{plotseling blijkbaar}\\

\haiku{Die beeldhouwer had!!}{het lichaam van zijn vrouw uit}{het hoofd gekend}\\

\haiku{Ja, ja, ich sage,,...}{Ihnen mein lieber Herr die}{g\"ottlichen R\"omer}\\

\haiku{Bij Volterra was.}{zo een hele kerk in de}{afgrond gesukkeld}\\

\haiku{Ik heb gisteravond.}{nog lang nagedacht en weet}{nu wat je doen moet}\\

\haiku{het gesprek was niet.}{anders dan een herhaling}{van het voorgaande}\\

\haiku{gesprek hoort men steeds '.}{zeldzamer en opt laatst}{helemaal nooit meer}\\

\haiku{{\textquoteright} {\textquoteleft}Dan w\`ordt ze zeekr.}{allemoale duur dat}{gat h\`enemieterd}\\

\haiku{{\textquoteright} {\textquoteleft}St\`elt oe ees vuur, da'j}{doar zatn te kiekng en ie zang}{oe eeng detta23 k\`ats}\\

\haiku{de Wieringermeer,.}{toe't d\`en pas dreuge was emaakt}{was doar nog niks bie}\\

\haiku{Toen we bij de rand,.}{van de krater aankwamen}{was het al donker}\\

\haiku{in je huis een steeds.}{wisselende bevolking}{van wilde vissen}\\

\haiku{hij was Hongaar en.}{had zich al door de hele}{wereld geslikt}\\

\haiku{Een stationshond.}{ging vlak voor de deur staan en}{blafte mij in slaap}\\

\haiku{Toen ik wakker werd,.}{liepen de muizen om het}{hardst over de balken}\\

\haiku{Het ging gelukkig,.}{goed want de stenen gingen}{allemaal verkeerd}\\

\haiku{En diezelfde dag.}{kwam er een ontzettende}{regenbui over mij}\\

\haiku{Maar wat ik ook deed,.}{ik kon die namen er niet}{bij ze uit krijgen}\\

\haiku{Aan de wand hing een, {\textquoteleft}{\textquoteright},.}{parapluParaplu zei}{ik en wees daarnaar}\\

\haiku{Ze kwamen in een.}{kring om me heen staan en het}{regende woorden}\\

\haiku{En midden in de.}{nacht werd ik opeens wakker}{van een luid gedreun}\\

\haiku{ik had me bij het.}{laatste politiebureau}{moeten afmelden}\\

\haiku{Dit gevoelsleven,.}{wordt losgelaten het wordt}{individueel}\\

\haiku{Aan deze kant van;}{Osnabr\"uck maakten ze er}{pas een echt feest van}\\

\section{Colla Bemelmans}

\subsection{Uit: Platbook 3. Gebaorte en doe\"ed}

\haiku{Door gesprekke m\`et ' '}{dee zaogch in tot wat}{ch in m'nen doed}\\

\haiku{Waat zalle weej dich.}{noow toch misse weej meuste}{vuu\"els te flot oetein}\\

\haiku{mer doe woors d'r neet}{diene stool is laeg en oos}{bed is te groeat}\\

\haiku{Bin bliej, hae veult noow,.}{gein pien Maar hae had zoe gaer}{nog thoes wille zien}\\

\haiku{aoje boum haet now.}{ein wong wao ie\"erst dae tak}{met bloesem hong}\\

\haiku{Toch, de insigste}{vakantie die mien moder}{oe\"ets haet gehad}\\

\haiku{Ze zeen gebore, '.}{op 24 December 1928t}{zeen mien Korstkindjes}\\

\section{Hans Berghuis}

\subsection{Uit: Niet naar de maan gaan}

\haiku{Ugo Claudio was toch,.}{wel een Romein die er zijn}{mocht dacht ik woedend}\\

\haiku{Dan is hij groot en,.}{sterk dan lacht zijn mond en dan}{schitteren zijn ogen}\\

\haiku{Van daaruit kun je. '}{de slaapkamerramen van}{het  kasteel zien}\\

\haiku{s Avonds zitten de.}{sterke knapen van het dorp}{urenlang op de bank}\\

\haiku{Ik parkeerde het.}{dampende autootje aan de}{rand van het vliegveld}\\

\haiku{Hier is niets aan de.}{hand. Mark zit immers niet in}{de gevangenis}\\

\haiku{De zaken van God,...}{gaan v\'o\'or de affaires van}{de mensen ofschoon}\\

\haiku{Aarzelt zij even bij?}{exit no. 7 vanwaar je naar}{Athenai kunt vliegen}\\

\haiku{Neen, zij gaat rustig.}{verder naar de uitgang voor}{Palma-Portmany}\\

\haiku{Verleden zondag.}{zit mijn vader toevallig}{met mij in de kerk}\\

\haiku{{\textquoteright} Ik begrijp het wel,.}{maar dat kan ik tegen mijn}{vader niet zeggen}\\

\haiku{s Morgens ligt het.}{pak van mijn vader over een}{stoel in de keuken}\\

\haiku{Hij doet zijn kaken.}{evenmin van elkaar maar ik}{hoor zijn hart praten}\\

\haiku{{\textquoteleft}Maak het niet te laat,,{\textquoteright}.}{Mark zei Nora zacht toen zij}{naar haar kamer ging}\\

\haiku{Wij zijn geboren.}{tussen 1922 en 1925 en wij}{zijn de Veertigers}\\

\haiku{Ik heb zolang al.}{de wens gehad eenmaal met}{Nora te dansen}\\

\haiku{{\textquoteleft}Mevrouw, een dichter;}{is alleen als dichter in}{dit land niet veel waard}\\

\haiku{hen te verleiden.}{zich zelf volkomen aan mij}{uit te leveren}\\

\haiku{De Middellandse,.}{Zee is trouwens altijd de}{moeite waard altijd}\\

\haiku{{\textquoteleft}Materi\"ele,.}{geschiedenis heeft er niets}{mee te maken Mark}\\

\haiku{{\textquoteright}, dan zou Mark vrede,;}{met hem hebben gehad maar}{dat doet Johan G. niet}\\

\haiku{dat nog eens,{\textquoteright} zegt hij.}{alsof hij vreest dat hij mij}{verkeerd verstaan heeft}\\

\haiku{En wij weten dat,,.}{ook wel wij Westerlingen}{wij Nederlanders}\\

\haiku{Wij gaan een nieuwe,.}{tegemoet maar de nieuwe}{wereld is ook klein}\\

\haiku{maar hij gelooft in,.}{de genade als in een}{verlossing ik niet}\\

\haiku{Alleen Mon mag hem,.}{storen alleen van Mon kan}{hij alles hebben}\\

\haiku{Mijn God, s\'oms hapt hij:}{in het lokaas maar zelfs d\'at}{doet hij met opzet}\\

\haiku{vanmiddag moesten wij,.}{elkaar pijn doen maar het was}{eindelijk weer goed}\\

\haiku{Hij had mij en de.}{kinderen naar het vliegveld}{in Palma gebracht}\\

\haiku{Van Portmany kun.}{je in de winter niet per}{vliegtuig vertrekken}\\

\haiku{de rust van alle.}{mensen die het opgeven}{verder te leven}\\

\haiku{ik ben natuurlijk;}{niet zijn Beatrice maar}{zijn Assepoester}\\

\haiku{Luister eens, Ludwig,.}{ik veracht mannen die over}{hun vrouwen klagen}\\

\haiku{Ik weet het niet maar ', '.}{k voel het zok word aan}{een kruis geslagen}\\

\haiku{Je tastte niet eens.}{in het donker rond als een}{kind dat verdwaald is}\\

\haiku{Ro had vandaag een.}{paar vel schrijfpapier van zijn}{tafel genomen}\\

\haiku{Ik heb u nodig,,.}{schone baronesa ik}{heb u zeer nodig}\\

\haiku{Excellentie zal}{het mij niet kwalijk nemen}{wanneer ik opmerk}\\

\haiku{Mijn eigen zoon is.}{het slachtoffer van deze}{getuige geweest}\\

\haiku{Ter informatie.}{van Uwe Excellentie volgt}{hier de vertaling}\\

\haiku{Laat mij nu schrijven.}{over de onmogelijkheid}{van mijn terugkeer}\\

\haiku{Wellicht wenste zij,.}{het ook hoewel zij het nooit}{heeft uitgesproken}\\

\haiku{Welk een zelfbedrog,,.}{welk een menselijke waan}{mijn goede Ysbrand}\\

\haiku{Het spreekt vanzelf dat.}{jullie de oude God niet}{meer nodig hebben}\\

\haiku{Jullie baan leidt naar.}{de technische ruimte in}{de buurt van de maan}\\

\haiku{En heb geduld met,,.}{mij Pieter het is toch nog}{een moeilijk verhaal}\\

\haiku{De kleinste is een.}{lieve hangoor die wel eens}{tegen mij aankruipt}\\

\haiku{Ik moest je nog veel,.}{meer schrijven maar plotseling}{ontbreekt mij de moed}\\

\haiku{Toen viel hij uit zijn,.}{stoel hij lag gekromd op de}{vloer van de patio}\\

\haiku{Misschien kunt u mij?}{een glas whisky on the rocks}{laten serveren}\\

\haiku{Je weet wel dat ik.}{even blij ben over de reis die}{jij nu moet maken}\\

\haiku{Wat heb jij ermee?}{te maken als wij elkaar}{de nek omdraaien}\\

\haiku{Verbijsterd bleef zij.}{zitten totdat de ober haar}{de rekening bracht}\\

\haiku{Mark vond het nodig;}{om op dat uur zijn zonen}{naar bed te sturen}\\

\haiku{Mijn auto de fe,,.}{was een voorbeeld een beeldspraak}{een vergelijking}\\

\section{Anton Bergmann}

\subsection{Uit: Brigitta}

\haiku{Hij kende zijne.}{kwaal in den grond en kon er}{uren over vertellen}\\

\subsection{Uit: Twee Rijnlandsche novellen}

\haiku{Dan vereenigen.}{zich al de gasten in de}{algemeene zaal}\\

\haiku{- Op de oevers niet,.}{het kleinste huisje niet de}{nederigste hut}\\

\haiku{- 't Is voortaan geene,.}{vriendenstem meer die zijn hart}{nog treffen kan}\\

\haiku{'t Was aandoenlijk,}{en tevens belachelijk}{de wijfjes te zien}\\

\haiku{Geen woord van smaad, geene.}{schaduw van verwijt hoorde}{ik ooit uit zijn mond}\\

\haiku{Het zong altijd zoo,.}{vroolijk en lief als wij hier}{praatten en lachten}\\

\haiku{Of was het mijne,;}{hoedanigheid van vrijen}{Belg die hen aantrok}\\

\haiku{{\textquoteright} besloot Karel, met.}{zijne hand eene dreigende}{beweging makend}\\

\haiku{Hij kwam op ons toe,,.}{groette beleefd en nam den}{puntigen helm af}\\

\haiku{{\textquoteleft}Gij weet, dat ik een,.}{Frankforterin ben die geen}{Pruisen lijden kan}\\

\section{J.H. Bergmans-Beins}

\subsection{Uit: Het bloed kruipt waar het niet gaan kan. Een vertelling uit het Drentsche boerenleven}

\haiku{Een vertelling uit}{het Drentsche boerenleven}{Colofon}\\

\haiku{Wiecher Luten is {\textquoteleft}{\textquoteright} {\textquoteleft}{\textquoteright}.}{achter in denhof bezig}{dezwa te haren}\\

\haiku{In het midden is,.}{de groote dorschvloer die tot}{de groote deuren gaat}\\

\haiku{Deze {\textquoteleft}baander{\textquoteright} wordt.}{gebruikt voor het inrijden}{van hooi en koren}\\

\haiku{{\textquoteright} {\textquoteleft}Jao wal, heur,{\textquoteright} zegt het, {\textquoteleft},.}{meisjeik denk er wal um}{ik wol hen jow hoes}\\

\haiku{{\textquoteright} Bedaard stappen Harm ',.}{en Roelfien naart hooiland}{waar Wiecher hen wacht}\\

\haiku{En als Wiecher de,.}{laatste streken heeft gedaan}{komt hij hen helpen}\\

\haiku{Zij is er als een,.}{dochter in huis alleen ze}{is er niet altijd}\\

\haiku{Van dienstvolk houdt hij,,.}{niet wat met Roelfiens hulp niet}{kan wordt uitbesteed}\\

\haiku{{\textquoteright} Allen lachen om,.}{het idee dat Roelfien naar huis}{gebracht moet worden}\\

\haiku{{\textquoteright} {\textquoteleft}O, dat kan wal, wij '}{hebt de vroggen opt nij}{laand en dat laot}\\

\haiku{d'r is nog gien ien,',?}{weggaon die ik holl'n}{wilt hadd ij dan wal}\\

\haiku{Langs den wand glijdt haar.}{blik en ze voelt zich een met}{alles wat daar is}\\

\haiku{Als Wiecher terug, '.}{komt in de keuken gaat hij}{bijt vuur zitten}\\

\haiku{{\textquoteleft}As ij trouwen mient, ',.}{dan zekg ijt niet goed}{te laot mien ij}\\

\haiku{de messen afveegt.}{en in de tafella bergt}{en verder opruimt}\\

\haiku{'t Is een oud lied,,.}{dat moeder haar leerde maar}{Roelfien zingt het graag}\\

\haiku{want Roelfien was een.}{knap meisje en in haar tand}{niet onbemiddeld}\\

\haiku{{\textquoteright} {\textquoteleft}En daacht moeder dan, '?}{datt met de borrel weer}{op zien plaos kwaamp}\\

\haiku{{\textquoteright} {\textquoteleft}Now volk,{\textquoteright} zegt Heling, {\textquoteleft}, '.}{ik wil maor verloopen}{zegen mett zwien}\\

\haiku{Steeds meer komt het in,.}{haar gedachten maar ze k\`an}{het niet aannemen}\\

\haiku{Wat naar het midden,.}{zit een groep meisjes waarbij}{Roelfien en Sina}\\

\haiku{Als Wiecher naar huis,.}{gaat is Roelfien's hart lichter}{dan in langen tijd}\\

\haiku{Hij doet beslag in ',.}{t ijzer klapt het dicht en}{legt het op het vuur}\\

\haiku{Ja, h\`em hadden ze,!}{behouden maar wat hadden}{ze veel verloren}\\

\haiku{{\textquoteleft}Elk is niet eev'n '.}{kerks en elk kant ok niet}{eev'n goed waachten}\\

\haiku{Hoe was ze ineen,.}{gekrompen van angst toen ze}{haar hoorde hoesten}\\

\haiku{Elk meende, dat hij '.}{ert rechte over wist en}{elk meende verkeerd}\\

\haiku{Mij ducht, ik mus is '.}{eem hier hen en kieken hoe}{oft er heer giet}\\

\haiku{Ik miende, dat oes}{Wiecher wat bejegent was}{en dat zie dij hier}\\

\haiku{{\textquoteright} Vrouw Eling kijkt naar de,.}{leemen vloer die nog als van ouds}{in de keuken ligt}\\

\haiku{Och jao, 'n tweibak, ',.}{die magk wal kowie is}{aans ok zoo enkeld}\\

\haiku{Niet te veel, dan wordt {\textquoteleft}{\textquoteright}, '.}{hetjoegel maar een beetje}{kant wel lijden}\\

\haiku{O, o, wat zul wij,.}{nog beleev'n aal niks as}{muite en z\"orgen}\\

\haiku{Ik hadd' niks in hoes.}{doe Hillechien kwaamp en dat}{was mij min genogt}\\

\haiku{Ze heeft nog geen tijd.}{gehad om te bedenken}{wat ze zal zeggen}\\

\haiku{'t Is met 't mark.}{aal opgaon met aal die}{kovviedrinkers}\\

\haiku{Hie kun jao wal 'n,?}{wicht had hebb'n daor zie}{wat meer in zagen}\\

\haiku{{\textquoteright} {\textquoteleft}Dat kan nooit tot zien.}{geluk weez'n as hie met}{zien minderman trouwt}\\

\haiku{Hij gaat 's middags.}{naar den akker om Wiecher}{koffie te brengen}\\

\haiku{Nee, vaoder, now,.}{niet meer daorveur zin wij}{te wied hen praot}\\

\haiku{{\textquoteleft}dan huef ij ok '.}{niet te schromen urn d'rn}{enn an te maoken}\\

\haiku{Langzaam gaat hij weer ',.}{aant werk zonder naar zijn}{vader te kijken}\\

\haiku{Dat zijn vader niet,,.}{toegeven zal neen dat lijkt}{hem onmogelijk}\\

\haiku{Hij heeft Wiecher nooit, '.}{wat geweigerd al kostte}{t nog zooveel geld}\\

\haiku{En jong van darteg.}{jaor kun ij toch de wet}{niet meer veurschriev'n}\\

\haiku{Och, als Harm het goed,.}{vindt zij zal Roelfien graag als}{dochter ontvangen}\\

\haiku{Ze bukt zich en neemt.}{het koffiekannetje op}{om in te schenken}\\

\haiku{{\textquoteright} vraagt Roelfien, {\textquoteleft}ik daacht,.}{dat ij aaltied tevree en}{opgeruumd wassen}\\

\haiku{Dat kun ik niet best '.}{overgeev'n en ik kunt}{in ien dag niet doen}\\

\haiku{Maor vaoder '.}{is wat nustreg naot mark}{en dat verdr\"ot mij}\\

\haiku{Heb ij d'r met je, ' '?}{volk over praot datt met}{t haarfst wezen zul}\\

\haiku{{\textquoteright} {\textquoteleft}En denk ij, Wiecher, ', '?}{datt mienens is of zul}{t wal overbeeter'n}\\

\haiku{{\textquoteright} vraagt Roelfien, terwijl, {\textquoteleft}?}{ze haar moeder aanzietwat}{zeg moeder d'r van}\\

\haiku{{\textquoteright} {\textquoteleft}En ik,{\textquoteright} zegt Wiecher, {\textquoteleft}.}{zie niet van Roelfien of as}{zie mij hebben wil}\\

\haiku{{\textquoteright} {\textquoteleft}O, Wiecher, dat weet, '.}{ij wal daor zul ikt}{niet um overgeev'n}\\

\haiku{blieft staon, dan zoo, '}{gaauw meugliek tenminsten as}{ij hiern stee veur}\\

\haiku{Beiden bepalen, '.}{zich tot een zwijgen over wat}{hunt hoogste ligt}\\

\haiku{Schoon alles vrede,,.}{lijkt voelen alle drie dat}{er een oorlog dreigt}\\

\haiku{Hij gaat om het huis {\textquoteleft}{\textquoteright},.}{heen naar dekamer die een}{deur naar buiten heeft}\\

\haiku{{\textquoteleft}Harm hef 'n groot haart,{\textquoteright}.}{en nog stiever kop is het}{algemeen oordeel}\\

\haiku{Tegen Harm doet men,.}{dat zoo niet maar tegen haar}{zal men niet zwijgen}\\

\haiku{Hij voedt zijn toorn en,.}{praat zich steeds voor dat het recht}{aan zijn zijde is}\\

\haiku{Och,{\textquoteright} zegt Rieks, {\textquoteleft}as ij, ', '.}{dunkt datt wal kan dan mag}{ikt ok wal li\^en}\\

\haiku{{\textquoteright} {\textquoteleft}En mensk giet al zien, '.}{leven hen schoel as ijt}{maor weet'n wilt}\\

\haiku{Dan schuren ze langs.}{de palen en stooten hun}{neus in het voeder}\\

\haiku{Wiecher komt haar na,.}{met een tweede emmer die}{hij bij haar neer zet}\\

\haiku{Jammer, dat de naam.}{aan vader Rieks toekomt en}{het geen Harm kan zijn}\\

\haiku{Niet haasten, wachten.}{tot de tijd daar is en dan}{komt alles terecht}\\

\haiku{As 't ies mooi is, '}{en helder weer dan giet wal}{metn hakkenkruk}\\

\haiku{ik hum ofzee, met.}{zun gelegenheid kun hie}{wal ies besluten}\\

\haiku{Als Roelfien terug, '}{gaat komt Wiecher haar tegen}{en samen gaan ze}\\

\haiku{En nu zoo ineens,.}{vertelt Jenne haar dat er}{een kind wordt verwacht}\\

\haiku{Ze doet een kooltje.}{in de stoof en schenkt haar dan}{een kop koffie in}\\

\haiku{Verleden jaar was,.}{zijn moeder nog pas hersteld}{toen kwam er niet van}\\

\haiku{Ze houden zooveel.}{van Wiecher en gunnen het}{hem zoo van harte}\\

\haiku{as wij eerder teeg'n', '.}{d aol toorn anruepen}{dan kwamt ok weerum}\\

\haiku{Wat Wiecher verkeerd,.}{dee dat zal Haarm van Wiecher}{weer terecht brengen}\\

\haiku{En as Fennechien, '.}{komm'n wol dan wast ja}{zoo weer  terecht}\\

\haiku{Enkele dagen {\textquoteleft}{\textquoteright}.}{later wordt inRieksen hoes}{een zoon geboren}\\

\haiku{{\textquoteleft}Vaoder, moeder, ';}{wij hebtn jonge zeun en}{fiksche dikke jong}\\

\haiku{Als hij den hof is,.}{doorgegaan blijft hij even bij}{den ouden oven staan}\\

\haiku{Hij neemt zijn zakdoek.}{en veegt er hard mee over zijn}{oogen en zijn gezicht}\\

\haiku{Laot je kinner'.}{maor even zekgen wat}{dag of d anern kunt}\\

\haiku{Even wacht hij en als,,:}{hij hoort dat ze rustig ademt}{vraagt hij fluisterend}\\

\haiku{Nu neemt Annechien.}{Harm uit de wieg en legt hem}{op buurvrouw's schoot}\\

\haiku{wal wellent daor, '.}{men van zee dat zie vant}{heksenvolk wassen}\\

\haiku{De mannen zijn op.}{het land en Roelfien is in}{het tuintje bezig}\\

\haiku{{\textquoteleft}Klein Haarm{\textquoteright} met zooveel,!}{liefde verwacht met zooveel}{blijdschap ontvangen}\\

\haiku{Het verlangen om,.}{het kind van haar zoon te zien}{wordt haar te machtig}\\

\haiku{Harm is den heelen,.}{dag weg naar zijn zuster die}{wat ongesteld is}\\

\haiku{Ze doet het weer dicht '.}{zooals zet vond en haast zich}{dan naar huis terug}\\

\haiku{Mocht Fennechien nog,.}{eens komen dan is is het}{nog tijd genoeg}\\

\haiku{En as ik er dan,.}{klair met zin dan mag ien van}{je hum wal krieg'n}\\

\haiku{Wat boeten is, dat {\textquotedblleft}{\textquotedblright},{\textquoteright}.}{isspiensters gaodeng en}{dat wet elk zegt Rieks}\\

\haiku{Ik zin aal daag' nog','}{blied dat ik Wiecher hebb en}{as ik hum missen}\\

\haiku{{\textquoteleft}Het kan best, ik kan,.}{wal op Haarm passen daor}{is niks met te doen}\\

\haiku{Ze gaan samen den.}{Brink over en bij den zandweg}{gaat Sina terug}\\

\haiku{'t Lijkt Annechien,.}{wel goed toe maar de mannen}{moeten het weten}\\

\haiku{Op de {\textquoteleft}voorkist{\textquoteright} van.}{den bruidegomswagen zit}{de wasschupsneuger}\\

\haiku{Harm lacht en grijpt naar.}{moeder's oorijzer en naar}{de gouden doekspeld}\\

\haiku{{\textquoteright} zegt Roelfien, {\textquoteleft}wat kun,?}{ij daor vinnen dat van}{moeder zul weez'n}\\

\haiku{{\textquoteleft}Jao,{\textquoteright} meent Annechien, {\textquoteleft} '{\textquoteright}. '}{teeg'n twee'j maanlue kan ikt}{niet holl'm        XXXIV}\\

\haiku{{\textquoteright} vraagt Annechien, {\textquoteleft}'t,.}{leek niet zoo goed as vleden}{joar naar Wiecher zee}\\

\haiku{Och, wat zal hij het,.}{aardig vinden als zijn kind}{bij hem kan loopen}\\

\haiku{Soms bekruipt haar wel,.}{eens een gevoel van angst als}{ze Wiecher aanziet}\\

\haiku{Als het mislukt, grijpt:}{hij er met beide handen}{op en roept heel hard}\\

\haiku{Ankomm'n jaor,, '.}{as ij de boks ankriegt dan}{krieg ij okn pet}\\

\haiku{Dan verzet hij zich,.}{een beetje zoodat hij van den}{weg niet te zien is}\\

\haiku{Hoe kan iemand het, ';}{vuur nu uit laten gaan als}{t koffietijd is}\\

\haiku{Nu, zoo net voor het {\textquoteleft}{\textquoteright}.}{nieuweverbouw haalt men de}{stoet bij den bakker}\\

\haiku{hij is, dat Wiecher,.}{zou meenen dat hij daar met}{een bedoeling loopt}\\

\haiku{Langzaam valt een traan,.}{tusschen zijn vingers door maar}{hij bemerkt het niet}\\

\haiku{Daarom was het hem.}{een welkome aanleiding}{om thuis te blijven}\\

\haiku{Dat Wiecher met haar,,.}{alleen wil wezen is maar}{een praatje meent ze}\\

\haiku{Dat hij wel eens een,.}{enkele maal hoest dat heeft}{niets te beteekenen}\\

\haiku{{\textquoteleft}Nee,{\textquoteright} zegt ze, {\textquoteleft}gaot, '.}{maor is met dan za'kt}{je wal zekgen}\\

\haiku{Dan neemt Wiecher haar:}{hand tusschen zijn handen en}{zegt met heesche stem}\\

\haiku{'t Is een beetje;}{donker geworden en dit}{is Wiecher welkom}\\

\haiku{'t Is wal is 'n, '.}{dag of wat over maort}{komp ieder keer weerum}\\

\haiku{{\textquoteright} Wiecher glimlacht en.}{strijkt den kleinen jongen door}{het krullende haar}\\

\haiku{Als Roelfien na een,.}{poosje gaat kijken ligt hij}{rustig te slapen}\\

\haiku{De pogingen om,.}{Harm tot reden te brengen}{zijn te erg voor haar}\\

\haiku{{\textquoteright} Sina tracht haar de.}{duistere gedachten uit}{het hoofd te praten}\\

\haiku{Met een bezwaard hart,}{keert ze naar huis terug en}{neemt zich voor zoo gauw}\\

\haiku{Ze wil er ook niet,.}{meer over denken het maakt haar}{maar dubbel bedroefd}\\

\haiku{het zal winnen en.}{ze zonder bitterheid aan}{alles kan denken}\\

\haiku{Van de aardappels,,.}{die er op gegroeid zijn heeft}{hij niet meer geproefd}\\

\haiku{Waar ze gaat, vervolgt, '.}{het haar nooit kan zet meer}{van zich afzetten}\\

\haiku{Hij slaat zijn armpjes.}{om grootvader's hals en kust}{hem opgetogen}\\

\haiku{Een vertelling uit.}{het Drentsche boerenleven}{Noten 1slanker}\\

\section{Ger Bertholet}

\subsection{Uit: Sjweitberg (onder ps. G. Rapaille)}

\haiku{Volges Feel is dat,.}{ein volkscultureel versjeinsel}{zo\"e oud wie de welt}\\

\haiku{Ziene boum sjting '.}{in de dil ent waor}{sjtil in de dil}\\

\haiku{En gans klein sjting ':}{opt breefke in die flesj}{gesjri\"eve}\\

\haiku{Es de sjo\"el oet, '.}{is geitt natuurlik auch}{mit de lif nao heim}\\

\haiku{Ich bekiek mich 't,,}{fruit de greunte en mich geit}{durg d'r kop wae dat}\\

\haiku{Loup durg flab, de mos.}{nogal gek zi\"en om zo\"e}{deep nao te dinke}\\

\haiku{Zou me in Sjweitberg?}{zich ins good h\"obbe kinne}{laote gaon}\\

\haiku{De tant van de mam '.}{van d'r god van de luj van}{Sjweitberg ist meug}\\

\haiku{Zonger zeiver druegt.}{de aerd oet en lache}{kint nog ummer}\\

\haiku{En me moos enne.}{ki\"er dekker bukke want}{nog enne wage}\\

\haiku{Eederein dae kaom,.}{sjtumme kreeg e sjiek oranje}{vlegsjke maedje}\\

\haiku{Dae sjrie\"ewde,.}{in do\"edsno\"ed wie}{e mager verke}\\

\haiku{Mennige ki\"er.}{hat hae zich dao al d'r kop}{mit gesjto\"ete}\\

\haiku{En veer mer dinke '.}{dat dat nogt insigste}{is wat hiej nog wirkt}\\

\haiku{{\textquoteright} Zo\"e noe en dan l\"op,,.}{get nao boete m\`e de riej}{blief lank aeve lank}\\

\haiku{Dat is neet om te,,.}{lache geluif ich want de}{riej reageert neet}\\

\haiku{{\textquoteleft}Nae kink, sjtrreng,,.}{neet m\`e ze h\"obbe waal d'rr}{kop los  ocherrm}\\

\haiku{En daorom mot hae,.}{zich zo\"ev\"a\"ol meugelik}{aansjaffe vingt ze}\\

\haiku{aan 't gelle om!}{zich zo\"e good meugelik te}{kinne verkoupe}\\

\haiku{t Is ech get te.}{lank om d'r hiej lang uever}{te vert\`elle}\\

\haiku{'t Nul-nummer.}{van d'r Sji\"em van Murge}{zal gek zi\"en}\\

\haiku{{\textquoteright} {\textquoteleft}All\`e, m\`e wat dan,}{auch waor dat zal zich toch}{waal al lang gelag}\\

\haiku{Volgend jaor zal.}{ich waal huere wie dat}{aafgeloupe is}\\

\haiku{Natuurlik waere, '.}{de winkels aafgeloupe}{m\`et is angesj}\\

\haiku{Biej 't sjonste waer '.}{zitt ongel\"ok dus auch nog}{altied in de loch}\\

\haiku{Die ze opluchde.}{vuer vief dubbelzout of}{enne negerzoen}\\

\haiku{gekaoze om.}{doezende posdoeve oet}{te laote vleege}\\

\haiku{Doe mos 't i\"esj nog!}{mer ins zi\"en daste drin}{gekaoze wuers}\\

\haiku{{\textquoteright} {\textquoteleft}M\`e j\`owaal, die mam is '}{n zuster van mien mam en}{mien mam is getrouwd}\\

\haiku{Ich zi\"en die mer.}{e paar ki\"er per waek es}{ze mich get oethulp}\\

\haiku{Missjiens dat ich 't.}{auch ins mot gaon h\"obbe}{uever erotiek}\\

\haiku{{\textquoteright} Dat mot me zich, biej ',.}{n Sjweitbergse b\"oshalte dao}{laote duje}\\

\haiku{Dan begin ik toch '.}{te geluive dat geer aan}{t vlooke versjlaafd zeet}\\

\haiku{De sjnoet van de.}{Li\'es zit aeve versjtopt}{achter d'r pilaer}\\

\haiku{Bobo kriet geine.}{tied om sjeloes te waere want}{hae mot troef make}\\

\haiku{{\textquoteleft}Manou, dae moste...,,,!}{van achter pakke om dae}{witte band laok punt}\\

\haiku{Manou{\textquoteright}, vingt de Els {\textquoteleft}{\textquoteright}, {\textquoteleft}.}{opvolksgezondheidich mein}{dat die in Mestreech geit}\\

\haiku{Dae is nog te meug '.}{om te griene baove}{t unnesj\`elle}\\

\haiku{W\`etste, want mit.}{mien vrouw is gel\"okkig neet mien}{wermde gesjtorve}\\

\haiku{{\textquoteright} 't Maedje wat pis,:}{bie h\"a\"or kaom kroog gein kans om}{nog v\"a\"ol te zegke}\\

\haiku{Ze wol mer ein dink....}{en dat waor nuuj gode}{kweke ingele}\\

\haiku{M\`e hae vert\`elde.}{dat hae pas enne ouwe}{vrund mit begraave haw}\\

\haiku{{\textquoteleft}Ich weit 't.{\textquoteright} En hae:}{haolde e verfroemeld}{breefke oet zien t\`esj}\\

\haiku{Dao d'r vrund van is,?}{al twi\"e jaor allein wat}{meinste dat dat is}\\

\section{Joan Bertrand}

\subsection{Uit: Vrung va Oze Leve Hier}

\haiku{Dao green-er daeks.}{datter sjnakde uever}{de zung va vreuger}\\

\haiku{einne f\`eine,.}{wiensjmaak kietelt zieng tong verfrisjt}{zien verhiemelte}\\

\haiku{M\`e zieng weeg sjtong,:}{in et land va zon pa\^ume}{en appelziene}\\

\haiku{Dao wour aevels nieks.}{mie uever es zieng po\^usje}{zag der herbergeer}\\

\haiku{Wat hawwe ze.}{de mam al daek geplaogd}{um ei nuej breurke}\\

\haiku{Wie gaen hawwe.}{ze-em mitgenomme nao}{et noviciaat}\\

\haiku{- {\textquoteleft}Morloot hier pater{\textquoteright}, {\textquoteleft}!}{zag der vrachriejergier zeet}{mich einne sjo\`ene}\\

\haiku{{\textquoteright} Dan sjnapde hae.}{zich de kap va genne kop}{en reej op heim aa}\\

\haiku{Hae duipde-n-em.}{en wees-em eing plaatsj aan}{urges bie de Maas}\\

\haiku{Dat wour aevels neet,.}{gemekkelich al wour hae}{dan auch al zoe sjterk}\\

\haiku{Huej wour et Zondig.}{en e Goonsdig zou et Sint}{Caesilia zie}\\

\haiku{Et Pieterke wo\`ed.}{bang en leep watter laope}{kos de trappe aaf}\\

\haiku{Sjterve wour zo\`e erg,!}{neet en begrave waede}{wour auch zo\`e erg neet}\\

\haiku{Zi\`ene awwe,.}{zilvere baad jeugt taege}{et klei gezichske}\\

\haiku{onge gen erm en....}{dan leep-er dich durch gen weije}{wat der vaege kos}\\

\haiku{en in die sjtar - gier -,.}{geluift et jao neet ei klei}{wit kingerh\"andje}\\

\haiku{Dan zou der tieger.}{naeve et lemke komme}{lieke in gen weij}\\

\haiku{Eine boerejong.}{kaom aa-gelaope en}{heel de kameel vas}\\

\haiku{Der heilige Sint.}{Joezep wour jues mit ze nao}{gen durp gegange}\\

\haiku{Sint Joezep sjtong:}{Heur nog i\"esj hi\"el bezurgd}{der waeg te wieze}\\

\haiku{Gei vi\"ezelke.}{aan die kling dinger of et}{trilde van plezeer}\\

\haiku{Et Wimke wreef zich:}{van sjpas in gen heng en}{der melder fluidde}\\

\section{Anna Blaman}

\subsection{Uit: De arme student}

\haiku{Het is wel zeker}{dat de student daar nooit zou}{terechtgekomen}\\

\haiku{Armoede maakt je.}{eenzaam en hongerig in}{alle opzichten}\\

\haiku{En daarop schoof hij.}{vrijmoedig bij in hun kring}{en hief zijn pot bier}\\

\haiku{En daarna greep haar.}{hand feilloos een deurknop en}{deed ze een deur open}\\

\haiku{Luister, zei ze, en}{ze drukte me daarbij in}{de enorme stapel}\\

\subsection{Uit: Drie romans}

\haiku{ik had lang geen spijt,.}{van graad of titel die ik}{misgelopen was}\\

\haiku{Maar ik geloof dat,,.}{hij door dat te zeggen zijn}{jeugd vergeten wou}\\

\haiku{Ik kijk alleen maar,.}{naar je gezicht waarvan ik}{afscheid nemen moet}\\

\haiku{Want nauwelijks op.}{de terugtocht hield hij me}{stil en wou me kwijt}\\

\haiku{Een schemerkamer - -,.}{eerst koffiedrinken dan thee}{en sigaretten}\\

\haiku{Voor het eerst had ik,,.}{nu met een ander Jonas}{over haar gesproken}\\

\haiku{Als het te bar werd.}{laveerde ze de kant uit}{van het boertige}\\

\haiku{De morgen had ik -.}{zoek gebracht met Jonas en}{een krankzinnige}\\

\haiku{Ik wist het wel, zij,.}{had wel door dat het met Saar}{en mij niet deugde}\\

\haiku{Het deed me denken,...}{aan de dreinende ritmiek}{van regen regen}\\

\haiku{Hij had wat moeten,.}{zeggen wat anders moeten}{zeggen dan hij deed}\\

\haiku{Een prinselijke,.}{pauper een pauper met een}{prinsenziel was hij}\\

\haiku{Fortuin en liefde -.}{in uw leven zullen u}{de blonden geven}\\

\haiku{Ook Kareltje en.}{zij waren ten slotte aan}{de dijk gaan zitten}\\

\haiku{De lente bracht die;}{jongen met zijn vrouwelijk}{verholen liefde}\\

\haiku{Ze deed een vrouw in ',,.}{t bad die zich bevuild had}{een zachtzinnig mens}\\

\haiku{Maar daarna is ons.}{samenzijn ellendiger}{dan ooit tevoren}\\

\haiku{Ik hield krampachtig.}{de blijdschap in mijn labiel}{gemoed in evenwicht}\\

\haiku{{\textquoteright} {\textquoteleft}Nou, ik denk altijd.}{toch nog beter over een man}{dan over vrouwentuig}\\

\haiku{Zonder warmte in.}{zijn ogen boog hij zich naar haar}{toe en kuste haar}\\

\haiku{Hij keek haar vlug, en.}{zonder uitdrukking in zijn}{te lichte ogen aan}\\

\haiku{Jonas boog zich over.}{het portier en braakte de}{zwarte koffie uit}\\

\haiku{Het ging er nou maar,.}{om het vol te houden tot}{hij weer boven zat}\\

\haiku{Marie zat aan de,.}{grasglooiing waar ze beschut}{was tegen de wind}\\

\haiku{Ontevreden keek.}{ze naar de karreploeg die}{op het veld werkte}\\

\haiku{Hij zag hoe zij zich.}{rekte om die boven in}{de kast te zetten}\\

\haiku{{\textquoteleft}Ik had de indruk,{\textquoteright}, {\textquoteleft}.}{zei hij ingetogendat}{u iets hinderde}\\

\haiku{Misschien roken we,.}{samen een sigaret dat}{neemt de moeheid weg}\\

\haiku{Ik ging de kamer,,.}{in en streelde terwijl ik}{langs haar liep haar arm}\\

\haiku{Ik moest verdwijnen,.}{en wel onmiddellijk en}{zonder aarzeling}\\

\haiku{Ik moest het weten,,,.}{ik had haar gezien niet waar}{die zondagmorgen}\\

\haiku{Had ze zo'n honger,?}{of was haar maag nu nog niet}{helemaal op streek}\\

\haiku{Ze schonk zich nog een,.}{kopje thee in en dronk dat}{haastig dorstig leeg}\\

\haiku{Schamper haalde ze:}{de schouders op en boog zich}{naar de tafel toe}\\

\haiku{Op een gegeven.}{ogenblik trok ze de voeten}{van die stoel terug}\\

\haiku{een eenzaam man, stram,,,.}{arrogant met een lege}{hoffelijke grijns}\\

\haiku{Elke morgen nam:}{ze aan de piano haar}{oefeningen door}\\

\haiku{En bovendien, een,.}{kind het leven geven wat}{een verantwoording}\\

\haiku{Die maandag nog was,.}{hij naar een concert geweest}{een Bachrecital}\\

\haiku{My life since,.}{I loved you has been one}{prolonged agony}\\

\haiku{Op de tast greep ze.}{een nieuwe sigaret en}{zoog het vuur erin}\\

\haiku{Droomde ze wel ooit,?}{dat ze zich er daarna iets}{van herinnerde}\\

\haiku{{\textquoteright} {\textquoteleft}Ik heb,{\textquoteright} zei hij, {\textquoteleft}een,.}{grammofoon met mooie platen}{zoals Tannh\"auser}\\

\haiku{ze greep zijn hand en.}{zo bleven ze zitten toen}{het weer donker werd}\\

\haiku{De lucht was duister,.}{minder sterren waren er}{dan hij gedacht had}\\

\haiku{Maar ook aan morgen,.}{moest hij nu niet denken nu}{was hij met Marie}\\

\haiku{Heel zijn leven was.}{\'e\'en lange hunkerende}{wacht geweest op haar}\\

\haiku{Zou het ooit vriendschap?}{kunnen worden als hij hem}{nu al kwijtraakte}\\

\haiku{Tot morgen,{\textquoteright} zei hij -.}{nog en op het trambalkon}{keek hij nog even om}\\

\haiku{Ik was gekleed en.}{driftig wou ik nu op mijn}{beurt naar beneden}\\

\haiku{De zolder stond in.}{strakke binten over heel de}{diepte van het huis}\\

\haiku{Het meisje begon,.}{gejaagd te schreien daarom}{stuurden we haar weg}\\

\haiku{Zodra de bel ging,.}{haastte ik me naar de trap}{en trok de deur open}\\

\haiku{Zulk huilen maakte,.}{me machteloos ik stond daar}{maar en wist geen troost}\\

\haiku{{\textquoteright} Zij wiste met de,.}{vrije hand haar tranen weg er}{was zoveel te doen}\\

\haiku{Zachtjes trok ik de '.}{buitendeur int slot en}{draalde op de stoep}\\

\haiku{Voor koffietijd bracht.}{ik nog even mijn artikel}{aan de directeur}\\

\haiku{Hij was zuinig waar,.}{het op waarderen aankwam}{maar hij had gelijk}\\

\haiku{Mevrouw De Watter.}{keek mijn verbazing met een}{zachte glimlach aan}\\

\haiku{{\textquoteright} vroeg ze kinderlijk,.}{terwijl haarzelf de tranen}{in de ogen welden}\\

\haiku{De stappen hielden,.}{stil voor mijn deur er klopte}{iemand zachtjes aan}\\

\haiku{Op dat moment greep,}{ik hem bij de arm troonde}{hem mee en wachtte}\\

\haiku{Het regende niet,,}{meer de weg was modderig}{en onbegaanbaar}\\

\haiku{Ik wist nu ook, dat.}{hij ternauwernood nog een}{gesprek verwachtte}\\

\haiku{langs de oevers, die,.}{hun tocht vervolgden bleef het}{steekspel onbeslist}\\

\haiku{- Toos was de brede.}{straat teruggelopen en}{ze stak het plein over}\\

\haiku{Ze was nu aan het.}{aarden pad gekomen dat}{naar boven voerde}\\

\haiku{Bij jonge oogst kreeg,.}{ze een ijl angstgevoel dat}{niet onprettig was}\\

\haiku{Haar slordig haar viel,.}{voor haar ogen dreigend liep ze}{op de spiegel toe}\\

\haiku{{\textquoteleft}O neen, blijf nu toch,,{\textquoteright}.}{zitten Saartje zei mevrouw De}{Watter iets te luid}\\

\haiku{{\textquoteleft}Dat meisje,{\textquoteright} zei ik, {\textquoteleft},.}{stroefdat is de zuster van}{mijn vriend van Jonas}\\

\haiku{Haar ogen lagen nu,.}{vlak onder me ik zag de}{irissen verblauwen}\\

\haiku{Ze staarde leeg en,.}{treurig weg met vochtige}{verblauwde irissen}\\

\haiku{Ze keek haar aan en:}{ze ontmoette een paar ogen}{vol verweer en schrik}\\

\haiku{Want als dat zo niet,.}{was dan hadden we elkaar}{niet zo gevonden}\\

\haiku{Zou je lust hebben,,?}{vanavond ergens heen te gaan}{of morgen Sara}\\

\haiku{- Het ziekenhuis was.}{een door tuinen omgeven}{kloosterlijk gebouw}\\

\haiku{Op de zaal met aan,.}{weerskanten bedden zag ik}{Toos en toen pas hem}\\

\haiku{{\textquoteleft}Ik heb geen tijd, vind ',?}{jet vervelend dat ik}{je alleen laat gaan}\\

\haiku{Er daverde een,.}{trein uit het perron er kwam}{er weer een binnen}\\

\haiku{Ze trok zich los en,,:}{keek me koud afwijzend aan}{haar ogen waren grauw}\\

\haiku{In de bossen nam,.}{elk vrouwtjesdier de vlucht voor}{hem hij was geen dier}\\

\haiku{{\textquoteright} Met een ruk hield ze,,:}{toen stil ze keek me aan in}{koude haat ze zei}\\

\haiku{Ik sloeg de dekens.}{van me af en ging weer op}{m'n bedrand zitten}\\

\haiku{{\textquoteright} Maar ze kijkt me aan.}{of ze me niet herkent en}{zegt natuurlijk neen}\\

\haiku{Wat er gebeurd was,.}{was te teer om aangeroerd}{te mogen worden}\\

\haiku{{\textquoteright} Als de tram komt steekt,.}{ze waarschuwend de hand op}{wat niet nodig is}\\

\haiku{Ze zei me dat ze.}{nooit geloofd zou hebben dat}{te durven zeggen}\\

\haiku{Haar ogen waren grijs,.}{haar mond was rijp en toch weer}{zacht als van een kind}\\

\haiku{{\textquoteright} - Ze lachte even in,.}{m'n ogen maar haar hand in de}{mijne bleef passief}\\

\haiku{{\textquoteleft}Het blauwe paleis,.}{heeft me zo triest gemaakt ik}{weet niets vrolijks meer}\\

\haiku{Mijn dieptelagen;}{zijn heus van dezelfde stijl}{als mijn fa\c{c}ade}\\

\haiku{Het was de eerste.}{keer dat zij hem toestond bij}{haar aan te komen}\\

\haiku{Gisteravond, in die,.}{dancing meende hij succes}{geboekt te hebben}\\

\haiku{Stel nu dat die echo...}{eens verschald zou zijn en ik}{haar niet zou weerzien}\\

\haiku{En deze middag,,.}{fantaseerde ik wist King}{dat ze niet thuis was}\\

\haiku{En hij wist zelfs waar.}{ze heen was en hoe laat ze}{zou terugkomen}\\

\haiku{In de hand hield hij.}{de nagemaakte sleutels}{van de deur gereed}\\

\haiku{{\textquoteright} - Hij keek me lang en:}{stil aan en antwoordde toen}{bijna fluisterend}\\

\haiku{Geen wonder dat hij,.}{bang werd ik zag er uit als}{een krankzinnige}\\

\haiku{Het meningloze,,;}{meisje Annie keek even op}{en glimlachte flauw}\\

\haiku{daarom... ik zou het,.}{goed maken met hem hij zou}{h\'a\'ar nooit meer moeten}\\

\haiku{Neen, Berthe was, waar,.}{het de erotiek betrof niet}{erg prinsesselijk}\\

\haiku{Maar hadden ze ook...}{ooit gedacht dat de vorstin}{zelf met een lakei}\\

\haiku{Yolande keerde.}{zich van de etalage af}{en keek Alide aan}\\

\haiku{De palmen waren,.}{breed de vingers kort en sterk}{met sterke nagels}\\

\haiku{Dat was alsof je.}{op een ziel speelde als op}{een edel instrument}\\

\haiku{Yolande stootte,.}{uitdagend Berthe uit wat}{haar voldoening gaf}\\

\haiku{ze hebben een ziel,,?}{dat is wel zeker maar wie}{begrijpt dat overdag}\\

\haiku{Annies hand was kil.}{en Berthes hand was reddend}{in een vaste greep}\\

\haiku{{\textquoteleft}God ja,{\textquoteright} zei ze, {\textquoteleft}die,;}{heb ik nog en daarom ging}{ik maar eens lopen}\\

\haiku{Daardoor, Alide, zat,.}{ik aan die slootkant daardoor}{wist ik het niet meer}\\

\haiku{Was het een boze,,,?}{droom die Peps toe zeg dat het}{een boze droom was}\\

\haiku{Je handen hebben,.}{me gestreeld en behoed je}{schoot is kuis en koel}\\

\haiku{En dan ga ik dat,,,.}{kleuren cynisch gewaagd in}{schreeuwende verven}\\

\haiku{Toen ik eindelijk,.}{voor Mon Repos stond wist ik}{niets en dacht ik niets}\\

\haiku{Van bij het venster,.}{klonk het snuiven van huilen}{geluidloos huilen}\\

\haiku{Hij zweette als een.}{otter en hij rolde als}{een vod de trap af}\\

\haiku{Hij hield zich  groot,:}{en daarom riposteerde}{ze genadeloos}\\

\haiku{Maar Kosta bleef het druk.}{hebben met roken en keek}{haar vooral niet aan}\\

\haiku{Maar ik ben jong en.}{heb een lichaam dat zelfs nooit}{nog aan de beurt kwam}\\

\haiku{En gaf ze daarvoor,,?}{ook om dat te kunnen doen}{iets van zichzelf prijs}\\

\haiku{Daar zette hij een,.}{vast half uurtje voor als hij}{zich stond te scheren}\\

\haiku{Als 't nodig was.}{deed hij zo braaf en dweepziek}{als een heilsoldaat}\\

\haiku{Er overkomt je bij.}{haar in een korte spanne}{tijds een massa goeds}\\

\haiku{King zou toch kunnen,,.}{wachten ondanks dat briefje}{of terugkomen}\\

\haiku{Hij wou die vervloekt,.}{begerenswaardige jas}{hebben en voorgoed}\\

\haiku{Hij dacht dat hij zijn,.}{drift beheerste maar zijn stem was}{fluisterend en heet}\\

\haiku{En dat gebeurde.}{toen we de koffers pakten}{om te vertrekken}\\

\haiku{Ze deed het net zo.}{zorgzaam en zo liefdevol}{als ooit tevoren}\\

\haiku{Maar ik kon deze,.}{taak niet van haar overnemen}{dat was al te hard}\\

\haiku{Ik zou misschien zo'n,,.}{zelfde aandacht winnen dacht}{ik al pratende}\\

\haiku{{\textquoteright} zouden de tranen.}{niet te stelpen tussen mijn}{vingers door vloeien}\\

\haiku{Je schrok, zette de.}{borden uit je handen en}{brak in snikken uit}\\

\haiku{Maar waar was dat, ja,.}{in Mon Repos had ik haar al}{een keer geslagen}\\

\haiku{Alide kwijnde weg,.}{in West-Europa nu}{ik er niet meer was}\\

\haiku{Haar wang lag aan haar '.}{hand en zo keek ze de kant}{vant venster uit}\\

\haiku{Zijn hart bonsde zo.}{zwaar dat hij verwonderd was}{het niet te horen}\\

\haiku{{\textquoteright} - En ze spreidde als.}{een deken de kamerjas}{over hen beiden uit}\\

\haiku{Ze was een tors van.}{edel marmer en een duister}{bloedwarm vrouwenhoofd}\\

\haiku{Zijn spel groeit hem hier ',.}{bovent hoofd hij is zijn}{eigen spelbreker}\\

\haiku{De angst dat hij die,.}{droom niet houden kan zit hem}{al op de hielen}\\

\haiku{Juliette behoeft.}{dan ook vanzelf niet meer in}{de gevangenis}\\

\haiku{Toch, juist die avond van.}{King's avontuur was er diep in}{haar weer iets gaande}\\

\haiku{{\textquoteright} Maar daarop strekte,,.}{de hospita bezwerend}{sussend een hand uit}\\

\haiku{Maar toch herhaalde,:}{ze omdat het King maar was}{die haar gekrenkt had}\\

\haiku{Ze drukte haar borst.}{tegen me aan en hield het}{hoofd wat achterover}\\

\haiku{{\textquoteleft}Een vrouw kan kopen.}{voor zichzelf alsof ze haar}{eigen minnaar is}\\

\haiku{Bijna herfst was het,,,,.}{een vroege herfstmiddag wijd}{zonnig zorgeloos}\\

\haiku{Ik zag de schutting.}{terug van de tuin waarin}{ik als kind speelde}\\

\haiku{Terwijl het toch om,.}{hem begonnen was scheen ze}{hem glad vergeten}\\

\haiku{Ze glimlachte, ze.}{sloot de ogen en bracht zo haar}{mond op zijn gezicht}\\

\haiku{En elke keer dat.}{zij haar tanden poetste moest}{ze daaraan denken}\\

\haiku{{\textquoteleft}Maar, jij, heb jij haar,?}{nooit gemist al liet je dat}{aan mij niet merken}\\

\haiku{Ze streelde met haar,.}{sterke vingers wonderlijk}{zacht zijn wang zijn haar}\\

\haiku{- Alide nestelde.}{zich op de divan en ging}{liggen nadenken}\\

\haiku{Ze schikte een paar '.}{kussens ondert hoofd en}{staarde voor zich uit}\\

\haiku{{\textquoteright} - Ze wist het, niets kon.}{zijn visie op haar storen}{of beschadigen}\\

\haiku{En wat was er van?}{hem geworden sinds ze hem}{niet meer beschermde}\\

\haiku{Het was een rijke,.}{blik die van het lokkende}{speelzieke wijfje}\\

\haiku{Ze zagen haar het,.}{postkantoor verlaten en}{zij liep spitsroeden}\\

\haiku{Niettemin schreed ze.}{rustig en soepel voort naast}{de gebrilde Peps}\\

\haiku{Maar het klonk in haar,.}{op als hol geluid een stem}{van gene zijde}\\

\haiku{Ze bleef zichzelf, ze.}{raakte niet vereenzelvigd}{met de natuur}\\

\haiku{Hoe na{\"\i}ef om zo.}{hartstochtelijk te hechten}{aan haar instemming}\\

\haiku{Ze keek nog steeds heel.}{ernstig en voelde een soort}{innerlijke pijn}\\

\haiku{Peps kwam die kamer.}{binnen en ontdekte zijn}{Alide op dat bed}\\

\haiku{Haar minnaar vocht een,;}{gevecht tegen twee tranen}{maar in elk oog \'e\'en}\\

\haiku{Ze voelde hoe zijn.}{hand die op de hare lag}{begon te zweten}\\

\haiku{{\textquoteright} Die lag soms uren op.}{de divan met een paar ogen}{waar je bang van werd}\\

\haiku{Ik vind hem trouwens,?}{ook veranderd de laatste}{tijd merk jij dat niet}\\

\haiku{{\textquoteleft}Nou, die meiden daar,,.}{die schudden je wel uit in}{alle opzichten}\\

\haiku{Die kennen kunsten,,.}{nou daar trekken ze het merg}{mee uit je botten}\\

\haiku{{\textquoteright} - Maar toch was er nog.}{iets anders wat haar dwars zat}{en wat ze niet zei}\\

\haiku{Niet dat haar dat zo,.}{hinderde want eigenlijk}{gaf ze er niets om}\\

\haiku{Kijk, dat begreep ze,?}{van zichzelf niet was ze dus}{zo onredelijk}\\

\haiku{Die had de dag weer,,;}{in die winkel achter die}{kassa doorgebracht}\\

\haiku{Maar toen vluchtte ze,,,.}{de kamer uit de trap op}{naar haar kamertje}\\

\haiku{Ze ging naar binnen,.}{sloeg de deur achter zich dicht}{en liep naar boven}\\

\haiku{{\textquoteleft}Natuurlijk weet ik,...{\textquoteright}.}{het maar En een weigering}{in ogen en gebaar}\\

\haiku{Anne zelf, die was.}{daarbij ternauwernood van}{werkelijk belang}\\

\haiku{Zo sterk had hij zich.}{dus ge{\"\i}dentificeerd}{met zijn eenzaamheid}\\

\haiku{Hij hief alleen de.}{hand op en wreef langzaam over}{zijn vermoeide ogen}\\

\haiku{{\textquoteright} - En daarop keerde '.}{hij zich van me af en ging}{voort venster staan}\\

\haiku{Toen ik de kamer.}{uit liep hield hij het hoofd weer}{van me afgekeerd}\\

\haiku{Toen kwam de blik tot.}{hem via de brilleglazen}{en hij werd gezien}\\

\haiku{{\textquoteleft}Als ze verliest, dan...{\textquoteright}}{zal dat toch pas zijn nadat}{ze heeft gewonnen}\\

\haiku{Ze zocht daar in en.}{op dat ogenblik stond ze met}{een gekromde rug}\\

\haiku{Ze nam een flesje}{uit haar tas en daarna greep}{ze een glas water}\\

\haiku{{\textquoteleft}Wees maar niet bang dat.}{ik iets minder vriendelijks}{over haar zeggen zal}\\

\haiku{Hij is lichtzinnig,.}{en brutaal en nu is hij}{ook nog gaan drinken}\\

\haiku{{\textquoteright} - De kastelein laat,.}{zich niet van de wijs brengen}{hij kent zijn mensen}\\

\haiku{Terwijl hij daar op.}{straat loopt denkt hij nog even door}{over de kastelein}\\

\haiku{Daar zat het hijgend,,,.}{bijna berstend weer op zijn}{plaats aan de aorta}\\

\haiku{Dat praten met haar,.}{kon ook straks misschien was dat}{niet eens meer nodig}\\

\haiku{Haar ogen knipperden,.}{alsof ze bang was maar ze}{bleef roerloos liggen}\\

\haiku{Ik tastte in het,.}{donker langs de muur geen deur}{was er te vinden}\\

\haiku{Ze zaten beiden,,,.}{roerloos zij Alide dwars op}{de schoot van Berthe}\\

\haiku{Haar huid was blank, haar.}{ogen grijs en troebel en haar}{mond was breed gewelfd}\\

\haiku{Ik keek weer naar haar,.}{voorhoofd een schild des hemels}{of een schild des doods}\\

\haiku{{\textquoteleft}Ik zou iets willen,{\textquoteright}, {\textquoteleft}.}{hebben dat je gestolen}{had zei ikvan Peps}\\

\haiku{{\textquoteright} - King knikte nog een.}{keer en hield daarop het hoofd}{rouwend gebogen}\\

\haiku{De kleren geurden.}{kamferachtig en ook een}{beetje naar tabak}\\

\haiku{{\textquoteleft}Schenk jij maar koffie,,,.}{in met een likeurtje kijk}{alles staat hier klaar}\\

\haiku{Boven die ogen stond.}{de rimpel waarin King een}{teken had gezien}\\

\haiku{Ze moest natuurlijk.}{blijven bij de idioot en}{mij met rust laten}\\

\haiku{Ik zag iets aan haar,,.}{maar ik wist niet wat er was}{iets aan haar gezicht}\\

\haiku{nog \'e\'en zo'n sc\`ene,,.}{en waarom dan ook en het}{is afgelopen}\\

\haiku{{\textquoteright} - Ik leunde met mijn.}{armen op de tafel en}{boog me naar haar toe}\\

\haiku{Ja, handen spelen.}{soms een gekke rol in die}{verdomde erotiek}\\

\haiku{Dat ik dat niet deed,.}{lag aan zijn gedrag en aan}{mijn medelijden}\\

\haiku{King  smeet wat geld.}{neer voor dat ene glas en liep}{het Parthenon uit}\\

\haiku{{\textquoteleft}Of houd je niet van,,.}{me zeg het maar  eerlijk}{want dan ga ik weg}\\

\haiku{Ik lach me toch de,.}{tranen in de ogen twee of}{drie dagen daarna}\\

\haiku{Een denkbeeld had zich.}{in me vastgezet waar ik}{niet meer van loskwam}\\

\haiku{{\textquoteright} - Ik ging naar binnen.}{en sloeg onbehouwen de}{deur achter medicht}\\

\haiku{{\textquoteright} - Hij hoorde aan mijn:}{toon dat ik de spot met hem}{dreef en antwoordde}\\

\haiku{Toen zag ik dat hij.}{haar wou optillen en naar}{het hakblok brengen}\\

\haiku{En overal, door heel,,,.}{het huis knipte ze het licht}{aan overal overal}\\

\haiku{Bij God, een mantel,!}{de mantel van Juliette}{heeft ze niet gezien}\\

\haiku{Dat was geen King, en.}{dat was evenmin een Kosta die}{de moeite waard was}\\

\haiku{Die lag nog steeds te.}{sterven aan de liefde die}{ongeneeslijk wondt}\\

\haiku{{\textquoteright} - {\textquoteleft}En waarom ook weer,?}{omdat je bang was dat ik}{je niet hebben wou}\\

\haiku{dat was, als je het,,.}{nauw neemt nog w\'el zo gemeen}{jij was niet dronken}\\

\haiku{En dat gebaar was,.}{bijna al te grollig na}{al wat gebeurd was}\\

\haiku{Goddank, dacht ze, dat,...}{ik zo oud ben want hij is}{betoverend}\\

\haiku{Wat je ook doet, je.}{haalt die twee niet uit elkaar}{en jullie ook niet}\\

\haiku{Heel mijn leven van.}{de laatste maanden had tot}{dit besluit geleid}\\

\haiku{Het was alsof ze,.}{me dood had gewaand  maar}{nu hervonden had}\\

\haiku{{\textquoteright} zei ze toen we daar, {\textquoteleft};}{op die kamer kwamenals}{een zieke zwerver}\\

\haiku{Als dat dan moet, dan,.}{maar de ergste dan moet ik}{jou verloochenen}\\

\haiku{Zo kwam het dat zijn,.}{arm begon te trillen van}{ingehouden lust}\\

\haiku{{\textquoteright} zei hij met een diep, {\textquoteleft}?}{teder geluiddat ik dus}{jou verraden had}\\

\haiku{Toen wierp ze nog een.}{paar blokken op het vuur en}{liet hen weer alleen}\\

\haiku{Hoe maak ik dan nog?}{van Uit het leven van een}{speurder een roman}\\

\haiku{Ze boog het hoofd en.}{liep schuw naar de trap die naar}{de hutten voerde}\\

\haiku{Maar onmiddellijk.}{daarop liep ze resoluut}{op Virginie toe}\\

\haiku{Maar ondertussen.}{bette ze toch de ogen en}{kamde ze haar haar}\\

\haiku{maar ga ik, dan valt,.}{er misschien iets te winnen}{terug te winnen}\\

\haiku{Aller ogen keken.}{snel van Virginie weg en}{er viel een stilte}\\

\haiku{{\textquoteright} - Ze keek terzijde.}{en zag het mooie gezicht van}{Louise Riffeford}\\

\haiku{Ze botsten, elk in,.}{eigen gedachten verstrikt}{tegen elkaar op}\\

\haiku{Ze reageerde.}{niet en wachtte af wat hij}{verder zeggen zou}\\

\haiku{Niemand wist dat ze.}{daarbinnen was en geen stem}{die haar terugriep}\\

\haiku{Zo fantaseerde,.}{Louise Riffeford ver op}{de feiten vooruit}\\

\haiku{Beiden keken ze,.}{naar Louise Riffeford die}{op de drempel stond}\\

\haiku{Een kort ogenblik bleef.}{Louise Riffeford nu nog}{op de drempel staan}\\

\haiku{Ze danste met die:}{razend knappe officier}{Sterreveld en zei}\\

\haiku{Toen schreef ik alles:}{wat hij tegen me gezegd}{had op en ik dacht}\\

\haiku{Een enkele blik.}{op haar was voldoende om}{dat vast te stellen}\\

\haiku{Nu neemt zo'n jongen,.}{een meisje en neem hem dat}{dan maar eens kwalijk}\\

\haiku{Met heimelijke.}{afkeuring keek hij naar de}{paren die swingden}\\

\haiku{Eerst voelde hij de.}{neiging om op te staan en}{naar hem toe te gaan}\\

\haiku{Zijn enige succes.}{was dat de buitenwereld}{hem dat niet aanzag}\\

\haiku{{\textquoteright} - {\textquoteleft}Kijk, daar is Louise,.}{Riffeford en alsof het}{haar eerste bal is}\\

\haiku{{\textquoteleft}Als dat niet langer.}{duurt dan een gesprek heb ik}{daar niets op tegen}\\

\haiku{U moest begrijpen.}{dat ze te stom waren om}{u G.G. te maken}\\

\haiku{Overigens, het moest,.}{een visioen zijn geweest}{het kon niet anders}\\

\haiku{{\textquoteright} - Hij hield op en keek.}{weer naar het trotse gezicht}{in de portretlijst}\\

\haiku{{\textquoteleft}Het lijkt me dat ik,.}{de moed daartoe moet hebben}{wat daar ook van komt}\\

\haiku{Stormachtig was het,.}{maar van een storm kon je toch}{nog lang niet spreken}\\

\haiku{En ik heb zo goed,,.}{als nooit slaap of beter ik}{slaap zo goed als nooit}\\

\haiku{Daarop scheurde hij:}{de bladzijden zorgvuldig}{in snippers en zei}\\

\haiku{Weten is iets wat,.}{je alleen doet en dat is}{ook maar het beste}\\

\haiku{Maar later denk je}{dat er in elke muurkast}{een kerel zit die}\\

\haiku{{\textquoteright} - Ze probeerde ook,.}{nog het elektrische licht maar}{dat deed het niet meer}\\

\haiku{Hij, de Alziende,.}{keek neer op alles wat op}{aarde geschiedde}\\

\haiku{Het bewoog om haar.}{heen als een  werveling}{van zwarte sluiers}\\

\haiku{Maar dat viel niemand,.}{op God niet en de speelse}{waterdieren niet}\\

\haiku{Louise Riffeford.}{had allang begrepen dat}{ze verloren was}\\

\haiku{Hij keek neer op de.}{plaats waar de Kruisvaarder ten}{onder was gegaan}\\

\haiku{Haar moeder liet een.}{verdrietig afkeurende}{blik op haar rusten}\\

\haiku{{\textquoteleft}Egbert is hier voor,,.}{je geweest je bent net te}{laat hij is net weg}\\

\haiku{Een golf van schaamte,.}{steeg in haar op schaamte en}{verontwaardiging}\\

\haiku{De kapitein van.}{het schip was opgestaan en}{kwam haar tegemoet}\\

\subsection{Uit: Droom in oorlogstijd}

\haiku{Het is helemaal.}{niet goed als Hansie daarmee}{vriendinnetje wordt}\\

\haiku{Maar ik, ik moet dan:}{nog geloven in mezelf}{en in mijn toekomst}\\

\haiku{dat leven van mij.}{is een tantalusbeker}{en ik stik van dorst}\\

\haiku{Straks, thuis, dacht Dolf, hoor,,.}{ik dat nog en morgen nog}{en overmorgen nog}\\

\haiku{{\textquoteright} De blik onder zijn,.}{ogen werd merkwaardig donker}{of die huilen ging}\\

\haiku{En je kan je niet.}{voorstellen hoe prachtig en}{lief ik die bok vond}\\

\haiku{Toen gebeurde er.}{plotseling zoiets als een}{verlossend wonder}\\

\haiku{{\textquoteright} - Toen greep een van de.}{mannen me beet en zette}{me op zijn schouder}\\

\haiku{Ze luisterde, ze,,.}{zocht en kon het niet verstaan}{niet achterhalen}\\

\haiku{Van vijf zes kanten.}{waren de helhonden op}{haar losgebarsten}\\

\haiku{{\textquoteleft}Moeder, ik geloof.}{dat Koba daar komt met haar}{vader en moeder}\\

\haiku{Ze draagt het mee in.}{haar hart terwijl ze kachels}{doet en gangen schrobt}\\

\haiku{En er zou nog meer.}{gezegd moeten worden in}{verband met die bloem}\\

\haiku{Nu stond ze daar nog.}{steeds te lachen voor die drie}{geliefde mensen}\\

\haiku{De zin des levens, -:}{dooreengelopen Mijn hart}{wijd-open Ze zegt}\\

\haiku{Het was een keurig.}{mantelpak dat ze al voor}{het derde jaar droeg}\\

\haiku{{\textquoteright} - De vrouw bleef wachten.}{totdat een stem haar verzocht}{boven te komen}\\

\haiku{De moeder trok met:}{een gebaar van onmacht de}{schouders op en zei}\\

\haiku{Een mens wordt ook zo.}{gemakkelijk zielig en}{beklagenswaardig}\\

\haiku{En daarop knikte...}{de vrouw en gleed haar blik weer}{schichtig langs haar heen}\\

\haiku{Zijzelf bijvoorbeeld.}{had vroeger  gedacht dat}{ze studeren zou}\\

\haiku{{\textquoteright} - De moeder en haar.}{dochter stonden voor het raam}{en keken haar na}\\

\haiku{{\textquoteleft}Nu sluit ik hierbij.}{in twee novellen die ik}{allang had liggen}\\

\haiku{Ze lijken eerder.}{in opdracht van dag- of}{weekblad geschreven}\\

\subsection{Uit: Eenzaam avontuur}

\haiku{Ze zei me dat ze.}{nooit geloofd zou hebben dat}{te durven zeggen}\\

\haiku{Haar ogen waren grijs,.}{haar mond was rijp en toch weer}{zacht als van een kind}\\

\haiku{{\textquoteright} - Ze lachte even in,.}{mijn ogen maar haar hand in de}{mijne bleef passief}\\

\haiku{{\textquoteleft}Het blauwe paleis,.}{heeft me zo triest gemaakt ik}{weet niets vrolijks meer}\\

\haiku{Toen het zomer werd.}{gingen we een paar maanden}{in een bos wonen}\\

\haiku{Mijn dieptelagen;}{zijn keus van dezelfde stijl}{als mijn fa\c{c}ade}\\

\haiku{Het was de eerste.}{keer dat zij hem toestond bij}{haar aan te komen}\\

\haiku{Gisteravond, in die,.}{dancing meende hij succes}{geboekt te hebben}\\

\haiku{Stel nu dat die echo...}{eens verschald zou zijn en ik}{haar niet zou weerzien}\\

\haiku{En deze middag,,.}{fantaseerde ik wist King}{dat ze niet thuis was}\\

\haiku{En hij wist zelfs waar.}{ze heen was en hoe laat ze}{zou terugkomen}\\

\haiku{In de hand hield hij.}{de nagemaakte sleutels}{van de deur gereed}\\

\haiku{{\textquoteright} - Hij keek me lang en:}{stil aan en antwoordde toen}{bijna fluisterend}\\

\haiku{Geen wonder dat hij,.}{bang werd ik zag er uit als}{een krankzinnige}\\

\haiku{Het meningloze,,;}{meisje Annie keek even op}{en glimlachte flauw}\\

\haiku{daarom... ik zou het,.}{goed maken met hem hij zou}{h\'a\'ar nooit meer moeten}\\

\haiku{Neen, Berthe was, waar,.}{het de erotiek betrof niet}{erg prinsesselijk}\\

\haiku{Maar hadden ze ook...}{ooit gedacht dat de vorstin}{zelf met een lakei}\\

\haiku{Yolande keerde.}{zich van de \'etalage af}{en keek Alide aan}\\

\haiku{blank haar gezicht en,.}{ver haar ogen alleen die}{glimlach was dichtbij}\\

\haiku{De palmen waren,.}{breed de vingers kort en sterk}{met sterke nagels}\\

\haiku{Dat was alsof je.}{op een ziel speelde als op}{een edel instrument}\\

\haiku{En Juliette stond.}{voor de spiegel en tooide}{zich met sieraden}\\

\haiku{Yolande stootte,.}{uitdagend Berthe uit wat}{haar voldoening gaf}\\

\haiku{ze hebben een ziel,,?}{dat is wel zeker maar wie}{begrijpt dat overdag}\\

\haiku{Annie's hand was kil.}{en Berthes hand was reddend}{in een vaste greep}\\

\haiku{{\textquoteleft}God ja,{\textquoteright} zei ze, {\textquoteleft}die,;}{heb ik nog en daarom ging}{ik maar eens lopen}\\

\haiku{Daardoor, Alide, zat,.}{ik aan die slootkant daardoor}{wist ik het niet meer}\\

\haiku{Was het een boze,,,?}{droom die Peps toe zeg dat het}{een boze droom was}\\

\haiku{in 't gras alsof,.}{ik zo zou opspringen de}{fiets grijpen en gaan}\\

\haiku{Je handen hebben,.}{me gestreeld en behoed je}{schoot is kuis en koel}\\

\haiku{En dan ga ik dat,,,.}{kleuren cynisch gewaagd in}{schreeuwende verven}\\

\haiku{Van bij het venster,.}{klonk het snuiven van huilen}{geluidloos huilen}\\

\haiku{Hij zweette als een.}{otter en hij rolde als}{een vod de trap af}\\

\haiku{Maar Kosta bleef het druk.}{hebben met roken en keek}{haar vooral niet aan}\\

\haiku{Maar ik ben jong en.}{heb een lichaam dat zelfs nooit}{nog aan de beurt kwam}\\

\haiku{En gaf ze daarvoor,,?}{ook om dat te kunnen doen}{iets van zichzelf prijs}\\

\haiku{Daar zette hij een,.}{vast halfuurtje voor als hij}{zich stond te scheren}\\

\haiku{Als 't nodig was.}{deed hij zo braaf en dweepziek}{als een heilsoldaat}\\

\haiku{Er overkomt je bij.}{haar in een korte spanne}{tijds een massa goeds}\\

\haiku{King zou toch kunnen,,.}{wachten ondanks dat briefje}{of terugkomen}\\

\haiku{Hij wou die vervloekt,.}{begerenswaardige jas}{hebben en voorgoed}\\

\haiku{Hij dacht dat hij zijn,.}{drift beheerste maar zijn stem was}{fluisterend en heet}\\

\haiku{En dat gebeurde.}{toen we de koffers pakten}{om te vertrekken}\\

\haiku{Ze deed het net zo.}{zorgzaam en zo liefdevol}{als ooit tevoren}\\

\haiku{Maar ik kon deze,.}{taak niet van haar overnemen}{dat was al te hard}\\

\haiku{Ik zou misschien zo'n,,.}{zelfde aandacht winnen dacht}{ik al pratende}\\

\haiku{{\textquoteright} zouden de tranen.}{niet te stelpen tussen mijn}{vingers door vloeien}\\

\haiku{Je schrok, zette de.}{borden uit je handen en}{brak in snikken uit}\\

\haiku{Maar waar was dat, ja,.}{in Mon Repos had ik haar al}{een keer geslagen}\\

\haiku{Alide kwijnde weg,.}{in West-Europa nu}{ik er niet meer was}\\

\haiku{Haar wang lag aan haar '.}{hand en zo keek ze de kant}{vant venster uit}\\

\haiku{Zijn hart bonsde zo.}{zwaar dat hij verwonderd was}{het niet te horen}\\

\haiku{{\textquoteright} - En ze spreidde als.}{een deken de kamerjas}{over hen beiden uit}\\

\haiku{Ze was een tors van.}{edel marmer en een duister}{bloedwarm vrouwehoofd}\\

\haiku{Zijn spel groeit hem hier ',.}{bovent hoofd hij is zijn}{eigen spelbreker}\\

\haiku{De angst dat hij die,.}{droom niet houden kan zit hem}{al op de hielen}\\

\haiku{Juliette behoeft.}{dan ook vanzelf niet meer in}{de gevangenis}\\

\haiku{Toch, juist die avond van.}{King's avontuur was er diep in}{haar weer iets gaande}\\

\haiku{{\textquoteright} Maar daarop strekte,,.}{de hospita bezwerend}{sussend een hand uit}\\

\haiku{Maar toch herhaalde,;}{ze omdat het King maar was}{die haar gekrenkt had}\\

\haiku{Ze drukte haar borst.}{tegen me aan en hield het}{hoofd wat achterover}\\

\haiku{{\textquoteleft}Een vrouw kan kopen.}{voor zichzelf alsof ze haar}{eigen minnaar is}\\

\haiku{Bijna herfst was het,,,,.}{een vroege herfstmiddag wijd}{zonnig zorgeloos}\\

\haiku{Ik zag de schutting.}{terug van de tuin waarin}{ik als kind speelde}\\

\haiku{Terwijl het toch om,.}{hem begonnen was scheen ze}{hem glad vergeten}\\

\haiku{Ze glimlachte, ze.}{sloot de ogen en bracht zo haar}{mond op zijn gezicht}\\

\haiku{En elke keer dat.}{zij haar tanden poetste moest}{ze daaraan denken}\\

\haiku{{\textquoteleft}Maar jij, heb jij haar,?}{nooit gemist al liet je dat}{aan mij niet merken}\\

\haiku{Ze streelde met haar,.}{sterke vingers wonderlijk}{zacht zijn wang zijn haar}\\

\haiku{- Alide nestelde.}{zich op de divan en ging}{liggen nadenken}\\

\haiku{Ze schikte een paar '.}{kussens ondert hoofd en}{staarde voor zich uit}\\

\haiku{{\textquoteright} - Ze wist het, niets kon.}{zijn visie op haar storen}{of beschadigen}\\

\haiku{En wat was er van?}{hem geworden sinds ze hem}{niet meer beschermde}\\

\haiku{Was deze vrouw  ?}{dan meer dan wat ze in dit}{genre had ontmoet}\\

\haiku{Het was een rijke,.}{blik die van het lokkende}{speelzieke wijfje}\\

\haiku{Ze zagen haar het,.}{postkantoor verlaten en}{zij liep spitsroeden}\\

\haiku{Niettemin schreed ze.}{rustig en soepel voort naast}{de gebrilde Peps}\\

\haiku{Maar het klonk in haar,.}{op als hol geluid een stem}{van gene zijde}\\

\haiku{Hoe na{\"\i}ef om zo.}{hartstochtelijk te hechten}{aan haar instemming}\\

\haiku{Ze keek nog steeds heel.}{ernstig en voelde een soort}{innerlijke pijn}\\

\haiku{Peps kwam die kamer.}{binnen en ontdekte zijn}{Alide op dat bed}\\

\haiku{Haar minnaar vocht een,;}{gevecht tegen twee tranen}{maar in elk oog \'e\'en}\\

\haiku{Ze voelde hoe zijn.}{hand die op de hare lag}{begon te zweten}\\

\haiku{{\textquoteright} Die lag soms uren op.}{de divan met een paar ogen}{waar je bang van werd}\\

\haiku{Ik vind hem trouwens,?}{ook veranderd de laatste}{tijd merk jij dat niet}\\

\haiku{{\textquoteleft}Nou, die meiden daar,,.}{die schudden je wel uit in}{alle opzichten}\\

\haiku{Die kennen kunsten,,.}{nou daar trekken ze het merg}{mee uit je botten}\\

\haiku{{\textquoteright} - Maar toch was er nog.}{iets anders wat haar dwars zat}{en wat ze niet zei}\\

\haiku{Niet dat haar dat zo,.}{hinderde want eigenlijk}{gaf ze er niets om}\\

\haiku{Kijk, dat begreep ze,?}{van zichzelf niet was ze dus}{zo onredelijk}\\

\haiku{Die had de dag weer,,;}{in die winkel achter die}{kassa doorgebracht}\\

\haiku{Maar toen vluchtte ze,,,.}{de kamer uit de trap op}{naar haar kamertje}\\

\haiku{Ze ging naar binnen,.}{sloeg de deur achter zich dicht}{en liep naar boven}\\

\haiku{{\textquoteleft}Natuurlijk weet ik,...{\textquoteright}.}{het maar En een weigering}{in ogen en gebaar}\\

\haiku{Anne zelf, die was.}{daarbij ternauwernood van}{werkelijk belang}\\

\haiku{Zo sterk had hij zich.}{dus ge{\"\i}dentificeerd}{met zijn eenzaamheid}\\

\haiku{Hij hief alleen de.}{hand op en wreef langzaam over}{zijn vermoeide ogen}\\

\haiku{Daarna greep hij naar.}{een pakje sigaretten}{en nam daar een uit}\\

\haiku{, maar het lijken me.}{begrijpelijke feiten}{die je daar vertelt}\\

\haiku{{\textquotedblright} - En daarop keerde '.}{hij zich van me af en ging}{voort venster staan}\\

\haiku{Toen ik de kamer.}{uit liep hield hij het hoofd weer}{van me afgekeerd}\\

\haiku{Toen kwam de blik tot.}{hem via de brilleglazen}{en hij werd gezien}\\

\haiku{{\textquoteleft}Als ze verliest, dan...{\textquoteright}}{zal dat toch pas zijn nadat}{ze heeft gewonnen}\\

\haiku{ik moet de stumper.}{die hier op de divan ligt}{zien te vergeten}\\

\haiku{Ze zocht daar in en.}{op dat ogenblik stond ze met}{een gekromde rug}\\

\haiku{Ze nam een flesje}{uit haar tas en daarna greep}{ze een glas water}\\

\haiku{{\textquoteleft}Wees maar niet bang dat.}{ik iets minder vriendelijks}{over haar zeggen zal}\\

\haiku{Hij is lichtzinnig,.}{en brutaal en nu is hij}{ook nog gaan drinken}\\

\haiku{{\textquoteright} - De kastelein laat,.}{zich niet van de wijs brengen}{hij kent zijn mensen}\\

\haiku{Terwijl hij daar op.}{straat loopt denkt hij nog even door}{over de kastelein}\\

\haiku{Daar zat het hijgend,,,.}{bijna berstend weer op zijn}{plaats aan de aorta}\\

\haiku{Dat praten met haar,.}{kon ook straks misschien was dat}{niet eens meer nodig}\\

\haiku{Haar ogen knipperden,.}{alsof ze bang was maar ze}{bleef roerloos liggen}\\

\haiku{Ik tastte in het,.}{donker langs de muur geen deur}{was er te vinden}\\

\haiku{Ze zaten beiden,,,.}{roerloos zij Alide dwars op}{de schoot van Berthe}\\

\haiku{Ik keek weer naar haar,.}{voorhoofd een schild des hemels}{of een schild des doods}\\

\haiku{{\textquoteleft}Ik zou iets willen,{\textquoteright}, {\textquoteleft}.}{hebben dat je gestolen}{had zei ikvan Peps}\\

\haiku{{\textquoteright} - King knikte nog een.}{keer en hield daarop het hoofd}{rouwend gebogen}\\

\haiku{De kleren geurden.}{kamferachtig en ook een}{beetje naar tabak}\\

\haiku{{\textquoteleft}Schenk jij maar koffie,,,.}{in met een likeurtje kijk}{alles staat hier klaar}\\

\haiku{Boven die ogen stond.}{de rimpel waarin King een}{teken had gezien}\\

\haiku{Ze moest natuurlijk.}{blijven bij de idioot en}{mij met rust laten}\\

\haiku{Ik zag iets aan haar,,.}{maar ik wist niet wat er was}{iets aan haar gezicht}\\

\haiku{nog \'e\'en zo'n sc\`ene,,.}{en waarom dan ook en het}{is afgelopen}\\

\haiku{{\textquoteright} - Ik leunde met mijn.}{armen op de tafel en}{boog me naar haar toe}\\

\haiku{Ja, handen spelen.}{soms een gekke rol in die}{verdomde erotiek}\\

\haiku{Dat ik dat niet deed,.}{lag aan zijn gedrag en aan}{mijn medelijden}\\

\haiku{{\textquoteleft}Ik  kan willen,.}{wat ik ook maar wil zolang}{ik maar alleen ben}\\

\haiku{Ik lach me toch de,.}{tranen in de ogen twee of}{drie dagen daarna}\\

\haiku{Een denkbeeld had zich.}{in me vastgezet waar ik}{niet meer van los kwam}\\

\haiku{{\textquoteright} - Ik ging naar binnen.}{en sloeg onbehouwen de}{deur achter me dicht}\\

\haiku{{\textquoteright} - Hij hoorde aan mijn:}{toon dat ik de spot met hem}{dreef en antwoordde}\\

\haiku{Toen zag ik dat hij.}{haar wou optillen en naar}{het hakblok brengen}\\

\haiku{En overal, door heel,,,.}{het huis knipte ze het licht}{aan overal overal}\\

\haiku{Bij God, een mantel,!}{de mantel van Juliette}{heeft ze niet gezien}\\

\haiku{Dat was geen King, en.}{dat was evenmin een Kosta die}{de moeite waard was}\\

\haiku{En Juliette, thuis,.}{gekomen had natuurlijk}{\'o\'ok haar bed gezocht}\\

\haiku{Die lag nog steeds te.}{sterven aan de liefde die}{ongeneeslijk wondt}\\

\haiku{{\textquoteright} - {\textquoteleft}En waarom ook weer,?}{omdat je bang was dat ik}{je niet hebben wou}\\

\haiku{dat was, als je het,,.}{nauw neemt nog w\'el zo gemeen}{j{\'\i}j was niet dronken}\\

\haiku{En dat gebaar was,.}{bijna al te grollig na}{al wat gebeurd was}\\

\haiku{Goddank, dacht ze, dat,...}{ik zo oud ben want hij is}{betoverend}\\

\haiku{Wat je ook doet, je.}{haalt die twee niet uit elkaar}{en jullie ook niet}\\

\haiku{En na zes borrels.}{stond ik op en liep ik naar}{de telefooncel}\\

\haiku{Ik was in vier, vijf'.}{sprongen in  t portaal}{en trok de deur open}\\

\haiku{Heel mijn leven van.}{de laatste maanden had tot}{dit besluit geleid}\\

\haiku{Het was alsof ze,.}{me dood had gewaand maar nu}{hervonden had}\\

\haiku{{\textquoteright} zei ze toen we daar, {\textquoteleft};}{op die kamer kwamenals}{een zieke zwerver}\\

\haiku{Als dat dan moet, dan,.}{maar de ergste dan moet ik}{jou verloochenen}\\

\haiku{Zo kwam het dat zijn,.}{arm begon te trillen van}{ingehouden lust}\\

\haiku{{\textquoteright} zei hij met een diep, {\textquoteleft}?}{teder geluiddat ik dus}{jou verraden had}\\

\haiku{Toen wierp ze nog een.}{paar blokken op het vuur en}{liet hen weer alleen}\\

\subsection{Uit: Fragmentarisch. Nagelaten proza}

\haiku{een prachtige \'en.}{afschuwelijke droom over}{tuinen en holen}\\

\haiku{King moet Juliette,,.}{die van een lustmoord verdacht}{wordt ontmaskeren}\\

\haiku{{\textquoteleft}Ik heb het leven,{\textquoteright}.}{in zijn essentie ontmoet}{en beleefd denkt hij}\\

\haiku{het conflict dat zich {\textquoteleft}{\textquoteright} {\textquoteleft}{\textquoteright}.}{kan voordoen tussenmuze}{enmenselijkheid}\\

\haiku{Alleen Erica Hart,,.}{begrijpt hem sterker nog zij}{is het met hem eens}\\

\haiku{- Ze begreep dat hij,.}{dronken was en gaf hem zijn}{zin ze keek ernstig}\\

\haiku{Hij tastte nerveus.}{naar zijn zakdoek en drukte}{die tegen de mond}\\

\haiku{Niettemin, de tand,.}{moest dienen als uitgangspunt}{hoe kon hij anders}\\

\haiku{Hij ontledigde:}{zijn mond van het teveel aan}{speeksel en begon}\\

\haiku{Ze informeerde,.}{nadrukkelijk naar Adriaan}{en naar Elsa zelf}\\

\haiku{Hoe laat begint ook? -,.}{maar dat bezoekuur Om twee}{uur antwoordde ze}\\

\haiku{Ze kon desondanks.}{een kind zijn dat in hem een}{volwassene zag}\\

\haiku{Eerst dacht hij dat dit.}{een droombeeld was dat tot zijn}{bewustzijn doordrong}\\

\haiku{Lieve Emmie, ik.}{heb vandaag rustig op mijn}{kamer gezeten}\\

\haiku{{\textquoteleft}dan wacht ik maar{\textquoteright} - de:}{lezer kan niet nalaten}{zich af te vragen}\\

\haiku{Het was, ik zal het,:}{maar bekennen moeilijk om}{daar niet te denken}\\

\haiku{In het telegram,,:}{dat hij opgaf maar dat niet}{verstuurd werd schreef hij}\\

\haiku{Ik loop het dorp uit,,.}{langs kleine wegen tot ik}{bij een groot bos kom}\\

\haiku{Om u de waarheid,.}{te zeggen ik weet niet veel}{van schilderkunst af}\\

\haiku{Omdat,{\textquoteright} zei ze, {\textquoteleft}het,.}{juist zo leuk is als je voor}{iets op moet passen}\\

\haiku{{\textquoteright} {\textquoteleft}Daarom bent u wel.}{haar dochter en toch niet de}{mijne bij voorbeeld}\\

\haiku{Hij sloeg zijn benen.}{over elkaar en vouwde zijn}{handen over zijn buik}\\

\haiku{Ze zei nog veel meer -}{van zulke dingen toen ze}{we stonden juistvoor}\\

\haiku{Weineen, het moest al.}{een miljonair zijn die dit}{zou willen hebben}\\

\haiku{{\textquoteright} Hoofdstuk II [Anna]:}{Blaman De hospes tikte}{op mijn deur en riep}\\

\haiku{Zijn mening over mij.}{was dat hij me verre van}{indrukwekkend vond}\\

\haiku{Al houd ik van je,,.}{lieve Jan ik besta toch}{ook nog wel alleen}\\

\haiku{Wat daar gebeurde,.}{of nog gebeurt ze willen}{het me niet zeggen}\\

\haiku{Maar dan heeft nu dat.}{huis misschien zijn bekoring}{voor u verloren}\\

\haiku{Zelfs al had ik zo'n,.}{relatie dan zou ik het}{je nog niet zeggen}\\

\haiku{Maar toen Moeke het,.}{bereid heeft heeft zij er zelf}{niets van willen eten}\\

\haiku{Mama is dood, en.}{jij en Anne-Marie}{moeten heel stil zijn}\\

\haiku{wie het was - en ik,}{doe het expres niet mijnheer}{want een patser ben}\\

\haiku{En Victorine,.}{die denkt dat hij mij misschien}{bij zich zou houden}\\

\haiku{Ik zocht overal, ja.}{zelfs onder het bed en in}{de nog lege kast}\\

\haiku{Ik heb daar een film.}{gezien waarin een man door}{zijn vrouw werd vermoord}\\

\haiku{Ik herinnerde.}{mij dat ik ontbijt en lunch}{had overgeslagen}\\

\haiku{Hij legde rustig:}{een hand in haar gebogen}{hals waarvan hij dacht}\\

\haiku{Ze wist dat ze zich,.}{had misdragen en had toch}{niet anders gekund}\\

\haiku{De een fluisterde,}{je met een verzuurde adem}{liefdewoorden toe}\\

\haiku{En ondertussen.}{verloor hij meer en meer greep}{op Victorine}\\

\haiku{Als ik vrijdagavond,;}{laat komen zou vond ik haar}{misschien al in bed}\\

\haiku{Hoeveel bloemen denk '?}{je dat er in \'e\'en baan van}{t behang zitten}\\

\haiku{Victorine en.}{haar zuster bleken veel te}{kunnen begrijpen}\\

\haiku{- Schneider liet  me.}{de kamers zien en Willem}{kwam achter ons aan}\\

\haiku{De troosteloosheid.}{in het vertrek voorzag haar}{van een aureool}\\

\haiku{Hij was niet alleen,.}{geprikkeld maar innerlijk}{behoorlijk overstuur}\\

\haiku{Immers, de mens deelt.}{deze appreciatie}{met geen enkel dier}\\

\haiku{{\textquoteright} Haar vriendin, Annie,,:}{houdt van Hilda omdat ze}{zo menselijk is}\\

\haiku{{\textquoteleft}echt een mens, een mens{\textquoteright},.}{vulgair van echtheid mede}{dank zij haar tandpijn}\\

\haiku{Te groot zijn ook de.}{blanke tanden van Sara}{Obreen in Vrouw en vriend}\\

\haiku{Maar zelfs wanneer zij,.}{haar gebit verloor zou hij}{nog van haar houden}\\

\haiku{In de zich daarna}{ontspinnende discussie}{uit een criticus}\\

\haiku{Herinneren we, ({\textquoteleft}{\textquoteright});}{ons Christiaan die zijn hart}{aan de dood schenktSchets}\\

\haiku{Dat de kaartlegster,;}{hem de dood voorzegde raakt}{hem eigenlijk niet}\\

\haiku{Wildhaas antwoordt met:}{een ook heden ten dage}{nog modern praatje}\\

\haiku{hij lijdt onder een:}{existentieel failliet dat}{alle mensen treft}\\

\haiku{Denkend aan Stella,,:}{krijgt hij opnieuw en voor het}{laatst een hartaanval}\\

\haiku{Het is niet goed dat{\textquoteright} (,)}{de mens alleen zijblz. 148}{vgl. Genesis 2:18}\\

\haiku{En daarom krijg ik,{\textquoteright} ().}{nu die hartkwaal die helpt me}{er wel uitblz. 63}\\

\haiku{Waarom moet je die?}{twee met alle geweld op}{mekaar betrekken}\\

\haiku{Anna Blaman, aan,;}{wie het verzoek gericht was}{reageerde prompt}\\

\haiku{Anna Blaman is.}{haar hele leven zwak en}{ziekelijk geweest}\\

\haiku{Zij beschreef deze ();}{ziekteperiode in}{De verliezers1960}\\

\haiku{En als ik nu eens,?}{een stap vooruit deed wat zou}{er dan gebeuren}\\

\haiku{Het bezorgde je,.}{kippevel het dreef je de}{tranen naar de ogen}\\

\haiku{{\textquoteleft}Een reisverslag van{\textquoteright},,,-.}{Anna Blaman De Gids 135e}{jaargang blz. 6066}\\

\haiku{7Jonas' naam verwijst,.}{wellicht naar Anna Blamans}{doopnaam Johanna}\\

\haiku{9In een passage,:}{waarin Jonas Marie voor}{het eerst een kus geeft}\\

\haiku{79{\textquoteleft}Alleen heiligen{\textquoteright} (,,.}{kunnen op water lopen}{blz. 159 160 vgl. Matth}\\

\haiku{{\textquoteleft}Misschien heb je al,.}{iets gehoord maar ik ben in}{Amsterdam gestrand}\\

\haiku{100Anna Blaman over (),.}{zichzelf en anderenMijn}{eigen zelf blz. 101}\\

\subsection{Uit: De kruisvaarder en Ontmoeting met Selma}

\haiku{Ze boog het hoofd en.}{liep schuw naar de trap die naar}{de hutten voerde}\\

\haiku{Maar onmiddellijk.}{daarop liep ze resoluut}{op Virginie toe}\\

\haiku{Maar ondertussen.}{bette ze toch de ogen en}{kamde ze haar haar}\\

\haiku{maar ga ik, dan valt,.}{er misschien iets te winnen}{terug te winnen}\\

\haiku{Aller ogen keken.}{snel van Virginie weg en}{er viel een stilte}\\

\haiku{{\textquoteright} - Ze keek terzijde.}{en zag het mooie gezicht van}{Louise Riffeford}\\

\haiku{Ze botsten, elk in,.}{eigen gedachten verstrikt}{tegen elkaar op}\\

\haiku{Ze reageerde.}{niet en wachtte af wat hij}{verder zeggen zou}\\

\haiku{Niemand wist dat ze.}{daarbinnen was en geen stem}{die haar terug riep}\\

\haiku{Zo fantaseerde,.}{Louise Riffeford ver op}{de feiten vooruit}\\

\haiku{Beiden keken ze,.}{naar Louise Riffeford die}{op de drempel stond}\\

\haiku{Een kort ogenblik bleef.}{Louise Riffeford nu nog}{op de drempel staan}\\

\haiku{Ze danste met die:}{razend knappe officier}{Sterreveld en zei}\\

\haiku{Toen schreef ik alles:}{wat hij tegen me gezegd}{had op en ik dacht}\\

\haiku{Een enkele blik.}{op haar was voldoende om}{dat vast te stellen}\\

\haiku{Nu neemt zo'n jongen,.}{een meisje en neem hem dat}{dan maar eens kwalijk}\\

\haiku{Met heimelijke.}{afkeuring keek hij naar de}{paren die swingden}\\

\haiku{Eerst voelde hij de.}{neiging om op te staan en}{naar hem toe te gaan}\\

\haiku{Zijn enige succes.}{was dat de buitenwereld}{hem dat niet aanzag}\\

\haiku{{\textquoteright} - {\textquoteleft}Kijk, daar is Louise,.}{Riffeford en alsof het}{haar eerste bal is}\\

\haiku{{\textquoteleft}Als dat niet langer.}{duurt dan een gesprek heb ik}{daar niets op tegen}\\

\haiku{Overigens, het moest,.}{een visioen zijn geweest}{het kon niet anders}\\

\haiku{{\textquoteright} - Hij hield op en keek.}{weer naar het trotse gezicht}{in de portretlijst}\\

\haiku{{\textquoteleft}Het lijkt me dat ik,.}{de moed daartoe moet hebben}{wat daar ook van komt}\\

\haiku{Stormachtig was het,.}{maar van een storm kon je toch}{nog lang niet spreken}\\

\haiku{En ik heb zo goed,,.}{als nooit slaap of beter ik}{slaap zo goed als nooit}\\

\haiku{Daarop scheurde hij:}{de bladzijden zorgvuldig}{in snippers en zei}\\

\haiku{Weten is iets wat,.}{je alleen doet en dat is}{ook maar het beste}\\

\haiku{{\textquoteright} - Ze probeerde ook,.}{nog het elektrische licht maar}{dat deed het niet meer}\\

\haiku{Hij, de Alziende,.}{keek neer op alles wat op}{aarde geschiedde}\\

\haiku{Maar dat viel niemand,.}{op God niet en de speelse}{waterdieren niet}\\

\haiku{Louise Riffeford.}{had allang begrepen dat}{ze verloren was}\\

\haiku{Hij keek neer op de.}{plaats waar de Kruisvaarder ten}{onder was gegaan}\\

\haiku{Haar moeder liet een.}{verdrietig afkeurende}{blik op haar rusten}\\

\haiku{{\textquoteleft}Egbert is hier voor,,.}{je geweest je bent net te}{laat hij is net weg}\\

\haiku{{\textquoteright} {\textquoteleft}Stil maar, stil  maar,,.}{moeder nu heb ik iemand}{het leven gekost}\\

\haiku{Een golf van schaamte,.}{steeg in haar op schaamte en}{verontwaardiging}\\

\haiku{De kapitein van.}{het schip was opgestaan en}{kwam haar tegemoet}\\

\haiku{Dat ging natuurlijk.}{over nicht Marie die een breed}{pad bewandelde}\\

\haiku{een lach echter die:}{zich bevrijdde tot warme}{vreugde toen ze zei}\\

\haiku{Ze ging een nichtje:}{van de boot halen en vond}{er Jeanne Brondag}\\

\haiku{Er was beslist in.}{haar moraal iets dat daarmee}{correspondeerde}\\

\haiku{{\textquoteright} zei Jeanne zwak, {\textquoteleft}.}{dat er een vriendschap tussen}{ons geboren was}\\

\haiku{Claartje van Dort zat,,.}{tegenover haar zwijgend met}{neergeslagen ogen}\\

\haiku{Ze had niets gezegd,.}{alleen maar koud geglimlacht}{en was weggegaan}\\

\haiku{{\textquoteleft}Ik vind het toch zo,{\textquoteright}.}{heerlijk dat je bij me bent}{zei Selma spontaan}\\

\haiku{Ze wist nog niet of.}{ze dat weten kon en toch}{van Selma houden}\\

\haiku{Daar zoende hij me.}{nog alsof hij dat die nacht}{nog niet gedaan had}\\

\haiku{En ik vroeg me af.}{of ik wel wijs gedaan had}{met dit huwelijk}\\

\haiku{Ze hervond Selma,,.}{slapend vredig met een mooie}{blos op het gezicht}\\

\haiku{zoals ik me dat -...}{altijd droomde zelfs al zou}{dat niet gebeuren}\\

\subsection{Uit: Ontmoeting met Selma}

\haiku{Dat ging natuurlijk.}{over nicht Marie die een breed}{pad bewandelde}\\

\haiku{een lach echter die:}{zich bevrijdde tot warme}{vreugde toen ze zei}\\

\haiku{Ze ging een nichtje:}{van de boot halen en vond}{er Jeanne Brondag}\\

\haiku{Er was beslist in.}{haar moraal iets dat daarmee}{correspondeerde}\\

\haiku{Claartje van Dort zat,,.}{tegenover haar zwijgend met}{neergeslagen ogen}\\

\haiku{Ze had niets gezegd,.}{alleen maar koud geglimlacht}{en was weggegaan}\\

\haiku{Ze glimlachte, streek...}{met een fijne hand over mijn}{haar en noemde Erik}\\

\haiku{Ze wist nog niet of.}{ze dat weten kon en toch}{van Selma houden}\\

\haiku{Daar zoende hij me.}{nog alsof hij dat die nacht}{nog niet gedaan had}\\

\haiku{Gert is goed, die houdt,}{zo eerlijk van me maar de}{nachten hier met}\\

\haiku{En ik vroeg me af.}{of ik wel wijs gedaan had}{met dit huwelijk}\\

\haiku{Ze hervond Selma,,,.}{slapend vredig met een mooie}{blos op het gezicht}\\

\haiku{zoals ik me dat -...}{altijd droomde zelfs al zou}{dat niet gebeuren}\\

\subsection{Uit: Op leven en dood}

\haiku{Hoewel ik doodmoe,.}{was kon ik niet besluiten}{om naar bed te gaan}\\

\haiku{De nette mensen}{hadden ze waarschijnlijk op}{de grond zien vallen}\\

\haiku{Ik voelde dat ik.}{haar hart niet trefzekerder}{had kunnen raken}\\

\haiku{Nu kan het wild langs.}{komen en het arme beest}{ruikt het niet eens meer}\\

\haiku{O Marie, als ik,!}{je toch eens vond als ik je}{toch eens tegenkwam}\\

\haiku{Die waarzeggerij,.}{heeft indruk op je gemaakt}{je gelooft erin}\\

\haiku{{\textquoteleft}Pas op, zeg dat niet,.}{te hard of ze gaat het er}{nog op aanleggen}\\

\haiku{dezelfde wanhoop,?}{vind maar magisch bezworen}{in een illusie}\\

\haiku{Haar mooie bezielde.}{ogen glinsterden alsof er}{vuur in water scheen}\\

\haiku{Ik nam een taxi naar,.}{huis terug want het was al}{ver over middernacht}\\

\haiku{Het had immers maar.}{een haar gescheeld of ze was}{mijn redding geweest}\\

\haiku{De vrouwen vormden;}{toen zeer gaarne clubs waaruit}{de man geweerd werd}\\

\haiku{Ze was jaloers, maar.}{ze exalteerde zich boven}{de jaloezie uit}\\

\haiku{{\textquoteright} En toen zette ik.}{ook dat portret weer terug}{en greep het laatste}\\

\haiku{Na die kus had ik,.}{dat ook kunnen doen maar er}{was iets veranderd}\\

\haiku{{\textquoteleft}Ik zou nooit beter.}{voor hem kunnen zijn dan ik}{de laatste tijd ben}\\

\haiku{Die hand kwam zwaar op.}{haar dij terecht en gleed af}{in haar vrouweschoot}\\

\haiku{net zolang tot de...}{schone schijn eraf sprong als}{glanslak van oudroest}\\

\haiku{Die onverwachte.}{bliksemsnelle beroering}{sloeg haar door het bloed}\\

\haiku{Dus,{\textquoteright} vroeg ik, {\textquoteleft}daaruit.}{kan ik niet afleiden of}{je iets om me geeft}\\

\haiku{En ik hoef niet eens.}{naar je te kijken om te}{weten hoe dat is}\\

\haiku{Ik voel er alles,.}{voor maar er is ook het een}{en ander tegen}\\

\haiku{Ik moest uitrusten:}{en ondertussen kalm en}{nuchter nadenken}\\

\haiku{{\textquoteright} zei ze, {\textquoteleft}Paul zei 'ga,.}{hem opvrolijken ik geef}{je carte blanche}\\

\haiku{Hoe het ook zij, om:}{te beginnen zou ik me}{daar maar aan houden}\\

\haiku{Feitelijk was hij.}{het dus die haar met een lijk}{in huis opscheepte}\\

\haiku{Dat was nu wel geen,.}{boos opzet maar hij wist dat}{hij een hartkwaal had}\\

\haiku{Ik weet het wel, je}{kunt er niets aan doen dat je}{me niet kunt nemen}\\

\haiku{Of misschien was het.}{enkel maar een verschijnsel}{van volwassenheid}\\

\haiku{Ze schrok er zelf van,,.}{ze kleurde er trok een vaag}{rood over haar voorhoofd}\\

\haiku{Ik strekte dus de,:}{rug keek hen vastberaden}{tegemoet en zei}\\

\haiku{genoeg was om te,.}{begrijpen waar het om ging}{als ik maar wilde}\\

\haiku{Ik zag nog juist hoe.}{vastberaden ze de deur}{van haar kamer sloot}\\

\haiku{Dus, wat had hij er?}{dan eigenlijk op tegen}{dat hij sterven moest}\\

\haiku{Waarom moet je die?}{twee met alle geweld op}{mekaar betrekken}\\

\haiku{Ze hijgde van drift,.}{en ze was nog lang niet aan}{het eind van haar wrok}\\

\haiku{Ik zag haar nog staan,.}{voor de piano op de}{portretten wijzend}\\

\haiku{Ik kon de hoop wel,.}{opgeven ik hoefde niets}{meer te verwachten}\\

\haiku{{\textquoteleft}En nou is 't uit,,!}{verdomme nou zal ik je}{eens wat vertellen}\\

\haiku{{\textquoteleft}Lieveling,{\textquoteright} zei ze, {\textquoteleft}}{toen met een lallende tong}{als je boos wordt is}\\

\haiku{Ik wuifde haar zelfs,.}{nog na zoals je iemand}{nawuift die emigreert}\\

\haiku{Gisteren Sally,,.}{vandaag Francisca dat was}{voorlopig genoeg}\\

\haiku{In haar jeugd had ze.}{hem gevolgd tegen de zin}{van haar moeder in}\\

\haiku{In wezen had ze,}{toen al met me afgedaan}{ze sloot me buiten}\\

\haiku{{\textquoteleft}Soms denk ik erover,?}{om weer te trouwen heeft Mea}{je dat niet verteld}\\

\haiku{En daarop zei ze:}{dan toch eindelijk wat ik}{wilde uitlokken}\\

\haiku{Als jonge vrouw was,.}{ze bepaald knap geweest dat}{was nog wel te zien}\\

\haiku{Als je toch maar de!}{vrijheid hield om er wel of}{niet op in te gaan}\\

\haiku{hij stond nog altijd.}{beter tegenover mij dan}{ik tegenover hem}\\

\haiku{Ik zag ervan af:}{om hem tegen te spreken}{en zei alleen maar}\\

\haiku{Ik schrok ervan, ik {\textquoteleft}!}{kon niet zeggenmaar dat is}{helemaal niet waar}\\

\haiku{wat je me daar hebt.}{zitten vertellen weet ik}{trouwens ook allang}\\

\haiku{Waar ik wel onder.}{geleden heb dat zijn heel}{andere zaken}\\

\haiku{Net zolang totdat.}{ik begreep dat het leven}{zelf het verraad was}\\

\haiku{{\textquoteright} zei Paul Stermunt, {\textquoteleft}in.}{de Brooklynbar als je daar}{niets op tegen hebt}\\

\haiku{Er stond me nog maar,,.}{\'e\'en ding te doen meende ik}{opstaan en weggaan}\\

\haiku{Maar voordat hij me.}{achterhaald had wist ik hem}{toch te ontduiken}\\

\haiku{{\textquoteleft}Als je dat mij voor ',.}{t zeggen geeft ga dan nog}{eerst maar even zitten}\\

\haiku{Misschien kwam haar mijn.}{wanhoop alleen maar klein en}{belachelijk voor}\\

\haiku{{\textquoteright} En daarop trok ik:}{mijn handen van mijn gezicht}{weg en zei heftig}\\

\haiku{Haar mooie gezicht was.}{koud en hard om te zien en}{haar ogen stonden groot}\\

\haiku{{\textquoteright} En hierop zei ik:}{natuurlijk wat iedereen}{zou gezegd hebben}\\

\haiku{En geen man die dat,.}{had kunnen verdragen maar}{hij natuurlijk wel}\\

\haiku{Het klonk smartelijk (?).}{en verwijtendkondet gij}{niet \'e\'en uur waken}\\

\haiku{{\textquoteright} zei ik, {\textquoteleft}dat ik je,}{niet heb gehoord ik heb je}{gehoord voor zover}\\

\haiku{Paul was dood, en dat.}{viel niet weg te praten of}{weg te huilen}\\

\haiku{de zee over, het strand,,.}{langs de duinen op maar hij}{ontdekte me niet}\\

\haiku{{\textquoteright} En toen voelde ik.}{plotseling de tranen over}{mijn wangen stromen}\\

\haiku{Zij was en bleef groot,.}{in haar liefde ik was en}{bleef klein in mijn trouw}\\

\haiku{Op een gegeven.}{dag kwam ik thuis en vond ik}{haar afscheidsbriefje}\\

\haiku{Hij stond op en stak,:}{me een hand toe maar toen zei}{hij nog aarzelend}\\

\haiku{Ik zei dus tegen:}{dat jongemeisje met haar}{bezorgde gezicht}\\

\haiku{We liepen toen juist:}{langs de laatste brits en daar}{bleef ze staan en zei}\\

\haiku{{\textquoteright} Maar daar was ik nu,.}{eenmaal te laat mee dat kon}{ik niet meer vragen}\\

\haiku{{\textquoteright} bouwde ze me na, {\textquoteleft}.}{dat je die gevoelens niet}{kunt beantwoorden}\\

\haiku{Vertel het maar,{\textquoteright} zei, {\textquoteleft} '?}{ik toenwaarom heb jet}{haar niet gegeven}\\

\haiku{Het is anders wel,?}{komisch om die twee samen}{te zien vind je niet}\\

\haiku{Het beste is dat.}{we spelen dat ik jou ook}{nog nooit heb gezien}\\

\haiku{{\textquoteleft}Die vriend van me die{\textquoteright}...}{door een auto-ongeluk}{omgekomen is}\\

\haiku{Ze liep de steiger:}{langs en las de namen van}{de boten luidop}\\

\haiku{{\textquoteright} zei ik scherp, maar toen:}{herstelde ik me en zei}{gemoedelijker}\\

\haiku{Ik weet precies hoe '.}{je bent en ik weet precies}{hoe jet bedoelt}\\

\haiku{{\textquoteright} {\textquoteleft}Neen, ik zal het je,,.}{maar zeggen hij ging voor me}{stelen en daarom}\\

\haiku{{\textquoteright} En dan was het best.}{mogelijk dat ik ook niets}{anders bedoelde}\\

\haiku{Ik kan dus alleen:}{nog maar vertellen wat er}{daarna gebeurde}\\

\subsection{Uit: De verliezers}

\haiku{Zuster Vos, ze liep,,.}{tegen de veertig jaar ze}{was gezond robuust}\\

\haiku{wie weet komt er van.}{de ene op de andere}{dag de ommekeer}\\

\haiku{Ze zouden bijna,.}{vergeten dat ze zijn vrouw}{was dertig jaar lang}\\

\haiku{De hand die hij eerst.}{op de kist had gelegd bracht}{hij nu naar zijn borst}\\

\haiku{Het enige dat er;}{op dit moment te horen}{was kwam van buiten}\\

\haiku{Hij had zich van de.}{kist afgekeerd en wilde}{de gang in stormen}\\

\haiku{Het was zo, er was.}{grijs tussen het donkere}{haar aan de slapen}\\

\haiku{- Ik had, zei Driekje,.}{een vriendje dat me nooit op}{tijd terug het gaan}\\

\haiku{Ondertussen zat,.}{de vermoeide man op de}{trap onbeweeglijk}\\

\haiku{hij zou in staat zijn,.}{om te gaan smeken op z'n}{knie\"en te vallen}\\

\haiku{{\textquoteleft}O, dat is wel in,{\textquoteright}.}{orde zei meneer Das en}{dat klonk schijnheilig}\\

\haiku{Helaas, daar kwam ze, '.}{niet toe al zou zet nog}{honderd keer denken}\\

\haiku{Ik zag alleen maar,.}{haar gezicht natuurlijk maar}{daar gaat het toch om}\\

\haiku{In waarheid ben ik,.}{vooral bang om het alleen}{te doen gewoon bang}\\

\haiku{Twee hoogtepunten,.}{en daartussen een donker}{mysterieus niets}\\

\haiku{- Ze nam ze \'e\'en voor.}{\'e\'en van hem over en legde}{ze op het dressoir}\\

\haiku{Ze hoopte van niet,,.}{want hoe dan ook ze moest hem}{in de steek laten}\\

\haiku{Misschien ook in de.}{enige mens die hier alleen}{verder moest leven}\\

\haiku{zich onder haar blik.}{betrapt alsof hij dat wel}{zou hebben gedaan}\\

\haiku{Hij zou haar nog wel,.}{eens op haar nummer zetten}{als hij de kans kreeg}\\

\haiku{{\textquoteleft}Zo,{\textquoteright} zei hij, {\textquoteleft}nou zuip.}{je de kolenkit maar leeg}{als je nog meer moet}\\

\haiku{{\textquoteright} {\textquoteleft}Dat weet je best,{\textquoteright} zei}{hij en hij keek haar met z'n}{fletse blauwe ogen}\\

\haiku{je keek ze rond, naar,.}{de meubels resten uit een}{oude inboedel}\\

\haiku{Haar begrip dan? - Ik...}{heb nooit een beroep  op}{haar begrip gedaan}\\

\haiku{Hield ze geen afstand,?}{door een gebrek aan moed aan}{dieper meeleven}\\

\haiku{Je las het brief je:}{nog eens dat ze voor je op}{tafel had gelegd}\\

\haiku{Er is iets dat me,...{\textquoteright}}{nerveus heeft gemaakt dat is}{toch niets bijzonders}\\

\haiku{Ze zou natuurlijk.}{in ieder geval wachten}{tot ze terug was}\\

\haiku{Weer pakte ze de.}{badhanddoek en droogde haar}{bezwete gezicht}\\

\haiku{Honderd keer kom je.}{in een sterf huis en het doet}{je zo goed als niets}\\

\haiku{Ze hield plotseling.}{op alsof ze zichzelf het}{zwijgen oplegde}\\

\haiku{{\textquoteright} zei ze, {\textquoteleft}ik ga nog,.}{maar zelden naar huis hoe graag}{ze me daar ook zien}\\

\haiku{{\textquoteleft}Die zou ik ge\"erfd.}{hebben van een dankbare}{rijke pati\"ent}\\

\haiku{{\textquoteleft}Ik begrijp er niets,,.}{van meneer Kostiaan maar}{komt u even binnen}\\

\haiku{Want wat was dat dan,!}{ook voor gekheid zoiets hier}{te willen laten}\\

\haiku{Maar Driekje hield zich.}{op een andere manier}{met de vraag bezig}\\

\haiku{{\textquoteright} {\textquoteleft}Dus u sliep zelf ook,?}{nog niet ik heb u dus niet}{uit de slaap gehaald}\\

\haiku{die heeft zich nergens,,.}{over te beklagen maar nou}{ja vergeet het maar}\\

\haiku{Hij dacht ik denk niet,,.}{aan Lucia niet aan zuster}{Vos niet aan morgen}\\

\haiku{{\textquoteleft}Ga naar je nest,{\textquoteright} zei, {\textquoteleft}.}{ze ruwje moet nodig naar}{de dokter lopen}\\

\haiku{Zo deed je met een.}{stuk in je kraag altijd iets}{waar je spijt van kreeg}\\

\haiku{Al was het alleen.}{maar omdat ze die niet van}{u gekregen heeft}\\

\haiku{Louis is nog altijd,.}{haar zoon hij heeft er recht op}{dat te weten}\\

\haiku{De zaak is, hij is,.}{gestolen of verloren}{dat weten we niet}\\

\haiku{Het bevrijdde hem,.}{het nam dat vreselijke}{ziekzijn van hem af}\\

\haiku{Het drong beslist niet.}{tot haar door dat die ring een}{kapitaal waard was}\\

\haiku{al was 't alleen,.}{maar om dat mooie weer zo is}{het toch bijna nooit}\\

\haiku{Je kunt er zelfs met.}{een zonnebril op nog niet}{tegen in kijken}\\

\haiku{Niet dat dat er z\'o,,.}{erg op aan komt die denkt toch}{nooit fraai dat weet je}\\

\haiku{{\textquoteright} Dat klonk van z{\'\i}jn kant,.}{als een fraze maar zo was}{het evenmin bedoeld}\\

\haiku{Maar ze begon er,:}{niet onmiddellijk over ze}{keek op en ze vroeg}\\

\haiku{Hij staarde haar aan.}{alsof ze hem een schrikbeeld}{had voorgespiegeld}\\

\haiku{{\textquoteright} En terwijl ze die:}{koffie inschonk zei hij van}{achter zijn handen}\\

\haiku{{\textquoteleft}Ik dacht het wel, zo,.}{ziet u haar Godzijdank niet}{zo w\'as het ook niet}\\

\haiku{En hij kwam ook nog,.}{met iets anders aan met een}{groot rond vergrootglas}\\

\haiku{Als ze bij me zou.}{logeren zou ik me wel}{schrap moeten zetten}\\

\haiku{Hij het haar wel diep.}{doordringen in dat voorbije}{leven van Lucia}\\

\haiku{Maar ik kende haar,,.}{vergeet dat niet al had ik}{haar dan verloren}\\

\haiku{Houdt u vast aan uw,.}{eigen indruk dat is voor}{mij van enorm belang}\\

\haiku{Ze staarde hem met,.}{diepe ernst aan eigenlijk}{zonder hem te zien}\\

\haiku{Ze hadden er niet,.}{meer naar omgekeken heel}{dat gesprek niet meer}\\

\haiku{Wat zou ze niet te!}{horen krijgen als ze straks}{alles had verteld}\\

\haiku{{\textquoteleft}Ik weet niet of ik,.}{dat zo hoog moet nemen ik}{weet het eerlijk niet}\\

\haiku{Als iemand zegt dat!}{je zo toegankelijk naar}{hem geluisterd hebt}\\

\haiku{Ik vraag Bertha misschien,.}{wel wat zij ervan denkt van}{zoiets waanzinnigs}\\

\haiku{Maar nu zie ik met,.}{m'n eigen ogen wat hij je}{stuurt uit dankbaarheid}\\

\haiku{Hij was nergens op,.}{bedacht het was op heel de}{trap volkomen stil}\\

\haiku{Niet arrogant, niet,.}{afkeurend alleen maar ziek}{en smekend om hulp}\\

\haiku{En bovendien werd.}{zijn suggestie bijzonder}{gunstig ontvangen}\\

\haiku{huiveren als op,!...}{een winteravond buiten niet}{warm genoeg gekleed}\\

\haiku{Je begon dus over, '.}{Loosje en dan had jet}{plotseling over haar}\\

\haiku{{\textquoteright} En toen zag hij ook;}{plotseling wat er met die}{ogen aan de hand was}\\

\haiku{Hij wierp er zich als ',:}{t ware met volle kracht}{tegen aan hij zei}\\

\haiku{Ze zei alleen maar,:}{koel en zakelijk alsof}{er niets gezegd was}\\

\haiku{Het leek erop dat.}{hij nog nooit werkelijk naar}{haar had gekeken}\\

\haiku{Ze wou immers geen,!}{attenties ze wou toch dat}{hij aan haar naam dacht}\\

\haiku{Lucia, je bent zo.}{ontzettend ver weg als je}{zo zit te kijken}\\

\haiku{Zit je verdriet niet,.}{zo alleen te vreten maar}{praat er met me over}\\

\haiku{Zijn blik zwierf over straat,.}{er liep er niet \'e\'en die bij}{Lucia halen kon}\\

\haiku{Ha, dat was geestig,:}{als Lucia het dan wel was}{dan gold voor Lucia}\\

\haiku{In die tijd moet het:}{geweest zijn dat hem voor de}{laatste keer ontviel}\\

\haiku{Ze dacht even na, ze,.}{zocht een vergelijking maar}{kon die niet vinden}\\

\haiku{- En daarop schudde.}{hij enkel maar het hoofd en}{dacht een tijdje na}\\

\haiku{Maar komt u gerust,}{eens naarboven voor een}{kopje koffie zo}\\

\haiku{- Ze had een bordje.}{soep voor hem gemaakt en ging}{daarmee naar binnen}\\

\haiku{{\textquoteleft}Het is z\'o gedaan,{\textquoteright}.}{zei ze nog en ze zette}{vlug twee kopjes uit}\\

\haiku{Er kon hier in huis,.}{wel gestolen worden er}{kon wel brand komen}\\

\haiku{- Ik zei natuurlijk:}{alleen maar wat je in zo'n}{geval altijd zegt}\\

\haiku{Ik leef, dus ik doe,.}{en beslis en dan nog naar}{mijn beste weten}\\

\haiku{Als hij niet ziek is,,!}{zei de zuster wat heb ik}{er dan te maken}\\

\haiku{Die vriendschap gaf me,!}{nu zoveel plezier maar wat}{gaat dat nog worden}\\

\haiku{{\textquoteright} zei ze zacht, {\textquoteleft}maar ik,?}{bedoel zijn er soms dingen}{uit de kamer weg}\\

\haiku{{\textquoteleft}Geen sterveling is,!}{precies wat hij zou moeten}{wezen o God neen}\\

\haiku{En leugens, welk mens,.}{leeft er nu zonder leugens}{meneer Kostiaan}\\

\haiku{Zou u bijvoorbeeld?}{God durven haten als u}{in God geloofde}\\

\haiku{Hij keek haar niet aan,.}{toen hij dat zei hij hield de}{ogen neergeslagen}\\

\haiku{Het hield maar niet op,,.}{het bleef maar aan de gang in}{haar ze wist nog meer}\\

\haiku{U hoeft er niet over.}{te piekeren of ze wel}{of niet van u hield}\\

\haiku{- En dan rijdt ze weg,,,}{op haar fietsje met kromme}{rug en dan denk je}\\

\haiku{En als ze hem dan.}{weer zag kon ze nog wel eens}{van hem opkijken}\\

\haiku{Goed, ze had haar de,!}{mond gesnoerd maar was het niet}{om je te schamen}\\

\haiku{Dat was plagen, en.}{aan de andere kant van}{de lijn bleef het stil}\\

\haiku{Ik zie niets, dat is,.}{nu eenmaal zo maar ik heb}{wel gevoel voor sfeer}\\

\haiku{{\textquoteright} En toen nam ze dat:}{gezicht tussen haar handen}{en zei hartelijk}\\

\haiku{{\textquoteleft}Ik weet het, ik ben,!}{een onmogelijk mens ik}{zeg de gekste dingen}\\

\haiku{Kijk nu eens Bertha, was,?}{die verstandiger dan zij}{of evenwichtiger}\\

\haiku{Halverwege hief.}{ze een hand op en legde}{die op het behang}\\

\haiku{{\textquoteleft}En daarmee h\`eb je ',{\textquoteright}, {\textquoteleft}.}{t gevraagd zei Berthamaar je}{mag het wel weten}\\

\haiku{{\textquoteright} En toen greep ze haar:}{bij de schouders en keek haar}{aan en zei innig}\\

\haiku{Rustig, een beetje,.}{plechtig precies zoals die}{paren die ik zag}\\

\haiku{hij legde in 't:}{voorbijgaan een hand op haar}{schouder en vroeg luid}\\

\haiku{- Maar hij merkte het,.}{al het was de verkeerde}{met wie hij opliep}\\

\haiku{Geen stoffer komt er,.}{op de vloer geen kopje wordt}{er omgewassen}\\

\haiku{Nou, nou, als je dat...}{doet kan je zo oud worden}{als Methusalem}\\

\haiku{Toen kreeg hij ook dat.}{holle zenuwachtige}{gevoel in z'n maag}\\

\haiku{Hij deed maar niet eens.}{meer een poging om wat te}{zien of te horen}\\

\haiku{Ik weet niet wat het,.}{is terecht op de wereld}{zijn en nodig zijn}\\

\haiku{Hij had er genoeg,!...}{van gezien en begrepen}{hij wist het nu wel}\\

\haiku{Je loopt daar alsof,.}{je ergens op af gaat maar}{je doet maar alsof}\\

\haiku{Hij had er nooit bij,.}{stilgestaan maar die Bertha vond}{hij niet sympathiek}\\

\haiku{- Hij vroeg, om maar iets,:}{te zeggen om haar niet te}{laten ontsnappen}\\

\haiku{Hij zei, terwijl hij:}{z'n hand terugtrok en om}{haar schouder legde}\\

\haiku{Maar het was misschien.}{verkeerd om daarover meteen}{al te beginnen}\\

\haiku{{\textquoteright} Maar daar wilde ze,.}{beslist nog niet over praten}{al zat het haar hoog}\\

\haiku{bega je nog vaak.}{de stommiteiten die in}{je natuur liggen}\\

\haiku{Is het geen pak van?}{je hart dat we er nu over}{gesproken hebben}\\

\haiku{{\textquoteleft}Geef die roos maar hier,{\textquoteright}, {\textquoteleft}.}{zei zedie zet ik wel even}{in een glas water}\\

\haiku{Ik zei het alleen,!}{maar voorzichtig maar ik vraag}{u ten huwelijk}\\

\haiku{{\textquoteleft}Ik heb er dagen,,.}{over nagedacht geloof me}{maar ik meen het echt}\\

\haiku{Maar Kostiaan, dat,}{is ernstiger ik zweer je}{dat ik dat oprecht}\\

\haiku{{\textquoteleft}Terwijl alles wat,?}{hier staat even afschuwelijk}{is weet je dat wel}\\

\haiku{Maar door jou is het,,!}{hier goed en gezellig en}{veilig noem maar op}\\

\haiku{{\textquoteright} {\textquoteleft}Natuurlijk,{\textquoteright} zei Bertha,, {\textquoteleft} '.}{maar haar stem hield afstandals}{jet zeggen wilt}\\

\haiku{En verder,  dat,,.}{was het mooiste was die zoon}{van hem die Louis Jr}\\

\haiku{De tranen sprongen:}{haar in de ogen en ze zei}{met verstikte stem}\\

\haiku{Ik laat je niet in,!}{de steek maar ik weet het toch}{net zo min als jij}\\

\haiku{Ik moet ook wat aan,...}{de tijd overlaten de tijd}{brengt wel weer evenwicht}\\

\haiku{Hij ging, hij wierp een:}{honende boosaardige}{blik op haar en zei}\\

\haiku{Ze dronk enkel een.}{kop thee en ging vroeger dan}{gewoonlijk op pad}\\

\haiku{{\textquoteright} De zuster gaf haar:}{de injectie in een beurs}{geprikt been en zei}\\

\haiku{Hij kan me toch niet!}{een hele dag zonder eten}{en drinken laten}\\

\haiku{Toen, met alle kracht,.}{die in haar was brak ze die}{barricade weg}\\

\haiku{Het was een kwestie,.}{van seconden het was een}{wedloop met de dood}\\

\haiku{{\textquoteleft}Ik begrijp het niet,,.}{want u bent een gewone}{zuster waar of niet}\\

\haiku{Maar toch moet u het.}{zijn want altijd voel ik me}{met u verbonden}\\

\haiku{Meneer Das merkte.}{natuurlijk dat ze haar fiets}{weer buiten zette}\\

\haiku{{\textquoteleft}Ik vraag me af wie,.}{er ongelukkiger is}{Kostiaan of ik}\\

\haiku{En toen liep hij naar.}{de tafel en sloeg met een}{vuist in het gebak}\\

\subsection{Uit: Vrouw en vriend}

\haiku{ik had lang geen spijt,.}{van graad of titel die ik}{misgelopen was}\\

\haiku{Maar ik geloof dat,,.}{hij door dat te zeggen zijn}{jeugd vergeten wou}\\

\haiku{Ik kijk alleen maar,.}{naar je gezicht waarvan ik}{afscheid nemen moet}\\

\haiku{Want nauwelijks op.}{de terugtocht hield hij me}{stil en wou me kwijt}\\

\haiku{Een schemerkamer - -,.}{eerst koffiedrinken dan thee}{en sigaretten}\\

\haiku{Voor het eerst had ik,,.}{nu met een ander Jonas}{over haar gesproken}\\

\haiku{Als het te bar werd.}{laveerde ze de kant uit}{van het boertige}\\

\haiku{De morgen had ik -.}{zoek gebracht met Jonas en}{een krankzinnige}\\

\haiku{Ik wist het wel, zij,.}{had wel door dat het met Saar}{en mij niet deugde}\\

\haiku{Het deed me denken,...}{aan de dreinende ritmiek}{van regen regen}\\

\haiku{Hij had wat moeten,.}{zeggen wat anders moeten}{zeggen dan hij deed}\\

\haiku{Een prinselijke,.}{pauper een pauper met een}{prinsenziel was hij}\\

\haiku{Fortuin en liefde -.}{in uw leven zullen u}{de blonden geven}\\

\haiku{Ook Kareltje en.}{zij waren ten slotte aan}{de dijk gaan zitten}\\

\haiku{De lente bracht die;}{jongen met zijn vrouwelijk}{verholen liefde}\\

\haiku{Ze deed een vrouw in ',,.}{t bad die zich bevuild had}{een zachtzinnig mens}\\

\haiku{Maar daarna is ons.}{samenzijn ellendiger}{dan ooit te voren}\\

\haiku{Ik hield krampachtig.}{de blijdschap in mijn labiel}{gemoed in evenwicht}\\

\haiku{{\textquoteright} {\textquoteleft}Nou, ik denk altijd.}{toch nog beter over een man}{dan over vrouwetuig}\\

\haiku{{\textquoteleft}My life, since,.}{I loved you has been one}{prolonged agony}\\

\haiku{Zonder warmte in.}{zijn ogen boog hij zich naar haar}{toe en kuste haar}\\

\haiku{Hij keek haar vlug, en.}{zonder uitdrukking in zijn}{te lichte ogen aan}\\

\haiku{Jonas boog zich over.}{het portier en braakte de}{zwarte koffie uit}\\

\haiku{Het ging er nou maar,.}{om het vol te houden tot}{hij weer boven zat}\\

\haiku{Marie zat aan de,.}{grasglooiing waar ze beschut}{was tegen de wind}\\

\haiku{Ontevreden keek.}{ze naar de karreploeg die}{op het veld werkte}\\

\haiku{Hij zag hoe zij zich.}{rekte om die boven in}{de kast te zetten}\\

\haiku{{\textquoteleft}Ik had de indruk,{\textquoteright}, {\textquoteleft}.}{zei hij ingetogendat}{u iets hinderde}\\

\haiku{Misschien roken we,.}{samen een sigaret dat}{neemt de moeheid weg}\\

\haiku{Ik ging de kamer,,.}{in en streelde terwijl ik}{langs haar liep haar arm}\\

\haiku{Ik moest verdwijnen,.}{en wel onmiddellijk en}{zonder aarzeling}\\

\haiku{Ik moest het weten,,,.}{ik had haar gezien niet waar}{die zondagmorgen}\\

\haiku{Had ze zo'n honger,?}{of was haar maag nu nog niet}{helemaal op streek}\\

\haiku{Ze schonk zich nog een,.}{kopje thee in en dronk dat}{haastig dorstig leeg}\\

\haiku{Schamper haalde ze:}{de schouders op en boog zich}{naar de tafel toe}\\

\haiku{Op een gegeven.}{ogenblik trok ze de voeten}{van die stoel terug}\\

\haiku{een eenzaam man, stram,,,.}{arrogant met een lege}{hoffelijke grijns}\\

\haiku{Elke morgen nam:}{ze aan de piano haar}{oefeningen door}\\

\haiku{Geenerveerd en moe.}{schoof ze de zilvervos wat}{van haar hals terug}\\

\haiku{En bovendien, een,.}{kind het leven geven wat}{een verantwoording}\\

\haiku{Die maandag nog was,.}{hij naar een concert geweest}{een Bachrecital}\\

\haiku{My life since,.}{I loved you has been one}{prolonged agony}\\

\haiku{Op de tast greep ze.}{een nieuwe sigaret en}{zoog het vuur erin}\\

\haiku{Droomde ze wel ooit,?}{dat ze zich er daarna iets}{van herinnerde}\\

\haiku{{\textquoteright} {\textquoteleft}Ik heb,{\textquoteright} zei hij, {\textquoteleft}een,.}{grammofoon met mooie platen}{zoals Tannh\"auser}\\

\haiku{ze greep zijn hand en.}{zo bleven ze zitten toen}{het weer donker werd}\\

\haiku{De lucht was duister,.}{minder sterren waren er}{dan hij gedacht had}\\

\haiku{Maar ook aan morgen,.}{moest hij nu niet denken nu}{was hij met Marie}\\

\haiku{Heel zijn leven was.}{\'e\'en lange hunkerende}{wacht geweest op haar}\\

\haiku{Zou het ooit vriendschap?}{kunnen worden als hij hem}{nu al kwijtraakte}\\

\haiku{Tot morgen,{\textquoteright} zei hij -.}{nog en op het trambalkon}{keek hij nog even om}\\

\haiku{Ik was ondankbaar,,.}{werkelijk dat ik me dat}{niet  zeggen liet}\\

\haiku{Ik was gekleed en.}{driftig wou ik nu op mijn}{beurt naar beneden}\\

\haiku{De zolder stond in.}{strakke binten over heel de}{diepte van het huis}\\

\haiku{Het meisje begon,.}{gejaagd te schreien daarom}{stuurden we haar weg}\\

\haiku{Zodra de bel ging,.}{haastte ik me naar de trap}{en trok de deur open}\\

\haiku{Zulk huilen maakte,.}{me machteloos ik stond daar}{maar en wist geen troost}\\

\haiku{{\textquoteright} Zij wiste met de,.}{vrije hand haar tranen weg er}{was zoveel te doen}\\

\haiku{Zachtjes trok ik de '.}{buitendeur int slot en}{draalde op de stoep}\\

\haiku{Voor koffietijd bracht.}{ik nog even mijn artikel}{aan de directeur}\\

\haiku{Hij was zuinig waar,.}{het op waarderen aankwam}{maar hij had gelijk}\\

\haiku{Mevrouw de Watter.}{keek mijn verbazing met een}{zachte glimlach aan}\\

\haiku{{\textquoteright} vroeg ze kinderlijk,.}{terwijl haarzelf de tranen}{in de ogen welden}\\

\haiku{De stappen hielden,.}{stil voor mijn deur er klopte}{iemand zachtjes aan}\\

\haiku{Op dat moment greep,}{ik hem bij de arm troonde}{hem mee en wachtte}\\

\haiku{Het regende niet,,}{meer de weg was modderig}{en onbegaanbaar}\\

\haiku{Ik wist nu ook, dat.}{hij ternauwernood nog een}{gesprek verwachtte}\\

\haiku{langs de oevers, die,.}{hun tocht vervolgden bleef het}{steekspel onbeslist}\\

\haiku{- Toos was de trede.}{straat teruggelopen en}{ze stak het plein over}\\

\haiku{Ze was nu aan het.}{aarden pad gekomen dat}{naar boven voerde}\\

\haiku{Bij jonge oogst kreeg,.}{ze een ijl angstgevoel dat}{niet onprettig was}\\

\haiku{{\textquoteleft}O neen, blijf nu toch,,{\textquoteright}.}{zitten Saartje zei mevrouw de}{Watter iets te luid}\\

\haiku{Haar ogen lagen nu,.}{vlak onder me ik zag de}{irissen verblauwen}\\

\haiku{Ze staarde leeg en,.}{treurig weg met vochtige}{verblauwde irissen}\\

\haiku{Ze keek haar aan en:}{ze ontmoette een paar ogen}{vol verweer en schrik}\\

\haiku{Want als dat zo niet,.}{was dan hadden we elkaar}{niet zo gevonden}\\

\haiku{Zou je lust hebben,,?}{vanavond ergens heen te gaan}{of morgen Sara}\\

\haiku{- Het ziekenhuis was.}{een door tuinen omgeven}{kloosterlijk gebouw}\\

\haiku{Op de zaal met aan,.}{weerskanten bedden zag ik}{Toos en toen pas hem}\\

\haiku{{\textquoteleft}Ik heb geen tijd, vind ',?}{jet vervelend dat ik}{je alleen laat gaan}\\

\haiku{Er daverde een,.}{trein uit het perron er kwam}{er weer een binnen}\\

\haiku{Ze trok zich los en,,:}{keek me koud afwijzend aan}{haar ogen waren grauw}\\

\haiku{In de bossen nam,.}{elk vrouwtjesdier de vlucht voor}{hem hij was geen dier}\\

\haiku{{\textquoteright} Met een ruk hield ze,,:}{toen stil ze keek me aan in}{koude haat ze zei}\\

\haiku{Ik sloeg de dekens.}{van me af en ging weer op}{m'n bedrand zitten}\\

\haiku{{\textquoteright} Maar ze kijkt me aan.}{of ze me niet herkent en}{zegt natuurlijk neen}\\

\haiku{Mijn adem stokte of,...}{ik huilen moest moorddadig}{wild had ik haar lief}\\

\haiku{Wat er gebeurd was,.}{was te teer om aangeroerd}{te mogen worden}\\

\haiku{{\textquoteright} Als de tram komt steekt,.}{ze waarschuwend de hand op}{wat niet nodig is}\\

\section{Rein Blijstra}

\subsection{Uit: Diagnose}

\haiku{Hier, neem die kruik mee,.}{en die twee glaasjes dan volg}{ik met de sherry}\\

\haiku{Want in de eerste.}{plaats zijn Vincent en ik niet}{verliefd op elkaar}\\

\haiku{Hoe waren ze toen,.}{eigenlijk beiden vraag ik}{me nu wel eens af}\\

\haiku{Dit primitieve,,.}{gevoel je weet wel dat ons}{helemaal niet past}\\

\haiku{{\textquoteleft}Heb jij Vincent ooit,?}{laten merken dat er iets}{tussen ons bestond}\\

\haiku{'s avonds over je vak,.}{te praten al is het dan}{ook maar zijdelings}\\

\haiku{Als je arts bent moet.}{je verantwoordelijkheid}{weten te dragen}\\

\haiku{Dan meen je misschien,.}{dat je helemaal niet meer}{bij ons kunt komen}\\

\haiku{{\textquoteright} En ik haastte me.}{naar mijn kamer en verliet}{kort daarop het huis}\\

\haiku{Je wilde dus nog?}{een R\"ontgenphoto maken}{om zeker te zijn}\\

\haiku{Was het sterven je,?}{dan toch gemakkelijker}{gevallen mijn vriend}\\

\subsection{Uit: Geheim archief}

\haiku{Wraak en vergelding{\textquoteright}.}{zijn organismen in een}{onvruchtbaren schoot}\\

\haiku{Munitie kan men.}{niet van dag tot dag naar een}{andere plaats sleepen}\\

\haiku{{\textquoteright} {\textquoteleft}Ik dacht dat je iets,{\textquoteright}.}{op het spoor was de laatste}{weken of maanden}\\

\haiku{We zijn immers met......}{zijn twee\"en en als je me}{niet begrepen had}\\

\haiku{{\textquoteleft}Zoo, zoo{\textquoteright}, lachte ik, {\textquoteleft},.}{een bevel van een koning}{die reeds onttroond is}\\

\haiku{{\textquoteright} {\textquoteleft}Om het laatste dep\^ot.}{wat ik je aangewezen}{heb te ontruimen}\\

\haiku{Lluch zal zich beter.}{weten te verdedigen}{dan mijn vriend Juan}\\

\haiku{{\textquoteleft}Ik ga even kijken,{\textquoteright},.}{waar mijn zoon blijft zegt hij en}{verlaat den salon}\\

\haiku{Medicijnmannen,,...?}{priesters geneesheeren is}{dat de evolutie}\\

\haiku{{\textquoteright} {\textquoteleft}Ik weet niet, hoe ik{\textquoteright},, {\textquoteleft}.}{er toe kwam zegt Dr. Arriens}{neem me niet kwalijk}\\

\haiku{Bovendien behoort{\textquoteright}.}{de zaal hiernaast immers ook}{tot de eerste klas}\\

\haiku{Hij heeft er zelf om{\textquoteright}.}{gevraagd en je kon het hem}{nu niet weigeren}\\

\haiku{{\textquoteleft}Als we naar boven,{\textquoteright}.}{moeten kunnen we beter}{meteen wegzwemmen}\\

\haiku{Het is bijna of.}{ik een onbestemden angst}{heb voor een vreemd land}\\

\haiku{beschouwde hij het?}{in zijn hart dan toch als een}{sollicitatie}\\

\haiku{{\textquoteright} {\textquoteleft}Wij kennen deze,.}{redenen niet omdat het}{doel te ernstig is}\\

\haiku{Maar dit alles was,;}{nog zoo ver dat het als het}{ware niet bestond}\\

\haiku{Ten slotte moest men.}{de feiten misschien ook kalm}{onder de oogen zien}\\

\haiku{Zijn opmerkingen.}{waren ten minste alleen}{van technischen aard}\\

\haiku{Ten slotte, het was.}{aardig een revolutie}{voor te bereiden}\\

\haiku{Maar gelukkig, het {\textquoteleft}{\textquoteright}.}{was over dag en een vrijwel}{openbare opdracht}\\

\haiku{Elken boom, dien hij,.}{passeerde was een mijlpaal}{naar het leven}\\

\haiku{Het bleek nu, dat de.}{chauffeur door kogels in het}{hoofd was getroffen}\\

\haiku{{\textquoteleft}Het is jammer, dat{\textquoteright},.}{we geen vindersloon kunnen}{eischen overwoog ik}\\

\haiku{Ik hield de parels.}{tegen het licht om me een}{houding te geven}\\

\haiku{Het bleef griezelig.}{stil en ik hoorde nu eerst}{den wind ruischen}\\

\haiku{Bovendien kon ik?}{niet meteen weggaan want jij}{was er immers nog}\\

\haiku{Maar ik vroeg naar een,.}{revolver omdat ik me}{onveilig voelde}\\

\haiku{Misschien heb je me,,.}{zelfs gered omdat je dacht}{dat je me noodig had}\\

\haiku{Boven gekomen.}{ging hij op het bed liggen}{in haar slaapkamer}\\

\haiku{Ik heb er nog eens,.}{over nagedacht maar ik kon}{geen besluit nemen}\\

\haiku{Er was niets van te,.}{zeggen ook in vijf dagen}{kan veel gebeuren}\\

\haiku{hij krabbelde een,.}{kort briefje waarin hij zich}{verontschuldigde}\\

\haiku{zij raakten elkaars.}{leven nergens dan in den}{huiselijken kring}\\

\haiku{het was een moord, een,.}{laffe moord omdat het geen}{terechtstelling was}\\

\haiku{Geen voldoening als,.}{een dier dat zijn prooi bespringt}{noch wraakgevoelens}\\

\haiku{er toch nog wel zin,.}{in zou hebben als het niet}{zoo gevaarlijk was}\\

\haiku{Maar nu zijn het er,.}{plotseling maar vijf van de}{tien die ontsnappen}\\

\haiku{ik bedoel niet de,.}{romantiek de Carmen of}{de fatale vrouw}\\

\haiku{Nu zal ik Maurits,{\textquoteright}.}{van te voren inlichten}{dat zal hem goed doen}\\

\haiku{{\textquoteright} {\textquoteleft}Ik kom pas op het{\textquoteright},, {\textquoteleft}.}{laatste oogenblik zei hij}{vlak voor het einde}\\

\haiku{beloofd had hem de.}{hand te zullen drukken als}{hij dienst weigerde}\\

\haiku{Het verwonderde,.}{Frederik dat hij geen vaag}{gevoel van angst had}\\

\haiku{Maurits en Armand.}{konden nu elk oogenblik}{naar buiten komen}\\

\haiku{Hij gaf zijn vriend de.}{hand en bleef met zijn rug naar}{de wereld gekeerd}\\

\haiku{Zij gaven elkaar {\textquoteleft}{\textquoteright},.}{nog eens de hand.Ga nu weg}{fluisterde Maurits}\\

\haiku{Daarop keerde hij,.}{zich snel om hij kon zoo niet}{door blijven loopen}\\

\subsection{Uit: Het planetarium van Otze Otzinga}

\haiku{Een toeval, de vondst,.}{van een codeboek had hem}{op het spoor gebracht}\\

\haiku{een ladder die vlak;}{achter hem kletterend op}{het plaveisel viel}\\

\haiku{{\textquoteright} {\textquoteleft}Je zou eens met mij,{\textquoteright}.}{naar een roulette moeten}{gaan stelde hij voor}\\

\haiku{Want, zoals je zegt,.}{wij zijn luchthartig en jij}{bent verleidelijk}\\

\haiku{Hij wist niet meer wat,.}{hij wilde noch  wat zij}{precies bedoeld had}\\

\haiku{{\textquoteright} {\textquoteleft}Zij onderscheiden,{\textquoteright}.}{zich niet erg van de onze}{zei hij ontwijkend}\\

\haiku{Zo eens in de tien.}{of twintig jaar komt er hier}{wel eens eentje langs}\\

\haiku{En ik was toch van:}{de andere kant van de}{Melkweg gekomen}\\

\haiku{Hij had zich beter.}{als William Johnson}{kunnen voorstellen}\\

\haiku{Ze was bijzonder;}{aardig en ik was heel lang}{onderweg geweest}\\

\haiku{{\textquoteright} Mimi lachte wat.}{raadselachtig en streelde}{me weer over mijn haar}\\

\haiku{Het is er niet meer.}{uit te houden voor iemand}{die zelfstandig denkt}\\

\haiku{Dan ben ik burger.}{van dit land en dan kan ik}{een baantje zoeken}\\

\haiku{{\textquoteleft}Als dat voor jou iets,.}{nieuws is zijn jullie nog niet}{erg opgeschoten}\\

\haiku{{\textquoteright} {\textquoteleft}In beginsel is,{\textquoteright}.}{alles bijzonder en niets}{bijzonder zei Ralph}\\

\haiku{{\textquoteright} verzekerde Ralph.}{zo plechtig als de leugen}{hem veroorloofde}\\

\haiku{Overigens had u.}{mij kunnen vragen mij te}{legitimeren}\\

\haiku{Nu, dan  heeft uw.}{broer u wel heel slecht op de}{hoogte gehouden}\\

\haiku{En je hoeft niet bang,.}{te zijn dat je het lot van}{Kolisar zult delen}\\

\haiku{, maar Peter, die aan,.}{het stuur zat had altijd haast}{als hij chauffeerde}\\

\haiku{Alice en Peter,.}{kunnen blijven zitten als}{ze geen zin hebben}\\

\haiku{Uw kwaliteiten:}{zijn in deze tijd voor ons}{niet meer van belang}\\

\section{Robert Bloemendal}

\subsection{Uit: Diefstal in de vliegtuigfabriek}

\haiku{Ongeriefelijk,,.}{warm maar niettemin in een}{uitstekend humeur}\\

\haiku{Kort beantwoordt hij,.}{haar groet en begeeft zich naar}{het bed van no. 4325}\\

\haiku{{\textquoteleft}Maar, mijnheer{\textquoteright}, zegt zij, {\textquoteleft}.}{verbaasdal die opgaven}{heeft U daar toch reeds}\\

\haiku{Ten slotte zijn wij,?}{geheel iets anders dan de}{politie niet waar}\\

\haiku{Dat is nu wel geen,.}{schande maar bepaald prettig}{vindt hij het ook niet}\\

\haiku{Iala Reichenbach:}{hem echter te hulp en zegt}{verduidelijkend}\\

\haiku{Dreigend en donker:}{klinkt de roep van de misthoorn}{over stad en haven}\\

\haiku{Ik heb toch over haar,?}{afgrijselijke accent}{gesproken niet waar}\\

\haiku{Hij mompelt zooiets {\textquoteleft}{\textquoteright}.}{alshandschoenen en bergt de}{kwitantie weer op}\\

\haiku{Daarna bijt hij de.}{punt van zijn sigaar af en}{steekt er den brand in}\\

\haiku{{\textquoteright} {\textquoteleft}Dat is werkelijk,{\textquoteright},.}{prachtig juffrouw zegt Holl en}{verdwijnt opgeruimd}\\

\haiku{Allemachtig, ik,!}{ben het hier die vragen stelt}{en niet omgekeerd}\\

\haiku{Dus hij moet zich nog.}{ergens in de omgeving}{verborgen houden}\\

\haiku{De directeuren.}{en afdeelingchefs van de}{vliegtuigenfabriek}\\

\haiku{Zijn kleeren zijn gescheurd.}{en lang zijn achterhoofd loopt}{een dun straaltje bloed}\\

\haiku{{\textquoteleft}Ik ben Jan Wuyster.}{van het politieposthuis}{aan den Adelaarsweg}\\

\haiku{Over anderhalf uur...?}{of zou het misschien morgen}{om kwart over twaalf zijn}\\

\haiku{Steeds lager cirkelt {\textquoteleft}{\textquoteright}.}{deKoetilang boven de}{hoofdstad van Siam}\\

\haiku{Nu rukt de man zich.}{echter los en wil het op}{een loopen zetten}\\

\haiku{Grant steekt een versche:}{sigaar op terwijl mister}{Hegwood verder praat}\\

\haiku{Ongeveer twee uur.}{voorbij Alexandri\"e moet u zich}{dan gereed houden}\\

\haiku{Mijn collega doet.}{Athene aan en zal u in}{Rhodos afzetten}\\

\haiku{De machine raast.}{met vol gas over de bergreuzen}{van het Turksche rijk}\\

\haiku{En elke druppel.}{heeft een andere vorm en}{andere randen}\\

\haiku{Van zijn huishoudster,.}{heeft hij reeds vernomen wie}{zijn bezoeker is}\\

\haiku{Ik herhaal nogmaals,,...{\textquoteright}}{dat het er hoofdzakelijk}{op aankomt dat wij}\\

\haiku{Ik heb gezegd, dat,.}{U geen tijd heeft maar hij laat}{zich niet afwijzen}\\

\haiku{Hij wil zijn naam niet.}{opgeven en ook niet het}{doel van zijn bezoek}\\

\haiku{In minder dan een.}{week heeft zij de heele boel}{prachtig opgeknapt}\\

\subsection{Uit: Rapaille}

\haiku{De inspecteur klopt.}{aan en wil onmiddellijk}{naar binnen stappen}\\

\haiku{Ik zal u...{\textquoteright} {\textquoteleft}Pardon,{\textquoteright}, {\textquoteleft}}{valt de telefoonjuffrouw}{hem in de rede}\\

\haiku{Dat weet u heusch,{\textquoteright}.}{wel antwoordt de colonel}{onverbiddelijk}\\

\haiku{Wanneer Baxter weer,.}{op straat staat is hij een stuk}{wijzer geworden}\\

\haiku{{\textquoteleft}Ja, dat vind ik ook,{\textquoteright}}{geeft mijnheer Grant doodleuk toe}{en kijkt den ander}\\

\haiku{{\textquoteright} {\textquoteleft}Hm, maar een van uw?}{beambten zal haar toch wel}{eens gezien hebben}\\

\haiku{Heeft hij zich vergist,.}{dan moet hij weer van voren}{af aan beginnen}\\

\haiku{Ik heb alleen maar.}{in groote trekken mijn rapport}{kunnen uitbrengen}\\

\haiku{Ik blijf hier toch nog.}{een heelen tijd en wacht hier}{op uw terugkomst}\\

\haiku{Een gek zal het - zooals -,.}{u reeds opmerkte misschien}{doen misschien ook niet}\\

\haiku{Zij zijn op z'n minst.}{een uur voor aankomst van het}{vliegtuig op Croydon}\\

\haiku{Maar het oog van den.}{beambte aan het stuur is}{nog sneller geweest}\\

\haiku{{\textquoteright} {\textquoteleft}U maakte na de?}{plundering van uw zaak uw}{aanspraken geldend}\\

\haiku{{\textquoteright} Met deze ietwat.}{cynische woorden neemt de}{juwelier afscheid}\\

\haiku{Deze kijkt bijna:}{geamuseerd wanneer hij}{de opmerking maakt}\\

\haiku{Rex Allan steekt een.}{sigaret op en gooit haar}{even later weer weg}\\

\haiku{{\textquoteleft}Ja, het is best in,{\textquoteright}.}{orde gekomen stelt Rex}{Allan hem gerust}\\

\haiku{Hij weet echter niet,.}{dat hij Baxter vandaag niet}{terug zal zien}\\

\haiku{{\textquoteright} {\textquoteleft}Ik weet niet wie het,.}{is maar ik weet w\`el wat hij}{op zijn kerfstok heeft}\\

\haiku{{\textquoteleft}Ik ben inspecteur,{\textquoteright}.}{Baxter van Scotland Yard zegt}{hij geruststellend}\\

\haiku{Mijn klanten zien er,.}{meestal anders uit zooals}{u zult begrijpen}\\

\haiku{Maar Curleigh komt ook een.}{handje helpen en nu is}{de strijd gauw beslist}\\

\haiku{Dan richt hij zich tot:}{den inspecteur en stelt de}{verrassende vraag}\\

\haiku{De balkondeuren.}{staan wagewijd open en hij}{rent naar het balkon}\\

\haiku{Nu weten wij ten,,}{minste wat er met onzen}{chauffeur is gebeurd}\\

\haiku{Als alles goed gaat,.}{krijg je een pond en als het}{mis gaat een kogel}\\

\haiku{- Geef mij even je naam.}{en adres op en kom om tien}{uur op Scotland Yard}\\

\haiku{De man, die een kop,.}{koffie zit te drinken kijkt}{onverschillig op}\\

\haiku{{\textquoteright} {\textquoteleft}Af en toe leek het,,{\textquoteright}.}{er wel op colonel zegt}{Baxter en grinnikt}\\

\haiku{Ik zou er niet veel,.}{kunnen opnoemen die u}{dat zouden nadoen}\\

\haiku{{\textquoteright} zegt Baxter, maar eer,:}{hij verder kan gaan valt Curleigh}{hem in de rede}\\

\haiku{{\textquoteright} De deur gaat open en,.}{op den drempel staat een groote}{slanke jongeman}\\

\haiku{{\textquoteright} {\textquoteleft}Het was een idee van.}{Daisy Stenton en zij heeft}{gelijk gehad ook}\\

\haiku{Hij deed het echter.}{niet en ging rustig naast zijn}{leege safe slapen}\\

\haiku{Zult u Jim Curleigh en?}{zijn verloofde werkelijk}{ongemoeid laten}\\

\haiku{{\textquoteright} {\textquoteleft}O, bedoel je dat.{\textquoteright}.}{En dan barsten zij alle}{twee in lachen uit}\\

\haiku{Daisy hoort hem naar.}{de deur gaan en meteen klinkt}{een vroolijke stem}\\

\haiku{{\textquoteleft}Kom onmiddellijk.}{scotland yard stop zeer dringend}{stop   rex allan}\\

\haiku{{\textquoteright} Terwijl zij de doos,:}{begint open te maken zegt}{Jim Curleigh glimlachend}\\

\section{Eug\`ene de Bock}

\subsection{Uit: Hendrik Conscience. Zijn persoon en zijn werk}

\haiku{De soldaten, die,.}{het kamp verlieten voegen}{zich bij hun makkers}\\

\haiku{Met De Laet nochtans;}{gebeurt de briefwisseling}{nog steeds in het Fransch}\\

\haiku{{\textquoteright} Maar dat gaat niet, de.}{gedachten blijven achter}{in zijn hoofd steken}\\

\haiku{Die vraagt Conscience,.}{uit neemt hem mee in zijn tuin}{en verdwijnt in huis}\\

\haiku{en het meisje werd.}{nog met nydigheid tegen}{het harnas gedrukt}\\

\haiku{Hij zal voor een dag.}{zijn klompen verlaten en}{weer een artiest zijn}\\

\haiku{Hij houdt een toespraak.}{en is secretaris van}{de feestcommissie}\\

\haiku{{\textquoteright} De details waar zich '.}{s schrijvers liefde aan hecht}{zijn weinig talrijk}\\

\haiku{Hij steekt de armen,.}{naar haar uit terwijl zij poogt}{afscheid te nemen}\\

\haiku{op haer hoofd stond eene.}{zilveren kroon van zeven}{blinkende sterren}\\

\haiku{Ah, hoe schatert gy,!}{van blydschap hoe magtig slaet}{gy uwe vlerken uit}\\

\haiku{Hij slaat de armen.}{om den herdersstaf zoodat hij}{er kan op leunen}\\

\haiku{hij vertellen moest,.}{en w\`at verzwijgen om zijn}{publiek te winnen}\\

\section{Boeka}

\subsection{Uit: Een koffieopziener}

\haiku{in vluggen draf het.}{erf afreed en den weg naar}{den kampong insloeg}\\

\haiku{Ieder moet vrij zijn,,.}{dus hoe eerder je op je}{zelf woont hoe beter}\\

\haiku{Prawiro die op,.}{meer gerekend had gaf niet}{dadelijk antwoord}\\

\haiku{- Weet jij misschien, of,?}{er ook eene vrouw is die als}{kokkie kan dienen}\\

\haiku{De menschen zouden.}{niet willen en dan werd de}{toewan besar boos}\\

\haiku{Maar daar ging hem een,:}{licht op die sinjo's waren}{allen hetzelfde}\\

\haiku{Ik was nog zeer klein,.}{toen ik in huis genomen}{werd door de nonja}\\

\haiku{Door groote spaarzaamheid.}{had hij een vijftig gulden}{kunnen overleggen}\\

\haiku{Hij ging dus naar de.}{kampong en kreeg een slaapplaats}{bij den hoofdmandoer}\\

\haiku{Tegen den avond kwam,.}{de inlander terug bij}{wien zij inwoonden}\\

\haiku{Medegesleept door,.}{het gat dat zeker daarom}{zoo breed gemaakt was}\\

\haiku{Eerst moesten ze door de.}{koffietuinen en kwamen}{daarna in het bosch}\\

\haiku{- De ouders zijn naar,.}{familieleden waar een}{bruiloft gevierd wordt}\\

\haiku{Weldra zou blijken,.}{dat zijne vermoedens niet}{zonder grond waren}\\

\haiku{- Het is mijn schuld,53 maar,.}{volgens den loerah mag het}{niet dat u hier is}\\

\haiku{Was die betrekking,.}{ver af dan zou de vrouw toch}{niet mede willen}\\

\haiku{De patih vond het,.}{niet noodig Karel dit alles}{mede te deelen}\\

\haiku{Van zijn rit naar huis.}{zou Karel later niet veel}{kunnen vertellen}\\

\haiku{- Neen, maar ik weet wel,.}{dat er den laatsten tijd veel}{vee gestolen wordt}\\

\haiku{- Als sobat dat wil,.}{doen en mij zoo wil helpen}{ben ik zeer dankbaar}\\

\haiku{Steeds dwaalden deze,.}{af naar het ongeval dat}{hem getroffen had}\\

\haiku{Van groote winsten, met,.}{veeteelt te behalen was}{ook niets gekomen}\\

\haiku{Zonder te groeten.}{ging hij heen en begaf zich}{naar zijne woning}\\

\haiku{Karel wist dan ook.}{op den man ten volle te}{kunnen rekenen}\\

\haiku{] Alphabetische.}{Lijst van vreemde woorden in}{dit werk voorkomend}\\

\subsection{Uit: P\`ahkasinum}

\haiku{Instelling eener;}{behoorlijke politie}{en goede rechtspraak}\\

\haiku{Straks in het midden '.}{van den Oostmoesson wordt het}{s nachts nog kouder}\\

\haiku{Na het middagmaal,}{gingen ze slapen om eerst}{wakker te worden}\\

\haiku{Pahkasinum wist in;}{het eerste oogenblik niet}{wat te antwoorden}\\

\haiku{{\textquoteright} zei de man wenkend,.}{zich snel met het ontvangen}{geld verwijderend}\\

\haiku{Vijftig of zestig.}{gulden zal ze er wel voor}{gekregen hebben}\\

\haiku{{\textquoteright} - Dat spreekt vanzelf,{\textquoteright} vond, {\textquoteleft}?}{Pahkasinumwanneer vindt de}{politie nu iets}\\

\haiku{{\textquoteright} Pahkasinum wierp het.}{eindje van zijn strootje weg en}{stak een ander op}\\

\haiku{de hoogte af, de.}{kali door en den weg op}{naar Kondanglegi}\\

\haiku{{\textquoteleft}Ga nu slapen, en!}{heb den moed niet vanavond meer}{buiten te komen}\\

\haiku{Het werd tijd, meende,.}{ze met haar man over een en}{ander te spreken}\\

\haiku{Deze zat in de.}{emper van de loods rustig}{sirih te pruimen}\\

\haiku{- Neen, de hoofdmandoer,.}{was er maar volgens zeggen}{was mijn kind er niet}\\

\haiku{{\textquoteright} - Daghuur is er op,.}{het oogenblik niet alleen}{op de droogbakken}\\

\haiku{Zoo gezellig als.}{op Djembierit was het dan}{ook bij lange niet}\\

\haiku{Pahkasinum vond, dat:}{ze daar erg lang talmde en}{daarom riep hij luid}\\

\haiku{{\textquoteleft}Het is niet goed, dat,.}{uw zoon naar Djembierit gaat}{sta dat niet meer toe}\\

\haiku{Nog een wijle werd,.}{gepraat waarna het tweetal}{opstond en vertrok}\\

\haiku{Die dieren heb ik,.}{met mijn laatste geld betaald}{ik bezit niets meer}\\

\haiku{Op vragenden toon,,:}{bij elke zinsnede even}{wachtend sprak hij zacht}\\

\haiku{- Als er geen geschenk,?}{gegeven wordt zal er dan}{wel iemand willen}\\

\subsection{Uit: P\`ah Troeno}

\haiku{Voor de gedoegan31,.}{hield hij stil steeg af en bracht}{het paard daarbinnen}\\

\haiku{Toen heb ik bevel.}{gekregen om een kantjil}{te laten vangen}\\

\haiku{{\textquoteleft}Ik heb met Sainum.}{in de warong te Kali}{Bidji gegeten}\\

\haiku{De beide lieden.}{zeiden dan ook niets meer en}{vervolgden hun weg}\\

\haiku{indien hij slechts v\'o\'or,.}{donker uit het bosch zou zijn}{was dit voldoende}\\

\haiku{Het hoofd naar den grond,,.}{gebogen de hoed in de}{hand naderde hij}\\

\haiku{Edele figuren,!}{die hij steeds met achting zou}{blijven gedenken}\\

\haiku{Bijna onhoorbaar.}{verklaarde Troeno niet weg}{te zullen loopen}\\

\haiku{Dit bleek gelukkig,.}{niet het geval want ver had}{hij niet durven gaan}\\

\haiku{Ziezoo, die zaak was,.}{voorloopig afgedaan}{dacht de wedono}\\

\haiku{Mijn man moest vrouwen,.}{zoeken maar hij wilde niet}{en nam zijn ontslag}\\

\haiku{vroeg Troeno, die zijn.}{maal ge\"eindigd had en nu}{ook naar buiten kwam}\\

\haiku{De gastvrouw had zich.}{bescheiden uit het vertrek}{teruggetrokken}\\

\haiku{Plotseling klinken.}{buiten zware stemmen en}{hoort men gejoel}\\

\haiku{Ondertusschen werd.}{het lichter en weldra zou}{de zon opkomen}\\

\haiku{Hij vond er een paar,.}{kapala's die evenals hij}{beschutting zochten}\\

\haiku{Hier in de stad zijn.}{de menschen heel anders dan}{in mijne woonplaats}\\

\haiku{- Heeft u iets gemerkt?}{of de loerah gedobbeld}{heeft in de kotta}\\

\haiku{- Ja, de god van de,.}{Chineezen is immers een}{slang zooals ze zeggen}\\

\haiku{Doch hij zeide niets,.}{want immers hijzelf schoof den}{laatsten tijd evenveel}\\

\haiku{- Dat is mijn bamboe,:}{antwoordde Troeno en liet}{er zacht op volgen}\\

\haiku{Dan zal mijnheer niet,.}{willen betalen viel de}{mandoer verstoord uit}\\

\haiku{Negen dubbeltjes,.}{op \'e\'en dag te verdienen}{dat was nog eens mooi}\\

\haiku{- Was het die Chinees,? -.}{die je uit het hok naar de}{kamer bracht Jawel}\\

\haiku{Hij liep toen even bij.}{den warong aan om een kop}{koffie te drinken}\\

\haiku{- Dat is zoo, maar van.}{mijne kinderen laat ik}{er geen meer weggaan}\\

\haiku{Uw kind werkt daar, als,?}{u nu daarvoor geld krijgt is}{dat dan verkoopen}\\

\haiku{ik sta niet toe, dat.}{mijne kinderen bij een}{Chinees aan huis zijn}\\

\haiku{Doch wanneer, zooals thans,,.}{de koffieoogst mislukt dan}{wordt geen geld verdiend}\\

\haiku{Het was haar kind en,,.}{daar had meende zij niemand}{iets over te zeggen}\\

\haiku{hij Toenggah, waar de.}{beesten later heengevoerd}{en geslacht waren}\\

\haiku{Onmiddellijk na,.}{zijn vertrek ontspon zich een}{levendig gesprek}\\

\haiku{Na eenigen tijd kwam.}{de wedono weder te}{voorschijn en wenkte}\\

\haiku{- Neen kandjeng,.}{fluisterde het inlandsche}{Hoofd schier onhoorbaar}\\

\haiku{- Jij woont immers in,.}{Toenggah vroeg de ambtenaar}{hem in goed Javaansch}\\

\haiku{- Als het geen lieden,.}{van hier zijn zijn het stellig}{menschen uit Toenggah}\\

\haiku{Bij zijne woning,.}{gekomen zag hij Sainum}{op het erf zitten}\\

\haiku{Met zijn maal gereed,:}{wierp Troeno het leege blad in}{een hoek en vroeg zacht}\\

\haiku{- Morgenochtend moet,.}{ik naar Tjandoeredjo daar}{zal ik eens vragen}\\

\haiku{De man bleek al te,.}{weten dat Sainum door haar}{man verlaten was}\\

\haiku{Als de assistent,.}{ontslagen is betaalt hij}{natuurlijk niet meer}\\

\haiku{Hij zeide niets meer,.}{nam zijn arit en ging naar het}{huis van den loerah}\\

\haiku{Wie het gestolen.}{heeft is bekend en er is}{al een getuige}\\

\haiku{Ik heb al verklaard,.}{dat ik niet gezien had dat}{het hout vervoerd is}\\

\haiku{Djogoreso, die man blijft,.}{hier laat hem niet weggaan en}{met niemand spreken}\\

\haiku{Van dat geheele.}{verhaal van den loerah was}{natuurlijk niets waar}\\

\haiku{- Die Wariokromo?}{is gisteren immers naar}{het distrikt gebracht}\\

\haiku{Of is het misschien,?}{een welgesteld man als hij}{niet eens een huis heeft}\\

\haiku{Mandoer Kasan was reeds,.}{naar het werk zijne dochter}{vond hij alleen thuis}\\

\haiku{Zijn tweede zoon ging.}{trouwen en nu werd zijn huis}{te klein voor allen}\\

\haiku{7Offermaaltijd op.}{den avond voorafgaande aan}{den 21sten Poewasa}\\

\haiku{8Offermaaltijd op.}{den avond voorafgaande aan}{den 27sten Poewasa}\\

\section{Martinus H. Boelen}

\subsection{Uit: Jongens uit Bergrust}

\haiku{wij met z'n twee\"en,,,,.}{Joop Piet Bertus en Johan vind}{ik meer dan genoeg}\\

\haiku{{\textquoteleft}Vrienden,{\textquoteright} vervolgde,,:}{Wim trots dat z'n idee zoveel}{bijval ondervond}\\

\haiku{En als jullie even,{\textquoteright}.}{je snater houden zal ik}{het je voorlezen}\\

\haiku{Dit aantal mag niet.}{overschreden worden dan met}{aller goedvinden}\\

\haiku{Waarin de club tot,.}{de ontdekking komt dat ze}{geen voorzitter heeft}\\

\haiku{Onze ouders zijn,.}{immers veel te bang dat er}{iets met ons gebeurt}\\

\haiku{Plots klonk de stem van:}{Karel benauwd van achter}{een hogen stapel}\\

\haiku{{\textquoteright} Behoedzaam, voetje voor,,.}{voetje met ingehouden adem}{gingen ze voorwaarts}\\

\haiku{Bijna hadden ze,.}{het kamertje weer bereikt}{toen Karel bleef staan}\\

\haiku{Hier drink maar even en.}{dan zullen wij meteen je}{ouders waarschuwen}\\

\haiku{{\textquoteright} Weifelend stond de.}{Baron stil en keek vragend}{naar den inspecteur}\\

\haiku{Grond er over en we.}{hebben de mooiste schuilplaats}{die je denken kunt}\\

\haiku{Die man lachte hen, '.}{ook nog uit vond hij net of}{t zo lollig was}\\

\haiku{Ze stonden er zelf,.}{verbluft van zo mooi als de}{hut verborgen was}\\

\haiku{Ze blijven nog een,.}{tijdje zitten bomen v\'o\'or}{de vreemde opstapt}\\

\haiku{Zover als Joop wist,,.}{had alleen Wim lucifers}{dat was de afspraak}\\

\haiku{Intussen hadden.}{de andere clubleden}{zich duchtig geweerd}\\

\haiku{{\textquoteright} {\textquoteleft}Vanzelfsprekend,{\textquoteright} vond, {\textquoteleft}.}{de anderedie zijn er}{bij inbegrepen}\\

\haiku{{\textquoteleft}Kijk eens jongens, er,?}{komt rook uit de schoorsteen zou}{de hut bewoond zijn}\\

\haiku{Hier woonde in dien.}{tijd Heer Jan van Lichtenberg}{met z'n gemalin}\\

\haiku{{\textquoteright} Haastig daalden ze.}{af in de richting van de}{aangewezen cel}\\

\section{Jan van Boendale}

\subsection{Uit: Boek van de wraak Gods}

\haiku{Want zoals David,.}{ons verzekert komt daaruit}{alle wijsheid voort}\\

\haiku{{\textquotedblleft}U zult geen zeven,{\textquotedblright}.}{maal vergeven maar zeven}{maal zeventig maal}\\

\haiku{Daartoe zijn ze op.}{grond van hun heilige plicht}{altijd gebonden}\\

\haiku{Weet dat prelaten.}{eropuit zijn alles in}{handen te krijgen}\\

\haiku{van al het bezit).}{is het meeste immers in}{handen van de Kerk}\\

\haiku{De vierde is de,.}{valse eed waarmee men God}{groot verdriet aandoet}\\

\haiku{Maar dit gebeurt niet,.}{overal nochtans gebeurt het}{vaker dan goed is}\\

\haiku{U kunt er evenwel.}{staat op maken dat ik u}{de waarheid zeg}\\

\haiku{Als zijn vader in,.}{aanzien staat is dat des te}{eervoller voor hem}\\

\haiku{Naderhand lieten.}{ze u het bezit na dat}{God hun had verleend}\\

\haiku{Ik betwijfel ten.}{zeerste of landsheren in}{de hemel komen}\\

\haiku{Dat deed hij omdat.}{zij verstandig was en een}{mooi voorkomen had}\\

\haiku{Dit was de moeder,.}{van Alexander zoals ik u}{hiervoor meedeelde}\\

\haiku{De zwaarste straf van.}{alle zal in het laatste}{tijdperk geschieden}\\

\haiku{5 Over de slechtheid:}{van de wereld Onze Heer}{Jezus Christus zegt}\\

\haiku{De mannen dragen,.}{bovendien korte kleren}{tot aan hun middel}\\

\haiku{Ze zullen niets meer.}{bezitten om hun lichaam}{mee te bedekken}\\

\haiku{Toch had hij daar de.}{beschikking over nauwelijks}{\'e\'en op vijf mannen}\\

\haiku{3 Een vertelling}{over een heer die een klooster}{overlast bezorgde}\\

\haiku{{\textquoteleft}Begrijpt u niet dat?}{Ik niet graag verlies wat Ik}{zo duur moest kopen}\\

\haiku{De een is dus de.}{ander zijn broer en ik ben}{hun beider moeder}\\

\haiku{Deel III [vervolg] 13}{Over een strijd die zich voordeed}{in het land van Luik}\\

\haiku{Daar viel degene.}{die de zak op zijn hoofd droeg}{en was op slag dood}\\

\haiku{{\textquoteleft}Vrouw,{\textquoteright} sprak hij, {\textquoteleft}ik ben.}{zo geschrokken dat ik niet}{durfde te spreken}\\

\haiku{Ga onverwijld de.}{kapel binnen die daar staat}{en draag de mis op}\\

\haiku{Zo moeten alle.}{zielen gewis naar de hel}{of naar de hemel}\\

\haiku{{\textquoteright} [...] De zwaarste straf van.}{alle zal in het laatste}{tijdperk geschieden}\\

\haiku{Daaruit volgt evenwel.}{dat de versie uit 1346 n{\'\i}et}{voor hem bedoeld was}\\

\haiku{Boendale moet zich.}{dit van tevoren hebben}{gerealiseerd}\\

\haiku{Hij zegt als gevolg.}{van zijn werk en leeftijd ziek}{te zijn geworden}\\

\haiku{Dat had hij zelf  .}{welbeschouwd ook gedaan met}{de eerste versie}\\

\haiku{zijn bronnen zijn te.}{vertrouwen en het publiek}{kan h\'em vertrouwen}\\

\haiku{zie Willem v, graaf,:}{van Holland Willem ii graaf}{van Henegouwen}\\

\subsection{Uit: Lekenspiegel}

\haiku{Het firmament draait.}{van nature altijd rond}{en mag nooit stilstaan}\\

\haiku{Want het firmament.}{draait binnen \'e\'en dag en \'e\'en}{nacht helemaal rond}\\

\haiku{En als zij dan in,.}{het oosten opkomt dan ziet}{men de dageraad}\\

\haiku{God schiep de mens niet,:}{alleen uit aarde maar uit}{de vier elementen}\\

\haiku{Dan verliest hij kracht.}{en verstand die Onze Heer}{hem gegeven had}\\

\haiku{Als ze verdrogen,.}{of omvallen gaat hun ziel}{eveneens te gronde}\\

\haiku{In Hem bevinden.}{zich altijd genoegens en}{volmaakte vreugde}\\

\haiku{Kijk nu toch eens wat.}{een ellende er voortkwam}{uit Eva's kwetsbaarheid}\\

\haiku{Hoe meer men met hen,.}{omgaat hoe meer  men te}{schande wordt gemaakt}\\

\haiku{{\textquoteright} Zie, kinderen, en}{merk allemaal op welke}{grote ellende}\\

\haiku{De rook ging kaarsrecht.}{hemelwaarts als een offer}{dat God behaagde}\\

\haiku{Men kan diegenen.}{niet erger vervloeken dan}{met een lang leven}\\

\haiku{En na de kwelling,.}{op aarde branden zij in}{het eeuwige vuur}\\

\haiku{En als zij sterven.}{geeft God hun het hemelrijk}{voor hun zaligheid}\\

\haiku{voor dergelijke,.}{smerige bedorven wijn}{moet men oppassen}\\

\haiku{Daarmee verliest de,,.}{man zijn bezit zijn eer zijn}{ziel en zijn leven}\\

\haiku{ze gebruiken hem.}{met mate en ze worden}{er scherpzinnig van}\\

\haiku{Zij vaardigden het:}{volgende statuut uit voor}{het hele volk}\\

\haiku{Het vierde rijk is,.}{het rijk van Rome dat bij}{Romulus begint}\\

\haiku{Babyloni\"e was,.}{het eerste het machtigste}{en het fraaiste rijk}\\

\haiku{Het Romeinse rijk,.}{was het laatste rijk van de}{vier maar het beste}\\

\haiku{Ieder rijk dat men,.}{uiteenscheurt zal van macht en}{kracht worden beroofd}\\

\haiku{Hij was getrouwd met,.}{Lavinia de dochter van}{koning Latinus}\\

\haiku{Dat zou jammer zijn,,.}{wees daarvan overtuigd en ik}{geloof het ook niet}\\

\haiku{Hoe men het ook wendt,.}{of keert bekering moet uit}{vrije wil geschieden}\\

\haiku{Toen de wereld nog,.}{maar net bestond heeft God heel}{snel wraak genomen}\\

\haiku{Maria bleef in de,.}{tempel waar een engel haar}{dagelijks brood bracht}\\

\haiku{V\'o\'or de geboorte:}{wil een vroedvrouw meten hoe}{het met Maria is}\\

\haiku{zij vertellen aan.}{de drie koningen dat de}{ster verschenen is}\\

\haiku{Het ene brood behoort,.}{aan de ziel toe dat is het}{geestelijke brood}\\

\haiku{Deze punten zal.}{ik achtereenvolgens aan}{de orde stellen}\\

\haiku{De priester die de,.}{mis opdraagt symboliseert}{Christus aan het kruis}\\

\haiku{heiligen schreven,.}{het epistel maar God zelf spreekt}{in het evangelie}\\

\haiku{De aarde schudde:}{bij zijn passie en de zon}{liet  verstek gaan}\\

\haiku{Wereldse wijsheid.}{is niets anders dan dwaasheid}{in de ogen van God}\\

\haiku{zelfs thuis bij de open.}{haard of in een woest bos waar}{niemand het kan zien}\\

\haiku{Ondankbaarheid is.}{de grootste lompheid die op}{de aarde bestaat}\\

\haiku{zowel God als de.}{mensen u. Hieruit valt veel}{winst te behalen}\\

\haiku{U moet zich evenmin.}{zorgen maken over de dood}{of er bang voor zijn}\\

\haiku{Daarom liet  ze,.}{een bed opzetten in een}{stal ver van de haard}\\

\haiku{In het volgende.}{voorbeeld zal dat duidelijk}{gemaakt worden}\\

\haiku{Aldus kan de ziel,.}{behouden blijven zoals}{de wijzen schrijven}\\

\haiku{Want kwaadheid verlamt.}{uw verstand zodat u de}{waarheid niet meer ziet}\\

\haiku{Er was eens een non.}{die al vele jaren in}{een klooster verbleef}\\

\haiku{We meenden nog een.}{keer pret en plezier met je}{te kunnen maken}\\

\haiku{daar doe ik mezelf,.}{een genoegen mee maar ik}{verlies me in God}\\

\haiku{{\textquoteright} - {\textquoteleft}Nee,{\textquoteright} antwoordde de, {\textquoteleft}.}{jongemanik ben nu een}{ander mens dan eerst}\\

\haiku{Wanneer u bij een,:}{vreemde aan tafel zit moet}{u niet veel praten}\\

\haiku{Dat zou hen immers.}{nog meer pijn doen en kwetsen}{dan al hun rampspoed}\\

\haiku{Aldus kan iemand.}{dat beter aan als hij op}{een vreemde plaats komt}\\

\haiku{Te allen tijde.}{moet u het advies van uw}{vrienden opvolgen}\\

\haiku{Iemand die ergens,.}{om vraagt wordt makkelijk op}{zijn teentjes getrapt}\\

\haiku{Houd geheimen voor,!}{haar verborgen dat is mijns}{inziens wijsheid}\\

\haiku{Raadslieden moeten:}{vooral op de volgende}{twee zaken acht slaan}\\

\haiku{Zo verspreiden zij,.}{het werk van de duivel wat}{de duivel graag hoort}\\

\haiku{Daarom moet niemand.}{zich inlaten met zaken}{die hem niet aangaan}\\

\haiku{Eertijds vroeg iemand:}{aan een zeer geleerde klerk}{wat het beste was}\\

\haiku{Want een arme is.}{dag en nacht bezig zich in}{leven te houden}\\

\haiku{Van zo'n vrouw zal men,.}{niets dan schade schande en}{ellende hebben}\\

\haiku{Dit is een van de,.}{belangrijkste geboden}{van God Onze Heer}\\

\haiku{het zou een koning,.}{het leven kunnen kosten}{als hij dat verdient}\\

\haiku{Ad 3 De derde:}{eigenschap gaat over wijsheid}{en bescheidenheid}\\

\haiku{Maar als de heer bang,.}{is en terugwijkt verliest}{hij meteen de strijd}\\

\haiku{Want het gespuis zal,.}{zijn land verlaten omdat}{het bang voor hem is}\\

\haiku{zij hebben gezien.}{en gehoord wat een heerser}{rechtens toebehoort}\\

\haiku{De mensen zouden,.}{nog als vee leven als de}{schrijfkunst niet bestond}\\

\haiku{daarin ligt immers.}{onze zaligheid en ons}{geloof besloten}\\

\haiku{In geen enkel vak.}{kan men zo hoog klimmen als}{in de wetenschap}\\

\haiku{Daarmee boeken ze.}{gemakkelijk succes en}{wordt hun naam bekend}\\

\haiku{Tot slot wil ik u.}{vertellen welke soorten}{dichters er bestaan}\\

\haiku{De eerste is de,.}{liefde tot God de Vader}{in het hemelrijk}\\

\haiku{2 weinig spreken;}{en alleen als het zin heeft}{over goede zaken}\\

\haiku{Wie zich aan deze,.}{tien punten houdt mag een wijs}{man genoemd worden}\\

\haiku{Wie zichzelf liefheeft,.}{bemint de deugd en alles}{wat zijn ziel verheft}\\

\haiku{Men vroeg eens aan een.}{geleerde of het zonde}{is lekker te eten}\\

\haiku{Laat nu ieder voor.}{zichzelf uitmaken met wie}{hij het eens is}\\

\haiku{Wie nooit iets verkeerds,.}{deed hoeft ook nooit zijn fouten}{te verbeteren}\\

\haiku{De duivel zal dit;}{alles op sluwe wijze}{bewerkstelligen}\\

\haiku{Spitsvondigheden.}{en argumenten zullen}{geen zin meer hebben}\\

\haiku{Op vele punten.}{wijkt zijn versie af van het}{Latijnse voorbeeld}\\

\haiku{In Boek iii kan men -}{leren volgens welke door}{God gegevenregels}\\

\haiku{Deze tekst lijkt op.}{het lijf gesneden voor een}{schepenfamilie}\\

\haiku{We hebben ervoor.}{gekozen om de structuur}{intact te laten}\\

\section{Jo Boer}

\subsection{Uit: Beeld en spiegelbeeld}

\haiku{Daarom moet je er,,.}{altijd trots op zijn dat je}{Franse bent Louise}\\

\haiku{Zijn zoon had hem zijn,.}{vele en zware zonden}{vergeven zei hij}\\

\haiku{{\textquoteright} Daarbij namen haar,.}{aandachtige heldere}{ogen ook mij scherp op}\\

\haiku{Hij was bang, dat hij.}{een ge{\"\i}nteresseerden}{indruk zou maken}\\

\haiku{Plotseling zei hij -,?:}{slotsom van welken weemoed}{van welk verlangen}\\

\haiku{Maar Charles had hem,.}{buitengesloten toen in}{dien tijd met Wiesje}\\

\haiku{daar rijdt een mijnheer,.}{met een langen baard op die}{ook een priksnor heeft}\\

\haiku{Zijn felle blik gaf,....}{slechts te lezen wat nobel}{was en zieleklaar}\\

\haiku{Een schande was het,,.}{maar zij had hem lief zij dacht}{aan niets anders meer}\\

\haiku{{\textquoteright} Maar in haar hart zal,:}{een kleine bevroren plek}{zijn waardoor zij denkt}\\

\haiku{Volgende maand ben.}{ik achtentwintig en mijn}{kansen gaan voorbij}\\

\haiku{Ik wist, dat ik weer,,.}{als zo vaak tevoren de}{mindere zou zijn}\\

\haiku{Integendeel, haar.}{spoedige vertrek was nog}{mijn enige  troost}\\

\haiku{Waarom laat ik de,?}{kinderen doorgaan met dit}{wrede vreemde spel}\\

\haiku{Ik zag mijn gezicht,.}{in den spiegel als drijvend}{op een donker meer}\\

\haiku{Hij had niet van pak,.}{gewisseld zoals anders}{zijn gewoonte was}\\

\haiku{Was het mogelijk?}{om zo te lijden en toch}{nog door te leven}\\

\haiku{Daarna werd de deur,.}{van de kamer zacht doch zeer}{beslist gesloten}\\

\haiku{Ook Marceline's.}{erfdeel ging onverdeeld in}{mijn bezit over}\\

\haiku{Als je hem zijn gang,.}{laat gaan brandt hij het hele}{Les Vignobles af}\\

\haiku{En mijn moeder is.}{er net de vrouw naar om dat}{huis te bewonen}\\

\haiku{{\textquoteleft}Jij zou begrijpen,,,.}{wat ik bedoelde Louise}{als je haar kende}\\

\haiku{In geen jaren had.}{ik Maman als een deel van}{mijn leven gezien}\\

\haiku{{\textquoteright} {\textquoteleft}Misschien begrijp je,,.}{het niet Louise maar ik houd}{werkelijk van haar}\\

\haiku{Overal, waar mensen,.}{samenkomen ontlaadt zich}{een spanning van leed}\\

\haiku{Van enigen wrok was;}{toen bij Jean-Paul al geen}{sprake meer geweest}\\

\haiku{dat gespring gaf het;}{rhythme van den tijd aan en}{werd tot Gods uurwerk}\\

\haiku{Want er zijn dingen,,.}{die je niet zeggen kunt zelfs}{tegen M\'em\'e niet}\\

\haiku{Aarzelend stak hij.}{zijn handen uit naar het vuur}{om ze te warmen}\\

\haiku{Maar hij wist ook, dat;}{Charles en hij niets van haar}{te vrezen hadden}\\

\haiku{Ik moet Jean-Paul.}{Les Vignobles aanbieden}{in ruil voor Maman}\\

\haiku{Dien middag dacht ik.}{werkelijk nog niet over mijn}{verraad als verraad}\\

\haiku{Ren\'e keek naar het:}{wit weggetrokken gezicht}{van Charles en dacht}\\

\haiku{Zij zijn weer veel te.}{verlegen om naar Gis\`ele's}{partijtje te gaan}\\

\haiku{Daarom moeten we,.}{het doorzetten dat ze aan}{Ren\'e beloofd wordt}\\

\haiku{Ze hadden het over,.}{dien aardigen mijnheer die}{bij jullie logeert}\\

\haiku{Links van hem zag hij,.}{een deur waarschijnlijk de deur}{van een klerenkast}\\

\haiku{{\textquoteright} Twee pijpekrullen.}{kriebelden over zijn hand. De}{kastdeur ging weer dicht}\\

\haiku{{\textquoteright}        VIII Ren\'e.}{de Saint-Vincent lag in}{zijn bed te lezen}\\

\haiku{27 November 19.. {\textquoteleft}.}{Midden in den nacht ben ik}{wakker geworden}\\

\haiku{Zo werd ik, voor \'e\'en,.}{enkelen avond althans de}{vrouw van Jean-Paul}\\

\haiku{- zat tegenover haar.}{met zijn arm om den hals van}{de geit geslagen}\\

\haiku{{\textquoteleft}Jullie moet je gauw,.}{gaan aankleden wij rijden}{om half negen weg}\\

\haiku{{\textquoteright} {\textquoteleft}Als de sneeuw wel z\'o,,?}{hoog ligt dan kunnen we niet}{verder h\`e tante}\\

\haiku{Er kwam geen einde,.}{aan den rit waarvan een leeg}{huis het einddoel was}\\

\haiku{Dat zij andere,.}{dingen zien dan wat hier den}{laatsten tijd gebeurt}\\

\haiku{Waarschijnlijk wel, want,,.}{ook de hond beneden in}{de hal sloeg niet aan}\\

\haiku{Ik holde de trap,.}{af de donkere hal door}{naar de buitendeur}\\

\haiku{De wijnen waren,;}{uitgelopen de appels}{en peren bloeiden}\\

\haiku{Ik wilde niet, dat.}{je eerste indrukken hier}{pijnlijk zouden zijn}\\

\haiku{Dan ben ik zeker,,.}{van je van een zekerheid}{die ruist door mijn bloed}\\

\haiku{{\textquoteleft}Neen,{\textquoteright} zei hij hardop, {\textquoteleft},.}{in de stiltedit is het}{niet wat ik bedoel}\\

\haiku{En met een diepen,,:}{zucht die van heel ver scheen te}{komen dacht hij weer}\\

\haiku{Waarom hadden zijn?}{broer en hij de estate}{eigenlijk verkocht}\\

\haiku{{\textquoteleft}Vraag je helemaal,,?}{niet waarom ik gekomen}{ben Laperade}\\

\haiku{Ik zag een middel - -;}{meende ik om jullie in}{de hand te houden}\\

\subsection{Uit: De erfgenaam}

\haiku{Iets dauwigs had die,.}{blik als van ongeplukte}{donkere druiven}\\

\haiku{Achter deuren en.}{ramen klonk het getik van}{lepels in kommen}\\

\haiku{Een uil zat in de,.}{spar aan den ingang lachte}{hoonend en vloog weg}\\

\haiku{Het maakte me zoo,:}{kribbig dat mijn stem oversloeg}{van drift toen ik vroeg}\\

\haiku{Zij waren aan haar.}{gestuurd en ze was mij een}{verklaring schuldig}\\

\haiku{Maar wanneer ik haar,.}{stilte hoorde wist ik dat}{ik zwijgen moest}\\

\haiku{Een zoon is nader,,.}{dan een zuster dat zegt het}{bloed dat zegt de wet}\\

\haiku{Ik wist zelf niet meer,,?}{was het een kind was het een}{dom levenloos ding}\\

\haiku{De woorden vormden.}{zich duidelijk en klaar in}{Clementine's hoofd}\\

\haiku{De boerderij lag.}{in een gehucht hier zeven}{uur sporen vandaan}\\

\haiku{Het is het eenige.}{contact dat ik eigenlijk}{met hem gehad heb}\\

\haiku{Mijn zoon was een klein,,.}{wild dier geworden bang voor}{slaag mager en valsch}\\

\haiku{Maria Bakkersvrouw.}{had een brief gekregen van}{haar zoon uit Detroit}\\

\haiku{Ik dacht aan Maman.}{en aan een kanten kleed dat}{zij eens had geklost}\\

\haiku{En soms denk ik, dat.}{David het betreurt dat hij}{mij genomen heeft}\\

\haiku{Er is immers geen,.}{zekerheid en je denkt zelf}{dat het niet waar is}\\

\haiku{de cel en dan de,.}{angst en dan op een morgen}{een deur die open gaat}\\

\haiku{Nee, dit keer zou zij.}{zich verdedigen op haar}{eigen wijze}\\

\haiku{Zelfs het maanlicht werd,.}{verborgen door nevels die}{opstegen uit zee}\\

\haiku{Het was zeven uur '.}{s avonds en alle luiken}{waren gesloten}\\

\haiku{Je bent gemaakt om.}{te schenken en zonder mij}{kan je dat niet eens}\\

\haiku{Het is toch niet een,,?}{zondige liefde die je}{gedreven heeft wel}\\

\haiku{{\textquoteright} En David in zijn:}{onredelijke lust tot}{kwellen vervolgde}\\

\haiku{In de plooien van.}{haar rok waren haar handen}{tot vuisten gebald}\\

\haiku{Maar een kind groeide.}{op tot jong meisje en van}{jong meisje tot vrouw}\\

\haiku{Hij had den indruk,.}{dat zij hem verwachtte al}{deed zij heel verrast}\\

\haiku{verpakt een kleine.}{noodelooze en venijnige wrok}{klaar om te bijten}\\

\haiku{{\textquoteright} ~ Mijnheer pastoor.}{wilde inzicht hebben in}{dat oude drama}\\

\haiku{Ja, hij zou me daar.}{sentimenteel gaan worden}{op zijn ouden dag}\\

\haiku{De buitenwereld.}{verandert om ons heen en}{wijzigt zich niet meer}\\

\haiku{{\textquoteright} Zijn vader en zijn;}{vaders vader en verder}{het verleden in}\\

\haiku{{\textquoteright} Een jaar later had.}{hij de rouwberichten huis}{aan huis rondgebracht}\\

\haiku{{\textquoteleft}Luister, Constance,.}{wij zullen den brief openstoomen}{boven den ketel}\\

\haiku{neen, zijn moeder was,,.}{dood en trouwens hij had nooit}{een moeder gehad}\\

\haiku{De cipier zei, dat,.}{hij geschreeuwd had dat hooren}{en zien je verging}\\

\haiku{Hij zendt een foto.}{en vraagt Clementine of}{zij haar zoon herkent}\\

\haiku{{\textquoteleft}Het is toch een mooi{\textquoteright}.}{bezit zei Clementine}{hardop tot zich zelf}\\

\haiku{zekerheid omtrent.}{de identiteit van No. A}{384178 bestaat er niet}\\

\haiku{Hierop berustte.}{dan ook de verdediging}{van zijn advocaat}\\

\haiku{De  eieren{\textquoteright},:}{waren stuk antwoordde zij}{vaag en ontwijkend}\\

\haiku{{\textquoteright} {\textquoteleft}Toch heeft hij mijn oogen{\textquoteright},,.}{had David geantwoord half}{boos en half lachend}\\

\haiku{Dat is een smaragd,.}{zei het Clementientje van}{de vele beesten}\\

\haiku{Het was nog steeds de.}{vliegende duif voor de kerk}{van Bartholomeus}\\

\haiku{Het is nog uit de.}{erfenis van die arme}{tante Ursule}\\

\haiku{De ratten renden.}{over den zolder en de trap}{kraakte van het vocht}\\

\haiku{Een blauwige rook.}{vulde de kamer en deed}{Constance hoesten}\\

\haiku{Een zenuwachtig.}{plapperend vuurtje overwon}{eindelijk den rook}\\

\haiku{{\textquoteright} Constance kende.}{David vanaf den dag dat}{hij geboren werd}\\

\haiku{{\textquoteright} {\textquoteleft}Zoo zwart{\textquoteright}, zei hij, en, {\textquoteleft}.}{nam een teug brandewijnzoo}{zwart als de zonde}\\

\haiku{Zij moest iets zeggen,.}{waardoor zij allemaal naar}{haar kijken zouden}\\

\haiku{Clementine drinkt.}{te veel en Anne is bang}{voor Clementine}\\

\haiku{Een schaduw gleed over,.}{den cupido een briefkaart}{viel uit den spiegel}\\

\haiku{Het kind was wakker.}{geworden van den slag en}{begon te huilen}\\

\haiku{Een donkere vrouw.}{zal Uw ergernis wekken}{en U bedriegen}\\

\haiku{Constance legde.}{alle kaarten in een krans}{om hartevrouw}\\

\haiku{De zevende kaart,,.}{nichtje leg je omgekeerd}{op hartevrouw}\\

\haiku{Daar zat zij stil en.}{in\'e\'engedoken naast}{Simon Vrachtrijder}\\

\haiku{Je stond hem op te,.}{wachten in zijn boomgaard toen}{we hier aankwamen}\\

\haiku{en sedert dien is.}{David nooit meer als vroeger}{tegen me geweest}\\

\haiku{Zij zou nu naar huis,:}{gaan haar hoofd op zijn knie\"en}{leggen en zeggen}\\

\haiku{Wij zullen alles,.}{samen doen maar laat het weer}{goed zijn tusschen ons}\\

\haiku{Maar bij god, als die,.}{oude dood is dan sleep ik}{jou voor het gerecht}\\

\haiku{Een beetje moe, een,.}{beetje geschrokken omdat}{ze gevallen was}\\

\haiku{{\textquoteright} Een glas trilde in,,.}{de glazenkast het was zoo'n}{gek klagend geluid}\\

\haiku{Steenen boonen... dan kon.}{je gelijk wel kiezelsteenen}{op je bord scheppen}\\

\haiku{Haar gezicht is weer.}{jong geworden en het is}{net of zij glimlacht}\\

\haiku{Zij lag daar in haar,.}{zilverglans het haar in twee}{vlechten gescheiden}\\

\haiku{Maar ik probeer mij.}{te onderwerpen aan den}{wil van mijn zuster}\\

\haiku{Hij herkende het.}{schip zooals hij bijna alle}{schepen herkende}\\

\haiku{Eerst wachten tot de.}{winkels open gingen om naar}{den barbier te gaan}\\

\haiku{Ik haal rat onder,.}{stolp weg laat loopen en rat}{gaat in gaatje drie}\\

\haiku{Want meer dan hij van,.}{zijn moeder gehouden had}{hield hij van zijn zoon}\\

\haiku{Hij wist dat hij het,.}{niet doen zou want hij was de}{vader van zijn zoon}\\

\haiku{Had ook zijn broer niet?}{een stuk van zijn wijnlanden}{moeten verkoopen}\\

\haiku{De dokter was heel,,.}{tevreden ja hij gaat op}{mijn moeder lijken}\\

\haiku{Er is zoo weinig;}{toe noodig om het onkruid te}{laten woekeren}\\

\haiku{{\textquoteright} {\textquoteleft}Zij lachte op een{\textquoteright},, {\textquoteleft}}{griezelige wijze zei}{de jongen later}\\

\haiku{Er waren zeker.}{al twintig menschen in de}{sterfkamer bijeen}\\

\haiku{Ik keek om mij heen}{in dat griezelige huis}{en ik bedacht mij}\\

\haiku{Herinner je je,,?}{hoe David zei dat wij haar}{gevonden hadden}\\

\haiku{Het was niet grooter.}{dan een sinaasappelkrat}{en niet half zoo diep}\\

\haiku{Zij heeft de kasten.}{gesloten en de sleutels}{op tafel gelegd}\\

\haiku{Zij heeft haar hand in.}{de mijne gelegd en zoo}{zijn wij blijven staan}\\

\haiku{In de stad of in,.}{mijn dorp of in een ander}{land waar je maar wilt}\\

\haiku{Zij was te krom of,,.}{te recht te dik of te dun}{te jong of te oud}\\

\haiku{Er is w\'el wind, zei,.}{de boom en dreig me niet zoo}{met die paraplu}\\

\haiku{De schemering wordt,.}{dieper en dieper en toch}{is het nog geen avond}\\

\subsection{Uit: Kruis of munt}

\haiku{Bruno werkte dag.}{en nacht om orde op zijn}{zaken te stellen}\\

\haiku{Haar was het immers:}{duidelijk wat Agatha en}{Robert nog ontging}\\

\haiku{De oude vrouw had.}{haar handen gevouwen over}{haar reticule}\\

\haiku{En vreemd, helder en:}{smekend vroeg de stem van de}{vroegere Agatha}\\

\haiku{daar had helemaal.}{geen uitdrukking onder haar}{geel zijden krullen}\\

\haiku{{\textquoteright} Hij zei m{\'\i}jn eisen,.}{en niet j\'ouw eisen of zelfs}{maar \'onze eisen}\\

\haiku{Het verveelde het,.}{kind nooit dat die schoenen zo}{weinig bewogen}\\

\haiku{De letters staan weer.}{in een vlakje apart met een}{randje er omheen}\\

\haiku{Voor Jules Verne,,.}{is Jopie die niet lezen}{kan nog veel te klein}\\

\haiku{Zij gaan naar haar toe,:}{kijken haar voor de eerste}{maal werkelijk aan}\\

\haiku{Toen de juffrouw dus:}{de volgende dag aan het}{kleine meisje vroeg}\\

\haiku{{\textquoteleft}En, Jopie, kun je,?}{me vertellen hoe onze}{wereld er uitziet}\\

\haiku{{\textquoteright} Dan weer het geluid,,,.}{van blote voeten het bed}{dat piepte stilte}\\

\haiku{Maar je Pa is bij,.}{je weggelopen omdat}{je Ma een spook is}\\

\haiku{Een ding slechts hadden:}{deze uiteenlopende}{karakters gemeen}\\

\haiku{Ik krijg ze pas te,,.}{zien dacht zij spottend wanneer}{zij afgericht zijn}\\

\haiku{De portretten, die,.}{er van haar uit die tijd over}{zijn zeggen van wel}\\

\haiku{In mijn huis wordt die,{\textquoteright}.}{vuiligheid niet gerookt zei}{de oude heer bars}\\

\haiku{Ongemerkt, zonder,.}{dat zij er op betrapt kon}{worden hielp zij hem}\\

\haiku{Zij verlangde naar,.}{Bernards thuiskomst als naar iets}{vaags iets onbestemds}\\

\haiku{De dokters denken,....}{dat je kleine meisje niet}{lang te leven heeft}\\

\haiku{Zijn zoon, die op de,.}{kade stond te wuiven had}{hij afgeschreven}\\

\haiku{meegenomen had,.}{naar de bergen zag zij het}{kindje nu veel meer}\\

\haiku{Is de schorpioen?}{verantwoordelijk voor het}{vergif in zijn staart}\\

\haiku{Kan het sombere,,?}{kleine meisje daar v\'o\'or haar}{werkelijk lachen}\\

\haiku{Hij kriebelt met zijn.}{lange hengel in de hals}{van de lantaarnpaal}\\

\haiku{Het kind wist heel goed,;}{dat zij het poesje niet mee}{naar huis kon nemen}\\

\haiku{Er werd, behalve,;}{de twee meiden een kokin}{in dienst genomen}\\

\haiku{Zij boog zich over de.}{trapleuning heen en keek de}{donkere gang in}\\

\haiku{De volgende dag.}{was zij deze avondlijke}{doodsangst vergeten}\\

\haiku{{\textquoteright} Zij wist niet, dat het.}{kind geruisloos achter haar}{aan geslopen was}\\

\haiku{Als het kind er niet....}{was om een oude vrouw wat}{op te vrolijken}\\

\haiku{Johanna's trotse {\textquoteleft}.}{besluitin dit huis wordt niets}{klandestien gekocht}\\

\haiku{Een oorlog was over....}{de wereld gegaan en hier}{was niets veranderd}\\

\haiku{Jij zult me toch niet,,?}{alleen laten is het wel}{m'n lieve jongen}\\

\haiku{XVII De oorlog had;}{sommige jarenlange}{vetes uitgewist}\\

\haiku{De zusters hadden,,;}{beledigd haar handen van}{hem afgetrokken}\\

\haiku{Ook bestond er geen,;}{twijfel of het kind zou de}{naam veranderen}\\

\haiku{het fototje is al,.}{een jaar oud maar zij is nog}{niet veel veranderd}\\

\haiku{Je kunt in deze,.}{natte kleren niet weer door}{de regen jongen}\\

\haiku{{\textquoteright} Maar hij was zelf te.}{ongedurig om op haar}{antwoord te wachten}\\

\haiku{{\textquoteleft}Kijk, moet je sien, s\`eg....{\textquoteright}:}{Voorzichtig lichtte hij de}{deksel van de doos}\\

\haiku{Schuw en aandachtig:}{nam hij haar met zijn vreemde}{schuinstaande ogen op}\\

\haiku{{\textquoteleft}Ma, je kunt je je.}{Sint Nikolaasfeestje wel}{uit je hoofd zetten}\\

\haiku{De lucht was donker.}{grijs en ongemerkt begon}{het weer te sneeuwen}\\

\haiku{Zij droeg haar paarse.}{kamerjapon en haar haar}{was onopgemaakt}\\

\haiku{in haar paars wollen.}{ochtendjapon bracht zij de}{dagen peinzend door}\\

\haiku{hoe moeilijk het kind,.}{was hypernerveus en zeer}{zwak van gezondheid}\\

\haiku{Het kind had nog steeds,....}{koorts haar arme moeder kon}{niet alleen blijven}\\

\haiku{{\textquoteright} Eerst was er nog geen;}{verandering merkbaar op}{haar moeders gelaat}\\

\haiku{Zelfs nu hij dood is,,.}{gelooft U het niet alleen}{omdat Ik het zeg}\\

\haiku{Aan de kant van het,.}{Frankenslag natuurlijk maar}{toch was het angstig}\\

\haiku{En daarom had haar:}{moeder ook altijd over haar}{vader gezwegen}\\

\haiku{Zij kwetste en wondde;}{Aletta met plagerijen}{van eigen vinding}\\

\haiku{nimmer met de naam.}{van Bernard Landman of met}{wat zij over hem wist}\\

\haiku{De tafel was niet,,,.}{gedekt op de schoorsteen lag}{dreigend een open brief}\\

\haiku{Als je je driftig.}{maakte werd je lelijk en}{verried je jezelf}\\

\haiku{{\textquoteleft}En toch ben ik er,....}{van overtuigd dat je het goed}{met mijn meisje meent}\\

\haiku{Dan ben je student,....}{dan beteken je al wat}{in de maatschappij}\\

\haiku{Mama mag niet te,.}{weten komen dat ik U}{geschreven heb}\\

\haiku{Als ik je nu een,?}{hond beloof zul je dan niet}{van me weglopen}\\

\haiku{{\textquoteright} {\textquoteleft}Dat betekent dus,,.}{dat je aan dingen denkt die}{ik niet weten mag}\\

\haiku{de dochter, die met....}{haar samenwoonde is uit}{het raam gesprongen}\\

\haiku{Boog zij niet naar haar,....}{moeders wensen dan zou het}{het laatste worden}\\

\haiku{Op dat moment, elf,.}{uur in de morgen koos het}{land de derde deur}\\

\haiku{'s Morgens in bed.}{reeds had zij het geluid van}{de misthoorn gehoord}\\

\subsection{Uit: De vertroosting van het troosteloze}

\haiku{WIJ liepen op den,.}{landweg over den olijfberg mijn}{kameraad en ik}\\

\haiku{Te midden van al.}{die planten had Jozef zijn}{graf laten maken}\\

\haiku{Dikwijls op warme.}{zomeravonden zat hij daar}{voor zijn graf en dacht}\\

\haiku{Hij keek mij rustig:}{met zijn donkere oogen aan}{en zeide nogmaals}\\

\haiku{Dagenlang speelde.}{hij in de zon voor de grot}{en was gelukkig}\\

\haiku{Zij antwoordde niet,.}{maar bewoog haar hand heen en}{weer in het water}\\

\haiku{Zij zag zijn schaduw.}{zich afteekenen tegen het}{glanzen der vensters}\\

\haiku{Zij was alleen nog.}{maar een brok koude in een}{hostiele wereld}\\

\haiku{naar die gebogen,,.}{zwartsatijnen figuur die}{ons zwijgend volgde}\\

\haiku{Hij kwam van een dorp,.}{dat vlak bij Parijs ligt en}{dat Bellevue heet}\\

\haiku{Wat begrijpen wij,?}{van de nooden die ons dwingen}{tot onze daden}\\

\haiku{Ik heb de wereld.}{gezien door het beperkte}{prisma van mijn oogen}\\

\haiku{Ik zag het als een.}{soort verraad van mijzelf aan}{de buitenwereld}\\

\haiku{Maar zij moet mij toch,.}{gehoord hebben want zij keek}{op en glimlachte}\\

\haiku{all\'e\'en het kind - was.}{de wereld anders dan ik}{haar ooit gekend had}\\

\haiku{Besefte ik toen?}{al wat dit bezoek in mij}{veranderen zou}\\

\haiku{Was het de weemoed,?}{van de oude vrouw die mij}{langzaam omhulde}\\

\section{Jan L. de Boer}

\subsection{Uit: De erfdochter van de Doorwerth}

\haiku{Het oogenblik scheen,....}{nabij dat dit zijn lichten}{last zou afwerpen}\\

\haiku{Hij nam met ruwen.}{zwier zijn breedgeranden hoed}{voor haar af en boog}\\

\haiku{{\textquoteright} Met een paar woorden;}{gaf Otto opheldering}{van het gebeurde}\\

\haiku{De steden en de.}{burgerstand konden echter}{vooruitgang boeken}\\

\haiku{De geestelijke.}{opende de deur en verdween}{in zijn kamer}\\

\haiku{hoe wreed leden van.}{\'e\'en huisgezin vaak tegen}{elkaar kunnen zijn}\\

\haiku{Lucht en bladeren -.}{en grachtwater alles scheen}{met vuur overgoten}\\

\haiku{In zijn gebogen.}{houding en bewegingen}{lag iets katachtigs}\\

\haiku{In zijn blauwe oogen.}{lag de uitdrukking van den}{peinzer en dichter}\\

\haiku{{\textquoteleft}Eens toefde ik bij,!}{menschen die het zingen en}{spelen liefhadden}\\

\haiku{Voor een laatst vaarwel.}{reed hij het bergpad op naar}{Rinaldi's sterkte}\\

\haiku{Het lieve kind is....}{voor vreugde en levenslust}{geboren en toch}\\

\haiku{Behendig in het.}{gevecht en even bedreven}{in de hoofsche kunst}\\

\haiku{Daar is geen dwaling,!}{mogelijk waar de Meester}{z\`elf aan het roer staat}\\

\haiku{Wie d\`at doet, keert zich.}{van Christus en de oude}{Gemeente-idee af}\\

\haiku{Maar welke plannen?}{rijpten er thans in het brein}{van den Jezu{\"\i}et}\\

\haiku{hij onopgemerkt.}{een gesprek van Walravia}{en haar grootmoeder}\\

\haiku{Hij droeg hooge laarzen,,;}{een donkere blauwe kleeding}{en bruinen mantel}\\

\haiku{zijn vuist viel krachtig:}{op de tafel en dreigend}{luidde zijn antwoord}\\

\haiku{Ik heb ruim de helft.}{verloren van hetgeen ik}{u reeds ter leen vroeg}\\

\haiku{In de dingen van.}{het geloof toonde zij een}{strenge opvatting}\\

\haiku{Ge weet gevoelens.}{van belangen en van wat}{recht is te scheiden}\\

\haiku{{\textquoteleft}Ik mag die dame,{\textquoteright}.}{niet zei Mevrouw Van Voorst na}{een korte stilte}\\

\haiku{Er was een tijd dat,.}{zij hem ontweek ondanks zijn}{openlijke hulde}\\

\haiku{{\textquoteleft}Als u d\`at gelukt, -!}{kom dan vrij tot mij ik zal}{u niet afwijzen}\\

\haiku{zij schrok wakker uit -.}{haar gemijmer Marten Loks}{stond tegenover haar}\\

\haiku{Hoe vaak heb ik je.}{al niet afgewezen en}{steeds kom je terug}\\

\haiku{Want het geloof in;}{de leer is het begin van}{allen vooruitgang}\\

\haiku{Dat gaf u het recht.}{om in den strijd hoogere}{hulp aan te roepen}\\

\haiku{dat is ook zoo, doch.}{slechts voor geestelijk  zeer}{ver gevorderden}\\

\haiku{De Stins van Sweder '.}{heette het onverwinbaarst}{Slot int Oversticht}\\

\haiku{En later, na zijn,.}{dood is hij aan menigen}{Van Voorst verschenen}\\

\haiku{{\textquoteleft}Het moet wel heerlijk!}{zijn met zulke mannen in}{het gevaar te gaan}\\

\haiku{{\textquoteleft}En gij, een zoo sterk,?}{man verlangt niet uw kracht in}{den strijd te toonen}\\

\haiku{{\textquoteleft}Ge behoort dus tot,?}{de nieuwe secte die voor}{de weerloosheid is}\\

\haiku{{\textquoteright} hield Ravenhorst vol. {\textquoteleft},.}{Wie wapens draagt maakt er al}{spoedig gebruik van}\\

\haiku{Plechtrude stond voor '.}{t raam der keminade}{en staarde hem na}\\

\haiku{Hij had een vijand, -.}{die hem bitter had gegriefd}{Heer Jan van Hoeckelom}\\

\haiku{Toen vloeiden tranen.}{van de rozewangen van}{de schoone Kunegond}\\

\haiku{Hij reed vergramd naar,.}{Dorenweerd vroeg een gehoor}{en werd ontvangen}\\

\haiku{{\textquoteleft}Wilt ge mij zweren,?}{dat ge voor immer met dien}{Guido breken zult}\\

\haiku{Toen ik wegreed, vroeg -!}{ik terloops naar de rest n\`og}{ruim een twintigtal}\\

\haiku{De oude Otto.}{onderwees Walravia en}{Ravenhorst keek toe}\\

\haiku{{\textquoteright} vroeg Walravia toen.}{zij teruggingen naar de}{huisgenooten}\\

\haiku{Wel-is-waar ken,.}{ik den weg niet maar gij kunt}{mij dien beschrijven}\\

\haiku{De Heilige Maagd,,.}{en de Heiligen zullen}{hem hoop ik bij staan}\\

\haiku{Ten onrechte weet.}{zij Jaspers weifelende}{houding aan lafheid}\\

\haiku{De grootmoeder liet;}{haar rozenkrans langzaam door}{de vingers glijden}\\

\haiku{De groote oogen van de;}{kleine burchtvrouw werden vol}{angst op hem gericht}\\

\haiku{Met goud wil hij zich.}{na de beleediging niet meer}{tevreden stellen}\\

\haiku{Dit was het portret,.}{van Walravia's vader}{eens Ravenhorsts vriend}\\

\haiku{Het is u, of ge.}{onzen Heer in het uur der}{beproeving verlaat}\\

\haiku{maar wie zich geeft en,,.}{offert gaat niet onder doch}{komt tot hooger rang}\\

\haiku{slechts Walravia's.}{snikken verbraken de rust}{in de ridderzaal}\\

\haiku{Hij zal den uitslag,.}{z\'o\'o maken dat deze uw}{geweten niet drukt}\\

\haiku{wat een breede borst -.}{dat is een gemakkelijk}{te treffen doelwit}\\

\haiku{ook ditmaal wilde!}{hij weer een proeve van zijn}{bekwaamheid geven}\\

\haiku{{\textquoteleft}Ge hebt mijn leven,,!}{gespaard maar denk niet dat ik}{u daar dank voor weet}\\

\haiku{hij opende de deur,;}{daarvan en trad in een klein}{vierkant kamertje}\\

\haiku{{\textquoteright} {\textquoteleft}Dat is mijn vaste,:}{overtuiging want ik zal je}{eens  wat zeggen}\\

\haiku{- die zou als Edelman.}{zeker nooit tot zulk geweld}{zijn toevlucht nemen}\\

\haiku{Toen ik wakker werd,.}{scheen er niets bijzonders te}{zijn voorgevallen}\\

\haiku{{\textquoteleft}In uw kleine kast?}{ligt dus het geld en links staat}{de doos met poeder}\\

\haiku{{\textquoteright} Van Ravenhorst sprak.}{snel en zijn gelaat werd door}{smart verwrongen}\\

\haiku{Achter hem klonk de,.}{schaterende helsche lach}{van de heidin}\\

\haiku{{\textquoteright} zei hij met zijn  ,.}{zachte gevoelvolle en}{welluidende stem}\\

\haiku{Ik toefde toen op.}{een Burcht in het Kleefsche en}{ontmoette er haar}\\

\haiku{Een tweeden keer kon,.}{het toeval eens niet te uwen}{gunste zijn zoo vreest ge}\\

\haiku{hoe valsch bleek nu het,.}{beeld dat hij zich van eigen}{wezen had gevormd}\\

\haiku{Johanna wierp zich.}{hiervoor op de knie\"en en}{bad lang en innig}\\

\haiku{Het volle maanlicht.}{bescheen haar schoon gelaat en}{edele gestalte}\\

\haiku{De roodharige.}{brulde als een wild dier toen}{hij op den grond lag}\\

\haiku{Misschien wordt u dit,,{\textquoteright}.}{later wel duidelijk mijn}{kind vervolgde hij}\\

\haiku{Gij meent dat zonder.}{tusschen-middelaar te}{kunnen bereiken}\\

\haiku{Toen hij haar naar haar,:}{voertuig teruggeleidde}{zei zij in de laan}\\

\haiku{hij is ook niet zoo.}{handig op de jacht en bij}{het visschen als Daem}\\

\haiku{Zij naderde een,.}{boomgroep waarachter zich twee}{mannen bevonden}\\

\haiku{In den avond, toen Daem,.}{vertrok bracht Walravia hem}{door de oprijlaan}\\

\haiku{In het poortgebouw,.}{wachtte haar Louise die haar}{onder den arm nam}\\

\haiku{Voor ditmaal waren.}{de Mennonieten aan de}{vervolging ontsnapt}\\

\haiku{{\textquoteright} {\textquoteleft}Vergeef mij, Heer Van,?}{Hamelen dus zie ik in}{u een Humanist}\\

\haiku{Kent ge de schoone?}{geestelijke liederen}{uit die  dagen}\\

\haiku{Maar de smart verwrong.}{zijn gelaat toen zijn oog het}{hare ontmoette}\\

\haiku{Hij vertelde haar.}{in het kort de voorvallen}{van den laatsten tijd}\\

\haiku{{\textquoteright} Zoo eindigde hun.}{gesprek en teleurgesteld}{trok Ravenhorst weg}\\

\haiku{Maar sterker zou ik,!}{zijn als ik mijn stem verhief}{als Heer Van Doorwerth}\\

\haiku{Vol vertrouwen reed.}{hij daarom nu de slotpoort}{van de Doorwerth door}\\

\haiku{Het is u bekend,?}{dat uw eigen dienaar met}{moordplannen rondloopt}\\

\haiku{Hoeveel schade men,.}{op die wijze doet aan zijn}{ziel wordt niet bedacht}\\

\haiku{dat mijn houding niet{\textquoteright},.}{edel en ridderlijk is zei}{hij na een stilte}\\

\haiku{Zoo wordt Johanna....}{u tot last en ge moet dan}{wel van haar scheiden}\\

\haiku{men moet Haar boven,.}{\`alles gehoorzamen wijl}{men niet anders m\`ag}\\

\haiku{Brieven uit Arnhem,....}{een brief van den kasteleyn}{van zijn goederen}\\

\haiku{Maar in dit verband.}{heb ik nog eens nagedacht}{over Jasper en jou}\\

\haiku{En ik geloof niet.}{dat je oom v\'o\'or dien tijd reeds}{kan worden beleend}\\

\haiku{Niemand mocht ook haar,....}{ware gevoelens kennen}{vooral Jasper niet}\\

\haiku{Gisteren vroeg ik.}{nogmaals hier of men onzen}{man niet had gezien}\\

\haiku{Terwijl ik met den,.}{waard praatte lette ik niet}{op dezen krijger}\\

\haiku{Bij mijn baard, hij is,.}{de eerste die zich op zoo}{iets kan beroemen}\\

\haiku{Hier vond men dus in;}{de verdrukking bij elkaar}{een kostbaren steun}\\

\haiku{En nu naderde.}{het vreeselijke gevecht}{zijn hoogtepunt}\\

\haiku{Pirot greep de vrouw.}{bij de haren en rukte}{heur hoofd achterover}\\

\haiku{De woorden die hij,.}{daarop liet volgen waren}{niet meer te verstaan}\\

\haiku{Zij ademde nog, maar.}{haar wonden gaapten diep en}{breed in de hartstreek}\\

\haiku{{\textquoteleft}Zeg mij, vrouw,{\textquoteright} zoo drong, {\textquoteleft},....}{hij aanspreek snel de Dood heeft}{u reeds in zijn macht}\\

\haiku{Kunnen wij niet het}{hoogste bereiken als wij}{werkelijk met ernst}\\

\haiku{Waarschuw Johanna,.}{dat zij vooral niet alleen}{moet gaan wandelen}\\

\haiku{Wat had zij dit jaar?}{met al die vreugde in de}{natuur te maken}\\

\haiku{Jij, roodharige,,?}{leelijke duivel wat denk}{je wel van je zelf}\\

\haiku{Johanna sloot de,.}{oogen om zijn rood blauw verhit}{gelaat niet te zien}\\

\haiku{De levensdrang kreeg.}{echter nog een oogenblik}{de overhand in haar}\\

\haiku{{\textquoteright} Voorzichtig opende.}{zij de deur en gluurde door}{een reet naar binnen}\\

\haiku{{\textquoteleft}Hier, bij haar doode,,.}{lijf past alleen de waarheid}{de volle waarheid}\\

\haiku{Enfin - daarover valt,.}{niet te twisten je bent jong}{en onbedachtzaam}\\

\haiku{Den volgenden dag.}{moest de Jonker voor goed het}{Kasteel verlaten}\\

\haiku{Nog ademde hij, maar.}{hij had hoogstens nog een klein}{half uur te leven}\\

\haiku{{\textquoteright} Een kwartier later.}{trad de priester met ernstig}{gelaat uit de zaal}\\

\haiku{Zij drukten elkaar.}{de hand en de Jezu{\"\i}et}{ging alleen verder}\\

\haiku{Zijn zonde is groot,,.}{want wie niet v\'o\'or de Kerk is}{is tegen Christus}\\

\haiku{Mijn eer verbood het -!}{mij een ander stond immers}{tusschen jou en mij}\\

\haiku{Wij hebben samen,}{gespeeld gejaagd en gevischt}{en nooit voelde ik}\\

\haiku{Zij greep den oude.}{bij het middel en danste}{met hem in het rond}\\

\haiku{dat is van God, de.}{zuivere Liefde tot al}{wat leeft medebrengt}\\

\haiku{Men heeft, ook nog in,.}{de laatste jaren veel naar}{deze gang gezocht}\\

\haiku{Jan de Bakker of,.}{Johannes Pistorius}{Pastoor van Woerden}\\

\haiku{M.C. Nieuwbarn O.P., {\textquoteleft}Het{\textquoteright};}{Heilig Misoffer en zijn}{ceremoni\"en}\\

\subsection{Uit: Het mysterie van het Veluwehuis (onder ps. J. van Callant)}

\haiku{Bij dien arbeid had,:}{hij ook het geduld geleerd}{dat hij nu toonde}\\

\haiku{Een oogenblik nog.}{en de trein stoof snuivend en}{rammelend binnen}\\

\haiku{{\textquoteleft}Ja, ik houd van de,{\textquoteright}.}{natuur zei Winkelman na}{een korte stilte}\\

\haiku{Hij zag haar eerst niet,.}{maar weldra vond hij haar toch}{in het kreupelhout}\\

\haiku{{\textquoteright} {\textquoteleft}Dus jij bent onze,?}{naaste buurman behalve}{de heer Verhoeven}\\

\haiku{De huisknecht maakte.}{een buiging en uitte een}{soortgelijken groet}\\

\haiku{{\textquoteright} vroeg hij, zich tot de.}{gastvrouw wendende met een}{lachje van twijfel}\\

\haiku{Maar dien neef van hen,!}{met dien moet ik eens kennis}{maken als hij komt}\\

\haiku{Toen herkende hij,:}{zijn vrouw streek met de hand over}{het voorhoofd en vroeg}\\

\haiku{Trap op, trap af scheen,.}{het te schuifelen maar geen}{der treden kraakte}\\

\haiku{Er verliepen drie,.}{dagen zonder dat er iets}{bijzonders voorviel}\\

\haiku{Op den middag na.}{zijn komst ging de dokter hem}{een bezoek brengen}\\

\haiku{De wetenschap is.}{op zielkundig gebied nog}{weinig gevorderd}\\

\haiku{{\textquoteright} Winkelman uitte.}{een gesmoorden kreet en trok}{zijn handen terug}\\

\haiku{Plotseling bleef zij -!}{staan en luisterde het kon}{geen vergissing zijn}\\

\haiku{Maar zijn overspannen,.}{toestand eischte dat er}{verandering kwam}\\

\haiku{{\textquoteleft}Dat hij zijn vuist naar,!}{de beide huizen ophief}{geeft wel te denken}\\

\haiku{Uit iemands boeken,}{kan men soms belangrijke}{conclusies trekken}\\

\haiku{{\textquoteright} Hij zweeg even en keek,:}{daarop den dokter weer scherp}{aan terwijl hij vroeg}\\

\haiku{Ik zou die dan graag, ',.}{alst niet onbescheiden}{is eens willen zien}\\

\haiku{{\textquoteleft}Het is gek, dat men!}{aan zulke kleine dingen}{zooveel waarde hecht}\\

\haiku{Herinner je, dat.}{de dokter geheimhouding}{van den verkoop vroeg}\\

\haiku{{\textquoteright} {\textquoteleft}Maar de naam dan in,;}{de passagierslijst in de}{hotelboeken enz.}\\

\haiku{{\textquoteright} vroeg Robbers toen ik, {\textquoteleft}?}{gereed was met de lectuur}{wat denk je daarvan}\\

\haiku{Ik wil straks alles -.}{nog eens beter opnemen}{ook dat geraamte}\\

\haiku{In de andere:}{kast naast het raam vonden we}{evenmin iets verdachts}\\

\haiku{{\textquoteright} Robbers wilde nog,}{een vraag stellen maar op dat}{oogenblik hoorden}\\

\haiku{het gebeurde scheen.}{een diepen indruk op hem}{te hebben gemaakt}\\

\haiku{Misschien zouden wij{\textquoteright},.}{dan samen kunnen gaan zei}{de heer Tellegen}\\

\haiku{Je zult je vriend Thom?}{toch zeker nog wel een paar}{jaar levens gunnen}\\

\haiku{Ik had mij in het.}{zand uitgestrekt en Robbers}{ging naast mij zitten}\\

\haiku{op een avond ziet hij,.}{iemand en ontstelt z\'o\'o dat}{hij een flauwte krijgt}\\

\haiku{wat men sterk wenscht,.}{en verwacht gaat op den duur}{vaak in vervulling}\\

\haiku{Ik vroeg hem of hij.}{de gillende geluiden}{wel eens had gehoord}\\

\haiku{Zal jongen sturen.}{om je den weg te wijzen}{naar mijn verblijfplaats}\\

\haiku{{\textquoteright} {\textquoteleft}Nu zullen we eens.}{zien wat de heer Tellegen}{ons kan vertellen}\\

\haiku{Hij heeft zich al een.}{paar maal bewogen en schijnt}{weer bij te komen}\\

\haiku{In de Indische.}{wouden liet hij den dokter}{de diamanten zien}\\

\haiku{De dokter stond met.}{een bleek en verschrikt gelaat}{bij zijn schrijfbureau}\\

\haiku{De dokter en zijn.}{vrouw hadden krankzinnig van}{angst kunnen worden}\\

\section{Emmanuel de Bom}

\subsection{Uit: Heldere gezichten}

\haiku{Hij kwam uit een storm.}{die zijn jong gemoedsleven}{hevig had doorschokt}\\

\haiku{Hij had zijn viool.}{weer uit de kast gehaald en}{had ze weer gestemd}\\

\haiku{hij voelde geen drang,...}{naar haar broze wezen die}{ziel zonder lichaam}\\

\haiku{het toekomende -;}{maand ging zijn maar het spookte}{tusschen de regels}\\

\haiku{onder 't bed school;}{een man met een blinkend mes}{tusschen zijn tanden}\\

\haiku{hij hoorde iemand,.}{de trap van het huis afgaan}{de vrouw die hoestte}\\

\haiku{hij onderbrak b.v....}{een heel ernstig gesprek met}{een opmerking over}\\

\haiku{Zij waren omtrent;}{hetzelfde tijdstip naar de}{haven gekomen}\\

\haiku{Bij hun verschijnen.}{op de Beurs werden zij met}{ontzag aangestaard}\\

\haiku{De oogen zijn grooter, '.}{geworden en schijnen scheef}{int hoofd te staan}\\

\haiku{En de meester schijnt...{\textquoteright} - {\textquoteleft},{\textquoteright},.}{er op te wakenJa vriend}{Lodewijk zei hij}\\

\haiku{De piano hing '...}{schrijlings doort raam van de}{eerste verdieping}\\

\haiku{Rechts flankeerde haar,;}{een slanke als een degen}{zoo fiksche page}\\

\haiku{En deze laatste {\textquoteleft} '{\textquoteright};}{man wasde zoekende naar}{t onbekende}\\

\haiku{En dan moet gij, in '}{gezelschap van Vlaanderens}{grootsten schrijver als}\\

\haiku{Zij kwamen, wijl de,.}{boot een half uur te laat}{kwam even lang te vroeg}\\

\haiku{In den aanvang der;}{15e eeuw begon de zee zich}{terug te trekken}\\

\haiku{gezamenlijk uit;}{te maken een fleurigen}{bond der jongeren}\\

\haiku{Maar ik wou, dat gij.}{dien dag Dr. Jef Spijkers aan}{de taak had gezien}\\

\haiku{En wij kijken toe,...}{en voelen ons tot zwijgen}{en peinzen genoopt}\\

\haiku{{\textquoteright} Maar... aan den hemel.}{was plotseling een vreemde}{klaarte gerezen}\\

\haiku{'t jaar traant weg... 't,..}{Is een week einde een naar}{en zielig einde}\\

\subsection{Uit: Wrakken}

\haiku{is al dadelijk.}{wilskrachtige gespitstheid}{van het intellect}\\

\haiku{Al ontginnen zij,.}{zich niet zij beseffen iets}{van het eigen zijn}\\

\haiku{Maar de mensch is nog{\textquoteright}.}{niet sterk genoeg om zonder}{ideaal te leven}\\

\haiku{onze geest moet  .}{de geest van onze tijd zijn}{en niet omgekeerd}\\

\haiku{Wij kunnen onze.}{handen niet adelen aan den}{zegenenden arbeid}\\

\haiku{Zijn gezicht was plat,;}{en schier loodkleurig het had}{iets van een Mongool}\\

\haiku{Hij wou iets anders,,...}{iets dat hij nog niet gekend}{had nergens ontmoet}\\

\haiku{Het verblijdde hem,.}{alsof zij nu dichter tot}{hem genaderd was}\\

\haiku{Het lag daar vermoeid,,.}{in zijn nachtjaponnetje}{als een mo\^e bloempje}\\

\haiku{- ik verleid u, ik,...}{slorp u op ben de schuld dat}{gij verloren gaat}\\

\haiku{buiten u is de -!...}{wereld mij le\^eg zonder u}{kan ik niet ademen}\\

\haiku{en altijd zoo blij, -!}{zoo goed gezind o ge zijt}{twee deugenietjes}\\

\haiku{Zij gingen door de.}{luidruchtige straat in het}{schoone avondweder}\\

\haiku{Richard dwaalde nog.}{eenigen tijd in mijmering}{langs de straten}\\

\haiku{Zij verbleekte en '.}{t was of haar harteklop}{plotseling stil hield}\\

\haiku{Zij herinnerde.}{zich hun afspraak en het was}{een nieuwe kwelling}\\

\haiku{Op dit oogenblik;}{voelde zij een oneindig}{misprijzen voor hem}\\

\haiku{Een weemoed zonk in.}{haar bij het herdenken van}{heel dat verleden}\\

\haiku{{\textquoteright} {\textquoteleft}Ja...{\textquoteright} en hij tastte.}{in zijn zak en haalde een}{doosje te voorschijn}\\

\haiku{Hij schrikte als hij.}{zijn doodsche waskleurige}{trekken gadesloeg}\\

\haiku{En opeens dacht hij,...}{dat nu een andere in}{zijn kooi zou liggen}\\

\haiku{En nu mocht alles - '...}{opnieuw beginnen zooalst}{altijd geweest was}\\

\haiku{maar met een zucht joeg.}{zij de vernevelende}{schim ver weg van haar}\\

\haiku{{\textquoteright} - Zij vroeg hem niet eens.}{waar die verschrikkelijke}{som voor dienen moest}\\

\haiku{Zij gaf voor dat zij,.}{naar huis moest maar hij wilde}{haar niet loslaten}\\

\haiku{Hier sta ik op het {\textquotedblleft}{\textquotedblright}{\textquoteright},.}{punt eengemengd bericht te}{plegen spotte hij}\\

\haiku{De schepen vaarden.}{in blij gewemel door het}{lavende briesje}\\

\haiku{Daar stond een man aan,:}{den achtersteven die met}{een sjerpje waaide}\\

\section{I.K. Bonset}

\subsection{Uit: Het andere gezicht van I.K. Bonset}

\haiku{De engelen - ja,.}{de engelen worden aan}{het spit gebraden}\\

\haiku{{\textquoteleft}H\'e, jelui daar, kun.}{je mij geen sigaret naar}{beneden gooien}\\

\haiku{Een belangwekkend.}{stilzwijgen nam al onze}{zinnen in beslag}\\

\haiku{Elke functie die,.}{op de fysica berust}{mechaniseren}\\

\haiku{Wij nieuwe mensen.}{willen met de kunstenaars}{een strijd beginnen}\\

\haiku{Massatoestand Het,.}{is niet prettig muis te zijn}{tussen twee katers}\\

\haiku{Het nieuwe inzicht ().}{het overzinnelijk gezicht}{heft dit contrast op}\\

\haiku{Der Dadaist{\textquoteright} - aldus - {\textquoteleft};}{Raoul Hausmannerleidet}{nicht die Welt kindlich}\\

\haiku{Elk eindje touw door.}{een knoop verbonden aan een}{ander eindje touw}\\

\haiku{23Dichterlijke.}{uitdrukking als resum\'e}{van Evola's werkje}\\

\haiku{27Bloeien uit de:}{tevredenheid met onze}{eigen gebreken}\\

\section{Joachim Bontius de Waal}

\subsection{Uit: Oorspronck en opkomst der stede Alckmaar, beginnende anno DL uyt een seer oud manuscript berustende ter Liberije deser Stadt gecopieert ende vervolgt tot MDCCLX}

\haiku{Alckmaer zonder}{die kercke altemael}{ende die lieten}\\

\haiku{Den 24 april is de.}{eersten steen geleydt aen de}{oude Vrieschepoort}\\

\haiku{men soude wat in,.}{de kerck doen hetwelck}{sij soo gedaen heeft}\\

\haiku{Is de Langenstraet}{deeser stadt Alckmaer uyt}{eenderhandt verhoogt.1587In}\\

\haiku{huys opgeregt,}{buyten de Kennemerpoort}{staet op die plaets daer}\\

\haiku{Dit jaer is de craen '.}{gesteldt op het endt vant}{ooster Fnidsen}\\

\haiku{10 augusti is.}{den eersten steen geleydt aan}{de Waagtooren}\\

\haiku{nae water, maer daer}{was niet.1651Is de heerlijckheydt}{van Schagen verkogt}\\

\haiku{Men is niet bewust}{of dat ongeluck door het}{swaeren donderweer}\\

\haiku{van hier trocken sij,.}{voorts door Noordt-Hollandt daer sij}{de beelden stormden}\\

\haiku{De schaeden was in,.}{Engenlandt Vranckrijck}{en Schotlandt seer groot}\\

\section{Henri van Booven}

\subsection{Uit: Kinderleven}

\haiku{Drie poppen, alleen.}{in de stille schaduw van}{de middagkamer}\\

\haiku{V\'o\'or alles moest zijn,.}{zwarte glanzende gaafheid}{ongerept blijven}\\

\haiku{Dat waren dan tw\'e\'e,,....:}{likken \'e\'en lange streek dus}{anapaesten}\\

\haiku{{\textquoteright} In de kasten, op.}{de bovenste planken rest}{nog een en ander}\\

\haiku{Zij heeft zeker wel}{eens in de kast gekeken}{en niet geweten}\\

\haiku{Wat heb ik dan toch,}{gedaan dat die wanden mij}{hier nog altijd zoo}\\

\haiku{{\textquoteright} Een ander maal, des,.}{morgens voor de lessen werd}{er heel lang getold}\\

\haiku{Want, zonder dat ik,,.}{het hoorde plotseling daar}{tripte zij binnen}\\

\haiku{Hier moet het verjaagd,;}{met ontstoken lampen met}{ons levend geluid}\\

\haiku{Er is niets dan dat....}{in de beslotenheid van}{deze duisternis}\\

\haiku{Het zal mij nu ook,....}{nimmer meer vreemd zijn noch kan}{het teloor gaan}\\

\subsection{Uit: De scheiding}

\haiku{{\textquoteright} En hij hamerde.}{met zijn bierglas op den rand}{van een stoel voor zich}\\

\haiku{Vooruit maar Rieke,,!}{m'n jonkske in het veen ziet}{men op geen kluitje}\\

\haiku{wij verschillen van.}{de Belgen voornamelijk}{in ras en godsdienst}\\

\haiku{Maar ik zeg je dan,,{\textquoteright}.}{mijn waarde dat wij d\'a\'ar nog}{lang niet aan toe zijn}\\

\haiku{Hij had afkeer van.}{den militair in Vincent}{op dat oogenblik}\\

\haiku{Weldra zou het spel,.}{in den tuin gedaan zijn en}{moesten zij naar binnen}\\

\haiku{Hij zou, naar hij schreef,.}{waarschijnlijk niet meer naar de}{tropen teruggaan}\\

\haiku{Miel was ziek, hij leed,.}{aan ingewandsziekte en}{had dikwijls koortsen}\\

\haiku{Maar een smidsjongen,:}{en een landman met een zeis}{drongen op zeggend}\\

\haiku{{\textquoteleft}Neen, dan zou ik u,{\textquoteright}.}{gewaarschuwd hebben ik heb}{geen Duitscher gezien}\\

\haiku{Maar Marius wist,.}{zich te beheerschen en hij}{hoorde nu zijn stem}\\

\haiku{De stationschef:}{meende dat het wel van een}{ontploffing kon zijn}\\

\haiku{Zij stonden voor een,}{spoorbrug en de weinige}{reizigers staken}\\

\haiku{En met gehaaste,.}{nerveuse stappen verliet}{hij de kamer}\\

\haiku{Was zij dan in die?}{enkele jaren z\'o\'oveel}{ouder geworden}\\

\haiku{{\textquoteleft}Neen, treurig genoeg,.}{want zij trekt zich dat alles}{verschrikkelijk aan}\\

\haiku{En hij stelde zich.}{voor wat hem te doen zou staan}{als dat gebeurde}\\

\haiku{Hij deed alles wat.}{hij kon om het gezin bij}{te staan in zijn nood}\\

\haiku{Het was hun alsof.}{elk woord dat zij spraken hun}{noodlottig kon zijn}\\

\haiku{Vanaf den zolder,.}{konden zij nu zien hoe het}{brandde in de stad}\\

\haiku{hij stond de eenige.}{overgebleven burgers der}{stad Leuven waren}\\

\haiku{Er was geen klacht uit.}{de zeventig mannen van}{Leuven opgegaan}\\

\haiku{Hij vermeed het ook,.}{hem nu dadelijk over zijn}{gezin te spreken}\\

\haiku{Zij hield van Vincent,?}{maar was Marius niet van}{heel anderen aard}\\

\haiku{Wat was zij dien avond.}{van de eene stemming in de}{andere gejaagd}\\

\haiku{Het dametje stond.}{nu op uit haar stoel en kwam}{nader bij Louise}\\

\haiku{zoudt ge er tegen,?}{op zien om op zulk een kar}{een poos te rijden}\\

\haiku{niet ver van de plek,.}{waar de brug vernield lag zou}{hij hen wachten}\\

\haiku{hij kon zonder door.}{te groote vaart in de steden}{argwaan te wekken}\\

\haiku{Hoe geheel anders.}{zagen zij er uit dan de}{infanteristen}\\

\haiku{Het was waar de rails.}{bij een splitsing van den weg}{naar rechts afbogen}\\

\haiku{{\textquoteright} {\textquoteleft}Dat hebt ge gezien,{\textquoteright}.}{hier staat het stempel van de}{Duitsche Legatie}\\

\haiku{Voelde hij nu meer?}{deernis met haar dan liefde}{en genegenheid}\\

\haiku{Maar opnieuw als een:}{plotselinge pijn sloop de}{twijfel in zijn hart}\\

\haiku{Zeker, zoo waren,.}{zij die Belgische vrouwen}{zij vergaten gauw}\\

\haiku{Dat was altijd de,.}{stille rustige trots van}{haar geslacht geweest}\\

\haiku{Dat was een heete,.}{dag een Zaterdag laat in}{Augustus geweest}\\

\haiku{maar ziende, dat zij,:}{zwaar ademhaalde kwam hij bij}{haar staan en zeide}\\

\haiku{Evenwel, ondanks dien;}{voorkeur had zij ontzag voor}{een helder verstand}\\

\haiku{Aanvankelijk kon;}{hij geen berusting vinden}{bij die overtuiging}\\

\haiku{Het gebeurde een.}{Zaterdagmiddag in het}{begin van Juli}\\

\subsection{Uit: Tropenwee}

\haiku{Een negermeisje,.}{stond daar een kind van een jaar}{of vijftien leek het}\\

\haiku{Hij nam zijn vouwstoel.}{uit den corridor en klom}{er mee naar boven}\\

\haiku{Haastig stappend en '.}{armen zwaaiend waadden die}{buitent bassin}\\

\haiku{De inspecteur zat.}{midden in deze kamer}{voor zijn schrijftafel}\\

\haiku{Met geduchte en:}{vijandige zekerheid}{voelde hij het toen}\\

\haiku{Vandaag was 't niet, '.}{zoo warmt zal binnen een}{uur wel ophouden}\\

\haiku{Het geluid van de.}{stoomfluit begon te grommen}{langaange-houden}\\

\haiku{bij de kazerne.}{stonden wat struiken en een}{enkele hooge boom}\\

\haiku{Later zette hij.}{zich bij den dooden boom op een}{uitstekenden steen}\\

\haiku{Een ijzeren pier.}{stak vlak bij die loodsen het}{gele water in}\\

\haiku{In de schemering.}{keerde de witte naar de}{factorij terug}\\

\haiku{De witte en zijn.}{reisgenoot bleven alleen}{op de verandah}\\

\haiku{Ronk had zijn witten.}{hoed reeds opgezet en scheen}{klaar om uit te gaan}\\

\haiku{{\textquoteleft}Dat zou ik bijna,?}{vergeten mag ik het er}{maar van afnemen}\\

\haiku{wat meer naar voren.}{lag het stationnetje}{met schaarsche lichten}\\

\haiku{De witte gaf geen,,:}{antwoord eerst zeide toen om}{toch wat te zeggen}\\

\haiku{Nu zijn we er gauw,{\textquoteright}.}{zeide een van de mannen}{die naast hem liepen}\\

\haiku{{\textquoteright} {\textquoteleft}Ja, daar lagen we,.}{als beesten op den planken}{vloer geen eens bedden}\\

\haiku{De man die vlak voor,;}{het licht zat had een roode}{vlek op het voorhoofd}\\

\haiku{Hij luisterde, nat.}{van zweet en trillend van de}{overgroote inspanning}\\

\haiku{Enkele uren voor.}{zonsondergang was alles}{voor de tocht gereed}\\

\haiku{Enkele uren, kort,.}{voor zonsondergang was hij}{van de jacht terug}\\

\haiku{Nous avons d\'ecouvert!}{encore trois bo{\^\i}tes de}{jambon conserv\'e}\\

\haiku{Toen dat gelukt was.}{zag hij de moustiquaire}{van binnen goed na}\\

\haiku{hij snakkend naar lucht......}{neer op de vuile lakens}{en wilde slapen}\\

\haiku{Hij vraagde een en,.}{ander van het land hoe ver}{hij wel gereisd had}\\

\haiku{Comme vous \^etes p\^ale,{\textquoteright}.}{zeide Fourneau toen hij den}{witte ontwaarde}\\

\haiku{Hij beproefde de.}{geluiden van die dieren}{te onderscheiden}\\

\haiku{En hij begon den,.}{dood die nu wel heel dicht bij}{moest zijn te schuwen}\\

\haiku{As je wat noodig heb,.}{seg-ut dan an die fent}{die we hier late}\\

\haiku{Hij at alles wat.}{de negerjongen hem bracht}{en vraagde om meer}\\

\haiku{Een schriklijke dorst,.}{folterde hem toch mocht hij}{maar weinig drinken}\\

\haiku{Een neger begon,,.}{stil in den gloed kijkend een}{donker  gezang}\\

\haiku{Zijne moeder, heel.}{zelden maar had zij zich hem}{welgezind getoond}\\

\haiku{Heel moeilijk liep de.}{witte naar het station}{en ging er binnen}\\

\haiku{De zwarte haalde.}{de koffers en bracht ze in}{het getimmerte}\\

\haiku{Een schandaal van je.}{huis om je hier als een beest}{te laten liggen}\\

\haiku{{\textquoteright} De witte begreep,.}{dat dit de nieuwe chef in}{Mataddi moest zijn}\\

\haiku{Als je mijn vraagt wat ', {\textquoteleft}.}{t beste is uitviere}{en zweete enEno's}\\

\haiku{De lange, zware,.}{droeg hem voort hoog boven de}{hoofden der menschen}\\

\haiku{Maar buiten de stad,.}{bij het bosch daar dansten de}{negers  Tam-Tam}\\

\haiku{Ik had neus en keel.}{vol geronnen bloed dat mij}{bijna stikken deed}\\

\haiku{Het stond op een klein,,....}{eiland aan de Vecht was het}{geweest meende hij}\\

\haiku{De heuvels duwden....}{het bange echo-gerucht}{aarzelend weerom}\\

\haiku{In den winkel vlak.}{bij het Hollandsche huis ging}{de witte binnen}\\

\haiku{{\textquoteleft}Weet u dan niet, dat?}{die al een paar weken dood}{en begraven is}\\

\haiku{Hij was gaan zitten,.}{en keek uit over de zee het}{boek in de handen}\\

\haiku{Een slag op de gong.}{zwaaide een luid gedreun sleepend}{heen over de dingen}\\

\haiku{Onder aan de trap.}{van de pier schommelde het}{stoom-barkasje}\\

\haiku{{\textquoteright} Zij gingen naar de,.}{trap toe links daarnaast was een}{andere doorgang}\\

\haiku{Ook over het dek van.}{de tweede klasse was een}{wit zeil gespannen}\\

\haiku{Druk pratend stapten.}{de passagiers in groepjes}{naar de eetsalon}\\

\haiku{Als het twaalf uur sloeg,,.}{dan wist iedereen zou dat}{mogen gebeuren}\\

\haiku{Wat was dit vreemd en,?}{ongewoon zou hij dit wel}{ooit meer beleven}\\

\haiku{Voor hij naar zijn hut.}{ging wandelde Jules nog}{eenmaal naar achter}\\

\haiku{Half wakker dacht hij.}{er aan dat het binnen een}{paar uren dag zou zijn}\\

\haiku{Het steeg, het geweld,.}{en builde uit en zwol en}{gierde als een storm}\\

\haiku{Druppels zweet voelde,.}{hij afglijden van zijn lijf}{zijn gezicht was nat}\\

\haiku{Hij had weer lang stil,,.}{gezeten wachtend beidend}{den langzamen tijd}\\

\section{F. Bordewijk}

\subsection{Uit: Bloesemtak}

\haiku{Maar wel heb ik iets.}{dergelijks ontdekt in de}{Rotterdammersloot}\\

\haiku{Hij vertoonde het,.}{nooit lang maar toch steeds weer als}{het zijn vak betrof}\\

\haiku{De kleuren heb ik,.}{al in mijn hoofd eindigde}{hij met een glimlach}\\

\haiku{Het was half zes, en.}{het verkeer over de straatweg}{reeds iets verminderd}\\

\haiku{- Zei Termunten daar? -.}{nog wat van Hij neemt dat ook}{voor zijn rekening}\\

\haiku{Op die manier kwam.}{haar vriendschap met Leo neer op}{het hoofd van Anton}\\

\haiku{Leo had bijna een,.}{verering voor de vriendin}{al verborg ze dit}\\

\haiku{dacht hij, zittend naast,.}{zijn chauffeur nog even terug}{aan de gevelsteen}\\

\haiku{Hij herinnerde,:}{haar daaraan en vervolgde}{tot haar speciaal}\\

\haiku{ik bedoel, u zult.}{de toestand niet helemaal}{in uw macht hebben}\\

\haiku{Het is blijkbaar al,.}{weer zover en dat begint}{me te vervelen}\\

\haiku{Ik heb het tot nog,.}{toe geslikt maar ik wil nu}{weten wat er is}\\

\haiku{Je zegt maar dat het,.}{een vergissing was dat je}{vader het niet wil}\\

\haiku{Ik moet het bij haar.}{weghalen om het aan jou}{te kunnen geven}\\

\haiku{Jaja, we lossen,.}{het nu wel heel aardig op}{hier onder elkaar}\\

\haiku{Dan zijn die nauwe.}{zijstraten openingen in}{de rijen boeken}\\

\haiku{Evenwel sprak ze over.}{zijn werk nooit anders dan in}{enkele woorden}\\

\haiku{Ze begreep te zijn.}{binnengedragen in een}{tapijtenwinkel}\\

\haiku{Even later hoorde.}{ze de onmiskenbare}{klik van haar deurslot}\\

\haiku{Hij verscheen nog op;}{de dag v\'o\'or zijn dood in het}{huis op de Hooigracht}\\

\haiku{Volgens de dokter.}{zou dit verschijnsel nog wel}{een poos aanhouden}\\

\haiku{Van der Gronden reed,,;}{voorzichtig niet alleen in}{de stad ook buiten}\\

\haiku{Het luik in de kap.}{bleef wegens het branden van}{de zon gesloten}\\

\haiku{Het was buiten kil.}{en allen zaten in de}{conversatiezaal}\\

\haiku{Verder ging hij niet,.}{maar er opende zich voor haar}{een nieuw gezichtspunt}\\

\haiku{Ditmaal werden er.}{in het vestibuletje}{kussen gewisseld}\\

\haiku{- Weet je wel dat ze?}{de laatste tijd in Den Haag}{erg over je kletsen}\\

\haiku{Hij rookte niet meer,;}{maar hield het pijpje in zijn}{tandeloze mond}\\

\haiku{Toch jammer dat je.}{die beoordelingen niet}{kunt laten drukken}\\

\haiku{Dat doe ik ook niet,.}{maar ik kan toch niet alles}{laten passeren}\\

\haiku{En nog stom op de,.}{koop toe want zo propageer}{je de middelmaat}\\

\haiku{E\'en visite aan.}{dit verwarde brein was hun}{meer dan voldoende}\\

\haiku{Zeg, even groeten, daar,.}{komt meneer Nathans onderbrak}{Aurora zichzelf}\\

\haiku{- Ik mag die meneer,,.}{Nathans niet erg zei ze gedempt}{terwijl Max verdween}\\

\haiku{Aurora kwam in,,.}{de ochtend niet ver van haar}{woning Max tegen}\\

\haiku{Hij viel haar door een.}{ongewoon gelige tint}{onmiddellijk op}\\

\haiku{Deze nietige.}{vreugde ging teloor onder}{het verder klimmen}\\

\haiku{Ze kwam tot zichzelf.}{doordat ze tranen over haar}{wangen voelde}\\

\haiku{Zelfs bracht de laatste.}{conclusie een zweem van een}{glimlach om haar mond}\\

\haiku{Hij was ook altijd.}{zo zichtbaar het hoofd van zijn}{kinderloos gezin}\\

\haiku{Aurora vond het.}{beter hierover niet met Van}{Marle te spreken}\\

\haiku{Bedekt met leien.}{ging in het dak voor hem het}{symbool verloren}\\

\haiku{Het was een ochtend.}{van beginnende winter}{in allerfijnst grijs}\\

\haiku{Hij wist dat zulke.}{abnormaliteiten niet}{in haar smaak vielen}\\

\haiku{vroeg de vriendin na.}{een ogenblik en steeds in de}{grootste verwarring}\\

\haiku{Blijkbaar toen het ene.}{niet opging omdat het al}{te doorzichtig was}\\

\haiku{Hij durfde haar niet,.}{meer in de ogen te zien hij}{zou zich doodschamen}\\

\haiku{De groene tafel}{Nu Aurora meende door}{de volkomen breuk}\\

\haiku{En dat ik haar die,?}{brief teruggaf vond je toch}{ook wel goed nietwaar}\\

\haiku{Het stond in haar brief,;}{en het vereenvoudigde}{de toenadering}\\

\haiku{Een antisemiet?}{zal toch niet bij voorkeur bij}{een Jood intrekken}\\

\haiku{En ik zal je nu.}{ook precies zeggen hoe de}{vork in de steel zit}\\

\haiku{- Jacob Effra{\"\i}m,.}{vanaf dit ogenblik kennen}{wij elkaar niet meer}\\

\haiku{Waarom toch kon de?}{mens op dit punt zijn leven}{niet terugdraaien}\\

\haiku{- O neen, antwoordde,.}{Aurora haastig die mag}{u niet gebruiken}\\

\haiku{Hoop jij maar dat ze,,.}{het winnen allebei en}{dat ik het verlies}\\

\haiku{Welnu, dat deed hij,,.}{op dit moment het kon nog}{niet te laat wezen}\\

\haiku{Aurora zat met.}{de rug naar de deur aan een}{damesbureautje}\\

\haiku{Er hing een sfeer van.}{haat in het huis en deze}{sfeer werd steeds dichter}\\

\haiku{Hij keek haar aldoor,.}{strak aan met kleine enigszins}{roodbelopen ogen}\\

\haiku{- Ik heb iemand een,,.}{slag gegeven gisteren}{met een bureaulamp}\\

\haiku{Ik weet het van een.}{vriendin die eens met hem te}{maken heeft gehad}\\

\haiku{Het was gegaan bij,.}{vlagen overeenkomstig haar}{grillig karakter}\\

\haiku{Zeg Nathans, voordat je,?}{begint mag ik even juffrouw}{Monterey spreken}\\

\haiku{En u betaalt  ,.}{me vijf procent rente over}{het jaar gerekend}\\

\haiku{hij leek haar zo echt.}{een stijgende geul tussen}{muren ingeklemd}\\

\haiku{- Mijn zoon heeft met die,.}{vrouw volkomen gebroken}{zei Termunten koel}\\

\haiku{Toen kwam Fronto op:}{zijn beurt naar voren en zei}{met de grootste klem}\\

\haiku{U vertelde me.}{dat de jonge Termunten}{zo is gepousseerd}\\

\haiku{Aurora wist er;}{niets van dat haar geval toch}{nog een nasleep had}\\

\haiku{En ondanks deze.}{redeneringen bleef de}{beklemming hem bij}\\

\haiku{Hij had gevreesd voor,,.}{een onbeslapen bed zij}{elders in huis ziek}\\

\haiku{Dit verontrustte.}{Iris veel meer dan de toestand}{van de kinderen}\\

\haiku{Met hem sprak ze wel,.}{eens over zijn moeder maar hij}{gaf weinig antwoord}\\

\haiku{Deze vrouw was een.}{toonbeeld van plichtsbesef in}{zakelijke vorm}\\

\haiku{Het kon hem tot zulk.}{een foltering worden dat}{hij de straat opliep}\\

\haiku{Hij besefte zeer.}{goed wat de kinderen aan}{hem te kort kwamen}\\

\haiku{Van Marle wilde,.}{weerstreven maar de vriend trok}{hem omhoog en mee}\\

\haiku{Wie zich verplaatst naar.}{de vierde dimensie is}{ineens onzichtbaar}\\

\haiku{Maar er volgt n\'og iets,.}{en daarmee kom ik tot het}{bekende grapje}\\

\haiku{Maar daarboven, in,...}{haar bureautje lagen toch}{de drie portretten}\\

\subsection{Uit: De doopvont}

\haiku{hij hoefde haar zelfs,.}{niet te bedwingen en ook}{dat was hem bekend}\\

\haiku{Eens had hij tegen,,:}{zijn oudste halfzuster Lea}{Bearda gezegd}\\

\haiku{het was Bearda,.}{en De Bleeck en het bleef dit}{door alle jaren}\\

\haiku{De Bleeck liep langs het.}{water met zijn normale}{snelle wandelpas}\\

\haiku{De Bleeck ging uit in,.}{zijn kleine bruine wagen}{men wist niet waarheen}\\

\haiku{- De wet hanteren.}{is het tegendeel van de}{wet ondervinden}\\

\haiku{Maar tegenwoordig, -:}{het Parlement bezit geen}{mensenkenners meer}\\

\haiku{Bij een vrouw zijn de;}{zenuwtoppen doorgegroeid}{tot in haar kleren}\\

\haiku{Dat moet u hebben,.}{gezien want vrouwen kijken}{alleen naar elkaar}\\

\haiku{Het is oud nieuws, maar.}{het schijnt altijd weer nuttig}{het te herhalen}\\

\haiku{hij kon er evenwel,.}{niets aan veranderen en}{het ging hem niet aan}\\

\haiku{Gebruik toch liever,.}{die hersenkwabben voor wat}{anders Brandenburg}\\

\haiku{De organen van.}{ons perceptievermogen}{zijn altijd te laat}\\

\haiku{wat heeft de wereld?}{er aan de Duitsers nog eens}{op stang te jagen}\\

\haiku{C'est ici que tombent -.}{en ruine les merveilles}{de la cuisine}\\

\haiku{En overigens was.}{hij ook liefst afgezonderd}{met zijn gedachten}\\

\haiku{Bij weten van De.}{Bleeck was Bearda nog nooit}{op de club geweest}\\

\haiku{En trouwens, was het,,?}{een wetenschap en zo ja}{een volledige}\\

\haiku{Je ziet het, ik heb.}{me daarginds ook al voor je}{ge{\"\i}nteresseerd}\\

\haiku{- Je weet het altijd.}{zo voor te stellen dat het}{een compliment lijkt}\\

\haiku{hij kon niet enkel;}{in frak met de winter naar}{huis zijn gereden}\\

\haiku{zeker niet zichtbaar,.}{en zijn hart bleef rustig en}{evenwichtig kloppen}\\

\haiku{En de enkele.}{inwoning was mogelijk}{slechts een eerste stap}\\

\haiku{Zijn op dit punt slecht.}{geweten deed hem zich van}{alles inbeelden}\\

\haiku{ze voelde zich ook,.}{vermoeid want ze had te snel}{teruggelopen}\\

\haiku{Ze bezat, al zei,.}{ze het ronduit van zichzelf}{een goede speurneus}\\

\haiku{Ze had hem verleid,.}{als vrouw naar de zaken en}{als vrouw naar de sexe}\\

\haiku{Hij werd gewapend,.}{met een revolver en vond}{het een hele eer}\\

\haiku{De sfeer, de stijl van.}{haar ouderlijk huis had haar}{te zeer doortrokken}\\

\haiku{De middag druilde.}{in dit vertrek voort met een}{saaiheid zonder eind}\\

\haiku{Hij had dat altijd,.}{gedaan en Sara had nooit}{aanmerking gemaakt}\\

\haiku{Lea was een goede.}{huisvrouw in die zin dat ze}{perfect leiding gaf}\\

\haiku{- De laatste jaren,.}{speel ik alleen nog maar C\'esar}{Franck zei Aleida}\\

\haiku{Dat is op zichzelf,.}{ook al weer een meesterwerk}{die transpositie}\\

\haiku{Sara kwam door haar,,.}{spel tot de componist voor}{hoe kort ook maar toch}\\

\haiku{- Ja, maar ik kan er,.}{toch niet van loskomen van}{zoiets onooglijks}\\

\haiku{daardoor toch verloor.}{haar tweede leven iets van}{zijn verschrikking}\\

\haiku{- Dat is dan een streek.}{zoals gepast is onder}{boezemvriendinnen}\\

\haiku{Eigenlijk had hij.}{respect voor het geheel van}{diens directoraat}\\

\haiku{dat hij zijn fabriek,,.}{hem dierbaarder nog dan zijn}{huis niet vinden kon}\\

\haiku{Het was een stil en,.}{luchtig grapje geheel te}{eigen genoegen}\\

\haiku{Maar dit hier is voor.}{mij typisch de fin  de}{si\`ecle-sfeer}\\

\haiku{Adeldom geeft ook nu,.}{nog iets al kan je het niet}{defini\"eren}\\

\haiku{Daar ze er niets aan,.}{toevoegde moest het hier dus}{haar slaapkamer zijn}\\

\haiku{De zware rode.}{gordijnen naar de serre}{waren gesloten}\\

\haiku{kon hij daarvoor een?}{beter bewijs verlangen}{dan wat zij nu gaf}\\

\haiku{Toen zag hij zich hier,,.}{in zijn stoel zitten achter}{een barri\`ere}\\

\haiku{Dit veranderde:}{echter toen hij zijn eerste}{grote werk volbracht}\\

\haiku{Met dat al was X.}{bij alle geslotenheid}{geen zwijgzaam persoon}\\

\haiku{De ochtend van die.}{dag vielen er nog verspreid}{enkele buien}\\

\haiku{het bleef droog en was,.}{opmerkelijk zoel des avonds}{windstil bovendien}\\

\haiku{Van het gezicht wist.}{hij niets dan dat het een grauw}{vrouwengezicht was}\\

\haiku{Tot nog toe had ze,.}{hem handig ontweken thans}{zat ze in de val}\\

\haiku{- Ja, vervolgde hij,,.}{wijzend op haar omgeving}{u bent als Algol}\\

\haiku{Maar wilt u op een?}{andere manier onze}{historie kennen}\\

\haiku{Mevrouw Van Fransen,.}{kwam voorbij aan de arm van}{de legatieraad}\\

\haiku{Hoe kleiner, des te.}{eer de gevolgen van de}{oorlog te boven}\\

\haiku{Foei, foei, meneer Ake,.}{dat u zoiets tegen een}{buitenlander zegt}\\

\haiku{Nu we toch over Poe, -.}{praten als jong meisje heb}{ik hem verslonden}\\

\haiku{Een volgende keer.}{moeten wij eens zorgen voor}{een vrolijker slot}\\

\haiku{De belastingen;}{waren op zichzelf zeker}{niet onredelijk}\\

\haiku{Als er volksvrouwen.}{getroffen waren zou het}{even erg zijn geweest}\\

\haiku{- Ja, dat verklaart me,.}{wel iets al blijft er nog veel}{onbegrijpelijks}\\

\haiku{Vroonhoven reed naast - -.}{zijn kleindochter in haar of}{zijn wagen naar huis}\\

\haiku{lichamelijke.}{aftakeling hadden een}{andere oorsprong}\\

\haiku{Maar ook wilde ze;}{onder geen voorwaarde haar}{plezier bederven}\\

\haiku{Een paar jaar later.}{zag ze die weg plotseling}{in een helder licht}\\

\haiku{- 't Is een goed kreng,.}{zei mevrouw Ulius wel van}{Sara Brandenburg}\\

\haiku{Op haar naam stond voorts,.}{een stukje dat misschien niet}{enig toch zeldzaam was}\\

\haiku{Ik wacht over een week,.}{een boot en dan vallen er}{weer zaken te doen}\\

\haiku{Dat alles was nu,.}{verdwenen en Katendrecht}{werd plaatsvervanger}\\

\haiku{Met de Kersttijd zou,.}{het zoontje van daarginds naar}{hier overkomen Jack}\\

\haiku{Haar aangeboren.}{trots hield haar lichaam overeind}{en onbewogen}\\

\haiku{Toen bood de kennis,.}{Colijn een sigaar aan maar}{Colijn weigerde}\\

\haiku{- En ikzelf ook, zei,.}{mevrouw Ulius weer naar de}{theetafel gaande}\\

\haiku{Dacht je dat ik je?}{d\'a\'arvoor helemaal naar hier}{zou laten komen}\\

\haiku{Kon die Brandenburg?}{het op een procedure}{laten aankomen}\\

\haiku{Anarchisten van.}{de oude tijd maakten wel}{meer zulke bommen}\\

\haiku{moederliefde was.}{nu eenmaal iets heel anders}{dan vaderliefde}\\

\haiku{Maar hoe ze heten,.}{dat zal uw zwager u wel}{kunnen vertellen}\\

\haiku{De reserve van.}{Lea en zijn zwager op dit}{punt stelde hij hoog}\\

\haiku{hij wist alleen dat,.}{daar een vrouw woonde en had}{haar naam vergeten}\\

\haiku{Daarop besloot hij.}{aan Van der Mark het weekblad}{terug te geven}\\

\haiku{Hij had behoefte.}{aan een degelijk gesprek}{en ging naar zijn club}\\

\haiku{Maar hij antwoordde,:}{en sprak daarbij uit wat hij}{vroeger had gedacht}\\

\haiku{- Knap gezegd, Til, en,.}{voor honderd procent juist zei}{de secretaris}\\

\haiku{Die onbescheiden.}{vraag heb ik zelfs niet aan mijn}{kleindochter gesteld}\\

\haiku{zijn huwelijk kon,.}{niet mislukt heten want het}{telde voor hem niet}\\

\haiku{Dan lijkt de tijd niet.}{ver meer dat de man ophoudt}{de vrouw te boeien}\\

\haiku{Als ik spreek van de,.}{mens als jabroer bedoel ik}{de grote massa}\\

\haiku{Ik vind dat je over.}{de grote massa toch wat}{onbillijk oordeelt}\\

\haiku{En wat moest er dan,?}{in haar kind omgaan als ze}{de waarheid hoorde}\\

\haiku{een halve meter,}{lager dan het eerste en}{wie zich daar ophield}\\

\haiku{Hij bracht haar de kop,.}{koffie en ze keek op en}{dankte vriendelijk}\\

\haiku{Ik bedoel niet hier,,.}{want hier is het o-kay}{maar ik meen de buurt}\\

\haiku{En ik hoor daar van.}{de buren dat ze pas om}{tien uur thuis komen}\\

\haiku{Hij zuchtte en keek,.}{haar aan met zijn gedachten}{bij die rij kroegen}\\

\haiku{Ik had wat gespaard,,.}{h\`e en mijn compagnon had}{de rest van het geld}\\

\haiku{Ze stond nog steeds te,.}{wachten een figuurtje van}{overmatig geduld}\\

\haiku{Dus als je liever... -,...}{nog een eindje omgaat Neen}{ik ga met je mee}\\

\haiku{- Neen, zei ze, geen acht, -,.}{slaand op zijn woorden en toon}{hier niet even verder}\\

\haiku{Van der Mark deed open.}{en kondigde mevrouw en}{meneer Hartman aan}\\

\haiku{Ze heeft het zelf nooit,.}{geweten maar ze was mijn}{bloedeigen dochter}\\

\haiku{Dat deze haar man.}{ontliep en daarin slaagde}{was haar onbekend}\\

\haiku{Nauwelijks zaten.}{zij toen Sara belde dat}{ze komen wilde}\\

\haiku{- Ze zijn allerliefst,,.}{zei Lea de sneeuwklokjes in}{een vaasje schikkend}\\

\haiku{Ze was zich bewust.}{van een conventioneel}{begin van gesprek}\\

\haiku{In haar gedachten.}{hield zij zich veel bezig met}{deze familie}\\

\haiku{Neen, inderdaad, het.}{was niet waar dat er hier geen}{overeenkomst bestond}\\

\haiku{Dat doet denken aan,.}{iets onvermijdelijks en}{het klinkt veel te zwak}\\

\haiku{Dan moet je toch blij.}{zijn dat je meer gewonnen}{dan verloren hebt}\\

\haiku{De besten wonnen,,.}{hem in het geestelijke}{en daarin alleen}\\

\haiku{Ze stond droefgeestig.}{en een beetje hulpeloos}{omlaag te kijken}\\

\haiku{En als hij wakker.}{mocht worden ziet hij dat er}{wat voor hem klaar staat}\\

\haiku{Hij vertaalde voor}{zijn gehoor de tekst van niet}{meer geheel klassiek}\\

\haiku{Aan het meisje dat.}{opendeed vroeg hij dan ook niet}{of mevrouw thuis was}\\

\haiku{Een weigering zou.}{deze zuil voor zijn ogen in}{stukken doen vallen}\\

\haiku{Maar denk u dan eens:}{in dat je als moeder van}{je kind horen moet}\\

\haiku{De voorzichtigste.}{vraag zou ongewoon zijn en}{achterdocht wekken}\\

\haiku{En het was een mooie,.}{middag echt zo'n weertje waar}{een mens van opfrist}\\

\haiku{- Ze hadden moeten.}{bedenken dat wij ook met}{ons twee\"en waren}\\

\haiku{Hij vond het echter.}{van geen belang en legde}{zich op de tafel}\\

\haiku{Later op de avond,.}{werd hij onrustig kreeg een}{inspuiting en sliep}\\

\haiku{het water dat tot.}{ontzaglijke knotten leek}{samengevlochten}\\

\haiku{Hij wist deze droom,.}{niet alleen op te roepen}{doch ook te leiden}\\

\haiku{Maar hij merkt heel ver,,.}{weg waar de kade eindigt}{nog iets anders op}\\

\haiku{De tijd van spelen.}{met zijn klasgenoten op}{de fiets was voorbij}\\

\haiku{Hij was blij met haar,.}{komst en sloeg geen enkele}{van haar jours meer over}\\

\haiku{Zijn secretaris.}{bracht en haalde hem in het}{kleine wagentje}\\

\haiku{Hij had het met een.}{allernaarste spitsvondigheid}{opzij geschoven}\\

\subsection{Uit: Fantastische vertellingen. Bundel 1}

\haiku{- Dat is niks voor u,,...}{meheer dat weet uwes eve zoo}{goed als ik zellef}\\

\haiku{- Nu, stuur het me dan,.}{morgenochtend en doe er}{de kwitantie bij}\\

\haiku{Een slecht mensch zal zich.}{meer geschokt voelen als wie}{hij slecht dacht goed blijkt}\\

\haiku{Ten slotte meende.}{ik de vervulling van mijn}{wensch nabij te zijn}\\

\haiku{Tusschen een en twee ',.}{uurs nachts had mijn vrouw een}{korten hollen hoest}\\

\haiku{Toen wende ik mij.}{aan een heete kruik mede}{te nemen naar bed}\\

\haiku{Ik had plotseling.}{een hevigen afkeer van}{mijn vrouw gekregen}\\

\haiku{Het duurde dan ook.}{niet lang of ik sliep weer in}{het geheel niet meer}\\

\haiku{De doodstraf was in.}{Nederland sinds tientallen}{jaren afgeschaft}\\

\haiku{Ik kleedde mij als.}{steeds zonder veel gerucht en}{in het duister uit}\\

\haiku{Wel tien maal reeds had.}{ik mij voorgenomen op}{haar toe te kruipen}\\

\haiku{Nog een, en nog een,.}{het tikte al dadelijk}{overal rondom mij}\\

\haiku{Uit een soort doezel,.}{werd ik wakker door een zacht}{glasachtig getik}\\

\haiku{De episode met,.}{de geit bijvoorbeeld daarvan}{rook ik de leugen}\\

\haiku{Mocht dit kunstmatig,.}{zijn aangezet dan was het}{onzichtbaar gedaan}\\

\haiku{- Als je niet later,,.}{komt dan zeven uur vind je}{me nog thuis Tinny}\\

\haiku{een schitterend en,.}{schitterend werk het verhaal}{van een strafproces}\\

\haiku{Langzaam wandelden.}{wij verder in de richting}{van Aerdenhout}\\

\haiku{Op het machtige;}{bordes stonden drie deuren}{wijd-noodend open}\\

\haiku{Ik voelde mij, of.}{ik mijn noodlot tegemoet}{liep en niet weg kon}\\

\haiku{Ik zag telkens om,.}{alsof ik mij in mijn rug}{aangevallen wist}\\

\haiku{De man had geen tijd.}{gehad om het geringste}{geluid te geven}\\

\haiku{Later op den avond,}{ruik ik het onloochenbaar}{en waarschuw ik u.}\\

\haiku{Drie kleine meisjes,,.}{gearmd liepen hem langzaam}{en schuw tegemoet}\\

\haiku{Alleen stonden aan.}{den ingang twee verzakte}{baksteenen kolommen}\\

\haiku{In een winkel kocht.}{hij een broodje dat hij aan}{de toonbank opat}\\

\haiku{Toen eindigde hij:}{met de eerste woorden van}{het Onze Vader}\\

\haiku{Van de overigen.}{bleef de overgroote meerderheid}{uit nieuwsgierigheid}\\

\haiku{De vreemdeling deed,.}{een paar stappen achteruit}{en bleef toen weer staan}\\

\haiku{Men praatte, lachte,,.}{joelde en schuifelde met}{voeten en stoelen}\\

\haiku{het koele licht van.}{het noorden viel er door drie}{hooge ramen binnen}\\

\haiku{Zij keken niet op,.}{terwijl het sinistere}{kloppen weerklonk}\\

\haiku{En toch hadden wij.}{bij die gelegenheid wel}{van ons doen spreken}\\

\haiku{Leiden is niet groot,.}{maar heeft de armoede}{van een wereldstad}\\

\haiku{Hij kwam pas later,.}{op den avond tegen een uur}{of tien aanzetten}\\

\haiku{Tot diep in den nacht.}{kon hij van die dooltochten}{door de stad houden}\\

\haiku{Zwijgend, den rug naar,.}{mij toe ontkleedde Jos van}{der Haerden zich}\\

\haiku{klonk het opeens uit.}{het bed aan het andere}{einde der kamer}\\

\haiku{Het duurde lang eer:}{ik mij goed bewust durfde}{worden van dit feit}\\

\haiku{Niet ten onrechte.}{spreekt men van de schatkamers}{van het geheugen}\\

\haiku{We herinneren.}{ons niet ze ooit gezien of}{gehoord te hebben}\\

\haiku{Nog voel ik daar de,.}{rimpels de vouwen van in}{mijn vingertoppen}\\

\haiku{Dienzelfden dag werd.}{Jos van der Haerden naar}{Endegeest gebracht}\\

\haiku{Ja, jongens, dat is.}{de geschiedenis van Jos}{van der Haerden}\\

\subsection{Uit: Fantastische vertellingen. Bundel 2}

\haiku{Hoe nietig en ver,.}{lijkt mij nu die tijd en hoe}{platvloersch mijn vreugde}\\

\haiku{Men zeide mij dat.}{de vrouw zelf het kind uit het}{raam geworpen had}\\

\haiku{Ik keek neer op den,.}{slaper die veel ellende}{moest hebben beleefd}\\

\haiku{En ik zal u ook,.}{wel eens mijn naam zeggen maar}{nu liever nog niet}\\

\haiku{In een behoefte}{te ontkomen aan den looden}{last der omgeving}\\

\haiku{Een dag later kon:}{de nieuwe kostganger op}{mijn kaartje lezen}\\

\haiku{Hij stak het gaslicht.}{op en wees mij een rieten}{leunstoel bij het raam}\\

\haiku{hij was verdwenen,.}{en toen ik geen antwoord kreeg}{stapte ik binnen}\\

\haiku{Ik merkte beleefd,.}{op dat hij den jongen zoo}{juist had ontslagen}\\

\haiku{Niet alzoo evenwel,.}{in de Jodenbuurt want ik}{kwam er maar weinig}\\

\haiku{schreven ze jiddisch, -}{met hebreeuwsche schrijfletters}{tusschen twee haakjes}\\

\haiku{'s Avonds speelde hij,.}{met mijn ouders kaart en zat}{ik toe te kijken}\\

\haiku{Hij zag op, schichtig,,.}{als betrapt en trok ijlings}{den sleutel terug}\\

\haiku{Maar ach, hij is er, ().}{al zie ikdit met een blik}{op het karretje}\\

\haiku{Ik nam schutterig.}{mijn hoed af en brabbelde}{iets onverstaanbaars}\\

\haiku{En een lachje klonk.}{uit haar keel of er linnen}{vaneen werd gescheurd}\\

\haiku{In de verte zag,.}{ik een rechte streep licht en}{vernam ik geluid}\\

\haiku{Nog een tweede schot,.}{klonk maar ik was reeds naar de}{kamerdeur geijld}\\

\haiku{Neen, het nachtpitje,:}{werd ons devies en daarbij}{versta men mij w\`el}\\

\haiku{het servituut van,.}{soortgelijke uitloozing maar}{dan straalsgewijze}\\

\haiku{En ik poogde het.}{jonge vrouwtje weer in mijn}{armen te trekken}\\

\haiku{Intusschen bleek het:}{geval van Testal haar toch te}{interesseeren}\\

\haiku{- Dat zal dan, poogde.}{ik aan deze gissingen}{een eind te maken}\\

\haiku{IJl klepte de klok.}{van Hornte in onzen rug}{het uur van half tien}\\

\haiku{De Reuzenpit wou,.}{dadelijk binnendringen}{maar ik weerhield hem}\\

\haiku{'s Anderen daags;}{kwam het parket over uit de}{naburige stad}\\

\haiku{grijnsde hij, opende,.}{het portier en trok mij aan}{mijn kraag naar binnen}\\

\haiku{Laat het u genoeg.}{zijn dat ik tot het leidend}{comit\'e behoor}\\

\haiku{De methode om.}{die bergplaatsen te vinden}{was heel eenvoudig}\\

\haiku{De menschen schenen.}{mij te behooren tot het}{uitvaagsel van Praag}\\

\haiku{Ondanks mijn afkeer.}{van paarden moest ik toch dit}{dier bewonderen}\\

\haiku{En toch bleef Praag nog.}{steeds iets van zijn bekoring}{voor mij behouden}\\

\haiku{En daarmee was het,.}{uit en de Pragers gingen}{vergenoegd verder}\\

\haiku{was het laatste wat...-,,,.}{ik dacht ~  Nou nou word es}{wakker jongenlief}\\

\haiku{Kalm aan maar, meneer,.}{van de Kasteelbergen sla}{uw oogen gerust op}\\

\haiku{Bovendien hoopte,.}{ik op afleiding en zoo}{begon ik de reis}\\

\haiku{De zeden- en.}{andere politie is}{verduiveld waakzaam}\\

\haiku{Dan kan ik u nog.}{een adres noemen waar men u}{misschien helpen kan}\\

\haiku{Ik heb hem al over,}{u gesproken maar omdat}{ik niet zeker wist}\\

\haiku{Achter mijn hielen.}{sloeg Talamon de voordeur}{dicht dat het dreunde}\\

\haiku{Intusschen, het had,,.}{mij met een flesch ouden port}{weer kracht gegeven}\\

\haiku{Onderaan stootte,.}{ik tegen iets hards dat mij}{den weg versperde}\\

\subsection{Uit: Fantastische vertellingen. Bundel 3}

\haiku{Zware luiken met.}{ijzeren bouten waren}{voor de twee vensters}\\

\haiku{Ik viel plat op mijn,.}{buik maar mijn handen voor mij}{plasten in water}\\

\haiku{- Ik zal je zeggen,.}{waar je bent dat is waar je}{altijd bent geweest}\\

\haiku{De oude was in,.}{deze buurt wel bekend maar}{trok niet de aandacht}\\

\haiku{Want op het bed in,.}{het midden der kamer lag}{de oude slapend}\\

\haiku{Maar dan had hij, hij,.}{alleen ook de hand gehad}{in haar verdwijning}\\

\haiku{Ik snelde naar den,.}{tuin maar de ramen waren}{alle gesloten}\\

\haiku{Zijn tijd was omstreeks,.}{vier uur en steeds bleef hij in}{de nabijheid}\\

\haiku{Dikwijls was ik 's,.}{avonds alleen en vrij om door}{de stad te dwalen}\\

\haiku{Dit  gruwelstuk.}{stelde het vorige nog}{weer in de schaduw}\\

\haiku{Gewoonlijk waren.}{dat habitueele dronkaards}{en zwakzinnigen}\\

\haiku{Aldus eindigde.}{gelijk een droom de stilste}{dag van mijn leven}\\

\haiku{Zelfs voelde ik mij.}{in dezen tijd relatief}{kalmer dan anders}\\

\haiku{Op Oudejaar werd,.}{Londen bezocht door een mist}{den echten ditmaal}\\

\haiku{Hij liep te rillen.}{en te klappertanden en}{zijn gelaat zag blauw}\\

\haiku{In vermetelheid.}{overtrof deze daad alle}{voorafgegane}\\

\haiku{Zij volgen hier voor:}{wie iets nader komen wil}{tot de verklaring}\\

\haiku{Van zijn afkomst, naam - -.}{en leeftijd we zeiden het}{reeds weten wij niets}\\

\haiku{De schrijftrant wijzigt.}{zich geheel waar het vers tag}{der moorden aanvangt}\\

\haiku{En neen, Bob, ik wil;}{je nu geen oogenblik meer}{in spanning laten}\\

\haiku{hoe zonderling ik}{met mijn aankondiging te}{willen gaan rusten}\\

\haiku{Ze lag zoo stil dat,.}{ik dacht dat ze sliep en zelf}{dommelde ik in}\\

\haiku{Met dat al weerhield.}{mij iets den vreemdeling mijn}{rug toe te wenden}\\

\haiku{mijn stappen knerpten,.}{ik rilde en het angstzweet}{stond op mijn voorhoofd}\\

\haiku{Dat geloof je toch,,,?...}{niet van me is het wel jij}{mijn beste vriendin}\\

\haiku{Lang bleef ik ook niet.}{beheerscht door het gevoel}{van gekrenkt te zijn}\\

\haiku{Lilian had me}{toen reeds zoo in haar macht dat}{ik daar zat gelijk}\\

\haiku{Ik zal de eenige,.}{vrouw ter wereld zijn die ooit}{met een aap trouwde}\\

\haiku{{\textquoteright} Het is een gissing,,.}{meer niet maar mijn gevoel zegt}{me dat ik juist raad}\\

\haiku{, en dat wanneer ik.}{het gekund had ik het niet}{zou hebben gewild}\\

\haiku{, en als eenig antwoord.}{van dit haast stomme gedrocht}{zijn galmenden hoest}\\

\haiku{, ik nam haar in mijn.}{armen  en ik droeg haar}{naar huis als een kind}\\

\haiku{Maar het zijn alleen.}{de gebeurtenissen die}{den afstand maken}\\

\haiku{Halverwege vond.}{ik een voetpad dat recht op}{de villa aanliep}\\

\haiku{Het goed dat aan het.}{leven zijn groote waarde geeft}{en zijn eenig doel}\\

\haiku{Mijn liefde had, van,;}{mijzelf ongeweten de}{vrees doen inslapen}\\

\haiku{Aan den rand van het.}{ravijn strekten wij ons uit}{op mos en varens}\\

\haiku{Het pad voerde door,.}{een bosch bedropen met een}{fel stralend herfstgeel}\\

\haiku{Het kerkhof, voorheen,.}{bolwerk van  den dood lag}{zonder verschrikking}\\

\subsection{Uit: Rood paleis}

\haiku{Hier Tijs, Tijs heet je,,.}{ik weet het nog bliksems goed}{neem een sigaret}\\

\haiku{Wij zijn niets en we,.}{weten dat we iets moesten zijn}{dus twijfelen we}\\

\haiku{Hij had altijd een.}{kleine voldoening als hij}{zooiets had gezegd}\\

\haiku{Maar de roode stoep,.}{bleef verlaten de mannen}{waren verzwolgen}\\

\haiku{Ze lachten langs hem,.}{met mooie bijtmonden een groote}{en een kleinere}\\

\haiku{Tijs hoopte dat hij.}{bij zijn impotentie zou}{kunnen volharden}\\

\haiku{Zij moederde op.}{de vervaarlijke manier}{van een dwingeland}\\

\haiku{In de vlammen zag.}{haar gezicht schrikwekkend licht}{en ongestadig}\\

\haiku{In het licht maakte.}{zijn haar het effect van een}{kleurloos aureool}\\

\haiku{Een heelen tijd bleef.}{hij zoo in de duisternis}{met twee dotten vuur}\\

\haiku{Dat voelde hij zelf,,.}{daarvoor spande hij zich in}{daarop was hij grootsch}\\

\haiku{- Pas op, zei hij, want.}{ze liep bijna tegen de}{mand op met den hond}\\

\haiku{Aan het raam zaten,.}{ze niet dat stond niet voor een}{respectabel huis}\\

\haiku{Het duurde nogal,.}{eer Tijs zijn geld besteed had}{want hij was secuur}\\

\haiku{Het caf\'e In de.}{groote zaal van het koffiehuis}{was het warm en kil}\\

\haiku{Vanzelf kwam hij toen.}{tot het voorstel van een nieuw}{bezoek aan het huis}\\

\haiku{Rond hem, ten allen,.}{kant orgelde Rood Paleis}{zijn regenkoraal}\\

\haiku{Hij dacht vaak het rood.}{van de hoeken te zullen}{zien wegbiggelen}\\

\haiku{Blind, toch veilig liep.}{hij achter den leihond van}{zijn gedachten aan}\\

\haiku{Dan kwam wel een lach,.}{een zwijgende lach van een}{paardengebit}\\

\haiku{Ze was een groote vrouw,,.}{deze Marie van Dam een}{eind in de dertig}\\

\haiku{Het doodsbed - Eh bien,,,,.}{regardez Marie-Laure}{c'est moi madame}\\

\haiku{Tranen kwamen in,.}{zijn oogen hij vond zichzelf zoo}{verschrikkelijk goed}\\

\haiku{Een zaalzuster zag.}{hem na en dacht het woord der}{omstandigheden}\\

\haiku{Het was erg ditmaal,,.}{zag Sauger maar als altijd}{niet onherstelbaar}\\

\haiku{Aldus had God den,.}{mensch geschapen niet slechts naar}{Zichzelf uit Zichzelf}\\

\haiku{Wij traden gereed,.}{uit ons lichaam wij traden}{niet uit onzen geest}\\

\haiku{En het was vreemd, maar.}{in Henry Leroy had de}{koude bewogen}\\

\haiku{Men nam het bestaan,.}{van den gemakkelijken}{kant maar met overleg}\\

\haiku{Ze had geen wrok, het.}{hinderde alleen dat daar}{kapitaal braak lag}\\

\haiku{Op deze slappe.}{heeren hadden zij en haar}{have niet veel greep}\\

\haiku{Hij rookte een paar,,.}{sigaretten bij het vuur}{las zijn krant stond op}\\

\haiku{Onlangs nog had hij,.}{er een begraven een der}{oppervlakkigste}\\

\haiku{Hij had het weer in.}{zijn hoofd gezet den zwaren}{Eduard te tergen}\\

\haiku{De leveranciers.}{en hun personeel kwamen}{nooit aan de hoofddeur}\\

\haiku{Het mysterie van.}{den bijkelder verdiende}{benut te worden}\\

\haiku{Het gaf vertrouwen.}{dat het een gehuwd man was}{met drie kinderen}\\

\haiku{Hij zei het ronduit,.}{want hij zat om den drommel}{niet onder de plak}\\

\haiku{Hij stak vol werklust,.}{maar was te zakelijk om}{zich te overschatten}\\

\haiku{Hij kon zijn triomf,.}{niet kwijt hij liep er dien avond}{mee naar Rood Paleis}\\

\haiku{Hij dronk niet heel veel,.}{champagne maar vandaag kon}{hij er slecht tegen}\\

\haiku{De waardin had hem.}{gezegd dat ze voor zaken}{naar Marseille ging}\\

\haiku{Hij opende een raam,.}{het rijtuig liep dicht achter}{de locomotief}\\

\haiku{Dan naar de oude.}{haven en de zeemansbuurt}{achter het stadhuis}\\

\haiku{E\'en ding maar wist hij,,:}{zeker en had hij stellig}{geweten altijd}\\

\haiku{Hij vond het groene,.}{gif niet lekker en liet het}{bij een enkel glas}\\

\haiku{Keurige oude.}{heeren als hij kwamen niet}{in het strafbankje}\\

\haiku{Op zijn ergst werden.}{ze opgeborgen in een}{sanatorium}\\

\haiku{Gemakkelijk wist.}{hij al deze gedachten}{uit te schakelen}\\

\haiku{Die in het zwart had,.}{al grijzend haar maar daarom}{was ze nog niet oud}\\

\haiku{En het scheen  of.}{de meisjes onrustiger}{waren dan vroeger}\\

\haiku{Tastenbreker en.}{Leroy hield stand temidden}{van de uitbarsting}\\

\haiku{De heeren hadden,.}{afgedaan en met hen de}{spelen der heeren}\\

\haiku{Hij was ten slotte,.}{vader en had een instinct}{van bezorgdheid}\\

\haiku{Een derden keer liet,.}{hij zich niet afschepen dat}{zat niet in zijn aard}\\

\haiku{Hij voerde haar in.}{een kleine spreekkamer aan}{het eind van de gang}\\

\haiku{Het gelaat was zeer,.}{groot en zeer hevig maar heel}{anders dan vroeger}\\

\haiku{Ze ging het land uit,,.}{naar Spanje ze had nog geen}{bepaalde plannen}\\

\haiku{Als hij het lang liet.}{staan dekte het nog niet de}{kosten van opslag}\\

\haiku{Ze had daar kunnen.}{ingaan en uitgaan zonder}{te zijn opgemerkt}\\

\subsection{Uit: Verzamelde verhalen}

\haiku{het werd niet gebruikt,.}{om te slapen en er kwam}{nooit iemand voorbij}\\

\haiku{Maar toen de leeftijd.}{van het meisje bleek legde}{hij zijn potlood neer}\\

\haiku{Hij zag hem haastig.}{een jas uittrekken en door}{het vertrek gooien}\\

\haiku{{\textquoteright} {\textquoteleft}Nee,{\textquoteright} antwoordde de, {\textquoteleft}}{jonge minnaar die nu zelf}{zekerheid verkreeg}\\

\haiku{Zijn inbreuk op de.}{huiselijke gewoonten}{kon argwaan wekken}\\

\haiku{Want hij keek eerst in,,:}{de tuin dan naar zijn vriend en}{antwoordde langzaam}\\

\haiku{Maar dat kon anders,.}{worden en hij voelde zich}{verre van gerust}\\

\haiku{Hij wist niet dat het.}{in de tussentijd opnieuw}{nacht was geworden}\\

\haiku{De ogen waren nu,,.}{wijd open maar er was geen licht}{in geen enkel}\\

\haiku{Het noemen van de.}{naam Moona Wilson werd voor}{beiden een noodlot}\\

\haiku{Deze bundel hield.}{ze echter zorgvuldig voor}{Graham verborgen}\\

\haiku{Kram werd na een poos.}{afgelost door Immerzeel}{voor de  nachtdienst}\\

\haiku{Maar wel verhuisde.}{het zo gauw mogelijk naar}{een andere buurt}\\

\haiku{Hij schaamde zich haast.}{voor zijn blanke benen van}{intellectueel}\\

\haiku{Een enkele maal;}{zonk de reiziger tot de}{enkels in het zand}\\

\haiku{het was hem bij zijn,.}{nuchterheid onverklaarbaar}{haast onaangenaam}\\

\haiku{Ja, zijn zoon kreeg een,.}{mooie begrafenis dat was}{tenminste een troost}\\

\haiku{Hier was de open deur.}{naar de achterkamer met}{het lijk van zijn zoon}\\

\haiku{Halman hoopte dat,.}{niemand iets merkte maar hij}{was ronduit verbluft}\\

\haiku{Natuurlijk, dacht hij,.}{weer in zijn beperktheid dat}{valt elke vrouw op}\\

\haiku{Dat is gebeurd in,,.}{dat andere land ik weet}{niet waar en ook hier}\\

\haiku{Zijn naam wordt me niet,.}{geopenbaard ook niet in}{de initialen}\\

\haiku{Het was met een bril,?}{begonnen en waarop was}{het uitgelopen}\\

\haiku{Daar viel niet aan te,.}{twijfelen en daar lag dus}{niet het gevaar in}\\

\haiku{{\textquoteleft}Jongen,{\textquoteright} zei ze, {\textquoteleft}zou,?}{je dat nu wel doen en dan}{vlak voor je examen}\\

\haiku{De ondergrondse.}{gruwelkamer viel hem per}{saldo toch niet mee}\\

\haiku{En dan waren er.}{toch enkele grappige}{momenten geweest}\\

\haiku{De massa bleef zo,.}{ongeveer op het oude}{peil ook geestelijk}\\

\haiku{Aldus ging het ook.}{in die richting tot in het}{oneindige voort}\\

\haiku{Nu bevond meneer.}{Kars zich in de eerste laan}{van de villabuurt}\\

\haiku{Hij wilde de streek.}{na meer dan twintig jaren}{nog eens terugzien}\\

\haiku{Diende het wellicht,?}{enig vrachtvervoer dit oude}{spoorwegstation}\\

\haiku{De man stond ervoor,.}{tussen de stammen van de}{rijzige houtwal}\\

\haiku{De chef zette zich.}{na een enkele hoofdknik}{tegenover de man}\\

\haiku{En ook de chef zelf.}{moest van zijn eigen diepte}{onkundig wezen}\\

\haiku{En daaroverheen vond.}{hij het wonderlijk deze}{figuur hier te zien}\\

\haiku{Het geruis leek een.}{soort fonetisch kiekeboe}{met hem te spelen}\\

\haiku{Ik moet er weer eens.}{uit en heb een week vervroegd}{verlof genomen}\\

\haiku{Waar ik naar toe ga,.}{weet ik nog niet precies maar}{ik zalje schrijven}\\

\haiku{Hier was de natuur.}{minder gevorderd dan daar}{waar hij vandaan kwam}\\

\haiku{Tot zover  dacht.}{hij onwetend wezenlijk}{als zijn schoonvader}\\

\haiku{Hij deed het met een.}{keurigheid nauwelijks meer}{van deze aarde}\\

\haiku{De reactie op.}{deze verrassing evenwel}{was zeer ongelijk}\\

\haiku{Gerrit, allerminst,.}{van de vlugsten bleef stokstijf}{op dezelfde plaats}\\

\haiku{We hebben Vermeers.}{Meisjeskopje overgedrukt}{over Rembrandts Nachtwacht}\\

\haiku{U hebt een huisvriend.}{die u vrezen doet voor uw}{huwelijksgeluk}\\

\haiku{Mocht er dan toch iets,.}{gebeuren dan bent u er}{tenminste zelf bij}\\

\haiku{Trouwens, er waren.}{reeds bepaalde vermoedens}{bij mij gerezen}\\

\haiku{De vraag die ons toen.}{ging intrigeren was w\'a\'ar}{Willem vi stilstond}\\

\haiku{dat er van onze.}{soort een onbegrensd aantal}{in de kosmos hing}\\

\haiku{We misten iets en.}{hielden het er voor dat we}{de kleine misten}\\

\haiku{Boven zijn zinnen.}{als instrumenten van nut}{rijst hij  nooit uit}\\

\haiku{Met dat al waren.}{we onder elkaar niet meer}{geheel dezelfden}\\

\haiku{Er werd zwaarmoedig,,.}{gekeken tersluiks gezucht}{witjes gelachen}\\

\haiku{Nee, het huis was niet,.}{rechtstreeks getroffen maar de}{luchtdruk wrong het stuk}\\

\haiku{Met grote moeite.}{wrikte hij uit de zijkant}{een portefeuille}\\

\haiku{Het laatste wat hij - -:}{dacht voordat hij droomloos voor}{het eerst insliep was}\\

\haiku{{\textquoteleft}Maar ze mag niet bij.}{me slapen en ook niet}{aan het eten komen}\\

\haiku{Want binnen het jaar.}{te zullen trouwen had hij}{moeten beloven}\\

\haiku{Hij trof dan ook in.}{Zwaagwakum allerminst een}{gespannen gehoor}\\

\haiku{{\textquoteright} zei de vrouw, zonder.}{te weten hoe precies juist}{ze zich uitdrukte}\\

\haiku{Ik heb twee diners,,.}{besteld geen drie want Louron}{eet niet van de kok}\\

\haiku{Hij had opnieuw het,.}{oude huis betrokken hij}{bleef de grote man}\\

\haiku{Nu ja, ze droegen.}{het haar veel korter en met}{meer afwisseling}\\

\haiku{Wellicht zou Falker.}{zonder de klok nooit tot zijn}{daad zijn gekomen}\\

\haiku{Aan de andere.}{kant zou dat misschien toch te}{ondeugend zijn}\\

\haiku{Allemaal nonsens,.}{maar hij zou haar eens flink aan}{het schrikken maken}\\

\haiku{Me dunkt eerder dat.}{dat hier vlotter zal gaan dan}{in het vorige}\\

\haiku{Maar die stip werd met.}{razende snelheid zwarter}{en duidelijker}\\

\haiku{Zelfs lag er daardoor.}{wat extra gezelligheid}{in het vooruitzicht}\\

\haiku{Ze hoeft alleen maar.}{te letten op de deuren}{zonder naamborden}\\

\haiku{Was al dat hout er?}{de oorzaak van dat hij niet}{kon worden gepeild}\\

\haiku{Er liep dwars door dit.}{benedenhuizencomplex}{een centrale gang}\\

\haiku{Aldus bleef Ada bij,.}{de bezoeken aanwezig}{maar ook zij alleen}\\

\haiku{Particuliere.}{klachten raken het bureau}{volkshuisvesting niet}\\

\haiku{toch volbracht hij, met,.}{veel pijn zijn dagelijkse}{gang naar het bureau}\\

\haiku{Hij zorgde er wel.}{voor dat er geen getuigen}{beschikbaar waren}\\

\haiku{En middelerwijl.}{deinden de passagiers als}{een volgzaam getij}\\

\haiku{vroeger moet het toch,.}{beter zijn geweest en in}{elk geval bonter}\\

\haiku{Ze bukte naar het,:}{koffertje op de treeplank}{maar hij was haar voor}\\

\haiku{Er moesten jasknopen,.}{sneuvelen er moest kwetsbaar}{garnituur scheuren}\\

\haiku{Onderwijl had hij.}{ontdekt waarin dat bij haar}{onbepaalde school}\\

\haiku{Het gaf haar iets wreeds,.}{en dat werd versterkt door die}{beitels van tanden}\\

\haiku{Ze werd aangedrukt.}{tegen de stang die ze niet}{had losgelaten}\\

\haiku{Coentje en Mieltje.}{Coentje was nu negen en}{Mieltje nog pas zes}\\

\haiku{geen machientje en.}{zijn broer moest ze toch wel aan}{een draad voorttrekken}\\

\haiku{Ook ontwaakt hij 's,,.}{nachts \'e\'en of twee keer voor een}{half uur een kwartier}\\

\haiku{Er bestaat nergens,.}{uitwijkmogelijkheid want}{er is ook geen links}\\

\haiku{Eva mag daar niets van.}{weten en hij verdiept zich}{nu in zijn werk}\\

\haiku{Ze weten dat hun.}{gedachten parallelle}{wegen zijn gegaan}\\

\haiku{Daar stampten dof in,.}{zijn rug stappen en weer stond}{hij en wendde zich}\\

\haiku{{\textquoteright} {\textquoteleft}Dat weet ik, maar ik.}{ben een oude vriend en ik}{kom hem iets brengen}\\

\haiku{We wachten u al,.}{een halve dag en langer}{meneer Reiziger}\\

\haiku{Enfin, dit huis was,,.}{toch een hol h\`et hol en de}{uitlegging nabij}\\

\haiku{{\textquoteright} {\textquoteleft}Daar vraag je me iets,.}{waarop ik geen antwoord zou}{weten Reiziger}\\

\haiku{{\textquoteright} vroeg de reiziger,,.}{want hij kende het en er}{was schrik in zijn toon}\\

\haiku{De zender was dus.}{op de een of andere}{manier ingelicht}\\

\haiku{Mocht dit een begin,.}{zijn van wereldondergang}{welnu in godsnaam}\\

\haiku{Het was geen aanloop;}{tot een polemiek over de}{wereldondergang}\\

\haiku{{\textquoteright} {\textquoteleft}Dat zou ik van de...}{omstandigheden willen}{laten afhangen}\\

\haiku{Hij hield echter de.}{vragen kort en zo waren}{ook zijn antwoorden}\\

\haiku{Ze liet de voordeur,.}{open want ze had zich al reeds}{enigszins herwonnen}\\

\haiku{In het ziekenhuis.}{bracht dokter Kramer beiden}{naar een wachtlokaal}\\

\haiku{Mogelijk had ze {\textquoteleft}{\textquoteright}.}{na de daad ervaren dat}{ze vanhem niet hield}\\

\haiku{Haar vlug verstand trok.}{uit het verhaalde een heel}{andere slotsom}\\

\haiku{Zijn medelijden.}{met de moeder verdrong haast}{zijn eigen verdriet}\\

\haiku{Zijn positie van.}{procuratiehouder liep}{geen moment gevaar}\\

\haiku{De standseer verbood.}{het en de standseer ging de}{Kas boven alles}\\

\haiku{De eigenlijke,.}{werklokalen het sanctum}{bereikte hij niet}\\

\haiku{Zijzelf had met de,.}{kinderen gegeten en}{zou nu opdienen}\\

\haiku{Doch eensklaps keerde,,.}{zij om en liep even terug}{een winkel binnen}\\

\haiku{Hun gesprek was nu,.}{wat meer alledaags maar ook}{vertrouwelijker}\\

\haiku{Om de benauwing.}{te verdrijven stond hij op}{en rekte zich uit}\\

\haiku{Daar, eenzame,{\textquoteright} zei,.}{ze met een glimlach die hem}{gelukkig maakte}\\

\haiku{Want deze was de,.}{enige die in het dolle}{plezier niet deelde}\\

\haiku{Nog lang kon Teyne.}{hun drukke stemmen horen}{klinken in de nacht}\\

\haiku{Het is tenslotte.}{even vervelend als altijd}{ernstig te wezen}\\

\haiku{Maar uit vrees nog meer.}{te bederven ging hij er}{niet verder op door}\\

\haiku{{\textquoteleft}Ik had permissie,.}{tot de beek met u mee te}{gaan juffrouw Heding}\\

\haiku{En denk u ook niet,.}{dat ik hierin verschil van}{andere mannen}\\

\haiku{ik zie je althans,,;}{zo en ik zou wel dwaas zijn}{als ik dat verzweeg}\\

\haiku{{\textquoteright} Teyne deelde mee,.}{dat zulk een artikel in}{zijn schaapskooi ontbrak}\\

\haiku{onze twee zoons vlak,,...}{na elkaar getrouwd het huis}{uit en het land uit}\\

\haiku{Zijn honden waren.}{gauw moe van het stoeien en}{liepen in zijn spoor}\\

\haiku{Toen riep de vrouw van.}{de houtvester hen binnen}{voor de koffie}\\

\haiku{Het was de eerste,.}{die hij meemaakte als lid}{van de toneelclub}\\

\haiku{{\textquoteleft}Ze gaat met de herfst,{\textquoteright}, {\textquoteleft}.}{weg schreef hijmet haar ouders}{weer naar Utrecht terug}\\

\haiku{daarvoor stond hij te.}{spoedig kritisch tegenover}{al wat hem overkwam}\\

\haiku{het was het gezond.}{vitale verlangen van}{de man naar de vrouw}\\

\haiku{{\textquoteleft}Herinner je je,,,}{dan niet dat je eens toen we}{samen wandelden}\\

\haiku{Maar Kersti, besef,,?}{je wel wat je nu op dit}{moment voor me doet}\\

\haiku{de kleine kerel;}{plaatste het ene been telkens pal}{voor het andere}\\

\haiku{De Tombelaine;}{is alleen van het oosten}{af te naderen}\\

\haiku{Verschoor nam uit een,.}{muurkast een klein voorwerp dat}{zwaar leek te wezen}\\

\haiku{{\textquoteleft}Ik zal beginnen.}{met de grootste diepte te}{fotograferen}\\

\haiku{even een schittering,,.}{er floot iets door de lucht en}{dan was alles stil}\\

\haiku{De eerste honderd.}{meter ging de duikerklok}{zeer langzaam omlaag}\\

\haiku{Naarmate het licht.}{afnam verzocht Verschoor de}{gang te versnellen}\\

\haiku{het staaldraad wond zich.}{geluidloos van de ene klos}{na de andere}\\

\haiku{En het was voor ons:}{beiden een verademing}{toen het antwoord kwam}\\

\haiku{Met een nog vage.}{ontzetting meende ik te}{zien dat het bewoog}\\

\haiku{{\textquoteright} Opnieuw had ik met.}{een blik van wilde angst door}{de sleuf gekeken}\\

\haiku{Ik moest mij buigen,,.}{niet slechts voor mijn meester maar}{ook voor het genie}\\

\haiku{De stukken pasten -.}{in elkaar tot een lucht en}{waterdicht geheel}\\

\haiku{Toen ik bij kennis,.}{kwam was ik op dek zonder}{mijn meester en vriend}\\

\haiku{Men zag en voelde.}{hem overal onmiddellijk}{als superieur}\\

\haiku{Och kaptein, als ik,.}{aan hem terugdenk dan kan}{ik bedroefd worden}\\

\haiku{Aan de bediening.}{van de wijn schonk hij altijd}{zijn volle aandacht}\\

\haiku{Bijna struikelde,.}{ik in het open graf maar er}{was daar niemand meer}\\

\haiku{En dit voornemen,,.}{vreemd als het zijn mocht deed mij}{toch sympathiek aan}\\

\haiku{Een ogenblik als het.}{huidige moest in stilte}{worden ondergaan}\\

\haiku{Kalkemeijer zag,.}{vanuit de diepte naar haar}{op bleek en ernstig}\\

\haiku{facile princeps,.}{leerde ik indertijd op}{het gymnasium}\\

\haiku{Dwaasheid dat hij met;}{een staf tegen de rots zou}{hebben geslagen}\\

\haiku{Het schoonhouden van;}{mijn bonte winkelvoorraad}{is zijn grootste trots}\\

\haiku{{\textquoteleft}Nee maar, da's \'o\'ok 'n,{\textquoteright}.}{bak zegt hij vrij luid achter}{zijn hand tegen mij}\\

\haiku{Maar zou u denke?}{dassij nou niks beginne}{ken om die cente}\\

\haiku{Ik kan er volstrekt,.}{niet uit wijs worden hoeveel}{moeite ik me geef}\\

\haiku{Maar ook als de wil:}{zich splitst zie ik toch dat \'e\'en}{de overhand behoudt}\\

\haiku{Ik druk vaster mijn,.}{arm om het wonderkind om}{mijn eigen jongen}\\

\haiku{jullie ken me de, -...}{moord stikke en medeen}{draai ik de bak in}\\

\haiku{Mijn hemel, dat zijn!}{nu de voldoeningen van}{de man uit het volk}\\

\haiku{Nee, in zijn hand is}{een bot mes waarmede hij}{eerst onbeholpen}\\

\haiku{Twee ontzaglijke.}{zeugen kwamen er knorrend}{over een hek kijken}\\

\haiku{{\textquoteright} informeerde de,.}{bezoeker die graag enige}{kunde wou tonen}\\

\haiku{{\textquoteright} Boer Berdeur draaide.}{nadenkend de steel van zijn}{pijp rond in zijn oor}\\

\haiku{Moar hai wou en hai.}{zou met die Azaintje van De}{Joager trouwe}\\

\haiku{Ik mag d\^a vollek, ',. '}{niet ben opscheppers main}{nie kerreks genog}\\

\haiku{Moar hai blaift met z'n,,...}{pote van d'r af en nog}{es gelaik het ie}\\

\haiku{Het werd steeds geplaatst,,.}{meest in feuilletonvorm en}{gaarne gelezen}\\

\haiku{Hij worstelde zich,.}{van de droom los en ging recht}{zitten in zijn bed}\\

\haiku{Hij trok aan de deur.}{die dadelijk toegaf en}{stond nu in de stal}\\

\haiku{de justitie had.}{ongetwijfeld de zaak nog}{niet opgegeven}\\

\haiku{Het meisje werd van.}{de weg getild en op een}{hand wagen gelegd}\\

\haiku{Van het meisje weet.}{ik alleen dat ik haar op}{de grond zag liggen}\\

\haiku{Hij moest ook zien een.}{of twee politiemannen}{mede te krijgen}\\

\haiku{In een oogwenk was.}{Piquillo beneden en}{bij de anderen}\\

\haiku{Hij had, gelijk men,.}{dat noemt een verborgen snaar}{in mij doen trillen}\\

\haiku{Mocht je zijn woorden,.}{geloven dan had hij er}{als een vorst geleefd}\\

\haiku{{\textquotedblright} Ik kon in de verste.}{verte niet raden wat hij}{me had voorgezet}\\

\haiku{Nu pauzeerde hij,.}{en hij sloeg zijn ogen op ter}{hoogte van mijn maag}\\

\haiku{- maar ik wil en ik}{zal toch ook eens werkelijk}{genieten van die}\\

\haiku{En toch viel zijn woord.}{in mij als het zaaizaad in}{de akkervore}\\

\haiku{Hoe ook gehecht aan}{aardse goederen hechtte}{hij niet het minst aan}\\

\haiku{Hij kon ook in de.}{eerste kamer gaan met niets}{dan het lichtpeertje}\\

\haiku{Nadat de ander.}{was vertrokken sloot hij het}{geld in zijn bureau}\\

\haiku{welke hen om zich.}{groepeerden tot een kleine}{klas om de meester}\\

\haiku{Denk er dan om, de...}{volgende keer donder ik}{je op de keien}\\

\haiku{Het was het grote,...}{ogenblik dat woorden tot niets}{dienden maar daden}\\

\haiku{En toch kwam hij slechts}{tot het besluit de woning}{voor welker behoud}\\

\haiku{Toen ging hij terug.}{naar zijn kamer en bleef daar}{de verdere dag}\\

\haiku{3Valt chronologisch.}{tussen de eerder openbaar}{gemaakte brieven}\\

\section{Henri Borel}

\subsection{Uit: Een droom}

\haiku{Ik hoorde niet veel.}{van hem toen ik weer terug}{was in Soerabaia}\\

\haiku{Ik schijn dan toch w\`el.}{in het rijk der wonderen}{te zijn aangeland}\\

\haiku{En ik begrijp maar.}{niet hoe ik ooit iets in haar}{gezien kan hebben}\\

\haiku{Ze zijn allen zoo.}{ongedwongen en zoo lief}{en zoo hartelijk}\\

\haiku{*** 's Ochtends, dikwijls,.}{al om tien uur begint een}{somber droomen-spel}\\

\haiku{Het lijkt nu alles,,....}{eeuwen eeuwen oud en van}{een v\`er verleden}\\

\haiku{Ik staarde, staarde,,.}{maar niets was te zien dan de}{dikke grijze mist}\\

\haiku{De rug licht-grijs.}{met een smal dons van wit bont}{als rand er omheen}\\

\haiku{Die zwartende sneeuw,.}{daalt over het duistere dal}{zwaar-gedragen}\\

\haiku{Om half zeven ging '.}{ik nog even wandelen in}{t Leverlaantje}\\

\haiku{Je voelt het pure.}{leven wijd-ademend in}{je lichaam vloeien}\\

\haiku{*** We\^er een wandeling,....}{in het lievelingspaadje}{bij vallenden avond}\\

\haiku{Ik begin mij zoo....}{aan dat vroolijke vrouwtje}{Annie te hechten}\\

\haiku{Maar al dichter en,,.}{dichterbij zonder te zien}{voel ik den krater}\\

\haiku{Alles is hier zoo,,.}{ganschelijk puur en recht en}{kuisch van wezen}\\

\haiku{Het gaan is door een,,.}{vaag grauw duister onzeker}{en mysterieus}\\

\haiku{{\textquoteright} {\textquoteleft}- Neen, heusch niet, je, '.}{weet wel dat ikt van jou}{wel verdragen kan}\\

\haiku{En zal je dan niet?}{v\'e\'el ongelukkiger zijn}{dan v\'o\'or je hier kwam}\\

\haiku{En dat is onrecht,,,,.}{Ru dat is wreed wreed onrecht}{dat kan nooit goed zijn}\\

\haiku{'t Is heelemaal.}{zonder iets onreins geweest}{wat ik voor haar voel}\\

\haiku{of wel, als ik haar, {\textquoteleft}!....}{tegenkom en hoe ze dan}{melodieusD\'ag}\\

\haiku{Rein als een jonge '.}{God rees Ardjoen\r{a} omhoog}{int morgenlicht}\\

\haiku{Als ze op een hoog.}{punt is gekomen staat ze}{stil en wacht op mij}\\

\haiku{Opeens, ergens, een,,.}{open plek waar boven lichte}{blauwe hemel schijnt}\\

\haiku{Ze is een zoete,,....}{rose roos in het groen rank}{oprijzend omhoog}\\

\haiku{Maar nu is alles,....}{weer goed en gewoon en er}{is niets verloren}\\

\haiku{Wild dooreen liggen,.}{groote rotsblokken waar stemmig}{een stroompje ruischt}\\

\haiku{{\textquoteleft}Waar is nu ergens,?}{dat mooie dennelaantje waar}{je zoo over uit was}\\

\haiku{{\textquoteleft}Dank je wel, hoor, ik,....}{ben er h\'e\'el blij me\^e ik zal}{ze goed bewaren}\\

\haiku{Ze gaat w\`eg van je,,,.....{\textquoteright}....}{Rudolf w\`eg w\`eg Maar het lijkt}{te onbestaanbaar}\\

\haiku{Ik voel dat ze mijn,....}{hand drukt en de tranen nu}{langs mijn wangen gaan}\\

\subsection{Uit: Karma}

\haiku{Moeten jullie, die,?}{zooveel  verder zijt dien}{stumpert uitlachen}\\

\haiku{alles wat je in.}{den Tijd rekent is slechts toover}{en begoocheling}\\

\haiku{Ja, dit is wel de}{volheerlijkste liefde die}{bestaan kan tusschen}\\

\haiku{Esdur)  J.S. Bach,.}{stond er boven maar ik}{kon het niet gelooven}\\

\haiku{Het vogeltje rolt,.}{om het kan niet meer precies}{op zijn rug liggen}\\

\haiku{niet dat het blijft is......}{de begoocheling maar dat}{het voorbij zou gaan}\\

\haiku{altijd zal je bij,,......}{mij zijn en ik bij jou al}{weten we het niet}\\

\haiku{Zoo, dan was meneer,, '.}{niet terecht anders had ze}{niett speet haar wel}\\

\haiku{Ja, meneer, weet u.....}{daar staat nu juist een lijkie}{boven de aarde}\\

\haiku{Zij waren elkaar,.}{nu niet meer onverschillig}{zij haatten elkaar}\\

\haiku{al de mis\`ere......}{van het leven komt door het}{willen aanraken}\\

\haiku{De laatste maanden.}{dacht het Moedertje om niets}{dan om het kindje}\\

\haiku{Dokters denken nooit,.}{over het Wonder omdat ze}{veel te veel weten}\\

\haiku{Het was een pracht van,.}{een haan dien ik bij mijn twaalf}{kippen had gekocht}\\

\subsection{Uit: Vlindertje}

\haiku{{\textquoteleft}Je hadt eigenlijk,{\textquoteright}, {\textquoteleft}}{een meisje moeten worden}{zei Ellie altijd}\\

\haiku{Hij was officier.}{geworden omdat Ellie}{het zoo gewild had}\\

\haiku{Hij was dan ook niets,}{grooter dan zij geweest want}{ze was vroeg gegroeid}\\

\haiku{En Ellie was het,,.}{zachtste fijnste teederste wat}{er voor hem bestond}\\

\haiku{{\textquoteright} Ook 's winters was.}{er genoeg te doen voor een}{damemeisje als zij}\\

\haiku{Wat durfde Lize,!}{van Elsmeet een breeden}{grooten hoed dragen}\\

\haiku{Ellie hield van hem,,.}{d\'at was genoeg en daarom}{hield hij hem  hoog}\\

\haiku{{\textquoteright} {\textquoteleft}- Nu, dat zal vooreerst,!}{wel niet gebeuren reken}{d\'a\'ar maar gerust op}\\

\haiku{Denk nu zelf eens de,,,.}{Sandt of den Bergh of Waalen}{die je daar noemde}\\

\haiku{Scheveningen lag.}{in al de glorie van een}{lichten zomerdag}\\

\haiku{Het leven w\`as niet,,!}{leelijk en duister het kon}{niet het kon niet zijn}\\

\haiku{En ze vond zich dan,.}{ook nog maar zoo'n nietig kind}{bij hem den sterke}\\

\haiku{Het leven was hem,}{niets meer waard zonder dat groote}{zalige genot}\\

\haiku{Het jonge mensch moest;}{nu maar eens een geregeld}{leven gaan leiden}\\

\haiku{Dat was al een h\'e\'el,.}{geschikte oplossing vond}{de oude mevrouw}\\

\haiku{Daar werd toch nooit iets,?}{aan veranderd al was ze}{nu ge\"engageerd}\\

\haiku{Ze had hem nu juist,.}{heel erg noodig want er kwam nu}{van alles te doen}\\

\haiku{Niets was veranderd,,,.}{alles stond nog als vroeger}{even te\^er even intiem}\\

\haiku{Ik vind je dan v\'e\'el,.}{te goed voor wien ook v\'e\'el te}{goed voor het Leven}\\

\haiku{{\textquoteright} Hij was op het punt,,!}{om het uit te gillen dat}{hij het k\'ende o}\\

\haiku{Toe, geef me een arm,.}{dan gaan we heel deftig naar}{papa beneden}\\

\haiku{D\'a\'ar was hij al te,.}{ervaren voor en had hij}{te veel me\^egemaakt}\\

\haiku{Zoo heel, heel vreemd was, '.}{het voor hem wat in haar aan}{t gebeuren was}\\

\haiku{Ik heb nu juist zoo'n.}{behoefte om mijn hart eens}{aan je te luchten}\\

\haiku{Je hebt daar het recht,.}{niet toe zoolang je met haar}{ge\"engageerd bent}\\

\haiku{Ik vind dat je een.}{gemeenen streek hebt gedaan}{tegenover Ellie}\\

\haiku{Liever dan haar te.}{besmetten met het vuil van}{de vuile wereld}\\

\haiku{Ik weet heel goed, dat.}{ik geen kerel ben voor een}{meisje als Ellie}\\

\haiku{Dan is het toch in.}{elk geval beter als ze}{nu \'e\'ens wat lijdt}\\

\haiku{de dochtertjes van ....}{fatsoenlijke ouders op}{straat niet meer veilig}\\

\haiku{Een straatjongen bleef,.}{even staan en trok een leelijk}{gezicht tegen haar}\\

\haiku{stil verdwijnen in,,}{het niet in die eindelijk}{weergevonden rust}\\

\haiku{Nu was zij op het,.}{uiterste eind gekomen}{ze kon niet verder}\\

\subsection{Uit: Het zusje}

\haiku{- Ze was heusch al,.}{achttien jaar al zag ze er}{zooveel jonger uit}\\

\haiku{{\textquoteright} En ze zei het hem,,,}{na lief alsof ze met haar}{stem het woord streelde}\\

\haiku{De mooie schitteroogen,,.}{die hem aanzagen waren}{die van een groot kind}\\

\haiku{En met mijn vader,.}{gaat het niet al te best ik}{zie hem bijna nooit}\\

\haiku{Hij was al den hoek,.}{van de straat om toen ze weer}{hard kwam aanloopen}\\

\haiku{Maar \'altijd dieper,,.}{zonk haar hoofd in de zee in}{de zee de groote zee}\\

\haiku{over de sombere,,....}{daken over de duistere}{huizen en verder}\\

\haiku{Ik wil gedenken,.}{een lieve doode zoo diep}{in mij begraven}\\

\haiku{Het is nog maar een,.}{vage tocht door den schijn die}{mij van u w\'eghoudt}\\

\haiku{Ja, het zal \'opgaan,.}{het zal alles \'opgaan in}{onbevlekten staat}\\

\haiku{{\textquoteright} Maar hij was erg blij.}{dat hij haar handjes in de}{zijne kon drukken}\\

\haiku{En hij schaamde zich.}{dat hij anders om dezen}{tijd nog in bed lag}\\

\haiku{'s Avonds om negen,.}{uur wachtte hij haar weer op}{en bracht haar naar huis}\\

\haiku{En den volgenden,,....}{morgen en den volgenden}{avond en trouw z\'o\'o door}\\

\haiku{En dan 's middags!}{het schaatsenrijden op de}{Vijvers in het Bosch}\\

\haiku{Om vier uur ging hij.}{heel zachtjes de school in om}{haar te verrassen}\\

\haiku{Aan het eind, dicht bij,,.}{de kachel zat Mientje het}{hoofd diep gebogen}\\

\haiku{{\textquotedblright}{\textquoteright} {\textquoteleft}- Ja, nu begrijp ik,,{\textquoteright}.}{het nu begrijp ik het pas}{fluisterde zij zacht}\\

\haiku{Ze had het hem nooit,.}{goed kunnen zeggen waarom}{ze dat had gedaan}\\

\haiku{Terwijl zij bezig,.}{was met het zetten zou hij}{wat voor haar spelen}\\

\haiku{Een weldadige,.}{goede vertrouwelijkheid}{lag over de dingen}\\

\haiku{En een broer moet toch........}{altijd verwachten dat zijn}{zuster \'e\'ens gaat}\\

\haiku{Zou zij wel ooit meer?}{van hem kunnen houden dan}{van een grooten broer}\\

\haiku{Ik bedoel dat het,.}{zoo niet altijd kan blijven}{zooals broer en zuster}\\

\haiku{{\textquoteleft}Zeker, Paul, ik vind.}{het allemaal heel mooi zooals}{jij dat nu voorstelt}\\

\haiku{Had het hem niet h\'e\'el,?}{alleen gelaten bijna}{alles wat hij wist}\\

\haiku{Was niet het beste,?}{van alles haar simpele}{hart dat van hem hield}\\

\haiku{Zij was voor hem meer.}{een menschelijke moeder}{dan een hemelsche}\\

\haiku{{\textquoteright} {\textquoteleft}Zeg, Paul, heb je, v\'o\'or,?}{mij wel eens van een ander}{meisje gehouden}\\

\haiku{Maar ik behoef nu.}{ook niet dood te gaan en mag}{wel bij je blijven}\\

\haiku{Hun liefde bleef nu,}{zoo veilig bewaard in hun}{eigen lief geheim}\\

\haiku{Maar nu kwam het, n\'u,.}{kw\'am het hij voelde dat het}{nu beginnen zou}\\

\haiku{{\textquoteright} En ze liepen langs,.}{den Parallelweg naar den}{overstap bij het spoor}\\

\haiku{dat ik je eens zal,.}{geven en dat nu wel graag}{zou willen komen}\\

\haiku{{\textquoteleft}Je kunt toch maar \'e\'en,,.}{ding in mijn oogen lezen Paul}{m\'e\'er staat er niet in}\\

\section{A.L.G. Bosboom-Toussaint, Cd. Busken Huet en Simon Gorter}

\subsection{Uit: Drie vergeten novellen}

\haiku{ik dank u.{\textquoteright} {\textquoteleft}Dan zal.}{ik de vrijheid nemen mij}{te verwijderen}\\

\haiku{{\textquoteright} {\textquoteleft}Omdat mijnheer De.}{Choiseul hem een gezantschap}{opdroeg naar Madrid}\\

\haiku{Gij verplettert uw,!}{chignon en de guirlande}{die er tegen rust}\\

\haiku{Ieder der boeren;}{van het dorp bracht daartoe iets}{van het zijne bij}\\

\haiku{Zij ook gaf hem veel,,.}{maar als eene gift die haar met}{woeker werd beloond}\\

\haiku{want ik wist niet eens,.}{werwaarts mijne moeder mij}{met zich heenvoerde}\\

\haiku{Ik ben de Graaf de,.}{Forbin en gij zult van nu}{aan mijn naam voeren}\\

\haiku{maar verneem gij dan,,....}{uit mijn naam wie het zijn mag}{die gekomen is}\\

\haiku{Geluid en stem, zijn,;}{\'e\'enigst herkenningsmiddel}{had hem getroffen}\\

\haiku{Met het ernstige,....}{plan om er aan te voldoen}{kwam ik naar Parijs}\\

\haiku{De meeste hunner;}{bekenden hadden voor dit}{hoogere geen oog}\\

\haiku{zes dagen zal ik,.}{strijden en den zevenden}{dag ter ruste gaan}\\

\haiku{Dokter Egbert vond;}{het niet streelend alzoo te}{worden afgescheept}\\

\haiku{Julia was op nieuw.}{gaan zitten en de Bijbel}{lag op haar knie\"en}\\

\haiku{Die mijn vleesch eet,.}{en mijn bloed drinkt die blijft in}{mij en ik in hem}\\

\haiku{{\textquoteright} zoover te brengen, dat:}{de dienstmaagd onder buien}{van lachen uitroept}\\

\haiku{Dien volgenden dag.}{was er theevisite bij}{de juffrouwen Zeulig}\\

\haiku{Denk je dat ik na,?}{andermans honden kijk of}{die nat zijn of droog}\\

\haiku{De vijf minuten.}{van hare deftigheid zijn}{bovendien ook om}\\

\haiku{{\textquoteleft}we waren niet in,.}{de wereld om te praten}{maar om wat te doen}\\

\haiku{of 't heusch waar,?}{zou wezen dat de menschen}{nog een ziel hadden}\\

\haiku{- Nou Koos, - had een van -?}{de vriendinnen gezegd hoe}{heb ik het met je}\\

\haiku{ik wou, dat ze ons,.}{leerden denken leerden \'e\'en}{ding goed te weten}\\

\haiku{{\textquoteright} - zei tante, die zich, {\textquoteleft}?}{heel dom kon houden als ze}{verkooswat zou dat}\\

\haiku{De juiste sterfdag.}{van onze lieveling is}{nooit bekend geweest}\\

\section{A.L.G. Bosboom-Toussaint}

\subsection{Uit: Engelschen te Rome. Romantische \'episode uit de regering van paus Sixtus V}

\haiku{, en zoo het in de,.....}{macht eener vreemdelinge}{stond u daarvoor te}\\

\haiku{De klok van eene der.}{naburige kapellen}{sloeg de tweede ure4}\\

\haiku{Eindelijk riep hij,,:}{uit op eenen toon die bijna}{zegepralend was}\\

\haiku{en geen bezoek van;}{den geliefde stelde haar}{gekrenkt hart gerust}\\

\haiku{o! vergeef, gij weet,.}{niet wat een minnaar lijdt die}{zich verbergen moet}\\

\haiku{{\textquoteright} {\textquoteleft}Gij moet inderdaad!}{veel geleden hebben met}{zulke denkbeelden}\\

\haiku{ik hoor den tred van,.}{Signora Respanti die door}{de galerij komt}\\

\haiku{Zijnen weldoener.}{mocht hij niet opofferen}{aan zijne liefde}\\

\haiku{de handelwijze,.}{van dien jongen man had iets}{dat haar ontrustte}\\

\haiku{Van de Prelaten,.}{klimt men tot de Gezanten}{de Roomsche Prinsen}\\

\haiku{{\textquoteright} riep de jongeling,.}{zich telkens tot hoogere}{drift opwindende}\\

\haiku{Reinier, schooner was de,.}{Moedermaagd niet dan zij zich}{huichelend voordoet}\\

\haiku{maar ziet gij iets in,....}{mij dat u eenen bepaalden}{afkeer inboezemt}\\

\haiku{En gesteld nu, dat,,!}{ik het konde dat ik het}{wilde Signora}\\

\haiku{Men prees uwe goedheid,,,.}{men zeide mij dat gij jong}{waart dat gij schoon waart}\\

\haiku{met eene ernstige;}{zwijgende buiging bleef hij}{op eenen afstand staan}\\

\haiku{zijn stem schoot te kort,,.}{om hoorbaar te vragen wat}{men van hem wilde}\\

\haiku{Gelukkige, die;}{eenmaal in de Eeuwige}{Stad geademd hebt}\\

\haiku{Ik visch somtijds in,.}{het kanaaltje dat links af}{den wijngaard omgeeft}\\

\haiku{Het bootje van den.}{opzichter der tuinen is}{altijd tot mijn dienst}\\

\haiku{Gij zult het weten,,;}{dat ik diep voel hoeveel ik}{heb goed te maken}\\

\haiku{Zijne Heiligheid.}{is snel in het uitvoeren}{eener bedreiging}\\

\haiku{heden echter vond,.}{hij het geraden zich te}{laten aandienen}\\

\haiku{Dat woord, hoe zacht ook,;}{uitgesproken klonk eensklaps}{als door eene echo rond}\\

\haiku{{\textquoteright} sprak hij ernstig, {\textquoteleft}uwen.}{jongen neef zoo onverhoeds}{te verpletteren}\\

\haiku{{\textquoteright} Inmiddels trachtte.}{Paolo den vreemdeling}{bij te brengen}\\

\haiku{{\textquoteright} {\textquoteleft}De grootouders van;}{Francesco zorgden voor}{mijne opvoeding}\\

\haiku{het purper had plaats,;}{gemaakt voor het violet}{de rouwkleur der kerk}\\

\haiku{al die monden, die,:}{schijnen te willen spreken}{en die toch zwijgen}\\

\haiku{ik spiegelde mij,....}{in het blauwend kristal gij}{wenddet het niet af}\\

\haiku{mijne gloeiende....}{lippen in aanraking met}{die zachte warmte}\\

\haiku{Hij vreesde daarom,.}{dat Anna veel zoude te}{ontdekken hebben}\\

\haiku{ook lispelde zij,,.}{bijna onhoorbaar woorden}{zonder samenhang}\\

\haiku{{\textquoteleft}Ik zal niet kunnen.}{terugkeeren voor den zondag}{van Quasimodo}\\

\haiku{Men zegt, dat Zijne;}{Eminentie een wild Arabisch}{ros heeft bereden}\\

\haiku{Scipione had.}{uit voorzichtigheid dien eed}{van haar afgeperst}\\

\haiku{Nog smeek ik u uwe,.}{hand af nog bied ik u den}{steun van mijnen arm}\\

\haiku{Nog bied ik u een,;}{leven van zorgelooze rust}{en kalm genoegen}\\

\haiku{{\textquoteright} {\textquoteleft}Gij hebt dan beslist,{\textquoteright},.}{hernam hij en wischte zich}{den laatsten traan af}\\

\haiku{Het was tegelijk.}{gericht tegen den Paus en}{tegen diens zuster}\\

\haiku{hij wist gehoorzaamd,.}{te zijn zoodra men hem}{niet meer tegensprak}\\

\haiku{En daar waren er,.}{velen die er belang bij}{meenden te hebben}\\

\haiku{De Graaf bezocht dan.}{den jongen Kardinaal en}{werd goed ontvangen}\\

\haiku{En de steek van zulk?}{een insect ontrust u en}{brengt u tot wanhoop}\\

\haiku{Beschik over al het,.}{mijne handel daarin zooals}{gij goed zult vinden}\\

\haiku{Laat ons hun toonen,.}{beter te zijn dan hunne}{lage verdenking}\\

\haiku{{\textquoteright} {\textquoteleft}Ik geloof, dat wij.}{die vrouw nog vooraf zullen}{moeten gebruiken}\\

\haiku{Wat ging het haar aan,?}{dat de Heer van Rome het}{anders beschikt had}\\

\haiku{Zij bevorderde;}{gaarne al de belangen}{der Spaansche ligue}\\

\haiku{deze  wraak is,.}{niet die van den mensch zij is}{die van den duivel}\\

\haiku{{\textquoteleft}Gij vindt eenen sleutel,;}{onder de tafel die opent}{u straks dit vertrek}\\

\haiku{{\textquoteright} Na deze woorden,.}{boog hij zich alsof hij niets}{meer te zeggen had}\\

\haiku{die zich grijsaard had,.}{gemaakt naar het hart in de}{volle kracht der jeugd}\\

\haiku{Hij scheen nu zonder,.}{kunst zoo oud als hij vroeger}{had willen schijnen}\\

\haiku{Hij ziet eenen man, die,.}{met drift toesnelt misschien kan}{het een helper zijn}\\

\haiku{Het zal eene laatste,;}{samenkomst zijn een afscheid}{voor deze aarde}\\

\haiku{Castagna speelde;}{verlegen met het gouden}{kruis op zijne borst}\\

\haiku{het geldt hier niet de:}{goed- of afkeuring}{van een Conclave}\\

\haiku{zij mocht zich eenmaal,.}{luide verheffen en het}{kon dan te laat zijn}\\

\haiku{{\textquoteright} zeide hem Sixtus,.}{en leunde het hoofd tegen}{zijne trouwe borst}\\

\haiku{{\textquoteright} {\textquoteleft}Heb dan moed tegen,.}{u zelven wees uwen grooten}{bloedverwant waardig}\\

\haiku{- daar alleen is de;}{veilige wijkplaats voor een}{afgestreden hart}\\

\haiku{Dat mijn William.}{aan mij denke als aan eene}{afgestorvene}\\

\haiku{Zij had altijd stil,;}{voor zich henen geleefd en}{niemand kende haar}\\

\haiku{Hij rekende op,;}{de versche troepen die hem}{waren toegevoegd}\\

\haiku{hij moest iets anders,;}{zijn geweest dan een mensch veel}{meer of veel minder}\\

\haiku{Ik, die nog jaren,!}{van kracht aan Veneti\"e had}{te geven misschien}\\

\haiku{uwe patrici\"ers,;}{hebben geen hart en uw volk}{heeft geen geheugen}\\

\haiku{De patrici\"ers;}{hadden te rekenen met}{hunne kli\"enten}\\

\haiku{de Senaat moest hun.}{de vrijheid van Pisani}{hebben toegezegd}\\

\haiku{Het moet immers den?}{martelaar niet zien waar het}{den admiraal vraagt}\\

\haiku{de grijze held moest.}{zich weer oefenen in het}{loopen als een kind}\\

\haiku{stouter in hare,;}{eischen naarmate zij die}{zag ingewilligd}\\

\haiku{met Rome den rug,;}{toe te keeren is men nog geen}{Christen geworden}\\

\haiku{Zijne kleeding was die,.}{van zijn rang en hij had den}{degen op zijde}\\

\subsection{Uit: De vrouwen van het Leycestersche tijdvak. Deel 2}

\haiku{De eerste overtrof,.}{zelf zijne stoutste hoop als}{wij hooren zullen}\\

\haiku{zullen we van die....?}{twee verliefden met der haast}{een verloofd paar zien}\\

\haiku{{\textquoteright} {\textquoteleft}In uw geval had,{\textquoteright}.}{ik mijn schoonzoon reeds omhelsd}{sprak Kiligrew zacht}\\

\haiku{zoo gij haar ernstig,,!}{meent zoo ik haar u geve}{maak haar gelukkig}\\

\haiku{{\textquoteright} {\textquoteleft}Zoo neem dan die hand,;}{die mijne zachte Ada u}{willig geven zal}\\

\haiku{{\textquoteright} {\textquoteleft}En hebben wij dan?}{hun groeten en glimlachen}{af te bedelen}\\

\haiku{{\textquoteright} {\textquoteleft}Die reuckeloosheid ',;}{ist juist welke ik u}{te verwijten heb}\\

\haiku{Zal het van hem ook,....?}{gezegd worden dat hij tot}{de Reingoudisten hoort}\\

\haiku{Mijn gehoor zal mij,.}{misleid hebben gelijk ik}{mijne oogen wantrouw}\\

\haiku{maar ik acht, dan zal....}{hij bevel geven tot een}{heimelijken dood}\\

\haiku{{\textquoteleft}aan die zijde houdt,.}{men volgaarne mits men niet}{behoeft te geven}\\

\haiku{ik zal vaststaan in '{\textquoteright}.}{t gedenken aan u. En}{hij ging van haar heen}\\

\haiku{maar dat spreekt vanzelf,{\textquoteright}, {\textquoteleft} '?}{riepen allen uit \'e\'en mond}{wie zout durven}\\

\haiku{{\textquoteleft}ik ook zou mij wel.}{genoeg geveiligd achten}{met zijne hulpe}\\

\haiku{gij zult niet gaan, om.}{ons aan uw Engelschman}{over te leveren}\\

\haiku{allen dus dankten.}{of verontschuldigden zich}{op eenige wijze}\\

\haiku{Reeds voelde, reeds zag,;}{het Ada welhaast ook zouden}{anderen het zien}\\

\haiku{Daarna, begrijpt ge,....}{zou zijne tuigenis voor}{u niet veel gelden}\\

\haiku{Genoeg voor ons, om,:}{u te bevestigen in}{hetgeen gij vermoedt}\\

\haiku{zij voelde afschuw.}{van zich zelve onder dit}{akelige masker}\\

\haiku{In dit kisje vindt,;}{ge goud tot omkooping van}{wie gij winnen wilt}\\

\haiku{maar, terwijl hij de,:}{trap opsteeg mompelde hij}{in het Italiaansch}\\

\haiku{want gij weet niet, hoe,....;}{men dien man haat  als men}{hem eens haten kan}\\

\haiku{alleen, hoe kon zij,?}{plaats vinden en vooral hoe}{kon zij stand houden}\\

\haiku{{\textquoteright} vroeg Barneveld, die.}{verlangde op het hoofdpunt}{terug te komen}\\

\haiku{{\textquoteright} {\textquoteleft}Dat is zekerlijk,!}{minst te betreuren door u}{heer van Barneveld}\\

\haiku{{\textquoteright} {\textquoteleft}Wie kan de mate,?}{der schuld berekenen waar}{men zulk lijden ziet}\\

\haiku{{\textquoteleft}Ik heb onder eed,{\textquoteright}.}{beloofd dat niet te zeggen}{hernam de krijgsman}\\

\haiku{een edelman met twee;}{dienaren overmocht toch wel}{een paar boosdoeners}\\

\haiku{van het weerzien van....}{Fabian zou ik veel voor}{haar durven hopen}\\

\haiku{ik heb er velen,,.}{als gij weet maar toch is dit}{eene die ik niet ken}\\

\haiku{Lady Margaret,.....}{was omdat ik zweeg en hun}{niet wilde zeggen}\\

\haiku{met mij, en tot het,,.}{doel dat gij weet begeef ik}{mij met haar van hier}\\

\haiku{sinds Gideon als,.}{leeraar te Utrecht staat is}{hij half uw hart kwijt}\\

\haiku{{\textquoteleft}Sinds gij mij toch niet,,!}{ten antwoord kunt staan ga ik}{u laten Elbert}\\

\haiku{{\textquoteright} {\textquoteleft}Ik ben als altijd ',!}{aan de zijde vant recht}{heer van Barneveld}\\

\haiku{Maar mijns bedunkens.}{is het nu bovenal een}{tijd van handelen}\\

\haiku{{\textquoteleft}maar het is soms van.....}{ongemeen groot nut en vrij}{minder gevaarlijk}\\

\haiku{{\textquoteright} {\textquoteleft}Voor de uitersten,{\textquoteright}.}{hoede ons de Alwijsheid}{sprak Leoninus}\\

\haiku{want schoon ik niets kwaad,....}{van hem wete veeleer niets}{dan goeds van hem zie}\\

\haiku{nu we niet meer in,.}{den Spaanschen tijd zijn kunnen}{we de Kenau's missen}\\

\haiku{{\textquoteright} zei Gideon, {\textquoteleft}gij,.}{zelf moogt oordeelen hoe u}{dit passen zoude}\\

\haiku{heb voor 't minst het.}{geduld te luisteren en}{u te overtuigen}\\

\haiku{maar dit bewijs van,.}{aandacht scheen haar genoeg te}{zijn om voort te gaan}\\

\haiku{zij scheen zelfs het leed,,;}{waarover zij geklaagd had niet}{meer te gevoelen}\\

\haiku{maar haar trotsch gemoed.}{bleek onvatbaar voor zulke}{gewaarwordingen}\\

\haiku{hij wendde den blik.}{af van de Prinses en sprak}{bijna gebiedend}\\

\haiku{{\textquoteright} {\textquoteleft}Blijf dan, maar onder,.}{conditie gij zult mij uwe}{kwelling meedeelen}\\

\haiku{ik kan niet anders,,,.}{dan blijven waar gij zijt dan}{volgen waar gij gaat}\\

\haiku{Daartoe was echter,.}{een voorwendsel noodig en dat}{voorwendsel bestond}\\

\haiku{want men moet billijk,.}{zijn op de eerste achtte}{Modet niet het meest}\\

\haiku{Ik ben geen vriend van,{\textquoteright}.}{Nieuwenaar hernam Norrits}{barsch en spijtig}\\

\haiku{{\textquoteright} {\textquoteleft}Onder de leiding,!}{van den goeden Keurvorst van}{Keulen mevrouwe}\\

\haiku{.... ik kan toch voor een....}{dienaar der Kerke kwalijk}{mijne deur sluiten}\\

\haiku{en of hij van hen, {\textquotedblleft}.}{was diede goede cause}{wilden vorderen}\\

\haiku{Zij weet zelve niet,,.}{hoezeer zij daarmede waar}{zegt dat zij u schaadt}\\

\haiku{Aan uw vromen vriend,.}{heb ik beloofd het leven}{getroost te dragen}\\

\section{F.R. Boschvogel}

\subsection{Uit: Niet wanhopen Maria Christina}

\haiku{Zijn ogen zitten hard.}{en koel achter de glazen}{van zijn gouden bril}\\

\haiku{die jongens van de.}{chirurgijn hebben te veel}{pit achter de oren}\\

\haiku{Gedurende de.}{vijfde partij gaat het nog}{makkelijker}\\

\haiku{Omdat hij beseft.}{dat hij het tegen hem zal}{moeten afleggen}\\

\haiku{Zijn linnenhandel,.}{schijnt weer op te brengen hij}{ziet op geen gulden}\\

\haiku{Om u is het dat.}{die twee loeders v\'o\'or uw deur}{gevochten hebben}\\

\haiku{En dat hij zijn mes,.}{trok om mij te lijf te gaan}{is zeer menselijk}\\

\haiku{Het eten van Andries.}{kan ze in de hete as}{warm en gaar houden}\\

\haiku{Hij had maar dit kind,.}{en een kind dat met alle}{gaven scheen bedacht}\\

\haiku{Ik ken Andries, en.}{laat jij de afgunstige}{luidjes maar kletsen}\\

\haiku{Hij heeft al een uur,.}{wakker gelegen wachtend}{op de dageraad}\\

\haiku{Ze zei het laatst, maar.}{je kon horen hoe het haar}{ten opperste lag}\\

\haiku{Hij zit met zijn ogen.}{in de wolken als het over}{de kinderen gaat}\\

\haiku{- Zo zal het wel zijn {\textquoteleft}{\textquoteright}.}{in het vaderland van je}{nieuwe gedachten}\\

\haiku{- Zeker, Dumouriez.}{staat met een leger klaar om}{ons te bevrijden}\\

\haiku{Lieven zit met een.}{smalende snuit strak naar een}{graanzak te kijken}\\

\haiku{Als de jongens groot,.}{worden kan een mens zich aan}{alles verwachten}\\

\haiku{En er is ook iets.}{van het dovende avondrood}{dat in haar ogen gloeit}\\

\haiku{Ze zet een stap naar.}{de tafel en grijpt het boek}{uit Lievens handen}\\

\haiku{Zij is al buiten,,.}{naar de stal of de schuur of}{naar de werkwinkel}\\

\haiku{Met vier glanzende, '.}{paarden wie jet vuur uit}{de lijven kunt slaan}\\

\haiku{Je krijgt kippenvel.}{als je er aan denkt wat er}{in Frankrijk gebeurt}\\

\haiku{Het felle gestraal.}{van haar vurige blik werkt}{op zijn mondhoeken}\\

\haiku{Kom, kom, hoe durft ze?}{de zaken dadelijk zulk}{aangezicht geven}\\

\haiku{Misschien vinden ze.}{binnenkort wel een eigen}{huis op Aartrijke}\\

\haiku{Herinneringen.}{die haar anders een kelk vol}{alsem en edik zijn}\\

\haiku{- Dat kunnen praatjes,,.}{zijn praatjes uit verdachte}{clubs bijt Andries af}\\

\haiku{Zijn handen liggen...}{klaar om recht te veren en}{dit verloren brood}\\

\haiku{Maar laat hij verder.}{zijn gezag uitoefenen}{als naar gewoonte}\\

\haiku{Het leven is al,.}{kort genoeg en zij is in}{weelde opgebracht}\\

\haiku{Het is een vrouw, die.}{je kan vertrouwen als een}{eigen  moeder}\\

\haiku{Hij staat op, knielt v\'o\'or.}{de Kruislievenheer en zegt}{een kort avondgebed}\\

\haiku{Maar hij stokt in zijn,.}{pleidooi die welbespraakte}{monsieur l'abb\'e}\\

\haiku{Ze durft niets anders.}{dan even te knikken als hij}{iets tegen haar zegt}\\

\haiku{- terug meedanst in,.}{de ronde moeten ze haar}{een poos meesleuren}\\

\haiku{Moet ik hem soms geen?}{vergiffenis vragen op}{mijn blote knie\"en}\\

\haiku{Op de drempel van.}{de grote vasten gaat zij}{gewoonlijk te biecht}\\

\haiku{Er is niet veel meer.}{nodig om hem helemaal}{te doen wegblijven}\\

\haiku{Die zitten nu een.}{carmagnole te fluiten}{achter de tralies}\\

\haiku{Een jongen, die zijn.}{hoofd liet op hol brengen door}{de slechte boeken}\\

\haiku{In een smalle gang.}{vol stof en zure lucht maakt}{zij haar opwachting}\\

\haiku{- haar beurt gekomen,.}{is voelt ze de slag van haar}{hart tot in haar hoofd}\\

\haiku{Of hij het van de,.}{eerste lettergreep niet beet}{had waarover het ging}\\

\haiku{Eerst eens zien wat er... -}{bekend is over dat lieve}{kind van u. Lieven}\\

\haiku{Ze kijkt op naar het,.}{hoge venster waarin de}{halletoren staat}\\

\haiku{En ontdekt meteen.}{tal van zaakjes die moeten}{opgeknapt worden}\\

\haiku{Kom, ze zullen de.}{kommode een einde van}{de muur wegslepen}\\

\haiku{- Het was niet moeilijk,.}{om bezigheid voor hem te}{vinden kapelaan}\\

\haiku{En het huis van een.}{kunstschilder tonen ze haar}{in een stille straat}\\

\haiku{Maarten legt het boek.}{v\'o\'or haar op tafel en keert}{traag de bladen om}\\

\haiku{Ze keert gejaagd de,,}{blaadjes van een album zoekt}{naar een handwerkje}\\

\haiku{- De Pruisen en de.}{Keizerlijken hebben hun}{de oorlog verklaard}\\

\haiku{Dit dorpse bier zal.}{de hoofdman van Aartrijke}{wel zelf betalen}\\

\haiku{De lach is het van,.}{de commissaire en van}{zijn secr\'etaire}\\

\haiku{Nu eerst, omdat het.}{de eerste keer is dat je}{moeder nodig hebt}\\

\haiku{En dan ja tegen.}{het ene en neen tegen het}{andere meisje}\\

\haiku{Ze  voelt dat hij,.}{niet stevig aanpakt dat hij}{in het duister tast}\\

\haiku{Nicht Isabella houdt.}{een catechismus- en}{letterschooltje open}\\

\haiku{Maar de twee paarden.}{van seigneur Jean heb ik niet}{teruggekregen}\\

\haiku{Hoe ver de nieuwe.}{gedachten over de wereld}{zijn doorgedrongen}\\

\haiku{- Dus loop je weer met,?}{dat vreemde volk mee tegen}{ons en allen in}\\

\haiku{Here Jezus, als.}{het monsieur l'abb\'e eens}{moet ter ore komen}\\

\haiku{Na de vespers van.}{tweede Paasdag gaat Anna}{naar het molenhuis}\\

\haiku{- Ge neemt de woorden,.}{van mijn lippen ma m\`ere}{sup\'erieure}\\

\haiku{Deze jaren zijn,.}{voorbij maar het jaagt een schoon}{geluk door haar ziel}\\

\haiku{Ze wil de jente.}{meiboom niet zien wuiven op}{de halletoren}\\

\haiku{De dood die dit land,:}{alom bespringt met vier vijf}{gesels tegelijk}\\

\haiku{De heren der kerk.}{kunnen lelijke dingen}{zeggen over haar zoon}\\

\haiku{Het is of hij op,,.}{haar woorden wacht op die vraag}{die haar achtervolgt}\\

\haiku{Een nieuwe gril, een.}{korte pennetrek en het}{systeem stort ineen}\\

\haiku{Hij heeft iets van de,.}{oude hoofdman over zich want}{ze vragen hem raad}\\

\haiku{Heel het dorp spreekt er,.}{schande over dat ze Andries}{hebben meegesleept}\\

\haiku{Ze is er weinig.}{op gesteld Andries tussen}{vier ogen te spreken}\\

\haiku{Andries zit grauw en.}{stom naar de toppen van zijn}{laarzen te kijken}\\

\haiku{Zeg haar hoezeer ik.}{verlang om bij  haar te}{komen inwonen}\\

\haiku{Maarten ziet hoe haar,.}{onderlip spant hoe ze vecht}{met haar ontroering}\\

\haiku{- Een paasweertje voor,!}{de Laetare-zondag}{Maarten Engelbrecht}\\

\haiku{Tot hiertoe was hem.}{alles zo onverschillig}{als de straatstenen}\\

\haiku{Hij zou die vreemde,.}{pastoor kunnen hangen zo}{kwaad is hij op hem}\\

\haiku{V\'o\'or de pastorij,.}{kruist hij monsieur l'abb\'e}{die naar de kerk gaat}\\

\haiku{Monsieur l'abb\'e.}{zit nog afwezig op het}{slagveld te kijken}\\

\haiku{Ze zullen zoveel.}{borreltjes drinken als de}{pastoors glazen wijn}\\

\haiku{Daar zijn gelukkig -,,!}{nog anderen kom ik schenk}{nog eens in Lowie}\\

\haiku{Ze zeggen dat het,.}{een Walekop is maar dat is}{een barre leugen}\\

\haiku{Zij kan alleen met.}{monsieur l'abb\'e praten}{in het tuinhuisje}\\

\haiku{De boerin, die een,.}{invroom vrouwmens is kan het}{niet langer dulden}\\

\haiku{Deze gewiekste.}{mannen halen het sluwste}{wild uit zijn schuilhoek}\\

\haiku{PASTOOR BOUCKAERT ZIT.}{NU AL DAGEN lang alleen}{in zijn pastorij}\\

\haiku{De pastoor knikt, doet,}{zoiets als wuiven met zijn}{moede moede hand.}\\

\haiku{De kerk gonst als een.}{bijenkorf van het gezang}{en het orgelspel}\\

\haiku{De kapelaan in.}{zijn boerenpak wipt op het}{paard en rijdt vooruit}\\

\haiku{Maar die knecht, mijne - -!}{mens de rest wordt gefluisterd}{dat was de pastoor}\\

\haiku{En te Torhout zit.}{er een verborgen in een}{hokje van de schouw}\\

\haiku{En de stille mis.}{bij het vroege schemeruur}{in de opkamer}\\

\haiku{Hij gaat met Adriaan '.}{naar Leopold Lievens op}{t Kanunnikse}\\

\haiku{Hij loopt de wolf in,.}{de muil want hij weet niets van}{die gendarmen af}\\

\haiku{Toen ze rakkerden.}{op de molenwal kon hij}{deze trap niet op}\\

\haiku{een onvolkomen,,.}{jongen te week en te bleek}{te meisjesachtig}\\

\haiku{Hij gaat de weg, die.}{honderden priesters gaan in}{dit verdrukte land}\\

\haiku{Ook zij vluchten de,.}{winterse kouters over de}{hemel weet waarheen}\\

\haiku{We hebben hem met.}{een vriendelijk schotje naar}{beneden gehaald}\\

\haiku{Want zij is een vrouw.}{die ze met hun laatste louis}{zouden betalen}\\

\haiku{Luister, als je 't, '.}{meent met je gevoelens moet}{jet bewijzen}\\

\haiku{Een der vrolijksten.}{onder hen is luitenant}{de la Touraine}\\

\haiku{Zij voelt hoe ze v\'o\'or.}{haar in aanbidding liggen}{nedergebogen}\\

\haiku{Ze heeft niet veel tijd.}{om zich zorgen te maken}{over Paul-Marie}\\

\haiku{Liefst van al zou ze,.}{haar op staande voet de deur}{wijzen hard en kort}\\

\haiku{Je kan het zien hoe.}{ze dit kind aan anderen}{heeft overgelaten}\\

\haiku{- De andere is,.}{niet meer teruggekeerd de}{la Touraine wel}\\

\haiku{- Het slechte wens ik,,.}{niet Maria-Christina}{alleen het goede}\\

\haiku{Een mens is er, aan.}{wie ik hen onbevangen}{durf toevertrouwen}\\

\haiku{Vader laat je heer,.}{en meester moeder is in}{alles steengerust}\\

\haiku{De nieuwjaarders, die,:}{Maarten v\'o\'or het raam zien staan}{knikken en wuiven}\\

\haiku{De hand van joffer.}{Serafina is vaardig}{in het briefschrijven}\\

\haiku{- Dood of levend, de,.}{\'emigr\'ee moeten we hebben}{zegt de luitenant}\\

\haiku{De grote piekhond,,.}{die onder de tafel ligt}{te soezen springt recht}\\

\haiku{El\'eonore, die zich een,.}{ogenblik schuil hield achter de}{deur is al buiten}\\

\haiku{En nog kruiven de '.}{krollen hoog en weerbarstig}{haar bovent hoofd}\\

\haiku{Als de mensen het,:}{horen zullen ze het hoofd}{schudden en zuchten}\\

\haiku{Hij ziet hoe Anna.}{nog eenmaal het kind in haar}{armen neemt en zoent}\\

\subsection{Uit: Zandstuivers}

\haiku{- 't Wordt hier stille,,.}{zeiden de menschen Steven}{is er ook van door}\\

\haiku{Ze gingen eten in.}{een restaurant en huurden}{een appartement}\\

\haiku{Ik zal  den zak.}{over mijn schoere smijten en}{naar de beeten gaan}\\

\haiku{Hij stond recht, en ging.}{traag en zwaar terug van waar}{hij gekomen was}\\

\haiku{Op 't hof zaten.}{ze te vergeefs naar Steven}{en Door te wachten}\\

\haiku{Over een week moest het.}{toegewezen worden aan}{een nieuwen pachter}\\

\haiku{Wegens de lage.}{huishuur heeft zoo'n dischhuisje}{altijd veel aantrek}\\

\haiku{De pastoor was geen.}{klein beetje preutsch met zulke}{parochianen}\\

\haiku{- Zeg op, want ik ben,.}{haastig zei de brouwer een}{beetje gekitteld}\\

\haiku{'s Anderen daags.}{echter ging de brouwer reeds}{naar Steven Dagraad}\\

\haiku{Heel de parochie.}{wachtte op madame van}{den secretaris}\\

\haiku{Ze meenden dat ik,.}{het niet hoorde maar een mensch}{hoort scherp aan dien kant}\\

\haiku{Maar we zullen 't.}{beleven dat hij nog een}{echte sinjeur wordt}\\

\haiku{Blondientje was nog,.}{geen acht jaar oud toen ze ook}{eens zoek geraakt was}\\

\haiku{Hij droeg Blondientje,,,.}{zijn zusje op zijn arm als}{een groote schoone pop}\\

\haiku{Ach, en maar \'e\'en kind,. '}{hebben en er zoovele}{mee tegen komen}\\

\haiku{Met eindeloos veel.}{deernis in dien zucht voor haar}{dochter Blondientje}\\

\haiku{- Och, mijnheer Adriaan,, '.}{maar \'e\'en kind hebben hoe zou}{t anders kunnen}\\

\haiku{- Boerin Sinnaeve,, '.}{zei hij voor de belooning}{moet get niet doen}\\

\haiku{Ze zeggen dat Ons.}{Heere u beloond heeft met}{dat schoon Blondientje}\\

\haiku{'s Namiddags trok.}{ze naar het kapelletje}{op den Koudepuid}\\

\haiku{Het is de eenigste.}{keer dat Adriaantje gezoend}{werd in zijn leven}\\

\haiku{Want ze hebben bij.}{ons al mizerie genoeg}{met de varkens}\\

\haiku{Ik ben er eens tot.}{op den drempel geweest met}{mijn schoolkameraad}\\

\haiku{Ze had het gehoord:}{hoe bot en bietebauwig}{de menschen zeiden}\\

\haiku{Maar een engel uit,,.}{Gods hoogen hemel dat was ze}{zoo goed en zoo schoon}\\

\haiku{Van Ieften was het,.}{een al te vlug dalen een}{sprong in den afgrond}\\

\haiku{Jaarlijksche bolling ',!}{int Boldershof om een}{panne hoofdvleesch}\\

\haiku{Mientje had haar deel.}{gekregen en moest het er}{nu maar mee stellen}\\

\haiku{Mientje was echter.}{de laatste om ook maar \'e\'en}{klachtje te uiten}\\

\haiku{En daarom was het.}{goed en gezellig voor hem}{in den werkwinkel}\\

\haiku{De pastoor gaf haar.}{vlug de absolutie en}{het heilig oliesel}\\

\haiku{Hij kreunde zacht, toen,.}{hij opstond een groote hoop was}{in hem gebroken}\\

\haiku{Over een paar weken,.}{zijt ge hier terug troostte}{kozijn notaris}\\

\haiku{Maar wij wisten niet,.}{dat zij ooit  zoo'n lieven}{naam gedragen had}\\

\haiku{Ik heb in heel mijn.}{bewaarschooljaren nooit meer}{geschrokken als toen}\\

\haiku{De vensterluiken,.}{zijn dicht en het is angstig}{stil om dit huisje}\\

\haiku{Korte djakke j - - -!}{appeltje trappeltje a}{twee beentjes n jan}\\

\haiku{De oudste wijven '.}{vant gehucht hebben ze}{nooit weten bouwen}\\

\haiku{De inspecteur hield}{de kleuters zoo lang bezig}{tot Zuster Marie}\\

\haiku{En ons jongste heeft.}{deze week vier keeren in de}{stuipen gelegen}\\

\haiku{Achter haar krommen.}{rug staat het golvende land}{in rooden zonnegloed}\\

\haiku{De meisjes sliepen.}{in \'e\'en bed en makkerden}{uitmuntend samen}\\

\haiku{s Morgens  was.}{haar programma tot in de}{puntjes opgemaakt}\\

\haiku{Als het zoo ver was,.}{kwam de barones naar de}{Godelieveschool}\\

\haiku{Geen week nadien hield.}{het koetsje weer stil v\'o\'or de}{Godelieveschool}\\

\section{Anthony Bosman}

\subsection{Uit: Witte zwanen zwarte zwanen}

\haiku{De boeren staken.}{bun spade in de grond en}{leunden er zwaar op}\\

\haiku{Trouwens, men kan slechts.}{geluk brengen indien men}{zelf gelukkig is}\\

\haiku{Het was alsof mijn,.}{spelen geen zin meer had nu}{zij was weggegaan}\\

\haiku{- Ach, zei ze, - ik heb.}{vele namen en elke}{naam is een verhaal}\\

\haiku{We gingen een trap,}{op en dan langs een brede}{gang met veel deuren}\\

\haiku{- Maar misschien is het,.}{beter dat je alleen het}{goede van mij ziet}\\

\haiku{Het kan lang duren,,.}{of kort speelman maar eens zul}{je mij verlaten}\\

\haiku{Veel was ik in de,.}{tuin omdat ik dan het dichtst}{bij de landweg was}\\

\section{Nanne Bosma}

\subsection{Uit: De emigrant}

\haiku{{\textquoteright} Onder luid gehoon.}{van de stakende slepers}{verdween de Duitser}\\

\haiku{Ze zei het niet, ze,}{had het nooit gezegd al die}{maanden niet maar diep}\\

\haiku{Pas goed op jezelf,,{\textquoteright}, {\textquoteleft}.}{Abe fluisterde zeook als}{ik er niet meer ben}\\

\haiku{Na deze laatste.}{omhelzing haastte Aukje}{zich zwijgend van boord}\\

\haiku{Hij wilde meteen.}{het raampje openen om uit}{de trein te hangen}\\

\haiku{{\textquoteleft}Houd die jongen toch,,.}{binnen Abe zo meteen valt}{hij nog uit de trein}\\

\haiku{{\textquoteleft}We moeten maar iets,.}{doen de hele dag binnen}{zitten is ook niets}\\

\haiku{Er ratelde een,.}{rijtuig over de brug verder}{gebeurde er niets}\\

\haiku{Arbeider worden.}{was wel de zwaarste slag die}{een boer kon treffen}\\

\haiku{Na het invallen.}{van de dooi mocht Abe bij de}{Atema's thuis komen}\\

\haiku{Hij wist zeker dat.}{het er de zestiende ook}{al gelegen had}\\

\haiku{Met grote letters:}{stond er CANADA en iets}{kleiner daaronder}\\

\haiku{De menigte ging,.}{tevreden uiteen het recht}{had gezegevierd}\\

\haiku{De jongens die het,.}{paard hadden laten lopen}{waren hem gevolgd}\\

\haiku{Beppe warmde voor:}{haar kleinzoons een restje op}{van het middagmaal}\\

\haiku{De hond holde naar.}{buiten en kwam Pier vrolijk}{blaffend tegemoet}\\

\haiku{Met grote stappen.}{klom hij de smalle trap op}{naar het sloependek}\\

\haiku{Maar ja, wat wist je,?}{in die tijd van Canada}{of van Amerika}\\

\haiku{deze twee zien eten,.}{dan zouden ze zeker groen}{uitgeslagen zijn}\\

\haiku{Even later weer veel.}{geschreeuw omdat het bier te}{lauw en te oud was}\\

\haiku{Abe liep een gang door,.}{een trap af en kwam zo op}{het derdeklasdek}\\

\haiku{De mensen op de.}{twee langs varende schepen}{keken en wuifden}\\

\haiku{De eerste keer is, '.}{het wel leuk maar nu heb ik}{t wel bekeken}\\

\haiku{Ook het stampen en.}{slingeren van het schip leek}{erger dan overdag}\\

\haiku{Langzaam zakte hij,.}{weg in een toestand van half}{slapen half waken}\\

\haiku{Canada, Quebec,.}{en boot konden ze elkaar}{niet veel vertellen}\\

\haiku{{\textquoteright} Nu begreep Molly.}{er niets van en ze lachten}{er allebei om}\\

\haiku{{\textquoteleft}I am from Torquay,{\textquoteright}:}{voegde ze er aan toe en}{verduidelijkend}\\

\haiku{Het was mooi weer, soms.}{zaten ze zelfs uit de wind}{aan dek in de zon}\\

\haiku{Ze was wat bleker,}{dan anders maar verder was}{er niets aan de hand.}\\

\haiku{Molly was bleek en,.}{had rood-behuilde ogen}{ze keek Abe niet aan}\\

\haiku{Hij wist niet goed wat.}{hij er mee aan moest en deed}{een stapje opzij}\\

\haiku{{\textquoteright} Abe probeerde haar.}{wat te kalmeren en van}{zich af te houden}\\

\haiku{Molly hield op met.}{huilen en omhelsde hem}{innig en dankbaar}\\

\haiku{Hij probeerde een.}{eend te pakken te krijgen}{aan de waterkant}\\

\haiku{De hond probeerde.}{uit alle macht terug te}{zwemmen naar de kant}\\

\haiku{Voorzichtig glipt hij,.}{uit het bed zonder zijn broer}{wakker te maken}\\

\haiku{{\textquoteright} {\textquoteleft}Da's mooi jongen,{\textquoteright} zei, {\textquoteleft}.}{zeen kijk eens wat ik voor}{je verjaardag heb}\\

\haiku{Aukje trok er nog,.}{wat aan onhandig over de}{bedsteerand}\\

\haiku{Pake stak zijn hoofd.}{om de hoek en hoorde wat}{er aan de hand was}\\

\haiku{Na schooltijd moesten Jan.}{en Pier hun spullen van de}{boerderij halen}\\

\haiku{{\textquoteleft}Ga nu maar, jongen,,{\textquoteright}.}{anders missen jullie de}{trein nog zei Aukje}\\

\haiku{Het was voor het eerst.}{in hun leven dat ze in}{een auto zaten}\\

\haiku{Jopie had altijd,.}{geld op zak en Pier leerde}{hoe hij daar aan kwam}\\

\haiku{Drummondville, 8,}{juli 1913 ~ Lieve vrouw}{en kinderen}\\

\haiku{We praten niet veel,}{meestal zijn we te moe}{om veel te zeggen}\\

\haiku{In Drummondville.}{heb ik een man ontmoet die}{Emile Vassal heet}\\

\haiku{Dit is voorlopig,.}{het einde van mijn verhaal}{ik ga nu wat eten}\\

\haiku{Het schetsboek hield Pier,.}{geheim alleen mem had het}{wel eens mogen zien}\\

\haiku{Na ruim een uur kreeg.}{hij zijn eerste zuivere}{toon uit de viool}\\

\haiku{Pier had de smaak te:}{pakken en hij had nu nog}{maar \'e\'en verlangen}\\

\haiku{Voorlopig hadden.}{ze het te druk om aan de}{toekomst te denken}\\

\haiku{Moe, vuil en grimmig.}{liepen de mannen rond de}{molen en het huis}\\

\haiku{Het bleef mogelijk.}{dat de ramp het gevolg was}{van blikseminslag}\\

\haiku{Ze spanden het beest.}{voor de wagen en reden}{naar Drummondville}\\

\haiku{Daarna gingen ze.}{uiteen om elkaar wellicht}{nooit meer te zien}\\

\haiku{Bij elk station,.}{nam hij een hapje dan deed}{je er langer mee}\\

\haiku{Een brief van Aukje.}{gaf haar de volgende dag}{meer informatie}\\

\haiku{De toevoeging {\textquoteleft}and{\textquoteright},:}{sons die hij kon vertalen}{en de plaatsnamen}\\

\haiku{Daar was haast niet aan:}{te komen en bovendien}{stond er dan nog op}\\

\haiku{{\textquoteright} {\textquoteleft}Het is alweer een,.}{hele tijd geleden nog}{voor het grote feest}\\

\haiku{{\textquoteleft}En ze hebben er,,.}{hele dikke kleden op}{de vloer mem zo zacht}\\

\haiku{P\`ere en m\`ere}{Vassal zijn erg goed voor me.}{P\`ere en m\`ere}\\

\haiku{{\textquoteleft}Is 't anders niet,{\textquoteright}.}{doe de mond maar open man en}{laat maar eens zien}\\

\haiku{Het is een goede,.}{les geweest ik ga niet meer}{zo ver het bos in}\\

\haiku{{\textquoteright} Zijn moeder had hem.}{over het haar gestreken en}{hem gerustgesteld}\\

\haiku{Ze sloot de brief af.}{en schreef met duidelijke}{letters het adres}\\

\haiku{Jan werkt bij pake,,.}{hij wordt groot je zult verbaasd}{zijn als je hem ziet}\\

\haiku{Het overlijden van.}{oude mensen ligt in de}{natuur der dingen}\\

\haiku{Het duurde enige.}{tijd voor hij reageerde}{op de twee brieven}\\

\haiku{Ik ben gewoon weer '.}{in dat kantoor gaan zitten}{ens avonds buiten}\\

\haiku{Op de laatste dag.}{kwam mijn vriend naar het kantoor}{vlak voor sluitingstijd}\\

\haiku{Je zult ook leven,.}{Aukje het komt allemaal}{nog best voor elkaar}\\

\haiku{{\textquoteleft}Dat is niet genoeg,.}{voor de trein je hebt nog maar}{zeven dollar over}\\

\haiku{Abe weigerde naar,.}{een hotel te gaan wegens}{de hoge kosten}\\

\haiku{{\textquoteright} Ze wandelden door,.}{de kleurloze saaie straten}{van de drukke stad}\\

\haiku{Er scheen geen school voor,.}{de kinderen te zijn ze}{waren altijd thuis}\\

\haiku{Aukjes brief deed er.}{meer dan drie weken over om}{hem te bereiken}\\

\haiku{Hij stond voor de deur.}{van het ziekenhuis toen zijn}{vrouw naar buiten kwam}\\

\haiku{Het was op het heetst.}{van de dag dat Abe in het}{rijtuigje stapte}\\

\haiku{Ver beneden, op,.}{het binnenplaatsje slofte}{een oude man rond}\\

\haiku{Het zou zeker een.}{halve dag duren eer de}{boel opgeruimd was}\\

\haiku{Ze gingen op een.}{bank voor het hotel zitten}{en dronken koffie}\\

\haiku{Rond het middaguur.}{stoomden ze de prairie in}{op weg naar Morse}\\

\haiku{De eerste dag op.}{zijn eigen grond was Abe al}{voor dag en dauw op}\\

\haiku{Piepend en krakend.}{hield het karretje stil bij}{het hout en huisraad}\\

\haiku{{\textquoteleft}Goejemorg'n,?}{sam'n kunn'n jullie nog wat}{naberhulp gebruik'n}\\

\haiku{Het eigenlijke.}{ontginningswerk moest wachten}{tot het volgend jaar}\\

\haiku{Molly zag er in.}{een lichtblauwe jurk prachtig}{uit en ze wist het}\\

\haiku{Een al dagenlang.}{durende sneeuwstorm dwong hen}{binnen te blijven}\\

\haiku{Abe niet omdat hij,.}{met een plan rondliep dat hij}{niet kon uitvoeren}\\

\haiku{Ze vroeg hem wel tien.}{maal of hij echt van plan was}{terug te komen}\\

\haiku{{\textquoteright} De vrachtboot was een,,.}{traag oud stoomschip geladen}{met superfosfaat}\\

\haiku{Hij liep achterom,.}{om niet met nieuwsgierigen}{te hoeven praten}\\

\haiku{Hij zag veel oude.}{bekenden en vertelde}{overal zijn verhaal}\\

\haiku{Voor de tweede maal,.}{in zijn leven voer Abe weg}{nu definitief}\\

\haiku{Nee, nee, ik kom niet,,.}{uit Maryland ik heet zo}{ik woon in Detroit}\\

\section{Taecke J. Botke}

\subsection{Uit: Het revier}

\haiku{Misschien kun je je.}{voorstellen hoe feestelijk}{die ervaring was}\\

\haiku{Maar in een hete.}{barak komen de dingen}{anders te liggen}\\

\haiku{Alleen in het brein {\textquoteleft}{\textquoteright}.}{derHerrenmenschen kon zo'n}{geestrijk idee ontstaan}\\

\haiku{Misschien ligt er toch '.}{een loopbaan voor mij int}{geestelijke}\\

\haiku{Deze eiste dat.}{Eddy bij de operatie}{aanwezig zou zijn}\\

\haiku{De opkomende.}{maan verbleekte de hemel}{tot een helder blauw}\\

\haiku{De mannen die het - -;}{leven met hun aandelen}{in de zak hebben}\\

\haiku{De ene snavel is,.}{ondernemend de ander}{veeleer vermoeid}\\

\haiku{Eer de stoet zich in:}{beweging zette werd het}{dagrantsoen verstrekt}\\

\haiku{Wie er al dood was,.}{en wie nog leefde viel niet}{meer uit te maken}\\

\haiku{zo'n man die beslist.}{het leven voor iets hogers}{wilde inzetten}\\

\haiku{hij vreesde ziekten.}{zodat wij vrij waren in}{ons doen en laten}\\

\haiku{Tandartsen waren{\textquoteright}.}{hoogstwaarschijnlijk geen echte}{Akademiker}\\

\haiku{Maar het was ook het.}{enige waarin zeker niet}{voorzien kon worden}\\

\haiku{Majesteit of geen, '.}{majesteit maart moet niet}{besmettelijk zijn}\\

\haiku{Die uit de fabriek,.}{stellen zich op in rij en}{gelid evenals wij}\\

\haiku{Zijn distinctie schiep,.}{afstand maar zijn schoonheid trok}{de blikken tot zich}\\

\haiku{Een boer die kwaad wordt,.}{is een oerfenomeen zo}{iets als een onweer}\\

\haiku{Waar ging dat heen als.}{dergelijke dieven vrij}{rond konden lopen}\\

\haiku{Toen hij echter de,.}{inhoud ontdekte was hij}{geheel verbijsterd}\\

\haiku{Ook de schoorstenen.}{van het crematorium}{tekenden zich af}\\

\haiku{Dit is, zo men wil,.}{een vorm van pech hebben die}{erger had gekund}\\

\haiku{Een gouden voortand.}{verleende aan de grijns een}{naargeestig accent}\\

\haiku{Het bestaan in de.}{quarantainebarakken}{was moeilijk genoeg}\\

\haiku{Iemand gooide de.}{deur open en ons vertrekje}{liep vol met Russen}\\

\haiku{{\textquoteleft}Heb jij misschien een?}{advertentie voor werksters}{in de krant gezet}\\

\haiku{Dat zou de onze,.}{worden met Joop en mij als}{geboortehelpers}\\

\haiku{Zwak nog, maar vrijwel.}{genezen had ik mijn bed}{kunnen verlaten}\\

\haiku{een aardig gebaar.}{dat echter geen einde maakt}{aan spuitend water}\\

\section{Ina Boudier-Bakker}

\subsection{Uit: Het beloofde land}

\haiku{dat hun werk, vooral,.}{dat van Adam al het werk in}{den omtrek overtrof}\\

\haiku{Maar den tweeden dag,,.}{na Jelle's dood gebeurde voor}{Eli iets wonderlijks}\\

\haiku{Het was alles in ',...}{t begin geweest als een}{heerlijk nieuw leven}\\

\haiku{Als de avond stil was;}{en schoon kwam het volk bijeen}{onder de linden}\\

\haiku{als de sneeuwstormen,...}{woeden en de ijzige}{noordoostenwind giert}\\

\haiku{Nu komt hij, en rust,...}{in het oude kamertje}{op den ouden stoel}\\

\haiku{Van kind af samen,;}{opgevoed waren zij sterk}{aan elkaar gehecht}\\

\haiku{Toen Eli opschrikte,.}{uit zijn gedachten zag hij}{plotseling Hester}\\

\haiku{Waarom moest altijd?}{elk gevoel weer verdrongen}{worden door een nieuw}\\

\haiku{Eli hield van haar en,,.}{van Maarten om de groote rust}{die van hen uitging}\\

\haiku{En niemand wist het,:}{geheim dat hij stil en trouw}{in zijn hart besloot}\\

\haiku{* * * ~ Moeder - ik weet,,.}{nu er is iets sterker dan}{al het andere}\\

\haiku{Wat mij geen rust liet,.}{eer ik zou voelen uw hand}{op mijn moede oogen}\\

\haiku{er was een onrust,;}{in Eli die hem dreef alleen}{te zijn met Hester}\\

\haiku{Maar je bent hem niet...{\textquoteright} {\textquoteleft} - '...{\textquoteright} {\textquoteleft}}{kwijtJawel hij hindert me}{niet meer int werk}\\

\haiku{Ik heb geleefd, de - -.}{dagen en nachten door met}{dat \'e\'ene Mijn Haat}\\

\haiku{Als zij klaagden over,,}{te veel arbeid dan dacht hij}{dat de arbeid dien}\\

\haiku{... langzaam versmolt hun.}{wrok voor een vaag kinderlijk}{gevoel van schaamte}\\

\haiku{Hij heeft gelijk - hij ',{\textquoteright},.}{w\'e\'ett zei de oude Brandt}{met stralende oogen}\\

\haiku{zijn wrok tegen het - -:}{volk zijn ellende om het}{werk bl\'e\'ef alleen dit}\\

\haiku{Waarom was deze!}{vreeselijke dag in zijn}{leven gekomen}\\

\haiku{ernstig, haar blijheid,}{geschokt maar rustig bereid}{alles te dragen}\\

\haiku{Een enkelen keer,;}{zeiden de anderen iets}{dat zijn aandacht trok}\\

\haiku{zijn woorden waren...}{een openbaring  geweest}{van iets vreeselijks}\\

\haiku{De smartkreet van het,.}{moederschepsel dat zich haar}{jongen ziet ontroofd}\\

\haiku{Het is een kreet van,.}{kracht een hartstochtelijke}{stem van strijd en smart}\\

\haiku{Hij zag het, zooals hij.}{nooit te voren een werk had}{gezien en doorvoeld}\\

\haiku{De ander zag hem,.}{schuin aan met een enkelen}{snellen oogopslag}\\

\haiku{Een poos bleef hij haar -.}{zwijgend aanzien toen schudde}{hij langzaam het hoofd}\\

\haiku{{\textquoteleft}Dat begrijp jij niet -, -.}{en niemand daarom kon ik}{er niet van spreken}\\

\haiku{Hester en niemand,.}{vermoedde dit hij verborg}{het nog volkomen}\\

\haiku{Een paar keer werd hij.}{wakker en herinnerde}{zich Berends woorden}\\

\haiku{Hij wist nu wel dat.}{hij bezweken was in den}{jaren-langen strijd}\\

\haiku{{\textquoteleft}Je hoort 't, jongens,...}{Eli geeft zijn woord dan moeten}{we dat vertrouwen}\\

\haiku{Z\`elfs de angst, die de,.}{vorige weken gevuld}{had was er niet meer}\\

\haiku{{\textquoteleft}Als Hester 't niet, ', -}{weet mo\`et zet nu weten}{waarom verzwijgen}\\

\haiku{toen niemand kwam, stak ',.}{hij een sleutel int slot}{en opende de deur}\\

\haiku{een poos zag hij hem,,,;}{aan zooals hij lag zijn hoofd op}{zijn arm zwaar snurkend}\\

\haiku{Ze dacht plotseling,.}{terug hoe ze vroeger zich}{Eli voorgesteld had}\\

\haiku{{\textquoteright} {\textquoteleft}Oude man - zij ook,.}{zullen ondergaan in den}{blanken dooden vrede}\\

\haiku{Maar de brieven van.}{Foeko waren nog altijd}{niet gekomen}\\

\haiku{zal 't me later,.}{spijten ik heb om u nog}{een poos doorgezet}\\

\haiku{Het was voor 't eerst '.}{weer na dien dag dat Elis}{avonds gekomen was}\\

\haiku{Hij zag naar Adam, die,.}{daar rustig zei de harde}{heldere woorden}\\

\haiku{Het kleine lampje...}{knetterde met steeds lager}{brandend vlammetje}\\

\haiku{Het was donker, lang - '.}{donkert was moeielijk om}{hier weer te komen}\\

\haiku{Daar zijn de sterken,,.}{met vasten mond het gelaat}{doorploegd van veel leed}\\

\haiku{{\textquoteright} Eli rilde in den,.}{kouden wind een scherpe hoest}{doorschokte zijn borst}\\

\haiku{er was iets in haar,.}{zitten zoo dat hem even aan}{Hester deed denken}\\

\haiku{als u me niet hadt,...{\textquoteright}}{kunnen helpen had ik naar}{vreemden moeten gaan}\\

\haiku{En hartelijk, met,.}{hun schaarsche woorden zeiden}{ze hem te blijven}\\

\haiku{{\textquoteleft}Zie je wel - 't is ', - '...{\textquoteright}}{goed datk gegaan ben nou}{zouk te ziek zijn}\\

\haiku{het gezicht in zijn,.}{handen verborgen weende}{hij stil over Eli Bag}\\

\subsection{Uit: Een dorre plant}

\haiku{Maar daarna begon.}{het als een zacht nieuw geluk}{in hem te leven}\\

\haiku{{\textquoteright} en kreeg Willem een,.}{duw of een klap dat hij haar}{met rust moest laten}\\

\haiku{en 't liet ook niet.}{als bij Bertus een goede}{herinnering na}\\

\haiku{Telkens, als ze naar,.}{het jonge paar keek kreeg ze}{tranen in de oogen}\\

\haiku{Hij was ieder vrij,.}{oogenblik bij haar hoewel}{ze hem niet kende}\\

\haiku{Wanneer Bertus van,.}{een reis thuiskwam  fleurde}{er altijd iets op}\\

\haiku{{\textquoteright} Toen Johanna dit,:}{aan Jonas overbracht zei de}{oude man norsch}\\

\haiku{D\`at vader toch zoo;}{weinig aardigheid had in}{de kleinkinderen}\\

\haiku{{\textquoteleft}Jij moet maar met ze,.}{naar het park gaan hoor als ze}{niet thuis kunnen zijn}\\

\haiku{{\textquoteleft}Laat ze maar liever ',.}{naart park gaan daar is meer}{ruimte dan bij mij}\\

\haiku{De periode,,.}{toen hij aan niets van vroeger}{ooit dacht was voorbij}\\

\haiku{Hij begon ook te,;}{merken dat de menschen op}{hem gingen letten}\\

\haiku{Hij werd alleen kwaad,.}{als zij hem beletten wou}{buiten te zitten}\\

\haiku{{\textquoteleft}'t Lijkt geen zier op -.}{Verbruggen denk je dat ik}{die niet zou kennen}\\

\haiku{En d\`at zou ze zich,!}{laten ontnemen zonder}{hem was zij ook niets}\\

\haiku{{\textquoteleft}Zou hij gaan spreken, ' '!}{waarachtigt hadm dan}{t\'och wel aangedaan}\\

\haiku{Toen hoorde hij ook.}{Johanna slaperig moe}{naar boven sloffen}\\

\haiku{En hij wist wel, hij '.}{zou ze int voorjaar niet}{meer zien opbloeien}\\

\subsection{Uit: Kinderen}

\haiku{nee, niet naar ma nou,{\textquoteright},.}{h\'e\'elemaal naar boven naar zijn}{eigen kamertje}\\

\haiku{{\textquoteright} Vader liep met hem ',.}{door naart eind van de gang}{waar een juffrouw stond}\\

\haiku{Jip keek verlegen,,,}{op zijn lei toen moedvattend}{stak hij zijn vinger}\\

\haiku{{\textquoteright} Weifelend dwaalde.}{zijn vochtig vingertje uit}{zijn mond over zijn lei}\\

\haiku{{\textquoteleft}Kom 'r maar tusschen,{\textquoteright}, {\textquoteleft},,!}{zei hijhier kan je wel staan}{maar goed vangen hoor}\\

\haiku{{\textquoteright} Geregeld ging de,,.}{bal rond tot ook Miel er uit}{viel en toen ook Ru}\\

\haiku{{\textquoteright} Jip knikte alleen,,;}{maar uiterst voldaan zonder}{eenige jaloezie}\\

\haiku{{\textquoteleft}Kom kinderen, twee!}{aan twee in de rij gaan staan}{en dan naar binnen}\\

\haiku{In den kring, op gang,,:}{nu liepen de kinderen}{zingend schel-valsch}\\

\haiku{{\textquoteleft}'n Ander gegooid, ' '.}{in de gangk zalm maar}{weer laten meedoen}\\

\haiku{{\textquoteleft}Denk erom, anders ';}{laatk je in een heele}{week niet meespelen}\\

\haiku{{\textquoteright} {\textquoteleft}Laten we alleen...{\textquoteright}}{maar es effetjes likken}{aan de achterkant}\\

\haiku{Ze likten, een voor, '. '}{een veegdent gauw met hun}{boezelaar weer droog}\\

\haiku{Roos, met \'even-spottend,.}{neergetrokken mondje keek}{in het kacheltje}\\

\haiku{dan heb je ook geen - '....}{mazelen geh\`ad je weet}{niet eens watt is}\\

\haiku{Vooruit nou - h\`e nee,,;}{nou moet je d\`oen wat ik zeg}{ik ben de mevrouw}\\

\haiku{{\textquoteright} {\textquoteleft}Nou, toen wel honderd, -.}{jaar later toen toen vonden}{ze z'n geraamte}\\

\haiku{{\textquoteleft}'t Smaakte w\`el \`erg,{\textquoteright} - {\textquoteleft}.}{leelijk zei Aleidaze heeft}{er niks an gehad}\\

\haiku{Maar ziet u, ik loop, '.}{een beetje hard wantt mocht}{es gaan regenen}\\

\haiku{voor zij kwam, vlak na,.}{moeders dood waren Riek en}{hij veel meer samen}\\

\haiku{{\textquoteleft}och nee, ik vergis,....}{me ik ga natuurlijk niet}{uit de stad morgen}\\

\haiku{Even voorbij den hoek.}{van den Oosterweg zag hij}{de bloemisterij}\\

\haiku{Hij stond doodstil, keek,}{hardnekkig den anderen}{kant uit met geweld}\\

\haiku{Maar later, toen moes - '.}{w\`eg was to\`en wast nog ve\`el}{na\`arder geworden}\\

\haiku{Nu haar mantel en, -,....}{hoed hingen keek ze even snel}{rond nee geen juffrouw}\\

\haiku{{\textquoteleft}W\'eten jouw ouders,?}{dat jij na vieren grachtjes}{omloopt met jongens}\\

\haiku{Ze had natuurlijk,.}{b\`est gezien dat ze altijd}{met dezelfde liep}\\

\haiku{- Die gemeene Loos - - ' -}{om zoo te spionneeren maar}{ze liett t\`och niet}\\

\haiku{dat mo\`est nu eenmaal -!}{hoe konden sommige dat}{toch prettig vinden}\\

\haiku{ik heb in geen twee -.}{keer een beurt gehad ik mo\`et}{er een krijgen nou}\\

\haiku{{\textquoteright} zei hij, met een blij,.}{even opglanzen in zijn oogen}{dankbaar voor den lof}\\

\haiku{{\textquoteleft}hij zou ze aan Frits,,....}{geve die vissies hij zou}{d'r wel blij mee zijn}\\

\haiku{Een poos was er niets,.}{dan het gulzig eten en het}{klikken der vorken}\\

\haiku{Hi\`er houe, hi\`er,,!}{zeg ik stommerik k\`a je}{uit je ooge niet zien}\\

\haiku{{\textquoteright} wees Tom. {\textquoteleft}Nee, ma houdt, ',.}{er niet van datr wat op}{zit hij moet glad zijn}\\

\haiku{{\textquoteleft}Zalle toch wel niet,'.}{allemaal zoo duur zijn la}{we d'r maar ingaan}\\

\haiku{Tom merkte het, wees.}{gauw stiekem opzij in de}{mand en dan naar ha\`ar}\\

\haiku{Nu is de mand leeg,{\textquoteright},.}{zei ma alsof ze dat}{heel natuurlijk vond}\\

\haiku{{\textquoteleft}Gelukkig, dat je,!}{er nu een gezond voor in}{de plaats hebt Tommie}\\

\subsection{Uit: De klop op de deur}

\haiku{De derde gast was.}{een toast begonnen op den}{gastheer en diens vrouw}\\

\haiku{Hij moest                     stilstaan.}{en midden op straat zich een}{zoen laten geven}\\

\haiku{De lange vader.}{en het kleine kind liepen}{er stil tusschendoor}\\

\haiku{Verslonden stond ze,.}{te kijken naar den poedel}{die door hoepels sprong}\\

\haiku{Nu ja, ze wist het.}{wel dat zij nu eenmaal mooi}{gevonden                     werd}\\

\haiku{Maar moeder lachte,.}{en daar was al een heele}{gerustheid in}\\

\haiku{{\textquoteright} {\textquoteleft}Maar 't is ook niet -.}{de aard van een vrouw dat ze}{zaken doet zooals ik}\\

\haiku{Over Het Water dreef.}{de wind de hooge tonen aan}{van de Boomklok}\\

\haiku{{\textquoteleft}Zeg Marie, toe geef?}{me nu meteen even dat geld}{van de handschoenen}\\

\haiku{Pas toen ze thuis haar,}{donkere trap weer opging}{bedacht ze kregel}\\

\haiku{De handschoenen had,.}{zij weggeborgen ze niet}{durvend toonen nog}\\

\haiku{{\textquoteright} En hij dacht aan de.}{nijpende zorgen in}{hun kleinen winkel}\\

\haiku{Bij het huis van Dr.,;}{Goldeweijn zei Leentje hem}{dat mevrouw uit was}\\

\haiku{En hier op eenmaal.}{stond Karel de Roos in een}{andere wereld}\\

\haiku{Ik had een briefje -,....}{van thuis Leentje zei ik mocht}{wel naar boven gaan}\\

\haiku{Vanavond moet ik ook,.}{pianospelen daarom}{studeer ik nog}\\

\haiku{{\textquoteleft}Vijf erwtjes zaten,{\textquoteright}.}{in een peul                    heelemaal}{van buiten kende}\\

\haiku{{\textquoteright} zei ze wat snibbig,.}{dat verloren ging in haar}{goedigen                     lach}\\

\haiku{Ze werken bij mij,.}{zestig uur in de week}{of tien uur per dag}\\

\haiku{haar goed moederlijk.}{gezicht keek van Ann\`etje}{naar Annebet}\\

\haiku{Och - zij had erbij.}{gezeten en gedacht dat}{zij het niet erg vond}\\

\haiku{Maar Fransje, als de,.}{pijn                     haar even losliet kon}{zelfs hierom lachen}\\

\haiku{'t Is toch een                          -!}{beestenziekte wat zou een}{mensch daar van krijgen}\\

\haiku{hooger loon kunnen.}{eischen door anderen}{worden vervangen}\\

\haiku{En zij stroomden de,.}{lokalen uit in den avond}{een schamele troep}\\

\haiku{Slecht betaalde                     ....}{pianolessen gaven}{m\`et vreemde talen}\\

\haiku{Ann\`etje hoorde.}{en begreep ook niet alle}{bizonderheden}\\

\haiku{{\textquoteright} Zij werd z\'o\'o bleek als.}{ze niet geweest was toen ze}{haar vonnis hoorde}\\

\haiku{hoe anders zijn de,....{\textquoteright}}{Fran\c{c}aises en hoe slecht}{ook kleeden ze zich}\\

\haiku{lachen om alles,.}{wat ze niet begrijpen wat}{ze z\`elf niet weten}\\

\haiku{Waarom - omdat ik....}{de   dingen niet weet die}{ik weten moest}\\

\haiku{neen u begrijpt het,:}{verkeerd anders zoudt u het}{z\'o\'o niet zeggen}\\

\haiku{s avonds er een paar -.}{boekhouderijtjes bij dat}{helpt                     allemaal}\\

\haiku{Alles deelden zij,.}{samen maar die                     dingen}{vertelde hij niet}\\

\haiku{{\textquoteleft}Kind, mijn kleintje, als,.}{ik niet terug kom weet je}{dat ik het goed vind}\\

\haiku{de bollen bloeien,{\textquoteright}.}{zei Fransje en snoof den}{lentegeur                     in}\\

\haiku{Handen, die rustig.}{en lief de dingen om}{je heen zouden doen}\\

\haiku{- dat is me te                     , -.}{opstandig te wild zoo raar}{van gedachten soms}\\

\haiku{{\textquoteright} {\textquoteleft}Is dat nu iets voor,...}{onzen jongen die ieder}{meisje krijgen kan}\\

\haiku{Vanavond had zij}{zeker gedacht dat zij niet}{hield van Frederik}\\

\haiku{een                     stortzee had}{hem in den laatsten zwaren}{storm in het Kanaal}\\

\haiku{{\textquoteleft}En Vondel krijgt nu.}{zijn standbeeld in het Rij-}{en                     Wandelpark}\\

\haiku{Hij heeft geen moment.}{aandacht ook gehad voor een}{kunst van Rembrandt}\\

\haiku{{\textquoteleft}Dat was Alberdingk -.}{Thym die daar met Van Lennep}{stond en da\`ar Pierson}\\

\haiku{Een lang stil leven.}{op den achtergrond viel}{in deze uren weg}\\

\haiku{{\textquoteright} {\textquoteleft}Aan liefde heeft het,{\textquoteright}.}{ons nooit ontbroken zei de}{domineesche hoog}\\

\haiku{Eveline is een,.}{beest van liefdeloosheid dat}{weet je nu eenmaal}\\

\haiku{{\textquoteleft}Alles goed,{\textquoteright} dacht ze,.}{in een snik van verlichting}{vloog hem tegemoet}\\

\haiku{{\textquoteleft}Moeder, hoort u eens,....?}{was u niet bang toen u met}{vader trouwen moest}\\

\haiku{Vooral Am\'elie van.}{Dugten kon haar moederlijk}{warm                     omhelzen}\\

\haiku{Ze hadden er geen,.}{benul van hier wat dat}{alles beteekende}\\

\haiku{Haar kinderjaren -.}{die goed                     geweest waren}{zij was een mooi kind}\\

\haiku{Veel minder goed ging.}{het tusschen Ann\`etje en}{Line Bergema}\\

\haiku{Ik moest den heelen ':}{avond denken aant woord van}{Gavarini}\\

\haiku{Dokter Bergema,.}{was                     er geweest die vond}{geen direct gevaar}\\

\haiku{Hij leerde uit zijn {\textquoteleft}{\textquoteright} - {\textquoteleft}.}{hoofd HeinesDie Weber}{Das Harfenm\"adchen}\\

\haiku{En d\`at alles bracht?}{Duitschland al in                     de}{veertiger jaren}\\

\haiku{Karel sprak er niet,.}{van dat hij Goldeweijn voor}{het raam had gezien}\\

\haiku{{\textquoteleft}Guerre \`a la,{\textquoteright},:}{Prusse dat hoor je overal}{vooral het slot}\\

\haiku{Nog v\'o\'or den winter.}{dreigde al armoede}{en werkeloosheid}\\

\haiku{De groote werkgevers,,,.}{de fabrikanten bezorgd}{zetten zich                     schrap}\\

\haiku{de vernietiging,.}{die aanschrijdt onweerhoudbaar}{door den                     Elzas}\\

\haiku{Naast hem zijn zoon, de.}{tengere                     knaap die nooit}{Keizer worden zal}\\

\haiku{Voor de stad dringen,.}{hen de Pruisen op dringen}{hen binnen de stad}\\

\haiku{Ja, als je trouwde,,.}{d\`an kwam je eruit dat was}{de                     eenige weg}\\

\haiku{Een vrouw is in den.}{avond de achterdeur van het}{paleis uitgevlucht}\\

\haiku{Pannen en schoorsteenen -.}{vliegen in de Keizersgracht}{ligt een boom geploft}\\

\haiku{In het paleis van.}{Versailles zetten zich de}{Duitschers aan tafel}\\

\haiku{ik kan tegen jou.}{wel eens                     praten over de}{dingen die ik denk}\\

\haiku{Maar nu - nu worden!}{de kinderen bedreigd die}{ze                     overhielden}\\

\haiku{Na een tweedaagschen....}{slag bij Belfort zijn ze}{teruggeworpen}\\

\haiku{Ze hokken in hun,.}{vuilste nauwe straatjes hun}{slechte woningen}\\

\haiku{En zij                     zaten,.}{weer als vroeger ieder aan}{een kant van het raam}\\

\haiku{Haar beide oudste.}{dochters waren                     eveneens}{in Indi\"e getrouwd}\\

\haiku{Als ik je vertel....}{dat in Fernande                     de}{dames dansen met}\\

\haiku{Maar ze had eerder,.}{haar tong afgebeten dan}{zelfs                     maar gevischt}\\

\haiku{{\textquoteright} ~ De volgende.}{dagen lachten Frederiks}{blauwe oogen schelmsch}\\

\haiku{want de vrouw, die                     .}{in haar eigen onderhoud}{kan voorzien is vrij}\\

\haiku{Jaren woonde zij -,,.}{hier een stille oude}{eenzame juffrouw}\\

\haiku{Da\`ar zullen we, als ',.}{t God blieft                     ons lieve}{land voor bewaren}\\

\haiku{En iederen                     .}{dag wachtte zij op wat hij}{nog niet gezegd had}\\

\haiku{En                     als zij bij -!}{haar moeder kwam hoe k\`on zij}{er over beginnen}\\

\haiku{{\textquoteleft}Ik ben bang tante,.}{dat het u veel zal kosten}{uit het huis te gaan}\\

\haiku{Hij wou haar geld                     ,.}{beheeren zorgen dat zij}{niet zooveel uitgaf}\\

\haiku{Hij kon in al wat,.}{het nieuwe huis betrof niet}{tegen haar op}\\

\haiku{Voelde zich                     ruw,.}{gesleurd in harde armen}{bezeerd en gekneusd}\\

\haiku{Op de bisbilles,.}{van zijn zusters onderling}{ging hij nooit in}\\

\haiku{{\textquoteright} {\textquoteleft}'t Is mijn huis niet,{\textquoteright}, {\textquoteleft} '.}{dacht Fransje Goldeweijnen}{t wordt                     het nooit}\\

\haiku{Frederik zag hen:}{de                     stoep opstommelen}{vanuit zijn kantoor}\\

\haiku{Maar 't is toch zoo'n,,{\textquoteright}.}{kleine slaapkop dat weet}{je niet zei Stance}\\

\haiku{De                     jongen is -!}{eruit jij zal d'r ook uit}{als je niet inbindt}\\

\haiku{Maar Am\'elie hield n\`og,,.}{een boek op haar schoot als iets}{kostbaars gevangen}\\

\haiku{Al die aaa's klinken.}{als een oorlogsgehuil uit}{een                     wijde keel}\\

\haiku{Daarom is dit zoo'n.}{gezegende tijd voor}{de jonge vrouwen}\\

\haiku{Heldt, de ziel van het.}{Algemeen Nederlandsch}{Werkliedenverbond}\\

\haiku{Officier vond hij - '.}{een stom                     baantjet had}{zijn sympathie niet}\\

\haiku{Toen de zusters weg.}{waren nam Frederik}{haar in zijn armen}\\

\haiku{'t W\`as ook niet dat....}{ze oma niet helpen wou of}{niet van haar                     hield}\\

\haiku{wat hij iederen.}{dag ervaren ging als iets}{verwonderlijks}\\

\haiku{Sinds twee jaar was hij.}{mede-directeur der}{Nederlandsche Bank}\\

\haiku{Craets minachtte de,.}{vrouwen werd furieus om}{de                     arbeiders}\\

\haiku{De strijd overigens,,.}{zooals die hier gestreden}{wordt boeit me weinig}\\

\haiku{En nu ze in                     ,.}{hun program de vrouw erin}{halen meer dan ooit}\\

\haiku{{\textquoteright} Toen ze pasten keek.}{Caroline plots geboeid naar}{Louises mooien hals}\\

\haiku{{\textquoteleft}Ze wou                     ook weer.}{eens meer bij Annette en}{Frederik komen}\\

\haiku{Daar stikte ze in,!}{of ze verdronk in                     een}{walgelijken poel}\\

\haiku{{\textquoteright} riep de oude vrouw,.}{geschrikt met haar schrille stem}{De dochter keek op}\\

\haiku{wij hebben alleen.}{een ongelukkig kind om}{van te vertellen}\\

\haiku{En tenslotte week:}{alles                     onbelangrijk}{terug voor de vraag}\\

\haiku{{\textquoteleft}Z\`eg 't me niet, z\`eg', '!}{t me niet want ik word}{gek alst zoo is}\\

\haiku{Goed is                     hij als,.}{een lam maar hij laat zich niet}{op zijn kop zitten}\\

\haiku{Vreemde                     stemmen.}{spraken woorden die niet meer}{verloren gingen}\\

\haiku{Over de liefde en{\textquotedblleft}{\textquotedblright}!}{den hartstocht gillen en}{weenen op een tooneel}\\

\haiku{Ze werd zoo doodelijk bleek,,.}{dat hij schrikte dacht dat ze}{flauw                     zou vallen}\\

\haiku{Ze aaide zijn haar,.}{trok het gedachteloos in}{plukjes uit elkaar}\\

\haiku{Annette zat met,.}{de brieven in haar schoot die}{ze moeder voorlas}\\

\haiku{{\textquoteleft}Je krijgt een broertje - '.}{of een                     zusje ik zie}{t aan je moeder}\\

\haiku{Ze was weer schuw en.}{stug en keek hem letterlijk}{de kamer                     af}\\

\haiku{Francientje wilde.}{een kopje nemen en oma}{schoof er haar een toe}\\

\haiku{Kwamen de k\`erels?}{er                     niet altijd met hun}{vrachtkarren overheen}\\

\haiku{Tumult ging op, maar.}{ook hier werden de vechters}{uit elkaar gejaagd}\\

\haiku{Een kerel en een,.}{groot wijf liepen de stoep over}{bleven voor haar staan}\\

\haiku{Om h\`em hield ze zich -.}{tenslotte in verlangend}{alleen te                     zijn}\\

\haiku{Dan verlangde ze -;}{dat Philip zou komen haar}{beste kameraad}\\

\haiku{Weet u nog hoe trotsch?}{uw Pa altijd in zijn}{hart was op mevrouw}\\

\haiku{Veltman,                     mevrouw -.}{Kleine-Gartman begon}{terug te treden}\\

\haiku{Haastig, opgeschrikt, '.}{was Pieter bijt jonger}{broertje gekomen}\\

\haiku{Haar eenige zorg was -!}{geweest en hoe erg had zij}{dat al gevonden}\\

\haiku{Ze ging zitten in.}{den                     fauteuil en staarde}{den stillen tuin in}\\

\haiku{al was zij                     haast,.}{vijftig en dat beduidde}{hier in Holland oud}\\

\haiku{Heel jong had zij een;}{liaison gehad met een}{veel ouderen man}\\

\haiku{Rustig was                     hij.}{door den vreemden Hollander}{naar zijn huis vervoerd}\\

\haiku{Nu ja, nu ja, ik, '.}{weet welt is allemaal}{in liefde geschied}\\

\haiku{Ze keek hem aan, over,,.}{de tafel haar gezicht half}{spot half ergernis}\\

\haiku{Zie je, ik ben toch,.}{consequent want                     ik w{\`\i}l}{er niet naar kijken}\\

\haiku{Alleen                     Mens, de,.}{koning van de Willemstraat}{hield zijn buurt rustig}\\

\haiku{Waarom vroeg hij haar -....}{niet                     te gaan zitten of}{binnen te komen}\\

\haiku{{\textquoteright} {\textquoteleft}Eigenlijk nog niet.}{zooals ik dat bedoelde en}{ook nog altijd zoek}\\

\haiku{{\textquoteright} {\textquoteleft}Maar de eerste de,}{beste haalde je weg en}{je vergat alles}\\

\haiku{Als jij                     er niet,.}{bent komen we dadelijk}{tot malle dingen}\\

\haiku{Caroline sinds de,.}{kinderen zoo groot werden}{was van hen vervreemd}\\

\haiku{de kamer binnen,.}{waar Louise aan de tafel}{zat voor haar handwerk}\\

\haiku{Geen                     Kamerlid;}{groette het nieuwe lid of}{liet zich voorstellen}\\

\haiku{Maar Annette, met,:}{den soberen ernst van haar}{vader had gezegd}\\

\haiku{Frederik raadde ' -....}{t nooit een asyl voor dieren}{wilde oprichten}\\

\haiku{Maar 's avonds zei de,:}{oude vrouw toen zij het oom}{Pieter verteld had}\\

\haiku{Het was                     zijn met -....}{voorkeur gekozen leven}{het onbekende}\\

\haiku{geen hoogvlieger,                     .}{en zonder geestelijke}{belangstellingen}\\

\haiku{Frederik lachte.}{bij dit alles zichzelf en}{de anderen uit}\\

\haiku{Het is sterker dan -....}{vrouw en kind                    en vader}{en moeder samen}\\

\haiku{als een vrouw                     gel\'o\'ofd,.}{heeft in een man en hij gooit}{dat geloof kapot}\\

\haiku{Ze dacht:                    hoev\'e\'el.}{dingen worden er in een}{vrouw kapot gemaakt}\\

\haiku{Annette wist het,,.}{en Frederik wist het maar}{hij vroeg er nooit naar}\\

\haiku{En misschien komt hij -.}{ook heelemaal                     niet meer}{uit hij wordt te zwaar}\\

\haiku{Hij was er te                     .}{jammerlijk aan toe om hun}{lachje te zien}\\

\haiku{Wonderlijk langzaam....}{lichtte ze het deksel van}{de                     terrine}\\

\haiku{Vaag en verward in,....}{haar gloeiend                     hoofd zag ze}{Frederik zitten}\\

\haiku{{\textquoteright} Annette keek naar,,.}{hem naar zijn smalle nog zoo}{jeugdige figuur}\\

\haiku{{\textquoteright} {\textquoteleft}Dan maar niet deftig -!}{ik ben net zoo goed bij den}{slager op den hoek}\\

\haiku{Waarover dachten en....}{spraken zij ooit dan van sport}{en                     wedstrijden}\\

\haiku{Wipte dan binnen,, '.}{bij Pieter die te werken}{zats avonds laat nog}\\

\haiku{Hoe                     dikwijls ze.}{ook bij hem gezeten had}{als hij benauwd was}\\

\haiku{En Annette dacht.}{aan den morgen toen Stance}{naar Indi\"e ging}\\

\haiku{Dan doe ik 't niet, ', '.}{dan doe ikt niet Dan doe}{ikt lekker niet}\\

\haiku{En in het Paleis.}{voor Volksvlijt ging de cyclus}{der Nibelungen}\\

\haiku{zelden heeft me iets,{\textquoteright}.}{zoo wonderlijk aangedaan}{zei                     Leedebour}\\

\haiku{En telkens ook was.}{in hem de                     triomf zich}{nog vrij te weten}\\

\haiku{Hij vond het leuk door,:}{zijn dorp te loopen                     en}{gegroet te worden}\\

\haiku{manchetje streek                     .}{liefkoozend over het donkere}{hoofd aan haar knie\"en}\\

\haiku{Fritsje - ik vind -?}{ze heel lief die versjes zal}{je ze nooit wegdoen}\\

\haiku{{\textquoteright} zei hij rampzalig, {\textquoteleft}.}{maar beloof dat u niets zegt}{van de                     verzen}\\

\haiku{{\textquoteright} de onderdrukte,:}{jubel in de stem haar}{trof barstte zij uit}\\

\haiku{Het had haar dieper.}{geraakt                     dan zij zichzelf}{wilde bekennen}\\

\haiku{een                     jong meisje.}{was het tengere blonde}{vrouwtje gebleven}\\

\haiku{Nog in veel later.}{jaren zou hij zich dien blik}{herinneren}\\

\haiku{Dikwijls dat jaar zat.}{ook op het matten stoeltje}{bij De Roos Frits Craets}\\

\haiku{Daaruit blijkt hoe de.}{ziel van het volk                     er niet}{door gegrepen is}\\

\haiku{Zij wist, zij                     was:}{niet voor hem geweest wat zij}{nog voor Philip was}\\

\haiku{Maar op zijn plaats in,;}{de stalles weer                     zat hij}{wat afgetrokken}\\

\haiku{Maar ik dacht, als 't,.}{niets is waarom zou ik u}{dan laten schrikken}\\

\haiku{{\textquoteright} {\textquoteleft}Hadt je een waarborg?}{voor Philip toen je dien zijn}{eigen weg liet gaan}\\

\haiku{zij zag haar eigen,.}{leven in de baan waarin}{het gegleden was}\\

\haiku{Zij keek naar de                     ,,:}{zilverkast de bekers de}{prijzen en geeuwde}\\

\haiku{Van Frits waren het.}{brieven zonder verslag van}{eigen ervaren}\\

\haiku{{\textquoteright} dacht De Roos, de                     ,.}{teere handen volgend die}{grepen en wezen}\\

\haiku{{\textquoteleft}Karel, ik moet toch.}{eens met je praten over die}{nieuwe verzen}\\

\haiku{Alleen Frederik.}{hoorde het blijkbaar niet en}{oom Pieter                     zweeg}\\

\haiku{Ze had verwacht een,.}{ongelukkig vrouwtje een}{tragischen aanblik}\\

\haiku{{\textquoteright} {\textquoteleft}Dan is er niets aan - '...}{t is zoo leuk om naar uw}{gezicht te kijken}\\

\haiku{En niet als                     haar {\textquoteleft}{\textquoteright}.}{zuster droomde Sophie}{in ha\`ar huis overthuis}\\

\haiku{Denk aan Die Weber, -.}{Der Biberpelz van Hauptmann}{zelfs Hannele}\\

\haiku{Het heeft zoo moeten -.}{komen op deze wijze}{en in dezen vorm}\\

\haiku{{\textquoteleft}En dat het beste,.}{wat die tijd te geven heeft}{da\`aruit voort zal komen}\\

\haiku{En de wereld is '.}{aant                     veranderen}{met dien nieuwen geest}\\

\haiku{Hij was een adept                     ,,,;}{van Proudhon als journalist}{even scherp vurig raak}\\

\haiku{Doe voor mijn part een -....}{reis naar                     Egypte of klim}{in een luchtballon}\\

\haiku{Annette wachtte,.}{tot zij verder                     spreken}{zou maar zij zweeg weer}\\

\haiku{In de volgende.}{weken ging Annette meer}{naar de Hartoniussen}\\

\haiku{Was het dit, dat haar....}{plotseling terugjoeg in}{de oude baan}\\

\haiku{{\textquoteright} Hij zweeg - hij dacht, in,.}{zijn liefde voor haar helder}{dat het d\`at niet was}\\

\haiku{Langzaam boog zij zich,.}{voorover en opeens schrok de}{slapende                     op}\\

\haiku{Maar nooit kwam terug:}{waar Louise heimelijk}{elk jaar op hoopte}\\

\haiku{En zij vroeg, met haar,:}{handschoenen nog aan wat zij}{iederen dag vroeg}\\

\haiku{Het meisje leek op,.}{haar maar de moeder was vroeg}{oud en afgetobd}\\

\haiku{Kleine donkere.}{zware                     vrouwen hadden}{zijn passie gehad}\\

\haiku{{\textquoteright} Mevrouw Annette.}{zat stil voor haar toilet en}{keek in den spiegel}\\

\haiku{Zij vocht all\'e\'en om -.}{hem te                     begrijpen niet}{vreemd aan hem te staan}\\

\haiku{Maar toen hij tegen:}{Philip uit wou barsten had}{die meteen gezegd}\\

\haiku{D\`at was schoonheid, zoo -.}{goed als Catherine zoo}{goed als De Dochter}\\

\haiku{Hij wist dit zeer goed,.}{maar verwerkte het niet in}{zijn                     gevolgen}\\

\haiku{Ze st\`ond een moment,.}{keerde zich toen om en}{ging de stoep weer af}\\

\haiku{Hij liep de kamer,.}{door zooals hij gewend was zijn}{handen op zijn rug}\\

\haiku{TOEN het morgen was.}{begon de dag voor Jetje Craets}{als een                     luister}\\

\haiku{Ergens op een stoep.}{lagen twee Janhageltjes}{netjes naast elkaar}\\

\haiku{In de armen van,,,!}{een vr\'e\'emden                     man moeder}{zoo dicht bij hem b\`ah}\\

\haiku{Nee waarachtig Jet,,.}{als je getrouwd bent is me}{dat een                     leven}\\

\haiku{Zij had eens gedacht.}{ze Caroline te geven}{als die                     trouwde}\\

\haiku{de                     hoofschheid,,,.}{de begrippen van fatsoen}{van   eer van stand}\\

\haiku{Wa\`ar het vandaan kwam,,;}{w\`at het hen inblies niemand}{zou                     het zeggen}\\

\haiku{Straks zich kleeden - een -.}{diner het                     hoeveelste}{al dezen winter}\\

\haiku{Met zijn altijd toch?}{zoo heldere scherpe oogen}{verkeerd gezien}\\

\haiku{{\textquoteleft}Ja, k\"onnen wir denn -.}{gar nichts dagegen machen}{das ist doch schrecklich}\\

\haiku{{\textquoteleft}Zoo, gelukkig - we,{\textquoteright}.}{dachten je al verloren}{lachte Eug\'enie}\\

\haiku{Pieter ging den kring,:}{rond zijn scherpe oogen ieder}{gezicht monsterend}\\

\haiku{En die heeft er een.}{rijkdom   van oude en}{nieuwe                     meesters}\\

\haiku{Ze waren allen,.}{zoo jong niet meer dat wisten}{ze                     plotseling}\\

\haiku{Dat deed ze altijd, '.}{als een machine die aan}{t afloopen was}\\

\haiku{En onmerkbaar trok.}{hij zich in zijn diepste}{innerlijk terug}\\

\haiku{Ze stond even op, de,.}{cadeautjes in haar armen}{toen hij binnenkwam}\\

\haiku{{\textquoteright} Hij glimlachte, zag.}{dan plotseling hoe een traan}{langs haar gezicht gleed}\\

\haiku{hier stond, eenzaam te.}{kijken naar wat voor hem}{onbereikbaar was}\\

\haiku{Och het was ook maar.}{idee dat je buiten eerder}{zou opknappen}\\

\haiku{Moeder, vader mag.}{in die hitte niet langer}{heen en weer reizen}\\

\haiku{En even staarde ze, -.}{ernaar als betooverd sloop dan}{katzacht                     terug}\\

\haiku{{\textquoteright} vroeg ze, onverwacht.}{den ouden schersenden}{toon terugvindend}\\

\haiku{Hij reisde heen en.}{weer terwijl Frederik}{zijn vacantie nam}\\

\haiku{oma Goldeweijn waar.}{hij met                     Frans spelen ging}{in den mooien tuin}\\

\haiku{Naast hem knuffelde -.}{Mies zijn arm hij keek met zijn}{glimlach op haar neer}\\

\haiku{wat de jongen zei,.}{het was een laatste echo}{uit verganen tijd}\\

\haiku{Belangrijk waren,.}{ze plotseling geworden}{de kinderen}\\

\haiku{Een hand kroop                     naar,.}{achter tot zij belandde}{in Francines schoot}\\

\haiku{Het werd nu tijd dat.}{hij haar opvoeding wat meer}{zelf                     ter hand nam}\\

\haiku{{\textquoteright} {\textquoteleft}En vader vindt hem,{\textquoteright}.}{juist zoo leuk ontsnapte haar}{in spijtig verweer}\\

\haiku{Zij stak kinderlijk.}{haar wollen handschoentje op}{als een laatste groet}\\

\haiku{het andere, het.}{heelemaal vreemde dat}{papa in haar bracht}\\

\haiku{En dan kreeg ze het.}{gevoel of ze alles van}{hier ineens kwijt was}\\

\haiku{{\textquoteleft}Wat die rare vent,',.}{Els vader vertelde}{kon nooit veel zaaks zijn}\\

\haiku{Hij vergat den tijd.}{terwijl zij samen zaten}{uit te kijken}\\

\haiku{Jetje die binnen kwam,:}{hollen alles moest weten}{van een                     concert}\\

\haiku{{\textquoteright} Haar hoofd lag aan zijn,.}{schouder haar hand omklemde}{zijn koude vingers}\\

\haiku{Een nieuw geluid, een:.}{nieuwe gedachte door}{heel Holland vliegen}\\

\haiku{{\textquoteright} Tegen Annette:}{in haar zwaren rouw zei De}{Roos met een glimlach}\\

\haiku{Het was toch al een.}{h\'e\'ele tijd geleden dat}{oom Philip dood was}\\

\haiku{De slag van Philips.}{dood had hem zwaar getroffen}{maar niet verslagen}\\

\haiku{of hij de vreugde.}{om die                     liefde thans nog}{dieper doorproefde}\\

\haiku{Gerda zou dat ook -.}{wenschen niet n\`og eens die hel}{van samenleven}\\

\haiku{Hij was te kiesch om!}{zoo maar meteen daarover te}{kunnen spreken}\\

\haiku{hij moet zwijgen van.}{zijn moeder en praten}{van zijn grootmoeder}\\

\haiku{XIX ANNETTE nog.}{pratende kwam de trap af}{van Carolines asyl}\\

\haiku{Zij lag al op haar,.}{knie\"en en trok voorzichtig}{het dekentje weg}\\

\haiku{Dat is nog van je,{\textquoteright}.}{overgrootmoeder Goldeweijn}{zei Caroline Craets}\\

\haiku{Als er in mijn                     ,.}{tijd z\'o\'o iets geweest was dan}{had ik d\`at gewild}\\

\haiku{of niemand meer                     .}{vast en overgegeven zijn}{leven leven kon}\\

\haiku{'t Had nu eenmaal.}{vast bij hem gestaan dat zijn}{zoon zou studeeren}\\

\haiku{{\textquoteright} Hartonius liep door naar,,.}{zijn kamer smeet de deur dicht}{viel in een stoel}\\

\haiku{Het zekere, de,.}{veiligheid dat                     zalig}{overtuigde was weg}\\

\haiku{Dit kon hij toch maar -....}{niet zonder meer aanvaarden}{elkaar niet zien}\\

\haiku{{\textquoteleft}Maar 't kan hem ook....}{heelemaal niet schelen of}{Jetje hier nog ooit komt}\\

\haiku{Alleen een gevoel.}{of met die tranen zijzelf}{was uitgedord}\\

\haiku{In oorlog heb je -'.}{aan papieren geld niets is}{t eenige munt}\\

\haiku{Hij moet -{\textquoteright} haar lippen - {\textquoteleft}.}{begonnen te                     trillen}{hij moet opkomen}\\

\haiku{Maar je kunt haast geen -.}{aansluiting krijgen ik reis}{van drie uur                     af}\\

\haiku{Verbeeld je dat we.}{niet bij mekaar waren als}{er wat                     gebeurt}\\

\haiku{{\textquoteright} ~ Toen hij thuis kwam,.}{stond hij plotseling in de}{gang voor Fred Melgers}\\

\haiku{{\textquoteright} De jongen zag bleek -.}{zijn                     branie-achtige}{houding verloren}\\

\haiku{Zoo was het immers?}{geweest met Alva en}{de Watergeuzen}\\

\haiku{toen                     boog het vlak,.}{langs de Nederlandsche grens}{af Belgi\"e binnen}\\

\haiku{{\textquoteleft}Ik denk aan al de.}{moeders die nu hun zonen}{verloren hebben}\\

\haiku{Ze keek voor zich, maar,.}{ze zag zijn gezicht zoo slap}{zoo onverschillig}\\

\haiku{De duizeling door.....}{je heen dat zijn gezicht niet}{niet oprecht                     was}\\

\haiku{{\textquoteleft}hoe is er aan ons,}{rustig leven ineens}{zoo'n eind gekomen}\\

\haiku{En den volgenden:}{dag zag Amsterdam het nooit}{gekende schouwspel}\\

\haiku{{\textquoteright} Toen glimlachte Jetje,.}{uit gewoonte maar tranen}{sprongen in haar oogen}\\

\haiku{Als Mies thuiskwam vond.}{die het een pretje zoo'n paar}{uur te helpen}\\

\haiku{zijn oogen dwingend in,.}{de hare riepen haar}{plotseling terug}\\

\haiku{De Roos sloot zijn deur.}{en keek rond met een gevoel}{van verlatenhid}\\

\haiku{Zij stond moeielijk op,.}{eindelijk en hij reikte}{haar zijn hand tot steun}\\

\haiku{XXX KRAUS reed naar huis,.}{van zijn concert het tweede}{in het                     seizoen}\\

\haiku{Hier liep je nu al -....}{die kapstokken langs een voor}{een al die jurken}\\

\haiku{Dat kind - wat frisch en -.}{jong staat je aan te kijken}{of ze een spook ziet}\\

\haiku{Je staat voor vader, -;}{je gaat straks naar je eerste}{bal ja vader}\\

\haiku{Hij wist het, zag het.}{alsof zij het hem zelf nog}{eenmaal gezegd had}\\

\haiku{dit                    oogenblik,,.}{den dood van hem in haar ziel}{niet te overleven}\\

\haiku{{\textquoteleft}Een ziekte, dat zij,,...?}{zoo zuiver zoo eerlijk zoo}{exclusief liefhad}\\

\haiku{Dag en nacht liep de,.}{dijkwacht er speurend                     naar}{de zwakke plekken}\\

\haiku{Hij was                     voor 't, '.}{eerst van zijn leven moet}{liep hem over den kop}\\

\haiku{{\textquoteleft}Alsof 't haar                     .}{heusch schelen kon wat hij}{ervan dacht of vond}\\

\haiku{Nu was hij blijkbaar,.}{hersteld ging een                     kunstreis}{doen in Amerika}\\

\haiku{De jongsten zaten,,.}{erbij wat verwonderd en}{meest onverschillig}\\

\haiku{Zijn oog                     gleed naar,.}{hun korte rokken die het}{been zichtbaar lieten}\\

\haiku{En hij zag helder:}{en scherp dit opgroeiende}{vrouwengeslacht}\\

\haiku{Dit alles dacht}{Frederik en hij voelde}{met al zijn liefde}\\

\haiku{de Engelsche                     ;}{ge{\"\i}nterneerden voerden}{daar het hoogste woord}\\

\haiku{hij keek naar Betsy,,.}{die zat daar nerveus de}{vuisten verknepen}\\

\haiku{Hij lichtte even zijn,.}{hoed de oude dame neeg}{nauw merkbaar het hoofd}\\

\haiku{Als zij 's avonds in,:}{bed lagen luisterden zij}{hoorden aan de klok}\\

\haiku{ze liet zich languit, {\textquoteleft}.}{vallenper slot is Van}{Loo \'o\'ok maar een man}\\

\haiku{Ik was als de dood.}{voor die koe waar we altijd}{langs moesten naar school}\\

\haiku{Ik kon niet studeeren.}{als er koffie gemalen}{werd in de keuken}\\

\haiku{In dit oogenblik,.}{besefte hij brak wat zijn}{leven waarde gaf}\\

\haiku{ze lag bij hem op,-}{haar knieen greep naar zijn hand die}{hij wegrukte}\\

\haiku{maar zijn moeder had '....}{er altijds avonds haar muts}{op gehangen}\\

\haiku{Ik heb 't vroeger,.}{veracht idioot gevonden}{en minderwaardig}\\

\haiku{Nu benijd ik ze -.}{nu ben ik jaloersch op}{iedereen die werkt}\\

\haiku{{\textquoteleft}De oude juffrouw,{\textquoteright}.}{Leedebour                    zeiden de}{kleinkinderen Craets}\\

\haiku{Maar haar gedachten.}{gingen daaroverheen den}{eigen ouden weg}\\

\haiku{{\textquoteright} {\textquoteleft}Er is m\'e\'er in 't.}{jonge leven van dezen}{tijd dan huwelijk}\\

\haiku{{\textquoteright} {\textquoteleft}Neem me niet kwalijk,.}{ik ben maar een huismusch en}{dacht da\`ar het laatst aan}\\

\haiku{Maar aan het front - de,,,,.}{Tsaar wankel geslingerd niet}{begrijpend blijft doof}\\

\haiku{{\textquoteright} Plotseling was ze -.}{vlak bij hem haar twee handen}{om zijn arm geklemd}\\

\haiku{Het was laat toen hij.}{de deur van zijn eigen huis}{eindelijk opensloot}\\

\haiku{Ze moest lachen, zooals.}{ze sinds hun kinderjaren}{gelachen hadden}\\

\haiku{Zij stond afgewend '.}{voort raam tot zij de}{deur hoorde sluiten}\\

\haiku{Met Els wandelde,,....}{hij voor Els had hij                     tijd}{met Els praatte hij}\\

\haiku{Ik                     zal minder.}{eenzaam zijn bij vreemden dan}{in mijn eigen huis}\\

\haiku{In de Vijzelstraat.}{voor een sigarenwinkel}{dromden de menschen}\\

\haiku{staarden uit het glas.}{de luxueuse slaapkamer}{in als een geheim}\\

\haiku{{\textquoteleft}Je hebt dit keer lang -?}{gewacht met antwoorden}{schrijf je nu weer gauw}\\

\haiku{Zij zei het niemand -.}{dat zij dikwijls benauwd was}{een kramp in haar borst}\\

\haiku{Hij                     meende een,.}{ooglid te zien trillen een}{lichte ademhaling}\\

\haiku{Tot Frederik met.}{geweld het gesprek bracht op}{litteratuur}\\

\haiku{Annette hoorde:}{in hun jonge stemmen het}{leven van den tijd}\\

\haiku{Hoe makkelijk, hoe.}{snel                     stelde zijn vrouw een}{ander in zijn plaats}\\

\haiku{de kleine oude.}{hand tegen haar gloeiend}{beschreid gezicht}\\

\haiku{Je hebt nu eenmaal,{\textquoteright}.}{je kinderen niet in je}{hand prevelde hij}\\

\haiku{Och man, er is geen!}{machteloozer ding op de}{wereld dan liefde}\\

\haiku{Dat zijn moeder een.}{ander                     stelde op de}{plaats van zijn vader}\\

\haiku{De jonge Seb, een,:}{leelijken trek om zijn mond}{keek naar zijn moeder}\\

\haiku{En daarna bleef hij,.}{innerlijk eenzaam met een}{diep gemis in zich}\\

\haiku{In zijn ego{\"\i}ste,:}{egocentrische ziel was de}{eene zachte                     plek}\\

\haiku{het volk in Rusland,.}{losgebroken heeft                     de}{regeering in handen}\\

\haiku{Dit alles zagen.}{Frederik Craets en zijn}{tijdgenooten}\\

\haiku{Want ook in eigen.}{huis randde de nieuwe geest}{hen ouderen aan}\\

\haiku{Op de boot waar het,.}{stampvol was voer ze het}{donkere IJ over}\\

\haiku{Tot                     Melgers in.}{de eerste plaats als de meest}{hulpbehoevende}\\

\haiku{Ze was te oud om.}{zich over die vreemde menschen}{te vermoeien}\\

\haiku{De waarde                      van.}{een mensch is zelden voor den}{tijd waarin hij leeft}\\

\haiku{{\textquoteright} {\textquoteleft}Ja, maar jij gaat met.}{die boerentrienen om als}{met je gelijken}\\

\haiku{Toen had de ander -.}{haar stom verbaasd aangezien}{en dan gelachen}\\

\haiku{Wat v\`onden zij er '?}{dan ins hemelsnaam voor}{onfatsoenlijks in}\\

\haiku{{\textquoteright} {\textquoteleft}Ach,{\textquoteright} zei de oude, {\textquoteleft}.}{Annettedat begrijp ik}{nu zelf ook niet meer}\\

\haiku{Hij zocht het - niet zooals;}{zij dachten om zich te doen}{bewierooken}\\

\haiku{Die al wat met                     ,.}{zijn huwelijk verbonden}{was ook hebben wou}\\

\haiku{Ann\`etje hing hem,;}{aan maar jaloerscher nog zag}{Betsy dat                     Seb}\\

\haiku{Of was dit alleen....}{ongegronde angst van}{te zwaar denkenden}\\

\haiku{Zij had                     aan haar}{vaders deur geklopt en geen}{antwoord krijgend was}\\

\haiku{Hester ging den kring -,}{rond zij allen ontroerden}{omdat zij haar zoo}\\

\haiku{Hij dacht aan wat hij.}{haar                     had hooren zeggen}{tegen Ann\`etje}\\

\haiku{Het kon hem niet veel,.}{schelen wat zij                     zeiden}{de oude menschen}\\

\haiku{Maar ze speurde snel,.}{sluw zijn                     verlatenheid}{zijn hulpeloosheid}\\

\haiku{Hij                     was zich ook}{nu zeer wel bewust dat zij}{niets leek op de vrouw}\\

\haiku{Hij zag thans in een:}{diepe vermoeidheid nog slechts}{dat eene                     verschiet}\\

\haiku{, en hij                     voelde.}{haar zachte vingers bevend}{aaien over zijn hoofd}\\

\haiku{{\textquoteright} Zij nam zangles, en.}{op een avond had zij Schubert}{voor hen gezongen}\\

\haiku{Vliegtuigen ziet men,,.}{waaruit de bommen de}{granaten ploffen}\\

\haiku{Maar                     dan is er:}{plotseling aan het front zelf}{een nieuwe vijand}\\

\haiku{De lamp brandde laag -.}{zij bleef stil                     zitten om}{hem niet te storen}\\

\haiku{maar die, als alles,.}{van haar familie haar}{toch ook beklemd heeft}\\

\haiku{Hij heeft nu reeds zoo.}{dikwijls aan het sterfbed van}{een vriend                     gestaan}\\

\haiku{en Fred herinnert....}{zich de aardige brieven}{die opa hem schreef}\\

\haiku{Haar sidderende.}{handen hieven het doode}{lichaam teeder op}\\

\haiku{Het eerste wat zij:}{zag was de kamer in de}{grootste wanorde}\\

\haiku{Sloeg en vernielde,.}{wat nog de wapens de}{honger had gespaard}\\

\haiku{O ja, alles met,.}{die pijn binnenin je die}{\`al                     erger werd}\\

\haiku{{\textquoteleft}Dacht je vader, dat?}{niet iedereen ergens een}{vooze plek heeft zitten}\\

\haiku{{\textquoteright} {\textquoteleft}U is geen Craets,{\textquoteright} zei.}{hij met den eersten glimlach}{door zijn moeiten heen}\\

\haiku{hij stelt zich met het.}{leger ter                     beschikking}{der nieuwe regeering}\\

\haiku{Op Zondagmorgen}{eindelijk den Duitschers in}{hun trein overhandigd}\\

\haiku{En toen was er nog,:}{die andere beroering}{in het eigen land}\\

\haiku{{\textquoteright} Mies overeind, schreide;}{als een ongelukkig kind}{tegen Jenny aan}\\

\haiku{Net als je eens                     !}{voor je plezier gezellig}{met je hond uit was}\\

\haiku{{\textquoteright} {\textquoteleft}Neen, dat ik een ezel....}{ben vertelden mijn vrouw en}{mijn dochters me al}\\

\haiku{Hun vader is in.}{een                     gevangenkamp in}{Frankrijk gestorven}\\

\haiku{Zij zelf zweeg wat zij.}{naderen zag alsof het}{haar reeds gezegd was}\\

\haiku{{\textquoteright} Haar stem sloeg over, haar.}{handen beefden toen ze den}{rouwsluier neerstreek}\\

\haiku{{\textquoteleft}Dacht je later soms?}{je huishouden ook te doen}{met zoo'n doek om}\\

\haiku{{\textquoteright} {\textquoteleft}Ik b\`en geen vrouw als.}{oma Craets en jouw moeder en}{als moeder Hartonius}\\

\haiku{ik zou alleen je;}{vrouw kunnen worden als je}{me                     d\'a\'arin vrij liet}\\

\haiku{{\textquoteright} Het ging door haar heen,.}{d\`at ze geen man wilde die}{zich aan ha\`ar vasthield}\\

\haiku{{\textquoteright} Toen hij de Leliegracht,.}{opkwam liep Ann\`etje hem}{haast in de armen}\\

\haiku{En tusschen Hartonius.}{en                     haarzelf groeide een}{nieuwe late band}\\

\haiku{M'n kleine dot - mijn -....}{snoezig diertje ben je}{dan mijn hartedief}\\

\haiku{{\textquoteleft}Het is ook mijn deel.}{niet meer die groote kinderen}{te begrijpen}\\

\haiku{{\textquoteright} En de jonge vrouw,,.}{de oogen helder stak haar}{arm door dien van Fred}\\

\haiku{{\textquoteright} Maar alleen Mies kreeg.}{een dieper inzicht in dit}{jonge huishouden}\\

\haiku{Aan haar zoon Pieter,.}{dacht Annette al noemde}{zij zelden zijn naam}\\

\haiku{Zij zag er in haar.}{donkere japon rustig}{bevredigd uit}\\

\haiku{Bij de beroering,,:}{de verbazing die het in}{de familie bracht}\\

\haiku{Er fluistert iets, dat -....}{niet meer is dan een ademtocht}{dat lijkt op een naam}\\

\subsection{Uit: Vrouw Jacob}

\haiku{Hij was waarschijnlijk,,.}{de eenige die dit inzag}{en daarom zweeg hij}\\

\haiku{in haar eerend het.}{nieuw opgroeiend ideaal van}{het ridderwezen}\\

\haiku{Een niet te breken}{wilskracht had zij ge\"erfd van}{vader en moeder}\\

\haiku{Arkel heimelijk.}{gesteund door zijn zwager Van}{Gelre en Gulik}\\

\haiku{Was niet zijn gansche!}{regeering bemoeilijkt door de}{vervloekte Arkels}\\

\haiku{Zoo kind - dat gebroed,.}{is ten onder gebracht v\'o\'or}{het jou kwaad kan doen}\\

\haiku{Beneden is een.}{ridder met een boodschap van}{Hare Majesteit}\\

\haiku{En zij voelde zich,....}{koud worden in een gevoel}{van verlatenheid}\\

\haiku{Vrouwe Jacoba.}{zal beter passen op den}{Franschen koningstroon}\\

\haiku{En zij, aanziende,.}{deze mannen wist zich op}{eenmaal geen kind meer}\\

\haiku{Zij luisterde naar,;}{dat diep en scheurend brullen}{beklemmend geboeid}\\

\haiku{in verteedering,.}{voor het kind in vereering voor}{de ontwaakte vrouw}\\

\haiku{En ik houd van de -,.}{Henegouwers die Franschen}{ik vertrouw er geen}\\

\haiku{In den nacht lagen,,.}{zij dicht tegen elkaar in}{het groote statiebed}\\

\haiku{{\textquoteleft}De Dauphin zal niet,!}{anders in Parijs komen}{dan met Bourgondi\"e}\\

\haiku{Hij tastte naar haar {\textquoteleft} -{\textquoteright}, {\textquoteleft}.}{hand.Jaque stamelde hijlaat}{me niet meer alleen}\\

\haiku{De berichten die,.}{uit Bouchain naar Quesnoy kwamen}{luidden weinig nieuws}\\

\haiku{haar teederste vriend,.}{de verafgode vader}{ging van haar heen}\\

\haiku{{\textquoteright} Jacoba hoorde,.}{het alles aan met niet te}{doorgronden kalmte}\\

\haiku{En even klopt haar hand,,:}{de sterke gespierde den}{hals  van haar paard}\\

\haiku{hier is - ho\`e is niet -;}{te vatten een andere}{wereld geworden}\\

\haiku{De uitkomst zoeken.}{uit deze verwarring van}{wenschen en afkeeren}\\

\haiku{De Bourgondi\"er;}{hield zichzelf bescheiden uit}{den familieraad}\\

\haiku{{\textquoteleft}Als uw oom zal ik,{\textquoteright}.}{ook in Brabant wel komen}{besloot hij milder}\\

\haiku{De laatst gesproken.}{zinnen bleven hangen in}{de stilte om hen}\\

\haiku{Jacoba ontving.}{hem plechtig en gratievol}{in de ridderzaal}\\

\haiku{Ge hebt mij,{\textquoteright} sprak hij -.}{verteederd ondanks zichzelf}{om haar lieve jeugd}\\

\haiku{Naar Dordrecht begaf,.}{hij zich en de stad haalde}{hem jubelend in}\\

\haiku{Gij hebt niet genoeg.}{aanhang om den Hoeken de}{macht te ontnemen}\\

\haiku{Oh, schooner, heerlijker,!}{beteekenisvoller aanval}{was er nooit gedaan}\\

\haiku{waarop elke man.}{zich bereid houdt voor haar zijn}{leven te geven}\\

\haiku{Ze zijn niet tegen,!}{te houden de Hoeken niet}{en de onzen niet}\\

\haiku{Voor haar staat de groote,.}{gestalte van Brederode}{tot den strijd gerust}\\

\haiku{Tot het laatste toe;}{heeft Arkel gewacht met het}{sein tot den aanval}\\

\haiku{In dit helsch gedrang.}{is Brederode verder van}{Arkel afgeraakt}\\

\haiku{Door het heele land.}{was de verslagenheid der}{Kabeljauwen groot}\\

\haiku{Zij trok naar Den Haag,,.}{Vrouw Jacob en alles viel}{haar jubelend toe}\\

\haiku{Den Beier fnuiken, -!}{wa\`ar zij kon had hij haar niet}{den liefste gemoord}\\

\haiku{Reeds driemaal waren.}{de landen door den keizer}{aan vrouwen beleend}\\

\haiku{Jacoba in haar,.}{overgevoeligheid ving het}{op en huiverde}\\

\haiku{De verovering;}{van Brielle kon alleen te}{water geschieden}\\

\haiku{Niet bezielde haar.}{vreugde en hoop als bij den}{tocht naar Gorcum}\\

\haiku{de belegering.}{van de machtigste stad in}{den wereldhandel}\\

\haiku{En een  nimbus;}{omgeeft het hoofd van Dame}{Jaque die dit aandurft}\\

\haiku{Daar binnen de stad;}{ziet Jan van Beieren de}{vijandin komen}\\

\haiku{en de burgers zijn.}{welgewapend en op een}{beleg voorbereid}\\

\haiku{Maar in het leger,.}{zelf wordt nu reeds gebrek aan}{proviand gevoeld}\\

\haiku{maar er is geen toorn,.}{in hen slechts radeloosheid}{om eigen onmacht}\\

\haiku{In den nacht nog gaat.}{de bode met den brief naar}{het Brabantsche kamp}\\

\haiku{De roem van Gorcum.}{is vergeten en door al}{dezen smaad gefnuikt}\\

\haiku{Rotterdam was bij.}{verrassing den Beier in}{handen gevallen}\\

\haiku{{\textquoteleft}Zoudt ge den Beier?}{tot een onderhoud kunnen}{bewegen met ons}\\

\haiku{Met scherpen blik mat,.}{hij den man die zich zwijgend}{op een knie neerliet}\\

\haiku{{\textquoteright} {\textquoteleft}Gijlieden verraadt?}{en verkoopt uw hertogin}{achter haren rug}\\

\haiku{Alleen gelaten,,.}{bleef hij lang stil zitten het}{hooge voorhoofd gefronsd}\\

\haiku{Zij begreep wat hem?}{hier had gebracht in opdracht}{van zijn heer vader}\\

\haiku{De Beier was er -,!}{meester van erfelijk leen}{wat beteekende d\`at}\\

\haiku{Maar l\'a\'at ons tijd, laat -.}{ons even rust we k\`unnen niets}{op dit oogenblik}\\

\haiku{de vurige, licht;}{opkomende blos liep tot}{in haar blanken hals}\\

\haiku{{\textquoteright} Dien avond liet Van den.}{Berghe zich bij den hertog}{verontschuldigen}\\

\haiku{{\textquoteleft}Zij zijn ontkomen,, -....}{Heer in de verwarring men}{heeft hen niet herkend}\\

\haiku{Hij zal zich, nu hij,.}{verward en verlaten is}{aan u vastgrijpen}\\

\haiku{Dame Jaque was de -!}{sterkste geweest en ho\`e had}{zij toegeslagen}\\

\haiku{Pas op, dat ge niet,.}{verliest wat ge met zooveel}{zorg hebt opgebouwd}\\

\haiku{Een verdrietige,,.}{booze uitdrukking rimpelde}{haar jong glad gelaat}\\

\haiku{- er werd daar wegens....}{een groote overstrooming zeer}{naar zijn komst verlangd}\\

\haiku{{\textquoteleft}Wat beklaagt ge u,,?}{Genadige Vrouwe over}{gebrek aan eerbied}\\

\haiku{het woord te nemen?}{zonder daartoe verlof te}{hebben gekregen}\\

\haiku{{\textquoteleft}Er valt aan eenmaal.}{uitgesproken besluit niet}{te veranderen}\\

\haiku{{\textquoteright} Jacoba stond zoo,.}{recht dat haar lichaam scheen een}{strak gespannen veer}\\

\haiku{Naakt en uitgeschud.}{bleef zijn lijk den ganschen nacht}{op de plek liggen}\\

\haiku{Gij zijt toch immers -!}{zoo'n krijgsoverste zoo beroemd}{in het oorlogsveld}\\

\haiku{Het zal op uw hoofd,!}{neerkomen z\'o\'o dat ge u}{niet meer te keeren weet}\\

\haiku{{\textquoteleft}Bekruisig u niet,,!}{tegen mij maar tegen uzelf}{tegen uw vrienden}\\

\haiku{Mijn gemalin had}{een beteren kijk op de}{toestanden dan gij}\\

\haiku{Toen bond Tserclaes in,,.}{voer in Jacoba's schuitje}{raadde verzoening}\\

\haiku{in dit warnet van.}{vijandelijkheid wist hij}{zich niet te redden}\\

\haiku{Zij ontmoette den,;}{trouwen bezorgden blik van}{Marie van Nagel}\\

\haiku{Zij zag opeens zich:}{te Woudrichem en hoorde}{Vianen zeggen}\\

\haiku{Hij wist, dat niemand.}{er haar toe zou krijgen dit}{te bezegelen}\\

\haiku{Ge hebt Dame Jaque,{\textquoteright}.}{ook op zij willen zetten}{troefde de knaap kwaad}\\

\haiku{{\textquoteright} dacht hij verlangend - {\textquoteleft}.}{het leven zou zoo veilig}{en vroolijk wezen}\\

\haiku{Deze bracht hem de....}{berichten van den Beier}{en zijn operaties}\\

\haiku{en tegelijk wist.}{hij hem de oorzaak van al}{deze ellende}\\

\haiku{{\textquoteleft}Uw Vrouwe is in,{\textquoteright},.}{Brabant zei Tserclaes een der}{volgende dagen}\\

\haiku{een verzoening tot.}{stand te brengen tusschen u}{en uw gemalin}\\

\haiku{Hij voelde zich hier:}{in Den Bosch rampzaliger}{nog dan in Brussel}\\

\haiku{Alleen daarom al.}{was het haar een genoegen}{hem te dwarsboomen}\\

\haiku{Hoek schonk - den mannen,....}{van haar Raad een oud vriend van}{het hof haars vaders}\\

\haiku{Hij had haar in zijn,;}{ridderlijke eerlijke}{jongensziel zeer lief}\\

\haiku{Ge komt mij te Mons -.}{bezoeken met het tournooi}{op Driekoningen}\\

\haiku{Deze nacht werd wel.}{misschien de bitterste uit}{dezen ganschen tijd}\\

\haiku{De zon gaat op, een,.}{stralende lentedag een}{gelukkig voorteeken}\\

\haiku{Nu - nu eindelijk.}{voelt zij zich de bruid van Humphrey}{van Glocester}\\

\haiku{des konings waren.}{Engeland's hoogste adel en}{de vorsten bijeen}\\

\haiku{hetgeen de zwaarste.}{kerkelijke straffen zou}{brengen over hun hoofd}\\

\haiku{Scherp, honend en uit,!}{de hoogte had het meiske}{gedaan tegen h\`em}\\

\haiku{Nu n\`og voelde zij!}{zich besmeurd ooit naast den knaap}{te hebben geleefd}\\

\haiku{In dien tijd zal de.}{dood den stokouden bisschop}{toch wel eens halen}\\

\haiku{Geen genegenheid,.}{eigenbaat alleen had hen}{hem doen aanhangen}\\

\haiku{Jan van Beieren.}{viel in de stad en dempte}{in bloed elk verzet}\\

\haiku{Zij liet liever het.}{land te gronde gaan dan dat}{zij het hem gunde}\\

\haiku{En feller joeg zij,}{de Hoeken die immer in}{grooter getale}\\

\haiku{Treedt gij alzoo den!}{laatsten wil van uw grooten}{koning met voeten}\\

\haiku{De Brabander heeft!}{de gravin van Holland tot}{zijn bijzit gehad}\\

\haiku{liet zonder verweer.}{daarna den toorn van Philips}{over zich heen woeden}\\

\haiku{{\textquoteleft}Doulce chose est,....}{que mariage Je le puis}{bien par moy prouver}\\

\haiku{Jacoba opnieuw.}{verloor zich in gemijmer}{over deze Odette}\\

\haiku{Aan den blaaskaak, dien,!}{hij haatte die hem als een}{knecht had behandeld}\\

\haiku{Echter bezwoer hij,.}{Winchester te zorgen dat}{Humphrey niets ondernam}\\

\haiku{Welnu,{\textquoteright} lachte hij - {\textquoteleft}?}{dan kortademigWat kost}{mij mijn zielerust}\\

\haiku{Hij noemde vlak en.}{zonder nadruk als het doel}{van zijn reis Holland}\\

\haiku{Hij had blijkbaar ook;}{daar in den hof te lang in}{de vocht gezeten}\\

\haiku{Ge begrijpt, zwager,;}{dat ge op mijn troepen niet}{kunt rekenen meer}\\

\haiku{{\textquoteleft}Well, 't lijkt me een -.}{goed land alleszins waard het}{te veroveren}\\

\haiku{Hij boog de knie en {\textquoteleft},,}{kuste haar hand.Mijn vriend mijn}{altijd getrouwe}\\

\haiku{Sinds Albrecht's regeering.}{waren zij vijanden van}{het Beiersche Huis}\\

\haiku{{\textquoteright} Een groote beweging;}{beroerde als een golfslag}{de Nederlanden}\\

\haiku{Hij had, wat hemzelf,;}{verwonderde geen wrok meer}{tegen Jacoba}\\

\haiku{Hij dacht, hoe weinig.}{hij haar bijzijn in rust had}{kunnen genieten}\\

\haiku{Haar hand op zijn arm,,;}{gebleven verlaten en}{hulpeloos gleed af}\\

\haiku{En daar op eenmaal.}{stond weer die ellendige}{Hollandsche kwestie}\\

\haiku{Straten, stadhuizen,.}{landschappen trokken langs zijn}{schemerende oogen}\\

\haiku{Het had haar van geen,.}{belang geleken indien}{zij Humphrey slechts bezat}\\

\haiku{De heele sfeer van.}{haar kinderjaren omspon}{haar vertrouwd en warm}\\

\haiku{Maar het Engelsche - -?}{leger dat de redding moest}{brengen waar bleef het}\\

\haiku{{\textquoteright} Goud rinkelt neer in.}{zijn haastig en begeerig}{gespreide handen}\\

\haiku{hij zou tegen een '.}{slag int open veld niet meer}{opgewassen zijn}\\

\haiku{Eindelijk zal ik.}{deze heele kwestie uit}{de wereld helpen}\\

\haiku{Toen plotseling in.}{dat ondragelijk zwijgen}{rees de hertogin}\\

\haiku{Die thans naar voren,.}{drong in een plotselingen}{heftigen argwaan}\\

\haiku{De weg naar Calais,{\textquoteright}.}{bracht haar gebroken stem op}{een oogenblik uit}\\

\haiku{Dacht, gepijnigd om,.}{haar aan het glorierijke}{begin van den tocht}\\

\haiku{Het werd een  spel,,.}{waarin al wat Hoeksch was zijn}{wilde vreugde vond}\\

\haiku{Thans is er voor u,.}{slechts \'e\'en weg die u in uw}{eer zal herstellen}\\

\haiku{Gij zegt Neef,{\textquoteright} zeide, {\textquoteleft}.}{Philipsdat ge den oorlog}{meer dan moede zijt}\\

\haiku{Hij was niet tegen,.}{haar opgewassen maar de}{Bourgondi\"er wel}\\

\haiku{Nu is hij binnen,.}{de stad gekomen en klaagt}{het vorstenvolk aan}\\

\haiku{zij nemen ook de.}{hertogin-moeder weer}{op in hun midden}\\

\haiku{Uit de strakke lucht,.}{brandt de heete zon en het}{regent nog steeds niet}\\

\haiku{En haar wettige!}{man staat voor de muren en}{belegert de stad}\\

\haiku{{\textquoteright} Jacoba voelt hun,.}{afval als een mist die haar}{kil van hen afsluit}\\

\haiku{er is ook gebrek -.}{in de stad er is vrees en}{wankelmoedigheid}\\

\haiku{{\textquoteright} Maar een razende.}{vloed van stemmen verdrinkt en}{overspoelt de zijne}\\

\haiku{Donderend valt zijn,,....}{stem waarvoor alles altijd}{zweeg in het rumoer}\\

\haiku{In haar slaapvertrek,,.}{ligt zij voorover haar gelaat}{diep in het kussen}\\

\haiku{hij zal u alles,...... ......}{uitvoeriger zeggen dan}{ik het schrijven kan}\\

\haiku{En haar gelaat is,....}{zeldzaam roerend als zij hen}{een voor een aanziet}\\

\haiku{Alles in den burcht,.}{lijdt onder de hitte de}{meedoogenlooze zon}\\

\haiku{Zij merkt alles op,,.}{er ontgaat haar niets bij hen}{die nog om haar zijn}\\

\haiku{Dat volk, ho\`e is niet,....}{te begrijpen opeens aan}{de macht gekomen}\\

\haiku{Even slaat zij den blik,,.}{naar boven naar den burcht waar}{zij woonde met h\`em}\\

\haiku{Doe het mij niet aan,!}{u te zien opgeofferd}{aan mijn vijanden}\\

\haiku{Zag den avond terug,.}{toen de Beier Arkel en}{hem had doen roepen}\\

\haiku{Dit alles wist hij,.}{zeer wel en hij moest hierin}{Egmond bijvallen}\\

\haiku{haar vorstelijken,,.}{tooi waarin zij hem overtrof}{haar te ontnemen}\\

\haiku{Alsof haar geest die.}{uren ontsnappen wilde aan}{martelend denken}\\

\haiku{Zij laat een kluwen,.}{vallen het rolt tot vlak aan}{Jacoba's voeten}\\

\haiku{Weg met Jacoba,,....}{naar Rijssel voorloopig}{de sterke vesting}\\

\haiku{Zij zal in Holland,,,.}{zijn en hem aanvallen den}{roover en overwinnen}\\

\haiku{Geen paard meer onder -....}{zich het langzame rijden}{van een boerenkar}\\

\haiku{Soms ontmoeten zij,;}{ruiters die even naar binnen}{kijken in de kar}\\

\haiku{Over het water, waar,.}{de riemen plassen rijst de}{toren van Gorcum}\\

\haiku{waaruit thans een hand.}{een klein vaandel heft met de}{Beiersche kleuren}\\

\haiku{Ze heeft geslapen -,!}{ho\`e goed hoe veilig diep en}{rustig geslapen}\\

\haiku{In Zeeland brengt Van.}{Haemstede een aanzienlijk}{leger op de been}\\

\haiku{Jacoba was in,.}{Gouda gebleven had er}{haar hof gevestigd}\\

\haiku{en de eilanden.}{en groote steden onder hen}{bleven vijandig}\\

\haiku{Zij is veranderd -.}{een machtig flu{\"\i}dum schijnt}{van haar uit te gaan}\\

\haiku{de weg naar den troon,.}{van Engeland het einddoel}{van al haar streven}\\

\haiku{hoe heeft ze gedaan,?}{in Henegouwen hoe heeft}{zij dien trouw beloond}\\

\haiku{XI Het volk in de:}{Nederlanden zag wat het}{nog nooit had gezien}\\

\haiku{De Hoeksche steden,,.}{al wat zich voor Jacoba}{verklaard had lachten}\\

\haiku{{\textquoteleft}Het is de Beier,,.}{geweest hertog Jan die het}{\`al heeft bedorven}\\

\haiku{{\textquoteleft}Ik voor mij w\'e\'et, dat,.}{wie zich aan uw inzicht houdt}{veilig is bewaard}\\

\haiku{{\textquoteright} Zij stond plotseling,.}{stil of zij daadwerkelijk}{voor een afgrond stond}\\

\haiku{{\textquoteright} {\textquoteleft}Als zij met goede....}{loodsen aan boord het juiste}{oogenblik kiezen}\\

\haiku{En n\`u eindelijk,:}{gebeurde wat Marie van}{Nagel verwachtte}\\

\haiku{Op d\`at oogenblik.}{davert de lucht van beider}{kanten krijgsgeschrei}\\

\haiku{Hoeksch of Kabeljauwsch -....}{wat doet het jong is zij en}{mooi en verlaten}\\

\haiku{Eindelijk rechtte;}{Van Nagel het grijze hoofd}{en wilde spreken}\\

\haiku{{\textquoteleft}Als ge haar beter,;}{hadt doen bewaken was dit}{alles niet geschied}\\

\haiku{een  licht rood op,.}{zijn bleekt wang bewees dat de}{pijl had getroffen}\\

\haiku{Uren en uren, tot laat,.}{in den nacht soms verbleef zij}{met haar getrouwen}\\

\haiku{Grepen hun wapens.}{van den muur en telden hun}{weerbare mannen}\\

\haiku{Het was een gloed, die.}{als een ketting van vuur liep}{door de gansche streek}\\

\haiku{en optrok met een.}{grooter en sterker leger}{dan ooit te voren}\\

\haiku{En \`op rukten van.}{alle zijden Jacoba's}{troepen naar de stad}\\

\haiku{v\'o\'or Philips met zijn,.}{groote macht aankwam moest de stad}{in hun handen zijn}\\

\haiku{Maria sta ons bij -!}{het schip met den standaard van}{Vrouw Jacob in top}\\

\haiku{Vrouw Jacob is hij,.}{toegedaan met lijf en ziel}{al kent zij hem niet}\\

\haiku{Bij de Kennemer,.}{boeren buiten de stad is}{de stemming gekeerd}\\

\haiku{In den laten avond,,;}{soms alleen vond haar Marie}{starend naar buiten}\\

\haiku{{\textquoteright} Het wordt groot - het vult - -.}{de kamer de gansche stad}{het geheele land}\\

\haiku{Niet haar trots, niet haar,,, -...?}{roem haar heerlijkheid haar kracht}{haar schoonheid maar dit}\\

\haiku{Hij ziet de boeren.}{terugdeinzen onder de}{moordende pijlen}\\

\haiku{geen Kennemer of.}{Westfries mocht ooit weer wapen}{of harnas dragen}\\

\haiku{{\textquoteright} Haar vuisten gebald,.}{tegen haar witte lippen}{geperst kreunde zij}\\

\haiku{{\textquoteright} Marie die zwijgend,.}{haar leed begroef en stil trouw}{naast haar voortleefde}\\

\haiku{Jacoba lachte.}{haar vreugdeloozen lach als zij}{dit alles hoorde}\\

\haiku{Oh, gruwelijker,!}{verbijsterender dan \'e\'en}{mensch op aarde wist}\\

\haiku{En voor 't eerst zag:}{zij de onbezonnenheid}{van die vlucht over zee}\\

\haiku{En haar rechten op.}{het hertogdom Luxemburg en}{het graafschap Chimay}\\

\haiku{{\textquoteright} ~ De pen kraste,,,.}{de woorden klonken helder}{bewust scherpzinnig}\\

\haiku{En hield dien thans aan,.}{haar borst gedrukt al den tijd}{dat zij met hem sprak}\\

\haiku{Op een dag, komen:}{burgers en dringen aan bij}{den bevelhebber}\\

\haiku{Jacoba zag ze,.}{uit haar raam en klemde de}{handen in den schoot}\\

\haiku{plicht, ja w\'a\'arlijk plicht,!}{te zorgen dat zijn erfgoed}{niet verloren ging}\\

\haiku{Een tweede schrijven.}{richtte hij aan de Lords van}{den Geheimen Raad}\\

\haiku{Hij weet, dat alle.}{tegenstand u slechts prikkelt}{om vol te houden}\\

\haiku{Mijn werk - uw schoone - -.}{droom de droom van een kroon in}{Engeland heeft uit}\\

\haiku{Een jonge vrouw in -.}{wapenrusting stond voor hem}{een z\'e\'er jonge vrouw}\\

\haiku{In de voorzaal zijn,,....}{bijeen Van Kijfhoeck Van de}{Merwede Montfoort}\\

\haiku{En Gode zij dank,....}{den Engelschman heeft zij}{opzij geworpen}\\

\haiku{Zij is koud tot in,.}{haar gebeente of zij nooit}{meer warm worden zal}\\

\haiku{Philips, op vertoon,;}{van praal en macht bedacht hield}{een groote wapenschouw}\\

\haiku{En als gijlieden,;}{uw handen uitbreidt verberg}{Ik Mijn oogen voor u}\\

\haiku{Maar als de eersten,,:}{omhooggevlogen boven}{aankomen zien zij}\\

\haiku{{\textquoteright} Zij ging aan het raam -.}{vol stortte de stralende}{zomerdag over haar}\\

\haiku{Iederen dag de,.}{jachtfeesten de tournooien}{te harer eere}\\

\haiku{Borre van Doirninck,,;}{die trouw haar gevolgd was zag}{haar verlorenheid}\\

\haiku{{\textquoteleft}Ik had in Gouda,.}{moeten blijven tot de groote}{rust van den dood kwam}\\

\haiku{Nu besta ik voort,.}{zonder rust in de engte}{van een dood leven}\\

\haiku{Hij stond recht, bevend {\textquoteleft}}{over zijn gansche lichaam met}{rooddoorloopen oogen}\\

\haiku{{\textquoteright} {\textquoteleft}Die zelf - d\`at moet Uwe -.}{Genade thans wel weten}{nooit uw vijand w\`as}\\

\haiku{{\textquoteleft}Vertelt gij mij, wat.}{aan Philip's hof wordt verhaald}{van La Pucelle}\\

\haiku{{\textquoteright} {\textquoteleft}En tenslotte als -.}{heks verbrand te Rouaan nu}{een week geleden}\\

\haiku{En hij wist ook, dat.}{Glocester's ontrouw haar had}{doen capituleeren}\\

\haiku{maar voelde hij meer?}{voor Vrouw Jacob of voor het}{welzijn van het land}\\

\haiku{Dan plotseling dacht,.}{hij hoe zij door haar Hoeksche}{vrienden alles wist}\\

\haiku{Wilde zij w\`el den,,?}{opstand om haar landen maar}{niet hem als gemaal}\\

\haiku{Hoe kwam tegen haar,.}{grove trekken Jacoba's}{fijne gratie uit}\\

\haiku{Er was een groene;}{kamer met wonderlijke}{beesten aan den wand}\\

\haiku{{\textquoteright} Hij was neergeknield,,.}{en kuste haar hand. Maar recht}{weer zag hij haar aan}\\

\haiku{De verluchting na.}{den Zoen van Delft was ook in}{zijn hart lang getaand}\\

\haiku{De eindelijke.}{vergelding voor al wat zij}{hem had doen lijden}\\

\haiku{Voorgoed van dezen -.}{aardbodem doen verdwijnen}{nooit meer in zijn weg}\\

\haiku{Met zonderlinge,....}{aandacht haar bezagen de}{zwarte stille oogen}\\

\haiku{De sterken balden,.}{de vuisten verlegden hun}{hoop naar beter tijd}\\

\haiku{Nu had zij ook Van.}{Borselen's leven gewaagd}{en vernietigd}\\

\haiku{Ho\`e, dat laat ik aan,.}{God en zijn geweten over}{te beoordeelen}\\

\haiku{Hij bracht belangrijk,{\textquoteright}.}{nieuws hernam zij plotseling}{met iets aanvallends}\\

\haiku{Men houdt hier niet van,{\textquoteright}.}{de Engelschen zei de Zeeuw}{Van Borselen hard}\\

\haiku{De helle hemel.}{met zijn blindend zomerlicht}{stortte op haar toe}\\

\haiku{Als werkelijk de ',?}{Engelschen int zicht zijn}{zullen zij vechten}\\

\haiku{Weet hij, of in 't?}{geheim de heks niet met hen}{gemeene zaak maakt}\\

\haiku{Met elken vriend gaat.}{heen voorgoed een moment uit}{het voorbije leven}\\

\section{Louis de Bourbon}

\subsection{Uit: Twaalf maal Azi\"e}

\haiku{Zij liep een spoorweg,.}{over sloeg rechtsaf en kwam op}{een groote boulevard}\\

\haiku{Er was geen verkeer.}{op de straat en ook in de}{huizen was het stil}\\

\haiku{- Dan weet ge misschien,?}{wel waar ik een betrekking}{zou kunnen vinden}\\

\haiku{- Zoo, zei de man en,.}{boog nu heelemaal naar haar}{over laat dat eens zien}\\

\haiku{Toen hij daar aankwam,.}{stond de Arabier juist op het}{punt te vertrekken}\\

\haiku{Loop hard, opdat hij.}{niet de gelegenheid heeft}{ze uit te geven}\\

\haiku{En bij Sitih had,.}{ieder gelijke rechten}{die betalen kon}\\

\haiku{- O, wat griezelig,,,!}{zei de dame die naast mij}{zat wat vreeselijk}\\

\haiku{Wij stonden nu op,.}{het bordes de dame hing}{nog steeds aan mijn arm}\\

\haiku{De Annamieten.}{werden kopschuw en trokken}{terug op Chi-hoa}\\

\haiku{Het was hard noodig de.}{noordelijke gebieden}{te consolideeren}\\

\haiku{Dupuis vertrok over,.}{zee naar Hai-Phong vandaar}{over land naar Hanoi}\\

\haiku{Zijn mooi zwart haar was.}{grijzend en op sommige}{plaatsen bijna wit}\\

\haiku{Maar om een leven.}{op te bouwen moest hij haar}{opnieuw verlaten}\\

\haiku{Hij dronk een tweeden.}{beker wijn en ging door de}{open deur naar buiten}\\

\haiku{Een onvoorzichtig,.}{marinier werd getroffen}{en stierf huilende}\\

\haiku{Hij had in Pha-ung.}{een schrander en toegewijd}{helper gevonden}\\

\haiku{Het is geen angst, maar,.}{onrust die zich onder de}{bezetting verspreidt}\\

\haiku{De vanen steken.}{scherp af tegen de wolken}{aan den horizon}\\

\haiku{Tegen de muur zat,,.}{een oude Chinees alleen}{en leek te slapen}\\

\haiku{Hij keek mij aan met,.}{een wreeden grijns en lachte}{spottend uitdagend}\\

\haiku{But look here, this,,.}{is a saphire Sir a}{real saphire}\\

\haiku{Nog twee seconden,,,.}{dacht Arthur als ik mijn sabel}{heb ben ik gered}\\

\haiku{- Kijk, daar loopt Sitih,,.}{fluisterde hij de vrouw die}{sakit hati is}\\

\haiku{en dat ik altijd.}{dezelfde gebleven was}{die ik geweest was}\\

\haiku{Ik stapte in de,.}{auto en wij reden weg}{ditmaal bergafwaarts}\\

\haiku{Doch dan deed hij het,,.}{ook geducht hij liet zich dan}{volloopen boordenvol}\\

\haiku{Wie hem de meeste,.}{in zijn bakje terugbracht}{mocht er \'e\'en opeten}\\

\haiku{Soms ook vertelde,.}{hij hun verhalen in zijn}{eenvoudig Maleisch}\\

\haiku{Ank van de overkant.}{is ook nog altijd vrij en}{praat veel over je}\\

\haiku{- U niets te vreezen,,,.}{zei hij ik U laten zien}{U moet weten}\\

\haiku{Daarbinnen lag, stijf,,.}{opgerold een glanzende}{grijze vilthoed}\\

\subsection{Uit: Vier onbekenden}

\haiku{Wat lijkt dat nu weer,.}{lang geleden het lijkt een}{heel andere tijd}\\

\haiku{Een erg mooie jonge.}{vrouw en die wordt dan verliefd}{op den luitenant}\\

\haiku{Hij liep mijn kant uit,.}{en ik ging naar hem toe om}{hem voor te lichten}\\

\haiku{Eventjes maar, toen ging.}{hij een loge binnen en}{ik weer op mijn plaats}\\

\haiku{Ik kwam nooit voor half,.}{\'e\'en thuis als moeder en broer}{al in bed lagen}\\

\haiku{Heel dicht lag ik naast,,.}{hem zijn adem ging over mijn bloote}{schouders als hij sprak}\\

\haiku{- Hoe kan het vandaag,,,.}{zijn dwaze maagd zei hij de}{dag is ten einde}\\

\haiku{Dat moest prachtig zijn.}{en ik had Arthur toch ook in}{den winter ontmoet}\\

\haiku{Ik geloof, dat ik.}{zoo een heele poos heb staan}{denken en droomen}\\

\haiku{Ik sloot opnieuw mijn.}{oogen en begon mij alles}{te herinneren}\\

\haiku{Het schijnt zoo, dat de.}{meesten zich om deze vraag}{niet bekommeren}\\

\haiku{Ze werkt immers zelf '.}{op een fabriek ens avonds}{is ze in City}\\

\haiku{Maar ik zou hem wel,.}{kunnen helpen als ie een}{keer m'n zwager was}\\

\haiku{Misschien ook wel is,.}{ze bang dat het gaan zal als}{de vorige keer}\\

\haiku{Hij werd een beetje,.}{geplaagd in de barak maar}{dat ging al gauw over}\\

\haiku{- Ik heb gehoord, dat,,.}{je rijk bent geweest zei ik}{zoo langs mijn neus weg}\\

\haiku{- Maar hoe is het dan,,.}{met jou vroeg ik je bent toch}{zelf arm geworden}\\

\haiku{Hij was trouwens voor,.}{elk werk ongeschikt dat wist}{hij uit ervaring}\\

\haiku{Ja, ik moet eerlijk,.}{bekennen dat ik er een}{beetje trotsch op was}\\

\haiku{De zon was al een,.}{poosje op in de boomen}{zongen de vogels}\\

\section{Menno ter Braak}

\subsection{Uit: D\'emasqu\'e der schoonheid}

\haiku{zijn bewonderaars.}{zijn de \'e\'endags-vrienden}{van de politiek}\\

\haiku{Omdat zij nog niet?}{gearriveerd waren op}{hun zestiende jaar}\\

\haiku{Hij zal den puber;}{niet zonder meer toelaten}{in zijn gemeenschap}\\

\haiku{het gaat tusschen de.}{schoonheid als soepsnuiverij}{en als bevrijding}\\

\haiku{Heeft God de wereld?}{ook op zulk een krukkige}{manier geschapen}\\

\haiku{8 Er is nog een;}{derde mogelijkheid in}{den strijd met den vorm}\\

\haiku{dat hun aesthetiek,?}{naar afval riekt wie zal er}{zich over verbazen}\\

\haiku{De kunst is thans door,;}{de natuur heengegaan zij}{is niet olympisch meer}\\

\haiku{in 't proza staan;}{zij in een klaar licht of als}{in een schemering}\\

\subsection{Uit: Dr. Dumay verliest...}

\haiku{Hij kende dat, het.}{onberedeneerbare}{verradersgevoel}\\

\haiku{{\textquoteleft}Zeg, waarde heer, zou?}{je nu eindelijk het woord}{eens willen nemen}\\

\haiku{en in de handen,.}{van Jean Wood zag hij zijn hoed}{tasch en wandelstok}\\

\haiku{Onafhankelijk,,...}{daarvan en van alles wat}{anderen aangaat}\\

\haiku{Nu zullen Victor,.}{en ik nooit meer zoo kunnen}{vechten als vroeger}\\

\haiku{hij wist niet, wat hij,,,...}{deed hem overkwam iets een daad}{een zoo-maar-iets}\\

\haiku{Ik weet heel zeker,.}{dat ik er eigenlijk geen}{behoefte aan heb}\\

\haiku{Hebt u tijdens mijn?}{absentie de Germania}{verder behandeld}\\

\haiku{Idee\"en zonder flair;}{behooren waarschijnlijk bij}{mijn constitutie}\\

\haiku{Het was Dumay niet,.}{volkomen helder waarom}{hij zelf gegaan was}\\

\haiku{Hij wist, dat hij op;}{het kerkhof nauwelijks om}{Jean Wood gerouwd had}\\

\haiku{Maar zegt u nou zelf,,!}{mijnheer Donner wat heb je}{dan nog aan je hond}\\

\haiku{{\textquoteleft}Mag ik de heeren,{\textquoteright}.}{even aan elkaar voorstellen}{zei hij onhandig}\\

\haiku{Zij liepen een paar.}{honderd meter onder een}{pijnlijk stilzwijgen}\\

\haiku{Later op den avond,:}{kwamen de vrienden waarover}{Max gesproken had}\\

\haiku{Het individu,:}{scheen samen te spannen met}{Souzie of liever}\\

\haiku{Wees maar blij, dat je.}{met vaderlijke zorgen}{niets te maken hebt}\\

\haiku{we laten op dat.}{punt onze log\'e's altijd}{maar aan hun lot over}\\

\haiku{{\textquoteright} {\textquoteleft}Als de zaken er,,.}{naar staan is er geen reden}{om nog te wachten}\\

\haiku{Een oude dame.}{raapte er een paar op en}{gaf ze haar terug}\\

\haiku{Dus, gegeven het,.}{aantal getrouwde mannen}{t\`och geen gewoon mensch}\\

\haiku{u zoo vriendelijk,?}{zijn even uit te kijken waar}{u uw voeten zet}\\

\haiku{Tegelijkertijd.}{onmoetten zijn oogen weer die}{van den dronken boer}\\

\haiku{En plotseling vond,.}{hij zichzelf liggen op een}{vreemden zak stompend}\\

\haiku{Het was duidelijk,.}{aan hem te zien dat hij zich}{zat te ergeren}\\

\haiku{Hij was een goede,,.}{brave man geweest al had}{hij ook zijn fouten}\\

\haiku{Dat kwam er van, als...}{het geloof er niet meer was}{om halt te roepen}\\

\haiku{Toe, ik heb nog niets,.}{van je gehoord hoe je het}{gehad hebt en zoo}\\

\haiku{{\textquoteright} {\textquoteleft}Ik zeg u toch, dat,!}{ik genoeg van hem heb dat}{ik hem niet meer wil}\\

\haiku{maar in het volgend.}{oogenblik had hij haar al}{weer opgevroolijkt}\\

\haiku{Zij bleven voor de.}{photo's staan en besloten}{de film te gaan zien}\\

\haiku{de titel van de.}{film was hij aan de kassa}{al weer vergeten}\\

\haiku{En hij weet, dat hij,.}{het liegt dat ik het eerlijk}{met hem gemeend heb}\\

\haiku{Soms is een dom woord.}{van juffrouw van der Wall voor}{mij een zaligheid}\\

\haiku{Alle gebouwen;}{lagen koel en scherp in het}{maanlicht te baden}\\

\haiku{Welneen, \`als hij het,.}{geweten had was hij het}{zeker vergeten}\\

\haiku{Kunt \`u hem nu niet?}{eens een duwtje geven in}{de goede richting}\\

\haiku{U had hem moeten,!}{zien toen hij voorzitter was}{van zijn H.B.S.-club}\\

\haiku{Nel had een massa,;}{kennissen die onderling}{veel avondjes hielden}\\

\haiku{Nu dan: ik heb me,:}{verloofd met iemand van wie}{je nooit gehoord hebt}\\

\haiku{Jawel, jullie doen,!}{maar tegenwoordig jullie}{denkt maar aan jezelf}\\

\haiku{De sportiviteit;}{veerde met Dumay trede}{na trede verder}\\

\haiku{Haar gezicht was wat,.}{opgezet zij hijgde van}{het trappen klimmen}\\

\haiku{{\textquoteleft}U moet hem redden,...,.}{juffrouw als u hem niet redt}{is hij verloren}\\

\haiku{Met mij spot hij toch,,}{maar ik ben niets voor hem van}{mij neemt hij niets aan}\\

\haiku{dit had toch best nog,...}{kunnen wachten meubels zijn}{in \'e\'en dag gekocht}\\

\haiku{{\textquoteright} zei hij grijnzend in):}{de richting van Dumay dan}{goedkoop uit kon zijn}\\

\haiku{Begrijp jij, dat er,?}{menschen zijn die in zoo'n}{bed willen slapen}\\

\haiku{Overigens, hij doet,,...{\textquoteright} {\textquoteleft}?}{in kikkers dat is zijn vak}{dusIs hij getrouwd}\\

\haiku{Een tasch viel met een,.}{smak op den grond papieren}{vlogen links en rechts}\\

\haiku{{\textquoteleft}Maar... wat praten we,,?}{toch allemaal we gaan toch}{trouwen wij twee\"en}\\

\haiku{Zonder veel aandacht.}{bladerde zij in \'e\'en der}{nieuwe aanwinsten}\\

\haiku{de baas groette hem,.}{vriendelijk hij tikte aan}{zijn hoed en reed weg}\\

\haiku{daarvoor komt zij op,,.}{en dat is mijn schuld dat is}{niet af te wasschen}\\

\haiku{als hij maar praat en!}{als Lydia maar theeschenkt en}{de baby verzorgt}\\

\haiku{De adem van den tuin;}{kwam door de groote openstaande}{deuren naar binnen}\\

\haiku{De baby, schoot hem,?}{door het hoofd er zal toch niets}{met de baby zijn}\\

\haiku{Je bent juist op tijd,.}{gekomen ik moet er met}{iemand over praten}\\

\haiku{Ik hoop van niet en.}{kom zonder tegenbericht}{morgenavond bij je}\\

\haiku{Hij betrapte zich,.}{op een sensatie die op}{teleurstelling leek}\\

\haiku{Met een paar sprongen.}{was Dumay boven en in}{Karin's slaapkamer}\\

\haiku{Wil je dat op je...?}{geweten hebben dat ik}{me om jou doodschiet}\\

\haiku{{\textquoteright} zei hij bevelend, {\textquoteleft},,.}{je weet dat ik niet meer te}{zeggen heb Karin}\\

\haiku{Maar ze is verliefd,...!}{op je dat ouwe mensch die}{vogelverschrikker}\\

\haiku{Maar even nog drong zich:}{tusschen de nevelen door}{een vraag naar voren}\\

\haiku{zij wist niet, of zij;}{Dumay geluk ging wenschen}{met zijn huwelijk}\\

\subsection{Uit: Hampton Court}

\haiku{Op dat oogenblik,.}{wist hij plotseling dat hij}{het park niet zou zien}\\

\haiku{Naast hem lagen een.}{Engelschman en zijn vrouw}{op hun dekstoelen}\\

\haiku{Toen hij dien avond naar,.}{boven ging had de slaap hem}{al bijna overmand}\\

\haiku{Andreas liep er;}{onwillekeurig heen en}{betastte den stam}\\

\haiku{Je leeft naast een vrouw,.}{alsof Hendrik VIII er}{geen zes had gehad}\\

\haiku{Dank, dank, hoesten en,!}{morgenschemering dank voor}{het pad naar den slaap}\\

\haiku{hij manoeuvreerde,.}{handig met zijn valies tot}{hij vlak bij haar was}\\

\haiku{Aan de trams, die over,;}{het stationsplein reden}{hingen vlaggetjes}\\

\haiku{De trambestuurder,.}{hernam zijn plaats er kwam weer}{schot in het verkeer}\\

\haiku{Ja, het was alles,;}{hetzelfde maar het had nu}{geen medelijden}\\

\haiku{De verteedering;}{verloste hem eensklaps uit}{zijn afwezigheid}\\

\haiku{Ze is weg en ik,.}{zou eigenlijk niet wenschen}{dat zij terugkwam}\\

\haiku{Onverwachts toch nog;}{stormden een paar hossende}{jongens langs hem heen}\\

\haiku{Diederik's juffrouw,.}{scheen aan den rol te zijn er}{werd niet opengedaan}\\

\haiku{Diederik, die hem,.}{tegemoet was gekomen}{zag hem verbaasd aan}\\

\haiku{Dat was waar ook, er,.}{was nog iemand dat was hem}{door het hoofd gegaan}\\

\haiku{{\textquoteleft}Ja, dat je het met,.}{mijn bewering niet eens zou}{zijn wist ik ook wel}\\

\haiku{Maar nu remde hem.}{de aanwezigheid van van}{Haaften hinderlijk}\\

\haiku{Hij boog diep voor haar,.}{zonder zich te storen aan}{hem uit de bioscoop}\\

\haiku{Trouwens, je hoorde,!}{zelf die flauwe kul die hij}{verkocht over dat Delftsch}\\

\haiku{Ja, klets maar jongen,,!}{van Haaften is er immers}{niet nu durf je wel}\\

\haiku{Toen trilde ergens.}{uit het niets een verlossend}{denkbeeld op hem toe}\\

\haiku{en daarop stond (zag?),.}{hij goed een soldatenpet}{studentikoos scheef}\\

\haiku{Toen drong het tot hem,.}{door dat het stil was en dat}{hij hier moest praten}\\

\haiku{Ik voor mij, ik ben.}{deze houten klaas dankbaar}{voor zijn openbaring}\\

\haiku{Hij leefde nog in,.}{een andere wereld die}{van het luisteren}\\

\haiku{en weer zweefde zijn:}{stem tusschen boosaardigheid}{en genegenheid}\\

\haiku{Daar is niets aan te,.}{doen en daarom moet ik me}{wel zoo uitdrukken}\\

\haiku{Dat de menschen hem,!}{ongevoelig en cynisch}{noemden wat zou dat}\\

\haiku{Daarmee had hij dus,.}{geleefd daarmee gingen dus}{ook veel menschen dood}\\

\haiku{Het is geen eerbied,,.}{het is geen slaafschheid het}{is nu eenmaal zoo}\\

\haiku{In dit opzicht, dacht,,,.}{hij ben ik het niet met hem}{eens goddank goddank}\\

\haiku{d{\`\i}t effect heeft d\`at,.}{ten doel d{\`\i}e stoot moet leiden}{tot d{\`\i}e positie}\\

\haiku{maar een hypocriet,!}{in den gewonen zin was}{hij niet volstrekt niet}\\

\haiku{Had zij eigenlijk,?....}{wel eens verteld op welke}{etage zij werkte}\\

\haiku{{\textquoteright} Maffie glimlachte,.}{zonder den glimlach van den}{klant op te vangen}\\

\haiku{Maffie bracht bloemen,,.}{mee verplaatste kleinigheden}{deed huisvrouwelijk}\\

\haiku{zij dronken ergens,,;}{koffie en ergens thee zij}{aten samen ergens}\\

\haiku{Hij trof van Haaften.}{later op den avond in het}{gewone kroegje}\\

\haiku{Dat is nu alles,.}{goed en wel maar je geeft geen}{antwoord op mijn vraag}\\

\haiku{Hij moest tegen wil,.}{en dank grijnzen terwijl hij}{de straatdeur dichttrok}\\

\haiku{laten we liever!}{ons entree tot de eerste}{acte uitstellen}\\

\haiku{Hij is mijn vriend, ja,,,;}{zeker dacht hij niet meer mijn}{patroon zooals vroeger}\\

\haiku{vergeefs, de lange,!}{vingers trillen als zij een}{kopje aanpakken}\\

\haiku{men zag een sober,.}{decor opgebouwd uit niet}{veel meer dan planken}\\

\haiku{de courtisane;}{scheen voor de nieuwe idee\"en}{te moeten wijken}\\

\haiku{Van Rees was zeker,,.}{een innemende man een}{aristocraat jawel}\\

\haiku{Iets heet al even gauw,.}{cynisch als symbolisch op}{een bepaald niveau}\\

\section{Karel Broeckaert}

\subsection{Uit: De sysse-panne: Borgers in den estamin\'e}

\haiku{Verboden wordt het.}{geestelik habijt in het}{openbaar te dragen}\\

\haiku{alors commenc\`erent.}{la spoliation et}{le vandalisme}\\

\haiku{{\textquoteleft}Even stellig is de:}{Gazette fran\c{c}aise bij}{de proklamatie}\\

\haiku{A.J.        Inleiding...}{I De Vlamingen komen}{altijd achteraan}\\

\haiku{Zijn optreden valt.}{samen met de ingang van}{een nieuw tijdsgewricht}\\

\haiku{- gy zoed niet gelooven!}{dat n\^en Bedelaer meer kan}{doen als n\^en Bankier}\\

\haiku{Zanne doet Mie op.}{staen geeft-ze leujens}{om kaffe t'haelen}\\

\haiku{aen hangt, om gelyk,.}{Gysken zegt de Menschen van}{hem schuw te maeken}\\

\haiku{Gysken. 't Es imand.}{van myn kennisse die my}{hier in gesleept heeft}\\

\haiku{Rentier kan worden,,;}{maer tog als hy het gaede}{slaet het helpt altyd}\\

\haiku{- De Kloosters doen meer ';}{dienst aent Publiek aes de}{Parochie Kerken}\\

\haiku{wy verfoyen in;}{onze Sermoenen alles}{wat de ryke doen}\\

\haiku{Wat deugden winnen?...}{zy by het verliezen van}{hunne Religie}\\

\haiku{Zyn zy geldzugtig,;}{zy verzuymen geenen middel}{om'er te krygen}\\

\haiku{want het Huwelyk.}{en is niet anders dan eene}{plegtige hoerery}\\

\haiku{naer verlangde, en,.}{noch lange van sprak alsze}{gepasseert waeren}\\

\haiku{en uweu Vader die, '.}{het in het hymelyk ziet}{zalt u loonen}\\

\haiku{(Pabula tyrannorum \&) ':}{sunt Plebs Rusticit zyn}{oude spreekwoorden}\\

\haiku{Boeren boet - Boeren;}{zyn loeren en schelmen uyt}{de natuere}\\

\haiku{Neen Koninginne;}{zeyd den Koning m\^e zullen'er}{Ossen m\'e koopen}\\

\haiku{'K en h\^ek eventwel.}{ze leven niet gezien of}{den Regen kwampter}\\

\haiku{Het vraegt eene sterke}{en overtuygende Penne}{om te bewyzen}\\

\haiku{Ploeg heeft twee Knegten.}{moet hebben en verscheyde}{Daghuermannen}\\

\haiku{5o Dat het klaer is,,}{wanneer den Grond niet alles}{opbrengt dat hy kan}\\

\haiku{Dat 'er maer een slag,.}{van Maeten Gewigten en}{van Wetten zyn moet}\\

\haiku{Ze keunen 't er,!}{veuren doen wat zegde vyf}{honderd Millioen}\\

\haiku{Het register van.}{de Burgerlike Stand werd}{in 1796 ingevoerd}\\

\haiku{het bestuur van het.}{huis werd die dag toevertrouwd}{aan het jongste kind}\\

\haiku{een pas geboren.}{kind in het kerkregister}{laten inschrijven}\\

\haiku{zig ten leur laten,.}{stellen zich met de vinger}{laten nawijzen}\\

\haiku{116De oude en -.}{de nieuwe Wetten31 Julie}{13 Thermidor IVBlz}\\

\section{Johan W. Broedelet}

\subsection{Uit: Hofstad. Deel 1}

\haiku{Z'n gang had dan ook ',.}{t zwevende van een die}{de aarde nauw voelt}\\

\haiku{Alleen moest-ie eens ',.}{n ander dasje koopen}{want dat rafelde}\\

\haiku{ik heelemaal niets{\textquoteright},.}{van hebben draaide zich om}{en wilde weggaan}\\

\haiku{Dat colporteeren.}{met advertenties is me}{niet meegevallen}\\

\haiku{De viool zette '.}{metn zucht z'n instrument}{weer onder z'n kin}\\

\haiku{Ze was bezig, 'r,.}{bloemen te begieten waar}{ze groote zorg voor had}\\

\haiku{Bij zulk 'n rijkdom '.}{van gemoed paste slechtsn}{eerbiedvol zwijgen}\\

\haiku{{\textquoteright} De barones van ' '.}{Liktum Priktum vroegt met}{r gewonen spot}\\

\haiku{Daarop keerde de,.}{meneer zich in eens om liep}{naar den vijver toe}\\

\haiku{{\textquoteright}, wees-ie meteen, '.}{op de twee damesn goed}{vrachtje voor terug}\\

\haiku{Er zich nu verder,.}{niet mee bemoeiend stapten}{de vriendinnen in}\\

\haiku{Vagebond keerde, ',.}{metn gezicht of-ie nooit}{anders gedaan had}\\

\haiku{Nou, versta jij, jij, ',.}{zorgtt komt weer heel of ik}{hou van jou loon af}\\

\haiku{Kaatje schreef met  'n, ' '.}{gemakkelijkheid datt}{r zelf verbaasde}\\

\haiku{Als z\'o\'o de entre\'e,?}{tot den salon was hoe moest}{die z\`elf dan wel zijn}\\

\haiku{O, ze zou 't nog,!}{uitschreeuwen van ontzetting}{als dat niet ophield}\\

\haiku{Of was dat juist 'n,.... {\textquoteleft},,,!}{teeken dat zeO mama}{mama z\`eg toch wat}\\

\haiku{En ze snelde naar,,.}{mama terug die de oogen}{flauw opsloeg steunend}\\

\haiku{De schoonheid van 't '.}{Herteveld trofr toch}{telkens weer opnieuw}\\

\haiku{Ik adoreer 't. 't,.}{Heeft me gisteravond he\`el he\`el}{gelukkig gemaakt}\\

\haiku{Denk niet, dat ik u,,.}{om die gewone vunse}{aardsche liefde smeek}\\

\haiku{{\textquoteright} woelde 't in den,.}{opgezetten rooien kop}{van den zweep-ridder}\\

\haiku{Als 't nu niet uit, '.}{was met die aardigheden}{riep-ien agent aan}\\

\haiku{{\textquoteright} Weg was Amie weer en '.}{Vagebond tuurde stil over}{den kop vant paard}\\

\haiku{Z'n houding was in.}{alle omstandigheden}{des levens correct}\\

\haiku{De w\`aarlijk hooge adel.}{bleef natuurlijk voor immer}{voor hem gesloten}\\

\haiku{Ja, niet alleen 't!}{m\`anlijk geslacht takelde}{af met de jaren}\\

\haiku{Struggle slenterde, '.}{terug bekeek zich van op}{zij inn spiegel}\\

\haiku{'t Was 'n mond vol.,.}{Vervelend dat dat goedje}{altijd zoo kleefde}\\

\haiku{Even wist ze nergens,,,.}{van soezend op gevaar af}{in slaap te knikken}\\

\haiku{Maar als-ie 'r nog, ' '?}{beminde opr ouden}{dag omr hand vroeg}\\

\haiku{Ook, de alt mocht dan, '.}{wezen wie ze wilde die}{toon stondm niet aan}\\

\haiku{Et \`a pr\'esent vous,....}{vous permittez des blagues}{des plaisanteries}\\

\haiku{De heerlijkheden;}{der wereld liet-ie kalm langs}{z'n mouw afzakken}\\

\haiku{Je phantasie krijgt,,,, '.}{de ruimte breekt uit zwelt}{barst wordtn volheid}\\

\haiku{De aster stak-ie ',.}{int knoopsgat Amie dicht aan}{z'n oor te hebben}\\

\haiku{O, kon ze 'm maar,,!}{heel even zien n\`u wat zou ze}{er niet om geven}\\

\haiku{Daarop was-ie nog,.}{even aan haar kamers geweest}{die ze op slot hield}\\

\haiku{Z'n ziel toog er in,,.}{huppelde mee dat-ie}{zichzelf niet meer was}\\

\haiku{Dan schoof ze ter zij,,.}{voor de danseuse noble}{die aan de beurt was}\\

\haiku{Geen haat gevoelde,.}{ze langer voor die Lydia}{twee loges verder}\\

\haiku{Onder haar leiding ' {\textquoteleft}{\textquoteright}.}{wast huisKlingeling zeer}{in bloei gestegen}\\

\haiku{En getinkel drong.}{door en eerst niet geziene}{menschen doken op}\\

\haiku{Ik geloof juist, dat '.}{hij int algemeen te}{copieus dineert}\\

\haiku{En de knecht van den '.}{baron van Priktum heeftr}{juist teruggebracht}\\

\haiku{O, ik beh\`oef die!}{bedwelming na mijn arbeid}{van zooveel dagen}\\

\haiku{Gelukkig echter.}{had Jacobi er zich nog}{altijd uit gered}\\

\haiku{'t Begon voor 'm,.}{te draaien dat-ie haast}{tegen den grond sloeg}\\

\haiku{Met de opening van.}{z'n salon moest-ie toch niet}{te lang meer wachten}\\

\haiku{t Beste dan ook!}{was nog niet goed genoeg voor}{d\'ezen cli\"ent}\\

\haiku{Vooral 'n vijf, zes ', '.}{kinderen warent die}{r aandacht vroegen}\\

\haiku{Laat ons daarom de,,.}{uurtjes die ik vandaag vrij}{heb daar doorbrengen}\\

\haiku{{\textquoteright} {\textquoteleft}Leo{\textquoteright} antwoordde 't,,.}{manneke dat nu maar wou}{dat dat koekje kwam}\\

\haiku{'t Liep mis met 'm. {\textquoteleft}{\textquoteright} '.}{DieKleine Courant vergde}{te veel vanr Wim}\\

\haiku{Hij was net op, toen ', '.}{z'n juffrouwm zei dat er}{iemand voorm was}\\

\haiku{Vagebond echter, ',.}{vans zangers gepeinzen}{niet wetend hield aan}\\

\haiku{Mam\`a, 't levendste!}{bewijs van de verfijning}{der van Hogenloo's}\\

\haiku{En zonder 'n woord,, '.}{meer draaide-ie zich om ging}{n eigen kant uit}\\

\haiku{Van de honderd pop.}{van vanmorgen had hij er}{nog vijf en twintig}\\

\haiku{Zonder haar zou z'n '.}{levent gewone zijn}{als van iedereen}\\

\haiku{Montrose nam z'n, '.}{sigaar weer op deedn paar}{geduchte trekken}\\

\haiku{Als ieder van te,,,!}{voren wist hoe-ie later}{zou worden nu nu}\\

\haiku{H\'el\`ene, zich in 'r,.}{nachttoilet te steken liep}{de spiegelkast langs}\\

\haiku{Bij kaarsen-schijn, ', ',.}{dan opt hooge bed wast}{dat hij haar bezat}\\

\haiku{Alleen, in mannen,.}{woelden zwarter dingen dan}{men vaak vermoedde}\\

\subsection{Uit: Hofstad. Deel 2}

\haiku{Doch Lydia, die zich, ' '.}{verveelde weemoedigde}{t meer mett uur}\\

\haiku{n Druk legde zich ',.}{opr waaraan ze zich te}{ontwentelen had}\\

\haiku{Waarlijk, ze hechtte ', ' '.}{zich aanm zoum niet graag}{missen inr huis}\\

\haiku{Ach, in Hofstad had. '!}{ieder zoo z'n vermaakjes}{n Stad van pleizier}\\

\haiku{Dat krioelde en.}{kringde en slinger-schoof}{zonder ophouden}\\

\haiku{'k heb je zoo goed.}{als nooit iets in allen ernst}{hooren beweren}\\

\haiku{Wat Eveline ook, ', '.}{zeim te weerhoudent}{bleef zonder gevolg}\\

\haiku{'t Mensch spartelde ', '.}{onderm alsof zem}{nog ontglippen ging}\\

\haiku{Op 'n soort standaard,, '.}{koper gehelmd vlotten}{costuumgedeelte}\\

\haiku{in 'm. Tien vingers.}{zette-ie verticalig}{op z'n schrijfbureau}\\

\haiku{En hij zette zich,.}{de handen stevig aan de}{leuning van z'n stoel}\\

\haiku{Ja, als-ie er over ', '!}{n piano beschikte}{zout nog w\`at zijn}\\

\haiku{De golven harer, '.}{verontwaardiging sloegen}{over datt kookte}\\

\haiku{Vanmiddag was-ie.}{andermaal gerefuseerd}{op de Oude Gracht}\\

\haiku{Intusschen, dit nam, ' '.}{niet weg datt geheeln}{zaak bleef tusschen h\`en}\\

\haiku{En hij ging door voor,.}{Titri's amant wat-ie in}{waarheid toch niet was}\\

\haiku{O, neen, daar kwam h{\`\i}j,,.}{de dokter van de  chic}{zoo goed als nooit in}\\

\haiku{Ach ja, als je in '!}{n nacht-societeit ook al}{ging critiseeren}\\

\haiku{O, 't kostte 'm.}{altijd nog veel meer dan-ie}{goed betalen kon}\\

\haiku{Maar kom, geheel voor!}{spek en boonen zat die er}{toch zeker niet bij}\\

\haiku{Hoe kwam-ie daarbij,, '?}{hij diet anders nooit over}{pati\"enten had}\\

\haiku{- daar, vanuit 't graf, '.}{om zoo te zeggen nogn}{zeer slimmen zet deed}\\

\haiku{In dienst moest alles,.}{bij tijds zijn vroeg naar bed en}{vroeg uit de veeren}\\

\haiku{H{\`\i}j, die zooveel voor,.}{de jongens deed de ziel van}{Purpurae Amori}\\

\haiku{Hij moest oppassen,, '.}{voelde-ie oft ging den}{gezochten kant uit}\\

\haiku{En z'n bezoeken.}{aan de Berghuisen werden}{steeds geregelder}\\

\haiku{{\textquoteright} Zelfs al stapte-ie -!}{over alles heen hij voelde}{er zich toe in staat}\\

\haiku{Wie echter zou 'n?}{gravin van Montrose de}{toegang weigeren}\\

\haiku{Hij zou er misschien.}{meer uit afleiden dan-ie}{noodig had te weten}\\

\haiku{Hemel, die stijve!}{Keutellanders ook met hun}{enge begrippen}\\

\haiku{Olga, 'r hoed af -.}{dien waagde ze er van avond}{in godsnaam maar aan}\\

\haiku{Kwam er weer niets van? ' '.}{t Leekm toch zoo'n groote ernst}{bij dien Amerikaan}\\

\haiku{Maar, o ja, voor die,,!}{dansen straks welke ze had}{helpen instudeeren}\\

\haiku{Ze voelde, dat 'r,.}{iets gebeuren ging dat ze}{niet wilde missen}\\

\haiku{We\`et  je 't nu,,,?}{mijn vlam mijn zomer-brand}{mijn lente-braam}\\

\haiku{Niet w\`eer eens mocht '.}{ze de kans loopen vann}{kortstondig geluk}\\

\haiku{er was niets tusschen.}{mevrouw de barones en}{meneer Vagebond}\\

\haiku{'t Was heel vroeg, nog,.}{donker toen Kaatje de deur der}{Dim's achter zich sloot}\\

\haiku{Als ze weer wachtte ',.}{tott volledig dag werd}{kwam er toch niets van}\\

\haiku{Dim, die maar 'n rol, '.}{gespeeld had naar bleek trachtte}{r deur te forceeren}\\

\haiku{Waarvoor haalde-ie ', ' ' {\textquoteleft}{\textquoteright}?}{r toch aan alst niet om}{tgewone was}\\

\haiku{George zat aan ', '.}{t andere eind van de}{kamer opn stoel}\\

\haiku{Ach, ja, wie begreep ',?}{m w\`el den al zieliger}{George Kapel}\\

\haiku{Hij heeft alleen maar '{\textquoteright}.}{n zieltje zei ze nog met}{verklaarbaren spot}\\

\haiku{De laatste kende,.}{ze enkel van gezicht den}{graaf wel van vroeger}\\

\haiku{Hoe kwam echter de?}{baron van Priktum tusschen}{dat troepje verzeild}\\

\haiku{Ik ben nog altijd,,.}{je bezit maar jij behoort}{mij ook voll\`edig}\\

\haiku{Hij verzette er, '.}{zich niet tegen keek zelfsn}{beetje toegevend}\\

\haiku{Ze escorteerden ' ',.}{alst waret drietal}{dat daar vooruit liep}\\

\haiku{{\textquoteright} Echtgenooten,.}{behoorden te weten wat}{elk hunner toekwam}\\

\haiku{Maar gaandeweg ging '.}{ze toch wat aandachtiger}{naarm luisteren}\\

\haiku{De roemruchtige,.}{schrijver kraakte geweldig}{z'n oogen sloten zich}\\

\haiku{En ze zweeg maar, want:}{luid werd ze overstemd door de}{al dringender kreet}\\

\haiku{Doch onverstoorbaar '.}{gingt mannetje in z'n}{hemd met dansen voort}\\

\haiku{C'est doux comme le.}{coeur de toutes ces jeunes}{filles adorables}\\

\haiku{In z'n vreugd om 't ' -,, '!}{eindigen vant seizoen}{ach Godt werd tijd}\\

\haiku{Hij voelde, hoe-ie,.}{weer omhoog ging zichzelf niet}{langer kon sturen}\\

\haiku{De tijdschriften en ' '.}{t publiek zaten met de}{handen int haar}\\

\section{Pol Brounts}

\subsection{Uit: Mien leef lui}

\haiku{'t Had get eweg vaan ':}{nen oonderwijzer dee aon}{Harieke vraog}\\

\haiku{Ouch al umtot ze.}{neet zoe good k\'os dinke wie}{d'n aptieker zelf}\\

\haiku{{\textquoteright} Wie ze 's aoves}{m\`et hunnen twieje bijein}{zaote kraog}\\

\haiku{Mesjiens had heer tev\"a\"ol.}{elend gezeen en gehuurd um}{n\'e\'et kontent te zien}\\

\haiku{Binne pakde iech.}{de kaart um ins te kieke}{wie tot iech zouw rije}\\

\haiku{Es me aajd woord en,,.}{me had gei geld mie mezjieu}{daan heel alles op}\\

\haiku{Heer beloerde miech.}{vaan bove tot oonder veur}{zoeveer es dat g\'ong}\\

\haiku{kleier zoonder ei.}{woord te z\`egke en maakde}{tot ze weg kwaome}\\

\haiku{{\textquoteright} {\textquoteleft}Veer mage,{\textquoteright} zag Leio, {\textquoteleft}.}{langsaameigelek gaaroet}{neet aon h\"a\"om koume}\\

\haiku{Es kr\"a\"olke h\"ob {\textquotedblleft}...}{iech altied mote zinge}{In Paradisum}\\

\haiku{En oonder 't dook {\textquoteleft}{\textquoteright},.}{m\`etI had a dream druimde}{ampa allermins}\\

\haiku{veendelkes in ze, ' {\textquoteleft}}{wegelke de hermenie}{sp\"a\"olde nogne kier}\\

\haiku{Dat huurste wel ins, '.}{d\`ekser tot ze daan ininsn}{opleving kriege}\\

\haiku{In eder geval 'nen,.}{interlektuweel wie Zjang}{dat altied neumde}\\

\haiku{Dat lielek mins in.}{de twiede rij zaog heer}{nog altied zitte}\\

\haiku{Dao kwaom dat lielek, '.}{st\`el m\`et t\"ossen hun int}{printsje vaan Rafael}\\

\haiku{Wee zouw tege die?}{lui gezag h\"obbe tot ze}{dat zoe m\'oste z\`egke}\\

\haiku{Die zwarte m\`es h\"ob,.}{iech zoe d\`eks gezoonge die}{kin iech vaan boete}\\

\haiku{Wat bleef dao nog veur,?}{ier m\`et in te l\`egke m\`et}{zoe'n begraffenis}\\

\haiku{Daan kin me nao de '.}{hoegm\`es koume en dao maak}{iechn st\`el m\`es vaan}\\

\haiku{Neet sjus wie heer 't,.}{ziech had veurgest\`eld meh gans}{anders es anders}\\

\haiku{En toen beg\'os de.}{begraffenis nog ins good}{op dreef te koume}\\

\haiku{Dao verklaore,{\textquoteright}.}{ze alles m\`et allewijl}{zag d'n andere}\\

\haiku{Koffie euver h\"a\"om,,}{eweg euver ze good pak z'n}{botram m\`et gelei}\\

\haiku{Sjoen netuur, prachteg,, '.}{weer lekker werm en zoe en}{ne sjoene camping}\\

\haiku{Ze had toen op h\"a\"or ',.}{batsn dikke roef sjijf wie}{e riestevl\"a\"ojke}\\

\haiku{Good, ze koume bij,:}{d'n dokter dee bekiek ziech}{dat geval en zeet}\\

\haiku{{\textquoteright} Heer duit ins hei op,,,:}{die bats knip ins dao geer kint}{dat wel heh zoe vaan}\\

\haiku{En deen Hollender.}{m\'oste ze dat allemaol}{oetl\`egke vaanzelf}\\

\haiku{Dat waor wie d'n.}{dirrekteur belde of Jean}{effe k\'os koume}\\

\haiku{Heuge waor nog.}{neet dao en heer had ouch niks}{laote hure}\\

\haiku{En zuuste tot dee?}{mins m\`et ziene linkervoot}{achter die ploej zit}\\

\haiku{Este veur ederein,...}{get origineels w\`els h\"obbe}{m\`et e leuk versje}\\

\haiku{Heer wis zeker tot!}{heer veur deen heilige ge{\'\i}n}{surpries had gemaak}\\

\haiku{Dao waoren 'rs, '.}{gen\'og die dat k\'oste gebruke}{al bracht niks op}\\

\haiku{N\'og e gel\"ok waor,.}{tot heer die ruimde gans k\'os}{gebruke ouch nog}\\

\haiku{Meh noe...{\textquoteright} {\textquoteleft}Noe goon veer '!}{op de Vriethof eve opn}{terreske zitte}\\

\haiku{Ze h\'ong wied euver '.}{de leuning en keek naot}{water oonder h\"a\"or}\\

\haiku{H\"a\"oren aosem rook nao.}{kauwgom en oonder zien hand}{voolt heer h\"a\"or sjouwer}\\

\haiku{Heer waor zeker,.}{devaan tot ze sleep meh heer}{keek zelfs neet nao h\"a\"or}\\

\haiku{Tegeneuver h\"a\"om.}{beg\'os heer de contoure vaan}{de hoezer te zien}\\

\haiku{'t Jakkere, 't, ', ', '...}{mopperet vloket}{lachet keke}\\

\haiku{{\textquoteright} zag ze altied, {\textquoteleft}en,,.}{noe wie iech aajd weur w\`el iech}{ins lekker niks doen}\\

\haiku{Zoe get veur op te,......{\textquoteright} {\textquoteleft}?!}{fietse meh daan aanders eh}{Veur op te fietse}\\

\haiku{Z\`egk miech mer wienie ' '.}{tott uuch oetkump en iech}{maakt in orde}\\

\haiku{Menier Hellergers ' '.}{woordne gere gezene}{gas int krinkske}\\

\haiku{Edere kier es heer.}{in de k\`erk kaom m\'os heer}{weer denao kieke}\\

\haiku{{\textquoteright} Noe meint geer mesjiens tot.}{Antonius neet zoe d\`eks}{in die k\`erk kwaom}\\

\haiku{d'n heilege zien?}{tot de lui hei speciaol}{tot h\"a\"om koume beie}\\

\haiku{Meh es mien vrouw dat... '?}{oets ter oere kump Kinstet}{diech veurst\`elle}\\

\haiku{Beter tot heer 't.}{zeet es eine dee bij ze}{volle verstand is}\\

\haiku{Es iech zeker wis ',.}{tot zijt neet zouw hure}{daan heel iech m'ch st\`el}\\

\haiku{{\textquoteright} {\textquoteleft}Luuster ins hei,{\textquoteright} zag, {\textquoteleft}.}{menier noe hel-opiech haw}{neet vaan die smoesjes}\\

\haiku{{\textquoteright} {\textquoteleft}Iech geluif tot iech,{\textquoteright}.}{h\"a\"om h\"ob aosemde de jong}{in mevrouw h\"a\"or oer}\\

\haiku{{\textquoteright} Ze g\'ongen achter.}{h\"a\"om stoon en loerden euver}{z'ne wielvinger}\\

\haiku{Blaajer vlogen in de,,...}{runde tek braoke struuk}{woorte plat getrooje}\\

\haiku{{\textquoteleft}Iech weit 't neet... Iech...{\textquoteright},.}{meinde vaan wel zag ze get}{understebove}\\

\haiku{En tot 't ouch neet '.}{umn verkrachting g\'ong k\'os}{heer ouch direk zien}\\

\haiku{{\textquoteleft}Neet wijer loupe,,{\textquoteright}.}{neet korterbij koume zag}{heer autoritair}\\

\haiku{Dao hingk eine in,{\textquoteright}.}{de boum constateerde de}{wandeleer neuchter}\\

\haiku{Godmieljaar, wie die,.}{ineens opst\'ong dach ik tot}{ik wat krijge moes}\\

\haiku{Ze zagte ziech wel, {\textquoteleft},,{\textquoteright},.}{goojendaagmorge middag}{zoe meh wijer niks}\\

\haiku{Wie zij eindelek k\'os,.}{opstoon en nao boete}{g\'ong waor heer eweg}\\

\haiku{Ze had h\"a\"or m\`et nao '.}{binne genome en h\"a\"or}{opre sjoet gepak}\\

\haiku{Boete beg\'os 't.}{noe ech te regene en}{hel te wejje}\\

\haiku{H\"ob geer mesjiens get aon,?}{eur veuj tot geer zoe langsaam}{en umziechteg lop}\\

\haiku{Dee perfesser daan,.}{heh dee vert\`elde tot veer}{gans verkierd loupe}\\

\haiku{Dus noe h\"ob iech hei ' '.}{nen have kilo en dao}{nen have kilo}\\

\haiku{Heer kretsde ziech ins '}{m\`ett potloed op z'ne}{kop en bedach ziech}\\

\haiku{Meh... wiev\"a\"ol gram is, ',...?}{noe en noe kumpt wiev\"a\"ol}{gram is noe e{\'\i}n oons}\\

\haiku{{\textquoteleft}Wee zelf niks deit, moot.}{neet meine tot zij alles}{wel zal ranzjere}\\

\haiku{{\textquoteleft}Huurt ins, geer kint toch,?}{dat aajd vaan bij miech in de}{straot vaan Berens}\\

\haiku{{\textquoteleft}Iech h\"ob 't uuch toch,{\textquoteright}.}{gezag constateerde die}{vaan Janse kontent}\\

\subsection{Uit: Zal iech uuch ins get vert\`elle...}

\haiku{Jao, iech moot wel bij,,.}{d'n dokter zien veur mie bein}{meh iech h\"ob d'n tied}\\

\haiku{Doog ze miech ope in!}{h\"a\"ore nachjepon of zoe get.}{Noe vraog iech uuch}\\

\haiku{Heer staok ziene.}{kop weer nao binne en keek}{h\"a\"or trouwherteg aon}\\

\haiku{Ze had h\"a\"ore peignoir.}{aongetrokke en stoond angsteg}{nao h\"a\"om te kieke}\\

\haiku{Mesjiens bin iech wel de '{\textquoteright},.}{verkierde kant  aont}{op loupe dach heer}\\

\haiku{Iech blijf hei zitte, '{\textquoteright}.}{tot ze m\"orge opkump daan}{weet iecht zeker}\\

\haiku{En tot 't dao 'ne.}{gekkeboel waor en tot}{ze dao beer droonke}\\

\haiku{kinder, tot die gen\'og}{hadde aon get water en}{get zand of modder}\\

\haiku{{\textquoteleft}Ze begraove.}{de lui allewijl in hun}{zoondags pekske}\\

\haiku{Heer droonk ze glaas leeg,,, ':}{stoond op puunde Tiny gaof}{Maxn hand en zag}\\

\haiku{Bij de Chineze ',.}{m\'o\'oste sjus rupse naot ete}{Dat is dao beleef}\\

\haiku{De juffrouw zal toch,?}{ouch wel ins rupse of e}{puupke laote}\\

\haiku{M\`et ein hand in de.}{tes vaan ziene regejas}{leep heer nao de deur}\\

\haiku{{\textquoteright} Heer keek W\"ollems}{sjalleks aon en beg\'os}{toen te keke vaan}\\

\haiku{Heer lag e st\"okske {\textquoteleft}?}{broed op zien oetgestoke}{hand.Zeetd'r h\"a\"or nog}\\

\haiku{Meh d\'a\'an, en dat is,{\textquoteright}.}{zoe daan kin iech die alles}{liere wat iech w\`el}\\

\haiku{Die had iech oet e.}{n\`es gehaold wat oet de boum}{waor gevalle}\\

\haiku{In 't begin veel,.}{miech dat tege huur dat kin}{iech uuch verklappe}\\

\haiku{Allein jaomer tot.}{me begraffenisse neet}{zoe in de hand heet}\\

\haiku{t Geb\"a\"orde op ' ',.}{n nach naonen aovend wie}{de meiste nachte}\\

\haiku{{\textquoteleft}Jeh, ze zal ouch wel{\textquoteright},.}{klem begonne zien zag heer}{get ge{\"\i}rriteerd}\\

\haiku{M\`et al deen ambras.}{zouwe ze dat nog haos}{vergete h\"obbe}\\

\haiku{{\textquoteright}        Nuits {\textquoteleft}Noe valle,{\textquoteright},.}{d'ch toch d'n sjeun oet este}{dat huurs zag de vrouw}\\

\haiku{M\`et mier es doezend.}{doeje in Zuid-Afrika en}{vief daog rotweer}\\

\haiku{Iech waor te zier{\textquoteright}, '.}{verdeep in mie book zagr}{get veroontsj\"oldegend}\\

\haiku{Wie de zakemaan ' '.}{h\"a\"or zaog kraog heerne}{kop wiene pieper}\\

\haiku{Meh iech daon gein,.}{oug mie touw es die aon de}{geng geit nondepie}\\

\haiku{Op de gezoondheid.}{vaan mevrouw Dezems vaan de}{sjeettent neve miech}\\

\haiku{En... Geer zouwt ouch nao{\textquoteright},.}{de gemeinte kinne goon}{bedach ze obbins}\\

\haiku{{\textquoteleft}Zouwe veer effe '?}{n tas koffie goon drinke}{dao op dat terras}\\

\haiku{{\textquoteright} Noe stoond ze haaf op,.}{oet h\"a\"ore stool meh m\`et ein hand}{heel heer h\"a\"or tege}\\

\haiku{t Zouw h\"a\"om neet zoe.}{verbaas h\"obbe es ze hel}{waor goon keke}\\

\haiku{De juffrouw woord e,:}{bitteke raos in h\"a\"or}{geziech meh ze zag}\\

\haiku{'nen Awwere maan,.}{dee zien jong vrouw content m\'os}{zien te hawwe}\\

\haiku{Mesjiens es iech links ouch ', '.}{n tes had tot iechm daan}{toch rechs leet zitte}\\

\haiku{Wie Zjo tr\"okkaom aon ',.}{t t\"a\"ofelke keek de maan}{h\"a\"om oonzeker aon}\\

\haiku{Heer droonk zie glaas leeg.}{en best\`elde e rundsje}{veur de taofel}\\

\haiku{'r Waor mager '.}{gewoorde enr waor}{gans oet zienen doen}\\

\haiku{{\textquoteleft}Hiere, toen... toen g\'ong 'r... '...}{nao de kis lufde z'ne}{poet op en deegt}\\

\haiku{{\textquoteleft}Jeh, es d'r miech neet, '...... '{\textquoteright}.}{koelek numpt eht heet}{wel get eweg vaan uuch}\\

\haiku{ze allemaol:}{en Gabri\"el kaom nao v\"a\"ore}{en zag hiel serjeus}\\

\haiku{Ze g\'ong h\"a\"om veur nao ', '.}{n kamer boe ze h\"a\"omne}{lere fauteuil wees}\\

\haiku{'t Waor daobij. '.}{vochteg en k\`ellegt Zouw}{vas goon regene}\\

\haiku{Nondepie, die had!}{heer netuurlek in de b\"os}{laote ligke}\\

\haiku{hunne fort waor, '}{hadde zet toch veerdeg}{gesp\"a\"old veur Sjarel}\\

\section{Johan Brouwer}

\subsection{Uit: De schatten van Medina-Sidonia (onder ps. Maarten van de Moer)}

\haiku{Het eenige wat me.}{interesseert is de mensch}{en zijn problemen}\\

\haiku{Huichelachtige,.}{wereld die niet in zijn rust}{wil gestoord worden}\\

\haiku{Zij willen thuis dat,.}{ik rechten afstudeer en}{het staat me tegen}\\

\haiku{Morgen kom ik wel,.}{even aan dan praten wij nog}{wel over je studie}\\

\haiku{En zonder dat ik,.}{het goed besefte begon}{ik stil te schreien}\\

\haiku{Zooals het geval met,.}{die snuifdoos en andere}{kleine gevallen}\\

\haiku{U weet wel, die groote,,....}{student hoe heet hij ook weer}{iedereen zegt Zeus}\\

\haiku{Mijn besluit zou hun,.}{verdriet doen maar ik kon niet}{anders handelen}\\

\haiku{Dat moet dat plein zijn,.}{waar in de Julidagen}{zoo gevochten is}\\

\haiku{Stel je voor dat zij.}{nu ineens Barcelona}{gingen bombardeeren}\\

\haiku{Drie matrozen en,.}{een onderofficier met}{een heel jong vrouwtje}\\

\haiku{Op het perron in.}{Tarragona was het een}{drukte van belang}\\

\haiku{Een vreemdeling, een,.}{indringer temidden van}{een gezin in rouw}\\

\haiku{Het was een jonge,,.}{man een arbeider blijkbaar}{in blauwe overall}\\

\haiku{Zij hebben beiden,.}{hun einde gevonden zooals}{zij dat wenschten}\\

\haiku{Er gaan er te veel,.}{naar Madrid alleen om hun}{eindje te vinden}\\

\haiku{De wapens waren.}{gekomen en er waren}{auto's beschikbaar}\\

\haiku{De Aragonees nam.}{hem op en legde hem aan}{den kant van den weg}\\

\haiku{Twee lijken heb ik.}{half opgericht om te zien}{of ik het zelf was}\\

\haiku{Deze linie schijnt.}{al op enkele plaatsen}{verbroken te zijn}\\

\haiku{Paco trekt me mee,.}{een trap af een station}{van de metro in}\\

\haiku{Kogels kletteren.}{tegen de steenen of maken}{kleine pluimpjes stof}\\

\haiku{Ik ben zoo moe, dat.}{ik mijn geweer nauwelijks}{meer kan vasthouden}\\

\haiku{De schilderijen.}{schenen zeer gevoelig voor}{die verandering}\\

\haiku{De San Francisco{\textquoteright}.}{is onkwetsbaar had ik reeds}{vaak hooren zeggen}\\

\haiku{De fiches die ik,.}{invulde legde ik op}{groote tinnen schalen}\\

\haiku{Die Oudejaarsavond.}{was een kortstondige vlucht}{in de illusie}\\

\haiku{Voorop liep een man.}{met een leeren windjack aan en}{een zwart mutsje op}\\

\haiku{Bovendien was er,.}{nog een zeer groote som ook in}{gouden dubloenen}\\

\haiku{Ik zag haar ook, een,.}{enkele maal zonder dat}{mij dat verschrikte}\\

\haiku{Dit probleem scheen hem.}{meer te interesseeren}{dan mijn proefneming}\\

\haiku{Zij gaf mij echter,.}{de hand zonder dat zij mij}{scheen te herkennen}\\

\haiku{dat ik wel Russisch,.}{kende maar redenen had}{dit te verzwijgen}\\

\haiku{Zelfs verscheidene.}{leden van de Alianza}{zaten gevangen}\\

\haiku{Er werd gesproken.}{van een dreigenden opstand}{in Barcelona}\\

\haiku{Die psychiater.}{had al in weken niets van}{zich laten hooren}\\

\haiku{Schmidt, Iwanow, of een,.}{van de anderen die weet}{dat ik alleen ben}\\

\haiku{- Ik geloof, dat er.}{een middel is om je van}{Iwanow te bevrijden}\\

\haiku{De psychiater.}{ontving me in het bureau}{van de inspectie}\\

\haiku{Ik heb ze zelf niet,.}{gezien ik heb er alleen}{over hooren spreken}\\

\haiku{- Hoeveel tijd zal er?}{noodig zijn voor het invullen}{van alle stukken}\\

\haiku{De jongen nam een.}{steentje en raakte de geit}{boven op den kop}\\

\haiku{Nadrukkelijk zei.}{ik hem dat die ontdekking}{geheim moest blijven}\\

\haiku{- Neen, er staat een groote,.}{steen voor maar daarachter moet}{nog een ruimte zijn}\\

\haiku{Wij kwamen in een,.}{vertrek dat iets kleiner was}{dan het andere}\\

\haiku{Zij voelden zich fier,,,.}{zij voelden zich soldaten}{strijders krijgslieden}\\

\haiku{Waar kwam ik met dien?}{doode vandaan en waar ging}{ik er mee naar toe}\\

\haiku{Ik wist slechts dat ik,.}{daar aankwam en anderen}{mijn last overnamen}\\

\haiku{Zijn lichaam scheen te,.}{ontspannen zijn hoofd zonk nog}{zwaarder op mijn arm}\\

\haiku{Terwijl ik wegzonk.}{werden groote stukken steen over}{mij heen geworpen}\\

\haiku{Zij hebben de zorg.}{voor de bibliotheek en}{de kerksieraden}\\

\haiku{Dertien steeds zwakker.}{wordende lichten waren}{dertien koningen}\\

\haiku{Zes rijen van vier.}{boven elkaar en nog twee}{boven den ingang}\\

\haiku{In wezen zal het.}{dan gedaan zijn met Spanje's}{zelfstandig bestaan}\\

\haiku{- Ik geloof, dat diep.}{in zijn hart niemand in zijn}{eigen dood gelooft}\\

\haiku{Het gevaar was een.}{wezenlijk element van het}{bestaan geworden}\\

\haiku{Men ranselde hem,.}{gruwelijk af en men nam}{het crucifix weg}\\

\haiku{Het Escuriaal{\textquoteright}.}{is het uitspansel van den}{geest van Philips II}\\

\haiku{Ik herinnerde.}{hem aan ons voorgenomen}{onderzoek aldaar}\\

\haiku{Het oorlogsgeweld,.}{kan hier slechts een dichterlijk}{een episch motief zijn}\\

\haiku{De veelvuldige.}{bombardementen waren}{als natuurrampen}\\

\haiku{Deze opening bleek.}{groot genoeg te zijn om een}{man door te laten}\\

\haiku{Er moest echter een;}{gemakkelijker toegang}{tot den Toren zijn}\\

\haiku{Zij onderzochten.}{op hun beurt de muren en}{den grond nauwkeurig}\\

\haiku{- Ik had een holte,.}{onder den grond verwacht zei}{de psychiater}\\

\haiku{Wij moeten probeeren.}{of deze steen ook zulk een}{mechaniek bezit}\\

\haiku{- Het wordt tijd dat nu.}{eens een rustig man hier de}{spa in den grond steekt}\\

\haiku{Het trof ons dat de.}{steen zelf en de aarde er}{om heen vochtig was}\\

\haiku{De zerk kwam een paar,.}{centimeter omhoog en}{brak toen midden door}\\

\haiku{Wij zijn reeds buiten.}{die magische sfeer van het}{gevaar gekomen}\\

\haiku{Mijn militaire.}{dienst was in Nederland een}{geheim gebleven}\\

\haiku{Ik zie de grenzen.}{niet tusschen mijn droomen en}{de werkelijkheid}\\

\section{Carry van Bruggen}

\subsection{Uit: Avontuurtjes}

\haiku{zoo hoog over heen en.}{jaagt het als in stukken naar}{alle kanten weg}\\

\haiku{en ze verbaast zich.}{dat ze er anders nooit aan}{denkt en het nooit voelt}\\

\haiku{hij zal toch niet met....?}{dien wild-vreemde over hun}{kwitantie spreken}\\

\haiku{maar in de keuken,....}{is nog warmte is nog lucht}{van eten en koffie}\\

\haiku{dat doet de oude,....}{als het vacantie is maar}{deze weet van niets}\\

\haiku{De dokter is nog,}{niet thuis gekomen ze weet}{het al voor de meid}\\

\haiku{Ze wil het beven,....}{van haar mond bedwingen want}{ze moet het zeggen}\\

\haiku{anders zou het haar....}{wel uit de verte tegen}{het schrikding helpen}\\

\haiku{Naar wat je zoo hoort,:}{om je heen bestaan er zes}{soorten van menschen}\\

\haiku{Ze kijkt gauw weer voor,,.}{zich uit want Kleij komt langs het}{pad haar achter-op}\\

\haiku{rillingen ritsen....}{over haar hoofd en haar oogen zijn}{ineens vol tranen}\\

\haiku{Zij hebben niet zoo -;}{heel goed geluisterd en niet}{zoo heel veel verstaan}\\

\haiku{Ze sluipt op haar teenen,,,....}{wrikt de grendels voorzichtig}{hun kokertjes in}\\

\haiku{vlak daarnaast woont Brons,,}{hij is een vriendelijke}{oude weduwnaar}\\

\haiku{Ze merkt ineens haar,.}{eigen loopen het loopen}{wordt er anders door}\\

\haiku{Nu moeten ze even,.}{zitten ze loopend op te}{eten zou zonde zijn}\\

\haiku{Je ziet de menschen,....}{al aan het dek je hoort de}{boot stampen en stoomen}\\

\haiku{alles aan hem wordt....}{elke seconde grooter}{en duidelijker}\\

\haiku{Het is louter voor,,!}{de grap dat Vader zoo praat}{dat Vader zoo doet}\\

\haiku{Ze is de deur uit}{en naar de kermis gedraafd}{en nooit leek de weg}\\

\haiku{en het lijkt wel of,....}{hij nu echter lacht maar dat}{kan natuurlijk niet}\\

\haiku{hij is niet kwaad en,....}{niet norsch alleen maar ruig}{en donker en oud}\\

\haiku{gelukkig voor die,,.}{lange bleeke vrouw want hij is}{natuurlijk haar man}\\

\haiku{ze opent ze weer, en.}{rondom isde volheid van}{het lachende licht}\\

\haiku{ze zoeken elk en,.}{vinden ze een voor een elk}{op een vreemde plek}\\

\haiku{Ze zitten een heel,....}{eind hooger dan de straat dat}{is alleen maar hier}\\

\haiku{hij is de eerste....}{en eenige die ze ooit in}{het speelveld vonden}\\

\haiku{je hoort ze niet, nu,.}{liggen ze op de hoop die}{al gevallen was}\\

\haiku{Men kijkt niet meer, naar,.}{Aap of Beer Maar alleen wij}{genieten de eer}\\

\haiku{en ook {\textquoteleft}Neen, mis{\textquoteright} en!}{dan even wachten en dan zingt}{de heele zaal mee}\\

\haiku{acht Want ze wisten,.}{van hun baboe dat je daar}{altijd krijgt wat toe}\\

\haiku{Maar moeder heeft hem {\textquoteleft}.}{al zoo lang en noemt hemmijn}{akker en mijn ploeg}\\

\haiku{en zij stond midden....}{in de kamer en keek hem}{aan terwijl hij las}\\

\haiku{acht naadjes boven,!}{je teekje en als het mooi}{is zelfs wel eens tien}\\

\haiku{De jongens staan er,....}{nog en lijken bijna in}{het riet verzonken}\\

\haiku{de witte streep breekt,....}{uit zijn bruin gezicht dat was}{zooeven strak en dicht}\\

\haiku{de mannen en de,....}{zonen hun praten kaatst in}{het kale licht}\\

\haiku{die boer dacht Heilbron,....!}{te bedotten maar Heilbron}{was dien boer te slim}\\

\haiku{, voelt zacht, doet zoete.}{schommeling door je leden}{gaan en in je keel}\\

\haiku{Neen, het jaar is kaal,....}{en zwart want Simchas-touro}{was de laatste bloem}\\

\haiku{van hoog uit valt de,}{stijve straal die is als een}{ijzen staaf en boort}\\

\haiku{Er was een orgel,,....}{er kwam een bedelman en}{moeder zette thee}\\

\haiku{Het duurt nooit lang, maar,.}{het moet zijn eigen beloop}{hebben als een gaap}\\

\haiku{Dat vroeg hij en ze,.}{moest er zelf om lachen zoo}{grappig als het klonk}\\

\haiku{en als het niet lukt,........}{dan kun je toch nog altijd}{doen alsof het lukt}\\

\haiku{Neen, ze zat roerloos,,}{neer ze gaf geen geluid want}{ze wil dat hij denkt}\\

\haiku{lukt het niet, krijg je,,.}{straf is dat voorbij denk je}{er ook niet meer aan}\\

\haiku{Zij-zelf hebben....}{thuis het eene kistje van Oom}{Elie nog bijna vol}\\

\haiku{En Sabbathisten,:}{gedenken den Sabbath want}{er staat geschreven}\\

\haiku{schemerig schommelt,.}{zijn gebogen lijf zijn stem}{komt van onder op}\\

\haiku{grijs het water, het:}{steigertje en levend als}{van stekelbaarsjes}\\

\haiku{Je zou bang kunnen,....}{zijn maar niet met Vader en}{niet met den schipper}\\

\haiku{en bijna wist ze....}{wat er dreef in het water}{en wat er zoo rook}\\

\haiku{ze hoeft zich niet zoo,....}{bang te maken zij is toch}{zelf niet op het schip}\\

\haiku{Het schip heeft er dus,!}{zeker niets mee te maken}{maar w\`el zij-zelf}\\

\haiku{het mag wel.... alleen,.}{er zijn zoo maar in de week}{geen appels in huis}\\

\haiku{hij vergeet haar, laat,....}{haar in den steek is met zijn}{vriendjes gaan spelen}\\

\haiku{je zusje van de,....}{trappen te duwen ze had}{wel dood kunnen zijn}\\

\haiku{je loopt erin, je,}{kunt ertegen vechten je}{hebt licht en menschen}\\

\haiku{{\textquoteright} en iets in je wordt....}{dadelijk glad en warm naar}{omlaag gestreken}\\

\haiku{wiekt enkel nog maar,....}{op schemerige grijze}{vlerken om haar heen}\\

\haiku{Of ze denken, dat,.}{Vader wel graag wil maar niet}{kan schaatsenrijden}\\

\haiku{waarom, want je noemt,....}{het rijden maar je kan het}{ook loopen noemen}\\

\haiku{H\`e.... kan nog iemand?}{anders zoo als Vader in}{de handen klappen}\\

\haiku{en van dien kant af.}{heb jullie nog nooit  in}{den trein gezeten}\\

\haiku{Maar liever niet, ze.}{staat er liever naast en houdt}{haar hand om den rand}\\

\haiku{je loopt en voelt geen,,....}{straat je zolen slapen zelf}{slaap je bijna ook}\\

\haiku{het ijs verlaten,,.}{dichtbij en ver de menschen}{allemaal naar huis}\\

\subsection{Uit: Een coquette vrouw}

\haiku{Ze was immers niet -!}{ziek en ze had hem zelf niet}{gevraagd te komen}\\

\haiku{Iedereen hier is.}{bedaard en beschaafd en heeft}{goede manieren}\\

\haiku{Het tegen deel van.}{alles wat ik mij voorstel}{dat een man moet zijn}\\

\haiku{Ik weet ook niet of -.}{u het mij zeggen mag maar}{u zegt het mij wel}\\

\haiku{En dat wist u dus,?}{allemaal precies toen u}{hier de trap op kwam}\\

\haiku{Ze wist allang van -,.}{niet en toch verwachtte zij}{het steeds weer opnieuw}\\

\haiku{- wat een kracht, wat een.}{hoogheid beduidde zooveel}{zelfverzekerdheid}\\

\haiku{Hij kuste haar weer,,,.}{maar ze verzette zich half}{in ernst half schertsend}\\

\haiku{ik zoo weinig voor.}{familie en heelemaal}{niets voor portretten}\\

\haiku{Sinds de geboorte;}{van het kind kwam Geerte weer}{vaker dan vroeger}\\

\haiku{Egbert trommelde,.}{met een vouwbeen op tafel}{glimlachte en zweeg}\\

\haiku{{\textquoteright} ijverde Geerte, {\textquoteleft}.}{er komen heel aardige}{en geschikte lui}\\

\haiku{Vooral als ze al,.}{haar vriendjes hier vindt om haar}{het hof te maken}\\

\haiku{Nu is het gauw uw,{\textquoteright}, {\textquoteleft}?}{beurt zei hijmag ik hier op}{u blijven wachten}\\

\haiku{Ze kon ook nu niet.}{plotseling weigeren met}{hem aan te zitten}\\

\haiku{{\textquoteleft}Ze is overspannen,{\textquoteright}, {\textquoteleft}.}{vond dieze zou eens een poos}{naar buiten moeten}\\

\haiku{natuurlijk vond, dat.}{ze elkaars bijzijn niet meer}{verdragen konden}\\

\haiku{wie kon die vreemde,?}{wezen met zijn beschaafde}{welluidende stem}\\

\haiku{Het is heelemaal.}{geen deugd of edelaardigheid}{of iets van dien aard}\\

\haiku{{\textquoteleft}{\"\i}k kan anders niet,.}{zeggen dat die waschbaas}{van jou mij bevalt}\\

\haiku{Ze was dien middag.}{bij Paul geweest en hij had}{den tijd vergeten}\\

\haiku{En ik heb je maar -,}{\'e\'en ding te zeggen dan kan}{je daarna razen}\\

\haiku{het echte geluk.}{van die anderen en hun}{eigen schijngeluk}\\

\haiku{Maar ik zal in elk,.}{geval wel zorgen dat je}{wat fatsoenlijks krijgt}\\

\haiku{Als ik een man was,,,.}{zou ik vrouwen geloof ik}{heel aardig vinden}\\

\haiku{Alle liefde en.}{alle vertrouwelijkheid}{waren immers \'e\'en}\\

\haiku{Van je-zelf, en,.}{van de heele wereld het}{geheele heelal}\\

\haiku{je kan wel zooveel{\textquoteright},.}{willen of een dergelijk}{onzinnig antwoord}\\

\haiku{{\textquoteright} riep Ina, bevend van,, {\textquoteleft}?}{drift met trillende lippen}{kan die nog huilen}\\

\haiku{Ze zouden haar niet,.}{langer zwak ze zouden haar}{niet in tranen zien}\\

\haiku{{\textquoteright} De andere vrouw.}{boog zich naar Paul over en trok}{hem even aan het oor}\\

\haiku{In het begin van:}{het jaar hadden ze nieuwe}{buren gekregen}\\

\haiku{Wij hebben meer dan,,,!}{het gemiddelde Egbert}{veel meer geloof ik}\\

\haiku{Jij zag den jongen,.}{niet jij zag alleen jezelf}{en wat je noodig had}\\

\haiku{{\textquoteright} {\textquoteleft}Het is heerlijk, te{\textquoteright} -,:}{spreken met weinig woorden}{zei Ina zacht en dan}\\

\haiku{Ik wil een thuis -, ik,.}{wil een vrouw ik wil met rust}{gelaten worden}\\

\haiku{{\textquoteleft}Ja -, Charley sprak over {\textquotedblleft}{\textquotedblright},.}{tweestroomingen in ons}{zooals ze het voelde}\\

\haiku{Ze zouden hem zelf -,.}{terugbrengen hij zal zoo}{meteen wel komen}\\

\haiku{{\textquoteright} zei Josefine.}{met een kleur van pijnlijke}{verontwaardiging}\\

\haiku{Kijk -, ik neem het voor,.}{Otto mee het zal hem wel}{interesseeren}\\

\haiku{En daarom doen mij,.}{die dingen zoo'n pijn die ze}{nu van je zeggen}\\

\haiku{Vertel eens even, weet,?}{Annie dat je naar mij toe}{bent en dat ik kom}\\

\haiku{Het gaat mij niet aan -,.}{en het interesseert mij}{eigenlijk ook niet}\\

\haiku{En de dames d\'a\'ar.}{verlangen blijkbaar heel erg}{naar uw terugkomst}\\

\haiku{Zoo kon het nog uren,,.}{duren den geheelen dag}{den ganschen nacht door}\\

\haiku{Wandelen -, alleen,,?}{in de mist in de kilte}{en daarna naar huis}\\

\haiku{{\textquoteright} vroeg Ina, {\textquoteleft}ik dacht een,.}{oogenblik aan wat anders}{ik verstond je niet}\\

\haiku{Die groote blonde vrouw,,?}{weet je wel die je kerstavond}{bij mij ontmoet hebt}\\

\haiku{Arnold heeft plaatsen,.}{voor een nieuwe operette}{dus wij zijn niet thuis}\\

\haiku{Waar had ze zich in -,?}{vastgewerkt hoe moest ze er}{zich weer uitredden}\\

\haiku{{\textquoteright} {\textquoteleft}Vertel eerst eens wat,{\textquoteright} {\textquoteleft}.}{van je zelf verzocht Charley}{die reis komt straks wel}\\

\haiku{{\textquoteright} {\textquoteleft}Van mezelf,{\textquoteright} zei Ina,, {\textquoteleft}}{mat en schouderophalend}{ze keek Charley aan}\\

\haiku{Hij plukte alleen.}{bloemen om ze iemand te}{kunnen aanbieden}\\

\haiku{de formule zoo,.}{zou hem niet erg aanstaan maar}{het is zijn praktijk}\\

\haiku{dit was het einde,.}{van haar laatsten droom van haar}{jongste illusie}\\

\haiku{Het woord klonk dof -, de.}{gedachte zette haar hart}{niet langer in gloed}\\

\subsection{Uit: Eva}

\haiku{Je hebt gewacht, je.}{hebt er een plaats in jezelf}{voor open gehouden}\\

\haiku{Ik zit nu op een,,.}{toren er is geen vuil dat}{zoo hoog spatten kan}\\

\haiku{Je hebt het ook met -,.}{wie het niet verdienen met}{den boozen padrone}\\

\haiku{een beetje beschaamd,.....}{een beetje verslagen en}{een beetje getroost}\\

\haiku{En zou er nu zoo,....}{iets bestaan als een zaad waar}{alles in klaar ligt}\\

\haiku{en je ziet,.... blauwe,....,....}{hemel blauw water gele}{bloemen bloeiend gras}\\

\haiku{In glinsterkringen.}{en zilverig schuim staat het}{kortgesneden riet}\\

\haiku{En de eerste dag,,!}{dien ze langs hun oogen voorbij}{zagen gaan zij zelf}\\

\haiku{Achter ze is de,,....}{lange donkere kap als}{een drukkende hand}\\

\haiku{Elken dag weer maakt,.... '}{haar de stad tot het zijne}{neemt haar in bezit}\\

\haiku{links en rechts een snoer.}{van gouden tientjes boven}{den donkeren stroom}\\

\haiku{Ik kan al niet eens,....}{goed tegen gymnastiekles}{om de commando's}\\

\haiku{Dus dat je je niet,....}{wou laten onderzoeken}{dat begrijp ik best}\\

\haiku{Reuk, laat mij los, plaag -,.}{mij niet om een naam ik kan}{hem je niet geven}\\

\haiku{o, maar je mag zoo,.}{niet kijken zoo onder je}{wenkbrauwen omhoog}\\

\haiku{later werd het een,....}{walgelijk woord  omdat}{er altijd iets van}\\

\haiku{tusschen buurman Bol,,.}{en buurman Bruin daar woont ze}{daar wordt ze verwacht}\\

\haiku{Maar kun je helpen {\textquoteleft}{\textquoteright}.}{dat je weet watperoe}{oe-rewoe beduidt}\\

\haiku{Ze zit achteruit,.}{in haar stoel laat het lamplicht}{in haar oogen bijten}\\

\haiku{Neen, van dat eene Hoofd,, {\textquoteleft}}{dien aardigen heer die zoo}{nadrukkelijk zei}\\

\haiku{hij wil voor alle,....}{examens later werken waar}{David nu voor werkt}\\

\haiku{Het is, als kwamen,.}{ze van een verre reis als}{waren ze vermoeid}\\

\haiku{Vergeet het nu even,,.}{leg het van u af dat u}{Grootvaders zoon bent}\\

\haiku{wrevelig knarsen.}{zijn zware voeten over den}{zanderigen vloer}\\

\haiku{Was dit staan over een?}{kom met rozen gebogen}{dan weer zoo iets geks}\\

\haiku{hij zei iets tegen,,....}{Selien hij lachte gooide}{zijn hoofd achterover}\\

\haiku{Even staat Joppe tot,.}{stilte beschaamd maar dat even}{is alweer voorbij}\\

\haiku{de oude, felle....}{oogen boren zoo recht en zoo}{diep de zijne in}\\

\haiku{een zee, zwart-grauw,.}{uit een lichtende toren}{met melk overgoten}\\

\haiku{dat ze je gevraagd,.}{heeft bij haar en Dora te}{komen inwonen}\\

\haiku{dat hij met Joop op,....}{schoot in het kamertje heeft}{gezeten dat hij}\\

\haiku{Het schijnt dat hij thuis,,:}{op niets meer aan niemand meer}{antwoord geeft hij zegt}\\

\haiku{Het is alsof je....}{een man den brand zag steken}{in zijn eigen huis}\\

\haiku{Ze zeggen alles,.}{verschillend ze bedoelen}{alles verschillend}\\

\haiku{daar stond Herman in,,.}{de deur met zijn guitaar met}{zijn nieuwe liedjes}\\

\haiku{{\textquoteright} Ernestien, zeg het,,.}{nu maar vraag het nu maar dan}{is het geleden}\\

\haiku{Leun roerloos tegen,........}{het regenraam sta stil in}{de stilte en wacht}\\

\haiku{het is ongepast,....}{en het is je geluk dat}{je maar alleen bent}\\

\haiku{{\textquoteright} Het laatste dat ze,,:}{Ernestien heeft geschreven}{was op een briefkaart}\\

\haiku{Ebner, arme man,,,....}{eenzame man oude man}{kom naar mijn lippen}\\

\haiku{boven hun hoofden....}{kwam Hugo zijn klas in en}{stemde zijn viool}\\

\haiku{met een enkelen,,....}{lach een handwuif achterom}{een blinkenden blik}\\

\haiku{d{\`\i}t prijsgeven van,,....}{jezelf is het grootste het}{mooiste het hoogste}\\

\haiku{En 's avonds kwam hij,....}{je halen en je praatte}{samen in het park}\\

\haiku{wijd-open word je....}{getrokken en er stijgen}{geruischen op}\\

\haiku{dit deeltje Hauptmann....,....}{en kijk hij houdt zooveel van}{zoute bolletjes}\\

\haiku{het blaasje richtte,,........}{zich op het zette uit het}{zwol en sloeg uiteen}\\

\haiku{ze kennen alleen,,....}{samen het eene de eene en}{dat kent hen allen}\\

\haiku{Het moet zoo geweest,,.}{zijn of jij waart hier nu niet}{met mij in mijn schoot}\\

\haiku{Krankzinnig zijn we,,,....}{immers wij menschen met den}{dood overal overal}\\

\haiku{David was na die.}{ziekte van voorverleden}{jaar nooit meer gezond}\\

\haiku{hier een man, die een....}{klein meisje vermoordt voor haar}{gouden belletjes}\\

\haiku{{\textquoteleft}Pas maar op, dat je{\textquoteright}.}{niet met paraplu en al}{de hoogte in gaat}\\

\haiku{{\textquoteright} {\textquoteleft}Nu ga je dan ook,,.}{allebei voor straf onder}{den mantel vandaan}\\

\haiku{Het sterft je af -, maar,.}{zoo is het niet je laat het}{achter op je weg}\\

\haiku{Zij zijn daarvoor nog....}{niet eens dicht genoeg bij het}{witte vuur geweest}\\

\haiku{Daar staat Heleen -, ze.}{heeft mij al gezien en ze}{weet dat ik Eva ben}\\

\haiku{maar eens hebben we....}{een eruit gevischt en mee}{naar huis genomen}\\

\haiku{Toen leek het ons ver,,....}{maar het is niet zoo ver we}{gaan de haven langs}\\

\haiku{Hij heeft op David,....}{geparasiteerd hij heeft}{hem uitgezogen}\\

\haiku{En dat David als....?}{dichter begaafd was heb jij}{dat ooit geweten}\\

\haiku{En David sprak het,,.}{ook niet tegen uit trots uit}{onverschilligheid}\\

\haiku{het staat buiten de.}{intellectueele en de}{ethische problemen}\\

\haiku{Ze wilde mij den,.}{brief in mijn handen geven}{maar ze bedacht zich}\\

\haiku{Ook de drift naar den,,,!}{Plicht de inhoudlooze doellooze}{redelooze plichtdrift}\\

\haiku{Omdat het ons zoo, -!}{aangrijpt waar het ons aanraakt}{als het onszelf raakt}\\

\haiku{Wonderlijk bestaan,,.}{waarin juist dit verzwegen}{wordt vergeten wordt}\\

\haiku{Maar laat het zoo zijn,.}{laat het kunnen dat de een}{voor den ander boet}\\

\haiku{Jaap en Ben deinsden,....}{ervoor terug en toen was}{Eddy geboren}\\

\haiku{Want eerst ga ik nu....}{naar Vader en Moeder en}{straks ga ik naar huis}\\

\haiku{Dat we er trotsch op,?}{zijn dat we ons haast schamen}{als het anders is}\\

\haiku{als het hunne in {\textquoteleft},{\textquoteright}....}{de spanning van hetHerder}{laat je schaapjes gaan}\\

\haiku{Ik liep zoo veilig,,:}{ik liep vlak achter Eddy}{en Claartje ik dacht}\\

\haiku{{\textquoteleft}Ik sta ervan te -, -,}{kijken ik hield jullie voor}{gelukkig getrouwd}\\

\haiku{{\textquoteleft}Kunnen we niet eens,,....?}{samen een heelen morgen}{een heelen middag}\\

\haiku{Misschien zoo ver zelfs,}{nog niet ik ben zoo snel van}{daar naar hier komen}\\

\haiku{Maar menschen dienen.}{t\`och behoorlijk aan elkaar}{te zijn voorgesteld}\\

\haiku{Je loopt sinds lang de,.}{kermistent voorbij je lacht}{om de muizenval}\\

\haiku{{\textquoteright} {\textquoteleft}De allergrootste,,.}{dwaasheid op die eene na op}{die andere na}\\

\haiku{maar in mijn begrip,.}{kon ik het toch niet binden}{nooit kreeg het een zin}\\

\subsection{Uit: Goenong-Djatti}

\haiku{Goenong-Djatti ().}{Een Indische roman door}{Carry van Bruggen}\\

\haiku{{\textquoteright} {\textquoteleft}Ja, zeker{\textquoteright} zei De, {\textquoteleft}.}{Klerk met overtuigingzeker}{ben ik gelukkig}\\

\haiku{Maar wie is d'r ten....}{slotte verantwoordelijk}{voor de directie}\\

\haiku{en me daar, nou ja, '....}{laatk zeggen amuseerde}{op plantersmanier}\\

\haiku{En dan voor alle, '.}{gerustheid de wachters}{maars waarschuwen}\\

\haiku{Amelie droeg een wit.}{japonnetje met weidsche}{lichtroode strikken}\\

\haiku{{\textquoteleft}dat je geheimen,}{ni\`et bewaard worden daar}{zorg je zelf wel voor}\\

\haiku{Kijk, haar tjelana.......}{kleeft haar om de beentjes laat}{nonnie nu zoet zijn}\\

\haiku{{\textquoteright} 'n Ommezientje ' '.}{wast dan stil en daarna}{t kind weer terug}\\

\haiku{Goenong-Djatti ().}{Een Indische roman door}{Carry van Bruggen}\\

\haiku{nu liep het tegen.}{den avond en nog waren ze}{niet geheel gereed}\\

\haiku{En ze had juist zoo,,...;}{graag gehad dat hij nu voor}{Charlotte  voor\'al}\\

\haiku{'t Nonnaatje-zelf, '.}{was er ook ontsteld van met}{n zweem van schaamte}\\

\haiku{{\textquoteright} Enkelen, die de,, '.}{vraag gehoord hadden wachtten}{benieuwdt antwoord}\\

\haiku{{\textquoteright} De bediende bood,.}{gekookte tongen Nelly's}{delicatesse}\\

\haiku{En daarna reden.}{met traag en droog grintgeknerp}{de andere voor}\\

\haiku{Vader Hans geeuwde.}{en Charlotte was al naar}{haar kamer gegaan}\\

\haiku{Zij-zelf was toen,,.}{ondanks haar vermoeidheid ook}{wakker gebleven}\\

\haiku{Ze dorst 't maar aan,, '.}{de oude dame en dat}{in zoon hitte}\\

\haiku{{\textquoteright} 't Nonnaatje zweeg, met,.}{een schouder-schok en er}{was even een pauze}\\

\haiku{{\textquoteright} {\textquoteleft}Ja, kind, vraag liever, -.}{wie er n{\`\i}et waren van de}{lui uit de buurt dan}\\

\haiku{En dan Hemming en,....}{Rutgers en De Leur en de}{dokter en Van Twist}\\

\haiku{En Wiesje kwam, van,}{den badkamerkant met een}{te midden van al}\\

\haiku{Amelie moest nou niet '.}{doen oft heelemaal uit}{de lucht kwam vallen}\\

\haiku{Hij wandelde door,.}{hooge zalen met portretten}{tegen de muren}\\

\haiku{Ze nam hem bij de,.}{hand en ze wandelden langs}{het water samen}\\

\haiku{toen scheurde hij 't,.}{vodje in kleine stukjes}{en strooide die rond}\\

\haiku{Goenong-Djatti ().}{Een Indische roman door}{Carry van Bruggen}\\

\haiku{En mevrouw Baarslag, van {\textquotedblleft}{\textquotedblright}....}{Kalipoeti heeft me al}{zoo dikwijls gevraagd}\\

\haiku{De Klerk had zich, als,.}{de anderen neergelegd}{om wat te slapen}\\

\haiku{ze was bij mevrouw.}{Van Houweningen maar heel}{koeltjes ontvangen}\\

\haiku{Hij sprak er nog niet,.}{over er was een andere}{oplossing denkbaar}\\

\haiku{Die hem 't geschikst,.}{leken noteerde hij even}{in z'n zakboekje}\\

\haiku{Meer had ze dan ook,}{niet te vorderen ze had}{vooruit geweten}\\

\haiku{Kolff tuurde weer neer,.}{naar z'n sigaar meer broze}{asch dan tabak nu}\\

\haiku{Wies was van morgen,.}{vroeg al weg uit spelen bij}{de meisjes Schaarbeek}\\

\haiku{Kijk nou 's naar de {\textquoteleft}{\textquoteright}, '.... '}{rubriekStadnieuws d\`at is dan}{tocht voornaamste}\\

\haiku{De krees waren nog,.}{niet neer maar heel gauw zou dat}{toch wel noodig worden}\\

\haiku{{\textquoteright} {\textquoteleft}H\`e nee, Amelie{\textquoteright} zei, {\textquoteleft}.}{Nel even booszeg niet z\`ulke}{verdachtmakingen}\\

\haiku{Al h\'e\'el vroeg gegaan, '.}{om voort ontbijt terug}{te kunnen wezen}\\

\haiku{{\textquoteright} {\textquoteleft}Ja, ik kan er niet,,.}{mee dwepen met dat erge}{blonde dat witte}\\

\haiku{en een aardig paar, '.}{zij een snoezig vrouwtje en}{hijn knappe man}\\

\haiku{{\textquoteright} {\textquoteleft}Dat kun je denken{\textquoteright},, {\textquoteleft}.}{spotte Ameliena zoo een}{hopelooze liefde}\\

\haiku{hij laat zich nogal, '.}{w\`el leiden maar je moetm}{zoo alles zeggen}\\

\haiku{Ik geloof, dat zoo.}{een man een ellendigen}{werkkring moet hebben}\\

\haiku{Nu ga ik heusch. '.}{t Zal buiten toch wel al}{om te stikken zijn}\\

\haiku{al naar dat we een '.}{geschikte hut vinden op}{n prettige boot}\\

\haiku{Die Charlotte van.... {\textquotedblleft}{\textquotedblright}?}{der Hoef blijft ze nog lang op}{Goenong-Djatti}\\

\haiku{En dan verbaasde.}{ze zich ook ineenen over haar}{eigen aarzeling}\\

\haiku{{\textquoteright} {\textquoteleft}O{\textquoteright} zei die, koeltjes, {\textquoteleft}.}{om mij hoef je niets te doen}{en niets te laten}\\

\haiku{Goenong-Djatti ().}{Een Indische roman door}{Carry van Bruggen}\\

\haiku{{\textquoteright} Charlotte bloosde.}{en verwarde zich in een}{ontwijkend antwoord}\\

\haiku{Ja, je weet, je voelt,{\textquoteright}}{wel dat ik me erg tot je}{aangetrokken voel}\\

\haiku{Maar van Singapore... weet....}{je wat d\`a\`ar m'n amusantste}{indruk geweest is}\\

\haiku{{\textquoteright} zei hij, {\textquoteleft}heerej\'e,,.}{wat een zieken ineenen wat}{een ontsteltenis}\\

\haiku{'k Wou alleen even ',.}{naart hospitaal de lui}{weten er van niets}\\

\haiku{Ik was vroeg-op.}{vanmorgen en alles komt}{nu ook tegelijk}\\

\haiku{Ze had zoo tekeer,.}{gegaan toen de oppassers}{d'r kwamen halen}\\

\haiku{een aangename,}{afwisseling als je zelf}{leefde van geluk}\\

\haiku{iedereen was nu}{even zenuwachtig en}{gespitst op gerucht}\\

\haiku{ze fluisterden nu...}{in de bijgebouwen over}{wat er was geschied}\\

\haiku{Dan is-ie ook wel,.}{uitgeput en kun je zien}{ho\`e die erin zit}\\

\haiku{en hij begreep, dat,.}{er iets erger moest zijn iets}{re\"eelers vooral}\\

\subsection{Uit: Heleen: een vroege winter}

\haiku{Van het huis-zelf.}{en de oude meubels hield}{ze bijkans evenzeer}\\

\haiku{Bijkans elken dag.}{stond Heleen daar en zag er}{dezelfde boeken}\\

\haiku{zoolang zij maar in,,.}{huis was konden ze loeren}{zooveel ze wilden}\\

\haiku{HELEEN werd twaalf jaar}{oud en kende van het woord}{vertrouwelijkheid}\\

\haiku{Heleen wist het niet,.}{draaide ongedurig in}{haar bank en zuchtte}\\

\haiku{Ze wist het niet, doch.}{immer dringender klopte}{de vraag bij haar aan}\\

\haiku{Ze hadden het lang,.}{geweten doch gingen het}{nu pas beseffen}\\

\haiku{Maar waarom dan.... en..........}{waaruit dan de \'e\'en zoo en}{de ander anders}\\

\haiku{Een enkele plek.}{in haar was door overgroei tot}{ouderdom gerijpt}\\

\haiku{Want waarom zou ze?}{van nature beter dan}{anderen wezen}\\

\haiku{een eeuwige wet.}{in de eindeloosheid der}{openbaringswijzen}\\

\haiku{- haar innerlijke,}{zorgen namen haar geheel}{en al in beslag}\\

\haiku{Wat ze toen zag, was,.}{de cirkel waarbinnen ze}{voortaan dolen zou}\\

\haiku{dien anderen dag.}{was tot den avond toe alreeds}{bezet en bestemd}\\

\haiku{Haar gelatenheid,.}{leerde haar dat ze dit in}{waarheid geloofde}\\

\haiku{Damp had zich onder,.}{de binten vergaard stil en}{zwaar als witte rook}\\

\haiku{Heleen bedwong haar.}{verwarring en dorst er niet}{verder op doorgaan}\\

\haiku{Wat was ze, wat had,?}{ze wat beteekende ze meer}{dan volslagen niets}\\

\haiku{Ze sloot half de oogen.}{en hief in den lichtschijn het}{gezicht naar hem op}\\

\haiku{Bij het ontwaken,.}{bevond ze dat ze bang was}{en trilde van leed}\\

\haiku{Heleen overdacht wat.}{haar vriend haar van zijn eigen}{leven had verteld}\\

\haiku{Nooit te voren had.}{Heleen zoo schrikkelijk haar}{eenzaamheid beseft}\\

\haiku{O, zij en haar vriend,.}{hoe verschilden ze nog naar}{hun groei en maaksel}\\

\haiku{Toch rondde hij zijn.}{vingers onder haar kin en}{kuste haar vluchtig}\\

\haiku{Verlangen had haar,,;}{zoolang zij het meester bleef}{op sluwheid gespitst}\\

\haiku{{\textquoteright} Heleen las dien brief.}{en liet hem daarna uit}{haar handen vallen}\\

\subsection{Uit: Het huisje aan de sloot}

\haiku{{\textquoteright} Glad-zwart is de,....}{vloer van het paleis roerloos}{staan de pilaren}\\

\haiku{hij zijn schuit naar den,....}{overwal doet er hetzelfde}{en keert weer terug}\\

\haiku{In breede, malsche.}{plooien wijkt aan weerszijden}{het water terug}\\

\haiku{half-feesten!}{zijn bijna nog prettiger}{dan heele feesten}\\

\haiku{Zeker, juist omdat,.}{het zoo prachtig brandde moest}{het het eerste uit}\\

\haiku{ze wil den draad nu,....}{haastig binnenpalmen maar}{het hoeft al niet meer}\\

\haiku{Ze weet wel beter,.}{ze heeft opzettelijk de}{groote domheid gezegd}\\

\haiku{De kachel moet wel,}{geweldig branden wamt in}{de smalte tusschen}\\

\haiku{Dat kan vader niet,.}{zijn vader rookt geen pijp en}{vader schatert niet}\\

\haiku{{\textquoteright} Oom Zeelik houdt even,,....}{op zijn oogen zijn klein hun oogen}{zijn groot ze denken}\\

\haiku{want het leven is.}{nu \'e\'en en al trillend en}{tintelend geluk}\\

\haiku{En zijn voeten staan!}{precies zoo scheef naar buiten}{gedraaid als anders}\\

\haiku{- en ze maakte de.}{kistjes open en keek alles}{na wat er in zat}\\

\haiku{Mijmerend kijkt ze....}{uit de hoogste ruiten naar}{de witte wolken}\\

\haiku{nu is 't maar weer,.}{voorgoed achter den rug de}{lente komt eraan}\\

\haiku{Plat tegen den grond,,.}{bijna nog zonder steel zooals}{altijd die eersten}\\

\haiku{Maar dat klonk nu net.}{alsof ze heel iets anders}{had willen zeggen}\\

\haiku{Van zijn tiende jaar.}{af is hij wees en woont bij}{zijn grootmoeder in}\\

\haiku{Hij leende geld aan -!}{den Keizer van Rusland want}{z\'o\'o rijk was hij wel}\\

\haiku{Ze zouden zoo graag,.}{willen dat ze David nog}{eens tegenkwamen}\\

\haiku{Eindelijk is dan....}{toch de klare waarheid wijd}{voor ze opengegaan}\\

\haiku{of lag hij dieper?}{in het water dan hij in}{de aarde zal zijn}\\

\haiku{maar het woord doet met,....}{haar als de reuk het verschrikt}{haar en het lokt haar}\\

\haiku{Hij bewijst Zijne.}{Gerechtigheid aan hen die}{slapen in het stof}\\

\haiku{Niets vriendelijk kijkt...}{hij nu meer wat maakt haar dat}{allemaal benauwd}\\

\haiku{de twee pilaartjes,..}{in volle werking grappig}{om naar te kijken}\\

\haiku{Zoo klein is ze toch,,.}{niet dat hij haar niet meer zag}{vlak onder de lamp}\\

\haiku{in den laten avond,.}{zoodat ze van slaap niet staan en}{niet kijken konden}\\

\haiku{Snoek - dat is net als,!}{bij hen vader en moeder}{schelen ook tien jaar}\\

\haiku{Het is dol aardig,!}{als de groote menschen overhoop}{liggen met elkaar}\\

\haiku{Ze moeten ineens..,.}{allebei lachen ja maar}{om h\'e\'el wat anders}\\

\haiku{{\textquoteleft}Ja maar, vader, als..!}{het toch regent en met ons}{mooie Sjabbesgoed aan}\\

\haiku{Ze staan allemaal.}{om mijnheer Hamel heen en}{kijken naar hem op}\\

\haiku{en dan komt het nog,!}{uit dat ze zoo'n heele plas}{hebben gemorst}\\

\haiku{Het eerste uur is,.}{voorbij de meester deelt nu}{de leesboekjes rond}\\

\haiku{Nu ze dat bedenkt,,.}{kan ze niet meer huilen kan}{ze niet meer boos zijn}\\

\haiku{Ja, ze doen het, ze,.}{doen het hun stoelen schoven}{ze al achteruit}\\

\haiku{En er blonk iets dat,.}{zon ving h\'e\'el fel en dat iets}{was aan den Keizer}\\

\haiku{Helpt het, als je hooi?}{heenspreidt over je voeten om}{ze af te koelen}\\

\haiku{.. is haar broertje dan?}{een arme jongen en zijn}{zij arme-lui}\\

\haiku{Aan hun dokter stuurt.}{Vader het geld in een brief}{als het jaar om is}\\

\haiku{Ze kreeg bijna de '..}{deur int gezicht zoo vlak}{als ze er achter}\\

\haiku{helpt ze moeder in,.}{de kamer dan hoor je haar}{zingen vlak-bij}\\

\haiku{Het huisje aan de.}{sloot    Aantekeningen}{1Joodsche gemeente}\\

\subsection{Uit: In de schaduw van kinderleven}

\haiku{{\textquoteleft}'k Zal Bram en Wietje,...}{wel meenemen dan ben-u}{meteen van ze af}\\

\haiku{{\textquoteright} {\textquoteleft}Nou, gane jullie,{\textquoteright}, '.}{nou maar ongeduldigde}{Geertr thee slurpend}\\

\haiku{Ze waren nu in, '.}{wijde woelige groep om}{m heen komen staan}\\

\haiku{Dromerig klonk hun.}{gemurmel door de kille}{stilte van de gang}\\

\haiku{{\textquoteleft}Geef alles maar hier,{\textquoteright},.}{zei de moeder wachtend met}{uitgestrekte arm}\\

\haiku{'t Kind was immers.}{niet verkouden en ze had}{ook de tering niet}\\

\haiku{{\textquoteright} {\textquoteleft}Nee, vader,{\textquoteright} zeien,,... {\textquoteleft}'.}{de kinderen verlegen}{t is anstons tijd}\\

\haiku{'t Was toch wel n\`et,, '...}{of meester Bom naar hem keek}{dachtt jongetje}\\

\haiku{Want de hoofdingang.}{v\'o\'or was voor de meesters en}{juffrouwen alleen}\\

\haiku{dat w\`as geen klikken....}{vond-ie en vechten mocht-ie}{niet voor z'n vader}\\

\haiku{Maar dat duurde kort,:}{en plichtmatig ving hij aan}{z'n stem schor-hokkend}\\

\haiku{{\textquoteleft}'t Is waarachtig ' '...}{ofkn jongetje van}{de bewaarschool ben}\\

\haiku{II Op 't ijs, de,.}{winter tevoren hadden}{ze mekaar ontmoet}\\

\haiku{{\textquoteleft}Wel nee,{\textquoteright} weerlegde ',... {\textquoteleft}.}{t kind ernstighelemaal}{niet en Th\'e ook niet}\\

\haiku{Hij studeert maar wat... ' '...}{hardk Hebm al in geen}{drie dagen gezien}\\

\haiku{en nu meenden ze.}{allebei niet buiten de}{ander te kunnen}\\

\haiku{Maar z'n brood vonden.}{ze lekker en allemaal}{kochten ze bij hem}\\

\haiku{Hij moest blij zijn, dat,.}{ze h\`em met vrede lieten}{hem begunstigden}\\

\haiku{De vrouw vertoonde,.}{zich nooit de man heel zelden}{meer buiten de deur}\\

\haiku{peren en noten...{\textquoteright}...}{En schalks-nieuwsgierig}{tegen z'n moeder}\\

\haiku{Hij verstond niet w\`at,.}{ze zei hoorde alleen de}{klank van de woorden}\\

\haiku{{\textquoteright}... Duidelijk door de...}{wind hoorden we zacht kloppen}{en zacht roepen}\\

\haiku{Daar ginds was een dorp... '......}{uitgemoord omk weet niet}{wat om niks misschien}\\

\haiku{Maar woede vlamde,:}{in Brams ogen en de handen}{saamknijpend riep hij}\\

\haiku{Veel hebben we hier,.}{niet ergens anders hebben}{we j\`a niemendal}\\

\haiku{Waar was hij no\`u weer......}{heengelopen misschien een}{eind met de Rus mee}\\

\subsection{Uit: Een Indisch huwelijk}

\haiku{Wist hij niet -, alleen {\textquoteleft}{\textquoteright}, {\textquoteleft}{\textquoteright}.}{dat hijiets wilde omdat}{zij nuniets wilde}\\

\haiku{Het effen, oude -}{stemmetje meldde hem dat}{de njonja ziek was}\\

\haiku{Ja, er was in dit.}{alles toch wel iets om een}{vrouw te bekoren}\\

\haiku{Hoog stond de maan, den.}{langen weg terug zou hij}{alleen moeten gaan}\\

\haiku{En samen gingen.}{ze de roodgelooperde}{bordestreden af}\\

\haiku{Mocht het ook verdriet,?}{heeten wat hij gevoeld had}{na den dood van Ruysch}\\

\haiku{t was al om 't {\textquoteleft}{\textquoteright}.}{even hol en dom geweest als}{zijn laatsteavontuur}\\

\haiku{{\textquoteleft}Ga je met ons mee,{\textquoteright},, {\textquoteleft}}{d\'ejeuneeren Feenstra vroeg ineens}{echter de dokter}\\

\subsection{Uit: In de schaduw (van kinderleven)}

\haiku{{\textquoteleft}'k Zal Bram en Wietje,....}{wel meenemen dan ben-u}{meteen van ze af}\\

\haiku{N\`et kom-je uit  ,,....}{et gewone mot je daar}{weer heen na dat hok}\\

\haiku{{\textquoteright} {\textquoteleft}Nou, gane jullie,{\textquoteright}, '.}{nou maar ongeduldigde}{Geertr thee slurpend}\\

\haiku{Ze waren nu in, '.}{wijden woeligen groep om}{m heen komen staan}\\

\haiku{En hij woonde 'n.... '....}{heel eind ver had nou nieteensn}{boterham gehad}\\

\haiku{Droomerig klonk hun.}{gemurmel door de kille}{stilte van de gang}\\

\haiku{{\textquoteleft}Geef alles maar hier,{\textquoteright},.}{zei de moeder wachtend met}{uitgestrekten arm}\\

\haiku{Ze streek 'r witte, '....}{schort glad veegde de vochte}{handen aann doek}\\

\haiku{Z\`onder appele{\textquoteright}.... {\textquoteleft},....}{zei de moederweet-je}{wat Schooner dervoor vroeg}\\

\haiku{'t Kind was immers.}{niet verkouden en ze had}{ook de tering niet}\\

\haiku{{\textquoteright} {\textquoteleft}Nee, vader{\textquoteright} zeien,,.... {\textquoteleft}'.}{de kinderen verlegen}{t is anstons tijd}\\

\haiku{'t Was toch wel n\`et,, '....}{of meester Bom naar hem keek}{dachtt jongetje}\\

\haiku{Want de hoofdingang.}{v\'o\'or was voor de meesters en}{juffrouwen alleen}\\

\haiku{dat w\`as geen klikken.....}{vond-ie en vechten mocht-ie}{niet voor z'n vader}\\

\haiku{dan zou er  wel '....}{stellig iets heel vreeselijks}{metm gebeuren}\\

\haiku{Maar dat duurde kort,:}{en plichtmatig ving hij aan}{z'n stem schor-hokkend}\\

\haiku{'t Is waarachtig ' '.....}{ofkn jongetje van}{de bewaarschool ben}\\

\haiku{{\textquoteright} {\textquoteleft}H\`e ja{\textquoteright} zei Juul, en....}{ze stak haar arm door die van}{de oude vrouw}\\

\haiku{en nu meenden ze.}{allebei niet buiten den}{ander te kunnen}\\

\haiku{Jezes, die was in '....}{de Paaschvacantie ook al aan}{t zasen gegaan}\\

\haiku{{\textquotedblleft}As-je nou eerst eris ',{\textquotedblright}, {\textquotedblleft}}{n lamp ansteekt lijsde}{Gerritdan kanne}\\

\haiku{In hevigen angst, '....}{voor ongeluk gildet}{ouwe menschje}\\

\haiku{slepend gleden ze,.}{naar beneden op de snee}{vochtig en frisch wit}\\

\haiku{Maar z'n brood vonden.}{ze lekker en allemaal}{kochten ze bij hem}\\

\haiku{Hij moest blij zijn, dat,.}{ze h\`em met vrede lieten}{hem begunstigden}\\

\haiku{De vrouw vertoonde,.}{zich nooit de man heel zelden}{meer buiten de deur}\\

\haiku{Hij verstond niet w\`at,.}{ze zei hoorde alleen den}{klank van de woorden}\\

\haiku{'t Was een Duitsche,....}{meid meegekomen met die}{menschen naar Rusland}\\

\haiku{Veel hebben we hier,.}{niet ergens anders hebben}{we j\`a niemendal}\\

\haiku{Moesten ze terug naar, '....}{Rotterdam dan werdt weer}{krimpen in een krot}\\

\subsection{Uit: Het Joodje}

\haiku{en die zet me apart.}{op een bank en geen een wil}{er met me spelen}\\

\haiku{Ben dorst niet opzien,.}{maar de bom sprong heel anders}{dan hij had verwacht}\\

\haiku{{\textquoteright} riep Ben, toen hij den,.}{brief gelezen had rood van}{verontwaardiging}\\

\haiku{Zoo vertelde ze,.}{oom-en-tante terwijl ze}{voor den spiegel stond}\\

\haiku{En meer en meer steeg.}{zijn angst dat iemand tot hem}{het woord zou richten}\\

\haiku{{\textquoteright} zei ze, aarzelend,.}{half-verlegen en}{half-brutaal}\\

\haiku{zoo maar voetstoots op,.}{te geven had de oude}{Nachbar toegestemd}\\

\haiku{{\textquoteright} {\textquoteleft}Natuurlijk, maar zou,?}{het dan wel geschikt zijn dat}{we vanavond komen}\\

\haiku{, ik wil zeggen, dat?}{we het waardeeren zouden als}{je in de club kwam}\\

\haiku{de zaak is immers,,.}{van de baan Ben wil niet in}{de club daarmee uit}\\

\haiku{{\textquoteright} {\textquoteleft}Ze komt niet, omdat.}{ik haar heb gezegd dat ze}{wel weg kon blijven}\\

\haiku{hij had alles van,.}{me kunnen krijgen wat z'n}{hart maar had begeerd}\\

\haiku{Het viel hem dus zwaar,,,.}{genoeg maar alles moest om}{harentwil beproefd}\\

\haiku{En hij overlegde,,,}{dat hij zijn werk wat boeken}{die hij kon veinzen}\\

\subsection{Uit: Maneschijn met koek en Al om een suiker balletje}

\haiku{Ze staan allemaal.}{om mijnheer Hamel heen en}{kijken naar hem op}\\

\haiku{en dan komt het nog,!}{uit dat ze zoo'n heele plas}{hebben gemorst}\\

\subsection{Uit: 'n Badreisje in de tropen}

\haiku{Ik meende, dat je....{\textquoteright} {\textquoteleft},{\textquoteright},.}{Heelemaal niet viel Gerda}{in opgewonden}\\

\haiku{'t Dek was geruimd.}{en dat gaf haar dadelijk}{een prettig gevoel}\\

\haiku{was dan ook meteen.}{klaar wakker en met een sprong}{uit haar stoel overeind}\\

\haiku{Nu reden ze de.}{heele stad door tot aan den}{voet van den heuvel}\\

\haiku{Stil en koel stond aan.}{weerszijden van den weg het}{roerlooze schaduwbosch}\\

\haiku{O.... wat was mevrouw,....}{toch begonnen heen te gaan}{en haar te brengen}\\

\haiku{Gerda begreep, dat....,,?}{ze licht zou moeten vragen}{maar hoe maar aan wie}\\

\haiku{Zij wist geen weg in, {\textquoteleft}{\textquoteright}.}{de wereld maar hij wist geen}{weg in degoedang}\\

\haiku{Die Gerda, net een,.}{kind dat moest en dat zou nou}{van huis en op reis}\\

\haiku{als Sarian, z'n,, '.}{kleine jongen sterven ging}{zout haar schuld zijn}\\

\haiku{Nee, de cholera,;}{zat nergens die kwam rechtstreeks}{van toean-Allah}\\

\haiku{En daarna, frisch in,.}{haar versche koele kleeren kwam}{ze weer naar buiten}\\

\haiku{Daar zag ze Riboe,, '.}{den krani van haar man al}{t erf opkomen}\\

\haiku{Ze liet even staan de.}{beide mannen en liep de}{buurvrouw tegemoet}\\

\haiku{Achter uit den tuin,,;}{kwam nu ook Soemon de}{koetsier aanloopen}\\

\haiku{Nj\`a{\textquoteright}, antwoordde  ,.}{zacht de jongen als teeken}{dat hij had verstaan}\\

\haiku{Sa{\"\i}na keek hem,.}{na haar kindergezichtje}{boos en d\'edaigneus}\\

\haiku{'t heden was als '....}{gisteren en morgen zou}{t als heden zijn}\\

\haiku{Daar stonden ze stil,.}{in de klare maan statig}{en onwezenlijk}\\

\haiku{Maar wat zei mevrouw,....}{van zulk een vrouw van zulk een}{ondankbaar wezen}\\

\haiku{Uit z'n borst klom de,}{donkere toorn-grom}{om den honger dien}\\

\haiku{Hij was 'n groot en, '.}{sterk beest met forsche armen}{enn stoeren kop}\\

\haiku{Oeri, die niet op de....}{klok kon kijken en bang was}{voor de telefoon}\\

\haiku{iedereen wist toch....}{dat hij geen enkele vrouw}{met rust kon laten}\\

\subsection{Uit: Om de kinderen}

\haiku{Ja, Jeantje was,.}{altijd een gemakkelijk}{volgzaam kind geweest}\\

\haiku{Hij wist het zelf wel,.}{en het hinderde hem al}{liet hij niets merken}\\

\haiku{Een enkele maal,.}{wij sympathiseerden nooit}{zoo heel bijzonder}\\

\haiku{Dat uitkijken naar,,.}{de post die slapeloosheid}{die zenuwbuien}\\

\haiku{Dien keer, dat we in....}{het begin dien Beyerman}{hier te eten hadden}\\

\haiku{Altijd zal er iets....}{in mij  voor hem pleiten}{en hem vrijspreken}\\

\haiku{Of er ergens daar -.}{een hondje blafte zoo keek}{hij even mijn kant op}\\

\haiku{En toch.... ik kan nog,.}{altijd niet gelooven dat ik}{hem zoo miskende}\\

\haiku{Wil je nu heusch,?}{niet blijven terwijl mama}{Van der Wal er is}\\

\haiku{{\textquoteleft}Maar als ik nu wat,.}{aan hem vroeg zou het van zijn}{moeder afmoeten}\\

\haiku{En dan zou ze nu,,?}{op haar ouden dag zich nog}{bekrimpen moeten}\\

\haiku{Emilie zal je de, -}{boeken terugsturen die}{ze nog van je heeft}\\

\haiku{{\textquoteright} Heen en weer stampend,:}{in de kamer begon hij}{luidkeels te zingen}\\

\haiku{{\textquoteright} Elk met een dik pak.}{chocolade in de hand}{stonden ze v\'o\'or haar}\\

\haiku{Den laatsten keer kon....}{je het in November nog}{zien aan mijn gezicht}\\

\haiku{introduc\'ees van.}{de gasten misschien en de}{zoon van den gastheer}\\

\haiku{Ja, nu vloog de drift -.}{weer in haar op en daarom}{wilde ze slapen}\\

\haiku{dat was de blijde,.}{zekerheid geweest waaraan}{ze zich staande hield}\\

\haiku{{\textquoteleft}Emilie, die in haar.}{eentje zit te snoepen van}{een verboden boek}\\

\haiku{Het was nu toch wel;}{volslagen pikdonker om}{haar heen geworden}\\

\haiku{De aanmatiging,.}{de verwaandheid van dat kind}{werd wel grenzeloos}\\

\haiku{{\textquoteleft}Fijnbesnaard{\textquoteright} - ja, 't.}{was aardig uitgedrukt van}{dominee Lette}\\

\haiku{Ver van hem af, zou.}{ze zijn heugenis uit haar}{leven verdrijven}\\

\haiku{wat zou hij, wat zou,....}{Ard Hettema zeggen als}{hij dat van haar wist}\\

\haiku{hij, dien ze zich niet?}{anders dan vlekkeloos kon}{en wilde denken}\\

\haiku{Ce portrait, tout beau,....}{que ce soit ne vaut pas un}{baiser du mod\`ele}\\

\haiku{{\textquoteright} Het boekerige,.}{woord misklonk maar Margo vond}{zoo gauw geen ander}\\

\haiku{Anders was ze toen,.}{juist blijven komen toen ze}{het merkte van Ard}\\

\haiku{Het staat allemaal,....}{zoo ver van mij af als een}{andere wereld}\\

\haiku{En hij zou dan toch.}{wel eenigen invloed hebben}{op moeder en zoon}\\

\haiku{Je vroeg je af hoe.}{die dingen dadelijk zoo}{werden overgebracht}\\

\haiku{{\textquoteright} Ditmaal wachtte hij,,.}{weer wel antwoord dicht bij de}{tafel zacht hijgend}\\

\haiku{Plas merkte ditmaal.}{nog minder dan anders hun}{binnenkomen op}\\

\haiku{{\textquoteright} Zoo onnatuurlijk,.}{kalm was nu weer zijn stem dat}{Plas er van opkeek}\\

\haiku{{\textquoteright} {\textquoteleft}Niets, ze wilde het.}{hem niet zeggen en ik mocht}{het hem niet zeggen}\\

\haiku{Ik ben de vriend van,....}{haar vader en ik beschouw}{mijzelf zoolang hij}\\

\haiku{het is volgens uw,.}{eigen beginselen dat}{ze trouwen moeten}\\

\haiku{Vanavond nog zou hij.}{kalm en vaderlijk met Frans}{over alles spreken}\\

\haiku{En toch, hij kon zich.}{geen blijvend geluk denken}{uit hun huwelijk}\\

\haiku{neen, het was toch wel,.}{altijd een verrukking zoo}{je macht te voelen}\\

\haiku{En nog zocht Frans de.}{schuld van zichzelven af op}{haar te wentelen}\\

\haiku{Ik zei, dat ik bij.}{hem de resultaten niet}{erg schitterend vond}\\

\haiku{{\textquoteright} {\textquoteleft}Maar hoe kom je er,?}{nu zoo plotseling bij nu}{we bijna thuis zijn}\\

\haiku{Henriet had ze reeds,.}{gewonnen Margo zou ze}{gewonnen hebben}\\

\haiku{Dokter Van 't Hoff.}{en ik hopen over een paar}{maanden te trouwen}\\

\haiku{Het was als werd haar,}{Leidsche leven door de vraag}{voor haar oogen gebeurd}\\

\haiku{{\textquoteright} {\textquoteleft}'t Lijkt allemaal,.}{al zoo lang geleden er}{is zooveel gebeurd}\\

\haiku{En Margo vooral,!}{was altijd zoo sceptisch na}{haar ervaringen}\\

\haiku{Toen die blik van het -,, -?}{kind spot minachting wrevel}{wat was het geweest}\\

\haiku{{\textquoteleft}Ik kan het, wat je,.}{divorce betreft volstrekt}{niet met je eens zijn}\\

\haiku{Ze zal nog wel eens....}{vaker en nog wel eens veel}{erger bedroefd zijn}\\

\haiku{Het is zoo goed dat....}{we het maar vooraf weten}{en op ons nemen}\\

\haiku{Hij schijnt haar zelf les.}{in allerlei dingen te}{hebben gegeven}\\

\haiku{Een man had zoo zijn.... {\textquoteleft}}{ijdelheidjes en De Wit}{was burgemeester}\\

\haiku{{\textquoteright} gooide Maddy's,.}{vader er uit een beetje}{grof-schertsend}\\

\haiku{Waarom mocht hij nu,,?}{niet dien eenen dag hier bij haar}{zijn en haar troosten}\\

\haiku{een besluit nemen,?}{en namen ook anderen}{zoo hun besluiten}\\

\haiku{Tot den toren ga '.}{ik mee en dant fietspad}{over de beek terug}\\

\haiku{Maar zelfcritiek zou.}{toch Jettie's redding wezen}{op den langen duur}\\

\haiku{Het eerste wat die.}{man denkt is natuurlijk dat}{hij duelleeren moet}\\

\haiku{Net als bij dien man,....?}{in het boek maar gelukkig}{niet in die mate}\\

\haiku{Hun blikken schampten,....}{langs elkaar heen iets dat zich}{schoof tusschen hun oogen}\\

\haiku{Alleen Oma praatte, -?}{onbevangen zag niets of}{hield ze zich maar zoo}\\

\haiku{Stijve Hollandsche....!}{schoonmama en nuchtere}{Hollandsche jongen}\\

\haiku{Hij was niet jong, moest,}{de vijftiger jaren al}{beklommen hebben}\\

\haiku{{\textquoteright} {\textquoteleft}Clich\'e,{\textquoteright} smaalde Frans,, {\textquoteleft}.}{schouderschokkendgeen aasje}{oorspronkelijkheid}\\

\haiku{Het was een tijd vol -....{\textquoteright}}{emoties en niet allemaal}{prettige emoties}\\

\haiku{Haar eigen dochter.}{is volwassen en nog vond}{ze het ongepast}\\

\haiku{\'e\'en woord en hij had,,.}{haar wankelend een tweede}{en overgegeven}\\

\haiku{Opnieuw begon, in,.}{wilden doelloozen ijver een}{eend te snateren}\\

\subsection{Uit: Prometheus}

\haiku{{\textquoteleft}We kunnen niet zien{\textquoteright} -,:.}{zegt men maar ook we kunnen}{niet onderscheiden}\\

\haiku{Zien we geen contrast,.}{dan onderscheiden we niet}{dan zien we dus niets}\\

\haiku{Denken we aan de {\textquoteleft}{\textquoteright}:}{curieuse vraag in de}{Tweede Hippias}\\

\haiku{Aan dien waan houdt hij,.}{zich staande deze is de}{steun van zijn zwakheid}\\

\haiku{Huiselijk gezegd,.}{men kan geen twee ruggen uit}{\'e\'en varken snijden}\\

\haiku{Rust toch is alleen.}{bij verblinding en bij de}{hoogste helderheid}\\

\haiku{onbegrensd,  naar,;}{alle kanten vervloeiend}{zelf-opheffend}\\

\haiku{Het eindpunt zal dus.}{steeds  tegelijkertijd}{weer uitgangspunt zijn}\\

\haiku{hij anderen niet.}{den weg wijst naar bronnen van}{eigen onderzoek}\\

\haiku{Gij hebt gehoord, dat - -{\textquoteright}}{tot de Ouden gezegd is}{maar ik zeg U enz.}\\

\haiku{de gedachte dat {\textquoteleft}{\textquoteright},.}{hijafbrekend werk doet zou}{hem ondraaglijk zijn}\\

\haiku{de (zinnelijke).}{liefde in de eerste plaats}{en daarmee de vrouw}\\

\haiku{Altijd looft de mensch.}{instinctief datgene wat}{zichzelf gelijk blijft}\\

\haiku{Ook die {\textquoteleft}men{\textquoteright} zijn we, {\textquoteleft}{\textquoteright}.}{allen zelfs eencontented}{pessimist als Shaw}\\

\haiku{Deze voorkeur heeft,.}{ook nog een anderen een}{positiever kant}\\

\haiku{Hij hield zich aan de,!}{Schrift maar wilde vooral niet}{redeloos heeten}\\

\haiku{Wie was armer dan,?}{de schrijver van Gil Blas Alain}{Ren\'e le Sage}\\

\haiku{Want het wezen der.}{dingen ontgaat hem door zijn}{egocentrischen aard}\\

\haiku{{\textquoteleft}le grand Cond\'e{\textquoteright}!) aan het -.}{wankelen bracht er is een}{innerlijk verschil}\\

\haiku{Gods recht op alle.}{dingen berust op zijn macht}{over alle dingen}\\

\haiku{, {\textquoteleft}zullen allicht meer,}{twisten rijzen dan tusschen}{meesters en slaven}\\

\haiku{Na Lodewijk XIV.}{is er in diens stijl geen groot}{heerscher meer geweest}\\

\haiku{ist das Wort todt, so.}{k\"onnen es keine Riesen}{aufrecht erhalten}\\

\haiku{Het gaat hem op het:}{allerbest zooals het Faust met}{den Aardgeest ging}\\

\haiku{in de Eenheid is.}{alle zelfherkenning reeds}{zelfvernietiging}\\

\haiku{Ook Lessing heeft de.}{officieele geleerdheid}{niet malsch behandeld}\\

\haiku{Maar een nieuw licht schijnt,.}{nu in en uit de lampen}{die menschen heeten}\\

\haiku{hij ziet {\textquoteleft}liefde{\textquoteright} als;}{een daarvan en zelfs niet steeds}{de belangrijkste}\\

\haiku{Zoo was het reeds, in,.}{beginsel gedurende}{de Renaissance}\\

\haiku{een zwakke en een,.}{sterke een falende en}{een overwinnende}\\

\haiku{Zoo spiegelt alle:}{litteratuur van den tijd}{hetzelfde streven}\\

\haiku{{\textquoteleft}Ist sein Hand wider,.}{jederman wird jedermans}{Hand sein wider ihn}\\

\haiku{Dit levert hier een,.}{intellectueel verschil}{geen zedelijk op}\\

\haiku{een noodzakelijk.}{kwaad tot het schoone oogmerk}{der Verzoening is}\\

\haiku{{\textquoteright} de Vicomte zich {\textquoteleft}.}{tot veelbeteekenend motto}{koos voor zijnTh\'eorie}\\

\haiku{ihn in Demut zu,,.}{beerben und viel zu schwach um}{ihm es gleich zu thun}\\

\haiku{Voor Lorenzo heeft.}{het leven voortaan geen zin}{en geen inhoud meer}\\

\haiku{La r\'ecompense,.}{est si grosse qu'elle les}{rend presque courageux}\\

\haiku{c'\'etait peut-\^etre.}{un p\`ere de famille}{qui mourait de faim}\\

\haiku{De redeneering.}{van alle dogmatici}{in alle tijden}\\

\haiku{Als dien na{\"\i}even -.}{schellinkjes-klant z\'o\'o}{ziet hij Prometheus}\\

\haiku{{\textquoteleft}Je ne leur ai pas,.}{donn\'e la pens\'ee car je}{suis un Dieu bon}\\

\haiku{elle pousse au{\textquoteright} -,:}{crime maar er volgt een raad}{een vertroosting op}\\

\subsection{Uit: Seideravond}

\haiku{over het bittere,,.}{over het offer en over het}{ongezuurde brood}\\

\haiku{n lolletje... 'n......}{relletje om de ouwe}{smous te treiteren}\\

\haiku{(rukt ineens de deur,)?}{open schreeuwend met schorre stem}{wat moeten jullie}\\

\haiku{jij deed het al toen,,.}{je tien jaar was niewaar in}{je grootvaders huis}\\

\haiku{Hier is het ei...  (,,).}{neemt het in de hand beschouwt}{het stem weemoedig}\\

\haiku{Je moet sterk zijn om,.}{daar tusschen te leven en}{toch vroom te blijven}\\

\haiku{Waarin is deze?}{nacht verschillend van alle}{andere nachten}\\

\subsection{Uit: Tirol}

\haiku{we hebben in een}{oudtirools Gasthof onze}{intrek genomen}\\

\haiku{er binnen te gaan.}{als bijvoorbeeld te leren}{shimmy-dansen}\\

\haiku{We zaten op de.}{bank onder een machtige}{beuk en we wachtten}\\

\haiku{keek hij rond of zich.}{ergens op zijn pad ook een}{jood dorst vertonen}\\

\haiku{Ten leste mocht het,,,...}{dan lukken maar vandaag och}{arme lukt het niet}\\

\haiku{Als hij nu, in zijn,.}{eigen belang maar niet al}{te eenvoudig is}\\

\haiku{Het is Maleis, en,.}{ik sprak het met sombere}{droeve stemklank uit}\\

\haiku{Thans loer ik op mijn.}{taalgevoelige vriend als}{een spin in het web}\\

\haiku{De muzikanten.}{zijn met hun hele boeltje}{naar buiten verhuisd}\\

\haiku{Ook wij zijn door de,.}{oude enge poort maar weer}{hierheen gekomen}\\

\haiku{Maar het zwijgende.}{jonge meisje is in geen}{geval zijn zuster}\\

\haiku{Plotseling zie ik...}{het als de bovenloop van}{een grote rivier}\\

\haiku{In de ene schaal het,...}{pasgenotene Fulpmes}{met zijn gletsjerschoon}\\

\subsection{Uit: Van een kind}

\haiku{Die konden zich nu,!}{heerlijk verbeelden dat zij}{in Parijs waren}\\

\haiku{De hele zomer.}{door bleven die vazen en}{manden daar staan}\\

\haiku{Soms had een knappe.}{tuinman er een jaartal of}{letters in gevormd}\\

\haiku{Wat had ze zo trots,?}{en zelfgenoegzaam wat had}{ze zo knap te zijn}\\

\haiku{En in afwachting.}{werden ze plotseling stil}{en keken haar aan}\\

\haiku{Verslagen lag die:}{jongenstrots en alles dat}{ermee samenging}\\

\haiku{die kwam om Bart nu,!}{het te laat was die kwam nu}{nog haar verdringen}\\

\haiku{ze voelde haar bloed,.}{bonzen door haar lijf maar ze}{hield zich parmantig}\\

\haiku{- dan kregen ze ook -}{koffie met beschuitjes en}{tegen Sinterklaas}\\

\haiku{Buiten zag ze hem.}{en Maup als Indianen}{dansen en grijnzen}\\

\haiku{Ook Leentje, vuurrood,,.}{naast de stoel dorst niet opzien}{zich niet verroeren}\\

\haiku{{\textquoteleft}Laat Abraham toch eerst.}{rustig gaan zitten en een}{kop koffie drinken}\\

\subsection{Uit: De vergelding (onder ps. Justine Abbing)}

\haiku{Even later liep ze}{dan door de winkelstraten}{en verbaasde zich}\\

\haiku{Moet alle zeilen.}{bijzetten om niet in haar}{klauwtjes te vallen}\\

\haiku{De bruuske toon van.}{Verkerks ondervraging bracht}{hem in verwarring}\\

\haiku{hij dacht anders den....}{laatsten tijd weinig meer aan}{Marietje de Geus}\\

\haiku{soms in het gevoel....}{dat hij wel nemen mocht wat}{hem gegeven werd}\\

\haiku{en dan, Jaap, dan ben....}{ik ineens zoo bang dat hij}{niet terugkeeren zal}\\

\haiku{En vind jij nu niet....,....}{in zoo'n vertoon van ja hoe}{zal ik het noemen}\\

\haiku{ik zeg het verkeerd,,....}{ik ben wezenlijk alles}{waarvoor hij mij houdt}\\

\haiku{Maar niet altijd kon,,....}{hij zich verdiepend in haar}{zoo rustig blijven}\\

\haiku{want welke man was?}{nu niet als jong student een}{beetje wild geweest}\\

\haiku{maar ze zou zich nooit.}{zoo dwaas en onvoorzichtig}{hebben gedragen}\\

\haiku{nu lag ze erin.}{en mocht ze met de heete}{kruik naar boven gaan}\\

\haiku{Alleen opnieuw in,....}{donker sloeg Hetty de oogen}{open en luisterde}\\

\haiku{ze hoorde nicht naar,.}{het jassenkamertje gaan}{waar het toestel hing}\\

\haiku{te weten of zij,!}{zeker van zichzelf was van}{haar gevoel voor hem}\\

\haiku{{\textquoteright} {\textquoteleft}Niet wat je hem hebt,....}{geschreven niet wat je nu}{verder denkt te doen}\\

\haiku{Laag over de hei woei,,....}{die wind met geluiden als}{snikken lang en droef}\\

\haiku{dan kunnen we het,.}{hem laten weten dat je}{hem liever niet ziet}\\

\haiku{Waarom zou ze zijn,?}{belangstelling afwijzen}{hem niet willen zien}\\

\haiku{of trok, nu al, Jaaps....}{geest van haar weg en liet haar}{wezenlijk alleen}\\

\haiku{nog begrijp ik het,}{niet en hij gaf me zijn woord}{dat hij gezien had}\\

\haiku{Stel je voor, hij gaat.}{weg van onze school omdat}{hij professor wordt}\\

\haiku{Maar ziet u, ze heeft,.}{w\`el kort haar korte krullen}{net als een jongen}\\

\haiku{{\textquoteleft}Ik dacht wel, dat het,!}{iets was waar je niet mee voor}{den dag kon komen}\\

\haiku{{\textquoteright} Hij lachte, een kort,,.}{stootend lachje sloeg met zijn}{krukstok in het zand}\\

\haiku{daarvan \'e\'en, die we,....}{toch niet houden die nu al}{naam maakt met zijn werk}\\

\haiku{ze zal ook wel eens....}{een kwartiertje naar Bolland}{geluisterd hebben}\\

\haiku{We hopen dan na.}{de groote vacantie nader}{kennis te maken}\\

\haiku{Ineens werd het haar,.}{te erg midden in een zin}{liet ze zich steken}\\

\haiku{het was haar, bij wat,!}{het schonk toch wel heel duur te}{staan gekomen ook}\\

\haiku{En.... en op eenmaal....}{sprongen haar gedachten meer}{dan tien jaar  over}\\

\haiku{en het wat aardig,....}{een gezellig dineetje voor}{hem ervan maken}\\

\subsection{Uit: Verhalend proza}

\haiku{'t Jongste broertje,.}{keek haar donker aan stompte}{haar tegen den arm}\\

\haiku{Jaren daarna stierf.}{de vader en had hem geld}{en zaak gelaten}\\

\haiku{Hier links moesten ze die,.}{straat in die prachtige straat}{met die mooie huizen}\\

\haiku{{\textquoteright} {\textquoteleft}N\'e\'e, ze zullen n{\'\i}et,{\textquoteright}.}{invallen zei met hoonenden}{nadruk de vader}\\

\haiku{Niemand zou aan hem,.}{kunnen zien dat hij geen}{schoolgeld betaalde}\\

\haiku{Ik wou je ook nog...{\textquoteright} ',,, {\textquoteleft}'}{zeggent meisje zweeg even}{zei dan snel en zacht}\\

\haiku{t was echt akelig,,.}{echt valsch zooals de directeur}{tegen je praatte}\\

\haiku{{\textquoteright} Jozef keek verschrikt.}{op en beurtelings zijn broer}{en zijn vader aan}\\

\haiku{De meisjes waren.}{op een wenk van haar moeder}{al naar bed gegaan}\\

\haiku{{\textquoteright} Met de anderen.}{verheugde Daantje zich op}{den komenden avond}\\

\haiku{Jozef was nog niet,;}{aan tafel moeder had zijn}{bord eten toegedekt}\\

\haiku{Esther groette met.}{een nog hooger kleur en het}{kind groette terug}\\

\haiku{Van moeder hield ze,,.}{wel maar om wat d{\'\i}e zei gaf}{ze heelemaal niet}\\

\haiku{En nu weer naar de,.}{vijfde met een prijs en de}{jongste in z'n klas}\\

\haiku{Maar de moeder lag.}{daar met gesloten oogen of}{ze niets had gehoord}\\

\haiku{de oude heer was.}{ongetwijfeld de rijkste}{en dat bleef hoofdzaak}\\

\haiku{Leuk zou dat wezen....}{als het tenminste blonde}{kinderen waren}\\

\haiku{De drie sloegen hun.}{jaskragen op en liepen}{onverstoord verder}\\

\haiku{Zoo zal het nou op,,!}{die nieuwe school als ik er}{kom ook wel weer gaan}\\

\haiku{Dan zouden ze toch!}{stellig niet meer Jodin en}{smaus durven schelden}\\

\haiku{al was het voelen,,,}{meer dan beseffen dat zij}{hoe broos haar bestaan}\\

\haiku{En zij alleen hier,,,...}{in huis moeder dood Esther}{weg Dani\"el weg}\\

\haiku{Voor de rest ging hij,?}{z'n gang maar hoefde vader}{daarvan te weten}\\

\haiku{Dani\"el richtte.}{z'n pijnlijke hoofd op en}{keek naar het meisje}\\

\haiku{Een souper van den,,,.}{kok van Wertheimer die w\'el}{fijn maar zoo duur was}\\

\haiku{Nathans is geen fijne,.}{kok Nathans is een gewone}{koosjere bakker}\\

\haiku{Schrille en rauwe '.}{roepen klonken uit boven}{t confuus rumoer}\\

\haiku{David luierde.}{in een hoek en las in z'n}{hemdsmouwen de krant}\\

\haiku{In waarachtigheid,,. '}{hij had gedacht dat vader}{toen gek ging worden}\\

\haiku{n Paar dagen was,.}{hij gebleven toen moest hij}{terug naar z'n werk}\\

\haiku{En sinds leidde ze.}{het gewone leven van}{zoovele vrouwen}\\

\haiku{Ze verlangde nu,;}{hevig naar het land dat ze}{tegemoetgingen}\\

\haiku{Van Gulik zou daar,,.}{den eersten tijd althans een}{rustkuur doormaken}\\

\haiku{Ze was ook blij, dat,.}{Jozef trouwen ging voor hem}{en voor Rebecca}\\

\haiku{Vader scheen nog maar,.}{voor \'e\'en ding te leven voor}{zijn geloof alleen}\\

\haiku{Hij had zich verhard}{en hij verhardde zich meer}{en meer en alles}\\

\haiku{Ze sprak over hem, vond,.}{Roosje alsof ze met hem}{was getrouwd geweest}\\

\haiku{belangstellend kwam.}{haar van z'n vroolijke oogen}{de blik tegemoet}\\

\haiku{Dat kind deed nog wel,.}{haar plicht dat kind zat nog wel}{vast in het geloof}\\

\haiku{Dan zou ze haar brood,.}{hebben en geen zorgen dan}{zou ze veilig zijn}\\

\haiku{Het leven had haar,.}{nog niets gegeven dit leek}{haar te veel opeens}\\

\haiku{Hij had dankbaarheid,.}{verwacht verblijding om wat}{er besloten was}\\

\haiku{Dien brief legde hij.}{den volgenden dag op Heins}{kamer en vertrok}\\

\haiku{In die momenten,}{voelde hij w\'elbewust welk}{een schrikkelijk spel}\\

\haiku{Debora stapte.}{achter de toonbank vandaan}{en haar in den weg}\\

\haiku{Maar toen het meisje,.}{eenmaal verdwenen was was}{gauw haar toorn gezakt}\\

\haiku{Vast van plan was ze,,?}{het geweest maar goed beschouwd}{wat had ze voor r\'echt}\\

\haiku{- en Roos trouwde, al,?}{had-ze geen cent al was}{ze een Jiddekind}\\

\haiku{Nee... nee... nee, ze zou,.}{zich nergens mee bemoeien}{niet hier en niet daar}\\

\haiku{En... w\'at zei\"en ze,,?}{tegen mekaar wat deden}{ze samen die twee}\\

\haiku{Dani\"el had hij,;}{weggestuurd in zijn huis geen}{Gods-lasteraar}\\

\haiku{Z'n heele leven,.}{was hij trouw ter sjoel gegaan}{nou kon hij niet meer}\\

\haiku{{\textquoteleft}Ja..., er zal toch met?}{je vader over gesproken}{dienen te worden}\\

\haiku{misschien slaat vader......{\textquoteright} {\textquoteleft}}{mij of jaagt mij uit het huis}{Als je vader je}\\

\haiku{Hij wist, dat het uur.}{gekomen was en dat hij}{eenzaam sterven zou}\\

\haiku{Rudi en Roosje,.}{ontwaarden hem z\'o\'o bij hun}{binnentreden}\\

\haiku{{\textquoteright} Glad-zwart is de,...}{vloer van het paleis roerloos}{staan de pilaren}\\

\haiku{hij zijn schuit naar den,...}{overwal doet er hetzelfde}{en keert weer terug}\\

\haiku{In breede, malsche.}{plooien wijkt aan weerszijden}{het water terug}\\

\haiku{half-feesten!}{zijn bijna nog prettiger}{dan heele feesten}\\

\haiku{De lichtjes tellen -,}{met de avonden op elken}{middag giet vader}\\

\haiku{Zeker, juist omdat,.}{het zoo prachtig brandde moest}{het het eerste uit}\\

\haiku{Ze weet wel beter,.}{ze heeft opzettelijk de}{groote domheid gezegd}\\

\haiku{De kachel moet wel,}{geweldig branden want in}{de smalte tusschen}\\

\haiku{Dat kan vader niet,.}{zijn vader rookt geen pijp en}{vader schatert niet}\\

\haiku{want het leven is.}{nu \'e\'en en al trillend en}{tintelend geluk}\\

\haiku{Nu is ze achter,.}{die deur nu is ze in de}{kerke-kamer}\\

\haiku{erover gebogen...,,...}{kijkend wijzend als lazen}{ze elkaar iets voor}\\

\haiku{En zijn voeten staan!}{precies zoo scheef naar buiten}{gedraaid als anders}\\

\haiku{- en ze maakte de.}{kistjes open en keek alles}{na wat er in zat}\\

\haiku{Die zegt niets..., maar haar,.}{oogen worden groot haar mond gaat}{open van verbazing}\\

\haiku{nu is 't maar weer,.}{voorgoed achter den rug de}{lente komt eraan}\\

\haiku{Plat tegen den grond,,.}{bijna nog zonder steel zooals}{altijd die eersten}\\

\haiku{Maar dat klonk nu net.}{alsof ze heel iets anders}{had willen zeggen}\\

\haiku{zoolang, dat het haar,.}{soms wel eens lijkt alsof ze}{het maar had gedroomd}\\

\haiku{Van zijn tiende jaar.}{af is hij wees en woont bij}{zijn grootmoeder in}\\

\haiku{Hij leende geld aan -!}{den Keizer van Rusland want}{z\'o\'o rijk was hij wel}\\

\haiku{Ze zouden zoo graag,.}{willen dat ze David nog}{eens tegenkwamen}\\

\haiku{Eindelijk is dan...}{toch de klare waarheid wijd}{voor ze opengegaan}\\

\haiku{maar het woord doet met,...}{haar als de reuk het verschrikt}{haar en het lokt haar}\\

\haiku{Hij bewijst Zijne.}{Gerechtigheid aan hen die}{slapen in het stof}\\

\haiku{Niets vriendelijk kijkt....}{hij nu meer wat maakt haar dat}{allemaal benauwd}\\

\haiku{de twee pilaartjes,...}{in volle werking grappig}{om naar te kijken}\\

\haiku{Zoo klein is ze toch,,.}{niet dat hij haar niet meer zag}{vlak onder de lamp}\\

\haiku{in den laten avond,.}{zoodat ze van slaap niet staan en}{niet kijken konden}\\

\haiku{Snoek - dat is net als,!}{bij hen vader en moeder}{schelen ook tien jaar}\\

\haiku{Met die zou 't wel,.}{goed afloopen als ze geen}{Joum Kippour meer hield}\\

\haiku{Hij heet Salomon.}{en wie hem Sal of Sallie}{noemt op school krijgt straf}\\

\haiku{Het is dol aardig,!}{als de groote menschen overhoop}{liggen met elkaar}\\

\haiku{Ze moeten ineens...,.}{allebei lachen ja maar}{om h\'e\'el wat anders}\\

\haiku{{\textquoteleft}Ja maar, vader, als...!}{het toch regent en met ons}{mooie Sjabbosgoed aan}\\

\haiku{Ze staan allemaal.}{om mijnheer Hamel heen en}{kijken naar hem op}\\

\haiku{en dan komt het nog,!}{uit dat ze zoo'n heele plas}{hebben gemorst}\\

\haiku{Het eerste uur is,.}{voorbij de meester deelt nu}{de leesboekjes rond}\\

\haiku{Nu ze dat bedenkt,,.}{kan ze niet meer huilen kan}{ze niet meer boos zijn}\\

\haiku{Ja, ze doen het, ze,.}{doen het hun stoelen schoven}{ze al achteruit}\\

\haiku{en juist v\'o\'ordat ze,,}{opnieuw warm over haar heele}{gezicht haar kijken}\\

\haiku{{\textquoteright}        *~         [20] Hooi voor {\textquoteleft}!}{warme voetenVader heeft}{het toch zelf gezegd}\\

\haiku{En er blonk iets dat,.}{zon ving h\'e\'el fel en dat iets}{was aan den Keizer}\\

\haiku{Helpt het, als je hooi?}{heenspreidt over je voeten om}{ze af te koelen}\\

\haiku{... is haar broertje dan?}{een arme jongen en zijn}{zij arme-lui}\\

\haiku{Binnenkomen en '.}{brood-eten moet ze en dan}{naart Joodsche school}\\

\haiku{Aan hun dokter stuurt.}{vader het geld in een brief}{als het jaar om is}\\

\haiku{het is Harm Blok, en!}{er zat bloed aan zijn handen}{en bloed in zijn haar}\\

\haiku{helpt ze moeder in,.}{de kamer dan hoor je haar}{zingen vlak-bij}\\

\haiku{Je hebt gewacht, je.}{hebt er een plaats in jezelf}{voor open gehouden}\\

\haiku{Ik zit nu op een,,.}{toren er is geen vuil dat}{zoo hoog spatten kan}\\

\haiku{Je hebt het ook met -,.}{wie het niet verdienen met}{den boozen padrone}\\

\haiku{In glinsterkringen.}{en zilverig schuim staat het}{kortgesneden riet}\\

\haiku{{\textquoteright} {\textquoteleft}Ja, luisteren wekt...{\textquoteright} {\textquoteleft}.}{de geluidenGeluid dat}{er misschien niet is}\\

\haiku{{\textquoteright} En de eerste dag,,!}{dien ze langs hun oogen voorbij}{zagen gaan zij zelf}\\

\haiku{Achter ze is de,,...}{lange donkere kap als}{een drukkende hand}\\

\haiku{Elken dag weer maakt,... '}{haar de stad tot het zijne}{neemt haar in bezit}\\

\haiku{links en rechts een snoer.}{van gouden tientjes boven}{den donkeren stroom}\\

\haiku{Ik kan al niet eens,...}{goed tegen gymnastiekles}{om de commando's}\\

\haiku{Dus dat je je niet,...}{wou laten onderzoeken}{dat begrijp ik best}\\

\haiku{Reuk, laat mij los, plaag -,.}{mij niet om een naam ik kan}{hem je niet geven}\\

\haiku{tusschen buurman Bol,,.}{en buurman Bruin daar woont ze}{daar wordt ze verwacht}\\

\haiku{Maar kun je helpen {\textquoteleft}{\textquoteright}?}{dat je weet watperoe}{oe-rewoe beduidt}\\

\haiku{Ze zit achteruit,.}{in haar stoel laat het lamplicht}{in haar oogen bijten}\\

\haiku{Neen, van dat eene Hoofd,, {\textquoteleft}}{dien aardigen heer die zoo}{nadrukkelijk zei}\\

\haiku{Het is, als kwamen,.}{ze van een verre reis als}{waren ze vermoeid}\\

\haiku{Vergeet het nu even,,.}{leg het van u af dat u}{Grootvaders zoon bent}\\

\haiku{midden in den nacht,...}{midden in de zee en z\'o\'o}{is ook dit van hier}\\

\haiku{wrevelig knarsen.}{zijn zware voeten over den}{zanderigen vloer}\\

\haiku{Was dit staan over een?}{kom met rozen gebogen}{dan weer zoo iets geks}\\

\haiku{hij zei iets tegen,,...}{Selien hij lachte gooide}{zijn hoofd achterover}\\

\haiku{Joppe, ik moet t\'och...... {\textquoteleft}}{om je lachen lachen met}{tranen in mijn oogen}\\

\haiku{of denk je soms dat......}{de heele buurt het niet weet}{goddelooze jongen}\\

\haiku{Even staat Joppe tot,.}{stilte beschaamd maar dat even}{is alweer voorbij}\\

\haiku{Binnen in mijn Lijf,.}{draag ik mijn Geweten ben}{ik mijn geweten}\\

\haiku{dat ze je gevraagd,.}{heeft bij haar en Dora te}{komen inwonen}\\

\haiku{Het schijnt dat hij thuis,,:}{op niets meer aan niemand meer}{antwoord geeft hij zegt}\\

\haiku{Kan Arjen Brand het?}{helpen dat Bauk hem aan den}{haak heeft geslagen}\\

\haiku{Ze zeggen alles,.}{verschillend ze bedoelen}{alles verschillend}\\

\haiku{daar stond Herman in,,.}{de deur met zijn guitaar met}{zijn nieuwe liedjes}\\

\haiku{Leun roerloos tegen,......}{het regenraam sta stil in}{de stilte en wacht}\\

\haiku{het is ongepast,...}{en het is je geluk dat}{je maar alleen bent}\\

\haiku{{\textquoteright} Het laatste dat ze,,:}{Ernestien heeft geschreven}{was op een briefkaart}\\

\haiku{Je hebt mij altijd......?}{tegen haar verdedigd laat}{je mij nu alleen}\\

\haiku{Dan zeg ik u, dat,...{\textquoteright} {\textquoteleft}...}{de man die dat schreefEn zelf}{geen haar beter is}\\

\haiku{Willy... Rebecca...,,...{\textquoteright} {\textquoteleft}...}{neem een kopje neem dit het}{is het slapsteDank je}\\

\haiku{dit prijsgeven van,,...}{jezelf is het grootste het}{mooiste het hoogste}\\

\haiku{O... ga jij nu ook,......}{al weg Ben en ik wilde}{je juist vertellen}\\

\haiku{naar kalenders met.}{jaren en dagen moeten}{ze kunnen vluchten}\\

\haiku{En 's avonds kwam hij,...}{je halen en je praatte}{samen in het park}\\

\haiku{het blaasje richtte,,......}{zich op het zette uit het}{zwolen sloeg uiteen}\\

\haiku{Het moet zoo geweest,,.}{zijn of jij waart hier nu niet}{met mij in mijn schoot}\\

\haiku{David was na die.}{ziekte van voorverleden}{jaar nooit meer gezond}\\

\haiku{hier een man, die een...}{klein meisje vermoordt voor haar}{gouden belletjes}\\

\haiku{{\textquoteleft}Pas maar op, dat je.}{niet met paraplu en al}{de hoogte in gaat}\\

\haiku{Nu ga je dan ook,,.}{allebei voor straf onder}{den mantel vandaan}\\

\haiku{Het sterft je af -, maar,.}{zoo is het niet je laat het}{achter op je weg}\\

\haiku{* Zij zijn daarvoor nog...}{niet eens dicht genoeg bij het}{witte vuur geweest}\\

\haiku{Daar staat Heleen -, ze.}{heeft mij al gezien en ze}{weet dat ik Eva ben}\\

\haiku{En dat David als...?}{dichter begaafd was heb jij}{dat ooit geweten}\\

\haiku{En David sprak het,,.}{ook niet tegen uit trots uit}{onverschilligheid}\\

\haiku{het staat buiten de.}{intellectueele en de}{ethische problemen}\\

\haiku{Ze wilde mij den,.}{brief in mijn handen geven}{maar ze bedacht zich}\\

\haiku{Omdat het ons zoo, -!}{aangrijpt waar het ons aanraakt}{als het onszelf raakt}\\

\haiku{Wonderlijk bestaan,,.}{waarin juist dit verzwegen}{wordt vergeten wordt}\\

\haiku{Maar laat het zoo zijn,.}{laat het kunnen dat de een}{voor den ander boet}\\

\haiku{Jaap en Ben deinsden,...}{ervoor terug en toen was}{Eddy geboren}\\

\haiku{Want eerst ga ik nu...}{naar vader en moeder en}{straks ga ik naar huis}\\

\haiku{Dat we er trotsch op,?}{zijn dat we ons haast schamen}{als het anders is}\\

\haiku{als het hunne in {\textquoteleft},...{\textquoteright}}{de spanning van hetHerder}{laat je schaapjes gaan}\\

\haiku{Ik liep zoo veilig,,:}{ik liep vlak achter Eddy}{en Claartje ik dacht}\\

\haiku{{\textquoteleft}Ik sta ervan te -, -,}{kijken ik hield jullie voor}{gelukkig getrouwd}\\

\haiku{{\textquoteleft}Kunnen we niet eens,,...?}{samen een heelen morgen}{een heelen middag}\\

\haiku{Misschien zoo ver zelfs,}{nog niet ik ben zoo snel van}{daar naar hier komen}\\

\haiku{Maar menschen dienen.}{t\'och behoorlijk aan elkaar}{te zijn voorgesteld}\\

\haiku{Je loopt sinds lang de,.}{kermistent voorbij je lacht}{om de muizenval}\\

\haiku{{\textquoteright} {\textquoteleft}De allergrootste,,.}{dwaasheid op die eene na op}{die andere na}\\

\haiku{Uit de starende,,}{sterren viel het woord mij toe}{maar dat was later}\\

\haiku{maar in mijn begrip,.}{kon ik het toch niet binden}{nooit kreeg het een zin}\\

\haiku{benaming voor de:}{eerste fietsmodellen.402l'Alsace}{et la Lorraine}\\

\haiku{de aangehaalde:}{dichtregel komt uit een van}{de sonnetten.514Ellen}\\

\haiku{alleen zijn zoon Cham,,.}{die zijn dronken vader had}{bespot werd vervloekt}\\

\haiku{6:4, die ook op het.}{rolletje perkament in}{de mezoezo staat}\\

\haiku{{\textquoteright} verklaarde ze in.}{1915 in een interview met}{Andr\'e de Ridder}\\

\haiku{Ook haar huwelijk}{met de sterk van zijn eigen}{gelijk overtuigde}\\

\subsection{Uit: De verlatene}

\haiku{'t Jongste broertje,.}{keek haar donker aan stompte}{haar tegen den arm}\\

\haiku{Jaren daarna stierf.}{de vader en had hem geld}{en zaak gelaten}\\

\haiku{Hijgend rende hij,,,.}{zonder omzien niet hoorend wat}{ze hem najouwden}\\

\haiku{{\textquoteright} Jozef keek verschrikt.}{op en beurtelings zijn broer}{en zijn vader aan}\\

\haiku{De meisjes waren.}{op een wenk van haar moeder}{al naar bed gegaan}\\

\haiku{{\textquoteright} Met de anderen.}{verheugde Daantje zich op}{den komenden avond}\\

\haiku{Jozef was nog niet,;}{aan tafel moeder had zijn}{bord eten toegedekt}\\

\haiku{Esther groette met.}{een nog hooger kleur en het}{kind groette terug}\\

\haiku{tot laat 's avonds zat.}{ze Jozefs uitzetje}{te beredderen}\\

\haiku{Van moeder hield ze,,.}{wel maar om wat di\`e zei gaf}{ze heelemaal niet}\\

\haiku{Daniel merkte z'n,.}{Joodsch accentje dat hij}{thuis niet had gehad}\\

\haiku{En nu weer naar de,.}{vijfde met een prijs en de}{jongste in z'n klas}\\

\haiku{Maar de moeder lag.}{daar met gesloten oogen of}{ze niets had gehoord}\\

\haiku{de oude heer was.}{ongetwijfeld de rijkste}{en dat bleef hoofdzaak}\\

\haiku{De drie sloegen hun.}{jaskragen op en liepen}{onverstoord verder}\\

\haiku{Zou\"en ze je dat,,....}{nou \`erg kwalijk nemen je}{familie en de}\\

\haiku{Zoo zal het nou op,,!}{die nieuwe school als ik er}{kom ook wel weer gaan}\\

\haiku{Dan zouden ze toch!}{stellig niet meer Jodin en}{smaus durven schelden}\\

\haiku{al was het voelen,,,}{meer dan beseffen dat zij}{hoe broos haar bestaan}\\

\haiku{Ik ga nou met Moos....}{praten en morgenochtend}{dan spreek ik met jou}\\

\haiku{Voor de rest ging hij,?}{z'n gang maar hoefde vader}{daarvan te weten}\\

\haiku{Een souper van den,,,.}{kok van Wertheimer die w\`el}{fijn maar zoo duur was}\\

\haiku{Nathans is geen fijne,.}{kok Nathans is een gewone}{koosjere bakker}\\

\haiku{Schrille en rauwe '.}{roepen klonken uit boven}{t confuus rumoer}\\

\haiku{David luierde.}{in een hoek en las in z'n}{hemdsmouwen de krant}\\

\haiku{Is het zijn schuld, dat?}{z'n broer en z'n zuster}{slecht zijn geworden}\\

\haiku{In waarachtigheid,,. '}{hij had gedacht dat vader}{toen gek ging worden}\\

\haiku{n Paar dagen was,.}{hij gebleven toen moest hij}{terug naar z'n werk}\\

\haiku{En sinds leidde ze.}{het gewone leven van}{zoovele vrouwen}\\

\haiku{Van Gulik zou daar,,.}{den eersten tijd althans een}{rustkuur doormaken}\\

\haiku{Ze was ook blij, dat,,}{Jozef trouwen ging voor hem}{en voor Rebecca}\\

\haiku{Vader scheen nog maar,.}{voor \'e\'en ding te leven voor}{zijn geloof alleen}\\

\haiku{Hij had zich verhard}{en hij verhardde zich meer}{en meer en alles}\\

\haiku{Ze sprak over hem, vond,.}{Roosje alsof ze met hem}{was getrouwd geweest}\\

\haiku{belangstellend kwam.}{haar van z'n vroolijke oogen}{de blik tegemoet}\\

\haiku{Dat kind deed nog wel,.}{haar plicht dat kind zat nog wel}{vast in het geloof}\\

\haiku{Dan zou ze haar brood,.}{hebben en geen zorgen dan}{zou ze veilig zijn}\\

\haiku{Het leven had haar,.}{nog niets gegeven dit leek}{haar te veel opeens}\\

\haiku{Hij had dankbaarheid,.}{verwacht verblijding om wat}{er besloten was}\\

\haiku{Dien brief legde hij.}{den volgenden dag op Heins}{kamer en vertrok}\\

\haiku{In die momenten,}{voelde hij w\`elbewust welk}{een schrikkelijk spel}\\

\haiku{Daniel voelde zich,}{een volslagen vreemde in}{dien familiekring}\\

\haiku{Debora stapte.}{achter de toonbank vandaan}{en haar in den weg}\\

\haiku{hou je in, hier op..,....}{straat ga mee naar mijn huis en}{vertel mij alles}\\

\haiku{Maar toen het meisje,.}{eenmaal verdwenen was was}{gauw haar toorn gezakt}\\

\haiku{Vast van plan was ze,,?}{het geweest maar goed beschouwd}{wat had ze voor r\`echt}\\

\haiku{- en Roos trouwde, al,?}{had-ze geen cent al was}{ze een Jiddekind}\\

\haiku{Nee.... nee.... nee, ze zou,.}{zich nergens mee bemoeien}{niet hier en niet daar}\\

\haiku{En.... w\`at zei\"en ze,,?}{tegen mekaar wat deden}{ze samen die twee}\\

\haiku{Nee, nee, Debora,....}{was een vreemde Debora}{had nergens mee noodig}\\

\haiku{Z'n heele leven,.}{was hij trouw ter Sjoel gegaan}{nou kon hij niet meer}\\

\haiku{{\textquoteleft}Ja...., er zal toch met?}{je vader over gesproken}{dienen te worden}\\

\haiku{Hij wist, dat het uur.}{gekomen was en dat hij}{eenzaam sterven zou}\\

\haiku{Rudi en Roosje,.}{ontwaarden hem z\'o\'o bij hun}{binnentreden}\\

\subsection{Uit: Het verspeelde leven (onder ps. Justine Abbing)}

\haiku{Ze droomde niet meer,,.}{vooruit het heden had haar}{weer ze liep vlugger}\\

\haiku{maar waarom niet, toen,?}{tante de eerste maal vroeg}{een smoesje bedacht}\\

\haiku{Een zuchtje wind deed,;}{het klimop lispen verre}{auto's toeterden}\\

\haiku{wat had hij dan dat?}{mensch in haar graf nog voor te}{trekken boven haar}\\

\haiku{Die pias in zijn,.}{zijden kuitenbroek zooals Wim}{hem altijd noemde}\\

\haiku{dan praatten ze over.}{politiek en hij wist het}{meest te vertellen}\\

\haiku{Ik vind wel eens dat '....}{hijt een beetje in de}{kleinigheden zoekt}\\

\haiku{{\textquoteright} Haar wangen bleven,.}{rood terwijl ze Ida naar de}{deur begeleidde}\\

\haiku{was iets van echte,.}{warmte en hartelijkheid}{nu hij haar groette}\\

\haiku{{\textquoteright} noodde de vader, {\textquoteleft} '?}{en tot zijn vrouw lachendKan}{t niet op moeder}\\

\haiku{nu zijn denken het, '.}{jongste meisje beroerde}{wast weer vlak bij}\\

\haiku{Tante liet vragen.....}{of Ied vannacht hier slapen}{mocht enkel vannacht}\\

\haiku{de eerste dag dat,....}{haar nichtje in de stad was}{haar bij wildvreemden}\\

\haiku{daar kan dedokter,....}{of de zuster mee meten}{hoe ziek of je bent}\\

\haiku{{\textquoteleft}'t Staat nog al in '!}{t gedeelte dat je voor}{je A2 moet kennen}\\

\haiku{{\textquoteleft}Maar nu gaan we naar,,!}{het dorp je weet wel bij Spoel}{en daar eten we ijs}\\

\haiku{- en dan nog boeken,!}{te lezen buiten voorschrift}{dokters-boeken}\\

\haiku{Wat zag je er toch,.}{best uit in dien tijd en wat}{kon je vroolijk zijn}\\

\haiku{Ze leek op mevrouw,.}{maar ze moest een heel ander}{karakter hebben}\\

\haiku{U zoudt eens zien, wat.}{hij zoo'n meisje tot haar recht}{zou laten komen}\\

\haiku{Zullen we haar dan,?}{eens op de koffie vragen}{een dezer dagen}\\

\haiku{Ze heeft wel eens een,.}{vrijen dag dien kan ze dan}{bij ons doorbrengen}\\

\haiku{Jawel, en ze ligt.}{voor de verandering weer}{eens met Coen overhoop}\\

\haiku{wat laat je toch die.}{kinderen zich blij maken}{met een doode musch}\\

\haiku{En waarom zou hij '?}{t niet uit fatsoenlijkheid}{gelaten hebben}\\

\haiku{Och neen{\textquoteright} hortte het, {\textquoteleft}....}{eindelijk over Ida's lippen}{zeg maar aan je ma}\\

\haiku{toch.... als ze niet zou,.}{hebben geweten wat ze}{gesproken hadden}\\

\haiku{{\textquoteright} Marie was blijven,,.}{staan verbluft haar verhaal nog}{lang niet ten einde}\\

\haiku{Ik geloof niet dat....}{Roomsch of Protestant er iets}{mee te maken heeft}\\

\haiku{Ze houdt van Ida, ze,.}{heeft alles voor haar over dat}{zegt Ied altijd zelf}\\

\haiku{Ze praatte meer tot,.}{zichzelf dan tot Ida die dan}{ook geen antwoord gaf}\\

\haiku{{\textquoteright} Marie raapte haar,.}{lorgnet uit het zand het eene}{glas was gebroken}\\

\haiku{en eigenlijk ook, {\textquoteleft}{\textquoteright} {\textquoteleft}.}{omdat een dame het had}{aangelegd metpa}\\

\haiku{Zie je, Brand is niet,.}{kwaad maar hij houdt wel wat heel}{erg van zijn gemak}\\

\haiku{{\textquoteright} {\textquoteleft}Ze behandelen ', '.}{ze overt algemeen toch}{goed zouk zeggen}\\

\haiku{O, hoe zou dat in,?}{elkaar zitten hoe mocht dat}{allemaal wel zijn}\\

\haiku{Niemendal dan een,....}{brief van een man die vroeger}{op je heeft getrapt}\\

\haiku{of het barsten zal,....}{zoo ellendig-kapot}{voel ik me ineens}\\

\subsection{Uit: De vier jaargetijden}

\haiku{Want al wist je wel,.}{dat ze naderden dit kwam}{toch weer onverwacht}\\

\haiku{Z\'o\'o eten zij het nooit,,,}{maar ze lachen niet want ze}{begrijpen het wel}\\

\haiku{Uit het booze land is, ...}{hij voortgestooten en is}{naar hier gekomen}\\

\haiku{Daar zwelt hij nu weer ...}{naar haar wangen en maakt haar}{mond tot stikkens vol}\\

\haiku{je spookt zoo benauwd}{en zoo rusteloos rond of}{mijn mond veel te klein}\\

\haiku{de wind doet buiten ...}{de boomen fluisteren en}{ritselend zuchten}\\

\haiku{Ze gaat naast ze en ...}{luistert en ziet haar schoenen}{onder zich stappen}\\

\haiku{En zulke zullen,,!}{er zijn en meer dan \'e\'en voor}{ieder op het feest}\\

\haiku{ik zal dit en dat,, ...}{ik ga hierheen en daarheen}{ik doe zoo en zoo}\\

\haiku{Nu is het al zoo,,}{dicht bij dat het wolkend stof}{tot haar lippen komt}\\

\haiku{Hij zal zijn paard niet, ...}{slaan dat kan ook best het mooie}{wagentje trekken}\\

\haiku{O neen, zoo niet, niet ...}{iets dat zoo dicht is en zoo}{afsluit als een deur}\\

\haiku{{\textquoteleft}Meteen schreeuw werd de,.}{zeeman wakker hij had zijn}{eigen graf herkend}\\

\haiku{het roepen van de, ...}{mannen die ze lossen en}{tot vlotten vormen}\\

\haiku{En daar komt het weer, ...}{aan en zwelt als een knop en}{die breekt voor haar open}\\

\haiku{hoe houd je het uit ...}{elkaar in de volheid van}{zoo'n overvollen dag}\\

\haiku{Ze staan nu samen ...}{voor het portretje tegen}{den donkeren muur}\\

\haiku{-, je voelt immers dat,}{Grootvader het heel goed merkt}{en mijnheer De Beer}\\

\haiku{Een woord schrijf je op, ...}{als je bang bent dat het je}{weer ontglippen zal}\\

\haiku{En als Jaap hem toch,!}{missen kan zou ze hem zelfs}{graag willen houden}\\

\haiku{De avond was nabij ...,}{en het huis straalde lamplicht}{uit al zijn kamers}\\

\haiku{Maar hij liet het niet.}{blijken en staat nu half met}{den rug naar haar toe}\\

\haiku{de menschen denken ....}{dat ze loopen kunnen dat}{loopen geen kunst is}\\

\haiku{Er wordt niet openlijk, ...}{gefluisterd niet rechtstreeks met}{vingers gewezen}\\

\haiku{Hij staat voor haar, hij ...}{reikt lang en breed boven haar}{en buiten haar uit}\\

\section{Kees van Bruggen}

\subsection{Uit: Als ge niet.... dan!}

\haiku{Vreemd bleef de jonge.}{vrijwilliger zijn vijand}{aan zitten kijken}\\

\haiku{Hij keek eens, en de.}{ander antwoordde met een}{pijnlijken glimlach}\\

\haiku{Als 't er aanzat '.}{namen wet er in de}{faubourg ook goed van}\\

\haiku{Die schiet je een gat,.}{in hun jas of je spit ze}{aan je bajonet}\\

\haiku{De soldaten die,.}{hier lagen waren niet van}{zijn eigen partij}\\

\haiku{Een zwarte ziekte.}{leek het levend gewas te}{hebben aangeteerd}\\

\haiku{- Het zal drie dagen,.}{geleden zijn hervatte}{de Engelschman}\\

\haiku{nu zouden woord en.}{begrip ook nimmer meer uit}{zijn hoofd verdwijnen}\\

\haiku{- Weet ik 't! - Misschien ',.}{ist mij ook zoo gegaan}{peinsde de ander}\\

\haiku{Ik meende ook te, {\textquoteleft}{\textquoteright}...}{zeggen dat  wat wij zoo}{beschaving noemen}\\

\haiku{- Mogelijk niet heel,,:}{en al wanneer we den draad}{volgen Denk eens na}\\

\haiku{Zou een generaal,?}{ons het v{\'\i}nden van deze}{plek verbeteren}\\

\haiku{Ze heeft mij alleen!}{geroepen toen ze mij noodig}{had om te moorden}\\

\haiku{De meester op school.}{deed z'n best een klassekind}{van me te maken}\\

\haiku{Ik mocht m'n meid niet.}{verdedigen tegen den}{agent die haar sarde}\\

\haiku{Ik moet - 't is om -!}{te stikken van den lach ik}{moet voor h\`a\`ar vechten}\\

\haiku{Kennelijk was de.}{vijfde man de aanvoerder}{der vier soldaten}\\

\haiku{Maar zijn verlamde,}{tong kon niet zeggen wat hem}{ontroerde en zoo}\\

\haiku{Door zijn trawanten,,:}{gevolgd ging de graaf op zoek}{aldoor roepende}\\

\haiku{Dat eene tergende, -...}{beest voor zijn oogen dat beest het}{wilde iets van hem}\\

\haiku{Hun werk en hun aard,.}{was vredig zij leefden een}{zachtzinnig leven}\\

\haiku{- waarin verschilden?...}{de nieuwe bewindvoerders}{van de vroegere}\\

\haiku{- Heeft de wereld \`ons...?}{of hebben wij de wereld}{teruggevonden}\\

\haiku{Men hoorde van den,.}{Franschman een woord dat veel}{op een vloek geleek}\\

\haiku{Om ons heen bleef hij;}{cirkelen bij al onze}{voorbereidingen}\\

\haiku{Mijn reisdagboek staat.}{vol met de merkwaardigste}{aanteekeningen}\\

\haiku{Zij drukte zijn hand,,:}{die over het bed hing en ging}{innemend verder}\\

\haiku{Aan onzen kant was.}{het inderdaad zooals we bij}{hen onderstelden}\\

\haiku{Geschreeuw... kreten... en,,...}{de lichamen gigantisch}{dansten in den mist}\\

\haiku{Een figuur was, toen,.}{zij dit zeide tusschen de}{tenten omgegaan}\\

\haiku{het geschenk harer.}{lieftalligheid onder hen}{allen verdeelde}\\

\haiku{Als vier wielen een,,.}{wagen zoo droegen zij den}{Graaf hunnen meester}\\

\haiku{Hij voelde iets van,.}{een kneep in de keel alsof}{hij zou gaan snikken}\\

\haiku{Hij, sinds de zieke,.}{genezen was voelde geen}{lust tot den arbeid}\\

\haiku{In zijn zelfverwijt.}{verachtte en benijdde}{hij hen tegelijk}\\

\haiku{Toch bleef er iets als,.}{verbazing in hem dat het}{ook hier gebeurde}\\

\haiku{Tegen onzen wil?}{u klinken in het staal of}{binden aan touwen}\\

\haiku{Want iedere straf,.}{die wij verzinnen voor u}{keert zich tegen \`ons}\\

\haiku{Schreeuwend met telkens,.}{overslaande stem eischte}{hij gerechtigheid}\\

\haiku{Groot en blond was hij,.}{zijn hoofd stond achterover van}{uitdagende jeugd}\\

\haiku{V\'o\'or het vertrek zou.}{ik echter nog Een spannend}{avontuur beleven}\\

\haiku{in hoeverre had?}{men hen te beschouwen als}{krijgsgevangenen}\\

\haiku{mondaine Franschen;}{met bonte slipdassen en}{gespleten baarden}\\

\haiku{men begeleidde.}{de sprekers met steenworpen}{naar het station}\\

\haiku{Achter de vaandels.}{van den Keizer barst een heet}{volk de grenzen over}\\

\haiku{Men zag een hunner:}{herhaalde malen met een}{ruk van zijn groot hoofd}\\

\haiku{Op een overheidlooze,.}{kolonie intusschen was}{niet gerekend}\\

\haiku{Hij had geleerd te, ().}{dulden en het woorddat men}{z\`elf sprak was \'o\'ok iets}\\

\haiku{Zou er iemand reeds,?}{wakende zijn met wien zij}{aanknoopen konden}\\

\haiku{Voor de eerste maal;}{in z'n leven nederig}{van binnen werd hij}\\

\haiku{De vergadering;}{vond veel pleizier in deze}{bedaarde leukheid}\\

\haiku{De gevolgen zijn ';}{voor u. Maar de kring was reeds}{aant verloopen}\\

\haiku{Vooral de Fransche,,.}{gezant een causeur had wil}{van zijn episode}\\

\haiku{De vingers plaatsend,,:}{tegen elkaar bedenkt hij}{zich en hij vervolgt}\\

\haiku{Ieder onzer - let! -...}{op krijgt twintig gepelde}{amandelen v\'o\'or zich}\\

\haiku{men had den feestmast;}{op het groene middenveld}{niet omvergehaald}\\

\haiku{Zijn lichaam, in de,.}{vaart gestuit stuipte over den}{grond en bleef stil}\\

\subsection{Uit: Fontana Marina}

\haiku{Het blonde meisje,.}{volgde glimlachend om de}{grote lieve man}\\

\haiku{Er  zou dan wel {\textquoteleft}{\textquoteright},.}{een juffrouwin het spel zijn}{wilden zij zeggen}\\

\haiku{In die zin was er,.}{Toet's aanvaarding van de man}{die haar Man zou zijn}\\

\haiku{een diepe voor, daar.}{stegen te weerskant wallen}{naar een dunne lucht}\\

\haiku{{\textquoteleft}Als er nu, Arend{\textquoteright} - zij - {\textquoteleft}...}{had zijn arm gegrepeneen}{grote vinger kwam}\\

\haiku{Er was nu niets meer.}{dan een trap van in de rots}{geplante scherven}\\

\haiku{Als een paard schraapte.}{hij de teelaarde van de}{harde ondergrond}\\

\haiku{Gedwee volgde zij.}{haar man door het poortje naar}{de weg terug}\\

\haiku{Tussen die korrels,.}{drong het water daarbinnen}{bleef het machteloos}\\

\haiku{Daar staan al twee of,...?}{drie passagiers die verder}{willen Zie je ze}\\

\haiku{Nu, toen waren wij.}{onverwacht genodigd bij}{een kikkerbruiloft}\\

\haiku{Zo lang zij samen.}{waren had zij hem nimmer}{reden gegeven}\\

\haiku{Evengoed kon ieder,,.}{ander vrouw of vriend hem met}{tegenspel dienen}\\

\haiku{{\textquoteright}, je hand beklopt haast,}{zonder dat je het weet zijn}{schouder dan voel je}\\

\haiku{Sinds trok zij met hem,,,.}{op ze reisden wandelden}{deelden hun armoe}\\

\haiku{Een naam hoorde er,.}{niet bij maar alles zag hij}{duidelijk terug}\\

\haiku{Wat zal ik 'n schat '!}{vann ouwe dame van}{vijfenvijftig zijn}\\

\haiku{{\textquoteright} De lange kerel.}{sprong dat zijn kop stiet tegen}{de zwarte hemelkruin}\\

\haiku{Arend vond het genoeg,.}{hij had geen zin zo met dat}{zotshoofd door te gaan}\\

\haiku{Hij zou straks eensklaps.}{uit de struiken opduiken}{en haar daar zien staan}\\

\haiku{Teleurgesteld rees,.}{hij omhoog hersjorde zijn vracht}{op  zijn bochel}\\

\haiku{Jaap Sparkels ezeltje.}{ijlbenig in het licht en}{droeg het schilderij}\\

\haiku{Ze moeten er zijn,.}{anders blijft alles los in}{de leegte hangen}\\

\haiku{{\textquoteleft}Toet zal vanmiddag,{\textquoteright}.}{voor me poseren bedacht}{hij zonder overgang}\\

\haiku{Chateaubriant had,{\textquoteright}:}{gelijk voer Arend zonder zijn}{spot te horen voort}\\

\haiku{Van het ogenblik dat.}{je weggaat blijft er een gat}{waar je geweest bent}\\

\haiku{{\textquoteright} zei Jaap geduldig, {\textquoteleft}.}{maar laat me niet afleiden}{van mijn onderwerp}\\

\haiku{Ik kan de koffie.}{van vanmiddag opwarmen}{met wat geitemelk}\\

\haiku{Het werd zo stil in,.}{de kamer of een gordijn}{was dichtgeschoven}\\

\haiku{{\textquoteright} {\textquoteleft}Ik zou er eens over,{\textquoteright}.}{moeten denken behield Jaap}{zich zijn mening voor}\\

\haiku{{\textquoteleft}Blijf me overigens;}{met de grootvaders van ons}{bedrijf van het lijf}\\

\haiku{Daarom vrijden de.}{Schuckenscheuer Schiefsalheims}{alleen bij kaarslicht}\\

\haiku{Neem dus in godsnaam,,.}{de blonde Lorelei die}{er nu toch is mee}\\

\haiku{De levende muis {\textquotedblleft}{\textquotedblright},.}{houdt de katin form zoals}{de sportslui zeggen}\\

\haiku{Hij leek nog verder,.}{te willen gaan maar stuitte}{plotseling zijn vaart}\\

\haiku{{\textquoteleft}Wat heb jij vandaag,?}{voor de onsterfelijkheid}{gedaan Don Carlos}\\

\haiku{Er wordt druk gedicht,,.}{en veel geschilderd dat wel}{als u dat bedoelt}\\

\haiku{Het tegendeel is,.}{waar die heren hebben het}{alleen over zichzelf}\\

\haiku{{\textquoteright} De dichter richtte.}{zijn duikerskop verwonderd}{naar alle kanten}\\

\haiku{Er zal niets anders,,.}{op zitten dichter dan op}{jezelf gaan wonen}\\

\haiku{Buiten in het licht.}{hing aan een dode twijg een}{vergeten amandel}\\

\haiku{Een amandel van het,?}{vorige jaar of kon het}{alreeds zijn van dit}\\

\haiku{Ze zou hem zeker,.}{te eten vragen al keek de}{man hem de deur uit}\\

\haiku{De hemel, of wie,,.}{daar aansprakelijk voor is}{bewaar me ervoor}\\

\haiku{Je moet de schuld nooit,.}{bij jezelf zoeken leerde}{de politieker}\\

\haiku{Banale wijsheid...?}{van een dubbeltje de tien}{wie zei dat ook weer}\\

\haiku{Je bent op mij, je,,,,.}{vrouw je kind je voetveeg je}{niets ben je verliefd}\\

\haiku{{\textquoteright} Zij tolde als een.}{mannequin op de spil van}{haar lange benen}\\

\haiku{{\textquoteright} {\textquoteleft}Zodat iemand in;}{een geestelijk vacuum niet}{zou kunnen denken}\\

\haiku{De beek ging voort te,.}{huppelen van de rotsen}{dat was haar bedrijf}\\

\haiku{{\textquoteright} {\textquoteleft}O, ons geweten,{\textquoteright}.}{staat erbuiten wist de man}{naast haar onverstoord}\\

\haiku{De man  boven.}{stond als een boomstomp in de}{wit-hete middag}\\

\haiku{Dag en nacht zwoegt hij.}{in de tredmolen zijner}{minderwaardigheid}\\

\haiku{Aandachtig was zij,.}{bezig met de primus haar}{eeuwige vijand}\\

\haiku{Het heeft geen zin het.}{bijzondere geval te}{veralgemenen}\\

\haiku{{\textquoteright} riep zij verbijsterd, {\textquoteleft},...{\textquoteright} {\textquoteleft}?}{Ugo de DuitserWat is er}{nu alweer met h\`em}\\

\haiku{Balorig werkte,.}{hij zich door zijn examens heen}{studeerde niet af}\\

\haiku{Zij glimlachte in,.}{herinnering maar de glans}{gleed van haar kaken}\\

\haiku{Arend Hobbe was de.}{eerste die woorden nodig}{oordeelde en vond}\\

\haiku{Het groepje mensen.}{stond daar als een disparaat}{Cooksgezelschap}\\

\haiku{Wat mij betreft, mijn,.}{postuur leent er zich weinig}{toe stel het je voor}\\

\haiku{{\textquoteright} hield Jaap querulant.}{vol. Zijn ogen stonden scherp in}{hun smalle kassen}\\

\haiku{Maar het ging niet om,.}{Hortense het ging om dat}{Blauwe Vrouwtje hier}\\

\haiku{Nu jullie eenmaal,.}{hier zijn verlang ik dat wij}{als vrienden leven}\\

\haiku{{\textquoteright} {\textquoteleft}Kom hier zitten, Toet,{\textquoteright},.}{nodigde Jaap begerig}{naar haar nabijheid}\\

\haiku{De eerwaarde heeft,.}{het goede met ons voor laat}{ons dat waarderen}\\

\haiku{Alleen de schilder,.}{zag het hij werd bedroefd en}{woedend tegelijk}\\

\haiku{Werd zij getrokken?}{of duwde een geheime}{drang haar van Arend weg}\\

\haiku{Op ieder ander,.}{uur misschien want je bent mooi}{en ik vind je lief}\\

\haiku{{\textquoteleft}Hortense was niet,,,?}{eens een sprookje en jij Toet}{bent jaloers van haar}\\

\haiku{God schiep de mensen,.}{en de kippen daar was geen}{tussenkomen aan}\\

\haiku{De Rector zelf met.}{zijn zwarte kapje stond voor}{het ijzeren hek}\\

\haiku{Nu wil je zeker,{\textquoteright}, {\textquoteleft}?}{giste hij achterdochtig}{ook weten waarheen}\\

\haiku{En wijselijk ook,.}{zij zouden genoodzaakt zijn}{naar hem te zoeken}\\

\haiku{Zeker betekent,{\textquoteright}.}{het niet dat jij te veel was}{verzekerde hij}\\

\haiku{{\textquoteright} smeekte het meisje,.}{een lieve glimlach schetsend}{die niet gelukte}\\

\haiku{Niets gedaan hier, in,.}{een natuur die alleen om}{de natuur gaf}\\

\haiku{Ik ben gebleven.}{die ik was op de dag toen}{ik ben gekomen}\\

\subsection{Uit: De freule}

\haiku{Door het floers van het,.}{witte tulle-gordijn}{verscheen de vreemde}\\

\haiku{Daar ginds... u kunt ze,...}{juist zien achter de gazons}{bij de oranjerie}\\

\haiku{En weer zweeg hij, naar '.}{t scheen gedompeld in zijn}{landschapbeschouwing}\\

\haiku{Uit haar reserve,,:}{die haar als kunstmatig bleef}{hinderen vroeg ze}\\

\haiku{Hij moet schilder zijn,,,,.}{maar hij heeft niets bij zich geen}{verf geen doeken niets}\\

\haiku{En in een woede,.}{die zij niet meer bedwong hief}{zij haar parasol}\\

\haiku{En weder, met zijn,:}{zekeren glimlach sprak hij}{haar gedachte uit}\\

\haiku{Ik leef ze, en waar,.}{ik tegenstand vind kom ik}{voor mijn rechten op}\\

\haiku{{\textquoteright} Maar dit zou wel het,.}{ergst zijn wat hij kon doen dat}{voelde hij ook wel}\\

\haiku{Nadenkend stond hij,.}{er mee sloeg het dan peinzend}{en gewichtig om}\\

\haiku{Mag ik mijn huis  ,?}{bewonen dat veel te groot}{is voor mij alleen}\\

\haiku{De dingen waren,.}{zoo en het was goed dat de}{dingen zoo waren}\\

\haiku{Wanneer jij nu ik, -?}{was en ik was bijvoorbeeld}{jou wat zou je dan}\\

\haiku{Want uit haar eigen:}{wezen begon zij opnieuw}{grenzen te bouwen}\\

\haiku{Hij zou zich wonden,.}{hij zou zijn overwinnenden}{glimlach verliezen}\\

\subsection{Uit: De geschiedenis van het huis. Een verhaal van vele avonturen}

\haiku{een zilveren bal,.}{in een schiettent en die danst}{op een fonteintje}\\

\haiku{De Architect ging.}{over tot een verhandeling}{over Ruimte-kunst}\\

\haiku{Men fluisterde dat'.}{de koning zich bedronk aan}{Samos zoeten wijn}\\

\haiku{Ook heeft hij reeds vaak,.}{met den architect gewerkt}{men begrijpt elkaar}\\

\haiku{Dat was een heele,.}{geschiedenis men kon nooit}{weten wie er kwam}\\

\haiku{riep de aannemer, '.}{een timmerman toe geef me}{s even je duimstok}\\

\haiku{Die hadden die groote! -,!}{scheur in haar tricot gezien}{Allemaal jongens}\\

\haiku{Het leek of ze haast,.}{hadden dupe te worden}{van het misverstand}\\

\haiku{Het was dan dat de.}{schoorsteen niet op de goede}{plek gemetseld was}\\

\haiku{Er is chromaat en.}{cadmium en siena en}{curcuma en oker}\\

\haiku{Want ook hij had zijn,,.}{meeningen al stond hem niet}{vrij ze te zeggen}\\

\haiku{Ze moeten wat, ze, {\textquoteleft},!}{kijken naar het neveltje}{ze zeggenwel wel}\\

\haiku{Aan den storm aan den,,.}{maneschijn aan den weg heen}{aan den we terug}\\

\haiku{Het loont niet, nog iets.}{omvangrijks aan te vangen}{en dat moet ook niet}\\

\haiku{Zou hem met honden.}{en geweren van het erf}{willen verjagen}\\

\haiku{Iets wat niemand, op,.}{straffe van er niet geweest}{te zijn mag missen}\\

\haiku{We komen weleens.}{terug als er kolen voor}{het proefstoken zijn}\\

\haiku{Misschien waren ze,.}{weleens kraamheer dan weten}{ze er alles van}\\

\haiku{Hij blafte als een.}{razende en sprong met vier}{pooten tegelijk}\\

\haiku{Het is dezelfde,.}{mijnheer alleen rekent hij}{een hooger tarief}\\

\haiku{De regisseur komt.}{alleen om te vertellen}{dat hij er niet is}\\

\haiku{De Bouwheer begon.}{zijn werk met alleenspraken}{te begeleiden}\\

\haiku{suste Jules, die.}{als wijsgeer afkeerig was}{van sterke woorden}\\

\haiku{De huisjes waren,.}{kleine rookfabriekjes ze}{brandden van binnen}\\

\haiku{- Nou, ik had nog wel,,, '}{voor een dag gehad maar het}{gaat een vaartje weet je}\\

\haiku{Maar de keuken is,.}{een werkplaats een museum}{en een boekerij}\\

\haiku{Die twee verstonden.}{elkander en elk woord was}{verder overbodig}\\

\haiku{hier zwaaide  een,.}{deur daar ginds kon het venster}{onmogelijk open}\\

\haiku{De stucadoor kwam,.}{om ze bij te strijken toen}{was de dag ten eind}\\

\haiku{Partijen en hun.}{vertegenwoordigers in}{rechten gingen heen}\\

\haiku{Dat hij dus den heer....}{Kantonrechter verzocht te}{willen bepalen}\\

\haiku{Hier krimpt een plank, daar....}{wringt zich een draadnagel als}{een pier naar buiten}\\

\haiku{Zij beleven de.}{disilluzie hunner}{laatste illuzie}\\

\subsection{Uit: Koning Adam}

\haiku{Heen en weder, heen,.}{en weder volgde hem ginds}{de glanzende schim}\\

\haiku{V De wereld om,.}{Adam was groot ontzaglijk vol}{vragen geworden}\\

\haiku{Hij scheen zijn evenwicht,,,.}{te hebben herwonnen en}{wijs vroeg zij hem niet}\\

\haiku{Hij kwam naar binnen,,.}{haar streelen over het klamme}{gemartelde hoofd}\\

\haiku{Den morgen bij het,.}{rijzen van het licht stond hij}{uit zijn leger op}\\

\haiku{In het zoekende:}{voortzwerven bezon hij de}{mogelijkheden}\\

\haiku{Het was haar nooit naar,.}{den zin te maken altijd}{grommen en brommen}\\

\haiku{Voorzeker had hij,.}{alle kipjes even lief en}{zij beminden hem}\\

\haiku{of de linde of,.}{de beuk die hun bloeisel in}{het blad verbergen}\\

\haiku{En uit het water}{glipten te allen kant de}{zilveren vischjes}\\

\haiku{Hoor, de wind staat stil,......!}{in de boomen een fluister}{gaat er van geluk}\\

\haiku{Haar verarmd lichaam,,.}{zij wist het ontstak hem niet}{meer in verrukking}\\

\haiku{daar zit alweer een, -?}{zusje meer wie zou me dat}{hebben gegeven}\\

\haiku{Zij gedroegen zich,.}{als meesters vroegen niet eerst}{en zeiden geen dank}\\

\haiku{{\textquoteleft}Weet je dan niet dat?}{het nest uit louter zulke}{balken gemaakt is}\\

\haiku{Wie zag het licht, wie,?}{de bergen wie de wouden}{vol geheimenis}\\

\haiku{Het gevaarlijke.}{wezen erkennen en voor}{de practijk negeeren}\\

\haiku{in Eva zou het niet.}{opkomen zich zelfstandig}{te laten gelden}\\

\haiku{Nog dieper viel de,.}{eenzaamheid om hem heen een}{regen van stilte}\\

\haiku{En vaster nam hij,.}{zich voor zijn exempel tot een}{wet te stellen}\\

\haiku{Zoo geviel het dat.}{alleen ouderen Adam's leer}{begrijpen konden}\\

\haiku{stammen over den stroom,.}{gesloten door lenige}{rieten verbonden}\\

\haiku{Zooals de arbeid, werd,;}{alras de jacht een doel om}{zichzelf nagestreefd}\\

\haiku{Heil de helden die!}{voluit gedijen kunnen}{naar het zonnelicht}\\

\haiku{Ik begrijp niets van,}{uw bedoeling en zie niets}{van uw heerlijkheid}\\

\haiku{{\textquoteright} zei Eefje nog eens,.}{en staarde in de leegte}{haar vraag achterna}\\

\haiku{Hij wenschte zeer.}{ernstig ziek te schijnen om}{te worden beklaagd}\\

\haiku{{\textquoteright} Nog eens, ijzig van,,.}{ontzetting gebood Adam hen}{alleen te laten}\\

\haiku{Uw oog, ten hemel,.}{gericht geeft de blinde voet}{aan de sprenkels sprijs}\\

\haiku{Men scheurde hem de,.}{kroon van het hoofd men wrijtte}{tegen zijn gezag}\\

\haiku{Wie het paradijs,.}{bezitten leven in vrees}{het te verliezen}\\

\haiku{Herinnert ge u,?}{Adam's wetten die van hemzelf}{hun oorsprong namen}\\

\haiku{- Wie nu haast heeft moet,,.}{weg gaan maar wie luisteren}{wil moet gaan zitten}\\

\haiku{Als het te koud is,.}{of te vuil of de bodem}{is droog geloopen}\\

\haiku{hij leeft, en het is.}{een hard lot de schuldenaar}{van een slaaf te zijn}\\

\haiku{- Het gezag doet er,.}{verkeerd aan die menschen hun}{gang te laten gaan}\\

\subsection{Uit: Het leven van Joost Welgemoed}

\haiku{De reep chocola,,.}{die bij zijn boterham was}{stak hij naar haar uit}\\

\haiku{Ze zou zeker niet.}{naar z'n raar hoedje kijken}{en wel naar het lint}\\

\haiku{Jooske meende dat.}{ze in dien snellen draai naar}{hem gekeken had}\\

\haiku{Hij meende alles,.}{te begrijpen al wist hij}{niet met woorden wat}\\

\haiku{Dat het scheen samen {\textquoteleft},,!}{te hangen met de plaat van}{ach vader niet meer}\\

\haiku{Joost was bang van die,.}{havelooze jongens en ook}{van dronken menschen}\\

\haiku{op het prentje greep:}{een groote vuist Klein Duimpje beet}{en hief hem hoog op}\\

\haiku{Joost vond de Duitschers,.}{schurken die in hem en zijn}{bedrog geloofden}\\

\haiku{Coba, de oudste,,:}{die het huishouden deed vond}{noodig Joost te zeggen}\\

\haiku{Den volgenden dag.}{kon hij bezonnen nog eens}{alles afspinnen}\\

\haiku{Weken later kwam.}{de kwitantie en Coba}{was razend geweest}\\

\haiku{Eens had hij vader {\textquoteleft}{\textquoteright}.}{naar de beteekenis van dat}{woordschande gevraagd}\\

\haiku{Hij zelf voelde zich.}{daar een beetje armoedig}{en verschoven bij}\\

\haiku{{\textquoteright} {\textquoteleft}Een schandaal om je,{\textquoteright}.}{te vertoonen voor alle}{menschen meende oom}\\

\haiku{Het leek of ook hij.}{behoefte had de schuld van}{zich af te praten}\\

\haiku{Niemand zou kunnen.}{zeggen dat de familie}{niet het hare deed}\\

\haiku{Jooske kon ook weer,.}{wat zeggen nogeens verder}{tasten met een vraag}\\

\haiku{Joost, zoo naar tante,.}{ziende uit zijn stilte werd}{daarvan iets gewaar}\\

\haiku{Joost had een tijd, een,.}{zeer langen tijd niet meer aan}{zijn vader gedacht}\\

\haiku{Daarom heb ik je, -?}{de eer aangedaan je op}{te merken vat je}\\

\haiku{{\textquoteleft}Als je zoet bent, mag{\textquoteright}.}{je een dans van me. Joost had}{geen dansen geleerd}\\

\haiku{Mijnheer Zandock, de,.}{eerste correspondent kwam}{even meelachen}\\

\haiku{Dan stelde hij zich.}{voor dat Lund en Boldering}{w\`el zouden durven}\\

\haiku{Joost, om zijn antwoord,.}{verlegen zag plotseling}{de situatie}\\

\haiku{Liever hief  hij.}{haar beeltenis in het licht}{zijner aanbidding}\\

\haiku{Smartelijk bedacht.}{hij weleens dat hij alle}{kosten alleen droeg}\\

\haiku{Daar hoort een heele.}{dosis energie en kennis}{en ervaring toe}\\

\haiku{{\textquoteright} Dreigend zag oom Joost,.}{aan toch niet zoo tevreden}{met zijn oratie}\\

\haiku{Handel is de kunst,.}{winst te maken om spoedig}{rijk te worden}\\

\haiku{De afnemer mag}{niet weten voor welken prijs}{zijn leverancier}\\

\haiku{Voor die alleen is.}{het leven de moeite van}{het leven waard}\\

\haiku{Het  gedurig.}{wachtend op den uitkijk staan}{sloopte zijn krachten}\\

\haiku{Oom zou er toch niet?}{over denken hem schriftelijk}{antwoord te geven}\\

\haiku{{\textquoteright} Dadelijk werd het,.}{een relletje waar Joost zich}{haastig uit redde}\\

\haiku{hij zocht zijn coup\'e,:}{terug waar een familie}{was komen zitten}\\

\haiku{Het vreemde land ving.}{zijn bestaan niet eerst aan met}{Joost Welgemoeds komst}\\

\haiku{In de breede straat,}{gingen weinig menschen meer}{of misschien dachten}\\

\haiku{Zijn ziel schaamde zich.}{en zijn mond plooide zich tot}{een schamper verweer}\\

\haiku{Gaf hij opnieuw zijn,?}{bestaan aan een waanbeeld een}{vorm zijner wenschen}\\

\haiku{Als een verloren,.}{iemand treuzelde hij nog}{een paar straten om}\\

\haiku{Het wonder van zijn.}{terugkeer verscheen levend}{voor zijn verbeelding}\\

\haiku{Hij zag zichzelf in,;}{dit tafreel vergevend in}{zijn grootmoedigheid}\\

\haiku{{\textquoteright} Een goed gehumeurd.}{spotje klaterde na in}{zijn lachende keel}\\

\haiku{Hij legde de hand.}{op den koperen knop om}{de deur te sluiten}\\

\haiku{De huisknecht poetste.}{in zwijgenden ijver aan}{zijn schoenen verder}\\

\haiku{{\textquoteright} 'n Fregat van 'n,.}{vrouw  en alles correct}{van beide kanten}\\

\haiku{Voor het eerst was het.}{wijde huis zijner ziel zijn}{eigen eigendom}\\

\haiku{Het ego{\"\i}sme  .}{zou van de menschen vallen}{als een leelijk kleed}\\

\haiku{Dan sprong Joost omhoog,,!}{uit zijn stoel hij speelde den}{flinkerd den frisscherd}\\

\haiku{Maar het gelaten.}{antwoord op zijn bezorgde}{vragen ontkende}\\

\haiku{In ieder geval,.}{zou het in orde komen}{stelde hij gerust}\\

\haiku{Zij demonstreerden.}{in optochten als cijfers}{in een optelsom}\\

\haiku{Je bent met ons niet.}{gelukkig omdat je thuis}{niet gelukkig bent}\\

\haiku{Ontmoedigd, zette, -.}{Joost den kleine neer hij moest}{maar naar z'n bedje}\\

\haiku{Hij kreeg een aanhang;}{van lachers en negatief}{gerichte geesten}\\

\haiku{Ze konden voor zijn.}{part een zaaltje huren en}{daar samenspannen}\\

\haiku{Den langen lieven.}{dag praatten zij erover en}{kwamen tot geen eind}\\

\haiku{Moe, verschuchterd, bleef, -.}{hij in zijn honk opgejaagd}{beest in zijn leger}\\

\haiku{{\textquoteright} Glimlachend zette.}{de vreemde zich over Joost's}{vijandigheid heen}\\

\haiku{Wat verlangt iemand,?}{terug die zich de moeite}{geeft mij te helpen}\\

\haiku{Tot mijn spijt,{\textquoteright} zeide, {\textquoteleft}.}{hijkan ik u niet te veel}{inlichting geven}\\

\haiku{Ze mogen den muur.}{niet meer zien waarbinnen zij}{opgesloten zijn}\\

\haiku{Alles zagen zij,.}{en toch waren die menschen}{niet  gelukkig}\\

\haiku{Maar ook dit kon hun,.}{honger niet verzadigen}{hun dorst niet lesschen}\\

\haiku{Wanneer zij kozen,!}{hoevele malen kozen}{zij het onrechte}\\

\haiku{In het bosch hieuwen,}{zij de opene plek van de}{gevelde stammen}\\

\haiku{Doch velen hunner;}{verdroegen op den duur die}{sterke dosis niet}\\

\haiku{John Hurst grijnsde,:}{gooide in de kranten den}{satanischen spot}\\

\haiku{ter  beurze stond {\textquoteleft}{\textquoteright}.}{het fondsJoost Welgemoed niet}{ongunstig bekend}\\

\haiku{Hij zou de zaal doen,,.}{sidderen van zijn kracht zijn}{verachting zijn hoon}\\

\haiku{De handen, in de,;}{mofzakken van zijn duffel}{grepen en wrongen}\\

\haiku{Daar strompelden ze...}{met geweld omhoog op hun}{stampende pooten}\\

\haiku{De kranten maakten,.}{zijn rekening op trokken}{het passief saldo}\\

\subsection{Uit: Tweestroomenland}

\haiku{Verder gaande op,,:}{mijn tocht dien middag had ik}{het veilig gevoel}\\

\haiku{ik had mijn schrijfblok,.}{op de tafel voor mij de}{vulpen afgeschroefd}\\

\haiku{zich honderdvoudig,, -.}{ik raakte den tel kwijt was}{niet gereed jij ging}\\

\haiku{Ik was waarachtig,,,,}{mooi interessant ik was}{zelfs zeide mij een}\\

\haiku{Weldra volgden dan.}{ook lauwe thee en verdacht}{besmeerde crackers}\\

\haiku{Hoe ik het had, wat,,.}{ik deed hoe ik voelde wat}{mijn plannen waren}\\

\haiku{hij bewondert dat,.}{volle rijke blond het doet}{hem zinnelijk aan}\\

\haiku{Je lipje verborg,,}{de teleurstelling niet maar}{het was toch feest}\\

\haiku{Boven de dunne, -}{ijzeren doorgang die geen}{poort had men liep daar}\\

\haiku{verbeterde jij,.}{een beetje weterig want}{het was namiddag}\\

\haiku{Meteen begrijpend,.}{hoe wreed je was begon je}{zwaar toe te lichten}\\

\haiku{Het werd een heerlijk.}{samenzijn en van Olga}{spraken wij niet meer}\\

\haiku{Zonder opstand, maar.}{ook zonder verlangen denk}{ik daaraan terug}\\

\haiku{Zeker, sportief, frisch,,.}{jong en vroolijk was ze een}{levenslustig kind}\\

\haiku{- Doe ik toch! - Tante () '?}{ze zei Tante vind je mijn}{sweater nietn snoes}\\

\haiku{In onze warme -?}{huiskamer staat hun kasje}{zijn ze niet zeldzaam}\\

\haiku{Misschien, al haastte,.}{ik mij zou ik haar alreeds}{gestorven vinden}\\

\haiku{Zijn geheugen was -?}{verzwakt dat van de oude}{doktersdame ook}\\

\haiku{Hij had gedaan, wat,.}{hij kon naar beide kanten}{was hij verantwoord}\\

\subsection{Uit: De verlaten man}

\haiku{Men vond daarin te,,.}{loven men vond erin te}{laken zooals dat gaat}\\

\haiku{Ook niet toen... ook niet.}{toen d\`at had opgehouden}{mij verdriet te doen}\\

\haiku{Gerard van Overen.}{kwam mij de handen drukken}{als krentebollen}\\

\haiku{Vroeger, komt het mij, {\textquoteleft}{\textquoteright}.}{voor zou dateen pijnlijke}{vraag zijn geweest}\\

\haiku{Neen, eigenlijk bent.}{u er de man niet naar om}{ouderwetsch te zijn}\\

\haiku{Alle intellect.}{is naar het onderst van de}{ruggegraat gezakt}\\

\haiku{- Ik moet u mijn van,.}{Goghs laten zien u gaat}{door voor een kenner}\\

\haiku{Muzikaal, ze zingt,,.}{dertig jaar zal ze zijn en}{niet onvermogend}\\

\haiku{Ik ben al een half,.}{uur met mijn ziel bij jou wie}{weet waar ik straks ben}\\

\haiku{de narigheid zit,, {\textquoteleft}{\textquoteright}, -:}{geloof ik ingeachte}{zonder franje dan}\\

\haiku{Dank u. Zij weet haar,.}{houding niet hamert maar wat}{los op de toetsen}\\

\haiku{- spring ik uit mijn stoel,.}{treed met deftige stappen}{Blok's kamer binnen}\\

\haiku{En ik weet ook, dat.}{ik niet meer zooveel lust heb}{naar Itali\"e te gaan}\\

\haiku{Kan ik heengaan en:}{het meisje bij mij laten}{komen en zeggen}\\

\haiku{Thans kijk ik kalm daar.}{op toe als een geleerde}{op zijn preparaat}\\

\haiku{wij nemen er nog,.}{een onder gelijken die}{den rookbak vullen}\\

\haiku{iemand op wien elk,}{land trotsch zou mogen zijn hij}{is een man met wien}\\

\haiku{Mijn groot, hoog venster.}{staat vlak voor de hooge gele}{boomen van den tuin}\\

\haiku{Jeanne's aandacht.}{blijft zich scherpen daar in de}{buurt van het portret}\\

\haiku{- Ik kan begrijpen,.}{dat je er tegen op ziet}{die te verlaten}\\

\haiku{Wat wilde dit sterk,,,,!}{zijn zoo doelvast zoo vrij zoo}{geleerd zoo zeker}\\

\haiku{En een gedachte,,:}{volgend waar ik geen deel in}{heb zegt zij ineens}\\

\haiku{nu heb ik toch mijn...}{hand terug genomen en}{zij bemerkt het niet}\\

\haiku{De trap af, zwijgend,.}{geef ik haar tot de straatdeur}{mijn geleide}\\

\haiku{Mijn oude drift om,,.}{te vermooien idealen}{te zien is niet dood}\\

\haiku{Maar, lieve jongen,,.}{ik heb geen soldo meer geen}{soldo zeg ik je}\\

\haiku{Bij Maarten trok ik.}{gummi handschoentjes aan voor}{al het grove werk}\\

\haiku{Zoo is zij naar haar,.}{bestemming zoo ging het mij}{naar mijn bestemming}\\

\haiku{indien zij keerde,,!?}{op dezen oogenblik hoe}{zwak zou ik weer zijn}\\

\haiku{zij staan groot boven.}{de menschen in haar lange}{rijzige lijven}\\

\haiku{Zeer langen tijd zijn.}{alle kantoorgeluiden}{wat zij moeten zijn}\\

\haiku{- Philine, als je,,,.}{zoo doet niets zegt niets verklaart}{ga ik de deur uit}\\

\haiku{De handen wrijvend,,.}{dat ze zingen dribbelt hij}{op z'n dikke beenen}\\

\haiku{Smaragdgroen schuimen.}{de kruinen in het kobalt}{der hemeldiepten}\\

\haiku{Het lijkt of allen,.}{hun plek gekozen hebben}{maar de plek koos hen}\\

\haiku{Bovendien - neen, zij,.}{is niet van het soort dat men}{behoeft te ontzien}\\

\haiku{Tja, ik zie het wel,,,.}{hij wil uitdagen vechten}{zijn kracht beproeven}\\

\haiku{- Waarom, Tonio? - Om,.}{t\`och zegt Tonio en keert zich}{onhebbelijk af}\\

\haiku{Haar opbellen, een,,?}{briefje schrijven bezoeken}{haar noodigen bij mij}\\

\haiku{Verlang ik toch weer?}{naar de bekoring harer}{gezamenlijkheid}\\

\haiku{Weer plooit haar mond zich,.}{dun en zuinig als aan \'e\'en}{kant wreed getrokken}\\

\haiku{Hoop die geen kans op, -.}{verwezenlijking heeft maar}{eenmaal dan toch hoop}\\

\section{Max de Bruin, Eug\`ene Coehorst, Paul C.H. van der Goor, Jan Notten en Lou Spronck}

\subsection{Uit: Mosalect. Bloemlezing uit de Limburgse dialectliteratuur}

\haiku{welke criteria,?}{leggen we aan wat nemen}{we op en wat niet}\\

\haiku{Het sloog krek drie oere, ':}{in Staiveswji\`ert wie}{t vriek\`or aafmersjeerde}\\

\haiku{En es toe deze ' '.}{kie\"er neet int prezong}{k\"oms weit icht neet}\\

\haiku{Waat is hie te doon,,?}{vroog ter det geer den hemel}{zwoa geweldj aan dootj}\\

\haiku{Dao zoot van alles....}{innet veldj en de b\"os t\`ot}{waerw\"olf toe}\\

\haiku{E\"es leef zeen en dan,.}{temptere Drek tr\"okpakke}{wat ze e\"es gaeve}\\

\haiku{Leefde is get van,,!}{twe\"e luuj Neet van eine is}{leefde Van twe\"e}\\

\haiku{Leefde is get van,,.}{twe\"e luuj Neet van eine van}{twe\"e is leefde}\\

\haiku{Zwart den auto, 't, '.}{kleed ovver zien kist de minse}{diem w\`egbrochte}\\

\haiku{{\textquoteright} En d'r Pierre zag, '.}{gans bedrufd datt met d'r}{pap neet ez\`o good g\`ong}\\

\haiku{zie werrek oach waal '. '}{ns aangesjoa\"ete of}{zats Vriedes zow}\\

\haiku{de begreffenis}{zie\"e um ellef oe\"er want}{um tie\"en oe\"er}\\

\haiku{En went mich ouwesj...}{zage ku\"e- me dat}{h\"on kinger sjturve}\\

\haiku{De heemkier van d'r}{verloare zoe\"en}{Heerlens  He\"e hool}\\

\haiku{En wie op l\'ochte}{k\'olke dreef blinjelings zie}{verlange wir durch}\\

\haiku{Op ins doa huer...}{iech jet hinger miech roespere}{en sjnoespere}\\

\haiku{Opins doa huer...}{iech jet hinger miech roespere}{en sjnoespere}\\

\haiku{'nne remmel \'onger ' '.}{nnen erm enn sjmeel teussen}{\'ongerlup en sjn\'or}\\

\haiku{De muskes*~		 duun.}{d'r moier zinge en de}{schorsteen \^anders hule}\\

\haiku{alweer ei gedich.}{in ech en \'onverzawte}{Limburger dialek}\\

\haiku{ze ginge l\^egge '...}{En de j\^onges v\"osjde in}{t Gelei ~ Mer}\\

\haiku{{\textquoteright} {\textquoteleft}Auwat, iech bin doch ',.}{mar alling mit der man en}{t kink wat sjlie\"eft}\\

\haiku{{\textquoteleft}Dat is va wie iech;}{dizze n\'ommedieg plantse}{ha oes-jesjtaapelt}\\

\haiku{En wie iech zaat - iech -}{how al allerhank jedoa}{det iech d\'at nit mie\"e}\\

\haiku{e kriet de zieng{\textquoteright} en '.}{j\'ong noan kirch i en zats}{ziech in de sjtalles}\\

\haiku{{\textquoteleft}Das ist der Protest;}{des Himmels f\"ur den Tyrannen}{der Chorknaben}\\

\haiku{sjuvet-e jans}{jemuutliech de dr\"o\"aet woa-e}{knoake en hals uvver}\\

\haiku{Jedemfals vroaret,,}{der pap \'os wie vier heem koame}{of der ze\"eje}\\

\haiku{h\"uet em et kink r\"oft!}{em Bevreij mich en sjpie\"el}{mit mich voe\"egel}\\

\haiku{t Sjuntse voong iech ' '.}{t sjtuk woa me op de naat}{ant waade is}\\

\haiku{Doatussje durch.}{is nog ing verloting mit}{sie\"er sjun prieze}\\

\haiku{op de weasjdr\"oa\"ed,!}{Jraad boaver mie hemme}{vilt mienne blik}\\

\haiku{Soe eine gooije}{sul moot waul meijne dat het}{Mastreegs get slegts is}\\

\haiku{H\"obstoe ins tie koeraazje,.}{Dan krijg tien kakedoe Ouch}{fris get op zien oere}\\

\haiku{Wee admireert neet,?}{hun braaun aaugen Hun sniewit}{velke wie satien}\\

\haiku{Aaug vindt me ter van,}{alle soorten En heet me}{ze wie dat me wilt}\\

\haiku{Te willen speulen;}{een presentie En onder}{d'aaugen van de maan}\\

\haiku{Van naau de mes wat,}{veul te goon En liever een}{de kerk te zitten}\\

\haiku{Anneke Scheuffers;}{is gestorven En met heur}{aaug tin aauwen tied}\\

\haiku{Ig bin altied vol,}{helgen iever En dikwils}{heub ig nog min daaug}\\

\haiku{De luy zien nou en,;}{daan zou aaurdig En t is aaug}{deks neet umme zus}\\

\haiku{De bron wou me zich;}{een kaan zuvren Wie voul en}{zwart tat men aug is}\\

\haiku{Veel mig toch ins ein,;}{out tin hiemel Ze kwaaum mig}{excellent te pas}\\

\haiku{Ig moes in huiske;}{bouten heubben Modest en}{propel gemeubleerd}\\

\haiku{Dou heubst te veul en;}{ig te min. D'ambitie zou}{mig aug neet kwellen}\\

\haiku{De witte greemsels;}{bleven likgen En reurden}{zich neet een de heuk}\\

\haiku{De baaum wet neet wee,;}{dat em plantde Et blaad wet}{neet wou dat et weyt}\\

\haiku{Us altied haauwen,.}{by et Good Op tat ver stil}{en zaaulig sterven}\\

\haiku{m\`et de eterneel*~		 :}{strikkous en heer sjreef in}{groete l\`etters}\\

\haiku{d'n ingaank van et {\textquoteleft}}{str\"a\"otsje ene paol en dao}{leunde noe Zjann\`et}\\

\haiku{Koelek had er evels,}{zene mond opegedoon of}{Janneske st\'ont veur}\\

\haiku{{\textquoteleft}zoene lieleke......{\textquoteright}, {\textquoteleft}{\textquoteright}.}{lieleke ze k\'os gei woord}{lielek gen\'og vinde}\\

\haiku{als een ei 3,5 cent,.}{kost hoeveel verdient dan wel}{die kip in een jaar}\\

\haiku{zoe vas opein, tot.}{zene mond achter de punt}{van z'n neus zaot}\\

\haiku{andere sloog ze.}{in de gawwigheid ene}{knien in zene nak}\\

\haiku{d'n onderein, um.}{ziech evekes te verpoeze}{van et dr\"ok get\'offel}\\

\haiku{heij z\`ette, daan '.}{g\'ong er kallek hoole en bleef}{miechn haaf oor eweg}\\

\haiku{{\textquoteright} - {\textquoteleft}Dan mooste miech ouch,.}{mer medein de sj\`elderije}{ophange Stiene}\\

\haiku{Iech waor ouch aon '.}{t witte wie noe en dao}{loerde dee sjijns op}\\

\haiku{Versjeije politieke,,,;}{haone Die zien ocherrem}{ouch mer maone}\\

\haiku{Dan geit 'ne sjt\'ok*~		 , '. '}{door aal wat leef Ent hart}{van blij emotie beef}\\

\haiku{Oraanje leech stijg;}{langs de gevel en sjittert}{in de hoegste roet}\\

\haiku{'n Ierste blaad luut los;}{en kantelt en r\"os veurgood}{op kawwe stein}\\

\haiku{Heer kraog middesijn,.}{en m\^os goon ligke en moch}{gei beer mie drinke}\\

\haiku{sjoen st\"ok preuf en k\"a\"ort.}{in ze gehiel en in zen}{apaarte motieve}\\

\haiku{{\textquoteright} Et \'onweer leet nao,.}{dee slaag neet aof ieder woort}{et nog erreger}\\

\haiku{Heer blaosde en '.}{t leve Dat greuide in}{bieste en plante}\\

\haiku{Meh es iech miech neet '.}{t'rin vergis Daan waort}{veur alle twie}\\

\haiku{{\textquoteright} infermeerde Jaan, ' '.}{ne keerel wiene boum en veur}{ter duuvel neet bang}\\

\haiku{er waor taamelek.}{zwoer en kleddere veel em}{neet toe erreg m\`et}\\

\haiku{Toen b\'okde-n-er.}{zech en voolt er tot er de}{kop in zen hand heel}\\

\haiku{Wie toen 'n {\textquoteleft}uigske{\textquoteright}...,...?}{ene glimlach oet deeg springe}{Wat toen geb\"a\"orde}\\

\haiku{zoe versjrikkelik es,}{mennigein dee nao d'n oos}{is goon vare neet}\\

\haiku{de s\`okkerbekker, ' '}{vaan de stad dee doort gans}{Zuid-Limburgs land}\\

\haiku{Dao passeerde v'r {\textquoteleft}'{\textquoteright},}{t Drifke dat altied eve}{levend ze water}\\

\haiku{De vroului pakde}{hun kedokes oet en de}{kerels droonke ziech}\\

\haiku{{\textquoteright} en inpessant leep, '.}{heer de deur oet umt keend}{te goon aongeve}\\

\haiku{Heij beg\^os Marie:}{weer haos obbenuits te}{kriete wie ze zag}\\

\haiku{Zen start t\"osse zen,.}{pu gedreve Sl\"op heer nao}{hoes tou met belump}\\

\haiku{Iech zaot al in!}{de twiede klas en Mia g\'ong}{al nao de veerde}\\

\haiku{Iech weet neet wat m\`et '.}{m'ch geb\"a\"ord is en iech h\"ob}{t noets gewete}\\

\haiku{Wie mie pijn tot 't.}{deeg wie beter tot iech m'ch}{beg\'os te veule}\\

\haiku{{\textquoteright} Iech keek verweze.}{op oet mie book en deeg of}{iech vaan Sint Jaan kaom}\\

\haiku{{\textquoteright} Iech doog de sjaw d'r, ':}{op en wis neet wat iecht}{ergste zouw vinde}\\

\haiku{Alle sjtel haw hae,.}{vol vieje zitte en de}{weije leepe vol deere}\\

\haiku{Dat is 't biste,.}{oonder famille allein}{kirremis hawte}\\

\haiku{Hae dach, 't zunt jong.}{luuj en hae deeg dat altied}{nao de kirremis}\\

\haiku{Och, doe moos weete, ', '.}{waet vieje haat kin ooch}{t vel verwachte}\\

\haiku{Job 'ns \'a reegel '.}{kristeliere enm d'r}{kee\"etel sjoore}\\

\haiku{Dae vergit neet, dae '.}{haett neet oonder de sjoon}{gesjrie\"eve}\\

\haiku{Slivvenhier dae sjat ',*.}{dich opn hawf oons dae knoept}{sjuus op d'r cent}\\

\haiku{Doe waors jao sjterk,,.}{wiej twieje sjterk wiej mosterd}{doe reets al biejeen}\\

\haiku{Doe wits toch, Job, dao:}{zunt driej dinger diej neet zunt}{te besjrieve}\\

\haiku{Is 't 'ne rieke, ' ',}{of istne erreme}{Dae in de hil k\"omp}\\

\haiku{Doew sjarde de aw ' ':}{kloek get drek int loo\"ek}{en zong zet sjlot}\\

\haiku{Ja, wae zaet 't. D'r.}{weend doog nog neet dae van d'r}{Waalekaant k\"omp}\\

\haiku{Kaakele is niks,,.}{kaakele kint eekereen}{mae eijer legke}\\

\haiku{haw naogesjikt es.}{dae zich te boete g\'ong en}{neit loestere w\'ol}\\

\haiku{Dao waore de.}{luuj oet den \'omtrek in de}{Krisnach haer komme}\\

\haiku{{\textquoteright} Sevrien kos niks meei.}{z\^egge en pakde Drikus}{mit einen erm vast}\\

\haiku{Hae sjtamelde get {\textquoteleft}?}{t\`egen Sevrien vanWat mot}{ich mit dat dink doon}\\

\haiku{in 't sjwart, mager,.}{en bleik koum d'r oet en ging}{dao nao gen hoes in}\\

\haiku{En d'r Hoeb\`e\`er ' '.}{kiektns \"a\"overn sjowwer}{g'n sjtraot aaf}\\

\haiku{Es Sliekkop miens waas, '...}{gewaes z\`ow haet inne}{politiek zeker}\\

\haiku{Sliekkop zitj oppe ',.}{hoogste plaats vant eilandj}{mit dieke klaagkael}\\

\haiku{Morgevreug gaon{\textquoteright},.}{v'r hae str\`ek zien vol pens oet}{en geit slaope}\\

\haiku{Oetgevlooktj,*}{zatte zich inne vaarleis}{en wajje}\\

\haiku{Ae zaat kwaolik*:}{of ein zwelped~		 kroop doort}{st\"ortgaat en zawgt}\\

\haiku{En gaeftj mich n\'ow,{\textquoteright}.}{eur zie\"el s\'onger van eur}{schatte te preuve}\\

\haiku{ae m\`et de vlegel,.}{det bi-j edere slaag ein}{krej oet de lougt veel}\\

\haiku{Kos-ter waal.}{hondert kriege En waor}{toch no\`ets kontent}\\

\haiku{Hae keek nog ins om, '.}{schaterde vant lachen}{en sneei zijn deur in}\\

\haiku{Veer sproken neet, veer,...}{zongen neet Veer hadden dao}{toe ouch gein raeijen}\\

\haiku{Toen w\`ol Oswald auch}{zie vader op dezelfde}{meneer oet d'n tied}\\

\haiku{En haaj doe de moel,{\textquoteright}*.}{doe b\"os nog te zeer eine}{gaaplaepel}\\

\haiku{Waat sjpeelt ei kiendj d\"ok, ', '.}{door de kop Zo ist braaf}{zo ist sjtout}\\

\haiku{{\textquoteright} Rrrroebedoebs, '.}{de trap op aan de k\`erk}{sjuut weern wej}\\

\haiku{*~		 taege, en noe ' '!}{maakde mich det uulskuuke van}{ne Wulmne sjlaag}\\

\haiku{- Allaa j\'onges, 'ne!}{sjteevige gedr\'onke op}{de goojen oetsjlaag}\\

\haiku{Hae zelf waar 't meis.}{in de wolke en sjpr\'ong wie}{ei j\'onk veule r\'ondj}\\

\haiku{En h\"obs-toe nog?}{waal ins geloesterd nao Roussels}{veerde symfonie}\\

\haiku{Alling de vrouw van ',.}{d'r Joe\"ehan witt ma die}{sjwiegt wie inne dup}\\

\haiku{{\textquoteright} {\textquoteleft}D'r gantse daag in ',?}{d'r winkel vant Lieske}{woa kals doe u\"uver}\\

\haiku{{\textquoteleft}Ich be Pi\`ejr}{va Fien Dabbs oet Geb\"osjelke en}{kom vraoge of ich}\\

\haiku{Waal veel ich d\`ek i,,.}{sjlaop meh dat kaom omtot Mam}{zoea dru\"ag veurb\`ede}\\

\haiku{En alles waat \`os.}{nog naobie waar Woord lankzaam}{vraemd en bleik en vaag}\\

\haiku{{\textquoteright} - dan waar ze weier '.}{gans Calypso en veilm}{in sjt\`orm in de erm}\\

\haiku{Mer hae maagde dat., {\textquoteleft}{\textquoteright},.}{In wirkelikheidau fond}{lachde hae mit h\"o\"or}\\

\haiku{en noe zuut me auch:}{al weier dat et is wie}{et gez\`ekde zaet}\\

\haiku{God bewaar, es et,.}{mer gein \`onwaer geef ee dat}{ze weier dao zeen}\\

\haiku{'t Haet niks eweg van,.}{Nikela zeliger God}{gaef h'm d'n hemel}\\

\haiku{nog riepe proeme.}{gegaef en noots h\"ob ich mit}{mien naober get ghad}\\

\haiku{De Richter waor.}{eine Mastreichtenaer en hoelj}{ens gaer van ein grap}\\

\haiku{H.D.        Mieng sj\"onste.}{sjiech   Spekholzerheids D'r}{Sjtaat tuutet tsing vuur zes}\\

\haiku{Hae waar al ens nao,.}{B\`ellevend gefietst maar dao}{waar niks te vinge}\\

\haiku{In de kortste tied.}{ware weej in Kastele}{beej de pastorie}\\

\haiku{{\textquoteright} - Eur houegmood steeg ten,, '}{top Es geer o minst gel\"ok}{haatj om te melje}\\

\haiku{{\textquoteright}, zag Berb, {\textquoteleft}kuns ies um '}{de wesj dru\"eg te kriege}{est de gansen}\\

\haiku{Harie vertrok en ',}{oft zoa moes zeen trof hee}{in de sjtraot}\\

\haiku{De beugel waor,.}{verros de gum van den dop}{bienao vergange}\\

\haiku{Obb\`ens geit de deur ':}{aop en kump de knech vant}{Kraneveld binne}\\

\haiku{Ens op ennen daag.}{had d'r zich eemus z\'on paar}{wanne weggepak}\\

\haiku{Mien Meelkop        Storm,;}{Venloos  zwart wu\"ert}{de l\'och kaolzwart}\\

\haiku{De z\^on leet nog in.}{de wol en de waereld zuut}{oet wie beschummeld}\\

\haiku{{\textquoteleft}Waat is d'r now weer{\textquoteright}, '.}{vru\"eg de vrouw en geit zich}{naev'nm zitte}\\

\haiku{Weer wiegelde de.}{wage met eine nieje}{zenuwschoek wiejer}\\

\haiku{{\textquoteright} De deur die al half, '.}{toe waar knalde met enne}{slaag int slaot}\\

\haiku{Petran was ennen,.}{ieverigen boer den zien}{land goed verz\"orgde}\\

\haiku{, mer Petran verstond, {\textquoteleft}{\textquoteright}.}{mer hallef wat ze zaei en}{riep d\"orrum merjao}\\

\haiku{Op \`ens draejde, '.}{de weg zich en jaowel door}{h\'addet kruuspunt}\\

\haiku{Petran was van d'n}{ie\"ene kant b\'ang en van}{d'n \^andere kant}\\

\haiku{Hej kreeg d'r zin ien, '.}{want ie haj enen dru\"ege}{k\`el vant snurke}\\

\haiku{\`en vur dat ie 't,.}{eiges ien de gate haj}{haj ie ze al \`op}\\

\haiku{{\textquoteleft}Stopt er \`owen b\`ok ien,{\textquoteright}, {\textquoteleft}.}{Pier zei Toe\"enik zal um}{ow goed betale}\\

\haiku{Der Pitter zaat n\"uks,:}{e bezoog der Joep ins es}{wente zage wool}\\

\haiku{{\textquoteleft}E kamp os verdaat,.}{v\"or e kiekt graat of wente}{kaffie sjmoekelet}\\

\haiku{Da zalle zage,:}{wente v\`e\"edig is mit}{diej va Ie\"epe}\\

\haiku{Herregod iggen,{\textquoteright}.}{himmel laot mich sjtil}{en geduldig zie\"e}\\

\haiku{Pastoe\"er ka gee...}{wo\`ad mie\"e oetbringe en dao}{in ins ene sjlaag}\\

\haiku{Noe geet 't op en '.}{aaf uvvern heen en probeere}{ze get te versjto\`a}\\

\haiku{V\"a\"or mees te lane}{hoft me ging sjl\`e\`eg toe te}{do\`a en da hat me}\\

\haiku{gank e-weg, en,.}{dan i zie hats nit twiefelt}{d\`e wet verhoe\"ed}\\

\haiku{Diej zonge \`eve hel ' '.}{naot oksaal truk en ezu}{gongt hin en weer}\\

\haiku{In der aavank ersjaffet.}{God himmel en \`e\`ed en e}{maket alles good}\\

\haiku{Zoste dao kunne,,?}{pr\`edige Barthelomee in}{de Sint Servaos}\\

\haiku{{\textquoteleft}al kost 't os twei,{\textquoteright}.}{doezend gulde v\`er zulle}{dat offer bringe}\\

\haiku{{\textquoteright} H\`e\"er Bussjep,, '!}{went d\`er ins wust wat inn}{zie\"el kan umgo\`a}\\

\haiku{Noe is 't 'n koo,.}{diej miskoft dan e p\`e\"ed}{dat vervangen is}\\

\haiku{de drukdje, 't, '.}{zinge de mes en alt}{licht in de vol kerk}\\

\haiku{Zoea sjuufse wie 'ne,}{sji\"em van i\"elenj langs de}{straot Woea d'anger}\\

\haiku{Et Wimke wreef zich:}{van sjpas in gen heng en}{der melder fluidde}\\

\haiku{handschrift in bezit, ();}{van Gilles Jaspars Gronsveld}{Gilles Jaspars1937}\\

\haiku{Z.p., z.j. Blz. 16-18 () (-);}{Spelling herzien door R. Geurts}{Jos Wetzels18971967}\\

\haiku{dikke ijzeren ():}{staaf om gaten te maken}{in de grond sjt\"odig}\\

\section{Johan de Brune (de Jonge)}

\subsection{Uit: Wetsteen der vernuften}

\haiku{Want het geeft hun een,.}{kalm zelfvertrouwen dat hun}{vorming vervolmaakt}\\

\haiku{Mensen die in het}{spel volleerd dachten te zijn}{overwon hij voor ze}\\

\haiku{Het inwendige,,.}{van apen zegt hij is gelijk}{aan dat van mensen}\\

\haiku{Vonnis dat men bij.}{Papon leest tegen hen die}{aan die kwaal lijden}\\

\haiku{Elke zonde die,,.}{een mens begaat zegt hij is}{buiten het lichaam}\\

\haiku{Maar in plaats van te:}{antwoorden zei hij tegen}{zijn opdrachtgeefster}\\

\haiku{En hij is bang voor.}{blauwe plekken onder de}{druk van zijn handen}\\

\haiku{Gekscherend vroeg ik?}{of iemand niet verliefd zou}{worden op zo'n beeld}\\

\haiku{Ziedaar lezer, ik.}{eindig een beetje anders}{dan ik gedacht had}\\

\haiku{Voor droomuitleggers.}{betekenen parels niets}{anders dan tranen}\\

\haiku{Michelangelo.}{vroeg de man hoe hij in zijn}{onderhoud voorzag}\\

\haiku{Geestig verhaal over.}{Demetrius Cynicus}{en een dansmeester}\\

\haiku{Van wateren die.}{in steen veranderen wat}{erin wordt gegooid}\\

\haiku{De Brune stijgt in.}{zijn dichtwerk niet boven het}{gemiddelde uit}\\

\haiku{Het proza van Jan;}{de Brune is vertaald in}{modern Nederlands}\\

\section{M.J. Brusse}

\subsection{Uit: Boefje}

\haiku{Dat zag je wel meer, - - '...}{dacht ze zacht voor zich zelf dat}{t dan slecht afliep}\\

\haiku{{\textquoteright} - {\textquoteleft}Och, werom nou, 'k...{\textquoteright} {\textquoteleft} '!}{doe je nou toch ommers niks}{D'r isn meheer}\\

\haiku{Z'n breede schouders;}{stonden wat naar voren als}{om zwaar te duwen}\\

\haiku{- Moe, geef Lientje 'n,...}{happie suiker ze het weer}{pijn in d'r kiesie}\\

\haiku{{\textquoteright} - zei die klabbak - {\textquoteleft}bij ',.}{je moer int secreet daar}{kan je ze scheppe}\\

\haiku{daar heije de meester!}{en da's die smeris die me}{laast het opgebrocht}\\

\haiku{{\textquoteright} - {\textquoteleft}Nou, 'k denk, ik ga, '.}{an de stroomtram hange dan}{benk er gauwer}\\

\haiku{Nou, d'r sting net 'n,:}{man  voor en toe had Jan}{zoo leukweg gezeid}\\

\haiku{Als die ouwe je,:}{pakte sloeg ie je raak op}{je tweede gezicht}\\

\haiku{Doch ie dat ie 't?}{fernuisie nogeris weer}{voor d'r zou poese}\\

\haiku{Maar Pukkie mos ook...}{altijd met bossies houtjes}{voor z'n moeder loope}\\

\haiku{{\textquoteleft}k\`o, snotneus, ga na, ';}{je moeder en vraag omn}{kouwe aardappel}\\

\haiku{- Fort, wat let me, of '...{\textquoteright}}{k schop je na Schiedam met}{je schele klosooge}\\

\haiku{- Maar 'n oogenblik,}{later belde nie weer want}{ie had stilletjes}\\

\haiku{{\textquoteleft}Gewerkt... de heele...!}{nacht bij de boere gewerkt}{en c\`ente verdiend}\\

\haiku{Ze moeder zei nie',...}{veel maar Onze Lieve Heer}{hoorde d'r bromme}\\

\haiku{Als 'n kind dat bang,;}{is stond ie klein dicht tegen}{z'n vader aan}\\

\haiku{Daar hadden ze uit ';}{een rijk huisn dienstmeid zien}{gaan om een boodschap}\\

\haiku{{\textquoteright} en hij was zoetjes ',...}{t portaal door de kamer}{binnengeslopen}\\

\haiku{en op z'n gezicht '.}{n zweem van narigheid t\`och}{onbillijk vinden}\\

\haiku{{\textquoteright} - Jawel, dat dorst dat;}{lieve Jantje tegen ze}{m\'oeder te zegge}\\

\haiku{s Middags was tie.}{effe na bove komme}{schreeuwe om ze brood}\\

\haiku{En toen 't rookte, ':}{tikte hij technisch metn}{vinger an z'n pet}\\

\haiku{hij 's van de trap...{\textquoteright} {\textquoteleft};}{gevalle en toe moeder}{kwam kijkeJawel}\\

\haiku{Want de st\`ad, met al,...}{die verleiding is de hel}{voor zoo'n degener\'e}\\

\haiku{lekkere dikke ',:}{scholletjes voorm en dan}{mot Lientje zegge}\\

\haiku{Nou, zij kon ook wee;}{van de zenuwe worde}{as ze d'r an docht}\\

\haiku{Hij zal zoo in z'n, ',...}{knolle wezet kind as}{ie z'n vader ziet}\\

\haiku{Even keek ie wat d\`at,,.}{nou weer zijn zou schuwe oogen}{dadelijk weer neer}\\

\haiku{En die jonge waar...}{we mee zitte is nog veel}{gemeender dan ik}\\

\haiku{Die andere vent, ', '.}{waark mee zit het gegriend}{toe iet hoorde}\\

\haiku{{\textquoteright} Och heire chut, d\`at;}{was nou altaad de r\`amp van}{d'r leife cheweist}\\

\haiku{ba meheir Dussert,, '.}{en \'alles four moe hour chein}{cintje voorm self}\\

\haiku{dan sting ie in 't, '...}{pikke donker zouwen ze}{opm afkomme}\\

\haiku{maar de Heilige,,.}{Maagd die kon alles hadde}{de manne gezeid}\\

\haiku{En dat vliegie op...}{de plank voor ze bankie mog}{ie nou niet vange}\\

\haiku{Toen Boefje uit de, '.}{gevangenis kwam wast}{een gesl\'agen kind}\\

\haiku{En 't meiske sloeg ', ':}{r armpjes om z'n hals en}{trokm naar zich toe}\\

\haiku{dat je zoo'n kind nou,,;}{toch maar most late gaan h\`e}{as ouwer zijnde}\\

\haiku{Ja, 't was de goeie,.}{trein hoefden v\'o\'or Rosendaal}{niet over te stappen}\\

\haiku{Ik heb ze tot nu - '.}{toe zelf les gegevenk}{heb mijn hoofdacte}\\

\haiku{Hij wist nog best den,?}{weg in  Rotterdam maar}{of ie werom wou}\\

\haiku{Of ik nog wel wist?}{van toe z'n heele boeltje}{was weggereje}\\

\haiku{Maar toe heb ik nog... '!}{v\'e\'el fainder geslape in}{n eerste  klas}\\

\haiku{{\textquoteright} Zoo, en of ie w\'e\'er, '?}{zoo zou beginnen als ie}{hier uitt huis kwam}\\

\haiku{zeg moeder gedag,!}{en vader en Lientje en}{Sientje en Mientje}\\

\subsection{Uit: Landlooperij}

\haiku{Vier dage heb 'k ' '.}{plat opn stoel enn stoof}{thuis motte zitte}\\

\haiku{k Had die kras \`an,,...... '}{kanne vliege zoo'n sloeber}{die zei da'k vet was}\\

\haiku{{\textquoteright} {\textquoteleft}De kaptein kwam an,,...}{boord viel over z'n eige beene}{zoo zat as tie was}\\

\haiku{En dat maakte mij.}{ook al niet monter om den}{tocht te beginnen}\\

\haiku{{\textquoteright} - proestte Toon uit, sloeg,...}{op z'n knie\"en en sprong m\`et}{van de leut overeind}\\

\haiku{'t Uitvaagsel krijg... ', '...}{je mee op die nachtboott}{Laagstet minste}\\

\haiku{In 't Nieuwediep......}{woont nog zoo'n stuk vrijer van}{mijn zeit die blonde}\\

\haiku{Maar half uitgekleed, '...}{lag hij nu ook op de bank}{en sliep alsn roover}\\

\haiku{zoo groezelbleek en.}{verflenst in die beslapen}{sjofele plunjes}\\

\haiku{Boog moeizaam z'n rug,.}{en waschte z'n gezicht}{met zwakke veegjes}\\

\haiku{{\textquoteleft}'t Is koffie, want.}{nou hebbe me meteen de}{grooste schooiers beet}\\

\haiku{{\textquoteleft}Och{\textquoteright} - galmde Hannes - {\textquoteleft} ' ';}{wat zak je daars van}{zegge me jonge}\\

\haiku{{\textquoteleft}Die ouwe het nou ' - -;}{alt vierde wijf onder}{z'n tande hoor Toon}\\

\haiku{daar drinke me 'n ',.}{slokkie en me  eten}{stukkie hoor Hannes}\\

\haiku{{\textquoteleft}Ja maar{\textquoteright} - suste de - {\textquoteleft} '.}{Mottigeje kan wels}{te driftig zijn maat}\\

\haiku{En er kwam 'n zwerm.}{groote bonten aanvliegen op}{den wind voor ons uit}\\

\haiku{As je 't nou hebt ',!}{overn vangst dan had ik er}{eentje vanochtend}\\

\haiku{Maar hij is niet te...{\textquoteright} {\textquoteleft}{\textquoteright} -;}{bezeileKappe dan maar}{galmt Hannes lijzig}\\

\haiku{{\textquoteright} - hitste gejaagd de - {\textquoteleft}...}{Mottigeze benne nog}{aldeur onder schot}\\

\haiku{{\textquotedblleft}Kom, jonges, zeg ik, '.}{nou je nog uitgesloofd voor}{n oogenblikkie}\\

\haiku{Nou zit je in 't ',;}{volst vant wild en nou}{hei je geen geweer}\\

\haiku{As je je oogen in,}{d'rlui dienst verspeeld had dan}{schopten de heeren}\\

\haiku{Zoo jaag jij ze op,,, '...}{zie je en as ik ze dan}{hoor legk ze neer}\\

\haiku{zie je, dan komt die,....}{duimspeling onder z'n kin}{deur en dan zit ie}\\

\haiku{Als Dirk nu strikken,.}{gaat planten dan stopt ie die}{tusschen z'n body}\\

\haiku{Maar toch \'o\'ok vond ik ';}{t wel gewichtig nu m\'e\'e}{te doen aan de jacht}\\

\haiku{Da's stroope, dus Piet en, '.}{Dirk die benne d'r bij zoo}{vast asn lessie}\\

\haiku{Want 'n veldwachter ';}{zien ik opn afstand van}{hier wel na Tessel}\\

\haiku{Toch hebbe me 's;}{eens uit zoo'n gat effetjes}{acht ramme gevierd}\\

\haiku{{\textquoteleft}Afnokke, maat, 't, '!}{is vijf uur in de morge}{me gevet op}\\

\haiku{Toe was ie weer blauw,,}{toe ie de kroeg uitgegooid}{wier en dan kon ie}\\

\haiku{een zak, z\'o\'o groot, dat.}{onze heele jol er wel}{in kon verdwijnen}\\

\haiku{En in een herberg.}{komen ze dra allemaal}{samen om te schooven}\\

\haiku{Ze slapen met vloed,.}{en visschen met eb zoolang}{de haringvangst duurt}\\

\haiku{En hunkerig liep:}{ie langs den steiger met de}{boot mee te schreeuwen}\\

\haiku{as zij dan snorke, '...}{vanaved het de Mottige}{t rijk weer alleen}\\

\haiku{Zoodra we weer,.}{zaten ging de Mottige}{met z'n verhaal voort}\\

\haiku{alles staat op, ze '...}{schreeuwen de kelners omm}{eruit te gooien}\\

\haiku{As 'k rijk was, keek ',}{k geen jenever meer \`an}{want dan bl\'e\'ef ik \`als}\\

\haiku{dus zij houden er,.}{zich buiten en laten de}{klungels maar prutsen}\\

\haiku{b\`e je nooi mal, want ',.}{zietn ander mijn drave}{dan holt ie me v\'o\'or}\\

\haiku{d\`an spring ik er af, '.}{en ga ast mot tot me}{strot onder water}\\

\haiku{Maar 'k zeg tege... ',.}{moeder de vrouw \'o\'okn poetje}{om van te bikke}\\

\haiku{Wat hier en gunder - '!}{al b\`enk nou in me \'e\'en}{en t\`achetigste jaar}\\

\haiku{{\textquoteright} - Die komt voor, en met ':}{t p'rtaalste snoet van de}{wereld zegt Leen toen}\\

\haiku{De justitie hieuw... ',?}{Neus daar all\'e\'enig nou voor}{t Kon wel zijn h\`e}\\

\haiku{Want de wind over zee,.}{is hun adem de golfslag h\`un}{heftige polsslag}\\

\haiku{Is nooit meer 'n zucht,...}{van gehoord geen krummel van}{an komme spoele}\\

\haiku{Als ze niets meer op.}{de wereld hebben om zich}{aan vast te houden}\\

\haiku{Dan zou 'k ook... dan ' '!...}{woukt an m'n eigen}{body ervaren}\\

\haiku{D\'a\'arom zat ie met zoo'n;}{ernstigen ijver al die}{dingen te wrijven}\\

\haiku{{\textquoteright} - gierde nie uit van, '.}{de pret en hij sloeg op z'n}{dij datt kletste}\\

\haiku{Van begeerte om, '.}{dat \`al dikker te maken}{staptek vlug door}\\

\haiku{Heel teer nam ie 'n.}{harmonikaatje tusschen z'n}{reuzige handen}\\

\haiku{Hoe kom ik d'r an,?}{as niemand wat neemt van me}{armoeiige koopwaar}\\

\haiku{of 'k jaag je de...{\textquoteright} '}{honde achter je gat Maar}{daar dichtbij kwam juist}\\

\haiku{{\textquoteright} - zei Toon hongerig,.}{toen ie weer zoo afgescheept}{werd door de boerin}\\

\haiku{want da 's nou niet,,?}{smakelijk wel waar jullie}{al van gehapt hebt}\\

\haiku{Maar in de dorpen,.}{heb je de winkels en dus}{verkoop je niet veel}\\

\haiku{maar nou heb 'k 'n}{jonge genome om mee}{de boer op te gaan}\\

\haiku{Genaved,{\textquoteright} - {\textquoteleft}manne{\textquoteright} - {\textquoteleft}{\textquoteright} - ', ' '.}{goeie kwamt loom terug in}{n zucht enn geeuw}\\

\haiku{daar kan je d'r 'n '}{veeg boter op krijge en}{n asempie kaas as}\\

\haiku{Maar moeder zei, met ':}{r schrille ruzie-stem}{liefjes geknepen}\\

\haiku{{\textquoteleft}Maar 'k zeg u, als,:}{den mensch kijkt in den spiegel}{dan kijkt hij eruit}\\

\haiku{{\textquoteright} {\textquoteleft}Dat gaat nogal{\textquoteright} - zei, ',.}{ik want nu dachtk dat ie}{gek was geworden}\\

\haiku{n wage bed\`ekt ',.}{metr koopwaar en  dat}{staat arremoeiig}\\

\haiku{of als die dronken?...}{jongen in een razende}{bui wil gaan vechten}\\

\haiku{De liedjeszanger ' '.}{was toch weln aardige}{kwast vann kerel}\\

\haiku{Hij keek mij triest aan,,:}{en terwijl ie somber z'n}{hoofd schudde zei ie}\\

\haiku{Buiten de slaapstee ' '.}{stond overt steegjen klaar}{gouden ochtend}\\

\haiku{Die welvarende;}{stallen slokten natuurlijk}{allen kooplust op}\\

\haiku{alsof ie plan had.}{onzen boel voor niemendal}{weg te gaan geven}\\

\haiku{daar opeens maar te '.}{staan improviseeren met}{n galmende stem}\\

\haiku{Want overal in de '.}{huizen scheent avondlicht al}{door de gordijnen}\\

\haiku{{\textquoteright} En ik antwoordde, ';}{alsof ikn kapitaal}{had te commandeeren}\\

\haiku{{\textquoteright} {\textquoteleft}Nou{\textquoteright} - zei de buurman - {\textquoteleft} '.}{erg geheimzinnigdan zal}{ik jes helpe}\\

\haiku{De kostbaas, 't hoofd,.}{in z'n handen op tafel}{wachtte ons slapend}\\

\haiku{En w\'a\'ar je je wendt,;}{raak je de lauwe kleeren van}{andere slapers}\\

\haiku{Maar ook zijn oogen zag,.}{ik langzaam verdraaien strak}{starend naar mij heen}\\

\haiku{Ik had 't raam al,.}{lang open en m'n valiesje}{er onder gezet}\\

\haiku{Waarom hangt ie dan ',?}{daar aant hout En waarom}{uit ie dan dien kreet}\\

\haiku{{\textquoteright} vroeg, 'n eind van mij, '.}{vandaann ouwelijke}{stem onderdanig}\\

\haiku{Maer na dertien jaer...}{boeten is mijn schaemte toch}{vrij wel verdwenen}\\

\haiku{'k Had toen nog nooit,.}{jenever geproefd en mijn}{vrouw nog veel minder}\\

\haiku{Dat maekte haer...}{jaloersch en toen is ze \'o\'ok}{dapper begonnen}\\

\subsection{Uit: Het rosse leven en sterven van de Zandstraat}

\haiku{Zaten er van die, ':}{opgeschoten werkjoggies}{tusschen dan wast}\\

\haiku{Dat zijn soms menschen, ':}{waar u of een andert}{niet van denken zou}\\

\haiku{Die er over blijven,;}{tobben met wrange vlagen}{van melancholie}\\

\haiku{Heel ding hoor, om daar.}{dan altijd zelf je jatten}{bij thuis te houden}\\

\haiku{water, voor nog 'n.}{paar centen heb je een lik}{groene zeep erbij}\\

\haiku{En zoo niet, dan den, ':}{boer maar weer op om buiten}{n slagje te slaan}\\

\haiku{Maar je wil toch ook.}{niet altijd aan je zaken}{worden herinnerd}\\

\haiku{En den anderen.}{nacht werd Dirk z'n deel in de}{Schavensteeg verzet}\\

\haiku{Changeerde direct,,.}{van hand in hand langs heelsters}{en opkoopers heen}\\

\haiku{dat opstootje daar,!}{da's nu toch net om den zoon}{van den Zot gedaan}\\

\haiku{Dronken kerel wordt,;}{deur uitgesmeten dat de}{ruiten rinkinken}\\

\haiku{We hadden dien avond;}{alweer lang genoeg in den}{Polder geloopen}\\

\haiku{{\textquoteright} De meid had een paar,}{aardige oogen en ze keek}{er den rechercheur}\\

\haiku{'t glipte zoo uit, '...,,?}{t lid hing er bij Ja h\`e}{hoe je zoo zijn kan}\\

\haiku{Ze is al nege, -,... '}{jaar met September geweest}{zoo'n lekkere meid}\\

\haiku{krijg de kel\'era{\textquoteright}, '.}{en zoo as dat dan ging as}{Chrisn brom in had}\\

\haiku{Maar ze dacht, dat ie... ';}{lag te stervent koude}{zweet op z'n gezicht}\\

\haiku{Nog maar twee jaar met,{\textquoteright} -.}{April dan komt ie weer los zei}{ze nadenkelijk}\\

\haiku{t Is verdorie...{\textquoteright} {\textquoteleft},,!}{temet half eenOch meid maak}{niet zoo'n kapsoones}\\

\haiku{Maar 's \'e\'en keer, dat ',':}{ie \`erg veel smul inr had}{toe had ie gezeid}\\

\haiku{en gezongen voor,...}{die gesjefte jonges van}{d'r die daar zitten}\\

\haiku{- Direct hoor ik 'n,,:}{raam open schuiven en Toosje}{die naar buiten roept}\\

\haiku{{\textquotedblleft}Bu'vrouw, bu'vrouw, kijk '?}{s effe wie daar voor de}{deur beneden staat}\\

\haiku{Dan zal je zien en.}{ondervinden Dat jij de}{Polder nie meer ken}\\

\haiku{Ze gaan de Zandstraat ' '}{netjes makent Wordtn}{kermenadebuurt}\\

\haiku{Bij Nielsen ken je.}{nie meer dansen Bij Charley}{zijn geen meisies meer}\\

\section{Cornelis de Bruyn}

\subsection{Uit: Reizen van Cornelis de Bruyn door de vermaardste deelen van Klein Asia, de eylanden Scio, Rhodus, Cyprus, Metelino, Stanchio, \&c., mitsgaders de voornaamste steden van Aegypten, Syrien en Palestina}

\haiku{Alle lofdichten,,.}{ook de twee Latijnse zijn}{dus weggelaten}\\

\haiku{s\c{c}avoir No. 16, No.,,,,, \&.}{21 No. 22B No. 22C No. 33}{No.37 No. 41 No. 187}\\

\haiku{Deze zwierf twee maal,.}{vijf jaren gene twee maal}{twee maal vijf jaren}\\

\haiku{Ce voyageur \'etoit,{\textquoteright}.}{un seigneur Allemand qui}{parcouroit lItalie}\\

\haiku{[BA/43] De genoemde:}{plaatsen in Europa zijn}{respectievelijk}\\

\haiku{Maar van wie deze,.}{anderhalve dichtregel}{is staat niet vermeld}\\

\haiku{[BA/49] Stambol zou een [][][].}{verbastering zijn vanCon}{stantinopolis}\\

\haiku{Cf. Gaspar (1998) 366,.}{en de erbij horende}{aantekening 36}\\

\haiku{Jean-Bernard, {\textquoteleft}.}{de VaivreAutour du grand}{si\`ege de 1480}\\

\haiku{Description de{\textquoteright}:}{Rhodes \`a la fin du XVe}{si\`ecle in}\\

\haiku{[BA/128] Deze obelisk.}{is omgevallen bij de}{aardbeving van 1301}\\

\haiku{{\textquoteright} [BA/169] Bedoeld is de ( {\textendash}).}{Franse classicus Claude}{Saumaise1588 1653}\\

\haiku{Ook voor hem was hun.}{winterverblijf tot dan toe}{een raadsel geweest}\\

\haiku{Zie Cornelis de {\textquoteleft},.}{Bruyn\={u}n Yakin-Dogu}{gezisi blz. 39}\\

\haiku{Ook in andere.}{woorden verandert vanaf}{boek II de spelling}\\

\haiku{II komt ze nog maar (),}{sporadisch voor9x swaar vs.}{25x zwaar in boek III}\\

\haiku{Opmerkelijk is:}{de veranderde spelling}{van een werkwoordsvorm}\\

\haiku{Boek IV vertoonde].}{geen enkele vindplaats met}{de klankwaarde s=[z}\\

\haiku{Hier scheidde van ons {\textquoteleft}{\textquoteright},.}{het schipNassau verordend}{na Scanderona}\\

\haiku{De jacht is hier een.}{dagelykze bezigheid}{en staat een yder vry}\\

\haiku{De mouwen van het.}{hembde zyn byzonder groot}{en met kant bewerkt}\\

\haiku{Doorgaans is dezen,.}{tulband van witte roode}{of geele zyde}\\

\haiku{verwyderd is van.}{de plaats daar die stad eertyds}{gelegen was}\\

\haiku{Dit woord van kiosk {\textquoteright} {\textquoteleft}{\textquoteright}.}{betekent int Turkscheen}{overdekte galdery}\\

\haiku{Mogelyk is het.}{Endimion en Diana}{die hem komt vinden}\\

\haiku{waaruit een yder zo.}{veel tapt als hy tot deze}{wassching van doen heeft}\\

\haiku{Fraaye gestalte der,}{Turken enz. Kleeding der Turksche}{mannen en vrouwen}\\

\haiku{Want hoe langer en,.}{fraayer van baard hoe een man}{meer in achting is}\\

\haiku{Gemeenlyk kan men.}{ze allerwegen by de}{Grieken bekomen}\\

\haiku{Doch hierover schiet my {\textquoteleft}.}{yts te binnentgeen ik hier}{tusschen moet voegen}\\

\haiku{toe schieten en valt {\textquoteleft}.}{onverhoeds in het nettgeen}{achter de schuit legt}\\

\haiku{yder die daar passeert,.}{moet geven onderhouden}{konnen worden}\\

\haiku{{\textquoteright}t Is een soort van,.}{zeevisch doch behoeft voor de}{baars niet te wyken}\\

\haiku{Aanmerkelyke.}{byzonderheden wegens}{de kameleons}\\

\haiku{Boek II   	Griekse.}{eilanden en Egypte}{XXXIIe Hoofdstuk}\\

\haiku{Ook geloof ik niet.}{dat er slechter zeelieden}{in de wereld zyn}\\

\haiku{Ik bemerkte wel}{dat het geschiedde omdat}{wy niet zouden zien}\\

\haiku{Tegenwoordig is,.}{het toegemetseld doch de}{plaats noch kennelyk}\\

\haiku{De Grieken roemen.}{dat zy noch een arm van dien}{heilig bewaaren}\\

\haiku{Ook is er in yder.}{een opening welke doorgang}{geeft na een kamer}\\

\haiku{welken naam zy ook.}{tot op den huidigen dag}{heeft behouden}\\

\haiku{{\textquoteright} Verschil tusschen de}{heer Thevenot en de heer}{Melton nopende}\\

\haiku{Ik hielp het op zyn.}{gezondheid eeten en vond}{het uitsteekend goed}\\

\haiku{Zy voerde de naam,.}{van Porta Superba of}{de Prachtige Poort}\\

\haiku{Deze vermaarde,.}{Vliet of Beek Kedron ziet men}{tegenwoordig droog}\\

\haiku{In deze zoude.}{hy zyne klaagliederen}{gemaakt hebben}\\

\haiku{Ziet, den engel des,:}{Heeren verschynt Joseph in}{den droom zeggende}\\

\haiku{5:1, alwaar de man,:}{Gods propheterende van}{den Messias zegd}\\

\haiku{Van hier heeft men een.}{schoon gezicht op Jerusalem}{en Bethlehem}\\

\haiku{dat men derhalven;}{geen voet tot een zo kwaaden}{misbruyk moest geven}\\

\haiku{dingen, achteden.}{zich ge\"eerd door dezelve}{te mogen wegneemen}\\

\haiku{Op denzelven, aan,}{het eynde van de zee legd}{het zo berugte}\\

\haiku{1193 Ter zyde, in,.}{een kleyne kapel ziet men}{drie altaaren}\\

\haiku{Beschryving van de.}{Galileesche Zee en de}{stad Tiberias}\\

\haiku{haar water is zoet,,.}{heel goed om te drinken en}{uitsteekend visryk}\\

\haiku{Ongemaklyke.}{ontmoeting van den auteur}{met den Aga van Tyrus}\\

\haiku{Ongemaklyke}{ontmoeting van den Autheur}{met den Aga van Tyrus}\\

\haiku{1268 Was een man van;}{vermaakelyken omgang en}{vrolyk van humeur}\\

\haiku{men zeyde dat zelf {\textendash}.}{1344 Zenobia een Joodin geweest}{is ontleend hadden}\\

\haiku{daar konnen er, als,.}{men maatig rekend niet min}{dan 560 geweest zyn}\\

\haiku{een gang liep, kruyswys {\textquoteright},.}{doort geheele gebouw}{regt in het midden}\\

\haiku{En het melden van.}{Decapolis maakt my nog}{meer verbystert}\\

\haiku{De datum daarvan,, {\textquoteright}:}{554 wyst aant begin van}{deze rekening}\\

\haiku{het beest daar ik op,,.}{zat nederzette bleeven}{er tien of twaalf dood}\\

\haiku{{\textquoteright}t Overige der.}{kleding is op de wyze}{der Turkinnen}\\

\haiku{Met den avond naamen wy.}{onze verblyfplaats aan een}{loopend water}\\

\haiku{Huys van Cajaphas258Huys.}{van Annas258Huys van Salomon259Huys}{van Lazarusibid}\\

\haiku{20Bijvoorbeeld bij,}{Nicolaas Heinsius Den}{vermakelyken}\\

\haiku{s\c{c}avoir No. 16, No.,,,,, \&.}{21 No. 22B No. 22C No. 33}{No.37 No. 41 No. 187}\\

\haiku{550linkshandigen.}{551Zie ook hofdstuk IX}{over dit onderwerp}\\

\haiku{718Het is duidelijk.}{dat De Bruyn hier de letters}{verkeerd geplaatst heeft}\\

\haiku{829Zie folio 219.}{voor de Bruyns toelichting op}{deze geschenken}\\

\haiku{1115En van die lampen.}{zijn er enkele op de}{afbeelding te zien}\\

\haiku{1226iets 1227terzijde}{stellen 1228uitgegeven}{1229voor het geval}\\

\haiku{1503zelf 1504heilige.}{persoon 1505in elk geval}{1506Zie folio 27}\\

\haiku{1598toen wij dichter ()}{bij de wind gingen zeilen}{om bij te draaien}\\

\section{Boudewijn B\"uch}

\subsection{Uit: Links!}

\haiku{De enige die na,.}{afloop vragen stelde was}{Militaire Jaap}\\

\haiku{Ik zou die knul maar.}{beetgrijpen voordat ze de}{politie bellen}\\

\haiku{Na verloop van tijd.}{kwamen we hem steeds minder}{tegen in de stad}\\

\haiku{De deur stond, zoals,.}{altijd open en blijkbaar was}{iedereen naar bed}\\

\haiku{Dat duurde maar kort,:}{want na een minuut zuchtte}{hij en sprak plechtig}\\

\haiku{We scholden Saul uit {\textquoteleft}{\textquoteright} {\textquoteleft}{\textquoteright}.}{voorkapitalistenhond}{enheterovriend}\\

\haiku{{\textquoteleft}Saul verhuist morgen.}{naar een kutkamertje in}{een sombere steeg}\\

\haiku{Even later kwam hij,.}{terug met een pick-up een}{elpee en twee boxen}\\

\haiku{Ik werd weer ernstig,:}{keek in mijn eigen exemplaar}{en merkte trots op}\\

\haiku{Sirius gaf het.}{miezerige kereltje}{een dreun op zijn neus}\\

\haiku{Een paar minuten.}{later zaten we in een}{politiebusje}\\

\haiku{Een aantrekkelijk.}{agentje met pukkeltjes sloot}{ons in en lachte}\\

\haiku{Hij kwam de kamer.}{in en klapte de deksel}{van zijn pick-up open}\\

\haiku{{\textquoteleft}Je krijgt op deze,!}{manier toch de lekkerste}{klap lucht binnen Boud}\\

\haiku{{\textquoteleft}Wat bedoel je met {\textquotedblleft}{\textquotedblright},?}{de Stones enpolitiek}{bezig zijn Polly}\\

\haiku{{\textquoteleft}Sier, geile pieper,.}{van me je krijgt die jongen}{\'echt niet uit de broek}\\

\haiku{{\textquoteleft}Jongen, als je er,.}{helemaal aan verslingerd}{bent zeg het me dan}\\

\haiku{Sirius las in.}{een Aula-pocket over iets}{sociologisch}\\

\haiku{{\textquoteright} {\textquoteleft}De stencils tegen.}{de opvoering van The boys}{in the band zijn klaar}\\

\haiku{Daarom moeten we...}{actie voeren tegen dat}{toneelstuk van die}\\

\haiku{Toen Karola binnen,:}{was gekomen stelde ze}{de bos blond haar voor}\\

\haiku{we waren jaloers.}{op haar dat ze zo'n man de}{hare mocht noemen}\\

\haiku{De ellende gaat,{\textquoteright}.}{voor het werkvolk ook door op}{kerstavond brieste Dolf}\\

\haiku{* dolf had ons dan snel,.}{ingepalmd maar lideke}{kon er ook wat van}\\

\haiku{{\textquoteright} Dolf antwoordde, zeer,.}{tegen zijn gewoonte niet}{op wat Polly zei}\\

\haiku{En zo stonden er.}{nog enkele tientallen}{punten op de lijst}\\

\haiku{Dolf kwam binnen, hij:}{keek mij nauwelijks aan en}{filosofeerde}\\

\haiku{Het licht ging aan en.}{Dolf stond op de drempel met}{een vuurrode kop}\\

\haiku{Een woord dat al lang.}{tot de spreektaal behoort in}{New York en Frisco}\\

\haiku{{\textquoteright} {\textquoteleft}Dat is geen wie, maar.}{de alternatieve naam}{voor San Francisco}\\

\haiku{Hij stond op en zei {\textquoteleft}{\textquoteright}.}{dat hijzo stoned als een}{afwasteiltje was}\\

\haiku{Hij vroeg zich af of.}{een kunstboom niet beter was}{dan een natuurhoorn}\\

\haiku{Nee, Polly, ik ben,.}{er niet geweest maar lees wel}{de Peking Review}\\

\haiku{{\textquoteright} Lideke ging met.}{de theekan rond en vond dat}{we maar moesten stemmen}\\

\haiku{{\textquoteright} Dolf was gaan zitten.}{en schoof zenuwachtig op}{zijn stoel heen en weer}\\

\haiku{Als jij zegt dat Mao,.}{dat geschreven heeft dan zal}{het heus wel zo zijn}\\

\haiku{Ik keek Dolf in de.}{ogen en zag dat h{\'\i}j dat zelfs}{niet geloofde}\\

\haiku{In Schimmert ken ik,.}{een groot leegstaand klooster dat}{we kunnen kraken}\\

\haiku{{\textquoteright} {\textquoteleft}Schimmel in Schimmert,{\textquoteright}.}{joelde Sirius die zijn}{derde joint bouwde}\\

\haiku{We hoorden iemand.}{heel uit de verte door een}{lange gang klossen}\\

\haiku{Dja's konijn heeft de.}{reis hierheen waarschijnlijk niet}{kunnen verdragen}\\

\haiku{Dolf was inmiddels.}{ook gearriveerd en stond}{wat streng te kijken}\\

\haiku{Ik haalde mijn hand.}{door Dja's pagekopje en}{keek in het doosje}\\

\haiku{We hebben Dja een.}{plechtigheid beloofd en die}{zal hij krijgen ook}\\

\haiku{Dolf begon met het.}{voorlezen van een gedicht}{van Voorzitter Mao}\\

\haiku{Polly kauwde op.}{een grasspriet en Lideke}{krabde in haar kruis}\\

\haiku{Volgens mij hebben,,{\textquoteright},.}{we platjes Dolf riep ze zich}{omdraaiend naar Dolf}\\

\haiku{{\textquoteright} Bono keek naar mij:}{en ik hoefde er niet eens}{over na te denken}\\

\haiku{Toen werd alles zwart.}{en stierf alle geluid en}{de laatste kleur weg}\\

\haiku{De zonnehitte.}{deed het zweet uit mijn lange}{haren weggutsen}\\

\haiku{Ik haalde het en.}{Sirius en ik gingen}{ermee aan de gang}\\

\haiku{Toen Bono met een, - {\textquoteleft}!}{laatste vermoeide kreet hij}{schreeuwdeAgnus Dei}\\

\haiku{Wat had plotseling {\textquoteleft}{\textquoteright}?}{een plenaire discussie}{te betekenen}\\

\haiku{Om elf uur betrad,,.}{ik benieuwd maar ook bang de}{centrale ruimte}\\

\haiku{Zag je zijn gezicht?}{toen ik dat fantastische}{stuk van Mao voorlas}\\

\haiku{{\textquoteright} {\textquoteleft}Hier, hier,{\textquoteright} riep ik en.}{gooide de tekening van}{Dja weer op de grond}\\

\haiku{dat je in die drie.}{maanden  met niemand spreekt}{en niet thuis zult eten}\\

\haiku{Ik heb geen zin meer.}{om die klootzakken waar dan}{ook te ontmoeten}\\

\haiku{Dat jongetje zal -.}{zonder mij verder moeten}{dacht ik wijselijk}\\

\haiku{Behalve dan dat [...].}{hij samen schijnt te wonen}{met een rechts wijf in}\\

\haiku{dat weten onze.}{Chinese vrienden heel goed}{op prijs te stellen}\\

\haiku{Je zal zien dat we.}{een ereplekje krijgen in}{de receptiezaal}\\

\haiku{Rare snijbonen,,;}{lijken die Chinezen mij}{maar zo'n cultuur h\`e}\\

\subsection{Uit: Links!}

\haiku{De enige die na,.}{afloop vragen stelde was}{Militaire Jaap}\\

\haiku{Ik zou die knul maar.}{beetgrijpen voordat ze de}{politie bellen}\\

\haiku{* ~ dat we niet meer.}{in de mensa konden eten}{was voor Saul het ergst}\\

\haiku{Na verloop van tijd.}{kwamen we hem steeds minder}{tegen in de stad}\\

\haiku{De deur stond, zoals,.}{altijd open en blijkbaar was}{iedereen naar bed}\\

\haiku{Dat duurde maar kort,:}{want na een minuut zuchtte}{hij en sprak plechtig}\\

\haiku{We scholden Saul uit {\textquoteleft}{\textquoteright} {\textquoteleft}{\textquoteright}.}{voorkapitalistenhond}{enheterovriend}\\

\haiku{{\textquoteleft}Saul verhuist morgen.}{naar een kutkamertje in}{een sombere steeg}\\

\haiku{Even later kwam hij,.}{terug met een pick-up een}{elpee en twee boxen}\\

\haiku{Ik werd weer ernstig,:}{keek in mijn eigen exemplaar}{en merkte trots op}\\

\haiku{Sirius gaf het.}{miezerige kereltje}{een dreun op zijn neus}\\

\haiku{Een paar minuten.}{later zaten we in een}{politiebusje}\\

\haiku{Een aantrekkelijk.}{agentje met pukkeltjes sloot}{ons in en lachte}\\

\haiku{{\textquoteright} brulde ergens aan.}{het einde van de gang een}{politieagent}\\

\haiku{Hij kwam de kamer.}{in en klapte de deksel}{van zijn pick-up open}\\

\haiku{{\textquoteleft}Je krijgt op deze,!}{manier toch de lekkerste}{klap lucht binnen Boud}\\

\haiku{{\textquoteleft}Wat bedoel je met {\textquotedblleft}{\textquotedblright},?}{de Stones enpolitiek}{bezig zijn Polly}\\

\haiku{{\textquoteleft}Sier, geile pieper,.}{van me je krijgt die jongen}{\'echt niet uit de broek}\\

\haiku{{\textquoteleft}Jongen, als je er,.}{helemaal aan verslingerd}{bent zeg het me dan}\\

\haiku{Sirius las in.}{een Aula-pocket over iets}{sociologisch}\\

\haiku{* ~ een dokter die.}{uit zijn mond riekte stond over}{mij heen gebogen}\\

\haiku{{\textquoteright} {\textquoteleft}De stencils tegen.}{de opvoering van The boys}{in the band zijn klaar}\\

\haiku{Daarom moeten we...}{actie voeren tegen dat}{toneelstuk van die}\\

\haiku{Toen Karola binnen,:}{was gekomen stelde ze}{de bos blond haar voor}\\

\haiku{we waren jaloers.}{op haar dat ze zo'n man de}{hare mocht noemen}\\

\haiku{De ellende gaat,{\textquoteright}.}{voor het werkvolk ook door op}{kerstavond brieste Dolf}\\

\haiku{{\textquoteright} Dolf antwoordde, zeer,.}{tegen zijn gewoonte niet}{op wat Polly zei}\\

\haiku{En zo stonden er.}{nog enkele tientallen}{punten op de lijst}\\

\haiku{Dolf kwam binnen, hij:}{keek mij nauwelijks aan en}{filosofeerde}\\

\haiku{Het licht ging aan en.}{Dolf stond op de drempel met}{een vuurrode kop}\\

\haiku{Een woord dat al lang.}{tot de spreektaal behoort in}{New York en Frisco}\\

\haiku{{\textquoteright} {\textquoteleft}Dat is geen wie, maar.}{de alternatieve naam}{voor San Francisco}\\

\haiku{Hij stond op en zei {\textquoteleft}{\textquoteright}.}{dat hijzo stoned als een}{afwasteiltje was}\\

\haiku{Hij vroeg zich af of.}{een kunstboom niet beter was}{dan een natuurhoorn}\\

\haiku{Nee, Polly, ik ben,.}{er niet geweest maar lees wel}{de Peking Review}\\

\haiku{{\textquoteright} Lideke ging met.}{de theekan rond en vond dat}{we maar moesten stemmen}\\

\haiku{{\textquoteright} Dolf was gaan zitten.}{en schoof zenuwachtig op}{zijn stoel heen en weer}\\

\haiku{Als jij zegt dat Mao,.}{dat geschreven heeft dan zal}{het heus wel zo zijn}\\

\haiku{Ik keek Dolf in de.}{ogen en zag dat h{\'\i}j dat zelfs}{niet geloofde}\\

\haiku{In Schimmert ken ik,.}{een groot leegstaand klooster dat}{we kunnen kraken}\\

\haiku{{\textquoteright} {\textquoteleft}Schimmel in Schimmert,{\textquoteright}.}{joelde Sirius die zijn}{derde joint bouwde}\\

\haiku{We hoorden iemand.}{heel uit de verte door een}{lange gang klossen}\\

\haiku{Dja's konijn heeft de.}{reis hierheen waarschijnlijk niet}{kunnen verdragen}\\

\haiku{Dolf was inmiddels.}{ook gearriveerd en stond}{wat streng te kijken}\\

\haiku{Ik haalde mijn hand.}{door Dja's pagekopje en}{keek in het doosje}\\

\haiku{We hebben Dja een.}{plechtigheid beloofd en die}{zal hij krijgen ook}\\

\haiku{Dolf begon met het.}{voorlezen van een gedicht}{van Voorzitter Mao}\\

\haiku{Polly kauwde op.}{een grasspriet en Lideke}{krabde in haar kruis}\\

\haiku{Volgens mij hebben,,{\textquoteright},.}{we platjes Dolf riep ze zich}{omdraaiend naar Dolf}\\

\haiku{{\textquoteright} Bono keek naar mij:}{en ik hoefde er niet eens}{over na te denken}\\

\haiku{Toen werd alles zwart.}{en stierf alle geluid en}{de laatste kleur weg}\\

\haiku{De zonnehitte.}{deed het zweet uit mijn lange}{haren weggutsen}\\

\haiku{Ik haalde het en.}{Sirius en ik gingen}{ermee aan de gang}\\

\haiku{Toen Bono met een, - {\textquoteleft}!}{laatste vermoeide kreet hij}{schreeuwdeAgnus Dei}\\

\haiku{Wat had plotseling {\textquoteleft}{\textquoteright}?}{een plenaire discussie}{te betekenen}\\

\haiku{Om elf uur betrad,,.}{ik benieuwd maar ook bang de}{centrale ruimte}\\

\haiku{Zag je zijn gezicht?}{toen ik dat fantastische}{stuk van Mao voorlas}\\

\haiku{{\textquoteright} {\textquoteleft}Hier, hier,{\textquoteright} riep ik en.}{gooide de tekening van}{Dja weer op de grond}\\

\haiku{dat je in die drie.}{maanden  met niemand spreekt}{en niet thuis zult eten}\\

\haiku{Ik heb geen zin meer.}{om die klootzakken waar dan}{ook te ontmoeten}\\

\haiku{Dat jongetje zal -.}{zonder mij verder moeten}{dacht ik wijselijk}\\

\haiku{Behalve dan dat [...].}{hij samen schijnt te wonen}{met een rechts wijf in}\\

\haiku{dat weten onze.}{Chinese vrienden heel goed}{op prijs te stellen}\\

\haiku{Je zal zien dat we.}{een ereplekje krijgen in}{de receptiezaal}\\

\haiku{Rare snijbonen,,;}{lijken die Chinezen mij}{maar zo'n cultuur h\`e}\\

\subsection{Uit: De rekening}

\haiku{{\textquoteright} Mijn moeder poetste:}{het zilver op een oude}{krant en zei zachtjes}\\

\haiku{{\textquoteright} Over de rest van de.}{inboedel werd nog een jaar}{geprocedeerd}\\

\haiku{{\textquoteright} vroeg de chef die ook,.}{een stofjas droeg voorzien van}{een extra biesje}\\

\haiku{{\textquoteright} lachte een jongen.}{met een benig gezicht en}{een tatoeage}\\

\haiku{{\textquoteleft}Slordig werk heb je,?}{geleverd moet ik daar een}{riks voor betalen}\\

\haiku{Die hebben we bij,.}{Pimmetje laten opslaan}{midden in de nacht}\\

\haiku{Morgen ben ik er,,{\textquoteright}.}{ook nog Lomar sprak Hetty}{gemaakt sensueel}\\

\haiku{maandenlang stond ze.}{mij voor ogen als ik in de}{loods aan het werk was}\\

\haiku{Ik begrijp niet dat.}{die man niet af en toe een}{graai in de kas doet}\\

\haiku{Voor moeder is het,.}{natuurlijk ook leuk komt die}{er ook nog eens uit}\\

\haiku{Kijk, jij gaat straks op.}{een school les geven en dan}{verdien je goud geld}\\

\haiku{Op een keer zei ze -, -:}{grappig bedoeld maar toch ook}{een beetje gemeend}\\

\haiku{{\textquoteright} {\textquoteleft}Dat is maar goed ook,.}{want anders flikkerde ik}{ze er direct uit}\\

\haiku{{\textquoteright} {\textquoteleft}Ik schrijf wel een brief.}{en nu gaan we het over die}{verslaving hebben}\\

\haiku{Ik heb er dankzij.}{mijn ouders in mijn jeugd al}{genoeg meegemaakt}\\

\haiku{{\textquoteright} {\textquoteleft}'t Is beter, voor,;}{de nodige gespreksrust}{gelooft u mij maar}\\

\haiku{{\textquoteright} {\textquoteleft}Wat kan voor u het?}{belang zijn om de naam van}{die kat te weten}\\

\haiku{Soms fietste ik wel.}{vijf keer per week naar het huis}{aan de stille laan}\\

\haiku{ik zal 'ns kijken,}{hoe we dat met die centen}{op kunnen lossen}\\

\haiku{Ga op zoek naar de,:}{schuldvraag pijnig je kop af}{naar de oplossing}\\

\haiku{Geld moet er zijn - dat -.}{weet ik ook wel maar het moet}{onzichtbaar blijven}\\

\haiku{Beneden begon.}{er iemand anders op de}{deur te beuken}\\

\haiku{Als je goed wakker,.}{bent ga je weer neuken en}{dan opnieuw slapen}\\

\haiku{Dana vroeg op een,:}{morgen de zon stond achter}{de gordijnen klaar}\\

\haiku{Soms zag ik Liphorst.}{een jaar lang niet en hoorde}{ik ook niets van hem}\\

\haiku{{\textquoteleft}Ik heb alleen een.}{tas met wat kleren en een}{koffer vol boeken}\\

\haiku{Een wijf om neer te,,.}{sabelen zeg maar niets ik}{praat wel even met haar}\\

\haiku{We hebben hier een.}{chic huis en dat zou ik graag}{zo willen houden}\\

\haiku{Nou ja, je bent de:}{eigenaresse van een}{chic huis dus je zegt}\\

\haiku{Het duurt nu al vijf.}{jaar en er lijkt nooit meer een}{eind aan te komen}\\

\haiku{Zodoende was het.}{aan mijn deur een komen en}{gaan van geldeisers}\\

\haiku{{\textquoteright} klinkt het opeens op.}{tussen angstaanjagende}{vogelgeluiden}\\

\haiku{De middenstand heeft.}{een klein beetje de neiging}{westers te denken}\\

\haiku{{\textquoteright} Na van de schrik te,:}{zijn bekomen antwoordde}{ik zenuwachtig}\\

\haiku{Goudstaart had een aardig,.}{gezicht was klein en had niets}{van een advocaat}\\

\haiku{{\textquoteright} {\textquoteleft}Dat zou natuurlijk.}{het eerste zijn geweest dat}{hij had moeten doen}\\

\haiku{In geen geval wil.}{ik meer dan drie ton voor een}{woning betalen}\\

\haiku{Hij onderhield het.}{contact nog even door middel}{van aanmaningen}\\

\haiku{ik heb geen zin om}{al die muntjes tussen de}{lakens uit te gaan}\\

\haiku{{\textquoteright} {\textquoteleft}Geef die jongen toch,}{gewoon een gulden en laat}{hem die bus houden}\\

\haiku{Mijn moeder wees naar:}{een kleine jongen en een}{nog kleiner meisje}\\

\haiku{Jugend in D. 1916- -.}{1937 krullerige letters}{met inkt geschreven}\\

\haiku{Werd mijn vader door?}{zijn toenmalige vrouw of}{verloofde gekiekt}\\

\haiku{Tussen mijn vader (:}{en mij was er ogenschijnlijk}{weinigmijn moeder}\\

\haiku{Mary werd in 1940,.}{geboren acht jaar voordat}{ik ter wereld kwam}\\

\haiku{De keren dat hij,.}{naar Nieuw-Zeeland kwam was}{hij ook niet zuinig}\\

\haiku{Je moet beloven.}{dat je ze nooit aan iemand}{anders laat lezen}\\

\haiku{Ik ken je moeder,.}{niet maar ik heb natuurlijk}{vaak over haar gehoord}\\

\haiku{Ik dacht plotseling.}{terug aan de jaren op}{de lagere school}\\

\haiku{je melk mag drinken,.}{krijgen jullie van Hare}{Majesteit cadeau}\\

\haiku{Toen ik besloten,:}{had dat ik D. voorgoed zou}{verlaten zei ze}\\

\haiku{Hij was beter af.}{geweest wanneer hij boven}{D. was neergehaald}\\

\haiku{op tien mei 1945 schreef.}{hij al zijn eerste brief om}{schadevergoeding}\\

\haiku{Dat is het enige:}{waar hij na de oorlog mee}{bezig is geweest}\\

\haiku{Weet je nog toen die?}{oorlogsmisdadigers vrij}{werden gelaten}\\

\subsection{Uit: De rekening}

\haiku{{\textquoteright} Mijn moeder poetste:}{het zilver op een oude}{krant en zei zachtjes}\\

\haiku{{\textquoteright} Over de rest van de.}{inboedel werd nog een jaar}{geprocedeerd}\\

\haiku{{\textquoteright} vroeg de chef die ook,.}{een stofjas droeg voorzien van}{een extra biesje}\\

\haiku{{\textquoteright} lachte een jongen.}{met een benig gezicht en}{een tatoeage}\\

\haiku{{\textquoteleft}Slordig werk heb je,?}{geleverd moet ik daar een}{riks voor betalen}\\

\haiku{Die hebben we bij,.}{Pimmetje laten opslaan}{midden in de nacht}\\

\haiku{Morgen ben ik er,,{\textquoteright}.}{ook nog Lomar sprak Hetty}{gemaakt sensueel}\\

\haiku{maandenlang stond ze.}{mij voor ogen als ik in de}{loods aan het werk was}\\

\haiku{Ik begrijp niet dat.}{die man niet af en toe een}{graai in de kas doet}\\

\haiku{Voor moeder is het,.}{natuurlijk ook leuk komt die}{er ook nog eens uit}\\

\haiku{Kijk, jij gaat straks op.}{een school les geven en dan}{verdien je goud geld}\\

\haiku{Op een keer zei ze -, -:}{grappig bedoeld maar toch ook}{een beetje gemeend}\\

\haiku{{\textquoteright} {\textquoteleft}Dat is maar goed ook,.}{want anders flikkerde ik}{ze er direct uit}\\

\haiku{{\textquoteright} {\textquoteleft}Ik schrijf wel een brief.}{en nu gaan we het over die}{verslaving hebben}\\

\haiku{Ik heb er dankzij.}{mijn ouders in mijn jeugd al}{genoeg meegemaakt}\\

\haiku{{\textquoteright} {\textquoteleft}'t Is beter, voor,;}{de nodige gespreksrust}{gelooft u mij maar}\\

\haiku{{\textquoteright} {\textquoteleft}Wat kan voor u het?}{belang zijn om de naam van}{die kat te weten}\\

\haiku{Soms fietste ik wel.}{vijf keer per week naar het huis}{aan de stille laan}\\

\haiku{ik zal 'ns kijken,}{hoe we dat met die centen}{op kunnen lossen}\\

\haiku{Ga op zoek naar de,:}{schuldvraag pijnig je kop af}{naar de oplossing}\\

\haiku{Geld moet er zijn - dat -.}{weet ik ook wel maar het moet}{onzichtbaar blijven}\\

\haiku{Beneden begon.}{er iemand anders op de}{deur te beuken}\\

\haiku{Als je goed wakker,.}{bent ga je weer neuken en}{dan opnieuw slapen}\\

\haiku{Dana vroeg op een,:}{morgen de zon stond achter}{de gordijnen klaar}\\

\haiku{Soms zag ik Liphorst.}{een jaar lang niet en hoorde}{ik ook niets van hem}\\

\haiku{{\textquoteleft}Ik heb alleen een.}{tas met wat kleren en een}{koffer vol boeken}\\

\haiku{Een wijf om neer te,,.}{sabelen zeg maar niets ik}{praat wel even met haar}\\

\haiku{We hebben hier een.}{chic huis en dat zou ik graag}{zo willen houden}\\

\haiku{Nou ja, je bent de:}{eigenaresse van een}{chic huis dus je zegt}\\

\haiku{Het duurt nu al vijf.}{jaar en er lijkt nooit meer een}{eind aan te komen}\\

\haiku{Zodoende was het.}{aan mijn deur een komen en}{gaan van geldeisers}\\

\haiku{{\textquoteright} klinkt het opeens op.}{tussen angstaanjagende}{vogelgeluiden}\\

\haiku{De middenstand heeft.}{een klein beetje de neiging}{westers te denken}\\

\haiku{{\textquoteright} Na van de schrik te,:}{zijn bekomen antwoordde}{ik zenuwachtig}\\

\haiku{Goudstaart had een aardig,.}{gezicht was klein en had niets}{van een advocaat}\\

\haiku{{\textquoteright} {\textquoteleft}Dat zou natuurlijk.}{het eerste zijn geweest dat}{hij had moeten doen}\\

\haiku{In geen geval wil.}{ik meer dan drie ton voor een}{woning betalen}\\

\haiku{Hij onderhield het.}{contact nog even door middel}{van aanmaningen}\\

\haiku{ik heb geen zin om}{al die muntjes tussen de}{lakens uit te gaan}\\

\haiku{{\textquoteright} {\textquoteleft}Geef die jongen toch,}{gewoon een gulden en laat}{hem die bus houden}\\

\haiku{Mijn moeder wees naar:}{een kleine jongen en een}{nog kleiner meisje}\\

\haiku{Jugend in D. 1916- -.}{1937 krullerige letters}{met inkt geschreven}\\

\haiku{Werd mijn vader door?}{zijn toenmalige vrouw of}{verloofde gekiekt}\\

\haiku{Tussen mijn vader (:}{en mij was er ogenschijnlijk}{weinigmijn moeder}\\

\haiku{Mary werd in 1940,.}{geboren acht jaar voordat}{ik ter wereld kwam}\\

\haiku{De keren dat hij,.}{naar Nieuw-Zeeland kwam was}{hij ook niet zuinig}\\

\haiku{Je moet beloven.}{dat je ze nooit aan iemand}{anders laat lezen}\\

\haiku{Ik ken je moeder,.}{niet maar ik heb natuurlijk}{vaak over haar gehoord}\\

\haiku{Ik dacht plotseling.}{terug aan de jaren op}{de lagere school}\\

\haiku{je melk mag drinken,.}{krijgen jullie van Hare}{Majesteit cadeau}\\

\haiku{Toen ik besloten,:}{had dat ik D. voorgoed zou}{verlaten zei ze}\\

\haiku{Hij was beter af.}{geweest wanneer hij boven}{D. was neergehaald}\\

\haiku{op tien mei 1945 schreef.}{hij al zijn eerste brief om}{schadevergoeding}\\

\haiku{Dat is het enige:}{waar hij na de oorlog mee}{bezig is geweest}\\

\haiku{Weet je nog toen die?}{oorlogsmisdadigers vrij}{werden gelaten}\\

\section{Thomas Fran\c{c}ois Burgers}

\subsection{Uit: Dorp in het onderveld}

\haiku{Geen behoeften wekt,;}{hij op hij die daar volstrekt}{geen behoefte toont}\\

\haiku{Wat moet het toch zwaar!}{zijn om de rente van angst}{en vrees te trekken}\\

\haiku{de een altijd vlug,.}{en rad de ander altoos}{langzaam en bedaard}\\

\haiku{Kom, sta eens hier, en,.}{zie die lange magere}{kerel hoe hij knikt}\\

\haiku{{\textquoteleft}Neef Willem,{\textquoteright} zei een, {\textquoteleft}.}{ou tantehier is een mooi}{pak kinderkleertjes}\\

\haiku{{\textquoteleft}Oom Willem, hier is,.}{nu voor uzelf ook wat een mooie}{ouderwetse broek}\\

\haiku{De bediende thuis.}{en de kinderen krijgen}{strengere orders}\\

\haiku{Hoe ik eruitzie,,}{weet ik niet maar als ik de}{gemeente overzie}\\

\haiku{Neem, ik bid u, het.}{mij ook niet kwalijk dat ik}{rondkijk in de kerk}\\

\haiku{De kerk is lang niet.}{zo vol als gisteren en}{de leraar ook niet}\\

\haiku{neef Hendrik springt van.}{zijn tafel op en vergeet}{bijna te danken}\\

\haiku{Ik weet nu alles,,.}{maar wees gerust uw geheim}{is bij mij veilig}\\

\haiku{hem die ander dag,{\textquoteright}.}{verbrand het met die warme}{lood bracht Mietje in}\\

\haiku{anders zou ik na.}{die water kijk en dan kon}{Dina kogels giet}\\

\haiku{Hij slooft zich af en.}{offert zichzelf werkelijk}{op voor zijn gasten}\\

\haiku{Hij wendde zich naar,.}{de hertog en fluisterde}{hem iets in het oor}\\

\haiku{Maar wacht maar, ik zal,{\textquoteright}.}{de boel inpeperen zo}{ging de spreker voort}\\

\haiku{allen, behalve,.}{oom Hendrik waren die dag}{slaven in dat huis}\\

\haiku{Een menigte van,,.}{allerlei kleur en geur ook}{stond er te kijken}\\

\haiku{Koffie en thee, wijn.}{en bier werden in ruime}{mate rondgediend}\\

\haiku{Zo had oom Hendrik;}{de danslust op een plaats met}{een slag uitgeblust}\\

\haiku{{\textquoteleft}Ja, zo hebt jij ons,.}{mos van dag beet gehad voel}{nou ook hoe het smaak}\\

\haiku{Neef Nols had reeds een,,,.}{bok Paul ook het werd mijn tijd}{en ik drong voorwaarts}\\

\haiku{De bok liep als had,.}{hij geen kwetsuur Tempest als}{had hij geen ruiter}\\

\haiku{{\textquoteleft}Maar Pietje, jij het,!}{gloo sommer van dag een dooie}{bok opgetel neh}\\

\haiku{Vat zomaar met de,,{\textquoteright}, {\textquoteleft}.}{hand kerel riep er eenen}{gebruik jou jachtmes}\\

\haiku{Geen sloot te breed, geen {\textquoteleft}{\textquoteright}.}{polleno te hoog voor de}{moedige paarden}\\

\haiku{de priester bericht.}{moet doen met het verzoek zijn}{kindje te dopen}\\

\haiku{Hij weet, er is geen.}{uitweg en beproeft ook niet}{eens die te zoeken}\\

\haiku{{\textquoteleft}Wat hebben wij nog.}{getuigenis van node?{\textquoteright}o}{vraagt de andere}\\

\haiku{tata betekent,.}{bij de kleurlingen vader}{outata grootvader}\\

\haiku{kos mijn arme ziel}{verlangt zeer naar voedsel van}{morre vanmorgen}\\

\haiku{Mededeling van,.}{het Meertens Instituut d.d.}{12 december 2003}\\

\haiku{jij hebt geloof ik,!}{vandaag zomaar een dode}{bok opgepakt niet}\\

\section{Andreas Burnier}

\subsection{Uit: Een tevreden lach}

\haiku{Zij ontdekte er,:}{nooit iets nieuws maar een machtig}{beeld welde er op}\\

\haiku{Geachte heer, ik,.}{ben tijdelijk in Parijs}{maar ken hier niemand}\\

\haiku{{\textquoteright} Simone begreep.}{niet waarom de uitgever}{haar dit vertelde}\\

\haiku{{\textquoteleft}Het is een meisje,,,.}{ja hoor het is een meisje}{dat kan je goed zien}\\

\haiku{waar ik op de een.}{of andere manier}{mee te maken had}\\

\haiku{Ze is heel geschikt.}{en ze schijnt zich voor je te}{interesseren}\\

\haiku{De hele hete.}{zomer repeteerde ik}{de schoolwiskunde}\\

\haiku{{\textquoteleft}Wij zijn van de g.g.d.{\textquoteright},.}{zei de verpleegster langzaam}{en nadrukkelijk}\\

\haiku{Ik kreeg een spuitje.}{en ontwaakte veel later}{op een grote zaal}\\

\haiku{Van nu af was er}{in de buitenwereld niets}{meer te verwachten}\\

\haiku{Er ontstond tussen:}{ons wat je een schijncontact}{zou kunnen noemen}\\

\haiku{Of eigenlijk over:}{de samenhang van de drie}{polariteiten}\\

\haiku{homoseksueel,,,.}{kunstenaar misdadiger}{in zich verenigt}\\

\haiku{Zij nam het kaartje aan,,.}{zocht overal maar kon de}{boeken niet vinden}\\

\haiku{We zullen op de,,{\textquoteright}.}{motor naar Helmond rijden}{Simone zei hij}\\

\haiku{{\textquoteleft}Het was de eerste{\textquoteright}....}{keer voor mij en sloot heel zacht}{de deur achter zich}\\

\haiku{Het was een rooster.}{onder in de muur waar het}{lichtschijnsel uitkwam}\\

\haiku{Ze dronken rode '.}{wijn ens nachts kwam Rainer}{nog drie keer bij haar}\\

\haiku{Nou ken je nog zo,.}{eenzaam wezen dat heb je}{dan maar te dragen}\\

\haiku{Je moet je eigen,.}{nou eenmaal accepteren}{zoals je bent h\`e}\\

\haiku{Haar vader was voor.}{de oorlog houthakker in}{Tsjecho-Slowakije}\\

\haiku{Maar wat weten zij?}{daar van onze verstarring}{en ontlediging}\\

\haiku{Als het niet anders,:}{kan dan in Holland dan zal}{ik er doorheen gaan}\\

\haiku{We konden moeilijk.}{weer weigeren en jij hebt}{nu net geen nachtdienst}\\

\section{Cyriel Buysse}

\subsection{Uit: Het 'ezelken', wat niet vergeten was}

\haiku{een pastoor in haar -.}{familie te bezitten}{aan hem was volbracht}\\

\haiku{Juffrouw Constance.}{was een oude vrijster van}{reeds bij de veertig}\\

\haiku{en daarop volgde,:}{een lange lange optocht}{andere vrouwen}\\

\haiku{{\textquoteleft}Geen vrijage, noch,,!}{binnenshuis noch buitenshuis}{of dadelijk weg}\\

\haiku{- K'n weet ik da niet,.}{antwoordde C\'eline dan}{ook heel natuurlijk}\\

\haiku{Eerst vreesde zij nog, -!}{met welken angst van twijfel}{en onzekerheid}\\

\haiku{loat ons hou\^en 't '.}{geen da w'h\^en en kontent zijn}{datt zeu wel goat}\\

\haiku{riep juffrouw Toria,.}{geschokt met grooten mond en}{uitgezette oogen}\\

\haiku{'t es doarover ', '.}{dak ou kome spreken}{beefdet Ezelken}\\

\haiku{Wie weet ook of het,?}{om haar geld alleen was dat}{hij haar gevraagd had}\\

\haiku{Tok tok tok, hoorde.}{zij de meid aan haar broeder's}{slaapkamer tikken}\\

\haiku{Er was een korte,,.}{poos volkomen doodsche als}{versteende stilte}\\

\haiku{C\'eline's gezicht,.}{en manieren bevielen}{haar niet dien middag}\\

\haiku{H\'e-je nou gezien ''?}{wa veurn schandoal da}{g in ou huis h\^et}\\

\haiku{'k h\`e heur gezeid!}{da ze morgen uchtijnk mee}{pak en zak wig moet}\\

\haiku{zij riep ten slotte;}{haar meid naar binnen en deed}{haar licht aansteken}\\

\haiku{- O, iefer Toria,,!}{iefer Toria wa zij-je}{gij toch broave}\\

\haiku{Zij sloot den brief in,.}{het couvert en Aamlie bracht}{hem naar de bus}\\

\haiku{- Ge meug gerust zijn,,.}{ieffreiwe antwoordde de}{meid reeds in de gang}\\

\haiku{ik u door Ivo uwe.}{koffers met alles er in}{wat u toebehoort}\\

\haiku{het spotgelach der,.}{grappenmakers die zich daar}{verscholen hadden}\\

\haiku{'t Puipken stond even,}{onthutst maar v\'o\'or hij er meer}{van kon vertellen}\\

\haiku{Langzamerhand was.}{het Ezelken aan den toestand}{gewend geraakt}\\

\haiku{Juffer Toria, en,.}{ook het Ezelken waren maar}{half gerustgesteld}\\

\haiku{Het scheelde weinig ' ' -!}{oft glas stortte uits}{Ezelkens hand. Watte}\\

\haiku{'t Ezelken had het, ',....}{alst ware instinktief}{voelen aankomen}\\

\haiku{Wat moest er van haar,,?}{worden waar moest ze heen als}{juffer Toria stierf}\\

\haiku{Mirza moest het eerst,.}{bediend worden die was de}{ongeduldigste}\\

\haiku{- Ik, da huis keupen,, ' ';}{wa peist-ekn h\`e}{doar gien geld veuren}\\

\haiku{'t Was of ze twee,.}{huizen bewoonden een aan}{elken kant der straat}\\

\haiku{- 'K zal de koster,;}{loate roepen zuchtte}{nog eens het Ezelken}\\

\haiku{Kan 't hij nou euk '! '}{al nie verdroagen datn}{kind zijne stiel liert}\\

\haiku{- Pas moar op da ge ',.}{niet te lankn wacht zei streng}{de geestelijke}\\

\haiku{En hij vertrok, den,}{koster verwittigend dat}{hij den volgenden}\\

\subsection{Uit: Het ezelken}

\haiku{Juffrouw Constance.}{was een oude vrijster van}{reeds bij de veertig}\\

\haiku{{\textquoteleft}Geen vrijage, noch,,!}{binnenshuis noch buitenshuis}{of dadelijk weg}\\

\haiku{- K'n weet ik da niet,.}{antwoordde C\'eline dan}{ook heel natuurlijk}\\

\haiku{Eerst vreesde zij nog, -!}{met welken angst van twijfel}{en onzekerheid}\\

\haiku{riep juffrouw Toria,.}{geschokt met grooten mond en}{uitpuilende oogen}\\

\haiku{'t es doarover, '.}{da ik ou kome spreken}{beefdet Ezelken}\\

\haiku{Wie weet ook of het,?}{om haar geld alleen was dat}{hij haar gevraagd had}\\

\haiku{Tok tok tok, hoorde.}{zij de meid aan haar broeder's}{slaapkamer tikken}\\

\haiku{Er was een korte,,.}{poos volkomen doodsche als}{versteende stilte}\\

\haiku{C\'eline's gezicht,.}{en manieren bevielen}{haar niet dien middag}\\

\haiku{H\'e-je nou gezien ''?}{wa veurn schandoal da}{g in ou huis h\^et}\\

\haiku{'k h\`e heur gezeid!}{da ze morgen uchtijnk mee}{pak en zak wig moet}\\

\haiku{zij riep ten slotte;}{haar meid naar binnen en deed}{haar licht aansteken}\\

\haiku{Zij sloot den brief in,.}{het couvert en Aamlie bracht}{hem naar de bus}\\

\haiku{- Ge meug gerust zijn,,.}{ieffreiwe antwoordde de}{meid reeds in de gang}\\

\haiku{ik u door Ivo uwe.}{koffers met alles er in}{wat u toebehoort}\\

\haiku{het spotgelach der,.}{grappenmakers die zich daar}{verscholen hadden}\\

\haiku{'t Puipken stond even,}{onthutst maar v\'o\'or hij er meer}{van kon vertellen}\\

\haiku{Juffer Toria, en,.}{ook het Ezelken waren maar}{half gerustgesteld}\\

\haiku{Het scheelde weinig ' ' -!}{oft glas stortte uits}{Ezelkens hand. Watte}\\

\haiku{Wat moest er van haar,,?}{worden waar moest ze heen als}{juffer Toria stierf}\\

\haiku{Op den drempel van,.}{de gang verscheen de koster}{die hem gevolgd had}\\

\haiku{Mirza moest het eerst,.}{bediend worden die was de}{ongeduldigste}\\

\haiku{- Ik, da huis keupen,, ' ';}{wa peist-ekn h\`e}{doar gien geld veuren}\\

\haiku{- 'K zal de koster,;}{loate roepen zuchtte}{nog eens het Ezelken}\\

\haiku{Kan 't hij nou euk, '! '}{al nie verdroagen datn}{kind zijne stiel liert}\\

\haiku{- Pas moar op da ge ',.}{niet te langn wacht zei streng}{de geestelijke}\\

\haiku{En hij vertrok, den,}{koster verwittigend dat}{hij den volgenden}\\

\subsection{Uit: Het leven van Rozeke van Dalen}

\haiku{Alfons trok de deur '.}{opt nachtslot en stak den}{sleutel in zijn zak}\\

\haiku{{\textquoteright} vroeg opnieuw de stem,.}{norsch-wantrouwend en nu}{heelemaal wakker}\\

\haiku{{\textquoteleft}'k Ben 't ik, boas,{\textquoteright}.}{Van Doalen antwoordde}{Alfons eindelijk}\\

\haiku{zij moesten ook maar eens ';}{om \'e\'en uurs nachts opstaan}{en mee gaan slijten}\\

\haiku{de groote waakhonden.}{blaften in het gerinkel}{van hun kettingen}\\

\haiku{De boer volgde, met.}{onder iederen arm een}{groote flesch jenever}\\

\haiku{Wat dachten ze wel?}{met hun lanterfanten en}{hun gekheid maken}\\

\haiku{Het was stikwarm '.}{in huis ent zweet brak uit}{op de gezichten}\\

\haiku{{\textquoteleft}dag mejonkvreiw en,{\textquoteright}.}{gezelschap en gingen druk}{voort met hun arbeid}\\

\haiku{De duisternis was}{bijna gansch gevallen en}{het onweer trok af}\\

\haiku{zweir mij dat er nie '!}{anders gebeurdn es en}{dat de sloeber liegt}\\

\haiku{{\textquoteleft}Vreiw Urzela van,}{de Weghe verkloart-e}{gij toe te stemmen}\\

\haiku{Het jong begijntje '.}{scheen haar vragend iets int}{oor te fluisteren}\\

\haiku{Ge goat 'n taske '}{seekelou drijnken enn}{boterkoeksken eten}\\

\haiku{Hij glimlachte zoet.}{en nam streelend hare hand}{onder de tafel}\\

\haiku{{\textquoteleft}La mee Rozeke.}{in de wijnkels en ik mee}{Fons ievers elders}\\

\haiku{men wist zelfs niet wie ';}{hij was en of hij opt}{kasteel vertoefd had}\\

\haiku{Zij was gelukkig,.}{door en met Alfons en dat}{maakte alles goed}\\

\haiku{{\textquoteright} verschrikte 't jong,.}{begijntje de handen in}{elkaar geslagen}\\

\haiku{Langzaam en triestig '.}{schudde hijt hoofd en week}{terug naar de deur}\\

\haiku{{\textquoteleft}De loatste coupon, '.}{es vervallen van van doag}{af meugtem knippen}\\

\haiku{Rozeke dankte,.}{maar wist nu verder geen woord}{meer te zeggen}\\

\haiku{Ook de baron, haar,,,.}{vader zag er bekommerd}{somber triestig uit}\\

\haiku{Zij kwamen binnen,:}{terwijl Smul het paard bij den}{stal ging uitspannen}\\

\haiku{{\textquoteright} orakelde hij ruw,.}{met een rechten blik op Dons}{uit de krib komend}\\

\haiku{eerst als een heel fijn,,;}{kleurloos stuifmeel nauw zichtbaar}{in de grijze lucht}\\

\haiku{leupt gij er achter, ';}{aangezien da ge toch nie}{mier verstandn h\^et}\\

\haiku{n speuk, dat rechte!}{lijk ne pijl uit nen bogen}{noar den bosch toe leupt}\\

\haiku{Het vroor en al de ';}{sterren tintelden ins}{hemels donkerblauw}\\

\haiku{Papa en mama.}{dachten dat ik hem op reis}{wel zou vergeten}\\

\haiku{{\textquoteright} riep Rozeke, hoe.}{langer hoe dieper door het}{voorstel afgeschrikt}\\

\haiku{{\textquoteright} {\textquoteleft}Ge meug gerust zijn,,{\textquoteright} '.}{bezinne antwoorddet}{Geluw Meuleken}\\

\haiku{{\textquoteright} zuchtte Rozeke,.}{met de hand het bonzen van}{haar hart bedwingend}\\

\haiku{Een aarzelende.}{voetstap bleef stil-schuivend}{op den drempel staan}\\

\haiku{Gelukkig was het.}{viertal nu reeds weer in druk}{gepraat en gezwets}\\

\haiku{Toen volgden snel, in,:}{rijke bonte kleurschakeering}{al de anderen}\\

\haiku{'t Was nu of nooit.}{het oogenblik om het hem}{te overhandigen}\\

\haiku{En zij waagde de:}{vraag die haar boven alles}{interesseerde}\\

\haiku{of hij voelde pijn.}{in de zij als iemand die}{te hard gerend heeft}\\

\haiku{Blijf maar zitten, blijf,{\textquoteright};}{maar zitten riep dringend de}{jonge barones}\\

\haiku{{\textquoteright} klaagde zij, {\textquoteleft}'k Geef,}{ik hem alles woar da zijn}{herte noar lust moar}\\

\haiku{{\textquoteright} {\textquoteleft}Gij moogt hem vooral,.}{niet laten werken nog van}{heel de zomer niet}\\

\haiku{Reeds lag het vroege;}{lentewerk dringend op den}{akker te wachten}\\

\haiku{en zij wilde ook.}{de min met het wagentje}{doen binnenkomen}\\

\haiku{En hoe moest het op?}{de boerderij ook gaan als}{hij eenmaal weg was}\\

\haiku{Om negen uur kwam '.}{een rijtuig vant kasteel}{Alfons afhalen}\\

\haiku{Maar het verveelde.}{haar toch ook en zij ging er}{een eind aan maken}\\

\haiku{{\textquoteright} riep zij nog, met het.}{Geluw Meuleken naast het}{rijtuig meehollend}\\

\haiku{Het was niet vreemd voor,,.}{haar zij was niet bang het scheen}{haar zoo natuurlijk}\\

\haiku{Smul en Vaprijsken,,.}{gingen er rechts en links als}{wakers naast zitten}\\

\haiku{Haw\`el joa, 't es '!}{precies doarmee datt}{uitgekomen es}\\

\haiku{{\textquoteright} {\textquoteleft}'K 'n weet 't nie,, ';}{bezinne moar iederien}{int dorp zegt het}\\

\haiku{{\textquoteright} Zij schrikte van 't.}{idee alleen dat zij hem zoo}{iets  vragen zou}\\

\haiku{{\textquoteright} En v\'o\'or ze den tijd.}{had nog een woord te spreken}{was hij de deur uit}\\

\haiku{{\textquoteright} {\textquoteleft}'K en weet 't nie,, '.}{bezinne hijn ziet er}{toch moar oardig uit}\\

\haiku{Zij huilde niet, maar.}{de oogen flikkerden vreemd in}{haar doodsbleek gelaat}\\

\haiku{Gij zijt een schurk en.}{uw plaats is niet hier maar in}{de gevangenis}\\

\haiku{- Dit is de eerste,.}{en de allerlaatste keer}{dat ik u waarschuw}\\

\haiku{{\textquoteleft}Het arme beest krijgt,{\textquoteright}.}{in mijn plaats de schoppen en}{de slagen dacht zij}\\

\haiku{alleen de vrees voor.}{andere ongelukken}{beangstigde haar}\\

\haiku{Doch haar hart sloeg kalm.}{en gelijkmatig en zij}{voelde geen emotie}\\

\haiku{Met strak-stukken.}{blik van niet-begrijpen}{staarde zij hem aan}\\

\haiku{En zij ging naar het,,.}{venster bij de wieg waarin}{haar jongste kind lag}\\

\haiku{Maar onverrichter.}{zake keerden zij naar de}{boerderij terug}\\

\haiku{En 't leven ging,,;}{opnieuw zijn tragen stillen}{dagelijkschen gang}\\

\haiku{en wanneer zij hem;}{nog zag was hij bijna als}{een vreemde voor haar}\\

\subsection{Uit: 'n Leeuw van Vlaanderen}

\haiku{Hij waardeerde juist:}{in zijn broeder datgene}{wat hem zelf ontbrak}\\

\haiku{Neen, waarachtig, nooit!}{had ik u met dien baard en}{dat lorgnet herkend}\\

\haiku{s Winters in zijn,;}{huis laat ik maar zeggen zijn}{paleis te Brussel}\\

\haiku{- Wat kunnen mij die!}{lekkere diners en die}{sigaren schelen}\\

\haiku{Bizonder knap, al!}{hebben zij tot nog toe zeer}{weinig gepresteerd}\\

\haiku{doolhof der kleine.}{kronkelige straatjes van}{het oude Brussel}\\

\haiku{Zij waren laat, de;}{anderen zouden reeds op}{hen zitten wachten}\\

\haiku{Dat was tot nu toe ';}{zijn  gedragslijn in}{t leven geweest}\\

\haiku{De eerste rijen,,:}{zeer goed gezeten maakten}{het zich gezellig}\\

\haiku{in om het even welk.}{gezelschap waar de menschen}{ook Vlaamsch verstonden}\\

\haiku{murmelde hij op,.}{zijn beurt met vrome stem en}{glinsterende oogen}\\

\haiku{een groot geheim van...}{onbewuste liefde was}{in hen geboren}\\

\haiku{Haast nergens konden.}{zij een zaal huren om er}{meeting te houden}\\

\haiku{Hij lag te bed, zwak,,,.}{bleek triestig zoowel moreel als}{lichamelijk ziek}\\

\haiku{Want hij dacht er in '!...}{t geheel niet aan den strijd}{nu op te geven}\\

\haiku{liet Desgen\^ets zich,.}{onwillekeurig als een}{angstkreet ontvallen}\\

\haiku{- Gij denkt toch niet, hoop,?}{ik dat ik persoonlijk er}{eenige schuld aan heb}\\

\haiku{Er was geen weerstand:}{mogelijk tegen een zoo}{brutalen aanval}\\

\haiku{Ja, d\`at was wel wat.}{hij zoolang had willen en}{niet durven zeggen}\\

\haiku{Er was geen woord aan,!}{toe te voegen geen woord aan}{te veranderen}\\

\haiku{Als ik zou denken.}{dat je te weinig vraagt zal}{ik je meer geven}\\

\haiku{Na een oogenblik,.}{was hij terug met een blik}{vol hout en steenkool}\\

\haiku{En een misnoegde,.}{bijna minachtende trek}{kwam op zijn gezicht}\\

\haiku{Waarom toch moest nu?}{weer die nare storing in}{zijn leven komen}\\

\subsection{Uit: Het recht van de sterkste}

\haiku{- Maria, laat ze maar,,.}{gaan sprak hij wij zullen ze}{wel achterhalen}\\

\haiku{- Theofiel, jongen,,.}{goa gij naor ou bedde}{dat zal beter zijn}\\

\haiku{En hij ging onder,.}{de sperrekens die rond het}{kapelletje staan}\\

\haiku{Die onverwachte.}{handelwijze liet Balduk}{stom van verbazing}\\

\haiku{riepen haar Witte.}{Manse en de andere}{vrouwen achterna}\\

\haiku{Het water, zwart, lag,.}{onbeweeglijk omringd van}{donker struikgewas}\\

\haiku{zij was reeds veertien,.}{dagen over tijd thans kon ze}{niet meer twijfelen}\\

\haiku{Hij had iets woest, iets,;}{overweldigends dat haar haast}{schrik inboezemde}\\

\haiku{Hij achtervolgde,,.}{haar hoog van gestalte met}{gesloten vuisten}\\

\haiku{Stom van schrik kreeg zij.}{die scheldwoorden als een slag}{in het aangezicht}\\

\haiku{Zij was als van een,.}{ander geslacht als van een}{ander bloed voor hem}\\

\haiku{Witte Manse en.}{haar moeder hadden hem op}{de drempel gevolgd}\\

\haiku{Donder de Beul, Klod.}{de Vos en Smuik Vertriest}{waren de eersten}\\

\haiku{Boef Verwilst drukte.}{met een vloek zijn spijt uit dat}{hij niet mocht schieten}\\

\haiku{Hij stak het plat, scherp.}{uiteinde van zijn hefboom}{eronder en hief}\\

\haiku{En terstond, de daad,.}{bij de woorden voegend liet}{hij zich neerplonsen}\\

\haiku{Zij werden allen,,.}{doch op zeer ongelijke}{manier veroordeeld}\\

\haiku{waarvan zouden zij,,?}{de oude moeder en het}{kind voortaan leven}\\

\haiku{Hij bleef nog enige,.}{stonden sprakeloos de blik}{op haar gevestigd}\\

\haiku{Als door een slag in '.}{t aangezicht kreeg zij het}{bewustzijn terug}\\

\haiku{Ontroerd, geschokt, met,.}{tranen in de ogen knielden}{de bezoekers neer}\\

\haiku{Allen keken op.}{en staarden snuffelend en}{zoekend om zich heen}\\

\subsection{Uit: De roman van den schaatsenrijder}

\haiku{Even buiten 't dorp,,.}{op korten afstand van ons}{huis lag de Lusthof}\\

\haiku{Het kraakt, er komen,.}{sterren in maar het schijnt toch}{te kunnen dragen}\\

\haiku{Dat alles reden.}{wij voortdurend langs en wij}{zagen dat alles}\\

\haiku{Het ijs lag er steeds;}{onbetrouwbaar en had er}{een vuilgele kleur}\\

\haiku{en z\'o\'o reuzesterk,.}{en taai was hij dat hij ons}{niet zelden overwon}\\

\haiku{Wij waren banger}{voor Guus dan voor zijn hond op}{het ijs en haastig}\\

\haiku{Er was een Peetse,,:}{Kins een Bruuntje Geelewie en er}{waren drie broeders}\\

\haiku{Ik keek en hoorde.}{dat alles aan met stillen}{weemoed en emotie}\\

\haiku{hij leek z\'o\'o sprekend,:}{dat ik naar hem toe ging en}{op den man af vroeg}\\

\haiku{Iedereen, oud of,,,.}{jong man of vrouw van klein tot}{groot was bang voor hem}\\

\haiku{- Stien Smijters h\^et de '!}{boantjes op de wal van}{t Oarmhuis vermeurd}\\

\haiku{En eigenlijk wist ':}{ik nooit precies wat mij wel}{t aangenaamst was}\\

\haiku{Er bleef mij trouwens.}{nog ruim voldoende liefde}{en illuzie over}\\

\haiku{En zoo kwam ik, als, ';}{altijd aant heerlijke}{Meylegem-Zuid}\\

\haiku{ik was als van de;}{aarde opgetild en op}{wieken gedragen}\\

\haiku{zij schoven verder ',,;}{overt ijs teeder omarmd}{amoureus-fluisterend}\\

\haiku{Even voorbij de bocht,.}{keek ik eens om en zag dat}{ze mij naoogden}\\

\haiku{Ik voelde dat ik,,.}{indruk had gemaakt ja dat}{ik overwonnen had}\\

\haiku{De Groote Schilder en:}{de Groote Musicus merkten}{het ook en schertsten}\\

\haiku{{\textquoteright} ~ *** ~ Zoo ging ik}{vele dagen wandelen}{en bewonderde}\\

\haiku{Ik verademde.}{alsof ik van een zware}{dreiging werd bevrijd}\\

\haiku{En dat men die toch,,!}{hebben moest en veel om daar}{te kunnen leven}\\

\haiku{en die houden er,.}{ook wel de vroolijkheid in}{wees dat maar zeker}\\

\haiku{Ik haastte mij, ik;}{hijgde en zwoegde door de}{glinsterende sneeuw}\\

\haiku{Ik had het prettig,.}{gevoel dat ik daar een van}{de besten zou zijn}\\

\haiku{- Neen, pardon, zoo niet,,.}{het linkerbeen naar achter}{professeerde ik}\\

\haiku{zij genoot, zij was....!}{tevreden en gelukkig}{gelukkig door mij}\\

\haiku{Het brandde op mijn,.}{lippen om het te vragen}{maar ik durfde niet}\\

\haiku{Het stroomde door mijn;}{heele lichaam heen als een}{electrische trilling}\\

\haiku{den moed hebben mij?}{voor altijd in het vreemde}{land te vestigen}\\

\haiku{Een mensch die stevig;}{gegeten heeft is dikwijls}{log en loom en zwaar}\\

\haiku{Ik liet de dames,.}{voor en trad in een salon}{door Papa gevolgd}\\

\haiku{Er kwamen tranen,....}{in mijn oogen die langzaam over}{mijn wangen vloeiden}\\

\haiku{Dan zeiden weer de.}{oogen wat de mond nog niet had}{durven uitdrukken}\\

\haiku{{\textquoteright} Dat waren uit hun.}{schaal gehaalde en in melk}{gekookte oesters}\\

\haiku{Welke jonge mooie?}{vrouw is niet gelukkig in}{een mode-winkel}\\

\haiku{En Maud beaamde,,.}{door een zwijgend hoofdgeknik}{haar tante's woorden}\\

\haiku{Een ma{\^\i}tre d'h\^otel.}{kwam naar mij toe en bood mij}{stil de wijnkaart aan}\\

\haiku{samen zes dollar '!}{voorn maaltijd die zoowat een}{zestig cent waard is}\\

\haiku{En af en toe was:}{het alsof de slapende}{reus even ontwaakte}\\

\haiku{Als ik mij repte.}{kon ik de electrische van}{half drie nog halen}\\

\haiku{een paar jongelui;}{die lachend pret maakten en}{wat dronken schenen}\\

\subsection{Uit: Tantes}

\haiku{Van Rysselberghe \& /, / [].}{Rombaut C.A.J van Dishoeck Gent}{Bussum z.j.1924}\\

\haiku{Het was een knappe,,.}{man met donker haar mooie snor}{en sprekende oogen}\\

\haiku{Je wist immers wel!}{waar ze te vinden waren}{als je ze noodig had}\\

\haiku{Hij had het moeten...!}{weten weten hoe en wat}{zij voor hem voelde}\\

\haiku{Hij zou eens eventjes;}{met een paar vrienden meegaan}{naar een koffiehuis}\\

\haiku{Een oogenblik had,,.}{Zijne Heiligheid ook naar}{hem Max gekeken}\\

\haiku{Meneer Dufour was.}{al aan de deur en hielp zijn}{zusters uitstijgen}\\

\haiku{Allen, even gestoord,.}{en bijna schrikkend keken}{met verbazing op}\\

\haiku{- Een eer die lastig,.}{is om te dragen meende}{tante Clemence}\\

\haiku{het leek wel of een;}{ongekende ramp over haar}{was neergekomen}\\

\haiku{Oe-Oe, op den grond,.}{Impikoko op een stoel}{en aten met hem mee}\\

\haiku{- Pas moar op da g' '!}{ou nietn verongelukt}{mee al da rijen}\\

\haiku{- Gee moar hier, Manse, ',.}{k zal gauwe genezen}{zijn glimlachte hij}\\

\haiku{riep hij verrast, als.}{naar gewoonte Fransch en Vlaamsch}{door elkaar mengend}\\

\haiku{Wat ons betreft,... wij.}{willen er absoluut niets}{gemeens mee hebben}\\

\haiku{- Wij kunnen hem toch!}{niet beletten hier te paard}{voorbij te rijden}\\

\haiku{zij kregen zelven;}{koffiebezoek van een paar}{dames uit haar dorp}\\

\haiku{voorloopig viel,.}{er niet te aarzelen maar}{te gehoorzamen}\\

\haiku{Het kwam hem voor of.}{een paar lui uit de buurt hem}{spottend nakeken}\\

\haiku{Hij duwde een van.}{zijn vensterluiken open en}{staarde in den nacht}\\

\haiku{Hij voelde 't nu,.}{ineens heel diep het hart vol}{wroeging en verwijt}\\

\haiku{zelve die hem de.}{gevoelens ingaf en de}{woorden dicteerde}\\

\haiku{Clement dus, zei Max,:}{glimlachend met als tweeden}{naam dien van papa}\\

\haiku{De doopsplechtigheid ',.}{int kleine stadje was}{iets buitengewoons}\\

\haiku{Zij zei hem niet, dat.}{zij bij meneer de pastoor}{was te biecht geweest}\\

\haiku{Vroeger, onder zijn,.}{blik sloeg ze dadelijk haar}{oogen neer en kleurde}\\

\haiku{Zij bekwamen er, ';}{niet van zij vielen alst}{ware uit de lucht}\\

\haiku{Max blikte koel in '.}{t onbestemde en streek}{zijn baard naar voren}\\

\haiku{Max deed of hij dat,;}{zeer betreurde hoewel hij}{het begrijpen kon}\\

\haiku{Max ging open doen en:}{zijn dienstmeisje stond voor hem}{en zei fluisterend}\\

\haiku{Zelfs de oude meid;}{Eemlie was met een aardig}{sommetje bedacht}\\

\haiku{zijn huis, zijn bedrijf,,!}{zijn goede gedienstigen}{zijn trouwe dieren}\\

\haiku{Hij zag haar roerloos, ',.}{staan int zwart gekleed met}{een krijtwit gezicht}\\

\haiku{De koetsier tikte.}{zijn paarden en ratelend}{reed het rijtuig weg}\\

\haiku{Zij liepen door een.}{lange gang en hielden bij}{de laatste deur stil}\\

\haiku{Het nonnetje boog '.}{even naart sleutelgat en}{scheen te luisteren}\\

\haiku{Een vraag brandde op,.}{Clara's lippen die zij haast}{niet durfde stellen}\\

\subsection{Uit: Verslagen over den gemeenteraad van Nevele}

\haiku{(Hij haalt zijne beurs):}{uit en geeft hem eenen cens}{H. MEGANCK}\\

\haiku{Ah Meneer L\'eonce, '';}{k en zou e k ik daar}{nie achterloopen}\\

\haiku{Ei, Meneer den Bron, ' ' '?}{est voort kortste of}{voort langste}\\

\haiku{(in eenen lach schietend),, '.}{Ah ja jat es daarom}{dat da voorskomt}\\

\subsection{Uit: Verzameld werk. Deel 1}

\haiku{We kunnen het dan:}{ook met De Cock ten volle}{eens zijn waar hij schrijft}\\

\haiku{Op 28 augustus,,.}{1897 wordt een zoontje Ren\'e}{Cyriel geboren}\\

\haiku{Niet dat we menen.}{dat het werk van Buysse een}{pleidooi nodig heeft}\\

\haiku{De beide jonkmans,,.}{koutend en schertsend stapten}{nevens de meisjes}\\

\haiku{- Theofiel, jongen,,.}{goa gij naor ou bedde}{dat zal beter zijn}\\

\haiku{En hij ging onder,.}{de sperrekes die rond het}{kapelletje staan}\\

\haiku{Die onverwachte.}{handelwijze liet Balduk}{stom van verbazing}\\

\haiku{riepen haar Witte.}{Manse en de andere}{vrouwen achterna}\\

\haiku{Het water, zwart, lag,.}{onbeweeglijk omringd van}{donker struikgewas}\\

\haiku{zij was reeds veertien,.}{dagen over tijd thans kon ze}{niet meer twijfelen}\\

\haiku{- Enfin, ik peins het,:}{toch verbeterde zij haar}{eerste gezegde}\\

\haiku{Om te zien of mijn. '! '!}{slinger daar goed voor wast}{Zal gaant Zal gaan}\\

\haiku{Hij had iets woests, iets,;}{overweldigends dat haar haast}{schrik inboezemde}\\

\haiku{Hij achtervolgde,,.}{haar hoog van gestalte met}{gesloten vuisten}\\

\haiku{Stom van schrik kreeg zij.}{die scheldwoorden als een slag}{in het aangezicht}\\

\haiku{Ongetwijfeld was.}{hij er op dat ogenblik reeds}{mee op weg naar Gent}\\

\haiku{Ditmaal geraakten.}{zij nog ongedeerd uit de}{klauwen van de Wet}\\

\haiku{Zij was als van een,.}{ander geslacht als van een}{ander bloed voor hem}\\

\haiku{Ze staat daar weeral! -,.}{riep Manse soms eensklaps de}{voordeur opentrekkend}\\

\haiku{Witte Manse en.}{haar moeder hadden hem op}{de drempel gevolgd}\\

\haiku{Donder de Beul, Klod.}{de Vos en Smuik Vertriest}{waren de eersten}\\

\haiku{Boef Verwilst drukte.}{met een vloek zijn spijt uit dat}{hij niet mocht schieten}\\

\haiku{Hij stak het plat, scherp.}{uiteinde van zijn hefboom}{eronder en hief}\\

\haiku{En terstond, de daad,.}{bij de woorden voegend het}{hij zich neerplonsen}\\

\haiku{bladeren, takken,.}{kruiden en stukken hout in}{zijn vaart meeslepend}\\

\haiku{Zij werden allen,,.}{doch op zeer ongelijke}{manier veroordeeld}\\

\haiku{waarvan zouden zij,,?}{de oude moeder en het}{kind voortaan leven}\\

\haiku{Hij bleef nog enige,.}{stonden sprakeloos de blik}{op haar gevestigd}\\

\haiku{Als door een slag in '.}{t aangezicht kreeg zij het}{bewustzijn terug}\\

\haiku{Ontroerd, geschokt, met,.}{tranen in de ogen knielden}{de bezoekers neer}\\

\haiku{Allen keken op.}{en staarden snuffelend en}{zoekend om zich heen}\\

\haiku{Een bloedkleurige,:}{vlam brandde beneveld in}{de rook de kreten}\\

\haiku{De verbrande plek.}{in het bed werd met een wit}{linnen doek bedekt}\\

\haiku{Een straal schoot uit zijn,.}{oog een glimlach van geluk}{kwam op zijn lippen}\\

\haiku{In een oogwenk stond,.}{het rijpaard gezadeld had}{hij zijn sporen aan}\\

\haiku{En Gilbert was ten;}{slotte gelukkig over de}{eerste uitkomsten}\\

\haiku{Terstond hielden de.}{gesprekken op en allen}{werden zeer ernstig}\\

\haiku{De jongelieden,,.}{zich neerzettend bestelden}{ieder een glas bier}\\

\haiku{Was hun tijdschrift daar,?}{wellicht niet gekomen dat}{niemand ervan sprak}\\

\haiku{Een glimlach was op;}{de lippen van enkele}{leden verschenen}\\

\haiku{Gilbert hoorde en.}{staarde hem aan met een}{strelend genoegen}\\

\haiku{{\textquoteleft}Bonjour ma{\^\i}t' De Roo{\textquoteright},,.}{il lui tendait son journal}{le Petit Flamand}\\

\haiku{Langzaam, steeds zonder,,.}{een woord zonder een gebaar}{volgde mijnheer haar}\\

\haiku{Wat kon het hem thans?}{nog schelen of zijn feest goed}{dan slecht gelukte}\\

\haiku{Een duizeling van;}{geluk bedwelmde zijn geest}{bij die gedachte}\\

\haiku{Hij herlas nog eens,.}{zijn brief vond hem goed en sloot}{hem in zijn omslag}\\

\haiku{zij voelt er zich ten;}{hoogste door vereerd en dankt}{u uit dien hoofde}\\

\haiku{Die dochter, naar het,.}{schijnt is van een zekere}{schoonheid niet ontbloot}\\

\haiku{de hand. Zijn zending,.}{was volbracht de jongeman}{wilde vertrekken}\\

\haiku{dat Ir\`ene verloofd.}{was en ging trouwen met haar}{neef Jozef De Moor}\\

\haiku{De datum voor de:}{echtverbintenis was}{nog niet vast bepaald}\\

\haiku{zij vernederen;}{de harten die zij zouden}{moeten verheffen}\\

\haiku{zij zijn de schuld, dat!}{het bloed van de mensen met}{beken heeft gestroomd}\\

\haiku{V\'o\'or hem  strekte,;}{zich een grasplein uit omzoomd}{met zwarte lovers}\\

\haiku{Bevend hief Gilbert.}{zijn bijna opgebrande}{lucifer omhoog}\\

\haiku{Doch hij moest, er viel,,...}{niet te aarzelen en hij}{ijlde hij ijlde}\\

\haiku{Drieghe deed op de.}{drempel zijn klompen uit en}{zij traden binnen}\\

\haiku{Maar nogmaals slaakte.}{zij een wilde kreet en viel}{terug in onmacht}\\

\haiku{Het schrikbeeld van 't,;}{gebeurde achtervolgde}{obsedeerde hem}\\

\haiku{Lauwereijnssens riep,.}{het artikel Drie op het}{woonhuis van Gilbert}\\

\haiku{anders krijgen wij, '!}{er vandaag niet mee gedaan}{ent moet gedaan}\\

\haiku{'t Was of eensklaps.}{een sluier van v\'o\'or Gilberts}{ogen werd getrokken}\\

\haiku{Hij ging, hij dwaalde.}{op goed geluk af zonder}{te weten waarheen}\\

\haiku{rampzaliger dan,,?}{ooit dodelijk getroffen}{om er te sterven}\\

\haiku{Bij plaatsen waren;}{de wegen er door onkruid}{en bramen bedekt}\\

\haiku{Ik ben z\'o zwak, z\'o,,}{ontroerd z\'o ziek dat ik u}{om zo te zeggen}\\

\haiku{dat zijn minnares,;}{ook wegliep met haar schreiend}{kind in de armen}\\

\haiku{Hij liep de spoorbaan,.}{over doorkruiste haastig het}{verlaten stadspark}\\

\haiku{De letters dansten;}{hem v\'o\'or zijn ogen en hij kon}{ook niet stilzitten}\\

\haiku{- O spreek tot mij, help,?}{me door uw raad zeg me wat}{mij nog te doen staat}\\

\haiku{Toen lispelde zij,,:}{eindelijk met een doffe}{troosteloze stem}\\

\haiku{En voor ons beiden,,:}{zal het genoegen zoals}{immer dubbel zijn}\\

\haiku{Och neen, het is geen,.}{kwaadspreken want ik heb ze}{in de grond wel lief}\\

\haiku{Tot morgen om drie, ',?}{uur dus aant station}{niet waar Ren\'e}\\

\haiku{dat het verleden,,;}{het lief verleden z\'o kalm}{en zo gelukkig}\\

\haiku{Helaas, tante was.}{nu dood en haar kinderen}{hadden zich verspreid}\\

\haiku{Daar keerde hij zich.}{om en staarde weer in de}{richting van de stad}\\

\haiku{Van beide rampen,,.}{bijna onvermijdbaar was}{dit nog de zachtste}\\

\haiku{Hij het haar los, en,.}{met een soort van schrik keken}{zij elkander aan}\\

\haiku{Maar plotseling hield:}{zij op en staarde hem aan}{met ogen  vol schrik}\\

\haiku{Vol onrust stak hij.}{de brief opzij en opende}{die van Raymonde}\\

\haiku{Zij kreeg opnieuw haar.}{onbeweeglijke houding}{van zieltogende}\\

\haiku{in de grijze lucht,.}{dreven loom en traag benden}{krassende raven}\\

\haiku{- Deze morgen moet.}{de overledene in haar}{kist gelegd worden}\\

\haiku{De hinderpaal die,;}{tussen beiden oprees was}{eensklaps verdwenen}\\

\haiku{aan alle kanten.}{om hem heen was het lijden}{en vertwijfeling}\\

\haiku{Nu begrijpt ge, dat?}{ik het recht verbeurd heb nog}{aan u te denken}\\

\haiku{elk ogenblik worden,.}{er onrechtvaardigheden}{misdaden gepleegd}\\

\haiku{zou d\'a\'ar toch waarlijk?}{nog een absolutie voor}{de zondaar liggen}\\

\haiku{Lange tijd had de.}{edelman van dergelijk plan}{niet willen horen}\\

\haiku{- Nathalie, zult gij,?}{alles wel sluiten en vuur}{en licht uitdoven}\\

\haiku{Zij draaide de lamp, '.}{een weinig lager deed haar}{deur int nachtslot}\\

\haiku{Buiten had de wind '.}{zich als het ware vant}{kasteel verwijderd}\\

\haiku{De dikke aderen;}{van haar kloeke hals zwollen}{onheilspellend op}\\

\haiku{Doch zij alleen vond:}{zijn handelwijze verkeerd}{en onnatuurlijk}\\

\haiku{Had ze gedurfd, ze;}{zou er aan haar tante van}{gesproken hebben}\\

\haiku{Hij zit daar pruilend;}{in een hoek met een boek dat}{hij zelfs niet meer leest}\\

\haiku{Zenobie bukte 't,;}{hoofd vuurrood van schaamte en}{tot wenens ontroerd}\\

\haiku{Focho was blijven,.}{staan verdween hij achter de}{donkere sparren}\\

\haiku{een tijdlang in het.}{huisje van de tuinman geen}{voet meer te zetten}\\

\haiku{Doch haar toestand, in,;}{plaats van te verbeteren}{verslechtte zichtbaar}\\

\haiku{- Nochtans gij hebt nu,,}{uw nichtje mejuffrouw de}{Rorick  bij u.}\\

\haiku{Zij heeft over haar mooie;}{trekken een zweem van ietwat}{treurige zachtheid}\\

\haiku{Doch niet in 't minst.}{voelde de oude jonkvrouw}{zich gerustgesteld}\\

\haiku{En daar Nina, meer,:}{en meer ontroerd en verbaasd}{zich excuseerde}\\

\haiku{Zij bleef daar staan en,;}{draalde bijna hopend hem}{te zien verschijnen}\\

\haiku{Zij riep Focho, die,,.}{vooruitrende bij zich en}{sloeg de zijlaan in}\\

\haiku{Zij schrikte er haast,}{van het kwam haar voor alsof}{hij daar ineens v\'o\'or}\\

\haiku{Ik zweer u dat ik.}{eerlijk en rechtschapen met}{u zal handelen}\\

\haiku{Intussen kunt ge,.}{mij voortdurend schrijven ik}{zal u antwoorden}\\

\haiku{Men zegt soms dat het ',;}{t Noodlot is dat over ons}{levenslot beschikt}\\

\haiku{Verdelg de jonge,;}{nog zo tengere bloempjes}{van de bomen niet}\\

\haiku{En 't is of de:}{natuur haar angstig gebed}{wilde verhoren}\\

\haiku{het Vermogen van...,!}{de Liefde O hoe innig}{voelt ze dit ook nu}\\

\haiku{Eerbiedig gingen.}{de dokter en Marie een}{weinig opzij staan}\\

\haiku{Hij wilde spreken,,,.}{vloeken schreeuwen en kon geen}{woord meer uitbrengen}\\

\haiku{- Nee, maar was ik er,,?}{niet gekomen het zou toch}{w\'el gebeurd zijn h\`e}\\

\haiku{- Waar? - In de keuken,.}{terwijl Pier-Cies bij u}{op de akker was}\\

\haiku{En nu was hij z\'o,.}{aan haar gewend dat hij haar}{niet meer missen kon}\\

\haiku{hij staakte het, en,.}{ging nu naar de hoogmis die}{Pol ook bijwoonde}\\

\haiku{Eens, toen ze van de,,:}{markt kwam had hij haar gekruist}{op een smal eenzaam}\\

\haiku{Ondanks al zijn aplomb,.}{keek de bezoeker enigszins}{verbauwereerd op}\\

\haiku{En, om het pijnlijk:}{gesprek in een andere}{wending te brengen}\\

\haiku{En ineens, bij dat,,.}{zicht ging er in hem een groot}{vreselijk licht op}\\

\haiku{Werktuiglijk, met een,.}{ruwe stoot duwde hij die}{open en trad binnen}\\

\haiku{En toch was er iets.}{onheilspellends in die slechts}{schijnbare vrede}\\

\haiku{'s Nachts vooral leed.}{hij nog heviger onder}{die vreemde kwelling}\\

\haiku{Men zou het er zich ':}{allergezelligst maken}{int warme stro}\\

\haiku{En langzaam buigt hij, '...}{zich de rechterhand int}{donker uitgestrekt}\\

\haiku{En in zijn woestheid,:}{schreeuwt hij woorden waarvan zij}{de zin niet begrijpt}\\

\haiku{Het spel is eerlijk,.}{zijn gang gegaan dat hebben}{wij allen gezien}\\

\haiku{Al wat ik weet is,;}{dat die schelm uit de pot een}{frank gestolen heeft}\\

\haiku{nu was zijn naam toch.}{voor altijd in het dorp met}{hoon en spot bedekt}\\

\haiku{Wat had zijn neef hem,?}{toch misdaan dat hij hem zo}{vreselijk haatte}\\

\haiku{Trouwens, hoe kon de,?}{jongeling het weten dat}{ze de zijne was}\\

\haiku{En in het dorp zelf, ',.}{kwam hij nooit meer zelfs niets}{zondags voor de mis}\\

\haiku{Hij sloot moedwillig;}{zijn ogen en zijn oren voor het}{akelig tafereel}\\

\haiku{Het landschap, om hem,:}{heen strekte zich heerlijk uit}{in zijn lentepracht}\\

\haiku{En even dacht hij aan,,.}{Rosa heel even maar zonder}{kwellend verlangen}\\

\haiku{- Dat komt er niet op,,.}{aan ik weet het toch klonk het}{besliste antwoord}\\

\haiku{Hij waardeerde juist:}{in zijn broeder datgene}{wat hemzelf ontbrak}\\

\haiku{Neen, waarachtig, nooit!}{had ik u met die baard en}{dat lorgnet herkend}\\

\haiku{- Wat kunnen mij die!}{lekkere diners en die}{sigaren schelen}\\

\haiku{Bijzonder knap, al!}{hebben zij tot nog toe zeer}{weinig gepresteerd}\\

\haiku{Zij waren laat, de;}{anderen zouden reeds op}{hen zitten wachten}\\

\haiku{Er is in u geen!}{schim van bewustzijn van uw}{hogere ikheid}\\

\haiku{De eerste rijen,,:}{zeer goed gezeten maakten}{het zich gezellig}\\

\haiku{murmelde hij op,.}{zijn beurt met vrome stem en}{glinsterende ogen}\\

\haiku{Werktuiglijk keek hij.}{op en zijn blik ontmoette}{die van Ghislaine}\\

\haiku{een groot geheim van...}{onbewuste liefde was}{in hen geboren}\\

\haiku{Hun rijtuig hield vlak,,;}{tegenover de kerk midden}{op de dorpsplaats stil}\\

\haiku{Want hij dacht er in '!...}{t geheel niet aan de strijd}{nu op te geven}\\

\haiku{liet Desgen\^ets zich,.}{onwillekeurig als een}{angstkreet ontvallen}\\

\haiku{- Gij denkt toch niet, hoop,?}{ik dat ik persoonlijk er}{enige schuld aan heb}\\

\haiku{Er was geen weerstand:}{mogelijk tegen een zo}{brutale aanval}\\

\haiku{Ja, d\'at was wel wat.}{hij zolang had willen en}{niet durven zeggen}\\

\haiku{Er was geen woord aan,!}{toe te voegen geen woord aan}{te veranderen}\\

\haiku{XVIII Twee dagen.}{gingen voorbij zonder dat}{er iets gebeurde}\\

\haiku{Als ik zou denken.}{dat je te weinig vraagt zal}{ik je meer geven}\\

\haiku{En een misnoegde,.}{bijna minachtende trek}{kwam op zijn gezicht}\\

\haiku{Ik eet zo weinig, '.}{mogelijk vlees ik hebn}{gruwel aan vleeseten}\\

\haiku{Waarom toch moest nu?}{weer die nare storing in}{zijn leven komen}\\

\haiku{- Kijk daar eens naar, sprak,.}{hij haar voor het beeld van de}{Gioconda brengend}\\

\haiku{En in 't bos zelf:}{ontmoette men slechts zeldzaam}{enkele wezens}\\

\haiku{De een, Cosaque, was een,,;}{mooie grote wit-en-bruin}{gevlekte barzoi}\\

\haiku{'t Is toch ook zo '!}{ontzettend rijk van kleur in}{t volle daglicht}\\

\haiku{en ik... zal mama... '.}{inviteren en misschien}{n paar vriendinnen}\\

\haiku{Florence trad met,.}{Maxime en Paul de trappen}{af hen tegemoet}\\

\haiku{- Maxime, jij kent ze,,?}{allen niet waar zal jij de}{voorstellingen doen}\\

\haiku{- een hansworst als die,.}{jonge snob een gansje als}{die dikke Elise}\\

\haiku{- O, ik had haar niet,!}{mogen laten gaan of ik}{had mee moeten gaan}\\

\haiku{Hij zou haar nu maar,.}{niet naar Oostende volgen}{dat zou zwakheid zijn}\\

\haiku{elle qui est si,!}{jeune  et si jolie}{et si distingu\'ee}\\

\haiku{en samen gingen.}{zij door bossen en lanen}{naar Far-West terug}\\

\haiku{Hij was echter vast.}{besloten haar met koelheid}{te bejegenen}\\

\haiku{- Zoals je wilt, beet,.}{zij eindelijk kortaf met}{sissende lippen}\\

\haiku{Hij k\'on het zo niet,.}{langer uithouden er moest}{een eind aan komen}\\

\haiku{- Liever de dood, ja,!}{liever de dood dan zulk een}{ellendig leven}\\

\haiku{Werktuiglijk, versterkt,}{door zijn besluit liep hij steeds}{verder langs het meer}\\

\haiku{Je zou ze haast zo.}{spoedig hebben dan als je}{ze zelf gaat halen}\\

\haiku{Soms  hield hij strak,;}{de hand aan zijn kin als in}{gespannen denken}\\

\haiku{Ik vind zelfs dat je '.}{r lang over gedaan hebt v\'o\'or}{het te ontdekken}\\

\haiku{Zij kan het enkel,.}{doen uit vrije wil indien haar}{aard er haar toe noopt}\\

\haiku{een aantal vrouwen;}{kunnen hebben in plaats van}{maar \'e\'en enkele}\\

\haiku{Enkele heren.}{bedienden zich en staken}{sigaretten op}\\

\haiku{Alfred hield zich of;}{dat vlugge besluit hem niet}{eens verwonderde}\\

\haiku{Hij stapte met Cosaque ',.}{int bootje en roeide}{naar het eilandje}\\

\haiku{En hij knelde zijn.}{vuisten ineen en sloot zijn}{tanden op elkaar}\\

\haiku{Zijn ogen strakstaarden,.}{erop als in bedwelming}{op cijfers van vuur}\\

\haiku{Ik dacht dat je bij,.}{de Loebmullers of bij de}{Berlaimonts was}\\

\haiku{- Madame, as 't, '...}{ou belieftk zoe zeu geirn}{iets van ou weten}\\

\haiku{Moar enfin, eefer;}{de Cuup\`ere es meschien heur}{hoemoaksterigge}\\

\haiku{want weet je, mooi heb, - '.}{ik haar nooit kunnen vinden}{zon modepop}\\

\haiku{Jouw naam, onze naam;}{moet aan de publieke spot}{onttrokken worden}\\

\haiku{Haar ziel, haar liefde,,;}{dat wist hij wel was al lang}{voor hem verloren}\\

\haiku{- De hoofdzaak is, dat '.}{alles nu gaat zoals we}{t hebben willen}\\

\haiku{Oneindig was de.}{stilte en de eenzaamheid}{die hem omringde}\\

\haiku{Waarom liet ze hem,?}{niet met rust nu toch alles}{tussen hen dood was}\\

\haiku{Hij had geen zin meer '.}{haar terug te nemen en}{daarmee wast uit}\\

\haiku{dacht hij, als ik haar?}{plotseling om de hoek van}{een straat ontmoette}\\

\haiku{Ik zie er zeker,.}{heel heel anders uit dan toen}{ik nog je vrouw was}\\

\haiku{Hoe vreemd en vals die,,.}{woorden klonken begreep hij}{niet voelde hij niet}\\

\haiku{het licht er als 't.}{ware van gouden glans over}{de bloemenvlakte}\\

\haiku{XXXIII En zachtjes...}{aan gaf hij zich aan de macht}{der bekoring over}\\

\haiku{Tussen tien en elf....}{zou hij bij Florence zijn}{Verzameld werk}\\

\subsection{Uit: Verzameld werk. Deel 2}

\haiku{De Strijd, de eerste,, \&,.}{uitgave Rotterdam Nijgh}{Van Ditmar 1918}\\

\haiku{- 'k Ben 't ik, boas,.}{Van Doalen antwoordde}{Alfons eindelijk}\\

\haiku{praatte hij met luid,;}{galmende stem alsof hij}{op de akker was}\\

\haiku{In het duistere;}{van de  nacht kon hij niets}{van haar gezicht zien}\\

\haiku{zij moesten ook maar eens ';}{om \'e\'en uurs nachts opstaan}{en mee gaan slijten}\\

\haiku{En alles om hen:}{heen kreeg nu ook meer en meer}{vaste vorm en kleur}\\

\haiku{Wat dachten ze wel?}{met hun lanterfanten en}{hun gekheid maken}\\

\haiku{Het was stikwarm '.}{in huis ent zweet brak uit}{op de gezichten}\\

\haiku{De klokkeslag van,:}{de lange trage uren was}{zo gauw geslagen}\\

\haiku{{\textquoteleft}dag mejonkvreiw en{\textquoteright},.}{gezelschap en gingen druk}{voort met hun arbeid}\\

\haiku{zweir mij dat er nie '!}{anders gebeurdn es en}{dat de sloeber liegt}\\

\haiku{Het jong begijntje '.}{scheen haar vragend iets int}{oor te fluisteren}\\

\haiku{- La mee Rozeke.}{in de wijnkels en ik mee}{Fons ievers elders}\\

\haiku{dat die jongen b'ron}{d'r euk bij woare en dat}{hij ons van den oavond}\\

\haiku{men wist zelfs niet wie ';}{hij was en of hij opt}{kasteel vertoefd had}\\

\haiku{En bij uw ouders,.}{Leo en Marie Dezen avond}{het avondmaal zult eten}\\

\haiku{Zij was gelukkig,.}{door en met Alfons en dat}{maakte alles goed}\\

\haiku{Zij haalde uit de,;}{eetkast twee grote witte}{koppen en een bord}\\

\haiku{verschrikte 't jong,.}{begijntje de handen in}{elkaar geslagen}\\

\haiku{Langzaam en triestig '.}{schudde hijt hoofd en week}{terug naar de deur}\\

\haiku{Hij vouwde een van,:}{de stukken open wees op het}{couponsblad en zei}\\

\haiku{- De loatste coupon, '.}{es vervallen van vandoag}{af meugtem knippen}\\

\haiku{Ook de baron, haar,,,.}{vader zag er bekommerd}{somber triestig uit}\\

\haiku{Zij kwamen binnen,:}{terwijl Smul het paard bij de}{stal ging uitspannen}\\

\haiku{orakelde hij ruw,.}{met een rechte blik op Dons}{uit de krib komend}\\

\haiku{eerst als een heel fijn,,;}{kleurloos stuifmeel nauw zichtbaar}{in de grijze lucht}\\

\haiku{leupt gij er achter, ';}{aangezien da ge toch nie}{mier verstandn h\^et}\\

\haiku{Het vroor en al de ';}{sterren tintelden ins}{hemels donkerblauw}\\

\haiku{Papa en mama.}{dachten dat ik hem op reis}{wel zou vergeten}\\

\haiku{riep Rozeke, hoe.}{langer hoe dieper door het}{voorstel afgeschrikt}\\

\haiku{- Ge meug gerust zijn,, '.}{bezinne antwoorddet}{Geluw Meuleken}\\

\haiku{zuchtte Rozeke,.}{met de hand het bonzen van}{haar hart bedwingend}\\

\haiku{- Hahaha!... 't Spijt ' '!}{mij dak nie liever op}{ou gewedn h\`e}\\

\haiku{Gelukkig was het.}{viertal nu reeds weer in druk}{gepraat en gezwets}\\

\haiku{Wat werd het eensklaps,}{stil in Rozekes leven}{na al de drukte}\\

\haiku{En zij waagde de:}{vraag die haar boven alles}{interesseerde}\\

\haiku{of hij voelde pijn.}{in de zij als iemand die}{te hard gerend heeft}\\

\haiku{- Blijf maar zitten, blijf,;}{maar zitten riep dringend de}{jonge barones}\\

\haiku{- Gij moogt hem vooral,.}{niet laten werken nog van}{heel de zomer niet}\\

\haiku{Reeds lag het vroege;}{lentewerk dringend op de}{akker te wachten}\\

\haiku{En toch... sommige:}{dingen kon noch mocht zij zo}{niet blijven dulden}\\

\haiku{en zij wilde ook.}{de min met het wagentje}{doen binnenkomen}\\

\haiku{En hoe moest het op?}{de boerderij ook gaan als}{hij eenmaal weg was}\\

\haiku{Om negen uur kwam '.}{een rijtuig vant kasteel}{Alf ons afhalen}\\

\haiku{, zijn beide ogen dof, '.}{en doods nut gelaat haast}{zonder uitdrukking}\\

\haiku{Maar het verveelde.}{haar toch ook en zij ging er}{een eind aan maken}\\

\haiku{riep zij nog, met het.}{Geluw Meuleken naast het}{rijtuig meehollend}\\

\haiku{Het was niet vreemd voor,,.}{haar zij was niet bang het scheen}{haar zo natuurlijk}\\

\haiku{Smul en Vaprijsken,,.}{gingen er rechts en links als}{wakers naast zitten}\\

\haiku{- Haw\`el joa, 't es '!}{precies doarmee datt}{uitgekomen es}\\

\haiku{Zij stond met hoge;}{kleur te beven en wist niet}{meer wat te zeggen}\\

\haiku{En v\'o\'or ze de tijd.}{had nog een woord te spreken}{was hij de deur uit}\\

\haiku{- 'k En weet 't nie,, '.}{bezinne hijn ziet er}{toch moar oardig uit}\\

\haiku{Zij huilde niet, maar.}{de ogen flikkerden vreemd in}{haar doodsbleek gelaat}\\

\haiku{- Gij zijt een schurk en.}{uw plaats is niet hier maar in}{de gevangenis}\\

\haiku{- Dit is de eerste,.}{en de allerlaatste keer}{dat ik u waarschuw}\\

\haiku{- Het arme beest krijgt,.}{in mijn plaats de schoppen en}{de slagen dacht zij}\\

\haiku{alleen de vrees voor.}{andere ongelukken}{beangstigde haar}\\

\haiku{Doch haar hart sloeg kalm.}{en gelijkmatig en zij}{voelde geen emotie}\\

\haiku{Met strak-stugge.}{blik van niet-begrijpen}{staarde zij hem aan}\\

\haiku{En zij ging naar het,,.}{venster bij de wieg waarin}{haar jongste kind lag}\\

\haiku{En 't leven ging,,;}{opnieuw zijn trage stille}{dagelijkse gang}\\

\haiku{en wanneer zij hem;}{nog zag was hij bijna als}{een vreemde voor haar}\\

\haiku{Nonkelken was rijk.}{en had nooit anders dan voor}{zijn plezier geleefd}\\

\haiku{het leek hem dat hij ';}{eensklaps zou genezen zijn}{zodra hijt had}\\

\haiku{hij wist nog niet waar,;}{de bouwlanden lagen waar}{de huizen stonden}\\

\haiku{- Doet er mee lijk of,,.}{ge wilt veur mij es alles}{goed zei meneer Vit\`al}\\

\haiku{ging plotseling een.}{zware stem aan de verste hoek}{van de tafel op}\\

\haiku{Da z' heur nonkel ', '.}{nien ha ze zoe moeten}{inn kleuster goan}\\

\haiku{Meneer Vit\`al was tot.}{nog toe in geen enkele}{dorpsherberg geweest}\\

\haiku{Het was te dom, hij,.}{had er eensklaps genoeg van}{van die stompe lui}\\

\haiku{De flinke tocht door;}{de avondkoelte had hem weer}{eetlust gegeven}\\

\haiku{Het meisje schudde,:}{langzaam ietwat verlegen}{glimlachend het hoofd}\\

\haiku{Hij was nu weer goed.}{en gelukkig gestemd en}{dacht een grapje uit}\\

\haiku{Hij boog zich over 't;}{tafeltje waaraan zij met}{hun vieren zaten}\\

\haiku{en haar uiterlijk '.}{was hem opt eerste zicht}{niet meegevallen}\\

\haiku{en ditmaal had ze.}{op hem een heel andere}{impressie gemaakt}\\

\haiku{Hij proefde er even,.}{van trok een zuur gezicht en}{schoof het dan opzij}\\

\haiku{hij voelde zelf een;}{soort gezelligheid om daar}{nog wat te blijven}\\

\haiku{Maar ook al proefde, ';}{hij er slechts even vant werd}{op de duur toch veel}\\

\haiku{Maar hij zag aan haar}{verbouwereerd gezicht dat}{ze zijn honende}\\

\haiku{had hij dan niet meer '?}{het recht te leven zoals}{t hem behaagde}\\

\haiku{Zou ze misschien van...}{de gelegenheid gebruik}{willen maken om}\\

\haiku{Meneer Vit\`al hield zich '.}{of hijt niet hoorde en}{liet maar volop gaan}\\

\haiku{Waarom nu weer dat!}{wilde en gevaarlijke}{snorren door de nacht}\\

\haiku{De Reu loerde even,;}{wantrouwig om naar Netje}{die in huis verdween}\\

\haiku{Meneer Vit\`al fronste.}{de wenkbrauwen en staarde}{peinzend v\'o\'or zich uit}\\

\haiku{- Hij kende niemand.}{hier die met hem voelen en}{genieten kon}\\

\haiku{Toen zag hij ook de,;}{enkele korte regels}{van het fijn geschrift}\\

\haiku{Kom moar mee lijk of}{ge zijt en zend iemand noar}{huis om te zeggen}\\

\haiku{zich boos gebarend.}{en haar hand uitslaand als om}{een klap te geven}\\

\haiku{herhaalde Taghon,.}{met nadruk de ogen rond en}{strak van overtuiging}\\

\haiku{'t was of nu ook;}{een vijand h\'a\'ar van hem zou}{kunnen wegvoeren}\\

\haiku{Ook vader Peutrus ':}{en moeder Lie waren heel}{int zwart gekleed}\\

\haiku{Al dat lawaai en.}{gewoel overweldigde haar}{en gaf haar hoofdpijn}\\

\haiku{Een zoete wraak zou ',.}{t voor hem zijn indien hij}{daarin slagen kon}\\

\haiku{en v\'o\'or 't naar bed,.}{gaan leidde hij haar even rond}{in al de kamers}\\

\haiku{- 'k Zal tegen ten!}{halver ien were thuis zijn}{om te dineren}\\

\haiku{- Toe, Nathelie, leupt;}{al gauwe noar ou keuken}{en schept de soep uit}\\

\haiku{Dat zou ook niet staan,.}{ginds in die achterhoek van}{De Groene Linde}\\

\haiku{- O, en ik die toch,.}{zeu geirne sampoande}{drijnke klaagde zij}\\

\haiku{De bontjes van 't;}{barontje waren rood en die}{van meneer Vit\`al groen}\\

\haiku{gilde plotseling.}{meneer Vit\`al met van woede}{uitpuilende ogen}\\

\haiku{Hij hoorde, als een, ';}{vaag echot geschreeuw van de}{menigte ginds ver}\\

\haiku{Het deed hem goed haar,.}{zo te zien haar dicht en trouw}{bij zich te voelen}\\

\haiku{As ik euk moar {\textquoteleft}'t{\textquoteright} ',.}{bolleken nien krijge}{gelijk Nonkelken}\\

\haiku{Was dat leven voor '?}{t meisje t\'och te saai en}{te eenzaam geweest}\\

\haiku{Marguerite.}{scheen geheel weer tot kalmte}{en rust gekomen}\\

\haiku{- 't Is de schande,,!}{de oneer de ondergang}{van de familie}\\

\haiku{Meer dan zulke en.}{dergelijke was er uit}{hem niet te krijgen}\\

\haiku{- Nog niet, antwoordde,.}{stil mevrouw Dudemaine even}{naar hem omkijkend}\\

\haiku{- Dwing niet, dwing niet, denk ',.}{aant verleden snikte}{mevrouw bleek van angst}\\

\haiku{Laten wij hem toch.}{eerst en vooral gezond en}{sterk zien te maken}\\

\haiku{- Kom mee, en papa,,!}{ook er is iets gebeurd en}{ik moet het weten}\\

\haiku{- Dat is er een van!}{de bediening of van de}{boerderij geweest}\\

\haiku{- Doe ou wa klieren,,.}{aan M\'edard en kom mee}{noar de boerderije}\\

\haiku{- O, M\'edard, as 't, ',,!}{u blieft ast u blieft helpt}{hem blijf toch bij hem}\\

\haiku{riep de man hees van,.}{schrik terwijl hij zijn geweer}{haastig terugtrok}\\

\haiku{En zij liep zelf met,.}{het glas naar de kelder haar}{moeder achterna}\\

\haiku{'t Gemoed van de ';}{mensen was in stemming met}{t omgevende}\\

\haiku{En langzaam liep hij,, '.}{verder vaag teleurgesteld}{naart dorpje toe}\\

\haiku{zij merkten, en met!}{welk een zoet gestreelde hoop}{en vertedering}\\

\haiku{dat kraken van het, '!}{ijs wat boordet door haar}{benen en haar hart}\\

\haiku{- 'k Wou maar dat 't,,.}{uit was en er sneeuw kwam om}{te kunnen arren}\\

\haiku{Hij was boos, zowel,;}{als Elsa die een hoge}{kleur kreeg van emotie}\\

\haiku{Wat wilde hij toch?}{eigenlijk en wat moest ze}{met hem aanvangen}\\

\haiku{Die avond, toen hij aan,.}{de buitenrand verscheen was}{zij er niet te zien}\\

\haiku{Op 't Landjuweel,,.}{galmde luid-vermanend}{de eerste etensbel}\\

\haiku{een man uit het dorp,.}{een jonge boer of knecht uit}{de omtrek binnen}\\

\haiku{maar het gebeurde:}{ook dat vreemdelingen zich}{daar even ophielden}\\

\haiku{- 't Is goed, M\'edard,,.}{jij hoeft niet verder mee te}{komen zei mevrouw}\\

\haiku{de Wet zou er zich,!}{mee bemoeien want het kind}{was minderjarig}\\

\haiku{Maar de nood dwong, er;}{moest onmiddellijk iets op}{gevonden worden}\\

\haiku{Zijn ogen weifelden,.}{richtten zich even strak en als}{bedwelmd ten gronde}\\

\haiku{Haar hart joeg sneller,.}{een weke emotie glansde}{in baar bleke ogen}\\

\haiku{Iets woelde in haar,,.}{scherp-kwellend diep van}{onrust en van angst}\\

\haiku{Wel voelde ze soms:}{af en toe een kort verwijt}{in zich opkomen}\\

\haiku{En ofschoon tegen, '.}{haar zin had zet hem niet}{kunnen beletten}\\

\haiku{Mevrouw Dudemaine.}{poogde er hem ook niet meer}{van af te houden}\\

\haiku{Zij zwegen beiden. '}{om naar het loeien van de}{wind te luisteren}\\

\haiku{riep hij, eensklaps haar.}{nijdig wegduwend en zelf}{de deur openrukkend}\\

\haiku{- Ala toe, moeder, kom, '!}{binnen en doe de deure}{toet es hier koud}\\

\haiku{Hij keek eens vluchtig,;}{naar de deur of hij daar ook}{soms iemand hoorde}\\

\haiku{een pastoor in haar -.}{familie te bezitten}{aan hem was volbracht}\\

\haiku{Juffrouw Constance.}{was een oude vrijster van}{reeds bij de veertig}\\

\haiku{- Geen vrijage, noch,,!}{binnenshuis noch buitenshuis}{of dadelijk weg}\\

\haiku{Dat was erg, z\'o erg, '.}{dat zet haast niet kon noch}{wilde geloven}\\

\haiku{riep juffrouw Toria,.}{geschokt met grote mond en}{uitpuilende ogen}\\

\haiku{'t es doarover ', '.}{dak ou kome spreken}{beefdet Ezelken}\\

\haiku{Wie weet ook of het,?}{om haar geld alleen was dat}{hij haar gevraagd had}\\

\haiku{Tok tok tok, hoorde.}{zij de meid aan haar broeders}{slaapkamer tikken}\\

\haiku{zij greep ineens de...}{knop van de keukendeur en}{duwde die ruw open}\\

\haiku{Er was een korte,,.}{poos volkomen doodse als}{versteende stilte}\\

\haiku{C\'elines gezicht,.}{en manieren bevielen}{haar niet die middag}\\

\haiku{H\'e-je nou gezien ''!}{wa veurn schandoal da}{g in ou huis h\^et}\\

\haiku{zij riep ten slotte;}{haar meid naar binnen en deed}{haar licht aansteken}\\

\haiku{Zij sloot de brief in,.}{het couvert en Aamlie bracht}{hem naar de bus}\\

\haiku{- Ge meug gerust zijn,,.}{ieffreiwe antwoordde de}{meid reeds in de gang}\\

\haiku{ik u door Ivo uwe.}{koffers met alles er in}{wat u toebehoort}\\

\haiku{Het Puipken, trouwens,;}{voelde zich daar ook blijkbaar}{niets op zijn gemak}\\

\haiku{'t Puipken stond even,}{onthutst maar v\'o\'or hij er meer}{van kon vertellen}\\

\haiku{Juffer Toria, en,.}{ook het Ezelken waren maar}{half gerustgesteld}\\

\haiku{Het scheelde weinig ' '.}{oft glas stortte uit de}{hand vant Ezelken}\\

\haiku{Wat moest er van haar,,?}{worden waar moest ze heen als}{juffer Toria stierf}\\

\haiku{Op de drempel van,.}{de gang verscheen de koster}{die hem gevolgd had}\\

\haiku{'t Was verwacht en '.}{alst ware reeds gebeurd}{v\'o\'or het gebeurd was}\\

\haiku{Mirza moest het eerst,.}{bediend worden die was de}{ongeduldigste}\\

\haiku{- Ik, da huis keupen,, ' ';}{wa peist-ekn h\`e}{doar gien geld veuren}\\

\haiku{- 'k Zal de koster,;}{loate roepen zuchtte}{nog eens het Ezelken}\\

\haiku{Kan 't hij nou euk, '! '}{al nie verdroagen datn}{kind zijne stiel liert}\\

\haiku{- Pas moar op da ge ',.}{niet te lankn wacht zei streng}{de geestelijke}\\

\haiku{De oude vrek zat.}{met uitgepuilde ogen op}{zijn stoel te schudden}\\

\haiku{- 'k Zal 't ulder, ' ',;}{teugenk zalt ulder}{teugen snikte Ivo}\\

\haiku{De hevige schok;}{had Guustje plotseling van}{zijn kwaal genezen}\\

\haiku{Een dikke schoof stro,.}{waarmee men de hond in zijn}{hok vastgepropt had}\\

\haiku{Toen liepen zij even,.}{rondom hun stallen of ook}{d\'a\'ar soms onraad was}\\

\haiku{- Haw\`el, bezinne, ' ', ' ':}{os get nien weetk}{zalt ou zeggen}\\

\haiku{- Ha, wilt-e nou,',:}{ne kier wa weten onz Mrie}{riep krijsend de vrouw}\\

\haiku{- 't Es goed, zei de,.}{politieman kortaf zijn}{boekje dichtklakkend}\\

\haiku{Machinaal streek hij.}{met de hand over zijn wang en}{keek naar zijn vingers}\\

\haiku{Eerst werd voorspeld dat;}{geen fatsoenlijk mens er nog}{de voet zou zetten}\\

\haiku{- Mrie, 'k ko... ko... 'k!}{kom ou vroagen of da ge}{mee mij wilt treiwen}\\

\haiku{Wat 'n emotie, toen!}{dat span voor de eerste maal}{op het hoevetje kwam}\\

\haiku{Guustje zag het en '.}{het beet hem even als een scherp}{venijn int hart}\\

\haiku{- 'k Begin al woarm,, {\textquoteleft}{\textquoteright}.}{te krijgen zei P\'e\'elzie haar}{katte loshakend}\\

\haiku{- Ieffreiwe, geef mij,.}{euk ne kier da boeksken dat}{doar op de toafel ligt}\\

\haiku{- Joa joa 't, 't es,.}{nou al firm k\^ewd beaamde}{Guustje bibberend}\\

\haiku{Maar eensklaps werd hun.}{aandacht door iets anders in}{beslag genomen}\\

\haiku{'k H\`e k\^ewwe, ',.}{k h\`e doanig k\^ewwe}{bibbertandde Ivo}\\

\haiku{{\textquoteleft}Heb je 't nu goed!}{begrepen of moet ik het}{nog eens herhalen}\\

\haiku{Er waren tal van,;}{acrobaten die blijkbaar een}{vertoning gaven}\\

\haiku{de orang-oetang, ook,;}{helemaal een mens die slechts}{de spraak te kort had}\\

\haiku{Als grote, witte.}{schepen dreven die wolken}{in de blauwe lucht}\\

\haiku{Elvire zou als;}{een verwend kindje haar zin}{krijgen met Fonske}\\

\haiku{Eerst had de moeder,,.}{zo onverhoeds gepakt wel}{enige aarzeling}\\

\haiku{En hij lachte heel.}{hard om zijn buitengewoon}{geestig gezegde}\\

\haiku{Hij hoopte wel, dat.}{hij er twee of drie van de}{hand zou kunnen doen}\\

\haiku{- Al was ie-hij de, ',.}{Paus wen kennen hem niet}{zei Sylvain smalend}\\

\haiku{hij begon ook in:}{andere kunstuitingen}{belang te stellen}\\

\haiku{hij huurde het per,;}{brief van de baron aan wie}{het toebehoorde}\\

\haiku{Hij haatte hem, had,.}{hem kunnen slaan hem van haar}{kunnen wegrukken}\\

\haiku{Zij waren aan zijn ',.}{huisje ent speet hem dat}{zij er reeds waren}\\

\haiku{Fonske wou, ondanks,.}{Van Belleghems aandringen}{niet binnenkomen}\\

\haiku{en, zoals ik in,.}{die tijd schreef zou ik thans niet}{meer k\'unnen schrijven}\\

\haiku{Boerke was een kort,,.}{dik ventje van om en bij}{de vijfenzestig}\\

\haiku{Zij keken allen,,;}{op verrast alsof zij hem}{pas nu ontwaarden}\\

\haiku{en Tibi, van zijn,:}{kant was nog geweldiger}{dan bij het heengaan}\\

\haiku{het erf bleef stil en,;}{leeg en leeg en stil lag ook}{de weg daarbuiten}\\

\haiku{Maar de ochtend leek,.}{dan zo lang dat er geen eind}{aan scheen te komen}\\

\haiku{Het gewoon kleindorpse,;}{alledaagse leven scheen}{ver en vergeten}\\

\haiku{Wat was het land schoon!}{met de alom rijpende}{oogsten en vruchten}\\

\haiku{Tibi volgde hem,,.}{met trage schreden en bleef}{staan waar Boerke stond}\\

\haiku{Die alledaagse.}{bejegening ontstemde}{Boerke vagelijk}\\

\haiku{Waarom hij nog kwam,,.}{ja dat kon hij zelf ook zo}{precies niet zeggen}\\

\haiku{Boerke schudde 't.}{hoofd en keek beteuterd naar}{het knappe meisje}\\

\haiku{Maar voor hun rust was}{plaats en ruimte nodig en}{reeds zo ontelbaar}\\

\haiku{Moar hij 'n kan mee '.}{heur nie treiwen omdat hij}{gien fortuunn h\'et}\\

\haiku{vroeg ze nog, ziende.}{dat hij zijn pijp aanstak en}{naar de voordeur ging}\\

\haiku{Machinaal deed hij.}{zijn klompen uit en nam die}{in de linkerhand}\\

\haiku{Een klein lampje was.}{aangestoken en brandde}{op een tafeltje}\\

\haiku{Meteen sloeg hij, van,.}{terzijde aandachtig de}{gezichten gade}\\

\haiku{Maar zelfs in een rijk.}{huwelijk had hij tot nog}{toe geen zin gehad}\\

\haiku{De levenloze,.}{dingen schenen te rusten}{evenals de mensen}\\

\haiku{menier den dokteur,, '.}{kreet Meerken alsof haart}{ergste werd gevraagd}\\

\haiku{Dat was toch nog een,,.}{afleiding iets eigens iets}{uit het verleden}\\

\haiku{Zij wilden weten,:}{w\'eten en Roomnietje drong}{opgewonden aan}\\

\haiku{Het leek wel of hij '.}{zelf int geheel niet meer}{wist wat hij wenste}\\

\haiku{Kom, Sefrien, help mij ', ''!}{n beetsen dak z in}{mijn oarms kan pakken}\\

\haiku{Zij kenden hen door ' '}{en door ent was ermee}{te leven zoals}\\

\haiku{de gasten liepen.}{langzaam langs de ramen en}{keken naar binnen}\\

\haiku{- As ik iets veur ou,, '!}{kan doen gelijk watte ge}{n moet moar spreken}\\

\haiku{- Zie je wel, klaagde, -!}{Reinilde we zijn al veel}{te lank gebleven}\\

\haiku{Het drong hem als een.}{prop de keel dicht en hij kreeg}{tranen in zijn ogen}\\

\haiku{Maar Keijsder wou.}{zijn paard niet alleen laten}{staan en bedankte}\\

\haiku{Daar stond ginds ver de.}{molen te draaien dichtbij}{zijn vroeger gehucht}\\

\haiku{Hij ademde diep en.}{voelde nieuwe kracht door heel}{zijn lichaam stromen}\\

\haiku{De knecht was een reeds;}{bejaarde man en deed hem}{aan Sefrien denken}\\

\haiku{Hij glimlachte bij:}{het idee en meteen was zijn}{besluit genomen}\\

\haiku{- D'r komen nog al,.}{dikkels brieven van Oscar}{berichtte Roze}\\

\haiku{Hij schoof zijn stoel dicht.}{bij de hare en sloeg zijn}{arm om haar middel}\\

\haiku{Hij liet haar los, en,.}{beiden keken naar elkaar}{met sprekende ogen}\\

\haiku{Weer steeg de gloedgolf.}{over Florimonds gelaat en}{bleekte langzaam weg}\\

\haiku{Wie het waagde zijn;}{vader te mishandelen}{was zijn leven kwijt}\\

\haiku{hij sloot zijn raam en,.}{holde de trappen af om}{dat bij te wonen}\\

\haiku{anderen namen,.}{lachend wat mee zonder hun}{beurs uit te halen}\\

\haiku{hij liep, hij liep maar, '.}{door alst ware door het}{noodlot voortgejaagd}\\

\haiku{Hij zag van verre ',;}{t huisje staan van Dikke}{Roze ongedeerd}\\

\subsection{Uit: Verzameld werk. Deel 3}

\haiku{werken waarin hij [}{vooral herinneringen}{ophaalt of verwerkt}\\

\haiku{{\textquoteleft}De 29ste juli de.}{begrafenis bijgewoond}{van Cyriel Buysse}\\

\haiku{Even buiten 't dorp,,.}{op korte afstand van ons}{huis lag de Lusthof}\\

\haiku{Het kraakt, er komen,.}{sterren in maar het schijnt toch}{te kunnen dragen}\\

\haiku{Dat alles reden.}{wij voortdurend langs en wij}{zagen dat alles}\\

\haiku{Het ijs lag er steeds;}{onbetrouwbaar en had er}{een vuilgele kleur}\\

\haiku{en z\'o reuzesterk,.}{en taai was hij dat hij ons}{niet zelden overwon}\\

\haiku{Wij waren banger}{voor Guus dan voor zijn hond op}{het ijs en haastig}\\

\haiku{Er was een Peetse,,:}{Kins een Bruuntje Geelewie en er}{waren drie broeders}\\

\haiku{Ik keek en hoorde.}{dat alles aan met stille}{weemoed en emotie}\\

\haiku{hij leek z\'o sprekend,:}{dat ik naar hem toe ging en}{op de man af vroeg}\\

\haiku{Iedereen, oud of,,,.}{jong man of vrouw van klein tot}{groot was bang voor hem}\\

\haiku{En eigenlijk wist ':}{ik nooit precies wat mij wel}{t aangenaamst was}\\

\haiku{Eigenlijk voelde,.}{ik mij daar nooit zoals ik}{was of wezen wou}\\

\haiku{Er bleef mij trouwens.}{nog ruim voldoende liefde}{en illusie over}\\

\haiku{zo zwart alsof het.}{open water was en niemand}{durfde er overheen}\\

\haiku{En zo kwam ik, als, ';}{altijd aant heerlijke}{Meilegem-Zuid}\\

\haiku{ik was als van de;}{aarde opgetild en op}{wieken gedragen}\\

\haiku{zij schoven verder ',,;}{overt ijs teder omarmd}{amoureus-fluisterend}\\

\haiku{Ik wist dat zij het, ',;}{was ik hoordet aan haar}{stem dat zij het was}\\

\haiku{Even voorbij de bocht,.}{keek ik eens om en zag dat}{ze mij naoogden}\\

\haiku{Ik voelde dat ik,,.}{indruk had gemaakt ja dat}{ik overwonnen had}\\

\haiku{'t Was ook te gek,.}{zoals die steedse meisjes}{zich daar aanstelden}\\

\haiku{Ik schreide goed en,;}{diep uit daar in de stilte}{en de eenzaamheid}\\

\haiku{Ik verademde.}{alsof ik van een zware}{dreiging werd bevrijd}\\

\haiku{Ik bestelde een.}{tweede pousje en stak een}{grote sigaar op}\\

\haiku{En dat men die toch,,!}{hebben moest en veel om daar}{te kunnen leven}\\

\haiku{en die houden er,.}{ook wel de vrolijkheid in}{wees dat maar zeker}\\

\haiku{Ik haastte mij, ik;}{hijgde en zwoegde door de}{glinsterende sneeuw}\\

\haiku{Ik had het prettig,.}{gevoel dat ik daar een van}{de besten zou zijn}\\

\haiku{- Neen, pardon, zo niet,,.}{het linkerbeen naar achter}{professeerde ik}\\

\haiku{zij genoot, zij was...!}{tevreden en gelukkig}{gelukkig door mij}\\

\haiku{Het brandde op mijn,.}{lippen om het te vragen}{maar ik durfde niet}\\

\haiku{Wat zal er van mij,,?}{worden dacht ik meteen als}{het eens niet meer vriest}\\

\haiku{Het stroomde door mijn;}{hele lichaam heen als een}{elektrische trilling}\\

\haiku{'t Was stralend mooi,.}{weer nog mooier dan al de}{vorige dagen}\\

\haiku{Een mens die stevig;}{gegeten heeft is dikwijls}{log en loom en zwaar}\\

\haiku{Zij vergezelden,.}{mij om van mijn verrukking}{mee te genieten}\\

\haiku{Ik liet de dames,.}{voor en trad in een salon}{door Papa gevolgd}\\

\haiku{Er kwamen tranen,...}{in mijn ogen die langzaam over}{mijn wangen vloeiden}\\

\haiku{Dan zeiden weer de.}{ogen wat de mond nog niet had}{durven uitdrukken}\\

\haiku{{\textquoteright} Dat waren uit hun.}{schaal gehaalde en in melk}{gekookte oesters}\\

\haiku{Welke jonge mooie?}{vrouw is niet gelukkig in}{een modewinkel}\\

\haiku{En Maud beaamde,,.}{door een zwijgend hoofdgeknik}{haar tantes woorden}\\

\haiku{Een ma{\^\i}tre d'h\^otel.}{kwam naar mij toe en bood mij}{stil de wijnkaart aan}\\

\haiku{En af en toe was:}{het alsof de slapende}{reus even ontwaakte}\\

\haiku{Vraag excuus, smeek om, '!}{genade ofk sla je}{de dikke kop in}\\

\haiku{Als ik mij repte.}{kon ik de elektrische van}{halfdrie nog halen}\\

\haiku{een paar jongelui;}{die lachend pret maakten en}{wat dronken schenen}\\

\haiku{De halve nacht heeft,.}{ze gehuild na het diner}{bij Delmonico}\\

\haiku{Het troebleerde hem,;}{gaf hem soms iets rusteloos}{en ontevredens}\\

\haiku{maar zij zouden 't:}{wel gelaten hebben hem}{ooit weg te sturen}\\

\haiku{Men zag hem telkens;}{weer door sleutelgaten en}{door reetjes loeren}\\

\haiku{en dan was hij wel.}{eens de geestigste man van}{de ganse fabriek}\\

\haiku{Het kloppen op zijn,;}{bochel was een gewone}{dagelijkse grap}\\

\haiku{Zij zat daar als een,.}{mooie bloem of luxeplant als iets}{dat er niet hoorde}\\

\haiku{Bruteijn had er nu.}{eenmaal genoeg van en was}{niet meer te vangen}\\

\haiku{Soms was het lief en,.}{vriendelijk maar ook wel eens}{bezorgd en gedrukt}\\

\haiku{Het meest geraden;}{was in elk geval daar niet}{te lang te toeven}\\

\haiku{Het kwam al gauw tot;}{wederzijdse scheldwoorden}{en dreigementen}\\

\haiku{Het enige wat hij:}{er blijvend bij gewonnen}{had was zijn bijnaam}\\

\haiku{schreeuwde Guustje met, {\textquoteleft}{\textquoteright}.}{een stem die men tot in het}{stampkot horen kon}\\

\haiku{Eleken, het tweede,.}{meisje diende met stille}{bewegingen op}\\

\haiku{De vrouwen droegen,;}{zonhoeden die hun gezicht}{en hals beschutten}\\

\haiku{Men kende ze, die!}{zondagsgangen van Berzeel}{naar zijn eigen dorp}\\

\haiku{- Ze zoen ons beter ',.}{elkn keiw kieken geven}{mee sloa meende hij}\\

\haiku{En plotseling sloeg,.}{hij hem neer zo hard als hij}{kon in Pierkens nek}\\

\haiku{Hij was groot en zwaar,.}{met een dikke hangsnor en}{geverfde haren}\\

\haiku{Iek beveel u nog,.}{eens uit te scheid herhaalde}{de burgemeester}\\

\haiku{De Witte{\textquoteright} kreeg een.}{tranencrisis alsof ze}{heel zou wegsmelten}\\

\haiku{- Hij zoe wek willen, ',.}{moar hijn mag nie van zijn}{moeder snikte zij}\\

\haiku{riep Feelken ieder,,;}{ogenblik uit louter dolle}{uitgelatenheid}\\

\haiku{Zij spraken nooit van, {\textquoteleft}{\textquoteright}.}{een liter altijd van een}{kilo jenever}\\

\haiku{Meneer De Beule,,.}{het hart vol wrok vermeed met}{zijn zoon te spreken}\\

\haiku{Nu was dat alles,.}{dood zoals zijn vreugde met}{haar weggaan dood was}\\

\haiku{Hij had het even over,,.}{zaken op een bezorgde}{chagrijnige toon}\\

\haiku{En dat troostend en,.}{sterkend gevoel bleef hem bij}{enkele dagen}\\

\haiku{Zou hij dan nooit die?}{ellendige kruk in de}{duisternis vinden}\\

\haiku{Daarbinnen in het.}{huisje was het plotseling}{doodstil geworden}\\

\haiku{Toen ging eensklaps een,,.}{stem op een vrouwenstem die}{ietwat angstig vroeg}\\

\haiku{Hij stond daar nu en.}{wist eensklaps niet meer wat te}{doen of te zeggen}\\

\haiku{Hij wenkte stil de}{moeder bij zich en stopte}{het haar in de hand.}\\

\haiku{En zou de zoon hem?}{bij de keel niet grijpen en}{hem buiten smijten}\\

\haiku{- Ge weet dat toch wel,,.}{antwoordde zij stil met een}{blos de ogen neerslaand}\\

\haiku{en Leo liet zijn  , {\textquoteleft}!}{wilde schreeuw horen zijn luid}{bulderendOajo\'aek}\\

\haiku{{\textquoteright} dat wellicht tot in.}{het woonhuis door meneer De}{Beule werd gehoord}\\

\haiku{Of hij ooit met haar;}{zou trouwen bleef geheel in}{het onzekere}\\

\haiku{Vader had al niet.}{veel in te brengen thuis en}{Meries nog minder}\\

\haiku{Daarbuiten lag de,,,.}{sneeuw de kilheid de onrust}{de onzekerheid}\\

\haiku{Maar eensklaps toornden.}{toch even de woede en de}{opstand in hem los}\\

\haiku{Het geluid van zijn.}{stem overheerste het gedonder}{van de heibalken}\\

\haiku{Zou ze wellicht reeds '?}{iets weten en zouden ze}{t daarover hebben}\\

\haiku{Als hij nu maar geen, '.}{mensen ontmoette dan zou}{hijt wel halen}\\

\haiku{Kaboel, die ook mee,.}{wou kreeg de deur tegen zijn}{neus en piepte even}\\

\haiku{streelde Lisatje, met.}{vertedering de zachte}{wangetjes aaiend}\\

\haiku{- 't Denkt mij dat 't,,.}{nog koel es buiten meende}{madam De Beule}\\

\haiku{Hij hoorde ook zijn.}{vader en zijn moeder loom}{de trap opklimmen}\\

\haiku{Aan tafel sprak hij;}{geen woord en keek zijn zoon geen}{enkele keer aan}\\

\haiku{Of wist meneer De?}{Beule soms dat hij weer bij}{Siednie was geweest}\\

\haiku{Zijn lange pijp hing,,;}{er tussen twee spijkers in}{de schoorsteenmantel}\\

\haiku{- Ha moar, meniere, '!}{k kom hier die ijzere}{roe were brengen}\\

\haiku{Dit avontuur liet in;}{hem een wrange desem van}{verbittering na}\\

\haiku{en Pierken stelde:}{voor dat een deputatie}{van drie arbeiders}\\

\haiku{er kwam leven in.}{de loom-gedrukte groep}{en de ogen blonken}\\

\haiku{riep Pierken eensklaps...}{met een soort van wilde trots}{op zijn borst kloppend}\\

\haiku{We willen meinschen,!}{worden meniere en gien}{lastdieren mier zijn}\\

\haiku{Maar Eleken zei nooit,.}{veel mengde zich liefst in geen}{verwikkelingen}\\

\haiku{Zij ging weer naar de;}{haverkist toe en vulde}{ditmaal gul de maat}\\

\haiku{Om halfzeven kwam,,.}{als elke dag meneer De}{Beule beneden}\\

\haiku{- Ha, om mijn wirk te,, '.}{doene meniere schrikte}{t meisje hevig}\\

\haiku{En hij verliet de '.}{stal om aant gesprek een}{eind te maken}\\

\haiku{Het drietal stond met;}{de rug naar hem toe en had}{hem niet zien komen}\\

\haiku{Joa joa, med\'am, past '!}{op os g'in d'handen vant}{slecht vreiwevolk zit}\\

\haiku{De mannen hadden;}{gegroet zonder een ogenblik}{hun werk te staken}\\

\haiku{tot aan de kirke ' '!}{willen kruipen ast nie}{gebeurdn woare}\\

\haiku{en hij kwam strak v\'o\'or:}{Sefietje staan en begon}{te bromneuri\"en}\\

\haiku{Het speet hun niet dat.}{Pier en Feelken nu uit de}{fabriek wegbleven}\\

\haiku{Was alles weer in,?}{orde nu of moest er nog}{over gepraat worden}\\

\haiku{Een laddertje stond,;}{naast hem langswaar hij blijkbaar}{opgeklommen was}\\

\haiku{en zijn gezicht leek,,.}{zwart met uithangende tong}{alsof hij walgde}\\

\haiku{Hij keek om zich heen,.}{als gehinderd door de te}{drukke omgeving}\\

\haiku{- Mais... nous n'avons plus,.}{rien \`a faire ici meende}{de burgemeester}\\

\haiku{- Est-ce que,?}{vous n'avez besoin de rien}{monsieur le cur\'e}\\

\haiku{- Wacht totdat menier,.}{de p\'aster wig es antwoordde}{meneer De Beule}\\

\haiku{- Ge zilt loater,.}{allemoal mijn veurbeeld}{volgen zei Pierken}\\

\haiku{en keek haar neef  ,.}{strak in de ogen aan terwijl}{zij hem de hand gaf}\\

\haiku{Het was een knappe,,.}{man met donker haar mooie snor}{en sprekende ogen}\\

\haiku{Raymond spaarde zijn;}{vriend de schimpscheuten en}{de verwijten niet}\\

\haiku{Hij had het moeten...!}{weten weten hoe en wat}{zij voor hem voelde}\\

\haiku{Hij zou eens eventjes;}{met een paar vrienden meegaan}{naar een koffiehuis}\\

\haiku{Zij hadden nog niet,.}{alle hoop verloren maar}{zij vreesden zozeer}\\

\haiku{Zo'n magnifieke,!}{stad en die drukte en dat}{vrolijke leven}\\

\haiku{Meneer Dufour was.}{al aan de deur en hielp zijn}{zusters uitstijgen}\\

\haiku{herhaalde tante,.}{Estelle haar handen vroom}{in elkander slaand}\\

\haiku{Allen, even gestoord,.}{en bijna schrikkend keken}{met verbazing op}\\

\haiku{- Een eer die lastig,.}{is om te dragen meende}{tante Clemence}\\

\haiku{het leek wel of een;}{ongekende ramp over hen}{was neergekomen}\\

\haiku{Oe-Oe, op de grond,,.}{Impikoko op een stoel}{en aten met hem mee}\\

\haiku{- Pas moar op da g' '!}{ou nietn verongelukt}{mee al da rijen}\\

\haiku{- Gee moar hier, Manse, ',.}{k zal gauwe genezen}{zijn glimlachte hij}\\

\haiku{riep hij verrast, als.}{naar gewoonte Frans en Vlaams}{door elkaar mengend}\\

\haiku{Wat ons betreft,... wij.}{willen er absoluut niets}{gemeens mee hebben}\\

\haiku{- Wij kunnen hem toch!}{niet beletten hier te paard}{voorbij te rijden}\\

\haiku{Het kwam hem voor of.}{een paar lui uit de buurt hem}{spottend nakeken}\\

\haiku{Hij duwde een van.}{zijn vensterluiken open en}{staarde in de nacht}\\

\haiku{Hij voelde 't nu,.}{ineens heel diep het hart vol}{wroeging en verwijt}\\

\haiku{Clement dus, zei Max -:}{glimlachend met als tweede}{naam die van papa}\\

\haiku{De doopplechtigheid, ',.}{int kleine stadje was}{iets buitengewoons}\\

\haiku{Zij zei hem niet, dat.}{zij bij meneer de pastoor}{was te biecht geweest}\\

\haiku{Vroeger, onder zijn,.}{blik sloeg ze dadelijk haar}{ogen neer en kleurde}\\

\haiku{Zij bekwamen er, ';}{niet van zij vielen alst}{ware uit de lucht}\\

\haiku{Max blikte koel in '.}{t onbekende en streek}{zijn baard naar voren}\\

\haiku{Max deed of hij dat,;}{zeer betreurde hoewel hij}{het begrijpen kon}\\

\haiku{Max ging opendoen en:}{zijn dienstmeisje stond voor hem}{en zei fluisterend}\\

\haiku{Zelfs de oude meid;}{Eemlie was met een aardig}{sommetje bedacht}\\

\haiku{zijn huis, zijn bedrijf,,!}{zijn goede gedienstigen}{zijn trouwe dieren}\\

\haiku{Hij zag haar roerloos, ',.}{staan int zwart gekleed met}{een krijtwit gezicht}\\

\haiku{zo gaan we kalmpjes...,...}{aan naar huis en Raymond blijft}{bij ons ja zeker}\\

\haiku{De koetsier tikte.}{zijn paarden en ratelend}{reed het rijtuig weg}\\

\haiku{Zij liepen door een.}{lange gang en hielden bij}{de laatste deur stil}\\

\haiku{Het nonnetje boog '.}{even naart sleutelgat en}{scheen te luisteren}\\

\haiku{- Ja, ze slaapt, z'is nog, ';}{wat moe zeit nonnetje}{zich weer oprichtend}\\

\haiku{Een vraag brandde op,.}{Clara's lippen die zij haast}{niet durfde stellen}\\

\haiku{Wellicht had ze soms,.}{andere verlangens doch}{zij uitte die niet}\\

\haiku{Zij was in 't zwart.}{gekleed en had haar hoed met}{roze bloemen op}\\

\haiku{Toen zij enkele,}{minuten had gelopen}{kwam haar bij een bocht}\\

\haiku{'k Zal mee iene, ';}{van de twie\"e treiwen as ik}{nie verdern kan}\\

\haiku{'t Was toch iets uit,;}{haar eigen leven dat zo}{wegstierf en verdween}\\

\haiku{riep Ulekens moeder, -!}{opgewonden we moen dat}{toch uek ne kier zien}\\

\haiku{en nu eerst voelde}{Uleken zo duidelijk de}{afstand en hoe wijs}\\

\haiku{Zonder een woord, met,.}{een kort gebaar sloeg ze ruw}{zijn hand van zich af}\\

\haiku{riep Broosp\`er, toen hij aan,.}{zijn hek was en ook zijn vrouw}{wenste goeden avond}\\

\haiku{Nu was 't de beurt.}{van Uleken om een plotse}{vuurkleur te krijgen}\\

\haiku{Als gij ooit iets te,.}{beleggen hebt vergeet niet}{naar hem toe te gaan}\\

\haiku{antwoordde Uleken, - ' '!}{zeer beslistkn goa in}{gien slechte huizen}\\

\haiku{Stanus bleef, zoals ',.}{t betaamde bij zijn vrouw}{de thuiswacht houden}\\

\haiku{het ingepropte.}{wicht dan in haar armen op}{en neer deed dansen}\\

\haiku{maar Uleken, evenmin, {\textquoteleft}{\textquoteright};}{als Natsen scheen haarsoorte}{te kunnen vinden}\\

\haiku{Uleken wist nu wel,;}{heel vast en zeker dat zij}{niet meer trouwen zou}\\

\haiku{was er toch ook een:}{andere troost in Ulekens}{leven gekomen}\\

\haiku{Eerst was het meer uit,.}{gevoel van plicht dat zij er}{zich mee bemoeide}\\

\haiku{Fielemiene en.}{Stanus bekeken elkaar}{en hun ogen straalden}\\

\haiku{'t Was donker in.}{de slaapkamer en nog steeds}{kwam er geen antwoord}\\

\haiku{- Als huefd van de ''.}{gemiente est zijn plicht}{ulder t helpen}\\

\haiku{De kleine zakte.}{op zijn stoel en sloeg met zijn}{vork op de tafel}\\

\haiku{- We zijn op wig noar,.}{de sampitter of hij ons}{wilt helpen zoeken}\\

\haiku{vroeg ze enkel, op,.}{nederige toon als om}{hem te vermurwen}\\

\haiku{Gedurende een.}{paar seconden waren angst}{en smart vergeten}\\

\haiku{het was alsof ze.}{daar zaten te bidden om}{een onzichtbaar lijk}\\

\haiku{Haar stem klonk schor en.}{diep in de stilte bij het}{vlammengeknetter}\\

\haiku{Er werd buiten aan.}{de deur geklopt en meteen}{waren zij binnen}\\

\haiku{Zij aarzelde om.}{te gaan zitten en bleef maar}{liever wachtend staan}\\

\haiku{- Ach joa, natuurlijk,;}{ge zij gij veel te jonk om}{mevreiwe te zijn}\\

\haiku{Zij liep in zichzelf '.}{te glimlachen en dacht weer}{aant verleden}\\

\haiku{Uleken was trouwens;}{bereid al de onkosten}{op zich te nemen}\\

\haiku{Als een vechthond vloog ' '}{Allewies Cesar opt}{lijf ent ogenblik}\\

\haiku{Maar ondertussen.}{was er iets nieuws in Ulekens}{leven gekomen}\\

\haiku{- 't Es het ginter ',!}{al verre lijk hier int}{kluester tante}\\

\haiku{maar een hoongelach.}{weerklonk en de kroegdeur werd}{hard dichtgeslagen}\\

\haiku{- Meschien zal onze,!?}{lieven Hier ons wijzen wat}{da we moeten doen}\\

\haiku{Van alles, wat zij;}{daar en op haar eigen dorp}{destijds geleerd had}\\

\haiku{vroeg ze, zo gewoon,.}{mogelijk doend zodra het}{meisje binnen was}\\

\haiku{Peis ne kier wat dat!}{de meinschen doarvan zoen}{keunen zeggen}\\

\haiku{Hij noemde haar ook,.}{al tante of hij reeds van}{de familie was}\\

\haiku{maar er werkte een,,...}{geheime fatale kracht}{over haar die haar dwong}\\

\haiku{dat hij niet gezeid ' ' '!}{n hee datk hem int}{Frans moe aanspreken}\\

\haiku{Telkens had hij 't,,.}{daarover de laatste tijd als}{hij in verlof kwam}\\

\haiku{Alles was eensklaps.}{zo heel anders en zoveel}{gemakkelijker}\\

\haiku{Met Lauwereyns had '.}{Allewies kennis gemaakt}{int regiment}\\

\haiku{Ook meneer Santiel,:}{hijgde hoorbaar maar dat was}{van de inspanning}\\

\haiku{Het nieuwe leven,.}{deinde om hen heen zonder}{hen aan te roeren}\\

\haiku{Doch hij had bijna;}{meer moeite met de vrouw dan}{met de woesteling}\\

\haiku{het Kasteel {\textquoteleft}Uputin{\textquoteright},;}{van de oude baron van}{Houren d'Uputin}\\

\haiku{Lowiezeken zou.}{ondankbaar zijn geweest over}{wat ook te klagen}\\

\haiku{Zij knikte van ja, '.}{t schaamterood gezicht achter}{de beide handen}\\

\haiku{Zij zaten daar als,.}{stugge rechters die een streng}{vonnis gaan vellen}\\

\haiku{'t 'n Zoe toch moar'.}{ou plicht zijn os g heur in}{heur ier herstelde}\\

\haiku{- Al kwam hij om mee,!}{heur te treiwen n\'og schupte}{ik hem aan de deur}\\

\haiku{Moeder Dort\'e zei,;}{dat ze toch nog liever een}{jongen had gezien}\\

\haiku{Als ze bij haar was,.}{keek ze voortdurend naar haar}{mooie ringenhanden}\\

\haiku{fluisterde, met een,.}{wantrouwige blik naar de}{deur neef Ga\"etan}\\

\haiku{De auto reed weg,.}{onder een verward tumult}{van de menigte}\\

\haiku{Meneer Aamid\'e stond op.}{zijn bordes te trillen en}{bijna te schreien}\\

\haiku{- Veronderstel, zei, -!}{hij dat de vijand ons hier}{t\'och komt verrassen}\\

\haiku{De man in de boom,.}{liet zich neerglijden vlug als}{een aap langs de stam}\\

\haiku{Wat hadden zij toch?}{misdaan om zulk een wrede}{ramp te verdienen}\\

\haiku{Leonard stapte.}{de akker op en ging met}{de mensen praten}\\

\haiku{maar, wat hen betrof, ',.}{t had slechter oneindig}{veel slechter gekund}\\

\haiku{De dokter slikte.}{haastig een kort lachje in}{en stond meteen op}\\

\haiku{maar wie dat deed werd.}{veracht en voor later met}{wraakneming bedreigd}\\

\haiku{Zulmatje was het.}{vertroetelde kind van de}{grenswacht geworden}\\

\haiku{herhaalde Tieste.}{met bevende vingers het}{bankje verstoppend}\\

\haiku{Hij reed er zo mee,,.}{door voor de aardigheid om}{eens te proberen}\\

\haiku{Zo ging dat. En zo.}{vervloog nog eens de droeve}{oorlogswinter}\\

\haiku{Het bleek, dat hij een.}{oude schoolmakker en vriend}{was van de dokter}\\

\haiku{klonk de rustige.}{stem van Leonard in de}{benauwde stilte}\\

\haiku{Hij trok zijn kar, die,, ' {\textquoteleft}{\textquoteright}.}{leeg was opzij bleef even v\'o\'or}{tKasteelken staan}\\

\haiku{Lowiezeken had:}{een wild gebaar van schrik en}{Dort\'e riep woedend}\\

\haiku{Tiestes antwoord ging.}{in het gewoel en in het}{gedrang verloren}\\

\haiku{Trillend lei Zulma.}{haar hand in de zijne en}{drukte die knellend}\\

\haiku{Nors nam zij het boek,.}{op en bladerde wat als}{in zwijgend protest}\\

\haiku{- Ze wilde zelve, ' '.}{noar ou toe komen moark}{h\`et heur belet}\\

\haiku{- Lowiezeken, zei, -.}{Jeannette ge zit doar}{nog mee ou schoens aan}\\

\haiku{Moeder Dort\'e kwam '.}{aanbellen ent ogenblik}{daarna ook Guustje}\\

\haiku{Nu, moeder, moet ik '.}{u het een ent ander}{van mij vertellen}\\

\haiku{XII - Lowieze, wil '?}{ik heur ne kier schrijven os}{ge gij nien wilt}\\

\haiku{Was de wereld dan,}{z\'o veranderd dat het niet}{meer als een schande}\\

\haiku{Maar Zulma weet gij:}{wat voor een gedacht dat ik}{altijd gehad heb}\\

\haiku{Maar de deftige.}{chauffeur reageerde in}{het geheel niet meer}\\

\haiku{zei hij op zijn beurt,.}{met uitgestrekte hand. Even}{aarzelde Zulma}\\

\haiku{vermoedelijk was.}{Leonard nog ergens in}{het dorp gebleven}\\

\haiku{Eerder last kon hij.}{van zijn jongere broer en}{zuster ginds krijgen}\\

\haiku{Hij scheen zijn zuster,.}{zwijgend om een beslissend}{antwoord te vragen}\\

\haiku{hij was drijfnat en, ':}{beslijkt maart scheen hem niets}{te kunnen deren}\\

\haiku{antwoordde Peetsen,.}{zonder aarzelen met de}{diepste overtuiging}\\

\haiku{Moeder schreide niet,.}{maar zij zag er zo bleekjes}{en zo mager uit}\\

\haiku{IV Somber strekte.}{nu de oceaan zijn woeste}{eindeloosheid uit}\\

\haiku{Het bijna zwarte.}{water deed denken aan een}{woelend berglandschap}\\

\haiku{men had geen kracht meer,.}{om nog iets te doen om nog}{aan iets te denken}\\

\haiku{Treinen reden op,,;}{en af over een soort dijk vrij}{hoog boven de straat}\\

\haiku{Ivan keek zijn broeder.}{aan of hij door hem voor de}{gek gehouden werd}\\

\haiku{- 'k Kenne doar ne.}{goeje restaurant woar da}{we goan dineren}\\

\haiku{Alvorens binnen:}{te treden nam Oculi hen}{nog eens kritisch op}\\

\haiku{zij klommen, langs een,.}{vrij sombere trap naar de}{derde verdieping}\\

\haiku{Alles verzwond tot.}{ijlheid in haar geest en zij}{sliep zoetjes in}\\

\haiku{Ze kregen meer dan.}{genoeg om er een volle}{dag van te leven}\\

\haiku{pochte Oculi en.}{hij wenkte de neger om}{af te rekenen}\\

\haiku{andere kofiers.}{kopen en die lelijke}{kisten verbranden}\\

\haiku{Clotilde vond het.}{akelig om aan te zien en}{aan te  horen}\\

\haiku{Zij voelde zich ook,.}{veiliger dat zij daar met}{hun twee\"en lagen}\\

\haiku{Ge verstoat toch wel '.}{da ze da niet buem veur buem}{n kosten uitdoen}\\

\haiku{De trein reed over een,.}{lange metalen brug die}{donderend dreunde}\\

\haiku{- d'r 'n es hier al.}{nie veel gelegenheid om}{Vloams te klappen}\\

\haiku{Hij vertelde van,:}{Ierland zoals hij het in}{zijn jeugd gekend had}\\

\haiku{En ook Oculi en,;}{Dzjeurens hoewel zij zo}{vervelend spraken}\\

\haiku{Zij sloot haar ogen en.}{vouwde de handen samen}{in een vroom gebed}\\

\haiku{Oculi, die in zijn ',:}{bord aant slurpen was keek}{met verbazing op}\\

\haiku{- Hoe vinde dat! vroeg,,.}{hem Maria levendig met}{schitterende ogen}\\

\haiku{Franklin keerde zich half.}{terzijde om zijn sigaar}{weer aan te steken}\\

\haiku{Een schei-dreunende.}{muziek barstte eensklaps los}{en het scherm ging op}\\

\haiku{- Ien van die keirels!}{die vandoage rijke}{zijn en morgen oarm}\\

\haiku{en onmiddellijk.}{plooide Mr. Watsons gezicht zich}{tot stemmige ernst}\\

\haiku{Een jonge dame,.}{mende heel alleen in het}{rijtuig gezeten}\\

\haiku{hij wist niet wat te.}{doen met al die lege uren}{van de lange dag}\\

\haiku{Er was daar wel geen,}{reden voor maar dat verschil}{van tijd stemde hen}\\

\haiku{het trilde in zijn.}{lichaam niet wanneer hij er}{zijn voet op drukte}\\

\haiku{Ik heb twee schoone.}{duivenjongens gekregen}{van Peetse Speybroeck}\\

\haiku{Hij had vijf dagen,.}{vrij die zij in Chicago}{zouden doorbrengen}\\

\haiku{Maria schaamde zich.}{en dadelijk straalde weer}{haar gezonde lach}\\

\haiku{Ge doe mij denken '.}{aann kennisse van ons}{die Dzjeurens hiet}\\

\haiku{Zij wilde wel heel.}{graag met haar mooie wagen in}{de stad eens geuren}\\

\haiku{Nu reeds was hij van!}{de derde wereld in de}{tweede overgestapt}\\

\haiku{en de weg verliet.}{de stroom en slingerde de}{steile hoogte op}\\

\haiku{De weg slingerde,,,.}{kronkelde met bochten die}{steeds scherper werden}\\

\haiku{Hij had de vrouw van!}{zijn baas genomen en hij}{zou haar nog nemen}\\

\haiku{- Kom binnen, jongen,,.}{kom binnen en ging hem zelf}{voor in het huisje}\\

\haiku{en haar ogen peilden,.}{hem zo dat hij met een kleur}{de blik afwendde}\\

\haiku{hun leven scheen er,.}{op de maat van de huizen}{die zij bewoonden}\\

\haiku{XIX Mr. Keane.}{kwam zijn vrouw in New York aan}{de boot afhalen}\\

\haiku{Oculi glunderde ':}{ent eerste wat hij zijn}{broeder toeriep was}\\

\haiku{De waarde van de!}{grond is alweer met veertig}{percent gestegen}\\

\haiku{en wij... Dear, ik moet!}{er onmiddellijk een nog}{mooiere hebben}\\

\haiku{hedde da nou nog?}{oeit van ou leven g'huerd}{van luelijkheid}\\

\haiku{Hij vond de schikking.}{nogal billijk en hij zei}{het aan zijn zuster}\\

\haiku{Op een gegeven.}{ogenblik moesten de tramcars voor}{het gedrang stoppen}\\

\haiku{- Ge keunt er nou mee! -?...}{treiwen en ou leven lang}{gelukkig zijn Nou}\\

\haiku{Zijn armen deden;}{pijn van verlangen om haar}{te omstrengelen}\\

\haiku{Gladys stond op en.}{reikte Moeder en Peetsen}{de hand tot afscheid}\\

\haiku{En eensklaps, zonder,.}{schijnbare reden barstte}{zij in tranen uit}\\

\haiku{doch haar huwelijk;}{met Mr. Keane had dat}{alles verbroken}\\

\haiku{en na een poos, als,,:}{in een droom hoorde hij h\'a\'ar}{stem die antwoordde}\\

\haiku{Geloof mij, Ivan, treur,.}{niet om mij ge zoudt met mij}{niet gelukkig zijn}\\

\haiku{Zo kwamen er zes.}{eentonige gelijke}{dagen in de week}\\

\haiku{allen voelden in.}{peinzend stilzwijgen een groot}{geheim om zich heen}\\

\haiku{Well het eenige dat}{ik hier somtijds beter vind}{dan in Amerika}\\

\haiku{Ik  denk dat gij.}{van uwe kant ook al meer dan}{genoefd hebt van oud}\\

\haiku{ik zal al doen wat.}{ik kan om op denzelfden}{boot te zitten}\\

\haiku{Lisatjes moeder was,.}{sluw Nonkel Justien was slim}{en flink voortvarend}\\

\haiku{Hij gaf een duwtje,.}{in de zij van Lisatje die}{ook eventjes lachte}\\

\haiku{Neen, hij verkocht niets,,.}{nog geen vierkante meter}{zolang hij leefde}\\

\haiku{Toen vroeg hij, zijn olijk:}{glinsterende ogen lachend}{op Ivan gevestigd}\\

\haiku{Hij aarzelde, hij,.}{twijfelde hij voelde zich}{zo diep ellendig}\\

\haiku{{\textquoteright} Geen liefelijke,!}{zachtheid geen nederige}{onderworpenheid}\\

\haiku{- Jaja, zei Smith, - maar,.}{gij hebt Luis niet gekend de}{Mexicaanse koetsier}\\

\haiku{Er broeide reeds lang.}{iets in haar en nu zou zij}{haar hart eens luchten}\\

\haiku{zueveel meugelijk.}{money moaken en de}{reste komt vanzelf}\\

\haiku{Ivan voelde iets als.}{een stille vijandigheid}{groeien om zich heen}\\

\haiku{Dat is toch iets, niet,!}{waar voor eenen jongen van zoo}{goede familie}\\

\haiku{Ivan schelde haastig.}{om een steward en leidde}{Peetsen in de kooi}\\

\haiku{Ivan trachtte hem aan ',.}{t verstand te brengen hoe}{dat in elkaar zat}\\

\haiku{- Dag, Maria! groette,,.}{hij heel gewoon alsof hij}{haar dagelijks zag}\\

\haiku{- 'k 'n Gelueve '!}{niet dak d\'a zoe keunen}{geweune worden}\\

\haiku{'t Leek alles nu,.}{zo vreemd zo ver wat eenmaal}{zo diep geleefd had}\\

\haiku{Toen kwam er een brief,,,:}{een korte een formele}{van Clotilde}\\

\haiku{Wel, lieve moeder}{en broeders en familie}{hiermede eindig}\\

\haiku{Naar Grover keek zij,.}{verlegen alsof hij er}{niet bij behoorde}\\

\haiku{zei Franklin glimlachend -!}{en drukte haar de hand. Ge}{zij welgekomen}\\

\haiku{Maria had voor hem.}{een diep bord gevraagd en twee}{rauwe eieren}\\

\haiku{- Wat 'n verschil met!}{Blue Springs om nog niet eens van}{New York te spreken}\\

\haiku{- U hebt b.v. geen idee,,.}{wat die kleine kerel daar}{Grover al niet weet}\\

\haiku{n Zoe nie keune,!}{zeggen wat dat er van de}{die geworden es}\\

\haiku{- Van 's gelijke,,' '.}{jongen en ziet da g ou}{geld nien verliest}\\

\haiku{en zij, Maria, kreeg.}{last van haar ogen en moest een}{sterke bril dragen}\\

\haiku{Aldus kwam Grover!}{naar het geboorteland van}{zijn ouders terug}\\

\haiku{Maria schreef, dat ze,;}{zo heel h\'e\'el weinig nieuws meer}{te vertellen had}\\

\haiku{maar 't feit dat hij {\textquoteleft}{\textquoteright}.}{nietneen zei mocht eigenlijk}{ja betekenen}\\

\haiku{Velde trok een vreemd.}{gezicht en begon achter}{zijn oor te krabben}\\

\haiku{- En ziet da ze nie '!}{wign goat achter d'ieste}{moand proboassie}\\

\haiku{En de nachtegaal,!}{zong maar aanhoudend om er}{dol van te worden}\\

\haiku{Vastberaden trad;}{hij haar tegemoet in het}{smalle gangetje}\\

\haiku{- Joa moar 'k 'n laa\"e, ' '! '!}{nietkn laa\"e niett Es}{huel serieus}\\

\haiku{Da moeder da moest,!}{weten ze zoe heur in heur}{kist ornme kieren}\\

\haiku{Zij gilden allen,.}{om het hardst alsof zij op}{de akker stonden}\\

\haiku{- In mijnen tijd was,!}{wirken troef ik spreek ik hier}{van nudig wirke}\\

\haiku{bij Marina was;}{het tot het vorige jaar}{ook aldus geweest}\\

\haiku{- 't Es koart veur den,,.}{achten hernam hij als een}{klacht van ongeduld}\\

\haiku{- Moeder, doe de die\"e.}{doar ne kier ophouwe mee}{heur zoagerije}\\

\haiku{ze weunt zij moar op, '.}{mijn koamer zen es zij van}{mijn familie niet}\\

\haiku{zei hij, de arme.}{man met bezorgde liefde}{in de ogen kijkend}\\

\haiku{verpoest,   charbons;}{et cokes ~ Links was er}{een Vlaamse Kamer}\\

\haiku{- Et quand aurons-nous?}{aussi une fois le plaisir}{de vous voir chez nous}\\

\haiku{De laatste stralen.}{van de lentezon schoten}{hun goud om hen heen}\\

\haiku{- Ha joa moar, 'k 'n,.}{ben ik doar nie op geklied}{menier Florimond}\\

\haiku{Hij laat haar los, doet,}{haar nog eens vast beloven}{dat ze komen zal}\\

\haiku{de hoge kruinen.}{van het stil-verlaten}{Park werden inktzwart}\\

\haiku{Prenez du coup au, '.}{moins deux meill da ge ern}{tijdje goe mee zijt}\\

\haiku{blomme woater te, '.}{geve aan de venster van}{n twiede stoassie}\\

\haiku{en wat dat er doar ' '!}{bui te gebeurtn trekke}{k ik mij nie aan}\\

\haiku{Plotseling keerde.}{de schaduw zich naar hem om}{en bleef palstil staan}\\

\haiku{- Ha, 'k 'n doe ekik,......!}{nie mier als Julien en nog}{vele andere}\\

\haiku{- Haw\`el joa, bel moar, ' ',.}{kn k\'an nie mier hijgde}{madame Verpoest}\\

\haiku{Sloapt er nen nacht,.}{op en zeg mij morge da}{ge wijs zult worde}\\

\haiku{- Wilde gij ne kier,',!}{zwijge joa g en ziere}{noar uw bedde goan}\\

\haiku{t Zijn leugens? - 'k ', '!}{Weinste datt woar woare}{datt leugens zijn}\\

\haiku{Allen nu keken,.}{hem strak aan merkten dat hij}{aangeschoten was}\\

\haiku{Ouais, ouais, c'est,.}{parce qu'il deviendrait trop tard}{savez-vous}\\

\haiku{Niemand kende haar,;}{niemand wist wie ze was noch}{waar ze vandaan kwam}\\

\haiku{Dadelijk vormde,,.}{zich onder haar voeten een}{sijpelend plasje}\\

\haiku{hij herleefde, vrij;}{van de zo lang en zo hard}{knellende banden}\\

\haiku{En hij trok er ruw,,.}{van onder als iedere}{avond naar zijn lief toe}\\

\haiku{- Hij wirkt bij Carels, '!}{op den atelier hij wintn}{schune daghure}\\

\haiku{Op het busseltje:}{van Euzekes kind stond er}{in drukletters}\\

\haiku{En in de loop van ',:}{t gesprek vroeg Euzeke}{schuchter-aarzelend}\\

\haiku{Deze, reeds gewend,.}{aan onbezorgde kost stond}{in verlegenheid}\\

\haiku{- Nuunt, nuunt mee hem op,,.}{stroate zei ze als wrang}{besluit in haarzelf}\\

\haiku{- Toe, toe, da zoe te, ' ';}{verre lien en uukkn}{zoek ekik da niede}\\

\haiku{Les fun\'erailles,,.}{auront lieu schreef Florimond}{steeds hardop lezend}\\

\haiku{met het papier in,.}{de hand de geest op zoek reeds}{naar een uitvluchtsel}\\

\haiku{Niet begrijpend, waar,:}{hij heen wilde liet ze zich}{vangen in de strik}\\

\haiku{XV Middelerwijl...}{kwam Florimond geregeld}{weer bij Euzeke}\\

\haiku{een dame, die met;}{een volgeladen blikken}{korf van de markt kwam}\\

\haiku{- Nous d\'esirons voir,.}{votre maison qui est \`a}{louer madame}\\

\haiku{Je ne regarde.}{pas \`a un franc pour avoir des}{gens pas crapuleux}\\

\haiku{Et puis, sales qu'ils,,!}{\'etaient n\'egligents qu'ils}{\'etaient les autres}\\

\haiku{- Al voir, madame,,,.}{al voir mossieur is het}{vriendelijk antwoord}\\

\haiku{J'ai presque tout compris,,.}{en devinant le reste}{naturellement}\\

\haiku{- Joa, joa, ge zul ne,,.}{kier hure wa da ze zegt}{meende Marina}\\

\haiku{Madame M\'edard,.}{stond te wachten in fluweel}{en bont ondanks april}\\

\haiku{sprak hij bewogen,.}{tot zijn moeder toen de deur}{achter hen dicht was}\\

\haiku{Zij hadden haar zo!...}{kortstondig en zo innig}{gelukkig gemaakt}\\

\haiku{- C'est vous qui avez l'air,.}{bien portante hernam ze}{tegen Aurore}\\

\haiku{Het was madame,:}{Verpoest met een opgewekt}{gelaat als altijd}\\

\haiku{antwoordde hij met.}{een komische buiging voor}{madame Verpoest}\\

\haiku{En Paulke volgde.}{weer het gevaarte als een}{schuitje op sleeptouw}\\

\haiku{propaganda met.}{de mond en met de pen was}{gemaakt geworden}\\

\haiku{- Papa zorgdege,.}{altijd veur de cro\^ute zei}{madame Verpoest}\\

\subsection{Uit: Verzameld werk. Deel 4}

\haiku{Buysse hanteert dit {\textquoteleft}{\textquoteright}:}{ik-perspectief van}{deconte vrij vaak}\\

\haiku{Schrik, uit Elsevier's,,,-;}{Ge{\"\i}llustreerd Maandschrift d.}{IV 1892 p. 488495}\\

\haiku{De gelukkige,,,,-;}{tijding uit Eigen Haard 1895}{nr. 27 p. 417421}\\

\haiku{Zonder de minste,.}{tegenstand te bieden liet}{hij zich meeleiden}\\

\haiku{een gesmoorde zucht}{ontglipte zijn beklemde}{borst en zijn armen}\\

\haiku{Zij was ook lang en,;}{slank van gestalte zoals}{hij doch iets kleiner}\\

\haiku{- zij zijn zeer braaf en ';}{komen hiers winters schier}{elke avond kaarten}\\

\haiku{Hij zal zich na de ';}{vespers aant kaarten wat}{vergeten hebben}\\

\haiku{'t Was zonderling,.}{hij scheen verheugd en ook toch}{enigszins verlegen}\\

\haiku{Maar zij ontsnapte:}{hem opnieuw en ging midden}{in de keuken staan}\\

\haiku{Elk droeg zijn lijn en '}{had het een oft ander}{pakje in de hand.}\\

\haiku{Ik zag mijn vangst met.}{onbeschrijflijke hoogmoed}{en ontroering aan}\\

\haiku{- Hoe! kreet ik verbaasd.}{rechtstaand en met geweld mijn}{uurwerk uittrekkend}\\

\haiku{ik verontwaardigd,.}{tot de garde die achter}{mij de deur toesloeg}\\

\haiku{Die betrouwt zich op;}{de vast bepaalde tijd en}{nut op zijn gemak}\\

\haiku{{\textquoteright} 't Is te zeggen,,,.}{huis verblijfplaats  toevlucht}{voor de reiziger}\\

\haiku{Ik werd eensklaps gans, ',.}{wakker ik richttet hoofd}{zette mij overeind}\\

\haiku{Ik keerde mij tot,.}{het jong meisje om dat mijn}{arm verlaten had}\\

\haiku{Als jachthonden, met,;}{het aangezicht tegen de}{grond volgden zij dit}\\

\haiku{Het vrouwvolk had zich,,;}{in de kamer achter de}{kleerkast verscholen}\\

\haiku{zijn handen, ontvleesd;}{en knokkelig als klauwen}{waren zwart en vuil}\\

\haiku{- Zie, mensen, dat weet,, -.}{ik niet sprak hij en dat zijn}{ook niet mijn zaken}\\

\haiku{riep hij met luider,,.}{stem tot mijn buur de eerste}{getuige knipogend}\\

\haiku{De notaris, zeer,;}{ernstig was begonnen zijn}{pen af te vegen}\\

\haiku{Eenieder sprak die, '.}{woorden na sloeg een kruis en}{richttet hoofd op}\\

\haiku{Langzaam als het dof,.}{getrappel van een kudde}{begon de aftocht}\\

\haiku{Doch de notaris.}{had reeds zijn zwaveldoosje}{te voorschijn gehaald}\\

\haiku{Gampelaarken 't,.}{Was in december bij het}{vallen van de avond}\\

\haiku{In diep stilzwijgen.}{had men het einde van dit}{verhaal aangehoord}\\

\haiku{juist stond ik voor de.}{woning van Beer Roetjes32}{de vogelman}\\

\haiku{Een reusachtige,,;}{kom nog halfvol pap bevond}{zich in het midden}\\

\haiku{Een diep misprijzen,.}{blonk in zijn blik een spotlach}{kwam op zijn lippen}\\

\haiku{- Hij is er bij, zeg,!}{ik u en wij moeten en}{zullen ze vinden}\\

\haiku{Wij trokken ons een.}{weinig achteruit nevens}{het tweede schuitje}\\

\haiku{- Komaan, zeg ik, laat.}{ons een wandelingetje}{maken op het dek}\\

\haiku{'t gerucht liep rond.}{dat men misschien dezelfde}{avond zou aanlanden}\\

\haiku{Weldra bevonden.}{wij ons als te midden een}{oceaan van lichten}\\

\haiku{Op een gegeven.}{ogenblik ontstond er daar een}{soort van worsteling}\\

\haiku{Ik had gedaan met,.}{eten rees van mijn zitplaats op}{en ging eens zien}\\

\haiku{- Zo niet, zend ik u.}{voor het overige van de}{reis in steerage37}\\

\haiku{Een schielijke, schier;}{plechtige stilte schorste}{de gesprekken op}\\

\haiku{Dringend wendde ik {\textquoteleft}{\textquoteright}.}{een uiterste poging bij}{decollector aan}\\

\haiku{Hun stappen klonken {\textquoteleft}{\textquoteright};}{ongelijk en hol over de}{planken van depier}\\

\haiku{Zijn makkers volgden,,.}{hem schielijk bedaard zijn aap}{sprong van zijn schouders}\\

\haiku{Ditmaal werd zijn borst.}{niet te zwak gevonden voor}{de soldatendienst}\\

\haiku{Werktuiglijk staat Miel.}{op en neemt ook zijn pakje}{dat op een stoel ligt}\\

\haiku{En moeder, doodsbleek,,,:}{met vergrote angstige}{ogen murmelt ohaar beurt}\\

\haiku{Het is zo vet en,,?}{wel gevoed als zijn meester}{zelf nietwaar Fido}\\

\haiku{Mul begreep, dat hij.}{in zulk een staat v\'o\'or zijn neef}{niet kon verschijnen}\\

\haiku{hij trok er binnen,.}{om wat uit te rusten en}{zich te verfrissen}\\

\haiku{Een dof gefluister,.}{greep plaats zij trokken enige}{stappen terzijde}\\

\haiku{Toen zag Perseijn in,:}{de maneschijn een der twee}{mannen op de rug}\\

\haiku{En hij legde Mul,,.}{die pots begon te beven}{de akte voor ogen}\\

\haiku{hij sprong naar die stem,;}{hij verworgde ze in de}{keel van de woestaard}\\

\haiku{Gij wilt ons daar iets!}{op de mouw spelden dat toch}{wat al te kras is}\\

\haiku{- 't Is tot mevrouw,?}{Montfort dat ik de eer heb}{het woord te richten}\\

\haiku{Maar zij begreep de '.}{betekenis van die blik}{en buktet hoofd}\\

\haiku{En op een avond was:}{het als een wind van schrik die}{woei over de hoeve}\\

\haiku{- Vader Slock, riep hij, -,,.}{geloof wel dat ik ernstig}{heel heel ernstig spreek}\\

\haiku{het is mijn plicht hun!}{het geluk te geven dat}{in mijn bereik staat}\\

\haiku{Zij rees v\'o\'or haar op,,.}{ellendig en wenend met}{haar kind op de arm}\\

\haiku{En zij slaakte een,;}{kreet van verrukking toen zij}{haar gans gekleed zag}\\

\haiku{en schertsend klopte '.}{de plaagzieke dokter nog}{eens opt buikje}\\

\haiku{Na het nagerecht.}{stonden de dames op en}{verlieten de zaal}\\

\haiku{- Adela{\"\i}de, nou;}{weet ik wat dat ze in het}{dorp tegen ons h\^en}\\

\haiku{En, spijtig met het,,:}{hoofd schuddend nam hij een vers}{blad papier zeggend}\\

\haiku{Men had voor hem geen;}{toornige blikken of geen}{hoongelach meer over}\\

\haiku{De secretaris,,.}{hevig ontsteld was opnieuw}{zeer bleek geworden}\\

\haiku{dat zij begreep wat.}{in hem omging en dat zij}{er gestreeld door was}\\

\haiku{De oude moeder.}{slaat wanhopig haar ogen en}{handen ten hemel}\\

\haiku{- En deze, hier, het,?}{kind van de meesters wat moet}{er daarvan worden}\\

\haiku{Opnieuw neemt zij de, '...}{vlucht naart diepste van de}{dennenbossen}\\

\haiku{Het kindje zwijgt en,....}{zuigt de ogen van de jonge}{moeder vallen toe}\\

\haiku{schreeuwde hij, verbaasd,.}{en woedend zich in volle}{lengte oprichtend}\\

\haiku{je sau...aurai bien....}{vous prot\'eger coontre}{cette canaille}\\

\haiku{Met nadruk drong hij.}{erop aan dat de prinsen}{t\'och zouden blijven}\\

\haiku{Dit bracht de heren.}{tot het bewustzijn van hun}{deftigheid terug}\\

\haiku{- De littekens zijn, ',?}{nog zichtbaar op mijn dijen}{ist niet waar vrouw}\\

\haiku{meneer Spittael en,,}{zijn twee magere gele}{juffrouwen meester}\\

\haiku{Maai dunkt nochtans dat.}{ik niets gezeid heb da nie}{gehoord mag worde}\\

\haiku{oed, het ogenblik is,.}{gekomen sprak Massijn met}{een stokkende stem}\\

\haiku{Meester De Vreught had.}{zijn zakdoek uitgehaald en}{snoot zich luidruchtig}\\

\haiku{Doch ik kwam terstond;}{tot het bewustzijn mijner}{deftigheid terug}\\

\haiku{Toch had hij zich, wat,;}{dat betreft de laatste tijd}{in acht genomen}\\

\haiku{Zonder de blaker,.}{los te laten stak hij de}{sleutel in het gat}\\

\haiku{Er was heel weinig,.}{kans dat hij langs die weg kon}{geholpen worden}\\

\haiku{Hij vond niets dan zijn,.}{sigarenkoker met nog}{drie havana's erin}\\

\haiku{D\'a\'arover alleen moest hij,.}{steeds blijven heersen steeds vrij}{blijven beschikken}\\

\haiku{Deze was een zeer,.}{vermogend man vrij ruw en}{stroef van uiterlijk}\\

\haiku{En toen haar man, die,:}{dat wel \'al te overdreven}{vond haar uitlachte}\\

\haiku{D\'a\'ar staat hij, d\'a\'ar brengt!...}{hij hun de grote liefde}{en het groot geluk}\\

\haiku{De lege flessen;}{worden vanuit de tent in}{het water gegooid}\\

\haiku{Hij stond op toen hij,:}{ons zag en zodra hij de}{gordel ontwaarde}\\

\haiku{- Ik zocht hem overal,,.}{ik kon niet denken waar hij}{toch gebleven was}\\

\haiku{Na een week liet ik ' '.}{hems ochtends ens avonds}{een paar uurtjes los}\\

\haiku{hij profiteert als.}{zij van de rijkdom en de}{overvloed der Natuur}\\

\haiku{men zou een weinig,...}{gaan bedelen dat was ook}{vroeger reeds gebeurd}\\

\haiku{Hij had de kracht niet,.}{meer om op te staan om zich}{nog te bewegen}\\

\haiku{hij woonde daar, met,;}{de geest afschuwelijke}{taferelen bij}\\

\haiku{- Bah joa 't, da goa,:}{nogal moar gisteren was}{de vangst toch beter}\\

\haiku{Ik zal  hem twee.}{vijffrankstukken geven en}{wij zullen eens zien}\\

\haiku{En nu weergalmt, door,,:}{het gejouw heen een kreet van}{haat alom herhaald}\\

\haiku{{\textquoteright} 't Is inderdaad.}{daarheen dat zich de stoet met}{rasse schreden wendt}\\

\haiku{Dan stijgt me als een.}{walg van verachting en van}{afkeer in de keel}\\

\haiku{De kikkers Om twee,,...}{uur na zijn dutje was de}{barbier vertrokken}\\

\haiku{Zij houden van wat,.}{hun groot schijnt van al wat ruim}{en overtollig is}\\

\haiku{Even vielen zijn ogen,.}{heel en al dicht en zijn hart}{hield op met kloppen}\\

\haiku{De groffe houten;}{lepel beefde sterker in}{het klein bruin handje}\\

\haiku{Hij zou nu naar de,;}{hoeve weer terugkeren}{gelijk zijn broeders}\\

\haiku{De knapen kropen;}{uit het water en kleedden}{zich haastig weer aan}\\

\haiku{In geen enkel was,.}{er nog een eendje alle}{waren bedorven}\\

\haiku{G'en peist toch zeker ' '?}{niet uitnen hoan enn}{oande te kwieken}\\

\haiku{Zij wierpen verse.}{kruimeltjes brood en verse}{brokjes aardappel}\\

\haiku{Toen deed de wachter:}{zich geweld aan om hem toch}{te kunnen zeggen}\\

\haiku{Elke morgen, zijn,}{wagen geladen tot aan}{het dekzeil verliet}\\

\haiku{dat zelfde ogenblik.}{verliet Merci\'e het huisje}{van de twee oudjes}\\

\haiku{In het geschokte,.}{brein van Merci\'e daagde een}{licht een gedachte}\\

\haiku{Al mier of 'n joar?}{dus woaren ze van gedacht}{mij wig te zenden}\\

\haiku{het waren zijn schrik:}{en zijn gruwel zelf die hem}{werkelijk stuwden}\\

\haiku{Ik volgde hem, kwam,.}{hijgend op een portaal waar}{hij mij reeds wachtte}\\

\haiku{In elk geval was '.}{t toch de moeite weird de}{kanse te woagen}\\

\haiku{De krieken, minder,,...!}{overvloedig raakten er nog}{door maar de pruimen}\\

\haiku{Dukske bloedde en, ':}{huilde vervaarlijk maart}{was nog niet gedaan}\\

\haiku{hoe kwam het toch, dat,!}{die steeds los liep terwijl hij}{steeds gebonden lag}\\

\haiku{Het ene deel viel op,;}{de grond met het gerinkel}{van een sleutelbos}\\

\haiku{Hij stapte rechtstreeks.}{binnen en dronk twee borrels}{aan de schenktafel}\\

\haiku{hij trok het huis van,,.}{Rosten Tjeef de verklikker}{de vijand voorbij}\\

\haiku{Die toestand in het.}{huisgezin kon echter zo}{niet blijven duren}\\

\haiku{Thans stond zij in 't.}{smal gangetje tussen de}{twee eerste bedden}\\

\haiku{Neen, zij had niet met;}{voorbedachten rade een}{kindermoord beraamd}\\

\haiku{den bueze geest,.}{van den duvel die altijd}{op de meinschen loert}\\

\haiku{je hield je stevig ':}{met de hand aant lijstwerk}{van de deur en zei}\\

\haiku{Hoe vreselijk als!}{iemand hem daar bij klaren}{dag zag aanschellen}\\

\haiku{En eensklaps zag hij,,.}{hem komen links uit een der}{deuren van de gang}\\

\haiku{We willen wij niets,}{weet van cur\'e of van kirk en}{doarom vallen}\\

\haiku{zijn geluk was zo,}{intens groot dat het hem haast}{natuurlijk voorkwam}\\

\haiku{De pastoor kwam op '.}{t geluid in de gang en}{naderde de deur}\\

\haiku{Alleen de pastoor,,.}{met zijn absolutie kon}{ze nog verlossen}\\

\haiku{En midden in de!}{nacht stond hij op en holde}{hij de velden in}\\

\haiku{Als een pak viel hij,;}{op zijn bed dadelijk in}{een loodzware slaap}\\

\haiku{Een groot ge eelte.}{van de dag bleef hij aldus}{stom-roerloos liggen}\\

\haiku{{\textquoteright} Natuurlijk kon hij.}{niet in gewijde aarde}{begraven worden}\\

\haiku{De lucht was zwoel, met,.}{laag-drijvende wolken}{broeiend van onweer}\\

\haiku{De vrouw ging naar het.}{achterhuis en nam er uit}{een hoek twee spaden}\\

\haiku{Hij, weer aan zijn werk,.}{scheen zich om haar niet langer}{te bekommeren}\\

\haiku{as 't nog iene, '!}{kier moest gebeuren dan es}{t veur altijd uit}\\

\haiku{Elk ogenblik kwamen;}{haar tranen in de ogen en}{hikken in de keel}\\

\haiku{maar hij staalde zich,.}{met wilskracht en antwoordde}{trots-minachtend}\\

\haiku{zij hadden geen aren,,:}{meer om te binden en met}{het wilde gejuich}\\

\haiku{En snikkend liep zij.}{haar muts opzetten en haar}{schoenen aantrekken}\\

\haiku{En ze zag dingen,...}{die ze niet meer kende die}{ze niet meer begreep}\\

\haiku{Permentier, veinzend,.}{dit niet te zien had haastig}{zijn deur gesloten}\\

\haiku{riep plotseling de,.}{president op dreigende}{toon met boze ogen}\\

\haiku{- G'n luept uek in,...?}{de bosschen niet g'n goat er}{gien heit85 roapen}\\

\haiku{Gedurende nen'.}{halve menuut zilt g hem}{op ou\"e kogel89 h\^en}\\

\haiku{hij aarzelde een,, '.}{ogenblik de ogen fonkelend}{t gehoor gespitst}\\

\haiku{de pijn als van een,.}{hagelslag die hem dwars door}{het hoofd zou boren}\\

\haiku{Wijl ze nu toch moest,.}{trouwen dan zo graag met hem}{als met een ander}\\

\haiku{Haar afgematte.}{kinderen konden weldra}{niet langer volgen}\\

\haiku{Dan valt hij in het '.}{oog en dooft ert laatste}{stipje leven uit}\\

\haiku{Hij was op 't punt,.}{er een brok van te vragen}{als een bedelaar}\\

\haiku{Hij had maar even de, {\textquoteleft}... '!}{klink op te tillen en met}{eenBrr watn weer}\\

\haiku{Dit gezicht, in plaats,.}{van hem te paaien maakte}{hem nog woedender}\\

\haiku{{\textquoteleft}Ik heb mijn moeder,,,!}{mijn oude goede brave}{moeder geslagen}\\

\haiku{Wat die barbaar, die,!}{bandiet niet gedurfd had dat}{had hij gedaan}\\

\haiku{Buiten stierven de.}{geluiden een voor een tot}{doodse stilte weg}\\

\haiku{Het koord in de hand.}{laat hij de lantaren naar}{beneden glijden}\\

\haiku{{\textquoteleft}Vreiwken, es 't '?}{mee ou\"e wille datt lijk}{uit den huize goat}\\

\haiku{Reinildeke bij,.}{haar broeder Leontientje}{bij haar schoonzuster}\\

\haiku{daarentegen scheen.}{er zachte rust en kalmte}{overal te heersen}\\

\haiku{En zij dankte de;}{hemel dat nu toch geen ramp}{meer op hen neerviel}\\

\haiku{- Ala toe, toe, g'n moet,.}{hier nie gegeneerd zijn drong}{de keukenmeid aan}\\

\haiku{Doodstil bleven zij,.}{alle drie met star op hem}{gevestigde ogen}\\

\haiku{Pover, ontsteld en,.}{bang hield zich schuil achter zijn}{kleingeruit raampje}\\

\haiku{Daar kroop hij in weg,.}{als een doodgejaagd beest in}{zijn laatste schuilplaats}\\

\haiku{Hij nam hem mee naar,.}{buiten en spande hem aan}{v\'o\'or zijn kruiwagen}\\

\haiku{Zij was nog van de,.}{oude oude tijd en was}{het steeds gebleven}\\

\haiku{En in de lange:}{winteravonden zaten zij}{zwijgend te werken}\\

\haiku{Leef nou liever wat, '.}{op ou gemakk zal ik}{veur ou wel wirken}\\

\haiku{- 'k H\`e al gezien ',,.}{datt goed es w'h\^en al da}{w'h\^en moeten sprak zij}\\

\haiku{- Joa, joa w', joa joa',,,!}{w zueveel of da g'r wilt}{en goeje zulle}\\

\haiku{Zij stemden allen:}{in die schikking toe en Ivo}{zelf voegde erbij}\\

\haiku{Iedere plak werd.}{boven op een dikke snee}{roggebrood gelegd}\\

\haiku{Wit blonk de vette '.}{plak met strepen rood erdoor}{opt zwarte brood}\\

\haiku{Hiere, ontfirmt ou over!}{de ziele van Coleta}{van den Bossche}\\

\haiku{{\textquoteleft}dewelke van de ' '?}{vier zoek nou pakken as}{k te kiezen h\^a}\\

\haiku{- Ha joa moar, kijk, da,?}{zijn dijngen die zoen keune}{gebeuren e-woar}\\

\haiku{- Dag menier Val\`ere,,.}{dag Sietje groette hij met}{brabbelende stem}\\

\haiku{Dan trok hij weer naar,.}{binnen en dicht flapten de}{metalen luikjes}\\

\haiku{Zij konden des te.}{vrijer en gezelliger}{met Sietje omgaan}\\

\haiku{Maar {\textquoteleft}la Zeun\`esse{\textquoteright} was.}{meer dan eens wreedaardig in}{haar flauwe grappen}\\

\haiku{- Ha moar 'k 'n weet, '.}{ik niet azue lijkn soorte}{van geroaktheid147}\\

\haiku{- Nie nie, nou nog niet,,.}{we zillen loater zien}{zei meneer Val\`ere}\\

\haiku{- Kijk zie, doar es den!}{toekomstigen nieuwen boas}{mee zijn schuenvoader}\\

\haiku{- Ha doar van achter,,.}{tegen de muur Philemon}{berichtte Sietje}\\

\haiku{verweet hij de smid,.}{gans bleek wordend van moeilijk}{ingehouden toorn}\\

\haiku{- Joa joa, we goan ze '!}{moar loate vliegen da}{zet hueren}\\

\haiku{- Nie nien 't, 't 'n, ' '!}{kan zeker nie zijn ent}{n z\'al nie zijn uek}\\

\haiku{Terwijl zorgde de.}{vrouw voor het huishouden en}{voor de kinderen}\\

\haiku{- Joa z', bezinne, '.}{ze stoan doar buiten op}{t hof te wachten}\\

\haiku{zei Liza naar het.}{onbeweeglijk-liggend}{konijntje wijzend}\\

\haiku{Binus vertelde,:}{een andere historie}{ook zeer wonderlijk}\\

\haiku{{\textquoteleft}Zoe we nie iest ne?}{kier moeten informeren}{of da ze nog leeft}\\

\haiku{die zij geofferd!}{had voor de genezing van}{moeders konijntjes}\\

\haiku{Felhoen ging in zijn, ',.}{zak legde vijf cent int}{schaaltje en voelde}\\

\haiku{Zie, ze kijken nog,.}{ne kier omme antwoordde}{hij geruststellend}\\

\haiku{Het was een ander,;}{type dan zijn vader dan}{zijn broers en zusters}\\

\haiku{Maar de zondag hield,.}{hij voor zich over wilde hij}{volkomen vrij zijn}\\

\haiku{- Joa ik,... 'k goa ne ', '.}{kiern beetse wandelen}{meet schuen were}\\

\haiku{En zenuwachtig '.}{tril-krabden zijn handen}{overt grauwe schort}\\

\haiku{Zenuwachtig hief.}{de geestelijke de klink}{op en was buiten}\\

\haiku{Een trapje met de,,.}{voet verwijdert hem en weer}{jankt hij heel eventjes}\\

\haiku{en 't laatste van:}{de vier huisjes stond bijna}{altijd gesloten}\\

\haiku{De heer ging achter.}{aan de wagen en hielp nu}{ook de dames uit}\\

\haiku{De fijne geur die.}{van hen uitging vulde heel}{het klein kamertje}\\

\haiku{- 'k H\^en toch al iets,, - '.}{zei hijk h\`e miene vel\'o}{kunne verkoapen}\\

\haiku{En strakker staren.}{alle ogen in de richting}{van het station}\\

\haiku{Reeds op een afstand,,:}{achter de heesters hoorde}{ik het hees gebrul}\\

\haiku{Ala, toe, kom hier, roept,.}{aanmoedigend de meid die}{zich heeft omgekeerd}\\

\haiku{Een dronkaard, die wat,.}{te veel drukte maakt wordt ruw}{opzijgestoten}\\

\haiku{steeds talrijker dringt.}{de menigte op naar de}{stroelende tonnen}\\

\haiku{Weldra zitten de.}{twee oude sukkelige}{goedaards aan tafel}\\

\haiku{Dan vertelt hem de,, '.}{zoon door zijn lachen heen hoe}{oft gegaan is}\\

\haiku{ne pak zien veurbij ' '.}{goan enk miende woarachtig}{datt veur ons was}\\

\haiku{De lauwe frisheid.}{van een vroege lentedag}{over het stille dorp}\\

\haiku{Soms is het niet te {\textquoteleft}{\textquoteright} {\textquoteleft}{\textquoteright}.}{onderscheiden wieer uit}{en wieer in is}\\

\haiku{Zijn makkers staken.}{het gevecht en gans bebloed}{tillen zij hem op}\\

\haiku{Een voldoende huis, ',.}{al wast ook maar een krot}{hadden zij allen}\\

\haiku{G' h\`et huizen lijk.}{kastielen en geld zueveel}{of da ge wilt}\\

\haiku{Hij liep, hij liep maar,.}{altijd verder door zonder}{te weten waarheen}\\

\haiku{Hij was al oud, toen,.}{ik hem leerde kennen wel}{bij de zeventig}\\

\haiku{Die vuile Fie, een, '! -?}{bloem van schoonheid hoe wast}{mogelijk En hij}\\

\haiku{Coralie vroeg of.}{ze zo wel allen samen}{over straat zouden gaan}\\

\haiku{Zij rukt en grabbelt,,.}{gillend van pijn en valt weer}{lamzieltogend neer}\\

\haiku{Belzemien haalde.}{zijn koperen snuifdoos uit}{en nam een snuifje}\\

\haiku{- Alles zou ervan;}{afhangen hoe het verder}{met Tante verliep}\\

\haiku{en ook, natuurlijk, '.}{van de tijd diet nichtje}{hier besteden mocht}\\

\haiku{Cord\'ula kon nog eens.}{gevaarlijk worden in haar}{onweersbuien}\\

\haiku{Doar 'n es niets in '!}{de weireld da zue dwoas es}{ofn mannemeins}\\

\haiku{Met hoge schouders,,,}{als drie schuldigen dropen}{de drie broeders af}\\

\haiku{- Moeten es dwang, doar ',.}{n es niets aan te doene}{sprak ze berustend}\\

\haiku{Maar eensklaps jankte,,.}{hij en wipte als onder}{een zweepklap half op}\\

\haiku{Zij stak haar fris en,:}{vrolijk-blozend blonde}{kopje uit zij riep}\\

\haiku{Van om de beurt met.}{Leontientje uit te gaan}{was niets gekomen}\\

\haiku{ja... Tante... die was!}{eigenlijk de gehele}{oorzaak van alles}\\

\haiku{'k 'n Wee niet hoe'!}{da g ulder op ulder}{hof nog tuegen durft}\\

\haiku{Zij glimlachte en,,.}{stak hem zonder wrok de hand}{ter verzoening toe}\\

\haiku{Hij begost nog ne.}{kier of twie\"e te weerluchten}{en te donderen}\\

\haiku{Informeer moar goed '.}{en blijf zue lange wig of}{datt nuedig es}\\

\haiku{en van dat ogenblik,.}{verlieten zij elkaar niet}{meer de hele dag}\\

\haiku{Doet er mee wat da.}{ge wilt en neemt er veuren}{da ge krijgen keunt}\\

\haiku{Men bleef niet te lang,;}{vooral niet's zondags wanneer}{Veel-Hoar het druk had}\\

\haiku{Daar hielden zij even.}{stil en schaterden er hun}{dolle pret wild uit}\\

\haiku{Je keek hem aan, je,:}{vroeg hem iets je praatte met}{hem een hele poos}\\

\haiku{Hij leek mij kleiner.}{en dikker dan ik hem in}{leven gekend had}\\

\haiku{Paffe!... doar neemt hij '...! -?...?}{zijne luep en sprijngt in}{t woater En en}\\

\haiku{- Haw\`el, hij lag-ie ',.}{hij int woater herhaalt}{deze doodgewoon}\\

\haiku{zijn diepliggende,;}{grijze ogen fonkelden in}{zijn steenrood gezicht}\\

\haiku{- Weet-e nog wel,,?}{Jan da ge mij ne kier ou}{strued verkocht h\`et}\\

\haiku{- Zij zelf voelden zich,.}{nu vaag belachelijk met}{al hun raar gefeest}\\

\haiku{De kerels waagden,:}{krasse toespelingen doch}{ook al vruchteloos}\\

\haiku{De boer getuigde.}{dat de jongen dikwijls in}{die schuur ging slapen}\\

\haiku{Zij lieten Jules,.}{los maar hun overtuiging stond}{onomstootbaar vast}\\

\haiku{Eensklaps en plechtig,!}{doodse stilte in plaats van}{het woelend rumoer}\\

\haiku{vader, een guitig,;}{gezicht altijd vrolijk en}{zeer jong van hart nog}\\

\haiku{Nu en dan bracht ik,;}{een bezoek op het kasteel}{waar vrienden woonden}\\

\subsection{Uit: Verzameld werk. Deel 5}

\haiku{Die blijde losheid.}{friste en knapte meester}{Gevers heerlijk op}\\

\haiku{Maar de zuster was;}{geheel ontredderd en wist}{hoegenaamd geen raad}\\

\haiku{eenmaal van wal zou, ',;}{het vanzelft zij goed of}{kwaad wel verder gaan}\\

\haiku{de geestelijke;}{had opgekeken en hem}{dadelijk herkend}\\

\haiku{viel hem de priester,.}{toornrood met fonkelende}{ogen in de rede}\\

\haiku{Zijn hersens waren,,}{leeg zijn wil was gebroken}{en reeds stamelde}\\

\haiku{maar meester Gevers,,,;}{was sluw noch slim noch strijder}{noch opportunist}\\

\haiku{verbaasde zich de.}{dikke vrouw met in elkaar}{geslagen handen}\\

\haiku{De meester beefde,,,.}{knikte stotterde wist niet}{wat te antwoorden}\\

\haiku{Mevrouw Speliers was;}{eventjes bij de meid in de}{keuken geroepen}\\

\haiku{Ik vraag u of kij?}{heb de brood gerefuseer}{aan meester Gevers}\\

\haiku{kreet de meester als,.}{waanzinnig begrijpend wat}{de oude ging doen}\\

\haiku{en middelerwijl;}{nam zijn verlangen en zijn}{liefde aldoor toe}\\

\haiku{En haar open schaar stond.}{als een bijtende snavel}{naar hem toegeprikt}\\

\haiku{Enkel op de dag...}{van de begrafenis ben}{ik er nog geweest}\\

\haiku{Maar Free was er de.}{kerel niet naar om zich te}{laten dwarsbomen}\\

\haiku{, en het overlijden.}{van zijn vader had zij hem}{niet kunnen melden}\\

\haiku{Doch de oude vrouw.}{was van vermoeienis en}{emotie uitgeput}\\

\haiku{Zij is weg, terug,.}{naar haar ouders en wij gaan}{wettelijk scheiden}\\

\haiku{Blijkbaar vragen ze,.}{weer naar de weg en die lui}{begrijpen hen niet}\\

\haiku{'t Is maandagavond,.}{de repetitie-avond}{van de dorpsmuziek}\\

\haiku{Het eerste antwoord:}{was van trage hoofdschudding}{en een afwijzend}\\

\haiku{Had hij nog ouders,,,?}{broeders zusters andere}{familieleden}\\

\haiku{Was dat {\textquoteleft}'t woar hier,{\textquoteright}...?}{woar doar geweest waarvan hij}{elke winter sprak}\\

\haiku{Ik weet een huisje,,...}{staan in Vlaanderen waar ik}{zou willen leven}\\

\haiku{men had hem maar voor,:}{de grap zo genoemd en het}{kon hem niet schelen}\\

\haiku{Het regende niet,.}{meer maar een gure wind blies}{snerpend om zijn oren}\\

\haiku{en hij ging hun door.}{de opengebleven deur met}{vlugge schreden voor}\\

\haiku{Zij beschouwden haar.}{nog als een kind dat buiten}{hun levenskring stond}\\

\haiku{De dikke vrouw kwam,,;}{soms maar o zo zelden in}{de omwaskamer}\\

\haiku{en ook de oevers:}{en de hemel trillen van}{schitterend leven}\\

\haiku{hoe graag zou ik hem,...}{geld willen geven om vlees}{veel vlees te kopen}\\

\haiku{Met alle kracht en,.}{aandacht is hij in zijn werk}{in zijn plicht verdiept}\\

\haiku{Wel neen, zij weten,?}{het niet meer maar wat komt het}{er nu ook op aan}\\

\haiku{- Hij is naar zijn land,,.}{terug antwoordde de baas}{blijkbaar gegeneerd}\\

\haiku{en daarbinnen, in,:}{de Rosbach was het als een}{pandemonium}\\

\haiku{Ja ja, dat lekker,!...}{bier eenmaal als men eraan}{was gewend geraakt}\\

\haiku{- Meneer, er zijn er}{daar al twee en ze zeggen}{dat ze hier komen}\\

\haiku{Ook de tweede man.}{kwam binnen en deed de deur}{achter zich  toe}\\

\haiku{Hij strompelde naar.}{zijn leunstoel en liet er zich}{zwaar in neervallen}\\

\haiku{Weer ging de deur open.}{en de zoon met de knappe}{meid traden binnen}\\

\haiku{Dadelijk deelde '.}{meneer Bollekens hunt}{gewichtig nieuws mee}\\

\haiku{In zijn jeugd moest hij,.}{zeker een flinke knappe}{kerel zijn geweest}\\

\haiku{en nu eens een die;}{men mooi vond en dan weer een}{die men lelijk vond}\\

\haiku{Wat kunnen ons de,?}{Duitsers de Fransen en de}{Engelsen schelen}\\

\haiku{Tuttuttut... dat zijn,!}{dingen die wij niet kennen}{die wij niet hebben}\\

\haiku{'t Was een nogal,.}{knappe blonde meid van een}{dertigtal jaren}\\

\haiku{Ik voelde, dat het.}{mens aan de praat wou en week}{langzaam achteruit}\\

\haiku{Maar ik zag dat de.}{vrouw hem een duw gaf en hij}{keek op en bleef staan}\\

\haiku{Ik nam mijn hoed af.}{voor Fietriene en wenste}{hun beiden geluk}\\

\haiku{hij liep gebogen.}{doch met forse wil tegen}{de windbuien in}\\

\haiku{meneer Cathoen en ';}{Fietriene dicht bij elkaar}{int mulle zand}\\

\haiku{Ik merkte slechts dat.}{zij kleiner en magerder}{was dan haar zuster}\\

\haiku{Hij sloot de deur en.}{draaide tweemaal de sleutel}{in het nachtslot om}\\

\haiku{De generaal, zijn,.}{wenkbrauwen fronsend bromde}{toornig in zichzelf}\\

\haiku{De generaal en.}{de twee dames spraken een}{hele poos geen woord}\\

\haiku{Het zwart en rood van.}{zijn uniform schitterde vaag}{in de duisternis}\\

\haiku{zijn uitspraak had een,.}{vreemde klank evenals die van}{het kleine meisje}\\

\haiku{Het waren eensklaps,.}{vrienden of zij elkander}{reeds jaren kenden}\\

\haiku{De beide dames.}{hadden zich omgekeerd en}{herkenden hem ook}\\

\haiku{Anderen wenkten.}{naar de meisjes en zonden}{dikke klapzoenen}\\

\haiku{ik ben de Dood en:}{mijn vraatzuchtige woede}{is afgrijselijk}\\

\haiku{Soms is het of er;}{duizenden en duizenden}{zwermden en zoemden}\\

\haiku{en Lovergem was slechts!}{anderhalf uur verwijderd}{van hun eigen dorp}\\

\haiku{Wat nu te Lovergem,:}{gebeurde zou straks ook te}{Bavel gebeuren}\\

\haiku{{\textquoteright} en weg waren ze,,:}{met rijwielen wagens en}{karren of te voet}\\

\haiku{Zelfs de oorlog had;}{hem zijn vermoeiend bedrijf}{niet doen  staken}\\

\haiku{het was de vlucht, de,!}{wilde uitzinnige vlucht}{op leven en dood}\\

\haiku{zij hoorden Vosken;}{aan met open mond en van schrik}{uitgezette ogen}\\

\haiku{Mijn officieren,.}{waren ernstig doch geenszins}{moedeloos gestemd}\\

\haiku{Soms hoorden wij vaag.}{geluid en gestommel in}{de donkere nacht}\\

\haiku{en men strekte de.}{vermoeide benen naar de}{rode vlammen uit}\\

\haiku{De doodsmare was,,;}{rondgestrooid niemand wist hoe}{niemand wist door wie}\\

\haiku{Angstig-gejaagd.}{schoolden zij samen op de}{dorpsplaats rond de kerk}\\

\haiku{De schoonzuster trok ',.}{t venster open keek in de}{schaars verlichte straat}\\

\haiku{en meteen bralde:}{hij onsamenhangende}{verwensingen uit}\\

\haiku{En feitelijk bleef:}{er maar \'e\'en slachtoffer in}{de gehele zaak}\\

\haiku{Met nieuwjaar werd hij.}{afgedankt en op een klein}{pensioen gesteld}\\

\haiku{Hun haat tegen het;}{nieuw vervoermiddel bleef stug}{en onveranderd}\\

\haiku{Zij had het Barontje, ';}{gehuwd om de titel en}{ook omt fortuin}\\

\haiku{anderen slechts heel,;}{weinig omdat ze te zeer}{gegeneerd waren}\\

\haiku{Meneer de pastoor.}{greep naar zijn beker en hief}{hem in de hoogte}\\

\haiku{meneer Fran\c{c}ois te '.}{voet ent Barontje in zijn}{mooie automobiel}\\

\haiku{Heliodoor stond '.}{int mooi gedeelte van}{de Grote Dorpsstraat}\\

\haiku{Pierke zei niets, maar.}{glimlachte en zijn sluwe}{oogjes tintelden}\\

\haiku{- Pakke moar, zei mijn '.}{oom Heliodoor hemt}{kopje aanreikend}\\

\haiku{Toen vulde hij de.}{lege mand met keien en}{keerde weer naar huis}\\

\haiku{- We zoen meniere,.}{wille spreken zei Pierke}{stil-neerslachtig}\\

\haiku{In \'e\'en lange lijn,.}{van de Zwitserse grens tot}{aan de Hollandse}\\

\haiku{Was de vijand daar,?}{reeds of waren het nog steeds}{meer vluchtelingen}\\

\haiku{En meteen, omdat,.}{hij zo rustig was voelden}{zij zichzelf veilig}\\

\haiku{De bosman blies zijn.}{lantaarntje uit en lachte}{in zijn zwarte baard}\\

\haiku{Zij hoorden niets dan.}{hun eigen geluid en dat}{werkte kalmerend}\\

\haiku{Meneer D\'esir\'e.}{opende zelf de flessen en}{vulde de glazen}\\

\haiku{wie niet gauw genoeg,.}{uit de weg ging werd gewoon}{omvergelopen}\\

\haiku{Men wees hem Het huis,.}{van Commercie waarin hij}{als een schicht verdween}\\

\haiku{Zij staakten hun spel.}{en keken de twee heren}{met verbazing aan}\\

\haiku{- D\'a\'ar, juist achter de,,.}{kerk de Vette Os een heel}{goed hotelletje}\\

\haiku{Os zag zitten een.}{wit couvert in de hoogte}{heen en weer zwaaide}\\

\haiku{Ik 'n mag\`e daar.}{niet op peinzen of ik zou}{beginnen schreemen}\\

\haiku{Nu dat is wel dat.}{gij in Oland zijt en wij zijn}{daar toch zoo blij om}\\

\haiku{Als het daar nu maar,.}{bij mocht blijven dan waren}{zij reeds half getroost}\\

\haiku{Ik aarzelde om.}{naar hen toe te gaan en om}{hen aan te spreken}\\

\haiku{Dat konden zij mij.}{eerst niet in duidelijke}{woorden vertellen}\\

\haiku{Heel zelden was er.}{een op jeugdige leeftijd}{ten grave gebracht}\\

\haiku{Er bloeiden altijd, ':}{mooie ouderwetse bloemen}{langst geveltje}\\

\haiku{Daarbuiten op de,.}{landweg stonden mensen in}{de lucht te kijken}\\

\haiku{Hij had de indruk,.}{dat hij aan een ontzettend}{gevaar was ontsnapt}\\

\haiku{Misschien zou God zich,,.}{over hun lot erbarmen hen}{helpen hen redden}\\

\haiku{de doodsangst stolde.}{hen in stijve en stomme}{onbeweeglijkheid}\\

\haiku{De steun, die hen zo,.}{dikwijls gespaard en gered}{had was er niet meer}\\

\haiku{'t Geen da ze bij '.}{ou nien nemen komen}{ze bij mij pakken}\\

\haiku{- Joa, Fielemiene, ';}{k zoe malgr\'e menier de}{p\'aster moete spreken}\\

\haiku{Meneer de pastoor.}{stond zwart en roerloos naast de}{kist en sprak geen woord}\\

\haiku{- 'k Gelueve da,.}{ze van den nacht goan deure}{breken zei Boerke}\\

\haiku{Het roggebrood, dat,;}{ze thans aten was niet slechter}{dan v\'o\'or de oorlog}\\

\haiku{- 'n Beetse lieger,, ' ' ',.}{Pierkn ziet nie mier}{klaagde de kleine}\\

\haiku{maar gij, Irma, moet,,,,...}{weg en gij ook Elodie en}{gij vooral Emma}\\

\haiku{zij betaalden goed,;}{evenveel en soms nog meer dan}{mevrouw de gravin}\\

\haiku{al hun gedachten,;}{al hun verlangens en hun}{hoop vlogen erheen}\\

\haiku{Toen bleef het bekje '.}{eindelijk wijd open staan en}{t oog verstarde}\\

\haiku{- We willen ne stien.}{veur Alineke geven in}{de nieuwe kirke}\\

\haiku{Zij togen naar 't.}{Kasteelken en vroegen om}{meneer te spreken}\\

\haiku{Wat was dat stil en,!}{verlaten zo'n groot dorp in}{nachtelijke rust}\\

\haiku{een troepje koeien,;}{rustig grazend als grote}{bloemen op het groen}\\

\haiku{Zij baden, de ogen,.}{strak op het grote roze}{gebouw gevestigd}\\

\haiku{veertien dorpen op,!}{en af asjeblief alles}{te voet heen en weer}\\

\haiku{Na een week was de;}{laatste zorg over Alinekes}{toestand verdwenen}\\

\haiku{Op hun uiterst best,,.}{gekleed als voor een kermis}{togen zij erheen}\\

\haiku{Dan golden daar nog {\textquoteleft}{\textquoteright};}{slechts de wetten en de geest}{van derepubliek}\\

\haiku{Hij es op 't land,,.}{antwoordde de moeder met}{inspanning wrijvend}\\

\haiku{- Ha, moeder, doar 'n...}{zijn nie veel gemienten in}{Vloanderen die}\\

\haiku{Buigend onder het.}{lage boogdeurtje trad ik}{achter hem binnen}\\

\haiku{de {\textquoteleft}burgemeester{\textquoteright} '!}{en de baas uit de Speurgaal}{t laatst van allen}\\

\haiku{Vandaar die blauwe,!}{plekken die men op het lijk}{geconstateerd had}\\

\haiku{Daar rustte dus Karl,,.}{de vijand die slecht geweest}{was voor de mensen}\\

\haiku{alleen het water...}{wist en zou voor eeuwig zijn}{geheim bewaren}\\

\haiku{Zij loeiden even, en,.}{ontlastten zich rustig de}{kop naar hem gekeerd}\\

\haiku{hij herhaalde het,,:}{schreeuwend snikkend jammerend}{op alle tonen}\\

\haiku{De burgemeester.}{trok zich met wethouder en}{veldwachter terug}\\

\haiku{'t zal misschien nog...!}{wel terecht komen gelijk}{de koe van Dons}\\

\haiku{vroeg ik na een poos,.}{om de benauwd wordende}{stilte te breken}\\

\haiku{Zij drongen aan, dat;}{ik even zou binnenkomen}{en iets gebruiken}\\

\haiku{Het was een mooi en,.}{vrij uitgestrekt meer geheel}{omringd door bossen}\\

\haiku{Hij is een ietsje.}{aan de drank verslaafd en drinkt}{altijd jenever}\\

\haiku{drinkt, ietwat morsend,.}{zijn borrel leeg en bestelt}{er een tweede}\\

\haiku{Hij keek naar iets, in,;}{de eerste rij stalles daar}{schuin beneden ons}\\

\haiku{Mijn illustere, '}{vriend sprak geen woord maart kwam}{mij voor alsof zijn}\\

\haiku{Instinctmatig, met,.}{een lichte huivering trok}{ik mij even terug}\\

\haiku{of mij te laten,}{aandienen de stoeptreden op}{en de hall binnen}\\

\haiku{en zij sloeg haar ogen,.}{neer terwijl een kleur even op}{haar wangen gloeide}\\

\haiku{Ik ken Marie reeds.}{zoveel jaren en heb ze}{in lang niet gezien}\\

\haiku{zo staan geweren, '.}{van soldaten in rust op}{t exercitieveld}\\

\haiku{'t is te zeggen;}{dag van schele hoofdpijn en}{van katterigheid}\\

\haiku{Daar kwam de Duitse.}{horde met kletterende}{hoeven aangestapt}\\

\haiku{maar nu hadden de:}{mensen niets geen pret meer in}{hun rare doening}\\

\haiku{De meeste boeren.}{en bewoners waren op}{hun erf gebleven}\\

\haiku{Op de ransel was,,}{een keteltje gebonden}{dat zij losmaakten}\\

\haiku{- Hot water, some,.}{hot water herhaalde de}{man ongeduldig}\\

\haiku{Terwijl hij sprak keek,.}{hij naar de omstaanders en}{ook naar de boeren}\\

\haiku{Toen wenkte hij de.}{boeren bij zich en deed hen}{de tombe vullen}\\

\haiku{Drie vreemde namen,,.}{stonden met zwarte letters}{erop geschilderd}\\

\haiku{riepen eensklaps de,.}{kerels onzacht het volk van}{de stoep wegduwend}\\

\haiku{Hij had een hok, vlak.}{v\'o\'or het huis en werd daar soms}{aan vastgeketend}\\

\haiku{wij waren daar reeds, '.}{z\'o aan gewend dat wet}{haast niet meer merkten}\\

\haiku{t Verleden kwam '.}{alst ware in golven}{naar hem toegestroomd}\\

\haiku{O, die kleren, die,!}{gezellige civiele}{kleren van weleer}\\

\haiku{Zij wrongen hem de.}{duimschroeven op de polsen}{en namen hem mee}\\

\haiku{te weten of hij,,!}{al dan niet tot kolonel}{zou promoveren}\\

\haiku{Met een pennetrek.}{konden zij hem voordragen}{en doen benoemen}\\

\haiku{Hij zuchtte kreunend.}{en zijn zwakke ogen keken}{zoekend om hem heen}\\

\haiku{En vol zachte zorg.}{drukte zij hem weer in de}{kussens achterover}\\

\haiku{En hij wil ook graag,.}{uw foto hebben als gij}{er een kunt missen}\\

\haiku{Toen werd hij eensklaps,,.}{wakker midden in de nacht}{en hoorde niets meer}\\

\haiku{De huizen waren,.}{er wit met rood pannendak}{en groene luiken}\\

\haiku{Hij zat te zingen}{op een heestertakje dat}{begon te groenen}\\

\haiku{ze noteerde en.}{zoals ook de moderne}{Theuriet ze opschreef}\\

\haiku{Ze komen zij hier,.}{binnen drijnken nen dreupel}{of twie\"e en zijn wig}\\

\haiku{'k Moak hem 't.}{beste eten geried dat op}{de weireld bestoat}\\

\haiku{Hij stond roerloos en ' '.}{keek int verschiet naar de}{kerk ent kasteel}\\

\haiku{Wij komen op een,.}{kruispunt waar ik even  naar}{de weg moet vragen}\\

\haiku{Ik denk dat het nu;}{wel langzaam aan tijd wordt om}{terug te keren}\\

\haiku{- Moar h\`et-e gij, ',?}{ulder nie gezeid datt}{wel woar es Gaston}\\

\haiku{Ge zoedt mij plezier,,.}{doen meniere mee hem nie}{mier mee te nemen}\\

\haiku{Zodra hij mij zag, {\textquoteleft}{\textquoteright}.}{komen begon dedikke}{luid te jubelen}\\

\haiku{- 't'n Ziet er giene,.}{gemakkelijke keirel}{uit merkte ik op}\\

\haiku{- Hein, monsieur le,?...}{vicomte qu'est-ce que}{vous dites de ca}\\

\haiku{Hun bedrijvige}{vingers rondden de bruine}{schillen als krullen}\\

\haiku{ik goa uek tegen,.}{den gevel zitten percies}{gelijk de meinschen}\\

\haiku{Een klokje luidde ',;}{ergens int verschiet heel}{dromerig en ver}\\

\haiku{vroeg hem grijnzend een.}{oude kerel met gouden}{kraag en grijze baard}\\

\haiku{Es da nou nie stom '?}{da ge mee d'ander meinschen}{nie gevluchtn zijt}\\

\haiku{V\'o\'or mij stond een man,,!}{een reus zoals ik er nog}{nooit een gezien had}\\

\haiku{vroeg mij, in vrij goed,,:}{Frans met een stem die zacht maar}{diep als koper klonk}\\

\haiku{Heren beginnen.}{altijd met elkander in}{een bar te brengen}\\

\haiku{Hij verslond haar met,,;}{zijn ogen hij snoof haar op}{hij smulde van haar}\\

\haiku{Wij hadden toch een.}{halve afspraak om nog eens}{samen uit te gaan}\\

\haiku{Wat 'n idee om daar!}{alleen te komen zitten}{in die luxeloge}\\

\haiku{de pauze viel in.}{en een aantal toeschouwers}{verlieten hun plaats}\\

\haiku{Zij stegen in de.}{wagen en Vien nam weer zijn}{plaats in achter hen}\\

\haiku{Je rustte uit, met;}{de ellebogen op het}{vieze tafeltje}\\

\haiku{De voeten waren,.}{plat en klein opvallend klein}{voor zulk zwaar lichaam}\\

\haiku{Ik zag hem zijn keus,,.}{doen betalen met een pak}{naar buiten komen}\\

\haiku{- Past op, sloeber, os!}{g'azue noar mij blijf kijken}{en mij blijf volgen}\\

\haiku{- Ha! 't'n ziet er nog,!}{giene gemakkelijken}{uit dienen dikke}\\

\haiku{een zoon, die klein was,,.}{als hijzelf en een dochter}{die een bochel had}\\

\haiku{- Ja of neen, wilt gij?}{ons de plaats aanwijzen waar}{uw voorraad meel ligt}\\

\haiku{Maar weer namen de ';}{gendarmen daar niet int}{minst notitie van}\\

\haiku{En weer galmde de:}{lang-gerekte kreet als}{een hanengekraai}\\

\haiku{Uzubup\'u kwam voor en.}{leidde de bezoeker in}{het spreekkamertje}\\

\haiku{Hij dacht daarover na.}{met sardonische glimlach}{en stekende ogen}\\

\haiku{Met een soort moedwil.}{keerde B\'er\'enice zich}{tot haar vader om}\\

\haiku{Toch voelden zij wel,.}{de grote plotselinge}{leegte om zich heen}\\

\haiku{de auto was al...}{voorbij en zij zagen het}{kruisje niet meer}\\

\haiku{Zijn vrouw is lang en,.}{mager met ingevallen}{borst en een scheel oog}\\

\haiku{hij heeft zijn flesje,;}{en zijn pijpje hij geniet}{en is gelukkig}\\

\haiku{Marcel keek naar het;}{huis in wording en schudde}{ook wel eens het hoofd}\\

\haiku{Jan-Sies wendde:}{zich vriendelijk tot Marcel}{en vroeg glimlachend}\\

\haiku{Maar hij moest wel iets;}{verzinnen om zijn bezoek}{te rechtvaardigen}\\

\haiku{- Ha, da zoe toch 'n!}{schande zijn dat da azue hiel}{de nacht moest duren}\\

\haiku{Het hanengeschrei.}{klonk verzwakt achter de tuin}{van meneer Alexander}\\

\haiku{Ik zag de oude -;}{schoolmeester De Brave al}{zoveel jaren dood}\\

\haiku{In die tijd kenden;}{wij geen onderscheid tussen}{mooi en lelijk Frans}\\

\haiku{ook zij was ervan,.}{overtuigd dat ik mijn straf ten}{volle had verdiend}\\

\haiku{ging de vrouw, die hem,.}{voorbij het raam zien komen}{had hem tegemoet}\\

\haiku{- Pijn 'n h\`e 'k niet,, ' '.}{menier Ag\'ust moarkn zie}{hoast nie  mier}\\

\haiku{- 'k 'n H\`e doar nog,,.}{niet op gepeisd Celestien}{antwoordde zij kil}\\

\haiku{Met diepe smeking.}{in zijn lodderige ogen}{staarde hij haar aan}\\

\haiku{{\textquoteleft}drijnkt ou iest 'n stik' '.}{in ou kroage os g}{azue nien durft goan}\\

\haiku{Octavie schudde.}{het hoofd en maakte grote}{armbewegingen}\\

\haiku{Hij haalde rustig.}{zijn koker te voorschijn en}{stak een sigaar op}\\

\haiku{Da Celestien hem!}{nou en dan komt zat drijnken}{in ou vroeger huis}\\

\haiku{Zij sprak erover met.}{Celestien en die nam het}{ook vrij kwalijk op}\\

\haiku{en zij spoedde zich,.}{op straat om te horen hoe}{de zaken stonden}\\

\haiku{riep ze gans ontdaan,;}{tot de oude tante die}{bij haar inwoonde}\\

\haiku{gilde Hortense,,.}{sidderend van toorn toen ze}{weer bij Tante was}\\

\haiku{'k zoe uek nog ne, '.}{kier willen dansen h\^en op}{n airken meziek}\\

\haiku{en as ik het doe ',;}{t es wel te wille van}{mevreiwe zulle}\\

\haiku{Zij hadden stille.}{pret om het zuur gedoe van}{de beide vrouwen}\\

\haiku{- Afijn, Hortense, as ' '....}{get liever nien h\`et}{hernam Anneke}\\

\haiku{Kamiel is er ten;}{zeerste op gesteld om mee}{te blijven spelen}\\

\haiku{Hij is z\'o lang, dat;}{hij meestal onder het}{hemd wordt weggestopt}\\

\haiku{Als hij u aankijkt,.}{weet gij eigenlijk nooit of}{hij u wel aankijkt}\\

\haiku{Talrijke manden;}{vol vis stonden er in een}{kring om het pleintje}\\

\haiku{h\`et-e gij geld? '!}{om te betoalent Es}{twie en tseventig fran}\\

\haiku{ik zou einden ver,.}{gelopen zijn om er niet}{meer langs te komen}\\

\haiku{En hij haastte zich,.}{uit het huis de bankjes in}{zijn broekzak stoppend}\\

\haiku{En bulderend sloeg.}{hij weer met beide handen}{op zijn knie\"en}\\

\haiku{riep Vertriest, de,.}{huisschilder die gekleed stond}{als om uit te gaan}\\

\haiku{Hij gaf vijf frank fooi,.}{aan de bediende die ten}{diepste dankte}\\

\haiku{riep eensklaps Daenens,.}{die tot nog toe de mond haast}{niet geopend had}\\

\haiku{- Ala, kom binnen en'.}{drijnk ne pot kaffee ier da}{g aan ou wirk goat}\\

\haiku{vloekte Allewies, '.}{die ookt beledigend}{gedoe gezien had}\\

\haiku{Hij woonde daar heel,.}{alleen met een oude knecht}{en een oude meid}\\

\haiku{Men zette haar even,.}{neer op het gras onder de}{bloeiende kruinen}\\

\haiku{De mannen namen.}{de kist weer op en tilden}{haar op hun schouders}\\

\haiku{Eigenlijk was het ' '.}{leven wel steedst zelfde}{opt platteland}\\

\haiku{Men stond vroeg op, men;}{gebruikte het ontbijt en}{ging naar de akker}\\

\haiku{boter, eieren,....}{ham en ook geregeld wat}{geld laten zenden}\\

\haiku{Zoals vanzelf spreekt,.}{verviel hij daarbij wel eens}{in herhalingen}\\

\haiku{Hij lachte mij uit,;}{omdat ik per rijwiel tot}{daar was gekomen}\\

\haiku{Men kon wel merken,;}{dat hij gaandeweg geler}{en magerder werd}\\

\haiku{Het was om zo te.}{zeggen een sociale}{betrekking voor hem}\\

\haiku{Zij mogen alles, '!}{hebben als ze mij maart}{leven laten}\\

\haiku{En de uitdrukking,:}{waarmee hij ons nakeek zal}{ik nooit vergeten}\\

\haiku{Alleen haar mond was.}{tandeloos geworden en}{zij liep gebogen}\\

\haiku{Hij zette hun 't;}{mes op de keel en eiste}{hun verborgen geld}\\

\haiku{Soarlewie schudde.}{het hoofd en liet moedeloos}{de armen zakken}\\

\haiku{- Mais non, zei hij, - c'est;}{la premi\`ere fois que}{je passe par ici}\\

\haiku{Meneer Alfred zag,.}{zeer bleek met een pijnlijke}{grijns om de lippen}\\

\haiku{Hij kwam binnen in.}{de Schone Warande om}{afscheid te nemen}\\

\haiku{Hij had het zo graag,;}{willen weten hij had het}{haar willen vragen}\\

\haiku{Een glimlach kwam over.}{zijn lippen en even praatte}{hij gewoon met haar}\\

\haiku{Geen mens meer in de.}{omtrek en geen geluid meer}{in de stille nacht}\\

\haiku{De dokter kon er;}{zo grappig en smakelijk}{zitten vertellen}\\

\haiku{wie Anzelieksken;}{kon doen lachen had succes}{en was tevreden}\\

\haiku{Hij lei zijn vette.}{poot tussen haar vingers en}{drukte die zwijgend}\\

\haiku{ook zij was gans in ',;}{t zwart gekleed in zwarte}{zij met git erover}\\

\haiku{Zij zag eruit als,,!}{een dame waarachtig als}{een echte dame}\\

\haiku{Zij droeg ook lange,.}{gouden oorbellen en een}{diamanten broche}\\

\haiku{Pruttelend draaide.}{zij de lichten uit en ging}{hem voor naar boven}\\

\haiku{maar gehoorzamen,,.}{zou hij en werken zou hij}{ook nog als een knecht}\\

\haiku{Tegen de oude {\textquoteleft}{\textquoteright}.}{stam van deSperre hing een}{klein kapelletje}\\

\haiku{Het was iets nieuws, iets,!}{onverwachts en onbekends}{iets hartstochtwekkends}\\

\haiku{joelden de bengels.}{in dolle pret  over Bruno's}{grappig gezegde}\\

\haiku{hijgde de dikke,.}{vrouw naar het ontdaan gelaat}{van haar man starogend}\\

\haiku{Hij nam het glas in.}{de hand en hield het naar het}{schaarse daglicht toe}\\

\haiku{en nog zachter, als,:}{schromend en zich schamend met}{bevende lippen}\\

\haiku{Hij liep gebogen,,.}{met hoge schouders tegen}{de gure wind in}\\

\haiku{Dat was wat hij het.}{meest ontbeerde en waaraan}{hij niet wennen kon}\\

\haiku{Micus kreeg een vreemd '.}{gevoel over zich en zijn hart}{ging aant jagen}\\

\haiku{Zij zat daar nog een,.}{ogenblik gedrochtelijk als}{een gruwelmonster}\\

\haiku{De kleine met zijn ';}{dikke snor en kikkerogen}{zat int midden}\\

\haiku{Hij keek dan niet naar,;}{het kind waarmee hij anders}{zo gaarne speelde}\\

\haiku{Andr\'e drukte haar.}{nauw tegen zich aan en sloeg}{de lok naar achter}\\

\haiku{Het was acht uur en.}{hij wist dat Marcela v\'o\'or}{donker moest thuis zijn}\\

\haiku{Zijn stem was eensklaps.}{week geworden en tranen}{stonden in zijn ogen}\\

\haiku{- Mematsjen, geef mij!}{nog nen dreupel en geeft er}{menier uek ienen}\\

\haiku{zijn vrouw, geboren,,.}{Bokkaert-Van Imme deed}{open Het mij binnen}\\

\haiku{Zijn wangen waren.}{ietwat ingevallen en}{zijn ogen stonden dof}\\

\haiku{Alleen zijn neus bleef.}{gloeien als een tomaat in}{zijn verlept gezicht}\\

\haiku{en hij weet heel goed,.}{dat hij niet mag en dat het}{hem verboden is}\\

\haiku{Men droomt er weg in;}{po\"ezie bij het gezang}{van de vogelen}\\

\haiku{Ik open de deur en,.}{als een kogel vliegt hij de}{trap af naar buiten}\\

\haiku{en vragen rijzen,...}{in mij op die ik niet kan}{beantwoorden}\\

\haiku{Het is zeer zeker;}{op zichzelf een feit van geen}{of weinig belang}\\

\haiku{- Ik niet hoor, ik heb...!}{er net zoveel leed over als}{u. Mysterie}\\

\haiku{De hoofden buigen,.}{een vage triestigheid komt}{over de gezichten}\\

\haiku{Ik dacht, dat broertje.}{er wellicht de weinige}{geur had afgelikt}\\

\haiku{- Wel, ik meen, dat ik!}{mij de teleurstelling niet}{erg zou aantrekken}\\

\haiku{Wij schaterden z\'o,.}{dat moeder bijna angstig}{voor haar raam kwam staan}\\

\haiku{Had ik maar gedurfd,.}{dan zou ik nog m\'e\'er op zijn}{bord gelegd hebben}\\

\haiku{'k 'n kwam moar ne, '.}{kier informeren hoe dat}{t mee Luusken es}\\

\haiku{Mijn vader was een,.}{bezadigd man  die van}{maat en orde hield}\\

\haiku{Stormen, oproeren,;}{oorlogen hadden over en}{om hem heen gewoed}\\

\haiku{Teleurgesteld ging '.}{ik er rondom heen en kwam}{terug opt plein}\\

\haiku{Ook een twintigtal!}{blaffende en brullende}{honden te water}\\

\haiku{En wreed blikte hij ',,.}{int ronde uitdagend}{met zijn schele ogen}\\

\haiku{geen wrede vuurflitsen.}{branden meer dreigend uit de}{grijze golven op}\\

\haiku{De chef van de wacht.}{stond op en keek reikhalzend}{over de hoofden heen}\\

\haiku{Een hele poos stond,.}{hij roerloos-luisterend stil}{om het te horen}\\

\haiku{Peet, door de woestheid,;}{van zijn aanval meegesleept}{viel boven op hem}\\

\haiku{Men hoeft het niet te.}{vragen of zij vers uit de}{loopgraven komen}\\

\haiku{Er zijn veel vrouwen.}{uit het volk en kinderen}{in die menigte}\\

\haiku{Die angst en droefheid.}{hangen voortdurend over ons}{gehele leven}\\

\haiku{'t Was tijdens de,,.}{noenstond tussen een en twee}{anders hun rustuur}\\

\haiku{Het ongeloof, de.}{slechtheid van de wereld is}{de schuld van alles}\\

\haiku{Zij patrouilleren,.}{te paard in kleine troepjes}{alom over het land}\\

\haiku{die kiekens liepen,!}{huele doagen zat zue}{zat as meinschen}\\

\haiku{D\'a\'ar, in die hoek, stond.}{de lijkwagen en zij moesten}{er rakelings langs}\\

\haiku{s Avonds v\'o\'or elke.}{terechtstelling had er een}{repetitie plaats}\\

\haiku{Het floddert tegen,.}{een van de ramen of het}{wou binnenkomen}\\

\haiku{Waarom is de Mens ';}{de natuurlijke vijand}{vant meesje}\\

\haiku{zij lopen nog eens, ',;}{hier en daar alst ware}{doelloos door elkaar}\\

\haiku{Ik zag het gaan in ',,...}{t felle zonnelicht steeds}{verder steeds verder}\\

\haiku{Daar stond ik weer met:}{mijn vraag aan de oever van}{het klotsend water}\\

\haiku{Eensklaps... wat kan een!}{beest toch raar doen en wat mag}{het soms bezielen}\\

\haiku{Dat was een van de.}{lievelingsgerechten van}{hun overleden vriend}\\

\haiku{Maar zijn stuur was hij.}{kwijt en dat zou hem wellicht}{duur komen te staan}\\

\haiku{Ik ga terug naar.}{de kleedkamer en kom weer}{gewoon te voorschijn}\\

\haiku{Maar hij bukt zich en.}{ik vlieg  door mijn eigen}{vaart tegen de grond}\\

\haiku{se lan\c{c}aient sur.}{la proie et l'avalaient}{avec gloutonnerie}\\

\haiku{Elle \'etait loin la!}{joie insouciante de}{ses jeunes ann\'ees}\\

\haiku{Affaiss\'ee entre les,,.}{bras d'Eiso Janke pleurait}{toujours abondamment}\\

\haiku{Betje l'aper\c{c}ut;}{mais fit semblant de ne pas}{le reconna{\^\i}tre}\\

\haiku{puis, avec un cri de,;}{jubilation elle}{se pr\'ecipita}\\

\haiku{Mais il marche, tu,!}{verras nous les battrons au}{prochain voyage}\\

\haiku{{\textquoteright} Il s'interrompit,,;}{brusquement la regarda en}{face dans les yeux}\\

\haiku{prosp\`ere et le,;}{114 allait revenir il}{devait revenir}\\

\haiku{Le chemin le plus,,:}{court pour rentrer chez lui \'etait}{d'obliquer \`a droite}\\

\haiku{Une seule chose:}{en lui \'etait pr\'ecise et}{toute puissante}\\

\haiku{Un \'elan de folle.}{jalousie et de rage}{le bouleversa}\\

\haiku{L'un, l'ouvrier, passait.}{inaper\c{c}u et on le}{laissait tranquille}\\

\haiku{se cramponnaient,.}{les uns aux autres pour ne}{pas \^etre balay\'es}\\

\subsection{Uit: Verzameld werk. Deel 6}

\haiku{Men kan het zonder{\textquoteright}.}{aarzeling vergelijken}{met deze stukken}\\

\haiku{Deze titel werd.}{geschrapt en vervangen door}{Familiedrama}\\

\haiku{Levensschetsen en.}{Portretten bijeengebracht}{door Mr. J. Kalff jr}\\

\haiku{Levensschetsen en.);}{Portretten bijeengebracht}{door Mr. J. Kalff jr}\\

\haiku{Wat staat dat frivool,,!}{op die oude deftige}{statige bomen}\\

\haiku{De zon komt door de.}{grijze wolken piepen en}{lonkt mij lachend toe}\\

\haiku{'t Gebeurde daar,.}{ergens in Vlaanderen op}{mijn geboortedorp}\\

\haiku{En voor de tweede.}{maal bleef meneer de pastoor}{hopeloos steken}\\

\haiku{De vader neemt het.}{in zijn armen op en kust}{het als waanzinnig}\\

\haiku{Ik trek aan het raam,:}{om hem weg te jagen maar}{het is reeds te laat}\\

\haiku{Voor de tweede maal.}{wordt hij naar een andere}{windrichting gekeerd}\\

\haiku{hij is geen profeet,,;}{zegt hij zelf en de mensen}{weten dat ook wel}\\

\haiku{Wat lijkt de oude,!}{grijze molen ijl en fijn}{in dat etherisch licht}\\

\haiku{Een ouderwetse,,;}{rammelende koets maar toch}{een luxe-en-erekoets}\\

\haiku{en tussen al dat.}{groen vonkt en spat het goud van}{de bloeiende brem}\\

\haiku{Het waait, maar hoe de,.}{wind ook zit geen tochtje kan}{mij daar hinderen}\\

\haiku{en de aarde, die,.}{dorst heeft krijgt te veel en kan}{alles niet slikken}\\

\haiku{Trouwens, het valse,,.}{het onechte grijnst u al}{dadelijk tegen}\\

\haiku{- Joa moar, Van Rompu,?}{h\`et-e gij die muerd}{nie zien gebeuren}\\

\haiku{Ik heb geleerd te,.}{doen gelijk het oud wijs paard}{van mijn molenaar}\\

\haiku{In \'e\'en adem stormde.}{ik de heuvel op en kwam}{ook buiten adem aan}\\

\haiku{Hoe het mij bekend,,.}{was dat hij zo heette kan}{ik niet verklaren}\\

\haiku{Alles is grijs, en,.}{ook guur en kil als op een}{novemberdag}\\

\haiku{Er ligt nog zulk een, '!}{schone rijke toekomst voor}{ons int verschiet}\\

\haiku{Het hoge koren.}{staat al opeens vol rode}{en blauwe bloemen}\\

\haiku{die hebben genoeg,;}{aan hun eigen stralende}{blozende schoonheid}\\

\haiku{Hier en daar is er;}{een wit hoofdje onder al}{die blauwe hoofdjes}\\

\haiku{vlak naast mij een zeer,;}{verliefd en jeugdig paartje}{dat heel veel pret had}\\

\haiku{'t Was in de buurt,.}{van Tripoli tijdens de}{zware gevechten}\\

\haiku{men wil en moet nog ':}{steeds en nogmaals de klank van}{t wonder horen}\\

\haiku{Wat is het geluid!}{van een sikkel gans anders}{dan dat van een zeis}\\

\haiku{Het ganse land trilt '.}{alst ware en wemelt}{van bedrijvigheid}\\

\haiku{Wat staan ze mooi en!}{rijk en warm goudbruin daar in}{de zon te gloeien}\\

\haiku{Toen sloot zich weer het,.}{grijs gordijn en eensklaps werd}{het avond triestig avond}\\

\haiku{Wat horen ze, sinds,!}{eeuwen al onveranderd}{bij onze volksaard}\\

\haiku{Ik heb hem aan de.}{trein gebracht en ben alleen}{weer thuis gekomen}\\

\haiku{ben ik genoodzaakt...}{voor uw zo gewaardeerde}{uitnodiging}\\

\haiku{daar verkondigt hij,.}{zelf een leer zodat ik niet}{hoef te antwoorden}\\

\haiku{De glorie van de.}{hoge bomen scheen hen gans}{te overweldigen}\\

\haiku{Eeuwigdurend zijn.}{ze en onveranderd in}{de loop der tijden}\\

\haiku{Mijn oude molen.}{heeft nieuw doek over twee van zijn}{wieken gekregen}\\

\haiku{Zij waren vuil en '}{vies en lelijk en int}{voorbijgaan roken}\\

\haiku{De molen heeft zijn.}{twee andere zeilen ook}{vernieuwd gekregen}\\

\haiku{Is het niet alsof?}{je levend in je donker}{graf werd neergelegd}\\

\haiku{{\textquoteleft}Mon \^ame est triste{\textquoteright}... {\textquoteleft}{\textquoteright} {\textquoteleft}{\textquoteright}.}{15 oktoberMon \^ame}{is niet meertriste}\\

\haiku{In het kerkje speelt;}{het orgel en weergalmen}{plechtige stemmen}\\

\haiku{- Omdat de mussen!...}{er anders piepen dan in}{ons eigen dorp}\\

\haiku{Nooit zou ik verwacht,.}{hebben in die wereld zo}{iets te aanschouwen}\\

\haiku{Het ogenblik daarna,.}{is alles stil alsof er}{niets meer gebeurde}\\

\haiku{Of is dit nu het?}{eind van alles en daagt er}{zelfs geen morgen meer}\\

\haiku{Ik moet hem nog eens,;}{van dichtbij aanschouwen hij}{leeft intens vanavond}\\

\haiku{de ganse natuur,,.}{schreit haar rouw haar droefheid}{haar ellende uit}\\

\haiku{Soms zie ik, in die,;}{nieuwe wereld mensen die}{ik hier gekend heb}\\

\haiku{de lansier draait en,.}{draait om er wee en razend}{van te worden}\\

\haiku{en er  was een,,.}{witte stenen molen met}{helrode wieken}\\

\haiku{De plek waar Sint Elooi,.}{eenmaal gestaan had kon hij}{niet terugvinden}\\

\haiku{Rechts lag de Rode,,,.}{Berg verder op de Franse}{grens de Zwarte Berg}\\

\haiku{Ik stond in de stad.}{Ieperen voor ik er mij}{rekenschap van gaf}\\

\haiku{Daar komt iets aan, in,...}{het verschiet over de grijze}{naaktheid van de weg}\\

\haiku{En zij valt uit in,.}{een vloed van verwijten die}{niet te stelpen zijn}\\

\haiku{Dat haar boerderij,.}{aan stukken geschoten is}{daarin berust ze}\\

\haiku{een draaitje aan de,.}{slinger en weg is hij naar}{betere oorden}\\

\haiku{even krijgen wij een, (,!}{mooie zachte grintwego welk}{een verademing}\\

\haiku{Waren we maar door!}{die ellendige trompet}{niet opgehouden}\\

\haiku{Weten die domme?}{lui in Bouillon dan nog niet}{wat sandwiches zijn}\\

\haiku{We hadden 't ook!}{nog wel gedeeltelijk uit}{zuinigheid gedaan}\\

\haiku{Je hebt het gevoel;}{of ze nu voortdurend weer}{zullen platlopen}\\

\haiku{De wegen zelf zijn,,!}{er helaas in Frankrijk niet}{op vooruitgegaan}\\

\haiku{Er is werkelijk.}{t\'e veel wijn en t\'e weinig}{water in die streek}\\

\haiku{Het was alsof 't,.}{alleen bestond boven en}{buiten alles om}\\

\haiku{De bomen en de.}{heesters om ons heen waren}{gitzwart geworden}\\

\haiku{Je suis venu ici,,.}{avec monsieur il y a}{cinq ans en auto}\\

\haiku{Spelende dames,,. '}{vooral winnende dames}{moet je niet storen}\\

\haiku{Wat klinkt die muziek,...!}{verleidend en meeslepend}{als de bank betaalt}\\

\haiku{Zij gaan even in een,.}{hoekje zitten ledigen}{hun beurs en tellen}\\

\haiku{Wat lijkt het klein, klein,,!}{armzalig klein van zo}{hoog en van zo ver}\\

\haiku{en, tot een man, die:}{juist toevallig op zijn fiets}{ons tegemoet kwam}\\

\haiku{Die ru{\"\i}neren.}{de weg en men heeft geen tijd}{hem te herstellen}\\

\haiku{- C'est bien beau quand on,,,.}{est l\`a-haut verzekerde}{kalmbewust de man}\\

\haiku{Wij reden... Maar 't;}{was een dag van tegenspoed}{en van vergissing}\\

\haiku{riepen mijn dames,.}{tot de eerste man die zij}{konden aanklampen}\\

\haiku{klonk het verbaasde,.}{antwoord net als op de brug}{van de Bidassoa}\\

\haiku{Nog v\'o\'or mijn dames,.}{daartoe bevel geven houd}{ik de wagen stil}\\

\haiku{Mensen en dieren,.}{begrijpen elkander nog}{niet of niet meer}\\

\haiku{- We zitten hier als,.}{schipbreukelingen zuchtte}{\'e\'en van mijn dames}\\

\haiku{Ik zie nauwelijks,.}{nog mijn weg een weg van slijk}{en modderplassen}\\

\haiku{Het orgel zweeg, de,.}{stemmen zwegen de kaarsen}{werden uitgedoofd}\\

\haiku{Er was de lange,, ';}{blonde weg lijnrecht tot in}{t onzichtbare}\\

\haiku{'t Zijn als hopen:}{en bundels linten van de}{prachtigste kleuren}\\

\haiku{In 't geheel niet,,;}{maar zodra ik champagne}{zie moet ik zingen}\\

\haiku{Zij springt, en meteen,!}{struikelt en valt zij een kort}{gilletje slakend}\\

\haiku{Die is het welke '.}{plotseling weer een schrik over}{t gezelschap jaagt}\\

\haiku{roep ik, verbaasd dat.}{le Dieu met geen enkel}{woord zijn hond beknort}\\

\haiku{En nu nog wel in ',!}{t gure najaar in een}{open automobiel}\\

\haiku{roept Maeterlinck,.}{met een gebaar van wanhoop}{vluchtig omkijkend}\\

\haiku{in Mont\'elimar;}{wordt even opgehouden om}{nougat te kopen}\\

\haiku{Hond en kat schijnen;}{elkaar met diepe aandacht}{te bestuderen}\\

\haiku{Zijn goed wordt gedroogd,;}{hij krijgt een ander pijpje}{en verse tabak}\\

\haiku{Hij komt nu weldra,.}{uit in Groot-Nederland}{en later in boek}\\

\haiku{{\textquoteleft}Nu mag, om het even:}{welke kolossaalsterke}{bruut mij aanranden}\\

\haiku{In welke plaats zal?...}{ik hem voor de zevende}{maal bezoeken}\\

\haiku{Zij hebben op het;}{slagveld de kogels om hun}{hoofd horen suizen}\\

\haiku{I done my small,.}{letter because y ave}{headache}\\

\haiku{De vormen van het,:}{grote schip tekenen zich}{af in het vage}\\

\haiku{Zal hij straks, als het,?}{schip uitvaart toespringen en}{de moorddaad plegen}\\

\haiku{tot eensklaps uit het:}{mistige grauw-grijs een}{vage schim opdoemt}\\

\haiku{O, wat zijn er veel,,!}{ouden bij en zwakken en}{half gebrekkigen}\\

\haiku{Het was een knappe,.}{jonge vrouw in de volle}{kracht van haar leven}\\

\haiku{De kleine krijgt een.}{soort van crisis en laat zich}{op de grond vallen}\\

\haiku{Met een angstgil neemt;}{ma hem in haar armen op}{en vlucht ermee weg}\\

\haiku{meten, die sinds een!}{eeuw op militair gebied}{werd afgelegd}\\

\haiku{en zei mij, dat het.}{streng verboden was enig licht}{te laten schijnen}\\

\haiku{Jean d'Aire, Jacques.}{et Pierre de Wissant en}{de drie anderen}\\

\haiku{Ik wou nog wel graag,,.}{verder maar begrijp dat het}{onmogelijk is}\\

\haiku{En spelend met zijn...}{stokje wenst hij ons verder}{goede reis toe}\\

\haiku{een stal, of schuur, of,.}{loods of boerenwoning die}{daar vroeger niet stond}\\

\haiku{En ook hier valt het,.}{mij op dat uiterlijk haast}{niets veranderd schijnt}\\

\haiku{Zo ziet men soms de,.}{straten van een stad in heel}{vroege morgenuren}\\

\haiku{- Diksmuide, zei hij,.}{langzaam de verrekijker}{aan zijn ogen zettend}\\

\haiku{De ruime schuur, de,,.}{hooizolders het wagenhok}{alles zit propvol}\\

\haiku{{\textquoteleft}'t Is vreemd{\textquoteright}, zegt hij, {\textquoteleft}!}{eindelijkzou er vandaag}{dan niets gebeuren}\\

\haiku{Zo reden wij een,.}{lange poos door absoluut}{verlaten oorden}\\

\haiku{- Gij zijt helemaal,,!}{uit de weg kapitein ge}{rijdt recht op D. af}\\

\haiku{Maar dat alles ging;}{zonder de minste uiting}{van smart of droefheid}\\

\haiku{Hij trachtte ons van,;}{het plan af te brengen ons}{te ontmoedigen}\\

\haiku{Ik weet niet hoeveel;}{de bevolking van Lens op}{dit ogenblik bedraagt}\\

\haiku{En wat 'n luxe in!}{een land dat heet verarmd en}{uitgeput te zijn}\\

\haiku{De Filosoof keert:}{zijn mistroostig gezicht naar}{ons om en antwoordt}\\

\haiku{Het sterke leven.}{van Frankrijk ligt begraven}{op zijn slagvelden}\\

\haiku{{\textquoteright} antwoordde mij een.}{Franse vrouw aan wie ik de}{opmerking maakte}\\

\haiku{Hoeveel honderden;}{karrewielen hebben over}{hem heen gereden}\\

\haiku{Hij stapte binnen, '.}{buigend ondert deurtje}{en wij volgden hem}\\

\haiku{Het meisje ging met,.}{vlugge vaste schreden naar}{de schenktafel toe}\\

\haiku{Gaston loopt langs de:}{wielen rond en al spoedig}{verneem ik zijn stem}\\

\haiku{De nieuwe band ligt.}{op en wij dalen naar het}{lieve stadje af}\\

\haiku{Hun debiet bestaat.}{uitsluitend te Parijs en}{in het buitenland}\\

\haiku{De grijze, leien.}{daakjes doezelen weg in}{geheimzinnigheid}\\

\haiku{Nogal wat kijkers,.}{op de drempels maar weinig}{of geen hoon en spot}\\

\haiku{Ik zie alleen de '.}{witte rookpluim bovent}{stationsgebouw}\\

\haiku{Wat is een auto,,,?}{ook de beste de duurste}{de meest volmaakte}\\

\haiku{- Gaston, da schijnen,.}{mij hier nogal geschikte}{meinschen zei  ik}\\

\haiku{Was het het vuur dat,...?}{mij bedwelmde of was het}{de geest van het huis}\\

\haiku{Geef ik wel aan de?}{rechte personen die ons}{geholpen hebben}\\

\haiku{Er hing een riempje,;}{te klepperen ergens in}{het bovenste bed}\\

\haiku{ook niet zo druk en.}{toch geeft het meer de indruk}{van een wereldstad}\\

\haiku{Zij schenen het land,.}{alleen meer kracht te schenken}{zijn woestheid delend}\\

\haiku{men is hier niet in,.}{een Spaans maar in een Engels}{landelijk hotel}\\

\haiku{Morgen weer tennis,.}{golf en tea en eten van de}{kok uit Engeland}\\

\haiku{Wat schijnen ze klein,,!}{op een afstand onder de}{geweldige rots}\\

\haiku{Omdat wij alleen!}{wensen weg te jagen en}{nooit zelf te wijken}\\

\haiku{Witte gestalten,.}{knielen biddend neer met het}{hoofd tegen de grond}\\

\haiku{{\textquoteright} met rollende r's:}{en buitengewoon-sterke}{klemtoon op de i}\\

\haiku{alleen het zwijgend.}{protest van hun totale}{onverschilligheid}\\

\haiku{De lange halzen,.}{bewegen zij vragend als}{zoekend heen en weer}\\

\haiku{- Dat zijn de bergen!}{en de sneeuwtoppen van de}{Atlas-keten}\\

\haiku{Wij komen v\'o\'or een,;}{hoge muur met prachtige}{Moorse ingangpoort}\\

\haiku{Beneden ons lag.}{de uitgestrekte stad in}{haar palmentuinen}\\

\haiku{Een luisterrijke;}{praal hier te midden van al}{dit vreemde bestaan}\\

\haiku{Een maanlandschap is!}{iets doods en dit leefde zo}{aangrijpend-intens}\\

\haiku{{\textquoteright} zei hij niet zonder,.}{trots alsof het iets was dat}{hem toebehoorde}\\

\haiku{Wat is dat alles!}{nog recent en wat lijkt het}{toch al lang voorbij}\\

\haiku{zeiden mijn dames,.}{opgetogen denkend mij}{daarmee te troosten}\\

\haiku{Een plannenmaker,, ':}{een illusionist een}{dromer int groot}\\

\haiku{Dat is iets, voer hij, -,.}{voort iets dat vanzelf spreekt maar}{lang nog niet alles}\\

\haiku{Er bestaat wel een!}{Trans-Siberian en}{een Trans-African}\\

\haiku{Zijn benen waren,.}{ietwat krom wat aan zijn gang}{iets waggelends gaf}\\

\haiku{Waarom bleef ik niet,?}{thuis waar ik het zo lekker}{en gezellig had}\\

\haiku{Ik wip uit mijn bed,,;}{trek de jaloezie\"en op}{gooi open de vensters}\\

\haiku{{\textquoteright} Il faut marcher, Et',:}{quand on veut fair des \'epates}{C'est peau d'z\'ebi}\\

\haiku{Toch lijken zij in.}{niets op wat wij bij ons van}{dat soort gewend zijn}\\

\haiku{Hun kleren hebben,.}{geen waarde zij zijn verslaafd}{aan drank noch tabak}\\

\haiku{Eenmaal geven is.}{meteen je rust verbeuren}{voor de hele reis}\\

\haiku{De man, die zulk een,!}{mooie klemtoon op dit laatste}{woord legt spreekt helaas}\\

\haiku{Kleiner worden de,.}{figuren op de oever}{die vaarwel juichen}\\

\haiku{Nu zit ik in de,.}{Zwitserse trein omringd door}{Zwitserse mensen}\\

\haiku{Hoededoos is het.}{meisje dat de hoed van de}{dame naar huis brengt}\\

\haiku{Het is de triomf?}{en de glorificatie}{van Hoededoos}\\

\haiku{het baantje v\'o\'or haar,.}{schoon opdat zij toch aan het}{doel zou geraken}\\

\haiku{Maar de thee zal dat.}{alles bij de anderen}{ook weer goedmaken}\\

\haiku{Jammer, dat de lucht, '.}{zo grijs is anders ware}{t schouwspel prachtig}\\

\haiku{Men hoort het aan de,.}{toon waarop zij de Franse}{woorden uitspreken}\\

\haiku{Om de strijdenden.}{van de niet-strijdenden}{te onderscheiden}\\

\haiku{Zoverre zijn we,,, '!}{helaas nog niet ook niet in}{t vrije Engeland}\\

\haiku{Of dit nu altijd {\textquoteleft}{\textquoteright}.}{evensympathiek aandoet is}{een andere zaak}\\

\haiku{Je gaat b.v. in een.}{winkel en bestelt er een}{of ander voorwerp}\\

\haiku{En ik keek rond of.}{ik nergens in de buurt een}{policeman zag}\\

\haiku{Dat treurig schouwspel.}{heeft het genoegen van mijn}{wandeling vergald}\\

\haiku{Terstond is daar een,:}{suppoost en zodra mijn vriend}{weer te voorschijn komt}\\

\haiku{Maeterlinck kocht,;}{het legde er nog een paar}{honderd duizend bij}\\

\haiku{Meestal mensen,.}{die nog maar ternauwernood}{wandelen kunnen}\\

\haiku{De ganse nacht, tot,.}{in de ochtenduren duurt de}{kruitverspilling voort}\\

\haiku{Hij nodigt ons uit;}{om zijn kerkje van binnen}{te bezichtigen}\\

\haiku{Nog smaller dan de,.}{eerste en afgronden om}{van te duizelen}\\

\haiku{k H\`e ik doar wa,.}{konijneten weest trekken}{langs de kanten}\\

\haiku{masco Joa ik, boas,.}{Van Poamel da es ienen veur}{ou os g'hem wilt}\\

\haiku{veur die troebels die.}{doar nou were zijn mee da}{wirkvolk in Gent}\\

\haiku{'t es 's morgens}{toch zuedoanig vroeg op te}{zijn en we lagen}\\

\haiku{Ge zij gulder ons.}{miesters en we moete}{wij g'huerzoamen}\\

\haiku{de baron Zie, Van,.}{Paemel ik zal kaart op}{tafel met u speel}\\

\haiku{de barones  (),,?}{heengaande Allons alzo}{verbleef  niewaar}\\

\haiku{En vergeet niet dat.}{ik Romanie maandag op}{de kasteel verwacht}\\

\haiku{de brigadier  ().}{tot de gendarme Blijf gij}{hier de wacht hou\^en}\\

\haiku{'k Weinsche dat 't ' '.}{leugens woarent gien da}{k hier zegge}\\

\haiku{moar d'r zijn nog}{ander en misschien beter}{hofstees te vinden}\\

\haiku{(Van Paemel schudt),?...}{krachtig het hoofd Wacht ne kier}{woar woaren we dan}\\

\haiku{zoo gelukki zijn}{en zien dat ik gelijk had}{van te handelen}\\

\haiku{masco 'k Kwam ik '.}{ne kier vroagen hoe datt}{goat mee Desir\'e}\\

\haiku{Allo, 't es goed,, '.}{leg ze doar moart zal ze}{wel iemand opeten}\\

\haiku{masco Wa da 'k,...}{goa doen om van te leven}{Menier de Paster}\\

\haiku{Hij wilde malgr\'e,.}{h\^en dat-e gij uek bij}{hem kwam Moeder}\\

\haiku{Met gedempte en) '?}{geheimzinnige stem Es}{t er glen belet}\\

\haiku{losgeloaten.}{worden die ulderen tijd}{uitgedoan h\^en}\\

\haiku{We 'n stoan nie,!}{ingeschreven op ulder}{boeken zeggen ze}\\

\haiku{die poait ou mee;}{wa zoete woorden uit de}{catechissemus}\\

\haiku{- Komt met nijdige ')!}{spotlach weer int midden}{van de keuken O}\\

\haiku{'k H\`e 't hem wel, ' ';}{gevroagd moar hijn hee}{t nie willen geen}\\

\haiku{Ge zilt mij hier iest,.}{al geen wa da g'h\`et tot de}{loaste cens}\\

\haiku{(Rosten Tjeef maakt een) '?}{gebaar van schrik Est gien}{woar da ge da peist}\\

\haiku{Peisde gij misschien '?}{da ze zij nien weet wat}{dat er gebeurd es}\\

\haiku{loat ons iest nog ',.}{n dreupelke pakken da}{geeft koeraze}\\

\haiku{Eerste toneel jan '}{en zulma Bij het opgaan}{vant gordijn zit}\\

\haiku{Alle menuten'.}{kan d ien of d'andere}{binnen komen}\\

\haiku{Boas, 't en es doar, '.}{nie da we sloapent}{es hier in de koamer}\\

\haiku{'k Wachtte tot dat, '.}{hij boven was en tons gijnk}{k uek noar boven}\\

\haiku{Hij 'n zal mij nie, '.}{antwoorden hijn zal noar}{mij nie kijken}\\

\haiku{of dat hij druemt... ' '!...}{t Wasn gerucht om d'r}{schouw van te worden}\\

\haiku{'n stikske van den.}{buikschotel mee eirdappels}{en saveu\"en}\\

\haiku{de buurvrouw Goa gij,... '.}{moar Zulmatjek komme}{seffens achter}\\

\haiku{Eindelijk neemt hij.}{een bord van tafel en gaat}{ermee om eten}\\

\haiku{cloet  (met holle,)?}{doffe stem H\`et-e nie}{wat te drijnken}\\

\haiku{(Beiden achtergrond)(,;}{af  de notaris}{links op rookt een pijp}\\

\haiku{stien  (als boven) ',.}{n Beetse te veel natte}{Menier de Juge}\\

\haiku{Uleken droogt met een.}{handdoek glazen af achter}{de schenktafel}\\

\haiku{(Uleken lacht hardop)?}{de notaris Woar es}{de juge dan}\\

\haiku{Dat 'n kan hem nie, '.}{schelen os de stamenee}{moar goedn marcheert}\\

\haiku{Ge moet weten, dat.}{die jonge juffer mij een}{weinig intrigeert}\\

\haiku{de kanterik Joa,,,.}{joa joa Menier den Docteur}{hij was zeker stout}\\

\haiku{de wrijver Menier, ' ';}{den Docteurt was azuen}{koartje noar den tienen}\\

\haiku{We zaten wij mee ',:}{twie toafels aant koarten in}{mijn hirbirge}\\

\haiku{vrouw roetjes  (met)!}{ten hemel opgeheven}{handen Ha jongens}\\

\haiku{vrouw roetjes  (naar)!?}{Grondnagel wijzend Wulder}{mee hem overienkomen}\\

\haiku{packal  (woedend)! '!}{Ocht es mee al die pruts}{van da Westvloams}\\

\haiku{(Muijshondt verschuift zich,.}{derwijze dat hij maar half}{op de stoel neerzit}\\

\haiku{Ik vraag niets beter.}{dan alles ten beste te}{zien eindigen}\\

\haiku{We zillen d'r nog,.}{ne kier goed over peizen en}{dan besluiten}\\

\haiku{'k Ben koperoal ',}{vant ieste rezement}{Doar ben ik kontent}\\

\haiku{stoute threse  (,)!}{woedend de vuisten op haar}{heupen Verdome}\\

\haiku{vrouw beert  (snelt naar,)!}{Maria gevolgd door Lisatje}{Verdomde slonse}\\

\haiku{Enkele mannen.}{en vrouwen vluchten rechts en}{links in de huizen}\\

\haiku{(Hij speelt en weldra)'}{zingen allen mee  k}{Ben koperoal van}\\

\haiku{lisatje  (kijkt naar) '.}{het raam Datn zoe mij nie}{verwonderen}\\

\haiku{(kijkt met gespannen '),?...}{aandacht doort raam Moar wie}{es dat doar dien hiere}\\

\haiku{Aftrekkend geraas)(;}{in de straat  slimke snoeck}{opgewonden}\\

\haiku{slimke snoeck  (komt),!}{dreigend op Reus af Veur ou}{niet nondedzju}\\

\haiku{As w'hem nie 'n h\^an ' ' '.}{tn zoe mijt leven}{nie mier weird zijn}\\

\haiku{(Gelach)  lisatje() ',.}{met het glas bij Beert As}{t ou blieft voader}\\

\haiku{(Gelach)  lisatje(),, '.}{schenkend Joa joak ken}{ou toeren wel}\\

\haiku{nen ambachtsman En '!}{niemandn weet er wa dat}{ik kan Moar habil}\\

\haiku{(barst plotseling in),!}{woedetranen uit en da}{we sakerdzju}\\

\haiku{(tot Witte Manse),, '.}{Ala toe Manske  geef gij}{onsn pijntje}\\

\haiku{d'r zaten gister.}{achternoene86 wel twintig}{honden achter}\\

\haiku{donder de beul, klod()!}{de vos en smuik vertriest}{tegelijk O}\\

\haiku{da ze zij ons nie ',.}{amoaln veracht hier in}{de Zijstroate}\\

\haiku{We zillen d'r van.}{donderdag oavend af al}{wa goan vangen}\\

\haiku{mijn huefd gerust, ' '.}{ofk verzeker ou dat}{t schief zal zitten}\\

\haiku{Hij sloat heur dued!... ()}{Allen vliegen naar de deur}{Vierde bedrijf}\\

\haiku{Aan beide kanten,.}{van de deur een gendarme}{die er de wacht houdt}\\

\haiku{reus balduk Keunt ge ' ',?}{t ons uek nien beetse}{zouten brugadier}\\

\haiku{reus balduk  (tot),,.}{Witte Manse Kom Manske}{kom gij bij mij}\\

\haiku{(Beert en zijn vrouw met)}{nog enkele anderen}{hollend rechts af}\\

\haiku{as hij Maria nog, '}{pebliek mishandelde moar}{in huis est toch}\\

\haiku{'k H\`e ou altijd, '}{zue geirn gezien ent zoe}{mij toch zue spijten}\\

\haiku{We zoen doar zue schuen,.}{en zue gelukkig geweund}{h\^en in Frankrijk}\\

\haiku{slimke snoeck  (tot)'?}{Reus Wa kan ou da schelen}{woar da z hangt}\\

\haiku{reus balduk  (komt)!}{met gebalde vuisten op}{Slimke Snoeck af O}\\

\haiku{Moeder, midden in,.}{de kamer naait iets vast aan}{Vaders binnenzak}\\

\haiku{hee...  guust  (met)...}{een blik op Philomene}{En mee spoaren}\\

\haiku{Buufstikken zillen,.}{we ginter eten zuevele}{of da we willen}\\

\haiku{We komen ulder,.}{nog ne kier bezoeken ier}{da ge wiggoat}\\

\haiku{(tot Rozeken) En,,...}{gij jongedochter goat er}{noartoe om te}\\

\haiku{pastoor  (doet haar),.}{weer zitten Blijf moar zitten}{gruetmoeder}\\

\haiku{Moeder neemt het hem.}{ruw uit de armen en geeft}{het aan Feelken}\\

\haiku{De burgemeester,.}{geeft hem vuur en hij steekt aan}{genoeglijk smakkend}\\

\haiku{Waar...,  moeder Och, '.}{Hiere schiedt er toch uitt es}{veel te triestig}\\

\haiku{moeder  (droevig),,:}{Ha moar ieffreiwe we zijn}{wij katholieken}\\

\haiku{ik zilde en da'.}{oarm schoapken die noar d}{helle zoe goan}\\

\haiku{n Kieken woar dat '!}{er nog gien halve vlere}{van gesneenn was}\\

\haiku{binst da we wulder.}{hier van den honger liggen}{te craveren}\\

\haiku{moeder  (huilend),,?}{Och Hiere voader wa goan ze}{toch mee hem doen}\\

\haiku{Gruetmoeder es.}{schouw van ou en ik voele}{mij toch zue ziek}\\

\haiku{en zue lank as da'.}{g hier zijt zal ik schuene}{veur ou zurgen}\\

\haiku{Meent gij misschien dat '?}{ikn wet voor u apart kan}{laten maken}\\

\haiku{Frankrijk staat hoog in,,;}{mijn waardering h\'e\'el hoog daar}{kom ik rond voor uit}\\

\haiku{(Af)  else  (),,...}{rent naar de deur O maar dat}{moet ik toch ook}\\

\haiku{Dat wij elkaar in!}{zulke omstandigheden}{moeten terugzien}\\

\haiku{Ons werk nu, mijn werk,.}{hier zal voortaan zijn dat weer}{goed te maken}\\

\haiku{jan bron Of mijn vrouw?}{en mijn dochter hier wel in}{veiligheid zijn}\\

\haiku{Denkt aan hen die op.}{de slagvelden voor onze}{vrijheid sterven}\\

\haiku{jan bron  (met een);}{vuistslag op de tafel Maar}{ik wil niet slapen}\\

\haiku{de mensen zouden.}{die kerels verscheurd hebben}{hadden ze gedurfd}\\

\haiku{eerste hollandse( ').}{soldaat  dadelijk bij}{t hek Paspoort}\\

\haiku{Eerste Belgische.}{soldaat patrouilleert lustig}{rokend heen en weer}\\

\haiku{eerste belgische()...!}{soldaat  verwonderd Tiens}{en ge spreekt Olands}\\

\haiku{Geheimzinnig) Ik.}{denk dat het een verbannen}{activist is}\\

\haiku{eerste belgische ',...}{soldaat Dat esn ander}{geval meniere}\\

\haiku{van veerdeghem()?}{tot Tweede Belgische}{soldaat Ook een}\\

\haiku{eerste hollandse() '!}{soldaat  verontwaardigd}{Een gek ist}\\

\haiku{In de achtergrond.}{een bordes met toegang tot}{een zonnige tuin}\\

\haiku{ik... Op dit ogenblik ':}{hoort men achtert toneel}{de stem van Dora}\\

\haiku{Dora zal er niet...}{tegen kunnen en ze zal}{niet toegeven}\\

\haiku{Veertiende toneel,()...}{castro dora  dora}{hijgend Vader}\\

\haiku{dat mag ik wel, na...!!}{alles wat u mij war u}{ons hebt aangedaan}\\

\haiku{om je de waarheid... '...}{te zeggen ik het liever}{n potje bier}\\

\haiku{nou wordt ie spinnig,...}{nou staat ie te trippele}{van kwajigheid}\\

\haiku{(terugkerend) maar,...}{je belooft me Raaks dat je}{niet aan Castro zegt}\\

\haiku{raaks O\'ok 'n vraag... as...}{iemand d'r uitziet als de}{dood van ieperen}\\

\haiku{Je moet 's nachts es ', '!}{n grokkie nemen dan slaap}{je alsn marmot}\\

\haiku{daar is Teunisse......!}{met de tuinstoelen en of}{uwe effies komt}\\

\haiku{w\'at heb je met die......}{brief voor meneer Daal gedaan}{die ik je laatst gaf}\\

\haiku{mevrouw  (die al,)}{die tijd als een standbeeld is}{blijven staan langzaam}\\

\haiku{D'r is maar \'e\'en man...,}{die hier helpen kan en dat}{is dokter Coertens}\\

\haiku{ik ken 't niet meer...()!!}{met gebalde vuisten}{Allemachtig}\\

\haiku{Castro is voor zijn)}{schildersezel gaan staan om niet}{gezien te worden}\\

\haiku{en toch ben ik 't... ', '...}{Zie ik eruit alsn dief}{en toch ben ikt}\\

\haiku{U wordt getroffen '......}{int liefste wat u hebt}{in uw kinderen}\\

\haiku{Misschien heb ik d\'a\'arom!}{wel zo veel gewerkt in m'n}{leven om die angst}\\

\haiku{zoveel dingen... die '... '}{men verschrikkelijke}{angst hebben bezorgd}\\

\haiku{Vijftiende toneel,()?!}{mevrouw castro  mevrouw}{jubelend Ja}\\

\haiku{Dan wendt zij het hoofd - -:}{weer terug pauze weer naar}{de tafel ziende}\\

\haiku{Sientje  (Wenkt haar).}{bij haar te komen Sientje}{komt naderbij}\\

\haiku{Juffrouw Elvire......,...}{slaapt toch niet meer hier wat}{mevrouw Nee dokter}\\

\haiku{elvire Kom an....}{Doortje wat doen die bloemen}{je nu voor kwaad}\\

\haiku{Jacques, in al de,......}{tijd dat je er niet was ben}{je bij me geweest}\\

\haiku{Dat moet het toch wel............?}{zijn Wat wat dacht jij dat er}{tussen ons stond}\\

\haiku{dan... nog es op 't,............?}{orgel zoals vroeger wil}{je dat voor me doen}\\

\haiku{blijft staan, kijkt als om.}{hulp werktuiglijk om en ziet}{Coertens in de deur}\\

\haiku{coertens Kom tot je...(,)}{zelf  castro  hem van}{zich afwerpend wild}\\

\haiku{castro  (laat haar)!}{ineens los en op Coertens}{afgaande Zwijg jij}\\

\haiku{op 'n stoel vallend)...}{Twintig jaar lang heb ik met}{dat kwaad gevochten}\\

\haiku{Existe-t-il dans?}{le monde vingt personnes}{qui la connaissent}\\

\haiku{Maeterlinck steekt.}{de vinger in de hoogte}{en fluit een deuntje}\\

\haiku{een drama waarin!}{werkelijkheid en ideaal}{\'e\'en zullen worden}\\

\haiku{Gedurende twee.}{volle jaren leed hij er}{honger en gebrek}\\

\haiku{{\textquoteright}, zegt de arme vrouw, {\textquoteleft},,{\textquoteright}.}{toch niet bewaar hem goed en}{wacht maar je zult zien}\\

\haiku{en meteen stelt hij,.}{hem voor aan Lacroix die het}{werk zal uitgeven}\\

\haiku{Gedurende acht.}{volle maanden werkt hij aan}{zijn reusachtig plan}\\

\haiku{maar  telkens toch '.}{dooft de grijze massat}{heerlijk licht weer uit}\\

\haiku{Die {\textquoteleft}charme{\textquoteright} werkte,:}{z\'o algemeen en z\'o sterk}{dat om het even wie}\\

\haiku{We'n keunen wij toch,.}{moar huel slecht roapen}{missen denkt het mij}\\

\haiku{De wandeling in ' '?}{t stadspark oft bezoek}{bij Westinghouse}\\

\haiku{mond en kin - naar rechts,,}{of links trekken terwijl hij}{sprak al naar gelang}\\

\haiku{Naast de stuurman zit,.}{een kerel gewapend met}{een enorme spreekbuis}\\

\haiku{Hun beleefdheid, hun.}{generositeit waren}{er geen des harten}\\

\haiku{Veel vreemdelingen,, '}{zei ik dies zomers bij}{ons buiten kwamen}\\

\haiku{De gevoelens, die,.}{zij ondergaat staan op haar}{gezicht te lezen}\\

\haiku{{\textquoteleft}Populierenhout{\textquoteright}:}{is goed lepelhout en wacht}{dan even en vervolgt}\\

\haiku{Het meisje was het,;}{kleinste het liefste en het}{blondste van de drie}\\

\haiku{Eens was hij, hoewel,.}{zwaar verkouden per rijwiel}{naar Gent gereden}\\

\haiku{Ik moet alles goed,.}{herdenken want het is zo}{plotseling gegaan}\\

\haiku{De slippen van zijn {\textquoteleft}{\textquoteright}.}{caban woeien lichtkens rechts}{en links van hem af}\\

\haiku{gisterenavond op, {\textquoteleft}{\textquoteright};}{zijn rijwiel met zijncaban}{en zijn rond hoedje}\\

\subsection{Uit: Verzameld werk. Deel 7}

\haiku{Maar de huldiging,,.}{in Antwerpen was hoe dan}{ook toch doorgegaan}\\

\haiku{Over de volgende:}{twee dramatische schetsen}{kunnen we kort zijn}\\

\haiku{De verhollandsing}{van het oorspronkelijk in}{dialect geschreven}\\

\haiku{De Nederlandsche.}{letterkunde in Belgi\"e}{sedert 1830 door Edw}\\

\haiku{Jean Tousseul, in Groot,,,,-;}{Nederland jg. XVIII 1920}{d. I p. 228235}\\

\haiku{Want laten we 't:}{in godsnaam niet vergeten}{of niet loochenen}\\

\haiku{Deze rubriek is '.}{en blijft dan ook het lijfstuk}{vant courantje}\\

\haiku{Voor eigenlijke.}{liefde van haar man is zij}{minder verlegen}\\

\haiku{Wat lijkt de Vlaamse,!}{daarentegen doorgaans flink}{vrolijk en gezond}\\

\haiku{'t Is om er bij, ':}{te huilen ent is om}{er voor te bidden}\\

\haiku{Het ene volk amuseert.}{zich veel meer en veel beter}{dan het andere}\\

\haiku{Wij, kinderen van,.}{de tegenwoordige tijd}{hebben niets misdaan}\\

\haiku{En ook de grote, - -:}{overledene Gezelle}{werd niet vergeten}\\

\haiku{'s Nachts trekken zij.}{erop uit en roven wat}{ze krijgen kunnen}\\

\haiku{{\textquoteright} De lezer zelf schudt:}{droevig het hoofd en murmelt}{met benepen hart}\\

\haiku{Beide gezinnen,;}{zijn onder elkaar bevriend}{meer zelfs dan bevriend}\\

\haiku{hij zat al nevens '!}{Nardje op den bok ent}{gespan was in gang}\\

\haiku{Zegene u de,.}{Alderhoogste want de navond}{is nabij komt bij}\\

\haiku{Zal Zola zich bij '?}{t vonnis neerleggen en}{de zaak opgeven}\\

\haiku{De wind, die zoo hoog.}{voorbijtrok had misschien over}{de heide gewaaid}\\

\haiku{Hij wil het mooie, het, '.}{rijke het schitterende}{vant uiterlijk}\\

\haiku{Hare voorpooten}{trokken soms aan een draadje}{en haar koppeken}\\

\haiku{De taal van Herman.}{Teirlinck lijkt ook heel sterk}{op die van Streuvels}\\

\haiku{Hij sprak eerst in het '.}{Vlaams en vertaalde daarna}{zijn toast int Frans}\\

\haiku{Ongelooflijk sterk;}{zijn haar sympathie\"en en}{antipathie\"en}\\

\haiku{maar het kan ook best,.}{gebeuren dat het heel iets}{anders betekent}\\

\haiku{Ik heb het hier reeds,:}{meer gezegd en ik moet het}{nog eens herhalen}\\

\haiku{Minnehandel door,,.}{Stijn Streuvels ~ 2 dln L.J.}{Veen Amsterdam}\\

\haiku{Hij duwt het weg maar ',.}{t komt terug hoe langer}{hoe hardnekkiger}\\

\haiku{- We kunnen er niets,,.}{aan doen Klara die dingen}{hangen in de lucht}\\

\haiku{te Rotterdam, en ().}{Vlaamsche BoekhandelLeo J.}{Krijn te Brussel}\\

\haiku{C'est son b\^aton qui}{repart le premier et fait}{le premier pas. Il}\\

\haiku{Het heet dat ze doof,.}{is maar alleen voor wat ze}{niet graag horen wil}\\

\haiku{Ook niet in Frankrijk.}{en in de grote steden}{van het buitenland}\\

\haiku{Het eerste zag het;}{licht met een voorrede van}{Emile Verhaeren}\\

\haiku{Ook wij weten niet.}{waar en hoe wij ons laatste}{lied zullen zingen}\\

\haiku{De vruchten van de.}{Natuur kunnen tijdelijk}{vernietigd worden}\\

\haiku{Hij is ook z\'o groot,.}{en hij staat z\'o hoog dat men}{hem maar moet lezen}\\

\haiku{Kort v\'o\'or het einde.}{ging Maeterlinck hem daar}{nog even opzoeken}\\

\haiku{Hij drukte 't niet,;}{verder in woorden uit wat}{je daar al niet kon}\\

\haiku{Wij kijken nog naar ',,;}{t Oosten wel zwaar bedroefd}{maar niet wanhopend}\\

\haiku{Jacques Muraille,,,.}{steenbikker is getrouwd met}{Marie een werkvrouw}\\

\haiku{Zijn leven is van.}{een doodse verlatenheid}{en melancholie}\\

\haiku{Zelden heb ik een.}{werk van groter en dieper}{emotie gelezen}\\

\haiku{Il restait un peu:.}{de soupe \`a l'h\^otesse}{je le lui offris}\\

\haiku{Dit geeft haar boek iets,.}{zeer bekorends van warme}{schone zuiverheid}\\

\haiku{daarvoor was ook de.}{omvang van het werk niet breed}{genoeg opgezet}\\

\haiku{Hij riep ons allen,;}{om zich heen wij moesten kijken}{en bewonderen}\\

\haiku{Het was een feest van,.}{elke dag een jubelen}{van ieder ogenblik}\\

\haiku{voor mij l\'e\'eft hij, als,!}{Vlaanderen zelf en zal hij}{blijven leven}\\

\haiku{Doch sprekender dan.}{alle beweringen zijn}{enkele cijfers}\\

\haiku{Hij vliegt u ijzig;}{in het aangezicht en poogt}{u te verblinden}\\

\haiku{Het volk wantrouwt hen.}{en men kan het volk niet gans}{ongelijk geven}\\

\haiku{Het heeft bij hem veel;}{meer weg van een houding dan}{van een overtuiging}\\

\haiku{Laat ik nu maar eens.}{heel oprecht en desnoods heel}{onbescheiden zijn}\\

\haiku{De drie anderen;}{kon ik slechts op de rug of}{van terzijde zien}\\

\haiku{De lijkstoet en de.}{rijtuigen staan onder de}{hoge bomen stil}\\

\haiku{en na het wonder:}{leek nu alles zo gewoon}{en onbelangrijk}\\

\haiku{en daar was ik aan.}{mijn tweede weddenschap van}{vijfduizend frank}\\

\haiku{Per auto komt men,.}{in plaatsen en streken waar}{men anders niet komt}\\

\haiku{Laten wij samen,.}{hopen dat die dag spoedig}{zal aanbreken}\\

\haiku{Een groot rumoer kwam.}{aangedreund van verre in}{de donkere nacht}\\

\haiku{Slechte berichten,.}{geloven zij niet willen}{ze niet geloven}\\

\haiku{ongelukkig door;}{wat zij onherroepelijk}{verloren hebben}\\

\haiku{Ik hoef er niets aan,:}{toe te voegen ik mag er}{veel uit weglaten}\\

\haiku{Het merk {\textquoteleft}Volkenbond{\textquoteright}...}{Gisteren is de man naar}{mij toegekomen}\\

\haiku{Ik heb het van 't.}{begin tot het einde met}{aandacht gelezen}\\

\haiku{Daar schoof ik een groen:}{gordijn weg en hij zag de}{twee ruime vakken}\\

\haiku{De zogenaamde '.}{vrede kwam en ik keerde}{int land terug}\\

\haiku{een grote bom viel.}{neer en de brug vloog in een}{rookwolk uit elkaar}\\

\haiku{En wat zij zijn op, '.}{h\'un dorp zijn huns gelijken}{overt hele land}\\

\haiku{en Duitsland met een.}{deel van Vlaanderen aan de}{andere zijde}\\

\haiku{Aan het lijden van,.}{de dieren dachten zij geen}{ogenblik die mensen}\\

\haiku{blijven doorvechten?}{of vrij en vreedzaam weer naar}{huis toe mogen gaan}\\

\haiku{Op wie wacht ze, en, {\textquoteleft}{\textquoteright} {\textquoteleft}{\textquoteright}?}{wie moet daar vandaan komen}{eenhij of eenzij}\\

\haiku{- De oorlog verslindt.}{elke dag duizenden en}{duizenden mannen}\\

\haiku{De Belg zijn spotlach,,.}{zijn stille schimp ontnemen}{is onmogelijk}\\

\haiku{en aan de verre,.}{einder nog een boerderij}{en dat is alles}\\

\haiku{D\'at was het wat hij!}{van zover en met zoveel}{zorg had meegebracht}\\

\haiku{Je zit er lang niet.}{onaardig en je eet er}{bepaald heel lekker}\\

\haiku{Tot haar verbazing.}{stond hij zelf reeds kant en klaar}{om te vertrekken}\\

\haiku{- Hebben ze onze!}{soep reeds uitgeschept v\'o\'or wij}{aan tafel zaten}\\

\haiku{- Wat ik ervan denk...... -...?}{herhaalde hij heel langzaam}{wat ik ervan denk}\\

\haiku{'t Diner was fijn,,.}{die avond en de wijnkelder}{werd aangesproken}\\

\haiku{Thans behoort men in.}{beschaafd gezelschap over grint}{en zand te praten}\\

\haiku{er was geen plaats voor,!}{hen omdat er niet genoeg}{geld voor hen was}\\

\haiku{Het roodborstje zingt;}{een melancholisch-kort}{vooisje en vliegt weg}\\

\haiku{Ik lachte ook, en,.}{grijnsde en wachtte wat nu}{verder komen zou}\\

\haiku{Zij voelen niets,  , {\textquoteleft}.}{naar zij beweren voor het}{begripvaderland}\\

\haiku{Zij zullen er niet,;}{binnenkomen nog in geen}{jaren en jaren}\\

\haiku{Deze keert zich om:}{en stelt de vraag aan een van}{zijn kameraden}\\

\haiku{Mieux on conna{\^\i}t la,{\textquoteright};}{vie plus on aime son chien}{zegt een Frans spreekwoord}\\

\haiku{Wat Deulin schreef is,.}{trouwens zo weinig zo veel}{te weinig bekend}\\

\haiku{Fran\c{c}oise was een,.}{vrome vrouw in de vrees des}{Heren opgevoed}\\

\haiku{Gillette, onder,.}{Belzebuths invloed gehuwd}{was niet gelukkig}\\

\haiku{{\textquoteleft}Ik ben niet rijk, maar.}{wil toch ook iets doen voor die}{ongelukkigen}\\

\haiku{Eenieder snakt naar.}{vrede en nagenoeg de}{ganse wereld vecht}\\

\haiku{u, vrouwen zoudt er.}{glimlachend mee trippelen}{en koketteren}\\

\haiku{Een dezer dagen.}{ontving ik het bezoek van}{een aardig jongmens}\\

\haiku{Het staat zo ver van.}{ons af en toch voelen wij}{het zo heel dichtbij}\\

\haiku{wat een geluk, dat!}{we die akelige dingen}{nu nog hebben}\\

\haiku{Er werd waarachtig;}{nog een derde maal tegen}{die open deur geklopt}\\

\haiku{- Maar, meneer, wat ik,!}{schrijf is toch de waarheid de}{zuivere waarheid}\\

\haiku{- Dat komt van buiten,, -,;}{zei zij van de schaduw uit}{die mooie hoge boom}\\

\haiku{De {\textquoteleft}zangeres{\textquoteright} ging ';}{echter eenigszins bedremmeld}{naart tooneel terug}\\

\haiku{Die zal er ook wel,!}{voor te vinden zijn of de}{tooneel-criticus}\\

\haiku{en in welk zalig?}{land van weelde en vrede}{is het dan gebeurd}\\

\haiku{Maar dat lijkt alles,,.}{nu zo ver zo heel h\'e\'el ver}{en lang geleden}\\

\haiku{- Voor wat moeten wij?}{onze kinderen na de}{oorlog opleiden}\\

\haiku{daar zijn de knoeiers;}{die heimelijk handel met}{de vijand drijven}\\

\haiku{veel en lekker eten,,;}{veel en lekker drinken veel}{en lekker roken}\\

\haiku{- Een zilverbon van,.}{fl. 2.50 herhaalde ik met}{vaste overtuiging}\\

\haiku{de stad gonst in de,.}{verte alsof ze zich van}{mij verwijderd had}\\

\haiku{- ik was juist van plan.}{te trouwen en mij ergens}{vast te vestigen}\\

\haiku{geef ze mij mee in,.}{een pakje in mijn doodkist}{zei Guerliche}\\

\haiku{Zij vinden er, als,.}{ik het zo mag uitdrukken}{geen waar voor hun geld}\\

\haiku{kerstdag komen ze,.}{vast en zeker bij elkaar}{vermaande hij nog}\\

\haiku{en ik begreep niet.}{goed waar hij zijn borstels wel}{vandaan kon halen}\\

\haiku{Er staan al heel wat.}{namen op en het aantal}{neemt dagelijks toe}\\

\haiku{- Trek er spoedig een,,.}{van aan amice mitsgaders}{andere laarzen}\\

\haiku{later, l\'ater, als,.}{ik zijn werk zijn plicht geheel}{volbracht zal hebben}\\

\haiku{zij konden er niets '.}{mee uitvoeren en hebben}{t hem gelaten}\\

\haiku{Goddank dat ze zijn!}{afgebeulde paardje hem}{gelaten hebben}\\

\haiku{Hij aarzelt even, drijft.}{zijn paard met ploeg de richting}{van de steenweg uit}\\

\haiku{- Gereed zijn, zulle,.}{of anders wordt ge met de}{gendarmen gehaald}\\

\haiku{Wij reden samen,,.}{de vreemdeling en ik per}{fiets door Gelderland}\\

\haiku{- Hoe komt het toch, vroeg, -?}{hij dat men in Holland geen}{omelet kan maken}\\

\haiku{Zeg haar dat zij er;}{precies dezelfde schotels}{mee moet bereiden}\\

\haiku{Het was Kerstavond en.}{een opgewekte stemming}{leefde in het dorp}\\

\haiku{Het was of er een.}{lamheid in mijn geest en in}{mijn hand en ogen kwam}\\

\haiku{- Pardon, meneer, maar.}{ik kan niet werken als er}{iemand naast mij staat}\\

\haiku{Ik meende dat dit,,}{weerzien ondanks alles wat}{gebeurd was voor mij}\\

\haiku{Ik vroeg met enige.}{verwondering of dat de}{nieuwe pasmunt was}\\

\haiku{De achterblijvers,;}{mopperen wat het geeft een}{beetje jaloezie}\\

\haiku{Zij leven nu al,,.}{zolang al zoveel jaren}{z\'onder te werken}\\

\haiku{En meteen gaf hij.}{bevel mijn molenaar de}{boeien om te slaan}\\

\haiku{zei mijn molenaar,.}{toen ze zowat twee uur lang}{gelopen hadden}\\

\haiku{dat zo'n stumperd de}{eindeloze tocht van ruim}{vijf uren die wij v\'o\'or}\\

\haiku{Ik spring van mijn plaats,,.}{op grijp naar mijn revolver}{ruk de voordeur open}\\

\haiku{Niet alleen hebben;}{zij al het vee en al de}{paarden meegesleept}\\

\haiku{Het is een oude,.}{boer met gebogen schouders}{en grijze haren}\\

\haiku{Ik vlucht weg, op mijn,,.}{rijwiel geel van slijk in de}{lichte manenacht}\\

\haiku{Hij maakt een gebaar,;}{om te beduiden dat het}{afgelopen is}\\

\haiku{Zij kunnen maar niet,,}{begrijpen dat hij nog leeft}{en soms zijn ze bang}\\

\haiku{Zij houdt zich heel, h\'e\'el,;}{stil om zijn weldadige}{rust niet te storen}\\

\haiku{Geluidloos komen,;}{zij binnen vragen eventjes}{of alles goed is}\\

\haiku{en daarop vertrekt;}{het jonge meisje zoals}{zij gekomen is}\\

\haiku{Zij werkten aan de,.}{wegen onder toezicht van}{Franse soldaten}\\

\haiku{Ik hoor dat ze zich -!}{ginds vervelen anders wel}{een heel goed teken}\\

\haiku{uitstekend!) en dat.}{ze er allemaal meer dan}{genoeg van hebben}\\

\haiku{Terstond, gehoorzaam,.}{van aard als ik ben rijs ik}{voorzichtig overeind}\\

\haiku{Waar zitten ze nu,,,?}{die helden die bevende}{blatende schapen}\\

\haiku{Zij keken over 't,;}{muurtje heen strak en stijf als}{waren zij van ijs}\\

\haiku{Ik haal er slechts een,.}{drietal aan die mij tot een}{obsessie worden}\\

\haiku{- Nog niet zo zeer door,;}{de oorlog sprak eindelijk}{de burgemeester}\\

\haiku{- Hier zie, meniere,,!}{vlak achter mijn hof tussen}{die twee wilgen doar}\\

\haiku{Dat alles had ik!}{reeds opgegeten en moest}{ik nog eens opeten}\\

\haiku{zoals de oude;}{Aristophanes het eeuwen}{geleden hoorde}\\

\haiku{zei een werkman, die,,.}{glimlachend in hemdsmouwen}{in zijn deurgat stond}\\

\haiku{Hij zou verbaasd zijn,.}{als hij zien kon hoe ze zich}{ontwikkeld hebben}\\

\haiku{La marraine, son,;}{poupon sur le bras pleure}{\`a chaudes larmes}\\

\haiku{Il n'y avait pas de,,:}{malheur pas de sang pas de}{mort dans sa maison}\\

\haiku{Quant au fond m\^eme,.}{de mon article je n'ai}{riep \`a y changer}\\

\haiku{l'a{\^\i}n\'ee, Rosalie,,}{si merveilleusement dou\'ee}{morte au d\'ebut}\\

\haiku{extr\^eme dans ses.}{affections comme dans}{ses antipathies}\\

\haiku{Et partout dans les:}{beaux p\^aturages se meuvent}{les riches troupeaux}\\

\haiku{Je ne savais et,,.}{quelquefois je souffrais}{de ne pas savoir}\\

\haiku{Elle est blanche.}{avec une plinthe noire et}{un toit de chaume}\\

\haiku{C'est vraiment par amour.}{pour la France qu'on y va}{et qu'on y donne}\\

\haiku{{\textquoteleft}D\`es que j'arrive,.}{en Flandre j'enrichis mon}{vocabulaire}\\

\haiku{Il \'eprouvait de la.}{peine \`a s'assimiler}{les vrais sons flamands}\\

\haiku{Je lui montrais mon.}{pays et lui m'apprenait \`a}{conna{\^\i}tre le sien}\\

\haiku{Overigens, hij dacht,.}{er ook niet aan daarover zijn}{beklag te maken}\\

\haiku{Hij maakte van een,:}{seconde rust gebruik om}{schreiend te roepen}\\

\haiku{Ik alleen had aan.}{al die uitgelatenheid}{geen deel genomen}\\

\haiku{- Een beetje vlugger,,.}{dan koetsier wij brengen den}{stoet in verwarring}\\

\haiku{Hij glimlachte even,,.}{en zijn glimlach liet zeer mooie}{witte tanden zien}\\

\haiku{- Als ik terugkom.}{zal de kudde reeds klaar staan}{om te vertrekken}\\

\haiku{Zwijgend, bedwelmd door,.}{rampgevoelens heb ik haar}{we\^er thuis gebracht}\\

\haiku{Zij was te goed, te.}{zacht voor deze wereld van}{smart en ellende}\\

\haiku{Ik begreep dat ik,.}{niet k\'on uitscheiden omdat}{ik nog niet dood was}\\

\haiku{De Schilder liep maar;}{aldoor babbelend met het}{brunetje vooruit}\\

\haiku{Hij sliep dien laatsten,.}{nacht in gelukzalige}{vergetelheid}\\

\haiku{De menigte, die,.}{haar prooi zag ontsnappen drong}{brieschend om ons heen}\\

\haiku{En ik liep verder,.}{en was al spoedig weer de}{brem vergeten}\\

\haiku{In een hoek lag een,.}{grauw linnen zakje flink met}{een touw dichtgesnoerd}\\

\haiku{Hij was een beetje,.}{schrokkig maar verder had}{hij geen gebreken}\\

\haiku{De meiden sloegen;}{de armen ten hemel en}{huilden wanhopig}\\

\haiku{Zij vertrokken op.}{een vroegen ochtend en von}{Varken was er bij}\\

\haiku{Ik kan niet bouwen.}{op den wankelen bodem}{eener vergissing}\\

\haiku{Ge zie wel, e-woar, ' '!}{datt nie neudign was}{van azeu te vloeken}\\

\haiku{halsstarrig v\'o\'or de.}{auto uit en is daar niet}{vandaan te krijgen}\\

\haiku{Haar zoon, - het kalf - is,.}{dat niet minder maar op een}{andere wijze}\\

\haiku{Het kind lag in 't ',.}{zand naastt rechter voorwiel}{onbewegelijk}\\

\haiku{Eigenlijk waren {\textquoteleft}{\textquoteright}.}{we allebeien d\'efaut}{ik zoowel als Andr\'e}\\

\haiku{Wij reden dus, op,.}{een prachtigen herfstdag van}{Yonkers naar New York}\\

\haiku{En nu, kindlief, heb.}{ik honger en wou wel heel}{graag iets gebruiken}\\

\haiku{- En krijg ik nu ook?}{zoo'n wagentje met vier of}{zes kaboutertjes}\\

\haiku{Een droom Helder en...}{duidelijk heb ik mijn droom}{gezien en doorleefd}\\

\haiku{Concreet en compleet:}{had hij mijn toekomstige}{psyche voorgevoeld}\\

\haiku{Meheus O neen neen ', '...}{t Meneere de Notaris}{t en zal nie zijn}\\

\haiku{Monsieur L\'eonce,.}{ces deux mots signifient la}{m\^eme chose}\\

\haiku{Dias Vinden ze daar,?}{geen koolmijnen in da land}{Meneer L\'eonce}\\

\haiku{Mr L\'eonce Mulle.}{De Terschueren ~ Drie}{uren zijn geslagen}\\

\haiku{Mulle Meneers de,.}{raadsheer nui ik verklaar ui}{de s\'eance open}\\

\haiku{Ik zou b.v. dijnk dat,......}{het niet waar contrarie van}{aan Vervijn daar ui}\\

\haiku{nous devons faire,.}{attention \`a Dias}{savez-vous}\\

\haiku{Bosschaert Ja maar alij,,...}{ten es maar est da Vreeze}{sekretaris}\\

\haiku{Sinds gij ze niet meer,.}{kunt gade slaan houden zij}{niet op met drinken}\\

\haiku{Schielijk glijdt de schel.}{hem uit de hand en valt op}{de zoldering}\\

\haiku{Lucy, Berthe en()!}{Laurence  geestdriftig}{toespringend Zero}\\

\haiku{Beiden aanschouwen),...}{elkander glimlachend Miss}{Jane mijn vriend}\\

\haiku{(Fernand staat recht) (tot),?}{hare dochters Maar niet te}{geweldig niet waar}\\

\haiku{Mevrouw Vandame(),.}{de hand openend Aan u}{Mijnheer Robert}\\

\haiku{Waer ic mij wend, waer.}{ic mij keer Ghij sijt alleen}{in mijn ghedachte}\\

\haiku{Mijnheer Lansing Neen,,.}{Mijnheer Fernand wij hebben}{hem niet gezien}\\

\haiku{ik ken ze, ik heb}{ze overwogen en doorgrond}{en reken ze als}\\

\haiku{'K heb reeds te veel.}{gezeid en zal er geen woord}{meer bijvoegen}\\

\haiku{Georges, Mijnheer.}{Lansing en Fernand ijlen}{naar de ingangdeur}\\

\haiku{Allen staan recht en,,.}{komen behalve Fernand}{op den voorgrond}\\

\haiku{Tot Haegen) Pa, we.}{gaan nu dat boodschapje doen}{bij de modiste}\\

\haiku{Als mijn schoonzoon zijt.}{ge solidair met mij en}{met ons allen}\\

\haiku{Het is hoogst zeldzaam.}{dat de kunst zooveel opbrengt}{als de handel}\\

\haiku{Papin-Dupont(),,.}{vriendelijk glimlachend}{Ja ja natuurlijk}\\

\haiku{(drukt zijne hand. Tot,)...}{Germaine die opgestaan}{is Germaineke}\\

\haiku{Plechtig) En indien?...}{ik u thans eens zegde dat}{ik haar bemin}\\

\haiku{Germaine O, dat.}{akelig geheim zou ik toch}{zoo gaarne kennen}\\

\haiku{Hij wacht,... hij wacht dat.}{gij hem de toelating geeft}{hier te komen}\\

\haiku{Hij is wijzer dan,.}{gij hij begrijpt wel dat ik}{zal vertrekken}\\

\haiku{Gisteren, tegen '.}{den avond heeft ie hier weln}{uur rondgeloerd}\\

\haiku{Daar net sloeg ie al '.}{weern haas kapot in de}{jacht van den baron}\\

\haiku{(woedend tot Labeer),!}{die weer naar de achterdeur}{wil Wel verdomme}\\

\haiku{Zij slaan een kruis en,.}{bidden even in stilte met}{gevouwen handen}\\

\haiku{Elodie  (kijkt om,,)?}{half norsch half glimlachend}{Wat heb je daar}\\

\haiku{Labeer  (lachend,,) '!}{tot Buck met den haas in de}{hand Zoone gannef}\\

\haiku{al zie ik het graag,.}{zelf er een pakken daar mot}{ik niks van hebbe}\\

\haiku{Elodie  (smeekend), '.}{O Frans doe het nou ins}{Hemelsnaam nooit meer}\\

\haiku{(Hij toont van verre) ' ',.}{den haas aan Labeer Kijkr}{s na vader}\\

\haiku{Toen dien ouwe met:}{zijn grijze bakkebaarde}{tegen Vader riep}\\

\haiku{Labeer loopt eensklaps '.}{naar de deur om die int}{nachtslot te draaien}\\

\haiku{en in z'n brief heet.}{ie an Elodie beloofd dat}{ie nie meer zal stroope}\\

\haiku{Een oogenblik staat.}{hij roerloos naar regen en}{wind te luisteren}\\

\haiku{tegen m'n zin ben -!}{ik eens verhuisd en het is}{m'n verandering}\\

\haiku{en ik zal nog 's.}{probeere of er voor ons geen}{recht te krijgen is}\\

\haiku{en ik zal met 'm......}{wegloopen ook zoo gauw we}{de kans maar schoon zien}\\

\haiku{'k ben er zeker.}{van dat we van jou niks te}{vreeze hebbe}\\

\haiku{Daar heb ik ook al,.}{zoo wat van gehoord meneer}{de notaris}\\

\haiku{Labeer probeert om,.}{op te staan maar zakt bevend}{op zijn stoel terug}\\

\haiku{Ge hebt 'm toch wel.}{hoore zegge dat ie naar}{de laatste trein moest}\\

\haiku{Je hebt ongelijk,, {\textquoteleft}{\textquoteright}.}{Cora datjuffrouw is niet}{om aan te hooren}\\

\haiku{Twee, drie stemmen  (,),?}{tegelijk verontwaardigd}{Hoe zo meneer}\\

\haiku{Zijn jullie zulke,?}{ego{\"\i}sten die alleen aan}{zichzelf denken}\\

\haiku{Je teekent dus die.}{aardige familie voor}{vier kaarten op}\\

\haiku{de costumes van;}{mijn oudste dochter en mijn}{zoon worden gehuurd}\\

\haiku{Barrois Woont U 't,?}{gansche jaar buiten meneer}{Van Raveschoot}\\

\haiku{Dat mensch heeft iets... iets,...}{griezeligs over zich iets waar}{je van huivert}\\

\haiku{Van Maanen  (kijkt schuw.}{Lien Van Thoorn na die in den}{achtergrond verdwijnt}\\

\haiku{Cora  (snikkend) ',... '.}{Ik kant niet helpen ik}{kant niet helpen}\\

\haiku{Angstig dringt Cora.}{zich nog dieper in den hoek}{van de veranda}\\

\haiku{Ik doe d'r mijn best,.}{voor maar Cora moet toch eerst}{nog ja zeggen}\\

\haiku{Arthur  (boos) Zeg, ben,,?}{jij mal Van Maanen of hou je}{me voor de gek}\\

\haiku{Wat is dat daar in,,.}{de keuken die nare lucht}{dat stoomgesis}\\

\haiku{Daarachter, rechts en,.}{links de oude baron en}{baron Maurice}\\

\haiku{Ik weet alleen dat.}{hij te Brussel is en naar}{hier komen z\`al}\\

\haiku{Barones Maurice,?}{Tiens-toi bien tranquille}{n'est-ce pas}\\

\haiku{u mijn vriend meneer.}{Franklin Van Paemel uit Blue}{Springs voorstellen}\\

\haiku{binnen, meneer Van}{Paemel en zet u.}{Oude baron Wees}\\

\haiku{Want anders laten.}{ze wel heel graag de mannen}{voor haar werken}\\

\haiku{Weet ge wel, dat ik?}{hier herhaaldelijk overheen}{gevlogen ben}\\

\haiku{ik geloof dat wij.}{voorloopig voldoende}{afgesproken zijn}\\

\haiku{Oude barones,.}{Wacht nog een beetje meneer}{Van Paemel}\\

\haiku{Van Gele Gedacht ' ',.}{n es eigentlijkt woord}{niet meneer den B'ron}\\

\haiku{We woaren amoal, ',!}{wrie schouw van hem wantt was}{nen dullen zille}\\

\haiku{Oude barones()...}{drukt Franklin de hand Meneer}{Van Paemel}\\

\haiku{Barones Maurice,.}{Nog een taske thee meneer}{Van Paemel}\\

\haiku{Oude baron Vous,...}{savez \c{c}a n'a pas beaucoup}{d'importance}\\

\haiku{Oude barones(),?}{schrikkend Qu'est-ce qui}{te prend Maurice}\\

\haiku{Denkt aan uw naam en...!}{maatschappelijken stand en}{aan de mijne}\\

\haiku{Een tijd geleden,?}{heb ik u hier luchtfotos}{laten zien niet waar}\\

\haiku{Ik ken weinig of,.}{geen Fransch maar toch begrijp ik}{die twee woorden wel}\\

\haiku{Bij ons hoeven ze '.}{zichs nachts niet in bosschen}{te verschuilen}\\

\haiku{Zij valt snikkend met.}{gevouwen handen v\'o\'or Franklin}{op haar knie\"en}\\

\haiku{Ik vraag u, meneer,,,.}{wat gij hier doet bij mijn vrouw}{bij mijn dochter}\\
