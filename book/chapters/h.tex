\chapter[28 auteurs, 5766 haiku's]{achtentwintig auteurs, vijfduizendzevenhonderdzesenzestig haiku's}

\section{Jacob Isra\"el de Haan}

\subsection{Uit: Pathologie\"en. De ondergangen van Johan van Vere de With}

\haiku{In de kamer naast,.}{de hare sliep haar man die}{het kind bij zich hield}\\

\haiku{Zij ging naar hare.}{eigen kamer terug om}{den dood te zoeken}\\

\haiku{{\textquoteright} Zijn vader verschrok,:}{zichtbaar voor Johan en de man}{dacht snel en beslist}\\

\haiku{{\textquoteright} Johan gevoelde, dat.}{de stem van zijnen vader}{brak in diens keel}\\

\haiku{Maar op dat juiste.}{oogenblik brak zijn lichaam}{in eene uitbarsting}\\

\haiku{{\textquoteleft}omdat ik juist voor,.}{vader zoo gevoel is ons}{leven toch verward}\\

\haiku{Johan zag het altijd.}{gaarne hoe de nacht in den}{dag veranderd werd}\\

\haiku{Johan gevoelde, dat.}{de linkerhand van zijnen}{vader heel zacht was}\\

\haiku{hoe de prachtige.}{rivier wezen zou met wind}{en met dat avondlicht}\\

\haiku{Johan zorgde geheel,.}{alleen dat zijn vader en}{hij nog thee kregen}\\

\haiku{{\textquoteleft}laat ik vader nu,.}{vertellen wat mijn groot en}{geheim verdriet is}\\

\haiku{Johan had twee dingen,.}{in dat gesprek gebracht die}{vreemd genoeg waren}\\

\haiku{Tusschen vieren en.}{half vijf moesten dan dikwijls de}{lampen nog licht op}\\

\haiku{{\textquoteright} Kor Koster sprak nu:}{met zacht-droeve stem van}{zeer ernstig verwijt}\\

\haiku{Johan wist, dat hij zoo,.}{gehavend was dat hij niet}{naar school kon komen}\\

\haiku{als je wilt mag je.}{gerust uit ons huis en uit}{onze stad weggaan}\\

\haiku{Ik dank je wel, dat,.}{je mij geschreven hebt dat}{het je vrij goed gaat}\\

\haiku{De vader van Johan:}{kwam uit Haarlem terug en}{daarna schreef hij Johan}\\

\haiku{Dat bovenhuis is,}{niet groot dus je kunt beslist}{niet meer dan \'e\'ene}\\

\haiku{Verder legde hij.}{de blonde vouwen van zijn}{haar goed in orde}\\

\haiku{De levenswijze.}{van het huis in Cuilemburg}{veranderde dus}\\

\haiku{Op eenen zonnigen,.}{middag zag Johan de groote zee}{die hij niet kende}\\

\haiku{Hij gaf in dien tijd,:}{verfijnde uitvoerige}{beschrijvingen van}\\

\haiku{Zij konden evengoed,.}{naar den bleeker gaan als}{ergens anders heen}\\

\haiku{Er kwam geen antwoord,.}{na zoovele biddende}{smekende brieven}\\

\haiku{{\textquoteleft}Het spijt mij, dat ze,...}{niet ruiken ik houd niet van}{bloemen zonder geur}\\

\haiku{{\textquoteright} Ren\'e werd nu diep,.}{teeder waardoor hij Johan nog}{veel meer verwarde}\\

\haiku{Hij wilde toen zelf.}{naar Londen gaan om Ren\'e}{naar huis te halen}\\

\haiku{Daarna zeide hij,.}{in de school dat hij thuis bleef}{omdat hij ziek was}\\

\haiku{{\textquoteright} {\textquoteleft}Dat weet ik niet... ik,...}{zeg je alleen eerlijk en}{precies wat ik voel}\\

\haiku{Maar ik kan nu niet,,}{dicht bij je blijven hier in}{huis dat voel ik wel}\\

\haiku{Zijn lichaam werd warm,.}{van licht trillend als fijne}{gevoelige vlam}\\

\haiku{dat is alles, wat.}{je met je onredelijk}{verlangen bereikt}\\

\haiku{Ook niet, omdat ik.}{dan wellicht van dit huis en}{Haarlem scheiden moet}\\

\haiku{Want Ren\'e zou hem.}{in het huis zijns vaders niet}{durven navolgen}\\

\haiku{En hoe dat einde,,.}{wezen zal dat hangt niet van}{jou af maar van mij}\\

\haiku{Ren\'e verhaalde.}{aan Johan van zijn leven in}{diepte van steden}\\

\haiku{{\textquoteleft}Kom Hannie,{\textquoteright} vleide, {\textquoteleft},}{Ren\'ehet is nu toch over}{twee\"en naar school gaan}\\

\haiku{maar 't is goed, ik,...}{zal je wel weer laten zien}{dat ik je liefheb}\\

\haiku{als u Ren\'e dwingt,....}{hier uit het huis te gaan dan}{ga ik ook beslist}\\

\haiku{het licht knipte in,,.}{zijne donkere oogen hij}{stond naast Johan die zat}\\

\haiku{Een avond van dien tijd,.}{van het leven van Johan was}{van prachtige rust}\\

\haiku{Rustig, met volmaakt,.}{bewerkte letters schreef hij}{hem eenen kleinen brief}\\

\haiku{Johan hoorde, dat hij.}{met water waschte en}{steenen dingen bewoog}\\

\haiku{{\textquoteright} {\textquoteleft}En dat je weer even,{\textquoteright}.}{ongelukkig worden zult}{zeide Ren\'e snel}\\

\haiku{... en daarom zul je...}{ten slotte aan je liefde}{voor mij ondergaan}\\

\haiku{en het is voor mij,.}{voordeelig dat het leven}{nu eenmaal zoo is}\\

\haiku{Hij wist niet, waarom,.}{hij Ren\'e had ontmoet door}{wien hij onderging}\\

\haiku{Maar daar heb ik op.}{het oogenblik niet zoo heel}{veel bezwaar tegen}\\

\haiku{Of de vrouw het eerst,,.}{daar ben ik nog benieuwd naar}{om dat te weten}\\

\haiku{{\textquoteleft}je denkt geloof ik,,.}{dat ik alles met je doe}{waar ik zin in heb}\\

\haiku{Hij rilde nu in.}{betrouwbare sterkte van}{eigen veiligheid}\\

\haiku{{\textquoteright} 30 Johan lag zonder.}{pijn en zonder ongeduld}{op zijne slaapplaats}\\

\haiku{De handen van Johan,.}{werden stukgeslagen en}{zoo ook zijn gezicht}\\

\haiku{Hij vroeg, of hij niet.}{voor eenige weken bij hen}{binnenwonen kon}\\

\haiku{{\textquoteright} {\textquoteleft}Dat geloof ik ook...,,?}{dus je vindt het niet noodig dat}{ik je dat voorlees}\\

\haiku{Zijn hoofd brandde van,.}{pijn en het stak zijne oogen}{aan die overspanden}\\

\haiku{Dat zijn de gaven,.}{des levens die ons met het}{leven verzoenen}\\

\haiku{De betekenis:}{van zijn dichterschap is nu}{algemeen erkend}\\

\haiku{Dit boek hoort bij het.}{levensverhaal van Jacob}{Isra\"el de Haan}\\

\subsection{Uit: In Russische gevangenissen}

\haiku{Maar zijn goede trouw.}{is boven den minsten}{twijfel verheven}\\

\haiku{De regeering beschikt,.}{over de verkeersmiddelen}{post en telegraaf}\\

\haiku{Daarvan maakt men zich.}{in het buitenland eene te}{zwarte voorstelling}\\

\haiku{Teleurstelling over,.}{het feit dat de Doema zoo}{weinig bereikte}\\

\haiku{Het zal duren tot,.}{de boeren begrijpen dat}{zij bedrogen zijn}\\

\haiku{en dat zij het weer,.}{zullen doen wanneer er geen}{ander bewind komt}\\

\haiku{Van Foinitzky en.}{van den Hoogleeraar en}{Senator Tag\'anzeff}\\

\haiku{In de kolonie.}{bij St. Petersburg volgt men}{het tweede systeem}\\

\haiku{De salarissen.}{in de koloni\"en en}{prioeten zijn laag}\\

\haiku{In de tuinen en.}{weiden waren de jongens}{nu aan  het werk}\\

\haiku{Russische knapen.}{hebben zuivere stemmen}{en een diep gevoel}\\

\haiku{Eene groote sterfte in.}{de gevangenissen is}{het gevolg hiervan}\\

\haiku{In Moscou vele.}{uit het zoo mogelijk nog}{meer beruchte Orel}\\

\haiku{Zij zijn gevangen:}{gezet door de civiele}{autoriteiten}\\

\haiku{in zijn oogen was de.}{vage dwaling die erger}{is dan eenige klacht}\\

\haiku{De redenen van.}{herroeping zijn talrijk en}{vaag geformuleerd}\\

\haiku{De gevangene.}{moest daar ook slapen wegens}{gebrek aan ruimte}\\

\haiku{Toen den vierden dag.}{een bewaarder kwam zag die}{het vuil in de cel}\\

\haiku{In zijn woede sloeg,.}{hij den man tegen den grond}{die in het vuil viel}\\

\haiku{Wij zaten samen.}{op den rand van haar bed als}{oude bekenden}\\

\haiku{De katorgisten,.}{werken twaalf uur per dag met}{een uur vrijen tijd}\\

\haiku{Z\'o\'oleverde 1907 reeds\%\%.}{24 meer op dan 1906 en 1908}{wederom 26 meer}\\

\haiku{Het gevolg is, dat.}{wij allen zenuwziek en}{ellendig worden}\\

\haiku{Ieder gesprek met.}{een gevangene alleen}{moest ik afdwingen}\\

\haiku{ach, krachtloos wordt hun,.}{verstand Dat het de dagen}{van hun strijd vergeet}\\

\haiku{Over eenigen hunner.}{moge in het bijzonder}{nog iets gezegd zijn}\\

\haiku{dat is al veel waard,,.}{dan vinden de lateren}{een steun een partij}\\

\haiku{ik was angstig voor,,.}{dien kalmen trotschen jongen}{nog niet twintig jaar}\\

\haiku{Nu ontbreekt ook de.}{schijn van recht die een vonnis}{toch altijd  heeft}\\

\haiku{Toen ik er was, werd.}{in Riga niet meer alle}{dagen geslagen}\\

\haiku{Ik antwoordde, dat.}{ik zijne carri\`ere}{niet bedreigen mocht}\\

\haiku{Ik ben 51 jaar, ik,.}{geloof niet dat ik nog vier}{jaar te leven heb}\\

\haiku{mijn voeten doen zeer,,.}{van die kettingen mijn maag}{is ziek mijn hoofd brandt}\\

\haiku{Ook Zjadanofski.}{werd geslagen tot hij zich}{niet meer verroerde}\\

\haiku{{\textquoteleft}Ik herinner mij.}{het boek van den Heer Harry}{de Wyndt nog zeer goed}\\

\haiku{Een half jaar later.}{was er te Riga weder}{een hongerstaking}\\

\haiku{Zij verzekerden,.}{mij dat de toestand in Orel}{schandelijk slecht is29}\\

\haiku{Zinge mijn lied nog,}{slechts van uw hartenbrekend}{lijden Vermijde}\\

\haiku{Daarom ben ik ook.}{zoo verbaasd over wat verder}{te Orel is gebeurd}\\

\haiku{Ze vlogen op mij.}{af en begonnen mij te}{slaan en te trappen}\\

\haiku{Hij genoot van het:}{schouwspel en hij hitste de}{bewakers nog aan}\\

\haiku{Ik blijf hier tot er.}{plaats is in een inrichting}{voor zenuwlijders}\\

\haiku{Ik verwachtte niet, {\textquotedblleft}{\textquotedblright}.}{anders of men zou ook mij}{eenbezoek brengen}\\

\subsection{Uit: Jerusalem}

\haiku{Het zal haar weer acht.}{pond kosten en iedereen}{zal haar uitlachen}\\

\haiku{Wij echter krijgen.}{een mooi biljet van goud op}{zijde-papier}\\

\haiku{Hamame zit als.}{een pop in wit met sluier}{van wit en zilver}\\

\haiku{En daartegen in,,.}{een kleine die gespannen}{luid kraait en schatert}\\

\haiku{{\textquoteleft}laat mij teruggaan.}{naar Jeruzalem en de}{bron alsnog openen}\\

\haiku{Gij glimlacht wellicht?}{over al dien eerbied en over}{al die etiquette}\\

\haiku{En Abdoel Salaam,,.}{die een wijs man is heeft hem}{laten omhakken}\\

\haiku{Ik hoop er nog wel.}{eens heen te gaan met Sa{\"\i}d}{Effendi samen}\\

\haiku{En Arabisch met den.}{vader en de andere}{familieleden}\\

\haiku{Een tafeltje bij.}{het raam met het mooie uitzicht}{op Jeruzalem}\\

\haiku{Natuurlijk komt het,,.}{niet te pas hem zonder meer}{daarnaar te vragen}\\

\haiku{Schatten aan boeken.}{zijn naar Engeland en naar}{Amerika gegaan}\\

\haiku{Een van mijn Joodsch (!}{vrienden te Amsterdamwat}{is Amsterdam ver}\\

\haiku{Maar achter al die.}{waarde-verschillen zal}{ik wel nooit komen}\\

\haiku{Maar er zijn hoeken,.}{waar de zon nooit komt en waar}{de lucht loodzwaar is}\\

\haiku{Op het terras van,,,,.}{het weeshuis vol vol vol van}{zon is nu bezoek}\\

\haiku{Van het Weeshuis gaan.}{we dus eerst weer het domein}{van de Russen over}\\

\haiku{Er is een rijweg,.}{door de Engelschen in den}{oorlog uitgelegd}\\

\haiku{Of opgezet als,.}{groote huizen en afgebouwd}{met een dwaas kort dak}\\

\haiku{Haar broer een van de.}{stoutsten en sterksten onder}{de Joodsche ruiters}\\

\haiku{Wie ook bestolen,.}{worden de gasten van den}{hoofdman zeker niet}\\

\haiku{Adil koopt ook bonbons.}{voor de vier vrouwen en de}{vele kinderen}\\

\haiku{En toch zijn de twee.}{slagen met een wreeden dood}{niet te duur betaald}\\

\haiku{Geen ander geluid.}{dan de stappende pooten}{en de echo daarvan}\\

\haiku{Maar nu Aboe Fares,.}{de sjech van allen is zijn}{er geene twisten meer}\\

\haiku{Waar wij rijden langs.}{zwarte tenten slaan de}{waaksche honden aan}\\

\haiku{Wij moeten elk van.}{de huisbedienden een half}{pond baksjisj geven}\\

\haiku{Ditmaal met een knecht,.}{die loopt en straks de paarden}{terugbrengen zal}\\

\haiku{{\textquoteright} Omdat de Polen.}{de lievelingen van mijn}{Volk hebben vermoord}\\

\haiku{Omdat de Polen.}{de lievelingen van mijn}{Volk hebben vermoord}\\

\haiku{Om twee uur zullen.}{wij bij den Klaagmuur komen}{voor de gebeden}\\

\haiku{Een geheelen dag.}{vasten is in dit heete}{weer wel al te zwaar}\\

\haiku{Verleden jaar  .}{is de Regeering het Weeshuis}{goed gezind geweest}\\

\subsection{Uit: Palestina}

\haiku{Deugdzaam sterven is,.}{mogelijk deugdzaam leven}{een ongerijmdheid}\\

\haiku{Voorloopig zeg,.}{ik dus alleen dat wij naar}{Hebron zullen gaan}\\

\haiku{Er is toch nu wel.}{iets veranderd bij een jaar}{of twee geleden}\\

\haiku{Een heel oud man met.}{het vertrouwen van een kind}{in zijnen vader}\\

\haiku{En zij zal water.}{morsen op de gordijnen}{en op de sofa}\\

\haiku{Wij zien van verre,.}{Bethlehem waar de rijke}{Christenen wonen}\\

\haiku{Abdoel Sala\"am heeft,.}{een witte warme deken}{medegenomen}\\

\haiku{De schavuit haalt een.}{diepe tabaksteug door zijn}{Turksche waterpijp}\\

\haiku{David deelde buit ().}{met de mannen van Hebron}{I Samuel 30:31}\\

\haiku{Godfried van Bouillon.}{gaf de stad in Leen aan den}{Heer van Avesnes}\\

\haiku{Vele kooplieden.}{dragen witte en groene}{doeken om hun fez}\\

\haiku{Vandaag zagen wij '.}{hen voort laatst gaan door de}{stad en door het dal}\\

\haiku{En wendt dan naar het,.}{Westen om bij Gaza in}{de zee te breken}\\

\haiku{Van boven af zien,,.}{wij het dal breed en edel waar}{de weg doorheenwindt}\\

\haiku{Wij kennen ook den,.}{onder-gouverneur uit}{Hebron gekomen}\\

\haiku{O, Berseba is.}{veel gemakkelijker dan}{Jeruzalem}\\

\haiku{De   B.F.C. regelt.}{de Dujet en de Qis\^as naar}{vaste tarieven}\\

\haiku{De Dujet hangt af.}{van rang en vermogen van}{beide partijen}\\

\haiku{Wij hopen ook een.}{bezoek te brengen bij de}{moeder van Jimmy}\\

\haiku{En zij is verloofd,.}{met den zoon van sjeikh Djadoeng}{die al zes jaar is}\\

\haiku{De twee leerjongens.}{blijven in hun holletje}{om af te werken}\\

\haiku{De h\^otelhouder.}{met vrouw en vijf kinderen}{in de andere}\\

\haiku{Weer verandert de.}{verhouding in den tijd van}{de Maccabee\"en}\\

\haiku{De Philistijnen.}{zijn onder de macht van de}{Grieken gekomen}\\

\haiku{De meesten zijn na.}{den oorlog uit Rusland in}{het Land gekomen}\\

\haiku{Abdoel Sala\"am is.}{h\'e\'el verheugd den ouden man}{mede te nemen}\\

\haiku{Hier, in dien tuin, waar,.}{het vernielde landhuis staat}{vestten de Turken}\\

\haiku{Zij duiken weg in.}{hunne schoudermantels van}{grijs en bruin gestreept}\\

\haiku{Wij willen vandaag,.}{nog rijden naar Esd\^ud het}{oude Ashdod}\\

\haiku{De schapen en de.}{geiten zullen er gaan in}{gemengde kudden}\\

\haiku{Door een prachtige.}{cactuslaan strompelen}{wij Yebnah binnen}\\

\haiku{Gelukkig werken.}{er sterke Joodsche jongens}{in de nabijheid}\\

\haiku{Kameelen trekken,.}{langs den weg die de kisten}{naar Jaffa voeren}\\

\haiku{Maar zij moeten ons,.}{vergeven dat wij er ons}{thans niet ophouden}\\

\haiku{En Allah zal ons,.}{leeren hoe mooi ons land ook in}{den regen kan zijn}\\

\haiku{Wij zouden een h\'e\'el.}{mooi uitzicht kunnen hebben}{van het platform af}\\

\subsection{Uit: Pijpelijntjes}

\haiku{{\textquoteright} In de kamer zit, '.}{de moeder mager ent}{dundonkere haar grijs}\\

\haiku{{\textquoteright} doofvraagt de oude.}{en haar woorden cadansen}{met de stopnaald mee}\\

\haiku{'t Kacheltje brandt,...}{een warme looming lijst}{in de kamer neer}\\

\haiku{{\textquoteright} {\textquoteleft}Ja... gesoesd, nou je ',, '...?}{t zegt hoor ik ook datt}{regent ben je klaar}\\

\haiku{{\textquoteright} {\textquoteleft}'t Regent zoo... 't, '.}{regent zoo laten wet}{morgenochtend doen}\\

\haiku{Een prettig gevoel.}{tintelde door mijn rug en}{langs mijn onderlijf}\\

\haiku{Toen vroolijker in '.}{t kleurige lampelicht}{dronken we koffie}\\

\haiku{Alles donker en,,.}{stil de straat leeg waardoor \'e\'en}{man zijn stappen klonk}\\

\haiku{{\textquoteright} Veertien dagen was '.}{Sam weg en ik wast huis}{niet uit geweest}\\

\haiku{... ik moet in eenen uit,.}{op deze brief wacht u maar}{niet met de koffie}\\

\haiku{nou op, dat Hec niet,.}{bijt als-ie los komt dan zorg}{ik voor m'neer Sam}\\

\haiku{{\textquoteright} {\textquoteleft}Kan me niks schelen, '....}{watr met dat beest gebeurt}{zijn m'n zaken niet}\\

\haiku{Uit de donkere,.}{huizenhoek naast ons doemde}{oud mannetje op}\\

\haiku{juffrouw Meks liet de,....}{gordijnen vallen met een}{harde rinkratel}\\

\haiku{Toosie ga jij naar bed,{\textquoteright},, '.}{strengde juffrouw Meks die vond}{datt verkeerd ging}\\

\haiku{welnou dan moet ze.}{maar deris bij dominee}{Deelman gaan hooren}\\

\haiku{{\textquoteright} {\textquoteleft}Zonder schandaal en,.}{die daalder betaalt u ook}{niet dat zal u zien}\\

\haiku{{\textquoteright} {\textquoteleft}Wat heb je d'r toch, '.}{mee noodigt lijkt waarachtig}{wel of je mee moet}\\

\haiku{Juffrouw Meks vond 't ', ' '.}{n goed id\'ee juffrouw Meks vond}{tn heel goed idee}\\

\haiku{{\textquoteright} {\textquoteleft}Nou{\textquoteright} zegt Hein {\textquoteleft}nou ga,.}{ik toch niet weg voor ik die}{ouwe moer ook heb}\\

\haiku{Om drie uur liep ik,,.}{de straat op die nog nat was}{maar zonder regen}\\

\haiku{En dan de drukke, '.}{stad-in-de-stad dwars}{door naart Rokin}\\

\haiku{Ze liepen met z'n,.}{twee\"en een ander was er}{bij dan gisteren}\\

\haiku{En dicht tegen Sam.}{aankreunend vertelde ik}{hem ellende}\\

\haiku{u kan ze gerust,.}{eten d'r zit geen margerien}{op en kaas van Noack}\\

\haiku{{\textquoteleft}Hoe vin-je me nou...,...}{laat Sam nou eerlijk zeggen}{of ik beter wor}\\

\haiku{morgenochtend zie '.}{ik je wel of vanavond als}{t niet te laat wordt}\\

\haiku{Maar dan zou ik nu,,.}{het andere nemen het}{gore niet goede}\\

\haiku{Maar ik wou toen niet,.}{met hem trammen hij was zoo}{gering en zoo vuil}\\

\haiku{{\textquoteright} In 't deurportaal,.}{stond Koos koudrillend met z'n}{handen in z'n zak}\\

\haiku{{\textquoteright} In de kamer stak,.}{ik de groote lamp ook op nu}{kon ik hem goed zien}\\

\haiku{Nou, toen woonde d'r ' ':}{bij ons opt dorpn oud}{vrouwtje en die zei}\\

\haiku{{\textquoteleft}Het is netjes, ik,,{\textquoteright}...}{moet u zeggen juffrouw Meks}{het is heel netjes}\\

\haiku{Sam moest naar Haarlem, '.}{een middag en ik had hem}{naart spoor gebracht}\\

\haiku{as m'neer Sam komt......}{zullen we wel zien steek u}{de lamp maar even op}\\

\haiku{eerst al die drukte.}{van je advocaat en zoo}{en van de rechtbank}\\

\haiku{d'r was niemand om,,}{me te halen nou en de}{rest dat snap je wel}\\

\haiku{{\textquoteleft}Ik heb nog wat geld,,?}{niet veel hoor zal ik nou maar}{bij jullie blijven}\\

\haiku{Met m'n slaapwarme '...}{voeten opt koudgladde}{vloerzeil rilde ik}\\

\haiku{{\textquoteleft}Zalle we niet naar,{\textquoteright}, {\textquoteleft}'.}{bed gaan fluisterde zijt}{is morgen vroegdag}\\

\haiku{Nee die is straks aan '.}{t station met Anna}{en de anderen}\\

\haiku{De kamer druk van,.}{menschen en ruw praten door}{mekaar kletsende}\\

\haiku{{\textquoteright} {\textquoteleft}Ja, dat wel, maar op.}{alle meniere hebt u}{z'n uitgaanskas nog}\\

\haiku{Anna help jij Stien, '.}{even voort naar beneden maar}{laatr niet vallen}\\

\haiku{Moet je begrijpen,,}{dat je daar nog na de kerk}{toe moet ook affijn}\\

\haiku{Hij dan weer langzaam.}{z'n woorden nou-kauwend}{en wikkewegend}\\

\haiku{ik zei gewoon, dat '...}{ikt verdomde en of}{ik ander werk kreeg}\\

\haiku{Maar ruw drukte ik ':}{m vlak tegen m'n beenen aan}{en zei heethijgend}\\

\haiku{maar zeg, ga d'ris even, ' '.}{kijkent is net of ze}{opt raam tikken}\\

\haiku{boven heb ik geen,.}{deel van leven meer ze doen}{daar niks as schelden}\\

\haiku{en we nemen wel...,,?}{goed afscheid vannacht voorgoed}{geloof u ook niet}\\

\haiku{------------- Maar rustig werd 't,.}{dan toch ze waren stil gaan}{liggen met mekaar}\\

\haiku{ik hou heel veel van,...}{d'r en ik weet gelukkig}{maar weinig van d'raf}\\

\haiku{ze h\`et in de Pijp,...}{gewoond zegt ze en es heeft}{ze niet dat weet ik}\\

\haiku{{\textquoteright} {\textquoteleft}Bij m'neer Driesse......,.}{ja ik kom dadelijk even}{m'n bloessie andoen}\\

\haiku{{\textquoteleft}Ziezoo, nou zijn we......{\textquoteright} {\textquoteleft}....}{d'r ik blijEn ik verdomd}{beroerd voor Overhoff}\\

\haiku{En lange dagen '.}{bleef ik bijm lijdend zijn}{onduurzieke onrust}\\

\haiku{Sprak niet tegen ons,, '.}{vroeg alleen maar wat geld als}{zet noodig had}\\

\haiku{Tegen het einde.}{van zijn boek noemt De Haan de}{Hoedemakersstraat}\\

\haiku{De naam van deze.}{straat werd later veranderd}{in Kuipersstraat}\\

\section{Hadewijch}

\subsection{Uit: Brieven}

\haiku{Misschien zou er met.}{den tijd wat meer licht in die}{duisternis opgaan}\\

\haiku{Een omgekeerde {\textsection}.}{D als paragraafteeken wordt}{hier door aangeduid}\\

\haiku{wat Gods goedheid hun (-).}{daar als wezenheid schenkt is}{hun waarheid1017}\\

\haiku{- De minnende ziel.}{wordt deelachtig gemaakt aan}{de klaarheid des Zoons}\\

\haiku{Daarom ook is zij,:}{niet alleen geroepen maar}{ook uitverkoren}\\

\haiku{Vooral zij die nog,}{jong is moet zich oefenen}{in alle deugden}\\

\haiku{zij verzekert haar:}{dat zij tot de Liefde}{uitverkoren is}\\

\haiku{omdat hij in zich,.}{heeft alles wat betaamt wat}{recht en rede is}\\

\haiku{Sijt blide altoes.}{in hope    om minne}{te vercrighene}\\

\haiku{Sijt op uwe hoede.}{ende in vreden van ||}{allen4041   dinghen}\\

\haiku{tegen een al te;}{stipte en te bepaalde}{levens-regel}\\

\haiku{Daer toe en constic.}{v niet bringhen dat    ghi}{mate daer ane hielt}\\

\haiku{Want de brief schijnt aan:}{te vangen te midden van}{een uiteenzetting}\\

\haiku{dat doet een deel der:}{trouwen gront dan herneemt zij}{deze uitdrukking}\\

\haiku{Er is hier nergens,.}{spraak van ontrouw nog minder}{van opstandigheid}\\

\haiku{De onvolmaakten;}{maken er hun liefdedienst}{afhankelijk van}\\

\haiku{Zij schikt en ordent,;}{zij drukt zich praegnanter en}{kernachtiger uit}\\

\haiku{als fragment G een.}{afschrift van den 6en en den}{10en brief volgens hs}\\

\haiku{Sine haken niet,.}{na smake Mer   335 si}{soeken orbere}\\

\haiku{tot de ontleding,:}{van een stemming waarin zij}{dikwijls verkeerd heeft}\\

\haiku{de overste zal doen;}{als Joseph die zijn broeders}{hoedde en leidde}\\

\haiku{zonder zich ooit op,}{zijn dienst te verheffen zal}{hij er naar streven}\\

\haiku{Liefde zoo trouw te.}{dienen is reeds het eeuwig}{leven beginnen}\\

\haiku{De MOTIEVEN die:}{daarbij ontwikkeld worden}{zijn voornamelijk}\\

\haiku{- Hier treffen wij de:}{grondgedachte aan van den}{strijd met de Liefde}\\

\haiku{de eeuwigheid die (-).}{de ware Liefde aan de}{ziel reeds schenkt1352}\\

\haiku{Mande heeft den brief;}{opgenomen tot r. 75}{tamelijk verkort}\\

\haiku{{\textquoteleft}blijdschap wordt ontsierd{\textquoteright}.}{door het vertrouwen op de}{duurzaamheid er van}\\

\haiku{zij kunnen er mooi,:}{over praten maar er zit wat}{anders in dan God}\\

\haiku{Ay en eest dan niet}{vrese- 424 425    lec}{roef dat wi vore}\\

\haiku{456 Dat ghenoeghet der,}{Minnen alre best datmen457}{458 te vollen}\\

\haiku{Aan het slot dan staat-,.}{r. 2531 om volgenden}{brief in te leiden}\\

\haiku{Want watmen dade,.}{buten   caritaten}{dat ware al niet}\\

\haiku{Hier omme steet wel,}{dat elc    mensche besie}{de gracie ende}\\

\haiku{Ende noch vraghet om,}{den wech dien die bi v}{502 sijn ende dien}\\

\haiku{om Minnen ere    .}{verdraghet den erren ende}{den onwetenden}\\

\haiku{geen menschelijke ().}{taal kan hemelsche dingen}{uitdrukken-122}\\

\haiku{Zoo werkt de ziel, door:}{de Liefde overheerscht en}{met Haar vereenigd}\\

\haiku{een prachtig beeld van:}{de hoogste waardigheid der}{ziel tegenover God}\\

\haiku{Ook in het Rike ().}{der Ghelieven wordt er op}{gezinspeeldc. XXV}\\

\haiku{de Volmaaktheid schenkt.}{aan de ziel het souverein}{gebied over haar zelf}\\

\haiku{Merk op dat er van.}{r. 69 af spraak is van de}{gansch onthechte ziel}\\

\haiku{ib. C. XLVI, blz.-,;}{9798 ook reeds C.XXXVI over}{Christus aan het Kruis}\\

\haiku{Hi velde sine,}{substancie dat was sinen}{heile-    ghen}\\

\haiku{wesene van.}{gode in een gheheel856}{gebruken comen}\\

\haiku{En toch schijnt deze:}{brief met den volgenden nauw}{verbonden te zijn}\\

\haiku{Ende alle die.}{v    belieghen die en}{weder segghet niet}\\

\haiku{Ay gheuoelt ende}{verstaet hoe    gherne ict}{saghe || ende}\\

\haiku{De gevoelens der ().}{ziel die met de Personen}{wandelt-64}\\

\haiku{toepassing van de?}{leer der circuminsessio}{personarum}\\

\haiku{Ik vlei me niet met:}{de hoop alles duidelijk}{te hebben gemaakt}\\

\haiku{God es mi met    .}{den vader gheheelleke}{met verweentheiden}\\

\haiku{dan komt 177 tot het,.}{einde dat ook het einde}{van zijn werkje is}\\

\haiku{Hier omme werden}{wi ghelettet in allen}{sinnen1106   Ende}\\

\haiku{Ende hier omme.}{en can nieman ande-}{ren ghehulpen}\\

\haiku{ieder vangt aan met,;}{een vers uit het Hooglied dat}{daarna verklaard wordt}\\

\haiku{Want  onse heer}{sedij selue eenen mensch Dat}{|| gerecht gebet1130}\\

\haiku{Die reden en can.}{god niet gesien dan in dat}{dat hi niet en is}\\

\haiku{Ende dat is daer,.}{om want si noch niet beset}{en sijn mit doechden}\\

\haiku{Want hi is god der.}{mynnen ende bekent wel}{die noet van mynnen}\\

\haiku{bijw. alleen, eeniglijk, -,.}{b.v. 14 39 e.e. niet 10 103}{e.e. alleinskine}\\

\haiku{soe sconen (schoon een),.}{beghin hebben zoo schoon}{beginnen 30 181}\\

\haiku{- znw. hare - n lien,,,,.}{dat waarover men beschaamd is}{schaamte 24 44 60}\\

\haiku{vgl. 3, 3 waar in.}{soortgelijk verband staat der}{heylegher doghet}\\

\haiku{- znw. vr. \'e\'enheid, het,,;}{in zichzelf vereenigd zijn}{b.v. van God 22 117}\\

\haiku{- der waerheit de,;}{genotvolle ervaring}{van wat is 15 14}\\

\haiku{bijw. naar wijze der;}{goddelijke natuur in}{den Vader 1 28}\\

\haiku{int - in het openbaar, ' (),.}{int algemeenin een}{gemeenschap 12 146}\\

\haiku{in ons selves - in, -,.}{eigen welbehagen 6}{286 na uwe 16 47}\\

\haiku{metten goede Gods,,,;}{met goddelijke dingen}{genade 10 67}\\

\haiku{- znw. vr. ook gratie,,,,;}{bovennatuurlijke gunst}{bijstand 10 13 61}\\

\haiku{- e ende ertsche,,;}{wat in den hemel en op}{aarde is 27 17}\\

\haiku{- maken een machtig, (),.}{geluid geven krachtig zijn}{van een stem 20 106}\\

\haiku{- znw. vr. {\textquoteleft}alreley{\textquoteright} (.);}{dair men wat af maken off}{besynnen kanTheut}\\

\haiku{- bnw. waarvoor geen naam,,.}{bestaat die geen naam hebben}{20 5 en daar pass}\\

\haiku{- gevolgd door inf. hem -,,;}{laten te iets verwaarloozen}{nalaten 4 42}\\

\haiku{dat ons ontbleven,;}{es waarin wij zijn te kort}{geschoten 6 337}\\

\haiku{- onverhaven bnw.,.;}{non elatus niet verheven}{22 22 en daar pass}\\

\haiku{hem - houden zich fraai,,,.}{onberispelijk houden}{gedragen 24 73}\\

\haiku{bi scoude ende,;}{bi rechte als rechtvaardig}{verschuldigd 17 90}\\

\haiku{met - n onderstaen,,,,.}{helpen steunen door vrede}{te schenken 5 4}\\

\haiku{werde 12, 229 en,;}{enkele dubieuze}{gevallen 6 68}\\

\haiku{ghi sijt te weec van.}{herten ende te kinsch}{in al uwen seden}\\

\haiku{Een brief, de XXXIe, uit.}{vroegere jaren werd er}{aan toegevoegd}\\

\haiku{Over Willem van St..}{Thierry zelf hoeven we niet}{lang uit te weiden1159}\\

\haiku{eam funditus et,,;}{interimens sicut mors}{interemit corpus}\\

\haiku{En zoo hebben de.}{L.S. ook voor de 41e preek wel}{Hadewijch gekend}\\

\haiku{en men wijst daartoe,.}{op het soort syllogisme}{waarmee hij aanvangt}\\

\haiku{met welk recht wordt zulk?}{een onbekende grootheid}{hier verondersteld}\\

\haiku{wat ook den stijl van.}{het slot kenmerkt tegenover}{dien van Hadewijch}\\

\haiku{Hoe komt het, dat ze:}{in al deze plaatsen op}{Richard terugvalt}\\

\haiku{Zoo ook fr. E, dat.}{immers nog uittreksels heeft}{uit den IVn Brief}\\

\haiku{Ziel is een weg waar;}{God door vaart in zijn vrijheid}{van uit zijn diepte}\\

\haiku{en dat juist maakt het,.}{mystiek in den ruimeren}{zin van het woord}\\

\haiku{dat is de Minne,,;}{die in de klaarheid het licht}{staat van de Godheid}\\

\haiku{Minne is de groote, (,).}{kracht die geheel de Schepping}{aandrijftXII 15 vlg.}\\

\haiku{Men heeft zelfs beweerd,:}{dat het Verlossingswerk voor}{Hadewijch niet leeft}\\

\haiku{Nu eens nu anders,,,.}{nu lief nu leet nu hier nu}{doer nu af nu ane}\\

\haiku{Ghi sult u proeven}{hoe ghi verdraghen moghet}{al dat u mescomt}\\

\haiku{Indien Hadewijch,}{zegt dat zij God haar zonde}{reeds zal bekennen}\\

\haiku{het is de hartstocht.}{van den H. Paulus om met}{Christus te zijn}\\

\haiku{Vereenigd, ja, hart;}{in hart en ziel in ziel en}{lichaam in lichaam}\\

\haiku{scone houdet na.}{de wet ende volcomen}{alsoe alst behoert}\\

\haiku{hoeveel meer we nog,!}{van Hem zouden ontvangen}{indien wij wilden}\\

\haiku{Daarom omgrijpt de.}{Vader den Zoon en den H.}{Geest in zijn enich recht}\\

\haiku{van zuiver leven,;}{in de liefde wat op zich}{zelf hoogst vruchtbaar is}\\

\haiku{Zoo geniet God ons.}{in zich zelven en alle}{heiligen met ons}\\

\haiku{Voert zij tot waarlijk?}{mystieke ervaringen}{en verschijnselen}\\

\haiku{de ingestorte,.}{mystieke Liefdestorm en}{orewoet van Beatrijs}\\

\haiku{Het is alles bij,,.}{Hadewijch zou men meenen}{natuur geworden}\\

\haiku{, dan zou men kunnen,.}{meenen dat de vrede er}{niet lang heeft geduurd}\\

\haiku{Heeft de hoogere?}{geestelijkheid zich ooit met}{het geschil gemoeid}\\

\haiku{zij heeft zich niet met,;}{hun kleederen getooid noch}{met hun verf en schijn}\\

\haiku{Ook den vreemden moet.}{wel wonderen van hare}{ende eysen}\\

\haiku{2Zie Inleiding op,.}{de Visioenen eerste}{en tweede hoofdstuk}\\

\haiku{hem selven ende,:}{allen creaturen =}{ten bate van voor}\\

\haiku{deen es vore 36}{dankes 37 B bezien 38 A}{glorilecheden}\\

\haiku{76 nemen met dank,.}{aanvaarden zonder er naar}{gezocht te hebben}\\

\haiku{130In allerlei te,.}{willen doen of vermijden}{wat vrijheid belet}\\

\haiku{119 seldi 120 B.}{zueten A haerre B}{harer 121 B warh}\\

\haiku{huedene 83 A.}{sine gratie B zijn B}{bezie84 B goet i.r}\\

\haiku{35829 b. 3592 ane B (.}{begert A gem. 3 haddic}{4 B noeytyt v.l.h}\\

\haiku{14 sonderlinghen?}{als comparatief gevoeld}{en daarom met dan}\\

\haiku{men..dech 30 soe dat;}{31 vanden B also 32}{dattene niemen}\\

\haiku{, zie 27 49 gronde (,.}{metdoch in A punt onder}{e in B doorgeh}\\

\haiku{) veruolghet Alsoe ();}{si nieti r. v. 50 no}{B gen. 52 B gen.}\\

\haiku{55 gheminneu 57.:}{A ghedincken 58 B}{so ongheeren}\\

\haiku{141 die (nose, nl.,):}{leed verdriet wij niet kunnen}{te boven komen}\\

\haiku{d.l.h.) egregie B;}{die eyghene d. 3 A}{heilegher B gen.}\\

\haiku{ververschen sonder,.}{vertraghen maar tot r. 43}{ontbreekt weer alles}\\

\haiku{eensgezindheid van,;}{gevoel die even als gij naar}{de Liefde streven}\\

\haiku{B hoge A wan...}{ge 104 selt 105 gi selt B}{van allen op ras}\\

\haiku{66 welk een zuiver;}{theocentrische reden}{voor onze blijdschap}\\

\haiku{27 dat ander waert.}{beteekent den Vader in de}{eenheid der natuur}\\

\haiku{onstech36 B dingen,.}{A ghebrukene in B}{is o uit u verb}\\

\haiku{of bet. dit = zij?}{die om hun zonden reeds ter}{helle behoorden}\\

\haiku{96-100 men ziet in:}{welken zin deze woorden}{moeten opgevat}\\

\haiku{naturlec 81 B;}{zien A twee 83 B gesien}{85 A gaet B ghaet}\\

\haiku{B vrems zeden 137:}{A na de werdecheit B}{nader werdecheit}\\

\haiku{niemanne 11 hen (.}{haren 12 B bezech wort}{in A iets onduid}\\

\haiku{De eerste is het.}{leven volgens Zijn eigen}{goddelijk wezen}\\

\haiku{B ia ende elc;}{wesen 353 A gheghoten}{es B ghegoten}\\

\haiku{met verw. naar hier).}{26 in confidentia opus}{871XLVJ-b}\\

\haiku{wat het is voor een.}{jonge herte Liefde te}{moeten ontberen}\\

\haiku{A te verwinne.}{B te verwinnen 9111 A}{iet 2 al ontbr}\\

\haiku{zere A diene:}{33 ziere 34 zinen B}{vart 35 na scone}\\

\haiku{A beziet zine.}{A ziere A ziere 13}{ziere w. 957.xxviij}\\

\haiku{De geestelijke.}{dronkenschap is een gewoon}{thema der mystiek}\\

\haiku{43 nedere B}{zinne 44 sijnse 46 B}{genuechten 49}\\

\haiku{142 ende dat... is:}{een nadere bepaling}{bij volwassene}\\

\haiku{A godelec B.}{godlec zijn 215 dingen 216}{wat ons C huus i.r}\\

\haiku{245 beteren =;}{met verzwegen vw. af te}{leiden uit wesen}\\

\haiku{5 tot enen mensche,.}{klaarblijkelijk tot haar zelf}{in een visioen}\\

\subsection{Uit: Visioenen}

\haiku{anders maken de,.}{letters C en B bekend}{dat ze uit die hss}\\

\haiku{twee sterretjes, dat.}{ze boven of onder aan}{de bladzijde staat}\\

\haiku{hieruit wordt de Lijst.}{der Volmaakten aangehaald}{door de letter 1}\\

\haiku{4de Boom (71-79) groot,,:}{met vele takken die door}{elkander groeiden}\\

\haiku{/ dan eneghen mensche /.}{die221 222    gheboren wart}{seder dat ic starf}\\

\haiku{Hij zegt niet, hoeveel.}{discipelen noch hoeveel}{talen er waren}\\

\haiku{I, 102, 1) dat dan.}{gedacht wordt op het hoogste}{der aarde te zijn}\\

\haiku{nochtan bleuen}{wi effenne/.    Ende}{si sal volwassen}\\

\haiku{heden/ ende met;}{di comen351   marghen}{in hare rike/}\\

\haiku{dat si in mi /;359 /;}{Ende dat si di cont}{van minen monde}\\

\haiku{De drie hemelen.}{worden haar dan verklaard als}{een beeld der Godheid}\\

\haiku{gaeft/, Ende ic noch}{niet doen en    bekinde}{uwe volcomene}\\

\haiku{Aan het slot worden, ':}{drie zaken vermeld doort}{visioen beteekend}\\

\haiku{vijf wegen, de eene,;}{hooger dan de andere}{leiden naar den top}\\

\haiku{Zoo neemt men Christus.}{enich op en beoefent men}{de eneghe Minne}\\

\haiku{Zij heeft Jezus in.}{zijn Menschheid en Godheid}{ten volle beleefd}\\

\haiku{Wat eunustus mag.}{beteekenen ben ik niet te}{weten gekomen}\\

\haiku{Eunuchus nu moet in:}{verband gebracht met het woord}{des Zaligmakers}\\

\haiku{een grauwen, jongen,,.}{en een blonden ouden met}{nieuwe vederen}\\

\haiku{Eindelijk heeft God;}{haar geleerd volcomene}{fierheit van Minnen}\\

\haiku{Als men nu overweegt,:}{dat God na te volgen in}{zijn Menschheid is}\\

\haiku{Dit kleed was gesierd,:}{met al die deugden die er}{hun naam op hadden}\\

\haiku{De uiteenzetting:}{ervan gaat duidelijk uit}{van het Schriftuurwoord}\\

\haiku{de zegels, die ze ',:}{langs buiten mett aanschijn}{verbonden waren}\\

\haiku{Sich wiltuus alsoe /}{voert meer ghebru-1110   ken}{alse ic Soe moestu}\\

\haiku{Verder wordt kort de;}{hoogheid en de klaarheid van}{dien troon aangeduid}\\

\haiku{want van een troon, of,;}{althans van een zetel wordt}{wel meer gesproken}\\

\haiku{Dat men alle dinc}{dore die claerheit van}{dien1153   trone sien}\\

\haiku{slechts 29 voor al de,!}{verloopen eeuwen maar 56}{voor haar eigen tijd}\\

\haiku{schreef, 6 liepen nog,.}{spelen en 5 zouden nog}{geboren worden}\\

\haiku{Weinig meer is er.}{te halen uit de meeste}{andere namen}\\

\haiku{naar verhouding, veel.}{minder namen van vrouwen}{noemt dan van mannen}\\

\haiku{Maar ook die kunnen,.}{kloosterzusters geweest zijn}{althans de maagden}\\

\haiku{verleyse ende.}{mijn vrouwe1328  nazaret}{kindese wel}\\

\haiku{afstaen enen -, zich op,,,;}{afstand van iem. houden hem}{ontrouw zijn 1 369}\\

\haiku{met - n werken met,,;}{de werken van innige}{vurigheid 7 19}\\

\haiku{F   familie,,,,,,.}{gevolg hofbedienden 9}{42 fenix 11 30}\\

\haiku{e.e. gheanxend, - der,,,,,.}{Minnen die in angst kwelling}{zich oefent 14 175}\\

\haiku{ghetrouwen, iem. iets,,,.}{toekennen daartoe in staat}{rekenen 8 91}\\

\haiku{hoe hi onse - in,,,;}{hemselven es ende ute}{hemselven 14 139}\\

\haiku{redeneerend verstand,,,,;}{en wat daardoor voortgebracht}{wordt 6 71 72 92}\\

\haiku{het Proza door Prof. (,.}{Dr. J. Vercoullie in 1895}{als n. 11 4e R.)1359}\\

\haiku{datten nieman soe.}{hertelike ghemint en}{heuet alse ic}\\

\haiku{Al heeft dus B bij '.}{t corrigeeren waarschijnlijk}{nog een ander hs}\\

\haiku{A telkens eke~,,,.}{C 10 ech A 9 ech 1}{ich in heilichBr}\\

\haiku{Men mag zeggen dat:}{in de rijmen de i klank}{veruit overheerscht}\\

\haiku{Door die schrijfwijze,,.}{ontstaat soms verwarring als}{goeds = goods godes}\\

\haiku{Enkele malen,, (.}{in A meer dan in C staat}{waert = woertB.v. Br}\\

\haiku{12, 98), enz. Deze.}{vormen worden gewoonlijk}{als ouder beschouwd}\\

\haiku{minne sout altoes,.}{thar wel duen waar E zoo}{schrikkelijken heeft}\\

\haiku{Daar is geen enkel.}{punt van overeenkomst in met}{E of F. ~ VI}\\

\haiku{Het kan zijn dat ik;}{hier en daar eenigszins anders}{had kunnen schikken}\\

\haiku{de redenen, die,.}{hij er voor aangeeft lijken}{me weinig dwingend}\\

\haiku{Dit voert mij tot een..}{ander bezwaar tegen de}{emendaties van Str}\\

\haiku{De algemeene:}{beschouwingen willen dit}{eenigszins toelichten}\\

\haiku{wat onwaarschijnlijk,.}{is daar ze niet aan \'e\'en stuk}{werden geschreven}\\

\haiku{Te voren nog had ();}{Hadewych gesproken van}{zijn dooddiere doet}\\

\haiku{wij moeten niet zoozeer,.}{Christus navolgen als wel}{Christus beleven}\\

\haiku{Dit zijn de twee groote,;}{mystieke toestanden die}{Hadewych vermeldt}\\

\haiku{noodzakelijke.}{eenheid van wil in minnen}{en haten met God}\\

\haiku{hare eenheid met,.}{hen die ten volle moeder}{Gods geworden zijn}\\

\haiku{de aandacht, omdat.}{daardoor het proza en de}{po\"ezie van Had}\\

\haiku{Een photographie.}{van dit zonnebeeld moet dit}{duidelijk maken}\\

\haiku{verklaard worden het.}{veelvuldig gebruik van het}{woord nuwe bij Had.1479}\\

\haiku{een nieuw sap vloeit door,.}{hun vezels en aderen een}{nieuw leven ontspruit1480}\\

\haiku{Et cum vidissem,.}{eum CECIDI AD PEDES}{EJUS tamquam mortuus}\\

\haiku{Sine HOEFT was groet;}{ende wijt ende KERSP van}{WITTER vaerwen}\\

\haiku{en dat zij niet naar,}{visioenen streefde niet}{in visioenen}\\

\haiku{In God is ook 's,:}{menschen ware leven van}{alle eeuwigheid}\\

\haiku{daar leven wij Gods,.}{eigen leven mede en}{zijn wij \'e\'en met God}\\

\haiku{- verderft helschen, hoedt,,;}{en voedt aardschen verheerlijkt}{hemelschen l. 68}\\

\haiku{{\textquoteleft}die terugkeer moet;}{gebeuren langs denzelfden}{weg als de uitgang}\\

\haiku{uitwerkselen, maakt,,;}{de ziel waardig om God te}{ontvangen 12 82}\\

\haiku{na zijn visioen,,;}{op Thabor heeft hij nooit meer}{gelachen 14 90}\\

\haiku{Op een Sinxendag {\textquoteleft}zong{\textquoteright} (,).}{men metten in de kerk en}{ik was daar7 1}\\

\haiku{Zij woonde niet in, {\textquoteleft}{\textquoteright}.}{een klooster dan warein}{de Kerk overbodig}\\

\haiku{De heeren van Diest.}{waren nauw verwant met de}{heeren van Breda}\\

\haiku{Uit een studie nu ':}{van dien katalogus van}{t Rooklooster blijkt}\\

\haiku{welnu ook deze.}{namen van de tweede laag}{komen daarin voor}\\

\haiku{Doch van veel grooter.}{belang is een andere}{plaats die in het hs}\\

\haiku{Uit die lofrede,.}{weten we mede niet slechts}{wat de kok over Had}\\

\haiku{De ipso Christo}{Sanctorum Sancto quidam}{animae illucens}\\

\haiku{Dit feest werd door Paus ({\textdagger});}{Joannes XXII 1334 voor heel}{de Kerk ingevoerd}\\

\haiku{in de rekening,,.}{rekening houdende met}{naar de mate van}\\

\haiku{vrees, bezorgdheid, niet;}{genoeg denzelfden dood te}{sterven als Christus}\\

\haiku{bij het uithouden, ().}{bij het volhardenvgl. te}{beghinnene}\\

\haiku{Men kan ook denken,}{aan het mystieke Lichaam}{van Christus tot wien}\\

\haiku{gij zult nog wel een ();}{beetje zulk een levenvan}{lijden doormaken}\\

\haiku{de lezing van C.,.}{geeft ook wel een zin maar dunkt}{me toch minder juist}\\

\haiku{enechleke vat ik:}{eerder op van den toestand}{van de zienares}\\

\haiku{309al vol onsteken.}{geheel deze voorstelling}{is apocalyptisch}\\

\haiku{dierre A omvaen (..):}{uit ontf 43 sinen 44 Doe}{45 ende alle}\\

\haiku{A gheheelre B.}{geheelre 53 B euweleke}{C nuwe ontbr}\\

\haiku{zij mag, zoo ze er,.}{eenig verschil in ontwaart den}{heerlijksten kiezen}\\

\haiku{de engel heeft haar,.}{zoo volkomen gemaakt dat}{zij niet meer twijfelt}\\

\haiku{A haer B har 83.}{sterke 345te sine of}{te siene ontbr}\\

\haiku{385na ons beide () (:}{n ghelijc = gelijk elk}{van ons beidenzooals}\\

\haiku{) 22 gene B zijn;}{23 A gratie B gescie}{25 B arbeyte}\\

\haiku{worden.dies van iets, ().}{om ietseenige verdienste}{dat ze niet hebben}\\

\haiku{is feitelijk een;}{verklarende herhaling}{van wat voorafgaat}\\

\haiku{samen (niet met Jc,):}{god ende mensche dit toch}{moet Had.'s leven zijn}\\

\haiku{hij had te lettel:}{minnen met affectien wat}{wel zal beteekenen}\\

\haiku{alle nakende;}{vervolgingen en smarten}{kunnen overwinnen}\\

\haiku{pen) ende heeren (.}{A heerscappentusschen die}{en ew geras}\\

\haiku{A auriolas (.)}{43 B geheele 44 B}{zijn B hierop ras}\\

\haiku{Wie tot die kennis,,.}{komt is zooals in 8,88 vlg. op}{weg ter heiligheid}\\

\haiku{hoe selke schinen.}{dolende ende nye ure}{daer ute en quamen}\\

\haiku{heilegen 123 A (.}{heilighen B heilegen}{B elkenop ras}\\

\haiku{811Dit is misschien,.}{ook een reden waarom ze}{niet \'e\'en met s. Aug}\\

\haiku{Daarin heeft zij nooit,.}{verpoozing volle vreugde en}{zaligheid gehad}\\

\haiku{ongrondelecheit;}{85 onderscedecheit A}{oercontse}\\

\haiku{bevestigd (in de).}{goddelijke Liefde door}{omgang met haar Zoon}\\

\haiku{door de middelste.}{vleugelen de geheele}{Minne aanschouwen}\\

\haiku{Hierdoor wordt misschien,.}{uitgedrukt dat de zeven}{gaven van den hl}\\

\haiku{223 B hogere ():}{A trouwe B geh. 224 B}{gewoutu uit n}\\

\haiku{1148sine zonder,.}{dat zij mij tormenten en}{mij niet tormenten}\\

\haiku{onder de Almacht,.}{der Liefde zitten rusten}{alle zaligen}\\

\haiku{De vriend had alles.}{over hare visioenen}{willen vernemen}\\

\haiku{Een populaire.}{heilige Brigida in}{de ME. was de hl}\\

\haiku{Men vindt, zegt hij, een.}{dergelijke signatuur}{in een aantal hss}\\

\haiku{Over de communie:}{onder beide gedaanten}{nog het volgende}\\

\section{Ben Haes}

\subsection{Uit: Moord op kasteel Valkensweerd}

\haiku{{\textquoteleft}Hendrik de Eerste{\textquoteright}!}{had ie zo maar voor de grap}{op bureau gezegd}\\

\haiku{{\textquoteright} Over 't gezicht van.}{den verdwijnenden agent gleed}{een air van gewicht}\\

\haiku{{\textquoteleft}We hebben maar niets{\textquoteright},.}{aangeraakt fluisterde de}{Indische dokter}\\

\haiku{Zo lang 't mooie weer '.}{aanhoudt blijven we nogn}{beetje hier hangen}\\

\haiku{Je weet wel h\`e, je,,!}{roept wat ik was helegaar}{in de war meneer}\\

\haiku{{\textquoteright} En de oude knecht,:}{keek den Inspecteur aan of}{hij zeggen wilde}\\

\haiku{{\textquoteleft}Je kunt tenslotte, '!}{nooit weten al ist maar}{voor de aardigheid}\\

\haiku{{\textquoteright} {\textquoteleft}Bracht de Jonker z'n '?}{nicht naar buiten of kwam ze}{all\'e\'ent huis uit}\\

\haiku{As ik ze tusse,,}{me klauwe krijg  zeg ik}{al tege Netje}\\

\haiku{Secretaris van.}{de directie der Hanborg}{Wapenfabrieken}\\

\haiku{{\textquoteleft}Tjonge, hoe is 't,.}{mogelijk z\`o iets moois heb}{ik nog nooit gezien}\\

\haiku{En toen is freule '?}{Julien poos later weer}{teruggekomen}\\

\haiku{We volgen hier 't,.}{Indische systeem ziet U}{bonnetjes schrijven}\\

\haiku{Losse verbinding?}{met officiele Britsche}{spionnagedienst}\\

\haiku{Mag ik weten, hoe '?}{U achtert feit van de}{moord gekomen bent}\\

\haiku{{\textquoteright} {\textquoteleft}Heus meheer, op ons, '.}{woord van gentelmenne me}{hebbet niet meer}\\

\haiku{{\textquoteright} {\textquoteleft}Thans eerst maar eens op{\textquoteright},.}{zoek naar Daan zonder Naam}{besliste Bolman}\\

\haiku{{\textquoteright}, bromde hij, terwijl ' ...... {\textquoteleft},?}{hijt toestel bediende}{Hallo Speur nog nieuws}\\

\haiku{Ik zal U van mijn, -.}{kant niet vragen hoe U aan}{die wetenschap komt}\\

\haiku{snoepreisje naar de '? ......}{lichtzinnige Stad aan de}{Amstel ent Y}\\

\haiku{Hij moest ook tanken '.}{enn slappe achterband}{laten oppompen}\\

\haiku{{\textquoteright} {\textquoteleft}Ik vrees, dat U de.}{ernst der situatie niet}{voldoende inziet}\\

\haiku{Maar ik twijfel er '.}{niet aan of haar onschuld zal}{aant licht komen}\\

\haiku{En nu even kalm he, ......}{anders kan ik m'n eigen}{woorden niet verstaan}\\

\haiku{{\textquoteleft}Ik zal U alles{\textquoteright},.}{vertellen hernam ze met}{licht bevende stem}\\

\haiku{er naar verlangde, '.}{z'n lege leven mett}{mijne te vullen}\\

\haiku{Freule Dudam:}{streek met de hand langs haar ogen}{en vervolgde dan}\\

\haiku{Maar jij Julie, kijk,,.}{me aan jij bent gered jij}{zult gered worden}\\

\haiku{Zoek de Kraaiepoot en ',!}{je bent de Panter ook op}{t spoor Bolletje}\\

\haiku{Weten jullie ook? ......}{onder welke naam ie daar}{staat ingeschreven}\\

\haiku{{\textquoteleft}Ik heb gehoord, dat '. '}{hiern ouwe kennis van}{me moet logeren}\\

\haiku{Allemaal \'e\'en pot,.}{nat op zeker ogenblik brand}{je je de vingers}\\

\haiku{{\textquoteright} {\textquoteleft}Niets, al bied je me '{\textquoteright},.}{n ton en hij deed een greep}{naar de tafel}\\

\haiku{Voorzichtig trachtte.}{hij het gesprek op Poolse}{Maria te brengen}\\

\haiku{Het licht flitste aan,.}{de kamer onthulde haar}{obscene geheim}\\

\haiku{{\textquoteright} {\textquoteleft}In orde Bol. 't?}{Arrestatiebevel heb}{je toch in je zak}\\

\haiku{{\textquoteright} {\textquoteleft}We zullen 'm G. ......{\textquoteright},, {\textquoteleft},?}{m\'eriten vloekte Speurik}{neem de auto h\`e}\\

\haiku{{\textquoteright} {\textquoteleft}Een goed rechercheur ',.}{moet ook een beetje vann}{wijsgeer hebben Speur}\\

\section{Paul Haimon}

\subsection{Uit: Gudela}

\haiku{{\textquoteright} {\textquoteleft}Was Ruprecht niet een,,?}{knappe jongen papa knap}{naar alle kanten}\\

\haiku{Veel bekwamer dan,.}{alle molenwerkers die}{ik ooit gehad heb}\\

\haiku{mensen laten hun?}{veld in de steek en willen}{met een geweer gaan}\\

\haiku{{\textquoteleft}Nu zal ik voor het!}{vreemde tuig tenminste niets}{behoeven te doen}\\

\haiku{Hij, die Ruprecht had, '.}{doorzien maar haar niet voort}{hoofd wilde stoten}\\

\haiku{Ze gaat weer naar haar.}{kamer om zich te kleden}{voor een lange dag}\\

\haiku{Zij wil zich haasten,.}{om het zeker te zien en}{het toch niet te zien}\\

\haiku{{\textquoteleft}Ai-ie, Fr\"aulein,{\textquoteright} en.}{ze laten de geweermond}{naar haar toedraaien}\\

\haiku{Hij kijkt lang en naar.}{alle kanten en dan draait}{hij zijn paard en wenkt}\\

\haiku{Hij rijdt rechtdoor, en.}{iedereen voorbij tot bij}{het gemeentehuis}\\

\haiku{Een meisje, dat op,.}{kostschool was geweest zag dat}{misschien maar alleen}\\

\haiku{Hij wenkte en dan:}{moet ze maar begrijpen en}{ze heeft begrepen}\\

\haiku{Ruprecht vereerde.}{de dood omdat hij een deel}{was van het leven}\\

\haiku{Het was of ze een,.}{wapen haalde om schielijk}{een aanslag te doen}\\

\haiku{Hoe kon zij nog aan,?}{iets anders denken dan aan}{haar eigen toestand}\\

\haiku{En eigenlijk was,.}{hij blij dat hij die Duitse}{knechten gezien had}\\

\haiku{het vaderland, dat,.}{nu was overgegeven was}{ook haar vaderland}\\

\haiku{Ja, dat doen ze, en.}{nu gooien zij hun bommen}{op mijn familie}\\

\haiku{Dan stond zij recht en,.}{keek de mensen aan waarvoor}{zij toch bang waren}\\

\haiku{En zij dacht, dat de.}{mens toch niet veel meer was dan}{een koe of een paard}\\

\haiku{Vader wil weten,.}{of ik in het geheim nog}{contact heb dacht ze}\\

\haiku{Het is of ik, sinds,.}{dat vliegtuig mijn naam draagt niet}{meer aard op de grond}\\

\haiku{hij gaat weer terug,.}{naar Compi\`egne als hij}{nog terug kan gaan}\\

\haiku{Het werd zo vreemd of:}{nu in haar lichaam alles}{reeds ging gebeuren}\\

\haiku{Het werd geen nacht, maar,.}{een wachten op een morgen}{die geen morgen werd}\\

\haiku{En zij hoorde weer:}{wat het laatste politiek}{gesprek was geweest}\\

\haiku{Hier hoor je niets van,.}{de oorlog dacht ze en ze}{liet ook het geweer}\\

\haiku{Heel het leven riep,.}{daar met de stem van iemand}{die haar nog lief had}\\

\haiku{{\textquoteright} en hij plantte een.}{paar stokken strak tegen de}{deur van de ingang}\\

\haiku{De wei werd als in:}{haar eerste groen gespreid rond}{hen en dan zei ze}\\

\haiku{De liefde kan men,.}{niet eens nadoen men kan ze}{nog niet vervangen}\\

\haiku{Hij toont haar aan ons,.}{soms gesluierd en soms in}{zilveren straling}\\

\haiku{Misschien ergeren,.}{zij zich daaraan het meest dat}{ze het zien kunnen}\\

\haiku{Christian heeft de,.}{brief in handen gehad maar}{hij kan niet lezen}\\

\haiku{{\textquoteright} ~ Nu zagen zij,.}{Grieta door de wei lopen en}{naar hen toe komen}\\

\haiku{Ik heb geen man meer,.}{nodig ik ben nooit aan \'e\'en}{man verslaafd geweest}\\

\haiku{Maar goed doorvoed en.}{gespierd al hadden ze dan}{geen boter gehad}\\

\haiku{{\textquoteright} De woudvrouw zwaaide.}{los en haast afwezig met}{haar hand ten afscheid}\\

\haiku{Hij dacht niet meer aan,.}{Herman die dit zijn leven}{lang zou onthouden}\\

\haiku{{\textquoteright} Haar moeder ging in,.}{de molen kijken zoekend}{of het meel er was}\\

\haiku{{\textquoteleft}Je liegt, o, je liegt,,{\textquoteright},.}{ik heb hem niet gehaald riep}{hij al rond lopend}\\

\haiku{Zich omdraaiend zag,.}{hij de ander naar binnen}{kijken naar zijn kind}\\

\haiku{{\textquoteleft}Ik heb nooit iemand,,.}{vermoord weet je maar vandaag}{zou ik het kunnen}\\

\haiku{In de kerk zou ik.}{misschien niet moe worden naar}{jou te luisteren}\\

\haiku{Was het misschien om?}{dit wat ik er later van}{terecht zou brengen}\\

\haiku{Hij had toch iets van.}{Ruprecht wanneer die over zijn}{nieuwe leer begon}\\

\haiku{Ze was ineens blij,.}{blij omdat deze man was}{zoals hij zijn moest}\\

\haiku{Ze was gelukkig,}{met dit voorbeeld van een mooi}{natuurlijk kind}\\

\haiku{diepe dingen van.}{het menselijk leven te}{kunnen bespreken}\\

\haiku{Er kwam een wilde.}{trek van teleurstelling in}{het gelaat der vrouw}\\

\haiku{De plek die ook de,.}{stroper had met een lijf zo}{koel als van een vis}\\

\haiku{{\textquoteright} {\textquoteleft}Het is Nicolaas,,.}{een boer die niet graag werkt en}{maar vast onderduikt}\\

\haiku{Misschien haalt hij wel,.}{een dokter die wil weten}{wie de vader is}\\

\haiku{Dan knipperde Grieta.}{en Gideon legde zijn}{steen op de slinger}\\

\haiku{{\textquoteright} Zijn gezicht ging open,.}{tot een grijns alsof er een}{masker overheen zat}\\

\haiku{Nu was zij alleen,.}{nog zijn zuster en niet meer}{in de eerste graad}\\

\haiku{Zij wees naar achter,,.}{naar een kleine dam door haar}{vader daar gelegd}\\

\haiku{Je komt vragen of.}{het goed is afgelopen}{met mijn ongeluk}\\

\haiku{Ze hield het niet lang,,.}{meer vol zag hij haar zware}{lichaam wankelde}\\

\haiku{Hij plaatste de lantaarn.}{op een verse hoop aarde}{en nam de schop vast}\\

\haiku{Misschien is het ook,.}{een gemene leugen want}{hij zegt de waarheid}\\

\haiku{{\textquoteleft}Neen,{\textquoteright} deed de jongen.}{alsof hij nu zijn vraag niet}{helemaal begreep}\\

\haiku{Het leek elkaar op,.}{te heffen als niet en niet}{dat vroeg en te vroeg}\\

\haiku{De auto's waren.}{reeds te zien en kwamen recht}{op de molen af}\\

\haiku{Hij was zo slecht niet,,.}{dacht ze weer hij is gewend}{te gehoorzamen}\\

\haiku{Als ik zou kunnen,.}{bidden wanneer het moet zou}{ik meer kunnen}\\

\haiku{Ze zag de rivier,.}{achter haar verglijden een}{weinig dof glanzend}\\

\haiku{Onder het oor was,.}{een bult en daaronder de}{kraag van een uniform}\\

\haiku{Die koude in haar,.}{steeg soms tot een gloeiend vuur}{dat haar verzengde}\\

\haiku{{\textquoteright} Dan knarsetandde,.}{de mond van de man die haar}{tot hier gebracht had}\\

\haiku{Jij bent de dokter,.}{niet jij bent Nicolaas en}{je bent weggegaan}\\

\haiku{Grieta is dat mens aan.}{de poort met haar geitenmelk}{en haar geitenstank}\\

\haiku{Waarom waren er,?}{zoveel achtervolgers die}{het wilden doden}\\

\haiku{Hij keek ook naar de,.}{zieke vrouw door de deur die}{was blijven openstaan}\\

\haiku{O, ik weet het al,.}{je staat reeds gereed hem aan}{hem over te geven}\\

\haiku{Hij begon nu te,.}{merken dat het bloed vlugger}{door zijn lichaam joeg}\\

\haiku{Twee ennen en een,.}{h de H van Hitler en}{de N van Nietsche}\\

\haiku{Ik heb hem horen.}{schreien en ze hebben hem}{niet meegenomen}\\

\haiku{In dat geval moet,{\textquoteright}.}{het kind uw naam dragen ging}{de dokter verder}\\

\haiku{Daarom zijn wij ook.}{zo blij na een verlossing}{met een goed einde}\\

\haiku{Dit is de stem van,.}{een die zeker vooraan in}{de kerk zit dacht hij}\\

\haiku{zo sterk als vroeger.}{vestingmuren er zouden}{hebben uitgezien}\\

\haiku{{\textquoteright} Gudela trok de;}{dekens nog meer om zich heen}{en staarde omhoog}\\

\haiku{Het was of hij met.}{de klauw van zijn hand een zwart}{duiveltje vastgreep}\\

\haiku{Maar ook andere:}{millioenen namen hem}{aan en doopten hem}\\

\haiku{En nog anderen,.}{namen hem aan en noemen}{hem rechtvaardigheid}\\

\haiku{Ze keek hem in het,.}{gelaat en zag dat hij op}{zijn lippen kauwde}\\

\haiku{Eduard had dat van.}{het kind gezegd voordat ze}{op zijn hoofd trapten}\\

\haiku{in ieder geval.}{is het geen schande wanneer}{men er geweest is}\\

\haiku{Waar het goed van eten,.}{en drinken was maar in het}{begin niet te veel}\\

\haiku{Ze begon met haar,.}{mond te trekken toen schudde}{ze neen met haar hoofd}\\

\haiku{Ik heb hem in het,.}{gezicht geslagen over zijn}{ogen en in zijn nek}\\

\haiku{Zij trok hem met een.}{schreeuw onder de resten van}{het voorhuis vandaan}\\

\haiku{Ik kon eerst in die.}{boom kruipen en dan naar de}{achterkant kijken}\\

\haiku{Het was zeker haar,.}{en zijn jongen een kind van}{vier jaar kon hij zijn}\\

\haiku{{\textquoteleft}Die ligt nu hier en.}{thuis wachten hem misschien een}{vrouw en kinderen}\\

\subsection{Uit: Jonkheid}

\haiku{het is vinnig koud,.}{en hij slaat met zijn handen}{onder de armen}\\

\haiku{Ginds komt al een troep,:}{herders de eerste van een}{heele karavaan}\\

\haiku{Hangt uwen lankrock voor, '.}{de wind Den voedstervader}{sorgt voort kind}\\

\haiku{Den lieven Jezus.}{krijt van dorst Zijn moeder geeft}{hem haere borst}\\

\haiku{Maer Joseph die was.}{heel verblijd Om dat het Kind}{niet meer en krijt}\\

\haiku{Vele anderen,:}{waren ook bang en wisten}{niet wat ze moesten doen}\\

\haiku{maar dat was nog zoo,.}{lang daar wilde zij liever}{niet meer op wachten}\\

\haiku{Het was bezaaid met,.}{een groen gewas waarover een}{man liep te eggen}\\

\haiku{Hij begroette oom.}{Gieljam hartelijk en ook}{Wale met warmte}\\

\haiku{Van daar uit gingen,.}{zij naar den stal waar het vee}{was ondergebracht}\\

\haiku{Uit de verte leek.}{zij heel mooi te zijn en haar}{tred was licht en vrij}\\

\haiku{Het jongste meisje.}{struikelde en viel hard met}{haar hoofd op den grond}\\

\haiku{Rein bekommerde,:}{zich daar echter niet om maar}{hij zei tot Wale}\\

\haiku{Het was misschien wel.}{daardoor dat hij altijd zoozeer}{leek op een diamant}\\

\haiku{Sint-Jan belooft mij,.}{veel ik zal vandaag nog meer}{gegeven krijgen}\\

\haiku{Hij was jong en blank.}{van gelaat met heel donker}{haar en wenkbrauwen}\\

\haiku{Die allen brachten.}{het er echter zonder groote}{ongelukken af}\\

\haiku{Ik dacht niet anders}{dan dat hij die daar speelde}{mij goed gezind was}\\

\haiku{Maar het was ook wel,.}{een zonderlinge zede}{die hier heerschte}\\

\haiku{Maar ik stond achter,.}{in de wei omdat ik pas}{laat was gekomen}\\

\haiku{Bij wijlen snorde.}{een tor of een kever met}{groote vaart den nacht in}\\

\haiku{- Niet goed voor u zou,.}{het zijn moest men u hier met}{mij te zamen zien}\\

\haiku{Hij wist nu geen woord,.}{meer te zeggen maar was ook}{de tranen nabij}\\

\haiku{hij zei echter geen,.}{woord maar keek daarna naar den}{kant waar Wale stond}\\

\haiku{Het werd heel stil toen:}{zij tot bij Rein en Wale}{ging en nu riep zij}\\

\haiku{{\textquoteleft}Ik zal altijd met,,.}{jou willen zijn Rein wat er}{ook mag gebeuren}\\

\haiku{Toen hij zich in zijn,,.}{bed uitstrekte merkte hij}{dat hij zeer moe was}\\

\haiku{De priester was reeds,.}{grijs aan zijn haren maar toch}{nog sterk en krachtig}\\

\haiku{Zij kropen beiden,;}{diep onder de mijt geheel}{tegen elkaar aan}\\

\haiku{Ze snelde toe, en.}{daar zag ze Rein weer verder}{draven op zijn paard}\\

\haiku{Neen, de kamillen,.}{deden het beest geen kwaad zij}{werkten zuiverend}\\

\haiku{Op het veld hing nog.}{een smoor en overal rook het}{naar gebrand koren}\\

\haiku{Ze sloeg haar armen.}{wild uit elkaar en toen nam}{haar vader haar vast}\\

\subsection{Uit: Het landgoed Solitudo}

\haiku{{\textquoteright} Johannes maande.}{hem dat hij voorzichtig moest}{zijn met zijn woorden}\\

\haiku{Dan zullen ze het.}{eerst denken aan hun vrienden}{onder de linde}\\

\haiku{Johannes die de,.}{bezoekers ontving was geen}{ogenblik onzeker}\\

\haiku{Johannes vergat.}{een ogenblik zijn honneursplichten}{toen zij bij hem kwam}\\

\haiku{Alsof vader door,.}{mijn blik gewaarschuwd was keek}{ook hij hun kant uit}\\

\haiku{Plotseling stond hij,.}{toen weer  op hij had nog}{haast niets genomen}\\

\haiku{Daar had hij plezier,.}{in die oude imker met}{zijn versleten toog}\\

\haiku{omdat hij niet eens;}{een poging had gedaan haar}{gerecht te proeven}\\

\haiku{Zebedeus was;}{weg en zou een vrouw die kon}{koken meebrengen}\\

\haiku{Kristie wees haar de.}{weg in het huis en hielp haar}{bij het uitpakken}\\

\haiku{Op onze vraag naar.}{haar naam antwoordde ze dat}{zij Philis heette}\\

\haiku{Op zijn kamer bleef.}{het licht naar Kristie's mening}{weer te lang branden}\\

\haiku{Nu denkt hij ook nog,{\textquoteright}.}{dat moeder hem is ontsnapt}{ging het door mijn hoofd}\\

\haiku{Hij had zeker nog,:}{het gordijn zien bewegen}{want opeens riep hij}\\

\haiku{Zebe zelf mocht mij.}{in opdracht van Johannes}{naar mijn bed dragen}\\

\haiku{{\textquoteright} Wist hij het nu, of?}{was hij nog verder weg van}{de werkelijkheid}\\

\haiku{Ik heb je gezegd.}{dat ook Johannes niet meer}{te vertrouwen is}\\

\haiku{Juist als na moeders,}{begrafenis liepen we}{met hem naar binnen}\\

\haiku{{\textquoteleft}Wat heb ik aan mijn,{\textquoteright},}{kinderen was zijn antwoord}{hij trok het zich aan}\\

\haiku{{\textquoteright} Onder die woorden:}{met potlood gekrabbeld had}{hij weer geschreven}\\

\haiku{Kristie zou er mij,.}{wel een vreugdig warm verslag}{van komen brengen}\\

\haiku{Dit was haar geschrift,.}{van gisteravond ze had iets}{met mij voorgehad}\\

\haiku{{\textquoteright} Zo bleef ze bezig}{en er kwam bovendien het}{een en ander uit}\\

\haiku{Dat glibberige?}{zuigdieren hun de lust uit}{merg en bloed zogen}\\

\haiku{Om Elza had op.}{Solitudo haast niemand}{zich zorgen gemaakt}\\

\haiku{Geloof jij ook dat?}{veel mannen door hun vrouwen}{groot zijn geworden}\\

\haiku{Wij doen niets dan hem.}{trachten te troosten met ons}{persoonlijk verdriet}\\

\haiku{Ik werd een feller.}{tegenstander van hem dan}{ik zelf vermoed had}\\

\haiku{Kristie fluisterde,:}{me nog toe terwijl ik naar}{een dik cahier zocht}\\

\haiku{De auto reed het.}{landgoed langzaam binnen en}{stopte bij de beuk}\\

\haiku{maar vroeg hij beleefd.}{het recept terug dat hij}{afgegeven had}\\

\haiku{Toen kwam hij weer bij,.}{hij herademde na een}{spookachtige droom}\\

\haiku{Ik ging op Kristie;}{toe die eerst als een versteend}{beeld was blijven staan}\\

\haiku{misschien ergert hem.}{dit en wil hij het dan zelf}{doen zoals het moet}\\

\haiku{Zebe knort om de;}{vele dorre takken die}{hij te ruimen krijgt}\\

\haiku{Nu heeft Johannes;}{hem gezegd dat hij spoedig}{geld moet verdienen}\\

\haiku{De eerste tijd moet,.}{hij beschrijven toen jullie}{gewonnen hadden}\\

\haiku{zijn gezicht wordt bleek,,.}{daarna rood van een grote}{spanning overtogen}\\

\haiku{We zien hem niet meer,.}{maar staren naar iets als uit}{een oud prentenboek}\\

\haiku{Misschien is hier een,{\textquoteright}.}{begrafenis nog mooi meent}{de vrouw van dertien}\\

\haiku{Als ik in de gang:}{kom zie ik het afscheid van}{vader en moeder}\\

\haiku{Ik zal mij nu al.}{met een bloemist verstaan die}{de hoogste prijs geeft}\\

\haiku{Het huis zal elke.}{vrouw te duur vinden als ze}{er nog geld bij krijgt}\\

\haiku{Ik wilde dat huis.}{echter als een present aan}{mevrouw aanbieden}\\

\haiku{Opeens horen we.}{vader met een stem die op}{een dreiging antwoordt}\\

\haiku{En daarom hebben.}{jullie de donderroede}{dus weggebroken}\\

\haiku{{\textquoteleft}Bemoei je nu ook.}{niet verder met de nieuwe}{kasteelbewoners}\\

\haiku{Ik vertrouw ze nog,{\textquoteright}.}{niet besluit vader als ze}{voorgoed weg zijn}\\

\haiku{Hij dacht bepaald iets,.}{heel slechts van mij alsof ik}{echt gevaarlijk was}\\

\haiku{{\textquotedblright} vroeg hij, ik huilde,.}{neen en hij vroeg wat ik dan}{van hem verwachtte}\\

\haiku{Zonder het moeras.}{kan het landgoed niet leven}{om zo te zeggen}\\

\haiku{Het was veel meer dan,.}{mooi het was vol majesteit}{om zo te zeggen}\\

\haiku{Bij de volgende:}{bocht houdt de chauffeur stil en}{vraagt aan zijn makker}\\

\haiku{Vanaf de wagen;}{grijpen de meisjes naar de}{zwarte vliertrossen}\\

\haiku{Nu is eindelijk.}{iets dat een succes voor hem}{zou worden mislukt}\\

\haiku{Hij steekt zijn handen.}{als vuisten even omhoog en}{slaat ze naar zijn hoofd}\\

\haiku{Het is de eerste.}{keer dat wij grote mannen}{zullen zien vechten}\\

\haiku{Hij schreeuwt en spartelt,.}{bijt mij in de handen dat}{het bloed eruit loopt}\\

\haiku{Het is of ik hem.}{naar haar zie lachen en zij}{die lach beantwoordt}\\

\haiku{Ik zie hoe de wind.}{tegen haar kleed slaat en haar}{figuur aftekent}\\

\haiku{Dat interesseert,.}{mij nog net maar dan is het}{ook genoeg geweest}\\

\haiku{Hebben ze je soms?}{ook geleerd hoe je een mooie}{vrouw moet verleiden}\\

\haiku{Zij hebben daarvoor.}{een Hubertusraam  aan}{de kerk gegeven}\\

\haiku{Dan rent moeder de.}{hut in en laat hem buiten}{voor de deur wachten}\\

\haiku{Zeker nog mooier,.}{dan toen ze kind was maar niet}{langer broos en vreemd}\\

\haiku{In haar ogen had ik.}{mij dus slechts geoefend voor}{mijn toekomstig werk}\\

\haiku{Alleen twijfel ik.}{eraan of je er ooit iets}{mee zult verdienen}\\

\haiku{Hij deed echter of;}{hij ons niet zag en verdween}{met grote stappen}\\

\haiku{was hetzelfde als.}{een mooie vrouw van haar schoonste}{sieraad beroven}\\

\haiku{Als iemand die zijn,.}{liefste verliezen gaat keek}{ik naar de eiken}\\

\haiku{Zij stonden er nog,.}{onbekend met wat er over}{hen werd bedisseld}\\

\haiku{Een gaai kon er in,,.}{dit seizoen niet zijn dan is}{het Zebe dacht ik}\\

\haiku{Alleen wil hij zich.}{zelf terugvinden tussen}{het geld en de macht}\\

\haiku{Ik zou notaris.}{moeten worden om voor jou}{een goudmijn te zijn}\\

\haiku{Ik zie een moeras}{dat bezig is zich over ons}{Goed uit te breiden}\\

\haiku{Ik beet eerst op mijn.}{tanden om de vraag nog}{tegen te houden}\\

\haiku{Johannes had weer.}{een papiertje met cijfers}{naast zich aan tafel}\\

\haiku{Ik geloof dat je.}{nog eens verliefd wordt op die}{boeren-Peter}\\

\haiku{Hij weet alles van,.}{ons veel meer dan je denkt als}{je hem bezig ziet}\\

\haiku{Stil, want Johannes.}{en Zebedeus mogen}{er niets van weten}\\

\haiku{Als hij het krijgt voor,.}{ons heb jij recht op een even}{groot deel als hij zelf}\\

\haiku{De vrouw trok zich aan.}{de kleren alsof zij ze}{wilde openscheuren}\\

\haiku{Het groene oog was,.}{het oog des verderfs dacht ik}{en wendde mij af}\\

\haiku{Als een lichtende,.}{kaars die beeft die zich verteert}{in haar binnengloed}\\

\haiku{Zo zou Johannes.}{ook gerekend hebben in}{mijn omstandigheid}\\

\haiku{In werkelijkheid.}{was er w\`el iets anders waar}{ik op moest letten}\\

\haiku{Ik wilde met haar,.}{meebidden maar ik had niets}{om aan te bieden}\\

\haiku{Het was of ik de.}{zwaarste boom uit het bos had}{horen neersmakken}\\

\haiku{{\textquoteright} {\textquoteleft}Zeg dat niet, Philis,,{\textquoteright}.}{er was nog niets gelukt kwam}{ik tussenbeide}\\

\haiku{5 Achter een beek.}{met een watermolen lag}{het bruine gebouw}\\

\haiku{Ze zwegen achter.}{mijn rug en ik voelde dat}{ze mij nakeken}\\

\haiku{Alleen aan zijn ogen,.}{zag ik iets vreemds alsof hij}{veel geleden had}\\

\haiku{De dokter zou het.}{nooit goed vinden dat ik u}{dit gesprek toestond}\\

\haiku{Mijn hoofd moet door een.}{lofwerk van spinnewebben}{als ik terugga}\\

\haiku{wilde ik vragen,.}{maar opeens waren we niet}{meer alleen binnen}\\

\haiku{In elk geval niet.}{om meneer De Roveren}{een plezier te doen}\\

\haiku{{\textquoteright} riep hij zo dat de.}{notaris het ook door het}{hout heen kon horen}\\

\haiku{De vogels die zo,;}{lang boven ons gecirkeld}{hadden vlogen weg}\\

\haiku{Wat kon Peter nog?}{meer van zijn historische}{dag hebben verwacht}\\

\haiku{Hij reikte hem mij,.}{over maar eigenlijk hoefde}{ik hem niet meer}\\

\subsection{Uit: Lentestorm}

\haiku{Waar is het rood dat,?}{verdween waar is het blauw dat}{duidt op een doode}\\

\haiku{De torens van den,.}{Wijngaardhof zijn ook rustig}{er waait geen vlag nog}\\

\haiku{Je hoort je ouden,,.}{geest die je zegt van toen van}{hoe het vroeger was}\\

\haiku{Maar Reinier van den,,.}{Branden knikt zijn zoon toe want}{hij weet dat het moet}\\

\haiku{De grijze priester,.}{die hen zegende beefde}{toen hij hen toesprak}\\

\haiku{Reinier begint van.}{het worstelen der jonge}{vrouw te vertellen}\\

\haiku{Geen moment slaat zij,.}{er acht op ofschoon hij ze}{niet kwaad gezegd heeft}\\

\haiku{ze kende immers.}{zoo Orbans gebaren en}{het tilde zoo licht}\\

\haiku{Zij waren vanzelf:}{stiller geworden in hun}{liefde-daden}\\

\haiku{{\textquoteleft}De ouders in een,{\textquoteright}:}{huis aan de kerk maar Machteld}{had hen verdedigd}\\

\haiku{Voor Machteld hoeft zij,.}{niet meer te vreezen zij is}{haar kamer niet uit}\\

\haiku{Er is even zoo iets.}{vreemds om zijn figuur en dat}{statige rijden}\\

\haiku{Nu hing alles van!}{dien koejongen af of hier}{zou gemaaid worden}\\

\haiku{Daar moest de boer weg;}{zijn en de oude boer te}{zwak om te maaien}\\

\haiku{Hij keek eens even op,,.}{naar de lucht naar de zon of}{zij nog niet weg ging}\\

\haiku{{\textquoteleft}Reinier,{\textquoteright} zuchtte zij,,.}{en staarde lang over hem heen}{als over haar leven}\\

\haiku{De meiden zeiden,.}{onder elkaar dat het nu}{wel mis zou loopen}\\

\haiku{Ze weet nu, dat ze,.}{zich dwaas gedragen heeft maar}{ze kon niet anders}\\

\haiku{Ze bleef er met heel:}{haar lichaam mee bezig en}{op eenmaal riep zij}\\

\haiku{Haar lichaam had het.}{willen inzuigen om haar}{ziel mee te voeden}\\

\haiku{{\textquoteleft}Heerlijk dier,{\textquoteright} dacht zij,.}{het wilde den nieuwen baas}{dus niet erkennen}\\

\haiku{'s Avonds komen de.}{witte avondnevels van den}{herfst om haar dwalen}\\

\haiku{eenzame uren, die....}{zij moest vullen met schoone}{vergeefsche beelden}\\

\haiku{Tot die op eenmaal,}{tezamen weefden om h\`em}{op eenmaal doortrok}\\

\haiku{En toen gebeurde,.}{alles wat in de Diepte}{maar gebeuren kon}\\

\haiku{zij was toen hij de}{koe geholpen had met het}{zware kalf en zij}\\

\haiku{Ze bijt haar tanden,.}{samen het is of ze op}{harde tranen bijt}\\

\haiku{{\textquoteright} {\textquoteleft}Moeder, kan Machteld,,{\textquoteright}.}{het helpen dat er oorlog}{is vroeg Walter dan}\\

\haiku{- Of zij vrouwen maar, ';}{namen zooals zij het eten tot}{zich nemens avonds}\\

\haiku{Zij werd heelemaal,.}{vervuld van iets zwaars iets dat}{grooter was dan zij}\\

\haiku{zijn liefde was met,.}{Frankrijk en zoo lang als dat}{was trof hem niemand}\\

\haiku{Ze slikte over haar,,:}{woorden heen een paar keer en}{dan ademde ze weer}\\

\haiku{Zou Godelief het,,?}{hebben afgestaan aan haar}{als het gekund had}\\

\haiku{{\textquoteleft}Het zal beter zijn,.}{als hij niet komt hij moet dit}{verdriet niet hebben}\\

\haiku{Zij heeft geprobeerd}{naar de grens te loopen om}{dan verder te gaan}\\

\haiku{Want haar hart schroeide.}{als werd het door twee sterke}{vuren aangetast}\\

\haiku{Meende Henricus,?}{met die nieuwe liefde dat}{de knecht haar liefhad}\\

\haiku{Ge weet het niet, oom,,?}{Henricus dat ik een knecht}{gedood heb om hem}\\

\haiku{telkens diep over het.}{lichaam of ze het met haar}{mond zou aanraken}\\

\haiku{{\textquoteleft}Het staat in de Schrift,{\textquoteright}.}{doch dan trok ze zich terug}{in een zijkamer}\\

\haiku{Hij had een diepe,.}{kap op zijn hoofd waardoor men}{zijn haar niet zien kon}\\

\haiku{Of zijn leger groot,.}{genoeg zal worden dat hij}{paarden kan krijgen}\\

\haiku{nu begint opnieuw,}{die geheime samenzweering}{en hij wenkte reeds}\\

\haiku{nooit, dacht zij, kwam hij ',.}{s nachts in dat zachte bed}{dat zij gemaakt had}\\

\haiku{Twee van haar moeten,.}{winnen en thans zijn beider}{oogen zwarte raadsels}\\

\haiku{iederen dag zag,,.}{zij den moord hier en thans thans}{brak hij uit dien mond}\\

\haiku{Men zei dat zij vuur,.}{konden eten maar dat ziet men}{wel wat zij kunnen}\\

\haiku{omdat Orban toch,!}{niet was gekomen terwijl}{hij toch vrij moest zijn}\\

\haiku{zoo ernstig en in,.}{de sfeer van zijn geval dat}{ze wel gelooven m\'oest}\\

\haiku{En staande naast zijn,,.}{paard met het hoofd geleund in}{den hals sliep hij in}\\

\haiku{de muren zijn stomp.}{als dateert hun verval van}{jaren geleden}\\

\haiku{zonder te spreken,.}{stond ze er maar z\'o\'o veel had}{zij hen nooit gezegd}\\

\haiku{het kind liet ze een.}{tijdlang huilen zonder dat}{ze er acht op sloeg}\\

\haiku{Maar dat reeds, dat hij,....}{bijna iemand gedood had}{dat reeds vond hij erg}\\

\haiku{Ik heb een mooie vrouw,.}{gezien wier man ik van het}{veld bij Leipzig droeg}\\

\haiku{Het was als in den,,.}{Bijbel Machteld en daarom}{ben ik weggegaan}\\

\subsection{Uit: De weg over de grens}

\haiku{Ook een weg over de,,.}{grens de duurste maar ook de}{gemakkelijkste}\\

\haiku{Ze werden er nog,.}{steeds rijker van en ook de}{trots zette goed aan}\\

\haiku{Je wende aan hun,,.}{taal aan hun discipline}{zelfs aan hun knevel}\\

\haiku{als er niemand was,}{en hun moeder niet op hen}{lette hadden ze}\\

\haiku{Als hij ze alleen,.}{laat gaan zal hij ze geen geld}{in hun zak geven}\\

\haiku{Plotseling ging ze.}{naar binnen en begon ze}{zich om te kleden}\\

\haiku{Bah!{\textquoteright} en de mannen.}{beginnen te schreeuwen}{als wild geworden}\\

\haiku{Zij kraaien vijftien.}{keer en dan zet de tater}{hen met moeite recht}\\

\haiku{Toen hij aan de balk.}{hing was zij trotser dan de}{haan ooit was geweest}\\

\haiku{of ben ik misschien?}{al verkocht en gebruikt hij}{mij als een slavin}\\

\haiku{Ik wou nu maar dat,.}{je die deur binnenkwam en}{dat je een paard had}\\

\haiku{Of hij dat gehoord,,.}{heeft vraagt Fr\"aulein Schaster en}{wat hij daar van denkt}\\

\haiku{{\textquotedblleft}Bitte, gn\"adige,{\textquotedblright},.}{Frau zei Drieka dan hoeft u}{dat niet meer te doen}\\

\haiku{Een verduiveld knap.}{wijfje heeft Peter Knarren}{als wijsvrouw gehaald}\\

\haiku{Peter ligt er naast,,.}{met vijftig groschen in zijn}{handen en snurkt}\\

\haiku{En hij zag dat een.}{Siegelbaron het met een}{kellner over hem had}\\

\haiku{Als het niet teveel,.}{is zullen we hem vragen}{dat hij ons laat gaan}\\

\haiku{En toen had hij zich.}{zo ver mogelijk van hen}{teruggetrokken}\\

\haiku{hij bevreesd dat zijn.}{eigen kinderen het hem}{ontroven zouden}\\

\haiku{Zij was de vrouw van,.}{Sep van Andr\'e geworden}{het kon raar lopen}\\

\haiku{Nu richtte ook Drik,.}{zich op om te tonen dat}{hij er ook mocht zijn}\\

\haiku{{\textquoteright} zei Peter nu en.}{de jongens keken hem met}{verraste ogen aan}\\

\haiku{Het was eigenlijk.}{te mooi geweest om na \'e\'en}{stuk gedaan te zijn}\\

\haiku{Zij zagen Sep die,.}{voorop liep een aanvoerder}{van een klein leger}\\

\haiku{Altijd was hij blind.}{en altijd was hij op tijd}{op de juiste plaats}\\

\haiku{Ze liep ermee naar,}{haar moeder het uit de korf}{tillend met een hand.}\\

\haiku{{\textquoteleft}Het zijn aardappels,.}{voor een varken die wij op}{kermisdag krijgen}\\

\haiku{Zij ging even naast Ria.}{zitten en legde haar een}{hand om de schouders}\\

\haiku{Hun broers kwamen thuis.}{en het eerst wat zij toonden}{waren hun handen}\\

\haiku{Roza had nog aan.}{een voorval deze middag}{niet willen denken}\\

\haiku{Ze had er niet om,.}{gehuild ze was er als door}{bevroren geweest}\\

\haiku{Hij was zo vrolijk,.}{en lustig als een jonge}{vrijer hoorde hij}\\

\haiku{Hij dacht na over wat.}{hij zou moeten doen en er}{viel hem niets meer in}\\

\haiku{Friedrich is niet.}{uit Duitsland gejaagd omdat}{hij heeft gestolen}\\

\haiku{Ze keerde zich met.}{de kandelaar in de hand}{om naar haar moeder}\\

\haiku{Eerst was ze er zelf;}{bij geweest en kon ze zien}{wat er gebeurde}\\

\haiku{Roza bleef over haar.}{vader gebogen en ze}{zag geen leven meer}\\

\haiku{{\textquoteleft}Je moet niet achten.}{dat ik zo iemand zijn}{widduwe naloop}\\

\haiku{{\textquoteleft}U moet willen dat,.}{hij leeft en voor u vecht en}{niet dat hij dood is}\\

\haiku{Wilde mijn hele.}{kop hebben om hem v\'o\'or op}{zijn kar te zetten}\\

\haiku{Hij greep naar zijn kiel,.}{stak de fles in zijn broekzak}{en verliet het veld}\\

\haiku{De broers geboden.}{Lotte dat ze zich niet als}{een wicht van elf jaar}\\

\haiku{Terwijl zij als een,.}{engel naar de kerk liep dacht}{ze aan wreedheden}\\

\haiku{Ze hield hem de soep.}{voor toen het scheen dat hij naar}{haar wilde grijpen}\\

\haiku{Immer zat,{\textquoteright} zei hij.}{en keek naar de andere}{kant dan waar zij was}\\

\haiku{anders was het Thies;}{altijd geweest die het bed}{voor hem opmaakte}\\

\haiku{Hij hield ze tegen,.}{zijn maagstreek alsof ze daar}{gekeurd moest worden}\\

\haiku{{\textquoteright} zei hij en hij schrok.}{een ogenblik omdat hij dat}{had durven zeggen}\\

\haiku{als haar vader ze.}{daar ging zoeken kon de man}{hem net ontvluchten}\\

\haiku{Op het gezicht van.}{Peter Knarren stond de angst}{in sterk reli\"ef}\\

\haiku{Hij voelde zich in.}{zijn recht aan de eigenaar}{van de hof gelijk}\\

\haiku{De boer stond met het,,.}{hoofd vooruit omdat hij te}{lang was in de deur}\\

\haiku{Emma knoopte haar.}{keurslijf je dicht en wilde}{de stal uitvluchten}\\

\haiku{Zichzelf aangeven.}{was soms even erg als iemand}{anders verraden}\\

\haiku{Hij gehoorzaamde,.}{op bevel niet omdat hij}{het er mee eens was}\\

\haiku{Het zal dan net zijn,.}{of ze voor ons luiden}{of we gaan trouwen}\\

\haiku{Dat was het lied van.}{de kermis als ze reeds meer}{dan een dag oud was}\\

\section{Ernest van der Hallen}

\subsection{Uit: Brieven aan Elckerlyc (onder ps. J. van den Wijngaerdt)}

\haiku{Draag het, indien het.}{voor uw werk of voor uw ziel}{noodzakelijk is}\\

\haiku{Wanneer de stof de.}{geest neerhaalt is de basis}{zelf der schoonheid weg}\\

\haiku{het venster zijner.}{ziel wijd open te werpen op}{het juichende licht}\\

\haiku{De opperste zin:}{en het uiteindelijk doel}{van het leven is}\\

\haiku{dat vele fouten;}{bedreven werden door hen}{die voorop gingen}\\

\section{Jacques Hamelink}

\subsection{Uit: Horror vacui}

\haiku{ze nog in leven).}{is weet ik niet waar ze woont}{en meneer Kobalt}\\

\haiku{De wagenwielen.}{knotsten en knersten over de}{ronde straatstenen}\\

\haiku{De wielen van de.}{kar rammelden daar knarsend}{en krakend overheen}\\

\haiku{Die zitten in hun.}{hol en horen toch wel waar}{het paard naar toe draaft}\\

\haiku{Daaronder was de.}{glimmende metalen knop}{van een riem te zien}\\

\haiku{Ik kon zien hoe ze,.}{slikte waarbij haar lippen}{iets vaneen gingen}\\

\haiku{Niet het geringste.}{ritselen van mijn kleding}{scheen haar te ontgaan}\\

\haiku{Het dier rekte zijn,.}{kop naar me toe en snoof}{onzeker leek het}\\

\haiku{Vertwijfeld schopte.}{en sloeg ik de geraakte}{poliep van me af}\\

\haiku{{\textquoteleft}Opgepast{\textquoteright} riep de.}{voerman met onnavolgbaar}{kwakende  stem}\\

\haiku{Af en toe gromde,.}{de voerman in zijn keel als}{een wolf hij zei niets}\\

\haiku{Wij zouden onze.}{vingers niet naar het toestel}{moeten uitsteken}\\

\haiku{Zijn stropdas hangt aan,,.}{een bedspijl zijn mouwen zijn}{los onopgerold}\\

\haiku{Ik zei dat ik het.}{doen zou als het de enige}{oplossing zou zijn}\\

\haiku{je bent ook gek net,{\textquoteright},.}{als allemaal net als die}{daar ze wijst naar mij}\\

\haiku{Hij is niet gerust,.}{er is een rimpel tussen}{zijn ogen gekomen}\\

\haiku{Dan kijkt hij naar me,,,.}{wil iets zeggen doet het niet}{knikt een paar maal kort}\\

\haiku{Nog dieper is de:}{zwarte sponsgrond die alles}{opzuigt en aanvaardt}\\

\haiku{Schijnbaar argeloos.}{preciserend met kleine}{haarfijne details}\\

\haiku{Een ochtendlijke,.}{nieuwsgierige kabouter}{gepensioneerd}\\

\haiku{{\textquoteleft}Ja{\textquoteright} zeg ik, {\textquoteleft}zeg me.}{welke ik moet gebruiken}{en hoeveel ervan}\\

\haiku{Ik word door de zon,;}{doodgestoken ruggelings}{val ik in water}\\

\haiku{Verder niet meer op,.}{de bedden overdag het is}{vaak genoeg gezegd}\\

\haiku{Zijn gezicht is nu.}{een glazig zwartstenen}{masker geworden}\\

\haiku{{\textquoteleft}dan schiet  ik je.}{met een gouden kogel en}{ik knip je snor af}\\

\haiku{De vijand at, aan,,.}{tafel met een arm op het}{blad om het bord heen}\\

\haiku{Daarna sloot ze de.}{balkondeuren en besloot}{een bad te nemen}\\

\haiku{Wat ze thans rook was.}{een tegelijk vager en}{penetranter lucht}\\

\haiku{Een ogenblik dreigde.}{mevrouw Siponelli in}{paniek te raken}\\

\haiku{Er was snachts een flink.}{pak sneeuw gevallen en het}{vroor dat het kraakte}\\

\haiku{Met het opknappen.}{van het huis bemoeide ze}{zich niet in het minst}\\

\haiku{{\textquoteright} Zijn stem had een vrij,.}{ironische klank wat ze niet}{scheen op te merken}\\

\haiku{Ze was toen op haar.}{vijftiende en als je goed}{keek nog steeds een kind}\\

\haiku{hoe die oom precies.}{aan die onderdelen kwam}{en waar hij woonde}\\

\haiku{En die had meer te.}{maken gehad met dingen}{als waarom het ging}\\

\haiku{Hij vroeg hun hoe het.}{met hun zaken ging en met}{het werk op het land}\\

\subsection{Uit: Het plantaardig bewind}

\haiku{Tenminste hij staat.}{met de armen omhoog naar}{de lucht te gillen}\\

\haiku{Ineens ren ik met.}{bemodderde voeten over}{kort fluwelig mos}\\

\haiku{Een rosse gloed rent,.}{laag over het water vreet zich}{in golven verder}\\

\haiku{Een soort walg om naar.}{beneden getrokken te}{worden bevangt mij}\\

\haiku{Bosneger, de Pok.}{en ik wendden voetenpijn}{en vermoeidheid voor}\\

\haiku{We hurkten bijeen.}{en zetten onze messen}{voor ons in het zand}\\

\haiku{Zijn handen wezen.}{iets aan ongeveer zo groot}{als een kokosnoot}\\

\haiku{Een geluid dat het,{\textquoteright}.}{maakte de stenen vlogen}{meters de lucht in}\\

\haiku{Het door kinderen.}{uitgegraven zand lag hoog}{om de zijkanten}\\

\haiku{Hij was krankzinnig,.}{maar op dat moment was hij}{er het dichtste bij}\\

\haiku{Er zwol iets in zijn.}{keel dat hij wegslikte en}{toen zat hij naast haar}\\

\haiku{(Hij had haar over zijn,.}{fantasie verteld ze had}{erom gelachen}\\

\haiku{Hij dacht vluchtig aan.}{de muur waarop de letters}{stonden geschreven}\\

\haiku{Morgenmiddag ga{\textquoteright}.}{je niet naar buiten hoorde}{hij nog achter zich}\\

\haiku{Hij bewoog alleen,.}{zijn lippen misschien zou ze}{hem ook zo verstaan}\\

\haiku{Haar haar was heel zwart,.}{hoog in dikke draaiingen}{om haar hoofd gelegd}\\

\haiku{Dan is het goed{\textquoteright} zei, {\textquoteleft}.}{zeje begrijpt de muziek}{beter dan wie ook}\\

\haiku{Hij durfde haar niet,}{aankijken alsof er nu}{iets tussen hen was}\\

\haiku{vrijdag Schichtig keek.}{hij door de deuropening van}{de school over het plein}\\

\haiku{Zijn pleegmoeder was,.}{alleen thuis en vroeg waar hij}{de schelp vandaan had}\\

\haiku{Een man die praatte.}{maar niet met de stem waarmee}{je gewoonlijk praat}\\

\haiku{Onder de tafel,.}{was een kleine vierkante}{koffer van wit leer}\\

\haiku{Mevrouw Daals was er,.}{niet over een paar dagen kwam}{ze weer zei tante}\\

\haiku{Het scheen hem nu veel}{minder iets zo gewoons als}{sigarettenlucht}\\

\haiku{Achter de deur was}{het zware monotone}{gepraat van de Stem}\\

\haiku{Haar mond die ik met.}{bramensap gekleurd heb is}{blauw en gesloten}\\

\haiku{Ik duw en trek tot,.}{het goed ligt met het gezicht}{naar boven gekeerd}\\

\haiku{En dan legt hij een:}{arm om mij heen en zegt met}{strakke ogen kijkend}\\

\haiku{Ervoor verbergen.}{gras en brandnetelvelden}{de lage toegang}\\

\haiku{Ik gaf niks om 'r.,.}{Gek maar dat met die ander}{kon ik niet hebben}\\

\haiku{Zijn dolzinnige.}{hoest klinkt tot me door als hij}{al op straat moet zijn}\\

\haiku{Haar mond is donker.}{en er zijn schaduwvegen}{om haar ogen en kin}\\

\haiku{Een dag of drie moet.}{hij toen al zo geweest zijn}{volgens de dokter}\\

\haiku{Hij ligt met de tong.}{uit de bek te reutelen}{als een stervende}\\

\haiku{je was bang, je kroop,{\textquoteright}.}{heel dicht tegen me aan het}{maakte mij ook bang}\\

\haiku{Deze gebaarde.}{Josias daarop voor hem}{uit te gaan lopen}\\

\haiku{{\textquoteleft}Josias Mure{\textquoteright}, {\textquoteleft}.}{zei hij bij zichzelfblijf wat}{er ook gebeurt kalm}\\

\haiku{De doos stond op de,.}{bovenste plank tussen twee}{stapels linnengoed}\\

\subsection{Uit: De rudimentaire mens}

\haiku{Misschien word ik wel,,.}{helemaal een sneeuwpop dacht}{hij goeie genade}\\

\haiku{Ze zuchtte diep en.}{stak hem haar lippen toe om}{gezoend te worden}\\

\haiku{Hun tot diep in de.}{aarde reikende wortels}{waren dezelfde}\\

\haiku{Op een goede dag.}{moest de proviandvoorraad}{aangevuld worden}\\

\haiku{Het vee was sterk en.}{vermenigvuldigde zich}{verwonderlijk snel}\\

\haiku{Geen hand reikte naar.}{een zorgeloos op het gras}{gegooid kledingstuk}\\

\haiku{De onderste tak.}{bevond zich meters boven}{de begane grond}\\

\haiku{het zou zich wel eens.}{kunnen ontpoppen in een}{fikse regenbui}\\

\haiku{Dat was al heel wat,.}{met dat feit als uitgangspunt}{werd veel mogelijk}\\

\haiku{In een duizel van.}{geluk en genot liet ze}{de handen begaan}\\

\haiku{Alles kon nog heel,,.}{goed terechtkomen ook nu}{juist nu welbeschouwd}\\

\haiku{De ruiters hadden.}{nietszeggende anonieme}{stoppelgezichten}\\

\haiku{Die boer pit terwijl,.}{hij op zijn poten staat als}{een paard verdomme}\\

\haiku{{\textquoteleft}Bedankt vrouw{\textquoteright} zei de.}{magere ruiter op niet}{onbeleefde toon}\\

\haiku{Naar zijn mening bleef:}{een mens altijd een mens en}{dat wilde zeggen}\\

\haiku{Maar het geweer dat.}{nog steeds tussen zijn knie\"en}{stond hinderde hem}\\

\haiku{Haar veestapel was.}{ze tengevolge van de}{oorlog kwijtgeraakt}\\

\haiku{Om hem van ons te.}{kunnen kopen liet ze hem}{die pot opgraven}\\

\haiku{De vrouw kon ze niet,.}{altijd zien hoe scherp haar ogen}{anders ook waren}\\

\haiku{Hij hervatte het.}{zoeken naar de verloren}{autosleutels niet}\\

\haiku{{\textquoteleft}Ik heb u dag aan,.}{dag om een teken gevraagd}{nu al twee jaar lang}\\

\haiku{ik heb het nu al:}{zo vaak gezegd dat ik het}{beu ben geworden}\\

\haiku{Nu eens hield hij hem}{zo dicht bij zijn gezicht dat}{het een wonder was}\\

\haiku{U zult wel gemerkt.}{hebben dat het zelfs al niet}{meer waait in dit bos}\\

\haiku{Hij stak zijn handen.}{uit om de regen beter}{te kunnen voelen}\\

\haiku{Met de angst van het}{voor bossen bevreesde kind}{dat hij geweest is}\\

\haiku{Nog steeds, maar flauwer,.}{al om zich heen slaand raakte}{hij snel uitgeput}\\

\haiku{De hoeven kwamen.}{recht op hem af en deden}{de grond daveren}\\

\haiku{Het hield zijn mondje.}{geopend en de oogjes}{waren gesloten}\\

\haiku{Wanneer de smaak haar.}{niet beviel spuwde ze het}{achteloos weer uit}\\

\haiku{Ook zij wierpen zich.}{op de grond en bedankten}{hem op hun manier}\\

\section{Maarten 't Hart}

\subsection{Uit: De droomkoningin}

\haiku{Het zijn er evenveel.}{als het volk Isra\"el dat}{door de woestijn trok}\\

\haiku{{\textquoteright} {\textquoteleft}Nee, maar wel heel licht,.}{en hij heeft moe even met een}{vleugel aangeraakt}\\

\haiku{Ik kon er alleen,}{maar naar luisteren ik kon}{het niet meezingen}\\

\haiku{{\textquoteright} In de kamer zat.}{een onbekende die me}{vriendelijk aankeek}\\

\haiku{Nu linksaf en dan,,.}{nog eens linksaf dacht ik maar}{toch aarzelde ik}\\

\haiku{{\textquoteleft}Ik woon op de dijk,{\textquoteright}, {\textquoteleft}.}{zei zekunnen we een heel}{eind samen lopen}\\

\haiku{Het was anders dan,.}{die andere muziek maar}{toch leek het erop}\\

\haiku{Ik weet het niet, ja,,.}{als het kon als we thuis een}{piano hadden}\\

\haiku{Voor het zover is,.}{probeert ze eerst om mij te}{leren klokkijken}\\

\haiku{Maar zolang als ik,:}{nog niet kan klokkijken kan}{zij ook niet zeggen}\\

\haiku{Ik verlaat mijn stoel,,.}{en loop met ingehouden}{adem naar de radio}\\

\haiku{Het huisje stond zo.}{dicht bij het pad dat ik de}{stemmen kon horen}\\

\haiku{zag ik er dan zo?}{onbenullig en weinig}{vreesaanjagend uit}\\

\haiku{{\textquoteright} {\textquoteleft}Nee,{\textquoteright} zei ze lachend,, {\textquoteleft},.}{en met iets van spot in haar}{stemnee dat kan niet}\\

\haiku{het gaat er alleen.}{maar om dat je van  de}{ander een Bach maakt}\\

\haiku{{\textquoteright} {\textquoteleft}Klassieke rotzooi,,!}{gadverdamme hadden we}{net zo leuk Elvis}\\

\haiku{her en der brandden.}{in het holst van de tuinen}{nog kleine lichtjes}\\

\haiku{Vandaag ben ik voor,.}{het laatst maandag komt jullie}{eigen bakker weer}\\

\haiku{Daar verliet ik het}{pad en omcirkelde ik}{zo snel mogelijk}\\

\haiku{Ik had les op de,.}{muziekschool bij een oude}{reusachtige vrouw}\\

\haiku{Het klonk als een oud,,.}{van ver voor mijn geboorte}{daterend bevel}\\

\haiku{hebben jullie er?}{weer \'e\'en betrapt die hier over}{het erf wou lopen}\\

\haiku{de wet van Metten.}{Anker is net zo'n wet als}{de wet van Hubble}\\

\haiku{Als mijn vader, vlak,:}{voor het avondeten de krant las}{en soms opmerkte}\\

\haiku{Voor het overige '.}{breide zes avonds als hij}{sliep zwarte sokken}\\

\haiku{Over mijn zuster kan.}{ik nog minder vertellen}{dan over mijn ouders}\\

\haiku{Toch bestaat ze niet,,:}{echt voor me of nee ik moet}{het anders zeggen}\\

\haiku{{\textquoteleft}Kom hier,{\textquoteright} en bleef staan.}{wachten terwijl de hond ging}{liggen op liet pad}\\

\haiku{dat ik mijn handen:}{vouwde en mijn ogen sloot en}{aldus God aanriep}\\

\haiku{bij een korter stuk.}{zou ze nooit kunnen laten}{horen wat ze kan}\\

\haiku{Daar moet je kunnen.}{horen dat de violist}{door smart overmand wordt}\\

\haiku{Zij speelde goed, maar.}{hij kraste erop los als}{een beginneling}\\

\haiku{Vervolgens vragen?}{of ze zin heeft om iets met}{me te gaan drinken}\\

\haiku{Ze rukte zich los, {\textquoteleft}{\textquoteright}.}{zeihou je handen thuis en}{vervolgde haar weg}\\

\haiku{die geur laat zich zelfs.}{niet door slecht trekkende open}{haarden verdrijven}\\

\haiku{{\textquotedblleft}Het is geen nette,,{\textquotedblright}.}{jongen Rens en dat vind ik}{zo verschrikkelijk}\\

\haiku{{\textquoteleft}Begrijp jij nou dat?}{de meeste mensen zo lang}{in hun bed blijven}\\

\haiku{Toch bleef zij hem slaan.}{met de riem totdat zij om}{een straathoek verdween}\\

\haiku{Het bloed gonsde in.}{mijn oren en mijn hart beukte}{tegen mijn borstbeen}\\

\haiku{En als je me nog,.}{twee geeltjes geeft maken we}{er echt een feest van}\\

\haiku{Vijftig gulden, dacht,.}{ik om te zien hoe iemand}{twee strikjes losknoopt}\\

\haiku{En als ze  over,,?}{me droomt sta ik er toch ook}{buiten waar of niet}\\

\haiku{het lijkt me niet de.}{weg om hoofdredacteur van}{een krant te worden}\\

\haiku{Dat artikel waar,}{ik het zo net over had was}{door jou geschreven}\\

\haiku{{\textquoteleft}We laten u eerst.}{het eerste deel uit het 22Ste}{trio van Haydn horen}\\

\haiku{Maar we zien elkaar,.}{misschien bij een overweg ik}{zal naar je wuiven}\\

\haiku{We liepen naar de.}{tafel waar de cassettes}{werden uitgereikt}\\

\haiku{{\textquoteright} {\textquoteleft}Nee, eigenlijk niet,.}{maar misschien begrijp ik niet}{goed wat je bedoelt}\\

\haiku{Wat vreemd,{\textquoteright} zei ik toen, {\textquoteleft}.}{we buiten warendat er}{niemand anders was}\\

\haiku{Het trok mijn aandacht.}{omdat het opeens geluid}{leek voort te brengen}\\

\haiku{{\textquoteright} En ook altijd zo.}{geweest en zelden zo sterk}{als op deze avond}\\

\haiku{Ze reikte me de,:}{rode wijn aan zocht naar de}{plaat en riep me toe}\\

\haiku{De kaart viel half open,,.}{vouwde zichzelf uit gleed toen}{van de tafel af}\\

\haiku{Een vrouw die mij even?}{zal laten merken dat ze}{sterker is dan ik}\\

\haiku{Je mag wel aan de,.}{dood denken maar je mag nooit}{aan zelfmoord denken}\\

\haiku{Maar het enige waar,.}{het opaan leek te komen had}{ik hem niet gezegd}\\

\haiku{Ik streelde haar nog,.}{steeds maar dat leek me niet de}{juiste methode}\\

\subsection{Uit: De vrouw bestaat niet}

\haiku{hoe ik ertoe kwam.}{er een bepaalde mening}{op na te houden}\\

\haiku{Zij was het enige.}{wezen op de wereld voor}{wie Krijnie bang was}\\

\haiku{Dat zorg en macht zo.}{nauw samenhangen is niet}{onbegrijpelijk}\\

\haiku{En eerst dan zal een.}{machthebber gebruik maken}{van zijn lichaamskracht}\\

\haiku{Is misschien dan toch:}{waar wat Harry Mulisch in}{een interview zei}\\

\haiku{Als hij een vaasje,.}{op de schoorsteen verschuift krijgt}{hij een grote mond}\\

\haiku{Volgens Joke Smit {\textquoteleft}:}{is depsychologische}{kern van het probleem}\\

\haiku{Voor de Goud-Elsje.}{serie liet ik de Bob Evers}{serie graag rusten}\\

\haiku{{\textquoteleft}De muziek behoort;}{tot eene geheel andere}{orde van zaken}\\

\haiku{Vrouwen konden hun.}{gecomponeerde werken}{ook publiceren}\\

\haiku{omdat ik wel, en,.}{zij niet later een beroep}{zou mogen kiezen}\\

\haiku{Overal om me heen.}{zie ik echter bewijzen}{van het tegendeel}\\

\haiku{Op 14 juni 1963:}{kocht ik een boek waar ik al}{lang naar had gezocht}\\

\haiku{dit soort werk diende.}{eerlijk over beide partners}{verdeeld te worden}\\

\haiku{Vooral Dostojewski is;}{in die dagen bij mij door}{de mand gevallen}\\

\haiku{Een dwaze maagd van,.}{Ida Simons maar daarna was}{het afgelopen}\\

\haiku{Toen ook daagde het.}{besef dat ik misschien zelf}{zou kunnen koken}\\

\haiku{Overigens zijn de.}{begrippen rotzooi en troep}{maar betrekkelijk}\\

\haiku{Maar, zou men kunnen,.}{zeggen de onderwerpen}{zijn toch heel anders}\\

\haiku{Ja, maar vergeet niet.}{dat Van der Meyden een heel}{snelle auto heeft}\\

\haiku{Ze weten het zelf:}{niet en daarom zal ik hen}{nu maar eens helpen}\\

\haiku{Overigens is het}{uitermate opvallend}{hoe merkwaardig groot}\\

\haiku{{\textquoteright} {\textquoteleft}Verbaast ons vaak{\textquoteright} klinkt {\textquoteleft}{\textquoteright}.}{toch bepaald anders dan het}{apodictischeis}\\

\haiku{{\textquoteright} Dat schreef Jung, deze.}{neo-platonische}{vernieuwer in 1934}\\

\haiku{Dat blijkt wel uit De.}{zwembadrnentaliteit van}{Andreas Burnier}\\

\haiku{Wie niet voor mij is,,.}{is tegen mij zei Jezus}{al zo mooi simpel}\\

\haiku{Op pagina 86:}{ongetwijfeld de mooiste}{uitspraak uit het boek}\\

\haiku{Alleen geloof ik.}{niet in haar oplossing of}{haar oplossingen}\\

\haiku{Zijn roman speelt in,.}{Boston aan het einde van}{de vorige eeuw}\\

\haiku{Daarom lijkt het mij.}{goed de raad van Virgina}{Woolf op te volgen}\\

\haiku{En een woord zonder,.}{betekenis is een dood}{woord een bastaardwoord}\\

\section{Henri Hartog}

\subsection{Uit: Sjofelen}

\haiku{- Ze kon 't pleizier,,.}{dat ze tot nogtoe had in}{d'r trouwen best op}\\

\haiku{As-t-ie ze nou,.}{niet kreeg kon zij ze voor zijn}{part wel opzouten}\\

\haiku{Maar morgen, dan zou,.}{ze gaan ze zou er geen gras}{over laten groeien}\\

\haiku{De meester, die sterk, '}{aan den draad trok wass avonds}{nog al eens gepoetst}\\

\haiku{Vrouw Scharrewou riep '.}{zachtjes zijn moeder en gaf}{diet lampje over}\\

\haiku{ze mosten van goeie,.}{huize zijn wouen ze hem}{d'r onder krijge}\\

\haiku{'t was toch al de,.}{vijfde dag ze most zich nou}{alleen maar redden}\\

\haiku{{\textquoteright} En Van Deesem, die,:}{ook buiten was gekomen}{zei tegen vrouw Muis}\\

\haiku{Maar nu kwam de Groef,.}{een woordje meepraten die}{naast vrouw Muis woonde}\\

\haiku{Hij was een beetje.}{in de olie en dan kon je}{niet voor hem instaan}\\

\haiku{Vrouw Van Deesem kwam,.}{nu terug met d'r neef die}{de huur ophaalde}\\

\haiku{Ze waren op 't.}{laatst allemaal van angst de}{deur uitgeloopen}\\

\haiku{'t jonge wijf had.}{ze an de buurvrouw verkocht}{voor een dubbeltje}\\

\haiku{Ze had jarenlang.}{op d'r zelve gewoond in}{een woning alleen}\\

\haiku{Ze kon toch niet over,.}{d'r hart verkrijgen om niet}{even te blijven staan}\\

\haiku{Toen vrouw Muis weer naar, '.}{binnen zou gaan kwam net de}{Groef langst portaal}\\

\haiku{Vrouw Muis was alweer,.}{weggesuft op haar stoel toen}{de jongen thuiskwam}\\

\haiku{Hei, scheeve, hoor'es even,,....}{ik geef-ie een kwartje as}{je op je broek trapt}\\

\haiku{Wel allemachtig,.}{d'r stond een groote knolraap in}{een pot v\`ol water}\\

\haiku{Ze kon niet velen,.}{dat-ie d'r an raakte}{door die rematiek}\\

\haiku{{\textquoteright} Jonge Miet werd kwaad, ', '.}{hoe had zet nou ze had}{t toch zelf gewild}\\

\haiku{hier en daar tegen, ';}{an geworpen en int}{bed zat ouwe Miet}\\

\haiku{Maar aan den overkant.}{waren de bovenhuizen}{in vieren verdeeld}\\

\haiku{{\textquoteleft}Daar leit een smeris ',,....{\textquoteright}}{int water verzuip nou}{maar verzuip nou maar}\\

\haiku{voeten slepen de,;}{straat voetpunten tjiepten als}{ingezet getjilp}\\

\haiku{Zij werd nuchterder,,;}{zij wist dat groote lui meisjes}{met cente namme}\\

\haiku{Die z'n vader of,.}{z'n moeder vermoord had was}{daar nog te goed voor}\\

\haiku{As 't een nette,?}{jongen was waarom bracht ze'm}{dan niet mee naar huis}\\

\haiku{En zonder dat zij,.}{er erg in hadden had-ie}{legge luisteren}\\

\haiku{Als ze 'm nou weg,.}{wou sturen dan most ze d'r}{maar veel over teemen}\\

\haiku{Toen die zestien jaar,.}{was liep-t-ie al na die}{verdomde hoeren}\\

\haiku{hij kon nog net met.}{zijne \'eene hand de Moer}{naar zijn keel grijpen}\\

\haiku{Louise boog zich naar,.}{de Brakel toe die zijn arm}{om haar hals legde}\\

\haiku{ze zal 'm wel voor.}{je in de watte legge}{op de beddeplank}\\

\haiku{dat is z'n vrouw, ze,...}{liep zoo driftig d'r na toe}{net of ze kwaad was}\\

\haiku{Nou mosten ze toch.}{met oneerlijke lui te}{doen gehad hebben}\\

\haiku{Geen een kwam d'r in,,.}{of hij most vooruit weten}{dat ze betaalden}\\

\haiku{dat is vroeger zoo,.}{geweest maar tegenwoordig}{is dat veranderd}\\

\haiku{{\textquoteright} En toen volgde hij}{zijne vrouw en achter haar}{pittigen opstap}\\

\haiku{{\textquoteright} Maar dat viel ook al,.}{tegen dat was alweer eene}{verkeerde tactiek}\\

\haiku{waar het zoo straks nog,.}{blauw was lagen er nu als}{schepen gegroepeerd}\\

\haiku{om haar oogen, die soms.}{heel dankbaar-zachtmoedig}{hem even a\`anschouwden}\\

\haiku{Hij vond, dat-ie,.}{meer vastigheid had als zij}{dat nu bepaalde}\\

\haiku{Dit verplichtte hem.}{op al deze wegen zijn}{aandacht te richten}\\

\haiku{zag veel jongens met,.}{meisjes die erg in hun schik}{leken met mekaar}\\

\haiku{Even, als een slok in.}{een nauwe keel was het dicht}{vallen van het slot}\\

\haiku{Dat ging dus in den.}{beginne allemaal van}{een leien dakkie}\\

\haiku{Ze kon moeilijk an ',.}{t fabriek blijven tot ze}{op alle dag liep}\\

\haiku{Ze zou natuurlijk,.}{zeggen dat de jongen te}{lang onderweg bleef}\\

\haiku{Maar die vent scheen maar,.}{niet beter te worden liet}{niks van zich hooren}\\

\haiku{En dan wisten ze,.}{wel hoe ze de pelisie}{op d'r hand kregen}\\

\haiku{anderen keken,,,.}{toe of ze kieskeurig ook}{fouten opmerkten}\\

\haiku{Post stond dicht bij haar,,,.}{in zijn boezeroen blootshoofds}{met Truus op zijn arm}\\

\haiku{{\textquoteright} vroegen ze, {\textquoteleft}wil je.}{vier dagen blijven of acht}{dagen of een maand}\\

\haiku{{\textquoteright} {\textquoteleft}Zeg voor mijn part wat,.}{je wil ze zijn allang naar}{de Godverdomme}\\

\haiku{As-t-ie dronken,,....}{was zocht-ie ruzie in de}{kroeg wou dan vechten}\\

\haiku{Zeg, zeit ze, die vent,,:}{van jou die lust ze ook koud}{en ik zeg nog zoo}\\

\haiku{{\textquoteleft}Ja maar, zoo as jij ',, '....}{t heb zoo'n enkel nachie}{jij hebt bed ruim}\\

\haiku{In Leie as meissie,'.}{zijnde was ze ook's bij een}{kaartlegster geweest}\\

\haiku{En dus besloot ze.}{morgen met de Sluische}{d'r man te prate}\\

\haiku{{\textquoteright} {\textquoteleft}'k Ben toch 's gaan,.}{hooren bij me zuster of}{die ergens van wist}\\

\haiku{hoe kort nog en dan,.}{liepen ze rond net zoo}{goed of ze dood was}\\

\haiku{Lekkerder dan al,.}{de menschen  die gedaan}{hadde gekrege}\\

\haiku{Je mocht 't wel, as ',.}{jet maar zoo dee dat ze}{je niks konden doen}\\

\haiku{Betaal de  lui,,,.}{maar die an je deur komme}{mane dweil stinkert}\\

\haiku{{\textquoteleft}ik hou wel van een,,, '}{schoone man wat jij ik wou}{dak er van nacht}\\

\section{Fran\c{c}ois Haverschmidt}

\subsection{Uit: Winteravondvertellingen}

\haiku{Deze gevaarten,,.}{bewogen maakten geluid}{en wezen op mij}\\

\haiku{Hoe de 1e kat hier,.}{te lande gekomen is}{is dood eenvoudig}\\

\haiku{Maar nu stond eensklaps,.}{Jelle's vader Henrij Gall op}{en vatte het woord}\\

\haiku{misschien wordt Klaauw nog,...!}{eens primus der provincie}{en dan welk een eer}\\

\haiku{Ach, ik dacht niet dat.}{kattenberekeningen}{meermalen falen}\\

\haiku{Het is wel slechts het,:}{woord van een  kat maar het}{is toch welgemeend}\\

\haiku{Vooraf evenwel nog, -,.}{\'e\'en woordje dat beteekent nog}{eenige volzinnen}\\

\haiku{Als gij er soms in,.}{mocht staan dan zal ik niets dan}{mooi's van u zeggen}\\

\haiku{Wie altoos met de,}{naakte waarheid voor den dag}{komt brengt het niet ver.}\\

\haiku{Plakken niet velen?}{bepaald valsche etiquetten}{op hunne flesschen}\\

\haiku{Ze tooit anderen.}{en wil ze wat meer laten}{schijnen dan ze zijn}\\

\haiku{Iemand, van wien ik,,,:}{durf wedden dat als hij sterft}{in de krant zal staan}\\

\haiku{Vooral voor iemand,.}{die getrouwd was met den man}{van jufvrouw Wawel}\\

\haiku{Iedereen weet, er,:}{zijn twee manieren om in}{opspraak te komen}\\

\haiku{Zij staat niet aan het;}{hoofd van philanthropische}{vereenigingen}\\

\haiku{- Wij kunnen deze.}{onbeduidende menschen}{dus gerust overslaan}\\

\haiku{Het zij verre van,,.}{mij hun talent hun genie}{te betwijfelen}\\

\haiku{Doch wat bewijst dit,, -, -?}{dan dat uw smaak niet vergeef}{het mij niet fijn is}\\

\haiku{Neen, schildknaap Kuno ',.}{wast veeleer Aan wien ze}{heur hartje gaf}\\

\haiku{Ziedaar een waardig!}{opvolger van den grooten}{zanger van zooeven}\\

\haiku{Zonder schoon te zijn,',.}{hadden zijn trekken iets edels}{iets beminnelijks}\\

\haiku{- Veel, zeide ik, was,,,.}{er dat hem aantrok veel ook}{wat hem van zich stiet}\\

\haiku{Hij koesterde een',.}{stille liefde voor alles}{wat schoon is en goed}\\

\haiku{- Wat zal ik er nog,?}{meer van zeggen om u zijn}{beeld te voltooien}\\

\haiku{Mogelijk was hij,,.}{zonder het te vermoeden}{aller geweten}\\

\haiku{Maar zijn plichtbesef,.}{noodzaakte hem van dezen}{wensch afstand te doen}\\

\haiku{Mijn geheele hart;}{is ingenomen door mijn}{academievrienden}\\

\haiku{Daar komen  toch.}{zeker alleen de echte}{letterkundigen}\\

\haiku{En toen heeft hij ook.}{een toast ingesteld op het}{jonge Vlaanderen}\\

\haiku{- Ik ben stom genoeg.}{om mij in verlegenheid}{te laten brengen}\\

\haiku{en als een ander}{wat zegt veroorlooft hij zich}{telkens ter zijde}\\

\haiku{Intusschen - ik heb.}{mij op den bewusten avond}{niet enkel verveeld}\\

\haiku{ik gaf wat, als ze.}{gedrukt waren en ik ze}{nog eens lezen kon}\\

\haiku{Maar, hoopte ik, gij,,.}{zoudt wel met iets dat minder}{was tevreden zijn}\\

\haiku{Maar terstond laat hij {\textquoteleft},.}{er op volgenKom ik ga}{er ook nog eens zien}\\

\haiku{- Zoolang haar Jan nog,...}{leefde had ze altoos nog}{goeden moed gehad}\\

\haiku{Maar Jan heeft nu wel.}{wat anders te doen dan zich}{te laten kussen}\\

\haiku{- Maar al genoeg - want}{ik kan mij nu niet langer}{met je ophouden}\\

\haiku{'t Was toch wat een,.}{deftige majoor en hij}{reed op een wit paard}\\

\haiku{Maar gij zeidet, dat.}{hij werk genoeg had om er}{niet af te vallen}\\

\haiku{Eigenlijk mag ik,.}{niet zeggen dat de jufvrouw}{haar meiden versleet}\\

\haiku{Zij deed er mee als.}{roekelooze jongeheeren}{met hun sigaren}\\

\haiku{Want versch waren ze,.}{juist niet de nieuwe meiden}{van mijn hospita}\\

\haiku{De jongste van uw.}{oudste kind is ouder dan}{uw eigen jongste}\\

\haiku{Of ik niet zie, dat?}{er een heele boel verkeerds}{in de wereld is}\\

\haiku{Kortom, wie zegt, dat,.}{gij geen gevoel hebt die moest}{zich liever schamen}\\

\haiku{Al was het de neus,,.}{van den burgemeester ik}{vrees dat hij plat ging}\\

\haiku{En nooit zondigde,.}{hij tegen het rijm evenmin}{als tegen de maat}\\

\haiku{{\textquoteleft}laat mij het om 's,}{hemels wil niet verraden}{wat voor een engel}\\

\haiku{- Gij hebt haar, zonder,;}{het te willen of het te}{weten gemankeerd}\\

\haiku{Hij liet zich dus niet;}{anders betitelen dan}{hem eerlijk toekwam}\\

\haiku{Want als zij in het,.}{water valt dan vertrekt de}{schavuit geen gezicht}\\

\haiku{Maar ach, hoe kan ik,?}{nog schertsen terwijl ik van}{den kleinen Bob spreek}\\

\haiku{- z\'o\'o iemand is noch,,.}{het een noch het ander is}{noch lui noch lekker}\\

\haiku{En dit laatste is,.}{het geval niet alleen met}{u maar ook met mij}\\

\haiku{Hoe ook aangelengd,.}{dat bedwelmend vocht voltooit}{onze ellende}\\

\haiku{Vertering had de.}{vreemde in het logement}{niet kunnen maken}\\

\haiku{Die roode kleur deed.}{vreemden wel eens vermoeden}{dat Mollemans dronk}\\

\haiku{die had dan een in '.}{t oog vallenden aanleg}{voor een beroerte}\\

\haiku{Het lijk opent de oogen,,}{de mond ontsluit zich en geeft}{een schreeuw de blaker}\\

\haiku{Hij moest met Trui en?}{den kleine bij Oom komen}{inwonen en Bram}\\

\haiku{De afwezigheid,.}{van den Baron is juist een}{goed teeken Mevrouw}\\

\haiku{Hoe d\`at: wat moet hij?}{op dit oogenblik met een}{geladen geweer}\\

\haiku{Hij stapte op de,:}{deur los belde en vroeg toen}{hem werd opengedaan}\\

\haiku{{\textquoteright} En met dat woord werd.}{hem de deur onzacht voor den}{neus toegesmeten}\\

\haiku{Bedenk gij zelf u,.}{maar eens of gij er niet wat}{op kunt verzinnen}\\

\haiku{wat trouwens ook niet,.}{behoefde want het was maar}{een meidenkamer}\\

\haiku{Z\'o\'o kreeg zij althans.}{een aanwijzing waar ze de}{baker vinden kon}\\

\haiku{Doch, alsof het er,}{Nol om te doen was geweest}{\`en den professor}\\

\haiku{{\textquoteright} De heer Frieseman had.}{een beetje moeite om het}{\'o\'ok niet te vinden}\\

\haiku{De tweede week na.}{zijn komst op het instituut}{kreeg Nol den kinkhoest}\\

\haiku{Zeker, er waren;}{sentimenteele ladie's}{en ook niet-ladie's}\\

\haiku{{\textquoteleft}zal je oppassen,,?}{buurman dat Johnny}{geen ongeluk krijgt}\\

\haiku{{\textquoteleft}Geef nu je moeder,,!}{maar een zoen ventje en dan}{gaan wij er van door}\\

\haiku{Zij dacht aan den prins,}{die beter geworden was}{en hoe gelukkig}\\

\haiku{En toen zag kleine.}{John wat geen mensch in de}{wereld gezien heeft}\\

\haiku{Er zijn menschen, die.}{liegen om u het geld uit}{den zak te kloppen}\\

\haiku{Toen proponeerde,.}{ik Heintje om haar in de}{waschkuip te smijten}\\

\haiku{Eindelijk werd ik,}{er melankoliek onder}{en op een goeien}\\

\haiku{En omdat we dat,;}{niet wisten deden wij dan}{ook geen van beide}\\

\haiku{Eenigen waren er,.}{blijkbaar voor maar anderen}{waren er tegen}\\

\haiku{Het beste was, je.}{maar in het geheel niet met}{hem te bemoeien}\\

\haiku{Het zat hem in zijn,;}{hart over die onverwachte}{pensioneering}\\

\haiku{Ik moest een ander,.}{in mijn plaats stellen maar dat}{ging niet op den duur}\\

\haiku{Als ik mijn boek voor,.}{mij heb dan kan ik naar geen}{praatjes luisteren}\\

\haiku{{\textquoteleft}Ferm, toe maar jongens,!}{laat hem geen duit houden van}{zijn gestolen geld}\\

\haiku{Of de advokaat,.}{het al buiten hem om deed}{dat gaf niet genoeg}\\

\haiku{Het is er mee, als.}{met de schilderstukken van}{mijn vriend van der Kwast}\\

\haiku{En gelukkig, als.}{men er niet midden in den}{nacht wakker van schrikt}\\

\haiku{{\textquoteleft}Het is een soort van,,:}{fantasie Oom en ik heb}{er boven gezet}\\

\haiku{ik U \'o\'ok toe (wel,).}{te verstaan dat bord pap op}{uw 80sten verjaardag}\\

\haiku{Als het maar mooi was,.}{en dat waren de verzen}{van Lucas stellig}\\

\haiku{{\textquoteleft}Pas eens op wat ik,,{\textquoteright};}{je zeg moeder als de wind}{straks niet gaat krimpen}\\

\haiku{ik was, hoe kort mijn,,}{geluk duurde en hoe diep}{diep ongelukkig}\\

\haiku{Zij was berekend,.}{voor elke aandoening voor}{iedere schakeering}\\

\haiku{Maar zijn dialoogstijl,,.}{die ook zonder de voordracht}{overeind blijft is uniek}\\

\haiku{Dit zijn er zoveel,.}{dat cursivering ervan}{storend zou werken}\\

\haiku{genie [der grootste \{\},,:}{sieraden van de balie}{genie p.164 r.7 r.14}\\

\haiku{In elk geval, zij,,.}{en ik zien elkaar zelden}{of juister nooit meer}\\

\section{Heere Heeresma}

\subsection{Uit: Vader vertelt}

\haiku{Het was inmiddels.}{wel duidelijk dat ik mijn}{stiel gevonden had}\\

\haiku{Hij stelde dan ook}{voor dat ik de rest uit de}{kassa zou gappen}\\

\haiku{- mijn sigaretten.}{tot de lippen schroeiende}{peuken verbrandde}\\

\haiku{Daarvoor ervaar ik.}{mijn verblijf aldaar als een}{te grote schande}\\

\haiku{128 p.  ~  ~Zesde,.}{druk  Onveranderde}{tweede druk 1971}\\

\haiku{88 p.~ ~Vijfde,.}{druk  Onveranderde}{derde druk 1969}\\

\haiku{108 p.~ ~Zesde,.}{druk  Onveranderde}{derde druk 1970}\\

\haiku{88 p.~ ~Achtste,.}{druk  Onveranderde}{derde druk 1971}\\

\haiku{96 p.~ ~Zesde,.}{druk  Onveranderde}{tweede druk 1971}\\

\haiku{Heere Heeresma,.}{vertelt over zichzelf  De}{Telegraaf 6.12.'69}\\

\haiku{3 Bijlemer Prinz' ();}{bekerzilmeta lezers}{uit de Bijlmermeer}\\

\haiku{Iets waar hij zich toen,.}{nameloos aan ge\"ergerd}{had miste hij nu}\\

\haiku{Richten kan je er,,!}{niet mee maar mensen wat gaat}{die gasgranaat hard}\\

\haiku{Dat krijg je ervan...}{wanneer je het kind met het}{badwater weggooit}\\

\haiku{Geen geleuter maar.}{to-the-point reispakketten voor}{jofele prijsjes}\\

\haiku{{\textquoteright} {\textquoteleft}Allemachtig, oom,{\textquoteright}, {\textquoteleft}?}{Nol riepen we in koorweet}{u wel wat u zegt}\\

\haiku{Gewoon een overall.}{aan en daar is de man van}{de luchtverversing}\\

\haiku{Gelukkig was ze.}{in de keuken bezig en}{deed hem meteen open}\\

\haiku{Een volk van dertien.}{miljoen zielen heeft geen recht}{op zoveel schrijvers}\\

\haiku{En zo kwam van het,.}{een het ander zoals u}{niet ontgaan zal zijn}\\

\haiku{En verder wordt de!}{heer Alberts voor mij nog steeds}{burgemeester}\\

\haiku{Daarvoor moet je niet.}{alleen heel positief zijn}{maar ook zeer modern}\\

\haiku{Proppers zijn het, die.}{het welzijn van hun gasten}{niets kan verdommen}\\

\haiku{Maar wat Marlboro.}{ervan bakte ging alle}{perken te buiten}\\

\haiku{mij diep toen ook ik...}{me omdraaide en deed of}{me neus bloedde}\\

\haiku{Geregeld treffen.}{we Kooimans titels eerder}{bij anderen aan}\\

\haiku{Zo kwam Rap op het.}{spoor van het tekenwerk van}{Beatrix Potter}\\

\haiku{Maar hoe maak ik het?}{de heren duidelijk dat}{we zigeuners zijn}\\

\haiku{We kombineren.}{daarvoor Leviticus 12}{en Lukas 2}\\

\haiku{Indachtig Jo Kal,.}{bij leven Godsdienstleraar}{te Amsterdam}\\

\haiku{Maar men weet hoe het.}{met uitgeverijen over}{het algemeen is}\\

\haiku{Zegt daar nu van {\textquoteleft}ach,.}{ik wist niet eens dat mijn naam}{op de cover stond}\\

\haiku{zuivel op zuivel,!}{is het werk van de duivel}{want zo is het toch}\\

\haiku{In de tram veerde.}{je op als de zitplaatsen}{volgeslibd waren}\\

\section{Herman Heijermans}

\subsection{Uit: Diamantstad}

\haiku{Was niet elk droomend?}{gezwijg g\'odlijker klank dan}{het puurste geluid}\\

\haiku{Achter de ruiten,.}{dichtst-bij was de starre}{aandacht gebroken}\\

\haiku{Jan antwoordde niet,.}{zacht schrapjes aaiend in den}{drekpoel beneden}\\

\haiku{Hij wil niet hebbe,, ',.}{v\`ader da'kn appel zoek}{die keerel van boven}\\

\haiku{Hou jij 'm teugen,,.}{met je lat Meijer anders}{flikkert-ie weer weg}\\

\haiku{De eene hand sloeg om, '.}{den deurpost vingerknekels om}{t molmende hout}\\

\haiku{Die eenzame m\'o\'est.}{in die dagen groot en stil}{hebben geleden}\\

\haiku{{\textquoteright} -, spotte Moppes, steen.}{zachtjens aanduwend over den}{zoetkring van zijn schijf}\\

\haiku{{\textquoteright} De schouders van den,:}{grijzen mageren slijper}{schokten ontkennend}\\

\haiku{'r droge lippen,.}{vrindlijk-rustig knikkend naar}{de zij van Bleazar}\\

\haiku{Nou ja, \`u heit goed, ':}{prate u weet niet watr}{komp-kijke}\\

\haiku{Achter het hoofdeind.}{was de deur van de kast waar}{de strontemmer stond}\\

\haiku{{\textquoteleft}Die is zoo g\'o\'ogem, '...}{zoo uitgeslape voorn}{kind van drie jare}\\

\haiku{Van morrege wor ':}{k wakker en daar zeit de}{gebenchte memme}\\

\haiku{Van 't Plein, dat zwart,.}{lag met krommende boomen}{kwam heftig gestuw}\\

\haiku{Moste ze Davy,!}{nie de darme uit z'n lijf}{trappe de pooiers}\\

\haiku{De vrouw van Semmie - '!}{die komp van de grach heit ze}{\`erg watr gebeurt}\\

\haiku{Soms sleepte z'n been,,.}{soms kon-ie niet loopen z'n}{water niet houden}\\

\haiku{Even lachte-die '.}{mal int geroes van de}{kijvende joden}\\

\haiku{Zelf was 'r mondje ',.}{n volwreven groezel z\'oo}{als ze gesmuld had}\\

\haiku{{\textquoteright} - Zij spande den duim.}{en den wijsvinger om de}{grootte te wijzen}\\

\haiku{{\textquoteright} {\textquoteleft}O jee zoo dikkels{\textquoteright},,:}{blufte ze weer blij dat ze}{die dingen mocht doen}\\

\haiku{Hij, etend 'n reepje, ', '.}{met stroop wenkter binnen}{vroeg naarr vader}\\

\haiku{- Wij haast geen vr\`ete - ' -!}{door de staking zij inn}{warreme Schoel bah}\\

\haiku{- As-je denk da-je ',,!}{mijnr tusschen neemt zeg mot}{je vroeger op staan}\\

\haiku{{\textquoteleft}toch zit 'r lucht in ',}{water en al zatr geen}{lucht in dan vin je}\\

\haiku{Het kind nam den visch,.}{in z'n zwart handje hield z'n}{pink dicht bij den bek}\\

\haiku{{\textquoteleft}godvergeefme de '!}{zonde wat ken jij metn}{glad smoel staan liege}\\

\haiku{Bij het station,.}{werkten ploegen met bezems}{schoppen en latten}\\

\haiku{{\textquoteleft}hoe komp iemand zoo, ' '!}{bemazzel ast water}{dich leit asn pot}\\

\haiku{Stop 'm derek in ',{\textquoteright}...}{t water anders kots je}{je hart uit je lijf}\\

\haiku{Geef mijn 'n kar met - '!}{negotie late z{\`\i}j zich}{n breuk sappele}\\

\haiku{{\textquoteleft}nee, danmod-u is -{\textquoteright}.}{wachte dan zal ik u wat}{anders late zien}\\

\haiku{Over 'r zittend aan ',}{t withouten tafeltje}{zag-ie eerst h\`oe wit}\\

\haiku{Geen dag ging voorbij '.}{of ze las opr bed als}{ze niet te moe was}\\

\haiku{Anders knikte ze,.}{anders herkende ze zijn}{manier van loopen}\\

\haiku{Met heet zand gong 't ' -.}{inn oogeblik had ze}{wel tienmaal beleef}\\

\haiku{Langzaam glee-ie in,,.}{rust z'n handen ontspanden}{z'n oogen knipten toe}\\

\haiku{Overdag hadden de '.}{kinderen al eerder bij}{m kennen kommen}\\

\haiku{{\textquoteleft}v\'oor Joozep na bed komp -{\textquoteright}.}{die zit nou nog lekker an}{tafel te slape}\\

\haiku{Wirrelend spetten.}{de vonkjes uit den dikken}{nek der machine}\\

\subsection{Uit: Droomkoninkje. Een verhaal voor groote kinderen}

\haiku{Jij heb 			 zeker,,?}{wat op je geweten dat}{je zoo bang ben h\`e}\\

\haiku{Ik zal op me teenen,....}{loopen en zoo zachies}{praten zoo zachies}\\

\haiku{{\textquoteright} {\textquoteleft}Dat denk 'k haast wel,, ' '.}{maar ze huilde zoo datk}{r niet 			 verstond}\\

\haiku{Heeft ze de mijne '.}{niet klaargemaakt met worstr}{op en met kaas}\\

\haiku{{\textquoteleft}Zeg moeder goeien,{\textquoteright},:}{dag zei vader die schik in}{z'n 			 bleuheid had}\\

\haiku{De rest zal 'k an,.}{de vogeltjes geven die}{d'r na 			 snakken}\\

\haiku{In de huiskamer - - '.}{de honnige wast}{schrikkelijk-eenzaam}\\

\haiku{{\textquoteright}   Toen vader van '.}{de fabriek kwam sliep \`alles}{int huisje}\\

\haiku{zeker de duim,   ;}{waarmee ze z'n 			 neus stuk}{had gesnoten}\\

\haiku{je mond houdt, ken je '....}{m ook hooren in de wind}{en in de regen}\\

\haiku{'t Is van mijn 'n....}{goedheid om hier te staan}{en geen verplichting}\\

\haiku{Toe nou, schat, toe nou{\textquoteright},,:}{zee 			 moeder nog liever}{dan anders lachend}\\

\haiku{{\textquoteleft}Ik wier ineene zoo,,{\textquoteright}:}{raar in me hoofd 			 moeder}{babbelde-ie na}\\

\haiku{{\textquoteright} {\textquoteleft}Omdat ik 'n brief{\textquoteright}....}{an tante Toos in de bus}{heb 			 gestoken}\\

\haiku{Hij, met 't hoofd op, '.}{de vrije hand trachttet boek}{te begrijpen}\\

\haiku{de 			 eerste trein,.}{en dan kom ik met de}{laatste werom}\\

\haiku{{\textquoteright}, vroeg de man, die zelf.}{kinderen had en bang}{was voor besmetting}\\

\haiku{{\textquoteleft}hij had niet 't lef '!}{motten hebbenn poot na}{me uit te steken}\\

\haiku{En wat zal vader,,}{as-ie uit Heerlen werom}{is en ze overtuigd}\\

\haiku{Zie je wel, dat je'....?}{niet huilen kan as ik}{t niet hebben wil}\\

\haiku{Moeder zat op den,,,.}{stoel bij de wieg 			 keek niet}{bewoog niet dacht niet}\\

\haiku{as je je man eens '!}{in de maand int Rooie Dorp}{mag opzoeken}\\

\haiku{{\textquoteright}, dreigde tante 't, ' ':}{aannemershuis metn}{kluif vann vuist toe}\\

\haiku{'k Geef je zoo van,,}{mijn armoei geverfde}{medam as je d'r}\\

\haiku{\'e\'en durft open te 			 ,.... '}{doen zoolang me zwager niet}{vrijgelaten wordt}\\

\haiku{Was 't reizen na '?}{Heerlen geen foefie geweest}{voorn allebie}\\

\haiku{{\textquoteleft}Ben ik daarvoor de?}{h\'e\'ele dag op de kinderen}{blijven passen}\\

\haiku{{\textquoteleft}maar ik 			 vind 't,}{schande en vooral zonde}{dat jij as moeder}\\

\haiku{{\textquoteright} En gebelgd op zijn.}{manier ging-ie naar buiten}{zitten 			 kijken}\\

\haiku{{\textquoteright} {\textquoteleft}Wel allemachtig,{\textquoteright}:}{zette de electrici\"en}{warm 			 loopend in}\\

\haiku{{\textquoteleft}Doet u 'r twintig,{\textquoteright}.}{gulden bij drong moeder bij}{de 			 leuning aan}\\

\haiku{{\textquoteright} {\textquoteleft}Daar mot je lekkers '....}{voor koopen en voor moeder}{n nieuwe 			 jurk}\\

\haiku{Nou zag-ie zichzelf,,,:}{ook een twee driemaal in de}{spiegels 			   en dacht}\\

\haiku{kles maar toe -			 en....}{toen het de kleine poes van}{mijn haar gegeten}\\

\haiku{Daar leit 'n hengst met....}{z'n vier pooten in de}{lucht te grabbelen}\\

\haiku{da-je 			 uit mot,....}{stappen en dat ze v\'o\'or in}{slaap zijn gevallen}\\

\haiku{{\textquoteright} {\textquoteleft}Laat is kijken, neef -,{\textquoteright}.}{ik heb ze nog nooit gezien}{soebatte-ie}\\

\haiku{En je scharrelt me!}{niet langer achter me rug}{in de keuken}\\

\haiku{Wie komt 'r onder, ',....?}{me hoed kijken hoe ikr}{uit 			 zie stommerd}\\

\haiku{{\textquoteleft}'n lolletje is ', ' ',,?}{t niet maark 			 wenr}{wel an wat jij Ko}\\

\haiku{{\textquoteright} {\textquoteleft}Prachtig{\textquoteright}, lachte de, ':}{electricien dier niet}{\`alles van snapte}\\

\haiku{Moeder, je mag wel,,?....}{meeluisteren maar je}{mag niks zeggen hoor}\\

\haiku{terwijl ik op 't,{\textquoteright} ':}{dak de 			 drajen hecht}{zei-ie tott kind}\\

\haiku{{\textquoteright}, gromde Kobus, niet,.}{goed wetend 			 wat-ie}{dee wat-ie zei}\\

\haiku{Toen sponsde moeder ',.}{snelt snuit van den jongen}{en kwam 			 omlaag}\\

\haiku{Toos, bijgekomen, '.}{zat mett hoofd op de}{tafel te grienen}\\

\haiku{Je zag niks van de, '.}{hooge 			 schoorsteenen niks vant}{wentelend schachtrad}\\

\haiku{Ze was zoo suf, dat ' '.}{ze opr 			 kousent}{tuintje door-stapte}\\

\haiku{Me niet eens na 			 ,....}{benejen dragen om me}{hier te laten eten}\\

\haiku{Hij was te laf 			 ,,.}{geweest om zich te bukken}{zich te overtuigen}\\

\haiku{naar 't ploffen 			  '.}{van de aardkluiten opt}{deksel geluisterd}\\

\subsection{Uit: Duczika}

\haiku{{\textquoteleft}'ns zien wie de baas ',...{\textquoteright} {\textquoteleft}!}{int leven is de man}{of de vrouwDe vrouw}\\

\haiku{{\textquoteleft}dat's gemeen en,!}{dat's laf om iemand z'n}{pols om te draaien}\\

\haiku{{\textquoteleft}want je heb me daar,,!}{gruwelijk beleedigd niewaar}{Poldi hahaha}\\

\haiku{{\textquoteleft}maar 'k hou niet van,!}{geneesmiddelen die je}{h\'e\'ele leven duren}\\

\haiku{zoo'n vroolijken avond,!}{hadden ze in geen maanden}{geen jaren gehad}\\

\haiku{{\textquoteright} klonk dadelijk de:}{tyranniek-schelle stem van}{onder den dameshoed}\\

\haiku{{\textquoteright}, zei Betty met 'n ' ':}{doodswit gezichtjen slok}{vanr glas nemend}\\

\haiku{hardnekkig als-ie, '.}{altijd was verdraaide-ie}{t toe te geven}\\

\haiku{{\textquoteright} {\textquoteleft}Omdat je met die, ',...{\textquoteright} {\textquoteleft}?}{ander metr nichtje nog}{meer flirtWat zou dat}\\

\haiku{Dat 's nonsens en - ' -,?...}{neurasthenisch gedoe weet}{k maar toch niewaar}\\

\haiku{Z\'o\'o beroerd na 't.}{drinken van wijn was-ie in}{geen tijden geweest}\\

\haiku{Om beurten zaten,,.}{de vrouwen de mannen om}{beurten sliepen ze}\\

\haiku{Nathan, kind{\textquoteright}, zei Semmy, ':}{Lubinsky z'n kin voort}{spiegeltje drogend}\\

\haiku{Drie-, viermaal had '.}{ze zacht-kermend gevraagd}{t raam te sluiten}\\

\haiku{{\textquoteright} Ze lachte 'r om,,.}{onoprecht met zorg om de}{bloedelooze lippen}\\

\haiku{Zacht bedrukte-ie,.}{den mouw van de blouse die}{ze zelf had gestikt}\\

\haiku{Ze hadden mekaar,.}{in de oogen gestaard hielden}{mekanders handen}\\

\haiku{Was 't 'n genot,...}{zoo adem te halen zoo voet}{naast voet te zetten}\\

\haiku{{\textquoteleft}en je geeft me je,, -,?}{woord dat als je morgen komt}{en dat d\'o\'e je niet}\\

\haiku{Goeien avond, meneer,{\textquoteright} ', '.}{Sch\"uler zein jong meisje dat}{n muziektasch droeg}\\

\haiku{Ook Erich Sch\"uler had 'r.}{bij z'n eerste bezoek een}{te zien gekregen}\\

\haiku{Dan zal ik morgen '!}{an Kr\"uger of an Neumann}{n briefie schrijven}\\

\haiku{Morgen z\'o\'o als ze,...}{wakker werd zou ze tienmaal}{zoo vlug beginnen}\\

\haiku{Maar ze haalde 't, '...}{in als Lotte en Betty}{n handje hielpen}\\

\haiku{Lotte zette 'n,.}{Tarantella in danste}{waarlijk voorbeeldig}\\

\haiku{Kwamen de menschen '?}{voorn piano  of}{voor de danseuse}\\

\haiku{Daar kan 'k Cl\'eo de,!...}{M\'erode Yvette Guilbert}{voor laten komen}\\

\haiku{de kat in Odessa,,}{uit den boom dan kan ik nog}{altijd nakomen}\\

\haiku{{\textquoteleft}Ook goed{\textquoteright}, babbelde.}{Duczika geduldig en}{innerlijkopgewekt}\\

\haiku{wat Schmink noodig hebben, '.}{als ze in Mannheim achter}{t voetlicht stonden}\\

\haiku{'k Zou d'r bij gaan ' '!}{zitten en ook isn duit}{int zakkie doen}\\

\haiku{{\textquoteright} Betty schonk z'n glas, ' ', ' '.}{in reiktetm over dronk}{r eerstn slok van}\\

\haiku{{\textquoteright} {\textquoteleft}Gedorie, steken!}{jullie niet allemaal je}{neus in mijn zaken}\\

\haiku{{\textquoteleft}en nou beweerde, '!}{ze vanmiddag nog wel dat}{zet om jou dee}\\

\haiku{Nacht Duczi{\textquoteright} - 'r hand' -:}{hield-ie vast of-ie  r}{niet los kon laten}\\

\haiku{Juffrouw Treibitz dee '.}{de wasch en Oom Ludwig hield}{t huishoudboek bij}\\

\haiku{{\textquoteleft}twee pond soepvleesch,...{\textquoteright} {\textquoteleft}!}{en twintig Pfennig beentjes}{vier augurkenNee}\\

\haiku{Nou dan weet ik 't - '{\textquoteright},.}{niet mot ut maar zeggen}{besloot de juffrouw}\\

\haiku{K\"onigsberger Klops...{\textquoteright} {\textquoteleft}{\textquoteright},.}{Met bloemkool knikte juffrouw}{Treibitz goedkeurend}\\

\haiku{Oom werd door juffrouw, ',!}{Treibitz dier ook patent}{uitzag vertroeteld}\\

\haiku{Zenuwbabbelen,,, '.}{aan een stuk door dee ze dat}{jer suf bij werd}\\

\haiku{{\textquoteleft}Ja, die kosten me, - -...}{Ludwig vier dochters en een}{zoon die eten wat op}\\

\haiku{Ze berukten de,.}{kettingen blaften dat je}{geen woord meer verstond}\\

\haiku{k Geloof dat 'k,!}{bedenkelijk boven m'n}{stand leef hahaha}\\

\haiku{{\textquoteleft}Zoo{\textquoteright}, sprak oom met 'n,:}{kleur van opwinding die z'n}{lippen verfletste}\\

\haiku{Ik geef je m'n woord,...{\textquoteright} {\textquoteleft}?}{dat-ie geen Hebreeuwsch leert}{Dat-ie niet wat}\\

\haiku{hoe laat 't op de - '.}{gangklok was telkens zat oom}{m op de hielen}\\

\haiku{Tenzij jij kans ziet...{\textquoteright} {\textquoteleft}!}{om met een van die lepels}{te racenChristus}\\

\haiku{Heb-ie geen kans, '?}{om hier of daar in de buurt}{n roeispaan te leenen}\\

\haiku{{\textquoteleft}op 't leven valt,,,...{\textquoteright}}{af te dingen naar meer dan}{\'e\'en kant maar toch toch}\\

\haiku{{\textquoteleft}Ja, ja{\textquoteright}, sprak Poldi, {\textquoteleft}{\textquoteright}:}{Nietzsche wonderlijk-vreemd}{in de oogen kijkend}\\

\haiku{Je ziet, 't klopt, klopt,...}{met de dingen waarover we}{zoo dikwijls spraken}\\

\haiku{Hij had 'r geen trek...}{in bij Duczi op heete}{kolen te zitten}\\

\haiku{{\textquoteright}, viel Erich geprikkeld - ' ' '?...}{uit zatt gedrochtmr}{tusschen te nemen}\\

\haiku{- naar 't z'n sigaar '.}{bepruimend monstertje bij}{t ladenbureau}\\

\haiku{{\textquoteright} Met oogen van diepste.}{walging keek Erich den in z'n}{stoel hijgende aan}\\

\haiku{Je stikte door de,,.}{bedomptheid den schemer de}{vadzige etenslucht}\\

\haiku{Al z'n leven was - '!}{de angst hoe lijkwit had-ie}{r niet uitgezien}\\

\haiku{Ik wou dat ik uw{\textquoteright},:}{jaren en uw gezondheid}{had zei ze goedig}\\

\haiku{Niettegenstaande ' {\textquoteleft}}{t bordjeUntertaillen}{besonders preiswert}\\

\haiku{Duizend tegen een...}{had-ie den knoop aan z'n vriend}{Poldi gegeven}\\

\haiku{Hij noemde 'r den ' -.}{naam vann restaurant in}{de buurt ze knikte}\\

\haiku{Zoolang 't rijtuig,.}{sjokkend door de straten glee}{spraken ze geen woord}\\

\haiku{Ben 'k niet altijd ' -?...}{n goed vriend voor jou en de}{anderen geweest}\\

\haiku{Dan sturen we 'n, '!...}{telegram dat je zuster}{n gekken kop heeft}\\

\haiku{van 'n leien dak,,!}{had je geen hulp van vreemden}{en wat voor vreemden}\\

\haiku{Wat lief, wat schattig,,' '.}{mijmerde ze en  r}{adem zong inr hoofd}\\

\haiku{Daar stond ze als 'n '.}{opgedirkte zottin met}{r zijden blouse}\\

\haiku{Twee-hoog ontmoette,.}{ze den Italiaan dien ze}{liefst mijlen ontweek}\\

\haiku{{\textquoteleft}zijn uw mama en...{\textquoteright} {\textquoteleft}?}{uw zusterIs Erich Sch\"uler in}{Friedrichshagen}\\

\haiku{... - schreef-ie, dat-ie...{\textquoteright}}{zich met uw nichtje Betty}{definitief had}\\

\haiku{De glazen staan in '...{\textquoteright} {\textquoteleft}{\textquoteright},,.}{t buffetGraag zei-ie blij}{dat ze op dreef kwam}\\

\haiku{En als-ie wel op,,!...}{de binnenplaats was was-ie}{even ver hahaha}\\

\haiku{{\textquoteright} En aan de overzij,, ':}{bij Caf\'e Bauer klonkt}{scherp en bevelend}\\

\haiku{Ze zei alleen dat, '.}{ze zoo moe was dat ze niet}{uitr oogen kon zien}\\

\haiku{Anders valt-ie...{\textquoteright} {\textquoteleft}!}{onder de termen vanWel}{God-allemachtig}\\

\haiku{Nee, onder deze. '}{omstandigheden mocht ze}{niet vervoerd worden}\\

\haiku{n Stoel bij 't bed, '.}{schuivend warmde-ier hand}{tusschen de zijne}\\

\haiku{Abfertigung der,!}{unter Kontrolle stehen}{den Personen hier}\\

\haiku{De eenige, die 'r,.}{niet met vragen lastig viel}{was Poldi R\"ose}\\

\haiku{Dat zou 'n laagheid, '... '}{zijnn afschuwelijke}{gemeenheidk Heb}\\

\haiku{Lieve Poldi, ik, '...}{smeek je laat mer niet meer}{aan terugdenken}\\

\haiku{Ze schudde 't hoofd - '}{ze hieldr omkringde oogen}{niet van de lamp af}\\

\subsection{Uit: Kamertjeszonde}

\haiku{Indien u mij in,:}{gemoede vraagt w\`at met het}{manuscript te doen}\\

\haiku{Nog lag ik niet in,.}{mijn stoel of de bel begon}{vinniger dan straks}\\

\haiku{'k Heb vergeten.}{de krenten te wasschen door}{de herrie van straks}\\

\haiku{we elkander het,,,.}{h\'e\'ele leven dag aan dag avond}{aan avond uur aan uur}\\

\haiku{maken - Juffrouw{\textquoteright}..... maar.}{dan weet-ie den naam van}{de oude vrouw niet}\\

\haiku{Bij den bakker aan,.}{de overzij haalt Dirk een versch}{p\`asgebakken brood}\\

\haiku{Van haar vingers druipt,.}{vet op de japon wat ze}{met haar mouw afveegt}\\

\haiku{{\textquoteright} - {\textquoteleft}Jij ben de heele!}{avond al zoo verrekt saai en}{vervelend geweest}\\

\haiku{{\textquoteright} - {\textquoteleft}'t Is gevaarlijk '.}{om iets an te halen met}{n getrouwde vrouw}\\

\haiku{{\textquoteleft}\`als het goed is voor,?}{jezelf waarom deugt het dan}{niet voor anderen}\\

\haiku{{\textquoteright} - Georgine laat,.}{mijn arm los gaat inloopen}{tusschen Guus en Duif}\\

\haiku{Toen ben ik voor hem '.}{en de kinderen weer an}{t zingen gegaan}\\

\haiku{Nou mot je weten,{\textquoteright}....}{da'k al om twaalf uur in me}{bed lag met maagpijn}\\

\haiku{Maar \`als ik niet te....}{eten had of mijn kinderen}{niet te eten hadden}\\

\haiku{{\textquoteleft}'k hoor je liever,....}{vertellen van honger en}{armoede dan van}\\

\haiku{{\textquoteright} roept de vrouw van den. - {\textquoteleft} ',?}{acteurDrink-ie mee van}{t rondje Sanet}\\

\haiku{Het kind staat in haar,.}{nachtpon op bloote voetjes zoo uit}{het bed gesprongen}\\

\haiku{Moosie, je vader meent ',,!}{t best met je Kom nou Moos}{wees nou verstandig}\\

\haiku{No\`u toen ik dat zei,.}{van dat kind had je oom en}{tante moeten zien}\\

\haiku{Hij me na een paar,?}{minuten achterna Waar}{is je huissleutel}\\

\haiku{{\textquoteright} {\textquoteleft}Voor die spijt 't me. '{\textquoteright}.... {\textquoteleft}'.}{Da'sn goeie vrouwk Zou ze}{twee woorden schrijven}\\

\haiku{Leipie holde de,, '.}{trap af pakte d'r in d'r}{nek smeetr op straat}\\

\haiku{Toen ben ik om half{\textquoteright}....}{zes even in de Bodega}{geweest om te zien}\\

\haiku{Wegens huwelijk,....}{van de tegenwoordige}{vraagt men te Arnhem}\\

\haiku{Hier ben ik, ik, zij ',.}{n chanteuse die lol in}{haar bordeelvak heeft}\\

\haiku{Ze is in rose.}{baltoilet met opstaanden}{kraag van geel satijn}\\

\haiku{'k Vind haar zoo wel.}{frisch met haar meisjesgezicht}{en haar zwarte oogen}\\

\haiku{Ik kijk naast Lisy, '.}{in den spiegel wrijft goed}{weg met m'n zakdoek}\\

\haiku{{\textquoteright}  {\textquoteleft}Zoo gingen wij{\textquoteright}}{naar de schouwburg heen Ad\`ele was}{niet erg vlug ter been}\\

\haiku{{\textquoteleft}Wacht u effen hier,{\textquoteright} - {\textquoteleft}.}{zegt de portieruw kamer}{is an de voorkant}\\

\haiku{Een, twee minuten,.}{alleen in de gang met de}{wuivende kaarsvlam}\\

\haiku{{\textquoteleft}Willen we is gaan '?}{toeren naar Scheveningen}{inn open bakkie}\\

\haiku{'k Maal om niks, om '....}{niks waarn hoop andere}{menschen om malen}\\

\haiku{Wat mot dat heerlijk.... '....}{zijn Watn akelig idee in}{zoo'n stad te leggen}\\

\haiku{Als je 'n man niet, '.}{volkomen vertrouwt mo\`et je}{niet metm trouwen}\\

\haiku{Hier h\`e-je mijn.... '{\textquoteright}....}{zakdoekk Heb d'r m'n neus}{nog niet in geveegd}\\

\haiku{{\textquoteright} {\textquoteleft}Als je an tafel.}{komt zitten en je hoed en}{je mantel uitdoet}\\

\haiku{{\textquoteleft}Dat zegt-ie Alf om,{\textquoteright}.}{niet met de deur in huis te}{vallen knipoogt Scherp}\\

\haiku{Ik kan niet gelooven}{dat u met v\'o\'orwetenschap}{m'n neef die ziek is}\\

\haiku{Ik verzeker je ',{\textquoteright},.}{da'kt je inpeper zei}{ik me inzeepend}\\

\haiku{{\textquoteright} {\textquoteleft}Wat doe je al niet.}{als de jonge meiden van}{rechts naar links dwijlen}\\

\haiku{Scherp sloeg de huisdeur, '.}{dicht kwam de keuken binnen}{metn hoop pakjes}\\

\haiku{{\textquoteright} {\textquoteleft}Wel nee, Sorsien, 'k '.... '....}{kom niet omt-geldt}{geld kan wel wachte}\\

\haiku{Kijk me goed anrecht -, -?}{in de oogen vrouw vr\'ouw vertrouw}{je me heelemaal}\\

\haiku{k dat-ie me - '.}{vermoorden zou stopt dek}{is achter me rug}\\

\haiku{- Toen h\'e'k drie weke '....}{geleefd metn officier}{van de marine}\\

\haiku{Dan kocht ik wel is ',....}{n zes pond paling ging die}{vrouwen tracteeren}\\

\haiku{Mok me dan zoolang}{verstoppe in je kelder}{of stel je me voor}\\

\haiku{Hij gaat met ons mee, '.}{alsk Georgine naar}{de Albert Cuyp breng}\\

\haiku{Als 'k nou met de, ' ' '....}{Ruhr begin benkr weer}{n end bovenop}\\

\haiku{Je mot is nagaan '....}{w\`atr an kolen verbruikt}{wordt in \'e\'en winter}\\

\haiku{In eens breekt-ie af,, ' {\textquoteleft}....}{loopt plomp naast me zegt nan}{poosje haast-woedend}\\

\haiku{{\textquoteleft}Daar kan jouw dikke -.}{kont niet tusschen kom maar n\`aast}{me in de leunstoel}\\

\haiku{Toen ik om twaalef,{\textquoteright}....}{uur opstond had ze nog niks}{gedaan die luie vlerk}\\

\haiku{Wie het jou geleerd?....}{ouwere mense in de}{rede te vallen}\\

\haiku{Met z'n kippige '.}{oogen zaagt-ie sneden vann}{halve vinger dik}\\

\haiku{- Meijer, de vader,,.}{van Stientje die was altijd}{verkouwen zei Duif}\\

\haiku{{\textquoteright} - {\textquoteleft}'k Heb schijt an jou,{\textquoteright} ':}{\`en an jou plappert Duif met}{n dikke tong uit}\\

\haiku{- Wat goddelek, om nou is ' -,?}{n heele nacht bij elkaar}{te zijn de \'e\'erste h\`e}\\

\haiku{{\textquoteleft}'t Is beter om '.}{Pijp-juffrouwe opn}{afstand te houen}\\

\haiku{{\textquoteright} Juffrouw Thomas brengt ',.}{n schotel bieten hel-rood}{in de witte schaal}\\

\haiku{Nou-ou, ik kan toch!}{niet op me bloote voeten naar}{beneje loopen}\\

\haiku{Maar als ze je weer,,:}{wat vrage van mamma hoor}{je moet je zegge}\\

\haiku{Geef an de kruier ',.}{t adres mee dan kom ik je}{van avond opzoeken}\\

\haiku{Voor 'n man is 't '.}{h\'e\'el gemakkelijk alsn}{vrouw z'n dienstmeid is}\\

\haiku{'t Is te voelen,.}{terwijl de leege woorden door}{de kamer drensen}\\

\haiku{{\textquoteleft}'k heb is eens 'n - '....}{kies late trekke dat zal}{k nooit vergete}\\

\haiku{en m'n eed toen - he'k ' '.}{je verteld datk toenn}{eed gezworen heb}\\

\haiku{Dat lijkt alles mal ' ',?}{opgewonden alskt}{hier zoo vertel h\`e}\\

\haiku{- Zij had gezegd dat ' '}{ze van me hield en datt}{r niks kon schelen}\\

\haiku{Over 'n paar weken.}{was ze over d'r tijd en was}{ze de machien kwijt}\\

\haiku{Maar 't grappigst was '.}{t eigenwijs opzetten}{van nieuwe regels}\\

\haiku{k Wil nou niet{\textquoteright} - en.}{ze ging in groote drukte de}{tafel afruimen}\\

\haiku{{\textquoteright} {\textquoteleft}Got, da's wel aardig,{\textquoteright},.}{zei juffrouw Doedelaar den}{doek verknuffelend}\\

\haiku{{\textquoteright} bobbelde juffrouw - {\textquoteleft},?}{Doedelaar met veel pretwat}{is v\`ochtig juffrouw}\\

\haiku{die kronkelt in de,,{\textquoteright}....}{verdieping da's arremoe}{siekte en hoofdpijn}\\

\haiku{Met z\`ulke kleine.}{oogjes stond je naar d'r bloote}{arme te kijke}\\

\haiku{En bij het licht, met,.}{woedend-harde rukken trok}{ze het portret stuk}\\

\haiku{Zij aan 't praten, ' '.}{aant overtuigen metn}{begin van berouw}\\

\haiku{Toen likte zij het, '.}{roervingertje af wou me}{t kopje brengen}\\

\haiku{{\textquoteleft}Asjeblief meneer '{\textquoteright} {\textquoteleft}.}{en w\`el bekommet u.}{Dank u wel mevrouw}\\

\haiku{heeft ze 't leven...}{zoo verzuurd dat-ie zich}{opgehangen heeft}\\

\haiku{maar tegenwoordig, '}{tegenwoordig benk toch}{wel een goed vrouwtje}\\

\haiku{Als ik me d'r niet, '.}{mee bemoeid had zout v\'eel}{beter geweest zijn}\\

\haiku{Wiel u asjeblief.}{er voortaan aan denke dat}{v\'oor die wienkel ies}\\

\haiku{ze zijn heel mooi,{\textquoteright} zei,.}{ik verheugd met meer honger}{dan lust tot gepraat}\\

\haiku{- {\textquoteleft}Oe, wat kn\`ars u oom,{\textquoteright},.}{zei Tilly de vingers in}{de ooren stoppend}\\

\haiku{{\textquoteright} - Zoo toonloos rustig ', '.}{als zet zei schriktet}{me w\`onderlijk op}\\

\haiku{Waarom was 'k maar?}{niet met Georgine \`en}{Scherp \`en Kaatje samen}\\

\haiku{Als je bericht kreeg - '?}{dat-ze d\'o\'od waren}{zout je wat doen}\\

\haiku{{\textquoteleft}'n jong mensch m\'ag iets ' -,?}{meer doen dann jong meisje}{wat zeg jij Jules}\\

\haiku{Zal 'k u ook is,{\textquoteright}.... - {\textquoteleft},,}{even legge meneerWel ja}{leg u mij ook is}\\

\haiku{{\textquoteright} - In 'n dronkenschap,, '.}{van herinnering springt ze}{op slaatn cancan}\\

\haiku{{\textquoteright} Om de lamp zaten,.}{we met z'n drie\"en pratend}{tot diep in den nacht}\\

\haiku{het was kaaremelk u}{begrijp jan was woedent en}{niet opgegeeten}\\

\haiku{{\textquoteleft}zij en Jan zijn nog,.}{niet bij me geweest z\'oolang}{ik met Alfred ben}\\

\haiku{Of 'k ook niet vond?}{dat Bij de wieg van Jan van}{Beers goddelijk was}\\

\haiku{Da's \'ook 'n mislukt,{\textquoteright} -....}{kaaretje meende schoonmama}{Kloos opnemend}\\

\haiku{Voor veertien dagen ' '{\textquoteright}.... {\textquoteleft}....}{zagkr nog gezond en}{welHartkwaal Meijer}\\

\haiku{Op het bruin geverfd,.}{haar spiegel-zwartte de}{glimmende hooge hoed}\\

\haiku{Guus, snikkerig, groot,.}{van vrouwelijk meelijden}{liep op Meijer toe}\\

\haiku{- Je \'eigen moeder,....}{die me niet gegund heeft je}{oogen toe te drukke}\\

\haiku{{\textquoteleft}Arme, arme Stien{\textquoteright} - - {\textquoteleft}.}{begon Meijer opnieuw\`arm}{ongelukkig kind}\\

\haiku{{\textquoteleft}'k Dacht niet dat 't,{\textquoteright} -.}{z\'oo gauw zou weze zei Dirk}{b\'ang voor die stilte}\\

\haiku{'k Wil die vent niet....{\textquoteright} {\textquoteleft} ';}{meer hier hebbeOf je di\`e}{neemt ofn ander}\\

\haiku{{\textquoteright} {\textquoteleft}Wat gezellig, h\`e,,?}{oome da'k u en mamma}{kan zien legge h\`e}\\

\haiku{{\textquoteright} {\textquoteleft}Dank u.{\textquoteright} 'k Stond al,. {\textquoteleft}......}{op de trap toen ze me nog}{even nariep Meneer}\\

\haiku{{\textquoteright}.... - {\textquoteleft}Over 't geld hoef u,{\textquoteright},.}{niet ongerust te zijn zei}{ik ongeduldig}\\

\haiku{k Sal nou maar gauw, '....}{weggaan h\`e anders ist}{w{\`\i}sselkantoor dicht}\\

\haiku{Bax was 's middags,.}{geweest had den toestand vrij}{gunstig gevonden}\\

\haiku{Ik keek haar alleen:}{maar aan en plots sloeg ze de}{armen om mijn hals}\\

\haiku{O God, Alf - je mot - '....}{niet d\`enke over wa'k zegk}{ben zoo ellendig}\\

\haiku{Want w\'e\'et je wel toen ',....}{jet \'e\'erst bij me kwam die}{avond bij juffrouw Bok}\\

\haiku{H\'e'k 'r nou n\`og niet?....}{twee weken geleden vijf}{dollars gegeven}\\

\haiku{U is te veel man '.}{van de wereld om niet te}{weten hoet hoort}\\

\haiku{Ja, al schud jij nog,, '.}{zoo je kop Dirkt kan me}{geen bliksem schelen}\\

\haiku{Wilt u nog trachten?}{wat te doen voor het behoud}{van de kinderen}\\

\haiku{Die vent en z'n vrouw '.}{moestk natuurlijk op mijn}{hand zien te krijgen}\\

\haiku{Iek weet niet wat voor.}{soort laarze dat zijn en Frits}{ies mienderjarig}\\

\haiku{{\textquoteright} Juffrouw Doedelaar ',.}{haalden sleutel uit haar}{zak dee mijn deur open}\\

\haiku{Da's 't onderscheid!}{en verder ontloope jullie}{mekaar geen flikker}\\

\haiku{Jullie scheppe hier.}{de peentjes op alsof je}{w\`onder wat inbrengt}\\

\haiku{Met 't h\`oofd in 't,,,.}{kussen moe toch-glimlachend}{luisterde ik weer}\\

\haiku{we ontvangen haar}{zoo goed en nu krijgen we}{geen taal of teeken}\\

\haiku{ik heb al teege pa}{gezegt als alfred soms een}{baantje voor hem heb}\\

\haiku{Nou leg ik leeg zoo,,.}{all\'e\'en zoo all\'e\'en op m'n}{bed in m'n bedstee}\\

\haiku{Zulleke schape,,?}{heelemaal geen lucht en pas}{ziek geweest niewaar}\\

\haiku{- Jeezis mierande, denk -!}{ik en ik de trap af as}{de verdommenis}\\

\haiku{toen ben 'k gauw met '!}{me kont op de tram gaan staan}{en nou weet jet}\\

\haiku{Maar 'k vin 't toch....}{beestachtig om zoo'n schaap nou}{weg te late gaan}\\

\haiku{Over 'n paar maande '....}{make zet zeker an}{Ka wijs da'k d\'o\'od ben}\\

\haiku{Ik verroerde me,.}{niet heeschwarm tintelend van}{zenuw-inspanning}\\

\haiku{De juffrouw vroeg of ' '.}{t voorn uurtje was of}{voor den heelen nacht}\\

\subsection{Uit: Kleine verschrikkingen (onder ps. S. Falkland)}

\haiku{Al dat geknoei most.}{ze in d'r nieuwe huis in}{Arnhem nie hebbe}\\

\haiku{Nog eens knikkend en '.}{glimlachend liep de brave}{kerel overt veld}\\

\haiku{Maar ze hadden een.}{anderen koers genomen}{of vreesden de kust}\\

\haiku{Het water gladde,.}{zonder geruisch zonder}{slag tegen het strand}\\

\haiku{De dag die zoo klaar,. '}{en luchtig begonnen was}{vaalde in scheemring}\\

\haiku{De menschen aan 't,.}{strand weken terug trokken}{de kinderen weg}\\

\haiku{dee\"en we al de '.}{povere dingen dien}{dokter gewoon is}\\

\haiku{Dan schuierde hij.}{weer en mijn handen trokken}{en hieven de polsjes}\\

\haiku{Met m'n vrouw samen,.}{maakte ze fleschjes klaar}{voor de zuigeling}\\

\haiku{Ja, dat zou je nie - '.}{geloovek hei in geen}{maande geslape}\\

\haiku{Dan bleekte de maan,.}{weer schuchtere glansjes door}{wolke-rand spinnend}\\

\haiku{Wij zijn 't!{\textquoteright} -, riepen,.}{we toen de buurvrouw angstig}{over de schutting keek}\\

\haiku{'k Ben gister den,{\textquoteright}.}{heelen dag van streek geweest}{zei mevrouw Dewaard}\\

\haiku{Op zoo'n manier zou.}{je je heele familie}{motten opsluiten}\\

\haiku{{\textquoteleft}'k Ben vast an 't,{\textquoteright}:}{kissie begonnen sprak-ie}{in afwezigheid}\\

\haiku{We hadden mekaar ' '.}{nog \'e\'ens ontmoet opt}{wegje naart strand}\\

\haiku{{\textquoteright} Hij knikte en 'n '.}{oogenblik later zatk}{mee aan de tafel}\\

\haiku{Even bekeek-ie 'm, '.}{van onder tot boven toen}{liet-iem vallen}\\

\haiku{Jan rookte, hevig,.}{de lucht doorhappend Chris en}{Trien vlochten bloemen}\\

\haiku{{\textquoteleft}Mijn krans is klaar,{\textquoteright} riep, ' '.}{Trien opspringendt groen van}{r boezelaar slaand}\\

\haiku{Toen wou Jan w\`eten ', '.}{hoe diept graf was maar Chris}{wout niet hebbe}\\

\haiku{{\textquoteleft}Ja - ja{\textquoteright}.  'r Oogen, '.}{gloeiden in de kassenr}{adem snoof gejaagd}\\

\haiku{{\textquoteright} {\textquoteleft}'n Kind blijft 'n kind - ',{\textquoteright}.}{n kind denkt niet na zei de}{wijze man nog eens}\\

\haiku{{\textquoteleft}{\`\i}k weet 't ook nie, ' '}{wantk mot binnent uur}{an de slachterij}\\

\haiku{{\textquoteright} sprak hij gelijk met,,.}{me op stappend linkervoet}{voor rechtervoet voor}\\

\haiku{Zij en Suus waren,.}{twee beesten bij tijjen}{varkes van meiden}\\

\haiku{{\textquoteleft}'t Is 'n slag... 'n '}{Moeder blijftn moeder al}{wordt ze n\`og zoo oud}\\

\haiku{De magere  .}{hand van knokels en vel scheen}{vleezig te worden}\\

\haiku{Vrouw Abel, verveeld, hield '.}{de boodschappenmand op de}{ronding vanr arm}\\

\haiku{De kist zal jou nie,{\textquoteright}.}{opvrete zei moeder met}{onrustige oogen}\\

\haiku{Den heelen avond was ', '.}{het daarn kwijlend gezoen}{n loddrig gegil}\\

\haiku{We krijgen nog nie '...}{eens de leege kisten vant}{goed na de zolder}\\

\haiku{Want 'n mensch ken d'r, '}{op en d'r af maarn kist}{al was die zoo klein}\\

\haiku{Ze was verzot op '.}{histories enn lijk had}{ze nog nooit gezien}\\

\haiku{Het deurgat zwalpte,.}{licht in de kamer tot bij}{de bruining der kist}\\

\haiku{{\textquoteleft}En hij is al weer, ' -,}{weg net watk zei hij is}{van hier gekommen}\\

\haiku{te bezwijke - zoo'n -.}{misselijke h\`ette}{zoo'n zon in je rug}\\

\haiku{Beter dat Bram 't}{nie beleeft dat zijn zoon met}{\`anzien terug komp}\\

\haiku{{\textquoteleft}dat is de afspraak ', ' '.}{dat weetk wel maar as jij}{tm zoo rauw zeit}\\

\haiku{de \`andere pijp{\textquoteright}... {\textquoteleft}',{\textquoteright}:}{van de schoorsteenn Sleepboot}{hardnekkigte oom}\\

\haiku{{\textquoteleft}Ik maak me nie blij ',{\textquoteright};}{metn dooie mosch zei oom}{wanhopig-kalm}\\

\haiku{{\textquoteleft}'t ken de boot van ' '.}{Sally zijn ent kenn}{\`andre boot weze}\\

\haiku{Ons mag-die t\`och,{\textquoteright}... {\textquoteleft}}{nie dadelijk zien hebbe}{we afgesproke}\\

\haiku{{\textquoteright} {\textquoteleft}As Mau 't 'm zoo - ' '{\textquoteright}... {\textquoteleft}!}{ra\'uw zeit as Mautr zoo}{maar uitpl\`appertN\`og}\\

\subsection{Uit: Schetsen. Deel 1 (onder ps. Samuel Falkland)}

\haiku{De violen, paars,.}{en geel fluweelen mollig}{naast de narcissen}\\

\haiku{Een breede ademplas.}{is als een aureool van}{bleekheid om zijn hoofd}\\

\haiku{Mijn God, wat zou er?}{van ons worden als er niet}{voor ons gekookt werd}\\

\haiku{Al de aschbakjes.}{van d'r familie leegde}{ze in een toetje}\\

\haiku{Ze liep rechtuit de,,.}{Weesperstraat in rondkijkend beduusd}{door zooveel menschen}\\

\haiku{Tot elf uur 's avonds.}{had-ie in het ruim van het}{kolenschip gewerkt}\\

\haiku{{\textquoteright} Angstig kroop-ie, ',.}{weg achtern reuzenzuil}{net in de schaduw}\\

\haiku{Je ziet alleen maar...,...}{z'n nakende voeten dan}{bont wit en rood bont}\\

\haiku{George begon.}{te huilen en de vrouwen}{keken krijtwit toe}\\

\haiku{Gister had zij ze,.}{gezien de werkeloozen in}{een langen optocht}\\

\haiku{De meneer, die met.}{de drie meisjes opkomt heeft}{niets buitengewoons}\\

\haiku{dikke kolommen,,.}{d\`an weer verwasemend tot}{mist dan vettig-wit}\\

\haiku{Als ze nou maar wat,.}{zure balletjes had wat}{zure balletjes}\\

\haiku{Maar dichtbij glijden,.}{ze effen weg als een mes}{dat over een plank slijpt}\\

\haiku{Ze had, als al de,.}{dames der gelegenheid}{een wit voorschootje voor}\\

\haiku{Pootig trapt-ie.}{met z'n hielen en slaat ze}{aan tegen z'n broek}\\

\haiku{Eerst 'n kind met 'n,.}{omslagdoek bleek onder de}{roodte van den doek}\\

\haiku{Onder den breeden,.}{stroohoed wuifde het haar dansend}{op den lentewind}\\

\haiku{* * * ~ Bij de huisdeur,, '.}{in een rieten stoel zat ze}{int zonnetje}\\

\haiku{Maar bij Wies was juist.}{het bijzondere aan het}{hoofd en de voeten}\\

\haiku{De twee lijvige,.}{paarse pompoenkoonen werden}{dus het eerst gezien}\\

\haiku{Als je fatsoenlijk,.}{man bent laat je je vrouw niet}{zoo berooid achter}\\

\haiku{h\`em, haar echtgenoot,.}{voor God den vader van haar}{pasgeboren kind}\\

\haiku{Dat heb je d'r van...* * *}{als je van die opvreters}{in huis neemt}\\

\haiku{ook zie je prachtig,.}{den Amstel met de bruggen}{rechts aan de overzij}\\

\haiku{Nou zal-ie blijven,...}{kijken naar zijn bord tot er}{w\'e\'er wat gezegd wordt}\\

\haiku{- Nou waren ze 'm '... '...}{ant beklagen ant}{bekla-a-age}\\

\haiku{nou is-ie dood voor.}{me. Achter de gordijnen}{zat ze sjiwwe}\\

\haiku{Ze hield de oogen niet,...}{af van de instrumenten}{stootend ademhalend}\\

\haiku{Je zou zwere dat '...}{d'r iemand int donker}{op de deur klopte}\\

\haiku{Zoo'n raam er uit, vind.}{ik het verschrikkelijkst van}{een verhuizerij}\\

\haiku{Hij zal toch wel wat,?}{tegen je gezeid hebben}{toen-ie nog hier was}\\

\haiku{Het wekkertje, dat,.}{om zes uur moest afloopen}{tikte kwaadaardig}\\

\haiku{Waarom heb je zoo -?}{lang met zulke groote oogen in}{het gas gekeken}\\

\haiku{Even staat-ie met het.}{touw om zijn middel op den}{rand van het bootje}\\

\haiku{Ik zat weer in mijn,.}{stoel zenuwachtig met het}{boek in de handen}\\

\haiku{Mijnheer Falkland,.}{ik ben maar zoo vrij geweest}{om mee te kommen}\\

\haiku{Zoo'n stilte moet er,.}{om Perrette geweest zijn}{toen de melkkan viel}\\

\haiku{Het is de zee bij,.}{het strand opkolkend golven}{in branding van schuim}\\

\haiku{Maar ze klapklapt naar.}{de gele bloemen boven}{de groene planten}\\

\haiku{{\textquoteright} De merrie stond schuin.}{op de tuintrap en snoof}{de biefstuklucht op}\\

\haiku{De meid maakte de,.}{buitendeur open veegde de}{voeten op de mat}\\

\subsection{Uit: Schetsen. Deel 2 (onder ps. Samuel Falkland)}

\haiku{Hij zat op de bank,.}{kijkend naar de kinderen}{die niet meer speelden}\\

\haiku{Ze zaten dicht bij,.}{nu hoofdjes gekeerd naar de}{zij van het boschje}\\

\haiku{Elken morgen reed;}{Daantje den rolstoel achter}{de aanrechtbank}\\

\haiku{{\textquoteleft}Wat zie jij me vuil,{\textquoteright}}{zij Suus en haar zakdoek met}{speeksel benattend}\\

\haiku{Ant nu gemaklijk.}{slobberde koffie en Suus}{roerde haar kopje}\\

\haiku{Aan het hoofd van de.}{tafels waren armstoelen}{voor de verpleegsters}\\

\haiku{Nu wou ik u nog '....}{eens laten zien hoet kind}{is bijgekomen}\\

\haiku{{\textquoteright}, informeerde Saar,.}{het gele gezicht dicht bij}{den schijn van het raam}\\

\haiku{z\'oo heeft iedereen -.}{ze zoo had ik ze al toen}{ik twintig jaar was}\\

\haiku{we waren nog steeds,,.}{verloofd verliefd genoten}{van de buitenlucht}\\

\haiku{Het tuintje, gesmoord,.}{in hooge houten schuttingen}{was triestig en zwart}\\

\haiku{In 't keukentje.}{hoorde hij haar scharrelen}{met den ketel}\\

\haiku{Inne...{\textquoteright} - verhaalde:}{l\`ang nadrukkelijk de vrouw}{van den herbergier}\\

\haiku{Toch begint ons paard.}{met groote opgewektheid van}{de ruif te vreten}\\

\haiku{{\textquoteleft}Wacht maar,{\textquoteright} zegt Lou en '.}{nog heeft hijt niet gezegd}{of Sorrie blijft staan}\\

\haiku{Alles is bij hem,,,.}{toegesmeerd zijn ooren zijn}{neusgaten zijn mond}\\

\haiku{Voor 'n daalder in,.}{de week maggie blij zijn als}{je onder dak ben}\\

\haiku{Wat mosten wij met?}{bij de tweehonderd gulden}{beestenpoeier doen}\\

\haiku{Want z\'o\'o'n bestelling '.}{geeft toch geen smid inn plaats}{van 60 inwoners}\\

\haiku{Wij hopen deze,.}{proefneming te herhalen}{hoogaanzienlijken}\\

\haiku{- Om dezen toorn te -}{motiveeren ge ziet dat}{ik wel zeer kalm ben}\\

\haiku{Die verweet je al, '}{de zonde van de wereld}{int bijzonder}\\

\haiku{Mevrouw het 'r wat '.}{saucijsjes enn stukkie}{klapstuk bijgedaan}\\

\haiku{Je mot rekenen, '...}{datk niks anders eet dan}{brood den heelen dag}\\

\haiku{Nu was zijn hoofd wat,,.}{meer gebogen in vreemde}{doffe gedachten}\\

\haiku{Het paard hinnekte,.}{zwiepte met den langen staart}{van witte haren}\\

\haiku{Ik zeg 't u nog, '.}{eens zegt u nog eens dat}{ik wil zijn all\'e\'en}\\

\haiku{In naam van God, in ',.}{naam vant kruis dat ik dien}{gebied ik n\'og eens}\\

\haiku{De hooge duin aan de.}{zeezij stompte op in den}{melkwitten hemel}\\

\haiku{'t Was de eerste, ' '.}{maal dat-iet land en}{t water zoo zag}\\

\haiku{Weer luisterde hij '.}{met starende oogen bijt}{klein stukje aarde}\\

\haiku{Edelachtbare, wat,?}{is er van den nacht van den}{Amsterdamschen nacht}\\

\haiku{Edelachtbare, wat,?}{is er van den nacht van den}{Amsterdamschen nacht}\\

\haiku{Mijnheer Duimelaar.}{leid met beslistheid z'n mes}{neer en luisterde}\\

\haiku{het zal noodig zijn u}{iets mede te deelen van}{mevrouw \`en meneer}\\

\haiku{'k Had de woning!}{wel ses-en-dertig keer}{kenne verhure}\\

\haiku{{\textquoteleft}... daar had 'k nou z\'o\'o.}{op gevlast om jullie me}{pertret te geve}\\

\haiku{- Als je dood ben, wat '?}{blijftr dan anders van je}{over dan je pertret}\\

\haiku{Boompjes en struiken,,.}{dor van leven harken in}{de blauwige lucht}\\

\subsection{Uit: Schetsen. Deel 3 (onder ps. Samuel Falkland)}

\haiku{dicht en z'n kouwe}{hand op haar w\'arme met den}{engagementsring}\\

\haiku{Je mot an vader '......}{zegge da'km niet achter}{me achter me lijk}\\

\haiku{en die kuiltjes van.......}{wit vel d'r tussche en z'n}{ongeschore kin}\\

\haiku{Vader stond altijd,.}{nog wat gebogen correct}{en stil naast moeder}\\

\haiku{Maar in ernstige,.}{gedachten ging ik naast mijn}{zwager die bleek zag}\\

\haiku{waaraan ik met groote,.}{moeite geholpen door de}{bemanning voldeed}\\

\haiku{Nooit zal ik den dag.}{vergeten toen de laatste}{kletskop verdeeld werd}\\

\haiku{Een ploeg arbeiders.}{van de sneeuwreiniging hield}{een vergadering}\\

\haiku{O, droevig is de.}{schimmende herinnering}{aan verre tijden}\\

\haiku{Van de gasten was '.}{de juffrouw van beneden}{t langst gebleven}\\

\haiku{Gek, h\`e.... als 'k me.}{waterchocola niet heb}{is me dag niet goed}\\

\haiku{De kamer had vier,.}{ramen zonder gordijnen}{m\`et jaloezie\"en}\\

\haiku{Komiek ook je moe\"e '}{hoofd inn hotelkussen}{en den heelen nacht}\\

\haiku{als je ligt met je,}{gesloten oogen op je warm}{ingedeukt kussen}\\

\haiku{Het is niet alles,.}{op te noemen wat er bij}{paarden te kijk is}\\

\haiku{Na de vijfde of.}{zesde lepel begint hij}{langzaam te drinken}\\

\haiku{{\textquoteright} Maar de moeder hief,:}{haar op kuste haar op oogen}{en wangen en zei}\\

\haiku{Elk oogenblik van.}{den dag kwam hij en plaatste zijn}{hand bij de tralies}\\

\haiku{De man wist er niets,.}{van en keek met verrassing}{toe toen hij thuis kwam}\\

\haiku{Leeuwtje zat over me., {\textquoteleft}.}{Leeuwtje is een klein ventje}{met grooteidealen}\\

\haiku{{\textquoteright} {\textquoteleft}Als je je mond had,.}{gehouden had je nu al}{alles geweten}\\

\haiku{Toen, Sam, je mag dat,.}{nou gek vinden toen ben ik}{ook stil geworden}\\

\haiku{Soo'n goszonde om '...}{n tapijt wat je pas krijg}{da\`alek neer te legge}\\

\haiku{Precies zoo iets van.}{ha\'ar om daar nou de heele}{nacht over te zeure}\\

\haiku{Ook zijn manchet was,,,.}{nieuw helder dragend een knoop}{verguld met rooden steen}\\

\haiku{Bij den preekstoel was.}{het zwart psalmenbordje naast}{de collectezak}\\

\haiku{Ze wist 't. Bruigom,,,.}{hoed in hand achter haar aan}{zette zich bij haar}\\

\haiku{{\textquoteright} Over de hoofden der,,}{hoorenden sprak hij half voor de}{vuist half aflezend}\\

\haiku{Zij zouden er een.}{flinke meid aan hebben en}{zorgzaam voor het kind}\\

\haiku{{\textquoteleft}wat 'n val... en dat...{\textquoteright} {\textquoteleft} '!}{je niet w\'e\'et waar je te land}{komtEn watn smak}\\

\haiku{{\textquoteleft}ik wou nou is van,...}{h\`em weten wat-ie met}{Botje wil zeggen}\\

\haiku{As je maar weet, als, '!...}{je maar weet da'kt an je}{vader zal zeggen}\\

\haiku{Wat dan! - Zouen we.... -, '!}{liever nog niet wat Ik zeg}{dat jet aanneemt}\\

\haiku{Bij de deur, buiten,:}{begon de student in een}{heel andere toon}\\

\haiku{bij elkaar krijg je.}{nog meer volgkoetsen dan de}{meneer die daar rijdt}\\

\haiku{Zoo zulks met gr\'o\'oten.}{tact geschiedde zou men niet}{licht er over spreken}\\

\haiku{Doode nummer wordt,.}{niet begraven geannexeerd}{door de wetenschap}\\

\haiku{{\textquoteright}, informeerde de,.}{salon-athleet Oliveira}{wakker geworden}\\

\haiku{De liedjes van den {\textquoteleft}{\textquoteright}.}{vroolijken beedlaar zou hij}{dien avond voordragen}\\

\haiku{{\textquoteright} zei hij mat, de oogen '.}{wrijvend die gloeiden vant}{staan boven het vuur}\\

\haiku{Lieve hemel 'k!}{ben zoo blas\'e en ik vind}{alles zoo banaal}\\

\haiku{Als ze me overdag,.}{konden zien zouden ze me}{n\`og meer bezingen}\\

\haiku{Dat l\'or van jou wil '!}{k nog niet-eens om me man}{z'n boterhamme}\\

\haiku{{\textquoteleft}'t Is me 'n reis{\textquoteright} -,.}{blaasde Saar terug terwijl}{ze hem knietjes gaf}\\

\haiku{Wanneer zou 't weer, {\textquoteleft}{\textquoteright}?}{gebeuren dat hij met z'n}{meissie buiten liep}\\

\subsection{Uit: Schetsen. Deel 4 (onder ps. Samuel Falkland)}

\haiku{Meer bij 't buffet,.}{kniezig en wreed stonden de}{tafels en stoelen}\\

\haiku{Je hoofd voelde zwaar '.}{en je neus was suf vant}{stof in de straten}\\

\haiku{Suf neemt het kind de,,}{nieuwe sigaar bijt ruw af}{de punt en mislijk}\\

\haiku{{\textquoteleft}Meneer,{\textquoteright} begon ze,;}{de worst weder in mijn kast}{plaatsend voor morgen}\\

\haiku{Gewichtig begon,.}{ze pillen te rollen de}{vingers zwart glimmend}\\

\haiku{Maar toen de kip al,.}{gepakt zat begon juffrouw}{Suzan te huilen}\\

\haiku{De melkboer die d\'e\'e ',.}{t \`achter in den tuin met}{een stevigen knauw}\\

\haiku{Op de stoep stond mijn,.}{palfrenier met de paarden}{een bruin en een zwart}\\

\haiku{Evenwel werd ik zeer,.}{onrustig niet wetend hoe}{ik er mee aan moest}\\

\haiku{Steviger dan straks,.}{bevestigde ik het touw}{keek over den gootrand}\\

\haiku{- Diep ademhalen - goed - - '.}{zoo best zoo wel dat loopt van}{n leien dakje}\\

\haiku{{\textquoteright} Zijn stem klonk hard, z\`elfs.}{in het kreunend lawaai van}{wielen en ruiten}\\

\haiku{Sien - in de keuken,?}{heb ik een pakje voor je}{klaar gezet hoor je}\\

\haiku{{\textquoteleft}Hebbe ze je niet '?}{gezegd d\`atr een soldaat}{hier op wacht mot staan}\\

\haiku{Het doek ging omhoog {\textquoteleft}{\textquoteright}.}{en dekomiek van het stuk}{maakte zijn entree}\\

\haiku{In 't koffiehuis,,.}{in den ouwen hoek zaten}{Pam en Hobbema}\\

\haiku{'n Mirakel zoo.}{snel als de vermicelli}{er in ronddraaide}\\

\haiku{Daar ging-die in.}{de kokende pan. En de}{boter werd al bruin}\\

\haiku{{\textquoteleft}'k Heb je nog eens, '.}{laten roepen omdatt}{z\'o\'o niet langer gaat}\\

\haiku{Evenzoo trachtten wij.}{relaties aan te knoopen}{met den kruidenier}\\

\haiku{Aan zieke boeren -.}{had hij een broertje dood als}{ze niet betaalden}\\

\haiku{Ze kauwden een poos -,.}{tegenover elkaar hij groote}{zij kleine happen}\\

\haiku{in 't koffiehuis - ' -}{bleef plakken zou jes avonds}{vroeger thuis komen}\\

\haiku{{\textquoteright} schaterde Jan en.}{de andere knechts stonden}{te schudden van pret}\\

\haiku{{\textquoteright} De hond trok stevig -.}{de man hield alleen maar z'n}{hand op den duwboom}\\

\haiku{Bij me eerste vrouw - '.}{tweemaal en bij deze is}{t de eerste keer}\\

\haiku{{\textquoteleft}Sjongen wat 'n pracht ',{\textquoteright}:}{vann krans hetgeen dezen}{vrijer deed zeggen}\\

\haiku{Bokje had rimpels.}{van grootemansgedachten om}{den gesloten mond}\\

\haiku{{\textquoteleft}Nou-nou,{\textquoteright} troostte ( '):}{Bet weerze moestt wel in}{zijn ooren schr\'eeuwen}\\

\haiku{Ze zou even naar huis,,.}{gaan haar meubeltjes nazien}{wat inkoopen doen}\\

\haiku{Vroeger hadden wij -.}{n\'a\'ast dat Caf\'e gewoond v\'e\'el}{jaren geleden}\\

\haiku{'t Is al mooi dat,,}{je elkaar zoo flauw-weg}{nog weet van vr\`oeger}\\

\haiku{Je dacht jong te zijn,, '.}{levensfuttig maar ann}{spiegel wende je}\\

\haiku{'t Is nou wel geen, ' '.}{sch\`ande maart had tochn}{boel beter gestaan}\\

\haiku{Voor vreemden leek 't.}{precies alsof niemand om}{de dooie  maalde}\\

\haiku{Wat bezielt je om?}{z\'oo je stervende vader}{toe te spreken}\\

\haiku{De erf laat ik na...}{aan de kinderen uit m'n}{tweede huwelijk}\\

\haiku{In den vijver, in,.}{de plassen op de blaeren}{ruischte het neer}\\

\haiku{Nou, ik ben Woensdag ' -?}{met opa uit geweest naart}{Rechthuis weel u wel}\\

\haiku{Ik veronderstel,.}{dat hij z\'e\'er vroeg getrouwd z'n}{vrouw verloren had}\\

\haiku{Onze jeugdige}{kruidenier begon op zijn}{stoel te bedenken}\\

\haiku{Ik hou 't geen uur,!}{meer met die kerel uit die}{ke\`e\`e\`e\`erel}\\

\haiku{Omdat zij deeg an ' '!}{t klaar maken was en hij}{r z'n neus in stak}\\

\haiku{De regen g\'o\'ot op,.}{je mantel op je hoed die}{zwaar  wer as lood}\\

\haiku{Schuw keek zij om zich.}{heen naar de schaduwen van}{boomen en heggen}\\

\haiku{En nou was 'r 'n,.}{plaasie binnen-in open waar}{ze kon uitrusten}\\

\haiku{Ruw schoof hij zijn stoel,,.}{bij het raam kwakte er op}{neer keek naar buiten}\\

\haiku{Nog eens trad hij op,.}{het bed toe betastte haar}{koudperlend voorhoofd}\\

\haiku{Schuw keek hij om of -.}{zij het niet zag of zij de}{oogen gesloten hield}\\

\haiku{liefst een dat honger.}{heeft opdat het v\'anz\'elf de}{armpjes uitstrekke}\\

\subsection{Uit: Schetsen. Deel 5 (onder ps. Samuel Falkland)}

\haiku{En datte is 'n -....}{j\`onge enne die jonge}{het zooveel gepraat}\\

\haiku{De schaar in z'n hand,}{wurmde voorzichtig tusschen}{de ijzerdraadjes}\\

\haiku{Je kan me daar de ',!}{waterleiding metn prop}{papier sluite och}\\

\haiku{Pan no. 6 stond op,.}{het vuur pan no. 7 op een}{petroleumstel}\\

\haiku{Ik ben kapot van,.}{dat groote kerkhof grooter dan}{P\`ere la Chaise}\\

\haiku{n zwakke borst had '....}{wast reden te meer om}{niet te gaan stake}\\

\haiku{Analyse van een.}{gemoedstoestand in verband}{met nieuwe schoenen}\\

\haiku{Raar. 't Kistje kwam.}{te staan op twee stoelen en}{er bij twee kaarsen}\\

\haiku{{\textquoteright}... {\textquoteleft}Ach Jeesis, J\`o,{\textquoteright} zei weer:}{de verteller van den moord}{op het dienstmeisje}\\

\haiku{Let wel, ik leg den,.}{nadruk op het bloed minder}{op de historie}\\

\haiku{{\textquoteleft}Kom aan mijn boesem,, '.}{lievelingk heb je is}{willen beproeven}\\

\haiku{Een postbode op,,.}{een kleine plaats vriend is een}{kerel van gewicht}\\

\haiku{Mijn vrouw slikte zijn,,.}{pillen zijn staaldrankjes werd}{met den dag zwakker}\\

\haiku{{\textquoteleft}je moet co\^ute qui.}{co\^ute voor versterkende}{middelen zorgen}\\

\haiku{{\textquoteright} {\textquoteleft}Vertel,{\textquoteright} zei ik, in,.}{perverse aandachtige}{Falklandluistring}\\

\haiku{Op eens, amice, was '.}{het mij alsof ik een klap}{int gezicht kreeg}\\

\haiku{Van avond zal-die.}{natuurlijk wel niet zingen}{van de vreemdigheid}\\

\haiku{Je doet ganschelijk.}{verkeerd tegen dezen tijd}{obscuur te worden}\\

\haiku{Als ze slapen ging,.}{d\'an op de logeerkamer}{en de deur op slot}\\

\haiku{- Karel die...{\textquoteright} {\textquoteleft}'k Vraag,{\textquoteright},.}{je excuses niet zei ze}{bits hem afwerend}\\

\haiku{'t Was een oude,.}{zwarte kater dien ze al}{meer dan tien jaar had}\\

\haiku{Dan hoorde je ze.}{telkens boven naar de kraan}{loopen en vloeken}\\

\haiku{{\textquoteright} Je moet weten dat '.}{Cateau voor geen geldn spin}{zou ged\'o\'od hebben}\\

\haiku{Den ganschen dag zie,.}{je er dames en heeren}{venters en koetsen}\\

\haiku{Ge weet niet w\'at eten,, -:}{w\'at praten w\'at kijken is}{en een mensch zien eten}\\

\haiku{Er zijn nog niet veel.}{bessen rijp en de warmte}{is zoo vermoeiend}\\

\haiku{En uitgelaten -,!}{als jongens draafden ze ja}{zoowaar tante draafde}\\

\haiku{We vonden niets dan,,.}{onrijp goed bessen die groen}{of verrot waren}\\

\haiku{Ik aarzelde in.}{de keus van het beroep dat}{ik zou aannemen}\\

\haiku{En van huis uit ben,.}{ik niet nerveus behalve}{in het voorjaar}\\

\haiku{Geeft u voorkeur aan?}{den eenen criticus boven}{den anderen}\\

\haiku{Johny - als jij, '?}{God als n\`eger z\`ag zou jij}{dan inm gelooven}\\

\subsection{Uit: Schetsen. Deel 6 (onder ps. Samuel Falkland)}

\haiku{de zich Marian,.}{verveeld-vies en kribbig}{den neus optrekkend}\\

\haiku{Moeder, goedig, t\`och -}{met angstig gebaar ruzie}{was zoo \`ell\`endig}\\

\haiku{Z'n broek, afgetrapt,,.}{slobbert om de groote plompe}{logge schoenen}\\

\haiku{{\textquoteright} {\textquoteleft}Zoo heet me broer ook,.}{die verleeje jaar an z'n hart}{gesturreve is}\\

\haiku{As-ie snurke wil.... ' '!...}{s\'al-ie snurket Lijkt wel}{n kertiermeester}\\

\haiku{v\'oor je 'n storm op, ', '.}{zee \`achter jen veilig}{hoteln warm bed}\\

\haiku{Hij zag er waarlijk,.}{ongunstig uit tot zelfs in}{z'n smerige kleeren}\\

\haiku{{\textquoteleft}z\'o\'o 'n w\`onder zou '!...}{t niet zijn voor menschen die}{dri\`e jaar getrouwd zijn}\\

\haiku{De phosphor-streep.}{vlamde zonderling in het}{donker der kamer}\\

\haiku{Hoe komt de duvel{\textquoteright} -.}{in z\'o\'o'n \`onschuldig kind zei}{die van beneden}\\

\haiku{{\textquoteleft}Wat zal Kenau 'r, '?}{wel van zeggen datt zoo}{laat geworden is}\\

\haiku{hij mevrouw den stok,, '.}{drukte op de veer bereid}{n moord te begaan}\\

\haiku{{\textquoteleft}.... Ik weet 'n ladder - ' -{\textquoteright}....}{as u dan bovenn raam}{openschuift ben u klaar}\\

\haiku{t Huisje had ze '.}{geruimd en moedern kop}{thee op bed gebracht}\\

\haiku{Die had 't ineens,.}{h\'ard te pakken en dat zoo}{kort voor de bruiloft}\\

\haiku{Dan met een vreemden, '.}{glimlach nam zen glas en}{\'e\'en der eieren}\\

\haiku{ze stilletjes om,.}{moeder die \`elk woordje van}{de krant las herlas}\\

\haiku{Niets leek veranderd.}{in de monotonie der}{kamertjesdingen}\\

\haiku{'t Was voor vier jaar '!}{n ruzie geweest tusschen}{Ant en de Snoepster}\\

\haiku{En 't dienstmeisje - '. '}{liep mee ast niet te ver}{uit de buurt voerde}\\

\haiku{As-die 't leeren, '.}{wou most-iet maar bij}{andren probeeren}\\

\haiku{{\textquoteleft}Nee,{\textquoteright} zei-die, z'n '.}{neus buigend tot onder den}{rand vant loket}\\

\haiku{Driftig lei ze 'r,.}{vinger op den mond toen-ie}{zoover de trap op was}\\

\haiku{De meneer wreef z'n, ' '.}{gezichtje waaidem lucht}{toe metn handdoek}\\

\haiku{As ze getrouwd was, -}{as moe at ze \`alles van}{chocola nou maar}\\

\haiku{'n Man kan boter '.}{zoo niet \`afkeuren of de}{vrouw stooftr stiekum mee}\\

\haiku{Net, toen de huisknecht, ':}{weerom keerde zag-iet}{l\'a\'atste gebeuren}\\

\haiku{{\textquoteright} {\textquoteleft}Juist,{\textquoteright} zei ik - en 'n.}{half uur later sprak ik den}{agent van de stoomtram}\\

\haiku{M'n vriend rukte het, '....}{roer om deedn paar extra}{slagen met z'n spaan}\\

\haiku{W\`at heb ik misdaan?}{om zoo in mijn eerst geslacht}{gestraft te worden}\\

\haiku{Bevend stak ze de, ',.}{lamp an liett gordijntje}{neer kleedde zich uit}\\

\haiku{De meester het 'r.}{an d'r tong getrokke en}{ze riep nie-eens au}\\

\haiku{Toen wou Jan w\`eten ', '.}{hoe diept gat was maar Chris}{wout niet hebbe}\\

\haiku{Santje grunnekte,.}{van pret doch met mate en}{onder de dekens}\\

\haiku{Dichtbij hoorde ze '.}{Anna's gegiegel die door}{n dekentuitje blies}\\

\haiku{As je moeder dood, '...}{was stopten ze je inn}{weeshuis bij vreemden}\\

\subsection{Uit: Schetsen. Deel 7 (onder ps. Samuel Falkland)}

\haiku{{\textquoteleft}... Hou je monden eens,{\textquoteright}.}{zei-ie en we luisterden}{allen glimlachend}\\

\haiku{Tegen negen, at '.}{man paar beschuitjes met}{marmelade}\\

\haiku{Nap dwaalde gebluft -, -.}{omlaag doorzocht de kamers}{de tuinkamer niets}\\

\haiku{Alles flapten ze '.}{in de kranten of int}{politierapport}\\

\haiku{Als 't 'n ander '?}{was zoue wijt toch niet uit}{de krant v\'oorleze}\\

\haiku{{\textquoteleft}je zal 'n fiets of '{\textquoteright}....}{n locomotief voor je}{neus gehad hebben}\\

\haiku{Voor 'n brievenbus - - - - {\textquotedblleft}!}{hield-ie stil toen toen z\`onder}{zweep z\`ondervort p\`erd}\\

\haiku{Of 'n evenpaard 'n - '?}{individu draagt of sleept}{w\`at ist verschil}\\

\haiku{Behoef ik nu nog,.}{te twijfelen dacht ik in}{vreugde en deemoed}\\

\haiku{De pen kroop over het '.}{blaadje enn krassende}{sluitstreep sloot den brief}\\

\haiku{Zes paar om d'r g\'o\'ed,{\textquoteright}.}{in te kommen van veertig}{cente zei moeder}\\

\haiku{Ze staarden naar 't, '.}{vierkant hok waarin het hooi}{lei ent lichaam}\\

\haiku{De vreemdeling zweeg,.}{een wijle bekeek me met}{loftuitende oogen}\\

\haiku{{\textquoteright} {\textquoteleft}Spausswasser ist Spausswasser,{\textquoteright},.}{snauwde ik uit m'n humeur}{wijd de deur openend}\\

\haiku{As-die niet in ',, '.}{t sterfhuis was zou-die}{aant graf kommen}\\

\haiku{'r Begon al heel '.}{wat stoppelgestuif opt}{laken te vallen}\\

\haiku{'k Draaide m'n stoel,.}{verzette den spiegel naar}{de  linkerzij}\\

\haiku{'t Zag 'r \`anders ' '.}{uit dankt wel bij den}{kapper gezien had}\\

\haiku{Pluimen en pluisjes ( ').}{dauwdenomt po\"etisch}{te zeggen omlaag}\\

\haiku{{\textquoteleft}Effen drinke,{\textquoteright} zei, '.}{ze de prop verslikkend die}{voorr keelgat wrong}\\

\haiku{Ja, daar schiet 'k mee, ',{\textquoteright} ':}{op w\`a\`art van komt snauwde}{de stem int gras}\\

\haiku{Mee-dreunend met 't,.}{gerommel in de verte}{kreunden de wielen}\\

\haiku{{\textquoteleft}de menschen van 't ' -{\textquoteright}.... {\textquoteleft}}{g\'asthuis zeggent en dan}{zou de politie}\\

\haiku{As de kindere ', ', '.}{t zeien dan w\`ast dan}{m\`ostt zoo wezen}\\

\haiku{D'r hoofd liep om van - '. '.}{de drukte en de schepen}{int zichtt Hielp}\\

\haiku{{\textquoteright}, schreeuwde Gerrit z'n:}{paarsbol gezichtje over de}{verschansing buigend}\\

\haiku{Gerrit en Adam en '.}{Gijs zwaaiden er mee datt}{spetterend knapte}\\

\haiku{En de kindren, bang '.}{voort donker liepen mee}{met de menschen}\\

\haiku{Angstig, alsof ze,:}{ze wekken wou riep ze met}{nadruk-accentjes}\\

\haiku{de pieters waren -.}{dood de pieters waren van}{h\`onger gestorven}\\

\haiku{De heele winkel,.}{was \'e\'en gezang \'e\'en zoet weeldrig}{getsilp en getril}\\

\haiku{En in 't voorjaar.}{zette meneer ze w\'e\'er bij}{mekaar in de broeikooi}\\

\haiku{{\textquoteright} {\textquoteleft}En je waterproof,{\textquoteright}.}{is opzij gescheurd lette}{m'n uitgever op}\\

\haiku{Neem 'n stoel en 'n ' '.}{stoof en gar bij zitten}{int zonnetje}\\

\haiku{{\textquoteright} {\textquoteleft}Zie je dan niet dat '?}{t de witte is met de}{veertjes an z'n poote}\\

\haiku{- N\`egen j\`onkies - 't ' - '{\textquoteright}....}{isn gezin ze hetr}{wat mee te stellen}\\

\haiku{{\textquoteleft}Suiker \`en boter -{\textquoteright},:}{en dan nog tw\'a\'alf eieren}{rekende besje}\\

\haiku{{\textquoteleft}me zakken zijn vol, - '{\textquoteright}....}{h\'e\'elemaal vol d'r was amper}{n plaasie voor me bril}\\

\haiku{Anders ben 'k zoo,{\textquoteright} ',:}{gezond klaagdet vrouwtje}{kurkig hijgend}\\

\subsection{Uit: Schetsen. Deel 8 (onder ps. Samuel Falkland)}

\haiku{{\textquoteright} {\textquoteleft}Dat ken 'n kl\`ein,{\textquoteright} {\textquoteleft}'!}{kind ruiken hield ma vol.n}{Klein kind in j\'o\'uw land}\\

\haiku{Toen ontspande ma,, ':}{minzamer blazend opn}{d\`erden stoel en zei}\\

\haiku{{\textquoteleft}Gebeurd,{\textquoteright} zei ma, 'r:}{voeten warmend op den rand}{van den kolenbak}\\

\haiku{zoo dikwijls van de,,}{hand in de tand zat je nou}{eens hier nou eens daar}\\

\haiku{Je zeg maar Owie en -{\textquoteright}....}{O-non en Merci de}{rest leer je vanzelf}\\

\haiku{{\textquoteleft}c'est tr\`es bon mais apr\`es...}{douze heures et moi avec pour}{le parle-ment}\\

\haiku{- {\textquoteleft}Zoo,{\textquoteright} kreunde ik m'n ':}{pijp bekauwend ent laatst}{slokje thee slurpend}\\

\haiku{Hier heb je geld en ' - '{\textquoteright}...}{loop opn draf dan krijg je}{n extra van me}\\

\haiku{{\textquoteright} 'n Nieuwe inval. '.}{n Schrijver z\`onder vrouw is}{geen sikkepit waard}\\

\haiku{{\textquoteright} {\textquoteleft}Dat is de br\'o\'er, die....{\textquoteright}:}{buiten woont en van z'n vrouw}{gescheiden is Of}\\

\haiku{{\textquoteleft}Nou, de huisheer van '.}{de overkant hett wel goed}{met z'n menschen voor}\\

\haiku{De een ging zijn weg,,.}{van lodderen bulderen}{stuiven en stampen}\\

\haiku{'n Stoomboot gulpte '.}{een roetstreep enn bom glee}{op zwarte vlerkjes}\\

\haiku{Het gulpte op den,.}{cementen bak toe om den}{kop van den zeehond}\\

\haiku{{\textquoteleft}Piet kijk is ga\`uw,{\textquoteright} zei, ' '.}{zet kommetje opt}{schoteltje smakkend}\\

\haiku{{\textquoteleft}En alles wat we,{\textquoteright}.}{v\`o\`orgelogen hebben zei}{ze haast hakkelend}\\

\haiku{Na zoo'n aanval van.}{bronchitis moest tante wat}{op verhaal komen}\\

\haiku{{\textquoteleft}'n Glaasje port of ' - '....}{n glas maderak heb}{van alles in huis}\\

\haiku{Vlamt het luidruchtig,,,.}{dan spreekt men van feest van een}{geboorte een bruid}\\

\haiku{{\textquoteleft}Luiken voor vensters ',{\textquoteright}.}{zijn de oogleden vann}{huis wijsgeerde ik}\\

\haiku{Verwonderd schelde - ' '.}{ik aann w\'e\'ek lang wask}{afwezig geweest}\\

\haiku{Maar hier coupeer ik.}{onmiddellijk \`elke op}{zichzelf misplaatste grap}\\

\haiku{En de keukendeur -.}{was dicht en stil \`en de deur}{van de goeie kamer}\\

\haiku{een draaimolen is,,.}{een beul een kwaadaardige}{liederlijke beul}\\

\haiku{En 's nachts, sliep-ie,.}{bij de waarzegster in de}{tent op de keien}\\

\haiku{Het gesprek lei 'n '.}{moment zoo plat alsn niet}{bewegende bot}\\

\haiku{je zoo maar niet je}{levens-hebben en houen}{en omgekeerd voelt}\\

\haiku{{\textquoteleft}Kinderen zijn geen,{\textquoteright}:}{getuigen zei ik en voor}{alle zekerheid}\\

\haiku{En zeg nou n\`og is - -....}{in me gezich wat jij in}{de krant heb gezet}\\

\haiku{De smid haalt je over '!,{\textquoteright}.}{t hekkie as-die je}{ziet gierde een-hoog}\\

\haiku{Vooral de groote meid - '!...}{de kwaje meid die opr}{broertje zou passen}\\

\haiku{{\textquoteleft}Laten we naar de,{\textquoteright}:}{Ringkade l\'oopen zeide}{mijn metgezellin}\\

\haiku{Eindelijk was-ie,.}{over z'n heesche woede heen}{kon-ie weer pr\`aten}\\

\haiku{Opgehitst, benauwd,,.}{knoopte hij de das rond z'n}{nek los ademde zwaar}\\

\haiku{Puf - 'k zit, blaasde, ' '.}{zer pappig handje als}{n waaier zwaaiend}\\

\haiku{{\textquoteleft}Marianne Prins,{\textquoteright},....}{zei ma de plombi\`ere}{verrast ne\`erzettend}\\

\haiku{Nou dan z\`al 'k 'r,{\textquoteright},.}{in d'r gezicht zien zei pa}{z'n vest toeknoopend}\\

\haiku{{\textquoteleft}heeft u iets tegen '?....}{n verkeering van mij met uw}{dochter Sophie}\\

\haiku{{\textquoteright} vroeg moe, die van 'r - '!}{fauteuil je zakter in}{weg van lekkerheid}\\

\haiku{De Haan, je doet je -:}{zaakjes toch nog best met je}{vijftig jaar dienst of}\\

\haiku{Dagen en weken ',,.}{leik in de kooi doodziek}{te lam om te eten}\\

\subsection{Uit: Schetsen. Deel 9 (onder ps. Samuel Falkland)}

\haiku{Er was daar een kof,.}{gezonken die ze zouden}{trachten te lichten}\\

\haiku{Zoo praatte vader ' -!}{en wij ant luistere}{dat ken je denke}\\

\haiku{Nou, toen kwam d'r 'n, '.}{motn gekrakeel van de}{andere wereld}\\

\haiku{die zit 'r in en ' -.}{die blijftr in daar was geen}{dichten mogelijk}\\

\haiku{de schipper is d'r - ' '.}{gloeiend bijt roer loopt as}{n hazewindje}\\

\haiku{{\textquoteleft}Ze zalle je niet,{\textquoteright},:}{opvrete zei dan Blanes}{zoo heette-die}\\

\haiku{Zij hadde kole - '.}{en koste en slijtagie}{hij deet met wind}\\

\haiku{En voor nimmendal -}{an de winkels levere}{nou dat zat zoo lang}\\

\haiku{{\textquoteright} En die simpele, -.}{woorden maakten me bang ik}{wist zelf niet waarom}\\

\haiku{As 'k 'm n\`og is, '.}{snap met jenever draat-ie}{r d\`adelijk uit}\\

\haiku{Op 'n Woensdag kwam ' '.}{k vant schaften in de}{machinekamer}\\

\haiku{En 'n dierlijke'.}{lodderlach schaterde uit}{Blanes zwarten bek}\\

\haiku{Daar dan,{\textquoteright} zei Zadok, ' '.}{t beestn afgekloven}{mergpijp toesmijtend}\\

\haiku{De eene Dubois had -.}{een aardig gezichtje de}{ander was leelijk}\\

\haiku{{\textquoteleft}zouen ze z\'o\'oveel,,!}{onrecht z\'o\'oveel moord z\'o\'oveel}{schandalen dulden}\\

\haiku{Bij onweer most je - -}{geen glimmende dingen \'open}{laten dat trok an}\\

\haiku{{\textquoteright}... 't Bevend tuitje '}{siepte koffiedik enn zwart}{strooperig straaltje}\\

\haiku{- want schudden mag je -.}{ze niet en op d'r kop staan}{mogen ze evenmin}\\

\haiku{Zelfs de kinderen,.}{zwegen angstig kijkend van}{vader naar moeder}\\

\haiku{En trokken naar een,.}{ander dorp toen de nacht de}{straten te schuil lei}\\

\haiku{Ze kromde  d'r,,.}{rug klauwde zich vast in de}{deur rukte en beet}\\

\haiku{Nou - ik heb 'n mand -.}{lekkere l\`evende schol}{ze sprong over de rand}\\

\haiku{We zatten mekaar...{\textquoteright} {\textquoteleft}}{moppen te vertellen tot}{de majoor langs kwam}\\

\haiku{r bij, Piet - as de ',....{\textquoteright} {\textquoteleft}.}{eenr praat van maakt praat de}{ander naJawel}\\

\haiku{{\textquoteleft}Geef u mijn is 'n '{\textquoteright}.}{ons zoetemelksche enn}{half ons leverworst}\\

\haiku{Op het podium.}{gromde geschuif van stoelen}{en taboeretten}\\

\haiku{van den \'e\'ersten tot;}{den l\`a\`atsten regel van het}{Lied von der Glocke}\\

\haiku{as pa weer na 't.... ' '.}{kantoor wast Wasr niet}{van gekomme}\\

\haiku{Z'n brilleglazen, '.}{schuchterden op haar toe b\`ang}{voorn uitbarsting}\\

\haiku{Ongetwijfeld was,.}{het een gentleman een die}{con amore werkte}\\

\haiku{- Wie zal d'r zoo gek '?}{zijn omn inbreker te}{herreberrege}\\

\haiku{Hij vond n\`ergens,,.}{logies de Duitscher vertrok}{met de laatste tram}\\

\haiku{Ik wil wel wete,{\textquoteright} ' - -:}{zeit boertje vinniger}{voor de tw\'e\'ede maal}\\

\haiku{Ik weet niets van 't '.}{lot vann bejaarde kip}{in de vrije natuur}\\

\haiku{De beschaving wijst,.}{bij bejaarde hennen en}{hanen naar soep kluif}\\

\haiku{Er kwam 'n flesch met.}{limonade en een met}{alcoholvrije-wijn}\\

\haiku{Zeg 'm gerust in{\textquoteright}....}{mijn naam dat vleesch de p\`est}{voor iedereen is}\\

\haiku{Thuis krijgen we 't, '{\textquoteright}....}{nooit as vader niet isn}{goeie verdienste het}\\

\haiku{Die met de zeere -.}{oogen stierf die met den boerekop}{viel van de trappen}\\

\haiku{D\`aar - in 't tuintje -.}{had je de ren met de acht}{kippen en den haan}\\

\haiku{Ze leek te drinken ',,.}{vant vuur de vonken de}{stuivende krinkels}\\

\haiku{n Baboe kon bij ' '!}{tijjen maller doen as}{n kind vann jaar}\\

\haiku{De natuur is een.}{materie-schalk en de}{menschen zijn blagen}\\

\haiku{Met 'n eenvoudig:}{gebaar wees-ie naar het}{oude pijpenrek}\\

\haiku{Als ze geen kop thee,.}{had gehouden zou ze in}{slaap zijn verzwommen}\\

\haiku{De kop hield 'r in, '.}{balans juist op de zotte}{limiet vant Zijn}\\

\haiku{{\textquoteright} zei ze bits, met 'n '.}{m\`a wakkerschrikkend j\`a enn}{brutaal-echo\"end p\`a}\\

\haiku{Jeanne haalde,.}{verveeld de schouders op dronk}{nijdige nipjes}\\

\haiku{\`op waren haakte.}{ze voor gelegenheden}{van liefdadigheid}\\

\haiku{'n Kleine twist smijt. '}{het huis-\'equilibre}{ondersteboven}\\

\haiku{Ze k\'e\'ek alleen st\`ar.}{naar pa's knie met de acht}{en veertig kolom}\\

\haiku{De kat loerde met,,.}{glazen blikkrende oogen klaar}{om z'n sprong te doen}\\

\subsection{Uit: Schetsen. Deel 10 (onder ps. Samuel Falkland)}

\haiku{Glimlachend liep de,,.}{juffrouw geen aanmerkingen}{makend niets zeggend}\\

\haiku{{\textquoteleft}Dag Corrie, snoesje,{\textquoteright},:}{zei Lies de schooltasch op den}{lessenaar leggend}\\

\haiku{daar was ze z\`oo bang -.}{voor  geweest en jawel}{niks as gekibbel}\\

\haiku{Het was een keurig,,.}{uitgezocht menu eenigszins}{ondeugend van toon}\\

\haiku{als ge Geldgebrek,,.}{hebt behoefte aan Brood schrijft}{dan Falklandjes}\\

\haiku{Stoffen voor 'n pak, '.}{kiezen als je niet uitt}{werk weet te komen}\\

\haiku{Schuif de boeken maar ',,.}{n beetje opzij Stom als}{ze je hinderen}\\

\haiku{188! 188! - Ja, als u, ' '.}{niet luistert ist niet noodig}{datk voorlees}\\

\haiku{Dan liep minstens 'n '!}{kwart van de gemeente met}{n gat in d'r buik}\\

\haiku{As ik hoofd van de, ' '.}{politie was steldek}{n vervolging in}\\

\haiku{Jij heb gisteravond! -}{mijn dienstmeid bekeurt op me}{\`eigen stoep bekeurd}\\

\haiku{Je blauwe potlood,, (,).}{commissarisluider daar}{Staal hem niet verstaat}\\

\haiku{In dat smaadstuk, in, '....}{dat sch\`andestuk heeftn zoon}{z'n \`eigen moeder}\\

\haiku{'k Zou haast zeggen, '.}{dat \`alle vrouwspersonen}{t mosten zien}\\

\haiku{de Duvel, voornoemd,.}{strooit z'n infaamheden in}{den reinsten akker}\\

\haiku{'t Ongelikte,.}{werd geaccentueerder}{dezelfde week nog}\\

\haiku{Kwam de slager aan '.}{het tuinhekje van Ib dan}{schoot-ier vandoor}\\

\haiku{{\textquoteright}, zei mevrouw, wakker.}{wordend in de linkerhelft}{van het lit jumeau}\\

\haiku{Dat geeft 'n dooie in,{\textquoteright}.}{de femilie zei mevrouw}{in de linkerhelft}\\

\haiku{Bij de visch - turbot -.}{sauce capres waren de}{aardappelen stijf}\\

\haiku{Daarom heb 'k m'n ' {\textquoteleft} '{\textquoteright}.... {\textquoteleft}}{souvenir ingezet met}{tToent kindje}\\

\haiku{We vonden 't best. {\textquoteleft} ' ',{\textquoteright}.}{Enn goospenning vann}{volle week zei ze}\\

\haiku{Zuchtend her-nam,.}{Barend de hand van z'n vrouw}{25 April 1903}\\

\haiku{En omdat moeder ' -}{meende datt gevaarlijk}{was te weigeren}\\

\haiku{Mie met de kaars die ',.}{straaltjes vet opt kleed spoot}{stonden voor m'n bed}\\

\haiku{Bij de {\textquoteleft}stedenten{\textquoteright} ' ' {\textquoteleft}}{wastn zoete inval}{en destedenten}\\

\haiku{Meneer - we hebben...{\textquoteright} {\textquoteleft}?}{centen bij u op de stoep}{gevondenW\`eer}\\

\haiku{Vrouwen verzetten - - '.}{zich dikwerf helaas tegen}{t best intellect}\\

\haiku{Je zweette van de,.}{morgen tot de avond maar je}{had wil van je werk}\\

\haiku{Een onderwijzer, '.}{of zoo dachtk. Een die de}{anderen waarneemt}\\

\haiku{Waarlijk, hij zei langs, '.}{z'n neus weg dingen diek}{absoluut niet wist}\\

\haiku{Daar staat 'n meester.}{bij met de tande in zijn}{mond as jij en ik}\\

\haiku{Jij ken 'r toch niet - ' - ' '.}{en {\`\i}k kenr nietk ken}{r zoo min als jij}\\

\haiku{Maar ze scharrelde, '.}{bij de kastn vergeten}{ding opbergend}\\

\haiku{Vanmorgen, nog geen,.}{drie uur gelejen was-ie}{\`even vochtig geweest}\\

\haiku{{\textquoteleft}Falkland - die man -{\textquoteright}.}{is je offer je heb dien}{man verjongejand}\\

\haiku{{\textquoteright} Kwaadaardig schoven ' '.}{r vingersn teen door de}{opstaande spanen}\\

\subsection{Uit: Schetsen. Deel 11 (onder ps. Samuel Falkland)}

\haiku{Nog grooter as 'n -?}{cocosnoot hei-jij wel}{is schijfies gekocht}\\

\haiku{{\textquoteright} {\textquoteleft}'k Zal me hand voor,, '.}{je ooge legge stommert dan}{zie jem niemeer}\\

\haiku{- Sientje met Bet en -...}{Zus an d'r hande Cor en}{Ansie d'r achter}\\

\haiku{Jessis as de zwaan ',!}{r op los vloog raakte ze}{onder de voete}\\

\haiku{Dat zag-ie nou ook -.}{z'n heele broek van binnen}{was geel van kaarsies}\\

\haiku{'t m\`ost - wou je 't.}{meissie niet in opspraak en}{ongeluk brengen}\\

\haiku{{\textquoteright} {\textquoteleft}Hoe wou jij negen!}{menschen en kindren van \'e\'en}{kip d'r buik vullen}\\

\haiku{Geef mijn 't maagie met '!}{wat sju enk doe me maal}{met aardappele}\\

\haiku{{\textquoteleft}en 'k wensch je veel,!}{genoegen maar ik zal niet}{van de partij zijn}\\

\haiku{De vrouwen konden '.}{z\`ulkn tekstverklaring niet}{laten passeeren}\\

\haiku{We\`ent gij, heerlijke:}{en machtige worstelaar}{voor kleine luyden}\\

\haiku{{\textquoteleft}'k Zel nog is luie,{\textquoteright},.}{zei de jongen superbe}{van demp-toon}\\

\haiku{Men zag er uw rok,, -.}{uw vest uw das ook uw kop}{door kaarsen bevlamd}\\

\haiku{'t Gebeurt meer, 'k.}{behoef er geen finesses}{van te vertellen}\\

\haiku{Ten slotte begrijp,.}{ik ook niet waarom ge die}{rooiekool zoo uitmaakt}\\

\haiku{Om me is lekker, '.}{dwars te zitten begon die}{metn advocaat}\\

\haiku{De uitmuntende,,:}{ziele-ontleedster freule}{Lohman vraagt terecht}\\

\haiku{{\textquoteleft}M'n honden geven,....}{mij hun liefde omdat ik}{hun geef de mijne}\\

\haiku{Als 'k me hier in -, ',....}{de schuld v\`erdrink ist voor}{jouw oogen jouw haar}\\

\haiku{(zij zwijgt nijdig, kijkt,)!}{vlug naar den waard giet haar glas}{over den grond Toe maar}\\

\haiku{Jij ben vanmorgen,.}{met je verkeerde been uit}{bed gestapt Marthaatje}\\

\haiku{(tast naar z'n das, zoekt, ',) ....}{snel de tafel af kijktr}{onder vraagt driftig}\\

\haiku{{\textquoteleft}Blijf u maar rustig ',,{\textquoteright}:}{int w\`arme graf mevrouw}{praatte-ie terwijl}\\

\haiku{{\textquoteright} {\textquoteleft}Betrekkelijk,{\textquoteright} zei:}{hij iets loslatend van de}{stemming van daar straks}\\

\haiku{{\textquoteright}, kommandeerde de,.}{hijgende man pogend zich}{te ontworstelen}\\

\haiku{'t Lukte zonder,,.}{aarzeling zonder gaping}{zonder incident}\\

\haiku{Koosje slobberde ', '.}{vanr kom bette met vrije}{handt achterhoofd}\\

\haiku{Vroeger dee tante,.}{Marretje dat vroeger zat}{ze mee aan tafel}\\

\haiku{De pijp in zijn hand,.}{rustte op z'n knie z'n hoofd}{boog wat naar voren}\\

\haiku{Hij was zoo helder,,. '}{van geest zoo bij de pinken}{meende oom Bernard}\\

\haiku{Mama leefde van '.}{wat Demoiselle enr}{zusters inbrachten}\\

\haiku{Nou zeg 'k niks meer,{\textquoteright}, ':}{redeneerde hij grimmig}{opn stoel ploffend}\\

\haiku{{\textquoteleft}As je \'e\'en kogel, ' '!}{vindt betaalk voor elke}{kogeln bankie}\\

\haiku{{\textquoteleft}ben je tevrejen? ' ' '!}{k Hadt motten weten}{v\'o\'ort huwelijk}\\

\haiku{d'r is niet \'e\'en meid,,!}{letterlijk niet \'e\'en of jij}{jaagt ze de deur uit}\\

\haiku{roemers, de borden.}{en schalen violette}{schulpen geworden}\\

\subsection{Uit: Vuurvlindertje}

\haiku{die had slaap, maar hield - ' ' -:}{zich koestt was weert naarste}{oogen-blikkie}\\

\haiku{{\textquoteleft}maar ik vin 't zoo '....}{fijnn andenken an me}{vader te hebben}\\

\haiku{De jaren hielpen ',,.}{t groote knagende verdriet}{langzaam vergeten}\\

\haiku{{\textquoteright} Met moeite hield ze,,.}{zich in om niet harder niet}{grover te schimpen}\\

\haiku{{\textquoteright}, klaagde moeder, die ':}{t nou zeker bedorven}{eten af had gezet}\\

\haiku{Ongezeggelijk.... '....}{wurmt Is elleke dag}{wat anders met je}\\

\haiku{niet mocht {\textquoteleft}uitvieren{\textquoteright},,.}{gaan uitrusten en in \'e\'en}{gierigheid huilen}\\

\haiku{{\textquoteright}, vroeg ze, 'r zoo naast, '.}{of-ie Engelsche woorden}{uitt Leerboek zee}\\

\haiku{{\textquoteleft}je zag enkel licht....{\textquoteright} {\textquoteleft},?}{van bovenHoe at je dan}{as je honger kreeg}\\

\haiku{{\textquoteright} gromde de vrouw, en '.}{r vingers tintelden van}{zenuw-opwinding}\\

\haiku{{\textquoteleft}Dat mot je heusch{\textquoteright}, ', {\textquoteleft}'}{niet meer doen praatte ze met}{tranen inr oogen}\\

\haiku{In de huiskamer '.}{leunde de grootmoeder in}{r stoel achterover}\\

\haiku{Koert probeerde 'r, '.}{in slaap te praten maart}{licht most opblijven}\\

\haiku{'r Was 'r niet een, '.}{bij geweest die van z'n hart}{n moordkuil maakte}\\

\haiku{{\textquoteright} De stoel niet tegen,.}{z'n drift bestand viel geknauwd}{ondersteboven}\\

\haiku{'t is links, toen weer ', - ' -....}{t is rechts en toen neemt}{me asje asje}\\

\haiku{En as je strakkies,....}{benejen komt zel je d'r}{van op staan kijken}\\

\haiku{zoo'n hoop verdomde, '....}{dingen dwars da'k me voel of}{k gek zal worden}\\

\haiku{'k Zal probeeren 't ' '....}{r uit te trappen al zal}{t niet glad zitten}\\

\haiku{Koert gaf geen antwoord, '.}{liett bed nog meer en nog}{knagender kreunen}\\

\subsection{Uit: Een wereldstad. Berlijnsche impressies en schetsen}

\haiku{{\textquoteright} {\textquoteleft}'k Zal Berlijn leeren{\textquoteright}, ',}{kennen namk me na een}{paar maanden t\`asten voor}\\

\haiku{Hier is 't,{\textquoteright} zei de,.}{wagenvoerder die beloofd}{had te waarschuwen}\\

\haiku{z'n oogenwit '.}{was bloedbeloopen als bij}{n opgejaagd dier}\\

\haiku{Drie-, viermaal, met,:}{maling an z'n slapende}{buren steunde-ie}\\

\haiku{De vagebond, die, '.}{zooeven z'n nood had geklaagd}{wast nijdigste}\\

\haiku{Dit nu is 'n h\'e\'el,{\textquoteright}:}{typisch hoekje van Berlijn}{zeide onze vriend}\\

\haiku{door schemerduister,.}{stapten we de deuren der}{bewaarplaatsen langs}\\

\haiku{Den Vater verliert,{\textquoteright}, ':}{man zei-ie als man die weet}{vant leven had}\\

\haiku{En 'k leef altijd,...}{in angst dat ze me van de}{straat zal wegnemen}\\

\haiku{We lagen opnieuw {\textquoteleft}{\textquoteright}.}{tegen de glooi{\"\i}ng van de}{B\"ohmischen W\"alder aan}\\

\haiku{Het eenige wat me.}{momenteel intresseert is}{Rusland in Berlijn}\\

\haiku{Morgen,{\textquoteright} zeide ik,:}{tot de maagd die er bleek bij}{geworden was}\\

\haiku{Je kunt nooit weten...{\textquoteright}}{welke curieuze vondst}{we gedaan hebben}\\

\haiku{Reuze-banket,,.}{Gister 10 December was}{het \`ongemeen druk}\\

\haiku{Voor 'n jour heb je.}{niet zoo schrikkelijk veel noodig}{en ook niet veel plaats}\\

\haiku{De punchgeur doortrok,.}{de kamer dee de mannen}{uitbundig praten}\\

\haiku{{\textquoteleft}heeft 'n sterveling...}{je wat gedaan da-je de}{pret gaat verstoren}\\

\haiku{Nou zeg 'k je \'e\'en{\textquoteright},, ':}{ding barstte mevrouw nan}{nieuwen ademhap los}\\

\haiku{{\textquoteleft}als je ons op die, '{\textquoteright}...}{manier blijft jagen benk}{nog in geen \'u\'ur klaar}\\

\haiku{Hoe de oogen van z'n...}{bruid bij die stomme woorden}{geglinsterd hadden}\\

\subsection{Uit: De wijze kater}

\haiku{Ik sta hier al een}{poos af te luisteren op}{wat voor manier jij}\\

\haiku{In mijn jeugd, toen ik,:}{nog graag op de schoot van mijn}{moeder sprong dacht ik}\\

\haiku{Die waren voor een!}{maand met bordpapier in plaats}{van met leer gezoold}\\

\haiku{Wat zou u in zo'n? '}{geval van schandelijke}{onverzorgdheid doen}\\

\haiku{(Wendt zich tot kater,.}{die zich met Jonathan heeft}{bezig gehouden}\\

\haiku{Helaas, Hoogheid, ik...}{moet mij aan de etiquette}{onderwerpen}\\

\haiku{Hoogheid, ik smacht naar!}{een t\^ete \`a t\^ete118 van}{meer tedere aard}\\

\haiku{(Af, uitgelaten,.}{door eerste lakei die hij}{genoeglijk toeknikt}\\

\haiku{als ik neerkniel in,;}{de kapel hoor ik hoe het}{altaar beknaagd wordt}\\

\haiku{De sallemanders,!}{en krengen hebben de koorts}{op het lijf Sire}\\

\haiku{Ik voel me ook als,}{een mens in een vreemd pakhuis}{maar de melk is best.}\\

\haiku{Ik had al het zeer,...:}{bijzondere genoegen}{Sire  Koning}\\

\haiku{Hoeveel ratten heeft?}{U hier sinds eergisteravond}{zelf niet gedood}\\

\haiku{Er zijn meer ratten,,.}{eerwaarde heer dan U en}{ik vermoeden254}\\

\haiku{Maar mijn poten heb.}{ik met groene zeep onder}{de pomp gewassen}\\

\haiku{Ik wou je alleen '...}{zeggen datt bijzonder}{verfrissend is}\\

\haiku{Ik zou hem op 't!...}{moment nog graag de ogen uit}{zijn kop halen}\\

\haiku{Juist, juist, Angorensis...,!...}{Geen tragische souvenirs}{arme kerel}\\

\haiku{Duizend tegen een,!}{dat hij zijn handtekening}{niet kan schrijven}\\

\haiku{Die was je wezen,...}{kopen bij een boer die ze}{al begraven had}\\

\haiku{Verdwijn uit onze!}{ogen en laat ons nooit meer iets}{van je horen}\\

\haiku{Als jij dat niet weet, '!}{heb je in rechten niett}{recht brood te bakken}\\

\haiku{96volgens het boek.}{Genesis schiep God op de}{zesde dag de mens}\\

\section{Albert Helman}

\subsection{Uit: Aansluiting gemist}

\haiku{Hij herinnerde.}{zich niet meer hoe lang hij daar}{was blijven zitten}\\

\haiku{Hij had zijn schouders.}{opgehaald en was hinkend}{de hut uit gegaan}\\

\haiku{Onwillekeurig.}{begint hij zo'n beetje te}{zoemen in zijn baard}\\

\haiku{voor het Consulaat.}{stond hij gelijk met een Turk}{of een Hottentot}\\

\haiku{De mannen echter,.}{praten gewoon terwijl ze}{naderbij komen}\\

\haiku{{\textquoteleft}Ik zou het prettig...}{vinden als we niet alle}{contact verloren}\\

\haiku{als hij maar een doel,,.}{kende of een halve gek}{was gelijk Baruch}\\

\haiku{Het is een dorp op,.}{zichzelf hier en ik ben bang}{voor indiscretie}\\

\haiku{de belediging.}{in zijn laatste gezegde}{steekt Baruch niet eens}\\

\haiku{Ik veronderstel...{\textquoteright} {\textquoteleft},{\textquoteright}.}{Laat ons er straks over spreken}{antwoordt Hornemann}\\

\haiku{{\textquoteright} {\textquoteleft}Och...{\textquoteright} Hornemann weet.}{eerlijk gezegd niet wat hij}{ervan denken moet}\\

\haiku{Sloeg zich er prachtig.}{doorheen met heilmassage}{en meer dergelijks}\\

\haiku{Je hebt het hele,.}{leven nog v\'o\'or je met}{allerlei kansen}\\

\haiku{Ongaarne... maar mijn.}{bangelijke zwerflust heeft}{me weer te pakken}\\

\haiku{Nu laat hij heel de.}{stroom van zijn onstuimigheid}{over haar heen storten}\\

\haiku{Ze zou  het niet.}{kunnen verdragen dat hij}{dit anders voelde}\\

\haiku{Als arts heeft hij een,.}{nederlaag geleden dat}{blijft hij zo voelen}\\

\haiku{Wie verstaat iets van?}{die dwaas-verwikkelde}{spaanse politiek}\\

\haiku{Hij wilde een rol,.}{spelen maar men vertelde}{hem niets van het spel}\\

\haiku{{\textquoteleft}Feitelijk is het.}{een driepersoons-huwelijk}{dat wij hier leiden}\\

\haiku{Schlauch echter wil.}{de dingen niet tragischer}{nemen dan ze zijn}\\

\haiku{nu zal het Schlauch.}{niet moeilijk vallen over hem}{te triomferen}\\

\haiku{Het is het enige.}{wat de ingehoudenheid}{der mensen verraadt}\\

\haiku{Daarom m\'oet hij met,.}{Hertha spreken voordat hij met}{de agenten meegaat}\\

\haiku{Daarom moedigt zij.}{Hornemann juist aan om zijn}{plan door te zetten}\\

\haiku{Hij heeft het zijne,.}{gezegd en zijn eten behoeft}{niet koud te worden}\\

\haiku{Des te groter zal,;}{haar wraak zijn op de dag dat}{afgerekend wordt}\\

\haiku{Hij kan haar alles,,,,.}{geven wat ze wil luxe zorg}{tederheid reizen}\\

\haiku{Er m\'oet een dokter,.}{komen zo kan hij niet de}{nacht in gaan met haar}\\

\haiku{En ergens verweg,.}{dreunen doffe slagen die}{misschien schoten zijn}\\

\haiku{Ze moet niet merken.}{dat hij al te veel prijs stelt}{op haar wederkeer}\\

\haiku{Schlauch wil gaarne.}{alles beloven wat de}{ander maar verlangt}\\

\haiku{Zijn woede moet eerst,.}{tot bedaren komen dan}{kan hij overleggen}\\

\haiku{Waar je ook komt, als.}{intellectuele jood}{word je slachtoffer}\\

\haiku{Van alle dingen {\textquoteleft}{\textquoteright};}{is niets betrekkelijker}{dan wat menrecht noemt}\\

\haiku{Ghetto-jood heeft hij.}{Mendel vaak verachtelijk}{bij zichzelf genoemd}\\

\haiku{{\textquoteleft}Dat soort natuurlijk,.}{dat in Duitsland nog altijd}{met de nazi's heult}\\

\haiku{maar de waarheid is,;}{dat hij zich best kan laten}{zien als zestiger}\\

\haiku{Eerst thans, in onze.}{nieuwe diaspora is het}{opnieuw ontsluierd}\\

\haiku{{\textquoteleft}En is dat ook in,?}{het kort uit te leggen hoe}{dit in elkaar zit}\\

\haiku{{\textquoteright} Lotte Bernstein kan.}{zich niet weerhouden een zucht}{van schrik te slaken}\\

\haiku{Wat ik er tot nu,.}{toe van ken bevalt me best.}{Ook van de mensen}\\

\haiku{Ik kan er helaas,.}{niet op wachten en ik heb}{er ook geen zin in}\\

\haiku{{\textquoteright} Sabine schudt van,:}{neen maar eerlijkheidshalve}{voegt ze eraan toe}\\

\haiku{Sabine gelooft,.}{echter dat ze daar op den}{duur wel overheen komt}\\

\haiku{Als Felipe haar,:}{bij het weggaan begeleidt}{vraagt hij hartelijk}\\

\haiku{De inboorlingen;}{proberen ieder geldstuk}{op zijn zilverklank}\\

\haiku{{\textquoteright} Ook Lotte Bernstein.}{spreekt op dit ogenblik zoals}{haar eigen moeder}\\

\haiku{Het is waar, ze heeft.}{dikwijls aan een verhouding}{met Tibor gedacht}\\

\haiku{maar die uitte hij.}{niet bij onverschillige}{gelegenheden}\\

\haiku{En toch zal ze het,,.}{doen om hem te bewijzen}{wat h{\'\i}j h\'a\'ar waard is}\\

\haiku{Sabine opent het.}{portier om hem naast haar te}{laten instappen}\\

\haiku{De contouren van.}{het fort op de top snijden}{hoekig in de lucht}\\

\haiku{Natuurlijk, alle.}{emigranten hebben hier de}{aansluiting gemist}\\

\haiku{een vrouw  kan hier,.}{niet komen in haar eentje}{zonder adjudant}\\

\haiku{{\textquoteleft}Ik geloof dat je,.}{niet erg gesteld bent op mijn}{gezelschap vanavond}\\

\haiku{En misschien ga je.}{me toch meer waarderen als}{je me beter kent}\\

\haiku{Ze haat zichzelf... en,.}{toch ze m\'oet en ze w{\`\i}l een}{spaans paspoort hebben}\\

\haiku{Dit is uiterste,...}{noodzaak want het gaat immers}{om leven en dood}\\

\haiku{Maar dat kan hij niet,.}{meer begrijpen de arme}{verdwaasde Tibor}\\

\haiku{Het overpeinzen komt,,.}{later Sabienchen als de}{strijd gestreden is}\\

\haiku{Onder hem ziet hij.}{nog slechts in het water een}{blauwwitte schuimkring}\\

\haiku{Een paar slaan een kruis, {\textquoteleft}{\textquoteright}.}{de overigen putten zich}{uit inme cago's}\\

\subsection{Uit: Afdaling in de vulkaan}

\haiku{Albert Helman}{Afdaling in de vulkaan}{Colofon}\\

\haiku{Het is al zo lang.}{geleden en je was nog}{geen vijf toen hij stierf}\\

\haiku{{\textquoteleft}Soms geloof ik, dat,{\textquoteright}.}{je mij moet haten sprak het}{meisje voor zich uit}\\

\haiku{Marjorie kwam naar,:}{me toe pakte mij bij de}{schouder en riep uit}\\

\haiku{Een hele poos moest,.}{er verlopen zijn toen ik}{mij hoorde roepen}\\

\haiku{Wanneer jij dan ook, -,.}{wat neemt de kralenketting}{die staat vast bij zwart}\\

\haiku{Haar blik viel op het.}{offermes dat nog boven}{in de koffer lag}\\

\haiku{De dame hield haar,;}{ogen meestentijds gesloten}{leek ietwat vermoeid}\\

\haiku{{\textquoteright} De betrekkelijk,.}{saaie stem die dit verteld had}{zweeg voor lange tijd}\\

\haiku{hoe kom ik nog ooit,;}{bijtijds terug aan het eind}{waar mijn coup\'e is}\\

\haiku{Daarom heb ik maar.}{gewacht tot wij elkander}{beter kunnen zien}\\

\haiku{En hij beriep zich.}{op al de dikke boeken}{die hij bij zich had}\\

\haiku{{\textquoteright} Een goed uur later,.}{pas wist ik wat er precies}{van mij verwacht werd}\\

\haiku{Er scheen hem iets te,:}{binnen te schieten want nu}{zei hij joviaal}\\

\haiku{{\textquoteright} zei Art\'egui op.}{een toon alsof hij mij de}{grootste dienst bewees}\\

\haiku{- was ik niet verdacht,.}{geweest en ik moest mij een}{ogenblik bezinnen}\\

\haiku{Ach, wat een trieste,.}{dag na de miserie van}{de nacht tevoren}\\

\haiku{Ik heb die oude.}{Amaral altijd voorspeld dat}{het mislukken zou}\\

\haiku{je krijgt van mij een,.}{eigen hond voor jou alleen}{wanneer je me trouwt}\\

\haiku{Ik zei - misschien ben -:}{ik onduidelijk geweest}{maar ik bedoelde}\\

\haiku{{\textquoteright} vroeg ik onderweg,.}{nog altijd gemelijk en}{terneergeslagen}\\

\haiku{Ik durf erom te,.}{wedden dat hij nog van zich}{zal laten spreken}\\

\haiku{{\textquoteright} vroeg ik, blij dat niet.}{de eeuwige Rancho meer}{het thema vormde}\\

\haiku{Het lot dat mij tot,.}{hier gevoerd had zou mij ook}{wel verder helpen}\\

\haiku{Niet alleen dat het,.}{mij niet verraste maar het}{was mij zelfs welkom}\\

\haiku{{\textquoteleft}De Alpen en de.}{Pyrenee\"en gaven mij}{datzelfde gevoel}\\

\haiku{{\textquoteright} Plotseling brak hij,.}{af en zweeg de handen op}{zijn knie gevouwen}\\

\haiku{In elk geval, er;}{was een meisje waar ik erg}{verliefd op raakte}\\

\haiku{Hoewel er toen ook.}{veel posada's en hotels}{te vinden waren}\\

\haiku{Zodat ze al met,.}{al toch met me trouwde en}{een brave vrouw was}\\

\haiku{Maar bultenaars zijn,.}{aanstonds driftig en ik had}{geen zin in ruzie}\\

\haiku{Ze vroegen of de...}{padr\'on het zich dan niet meer}{herinneren kon}\\

\haiku{Het was dezelfde}{als die van de veel minder}{wilde stroom waarlangs}\\

\haiku{Zelfs het brouwsel van.}{Remigio bood weinig baat}{meer op die hoogte}\\

\haiku{Nu ik echter weer,}{wat uitgerust was kropen}{een voor een al de}\\

\haiku{Maar het was geen raad, -.}{alleen dan zou ik het toch}{niet hebben gedaan}\\

\haiku{En het is goed om,.}{zo'n doel te hebben groter}{dan je eigenbaat}\\

\haiku{{\textquoteleft}Ja, mi jefe, hij,.}{die aanvoert moet zeer hard zijn}{en illusieloos}\\

\haiku{{\textquoteright} {\textquoteleft}Ik ben klaar wakker,{\textquoteright}, {\textquoteleft}.}{antwoordde ikna deze}{hele dag van rust}\\

\haiku{In de namiddag.}{van de tweede dag scheidden}{onze wegen zich}\\

\haiku{Het kon niet anders,.}{het moest een menselijke}{aanwezigheid zijn}\\

\haiku{Ik verhaastte mijn;}{stappen over de weg die naar}{het schijnsel voerde}\\

\haiku{Wij sliepen wakend,,.}{wachtten zonder haast nu het}{einddoel was bereikt}\\

\haiku{Zijn snor stond schriel en,.}{rafelig aan de punten}{dwars over zijn gezicht}\\

\haiku{{\textquoteleft}Was hij zo bang, dat?}{hij een aspirine nam}{als hij moest vechten}\\

\haiku{Moctezuma moest,:}{me wel erg na{\"\i}ef vinden}{want hij antwoordde}\\

\haiku{{\textquoteright} Wat later ging ik.}{met de Ingeniero het}{terrein verkennen}\\

\haiku{{\textquoteleft}Als we weer bij het,.}{woonhuis zijn zal ik je ook}{d\`at nog laten zien}\\

\haiku{{\textquoteright} {\textquoteleft}En je weet niet dat?}{er nog een slijpsteen staat in}{de grote corral}\\

\haiku{Als het waar is, zal,...}{hij het zeker merken nog}{v\'o\'or ieder ander}\\

\haiku{Daarbuiten is naar.}{mijn opvatting de wereld}{dood gelijk een berg}\\

\haiku{Door hielgeschop dreef;}{hij de onwillige ezel}{naar de bidkapel}\\

\haiku{Eerst had ik nog wel,.}{geduld toen ik slechts weinig}{wist en verweg was}\\

\haiku{Denk niet dat ik hier,.}{alleen maar rondkijk enkel}{maar beveel en vit}\\

\haiku{Zo kalm en rustig,.}{als de beste fokstier ook}{toen ik hem losliet}\\

\haiku{Dus nam ik hem het,.}{medaljon weer af wat hij}{braaf liet geschieden}\\

\haiku{{\textquoteright} {\textquoteleft}Als een bewijs dat,{\textquoteright}.}{u ook practisch werk doet op}{de rancho zei ik}\\

\haiku{{\textquoteleft}Geen echte zelfmoord,.}{natuurlijk maar zoals de}{ezels in Yucat\'an}\\

\haiku{Arme Antonio...,.}{Denk niet dat hij daarbij ook}{in iets tekort schoot}\\

\haiku{Hij kroop heel dikwijls,}{in mijn armen legde}{zich gelijk een kind}\\

\haiku{en wat een moeite,,.}{en verdriet om slechts tot hier}{vandaag te komen}\\

\haiku{Want wat ik in die,.}{flits gezien had overtrof mijn}{ergste fantasie}\\

\haiku{Misschien vergis ik,,.}{mij dacht hij of heb ik het}{in mijn slaap gedaan}\\

\haiku{hij had een nieuwe,.}{meegenomen en nu stond}{er maar een stompje}\\

\haiku{Daarna ging hij naar,,.}{binnen alle kamers door}{en zag alweer niets}\\

\haiku{De betovering}{van de hut scheen er juist door}{te verergeren}\\

\haiku{Maar geef ons vruchtbaar.}{zaad en goede lendenen}{aan onze vrouwen}\\

\haiku{De eerste keer maar,.}{enkele dagen later}{enkele weken}\\

\haiku{Toen was het blijkbaar.}{juist de hoogste uiting van}{liefde en fatsoen}\\

\haiku{En dacht u dat ik?}{voor mariachi langs de straat}{zou willen spelen}\\

\haiku{Zijn zuster die in,.}{Mexico gebleven was mocht}{zeggen wat ze wou}\\

\haiku{{\textquoteright} {\textquoteleft}Dat is heel ver. Maar,.}{ik ben blij dat je weer bij}{ons bent mi capit\'an}\\

\haiku{Terwijl ze nu zelf.}{op de vloer ernaast zich in}{hun deken rolden}\\

\haiku{Mijlen, denk ik, want,,.}{er kwam geen einde aan in}{geen half uur geen uur}\\

\haiku{Het klamme angstzweet,.}{brak mij uit om van het paard}{maar niet te spreken}\\

\haiku{Maar wanneer Ort{\'\i}z hem,.}{aangesteld heeft zal hij zijn}{verdiensten hebben}\\

\haiku{Wanneer ik daarbij;}{dan maar zorgde Veyt{\'\i}a uit}{de weg te blijven}\\

\haiku{Het is zeker dat.}{wij nu nog sterker zijn dan}{de cabrones ginds}\\

\haiku{aarzelend - wie gaat -:}{er graag voor d\'efaitist door}{maar bekommerd zei}\\

\haiku{Ginds, waar nu dat huis,.}{staat met dat ongewone}{dak je kent het wel}\\

\haiku{De indio is veel.}{te zeer verslaafd aan dit wat}{hem juist ondermijnt}\\

\haiku{{\textquoteright} {\textquoteleft}Zeker is geen woord,,{\textquoteright}.}{meer als u mij dat toestaat}{antwoordde Veyt{\'\i}a}\\

\haiku{Ik had het kunnen...,.}{weten dat ik nooit ach laat}{ik daarvan zwijgen}\\

\haiku{Maar mijn uur was zelfs,.}{toen nog niet geslagen ook}{dat stond geschreven}\\

\haiku{Ik zal hem een keer,...}{tortilla's bakken die hij}{daarna nooit meer eet}\\

\haiku{Aan Teobaldo, die,.}{toevallig in mijn buurt kwam}{vroeg ik wat het was}\\

\haiku{Een paar commando's,.}{met zijn stem die plotseling}{schril klonk in de nacht}\\

\haiku{Terwijl ik juist de - -.}{situatie hopeloos}{hier wilde redden}\\

\haiku{{\textquoteleft}En wij winnen de,.}{verbinding met Ort{\'\i}z die veel}{belangrijker is}\\

\haiku{{\textquoteright} En op het lichte:}{schouderophalen van don}{Salustiano}\\

\haiku{Daarom had hij al;}{die dagen nutteloos}{voorbij laten gaan}\\

\haiku{Slecht nieuws heeft de tijd,{\textquoteright}.}{sprak don Salustiano}{bijna fluisterend}\\

\haiku{Het was alsof er.}{buiten slechts gefluisterd werd}{en zacht geslopen}\\

\haiku{de maat die hem op.}{deze laatste verre reis}{moest begeleiden}\\

\haiku{{\textquoteleft}Ik wist niet,{\textquoteright} lachte, {\textquoteleft}.}{hijdat er in u ook nog}{een generaal stak}\\

\haiku{Een vriend in goede,?}{dagen moest het zeker zijn}{in slechte nietwaar}\\

\haiku{{\textquoteright} {\textquoteleft}Even duidelijk als,,.}{het feit dat dit uw dood zou}{zijn binnen het uur}\\

\haiku{Wie anders dan die?}{zoutelozen die het zout}{der aarde moesten zijn}\\

\haiku{{\textquoteright} Hoogst nieuwsgierig ging.}{ik naar de kamer van don}{Salustiano}\\

\haiku{Deze man was een,.}{van Art\'egui's verraders}{dat scheen vast te staan}\\

\haiku{Het lijkt misschien dat,....}{ik verloren heb maar ik}{zal verder zwijgen}\\

\haiku{Het gegeven woord.}{van onze jefe moeten}{wij eerbiedigen}\\

\haiku{{\textquoteleft}Juanito is.}{al onderweg om het u}{te komen zeggen}\\

\haiku{de grootste moed is,.}{misschien die waarbij je beeft}{maar op je post blijft}\\

\haiku{Hij jammerde steeds,}{maar hetzelfde tot opeens}{de negerkoning}\\

\haiku{het hoorde, hem naar:}{zich toe wenkte en hem in}{deugdelijk spaans vroeg}\\

\haiku{Don Nicol\'as was het,;}{hiermee volkomen eens en}{schoot het geld graag voor}\\

\haiku{Helaas, mijn trouwe.}{Candelario ontbrak daar}{in de voorhoede}\\

\haiku{Ik zal je op een.}{keer misschien bewijzen dat}{ik juist gegist heb}\\

\haiku{en ik stapte over,,.}{houtskool vastgekoekte as}{die zelfs nu nog stonk}\\

\haiku{Het alledaagse,.}{lot van zoveel meisjes in}{dit land bedacht ik}\\

\haiku{Want dan zullen ze,,{\textquoteright}.}{toch weer eens komen vroeg of}{laat verzuchtte zij}\\

\haiku{Wel, ik ben blij dat,.}{dit een droge streek is waar}{zoiets niet voorkomt}\\

\haiku{{\textquoteleft}Eindelijk is het,.}{toch gebeurd waarvoor ik al}{die tijd al bang was}\\

\haiku{{\textquoteright} {\textquoteleft}Dit is maar de helft,{\textquoteright}.}{van de waarheid meende don}{Salustiano}\\

\haiku{Daarom wordt elke.}{oorlog weer gevolgd door een}{of nog meer nieuwe}\\

\haiku{En als hij hem nu,.}{maar teruggeeft dan zijn wij}{beiden geholpen}\\

\haiku{{\textquoteleft}Maar denkt erom, je,,?}{steekt geen poot uit als ik het}{niet zeg begrepen}\\

\haiku{het kan voor hem ook,.}{nuttig zijn dit zonderling}{verhaal te horen}\\

\haiku{En toch...{\textquoteright} {\textquoteleft}U wordt de,{\textquoteright}.}{oudste hacendado van}{Tamaulipas zei Isidro}\\

\haiku{Ik heb er ook voor;}{gezorgd dat niets meer over is}{van al dat prutswerk}\\

\haiku{De breuknaadjes zijn,...}{weliswaar nog te zien maar}{als een haar zo fijn}\\

\haiku{Op de Rancho der.}{Drievuldigheid werd het zeer}{rustig en vertrouwd}\\

\haiku{{\textquoteleft}Dat heeft bovendien.}{niets met oude of nieuwe}{mensen te maken}\\

\haiku{Hij scheen een echte;}{fat en het beste was hem}{nog niet goed genoeg}\\

\haiku{Wij verzwegen dat,:}{het beest nog ongebroken}{was maar zeiden wel}\\

\haiku{{\textquoteright} {\textquoteleft}Marjorie ziet er,,}{heus niet naar uit dat ze een}{gebroken hart heeft}\\

\haiku{Maar ik stelde met:}{een soort deskundigheid reeds}{van tevoren vast}\\

\subsection{Uit: Amor ontdekt Aruba}

\haiku{{\textquoteleft}Maar u behoeft zich.}{gedurende de wachttijd}{niet te vervelen}\\

\haiku{U kunt dan daarmee,;}{naar de stad gaan of iets van}{het eiland gaan zien}\\

\haiku{Maar denkt u erom.}{dat wij al over een paar uur}{kunnen vertrekken}\\

\haiku{Er woei een adem van:}{oneindigheid door deze}{ritselende rust}\\

\haiku{Want ik ben verliefd,.}{geworden op uw eiland}{verliefd op Aruba}\\

\haiku{Landelijk en toch,.}{niet boers afgetrokken en}{toch niet eenzelvig}\\

\haiku{Toch werd hier in de,{\textquoteright}.}{buurt vrij veel goud gevonden}{vertelde Sjon Eli}\\

\haiku{{\textquoteright} {\textquoteleft}Het is een flauwe,,{\textquoteright},.}{rottige grap zei Cynthia}{ook even buiten adem}\\

\haiku{Integendeel, dan.}{tonen zij zich juist in hun}{ware gedaante}\\

\haiku{{\textquoteleft}Hij en ik geven.}{tenminste al onze vrije}{tijd aan dit eiland}\\

\haiku{{\textquoteleft}Eilandbewoners,{\textquoteright}.}{zijn gek op de zee merkte}{Cynthia lachend op}\\

\haiku{{\textquoteright} De steile, platte.}{kust bleef geaccidenteerd}{in zijn contouren}\\

\haiku{Nog standhoudend, een,,.}{eeuw twee eeuwen misschien maar}{zeker niet voorgoed}\\

\haiku{{\textquoteright} vroeg Cynthia, meer aan.}{zichzelf dan aan Sjon Eli tot}{wien zij zich richtte}\\

\haiku{{\textquoteleft}Er woei een adem van:}{oneindigheid door deze}{ritselende rust}\\

\subsection{Uit: De dolle dictator. Het ondoorgrondelijke leven van Juan Manuel de Rosas}

\haiku{Ook zij wier beide.}{ouders uit Europa naar}{Amerika kwamen}\\

\haiku{Mesties- halfbloed,.}{nakomeling van blanken}{en indianen}\\

\haiku{Misschien zijn ze nog,,.}{vijf leguas ver maar je}{zult zien ze komen}\\

\haiku{Maar don Clemente,.}{weet ze \'o\'ok te ruiken en}{langer dan jij}\\

\haiku{Don Clemente weet.}{dat zijn paard verloren is}{en springt op de grond}\\

\haiku{eerst zuidwaarts naar de, '.}{heuvels dan naart Oosten}{waar de zee moet zijn}\\

\haiku{{\textquoteright} De jongen houdt een,.}{harde kop een etmaal blijft}{hij opgesloten}\\

\haiku{Er stijgt geloei op,.}{dat verlangend uitklinkt in}{de oneindigheid}\\

\haiku{Hij zit toch zeker.}{even vast te paard gelijk de}{beste onder hen}\\

\haiku{De dieren paren,?}{toch zodra hun jonge kracht}{begint te rijpen}\\

\haiku{ze weet dat het niet,.}{waar is wat ze zei ze kent}{haar oudste immers}\\

\haiku{En ze weet ook dat.}{hij nu het toppunt van zijn}{woede heeft bereikt}\\

\haiku{Want ook Europa.}{beleeft weer een hausse in}{haar koningsschappen}\\

\haiku{Goed, waarvoor heeft hij?}{anders de lange avonden}{op Los Cerrillos}\\

\haiku{E\'enmaal in de;}{week brengt een koerier hem zijn}{brieven en zijn krant}\\

\haiku{Deze laatsten, en,.}{vooral zijn moeder nemen}{het hem hoogst kwalijk}\\

\haiku{Men schiet niet op, want:}{L\'opez is onvermurwbaar in}{een van zijn eisen}\\

\haiku{Verder laat hij niets,.}{uit en Rosas voelt zich steeds}{meer gekrenkt door hem}\\

\haiku{Die zwaait hij in de.}{wind met het bosje hooi dat}{plotseling \`opvlamt}\\

\haiku{Ten Noorden daarvan.}{mogen nu de christenen}{hun forten bouwen}\\

\haiku{Terwijl dit gaande,.}{is versterkt Rosas de grens}{zoveel mogelijk}\\

\haiku{Ook Rosas wordt zich.}{nu de herhalingsdwang in}{zijn leven bewust}\\

\haiku{zijn blonde haren;}{aan de conquistadores}{uit Noord-Spanje}\\

\haiku{Was het niet billijk?}{dat er meer indianen}{stierven dan koeien}\\

\haiku{Ze mogen zich bij;}{de regering vervoegen}{als ze lust hebben}\\

\haiku{Het zijn kisten vol:}{die hij meesleept aan boord om}{te bestuderen}\\

\haiku{Het is ontzaglijk}{hoeveel hij reeds weet en hoe}{dit vele weten}\\

\haiku{Hij is een gaucho;}{en door niemand laat hij zich}{in de kaart kijken}\\

\haiku{het spreekt vanzelf dat,.}{hij degene is bij wie}{de beslissing ligt}\\

\haiku{{\textquoteright} Juan Manuel.}{is gedwongen wederom}{hoog spel te spelen}\\

\haiku{Nu is Quiroga,:}{wakker en gebogen uit}{het raampje roept hij}\\

\haiku{Het antwoord is een:}{half-gefluisterde}{roep uit aller mond}\\

\haiku{Allen vliegen op,.}{haar wenken beven voor haar}{striemende woorden}\\

\haiku{In de familie.}{Rosas aarden de mannen}{steeds naar hun moeders}\\

\haiku{Hij moet er niet te,.}{veel aan denken het zou hem}{weemoedig maken}\\

\haiku{Hoewel het maar een.}{vodderig  berichtje}{van Cuiti\~no is}\\

\haiku{Maar hij wil niet de,,.}{enige zijn niet de enige}{verrader niet hier}\\

\haiku{{\textquoteright} Daarop gaat Ram\'on snel.}{naar huis om zich gereed te}{maken voor de reis}\\

\haiku{Het gaat van mond tot.}{mond dat Ram\'on Maza een der}{hoofd-raddraaiers is}\\

\haiku{Er is niets aan te,.}{doen al moet de halve stad}{eraan geloven}\\

\haiku{Rosas kijkt koel naar,,:}{haar handenwringen haalt de}{schouders op en zegt}\\

\haiku{{\textquoteright} Hiermee heeft hij de.}{ander nog eens extra aan}{het schrikken gemaakt}\\

\haiku{Ze imiteren hem.}{en weten hun lot aan het}{zijne gebonden}\\

\haiku{Hij ziet het vochtig;}{schitteren van haar ogen en}{haar lieve glimlach}\\

\haiku{Ze liegen er niet,,.}{om ze spreken klare taal}{tot in hun titels}\\

\haiku{De sluwheid waarmee,.}{hij daarbij te werk gaat zegt}{niets van zijn verstand}\\

\haiku{Hij wil lijken om,.}{zich heen hebben het komt er}{niet op aan van wie}\\

\haiku{Onmiddellijk had.}{de dikke mulat aan dit}{bevel gehoorzaamd}\\

\haiku{Las Balchitas, is?}{dat geen oord in het diepste}{van Patagoni\"e}\\

\haiku{{\textquoteright} - {\textquoteleft}Dan zullen wij hem,{\textquoteright}.}{waardig ontvangen antwoordt}{Juan Manuel}\\

\haiku{Juan Manuel:}{laat deze troep aantreden}{en verklaart plechtig}\\

\haiku{Daarna zullen wij.}{overgaan tot de plechtigheid}{van de wederdoop}\\

\haiku{Vaag, onkenbaar vaag.}{schemert hem het beeld van zijn}{moeder voor de ogen}\\

\haiku{Juan Manuel.}{heeft zich ook tegen deze}{liefde fel gekant}\\

\haiku{Nog wordt hij door de!}{commandant behandeld als}{een hoge eregast}\\

\haiku{En het andere...?}{dan haar vijf kinderen die}{ook de zijne zijn}\\

\haiku{Vrienden, vijanden,.}{ze zijn haast allen van het}{toneel verdwenen}\\

\haiku{- {\textquoteleft}Er is geen die de,,.}{wet stelt geen die eist alleen}{ik zelf tot mijzelf}\\

\haiku{bij sneeuw en wind blijft,.}{hij buiten werken net als}{op een lentedag}\\

\haiku{En Rosas met zijn;}{ijzeren energie begint}{weer te bevelen}\\

\haiku{En gezuiverd in,.}{de oude onbedorven}{indiaanse schoot}\\

\haiku{De trotse Pampas,?}{de Tehueltsches en}{de Pueltsches}\\

\subsection{Uit: Het euvel Gods}

\haiku{Maar jij, mijn vriend, ik.}{weet dat jij de jouwe kent}{als ik de mijne}\\

\haiku{Tot zoo laat was er;}{tusschen al Gods dagen geen}{verschil voor Jukkers}\\

\haiku{Spranger belde mij....}{op voor een spoedoperatie}{voor Arthur Christensen}\\

\haiku{De ander trachtte,.}{hem te kalmeeren maar hij}{wond zich steeds meer op}\\

\haiku{het gewelf was zoo,.}{hoog dat het in het midden}{niet meer te zien was}\\

\haiku{Als menschen uit een.}{andere wereld staarden}{wij elkander aan}\\

\haiku{Ik meen te mogen:}{veronderstellen dat hij}{heeft uitgeroepen}\\

\haiku{In ieder geval,.}{was het beestje omnivoor}{stelde Fowler vast}\\

\haiku{Het brak tusschen mijn,.}{tanden en ik bekeek het}{stuk dat ik vasthield}\\

\haiku{Ik zag hoe Fowler,.}{kokhalsde en was niet in}{staat te antwoorden}\\

\haiku{Vermoeid was zij nu,.}{na drie jaren geworden}{ten doode vermoeid}\\

\haiku{Vragen durfde hij, ':}{niet want wat hijt eerst zou}{willen vragen was}\\

\haiku{En omdat je uit.}{eigen ervaring reeds weet}{hoe het daar zijn kan}\\

\haiku{Maar de oude vrouw.}{trok me aan de mouw mee naar}{een andere reet}\\

\haiku{{\textquoteleft}Wij moeten nu de,.}{andere kant uit want het}{laatste huis is leeg}\\

\haiku{En weer is schemer.}{gevallen over de wankel}{voortstappende man}\\

\haiku{Ebenhezer rookte.}{uit een steenen pijpje en zon}{op al zijn godsdienst}\\

\haiku{Ze komen in het,.}{huis waar het niet noodig is het}{licht te ontsteken}\\

\haiku{Bijna is het dag,{\textquoteright},.}{zegt hij naar buiten stappend}{waar de paarden staan}\\

\haiku{juist zooals het op diens;}{minutieuze kaarten}{was aangegeven}\\

\haiku{De noordelijkste!}{kaart van pater Martinus}{was onnauwkeurig}\\

\haiku{aan sterke takels.}{rijdt de groote ketel gloeiend}{ijzer uit het vuur}\\

\haiku{Zoo moest ik het ook,.}{in mijn handen  kunnen}{vasthouden dacht Cobbs}\\

\haiku{{\textquoteright} Maar slap en willoos.}{liet ze zich doorgolven en}{Cobbs verloor de moed}\\

\haiku{Waarom heb ik ook,.}{geen moeder fluisterde het}{binnenste van Cobbs}\\

\haiku{Een mensch bewaart het,}{schoonst de herinneringen}{aan bergen zoolang}\\

\subsection{Uit: Facetten van de Surinaamse samenleving}

\haiku{Hun stemmen spreken,.}{door de puyai heen als}{bij een buikspreker}\\

\haiku{De therapie volgt.}{dan uit de diagnose en}{de etiologie}\\

\haiku{Dit hangt vaak meer van,.}{de pati\"ent af dan van}{de middeltjes zelf}\\

\haiku{Niet m\'e\'er overigens;}{dan de stadsbewoners van}{menig ander land}\\

\haiku{immers spoedig, als,:}{jonge vrouw had de dochter}{maar \'e\'en bestemming}\\

\haiku{{\textquoteleft}Avonturen{\textquoteright} winden.}{hem zelden op en hij zoekt}{ze ook niet bepaald}\\

\haiku{Ook bij andere;}{gelegenheden zingen}{ze spotliederen}\\

\haiku{Het huisraad staat steeds.}{glimmend te pronken in het}{halflicht van de hut}\\

\haiku{Het taboe van de...}{menstruerende vrouw zit}{ook nog vrij diep in}\\

\haiku{Een hangmat is voor.}{Indianen uiteraard}{een primair bezit}\\

\haiku{Dit dient hij op zijn.}{minst tot aan de geboorte}{van het kind te doen}\\

\haiku{{\textquoteleft}Natuurlijk, meneer,.}{de Voorzitter zodra er}{kinderen komen}\\

\haiku{een korjaal, een huis,:}{of de symbolisering}{van zijn gevoelens}\\

\subsection{Uit: De G.G. van Tellus}

\haiku{{\textquoteleft}Lijken net van die,;}{romanschrijvertjes kunnen}{niet samenvatten}\\

\haiku{{\textquoteleft}Die stommelingen.}{maken hun rapporten nooit}{volledig genoeg}\\

\haiku{Moet ik ze een voor?}{een bij me laten komen}{en met ze praten}\\

\haiku{Men kan alles uit,.}{de weg lopen behalve}{de medemensen}\\

\haiku{Dan verdwijn ik nu,{\textquoteright},,.}{maar zegt ze ondanks zichzelf}{maar toch half vragend}\\

\haiku{Hij vindt dat Edm\'ee.}{Duval en deze bloemen}{bij elkaar horen}\\

\haiku{Ze heeft een soort van,,.}{weerstand trots of een restant}{conventie misschien}\\

\haiku{Ach, moet nu reeds haar?}{betovering verbroken}{worden door een bruut}\\

\haiku{Op 19 dezer is.}{zijn gemelde novelle}{gereedgekomen}\\

\haiku{Inderdaad in een.}{toestand die weinig verschilt}{met die van beesten}\\

\haiku{{\textquoteleft}O, mon dieu, van.}{een bullebak zou ik niet}{erger verschrikken}\\

\haiku{Rosy denkt alleen.}{aan de vorige romans}{van Norman Angus}\\

\haiku{Ja, ja!) dan moet ik.}{u zeggen dat ik het een}{immoreel werk vind}\\

\haiku{{\textquoteright} vraagt Tom op een toon.}{waarin zijns ondanks toch een}{tikje spot doorklinkt}\\

\haiku{Ze huilen mee met...}{de wolven met wie ze het}{aas mogen delen}\\

\haiku{Het werken zelf is,.}{geen slavernij al is het}{nog zo zwaar en veel}\\

\haiku{Ouders zijn niet oud,.}{zolang de kinderen nog}{hun zorg behoeven}\\

\haiku{Nu beginnen de.}{jongens te praten en wordt}{over en weer verteld}\\

\haiku{Dan wordt het direct,.}{moord en doodslag echtscheiding}{en advocaten}\\

\haiku{een wijffie dat nog,{\textquoteright}.}{best te zoenen valt vertelt}{Eddy openhartig}\\

\haiku{Het brengt hem zelf ook,:}{in een speelse stemming en}{plagerig zegt hij}\\

\haiku{{\textquoteleft}Ik zit toch altijd.}{zo'n beetje te denken over}{wat ik zetten moet}\\

\haiku{Een mooie vrouw weet dat.}{ze een bankrekening heeft}{bij iedere man}\\

\haiku{Maar hij ontdekte,.}{dat dit gevoel zelf fout was}{een zonde in hem}\\

\haiku{Ze had hem kunnen,.}{schrijven misschien zou hij haar}{geholpen hebben}\\

\haiku{wanneer je je in.}{staat voelde zelfs je eigen}{moeder te haten}\\

\haiku{wist hij slechts hoe het,.}{te bereiken hoe het het}{diepst te treffen}\\

\haiku{hij is spontaner,,.}{ondernemender net als}{de vlam van een brand}\\

\haiku{Ten langen leste,.}{voor een spiegel waarin hij}{keek zonder te zien}\\

\haiku{Hij weet ook niets van:}{die andere smerige}{geschiedenissen}\\

\haiku{Alles aan hem is,,.}{rein onaangetast ofschoon}{hij een lamme is}\\

\haiku{En hoe vlijmend en.}{geestig kunnen zijn woorden}{een vijand wonden}\\

\haiku{{\textquoteleft}Neen, lieveling, je.}{weet dat ik op zee nooit zin}{heb om te lezen}\\

\haiku{Dacht je dat ik een?}{half uur kon werken zonder}{aan je te denken}\\

\haiku{De jonge Joachim,.}{Dijkman dan was nog maar een}{paar jaar in het vak}\\

\haiku{Niet voor niets had hij...}{aan het leven eigenlijk}{de brui gegeven}\\

\haiku{{\textquoteleft}Maar verzen kun je.}{toch nooit helemaal kennen}{door lezen alleen}\\

\haiku{Wat was de inhoud?}{van dit prettig gedrukte}{deemoedig dulden}\\

\haiku{Behalve dan {\textquoteleft}De{\textquoteright},.}{wapens neer een groots epos dat}{zijn jeugd had verblijd}\\

\haiku{Hij herinnerde.}{zich niet dat hij hem ooit was}{tegengekomen}\\

\haiku{Geen ogenblik was ze.}{Joachim Dijkman meer uit de}{gedachten geweest}\\

\haiku{En telkens opnieuw.}{trachtte hij zich haar beeld voor}{de geest te brengen}\\

\haiku{Dit bracht haar op een,.}{idee dat zij zonder dralen}{wilde uitvoeren}\\

\haiku{{\textquoteright} juichte Dijkman, en.}{hij snelde reeds weg om het}{boek te gaan halen}\\

\haiku{Er ontsnapte een,.}{woord aan Joachim zonder dat}{hij wist hoe het kwam}\\

\haiku{Hij zag duidelijk:}{de visioenen die het}{boek voor hem opriep}\\

\haiku{Soms schaamde hij zich.}{bijna over de nukkigheid}{van zijn denkfuncties}\\

\haiku{Ik heb een hekel,.}{aan romans maar in dit werk}{steekt veel waardevols}\\

\haiku{{\textquoteleft}Zeer ten onrechte,,{\textquoteright}.}{geloof ik voegde Walstra er}{gewichtig aan toe}\\

\haiku{{\textquoteright} {\textquoteleft}Vertalen is op,{\textquoteright}.}{zichzelf een grote kunst viel}{de leraar hem bij}\\

\haiku{Zijn zelfkritiek zei.}{hem dat hij beter schrijven}{kon dan vertellen}\\

\haiku{{\textquoteleft}Is dat niet een van?}{de zondagsgetijden van}{dominee Stuurman}\\

\haiku{{\textquoteleft}Het wordt tijd  dat.}{er weer eens goede romans}{geschreven worden}\\

\haiku{6307-18a (inzake).}{de voorstellingswereld van}{geobserveerde}\\

\haiku{Jesses, ze wilde.}{vooral niet meer denken aan}{die beroerde school}\\

\haiku{Dat juist de dingen,.}{waar je zin in had je zo}{verpest moesten worden}\\

\haiku{{\textquoteleft}God,{\textquoteright} zei mevrouw met, {\textquoteleft}.}{verbazingmisschien is ze}{niet goed geworden}\\

\haiku{{\textquoteright} vroeg de moeder, boos.}{nu ze niet meer ongerust}{behoefde te zijn}\\

\haiku{Mijnheer Steelinks voet.}{stiet tegen het open op de}{grond gevallen boek}\\

\haiku{{\textquoteright} {\textquoteleft}Zo draai je in een,,{\textquoteright}.}{cirkeltje rond mijn beste}{sprak de G.G. waardig}\\

\haiku{Voelde ze daarom?}{tegenwoordig zoveel voor}{reclasseringswerk}\\

\haiku{Maar hij weet dat hij.}{de zaak nog niet zo gauw}{eraan kan geven}\\

\haiku{{\textquoteright} De oude heer maakt,.}{zich nu weer boos en trappelt}{voor zich uit van drift}\\

\haiku{{\textquoteright} Dani\"el antwoordt,.}{maar niet buigt enkel het hoofd}{een klein beetje meer}\\

\haiku{{\textquoteleft}Ik moet, wou, zou graag.}{weten wie Ja\"el precies}{moet optrommelen}\\

\haiku{Dan zijn jullie een,.}{waardeloze bende aan}{de ontbinding toe}\\

\haiku{{\textquoteleft}Ja, wel zonderling,{\textquoteright},.}{zegt Rose Clark terwijl ze}{Angus de hand drukt}\\

\haiku{{\textquoteright} {\textquoteleft}Interessant kind,{\textquoteright},.}{bromt Van Stolwijk zonder het}{meisje aan te zien}\\

\haiku{Willy Roosendaal.}{staat er ook maar verlegen}{en eenkennig bij}\\

\haiku{Maar hij antwoordde:}{niet op de onderbreking}{en ging kalm verder}\\

\haiku{Iets waarover de G.G..}{zeer weinig gesticht is op}{velerlei gronden}\\

\haiku{Om zoiets zomaar,.}{openbaar te maken erger}{straf is ondenkbaar}\\

\haiku{Het wordt stiller en.}{stiller in de grote zaal}{met al die mensen}\\

\haiku{een oneindige,.}{levend-dode wereld}{die voor hem openligt}\\

\haiku{dat ondervindt zij,,.}{de voormalige Rosy}{Fiddle dubbel erg}\\

\haiku{Al haar verdrongen.}{afkeer van vroeger komt nu}{naar boven als haat}\\

\haiku{{\textquoteleft}Ik geloof niet dat.}{dit het goede moment is}{voor plichtplegingen}\\

\haiku{Hij had er een boek,.}{over geschreven zoveel had}{ze wel begrepen}\\

\haiku{{\textquoteright} {\textquoteleft}Het was niet gemeen,}{genoeg of u wist er een}{boek over te schrijven}\\

\haiku{{\textquoteright} {\textquoteleft}Troost is surrogaat,{\textquoteright} {\textquoteleft}.}{antwoordt de G.G.Steek uw hand}{in uw zak en voel}\\

\haiku{De wetsvoltrekking,.}{de bekrachtiging van je}{geschonden moraal}\\

\haiku{In het begin was,.}{hij goed voor mij hij hield op}{zijn manier van mij}\\

\haiku{uit wie ik voortkom;}{en bij wie ik eindelijk}{teruggekeerd ben}\\

\haiku{Hij heeft goed praten...}{tegen al de mensen die}{nu hij hem komen}\\

\haiku{Dat moet goed zijn, want,.}{ze wil niets meer er is niets}{meer waarop ze wacht}\\

\haiku{Ze wil zich op de.}{grond laten glijden om dit}{werkelijk te doen}\\

\haiku{{\textquoteright} brengt Simone in, {\textquoteleft}?}{het middenweet u dan ook}{waar Ren\'e uithangt}\\

\haiku{Wat ze ook wel doen,,.}{maar vaak te stuntelig te}{grillig of averechts}\\

\haiku{je moet er met een,.}{voorhamer op slaan eer ze}{wat weerklank geven}\\

\subsection{Uit: Hart zonder land}

\haiku{- {\textquoteleft}Nu zullen wij niet{\textquoteright},.}{op de anderen wachten}{sprak Anne halfluid}\\

\haiku{De gewoonheid van.}{haar kinderspel had niemand}{kunnen verbazen}\\

\haiku{In bed dacht zij dan,:}{nog dikwijls aan de pop en}{het was wonderlijk}\\

\haiku{Groucha antwoordde,:}{niet maar de student pakte}{de mand op en zei}\\

\haiku{Toen vouwde hij de.}{blaadjes voorzichtig dicht en}{stak ze in zijn zak}\\

\haiku{een die ik ben, en,.}{die toch meer is dan ik want}{ook een deel van jou}\\

\haiku{{\textquoteright} - {\textquoteleft}Onze plantages,{\textquoteright}.}{zijn meer Afrika dan je denkt}{sprak de oude man}\\

\haiku{Zonder dat ze 't,}{weten spreken die kerels}{van hun vaderland}\\

\haiku{De volgende dag.}{vroeg ik of Lindor met mij}{mee mocht gaan jagen}\\

\haiku{Eens had Wassilj aan.}{een man gevraagd of het nog}{ver was naar Moskou}\\

\haiku{Er vloeide geen bloed,,.}{maar de zwerm week achteruit}{verder en verder}\\

\haiku{de takken van een,!}{heester zie ik de open plek}{en welk een schouwspel}\\

\haiku{hoe grijze nevels.}{hangen in een huis waar licht}{en liefde woonden}\\

\haiku{In Alec's arm leunde,.}{Kesini zwaarder en vermoeid toen}{zij huiswaarts keerden}\\

\haiku{En omdat ik bang,.}{ben je te verliezen zooals}{Sunanda je verloor}\\

\haiku{{\textquoteright} Doodsbleek ging zij naar,,.}{binnen haar zuster trad haar}{reeds tegemoet blij}\\

\haiku{en zich omdraaiend,.}{spuwde hij in een nis waar}{een ikoon moest hangen}\\

\haiku{Het leven dat je.}{bij een stervende bijna}{als een onrecht voelt}\\

\haiku{{\textquoteleft}Laten we hem naar,.}{het hospitaal brengen hij}{is bewusteloos}\\

\haiku{Hijgend wilde hij,.}{roepen maar zijn stem was slechts}{een schor gerochel}\\

\haiku{Mijn lippen raken,,,;}{u Ichthus witte visch die}{geslacht zijt om ons}\\

\haiku{aanstonds, morgen, pijpt,.}{hij voor en allen dansen}{wij dien Orpheus na}\\

\haiku{Zij zag hoe of hij,.}{leed en nam de kevie mee}{naar de biljartzaal}\\

\haiku{het bloed ruischt aan,,.}{zijn slapen iets hamert in}{zijn hoofd doorschokt hem}\\

\haiku{En toen de avond kwam,.}{begaf de jonge vrouw zich}{reeds vroeg ter ruste}\\

\subsection{Uit: Kleine kosmologie}

\haiku{De dubbelster volgt.}{snel en weifelloos haar weg}{en stelt geen vragen}\\

\haiku{Is het leven een?}{samoem die opkomt uit de}{verre horizon}\\

\haiku{ruimte die gevuld,.}{moet worden wijl begrenzing}{om vervulling vraagt}\\

\haiku{integratie der;}{atomen en versplintering}{der fantasie\"en}\\

\haiku{dommekracht waarmee;}{een zwak verstand heelallen}{in hun assen tilt}\\

\haiku{{\textquoteright} Opziend naar het jong,}{gezicht vol levenslust en}{overmoed glimlachte}\\

\haiku{Maar nooit tevoren,:}{merkte ik wat ik thans met}{blij verbazen zag}\\

\haiku{hoe welgevormd zij,,...}{was hoe lieflijk haar gelaat}{hoe rank haar leden}\\

\haiku{Door de opening zag.}{ik de sterren schijnen over}{het bedauwde veld}\\

\haiku{Uranus Lang reeds was;}{hij bekend als een ster van}{bescheiden grootte}\\

\haiku{Of missen wij  ,,?}{het goddelijk span van Hoop}{Volharding Liefde}\\

\haiku{Gelaten kreunt de.}{oude aarde en heft zich}{uit haar slaap verjongd}\\

\haiku{Zo weegt de grote,:}{Waagschaal van de dood waarop}{wij alles zetten}\\

\haiku{nacht van diepe slaap,,.}{weerbarstig ondoordringbaar}{voor den vreemdeling}\\

\haiku{Het lukt misschien nog,{\textquoteright}.}{beter als je ook mijn kleed}{aantrekt zegt Yahyah}\\

\haiku{Wat zulk een oude?}{wijsgeer uit kan kuren als}{hij wordt ondervraagd}\\

\haiku{nagezeten door,.}{de schimmen van een beer een}{duivel en een draak}\\

\haiku{Wend u Oost-.}{en wend u Westwaarts langs uw}{eeuwenoude baan}\\

\haiku{Tot Herakles kwam,...}{met zijn stierkracht en allen}{verbaasde ook u}\\

\haiku{Tevergeefs welft de,.}{ruiter zijn borst zit  ik}{fier in het zadel}\\

\haiku{Steeds als ik hem weer,,:}{ontdek Dolfijn en redder}{van de kunst denk ik}\\

\haiku{Hoger, hoger dan.}{het altaar moet het beeld staan}{van den lieveling}\\

\haiku{En dat ik in ruil;}{daarvoor stil en alleen als}{een ster mag vergaan}\\

\haiku{Het gefluister der,;}{sferen werd duidelijker}{allengs verstaanbaar}\\

\haiku{{\textquoteright} Hen naderde ik, '.}{ondert afgrijselijk}{gesnurk van hun slaap}\\

\haiku{En dit alleen scheen.}{al voldoende om mij de}{arm te verlammen}\\

\haiku{Druppelend viel uit;}{de zak aan mijn zijde het}{bloed van Medusa}\\

\haiku{en iedere drup.}{in het zand der woestijn werd}{een giftig serpent}\\

\haiku{En langzaam nadert;}{Thuban weer het middelpunt van}{ons beperkt heelal}\\

\haiku{Het kind echter geeft;}{aan de nimf het geschenk met}{een glimlach terug}\\

\haiku{dat werd geboren,;}{om te branden en nimmer}{doven zal tot as}\\

\haiku{Het zonlicht slurpt de;}{nevelrafels en vlijt zich}{op de heuvelflank}\\

\haiku{Indien een man niet,,.}{komt Vindemiatrix moet}{een god u plukken}\\

\haiku{Vliegende Velox,...}{die vliedt en toch niet uit de}{kijker kunt lopen}\\

\haiku{Vliegende Velox,...}{die vliedt en toch niet uit de}{kijker kunt lopen}\\

\haiku{Vliegende Velox,...}{die vliedt en toch niet uit de}{kijker kunt lopen}\\

\haiku{Leer van de lenzen,;}{hoe gauw onze wensen de}{daad achterhalen}\\

\haiku{Vliegende Velox,...}{die vliedt en toch niet aan de}{wacht kunt ontlopen}\\

\haiku{sinds erger schande,.}{van herroepen wat hij w\'e\'et}{hiermee bespaard wordt}\\

\haiku{'t Groot gezang der:}{sferen overstemt het bidden}{der Inquisiteurs}\\

\haiku{Drie, vier eeuwen zijn.}{maar drie of vier seconden}{in de wereldtijd}\\

\haiku{Wanhoopt niet, het licht!}{zal eens de duisternis voor}{altoos overwinnen}\\

\haiku{Het ware weten,...}{springt te voorschijn in grote}{stilte binnenshuis}\\

\haiku{Het ware inzicht ';}{gaat slechts open voor wie int}{licht een blinde was}\\

\haiku{Ik echter kan thans,;}{niet meer falen mijn ster dwaalt}{nimmer van haar spoor}\\

\haiku{kringen vormend in.}{een spel dat even zinloos is}{als alle spelen}\\

\haiku{De boordeloze.}{ruimte heeft voor dit drijvend}{schip geen ankerplaats}\\

\haiku{En eerst - hoe lang is,}{dat geleden en hoe lang}{moest dat nog duren}\\

\haiku{met een plotseling,:}{weer hoger laaien om dan}{weer saam te trekken}\\

\haiku{gij zijt de oorzaak,;}{der geboorten verzoent ons}{met de reis des doods}\\

\haiku{het weten dat elk.}{aards beminnen een jacht ter}{zelfbevrijding is}\\

\haiku{ik ben nog zaad, nog.}{zucht van een gedachte die}{zich straks pas openbaart}\\

\haiku{hoezeer dit leven;}{in zijn broosheid dat van het}{heelal weerspiegelt}\\

\subsection{Uit: De laaiende stilte}

\haiku{Dan moet ik dubbel,.}{voorzichtig zijn want dat mag}{nooit grijpbaar worden}\\

\haiku{En Josephine...}{grijpt daarbij zijn hand vast en}{kijkt gelukzalig}\\

\haiku{Zij hebben gelijk.}{met te denken dat ook ik}{hier gelukkig ben}\\

\haiku{Ik koester mijn pijn.}{als een kind dat mijn borst bijt}{terwijl het zich voedt}\\

\haiku{Nachtenlang heeft mij.}{dit akelig gezicht tot in}{mijn slaap achtervolgd}\\

\haiku{Pas veel later ben,}{ik erover gaan nadenken}{hoe het mogelijk}\\

\haiku{want de dorre hand.}{van plicht en schijn drukt op mijn}{mond en smoort mijn stem}\\

\haiku{Het enige dat een,.}{kortstondig licht brengt in de}{brede wrede nacht}\\

\haiku{{\textquoteleft}Integendeel, zijn.}{strijdbare natuur dwingt hem}{tot gemeenzaamheid}\\

\haiku{er is geen plaats voor,?}{liefde meer en waar moet ik}{de mijne bergen}\\

\haiku{Morhang voor Raoul.}{kunnen beheren tijdens}{zijn afwezigheid}\\

\haiku{angst en onrust die.}{heel onze onzekere}{jeugd begeleid heeft}\\

\haiku{Maar zal Raoul ooit,?}{kunnen zullen wij vrouwen}{daartoe in staat zijn}\\

\haiku{Terwijl ik toch, in,.}{de grond van mijn hart elke}{verandering vrees}\\

\haiku{Toch is Amsterdam,.}{een drukke stad de drukste}{die ik ooit bezocht}\\

\haiku{Onrust drijft hem uit,.}{het woonvertrek het huis uit}{en de straten op}\\

\haiku{Toch is het beter,.}{geweest dit hier te horen}{dan nog op Morhang}\\

\haiku{Elk van ons is zo,,,.}{geslagen dat geen woord geen}{klacht geen troost meer klinkt}\\

\haiku{Met hem was het iets,.}{anders en daarom is hij}{van mij weggegaan}\\

\haiku{De storm die bijna,.}{een week lang geduurd heeft is}{eindelijk voorbij}\\

\haiku{Ik heb er altijd;}{naar verlangd en het gezocht}{met mijn verbeelding}\\

\haiku{Als ik dat nu ook;}{maar kon begrijpen v\'o\'or het}{einde van de reis}\\

\haiku{Er was niet enkel,.}{welkom maar ook afweer van}{het onbekende}\\

\haiku{Zoals ik toen al,.}{dadelijk verwachtte viel}{Willem Das hen bij}\\

\haiku{zijn gedragingen;}{tegenover de negers en}{de negerinnen}\\

\haiku{Tot mijn verbazing.}{vroeg hij toen zelfs niet waar mijn}{paard gebleven was}\\

\haiku{Wantrouwen, stomme,;}{woede en wrok ze loeren}{aan alle kanten}\\

\haiku{Een geldbedrag dat.}{hij bovendien bij lange}{na nog niet bezit}\\

\haiku{Het is zodoende.}{bijna tot een twist tussen}{de twee gekomen}\\

\haiku{Was dat echter waar,.}{hij moet mij dan al dikwijls}{hebben bijgestaan}\\

\haiku{Want er komt een dag.}{dat elke zoon ook op zijn}{beurt de vader wordt}\\

\haiku{Hij is veel vrijer,,.}{innerlijk dan wie ook hier}{op de plantage}\\

\haiku{Meer dan eens heb ik -}{mijn blikken voor de zijne}{neergeslagen ach}\\

\haiku{Het is wel ver met,}{mij gekomen dat ik dit}{zo onomwonden}\\

\haiku{Niemand behoeft meer,.}{te spreken niets behoef ik}{meer op te schrijven}\\

\haiku{Zolang hij er is,,.}{zullen zij niets verkeerds doen}{heeft hij mij beloofd}\\

\haiku{hun begeerte naar,.}{weelde onverzadigbaar}{door domme spilzucht}\\

\haiku{Hoe kon het anders?}{dan dat dit het laatste was}{wat hij deed en zei}\\

\haiku{Had ik toen iets kwaads,?}{gedaan wie zou er zich om}{bekommerd hebben}\\

\haiku{De enige die het,;}{zou hebben geweten zou}{Isidore zijn geweest}\\

\haiku{de wereld hier kent,.}{geen verleden zomin als}{zij een toekomst heeft}\\

\haiku{Nog altijd zie ik,;}{die blik van Isidore welke}{mij dit geleerd heeft}\\

\haiku{Nu is hij gevlucht,,.}{in de dood en keen telkens}{terug trouw te trouw}\\

\haiku{Zij willen immers,.}{breken nedersmakken en}{het diepste treffen}\\

\haiku{Het is alleen maar}{onbegrijpelijk dat er}{nog zovelen zijn}\\

\haiku{Ik vergat dat het.}{niet eender en niet even lang}{is voor een ieder}\\

\haiku{Met Raoul zijn vrees,....}{te delen zijn bezorgdheid}{voor het kind misschien}\\

\haiku{Ik mag blij zijn dat,.}{ik dienen kan daar waar die}{dienst nog wordt aanvaard}\\

\haiku{Niet langer verwringt.}{de succubus van hete}{begeerte hun vorm}\\

\haiku{Wel verliest het door.}{te steken en te snijden}{van zijn scherpte veel}\\

\subsection{Uit: Leef duizend levens (onder de naam Lou Lichtveld)}

\haiku{Hij is opgejaagd:}{als een rijwielkampioen}{en heeft slechts \'e\'en zorg}\\

\haiku{Natuurlijk alleen,.}{van het bizondere het}{belangwekkende}\\

\haiku{Een ander houdt van.}{haar en is aanstonds bereid}{voor haar te sterven}\\

\haiku{hij leeft niet geheel,.}{van eigen willekeur niet}{geheel autonoom}\\

\haiku{het liefdeleven,!}{de confrontatie met den}{erotischen partner}\\

\haiku{Gelijk de duivel.}{door zijn inblazingen en}{bedriegerijen}\\

\haiku{De meeste daden,;}{blijken heel anders te zijn}{dan wat ze lijken}\\

\haiku{{\textquoteleft}Allons, enfants de,{\textquoteright} {\textquoteleft}{\textquoteright}.}{la patrie of kan hetjij}{een veelvoud worden}\\

\haiku{De oude, ruwe;}{krijgsverhalen blijven voor}{het gewone volk}\\

\haiku{Knut de Snijder bleef,.}{lange tijd te Austegaard}{en werd een oud man}\\

\haiku{{\textquoteright} Moge een enkel.}{voorbeeld verduidelijken}{wat er bedoeld wordt}\\

\haiku{Het verhaal ontrolt, {\textquoteleft}.}{zich voor ons het is onder}{het lezenwordend}\\

\haiku{Want in het snelste.}{geval verhaalt men van het}{zojuist gebeurde}\\

\haiku{Niets is minder waar, {\textquoteleft},{\textquoteright}:}{zolang alle kunst slechts een}{herscheppen dat is}\\

\haiku{Wij worden attent,.}{gemaakt en de contr\^ole kan}{achteraf komen}\\

\haiku{zij verblinden niet,,,.}{zij winnen stap voor stap boek}{voor boek jaar na jaar}\\

\haiku{Beschouwing is een.}{confrontatie van het Ik}{met het Andere}\\

\haiku{Het kunstwerk wordt uit.}{een wisselwerking tussen}{beide geboren}\\

\haiku{Dit bewegende,,:}{voortstuwende zijn wij al}{tegengekomen}\\

\haiku{De fabel toont hem,.}{aan welk standpunt dat van den}{romanschrijver is}\\

\haiku{De oplossing der:}{opgeworpen problemen}{kan dan alleen zijn}\\

\haiku{hoeveel jaren haar,.}{nog vergund waren en het}{doet er ook niet toe}\\

\haiku{Het beantwoordt slechts - -.}{zoals iedereen weet aan}{een wensfantasie}\\

\haiku{Draagt het geen kiemen?}{in zich die het maar al te}{snel vernietigen}\\

\haiku{Hij was gekleed in;}{een flesgroene jas met een}{zwart-fluwelen kraag}\\

\haiku{en eindelijk in,.}{een afgelegen streek waar}{hij wederom faalt}\\

\haiku{{\textquoteleft}U beweert dus in?}{volle ernst een zelfstandig}{karakter te zijn}\\

\haiku{Het leesgezelschap{\textquoteright}.}{van Diepenbeek van P. van}{Limburg Brouwer voert}\\

\haiku{Sarcastisch zelfs geeft:}{hij midden in het boek van}{Mathilde toe}\\

\haiku{Zijn bedoeling blijkt,:}{duidelijk wanneer hij iets}{verder doorgaat}\\

\haiku{In een geval als {\textquoteleft}{\textquoteright}:}{Dostojewski'sSchuld en boete is}{voor hem de vraag niet}\\

\haiku{Van dezen laatste,:}{uit gezien moet het vraagstuk}{zo vertaald worden}\\

\haiku{Zulk een idee is op.}{zijn best een vaststelling en}{op zijn minst een vraag}\\

\haiku{Het ware, althans,.}{verifieerbare komt}{nu op de voorgrond}\\

\haiku{Bij ons maakte Van {\textquoteleft}.}{Lennep een aanvang door zijn}{Klaasje Zevenster}\\

\haiku{Stevenson in {\textquoteleft}Dr.{\textquoteright} {\textquoteleft}}{Jekyll and Mr. Hyde of}{Oscar Wilde in}\\

\haiku{The picture of{\textquoteright}.}{Dorian Gray op een veel}{verfijnder manier}\\

\haiku{Een klassiek voorbeeld {\textquoteleft}{\textquoteright}.}{daarvan hebben wij inUncle}{Tom's cabin van mrs}\\

\haiku{{\textquoteleft}Nederigheid is.}{niet de overheersende deugd}{der romanschrijvers}\\

\haiku{Zij zijn niet bevreesd.}{aanspraak te maken op de}{titel van scheppers}\\

\haiku{In dit opzicht is;}{het aesthetische echter}{onproblematisch}\\

\haiku{) Ongelukkigen,.}{of vervolgden te lijf gaan}{De zeden kwetsen}\\

\haiku{Ons 20ste-eeuws standpunt.}{is heel wat eenvoudiger}{en radicaler}\\

\haiku{Sommige tergen,,;}{achtervolgen ons laten}{de geest niet met rust}\\

\haiku{En hun oeuvre werkt, {\textquoteleft}.}{nog steeds na zo goed als dat}{van deklassieken}\\

\haiku{Degene die de {\textquoteleft}{\textquoteright} {\textquoteleft}{\textquoteright},.}{vraag en dus deprijs bepaalt}{dat is de lezer}\\

\haiku{Literatuur ter:}{verdere ori\"entatie}{Algemeen}\\

\haiku{Balzac 13, 53, 64,,,,,,,,,,,.}{65 128 149 v 267 294 296}{297 303 308 310 315}\\

\haiku{Boccaccio 79 v,,,,,,,,,,.}{81 v 158 v 161 255 269}{295 323 333 334 341}\\

\haiku{Mann (Thomas) 56, 59,,,,,,,,.}{123 191 208 222 227 262 297}{310 Manzoni 55}\\

\subsection{Uit: De medeminnaars}

\haiku{Een ogenblik stond hij.}{aarzelend te draaien op}{zijn lange benen}\\

\haiku{Toen, reeds achter de,:}{stoel van zijn moeder even naar}{haar overgebogen}\\

\haiku{hij paste niet meer.}{in de schoolbanken met zijn}{groot en hoekig lijf}\\

\haiku{Misschien zou ze hem,,.}{als hij haar weerzag niet eens}{meer willen kennen}\\

\haiku{Zie ik er uit als...{\textquoteright} {\textquoteleft},}{een van die vrouwen dieZo}{bedoel ik het niet}\\

\haiku{{\textquoteright} {\textquoteleft}Ik ben misschien niet,{\textquoteright}.}{zo ervaren bekende}{Joachim nederig}\\

\haiku{{\textquoteleft}Ken je dat grote?}{caf\'e op de hoek van de}{Markt en de Bredestraat}\\

\haiku{Ze had de afspraak.}{toch stellig niet gemaakt om}{van hem af te zijn}\\

\haiku{Een schok gaf hem de.}{ontdekking dat ze nu toch}{gekomen was}\\

\haiku{{\textquoteright} Carla zei niets meer,.}{hij rekende af en zij}{gingen de straat op}\\

\haiku{{\textquoteright} Met wijsneuzige:}{nadrukkelijkheid begon}{Joachim te oreren}\\

\haiku{Maar beloof me dan,....}{dat je mij een berichtje}{stuurt met een afspraak}\\

\haiku{In zijn handpalmen.}{voelde hij nog de warmte}{van haar polsen na}\\

\haiku{Zoals nu, hier voor,.}{zijn vader die hoorbaar aan}{zijn sigaar pufte}\\

\haiku{Maar hij had bergen,.}{werk te verzetten want zijn}{achterstand was groot}\\

\haiku{Zijn moeder kon hem,}{weer horen fluiten wanneer}{hij door het huis liep}\\

\haiku{Hoewel de leraar,,.}{er oudergewoonte geen}{acht op scheen te slaan}\\

\haiku{Daarna verdween ze,.}{achter de bomen waar hij}{haar niet meer kon zien}\\

\haiku{Maar hij was bezig,.}{zijn doel te bereiken weer}{een stap dichterbij}\\

\haiku{Ze spande zich in,.}{een ladder uit een zijden}{kous op te halen}\\

\haiku{maar wat wist het park?}{zelf van deze vluchtige}{wandelaarster af}\\

\haiku{een uitgetrokken.}{kledingstuk dat zij onder}{zijn ogen hanteerde}\\

\haiku{Het is dwaasheid niet.}{roekeloos te nemen wat}{het leven ons geeft}\\

\haiku{met een vinger op.}{de eigen mond beduidde}{ze hem te zwijgen}\\

\haiku{Een zacht fluitje klonk,.}{als een signaal waarmee men}{een bekende roept}\\

\haiku{Onmogelijk, - bij;}{zulke vrouwen ging alles}{immers anders toe}\\

\haiku{Duidelijk kon hij:}{de driftige woorden van}{zijn vader verstaan}\\

\haiku{Stel je eens voor, dat,:}{het Carla geweest was die}{hem had opgebeld}\\

\haiku{Als het tenminste.}{een vrouw was geweest en geen}{ondergeschikte}\\

\haiku{Eventueel,{\textquoteright} hernam, {\textquoteleft}.}{hijzou je bij mij in de}{zaak kunnen komen}\\

\haiku{Maar iets, iemand, een.}{gevaar verhinderde haar}{misschien te roepen}\\

\haiku{De duisternis viel,.}{reeds in toen Joachim bij het}{huis kwam aangefietst}\\

\haiku{{\textquoteright} {\textquoteleft}Het is geen zaak, het,{\textquoteright}.}{is iets doodonschuldigs zei}{Joachim met nadruk}\\

\haiku{Je wist toch heel goed.}{dat dit geen boodschap aan een}{kleine jongen was}\\

\haiku{{\textquoteright} {\textquoteleft}En jij bent zeker?}{in het geheel niet vals en}{niet bedriegelijk}\\

\haiku{de zorg om jou, om.}{wat je misschien belet heeft}{je woord te houden}\\

\haiku{{\textquoteleft}Laten wij elkaar,.}{niets vragen ieder zwijgen}{over zijn geheimen}\\

\haiku{{\textquoteright} Joachim trok haar hoofd.}{naar zich toe en legde zijn}{arm om haar schouder}\\

\haiku{{\textquoteleft}Ik laat me in met;}{een kinderachtigheid die}{me niet helpen kan}\\

\haiku{Zwijgend en met een.}{mismoedig gebaar gaf hij}{haar de brief terug}\\

\haiku{Het voertuig kwam gauw,.}{genoeg hij stapte in en}{noemde Carla's adres}\\

\haiku{{\textquoteleft}Het is goed dat het,{\textquoteright}.}{binnenkort afgelopen}{is zei de moeder}\\

\haiku{Wat moet ik dan doen?}{om mij nu voorgoed van haar}{te verzekeren}\\

\haiku{Deed ze tegenover?}{anderen misschien net zo}{als tegenover hem}\\

\haiku{een tijd van lange.}{zoete vrede zal misschien}{voor ons beginnen}\\

\haiku{Ze wisten, hoewel,.}{huisgenoten zo weinig}{van elkander af}\\

\haiku{Hij moest haar tonen,,.}{hoezeer hij dit samenzijn}{ook hier waardeerde}\\

\haiku{Vooral de oude,{\textquoteright}.}{heer mag hiervan niets weten}{vermaande Willem}\\

\haiku{Als een dorstige;}{temidden van de oceaan}{gevoelde hij zich}\\

\haiku{Velen woonden hier,.}{zoals de wereld ook door}{velen werd bewoond}\\

\haiku{{\textquoteright} en aanstonds daarop,:}{met  een gebaar naar zijn}{boeket vervolgde}\\

\haiku{Kroner's vakken in,.}{het bizonder als die hem}{zou willen treffen}\\

\haiku{Maar was het niet voor,?}{haar eigen bestwil om haar}{te kunnen helpen}\\

\haiku{In plaats van zich te.}{ontwarren werd haar raadsel}{ingewikkelder}\\

\haiku{Er viel voor Joachim.}{niets beters te doen dan hun}{voorbeeld te volgen}\\

\haiku{Die had dus Carla,...}{weggebracht en met een soort}{vanzelfsprekendheid}\\

\haiku{Maar zijn voorgevoel,.}{bevestigde hem dat ze}{ditmaal er zou zijn}\\

\haiku{{\textquoteleft}Maar je weet toch, het...}{ontbrak mij in de laatste}{tijd aan een bedrag}\\

\haiku{{\textquoteright} {\textquoteleft}Een vrouw als jij, met...?}{zoveel vrienden Zijn er dan}{geen rijke onder}\\

\haiku{{\textquoteright} Mainteneren wou,}{hij zeggen maar dat slikte}{hij natuurlijk in}\\

\haiku{{\textquoteleft}Niet bizonder,{\textquoteright} was.}{zijn antwoord op de vraag hoe}{hij het gemaakt had}\\

\haiku{Leo Dekking die het,.}{examen nog voor de boeg had}{kwam al op hem af}\\

\haiku{Geef me dan maar de.}{genadeslag en laat het}{afgelopen zijn}\\

\haiku{{\textquoteleft}Ik zie dat je al,,{\textquoteright}.}{begrijpt waar het op uitdraait}{ging de rector voort}\\

\haiku{Dit, wat erger was,.}{dan ronduit zakken weten}{waar je aan toe was}\\

\haiku{Van twee kwaden is,{\textquoteright}.}{dit nog het minste zei zijn}{vader laconiek}\\

\haiku{Een dergelijke.}{bezorgdheid was hij van zijn}{vader niet gewend}\\

\haiku{Twee maanden zijn een,,}{hele boel tijd als je die}{goed gebruikt Warden}\\

\haiku{{\textquoteleft}En je moet het doen,.}{omdat de eer van de school}{ermee gemoeid is}\\

\haiku{En deze had hem.}{bijna gesoebat om het}{herexamen te doen}\\

\haiku{Hij had de grootste.}{moeite om zijn opwinding}{te onderdrukken}\\

\haiku{Er zit niets anders,.}{op dan dat ik inderdaad}{het herexamen doe}\\

\haiku{Hij had geen recht van,.}{spreken meer wanneer hij er}{geen gevolg aan gaf}\\

\haiku{In ieder geval,}{zit je hier beter dan in}{zo'n kazerneflat}\\

\haiku{Deze ontmoeting.}{moet nu maar over ons beider}{leven beslissen}\\

\haiku{Hoe kon hij er \'e\'en?}{moment over gedacht hebben}{haar los te laten}\\

\haiku{De man die naar de.}{hond floot op het trottoir leek}{op meneer Rocquet}\\

\haiku{Practisch werken en,{\textquoteright}.}{alvast wat verdienen had}{Joachim geantwoord}\\

\haiku{In zijn dooie eentje.}{zou hij er zelfs van kunnen}{leven als het moest}\\

\haiku{Hoewel hij tevens,.}{blij was dat althans Carla}{niet ter sprake kwam}\\

\haiku{Hij liet zijn fiets maar,.}{staan en riep een taxi als een}{echte zakenman}\\

\haiku{Wie weet hoeveel er?}{waren en hoe weinigen}{slechts werden betrapt}\\

\haiku{Carla streek Joachim,.}{over zijn haren met een haast}{moederlijk gebaar}\\

\haiku{Hoe staat het met je?}{nieuwe functie op kantoor}{bij de oude heer}\\

\haiku{Er kwamen immers.}{genoeg aantrekkelijke}{langs of in hun buurt}\\

\haiku{Maar dat moet ik nog,{\textquoteright}.}{tegenkomen vertelde}{Willem goedgemutst}\\

\haiku{Dat heeft hij zeker;}{aan de een of andere}{zakenvriend geleend}\\

\haiku{Leo moest eens weten,,,... {\textquoteleft}}{hoe hij Joachim aan deze}{dingen was ontgroeid}\\

\haiku{Hoe had hij ook maar!}{\'e\'en ogenblik kunnen denken}{dat het Carla was}\\

\haiku{Nu zonk het terug.}{tot teleurstelling die op}{zwaarmoedigheid leek}\\

\haiku{het oude ideaal.}{dat zo heftig door al zijn}{zinnen verlangd werd}\\

\haiku{{\textquoteright} En hij loog erbij,,:}{hij wist zelf niet waarvoor het}{kwam automatisch}\\

\haiku{Maar nu is het mijn,.}{allerlaatste souvenir}{geworden weet je}\\

\haiku{de minnaars bleven.}{toch achter en zouden haar}{gauw doen vergeten}\\

\haiku{{\textquoteleft}Een daarvan is een,.}{van mijn leraren een van}{mijn examinators}\\

\haiku{Hij was er zelfs bij.}{toen ze mij zo ongeveer}{de deur uit zette}\\

\haiku{En als hij weet dat,.}{je erop verdacht bent zal}{hij het wel laten}\\

\haiku{Moeilijker dan voor?}{u om u aan het examen}{te onderwerpen}\\

\haiku{Niet dat de tijd die,.}{hij aldus won hem nog veel}{zou kunnen baten}\\

\haiku{Hij ging zitten op.}{een van de stoeltjes die nog}{steeds op hen wachtten}\\

\haiku{Zich ontworsteld had.}{aan Kroner's greep en aan de}{greep van iedereen}\\

\haiku{Of hij schoot haar neer,.}{en daarna ook zichzelf een}{kogel door de kop}\\

\haiku{En eerst misschien nog.}{wie zich tussen haar en hem}{zou willen dringen}\\

\haiku{Want dit was Carla,.}{maar toch niet de vrouw voor wie}{hij was gekomen}\\

\haiku{Zijn trage, vlakke,:}{stem zei toen de jongeman}{hem vragend aankeek}\\

\haiku{had Joachim zijn stem,?}{hier niet al eens gehoord door}{alle wanden heen}\\

\haiku{{\textquoteleft}Schiet maar op,{\textquoteright} zei de,.}{inspecteur nog zonder zelfs}{een zweem van boosheid}\\

\subsection{Uit: De medeminnaars}

\haiku{Toen, reeds achter de,:}{stoel van zijn moeder even naar}{haar overgebogen}\\

\haiku{hij paste niet meer.}{in de schoolbanken met zijn}{groot en hoekig lijf}\\

\haiku{Misschien zou ze hem,,.}{als hij haar weerzag niet eens}{meer willen kennen}\\

\haiku{Zie ik er uit als...{\textquoteright} {\textquoteleft},}{een van die vrouwen dieZo}{bedoel ik het niet}\\

\haiku{{\textquoteright} {\textquoteleft}Ik ben misschien niet,{\textquoteright}.}{zo ervaren bekende}{Joachim nederig}\\

\haiku{{\textquoteleft}Ken je dat grote?}{caf\'e op de hoek van de}{Markt en de Bredestraat}\\

\haiku{Ze had de afspraak.}{toch stellig niet gemaakt om}{van hem af te zijn}\\

\haiku{{\textquoteright} Carla zei niets meer,.}{hij rekende af en zij}{gingen de straat op}\\

\haiku{{\textquoteright} Met wijsneuzige:}{nadrukkelijkheid begon}{Joachim te oreren}\\

\haiku{Maar beloof me dan,....}{dat je mij een berichtje}{stuurt met een afspraak}\\

\haiku{In zijn handpalmen.}{voelde hij nog de warmte}{van haar polsen na}\\

\haiku{Zoals nu, hier voor,.}{zijn vader die hoorbaar aan}{zijn sigaar pufte}\\

\haiku{Maar hij had bergen,.}{werk te verzetten want zijn}{achterstand was groot}\\

\haiku{Zijn moeder kon hem,}{weer horen fluiten wanneer}{hij door het huis liep}\\

\haiku{Hoewel de leraar,,.}{er oudergewoonte geen}{acht op scheen te slaan}\\

\haiku{Daarna verdween ze,.}{achter de bomen waar hij}{haar niet meer kon zien}\\

\haiku{Maar hij was bezig,.}{zijn doel te bereiken weer}{een stap dichterbij}\\

\haiku{Ze spande zich in,.}{een ladder uit een zijden}{kous op te halen}\\

\haiku{maar wat wist het park?}{zelf van deze vluchtige}{wandelaarster af}\\

\haiku{een uitgetrokken.}{kledingstuk dat zij onder}{zijn ogen hanteerde}\\

\haiku{Het is dwaasheid niet.}{roekeloos te nemen wat}{het leven ons geeft}\\

\haiku{met een vinger op.}{de eigen mond beduidde}{ze hem te zwijgen}\\

\haiku{Een zacht fluitje klonk,.}{als een signaal waarmee men}{een bekende roept}\\

\haiku{Onmogelijk, - bij;}{zulke vrouwen ging alles}{immers anders toe}\\

\haiku{Duidelijk kon hij:}{de driftige woorden van}{zijn vader verstaan}\\

\haiku{Stel je eens voor, dat,:}{het Carla geweest was die}{hem had opgebeld}\\

\haiku{Als het tenminste.}{een vrouw was geweest en geen}{ondergeschikte}\\

\haiku{Eventueel,{\textquoteright} hernam, {\textquoteleft}.}{hijzou je bij mij in de}{zaak kunnen komen}\\

\haiku{Maar iets, iemand, een.}{gevaar verhinderde haar}{misschien te roepen}\\

\haiku{{\textquoteright} {\textquoteleft}Het is geen zaak, het,{\textquoteright}.}{is iets doodonschuldigs zei}{Joachim met nadruk}\\

\haiku{je wist toch heel goed?}{wat je deed toen je mij schreef}{dat ik kon komen}\\

\haiku{Je wist toch heel goed.}{dat dit geen boodschap aan een}{kleine jongen was}\\

\haiku{de zorg om jou, om.}{wat je misschien belet heeft}{je woord te houden}\\

\haiku{{\textquoteleft}Laten we elkaar,.}{niets vragen ieder zwijgen}{over zijn geheimen}\\

\haiku{{\textquoteright} Joachim trok haar hoofd.}{naar zich toe en legde zijn}{arm om haar schouder}\\

\haiku{{\textquoteleft}Ik laat me in met;}{een kinderachtigheid die}{me niet helpen kan}\\

\haiku{Zwijgend en met een.}{mismoedig gebaar gaf hij}{haar de brief terug}\\

\haiku{Het voertuig kwam gauw,.}{genoeg hij stapte in en}{noemde Carla's adres}\\

\haiku{Hij behoefde de,.}{brief  slechts open te maken}{en deed het dan ook}\\

\haiku{{\textquoteleft}Het is goed dat het,{\textquoteright}.}{binnenkort afgelopen}{is zei de moeder}\\

\haiku{Wat moet ik dan doen?}{om mij nu voorgoed van haar}{te verzekeren}\\

\haiku{Deed ze tegenover?}{anderen misschien net zo}{als tegenover hem}\\

\haiku{een tijd van lange.}{zoete vrede zal misschien}{voor ons beginnen}\\

\haiku{5 Met een spinsel;}{van leugens had hij thuis zijn}{wegblijven verklaard}\\

\haiku{Ze wisten, hoewel,.}{huisgenoten zo weinig}{van elkander af}\\

\haiku{Hij moest haar tonen,,.}{hoezeer hij dit samenzijn}{ook hier waardeerde}\\

\haiku{Vooral de oude,{\textquoteright}.}{heer mag hiervan niets weten}{vermaande Willem}\\

\haiku{Als een dorstige;}{temidden van de oceaan}{gevoelde hij zich}\\

\haiku{Velen woonden hier,.}{zoals de wereld ook door}{velen werd bewoond}\\

\haiku{Kroner's vakken in,.}{het bijzonder als die hem}{zou willen treffen}\\

\haiku{Maar was het niet voor,?}{haar eigen bestwil om haar}{te kunnen helpen}\\

\haiku{In plaats van zich te.}{ontwarren werd haar raadsel}{ingewikkelder}\\

\haiku{Er viel voor Joachim.}{niets beters te doen dan hun}{voorbeeld te volgen}\\

\haiku{Die had dus Carla,...}{weggebracht en met een soort}{vanzelfsprekendheid}\\

\haiku{Maar zijn voorgevoel,.}{bevestigde hem dat ze}{ditmaal er zou zijn}\\

\haiku{{\textquoteleft}Maar je weet toch, het...}{ontbrak mij in de laatste}{tijd aan een bedrag}\\

\haiku{{\textquoteright} {\textquoteleft}Een vrouw als jij, met...?}{zoveel vrienden Zijn er dan}{geen rijke onder}\\

\haiku{{\textquoteright} Mainteneren wou,}{hij zeggen maar dat slikte}{hij natuurlijk in}\\

\haiku{Niet bijzonder,{\textquoteright} was.}{zijn antwoord op de vraag hoe}{hij het gemaakt had}\\

\haiku{Leo Dekking die het,.}{examen nog voor de boeg had}{kwam al op hem af}\\

\haiku{Geef me dan maar de.}{genadeslag en laat het}{afgelopen zijn}\\

\haiku{{\textquoteleft}Ik zie dat je al,,{\textquoteright}.}{begrijpt waar het op uitdraait}{ging de rector voort}\\

\haiku{Dit, wat erger was,.}{dan ronduit zakken weten}{waar je aan toe was}\\

\haiku{Van twee kwaden is,{\textquoteright}.}{dit nog het minste zei zijn}{vader laconiek}\\

\haiku{Een dergelijke.}{bezorgdheid was hij van zijn}{vader niet gewend}\\

\haiku{{\textquoteleft}En je moet het doen,.}{omdat de eer van de school}{ermee gemoeid is}\\

\haiku{En deze had hem.}{bijna gesoebat om het}{herexamen te doen}\\

\haiku{Hij had de grootste.}{moeite om zijn opwinding}{te onderdrukken}\\

\haiku{Er zit niets anders,.}{op dan dat ik inderdaad}{het herexamen doe}\\

\haiku{Hij had geen recht van,.}{spreken meer wanneer hij er}{geen gevolg aan gaf}\\

\haiku{In ieder geval,}{zit je hier beter dan in}{zo'n kazerneflat}\\

\haiku{Deze ontmoeting.}{moet nu maar over ons beider}{leven beslissen}\\

\haiku{Hoe kon hij er \'e\'en?}{moment over gedacht hebben}{haar los te laten}\\

\haiku{De man die naar de.}{hond floot op het trottoir leek}{op meneer Rocquet}\\

\haiku{Praktisch werken en,{\textquoteright}.}{alvast wat verdienen had}{Joachim geantwoord}\\

\haiku{In zijn dooie eentje.}{zou hij er zelfs van kunnen}{leven als het moest}\\

\haiku{Hoewel hij tevens,.}{blij was dat althans Carla}{niet ter sprake kwam}\\

\haiku{{\textquoteright} Hij wachtte even, keek,:}{of Joachim hem wel begreep}{en  ging toen voort}\\

\haiku{Hij liet zijn fiets maar,.}{staan en riep een taxi als een}{echte zakenman}\\

\haiku{Ik beloof je, dat,.}{je het over uiterlijk een}{dag of drie vier hebt}\\

\haiku{Wie weet hoeveel er?}{waren en hoe weinigen}{slechts werden betrapt}\\

\haiku{Hoe staat het met je?}{nieuwe functie op kantoor}{bij de oude heer}\\

\haiku{Er kwamen immers.}{genoeg aantrekkelijke}{langs of in hun buurt}\\

\haiku{Maar dat moet ik nog,{\textquoteright}.}{tegenkomen vertelde}{Willem goedgemutst}\\

\haiku{Dat heeft hij zeker;}{aan de een of andere}{zakenvriend geleend}\\

\haiku{Leo moest eens weten,,,... {\textquoteleft}}{hoe hij Joachim aan deze}{dingen was ontgroeid}\\

\haiku{Hoe had hij ook maar!}{\'e\'en ogenblik kunnen denken}{dat het Carla was}\\

\haiku{Nu zonk het terug.}{tot teleurstelling die op}{zwaarmoedigheid leek}\\

\haiku{het oude ideaal.}{dat zo heftig door al zijn}{zinnen verlangd werd}\\

\haiku{{\textquoteright} En hij loog erbij,,:}{hij wist zelf niet waarvoor het}{kwam automatisch}\\

\haiku{Maar nu is het mijn,.}{allerlaatste souvenir}{geworden weet je}\\

\haiku{de minnaars bleven.}{toch achter en zouden haar}{gauw doen vergeten}\\

\haiku{{\textquoteleft}Een daarvan is een,.}{van mijn leraren een van}{mijn examinators}\\

\haiku{Hij was er zelfs bij.}{toen ze mij zo ongeveer}{de deur uit zette}\\

\haiku{En als hij weet dat,.}{je erop verdacht bent zal}{hij het wel laten}\\

\haiku{Moeilijker dan voor?}{u om u aan het examen}{te onderwerpen}\\

\haiku{Niet dat de tijd die,.}{hij aldus won hem nog veel}{zou kunnen baten}\\

\haiku{Je ziet eruit als,.}{een geest en het kind is een}{schatje om te zien}\\

\haiku{Zich ontworsteld had.}{aan Kroners greep en aan de}{greep van iedereen}\\

\haiku{Of hij schoot haar neer,.}{en daarna ook zichzelf een}{kogel door de kop}\\

\haiku{En eerst misschien nog.}{wie zich tussen haar en hem}{zou willen dringen}\\

\haiku{Want dit was Carla,.}{maar toch niet de vrouw voor wie}{hij was gekomen}\\

\haiku{Zijn trage, vlakke,:}{stem zei toen de jongeman}{hem vragend aankeek}\\

\haiku{{\textquoteleft}Schiet maar op,{\textquoteright} zei de,.}{inspecteur nog zonder zelfs}{een zweem van boosheid}\\

\subsection{Uit: Mijn aap schreit. Het euvel gods}

\haiku{Twee dingen waren,:}{het die mij aangetrokken}{hadden in de aap}\\

\haiku{De aap sprong rond in.}{mijn huis alsof hij er al}{jaren geweest was}\\

\haiku{Tevens was ik blij.}{dat het probleem zichzelve}{uitgewezen had}\\

\haiku{Mijn aap heeft een paar;}{eikeltjes gestolen die}{langs de weg lagen}\\

\haiku{was hij al aanstonds,.}{goede maatjes met haar toen het}{meisje binnen kwam}\\

\haiku{Ik zette het raam,.}{open want de kamer werd mij}{te eng en benauwd}\\

\haiku{Ik schold op hem, maar,.}{sloeg hem niet meer want hij was}{nog zwak en mager}\\

\haiku{Je moet niet alleen,.}{de verschillen leren maar}{ook de overeenkomst}\\

\haiku{In geen geval moest;}{ze zich echter met het beest}{kunnen bemoeien}\\

\haiku{{\textquoteright} {\textquoteleft}O, het is zo erg,{\textquoteright},.}{niet zei ik en bond mijn pols}{af met m'n zakdoek}\\

\haiku{Direct toen het licht,,.}{aanging zag ik hem zitten}{in zijn mand mijn aap}\\

\haiku{Stijf en koud ben ik.}{naar boven gegaan toen het}{haast ochtend was}\\

\haiku{de nieuwsgierigheid:}{die we allen hebben voor}{het einde van iets}\\

\haiku{Reeds enige dagen.}{was hij gehuwd en woonde}{hij in het paleis}\\

\haiku{Nauwelijks was hij,}{bij de vijver gekomen}{of hij wierp zo snel}\\

\haiku{Dan hief hij be{\^\i} zijn:}{armen op en begon een}{lied te zingen}\\

\haiku{{\textquoteright} De jager zette.}{zijn hoed recht en kruiste de}{armen over z'n borst}\\

\haiku{Tot zo laat was er;}{tussen al Gods dagen geen}{verschil voor Jukkers}\\

\haiku{Het eten was van avond,...}{slecht maar uwe liefkozingen}{zijn al mijn voedsel}\\

\haiku{anders was zoveel.}{leugen en bedrog immers}{niet nodig geweest}\\

\haiku{Spranger belde mij....}{op voor een spoedoperatie}{voor Arthur Christensen}\\

\haiku{De ander trachtte,.}{hem te kalmeren maar hij}{wond zich steeds meer op}\\

\haiku{Als mensen uit een.}{andere wereld staarden}{wij elkander aan}\\

\haiku{In ieder geval,{\textquoteright}.}{was het beestje omnivoor}{stelde Fowler vast}\\

\haiku{Het brak tussen mijn,.}{tanden en ik bekeek het}{stuk dat ik vasthield}\\

\haiku{Ik zag hoe Fowler,.}{kokhalsde en was niet in}{staat te antwoorden}\\

\haiku{Vermoeid was zij nu,.}{na drie jaren geworden}{ten dode vermoeid}\\

\haiku{Alles was van een.}{uiterste zindelijkheid}{en scheen te stralen}\\

\haiku{Vragen durfde hij, ',:}{niet want wat hijt eerst zou}{willen weten was}\\

\haiku{Ik knikte, nam een:}{slok van het onsmakelijk}{mengsel  en zei}\\

\haiku{Maar de oude vrouw.}{trok me aan de mouw mee naar}{een andere reet}\\

\haiku{{\textquoteleft}Wij moeten nu de,.}{andere kant uit want het}{laatste huis is leeg}\\

\haiku{Er zijn misdaden,.}{zo verschrikkelijk dat ze}{geen naam meer hebben}\\

\haiku{Bijna is het dag,{\textquoteright},.}{zegt hij naar buiten stappend}{waar de paarden staan}\\

\haiku{De noordelijkste!}{kaart van pater Martinus}{was onnauwkeurig}\\

\haiku{Het jongetje en.}{het meisje aan haar zijde}{zien nieuwsgierig rond}\\

\haiku{{\textquoteright} Maar slap en willoos.}{liet ze zich doorgolven en}{Cobbs verloor de moed}\\

\haiku{Waarom heb ik ook,.}{geen moeder fluisterde het}{binnenste van Cobbs}\\

\haiku{Want hij geloofde,.}{niet meer omdat er niets meer}{te geloven viel}\\

\subsection{Uit: Mijn aap lacht}

\haiku{Iedereen erkent.}{van sommige gevoelens}{de verhevenheid}\\

\haiku{Je leert ook al de.}{geluiden in de verte}{duidelijk verstaan}\\

\haiku{Bijen, wespen en.}{veel andere insekten}{zoemden er omheen}\\

\haiku{Onderwijl kneep ons,.}{de honger aangewakkerd}{door de lange tocht}\\

\haiku{V Diepdonkere.}{nacht stond geurig om mij heen}{toen ik weer bijkwam}\\

\haiku{lang zaten wij in}{het struikgewas waaronder}{onophoudelijk}\\

\haiku{Alleen dat \'ene was,:}{er nog dat \'ene dat ik moest}{en zou bereiken}\\

\haiku{De hele wereld.}{droop en uit de kruin lekten}{nog lange beken}\\

\haiku{Uit de moeder steeg,.}{een rauwe kreet op maar ze}{bood geen tegenstand}\\

\haiku{maar dan zijn ze snel,.}{en haastig gaan en komen}{botsend langs elkaar}\\

\haiku{, en waar ik tastte,.}{voelde ik koud gesteente}{gepolijste rots}\\

\haiku{Het was een hele,}{opgaaf waaraan ik mij toch}{niet kon onttrekken}\\

\haiku{Want als het nieuwtje,.}{eraf is komt de hele}{last toch op mij neer}\\

\haiku{{\textquoteright} Zijn aard was toch wat.}{minder ruw dan zijn woorden}{gewoonlijk klonken}\\

\haiku{{\textquoteright} Toen Mauriesje het,.}{zag werd hij boos en duwde}{de jongen opzij}\\

\haiku{{\textquoteleft}Je houdt je koest, hoor, '.}{anders draai ik je je nek}{om voor jet weet}\\

\haiku{Allemaal zijn wij,,.}{verschillend dat is waar maar}{willen  vriendschap}\\

\haiku{Want ze was niet bleek,;}{maar juist erg zwart van haar en}{van gezicht en ogen}\\

\haiku{Veeg je poten af,.}{en salueer tenminste}{voordat je vertrekt}\\

\haiku{{\textquoteleft}Je zult hem niet veel,.}{pleziertjes meer kunnen doen}{mijn beste mensen}\\

\haiku{{\textquoteright} Louise was, de vuist,.}{tegen haar mond gedrukt de}{kamer uit gevlucht}\\

\haiku{Zo kwamen zij al.}{smoezend tot onder aan de}{dakrand waar ik zat}\\

\haiku{En alsof er niets,.}{gebeurd was spraken zij van}{andere dingen}\\

\haiku{Toen zij haar uitliet,,.}{lette mevrouw Hudson op}{dat ik niet meeliep}\\

\haiku{iemand die mijn ziel,.}{begreep en ook al was hij}{ver dicht bij me stond}\\

\haiku{de tijden zijn er,.}{toch niet op vooruitgegaan}{de blanken ook niet}\\

\haiku{Al was het ook een,.}{krokodil geweest ze had}{hem aangevallen}\\

\haiku{Ik heb mijn adem in.}{uw borst geblazen en u}{met mijn geest verwekt}\\

\haiku{terwijl ze even zweeg,,.}{zoals ze vaker deed ook}{als ze alleen was}\\

\haiku{Zo'n schepsel denkt dat.}{er wel wat te halen valt}{of weet-ik-wat}\\

\haiku{Dacht je soms dat je,?}{de enige meid was die een}{kind  moest krijgen}\\

\haiku{Ze scheen hem iets toe,:}{te fluisteren want Winters}{antwoordde opeens}\\

\haiku{En nu gebeurde.}{iets waarop ik allerminst}{verdacht geweest was}\\

\haiku{{\textquoteright} Hetgeen mij niet veel,.}{hoop gaf haar bedrieglijke}{natuur kennende}\\

\haiku{{\textquoteleft}Ik heb in huis te,.}{veel oud porselein staan om}{het al te wagen}\\

\haiku{Het was vroeg in de,.}{morgen een troepje jongens}{ging op weg naar school}\\

\haiku{Terwijl ze met haar,:}{ene vrije hand mijn kooi opendeed}{riep ze bijna hees}\\

\haiku{Daar zat stellig heel,.}{wat achter dat ik moest zien}{uit te  vorsen}\\

\haiku{Maar toch, kijk eens naar!}{het schuldbesef van deze}{jeugdige zondaar}\\

\haiku{u is de priester,{\textquoteright}.}{antwoordde juffrouw Elfriede}{maar half tevreden}\\

\haiku{{\textquoteright} {\textquoteleft}Als een beest... als een,{\textquoteright}.}{beest kreunde Rosalina}{zonder op te staan}\\

\haiku{Planten leken ze,,.}{mij toe op weg om dieren}{te worden meer niet}\\

\haiku{Als je nog ziek mocht,.}{blijven kom ik je stellig}{weer eens opzoeken}\\

\haiku{{\textquoteright} Waarop ze weer een:}{keel opzette en opnieuw}{begon te roepen}\\

\haiku{Ze gaf een schamper,.}{lachje terwijl Preiselbeer}{op haar toe stapte}\\

\haiku{Ik naar binnen, door.}{de opening boven een van}{de lage deurtjes}\\

\haiku{Ik had moeite mij.}{vast te houden en voelde}{een flauwte nabij}\\

\haiku{weg van de mijnen;}{onder wie ik vreedzaam en}{gelukkig leefde}\\

\haiku{En zo vergeten;}{zij rechtuit te denken en}{rechtuit te voelen}\\

\haiku{In de verte klonk,,,.}{twee- driemaal ook gefluit}{maar anders korter}\\

\haiku{Nu wist ik pas wat,.}{leedvermaak wat plezier om}{andermans pijn was}\\

\haiku{En het was alsof.}{de bomen zelf zich rekten}{met een lui gebaar}\\

\haiku{Nooit had ik het bos,.}{zo mooi gezien als nu bij}{deze wederkeer}\\

\haiku{II Als een grote.}{blanke spiegel lag het meer}{tussen de bomen}\\

\haiku{Kwamen andere?}{dieren net als ik ooit tot}{eenzelfde besef}\\

\haiku{Een jong volk op zoek,.}{naar taken kon nergens meer}{voor terugschrikken}\\

\haiku{Iets moest ik doen, iets.}{geweldigs en groots om de}{troep te bezweren}\\

\haiku{Hoe smadelijk was.}{ik door mijn Mandrillen in}{de steek gelaten}\\

\haiku{Mijn hele leven,.}{was een droom geweest als dit}{hier ook een droom was}\\

\haiku{Heb je dan niet om,?}{je heen gekeken in woud}{en steppe warhoofd}\\

\haiku{Want even zeker als,}{ik die in deze laatste}{dag van je bestaan}\\

\subsection{Uit: Omnibus}

\haiku{Ziehier de fijne,,;}{schriele sprinkhaan het voedsel}{van Sint Jan Baptist}\\

\haiku{Twee dingen waren,:}{het die mij aangetrokken}{hadden in de aap}\\

\haiku{De aap sprong rond in.}{mijn huis alsof hij er al}{jaren geweest was}\\

\haiku{Tevens was ik blij.}{dat het probleem zichzelve}{uitgewezen had}\\

\haiku{Mijn aap heeft een paar;}{eikeltjes gestolen die}{langs de weg lagen}\\

\haiku{was hij al aanstonds,.}{goede maatjes met haar toen het}{meisje binnenkwam}\\

\haiku{Ik zette het raam,.}{open want de kamer werd mij}{te eng en benauwd}\\

\haiku{Ik schold op hem, maar,.}{sloeg hem niet meer want hij was}{nog zwak en mager}\\

\haiku{Je moet niet alleen,.}{de verschillen leren maar}{ook de overeenkomst}\\

\haiku{In geen geval moest;}{ze zich echter met het beest}{kunnen bemoeien}\\

\haiku{{\textquoteright} {\textquoteleft}O, het is zo erg{\textquoteright},.}{niet zei ik en bond mijn pols}{af met m'n zakdoek}\\

\haiku{Direct toen het licht,,.}{aanging zag ik hem zitten}{in zijn mand mijn aap}\\

\haiku{Stijf en koud ben ik.}{naar boven gegaan toen het}{haast ochtend was}\\

\haiku{XVIII Aan het eind;}{van alle overpeinzingen}{stond mijn besluit vast}\\

\haiku{de nieuwsgierigheid:}{die we allen hebben voor}{het einde van iets}\\

\haiku{Reeds enige dagen.}{was hij gehuwd en woonde}{hij in het paleis}\\

\haiku{Nauwelijks was hij,}{bij de vijver gekomen}{of hij wierp zo snel}\\

\haiku{Dan hief hij be{\^\i} zijn:}{armen op en begon een}{lied te zingen}\\

\haiku{{\textquoteright} De jager zette.}{zijn hoed recht en kruiste de}{armen over z'n borst}\\

\haiku{De gewoonheid van.}{haar kinderspel had niemand}{kunnen verbazen}\\

\haiku{In bed dacht zij dan,:}{nog dikwijls aan de pop en}{het was wonderlijk}\\

\haiku{Groucha antwoordde,:}{niet maar de student pakte}{de mand op en zei}\\

\haiku{Daar nam zij de pop,.}{die hard voelde en stijf van}{de oude kleren}\\

\haiku{des avonds stierf Groucha,,:}{glimlachend omdat zij zacht}{en ver nog hoorde}\\

\haiku{Toen vouwde hij de.}{blaadjes voorzichtig dicht en}{stak ze in zijn zak}\\

\haiku{{\textquoteright} {\textquoteleft}Onze plantages,{\textquoteright}.}{zijn meer Afrika dan je denkt}{sprak de oude man}\\

\haiku{De volgende dag.}{vroeg ik of Lindor met mij}{mee mocht gaan jagen}\\

\haiku{{\textquoteleft}dan kun je wel zes.}{dragers meenemen om ze}{naar huis te dragen}\\

\haiku{Eens had Wassilj aan.}{een man gevraagd of het nog}{ver was naar Moskou}\\

\haiku{Er vloeide geen bloed,,.}{maar de zwerm week achteruit}{verder en verder}\\

\haiku{hoe grijze nevels.}{hangen in een huis waar licht}{en liefde woonden}\\

\haiku{En terwijl Sunanda,:}{met haar hoofd in zijn schoot lag}{fluisterde zij zoet}\\

\haiku{In Alec's arm leunde,.}{Kesini zwaarder en vermoeid toen}{zij huiswaarts keerden}\\

\haiku{{\textquoteright} En zich omdraaiend,,.}{spuwde hij in een nis waar}{een ikoon moest hangen}\\

\haiku{Het leven dat je.}{bij een stervende bijna}{als een onrecht voelt}\\

\haiku{{\textquoteleft}Laten we hem naar,.}{het hospitaal brengen hij}{is bewusteloos}\\

\haiku{Hijgend wilde hij,.}{roepen maar zijn stem was slechts}{een schor gerochel}\\

\haiku{Mijn lippen raken,,,;}{u Ichthus witte vis die}{geslacht zijt om ons}\\

\haiku{Buiten was alles,,.}{stil geworden een kille}{ruisende stilte}\\

\haiku{aanstonds, morgen, pijpt,.}{hij voor en allen dansen}{wij dien Orpheus na}\\

\haiku{Zij zag hoezeer hij,.}{leed en nam de kevie mee}{naar de biljartzaal}\\

\haiku{{\textquoteright} {\textquoteleft}Maak je niet bezorgd,{\textquoteright}, {\textquoteleft}.}{lachte Alfredik ben zo}{safe als de paus}\\

\haiku{Of mijn dochter,{\textquoteright} dacht,.}{hij nog maar hij sprak deze}{aanvulling niet uit}\\

\haiku{En toen de avond kwam,.}{begaf de jonge vrouw zich}{reeds vroeg ter ruste}\\

\haiku{{\textquoteright} Zij leefden ieder,;}{apart teruggetrokken en}{vervreemd van allen}\\

\haiku{Zij dronken de wijn.}{die rode warmte geeft en}{stille vertroosting}\\

\haiku{Daar zegt de nachtwacht:}{opeens luidop de jongste}{van zijn gedachten}\\

\haiku{Als ik bewegen,,.}{kon denkt de nachtwacht zou ik}{zeker gaan kijken}\\

\haiku{De ratten namen.}{een doden gijzelaar mee}{in hun dodenrijk}\\

\haiku{En voor de eerste,:}{maal terwijl het dag is ziet}{ge hun gedaante}\\

\haiku{Nergens meer een schat.}{die niet hun scherpe tanden}{hebben beschadigd}\\

\haiku{Hij verschanst zich, duwt,.}{de tafel voor de deur de}{nachtwacht is bevreesd}\\

\haiku{{\textquoteleft}Hier zijn geen ratten,{\textquoteright},.}{gromt de muzikant terwijl}{de riemen plassen}\\

\haiku{Allen volgen den,.}{pastoor in een tumult dat}{sneller sneller gaat}\\

\haiku{{\textquoteright} bedenkt hij, en den '.}{bode reikt hijt glas om}{weer te vullen}\\

\haiku{Een koude is er,,.}{die door alles heendringt in}{de kern van je hart}\\

\haiku{Hij haakt de keten,,.}{los die om zijn hals hangt legt}{hem neer op tafel}\\

\haiku{{\textquoteright} Maar de muzikant.}{knoopt reeds zijn wambuis dicht en}{keert zich naar de deur}\\

\haiku{Het bootje is een,,....}{stip verdwenen nog te zien}{voorgoed verdwenen}\\

\haiku{Tot zo laat was er;}{tussen al Gods dagen geen}{verschil voor Jukkers}\\

\haiku{Het eten was vanavond,...}{slecht maar uwe liefkozingen}{zijn al mijn voedsel}\\

\haiku{anders was zoveel.}{leugen en bedrog immers}{niet nodig geweest}\\

\haiku{Spranger belde mij....}{op voor een spoedoperatie}{voor Arthur Christensen}\\

\haiku{De ander trachtte,.}{hem te kalmeren maar hij}{wond zich steeds meer op}\\

\haiku{Als mensen uit een.}{andere wereld staarden}{wij elkander aan}\\

\haiku{In ieder geval,{\textquoteright}.}{was het beestje omnivoor}{stelde Fowler vast}\\

\haiku{Het brak tussen mijn,.}{tanden en ik bekeek het}{stuk dat ik vasthield}\\

\haiku{Ik zag hoe Fowler,.}{kokhalsde en was niet in}{staat te antwoorden}\\

\haiku{Vermoeid was zij nu,.}{na drie jaren geworden}{ten dode vermoeid}\\

\haiku{Alles was van een.}{uiterste zindelijkheid}{en scheen te stralen}\\

\haiku{Maar de oude vrouw.}{trok me aan de mouw mee naar}{een andere reet}\\

\haiku{{\textquoteleft}Wij moeten nu de,.}{andere kant uit want het}{laatste huis is leeg}\\

\haiku{Er zijn misdaden,.}{zo verschrikkelijk dat ze}{geen naam meer hebben}\\

\haiku{Bijna is het dag,{\textquoteright},.}{zegt hij naar buiten stappend}{waar de paarden staan}\\

\haiku{De Noordelijkste!}{kaart van pater Martinus}{was onnauwkeurig}\\

\haiku{Het jongetje en.}{het meisje aan haar zijde}{zien nieuwsgierig rond}\\

\haiku{{\textquoteright} Maar slap en willoos.}{liet ze zich doorgolven en}{Cobbs verloor de moed}\\

\haiku{Waarom heb ik ook,.}{geen moeder fluisterde het}{binnenste van Cobbs}\\

\haiku{Want hij geloofde,.}{niet meer omdat er niets meer}{te geloven viel}\\

\haiku{Opnieuw verlaat een,.}{vogel de aarde-zee}{de tweede laag}\\

\haiku{Ik zag de jonge,.}{zachtgrijze dag achter de}{ijsbloemen wachten}\\

\haiku{Je mengt je onder,.}{de schreeuwers de schelders en}{de hoera-roepers}\\

\haiku{Desnoods reis je om,,,.}{over Frankrijk en Engeland}{Noorwegen Zweden}\\

\haiku{Niettemin ben ik,.}{glimlachend op weg gegaan}{zo lijkt het me nu}\\

\haiku{Ach, die dame die,.}{zich nimmer verbidden laat}{dat weet ik vandaag}\\

\haiku{Als ik hem over de...}{bomen of in de rivier}{had kunnen gooien}\\

\haiku{Zwijgzaam als altijd,.}{en zonder verwondering}{bemerkten ze mij}\\

\haiku{Men kon zien dat hij.}{slechts verachting voelde voor}{hun botte smoelen}\\

\haiku{Ik vertelde hun, -}{dat ik ongelukkig was}{maar niet van de brief}\\

\haiku{{\textquoteright} Een weinig verbaasd:}{over deze onverwachte}{vraag antwoordde ik}\\

\haiku{Ik greep hem uit haar,,.}{handen kuste hem stak hem}{jubelend omhoog}\\

\haiku{Ik bezag slechts mijn,.}{brief van alle kanten dat}{kostbare kleinood}\\

\haiku{Maar deze brief is,,?}{mijn manneneer mijn roeping}{mijn al. Begrijp je}\\

\haiku{Kwamen ze niet in,?}{de kast waar Marja mijn jas}{had weggehangen}\\

\haiku{Voorzichtig sloop ik,.}{de trap af dat haar tante}{mij niet zou horen}\\

\haiku{Ik zou gaan, als de.}{koperen pijl op de spits}{naar den vijand wees}\\

\haiku{De heuvelrug die,.}{ik beklimmen moest was steil}{en terrasvormig}\\

\haiku{Orion, die langzaam,.}{klom en zijn arm hief naar de}{horens van de Stier}\\

\haiku{- een sterke arm die,.}{mij tilde een warme borst}{waaraan ik insliep}\\

\haiku{Maar is het geen dwaas -,!}{die z\'oveel onderneemt en}{Marja en Marja}\\

\haiku{Het zou lijken op,.}{een verzoek om steun en ik}{had geen steun nodig}\\

\haiku{Stad met de vele,.}{bruggen over het blinkende}{troebele water}\\

\haiku{De onrust knaagde,,.}{de vertwijfeling die vroom}{maakt deed mij vloeken}\\

\haiku{Altijd dezelfde,,.}{straten dezelfde hoeken}{dezelfde huizen}\\

\haiku{En van het toeval,,.}{indien dat bestaat moeten}{we profiteren}\\

\haiku{Voor de dag ermee,!}{met de brief die je daar in}{je binnenzak hebt}\\

\haiku{ik leed meer dan ooit,.}{te voren onder dit steeds}{dringender gevraag}\\

\haiku{De onbekende,.}{best\'a\'at en daarom zal hij}{ook zeker komen}\\

\haiku{Hij kon zich haar niet,.}{goed meer voorstellen had haar}{maar weinig gezien}\\

\haiku{Het w\`as natuurlijk,,.}{zo en geen middel bestond}{daartegen geen troost}\\

\haiku{In mijn tijd had een.}{meisje als Genia niet lang}{hoeven te wachten}\\

\haiku{Je moeder was nog,.}{geen vijfentwintig toen ze}{trouwde ga maar na}\\

\haiku{En hij dacht [maar zei].}{het toch niet dat dit best de}{laatste keer kon zijn}\\

\haiku{{\textquoteleft}Met zo'n jong hart als.}{dat van Oom Lucas kan je}{honderd jaar worden}\\

\haiku{Ik ben vroeger ook,.}{in betrekking geweest toen}{moeder nog leefde}\\

\haiku{Een heide waarin.}{zich het pad verloren heeft}{tussen de bosjes}\\

\haiku{{\textquoteleft}Wat een zonde dat.}{Oom Lucas dit niet meer heeft}{kunnen meemaken}\\

\haiku{{\textquoteleft}Ik ben het, Cesar,{\textquoteright},.}{fluisterde hij opdat ze}{niet verschrikken zou}\\

\haiku{Had hij soms willen?}{scheppen toen hij met Genia}{naar het stadhuis ging}\\

\haiku{de boreling van.}{zijn moeder was een ander}{schepsel dan hijzelf}\\

\haiku{Naast een man zoals?}{Cesar die niets wilde en}{geen ambities had}\\

\haiku{Het  zijn was voor.}{hem nog slechts een ogenblik van}{oneindige duur}\\

\haiku{Hij droomde dan, dat.}{een klein en wrak scheepje met}{hem afdreef naar zee}\\

\haiku{hij vergenoegde.}{zich met zijn uitkijkpost op}{de vestingmuur}\\

\haiku{In werkelijkheid,:}{verkende ze hem en ze}{kirde aan zijn oor}\\

\haiku{Maar het was er niet,.}{minder levendig om het}{liet mij rust noch duur}\\

\haiku{Wel, vandaar kwam ik.}{weer bij de gereedschappen}{en de machines}\\

\haiku{Ik kreeg precies de.}{nodige brandstof om aan}{de gang te blijven}\\

\haiku{Hij vertelde, dat.}{hij ziek was geweest en nog}{niet geheel hersteld}\\

\haiku{We kunnen daarvan.}{profiteren door vanavond}{samen uit te gaan}\\

\haiku{Het is vast een stuk.}{vlees dat van de haak gescheurd}{is en gevallen}\\

\haiku{{\textquoteleft}Hoor eens lieveling,.}{je hebt vergeten om de}{ijskast dicht te doen}\\

\haiku{De trekken van den.}{dodelijk gewonden man}{gaan zich ontspannen}\\

\haiku{hij weet vooruit dat.}{deze bij na-dode}{vastbesloten is}\\

\haiku{Nog lang v\'o\'or het dag,,.}{is kraait er zelfs een haan hier}{midden in de stad}\\

\haiku{na jaren is haar....}{parfum het enige wat je}{bijgebleven is}\\

\haiku{Nog geen week later.}{werd hij telegrafisch naar}{Moskou ontboden}\\

\haiku{een hond blafte, de,.}{zee ruiste een frisse bries}{woei onverstoorbaar}\\

\haiku{De gardiaan van.}{de Capucijnen is al}{urenlang op het fort}\\

\haiku{{\textquoteright} brulde hij, zo hard.}{dat de anderen bij de}{toonbank omkeken}\\

\haiku{De terechtstelling.}{van de Madrileense}{opstandelingen}\\

\haiku{{\textquoteleft}Ik trek af,{\textquoteright} denkt hij, {\textquoteleft},.}{bevel of geen bevel het}{kan mij niet schelen}\\

\haiku{Als was niet de man,.}{met de sigaar maar hij de}{ge\"executeerde}\\

\haiku{Uit zichzelf alleen.}{zou hij toch geen moed vinden}{voor zulk een besluit}\\

\haiku{{\textquoteright} Alfredo kwam op.}{hem toe en legde zijn hand}{op Tristan's schouder}\\

\haiku{En Tristan nu,:}{met ferme slagen op de}{rug kloppend zei hij}\\

\haiku{in al de jaren,,.}{heb ik geloof ik ook nooit}{meer aan hem gedacht}\\

\haiku{Je hebt gelijk, we.}{zijn al tien jaar getrouwd en}{er is veel gebeurd}\\

\haiku{Het had geen zin de,.}{tijd te tellen z\'o dicht bij}{het tijdeloze}\\

\haiku{Maar allen waren,.}{even apathisch allen schenen}{te slaapwandelen}\\

\haiku{{\textquoteleft}Ik geloof dat wij.}{op weg zijn veel te grote}{vrienden te worden}\\

\haiku{{\textquoteleft}En welke liegt er,,?}{niet om wanneer ze zich te}{jong waant of te oud}\\

\haiku{Of wat jij misschien?}{niet reeds tegen andere}{vrouwen gezegd hebt}\\

\haiku{{\textquoteleft}Rubbermelk is het,}{grootste goed dat de hemel}{ons gegeven heeft}\\

\haiku{Je veegt de room van....}{je lippen af en begint}{opnieuw te zoeken}\\

\haiku{Die het met minder,.}{doen zijn beslist pechvogels}{of onwilligen}\\

\haiku{Een beste leerling, -,.}{die dan ook de kans heeft een}{groot man te worden}\\

\haiku{{\textquoteright} {\textquoteleft}Ze zijn nu ook veel,{\textquoteright}.}{langer antwoordde Carmen}{verontschuldigend}\\

\haiku{Zo wordt een viool, -.}{doortinteld van een toon als}{ik van Carmen was}\\

\haiku{Maar welk een klein mens,,!}{ben ik was ik dat zo-iets}{hem kon ombrengen}\\

\haiku{Zo stonden wij daar.}{tegenover elkander in}{de vestibule}\\

\haiku{Daarom ging ik maar.}{de kapel binnen om tot}{mijzelf te komen}\\

\haiku{thans ben ik in een,, -.}{ander landschap hier ben ik}{vreemd nochtans bekend}\\

\haiku{onherroepelijk.}{komt achter elke hoogste}{top weer een vallei}\\

\haiku{Zweetdruppels kwamen.}{als kleine torren over mijn}{voorhoofd gekropen}\\

\haiku{Ze kruipen op langs,,.}{de wangen de neusvleugels}{naar de oogkassen}\\

\haiku{Met een juichkreet zag,,.}{ik het bos de weide de}{hemel om mij heen}\\

\haiku{Met dit alles had,!}{ik mijn ogen gebet om weer}{ziende te worden}\\

\haiku{Namen en wegen,.}{ging ik weten om toch het}{spoor bijster te zijn}\\

\subsection{Uit: Orkaan bij nacht}

\haiku{die zeker van mij.}{denkt  zoals ik toenmaals}{van mijn vader dacht}\\

\haiku{Hij is mijn zoon, maar.}{ach hoe bitter weinig van}{mijzelf is in hem}\\

\haiku{Zou het mogelijk?}{zijn dat mensen werkelijk}{kunnen liefhebben}\\

\haiku{En je bewustzijn.}{van dit alles zoetjes maar}{in slaap te zingen}\\

\haiku{Immuun zijn en de.}{stoten op te vangen met}{voldoende weerstand}\\

\haiku{, weer eens andere,.}{gezichten om mij heen zag}{andere huizen}\\

\haiku{Ze heeft geloofd dat.}{het een plotseling en kloek}{besluit van mij was}\\

\haiku{Maar de leegte, de,.}{onnozelheid de weerzin}{is ook hier gelijk}\\

\haiku{{\textquoteleft}Weet je dan niet dat?}{er miljoenen meisjes zijn}{die Maria heten}\\

\haiku{Maar m{\'\i}j zou ze niet,.}{vangen al had ze dan geen}{slechte kijk getoond}\\

\haiku{Het medelijden,,.}{dat tot het Ik zijn punt van}{uitgang wederkeert}\\

\haiku{Hij is niets dan een,.}{schaduw dat heb ik je reeds}{vaak genoeg gezegd}\\

\haiku{Wat baat het om te?}{zeggen dat ik het vooruit}{had kunnen weten}\\

\haiku{nu ga ik morgen,.}{zeker weg nu kan ik het}{niet meer uitstellen}\\

\haiku{Kloppen niet alle?}{toeristenharten sneller}{bij deze overvaart}\\

\haiku{Het komt er net op ',,}{aan hoe jet wilt zien zegt}{hij tegen zichzelf}\\

\haiku{Maar Minne is toch.}{blij dat hij bij hen zit en}{een weinig meepraat}\\

\haiku{En om niet al te:}{vriendschappelijk te schijnen}{voegt hij er aan toe}\\

\haiku{De dame heeft nog,.}{een jongetje bij zich van}{een jaar of vier vijf}\\

\haiku{Ieder jaar heeft hij;}{het minder begrepen op}{de Afrikaanse zon}\\

\haiku{Daarom zal hij het.}{de tijd die hem rest hier ook}{nog wel uithouden}\\

\haiku{Doch hij zou het wel,.}{willen ontkennen want hij}{is tegelijk kwaad}\\

\haiku{zulke vrouwen zijn,,.}{zelf net grote lastige}{wrede kinderen}\\

\haiku{Want hij peutert nog.}{steeds met zijn zakmes in het}{automobieltje}\\

\haiku{je hebt natuurlijk.}{kou gevat met dat natte}{pak van gisteren}\\

\haiku{Dan legt ze haar hand.}{op zijn voorhoofd en voelt hoe}{heet en klam het is}\\

\haiku{Tot ziens...{\textquoteright} En ze vraagt,.}{zich of hoe ze ertoe komt}{juist dit te zeggen}\\

\haiku{hoe zou er anders.}{een saamhorigheid tussen}{hen kunnen bestaan}\\

\haiku{Hij herinnert zich.}{het aarzelende in haar}{spreken en kijken}\\

\haiku{Natuurlijk, ze heeft;}{het kind meegenomen om}{zich te wapenen}\\

\haiku{En nu zijn ze met,.}{hun beiden alleen voor de}{allereerste maal}\\

\haiku{zeker is het dat;}{onze sterren ieder hun}{eigen baan volgen}\\

\haiku{Dit is de enige:}{hoop die onverwoestbaar in}{mij is gebleven}\\

\haiku{Minne probeert zich.}{rekenschap te geven van}{deze verdwazing}\\

\haiku{Een tussenweg... dus,,.}{halfheid w\'e\'er een compromis}{w\'e\'er een mislukking}\\

\haiku{hoe kun je eerlijk,?}{zijn zonder eerst te weten}{wat eerlijkheid is}\\

\haiku{Het leven heeft niets,.}{goed met hem voorgehad en}{nu is het te laat}\\

\haiku{{\textquoteright} {\textquoteleft}Natuurlijk, daarin,{\textquoteright}.}{heb je volkomen gelijk}{zegt Minne beschaamd}\\

\haiku{Daar is de brede;}{vlakke kust en het wijde}{blauw van de oceaan}\\

\haiku{Maar als Claire het,,?}{zelf uithoudt als z{\'\i}j het kan}{waarom dan h{\'\i}j niet}\\

\haiku{Het past bij Claire's,,;}{uiterlijk maar niet bij haar}{wezen denkt Minne}\\

\haiku{laat hij zich nu maar.}{tevreden stellen met zijn}{commandantsdochter}\\

\haiku{de lange uren dat;}{hij alleen is en Claire}{verdiept in haar werk}\\

\haiku{De Noordafrikaanse.}{lente is ondertussen}{warmer geworden}\\

\haiku{Het zonlicht trillert.}{al over de heuvels en staat}{schel boven de zee}\\

\haiku{Dit is een nieuwe.}{lievelingsplek die Minne}{moet leren kennen}\\

\haiku{in het binnenste;}{gedeelte van de baai dringt}{slechts heel weinig zon}\\

\haiku{De drift die thans in,.}{hem is geeft hem lust op de}{rotsen te beuken}\\

\haiku{tot het uiteenscheurt.}{en wij in tijdeloze}{diepte verdwijnen}\\

\haiku{maar is de zuiverste?}{liefde geen concentratie}{van levensgevaar}\\

\haiku{De vrouw hoort het aan:}{het iets-gejaagde van}{zijn praten en zegt}\\

\haiku{Toch heeft dit lijflijk.}{contact met de storm hem goed}{gedaan voor een poos}\\

\haiku{Zij voelt haar warmte,.}{in de zijne overgaan zijn}{gloed in de hare}\\

\haiku{Het duister brandt aan,.}{hun randen maar het is dicht}{en ondoordringbaar}\\

\haiku{En vernieuwt zich het,.}{vuur in hem uitlaaiend in}{de zwartere nacht}\\

\haiku{Hij is jaloers, en.}{het is spijt die hem stokstijf}{doet staan bij de deur}\\

\haiku{En eens moet t\'och de.}{droom gekoppeld worden aan}{het alledaagse}\\

\haiku{Boos stampt hij met zijn:}{voet op de grond en zegt met}{tranen in zijn stem}\\

\haiku{De hoge trotse,;}{schoorsteen is gevallen de}{tqaf is gebroken}\\

\haiku{{\textquoteright} zegt de vrouw, veel meer.}{ge{\"\i}nteresseerd nu ze}{zich tot Minne wendt}\\

\haiku{Ik hoop dat je het,{\textquoteright}.}{eens bent met mijn voorstellen}{zegt Minne lachend}\\

\haiku{Hoe jong hij ook is,.}{zij ziet dat hij zijn eigen}{weg begint te gaan}\\

\haiku{Daarom juist wil hij.}{liever twijfelen en het}{lezen uitstellen}\\

\haiku{Marc heeft veel dingen,.}{in zijn kleine hoofd die hij}{zou willen zeggen}\\

\haiku{Hij kan Minne niet;}{zomaar verachten net als}{hij het Suze doet}\\

\haiku{Het begrip {\textquoteleft}kind{\textquoteright} is.}{groot en dreigend opgestaan}{in zijn bewustzijn}\\

\haiku{Zijn wrakhout verder,;}{spoelen achterlaten op}{verlaten kusten}\\

\haiku{Wat heeft hun liefde,?}{aan een halfslachtig dienen}{aan gebondenheid}\\

\haiku{En onderwijl vindt:}{Postma bij Hopkins het onthaal}{dat hij verwachtte}\\

\haiku{{\textquoteright} {\textquoteleft}Ja, onze harems...{\textquoteright}.}{zegt de oude Arabier even}{naar hem toegewend}\\

\haiku{De nacht geeft Sam iets,.}{milds het grootsprekerige}{is een beetje weg}\\

\haiku{Hij houdt de handen,;}{aan zijn hoofd alsof hij in}{vervoering luistert}\\

\haiku{Gevaarlijker dan.}{opium of hennepzaad is}{westerse muziek}\\

\haiku{De toon van Minne's.}{telefoongesprek geeft hem}{volle zekerheid}\\

\haiku{{\textquoteleft}En...{\textquoteright} stottert Minne, {\textquoteleft}...?}{en wat wil je zeggen met}{die combinatie}\\

\haiku{In de verte is.}{nog slechts het laatste restje}{zonnegloed te zien}\\

\haiku{Of zij in deze.}{levenskwestie niet zo denkt}{als Sam bijvoorbeeld}\\

\haiku{Geen enkele maal,.}{heeft hij haar naam genoemd al}{kent hij die zeer goed}\\

\haiku{Enkel wacht ze nog,.}{dat Minne het zal zeggen}{wat ze reeds vermoedt}\\

\haiku{de kinderen die,.}{de toekomst zijn ze willen}{ook niet dat ik blijf}\\

\haiku{En suizelend zwenkt.}{de auto door de tuinpoort}{naar de grote weg}\\

\haiku{{\textquoteleft}Is het mogelijk?}{dat een mens gelukkig is}{en het zelf niet weet}\\

\haiku{Wel gaan zijn voeten,;}{licht alsof ze nog niet goed}{de aarde raken}\\

\haiku{het kind was lastig.}{en Madame kon driftig}{worden als een man}\\

\haiku{Hij is weggegaan,.}{ik heb nooit meer gehoord waar}{hij gebleven is}\\

\haiku{Ik hoopte dat ik,:}{ooit zo'n mens ontmoeten zou}{om maar te weten}\\

\haiku{Ik wist alleen wat,,.}{sexe was ik wist dat liefde}{\'anders zijn moest m\'e\'er}\\

\haiku{Hij vertelt nu ook,.}{van zijn eigen leven met}{slechts weinig woorden}\\

\haiku{alle dreiging heeft,.}{het thans verloren en het}{raakt de diepte niet}\\

\haiku{Er blijft nog slechts het,.}{kleine menselijke het}{uiterlijk daarvan}\\

\haiku{Maar madame zal?}{nu misschien wel weer het heft}{in handen nemen}\\

\haiku{Ja, mijn waarde, zo,....}{z{\'\i}jn ze de vrouwtjes spreekt hij}{hem in stilte toe}\\

\haiku{Hij slaat zijn kleine,:}{armen om Minne's hals en}{zoent hem en zegt lief}\\

\haiku{{\textquoteright} De heerlijkheid van.}{hun geheim blijft veilig in}{zijn hoede achter}\\

\haiku{Haar hart klopt zwaar en ':}{t is of elke harde}{bloedstoot weer herhaalt}\\

\haiku{Niemand, indien hij.}{niet voluit de prijs van zijn}{hart durft betalen}\\

\haiku{Ik kan niet meedoen.}{aan het verder kweken van}{de oude leugens}\\

\haiku{Dichterbij nog is.}{het zachte praten van een}{welbekende stem}\\

\haiku{Eens heb ik naar de,...}{dood verlangd het storten in}{een donkere kloof}\\

\subsection{Uit: Peis noch vree}

\haiku{wat hij gewild had,.}{dat voor eeuwig zou bestaan}{nu was het genoeg}\\

\haiku{Man en vrouw in bed,.}{ze nemen weinig m\'e\'er plaats}{dan een enkeling}\\

\haiku{Ze gaan erin, en,,!}{blijven tot ze betere}{mensen zijn erin}\\

\haiku{Ook dit kommerlijk.}{bestaan echter zou hun niet}{lang vergund blijven}\\

\haiku{De vrouw haalde de:}{schouders op en antwoordde}{zacht maar resoluut}\\

\haiku{Als wij iets hadden,.}{ik zou het gezegd hebben}{om h\'a\'ar te sparen}\\

\haiku{Zwijgend boog Josef.}{zich over zijn vrouw en tilde}{haar in zijn armen}\\

\haiku{Koffie,{\textquoteright} sprak de man.}{en trachtte monter wat licht}{in de lucht te zien}\\

\haiku{{\textquoteleft}Honden, bedelaars.}{en lieden van het vreemde}{ras hier ongewenst}\\

\haiku{Gelukkig, dacht zij,.}{dat Davidson tenminste}{dit niet heeft gemerkt}\\

\haiku{{\textquoteright} Toen had zijn moeder:}{het alweer weggetrokken}{met de berisping}\\

\haiku{waar het alles vond:}{waarmee het van meet af aan}{vertrouwd geraakt was}\\

\haiku{Wat een dwaasheid om.}{je op een tocht als deze}{te gaan poeieren}\\

\haiku{Tenslotte kun je,.}{alles overdrijven dat moest}{je toegeven}\\

\haiku{En aan je mooiheid,.}{komt gauw een eind vooral als}{je kinderen krijgt}\\

\haiku{Hij keek alleen een.}{beetje schuinsweg naar haar ogen}{en knikte zachtjes}\\

\haiku{Ik was toch met hem.}{getrouwd omdat ik op mijn}{manier van hem hield}\\

\haiku{De kalk was niet heel,.}{wit meer scheen opgelost in}{het schemergrauwen}\\

\haiku{De visser zette.}{zijn hengel tegen de wand}{en stiet de deur open}\\

\haiku{Ik heb er mijn loon,.}{voor gehad zo zou je het}{kunnen opvatten}\\

\haiku{Hier, waar het vredig.}{is en ik u lastig val}{met mijn geraaskal}\\

\haiku{{\textquoteright} {\textquoteleft}Zo zijn vrouwen,{\textquoteright} vond,.}{de hengelaar terwijl hij}{het bed klaarmaakte}\\

\haiku{{\textquoteleft}Al wat er is, al,.}{wat er leeft houdt Hij in zijn}{versteende handen}\\

\haiku{Achter haar hoorde:}{ze de scherpe neuriestem}{van Henri zingen}\\

\haiku{Hij droeg een Duitse,.}{uniformjas maar zijn broek was}{die van een burger}\\

\haiku{Een ijlheid die zelfs.}{met het koudste water niet}{viel af te wassen}\\

\haiku{Daar lag het landschap,.}{zoals zij het gisteren}{nog niet gezien had}\\

\haiku{Je bent nog aan 't -.}{begin alleen een kind ziet}{tegen jaren op}\\

\haiku{Hoe glunder had hij.}{haar niet aangekeken toen}{hij haar kwam wekken}\\

\haiku{Zo zullen het ook,.}{de landen moeten doen en}{de werelddelen}\\

\haiku{{\textquoteleft}Vrijwillig,{\textquoteright} troostte,.}{ik mijzelf want ik had toch}{kunnen weigeren}\\

\haiku{{\textquoteleft}Jullie zouden pas,{\textquoteright}.}{gelukkig zijn wanneer je}{een kind had zei hij}\\

\haiku{Daarom weet ik nu.}{ook heel precies hoe het toen}{moet zijn gelopen}\\

\haiku{ze) daar straffeloos.}{en haast bij wijze van sport}{worden uitgemoord}\\

\haiku{Ik weet niet of er.}{gevolg gegeven is aan}{haar gratieverzoek}\\

\haiku{je vader,{\textquoteright} zei de.}{kluizenaar toen de grijsaard}{eindelijk klaar was}\\

\haiku{{\textquoteleft}Niet onze woorden,.}{maar onze daden vormen}{onze verdienste}\\

\subsection{Uit: De rancho der X mysteries}

\haiku{Terwijl hij naar het,:}{geld zocht trok hij een couvert}{voor de dag en zei}\\

\haiku{Maar ik zal hem wel.}{zorgvuldig bewaren tot}{aan mijn eigen dood}\\

\haiku{Dat is een van de.}{hoofdopgaven van wat men}{hier beschaving noemt}\\

\haiku{Die ligt nog altijd.}{achter slot in een lade}{van dit schrijfburo}\\

\haiku{{\textquoteleft}Want inderdaad heb,,:}{ik een ogenblik niet langer}{dan een flits gedacht}\\

\haiku{Hoe mijn geest zich zou,.}{voelen leek mij voorlopig}{nog raadselachtig}\\

\haiku{Toen viel mijn oog op,.}{een langwerpige krat die}{geheel anders was}\\

\haiku{{\textquoteleft}Het is toch jammer.}{dat wij de politie niet}{geroepen hebben}\\

\haiku{{\textquoteright} De knecht had weer moed,.}{gevat liet den Amerikaan}{niet alleen werken}\\

\haiku{{\textquoteright} {\textquoteleft}Toen is de {\textquoteleft}Timor{\textquoteright}.}{gezonken van de zwaarte}{van al die vlinders}\\

\haiku{Ik ging dus alleen.}{en moet bekennen dat ik}{teleurgesteld werd}\\

\haiku{Dan schenk ik hem u..}{Een vriendelijk geschenk mag}{men niet weigeren}\\

\haiku{Gelijk zo vaak in}{mijn leven verweet ik een}{onverschrokkener}\\

\haiku{Ik was nog niet veel,:}{wijzer maar de conducteur}{die langs kwam en riep}\\

\haiku{Ik moet even verschrikt,:}{gekeken hebben want hij}{verklaarde aanstonds}\\

\haiku{Geloof niet, dat ze.}{er iets bijgeleerd hebben}{in al die eeuwen}\\

\haiku{Het enige wat zij,.}{leerden is bidden in de}{kerk en berusten}\\

\haiku{en voelde ik mijn,,.}{koffer die ik nog in de}{hand hield weggerukt}\\

\haiku{Er waren meer dan.}{dertig doden en meer dan}{honderd gewonden}\\

\haiku{Mijn verstand zei me,,,.}{dat ik om dit te kunnen}{doen moest hertrouwen}\\

\haiku{Sommigen slaakten,.}{een zucht van verlichting of}{misschien wel van spijt}\\

\haiku{hij wankelde en.}{de omstaanders begonnen}{weer op te dringen}\\

\haiku{Dat is verboden,{\textquoteright}.}{verklaarde Candelario}{met plechtige ernst}\\

\haiku{{\textquoteleft}Zo'n auto is voor.}{hen een mysterie en voor}{ons iets alledaags}\\

\haiku{{\textquoteleft}Uw vader heeft een,.}{kortgeknipte snor als van}{zwarte kokosbast}\\

\haiku{Het liep tegen de,.}{middag de tijd dat alles}{onbeweeglijk wordt}\\

\haiku{Maar dan onder de,.}{mensen zodat ze worden}{als redeloos vee}\\

\haiku{{\textquoteleft}Ik wou dat alles.}{zo licht op mijn geweten}{woog als deze Yaqui}\\

\haiku{Het zal wel een uur,.}{geduurd hebben voordat de}{kleine deur open ging}\\

\haiku{Daarna heb je mij,.}{verschrikt en dan krijg ik soms}{van die toevallen}\\

\haiku{maar de rest van de.}{droom was overduidelijk en}{liet mij niet meer los}\\

\haiku{{\textquoteleft}Ik zal nooit al die.}{personen en stromingen}{leren ontwarren}\\

\haiku{Hij houdt er alleen,.}{niet van zijn paarlen voor de}{zwijnen te werpen}\\

\haiku{het zal het eerste,.}{zijn wat ik doe zodra ik}{mijn geld gebeurd heb}\\

\haiku{Ik probeerde de.}{panama en daarna zo'n}{weke vilten}\\

\haiku{{\textquoteright} zei ik onderweg,.}{nog lachend om het dwaze}{visioen van Efra{\'\i}n}\\

\haiku{De lange golfslag.}{der regenvlagen zong mij}{eindelijk in slaap}\\

\haiku{{\textquoteright} Hij leek opgelucht,:}{door dit antwoord wendde zich}{af en zei schamper}\\

\haiku{Het zou van alles, -...}{kunnen zijn van een wapen}{tot een stuk speelgoed}\\

\haiku{maar nauwelijks ziet,......}{hij het hij hoort alleen en}{sterft twintig doden}\\

\haiku{maar ik kon alleen,.}{inwendig vloeken omdat}{ik zo weerloos was}\\

\haiku{Daar hielden de vier.}{bandieten stil en stegen}{van hun paarden af}\\

\haiku{Maar toen deze hem,:}{zag aankomen fluisterde}{hij geruststellend}\\

\haiku{slechts een kleine haag.}{van stevige rietstengels}{scheidde mij ervan}\\

\haiku{Harris, toen zij zich.}{opzettelijk door mij had}{laten inhalen}\\

\haiku{Met zachte klopjes.}{begon ik de hand van het}{meisje te strelen}\\

\haiku{Maar het hart dat hij,.}{bezat was nog wijder en}{opener dan dit land}\\

\haiku{Het huttendorp was.}{echter uitgestorven toen}{wij er aankwamen}\\

\haiku{Nauwelijks was de,.}{Indio goed en wel weg of}{de vrouw kwam terug}\\

\haiku{Vandaag zal ik u.}{niet lastig vallen met zo'n}{afgodendienaar}\\

\haiku{Agapito heeft ook;}{kruiden gegeven aan de}{vrouw die mij verzorgt}\\

\haiku{{\textquoteright} Maar de Schoolmeester,,:}{schudde het hoofd en zei half}{lachend half spijtig}\\

\haiku{Hij deed het rustig,,;}{vleiend bijna maar op zeer}{suggestieve toon}\\

\haiku{Ze hadden best een.}{spelletje met Felipe}{willen beginnen}\\

\haiku{En nog verder moesten.}{bruine  adobe-hutten}{van dc armen staan}\\

\haiku{Ik antwoordde van,.}{ja maar voegde er aan toe}{dat hij bezoek had}\\

\haiku{{\textquoteleft}Neen,{\textquoteright} zei hij, {\textquoteleft}ik zal.}{je liever laten zien dat}{ook het licht niets baat}\\

\haiku{{\textquoteleft}Juana Sierra,{\textquoteright}.}{weer met hetzelfde verzoek}{om een gebedje}\\

\haiku{uit alles bleek, dat.}{hij een even goed theoloog}{als doodgraver was}\\

\haiku{{\textquoteright} {\textquoteleft}De Heilige Maagd,}{heeft zeker niets met deze}{dingen te maken}\\

\haiku{De figuur droeg  .}{een kroon en hield de armen}{op de borst gekruist}\\

\haiku{het is een soort van,.}{leegte een verlangen naar}{niet-verlangen}\\

\haiku{Hijzelf weigerde.}{rond te hangen bij zulk een}{uitgestorven boel}\\

\haiku{Baboso nogmaals.}{de magische strepen van}{zijn toorn over de grond}\\

\haiku{Maar laat hem eens aan...}{zijn lot over en kom dan over}{zes maanden terug}\\

\haiku{Maar niemand weet of,.}{ik mijzelf straks kan redden}{laat staan een ander}\\

\haiku{{\textquoteright} Met alles wat in,.}{mij was verzette ik mij}{tegen dit denkbeeld}\\

\haiku{Zie je nu wel, dat?}{mijn waarschuwing maar al te}{goede gronden had}\\

\haiku{Maar Mexico heeft bij.}{al het boosaardige iets}{van de eeuwigheid}\\

\haiku{{\textquoteleft}Er is een man van,.}{je rancho gekomen die}{je dit gebracht heeft}\\

\haiku{Morgenochtend komt.}{hij bij mij op kantoor om}{je te ontmoeten}\\

\haiku{Zo niet, dan is het,.}{misschien nog wel het beste}{dat gij ze niet weet}\\

\haiku{Groet mijn neef Isidro en,.}{zeg hem dat hij gelijk heeft}{om weg te blijven}\\

\haiku{Het viel moeilijk uit,.}{te maken maar tenslotte}{kon het heel best zijn}\\

\haiku{Maar wat is er nog,?}{anders in te vinden dan}{stof stof en knekels}\\

\haiku{Om generaal te,.}{zijn moet men in Mexico ook}{zo\"oloog wezen}\\

\subsection{Uit: Serenitas}

\haiku{wat al die vrienden,.}{zochten leek hem hazardspel}{onzindelijkheid}\\

\haiku{{\textquoteright} Slechts eenmaal had zijn,.}{baas gedacht dat het ook met}{Dorus mis zou lopen}\\

\haiku{Na dagen werk sprak,.}{zij hem weer aan een avond toen}{hij van zijn werk kwam}\\

\haiku{Nou ja, je hebt ook{\textquoteright},, {\textquoteleft}'.}{gelijk zei het meisjet}{Is nog niet eens aan}\\

\haiku{Als je kwam moest je.}{maar meteen zeggen wanneer}{we gingen trouwen}\\

\haiku{{\textquoteright} {\textquoteleft}Een man heeft altijd{\textquoteright},.}{bedoelingen met een vrouw}{zei Marietje bits}\\

\haiku{Met een wonderlijk,:}{hoge jongensstem die even}{maar schor klonk zong hij}\\

\haiku{Ze zuchtte verlicht,.}{toen ze hem daar nog zag staan}{geduldig wachtend}\\

\haiku{Aan onszelf mankeert,.}{ook wat en je moet trachten}{samen goed te zijn}\\

\haiku{De liefde is een.}{geluk en een offer had}{de pastoor gezegd}\\

\haiku{Toen hij de pastoor, '.}{weer sprak vertelde Dorus van}{t mislukte plan}\\

\haiku{Wat zal het moeilijk.}{zijn om die te openen voor}{een vriendelijk woord}\\

\haiku{ze was nog zwak, en,:}{toen Dorus weer alleen zat met}{zijn tante sprak hij}\\

\haiku{{\textquoteright} {\textquoteleft}Een mens kan zoveel,{\textquoteright}.}{honger hebben dat het hem}{niet meer kan schelen}\\

\haiku{ze zal geloven,.}{dat het medelijden van}{je is of zoiets}\\

\haiku{{\textquoteright} En toen zij bezig:}{was te drinken van het glas}{dat hij nog vasthield}\\

\haiku{Je ziet het wel, ze{\textquoteright},.}{is niet erg toeschietelijk}{zei de oude vrouw}\\

\haiku{{\textquoteleft}Je moest een ander,.}{huis zien te krijgen als we}{ooit zover komen}\\

\haiku{Want zo gesloten,,:}{zo afwezig was ze soms}{dat hij moest denken}\\

\haiku{'t Is alleen erg,.}{dat je zo verloren loopt}{en zonder veel doel}\\

\haiku{En ik ben juist hier{\textquoteright}.}{komen wonen om van veel}{geklets af te zijn}\\

\haiku{zoals het gaan moet.}{en die maakt dat alles ten}{goede terecht komt}\\

\haiku{Een man die nooit door;}{hartstocht weifelde in zijn}{koninklijk gebaar}\\

\haiku{Dorus echter zat in.}{diep nadenken verzonken}{toen Winters thuis kwam}\\

\haiku{Het wil me niet uit{\textquoteright},, {\textquoteleft}.}{de gedachten sprak hijdat}{je ongelijk hebt}\\

\haiku{{\textquoteleft}Het is iets prachtigs.}{als je zo zorgeloos en}{zo gelukkig bent}\\

\haiku{{\textquoteleft}Als ze nog van mij,{\textquoteright}.}{houden dan zou u ook van}{me moeten houden}\\

\haiku{Zo'n huichelaar, zo'n,!}{stiekemerd zo'n vuilpoes van}{een farizee\"er}\\

\haiku{De kinderschaar aan,,.}{uwe voeten O Jezus komt}{U blij begroeten}\\

\haiku{Zonde van iemand!}{die zo van het leven zou}{kunnen genieten}\\

\haiku{Als de knechts reeds zijn,.}{schaften staat Dorus nog steeds in}{de deur en luistert}\\

\subsection{Uit: De stille plantage}

\haiku{En wie ze ooit van,.}{te voren zag hij hervindt}{ze                         nimmermeer}\\

\haiku{{\textquoteright}  Josephine.}{sloeg haar arm om hem heen en}{streelde zijn haren}\\

\haiku{Raoul wil niet naar, '.}{Gen\`eve bij alt getwist}{om bijbelwoorden}\\

\haiku{{\textquoteright}                        Maar C\'ecile,:}{zag de blaren vallen en}{in haar zei een stem}\\

\haiku{{\textquoteright}  {\textquoteleft}Daarom heb ik,.}{ook niet omgezien maar in}{mij en nu vooruit}\\

\haiku{{\textquoteleft}Dit land is te klein,.}{voor ballingen te rijk}{voor armen als wij}\\

\haiku{los genoeg wezen;}{om daarheen te gaan waar het}{schoonste ons                     roept}\\

\haiku{En stellig spreken,.}{wij elkander nader}{over vijf of tien jaar}\\

\haiku{{\textquoteright}  {\textquoteleft}Zoudt gij denken?}{dat er een dusdanige}{voorbestemming was}\\

\haiku{{\textquoteright}  {\textquoteleft}O neen, ik ga,{\textquoteright}.}{om niet weer te keren zei}{Raoul beraden}\\

\haiku{Wantrouw de hitte,,.}{wantrouw het land wantrouw zelfs}{wat u schoon                     lijkt}\\

\haiku{Reeds woog de harde.}{scherpe hitte op hun hoofd}{en op hun handen}\\

\haiku{{\textquoteright}  {\textquoteleft}De grond is hier,{\textquoteright}, {\textquoteleft},.}{ook goed sprak Daszo goed dat}{het een nadeel heeft}\\

\haiku{De heer                     Morhang.}{zal zien dat goedheid ondeugd}{is voor dit zwart vee}\\

\haiku{Een stoel die niet deugt,}{hak je tot brandhout                     een}{slaaf die je ergert}\\

\haiku{Hij kon aan wal gaan.}{om het terrein nauwkeurig}{te                     verkennen}\\

\haiku{Hoe verder van                     ,.}{de mensen hoe groter de}{kansen voor geluk}\\

\haiku{{\textquoteleft}Verbannen zult ge,.}{u                     zeker voelen zo}{bij tijd en wijle}\\

\haiku{met de                     aanleg.}{van tabaksaanplantingen}{was reeds begonnen}\\

\haiku{{\textquoteleft}Deze plantage.}{zal eerst over twintig jaar ten}{volle bloeien}\\

\haiku{op wie de stilte.}{der plantage zwaarder weegt}{dan nodig is}\\

\haiku{het zwart van                     zijn.}{halfnaakte lichaam leek slechts}{schaduw van een wolk}\\

\haiku{Er is een blanke,.}{geestelijke dicht hier in}{de buurt verscholen}\\

\haiku{De angsten van haar.}{verbeelding                     werden er}{nog tastbaarder door}\\

\haiku{Op de plantages.}{was de regentijd die van}{het grootste gevaar}\\

\haiku{Hier was het een park;}{dat wachtte op het rustig}{treden van een mens}\\

\haiku{Ons hart is een boot;}{die geankerd ligt in een}{zeer                     stille baai}\\

\haiku{Een deel slechts van 't,.}{lied verstond Raoul en het}{ergerde                     hem}\\

\haiku{{\textquoteright}  De anderen.}{noemden hem indringer en}{hoogmoedige dwaas}\\

\haiku{Slechts                     dat kon hem,:}{redden van de ondergang}{het doemwaardige}\\

\haiku{De vruchtbare geest,.}{van het woud houdt zich verre}{van ons dacht zij}\\

\haiku{Als zelfs                     hij het,...}{niet begreep hoe zouden dan}{ooit de anderen}\\

\haiku{Agnes was nu de;}{uiterste grenzen van de}{velden genaderd}\\

\haiku{Bijna verlaten.}{lagen de kostgronden en}{de tabaksvelden}\\

\haiku{Wat nog gered werd,.}{was verweg het kleinste en}{schamelste deel}\\

\haiku{Als Willem Das weer,.}{iets d\'a\'arover zeggen zou kon hij}{hem                     antwoorden}\\

\haiku{Zonder te schreeuwen '.}{kromden zij zich ondert}{snerpen van zijn zweep}\\

\haiku{Heeft een meisje eens,,?}{geen woorden                     los daaruit}{verwaaid gestameld}\\

\haiku{Recht boven hun hoofd,.}{scheen de koperen zon en}{geen vogel zong meer}\\

\haiku{Zelf droeg hij haar, zwaar,.}{op zijn armen maar zonder}{te hijgen en kalm}\\

\haiku{Snel doet de dood zijn.}{werk in                     het land van het}{brandende leven}\\

\haiku{{\textquoteright} stond op en wenkte.}{de negers dat zij hem weg}{konden                     dragen}\\

\haiku{Waanzinnig stormde,:}{Agnes                     naar binnen de}{haren verwilderd}\\

\haiku{Duizend geluiden;}{omvingen het blokhuis en}{heel de plantage}\\

\haiku{Een                     vlaag joeg haar,.}{de kamer uit naar het bed}{van de opzichter}\\

\haiku{een man die zij had,.}{kunnen                     liefhebben die}{rechten had op haar}\\

\haiku{Des avonds speelde zij.}{vaak met het bruine kind dat}{nu een vrije was}\\

\haiku{Een harde onwil.}{deed Raoul de tanden op}{elkander klemmen}\\

\haiku{meer grond dan hij                     ,,,,...}{betreden kon veel huizen}{slaven vrouwen macht}\\

\haiku{Hij zegt dat ik niets,.}{van het                     werk begrijp en}{dat is misschien waar}\\

\haiku{Er was geen denken,.}{aan                     ze in de bossen}{te achtervolgen}\\

\haiku{{\textquoteleft}Als het moet, kunnen.}{wij hier later altijd weer}{terug                     komen}\\

\haiku{Dit licht was nieuw en.}{luisterrijker dan zij ooit}{tevoren zagen}\\

\haiku{Het was niet in te.}{denken dat je dit alles}{nooit meer zou zien}\\

\haiku{Niet met de dagen,.}{leef je daar maar met de}{snelheid van dromen}\\

\haiku{Des ochtends was zij,.}{het eerste buiten op de}{voorplecht van de boot}\\

\haiku{Altijd door spoelde;}{de zee langs                     het schip en}{wast zijn wanden schoon}\\

\haiku{{\textquoteleft}Soms was het toch nog,{\textquoteright}.}{mooier ginds sprak zij als tot}{zichzelve}\\

\haiku{{\textquoteright}  {\textquoteleft}Ja, je had er,{\textquoteright}.}{prachtige kansen bromde}{Raoul in zijn baard}\\

\haiku{eeuwig is de                     ,.}{stroom de enige die deze}{wildernis trotseert}\\

\haiku{Ik heb een oude...}{neger hier eens er over}{horen vertellen}\\

\haiku{Doch met de grootste,:}{stelligheid antwoordde}{de oude neger}\\

\haiku{Maar gaat gerust uw,,.}{gang                     heren producers}{mij maakt het niets uit}\\

\subsection{Uit: De stille plantage}

\haiku{Een zonnescheut, en:,,.}{zij denken verder verder}{verder moet het zijn}\\

\haiku{zou zijn leven niet...}{verloren zijn wanneer dit}{alles er niet was}\\

\haiku{Raoul wil niet naar, '.}{Gen\`eve bij alt getwist}{om bijbelwoorden}\\

\haiku{{\textquoteleft}Je begrijpt wat het.}{is voor Raoul om alles}{achter te laten}\\

\haiku{Zij bleef staan en trok.}{haar bij zich op het lage}{muurtje langs de weg}\\

\haiku{{\textquoteleft}Ik moet je nog wat,{\textquoteright}, {\textquoteleft}.}{vertellen sprak hijwat ik}{straks verzwijgen moest}\\

\haiku{D\'a\'ar was Raoul, die;}{buiten nog haastig stond te}{spreken met zijn oom}\\

\haiku{D\'a\'ar Josephine,;}{die haar arm om C\'ecile}{heen geslagen had}\\

\haiku{die luistert en ver;}{staat de gefluisterde taal}{van wind en wolken}\\

\haiku{{\textquoteleft}Bekroop u nog nooit,?}{de lust om daarginder te}{blijven kapitein}\\

\haiku{Reeds woog de harde.}{scherpe hitte op hun hoofd}{en op hun handen}\\

\haiku{Maar als ge wilt, zal.}{ik gaarne van uw raad en}{dienst gebruik maken}\\

\haiku{Met statig en toch;}{vriendelijk gebaar ontving}{de landvoogd  hem}\\

\haiku{En ik denk dat gij,}{die het leven hier niet kent}{en niet gewend zijt}\\

\haiku{Of zegt de bijbel?}{niet dat elke meester zijn}{knecht kastijden mag}\\

\haiku{{\textquoteleft}Vergeet morgen niet,{\textquoteright}:}{de dijken op te hogen}{of tegen Raoul}\\

\haiku{Maar ik denk toch niet,.}{dat hij de rechte man is}{hier op deze plaats}\\

\haiku{{\textquoteleft}'t Is niet goed in.}{deze oorden dat een man}{te lang alleen zij}\\

\haiku{Wanneer je met een,.}{vrouw wilt wonen zal ik het}{de meester vragen}\\

\haiku{{\textquoteright} {\textquoteleft}Laat mij het nog niet,,{\textquoteright}.}{zeggen misses smeekte hij}{het hoofd gebogen}\\

\haiku{Er is een blanke,.}{geestelijke dicht hier in}{de buurt verscholen}\\

\haiku{Op de plantages.}{was de regentijd die van}{het grootste gevaar}\\

\haiku{Hier was het een park;}{dat wachtte op het rustig}{treden van een mens}\\

\haiku{Waar zij was liet zij,.}{iets dierbaars achter stierf een}{deel af van haar zelf}\\

\haiku{Dit was zijn rijk, dat.}{hij regeren zou naar zijn}{eer en geweten}\\

\haiku{Wat heb je eraan.}{met alle anderen in}{onvrede te zijn}\\

\haiku{Steeds herkende zij.}{weer een hoek of een boom of}{een dak van Bel Exil}\\

\haiku{{\textquoteleft}Wie honger heeft moet.}{de aarde bebouwen in}{het zweet zijns aanschijns}\\

\haiku{Agnes was nu de;}{uiterste grenzen van de}{velden genaderd}\\

\haiku{{\textquoteright} {\textquoteleft}Kan je geduldig,?}{zijn als je eensklaps het spoor}{bijster geraakt bent}\\

\haiku{Bijna verlaten.}{lagen de kostgronden en}{de tabaksvelden}\\

\haiku{Lang nog blijven ze,.}{wanneer uw woning reeds leeg}{en verlaten is}\\

\haiku{Zonder te schreeuwen '.}{kromden zij zich ondert}{snerpen van zijn zweep}\\

\haiku{Recht boven hun hoofd,.}{scheen de koperen zon en}{geen vogel zong meer}\\

\haiku{Zelf droeg hij haar, zwaar,.}{op zijn armen maar zonder}{te hijgen en kalm}\\

\haiku{Rustig, zonder te.}{haasten trad hij naar buiten}{toen Raoul hem riep}\\

\haiku{Duizend geluiden;}{omvingen het blokhuis en}{heel de plantage}\\

\haiku{En daarin, opeens,,.}{tjuikte een vogel kraaide}{de eerste haan}\\

\haiku{Het leek of ze niet.}{het minst begrepen wat er}{gedaan moest worden}\\

\haiku{niemand stoorde meer.}{het nachtelijk leven van}{de negerinnen}\\

\haiku{daarin was zij zelf.}{somtijds een vreemde die zij}{plotseling hervond}\\

\haiku{Een harde onwil.}{deed Raoul de tanden op}{elkander klemmen}\\

\haiku{Te weinig om de.}{strijd tegen de wildernis}{voorgoed te winnen}\\

\haiku{En hij nam haar in,.}{zijn armen kuste haar de}{tranen uit de ogen}\\

\haiku{{\textquoteright} {\textquoteleft}Zijn jeugd...{\textquoteright} glimlachte,.}{Josephine vrolijk door}{Raouls zekerheid}\\

\haiku{Josephine dacht.}{niet langer aan heengaan en}{niet meer aan blijven}\\

\haiku{Haar eindelijk weer.}{te paard te zien gaf hem een}{onverwacht plezier}\\

\haiku{Dit licht was nieuw en.}{luisterrijker dan zij ooit}{tevoren zagen}\\

\haiku{Langs onbekende,:}{bossen varen de boten}{en toch denkt een mens}\\

\haiku{Des ochtends was zij,.}{het eerste buiten op de}{voorplecht van de boot}\\

\haiku{{\textquoteleft}'t Is bij wijze,.}{van vergiffenis die ik}{u vraag mejuffer}\\

\haiku{{\textquoteright} {\textquoteleft}Ach, zeg toch liever,}{nog eens hoe die tijgers uit}{de bossen kwamen}\\

\haiku{{\textquoteright} Maar Josephine,:}{verdedigde haar zoon en}{spotte op haar beurt}\\

\haiku{{\textquoteright} {\textquoteleft}Als u dat bedoelt,{\textquoteright}, {\textquoteleft}.}{zei toen Gastonik ga er}{later vast naar toe}\\

\haiku{Maar gaat gerust  ,,.}{uw gang heren producers}{mij maakt het niets uit}\\

\subsection{Uit: De stille plantage}

\haiku{Een zonnescheut, en:,,.}{zij denken verder verder}{verder moet het zijn}\\

\haiku{zou zijn leven niet...}{verloren zijn wanneer dit}{alles er niet was}\\

\haiku{Raoul wil niet naar, '.}{Gen\`eve bij alt getwist}{om bijbelwoorden}\\

\haiku{{\textquoteleft}Je begrijpt wat het.}{is voor Raoul om alles}{achter te laten}\\

\haiku{Zij bleef staan en trok.}{haar bij zich op het lage}{muurtje langs de weg}\\

\haiku{{\textquoteleft}Ik moet je nog wat,,.}{vertellen sprak hij wat ik}{straks verzwijgen moest}\\

\haiku{D\'a\'ar was Raoul, die;}{buiten nog haastig stond te}{spreken met zijn oom}\\

\haiku{D\'a\'ar Josephine,;}{die haar arm om C\'ecile}{heen geslagen had}\\

\haiku{Bekroop u nog nooit,?}{de lust om daarginder te}{blijven Kapitein}\\

\haiku{Reeds woog de harde.}{scherpe hitte op hun hoofd}{en op hun handen}\\

\haiku{Maar als ge wilt zal.}{ik gaarne van uw raad en}{dienst gebruik maken}\\

\haiku{En ik denk dat gij,}{die het leven hier niet kent}{en niet gewend zijt}\\

\haiku{Of zegt de bijbel?}{niet dat elke meester zijn}{knecht kastijden mag}\\

\haiku{Maar ik denk toch dat,.}{hij niet de rechte man is}{hier op deze plaats}\\

\haiku{{\textquoteleft}'t Is niet goed in.}{deze oorden dat een man}{te lang alleen zij}\\

\haiku{Krijgen ze uitbouw,?}{aan hun loods waarover ik je}{laatst gesproken heb}\\

\haiku{Wanneer je met een,.}{vrouw wilt wonen zal ik het}{de meester vragen}\\

\haiku{{\textquoteright} {\textquoteleft}Laat mij het nog niet,,{\textquoteright}.}{zeggen misses smeekte hij}{het hoofd gebogen}\\

\haiku{Er is een blanke,.}{geestelijke dicht hier in}{de buurt verscholen}\\

\haiku{Op de plantages.}{was de regentijd die van}{het grootste gevaar}\\

\haiku{Hier was het een park;}{dat wachtte op het rustig}{treden van een mens}\\

\haiku{Waar zij was liet zij,.}{iets dierbaars achter stierf een}{deel af van haar zelf}\\

\haiku{{\textquoteleft}Wat zijn zij nog ver,.}{van het Christendom zei hij}{tegen C\'ecile}\\

\haiku{Dit was zijn rijk, dat.}{hij regeren zou naar zijn}{eer en geweten}\\

\haiku{Wat heb je eraan.}{met alle anderen in}{onvrede te zijn}\\

\haiku{Steeds herkende zij.}{weer een hoek of een boom of}{een dak van Bel Exil}\\

\haiku{Wie honger heeft moet.}{de aarde bebouwen in}{het zweet zijns aanschijns}\\

\haiku{Bijna verlaten.}{lagen de kostgronden en}{de tabaksvelden}\\

\haiku{Lang nog blijven ze,.}{wanneer uw woning reeds leeg}{en verlaten is}\\

\haiku{Ik geloof niet dat.}{ze gaarne naar andere}{meesters verhuizen}\\

\haiku{Recht boven hun hoofd,.}{scheen de koperen zon en}{geen vogel zong meer}\\

\haiku{Zelf droeg hij haar, zwaar,.}{op zijn armen maar zonder}{te hijgen en kalm}\\

\haiku{Rustig, zonder te.}{haasten trad hij naar buiten}{toen Raoul hem riep}\\

\haiku{Duizend geluiden;}{omvingen het blokhuis en}{heel de plantage}\\

\haiku{En daarin, opeens,,.}{tjuikte een vogel kraaide}{de eerste haan}\\

\haiku{t Leek of ze niet.}{het minst begrepen wat er}{gedaan moest worden}\\

\haiku{niemand stoorde meer.}{het nachtelijk leven van}{de negerinnen}\\

\haiku{daarin was zij zelf.}{somtijds een vreemde die zij}{plotseling hervond}\\

\haiku{Een harde onwil.}{deed Raoul de tanden op}{elkander klemmen}\\

\haiku{Te weinig om de.}{strijd tegen de wildernis}{voorgoed te winnen}\\

\haiku{Het zweet droop langs zijn,.}{hoofd en armen zo snel liep}{hij naar het blokhuis}\\

\haiku{En hij nam haar in,.}{zijn armen kuste haar de}{tranen uit de ogen}\\

\haiku{{\textquoteright} {\textquoteleft}Zijn jeugd...{\textquoteright} glimlachte,.}{Josephine vrolijk door}{Raouls zekerheid}\\

\haiku{Haar eindelijk weer.}{te paard te zien gaf hem een}{onverwacht plezier}\\

\haiku{Langs onbekende,:}{bossen varen de boten}{en toch denkt een mens}\\

\haiku{Des ochtends was zij,.}{het eerste buiten op de}{voorplecht van de boot}\\

\haiku{Slechts Agnes stond nog '.}{laat opt achterdek te}{staren in dit zwart}\\

\haiku{Hij heeft ook u doen,,}{ontwaken v\'o\'or het te laat}{was en mogelijk}\\

\haiku{{\textquoteleft}'t Is bij wijze,.}{van vergiffenis die ik}{u vraag mejuffer}\\

\haiku{{\textquoteright} {\textquoteleft}Ach, zeg toch hever,}{nog eens hoe die tijgers uit}{de bossen kwamen}\\

\haiku{{\textquoteright} Maar Josephine,:}{verdedigde haar zoon en}{spotte op haar beurt}\\

\haiku{{\textquoteright} {\textquoteleft}Als u d\`at bedoelt,,.}{zei toen Gaston ik ga er}{later vast naar toe}\\

\subsection{Uit: Het vergeten gezicht}

\haiku{Ze hadden onrust,;}{in zijn leven gebracht en}{maar schamel plezier}\\

\haiku{Het was trouwens al;}{de zoveelste keer dat hij}{in Veracruz was}\\

\haiku{Maar de ander zei,:}{lachend om de ergernis}{van zijn kameraad}\\

\haiku{Kon hij niet altijd?}{weer een schip vinden wanneer}{het hem berouwde}\\

\haiku{Hij schaamde zich, maar.}{hij wist dat de Cubaan hem}{niet verraden zou}\\

\haiku{Slechts een enkele:}{maal vond hij gezelschap dat}{hem aangenaam was}\\

\haiku{Hij zocht de aarde,,;}{maar niet de mensen dat wist}{hij nu al zeker}\\

\haiku{Rufino haatte,;}{het gezicht daarvan dat hem}{protserig toescheen}\\

\haiku{Hij was dom geweest;}{zonder geleide ook dit}{stuk af te leggen}\\

\haiku{De enige kleine,:.}{zekerheid die hem nog bleef}{was weer teruggaan}\\

\haiku{Zijn weg moest westwaarts,.}{voeren dat was het enige}{wat hij nog wist}\\

\haiku{Tenochtitl\'an, dat {\textquoteleft}{\textquoteright}.}{later Spaans en pronkerig}{Mexico genoemd werd}\\

\haiku{Zo kwam hij, door het,.}{licht geleid waar het centrum}{van de stad moest zijn}\\

\haiku{misschien vond  hij.}{voor deze nacht een bank in}{een van de parken}\\

\haiku{Lang kon hij niet meer, -.}{zonder werk lopen hoogstens}{enkele weken}\\

\haiku{Maar toen hij op straat,,.}{kwam sloeg het licht hem tegen}{evenals het geraas}\\

\haiku{Met een warme blik,;}{een glimlach die het begin}{werd van herkennen}\\

\haiku{een middelpunt van,;}{alle leven van alle}{wereldgebeuren}\\

\haiku{lichtgrijze nevels,.}{die het helle verkleurde}{zonlicht verstoorden}\\

\haiku{Wat was er nog over?}{van de reine hoogvlakte}{die hij verwacht had}\\

\haiku{Ondanks alles was,...}{hij er misschien toch beter}{aan toe geweest daar}\\

\haiku{iemand was die zich,,.}{al was het maar voor kort om}{hem bekommerde}\\

\haiku{Rufino meende,.}{dat zij een vreemdelinge}{moest zijn net als hij}\\

\haiku{- en ook, dat hij  ...}{een portret bezat van het}{meisje van zijn oom}\\

\haiku{ze kon dat met haar.}{ervaring meteen aan het}{soort bezoeker zien}\\

\haiku{Rufino voelde,.}{haar harde lichaam dat warm}{was in zijn handen}\\

\haiku{Ze kon op deze;}{wijze alleen maar van kwaad}{tot erger komen}\\

\haiku{{\textquoteright} Matilde was voor,.}{hem komen staan en schudde}{hem bij de schouders}\\

\haiku{Vervolgens ging hij,.}{met langzame dreunende}{schreden naar buiten}\\

\haiku{Hij had gedacht met,.}{zijn hulp aan haar zichzelf te}{kunnen bevrijden}\\

\haiku{{\textquoteleft}Maar haar Agust{\'\i}n is,.}{een goed zakenman dat moet}{ik hem nageven}\\

\haiku{{\textquoteleft}Het is goed dat je,,}{nu gekomen bent en niet}{later in de avond}\\

\haiku{Rufino begon.}{te geloven dat hij haar}{onrecht had gedaan}\\

\haiku{En hij had zich door;}{zijn illusies omtrent haar}{laten misleiden}\\

\haiku{{\textquoteright} Dan zich plotseling,:}{weer oprichtend zonder de}{doos los te laten}\\

\haiku{Met zijn vaardige.}{linkerhand wierp hij achter}{zich de huisdeur dicht}\\

\haiku{{\textquoteright} Eerst toen gaf hij zich.}{rekenschap dat Rufino}{ongewapend was}\\

\haiku{Daarna wendde hij,:}{zich weer tot Rufino en}{vroeg verachtelijk}\\

\haiku{{\textquoteleft}Haast je wat,{\textquoteright} beval,.}{Agust{\'\i}n zijn revolver weer}{opgericht houdend}\\

\haiku{dat geleefd had in...}{een ver stadje van een streek}{die hij niet kende}\\

\haiku{Niet de lafheid van,,{\textquoteright}.}{een klein flikkerend schiettuig}{bedacht Rufino}\\

\haiku{Je kunt Matilde,,}{de groeten van mij doen als}{je haar nog ooit ziet}\\

\haiku{Het is jammer dat.}{je wegloopt en hier niet in}{zaken bent gegaan}\\

\haiku{Ik reken op je,.}{vriendschap het enige waaraan}{ik nog durf denken}\\

\haiku{Maar daarvan wist zij,.}{niets af en hij kon het haar}{ook niet bijbrengen}\\

\haiku{Niemand wist wanneer,.}{of waar hij schering  was}{en wanneer inslag}\\

\haiku{het werd geboren.}{uit de zee en gaat later}{weer daarin onder}\\

\haiku{{\textquoteright} {\textquoteleft}Ik ben een klant als,.}{iedere andere en}{betaal voor een nacht}\\

\haiku{Hij ging voorzichtig,}{naar binnen om te kijken}{wat er gebeurde}\\

\haiku{Na de ontzetting,;}{van het drama alleen te}{zijn met de dode}\\

\haiku{Weldra viel de nacht,.}{bij het naderen van een}{volgend gebergte}\\

\haiku{de chocolade,.}{ging daar van hand tot hand en}{zij beraadslaagden}\\

\haiku{{\textquoteleft}Denk je dat {\`\i}k me?}{nog veel herinner van waar}{ik geboren ben}\\

\haiku{Niemand sprak een woord.}{toen Rufino zich bij het}{troepje kwam voegen}\\

\haiku{De uitreis was niets,.}{dan zoete-koek geweest}{gesmeerde boter}\\

\haiku{Ik vraag me alleen,,}{af wat de gil was die van}{het achterdek kwam}\\

\haiku{Toch klonk het hem zelf,.}{potsierlijk toe nu hij het}{eenmaal gezegd had}\\

\haiku{Hij hield zich aan de,;}{reling vast toen hij langzaam}{naar achteren liep}\\

\haiku{Hij zei weer tegen;}{zichzelf dat het tenslotte}{maar inbeelding was}\\

\haiku{het was de eerste.}{keer dat iemand hier iets van}{zijn dochter ervoer}\\

\haiku{Bij alle mensen,.}{is het zo maar alleen bij}{mij kun je het zien}\\

\haiku{In dat geval was,.}{hij werkelijk gelukkig}{heel wat dagen lang}\\

\haiku{{\textquoteleft}Integendeel, ik.}{ben het met je eens dat je}{niet voor vrouwen voelt}\\

\haiku{De lichtmatroos riep,:}{hem hoog en beangst zijn nog}{kinderlijke stem}\\

\haiku{Maar Rufino vond;}{het eigenlijk heel rustig}{en plezierig zo}\\

\haiku{Te omhelzen wat,.}{verzwelgt te baren wat voor}{altijd ons omhult}\\

\haiku{Dan zou dat overal,.}{hetzelfde wezen op geen}{enkel schip beter}\\

\haiku{Rufino zat met,.}{het hoofd in de handen op}{de rand van zijn brits}\\

\haiku{{\textquoteright} Rufino's gezicht.}{vertrok zich nogmaals tot een}{pijnlijke glimlach}\\

\haiku{{\textquoteleft}Er is niet altijd,.}{rechtvaardigheid in deze}{wereld mijn jongen}\\

\haiku{Hij keerde terug;}{met de vrouw die hij haar mand}{afgenomen had}\\

\haiku{Nu zwierf hij wie weet,.}{waar en don Cosme maakte}{zich zorgen over hem}\\

\haiku{en de andere,{\textquoteright}.}{zoon is nu weg sprak do\~na}{Anita met een zucht}\\

\haiku{Uw dienaar... en hoogst...}{erkentelijk als ge u}{bij ons wilt zetten}\\

\haiku{Hij denkt dat ik hier,......}{slaap zoals vaker en jij}{in je kamertje}\\

\haiku{Zij vatte hem bij:}{de beide armen en drong}{hem zachtjes terug}\\

\haiku{Maar hier had ze zelfs,...}{geen flauwe kans gezien bij}{gebrek aan vrouwen}\\

\haiku{Opeens kwam het hem;}{voor dat hij dit alles reeds}{tienmaal beleefd had}\\

\haiku{Verwijfd leken die,;}{vreemdelingen zo groot en}{blank als zij waren}\\

\haiku{Al het andere,.}{was bijzaak slechts theater}{en fantasterij}\\

\haiku{Daar moet je vrouw voor,.}{zijn om zo'n subtiel bedrog}{meteen te doorzien}\\

\haiku{Laat ons graven waar,.}{wij staan want waar wij staan is}{altijd Klondyke}\\

\haiku{in zo korte tijd.}{raakte niemand ontwend aan}{het steedse leven}\\

\haiku{Tot elke prijs zou.}{zij trachten voorlopig bij}{Anita te blijven}\\

\haiku{{\textquoteleft}We hadden nooit met.}{deze aangelegenheid}{moeten beginnen}\\

\haiku{Hij liet de waard bij.}{zich komen en vroeg naar het}{adres van een bordeel}\\

\haiku{Het spijt me van dat,,{\textquoteright}.}{litteken Palomino}{zei ze vriendelijk}\\

\haiku{Mijn vriend Rufino....}{L\'opez is ook gebleven toen}{hij er zin in had}\\

\haiku{hij had niet eens de,.}{poncho over zich getrokken}{zich niet uitgekleed}\\

\haiku{De ochtenden hier.}{vervulden haar steeds met een}{wonderlijk ontzag}\\

\haiku{Ook de rechtste lijn,.}{is zo krom dat hij weer tot}{zichzelf terugkeert}\\

\haiku{Alleen Rufino.}{was een kind geweest zonder}{kinderachtigheid}\\

\haiku{Nu zag hij er, zo,.}{stram als hij daar stond meer als}{een militair uit}\\

\haiku{we zullen een heel,.}{stuk kunnen zeilen met uw}{beider welnemen}\\

\haiku{Macario was het,.}{eerst boven rekte zich en}{spiedde om zich heen}\\

\subsection{Uit: Waarom niet}

\haiku{Als hij heelemaal,}{buiten is waar het daglicht}{altijd heller schijnt}\\

\haiku{Hij is de sterkste,.}{van hun drie\"en vast nog veel}{sterker dan Rientje}\\

\haiku{Drie zweetdroppeltjes.}{komen in de bocht van haar}{wipneus te zitten}\\

\haiku{Eensgezind waren,.}{ze alleen in het gevaar}{als ze bang waren}\\

\haiku{Zoolang het leven,.}{zijn gewone gangetje}{ging kibbelden ze}\\

\haiku{Het woord was taboe,;}{maar je mocht het gerust in}{je eentje zeggen}\\

\haiku{Haar wipneusje vroeg,,:}{en ze besliste nog een}{beetje onzeker}\\

\haiku{Daar eindigden de.}{dagen en begonnen de}{nieuwe ochtenden}\\

\haiku{{\textquoteright} Rientje haalde de.}{bladeren uit haar blonde}{verwarde haren}\\

\haiku{Hij liet zijn armen;}{over de oppervlakte van}{het water zweven}\\

\haiku{de groepjes boomen;}{die in het wilde weg hier}{en daar uitstulpten}\\

\haiku{- {\textquoteleft}Fijn is dat gruis daar{\textquoteright},,.}{beneden zei Jan en wees}{naar de fjordenkust}\\

\haiku{De zon scheen bijna.}{loodrecht omlaag en om hen}{heen danste de lucht}\\

\haiku{Zelfs het grassprietje.}{waarmee hij ze plaagde bracht}{ze niet daarvan af}\\

\haiku{Ze stond het dichtst bij.}{Karel en rukte hem zijn}{steen uit de handen}\\

\haiku{Zij timmerde er,.}{op los en Jan verdween luid}{krijtend uit de grot}\\

\haiku{{\textquoteright} - {\textquoteleft}Jouw neus is net zoo{\textquoteright},.}{leuk als een kromme bloem zei}{Karel onverwachts}\\

\haiku{Bovendien konden;}{ze vanaf de grot precies}{zien wat je er deed}\\

\haiku{Een andere kracht.}{echter vocht tegen zijn lust}{om op zee te gaan}\\

\haiku{de gekartelde.}{rand van rotsen waartusschen}{wit waterschuim lag}\\

\haiku{Twee dagen bleef het,.}{zoo voordat het stormen en}{de regen ophield}\\

\haiku{Karel had wel aan,.}{Jan willen vragen waar ze}{was maar hij dorst niet}\\

\haiku{Voor de tweede maal.}{voelden de jongens zich een}{beetje opgelucht}\\

\haiku{hij dacht alleen dat.}{nu de kooi uiteensloeg en}{hij zou verdrinken}\\

\haiku{Maar zijn kleeren plakten.}{stijf en plankerig vast en}{maakten hem benauwd}\\

\haiku{De menschen die hier,.}{wonen weten stellig ook}{hoe weg te komen}\\

\haiku{Maar God die hem tot,.}{hier geholpen had zou hem}{ook verder helpen}\\

\haiku{Een tak met een soort.}{dennenappels roosterde}{en was half verbrand}\\

\haiku{En weer begon hij;}{te hopen dat er toch wel}{menschen zouden zijn}\\

\haiku{Hij zou dat kale,.}{weleens willen aanraken}{als hij maar durfde}\\

\haiku{Maar Rientje had het,;}{zoo besloten en Karel}{was het er mee eens}\\

\haiku{Haar eigen vader;}{zou geen gemakkelijke}{aan  haar hebben}\\

\haiku{de Paps van vreemde,.}{kinderen nog minder dat}{beloofde ze hem}\\

\haiku{Maar de man sloeg er,.}{geen acht op en dat maakte}{Rientje nog veel boozer}\\

\haiku{Zij sprong op en liep.}{hem achterna tot aan de}{ingang van de grot}\\

\haiku{Tenslotte begon:}{ze de jongens allerlei}{verwijten te doen}\\

\haiku{Zij namen ieder,.}{een van de dieren die de}{man ze aanreikte}\\

\haiku{Hij rukte eens aan,.}{het stuk dat af was en liet}{de jongens trekken}\\

\haiku{Karel raapte een;}{paar vezels op en begon}{het ook te probeeren}\\

\haiku{Het scheen dat er toch.}{altijd iemand eenzaam moest}{zijn op het eiland}\\

\haiku{Hij wist dat hij op,;}{deze wijze zwak was en}{niet mocht toegeven}\\

\haiku{En hij trachtte zich,.}{te vermannen zette de}{tanden op elkaar}\\

\haiku{Manuel schudde,;}{het hoofd en bedacht dat hij}{weer dom had gedaan}\\

\haiku{- {\textquoteleft}Neen, het is een hoed{\textquoteright},.}{zei ze toen de man haar het}{schort wou ombinden}\\

\haiku{Wel was hij al flink.}{versleten en zaten er}{overal gaten in}\\

\haiku{{\textquoteright} vroeg Manuel toen. - {\textquoteleft},.}{alles op wasUit de beek}{natuurlijk dommert}\\

\haiku{Die vertelde ook.}{zulke verhalen van een}{orang-weet-ik-veel}\\

\haiku{Het leek wel de stem,.}{van Paps en die stem scheen uit}{de rots te komen}\\

\haiku{Iets donkers kwam naar,.}{voren en de man bad met}{verdubbelde kracht}\\

\haiku{Maar de eerste naam.}{maakte  het spelletje}{aanlokkelijker}\\

\haiku{- {\textquoteleft}Als jij in de grot,{\textquoteright},.}{komt slapen gaan wij in de}{hut zei Rientje slim}\\

\haiku{Hij, de groote-mensch,,,,!}{de dwaas de betweter de}{indringer de tyran}\\

\haiku{dat het van deze.}{wachtpost zou afhangen of}{hij wegkwam of niet}\\

\haiku{Toch merkte ze heel;}{goed dat de man vlak achter}{haar was gekomen}\\

\haiku{Hij had ze dit woord;}{vaak hooren gebruiken voor}{iets verschrikkelijks}\\

\haiku{{\textquoteright} - {\textquoteleft}Je ben eigenwijs,{\textquoteright},.}{je w{\`\i}lt het niet begrijpen}{riep Manuel uit}\\

\haiku{Ze bezorgden hem.}{ook allerlei gemak met}{kleine karweitjes}\\

\haiku{- {\textquoteleft}Als hij de vesten{\textquoteright},.}{meeneemt kunnen we zelf nooit}{meer weg bedacht Jan}\\

\haiku{nu en dan bracht hij.}{de hand boven zijn oogen en}{tuurde naar de zee}\\

\haiku{{\textquoteright} En hij schopte en.}{begon om zich heen te slaan}{als een dolleman}\\

\haiku{De fladderende.}{haren van Rientje voor hem}{uit maakten hem dol}\\

\haiku{In de arm knaagde,.}{een stekende pijn en ze}{hing nog altijd slap}\\

\haiku{Het begon nacht te,;}{worden en ze bleven maar}{op dezelfde plaats}\\

\haiku{Hij was heet, overal,.}{waar je hem aanpakte en}{lag maar te kermen}\\

\haiku{{\textquoteright} - {\textquoteleft}Als hij beter is,{\textquoteright},.}{beginnen we direct aan}{die kuil zei Rientje}\\

\haiku{Jan dacht dat hij met.}{een groote stok gekomen was}{om hem dood te slaan}\\

\haiku{Ik dacht eerst dat het,.}{een dorre tak was en wou}{hem eruit trekken}\\

\haiku{{\textquoteright} - {\textquoteleft}Ik zal er ook over{\textquoteright},.}{denken zei Rientje met een}{gezicht vol modder}\\

\haiku{{\textquoteright} - {\textquoteleft}Nou ja{\textquoteright}, zei Karel.}{die op dat oogenblik geen}{ander antwoord wist}\\

\haiku{Ze kwam tot bij de,.}{grot keek voorzichtig om het}{hoekje naar binnen}\\

\haiku{Hij bleef ook niet lang.}{aandringen dat zij in de}{grot zou overnachten}\\

\haiku{hij trok er zich niets;}{van aan dat Jan nu met een}{lamme arm rondliep}\\

\haiku{Er bestond ook geen.}{twijfel omtrent de plaats die}{ze had aangeduid}\\

\haiku{Het gelig beetje,;}{licht dat door de nevels scheen}{kroop alweer lager}\\

\haiku{Ze konden nog zien,.}{waar Manuel geloopen}{had zijn groote stappen}\\

\haiku{Het was veel beter.}{om nu maar een onschuldig}{gezicht te trekken}\\

\haiku{het had hem geleerd.}{om op zijn hoede te zijn}{en snel te vluchten}\\

\haiku{Op zoo'n klein eiland.}{zou het ventje hem toch niet}{kunnen ontloopen}\\

\haiku{Maar het kon ook iets,.}{anders zijn wat hij niet wist}{iets ongeneeslijks}\\

\haiku{Hij droomde dat het.}{verpletterd werd onder een}{geweldig rotsblok}\\

\haiku{Even maar, en hij zou,.}{het wanhopigste doen het}{uiterste riskeeren}\\

\haiku{je ging tenlaatste de.}{dingen zien waarop je zoo}{hardnekkig hoopte}\\

\haiku{Onderwijl stonden.}{Karel en Rientje maar met}{groote oogen te kijken}\\

\haiku{{\textquoteright} Hij was veel te bang.}{dat Manuel zich misschien}{nog bedenken zou}\\

\haiku{{\textquoteright} - {\textquoteleft}Morgen ga ik mijn{\textquoteright},.}{sabel halen liet hij er}{meteen op volgen}\\

\haiku{Ze vond dat jongens.}{nooit erg veel verder denken}{dan hun neus lang is}\\

\haiku{Val me niet lastig,.}{met al die histories met}{de  kinderen}\\

\haiku{Ik moet beginnen,,.}{te schreeuwen te wuiven dat}{ze mij opmerken}\\

\haiku{Vroeger dachten ze,.}{een paar duizend maar dat is}{nonsens natuurlijk}\\

\haiku{- {\textquoteleft}Vraag of dat eiland{\textquoteright},.}{heelemaal onbewoond was}{zei de kapitein}\\

\haiku{In Durban vind je.}{zeker een paar landslieden}{en jullie consul}\\

\haiku{Om een jongen die;}{nog steeds niet met zijn aanzoek}{op de proppen komt}\\

\haiku{dit is de eerste,.}{vrouw sedert mijn terugkomst}{die ik z\'o\'o bekijk}\\

\haiku{Je kunt net zoo goed.}{met ongewasschen voeten}{een meisje aaien}\\

\haiku{Eerst insisteerde.}{Lola om er meer van te}{weten te komen}\\

\haiku{Ik wou dat je wat{\textquoteright},.}{ego{\"\i}stischer dacht viel Gil}{hem in de rede}\\

\haiku{Integendeel, het;}{huwelijksaanzoek was haar}{een pak van het hart}\\

\haiku{Als je nu naar al,.}{die booten toe moet kan het}{je te pas komen}\\

\haiku{ik heb het immers,;}{niet gewild het kind moet het}{mij maar vergeven}\\

\haiku{Waarom moet ik nu?}{al gestraft worden dat ik}{hem niet gewild heb}\\

\haiku{Als het niet iets heel,.}{gewoons was had ik immers}{mijn mond gehouden}\\

\haiku{{\textquoteright} Manuel stond aan,,.}{de andere kant van de}{tafel bleek nerveus}\\

\haiku{hoeveel tijd zal er?}{noodig zijn voordat ik ervan}{zal kunnen houden}\\

\haiku{Hij ging werkelijk.}{naar de slaapkamer waar de}{wieg stond met het kind}\\

\haiku{Maar waarom zeggen,?}{die kinderen hetzelfde}{precies hetzelfde}\\

\haiku{Ik heb hem nadien,,.}{maar eens gezien en hij had}{een lam smal armpje}\\

\haiku{Je hebt een duivel,.}{uit mijn lichaam gerukt een}{duivel van verdriet}\\

\haiku{Trek ze toch uit, je,.}{bent almachtig nu en ik}{sterf aan deze pijn}\\

\haiku{Maar nog had hij geen,.}{besluit kunnen nemen wist}{hij niet wat te doen}\\

\haiku{Manuel kon er.}{niet toe besluiten zich nog}{meer bloot te geven}\\

\haiku{Met hem kunnen wij,.}{opschieten hij vindt al die}{gevoelens onzin}\\

\haiku{{\textquoteright} - {\textquoteleft}Het hangt er van af.}{waar je zelf gaat staan om die}{zaak te bekijken}\\

\haiku{{\textquoteright} - {\textquoteleft}Heb je nooit gemerkt?}{dat iets in ons zich verzet}{om na te denken}\\

\haiku{Natuurlijk vroeg ze. - {\textquoteleft}{\textquoteright},.}{of hij ook van reizen hield}{Neen zei Manuel}\\

\haiku{Jammer genoeg, zou,.}{vader zeggen ofschoon hij}{er nooit geweest is}\\

\haiku{Je zou in staat zijn.}{om uit wraak stilletjes met}{ons weg te varen}\\

\haiku{Wat is er precies?}{met die verloren korrels}{aarde aan de hand}\\

\haiku{- {\textquoteleft}Ik wed dat het een{\textquoteright},.}{citaat is uit Epictetus}{juichte Edwina}\\

\haiku{Lang stond Manuel,:}{ze nog na te wuiven tot}{Gil eindelijk zei}\\

\haiku{{\textquoteright} - {\textquoteleft}Zoomin als jij een{\textquoteright},.}{volmaakt anarchist plaagde}{Manuel terug}\\

\haiku{En het was alsof.}{ze hier op de boot eens zoo}{gek eruit zagen}\\

\haiku{Zij bedacht niet dat.}{het meisje ongeveer zou}{oud moest zijn als zij}\\

\haiku{Hij voelde nogmaals.}{of hij zijn revolver bij}{zich had gestoken}\\

\haiku{Dat ziet eruit als,.}{een schietstrik of het moet een}{verdraaid toeval zijn}\\

\haiku{{\textquoteleft}Zelfs tegen onze.}{dwaasheden is hun vernuft}{niet opgewassen}\\

\haiku{Als medemenschen.}{hebben wij geen enkel recht}{meer tegenover hen}\\

\haiku{Hij was mokkend in,}{de hoek gaan staan waar hij met}{heel scherp toekijken}\\

\haiku{als je ook met een....}{heele troep kerels op ze}{afkomt geen wonder}\\

\haiku{- {\textquoteleft}Ik geloof dat w{\`\i}j{\textquoteright},.}{wel vrienden zullen worden}{zei de oude man}\\

\haiku{{\textquoteright} De golven komen;}{en nemen het schip op en}{dragen het verder}\\

\haiku{{\textquoteleft}Ik ken daar een man,.}{die op ons wacht en die wel}{voor ons zorgen zal}\\

\haiku{hij was, hoe prachtig,,...}{hij zijn armen bewegen}{kon springen duiken}\\

\haiku{We hebben het hem{\textquoteright},.}{dikwijls genoeg gezegd zei}{Karel dan telkens}\\

\haiku{Er is geen erger.}{overwonnene dan de man}{die niets meer verwacht}\\

\haiku{En dan moest Rientje:}{tenslotte toegeven en}{uit zichzelf zeggen}\\

\haiku{Drie jaar later werd,.}{een jongen geboren die}{Jan Willem heette}\\

\haiku{Nu weet ik waarvoor.}{ik al deze jaren heb}{moeten doormaken}\\

\haiku{Ik begrijp...{\textquoteright} - {\textquoteleft}Op de{\textquoteright},.}{knie\"en zou ik u moeten}{danken zei Voorberg}\\

\haiku{Hij begreep dat de.}{vader hem dat waarschijnlijk}{kwalijk zou nemen}\\

\haiku{Ik kan in weelde;}{mij natuurlijk niet meten}{met Lord Curdington}\\

\haiku{{\textquoteleft}Kom-kom, wie zelf.}{zijn fouten inziet heeft ze}{al half verbeterd}\\

\haiku{{\textquoteleft}Je weet toch dat de?}{isra\"elieten niet aan}{den Christus gelooven}\\

\haiku{Je houdt van iemand,.}{of je houdt niet van hem daar}{kun je niets aan doen}\\

\haiku{Ik hield meteen van.}{Winny en van Edgar en}{van de kapitein}\\

\haiku{Maar als ik ze niet,.}{begrijpen kan heb ik er}{niets mee te maken}\\

\haiku{De god tot wie ze,;}{baden rook naar kamfer en}{zat \`onder de stof}\\

\haiku{{\textquoteright} - {\textquoteleft}Het kan me schelen.}{omdat de banden van het}{bloed ons verbinden}\\

\haiku{Je hebt geen enkel,,.}{recht want ik heb niets van je}{en niets van je noodig}\\

\haiku{En dat is hoog noodig,,.}{want veel soeps is het niet wat}{je om je heen ziet}\\

\haiku{wij beiden hierom,.}{en veroordeelden het als}{zeer onwelvoeglijk}\\

\haiku{Het is onzin om.}{in belangrijke dingen}{niet rechtuit te zijn}\\

\haiku{{\textquoteright} - {\textquoteleft}Er zijn menschen die;}{spoorwegen bouwen uit een}{soort van amusement}\\

\haiku{Het hoogste wat wij.}{bereiken kunnen is het}{weten hoe ze zijn}\\

\haiku{Maar het zal eerder,.}{goed zijn dan slecht dat we van}{elkander houden}\\

\haiku{Hij wilde na een,;}{groet gewisseld te hebben}{haastig doorloopen}\\

\haiku{Wat kon een jonge,?}{onwijze knaap als Paco}{daarvan begrijpen}\\

\haiku{Maar een bepaalde,,?}{tactiek een bepaalde lijn}{geven ze die aan}\\

\haiku{{\textquoteleft}... beken jezelf tot...:}{het proletariaat durf}{te zijn wat je bent}\\

\haiku{Ze maken je zelf,.}{tot hun tegenstander die}{verdommelingen}\\

\haiku{aan menschenliefde.}{ontspruit onze haat tegen}{de onderdrukkers}\\

\haiku{Elk plantje moet zijn.}{plekje hebben en ieder}{beest zijn holletje}\\

\haiku{En opeens, als kwam,:}{hij eindelijk tot zichzelf}{zei hij resoluut}\\

\haiku{Koeprow woonde niet,.}{anders dan de anderen}{niet anders dan Jan}\\

\haiku{Koeprow had het ook:}{wel eens als een verwijt te}{hooren gekregen}\\

\haiku{De ander, gewend,;}{aan Koeprow's manier van doen}{doorstond ze rustig}\\

\haiku{Er waren een paar,.}{aardige kameraden}{daar anarchisten}\\

\haiku{Sinds ik partijlid,.}{ben praat ik immers niet meer}{over vrijbuiterij}\\

\haiku{Bolsjewisme moet,.}{je in het bloed zitten in}{het protoplasma}\\

\haiku{- {\textquoteleft}Dat beteekent koppen{\textquoteright},,.}{zei Iwan en hij maakte een}{gebaar naar zijn hals}\\

\haiku{In een millioen.}{jaren heeft de menschheid}{nog niets bijgeleerd}\\

\haiku{Hoe kon ze over hen.}{praten terwijl het verdriet}{nog zoo versch schrijnde}\\

\haiku{{\textquoteleft}Ook wij gelooven, om,{\textquoteright},.}{te kunnen vechten vader}{Rosenblum zei hij}\\

\haiku{Zijn verbeelding was.}{reeds bezig met de nieuwe}{taak die hem wachtte}\\

\haiku{{\textquoteright} - {\textquoteleft}We zijn met veel te.}{weinigen om nog hard van}{stapel te loopen}\\

\haiku{en dit zou dan de.}{plek wezen waarop Jan zijn}{aanval kon richten}\\

\haiku{van nabij had hij.}{nooit veel sterk effect gezien}{van al dat geschrijf}\\

\haiku{Een waarheid die geen,;}{geld oplevert is niets waard}{voor de krantenman}\\

\haiku{een luizenleger.}{van millioenen kleine}{gemeene leugens}\\

\haiku{{\textquoteleft}Je onderschrijft het,}{negatieve programma}{je doet mee zoolang}\\

\haiku{Onbewust merkte;}{ook Jan dit spiegelbeeld van}{zichzelf aan haar op}\\

\haiku{{\textquoteright} Jan had niet zooveel.}{vertrouwen in die klasse}{van medemenschen}\\

\haiku{Ik ben een man uit,?}{het publiek en wat is een}{krant zonder publiek}\\

\haiku{Waarschijnlijk hielden.}{ze er in het geheel geen}{theorie op na}\\

\haiku{Wat wij nastreven.}{moet in de eerste plaats een}{mogelijkheid zijn}\\

\haiku{- {\textquoteleft}Wat men ze geven,,:}{moest inplaats van geld is een}{briefje waarop staat}\\

\haiku{Deze Hollandsche?}{episode zou dus nu reeds}{afgeloopen zijn}\\

\haiku{{\textquoteleft}Luister  eens, je.}{hebt die zoon van mij een heel}{eind ver gekregen}\\

\haiku{{\textquoteright} Toch klampt ze zich aan,.}{de mogelijkheid van dit}{toeval vast dacht Jan}\\

\haiku{Ongenaakbaar voor,.}{al het kleine gemeene}{en vuile rondom}\\

\haiku{Reeds lang deden ze;}{daar hun best de koning een}{beentje te lichten}\\

\haiku{Als je hard ging, liep;}{je vast tusschen de dijken}{en de duinen hier}\\

\haiku{Men komt zelden in,.}{het kamertje waar men niet}{van oudsher thuishoort}\\

\haiku{{\textquoteright} vroeg Jan zuurzoet en.}{in de parabel-stijl}{die hier mode was}\\

\haiku{{\textquoteright} - {\textquoteleft}En dacht je soms dat?}{ze in werkelijkheid niet}{op elkaar leken}\\

\haiku{Anders sukkelen.}{we in enkele maanden}{tien jaar achteruit}\\

\haiku{ze schreef dat zij met;}{haar handelsdelegatie}{naar Parijs zou gaan}\\

\haiku{Waarom zijn jullie?}{noorderlingen toch zoo van}{vrouwen geschrokken}\\

\haiku{{\textquoteright} - {\textquoteleft}Zelfs nu behoeft het{\textquoteright},.}{er niet te zijn zei Jan een}{beetje achteloos}\\

\haiku{{\textquoteright} - {\textquoteleft}Dat is het juist{\textquoteright}, zei.}{Gil terwijl hij een gebaar}{van wanhoop maakte}\\

\haiku{Zon, palmen, felle,,...}{kleuren eindelooze ruimten}{van land zee en lucht}\\

\haiku{hoe anders dan ik.}{denk zou de werkelijkheid}{misschien geweest zijn}\\

\haiku{{\textquoteright} - {\textquoteleft}Zoolang ik je ken,.}{is het of je zoekt naar iets}{wat onvindbaar blijft}\\

\haiku{Daarom begon Jan,:}{maar te vertellen hoe het}{alles geweest was}\\

\haiku{zelfs de gitaar was.}{in zijn onzichtbare tuin}{in slaap gevallen}\\

\haiku{met behulp van de.}{arbeid van een paar honderd}{arme inlanders}\\

\haiku{In Engeland, of,.}{als je met alle geweld}{zou willen ook hier}\\

\haiku{Jan begon driftig.}{op en neer te loopen toen}{de ander weg was}\\

\haiku{en altijd bleek dat.}{je in je wezenlijkste}{verlangens faalde}\\

\haiku{En nu vooral, nu.}{ik aan zooveel andere}{dingen moet denken}\\

\haiku{Hoe zeer moest ze hem.}{dan niet aan zich gebonden}{en geknecht hebben}\\

\haiku{Het zijn mystici, -.}{of desperado's of}{beide tegelijk}\\

\haiku{{\textquoteright} - {\textquoteleft}Je liefde voor de.}{partij was dus grooter dan}{je liefde voor Jan}\\

\haiku{Ze vond het gek van.}{zichzelf tot deze slotsom}{gekomen te zijn}\\

\haiku{Tot zoo lang is het.}{beter al die dingen uit}{de weg te blijven}\\

\haiku{Je hebt nog liever.}{die hut stukgemaakt en ons}{plezier bedorven}\\

\haiku{Anders sterven we.}{aan het gevaar van onze}{eigen wapenen}\\

\haiku{{\textquoteright} Van wie het nieuwe,.}{plan afkomstig was kon hij}{moeilijk  zeggen}\\

\haiku{Hij heeft er zich dood,.}{op geknutseld maar drie van}{die dingen zijn klaar}\\

\haiku{lieden die zich een.}{geluk droomen dat niet van}{deze wereld is}\\

\haiku{{\textquoteright} vroeg Jan, zonder nog.}{zijn blikken af te wenden}{van de zoldering}\\

\haiku{Jan verdedigen,.}{wilt ook terwijl hij niet meer}{verdedigbaar is}\\

\haiku{hem, terwille van,.}{jou zal ik hem verdragen}{wat er ook gebeurt}\\

\haiku{De rest komt er niet,.}{op aan een menschenleven}{komt er niet op aan}\\

\haiku{Alles heeft dwaze,.}{namen de eene is zoo slecht}{als de andere}\\

\haiku{Maar nu, op deze.}{eigen oogenblikken moest}{het toch gebeuren}\\

\haiku{Zelfs met de wind mee.}{was dat onmogelijk in}{deze drukke stad}\\

\haiku{Dan drukte hij, nog,}{steeds de linker hand in zijn}{zak de knop zoo diep}\\

\haiku{Hoe was dan z\'o\'o snel?}{de revolver uit zijn zak}{omhoog gekomen}\\

\haiku{Waren nu reeds uit?}{die massa enkelen op}{hem losgelaten}\\

\haiku{Dan keek hij overal,;}{om zich heen er was niemand}{in de nabijheid}\\

\haiku{Hard trok hij de deur,.}{achter zich dicht dat het lang}{na-dreunde in huis}\\

\haiku{Als ik vooreerst maar,.}{wegkom uit Madrid dan zie}{ik wel weer verder}\\

\haiku{dat is het eenige:}{wat je te doen hebt als vrij}{en redelijk mensch}\\

\haiku{Over de vlakke weg,.}{die nu volgde hernam hij}{het dolle tempo}\\

\haiku{Alle andere.}{zijn verloren droomen of}{nog ongeboren}\\

\haiku{wat lang voor de eerst.}{bekende geschiedenis}{op aarde plaatsgreep}\\

\haiku{Straks zou het alweer,.}{te laat zijn dan kwam al het}{andere terug}\\

\haiku{En ze durfde nu.}{niet te telefoneeren met}{Manuel's woning}\\

\haiku{{\textquoteright} - {\textquoteleft}In naam van vand\'a\'ag zijn.}{er offers die we met recht}{mogen weigeren}\\

\haiku{Rien had niet meer de;}{moed om aan te dringen dat}{Mirjam heen zou gaan}\\

\haiku{{\textquoteleft}Te wonen in wat,.}{er buiten ligt zal meer pijn}{doen dan hier te zijn}\\

\haiku{Maar zijn werk was er,,?}{toch nog zijn idee het beste}{en edelste aan hem}\\

\haiku{Alleen, voor haar was;}{hij nooit enkel de drager}{van een idee geweest}\\

\haiku{D\`an zal ik danken,:}{wanneer er dank is verdiend}{en ik zeggen kan}\\

\haiku{{\textquoteright} - {\textquoteleft}Dat wat wij toeval,.}{noemen is misschien juist de}{zin van ons leven}\\

\haiku{Elk had zijn deel in,.}{het werk in het slagen en}{in de mislukking}\\

\haiku{{\textquoteright} Rien ving zijn gezicht.}{binnen de glansboog van haar}{groote blauwe blikken}\\

\haiku{Ik moet hem toch zijn;}{plaats en zijn kans geven in}{dit ondermaansche}\\

\haiku{zoo was zijn aard, dat.}{had hij noodig om zichzelf te}{verwezenlijken}\\

\haiku{zulke vrouwen moest...}{ik in mijn leven nog eens}{kunnen ontmoeten}\\

\haiku{Vroeger ben ik vaak}{terneergeslagen geweest}{omdat ik bang was}\\

\haiku{Het gaf haar iets van.}{een boerin en iets van een}{wild ongetemd dier}\\

\haiku{Het is een soort van.}{idealisme waarvoor ik}{geen begrip meer heb}\\

\haiku{moeten wij ze ook,,?}{nog op papier in verf in}{marmer namaken}\\

\haiku{En dan moet ik die,.}{droom maar weer opschrijven om}{te kunnen leven}\\

\haiku{Als ik leerlooier.}{was of landlooper zou ik}{het \'o\'ok weigeren}\\

\haiku{Het eenige wat wij,.}{meenemen op de vlucht is}{onze oprechtheid}\\

\haiku{Elk besef van tijd,.}{heeft opgehouden er is}{slechts aanwezigheid}\\

\haiku{met het spelletje.}{van de conjecturen was}{het evenzoo gesteld}\\

\haiku{Het was altijd heel,,.}{nabij geweest en toch ik}{heb het niet herkend}\\

\haiku{En bij haar laatste,,:}{zwijgen nadat ook dit was}{gezegd  dacht ik}\\

\haiku{je denkt aan afscheid,;}{ouderdom en doodgaan met}{een voldaan gevoel}\\

\haiku{wat was er anders,?}{noodig om gelukkig dankbaar}{en voldaan te zijn}\\

\haiku{mijn bizonder deel.}{van uitgestooten-zijn}{dat ik moet dragen}\\

\haiku{Toen ik haar begon,,.}{te volgen werd ze grooter}{groeide tot een berg}\\

\haiku{Zoolang... zoolang... tot,.}{later het geschikte uur}{zou komen later}\\

\haiku{En dan, laat ik je.}{n\`ogmaals herinneren aan}{Robinson Cruso\"e}\\

\subsection{Uit: Waar is Vrijdag gebleven?}

\haiku{Een ernstig conflict.}{tussen de naburen was}{onvermijdelijk}\\

\haiku{De juiste toedracht.}{van zaken valt moeilijk meer}{te achterhalen}\\

\haiku{en als men hem niet, (...).}{had laten gaan zou hij in}{zee gesprongen zijn}\\

\haiku{het achterwege,{\textquoteright}.}{te laten ofschoon hij het}{ten slotte wel deed}\\

\haiku{{\textquoteleft}in vrede{\textquoteright} zijn de.}{twee allerlaatste woorden}{in Crusoe II}\\

\subsection{Uit: Zomaar wat kinderen}

\haiku{de mooie nieuwe fiets.}{van meester Tjon Sie Kwie te}{mogen schoonpoetsen}\\

\haiku{{\textquoteright} Het was laat in de.}{namiddag eer zij terug}{in de stad waren}\\

\haiku{Ik zie je nooit iets,.}{lezen je doet alsof er}{geen boeken bestaan}\\

\haiku{Zelfs al had Pa {\textquoteleft}met{\textquoteright},.}{geen sterveling gezegd op}{zo'n dreigende toon}\\

\haiku{Ik zal je maar niet,.}{zeggen hoe want dat kan je}{toch niet begrijpen}\\

\haiku{{\textquoteleft}Zet ze maar achter!}{het huis in die baskiet met}{gebroken kleren}\\

\haiku{{\textquoteleft}Al zit je vol, je,?}{kunt toch wel een plaatsje voor}{hem inruimen h\`e}\\

\haiku{de meesten echter,.}{liepen onverschillig al}{pratend lang hem heen}\\

\haiku{Of zoals je dat {\textquoteleft}{\textquoteright}.}{zelfs bijhalve-Chinezen}{onmiddellijk zag}\\

\haiku{{\textquoteleft}Er zijn in onze.}{groep ook drie of vier Turkse}{jongens en meisjes}\\

\haiku{Nou, met zo iemand,.}{kon je toch nooit omgaan waar}{hij ook vandaan kwam}\\

\haiku{zo lekker smaakte.}{hem deze schotel nog in}{zijn herinnering}\\

\haiku{Anders verlies ik,?}{misschien deze baan en wat}{beginnen we dan}\\

\haiku{Dit moest sneeuw zijn, wist,.}{hij maar nog niet hoe het er}{van nabij uitzag}\\

\haiku{Van Annemiek die {\textquoteleft}!}{nog steeds met hem aanpapte}{en ook altijdDoei}\\

\haiku{Misschien kom ik nog,}{wel eens kijken bij jullie}{in Suriname}\\

\subsection{Uit: Zuid-Zuid-West}

\haiku{eenzaam te zijn, want.}{alleen de eenzame geeft}{acht op de stilte}\\

\haiku{Arme Inca, die.}{vijftien eeuwen wachten moest}{voor het stalleke}\\

\haiku{Geeft rekenschap van,.}{uw rentmeesterschap gij die}{het goud begeerd hebt}\\

\haiku{Zijn ogen spiegelen;}{zich in het kalme zwarte}{water der kreken}\\

\haiku{Zij leggen die over,.}{elkaar en weten niet dat}{het steeds een kruis wordt}\\

\haiku{Het hart van dit land;}{is een stille kern waar geen}{geluid meer kan zijn}\\

\haiku{{\textquoteright} En in zijn blijde.}{verwondering citeerde}{hij ganse verzen}\\

\haiku{Na een paar uren kwam.}{de heerlijke sensatie}{van het uitstappen}\\

\haiku{Geen van de planten,.}{die groeien in dit zand zijn}{te zien in de stad}\\

\haiku{De verpoeierde.}{regen was zilver in de}{ongeboren dag}\\

\haiku{De dag is een luie,{\textquoteright}.}{en trage mulat zei hij}{tegen de bomen}\\

\haiku{hoe alles groeide,,.}{hoe groot het land was dat nu}{weer ontgonnen werd}\\

\haiku{De zon scheen de glans.}{van versgebakken brood over}{zijn kop en handen}\\

\haiku{Nu is het als een,,.}{wijde vijver waar ik wacht}{in een schaduwplek}\\

\haiku{Dan zag je de nacht,.}{door het venster als een heel}{lelijk zwart gezicht}\\

\haiku{terwijl je wachten.}{moest onder een balkon werd}{ze zelfs feestelijk}\\

\haiku{Maar binnen in het.}{gesloten huis was alles}{helemaal anders}\\

\haiku{Ik bid u, denk met.}{een liefdevol hart aan de}{jeugd van Maldoror}\\

\haiku{ik herinner mij.}{eigenlijk geen enkel feest}{in ons oude huis}\\

\haiku{Dit alles hangt heel,,.}{nauw samen met het huis met}{de stad met het land}\\

\haiku{Wie bevreesd is, kan.}{nimmer het binnenste van}{mijn land betreden}\\

\haiku{Uitgeteerd keren,.}{ze stadwaarts en hun schat is}{na \'e\'en dag een droom}\\

\haiku{Onontkoombaar zijn.}{wij ingesloten in zulk}{een benauwenis}\\

\haiku{{\textquoteleft}Op de toppen des}{levens ben ik gestegen}{om u te zingen}\\

\haiku{Een fijngekauwde;}{wortel spuwen zij allen}{in de lege boot}\\

\haiku{De donder verliest.}{zich in deze ruimte tot}{een klein gemorrel}\\

\haiku{Uit zijn schuilplaats komt,,.}{een tijger te voorschijn en}{gluurt naar links naar rechts}\\

\haiku{Ik schoof de grijsaard.}{weg van de piano en}{speelde de tango}\\

\haiku{En een kind dat te,.}{jong op reis gaat komt soms te}{zeer als man terug}\\

\haiku{De stad beweegt zich,.}{om een lichaam de zee woelt}{rond een dood lichaam}\\

\haiku{Niet in de landen,.}{zijn wij vreemdelingen maar}{in elkanders droom}\\

\haiku{Kropina-kreek,.}{een kleine zijtak van de}{Para-rivier}\\

\haiku{Hij schreef ook enige.}{merkwaardige boeken tot}{hun verdediging}\\

\subsection{Uit: Zusters van liefde}

\haiku{het was niet besteed,....}{aan zo'n wezen geen parels}{voor zulke nou ja}\\

\haiku{Roerloos, ongestoord.}{door het binnenkomen van}{de twee bezoekers}\\

\haiku{Isolatie in een.}{afzonderlijke barak}{was onafwendbaar}\\

\haiku{Toen al waren zijn,.}{gedachten bij dat meisje}{maar hij vroeg nog niets}\\

\haiku{Hij merkte dat haar,,.}{haren gitzwarte nog niet}{waren afgeknipt}\\

\haiku{Want inderdaad moesten,.}{dat er meerdere geweest}{zijn stelde men vast}\\

\haiku{Maar t\'och, ook zo in.}{zijn eentje doet een jong mens}{veel ervaring op}\\

\haiku{Je ziet het ze al:}{schrijven en je hoort het de}{mensen al zeggen}\\

\haiku{Vertrouwensman van,.}{de Cristero's of de hemel}{mag weten wat meer}\\

\haiku{haar werkzaamheden,}{kon hij niet beoordelen}{die kende zij zelf}\\

\haiku{Daarom heb ik je?}{immers aangeraden je}{daar te vestigen}\\

\haiku{Een staat binnen de.}{staat is onduldbaar volgens}{elke logica}\\

\haiku{Hij is moeilijk voor....}{iedereen en eigenlijk}{mag je niet klagen}\\

\haiku{Hoe dikwijls ben ik,{\textquoteright}.}{zelf niet opnieuw begonnen}{troostte Amaral hem}\\

\haiku{{\textquoteleft}Nou, de afloop van.}{deze boevenstreek kunt u}{zich wel voorstellen}\\

\haiku{Zonder veel geestdrift,.}{maar ook zonder merkbare}{onvriendelijkheid}\\

\haiku{Anders zat je hier.}{nu niet te luisteren naar}{mijn gelamenteer}\\

\haiku{Maar ondertussen,.}{zit uw vrouw in spanning te}{wachten Manolo}\\

\haiku{Je ziet, het zijn en.}{blijven toch heidenen in}{hun diepste wezen}\\

\haiku{op hun manier het.}{dodenfeest vieren voor de}{gestorven Christus}\\

\haiku{nieuw inheems broedsel,....}{zoals ik zelf en Lino}{en wie je maar wilt}\\

\haiku{Wat zij te doen had,;}{was zichzelf en ook Lino}{de tijd te gunnen}\\

\haiku{De wijsheid woont bij,.}{het simpele volk niet bij}{de professoren}\\

\haiku{Daar was het mij \'o\'ok,.}{om begonnen hoewel niet}{op de eerste plaats}\\

\haiku{degene die hij,?}{naar de kliniek gebracht had}{waaruit zij vluchtte}\\

\haiku{Want de landmeter.}{toonde zich uitermate}{nerveus en beangst}\\

\haiku{En om zijn vriend in,:}{dezelfde trant van repliek}{te dienen zei hij}\\

\haiku{Wat natuurlijk nooit.}{gebeurt als deze dingen}{hier blijven doorgaan}\\

\haiku{{\textquoteleft}Laten we in 's;}{hemelsnaam ophouden over}{dat nonnengedoe}\\

\haiku{Iets verder hingen:}{ook allerlei andere}{folterwerktuigen}\\

\haiku{{\textquoteleft}Maar wat versta je?}{dan onder het vinden van}{een veilige weg}\\

\haiku{Hun gesprek stokte.}{dan ook verder en liep even}{later ten einde}\\

\haiku{Hopelijk zou hij;}{Isidro daar zonder veel moeite}{weten te vinden}\\

\haiku{Wie het land bebouwt,.}{weet er alles van en houdt}{er rekening mee}\\

\haiku{Maar mensen uit de.}{stad zijn alles vergeten}{wat van belang is}\\

\haiku{{\textquoteleft}We kwamen er pas,{\textquoteright}.}{achter door een stom toeval}{ging Isidro opeens voort}\\

\haiku{Wat tegenwoordig,.}{verboden was behalve}{voor de navelstreng}\\

\haiku{Toch wil ik er nog,{\textquoteright}, {\textquoteleft}.}{weieens naar toe zei Hector}{maar dat heeft de tijd}\\

\haiku{Hij deed zijn zegje,,{\textquoteright}.}{hard en duidelijk dat vind}{ik leuk sprak Chole}\\

\haiku{H\'a\'ar scheppingsdrang is.}{een andere dan die van}{de meeste mannen}\\

\haiku{Maar die andere.}{naam heeft de zuigelingen}{geen geluk gebracht}\\

\haiku{{\textquoteleft}Wij houden onze,.}{rancho's opzettelijk klein}{zoals ze nu zijn}\\

\haiku{Nog maar twee of drie.}{zittingen schatte hij en}{het zou voltooid zijn}\\

\haiku{Maar hoe Chole dit.}{alles zo aanstonds opneemt}{is mijn grootste zorg}\\

\haiku{De kamer met het;}{beste licht bestemde hij}{meteen tot atelier}\\

\haiku{Daarenboven was.}{zij hier in dit gehucht van}{onschatbaar veel nut}\\

\haiku{oudere vrouw was.}{al opgestaan toen hij met}{zijn relaas begon}\\

\haiku{{\textquoteleft}Se\~nor,{\textquoteright} zei ze, nu, {\textquoteleft}.}{bijna fluisterendik moet}{u nog wat zeggen}\\

\haiku{{\textquoteright} Na hem even zwijgend,:}{te hebben aangezien ging}{Leocadia voort}\\

\haiku{En misschien zou hij,,.}{haar als het meezat toch nog}{kunnen bepraten}\\

\haiku{De enige kerk van,.}{de geheel verlaten plaats}{zoals hij merkte}\\

\haiku{ook het kerkhof waar.}{al tientallen lichtjes bij}{de graven brandden}\\

\haiku{Zo lang zij hem nog,.}{dienen kon vormden zij een}{ware twee\"eenheid}\\

\haiku{Hij voelde zich niet.}{al te zeer bezwaard door haar}{aanhankelijkheid}\\

\haiku{{\textquoteright} veinsde Hector met.}{het onschuldigste gezicht}{dat hij kon trekken}\\

\haiku{Wat niet wegnam dat.}{er soms heel vervelende}{dingen gebeurden}\\

\haiku{Hij was er slecht aan,}{toe en is naar het grote}{hospitaal vervoerd}\\

\haiku{En schenk ons een goed,...{\textquoteright} {\textquoteleft}?}{leven een nuttig bestaan}{Is dat niet prachtig}\\

\haiku{Nu moest ze gaan, er,...}{wachtten zoveel anderen}{zoals Ena wel wist}\\

\haiku{Ofschoon hij in zijn.}{eentje al moeite genoeg}{had rond te komen}\\

\haiku{{\textquoteleft}Laat toch eens zien wat.}{je hier zoal gemaakt hebt}{in deze chaos}\\

\haiku{Waarop Laurette,:}{alsof zij hem begreep en}{zich excuseerde}\\

\haiku{hoe hij dit doen kon:}{en keek verrast op toen zijn}{vrouw plotseling zei}\\

\haiku{{\textquoteleft}Het is hier niet groot.}{en er is niets anders in}{dit huis beschikbaar}\\

\haiku{Samen gingen zij;}{hun eerste inkopen doen}{in de stadsdrukte}\\

\haiku{Maar met haar zuiver...}{instinct stelde zij immers}{zuivere daden}\\

\haiku{Ik zou hier misschien...{\textquoteright}.}{Ena's ogen keken de kant uit}{van het lazaret}\\

\haiku{Zij behielden hun.}{waardigheid totdat zij uit}{het gezicht waren}\\

\haiku{Een vreemde dame,...}{eigenlijk maar ijverig}{genoeg en nuttig}\\

\haiku{Ze begrijpen niet.}{dat de natuur veel helpers}{voor ons heeft klaarstaan}\\

\haiku{Het leek wel alsof,.}{hij altijd iets in het schild}{voerde jazeker}\\

\haiku{Intussen steeg de}{drukte op het marktplein waar}{tientallen kraampjes}\\

\haiku{Alleen wat de Kerk.}{aangaat geef ik toe dat je}{gelijk kunt hebben}\\

\haiku{Ik heb vanavond nog,,}{veel te doen maar moet jullie}{dringend iets melden}\\

\haiku{Dit was immers iets.}{van nog veel groter belang}{dan een Sint-Jansfeest}\\

\haiku{Of hij soms meende?}{dat men op een feestelijk}{pleziertochtje ging}\\

\haiku{Niet allen echter.}{genoten toen al van zo'n}{vreedzame sluimer}\\

\haiku{Wat moest hij van haar,?}{nu hij immers wist dat zij}{toch maar een vrouw was}\\

\haiku{Ik heb het niet, heb.}{het nooit zo maar gegeven}{aan een gezonde}\\

\haiku{Zie dan maar verder.}{te komen op deze reis}{die nog weken duurt}\\

\haiku{als zij zich al te.}{hoog en vermetel in zijn}{nabijheid wagen}\\

\haiku{En don Benito,.}{was toch de slimste al was}{hij maar een Indio}\\

\haiku{zijn tweekleurige.}{glazige blik groot en star}{op haar gevestigd}\\

\haiku{Luider dan eerst ging}{hij voort met haar te zeggen}{dat hij gezien had}\\

\haiku{Hij is slim, mij te.}{slim af moest Urbina L\'opez}{zichzelf bekennen}\\

\haiku{{\textquoteleft}Hij dacht zeker er,.}{naar toe te kunnen vliegen}{met al zijn grootspraak}\\

\haiku{Ik geloof dat het.}{waar is wat er in dat boek}{van mijn maestro stond}\\

\section{Albert Helman en Albert Kuyle}

\subsection{Uit: Van pij en burnous}

\haiku{de dingen met vijf,.}{zinnen te benaderen}{en niet met het bloed}\\

\haiku{Overal te zoeken.}{naar de \'e\'ene weg die nog}{onversperd moet zijn}\\

\haiku{Aan niets wordt zooveel;}{zekerheid geofferd als}{aan nieuwsgierigheid}\\

\haiku{Hoor nu het nacht is,.}{hoe dit zwarte water klotst}{tegen den steiger}\\

\haiku{Met een vaartje neem ik,.}{de hoeken en kom buiten}{adem in de dorpsstraat}\\

\haiku{Het hindert weinig.}{op welken tijd van het jaar}{gij in Umbri\"e komt}\\

\haiku{Maar Umbri\"e ligt van.}{de verdere wereld ook}{zoo afgezonderd}\\

\haiku{uit alle stemmen;}{klokte een donkere roep}{naar nieuwe broeders}\\

\haiku{Zij slaan hun handen,.}{tegen de zwarte muur die}{korrelig aanvoelt}\\

\haiku{Een nieuwe omdraai.}{geeft uitzicht op de heele}{vlakte van Umbri\"e}\\

\haiku{Totdat om de hoek}{van de straat het heldere}{zingen van water}\\

\haiku{En weer spiegelt het,,,.}{water de tijd de liefde}{de Gerechtigheid}\\

\haiku{De stad van Petrus,}{die niet warm voor u open ligt}{als ge komt van ver}\\

\haiku{Wij wachten op Sint,.}{Petrus die zegenend ons}{allen voorbijtrekt}\\

\haiku{Hij zegent allen.}{met hostie en kelk die hij}{in zijn handen draagt}\\

\haiku{Zijn woorden slaan neer.}{als zware keien voor de}{voeten van Arius}\\

\haiku{De abt is een groote,.}{rijzige man als hij daar}{voor zijn zetel staat}\\

\haiku{De menschen loopen,.}{er heen en weer de jongens}{ravotten er wat}\\

\haiku{Wie kent Amsterdam?}{als hij noch de Jordaan noch}{de Kolk gezien heeft}\\

\haiku{Verderop voor een.}{der huisjes zit een oude}{man op een bankje}\\

\haiku{Ze reizen langs de.}{vooruitgeschoven posten}{van het moederland}\\

\haiku{De dienstgang is vol}{leven en vlak achter een}{troepje matrozen}\\

\haiku{Salammb\^o echter {\textquoteleft}{\textquoteright};}{bleef voor mij het maximum van}{wat wijstijl noemen}\\

\haiku{De tambourijn schuift,.}{schelle ronde vlakjes af}{als email dat schilfert}\\

\haiku{Een slank, fransch meisje.}{schenkt ons den gloeiend-rooden wijn}{van M\'egrine in}\\

\haiku{Warme bronnen met ',.}{chloorzure soda int}{water of zooiets}\\

\haiku{de wegen waarlangs.}{hij ons voert naar het feest der}{Verrijzenis}\\

\haiku{Door de open deur zie.}{je een stuk mimosa-laantje}{als een schilderij}\\

\haiku{Arme Vincent; ik,.}{bid voor hem en voor alle}{schilders die ik ken}\\

\haiku{Het is oud als een.}{bedoe{\"\i}nentent en jong}{als deze lente}\\

\haiku{Met gorgel- en.}{sis-geluiden leidde}{hij de kameelen}\\

\haiku{Dan verdwijnt alles.}{in het trillend zoemen van}{bewegende lucht}\\

\haiku{Hij keek op naar het,.}{vroom gelaat den vollen baard}{en de wijze oogen}\\

\haiku{Het jonge hoofd naast.}{hem boog zich nog dieper naar}{den gelen stofweg}\\

\haiku{De bedoe{\"\i}nen.}{praten fluisterend met scherp}{schrapen van de keel}\\

\haiku{Want de tapijten.}{zijn hier de rijkdom en de}{welvaart van de stad}\\

\section{Emiel van Hemeldonck}

\subsection{Uit: Agnes}

\haiku{Lange uren, waar geen,;}{eind aan scheen te komen en}{ik kon niet klagen}\\

\haiku{{\textquoteleft}Ziet ge, het zal niet!}{lang duren of hij is een}{eerste klas koetsier}\\

\haiku{En ook moeder moet,.}{het gehoord hebben want zij}{antwoordde niet meer}\\

\haiku{En het meisje met.}{het stroohoedje en de blauwe}{korenbloemen}\\

\haiku{Dat ik een haas kon,.}{verrassen snoek stroppen en}{een eekhoorn vangen}\\

\haiku{Geluk er mee, en...{\textquoteright}}{leer het later zorgvuldig}{aan uw kinderen}\\

\haiku{Een magere, hooge;}{stem riep uit de verte en}{het meisje verschrok}\\

\haiku{Ik voelde de oogen,.}{van het meisje op mij maar}{ik vond geen woorden}\\

\haiku{Hij knikte, en weer,.}{lag er spot in zijn stem maar}{het deed mij geen pijn}\\

\haiku{Het was of ik uit,.}{een tooverwereld kwam uit een}{vreemden droom ontwaakt}\\

\haiku{In den stal stampten,.}{de paarden het dreunde hol}{in het groote gebouw}\\

\haiku{Tist betrapte mij.}{bij dit werk en het was hem}{niet onaangenaam}\\

\haiku{in het licht van de.}{kaars danste zijn grillige}{schaduw op den muur}\\

\haiku{De klauw van zijn hand.}{greep naar de kaarsvlam en het}{was donkere nacht}\\

\haiku{ik leefde feller,.}{heviger dan ik het tot}{nog toe gedaan had}\\

\haiku{Er was geen tijd meer,,.}{dorpen hadden geen naam er}{was geen herkennen}\\

\haiku{maar er was iets in,,.}{zijn blik houding en gebaar}{dat ik niet kende}\\

\haiku{Het was als een even,.}{afgebroken gesprek dat}{hij thans voortzette}\\

\haiku{maar dan verstarde:}{zijn glimlach tot een grimas}{en hij fluisterde}\\

\haiku{met moeite kon ik,.}{in de maat van zijn tragen}{zwaren stap blijven}\\

\haiku{{\textquoteleft}Er komt onweer,{\textquoteright} zei,.}{hij en het was alsof hij}{van man tot man sprak}\\

\haiku{Dan voelde ik het.}{licht verzwakken en opende}{voorzichtig de oogen}\\

\haiku{Het geldstuk brandde.}{in mijn hand. Als een golf sloeg}{de woede over mij}\\

\haiku{Zijn magere hand.}{rilde op den dikken knop}{van zijn wandelstok}\\

\haiku{Ik had het vroeger,;}{wel gezien maar er nooit veel}{aandacht aan besteed}\\

\haiku{Op de werktafel:}{stond een bronzen beeldje op}{marmeren voetstuk}\\

\haiku{ik wist alleen dat,.}{ik niet buigen zou wat er}{ook mocht geschieden}\\

\haiku{Toen  hoorde ik,.}{den naderenden stap van}{Ben mijn oudsten broer}\\

\haiku{{\textquoteright} Wanneer ik naar hem,.}{opkeek zag ik dat hij de}{schouders ophaalde}\\

\haiku{Wij hadden tegen,;}{het bosch aan een plek liggen}{die moerassig was}\\

\haiku{Ik begreep en schreef,.}{ook de enkele woorden}{die hij dicteerde}\\

\haiku{Achter haar beklom,,.}{ik de smalle trap \'e\'en twee}{verdiepingen hoog}\\

\haiku{Schouderophalend,}{keek zij mij aan en op dit}{oogenblik hoorden}\\

\haiku{de geboorte van.}{het  eenige kind had haar}{het leven gekost}\\

\haiku{Louis is aan 't werk,,...{\textquoteright}}{en zoodra hij verlof}{kan krijgen komt hij}\\

\haiku{Wanneer ik aan de,.}{deur stond zag ik dat hij nog}{iets zeggen wilde}\\

\haiku{In kladden drijven,;}{de vogels over de wijde}{rustende akkers}\\

\haiku{De stilten tusschen.}{ons waren vaak langdurig}{en soms ondraaglijk}\\

\haiku{En ik voelde haar,.}{tegenwoordigheid al zag}{noch hoorde ik haar}\\

\haiku{De buiging die hij,.}{maakte kwam mij houterig}{en potsierlijk voor}\\

\haiku{Met zeer gemengde.}{gevoelens bracht ik den avond}{op mijn kamer door}\\

\haiku{Hij groette verstrooid,.}{keek even naar de pakkage}{en nam zijn plaats in}\\

\haiku{{\textquoteright} Plots kwam hij terug,:}{bij de auto greep een paar}{valiezen en vroeg}\\

\haiku{Ik moest verhalen,.}{en de vinnige vragen}{doorflitsten mijn relaas}\\

\haiku{maar meer dan deze,.}{woorden trof mij de wijze}{waarop hij ze sprak}\\

\haiku{Over de binnenkoer,.}{kwam Jos\'e gegaan den arm}{reikend aan Spitsmuis}\\

\haiku{Dit alles kwam mij,.}{zoo raadselachtig voor dat}{het mij verwarde}\\

\haiku{Doelloos liep ik door,.}{de straten alleen om den}{tijd om te krijgen}\\

\haiku{zonder mij nog \'e\'en,.}{woord te gunnen zette zij}{het op een loopen}\\

\haiku{Onzeker was zijn,:}{blik en vaag het gebaar van}{zijn hand als hij zei}\\

\haiku{De geur van dieren,.}{gestapeld hooi en versch groen}{vergezelde mij}\\

\haiku{Dit woord greep mij aan,.}{geen schooner lof had mij dieper}{kunnen ontroeren}\\

\haiku{Agnes zal maar eerst,...}{dezen avond komen tenzij}{gij ze laat roepen}\\

\haiku{Alleen de jonker,.}{bewoog niet maar ik kon niet}{gelooven dat hij sliep}\\

\haiku{Een kleine mis, met.}{wat oude moederkens en}{een paar kinderen}\\

\haiku{De grafmaker keek.}{op toen ik bleef staan tot hij}{den kuil gevuld had}\\

\haiku{Zij keek naar mij op.}{en ik meende argwaan in}{haar oogen te lezen}\\

\haiku{in den koelen avond.}{werden de slachtoffers naar}{het gasthuis gebracht}\\

\haiku{Ik gehoorzaamde.}{werktuiglijk en de auto}{schoot den zijweg in}\\

\haiku{Boven een donker;}{massief rees de scherpe spits}{van een kerktoren}\\

\haiku{Na een wijl zette,.}{hij zich in het gras leunend}{tegen den schuurmuur}\\

\haiku{{\textquoteleft}Ik heb hem bij ons,.}{gehaald wanneer hij gered}{moest worden diefstal}\\

\haiku{De auto schoot in,;}{gang ik sloeg den eersten den}{besten weg  in}\\

\haiku{Ik schoof dichter naar,.}{haar toe en nam haar hand die}{zij niet terugtrok}\\

\haiku{Beter en dieper,.}{dan ik dit ooit gevoeld had}{wist ik mij hier thuis}\\

\haiku{Toen ik haar vroeg hoe,}{haar jonge meesteres het}{stelde schudde zij}\\

\haiku{Misschien had hij dit,:}{wel bemerkt want naar Spitsmuis}{omkijkend zei hij}\\

\haiku{Dan schudde zij het,.}{hoofd en ging zonder mij een}{antwoord te gunnen}\\

\haiku{Maar hij gebaarde.}{van mijn aanwezigheid niet}{meer bewust te zijn}\\

\haiku{{\textquoteright} Werktuiglijk nam ik,.}{het stuk in handen en plots}{doorvoer mij een schok}\\

\haiku{In de garage:}{wees ik den chauffeur naar het}{hoogste schap en zei}\\

\haiku{het was of zij mij,.}{iets zeggen wilde maar ik}{wachtte vruchteloos}\\

\haiku{Misschien was het wel,.}{goed dat ge intusschen naar}{iets anders uitzaagt}\\

\haiku{ik zag en hoorde,.}{alles maar de beteekenis}{drong niet tot mij door}\\

\haiku{Achter de tafel,,.}{zat rustig zijn pijp rookend}{een politieman}\\

\haiku{Honger naar kennis,,.}{honger naar macht en dit is}{zijn verderf geweest}\\

\haiku{Het verlangen dat,;}{mij verteerde heb ik niet}{kunnen verbergen}\\

\haiku{Door de wolken brak,.}{de bleeke zon en het schrille}{licht deed mijn oogen pijn}\\

\subsection{Uit: De harde weg}

\haiku{Zij antwoordde niet,.}{maar hij meende een glimlach}{gezien te hebben}\\

\haiku{{\textquoteleft}Niet waar, heer pastoor,?}{we krijgen wonderlijke}{dingen te hooren}\\

\haiku{{\textquoteright} Haar stem was z\'o\'o, dat.}{het bevel geen tweede maal}{moest herhaald worden}\\

\haiku{{\textquoteright} Haar spot was scherp en,,.}{neerhalend maar geen lach geen}{glimlach trad haar bij}\\

\haiku{In het rijtuig, dat,.}{zij zelf voerde zat de knaap}{Micha\"el naast haar}\\

\haiku{En dan bereikte.}{het magere geluid van}{een vedel haar oor}\\

\haiku{{\textquoteright} kraaide hij, maar dan.}{was zijn lied uit en hij kroop}{langs de banken weg}\\

\haiku{Er was geruisch.}{van kleeren en het geluid van}{een gedempten stap}\\

\haiku{Hij ging haar v\'o\'or, de,.}{trap af boog naar de deur van}{de gastenkamer}\\

\haiku{een baken voor het,.}{kleine werkzame volk van}{stad en platteland}\\

\haiku{En dit kind... {\textquoteleft}Er kan,.}{geen dringende haast bij zijn}{hoogedele Vrouwe}\\

\haiku{Buiten zijn uniform.}{was niets martiaals meer aan}{hem te bespeuren}\\

\haiku{Hij stond in de deur:}{van het prieeltje en riep}{de kinderen toe}\\

\haiku{fluweelen buis en.}{zilveren beengespen op}{de witte kousen}\\

\haiku{{\textquoteright} Door de groote poort, langs,,.}{de hal waar reeds wakend licht}{brandde naar de trap}\\

\haiku{{\textquoteright} {\textquoteleft}Cerberus die zijn,{\textquoteright}.}{schatten goed bewaakt spotte}{de provisor mild}\\

\haiku{de hooge muren, bij,.}{dit geluidloos sneeuwen dat}{de wereld afsloot}\\

\haiku{de zotte wind, die,.}{onder de sneeuwvlokken joeg}{wervelde ze mee}\\

\haiku{tegen de bermen.}{bij de poelen sterden de}{eerste anemonen}\\

\haiku{{\textquoteright} Vaag vermoedde hij,.}{wat van hem verlangd werd en}{hij knikte nogmaals}\\

\haiku{{\textquoteright} knikte zij spottend.}{en het was alsof zij hem}{als een vod wegsmeet}\\

\haiku{haar lippen werden,.}{scherp haar hand trilde wanneer}{zij het mes neerlei}\\

\haiku{Het kon de oude,:}{gewoonte zijn die plots in}{hem weer leven kreeg}\\

\haiku{{\textquoteleft}Gij hebt uw werk goed,,.}{verricht ma m\`ere en uw}{opzet is geslaagd}\\

\haiku{Nu moest zij niet meer;}{wachten op den koerier van}{Hare Majesteit}\\

\haiku{Het was niet noodig dat;}{zij het angstige meisje}{op den voet volgde}\\

\haiku{Wanneer zij stil hield,;}{op het voorplein brandde de}{kramp in hand en voet}\\

\haiku{De eenige weg, heeft...}{Hij gezegd die de waarheid}{is en het leven}\\

\haiku{Als gravin Martha,.}{omkeek zat pastoor Grang\'e}{daar als vernietigd}\\

\haiku{{\textquoteleft}Neen,{\textquoteright} schudde hij, als.}{de koetsier het portier van}{het rijtuig open wierp}\\

\haiku{Het rhythme van zijn,.}{tragen stap vertraagde nog}{hij moest zich voortsleepen}\\

\haiku{vaag schoof het landschap,,,.}{voorbij struikgewas boomen}{een boerenwoning}\\

\haiku{dat een gezonde.}{ziel alleen in een gezond}{lichaam kan wonen}\\

\haiku{{\textquoteright} Maar hij voelde den,:}{lichten hoon niet knikte naar}{Micha\"el en zei}\\

\haiku{Hij had tijd noch kans.}{om zich van zijn vermoeden}{te vergewissen}\\

\haiku{Hij had een zwaren,.}{nacht gehad in zijn gewond}{been knaagde de pijn}\\

\haiku{{\textquoteright} raadde Alexander, maar.}{zij gebaarde alsof zij}{hem niet gehoord had}\\

\haiku{Als de kamerdeur,.}{geopend werd spoelde het}{licht de gang binnen}\\

\haiku{Het afscheidnemen,.}{zou hem niet t\'e zwaar vallen}{zijn hart was niet hier}\\

\haiku{Binnen een kwartier,.}{wordt de klok geluid en dan}{schiet hij toch wakker}\\

\haiku{Hij vond een glimlach,,.}{derwijze berusting en}{schudde traag het hoofd}\\

\haiku{In de kamer naast.}{hem ging het geluid van oom}{Alexanders licht gesnork}\\

\haiku{{\textquoteright} zei hij, maar het klonk.}{of hij daar heelemaal niet}{zoo zeker van was}\\

\haiku{Maar Claire stond er,.}{op te vertrekken hoe ook}{aangedrongen werd}\\

\haiku{de vlakten plooiden.}{open onder het magere}{licht van den sneeuwnacht}\\

\haiku{{\textquoteright} vroeg hij, opkijkend.}{en haar het dichtgevouwen}{papier toereikend}\\

\haiku{Als er geluid van,:}{een stap klonk en Amelie bij}{het vuur stond zei hij}\\

\haiku{Wij kunnen naaien,, ',.}{en alst moet is er nog}{andere arbeid}\\

\haiku{{\textquoteright} De Bellincourt keek.}{hem scherp aan en zijn lachje}{was medelijdend}\\

\haiku{{\textquoteleft}Ik denk dat het niet.}{zoo moeilijk zou zijn om den}{ketting op te gaan}\\

\haiku{Hij riep haar, wees op,.}{het vertrappelde gras maar}{zij lachte spottend}\\

\haiku{{\textquoteright} Misschien had hij haar.}{niet verstaan of raakte haar}{dwaas bevel hem niet}\\

\haiku{Zonder eenig dankwoord,.}{nam zij den kroes aan en dronk}{zonder op te zien}\\

\haiku{De vraag verraste.}{hem en hij begreep niet wat}{zij van hem wilde}\\

\haiku{Dezen avond kunt gij,{\textquoteright}.}{te Loven zijn zei hij en}{Micha\"el begreep}\\

\haiku{Van der Noot maakte;}{zich driftig bij de lezing}{van de missive}\\

\haiku{{\textquoteleft}De Keizer toont eens.}{te meer dat hij het goed meent}{en het verkeerd doet}\\

\haiku{Tot mijn spijt zal ik.}{echter de plechtigheid niet}{kunnen bijwonen}\\

\haiku{In lange rijen;}{zaten de fraters aan de}{withouten tafels}\\

\haiku{Hij ging hen v\'o\'or, en.}{na hem volgde de lange}{rei hem naar de kerk}\\

\haiku{Langen tijd bleef hij,.}{haar achternastaren}{maar zij keek niet om}\\

\haiku{ergens weerklonk het.}{lustige geklepper van}{een watermolen}\\

\haiku{Zij keek naar haar zoon,:}{en er klonk fierheid in haar}{stem wanneer zei zij}\\

\haiku{{\textquoteleft}Het doel van mijn reis.}{zou den leeraar wellicht}{interesseeren}\\

\haiku{Zoo zette dan 's.}{anderen daags Micha\"el}{alleen de reis voort}\\

\haiku{Hij keek haar vragend,:}{aan en daar zij zwijgen bleef}{zei hij gebiedend}\\

\haiku{Maar hij kon het niet,.}{over zijn hart krijgen terug}{naar haar toe te gaan}\\

\haiku{{\textquoteleft}Ik ben er verheugd,}{om in u den advokaat}{van ons volk te zien}\\

\haiku{{\textquoteleft}Nu, als de oude......}{heer zich nog interesseert}{aan zijn zeg hem dan}\\

\haiku{Micha\"el wuifde,.}{met de hand maar zijn groet werd}{niet beantwoord}\\

\haiku{{\textquoteleft}Mijn verblijf alhier.}{duurt langer dan wel eenigszins}{kon voorzien worden}\\

\haiku{Maar oom Alexander had.}{reeds rechtsomkeer gemaakt en}{traag gleed de deur dicht}\\

\haiku{De broeder-portier,.}{kende hem niet liet hem in}{de groote spreekkamer}\\

\haiku{Er is bezoek dat.}{u mogelijk belang}{kan inboezemen}\\

\haiku{{\textquoteright} {\textquoteleft}Voor ons was het ook,{\textquoteright}.}{een heele verrassing ging}{prelaat Hermans voort}\\

\haiku{Huygens gaf hij een.}{omstandig relaas van hun}{belevenissen}\\

\haiku{Ja, ja, ze mochten,.}{hem zonder fout verwachten}{lachte de prior}\\

\haiku{Samen bogen zij,.}{over de wieg over het bed waar}{Aim\'ee sluimerde}\\

\haiku{Zelden kon hij zich.}{losrukken om weer naar zijn}{werktafel te gaan}\\

\haiku{wat hij te zeggen,.}{had kon ook met andere}{woorden geschieden}\\

\haiku{Zoo trok hij door de.}{dreef naar het bosch en verder}{plooide de hei open}\\

\haiku{De honden liepen,.}{snuffelend in het grauwe}{verregende kruid}\\

\haiku{In hun nood namen,}{de boeren hun toevlucht tot}{Micha\"el waarvaf}\\

\haiku{{\textquoteleft}Als honden over de...,.}{wegen gejaagd Wie ons vangt}{heeft een belooning}\\

\haiku{{\textquoteleft}Neen, maar misschien hebt?}{ge wel een drukker onder}{uw goede vrienden}\\

\haiku{Een leurder neemt de.}{oude kleeren over en draagt den}{dood en weet het niet}\\

\haiku{{\textquoteleft}Haal bij wie naar uw,{\textquoteright}.}{meening helpen kan gebood}{hij den chirurgijn}\\

\haiku{Hij liep de gang op,.}{en neer onrustig als een}{wild dier in de kooi}\\

\haiku{* * * ~ 's Anderen.}{daags schelde hij vruchteloos}{om het dienstmeisje}\\

\haiku{{\textquoteright} Perrin wenkte de.}{gendarmen en stiet moedig}{de achterdeur open}\\

\haiku{De bedreiging, dit,:}{wapen der zwakkelingen}{hanteerend zei hij}\\

\haiku{{\textquoteright} {\textquoteleft}Ha,{\textquoteright} fluisterde zij.}{en het was of zij onder}{een slag ineenkromp}\\

\haiku{{\textquoteright} {\textquoteleft}Waar een wil is, is,{\textquoteright}.}{ook een weg antwoordde de}{Choiseul geprikkeld}\\

\haiku{Meer wist Micha\"el,.}{niet maar het was voldoende}{om zijn rust te rooven}\\

\haiku{de naam klonk in hem,.}{na dan verhelderde een}{glimlach zijn gelaat}\\

\haiku{{\textquoteright} Geen antwoord hoorend, keek,.}{hij naar Micha\"el op zag}{hoe zwak hij nog was}\\

\haiku{Als zij stilstonden,;}{verraste hen het geluid}{van een geweerschot}\\

\haiku{{\textquoteright} Zijn oogen rustten op,.}{haar gelaat maar onder zijn}{blik roerde zij niet}\\

\subsection{Uit: Johan van der Heyden, magister}

\haiku{We zullen u die}{oneerbiedige woorden}{niet aanrekenen}\\

\haiku{De Zwarte Molen?}{is nog altijd eigendom}{van de van Wessem's}\\

\haiku{{\textquoteleft}Geert Dielis, met wat.}{spellen en een paar liedjes}{is geen school te doen}\\

\haiku{Altijd tegen avond,,.}{zegt hij zoo dan speel ik mijn}{geestelijk schaakspel}\\

\haiku{Hij moest de vraag niet,.}{stellen hij las genoeg op}{de aangezichten}\\

\haiku{de deken gaf hem.}{twee kriaaltjes mee om hem}{den weg te wijzen}\\

\haiku{Dat zal zoo...{\textquoteright} {\textquoteleft}Neen, dat{\textquoteright},.}{zal nu niet deed de nieuwe}{meester opmerken}\\

\haiku{die vreugd van elken,.}{dag haar broertjes en zusjes}{bezorgd te weten}\\

\haiku{Zoo lag ze daar bleek,.}{en stil tot geen enkele}{inspanning bekwaam}\\

\haiku{De grootste hoorde;}{ze roepen en tieren in}{de lage bosschen}\\

\haiku{Ik doe mijn best.{\textquoteright} Als,:}{hij van der Heyden tot aan}{de deur bracht vroeg hij}\\

\haiku{In {\textquoteleft}'t Heibloemke{\textquoteright}...{\textquoteright}:}{vinden wij elkaar Johan keek}{hem glimlachend aan}\\

\haiku{het praatje over weer.}{en wind en aan tafel de}{nieuwtjes van de stad}\\

\haiku{{\textquoteleft}Mijn vader, die in ',.}{t leger diende bracht hem}{mee van zijn tochten}\\

\haiku{Als hij terug op,.}{de binnenkoer kwam was het}{venster gesloten}\\

\haiku{{\textquoteright} Johan keek hem strak aan,, {\textquoteleft}!}{maar zijn mond bleef gesloten}{Hoed u voor dien man}\\

\haiku{de wind speelde in.}{haar hoofddoek en hij wuifde}{haar van verre toe}\\

\haiku{Tegen avond maakte.}{hij den hond los en trok langs}{het Akkerpad op}\\

\haiku{{\textquoteright} {\textquoteleft}Ga met mij nu mee,.}{ik moet zoo rap als ik kan}{terug naar Turnhout}\\

\haiku{{\textquoteright} Johan meende verder,.}{te vragen maar daar was de}{oude met Rina}\\

\haiku{{\textquoteleft}Zeg maar dat Marten!}{van Rossum het beestje zal}{komen betalen}\\

\haiku{Zoo geraakten ze.}{aan de eerste huizen van}{de Pottersstraat}\\

\haiku{Hij kende de de, -.}{Roode's al van langer hij}{was zelf van Turnhout}\\

\haiku{Hij kende er wat,.}{van had veel van die boomen}{geplant en ge\"ent}\\

\haiku{En plots doorstroomde.}{breed orgelspel den wijden}{beuk van de kapel}\\

\haiku{hier in de abdij,.}{is hulp Rina zal haar het}{brood toereiken}\\

\haiku{Dit kind en...{\textquoteright} {\textquoteleft}Toe nu,{\textquoteright},.}{Johan drong ze aan en dat klonk}{haast als een bevel}\\

\haiku{In de kamer nam.}{hij den kandelaar op en}{hield hem over de wieg}\\

\haiku{De stemmen klonken,,;}{en daalden wentelden en}{keerden stegen dan}\\

\haiku{{\textquoteleft}Ach ja{\textquoteright}, glimlachte, {\textquoteleft}?}{hijdie  Geert Dielis van}{den Zwarten Molen}\\

\haiku{Geert had niet gezien,.}{wat het boek was hij had het}{op de schouw gelegd}\\

\haiku{Zij liet een schreeuw en.}{viel op de knie\"en v\'o\'or het}{kruis boven den haard}\\

\haiku{ik heb het al zoo,,...}{dikwijls gezegd hij mag dat}{niet doen hij mag niet}\\

\haiku{Hij keek van boven.}{zijn brilleglazen uit en}{lei zijn brevier weg}\\

\haiku{{\textquoteright} {\textquoteleft}De Bijbel van Geert...{\textquoteright}.}{Dielis Van Wessem keek hem}{ongeloovig aan}\\

\haiku{Hij is geen kind meer,.}{dat heeft hij trouwens al meer}{dan eens bewezen}\\

\haiku{hier en eet eerst, ik{\textquoteright},.}{zorg wel voor de twee klassen}{zei hij vriendelijk}\\

\haiku{Midden de kamer;}{hing een bronzen luchter met}{brandende kaarsen}\\

\haiku{donker en dreigend.}{rolde de holle roffel}{door de leege stede}\\

\haiku{Hij scheen op iets te,.}{dubben de vage blik was}{naar binnen gekeerd}\\

\haiku{Aan een der hutten.}{stak hij zijn witten stok in}{het riet van het dak}\\

\haiku{Ginder zag hij den.}{glimmenden waterspiegel}{van een droomend veen}\\

\haiku{Zijn bloed stond stil, nog.}{nooit had zij al die jaren}{z\'o\'o v\'o\'or hem gestaan}\\

\haiku{Johan hoorde zijn roep.}{over de hei en er kwam een}{warm gevoel over hem}\\

\haiku{Dan brak geestdriftig.}{handgeklap en stormachtig}{hoerageroep los}\\

\haiku{het is nutteloos,.}{op te blijven het zou wel}{laat kunnen worden}\\

\haiku{Ge ziet, als d'ijzers,.}{afgetrokken zijn hebt ge}{niets meer te zeggen}\\

\haiku{Dat is nu jaren, '.}{voorbij ent is of het}{maar pas gebeurd is}\\

\haiku{Laatst in {\textquoteleft}De Roode{\textquoteright}.}{Schild heeft een heel gezelschap}{het kunnen hooren}\\

\haiku{De Roode Schild{\textquoteright}, waar.}{de bode voor Herentals}{en Lier uitspande}\\

\haiku{{\textquoteleft}Nooit, nooit meer zeggen,...{\textquoteright},,.}{jongen Traag met starre oogen}{knikte het kind}\\

\haiku{Maar wie zou hem in?}{den avond laten gaan hebben}{langs de moerassen}\\

\haiku{Nog een kind, had zijn.}{scherp verstand het doorzicht van}{den volwassene}\\

\haiku{Het verheugt mij dat{\textquoteright},.}{mijn zoon de waarheid liefheeft}{zei hij kort en hard}\\

\haiku{De Magistraat stond.}{op de pui van het stadhuis}{en schouwde den stoet}\\

\haiku{de landvoogdes is:}{onwetend van de nooden die}{het volk martelen}\\

\haiku{{\textquoteleft}Nooit werd met scherper.}{ironie de ondergang van}{de Kerk geschilderd}\\

\haiku{Een hoek der kamer:}{werd vrij gemaakt en daar nam}{het gezelschap plaats}\\

\haiku{Het was vreemdstil in.}{de gang en de stilte in}{haar was nog grooter}\\

\haiku{Hij zag wel dat hij.}{oud geworden was en hoe}{dof die oogen stonden}\\

\haiku{haar handen lei ze.}{op zijn schouders en haar oogen}{lieten hem niet los}\\

\haiku{We zijn reeds in de...{\textquoteright} {\textquoteleft}{\textquoteright},.}{VastenJa knikte D'Hose}{en keek Johan strak aan}\\

\haiku{Hij sloeg de oogen neer.}{onder den vragenden blik}{van van der Heyden}\\

\haiku{Want de kettersche,.}{leering die u betooverd heeft kan}{mij niet bekoren}\\

\haiku{{\textquoteright} De kamerdeur werd:}{opengeworpen en een blije}{stem verwelkomde}\\

\haiku{V\'o\'or het hoogaltaar.}{zonk koordeken Coomans}{neer en boog het hoofd}\\

\haiku{Maar ik kan me al}{voorstellen hoe hij ginder}{in zijn fluweelen}\\

\haiku{{\textquoteright} Vruchteloos trachtte.}{de jonge man de kerels}{tegen te houden}\\

\haiku{Onwillekeurig.}{vouwde hij zijn handen en}{begon te bidden}\\

\haiku{{\textquoteright} En de volgende '.}{week was de gezant van duc}{d'Alf opt Kasteel}\\

\haiku{Patroeljes ruiters,.}{reden door de straten zoo}{bij dag als bij nacht}\\

\haiku{De wacht bracht hem op,.}{het binnenplein zoo naar de}{kleine wachtkamer}\\

\haiku{het was haast of er,.}{vreugde lag in haar stem om}{die onzekerheid}\\

\haiku{De ziekte van haar...,.}{moeder Vrees niet het heeft haar}{aan niets ontbroken}\\

\haiku{Heel 't gezin is.}{groot en niemand die niet voor}{zichzelf zorgen kan}\\

\haiku{Wel viel de gang hem,.}{zwaar er was zooveel dat hij}{niet vergeten kon}\\

\haiku{Duc d'Alf's strengheid had;}{de rumoerigste geuzen}{voorzichtig gemaakt}\\

\haiku{En ze hebben hem,...{\textquoteright} {\textquoteleft}{\textquoteright},.}{gepijnigd gemarteldJa}{zei de priester stil}\\

\haiku{{\textquoteright} ~ V\'o\'or hij zich ter,:}{ruste legde had Rina}{hem nogmaals gevraagd}\\

\haiku{{\textquoteright} {\textquoteleft}Neen{\textquoteright}, antwoordde hij, {\textquoteleft}...}{zachtmisschien is de tros naar}{Limburg afgezakt}\\

\haiku{Zoo den langen nacht,.}{door met de gedachten die}{hem niet loslieten}\\

\haiku{Het was als de stem,.}{van een angstig kind dat plots}{zijn moeder ontwaart}\\

\subsection{Uit: Kroniek}

\haiku{De heer klopt hem op,,;}{den schouder hij moet mee gaan}{in de groote kamer}\\

\haiku{Fien zijn vrouw, ze is,.}{te vroeg gegaan ze had dat}{moeten beleven}\\

\haiku{Stanske doet dat niet,.}{ze heeft haar werk en ze zal}{dat niet verlaten}\\

\haiku{Maar dat hij dan juist,.}{voorgoed zou heengaan daar heeft}{ze nooit aan gedacht}\\

\haiku{De eenzaamheid die.}{haar aangrijnst en de armoe}{die aan haar hart bijt}\\

\haiku{Een goed werk deed ze,;}{er mee dat is heel het dorp}{door verteld geweest}\\

\haiku{{\textquoteleft}Wat buurten, bazin{\textquoteright}.}{Langs het poortje komt hij den}{hof ingedrenteld}\\

\haiku{dreef naar de meerschen,.}{waar de burgemeester zijn}{nieuwe hoeve bouwt}\\

\haiku{Een meester, dat zit.}{niet alleen binnen de vier}{muren van de school}\\

\haiku{En toekomende,,}{week begint de school ja de}{volgende week al.}\\

\haiku{Hij staat in de deur,,.}{de burgemeester en ziet}{den jongen man na}\\

\haiku{Hij blijft daar achter.}{in de klas staan en zegt geen}{gebenedijd woord}\\

\haiku{Hij ziet dat meester.}{Van Deun geboeid toeluistert}{en hij gaat maar door}\\

\haiku{Ze glimlachen maar,,.}{hij verschiet daar niet van hij}{kent die zwijgers wel}\\

\haiku{{\textquoteright} {\textquoteleft}Ja,{\textquoteright} glimlacht Karel, {\textquoteleft},...{\textquoteright}}{ik heb op een anderen}{akker geploegd maar}\\

\haiku{Hij trok voor een dag,.}{of acht naar zijn zoon die te}{Antwerpen woonde}\\

\haiku{Hij trekt  driftig,:}{aan zijn pijp hij staat recht en}{zegt kort en beslist}\\

\haiku{duidelijk kan hij,.}{den timmerman hooren dat}{donkere geluid}\\

\haiku{Hij tikt er met den.}{nagel tegen en past ze}{tusschen zijn tanden}\\

\haiku{als ze thuis komen,.}{staat Stanske al in de deur}{ze wacht met haar eten}\\

\haiku{hij moet zien hoe schoon,.}{dat koren op wil op dien}{mageren heigrond}\\

\haiku{Nieuw is ze niet, maar.}{ge hebt zelf wel gehoord dat}{er nog klank in zit}\\

\haiku{en de piano.}{is honderd maal meer waard dan}{den vorigen keer}\\

\haiku{{\textquoteright} Zoo, dat is niet waar,?}{en waar heeft Goor dat anders}{gehoord of gezien}\\

\haiku{Hij kent dien stap langs.}{het lage venster en het}{geruttel van de deur}\\

\haiku{De kleintjes zwijgen,.}{en kruipen weg ze krijgen}{hun deel van de bui}\\

\haiku{Hij zit aan tafel,.}{en haalt papier en potlood}{boven en zijn boek}\\

\haiku{Goor schuift papieren,.}{en boek bijeen hij sluipt haast}{de zoldertrap op}\\

\haiku{Zij zet een lied aan,,;}{er zijn geen woorden bij noodig}{haar stem klimt en daalt}\\

\haiku{Haar handen beefden.}{zoo sterk dat ze de tas haast}{niet vasthouden kon}\\

\haiku{Het licht van de lamp,,.}{speelt in haar oogen die zacht}{zijn hij weet dat wel}\\

\haiku{Hij hoort alleen den.}{hollenden wind als een ver}{gezoem in de schouw}\\

\haiku{De koorts verteert dit,.}{jonge lichaam hij voelt die}{kleine hand branden}\\

\haiku{Bij waken,{\textquoteright} zegt hij, {\textquoteleft}.}{en binnen een paar uur nog}{een lepel geven}\\

\haiku{Hij buigt over het kind,.}{er is een zware eerbied}{over hem gekomen}\\

\haiku{en dat gelaat is.}{hard met die scherpe lijnen}{van pijn en armoe}\\

\haiku{Hij moet weer buiten,,,.}{zijn in de zon de huizen}{uit de velden in}\\

\haiku{Als hij opkijkt, blikt,.}{hij in Anna's oogen die hem}{angstig aanstaren}\\

\haiku{meester geeft hem de.}{verhalenboekjes met de}{na{\"\i}eve prentjes}\\

\haiku{Meester kijkt ze na,;}{en dan verglijdt de glimlach}{van zijn aangezicht}\\

\haiku{De Ruyck voelt zich,.}{onzeker hij weet niet wat}{hij nog zeggen zal}\\

\haiku{De ondervraging.}{is komen bevestigen}{wat ik vermoedde}\\

\haiku{{\textquoteright} Meester Van Deun kijkt,.}{op zijn oogen worden klein en}{hij verzet zijn klak}\\

\haiku{Dan merkt hij plots het.}{piepen van een poortje en}{een stap op het pad}\\

\haiku{Ze kijkt hem strak aan,.}{dan worden haar oogen klein en}{de harde blik breekt}\\

\haiku{Na 't middageten,.}{wachten de schriften hij mag}{dat niet verwaarloozen}\\

\haiku{E\'en schakel uit den.}{ketting en heel het ding valt}{rinkelend dooreen}\\

\haiku{Hij blijft er wat naar,.}{kijken het is of hij staat}{op iets te wachten}\\

\haiku{De burgemeester.}{herhaalt het nog eens en stopt}{dan een versche pijp}\\

\haiku{Een ruime voorplaats,,;}{daar moet de piano staan}{een studeerkamer}\\

\haiku{Hij gaat eerst naar zijn.}{vader en hij weet niet hoe}{hij het zeggen zal}\\

\haiku{ginder is 't zand,,.}{zand en nog zand vliegakkers}{waar de wind mee speelt}\\

\haiku{En weer een nieuwe.}{dag en de arbeid is daar}{als een bedreiging}\\

\haiku{Als de furie over,.}{was heeft Barbara er nog}{eens mee gelachen}\\

\haiku{en de kinderen,.}{de eeuwige zorg en de}{slapelooze nachten}\\

\haiku{Een vrouwmensch om aan.}{het roer van een schip te staan}{en te bevelen}\\

\haiku{Een goed mensch, dat mag,,{\textquoteright}.}{gezegd worden een sterk mensch}{zegt meneer pastoor}\\

\haiku{Ge moogt me gelooven...{\textquoteright} {\textquoteleft}?}{dat ik ze gebruiken kan}{Maar wat wilt ge dan}\\

\haiku{{\textquoteleft}Er is geen geschrift,...}{en als ge geen geloof hecht}{aan wat wij zeggen}\\

\haiku{Hij wandelt met haar,,.}{langs de veldpaden door de}{bosschen door de hei}\\

\haiku{Hij loopt eens naar de,;}{school de bloemen mogen niet}{vergeten worden}\\

\haiku{Meester heeft zijn zoon,.}{in den arm genomen z\'o\'o}{voor Anna gestaan}\\

\haiku{Het kind groeide, de,,.}{donkere zoekende oogen}{het zwarte krulhaar}\\

\haiku{Soms gebeurt dat wel, '.}{als hij aan die bemoste}{steenen aant werk is}\\

\haiku{Het krijgt eten, zwarte,,.}{blinkende steenen en dan moet}{het terug binnen}\\

\haiku{Hij hoort den sleutel.}{in het slot knarsen en het}{gepiep van de poort}\\

\haiku{Een roode bloem, een,,.}{gele bloem een witte bloem}{nog veel bloemen}\\

\haiku{Hij staat daar met de.}{handen in de zakken aan}{zijn pijp te trekken}\\

\haiku{{\textquoteleft}Ja,{\textquoteright} zegt zijn vader, {\textquoteleft}.}{zoodaarmee gaan we weerom}{naar den winter toe}\\

\haiku{Dat was de laatste.}{maal dat hij ginder op de}{hoeve geweest was}\\

\haiku{De tijd verglijdt, er,.}{is geen pijn meer zij heeft dit}{geluk nooit gekend}\\

\haiku{Wat, dat weet hij niet,.}{juist maar het is als een roes}{over hem gekomen}\\

\haiku{Het kan het voorjaar,,.}{zijn of de voorbije stormnacht}{of dit ongeluk}\\

\haiku{Hij dronk niet, en we.}{geraakten elken winter}{door zonder vragen}\\

\haiku{Hij ziet het smalle,,.}{gelaat van het kind en de}{harde grijze oogen}\\

\haiku{Hij was een van die,,.}{kinderen die veel nemen}{maar weinig geven}\\

\haiku{Het is hem als een;}{verlichting als hij hoort dat}{ze liefst boven blijft}\\

\haiku{Zijn handen in zijn,.}{zakken hij fluit in den avond}{en is gelukkig}\\

\haiku{Niets moet hij vragen,.}{ze zitten gereed om hem}{alles te geven}\\

\haiku{De zon werkt in den,.}{jongen mast de geur van warm}{hars hangt in de lucht}\\

\haiku{Dat is de meester,,.}{van Rielen en dat is zijn}{vrouw en daar het kind}\\

\haiku{Tweemaal moet ze de,.}{vraag stellen Jane kijkt haar}{met koele oogen aan}\\

\haiku{Hij kijkt haar zwijgend,.}{aan maar die blik is een vraag}{en zij voelt dit zoo}\\

\haiku{{\textquoteleft}Ik zal 't eens aan,{\textquoteright}.}{meneer pastoor vragen zegt}{hij ten einde raad}\\

\haiku{{\textquoteleft}En we hebben thuis,...{\textquoteright}.}{een schoon katje zoo'n schoon jong}{katje en        XXII}\\

\haiku{Hij kan toch al die?}{mannen niet d'een na d'ander}{gaan te voet vallen}\\

\haiku{{\textquoteleft}Ik zal Jane een,{\textquoteright}.}{tas melk warmen zegt hij en}{Anna knikt hem toe}\\

\haiku{hij zet een ernstig.}{aangezicht of hij iets heel}{gewichtigs bedenkt}\\

\haiku{Wat ze naar St. Job,.}{kunnen brengen kunnen ze}{ook naar Vossel\`er doen}\\

\haiku{Hij werpt de kachel,.}{open de vlammengloed danst op}{de hooge zoldering}\\

\haiku{Meester De Ruyck.}{heeft hem trouw verslag over zijn}{poging gegeven}\\

\haiku{En op den koop toe, -.}{heiboeren daar doe'de nog}{niet mee wat ge wilt}\\

\haiku{Het vriest vinnig en.}{op het oksaal tocht het van}{alle windstreken}\\

\haiku{Zijn oogen gaan naar de,?}{mannen rond de tafel wat}{zullen ze zeggen}\\

\haiku{Maar dan wordt zijn blik,.}{weer streng en in dit gelaat}{komt iets hard hoekig}\\

\haiku{Er staat een lampje, -.}{op de tafel en daar dat}{rustbed in den hoek}\\

\haiku{Ze hooren gerucht.}{in de kamer en Anna}{Wouters luistert toe}\\

\haiku{het wilde lied van;}{een dronken vogel op den}{eersten lentedag}\\

\haiku{Als hij weer thuis is,,.}{ziet hij haar hoogrood gelaat}{en haar oogen glanzen}\\

\haiku{Zijn oogen zoeken, hij.}{vindt haar oogen die glanzen en}{onbeweeglijk staan}\\

\haiku{Er is geen tijd meer,,.}{zijn oogen doen pijn het gebeurt}{alles buiten hem}\\

\haiku{Hij schreit niet en toch.}{voelt hij de tranen over zijn}{aangezicht vloeien}\\

\haiku{In den namiddag:}{na de begrafenis heeft}{zijn vader gezegd}\\

\haiku{Als hij buiten gaat,,.}{volgt Karel hem als een hond}{met geslagen oogen}\\

\haiku{Elke slag gonst met,.}{een donker geluid in den}{stam gonst in zijn hoofd}\\

\haiku{Maar dan voelt hij dat,;}{de weerstand vermindert het}{is haast onmerkbaar}\\

\haiku{Hij hoort de stappen,.}{op de trap er is nog dit}{gerucht in den stal}\\

\haiku{{\textquoteright} Hij wacht niet op een,.}{antwoord hij is de heer en}{meester en beveelt}\\

\haiku{Het leven, hij heeft,.}{het ondergaan hij heeft er}{over gezegevierd}\\

\haiku{Maar het kan dan als.}{een storm over hem komen en}{hij ligt verslagen}\\

\haiku{De lauwe dooiwind,.}{speelt in zijn haren het kind}{is ongeduldig}\\

\haiku{Het kleine, scherpe;}{aangezichtje en de groote}{onrustige oogen}\\

\haiku{Meester kan dit kind,.}{niet benaderen hij voelt}{dat en het pijnt hem}\\

\haiku{hij zit met starre.}{droomoogen aan tafel en zijn}{glimlach is geluk}\\

\haiku{{\textquoteright} Mester knikt, hij heeft.}{het luisterend kind in de}{deuropening zien staan}\\

\haiku{En... dat ge hier nu,,?}{zijt Janneke daar is toch}{niets met de jongens}\\

\haiku{Zal ik hem eens over,?}{dien grond spreken daar is nu}{toch nog geen haast bij}\\

\haiku{{\textquoteright} {\textquoteleft}Als ge dat eens wilt,{\textquoteright}.}{doen zegt Janneke Berten}{en hij knikt voldaan}\\

\haiku{{\textquoteleft}Dat was wel eens om, '!}{te doen alsne mensch niet op}{geld of hulp moest zien}\\

\haiku{In dien man leeft het,.}{kind dat hij gekend heeft het}{is lang geleden}\\

\haiku{Den volgenden keer,.}{nemen we een blijspel of}{maar weer een drama}\\

\haiku{Rijpe bessen en.}{de vochtige geuren van}{mos en rottend hout}\\

\haiku{Misschien ontdekken,:}{ze hem wel en glimlachen}{ze hem toe zeggend}\\

\haiku{Er staat werkelijk.}{iemand over hem gebogen}{en zijn hart valt stil}\\

\haiku{{\textquoteright} De blauwpurpere.}{bessen reiken tot den rand}{van de kleine kan}\\

\haiku{{\textquoteleft}We moeten naar huis,,.}{gaan het is middag Stanske}{heeft al geroepen}\\

\haiku{Hij gaat met lichten,, -.}{stap zoo in den droom hij heeft}{kabouters gezien}\\

\haiku{{\textquoteright} En plots is daar een.}{lang gerekte angstkreet en}{onderdrukt gesnik}\\

\haiku{Hij luistert scherp toe,.}{of hij dat ingehouden}{geschrei niet meer hoort}\\

\haiku{{\textquoteright} In den zoon heeft hij,.}{den vader ontdekt het is}{niet te loochenen}\\

\haiku{ze maakt zijn sjaal los,.}{ze heeft zijn sloffen onder}{de kachel gezet}\\

\haiku{ze staat aan 't vuur, '.}{ze bukt en zit wat int}{vuur te leuteren}\\

\haiku{Meester heeft het maar.}{\'e\'ens moeten vragen aan}{Janneke Berten}\\

\haiku{Zij steekt haar handen,,.}{uit zij ziet dit glimlachend}{mager aangezicht}\\

\haiku{Zijn moeder die naar.}{hem toekomt en haar hand die}{op zijn voorhoofd ligt}\\

\haiku{{\textquoteleft}Ja, een erg geval,{\textquoteright}.}{knikte de baas en zijn stem}{glijdt er licht over heen}\\

\haiku{Fik van Sooi Delles,.}{zit thuis het gaat ellendig}{traag met die wonde}\\

\haiku{de menschen zetten.}{dat op een kast of hangen}{het tegen den muur}\\

\haiku{Fik heeft er zoo maar,.}{wat verf aangestreken zoo}{wat voor zijn plezier}\\

\haiku{Als ze buiten is,,.}{gaat meester naar de kast daar}{staat zijn papierke}\\

\haiku{{\textquoteleft}Ge moet niet van de.}{slimsten zijn om te weten}{vanwaar die wind komt}\\

\haiku{Na d'hoogmis is hij,.}{er mee begonnen het is}{geen kleine karwei}\\

\haiku{En hoe legt Fienke?}{van Sooi Delles dat aan boord}{met al dat klein volk}\\

\haiku{De Ruyck heeft geen,.}{last met de melodie ze}{hangt in zijn vingers}\\

\haiku{Maar hij wordt naar haar.}{gezogen en hij weet het}{en verweert zich niet}\\

\haiku{Als hij weer over het,,}{klavier gebogen zit zal}{hij piet opzien}\\

\haiku{Over den toog ligt de.}{herbergier gebogen in}{gesprek met \'e\'en klant}\\

\haiku{Hij ziet nog licht door.}{de raamspleten als hij in}{het gangetje komt}\\

\haiku{Hij staat ginder aan.}{den toog en luistert naar haar}{los en luchtig woord}\\

\haiku{Als hij buiten staat.}{in de eenzame straat moet}{hij zich losrukken}\\

\haiku{Met harde oogen staart.}{hij in den nacht en uren lang}{ligt hij wakker}\\

\haiku{Dat rooft wel veel tijd,,.}{en zijn kinderen die zijn}{nogal veel alleen}\\

\haiku{Zijn oogen gaan over de,,.}{werf de lage gebouwen}{den rustigen dries}\\

\haiku{De zegenende hand.}{gaat over hem en deemoedig}{maakt hij het kruisteeken}\\

\haiku{Er is een rust die,.}{hem aangrijpt in de lucht hangt}{een blije verwachting}\\

\haiku{Als 't schooljaar uit,.}{is komt Fred van den meester}{met prijzen naar huis}\\

\haiku{Hij staat v\'o\'or het volk.}{en toont den bloedbevlekten}{mantel van Cesar}\\

\haiku{Ze zit nog niet op,.}{den hoogsten kant maar dat hij}{zoo iets durft zeggen}\\

\haiku{{\textquoteright} Hij rilt onder de.}{harde woorden die hem als}{zweepslagen treffen}\\

\haiku{En zijn plaats op de,,.}{eerste bank blijft open vader}{tot hij terug is}\\

\haiku{Als ze daar met de.}{appels zijn moet de bende}{op een lange rij}\\

\haiku{{\textquoteright} De laatste woorden,.}{fluistert hij of een pijn hem}{onverhoeds overvalt}\\

\haiku{Zijn blokken gaan naar,,.}{Turnhout naar Herentals tot}{in de Walen toe}\\

\haiku{Ze komt voor haar kind.}{vragen wat zij vroeger voor}{zichzelf gevraagd heeft}\\

\haiku{s Anderdaags vroeg.}{in den morgen vertrekt hij}{met de eerste tram}\\

\haiku{Hij zit in de stad,.}{en dat is ver. En een mensch}{met een geleerdheid}\\

\haiku{Het is begonnen,.}{met de nieuwe stallen nu}{jaren geleden}\\

\haiku{Hij zit er over na '.}{te peinzen als zes avonds}{in den heerd zitten}\\

\haiku{Merieke kent zijn {\textquoteleft}!}{stap al.Die\"e van Janneke}{Berten is daar weer}\\

\haiku{{\textquoteleft}Merieke van den.}{meester en dien jongste van}{Janneke Berten}\\

\haiku{{\textquoteleft}En als ge 't nog,.}{beter wilt weten moet'te}{maar eens komen zien}\\

\haiku{De vacantie is, '.}{daar en int meestershuis}{wordt trouwfeest gevierd}\\

\haiku{Hij kuiert wat in,.}{het tuintje hij loopt daar als}{een leeuw in de kooi}\\

\haiku{Hij vraagt naar Sooi en.}{hij verneemt dat hij met den}{ploeg in d'akkers zit}\\

\haiku{Hij slaapt als een steen,,.}{maar als de eerste haan kraait}{staat hij op de werf}\\

\haiku{Een goeien dag en,,.}{anders niets wel vrank en open}{maar daarmee gedaan}\\

\haiku{{\textquoteleft}Ik ben nu toch nat, '.}{k zal die medicijnen}{maar eerst wegdragen}\\

\haiku{Steeds begrijpend, - maar.}{nu is zijn jongen daar en}{d\`at begrijpt hij niet}\\

\haiku{{\textquoteright} Nelleke kijkt om.}{naar het natte spoor van haar}{voeten op den vloer}\\

\haiku{Hij glimlacht als hij,.}{het kind ziet zoo'n leerlinge}{heeft hij nooit gehad}\\

\haiku{Misschien van Rielen,, '.}{ook wel of neen daar zijns}{avonds zoo laat geen trams}\\

\haiku{Zijn oogen glijden over,.}{zijn stoere krijgers over het}{jammerende kind}\\

\haiku{Hij legt alles op.}{het tafeltje en zijn oogen}{hangen aan het kind}\\

\haiku{In de deur van de.}{kamer staat een meisje en}{hij herkent ze niet}\\

\haiku{Meester is langs de.}{akkers gegaan en hij heeft}{de weien gezien}\\

\haiku{Dat is het niet, dat,,,!}{is het niet maar onze Peer}{onze Peer zoo vroeg}\\

\haiku{Fred jongen, als hij,,.}{nu nu op dit oogenblik}{in de stad kon zijn}\\

\haiku{Rielen, Rielen, het,.}{doode Rielen dat aan zijn}{oogen voorbij wandelt}\\

\haiku{Voor een artiste;}{gaat het van het eene seizoen}{in het andere}\\

\haiku{En Mieke moet de.}{kleine mannen eten geven}{en dan naar bed doen}\\

\haiku{Hij kan niet vluchten,.}{in een melodie in de}{wereld van een boek}\\

\haiku{{\textquoteright} De stem is zeer zacht,.}{een donker gefluister als}{van een die moe is}\\

\haiku{{\textquoteright} vraagt meester, hij kan.}{dit angstige vermoeden}{haast niet verbergen}\\

\haiku{Hij zal met een lied,.}{beginnen dat brengt stemming}{en bindt de aandacht}\\

\haiku{Het moet hem zoet zijn...}{terug te schouwen op den}{afgelegden weg}\\

\haiku{En Lina van den,?}{burgemeester wat heeft hij}{hooren vertellen}\\

\haiku{Zijn eigen fout, neen,,.}{niet Freds fout maar hijzelf mag}{dat op zich nemen}\\

\haiku{Waarom heeft hij hem,?}{laten gaan waarom heeft hij}{steeds toegegeven}\\

\haiku{Meester trekt er de.}{eerste vacantiedagen}{mee naar de hoeve}\\

\haiku{{\textquoteright} Meester hoort hoe die,.}{stem mild geworden is en}{dat grijpt naar zijn hart}\\

\haiku{En zij zelf koopt maar,,.}{kinderen niets dan meisjes}{alle jaren \'e\'en}\\

\haiku{Hij ligt alleen en.}{ziet de drijvende wolken}{en den hoogen hemel}\\

\haiku{Het leven dat hem,.}{geraakt heeft hij weet niet wat}{er van komen moet}\\

\haiku{De avond daalt, er ligt.}{een onzeglijke teerheid}{over al de dingen}\\

\haiku{Dat is lijk geld, als '.}{get kunt gebruiken hebt}{g'er gewoonlijk geen}\\

\haiku{Meester kijkt van zijn,.}{schrijfboeken op de zware}{stilte is voelbaar}\\

\haiku{{\textquoteright} Meneer de deken,.}{knikt hij kent Rielen en hij}{kan niet weigeren}\\

\haiku{Hij ziet wel hoe zijn.}{vicaire wenkt en hij schudt}{glimlachend het hoofd}\\

\haiku{Hij leeft er veertien.}{dagen van en dan is het}{verlangen daar weer}\\

\haiku{{\textquoteright} zegt Tistje en hij,.}{prakkezeert nog wat over wind}{regen en slecht weer}\\

\haiku{Hij weet wel dat hij,.}{beter zwijgen zou maar hij}{kan er niets tegen}\\

\haiku{Dan buigt Tistje, hij;}{is heel en al deemoed en}{onderdanigheid}\\

\haiku{Het verrast hem niet,.}{hij glimlacht vreemd als hij den}{brief in handen heeft}\\

\haiku{de nieuwe meester,.}{wacht al ongeduldig zijn}{leven begnit}\\

\haiku{{\textquoteleft}Jongens,{\textquoteright} zegt meester,.}{de Ruyck maar hij voelt dat}{hij niet verder kan}\\

\haiku{Hij kent hun namen,;}{ze hebben bij hem op de}{banken gezeten}\\

\haiku{Volk dat hier geweest,,.}{is een echte processie}{dat mag ik zeggen}\\

\haiku{En ja, ik vergeet,...}{er nog ze zullen mij wel}{te binnen vallen}\\

\haiku{In de kerk is dit.}{donker geroezemoes}{van de hoogdagen}\\

\section{Fr. Hendrichs}

\subsection{Uit: De maaier van den dood (onder ps. Edw. Halliwells)}

\haiku{Ik zal Uw koffers,.}{halen ik ben blij dat het}{zoo onschuldig is}\\

\haiku{{\textquoteright} {\textquoteleft}Ik weet nog niets van{\textquoteright},,}{een gestolen kapdoos af}{verbaasde hij mij}\\

\haiku{Maar de hooge Schot kwam:}{eerst met uitgestoken hand}{op mij toe en sprak}\\

\haiku{{\textquoteright} {\textquoteleft}Alles hangt er van,{\textquoteright},, {\textquoteleft}.}{af Edward verstomde hij}{mijluister verder}\\

\haiku{Miss Allwough is,;}{inderdaad zijn nicht en ik}{weet dat zij hem vreest}\\

\haiku{Waarom de nicht haar,;}{honingzoeten neef ducht kon}{ik niet ontdekken}\\

\haiku{Den geheelen dag,}{had ik geen verstandig woord}{met hem gewisseld}\\

\haiku{Het was werkelijk,.}{geen verlokkend vooruitzicht}{dat hij mij opende}\\

\haiku{ik verwacht, zal hij,,;}{je te lijf gaan indien hij}{begrijpt wat je zoekt}\\

\haiku{ik voelde dat mijn.}{lach niet oprecht was en ik}{rilde als van kou}\\

\haiku{De tuinarchitect.}{Allcott een nog grootere}{geheimzinnigheid}\\

\haiku{het stond bij mij vast,;}{dat ik niet in zou breken}{bij miss Rudgewood}\\

\haiku{{\textquoteleft}Iemand klom op een,,.}{stoel om die schilderij te}{verschuiven zegt hij}\\

\haiku{Was deze oude,?}{David zulk een huichelaar}{dat hij Ethel verried}\\

\haiku{eerst toen hij wist, dat....}{Ethel's papieren open en bloot}{op tafel lagen}\\

\haiku{De persoon, die zij,.}{aldus in vertrouwen nam}{had haar verraden}\\

\haiku{Nu werd mijn ijver.}{eerlijker beloond dan in}{de vorige uren}\\

\haiku{{\textquoteleft}Mag ik weten hoe,,?}{je werkdag gelijk je het}{uitdrukt werd besteed}\\

\haiku{een krachtig meisje,.}{voorvoelt instinctmatig het}{gevaar dat haar dreigt}\\

\haiku{{\textquoteleft}Ik weet het niet meer,.}{ik weet ook niet met welk recht}{U mij ondervraagt}\\

\haiku{Doch mijn pleegvader,.}{is niet in staat tot eenige}{laagheid Mijnheer Gould}\\

\haiku{even lichtte zij het,.}{hoofd om mij een dankbaren}{blik na te zenden}\\

\haiku{Een pleegvader, die!}{zijn kind zielslief heeft en het}{rampzalig maakte}\\

\haiku{Ik geloof, dat mijn.}{vriend een spoor ruikt en dan is}{hij verschrikkelijk}\\

\haiku{dan sprak Elliot:}{met zijn aantrekkelijke}{gemoedelijkheid}\\

\haiku{{\textquoteright} {\textquoteleft}Nog ongeveer een,.}{jaar toen vonden zij elders}{een betere plaats}\\

\haiku{{\textquoteleft}Ik zal haar schrijven,,,.}{dat je haar bedroog Joe er}{rest mij niets anders}\\

\haiku{{\textquoteright} {\textquoteleft}Ik zal open met U,?}{spreken U gelooft immers}{niet in Harold's schuld}\\

\haiku{Gingen er geheel?}{nieuwe personen in het}{drama optreden}\\

\haiku{{\textquoteright} ~ {\textquoteleft}Ik denk, dat het;}{gele boekje mij nog even}{te pas zal komen}\\

\haiku{{\textquoteright} {\textquoteleft}Is de rijksweg de?}{eenige weg van Bathhurst}{naar de grasvelden}\\

\haiku{{\textquoteright} {\textquoteleft}De Draak{\textquoteright} lag eveneens {\textquoteleft}{\textquoteright},?}{rechts vanDe Schotsche Leeuw als}{ik mij niet bedrieg}\\

\haiku{Dat was alles wat.}{hij losliet en ik bleef even}{wijs als tevoren}\\

\haiku{Vrouwen verbeelden,{\textquoteright}.}{zich wel meer wat wierp ik er}{oppervlakkig uit}\\

\haiku{het is meer dan eens.}{een vraag van wichelroede}{en telepathie}\\

\haiku{Na een paar honderd:}{schreden kwam Joe ons uit een}{zijpad in de flank}\\

\haiku{{\textquoteright} Dan drukte hij mij {\textquoteleft},{\textquoteright}, {\textquoteleft}.}{hartelijk de hand.Om Gods}{wil sprak ikspaar je}\\

\haiku{hij nam Mac Pherson's}{papieren uit diens zakken}{en wij verlieten}\\

\haiku{Omdat de spoorlijn,,;}{waar Doberney aan ligt toen}{nog niet gebouwd was}\\

\haiku{Joe zal er wel niet,.}{op tegen hebben dat ik}{een pijp bij hem rook}\\

\haiku{Ik vroeg, waarom Mr..}{Mac Pherson zooveel belang}{in ons kon stellen}\\

\haiku{ik vond de vraag zoo,.}{zot dat ik weer uitbarstte}{in een schaterlach}\\

\haiku{{\textquoteleft}Het kon immers niet,,{\textquoteright}.}{anders zijn Joe vervolgde}{hij gemoedelijk}\\

\haiku{{\textquoteleft}Neen, Joe, kijk mij maar,.}{frisch en vroolijk aan want ik}{heb goed nieuws voor je}\\

\haiku{Dan stond Elliot,.}{op greep zijn glas vast en hief}{het bevend omhoog}\\

\haiku{Blijf onze vriend, wij,.}{blijven de Uwe onze dank}{is onsterfelijk}\\

\section{Leo Herberghs}

\subsection{Uit: Gehuchtenboek}

\haiku{Het kloostertje is.}{te groot voor de weinige}{zusters die er zijn}\\

\haiku{Hotels in het groen.}{verborgen die je bereikt}{langs stenen trappen}\\

\haiku{Oud-Eijsden.}{is grijzig en verward in}{dromen van eertijds}\\

\haiku{{\textquoteleft}'n Varken haalt zelfs{\textquoteright},.}{uit deze modderbrei nog}{voedsel zegt de boer}\\

\haiku{De zon gaat onder,.}{achter wolken hoewel het}{nog lang geen avond is}\\

\haiku{Naast de sleepboot rust.}{een Kempenaer die in}{reparatie is}\\

\haiku{Die berg met aarde.}{die u hier ziet hebben we}{naarboven gehaald}\\

\haiku{Een derde boer zit.}{op de hooizolder waar hij}{hooi bij elkaar veegt}\\

\haiku{Ik heb er een die:}{helemaal van hout is maar}{die staat op zolder}\\

\haiku{In de stoel bij de,,.}{haard bij een breiwerkje zit}{een slapende kat}\\

\haiku{bosjes die vermist.}{aan de grens van vindbaar en}{onvindbaar liggen}\\

\haiku{Soms denk je dat een.}{rij bomen de plaats aangeeft}{waar je zoeken moet}\\

\haiku{Rott*~         Overheersend.}{grijze daken die diep naar}{beneden buigen}\\

\haiku{Er zijn antieke, {\textquoteleft}{\textquoteright} {\textquoteleft}{\textquoteright}.}{gelige treinstellen met}{Raucher enBuffet}\\

\haiku{Een kindergrafje,,.}{aandoenlijk blauw geschilderd}{ligt half gekanteld}\\

\haiku{{\textquoteright} De vrouw brengt ons naar.}{het bakhuisje en doet de}{sluitpen van de deur}\\

\haiku{De boerenvrouw wijst.}{naar het huis tegenover het}{hare dat leeg ligt}\\

\haiku{Achter de heuvels,,.}{links ligt Cottessen en rechts}{ook onzichtbaar Epen}\\

\haiku{Op het uiteinde.}{van de bebouwing wisselt}{het land de stad af}\\

\haiku{zodat ik haastig.}{retireer als een soldaat}{die vlucht voor spervuur}\\

\subsection{Uit: De laatste nachtegaal}

\haiku{Geen stank, geen lawaai!}{en zelfs geen toeristische}{route voor je deur}\\

\haiku{De burgemeester,.}{had toevallig net spreekuur}{dus dat kwam goed uit}\\

\haiku{De burgemeester ().}{zijn naam was Grotermaar was}{blij dat hij hen zag}\\

\haiku{In ons nabuurdorp.}{Rode trekken ze zich niks}{van de natuur aan}\\

\haiku{Die goeie raad van zijn.}{collega had hij zelf ook}{kunnen bedenken}\\

\haiku{Maar buurman Teuge.}{interesseerde zich niet}{voor het weggetje}\\

\haiku{Boer Strijdbaar, de man,.}{van Mien had suffend bij de}{kachel gezeten}\\

\haiku{Heeft hij ook gedacht?}{aan een dienstencentrum met}{sjoelbakkenkamer}\\

\haiku{Als ik jou was zou{\textquoteright}.}{ik maar eens met de heren}{kontakt opnemen}\\

\haiku{{\textquoteleft}Zo,{\textquoteright} zei Krombrood, {\textquoteleft}dat,.}{wist ik niet maar ik geloof}{wel dat het goed is}\\

\subsection{Uit: Mijn gezamenlijk geknor}

\haiku{Ik ga dan naar die}{persoon toe en duw hem er}{een paar in de hand.}\\

\haiku{Hij weet niet eens of.}{de waarheid een staart heeft of}{alleen maar een rug}\\

\haiku{Met het gevolg dat.}{je nooit een vis boven op}{de trap ziet staan}\\

\haiku{Nu eens slaat de deur,.}{te hard dicht dan weer klemt het}{raam of lekt het dak}\\

\haiku{en je je kieuwen.}{en je zwemvliezen niet hoeft}{te gebruiken}\\

\haiku{Dat moeten ze zelf.}{weten en je moet je er}{niet mee bemoeien}\\

\haiku{Ik zeg vaak knie als.}{ik voet bedoel en neus als}{ik oor wil zeggen}\\

\subsection{Uit: Wie zwemt is keg. Een handleiding voor actief niet-zwemmen}

\haiku{Als het niet naar school,.}{had moeten gaan had het ook}{geen schoolslag geleerd}\\

\haiku{Mij hebben ze het.}{tenminste nooit gevraagd en}{ik ken de schoolslag}\\

\section{Jacob Hiegentlich}

\subsection{Uit: 1907-1940. Een joods artist tussen twee oorlogen}

\haiku{{\textquoteleft}Kort geding tegen {\textquotedblleft}{\textquotedblright}{\textquoteright} (,), {\textquoteleft}}{deJongerenDe Nieuwe}{Gids 1931 I p. 615}\\

\haiku{Wat er omgaat in:}{deze jonge dichter is}{zuiver getekend}\\

\haiku{De compositie.}{van zijn romans is lang niet}{onberispelijk}\\

\haiku{C.L. Sciarone.}{heeft echter in Den Gulden}{Winckel van Sept}\\

\haiku{Siegfried was verheugd,;}{al sprak Pastoor wellicht ook}{tot anderen zo}\\

\haiku{Hij was ontwaakt en,:}{uit z'n roes herrezen z'n}{ogen stonden ernstig}\\

\haiku{Het was misschien een, - '.}{gave zoals ook dichter}{zijnn gave was}\\

\haiku{Op school hadden de, '}{leerlingen er van gehoord}{ze vroegen oft}\\

\haiku{ze ziet en 's avonds.}{schreef hij verzen en toch was}{hij niet gelukkig}\\

\haiku{Hij betreurde nog,.}{steeds de loopbaan die men voor}{hem gekozen had}\\

\haiku{Maar in 't begin.}{van de begrafenis was}{hij zeer geschrokken}\\

\haiku{{\textquoteleft}Er is maar \'e\'en God.}{en dat is die van ons en}{dat is de beste}\\

\haiku{Dan zitten ze op.}{een verwarmd terras bij de}{grote potkachel}\\

\haiku{Zelfs in Frankrijk schrijft,:}{men over hem vertaalt men een}{gedicht aan zijn vrouw}\\

\haiku{men lacht zuurzoet, uit '.}{angst voorn slecht artist te}{worden gehouden}\\

\haiku{{\textquoteleft}Nee, 't is nog zo'n, '.}{mooie avond ik ga nog liever}{wat opt terras}\\

\haiku{Onder de douche,:}{viel hem Baudelaire in}{hij declameerde}\\

\haiku{onafhankelijk.}{van elkaar hier en b.v. in}{China opduiken}\\

\haiku{De Jodenhaat is ',.}{n gevoel dat bij velen}{onverdelgbaar tiert}\\

\haiku{(Wunderhorn) Maar ook.}{deze Stolz zal Hitler c.s.}{ten kwade duiden}\\

\haiku{De cultuur van den.}{overwinnaar overstroomt die van}{den overwonnene}\\

\haiku{Het gezicht uit mijn}{raam op de uitgestrekte}{groenwitte velden}\\

\haiku{Welnu, bij dezen;}{vitalen Oostjood is van}{waanzin geen sprake}\\

\haiku{(Posthuum verschenen,).}{met een Voorwoord door Roelfien}{van Blokhuysen}\\

\section{Henri van der Hoek}

\subsection{Uit: De man van Timbuctu}

\haiku{{\textquoteright} {\textquoteleft}Dat kan best zijn, maar.}{ik begrijp toch niet hoe gij}{dat zoo raden kunt}\\

\haiku{Aldus kwam er eene.}{geheele verandering}{in het huishouden}\\

\haiku{{\textquoteleft}Ik hoor daar iets, wat,}{niet voor eene goede wijze}{van opvoeding pleit}\\

\haiku{ligt nog z\'o\'o versch in,.}{mijn geheugen alsof het}{pas gister voorviel}\\

\haiku{Het leventje, dat,.}{ik nu leidde beviel mij}{buitengewoon goed}\\

\haiku{{\textquoteright} vroeg ik mijn oom, die.}{al mijne bewegingen}{had gadeslagen}\\

\haiku{Was hij vroeger al,.}{niet erg mededeelzaam thans}{sprak hij maar zelden}\\

\haiku{Ik besloot dus maar.}{om niets meer te vragen en}{zoo bleef het er bij}\\

\haiku{Er stond een groote C {\textquoteleft}.}{boven aan het papier en}{daaronder stondPars}\\

\haiku{{\textquoteright} Hierop liep hij op.}{het bureau toe en plaatste de}{lamp op het schrijfblad}\\

\haiku{Vervolgens schoof hij.}{een stoel bij en bekeek het}{slot van het bureau}\\

\haiku{Iemand die rechts is!}{zal toch aan den linker kant}{geen afdruk maken}\\

\haiku{{\textquoteright} Hierna wees hij mij,.}{naar de plaats waar de pinnen}{behoorden te zijn}\\

\haiku{Daar is, geloof ik,,{\textquoteright}.}{de vijver hernam Born in}{de verte ziende}\\

\haiku{Indien gij even hier,.}{wilt komen kunt gij  er}{u van overtuigen}\\

\haiku{Het is de indruk,.}{van een touw waarmede de}{boot was vastgemaakt}\\

\haiku{Ondertusschen zijn.}{wij de lengte der boot te}{weten gekomen}\\

\haiku{Neen, gij behoeft niet,,{\textquoteright}.}{te zoeken dat heb ik al}{gedaan vervolgde}\\

\haiku{{\textquoteleft}Indien gij de lamp,,}{wilt vasthouden zal mij dit}{hoogst aangenaam zijn}\\

\haiku{{\textquoteleft}Er valt hier niets meer,,.}{te verrichten meester we}{zullen naar huis gaan}\\

\haiku{Maar nu hoorde ik,:}{de stem van mijn hospita}{die tot de meid riep}\\

\haiku{Ik was natuurlijk;}{heel vriendelijk voor haar en}{vroeg hoe zij heette}\\

\haiku{Ik herhaal het, om.}{\'een of twee uur was het nog}{tijd genoeg geweest}\\

\haiku{doe dus of gij thuis,.}{waart steek eens op en neem er}{dan uw gemak van}\\

\haiku{{\textquoteleft}Ja, en wij hebben -.}{allen kans van slagen dank}{zij het vroege uur}\\

\haiku{Niet \'e\'en enkele;}{keer behoefde Born hem een}{order te geven}\\

\haiku{En nu stond hij te;}{ginne-gappen met mijn}{sigaar in zijn mond}\\

\haiku{{\textquoteright} {\textquoteleft}Dat zag ik aan het.}{eindje sigaret dat hij}{weggeworpen had}\\

\haiku{En vertrek nu, mijn,'!}{brave en dat des Heeren}{zegen op u daal}\\

\haiku{Ik bemerkte een.}{ondeugende flikkering}{in de oogen van Born}\\

\haiku{{\textquoteleft}Neem plaats, inspecteur,{\textquoteright};}{sprak hij op zijn gewonen}{vriendelijken toon}\\

\haiku{Maar ik liet niets van.}{mijn vermoeden blijken en}{liep kalm naast haar voort}\\

\haiku{Het papier, dat eerst,;}{op de borst van den doode}{lag was verdwenen}\\

\haiku{En toen nam ik haar.}{mede naar het bureau en}{daar zit ze nu nog}\\

\haiku{want om de eerste.}{nacht reeds te laten waken}{vond ik overbodig}\\

\haiku{Full Speed rende hij.}{nu de Singel op en \'een}{der zijstraten in}\\

\haiku{Dus eindelijk zou!}{dan de oude heer Dubois}{gewroken worden}\\

\haiku{{\textquoteright} {\textquoteleft}Of hij heeft een meer,{\textquoteright}.}{winstgevend baantje kunnen}{krijgen ging ik voort}\\

\haiku{{\textquoteleft}Inspecteur,{\textquoteright} ging hij, {\textquoteleft};}{voorter zal natuurlijk wel}{een loods aan boord zijn}\\

\haiku{Na hem volgden ik,;}{inspecteur van Noort en ten}{laatste de agenten}\\

\haiku{Het was, alsof er.}{een troep wilde dieren aan}{het vechten waren}\\

\haiku{Wij legden hem bij}{het vuur neder en deden}{zooveel mogelijk}\\

\haiku{Op de plaats waar hij,.}{met het hoofd gelegen had}{ontdekte ik bloed}\\

\haiku{Weldra kwamen wij,.}{aan het station waar de}{trein juist gereed stond}\\

\haiku{ik wachten v\'o\'or ik.}{hem het doodende staal door}{het hart kon stooten}\\

\haiku{Met Dubois had ik.}{afgesproken  daar om}{drie uur te komen}\\

\haiku{{\textquoteright} Deze raad was nog.}{zoo slecht niet en vol vreugde}{spoedde ik mij heen}\\

\haiku{ik was en uitte.}{vol dankbaarheid een niet te}{weerhouden juichkreet}\\

\haiku{Wij drongen het bosch.}{binnen en spoedden ons zoo}{snel mogelijk voort}\\

\haiku{Ondertusschen zocht.}{ik naar een gelegenheid}{om te ontsnappen}\\

\haiku{Welnu dan kon het.}{mij ook niet schelen of hij}{leefde of dood was}\\

\haiku{Ik was koortsig en,.}{opgewonden wat niet te}{verwonderen was}\\

\haiku{Tot mijn groote blijdschap,.}{bemerkte ik dat wij naar}{het noorden trokken}\\

\haiku{Ik zal als tolk voor,.}{u dienen want mijn vader}{verstaat uw taal niet}\\

\haiku{En toch, bij nader,.}{inzien moest ik hem wel van}{den moord verdenken}\\

\haiku{Ikzelf was nu heer.}{en meester en stond dus in}{rang gelijk met hem}\\

\haiku{{\textquoteleft}Als vader niet goed,,{\textquoteright},}{is zal ik je wel komen}{waarschuwen zei ik}\\

\haiku{{\textquoteright} {\textquoteleft}Neen,{\textquoteright} drong ik aan, {\textquoteleft}de,;}{weg lijkt mij nu vrij ik kan}{niet langer wachten}\\

\haiku{Hadt je nu ooit wel,?}{kunnen denken dat ik je}{nog eens vinden zou}\\

\haiku{Vandaar zal hij wel.}{naar het ooglijdersgesticht}{worden overgebracht}\\

\section{B\`er Hollewijn}

\subsection{Uit: Als ratten in de kerk. Een dorpsgeschiedenis uit een nog jong verleden}

\haiku{Maar in plaats van zijn,:}{toorn de vrije loop te laten}{zei pastoor Klabbers}\\

\haiku{In de naam van de,.}{Vader en de Zoon en de}{Heilige Geest Amen}\\

\haiku{Ze liet hem in de.}{wachtkamer en spoedde zich}{naar haar heer-broer}\\

\haiku{Je kunt niet op straat.......}{lopen of je wordt door u}{weet wel overvallen}\\

\haiku{Maar ja, de freule.}{was raar en rare mensen}{doen rare dingen}\\

\haiku{Als je een kat bij,.}{een stuk vlees zet moet het al}{een sterke kat zijn}\\

\haiku{Doe de groeten van.}{mij thuis en zeg dat ik eens}{gauw kom aanlopen}\\

\haiku{{\textquoteleft}Stop,{\textquoteright} zei Verstegen,}{nog voordat mijnheer pastoor}{was uitgesproken}\\

\haiku{{\textquoteright} {\textquoteleft}Wis en waarachtig,{\textquoteright}.}{protesteerde de boer weer}{met de oude gloed}\\

\haiku{{\textquoteright} Zijn vrouw verkeerde.}{in een vergevorderde}{staat van opwinding}\\

\haiku{Maar je ervaring}{met de jongemannen hier}{komt mij wel van pas.}\\

\haiku{Men had de huizen.}{geverfd en gewit en de}{straten schoongeveegd}\\

\haiku{Maar als er iets is,.}{gebeurd heb  je het aan}{je zelf te wijten}\\

\haiku{{\textquoteleft}Ik heb weinig lust.}{om me door jou in het gips}{te laten drukken}\\

\haiku{{\textquoteleft}Denkt u misschien,{\textquoteright} vroeg, {\textquoteleft}.}{hijdat zo'n nacht me op de}{knie\"en zou krijgen}\\

\haiku{Het leven begint.}{bij veertig en ik ben nog}{geen vijfenvijftig}\\

\haiku{Totdat u me zei,.}{dat ik hen zelf hun leven}{moest laten maken}\\

\haiku{Ik blijf u eeuwig.}{dankbaar voor de raad die u}{mij hebt gegeven}\\

\haiku{Evengoed als ik de,.}{kerk respekteer dient u mijn zaak}{met rust te laten}\\

\haiku{{\textquoteleft}Laat me met rust,{\textquoteright} klonk.}{een schor baritongeluid}{tussen de knie\"en}\\

\haiku{{\textquoteright} zei de pastoor, om.}{het gesprek een andere}{wending te geven}\\

\haiku{Zachtjes sloop ze de.}{trap op en luisterde aan}{de slaapkamerdeur}\\

\haiku{{\textquoteright} hoorde ze gedempt, {\textquoteleft}?}{mevrouw jammerenwaarom}{komt er geen kind in}\\

\haiku{Het kristendom is,,.}{de bezieling het zuurdeeg}{dat alles doordringt}\\

\haiku{Met een onbestemd.}{gevoel in zijn binnenste}{zwoegde hij verder}\\

\haiku{De oorzaak lag nu,.}{niet bij de voetbalklub maar bij}{mijnheer pastoor zelf}\\

\haiku{Je hebt in een uur.}{tijd zes sigaretten de}{lucht in geblazen}\\

\haiku{{\textquoteleft}Wanneer het nog ooit,.}{zover komt zal ik  u}{nader berichten}\\

\haiku{Vier en vier... zestien...,...}{en zestien tien en tien is}{twintig zes en zes}\\

\haiku{Lena hoorde hem.}{naar boven strompelen en}{kwam de keuken uit}\\

\haiku{Ik nam haar onder.}{de armen en probeerde}{haar op te trekken}\\

\haiku{{\textquoteleft}Ik zou niet graag in,{\textquoteright}.}{de schoenen van Suske staan}{meende Verstegen}\\

\haiku{Mijnheer pastoor had.}{zich met een dansleraar in}{verbinding gesteld}\\

\haiku{{\textquoteright} {\textquoteleft}Staat u boven de?}{wet en de plaatselijke}{verordeningen}\\

\haiku{Het optreden van.}{de pastoor begon Suske}{knap te vervelen}\\

\haiku{Hij trok zijn handen.}{van het buffet weg en kwam}{er achter vandaan}\\

\haiku{{\textquoteright} Radeloos liep ze,,.}{de kamer uit de gang op}{naar de buitendeur}\\

\haiku{De dokter...{\textquoteright} Bevend,...}{rende ze weer naar binnen}{naar de telefoon}\\

\haiku{Ze stond op en ging.}{naar de keuken om een glas}{water te halen}\\

\haiku{{\textquoteleft}Ik geloof dat het,{\textquoteright}.}{wel los zal lopen stelde}{hij Lena gerust}\\

\haiku{Geloof me pastoor,.}{het wordt de hoogste tijd dat}{u tot u zelf komt}\\

\haiku{{\textquoteleft}Morgen gaan we een.}{eind verder en dan heb je}{mij niet meer nodig}\\

\haiku{Als je altijd in,.}{de weer bent geweest kun je}{niet zo maar niets doen}\\

\haiku{Het was duidelijk.}{dat de lange priester het}{flink te pakken had}\\

\haiku{Wenen is voor ons.}{een danszaal met heerlijke}{Straussmelodie\"en}\\

\haiku{{\textquoteleft}Als hij mij dat had,.}{geleverd was er geen ruit}{meer heel gebleven}\\

\haiku{Ze hebben u een,{\textquoteright}.}{flink kopje kleiner gemaakt}{stelde de boer vast}\\

\haiku{Voor mij staat vast, dat.}{de burgemeester eerlijk}{en betrouwbaar is}\\

\haiku{In mijn jeugd ging ik.}{altijd bij de oude Driek}{in ons dorp kijken}\\

\haiku{{\textquoteright} {\textquoteleft}Wat hij wil, weet ik,{\textquoteright}.}{niet richtte de pastoor zich}{weer tot de meester}\\

\haiku{Ik heb zo'n idee, dat.}{je het hele gezin hier}{op stang hebt gejaagd}\\

\haiku{Misschien is dit het,{\textquoteright}.}{beste gaf deze met een}{somber gezicht toe}\\

\haiku{Pastoor Klabbers keek.}{ontstemd vanaf de preekstoel}{naar het tafereel}\\

\haiku{Men zag haar knie\"en.}{en ongedekte dijen}{beven en trillen}\\

\haiku{Ze knagen aan de!}{eer en de goede naam van}{hun medemensen}\\

\haiku{Maar in plaats van zijn,:}{toorn de vrije loop te laten}{zei pastoor Klabbers}\\

\haiku{Abraham, Isaak en.}{Jakob leefden duizenden}{jaren geleden}\\

\haiku{Als we de kerk trouw,.}{blijven beloont de hemel}{ons duizendvoudig}\\

\haiku{Dank u hartelijk,{\textquoteright}.}{was alles wat de pastoor}{wist  te zeggen}\\

\subsection{Uit: Betsche (onder ps. Orenz)}

\haiku{Met 'r korte lijf '.}{komt ze nauwelijks boven}{t tafelblad uit}\\

\haiku{Dadelijk begint. ' '.}{dat jong weer overnieuwt Is}{n bodemloos vat}\\

\haiku{Lewieke was 'n,,.}{gave gezonde jongen}{van bijna acht pond}\\

\haiku{Ze moest maar zien met '.}{haar vijf kinderen int}{leven te blijven}\\

\haiku{Leen spaarde alles '.}{wat ze krijgen kon en gaf}{t d'n Hollander}\\

\haiku{Ze was met twee van ',.}{bij hun opt portaal naar}{de kanaal gegaan}\\

\haiku{Dat is niet zoveel.}{werk en ze heeft maar de helft}{leverworst nodig}\\

\haiku{Van Dorus van uit de, '.}{eerste bouw heeft hij ook eens}{n portret gemaakt}\\

\haiku{Bij hen onder woont ', '.}{ookn schilder maar daar kun}{jet niet aan zien}\\

\haiku{Ze kw\'amen, met 'n '.}{deftigheid alsof ze naar}{n soiree gingen}\\

\haiku{'t Viel niet mee om '.}{int hartstikke duister}{de weg te vinden}\\

\haiku{Ze moest niet denken, ' ',.}{dattne salon de luxe}{was zoals bij hun}\\

\haiku{Opeens schrok mevrouw ',.}{doorn geluid dat uit de}{hoek aan de deur kwam}\\

\haiku{Ze had kippenvel.}{en hing haren pelsmantel}{los over de schouders}\\

\haiku{Sodemerakel, wat.}{had die schullever op z'n}{geweten gehad}\\

\haiku{De blanken moesten over '.}{t algemeen niet veel van}{de negers hebben}\\

\haiku{Als ze 'n meisje ' {\textquoteleft}{\textquoteright},.}{hadden enn flessnaps kon}{je over hen lopen}\\

\haiku{Laatst hebben ze hem '.}{metn meisje achter de}{walmuur uitgehaald}\\

\haiku{Ze had gezegd, dat ',.}{t godgeklaagd was wat hij}{allemaal aanving}\\

\haiku{Bij die briefjes maakt,.}{de juffrouw tekeningen}{die ze met krijt kleurt}\\

\haiku{De hele middag{\textquoteright} {\textquoteleft}?}{al.Waarom ben je dan niet}{naar huis gekomen}\\

\haiku{Harie en Lambeer.}{slapen samen in \'e\'en bed}{onder de pannen}\\

\haiku{Ze zagen er uit '.}{alsof ze uitn brandend}{huis waren gevlucht}\\

\haiku{Ze zijn vergeten,,.}{dat ik hun op de bok heb}{geholpen zegt ze}\\

\haiku{Daar konden ze zich '... {\textquoteleft}}{n plaats voor reserveren}{Wat denken ze wel}\\

\haiku{Of 't verstandig,,.}{is dat ie zich Christien pakt}{valt nog te bezien}\\

\haiku{Je hoort 't aan 't,.}{spektakel dat ze nu vlak}{bij de poort maken}\\

\haiku{En ook omdat hij.}{dan weer werken kan en hij}{z'n volle loon krijgt}\\

\haiku{t Is voor Betsche '.}{n hele rekenarij}{om er te komen}\\

\haiku{t Is nu toch goed,{\textquoteright} '.}{wast enige wat ze had}{kunnen antwoorden}\\

\haiku{Z'n vrouw brengt hem z'n.}{eten en dikwijls brengt ze dan}{ook het hare mee}\\

\haiku{Daarom moet hij naar.}{beneden kijken als ie}{tegen Betsche spreekt}\\

\haiku{'n Langgerekte.}{geeuw van Manus doet beide}{vrouwen opkijken}\\

\haiku{Hoe ouder Manus,.}{werd hoe ongelukkiger}{dat hij zich voelde}\\

\haiku{Snotverdutju, dan.}{krijgen ze me met geen tien}{paarden meer in bed}\\

\haiku{{\textquoteright} {\textquoteleft}Als ze dan ook maar,.}{weet dat ze me stikken kan}{met haar gezauwel}\\

\haiku{{\textquoteright} {\textquoteleft}Daar heb je met de...{\textquoteright}.}{Pater niet over te klagen}{Betsche heeft gelijk}\\

\haiku{De Pater heeft nog.}{nooit zo iets gezegd als wat}{Christien er uit flapt}\\

\haiku{Als ie er kans voor, '.}{kreeg zocht hij er altijdn}{stil plekje voor uit}\\

\haiku{Gisterenavond had ', {\textquoteleft}{\textquoteright}.}{iet dan toch gemeend toen}{hijeindelijk zei}\\

\haiku{Je hebt me 'n les '.}{gegeven en toen kreeg ik}{n echte sigaar}\\

\haiku{Dat was trouwens wat,.}{van de conferentie zei}{hij en niet van hem}\\

\haiku{De Pater hielp hem.}{met flessen aangeven en}{de melkkruik vullen}\\

\haiku{Z'n borstkas, die nu,;}{met ribben is getekend}{zal dan gevuld zijn}\\

\haiku{Zo ongeveer is,.}{de voorstelling die Manus}{zich van z'n zoon maakt}\\

\haiku{Lewieke heeft z'n.}{broek aangetrokken en helpt}{haar ijverig mee}\\

\haiku{Als ze 's avonds naast, '.}{elkaar in bed liggen komt}{n hand aan Betsche}\\

\haiku{Als je de tering,.}{hebt moet je je toch al in}{de gaten houden}\\

\haiku{Begin maar eens met.}{vijf kinderen als je de}{hele dag weg bent}\\

\haiku{Ze komen ook niet, ' {\textquoteleft}{\textquoteright}.}{bij haar want dan zouden ze}{gauwnnaam krijgen}\\

\haiku{Als er geholpen,.}{moet worden hoef je niet met}{geld te rammelen}\\

\haiku{De twee halen de...}{schouders op en geven de}{zatlap geen antwoord}\\

\haiku{Enfin, ik ga eerst ',.}{evenn pot bier drinken dan}{kom ik naar je toe}\\

\haiku{Daar werd 't opeens '. '}{stil en toen hoorde Betsche}{n hels spektakel}\\

\haiku{Hij lag tussen z'n,.}{wimpers door te kijken toen}{Betsche binnenkwam}\\

\haiku{Ze wist wel niet wat,.}{saldo en rente was maar}{wat kon dat schelen}\\

\haiku{Hier...{\textquoteright} Op dat moment.}{ging de kaffeedeur open en}{verscheen de Pater}\\

\haiku{Geloemige nog,.}{aan toe wat had Betsche in}{haar rats gezeten}\\

\haiku{Als Manus dood ging.}{kregen ze honderd gulden}{van het dooiefonds}\\

\haiku{Dat zijn wat wilde,,.}{haren die gaan er wel uit}{redeneerde ie}\\

\haiku{Ze keek niet om, maar '.}{begon te schreeuwen alsof}{ze aann mes hing}\\

\haiku{{\textquoteleft}Vuile sokus,{\textquoteright} {\textquoteleft}.}{griende zewaar heb je zo}{lang uitgehangen}\\

\haiku{Met \'e\'en scheur rijt de.}{een de ander z'n hemd tot}{aan de broeksrand open}\\

\haiku{{\textquoteleft}Was je eerst,{\textquoteright} gebiedt, {\textquoteleft}.}{Betsche noganders maak je}{het hele bed geel}\\

\haiku{In het straatje giert.}{de storm van gemaskerden}{naar het hoogtepunt}\\

\haiku{{\textquoteleft}'t Is nog maar tien,{\textquoteright}, {\textquoteleft},.}{minuten zegt Betschekom}{O.L. Heer wacht op ons}\\

\haiku{Gij zijt allemaal,.}{erg bedankt dat ge zo veel}{voor ons gedaan hebt}\\

\subsection{Uit: Brandende aarde. Een brok geschiedenis van de mijnstreek}

\haiku{Het hout kraakte nu.}{achter hen en dezelfde}{holle lach weerklonk}\\

\haiku{Ze kwamen op de.}{ontgonnen heuveltop en}{zochten naar hun hut}\\

\haiku{Ze trachtte hem door.}{gebaren te bewegen}{met hen mee te gaan}\\

\haiku{Enkele dagen.}{later werd er op de deur}{van de hut geklopt}\\

\haiku{Voor hen stond het vast,.}{dat het uitwerpselen van}{de hel waren}\\

\haiku{Door zijn verstorven.}{en harde leefwijze werd}{hij vroegtijdig oud}\\

\haiku{De zwelling van zijn.}{spierballen tekende zich}{onheilspellend}\\

\haiku{{\textquoteright} Christel sloeg angstig '.}{n kruis en schoof schuw langs haar}{broer door naar moeder}\\

\haiku{Erik trok de geldzak.}{te voorschijn en legde hem}{voor zich op tafel}\\

\haiku{Maar toen de Schele,.}{de daad bij het woord voegde}{riep hij hem terug}\\

\haiku{'t Liefst zou hij nooit '.}{\'e\'en brok van z'n glorie aan}{n ander overdoen}\\

\haiku{Hij zou de oude,.}{laten zien dat hij niet bang}{was voor de duivel}\\

\haiku{{\textquoteright} In 't vertrek trok.}{hij zijn buis uit en ging door}{het raam staan turen}\\

\haiku{Hij zocht gezelschap.}{en dronk pinten bier in een}{onmatig aantal}\\

\haiku{Beneden zocht de,,.}{Leeuw stotend en struikelend}{naar de  uitgang}\\

\haiku{In plaats van de vrouw.}{kwam daarna een tweede man}{aan de lier te staan}\\

\haiku{Aan zijn knie\"en en.}{ellebogen vertoonden}{zich schilferplekken}\\

\haiku{Hijgend wreef hij zijn.}{ogen vrij en keek voor zich uit}{zonder iets te zien}\\

\haiku{Nooit meer zou hij een '.}{hand uitsteken omn stuk}{kool te verwerven}\\

\haiku{Ga maar terug van.}{waar je gekomen bent met}{je duivelsgebroed}\\

\haiku{Twee jaren later.}{werd de oude naast zijn vrouw}{in het graf gelegd}\\

\haiku{'t Waren ruwe,.}{kerels voor wie slechts drank en}{vrouwen bestonden}\\

\haiku{Lang heb ik gewacht.}{om de bezitster mijn wraak}{te doen gevoelen}\\

\haiku{Met 'n minachtend.}{lachje draaide hij zich om}{en spoog op de grond}\\

\haiku{'n Vrouw zuchtte en.}{een man liep mopperend de}{kamer op en neer}\\

\haiku{{\textquoteright} Toen hij de ladder,:}{naar de galg beklom bad hij}{met trillende stem}\\

\haiku{Ik zou ook liever,.}{boer blijven als jullie maar}{niets te kort kwamen}\\

\haiku{De zon bleef echter,.}{onberoerd doorschijnen fel}{en genadeloos}\\

\haiku{Voldoening blonk in:}{zijn zwartomrande ogen toen hij}{haar verzekerde}\\

\haiku{Wilhelm liet de arm.}{van z'n vader los en ging}{op de bank liggen}\\

\haiku{Hij voelde zich als '.}{n vreemdeling aan de deur}{van zijn eigen huis}\\

\haiku{Wilhelm lag op de.}{bank tegen de muur met zijn}{slaap te worstelen}\\

\haiku{Er brak een rad en.}{een der mijnen moest geheel}{worden stilgelegd}\\

\haiku{{\textquoteleft}Moeder ook proeven,{\textquoteright} '.}{beet ze eveneens een stuk van}{t veevoeder af}\\

\haiku{Dan verdien ik met {\textquotedblleft}{\textquotedblright}.}{de jongens genoeg om spek}{enweck te kopen}\\

\haiku{In de nabijheid.}{van de Holz werden nieuwe}{schachten aangelegd}\\

\haiku{Met een diepe zucht,.}{liet ze de jongen los die}{blij naar buiten liep}\\

\haiku{Zelfs de winkeliers.}{zagen liever haar rug door}{de deur verdwijnen}\\

\haiku{Ze leken meer op.}{de Tsaar van Rusland dan op}{goede christenen}\\

\haiku{Voorzichtig legden.}{de Ratten hun hoofdman voor}{het huis op de grond}\\

\haiku{Mensen zagen haar '.}{gebogen overt lichaam}{van Rudolf liggen}\\

\haiku{{\textquoteright} Tegen de sterke.}{armen van de pumper was}{Rudolf machteloos}\\

\haiku{Als de baas alles, '.}{weten wilde zou iet}{te horen krijgen}\\

\haiku{Met een nors gezicht, {\textquoteleft}{\textquoteright}.}{zei de man dat hij in het}{vervolgslepen moest}\\

\haiku{Het geweld barstte;}{los en de citadel werd}{in puin geschoten}\\

\haiku{Altijd door zag hij.}{haar lachend gezicht met de}{blinkende tanden}\\

\haiku{Z'n verlangen dreef,.}{hem vooruit in de richting}{van de boerderij}\\

\haiku{Is er bij jullie '?}{op de boerderij nooitn}{ongeluk gebeurd}\\

\haiku{Als hij tot U komt,;}{gehuld in schapenvacht en}{aan Uw oren femelt}\\

\haiku{In ieder geval. '}{had hij de directeur voor}{de poort laten staan}\\

\haiku{Z'n broer Anton had.}{een roodharig kind met een}{bleek sproetengezicht}\\

\haiku{Driekus Joep werd bij.}{deze gelegenheid als}{portier aangesteld}\\

\haiku{De scheldende stem.}{van het rode vrouwmens klonk}{op het kamertje}\\

\haiku{Eindelijk zou het {\textquoteleft}{\textquoteright}.}{in de crypte vanzijn huis}{worden bijgezet}\\

\haiku{Liesbeth kende de '.}{oorzaak van deze spot en}{sloegn zijpad in}\\

\haiku{Als de vroedvrouw kwam,.}{zou het kind niet lang meer op}{zich laten wachten}\\

\haiku{{\textquoteright} zei Driekus Joep fier,.}{tegen Jozef terwijl hij}{z'n jas dichtknoopte}\\

\haiku{Ze had hem een zoon. ',.}{geschonkent Mooiste wat}{een vrouw geven kan}\\

\haiku{De ellende in.}{de arbeidersgezinnen}{groeide met de dag}\\

\haiku{Ze was een flinke, '.}{jonge vrouw geweest metn}{welgevormd lichaam}\\

\haiku{De ranke klipgeit '.}{van weleer groeide uit tot}{n kamerolifant}\\

\haiku{In een winkel kocht ',.}{hijn paar broodjes die hij}{onderweg opat}\\

\haiku{Hans, de jongste, 'n,.}{stumperig ventje dat nooit}{had kunnen lopen}\\

\haiku{'t Was een proper,,.}{huisje goed in de verf met}{witte gordijnen}\\

\haiku{{\textquoteleft}'t Is rotzooi,{\textquoteright} zei,.}{Wilhelm terwijl hij weer op}{de stoel ging zitten}\\

\haiku{De mannen zaten.}{met grimmige gezichten}{aan het tafeltje}\\

\haiku{Dat moet de Bond maar,{\textquoteright}.}{opknappen bromde hij en}{ging naar de voordeur}\\

\haiku{Uit de kooi stapte.}{Wilhelm op de losvloer in}{de grote schachtplaats}\\

\haiku{De Dodekop kwam.}{als laatste door het gat en}{zag de etende mannen}\\

\haiku{Degene, die hij,.}{betrapte kon minstens op}{ontslag rekenen}\\

\haiku{Nauwelijks was hij,,:}{terug of weer klonk het hoog}{en laag hard en hees}\\

\haiku{{\textquoteright} 's Avonds zaten ze.}{samen en spraken over de}{mijn en de lonen}\\

\haiku{De oude Jozef,.}{meende dat hij het gesprek}{wilde afbreken}\\

\haiku{Zelfs in de pijler,,.}{als hij de kool loswroette}{dacht hij aan Greetje}\\

\haiku{{\textquoteright} Met gestrekte arm.}{en bevende vinger wees}{Gerard naar de deur}\\

\haiku{In z'n gedachten.}{had hij met haar geleefd in}{een glanzend huisje}\\

\haiku{Elke Zondag ging '.}{hij naar het huisje op de}{hoek vant pleintje}\\

\haiku{'n Witte zuster.}{begeleidde hem naar de}{r\"ontgenafdeling}\\

\haiku{{\textquoteright} De zuster verstond,.}{wel wat de soldaat zeide}{maar gaf geen antwoord}\\

\haiku{'n Onwillige.}{motor aan de schudgoot had}{hem opgehouden}\\

\haiku{Het ontstuimige.}{bloed van zijn voorvaderen}{joeg door z'n aderen}\\

\haiku{{\textquoteright} Greetje legde haar.}{verstelwerk op tafel en}{kwam bij Wilhelm staan}\\

\haiku{Hij zag haar aan de,.}{deur staan met opgeheven}{hoofd en harde ogen}\\

\haiku{Tranen welden in.}{z'n ogen en drupten door de}{groeven langs zijn neus}\\

\section{P.C. Hooft}

\subsection{Uit: Liederen en gedichten}

\haiku{niet langer dan het,.}{weigeren duurt niet langer}{duurt het minnen}\\

\haiku{Maar 't schijnt wel wie,.}{geen rust en waagt kan kwalijk}{lust gewinnen}\\

\haiku{Indien dit bosje, '!}{klappen kon wat melddet}{al boelage}\\

\haiku{wie geboden dienst, '.}{versmaadt wenst er wel om als}{t is te laat}\\

\haiku{{\textquoteright} ~ {\textquoteleft}Reine liefd' van',{\textquoteright}, {\textquoteleft},!}{d allerreinste zei hij}{Sijbrech bolle meid}\\

\haiku{{\textquoteright} ~ Zij heeft een zweep,;}{ontbo\^on uit Polen die ze}{bij haar kammen hangt}\\

\haiku{Zij mogen niet, uit,,.}{heter borst gemind zijn maar}{slechts aangebeden}\\

\haiku{Zo gaat het waar men.}{naar waardij de godlijkhe\^en}{verzuimt te vieren}\\

\haiku{Uw dubbel nat, door,.}{deze nood verijsd wordt stijf}{als stenen vonder}\\

\haiku{Treurt rozelaar, treurt,,.}{bollen treurt laat vrij de mol}{uw plaats verwild'ren}\\

\haiku{Om daar een parel,,.}{af te halen en streeft zo}{niet door duizend do\^on}\\

\haiku{Zijn hoogste lof, in,,;}{mensenkelen noch donder}{is noch bliksemjacht}\\

\haiku{Hoe kunt gij hen, die,,,.}{u niet zoeken bestoken}{in hun voordeel gaan}\\

\haiku{hij gaat dan ook weer.}{meer liefdespo\"ezie voor}{Christina schrijven}\\

\haiku{Zo noemt hij zichzelf, {\textquoteleft}{\textquoteright}.}{regelmatig Cephalo}{dat is Grieks voorhoofd}\\

\haiku{Amaryl, de deken.}{zacht 17 Opnieuw een lied voor}{Ida Quekels uit 1604}\\

\haiku{7 hem... ga hem (nl.);}{Narcissus bij gebrek aan}{een levensgezel}\\

\haiku{1 Periosta,;}{een rivier mogelijk de}{Amstel of de Vecht}\\

\haiku{5 dode... ruimen ();}{de doden verenvan het}{bed te verlaten}\\

\haiku{In 1630 stuurde hij:}{haar de eerste kersen met}{dit korte versje}\\

\haiku{1-2 plicht... staat de;}{taken bevat van een mens}{die in aanzien staat}\\

\haiku{2 eerst... schoot eerst op,.}{Delfts schoot tenslotte in de}{schoot van de aarde}\\

\section{Samuel Coster en P.C. Hooft}

\subsection{Uit: Warenar}

\haiku{beter eeren,30 En,.}{ick in plaets des giericheyts}{dit huys beheeren}\\

\haiku{gheslaghen,75 Dat hy?}{mijn thienmael op een dach ten}{huys uyt gaet jaghen}\\

\haiku{Een hielen dach sit,}{hy in huys ghelijck as}{op de winckels}\\

\haiku{Gelijc hier Warnar,.}{buers dochter die mocht mijn}{kommer stelpen}\\

\haiku{eenich dralen, Keunje,:}{me niet helpen soo moet ick}{een ander halen}\\

\haiku{Daer mat hy't feyt ten, //}{breetsten uyt en gingh staen met}{veel menty temen,289}\\

\haiku{Ien doodts-hooft seydse,,}{ien doots-hooft wat binje ien}{nuwelijck bloedt?498}\\

\haiku{warnar En as je,?}{ghelt ontfanghen zoudt wat}{doeje dan hier}\\

\haiku{geertruyd Jae wel 'tis, //}{te wonder wat rancken}{datter om gaen.580}\\

\haiku{Wat komt my over, ick //,:}{bin ien bedurven man.635 Hout}{den dief hout den dief}\\

\haiku{al gaf jou de Beul,?}{een kerf.685  ritsert Ic een}{Pot waer van daen}\\

\haiku{De Heer heb heur ziel,,}{wy warent soo wel iens We}{villen nimmermeer}\\

\haiku{die aan het eind van.}{het spel van zijn behoudzucht}{zal zijn genezen}\\

\haiku{De voorrede houdt:}{de afloop van het spel met}{moeite verborgen}\\

\haiku{Giericheydt kondigt.}{aan dat hij de vrek nog \'e\'en}{lesje zal leren}\\

\haiku{Essentieel in.}{morele zin is ook de}{afloop van het spel}\\

\haiku{{\textquoteleft}kaleteyt{\textquoteright} is een {\textquoteleft}{\textquoteright},.}{woordspeling metkaal wat aan}{berooid doet denken}\\

\haiku{hoewel hij er nu:}{volstrekt niet toe gedwongen}{zou zijn 21stede}\\

\haiku{nog geen duit (op veel)...?:}{munten stond een kruisvormig}{teken 86Of taecken}\\

\haiku{aan de voorname,:}{kant de rechterkant 296hebbe}{kick gheen ghenuyghen}\\

\haiku{per pond een reaal (),!}{gouden munt ter waarde van}{ca 2{\textonehalf} stuiver kind}\\

\haiku{alsof (het lot van):}{Holland ervan afhing 587op}{waren ghenomen}\\

\haiku{dubloenen (Spaanse,):}{dukaat ter waarde van ca.}{zeven gulden 634of}\\

\haiku{anders had men in}{het huis van Warnar niet zo}{lang op mij gewacht}\\

\section{C.P. Hoogenhout}

\subsection{Uit: Geskiedenis van Josef en Catharina, die dogter van die advokaat}

\haiku{Ik voel al 'n paar.}{dagen of ik niet lang meer}{op aarde zal zijn}\\

\haiku{{\textquoteleft}Ek praat eendag met,}{vrind Pannevis daaroor en}{laat vir mij ontval}\\

\haiku{En toe hulle ver,,.}{Hem sien bid hulle Hem aan}{maar party twyfel}\\

\haiku{Veelseggend is die,:}{Voorrede wat ons hier in}{sy geheel laat volg}\\

\haiku{{\textquoteright} Vader en skoonseun,.}{gesels in die Ne\"ende}{Gesprek Z.A. 5 Des}\\

\haiku{{\textquoteright} Die geskiedenis.}{van die G.R.A. moet as bekend}{veronderstel word}\\

\haiku{{\textquoteright} Soms trek hy liewers ' {\textquoteleft}{\textquoteright}.}{nsluier oor die sielelewe}{van sy karakter}\\

\haiku{Hy beveel jou ook,, {\textquoteleft}}{nie maar s\^e net wat hy dink}{die beste is bv.}\\

\haiku{{\textquoteright} {\textquoteleft}Hoe sal die arme;}{Wagner sig nou verbly in}{syn salige rus}\\

\haiku{{\textquoteright} Die perdeboer word,.}{ook nie vergeet nie want die}{papies is lastig}\\

\haiku{(2) C.P. Hoogenhout.}{in Die Nuwe Brandwag deel}{I nrs. 1 en 2}\\

\haiku{Laaste Stem uit die,.}{Genootskap van Regte}{Afrikaners 1918}\\

\haiku{{\textquoteleft}Das tog al te erg.}{om ons broer van honger en}{dors te laat doot gaan}\\

\haiku{Die ander kinders.}{was nie van Ragel nie maar}{van een ander vrou\"e}\\

\haiku{God die alles weet,,}{die alles siet kan ons ook}{net soo goed sien as}\\

\haiku{{\textquoteleft}Ons het niks meer om,,.}{te eet nie gee voor ons brood}{anders gaat ons dood}\\

\haiku{Hulle weet goed, dat.}{Jakob voor Benjamin nooit}{sal laat saam gaan nie}\\

\haiku{En toe hulle nou.}{regtig na die tronk gaan toe}{skrik hulle nog meer}\\

\haiku{Voor Benjamin wort.}{vyf maal soo veul opgeskep as}{voor die andere}\\

\haiku{Josef was bly om,.}{al syn broers by hem te heh}{vooral Benjamin}\\

\haiku{{\textquoteright} Toe die broers dat hoor,:}{was hulle soo maar lam van}{skrik en hulle seh}\\

\haiku{{\textquoteleft}Og meneer mot tog.}{nie kwaad wort nie dat ek nog}{een woordje wil seh}\\

\haiku{Myn kinders mot tog!}{ook altyd ou mense met}{agting behandel}\\

\haiku{en as myn storie,.}{nie na jou sin is nie stuur}{hom dan maar terug}\\

\haiku{'n Afrikaander, wat,;}{geen Engels verstaat sou daar}{min an gehad h\^e}\\

\haiku{Hulle stemme gaat, '.}{te hoog dit lyk of daarn}{twis of rusie is}\\

\haiku{Appelkoos, perskes,,,,.}{appels pere pruime was}{by hom in o'ervloed}\\

\haiku{want Ma roep dat die.}{karn af is en ik moet die}{botter gaat af haal}\\

\haiku{Die rente  was.}{al 3 maande verval en}{nog nie betaal nie}\\

\haiku{{\textquoteright} {\textquoteleft}Weet jy nog, wat jy}{eendag op skool geseg het}{toen ik jou geplaag}\\

\haiku{By 'n draai in die '}{groot pad terwyl hulle weer}{opn stap getrek}\\

\haiku{Maar David dit was,!}{daarom nie jou skuld as dit}{so gekom het nie}\\

\haiku{n Heerlike troos,}{lesers wat kerkerade is}{of familie het}\\

\haiku{Die laaste handdruk.}{toen hy al op die wage}{sit was gewissel}\\

\haiku{Die voorperde trap, ',.}{reg. Een klapn harde klap}{en die wagens rol}\\

\haiku{- twe elemente, wat.}{met makaar mar volstrek nie}{kan verenig nie}\\

\haiku{dat hy gerus mag,.}{slaap en dat syn reis verder}{voorspoedig mog wees}\\

\haiku{dit was mar so haar -.}{gewoonte tant Mimie is}{dit al gewoon}\\

\haiku{{\textquoteright} Toen sy dit geseg,, '}{het sy kombuis toe en net}{nou kann mens hoor}\\

\haiku{In kort, hy is weg.}{en die onrusstokers het}{hulle sin gekry}\\

\haiku{as dit so kom nie,,.}{maar meer mag ik nie se nie}{en wil ik oek nie}\\

\haiku{David was in die.}{Kaap en siet Catharina}{en wie weet wat meer}\\

\haiku{maar die goeje siel,;}{had nie anders had sy dit}{seker gegewe}\\

\haiku{E\'en weg is daar voor,,;}{myn Vrystaat of Transvaal toe daar}{is ik onbekend}\\

\haiku{Sy weet hoe David.}{moet voel as hy dit lees in}{die onderveld}\\

\haiku{Ver 'n man, wat deur ',!}{n ongeluk bankrot raak}{het ik reg jammer}\\

\haiku{Hulle wa'ens, karre.}{en negosiegoed was goed}{verkoop en verruil}\\

\haiku{dat hulle brandma'er.}{bowe gekom was en dit}{sou nie betaal nie}\\

\haiku{Juis dit doet jou hart,.}{eer an dat jy nou myn woord}{terug wil gewe}\\

\haiku{insuur en o'ersuur,;}{gaat al goed brood knee is nog}{bietjie swaar seg sy}\\

\section{Cor de Hoon}

\subsection{Uit: Bitter lemon}

\haiku{Hij dacht aan de twee.}{bedden en de spiegel met}{de kromme pootjes}\\

\haiku{Maar daar praten we{\textquoteright}.}{nog wel eens over als je in}{de vijfde klas zit}\\

\haiku{, maar voor hij naar de,.}{docentenkamer ging moest}{hij naar het toilet}\\

\haiku{Dat citeren is,{\textquoteright}.}{een beroepsziekte aan het}{worden geloof ik}\\

\haiku{Er was een pijnlijk.}{plezierige steekvlam door}{hem heengeschoten}\\

\haiku{Voorzichtig liet hij.}{zich terugzakken in zijn}{vorige houding}\\

\haiku{Bovendien had hij.}{at genoeg zorgen om zijn}{blindwordende ogen}\\

\haiku{De rector keek hem.}{vragend aan in afwachting}{van een verklaring}\\

\haiku{Vincent kan dan aan.}{het strand liggen terwijl ik}{de musea bezoek}\\

\haiku{Ik zag vanmiddag{\textquoteright}.}{een vlieg over de rand van een}{flatgebouw lopen}\\

\haiku{Terwijl hij daar stond}{begon de klas langzaam voor}{zijn ogen te draaien}\\

\haiku{Maar ik kan niks doen{\textquoteright}.}{met een valpeip die te kort}{afgeschneden is}\\

\haiku{Misschien was het toch.}{niet zo vreemd dat Fred er zich}{niet rijp voor voelde}\\

\haiku{Er zit te veel stof.}{in de lucht en dat is niet}{goed voor stoflongen}\\

\haiku{Er klonk een zweem van.}{trots in haar stem en de man}{glimlachte hijgend}\\

\haiku{Uit de struiken kwam,,.}{de egel te voorschijn bleef even}{staan zijn snuit omhoog}\\

\haiku{Het had geen zin in.}{bed te liggen en naar het}{plafond te staren}\\

\haiku{{\textquoteleft}Mijn vader zegt dat{\textquoteright}.}{een dokter nog aanzien heeft}{in de maatschappij}\\

\haiku{Vincent vouwde de,:}{kaart op schonk een glas whisky}{in en mompelde}\\

\haiku{De rector zegt dat{\textquoteright}.}{ik te veel de neiging heb}{om te citeren}\\

\haiku{Vincent voelde zich.}{verlegen en begon nog}{harder te zweten}\\

\haiku{Ze bogen samen,.}{het hoofd lachten geruisloos}{en dronken whisky}\\

\haiku{Zeg maar dat ik bij.}{de dokter geweest ben en}{lage bloeddruk heb}\\

\haiku{Hij grijnsde tegen,.}{zijn spiegelbeeld het grijnsde}{minzaam terug}\\

\haiku{Hij liep tussen de.}{rekken door op zoek naar de}{bleke jongeman}\\

\haiku{In de tropen aten.}{de meeste mensen flink en}{dronken overvloedig}\\

\haiku{De ergste hitte.}{zou nu langzaam wegebben}{uit het schoolgebouw}\\

\haiku{{\textquoteleft}O, ik dacht dat het{\textquoteright}.}{weer een voorbeeld was van jouw}{gevoel voor humor}\\

\haiku{Hij kleedde zich uit.}{in het donker en ging naakt}{op bed liggen}\\

\haiku{De laaste druppel.}{die er aan hing spatte uit}{elkaar in zijn oog}\\

\haiku{De bak liep vol en.}{zijn theorie werd door de}{praktijk bevestigd}\\

\haiku{Hij wist ook zeker.}{dat hij dit werk geen dag zou}{kunnen volhouden}\\

\haiku{Nou ja, we hebben.}{elkaar toch ook weer niet zo}{veel te vertellen}\\

\haiku{Het eerste ogenblik.}{deinsde Vincent terug toen}{hij de hal betrad}\\

\haiku{Was beschteld door zo'n.}{kakmadam die ieder jaar}{een nieuw toilet wil}\\

\haiku{De kans zat er dik.}{in dat sommige mensen}{overgeplaatst werden}\\

\haiku{Toen Eric aandrong en;}{haar wilde kussen duwde}{ze hem weg en zei}\\

\haiku{Met deze woorden.}{liet ze Eric staan en voegde}{zich weer bij de groep}\\

\haiku{Vincent wachtte tot.}{hij uitgesproken was en}{ging naar de keuken}\\

\haiku{Hij zou hem vragen.}{een borrel met hem te gaan}{drinken in de stad}\\

\haiku{Maar Vincent voelde.}{diep in zijn binnenste dat}{hij boete moest doen}\\

\haiku{Ik heb een hekel{\textquoteright}.}{aan die krantenberichten}{over Muis in Melkfles}\\

\haiku{Hij moest eigenlijk,,.}{niet meer rijden maar ja wat}{moeten we met hem}\\

\haiku{Hij zette zijn bril.}{af en trok zijn onderste}{oogleden omlaag}\\

\haiku{Ben je van plan om?}{ook over te schakelen op}{natuurlijk voedsel}\\

\haiku{of ze ook wel weer.}{te voorschijn kwamen als ze}{gepasseerd waren}\\

\haiku{Terwijl hij hier nog.}{mee bezig was ging bij de}{buren de deur open}\\

\haiku{Ik had een virus{\textquoteright}.}{infectie en mijn dikke}{darm was onrustig}\\

\haiku{Glas kraakte onder,.}{zijn voeten het wegdek was}{nat en glibberig}\\

\haiku{De routine van.}{dertig jaar kon niet zonder}{uitwerking blijven}\\

\haiku{De jongen in de.}{derde bank van de tweede}{rij trok zijn aandacht}\\

\haiku{Jou herken ik ook{\textquoteright},.}{zei hij tegen een jongen}{achter in de klas}\\

\haiku{not be ever again The,,{\textquoteright}.}{marked of many loved}{of one Gentlemen}\\

\haiku{{\textquoteleft}Waarom trekt U uw,,?}{jasje niet uit als U het}{zo warm hebt meneer}\\

\haiku{Het zou beter zijn.}{als er iemand bij was die}{hij kent en vertrouwt}\\

\haiku{En laatst belde de}{postbode bij ons omdat}{hij geen gehoor kreeg}\\

\haiku{{\textquoteleft}Ik zal jullie nog.}{\'e\'en sc\`ene laten horen}{en dan stoppen we}\\

\haiku{Uit het verwarde.}{geschreeuw ving hij nog slechts hier}{en daar een kreet op}\\

\haiku{Ik heb me \'e\'en keer,.}{naakt vertoond maar voortaan hou}{ik mijn kleren aan}\\

\haiku{Vincent wilde niet.}{meer naar hen luisteren en}{verliet zijn kamer}\\

\haiku{{\textquoteleft}But you must know, your,,{\textquoteright}.}{father lost a father that}{father lost lost his}\\

\haiku{Pas toen hij vlak voor,:}{Vincent stond keek hij op stak}{zijn hand uit en zei}\\

\haiku{Jouw vreemde gedrag.}{was natuurlijk een signaal}{van onbehagen}\\

\haiku{{\textquoteleft}Ik heb mijn hele,{\textquoteright}.}{opvatting over het leven}{gewijzigd Vincent}\\

\haiku{Dat zou Kyle of.}{Lochalsh kunnen zijn en}{dat Killiecrankie}\\

\subsection{Uit: Zij op het nachtkastje}

\haiku{Het ging tenslotte,.}{niet om de pati\"ent maar}{om rust en orde}\\

\haiku{Hij was opgestaan,.}{had zich aangekleed en was}{naar buiten gegaan}\\

\haiku{Hij keek nog eenmaal.}{naar de gang maar zag niets dat}{hem verontrustte}\\

\haiku{Het was in zijn ogen,.}{een Grote Zonde zo met}{de tijd te knoeien}\\

\haiku{De man vroeg zich af,.}{hoe lang het nog zou duren}{voor de dag aanbrak}\\

\haiku{Ze vroeg zich af wat.}{de man deed in die lange}{ziekenhuisnachten}\\

\haiku{Ze begon door de.}{nog bijna verlaten straat}{naar huis te lopen}\\

\haiku{Hij bestuurde zijn.}{parochie met vaste hand}{en genoot ervan}\\

\haiku{De man had in de;}{kudde een onbestemde}{plaats ingenomen}\\

\haiku{Hij had achteraf, {\textquoteleft}}{besloten dat de titel}{van haar boek moest zijn}\\

\haiku{Hij gluurde even, toen.}{hij het achteruit schuiven}{van haar stoel hoorde}\\

\haiku{Zo stond ze voor hem.}{in haar jurk en zo had ze}{haar nachtgewaad aan}\\

\haiku{een deur, die zo maar,,.}{ineens in het slot sprong een}{plank die zich voegde}\\

\haiku{Het was triest in een,.}{straat gewoond te hebben waar}{nooit iets was gebeurd}\\

\haiku{Zonder een kik te.}{geven zakte het als een}{pudding in elkaar}\\

\haiku{Deze liep de weg,,.}{op maar keek dan om om te}{zien waar zijn hond bleef}\\

\haiku{Het ritme van de.}{tijd werd hier bepaald door de}{thema's dag en nacht}\\

\haiku{Het was een hele.}{vooruitgang weer een hoofd op}{het kussen te zien}\\

\haiku{Alle gesprekken.}{verstomden en iedereen}{keek vol spanning toe}\\

\haiku{Niemand wist hoe ze,.}{heette wat ze dacht of wat}{ze met het bloed deed}\\

\haiku{Onder het scheren.}{had hij besloten te voet}{naar kantoor te gaan}\\

\haiku{het dennebos ging.}{over in loofhout en hij liet}{het bos achter zich}\\

\haiku{Die zou wel lauw zijn,}{en het was bovendien geen}{drank die geschikt was}\\

\haiku{Bovendien kon hij.}{de vrouw naast hem niet zo maar}{in de steek laten}\\

\haiku{Even bleef ze staan om.}{op adem te komen en zich}{te ori\"enteren}\\

\haiku{Hij had gezien, dat.}{ze een blouse droeg met een}{hele rij knoopjes}\\

\haiku{Hij wilde, dat ze.}{ophield met dat vervloekte}{heen en weer wiegen}\\

\haiku{Toch ging hij 's avonds.}{vaker dan gewoonlijk}{alleen wandelen}\\

\haiku{Waarom dit vreemde.}{mannetje liep te dwalen}{was hem ontschoten}\\

\haiku{Zijn vrouw zat nog steeds.}{in dezelfde houding over}{hem heen te staren}\\

\haiku{Het was kort na die,.}{nacht geweest waarin hij niet}{thuis gekomen was}\\

\haiku{Er kwam geen trein aan.}{en er was geen trein waar zij}{mee vertrekken kon}\\

\haiku{Het was duidelijk,.}{dat ik de hoofdpersoon van}{de bijeenkomst was}\\

\haiku{Wie weet hoe vaak dit.}{hoofd al boven de modder}{uitgekomen was}\\

\haiku{Fluweelachtig blauw,.}{en rood tere roze en}{zeegroene kleuren}\\

\haiku{In een oogwenk stond.}{er een hele verdieping}{op zijn bungalow}\\

\haiku{Ik wist niet of ik.}{kon helpen en waar ik het}{eerst naartoe moest gaan}\\

\haiku{Ze keken niet op.}{of om en lazen uit hun}{heilige boeken}\\

\haiku{De tafel, waaraan,.}{hij zat te werken zag er}{keurig netjes uit}\\

\haiku{Op het scherm waren.}{nu alle inwendige}{organen zichtbaar}\\

\haiku{Zo verdween hij  .}{in het inwendige van}{het medisch centrum}\\

\haiku{Het water was lauw.}{en ik bereikte zonder}{moeite het midden}\\

\haiku{Dit werd bevestigd.}{toen er een stroom van kritiek}{losbrak in de zaal}\\

\haiku{Als ik mezelf in,.}{veiligheid wilde brengen}{moest ik nu vluchten}\\

\haiku{Ik ben me er niet.}{van bewust dat er iets aan}{mijn paspoort mankeert}\\

\haiku{Hij is natuurlijk.}{een gevolg van alles wat}{ik meegemaakt heb}\\

\section{Roel Houwink}

\subsection{Uit: Novellen (1920-'22)}

\haiku{Tevergeefs zoeken.}{hun vingers steun tusschen het}{welkend tafelgroen}\\

\haiku{Eerst toen zij weken,:}{lag trok om haar zwerend lijf}{den zachten sluier}\\

\haiku{Maar toen de duinen,.}{waren weggevloeid in mist}{bukte  hij zich}\\

\haiku{Dagen ontweek hij,.}{de menschen hongerend om}{de norsche hoeven}\\

\haiku{Hoe zij hem vingen....}{strompelend door den laatsten}{akker voor de grens}\\

\haiku{Hij wordt gewaarschuwd,.}{en slentert ter zijde elk}{bewegen gestremd}\\

\section{Th\'er\`ese Hoven}

\subsection{Uit: Naar Holland en terug}

\haiku{Het Noodlot was hem,,.}{op wonderlijke wijze}{te hulp gekomen}\\

\haiku{de keuken vlak naast -.}{de eetkamer hoort toch in}{de bijgebouwen}\\

\haiku{Niet meer naar Indi\"e,.}{kunnen gaan dat was altijd}{in Holland blijven}\\

\haiku{Zoo, nu en dan deed, '.}{Jan maar als of hij dacht dat}{t een grapje was}\\

\haiku{jij hoeft je er niet,.}{op te verheugen want jij}{gaat zeker niet mee}\\

\haiku{{\textquoteright} Jan stond met een geeuw,,:}{op rekte zich een paar maal}{uit bromde iets van}\\

\haiku{Wat een baas toch, die, '.}{Njo van haar zoo deftige}{meneer int zwart}\\

\haiku{{\textquoteleft}Foei Gerarda, wat,.}{beoefen je slecht wat je}{zooeven gehoord hebt}\\

\haiku{{\textquoteright} vroeg de Kapitein,.}{die zich met zijn nichtjes had}{beziggehouden}\\

\haiku{iedereen weet toch,.}{zij een inlandsche vrouw maar}{hier heel wat anders}\\

\haiku{Ik had het al zoo,.....}{lang willen doen maar ik kon}{er niet toe komen}\\

\haiku{Een neef van haar, die,.}{naar Indi\"e ging had haar ten}{huwelijk gevraagd}\\

\haiku{De tijd, die nu kwam,.}{was een gezegende voor}{de arme L\'eonie}\\

\haiku{{\textquoteleft}Weet u.... begon hij...., {\textquoteleft}'....}{aarzelendt Is toch een}{lamme gewoonte}\\

\haiku{Mijn arme moeder,,....}{kassian vindt me een echt}{Hollandschen jongen}\\

\haiku{{\textquoteleft}Vindt u dat Pa het....?}{leven van Ma bedorven}{heeft of omgekeerd}\\

\haiku{In Indi\"e had ze,.}{nooit gewandeld in Holland}{zoo min mogelijk}\\

\haiku{in den Haag was 't,.}{Jan geweest die haar nog wel}{eens mee had getroond}\\

\haiku{je heele leven.}{te moeten boeten voor een}{dwaasheid van je jeugd}\\

\haiku{{\textquoteleft}Ik ga met Pa een,{\textquoteright}.}{voetreis in den Hartz maken}{kondigt de een aan}\\

\haiku{En Jan Weitinga,.}{gaat treurig naar huis omdat}{hem niets is beloofd}\\

\haiku{De berichten uit.}{Berchtesgaden werden hoe}{langer hoe slechter}\\

\haiku{{\textquoteright} {\textquoteleft}Maar zijn arme vrouw,,.}{die geen woord Duitsch kent en zijn}{jongen die hier is}\\

\haiku{Nee, als ik u een,.}{raad mag geven zou ik hem}{maar stil hier laten}\\

\haiku{Nou, 't beste zou,.}{dan maar zijn dat hij met zijn}{Ma naar Indi\"e ging}\\

\haiku{Een paarsroode blos....}{bedekte het gelaat der}{zedige Tony}\\

\haiku{zoo'n jongen van 18....,....}{jaar zonder vader met een}{inlandsche moeder}\\

\haiku{Naar frankfort is al?}{bijna een heele dag en}{dan nog naar M\"unchen}\\

\haiku{{\textquoteright} Nu is het de beurt.}{van den Notaris om er}{niets van te vatten}\\

\haiku{Ik wou eigenlijk,.}{eens met u overleggen wat}{het beste zou zijn}\\

\haiku{Ik ken ze wel niet,.}{allemaal maar beter dan}{jij kan er geen zijn}\\

\haiku{Ook nu wil ze, zoo, '.}{spoedig mogelijkt weer}{goed maken bij Jan}\\

\haiku{Maar L\'eonie kan er,,.}{bij verder nadenken niets}{aardigs in vinden}\\

\haiku{Thuis, als kind, Gondang,.}{Legi werd er niet veel acht}{op haar geslagen}\\

\haiku{Maar ze voelde zich.}{nog ongelukkiger en}{nog meer verlaten}\\

\haiku{{\textquoteright} Als zieke eenmaal, '....}{vindt niet lekker ist al}{bedorven voor hem}\\

\haiku{Hun verhouding van.}{broeder en zuster had toch}{niet kunnen duren}\\

\haiku{Zij kleedt zich nu weer,,?}{heelemaal op zijn Indisch}{toch zoo lekker ja}\\

\haiku{{\textquoteleft}Hoor eens, kameraad, '.}{lan we mekaar nu eens}{klaren wijn schenken}\\

\haiku{Als dokter moogt u,....}{niet oververtellen wat uw}{pati\"enten scheelt}\\

\haiku{ik ga naar huis, naar.}{L\'eonie dan en ik zet oogen}{en ooren goed open}\\

\haiku{Nu ja, als 't er,....}{op aan kwam had hij wel zijn}{zinnen bij elkaar}\\

\haiku{We blijven bij je,,,?}{tot neef Adriaan komt Majin}{als je het goed vindt}\\

\haiku{De laatste jaren;}{in Holland waren ook al}{heel ongelukkig}\\

\haiku{{\textquoteright} L\'eonie poogde de.}{schim van een lachje op haar}{gelaat te tooveren}\\

\haiku{Geen enkele man,,.}{zet zich zonder wrok of spijt}{over een blauwtje heen}\\

\haiku{{\textquoteright} {\textquoteleft}Voor mij was ze de,,.}{trouwste vriendin die ik ooit}{gehad heb Clara}\\

\haiku{Ze verzocht Majin:}{op een ochtend bij haar te}{komen en zei toen}\\

\haiku{Dan hadden wij ons.}{allebei verveeld en u}{hier in uw eentje}\\

\haiku{Ze lacht wel zoo en -.}{zegt lieve woordjes maar ze}{kan toch niet gelooven}\\

\haiku{Jammer, dat zij niet.}{wat vertrouwelijker met}{zijn moeder omging}\\

\haiku{Hij lachte - omdat '....:}{zet zoo raar zei zuchtte}{toen en fluisterde}\\

\haiku{Mijn vader heb ik;}{nooit gekend en Mama sprak}{bijna nooit over hem}\\

\haiku{ik was nog een kind,, '.}{ik wist niet wat liefde was}{ik weett nog niet}\\

\haiku{Voor beiden was 't,.}{een foltering ter wille}{van Jan doorgestaan}\\

\haiku{Ik dacht maar zoo, dat....}{u toch wel weer verlangen}{zoudt naar uw huisje}\\

\haiku{Menschen werden wel.}{eens gek van veel en vooral}{van moeielijk denken}\\

\haiku{{\textquoteleft}Maar nu, ik begrijp,?}{nog niet waarom jij haar in}{mijn huis wil brengen}\\

\haiku{De jonge moeder,;}{probeerde van alles en}{alles mislukte}\\

\haiku{{\textquoteright} Jan had er in al;}{dien tijd niet met zijn moeder}{over durven spreken}\\

\haiku{Jan, die den dokter,...}{had uitgelaten was weer}{binnengekomen}\\

\subsection{Uit: Van Koningsplein naar Gang Ket\`apan}

\haiku{Ze kwamen in den, '.}{schouwburg waart tamelijk}{vol en heel warm was}\\

\haiku{{\textquoteleft}Ja, werkelijk, en, '.}{vooral nu dat ikt niet}{meer behoef te doen}\\

\haiku{{\textquoteright} {\textquoteleft}Onverschillig, nee, '....}{maart Indische klimaat}{werkt zoo verslappend}\\

\haiku{{\textquoteleft}Waarom zijn jelui,,?}{in al dien tijd nooit eens naar}{Europa gegaan}\\

\haiku{Het meenemen van.}{een der kinderen was voor}{de convenances}\\

\haiku{Het was haar, als was,}{zij de bezitster van een}{kostbaren schat dien}\\

\haiku{ik zou willen' dat,....}{ik jou ontmoet had in plaats}{van Catherine}\\

\haiku{Waarop hem gevraagd,.}{werd wie die stommiteit op}{zijn rekening kreeg}\\

\haiku{Trouwens, die zuster,.}{is nog zoo lang geleden}{niet uitgekomen}\\

\haiku{Ik vind niet, dat we....}{nu juist in een stemming zijn}{om feest te houden}\\

\haiku{Pa was mooi boos, dat.}{je op je verjaardag niet}{eens thuis wou komen}\\

\haiku{{\textquoteright} {\textquoteleft}Ga je gang, als je,{\textquoteright},.}{er pleizier in hebt zei Jack}{op smalenden toon}\\

\haiku{thuis was of als de,}{kinderen er waren dan}{lachte ze wel weer}\\

\haiku{ze trapte er even,:}{met haar blooten voet op toen}{zei ze heel gewoon}\\

\haiku{op de scholen was;}{er voor de kleintjes gestrooid}{door een heuschen Sint}\\

\haiku{{\textquoteright} {\textquoteleft}Dat niet, maar Liekie....}{en ik hebben nog wel wat}{in onzen spaarpot}\\

\haiku{{\textquoteright} Gonne luisterde,;}{niet verder en ging er ook}{niet verder op door}\\

\haiku{Wat hoefde hij zich '....}{altijd muizenissen in}{t hoofd te halen}\\

\haiku{Gonne zei, dat ik, '.}{vroeg thuis moest zijn omdatt}{zoo vol is op straat}\\

\haiku{'t Was zoo leeg in, '....}{huis nat vertrek van haar}{man en van Liekie}\\

\haiku{Hij zou Liekie thuis,.}{brengen dat beteekende}{bij Catherine}\\

\haiku{en wilde hij 't,.}{pak niet weer meenemen moest}{hij haar wel volgen}\\

\haiku{{\textquoteright} In \'e\'en seconde.}{trachtte de zakenman den}{toestand te overzien}\\

\haiku{naar die andere........}{dat ze hem misschien in lang}{niet terug zou zien}\\

\haiku{Hij mocht haar niet meer,.}{steunen en och arme ze}{had dien steun zoo noodig}\\

\haiku{'t Was zijn zoon, hij,....}{moest hem vormen tot een braaf}{tot een bruikbaar mensch}\\

\haiku{hij was anders wel,.}{wat ongenaakbaar die heer}{en meester van haar}\\

\haiku{ze wilden een roes, '.}{en Catherine moestt}{gelag betalen}\\

\haiku{Een schok ging haar door ',!}{t lichaam een schaterlach}{ontsnapte haar keel}\\

\haiku{Ze streek zich met de ',....}{hand overt voorhoofd ze kon}{er niet aan gelooven}\\

\haiku{jammer, stomme zet,,.}{begreep nog niet hoe hij er}{toe gekomen was}\\

\haiku{Hij zag van uit de,.}{verte dat er overal licht}{aan was in zijn huis}\\

\haiku{{\textquoteleft}Pien moet der na toe,, '.}{als iemand nog der wat an}{kan doen ist Pien}\\

\haiku{Als toewan53 dokter,.}{obat geeft en niet gelpt dan laat}{Pien nenn\`eh komen}\\

\haiku{Ik sukkelde veel,,.}{ik zag er slecht uit ik was}{leelijk geworden}\\

\haiku{Die overtuiging, dat....}{ze hem haten moest en het}{bewustzijn dat ze}\\

\haiku{Zij, de willooze, de........}{apathische die hem van zich}{had laten weggaan}\\

\haiku{als je nu tot mij,.}{zoudt terugkeeren zou je even}{schuldig zijn als toen}\\

\haiku{En van de hoogste.}{illusie verzonk ze in}{de diepste wanhoop}\\

\haiku{Na dien nacht, wreed als,.}{een terechtstelling stortte}{Catherine in}\\

\haiku{{\textquoteright} {\textquoteleft}Ik begrijp, dat jij,!}{er naar snakt om van hier weg}{te komen maar o}\\

\haiku{Integendeel, ik,.}{zou je zoo dankbaar zijn als}{je mij met rust liet}\\

\haiku{{\textquoteleft}Ik kom obat brengen,,,.}{zoo goede obat wekt mevrouw}{gier heelemaal op}\\

\subsection{Uit: Zoo men zaait}

\haiku{na eenigen tijd zond,,}{hij haar naar Buitenzorg bij}{een dokter in huis}\\

\haiku{Nee, daar behoorde,,.}{meer moed toe dan hij op zijn}{leeftijd nog bezat}\\

\haiku{en toen volgde een '.}{opsomming vant geen ze}{zich herinnerde}\\

\haiku{Mijn arm kind, kon ik,,.}{je slechts een tehuis als mijn}{vrouwtje aanbieden}\\

\haiku{hij behandelt me,.}{zoo als een jongetje een}{burgerjongetje}\\

\haiku{Ze wist niet wat te,.}{zeggen ze kon zich toch niet}{aan hem opdringen}\\

\haiku{Meneer komt terug,,!}{met twee dochters de eene zoo}{mooie meisje prachtig}\\

\haiku{Maar.... om nu nog eens.}{op de bedden-quaestie}{terug te komen}\\

\haiku{We hebben, v\'o\'or ons,.}{vertrek een maand met hem in}{Parijs doorgebracht}\\

\haiku{{\textquoteleft}Nee,{\textquoteright} zeg mijn man, {\textquoteleft}veel,.}{te omslachtig en te duur}{zoo lang een wagen}\\

\haiku{ze had ook zoo vroeg.}{al de ellende van haar}{moeder meegemaakt}\\

\haiku{Ze zou er nu niet,.}{aan toegeven doch rechtstreeks}{op haar doel afgaan}\\

\haiku{Waarom zendt u die....,,?}{weduwe Janssens met haar}{troepje dan niet weg}\\

\haiku{Het was de tweede,.}{dien ze sedert haar aankomst}{in Indi\"e ontving}\\

\haiku{Ik ben, in vele,.}{opzichten niet zoo sterk en}{niet zoo flink als jij}\\

\haiku{vindt hij 't beter,:}{dat hij 2e klasse reist dan}{seint hij eenvoudig}\\

\haiku{als altijd, enkel.}{en alleen vervuld van haar}{eigen persoontje}\\

\haiku{Als 't er een van, '.}{Jack was geweest zou ikt}{ook gedaan hebben}\\

\haiku{Ik hoef er toch niet, ',}{bij te zetten dat ikm}{weggenomen heb}\\

\haiku{Het scheen, dat Nonna,,:}{Pien haar gedachte raadde}{ten minste ze zei}\\

\haiku{der zat haast niks in}{me kisten en ik kon me}{niks anschaffen ook}\\

\haiku{{\textquoteright} Liekie bracht hem tot.}{aan de voorgalerij en}{nam lachend afscheid}\\

\haiku{{\textquoteleft}En dan... denk er aan,.}{dat je niet te luidruchtig}{of te vroolijk bent}\\

\haiku{ze vond dat het te.}{fleurig en te frisch voor de}{gelegenheid was}\\

\haiku{die had haar, zoo ze,.}{op was natuurlijk naar de}{badkamer zien gaan}\\

\haiku{Ze zou er zich dus '.}{maar zelve van overtuigen}{en gingt huis in}\\

\haiku{En nu was ze z\'o\'o,.}{stil geweest dat Gonne er}{niets van gemerkt had}\\

\haiku{{\textquoteleft}Je weet toch, hoeveel '.}{eenvoudiger ikt in}{Holland gewend was}\\

\haiku{Je moet niet denken, '.}{dat ik totaal onbekend}{ben mett Maleisch}\\

\haiku{Ze moest voor hem de,.}{eerste zijn gelijk hij voor}{haar de eerste was}\\

\haiku{Henk.... heusch, tracht mij,.}{en mogelijk ook je zelf}{niet te misleiden}\\

\haiku{{\textquoteright} Hendrik de Berg was '.}{reeds naar buiten gesneld en}{naart rijtuig toe}\\

\haiku{De ongewone,....}{omgeving de slapende}{Nonna voor haar bed}\\

\haiku{Ik heb niets meer van ',{\textquoteright}.}{t leven te wachten zei}{Gonne gelaten}\\

\haiku{werd ze opgeschrikt.}{door haastige stappen in}{de voorgalerij}\\

\haiku{{\textquoteleft}Ik luisterde niet,....}{verder doch voelde me wel}{honderd pond lichter}\\

\haiku{Maar Jack dacht er niet,,,.}{aan ook maar \'e\'en haarbreed voor}{wie ook te wijken}\\

\haiku{Zij, die zelve hoog,.}{stond en achtenswaardig was}{had hem geminacht}\\

\section{Marijke H\"oweler}

\subsection{Uit: Bij ons schijnt de zon}

\haiku{Daar heb jij bij mijn.}{weten nooit zo over in de}{zorgen gezeten}\\

\haiku{Maar de bank was leeg.}{en Arnold liep ongerust}{naar hun slaapkamer}\\

\haiku{zo zit het,{\textquoteright} zei Roos,.}{toch enigszins tevreden over}{de samenvatting}\\

\haiku{{\textquoteleft}Nee ik bedoel dat,{\textquoteright}.}{van die huisgenoot van je}{zei mevrouw De Zeeuw}\\

\haiku{een huisgenoot van.}{een kennis van mij heeft last}{van aanstellerij}\\

\haiku{{\textquoteright} {\textquoteleft}H\`e, zou jij even het,.}{raam open willen zetten je}{ziet er zo warm uit}\\

\haiku{{\textquoteright} {\textquoteleft}Nee, dat bedoel ik,{\textquoteright}.}{niet wuifde mevrouw De Zeeuw}{haar ongemak weg}\\

\haiku{{\textquoteleft}Mijn ogen kleven en.}{mijn bril en mijn zakdoek en}{mijn broekzak kleven}\\

\haiku{{\textquoteleft}'n Jaar of zes,{\textquoteright} zei,.}{Roos wetend dat dit een wat}{lage schatting was}\\

\haiku{{\textquoteleft}Ja,{\textquoteright} zei moeder, {\textquoteleft}en,.}{voor je ophangt h\`e breng even}{wat geld voor me mee}\\

\haiku{{\textquoteleft}Het is jammer dat.}{de mensen tegenwoordig}{geen melk meer willen}\\

\haiku{{\textquoteright} vroeg Leo zo neutraal,.}{mogelijk ofschoon Roos het}{toch te gretig vond}\\

\haiku{Hij bekeek Rosa.}{en scheen daar inspiratie}{aan te ontlenen}\\

\haiku{{\textquoteright} {\textquoteleft}Maar waarom ga je.}{dan zo vaak naar haar toe als}{je niet van haar houdt}\\

\haiku{{\textquoteright} vroeg Arnold en liet.}{zich verbouwereerd op de}{eetkamerstoel neer}\\

\haiku{{\textquoteright} {\textquoteleft}Weet je wat je doen,,{\textquoteright}.}{moet in zo'n geval Arnold}{fluisterde Mattheus}\\

\haiku{{\textquoteright}        13 Het waarom}{van een innerlijk als men}{een uiterlijk heeft}\\

\haiku{deze was dat het.}{zo prettig verlopen was}{met de inboedel}\\

\haiku{{\textquoteleft}Je hebt er niets aan.}{om het allemaal zo hoog}{te willen spelen}\\

\haiku{{\textquoteright} Autorit\"atsgl\"aubig,.}{als Hugo was maakte het}{wel enige indruk}\\

\haiku{Dat had Hugo nog.}{niet zo gezien en hij keek}{Rosa vragend aan}\\

\haiku{wilde ik het ook,.}{nog met je over hebben ze}{heeft t\'och iets aardigs}\\

\haiku{En Leo begreep maar.}{al te goed wat zijn moeder}{daarmee bedoelde}\\

\haiku{Heb jij dat nou ook?}{soms dat je eenvoudige}{dingen niet verstaat}\\

\haiku{Maar moeder maakte.}{dat Leo niet aan kiezen toe}{hoefde te komen}\\

\haiku{{\textquoteleft}Dat zou dat moeder,{\textquoteright}.}{zich allerlei dingen in}{het hoofd haalt zei Leo}\\

\haiku{{\textquoteleft}Dat ze er niet van,.}{slapen kan dat ze er dag}{en nacht mee rondloopt}\\

\haiku{In de keuken sneed}{Rosa de uien en als}{men haar vragen zou}\\

\haiku{Arnold aarzelde.}{zoals hij de halve nacht}{had liggen dubben}\\

\haiku{Arnold probeerde.}{een snikachtig geluid in}{zijn stem te leggen}\\

\haiku{{\textquoteright} Arnold legde het.}{boterhamtrommeltje op}{bed en ging zitten}\\

\haiku{{\textquoteright} {\textquoteleft}Ga met \'ons mee,{\textquoteright} zei, {\textquoteleft}.}{Arnold spontaanwij willen}{naar het buitenland}\\

\haiku{{\textquoteleft}Wat enig om je nu,...}{eens in een gewoon kostuum}{te zien nou gew\'o\'on}\\

\haiku{{\textquoteright} {\textquoteleft}Koken, oh nee kind,,,,.}{voor geen goud ik kook nooit meer}{n\'o\'oit stel je voor zeg}\\

\haiku{zou ze w\'el, zou ze,?}{niet zou Van der Loo zijn mond}{dicht kunnen houden}\\

\haiku{Nu moest hij ze nog.}{met alle twee zijn voeten}{zien op te tillen}\\

\haiku{{\textquoteleft}Mattheus heeft nog maar,{\textquoteright}.}{pas z'n rijbewijs gehaald}{aarzelde Arnold}\\

\haiku{Sorry dat ik het,.}{zeg maar meer is het niet die}{schoonmoeder van jou}\\

\haiku{Toch keek Van der Loo '.}{zijn prachtige map int}{geheim nog even na}\\

\haiku{Al z\'o lang, dat ik.}{was gaan geloven dat ik}{het verkeerd begreep}\\

\haiku{Omdat ik wist dat.}{je vergeten zou om het}{aan ze te vragen}\\

\haiku{Ik wil niet dat je ',{\textquoteright}, {\textquoteleft}.}{r treitert zei Arnolddat}{wil ik niet hebben}\\

\haiku{In elk geval, Leo.}{was altijd al een beetje}{onstuimig geweest}\\

\haiku{{\textquoteright} {\textquoteleft}Oh lieve mamaatje,{\textquoteright}, {\textquoteleft}}{zongen ze verderzeg het}{niet tegen papaatje}\\

\haiku{Hij nam haar handen.}{in de zijne en streek er}{met zijn voorhoofd langs}\\

\haiku{{\textquoteleft}Wil je wel zorgen,{\textquoteright}.}{dat je met je vlerken van}{me afblijft zei Roos}\\

\haiku{En wat mevrouw De}{Zeeuw vooral zo waardeerde}{in de kinderen}\\

\haiku{Nee, mevrouw De Zeeuw.}{was niet ontevreden over}{haar investering}\\

\haiku{{\textquoteleft}Kun je misschien een?}{briefje geven waarop staat}{wat je geleend hebt}\\

\haiku{Bovendien moest men.}{een kind vroeg leren om iets}{af te kunnen staan}\\

\haiku{Dat maakte de sfeer.}{in die mooie Citroen er niet}{gezelliger op}\\

\haiku{Ze had zich een paar.}{jurken aangeschaft om dat}{te onderstrepen}\\

\haiku{{\textquoteleft}We moesten eigenlijk,{\textquoteright}.}{naar bed vond Mattheus en keek}{eens op naar Arnold}\\

\haiku{Dat die cognac blijft.}{stilstaan in je glas terwijl}{je ermee ronddraait}\\

\haiku{{\textquoteright} {\textquoteleft}Nee,{\textquoteright} gaf Hugo na,.}{enig nadenken toe het was}{een nieuw gezichtspunt}\\

\haiku{Zo, zo,{\textquoteright} zei mevrouw, {\textquoteleft}?}{De Zeeuwlopen jullie niet}{te hard van stapel}\\

\haiku{{\textquoteleft}Kijkt u eens hier,{\textquoteright} en.}{hij haalde een pakketje}{uit zijn binnenzak}\\

\haiku{{\textquoteright} En mevrouw De Zeeuw.}{wilde elk een pakketje}{in de hand stoppen}\\

\haiku{{\textquoteleft}Zo,{\textquoteright} zei Arnold toen.}{mevrouw De Zeeuw haar laatste}{slokje thee op had}\\

\haiku{Loof de Heer, dacht Leo,.}{Gij zijt de gezegende}{onder de vrouwen}\\

\haiku{En Thomas vroeg zich.}{af of hij jaloers was op}{Leo en dat was zo}\\

\subsection{Uit: Tobias}

\haiku{De kleinste ja, als,,.}{die zo oud was geweest maar}{de Tobias nee}\\

\haiku{Waar het alleen om,,.}{gaat dat is dat hij er niet}{gezien zal worden}\\

\haiku{Rachel hoort haar niet,,.}{die lijkt niet hier te zijn zo}{staart die voor zich heen}\\

\haiku{Hij haalt de schouders,:}{op kijkt haar dan aan en wil}{weer verder lopen}\\

\haiku{Dan sluipt hij naar de,,.}{deur links van de tap en doolt}{wat door de gangen}\\

\haiku{{\textquoteright} En Tobias voelt.}{het paard onder zich rillen}{en geeft de sporen}\\

\haiku{dat hoef je niet te,{\textquoteright},.}{zeggen en hij neemt de lap}{legt die op de grond}\\

\haiku{{\textquoteright} {\textquoteleft}Dat kan wel zijn, maar,,}{nooit zo achterlijk als de}{Tobias dat blijkt}\\

\haiku{Ik ben al ruim vier.}{jaar gered en elk jaar is}{een jaar geredder}\\

\haiku{Ze keken wel mooi.}{uit om ze een strobreed in}{de weg te leggen}\\

\haiku{Hoeveel  moet ik?}{jou bieden als hij vannacht}{bij jou zou slapen}\\

\haiku{Ik heb er wel vijf ',{\textquoteright}, {\textquoteleft}.}{gezien diet konden houdt}{Joshua volof zes}\\

\haiku{Als Belle er maar,.}{een beetje tevreden mee}{was met die uitleg}\\

\haiku{Een theepot en een,,.}{paar boeken met een touwtje}{erom komen mee}\\

\haiku{{\textquoteright} {\textquoteleft}Trouwen zeker,{\textquoteright} snikt,.}{Judith maar ze laat nu haar}{arm tenminste los}\\

\haiku{En nu hij haar kust,.}{blijken zijn tranen zouter}{dan de hare}\\

\haiku{Dat kwam, omdat hij.}{vergeten was dat hij niet}{met paard en kar was}\\

\haiku{{\textquoteleft}Waar was je dan zo,{\textquoteright}, {\textquoteleft}?}{bezorgd over vraagt zewaar til}{je dan zo zwaar aan}\\

\haiku{{\textquoteright} {\textquoteleft}Dat weet ik niet,{\textquoteright} zegt.}{Tobias en ze lacht en}{strijkt hem door zijn haar}\\

\haiku{{\textquoteleft}Maak open,{\textquoteright} zegt ze, {\textquoteleft}en.}{kijk of je niets hebt wat als}{een doek kan dienen}\\

\haiku{{\textquoteright} {\textquoteleft}Ach God,{\textquoteright} zegt ze, {\textquoteleft}staat.}{de Tobias jou daar voor}{niets op te wachten}\\

\haiku{{\textquoteright} Ze knikken, vegen.}{het laatste brood nog van hun}{mond en staan dan op}\\

\haiku{{\textquoteright} {\textquoteleft}De Joshua is al,.}{in het dorp de Joshua die}{kan toch niet zwijgen}\\

\haiku{Het klinkt als de wind.}{waarin men alles horen}{kan wat of men vreest}\\

\haiku{Nooit was ze  nog.}{zo ongerust geweest over}{geluid dat uitbleef}\\

\haiku{Ooit,{\textquoteright} zegt ze, {\textquoteleft}heb ik,.}{veel gepraat voornamelijk}{tegen de dingen}\\

\haiku{{\textquoteright} Maar Tobias schijnt.}{niet te willen begrijpen}{waar hun redding ligt}\\

\haiku{Hij is bewaker,.}{zogenaamd dat niemand het}{mijnenveld betreedt}\\

\haiku{{\textquoteleft}Nee,{\textquoteright} zegt hij dan, {\textquoteleft}jou,{\textquoteright}.}{laat ik niet weer gaan en kijkt}{weer naar zijn handen}\\

\haiku{Als Judith haar het}{water brengt en het nadien}{weer op wil halen}\\

\haiku{{\textquoteright} {\textquoteleft}Een heks, een heks,{\textquoteright} zegt}{Joshua en aan de dekens}{valt te zien hoe kort}\\

\subsection{Uit: Tranen van niemand}

\haiku{Kijk dan naar die kast,,?}{tegenover je die wil je}{toch hebben nietwaar}\\

\haiku{Winnie sloeg de ogen.}{neer en begon zedig haar}{taartje op te eten}\\

\haiku{En langzamerhand,}{werden de bomen kaal want}{hoe groter de plicht}\\

\haiku{Hij liep de deur uit,,}{deed zijn jas aan en voordat}{ze precies begreep}\\

\haiku{Het lag ook wel aan,;}{mij ik kon nooit eens meedoen}{als zij gek deden}\\

\haiku{Zij hoorde hem zijn.}{step uit de schuur halen en}{knarsend wegrijden}\\

\haiku{III Zijn kind had nooit,.}{geleefd voor hem als nu nu}{het gestorven was}\\

\haiku{Waarom wilden wij,?}{ons voortzetten zijn wij z\'o}{op ons zelf gesteld}\\

\haiku{Marie-Jeanne,,,.}{gaat het nog het is nu tijd}{ik kom zo terug}\\

\haiku{Kom, zei het meisje,,}{laten we ons plezier niet}{laten bederven}\\

\haiku{Nee, zei hij, laten.}{we eerst gaan kijken hoe de}{mensen het hier doen}\\

\haiku{Ze keken in de,.}{emmer hoogmoedig volgde}{de man hun blikken}\\

\haiku{De obers zagen het.}{en namen niet de moeite}{ze weg te jagen}\\

\haiku{Het was vier uur, hij.}{rekte zich en bleef voor zich}{uit liggen kijken}\\

\haiku{He, doe nou niet zo,,,!}{triest we zijn hier toch voor ons}{plezier zei ze kom}\\

\haiku{Laat kwamen ze aan,,,}{iedereen sliep ze maakten}{de leider wakker}\\

\haiku{de voorzitter ons}{op en vroeg of wij  het}{gevonden hadden}\\

\haiku{, ik werd voorgesteld,.}{aan zijn vrouw gewogen en}{te licht bevonden}\\

\haiku{Misschien niet vandaag,,.}{maar morgen zeker hoorde}{ik mezelf zeggen}\\

\haiku{Ik weet niet, begon,,;}{ze ik ben bang dat hij niet}{goed ziet en dan snel}\\

\haiku{We weten het niet,.}{het is zo hopeloos ver}{van huis allemaal}\\

\haiku{De recensie van,,.}{een toneelstuk enkele}{films een cabaret}\\

\haiku{K\'erels, dan kon je.}{echt horen dat ze de pest}{had aan het leven}\\

\haiku{Evengoed vroeg je je,}{af hoe zo'n griet zo stom kon}{wezen dat wordt vast}\\

\haiku{Ach schat, zei ze, trek,?}{je er niks van aan je heb}{het nou toch lekker}\\

\subsection{Uit: Van geluk gesproken}

\haiku{{\textquoteright} Zij bedoelde dan.}{niet alleen de lucht maar zijn}{hele verschijning}\\

\haiku{Soms  ontbrak de}{hele week het toetje aan}{moeders eten nu eens}\\

\haiku{Het lag Wouter op.}{de tong om te vertellen}{wat hij van plan was}\\

\haiku{Martijn heeft een 6.}{voor staatsrecht en ik heb een}{4 voor statistiek}\\

\haiku{Eerst klonk er geklos.}{op de trap en daarna de}{stem van meneer Kalk}\\

\haiku{{\textquoteleft}Dat heb jij toch al,{\textquoteright}.}{gedaan zei Martijn en dat}{was natuurlijk zo}\\

\haiku{{\textquoteright} {\textquoteleft}Als jij m'n petje ',}{van de achterbank haalt dan}{doe ikt dak open}\\

\haiku{{\textquoteleft}Ik wilde niet aan,{\textquoteright}.}{\'e\'en stuk doorrijden zei Leo}{verontschuldigend}\\

\haiku{Hij kocht zelfs nog een {\textquoteleft}{\textquoteright}.}{flesje wijn omthuis bij de}{tv op te drinken}\\

\haiku{Hij ging daarom maar.}{aan de grote tafel in}{het midden zitten}\\

\haiku{Het bericht kwam toch:}{niet helemaal goed door want}{meneer de Bruin riep}\\

\haiku{Jij moet je er niet.}{mee bemoeien als je er}{geen verstand van hebt}\\

\haiku{{\textquoteright} {\textquoteleft}Ik mag geen verstand,}{van stereo hebben maar van}{jou weet ik alles}\\

\haiku{{\textquoteright} {\textquoteleft}Geen nieuws,{\textquoteright} zei De Bruin.}{en probeerde zich uit de}{voeten te maken}\\

\haiku{Thomas beloofde,.}{Thomas bezwoer hen dat hij}{zich beheersen zou}\\

\haiku{{\textquoteright} Thomas keek op en.}{zag hoe Martje de tafel aan}{het afruimen was}\\

\haiku{{\textquoteleft}Laat toch staan kindje,,.}{morgen gaan we verder ga}{toch lekker naar bed}\\

\haiku{Ik geloof dat we,{\textquoteright}.}{beter naar bed kunnen gaan}{zei Thomas praktisch}\\

\haiku{Dat kunt u zich niet,{\textquoteright}.}{voorstellen zei mevrouw de}{Bruin nu iets kalmer}\\

\haiku{{\textquoteleft}Heeft hij u,{\textquoteright} Martijn, {\textquoteleft}?}{kon het woord maar niet vinden}{geslagen of zo}\\

\haiku{{\textquoteright} zei ze toen weer en.}{voor Martijn op kon staan was}{het haar al gelukt}\\

\haiku{Barbara had op:}{zich genomen met Martje te}{praten of beter}\\

\haiku{{\textquoteleft}Ik ga even vragen,{\textquoteright}.}{of ze iets wil zei Thomas}{en liep naar boven}\\

\haiku{Niettemin deed het.}{hem goed in de vertrouwde}{omgeving te zijn}\\

\haiku{Leo scheurde door de.}{drie bochten welke zijn huis}{van zijn werk scheidden}\\

\haiku{Van vrouwen wist je.}{eigenlijk nooit hoe preuts ze}{nou precies waren}\\

\haiku{Hij had daar nooit veel.}{goeds van verwacht want van het}{een kwam het ander}\\

\haiku{Nog nooit had Martje een!}{snackbar gezien met zoveel}{Telegraaflezers}\\

\haiku{Daardoor kwam het ook.}{dat ze de auto's en zelfs}{de bus niet hoorde}\\

\haiku{{\textquoteright} zei Leo geschrokken.}{en voelde driekwart van zijn}{energie wegzakken}\\

\haiku{{\textquoteleft}Welnee kerel,{\textquoteright} zei.}{Van Zutphen en zwaaide de}{rook uit z'n buurt weg}\\

\haiku{Hij legde z'n hand' {\textquoteleft},,}{op Thomas mouw en schonk diens}{glas vol.Hier drink op}\\

\haiku{{\textquoteright} {\textquoteleft}Nee,{\textquoteright} zei Thomas, {\textquoteleft}uit,,.}{haar briefje v\'o\'ordat ze naar}{ons toekwam bleek niets}\\

\haiku{{\textquoteleft}Ik heb ook nog aan,{\textquoteright}.}{de studentenpsycholoog}{gedacht zei Thomas}\\

\haiku{Nee, niet wat me te,,.}{wachten staat wat er gebeurt}{tegelijk gebeurt}\\

\haiku{{\textquoteright} {\textquoteleft}Nee,{\textquoteright} zei Roos, {\textquoteleft}dat hoeft,.}{niet ik hoef het deze keer}{niet meer te horen}\\

\haiku{Martje maakte zich los.}{van haar eigen avontuur en}{knikte naar Geesje}\\

\haiku{Ik kan toch beter,.}{gewoon bij m'n verslaafde}{zootje blijven dacht Gees}\\

\haiku{En Thomas was er.}{al helemaal niet voor in}{het goede humeur}\\

\haiku{{\textquoteleft}Meisje,{\textquoteright} zei Thomas, {\textquoteleft},{\textquoteright}.}{wat fijn dat je meeging en}{verder zwegen ze}\\

\haiku{'t Is trouwens niet,,!}{warm ook schiet eens een beetje}{op ouwe jongen}\\

\haiku{Net op tijd om het.}{eten om zes uur op tafel}{te kunnen zetten}\\

\haiku{{\textquoteright} Leo wist zich geen raad,.}{en ging maar weer eens achter}{z'n bureau zitten}\\

\haiku{{\textquoteright} {\textquoteleft}Nee,{\textquoteright} zei Thomas, {\textquoteleft}weet?}{je waarom het een rare}{vraag is geworden}\\

\haiku{{\textquoteright} Barbara stond nog.}{steeds met de sandwiches en}{de limonade}\\

\haiku{Maar ze had het nog.}{niet opgeborgen of het}{meisje wilde m\'e\'er}\\

\haiku{Misschien een wieg of....}{luiers of Leo's hoofd liep}{om van de zorgen}\\

\haiku{{\textquoteleft}Dag Anke,{\textquoteright} zei Leo, {\textquoteleft}?}{vriendelijkwat heb je voor}{me te doen vandaag}\\

\haiku{En mevrouw Elisa {\textquoteleft}{\textquoteright}.}{wiegde het kindbiem-bahm}{op het klokgelui}\\

\haiku{Nee meneer Evers, niet,...}{doen komt u hier even zitten}{en neemt u meneer}\\

\haiku{Het was een eindje,,.}{om en ja zo was het het}{was een eindje om}\\

\haiku{{\textquoteright} En ze begon aan,.}{Wouter te trekken die heel}{langzaam overeind kwam}\\

\haiku{Dus De Bruin knikte.}{maar een beetje dat het een}{goed idee van haar was}\\

\haiku{{\textquoteleft}En je potloodje,{\textquoteright},.}{zei De Bruin maar dat verzon}{hij in zijn overmoed}\\

\haiku{{\textquoteright} zei de man die het.}{telefoonnummer bijna}{goed had gelezen}\\

\haiku{We zouden misschien,,}{het huis in twee\"en kunnen}{delen jij boven}\\

\subsection{Uit: Van geluk gesproken}

\haiku{{\textquoteright} Zij bedoelde dan.}{niet alleen de lucht maar zijn}{hele verschijning}\\

\haiku{Soms  ontbrak de}{hele week het toetje aan}{moeders eten nu eens}\\

\haiku{Het lag Wouter op.}{de tong om te vertellen}{wat hij van plan was}\\

\haiku{Martijn heeft een 6.}{voor staatsrecht en ik heb een}{4 voor statistiek}\\

\haiku{Eerst klonk er geklos.}{op de trap en daarna de}{stem van meneer Kalk}\\

\haiku{{\textquoteleft}Dat heb jij toch al,{\textquoteright}.}{gedaan zei Martijn en dat}{was natuurlijk zo}\\

\haiku{{\textquoteright} {\textquoteleft}Als jij m'n petje ',}{van de achterbank haalt dan}{doe ikt dak open}\\

\haiku{{\textquoteleft}Ik wilde niet aan,{\textquoteright}.}{\'e\'en stuk doorrijden zei Leo}{verontschuldigend}\\

\haiku{Hij kocht zelfs nog een {\textquoteleft}{\textquoteright}.}{flesje wijn omthuis bij de}{tv op te drinken}\\

\haiku{Hij ging daarom maar.}{aan de grote tafel in}{het midden zitten}\\

\haiku{Het bericht kwam toch:}{niet helemaal goed door want}{meneer de Bruin riep}\\

\haiku{Jij moet je er niet.}{mee bemoeien als je er}{geen verstand van hebt}\\

\haiku{{\textquoteright} {\textquoteleft}Ik mag geen verstand,}{van stereo hebben maar van}{jou weet ik alles}\\

\haiku{{\textquoteright} {\textquoteleft}Geen nieuws,{\textquoteright} zei De Bruin.}{en probeerde zich uit de}{voeten te maken}\\

\haiku{Thomas beloofde,.}{Thomas bezwoer hen dat hij}{zich beheersen zou}\\

\haiku{{\textquoteright} Thomas keek op en.}{zag hoe Martje de tafel aan}{het afruimen was}\\

\haiku{{\textquoteleft}Laat toch staan kindje,,.}{morgen gaan we verder ga}{toch lekker naar bed}\\

\haiku{Ik geloof dat we,{\textquoteright}.}{beter naar bed kunnen gaan}{zei Thomas praktisch}\\

\haiku{Dat kunt u zich niet,{\textquoteright}.}{voorstellen zei mevrouw de}{Bruin nu iets kalmer}\\

\haiku{{\textquoteleft}Heeft hij u,{\textquoteright} Martijn, {\textquoteleft}?}{kon het woord maar niet vinden}{geslagen of zo}\\

\haiku{{\textquoteright} zei ze toen weer en.}{voor Martijn op kon staan was}{het haar al gelukt}\\

\haiku{Barbara had op:}{zich genomen met Martje te}{praten of beter}\\

\haiku{{\textquoteleft}Ik ga even vragen,{\textquoteright}.}{of ze iets wil zei Thomas}{en liep naar boven}\\

\haiku{Niettemin deed het.}{hem goed in de vertrouwde}{omgeving te zijn}\\

\haiku{Leo scheurde door de.}{drie bochten welke zijn huis}{van zijn werk scheidden}\\

\haiku{Van vrouwen wist je.}{eigenlijk nooit hoe preuts ze}{nou precies waren}\\

\haiku{Hij had daar nooit veel.}{goeds van verwacht want van het}{een kwam het ander}\\

\haiku{Nog nooit had Martje een!}{snackbar gezien met zoveel}{Telegraaflezers}\\

\haiku{Daardoor kwam het ook.}{dat ze de auto's en zelfs}{de bus niet hoorde}\\

\haiku{{\textquoteright} zei Leo geschrokken.}{en voelde driekwart van zijn}{energie wegzakken}\\

\haiku{{\textquoteleft}Welnee kerel,{\textquoteright} zei.}{Van Zutphen en zwaaide de}{rook uit z'n buurt weg}\\

\haiku{Hij legde z'n hand' {\textquoteleft},,}{op Thomas mouw en schonk diens}{glas vol.Hier drink op}\\

\haiku{{\textquoteright} {\textquoteleft}Nee,{\textquoteright} zei Thomas, {\textquoteleft}uit,,.}{haar briefje v\'o\'ordat ze naar}{ons toekwam bleek niets}\\

\haiku{{\textquoteleft}Ik heb ook nog aan,{\textquoteright}.}{de studentenpsycholoog}{gedacht zei Thomas}\\

\haiku{Nee, niet wat me te,,.}{wachten staat wat er gebeurt}{tegelijk gebeurt}\\

\haiku{{\textquoteright} {\textquoteleft}Nee,{\textquoteright} zei Roos, {\textquoteleft}dat hoeft,.}{niet ik hoef het deze keer}{niet meer te horen}\\

\haiku{Martje maakte zich los.}{van haar eigen avontuur en}{knikte naar Geesje}\\

\haiku{Ik kan toch beter,.}{gewoon bij m'n verslaafde}{zootje blijven dacht Gees}\\

\haiku{En Thomas was er.}{al helemaal niet voor in}{het goede humeur}\\

\haiku{{\textquoteleft}Meisje,{\textquoteright} zei Thomas, {\textquoteleft},{\textquoteright}.}{wat fijn dat je meeging en}{verder zwegen ze}\\

\haiku{'t Is trouwens niet,,!}{warm ook schiet eens een beetje}{op ouwe jongen}\\

\haiku{Net op tijd om het.}{eten om zes uur op tafel}{te kunnen zetten}\\

\haiku{{\textquoteright} Leo wist zich geen raad,.}{en ging maar weer eens achter}{z'n bureau zitten}\\

\haiku{{\textquoteright} {\textquoteleft}Nee,{\textquoteright} zei Thomas, {\textquoteleft}weet?}{je waarom het een rare}{vraag is geworden}\\

\haiku{{\textquoteright} Barbara stond nog.}{steeds met de sandwiches en}{de limonade}\\

\haiku{Maar ze had het nog.}{niet opgeborgen of het}{meisje wilde m\'e\'er}\\

\haiku{Misschien een wieg of....}{luiers of Leo's hoofd liep}{om van de zorgen}\\

\haiku{{\textquoteleft}Dag Anke,{\textquoteright} zei Leo, {\textquoteleft}?}{vriendelijkwat heb je voor}{me te doen vandaag}\\

\haiku{En mevrouw Elisa {\textquoteleft}{\textquoteright}.}{wiegde het kindbiem-bahm}{op het klokgelui}\\

\haiku{Nee meneer Evers, niet,...}{doen komt u hier even zitten}{en neemt u meneer}\\

\haiku{Het was een eindje,,.}{om en ja zo was het het}{was een eindje om}\\

\haiku{{\textquoteright} En ze begon aan,.}{Wouter te trekken die heel}{langzaam overeind kwam}\\

\haiku{Dus De Bruin knikte.}{maar een beetje dat het een}{goed idee van haar was}\\

\haiku{{\textquoteleft}En je potloodje,{\textquoteright},.}{zei De Bruin maar dat verzon}{hij in zijn overmoed}\\

\haiku{{\textquoteright} zei de man die het.}{telefoonnummer bijna}{goed had gelezen}\\

\haiku{We zouden misschien,,}{het huis in twee\"en kunnen}{delen jij boven}\\

\section{J. Huf van Buren en Maurits Sabbe}

\subsection{Uit: De twee invasies. Hoe grootvaders broer uit den oost weerkeerde}

\haiku{En 't is ook een'.}{ijselijke bazar op}{straat met al da volk}\\

\haiku{Een luide lach, en {\textquoteleft},!}{eenmerci monsieur ne}{vous d\'erangez pas}\\

\haiku{Water in korten,;}{tijd gesproken werd zal ik}{hier niet herhalen}\\

\haiku{Albertine gaf;}{ten antwoord dat hij dan maar}{bij haar blijven moest}\\

\haiku{Dan spijt het mij dat,.}{het zoo geloopen is zei}{de jonge baron}\\

\haiku{Ik wil  niet, dat.}{er ooit over gesproken of}{om getreurd worde}\\

\section{Gerard van Hulzen}

\subsection{Uit: De ontredderden. Eerste bundel}

\haiku{Wie zal de lijn hier?}{trekken en wie heeft het bij}{het goede eind}\\

\haiku{'t Ging belabberd,, ',.}{vanochtend merkte-iet}{dadelijk al goed}\\

\haiku{- Welja, ik mot maar,.}{altoos hellepe weerde}{hij onwillig af}\\

\haiku{Het Jantje voelde,,.}{toch iets ervan bleef staan en}{keek om verwonderd}\\

\haiku{Ze schudde haar door,.}{elkaar om haar gauwer te}{laten bekennen}\\

\haiku{- Nou jij of je broer...}{dacht je soms dat ik kledder}{in m'n ooren heb}\\

\haiku{- Jawel, je hebt 'et,.}{maar voor kommandeere schampte}{Hein brutaal terug}\\

\haiku{Betje bedelde.}{nu alleen en waagde zich}{weer in de avondstraat}\\

\haiku{- Ja zeker, ik meen,.}{het we kunnen je toch het}{heele jaar niet ho\^ue}\\

\haiku{Op zijn verslonsde;}{gele haren kleefde een}{beetje schuin de pet}\\

\haiku{Voor de bewoners '.}{waren die schelknoppen in}{t geheel niet noodig}\\

\haiku{Hij bromde nog eens,.}{hum en zij werkte en wreef}{weer voort aan haar stoel}\\

\haiku{- Je kon anders best,.}{wachten tot ik weg was dan}{heb je de ruimte}\\

\haiku{- Hier heb je de prijs.}{die je koopt en hier is de}{primie die je wint}\\

\haiku{Nou hoef ik je niks.}{meer te zeggen wat bloemen}{wel tot stand brengen}\\

\haiku{Hij rekte dit woord,,:}{primie rekte opnieuw de}{woorden en bralde}\\

\haiku{'t Gaat hier net as,! '}{bij de staatsloterij zoo}{eerelik als goud}\\

\haiku{Maar ze liet zich niet,;}{verdringen duwde met haar}{achterste terug}\\

\haiku{t Is heelemaal......?}{zes-en-twintig cente}{wat wou je d'ervan}\\

\haiku{Ze had het hem nog,.}{niet verweten nee dat moest}{ze ook eens lappen}\\

\haiku{Mies, nu ook wakker,.}{geworden kwam eruit en}{kuste haar moeder}\\

\haiku{O, hij begreep het, ',.}{t kwam door dat vriesweer dat}{alles zoo opklonk}\\

\haiku{zijn oogen staarden blind.}{tegen haar strak-harde rug}{als tegen een muur}\\

\haiku{Dan kwakte hij de,:}{gevulde schepper in de}{bak terug gromde}\\

\haiku{zeg ik je, as we,.}{op straat komme te staan trek}{ik ertusschen uit}\\

\haiku{een loopende hond.}{valt allicht wat in de mond}{was toch het spreekwoord}\\

\haiku{En toch 't moest, als, '.}{hij langer wachtte gingt}{heelemaal niet meer}\\

\haiku{Hij liet zijn jas los,,:}{die op de stoel gleed schreeuwde}{dan ineens gedurfd}\\

\haiku{t vriest toch veel te...,!!}{hard om te kunne plakke}{ja maak mijn dat wijs}\\

\haiku{Hij holde door, was,.}{de onderste trap al af}{buiten haar bereik}\\

\haiku{Wat kocht ze in al,, '.}{die tijd bij een bedroefd drupje}{t meeste voor h\`em}\\

\haiku{Thuis hadden ze 't,.}{niet breed maar de kast zag er}{toch behoorlijk uit}\\

\haiku{De kwaadheid, zoo lang,,.}{bedwongen ziedde brak uit}{naar alle kanten}\\

\haiku{je zwijgt en zwijgt, doet ',.}{of jet niet merkt om de}{vree te bewaren}\\

\haiku{Ach, ach, wat maakte,?}{een vrouw niet mee wat kreeg ze}{al niet te doorstaan}\\

\haiku{wel tien keer streek ze ',;}{t zelfde vouwtje uit en}{haar oogen staarden blind}\\

\haiku{Een oogenblik dacht,,.}{ze dat Baller er aan kwam}{maar ze zag verkeerd}\\

\haiku{Ze voelde haar hoofd.}{sufzwaar worden van al dat}{moeizaam overleggen}\\

\haiku{Van zoo'n steggel, zoo'n.}{penningfokker had ze niet}{veel te verwachten}\\

\haiku{Lekker zou-ie haar ',!}{noues troeven nou had ze}{net niks te zeggen}\\

\haiku{Een vermoeden van,.}{vuilheid golfde in haar op}{maakte haar dol}\\

\haiku{Gejacht liep ze de,.}{straat ten einde zonder te}{beseffen waarheen}\\

\haiku{Een huiver van kou,.}{en killigheid omving haar}{drong door alles heen}\\

\haiku{In elk geval eerst!}{die zak ergens neerzetten}{en dan verder zien}\\

\haiku{Wie kan dat zeggen, '....}{t gaat van zelf bijna als}{geboren-worden}\\

\haiku{- Wat is er gebeurd,,.}{herhaalde Greet om haar tong}{wat los te krijgen}\\

\haiku{'t Viel haar on-. '}{noemlijk zwaar tot haar ouwe}{doen terug te keeren}\\

\haiku{De glad gevroren.}{straten maanden haar weer aan}{tot voorzichtig gaan}\\

\haiku{Ze hoorde of zag,.}{niets meer van wat om haar heen}{liep keek niet meer uit}\\

\haiku{O, hij zou zich wel,,!}{redden hij had haar niet noodig}{o nee volstrekt niet}\\

\haiku{Wat gaf het of hij,.}{hier al vuur aanle{\^\i} als ze}{toch niet opdaagde}\\

\haiku{Het voorgevoel nam '.}{in een paar telt begrip}{van zekerheid aan}\\

\haiku{De mannen raakten,.}{in verwarring wisten niet}{wat ze moesten zeggen}\\

\haiku{De werkelijkheid.}{leek hem minder erg dan al}{die grijnsgezichten}\\

\haiku{Waarom wou-ie zich?}{toch in eigen oogen beter}{maken dan hij was}\\

\haiku{De kroegen zouden,.}{gauw gaan sluiten nu kon hij}{nog eentje nemen}\\

\haiku{Achter hem in 't.}{dorre hout meende hij te}{hooren ritselen}\\

\haiku{Een afgezakte,......}{een afgetrapte was-ie}{kostte geld voor niets}\\

\subsection{Uit: De ontredderden. Tweede bundel}

\haiku{Nu werd zijn aandacht ' '.}{getrokken doorn groepje}{menschen opt strand}\\

\haiku{Met 'n enkele ',:}{handstoot duwde hijn paar}{jongens op zij riep}\\

\haiku{Dat was een leelijk,.}{geval alles behalve}{een buitenkansje}\\

\haiku{Maar terwijl hij zich.}{dit inpraatte voelde hij}{zich  niet zeker}\\

\haiku{'t Zou 't beste.}{zijn even te gaan kijken om}{zich te overtuigen}\\

\haiku{- Daarom hoef je niet,...!}{dadelijk te ranselen}{je lijkt wel een beul}\\

\haiku{- Wat zeur je dan... je '...}{mott natuurlik zoolang}{mogelijk volho\^ue}\\

\haiku{'t Was 't relaas,,.}{dat ze elkaar aldoor voor}{ze{\^\i}en al jaren}\\

\haiku{- Ik denk 't wel... we...}{kunne natuurlik bij de}{direktie probeere}\\

\haiku{Naar alle kanten,.}{vluchtten de menschen in trams}{in koffiehuizen}\\

\haiku{Hij voelt de raakheid,.}{van die woorden doch wil er}{niet aan toegeven}\\

\haiku{De vrouw, stil in het,.}{achtervertrek voelt een glimp}{van verheugenis}\\

\haiku{Nou  hebben ze,,.}{de tijd blijven ze plakken}{verteren toch niet}\\

\haiku{Zij wil wel weg, blijft '.}{toch staan en weet niet hoe ze}{t aanleggen zal}\\

\haiku{Affijn, eerst probeeren -!}{het zaakje te verkoopen}{en dan verder zien}\\

\haiku{Weer ging een schijfje;}{appel over de lip van de}{meneer naar binnen}\\

\haiku{Nee, niet dadelijk,.}{gaan als je de centen hebt}{zei hij bij zichzelf}\\

\haiku{kijk wat 'n lef, liet,.}{daarop de oogen weer zakken}{als bedacht hij zich}\\

\haiku{Onderwerping, haat,.}{en weer onderwerping ze}{kwamen in \'e\'en tel}\\

\haiku{Hij merkte ook, dat,.}{ze op straat bleven staan dat}{ze naar hem keken}\\

\haiku{waar-je je hoofd kon,;}{neerleggen alle avonden}{op dezelfde plek}\\

\haiku{Be menschen weten.}{bij lange niet wat voor goeds}{in een borrel zit}\\

\haiku{Och, als hij nou weer, '.}{geld had zout zeker niet}{anders gaan als toen}\\

\haiku{Laatst nog, nee nu al,!}{weer drie jaar geleden met}{die zes week ziekte}\\

\haiku{Triestig staarde hij '.}{voor zich uit overt water}{van de Vijverberg}\\

\haiku{zoo'n vent was 't niet.}{waard om er twintig jaar voor}{achter slot te gaan}\\

\haiku{Meer met smeekende. '}{blikken dan met woorden zocht}{hij te vermurwen}\\

\haiku{- Drink ook 'es effe,,.}{riep er \'e\'en die gulzig aan}{de emmer slokte}\\

\haiku{- Werk je anders maar,, '...}{niet kapot schreeuwde Jaap die}{nu aant vuur stond}\\

\haiku{ze mochten Roelf graag,.}{lijden wilden hem niet over}{de kop jakkeren}\\

\haiku{Er valt altijd wat.}{te vinden of te vitten}{als ze d\`at willen}\\

\haiku{- Dat geloof ik... heb... '!}{motte anpietsek was}{heel wat ten achter}\\

\haiku{Tegelijk maakte:}{hij een beweging waarmee}{hij wilde zeggen}\\

\haiku{- Je hebt gelijk, ik,}{heb er ook al drie binne}{en morge vroeg is}\\

\haiku{- Het is niet goed te,.}{drinken waagde ze enkel}{schuchter te zeggen}\\

\haiku{het bleeke vermoeden,.}{dat ook haar jongen eens zoo}{iets zou overkomen}\\

\haiku{- 't Is zonde hem,,!}{te roepen zei ze in haar}{zelf maar het moet toch}\\

\haiku{De baas stond naar hem,;}{te gluren en hij voelde}{dat-ie knoeide}\\

\haiku{'t Was of met de.}{komst van Trees een heele blije}{wereld binnen schoof}\\

\haiku{Ik ben er niks op!}{gesteld wat jullie onder}{elkaar verknussen}\\

\haiku{Van half drie, tot half,,,}{vier van half vier tot half vijf}{zat ze aangekleed}\\

\haiku{Tegen vijf uur trok,.}{ze d'erop uit nam de tram}{naar Scheveningen}\\

\haiku{Maar hij vatte haar ',,:}{omt midden kuste haar}{wild op de mond zei}\\

\haiku{- En misschien nog wel,.}{wat anders ook giechelde}{Marie niet erg kiesch}\\

\haiku{Zijn oogen waren zoo,.}{schelvischachtig-grauw maar}{zijn neus vond ze leuk}\\

\haiku{ze moest het zich nog,.}{eens inprenten dat ze niet}{weer met hem uitging}\\

\haiku{Hij vroeg zich aldoor.}{af waar ze gisteravond naar}{toe kon zijn gegaan}\\

\haiku{Alwat ze zei of '.}{deed zou immerst geval}{enkel vergrooten}\\

\haiku{Maar een jong, wild ding,.}{als Trees die wil niet altijd}{stemmig thuis zitten}\\

\haiku{Ze wist van haar zelf.}{wat het beteekent als een vrouw}{zoo'n zwakke man krijgt}\\

\haiku{Met een strakke blik,.}{stond ze over hem gebogen}{de lippen bevend}\\

\haiku{- H\`e, gelukkig dat,!}{het eruit is dat h\^et me}{lang genog benauwd}\\

\haiku{Zij kon geen antwoord,;}{vinden suste dat hij zich}{rustig zou houden}\\

\haiku{Dan zakte ze op,.}{de stoel neer verborg haar moe\"e}{hoofd in de handen}\\

\haiku{Haar jongen had toen.}{overal loopen zoeken en}{haar niet gevonden}\\

\haiku{goed, jubelde Trees,.}{harte-gereed blij met}{deze oplossing}\\

\haiku{'s Nachts in haar bed, '.}{dacht ze toch aan Roelf en vond}{zet doodjammer}\\

\haiku{zijn moeder en 'n,.}{tante en nog een nichtje}{stonden er eveneens}\\

\haiku{Soms mocht-ie al op ',.}{t balkon komen en in}{de tuin wandelen}\\

\haiku{Een vochtglinstering,.}{in haar oude oogen weersprak}{dit goed vertrouwen}\\

\haiku{- Jij, vroeg z'n moeder,,!}{ontzet welnee jongen dat}{zal niet gebeure}\\

\haiku{- Zoo, maar ik heb er... ',!}{genog vant hangt me de}{keel uit d\`at zoeken}\\

\haiku{Vijlen beraspten,.}{de doorgeslagen nagels}{werkten de hoef bij}\\

\haiku{'t maakte haar zelf.}{week  en onmachtig om}{verder te kunnen}\\

\haiku{Ze hield de wagen.}{in en liet het kind van haar}{moede arm zakken}\\

\haiku{al hoorde ze niet,.}{de woorden ze voelde toch}{wat daar werd gezegd}\\

\haiku{Wat stapten ze vlug,.}{ook al droeg die juffer nog}{zulke hooge hakken}\\

\haiku{Hard ratelden de.}{wrakke radertjes over de}{vlakke klinkerweg}\\

\haiku{Smadelijk keek de,.}{inspekteur naar haar dan naar}{de vier kinderen}\\

\haiku{Die mooie lucht zag ze.}{dreigend met zijn rosse en}{violette pracht}\\

\subsection{Uit: Wrakke levens}

\haiku{Zoo was het lange, '.}{tijd maart gaat gelukkig}{nu veranderen}\\

\haiku{Die moest ze nog eerst,....}{afmaken maar dan dan kon}{ze wat uitrusten}\\

\haiku{'t kwam zeker van!}{al dat gewoel langs zijn oogen}{en van de warmte}\\

\haiku{- Ja, klakte Sijpert,,?}{het lijkt er wel wat op maar}{wat zul je eran doen}\\

\haiku{- Dat meen ik ook! - 't,,,?}{Zou gaan maar o mijn vrouw waar}{moet die dan blijven}\\

\haiku{- Zeker, je hebt hier, ',}{ook kans van genezen maar}{wat ist geval}\\

\haiku{Hij moest het nog eens,.}{aanzien d\`at groeide nu van}{zelf weer in hem aan}\\

\haiku{En dan, hij wil je,.}{ook graag zelf genezen dat}{begrijp je toch wel}\\

\haiku{- Hou-je goed kerel,,.}{tot het volgende jaar dan}{zei luchtig De Greef}\\

\haiku{ze mocht niet bij hem,.}{blijven al had hij het hard}{te verantwoorden}\\

\haiku{- Gaat u maar naar hem... '.}{toe hij h\`et zoo-evenn}{benauwdheid gehad}\\

\haiku{Maar ze kwam weer in,,.}{beklag en dat troostte even}{vergoedde wel wat}\\

\haiku{Als je in Zurich,.}{overnacht spreek je toch van geen}{poste-restante}\\

\haiku{Het eene voorwendsel '.}{leek nog onwaarschijnlijker}{dant andere}\\

\haiku{Ze wilde de brief,.}{aan haar lippen drukken maar}{ze hield zich nog in}\\

\haiku{Uit haar ooghoeken,,;}{zoo pijnlijk van vermoeidheid}{perelde een traan}\\

\haiku{Ze vond zich zelf weer.}{kinderachtig met haar}{ingebeelde angst}\\

\haiku{Zijn patroon had hem,,.}{door goede menschen gesteund}{hier heen gezonden}\\

\haiku{voor Rekeltje bleef.}{nagenoeg geen kansje om}{een plaats te vinden}\\

\haiku{Vooruit Rekeltje,,!}{kommandeerde de baas nou}{tegen mij jouw werk}\\

\haiku{Maar dat zag de boer ',!}{in de spiegel en die aan}{t opspele nou}\\

\haiku{Maar het afloopen '.}{ging toch gemakkelijker}{dant opkomen}\\

\haiku{ze wist dus niet of.}{ze wel zoo goed vooruitging}{als ze zelf meende}\\

\haiku{Soms keerde een zich ',.}{om of sprakn paar woorden}{anders bleef het stil}\\

\haiku{Maar Fiene Tas, die,,:}{beter wist vinnigde al}{op haar in schampte}\\

\haiku{Zij deed het alleen,.}{uit angst om de praatsters een}{beetje in te toomen}\\

\haiku{ik zou wel zegge,... '.}{dat het niet zoo erg meer is}{alleens morgens}\\

\haiku{Ik zou natuurlijk,...}{graag terug gaan maar als u}{denkt dat het niet kan}\\

\haiku{Ze verzond de brief, '.}{nog dezelfde avond maart}{gaf haar weinig rust}\\

\haiku{- Omdat jij nou niet,!... -,...?}{weg mag daarom scheld je op}{mijn Niet weg niet weg}\\

\haiku{Je moest het toch al! -,! -? - '...}{hebben Ja n\`et gekregen}{En Ik weetet niet}\\

\haiku{ze bleven ieder, '.}{in haar eigen hoekje voor}{t eigen raampje}\\

\haiku{De jongen boven,,.}{dat wist ze nu zeker ging}{langzaam achteruit}\\

\haiku{Een tweede snit werd.}{gedaan en ongemerkt trok}{de zomer voorbij}\\

\haiku{Grietje Groen, weer wat,.}{opgekwakkeld luchtkuurde}{nu en dan maar eens}\\

\haiku{Vlijm en scherp hoorde,:}{ze weer de woorden van de}{dokter die toch zei}\\

\haiku{Maar 't gerucht trok.}{aldoor zijn blikken en dat}{maakte hem nerveus}\\

\haiku{- Wat denkt u ervan,.}{vroeg hij eindelijk toen de}{dokter bleef zwijgen}\\

\haiku{- Dat in geen geval... -... -,!}{En ik dacht Beste jongen}{laat mij nu denken}\\
