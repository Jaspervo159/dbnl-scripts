\chapter[19 auteurs, 3311 haiku's]{negentien auteurs, drieduizenddriehonderdelf haiku's}

\section{Rink van der Velde}

\subsection{Uit: Feroaring fan lucht}

\haiku{D\^er wol ik libje,.}{wenjen bliuwe solang my}{God dit libben skinkt}\\

\haiku{Foekje hat noch wat,}{giele beantsjes en tusearte}{boud mar dy binne}\\

\haiku{Se tuge inoar \^of.}{dat it in lust is en Durk}{bliuwt der foar stean}\\

\haiku{Durk stapt op de pomp.}{ta en tapet himsels in}{m\^ulfol k\^ald wetter}\\

\haiku{It is sa'n moaie.}{blakstille septimberdei}{en dan klinkt it fier}\\

\haiku{{\textquoteright} Durk stelt him foar it.}{front fan syn h\'ush\^alding op}{en sjocht nei Foekje}\\

\haiku{Dit is Johan, heit,{\textquoteright} seit, {\textquoteleft}.}{Gryt sljochtweien hy hat hjir}{noch net earder west}\\

\haiku{{\textquoteright} {\textquoteleft}Wel ja, sprek mar op,.}{likegoed sille ik en}{mem it der oer ha}\\

\haiku{{\textquoteright} {\textquoteleft}In penty is wat,.}{oars dat is in broekje mei}{hoazzen der oan}\\

\haiku{{\textquoteright} Durk giet even op it.}{potrak neist de pomp sitten}{te bekommen}\\

\haiku{Akkoart, ha 'k sein,.}{mar dan mei hjir en d\^er in}{ferklearring derby}\\

\haiku{Dat ienkear in dief,.}{altiten in dief wie al}{wie it mar in fyts}\\

\haiku{Nee, neat op tsjin, mei, '.}{de tiid meigean hak}{altiten al sein}\\

\haiku{Sa hawwe wy,.}{jim mem ek opbrocht tiden}{hawwe tiden}\\

\haiku{En wy mochten net,.}{fuotbalje wy wienen der}{om wat te learen}\\

\haiku{Mar om de wille,.}{achter sa'n bal oandrave}{is sa dom as wat}\\

\haiku{Yn de earste jierren.}{nei de oarloch hienen wy}{ek noch bonnen mei}\\

\haiku{As jo wat leard ha,,.}{stiet de wr\^ald foar jo iepen}{sis ik altiten}\\

\haiku{Ik wie der krekt op '.}{e tiid by en koe dat}{knyntsje noch r\^ede}\\

\haiku{En as it nypt en.}{wedernypt dan binne se}{o sa ienriedich}\\

\haiku{Dit wie fansels al '.}{wat wreed en Foekje g\^ulde}{it opt l\^est \'ut}\\

\haiku{Dizze kin wol tsien, '.}{kear om Anema hinne dat}{hak him al \^ofloerd}\\

\haiku{W\^es bliid dat wy,.}{it bewenje sa kriget}{it syn \^underh\^ald}\\

\haiku{Hy soe it lykwols,,}{r\^eden ha seit hy der sels}{fan as er mar jild}\\

\haiku{Dat is te sizzen,.}{se wolle dat er oar en}{lichter wurk siket}\\

\haiku{Durk is doe knap lulk.}{wurden en hat it oanbrocht}{by de direkteur}\\

\haiku{Dy hie it bern der.}{al ynhelle ear't Durk der}{wat fan sizze koe}\\

\haiku{It liket derop, '.}{dat Gryt har fant winter}{yn Drachten deljout}\\

\haiku{Hy wit wol dat ik,}{\'ut en troch in knyntsje snip}{mar hy wit ek wol}\\

\haiku{Ik meitsje it dien,{\textquoteright}, {\textquoteleft}.}{sei ermar rekkenje dan}{mar net mear op my}\\

\haiku{{\textquoteright} {\textquoteleft}Noch twa jier dan kin ',{\textquoteright}.}{k yn de sanearing}{sei L\'utzen Bouma}\\

\haiku{{\textquoteright} {\textquoteleft}Dat r\^ede wy wol, ' '.}{25 g\^une kink wol \'ut}{e boeken h\^alde}\\

\haiku{As dy poat fan dy,.}{better is geane wy}{wer tegearre}\\

\haiku{En ik kin der net,.}{deun by lizzen gean do witst}{hoe skerp sa'n bok r\^ukt}\\

\haiku{{\textquoteright} {\textquoteleft}Do fertroust dysels,{\textquoteright}.}{ek net sa bot mear sei Ids}{spytgnyskjend}\\

\haiku{Oare wike hast him,.}{wer Ids en ik nim dy in}{pear p\^un reefleis mei}\\

\haiku{Yn 't hok klaut er.}{in steal \^alde ielf\^uken}{fan it souderke}\\

\haiku{{\textquoteright} {\textquoteleft}Wy kinne der skea,.}{fan ha kinst him better wat}{temjitte komme}\\

\haiku{{\textquoteright} {\textquoteleft}Welja, oars helje,{\textquoteright}.}{jim der noch in amtner by}{oan seit Durk haatlik}\\

\haiku{Der stiet net folle, '.}{stream mar d\^er ist neffens}{Durk net minder om}\\

\haiku{{\textquoteleft}Ja, dat is moai,...{\textquoteright}.}{mar Klaske sei niis Durk}{l\^est it briefkaartsje}\\

\haiku{It is in goede '.}{man en hy is opt l\^est}{mei \'us Aaltsje troud}\\

\haiku{Der komt in stuit dat.}{Durk de l\^este kofje der\'ut}{nimt en even stil falt}\\

\haiku{As wy de winter}{troch moatte mei njoggen}{tsientsjes en as se}\\

\haiku{Witst wol wat in stik '?}{klean opt heden kostet}{en wat it fleis jildt}\\

\haiku{Asto de boel wei, ' '.}{griemste h\^ald ik dy ope}{harsens ynt Djip}\\

\haiku{Eartiids hat er.}{Foekje wolris in klap j\^un}{as hja spul hienen}\\

\haiku{Mar as it dan dochs,{\textquoteright}.}{net mear te kearen is}{mimert Durk fierder}\\

\haiku{{\textquoteright} {\textquoteleft}D\^er hast gelyk oan,,?}{mem mar moat dat no mei}{keunstmiddels en sa}\\

\haiku{Ik wie sels earst ek,.}{net wizer mar ik ha der}{in soad oer neitocht}\\

\haiku{It hontsje jout ien kear,.}{l\^ud in s\^eft mar ferheftich}{f\^ul piipjen}\\

\haiku{Hy reaget de '.}{t\^uken oane kant en docht}{in stap nei foaren}\\

\haiku{It swit prikelt him.}{\^under de pet en hy smyt}{it ding fan him \^of}\\

\haiku{It is oars goed iten,.}{sa'n grouwe pike dy't noch}{krekt net flechtich is}\\

\haiku{Bijke giet tsjin him.}{oan lizzen en it hontsje jout}{aardich waarmte \^of}\\

\haiku{{\textquoteright} Bijke sn\'uft yn 'e,.}{sleatsw\^al om hy hat lucht}{fan in rot of sa}\\

\haiku{As Feike Dam d\^er,.}{west hie soed er no al wat}{\^undernommen ha}\\

\haiku{Bouwe syn bromfytsark,,.}{dat oeral omslingeret}{smyt er op in bult}\\

\haiku{{\textquoteright} Tsjerk is achter\'ut nei.}{de reed riden en dan giet}{it oer de br\^ege}\\

\haiku{Duvel, wat sil er '.}{lulk w\^eze as er dit yn}{e gaten kriget}\\

\haiku{It is in edel dier,,.}{sa'n reebok soks moat mei}{ferdrach dien wurde}\\

\haiku{De iene is in, '.}{bytsje stikken mar d\^er kin}{k ek neat oan dwaan}\\

\haiku{De bern meie it net,.}{br\^uke want  oars soe it}{samar op w\^eze}\\

\haiku{{\textquoteright} {\textquoteleft}Dat hoecht no net mear,{\textquoteright}.}{seit Durk wylst hy stean bliuwt}{en in sjekje draait}\\

\haiku{Dat jout neat, mar hy.}{sit as de duvel achter}{de streuperij oan}\\

\haiku{Durk trouwens ek, mar.}{hy wol dat gejeuzel net}{langer oanhearre}\\

\haiku{Mar jo dogge wat,.}{sokke tiden jo smite}{josels heal wei}\\

\haiku{{\textquoteright} {\textquoteleft}Ik soe it der fan ',.}{e wike mei har oer ha mar}{se lake my \'ut}\\

\haiku{It wurdt minder,{\textquoteright} seit, {\textquoteleft}}{er karmasterjendder leit}{in b\^este bocht yn}\\

\haiku{Yn \'us gemeente,{\textquoteright}.}{hie it der al lang \^of west}{ferklearret Gerard}\\

\haiku{{\textquoteright} Durk docht oprjocht, '.}{syn b\^est mar it wol him net}{ynt sin komme}\\

\haiku{om stean litte as.}{ik dy oanjaan soe by de}{boargerlike st\^an}\\

\haiku{{\textquoteleft}De oare deis soest.}{de middeis th\'us komme en}{dan nei de Sweach ta}\\

\haiku{{\textquoteleft}Goed \^unth\^alde, Feike,{\textquoteright}, {\textquoteleft}:}{seit er mei triljend l\^uddit}{moatst goed \^unth\^alde}\\

\haiku{{\textquoteleft}It fel ek noch fol,.}{hagel no bar ik der net}{iens twa kwartsjes foar}\\

\haiku{No moatst \'ut 'e,,{\textquoteright}.}{wei w\^eze oars falle der}{deaden raast Durk}\\

\haiku{Durk pakt syn fyts en.}{trapet twa kear sa hurd as}{oars nei de Sweach ta}\\

\haiku{{\textquoteright} {\textquoteleft}Wat binne jo in,{\textquoteright}.}{min persoan seit Durk mei}{de klam op elk wurd}\\

\haiku{{\textquoteright} It binne sokke,.}{liepe praters dy minsken}{fan de gemeente}\\

\haiku{{\textquoteleft}It moat net te,.}{leabrekkend w\^eze ik sil}{der achteroan}\\

\haiku{Der wurdt oer praat om,{\textquoteright}.}{my op fyftich persint te}{setten seit er th\'us}\\

\haiku{moat fansels earst,.}{wol wenne benammen foar}{Kekke en Bijke}\\

\haiku{It leit my sa by,,.}{Ids dat it mei my yn dat}{Drachten net goed komt}\\

\haiku{Dy strewellen fan,}{jo hingje in heul ein \'us}{kant oer dat it stiet}\\

\haiku{Ik kin it op 't,.}{heden net byh\^alde de}{holle rint my om}\\

\haiku{Skoalle tolve,}{is foar dizze strjitten}{mar it is de fraach}\\

\haiku{{\textquoteright} Hja sil it earst wol.}{betelje en letter mei}{mem ferrekkenje}\\

\haiku{Jo stjoere de bern.}{moarnier mar en der sil}{mei r\^eden wurde}\\

\haiku{It is my krekt as ' '.}{hiek in heule lange}{dei yne put stien}\\

\haiku{As se in oar der,.}{mei pakke sykje se \'ut}{w\^er't er wei komt}\\

\haiku{It soe in moaie.}{fertoaning wurde en g\^ans}{opskuor jaan}\\

\haiku{De jonge hie oan.}{de S\^anleane altiten}{aardich syn slinger}\\

\haiku{Ik ha hjir hjoed twa '.}{kear dat minske fan hjir neist}{wei oane doar h\^an}\\

\haiku{En heit soe yn 'e,.}{keamer de skuon \'utdwaan}{it w\^adet sa yn}\\

\haiku{Earst in pear dagen ',{\textquoteright}.}{ynt hok en lit him dan}{mar rinne seit Durk}\\

\haiku{{\textquoteright} {\textquoteleft}En ik fernim no.}{al hoe goed it my dwaan sil}{om der even by wei}\\

\haiku{Ik wol dy winkels,.}{ek wolris besjen ik ha der}{noch gjin stap \'ut west}\\

\haiku{{\textquoteleft}Oars hie myn jonge, '.}{net mei in stik izer slein dat}{hak se net leard}\\

\haiku{{\textquoteleft}Gjin sprake fan, dy.}{bern hawwe leard om fan}{har \^of te biten}\\

\haiku{Hy raast twa slaggen ',:}{ome pomp hinne lyk as}{eartiids en ropt}\\

\haiku{Se hawwe der.}{rekken mei holden dat it}{nachts aardich fris wurdt}\\

\haiku{Se kr\^upe s\'untsjes by.}{de w\^al op en geane}{in eintsje tebek}\\

\haiku{{\textquoteright} {\textquoteleft}Dan keapje wy,{\textquoteright}.}{ien seit Meine en hy lit}{de flesse omgean}\\

\haiku{{\textquoteleft}Ik tink dat wy wol,,{\textquoteright}.}{in santich ielen ha heit}{seit Willem optein}\\

\haiku{Ast op elke trije, '.}{d\^obers in iel hast ist}{b\^est en wy ha mear}\\

\haiku{Hy set de fyts dwars '.}{oert paad sadat se der}{wol \^of moatte}\\

\haiku{Hy kwakt de fyts tsjin '.}{e beam en is yn in pear}{stappen by Feike}\\

\haiku{Mar de chef seit ek.}{dat jo gauris in skoftsje}{steane te sjen}\\

\haiku{Tsjerk wol har ek sa'n,:}{krantsje ta ha hy is der}{rejaal mei en seit}\\

\haiku{{\textquoteleft}Hy falt gewoan ',.}{yne klean fan lytse Durk}{hy kostet \'us neat}\\

\haiku{En de krystbeam is,.}{sa grut \'utfallen hy kin}{amper troch de doar}\\

\haiku{Mar dan moatte:}{sokke mannen net l\^ebich}{wurde en sizze}\\

\haiku{Foekje en Klaske.}{knikke wakker by alles}{wat de man oanfiert}\\

\haiku{Mar hy kaam mei de,.}{lep achter my oan dat doe}{moast ik ek wol}\\

\haiku{Jo ha fansels wol '.}{yne rekken dat ik mei}{jo yn proses gean}\\

\haiku{En wa is dizze?}{skriuwer dat er de skiednis}{ferfalse mei}\\

\haiku{D\^er tinke sokke,.}{mannen fansels net oan mar}{in oar sit der mei}\\

\haiku{Dit hat Drachten op.}{syn minst in middelgrutte}{yndustry koste}\\

\haiku{Dit der even tusken,.}{troch al moat hjir net te}{licht oer praat wurde}\\

\haiku{Ik rin hjir yn myn.}{frije tiid ek mei de siel}{\^under de earm om}\\

\haiku{it sa swier hast Durk,.}{dan moatte wy dochs wat}{foar dy \'utfine}\\

\haiku{It is Durk min nei ',.}{t sin mar hy kin him der}{net foarwei wine}\\

\haiku{Hy fotografeart}{fan moarns oant j\^uns en plakt}{de brut yn grutte}\\

\haiku{in eigenwiis.}{man dy't mient dat er al hast}{boargemaster is}\\

\section{Paul van der Velden}

\subsection{Uit: Darrenslacht}

\haiku{Ik vraag haar waarom.}{niemand mij heeft verteld over}{mijn dansende opa}\\

\haiku{Toch klopte ze aan.}{en op een brommend antwoord}{opende ze de deur}\\

\haiku{Maar het bedrag van.}{f23,50 was niet helemaal bij}{elkaar gesprokkeld}\\

\haiku{Moest ze de tand over?}{haar linkerschouder gooien}{of over haar rechter}\\

\haiku{Dat was een strenge,.}{die je ongezouten de}{waarheid kon zeggen}\\

\haiku{De volgende dag,.}{vertrek ik naar Fatima}{een bedevaartsoord}\\

\haiku{Ik hoor het kraken.}{van zijn botten en dan zakt}{hij door zijn pootjes}\\

\haiku{Zijn vingers schreven.}{voorbeeldletters in de lucht}{bij het schoonschrijven}\\

\haiku{Ik hoor niets, alleen.}{het krassen van haar nagel}{tegen de potwand}\\

\haiku{{\textquoteright} ~ Veel te laat, tien,.}{jaar later kom ik haar weer}{tegen in de stad}\\

\haiku{{\textquoteright} ~ {\textquoteleft}Twee weken was,.}{hij weg toen de zandziener}{zijn aanzegging deed}\\

\section{Adriaan Venema}

\subsection{Uit: Lemmingen (onder ps. A. ten Hooven)}

\haiku{Ik draaide bij Utrecht.}{de weg af en keerde naar}{Amsterdam terug}\\

\haiku{waar het lag tot de.}{leraar opstond en langzaam}{zijn richting uitkwam}\\

\haiku{Dani\"el kreeg een.}{kleur en glipte langs de man}{naar de kleedkamer}\\

\haiku{Hij sperde zijn ogen.}{wijd open om maar niets van het}{schouwspel te missen}\\

\haiku{Ze stonden roerloos,,.}{een somber groepje en ze}{keken hem strak aan}\\

\haiku{{\textquoteright} Een van de mooiste.}{ogenblikken uit zijn leven}{zou nu voorbijgaan}\\

\haiku{De monoloog aan.}{de andere kant van de}{lijn duurde niet lang}\\

\haiku{{\textquoteright} {\textquoteleft}Nou, mama, misschien,.}{zijn ze aardig tegen u}{maar niet tegen ons}\\

\haiku{Zijn vader stapte,.}{in gevolgd door een van de}{twee marechaussees}\\

\haiku{Er is niets tegen.}{me gezegd over mensen die}{rond willen lopen}\\

\haiku{Iedereen doet zo,.}{optimistisch maar ikzelf}{zie het somber in}\\

\haiku{Nu zijn mijn boeken,.}{me lief maar mijn hachje heb}{ik er niet voor over}\\

\haiku{Hij verwachtte dat,.}{het uiteen zou spatten maar}{dat gebeurde niet}\\

\haiku{Dani\"el zou het,.}{huis binnengelopen zijn}{achter de man aan}\\

\haiku{\'e\'en dode voor zo'n.}{middag zou hem voldoende}{bevredigd hebben}\\

\haiku{wie in je weg komt,,.}{te staan die sla je neer die}{hoor je te doden}\\

\haiku{Zij hebben het met,.}{Kwik te zamen gezongen}{kort geleden nog}\\

\haiku{Ik weet het niet - zijn,?}{ouders zouden die dat goed}{gevonden hebben}\\

\haiku{Ik blijf er geen uur -.}{langer door leven als ik}{dat al zou willen}\\

\haiku{Bovendien had hij.}{de man nodig om een krant}{te bemachtigen}\\

\haiku{Zoals een leeuw zich.}{brullend opricht als hij wordt}{lastig gevallen}\\

\haiku{Even later kwam ze.}{terug met een gaasje en}{een grote pleister}\\

\haiku{Dit zou zijn eerste.}{actie zijn die hij los van}{Micha ondernam}\\

\haiku{{\textquoteright} Alleen was de kans}{dat hij hem met de asbak}{zou kunnen doden}\\

\haiku{Tot welk niveau moet?}{ik me dan wel verlagen}{om quitte te staan}\\

\haiku{Hij staat op en geeft {\textquoteleft}?}{me een hand.Is de waarheid}{nu nog relevant}\\

\haiku{Hij heft de arm met.}{het zwaard op het moment dat}{hij vindt dat het moet}\\

\haiku{{\textquoteleft}En ik blijf erbij.}{dat we het plebs buiten ons}{huis moeten houden}\\

\haiku{{\textquoteleft}Ik had eens op een.}{middag een meisje uit de}{buurt meegenomen}\\

\haiku{Ik kleed me aan en.}{loop terug naar het bed om}{af te rekenen}\\

\haiku{{\textquotedblleft}We zijn niet gek, we}{geven jullie door aan de}{zedenpolitie}\\

\haiku{{\textquoteleft}Weet u nog dat we?}{over een portret van de F\"uhrer}{hebben gesproken}\\

\haiku{De cops grijpen je,,?}{zodra ze daar maar een kans}{voor zien weet je dat}\\

\haiku{Maar als je daar de,.}{krant staat te lezen kunnen}{ze je niet grijpen}\\

\haiku{Er zijn daar een paar.}{kiosken en die krant heb}{je er net gekocht}\\

\haiku{Ga maar naar Mama,.}{Meyer op 2nd Avenue vlak}{bij het Bay Hotel}\\

\haiku{Hij probeerde een.}{martiale trek op zijn}{gezicht te krijgen}\\

\haiku{Ik weet niet meer wat,:}{hij allemaal zei maar \'e\'en}{ding weet ik nog wel}\\

\haiku{{\textquoteleft}Geef me dan maar bier,{\textquoteright},.}{zei Roskam op zijn beurt het}{klonk heel grootmoedig}\\

\haiku{Zijn gezicht stond kwaad,.}{want hij begreep Dani\"els}{vader nu heel goed}\\

\haiku{Dani\"el gaf de.}{anderen een teken dat}{ze moesten blijven staan}\\

\haiku{Hij was nauwelijks,.}{trots maar dat nam mijn eigen}{voldoening niet weg}\\

\haiku{Hij had een zachte.}{stem en hij sprak Duits met een}{zangerig accent}\\

\haiku{De graaf verbrak het:}{stilzwijgen door te vragen}{hoe het op school ging}\\

\haiku{Dani\"el voelde.}{een hevige opwinding}{in zich opkomen}\\

\haiku{Hij viel in zijn stoel.}{terug en dronk snel een paar}{teugen uit zijn glas}\\

\haiku{{\textquoteleft}We praten over strijd,{\textquoteright}, {\textquoteleft}.}{zei hij langzaammaar het gaat}{daar niet alleen om}\\

\haiku{{\textquoteright} Dani\"el hoorde.}{de zoemer en het klikken}{van de deuropener}\\

\haiku{De ene man liep naar.}{de auto en sjorde de}{bestuurder eruit}\\

\haiku{En kogels fluiten,,:}{toch dacht hij want dat had hij}{uit westerns geleerd}\\

\haiku{Hij probeerde te,.}{lachen zodat zijn rotte}{tanden bloot kwamen}\\

\haiku{Hij keek in de wieg.}{naar het rode hoofd van het}{pasgeboren kind}\\

\haiku{Hup, jongen, smeer 'm,{\textquoteright}.}{beet de man hem met een stem}{vol minachting toe}\\

\haiku{De vrouw schuift met een.}{ruk haar stoel naar achteren}{en zakt onderuit}\\

\haiku{Eigenlijk leefde.}{hij alleen nog maar om zijn}{tijd uit te dienen}\\

\haiku{De wereld is niet,.}{gebrand op de waarheid de}{wereld kijkt naar kracht}\\

\haiku{Hij vroeg Dani\"el.}{rechtop te gaan staan en zich}{niet te bewegen}\\

\haiku{Toen sprong Dani\"el.}{met de kat in zijn armen}{van de gaanderij}\\

\haiku{Ze schreeuwden hoog en.}{schel zodra de vlammen hun}{veren bereikten}\\

\haiku{Hij vloekte, want tijd.}{was juist op dit moment erg}{belangrijk voor hem}\\

\haiku{Hij liep langs de trap.}{de grote zitkamer in}{en deed het licht aan}\\

\haiku{In het licht van de.}{maan zag hij haar op het pad}{naast het huis liggen}\\

\haiku{{\textquoteright} Dani\"els moeder.}{begon te huilen en hij}{bracht haar naar boven}\\

\haiku{{\textquoteleft}Wat let me om de?}{gehele wereld aan mijn}{voeten te werpen}\\

\haiku{Toen de vrouw opkeek,.}{legde hij zijn hand haastig}{op het tafelblad}\\

\haiku{Het was alsof hij.}{begreep dat aan alles een}{einde moest komen}\\

\haiku{Trams die van links naar.}{rechts reden en trams die van}{rechts naar links gingen}\\

\haiku{Even later kwamen.}{de mannen de trap weer af}{en de keuken in}\\

\haiku{Het was laat in de.}{avond toen zijn vader en zijn}{zuster thuiskwamen}\\

\haiku{Even later sloeg de.}{motor van hun auto aan}{en het was voorbij}\\

\haiku{Hij was nu weer een.}{lafaard om daar niet voor te}{durven uitkomen}\\

\haiku{Ver weg hoorden ze.}{de klagende tonen van}{een accordeon}\\

\haiku{Ze stond op zonder.}{een woord af te wachten en}{liep naar de keuken}\\

\haiku{De mannen sleepten.}{zijn vader nu tussen hen}{in over het tuinpad}\\

\haiku{Hij ging naar een groot,.}{gebouw waar militairen}{in en uit liepen}\\

\haiku{Er mag geen steen of.}{welke naamsaanduiding dan}{ook worden geplaatst}\\

\haiku{Aan de rand van het.}{tuinpad stond een spade in}{de grond gestoken}\\

\haiku{Haar armen gingen,.}{even omhoog maar zakten toen}{slap naast haar lichaam}\\

\haiku{Tevreden bleef hij.}{enkele ogenblikken naar}{het lichaam kijken}\\

\haiku{Twee mannen trokken.}{haar broek uit en smeerden haar}{onder met menie}\\

\haiku{Ze liepen naar een.}{duinpan waar een peloton}{stond aangetreden}\\

\haiku{{\textquoteleft}Er zijn er niet veel,{\textquoteright}.}{meer uit die jaren die nog}{leven zei de graaf}\\

\haiku{Soms kostte het hem.}{moeite zich zijn vader voor}{de geest te halen}\\

\haiku{Zijn broer liet nooit iets.}{na om zijn minachting voor}{Micha te tonen}\\

\haiku{Hij lag ruggelings,.}{op het vloerkleed een gaatje}{in het voorhoofd}\\

\haiku{Een verpleegster geeft.}{me een injectie om de}{pijn te verlichten}\\

\haiku{{\textquoteleft}En mag ik u nu,{\textquoteright}.}{groeten zei de man en hij}{liep het hotel uit}\\

\section{Gustaaf Vermeersch}

\subsection{Uit: De last}

\haiku{Daar peuterde hij.}{een wijle en dan kwam zijn}{uurwerk te voorschijn}\\

\haiku{Ze gingen voort en '.}{t wijf stak geniepig de}{druppelglazen weg}\\

\haiku{'t opgezweept bloed.}{verdoofde zijn blikken en}{bezwaarde zijn hoofd}\\

\haiku{- Ge zegt dat lijk of!}{iemand verplicht ware u}{die kans te geven}\\

\haiku{Ze bedolven zich,.}{daar in een hoek en keken}{toe in stilzwijgen}\\

\haiku{Dan vaagde hij met.}{zijn zakneusdoek zijn zweet af}{en bezag haar schuins}\\

\haiku{De ouden stonden.}{op en torden stijf en zich}{rekkend naar de deur}\\

\haiku{Soms voelde hij wat,;}{steken in zijn kop maar gaf}{daar weinig acht op}\\

\haiku{Plots scheidde hij weer.}{uit en hield de oogen star in}{de ruimte gericht}\\

\haiku{Ze had heur schorte.}{vol boterkoeken die ze}{op tafel gooide}\\

\haiku{Ze verwijderden.}{zich en hij werd razend van}{machtelooze kwaadheid}\\

\haiku{Moeder sprak niet over ',;}{t meisje haar armoe ze}{waren zij ook arm}\\

\haiku{Hij sprak met volle ';}{overtuiging omdat hij wist}{datt niet waar was}\\

\haiku{Ik kwam u vragen.}{of ge soms niet wenschte}{getuige te zijn}\\

\haiku{hij verplicht zich op,}{een dorpel neer te zetten}{hij kon niet meer voort}\\

\haiku{hij voelde een juk.}{op zijn schouders wegen dat}{hem terneerdrukte}\\

\haiku{'t Wijf vond dat ook ';}{geweldig geestig en ze}{giecheldet uit}\\

\haiku{als ze op zijn teenen,}{torden of hem begoten}{zoo v\'eel avonden moest}\\

\haiku{Toen hij eindelijk.}{op straat kwam voelde hij nog}{zijn moed verflauwen}\\

\haiku{- Ja. - 't Is rap, 't!}{is nog maar gewillig een}{maand dat ge verkeert}\\

\haiku{Dit zicht prikkelde.}{hem zoo zeer dat hij niet meer}{kon blijven liggen}\\

\haiku{Moeder sprak tegen '.}{datt niets was en begon}{bijeen te vagen}\\

\haiku{Jan zei niet veel, hij,.}{keek onophoudelijk de}{straat op ongerust}\\

\haiku{De brandewijn werd.}{opnieuw uitgeschonken en}{men klonk en dronk weer}\\

\haiku{Bij pozen keek hij,.}{langs weerskanten de straat op}{doch niemand kwam af}\\

\haiku{De plechtigheid liep.}{nu snel af en ze gingen}{van daar naar de kerk}\\

\haiku{Dan barstten ze in! -}{handgeklap los en schreeuwden}{luidruchtig bravo}\\

\haiku{ik bezit niets, gaf, '.}{Jan voor antwoordk heb al}{mijn geld verdronken}\\

\haiku{Jan, die de laatste,}{ging was alleen achter en}{toen hij weerkwam botste}\\

\haiku{Een vlammetje blonk,:}{knetterend vlamde wijder}{uit en verblankte}\\

\haiku{De eenigste deugd die.}{hij had was dat hij veel geld}{zou binnenbrengen}\\

\haiku{t'enden adem, tegen.}{hem mocht leunen met heel de}{zwaarte van haar lijf}\\

\haiku{haar jak stond open en;}{liet een vuil-roode baai zien}{en een vuil wit hemd}\\

\haiku{Zij, ze was nog wel.}{verdragelijk omdat ze}{een vrouw was en jong}\\

\haiku{wee als Romme hem,!}{niet voldeed hij zou haar wel}{weten te straffen}\\

\haiku{Daarboven-op droeg.}{ze een even rosse halsdoek}{met bruine froeien}\\

\haiku{Ze legden zich weer,.}{te bed lijk gewoonte met}{hun rug naar elkaar}\\

\haiku{Doch alles stond in}{een bekende schikking en}{daaraan erkende}\\

\haiku{Daar zocht hij een bank,.}{op doch de eerste welke}{hij vond was bezet}\\

\haiku{Ze woonde ginder;}{ieverst in een klein straatje}{aan d'andere poort}\\

\haiku{Ze klapten nog wat '.}{over-en-weer maart wijf was}{niet te bepraten}\\

\haiku{Dan ging hij ineens,.}{op de deur toe en stiet er}{tegen ze vloog open}\\

\haiku{Hij schudde zijn hoofd,.}{loerde nogeens naar de deur}{en vond ze goed toe}\\

\haiku{Dat miek hem razend.}{op zijn eigen omdat hij}{haar genomen had}\\

\haiku{- 't is altijd 't,, ':}{zelfde zei hij de naaste}{maand zalt weer zijn}\\

\haiku{Langzaam rechtte de.}{razernij zijn gekrompen}{gestalte weer op}\\

\haiku{Hij tastte in zijn,.}{zak vond nog wat enkele}{kluiten en gaf ze}\\

\haiku{Toen hij nader kwam.}{bleef hij weer staan en overzag}{dat alles nog eens}\\

\haiku{Hij doopte er zijn,,.}{vingeren in sloeg zich een}{kruis en ging slapen}\\

\haiku{Hij boog en Romme,.}{bedankte hem vurig toen}{trok hij de deur uit}\\

\haiku{'t Begon hem reeds.}{te vervelen en hij stond}{er verrammeld mee}\\

\haiku{Jan ging open doen en.}{vond Stant v\'oor de deur staan in}{zijn piottenpak}\\

\haiku{Een oogenblik liep,:}{ze als zinneloos rond luid}{huilend en snikkend}\\

\haiku{die een eindelijk;}{begrip was van hun eigen}{en van elkander}\\

\haiku{- 't Is wreed, gij gaat!}{naar uw ouders en ik mag}{de mijne niet zien}\\

\haiku{Zijn wijf liep rond, ze.}{had wafels gebakken en}{sjokola gemaakt}\\

\haiku{Dan ging dat alles.}{weer op in een razernij}{tegen die wijven}\\

\haiku{- Dat gaat mij niet aan,, '!}{zei de vent kalmt is}{maar een bedenking}\\

\haiku{Jan bekeek hem, zijn.}{verbeest gezicht en zijn blik}{zonder bewustzijn}\\

\haiku{- Voor mijn kind zal ik ',.}{t wel doen dat kan er toch}{niets aan verhelpen}\\

\haiku{Hij bewoog zijn hoofd '.}{op-en-neerwaarts en bleef}{int zand kijken}\\

\haiku{Bij dat bedrijf ging '}{zijn gemoed aant hollen}{en hij ademde zwaar}\\

\haiku{Ontzet bezag hij. '}{dat en voelde een koude}{om zijn herte slaan}\\

\haiku{Eindelijk wierp hij.}{het weer weg en ging heen door}{droefenis overmand}\\

\haiku{Doch ineens begon.}{ze luid te gillen en hij}{liet haar ontzet los}\\

\haiku{De smoor rekte zich,,:}{daarover uit werd steeds dichter}{overwelfde de plaats}\\

\haiku{Een heele tijd liep.}{hij en bedaarde maar als}{hij geen adem meer vond}\\

\haiku{Dan bleef hij staan voor.}{een plotse ingeving en}{staakte zijn geschrei}\\

\subsection{Uit: Nazomer}

\haiku{Er was een valsheid:}{in haar toestand die ze zo}{duidelijk voelde}\\

\haiku{Ze geloofde 't.}{zelf eindelik omdat ze}{niemand bemind had}\\

\haiku{Matielde vooral.}{koesterde een innige}{nijd tegen ieder}\\

\haiku{Het uitwerksel was,.}{telkens verschrikkelik haar}{maag was zeer ontsteld}\\

\haiku{Toevallig bleven.}{ze niettemin staan kijken}{tot hij voorbij kwam}\\

\haiku{Daardoor kwam het dat {\textquoteleft}{\textquoteright}.}{hij ook reeds hetmierenest van}{de Deman's kende}\\

\haiku{En hoe meer hij ze,}{van nabij beschouwde hoe}{meer gelijkenis}\\

\haiku{Het was zonde dat,,!}{zo een kerel die zoveel}{geld won jongman bleef}\\

\haiku{Afschuwelike,.}{dingen schichtten door haar brein}{ze voldeed eraan}\\

\haiku{Neen, zo kon ze hem,.}{zich niet verbeelden ze had}{geen zinnen daarnaar}\\

\haiku{Hij was half bedwelmd.}{en keek haar glimlachend aan}{met gulzig genot}\\

\haiku{daar is niets dommer,!}{een lintje en een strekje}{kan hen verleiden}\\

\haiku{ze begreep het niet,, '.}{ze had niets gedaant was}{van zelf gekomen}\\

\haiku{ze wankelde naar.}{binnen doch was niet in staat}{een woord te spreken}\\

\haiku{Haar oren begonnen.}{te ruisen en de dingen}{benevelden zich}\\

\haiku{Wel mogelik, ik.}{neem geen stand in een geschil}{waarvan ik niets ken}\\

\haiku{- Om het even het moet! -,.}{uitgeroeid worden Neen maar}{genezen worden}\\

\haiku{- Maar het deed hem erg.}{onaangenaam aan en hij}{lachte bedwongen}\\

\haiku{Maar misschien drink je,?}{nogal veel azijn of liever}{gebruik je er veel}\\

\haiku{- Hier kan niemand ons,.}{zien of overvallen zei hij}{met gesmoorde stem}\\

\haiku{Ze vloog opniew aan.}{zijn hals en bezwoer hem toch}{met haar te trouwen}\\

\haiku{De eerste stonden.}{verbluft en de drie kwamen}{bij Van Riebeeck staan}\\

\haiku{- Ga stilletjes naar,,,}{huis jongen met je dochter}{en zwijg dat je zweet}\\

\haiku{Matielde wilde.}{haar zuster weg hebben doch}{Stiena wilde niet}\\

\haiku{- Je overdrijft, zei haar,?}{Mevrouw Bollekens waarom}{zou dat jou schuld zijn}\\

\haiku{In hem had zich de.}{warmte gelegd van Stiena's}{tegenwoordigheid}\\

\haiku{Het was een zachte.}{bekoring die nooit meer uit}{hem zou verdwijnen}\\

\haiku{Maar hij kon haar niet.}{zo treurig zien omdat zulks}{zijn genot benam}\\

\haiku{- Je weet dat moei aan?}{Van Riebeeck gevraagd heeft om}{met hem te trouwen}\\

\haiku{Hij werd een wijle.}{woedend en zou de oudste}{wel verwenst hebben}\\

\haiku{er zijn er soms die!}{naar je verlangen zonder}{dat je het zelf weet}\\

\haiku{Van Riebeeck had even,.}{naar haar opgezien hun ogen}{ontmoetten elkaar}\\

\haiku{'t is 'n schande '.}{datn mens alzo onder}{bewaking moet staan}\\

\haiku{We zijn gelukkig.}{omdat we gestraft worden}{voor onze zonden}\\

\haiku{In dit te weten.}{had ik een zalig en toch}{onschuldig genot}\\

\haiku{Ja, dan dachten we.}{toch ook wat we al zouden}{doen met die boel geld}\\

\haiku{Nu zagen we dat,.}{ook op h\'un kamer rook was}{hoewel veel minder}\\

\haiku{Eindelik, 't was '...}{blijkbaar dat zet niet meer}{konden herden en}\\

\haiku{we hadden immers.}{al zoveel ontroeringen}{en schokken doorstaan}\\

\haiku{'t Was 'n vreemde,.}{vandaar haar medelijden}{dat rechtzinnig was}\\

\haiku{hoe meer karakter,.}{en hoe minder gevoel hoe}{minder geweten}\\

\haiku{Zo had ik reeds vrees '.}{datn grotere dief me}{de baard zou afdoen}\\

\haiku{Bij die mensen die.}{ons opgenomen hadden}{zou ik eten krijgen}\\

\haiku{Hij was een schavuit,,.}{lijk ik Piekvogel doch de}{waarheid voor alles}\\

\haiku{Was ik niet een mens?}{gelijk ieder ander mens}{en dus een lafaard}\\

\haiku{t Is zo'n zeldzaam!}{genot op de kosten van}{een ander te eten}\\

\haiku{En zouden we nu,,?}{nog na al die dingen de}{mensen bedriegen}\\

\haiku{De dagen gingen,,,.}{voorbij dat is geloof ik}{toch de gewoonte}\\

\haiku{Hij gelastte zich.}{zulks te doen mitsgaders een}{dertigtal franken}\\

\haiku{Hij is immers wat.}{achterbaks en zal dit met}{mij ondervinden}\\

\haiku{al die tieraden,}{op God die hierboven staan}{beangstigen me.}\\

\haiku{ik verleng steeds het.}{stuk enkel en alleen om}{nog meer te trekken}\\

\subsection{Uit: Het rollende leven. Deel 1}

\haiku{Haar raadgevingen:}{waren geen bevel meer doch}{bijna een smeken}\\

\haiku{Hij werd beschaamd over.}{zijn onbeholpenheid en}{nodeloze schrik}\\

\haiku{Langzaam echter werd,.}{het zo klein dat men nog slechts}{de hellingen zag}\\

\haiku{- Als de loeders het '!}{nu nog niet weten kunnen}{zet gaan rieken}\\

\haiku{Arie's hand was niet vast,.}{genoeg hij beefde wat en}{de duw was te flauw}\\

\haiku{Zijn houding stelde,.}{Arie wat gerust zodat hij}{zijn werk hervatte}\\

\haiku{ge zoudt wel mogen!}{trakteren om met ons mee}{te mogen rijden}\\

\haiku{De eerste maal dat '.}{get weer verzuimt zult ge}{weten aan wat prijs}\\

\haiku{Block sloeg geweldig:}{de andere dicht en gaf}{een krakende vloek}\\

\haiku{wat het was - een macht.}{van blinkende knoppen langs}{de eenzame weg}\\

\haiku{Het deed hem deugd, maar:}{seffens dreef die goedheid weg}{bij de gedachte}\\

\haiku{wat wilde, hij had.}{toch gedaan wat hij kon en}{zijn plichten vervuld}\\

\haiku{Deze schold hem uit.}{en had een kwade grijns van}{voldane vraakzucht}\\

\haiku{s Avends vernam.}{hij weer nieuws over de zaak uit}{de mond van Schoonheydt}\\

\haiku{Toen 't gedaan was!}{zag hij ontsteld dat hij te}{ver ingevuld had}\\

\haiku{Daartoe vonden ze.}{gauw genoeg een reden in}{alles wat hij deed}\\

\haiku{stil de portel met.}{een ger opengetrokken en}{binnengeslopen}\\

\haiku{de trein stille stond,}{liep een eindje verder om}{de schijn te geven}\\

\haiku{Hij dronk stevig door,,.}{samen met de andere}{al zijn geld moest op}\\

\haiku{Dit plotse voorstel.}{doorschokte Arie met vreemde}{wellustgevoelens}\\

\haiku{Hij had zich als een.}{redeloze ezel laten}{meelimperen}\\

\haiku{Ze verstaan niets van, '.}{u en willen u buiten}{daarmee ist uit}\\

\haiku{- Ge zijt nog een duts,,,}{zei hij Arie en veel te veel}{te goeder trouwe}\\

\haiku{Maar hij trok algauw,.}{zijn plan was blij dat het zo}{afgelopen was}\\

\haiku{Schoonheydt zoog gulzig,,.}{aan dat genot waar hij Arie}{vergeefs naar smachtte}\\

\haiku{O ja, het trok hem,,!}{wel aan dat genot hij had}{het bijna gesmaakt}\\

\haiku{het rijtuig waar de.}{dame was ingestegen}{doch ging er niet bij}\\

\haiku{De dagen zouden.}{hem alzo meedragen en}{hij herwerd zichzelf}\\

\haiku{Met de gewone,.}{trein reed hij af er mocht van}{komen wat wilde}\\

\haiku{vragen, hij gaf haar!}{de toelating om mee te}{reizen op voorhand}\\

\haiku{- naar mij moet ge niet,,.}{zien ge kunt doen wat ge wilt}{ik ga niet meer mee}\\

\haiku{Maar neen, dat ook was,;}{zo ontzettend leeg hij had}{er iets bijgevoegd}\\

\haiku{Er schortte hem iets,,.}{hij zocht het overal maar hij}{wist niet wat het was}\\

\haiku{'t Was een zondag.}{en daar stonden lieden om}{een uchtenddruppel}\\

\haiku{Deze gedachte.}{kwam  in hem op met een}{schok van ontzetting}\\

\haiku{Ja, hij moest het uit,.}{zijn hoofd steken die liefde}{was zonder toekomst}\\

\haiku{Waarom heeft hij u?}{niet gezegd dat gij zoiets}{niet mocht antwoorden}\\

\haiku{Maar de gevolgen...}{zijn een jaar vertraging in}{de bevordering}\\

\haiku{Hij was alzo naar;}{de burelen gegaan van}{de stasiemeester}\\

\haiku{Arie kon er niets aan,!}{doen maar het deed hem deugd hij}{was nu gevroken}\\

\haiku{Daar straks had hij in,.}{een roes geleefd nu volgde}{de ontnuchtering}\\

\haiku{hij had haar kunnen,,.}{helpen broederlik doch niet}{haar bezoedelen}\\

\haiku{De heiligschennis,?}{was gepleegd waarom zou hij}{ze niet herhalen}\\

\haiku{er moest ergens een.}{vermaledijding op zijn}{hoofd terecht komen}\\

\haiku{die kerel met die...}{slappe hoed die de pakken}{had afgegeven}\\

\haiku{om de voldoening!}{te smaken te weten dat}{men naar hem smachtte}\\

\haiku{Waarom zou dit voor?}{hem alleen noodlottige}{gevolgen hebben}\\

\haiku{Waarom sloeg men hem '?}{niet eens int gezicht en}{wierp men hem buiten}\\

\haiku{Al het lage van:}{die rol kwam op hem af als}{een donkere wolk}\\

\haiku{Hij schreef haar dat zijn.}{dienst plots veranderd was en}{hij niet kon komen}\\

\haiku{Stilaan begon hij,,}{evenwel te verademen}{er was niets gebeurd}\\

\haiku{De brief was reeds drie,,.}{weken oud sedert niets meer}{een plotse stilte}\\

\haiku{Zijn krachtige stem '.}{overheerstet geratel van}{het rollende tuig}\\

\haiku{Een snoodheid te meer:}{zou hij begaan hebben en}{toch twijfelde hij}\\

\haiku{- Ziet ge, grinnikte, {\textquoteleft}{\textquoteright}!}{hij tegen Arie altijd die}{hoopquand-m\^eme}\\

\haiku{Block ging voort daarop.}{en Arie droomde de hele}{nacht over Block's vrijheid}\\

\haiku{De vroeging bleef om.}{wat gebeurde iedere}{dag daar ver van hem}\\

\haiku{Met een kort gebaar,.}{eiste hij zijn Calepin}{die in orde was}\\

\haiku{hij volbracht slechts een,.}{nieuwe moord bracht een derde}{wezen ten onder}\\

\haiku{De mensen hadden.}{gelijk dat ze ermee geen}{rekening hielden}\\

\haiku{Eindelik, daar had,,!}{hij de vrije straat de reken}{huizen de vrije lucht}\\

\haiku{Ze keerden weerom,,.}{en stapten voort stapje voor}{stapje wandelend}\\

\haiku{De leugen die hij:}{had uitgevonden had een}{dubbel doel gehad}\\

\subsection{Uit: Het rollende leven. Deel 2}

\haiku{Nu moest hij niemand,.}{meer schoon spreken haalde zelf}{hout en kolen bij}\\

\haiku{waarheen hij maar keek '}{joeg hem de angst op het lijf}{en beter ware}\\

\haiku{ze liepen naar vo\'or,.}{het was de masjieniest die}{water moest nemen}\\

\haiku{'t Was wonderlik.}{om zien hoe de mannen daar}{stonden te blinken}\\

\haiku{dat, zo ik u voor, '}{uw nieuwjaar uw opslag niet}{kan ter hand stellen}\\

\haiku{Straks ging de trein aan '.}{t wiegen en hij strekte}{zich uit op de bank}\\

\haiku{Waar had hij gemeend?}{die steunstok te vinden}{die hij nodig had}\\

\haiku{Steeds loerde hij in.}{de verte of hij Baeyens}{niet naderen zag}\\

\haiku{Op de mand met klein.}{materiaal lagen de}{sleutels van de kast}\\

\haiku{dat hij meekreeg smeet.}{hij weg om niet te laten}{zien dat hij niet at}\\

\haiku{Block werd dik en hij.}{blies lijk een genter van dat}{trappen opklimmen}\\

\haiku{Hij moest zich op een}{stoel laten vallen om tot}{zichzelf te komen}\\

\haiku{'t Is erg, heel erg,.}{maar ge moet u alzo niet}{laten teneerslaan}\\

\haiku{En Arie vertelde.}{hem nog wat hij al wist van}{zijn loense gezel}\\

\haiku{Ze spraken wat over,.}{die ziekte over het geld dat}{dit alles kostte}\\

\haiku{Hij heeft er zeven, '!}{honderd opgemaakt op drie}{maandent kan gaan}\\

\haiku{Hij liep recht naar 't,.}{huis van Block waar zijn vrouw was}{maar ze wist het reeds}\\

\haiku{Hoewel uitgeput,.}{had Arie toch nog de kracht om}{blijheid te voelen}\\

\haiku{Hij kon zijn kind niet:}{zien zonder dat hij er twee}{beelden achter zag}\\

\haiku{voor hem was niets toch,.}{dan miezerie overal hij}{was maar liever dood}\\

\haiku{Hij was al buiten ',.}{zonder hijt wist in een}{vlaag van verstomming}\\

\haiku{Zie hoe grillig die.}{plant daar groeide in kromten}{en ellebogen}\\

\haiku{In zijn huis was ook ' '.}{iets aant gisten ent}{zou uiteenvallen}\\

\haiku{hij stortte nu weer '.}{een andere reddeloos}{int ongeluk}\\

\haiku{Ja, madam, 't is,.}{dat Arie een goeie jongen is}{dat hij zich l\'a\'at doen}\\

\haiku{Maar had hij anders, ':}{gevaren dan waret}{onverdiend geweest}\\

\haiku{Laermans leefde,:}{ervoor zijn leven bestond}{enkel uit dit \'een}\\

\haiku{Na een twintigtal.}{vruchteloze pogingen}{gaf hij het op}\\

\haiku{zou men zelf er nooit,!}{van genieten daar was nu}{juist niets aan te doen}\\

\haiku{De gladde baan en,,.}{vaag als spoken daarboven}{een bos van masten}\\

\haiku{Weg dat alles, weg,.}{zich steeds dieper in zichzelf}{keren en zwijgen}\\

\haiku{Zijn drijfveren zou,.}{ze nooit kennen ze zou hem}{dus nooit begrijpen}\\

\haiku{Er is maar Spirou,,... '.}{die vent zal u niets doen maar}{t is toch d\'at niet}\\

\haiku{Irma had een en,!}{ander nodig moest uit en}{kende geen woord Frans}\\

\haiku{ze verwilderden,.}{zienderogen zonder dat er}{iets aan te doen was}\\

\haiku{sedert ze getrouwd.}{waren had Irma nog maar}{e\'en nieuw kleed gehad}\\

\haiku{Met de arm in de}{draagband liep hij rond om de}{pijn te verbijten}\\

\haiku{hij kon  en sprak.}{zachter tegen haar dan hij}{in lang gedaan had}\\

\haiku{het scheen hem een tijd,!}{zonder einde het was toch}{reeds meer dan een uur}\\

\haiku{Gelukkig duurde,:}{zijn kwaadheid niet lang hij zag}{gauw klaar in hun spel}\\

\haiku{ze zouden hem nu}{weldra belet hebben naar}{zijn moeder te gaan}\\

\haiku{Niet gemakkelik}{om iets te vinden met die}{jongens en nog had}\\

\haiku{'k Zal er eens over,,.}{spreken als ge wilt aan een}{heel invloedrijk man}\\

\haiku{Hij was nu van een,.}{wonderlike zachtheid steeds}{van een groot geduld}\\

\haiku{En toch las hij in;}{Irma's ogen nog een angst van}{een andere soort}\\

\haiku{Maar zij zag in dat;}{alles slechts de gewone}{gang van de wereld}\\

\haiku{Zware stappen op '.}{de trap die ploften in de}{stilte vant huis}\\

\haiku{Zij waren beter.}{weg en hij alleen met zijn}{gesloten wezen}\\

\haiku{Hij verhaalde hen,:}{zijn hopen en verwachten}{doch schrikte toen plots}\\

\haiku{Zo was hij roerloos.}{gebleven en was alles}{over hem heengestroomd}\\

\haiku{Toch zou hij maar gaan '.}{slapen want morgen wast}{reeds om drie ure dag}\\

\haiku{ik kom binnen acht.}{dagen weer en we zullen}{het zien te schikken}\\

\haiku{Maar toch hoorde hij,}{die troostwoorden gaarne hij}{hoorde ze zelden}\\

\haiku{Drola gaf order te.}{vertrekken en ging mee met}{hen om een borrel}\\

\haiku{Hier ook, wist hij, lag.}{hij gewoonlik met alle}{leerlingen overhoop}\\

\haiku{het was de weervraak.}{van de dingen die erin}{lag en niets anders}\\

\haiku{Of deze bij de,.}{minister geweest was was}{niet uit te maken}\\

\haiku{zei hij en trok de.}{klink over van de rem die met}{heftig geraas sloot}\\

\section{Edward Vermeulen}

\subsection{Uit: Dagboek van een banneling}

\haiku{- Kijk eens hier, riep de,.}{hulpsecretaris die er}{aan het schrijven zat}\\

\haiku{Als het morgende,:}{was ik tot een en zelfde}{besluit gekomen}\\

\haiku{'t Was natuurlijk,.}{Louize die de eerste haar}{woorden gereed had}\\

\haiku{Ze loeg algelijk;}{en d'r kwamen tranen uit}{hare oogen gerold}\\

\haiku{'k Wist wel van waar.}{die tranen kwamen en wat}{er op volgen zou}\\

\haiku{we zagen er het;}{volk geerne en het volk was}{ons even genegen}\\

\haiku{Die week duurde lang,.}{want ik verlangde mijn hert}{uit naar den zondag}\\

\haiku{Augusta, morgen is ' ',}{t uw feest Enk wensch met}{dees gelegenheid}\\

\haiku{Voeg ik het mijne.}{opeengelaagd En in een}{bundel toegeknoopt}\\

\haiku{Vooruit maar, langs den,.}{Brugschen steenweg al het Land}{van Beloften rond}\\

\haiku{- Die Moezelwijn komt,,.}{van mijnheer Herbau's kasteel}{zegt Beck knipoogend}\\

\haiku{De duivelsdans houdt '.}{op ent rijk Van God rijst}{vreedzaam uit het slijk}\\

\haiku{we krijgen onze,.}{passen terug naarmate}{de inschrijvingen}\\

\haiku{Weerom al Duitsche,.}{huichelarij zijpelend}{van schijnheiligheid}\\

\haiku{Doe maar, 't gaat er,.}{allemaal door ze'n wegen}{dat ginder toch niet}\\

\haiku{'k Heb ze lang en,,,.}{veel bekeken die schoone}{edele sterke vrouw}\\

\haiku{de jongens meest op....}{een opengelegd bedde en}{ik in mijn zetel}\\

\haiku{Een leelijke Duits...}{nadert barsch en jaagt die}{damen brutaal weg}\\

\haiku{En als het hi\'er al,?}{vernield is links en rechts wat}{is het dan verder}\\

\haiku{De trein houdt in en,,.}{een treinwachter geeft ons op}{aanvraag de landskaart}\\

\haiku{slaap en mij noodigen,,!}{zoo zoet dringend en ik zou}{U vinden vinden}\\

\haiku{Ze legden liever,.}{hun zaken stil dan met den}{vijand te heulen}\\

\haiku{In den namiddag.}{wandelde ik eenzaam door}{de dennebosschen}\\

\haiku{Doch, duivelszak is.}{nooit vol en Nooitgenoeg is}{op den wereldbol}\\

\haiku{In elke landsche}{parochie kan men immers}{zonder te missen}\\

\haiku{Mijnheer pastoor van,,}{Bell ik zie u nog staan zooals}{toen met uw blauwen}\\

\haiku{- Die worden warm met,,,.}{te spelen kindje loech Dr}{Meeus blijf en speel maar}\\

\haiku{- Nee, nee, mijnheer de,,.}{Bestuurder wij verkochten}{er geen kaas geen kaas}\\

\haiku{We daalden af en,.}{stapten in de keuken waar}{de verpleegster was}\\

\haiku{- Maar ik zou toch zoo!}{harstochtelijk graag Onzen}{Lieven Heer vinden}\\

\haiku{- Blesse zei ze, de}{boerinne zou wel willen}{dat ik melk gaf lijk}\\

\haiku{'t Was ginder al '.}{voor de vette koe ent}{vet zwijn en hier ook}\\

\haiku{Nu op einde 1914:}{ligt dat rood ornement nog}{in het huis Mortier}\\

\haiku{Onze kanonnen.}{bulderden buitengewoon}{geweldig dien dag}\\

\haiku{Nog altijd lagen,.}{de vijanden voor de stad}{dicht langs den IJser}\\

\haiku{maar er werden meest.}{seminaristen van ons}{Vlaanderen gewijd}\\

\haiku{Il faut esp\'erer:}{que ce d\'epart ne sera}{pas n\'ecessaire}\\

\haiku{Hier in St-Joseph,;}{krijgen wij gewoonlijk noch}{melk noch boter meer}\\

\haiku{Hij stelde voor een.}{mijner onderpastors naar}{Vrankrijk te zenden}\\

\haiku{alle schikkingen:}{waren genomen voor de}{weerkomst naar Belgi\"e}\\

\haiku{'t Is algelijk;}{te peizen dat het al zoo}{zeere niet zal gaan}\\

\haiku{Wat er daar al van,.}{geworden is weten wij}{tot nu toe nog niet}\\

\haiku{Rusland, Vrankrijk en;}{Engeland veranderden}{van ministerie}\\

\haiku{In 't algemeen.}{zijn de soldaten moedig}{en vol betrouwen}\\

\haiku{Maar stillekens aan:}{komt men er van langs om meer}{tegen die zeggen}\\

\haiku{Sommige, in 't,;}{zuiden van Vrankrijk werken}{in de wijngaarden}\\

\haiku{Gelukkig dat men.}{nog boonen en rijst vindt om}{ze te vervangen}\\

\haiku{wij gerochten op}{het einde der maand en de}{zaken bleven lijk}\\

\haiku{zoo niet zal hij eene.}{heelkundige operatie}{moeten ondergaan}\\

\haiku{Het mag hun kosten,,;}{wat het wil zeggen zij ze}{moeten het hebben}\\

\haiku{Veurne heeft nog al.}{schade geleden en telt}{rond de twintig dooden}\\

\haiku{hij is eindelijk.}{uit zijn vel gesprongen en}{heeft vrake gevraagd}\\

\haiku{Veel armen worden.}{rijk en veel welstellende}{menschen worden arm}\\

\haiku{ook nog wat sous en.}{niet zelden zilvergeld zooals}{franks en halve franks}\\

\haiku{Geen twijfel of de '.}{Duitsch was een nieuw offensief}{aant bereiden}\\

\haiku{De Italianen.}{schijnen dezen keer wel te}{zullen wederstaan}\\

\haiku{De ceremonie;}{begint om 9 ure en duurt}{hier tot rond 11 ure}\\

\haiku{Ondertusschen wordt.}{het hier en elders van langs}{om meer dure tijd}\\

\haiku{De verbondenen.}{integendeel jubelden}{van vreugd en van hoop}\\

\haiku{{\textquoteright} De oorlog heeft veel;}{langer geduurd dan gelijk}{wie het voorzien had}\\

\haiku{maar in den grond is.}{de stad leelijk gekwetst en}{ellendig gesteld}\\

\haiku{hoogte die niemand.}{v\'o\'or den oorlog had kunnen}{voorzien noch droomen}\\

\haiku{een dag van vreugde,.}{van vriendschapsbetooging en}{van gelukwenschen}\\

\haiku{De zusters die met.}{de ouderlingen in Vend\'ee}{waren zijn er nog}\\

\subsection{Uit: De reis door het leven}

\haiku{Mijn leven duurde:}{lang en nochtans jammer ik}{niet met den profeet}\\

\haiku{God weet ons op tijd,.}{en stond te doen voelen dat}{wij voor hier niet zijn}\\

\haiku{God en de menschen.}{niet volmaakter bemind en}{gediend te hebben}\\

\haiku{Dan luidden de groote:}{klokken en we voelden er}{de zindering van}\\

\haiku{we waren immers;}{echte natuurkinders en}{de natuur is wild}\\

\haiku{De plaatsejongens:}{staken de koppen bijeen}{en sloten akkoord}\\

\haiku{De helft er van gaf,;}{ik aan moeder zonder den}{oorsprong te melden}\\

\haiku{dat zijn gespogen;}{menschen en de groote -n}{haan is Pickavet}\\

\haiku{Talia, vermaande,,.}{ze stil neem het gerust me\^e}{maar doe dat nooit meer}\\

\haiku{ons merrieke wil.}{dat leelijk spektakel van}{een vent niet kennen}\\

\haiku{Nu zit ik soms op.}{die tijden te peizen en}{den kop te schudden}\\

\haiku{Eens nochtans zag ik '.}{hem stopgezet ent was}{een vies geval ook}\\

\haiku{dan hadden wij toch...}{de leute toe en leute}{misten wij niet}\\

\haiku{Nu nog bederf ik:}{buiten eenieders wete}{een boeremeisje}\\

\haiku{dat zal wel in uw,,.}{gedacht schieten smeekte hij}{mij tegenhoudend}\\

\haiku{hij was een blok van,,;}{een jongen breed geschouderd}{geplant op zijn beenen}\\

\haiku{een wijden broek en,.}{een kort sarrootje op den kop}{een zijdene klakke}\\

\haiku{de plas opsprong en.}{al tusschen mijn beenen verdween}{door de weidehaag}\\

\haiku{Ik vertrok van daar,.}{lijdend als een verdoemde}{versuft en verstompt}\\

\haiku{Al voor ons hof, langs,.}{den wal hoorde ik een man}{over den weg stormen}\\

\haiku{Na een paar dagen:}{was er volle licht over de}{moordzaak gekomen}\\

\haiku{Nogeens gezeid, 'k.}{heb slechts enkele brokken}{kunnen bewaren}\\

\haiku{k zag hem toch zoo - ';}{geren Enk streelde hem}{op zijn dikken kop}\\

\haiku{Wie zou 't geweld '?}{ent gevaar der jonge}{jaren loochenen}\\

\haiku{We 'n hadden geen,.}{tien stappen gesteld of daar}{viel hij we\^erom stil}\\

\haiku{Ze verschoot, sloeg haar.}{naaiwerk af en keek mij met}{bewelkte oogen aan}\\

\haiku{zuchtte hij eens, we.}{waren altijd zoo bevriend}{met die familie}\\

\haiku{- Dat pakt mij, Gusten,,.}{en ik bedank u besloot}{de burgemeester}\\

\haiku{'t Nam al rond mij,.}{zijn vormen aan Van weif'lend}{trillend licht omdaan}\\

\haiku{hoe lavend, Hoe stil,,!}{hoe schoon hoe droom'rig zinkt Rond ons}{de trissche navend}\\

\haiku{Om 't eindigen,}{een hoftafereeltje dat}{ik honderden keeren}\\

\haiku{het voorgevoel was:}{van het ijselijkste wat}{mij kon overkomen}\\

\haiku{moeder, ge ziet er... - ',.}{mij een beetje ding uitk}{B\'e z-iek zei ze}\\

\haiku{Geen uur nadien was;}{moeder berecht en lag ze}{gerust op haar bed}\\

\haiku{Na jaren studie,,:}{na veel zielen doorgrond te}{hebben besluit ik}\\

\haiku{Mijn taak was, is en.}{blijft de kunste kennen van de}{volksziel te grijpen}\\

\haiku{de jongen zette.}{zijn kip open en moorelde}{zijn vooizeken uit}\\

\haiku{Weinige dagen.}{later was geheel de streek}{met Duitschers overstroomd}\\

\haiku{hij nam een stoel, plaatste.}{dien tegen den mijne en}{zat zoo nevens mij}\\

\haiku{- Ge kunt Oberst melden,,.}{dat het een uitdeeling van}{brandstof geldt zei ik}\\

\haiku{- Mijnheer, in volle,,?}{rechtzinnigheid zeg wat peist}{ge van den oorlog}\\

\haiku{We waren in 1917.}{en we begonnen het aan}{God op te geven}\\

\haiku{d'r kwamen stabel:}{meer soldaten aan en die}{klapten te luide}\\

\haiku{- Mijnheer Vermeulen,,.}{zei de Orts ik moet u tot}{mijn spijt aanhouden}\\

\haiku{Ik volgde hem en,.}{werd geleid in een plaats waar}{een jonge heer was}\\

\haiku{- 'k Had sedert mijn;}{gevangzetting twaaf kilos}{gewicht verloren}\\

\haiku{Ge kunt dan morgen.}{vroeg uit Roeselare naar}{Hooglede reizen}\\

\haiku{Hij stond recht, rood van,...}{opgewondenheid reikte}{de hand en vertrok}\\

\haiku{- Neen, steigerde hij,.}{alle menschen moeten zich}{kunnen gerieven}\\

\haiku{En midden deze;}{smertelijkheid wrocht ik bij}{dage als een peerd}\\

\haiku{'k zou mogelijks.}{ook met den daveraar}{op het lijf zitten}\\

\haiku{- 't Is er eene van,,.}{Doktoor Vandeputte zei}{ze een glas vullend}\\

\subsection{Uit: De vracht}

\haiku{- Om geheel heilig,,.}{mogelijks om vermuft te}{worden gekte hij}\\

\haiku{Lode kwam terug.}{in de keuken en ontstak}{een versche sigaar}\\

\haiku{De menschen noemen,,.}{dat hier anders ja met een}{gruwelijken naam}\\

\haiku{hang haar wat leugens,.}{op want ze zal u lastig}{vallen met vragen}\\

\haiku{Een weinig later.}{kwam hij terug ontbeet en}{was reizensgereed}\\

\haiku{Mijnheer Lode is;}{dan gaan loeren en heeft den}{hond kapot gemaakt}\\

\haiku{Ze vielen toe in;}{De Kollebloem als een hond}{in den hutsepot}\\

\haiku{er waren  er,;}{aan de bolle anderen}{aan het kaartspelen}\\

\haiku{- Naar 't Berenhol,,.}{fluisterde hij zijn makker}{een handstoot gevend}\\

\haiku{Lucas zweeg en scheen.}{de teljooren te tellen}{op den kavebank}\\

\haiku{En wel opletten,.}{dat ge nooit verre van hem}{af zijt als hij swalpt}\\

\haiku{De huisbaas schoot in.}{een kletterenden lach en}{hield zich den buik vast}\\

\haiku{- Die duivel van een,,, ',!}{deugniet loech de vrouw m\^ee ja}{hij wast. Toe Klaas}\\

\haiku{Klunten verhaalde.}{hem wat dien namiddag en}{ook des avonds voorviel}\\

\haiku{- Hem in 't water,,;}{gooien snakte Urbaan hem}{te koelen leggen}\\

\haiku{Nog binst dienzelfden;}{nacht klopten lieden van het}{dorp bij Lucas aan}\\

\haiku{- Of omdat hij geen ',.}{maag meern heeft wierp er een}{boertje brutaal op}\\

\haiku{Ons lichaam is een.}{rijk en de maag is er het}{goevernement van}\\

\haiku{De deur ging open en.}{Klaas De Zwingel toonde zijn}{leelijke tronie}\\

\haiku{En d' echo rolde ',.}{t roepen voort Van deur tot}{deur en poort tot poort}\\

\haiku{Klaas sprak geen woord, maar,.}{trok de schouders op zonder}{naar hem te kijken}\\

\haiku{- Haja, precies, daar,!}{hebben wij djanters nog een}{akkoord waarachtig}\\

\haiku{- Ik weet toch alles,;}{wat u pijnigt doch zeg maar}{op en verzwijg niets}\\

\haiku{met Leona trouw,!}{ik en niemand kan noch zal}{het verhinderen}\\

\haiku{- Wat meer is, 'k ben.}{mo\^e van bespot te zijn om}{mijn overtuigingen}\\

\haiku{- Z'is algelijk ook,.}{nog wat geschonden aan den}{neus fluisterde hij}\\

\haiku{... ketterde Lode,.}{zich onderbrekend met een}{vuistslag op den knie}\\

\haiku{Lode sprak niet, bracht; '}{den avond ingesloten door}{en ging vroeg slapen}\\

\haiku{Klaas reikte zijn glas ' -,!}{en Lode vuldet. Ziet}{ge waterkwezel}\\

\haiku{hij is hooge geleerd,,:}{menheere en de menschen}{zijn zot achter hem}\\

\haiku{en zoo krijgt hij een.}{druppeltje langs hier en een}{druppeltje langs daar}\\

\haiku{Op een avond, daar hij,.}{wel alleen met zijn broer was}{pakte hij hem aan}\\

\haiku{gilde Lode, zoo:}{bleek als de dood en hij sprong}{op en raasde}\\

\haiku{Geheel dien dag en;}{volgende dagen woog druk}{in het huishouden}\\

\haiku{En of er nu nog?}{een helle ware wat kan}{dat ons verdoemen}\\

\haiku{Op het zicht van den.}{jonkheid grinnikte hij en}{reikte hem de band}\\

\haiku{- Haja, precies, 'k.}{peisde dat ge van entwat}{anders doende waart}\\

\haiku{Hoe ellendiger,.}{die werd hoe venijniger}{en gevaarlijker}\\

\haiku{Na de mis zag hij}{het volk getroppeld op het}{kerkeplein en ving}\\

\haiku{Nu stonden beide:}{echtgenooten voor een}{afgewrochte taak}\\

\haiku{- Zorg, dat er bij ons.}{afstappen te Ranck een}{auto gereed sta}\\

\haiku{- t' Akkoord, knikte,?}{Lode maar wie verwittigt}{er papa en Jan}\\

\haiku{Fernandje ging naar.}{oom Urbaans kamer en ging}{bij het bed zitten}\\

\haiku{In uwe plaats zou ik.}{rustig blijven en zooveel}{mogelijk slapen}\\

\haiku{Hij wendde zich om,,}{en zijn blikken vielen op}{Urbaan die stijf lag}\\

\haiku{Ondertusschen ging.}{Lode naar De Zwingels en}{spelde hem de les}\\

\haiku{Klaas knikte, altijd,.}{gedwee doch met een groote neep}{tusschen de leepoogen}\\

\haiku{Benauwdheden van.}{dit en van dat. Alsof ik}{nog een kind ware}\\

\haiku{met het groot verlof;}{kwam hun jongen naar huis als}{primus van zijn klas}\\

\haiku{De glim vervloog van '.}{s jongens wezen en hij}{keek staal en doelloos}\\

\haiku{daar streek hij de hand,:}{over zijn voorhoofd bewreef zich}{de oogen en besloot}\\

\haiku{- Trek gij het u aan,,:}{smeekte hij ge kunt geheel}{de schreef uitvagen}\\

\haiku{Enfin... ja toch, die...}{kerk en dat orgespel en}{al die plechtigheid}\\

\haiku{Als Helena naar,.}{huis ging kwam Lode haar toch}{gejaagd te gemoet}\\

\haiku{Ze liet hem, ging naar.}{de ziekenkamer en bleef}{voor het bedde staan}\\

\haiku{hij moest haar hooren.}{en zien en dikwijls binst den}{dag met haar spreken}\\

\haiku{zondag te nonkel '.}{Jans ens anderdaags bij}{notaris Pierszoon}\\

\haiku{Het duurde al niet - '}{lang eer de plagers int}{bijzonder nonkel}\\

\haiku{Wat beteekende dan:}{uwe vermaning van tijdens}{uw doodelijke ziekte}\\

\haiku{Helena bezag.}{hem weemoedig en tranen}{sprongen uit haar oogen}\\

\haiku{Er lag een kwade.}{grijns op Fernands wezen en}{Helena zag het}\\

\haiku{monkelde Fernand,.}{zich bij die eenzijdige}{beschouwing houdend}\\

\haiku{- Zeg maar, stamelde ' - ',.}{t.k Moest u zeggen dat}{ik u geerne zie}\\

\haiku{Wat aangaat de woonst,:.}{daarmee zit ik ook niet in}{we kunnen wachten}\\

\haiku{Weet ge dan niet dat?}{liefde de grootste aller}{babbelkousen is}\\

\haiku{- beklagen u, want....}{Fernand lei zijn meestergast}{stil met een tongslag}\\

\haiku{De zekerheid van.}{die komende zaligheid}{miek hem overmoedig}\\

\haiku{Fernand werd met den.}{dop zoo bleek als een lijk en}{kon geen woord uiten}\\

\haiku{In Gods naam, smeekte,,?}{hij zeg mij hebt ge wel de}{waarheid gesproken}\\

\haiku{De overste dankte.}{hertelijk en stond op om}{afscheid te nemen}\\

\haiku{'k zie het, maar Ons.}{Heer is bermhertig en wij}{moeten het ook zijn}\\

\haiku{Hij bezag lachend.}{zijn neef en de lach verdween}{rasser dan hij kwam}\\

\haiku{Ziet ge wel, vriend, dat.}{iemand er onder bezweek}{en zoo uitboette}\\

\haiku{- Ja en trek het daar,.}{niet lang in dat krottekot}{vermaande Lode}\\

\haiku{hij stak beide zijn.}{armen in de hoogte en}{rekte zijn lijf uit}\\

\haiku{Hij miek mij wijs, dat.}{hij gespeculeerd en groote}{winsten gedaan had}\\

\haiku{alles goed maken,...}{en hem vrede bezorgen}{beaamde Fernand}\\

\haiku{Vanher bedekte.}{hij zijn wezen en wachtte}{geen antwoord meer af}\\

\haiku{'k Wist niet dat een,.}{mensch zoo lijden kon zonder}{er van te sterven}\\

\haiku{Lode weerde die.}{hulde met armgezwaai af}{en bleef gezeten}\\

\haiku{- Hij was ons altijd,.}{een trouwe ziel herhaalde}{hij meermaals dien avond}\\

\haiku{- Ge zult ten minste,,,.}{rusten slaapt ge niet zei ze}{den stoel aanwijzend}\\

\haiku{Hoe fijn en kiesch ook,.}{ze te werk gingen vatte}{hij goed hun inzicht}\\

\section{Hugo Verriest}

\subsection{Uit: Op wandel}

\haiku{Maar hoe droef blijft mijn.}{hert voor die zee van wee die}{ik niet dijken kan}\\

\haiku{En toch is hij een,.}{fatsoenlijk man om zien en}{is het inderdaad}\\

\haiku{- Een gevoelen van,.}{ruimte spreidt rondom ons en}{vrede walmt binnen}\\

\haiku{niet alleen in de,.}{oogen maar tot in den vorm van}{hoofd en mond en kin}\\

\haiku{Het mos, hier en daar,.}{priemt er door en groeit nevens}{en rond de boomen}\\

\haiku{Hij lacht en knikt en.}{in den gang zwakt in lichten}{zwier door zijn leden}\\

\haiku{en nog, nog is die,.......}{zoetheid niet doorstralende}{maar wat versmolten}\\

\haiku{Gij hebt ongelijk -.}{van dat te gevoelen maar}{ik gevoel het}\\

\haiku{Geheel die kerke.}{is een verre nichte van}{Shakespeare}\\

\haiku{Een vlaamsche vrouw houdt,;}{van geen pronken Zij is vol}{moed en lacht zoo schalk}\\

\haiku{Houdt u dus wel van.}{t'huis te ronken En komt bij}{tijds tot in de Valk}\\

\haiku{Daaruit komt en wendt.}{en keert de nieuwe eeuwe}{en de nieuwe tijd}\\

\haiku{- Een lichte smoor, een,,.}{doorschijnende mist hangt in}{de lucht over de zee}\\

\haiku{'t Is vlottend en,!}{waaiend mat maar van peerlen}{glinsterend zilver}\\

\subsection{Uit: Regenboog uit andere kleuren}

\haiku{Ik brenge U wat:}{ik hier en daar gevonden}{hebbe en gezien}\\

\haiku{Over het huizeken,;}{gebogen waaien hunne}{ijdele kruinen}\\

\haiku{Hoe schoon ligt zij daar,,,.}{in haar beddeken bleek wit}{in witte lakens}\\

\haiku{maar daar heeft men hun ';}{int stille geraden}{van te verhuizen}\\

\haiku{De wind moet waaien.}{rond zijnen kop en hoed en}{door zijne kle\^eren}\\

\haiku{en als ik hem ging:}{bezoeken schudde hij zijn}{hoofd traagzaam en zei}\\

\haiku{Vader en moeder.}{zijn te neerstig en moeten}{te zeere werken}\\

\haiku{Het is bewoond, en.}{schijnt altijd sedert eenige}{dagen verlaten}\\

\haiku{Tichels en pannen,}{en bouwstoffen verkoopt hij}{en bedriegt zoo veel}\\

\haiku{Hij is christelijk, {\textquoteleft},{\textquoteright}, -;}{enkwijt zijne plichten wel}{hij gaat naar de kerk}\\

\haiku{Paschen, den eenen,.}{of anderen wekedag}{als er min volk is}\\

\section{Frans Verschoren}

\subsection{Uit: Jeugd}

\haiku{Deezeke schudt zijn....}{beddeken uit en laat de}{pluimekes vliegen}\\

\haiku{hij stond tegen den '.}{mesthoop en was een groot ei}{aant uitblazen}\\

\haiku{Daar kwam iets piepen,, '.}{rozig-geel stukske lijf}{vlak voort gaatje}\\

\haiku{{\textquoteright} De Pad liep zijn beurs,;}{kassers halen onder in}{zijn bed verborgen}\\

\haiku{Ze gooiden ze weg,,.}{in een gracht en beenden toen}{rap voort de hei-in}\\

\haiku{{\textquoteright} De Sooi neep zijn vuist.}{dichter toe om het dierke}{te doen stilblijven}\\

\haiku{ze brandgloeiden op.}{zijn kaken en pitsten de}{tranen uit zijn oogen}\\

\haiku{gelukkiglijk dat.}{de boschwachter in-tijds}{nog op-zij sprong}\\

\haiku{Twee keeren had kromme, '.}{Pol het hem voorgedaan in}{t Gasthuiskerkske}\\

\haiku{de snikken rukten,.}{scheurend uit zijn keel koortsig}{schokkend heel zijn lijf}\\

\haiku{Sooike verstond heur;}{niet en liet zich gewillig}{naar boven leiden}\\

\haiku{Met een luiden gil;}{schokte hij wakker uit zijn}{akeligen  droom}\\

\haiku{Rondom, de ouders,,.}{loerend naar hun kinderen}{aangedaan en fier}\\

\haiku{En Verboven had,,.}{straf de stommerik voor dat}{smijten met zijn klak}\\

\haiku{Maar ze twijfelden, ',!}{al gauw weer wantt was zoo'n}{gelukszak die Sooi}\\

\haiku{{\textquoteright} {\textquoteleft}'k Heb niks gedaan,,{\textquoteright};}{Meester stamelbrabbelde}{de bange jongen}\\

\haiku{Ge zijt zeker wel,,?}{tevreden over Sooi niet waar}{Mijnheer Van Truijen}\\

\haiku{ze spraken af, met,,;}{hun oogen en ze bleven stil}{den heelen morgen}\\

\haiku{lustig verkondend.}{dat het uit was met leeren en}{stil zijn voor vandaag}\\

\haiku{Recht naar de vitrien,.}{van Fien Savelkoel die een}{snoepwinkeltje hield}\\

\haiku{Maar hij hield zich kloek,.}{als een bestige baas die}{er alles afwist}\\

\haiku{Nu was het spoedig.}{klaar waar de gast de centen}{vandaan had gehaald}\\

\haiku{Niks aan gedacht en,!}{zuiver geenen honger gevoeld}{met dat meuleke}\\

\haiku{en hij drong door 't.}{volk en stond te gapen met}{begeerige oogen}\\

\haiku{dat was voor rijke,;}{menschen alleen die veel geld}{konden verteren}\\

\haiku{Lenig boog zijn lijf.}{en zijn vuist omknelde de}{zware  barre}\\

\haiku{niemand is verplicht,;}{te geven daar g'allemaal}{uw plaats hebt betaald}\\

\haiku{Dat hadden ze nog.}{nooit gezien en konden ze}{zeker niet krijgen}\\

\haiku{Nu was het, op school,.}{een stoeffen en boffen op}{zijn tamme kouwke}\\

\haiku{Ze geloofden er,;}{niets af van al wat hij}{hun wijsmaken wou}\\

\haiku{{\textquoteright} schreeuwde de vogel ',.}{daar boven opt dak en}{tjippelde verder}\\

\haiku{Tuurke streelde zijn,,,:}{brave kouwke en nu klonk}{het fier gewichtig}\\

\haiku{Dat hadden ze nog,,.}{nooit gezien zoo'n tamme kraai}{zoo slim en zoo tam}\\

\haiku{Al heel vroeg had ze,;}{hem naar school gezonden met}{hoop op beterschap}\\

\haiku{Ze beriepen zich,;}{op de omstaande kijkers}{die moesten getuigen}\\

\haiku{'k Ben van den nacht, ';}{naar d'hel geweest de duvels}{waren aant eten}\\

\haiku{En nu zouden ze,,!}{eens zien peinsde Sooike wie}{baas was boven-al}\\

\haiku{Daar bleven ze staan,,!}{keken rond en ze zagen}{geen levende ziel}\\

\subsection{Uit: Van een jongen die geluk had}

\haiku{En Verboven had,,.}{straf de stommerik voor dat}{smijten met zijn klak}\\

\haiku{Maar ze twijfelden, ',!}{al gauw weer wantt was zoo'n}{gelukszak die Sooi}\\

\haiku{{\textquoteright} {\textquoteleft}'k Heb niks gedaan,,{\textquoteright};}{Meester stamelbrabbelde}{de bange jongen}\\

\haiku{Ge zijt zeker wel,,?}{tevreden over Sooi niet waar}{Mijnheer Van Truijen}\\

\haiku{ze spraken af, met,,;}{hun oogen en ze bleven stil}{den heelen morgen}\\

\section{Hans Vervoort}

\subsection{Uit: Eerlijk is vals}

\haiku{Ik kwam vrij snel in.}{mijn e-mail terecht en}{er was een bericht}\\

\haiku{De baan vond ze leuk.}{en hij was ook een goede}{baas als hij niet dronk}\\

\haiku{{\textquoteright} {\textquoteleft}Zo is dat. Het is.}{uit met het verwennen van}{zielige oudjes}\\

\haiku{Toen ik hem na de,.}{oorlog eindelijk bewust}{zag was ik zes jaar}\\

\haiku{Ik zag die dag mijn,.}{moeder weer in haar oude}{doen vief en vol praats}\\

\haiku{De vader van mijn.}{vader woonde inderdaad}{in het pension}\\

\haiku{Ik werd er na de.}{zoveelste bezoeker zelfs}{wat korzelig van}\\

\haiku{Die Janine had.}{toch maar een goed inzicht in}{de kantoorpsyche}\\

\haiku{Volgens Janine,}{was ze vijfentwintig maar}{wat mij betreft had}\\

\haiku{{\textquoteleft}Laat ze gaan,{\textquoteright} riep Jaap, {\textquoteleft}.}{ons achternaze moeten}{morgen weer vroeg op}\\

\haiku{We sliepen uit, en.}{ik bracht haar een uitgebreid}{ontbijt op bed}\\

\haiku{H\`e lekker, zo'n warm,{\textquoteright}, {\textquoteleft}.}{bankje zei ikdie mensen}{hebben goed zitvlees}\\

\haiku{Die springen wat rond,.}{in het weiland floepen hun}{tong om een muskiet}\\

\haiku{{\textquoteright} Marijke kraaide,.}{even goddank kon ik haar nog}{steeds laten lachen}\\

\haiku{En d\'at schijnt wel kans,.}{te hebben die knobbeltjes}{kun je wel fokken}\\

\haiku{De heer Faverey.}{van Holland Recherche was}{zo goed als zijn woord}\\

\haiku{{\textquoteright} {\textquoteleft}Ja, natuurlijk, we,.}{hebben samen ontbeten}{om een uur of acht}\\

\haiku{{\textquoteright} vroeg Janine, die.}{een paar seconden na hun}{vertrek binnenkwam}\\

\haiku{Want alhoewel zij,.}{geen steek kon naaien wist zij}{alles van mode}\\

\haiku{Geen enkele prijs,.}{was voor haar heilig er kon}{altijd wel wat af}\\

\haiku{Die Soedarso heeft je,.}{vader vermoord en nu zoekt}{de dochter contact}\\

\haiku{Hoe kom je erbij?}{dat mijn vader door Soedarso}{is neergeschoten}\\

\haiku{Violet at iets,.}{vegetarisch ik had de}{gevulde kalkoen}\\

\haiku{{\textquoteright} {\textquoteleft}Ja, dat stond ook in,...}{het verslag van Faverey}{maak je geen zorgen}\\

\haiku{Als u mij haar adres,.}{geeft dan neem ik zelf verder}{contact met haar op}\\

\haiku{En als wij eruit,,.}{stappen zoals nu missen}{we die controle}\\

\haiku{Dat was nodig voor}{haar genezing en ik wist}{dat wat ze opschreef}\\

\haiku{Ze hield plotseling,?}{op met schrijven had ze mijn}{nabijheid gevoeld}\\

\haiku{Ze kenden het adres,.}{in Amsterdam wel want daar}{zaten ze vroeger}\\

\haiku{{\textquoteright} Zo makkelijk liet.}{hij zich natuurlijk niet in}{de luren leggen}\\

\haiku{Jullie gaan wel ver.}{in het beschermen van je}{nieuwe directeur}\\

\haiku{zich ziek, ze durfde.}{zich niet te vertonen met}{die blauwe plekken}\\

\haiku{{\textquoteright} Hij stapte op en.}{was de deur uit voordat we}{pap konden zeggen}\\

\haiku{Hij was somber en.}{ik vroeg wat hem dwarszat en}{bleef daaraan trekken}\\

\haiku{Een kalverliefde,.}{noemde hij het hij moest nog}{volwassen worden}\\

\haiku{In de drukte en.}{het geroezemoes sprak ik}{ook Anton nog even}\\

\haiku{Ze schrok er zelf even.}{van en zond mij een kleine}{schuldige blik toe}\\

\haiku{{\textquoteleft}Ik heb vierhonderd,?}{dollar bij me geef jij dat}{straks aan die pastoor}\\

\haiku{{\textquoteleft}Ik was nieuwsgierig,.}{ik wilde weten of ik}{nog familie had}\\

\subsection{Uit: Encyclopedie van op het nippertje geredde kennis (en andere stukjes om te lezen)}

\haiku{Nipper 3 In mijn:}{jeugd riepen opgeschoten}{jongens naar elkaar}\\

\haiku{Het was ook niet op,.}{vrouwen gericht het was een}{algemene yell}\\

\haiku{Ik kwam in 1956 op,.}{een accountantskantoor te}{werken 17 jaar oud}\\

\haiku{de Tweede Kamer....}{had weer weinig in de melk}{te brokkelen}\\

\haiku{Tot ze op een dag,:}{vertrokken is met een kort}{vaarwel-briefje}\\

\haiku{nooit had hij op zijn.}{leeftijd deze liefde nog}{mogen verwachten}\\

\haiku{Ik zat met hem in.}{een forum en na afloop}{praatten we nog wat}\\

\haiku{Maar Jaspars werd ziek.}{en wond daar op radio en}{TV geen doekjes om}\\

\haiku{Soms heeft blindheid z'n.}{goed kanten want hij had zijn}{uiterlijk niet mee}\\

\haiku{Op zolder stond een,,:}{heel groot bed Daar sliep een kind}{in opgelet}\\

\haiku{Er kwam een zeehond.}{uit de zee En gleed in bed}{als nummer twee}\\

\haiku{Rare snuiters over,.}{het algemeen met wie je}{weinig contact had}\\

\haiku{wat moest je vroeger!}{timmeren om letters op}{papier te krijgen}\\

\haiku{Wij zetten prompt ook,.}{onze koffers even neer om}{mee te luisteren}\\

\haiku{Dan kan je daarna.}{zonder schuldgevoel doen wat}{je zo graag wilde}\\

\haiku{U weet namelijk!}{al precies dat uw vlucht twee}{uur vertraging heeft}\\

\haiku{Ze kan het toch niet,.}{horen heb ik inmiddels}{wel begrepen}\\

\haiku{En toch, vreemd genoeg,.}{kun je niet toe met \'e\'en van}{deze twee woorden}\\

\haiku{Het zal 2010 geweest}{zijn toen ik met uitgever}{Vic van de Reijt}\\

\haiku{Toch enigszins geschokt.}{bereikte de oude heer}{veilig zijn huis}\\

\haiku{Ze zijn niet aaibaar,.}{en je ziet ze hooguit als}{een schicht wegvliegen}\\

\haiku{En de vis hadden.}{we al geselecteerd bij}{de vermaaksfunctie}\\

\haiku{Maar de hond hadden,.}{we al geselecteerd dus}{dat is geen probleem}\\

\haiku{hun onderbeen is.}{een paar ons lichter dan dat}{van blanke renners}\\

\haiku{{\textquoteleft}Hou dan toch op met,{\textquoteright}.}{dat schoonmaken fluisterde}{mijn vader terug}\\

\haiku{Ik liep op haar af,.}{bukte me en drukte mijn}{lippen op haar wang}\\

\haiku{{\textquoteright} Mijn vader, achter,.}{het stuur ging ongemerkt wat}{harder rijden}\\

\haiku{Zuster van Anken.}{vroeg mij om te helpen met}{de begrafenis}\\

\subsection{Uit: Geluk is voor de dommen}

\haiku{Ik vreesde dat het.}{kwam omdat mijn hand zweette}{als ik nerveus was}\\

\haiku{John was bijna,.}{twee meter lang en werkte}{graag met gewichten}\\

\haiku{Totdat ze een keer.}{een suikerspin bij me kocht}{en nog wat candy}\\

\haiku{Maria kocht \'e\'en keer.}{een suikerspin bij mij en}{ik zag het meteen}\\

\haiku{Dus had ik ook geen,.}{concurrentie Maria was}{helemaal van mij}\\

\haiku{{\textquoteright} {\textquoteleft}Misschien wel, maar in.}{elk geval moet ik dan toch}{in de buurt blijven}\\

\haiku{Geschrokken deinsde.}{ik achteruit terwijl de}{deuren opengingen}\\

\haiku{Daar stond Charlotte,,.}{in compleet tenue ze had}{zelfs haar hoedje op}\\

\haiku{Ik trok haar haastig.}{naar binnen en deed de deur}{achter haar op slot}\\

\haiku{Ze liep op me af.}{en trok de schouders van mijn}{regenjas omhoog}\\

\haiku{Hij ontsloot het hek,:}{en terwijl ze het tuinpad}{betraden riep hij}\\

\haiku{Ik deed de deur dicht.}{en duwde ze de gang door}{naar de woonkamer}\\

\haiku{Steeds weer kwam bij mij.}{de herinnering aan Muf}{Engel naar boven}\\

\haiku{Ik wist wat voor kind,.}{ik in huis had wat hij kon}{en wat hij niet kon}\\

\haiku{Alleen de angst voor.}{een ongeluk kon ik niet}{uit mijn hoofd zetten}\\

\haiku{een moederkloek dan.}{Melanie en ik samen}{ooit geweest waren}\\

\haiku{waarom ik hem dat,.}{niet zou gunnen met Peter}{deed ik het elk jaar}\\

\haiku{Dus heb ik verteld,.}{dat autorijden niet van}{jou mocht en waarom}\\

\haiku{Nooit eerder was tot.}{me doorgedrongen dat dat}{moment zou komen}\\

\haiku{Dit zijn de longen,..{\textquoteright},!}{van de stad jongen Goh de}{longen van de stad}\\

\haiku{Maar Jeannette:}{had me nu het vuur na aan}{de schenen gelegd}\\

\haiku{Neuken, hij zei het,.}{een beetje geaffecteerd}{met gespitste mond}\\

\haiku{Meestal moest er.}{harder gewerkt worden voor}{zo'n geluksgevoel}\\

\haiku{{\textquoteleft}Een Ouwenaardje?}{of een Veluwse Durk of}{de Krielknaber}\\

\haiku{{\textquoteright} Nee, dat viel niet uit,.}{te leggen dat was ze bij}{Koos ook niet gelukt}\\

\haiku{Vier poten en een,.}{rond middenstuk voorzien van}{leren bekleding}\\

\haiku{Dat  houdt het bloed.}{draaiende en wekt de gal}{in de lever op}\\

\haiku{{\textquoteright} En tegelijk kreeg.}{ik een telefoontje van}{mijn oude vriend Carl}\\

\haiku{Want hij was toch de,?}{man achter Manuel in}{die tv-beelden}\\

\haiku{Dit spel reageert.}{op de klank van opperste}{gelukzaligheid}\\

\haiku{Dick deed ons verslag.}{van het heen en weer gedraaf}{tussen de scores}\\

\haiku{Maar ik wil niet dat,.}{mijn zoons die foto's zien als}{ik er niet meer ben}\\

\haiku{hoeveel moeite ons.}{bedrijf deed om discretie}{te verzekeren}\\

\haiku{Maar het maakte ons.}{liefdesleven daarna wel}{realistischer}\\

\haiku{{\textquoteleft}Pijnstiller{\textquoteright} stond er, {\textquoteleft}{\textquoteright}.}{met de hand bij geschreven}{max. 6 per etmaal}\\

\haiku{En na de foto's:}{van Tante Bettina deed}{ik dat altijd braaf}\\

\haiku{in dat huwelijk.}{blijven zitten en jou niets}{te laten weten}\\

\haiku{In het caf\'e kwam.}{dat na een pilsje of wat}{altijd naar boven}\\

\haiku{Gelukkig had ik.}{voldoende pils in  huis}{voor een lange avond}\\

\haiku{Maar die moesten dan wel.}{opvallen en een beetje}{telegeniek zijn}\\

\haiku{Grashoek stuurde mij.}{via de mail een bedankje}{voor mijn recensie}\\

\haiku{Ik ben zo dankbaar.}{voor wat aandacht dat ik niets}{durf te weigeren}\\

\haiku{Ten slotte waagde.}{hij het er maar op en bracht}{de fiets weer op gang}\\

\haiku{Een gevoel dat zo?}{snel wegebt kan toch nooit iets}{betekend hebben}\\

\haiku{Het meisje zoog op.}{haar duim en hield een pop in}{haar armen geklemd}\\

\haiku{Toen hij wegfietste,,.}{keek hij nog even om maar ze}{was niet meer zichtbaar}\\

\haiku{{\textquoteright} Wat een onzin kraam,,,?}{ik uit dacht hij wie is zij}{waarom zeg ik dit}\\

\haiku{Hij transpireerde,:}{en beefde en stond op sloeg}{zijn broek af en zei}\\

\haiku{Hij voelde hoe ze.}{zijn broekriem lostrok en zijn}{geslacht liet zwellen}\\

\haiku{Haar handen woelden.}{door zijn haar en hij kreeg het}{warmer en warmer}\\

\haiku{{\textquoteright} zei ze ineens en.}{hij richtte zijn hoofd op en}{keek haar verbaasd aan}\\

\haiku{Een paar jaar later:}{begon vanuit Amerika}{de revolutie}\\

\haiku{In haar loge stond,.}{Elsa overeind haar hand voor}{de mond geslagen}\\

\haiku{Misschien was ze bang.}{dat zijn ouders haar handschrift}{zouden herkennen}\\

\haiku{Nou ja, twee dagen,.}{Londen z\'o bijzonder was}{dat nu ook weer niet}\\

\haiku{{\textquoteleft}Auf wiederseh'n,,,.}{auf wiederseh'n we'll meet}{again  someday}\\

\haiku{Van de drie flessen.}{wijn was minstens de helft in}{zijn keelgat beland}\\

\haiku{De euforie van.}{de nachtelijke zoen was}{allang verdwenen}\\

\haiku{De zorgen zijn voor,.}{morgen en hij was al met}{al wel een lieverd}\\

\haiku{{\textquoteright} riep hij vanuit de.}{kooi die Christiaan op de}{achterbank zette}\\

\haiku{{\textquoteright} {\textquoteleft}Koffer drie,{\textquoteright} riep ze.}{en ze begon zich in de}{rolstoel te hijsen}\\

\haiku{Ze durfde er niet,.}{om te vragen omdat ze}{het antwoord wel wist}\\

\haiku{Alsof hij het elk.}{moment tot een ru{\"\i}ne}{kon samenknijpen}\\

\haiku{{\textquoteright} {\textquoteleft}Kijk eens om je heen,,.}{de planten hebben er geen}{last van zo te zien}\\

\haiku{Ze was te saai voor,.}{hem te zeer tevreden met}{een rustig bestaan}\\

\haiku{Alleen als ze dood,.}{was zou hij het krijgen en}{zijn gang kunnen gaan}\\

\haiku{Ze moest hier zo snel,.}{mogelijk weg tijd vinden}{om na te denken}\\

\subsection{Uit: Heden mosselen, morgen gij}

\haiku{{\textquoteright} Dat luchtte op, maar,.}{toch vijftienduizend gulden}{was ook een smak geld}\\

\haiku{{\textquoteleft}Kanonnier Dieriks.}{meldt zich met 1 paar schoenen}{voor reparatie}\\

\haiku{Het regende al.}{de hele dag op en af}{en niemand was droog}\\

\haiku{{\textquoteleft}Poef{\textquoteright}, riep ik terwijl,.}{ik de trekker overhaalde}{maar het deed hem niets}\\

\haiku{is het hele pand}{nog een keer uitgebrand met}{veel Marokkanen}\\

\haiku{Ze keek me moeizaam.}{aan en maakte een kleine}{kokhalsbeweging}\\

\haiku{{\textquoteright} Ik kuste haar, ze,.}{bleef in bed zwaaide nog met}{een hand en vertrok}\\

\haiku{Of je nu bruine,.}{bonen of opinies inblikt}{dat maakt toch niets uit}\\

\haiku{Hij zag er fris uit,:}{als altijd de man zonder}{kleine zwakheden}\\

\haiku{Maar zijn ogen waren.}{wat te vochtig en keken}{je te vragend aan}\\

\haiku{{\textquoteright} {\textquoteleft}Tjongejonge{\textquoteright}, zei,.}{Henk en goot voorzichtig wat}{Cassis naar binnen}\\

\haiku{Een grote vrouw en.}{3 kinderen zaten thuis}{op hem te wachten}\\

\haiku{Ik zag het meteen,.}{zitten maar John moest niet}{geforceerd worden}\\

\haiku{twee krissen die ik,.}{nog kende souvenirs uit}{de tijd van weleer}\\

\haiku{Laatst mochten Jan en.}{vriend mee om een nieuw gemaal}{te bezichtigen}\\

\haiku{Met grote angstogen.}{keek Jantje naar de dokter}{die kwam toegesneld}\\

\haiku{{\textquoteleft}I am from Holland{\textquoteright},, {\textquoteleft}.}{zei ikand I am here}{for the first time}\\

\haiku{Wat ik u vragen{\textquoteright},, {\textquoteleft}.}{wilde zei ikis hoe ik}{in Bokor kan komen}\\

\haiku{{\textquoteright} Als op afroep kwam.}{een donkere Boerdosser}{binnen wandelen}\\

\haiku{{\textquoteright} {\textquoteleft}Haha{\textquoteright}, zei ze, {\textquoteleft}dacht?}{u dat meneer Henrix dat}{prettig zou vinden}\\

\haiku{{\textquoteleft}Mandi\"en{\textquoteright}, riep ik,.}{verrast maar de bediende}{kende het woord niet}\\

\haiku{Ze hoopt natuurlijk,.}{dat Oeson niets zal doen}{zolang ze hier zit}\\

\haiku{Het  was nog vrij,.}{warm maar de hitte was nu}{toch wel te dragen}\\

\haiku{Hij was tenger en.}{had koolzwarte ogen in een}{zachtzinnig gezicht}\\

\haiku{Hij zag er sereen.}{uit maar gaf me nog een lel}{met zijn revolver}\\

\haiku{{\textquoteright} {\textquoteleft}Tegen de prijs die.}{wij voor de hele partij}{geboden hebben}\\

\haiku{Mani par ono{\textquoteright}, zei, {\textquoteleft}.}{ik conversationeel}{won tara codex}\\

\haiku{Het duurde een paar.}{minuten voordat ik weer}{een beetje bijkwam}\\

\haiku{Margreet Oeson,.}{keek me steeds vragend aan maar}{ik vermeed haar blik}\\

\haiku{{\textquoteleft}Nog een paar dagen,,{\textquoteright},:}{wachten jongen zei ze en}{tegen tante Aal}\\

\haiku{Aan het begin en.}{het eind van de brug kon je}{van alles kopen}\\

\haiku{Je kwam binnen en,.}{het was een groot huis met warm}{licht uit de lampen}\\

\haiku{Het was stil buiten.}{en gezellig binnen en}{al ver over bedtijd}\\

\haiku{Toen hij dood was, mocht.}{mijn moeder de poort uit om}{hem te begraven}\\

\haiku{Nu niet, Hans{\textquoteright}, zei ze,.}{nijdig toen ik tegen haar}{begon te zeuren}\\

\haiku{Nu rechts aanhouden{\textquoteright},, {\textquoteleft}}{zei hij zelfverzekerdals}{we goed luisteren}\\

\haiku{{\textquoteright} Maar na een half uur,.}{zag ik ze omhoog klimmen}{Els en haar zusje}\\

\haiku{Ik kreeg een keurig.}{briefje terug en om vier}{uur moet ik er zijn}\\

\haiku{{\textquoteright} Ook bij het eten was.}{hij nog zeer korzelig en}{hij ging vroeg naar bed}\\

\haiku{Gehoorzaam ging ik,.}{liggen rillend van de kou}{en viel prompt in slaap}\\

\haiku{{\textquoteright} {\textquoteleft}Gooi dat geweer op,{\textquoteright},, {\textquoteleft}.}{de grond Frans riep mijn oomen}{kom dan dichterbij}\\

\haiku{Vandaar dat Sjoukje,?}{zo ge{\"\i}nteresseerd was}{in je brieven h\'e}\\

\haiku{Vader richtte het.}{geweer op zijn hoofd en kwam}{langzaam dichterbij}\\

\subsection{Uit: Kleine stukjes om te lezen}

\haiku{Hij moest er een tijd.}{naar zoeken maar toen had hij}{het toch gevonden}\\

\haiku{Hij stapte op zijn.}{bromfiets en liet zich langzaam}{naar kantoor rijden}\\

\haiku{Even het gebit uit,.}{en de kaken op en neer}{blaasbalgje spelen}\\

\haiku{Pieter verlegen.}{met zijn pukkels in zijn hoofd}{naast zijn staketsels}\\

\haiku{Nou Pieter,{\textquoteright} zei ik, {\textquoteleft},.}{Gut ik wist niet dat je al}{zoveel gemaakt had}\\

\haiku{{\textquoteleft}Henk, het meisje was,.}{hier geen twee weken of ze}{kreeg er genoeg van}\\

\haiku{{\textquoteright} Mijn vader, achter,.}{het stuur ging ongemerkt wat}{harder rijden}\\

\haiku{Ik hoorde het uit,.}{de tweede hand misschien sterk}{gedramatiseerd}\\

\haiku{Behalve de twee.}{schildwachten was alleen de}{priester op het plein}\\

\haiku{Mevrouw Elsa Cruijff,.}{ze droeg de naam nog steeds een}{beetje onwennig}\\

\haiku{Ze liep resoluut,.}{op de stille figuur af}{knielde erbij neer}\\

\haiku{Ik was er nog niet,}{geweest deze kans om er}{gechaperonneerd}\\

\haiku{ik heb toch liever,.}{dat je weer snel terug gaat}{anders krijg je straf}\\

\haiku{Ik was blij met de.}{grote helm die mijn gezicht}{in het donker liet}\\

\haiku{Het was al drie uur.}{in de morgen voordat hij}{tot een besluit kwam}\\

\haiku{Ze stond hem op te,.}{wachten bovenaan de trap}{blond en iets te dik}\\

\haiku{Hij besloot er toch,.}{een te plaatsen heel terloops}{en vanzelfsprekend}\\

\haiku{Marijke dacht dat...{\textquoteright} {\textquoteleft},,{\textquoteright}.}{jeJa het is goed ik neem}{haar wel mee zei hij}\\

\haiku{Een mooie kamer{\textquoteright} zei.}{ze al terwijl hij nog zocht}{naar het lichtknopje}\\

\haiku{{\textquoteleft}Geachte Juffrouw,}{zoudt u mij het genoegen}{willen bereiden}\\

\haiku{Hij dronk een klein glas.}{sherry en keek met nieuwe}{ogen zijn kamer rond}\\

\haiku{De dood kun je je:}{niet voorstellen omdat het}{te absoluut is}\\

\haiku{s Middags als de.}{school uitging achtervolgden}{zij ons in groepjes}\\

\haiku{Ze zouden een nacht,.}{wegblijven ik had het rijk}{alleen in ons huis}\\

\haiku{Ik sliep al toen de.}{huizen wakker werden en}{hulp kwam toegesneld}\\

\haiku{Achter mij was het:}{onstuitbare geluid van}{Winnie te horen}\\

\haiku{{\textquoteleft}Ja, dat kunnen we,{\textquoteright}.}{vooraf wel doen zei ik vol}{zelfvertrouwen}\\

\haiku{{\textquoteleft}Laat maar, laat maar,{\textquoteright} zei, {\textquoteleft},.}{ikga jij de herdertjes}{maar verrassen man}\\

\haiku{Een bus brengt hem naar,.}{de buitenwijk van de stad}{het is er vrij druk}\\

\haiku{een beetje dom of.}{een beetje labiel of een}{beetje ongezond}\\

\haiku{De cirkel is dan.}{rond en iedereen heeft wat}{hij wilde hebben}\\

\haiku{Zo lijdt A dus zijn.}{tweede verlies en B raakt}{het ding nooit meer kwijt}\\

\subsection{Uit: Met stijgende verbazing}

\haiku{Op de vloer van de.}{Mercedes lagen een paar}{sigarenpeuken}\\

\haiku{{\textquoteleft}Ik heb vooral in,{\textquoteright}, {\textquoteleft}.}{het bos gewandeld zei ik}{dan word je niet bruin}\\

\haiku{En toen kon hij wat.}{geld lenen en bood aan om}{zich in te kopen}\\

\haiku{{\textquoteleft}Dus,{\textquoteright} zei de jongen, {\textquoteleft}.}{met het baardjehet gaat om}{een man of een vrouw}\\

\haiku{Hij denkt eerst nog aan,.}{een luchtspiegeling maar hij}{kruipt er toch naar toe}\\

\haiku{Hij veegde het zweet.}{van zijn voorhoofd en poetste}{zijn brilleglazen}\\

\haiku{{\textquoteright} Haastig pakte ik.}{mijn jas en struikelde de}{donkere trap af}\\

\haiku{Ik stak het stokje.}{zorgvuldig in mijn broekzak}{om te bewaren}\\

\haiku{Ze programmeerde, '.}{al onze tijd ens avonds}{waren we doodmoe}\\

\haiku{Na de receptie.}{en het etentje bracht ik Maartjes}{getuige naar huis}\\

\haiku{Ik wilde haar en,.}{Jimmy per se nog zien maar}{Bert hield me tegen}\\

\haiku{Morgen koop ik een,{\textquoteright}, {\textquoteleft}.}{croquet voor je beloofde}{ikvan Kwekkeboom}\\

\haiku{Een andere vrouw,.}{wilde interrumperen}{maar ze kreeg geen kans}\\

\haiku{En dan heb ik wat,.}{van  die blikjes in huis}{voor het geval d\'at}\\

\haiku{{\textquoteright} zei hij snel tegen, {\textquoteleft}.}{mijmarkt voor alleenstaande}{oudere vrouwen}\\

\haiku{Ik was weer geheel,,.}{nuchter de zak geledigd}{de druk verdwenen}\\

\haiku{Ik keek haar aan, ze.}{had een mooi schrander hoofd en}{een ironische lach}\\

\haiku{{\textquoteright} {\textquoteleft}Welnee, eigenlijk.}{informeerde hij alleen}{hoe het met Bert ging}\\

\haiku{Onaangename.}{herinneringen dreigden}{boven te komen}\\

\haiku{Hij kwam om elf uur,}{op kantoor knikte in het}{voorbijgaan naar me.}\\

\haiku{{\textquoteleft}Mijn god Bert,{\textquoteright} zei ik, {\textquoteleft}.}{toen Jenny even weg waswat}{ben je toch een lul}\\

\haiku{{\textquoteright} Hij betastte zijn,.}{zakken vond een sigaar en}{stak hem in zijn mond}\\

\haiku{{\textquoteright} {\textquoteleft}Niets, je wilde me.}{laten zien dat je alles}{op moet  schrijven}\\

\haiku{En heren, de bar.}{is open als u straks nog een}{glaasje wilt drinken}\\

\haiku{Seizoenstrends en trends.}{per weekdag worden in de}{doelstelling verwerkt}\\

\haiku{Allicht raadpleeg je,.}{dan de kaart en het water}{loopt je in de mond}\\

\haiku{Hij keek me even aan,.}{maar er viel aan zijn gezicht}{niets af te lezen}\\

\haiku{Hij zette de fles,.}{op tafel pakte zijn bril}{op en wreef zijn ogen}\\

\haiku{Ik was er zelf net,.}{uit ik wilde er niet in}{betrokken raken}\\

\haiku{{\textquoteright} vroeg ik, {\textquoteleft}wil je hem,?}{wegsturen net zoals Van}{Lier dat bij jou deed}\\

\haiku{{\textquoteright} Grote opluchting.}{brak los en ik vertrok met}{veel goede wensen}\\

\haiku{Maartje was nog geen twee.}{dagen weg en hij begon}{al te verslonzen}\\

\haiku{Hij slokte haastig,.}{een glas weg boerde en kwam}{langzaam bij kennis}\\

\haiku{{\textquoteleft}Ik moest die auto.}{dus aan de kant zetten en}{mee naar het bureau}\\

\haiku{Maar hij zag het h\`e,.}{dat ik dat nodig had en}{dat ik een heer was}\\

\haiku{Trut, mijn broer in de,.}{steek laten op zo'n moment}{een schande was het}\\

\haiku{{\textquoteleft}Ik heb haar een paar,.}{keer gebeld maar ik kom niet}{voorbij nicht Carla}\\

\haiku{{\textquoteleft}Hier,{\textquoteright} zei Theo en stak.}{hun een kurketrekker toe}{die hij op zak had}\\

\haiku{We wachtten tot Ruud.}{de fles geopend had en}{drie glazen inschonk}\\

\haiku{Bert ledigde een,.}{blik frisdrank in zijn glas zag}{ik met opluchting}\\

\haiku{ze mijn klanten, dat.}{zal ze met de helft lukken}{en de rest vertrekt}\\

\haiku{Ze stond snel op en,.}{nam haar tasje mee Govers kon}{het wel vergeten}\\

\haiku{{\textquoteright} vroeg ik en merkte.}{aan de beweging van haar}{hoofd dat ze knikte}\\

\subsection{Uit: Het tekort}

\haiku{Heren, we hebben,.}{al drie minuten gepraat}{nu eerst een plaatje}\\

\haiku{Na een kwartier was.}{hij gezakt tot 150 en ik}{gestegen tot 100}\\

\haiku{De een zie je nooit,.}{de ander komt nog wel eens}{een praatje maken}\\

\haiku{Die avond vrijden we,.}{als vanouds ongeduldig}{en met veel lawaai}\\

\haiku{We stapten de lift,.}{in waar we elkaar snel wat}{beter opnamen}\\

\haiku{Misschien had hij toch.}{wel gelijk dat er op mij}{te bouwen viel}\\

\haiku{{\textquoteleft}Ik ben je vader,.}{niet het is de eerste keer}{in al die jaren}\\

\haiku{{\textquoteleft}Waarom wil je niet?}{dat er in mijn bijzijn over}{Robin gepraat wordt}\\

\haiku{En toen ik er eind,.}{van de middag naar vroeg had}{Jaap het al verstuurd}\\

\haiku{Dat is verdomme.}{de tweede keer dat Dries een}{kloterapport krijgt}\\

\haiku{Van Maurik klaagt echt,.}{niet gauw en nu al twee keer}{in een paar weken}\\

\haiku{bent is iets van 1,.}{gulden 97 plus afschrijving}{fiets een rijksdaalder}\\

\haiku{Maar ik merkte dat.}{hij ineens verstrakte en}{liet de spriet vallen}\\

\haiku{{\textquoteright} {\textquoteleft}Maar denk eens aan de.}{invloed die dat heeft op het}{opgroeiende kind}\\

\haiku{En desnoods liever.}{een kwart vogel in de hand}{dan twee in de lucht}\\

\haiku{We deden de drie}{verplichte zoenen op de}{wang en ik rook vaag}\\

\haiku{Ik ben op zoek naar.}{mijn echtgenoot die weer eens}{geheel foetsie is}\\

\haiku{Dan zal ik bij jou,?}{moeten logeren heb je}{een logeerkamer}\\

\haiku{Ik vond haar lief en.}{ze had recht op warmte na}{zo'n mislukte avond}\\

\haiku{{\textquoteright} {\textquoteleft}Mwah,{\textquoteright} zei Robin met, {\textquoteleft}.}{getuite lippenmij nooit}{zo opgevallen}\\

\haiku{{\textquoteright} {\textquoteleft}Suit yourself,{\textquoteright} zei ze en.}{begon met smaak te eten van}{haar babi panggang}\\

\haiku{Ze vond het contact.}{dat we nu hadden juist heel}{prettig en rustig}\\

\haiku{{\textquoteright} Ze lachte naar me.}{om te laten zien dat het}{maar half gemeend was}\\

\haiku{{\textquoteright} {\textquoteleft}Meneer De Leeuw, ik.}{zeg altijd maar je bent zo}{oud als je je voelt}\\

\haiku{Triomfantelijk.}{legde hij ze \'e\'en voor \'e\'en}{op het tafeltje}\\

\haiku{{\textquoteright} Nu even doorbijten,?}{had ik het in me om hem}{erin te luizen}\\

\haiku{{\textquoteleft}Verdomme, de pils,.}{is op Marjolein is ook}{niet meer wat ze was}\\

\haiku{{\textquoteright} {\textquoteleft}Ja man, je kan het.}{niet altijd krijgen zoals}{je het hebben wilt}\\

\haiku{{\textquoteleft}Ik zei al tegen,,{\textquoteright}.}{je dat het niks was John}{zei de directeur}\\

\haiku{En als je zoveel,.}{van Robin houdt moet je hem}{nog een kans geven}\\

\haiku{Onwillekeurig.}{keek ik in de spiegel en}{schrok van wat ik zag}\\

\haiku{{\textquoteleft}Marjolein en ik.}{hebben een probleem en daar}{speel jij een rol bij}\\

\haiku{En zodra jij zelf.}{even aan het kortste eind trekt}{is het huis te klein}\\

\haiku{iets was hem in geen.}{jaren verteld en hij had}{het er moeilijk mee}\\

\subsection{Uit: Vanonder de koperen ploert}

\haiku{in Indonesi\"e.}{zijn prijzen een favoriet}{gespreksonderwerp}\\

\haiku{Hij kijkt lang naar het,.}{geld en mijn boze hoofd en}{ik kijk lang terug}\\

\haiku{Even later komt een,}{krantenjongen binnen door}{Engelse ziekte}\\

\haiku{Netjes geef ik haar {\textquoteleft},{\textquoteright}.}{een hand.Nama saja Hans}{Vervoort zeg ik trots}\\

\haiku{Ostentatief laat.}{ik de tas met camera}{op tafel liggen}\\

\haiku{Ziet u, de cheque.}{uit Holland komt maar \'e\'ens}{in de drie maanden}\\

\haiku{Hij knikt en dan lijkt.}{me het moment gekomen}{om te vertrekken}\\

\haiku{Ons geroep helpt niets,.}{ze blijven in een stevig}{tempo doorsjouwen}\\

\haiku{Na een week zingt hij {\textquoteleft}{\textquoteright}.}{geheel zelfstandigHappy}{birthday to joe}\\

\haiku{Vanzelf ga ik ook}{naar de mensen kijken en}{onder een boom zit}\\

\haiku{{\textquoteleft}Bij een boom zo vol,.}{geladen geven \'e\'en twee}{pruimpjes extra niet}\\

\haiku{Een nummer weet ik,.}{niet maar het was een hoekhuis}{volgens mijn ouders}\\

\haiku{Een wit kerkje, veel,,.}{stenen huizen een schooltje}{een stenen passar}\\

\haiku{De trein zal om \'e\'en,.}{uur vertrekken maar liever}{te vroeg dan te laat}\\

\haiku{Uit de stortbak groeit,.}{een schaduwplant dat is dan}{weer typisch Indisch}\\

\haiku{We blijven rustig,.}{wachten en na enige tijd}{komen ze terug}\\

\haiku{Een oude kelner.}{met \'e\'en tand komt ons vragen}{wat we willen eten}\\

\haiku{een straat die vroeger,.}{de Palmenlaan heette als}{ik me niet vergis}\\

\haiku{Ik ga het straatje.}{in waar ik in mei 1953 voor}{het laatst ben geweest}\\

\haiku{Hier kom ik nooit meer,.}{terug dacht ik heel bewust}{toen ik wegfietste}\\

\haiku{Ik zocht Lenta op,.}{samen liepen we naar de}{rivier voor de school}\\

\haiku{Ik knikte, keek de.}{grote zaal rond en nam in}{gedachten afscheid}\\

\haiku{Fietsers, betja's, auto's.}{rijden in brede stromen}{tegen elkaar in}\\

\haiku{Aan het eind van de.}{rit treffen we een poortje}{met een bewaker}\\

\haiku{Hij flikt het glaasje,.}{leeg in zijn keelgat schenkt zich}{meteen opnieuw in}\\

\haiku{Direct na onze:}{aankomst is een paniek van}{gastvrijheid ontstaan}\\

\haiku{De gulheid van de.}{familie zal zich wel niet}{over hem uitstrekken}\\

\haiku{Ze liggen keurig.}{gerangschikt in een richel}{boven de bouillon}\\

\haiku{Onverwacht wordt het.}{de mooiste rit die we op}{Java meemaken}\\

\haiku{waarom filmhelden}{er altijd zo authentiek}{zweterig en vies}\\

\haiku{Nu begrijpt hij het,.}{beter een newspaperman}{van voor de oorlog}\\

\haiku{Naast de chauffeur zit,.}{de eigenaar wiens rol ons}{niet duidelijk wordt}\\

\haiku{Zodra er twee bij.}{elkaar in de buurt komen}{ontstaat het gevecht}\\

\haiku{Even de paspoorten,.}{oppikken en dan met het}{vliegtuig naar Bali}\\

\haiku{Eerst moeten we even,.}{langs het huis van Mama voor}{een afscheidshapje}\\

\haiku{Mama Pungut barst.}{in tranen uit en laat ons}{maar met moeite gaan}\\

\haiku{De prijs is laag, daar.}{zal onze begeleider}{niet rijk van worden}\\

\haiku{Al die tijd heeft hij,.}{niet gelachen alleen zijn}{tanden ontbloot}\\

\haiku{Na de reis zal hij.}{ze terugkrijgen wordt hem}{vriendelijk beloofd}\\

\haiku{Jakarta is de.}{plek waar Maja's voorouders}{begraven liggen}\\

\haiku{Tijdens het gesprek:}{herinnert hij zich iets en}{zegt tegen zijn vrouw}\\

\subsection{Uit: Een zomer apart}

\haiku{Ik plaatste het bord op.}{zijn schoot en de koffie voor}{hem op het muurtje}\\

\haiku{We keken even naar.}{de stralende hemel maar}{er kwam geen teken}\\

\haiku{Als je moeder het.}{maar af en toe mag lenen}{voor haar boekhouding}\\

\haiku{Het hinderde me,?}{dat ik niets kon dat moest toch}{een keer uitkomen}\\

\haiku{Angela was in.}{de keuken bezig met het}{snijden van groente}\\

\haiku{O, dat kostte me,.}{wat want ik was bang hoe hij}{zou reageren}\\

\haiku{Na een tijd ging het,.}{wat beter de afstand naar}{mijn bed leek haalbaar}\\

\haiku{Het duurde een half.}{uur voordat er geluiden}{op de trap klonken}\\

\haiku{In de koele hal,.}{hing een zwaard ik kon nu geen}{uitstel meer velen}\\

\haiku{Ik hou hem maar op,.}{dan kun je er niet nog een}{keer op gaan zitten}\\

\haiku{z'n tol eiste en.}{dat hij er goed aan zou doen}{een vrouw te vinden}\\

\haiku{Ik verander er,,.}{nog aan ik zal haar Hetty}{noemen of zo iets}\\

\haiku{Ik bracht haar terug.}{naar de burcht en liet haar bij}{de ingang alleen}\\

\haiku{{\textquoteright} En met z'n drie\"en.}{wuifden ze me na toen ik}{in de taxi verdween}\\

\haiku{Het was koud en ik.}{rilde en verloor ineens}{mijn zelfbeheersing}\\

\haiku{Een bord verwees naar,.}{Engeland er stond een boot}{met een rookpluim op}\\

\haiku{{\textquoteright} {\textquoteleft}Jawel, ze knikt ja.}{of nee en ze glimlacht en}{ze maakt gebaren}\\

\haiku{Ik stond op en ging.}{buiten in de schaduw op}{een bankje zitten}\\

\haiku{Ze had mijn boeken.}{gelezen en ze veegde}{de vloer met me aan}\\

\haiku{Hij hield zijn glas naar.}{achteren en ik schonk hem}{in en wachtte af}\\

\haiku{{\textquoteleft}Whisky,{\textquoteright} zei hij en.}{zakte neer op een stoel aan}{de keukentafel}\\

\haiku{En ik heb iemand.}{nodig die kan rekenen}{en zaken kan doen}\\

\haiku{Maar dan had zij mij,?}{toch kunnen zien en wenken}{me kunnen volgen}\\

\haiku{Wat had die toerist?}{gezegd of gedaan om hem}{zo kwaad te maken}\\

\haiku{Ik deed mijn ogen dicht.}{en hoopte maar dat men zou}{denken dat ik bad}\\

\haiku{Maar al lezend vroeg.}{ik me af wat er erg was}{aan onoprechtheid}\\

\haiku{{\textquoteright} {\textquoteleft}Is dat je nieuwe,,?}{vrouw maar dat is toch niet een}{nieuwe moeder h\`e}\\

\haiku{{\textquoteright} {\textquoteleft}Wat is er nou aan,?}{de tuin wat moeten we nou}{met haar in de tuin}\\

\haiku{Er mocht wel bij maar,.}{er mocht niets af niemand en}{niets mocht weg of dood}\\

\haiku{Een goeie huishoudster,,}{heb ik in elk geval dacht}{ik en ze ving het}\\

\subsection{Uit: Zonder dollen}

\haiku{Hij aarzelde even {\textquoteleft}?}{en drukte me zijn glas in}{de hand.Blijf je hier}\\

\haiku{Als de nood aan de,.}{man kwam was de kalmte in}{elk geval nabij}\\

\haiku{Het kwam gewoon uit,.}{de lucht vallen mijn vader}{praatte daar nooit over}\\

\haiku{Lucas leidde me.}{naar  een gezelschapje}{van een man of tien}\\

\haiku{We wachtten tot het.}{licht groen werd en staken de}{doodstille weg over}\\

\haiku{{\textquoteleft}Is hier geen normaal,?}{caf\'e waar je rustig een}{pilsje kunt drinken}\\

\haiku{{\textquoteright} vroeg ik en tilde,.}{haar even op waarbij ik haar}{middel indrukte}\\

\haiku{Eefje had haar ogen,.}{dicht kleine blauwe adertjes}{op de oogschelpen}\\

\haiku{{\textquoteright} Zes jaar later, in,.}{Budapest zat het nog exact}{in mijn geheugen}\\

\haiku{{\textquoteright} {\textquoteleft}Nog zo'n geinige.}{opmerking en ik gooi me}{van het balkon af}\\

\haiku{Ik lachte hem uit.}{en tenslotte begon hij}{ook te grinniken}\\

\haiku{De poes was nergens.}{te vinden en ik voelde}{me zo erg alleen}\\

\haiku{Dat geroezemoes,.}{van al die stemmen daar werd}{ik knettergek van}\\

\haiku{Als ik weer OK ben,.}{kom ik naar je toe want Jack}{is maar tijdelijk}\\

\haiku{Het stomme is, toen.}{ik wegging dacht ik dat het}{erg goed met haar ging}\\

\haiku{Of misschien was ze,.}{wel gewoon bang voor de reis}{god mag het weten}\\

\haiku{Mevrouw Overeem vertrok.}{om een uur of drie en toen}{waren ze er nog}\\

\haiku{De hoofdpijn was weg,.}{maar ik voelde me nog steeds}{moe en bezopen}\\

\haiku{E\'en druppel op het.}{tapijt en de huisvrouw kan}{het wel vergeten}\\

\haiku{Ik ging verstrooid op.}{het krukje zitten en keek}{naar haar lange lijf}\\

\haiku{Haar kleine borsten.}{wipten telkens boven de}{waterspiegel uit}\\

\haiku{Celia schokte nog,.}{een paar keer door terwijl ik}{wat nadruppelde}\\

\haiku{De roomservice.}{bracht wat toast en sloot de deur}{discreet achter zich}\\

\haiku{Ik had hem later,.}{op de avond gepland maar hoe}{eerder hoe beter}\\

\haiku{het is gewoon waar,.}{ik ben niet zo'n genieter}{als Lucas of jij}\\

\haiku{{\textquoteright} {\textquoteleft}O, was jij dat{\textquoteright}, zei, {\textquoteleft},.}{ikja dat is het leuke}{van een erfenis}\\

\haiku{{\textquoteright} {\textquoteleft}Jullie waren niet?}{meer thuis toen Eefje uit het}{ziekenhuis langskwam}\\

\haiku{{\textquoteright} {\textquoteleft}Misverstanden{\textquoteright}, zei, {\textquoteleft},.}{ik schamperjongen ik ken}{je van binnenuit}\\

\haiku{Die pikken het niet.}{als zo'n nieuwkomer ineens}{de grote baas is}\\

\haiku{Hij was toch alweer.}{heel ver heen en dat viel me}{een beetje tegen}\\

\haiku{Dat was kennelijk.}{een standaard-gimmick}{hier in Hongarije}\\

\haiku{Maar daar tegenover.}{staat dat ik nogal zwaar op}{hem ben gaan zitten}\\

\haiku{{\textquoteleft}Lekker wandelweer{\textquoteright},, {\textquoteleft}.}{zei ikals het nou maar niet}{gaat  regenen}\\

\haiku{Ik maakte een klein.}{trippelpasje en knalde}{tegen een deur aan}\\

\subsection{Uit: Zonnige perioden}

\haiku{Terwijl ik de rest.}{van de kantoorpost doornam}{ging de telefoon}\\

\haiku{Poe haar bovenlijf.}{tegen mijn been en liet haar}{hongerklaag horen}\\

\haiku{Als ik niet zo stuurs,.}{was en zo weinig tijd had}{zou ik het zelf doen}\\

\haiku{Gelukkig kreeg ik.}{na de hbs een goede baan}{bij de Postgiro}\\

\haiku{En deze week kreeg,.}{ik een kaart uit Florida}{St. Augustine}\\

\haiku{{\textquoteleft}U bent een oude.}{schoolvriend en u weet dus dat}{Peter nooit veel zei}\\

\haiku{Het is zacht gezegd.}{niet leuk om hier te zitten}{en niets te weten}\\

\haiku{{\textquoteleft}Ik begreep uit zijn.}{kaartje dat hij op zoek is naar}{oude bekenden}\\

\haiku{Lilian zat voor,.}{in de klas en ik een paar}{banken achter haar}\\

\haiku{Dat had hij me nooit:}{vergeven en elke keer}{kwam het weer terug}\\

\haiku{{\textquoteleft}Ik zit hier nu met.}{een probleem van jou waar ik}{echt geen trek in heb}\\

\haiku{Daar zat een vrouw te,.}{slapen hoofd op de armen}{op het tafelblad}\\

\haiku{Ja, je zou misschien.}{wat minder tijd aan je werk}{kunnen besteden}\\

\haiku{Hier nooit klevers, nooit,.}{werd je gesneden men gaf}{elkaar de ruimte}\\

\haiku{Ook als ik hier niet.}{met een doel gekomen was}{zou ik er graag zijn}\\

\haiku{I always thought, ik.}{dacht altijd wel dat je iets}{met schrijven zou doen}\\

\haiku{In het hoogseizoen.}{verhuur ik het en woon ik}{zelf in het motel}\\

\haiku{Maar hij moet wel een.}{week in het ziekenhuis met}{een hersenschudding}\\

\haiku{{\textquoteright} {\textquoteleft}Ik garandeer het,.}{voor een week daarna sta ik}{niet voor mezelf in}\\

\haiku{Het stond laatst nog in,.}{Management Team papier}{en inkt zijn killers}\\

\haiku{{\textquoteright} Hij produceerde,.}{van onder de deken een}{hand die ik schudde}\\

\haiku{Een half uur later.}{kreeg ik belet in zijn huis}{aan het Vondelpark}\\

\haiku{Daar lag Flamingo,.}{Resort een oase van rust}{met uitzicht op zee}\\

\haiku{Je stuurde me een.}{ansichtkaart dat je op zoek}{bent naar Lilian}\\

\haiku{Ik denk veel te veel.}{aan jou en aan vroeger en}{hoe het verder moet}\\

\haiku{Ze omarmde me.}{warm en al mijn twijfels en}{angsten verdwenen}\\

\haiku{Een onthutsende,.}{ervaring die mij altijd}{bijgebleven was}\\

\haiku{{\textquoteleft}Ik heb wel gemerkt.}{dat Lilian en jij iets}{met elkaar hebben}\\

\haiku{Maar hij had nog steeds.}{het talent om anderen}{te ontregelen}\\

\haiku{{\textquoteright} Hij zweeg, zoog aan zijn.}{sigaret en begon weer}{ernstig te hoesten}\\

\haiku{{\textquoteright} {\textquoteleft}Mevrouw Van Eerden,{\textquoteright}, {\textquoteleft}.}{zei Mackaywas een echte}{Hollandse mevrouw}\\

\haiku{Dus dit fotootje hoef,{\textquoteright}.}{ik niet in te lijsten zei}{Peter ten slotte}\\

\haiku{Hoor eens, ik heb een.}{jobje voor een paar weken}{in het buitenland}\\

\haiku{Als honderd kikkers.}{op je zeilplank springen wil}{je wel wegwezen}\\

\haiku{Wil je haar zeggen,?}{dat het me spijt van die keer}{in de magnetron}\\

\haiku{Maar een goed einde.}{voor Peter en Christine}{hoorde er ook bij}\\

\haiku{De grond veerde van.}{het bladerafval van de}{afgelopen eeuw}\\

\haiku{Het was ook dom van.}{mij om te proberen het}{te repareren}\\

\haiku{{\textquoteright} Clich\'e-time,.}{wisten mijn hersens en ik}{begon te zwoegen}\\

\haiku{Vader aan zoon die:}{na auto-ongeluk in}{het ziekenhuis ligt}\\

\haiku{{\textquoteright} De mangrove liet.}{ons gaan en we kanoden}{in stilte verder}\\

\haiku{En wij willen je.}{hier niet terug zien komen}{met een lang gezicht}\\

\haiku{Zo'n jonge vrouw met,.}{zo'n ouwe zakenlul dat}{kan niet  duren}\\

\haiku{Ik was ten einde.}{raad en ik legde hem de}{situatie uit}\\

\haiku{Ze schrok niet, en stak.}{haar hand uit om me op de}{kant te trekken}\\

\subsection{Uit: Zwarte rijst}

\haiku{Financi\"en Jan,{\textquoteright},.}{zei ik scherp hij moest nu niet}{opnieuw gaan huilen}\\

\haiku{Baboe zouden we,,.}{vroeger zeggen maar dat mocht}{niet meer dat wist ik}\\

\haiku{{\textquoteright} Victor, Yvonne en.}{Eric lieten zich schaapachtig}{grijnzend voorstellen}\\

\haiku{Nee oom, we hebben,.}{geen bagage bij ons die}{staat in het hotel}\\

\haiku{Als je tenminste.}{genoegen wilt nemen met}{eenvoudige kost}\\

\haiku{We keken elkaar,.}{aan het was nog niet in ons}{hoofd opgekomen}\\

\haiku{Ik was er al bang,.}{voor maar het onderwerp was}{niet te vermijden}\\

\haiku{Ik ben in Birma.}{terechtgekomen en zij}{in Ambarawa}\\

\haiku{{\textquoteleft}Waarom bent u niet,?}{in Holland gebleven na}{uw pensioen oom}\\

\haiku{Er is hier nog een,,.}{Hollander Carl Swerts die komt}{regelmatig langs}\\

\haiku{Er viel een hoop te.}{leren over anatomie in}{het huis van Hanny}\\

\haiku{Een doorgangskamp in.}{Batavia waar ik voor het}{eerst ijskoffie dronk}\\

\haiku{Hoor eens, je moeder,.}{was een heel mooi meisje maar}{een beetje verwend}\\

\haiku{We dronken ons glas,.}{leeg en gingen weg na een}{hartelijk afscheid}\\

\haiku{{\textquoteleft}Ja,{\textquoteright} zei Swerts, maar hij.}{liet zich niet afleiden en}{bleef me aankijken}\\

\haiku{Zijn bediende was,.}{een oud-leerling om maar}{iemand te noemen}\\

\haiku{{\textquoteright} riep ik, {\textquoteleft}Jimmie, ga?}{nu slapen of wil je een}{tik voor je billen}\\

\haiku{De groep dunde uit.}{en uiteindelijk kwam er}{niets meer naar boven}\\

\haiku{{\textquoteright} Ik aarzelde, het.}{vergde een oversteek waar ik}{weinig lust in had}\\

\haiku{{\textquoteright} {\textquoteleft}Ach shit,{\textquoteright} zei Victor, {\textquoteleft}?}{wat jullie vroeger deden}{kunnen wij toch ook}\\

\haiku{Hij stapte op de.}{boomstam en begon naar de}{overkant te lopen}\\

\haiku{{\textquoteright} Hij sloeg met zijn vuist,.}{op het kleine tafeltje}{de glaasjes dansten}\\

\haiku{En dan ben ik zo}{stom om hem uit die sloot te}{halen en verzwik}\\

\haiku{Ik trok haar nog wat.}{dichter naar me toe en ze}{zuchtte behaaglijk}\\

\haiku{{\textquoteright} Hij keek de kring langs.}{en zijn blik bleef rusten op}{de vier getrouwen}\\

\haiku{De revolutie,,.}{vindt op dit moment plaats in}{de stad op het land}\\

\haiku{En daarna keren.}{we terug naar het goede}{leven van vroeger}\\

\haiku{Een politieman.}{stond wijdbeens boven hem en}{richte zijn pistool}\\

\haiku{Al die jaren dat,.}{we gewacht hebben tot het}{zou overgaan huhu}\\

\haiku{Daar moest ik mezelf.}{maar aan vastklampen en niet}{verder nadenken}\\

\haiku{Uit de donkere.}{middentuin steeg een chronisch}{en scherp gonzen op}\\

\section{Simon Vestdijk}

\subsection{Uit: Verzamelde romans. Deel 9. Aktaion onder de sterren}

\haiku{Hyperenoor, de,;}{geweldige bleef zo lang}{mogelijk moorden}\\

\haiku{{\textquoteright} Op zo scherpe toon,.}{had hij gesproken dat de}{jongen vuurrood werd}\\

\haiku{{\textquoteright} {\textquoteleft}Wou je een derde?}{reden hebben om me weg}{te laten jagen}\\

\haiku{En negen maanden.}{later werd Orion door de}{aarde geboren}\\

\haiku{Veeteelt en landbouw;}{waren innig verstrengeld}{in deze lezing}\\

\haiku{Van deze vrouwen.}{was de machtige moeder}{op de burcht er \'een}\\

\haiku{De bezoeken aan.}{de hut had hij gestaakt uit}{een dof schuldbesef}\\

\haiku{Zij zal je niet meer.}{erkennen als waardig om}{haar dienst te leiden}\\

\haiku{{\textquoteleft}Was er, o zoon van,,?}{Apollo een antwoord bij dat}{uw goedkeuring heeft}\\

\haiku{Treedt de Kentauros ',!}{int licht want golven zijn}{dansende paar den}\\

\haiku{{\textquoteleft}Denkt gij, dat onze?}{mensen deze verklaring}{zullen aanvaarden}\\

\haiku{Maar ik ben moeilijk,}{van de gedachte af te}{brengen dat het t\'och}\\

\haiku{mij getrouwd te zien,!}{met Timandra de kroon op}{uw opvoederswerk}\\

\haiku{laag, lemig en met.}{grote gaten die wellicht}{poorten voorstelden}\\

\haiku{Waarom Aktaion?}{zelf niet gekomen was om}{haar dit te zeggen}\\

\haiku{Een ogenblik stond zij.}{verwezen te kijken bij}{deze ontdekking}\\

\haiku{Cheiron beefde over.}{zijn gehele lichaam en}{kon niet verder gaan}\\

\haiku{Cheiron nodigde,:}{hem opnieuw uit binnen te}{treden maar hij riep}\\

\haiku{{\textquoteleft}Neen, vader Cheiron,,.}{wat ik te zeggen heb is}{vlug genoeg gezegd}\\

\haiku{Met deze zelfde.}{Hermesianax sprak ik}{veel over de goden}\\

\haiku{{\textquoteleft}Ik wil niet de boom, -, -?}{in ik wil niet waarom moet}{ik nu de boom in}\\

\haiku{{\textquoteright} vroeg Aktaion, de.}{blik onafgebroken op}{de klimmer gericht}\\

\haiku{Zodra zij hem naar,.}{Artemis trachtte heen te}{dringen week hij uit}\\

\haiku{Bedenkt, dat jonge;}{mensen de wereld anders}{zien dan wij grijsaards}\\

\haiku{Aktaion keerde.}{een eigenaardig pruilend}{gezicht naar hen toe}\\

\haiku{de afbeelding van,.}{een man die zijn gezicht met}{krijt had ingesmeerd}\\

\haiku{{\textquoteright} {\textquoteleft}Tot de zon en de.}{maan zich verenigen om}{je te verblinden}\\

\haiku{Voor iemand volleerd.}{als jij is dit geen plaats om}{binnen te treden}\\

\haiku{Maar Karion kan,.}{slecht oordelen omdat hij}{zelf achterlijk is}\\

\haiku{En weer keek hij om,,.}{zich heen langzaam geen spleet in}{de rotsen overslaand}\\

\subsection{Uit: Verzamelde romans. Deel 19. Bevrijdingsfeest}

\haiku{men was als mens, en.}{hoe groot de waarheid waarvoor}{men had te strijden}\\

\haiku{- {\textquoteleft}Ik hoop niet, dat hij.}{op deze terreinen de}{beest gaat uithangen}\\

\haiku{Ook toen zij voor de,.}{afgravingen stonden sprak}{hij niet dadelijk}\\

\haiku{Je kunt het alleen,.}{goedmaken als je naar dat}{oude mens toegaat}\\

\haiku{{\textquoteright} {\textquoteleft}Daar schiet je wat mee...{\textquoteright} {\textquoteleft}?}{opWas die boswachter met}{je meegelopen}\\

\haiku{Hij was zeer stipt in,;}{alles en aan zijn aandacht}{ontsnapte weinig}\\

\haiku{ik heb er heel wat,...}{gekend al wil ik daar nu}{niet over uitweiden}\\

\haiku{Op zichzelf prachtig,.}{want de regering  zal}{daar nooit voor zorgen}\\

\haiku{{\textquoteright} {\textquoteleft}Dat komt omdat er.}{geen spoor van hysterie in}{deze muziek is}\\

\haiku{Ten slotte liep zij.}{naar hem toe om zijn mond te}{sluiten met een kus}\\

\haiku{En een psychische,,;}{behandeling nou ja wat}{betekent dat nou}\\

\haiku{{\textquoteright} {\textquoteleft}Maar meneer Hoeck, is?}{er dan nooit eens iemand naar}{hem komen kijken}\\

\haiku{Een neef van hem heeft,;}{me geschreven of hij niet}{terug kan komen}\\

\haiku{wel de moeite waard.}{eens te bezoeken voor een}{niet-Amsterdammer}\\

\haiku{{\textquoteleft}Het is hier alle,{\textquoteright}.}{dagen keet zei Lucy op}{hartelijke toon}\\

\haiku{zij liet zich kalmweg,.}{lasseren verwijderde}{toen lachend de haak}\\

\haiku{{\textquoteleft}Ik ben eigenlijk,;}{dichter maar van gedichten}{kun je niet leven}\\

\haiku{als hij vanavond bij,.}{tante Gien was geweest had}{u hem kunnen zien}\\

\haiku{, en dan was ik juist,.}{g\'een kerel want kerels gaan}{tegen de wind in}\\

\haiku{{\textquoteright} schaterlachte hij, {\textquoteleft},!}{maar je hebt me uitstekend}{geholpen prachtig}\\

\haiku{Het leek me meer iets,;}{abnormaals zoals mannen}{wel vaker hebben}\\

\haiku{{\textquoteleft}Ik zal je een zoen,!}{geven omdat je me zo}{goed geholpen hebt}\\

\haiku{{\textquoteright} Niet te snel drong Evert.}{de plantaardige berg van}{duisternis binnen}\\

\haiku{Bij hem evenwel had.}{dit gebaar niets herderlijks}{of patriarchaals}\\

\haiku{maar hij bleef, zoals,.}{hij zich uitdrukte in elk}{geval fatsoenlijk}\\

\haiku{Door alles heen het;}{geduldige getinkel}{op de piano}\\

\haiku{De hele stad stonk,.}{een beetje maar dat hielp weer}{tegen de honger}\\

\haiku{{\textquoteright} - Petit keek naar zijn,,.}{schoenen hief het hoofd weer op}{en keek langs Evert heen}\\

\haiku{de zwarthandelaars.}{leken zelfs uitgesproken}{melancholici}\\

\haiku{Die arme tante,{\textquoteright}, {\textquoteleft}.}{Gien zei Maymaar ze zal wel}{iets anders vinden}\\

\haiku{Tijdens zijn gesprek;}{met Drost was hij dicht bij een}{duizeling geweest}\\

\haiku{{\textquoteright} Licht wenkbrauwfronsend.}{schudde zij het hoofd en zat}{nog te luisteren}\\

\haiku{Zijn kleren had hij,.}{aangelaten hij moest nu}{toch kunnen opstaan}\\

\haiku{Ik kan hem zeker,...{\textquoteright} {\textquoteleft}?}{wel even meenemen als ik}{weggaZal ik even}\\

\haiku{Belachelijk om.}{je op te winden over een}{dergelijke zaak}\\

\haiku{{\textquoteright} {\textquoteleft}Vernietigen,{\textquoteright} zei, {\textquoteleft}.}{Markman op peinzende toon}{dat is een heel ding}\\

\haiku{{\textquoteright} vroeg Markman, terwijl.}{hij ietwat treuzelend de}{theekop van Evert overnam}\\

\haiku{Bijna roerloos hing.}{het gele gebladerte}{van de grachtbomen}\\

\haiku{Die ochtend achter,.}{het stuur schudde hij zich als}{een morsige hond}\\

\haiku{De oude man kon.}{tegenover Rie Bentz alleen}{maar gesnoefd hebben}\\

\haiku{zij waren al op,;}{weg naar elders zij zweefden}{reeds een klein weinig}\\

\haiku{Men wentelde een,,.}{steen daaronder lag zand een}{halve meter dik}\\

\haiku{Maar nooit beknorde,,.}{zij ze liet dit aan Evert over}{die het evenmin deed}\\

\haiku{maar stemmen uit het.}{dorp waren natuurlijk veel}{aannemelijker}\\

\haiku{We zullen spelen,,.}{dat ik jullie doodschiet net}{als de rotmoffen}\\

\haiku{Ook hij lette op,.}{de kinderen het was zijn}{plicht dit te doen}\\

\haiku{En de dokter, dit?}{levend geweten in zijn}{steenrood omhulsel}\\

\haiku{Ze gunt hem mij dus,,.}{n{\'\i}et dacht Jeanne en toch}{is ze niet jaloers}\\

\haiku{Maar ik zou jou niet,.}{eens kunnen haten wanneer}{Evert met je wegliep}\\

\haiku{Maar een krachtproef zou,,.}{het in elk geval blijven}{morgen overmorgen}\\

\haiku{En oma zal het ook,.}{wel gehoord hebben en tobt}{er nu misschien over}\\

\haiku{Evert klopte hem op,;}{de schouder het werd nu een}{echt mannengesprek}\\

\haiku{Zij geloofde nu,;}{wel dat alles toch goed was}{tussen hen beiden}\\

\haiku{Ook de latere,,.}{Okke een grote grijze}{was ongesneden}\\

\haiku{De kater keek hem.}{ondoorgrondelijk aan en}{begon te spinnen}\\

\haiku{Ryers en Els moesten.}{geen vijf minuten na hem}{naar bed zijn gegaan}\\

\haiku{hij zag het rose;}{lampje en de lichtglans op}{zijn kale schedel}\\

\haiku{Het lange haar gaf;}{iets vrouwelijks aan zijn toch}{niet verwijfd gezicht}\\

\haiku{Anna, voor wie zijn,.}{ziel zo doorzichtig was zou}{het hebben geloofd}\\

\haiku{Het laatste wrakstuk.}{van een uiteengeslagen}{wereldbeschouwing}\\

\haiku{zij moest nu vooral,.}{niet menen dat hij in zijn}{wiek geschoten was}\\

\haiku{Nog niet eens zijn ziel,,;}{of zijn geest want die heeft hij}{misschien al niet meer}\\

\haiku{Iemand die \'een vrouw,;}{heeft komt er niet licht toe een}{tweede te nemen}\\

\haiku{Daar het dienstmeisje,:}{goed getraind was was er maar}{\'een mogelijkheid}\\

\haiku{Het was te groot voor,.}{een receptenpapiertje}{te klein voor een brief}\\

\haiku{{\textquoteleft}Nou zullen ze mij!}{godverdomme ook nog voor}{een flikker houden}\\

\haiku{Dat Evert hem ooit nog,.}{geld zou geven geloofde}{hij allang niet meer}\\

\haiku{En niet ver van hem,.}{zat Louis Drost met vier jonge}{mannen te praten}\\

\haiku{{\textquoteleft}Je zei toen iets over.}{wat Markman in de oorlog}{uitge1haald zou hebben}\\

\haiku{Zijn moeder vond hij.}{lijkbleek en moeilijk ademend}{op haar slaapkamer}\\

\haiku{hoe hij dit effect,.}{bereiken kon wanneer hij}{Markman's naam verzweeg}\\

\haiku{- {\textquoteleft}Het maakt op mij meer.}{de indruk van chantage}{dan van amusement}\\

\haiku{hij bracht de vinger.}{aan de lippen en wees op}{de suitedeuren}\\

\haiku{Maar het wordt anders.}{wanneer ze hem onder zijn}{armen gaan kijken}\\

\haiku{mompelen, dat veel:}{weghad van een herhaling}{dierzelfde woorden}\\

\haiku{Bijzonderheden,}{weet ik niet tenminste niet}{uit de eerste hand.}\\

\haiku{Er lag ongeveinsd,:}{medegevoel in zijn stem}{toen hij zachtjes zei}\\

\haiku{Toen hij klaar was, vroeg,:}{Mahrholtz op een toon die}{hem deed ophoren}\\

\haiku{daarmee heb je een...{\textquoteright} {\textquoteleft}...}{zeer groot gedeelte van je}{schuld afbetaaldSchuld}\\

\haiku{Schrijf mij een brief, als,.}{je aangekomen bent ik}{moet je adres weten}\\

\haiku{Mahrholtz had hem,;}{veel te verwijten en veel}{aan hem te danken}\\

\haiku{Rundstedt und Rommel, -,...}{haben gesagt ja was haben}{die nicht alles gesagt}\\

\haiku{Hij kroop erheen, en:}{onderzocht het wapen met}{bevende vingers}\\

\haiku{Maar Mahrholtz zag,,;}{h\'em zijn heer en meester en}{verleider heel goed}\\

\haiku{Maar ik zou toch even.}{tijd gevonden hebben bij}{je aan te komen}\\

\haiku{{\textquoteleft}Het is ellendig,{\textquoteright}, {\textquoteleft}.}{zei hij moeilijkik weet niet}{wat ik zeggen moet}\\

\haiku{{\textquoteright} - Zonder haar aan te,,.}{kijken zag hij dat zij de}{schouders ophaalde}\\

\haiku{En vertel het in,,,!}{Godsnaam niet aan Anna Evert}{toe beloof me dat}\\

\haiku{Beloof me, als het,.}{maar enigszins doenlijk is dit}{te verhinderen}\\

\subsection{Uit: Else B\"ohler, Duitsch dienstmeisje}

\haiku{Simon Vestdijk,}{Else B\"ohler Duitsch dienstmeisje}{Colofon}\\

\haiku{Ik zou nauwelijks,:}{klaarkomen in de weken}{die mij nog resten}\\

\haiku{Maar opgeschreven, -.}{heb ik het nog nooit en dat}{moet nu gebeuren}\\

\haiku{Zeven bladzijden,:}{zijn verstreken nu zal ik}{woord moeten houden}\\

\haiku{Nooit heb ik van een.}{mannelijk lichaam gewalgd}{als van het zijne}\\

\haiku{ik meende alleen.}{een onbedwingbare trek}{te hebben in thee}\\

\haiku{Als ik mijn moeder,:}{vroeg wat minder door het huis}{te schreeuwen zei ze}\\

\haiku{Meneer Steketees,,.}{platje gelijkvormig aan}{het onze was leeg}\\

\haiku{In drie stappen was,.}{ik bij de suitedeuren}{wrong me er doorheen}\\

\haiku{Moeilijkheden van.}{zinsbouw en idioom nam ik}{als hindernissen}\\

\haiku{Onafgebroken.}{voelde ik haar zwarte ogen}{op mij gevestigd}\\

\haiku{Ik begreep, dat ik,.}{het onderspit delven zou}{als het zo doorging}\\

\haiku{Maar ook wist ik, dat...}{die slag geen minuut later}{had moeten komen}\\

\haiku{tegenover mezelf,.}{bleef die even groot welke weg}{ik ook zou inslaan}\\

\haiku{dann lege ich drei,.}{Steine hin auf den Balkon}{dann weisst du Bescheid}\\

\haiku{Raakte ik haar borst,}{aan dan gedoogde ze dat}{net zo lang tot ik}\\

\haiku{trouwen konden we,.}{in Duitsland of Belgi\"e als}{Utrecht onwillig bleek}\\

\haiku{Tevergeefs hield ik,;}{haar voor hoe zalig het nu}{in het park zou zijn}\\

\haiku{Else B\"ohler werd nu;}{ontoegankelijk voor al}{mijn opmerkingen}\\

\haiku{druk babbelend sprong;}{ze van het ene onderwerp}{op het andere}\\

\haiku{ook het uitspreken.}{van mijn eigen voornaam ging}{nooit zonder kleuren}\\

\haiku{Ik was ineens diep,.}{gelukkig ik had kunnen}{huilen als een kind}\\

\haiku{Mijn duiveltje dook,.}{weer op  met een dikke}{tong van de warmte}\\

\haiku{{\textquoteright} - {\textquoteleft}M\"ochtest du dann, dass?!}{ich auf meine Kniee fiel wie}{Fr\"aulein Erkelens}\\

\haiku{Overigens heb je!}{er niets van begrepen wat}{ik toen beweerde}\\

\haiku{God, o god, wat een...{\textquoteright} {\textquoteleft},?}{afschuwelijk levenLaat}{u me door of niet}\\

\haiku{{\textquoteright} Bliksemsnel streek mijn,.}{moeders hand mij door het haar}{toen ik mijn sprong nam}\\

\haiku{Zij had zich van me.}{losgemaakt en zwaaide met}{haar lange armen}\\

\haiku{Het was duidelijk,;}{dat er geen zuinigheid bij}{haar voor kon zitten}\\

\haiku{een demonstratief.}{gekraak bewees dat ze haar}{bed had opgezocht}\\

\haiku{Die inktvis met 't,!}{achterwerk van een kreeft wat}{zit daar al niet in}\\

\haiku{{\textquoteright} Wat ik nu ging doen,,.}{verbaast mij nu ik er aan}{terug denk nog steeds}\\

\haiku{Na alles wat je,!}{me zo juist verteld hebt ligt}{het toch voor de hand}\\

\haiku{Langzaam viel de deur.}{achter mij dicht en kleefde}{weer in de posten}\\

\haiku{Het was mijn moeder,...}{dus niet maar Eg die nu in}{de huiskamer zat}\\

\haiku{Ich h\"atte Dir schon,...?}{l\"angst ehergeschrieben aber Wie}{geht es Dir sonst noch}\\

\haiku{Verlange nur nach,.}{Dir uw Dich einmal gl\"ucklich}{machen ta kannen}\\

\haiku{Heute Abend blutet,!}{mein Hert es schreit nach etwas}{Unmdgliches}\\

\haiku{Bis Heute darf ich.}{frei and ehrlich noch jeden}{gegen\"uberstehen}\\

\haiku{Ik wilde het doen:}{voorkomen alsof de brief}{mij geheel koud liet}\\

\haiku{Een naaimandje op.}{de tafel was zo groot als}{een boodschappenmand}\\

\haiku{Ik begreep niet, hoe.}{ik het zo lang bij die stem}{uitgehouden had}\\

\haiku{Peter zou mij nu...}{een lesje willen geven}{met de vrouw van 43}\\

\haiku{Hij stond in de deur.}{van het atelier verbaasd naar}{boven te kijken}\\

\haiku{Toen ik zei, dat ik,.}{weg ging bedoelde ik niet}{alleen hier vandaan}\\

\haiku{{\textquoteleft}U had haar bij ons,;}{thuis moeten ontvangen uit}{eigen beweging}\\

\haiku{Ik zocht naar meisjes;}{met vlechten die op Else}{B\"ohler zouden lijken}\\

\haiku{Ik moest enkele.}{malen slikken voordat ik}{naar binnen durfde}\\

\haiku{van de vijf winkels}{die ik hier overzag koos ik}{om te beginnen}\\

\haiku{die welke zo ver;}{mogelijk van de winkel}{van Steinmann aflag}\\

\haiku{{\textquoteright} vroeg ik ademloos, na.}{haar een paar pas achterna}{gelopen te zijn}\\

\haiku{hoe ik Else zou,;}{kunnen vinden en waar\`om ik}{haar wilde vinden}\\

\haiku{Und Moskau zahlt mir,.}{all den Schund So fahre ick}{gang Deutschland rund}\\

\haiku{Ik stelde vast, dat,.}{men mij overwonnen had dat}{ik terug moest gaan}\\

\haiku{Het woord {\textquoteleft}Sch\"utzenfest{\textquoteright},.}{kende zij w\`el maar dat is}{het \'o\'ok niet geweest}\\

\subsection{Uit: Verzamelde romans. Deel 35. De filosoof en de sluipmoordenaar}

\haiku{wie weet hadden zij:}{zelfs iets met de letteren}{uitstaande gehad}\\

\haiku{Ik heb gezien, dat,,.}{u een dapper man bent die}{recht heeft op mijn naam}\\

\haiku{Beauregard was naar;}{zijn regiment vertrokken}{in de Provence}\\

\haiku{Niemand scheen Karel,.}{als een levend eens levend}{mens te beschouwen}\\

\haiku{De gezochte stond,.}{bij een bloemenkar vijf of}{zes huizen verder}\\

\haiku{- {\textquoteleft}Lef\`evre is geen,,.}{naam waarvoor u zou moeten}{vluchten kolonel}\\

\haiku{Wat zoudt u mij al!}{niet over de Zweedse koning}{kunnen verhalen}\\

\haiku{Natuurlijk waren:}{er de paspoortachtige}{uiterlijkheden}\\

\haiku{In uw brief repte...{\textquoteright} {\textquoteleft},!}{u van een half uurGaat u}{zitten kolonel}\\

\haiku{Had ik verder mijn,.}{mond gehouden dan had er}{geen haan naar gekraaid}\\

\haiku{h\'ad u het gedaan,.}{dan zou u het mij gerust}{kunnen vertellen}\\

\haiku{Dit zijn de Pens\'ees,.}{van Blaise Pascal die u}{van naam zult kennen}\\

\haiku{dat had hij aan de;}{zoveel jongere Alexander}{moeten overlaten}\\

\haiku{Met een krachtig woord.}{riep Arouet zijn factotum}{tot de orde}\\

\haiku{Het eerste wat hem;}{bereikte was de geur van}{zeer goede koffie}\\

\haiku{Holm aan het werk, en:}{nog geen drie weken later}{komt er een wagen}\\

\haiku{Ik zal aanstonds de,.}{knecht roepen dan kan hij ons}{koffie inschenken}\\

\haiku{Wij gaan met niemand,...}{om en binnenkort gaan we}{naar Zweden terug}\\

\haiku{{\textquoteright} {\textquoteleft}Pardon, mevrouw,{\textquoteright} zei, {\textquoteleft}.}{Arouethet boek handelt over}{Karel de Twaalfde}\\

\haiku{Het wordt allemaal,...}{later geregeld tot ons}{aller genoegen}\\

\haiku{Past u op, anders!}{laat hij mijn kopje vallen}{of dat van de graaf}\\

\haiku{{\textquoteright} {\textquoteleft}Wanneer ik u zo,...}{hoor zou ik ertoe kunnen}{komen een boek over}\\

\haiku{{\textquoteright} {\textquoteleft}Is het waar, mijnheer,?}{dat hij zichzelf de kroon op}{het hoofd heeft gezet}\\

\haiku{Het komt eigenlijk.}{alleen doordat zij en ik}{met niemand omgaan}\\

\haiku{laat h\'em dan schrijven, -, -;}{in dat boek dus dat hij je}{voor onschuldig houdt}\\

\haiku{{\textquoteright} {\textquoteleft}Ik vermoed, dat hij,.}{zich verveelde en wilde}{filosoferen}\\

\haiku{Een staatsieportret,;}{was het niet de koning was}{niet in groot ornaat}\\

\haiku{{\textquoteright} zei Holm, die over een,, {\textquoteleft}}{zeer zachte wat hese stem}{bleek te beschikken}\\

\haiku{In 1718 was het in,.}{Stockholm de schilder heb ik}{toen nog even ontmoet}\\

\haiku{Er schuilt in deze.}{gelaatstrekken iets van een}{gevallen engel}\\

\haiku{Deze verschijning.}{had geglimlacht en op zijn}{gezicht gewezen}\\

\haiku{Ziekte scheen hij als.}{ernstiger te beschouwen}{dan verwondingen}\\

\haiku{Over zijn karakter.}{schrijf ik natuurlijk alleen}{wat ik zeker weet}\\

\haiku{De koning stond stil,,.}{keek naar D\"uring en begon}{zachtjes te lachen}\\

\haiku{Of eigenlijk nog:}{meer door wat u mij over de}{gravin vertelde}\\

\haiku{Maar tevens zou het}{erg interessant  voor}{u zijn om te zien}\\

\haiku{{\textquoteleft}U te beloven.}{het in mijn boek openlijk voor}{u op te nemen}\\

\haiku{{\textquoteright} {\textquoteleft}Dat is niet nodig,{\textquoteright}, {\textquoteleft}.}{zei Siquier droogjesik}{ken die gevoelens}\\

\haiku{U moet vooral ook,.}{niet menen dat mijn leven}{dan verwoest zou zijn}\\

\haiku{Zij schreef, dat zij hem.}{gaarne deze zelfde avond}{nog zou ontvangen}\\

\haiku{En zij van haar kant:}{was ge{\"\i}mponeerd door de}{militaire stand}\\

\haiku{Dan moet Octave,.}{er maar voor zorgen dat zijn}{naam gezuiverd is}\\

\haiku{{\textquoteright} viel de graaf in, {\textquoteleft}mijn.}{zuster riep mij om koffie}{te komen drinken}\\

\haiku{{\textquoteleft}In elk geval maakt.}{het niet het minste verschil}{voor mijnheer Arouet}\\

\haiku{dat de kolonel.}{Zweden waarschijnlijk spoedig}{weer zou verlaten}\\

\haiku{Jij hebt Octave,!}{zelf verzekerd dat je hem}{voor onschuldig hield}\\

\haiku{Even later was hij,.}{misschien slaperig toen hij}{het zei beslist niet}\\

\haiku{Ook dat hielp hem niet,.}{al is daar tenminste nog}{een maand naar genoemd}\\

\haiku{ik heb twee gekken,,.}{als zoons de een schrijft proza}{de ander verzen}\\

\haiku{In de open deur stond,.}{de huishoudster hijgend van}{het trappen lopen}\\

\haiku{Siquier kwam de.}{kamer binnen met \'e\'en hand}{tastend naar voren}\\

\haiku{Toch alleen maar dat?}{zij tegenover mij haar mond}{voorbij heeft gepraat}\\

\haiku{Maar ik kan hierna.}{geen vertrouwen meer hebben}{in haar karakter}\\

\subsection{Uit: Verzamelde romans. Deel 37. De held van Temesa}

\haiku{ik zeg dit waarlijk,':}{niet omdat ik Polites}{priester ben geweest}\\

\haiku{{\textquoteright} Nooit heb ik iemand,,.}{zo zien verbleken ik dacht}{dat hij mij zou slaan}\\

\haiku{Periphas was,;}{wel een goede jongen en}{verstandig genoeg}\\

\haiku{Toen ik terugkwam,,,,;}{stond hij er nog zwaar blank en}{slaperig oeroud}\\

\haiku{{\textquoteleft}Roeit elkander uit,,.}{Kroton en Sybaris mijn}{volk blijft erbuiten}\\

\haiku{Het mijne vond ik;}{waar mijn makkers het hadden}{achtergelaten}\\

\haiku{De zegswijze {\textquoteleft}hij{\textquoteright}.}{vecht als een Krotoni\"er}{stamt uit die dagen}\\

\haiku{Toen richtte zij de,,:}{ogen weer op mij heel rustig}{niet onwelwillend}\\

\haiku{Maar het was waar, een.}{ander antwoord kon zij niet}{van mij verwachten}\\

\haiku{mijn zoon, want je schijnt.}{wel wat vaderlijke steun}{nodig te hebben}\\

\haiku{Ik zou schatrijk zijn,.}{als ik deze boosdoeners}{niet om mij heen had}\\

\haiku{hier komt Hekate,,, -,.}{brrr boe een luimige}{vrouw mag ik zeggen}\\

\haiku{Zij hielden het op,;}{het oude maar stonden ook}{voor het nieuwe open}\\

\haiku{Maar ik had niet eens.}{het arglistig groen in zijn}{ogen zien verschieten}\\

\haiku{Het was ook niet in;}{mijn voordeel al te tergend}{met hem te spelen}\\

\haiku{Bij mijn rustbed stond,.}{een man niet te herkennen}{in de schemering}\\

\haiku{Maar toen, alsof het,.}{erop gewacht had verscheen}{het valkenkopje}\\

\haiku{hoe w\'eet de Held, of,?}{ze nog leven of een heel}{klein beetje leven}\\

\haiku{Heel Hellas legde.}{de kiemen van ontbinding}{in onze boezem}\\

\haiku{Haar meesterlijke.}{onzijdigheid richtte zich}{nu op mij alleen}\\

\haiku{Ik heb met een der,.}{beambten gesproken en}{ik ben geschrokken}\\

\haiku{Sinds mijn aanstelling,,.}{is hij naar het schijnt weer met}{wurgen begonnen}\\

\haiku{Orion jaagt daar nog,.}{steeds Achilles trekt ten strijde}{of zit in zijn tent}\\

\haiku{Ook als je niet bij,,}{ons komt neem dan toch ontslag}{als priester doe dat.}\\

\haiku{{\textquoteright} Midden in de nacht,.}{werd ik gewekt omdat het}{hero\"on in brand stond}\\

\haiku{Het kostte mij heel.}{wat moeite hem tot beter}{inzicht te brengen}\\

\haiku{2 De volgende:}{ochtend begaven zich naar}{het prytaneion}\\

\haiku{Maar om de Held te.}{kunnen wegjagen moet men}{hem eerst ontmoeten}\\

\haiku{Dat Praxidamas zijn.}{naam wil lenen is misschien}{toch niet voldoende}\\

\haiku{Nu staat het vast, dat.}{Theagenes in diens}{bestaan geloofde}\\

\haiku{Er zou wat bloed van':}{Polites aan Euthymos}{zwaard moeten kleven}\\

\haiku{{\textquoteright} {\textquoteleft}Het voorschrift wil, dat.}{zowel het offer als de}{priester in slaap zijn}\\

\haiku{{\textquoteleft}Men had mij hierop,.}{voorbereid men schilderde}{u af als koppig}\\

\haiku{Later bleek hij van.}{het Lokrisch gezelschap toch}{deel uit te maken}\\

\haiku{{\textquoteright} {\textquoteleft}Ja...{\textquoteright} - Theagenes,.}{liet al zijn voorhoofdsrimpels}{spelen links en rechts}\\

\haiku{De exegeet had hier,.}{veel verdriet van hij begreep}{het niet helemaal}\\

\haiku{dat spel ken ik niet,{\textquoteright},.}{en stak zijn hand uit als om}{ze aan te raken}\\

\haiku{Maar bovendien zag.}{ik er tegenop om met}{hem alleen te zijn}\\

\haiku{ze had Euthymos,!}{gezien  en er nooit met}{mij over gesproken}\\

\haiku{maar onderwijl dacht.}{ik voortdurend na over wat}{ik hoorde en zag}\\

\haiku{{\textquoteleft}O, is het dat,{\textquoteright} zei, {\textquoteleft}.}{ik met een lachjemaar nu}{vergist u zich toch}\\

\haiku{Ik wist alleen, dat,.}{u een dochter had die nu}{veertien jaar moet zijn}\\

\haiku{{\textquoteright} Terwijl hij zich naar,:}{de deur gewoog hoorde ik}{hem nog prevelen}\\

\haiku{Op een middag, toen,.}{zij bij die weduwe was}{doorzocht ik de hoop}\\

\haiku{Eindelijk trok ik,,.}{hem naar binnen sloot de deur}{maar maakte geen licht}\\

\haiku{het duurde een tijd,:}{voordat ik begreep dat het}{betekenen moest}\\

\haiku{Later hoorde ik,.}{dat hij de bewakers in}{zijn huis niet verdroeg}\\

\haiku{Men schreeuwde nog niet,.}{aan \'een stuk door maar men was}{bereid dit te doen}\\

\haiku{Toen de agora nog,.}{natgeregend werd scheen bij}{mij de zon alweer}\\

\haiku{Staken van de dienst,,, -.}{nooit  meer een offer nooit}{meer meer wist hij niet}\\

\haiku{En deze Held zoudt?}{gij laten wegboksen door}{een onbevoegde}\\

\haiku{Zoals gij weet, zijn.}{hier twee betekenissen}{aan te verbinden}\\

\haiku{Is hij niet moedig,.}{genoeg dan mogen wij geen}{toestemming geven}\\

\haiku{Bij de uitlegging.}{van het orakel heeft hij de}{doorslag gegeven}\\

\haiku{men weet het maar al,.}{te  goed alleen niet in}{bijzonderheden}\\

\haiku{Zijn haar was bijna,,.}{spierwit en zeer dun ofschoon}{hij nog niet kaal was}\\

\haiku{Ziet, hij springt over de,.}{muur en in het bos kraken}{de dorre takken}\\

\haiku{Want ik geloof, dat,,.}{u moedig bent Euthymos}{ik heb het gezien}\\

\haiku{Onder het volk gaat,.}{het gerucht dat Euthymos}{met haar trouwen zal}\\

\haiku{Er zijn ook nieuwe,.}{gezichten onder u als}{ik mij niet bedrieg}\\

\haiku{zo nam ik de taak, {\textquoteleft}.}{van Theagenes overwaar}{anderen bij zijn}\\

\haiku{Nu en dan wierp hij;}{een ontevreden blik over}{de aanwezigen}\\

\haiku{Hier had hij zijn zwaard,.}{neergelegd niet ver van dat}{van zijn belager}\\

\haiku{{\textquoteright} {\textquoteleft}Goed, dan een schim,{\textquoteright} zei, {\textquoteleft}.}{Euthymos wreveligmaar}{laat mij uitspreken}\\

\haiku{Ik wilde Plexippos,.}{niet wekken dat hadden wij}{niet afgesproken}\\

\haiku{Hij schuifelde wat,,.}{met \'een voet en keek langzaam}{opzij onderuit}\\

\haiku{Of misschien dacht u,...{\textquoteright} {\textquoteleft},{\textquoteright}}{niet eens dat ik loogDat dacht}{ik inderdaad niet}\\

\haiku{{\textquoteright} {\textquoteleft}Daar kunt u mij ook,.}{onder rekenen want ik}{begrijp er niets van}\\

\haiku{Gestenigd zou u,.}{niet meer worden maar er zijn}{andere straffen}\\

\haiku{{\textquoteright} {\textquoteleft}Ik dan rechtstreeks naar,.}{de agora waar ik u zal}{kunnen aanklagen}\\

\haiku{u heeft de Held niet,.}{gezien en u heeft hem niet}{kunnen verjagen}\\

\haiku{Ik vind het alleen,.}{maar rechtvaardig dat u het}{nu zelf ondervindt}\\

\haiku{{\textquoteright} {\textquoteleft}Maar wie waarborgt mij,?}{dat u na afloop niet t\'och}{een aanklacht indient}\\

\haiku{Maar werd het laat, dan.}{zou ik mijn plannen een dag}{uit moeten stellen}\\

\haiku{Hij zou daar aan huis.}{moeten komen om zijn}{verhaaltje te doen}\\

\haiku{Ziet men Euthymos?}{en Krokinas naast elkaar}{voor de rechter staan}\\

\haiku{Geen weldenkend mens,}{zal ontkennen dat wat ik}{ging ondernemen}\\

\haiku{Wij samen weten,...}{wel beter maar daar heeft z{\'\i}j}{niets mee te maken}\\

\haiku{E\'enmaal had hij,.}{de hand bewogen een niet}{onsierlijk gebaar}\\

\haiku{Hij moest weten, dat.}{de bewaker vijf of zes}{pas achter hem stond}\\

\haiku{Hij voelde zich niet,;}{op zijn gemak dat was aan}{alles te merken}\\

\haiku{Zij had de armen,.}{om mij heengeslagen en}{wilde nooit meer weg}\\

\subsection{Uit: Verzamelde romans. Deel 2. Meneer Visser's hellevaart}

\haiku{fusilleert alle!!}{voorradige kolonels}{als landverrader}\\

\haiku{Proberen vanavond,,,,,...}{met Anton dag broer bgoer bgoer}{dag bgoer dag bgoerie}\\

\haiku{{\textquoteright} Dat loeren kan ze,.}{niet laten ze w\'eet wie er}{om vijf voor half belt}\\

\haiku{Wanneer Visser over,;}{zijn vrouw begon antwoordde}{hij met de jongens}\\

\haiku{Hij ging bijvoorbeeld,.}{zo staan dat ze telkens een}{omweg maken moest}\\

\haiku{En zoals altijd:}{liet hij zijn oog dreigend over}{de tafel glijden}\\

\haiku{Op de loper \'een,, - ', '!}{twee alst oneven is}{zalt geen pijn doen}\\

\haiku{Dadelijk, jongen,.}{maar laat me eerst je moeder}{een handje geven}\\

\haiku{Jansonius tonnenman,...}{zo ben ik trouwens aan de}{heren gekomen}\\

\haiku{Gelukkig is de,,...}{keukendeur dicht Marie weet}{dat ik niet gestoord}\\

\haiku{Zijn angst met geweld,.}{wegduwend concentreerde}{hij zich op die taak}\\

\haiku{het kleine dorpje,.}{in de buurt van Weulnerdam}{waar hij vandaan kwam}\\

\haiku{wie laat er nou in!}{christesnaam een bouffante}{op een stoel hangen}\\

\haiku{Hij boog zich omlaag.}{en richtte zijn gezicht scheef}{op naar de hemel}\\

\haiku{Een kaatsbal of een '.}{tol door de ruiten was nog}{niet eenst slimste}\\

\haiku{De tuin, die er bij,;}{hoorde zette zich voort tot}{aan de Achterweg}\\

\haiku{misschien weet ie wat...!}{positiefs over die schorsing}{Niet-aan-denken}\\

\haiku{Je moet 't hier dan,.}{maar zien te bolwerken met}{inspecteur Blanksma}\\

\haiku{Wie anders dan ik?}{was de belangrijkste man}{op dat uniek moment}\\

\haiku{Het welbehagen,.}{was blijven duren maar werd}{nu bijna drukkend}\\

\haiku{t Zal u anders,!}{wel amuseren dat gedoe}{in een kleine plaats}\\

\haiku{Maar toch deed het hem,...}{goed iemand anders in zijn}{val mee te slepen}\\

\haiku{Hij moest eens precies!}{weten hoe of een hart nou}{eigenlijk smaakte}\\

\haiku{een ziekte na de, ':}{dood een ziekbed in vet of}{boter opt vuur}\\

\haiku{Maar dat hameren '.}{op die werf is toch weln}{opwekkend geluid}\\

\haiku{Achter, naast me. Daar,... '...}{helemaal links voorbijt}{been van m'n oogkas}\\

\haiku{Bah... Zorgen dat ik...,;}{om drie uur thuis ben Zware}{laarzen die pummel}\\

\haiku{En omdat ze maar... {\textquoteleft}}{niet ophield met zeuren wat}{kon ik toen anders}\\

\haiku{{\textquoteleft}Nee, dat was tegen,,.}{jou Cohenn hij vind die mop}{aardiger dan ik}\\

\haiku{In een auto, in,....}{een auto je stapt er in}{Formidabel wijf}\\

\haiku{De rol van Lehmans,...}{nog even repeteren als}{ik van Eveking af}\\

\haiku{niemand durft weggaan,.}{om de anderen net als}{op een visite}\\

\haiku{aan het strand... grote......}{warme hand warm poposant}{blauw glanzig schepje}\\

\haiku{Daar uit de Raamstraat.}{komen er nog twee van die}{aangeklede apen}\\

\haiku{In d\'eze angst, door,.}{hemzelf uitgelokt was hij}{volkomen veilig}\\

\haiku{Ik zou nou alleen '.}{wel willen weten hoe of}{t nu verder moet}\\

\haiku{{\textquoteleft}U zult 't wel gek,, '.}{vinden meneer maart was}{de commissaris}\\

\haiku{Volgens het Wetboek...{\textquoteright}}{van Strafrecht en het Wretboek}{van Strafvordering}\\

\haiku{De commissaris ',.}{staarde hem int gezicht}{toen naar zijn colbert}\\

\haiku{Er waren zoveel,...}{ijle stemmen in de lucht}{maar die ene bleef weg}\\

\haiku{{\textquoteright} Cohen begon aan.}{de mop van de man in de}{trein die stotterde}\\

\haiku{{\textquoteleft}Zeg lui, nou heb ik...{\textquoteright} {\textquoteleft}!}{de vrouw er weer eens \'a\'ardig}{tussen gehadAha}\\

\haiku{{\textquoteleft}Vertel me maar eens,!}{wat over dat boek dat je van}{me gekregen hebt}\\

\haiku{{\textquoteleft}Hebben u en uw,?}{man militairistische}{neigingen mevrouw}\\

\haiku{elf min drie, elf min,?}{twee ja ja ja waar was ik}{ook weer gebleven}\\

\haiku{gewoon een bol, een,:}{ballon die je kan blazen}{en bij je steken}\\

\haiku{Hij draaide zich om.}{en bewoog zich aarzelend}{naar het rechtse bed}\\

\haiku{{\textquoteright} {\textquoteleft}Ik kan je zo niet,,,!}{ontvangen Visser d'r is}{niets in orde niets}\\

\haiku{Ik ben niet van plan,!}{me voor de gek te laten}{houden verduiveld}\\

\haiku{{\textquoteleft}Ik geef het woord aan,!}{Dr. Touraine geneesheer}{te dezer stede}\\

\haiku{ze zouden misschien...}{wel niet eens begrijpen wie}{daar weggebracht werd}\\

\haiku{Er lopen heel wat '.}{meneer Vissers rond zonder}{t zelf te weten}\\

\haiku{- Weer speelde, in de,.}{slaapkamer het kaarslicht in}{twee vragende ogen}\\

\subsection{Uit: Verzamelde romans. Deel 6. De nadagen van Pilatus}

\haiku{Hij keek nog eens, en,:}{zei met nadruk de hand schuin}{naar omlaag gestrekt}\\

\haiku{De oude man blies,.}{als een panter maar waagde}{zich niet dichterbij}\\

\haiku{Even tevoren had...}{hij die beweging bij de}{keizer opgemerkt}\\

\haiku{ik klim iedere,;}{dag naar het Capitool hij}{stroomt naar de Tiber}\\

\haiku{Over uw indrukken...{\textquoteright} {\textquoteleft},!}{Natuurlijk van dat proces}{deugde geen jota}\\

\haiku{In Rome wonen,...}{betrekkelijk weinig}{Romeinen helaas}\\

\haiku{\'e\'en ervan was nu.}{onbeweeglijk geworden}{onder zijn lippen}\\

\haiku{Ik wil je straks je,,}{handen laten wassen ik}{zie het weer voor me}\\

\haiku{dat zullen we de,!}{kinderen op straat leren}{als aftelversje}\\

\haiku{{\textquoteleft}Nog \'e\'en volksopstand,,.}{mijn beste Pilatus en}{je vrouw gaat eraan}\\

\haiku{Claudia, dat is mijn,,;}{vrouw ze is nu dood was niet}{over hem uitgepraat}\\

\haiku{Ik had niets tegen,,...}{de man Antipas koning}{Antipas ook niet}\\

\haiku{Wij vonden iemand, -:}{bereid een graf af te staan}{je herinnert je}\\

\haiku{Ik kan het alleen,... -,{\textquoteright}}{wel af ik Ik zal mijn best}{doen voor je vader}\\

\haiku{Kijk eens, ik had een.}{opstand moeten dempen om}{uw man te redden}\\

\haiku{Hij herhaalde slechts,.}{wat Maria hem had gezegd}{de vorige nacht}\\

\haiku{{\textquoteright} riep Caligula, {\textquoteleft}...}{bestraffendhet was alleen}{wat onvolledig}\\

\haiku{Of ik heb gedroomd,, -...}{dat ik hem volschreef ik droom}{vaak van de goden}\\

\haiku{Secundo, was hij,?}{ervan overtuigd dat hij na}{de dood zou opstaan}\\

\haiku{{\textquoteright} {\textquoteleft}Het zal prins Agrippa.}{wellicht interesseren}{met haar te praten}\\

\haiku{De beledigers.}{waren overigens al lang}{uit de weg geruimd}\\

\haiku{wilde ze niet, dan;}{zou hij er haar desnoods met}{geweld heenbrengen}\\

\haiku{Als ik Barachius,.}{niet geholpen had had ik}{het nu rustiger}\\

\haiku{Maar hoe meer ik wil,,.}{hoe meer mijn lichaam zegt dat}{er niets van inkomt}\\

\haiku{De gehele dag;}{had hij gedaan alsof hij}{alleen in huis was}\\

\haiku{de heros werd tot,;}{demon die dan toch altijd}{nog sterfelijk was}\\

\haiku{Ik nooit, en toch zegt,;}{het orakel dat ik eerder}{zal sterven dan hij}\\

\haiku{Overigens nam men.}{hem dit evenmin kwalijk als}{een rots of een wolk}\\

\haiku{Als ik je gelast,,?}{dit schip in brand te steken}{doe je het dan ook}\\

\haiku{Samenscholende.}{gasten wachtten hun beurt af}{voor het rouwbeklag}\\

\haiku{Ik ben zelfs bereid.}{om mij in uw liggende}{houding te schikken}\\

\haiku{Maar de lamp scheerde:}{over het voeteneinde en}{boetseerde een iets}\\

\haiku{Bedienden snelden,,,;}{toe eunuchen Ethiopische}{slaven een lijfarts}\\

\haiku{Zelfs de rennen en,.}{de bordelen liet hij in}{de steek of zij hem}\\

\haiku{Tenslotte heeft hij...}{het beeld een kinnebakslag}{gegeven ook nog}\\

\haiku{Het bewijs van de;}{moord zou niet gemakkelijk}{te leveren zijn}\\

\haiku{Mijn vertrouwen in.}{hem is sterk verminderd na}{deze schapebrief}\\

\haiku{Met een paar stappen,.}{was hij bij haar maar zonder}{haar aan te raken}\\

\haiku{De tribuun nam het,.}{woord toen Gajus zich reeds schuin}{achter hem bevond}\\

\haiku{Niet veel beter bracht.}{Quirinius Fannius}{Piso het eraf}\\

\haiku{Jij moet weten wie,.}{jouw mannen omkopen ook}{als ik het zelf ben}\\

\haiku{een sfinxfiguur, een,?}{obelisk tempelbogen en}{tempelpylonen}\\

\haiku{men wil Mnester, doch,.}{een grote bruine beer die}{de slaper besloop}\\

\haiku{dat dit zijn eerste,!}{kruisiging was de eerste}{die hij bijwoonde}\\

\haiku{Niet iedereen was;}{zo spoedig op de hoogte}{als Caligula}\\

\haiku{Van dit vroom bedrog.}{was hij zelf in de eerste}{plaats het slachtoffer}\\

\subsection{Uit: Verzamelde romans. Deel 49. Het schandaal der Blauwbaarden}

\haiku{- {\textquoteleft}Signor Bohlen heeft een,.}{roman geschreven die hier}{in Florence speelt}\\

\haiku{{\textquoteright} {\textquoteleft}Als Mr. Bohlen niet wil,,{\textquoteright}.}{doe {\'\i}k het zei Wilkie met}{merkbare zelfspot}\\

\haiku{waarschijnlijk zaten.}{zij in hun torens op hun}{saffraan te knagen}\\

\haiku{Met geweld dwong ik.}{mij mijn verwachtingen niet}{te hoog te spannen}\\

\haiku{Een historicus.}{zal de familie wel niet}{hebben voortgebracht}\\

\haiku{waarom bouwde men?}{eigenlijk al die torens}{in San Gimignano}\\

\haiku{Dat neemt niet weg, dat.}{ik het verhaaltje nu zelf}{wel af kan maken}\\

\haiku{Ben ik graaf Giorgio,}{ter wille ontdek ik hier}{in San Gimignano}\\

\haiku{Witte dwergen zijn,,...}{niet jaloers Signor Wilkie}{dat is toch bekend}\\

\haiku{Hij zei tenminste,:}{kennelijk blij weer Engels}{te kunnen spreken}\\

\haiku{{\textquoteright} - Met deze woorden,.}{keek ik hem strak aan maar hij}{reageerde niet}\\

\haiku{Veeleer waren de,,.}{blikken die ons bereikten}{uitgesproken nors}\\

\haiku{Maar ik vraag mij af,.}{of iemand anders het stuk}{vervalst kan hebben}\\

\haiku{{\textquoteright} Met een kort gebaar.}{veegde Lampugnani de}{tegenwerping weg}\\

\haiku{Misschien vindt u het.}{verslag van het proces en}{van de foltering}\\

\haiku{Lampugnani liet.}{geen twijfel bestaan aan zijn}{mening daaromtrent}\\

\haiku{Ik moet een berucht!}{vrouwenmoordenaar vragen}{of hij het wel is}\\

\haiku{Waarmee niet gezegd,.}{wil zijn dat hij zich op straat}{bijzonder haastte}\\

\haiku{Ik had niet nodig}{naar zijn laarzen te kijken}{om te begrijpen}\\

\haiku{Zijn rechtermouw was,,.}{met bloed bevlekt tenminste}{daar hield ik het voor}\\

\haiku{Dat heb ik u al,,.}{gezegd Signore de slaap}{had mij overmeesterd}\\

\haiku{{\textquoteleft}Uw vriend - ik neem aan, -.}{dat het uw vriend is denkt graag}{kwaad van de mensen}\\

\haiku{{\textquoteright} Na even hulpeloos,:}{om zich heengekeken te}{hebben zei de man}\\

\haiku{Wij beloofden hem.}{niet aan de politie te}{zullen verraden}\\

\haiku{{\textquoteright} stelde ik voor, {\textquoteleft}men...{\textquoteright} {\textquoteleft}?}{hoort wel van die dingenBij}{het scheren toch niet}\\

\haiku{Volgens de dokter,.}{was dit onmogelijk maar}{die wist er niets van}\\

\haiku{Het hoeft u niet de,,.}{minste moeite te kosten}{geen toelichting niets}\\

\haiku{Bel ik bij zijn klein,,.}{bouwvallig Palazzo aan}{dan geeft hij niet thuis}\\

\subsection{Uit: Verzamelde romans. Deel 24. De verminkte Apollo}

\haiku{De westelijke.}{lag reeds van de namiddag}{af in het duister}\\

\haiku{Enkele priesters.}{stapten haastig naar binnen}{en vertrokken weer}\\

\haiku{Voor het altaar van.}{Poseidon zouden mollen}{worden geofferd}\\

\haiku{- {\textquoteleft}Had zij verbanning,.}{ge\"eist men zou de straf niet}{hebben voltrokken}\\

\haiku{Vergeef mij, Aletes,:}{wij hebben allen onze}{eigen geaardheid}\\

\haiku{In het vierde jaar}{na de val van Krisa had}{hij van zijn vader}\\

\haiku{Toen het karige,.}{winterlicht hun bed bescheen}{was zij er niet meer}\\

\haiku{Voor de avond hoopten;}{de mannen haar in Delphi}{te hebben gebracht}\\

\haiku{Door een heraut met.}{een gouden skepter heeft hij}{het laten weten}\\

\haiku{Je zult niet rusten,}{voor je Hem tot God van de}{oorlog hebt gewijd}\\

\haiku{Weer rommelde het.}{en het vertrek werd doorschokt}{van kalkwit licht}\\

\haiku{Ik maak mij niet mooi,.}{voor andere mannen zelfs}{niet voor je vader}\\

\haiku{Men liet haar lopen,,.}{maar bleef haar gangen volgen}{enige dagen lang}\\

\haiku{Dat was nieuw voor ons,!}{die onderdrukking en al}{het nieuwe trekt aan}\\

\haiku{Wie waarborgt ons, dat,?!}{Apollo niet wilde dat het}{beeld gestolen werd}\\

\haiku{de tekens voor de.}{woeste en rochelende}{geluiden kent gij}\\

\haiku{Een bronzen beeld naar,,.}{Uw gelijkenis Gij weet}{het is gestolen}\\

\haiku{De eerste Pythia,.}{uit een voornaam geslacht sinds}{mensenheugenis}\\

\haiku{{\textquoteleft}Diomos, Diomos, zal (:}{verzoend worden zonder bloed}{dat betekende}\\

\haiku{Diomos vreesde het,.}{ergste en nam inderhaast}{afscheid van Leont{\'\i}on}\\

\haiku{Er liggen heel wat,.}{gegevens ter beschikking}{ook over het beeld zelf}\\

\haiku{Iemand als gij zou,...}{in staat zijn een stad uit te}{moorden met helpers}\\

\haiku{{\textquoteright} - Zij begonnen te.}{schateren en wierpen de}{armpjes in de lucht}\\

\haiku{al is het dan geen,;}{bloedvergieten het komt toch}{op onze hoofden}\\

\haiku{De laatste maal, dat,:}{ik hier vertoefde zei de}{Pythia tegen mij}\\

\haiku{Voor echte wreedheid';}{waren Kleisthenes lippen te}{vol en te goedlachs}\\

\haiku{Het vergiftigen;}{van waterleidingen is}{voortaan goddeloos}\\

\haiku{Ik noem ze Pythisch,.}{omdat ze om de vier jaar}{gehouden worden}\\

\haiku{Maar Periandros.}{wilden ze in geen geval}{in hun land hebben}\\

\haiku{De opvolger is,.}{nu die half Egyptische neef}{Hellas welgezind}\\

\haiku{Vandaar immers, dat.}{Periandros oorlog voert}{tegen Epidauros}\\

\haiku{Maar laat deze wijn,.}{uitgisten Kleisthenes drinken}{wij \'o\'ok nog wel op}\\

\haiku{{\textquoteright} {\textquoteleft}Het zijn demonen,{\textquoteright},.}{zei Gylidas en zuchtte}{ten tweede male}\\

\haiku{hij stond toch minder.}{onder haar invloed dan men}{altijd had gemeend}\\

\haiku{De wagenmenners;}{en Hippias wilden hem}{de weg versperren}\\

\haiku{voor zichzelf zag hij.}{geen verschil met meisjes van}{dezelfde leeftijd}\\

\haiku{{\textquoteright} - Hij stond op, en keek,.}{in de gele waakzame}{ogen van Gylidas}\\

\haiku{De enige die mij,!}{als een mens toespreekt en niet}{als een dwingeland}\\

\haiku{Op de drievoet zat.}{met wijdopengesperde ogen}{de blonde Pythia}\\

\haiku{Maar zou hij, door te,?}{weigeren Kleisthenes niet tot}{zijn vijand maken}\\

\haiku{Hierover zou men de,.}{Pythia kunnen raadplegen}{zei Onomakriton}\\

\haiku{En mijn vrouw kan ik,.}{niet vier jaar alleen laten}{zonder berichten}\\

\haiku{{\textquoteright} {\textquoteleft}Het meeste heb ik,{\textquoteright}.}{van mijn vader geleerd zei}{Diomos lusteloos}\\

\haiku{Zoals jij later,}{vergeten zult dat je vrouw}{je belasterde}\\

\haiku{Toch, alsof een stem,.}{het hem toegalmde wilde}{hij naar Messeni\"e}\\

\haiku{bijna steeds meende:}{men een stofwolk achter hem}{te zien opdoemen}\\

\haiku{Rechts ervan kroop de,.}{Helikon als een zwarte}{kartelige slang}\\

\haiku{Thraki\"ers zagen de:}{twee bezoekers voor het eerst}{als m\'e\'er dan slaven}\\

\haiku{zelfs ging hij zo ver:}{een hoofse toespeling te}{wagen op Achilleus}\\

\haiku{Uit een der tempels,.}{kwam jankend gegil gevolgd}{door vrouwengelach}\\

\haiku{Aletes mocht bloed doen, -...}{vloeien hij niet die tien maal}{beter doden kon}\\

\haiku{{\textquoteright} vroeg hij, toen de straat.}{zich verbreedde en sterker}{begon te stijgen}\\

\haiku{Het betekent, dat,}{het nabootsingen zijn van}{Dionysostempels}\\

\haiku{Hoe beijverde.}{Aletes zich om het hem naar}{de zin te maken}\\

\haiku{voor hen uit, ergens,.}{in de verte begon de}{hoofdweg naar Korinthe}\\

\haiku{Maskers op lange.}{staken schommelden nader}{uit de tempelstraat}\\

\haiku{Het was of er een,.}{zacht veelkleurig vocht over die}{oogbollen heenliep}\\

\haiku{{\textquoteleft}Blijf niet hier, moeder,{\textquoteright}, {\textquoteleft},}{zei de aanvoerder zachthoort}{de koning ervan}\\

\haiku{{\textquoteright} - Hij reikte Aletes,,,.}{de hand daarop na enige}{weifeling Diomos}\\

\haiku{{\textquoteright} snauwde Diomos, en, {\textquoteleft}}{greep naar de Delphische penning}{onder zijn kleren}\\

\haiku{En dat Delphi ons,.}{heeft uitgezonden blijkt uit}{onze papieren}\\

\haiku{- {\textquoteleft}Bijna iedereen,.}{houdt er rekening mee dat}{ik een grijsaard ben}\\

\haiku{Alsof Herakles!}{jullie Apollo niet allang}{doodgeslagen had}\\

\haiku{Met een waakzame.}{blik op zijn vriend bracht Aletes}{de hand aan het zwaard}\\

\haiku{Nog meer treden, nog,;}{meer voorbijgangers die geen}{acht op hen sloegen}\\

\haiku{Wij hebben er al,.}{onze hoop op gevestigd}{dat Lykophron nog leeft}\\

\haiku{{\textquoteleft}Zegt u die dingen,,{\textquoteright}, {\textquoteleft}}{toch niet jonge heer klaagde}{de oudere vrouw}\\

\haiku{{\textquoteright} vroeg zij aan Diomos, {\textquoteleft},...}{ik heet Mestra ik ben hier}{enkele maanden}\\

\haiku{Ik geloof, dat dit...{\textquoteright} {\textquoteleft}?}{beeld bestaat en opgespoord}{Waar komt gij vandaan}\\

\haiku{Hier was een man, die.}{zich beklaagde en daar zijn}{redenen voor had}\\

\haiku{ik doe een beroep.}{op het door Zeus beschermde}{heilige gastrecht}\\

\haiku{Zijn pijnlijk hoofd en.}{het bloed tussen zijn lippen}{gaven het antwoord}\\

\haiku{de enige straffen,.}{waren honger en ziekte}{zelden geseling}\\

\haiku{Iedereen mocht met;}{hem worstelen om de zweep}{te bemachtigen}\\

\haiku{Krekels en vogels,.}{deden zich horen en het}{gemurmel der stad}\\

\haiku{Daar zij de rang van,.}{beambte bekleedde mocht}{men haar niet doodslaan}\\

\haiku{En wie een gedicht,,.}{opzei zei een God op en}{geloofde in Hem}\\

\haiku{Haar gezicht had hij,.}{in zijn vuist kunnen nemen}{als een vogelei}\\

\haiku{Beter nog was het,,;}{wanneer Aletes bleef en aan}{de buitenkant sliep}\\

\haiku{Na  de koude.}{werden geen gevangenen}{meer binnengebracht}\\

\haiku{Het werd iets lichter,.}{de Kithairon hervatte}{zijn duister gegloei}\\

\haiku{{\textquoteright} {\textquoteleft}Ik ben bang zonder,.}{jou je weet niet hoe ik je}{vereer en liefheb}\\

\haiku{gedronken zou zij,.}{wel hebben maar de wijn was}{zuiver in haar bloed}\\

\haiku{ondraaglijke smart.}{werd weggeschoven achter}{bloedrode woede}\\

\haiku{Het duurde een tijd,.}{voor hij het zag liggen rood}{tot aan het gevest}\\

\haiku{Eens, bij fakkellicht,...}{had de vlek zich onzichtbaar}{weten te houden}\\

\haiku{Bij de eerste stap,.}{die hij deed moest hij zich aan}{een struik vasthouden}\\

\haiku{Het meisje keek naar,.}{het lichaam van haar vader}{en dan weer naar hem}\\

\haiku{E\'enmaal, trappend,:}{op vlijmscherpe distels liet}{hij het beeld vallen}\\

\haiku{Het beeld trok aan zijn,.}{schouder zodat hij vaak van}{hand verwisselde}\\

\haiku{{\textquoteleft}Ga naar de priester,,.}{die toezicht houdt of die in}{de tempel vertoeft}\\

\haiku{De zon gleed rond in,.}{zijn pas uitgebroken zweet}{beet en blakerde}\\

\haiku{Trixas was eigenlijk,...}{nog jong misschien had hij maar}{twee grijze haren}\\

\haiku{mijn grootste schuld, want...}{ik had hem niet uit het oog}{mogen verliezen}\\

\haiku{Het kon zijn, dat men.}{hem nooit goed geleerd had wie}{en wat Apollo was}\\

\haiku{Stellig en zeker.}{was Agetoros doordrongen}{van deze waarheid}\\

\subsection{Uit: Verzamelde romans. Deel 48. De h\^otelier doet niet meer mee}

\haiku{Hij bleef recht voor zich,.}{uitkijken en strompelde}{met zijn stok verder}\\

\haiku{- was de enige die.}{ronduit blijk scheen te geven}{van het tegendeel}\\

\haiku{Niets is zo goed voor,.}{het optreden de maintien}{van jonge mensen}\\

\haiku{{\textquoteright} {\textquoteleft}Dat is hij juist n{\'\i}et,{\textquoteright}, {\textquoteleft}...}{glimlachte ikmaar hij kan}{best gezegd hebben}\\

\haiku{jij kent hem beter,.}{dan ik jij zult het misschien}{geen leugens noemen}\\

\haiku{Maar ik denk nu aan,.}{de opstand van Didier van}{vier jaar geleden}\\

\haiku{Ik ben zelfs bereid,.}{te verklaren dat ik mij}{vergist kan hebben}\\

\haiku{{\textquoteleft}Monsieur Trublet,.}{had mij niets opgedragen}{Mademoiselle}\\

\haiku{Een verklaring van {\textquoteleft}{\textquoteright};}{mijnspionnage kon ik}{hem moeilijk geven}\\

\haiku{{\textquoteleft}Ik zal Lamoignon,.}{nu naar u toesturen wij}{blijven nog maar kort}\\

\haiku{twee ruimten op de,:}{tweede verdieping die als}{kantoor dienst deden}\\

\haiku{ik had geen lust om;}{mijn nieuwsgierigheid aan de}{kaak gesteld te zien}\\

\haiku{Had ik er flink wat,:}{bij elkaar dan kon ik hem}{ermee overvallen}\\

\haiku{maar de haastige:}{beweging van  haar hand}{was mij niet ontgaan}\\

\haiku{Daarentegen keek,:}{hij terstond op toen zij een}{half uur later vroeg}\\

\haiku{hij is overigens,...}{in het geheel geen graaf ik}{had ge{\"\i}nformeerd}\\

\haiku{Maar dat is niet zo,,.}{Duitsertje het is een veel}{belangrijker zaak}\\

\haiku{ook kon Marie dit,.}{all\'e\'en hebben gedaan of}{Andilly alleen}\\

\haiku{Hij zat daar altijd,.}{om deze tijd wanneer hij}{niet uitgegaan was}\\

\haiku{{\textquoteleft}Eruit,{\textquoteright} waarop de,.}{Zuidfransman naar de gang liep}{en zo naar de trap}\\

\haiku{De keizer moest niets,.}{van hen hebben hij heeft er}{ook heel wat gestraft}\\

\haiku{Ik zou bijvoorbeeld...}{niets ten nadele van uw}{oom willen zeggen}\\

\haiku{U geniet hier de......{\textquoteright} {\textquoteleft}}{gastvrijheid van de Franse}{staat van de koning}\\

\haiku{{\textquoteright} {\textquoteleft}Monsieur Trublet.}{verleent onderdak aan zijn}{eigen reizigers}\\

\haiku{{\textquoteright} {\textquoteleft}Wanneer neemt u een?}{Bonapartistische}{opstand w\'el ernstig}\\

\haiku{zo ja, dan was er.}{sprake van plaatselijke}{eigengereidheid}\\

\haiku{Mademoiselle:}{Nathalie veroorloofde}{zich de opmerking}\\

\haiku{Ik was nu niet bang,,}{meer en ik zei dat beren}{niet van gezouten}\\

\haiku{en ze wisten, dat.}{ik mijn pistool momenteel}{niet gebruiken kon}\\

\haiku{{\textquoteright} {\textquoteleft}Wij hebben \'o\'ok last,{\textquoteright}.}{van het spit werd er uit de}{ingang geroepen}\\

\haiku{{\textquoteright} vroeg Mionnet, met.}{een bedenkelijke plooi}{in zijn hoog voorhoofd}\\

\haiku{Het kan zijn, dat zij,,.}{met hun dialect de Franse}{taal slecht verstonden}\\

\haiku{Ik kreeg de indruk,.}{dat hij zich niet meer tegen}{mij zou verzetten}\\

\haiku{Met het pistool in.}{de hand stond hij reeds bij de}{pakken en zakken}\\

\haiku{Als je honger hebt,,.}{ga je maar werken bij de}{fortificaties}\\

\haiku{Ik bedoel dus niet.}{zozeer je meningen als}{Erlanger student}\\

\haiku{Verder had ik als;}{alibi het huwelijk van}{mijn oudste dochter}\\

\haiku{En dat zette zich.}{voort in alle latere}{gebeurtenissen}\\

\haiku{{\textquoteleft}Maar als verrader...{\textquoteright} {\textquoteleft}.}{Een impulsief man als Ney}{{\'\i}s geen verrader}\\

\haiku{Hij zal dankbaar zijn.}{in Frankrijk te vertoeven}{z\'onder politiek}\\

\haiku{Daarin was maar heel,.}{weinig mislukt en stoute}{stukjes in overvloed}\\

\haiku{{\textquoteleft}Mijn gevoelens voor.}{de keizer kunt u buiten}{beschouwing laten}\\

\haiku{En er is nu geen.}{generaal Ney meer om naar}{hen over te lopen}\\

\haiku{Ik ben niet meer dan.}{de geldschieter en de chef}{de r\'eception}\\

\haiku{Het is zelfs de vraag,...}{of men mij bij mislukking}{zal kunnen straffen}\\

\haiku{Dumoulin ook,.}{over hem hebben we meen ik}{al eens gesproken}\\

\haiku{Hiermee wil ik niet,;}{zeggen dat hij weer dingen}{voor mij geheim hield}\\

\haiku{geld geven, en zijn.}{nest in de Alpen voor de}{adelaar inrichten}\\

\haiku{Uw dochters kunnen...}{heel goed eens de handen uit}{de mouwen steken}\\

\haiku{Zwaar, dat kan ik niet...{\textquoteright} {\textquoteleft}.}{zeggenOmdat zij in die}{trommel was gezakt}\\

\haiku{Natuurlijk liep hij,,.}{met al die wapens naar de}{vrijgekomen strook}\\

\haiku{Maar de strook bleek niet,?}{geheel vrij te zijn want wat}{zag hij daar opeens}\\

\haiku{Ik was dan wel een,,}{Duitser maar ook zo goed als}{een Elzasser etc.}\\

\haiku{{\textquoteleft}K\'an Johnston het,,?}{niet of wil hij niet of krijgt}{hij niet genoeg geld}\\

\haiku{Twistgesprekken als;}{tussen Mionnet en mij}{kwamen niet meer voor}\\

\haiku{Mijn voorstel om de.}{taak van hem over te nemen}{werd afgeslagen}\\

\haiku{verder dan dat scheen.}{men het in Amerika niet}{gebracht te hebben}\\

\haiku{zij konden heel goed:}{voor grapjes doorgaan om de}{tijd te verdrijven}\\

\haiku{Hij was al weer twee,.}{keer bij hen geweest zonder}{naar mij te vragen}\\

\haiku{om opschudding te.}{voorkomen was het bericht}{achtergehouden}\\

\haiku{Monsieur Trublet,.}{wist ervan maar sloot zich op}{in zijn kantoor}\\

\haiku{{\textquoteright} Johnston bedacht,.}{de vragen zelf al stond hij}{open voor suggesties}\\

\haiku{{\textquoteright} Het woord {\textquoteleft}zigeuners{\textquoteright} {\textquoteleft}{\textquoteright},.}{was vervangen doorgypsies}{het Engelse woord}\\

\haiku{Nadat zijn taak hem,:}{weer had opge\"eist kwam het}{antwoord vlug genoeg}\\

\haiku{Johnston had het,,.}{maar dan geheel toevallig}{bij het rechte eind}\\

\haiku{{\textquoteleft}Geloofde hij in?}{dit nieuwe bericht over de}{dood van de keizer}\\

\haiku{Monsieur Trublets{\textquoteright}.}{gastvrijheid kent u. Allen}{knikten instemmend}\\

\haiku{Hij moest toch heel goed,.}{weten dat de heren niet}{d\'urfden uit te gaan}\\

\haiku{{\textquoteright} Met deze woorden,.}{liep hij naar een stoel waarin}{hij zich liet vallen}\\

\haiku{Hij sloeg de handen,.}{voor het gezicht en barstte}{in snikken uit}\\

\subsection{Uit: Verzamelde romans. Deel 52. Het proces van meester Eckhart}

\haiku{Zij werden meer op,}{de vingers getikt konden}{minder ongemoeid}\\

\haiku{En als ik het wist,.}{was het maar de vraag of ik}{het zou begrijpen}\\

\haiku{Maar misschien is men,.}{in de war misschien vliegt de}{Duivel naar hem toe}\\

\haiku{Het wachten is nu.}{allereerst op de stappen}{van de aartsbisschop}\\

\haiku{{\textquoteright} Deze overweging,:}{scheen hem iets hoopvoller te}{stemmen en hij zei}\\

\haiku{{\textquoteleft}Mijn bezoek hier op.}{het Domplein kan ik u}{beter toelichten}\\

\haiku{Hij las zonder bril,.}{en met het papier niet al}{te dicht bij de ogen}\\

\haiku{En niemand die het,.}{ooit zag aankomen zelfs op}{de laatste dag niet}\\

\haiku{{\textquoteright} {\textquoteleft}Misschien is er in,.}{zijn ziel een kleine uithoek}{die erop rekent}\\

\haiku{Veeleer duidden zij.}{op een afwachten van het}{gunstigste moment}\\

\haiku{Vermoedelijk had.}{hij haast om alles achter}{de rug te hebben}\\

\haiku{Reyner heeft zes,,.}{klerken voor verschillende}{zaken en zaakjes}\\

\haiku{Mijn zonden, meende,}{ik lagen op het terrein}{der formulering}\\

\haiku{Op dit schreien keek.}{Konrad van Halberstadt voor}{de tweede maal op}\\

\haiku{In de eerste plaats,,.}{zoals u wel begrijpen}{zult de herroeping}\\

\haiku{Ik heb alleen met,.}{de Paus te maken en ik}{ga naar de Paus toe}\\

\subsection{Uit: Het vijfde zegel}

\haiku{Niemand had er aan.}{gedacht hem van ketterij}{te beschuldigen}\\

\haiku{De aanvoerder, wiens,.}{neus afgeschoten was sprak}{met een licht accent}\\

\haiku{Nu het paleis zo,.}{dichtbij was durfde hij er}{wel naar te kijken}\\

\haiku{{\textquoteleft}Exeat aula, qui vult{\textquoteright}, -:}{esse pius aangevuld}{nog door Ovidius}\\

\haiku{Het leek de Griek niet,:}{de onaangenaamste van}{de vijf die dit zei}\\

\haiku{{\textquoteright} {\textquoteleft}De kleuren zijn te,!}{licht en te vrolijk er is}{te veel blauw en geel}\\

\haiku{De moeilijkheden;}{waren niet zo groot geweest}{als hij verwacht had}\\

\haiku{een overredende,.}{woordenstroom zonder enige}{belemmering nu}\\

\haiku{er zeker van kon,.}{zijn dat de wagen met het}{schilderij klaar stond}\\

\haiku{Vijf dagen later.}{kwam een brief uit Aranjuez}{zekerheid brengen}\\

\haiku{Nog verwonderde,.}{het hem dat ze zich later}{nooit gewroken had}\\

\haiku{In de namiddag}{van de 16e Mei verhief hij}{zich uit die vadsig}\\

\haiku{Een ontmoeting, zeer,;}{toevallig met de vrouw met}{de kanthanden}\\

\haiku{, en onmiddellijk,.}{schoof zich iets achter hem open}{onhoorbaar glijdend}\\

\haiku{Want ook de botste uit:}{het gezelschap besefte}{waar het hier om ging}\\

\haiku{en Taddeus, dat,!...}{was Scorpio de vinger in}{eigen borst borend}\\

\haiku{Het spijt mij, dat ik, ...{\textquoteright} {\textquoteleft}?!}{u stoor zo laatMaar hoe komt}{u in Toledo}\\

\haiku{{\textquoteright} - De man moest Greco's.}{vraag van de vorige avond}{onthouden hebben}\\

\haiku{Je zult zien, Juan,,,!}{hij jaagt hem weg hij ontvangt}{hem niet de trotsaard}\\

\haiku{{\textquoteleft}Jij Preboste zorgt,.}{ervoor dat ik het eerste}{uur niet gestoord word}\\

\haiku{Hij gaf Miguel.}{instructies en volgde even}{later met de gast}\\

\haiku{(Het was kenmerkend,.}{voor zijn openluchtnatuur dat}{hij de deur oversloeg}\\

\haiku{{\textquoteleft}Ik wens u geluk,!}{met de verjaardag van de}{koning Don Martin}\\

\haiku{het huis is tegen.}{de heuvel opgebouwd door}{die oude Moren}\\

\haiku{De laatste tijd denk.}{ik meer aan Creta terug}{dan ooit te voren}\\

\haiku{Deze achtertrap,.}{liep langs provisiekasten}{die Greco afsloot}\\

\haiku{{\textquoteright} {\textquoteleft}Dat wist ik niet,{\textquoteright} zei,.}{Esquerrer de schilder enigszins}{bezorgd aankijkend}\\

\haiku{Eerst op het plein voor.}{het stadhuis wachtte hem een}{groter machtsvertoon}\\

\haiku{{\textquoteleft}Noem hem geen Jood, of.}{het Heilig Officium}{bemoeit er zich mee}\\

\haiku{Het bestaat evenmin!}{als tussen Maria en een}{andere Maria}\\

\haiku{zolang men jullie,!}{kan uitschilderen gebeurt}{er niets van die aard}\\

\haiku{Een opgejaagde.}{vleermuis beschreef trillende}{kringen en verdween}\\

\haiku{Tot alles zou hij.}{bereid zijn geweest in ruil}{voor m\'e\'er verhalen}\\

\haiku{en ieder huis, dat,.}{niet op de gangen uitkwam}{was een muizenval}\\

\haiku{In deze maanden.}{had hij meer geschilderd dan}{anders in jaren}\\

\haiku{Hij nam zich voor haar.}{steeds te blijven eren om haar}{onderworpenheid}\\

\haiku{Er was geen sprake,}{van verdienste laat staan van}{opofferingen}\\

\haiku{Onder de vrienden}{die hij zich in Rome had}{gemaakt was er \'e\'en}\\

\haiku{men zocht dagen uit,;}{dat de anderen ziek of}{verhinderd waren}\\

\haiku{Die hij misschien zal!}{mogen helpen martelen}{in een zwembroekje}\\

\haiku{Greco ving dingen;}{op die zijn factotum hem}{steeds verzwegen had}\\

\haiku{Dan kwam hij met een,;}{kreet te voorschijn en het spel}{was afgelopen}\\

\haiku{tussen zijn lippen.}{vertoonde zich het puntje}{van een bleke tong}\\

\haiku{{\textquoteleft}Ik ben te bang voor, ...{\textquoteright} {\textquoteleft},}{hem en ook dat jijBijten}{gaat z\'eker te ver}\\

\haiku{{\textquoteright} {\textquoteleft}Dat is ver gezocht,{\textquoteright}, {\textquoteleft}?}{zei Grecowat heeft Ayala}{daarmee te maken}\\

\haiku{in het atelier, waar,.}{Greco ze bewaarde was}{in lang niet gestookt}\\

\haiku{In het donker, want, ...}{ook mijn lamp was uitgegaan}{was hij onvindbaar}\\

\haiku{Maar wat hij nu doen.}{ging was zo mogelijk nog}{raadselachtiger}\\

\haiku{Miguel wachtte.}{met iets te zeggen tot zijn}{meester voor hem stond}\\

\haiku{{\textquoteright} {\textquoteleft}Ik meende, dat ik,!}{hier als getuige zat niet}{als beschuldigde}\\

\haiku{Ik diletteer in,.}{kunsttheorie\"en en u}{theologiseert}\\

\haiku{Midden in de nacht,.}{zocht hij Ger\'onima op voor}{het eerst na maanden}\\

\haiku{Slechts aarzelend was,.}{men begonnen men durfde}{niet goed voor elkaar}\\

\haiku{vrouwen tilden hun.}{kinderen op om te zien}{wat er gebeurde}\\

\haiku{Hoe weinig misbruik,!}{had Esquerrer daarvan gemaakt}{van  dit laatste}\\

\subsection{Uit: Verzamelde romans. Deel 5. Het vijfde zegel}

\haiku{hij liep, begerig.}{toe om  het berouw in}{ontvangst te nemen}\\

\haiku{Niemand had er aan.}{gedacht hem van ketterij}{te beschuldigen}\\

\haiku{De aanvoerder, wiens,.}{neus afgeschoten was sprak}{met een licht accent}\\

\haiku{Nu het paleis zo,.}{dichtbij was durfde hij er}{wel naar te kijken}\\

\haiku{{\textquoteleft}Exeat aula, qui vult{\textquoteright}, -:}{esse pius aangevuld}{nog door Ovidius}\\

\haiku{Het leek de Griek niet,:}{de onaangenaamste van}{de vijf die dit zei}\\

\haiku{{\textquoteright} {\textquoteleft}De kleuren zijn te,!}{licht en te vrolijk er is}{te veel blauw en geel}\\

\haiku{De moeilijkheden;}{waren niet zo groot geweest}{als hij verwacht had}\\

\haiku{een overredende,.}{woordenstroom zonder enige}{belemmering nu}\\

\haiku{er zeker van kon,.}{zijn dat de wagen met het}{schilderij klaar stond}\\

\haiku{{\textquoteleft}...een tijd en tijden,...}{en een halve tijd buiten}{het gezicht der slang}\\

\haiku{Nog verwonderde,.}{het hem dat ze zich later}{nooit gewroken had}\\

\haiku{Een ontmoeting, zeer,;}{toevallig met de vrouw met}{de kanthanden}\\

\haiku{, en onmiddellijk,.}{schoof zich iets achter hem open}{onhoorbaar glijdend}\\

\haiku{Want ook de botste uit:}{het gezelschap besefte}{waar het hier om ging}\\

\haiku{en Taddeus, dat,!...}{was Scorpio de vinger in}{eigen borst borend}\\

\haiku{Het spijt mij, dat ik,...{\textquoteright} {\textquoteleft}?!}{u stoor zo laatMaar hoe komt}{u in Toledo}\\

\haiku{{\textquoteright} - De man moest Greco's.}{vraag van de vorige avond}{onthouden hebben}\\

\haiku{Je zult zien, Juan,,,!}{hij jaagt hem weg hij ontvangt}{hem niet de trotsaard}\\

\haiku{{\textquoteleft}Jij Preboste zorgt,.}{ervoor dat ik het eerste}{uur niet gestoord word}\\

\haiku{Hij gaf Miguel.}{instructies en volgde even}{later met de gast}\\

\haiku{(Het was kenmerkend,.}{voor zijn openluchtnatuur dat}{hij de deur oversloeg}\\

\haiku{het huis is tegen.}{de heuvel opgebouwd door}{die oude Moren}\\

\haiku{De laatste tijd denk.}{ik meer aan Creta terug}{dan ooit te voren}\\

\haiku{Deze achtertrap,.}{liep langs provisiekasten}{die Greco afsloot}\\

\haiku{{\textquoteright} {\textquoteleft}Dat wist ik niet,{\textquoteright} zei,.}{Esquerrer de schilder enigszins}{bezorgd aankijkend}\\

\haiku{Eerst op het plein voor.}{het stadhuis wachtte hem een}{groter machtsvertoon}\\

\haiku{{\textquoteleft}Noem hem geen Jood, of.}{het Heilig Officium}{bemoeit er zich mee}\\

\haiku{Het bestaat evenmin!}{als tussen Maria en een}{andere Maria}\\

\haiku{zolang men jullie,!}{kan uitschilderen gebeurt}{er niets van dien aard}\\

\haiku{Een opgejaagde.}{vleermuis beschreef trillende}{kringen en verdween}\\

\haiku{Tot alles zou hij.}{bereid zijn geweest in ruil}{voor m\'e\'er verhalen}\\

\haiku{een ieder huis, dat,.}{niet op de gangen uitkwam}{was een muizeval}\\

\haiku{In deze maanden.}{had hij meer geschilderd dan}{anders in jaren}\\

\haiku{Hij nam zich voor haar.}{steeds te blijven eren om haar}{onderworpenheid}\\

\haiku{Er was geen sprake,}{van verdienste laat staan van}{opofferingen}\\

\haiku{de beoefening;}{ervan gereglementeerd}{door de nieuwe Paus}\\

\haiku{Onder de vrienden}{die hij zich in Rome had}{gemaakt was er \'e\'en}\\

\haiku{Veel was er voor een,.}{ontbinding te zeggen er}{tegen sprak ook veel}\\

\haiku{men zocht dagen uit,;}{dat de anderen ziek of}{verhinderd waren}\\

\haiku{Die hij misschien zal!}{mogen helpen martelen}{in een zwembroekje}\\

\haiku{Greco ving dingen;}{op die zijn factotum hem}{steeds verzwegen had}\\

\haiku{Dan kwam hij met een,;}{kreet te voorschijn en het spel}{was afgelopen}\\

\haiku{tussen zijn lippen.}{vertoonde zich het puntje}{van een bleke tong}\\

\haiku{{\textquoteleft}Ik ben te bang voor,...{\textquoteright} {\textquoteleft},}{hem en ook dat jijBijten}{gaat z\'eker te ver}\\

\haiku{{\textquoteright} {\textquoteleft}Dat is ver gezocht,{\textquoteright}, {\textquoteleft}?}{zei Grecowat heeft Ayala}{daarmee te maken}\\

\haiku{in het atelier, waar,.}{Greco ze bewaarde was}{in lang niet gestookt}\\

\haiku{In het donker, want,...}{ook mijn lamp was uitgegaan}{was hij onvindbaar}\\

\haiku{Maar wat hij nu doen.}{ging was zo mogelijk nog}{raadselachtiger}\\

\haiku{Miguel wachtte.}{met iets te zeggen tot zijn}{meester voor hem stond}\\

\haiku{{\textquoteright} {\textquoteleft}Ik meende, dat ik,!}{hier als getuige zat niet}{als beschuldigde}\\

\haiku{Diego was de enige,.}{die hem die Maandagochtend}{het vlees had zien eten}\\

\haiku{Ik diletteer in,.}{kunsttheorie\"en en u}{theologiseert}\\

\haiku{Midden in de nacht,.}{zocht hij Ger\'onima op voor}{het eerst na maanden}\\

\haiku{Slechts aarzelend was,.}{men begonnen men durfde}{niet goed voor elkaar}\\

\haiku{vrouwen tilden hun.}{kinderen op om te zien}{wat er gebeurde}\\

\haiku{16 Sola fides -.}{justificat Alleen het}{geloof rechtvaardigt}\\

\haiku{41 Patio de los -.}{Evangelistas Binnenplaats}{der Evangelisten}\\

\haiku{120 Index Librorum -.}{Prohibitorum Lijst van}{verboden boeken}\\

\haiku{226 Compa\~n{\'\i}a de -,.}{Jes\'us Leger van Jezus}{de Jezu{\"\i}eten}\\

\haiku{256 Convento de -.}{la Concepci\'on Klooster van}{de Ontvangenis}\\

\haiku{258 Panthe{\"\i}sme -,.}{De leer dat God alles is}{zonder onderscheid}\\

\haiku{304 Generale -.}{biecht Biecht van alle zonden}{en overtredingen}\\

\haiku{317 Virgen de la -.}{Misericordia Maagd van}{de Barmhartigheid}\\

\subsection{Uit: Verzamelde romans. Deel 22. De vijf roeiers}

\haiku{Het spiegeltje had,.}{hij al eerder van de hand}{gedaan gewaagd vroeg}\\

\haiku{En het meisje was,.}{bleek en droefgeestig dus zat}{het ook in het hart}\\

\haiku{{\textquoteright} - Scherp lette hij op.}{de gelaatsuitdrukking van}{de seminarist}\\

\haiku{Maar die is \'ook van,.}{de overkant en hij heeft een}{hard leven geleid}\\

\haiku{met die voetjes, en die,.}{stenen hals met barsten en}{het losse hoofdje}\\

\haiku{{\textquoteright} - Geleidelijk aan,,.}{half buiten zijn wil had hij}{het hoofd gebogen}\\

\haiku{Een koopman, wie ik,.}{geld heb gegeven omdat}{hij slecht gekleed was}\\

\haiku{Hij zal je vragen.}{waarom je mijn bezorgdheid}{niet wegnemen wilt}\\

\haiku{{\textquoteleft}Ik heb nooit iets bij.}{je gemerkt van voorkeur voor}{die rechtenstudie}\\

\haiku{Hoe donker was het,.}{stadje wanneer men er van}{bovenaf inkeek}\\

\haiku{Schone donkere,,.}{kleren een schoon hemd en zijn}{haar zonder schubben}\\

\haiku{{\textquoteleft}Ik dank je, lieve,.}{Moyna voor die bloeiende}{brem op mijn kussen}\\

\haiku{Een man van eer kon.}{niet doodgaan in een plaatsje}{als Lomanagh}\\

\haiku{{\textquoteright} {\textquoteleft}En ik wil hier weg,{\textquoteright},:}{hield de oude O'Flanagan aan}{en tegen Maurice}\\

\haiku{jou, dan zou ik van,.}{ze verwachten dat ze mij}{de strot afsneden}\\

\haiku{{\textquoteright} Op dit oordeel was.}{de marskramer volledig}{voorbereid geweest}\\

\haiku{Pat praatte genoeg,?}{tegen hem en zou hij dit}{verzwegen hebben}\\

\haiku{zelfs een gesprek met.}{Vader Kearny had hem}{niet wijzer gemaakt}\\

\haiku{Van die tijd stamde,,.}{zijn angst voor aanrakingen}{strelingen blikken}\\

\haiku{De twee meisjes, die,.}{hem het leven vergalden}{waren althans stil}\\

\haiku{met zijn neus, lippen,,,...}{en tong met zijn voedsel met}{de regen de zee}\\

\haiku{{\textquoteleft}Innisbavan,{\textquoteright} zei,,}{Shaun met een verlegen blik}{op John van wie}\\

\haiku{Hij begreep, dat de.}{late wandelaar het erf}{was opgelopen}\\

\haiku{Houdt u een oogje...{\textquoteright} -.}{in het zeil Hij hikte en}{zijn gezicht vertrok}\\

\haiku{Zijn vaders hand was,.}{de laatste geweest die hulp}{had kunnen bieden}\\

\haiku{En meteen was er.}{iets van de feestelijkheid}{verloren gegaan}\\

\haiku{Ik werk hier harder.}{dan iemand zo vet als jij}{zich kan voorstellen}\\

\haiku{Hij zag de nacht als.}{een door zingende dieren}{belaagd paradijs}\\

\haiku{De Fenians aan de,:}{overkant loeren op hem en}{ik houd ze tegen}\\

\haiku{Maar, bij God, ik doe!}{wat ik doe en wat de stem}{van God mij ingeeft}\\

\haiku{Gebiedend snauwde:}{hij de anderen toe zijn}{voorbeeld te volgen}\\

\haiku{Hij had een beetje,.}{medelijden met Moyna}{en hij hield van haar}\\

\haiku{En dan waren ze,.}{nog een beetje dronken ook}{behalve Moyna}\\

\haiku{ik haal ze uit de,.}{zee de seminaria en}{de paardestallen}\\

\haiku{Hoewel, als ik die...}{Eileen van jullie eens}{onder het mes nam}\\

\haiku{Dat kon van boosheid,;}{zijn Maurice's  laatste uur}{kon geslagen zijn}\\

\haiku{{\textquoteleft}Jongens, als het hard,.}{gaat waaien vergeet dan de}{arme Moyna niet}\\

\haiku{Wat haar bijzonder.}{hinderde was dat zij zich}{driftig had gemaakt}\\

\haiku{Een tijdlang stond hij.}{te luisteren naar een ver}{verwijderd gesnurk}\\

\haiku{in het testament,,}{stond alles zei hij mooier}{dan zelfs de pastoor}\\

\haiku{Wie heeft toen een van?}{de beulen een steen onder}{de neus gehouden}\\

\haiku{Een van de beste,,.}{zonen van Erin heren}{en een hart van goud}\\

\haiku{Jullie weet m\'eer van,.}{die vent of \'een van jullie}{weet meer van die vent}\\

\haiku{Ik,{\textquoteright} zei John, {\textquoteleft}en,.}{ik kreeg de indruk dat hij}{een Fenian was}\\

\haiku{Dit is overigens.}{niets vergeleken bij dat}{draaiertje van mij}\\

\haiku{{\textquoteright} {\textquoteleft}Als je er meer hebt,,.}{dan betekent dat dat je}{ons hebt bedrogen}\\

\haiku{De beweging had.}{hij volvoerd alsof het zijn}{dagelijks werk was}\\

\haiku{Shaun slaakte een gil,;}{en kromp hijgend ineen voor}{wat er volgen kon}\\

\haiku{een stille vrouw, wier,.}{ogen hij had ge\"erfd en de}{pijn gold zijn moeder}\\

\haiku{hij verwachtte het,,,.}{tweede houtje diep in zijn}{bloed zijn vlees zijn ziel}\\

\haiku{{\textquoteright} {\textquoteleft}Wij allemaal...{\textquoteright} {\textquoteleft}Kom,,,?}{Shaun je zei toch dat Cork een}{meid in het hoofd had}\\

\haiku{Hier was tenminste,...}{een indruk zoals Jimmy}{van hem verlangde}\\

\haiku{Was Conic soms al,?}{dood door die vreselijke}{slagen op zijn hoofd}\\

\haiku{{\textquoteright} riep John, die met, {\textquoteleft}.}{Shaun gearmd liepik stel voor}{wat aan te plakken}\\

\haiku{Het greep bij de keel,,.}{het gaf wel kracht maar men werd}{er ook doodsbang van}\\

\haiku{er kon immers van;}{alles gebeuren in een}{bos bij volle maan}\\

\haiku{Zijn stem klonk nog zwak,,:}{maar sarcastich genoeg toen}{hij naar voren riep}\\

\haiku{{\textquoteleft}H\'e, jullie eksters,?}{daar gaan we nu nog ruiten}{insmijten of niet}\\

\haiku{Een lichtende vorm.}{schoot omhoog en verdween in}{het gebladerte}\\

\haiku{Moet ik soms weken,?...}{en maanden tegen planken}{aankijken bij God}\\

\haiku{Aan het begin van.}{het voorterras zie je de}{baai voor je liggen}\\

\haiku{Snauwend gelastte.}{Keane de andere}{vier hem te volgen}\\

\haiku{Dat maakt een slechte,}{indruk en bij Mr. Coyne}{sta je in de pas.}\\

\haiku{{\textquoteright} {\textquoteleft}Paramentiek is,{\textquoteright}, {\textquoteleft}}{nooit mijn sterkste punt geweest}{zei John ironisch}\\

\haiku{die wisten, dat wij,.}{er waren en ze hadden}{hun wapens niet meer}\\

\haiku{Ja, de Fransen zijn,.}{geen dom volk zoals ik ze}{heb leren kennen}\\

\haiku{Maar beste jongen,.}{ik ben nog niet anders dan}{kwaad op je geweest}\\

\haiku{Wie waarborgde, dat,?}{hij de bruiloft overleven}{zou dat alleen al}\\

\haiku{Het is beroerd voor,,.}{je maar je mag me doodslaan}{als het niet waar is}\\

\haiku{{\textquoteright} {\textquoteleft}Keane...{\textquoteright} {\textquoteleft}Als je,.}{langer gebleven was had}{je het zelf gezien}\\

\haiku{d\'at juist maakte het.}{geluid zo verschrikkelijk}{om aan te horen}\\

\haiku{{\textquoteright} {\textquoteleft}Ik denk niet meer aan,{\textquoteright}, {\textquoteleft}:}{die teef zei Conic koelen}{h{\'\i}j heeft zijn portie}\\

\haiku{jouw moeder heeft er,.}{bepaald niet voor gezorgd dat}{jij de grote kreeg}\\

\haiku{Ze was toen nog te.}{jong om overlast te hebben}{van de soldaten}\\

\haiku{{\textquoteright} Keane dacht na. - {\textquoteleft}?}{En geen moeite gedaan om}{hem te ontzetten}\\

\haiku{{\textquoteright} {\textquoteleft}Waar ik u voor houd,{\textquoteright}, {\textquoteleft}.}{zei Keaneis van niet}{het minste belang}\\

\haiku{en dat jullie me,.}{niet gezegd hebben dat er}{een moord was gepleegd}\\

\haiku{Ze hadden zwarte,.}{maskers voor de moordenaars}{van Mr. Coyne ook}\\

\haiku{Pat liep naar het raam.}{om de spijker nog eens in}{ogenschouw te nemen}\\

\haiku{Moyna Donovan,.}{kan getuigen dat wij niet}{op moord uit waren}\\

\haiku{maar ik wil alleen,...}{maar zeggen dat God er niet}{om verlegen zit}\\

\haiku{Onder de plak van.}{zo'n schreeuwlelijk kun je nooit}{een kerel worden}\\

\haiku{hij schudde een paar,.}{maal het hoofd alsof hij het}{beter wist dan zij}\\

\haiku{Jullie moet me niet,,.}{zo aanstaren jongens ik}{ben ook maar een mens}\\

\haiku{Van het begin af.}{aan was zij zeer scherp tegen}{hem opgetreden}\\

\haiku{Toen hij zich moeizaam,.}{omdraaide lag de zwerver}{al weer op zijn rug}\\

\haiku{Slaan vernietigt het,.}{moreel schijnt een groot denker}{gezegd te hebben}\\

\haiku{nu zal ze wel hees,,;}{zijn van de hoerenziekte}{dat geeft vaak heesheid}\\

\haiku{Maar dat heb ik nooit,.}{gedaan ik ben het ook nooit}{echt van plan geweest}\\

\haiku{ze namen me voor,...}{\'een dag zoals de schooiers}{Molly voor \'een nacht}\\

\haiku{Uit de mist kwamen.}{meedogenloos schreeuwend de}{meeuwen aangevlerkt}\\

\haiku{Keane boeit ons,,.}{niet omdat hij weet dat we}{niet zullen vluchten}\\

\haiku{Toen zocht hij naar ogen,.}{die strak of dreigend op hem}{gericht konden zijn}\\

\haiku{- {\textquoteleft}Eerst moet aan het licht.}{worden gebracht wie de schuld}{dragen aan zijn dood}\\

\haiku{{\textquoteleft}Mr. Coyne is een,,.}{gentleman jongen hij laat}{het lijk aan ons over}\\

\haiku{Wanneer hij nu eens.}{niet de kracht had n{\'\i}et tegen}{Moyna te lachen}\\

\haiku{Vader heeft zich nog,,?}{boos gemaakt om mijnentwil}{maar wat kon hij doen}\\

\subsection{Uit: Verzamelde romans. Deel 11. De zwarte ruiter}

\haiku{De vrouw, die met een,;}{haakwerkje bezig was zag}{ik van achteren}\\

\haiku{{\textquoteleft}Wanneer u graag rookt,.}{weest u dan voorzichtig met}{vuur in de bossen}\\

\haiku{Inmiddels hadden,,.}{de vrienden dorstend naar wraak}{de brug opgehaald}\\

\haiku{Hij had een lange,,.}{gebogen neus en sterke}{lijnen om de mond}\\

\haiku{de dochter van de....}{bewoner van Ruiterstein}{alles symboliek}\\

\haiku{Onder geen beding.}{zou ik in een verzoening}{hebben toegestemd}\\

\haiku{Haar ogen vreesde ik,;}{nu maar die ogen waren niets}{zonder haar mankheid}\\

\haiku{En toch, hij mocht groot,.}{zijn terzelfder tijd was hij}{nog niet groot genoeg}\\

\haiku{We zouden elkaar,,}{eens kunnen ontmoeten waar}{mijn man niet bij is}\\

\haiku{{\textquoteright} Na even nagedacht,.}{te hebben knikte zij maar}{verroerde zich niet}\\

\haiku{Het hoeft maar heel klein,, -!}{te zijn \'e\'en vlam maar ter ere}{van onze vriendschap}\\

\haiku{ze wilde vuurwerk,,:}{en ze had het en ze gaf}{het niet uit handen}\\

\haiku{Ik stond op van de.}{plaats waar ik op de heide}{neergezonken was}\\

\haiku{Ik was al niet meer,.}{overtuigd dat ik Digna}{Raecke gered had}\\

\section{Cornelis Veth}

\subsection{Uit: Prikkel-idyllen. Deel 1}

\haiku{Een bezoeker werd,.}{aangediend en dadelijk}{binnengelaten}\\

\haiku{Laat de avonturen,,.}{nieuw zijn spannend en laat het}{er dik op liggen}\\

\haiku{Nu d{\`\i}e zal ik wel,.}{eens lezen als ik eens niets}{beters te doen heb}\\

\haiku{De heer Dainty gaf.}{de zaak onmiddellijk bij}{de politie aan}\\

\haiku{Broadstreet stelde zich!}{nu in verbinding met den}{beul van Nantes}\\

\haiku{{\textquoteleft}Maar dat is bijna!}{precies de advertentie}{der roodharigen}\\

\haiku{ik begreep, dat ik.}{de andere dame in}{het oog moest houden}\\

\haiku{{\textquoteleft}Mr. Noppes,{\textquoteright} zei Sir,.}{Sherlock Holmes op onzen}{client toetredende}\\

\haiku{Een nihilistisch,!}{complot nu behoef ik u}{niets te vertellen}\\

\haiku{De naden deden.}{hem telkens pijn in dezen}{nerveuzen toestand}\\

\haiku{Toen, voor wij er op,,.}{verdacht waren keerde hij}{zich om en vluchtte}\\

\haiku{{\textquoteleft}Lord Cookerville,{\textquoteright}, {\textquoteleft}.}{sprak hijdeze zaak is van}{uiterst kieschen aard}\\

\haiku{Ik onderzocht nu.}{de materie die aan de}{couverten kleefde}\\

\haiku{{\textquoteright} zei Harry Wilson,.}{bij zich zelf terwijl hij zijn}{revolvers laadde}\\

\haiku{Gij hebt te kiezen!}{tusschen den dood door vergif}{en dien door den strop}\\

\haiku{Op de tafel v\`o\`or,....}{dit vreemde gezelschap lag}{in een lauwerkrans}\\

\haiku{Ik kies dus....{\textquoteright} Allen.}{wachtten nieuwsgierig op zijn}{verdere woorden}\\

\haiku{Ik dank u.{\textquoteright} {\textquoteleft}Maar er!}{moet toch een einde komen}{aan dezen toestand}\\

\haiku{Ten eerste werpt gij....}{mij den sleutel toe van de}{deur achter  mij}\\

\haiku{A propos, als je,!}{nog eens op karwei gaat neem}{dan een revolver}\\

\haiku{Hij herkende de,.}{stem van den man met wien hij}{juist had gesproken}\\

\haiku{Dit portret maakte.}{in dit ouderwetsch vertrek}{een vreemden indruk}\\

\haiku{Converseert in drie,,!}{talen kan lezen schrijven}{en rekenen}\\

\haiku{Waar de gaatjes in,!}{het portret toe dienden was}{eveneens duidelijk}\\

\haiku{Nog slechts twee van de!}{huilende scalp-dieven}{waren in leven}\\

\haiku{{\textquoteright} {\textquoteleft}Graaf van Zwartburg, mijn!}{boodschap betreft uw titel}{en eigendommen}\\

\haiku{Daar rollen vele,....}{juweelen benevens een}{papier over den vloer}\\

\haiku{Het harde geluid,,!}{dat Jeanne's oor trof deed}{haar oog schitteren}\\

\haiku{De gramofoon werd aan,.}{den gang gebracht maar was niet}{geheel in orde}\\

\haiku{Een verborgen deur,!}{sprong open toen hij toevallig}{er tegen leunde}\\

\subsection{Uit: Prikkel-idyllen. Deel 2}

\haiku{Mijn dochterke, de,!}{zoete Sperata wierd mij}{afgetruggeld}\\

\haiku{Onze pen is niet,,!}{bekwaam het tafereel dat}{volgde te malen}\\

\haiku{Vooraleer zij het,}{wist bevond Ewalda zich te}{midden der t'accoord om}\\

\haiku{Den ganschen dag zwierf....}{hij met den slangenman Pol}{omheen het serail}\\

\haiku{Men bevindt zich in!}{tegenwoordigheid van een}{ijselijk drama}\\

\haiku{menschweerdig bestaan!}{te winnen krijgt mijns zeggens}{steeds meer dringendheid}\\

\haiku{{\textquoteleft}In den geest,{\textquoteright} besloot, {\textquoteleft}!}{hij al weenendzal ik steeds}{een der Uwen blijven}\\

\haiku{Met een groote overmacht,!}{was hij nader geslopen}{en viel de roovers aan}\\

\haiku{daar bevond zij zich.}{plotseling aan den rand van}{een gapend ravijn}\\

\haiku{Wanneer de schoone,}{beestendresseersteer weer bij}{kwam hing zij tusschen}\\

\haiku{uwerzijds, mijn zoete{\textquoteright} {\textquoteleft}.}{engel spreekt hijden rokbroek}{hier in te voeren}\\

\haiku{Ik liep juist op de,.}{gang toen Z.K.H. daarnaar toe}{Bij het ontbijt}\\

\haiku{Wat een raar idee van.}{de menschen om op eens zoo}{eigen te worden}\\

\haiku{{\textquoteright} {\textquoteleft}Zeg eens, Pa,{\textquoteright} zei de, {\textquoteleft},!}{kroonprinsU spreekt tegen een}{troonopvolger hoor}\\

\haiku{Denk je, dat Wij je,?}{niet al lang in de gaten}{hebben kereltje}\\

\haiku{{\textquoteright} {\textquoteleft}Sire, ik ben naar,,.}{hem toe gegaan en heb hem}{verteld zoo en zoo}\\

\haiku{En verder heb ik.}{niets meer van de liefde van}{den Kroonprins gehoord}\\

\haiku{Een voldoend bedrag.}{werd tevens overgemaakt tot}{dekking der kosten}\\

\haiku{Merkwaardigheden.}{waren in het dorp niet veel}{te bezichtigen}\\

\haiku{Dat geboomte en.}{uitspansel den bodem}{De vergissing}\\

\haiku{{\textquoteright} roept Unac uit, {\textquoteleft}heeft mijn?}{broeder met de vogeltong}{zijn scalp verloren}\\

\haiku{'k Jok niet, ik jok,,!}{niet Utah nog is de wond Warm}{als uw eigen bloed}\\

\haiku{Onze patrouille.}{had een geheelen dag en}{nacht vergeefs gewacht}\\

\haiku{Goed ordinair zwart,.}{en blond stemming  vast rood}{aan den flauwen kant}\\

\subsection{Uit: Prikkel-idyllen. Deel 3}

\haiku{Armand en Pierre, ().}{de beide laatsten achter}{boom verscholen}\\

\haiku{Al met zijn neus in,,!}{den wind ja wind Al met zijn}{neus in den wind}\\

\haiku{Mourden sullen plaats,!}{hibben onschuldigen in}{de gefengenis}\\

\haiku{Zoetjes an, kommen,,.}{me waar we wezen motten}{meneer de avekaat}\\

\haiku{Na haar grootje, van,,.}{vaders kant van zelfsprekend}{meneer de avekaat}\\

\haiku{Je kwam bij ons in.}{de keuken met een boodschap}{van meneer Serlier}\\

\haiku{Het gerecht behoeft!}{zich toch niet levendig te}{laten verbranden}\\

\haiku{Elk bruigom heeft zijn,,!}{bruid De galg heeft ook haar buit}{Nu snij ik uit}\\

\haiku{Bessie springt met hem.}{in den vloed en zwemt naar een}{onbewoond Eiland}\\

\haiku{Wilt U zoo beleefd,?}{zijn om vooraan te gaan staan}{bij de famielje}\\

\haiku{Een ridder geheel,.}{in zwarte wapenrusting}{komt aangereden}\\

\haiku{Ik zie precies voor,.}{me wat je dee alsof ik}{er bij geweest was}\\

\haiku{{\textquoteright}  (Tot Jut.) Heidaar,,,!}{jij analphabeet Schrijf op voor}{ik het vergeet}\\

\haiku{Vorigen, Marco,,.}{met tamboer trompetter en}{twee soldaten}\\

\haiku{De Jut trekt af met,, '.}{schand Wij gaan hem na tot aan}{de grens vant land}\\

\section{L\'eon Veugen}

\subsection{Uit: Es God bleef. Bundel sjetse m\`et verhaolende inslaag en romantiese oetslaag}

\haiku{Geer z\"olt leefh\"obbe,.}{m\`et hart en ziel m\`et passie}{en m\`et jaloezij}\\

\haiku{In eur sympathie{\textquoteright}.}{en antipathie zeet geer}{intu{\"\i}tief}\\

\haiku{Ama waor 'n hiel, ' '}{good mins meh ze waorne}{kaptein en ze k\'os}\\

\haiku{'t Is netuurlik,}{ouch hiel good meugelik tot}{es eine vaan us}\\

\haiku{Iech ammezeerde.}{m'ch dao m\`et e potloed en}{e st\"okske pepier}\\

\haiku{Tant Lie waor gein,.}{echte Tant vaan us meh dat}{wiste veer toen neet}\\

\haiku{Meh iech waor 'rs.}{ruim drei en iech waor gans}{oetgelaote}\\

\haiku{{\textquoteleft}Wat, m\'os tiech heij de '!}{zaak op stelte z\`ette}{\'ondert gebed}\\

\haiku{Heer keek miech streng aon,,:}{nog strenger es aanders es}{of heer z\`egge wou}\\

\haiku{es flink breidoet te.}{goon stoon en de zaak mer te}{laote loupe}\\

\haiku{Iech loerde neet op '}{e paar cent est g\'ong um}{de kwaliteit vaan}\\

\haiku{Iech weet neet wat m\`et '.}{m'ch geb\"a\"ord is en iech h\"ob}{t noets gewete}\\

\haiku{Wie mie pijn tot 't.}{deeg wie beter tot iech m'ch}{beg\'os te veule}\\

\haiku{Ze waore nog,... '}{mer koelik te zien en nog}{zoe greun wie graas meh}\\

\haiku{{\textquoteright} Iech v\'ond 't fijn, tot.}{Jean miech neet bij de kleiner}{kinder t\`elde}\\

\haiku{Iech v\'ond 't e leuk.}{werkske en beg\'os direk}{m\`et te hellepe}\\

\haiku{'t sjerm h\'ong tege, ',}{de vinstert sjerm m\`et dee}{sjoene rand boe Pa}\\

\haiku{Komp geer dus oets in}{de buurt vaan Portugees Oost}{Africa en huurt}\\

\haiku{Umtot ze toch de}{leeftied had kraog ze op}{h\"a\"or b\`edsje h\"a\"or ierste}\\

\haiku{Toen bin iech nao h\"a\"or {\textquoteleft}}{tougegaange en h\"ob get}{gebazeld euver}\\

\haiku{Ze st\'ond toen vlak bij}{de kas boe iech in zaot}{en iech heel m'ch mer}\\

\haiku{Iech probeerde de,:}{beuvenste k\"orf op te duije}{meh dat veel neet m\`et}\\

\haiku{Dao hadste 't al. '.}{Mia hadt gekraak gehuurd}{en kaom aonrenne}\\

\haiku{ziech neet zoe gaw op '!}{z'ne kop zitte doornen}{\'onderwijzer}\\

\haiku{waor, beg\'os heer ', '.}{t leedsje te speule wat}{dao opt bord st\'ond}\\

\haiku{En Pa betaolde,.}{dus mer alles zellef de}{studie en de beuk}\\

\haiku{Toen heet ze h\"a\"ore l\`este.}{aosem oetgebloze en}{iech waor daobij}\\

\haiku{{\textquoteleft}De z\`eks jummers zelf, '!}{tot iechm neet zoe lang druug}{maag laote stoon}\\

\haiku{tot d'r neet altied;}{gelegenheid waor um}{de sjeun te p\'otse}\\

\haiku{Meh 't waor 'n.}{gemein streek en de vrundsjap}{waors kepot}\\

\haiku{Nou dat waor wel.}{de gammelsten auto dee}{iech oets gezeen h\"ob}\\

\haiku{Meh noe had heer 't.}{naodeil tot ouch de remme}{niks weerd waore}\\

\haiku{Miestal waor Ma '.}{erg laat daom\`et en daan m\'os}{iecht insl\'okke}\\

\haiku{Harie - zoe h\`edde -.}{m'ne nuije vrund woende aon}{de rand vaan de stad}\\

\haiku{Heer wou iers ins m\`et '.}{ne verstendige Priester}{d'reuver praote}\\

\haiku{de slumste, meh ze.}{zaog d'r nog altied eve}{verleidelik oet}\\

\subsection{Uit: 'ne Z\"och vaan de ieuwigheid}

\haiku{en in een droombeeld:}{overziet hij in \'e\'en enkel}{teken zijn leven}\\

\haiku{Gans gel\"okkig waor ',!}{r mesjien nog neet meh wel op}{weeg denau tou}\\

\haiku{Heer wis neet good wat '.}{ze veur had en keekr e}{bitteke suf aon}\\

\haiku{Veur d'n ierste kier had ':}{N\`eske get gezag m\`etne}{rechte strakke m\'ond}\\

\haiku{In Persessies m\'os '.}{r m\`etloupe en m\`et}{de Heiligdomsvaart}\\

\haiku{Aon Slevrouw waos.}{heer tougewijd bij mie es}{ein gelegenheid}\\

\haiku{meh es ze weg goon,.}{bestoon ze allein nog}{in mien gedachte}\\

\haiku{{\textquoteleft}God, God{\textquoteright}, b\`ejde, {\textquoteleft}!}{heerlaot m'ch toch neet deen}{Anti-Christ weure}\\

\haiku{Pa had veugelkes, '.}{gefok in groete kowwe}{dier zelf maakde}\\

\haiku{Die had 'r m\`et nao '.}{hoe s genome en in}{n sjeundoes gezat}\\

\haiku{Heer waos verbaas,.}{want Mestreechtenere zien neet}{zoe oranje gezind}\\

\haiku{h\'onger oetst\`elle.}{es de rutse en abele}{beg\'oste te pakke}\\

\haiku{Heer m\'os 't toch ins ';}{opzeuke en zien ofr}{deen teks k\'os einde}\\

\haiku{'t Iezer waos' ':}{nog neet d roet oft blood}{beg\'os te loupe}\\

\haiku{Gel\"okkig had eine '.}{ne zakdook boe neet al te}{v\"a\"ol sn\'ots in zaot}\\

\haiku{Heer leet 'r mer vas,:}{d'n trap op goon en vroog h\"a\"or}{hand vashawwend}\\

\haiku{mesjien laog ze get.}{te leze of zoemer veur}{z'ch oet te stare}\\

\haiku{Heer had noets get m\`et '.}{r te doen gehad of zelfs}{mer w\`elle h\"obbe}\\

\haiku{Lier  noe mer iers!}{ins um vief menute neet}{aon Ape te dinke}\\

\haiku{Dat vaan die rots en '.}{t veugelke maakde op}{h\"a\"om neet v\"a\"ol indr\"ok}\\

\haiku{Met groete aondach had ' ':}{rt werk gedoon en bij}{eder st\"okske gedach}\\

\haiku{meh dat m\'os wel good,.}{geb\"a\"ore aanders kraog me}{later geel vlekke}\\

\haiku{Ins had 'r Charles (!):}{aongesproke op straot}{heer wis nog zjus boe}\\

\haiku{'t had mie weg vaan ', '.}{ne zaol m\`et steulkes wie}{inne cinema}\\

\haiku{Toch laos heer veur, ':}{al m\'osr edere kier nao}{aosem snakke}\\

\haiku{{\textquoteright} Heer had de spanning ':}{gebroke m\`et opn druug}{meneer te z\`egge}\\

\haiku{Meh, zou 'r dat wel,?}{kinne zoelang es Penny}{d'r nog waos}\\

\haiku{De taofel st\'ond ' '.}{vol m\`et blomst\"okker es oft}{umn broelof g\'ong}\\

\haiku{{\textquoteright} Tante Yvonne m\'os ' ';}{t teske tr\"okz\`ette}{opt sjeutelke}\\

\haiku{teskes en sjeutelkes '.}{en tleurkes m\`et forsj\`etsjes}{veurt gebekske}\\

\haiku{Rechs bove en links!}{\'onder opnaomes vaan zien}{d\"ochterke Beppie}\\

\haiku{M\`et e bitsje ' '.}{funkele kraogrt}{vuur weer good op gaank}\\

\haiku{Heer had ins e book.}{geleze vaan Arthur Koestler}{euver  humor}\\

\haiku{Heer woort d'r gans werm, ';}{vaan meht waos gein zwoer}{passie deze kier}\\

\haiku{tr\`ek me rechs daan geit, '.}{de baj nao rechs tr\`ek me links}{daan geitr nao links}\\

\haiku{De veermaan sjuifde '.}{opzij en Charles moch aon}{t raad drejje}\\

\haiku{{\textquoteright} Werechtig, dao ging.}{zoe'n groete r\"os oet vaan al}{dat greun r\'ontelum}\\

\haiku{Want wat heer noe wou,!}{doen waos zun m\`et volle}{kinnis en vrije w\`el}\\

\haiku{de zuus al d'n vrun.}{trouwe  en z'ch e leuk}{huiske inriechte}\\

\haiku{gein twie vlemkes zien ',.}{t zelfde al liekene}{ze nog zoe op ein}\\

\haiku{{\textquoteright} had 'r gevlook,.}{meh dao waos die geert neet}{m\`et gerippereerd}\\

\haiku{heij is eine dee.}{probeert e forellekoer te}{dirigere}\\

\haiku{Gein al te groete, ',;}{wandeling dachr meh toch}{evekes aajd Mestreech in}\\

\haiku{{\textquoteleft}W\`elt g'r geluive,?}{tot ze m'ch nog eder jaor}{m\`et Keersmis get sjik}\\

\haiku{{\textquoteright} {\textquoteleft}W\`el iech d'ch ins get,......}{z\`egge jong jao nump m'ch de}{vrechheid neet koelik}\\

\haiku{Want 'n vrouw is e,.}{raar weze dao kin me noets}{staot op make}\\

\haiku{Ze had geveuld tot ', '.}{r get in de loch zaot}{int zonneleech}\\

\haiku{in de riechting vaan,}{Sint Pieter m\`et de berg in}{de riechting  vaan}\\

\haiku{Hier, probeert U maar!}{eens deze sabel in de}{schede te steken}\\

\haiku{Charles m\'os 't frans,:}{leedsje zinge wat ze noets}{k\'oste \'onthawwe}\\

\haiku{2 Jeh, veurwat had?}{Miep z'ch veur die secte goon}{interessere}\\

\haiku{Daan probeerde ze;}{te doen of heer de taol}{neet good beheersde}\\

\haiku{Toen pakde 'r ze.}{k\"offerke en puunde vrow}{en kinder adie}\\

\haiku{Allein 'nen dikke, ',;}{Duitserne bl\'onde reus oet}{Beieren zag niks}\\

\haiku{Dee veurige breef,, '.}{dee tr\"ok waos gekoume}{staokr debij}\\

\haiku{Ze had 'm vert\`eld,.}{vaan h\"a\"or \"a\"ontsje dee op de}{b\`este sjaol zaot}\\

\haiku{Daan stoonte d'r e:}{paar damessjeun boete}{op de vinsterbaank}\\

\haiku{{\textquoteright}, zagte ze daan, {\textquoteleft}de!}{kins wel zien tot d'r niks geit}{bove eige teelt}\\

\haiku{ne Mins m\'os toch wel.}{volslage immoreel zien}{um zoeget te doen}\\

\haiku{D'r waore sjijns.}{al hiel get lui verdr\'onke}{in die zwijnerij}\\

\haiku{M\`et groete stappe.}{leep heer in de riechting vaan}{de Poort Waarachtig}\\

\haiku{Pas wie d'n trein gans '.}{st\`el st\'ond zaog ze Amersfoort}{opt b\"ordsje stoon}\\

\haiku{zoe good wou zien en '.}{m neet op heite kole}{laote zitte}\\

\haiku{Ze st\`elde h\"a\"om get, ':}{veur watr z'ch mer ins good}{m\'os euverdinke}\\

\haiku{Wat heer 't ergste '.}{v\'ond waost verbranne}{vaan alle pep\`erre}\\

\haiku{(wat e geld hadde!)....}{ze mote oetgeve vaan}{dat klein inkoume}\\

\haiku{Trouwens g'r kint 'm {\textquotedblleft}{\textquotedblright}, '!}{beterNozak numme noe}{r gecastreerd is}\\

\haiku{H'r wou allein nog.}{get meziek opz\`ette}{en naodinke}\\

\haiku{Noe ins leek 't op, '. '}{e kruiske daan weer opn}{viefpuntige staar}\\

\haiku{t Losde z'ch op,}{in kleure sjoen greun wie vaan}{e koreveld es}\\

\section{Bea Vianen}

\subsection{Uit: Het paradijs van Oranje}

\haiku{Mohan had zich voor de.}{gelegenheid wel erg}{smaakvol opgedoft}\\

\haiku{Hij stapte kwiek uit,.}{de auto stak een hand uit}{en noemde zijn naam}\\

\haiku{Maar tussen haakjes,?}{heb je het met hem over mijn}{inkomsten gehad}\\

\haiku{Hij wil belangrijk.}{figuur zijn en daarom zegt}{hij de gekste dingen}\\

\haiku{Ze was blij met de.}{aandacht maar durfde dat niet}{te laten blijken}\\

\haiku{Of was er bij hem?}{n\'u al sprake van stipjes}{onder het glazuur}\\

\haiku{Zijn diagnose was,.}{negatief hij wist dat hij}{het zou verliezen}\\

\haiku{Hij begon opeens.}{hevig te twijfelen aan}{het doel van zijn reis}\\

\haiku{{\textquoteleft}Niets is moeilijk,{\textquoteright} zei,.}{Sirdjal en legde hem uit}{hoe er te komen}\\

\haiku{het keukentje als.}{hij het luik naar beneden}{had laten zakken}\\

\haiku{Ze zag op hem neer,.}{maar voelde zich niet tegen}{hem opgewassen}\\

\haiku{En men vergat dat,.}{terugkeer moeilijk was ja}{onmogelijk zelfs}\\

\haiku{Sirdjal pakte het {\textquoteleft}{\textquoteright}.}{glasspeciaal uit de kast}{en schonk hem thee in}\\

\haiku{Had hij genoeg van?}{het pootjebaden op z'n}{zolderkamertje}\\

\haiku{{\textquoteright} {\textquoteleft}Als je me eerder,.}{had gebeld had ik wel even}{voor je gekeken}\\

\haiku{Je moet het zoveel.}{betalen en dan moet je}{ook nog olie kopen}\\

\haiku{Kijk Firoz, niemand dwingt.}{je om wel of niet te gaan}{met een vrouw van hier}\\

\haiku{Als je het niet wilt,,.}{als je het niet kan dan moet}{je het ook niet doen}\\

\haiku{Ik ben ook blij met.}{het vlees dat  je voor me}{hebt meegenomen}\\

\haiku{want hij was verknocht.}{aan zijn plekje achter het}{waterreservoir}\\

\haiku{Hoe kan je dan nog?}{volhouden dat je me ooit}{nodig hebt gehad}\\

\haiku{Kort en goed, hij moest.}{zich maar niet verbeelden dat}{ze hem niet kenden}\\

\haiku{Hij bleef over zijn geld,.}{pochen zijn toekomstplannen}{in Suriname}\\

\haiku{Zachtjes drukte hij.}{de hendel van de deur naar}{de keuken omlaag}\\

\haiku{Hij deed het licht aan.}{en concentreerde zich op}{de kruiden voor hem}\\

\haiku{Ze hebben me op.}{een dag staan opwachten in}{de Tourtonnelaan}\\

\haiku{De sla was op, maar.}{er was nog wat rijst over en}{vlees was er genoeg}\\

\haiku{{\textquoteleft}Het was erg lekker,,,.}{bhai maar ik kan niet meer mijn}{maag is gekrompen}\\

\haiku{Hij wist nu zeker.}{wie er aan de andere}{kant van de lijn was}\\

\haiku{Voor het laatste was.}{bapa een veel te oude}{en zuinige man}\\

\haiku{Het was juist altijd.}{zijn bedoeling geweest de}{dood uit te stellen}\\

\haiku{Om half vijf was hij,.}{weer thuis hij blies en hijgde}{van de boodschappen}\\

\haiku{{\textquoteleft}Ik neem aan dat u?}{vergeten bent het licht op}{de trap uit te doen}\\

\haiku{Om kwart over zeven.}{pakte hij de tram naar het}{centrum van de stad}\\

\haiku{{\textquoteright} {\textquoteleft}We zijn allemaal,.}{gevangen en iedereen}{in zijn eigen kooi}\\

\haiku{{\textquoteleft}Weet je, bhai, ik heb.}{stilletjes gelezen wat}{je hebt geschreven}\\

\haiku{{\textquoteright} Op dergelijke.}{aantijgingen was Sirdjal}{altijd voorbereid}\\

\haiku{Hij was er wel bang,.}{voor maar liet er zich toch niet}{echt door bedrukken}\\

\haiku{Bisoenlal zat met.}{een half verdwaasd gezicht voor}{zich uit te kijken}\\

\haiku{D\'an wist je echt niet.}{of hij zich voor wie dan ook}{interesseerde}\\

\haiku{Hij kwam er nog maar,.}{zelden en het deed hem goed}{dat hij er weer was}\\

\haiku{{\textquoteright} {\textquoteleft}Ik bel je nog op,{\textquoteright}.}{zei Sirdjal en richtte zijn}{schreden huiswaarts}\\

\haiku{Hij had daar trouwens,.}{alle recht toe want hij had}{op Den Uyl gestemd}\\

\haiku{U bent bang, omdat.}{zoiets in uw eigen kring}{voor onbeschaafd geldt}\\

\haiku{Hij hijgde, voelde.}{opnieuw een koude rilling}{over zijn lichaam gaan}\\

\haiku{Hij deed de deur dicht,.}{rookte de ene sigaret}{na de andere}\\

\haiku{Hij schraapte zijn keel,.}{spuugde in de wasbak en}{keek in de spiegel}\\

\haiku{Hij kon zijn ogen niet.}{geloven en liet hem als}{in een droom binnen}\\

\subsection{Uit: Sarnami, hai}

\haiku{Zij wordt nieuwsgierig,,.}{kijkt op terwijl zij met de}{kam over haar hoofd strijkt}\\

\haiku{De oude werpt het.}{tweede kale stengeltje}{op de houten vloer}\\

\haiku{Het zijn altijd weer.}{dezelfde vragen waarop}{nooit een antwoord komt}\\

\haiku{Van het groen van de,.}{palmen de manjebomen}{en de guave}\\

\haiku{Pas na een hele.}{tijd ontspant zij zich en doet}{zij haar ogen wijd open}\\

\haiku{Een donkere hand.}{tilt met een vuile vaatdoek}{het deksel omhoog}\\

\haiku{Muskieten zwermen.}{aan en beginnen voor het}{venster te gonzen}\\

\haiku{Het heeft geen zin haar.}{nog langer te sarren met}{haar aanwezigheid}\\

\haiku{Het onverschillig,.}{antwoord van de vrouw krenkt haar}{maar zegt haar genoeg}\\

\haiku{Roekmien, nu weer in,.}{het atelier neemt plaats achter}{haar trapmachine}\\

\haiku{Zij staat meteen weer,.}{op doet de radio aan en}{gaat dan aan het werk}\\

\haiku{Ram ziet S. zitten,.}{groet vriendelijk en vraagt of}{zij al lang hier is}\\

\haiku{Zij buigt zich over het.}{boek maar kan zich niet langer}{concentreren}\\

\haiku{Onder het licht van.}{een lantaarn blijft zij staan om}{ze te bekijken}\\

\haiku{Voorin zitten twee.}{luid pratende vrouwen met}{bonte hoofddoeken}\\

\haiku{Maar er is nog een.}{andere werkelijkheid}{om voor te vechten}\\

\haiku{Zij loopt weg van het,.}{raam laat het muskietenkleed}{neer en kleedt zich uit}\\

\haiku{{\textquoteright} Dan zit zij in een,,.}{motorboot nog altijd naakt}{op weg naar het schip}\\

\haiku{De eersten zijn meer.}{bij hun vader dan bij hun}{eigen moeder thuis}\\

\haiku{Zij laten S. en.}{Soekhia in een bedrukte}{stemming achter}\\

\haiku{Zei haar vader niet?}{dat zij de laatste tijd zo}{lang van huis wegbleef}\\

\haiku{S. heeft boodschappen.}{gehaald bij de Chinees en}{is op weg naar huis}\\

\haiku{Na die bewuste.}{vrijdag heeft zij de vrouw niet}{meer teruggezien}\\

\haiku{Haar haren, haar rug.}{en haar buik zijn drijfnat van}{het transpireren}\\

\haiku{s Avonds na het bad.}{studeren zij verder op}{haar kamer boven}\\

\haiku{Angst, verdriet gemengd,.}{met gevoelens van vreugde}{overweldigen haar}\\

\haiku{Zij kent zijn drift en.}{ziet aan zijn gezicht dat het}{ergste voorbij is}\\

\haiku{Hij kijkt niet op, zelfs.}{niet als de bliksem knalt en}{huis en erf verlicht}\\

\haiku{Het meisje wil iets,,.}{zeggen doet haar mond open maar}{er komt geen woord uit}\\

\haiku{Vervolgens wordt hem.}{gevraagd waarmee hij haar wenst}{te accepteren}\\

\haiku{Na vier maanden is.}{het afgebouwd en kunnen}{zij het betrekken}\\

\haiku{Radj is de laatste,.}{tijd erg prikkelbaar heeft zij}{tegen S. gezegd}\\

\haiku{Zij vindt de stof erg.}{mooi en vraagt of zij haar heel}{even mag betasten}\\

\haiku{{\textquoteright} vraagt S. {\textquoteleft}De dokter,{\textquoteright}.}{zegt over anderhalve maand}{antwoordt Selinha}\\

\haiku{S. volgt haar naar de,.}{keuken waar zij tot half acht}{zitten te praten}\\

\haiku{Zij kijkt naar de lucht,.}{zwaait met de stok en begint}{vlugger te lopen}\\

\haiku{Hij pakt de stok uit.}{haar hand en werpt hem naar het}{andere trottoir}\\

\haiku{Het betekent dat.}{zij moet wachten totdat haar}{vader terug is}\\

\haiku{Er bestaat zo iets.}{als een paspoort dat je niet}{zelf hebt getekend}\\

\haiku{Zij wil niet met een.}{man gezien worden en dan}{liefst in een auto}\\

\haiku{Achter haar klinkt de.}{motor van de achteruit}{rijdende auto}\\

\haiku{Zij had het zo druk.}{met de huishoudelijke}{beslommeringen}\\

\haiku{Het ene moment voelt.}{zij zich vertederd door de}{herinneringen}\\

\haiku{{\textquoteleft}Je weet dat ik nog,{\textquoteright}.}{op school zit antwoordt zij om}{hem te ontwijken}\\

\haiku{Sindsdien is er in.}{haar leven een nieuwe angst}{bijgekomen}\\

\haiku{Straks zal het tot haar.}{doordringen dat zij nooit meer}{naar huis terugkeert}\\

\haiku{Islam zou kunnen.}{ontkennen en weigeren}{met haar te trouwen}\\

\haiku{Zij kan aan niemand,.}{anders laten merken wat}{zij van binnen voelt}\\

\haiku{De dief had zich met.}{olie ingesmeerd maar werd ten}{slotte toch gepakt}\\

\haiku{Zij moet van de rand,.}{zijn afgegleden toen hij}{naast haar kwam liggen}\\

\haiku{Kort voordat zij naar.}{bed ging liet zij de jongens}{van Soekhia binnen}\\

\haiku{Ata's aanwezigheid.}{dwingt haar tot het opvoeren}{van een toneelspel}\\

\haiku{Zij knipt het licht uit,.}{trekt zachtjes de deur achter}{zich dicht en is weg}\\

\haiku{Hij slaat de druppels.}{uit de kam en stopt die in}{de zak van zijn hemd}\\

\haiku{De jongen had een.}{grote bewondering voor}{zijn fysieke kracht}\\

\haiku{Zij betaalt hem de.}{rekening en gaat naar de}{kamer van het kind}\\

\haiku{Misschien bedoelt hij,{\textquoteright}.}{dat hij je zal verstoten}{zegt zij ten slotte}\\

\haiku{Hij doet de deur open,.}{glimlacht nerveus en neemt de}{jongen van haar over}\\

\haiku{Hij komt en gaat met.}{het besef dat zij afscheid}{nemen van elkaar}\\

\haiku{Hetzelfde deed zij.}{de avond dat zij hier voor het}{eerst had geslapen}\\

\section{S.G. van der Vijgh jr.}

\subsection{Uit: Werkers}

\haiku{Mijn tranen heet gaan}{om hun doode trekken neer}{en in mijn armen}\\

\haiku{loeide het uit dat,;}{ze gezocht werd dat ze niet}{meer wou in de keet}\\

\haiku{eindeloos blanke,.}{zee van landen vlagend om}{de hooge machtfabriek}\\

\haiku{Het is van de kerk,:}{uit dat plechtigheid en ernst}{waren over het dorp}\\

\haiku{Als er zieken zijn,;}{hooren de smidslui het bij}{toeval van vreemden}\\

\haiku{In den hoek hoestte,.}{oude Piet wakker schrikkend}{uit zijn gedommel}\\

\haiku{Wel weerlichts,{\textquoteright} zeit-ie, {\textquoteleft}' ',?}{t is zonde da'kt zeg}{is da nou stoke}\\

\haiku{{\textquoteright} vroeg Kees minachtend,.}{dadelijk partij nemend}{voor de dagstokers}\\

\section{Simon Vinkenoog}

\subsection{Uit: Liefde. Zeventig dagen op ooghoogte}

\haiku{Maar ik heb alle:}{zekerheden verloren}{behalve die ene}\\

\haiku{redakteuren Claus,,,):}{Michiels Mulisch Vinkenoog}{zijn notities stuurt}\\

\haiku{Ook de zeventig,.}{dagen die ik sleet ben ik}{in staat te overzien}\\

\haiku{ik heb die liefde, (}{vele namen gegeven}{\'e\'en daarvan is God}\\

\haiku{Ik zal ze wieden,.}{uit mijn dagboekbladen en}{toch niets weglaten}\\

\haiku{Ik heb een afstand,.}{geschapen kregen die ik}{wilde overbruggen}\\

\haiku{Kijken, of ik het,.}{nog kan rekonstrueren en}{tot hoever terug}\\

\haiku{s avonds komt zij thuis,.}{langs terwijl Elize en Don}{naar de bioskoop zijn}\\

\haiku{De schrijver geplaagd,.}{door muggen drie dagen voor}{de terechtzitting}\\

\haiku{de godgeworden.}{mens eerder dan om een van}{de middelen}\\

\haiku{Van nu af aan slechts,.}{d\'eze regels mijn gids en}{begeleider}\\

\haiku{Zij kon het weten,, {\textquoteleft}{\textquoteright}.}{zij had het meegemaakt zij}{hadermee geleefd}\\

\haiku{God is here, and,:}{I am a witness to the}{fact and so are you}\\

\haiku{Het ritme is een, (.}{eenvoudige anapest/trochee}{lees HopkinsPenguin-ed}\\

\haiku{Het motto grijpt me,:}{meer dan tien jaar later bij}{de strot zelf zegt hij}\\

\haiku{ik ben blij en trots,.}{het meegemaakt te hebben}{als vriend des huizes}\\

\haiku{Alsof een voorwerp,!}{een betekenis heeft die}{bevrijd kan worden}\\

\haiku{Hoeveel verwijten,:}{tegen mij die ik nog zal}{moeten uitbannen}\\

\haiku{Weet dat je minder,,.}{ziet zodra je kwaad wordt de}{perceptie mindert}\\

\haiku{Maak van je kind een,.}{heilig huis dat geen ander}{heilig huis erkent}\\

\haiku{Geen wetenschap is,.}{mogelijk zonder geloof}{geen leven of dood}\\

\haiku{Niet op alles het,.}{antwoord dat ligt in ieder}{voor zich besloten}\\

\haiku{Ik bevrijd je van,.}{de angst ook voor de dood want}{ik ben het leven}\\

\haiku{zelfs de vlooien van.}{m'n buren spelen een rol}{in de eeuwigheid}\\

\haiku{Zij weten wie ik,.}{ben al weet ik het met een}{ander lichaam}\\

\haiku{De normale mens - - {\textquoteleft}.}{funktioneert volgens hem}{dans les limites}\\

\haiku{zo helpe mij God -.}{waarachtig het is ook nog}{nieuw realisme}\\

\haiku{omdat hij niet wist,.}{te vertellen of hij van}{z'n vrouw hield of niet}\\

\haiku{het Evangelie van), (;}{Johannes de manen van}{de leeuwde hartstocht}\\

\haiku{Ik loop de gang op,.}{haar Chinese ochtendjas}{half omgeslagen}\\

\haiku{Ik heb geluisterd, ().}{ik kan het helpen dat zij}{huildeopluchting}\\

\haiku{Alles is goed, ja,.}{als de mensen hun weten}{kunnen gebruiken}\\

\haiku{Moet de weg vrij zijn,?}{of begeven we ons met}{goed en kwaad op weg}\\

\haiku{Schuin tegenover mij,.}{op de derde verdieping}{een raam stond halfopen}\\

\haiku{Ik weet het, ik was,.}{de langste ik droeg de vlag}{trots lopend voorop}\\

\haiku{Ik funktioneer.}{als beeld in de hersenen}{van mijn bezoekers}\\

\haiku{gerrit achterberg,[ -]}{dichter   20 mei 1905 17}{januari 1962}\\

\haiku{zijn zoon/mijn zoon, het (:}{gezamenlijkeik hoor}{me praten en denk}\\

\haiku{alle andere,,.}{dingen die ik zie zijn even}{grote wonderen}\\

\haiku{{\textquoteleft}Je openstellen voor,,.}{de waarheid die in je ligt}{en die uitdragen}\\

\haiku{Alle po\"ezie.}{wordt geschreven om dit doel}{nader te brengen}\\

\haiku{er is geen einde -:}{aan het weten godzijdank}{is er een begin}\\

\haiku{We waren bij Huub,,,.}{weggevlucht hij m'n vriend was}{niet te verdragen}\\

\haiku{Reineke kwam thuis,.}{bij de omhelzing gleden}{onze kleren weg}\\

\haiku{Met de stoomkursus,}{van de vertaling doorloop}{ik jaren ineens}\\

\haiku{* ~ Met Reineke (;}{en Stepheneen andere}{route achter mij}\\

\haiku{een brandweerman, twee,.}{jongetjes anderen die}{UFO's hadden aanschouwd}\\

\haiku{Het echtpaar Van O.,:}{jaag ik weg zo ook S. die}{ik schrijvend toevoeg}\\

\haiku{Zijn boodschap aan mij.}{een gans andere dan de}{door Ruud gehoorde}\\

\haiku{(Hier wordt de lezer.}{gevraagd zich over het witte}{papier te buigen}\\

\haiku{Benodigdheden.}{een gemakkelijke stoel}{en een telefoon}\\

\haiku{Misschien komt het geld,...:}{nog eens met sneeuwbaleffekt}{terug wie weet Reis}\\

\haiku{Mijn verzameld werk, (-).}{de eerste gedichten1948}{1964 en de laatste}\\

\haiku{We betrekken een,,.}{nieuwe woning voor een jaar}{weer op de Bloemgracht}\\

\haiku{zij absorberen,.}{slechts wat hoogst nodig is het}{meest voor de hand ligt}\\

\haiku{in oprechtheid, met,.}{geduld in onderwerping}{en vertrouwen}\\

\haiku{Ik leef het meeste,{\textquoteright} -:}{als ik alleen ben met jou}{bijna verwijtend}\\

\haiku{ook Elize zal de,.}{man vinden die haar geeft wat}{ik niet kon geven}\\

\haiku{Neem de mensen hun,.}{gereedschap niet uit handen}{geef ze te spelen}\\

\haiku{Een schuin inzicht in,;}{hun portiek waar zij woonden}{op een beletage}\\

\haiku{{\textquoteleft}Als Simon nu maar,...}{niet dacht dat hij altijd naar}{woorden moet zoeken}\\

\haiku{Reineke wakker -.}{ik breng haar achterop de}{scooter naar haar werk}\\

\haiku{{\textquoteleft}Anderen zouden,.}{verslaafd raken als ze maar}{niet zo'n angst hadden}\\

\haiku{Ik ga een deel van,.}{de gang met ze mee wijs ze}{de weg naar boven}\\

\haiku{{\textquoteright} {\textquoteleft}Ik moet de waarheid.}{ervan met mijn gehele}{wezen ervaren}\\

\haiku{Ik weet het niet, wat,?}{moet ik zeggen welke kant}{gaat het sprookje uit}\\

\haiku{Ik ben geen - nee, niet,:}{een van al die andere}{dingen of ook toch}\\

\haiku{Het verbaast me, als ():}{een anderStephen bewust}{en duidelijk zegt}\\

\haiku{Hij heeft een antwoord,:}{van de Bijenkorf gehad}{dat hem bevredigt}\\

\haiku{Als je maar niet stil,,}{blijft staan nalaat vooruit te}{gaan als je maar weet}\\

\haiku{vrouw, vrienden, muziek,,,,?}{geluk werk wat te roken}{wat wil ik nog meer}\\

\haiku{de mensen eerst op,.}{weg naar zichzelf dan kunnen}{we verder spreken}\\

\haiku{Over de mens, die schrijft, {\textquoteleft}{\textquoteright}.}{de levende schrijver die}{jijin handen hebt}\\

\haiku{alles wat ik nu}{begrijp alles waaraan ik}{nu weet te denken}\\

\haiku{zij maakt aanstalten,.}{naar bed te gaan de heer des}{huizes tikt en vrijt}\\

\haiku{Ik heb gehandeld,.}{in ijdelheid de tijd met}{mij vereenzelvigd}\\

\haiku{HH theologen,,!}{dichters mathematici}{en filosofen}\\

\haiku{Hanterend zoveel.}{mogelijkheden als ik}{bemachtigen kan}\\

\haiku{Een maatschappij dient.}{op solider maatschappij}{te worden herbouwd}\\

\haiku{ik raak nauwelijks,.}{iets aan ik raak verzeild in}{de pagina's zelf}\\

\haiku{Hij trekt z'n laarzen,):}{weer aan maandag vertrekt hij}{naar de Spaanse vlieg}\\

\haiku{Hij deelde me mee.}{niet te weten wat hij met}{z'n toekomst moest doen}\\

\haiku{Ik denk van een plicht,.}{te zijn ontslagen maar dit}{blijkt niet het geval}\\

\haiku{{\textquoteright} (Hij woont tijdelijk.) {\textquoteleft}:}{op een van de WallenEn}{in de folders staat}\\

\haiku{De onomatopee,:}{in het Engels vaak mooier}{dan het Nederlands}\\

\haiku{De stem bij de bel. (.}{Ik was de eigenaar van}{de stem vergeten}\\

\haiku{Hij konfronteert ook,?}{mij want wie is volkomen}{onbevooroordeeld}\\

\haiku{ik dacht dat ze zo'n.}{monsterlijk gezicht zette}{om me te plagen}\\

\haiku{Ik heb het nog, nog,.}{\'e\'en keer laten meemaken}{nooit meer aangeraakt}\\

\haiku{al een week lang was.}{G. met dochter Rosalie}{in dit land terug}\\

\haiku{Blijf vragen en het.}{antwoord zal je in de schoot}{worden geworpen}\\

\haiku{Ondertussen zijn ();}{Huub en Olvertvandaag 24}{binnengekomen}\\

\haiku{De mens aan de mens,.}{gelijk met een andere}{taak te vervullen}\\

\haiku{tussen goed en kwaad,:}{tussen deze wereld en}{de schijnwerelden}\\

\haiku{Het heden is de,}{toestand waarin ik het liefst}{verkeer door wat komt}\\

\haiku{{\textquoteleft}Beste Simon, ik...{\textquoteright}.}{geloof niet dat En ik zal}{je zeggen waarom}\\

\haiku{heiligen zijn op,.}{weg naar de volmaaktheid en}{heel ver gekomen}\\

\haiku{Ik zou uren over Jan,.}{willen spreken over alles}{wat hij zegt en doet}\\

\haiku{de van 10 tot 75, -.}{sekonden vrij vallende}{skydivers 250 km/u}\\

\haiku{de armen gespreid,,,.}{op de buik het hoofd in de}{nek als een vogel}\\

\haiku{Die denken dat het,,.}{de ziel is of zoiets ja}{ze noemen het ziel}\\

\haiku{ik probeerde wat, {\textquoteleft}{\textquoteright};}{ik z\'o kreegtoevallig nog}{uit te spreken ook}\\

\haiku{het cybernetisch.}{denken toegepast aan de}{omstandigheden}\\

\haiku{Dit is nu, dit is,,.}{altijd dit is van hieruit}{waar dit beleef ik}\\

\haiku{E\'en groot toneelstuk,,.}{iedereen doet maar wat maar}{hij weet wel b\'eter}\\

\haiku{Ik wil dat je aan,,.}{me werkt zoals ik aan dit}{boek werk met liefde}\\

\haiku{Van hieruit ga ik, -:}{verder opnieuw als het moet}{telkens weer opnieuw}\\

\haiku{Het huis aan de Oudezijds.}{verleende onderdak aan}{vele bezoekers}\\

\haiku{Dit ben ik, dit ben,.}{jij nog niet ik ben niet aan}{je toegekomen}\\

\haiku{{\textquoteright} Rudolf leerde het,:}{Boek der Veranderingen}{kennen hij wierp 13}\\

\haiku{Ik rijd scooter op,.}{routine haal I Tjing voor}{vrouwen en vrienden}\\

\haiku{Zoveel, dat het een.}{wonder is dat er nog naar}{me wordt geluisterd}\\

\haiku{hij ziet het zwarte, ().}{leer van mijn jas kijkt niet tot}{in mijn ogenik wacht}\\

\haiku{Het valt me prettig,}{dit verhaal op dit ogenblik}{tot jou te richten}\\

\haiku{Ik heb aandacht voor,.}{de details zij maken het}{leven  zoeter}\\

\haiku{{\textquoteleft}Daar ben ik, geloof,,.}{ik niet zo geschikt voor voor}{document humain}\\

\haiku{het wonder van de,.}{elektriciteit verdwenen}{het wonder daglicht}\\

\haiku{Want, zeg ik, als ik,}{mij bewijs bewijs ik door}{mijn ontdekkingen}\\

\haiku{Vrijheid is sneller,.}{dan het licht vrijheid is je}{laatste dimensie}\\

\haiku{Je bent vrij, als je,,.}{sterft vrij als je ademt in slaap}{of bij bewustzijn}\\

\section{Ab Visser}

\subsection{Uit: Leven van de pen}

\haiku{{\textquoteleft}Ga daar maar rustig.}{staan vader en zorg dat je}{d'r niet uitlazert}\\

\haiku{{\textquoteright} Mijn permanente.}{eenmansshow levert mij geen}{enkel voordeel op}\\

\haiku{Hij was het type:}{waarvan de mannen onder}{elkaar zeggen}\\

\haiku{Agatha doorliep geen,.}{enkele school maar ontving}{haar opleiding thuis}\\

\haiku{hij bleef ongetrouwd,.}{vermoedelijk omdat hij}{homosexueel was}\\

\haiku{de gevulde lijn ().}{plooit zichin de confectie}{naar de plompe vorm}\\

\haiku{{\textquoteleft}wordt eikenschors bij ', '.}{t pond gewogen men weegt}{kaneel bijt lood}\\

\haiku{Een boek hoeft niet eens.}{goed te zijn als de titel}{maar suggestief is}\\

\haiku{Ik heb zelf pas een {\textquotedblleft}{\textquotedblright}.}{boek uitgegeven datKaal}{met een kuifje heet}\\

\haiku{Hij was tenslotte.}{op alle terreinen des}{levens de pineut}\\

\haiku{Het was tevens ons,.}{afscheidsgesprek al wisten}{wij dat toen nog niet}\\

\haiku{{\textquoteleft}Vergeet niet, dat ik!}{menige schuit door zware}{stormen geloodst heb}\\

\haiku{{\textquoteleft}Kop dicht en kom mee,.}{naar beneden dan zal ik}{het je bewijzen}\\

\haiku{{\textquoteleft}Het komt er toch niet,,...{\textquoteright} {\textquoteleft}}{meer op aan moeder hij was}{een gelukkig mens}\\

\section{Lodewijk Vleeschouwer}

\subsection{Uit: Het boek der vertellingen en andere kuizelarijen (onder ps. Reinaert de Vos)}

\haiku{Maar het ongeluk.}{had nog niet opgehouden}{hem te vervolgen}\\

\haiku{de honden moeten.}{u niet dikwijls onder den}{tand gehad hebben}\\

\haiku{De moeder ontstak,.}{een klein lampken en zette}{het op de tafel}\\

\haiku{Bazin, geeft daar eens, '.}{een ouweltjen en eenen}{knoop voort cachet}\\

\haiku{maar wat moeite hij,.}{ook nam er kwam geen druppel}{melk te voorschijn}\\

\haiku{maar ik wil toch de,}{schuld niet zijn dat gij in het}{ongeluk stort.-}\\

\haiku{Ik slijp de scharen,.}{en draai gezwind En hang mijn}{mantel naar den wind}\\

\haiku{- Het gaat u zeker,?}{zeer goed daar gij zoo lustig}{bij uw slijpen zijt}\\

\haiku{- Ja, antwoordde de,.}{schaarslijper het handwerk heeft}{een gouden bodem}\\

\haiku{Mijn twee matanten,,.}{moet gij weten Die waren}{geen van bei getrouwd}\\

\haiku{het is uitmuntend......}{om verkenskarbonaden}{in te braden}\\

\haiku{- De visch zwom weg, maar.}{kwam spoedig weer terug en}{wierp den ring aan land}\\

\haiku{Zoo tij ik op de,,....,!}{straat kwam de enveloppe}{af en raad eens}\\

\haiku{Twee honderd francs,,,,!}{zei mijn pere wel jongen}{dat is kapitaal}\\

\haiku{Elken burger der.}{stad zond hij een gebraad en}{eene kruik spaanschen wijn}\\

\haiku{Toen hij aanklopte,:}{vroeg de deurwachter met eene}{diepe forsche stem}\\

\haiku{Daarom zou ik wel,}{wenschen nog een paar uren te}{mogen slapen.-}\\

\haiku{En zoo, vaarwel.-.}{Frans draafde nu welgezind}{op Antwerpen toe}\\

\haiku{Op gindsch kasteel zijn,.}{kamers genoeg en ik heb}{er de sleutels van}\\

\haiku{Eindelijk werd de,;}{echte sleutel gevonden}{en het slot ging open}\\

\haiku{Mijn geest verliet het,.}{afgeteerde  lichaam}{maar bleef hier verband}\\

\haiku{- Zoo sprak hij, en kwam,,.}{verzadigd en dik eerst laat}{in den avond naar huis}\\

\haiku{- Dit zal u ook niet,,.}{bevallen zei de Kater}{het heet Heeluit}\\

\haiku{- Heb ik de stukken '?}{niet gelezen en ers}{nachts over nagedacht}\\

\haiku{- Als ik maar eens een,.}{mensch te zien kreeg ik zou hem}{toch wel aanvallen}\\

\haiku{- Nu, broeder Wolf, sprak,?}{de Vos hoe zijt gij met den}{Mensch uitgemeten}\\

\haiku{- Ja maar..... - En ge zijt,.}{te rechtvaardig om er niet}{in toe te stemmen}\\

\haiku{En ziet, hunne hoop,.}{bedroog hen niet want Jan kwam}{wezenlijk terug}\\

\haiku{Hij zag wel in, dat;}{hij hier een geheel ander}{leven moest leiden}\\

\haiku{Eens was hij met zijn,.}{vader in het woud gegaan}{om hout te vellen}\\

\haiku{tien of twintig van,}{hun doen mij niets als God niet}{tegen mij is.-}\\

\haiku{want het gaf een slag.}{als of er de donder op}{gevallen ware}\\

\haiku{Jan leerde spoedig,.}{hoe men het scheepstuig roer en zeil}{moest behandelen}\\

\haiku{- Verhoede God en,!}{mijn goede degen dat de}{prinses zou sterven}\\

\haiku{maar hij was slechts de,.}{zoon van eenen dorpschout en zij}{eene koningsdochter}\\

\haiku{- Nu, God zij dank, dat!}{gij weder daar zijt en een}{koning geworden}\\

\haiku{Of gij gaarne op,.}{dien armstoel zit of niet daar}{lachen wij mede}\\

\haiku{Weetde wel, dat ge?}{daar heel fraaie vraagskens op uwen}{program gezet hebt}\\

\haiku{- wel te verstaan, als.}{een lantaarnpaal een ander}{hart had dan van steen}\\

\haiku{- Op mijn woord van eer, -,.}{vervolgde ik tot mijne}{makkers sprekende}\\

\haiku{Eindelijk scheen de,;}{meester te denken dat ik}{genoeg had gehad}\\

\haiku{maar zij mochten mij,.}{toch niet lijden zoo jaloersch}{waren zij op mij}\\

\haiku{Welk een slag, voor een,!}{jongman die zulk een uniform}{bekostigen moest}\\

\haiku{Dobble bleef roerloos.}{op den kapblok zitten en}{huilde als een kind}\\

\haiku{Ik zegde niemand,, -?}{dat de hond eene zelfmoord had}{gepleegd waartoe dit}\\

\haiku{als gij er een hebt,.}{dan moest  gij er liever}{maar in gaan wonen}\\

\haiku{Hij schreef met veel zorg.}{het verlangde bewijs van}{ontvangst en ging heen}\\

\haiku{Had hij ze aan uw.}{geld blauw geteld dan zou het}{erger geweest zijn}\\

\haiku{Die snaak Waters was.}{op haar verliefd geworden}{en had haar getrouwd}\\

\haiku{- n'est-ce pas a?}{un homme comme il faut}{que j'ai affaire}\\

\haiku{- Bien du succ\`es, mon, -;}{cher Lovelace wierp hij hem}{als een vaarwel toe}\\

\haiku{Ge moet nochtans niet,.}{denken dat ze nu nog in}{die positie staan}\\

\haiku{- Je te livre le,. -?}{pigeon tu en feras ce}{que tu pourras Quand}\\

\haiku{vous avez le accent,... -,.}{aigiou le accent gr\`eve}{Grave milord}\\

\haiku{L\`a on ne trouve;}{que l'ignorance greff\'ee}{sur la sottise}\\

\haiku{monsieur ne sait?}{pas qu'il y a de grandes}{r\'ejouissances demain}\\

\subsection{Uit: Stukken en brokken}

\haiku{zy worden niet met,.}{de hand voortgeduwd maer door}{een peerd getrokken}\\

\haiku{ikke heb betael,;}{ikke vrag une voiture}{suppl\'ementaire}\\

\haiku{en hy bezag my,;}{met roedelyden omdat}{ik zoo lang werk had}\\

\haiku{want hy was even zoo,;}{gauw met zyn middagmael klaer}{als de anderen}\\

\haiku{en Ka{\"\i}n, hem de... -,?}{schurk Gelyk al de witte}{veronderstel ik}\\

\haiku{majoor, vertelt ons, -.}{dat eens werd er aen alle}{kanten geroepen}\\

\haiku{Nooit zult gy raden,.}{wat wy in den buik van het}{wangedrocht vonden}\\

\haiku{en als ik niet mis,.}{ben zal myn kunstgreep ons plaets}{genoeg bezorgen}\\

\haiku{Inderdaed, nu ik,,}{er op lette zag ik den}{tooneelspeler zoo haest}\\

\haiku{Toen de beweging,;}{gestild was speelde het stuk}{nog eenigen tyd voort}\\

\haiku{Dit bevonden wy,.}{toen wy ons van het schouwburg}{naer huis begaven}\\

\haiku{Zyn neus was deerlyk,!}{opgezwollen en zyn hoofd}{deed hem och arme}\\

\haiku{- Rydt naer Lancaster, -,;}{zeg ik u. Wel ik zal naer}{Lancaster ryden}\\

\haiku{Als de doctor de,,:}{tong gezien had zegde hy}{het hoofd schuddende}\\

\haiku{De minister had',.}{zyn tekst genomen uit St.}{Jans Evangelie kap}\\

\haiku{Hy hoort van onze,.}{inrigtingen gewagen}{en hy schokschoudert}\\

\haiku{men maekte een'';}{gemeenteraedsheer van hem of}{zelfs een minister}\\

\haiku{- Zeer wel, - zegde ik, -;}{ik zie dat ge reeds zeer sterk}{zyt in het latyn}\\

\haiku{Er moest een renloop,,.}{van peerden niet verre van}{de stad plaets grypen}\\

\haiku{Ik verhaelde hem,}{het tooneel dat ik bygewoond}{had en hy snelde}\\

\haiku{Ik verhaelde wat,:}{ik gezien had waerop de}{advokaet my vroeg}\\

\haiku{men aenzag haer als;}{beneden de weerdigheid}{van een beschaefd man}\\

\haiku{Deze afkeer was,}{zoo groot dat weinigen het}{mogelyk achtten}\\

\haiku{- De man die dat in,!}{zyn hoofd heeft gekregen is}{van onze eeuw niet}\\

\haiku{Waerin bestaet dan,?}{die fransche educatie die}{gy hebt ontvangen}\\

\haiku{Hoe ik met den held.}{van deze geschiedenis}{kennis maekte}\\

\haiku{ie bezien, had ik'.}{een makker nevens my op}{de bank gekregen}\\

\haiku{Nog verscheidene.}{malen ontmoette ik hem}{op de zelfde plaets}\\

\haiku{Met diepen weemoed,}{en het hart vol benauwdheid}{verliet ik myne}\\

\haiku{Het leven aldaer,}{vond ik geheel iets anders}{dan hetgene ik}\\

\haiku{In het begin ging.}{de zaek beter dan ik er}{my aen verwachtte}\\

\haiku{Nauwelyks was zy,.}{de deur uit of eene dienstmeid}{trad in den winkel}\\

\haiku{Het was de meid, die.}{vroeger een half pond suiker}{was komen koopen}\\

\haiku{men had myn' hoed naer,.}{huis gezonden en de meid}{had alles verteld}\\

\haiku{hoewel hy volstrekt':}{geene achterdocht wegen zyn}{toestand gevoelde}\\

\haiku{Moeijelyk zyn de,.}{tyden geweest in welke}{gy hebt geheerscht}\\

\haiku{Was alle kennis?}{en alle wetenschap niet}{in hem opgevat}\\

\haiku{maer Juno, de vrouw,.}{van Jupiter was er ook}{zot naer geworden}\\

\haiku{De eerste helft van,;}{dit woord is latyn en de}{andere helft ook}\\

\haiku{- Maer daertoe is het,,;}{fransch te verheven te edel}{en vooral te ryk}\\

\haiku{- Pos, - zei Jan, - en ze.}{schreef het woord anthropos op}{een stuksken papier}\\

\haiku{maer ge kunt misschien,.}{geen  grieksch genoeg om dit}{fransch woord te verstaen}\\

\haiku{quatre \`a point, deux,,;}{\`a quatre en zoo voorts tot}{vier en twintig toe}\\

\haiku{{\textquoteleft}Avignon is eene hel,;}{geworden de schuilplaets van}{alle gruwelen}\\

\section{Bert Voeten}

\subsection{Uit: Doortocht. Een oorlogsdagboek 1940-1945}

\haiku{Ook ditmaal heb ik.}{weer geruimen tijd in een}{greppel gelegen}\\

\haiku{Maar nu we het front,.}{hier vlak bij ons zien k\`an het}{niet lang meer duren}\\

\haiku{Even later hoort men.}{ze donkere roffels slaan}{op de pontonbrug}\\

\haiku{een van de zangers.}{zeggen en dan zetten zij}{het Wilhelmus in}\\

\haiku{Het was of alles.}{wachtte op het gedonder}{der luchteskaders}\\

\haiku{{\textquoteright} {\textquoteleft}Gelukkig niet{\textquoteright}, zei.}{ik en we gaven elkaar}{een knipoogje}\\

\haiku{Ik zag die wanhoop,.}{in haar oogen toen zij naast mij}{ging over het Damrak}\\

\haiku{{\textquoteleft}Joodsche gasten niet{\textquoteright},.}{gewenscht hing bij de deur van}{een caf\'etaria}\\

\haiku{Mijn voeten hield ik.}{stijf tegen elkaar als om}{mij schrap te zetten}\\

\haiku{De angst woont achter,.}{geblindeerde ramen in}{gangen en sloppen}\\

\haiku{Seyss heeft de kaarten,.}{op tafel gelegd langzaam}{en nadrukkelijk}\\

\haiku{{\textquoteleft}Wij moeten tot het.}{laatste lid uiteengerukt}{en verstrooid worden}\\

\haiku{Ik bemerkte dat.}{zijn spot hem evenzeer wondde}{als hij het mij deed}\\

\haiku{Geuren van versch hooi.}{en windbewogen water}{woeien ons tegen}\\

\haiku{De boerin schonk ons,.}{koffie in wijde witte}{kommen zonder oor}\\

\haiku{De {\textquoteleft}claque{\textquoteright} zette.}{in en vijf minuten lang}{brulde de kelder}\\

\haiku{De deuren gaan open....}{en we komen achter een}{sleepboot te hangen}\\

\haiku{Misschien rijten de.}{rotspieken van verzonken}{bergketens haar open}\\

\haiku{Wat er van onze,.}{vloot nog over was is bij dit}{treffen vernietigd}\\

\haiku{Nieuw-Guinea moet.}{hun springplank naar het vijfde}{werelddeel worden}\\

\haiku{Toen ik naar den trein,.}{ging keken de verkoolde}{dakspanten me na}\\

\haiku{Hoe langer hoe meer.}{gaan de omstandigheden}{op mij invreten}\\

\haiku{Aan het station.}{probeerden vrouwen door het}{cordon te komen}\\

\haiku{Maar die velen zijn,.}{stil verslagen en ontzet}{naar huis gegaan}\\

\haiku{En in den nacht zag.}{ik weer de lichten dooven op}{het emplacement}\\

\haiku{16 Juni Kort zijn,.}{de dagen al reikt het licht}{tot ver in den avond}\\

\haiku{{\textquoteleft}Denk je werkelijk,?}{dat Engeland het z\'o\'o ver}{zal laten komen}\\

\haiku{Alleen maar om weer;}{eens te kunnen praten met}{een ouden makker}\\

\haiku{Hij zag de oogen van.}{de donkere vrouw aan de}{bar van Caliente}\\

\haiku{Voor \'e\'en nacht vergeet.}{je den heelen beestenboel}{met zoo iets naast je}\\

\haiku{Dat stadium is.}{bij Leningrad al maanden}{geleden bereikt}\\

\haiku{Mussert werd door Seyss {\textquoteleft}{\textquoteright}.}{benoemd totleider van het}{Nederlandsche volk}\\

\haiku{Ik dacht, dat u het.}{op het laatste  moment}{niet aangedurfd had}\\

\haiku{{\textquoteleft}Je moet wel dwaas zijn,.}{als je je zonder meer aan}{dat tuig uitlevert}\\

\haiku{Hoe goed was het nog,,.}{amper een week geleden}{toen je bij me was}\\

\haiku{Ze zullen wel gauw,.}{opengaan want met een schop heb}{je nog nooit gewerkt}\\

\haiku{In de doos van mijn.}{kamer werden beelden en}{geluiden anders}\\

\haiku{Dan kwamen er twee;}{die met stukken hout in een}{blikje rammelden}\\

\haiku{Wat achterbleef is:}{een jong mensch die in zijn geest}{alleen bewaard heeft}\\

\haiku{er waren leeraren,:}{machinisten en weer twee}{letterkundigen}\\

\haiku{Ik hoorde hoe zij,.}{gevlucht was van het eene adres}{naar het andere}\\

\haiku{11 November Het.}{Roode Leger rukt op over}{het geheele front}\\

\haiku{Maar de soldaat kan?}{immers in het kansspel van}{den veldslag winnen}\\

\haiku{Deze tijd laat zich.}{slechts verbannen voor den duur}{eener omhelzing}\\

\haiku{Hij rukt mij los uit {\textquoteleft}{\textquoteright}.}{de Tweede Wereld met een}{enkel brandend feit}\\

\haiku{Duizend geruchten.}{leven een wisselend doch}{hardnekkig bestaan}\\

\haiku{grenzen, steden en,.}{landschappen uitwisschend tot}{chaos brandend}\\

\haiku{16 Juni Tot den.}{avond met zware hoofdpijnen}{te bed gelegen}\\

\haiku{De zwaarte van de,.}{kogels trok mijn bovenlijf}{omlaag steeds dieper}\\

\haiku{Gr\"une, Gestapo.}{en S.D. maken elken dag}{nieuwe slachtoffers}\\

\haiku{Fransche hoeren en.}{verraders trokken met den}{vluchtenden mof mee}\\

\haiku{Maar de Duitscher geeft, {\textquoteleft}{\textquoteright}.}{toe dat hij zijn troepen heeft}{teruggenomen}\\

\haiku{De gruwelen van.}{den t\`echnischen krijg zullen}{intenser worden}\\

\haiku{Een enkele maal.}{profiteeren we van het}{kaarslicht bij vrienden}\\

\haiku{En Goebbels haalt er.}{uit wat er agitatorisch}{uit te halen valt}\\

\haiku{Ze krijgen dien mof{\textquoteright},.}{er toch maar niet onder deed}{vandaag weer opgeld}\\

\haiku{De oude gevels.}{blikken meewarig in den}{triesten schemeravond}\\

\haiku{stootend en bassend.}{en gillend kondigden zij}{het nieuwe uur aan}\\

\haiku{Gloeiende lava...}{schuift steden voor zich uit of}{het schaakstukken zijn}\\

\haiku{Het medelijden,,.}{dat diep in je woont krijgt geen}{kans op te komen}\\

\haiku{Straks vervallen we.}{tot een staat van volmaakte}{onverschilligheid}\\

\haiku{Gisteren knalden.}{hier op de Nieuwmarkt nog de}{pistolen der groenen}\\

\haiku{15 April Roosevelts.}{overlijden is een slag voor}{de democratie}\\

\section{Theun de Vries}

\subsection{Uit: De freule. De bijen zingen}

\haiku{Over de vader van.}{het ongeboren kind werd}{niet meer gesproken}\\

\haiku{De andere hand.}{van de freule gleed door de}{zijsleuf van haar rok}\\

\haiku{Het rumoer om de.}{doop van Justus Wiarda}{wilde niet sterven}\\

\haiku{Ze zweefde boven.}{de beklemmende volte}{van de kerkdiensten}\\

\haiku{Het scheen veeleer, of.}{Bely haar misstap met de}{dood bezegeld had}\\

\haiku{Hester wordt ouder,{\textquoteright},.}{zeiden de boden die haar}{van nabij zagen}\\

\haiku{Toen zag Justus haar;}{mondhoeken smartelijk en}{verwonderd beven}\\

\haiku{Justus kleurde en.}{veegde de handen schoon aan}{zijn ruwe hozen}\\

\haiku{Toen zag hij ook, dat,.}{de schaduw een vrouw was al}{leek dat eerst niet zo}\\

\haiku{Hij zag de verre;}{ruitergestalte voor zich}{uit op het vrije veld}\\

\haiku{Op de grond lagen,,.}{scherven her en der gespat}{beslagen met smook}\\

\haiku{Hester Wiarda.}{ontwaakte niet meer tot het}{gewone leven}\\

\haiku{Dan luisterden de.}{bewoners van de state}{met een huivering}\\

\haiku{Zij haalden er het,.}{bezit vandaan dat de mens}{eigenwaarde geeft}\\

\haiku{De anderen zien:}{verwonderd naar de boer en}{houden op met eten}\\

\haiku{Niemand let op de,,.}{grootmoeder die alleen zit}{in haar stoel en zwijgt}\\

\haiku{{\textquoteleft}Geen gasten om te,,.}{drinken Aesger Wiarda}{en geen wigeters}\\

\haiku{Als scherpe stekels.}{vliegen de eerste splinters}{rechtop uit het hout}\\

\haiku{Haar handen liggen.}{gekromd en roerloos op de}{leuning van de stoel}\\

\subsection{Uit: 60. Keuromnibus}

\haiku{David verdiepte;}{zich niet in het intieme}{leven der blanken}\\

\haiku{veranderingen.}{waren desondanks voor hem}{en allen op til}\\

\haiku{daarop verdwenen.}{de twee onbekenden snel}{over de terreingolving}\\

\haiku{er ging alleen een.}{onbehaaglijk deinen door}{hun gelederen}\\

\haiku{de nacht zoog alle;}{geluiden van hemel en}{zee en eiland in}\\

\haiku{Ze schreed met een zwaard,}{in de ene een fakkel in}{de andere hand.}\\

\haiku{ze was de vrijheid -.}{van de negers het einde}{van de willekeur}\\

\haiku{Het nerveus gebaar.}{liet de ganzeveren pen}{op de grond rollen}\\

\haiku{David zag hem een:}{der laden opentrekken en}{deed een stap terug}\\

\haiku{Maar hij had geen zin;}{terug te keren naar de}{keuken vol vrouwen}\\

\haiku{Ik nodig je uit;}{morgen in de voormiddag}{bij mij te komen}\\

\haiku{Massou keerde zich,.}{om Scipion trok hem een}{paar pas ter zijde}\\

\haiku{Voor de tweede maal.}{op die dag vroeg men hem of}{hij betrouwbaar was}\\

\haiku{Ofschoon je misschien.}{niet met al onze plannen}{op de hoogte bent}\\

\haiku{Justin was alleen.}{zwijgzaam en bleek en David}{wist nu de reden}\\

\haiku{{\textquoteright} M. Beaupr\'e liet zijn.}{hagelwit gebit in een}{hernieuwd lachen zien}\\

\haiku{{\textquoteleft}Monsieur, ik zie,....}{in dat ik verkeerd deed ik}{kon het niet helpen}\\

\haiku{{\textquoteleft}Zo zie je, wat het...}{weldoen van deze zwarte}{schurken oplevert}\\

\haiku{M. Falconet scheen;}{te voelen dat zijn laatste}{opmerking doel trof}\\

\haiku{maar zoveel kan ik{\textquoteright} ():}{je wel zeggenzijn stem rees}{triumfantelijk}\\

\haiku{Was M. Malenfant?}{feitelijk niet voor hem in}{de bres gesprongen}\\

\haiku{maar het opschuiven.}{naar de zijwaarts gestrekte}{tak was moeilijker}\\

\haiku{Hij voelde Justins blik,.}{krampachtig op zijn gezicht}{dwingend om antwoord}\\

\haiku{klaarblijkelijk had.}{men daardoor de bagage}{naar buiten gesleept}\\

\haiku{Maar hij bedacht dat;}{de brand niet bij het koetshuis}{zou blijven stilstaan}\\

\haiku{Een van de spelers;}{zat met het hoofd tussen de}{schouders gedoken}\\

\haiku{Massou's gezicht was.}{star op het gezicht van de}{zingende gericht}\\

\haiku{zij wankelde, haar,;}{armen zakten haar zingen}{brak middenin af}\\

\haiku{II David voelde.}{hoe een naakte voet hem in}{de ribben schopte}\\

\haiku{Het duurde maar kort.}{of er kwamen enkele}{negers uit het bos}\\

\haiku{{\textquoteright} David wilde een,.}{kreet slaken een beweging}{van verzet maken}\\

\haiku{{\textquoteright} zei David, schor van, {\textquoteleft}!}{een nieuwe aandoeningmaak}{mijn handen toch los}\\

\haiku{David voelde zich;}{opnieuw voortgestompt en liet}{zich willig drijven}\\

\haiku{Op dit ogenblik kwam,:}{de schildwacht terug en zei}{naar Massou gekeerd}\\

\haiku{{\textquoteright} {\textquoteleft}Men spreekt mij niet aan...}{alsof ik een royalist}{en een planter ben}\\

\haiku{Maar ik herinner;}{me het aantal zwarten hier}{ruw genomen wel}\\

\haiku{Ik heb nooit iets om,.}{mijn meester gegeven of}{de vrijheid veracht}\\

\haiku{Eindelijk boog zich,:}{Massou naar hem over schraapte}{zacht zijn keel en vroeg}\\

\haiku{En iedereen weet.}{nu dat wij beiden zonen}{van Boucman zijn}\\

\haiku{Scipion had ze,.}{en ik heb ze aan massa}{Hugues gegeven}\\

\haiku{er was geen ruimte;}{voor de tegenstanders meer}{om zich te weren}\\

\haiku{David wendde de,.}{blik af van het bebloede}{vertrokken gezicht}\\

\haiku{Hij deed het als had.}{hij de hele ochtend op}{dit ogenblik gewacht}\\

\haiku{Er lagen messen,,;}{naast hen op stokken gesnoerd}{een enkel pistool}\\

\haiku{Hij keek de kamer;}{niet in voor alle luiken}{geopend waren}\\

\haiku{Ik zal mijn best doen,.}{het u aan de rest niet te}{laten ontbreken}\\

\haiku{Er bestaan veel te -?}{weinig afbeeldingen van}{onze West niet waar}\\

\haiku{{\textquoteright} zei David bitter, {\textquoteleft} '.}{omdat ik int geheel}{niet voor hem bestond}\\

\haiku{Hugues bleef voor de.}{kaart staan en sloeg er driftig}{met de hand tegen}\\

\haiku{als zij hun zaken.}{besproken hebben is het}{tijd voor het souper}\\

\haiku{{\textquoteleft}Hm,{\textquoteright} zei hij, {\textquoteleft}als je,...}{toch drinkt kunnen we het net}{zo goed samen doen}\\

\haiku{Het was niet in de.}{eerste plaats angst die hem dreef}{om licht te maken}\\

\haiku{{\textquoteleft}Ja,{\textquoteright} zei ze, de blik, {\textquoteleft}.}{op David richtendzo zag}{de kamer er uit}\\

\haiku{Hij droeg de blauwe.}{jas niet die de blanken hem}{gegeven hebben}\\

\haiku{Na die ene avond had.}{David evenmin meer iets van}{Perle vernomen}\\

\haiku{Wij zijn het die haar,.}{moeten veranderen door}{zelf anders te zijn}\\

\haiku{Hij stond naast haar en.}{wees haar alles zoals men}{een kind iets uitlegt}\\

\haiku{{\textquoteleft}Maar we zouden weer,.}{vijanden kunnen worden}{als je niet oppast}\\

\haiku{{\textquoteleft}Maar ik wist niet dat.}{de vaudoux jou tot koning}{gekozen hadden}\\

\haiku{Jij wist het niet, en,,.}{Bastiat die een zeldzaam}{mens was wist het niet}\\

\haiku{Er is er \'e\'en bij,.}{die ik in mijn  leven}{niet weer hoop te zien}\\

\haiku{David herkende,.}{het eerst de lange zorgzaam}{geklede Chr\'etien}\\

\haiku{David hief ook het,.}{glas maar minder overmoedig}{dan de anderen}\\

\haiku{Ik geloof niet dat,}{een kunstenaar dat ooit van}{zichzelf kan zeggen}\\

\haiku{Ik moet nog heel veel,...}{leren voor ik een meester}{in de kunst kan zijn}\\

\haiku{Perle was volmaakt,.}{kalm en wat zij deed deed zij}{trouw aan een bevel}\\

\haiku{{\textquoteright} Scipion wreef zich,.}{langs de wang waarbij hij de}{ene mondhoek optrok}\\

\haiku{Als ze gegrepen.}{worden maken ze grote}{kans op het schavot}\\

\haiku{Af en toe bleef hij,.}{staan hield zich de zijden vast}{en schudde het hoofd}\\

\haiku{Het stemde hem niet.}{gerust terug te vallen}{op zijn verachting}\\

\haiku{Hij hees zich op toen,.}{de magere strogele}{jongen voor hem stond}\\

\haiku{{\textquoteleft}Het schijnt niet, idioot,,{\textquoteright}:}{het is waarna hij zichzelf}{had toegedronken}\\

\haiku{Kankrin bemerkte:}{dat Gleb Michailowitsj hem van}{terzijde opnam}\\

\haiku{Monumentale,.}{jonge kerel u moet hem}{zich herinneren}\\

\haiku{Hij zwaaide een keer.}{met de hoed alvorens bij}{die weer opzette}\\

\haiku{{\textquoteleft}Maar George - als,!}{dat goddome zo is moet}{je naar de dokter}\\

\haiku{geklodderd als bij, '.}{een kind dat voort eerst een}{penseeltje vasthoudt}\\

\haiku{{\textquoteleft}Afgezien nog van,,.}{de vraag of het geen schep geld}{heeft gekost dat raam}\\

\haiku{van Theo natuurlijk,).}{jij zult nooit een cent met je}{geknoei verdienen}\\

\haiku{ze hadden de kleur;}{en bijna de vorm van oud}{knoestig geboomte}\\

\haiku{Est-ce que,}{je ne m\'erite pas mon}{pain parce que je}\\

\haiku{Hij was zijn eigen,.}{leermeester geworden nu}{niemand hem meer hielp}\\

\haiku{In het hospitaal;}{drong de zomerhitte langs}{de gangen binnen}\\

\haiku{Wat schokte hem, wat?}{vonkte in zijn blik als een}{onveilig signaal}\\

\haiku{naar het atelier, de,.}{omgeving het nieuwe huis}{keek hij nog steeds niet}\\

\haiku{{\textquoteleft}Ik geef het toe, broer -;}{het leven was niet bijster}{moederlijk voor je}\\

\haiku{Hij drukte die kort,,.}{vermanend nu Vincents stem}{weer knersend aanzwol}\\

\haiku{Ik zou gelukkig, -.}{willen zijn op hun manier}{Theo maar ik kan niet}\\

\haiku{Al was er geen peil,.}{op te trekken waarheen dit}{alles zou leiden}\\

\haiku{Goed, ik ben weer in,.}{actie ik zal nog veel}{actiever worden}\\

\haiku{Ik wil de mensen,,.}{bereiken hen bew\'egen}{hun harten raken}\\

\haiku{Theo had er ditmaal,.}{om gevraagd hij was er in}{geen jaren geweest}\\

\haiku{De kleur vliegt tussen...}{je vingers door terwijl je}{haar denkt te grijpen}\\

\haiku{{\textquotedblright} en ze jankte en,,,:}{zei dat ze haar best zou doen}{en ik zweer je Theo}\\

\haiku{Uit een zijpad in.}{de Bosjes kwam het geluid}{van een belletje}\\

\haiku{{\textquoteleft}Nu, zoek dat dan maar...,{\textquoteright}.}{uit wat Vincent al bijna}{weer kwaad had gemaakt}\\

\haiku{{\textquoteright} Carolus stond nog,.}{op dezelfde plek leunend}{op de paraplu}\\

\haiku{de marines, de,,.}{straatschetsen de bosstudies}{de olieverfschetsen}\\

\haiku{Zij dachten daar in...}{Nuenen dus niet enkel in}{vijandschap aan Sien}\\

\haiku{vooral Carolus -;}{stelde zich daarbij weer aan}{als rijkemans aap}\\

\haiku{Hij riep Maria aan;}{tafel en zette haar de}{aardappelen voor}\\

\haiku{het ogenblik waarop,.}{het jaar 1882 in het graf zou}{glijden naderde}\\

\haiku{hij luisterde met;}{enige angst of Maria soms}{wakker zou worden}\\

\haiku{Theo zal haar daaraan.}{laten opereren zodra}{ze is aangesterkt}\\

\haiku{Vincent zelf at vaak,.}{al minder om haar meer te}{kunnen opscheppen}\\

\haiku{Hij had Theo voor de.}{zoveelste maal om extra}{hulp moeten vragen}\\

\haiku{alles was hem zwart,,}{voor ogen geworden hij gleed}{een afgrond binnen}\\

\haiku{Ja, een home had,.}{hij al leek het vaak meer op}{een schip in noodweer}\\

\haiku{Zij scheen ten slotte,.}{voorbestemd te zijn Theo's}{leven te delen}\\

\haiku{Hij lag vaak wakker,;}{met de vrees dat hij niet door}{de muur kon breken}\\

\haiku{En als ik Sien nou...{\textquoteright}.}{in huis nam Vincent rukte}{zich eindelijk los}\\

\haiku{Hij schoof het kussen,.}{onder haar hoofd en wachtte}{tot zij bedaard was}\\

\haiku{Vincent was vaak al,.}{bij het eerste licht op de}{weg en tekende}\\

\haiku{Vincent kreeg al in,.}{de trein maagpijn de overvloed}{was te groot geweest}\\

\haiku{Het kostte hem twee.}{etmalen om weer in het}{gareel te komen}\\

\haiku{wij schilders zijn net...}{zo grillig in ons wel en}{wee als de vrouwen}\\

\haiku{{\textquoteright} Breitner  keek hem,.}{nog steeds verbaasd aan tot hij}{lichtjes grinnikte}\\

\haiku{Wat vond oom ervan,?}{was er in Amsterdam voor}{zo iets geen afzet}\\

\haiku{Hij ging met het pak.}{onder de arm op loden}{schoenen naar de Plaats}\\

\haiku{Zijn ingewanden.}{krompen samen van vrees voor}{het onbekende}\\

\haiku{turfdragers - dus De!}{Bock had ook die veenderij}{in het duin ontdekt}\\

\haiku{het beste is dat...}{soort maar te belazeren}{voor ze het jou doen}\\

\haiku{deze armoede!}{is een dorenheg waaraan}{ik mij kapot schram}\\

\haiku{Theo kwam naast hem staan,.}{een tengere stille hand}{op Vincents schouder}\\

\haiku{{\textquoteleft}Ik blijf kort deze,,.}{keer Vinc morgenavond ga ik}{naar Nuenen terug}\\

\haiku{Maar Brabant kan mij.}{plotseling verschijnen als}{het rustgevende}\\

\haiku{Ik besef het en...{\textquoteright}.}{ik verzet me nog steeds Theo}{trok hem naar zich toe}\\

\haiku{Maar men jaagt niet naar,.}{de straat terug wat men van}{de straat heeft gered}\\

\haiku{Hij zag er dubbel:}{sterk en gezond uit met zijn}{diep gebruinde huid}\\

\haiku{Vincent begon Van:}{der Weele opnieuw uit te}{vragen over Drente}\\

\haiku{hij wendde de ogen.}{weg voor de aanblik van die}{bleekrode open mond}\\

\haiku{Hij herinnerde,.}{zich zijn kinderangst voor de}{nacht lang geleden}\\

\haiku{{\textquoteright} De bazige vrouw.}{snipte verachtelijk twee}{vingers langs elkaar}\\

\haiku{er kwam een eigen,.}{sentiment in de stukken}{die hem weer hoop gaf}\\

\haiku{sommigen kwamen;}{om hem heen staan als hij zijn}{aquarellen maakte}\\

\haiku{{\textquoteleft}Een vrouw van dertig.}{wordt niet meer door vader of}{moeder gegeven}\\

\haiku{De illusie, dat,.}{zij elkaar pas gevonden}{hadden werd sterker}\\

\haiku{Sien had hem al tot,.}{zich genomen terwijl zijn}{trots zich nog weerde}\\

\haiku{{\textquoteleft}We zien elkaar dus,.}{nog aan het station als}{ik naar Drente ga}\\

\haiku{{\textquoteleft}Als ik hier in de,;}{Wildhoek kom kan ik Lolkje}{niet voorbijlopen}\\

\haiku{ja, de laatste maal.}{had de mijnheer Wigle niet}{eens meer ontvangen}\\

\haiku{en het eerste jaar.}{wou hij ook nog gratis mest}{over het nieuwe land}\\

\haiku{Alle Wildhoekster;}{jongens en meisjes kenden}{elkaar van die school}\\

\haiku{wij van dat goedje,;}{hebben want wanbetalers}{zijn het dikwijls ook}\\

\haiku{Maar meteen waren;}{ook weer de hoogmoed en het}{boos verzet in haar}\\

\haiku{{\textquoteright} Op Anders' lippen,:}{drong de vraag zo vaak hij langs}{Jannina zwenkte}\\

\haiku{{\textquoteright} begon Anders, en:}{onverhoeds daalde zijn stem}{schor en verbeten}\\

\haiku{daarachter rezen;}{als pluimpjes stilstaand gras de}{laatste restjes bos}\\

\haiku{{\textquoteleft}Jongens, is het waar,?}{dat Floris Hoogwolda werk}{maakt van Jannina}\\

\haiku{het water liep traag.}{van zijn gezicht terwijl hij}{naar hen luisterde}\\

\haiku{en toen geen van hen:}{meer iets te vertellen had}{lachte hij en zei}\\

\haiku{Het was duidelijk.}{dat hij op de aftocht van}{de oude wachtte}\\

\haiku{zij, Jannina, met, -?}{de rug tegen de deurpost}{haar ogen in de nacht}\\

\haiku{{\textquoteright} Jannina's adem bleef,.}{een oogwenk dralen daarop}{rukte ze zich los}\\

\haiku{Maar als ik zeg dat,.}{ie beter was dan jij dan}{spreek ik de waarheid}\\

\haiku{Hij zwaaide met de,,:}{korte armen wenkte om}{stilte betoogde}\\

\haiku{en toen ging Fok naar '...}{de voorkamer en hij kwam}{terug mett geld}\\

\haiku{Hij draaide zich om,;}{Floris mompelde ook iets}{dat op een groet leek}\\

\haiku{Zij zaten er in;}{het smalle reepje gras dat}{langs het water liep}\\

\haiku{Ze stapte langs de;}{eerste groep die haar zwijgend}{opnam en doorliet}\\

\haiku{In de stoffige.}{hete avondval stonden de}{turfstekers bijeen}\\

\haiku{Het lachen waarmee.}{de arbeiders hem groetten}{had iets goedmoedigs}\\

\haiku{De veldwachter liep,.}{drie vier passen met de stoet}{mee en bleef weer staan}\\

\haiku{Soms ging ze, als nam,.}{ze een onverhoeds besluit}{naast hem mee verder}\\

\haiku{Onder de hoge.}{bomen van de oprit bleef}{men voor het eerst staan}\\

\haiku{Ze kwamen in breed.}{gelid aandringen en nu}{stokten ze niet meer}\\

\haiku{Wybren Post vocht.}{tevergeefs om Jan Herder}{tegen te houden}\\

\haiku{Hij lag 's nachts niet,;}{van ellende op zijn brits}{te vloeken maar sliep}\\

\haiku{'t is geen zomer,{\textquoteright}.}{meer zei een stem vanachter}{het kale buffet}\\

\haiku{alles staat verkeerd,,,.}{op zijn kop een vergissing}{een bespotting}\\

\haiku{Hij zat nog stil toen,.}{zij het verhaal eindigde}{en dacht moeizaam na}\\

\haiku{we moeten met man...}{en macht ons recht op een stuk}{brood verdedigen}\\

\haiku{Kankrin ziet dan in.}{dat hij en zijn soort aan de}{dood vervallen zijn}\\

\haiku{Het werk werd voor mij.}{onder het schrijven het boek}{der ontdekkingen}\\

\haiku{In 1938 publiceert,}{hij Het rad der fortuin en}{in dat zelfde jaar}\\

\subsection{Uit: Noorderzon}

\haiku{mismaakte priesters,}{duldt de grote Alvader}{niet hij zelf is trots}\\

\haiku{Wierd Langskonk vroeg zijn;}{zoon nooit naar de waarheid van}{al dit beweren}\\

\haiku{Ik heb twee zonen,.}{verloren ik neem Rikelt}{aan in st\^ee van hen}\\

\haiku{Richt en Una stonden.}{bij de omvreding van de}{sate als beelden}\\

\haiku{En een enkele,,:}{zeer schrandere voegde er}{in zichzelf aan toe}\\

\haiku{Sommigen bouwden;}{mettertijd een eigen hut}{of onderkomen}\\

\haiku{Ywert was vrijwel zo,,.}{groot als Una zelf mager van}{bouw maar schouderbreed}\\

\haiku{en het kwam de smid,.}{voor dat ze dunner was en}{vol schroomvalligheid}\\

\haiku{Ze keerde naar huis,.}{terug nog ontdaner dan}{ze gekomen was}\\

\haiku{Toen de dag doorbrak,;}{stond het hele gewest in}{zijn landholten blank}\\

\haiku{gedurende die.}{blinde nacht dacht zij lange}{tijd aan niets anders}\\

\haiku{Zij baggerden door.}{blauwe zeedrab terug naar}{hun vernield bezit}\\

\haiku{Toen richtte zij zich,.}{op zodat alleman haar}{gezicht goed kon zien}\\

\haiku{Vrouwen en maagden,:}{breng palen en puntstokken}{en leg vuren aan}\\

\haiku{Er was \'e\'en van hen,,;}{een ouder man die dienst scheen}{te doen als priester}\\

\haiku{{\textquoteleft}Mij is aan Gabbe!}{noch aan welke kruisslaaf ook}{maar d\'at gelegen}\\

\haiku{Ze bespeurde de,;}{weifeling in Ingele's}{hand \'e\'en kort ogenblik}\\

\haiku{Una keerde zich in.}{de richting van de westwaarts}{gewentelde zon}\\

\haiku{Een ogenblik kwam het,;}{haar voor dat zij dit alles}{eerder beleefd had}\\

\haiku{{\textquoteright} Una stremde met een;}{wijd gebaar de woeling en}{het vormloos geschreeuw}\\

\haiku{{\textquoteleft}Als Thorlik jou tot,{\textquoteright}, {\textquoteleft}?}{vrouw nam zei zezou je hem}{als man begeren}\\

\haiku{{\textquoteright} Ingele hoestte,.}{weer en vertaalde grommend}{en met tegenzin}\\

\haiku{{\textquoteright} In het zeegewest.}{keerde men terug naar de}{hoeven en het werk}\\

\haiku{maar Orp zelf liet zich;}{de gunst van de wijze vrouw}{graag welgevallen}\\

\haiku{zij overlaadde hem,,.}{met sieraden kleren en}{geld als een edeling}\\

\haiku{Over de vader van.}{het ongeboren kind werd}{niet meer gesproken}\\

\haiku{De andere hand.}{van de freule gleed door de}{zijsleuf van haar rok}\\

\haiku{Het rumoer om de.}{doop van Justus Wiarda}{wilde niet sterven}\\

\haiku{{\textquoteleft}Hester wordt ouder{\textquoteright},,.}{zeiden de boden die haar}{van nabij zagen}\\

\haiku{Toen zag Justus haar;}{mondhoeken smartelijk en}{verwonderd beven}\\

\haiku{Justus kleurde en.}{veegde de handen schoon aan}{zijn ruwe hozen}\\

\haiku{Toen zag hij ook, dat,.}{de schaduw een vrouw was al}{leek dat eerst niet zo}\\

\haiku{Hij zag de verre;}{ruitergestalte voor zich}{uit op het vrije veld}\\

\haiku{Op de grond lagen,,.}{scherven her en der gespat}{beslagen met smook}\\

\haiku{Hester Wiarda.}{ontwaakte niet meer tot het}{gewone leven}\\

\haiku{Dan luisterden de.}{bewoners van de state}{met een huivering}\\

\haiku{Zij haalden er het,.}{bezit vandaan dat de mens}{eigenwaarde geeft}\\

\haiku{De anderen zien:}{verwonderd naar de boer en}{houden op met eten}\\

\haiku{Niemand let op de,,.}{grootmoeder die alleen zit}{in haar stoel en zwijgt}\\

\haiku{{\textquoteleft}Geen gasten om te,,.}{drinken Aesger Wiarda}{en geen wigeters}\\

\haiku{Als scherpe stekels.}{vliegen de eerste splinters}{rechtop uit het hout}\\

\haiku{Haar handen liggen.}{gekromd en roerloos op de}{leuning van de stoel}\\

\subsection{Uit: Het rad der fortuin}

\haiku{En Tjalling wist dat.,.}{Zo is de loop der dingen}{bedoelde Reinou}\\

\haiku{D\'aarom en daarom,.}{all\'een duldt ze hem hier op}{de begrafenis}\\

\haiku{Het staat voor Tjalling's,.}{ogen hij ziet zich opnieuw op}{de stelling zitten}\\

\haiku{Ze draaide zich op.}{de hakken om en verdween}{met lichte voeten}\\

\haiku{De domin\'e stond,.}{aan het hoofdeinde van de}{kist hoog opgericht}\\

\haiku{Op de gebulte}{diep-doorploegde inrit}{van het erf wachtte}\\

\haiku{- En Ekke... voor den.}{jongen zal ik dan een plaats}{bij den boer zoeken}\\

\haiku{Korzelig en met.}{spijt over zijn geldbelofte}{greep hij naar zijn pet}\\

\haiku{De winterlucht was,.}{geselend scherp ze wekte}{Tjalling nog eenmaal}\\

\haiku{Hij deed zijn zaken,,.}{en hij hielp mee in het werk}{al naar het seizoen}\\

\haiku{Het was een coup\'e;}{vol mannen in hun beste}{lakense pakken}\\

\haiku{waarheen het eerst te,,...}{gaan en hoe en of men bij}{elkaar zou blijven}\\

\haiku{Hij bleef des Vrijdags.}{langer in Leeuwarden en}{dronk meer dan voorheen}\\

\haiku{De scheidslijn met de.}{oude geboortestreek sneed}{dieper en dieper}\\

\haiku{Waar geld met geld zich{\textquoteright} -;}{paart zeiden de ouden en}{wijzen van de streek}\\

\haiku{Haar ademhaling in.}{hetzelfde bed lokte toch}{bij tussenpozen}\\

\haiku{de stad, offertes,,, -.}{beurs caf\'e partij biljart}{tot hij weer thuiskwam}\\

\haiku{hij praatte vaag over,.}{zijn zaak hij liep met Tjalling}{de boerderij door}\\

\haiku{Hij verlangde naar,.}{Rudmer met de weke zorg}{van ouderen broer}\\

\haiku{- Rudmer van Tjalling,;}{en Reinou wordt domin\'e}{zei men in de streek}\\

\haiku{- hij had de geest van,.}{het menniste volk hij zou}{zijn Meester vinden}\\

\haiku{emmers stonden op,.}{een plank die juist boven het}{water uitreikte}\\

\haiku{Hij gleed langs de wand,}{in het water de emmer}{nog steeds in de hand.}\\

\haiku{Hij schaamde zich en.}{was heimelijk trots op zijn}{nieuwe heldendom}\\

\haiku{Rudmer bemerkte, {\textquoteleft}{\textquoteright};}{datViet niet bizonder in}{tel was bij de rest}\\

\haiku{als je eerlijk wilt.}{zijn en de werkelijkheid}{onder het oog zien}\\

\haiku{- misschien word ik het...,...}{nog eens als ik de moed houd}{consequent te zijn}\\

\haiku{ik dacht 'n moment, ' -}{dat je heit metn heerschap}{uit de stad thuis kwam}\\

\haiku{aarzelende, half,;}{onderdrukte van Tjalling}{opener van Reinou}\\

\haiku{Hij lachte, en gaf:}{Rudmer een broederlijke}{klap op de schouder}\\

\haiku{- Kijk niet meer naar haar,....}{mijn jongen je moet straks naar}{Groningen terug}\\

\haiku{Rudmer vernam, dat}{zij alleen nog maar enige}{twijfel koesterden}\\

\haiku{kom, 't is zulk mooi,... -?}{weer ik neem de fiets En je}{ging naar Oostermeer}\\

\haiku{- Ze keerde zich naar,.}{hem toe zonder hem nog aan}{te durven kijken}\\

\haiku{Als ze verdomme,...!}{dan maar inzien dat ze geen}{vat op mij hebben}\\

\haiku{Antje's gezicht was.}{gaan gloeien bij de laatste}{woorden van Herre}\\

\haiku{Zijn knie\"en knakten,;}{de ouderdom zat al diep}{in zijn gebeente}\\

\haiku{- Zo -, zei hij, langzaam,,.}{en het drong tot hem door dat}{hij niet verheugd was}\\

\haiku{Totdal hij niet meer,:}{antwoordde op haar bange}{verwachtende vraag}\\

\haiku{Een man, die getrouwd, - -: - '? -.}{was en die Haastig opnieuw}{Kende Memm Nee}\\

\haiku{Het hield hem zozeer,.}{bezig dat hij het zelfs aan}{Antje vertelde}\\

\haiku{Het was, of er iets,.}{in Tjalling doorbrak dat hij}{er nooit had vermoed}\\

\haiku{- nee, zij is blond en,.}{groot het is het meisje van}{de harddraverij}\\

\haiku{Antje Adzers droeg;}{haar zwangere lijf als een}{schaamachtige last}\\

\haiku{De jongen beviel,.}{hem recht en slecht en niet van}{het brutale soort}\\

\haiku{Zijn magere hals.}{draaide onrustig boven}{het lage boordje}\\

\haiku{De lange trage.}{hand van Pieter lag op het}{dijbeen van de vrouw}\\

\haiku{De boeren klaagden -.}{nog al eens dat had mijnheer}{zeker ook gehoord}\\

\haiku{- Maar als-ie d'r laat, ' -!}{zitten dan zal ikm met}{mijn eigen handen}\\

\haiku{Aan het einde van.}{de tweede week vernam hij}{het allerlaatste}\\

\haiku{Je hebt centen voor -,,.}{het grijpen op die fabriek}{grijp ze dan stommerd}\\

\haiku{En Herre hoeft zo.}{hoog niet op te geven van}{Pieter's ongeluk}\\

\haiku{Dit was de kans, die,.}{hij nodig had die hij zo}{lang had nagejaagd}\\

\haiku{met een lang kleed om.}{zich heen en een wonderlijk}{gezicht zonder ogen}\\

\haiku{maar eindelijk kon,}{ze niet meer en gilde ze}{heel hard en rende}\\

\haiku{Hij leek op vader,;}{maar bij hem was alles twee}{keer zo breed en dik}\\

\haiku{maar Egmont jachtte.}{elke dag zijn huiswerk af}{en ging de straat op}\\

\haiku{Maar des avonds verscheen;}{tante Flora onverhoeds}{aan de Oude Gracht}\\

\haiku{Ze had gezien, dat;}{tante Flora en oom Lex}{het heel arm hadden}\\

\haiku{...elle me comprend,,!}{et mon coeur transparant pour}{elle seule b\'elas}\\

\haiku{Misgunde hij den?}{boerenzoon een dochter van}{het geslacht d'Aby}\\

\haiku{ze trilde korte,.}{tijd zo heftig in zijn arm}{dat hij er van schrok}\\

\haiku{Ze liepen in de,.}{rossige schemer naar huis}{verdoofd van geluk}\\

\haiku{Egmont was er bij.}{en oom Julien en Carla}{met Van Everdingen}\\

\haiku{Ze legt haar mes en,.}{vork harder naast het bord neer}{dan ze bedoeld had}\\

\haiku{Kleren... Natuurlijk,.}{voor een vrouw betekenen}{kleren altijd meer}\\

\haiku{Ik dacht, dat je in!}{de eerste plaats naar Holland}{wou voor je m\'oeder}\\

\haiku{wat ben je goed en!}{gewillig met me naar de}{eenzaamheid gegaan}\\

\haiku{Hij begon alle ',.}{bezwaren int veld te}{brengen die hij zag}\\

\haiku{Toen Ruth's noodschreeuw twee,.}{keer achtereen door het huis}{sneed kromp hij samen}\\

\haiku{{\textquoteleft}Ik denk nog net zo,.}{over de finanti\"en als}{toen je student werd}\\

\haiku{het gezin zat in,;}{de keukenkamer aan het}{tweede ochtendbrood}\\

\haiku{Gelaten en zwaar.}{dansten de paardeschoften}{bij zijn zweepslagen}\\

\haiku{Maar zij waren in,;}{de meerderheid en hier was}{hun spreekwijze wet}\\

\haiku{de wijdte, het dorp,,;}{de vaart waar de donkere}{tjalken door voeren}\\

\haiku{Hij wil je niet meer,,.}{zien anders begaat ie een}{moord aan je zeit ie}\\

\haiku{Maar jongen, jongen,:}{het leven ligt er voor \'ons}{nu eenkeer zo voor}\\

\haiku{Tjalling's mondgroeven,.}{verdiepten zich terwijl hij}{de weg aftuurde}\\

\haiku{Van Februari.}{af gaat dat nou al. Klappen}{krijgen ze niet meer}\\

\haiku{hoe vinden ze  ...?}{nou bij jullie de stappen}{van Abraham Kuyper}\\

\haiku{- En ik geloof, dat ',...}{Ryken van Sake Nieuwboer}{n plaats zoekt met Mei}\\

\haiku{En toch was er in,.}{hem iets dat oom en tante}{niet vergeven kon}\\

\haiku{Hij sliep in bij het.}{regelmatig metalen}{zingen der wielen}\\

\haiku{maar kon men op zo,?}{iets een fabriek bouwen die}{duizenden kostte}\\

\haiku{Oom Gevert is dood,,...}{maar oom Hebbert leeft nog zei}{mijn ouwe pake}\\

\haiku{Hij hoorde haar de:}{adem even inhouden achter}{de kinderwagen}\\

\haiku{Van zulken hoeven!}{wij ons toch niet op de kop}{te laten zitten}\\

\haiku{Maar allengs trok toch;}{het bedrijf hem weer in zijn}{opwindende sleur}\\

\haiku{Herre zweeg, en zijn.}{hand gleed onzeker langs de}{knopen van zijn vest}\\

\haiku{- ...Jij denkt nog steeds, dat,,.}{ik gek ben zei Rudmer die}{weer op en neer liep}\\

\haiku{Hij keek Herre niet,.}{aan zijn vingers draaiden de}{sleutelring sneller}\\

\haiku{Het was al laat, toen.}{hij Rudmer de weg naar de}{logeerkamer wees}\\

\haiku{Voor 't overige.}{bleef natuurlijk elk van hen}{baas in eigen huis}\\

\haiku{Ze dorst niets zeggen,.}{ofschoon bevreemd verwijt haar}{naar de lippen kwam}\\

\haiku{Het had hem koud en,.}{warm langs de rug gelopen}{terwijl Tjisse sprak}\\

\haiku{Maar d'r mankeert aan.}{je redenering toch nog}{het een en ander}\\

\haiku{Het zwijgen van den:}{geslepen schraper was m\'e\'er}{dan een afwachting}\\

\haiku{de mismoedigheid.}{over het geval met Tjisse}{woog enkel zwaarder}\\

\haiku{Jezus, moeten we?}{dan door zo'n schietding in de}{ellende komen}\\

\haiku{Dat is 'm. Hij heeft,!}{tegen me gezegd dat ik}{er op passen moest}\\

\haiku{Een wild wraakzuchtig.}{besef van vrijheid en macht}{gloeide in Ekke}\\

\haiku{- 't Is droog, geloof,;}{ik zei hij in sullige}{onderworpenheid}\\

\haiku{De laatste woorden;}{van den vreemde drongen pas}{nu tot Ekke door}\\

\haiku{Hij wierp zich om, op,.}{zijn buik lag hij het gezicht}{in het beddegoed}\\

\haiku{De kruik in haar hand.}{trilde en spilde droppels}{op het tafelzeil}\\

\haiku{zijn hoofd suisde, hij,;}{kreeg een schok door het hele}{lijf zijn voet schoot uit}\\

\haiku{Hij waagde het niet,.}{op te staan in de boot en}{haar toe te roepen}\\

\haiku{Arjen Taekes sloeg,.}{de hand op tafel naast de}{overbekende kruik}\\

\haiku{Het huis was hem een -.}{verschrikking geworden de}{mensen nog veel meer}\\

\haiku{Hij dacht aan At, hun,.}{snelle wilde avondliefde}{en aan de toekomst}\\

\haiku{Toen hij nog op het,.}{richelend weidepad liep}{ging de deur al open}\\

\haiku{de kleine hond liep;}{onrustig rond op erf en}{weg en zocht zijn baas}\\

\haiku{en voor de harde.}{luidruchtigheid van Jel schrok}{hij niet meer terug}\\

\haiku{Ekke zat daar lang;}{en geslagen en wist met}{zijn handen geen raad}\\

\haiku{Zodra we het voor '.}{elkaar hebben mett huis}{ga ik naar hen toe}\\

\haiku{en het scheen, of  .}{hij met de gedachten steeds}{bij iets anders was}\\

\haiku{- Tjalling woelde bij.}{zulke gedachten en hield}{Reinou uit de slaap}\\

\haiku{Herre was ook geen,.}{boer gebleven en hem ging}{alles fortuinlijk}\\

\haiku{wat was zijn schuld, al,?}{had hij dan nooit geleerd in}{schuld te geloven}\\

\haiku{Op Tjisse Landman;}{had hij reeds onmiddellijk}{moeten vertrouwen}\\

\haiku{- 't Lijkt verdomd wel,!}{of ik alleen maar voor dien}{grootboer werken mag}\\

\haiku{- Marten Offinga,:}{sprak het speeksel in zijn pijp}{pruttelde driftig}\\

\haiku{Hij stond nog steeds op,.}{dezelfde plek verslagen}{en onberaden}\\

\haiku{Wychman wilde.}{voor hem zijn oefeningen}{niet onderbreken}\\

\haiku{Een goed pianist,!}{oefent minstens vijf uur per}{dag zegt meneer Wolf}\\

\haiku{Het was de taal van,.}{een andere wereld die}{daar gesproken werd}\\

\haiku{- Het was November,,.}{het regende de avonden}{waren triest en lang}\\

\haiku{Herre deed een stap,.}{in zijn richting maakte een}{bevelend gebaar}\\

\haiku{'t Schijnt, dat u hem.}{terwille van dien kleine}{het spelen verbiedt}\\

\haiku{VIII Wychman,.}{speelde Wychman volgde}{muziektheorie}\\

\haiku{E\'en blijft er, naast hem,.}{op wie tenslotte alle}{zorgen neerkomen}\\

\subsection{Uit: Sla de wolven, herder!}

\haiku{Urukagina opende.}{flauw de ogen en staarde over}{het weghellend land}\\

\haiku{hij mengde alles.}{met koude adem dooreen en}{schudde het weer neer}\\

\haiku{Maar zij dachten aan;}{hun kinderen en waren}{bang voor hun vrouwen}\\

\haiku{en iedereen prees;}{den oude om zijn goedheid}{jegens den jongen}\\

\haiku{Naarmate hij het,.}{dorp naderde werd het gras}{hoger en ruiger}\\

\haiku{Het duurde niet lang,,;}{of er volgde een tweede}{dorp en een derde}\\

\haiku{Urukagina zag, dat;}{er een veel oudere man}{naast de kudde liep}\\

\haiku{Maar hij wist, dat het -?}{onmogelijk was was hij}{niet als slaaf verkocht}\\

\haiku{{\textquoteright} Urukagina begon,.}{pas nu zelf te lezen wat}{er in de klei stond}\\

\haiku{Urukagina hoorde,;}{tot de oudste jongens die}{in het dorp bleven}\\

\haiku{Toen Zarzari weer,.}{opkeek staarde zijn blik in}{een andere tijd}\\

\haiku{{\textquoteleft}Ik geloof, dat die,{\textquoteright}.}{daarnaast beter in zijn vet}{zit merkte hij op}\\

\haiku{Zelfs in het duister,;}{kon de jongen zien dat zij}{mistroostig liepen}\\

\haiku{Je hebt niet anders.}{gedaan dan wat de priester}{je bevolen heeft}\\

\haiku{Er bleven meer dan -.}{twintig dieren dood de helft}{daarvan was van mij}\\

\haiku{De man, die tegen,.}{de vrouw had gesproken kwam}{hen traag tegemoet}\\

\haiku{- Een tweede priester;}{had zijn kleed afgeworpen}{en trad naderbij}\\

\haiku{alleen de rivier,;}{stroomde de rook kronkelde}{ijl en scherp van geur}\\

\haiku{en in zijn gehoor.}{wies het veelvormig gerucht}{tot een dolle plaag}\\

\haiku{Hij bukte zich en.}{nam uit een aarden schaal een}{verdorde wortel}\\

\haiku{de edele heer was,.}{slechts met nadruk gevraagd de}{boot te verlaten}\\

\haiku{{\textquoteright} - Zarzari's gezicht;}{kneep bijeen als een oude}{rimpelige vrucht}\\

\haiku{het water joeg hen.}{ver in het troosteloze}{binnenland terug}\\

\haiku{{\textquoteright} zei hij, {\textquoteleft}dan kun je.}{er op uit trekken tegen}{het wild gedierte}\\

\haiku{Als een man zal hij!}{te voorschijn treden uit de}{donkere kameren}\\

\haiku{In zijn jeugd lag hij,}{in het zinkende schip als}{volwassene lag}\\

\haiku{ze rende blind, nu,.}{eens naar de ene dan naar de}{andere zijde}\\

\haiku{de eerste trots op.}{zijn naakte mannenkracht kwam}{er voor in de plaats}\\

\haiku{Thuaa lag stil, als,.}{een schelp die van binnen uit}{wordt dichtgehouden}\\

\haiku{hij voelde, hoe zij,,.}{zich verhardend de knie\"en}{over elkaar klemde}\\

\haiku{vrouwen spoelden er;}{wat wasgoed en droegen het}{haastig naar binnen}\\

\haiku{- {\textquoteleft}'t Is immers nog,...}{geen jaar geleden dat ik}{zelf met wasgoed liep}\\

\haiku{Hij begreep het zo,.}{plotseling dat hij zich met}{een schok oprichtte}\\

\haiku{in het halfduister.}{van hun veld-bed sperden}{haar ogen zich stralend}\\

\haiku{{\textquoteright} vroeg hij niet zonder,.}{spot maar de welwillendheid}{was niet verdwenen}\\

\haiku{beiden deden ze,;}{alsof ze niets van Thuaa's}{zwangerschap wisten}\\

\haiku{Abi-ishar schreed, met,.}{Thuaa aan de hand langzaam}{naar Urukagina's hut}\\

\haiku{De koning, die met -;}{cederengeur is vervuld}{daalt af in zijn park}\\

\haiku{- \'een maagdje heb ik -;}{hem toegeleid wier hart een}{snarenspel gelijkt}\\

\haiku{ik stelde haar een,,}{bezwering voor opdat je}{een zoon zou hebben}\\

\haiku{Urukagina woelde,.}{als zat hij in een dichte}{doornstruik gevangen}\\

\haiku{zij banden hem weer,,.}{hielden hem in hun mazen}{braken zijn opvlucht}\\

\haiku{Hij hief de hand op,.}{alsof hij een lijflijke}{slag moest afweren}\\

\haiku{Iemand kwam na een.}{poos bij hem en trok hem weg}{van de dode vrouw}\\

\haiku{Zijn zwager sloeg de.}{ogen neer en sloot de handen}{krampachtig ineen}\\

\haiku{De priester sloot de.}{ogen en trachtte zich met een}{ruk te bevrijden}\\

\haiku{De razernij van, -!}{Adad25 bezweer ik over je als}{je niet loslaat Ai}\\

\haiku{Het is je eigen,...}{aanmatiging die Thuaa's}{dood heeft veroorzaakt}\\

\haiku{{\textquoteright} Hij lachte kort en,,.}{snauwend en het oog ging weer}{open vals als voorheen}\\

\haiku{Onder het werkvolk,...}{en de boeren zijn sterke}{vaardige gasten}\\

\haiku{Korte tijd overwoog,.}{hij de wegen der mensen}{weer te verlaten}\\

\haiku{{\textquoteright} Urukagina keek hem,.}{aan verslagen door deze}{nieuwe beproeving}\\

\haiku{{\textquoteleft}er was arbeid voor,{\textquoteright}.}{honderd handen meer had de}{opzichter gezegd}\\

\haiku{De opzichter stond,,.}{kalm en bruin voor Urukagina}{en schudde het hoofd}\\

\haiku{Ze wilden juist slaags,.}{raken toen de opzichter}{tussenbeide kwam}\\

\haiku{Een oudere man,,.}{met iets schichtigs kwaadaardigs}{en groens in zijn ogen}\\

\haiku{{\textquoteleft}Heb ik niet gezegd,?}{dat het een onbeschaamde}{steppenrekel was}\\

\haiku{- Urukagina keek naar,.}{de grote sterren met hun}{beweeglijk blauw vuur}\\

\haiku{maar zij hadden hun.}{stille en verheven macht}{over hem verloren}\\

\haiku{Eindelijk gingen,.}{ze verder rechtstreeks naar het}{kamp der kamelen}\\

\haiku{Urukagina was zo,;}{verwonderd dat hij zich niet}{van de plek roerde}\\

\haiku{er liep alleen een.}{vaag-ontsteld fluisteren}{door de omstanders}\\

\haiku{Hamman, die juist een,:}{gedroogde vijg in de mond}{stak zei slaperig}\\

\haiku{{\textquoteright} Samsunu wendde,,;}{zich naar den metgezel en}{deed of hij kwaad werd}\\

\haiku{{\textquoteleft}Waarom, poortwachter,?}{nam je van mij de schaamdoek}{mijner lendenen}\\

\haiku{{\textquoteleft}Ishtar daalde neer in,!}{de onderwereld maar zij}{kwam niet weer boven}\\

\haiku{- Besprenkel Ishtar met,!}{het water des levens en}{breng haar weg van mij}\\

\haiku{kooplieden drukten...}{hun zegel onder brieven}{en warenlijsten}\\

\haiku{Het gesprek scheen hem.}{op boosaardige wijze}{naar de zin te zijn}\\

\haiku{meestal hield hij;}{zich op een afstand van de}{gewone mannen}\\

\haiku{Maar deelt de tempel?}{van Ningirsu er niet in}{de eerste plaats in}\\

\haiku{De negers vroegen,;}{om vinnen en staart die hun}{werden toegestaan}\\

\haiku{de lucht werd weer leeg,,;}{half door wolken gerafeld}{half met zon bespeeld}\\

\haiku{Deze vervloekte -{\textquoteright},.}{oorden waar een boze geest}{heerst zei de leider}\\

\haiku{een voorstel daartoe...}{zou spoedig in de hofraad}{behandeld worden}\\

\haiku{hoe is zijn naam ook...}{weer die jonge man met het}{wolfsvel bedoel ik}\\

\haiku{daarop suisde hij.}{sprongsgewijs naar de hals van}{de krijsende kip}\\

\haiku{Waarom zouden wij?}{dan bij het leven aan die}{verschillen denken}\\

\haiku{Papsukal gaf hem;}{een tikje op de wang en}{verliet de kamer}\\

\haiku{{\textquoteright} Papsukal leegde.}{zijn schaal met palmwijn en nam}{hem aandachtig op}\\

\haiku{{\textquoteright} Papsukal boog zich.}{naar hem toe en legde een}{arm om zijn schouder}\\

\haiku{{\textquoteleft}Ik ben altijd goed,{\textquoteright}.}{geweest in het oplossen}{van raadsels zei hij}\\

\haiku{Urukagina kwam langs;}{de hoge achtermuur van}{Ningirsu's tempel}\\

\haiku{{\textquoteright} Papsukal stond snel.}{op en legde zijn handen}{op Urukagina's hoofd}\\

\haiku{Maar hij heeft het niet,:}{eens aan willen nemen en}{hun toegesproken}\\

\haiku{De mannen van de...}{Kraanvogelkreek konden hun}{verbazing niet op}\\

\haiku{Maar wie had hij dan,,?}{in de schemering van zijn}{hart in Shaksagh herkend}\\

\haiku{{\textquoteright} zei Urukagina zacht,.}{om de vreemd-wordende}{stilte te breken}\\

\haiku{Hij slikte het vlees,.}{weg stond op van de ligbank}{en liep naar haar toe}\\

\haiku{Hij schudde het hoofd,.}{wachtend tot zij haar angst zou}{hebben overwonnen}\\

\haiku{De aardezang der.}{krekels gonsde van alle}{zijden op hen aan}\\

\haiku{{\textquoteright} vroeg ze, een oogwenk;}{machteloos en moe in zijn}{omarming hangend}\\

\haiku{Voor het overige.}{bootste hij Papsukal in}{alle dingen na}\\

\haiku{Ook was het waar, dat;}{een dergelijk heldenstuk}{zich nooit had herhaald}\\

\haiku{bevestigde hij,,.}{schel waar hij indrukwekkend}{bedoelde te zijn}\\

\haiku{Het is mij gelukt,.}{om den voorlezer aan het}{spreken te krijgen}\\

\haiku{Je hebt een openbaar.}{ambt ontvangen en misbruik}{gemaakt van je macht}\\

\haiku{De vertedering,.}{draalde nog in zijn ogen maar}{zijn mond bleef even hard}\\

\haiku{hem thans van dichtbij.}{te ontmoeten scheen althans}{een gebeurtenis}\\

\haiku{fijne, striemende.}{sneeuwvlagen joegen af en}{toe uit de hemel}\\

\haiku{het was noen-tijd,.}{en de honger knaagde in}{de meeste magen}\\

\haiku{{\textquoteleft}Ik verklaar, dat het.}{ten laste gelegde mij}{als waarheid voorkomt}\\

\haiku{hij had slechts toe te...}{stemmen op wat hem in de}{mond werd gegeven}\\

\haiku{Bij den aanvang der.}{plechtigheid leek alles op}{vorige jaren}\\

\haiku{de troonopvolger.}{smeet den hoofdman der lijfwacht}{iets naar het gezicht}\\

\haiku{zijn mond sputterde,.}{zijn aderen werden donker}{boven de slapen}\\

\haiku{iedereen drong naar,,.}{voren op om te zien wat}{er gebeuren zou}\\

\haiku{Een veelvoudige,,}{verschrikte hier en daar met}{overspannen lachen}\\

\haiku{wat deert het mij, wie...}{er patesi is en wie}{de ambten verdeelt}\\

\haiku{{\textquoteleft}Ik kwam hier, om een,.}{vriend terug te vinden niet}{een tegenstander}\\

\haiku{{\textquoteright} Hamman klakte met.}{de tong en Sun-nasir's}{vuist viel op tafel}\\

\haiku{{\textquoteleft}Het ga je goed,{\textquoteright} zei,;}{hij ten slotte den ander}{de handen drukkend}\\

\haiku{hij keek naar het slijk:}{en de stenen onder zijn}{voet en mompelde}\\

\haiku{Maar de tweede schok.}{was van ingrijpender en}{noodlottiger aard}\\

\haiku{er dreigde voor den.}{patesi en de zijnen}{het hoogste gevaar}\\

\haiku{en dat alles door...{\textquoteright}}{een stompzinnig verzinsel}{van een zwarten hond}\\

\haiku{Lu-enna's duimen;}{begonnen wrevelig langs}{elkaar te schuiven}\\

\haiku{Er was niets meer, dat;}{Urukagina uitdreef naar de}{roerige wereld}\\

\haiku{Welke gedachten?}{gingen hem bij al deze}{dingen door het hoofd}\\

\haiku{Hij deed het, omdat;}{Shaksagh schijnbaar stil en peinzend}{aan zijn zijde lag}\\

\haiku{de opwelling van,...{\textquoteright}}{uw innerlijk spreek ze niet}{onmiddellijk uit}\\

\haiku{Laat het winnen van,...!}{dochters aan hen over die niet}{beter verdienen}\\

\haiku{De godin kent mij -{\textquoteright} -;}{niet meer Urukagina lei snel}{de hand op haar mond}\\

\haiku{Nu kan mijn borst het,:}{niet langer bergen nu moet}{ik er van spreken}\\

\haiku{Een der donkerste:}{bleef Urukagina's geheime}{zorg aangaande Shaksagh}\\

\haiku{Hij betoogde, dat.}{hij Urukagina's plannen met}{alle kracht steunde}\\

\haiku{Een verwarrende.}{beklemming dreef het bloed naar}{Urukagina's gezicht}\\

\haiku{slank, gespierd, fier, het.}{gezicht als brons onder de}{zilveren haardos}\\

\haiku{Hij bedekte de {\textquoteleft}?}{ogen met de hand.Jij en ik}{hebben hem gedood}\\

\haiku{Een tijdlang stonden.}{zij tegenover elkaar en}{bewogen zich niet}\\

\haiku{Zo even wist Bada,,;}{niets van de steden af ja}{hij verfoeide ze}\\

\haiku{Nadat ik deze,}{middag van u hoorde en}{u zag en wist wien}\\

\haiku{{\textquoteleft}Als men jong is, leeft,}{men in de verwachting naar}{het volkomene}\\

\haiku{Maar wie geroepen,.}{is en zich verbergt voor zijn}{god maakt zich schuldig}\\

\haiku{Eens heb je jezelf,.}{ingezet maar men heeft je}{daarin verhinderd}\\

\haiku{- Zij wendde het hoofd,,;}{af als hij die dingen zei}{en lachte niet meer}\\

\haiku{Dit was een dieper,.}{bezit het bezit dat van}{een vrouw een vrouw maakt}\\

\haiku{Maanden heb ik er,.}{over gepeinsd het verworpen}{en weer opgevat}\\

\haiku{{\textquoteright} - Hij zag haar lippen.}{zich vastberaden sluiten}{en weer vaneen gaan}\\

\haiku{Zij hebben dit met,...}{de ouderdom gemeen dat}{zij kunnen wachten}\\

\haiku{Van de sterren ging,.}{zijn blik naar de aarde nog}{eenmaal keek hij om}\\

\haiku{Ik weet niet, wat het.}{is en je behoeft het mij}{niet te vertellen}\\

\haiku{Intussen lichtte;}{de wind langzamerhand het}{neergestoven zand}\\

\haiku{Urukagina had er,;}{zelf nog geen voorstelling van}{hoe alles zou zijn}\\

\haiku{Zij hadden gezaaid,;}{en geplant maar de regen}{was uitgebleven}\\

\haiku{En Sun-nasir:}{gaf het met luid spotgehuil}{begroete antwoord}\\

\haiku{hij overwoog nog, of,.}{men niet terug zou gaan toen}{het al te laat was}\\

\haiku{Een van Urukagina's.}{metgezellen keerde zich}{om en rende weg}\\

\haiku{Vlak daarop floot een.}{speer en stiet naast hen op de}{harde bodem af}\\

\haiku{Het feit, dat hij nog,.}{leefde vervulde hem eerst}{met blijdschap en hoop}\\

\haiku{Het volgend ogenblik.}{landden zij temidden van}{veelvormig gerucht}\\

\haiku{- Hij zag dit, kalm en,,.}{koud vrij van wanhoop zij het}{niet van bitterheid}\\

\haiku{maar hij had die dood.}{niet verzoend door het geluk}{van een misbruikt volk}\\

\haiku{- De stilte en het,;}{nu weer gezeefde gele}{licht verrieden niets}\\

\haiku{een zuur, verbeten.}{lachje rimpelde als van}{ouds zijn spits mondwerk}\\

\haiku{Het was het wolfsvel,.}{dat men hem in Ab-Enki}{afgenomen had}\\

\haiku{Urukagina bracht snel,.}{de hand naar de gordel naar}{het vertrouwde mes}\\

\haiku{Urukagina bleef staan,.}{en zag dat het de mannen}{der ilku's waren}\\

\haiku{- Hij is Eanatum,...,!}{die herrezen is misschien}{Ur-Nina zelf}\\

\haiku{Bada knikte nog.}{eens en bracht toen de handen}{tegen het voorhoofd}\\

\haiku{nu zal het niet lang,...}{meer duren of ik reik nog}{maar tot je schouder}\\

\haiku{Daarop boog ze haar,.}{zware rode hoofd en ging}{voor de tweede maal}\\

\haiku{Ik werd onderweg,...}{overvallen door een vrouw zij}{sprak zo gebiedend}\\

\haiku{men ried onder de,.}{flardenmantel haar trotse}{rechte gestalte}\\

\haiku{{\textquoteright} Van vriendschap spreekt hij,,.}{niet meer dacht Urukagina een}{oogwenk verbitterd}\\

\haiku{{\textquoteleft}Ik heb niets meer te,,...{\textquoteright}}{wensen Urukagina dan een}{dood zonder pijnen}\\

\haiku{Zich daarom in hun,.}{stijl te vernederen zou}{dollemanswerk zijn}\\

\haiku{{\textquoteleft}Je weet, mijn gouden,,...}{ree dat ik oprecht tegen}{je spreek zoals steeds}\\

\haiku{Het kon geen kwaad, als...}{zij in de tussentijd niet}{verwaarloosd werden}\\

\haiku{Het leven, dat de,.}{stedelingen leidden was}{voor hem gesloten}\\

\haiku{Tegen hem leunend,;}{sloot ze de handen bijeen}{om het amuletje}\\

\haiku{{\textquoteright} Haar ogen fonkelden,.}{hem tegen opgetogen}{en bewonderend}\\

\haiku{dacht hij, de trap naar.}{het inwendige van het}{paleis afdalend}\\

\haiku{Je hebt de macht, hem,.}{uit zijn ambt te ontzetten}{als je dat verkiest}\\

\haiku{dat ze met jou en;}{je vrienden in holen en}{moerassen overnacht}\\

\haiku{Urukagina wachtte.}{een dag of wat en sprak toen}{met Sun-nasir}\\

\haiku{geloof jij, dat er?}{iets in Shaksagh's houding valt}{te veroordelen}\\

\haiku{Ook werd het alras,.}{bekend dat Shaksagh dit keer niet}{ingegrepen had}\\

\haiku{- Sun-nasir het.}{den schrijver gaan en begaf}{zich naar Urukagina}\\

\haiku{Sun-nasir en.}{Urukagina kwamen in de}{zuilengaanderij}\\

\haiku{De afwezigheid.}{van levende wezens was}{hier bijna drukkend}\\

\haiku{{\textquoteright} Urukagina zei het,.}{weifelend maar niet geheel}{zonder overtuiging}\\

\haiku{Urukagina hief de.}{sluier herkennend met de}{punt van de teen op}\\

\haiku{De stedelingen:}{hadden overal langs de weg}{naar de stoet gestaard}\\

\haiku{Hij hield even op, haar,:}{vingers te wrijven en ze}{vervolgde haastig}\\

\haiku{Hij hield de adem in,.}{als kon dit nietig geruis}{haar wakker maken}\\

\haiku{Waarom had zij zo?}{dringend naar de liefde van}{haar ouders gevraagd}\\

\haiku{Urukagina eist zelfs,!}{schadeloosstelling voor de}{vrouw die men verstoot}\\

\haiku{de priesters grijpen,, -{\textquoteright} {\textquoteleft}}{het gebod verachten dat}{zij ons overbrachten}\\

\haiku{Hij liet de handen.}{langs de zijden afhangen}{en knikte langzaam}\\

\haiku{Zij droeg voor het eerst,}{de brede met juwelen}{doorvlochten haarkroon}\\

\haiku{Achter die deuren.}{zou Amat-Bau uit zijn}{bestaan verdwijnen}\\

\haiku{De gevoelloosheid,.}{doordrong zijn denken maar ook}{de herinnering}\\

\haiku{Zij hingen over zijn,.}{gevoel als hadden zij een}{tastbare zwaarte}\\

\haiku{De gebeurtenis.}{bracht de uiteenlopendste}{beroering te weeg}\\

\haiku{In den beginne,.}{had hij gemeend Urtar te}{kunnen gebruiken}\\

\haiku{Hij keek den priester,.}{aan die de ogen veelzeggend}{naar buiten wendde}\\

\haiku{hij deed een grote,,.}{pas voorwaarts naar de deur om}{de hemel te zien}\\

\haiku{De zware trage;}{olie had het water met een}{goudtint doortrokken}\\

\haiku{Bijna zou ik het,;}{veroordelen dat mensen}{dromen en denken}\\

\haiku{Urtar hief zich op.}{de tenen en schreeuwde een}{langgerekt bevel}\\

\haiku{Hij rekte zich op.}{zijn rijdier en balde de}{vuist boven het hoofd}\\

\haiku{het wilde echter,;}{niet vol blijhartig en ruim}{worden als voorheen}\\

\haiku{Urukagina sprong van.}{zijn ezel en hief den grijsaard}{van de bodem op}\\

\haiku{Hij  trachtte het,.}{te overwinnen door veel met}{Bada te spreken}\\

\haiku{Je tracht voor me te,,}{verbergen dat het er slecht}{met de stad voor staat}\\

\haiku{hij had dit geluk,.}{verloren hij had zich niet}{kunnen verdelen}\\

\haiku{{\textquoteleft}Wat zoekt men hier in,?}{de nacht bij het altaar en}{verstoort de stilte}\\

\haiku{hij was er nu wel,.}{zeker van dat zij niet meer}{in de tempel was}\\

\haiku{de stank van dood stof,;}{en ontbinding zat in zijn}{haar in zijn gewaad}\\

\haiku{Niet de liefde van,...}{een vrouw of het bezit van}{een kind maar zijn doel}\\

\haiku{het beitelwerk, dat,.}{de muurrand vulde was vol}{grillige schaduw}\\

\haiku{Hier en daar liepen,.}{wagensporen tamelijk}{vers en herkenbaar}\\

\haiku{Nog een halve mijl.}{verder was het bevaarbaar}{voor een kleine boot}\\

\haiku{Bada zocht een taai.}{grasje en peuterde een}{graatje uit zijn kies}\\

\haiku{Bada wiste de,,.}{naam Urukagina die hij had}{geschreven weer uit}\\

\haiku{gemalin van god,}{Ningirsu beschermvrouw van}{Shirpurla B\^elili}\\

\haiku{63Talent = \ensuremath{\pm};}{30 kg. 64Weg van Ea =}{de zuiderhemel}\\

\subsection{Uit: W.A.-man}

\haiku{de haat doordrong hem,.}{diep en innig zij werd het}{merg van zijn bestaan}\\

\haiku{Hij moest zich met kracht,.}{beheersen diep ademhalend}{kwam hij tot zich zelf}\\

\haiku{Op dat ogenblik was,,.}{ook zijn moeder er vermoeid}{bits en geprikkeld}\\

\haiku{Frans' moeder keek strak.}{en behekst naar het mondstuk}{van de luidspreker}\\

\haiku{Misschien is het tot...}{mijnheer doorgedrongen dat}{we in oorlog zijn}\\

\haiku{Vogel liet de blik:}{veelzeggend van de een naar}{de ander gaan}\\

\haiku{Frans en Sudderam.}{wierpen zich tegelijk naar}{voren om to zien}\\

\haiku{Zij namen opnieuw,.}{een bocht Trewes reed als de}{baarlijke duivel}\\

\haiku{{\textquoteright} Zijn haat golfde loom,.}{op maar legde zich toen ze}{weer over straat liepen}\\

\haiku{zij slopen naar een;}{andere schuilplaats door de}{verduisterde stad}\\

\haiku{Zij gingen een voor.}{een naar buiten en klommen}{onder het zeildoek}\\

\haiku{Weer werd het donker,.}{een lucht vol stof en molm en}{teer viel zwaar op hem}\\

\haiku{Het deed pijn, maar hij.}{had niet de wil en de macht}{zich te verroeren}\\

\haiku{De sterren hingen.}{in tintelende reeksen}{boven Amsterdam}\\

\haiku{De mannen achter, {\textquoteleft}{\textquoteright}.}{het luik hoorden maar \'e\'en woord}{capitulatie}\\

\haiku{{\textquoteright} zei hij eindelijk, {\textquoteleft}.}{we m\'o\'eten weten wat er}{aan de knikker is}\\

\haiku{Opnieuw begon hij,,.}{te trillen hij moest blijven}{staan een paar tellen}\\

\haiku{vlak om de hoek was.}{het hellende smalle huis}{met het winkeltje}\\

\haiku{Hij trilde opnieuw,.}{de afschuw jegens zich zelf}{steeg hem naar de keel}\\

\subsection{Uit: Wilde lantaarns}

\haiku{{\textquoteleft}Als ik hier in de,;}{Wildhoek kom kan ik Lolkje}{niet voorbijlopen}\\

\haiku{ja, de laatste maal.}{had de mijnheer Wigle niet}{eens meer ontvangen}\\

\haiku{en het eerste jaar.}{wou hij ook nog gratis mest}{over het nieuwe land}\\

\haiku{Alle Wildhoekster;}{jongens en meisjes kenden}{elkaar van die school}\\

\haiku{in zijn hart was hij;}{bang voor het bezittende}{volk van de polder}\\

\haiku{Wieger naar stad, om,;}{er zelf bij te zijn als de}{ouwe wat insloeg}\\

\haiku{wij van dat goedje,;}{hebben want wanbetalers}{zijn het dikwijls ook}\\

\haiku{Maar meteen waren;}{ook weer de hoogmoed en het}{boos verzet in haar}\\

\haiku{{\textquoteright} - Op Anders' lippen,:}{drong de vraag zo vaak hij langs}{Jannina zwenkte}\\

\haiku{- {\textquoteleft}E\'en advocaatje,{\textquoteright},.}{drong Anders die het meisje}{voelde wankelen}\\

\haiku{{\textquoteright} begon Anders, en:}{onverhoeds daalde zijn stem}{schor en verbeten}\\

\haiku{daarachter rezen,,;}{als pluimpjes stilstaand gras de}{laatste restjes bos}\\

\haiku{{\textquoteleft}Jongens, is het waar,?}{dat Floris Hoogwolda werk}{maakt van Jannina}\\

\haiku{het water liep traag,.}{van zijn gezicht terwijl hij}{naar hen luisterde}\\

\haiku{en toen geen van hen,:}{meer iets te vertellen had}{lachte hij en zei}\\

\haiku{Het was duidelijk,.}{dat hij op de aftocht van}{den oude wachtte}\\

\haiku{die  leken op,.}{niets dat haar verbeelding of}{geheugen kende}\\

\haiku{zij, Jannina, met, -?}{de rug tegen de deurpost}{haar ogen in de nacht}\\

\haiku{{\textquoteright} - Jannina's adem bleef,.}{een oogwenk dralen daarop}{rukte ze zich los}\\

\haiku{Maar als ik zeg, dat,.}{ie beter was dan jij dan}{spreek ik de waarheid}\\

\haiku{Hij zwaaide met de,,:}{korte armen wenkte om}{stilte schreeuwde schor}\\

\haiku{Zij zaten er in,;}{het smalle reepje gras dat}{langs het water liep}\\

\haiku{de meesten groetten.}{niet terug en keken}{haar onwillig na}\\

\haiku{Ze stapte langs de,;}{eerste groep die haar zwijgend}{opnam en doorliet}\\

\haiku{In de stoffige.}{hete avondval stonden de}{turfstekers bijeen}\\

\haiku{Het lachen, waarmee,.}{de arbeiders hem groetten}{had iets goedmoedigs}\\

\haiku{De veldwachter liep,.}{drie vier passen met de stoet}{mee en bleef weer staan}\\

\haiku{Soms ging ze, als nam,.}{ze een onverhoeds besluit}{naast hem mee verder}\\

\haiku{Onder de hoge '.}{bomen van de oprit bleef}{men voort eerst staan}\\

\haiku{Ze kwamen in breed.}{gelid aandringen en nu}{stokten ze niet meer}\\

\haiku{Wybren Post vocht,.}{tevergeefs om Jan Herder}{tegen te houden}\\

\section{Beb Vuyk}

\subsection{Uit: Gerucht en geweld}

\haiku{We zaten onder.}{die regenboom en keken}{op naar de takken}\\

\haiku{Er zijn plaatsen waar.}{de tijdelijke dingen}{bestendigd worden}\\

\haiku{Als een verpleegster,.}{die bij een zieke waakt had}{ze toen gedacht}\\

\haiku{{\textquoteleft}Wat hebt u tegen,,?}{die zwarten kapitein u}{bent toch zelf ook zwart}\\

\haiku{{\textquoteright} zei hij tegen de.}{commandant die jaren op}{Java gewoond had}\\

\haiku{De puisten liepen,.}{door tot in de hals die iets}{vogelachtigs had}\\

\haiku{Alles wat bruin was.}{en aan de Hollandse kant}{stond noemden zij zo}\\

\haiku{Etty moet zich daar.}{tussen die Hollanders dood}{ge\"ergerd hebben}\\

\haiku{Hij zat op een stoel,.}{groenachtig in zijn gezicht}{en mompelde wat}\\

\haiku{In betere staat.}{zou het een seigneurlijke}{woning zijn geweest}\\

\haiku{{\textquoteleft}Ik begrijp dat dit.}{landschap je boeit en dat je}{voorlopig hier blijft}\\

\haiku{{\textquoteleft}Vroeger offerden,.}{wij geen dieren maar slaven}{en gevangenen}\\

\haiku{Het oude dorpshoofd.}{nodigde ons uit om het}{huis te betreden}\\

\haiku{Het dreunen van de.}{gongs deed alle begrip van}{tijd verloren gaan}\\

\haiku{Voor het helemaal.}{licht zou zijn moest ik in de}{rivier gaan baden}\\

\haiku{{\textquoteleft}Vroeger offerden,.}{wij geen dieren maar slaven}{en gevangenen}\\

\haiku{We bleven staan in.}{de hete zon aan de rand}{van de begraafplaats}\\

\subsection{Uit: Het laatste huis van de wereld}

\haiku{Max Havelaar in.}{zijn toespraak tot de Hoofden}{van Lebak}\\

\haiku{Menschen verdringen.}{zich aan de steiger en rond}{een ijzeren loods}\\

\haiku{De K.P.M.-boot is,.}{iets bijgedraaid ligt nu schuin}{voor onze Tandjong}\\

\haiku{De ramen zijn klein,,,.}{hoog en smal zonder glas maar}{met dikke luiken}\\

\haiku{Gaba-gaba.}{is de hoofdnerf van het blad}{van de sagopalm}\\

\haiku{Het water is hier,.}{zeer vischrijk maar de opbrengst}{in geld is gering}\\

\haiku{Wij vragen hem of.}{hij niet met zijn vijf kontjo's ons}{erf wil schoonmaken}\\

\haiku{Een van die oude.}{Hindoe-hoofden was een vriend}{van mijn schoonvader}\\

\haiku{Zesde hoofdstuk ONS.}{erf grenst aan het erf van den}{Bestuursambtenaar}\\

\haiku{Dan klimmen zij langs,.}{een smal pad de heuvel op}{bijna zonder groet}\\

\haiku{Twee dagen later, ',.}{s morgens met de zeewind}{steken we van wal}\\

\haiku{Veertig ketels heeft,.}{Batoeboi het wordt een tocht}{van twee dagen}\\

\haiku{Jij bedriegt en ik,?}{bedrieg waarom zouden we}{geen vrienden blijven}\\

\haiku{Er wordt geen rijst meer,,,;}{uitgegeven noch koffie}{noch geld noch pisang}\\

\haiku{Hij loopt met een groote.}{mand heen en weer tusschen de}{bloeboer en het vat}\\

\haiku{De jongens hebben.}{zich op hun zij gedraaid en}{ademen diep en luid}\\

\haiku{Toen we weggingen,,.}{ver na middernacht zat de}{bruid nog voor haar bed}\\

\haiku{het beest echter slaat.}{op de vlucht in de richting}{van de groote rivier}\\

\haiku{De linkerhelft van;}{onze tuin op de Tandjong}{is zeer onvruchtbaar}\\

\haiku{Heintje, zijn vrouw en,,.}{kinderen de tuinjongen}{iedereen rept zich}\\

\haiku{Maar als oplossing.}{van het kleedingvraagstuk is het}{aardig gevonden}\\

\haiku{Niet te begrijpen, {\textquoteleft},,{\textquoteright}.}{klankensenangal\'e}{agoos moelin\'ekrois}\\

\haiku{Wie kinderen krijgt,.}{moet ervoor betalen met}{geld of met de dood}\\

\haiku{Er staat een flinke.}{deining en de leege prauw zwalkt}{hevig op en neer}\\

\haiku{Regen dringt door het,,.}{atapdak twee- driemaal slaat}{een golf naar binnen}\\

\haiku{Ode Madi zit met.}{een ploeg op Toebahoni}{en stookt twee toengkoe's74}\\
