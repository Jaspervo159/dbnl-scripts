\chapter[15 auteurs, 1144 haiku's]{vijftien auteurs, elfhonderdvierenveertig haiku's}

\section{D.L. Daalder}

\subsection{Uit: Schimmenspel}

\haiku{over het gladde haar....}{sluit een wit plooikapje met}{heel fijne kantjes}\\

\haiku{Geen wonder, dat ze....}{een stoof nodig heeft om in}{evenwicht te blijven}\\

\haiku{het glas met water....}{vullen voor den dominee}{op Zondagmorgen}\\

\haiku{{\textquoteleft}Dominee is een -.}{grote man die moet goed eten}{om sterk te blijven}\\

\haiku{Ze kibbelen wel.}{graag samen en liefst over hun}{verschil in geloof}\\

\haiku{Drift flitst in mij op,,.}{maar ik heb niet de moed iets}{terug te zeggen}\\

\haiku{Bij Meester Douma;}{dringen de kinderen naar}{binnen met geweld}\\

\haiku{{\textquoteleft}Maar ik loop mank en.}{t\`och speel ik de baas over al}{die kwajongens hier}\\

\haiku{Ze kijkt met strenge.}{ogen over de klas en ze zegt}{heel scherpe dingen}\\

\haiku{Ik zie duidelijk,.}{dat Kees Keizer knikkers ruilt}{met Sime Vlaming}\\

\haiku{Dan valt m'n oog op.}{Rika Burger en het wordt}{me warm om het hart}\\

\haiku{En als ik op een,.}{van de knoesten klim kan ik}{net door het raam zien}\\

\haiku{Maar ik weet wel een.}{middel om hem aan zijn plicht}{te herinneren}\\

\haiku{dan krijgt ieder kind ' '.}{een cent voors morgens \`en}{\'e\'en voors middags}\\

\haiku{gekke Wullem van;}{Reyer Verdompeltje brengt}{ze mee uit Spiekdorp}\\

\haiku{{\textquoteright} En piekerend over.}{deze zonderlinge zaak}{vervolg ik mijn weg}\\

\haiku{ze zien duidelijk,.}{dat de blinkende koppen}{naderbij komen}\\

\haiku{Aai Burger zegt, dat:}{je helemaal onder de}{dekens moet kruipen}\\

\haiku{Ook mag ik hem niet,....}{omdat hij altijd zo zuur}{en benepen kijkt}\\

\haiku{Natuurlijk omdat.}{ik een jongen ben van den}{groven schoolmeester}\\

\haiku{je moet 't dak maar,....}{es op want grootmoeder klaagt}{over de  spreeuwen}\\

\haiku{je kunt op het dak.}{de schuiten bij Nieuweschild}{in zee zien liggen}\\

\haiku{En daarop schrijven,:}{ze een gedicht dat begint}{met de woorden}\\

\haiku{Schaduw zal de nacht. ', ';}{hun bi\^ent Grauwt almeet}{wordt stil en duister}\\

\haiku{Tot ik plotseling:}{de spottende blik ontdek}{van meester Douma}\\

\haiku{Ik val grootje om -:}{de hals ze weert me lachend}{af met de woorden}\\

\haiku{Waarom weet ik niet,.}{precies want niet alles is}{mij even duidelijk}\\

\haiku{Over den duivel, die,}{rondgaat als een briesende}{leeuw zoekende wien}\\

\haiku{Ik rilde van een.}{vreemde zaligheid bij dit}{schrikwekkend verhaal}\\

\haiku{Alle melk was op.}{en daarom moest Willem die}{avond cognac schenken}\\

\haiku{{\textquoteright} Maar Moeder offreert.}{een oud vloermatje en een}{bos ouwe kranten}\\

\haiku{Er gaat een gejuich,.}{op onder de jongens als}{de buit binnen is}\\

\haiku{Moeder zal wel kwaad,.}{wezen vanavond maar ook d\`a\`ar}{is niets aan te doen}\\

\haiku{De vete tegen {\textquoteleft}{\textquoteright} {\textquoteleft}{\textquoteright} '.}{de fijnen ende Roemsen}{zit haar int bloed}\\

\haiku{{\textquoteleft}As jullie 't weer,,....}{waagt hier te komme stuur ik}{de hond op je of}\\

\haiku{De fijnen moesten maar,....}{proberen nieuwe voorraad}{te verzamelen}\\

\haiku{en t\`och is er iets,}{geschonden in het  beeld}{van mijn Vader dien}\\

\haiku{Duizenden zijn er -.}{op het eiland je hoort ze}{altijd en overal}\\

\haiku{Alleen Kwantes, de,.}{rijksveldwachter is tegen}{hem opgewassen}\\

\haiku{Ik hol de dijk af,.}{de brug over en wip over het}{hek van het boestuk}\\

\haiku{Maar ik ervaar gauw,}{genoeg dat ze nooit \`echt raak}{schieten en voel me}\\

\haiku{er veiliger dan.}{in de straten van het vaak}{vijandige dorp}\\

\haiku{je moet een d\`ame....}{anders behandelen dan}{een boerendochter}\\

\haiku{Aan Oosterend is,,.}{dat niet veel bijzonders maar}{w\`at er is is goed}\\

\haiku{Als er alleen maar.}{boeren zijn krijg je hoogstens}{een tik met de zweep}\\

\haiku{er is iets vernield,....}{dat nooit meer volkomen is}{te herstellen}\\

\haiku{Tot plotseling een....}{hevige beweging golft}{door de mensenhoop}\\

\haiku{Trijn loopt gearmd met,,.}{haar moeder Nantje die}{hartstochtelijk huilt}\\

\haiku{Hij is heel bleek en....}{ik zie dat de hoge hoed}{in z'n handen beeft}\\

\haiku{voor den zoon van den....}{groven schoolmeester zal er}{geen genade zijn}\\

\haiku{bij Kinnebakkie,....}{ligt stro op de straat omdat}{er een zieke is}\\

\haiku{Ik ga naar grootje,.... '}{waar vader en moeder op}{de verjaring zijn}\\

\haiku{Na 11 November.}{volgen de feesten elkaar}{in snel tempo op}\\

\haiku{{\textquoteleft}Je moet maar an de, '}{Sunterklaas frage of die}{t een beetje goed}\\

\haiku{D\`an is er altijd.}{geld om mooie en lekkere}{dingen te kopen}\\

\haiku{Tegen vijf uur, als ',.}{t flink donker wordt ben ik}{niet meer te houden}\\

\haiku{de {\textquoteleft}streetfeger{\textquoteright} laat....}{de sarrende uitdaging}{niet over z'n kant gaan}\\

\haiku{Kinderen zijn er....}{niet meer geboren in dit}{late huwelijk}\\

\haiku{hij draagt een sikje,.}{dat hevig op en neer wipt}{als hij praat of eet}\\

\haiku{Verwoed vecht ik 's:}{middags met aardrijkskunde}{en geschiedenis}\\

\haiku{Later hoor ik, dat {\textquoteleft}{\textquoteright}.}{dit kosthuis bekend staat als}{de blikken emmer}\\

\haiku{De slaapkamers zijn,....}{klein en ongezellig in}{de nok van het huis}\\

\haiku{Schwantje houdt zich goed,:}{als zijn vader hem stevig}{op de schouder slaat}\\

\haiku{De voldoening van.}{gedaan werk begeleidt mij}{altijd en overal}\\

\haiku{verstolen kijk ik,....}{naar meisjes die lachend en}{pratend voorbijgaan}\\

\haiku{Hij staat v\`o\`or de klas,.}{geleund tegen een bordstijl}{en spreekt als een boek}\\

\haiku{Wij zien uit welke.}{delen die bestaat en hoe}{die zijn verbonden}\\

\haiku{Duidelijk zie ik,:}{dat Jan Lenstra piekert over een}{wiskundevraagstuk}\\

\haiku{z'n nagel krast nu.}{en dan een lijntje in het}{vernis van de bank}\\

\haiku{opstellen maken.}{is van ouds een geliefde}{sport in mijn leven}\\

\haiku{Het geritsel van.}{zijn papier accentueert}{de doodse stilte}\\

\haiku{{\textquoteleft}En bedenk wel, dat!}{het gewone werk daar niet}{onder lijden mag}\\

\haiku{En plotseling wordt,.}{de baas zich bewust welke}{fout hij heeft gemaakt}\\

\haiku{Hij gaat werken als,,.}{wij haastig en met grote}{grove contouren}\\

\haiku{ik verander van,.}{kosthuis ik word gedoopt en}{ik krijg een meisje}\\

\haiku{Op een meeting, vlak,,.}{bij ons huis op het land van}{Piet Dros spreekt Staalman}\\

\section{P.A. Daum}

\subsection{Uit: Nummer elf (onder ps. Maurits)}

\haiku{Na het eten ging hij.}{zijn partijtje maken in}{de soci\"eteit}\\

\haiku{{\textquoteright} {\textquoteleft}Laat me met rust{\textquoteright} was,.}{het onvriendelijk antwoord}{met een knoop er op}\\

\haiku{{\textquoteleft}Het is alles goed,.}{en wel maar met mij is het}{een ander geval}\\

\haiku{Wat schulden maken,.}{was en ze niet betalen}{dat wist ze precies}\\

\haiku{Men zegt dat hij knap.}{is voor zijn zaken en dat}{geloof ik ook wel}\\

\haiku{De heren rookten,.}{een havanna genietend}{als goede rokers}\\

\haiku{dan eens wat beter,.}{dan weer wat slechter en op}{den duur achteruit}\\

\haiku{Ik heb het gehoord, ....}{al haast een week geleden}{van onze baboe}\\

\haiku{Zij draaide de lamp,,.}{op keek in een handspiegel}{en zag hem die trek}\\

\haiku{{\textquoteright} Lena sliep die nacht,.}{zoals ze het in lange}{tijd niet had gedaan}\\

\haiku{Wie weet of het niet.}{een kleinigheid was over de}{medicijn of zo}\\

\haiku{De dokter kon er.}{niet veel meer aan doen dan de}{ziekte waarnemen}\\

\haiku{Zo'n kerel was nu.}{maar niet tot het geringste}{besef te krijgen}\\

\haiku{{\textquoteleft}Als ik er langs die,.}{weg zou moeten komen dan}{bedank ik er voor}\\

\haiku{Want hijzelf was weer.}{gewoon teruggebracht tot}{zijn pensioentje}\\

\haiku{Het scheen dat deze.}{neef een onbe- schaamde}{berenmaker was}\\

\haiku{enfin, dat het niet.}{rooskleurig gesteld is met}{de financi\"en}\\

\haiku{Hij zou er graag het,.}{zijne van gehad hebben}{maar dat was moeilijk}\\

\haiku{Hijzelf was ook niet ....}{eenkennig en als het maar}{buiten hem omging}\\

\haiku{Ik hoop niet dat je...{\textquoteright} {\textquoteleft},.}{me voor zo aartsdom aanziet}{Welnee zeker niet}\\

\haiku{Nu, als het u te,.}{Batavia niet mocht lukken}{schrijf me dan maar eens}\\

\haiku{Zij haalde het geld.}{uit haar eigen trommel en}{gaf het haar vader}\\

\haiku{{\textquoteleft}Breng het hem morgen,.}{zelf pa maar asjeblieft een}{bewijs op zegel}\\

\haiku{Het was een woord dat.}{de oude Bruce razend}{maakte van woede}\\

\haiku{Men begreep niet waar,. '}{de mensen vandaan kwamen}{maar ze waren er}\\

\haiku{Een appel en een,!}{ei dat zou ook hier wel het}{resultaat wezen}\\

\haiku{{\textquoteleft}Ik hoop,{\textquoteright} zei Vermey, {\textquoteleft}.}{dat je een beetje beter}{over me denken zult}\\

\haiku{{\textquoteright} Te Batavia moest.}{de oude Bruce in een}{draagstoel naar de wal}\\

\haiku{Ik zou wel een klein.}{souvenir van de oude}{heer willen hebben}\\

\haiku{Hij had meer dan ooit.}{vues op Lena of liever}{op haar vermogen}\\

\haiku{{\textquoteright} {\textquoteleft}Welnu, de waarheid.}{is dat hij me met Vermey}{getrouwd wilde zien}\\

\haiku{Wij hebben immers.}{zonder dat wel meer verschil}{van mening gehad}\\

\haiku{je herinnert je?}{nog wel de manoeuvre van}{de oude Bruce}\\

\haiku{Dat is te zeggen...,.}{ziet u ik ben geen jongen}{van achttien jaar meer}\\

\haiku{Men weet nooit of men.}{iemand wel ergens mee kan}{feliciteren}\\

\haiku{Ga jij nu naar huis,.}{dan schrijf ik nog dadelijk}{naar neef Voirey}\\

\haiku{{\textquoteright} Hij had een pret van.}{belang en lachte zoals}{hij maar zelden deed}\\

\haiku{Voirey was er ook.}{net een die dacht dat alles}{voor geld te koop was}\\

\haiku{{\textquoteright} {\textquoteleft}Maar er kon wel eens,{\textquoteright}.}{geen volgend maal komen zei}{ze sentimenteel}\\

\haiku{Zelfs in gedachten.}{wilde hij niet terug naar}{die Boheemse tijd}\\

\haiku{{\textquoteleft}Zo'n slet,{\textquoteright} dacht hij, stond,.}{op draaide het licht uit en}{ging naar zijn kamer}\\

\haiku{{\textquoteright} Tot een besluit was.}{Vermey ook de dag daarna}{nog niet gekomen}\\

\haiku{Ik heb geen plezier.}{zo alleen naar de muziek}{te gaan luisteren}\\

\haiku{Er waren geen lui,!}{op de weg tenminste haast}{niet dan inlanders}\\

\haiku{De postbode had.}{intussen zijn brieven en}{couranten gebracht}\\

\haiku{{\textquoteleft}Intussen,{\textquoteright} voegde, {\textquoteleft}}{Vermey er heel bedaard bij}{hoor ik wel van je}\\

\haiku{{\textquoteright} vroeg mevrouw Vermey,.}{opkijkend in de richting}{der bijgebouwen}\\

\haiku{Hij baadde, kleedde.}{zich en liet zijn bediende}{de koffers pakken}\\

\haiku{Zachtjes beurde hij.}{het hoofd op en schoof er zijn}{brede arm onder}\\

\haiku{Als men een slang vindt,.}{op zijn erf dan pakt men die}{niet met de handen}\\

\haiku{{\textquoteright} Haar moeder haalde:}{de schouders op en zei op}{minachtende toon}\\

\subsection{Uit: Verzamelde romans. Deel 3 [alleen Goena-goena]}

\haiku{Nijgh \& Van                     Ditmar,,-,-,-.}{Amsterdam 1998 p. 7208}{979985 10171027}\\

\haiku{Naast hem stonden op;}{een marmeren knaap{\textdegree} twee}{kopjes koffie}\\

\haiku{alleen miste hij.}{dat van zichzelve voor zijn}{eigen zaken}\\

\haiku{Als men 'n man heeft,,...{\textquoteright} {\textquoteleft}?}{die notaris is en men}{vertrouwt hem nietNu}\\

\haiku{{\textquoteright} {\textquoteleft}Ja,{\textquoteright} was 't antwoord, {\textquoteleft} '.}{met een zucht vol staatszorgdat}{isn lelijk ding}\\

\haiku{hoe haar zuster van;}{die ellendige kerel}{zou verlost raken}\\

\haiku{Reeds van de eerste;}{dag waren ze elkaar}{tegengevallen}\\

\haiku{{\textquoteright} De oude gaf door,.}{niets te kennen dat ze dit}{verstond of begreep}\\

\haiku{Ze durfde 'n                     .}{ogenblik haar gedachten niet}{te laten voortgaan}\\

\haiku{{\textquoteleft}Je bent erg pinter,,,.}{n\`eh dat je raden kunt wat}{anderen schrijven}\\

\haiku{De oude zal hem,.}{wel goed verzorgen en je}{kunt hier toch niets doen}\\

\haiku{{\textquoteright} {\textquoteleft}Mijn hemel, Borne,?}{je hebt toch op het kerkhof}{geen standjes gemaakt}\\

\haiku{We gaan anders niet,.}{om te spelen maar enkel}{en alleen voor Bets}\\

\haiku{{\textquoteleft}Ajakkes,{\textquoteright} zei Betsy,;}{zich woedend van dit Ezau{\"\i}sch}{schouwspel afwendend}\\

\haiku{Hij was bekend als {\textquoteleft}{\textquoteright},.}{eenhaai die altijd met}{de winst ging strijken}\\

\haiku{{\textquoteright} {\textquoteleft}Nu, 't was zo erg,...}{niet en als Pr\'edier me}{niet zo had verveeld}\\

\haiku{Ik zie niet, dat er.}{enige reden bestaat om}{hem te beklagen}\\

\haiku{Zij reikte hem over,.}{de tafel haar hand die hij}{maar flauwtjes drukte}\\

\haiku{Zij stond op, kwam naar;}{hem toe en stak haar gezicht}{vooruit om een kus}\\

\haiku{{\textquoteleft}Als hij rijk is en,?}{hij wil waarom zou hij dan}{niet als hij kon}\\

\haiku{en wat erop groeit,.}{beter dan de blanda's{\textdegree} die}{er overheen rijden}\\

\haiku{maar ze had                     er;}{nooit enig gevolg van gezien}{of ondervonden}\\

\haiku{Daarom achtte                     .}{Pr\'edier zich genoopt een}{besluit te nemen}\\

\haiku{{\textquoteleft}Waarom zeg je dat '?}{opn manier alsof het}{iets bijzonders was}\\

\haiku{Misschien is er iets.}{waars in en                     zou men nog}{verder kunnen gaan}\\

\haiku{{\textquoteleft}Ik vind wel, dat je.}{vanochtend gruwelijk zwaar}{op                     de hand bent}\\

\haiku{Als zij zich in het,.}{ongeluk willen storten}{is het hun zaak}\\

\haiku{{\textquoteright} Met saamgeknepen;}{lippen hief Bronkhorst het hoofd}{op en zag haar aan}\\

\haiku{{\textquoteright} Bronkhorst stond op en {\textquoteleft}....}{reikte haar de hand.Dus tot}{over een paar dagen}\\

\haiku{Je begrijpt toch wel,,.}{oudje dat ik mijn geld niet}{kan weggooien}\\

\haiku{Hoe lief zat ze daar,!}{met zijn ziek                     kind en hoe}{rustig sliepen ze}\\

\haiku{{\textdegree} Zacht en stil schoof;}{ze kleine Jean van haar}{arm in zijn bedje}\\

\haiku{Toen ze weg waren,.}{keek kapitein Borne hen}{na met aandoening}\\

\haiku{Betsy nam met haar.}{oude meid haar intrek}{bij de Bronkhorsten}\\

\haiku{Zij had met Sidin,.}{de conferentie gehad}{bij haar zoon Ketjil}\\

\haiku{De notaris was.}{kwistig                     geweest met zijn}{uitnodigingen}\\

\haiku{{\textquoteleft}Het is niet om de,.}{soesah maar omdat}{je niet brani bent}\\

\haiku{Alles moest zijn                     ,.}{tijd hebben en het zou nu}{niet lang meer duren}\\

\haiku{U kunt zo lekker,.}{kwee-kwee                     maken en daar}{houdt meneer zo van}\\

\haiku{het is zijn geheim,,.}{dat niemand aangaat omdat}{het van hem                     is}\\

\haiku{Ze hadden het 's;}{avonds daar altijd over in de}{bediendenkamers}\\

\haiku{Maar op haar schijnen.}{jouw duivelskunsten                     geen}{invloed te hebben}\\

\haiku{Maar Betsy trok de.}{wenkbrauwen hoog op en stak}{de lippen vooruit}\\

\haiku{{\textquoteleft}Ik had daar haast 'n,{\textquoteright}.}{brief opengemaakt aan jouw adres}{zei hij tot Marie}\\

\haiku{als vrouw stelde zij.}{temperament meer op}{prijs dan logica}\\

\haiku{Zoals                     ze haar,.}{plichten had vervuld zou ze}{staan op haar rechten}\\

\haiku{{\textquoteleft}Het scheelt, hoe men ook,.}{doet de ene dag toch altijd}{bij de andere}\\

\haiku{Dit was                     van een.}{andere hand dan dat van}{de vorige avond}\\

\haiku{{\textquoteright} {\textquoteleft}Komaan,{\textquoteright} zei ze met, {\textquoteleft}!}{bleke lippenen dat wordt}{mij gevraagd door jou}\\

\haiku{maak nu asjeblieft,,...{\textquoteright} {\textquoteleft}?}{geen bezwaren want als het}{nodig was danDan}\\

\haiku{d\'at nam zij                     zich,;}{ernstig voor om zijn goede}{naam te sauveren}\\

\haiku{Er werd niet verder,;}{over gesproken maar toch vond}{Marie het                     vreemd}\\

\haiku{Misschien wat vermoeid.}{van dat langdurig hossen}{tussen de wielen}\\

\haiku{zij wist dat hij                     .}{daarvan hield als hij uit was}{geweest en dorst had}\\

\haiku{De mensen mogen.}{voor mijn part precies zeggen}{wat zij willen}\\

\haiku{Het is niet                     waar,.}{en het zal op zo'n manier}{ook niet waar worden}\\

\haiku{en naarmate de,.}{klank                     verzachtte werden}{de woorden weker}\\

\haiku{van indiscretie,.}{geen spoor maar iedereen wist}{het niettemin}\\

\haiku{Hij had 't niet                     .}{louter gedaan om haar te}{contrari\"eren}\\

\haiku{'t Was er ditmaal.}{juist een die ze vertrouwde}{en gaarne                     mocht}\\

\haiku{integendeel, zij,.}{lachte allerliefst en}{stond dadelijk op}\\

\haiku{{\textquoteright} vroeg de oude, die.}{op een bal\'e-bal\'e}{haar hazenslaap sliep}\\

\haiku{En nu                     ving hij;}{aan met gemaakte kalmte}{te redeneren}\\

\haiku{Hij had dan toch ook '.}{wel eens                     meern sabel}{in de hand gehad}\\

\haiku{{\textquoteright} {\textquoteleft}Dus denkt u, dat het...{\textquoteright} {\textquoteleft}.}{van mijn kant niet nodig is}{Ik vermeen van neen}\\

\haiku{{\textquoteright} Daar de kapitein,.}{er momenteel althans geen}{raad op wist zweeg hij}\\

\haiku{{\textquoteright} {\textquoteleft}'t Is jammer, dat.}{ik me er persoonlijk niet}{mee kan bemoeien}\\

\haiku{En toen hij zich tot,.}{haar wendde ontroerde zij}{van zijn ontroering}\\

\haiku{Dit teken verwijst}{naar de woordenlijsten op}{blz. 1017                     e.v.}\\

\section{August Defresne}

\subsection{Uit: Het gehucht}

\haiku{In de schommeltent.}{ontstond enige tijd later}{een soortgelijk feit}\\

\haiku{Zij rukte zich met.}{een gil los en snelde om}{hulp naar haar vader}\\

\haiku{Het lach-kabaal,,.}{dat losbrak overstemde de}{kermisgeluiden}\\

\haiku{Zij holden door de,,.}{jankende jengelende}{schreeuwende kermis}\\

\haiku{Die oom was goed voor,.}{me maar na twee jaar verdronk}{hij op een zeiltocht}\\

\haiku{Door dat trekken ging.}{de koperen stang in haar}{steunpunten draaien}\\

\haiku{Maar naarmate hij,.}{onrustiger werd werd zij}{kalmer en kalmer}\\

\haiku{Zij waren een flits.}{van de oneindige tijd}{\'e\'en mens geweest}\\

\haiku{Vanuit zijn kamer.}{kon hij door een kleine gang}{in de kerk komen}\\

\haiku{Enige kinderen.}{hingen uit de donkere}{ramen te kijken}\\

\haiku{{\textquoteleft}Wat is het hier mooi{\textquoteright},.}{en stil zuchtte verlicht de}{opgejaagde man}\\

\haiku{{\textquoteright} De boeienkoning.}{keek met angstig ontzag naar}{het tabernakel}\\

\haiku{En voor mijn dochter.}{ben ik ook altijd zo goed}{mogelijk geweest}\\

\haiku{{\textquoteright} Tussen twee kramen.}{voor het logement stond een}{handkar schuin omhoog}\\

\haiku{Soms liep er een een.}{paar passen en ging dan weer}{naar zijn plaats terug}\\

\haiku{{\textquoteleft}Weet je{\textquoteright}, fluisterde, {\textquoteleft}.}{ze met gebogen hoofdik}{ben nog een meisje}\\

\section{Wies Defresne}

\subsection{Uit: Klanten}

\haiku{- Wie praat hier van door,!}{den neus boren dat gebeurt}{jou toch alleen maar}\\

\haiku{- Hij gapt ze op De,,?}{Porceleina de stinkerd}{en hoe doet hij dat}\\

\haiku{De capsules kan;}{ik aan de nonnen geven}{voor de nikkertjes}\\

\haiku{dan is je broertje, '.}{anders die brengts Zondags}{taartjes voor me mee}\\

\haiku{Mijn broer keek af en.}{toe door de spleten van de}{\'etalage-kast}\\

\haiku{- En jij komt voor de,,.}{honden ik zal ze halen}{zei kapelaan Roets}\\

\haiku{- Ik had eigenlijk,.}{in de kunst moeten  gaan}{vertelt hij Bertus}\\

\haiku{Ach arme, wat is!}{hij met zijn zwaan tegen de}{lamp geloopen}\\

\haiku{in 't museum.}{hangen de schilderijen}{ook op oogshoogte}\\

\haiku{Hij staat nu op en.}{gaat in den winkel tegen}{de vaten kloppen}\\

\haiku{Ik heb dikwijls zin,:}{om te doen wat Lamberts de}{vorige week zei}\\

\haiku{- Je zult er altijd,,.}{spijt van hebben zei ze dat}{is geen vrouw voor jou}\\

\haiku{- Och foei neen, Marie,,.}{dat moet je zoo niet zeggen}{dat arme Toontje}\\

\haiku{De vroolijke Sjef,.}{maakt zich daarover geen zorgen}{al is hij spion}\\

\haiku{En dat is zoo, want,.}{die krijgt vlagen van waanzin}{als ze op reis is}\\

\haiku{Je moet niet veel meer,,.}{hebben  ouwe je bent}{nou al lazerus}\\

\haiku{Toen de kwaliteit {\textquoteleft}{\textquoteright},!}{drank slechter werd voelden we}{wel dat het mis ging}\\

\haiku{Moeder griste hem:}{den brief uit de vingers en}{zei tegen Claesen}\\

\haiku{En ze bewoog haar '.}{handen rond haar hoofd alsof}{zem al op had}\\

\haiku{- Geef mijn peignoir en.}{pantoffels eens aan en laat}{me maar even met rust}\\

\haiku{wij vinden het fijn, '.}{wij vindent weer eens wat}{anders dan anders}\\

\haiku{Van onze klanten.}{ging de Engelsche spion}{wel eens mee fuiven}\\

\haiku{De fleschjes Stout,;}{werden achter onder langs}{de toonbank gezet}\\

\haiku{De dirigent loopt.}{langs de eerste rij met een}{fluitje in den mond}\\

\haiku{Zij is in het wit.}{gekleed en heeft een boquet}{rozen in den arm}\\

\haiku{- Bis, bis, en na nog.}{een klein toegiftje is dat}{moois ook weer voorbij}\\

\haiku{Jouez avec nous une,.}{partie de Whiste nous vous}{serons tr\`es oblig\'es}\\

\haiku{Maar ik bevries toch.}{nog  liever dan dat ik}{de klanten bedien}\\

\section{Maurits Dekker}

\subsection{Uit: Waarom ik niet krankzinnig ben (onder ps. Boris Robazki)}

\haiku{Immers is de mensch,,.}{in volstrekte eenzaamheid}{de juiste mensch niet}\\

\haiku{Uw argeloosheid.}{en zelfbewustheid zullen}{U parten spelen}\\

\haiku{Tijdens mijn ziekte,.}{had Dmitri Tomitch zich van}{het leven beroofd}\\

\haiku{Menigmaal heb ik.}{later alleen met mijn oogen}{een gesprek gevoerd}\\

\haiku{Dit is de laatste.}{herinnering die ik van}{mijn dorp bewaard heb}\\

\haiku{Ik kon nog geen werst,.}{hebben afgelegd toen de}{storm opnieuw opstak}\\

\haiku{Ik zou doen wat mijn.}{vader gezegd had en mij}{naar M. begeven}\\

\haiku{- Je bent moe, hernam,.}{hij de pit van zijn lamp nog}{wat hooger draaiend}\\

\haiku{Ik viel in slaap op,.}{den top van de draaiende}{golvende wereld}\\

\haiku{Het wezen van het (:}{meslet vooral op het woord}{dat ik hier gebruik}\\

\haiku{Ik ging in mijn bed.}{overeind zitten en begon}{hevig te huilen}\\

\haiku{Naar buiten riep ik,:}{neen maar van binnen uit kwam}{steeds weer het antwoord}\\

\haiku{Op dat oogenblik.}{betraden Marja en de}{vreemde haar kamer}\\

\haiku{Eens, toen ik 's avonds,.}{weer de wacht hield voor Marja's}{deur viel ik in slaap}\\

\haiku{Ik kon echter niet,.}{verhinderen dat mijn vrees}{hand over hand toenam}\\

\haiku{Dien avond wachtte ik.}{zijn komst met de revolver}{in mijn  hand af}\\

\haiku{Ja vriend, als je er,}{later over nadenkt begrijp}{je niet dat je zoo}\\

\haiku{dat Gij menschen zijt,.}{van vleesch en beenderen}{zooals ieder ander}\\

\haiku{Ik verstijfde, mijn.}{voeten werden zwaar en ik}{kon geen stap meer doen}\\

\haiku{Van hier uit was het.}{niet moeilijk de bakkerij}{terug te vinden}\\

\haiku{Wel herkende ik:}{de plaats waar ik mij op dat}{oogenblik bevond}\\

\haiku{Mijn ledikant stond.}{in een hoek en daarom had}{ik maar \'e\'en buurman}\\

\haiku{de mijne liggen.}{en ik begrijp waarom je}{dikwijls verdriet hebt}\\

\haiku{Ik hou van je, zoo.}{groot als de wereld en zoo}{diep als je oogen zijn}\\

\haiku{Na vijf dagen werd;}{ik genoodzaakt afscheid van}{mijn vriend te nemen}\\

\haiku{mijn medewerking.}{bij het ledigen zijner}{flesschen te verleenen}\\

\haiku{er omheen, welke.}{teekening er maanden lang}{op is blijven staan}\\

\haiku{Uw arm beweegt zich.}{onverbiddelijk in de}{richting van het wiel}\\

\haiku{Alles, van groot tot;}{klein en van hoog tot laag is}{daar middelmatig}\\

\haiku{Eens per maand, als de, '}{salarissen uitbetaald}{moesten worden kwam hij}\\

\haiku{s morgens reeds, maar.}{op die dagen bleef hij bij}{het theedrinken weg}\\

\haiku{- Een hulp-schrijver,.}{k\`an geen partij schaak van een}{klerk winnen zei hij}\\

\haiku{- Neem toch, vervolgde,.}{zij mij een schoteltje en}{een mes toeschuivend}\\

\haiku{Mijn oudste dochter.}{maakt shawls en dassen voor een}{modemagazijn}\\

\haiku{Jammer dat zij niet,.}{thuis is anders zou u haar}{werk eens kunnen zien}\\

\haiku{Het lawaai was nu.}{zelfs achter het gesloten}{venster onhoudbaar}\\

\haiku{- Ik ben heelemaal,,}{niet kwaad antwoordde ik wat}{je gezegd hebt is}\\

\haiku{Je moet dit alleen.}{inzien en begrijpen dat}{verzet je niets helpt}\\

\haiku{Zij waardeerde en,.}{beoordeelde mij geheel}{zooals ik verwacht had}\\

\haiku{Voorzichtig kroop ik:}{naar het bed en drukte zacht}{een kus op haar hand}\\

\haiku{- Neen, Vladimir, maar.}{ik geloof dat hij je spel}{min of meer doorziet}\\

\haiku{Volkomen waar, maar.}{dan toch in hoofdzaak voor hen}{die niets bezitten}\\

\haiku{Als iemand het ooit,.}{goed met mij gemeend had dan}{was hij het geweest}\\

\haiku{Welke pogingen,}{ik ook aanwendde ik bleek}{onmachtig te zijn}\\

\haiku{Ik werd pas stil, toen.}{ik haar gelaat plotseling}{veranderen zag}\\

\haiku{het was op haar drift.}{in te gaan en het geval}{ernstig te nemen}\\

\haiku{ik ben maar een mensch.}{en zelfs aan mijn vermogens}{zijn grenzen gesteld}\\

\subsection{Uit: De wereld heeft geen wachtkamer}

\haiku{Toneel en kerk zijn.}{in de loop der eeuwen van}{elkaar afgegroeid}\\

\haiku{Het Wereldvenster - -.}{het wereldgebeuren brengt}{ze weer bij elkaar}\\

\haiku{Alles behalve,:}{een vrolijke wetenschap}{maar in elk geval}\\

\haiku{De Wereld heeft geen.}{Wachtkamer stelt ons allen}{verantwoordelijk}\\

\haiku{- Als Mary mij straks,.}{een handje helpt zijn wij er}{vlug genoeg doorheen}\\

\haiku{Maar eerst ga ik even,.}{naar de cantine een}{kop koffie drinken}\\

\haiku{Dichtbundels op de,.}{plaats waar hij zijn rapporten}{had moeten vinden}\\

\haiku{Welke zekerheid,?}{hebt u dat die storing zich}{niet herhalen zal}\\

\haiku{Hij draaide zich om,:}{en zei met zijn duim naar de}{luidspreker wijzend}\\

\haiku{Misschien valt het toch,.}{nog mee en breng ik er het}{leven af dacht hij}\\

\haiku{Zijn ironische toon.}{ontging haar en zij keek hem}{bewonderend aan}\\

\haiku{- Dat zoiets nu juist,,.}{moest gebeuren terwijl ik}{er niet was zei hij}\\

\haiku{Het is niet anders.}{en wij zullen er ons bij}{neer moeten leggen}\\

\haiku{Merkwaardig ook, dat.}{hij nu reeds alle lampen}{had aangestoken}\\

\haiku{U bent toch ook maar,?}{een gewone vrouw al bent}{u dan een dokter}\\

\haiku{Is het soms gek van,?}{me als ik mij bezorgd maak}{over mijnheer James}\\

\haiku{Trek je in ieder.}{geval niets van de praatjes}{van de mensen aan}\\

\haiku{Jouw deel van het werk,.}{kun je rustig doen ik ben}{tot je beschikking}\\

\haiku{Neerslachtiger dan,.}{hij was weggegaan keerde}{hij naar huis terug}\\

\haiku{Zoiets deed je toch,.}{niet als iemands mening je}{onverschillig liet}\\

\haiku{Begrijp toch wat het:}{voor mij betekenen zou}{als ik weigerde}\\

\haiku{- Dag mijn jongen, zei.}{hij zacht en verliet langzaam}{het ziekenvertrek}\\

\haiku{En tenslotte  .}{help ik door mijn sterven ook}{de wetenschap nog}\\

\haiku{- Ik dank je, zei hij,,,.}{eindelijk voor je vriendschap}{voor alles Mary}\\

\haiku{Hij stond op, liep naar.}{het hoofdeinde van het bed}{en keek zijn zoon aan}\\

\haiku{Benner stond voor het.}{geopende venster en}{staarde naar buiten}\\

\haiku{Maar ook zijn geduld.}{had grenzen en men moest nu}{tot een eind komen}\\

\section{Grazia Deledda}

\subsection{Uit: De weg van het kwaad}

\haiku{Houd je voet schrap en.}{je zult zien dat hij je geeft}{wat je wilt hebben}\\

\haiku{{\textquoteright} - {\textquoteleft}Maar wat voor roddels,,?}{Zia Luisa wat valt er over}{mij te beweren}\\

\haiku{- {\textquoteleft}Morgen moet je naar,?}{ons weiland in de vallei}{weet je waar het is}\\

\haiku{{\textquoteright} Zio Nicola ging,.}{achter een klein ongedekt}{tafeltje zitten}\\

\haiku{Hij merkte dat ze,.}{hem niet alleen minachtte}{maar ook wantrouwde}\\

\haiku{- {\textquoteleft}Dat is \'e\'en{\textquoteright}, zei hij.}{en liet zich lenig uit de}{perenboom glijden}\\

\haiku{Een vluchtige blos.}{kleurde het bleke gezicht}{van het dienstmeisje}\\

\haiku{De hond was naar de,.}{poort toegerend en krabde}{eraan blij jankend}\\

\haiku{- {\textquoteleft}Dat werd tijd{\textquoteright}, zei ze,. -}{een punt van haar hoofdband naar}{haar schouder trekkend}\\

\haiku{Voor een poosje was}{het eentonige geluid}{van de koffiemolen}\\

\haiku{Ze lachte en schold.}{en dreigde het tegen Zia}{Luisa te zeggen}\\

\haiku{-{\textquoteleft}Baas, is dat?}{een manier om de mensen}{te laten werken}\\

\haiku{Droevige zaak als,.}{je niet rijk geboren bent}{uit een machtig ras}\\

\haiku{Misschien met een heer,,.}{een gestudeerd heer misschien}{met een rijke boer}\\

\haiku{Ze past niet bij me.}{en mijn verlangen kan me}{tot waanzin brengen}\\

\haiku{Geest van mijn moeder,.}{help me.  Bevrijd me van}{slechte denkbeelden}\\

\haiku{Kerstmis komt en dan.}{gaan we zingen en drinken}{met Zio Nicola}\\

\haiku{Hij werd midden in.}{de nacht wakker en trok zich}{in de schuur terug}\\

\haiku{eenden, schaakstukken,,.}{piramides kruisen en}{zelfs priestermutsen}\\

\haiku{Daar was Nuoro al,,.}{omgeven door de wind in}{de sombere avond}\\

\haiku{Het is koud, maar we.}{zijn niet zo fijn gebouwd als}{de hoge heren}\\

\haiku{horen de vrouwen,.}{in bed en nergens anders}{zo denk ik erover}\\

\haiku{Die liefde was als;}{de vlam die langs het droge}{mos had gestreken}\\

\haiku{Ik kan je kwaad doen,,.}{maar ik wil het niet het komt}{zelfs niet bij me op}\\

\haiku{- {\textquoteleft}Zo net brandde het,.}{nog ik begrijp niet waarom}{het is uitgegaan}\\

\haiku{Ze stribbelde een,.}{beetje tegen maar zonder}{geluid te maken}\\

\haiku{Hij stond op en ging:}{naar bed met steeds dezelfde}{vreugde in zijn hart}\\

\haiku{Ik verkoop meteen,,.}{het huis en het land alles}{en ik word koopman}\\

\haiku{En hij wilde haar,,.}{puur trouwen hoogstens gezoend}{en alleen door hem}\\

\haiku{in een goede biecht.}{wordt de geest gewassen als}{een doek in de bron}\\

\haiku{En hoe had hij het?}{gewaagd zijn  oog op haar}{te laten vallen}\\

\haiku{Wie van hen is het{\textquoteright}?}{waard Francesco Rosana}{s schoen te strikken}\\

\haiku{Maria bleef een paar.}{dagen op haar eigen land}{en bloeide er op}\\

\haiku{Uit elk dorp uit de;}{omgeving kwamen mensen}{naar Gonare}\\

\haiku{een blauwige damp.}{omhoog als de adem van de}{koortsige aarde}\\

\haiku{Maar ze durfde haar.}{duistere hartsgeheimen}{niet uit te spreken}\\

\haiku{{\textquoteright} - {\textquoteleft}Het spijt me, beste,.}{meid maar we hebben het langs}{de weg verloren}\\

\haiku{Het viel iedereen.}{op en hij deed geen moeite}{het te verbergen}\\

\haiku{Het is treurig om,.}{aan te zien zo mager als}{hij is geworden}\\

\haiku{{\textquoteright} De reiziger ging,.}{weg maar Pietro floot om hem}{terug te roepen}\\

\haiku{Ze kan me niet op{\textquoteright}.}{zon Judasachtige}{manier verraden}\\

\haiku{Hij greep hem bij zijn.}{halsband en trok hem achter}{de hoek van de muur}\\

\haiku{{\textquoteright} Maria voelde dat.}{zijn woorden voor haar bestemd}{waren en werd bang}\\

\haiku{Maar hij herhaalde.}{zijn minachtende gebaar}{en brak de zin af}\\

\haiku{Toen hij alleen was.}{gaf Pietro zich over aan zijn}{woede en wanhoop}\\

\haiku{Hoe graag had hij het!}{zaaigraan vergiftigd of in}{de wind gegooid}\\

\haiku{De weg hield niet op.}{en hij wist trouwens ook niet}{waar hij heen ging}\\

\haiku{-{\textquoteright}Als de winter.}{komt slaap ik weer onder dat}{rampzalige dak}\\

\haiku{- {\textquoteleft}Weet je zeker dat?}{ik me niet met opzet heb}{laten oppakken}\\

\haiku{{\textquoteright} - {\textquoteleft}Ja{\textquoteright}, zei Pietro, - {\textquoteleft}maar.}{bezit is niet genoeg om}{gelukkig te zijn}\\

\haiku{Na de nacht van zijn.}{vrijlating was Pietro niet}{meer langsgekomen}\\

\haiku{En ja, als je de,.}{spullen hebt dan moet je je}{niet karig tonen}\\

\haiku{Ze keerde terug.}{in haar kamer en legde}{haar bruidskleren klaar}\\

\haiku{{\textquoteright} Maar hij sloeg zijn arm{\textquoteright}.}{om Marias middel en}{wilde haar kussen}\\

\haiku{Ja, nu was alles,.}{voorbij nu was er niets meer}{om bang voor te zijn}\\

\haiku{Of moest hij naar de?}{keuken om op zijn plaats van}{knecht te gaan zitten}\\

\haiku{Het cadeau heb ik,.}{haar gegeven maar die zoen}{moet ik nog hebben}\\

\haiku{{\textquoteright} - {\textquoteleft}Als jij haar een zoen{\textquoteright},.}{geeft doe ik het ook zei de}{jonge eigenaar}\\

\haiku{De uitgestrekte.}{tanca was omgeven door}{begroeide muurtjes}\\

\haiku{- {\textquoteleft}Nee, ik vergis me{\textquoteright},.}{niet dacht Maria en keerde}{naar de hut terug}\\

\haiku{Wat gebeurde er,?}{daarginds in het nu geheel}{zwart geworden bos}\\

\haiku{dan kan hij zich op.}{zijn beurt ongerust maken}{als hij terugkomt}\\

\haiku{Wat had ik gelijk,!}{om ongerust te zijn ik}{was te gelukkig}\\

\haiku{Francesco zou,.}{niet dood zijn ik zou niet zo}{geleden hebben}\\

\haiku{Als ik hem kwaad had,.}{willen doen dan had ik het}{eerder kunnen doen}\\

\haiku{- {\textquoteleft}Daar ben ik{\textquoteright}, zei ze,, - {\textquoteleft}?}{vriendelijk maar niet teder}{Wat wil je van me}\\

\haiku{Ik ben klaar met het.}{inzaaien van het graan en}{ik ga hout kappen}\\

\haiku{{\textquoteright} - {\textquoteleft}Praat daar niet over{\textquoteright}, zei,.}{Pietro en beet zich in zijn}{vuist -{\textquoteright}Hou je mond}\\

\haiku{- {\textquoteleft}Daarna doe ik wel{\textquoteright},.}{het offici\"ele verzoek}{lachte Antine}\\

\haiku{Soms vroeg ze zich af.}{of ze opnieuw van Pietro}{was gaan houden}\\

\haiku{vandaag is het feest{\textquoteright},.}{en de boeren zijn in het}{dorp zei Antine}\\

\haiku{Hij was zo knap, zo,.}{goed gekleed met ogen die van}{geluk fonkelden}\\

\haiku{Een krachtige klop.}{op de poort wekte haar uit}{deze droomtoestand}\\

\haiku{{\textquoteright} En naarmate het.}{kwaad zich ophoopte weerklonk}{die vraag steeds sterker}\\

\haiku{Voor de mensen moest,..}{zij zich opofferen haar}{hele leven lang}\\

\haiku{Ze is geboren,,.}{om te vechten te strijden}{verraad te wreken}\\

\haiku{Misschien de eerste,:}{avond toen hij haar had omarmd}{en haar had gezegd}\\

\haiku{Pietro had een stap.}{achteruit gedaan en bleef}{naar haar kijken}\\

\section{Val\`ere Depauw}

\subsection{Uit: Het late geluk van Remi Zwartekens}

\haiku{waarom heeft liefje}{een dokus en eksteroog}{helemaal geen}\\

\haiku{Meneer Zwartekens,}{vooraleer gedichten uit}{te geven zoudt}\\

\haiku{Bestonden er dan?}{twee manieren om zonder}{fouten te schrijven}\\

\subsection{Uit: Niet versagen, Mathias}

\haiku{Val\`ere Depauw,}{Niet versagen Mathias}{Colofon}\\

\haiku{Hoe verheugd zou ze.}{nu zijn met dit boek van haar}{geliefden auteur}\\

\haiku{{\textquoteleft}Ik word opgejaagd...}{als wijd en dan moet ik nog}{wachten op mevrouw}\\

\haiku{{\textquoteleft}Ik had u nochtans,{\textquoteright}.}{gevraagd om zeven uur hier}{te zijn overdreef zij}\\

\haiku{En als het moest... ~ {\textquoteleft},{\textquoteright}.}{Maar hij moet voorzichtig zijn}{herhaalde Simon}\\

\haiku{Waarom?{\textquoteright} {\textquoteleft}Ja, hoe moet,{\textquoteright}.}{ik dat uitleggen begon}{Simon voorzichtig}\\

\haiku{Als de anderen,.}{de eerste gebruiken des}{te erger voor hen}\\

\haiku{{\textquoteleft}Nog niet, Lagrange...{\textquoteright} {\textquoteleft},?}{laat niet afMaar het komt wel}{in orde nietwaar}\\

\haiku{Maar Pieter-Jan bleek.}{aan dat doktersbezoek geen}{belang te hechten}\\

\haiku{Er was een nieuwe,.}{kracht in hem hij was sterk en}{niet te overwinnen}\\

\haiku{Hij had besloten:}{dat er een nieuwe fabriek}{gebouwd moest worden}\\

\haiku{{\textquoteright} 's Middags gingen,.}{zij samen naar Simon die}{den brief vertaalde}\\

\haiku{Hij was het evenbeeld,.}{van zijn moeder zoowel physiek}{als van karakter}\\

\haiku{En als er ook in...}{Anton niet veel liefde voor}{het bedrijf zijn moest}\\

\haiku{Als ge het geld weer,.}{missen kunt brengen wij het}{weer naar de spaarkas}\\

\haiku{Maar Maria kon zich.}{hierom niet verheugen en}{ze kon niet trotsch zijn}\\

\haiku{Het was een dure.}{les geweest en nooit meer zou}{hij zich zoover wagen}\\

\haiku{En zoo er toch iets,?}{moest gebeuren zoo er toch}{iets moest gebeuren}\\

\haiku{En het was ook lang,.}{geleden dat ze zoo had}{kunnen glimlachen}\\

\haiku{Lagrange was niet,.}{op zijn gemak dat kan ik}{u verzekeren}\\

\haiku{Over dien eersten Mei,:}{was er nog niet gesproken}{maar nu vroeg Simon}\\

\haiku{Hij was groot en sterk,.}{gebouwd maar hij had niet de}{fijnheid van Lambert}\\

\haiku{Henri Lagrange,.}{de kleinzoon van den ouden}{Charles Lagrange}\\

\haiku{Louis Lagrange moest.}{dicht bij de vijftig zijn en}{hij was weduwnaar}\\

\haiku{In 1905 brak tusschen...}{Lagrange en Wieringer}{het groote conflict los}\\

\haiku{En dat weten de.}{wevers en daar moeten wij}{gebruik van maken}\\

\haiku{En er is ook nog...}{de manier waarop die tien}{percent ge\"eischt wordt}\\

\haiku{In elk geval was.}{De Canni\`eres op de}{hand van Lagrange}\\

\haiku{{\textquoteleft}Wat er m\'e\'er kan mee?}{bereikt worden dan met een}{gewoon armuur}\\

\haiku{Op een morgen kwam.}{Simon Mathias op zijn}{bureau opzoeken}\\

\haiku{Het gebeurde in.}{het groote bureau en alleen}{de experts werkten}\\

\haiku{Ik, als griffier, heb.}{hier niet te bevelen en}{niet te verbieden}\\

\haiku{Het brevet, dat hij,,...}{den voorzitter voorlegt kan}{waardeloos zijn valsch}\\

\haiku{{\textquoteright} En het was goed, dat.}{Maria van alles op de}{hoogte werd gesteld}\\

\haiku{Meneer Lagrange,{\textquoteright}.}{heeft het recht de huiszoeking}{te doen zegde hij}\\

\haiku{En gingen zij ten,,!}{onder dan was het vechtend}{verbeten vechtend}\\

\haiku{En ginder tusschen...}{de getouwen stapte de}{oude Lagrange}\\

\haiku{Maar Wieringer zal,.}{het wel volhouden denkt men}{te Steenbrugge}\\

\haiku{En Anton, die nu.}{zestien was en steeds meer op}{zijn vader geleek}\\

\subsection{Uit: Toch lammeren, broers! (onder ps. Piet Canneel)}

\haiku{{\textquoteleft}Zij zijn moedig en,.}{sterk en zij hebben nog een}{lange tijd voor zich}\\

\haiku{En hopen, dat het,,.}{eens beteren zal ach dat}{kunnen zij niet meer}\\

\haiku{Zij was de laatste,...}{dagen alleen wat vermoeid}{maar dat ging wel over}\\

\haiku{Een eigen huis, dat,.}{prachtig ingericht was en}{heel wat contanten}\\

\haiku{Ge kunt het niet zien,:}{zoals hij daar zit maar hij}{heeft geen voeten meer}\\

\haiku{{\textquoteleft}En die andere,{\textquoteright}.}{jonge kerel aan het raam}{vervolgde Lode}\\

\haiku{Neen, neen{\textquoteright} zegde ik, {\textquoteleft},,}{op mijn beurtniet weigeren}{Mon niet weigeren}\\

\haiku{Zij dacht er ernstig.}{over na en dat was wellicht}{de beste uitkomst}\\

\haiku{{\textquoteleft}En nu wordt het tijd,.}{dat ge naar de steenweg gaat}{of ge komt te laat}\\

\haiku{{\textquoteleft}En nu stappen wij,.}{gauw op want het is te koud}{om te blijven staan}\\

\haiku{De kinderen zijn.}{bij mijn zuster en ik werd}{ook uitgenodigd}\\

\haiku{Hij hoefde echter,.}{niet te antwoorden want hij}{duwde het hek open}\\

\haiku{{\textquoteleft}Ik heb tot in de.}{minste bizonderheden}{alles meegemaakt}\\

\haiku{Hoogopgericht, trots,.}{alsof zij ook aan de paal}{vastgebonden was}\\

\haiku{Rikus, als tweede,,:}{oudste vond dat hij ook een}{woordje mocht plaatsen}\\

\haiku{Ivo stond voor hem en,.}{die grote sterke kerel}{weende als een kind}\\

\haiku{{\textquoteleft}En ze laat u nog.}{eens vragen of gij er nooit}{spijt zult om hebben}\\

\haiku{{\textquoteleft}En ik zal er nooit...}{spijt over hebben met Elza}{gehuwd te zijn}\\

\haiku{Een schoon kindje, het,...}{evenbeeld van de moeder trek}{voor trek het evenbeeld}\\

\haiku{En nu moet ge gaan,,.}{Lowieke want ik heb nog}{enkele klanten}\\

\haiku{laet ons singhen blij, Daer...}{meed oock onse leisen}{beghinnen vrij}\\

\section{Lodewijk van Deyssel}

\subsection{Uit: De scheldkritieken}

\haiku{Ik zet het je, ze.}{op een geschikte manier}{terug te leggen}\\

\haiku{De Europeesche,,.}{naties die heden stilstaan}{zijn morgen te niet}\\

\haiku{Thands behoort deze,;}{schrijver tot dien kliek gene}{tot een andere}\\

\haiku{Ontzachlijk in zich.}{zelf en ontzachlijk in het}{begrip der menschen}\\

\haiku{Wij willen Holland.}{hoog opstooten midden in}{de vaart der volken}\\

\haiku{Mij is het een lust,}{te schimpen op uw rimpels}{die geen groote Idee u}\\

\haiku{O, kon ik u zoo,!}{krenken dat uw hersens van}{den schrik verlamden}\\

\haiku{'t Is altijd feest, '.}{in onze Letterent}{is altijd kermis}\\

\haiku{Het wordt tijd om tot.}{een duidelijk begrip te}{komen van dit woord}\\

\haiku{{\textquoteleft}de meesten stelden{\textquoteright}.}{het uit tot morgen zegt een}{porder tot zijn vrouw}\\

\haiku{middelmatigheid,.}{en wankunst zich vertoonen}{ze te signaleeren}\\

\haiku{Gij vertoont den mensch,,...}{zeker maar als denkeres}{en kunstenares}\\

\haiku{Ja, malen, dat is ',;}{t wat gij doet en met een}{verheven penseel}\\

\haiku{Mag ik er dus op,?}{rekenen dat u niet meer}{zoo hard werken zult}\\

\haiku{welken indruk heb?}{ik van dat boekje in zijn}{geheel ontvangen}\\

\haiku{{\textquoteleft}er gaapte als een{\textquoteright};}{onoverkomelijke}{klove tusschen hen}\\

\haiku{je bent, ja, je bent,...,,...!}{een auteur een schrijver een}{romancier ondeugd}\\

\haiku{14 Juni 1890        [.}{Romans van Maurits 58}{Goena-Goena}\\

\haiku{wat duivel kan 'et,!}{m{\'\i}jn ook biete het raakt me}{eigelijk geen lor}\\

\haiku{Waarin Amor als naar,,.}{gewoonte bewijzen van}{zijn blindheid geeft IV}\\

\haiku{De opschriften dus.}{bederven eenigermate}{het geheele spel}\\

\haiku{De verhalende;}{gedeelten van dit werkje}{zijn niet de beste}\\

\haiku{Haar zuster 68~		 [,.}{Tooneelspel in vier bedrijven}{door Marcellus Emants}\\

\haiku{niet zoozeer om aan eene,;}{dame te zenden ook niet}{voor de kinderen}\\

\haiku{Doch, het is waar ook,.}{wij zouden een overzicht van}{het verhaal geven}\\

\haiku{Te gelijk is er.}{een boekje met gedichten}{van den heer G.H. Priem}\\

\haiku{Toen ben ik gretig.}{met mijn neus er op en er}{in gaan snuffelen}\\

\haiku{En... e... ze had - we -,?}{zijn onder  ons nog al}{fortuin geloof ik}\\

\haiku{In 't algemeen:}{zijn de waarheden omtrent}{den heer Byvanck}\\

\haiku{Byvanck dont le...}{livre sur Paris a \'et\'e}{publi\'e en fran\c{c}ais}\\

\haiku{Zij, de mannen van,.}{den dijk werden allengskens}{helden in mijn droom}\\

\haiku{Zij kenden geen God,.}{dan de God der waarheid zij}{dienden geene partij}\\

\haiku{Gisteren heb ik.}{een artikel van 3 blz.}{over Quack geschreven}\\

\haiku{maar ik, ten minste,,,.}{heb geld noodig niet voor mij maar}{voor mijn kinderen}\\

\haiku{Goes, die kan zulke.}{dingen terdege aardig}{aan de man brengen}\\

\haiku{Hoe gaat het in je.}{tweezaamheid en wanneer komt}{nu toch je roman}\\

\haiku{4, april 1888, blz. 163-,;}{169 en aldaar gesigneerd}{met de letters F.H.}\\

\haiku{(Een {\textquotedblleft}boekbe\"oordeeling{\textquotedblright}).}{van het werkje van dien naam}{door Soera Rana}\\

\haiku{die oude fraze,:}{die geen schoolkind van Delfzijl}{meer op zo\^u schrijven}\\

\haiku{Daarin onthulde.}{hij dat Marie de naam was}{van zijn verloofde}\\

\haiku{Netscher, niet in staat,.}{te troosten liet Professor}{ten leste alleen}\\

\haiku{Eindelijk, niet dan,...}{toevallig viel zijn blik op}{het leelijke woord}\\

\haiku{De van Eeden die,.}{erin komt is mijn vader}{de botanicus}\\

\haiku{Zoo een bruidegom;}{kan wanhopig in een hoek}{gaan zitten staren}\\

\haiku{Studi\"en op het,.}{gebied der Letterkunde}{door W.G.C. Byvanck}\\

\haiku{Il se donne tout,.}{entier voil\`a le secret de}{sa puissance}\\

\haiku{Gij zijt hier te dicht,,.}{te dicht zijt gij hier bij wat}{mij heilig is}\\

\haiku{Zien zij er niet uit?}{als een menigte kleine}{ovale grafzerkjes}\\

\haiku{-{\textquoteright} ~ 123 Hierachter (:}{in het handschriftvoorzien van}{de kanttekening}\\

\haiku{of ik dit ben of,,,...}{dat ben of niet ben of hoe}{het nu met mij z{\'\i}t}\\

\haiku{82, 249-56, 290, 350-,:}{3 ~ Calderon de}{la Barca Pedro}\\

\haiku{297 Hilman, J.: 161,,,:,,-,:}{183 328 Hofdijk W.J. 33 211}{2869 343 Holda}\\

\haiku{30, 33, 41, 52, 71-,,-,,,,,:}{3 116 1224 145 186 203}{343 Murger Henri}\\

\haiku{F. van: 80-84, 312,:,,,,,:}{Reijnders Karel 290 293 339}{347 348 358 Rembrandt}\\

\haiku{Een aangrijpend schoon.}{panorama lag dan aan}{mijn voeten ontrold}\\

\subsection{Uit: Verzamelde werken. Deel 2. De kleine republiek}

\haiku{Het portier stond al.}{open en er was bijna geen}{tijd meer voor Mietje}\\

\haiku{De vader van zijn,.}{vrouws bezorgdheid Willem ziek}{dadelijk schrijven}\\

\haiku{En er kwamen er,.}{achter Willem aan waarnaar}{hij niet dorst omzien}\\

\haiku{Maar het was half elf.}{en de klok bielebangde}{door de gebouwen}\\

\haiku{alles was stil, de,;}{slaapzaal le\^eg de jongens al}{lang naar beneden}\\

\haiku{Daarna keek Willem.}{tegen zijn spiegeltje of}{zijn h\'aar wel goed zat}\\

\haiku{Vertel nu eens, denkt,.}{ge dat ge U hier nogal}{zult kunnen schikken}\\

\haiku{- Zoo, en wat was dat,,?}{dan wel die ondeugendheid}{wat deed ge dan wel}\\

\haiku{- Och, dat weet ik niet,,,.}{meneer ja ze vonden het}{wel eens nog al erg}\\

\haiku{Pieterse, weet gij,}{er niet iets meer van gij zit}{daar zoo ernstig v\'oor}\\

\haiku{- Zoo, mot je scheiten,,.}{alle mattessen zitten}{vol eerst met me me\^e}\\

\haiku{Hij kauwde achter,.}{zijn lippen met chocola}{aan de mondhoeken}\\

\haiku{Kom, dat neukt niet, 't ' ',.}{isnn goeye hier die doet}{net of-i niks merkt}\\

\haiku{hee, zou Willem nog,.}{niet uit school komen het is}{toch al laat genoeg}\\

\haiku{Stelhuis en Brik, zijn,, '}{oudste plagers pakten hem}{beet brachten hem in}\\

\haiku{Hier tegen had hij:}{nog nooit in zijn leven iets}{gedaan of gebiecht}\\

\haiku{Maar er was niets, en.}{de lampen waren nog niet}{ne\^ergedonkerd}\\

\haiku{Hij zag den duivel,,.}{niet maar de duivel was een}{geest dus onzichtbaar}\\

\haiku{Hij lei zijn hand uit',:}{in de buiging van Mets arm}{zei tot hem en Breg}\\

\haiku{De groote nachtmis van '.}{den eersten Kerstdag was om}{vijf uurs ochtends}\\

\haiku{Hij was hier nu thuis,,.}{gemeenzaam onverschillig}{voor al het nare}\\

\haiku{Daar was 't weer, als,.}{een groot-lieve vogel}{die kwinkeleerde}\\

\haiku{Zij was niet dood, zij,,.}{was niet weg zij leefde nog}{zij zou bij hem zijn}\\

\haiku{Eindelijk was hij,.}{aangeheschen de kraag nog}{\'op tegen den hals}\\

\haiku{hier samenzijnd om,.}{dan los te gaan  ieder}{naar zijn kompanjie}\\

\haiku{En daar brandden de.}{zondaren eeuwig zonder}{ooit verteerd te zijn}\\

\haiku{Hij bad verder, maar:}{zijn gedachten bleven stil}{bij die woorden van}\\

\haiku{Zoudt ge dan rein zijn?}{geweest en in den hemel}{hebben kunnen gaan}\\

\haiku{- Ja, 't is toch h\'eel, ',.}{mooi je zalt n\'ooit zien dat}{mot ik bekenne}\\

\haiku{voor geen geld van de,...}{wereld we\^er wille slape}{in dat kamertje}\\

\haiku{- Nou, as ge d'r toch,.}{niet van krijgt dan hoefde dat}{ook niet te wete}\\

\haiku{Het was een zoete,, ',.}{vreugde ja nu voelde hij}{t wel langzaamaan}\\

\haiku{Na de wandeling,.}{veindse Willem vreeselijk}{hoofdpijn te hebben}\\

\haiku{Ze hadden gedamd,,:}{om wat te doen en telkens}{had Willem gezeid}\\

\haiku{Hij voelde de school,.}{nu in de verte in de}{vlak-bij-toekomst}\\

\haiku{- Hij is niet op school,,.}{geweest zei ze hij heeft een}{goeverneur gehad}\\

\haiku{De stem van Hoeffel.}{joeg hoog op voor de vensters}{achter Willems rug}\\

\haiku{Eindelijk waren,.}{ze in het bosch op den berg}{waar de kapel stond}\\

\haiku{Zij lachten om den,,,.}{kleine den leelijke den}{bleeke den gesarde}\\

\haiku{zeg-gij 't maar,,:,, ';}{Tiessen of kom Tiessen gij}{zultt wel weten}\\

\haiku{Pauli schuurde de.}{pop en en de proppen af}{van den katheder}\\

\haiku{Toen Pauli we\^er op,.}{z'n plaats zat kwam Piet-Suf}{we\^er voor de klas staan}\\

\haiku{En hij stond even stil.}{in de wijde nis van den}{groen-gouden dag}\\

\haiku{je, anders kan je '.}{wanneer je maar wiln pak}{slaag van me krijge}\\

\haiku{- Nou, als je maar wil,,,.}{van-daag morrege of}{met de vakancie}\\

\haiku{Er moest alleen op,.}{gelet worden dat Blaise}{d'r niks van merkte}\\

\haiku{Dani\"el had ook.}{allerlei streekjes waarme\^e}{hij Hoeffel plaagde}\\

\haiku{nu we\^er minder bij,.}{Scholten werd vernederd en}{geslagen zijnde}\\

\haiku{Hij nam zijn handen.}{uit zijn broekzakken en deed}{er een aan zijn kin}\\

\haiku{Dit was een gevoel,}{dat Willem niet gehad had}{sints het eerste jaar}\\

\haiku{- A jesses wat 'n,... -}{keerel zei Willem as je nou}{toch vooruit beloofd}\\

\haiku{- Kom, beloof 'et nou,:}{dan krijg je wat moois van me.}{Dani\"el lachte}\\

\haiku{- Nou, daar mot toch ies,,.}{voor zijn zei Willem late}{we wat verzinne}\\

\haiku{Hij trad voort tot aan,,.}{de stoep ging langzaam de stoep}{af en wachtte we\^er}\\

\haiku{die jongen... Zo\^u hij?...}{misschien ook al een beetje}{van Willem houden}\\

\haiku{hem, hij keek ne\^er en.}{het was of het windje de}{aandacht had verwaaid}\\

\haiku{die le\^e nu ook school;}{en ha\'ar vakancie was toen}{nog niet begonnen}\\

\haiku{Hij stak zijn tong uit.}{en trok rare gezichten}{tegen het kajee}\\

\section{Els Diederen}

\subsection{Uit: Platbook 5. Moder Maas}

\haiku{Ich hei aevel wel,,.}{mien druime en verlanges}{veural veul twiefel}\\

\haiku{M\`et mien kleinkinger '.}{veur oppe fiets geitt wie}{vanzelf richting Maas}\\

\haiku{Wat haet de Pletsjbaek '.}{toch gel\"ok en wat b\`en ichne}{gel\"okkige miensj}\\

\haiku{zoe st\`ellekes,.}{duide ritselend door de}{wind in cadans}\\

\haiku{Ik waas zoe\"e blie wie ',,.}{n kind wat neet zoe\"e maf waas}{ik waas nog ein kind}\\

\haiku{Miene pap dach dao,.}{aevel anders euver maar}{zag niks taege mam}\\

\haiku{Ich drejdje 't,:}{stuur verkie\"erdj precies d'n}{angere kantj op}\\

\haiku{zien bein kwoeam gekneldj.}{t\"osse de grindjbak en}{de baggermuuele}\\

\haiku{Det zien vroumes neet?}{k\'os verdrage det hae d'n}{hie\"ele daag thoes waas}\\

\haiku{door 't landjsjap struimp  ,:}{Waar de brede stroom der Maas}{statig zeewaarts vloeit}\\

\haiku{De Pruse h\"obbe}{de streek inne zestiger}{en zeventiger}\\

\haiku{Ich woondje doe in '.}{t centrum van Remunj op}{get mie\"e dan 20 m+NAP}\\

\haiku{De baek klatert de, ',}{wiejert inn M\"osj l\`esjt h\"a\"oren}{doorsj Woa birk en \`esj}\\

\haiku{'t Is weer voorbij,.}{die mooie zomer zong d'r Jerard}{Cox \'op d'r radio}\\

\haiku{En det geit 't b\`es '.}{mit angere same op}{t grote vaer}\\

\haiku{Doa brent nog jee lit. ' '.}{t Is jraad oft de sjtem}{van d'r Wullem huet}\\

\section{Jules Dister}

\subsection{Uit: Vermorzelde honden}

\haiku{De betrokkene.}{zou vaker met een koffer}{gesignaleerd zijn}\\

\haiku{Ergens tussen Luik.}{en Maastricht houden ook zij}{op te bestaan}\\

\haiku{Het enige wat hij,.}{niet kan is spreken anders}{was hij net een mens}\\

\haiku{{\textquoteright} Mijn oom Pie is geen.}{partij voor deze monsters}{van de didactiek}\\

\haiku{Wit haar wolkt om zijn,,.}{hoofd een stralenkrans een}{zilveren monstrans}\\

\haiku{Emoties zijn niet zijn,.}{sterkste kant laat staan dat hij}{ze zich herinnert}\\

\haiku{Aan het eind van de.}{oorlog deed ze opnieuw iets}{verbazingwekkends}\\

\haiku{En wat wil nu het,?}{verhaal ongelofelijk}{als het leven zelf}\\

\haiku{Het is een verhaal,?}{dat tot de verbeelding spreekt}{maar is het ook waar}\\

\section{R. Dobru}

\subsection{Uit: Wan monki fri. Bevrijding en strijd}

\haiku{De mensen die toen!}{de bovenlaag in de}{maatschappij vormden}\\

\haiku{Jozua pakt zijn tas.}{onder de arm en                     voegt}{zich bij de stakers}\\

\haiku{Hij is toen in de,.}{handel gegaan met het}{noodlottig gevolg}\\

\haiku{Bij springvloed konden.}{wij met de korjalen}{tot in huis komen}\\

\haiku{Een goede oogst van,;}{wie                     dan ook was voor de}{hele plantage}\\

\haiku{De gracht, een goot, werd.}{aan weerskanten begrensd door}{twee                     zandwegen}\\

\haiku{Enkele maanden.}{later kon het gezin weer}{bijeen gaan wonen}\\

\haiku{En u hebt God lief,.}{wanneer u uw                     naaste}{lief hebt als uzelve}\\

\haiku{Hij                     vertelde.}{ons dat zijn familie dat}{van hem had ge\"eist}\\

\haiku{Veredelen is.}{iets voeren van primitief}{naar                     ontsloten}\\

\haiku{Wij hadden allen.}{een kruidbad gekregen van}{mijn                     grootmoeder}\\

\haiku{Van mijn nicht waren.}{niet eens de                     haren aan}{haar benen verschroeid}\\

\haiku{De oude vrouw kwam(!).}{zowaar met een Bijbel in}{de hand aan tafel}\\

\haiku{Terwijl ik met hem,.}{zat te praten krijgt hij \'e\'en}{van zijn                     winti's}\\

\haiku{Ik vond het jammer.}{dat ze te                     arm was om}{het te betalen}\\

\haiku{Tot nu toe zal je.}{protesten horen van de}{Kreoolse meisjes}\\

\haiku{Tegenwoordig moet!}{je de Hindostaanse}{meisjes zien dansen}\\

\haiku{Wat je altijd bij;}{de Kreoolse meisjes vindt}{is bille baja}\\

\haiku{Een heleboel                     .}{oude dogma's waren weer}{kapotgeslagen}\\

\haiku{Wi Egi Sani                     .}{hield toen in 1957 haar eerste}{kultureel kongres}\\

\haiku{Naarmate de man,.}{sprak                     kreeg ik meer en meer}{sympathie voor hem}\\

\haiku{Wij maakten slogans,.}{en                     bralden die uit om}{gehoord te worden}\\

\haiku{De werkgever die.}{een paradijs had tijdens}{Pengels regiem}\\

\haiku{- De eenwording van;}{het Surinaamse volk wordt}{erdoor belemmerd}\\

\section{Neel Doff}

\subsection{Uit: De avond dat Mina mij meenam}

\haiku{De huurders die er ',;}{t best van konden komen}{woonden beneden}\\

\haiku{Intussen maakten,;}{de keukenmeid de min en}{ik de bedden op}\\

\haiku{{\textquoteright} En met haar handen.}{als kolenschoppen gaf ze}{mij een balletje}\\

\haiku{Ze liepen in een.}{wolk van Eau de Cologne}{en pepermuntgeur}\\

\haiku{Mijn zuster en de.}{andere meisjes droegen}{zulke stoffen niet}\\

\haiku{{\textquoteright} {\textquoteleft}Hoor eens, meisje, je,.}{komt hier nu al vijftien jaar}{altijd bij vlagen}\\

\haiku{Zo, en dat was dat..}{U moet niet denken dat u}{mij ooit terugziet}\\

\haiku{{\textquoteleft}Hier ouwe,{\textquoteright} zei ze, {\textquoteleft}.}{danhet is niet jouw schuld als}{je baas een hond is}\\

\haiku{thuis had hij het weer.}{te voorschijn gehaald en aan}{zijn vrouw gegeven}\\

\haiku{als je niets vertelt,.}{van dat wasgoed geef ik je}{de accordeon}\\

\haiku{Zijn benen waren,;}{hoog en gespierd zijn lange}{handen beweeglijk}\\

\haiku{Zindelijk houden.}{en naar school sturen kost echt}{niet teveel moeite}\\

\haiku{Bij tijd en wijle,.}{kwam hij naar huis maar alleen}{als ik er niet was}\\

\haiku{Op een morgen kwam.}{ze triomfantelijk bij}{ons binnentrippen}\\

\haiku{{\textquoteright} ~ Het dienstmeisje,,.}{waar Dirk hartstochtelijk van}{hield schonk hem een zoon}\\

\haiku{Bij de anderen,.}{zijn het de mannen waar ze}{niet tegen kunnen}\\

\haiku{Ik heb vijfhonderd,.}{frank ik wil alles voor m'n}{geld wat ervoor staat}\\

\haiku{Maar toen ik beter,.}{was is hij direkt bij me}{teruggekomen}\\

\haiku{{\textquoteright} {\textquoteleft}Dat kun je nu wel,...}{zeggen maar of Zouzou dat}{zo leuk zou vinden}\\

\haiku{Hier ben je wat, de,,}{mensen kennen je je staat}{goed aangeschreven}\\

\haiku{Jij lijkt haar te veel,!}{op een man en van mannen}{heeft ze al genoeg}\\

\haiku{een kind is een kind,.}{ik zal het niet verstoten}{om zijn velletje}\\

\haiku{{\textquoteleft}Als ze mijn kind was,.}{geweest had ik haar niet in}{het vak laten gaan}\\

\haiku{De rode is voor,.}{Fifi die is zo bruin als}{beukenootjes}\\

\haiku{{\textquoteright} {\textquoteleft}Als we allemaal,?}{een roos hebben zingt u dan}{een liedje met ons}\\

\haiku{Met de vuisten op:}{tafel leunde zij voorover}{en siste ademloos}\\

\haiku{ze lachte met een.}{helder geluid dat ver over}{het water schalde}\\

\haiku{{\textquoteright} En toen ze maar \'even,.}{een hand durfde losmaken}{gaf ze hem een klap}\\

\haiku{{\textquoteright} {\textquoteleft}Nou, Leen, als het aan:}{jou lag deed je nog zaken}{met de vissen hier}\\

\haiku{{\textquoteleft}Een vrouw met geld en;}{een neus voor zaken zou een}{schip kopen als dit}\\

\haiku{{\textquoteright} {\textquoteleft}Geef me tenminste,,.}{wat te drinken twee glazen}{dan heb ik weer moed}\\

\haiku{Ze keek even steels om,;}{zich heen alsof ze iemand}{zocht en ging toen mee}\\

\haiku{Je schaamt je omdat,.}{je een standje krijgt maar niet}{omdat je vuil bent}\\

\haiku{Ik bond het om haar;}{hoofd en met de strik midden}{tussen haar krullen}\\

\subsection{Uit: Dagen van honger en ellende}

\haiku{Geen van de kwalen.}{der armoede gaat aan de}{Oldema's voorbij}\\

\haiku{in hun huwelijk;}{kwamen twee ongerepte}{lichamen samen}\\

\haiku{hij toonde te veel,.}{dat domheid en gemeenheid}{hem tegenstonden}\\

\haiku{Een schok van de kar.}{deed de groote koffiemolen}{op mijn neus vallen}\\

\haiku{De hoofdgrachten van:}{Amsterdam boezemden mij}{grooten eerbied in}\\

\haiku{de roofjes op mijn.}{hoofd gingen open en het bloed}{liep mij in den hals}\\

\haiku{Ik had dus mijzelf,.}{beloofd dit jaar mijn eerste}{communie te doen}\\

\haiku{Ik koos een stoep aan,;}{een der grachten uit om mijn}{les te bestudeeren}\\

\haiku{Zij gaf antwoord met,.}{zoo'n zachte stem dat ik het}{nauwelijks verstond}\\

\haiku{wat verder was een.}{oase van boomen en een}{grasveld met bloemen}\\

\haiku{{\textquoteright} {\textquoteleft}Vader{\textquoteright}, zei ik, {\textquoteleft}laat;}{mij vannacht tusschen moeder}{en u in slapen}\\

\haiku{Heeft zij je nog niet,?}{verteld wanneer ze weer een}{kindje gaat koopen}\\

\haiku{onder 't draaien.}{net zooveel leven maakte}{als duizend bijen}\\

\haiku{Zijn gezicht straalde,,;}{zijn blauwe oogen werden zwart}{zijn lippen vochtig}\\

\haiku{De achterdeur van;}{een huis op den Nieuwendijk}{kwam in het slop uit}\\

\haiku{Wij legden hem in,.}{de wieg waar hij den heelen}{nacht bleef doorslapen}\\

\haiku{ik had opgemerkt,.}{dat rijke menschen spreken}{als in de boeken}\\

\haiku{Als er sprake was,.}{van reizen verloor vader}{alle bezinning}\\

\haiku{Aan den overkant van,.}{de gracht kwam een vrouw aan die}{iets in haar schort droeg}\\

\haiku{{\textquoteright} {\textquoteleft}Toe moeder, omdat.}{ie zoo geschrokken is toen}{ie van zoo hoog viel}\\

\haiku{Jij en ik wachten,,.}{nooit om wat voor ons staat te}{laten verdwijnen}\\

\haiku{Tegen het voorjaar,'.}{werd Baatje zoo dik en vet dat}{t een lust was}\\

\haiku{{\textquoteleft}En dan, je begrijpt,.}{de muizen loopen zoo maar}{tusschen zijn pooten}\\

\haiku{op het oogenblik;}{van koopen keert ze hem om}{en ontdekt een barst}\\

\haiku{Toen ging ik terug,.}{naar de volksstraten waar de}{verkoop beter ging}\\

\haiku{maar hij was nog zoo'n,;}{kind dat hij er nauwelijks}{eenig verdriet van had}\\

\haiku{hij volgde haar in.}{het donkere gangetje}{v\'o\'or onze kamer}\\

\haiku{als ik doortrokken;}{was geweest van parfums en}{in kanten gehuld}\\

\haiku{V\'o\'or hij wegging, gaf:}{hij me een paar gulden en}{herhaalde nog eens}\\

\haiku{Op de trap sprak hij ',.}{me int Fransch  aan maar}{ik verstond hem niet}\\

\haiku{{\textquoteleft}Er schiet niets anders,}{over dan den heelen nacht te}{blijven rondloopen}\\

\haiku{hij liep om de bank,;}{heen raapte het pakje op}{en ging langzaam weg}\\

\haiku{Ontmoedigd was ik.}{tot laat in den avond bij die}{vriendin gebleven}\\

\haiku{Hij kwam voorzichtig,.}{weer rechtop het geldstukje}{tusschen de tanden}\\

\haiku{{\textquoteright} {\textquoteleft}Moeder, nou haalt die!}{valsche meid me naar zich toe}{om me pijn te doen}\\

\haiku{{\textquoteleft}Ik denk, dat 't op,,}{vader neer zal komen als}{de zaak vervolgd wordt}\\

\haiku{Hij zag er bleek en;}{vervallen uit als een}{kleine vagebond}\\

\haiku{{\textquoteright} Wat deed 't mij goed,!}{tegen die geweldige}{borst aan te liggen}\\

\subsection{Uit: Dagen van honger en ellende}

\haiku{Zo kwam mijn broertje,.}{Kees altijd bij ons terug}{toen we klein waren}\\

\haiku{verschrikkelijk op.}{tot het niet veel scheelde of}{mijn ogen waren dicht}\\

\haiku{De hoofdgrachten van:}{Amsterdam vervulden mij}{met diepe eerbied}\\

\haiku{{\textquoteleft}Lieve hemel, waar?}{haalt dat kleine mirakel}{die woorden vandaan}\\

\haiku{Ik stond daar naast de,.}{zuster bevend van schaamte}{en vernedering}\\

\haiku{Al onze arme.}{kleintjes zijn op die manier}{op school behandeld}\\

\haiku{En met een wit bord.}{erop als deksel leek het}{ons heel behoorlijk}\\

\haiku{Ik had nooit 's nachts;}{bloemen gezien en kende}{dat verschijnsel niet}\\

\haiku{{\textquoteright} {\textquoteleft}Vader,{\textquoteright} zei ik, {\textquoteleft}mag?}{ik vannacht tussen moeder}{en u in slapen}\\

\haiku{Geholpen door een}{van de dienstboden van het}{huis pakte moeder}\\

\haiku{Wij legden hem in,.}{de wieg waar hij de hele}{nacht bleef doorslapen}\\

\haiku{{\textquoteright} Gelukkig stond er.}{een kamerschut tussen ons}{en de anderen}\\

\haiku{Als er sprake was,.}{van reizen verloor vader}{alle bezinning}\\

\haiku{Ze moesten mij in bed,.}{stoppen de opwinding had}{me koortsig gemaakt}\\

\haiku{Aan de overkant van,.}{de gracht kwam een vrouw aan die}{iets in haar schort droeg}\\

\haiku{de klompjes brandden;}{maar heel langzaam doordat ze}{kleddernat waren}\\

\haiku{Maar ze dachten dat.}{ik een arme duvel was}{die geen cent bezat}\\

\haiku{En wat jou aangaat,,.}{vrouwtje het is hoog tijd dat}{je goed verzorgd wordt}\\

\haiku{Aangezien ik vaak,:}{niet kon slapen hoorde ik}{ze soms overleggen}\\

\haiku{ik wist dan wat ze.}{voorhadden en deelde in}{hun ongerustheid}\\

\haiku{{\textquoteleft}Kijk eens of Keetje,.}{slaapt die meid ligt soms hele}{nachten te woelen}\\

\haiku{{\textquoteright} {\textquoteleft}Nou, dan moest je maar.}{eens een paar flessen wijn voor}{moeder meenemen}\\

\haiku{{\textquoteright} {\textquoteleft}Ja, zo zou het wel,{\textquoteright}.}{kunnen zei Mina na een}{ogenblik nadenken}\\

\haiku{Zij was bang dat de.}{klant zou denken dat ik al}{in het leven zat}\\

\haiku{Op de trap sprak hij,.}{me  in het Frans aan maar}{ik verstond hem niet}\\

\haiku{Hij draaide rond op,:}{zijn hakken sloeg zich op zijn}{dijen en brulde}\\

\haiku{Ik ben het zat, een.}{menselijk wezen hoort niet}{tussen jullie thuis}\\

\haiku{De eerste avond dat,:}{zij van hun werk thuiskwamen}{schrokken wij van hen}\\

\haiku{ik dommelde een,.}{beetje telkens oplettend}{of er onraad was}\\

\haiku{Hein en ik keken.}{elkaar aan en begrepen}{elkaars gedachten}\\

\haiku{{\textquoteleft}Ik weet wel dat het,,!}{onzin is maar vertel toch}{door het is zo leuk}\\

\haiku{daar heeft hij gelijk,.}{in ze heeft al even weinig}{beenderen als vlees}\\

\haiku{{\textquoteright} Wat deed het mij goed,!}{tegen die geweldige}{borst aan te liggen}\\

\haiku{De kinderen aten,;}{op tijd werden gewassen}{en gingen naar school}\\

\haiku{Maar opeens, als zij,.}{de vijftig al gepasseerd}{is komt het terug}\\

\section{A. den Doolaard}

\subsection{Uit: De druivenplukkers}

\haiku{wanneer ze lastig.}{werden zou hij ze wel in}{een hoekje drukken}\\

\haiku{niets te beginnen}{viel omdat hij een erfstuk}{was van de vader}\\

\haiku{- Dat meisje in de,,?}{keuken met dat donkere}{haar is dat jouw vrouw}\\

\haiku{Ze slenterden langs.}{de fontein omlaag naar het}{arbeiderslogies}\\

\haiku{- Je kunt vloeken, je,,.}{kunt zuipen zooveel als je}{wilt en ik doe mee}\\

\haiku{- zei hij, terwijl hij.}{zijn vuisten uit zijn warme}{broekzakken haalde}\\

\haiku{Zoo gauw ik geld heb,:}{voor een nieuw glazen oog laat}{ik kaartjes drukken}\\

\haiku{Ik wist dat hij dien, ',.}{avondt was een week daarna}{met haar slapen ging}\\

\haiku{Boven op de muur.}{stonden ze tegen de maan}{als twee groote katten}\\

\haiku{De zon zonk langzaam.}{naar de schuin toeloopende}{wanden van de schelf}\\

\haiku{Andr\'e lachte zoo.}{hard dat de hofhond er van}{begon te bassen}\\

\haiku{Heb jij er dan geen,,?}{verdriet van onding dat je}{druiven verrotten}\\

\haiku{- De markies is bang -.}{zich te compromitteeren}{werd er gefluisterd}\\

\haiku{Was zij iets anders,,?}{dan een koude vrouw die slechts}{nam maar nimmer gaf}\\

\haiku{Eindelijk kwam er:}{bericht uit Marseille van}{een arrestatie}\\

\haiku{bijen zoemden door;}{de wuivende schaduwen}{der dadelpalmen}\\

\haiku{Zoodra hij kon.}{nam hij afscheid en ging zijn}{lijfwacht bestellen}\\

\haiku{Wat hebben wij te?}{maken in het land waar de}{zon een krater is}\\

\haiku{de jonge Saporta,.}{ligt bij Verdun de jonge}{Blondel in het Rif}\\

\haiku{- troostte Pepe, - ik,,.}{ga m'n geweer halen en}{m'n hengel zoet maar}\\

\haiku{maar wanneer Vladja,.}{zijn handen hard in elkaar}{klapt fluit ze nijdig}\\

\haiku{Laat mij gelukkig,,!}{worden terwijl ik in U}{onderga natuur}\\

\haiku{Op je hersens moet,!}{je dragen dat houdt de booze}{gedachten binnen}\\

\haiku{- Sta niet te kijken!}{als een echte Hottentot}{met een eendenhoofd}\\

\haiku{Ze leken op groote;}{harten zooals je op plaatjes}{van heiligen zag}\\

\haiku{De leege tobbe was.}{licht en Vladja wandelde}{tevreden terug}\\

\haiku{Twee en tachtig, en,.}{zoo rechtop al leunde hij}{zwaar op zijn stokje}\\

\haiku{- 't Is Monsieur, ' -.}{Laforgue die opt}{kantoor werkt zei hij}\\

\haiku{Achter het strooblok.}{hoorden ze de Haaientand}{zich overeindwoelen}\\

\haiku{Hij zag Alice, die.}{het kralengordijn van de}{keuken opzij sloeg}\\

\haiku{Maar haar groote zwarte:}{oogen keken donker en ze}{fluisterde verschrikt}\\

\haiku{Gisteren heeft hij,}{mij de hand gedrukt en ik}{zag dat hij blij was}\\

\haiku{Wanneer je zoo'n mooi,;}{khakihemd bezit dan heb}{je geen tweede noodig}\\

\haiku{het leven is de;}{weerschijn der wolken in een}{vijver geworden}\\

\haiku{- Maar waarom heeft U,?}{de tuinspiegel omgegooid}{Monsieur Hu{\^\i}tre}\\

\haiku{Hoe nederiger,!}{de ondergeschikte des}{te strenger het recht}\\

\haiku{Voor elke kus die;}{ik U niet toestond een dag}{van brandend verdriet}\\

\haiku{een edelhert, zooals er;}{nog nooit een gezien werd in}{de heele vallei}\\

\haiku{hij heeft gezien, hoe.}{ik veertien jaar geleden}{het gewei ophing}\\

\haiku{Er was een scherpe,.}{pijn in zijn keel dat was van}{verdriet om Andr\'e}\\

\haiku{Mis mannetje, er,.}{is er nog altijd een die}{springlevend rondloopt}\\

\haiku{Lieg tegen hem en! '!}{kijk hem dan aant Is of}{de bliksem je treft}\\

\haiku{De nachten zijn koel,.}{en donker maar mijn hart blijft}{wakker en slaat wraak}\\

\haiku{hij was reeds in het.}{stroomgebied dat hij twaalf jaar}{lang ontweken had}\\

\haiku{Hij had de kijker.}{snel vastgedraaid en keek in}{haar donkere oogen}\\

\haiku{Drie dagen later.}{verraste de Saporta hun}{schuldige blikken}\\

\haiku{Voor zoover ik weet, is...}{de afstand tusschen deze}{plaatsen zeer gering}\\

\haiku{maar nu keek ze hem,.}{dringend aan en hij moest de}{oogen wel opheffen}\\

\haiku{Dit was de kleine,;}{gravin niet meer zooals hij haar}{vroeger gekend had}\\

\haiku{Dit was het water,!}{waaraan hij zijn gedachten}{had meegegeven}\\

\haiku{Ben ik het geweest,?}{die haar zooveel pijn deed dat}{zij zoo kijken moet}\\

\haiku{Hij boog zich langzaam,.}{voorover alsof het hem groote}{inspanning kostte}\\

\haiku{Jullie kunnen geen.}{glas ophebben of jullie}{bulken als stieren}\\

\haiku{De acrobaat breidde,.}{zijn armen uit en zette}{een hooge triller aan}\\

\haiku{Vladja lag kauwend.}{op zijn ellebogen en}{keek naar de verte}\\

\haiku{In de nevel werd,,,.}{gevochten en Stephane}{jouw man was de held}\\

\haiku{en hij kon haar niet,.}{afweren want zij greep als}{een verdrinkende}\\

\haiku{Ik heb tegen hem...}{gezegd dat we elkaar reeds}{lang weer ontmoetten}\\

\haiku{Ineens bukte hij '.}{zich en greep de wagenas}{int midden beet}\\

\haiku{Hij vroeg alleen aan.}{Monsieur Hu{\^\i}tre wie die}{Andr\'e geweest was}\\

\haiku{Hij nam vlug een slok,.}{water om de andere}{kant uit te kijken}\\

\haiku{Hij raakte haar zoo,.}{voorzichtig aan alsof ze}{een heilige was}\\

\haiku{Zooiets dragen ze,.}{bij ons thuis aan deze kant}{van de Karpathen}\\

\haiku{In het The\^atre de.}{Dix Heures zag ik op een avond}{Mr. en Miss Joering}\\

\haiku{Waarom moest iemand?}{die twee frank meer verdiende}{zoo'n kijftoon aanslaan}\\

\haiku{Opeens hoorden zij;}{een harde kreet en vlogen}{bevend uit elkaar}\\

\haiku{- Zij hielpen hem naar,.}{het raam waar hij loodzwaar in}{hun armen leunde}\\

\haiku{Andr\'e vloog overeind,.}{en zakte toen langzaam weer}{naar zijn plaats terug}\\

\haiku{- bromde Henri, - hij, '?}{is beter zag je hem dan}{niet aant raam staan}\\

\haiku{Een auto met  ,.}{lichten op stond voor de stoep}{het portier hing open}\\

\haiku{Buiten de parkmuur.}{ging Andr\'e mismoedig aan}{de wegkant zitten}\\

\haiku{- Jammer - fluisterde, -,.}{hij nu weet hij niet meer dat}{er recht gedaan is}\\

\haiku{Het is nu voorbij -,.}{mijn vader stierf in vrede}{dank zij ons bedrog}\\

\haiku{Twintig jaar kalme,?}{liefde is dat meer waard dan}{de roes van een maand}\\

\haiku{Door de leege velden,.}{togen de plukkers weg bij}{twee\"en en drie\"en}\\

\haiku{- En toch zal ik hem -.}{zien uitvaren knerste hij}{tusschen zijn tanden}\\

\haiku{Van dertig meter ',,}{hoog int  water dat}{overleefde je niet}\\

\haiku{Zoo heet het eiland,.}{waar Andr\'e naar toe gaat en}{de wijn heet ook zoo}\\

\haiku{Als hij \'e\'en vinger,.}{uitsteekt dan schiet ik hem een}{kogel in zijn beenen}\\

\subsection{Uit: Ori\"ent-Express}

\haiku{Drie uur lang bleven;}{zij samen opgesloten}{in de woonkamer}\\

\haiku{{\textquoteleft}Klimmen jullie naar.}{beneden met de kaars en}{laat het luik vallen}\\

\haiku{{\textquoteleft}Denk je dat ik de?}{minste hoop heb om nog drie}{maanden te leven}\\

\haiku{De pope schraapte,.}{een paar maal zijn keel dat het}{hol door de kerk klonk}\\

\haiku{Het licht viel op de.}{landkaart van rimpels in zijn}{donkerroode nek}\\

\haiku{Dat zal morgenavond!}{een vlam geven van hier tot}{Constantinopel}\\

\haiku{Dat had Damian,.}{toch maar slecht uitgerekend}{vlak voor de opstand}\\

\haiku{En de opstand bij!}{ons zal net zoo vallen als}{die vervloekte vlam}\\

\haiku{Kroum schreeuwde door, maar.}{Kosta kon in het laaien zijn}{woorden niet verstaan}\\

\haiku{{\textquoteright} {\textquoteleft}Ze heet Todor,{\textquoteright} zei, {\textquoteleft}.}{Milja glimlachenden geef}{nu die schaar maar hier}\\

\haiku{Er is geen golving,.}{in ze ligt precies vlak en}{oneindig rustig}\\

\haiku{Tusschen Prilep en het.}{klooster Svati Archangel}{ligt het dorp Markov Grad}\\

\haiku{De vleugelen van,.}{den aartsengel waren groot}{ruim en genadig}\\

\haiku{de heele tcheta.}{had dekking gezocht in de}{rotsen achter hem}\\

\haiku{morgen zouden het,.}{er negen zijn overmorgen}{een heel bataljon}\\

\haiku{De Turken waren.}{driester geworden en gingen}{tot den aanval over}\\

\haiku{maar voor het eerst van,.}{zijn leven was hij bang bang}{voor de kleeren der dooden}\\

\haiku{met het salvo der.}{anderen verdwenen drie}{andere schimmen}\\

\haiku{{\textquoteright} Zijn stem was kort en.}{hij schoot de woorden telkens}{met een hik eruit}\\

\haiku{Maar niemand kon Kroum,.}{terug houden voor Kroum was}{de vrijheid alles}\\

\haiku{Af en toe keek hij.}{naar de bergen om te zien}{of zij opschoten}\\

\haiku{Er stonden nu veel;}{verlaten boerderijen}{in Macedoni\"e}\\

\haiku{Toch was het land hier;}{langs de zijpaden leeg en}{er klonk geen gerucht}\\

\haiku{Zoo was hun leven.}{geweest het laatste jaar voor}{de revolutie}\\

\haiku{De zon glansde stil,.}{op zijn geel verdroogd hoofd en}{de roode tulband}\\

\haiku{Milja was van den.}{ezel gegleden en suste}{Stana die huilde}\\

\haiku{Want het waren net.}{uitgerekte huizen met}{hun ritsen raampjes}\\

\haiku{Ik heb er geen een,.}{en in die trein gezeten}{heb ik ook nog niet}\\

\haiku{Hij ging rustig door,.}{met priemen en had haar zelfs}{niet aangekeken}\\

\haiku{Mile bromde ook.}{en daarom maakte ze zich}{zoo klein mogelijk}\\

\haiku{Want nu zijn we in.}{Servi\"e en deze rivier}{heet de Morava}\\

\haiku{{\textquoteleft}Vervloekt kind,{\textquoteright} en sloeg.}{toen ineens spijtig met zijn}{vuist tegen zijn mond}\\

\haiku{Hij gaf haar een klap:}{in den rug met den steel van}{zijn zweep en snierde}\\

\haiku{Misschien gaat hij wel,.}{naar een gevangenis waar}{hij nooit meer uitkomt}\\

\haiku{Er waren wel tien.}{man noodig om hem opzij van}{den weg te trekken}\\

\haiku{Met dat al blijft een.}{paal een paal en Ristitch was}{een voorzichtig man}\\

\haiku{{\textquoteright} En tegelijk kijk;}{je om naar de deur of er}{niet iemand luistert}\\

\haiku{Dwars door het schelle,.}{licht dat hard tegen haar oogen}{aansprong liep een man}\\

\haiku{Het was alsof de.}{saaie heuvels ineens wakker}{geworden waren}\\

\haiku{vroeger was de muur,{\textquoteright}.}{hooger hij staat er al over}{de vijfhonderd jaar}\\

\haiku{Ik weet niet of je,.}{dit begrijpt maar ik kan het}{niet anders zeggen}\\

\haiku{Ik verlangde naar.}{m{\`\i}jn jonge vrouw en naar ons}{ongeboren kind}\\

\haiku{ik kan niet anders,.}{omdat ik eens onder mijn}{eigen dooden wegkroop}\\

\haiku{Een witte vlinder.}{fladderde langs haar heen maar}{ze merkte het niet}\\

\haiku{Ze trok haar gele.}{hoofddoek van haar haren en}{wuifde er woest mee}\\

\haiku{Jij denkt natuurlijk;}{aan woeste ritten te paard}{en God weet wat meer}\\

\haiku{Ja, voor het jonge,;}{geslacht is het beter in}{dat land zonder dwang}\\

\haiku{Dit was het nieuwe.}{dorp der kolonisten uit}{Noord-Servi\"e}\\

\haiku{{\textquoteright} zei hij met een zwaar.}{hoofdknikken terwijl hij zijn}{oogen bijna dicht deed}\\

\haiku{{\textquoteright} {\textquoteleft}Maar waarom leef je?}{dan in dit dorp Oom Kosta en}{niet in Radovo}\\

\haiku{Toch was het niet voor,...{\textquoteright}}{niets want ik vond dit dorp en}{ben er tevreden}\\

\haiku{Hij krabde aan zijn,.}{bult nam langzaam een stap en}{kwam breed voor haar staan}\\

\haiku{Het waren niet de,:}{oogen van een dichter maar van}{een man van de daad}\\

\haiku{en naar zijn oogen die,,.}{zooals hij plotseling zag zeer}{verschillend keken}\\

\haiku{Ik stond op die lijst.}{als No. 4 en ik heb die}{plaats niet gestolen}\\

\haiku{Want wanneer je door,:}{een slang gebeten bent dan}{geeft sabbelen niets}\\

\haiku{de vos van Stankovitch';}{en de schimmelponey van}{Stankovitch zoon Christo}\\

\haiku{Dooden spreken niet en.}{zoo was Damianovitch}{ten minste gered}\\

\haiku{Hij kende dat soort.}{gevoelens en hij wist hoe}{lang ze door vraten}\\

\haiku{Het was natuurlijk.}{een kleinigheid dit kind te}{laten verdwijnen}\\

\haiku{Een dubbele kreet.}{kwam uit de twee geschonden}{kindergezichten}\\

\haiku{Ze wist dat de zon;}{nog een uur noodig had tot den}{uitersten raamstijl}\\

\haiku{Daarna werd de zon.}{een laat vuur op den heuvel}{en de nacht begon}\\

\haiku{het rood van wilde,.}{zuring papaverrood en}{zonsondergangrood}\\

\haiku{Stankovitch nam haar mee!}{het bosch in en twee dagen}{later was ze dood}\\

\haiku{Dan gooide je zoo'n,.}{dop weg omdat hij leelijk}{en nutteloos was}\\

\haiku{hoe zijn vader die,.}{hij voor een heilige hield}{door en door slecht is}\\

\haiku{Doch de jongen sprak,}{meteen langzaam door alsof}{hij zijn woorden \'e\'en}\\

\haiku{Hij hing langzaam zijn.}{patroongordel en zijn tasch}{met papieren om}\\

\haiku{Er kwam alleen een.}{kleine rimpel tusschen haar}{strakke wenkbrauwen}\\

\haiku{Todor Alexandrov.}{had trouwens gezegd dat hij}{dit niet wenschte}\\

\haiku{De trein gaat om half,,.}{twaalf want je reist niet over Nisch}{maar over Bucarest}\\

\haiku{Naar dompig vette,.}{schapenvachten naar zuur zweet}{en scherpe uien}\\

\haiku{En geen wonder dat.}{hij eindelijk begreep hoe}{groot haar liefde was}\\

\haiku{Beheerschte hij?}{haar dan niet zoo volkomen}{als hij gedacht had}\\

\haiku{met oogen die te groot.}{en neusvleugels die te hoog}{en te rond waren}\\

\haiku{{\textquoteright} Het laatste woord was.}{een langgerekte kreet die}{wegzwierf in den wind}\\

\haiku{Er zijn honderden.}{handen die de leeuwenvlag}{willen borduren}\\

\haiku{Het was een vreemd en,;}{onaardsch licht niet van den dag}{en niet van den nacht}\\

\haiku{Zij volgde hem, en.}{scheurde haastig een tak vol}{roode bloesems af}\\

\haiku{De steenen hier waren;}{ook heel anders dan in de}{heuvels rond Kounovo}\\

\haiku{maar voor de rest leek:}{Chandanov op een pop in}{een panopticum}\\

\haiku{{\textquoteright} Meteen boog hij zich}{naar Christo toe en keek hem}{strak aan als wilde}\\

\haiku{De rest zullen we.}{in de groote Europeesche}{couranten lezen}\\

\haiku{Want wat je daar aan,!}{je kin hebt kan je heusch}{niet langer dragen}\\

\haiku{Uit de verte zag.}{hij enkel haar rug en haar}{breede bloote voeten}\\

\haiku{Hij keek haar vragend,:}{aan maar ze pakte zijn hand}{beet en zei haastig}\\

\haiku{De populieren.}{rekten zich nog kaal tegen}{het vale avondrood}\\

\haiku{Hij draaide haar om,.}{en duwde haar weg in de}{richting van het dorp}\\

\haiku{De eerste tijd had;}{het zware ding hem bij het}{zitten gehinderd}\\

\haiku{{\textquoteright} Meteen nam hij zijn.}{oogen van Christo weg en keek}{vluchtig de kring rond}\\

\haiku{Uit het kleine park.}{er naast kwam de zware}{geur van gesproeid gras}\\

\haiku{Het licht glansde in;}{de zwartzijden jurk die strak}{rond haar heupen lag}\\

\haiku{Je hebt best tijd om.}{een glas pruimen jenever}{met mij te drinken}\\

\haiku{Voivoda Petrov,?}{wie is de moordenaar van}{Todor Alexandrov}\\

\haiku{Ze stribbelde niet,.}{tegen maar legde haar hoofd}{tegen zijn schouder}\\

\haiku{{\textquoteright} Haar oogen stonden groot.}{en star en hij moest weer aan}{haar moeder denken}\\

\haiku{{\textquoteleft}Ik wil je helpen,,;}{eggen Oom Kosta en ploegen}{daar op je akker}\\

\haiku{maar ik ben bang dat!}{je naar vuur zoekt in de asch}{van verleden jaar}\\

\haiku{en ook zij had een,.}{dierbare doode en hij}{stond achter elk raam}\\

\haiku{Hij trad voor de deur.}{en keek treurig rond over het}{bijna boomlooze land}\\

\haiku{Hij besloot eerst het,;}{graf te graven nu terwijl}{iedereen weg was}\\

\haiku{Weldra moest hij de.}{losgewoelde steenen al hoog}{over de rand werpen}\\

\section{Aernout Drost}

\subsection{Uit: Hermingard van de Eikenterpen. Een oud vaderlands verhaal}

\haiku{Enige beknopte;}{toelichtingen kwamen mij}{noodzakelijk voor}\\

\haiku{Vrouwe wats geschied,,.}{Ik wint of ik bliver dood}{Nu blijft gezond}\\

\haiku{Zij leefden sinds in;}{gedurige worstelstrijd}{met hun overwinnaars}\\

\haiku{Ongeduldig sloeg;}{de Bard de grijsaard en de}{heldenzoon gade}\\

\haiku{thans durfde hij zich.}{in de toekomst weinig met}{Freya's gunst vleien}\\

\haiku{Het vroom gebed en.}{oprechte offer zijn hun}{welbehagelijk}\\

\haiku{Na de nacht doorwaakt,.}{te hebben trokken wij met}{de dageraad voort}\\

\haiku{De Goden zonden,;}{mij deze nacht een droom een}{vreselijke droom}\\

\haiku{Deze droom is een.}{vreselijke bode van}{de toorn der Goden}\\

\haiku{{\textquoteright} riep hij uit, en de.}{vreugde schitterde in het}{levendig bruin oog}\\

\haiku{de Usipeter is;}{niet terug en het lot der}{onzen onbeslist}\\

\haiku{-{\textquoteright} Met kinderlijke;}{trots zwaaide de grijsaard het}{oude krijgstuig}\\

\haiku{gij haatte mij, ik,.}{weet het de wil uwer Goden}{is u een wellust}\\

\haiku{Mijn ziel is los van '.}{de aarde en bovent}{aards verheven}\\

\haiku{ik gevoel grote,;}{zeer grote en verheven}{gewaarwordingen}\\

\haiku{Maar algemene,.}{schrik verspreidde zich toen de}{Bard terugkeerde}\\

\haiku{Men sleurde mij naar,,!}{de hal en wij vonden u}{aldaar gebiedster}\\

\haiku{Telkens heeft zij met;}{dringende belangstelling}{naar u vernomen}\\

\haiku{Huwelijksmin en!}{vaderliefde deden hem}{alles vergeten}\\

\haiku{De beker bleef op;}{de hertogelijke dis}{onaangeroerd staan}\\

\haiku{door de struwelen,.}{baande hij zich een weg want}{hij herkende mij}\\

\haiku{toen mijn krachten mij,,.}{begaven sleurde men mij}{met zacht geweld voort}\\

\haiku{Beklaag, beklaag u,!}{nimmer hem de naamloze}{te moeten noemen}\\

\haiku{Gaarne zou ik hem,.}{troosten maar nimmer wil hij}{mij zijn leed klagen}\\

\haiku{- Zo de reine Geest,{\textquoteleft}}{der Godheid Die zich uitstort}{van heur troon?-}\\

\haiku{Timotheus was op,:}{de knie\"en gezonken geen}{hunner die niet bad}\\

\haiku{de godsdienst, die wij,!}{heden beleden bij elk}{lotgeval heilig}\\

\haiku{Hoe beleidvol de,;}{Bard haar volgde de jonkvrouw}{ontdekte hem toch}\\

\haiku{eens droomde ik van,:}{het kind hetwelk mijn gade}{onder het hart droeg}\\

\haiku{een onvermengde.}{zaligheid zal de aardse}{onspoed vervangen}\\

\haiku{{\textquoteright} aldus ving zij aan, {\textquoteleft};}{mijn gebiedster bevindt zich}{niet in haar woning}\\

\haiku{Daarom was Hermingard;}{in de toren van Witte}{Geertrud gekerkerd}\\

\haiku{Schoon 't duister heers',, ',!}{geen licht ons oog mag treffen}{k Zal steeds min God}\\

\haiku{de Goden laten;}{niet toe dat hun gewijden}{bedrogen worden}\\

\haiku{{\textquoteright} riep zij uit, en wierp.}{zich voor de ontzaglijke}{vrouw op de knie\"en}\\

\haiku{{\textquoteleft}Bedrieg u niet,{\textquoteright} ging, {\textquoteleft};}{vrouw Geertrud voortik wil u}{niet  bedriegen}\\

\haiku{gij zult u in het;}{aanstaand geluk v\'o\'or deszelfs}{bestaan verheugen}\\

\haiku{Hem alleen vrees ik,.}{tot Hem alleen bepaalt zich}{mijn Godsdiensthulde}\\

\haiku{{\textquoteright} Hermingard keerde naar,:}{Geertruds verblijf terug en}{trad aan het venster}\\

\haiku{Nu toog hij naar de.}{schuilplaats der rovers en werd}{door hen gevangen}\\

\haiku{Toen Timotheus zweeg,.}{strekte zijn pleegvader de}{handen naar hem uit}\\

\haiku{Hij hoort mijn gebed......}{die bede moge in de}{oordeelsdag pleiten}\\

\haiku{De grijsaard was als,;}{versteend en luisterde van}{boezemangst vervuld}\\

\haiku{Weldra volgde er.}{een zeer treffend toneel van}{mannelijke rouw}\\

\haiku{als de maan aan de,;}{blauwe hemel bleekt zal uw}{as verzameld zijn}\\

\haiku{{\textquoteright} wendde Welf zich tot, {\textquoteleft}?}{haarweet gij de schuilplaats van}{de zoon der ondeugd}\\

\haiku{{\textquoteright} {\textquoteleft}Christus zal met haar,{\textquoteright}.}{zijn hernam Marcella met}{betraande blikken}\\

\haiku{nog stormt het daar wild,;}{en woedend wanneer ik aan}{het verleden denk}\\

\haiku{{\textquoteright} Zo sprekende, had;}{hij zich op de zodenbank}{nedergeworpen}\\

\haiku{zijn ouderdom bleef,?}{gespaard waarom zou die mijns}{vaders geknot zijn}\\

\haiku{er       Landbouw der*.}{Batavieren   ~         bleef}{hem geen twijfel over}\\

\haiku{aan de andere.}{zijde van het sterfbed stond}{de wichelares}\\

\haiku{Eindeloze troost.}{schenkt mij deze genade}{in het laatste uur}\\

\haiku{{\textquoteleft}Waarom zouden wij,,,!}{wenen als zij die geen hoop}{hebben Timotheus}\\

\haiku{Marcella had met;}{hem de weleer bewoonde}{woning betrokken}\\

\haiku{hoe gelukkig maakt,!}{mij uw leven uw liefde}{en uw onderwijs}\\

\haiku{De grote God schonk.}{mij zijn genade door het}{dierbaarste werktuig}\\

\haiku{mij een waardiger!}{tolk der hemelse leer en}{verlichtte mijn geest}\\

\haiku{Ik werd een Christen,,!}{geloofd zij het Lam dat voor}{onze zonden stierf}\\

\haiku{Wij vierden aldaar '.}{het twintigste jaar vans}{keizers regering}\\

\haiku{{\textquoteright} Met verrukking en.}{deelneming hadden allen}{het verhaal gehoord}\\

\haiku{sedert ongeveer.}{tien jaar was zijn naam hier te}{lande een begrip}\\

\haiku{Toen Drost Hermingard schreef,.}{kon hij dus putten uit een}{rijke traditie}\\

\haiku{de kardinale.}{christelijke deugden zijn}{reeds in hun bezit}\\

\haiku{namelijk toen zij (-).}{de apostel Paulus hoorde}{sprekenHand. 16:1415}\\

\haiku{uit lied cxvi van}{Ecclesiasticus oft}{de wyse sproken}\\

\haiku{ongetwijfeld wist (,,).}{E. Gibbon Decline and}{Fall ii Ch. 20}\\

\haiku{uit het gedicht {\textquoteleft}Aan{\textquoteright}.}{de ontluikende jeugd door}{Willem Bilderdijk}\\

\haiku{Camillus heette.}{de bruilofttochtgeleider}{bij de Romeinen}\\

\section{E. du Perron}

\subsection{Uit: Het land van herkomst}

\haiku{hoe ze ons kunnen?}{beletten om bijvoorbeeld}{de maan te voelen}\\

\haiku{Geloof H\'everl\'e;}{niet als hij zegt dat je veel}{van een Fransman hebt}\\

\haiku{Zeg maar eens eerlijk.}{wat voor jou re\"eel is in}{deze omgeving}\\

\haiku{nog anders dan je!}{gedacht had het persoonlijk}{te zullen maken}\\

\haiku{Als hij mijn vriend is,}{waarom ben ik dan niet met}{hem op mijn gemak}\\

\haiku{Hij stelt mij voor de:}{bar van Poccardi in}{te gaan en bestelt}\\

\haiku{Misschien heeft hij zelf,;}{te veel vadergevoelens}{oppert Goera\"eff}\\

\haiku{ik ben werkelijk,.}{zuiver een senorito}{een bourgeoiszoontje}\\

\haiku{De verandering,}{smolt langzaam weg ik lag naar}{mijzelf te kijken}\\

\haiku{de ziekte van mijn,.}{moeder zich spoedig genoeg}{moest laten gelden}\\

\haiku{{\textquoteleft}In die tijd, als je,{\textquoteright}.}{acht-en-twintig was was je}{meteen veertig}\\

\haiku{er was bijna geen,.}{weerstand meer in het kleine}{verzwakte lichaam}\\

\haiku{Hij is helemaal,!}{van streek want hij heeft nooit aan}{dit alles gedacht}\\

\haiku{heel broos en fijn, met.}{een prachtig teint en destijds}{gitzwart haar en ogen}\\

\haiku{altijd vertelde,,}{hij mij van mijn vader met}{een soort oude vrees}\\

\haiku{(Als mijnheer Ducroo,.)}{zijn knevels maar opdraaide}{beefden wij allen}\\

\haiku{omdat men degeen.}{is die de bekentenis}{heeft aangetrokken}\\

\haiku{De tweede was een {\textquoteleft}{\textquoteright};}{blanke nona en heette}{Jeanne Ende}\\

\haiku{Op een dag werd ik:}{daar door een dikke Duitser}{gefotografeerd}\\

\haiku{Ik ben vergeten;}{dat het licht werd uitgedraaid}{toen de film begon}\\

\haiku{Zij was een lange.}{vrouw met bijna spierwit haar}{maar een glad gezicht}\\

\haiku{{\textquoteright} Yoeng had vooral last;}{van geesten wanneer hij de}{ramen moest sluiten}\\

\haiku{De vensters ervan;}{waren toen smal en zwart en}{van tralies voorzien}\\

\haiku{De laatste ronde.}{werd tussen Tjang Panel en}{Lies uitgevochten}\\

\haiku{Bella is op haar:}{lach doorgegleden in een}{heel ander verhaal}\\

\haiku{Ongeveer in het.}{midden van de drijvende}{bal\'e was het dorp}\\

\haiku{het behoorde toe (),;}{aan de loerahdorpshoofd die}{Pa Djoewi heette}\\

\haiku{Toen hij stierf, was er.}{niemand bij hem en hijzelf}{was geheel vervuild}\\

\haiku{Mijn vader trok zich,;}{toen al bij voorkeur in bed}{terug of hij las}\\

\haiku{Ik kende deze.}{listen van weggaan trouwens}{al van heel jong af}\\

\haiku{\`en omdat zij niet,,}{alleen kon zijn zouden een}{boek kunnen vullen}\\

\haiku{Mijnheer was uit, zei,;}{het dienstmeisje maar zou wel}{dadelijk komen}\\

\haiku{Maar de liefde van.}{de personages gaat ons}{dan haast niet meer aan}\\

\haiku{Als de man van wie,.}{zij houdt evenzeer bedreigd wordt}{gaat ook dit niet op}\\

\haiku{Haar moeder trouwens,;}{was een Javaanse en kwam}{haar soms opzoeken}\\

\haiku{En wat krijg ik van?}{je als oom Edwin en ik}{later gaan trouwen}\\

\haiku{Door de tralies kon,;}{ik haar toch soms zien bidden}{languit op de vloer}\\

\haiku{ik proberen mij,.}{met hen te amuseren wat}{zelden gelukte}\\

\haiku{De vrouwen waren,:}{opgetogen over hem en}{mijn moeder zei mij}\\

\haiku{{\textquoteleft}Nah, en je zult niet,,,...}{gelo-oven maar oom ja-a}{droomde van een pauw}\\

\haiku{{\textquoteright} Mijn moeder was ook;}{in Brussel en Grouhy door}{hem teleurgesteld}\\

\haiku{{\textquoteleft}kijk, ik verdedig,,.}{mij niet eens ik heb er geen}{lust zelfs geen tijd voor}\\

\haiku{Hij liep mij hardop;}{pratend achterna om mij}{te laten lachen}\\

\haiku{{\textquoteright} Toen draaide ik hem.}{mijn rug toe en ging naar het}{karretje terug}\\

\haiku{verder had men hem;}{rijk getooid op een paardje}{laten rondrijden}\\

\haiku{op een dag stelde.}{ik hem voor deze kunst op}{mij te beproeven}\\

\haiku{omdat ik kans had.}{er de chef-corrector}{nog aan te treffen}\\

\haiku{Als mijn ouders dan,.}{in de fabriek waren kwam}{Pieng mij bekrijsen}\\

\haiku{{\textquoteleft}Wacht, Isnan, tot ik!}{groot ben en ik zal je net}{zo behandelen}\\

\haiku{Soms kwam hij toch met,:}{gaten in zijn vel thuis eens}{midden in zijn kop}\\

\haiku{braaf beest, hij heeft er{\textquoteright}.}{zelfs niet over gedacht om een}{van ons te bijten}\\

\haiku{Het zag er vreemd uit.}{en was toch nog een beetje}{tijger gebleven}\\

\haiku{Ik herinner mij.}{van deze laatste tijd niet}{veel meer dan een sfeer}\\

\haiku{hij schrijft terug dat.}{hij uitstel heeft aangevraagd}{en verkregen}\\

\haiku{ik wist de toon niet.}{te treffen die succes in}{grotere kring had}\\

\haiku{Het kwam niet in mij;}{op verlegen te zijn voor}{deze getuige}\\

\haiku{Met mijn boekentas;}{als gewoonlijk over de vrije}{arm sprong ik eruit}\\

\haiku{zij werd in zijn plaats,.}{door de agenten gegrepen}{maar zij hield van hem}\\

\haiku{hij zei dat hij er:}{niet over dacht om kostgeld voor}{hem te betalen}\\

\haiku{toch is zijrond in,.}{haar vlees geworden zooals gij}{haar niet gekend hebt}\\

\haiku{ik verbeeldde mij,:}{niet in de plaats van Darma}{te zijn maar ik dacht}\\

\haiku{Meneer Ducroo, heb,.}{ik gemerkt leert veel buiten}{de H.B.S. en hier niets}\\

\haiku{Later ontmoette}{ik hem bij de kapper en}{viel het mij weer op}\\

\haiku{Toen ik mijn moeder,.}{getroost had fietste ik naar}{de gevangenis}\\

\haiku{De cipier was een;}{Europeaan die eruit}{zag als een Alfoer}\\

\haiku{Het hinderde mij;}{toch dat niemand van ons bij}{Leni verder kwam}\\

\haiku{als zij die niet zelf.}{wilde uitkiezen moesten wij}{er iets op vinden}\\

\haiku{Junius, die nooit,;}{bij ons kwam zitten zag het}{uit de verte aan}\\

\haiku{zodra hij mij zag}{reed hij met het ringetje}{zwaaiend naar mij toe}\\

\haiku{Er was maar \'e\'en ding,,:}{merkte Junius op dat}{wel raar in hem was}\\

\haiku{Op een Zondagavond;}{in de soos kwam ik weer in}{de buurt van Hetty}\\

\haiku{Maar het ophalen.}{van de verloren beurten}{lukte mij niet meer}\\

\haiku{grijsachtig wit en:}{koel in zijn grote tuin vol}{romantisch lommer}\\

\haiku{Toen ik volwassen;}{was zocht ik in gesprekken}{soms toenadering}\\

\haiku{{\textquoteleft}want alleen door oom{\textquoteright}.}{Van Kuyck hebben we ons}{fortuin behouden}\\

\haiku{{\textquoteright} Welnu, Mad, Ida is,,:}{hier zij heeft mij leeren kennen}{ze zeide alleen}\\

\haiku{hier zag ik Trude.}{die zeker een halfuur om}{mij had gelachen}\\

\haiku{{\textquoteright} De jonge man, met:}{glinsterende ogen en een}{glimlach vol snaaksheid}\\

\haiku{Het was even donker;}{en omstreeks half 7 toen wij}{er terugkwamen}\\

\haiku{De derde maal was.}{ik 17 en het gebeurde}{in Tjitjalengka}\\

\haiku{Zij had mij gezegd}{dat zij zeker dagen zou}{laten voorbijgaan}\\

\haiku{ik was bij oom Van,}{Kuyck gelogeerd om 11}{uur nam ik op straat}\\

\haiku{ik kon naar Arthur, die.}{in de buurt woonde en in}{een tuinkamer sliep}\\

\haiku{die heeft je, ik weet...{\textquoteright}.}{niet die heeft je gemaakt tot}{een mensenhater}\\

\haiku{{\textquoteleft}Zie je, Ducroo, ik,!}{ben een goed mens maar d\`at kan}{ik nou niet hebben}\\

\haiku{ik bedacht dat ik;}{er kon komen met het geld}{dat ik nog over had}\\

\haiku{Wij hadden zolang;}{gewacht dat wij de kleine}{buit voor lief namen}\\

\haiku{{\textquoteleft}Als je niet oppast,,{\textquoteright},.}{An ga jij er eens ook zo}{uitzien zei Taco}\\

\haiku{{\textquoteright} heb ik mij dikwijls;}{afgevraagd als ik met hem}{samen was geweest}\\

\haiku{Voor een heel ras is.}{een groot man trouwens altijd}{de grote acteur}\\

\haiku{Zoals je op de{\textquoteright}.}{dansles ook van een meisje}{kunt leren dansen}\\

\haiku{hij kan er niet meer{\textquoteright}.}{aan doen dan ik en heeft er}{al last genoeg van}\\

\haiku{Hij was dol op zijn,;}{enige zoon die als kind een}{tenger knaapje was}\\

\haiku{Hij werd rood in zijn,:}{gezicht en ik ook maar hier}{lachte ik hem uit}\\

\haiku{Hij liep weg, zeggend,.}{dat het best was en die dag}{zag ik hem niet meer}\\

\haiku{We hebben daarbij,.}{17 man verloren het was}{een vrij grote troep}\\

\haiku{De rest is alleen}{thuisgekomen omdat een}{inlands sergeant}\\

\haiku{Van de aanvallers,,,.}{er waren er 11 geweest}{10 dood \'e\'en gesmeerd}\\

\haiku{je hebt niet de tijd.}{aan iets te denken als je}{met hem bezig bent}\\

\haiku{{\textquoteright} {\textquoteleft}Na Elly's dood... en,.}{ikzelf achteraf heeft het}{me toch aangepakt}\\

\haiku{hij heeft het gemerkt:}{toen hij dagelijks reisde}{per vliegmachine}\\

\haiku{hij is nu al 3.}{maanden hier en dit is dus}{zijn eerste rapport}\\

\haiku{In mijn schoongeboend:}{hotelletje een brief van}{Jane uit Bretagne}\\

\haiku{Ik vond zelf de grap.}{een beetje ver gedreven}{en ging naar Grouhy}\\

\haiku{en met 40 colli,.}{bagage die telkens moesten}{worden nageteld}\\

\haiku{en bleek spoedig mijn.}{moeder niet zo goed meer te}{kunnen masseren}\\

\haiku{de zogeheten,,.}{meesters de bedienden de}{log\'e's en ikzelf}\\

\haiku{zijn opmerkingen;}{over toestanden klonken zelfs}{niet meer als vroeger}\\

\haiku{In de eerste plaats;}{was er natuurlijk haar werk}{van iedere nacht}\\

\haiku{En bijwijze van...{\textquoteright}}{navelstreng een touwtje als}{bij kinderballons}\\

\haiku{Het is of ik een,.}{satire schrijf en toch is}{het niet geheel waar}\\

\haiku{Nu is het zelfs geen,;}{voorstad meer maar een wijk op}{de grens van Parijs}\\

\haiku{Ik zou haar een ring,?}{willen geven maar zou ze}{die niet inslikken}\\

\haiku{Je stelt een wet in,,.}{zei H\'everl\'e nog daarmee}{sta je altijd sterk}\\

\haiku{Toen ik zowat elf,;}{was vond ik op een dag mijn}{moeder huilende}\\

\haiku{Mijn angst dat ik d\`at!}{in die prachtige salon}{kon achterlaten}\\

\haiku{{\textquoteleft}Mm... en toen ik in,.}{Parijs was kreeg ik ook hier}{opeens genoeg van}\\

\haiku{je zult merken dat,{\textquotedblright}.}{ik gelijk heb maar ik laat}{je de plaats nu vrij}\\

\haiku{Ik ging met Sjoera:}{samenwonen en wilde}{Tanja niet meer zien}\\

\haiku{Toen zij wakker werd.}{had men haar tasje met het}{geld weggenomen}\\

\haiku{hij had haar in het;}{Schwarzwald ontmoet en gepoogd}{haar te verleiden}\\

\haiku{Ik bracht het gesprek:}{op Tanja en hij ging er}{dadelijk op in}\\

\haiku{in deze tijd zou.}{het tegendeel immers zo}{normaal zijn geweest}\\

\haiku{Ik kan er nu om}{lachen als ik bedenk met}{welke gevoelens}\\

\haiku{op het kwaadspreken.}{dat al begon zodra zij}{het hek uit waren}\\

\haiku{Wat er aan strijd was;}{geweest tussen de mama}{en haar ontging mij}\\

\haiku{{\textquoteleft}Ach, je weet er niets,,{\textquoteright}.}{van zei ze dat komt omdat}{het nog zo klein is}\\

\haiku{maar soms denk ik weer.}{dat ook deze formule}{te eenzijdig is}\\

\haiku{hoe kan men z\'oveel?}{verdriet hebben om iets dat}{z\'o vaag worden kan}\\

\haiku{hij wijdde er lang}{over uit en terwijl ik naar}{hem luisterde was}\\

\haiku{{\textquoteleft}Kom dichter bij me,.}{en zeg gerust als je het}{idee niet prettig vindt}\\

\haiku{tegen het licht van.}{de gang zag ik dat zij in}{een avondmantel was}\\

\haiku{maar, zei ze, ik moest.}{daarom niet denken dat zij}{niets had meegemaakt}\\

\haiku{Ik was niet langer,:}{dan vijf minuten weg en}{ik dacht onderwijl}\\

\haiku{In de Deux Magots,,;}{waar gevochten is spreekt men}{er nauwelijks over}\\

\haiku{Er zijn meer dan 5000,.}{manifestanten ditmaal}{en zij groeien aan}\\

\haiku{All\'e\'en een goede...{\textquoteright}.}{economische ordening}{Bij de H\'everl\'e's}\\

\haiku{Ik kwel mij met de -}{gedachte dat mensen als}{wij ik bedoel nu}\\

\haiku{Om dit leven uit,.}{te schakelen zou men niet}{meer moeten denken}\\

\haiku{je zou Deterding,:}{willen opblazen je kan}{het niet en je denkt}\\

\haiku{iets heel merkwaardigs,.}{werkelijk in het soort van}{je notarissen}\\

\haiku{{\textquoteright} {\textquoteleft}In de eerste plaats,.}{als niet-vee daarna als}{alles wat je wilt}\\

\haiku{we zoeken ze thuis,,;}{op volgens het adresboek en}{we vermoorden ze}\\

\haiku{binnen een halfuur.}{had hij drie verschillende}{reacties gewekt}\\

\haiku{en het was vroeger,...{\textquoteright}}{z\'o'n knappe jongen dat kan}{men ook nog wel zien}\\

\haiku{Ik sta tegenover.}{zulke woorden met een leeg}{hart en een leeg hoofd}\\

\subsection{Uit: Manuscript in een jaszak gevonden. Kroniek van de bekering van Bodor Gu{\'\i}la Buitenlander}

\haiku{Je suis all\'e \`a;}{Montmartre pour y chercher}{un ext\'erieur}\\

\haiku{Car malgr\'e tous mes.}{efforts je me suis compris}{en l'\'ecrivant}\\

\haiku{Want hoe ik ook mijn,.}{best deed ik begreep mezelf}{toen ik het schreef}\\

\haiku{Willekeurig in.}{stukken gehakt en onder}{elkaar geschreven}\\

\haiku{Alors tous deux furent.}{apais\'es et l'un ne faisait}{plus rien que baver}\\

\haiku{Mariette,~~~~~n'est,.}{pas ici elle d\'ejeune}{l'avocat r\^eveur}\\

\haiku{je vous remercie}{pour vos le\c{c}ons cher ma{\^\i}tre}{mais vous ne m'avez}\\

\haiku{reconnaissant sans}{cendrars sans moi tandis que}{j'\'ecris son portrait}\\

\haiku{de hemel is de}{rand van een horloge dat}{op kwart over elf staat}\\

\haiku{est-ce tout?}{simplement la langue de}{Poulo-Djawa}\\

\haiku{Ik heb een jeboek.}{gevenschre dat le zenprij}{noch le kenla is}\\

\haiku{de te logische!}{opstandeling tegen mijn}{jeugdige overmoed}\\

\subsection{Uit: Tegenonderzoek}

\haiku{Bloem's requisitoir,,.}{kent geen genade zegt Ter}{Braak en het is juist}\\

\haiku{Wilt u overigens?}{Clair-Obscur en Saturnus}{eens met mij doorzien}\\

\haiku{De Bezoeker heeft,,;}{Marsman zelf geloof ik als}{een aanwinst erkend}\\

\haiku{Ik verbeeld mij ook}{geenszins dat mijn streven om}{precies te zeggen}\\

\haiku{Z\'o afgepast en:}{verantwoord hoeft ieder woord}{hier toch niet te zijn}\\

\haiku{Geantwoord dat ik:}{deze psychologiese}{trouvaille zie als}\\

\haiku{Van Ostaijen wordt -:}{er bij gehaald en Ter Braak}{zei het immers reeds}\\

\haiku{Dit dekking zoeken;}{achter Van Ostaijen vind}{ik ook al zo lam}\\

\haiku{Men zou daar telkens,,.}{op terug komen en toch}{daar gaat het niet om}\\

\haiku{Sedert enige tijd,:}{kondigt hij aan onder de}{lijst van zijn werken}\\

\haiku{De soberheid van.}{toon is in het eerste deel}{bijna misleidend}\\
