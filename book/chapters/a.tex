\chapter[11 auteurs, 912 haiku's]{elf auteurs, negenhonderdtwaalf haiku's}

\section{Bertus Aafjes}

\subsection{Uit: Een laars vol rozen}

\haiku{Er kwam een kleine:}{tinteling in zijn groene}{ogen toen hij zeide}\\

\haiku{{\textquoteleft}Zou je (hij kuchte) ()?}{misschienhij kuchte weer van}{mij kunnen houden}\\

\haiku{Ik voelde hoe ik.}{bleek werd toen hij de naam van}{mijn engel uitsprak}\\

\haiku{De hete dag EEN.}{keer heeft een muilezeltje mij}{het leven gered}\\

\haiku{Vlak voor mijn ogen hing.}{een stenen kruik en daaruit}{sijpelde water}\\

\haiku{Nu sloeg hij weer een.}{zijweg in en keek even of}{ik hem gevolgd was}\\

\haiku{Honderden biezen.}{en twijgen lagen verspreid}{op de lemen vloer}\\

\haiku{{\textquoteleft}Het is niet prettig.}{een Franse stoelenmatster}{te zijn in Itali\"e}\\

\haiku{denk ik, terwijl ik.}{langs de hals van mijn engel}{naar beneden tuur}\\

\haiku{De persen zijn al{\textquoteright},.}{gesmolten zegt hij en hij}{deint met zijn vleugels}\\

\haiku{Wellicht werd hij door.}{zijn onrust naar andere}{zee\"en gedreven}\\

\haiku{Ik verhaast mijn pas.}{en win het van de langzaam}{sjokkende muilezel}\\

\haiku{Ik zet mijn ransel.}{voorzichtig op een open plaats}{tussen de tonnen}\\

\haiku{En dan gaat hij met.}{een kan naar de wijnton en}{draait de kraan open}\\

\haiku{Doch wie ook zou dit?}{bankroet van de bedelstand}{kunnen forceren}\\

\haiku{IJlings verdween zij.}{met haar melkemmer en liet}{de koe achter}\\

\haiku{H\`em vertrouwde hij.}{toe dat hij immer in dit}{park achtervolgd werd}\\

\haiku{Ik wil den lezer:}{een dezer vertalingen}{niet onthouden}\\

\haiku{Geheel de klas volgt.}{hem daverende op \'e\'en}{lettergreep lengte}\\

\haiku{Hij slaat het eerst het:}{kruis en in zijn donkere}{ogen staat te lezen}\\

\haiku{Buiten beginnen.}{ineens alle klokken van}{het dorp te luiden}\\

\subsection{Uit: De wereld is een wonder}

\haiku{Ik wil niet zeggen.}{dat een Nederlander niet}{chauvinistisch is}\\

\haiku{Oostermeer strekt zich.}{een der prachtigste tuinen}{van Nederland uit}\\

\haiku{Er ligt een jonge.}{stier van nog maar vier dagen}{tussen de planken}\\

\haiku{Groter bouwers dan.}{de Egyptenaren heeft de}{wereld nooit gekend}\\

\haiku{Volgens moderne.}{begrippen had het slechts een}{handjevol mensen}\\

\haiku{Hij had nog de man,.}{gekend die door het oog van}{God gekeken had}\\

\haiku{Het verhaal van de,.}{man die door het oog van God}{in de Sint Jan keek}\\

\haiku{Als het donker wordt.}{gaan er rode scheepslantaarns}{aan in de herberg}\\

\haiku{Zij heeft hun gezegd,.}{dat de duivel op aarde}{rond gaat en God ook}\\

\haiku{Als ratelende.}{wagens op de toppen der}{bergen springen zij}\\

\haiku{De zon en de maan.}{worden zwart en de sterren}{trekken haar glans in}\\

\haiku{Ik moet een glazen.}{pot hebben en een handvol}{groene bladeren}\\

\haiku{De volgende dag.}{gingen zij met ons mee in}{de trein naar Hoensbroek}\\

\haiku{Het was vier uur in.}{de nacht toen zijn voelhorens}{voorgoed verstarden}\\

\haiku{In de pastorie.}{van zijn ouders werkt hij als}{een bezetene}\\

\haiku{De koets springt over een,,.}{steen de wereld buitelt de}{wereld trekt weer recht}\\

\haiku{Hij maakte de fout,.}{die later zo velen na}{hem zouden maken}\\

\haiku{Ergens op de kim.}{stond een gele hooiopper}{als een negerhut}\\

\haiku{Zij verwelkomden.}{mij hartelijk alsof ik}{nooit was weggeweest}\\

\haiku{Een soort heiligheid,.}{waarvoor overigens veel te}{zeggen valt vind ik}\\

\haiku{Er staat er immers.}{een te bloeien in de tuin}{waarin ik dit schrijf}\\

\haiku{De Zenmeester stelt,.}{de vraag de Zenleerling moet}{het antwoord geven}\\

\haiku{Dat is typisch Zen,,.}{zichzelf vergeten zijn hoe}{ziek men ook is}\\

\haiku{Ik zag de tuin en.}{al mijn gevoelens waren}{op slag ontwapend}\\

\haiku{Er werd thee gebracht,.}{door een andere monnik}{dikke groene thee}\\

\haiku{Ik heb niet kunnen.}{uitmaken of de man in}{ernst sprak of schertste}\\

\haiku{Die vonden zij in.}{de persoon van kardinaal}{Agagianian}\\

\haiku{De Garde marcheert,,.}{aan hellemboswuivend met}{wapengekletter}\\

\haiku{En zij die te ziek.}{of te zwak zijn brengen hun}{stem uit op hun bed}\\

\haiku{En ineens is de.}{spreeuwenzwerm opgelost in}{de hemel als rook}\\

\section{C.S. Adama van Scheltema}

\subsection{Uit: Itali\"e}

\haiku{het kan, dan kunnen...}{wij het toch z\'eker en met}{nieuwen moed droogden}\\

\haiku{wij stonden voor de -,!}{Giovanni Bellini's zoo}{re\"eel zoo modern}\\

\haiku{En dat was de broer,}{van Gentile Bellini}{van wien wij later}\\

\haiku{{\textquoteright} - en dikwijls daarna,.}{heb ik dat woord zoo gehoord}{van verscheiden mond}\\

\haiku{En daar, na langen,....}{tijd haalt hij het weer op en}{d'er zit weer niets in}\\

\haiku{das sch\"one Florenz{\textquoteright}...}{keek hij ons verwonderd en}{beleefd spottend aan}\\

\haiku{Hij schreef links en rechts, -:}{op advertenties en na}{een maand had hij beet}\\

\haiku{de kerk, de markt, de,,, -.}{huisjes de menschjes de}{poorten de wallen}\\

\haiku{plak ze er niet op,!}{want het komt immers toch niet}{terecht uit Itali\"e}\\

\haiku{Inderdaad, hoeveel!}{hebben de Italianen}{op de Franschen voor}\\

\haiku{Het zijn alleen maar,.}{kinderen kinderen met}{een boefjesaanleg}\\

\haiku{{\textquoteright} - dan wisten wij, dat...}{het minstens een half uur lang}{zoo zou  duren}\\

\haiku{- te zitten, de jeugd.}{als bruin-rose diertjes links}{en rechts wegvluchten}\\

\haiku{mooier dan die berg,?}{alleen met de hooge muren}{van grauwe tufsteen}\\

\haiku{kijk nou daar en dat -, -!}{z\'o\'o is het geweest je ziet}{het heel duidelijk}\\

\haiku{Zie bij ons eens naar:}{een heerenhuis of grooter}{complex in afbraak}\\

\haiku{Goethe heeft er in.}{ieder geval beter den}{tijd voor genomen}\\

\haiku{Die Gothiek trouwens -,.}{curieus dat Itali\"e haar}{nooit begrepen heeft}\\

\haiku{De andere, de,....}{S. Maria in Cosmedin}{was weer heel anders}\\

\haiku{- doch dat kwam misschien.}{enkel maar omdat wij de}{beschaafdsten waren}\\

\haiku{Wat een pracht zou dat -!}{kunnen zijn en hoe droevig}{armoedig was het}\\

\haiku{Waarom moeten wij?}{altijd grijpen naar Fransche}{en Duitsche boekjes}\\

\haiku{hoe zou het oordeel,?}{over hem zijn wanneer eens niets}{dan dit was bewaard}\\

\haiku{- als het nu nog de!}{muze met den inktkoker}{was bij den dichter}\\

\haiku{Wij hadden ook een {\textquoteleft}{\textquoteright}:}{profane liefde onder}{die verzameling}\\

\haiku{Angelico zou.}{daar zeker een zoeten lach}{bij hebben geschreid}\\

\haiku{- Maar ook in het  :}{bouwen schijnt nog zooveel aan}{het oude gelijk}\\

\haiku{dat is tegen een, -!}{stijven hals begrijpt ge en}{men ziet er best in}\\

\haiku{de ontdekking van - -;}{den mensch van een  ziel van}{een menschen-ziel}\\

\haiku{En juist daarom, zou,;}{ik meenen staat hij tevens}{een toekomst nader}\\

\haiku{Met dat al voelen:}{wij nog iets ontsnappen aan}{de vergelijking}\\

\haiku{27Vosmaer noemt in {\textquoteleft}{\textquoteright}:}{zijnInwijding Rembrandt hier}{maar stoutmoedig weg}\\

\haiku{(Het Rome van den.)}{keizertijd wordt meest op 1.500.000}{inwoners geschat}\\

\haiku{S.) hat er einem,,{\textquoteright}.}{neuen Stil dem Barock den}{Boden bereitet}\\

\section{Catharina Alberdingk Thijm}

\subsection{Uit: Den harem ontvlucht. Een Turksch verhaal uit onze dagen (juni 1904-1906)}

\haiku{maar hij keerde tot,:}{de sofa terug strekte}{zich uit en hernam}\\

\haiku{Niet alleen schonk hij;}{een ruime lijfrente aan}{den ouden priester}\\

\haiku{ik begrijp meer en.}{meer wat mijn grootvader hier}{heeft moeten lijden}\\

\haiku{Ik ben bitterder,,!}{gestemd dan ooit Ella en}{ik heb er helaas}\\

\haiku{Gij hadt het gelaat.}{van den prins moeten zien bij}{onze verschijning}\\

\haiku{{\textquoteleft}Als ik dat voor mijn,.}{zieke geweest was zou hij}{gered zijn geweest}\\

\haiku{Maar vergeet niet dat:}{gij steeds omgeven zijt van}{ondergeschikten}\\

\haiku{dan zal ik altijd.}{zorgen dat je aangenaam}{gezelschap ontmoet}\\

\haiku{maar ik ben niet voor;}{niets aan den schoot eener groote}{liefde opgegroeid}\\

\haiku{ik er zorg voor, haar.}{eenige lieve kennissen}{te doen ontmoeten}\\

\haiku{Het korte leven.}{schijnt nog veel te lang te zijn}{voor trouwe vriendschap}\\

\haiku{maar als de  hoop,.}{op verlossing komt neen dat}{kan ze niet dragen}\\

\haiku{Zij hief zich met een.}{zenuwachtig lachje van}{haar rustbank overeind}\\

\haiku{Ik ben overtuigd, dat.}{hij haar omgekocht heeft om}{mij te bespieden}\\

\haiku{Wat zij echter ook,;}{mag hebben geleden zij}{droeg haar last zwijgend}\\

\haiku{Ik dank u voor al,;}{wat gij voor ons beiden deedt}{voor al wat gij zijt}\\

\haiku{, want ik ben zeker,.}{dat het uit jouw vruchtbaar brein}{is opgerezen}\\

\haiku{Onmogelijk dus.}{voor Nourya zich te laten}{verontschuldigen}\\

\haiku{Het eenige wat mij.}{kwelt is bezorgdheid omtrent}{Zeanor's toestand}\\

\haiku{{\textquoteleft}En dan zal ik geen.}{stap meer kunnen doen zonder}{bewaakt te worden}\\

\haiku{die in duigen zou,.}{zijn gevallen als je er}{over gesproken hadt}\\

\haiku{Marcelle zal je;}{beiden in alles steunen}{zooveel zij slechts kan}\\

\haiku{Zij meende een droom.}{te doorleven en liet zich}{gewillig kleeden}\\

\haiku{Voor ons kind behoef,,{\textquoteright}, {\textquoteleft}}{je niet te vreezen Djalah}{gaf hij ten antwoord}\\

\haiku{dat Allah even goed,.}{hier als elders herstel van}{krachten kan schenken}\\

\haiku{{\textquoteleft}Ahmed,{\textquoteright} smeekte zij, {\textquoteleft},}{wees niet hard voor ze als gij}{ze weer mocht vinden}\\

\haiku{men hoorde ze al,;}{van verre kijven wanneer}{iets haar mishaagde}\\

\haiku{Men verzekerde,.}{mij dat alle paspoorten}{in orde waren}\\

\haiku{Zoo bereikten zij:}{dien avond tegen zeven uur}{Bulgarije's hoofdstad}\\

\haiku{Ik ken de Turken.}{hier niet en wensch geen kennis}{met ze te maken}\\

\haiku{{\textquoteright} {\textquoteleft}Maar indien ik u,?}{mijn eerewoord gaf dat dit}{niet geschieden zal}\\

\haiku{{\textquoteleft}Waart gij niet zoo jong,.}{gij zoudt den toestand minder}{rooskleurig inzien}\\

\haiku{maar wat hij zonder,.}{plichtverzaking doen kan zal}{hij niet nalaten}\\

\haiku{Hij speelde zijn rol,{\textquoteright}, {\textquoteleft}}{waarlijk niet kwaad antwoordde}{Marcelle lachend}\\

\haiku{En, lieveling, wees;}{niet bang dat je last van mij}{zult hebben op reis}\\

\haiku{Trouwens het is geen.}{verre reis die ik van je}{krachten vergen zal}\\

\haiku{{\textquoteright} Zeanor drukte.}{de handen harer zuster}{tegen haar gelaat}\\

\haiku{{\textquoteright} fluisterde zij, {\textquoteleft}goed.}{zelfs mij wederom op het}{ziekbed te werpen}\\

\haiku{Het is de eenige,.}{weg of ik zou u dien niet}{voorgesteld hebben}\\

\haiku{{\textquoteright} Heel die eerste dag,,.}{een Zaterdag verliep zoo}{vreedzaam mogelijk}\\

\haiku{{\textquoteright} antwoordde zij, {\textquoteleft}al}{hebben zij mij nog dieper}{getroffen dan u.}\\

\haiku{Nourya, ik ben z\'o\'o.}{gelukkig deze natuur}{te hebben gezien}\\

\haiku{{\textquoteright} fluisterde zij, als.}{vreesde zij haar gedachten}{luid uit te spreken}\\

\haiku{Hoe zou ik echter?}{den moed hebben gevonden}{haar dat te zeggen}\\

\haiku{dat is het eenige.}{wat onze vlucht in mijn oogen}{kan rechtvaardigen}\\

\subsection{Uit: Koningsliefde. Het drama in Servi\"e}

\haiku{{\textquoteleft}maar zoo hij opgroeit;}{zal hij een man worden van}{zeldzame geestkracht}\\

\haiku{{\textquoteright} {\textquoteleft}Papa is zeker,,?}{vreeselijk wijs dat u hem}{alles vraagt niet waar}\\

\haiku{{\textquoteright} {\textquoteleft}Daarvan vermoedde,.}{ik niets ik zweer het u bij}{het hoofd van ons kind}\\

\haiku{De politiek van,,;}{zijn land was in meerderheid}{Oostenrijksch gezind}\\

\haiku{{\textquoteright} Overste Deludoff.}{uitte een gesmoorden vloek}{tegen den koning}\\

\haiku{Ik beloof u een,.}{troonopvolger op wien gij}{allen trotsch zult zijn}\\

\haiku{{\textquoteright} Uitgeput zonk het.}{prinsje weer op zijn kussen}{neer en sloot de oogen}\\

\haiku{Van lieverlede;}{echter geraakte hij aan}{dit alles gewoon}\\

\haiku{De orgi\"en van.}{zijn vader volgde hij met}{sombere blikken}\\

\haiku{Als ik eens koning,?}{ben komt u weer terug bij}{uw Sascha niet waar}\\

\haiku{Het voorstel werd ten.}{slotte met algemeene}{stemmen verworpen}\\

\haiku{Wat zou hij anders?}{doen als zich geheel aan de}{studie overgeven}\\

\haiku{Maar thans ben ik tot,.}{u gekomen om u niet}{weer te verlaten}\\

\haiku{Zoo menig vorst blijft,.}{ongehuwd volgt alleen de}{neiging zijns harten}\\

\haiku{{\textquoteright} {\textquoteleft}Onze afspraak is,{\textquoteright}.}{vervallen antwoordde de}{jonge vrouw kortaf}\\

\haiku{Nathalie sedert...?}{jaren geen schooner dag hebben}{gekend en Alexander}\\

\haiku{Dat is een woord, dat,,{\textquoteright}.}{ik niet licht zal vergeten}{Sascha zeide hij}\\

\haiku{Nicolesco.}{wierp zich op als den tolk van}{aller gevoelens}\\

\haiku{{\textquoteleft}Draga,{\textquoteright} zeide zij,.}{het op de borst gezonken}{hoofd opheffende}\\

\haiku{De vrouw tegenover.}{hem vergat bij dien aanblik}{dat hij koning was}\\

\haiku{Welnu, ik verklaar.}{u dat ik zulk een misdaad}{niet toelaten zal}\\

\haiku{Sascha was nog te,.}{jong om in zulk een tweestrijd}{te worden gebracht}\\

\haiku{{\textquoteleft}Ik heb het mijn plicht.}{geacht den Czaar de waarheid}{mede te deelen}\\

\haiku{{\textquoteright} Tot eenig antwoord wierp.}{hij zich aan haar voeten en}{beleed haar alles}\\

\haiku{Meer dan ooit was hij.}{besloten de roepstem te}{volgen van zijn hart}\\

\haiku{Gij zijt waanzinnig.}{ook maar een oogenblik aan}{zoo iets te denken}\\

\haiku{{\textquoteright} hernam de jonge,.}{vrouw hem een uitdagenden}{blik toewerpende}\\

\haiku{Hoe weinig kost het!}{u niet en hoeveel zal het}{voor mij beteekenen}\\

\haiku{{\textquoteleft}Ik zou mij schamen.}{het leger van zulk een goed}{soldaat te berooven}\\

\haiku{Van dat oogenblik,.}{af werd de jonge vrouw zoo}{goed als doodverklaard}\\

\subsection{Uit: Kroonprinses}

\haiku{Catharina}{Alberdingk Thijm Kroonprinses}{Colofon}\\

\haiku{{\textquoteleft}Maar hoe is het dan,?}{mogelijk dat ik het stuk}{niet gelezen heb}\\

\haiku{{\textquoteleft}Neen, Monseigneur, maar.}{wel Uwe Hoogheid noodeloos leed}{willen besparen}\\

\haiku{Zelfs het oog van een.}{tuinman kon hij thans niet op}{zich voelen rusten}\\

\haiku{Indien gij kalmer,}{waart zoudt gijzelf begrijpen}{hoe noodzakelijk}\\

\haiku{maar ik verzoek u.}{allereerst rechtvaardig te}{zijn tegenover mij}\\

\haiku{{\textquoteright} {\textquoteleft}Heeft u of moeder,?}{dan nooit begrepen waarom}{zij Vasthi haatte}\\

\haiku{{\textquoteright} {\textquoteleft}Door mij van haar te,{\textquoteright}:}{laten scheiden antwoordde}{de kroonprins bevend}\\

\haiku{Maar eerbiedig dan,;}{ook voortaan de vrouw die ik}{niet vergeten kan}\\

\haiku{maar als ik ze in,.}{mijn armen neem denk ik aan}{mama met Bertie}\\

\haiku{{\textquoteright} ~ Thesa had druk.}{bij het opstellen van dit}{epistel geholpen}\\

\haiku{zeker, en als ik,.}{het groote raam opendeed zou men}{ons misschien hooren}\\

\haiku{Hoe jammer voor u,!}{en mij dat anderen zich}{tusschen ons plaatsten}\\

\haiku{Omdat hij trouw bleef,?}{ondanks alles zou hij}{geschandvlekt wezen}\\

\haiku{Buitendien beweeg.}{ik mij vooral onder de}{gevallen vrouwen}\\

\haiku{Ik weet het wel u;}{behoort tot eene andere}{kerk dan de mijne}\\

\haiku{{\textquoteleft}Uwe houding bewijst,.}{mij hoe ver gij nog van}{Huis verwijderd zijt}\\

\haiku{{\textquoteright} Nog mistroostiger.}{dan te voren gevoelde}{zij zich thans gestemd}\\

\haiku{{\textquoteright} Elk hunner woorden '.}{drong den jongen vorst als een}{dolksteek int hart}\\

\haiku{{\textquoteright} De handen van den.}{kroonprins wrongen zich om de}{armen van zijn stoel}\\

\haiku{Ik dank u uit den.}{grond van mijn hart voor uwe daad}{van barmhartigheid}\\

\haiku{u liefhebben met}{geheel mijn  ziel en denk}{voortdurend aan u.}\\

\haiku{Bij oogenblikken,.}{is het mij alsof ik u}{aan het hart drukte}\\

\haiku{zelfs verlangde hij;}{niet dat zij eene prinses van}{den bloede zou zijn}\\

\haiku{{\textquoteleft}Het wordt dat, waar het,}{ophoudt menschelijk te zijn}{zoo althans oordeel}\\

\haiku{ik er over, en men.}{mag niet tegen zijn eigen}{geweten ingaan}\\

\haiku{{\textquoteleft}maar met u valt niet,.}{te spreken zoodra die}{vrouw in het spel komt}\\

\haiku{{\textquoteright} Den volgenden avond.}{ontving Vasthi een brief die}{haar deed verbleeken}\\

\haiku{{\textquoteright} Hij had gevoeld hoe.}{edel dat antwoord was en er}{haar om leeren achten}\\

\haiku{{\textquoteleft}Komt eens om mij heen,,{\textquoteright}:}{staan kinderen sprak zij met}{vriendelijken ernst}\\

\haiku{maar zij legde er,.}{een pathos in die hem tot}{in de ziel roerde}\\

\haiku{Ik heb zelfs nooit iets,.}{in haar opgemerkt dat mij}{onaangenaam was}\\

\haiku{Hij bleef dezelfde,,;}{voor haar altijd vol goedheid}{altijd welwillend}\\

\haiku{Door u verloor ik,;}{alles wat waarde heeft in}{de oogen eener vrouw}\\

\haiku{Die kinderen, die,;}{gij niet vergeten kunt gij}{zult ze niet weerzien}\\

\haiku{{\textquoteright} De jonge man wierp,.}{haar een somberen blik toe}{vol bedreigingen}\\

\haiku{gij zult naar liefde:}{hunkeren en betreuren}{wat gij hebt versmaad}\\

\haiku{maar zelfs de rouw kon,.}{de vete niet uitwisschen}{die men hem toedroeg}\\

\haiku{Zij voltooide haar,.}{volzin niet en waagde het}{niet hem aan te zien}\\

\haiku{maar de jonge oogen.}{die hem aanblikten zagen}{het ternauwernood}\\

\haiku{Ik had het recht niet.}{haar hier tegen wil en dank}{terug te voeren}\\

\haiku{Weenend van smart en,.}{woede kwamen zij tegen}{zulk een onrecht op}\\

\haiku{{\textquoteleft}als zij maar niet te,.}{vroeg verstaan haar niet te hard}{te veroordeelen}\\

\haiku{Later was het eene;}{verlichting toen Luta in}{het huwelijk trad}\\

\haiku{hij is heel goed, heel,,,;}{edel heel hoog Monica hij}{leeft in Duitschland}\\

\haiku{hadden zeker haar....}{vader gedwongen van de}{moeder te scheiden}\\

\haiku{{\textquoteright} {\textquoteleft}Gerust, ik heb een.}{schel bij me en wil niet dat}{ge voor me opblijft}\\

\haiku{het is vreemd dat hij,.}{eerst nu na zooveel jaren}{tegen mij optreedt}\\

\haiku{{\textquoteleft}Ik was zoo in mijn,}{lectuur verdiept dat ik doof}{schijn te zijn geweest}\\

\haiku{{\textquoteleft}maar nu gij het zegt,.}{ja daar buiten heerscht een}{zonderling rumoer}\\

\haiku{Twee of drie hunner.}{kunnen immers het woord voor}{de anderen doen}\\

\section{J.A. Alberdingk Thijm}

\subsection{Uit: Karolingsche verhalen}

\haiku{Maar de Engel, die,:}{van God gezonden was sprak}{nu tot den Koning}\\

\haiku{{\textquoteleft}Zult gij Gods gebod,,.}{in den wind slaan Koning zoo}{zijt gij verloren}\\

\haiku{Zoo zorgt hij voor zijn,.}{onderhoud waar hij rijke}{lieden kan vinden}\\

\haiku{Ik bid Gode te,!}{waken dat deze mij geen}{kwaad of oneer doe}\\

\haiku{{\textquoteleft}Dat is iemant, die.}{in dit bosch verdwaald is en}{van den weg geraakt}\\

\haiku{{\textquoteright} Toen zij elkander,.}{voorbijkwamen reden zij}{d\'oor zonder groeten}\\

\haiku{Liever zullen we -.}{vechten dan dat ik mij tot}{antwoord dwingen liet}\\

\haiku{{\textquoteleft}Spreekt eerst tot mij - dan,;}{zal ik u zeggen wat ik}{hier zoek en jage}\\

\haiku{zegt me nu, zoo 't,.}{u gelieft hoe gij in uw}{onderhoud voorziet}\\

\haiku{Is hij van zulke,?}{machte dat gij de nacht tot}{rijden moet kiezen}\\

\haiku{Elegast, die dit,.}{alles had ga\^ageslagen}{kroop er zachtkens heen}\\

\haiku{{\textquoteright} ging hij voort, {\textquoteleft}ziet hier,.}{het zadel waar ik u zoo}{even van verhaald heb}\\

\haiku{Ik was daar en had,.}{het gadegeslagen en}{kroop er zachtkens heen}\\

\haiku{Men sleepte Eggheric -:}{voort en hing hem en alle}{verraders tevens}\\

\haiku{zij duchtte, dat hij ', ', '.}{t zoude dooden waret}{dat hijt vernam}\\

\haiku{maar toen zij vijftien,.}{jaar oud waren ontwies Reinout}{Lodewijk een voet}\\

\haiku{Valt het wat zwaar en, '.}{verdrietigt is nochtans}{met eere gedaan}\\

\haiku{en, om uw eigen,:}{eer wilt mijnen magen en}{den uwen andwoord geven}\\

\haiku{liever offerde,.}{ik alles op dan dat mijn}{goed hun blijven zo\^u}\\

\haiku{Als dat Haymijn zag,,,}{vervaerde hij hem daar}{hij ter aarde lag}\\

\haiku{Haymijn gordde hem ',:}{t zwaerd en sloeg hem in}{den hals zeggende}\\

\haiku{daar sprong Reinout op, en, '.}{sprong het de lendenen aan}{stukken datet stierf}\\

\haiku{{\textquoteleft}Ik rade u, dat,,.}{gij u wapent want het Ros}{is groot fel en sterk}\\

\haiku{{\textquoteright} Met dat Haymijn die,.}{woorden tot Reinout sprak ontsloot}{men de staldeure}\\

\haiku{Vrouw Aye, dat ziende,,:}{liep haastig toe en wrong haar}{handen zeggende}\\

\haiku{Ik zal beproeven,.}{in korten tijd of Reinout mijn}{neve is of niet}\\

\haiku{{\textquoteright} Toen kwam er klachte,;}{voor den Koning dat zijn Kok}{doodgeslagen was}\\

\haiku{{\textquoteright} Na de maaltijd ging,:}{men dansen en spelen en}{daar was groote vreugde}\\

\haiku{Men schonk er den wijn.}{overvloedig in gouden en}{zilveren vaten}\\

\haiku{{\textquoteright} Haymijn ging nu tot.}{zijne Kinderen en bracht}{ze voor den Koning}\\

\haiku{... Laat ze hier komen,!}{uw Kinderen en proeven}{hun macht met den steen}\\

\haiku{{\textquoteright} sprak hij, {\textquoteleft}gij zijt niet,:}{zoo koen dat gij uw hand zoudt}{slaan aan mijnen baard}\\

\haiku{Haymijn zag naar Reinout -.}{dat hij zijn krachten aan den}{steen zoude toonen}\\

\haiku{En Lodewijk stond:}{daar met grooten nijd in het}{harte en zeide}\\

\haiku{alleen bid ik u,.}{dat gij niet meer speelt om zoo}{kostelijken pand}\\

\haiku{Ik zal blijven op ',.}{et veld en verwachten wat}{mij overkomen mag}\\

\haiku{{\textquoteleft}Dede ik dat, Heer,{\textquoteright}, {\textquoteleft}.}{Koning zeide Reinoutzoo}{ware ik een dwaas}\\

\haiku{{\textquoteright} - {\textquoteleft}Dat waar kwaad pand voor,{\textquoteright}, {\textquoteleft}!}{onzen schat zeide Adelaert}{ik nam wat beters}\\

\haiku{al had ik goud in,.}{mijne hand het werd koper}{eer het daaruit kwam}\\

\haiku{{\textquoteright} Reinout sloeg hem 'et hoofd ',:}{af en gafet zijn broeder}{Adelaert en zeide}\\

\haiku{In het kasteel van.}{Vaucloen aan de Dordone}{woonde Koning Ywein}\\

\haiku{wat raad geeft gij mij,?}{in deze dat ik mijne}{eere behoude}\\

\haiku{Naar mijn oordeel, zult,,.}{gij ze behoudens lijf en}{goed uitleveren}\\

\haiku{Nu riep Reinout door het:}{landschap velen op om tot}{de rots te komen}\\

\haiku{nu hebben wij groote,,!}{honger en dorst dus bidden}{wij om Gods wille}\\

\haiku{dat onze Heeren,,,....}{gevangen zouden zijn Ritsaert}{Writsaert Adelaert en Reinout}\\

\haiku{Koning Carel was,, {\textquoteleft}}{onverbidbaar maar zeide}{dat hij ze houden}\\

\haiku{{\textquoteright} zeide de bode, {\textquoteleft}:}{gij zijt Koning en moogt uw}{woord niet herroepen}\\

\haiku{en zoo zij 't uit ' -.}{et oog verliezen doet ze}{stokslagen geven}\\

\haiku{{\textquoteleft}Ik beveel u dit,.}{Ros op zulke straffe als}{Roelant gezeid heeft}\\

\haiku{ik vond zoo schoonen,,!}{man als gij zijt bevangen}{met zoo groote rouwe}\\

\haiku{{\textquoteleft}God loon u,{\textquoteright} en stak,;}{ze in zijn reiszak en scheen}{blijde te wezen}\\

\haiku{Madelgijs en Reinout,;}{zagen eene schure openstaan}{daar veel stroois in was}\\

\haiku{{\textquoteleft}O lieve gezel,{\textquoteright},,}{zeide hij tot Reinout dat de}{lieden het hoorden}\\

\haiku{{\textquoteright} Meteen is daar een,.}{man bij hen gekomen die}{uit de kerke kwam}\\

\haiku{Toen gaf Madelgijs -,:}{Reinout weder zijne sporen}{van goud en zeide}\\

\haiku{{\textquoteright} - {\textquoteleft}Helaas,{\textquoteright} zeide Reinout, {\textquoteleft},,:}{gij doet kwalijk oom dat gij}{den spot met mij drijft}\\

\haiku{{\textquoteleft}'t Is een Ridder,, '.}{genaamd Reinout en mag hier in}{et land niet komen}\\

\haiku{de twaalf knechten, wien, '.}{hij bevolen was hadden}{et elk aan een koord}\\

\haiku{En als zij op de,:}{baan waren zeide Koning}{Carel tot Roelant}\\

\haiku{Pelgrim rijden op, '!}{Beyaert datet aan zijn herstel}{bevorderlijk zij}\\

\haiku{De knechten, dien 'et,.}{Ros bevolen was hielden}{kwalijk de koorden}\\

\haiku{daarna de Hertog;}{van Beieren en Samsoen}{van Borgondi\"en}\\

\haiku{{\textquoteright} zeide Roelant, {\textquoteleft}wij.}{en dachten niet dat wij \'u}{hier vinden zouden}\\

\haiku{{\textquoteleft}Wat zullen wij van,?}{dezen Schildknaap zeggen dien}{Reinout verslagen heeft}\\

\haiku{in zijn heldenmoed ',;}{kan niemantet bedwingen}{noch achtervolgen}\\

\haiku{en gij haalt vele,.}{rampen over u zoo gij ze}{ter dood laat brengen}\\

\haiku{gij zet u tegen -, '.}{mij wij zullen zien wie hier}{t meeste vermag}\\

\haiku{{\textquoteright} Bij deze woorden:}{werd des Konings herte nog}{heftiger geschokt}\\

\haiku{De liefste, daar ik,!}{mijn betrouwen op stelde}{heeft mij begeven}\\

\haiku{want Koning Asises van,;}{Keulen doet u bidden dat}{gij hem hulpe zendt}\\

\haiku{{\textquoteright} Reinout nu hadde een ', ';}{verspieder ins Konings}{Hof dieet hoorde}\\

\haiku{Madelgijs ging, en.}{kocht de beste spijze die}{hij op de markt vond}\\

\haiku{Als de broeders dit,,:}{zagen lachten zij er om}{en Adelaert zeide}\\

\haiku{{\textquoteright} Aldus scheiden Reinout,.}{en Madelgijs van hem en}{reden naar Parijs}\\

\haiku{of men Reinout ergends,!}{vernam dat men hem vinge}{en tot u voerde}\\

\haiku{{\textquoteright} Toen zij aldus met,:}{hem spotteden zeide Reinout}{met zoete woorden}\\

\haiku{ware 't zoo wel, '.}{zwart als wit ik zo\^u zeggen}{datet Beyaert ware}\\

\haiku{gij zult varen te,:}{Vaucoloen en wachten daar}{Reinout en zijn broeders}\\

\haiku{Als zij buiten 'et,;}{bosch kwamen zagen zij een}{teeken aan de lucht}\\

\haiku{{\textquoteright} zeide de Vrouwe, {\textquoteleft} ':}{toendoet dan voort minst wat}{ik u zeggen zal}\\

\haiku{Maar is 't dat hij, -.}{mede rijden wil zoo gaat}{gij en uw broeders}\\

\haiku{eilaas, eilaas, ik,.}{zegge u dat mijn vader}{u verraden heeft}\\

\haiku{de Vrouwe weende!}{zeer en bad dat ze God in}{zijne hoede nam}\\

\haiku{{\textquoteright} Als de Edele Reinout, '.}{Florenberge zag werd hem}{et herte lichter}\\

\haiku{en vlieden wij - want!}{wij worden gevangen of}{moeten sneuvelen}\\

\haiku{Madelgijs trad juist:}{uit een kamer en riep tot}{den Kok en Drossaart}\\

\haiku{Dus draafden zij zoo, ';}{lang dat ze kwamen inet}{dal van Vaucoloen}\\

\haiku{{\textquoteright} - {\textquoteleft}Ik moet 'et eerst den,{\textquoteright}, {\textquoteleft};}{eed doen zeide Wouterwant}{ik aanlegger ben}\\

\haiku{Ik zeg Ogier aan, dat;}{hij verradenis gepleegd}{heeft te Vaucoloen}\\

\haiku{hij bracht daar met zich,.}{1500 man die deden wonder}{met den wapenen}\\

\haiku{{\textquoteleft}Ik en 500 mijner,,.}{mannen die van groote krachte}{zijn varen mede}\\

\haiku{hij leefde, en geen,.}{toevluchtsoord in wat land hij}{zich begeven mocht}\\

\haiku{Dat hoorde Ogier, de,:}{koene man hij sprong toornig}{vooruit op Fortsier}\\

\haiku{en stak op Ritsaert, en,,;}{hij weder op hem zoo dat}{hij Fortsier doorstak}\\

\haiku{des hadde Koning,!}{Carel groote toorn en riep zijn}{krijgsleuze Mont-joye}\\

\haiku{En Madelgijs lag ',:}{gevangen ins Konings}{tente en zeide}\\

\haiku{Maar toen de broeders,,:}{wechdraafden zag ze Koning}{Carel en zeide}\\

\haiku{{\textquoteleft}Broeder, hoe durft gij:}{dusdanige dingen ons}{te voren leggen}\\

\haiku{{\textquoteright} De Heremijt nu.}{dede zijn gebed tot den}{Almogenden God}\\

\haiku{Braes;62 daar vond hij}{schepen en voer in het land}{van den Islamme.63}\\

\haiku{Reinout hadde verstaan.}{wie tegen zijn zone den}{kamp zoude vechten}\\

\haiku{REINOUT ging tot Koning,.}{Carel en stond v\'oor hem als}{een arme pelgrim}\\

\haiku{Geen betere voor,{\textquoteright}, {\textquoteleft}.}{hem zeide de meesterdan}{Sint-Pietersman}\\

\haiku{uw neve Reinout kwam,;}{dienen de metselaars en}{niemant kende hem}\\

\haiku{{\textquoteright} zoo fluistert de knecht, '.}{Terwijl hij zich krampig aan}{t monnikskleed hecht}\\

\haiku{en als ge in dit,, ....}{bosch Alleen u verv\'eelt in}{uw luchtigen dos}\\

\haiku{Die pronkt met eens krans,;}{van robijn esmeraud En}{keurdiamanten}\\

\haiku{Met grooter eere,;}{ontving men den Koning zoo}{Heeren als Vrouwen}\\

\haiku{{\textquoteleft}Ai Heere,{\textquoteright} zegt zij, {\textquoteleft}?}{door wat oorzaak zullen wij}{ons kind verliezen}\\

\haiku{Zij dankten hem, en.}{namen oorlof en ruimden}{met blijdschap het hof}\\

\haiku{Die er onder stond,, '.}{hem dachte dat hij int}{Paradijs ware}\\

\haiku{want dat de rechte,,,.}{waarheid is dat Blancefloer}{zijne vriendin leeft}\\

\haiku{Zie echter wel toe,.}{dat gij uwen gouden beker}{niet op het spel zet}\\

\haiku{Maar zal de Emir naar -.}{recht uitspraak doen zoo zult gij}{de dood ontkomen}\\

\haiku{Om mij verliet gij.}{uw ouderlijk huis en zijt}{hiertoe gekomen}\\

\haiku{Ik zal zelf mij wraak,.}{verschaffen van den smaad die}{mij is aangedaan}\\

\haiku{om Gods en dezer,.}{Heeren wilde schenk ik u}{beide het leven}\\

\haiku{2De plaats van reg. 19.}{tot 24 houdt Dr. Jonckbloet}{voor ingeschoven}\\

\haiku{'t Is jammer, dat,:}{Dr. J.C. Matthes alleen Reinout}{op Beyaert laat zitten}\\

\section{A. Alberts}

\subsection{Uit: De vergaderzaal}

\haiku{Dek dan voor zeven,.}{en zanik niet verder zei}{de secretaris}\\

\haiku{Meneer moest naar een,.}{andere vergadering}{zei de concierge}\\

\haiku{Dag Beuzekom, zei}{meneer Dalem en hij gaf}{de ander een hand.}\\

\haiku{De oude heer wordt,,:}{kwaad staat op geeft die vent een}{rijksdaalder en zegt}\\

\haiku{Jullie hadden dat.}{smoel van die vent achter zijn}{bureau moeten zien}\\

\haiku{Kunnen de heren?}{aan de overzijde ons geen}{garantie geven}\\

\haiku{Hij keek om zich heen.}{en hij zag dat er niemand}{meer stond te wachten}\\

\haiku{Hij stak haastig de.}{rijweg over en stapte in}{de voorste wagen}\\

\haiku{Als het nodig is,.}{zal ik gevonden worden}{zei meneer Dalem}\\

\haiku{Hij draaide zich weer.}{om en liep de gang nog eens}{door tot het einde}\\

\haiku{Hij ging achter het.}{bureau zitten en legde}{de map open voor zich}\\

\haiku{Hij sloeg de deur van,.}{het caf\'e zo hard dicht dat}{de ruit rinkelde}\\

\haiku{Meneer Dalem zag.}{ze reusachtig groot tegen}{de ringdijk zitten}\\

\haiku{In het westen was.}{de lucht nu ook helemaal}{donker geworden}\\

\haiku{Hij hield zijn handen.}{op zijn buik en rolde op}{zijn rug heen en weer}\\

\haiku{Hij hield zijn hand voor.}{zijn ogen en probeerde het}{pad af te kijken}\\

\haiku{Toen hij weer voor zich.}{uit keek zag hij zijn schaduw}{over de brug vallen}\\

\haiku{Een buitengewoon,.}{verstandige opmerking}{zei meneer Dalem}\\

\haiku{Hij draaide zich om.}{naar de kant van waar hij was}{binnengekomen}\\

\haiku{En als u dat soms,:}{beter verstaat zei de vriend}{van de dikke man}\\

\haiku{Hij sneed vlees vlak langs.}{het bot af en maakte er}{dobbelsteentjes van}\\

\haiku{Hij hoorde achter.}{zich stemmen die zeiden dat}{het niet anders kon}\\

\haiku{Twee waren leeg en.}{hij zag dat ze in de vier}{anderen lagen}\\

\haiku{Je zou ze ook naar,.}{boven kunnen brengen zei}{de secretaris}\\

\section{Jo van Ammers-K\"uller}

\subsection{Uit: Mijn Amerikaansche reis}

\haiku{Die vele, en laat:}{mijn Hollandsche nuchterheid}{eerlijk bekennen}\\

\haiku{* * * ~ Armoede doet;}{zich het ergste gevoelen}{in de behuizing}\\

\haiku{Het is merkwaardig.}{hoeveel minder vermoeiend}{hier het reizen is}\\

\haiku{Hier zijn de huizen}{het hoogst en staan zij het dichtst}{op elkander}\\

\section{Seerp Anema}

\subsection{Uit: De Aetiopische}

\haiku{Voorsmaak van de rust,,.}{in Masjiaach gewaarborgd}{wilde Hij geven}\\

\haiku{Die twaalf leeuwen op.}{de zes trappen zijn de twaalf}{stammen Jisra\"eels}\\

\haiku{Het geboomte is,.}{nog te jong om die stralen}{te ondervangen}\\

\haiku{Alleen het prachtig.}{lazuur was weer uw gave}{aan de volkeren}\\

\haiku{{\textquoteright} {\textquoteleft}Wesir Senmoet, de.}{Sidonische vondeling}{wijst die hulde af}\\

\haiku{- Hij noemde zich een,.}{broze staf die een sterke}{zuil moest vervangen}\\

\haiku{Haastig wekte hij:}{een kamerdienaar in het}{rechter zijvertrek}\\

\haiku{dat der achter hen.}{neerstortende golven van}{deze zeeboezem}\\

\haiku{Zoo liet hij dan zijn.}{wagens aanspannen en nam}{zijn krijgsvolk met zich}\\

\haiku{Daar over heen spreiden.}{ruige reuzenpalmen hun}{zware schaduwen}\\

\haiku{Vaak zullen we dat,{\textquoteright}.}{op dezen tocht niet kunnen}{doen sprak de koning}\\

\haiku{{\textquoteright} {\textquoteleft}Voorzeker, mijn vorst,:}{maar zegt Asaaf ook niet in}{\'e\'en zijner psalmen}\\

\haiku{Strak en steil steeg hij.}{boven hen uit als van een}{gansch ander geslacht}\\

\haiku{En gij zult mij een.}{priesterlijk koninkrijk en}{een heilig volk zijn}\\

\haiku{{\textquoteright} {\textquoteleft}Gij zult tegen uw.}{naaste niet spreken als een}{valsche getuige}\\

\haiku{Ik uwer vijanden,.}{Vijand zijn en benauwen}{die u benauwen}\\

\haiku{Zij reizen op naar,;}{den hemel zij dalen neer}{tot de afgronden}\\

\haiku{Neemt deze woorden,.}{mee op uw reis die Jahw\`e}{voorspoedig make}\\

\haiku{Sjaoels koningschap!}{had het bewezen in een}{ontroerend drama}\\

\haiku{{\textquoteleft}Jahw\`e zij met u,{\textquoteright}.}{klonk het hem tegen uit den}{mond der komenden}\\

\haiku{Maar als gij uw jeugd,.}{en flinkheid niet kent wijt het}{uw eigen blindheid}\\

\haiku{Boden op snelle:}{kemels hadden het nieuws ten}{paleize gebracht}\\

\haiku{Een der dienstvrouwen.}{van Tafaths kleine hof}{kwam den knaap halen}\\

\haiku{Opnieuw wandelde.}{Sjalomo een wijle heen}{en weer in de zaal}\\

\haiku{De berekening;}{had in hoogte en richting}{slechts weinig gefaald}\\

\haiku{- Ik zie geen sporen......}{meer van de heilige bron}{en het heilig bosch}\\

\haiku{Alleen de tijd van.}{zes maanden moest tot op de}{helft bekort worden}\\

\haiku{Groot is Jahw\`e en,.}{hoog te loven onze God}{om zijn koningsstad}\\

\haiku{{\textquoteright} {\textquoteleft}Dat eischt een tot.}{het uiterste doorgevoerd}{belastingstelsel}\\

\haiku{{\textquoteright} Hij glimlachte en.}{dreef de rossen de Dal-}{en Hoekpoort voorbij}\\

\haiku{Ze stamden af van.}{onze goden en hebben}{groote daden gedaan}\\

\haiku{{\textquoteleft}Maar hoe heette dan,.}{uw volk want geen gerucht van}{hen drong tot ons door}\\

\haiku{Het werd gevonden.}{in een geheim vertrek in}{den burcht onzer stad}\\

\haiku{Ze vroegen zeven,.}{dagen die hun glimlachend}{werden toegestaan}\\

\haiku{machtloos poogt dan 't,,:}{ontwijken de \'e\'ene den}{ander vervolgend}\\

\haiku{De witroode pluim.}{op hun bronzen helm was hun}{herkenningsteeken}\\

\haiku{- Toen kwam men aan een,.}{vijf uur breede vlakte uit}{krijtrots bestaande}\\

\haiku{En de sluiting met,:}{de plechtige bede zacht}{en gedragen}\\

\haiku{zij brengen goud en.}{wierook en verkondigen}{den lof des Heeren}\\

\haiku{Hoe lief hij dat ook,.}{bedoelde het gelukte}{slechts gedeeltelijk}\\

\haiku{De priesterkoren.}{van Amon-Re zingen slechts}{te samen \'e\'en lied}\\

\haiku{Hier hoor ik als het......}{ware meerdere zangen}{dooreen gevlochten}\\

\haiku{{\textquoteleft}Mijn zuster,{\textquoteright} sprak de,:}{koning toen zijn blikken het}{al hadden omvat}\\

\haiku{Als Hij dat Zelf doet,.}{verschijnt Hij onder ons in}{menschengestalte}\\

\haiku{Hij komt, om Zich ten,.}{offer te geven voor wie}{zijn voeten kussen}\\

\haiku{Hier is de baan tot,.}{Jahw\`e die de zwaarden der}{cherubs vrij laten}\\

\haiku{Jahw\`es, die alles,.}{overwint wat zich tegen dit}{heil wil verzetten}\\

\haiku{God verplettert den,,.}{harden kop van wie daarheen}{gaan trots op hun schuld}\\

\haiku{Het kon niet anders,.}{of ze moest  vroeg of laat}{naar Koesj terugkeeren}\\

\haiku{{\textquoteright} {\textquoteleft}In zijn psalm noemt mijn.}{vader Malchizedeqs}{priesterschap eeuwig}\\

\haiku{Het rosse licht deed.}{het goudpoeder op pruik en}{mantel schitteren}\\

\haiku{Jahw\`e verheffe.}{zijn aangezicht over u en}{geve u vrede}\\

\haiku{Den vierden tammoez,{\textquoteright}, {\textquoteleft}.}{zegt de Palestijnlegt de}{zon haar sluier af}\\

\haiku{Is mijn lot eenig, zooals,!}{gij zeidet hoe moet dan het}{uwe worden genoemd}\\

\haiku{Als de geuren in,......}{bloem en kruid ontwaakten met}{zoete bedwelming}\\

\haiku{De weg over de brug.}{leidde achter de heesters}{rechts naar de bronnen}\\

\haiku{Rechts van de beek was,.}{een grasveld waarop het spel}{zou worden gespeeld}\\

\haiku{O Maak in het oog!}{uwer vrienden uw liefste niet}{tot een vermomde}\\

\haiku{Een hoofdstel van goud.}{laat ik maken met beng'lende}{zilveren klokjes}\\

\haiku{Fluiten met nu en.}{dan trompetten en cymbels}{namen het lied over}\\

\haiku{Trompet en cymbel.}{vervingen even melodie}{en begeleiding}\\

\haiku{aan zijn arm gaat een,,.}{bejaarde slanke vrouw de}{moeder van Abida}\\

\haiku{Waar is uw liefste,?}{heengegaan o gij schoonste}{onder de vrouwen}\\

\haiku{van de trali\"en.......}{trok hij zijn hand terug zijn}{stappen verstilden}\\

\haiku{- En, misschien \'e\'en of.}{meer der beste leerlingen}{van de Wijsheidsschool}\\

\haiku{En zou dat den roep?}{van Sjalomo's wijsheid niet}{blootstellen aan spot}\\

\haiku{{\textquoteright} En Nofernere:}{had haar moed ingesproken}{met de opmerking}\\

\haiku{Dan droeg de vorstin,,.}{na Loema's stiltewenk haar}{tweede raadsel voor}\\

\haiku{Doet de gloed haar oog,.}{druipen zoo lacht zij en haar}{traan glipt over haar wang}\\

\haiku{{\textquoteleft}Dank lieve vrienden,!}{dat gij haar hart zoo wijd voor}{mij hebt geopend}\\

\haiku{- Hij staart omhoog in:}{de schemering der vensters}{daarboven en bidt}\\

\haiku{Dr J. Ridderbos {\textquoteleft}{\textquoteright}:}{zegt inKorte Verklaring}{bij Jesaja 43:3}\\

\subsection{Uit: De Aulische}

\haiku{- En vergeet Joaab,.}{niet uws vaders moordenaar}{en die van Amasa}\\

\haiku{{\textquoteleft}Uit eerbied voor uw,:}{keus mijn vriend uit Benjamien}{doe ik dezelfde}\\

\haiku{- Al waren ze niet,.}{enkel vleierij enkel}{waarheid nog minder}\\

\haiku{Hoe zou ze voor haar:}{godin kunnen verschijnen}{zonder de woorden}\\

\haiku{{\textquoteright} {\textquoteleft}Zoo hoorde ik den.}{chakaam-vrouwenkenner}{nog nimmer spreken}\\

\haiku{- {\textquoteleft}Wat een schat,{\textquoteright} sprak ze,.}{glimlachend toen ze de deur}{weer had gesloten}\\

\haiku{Maar, hoewel ik het,!}{waardeerde wat bracht mij dat}{mijn roeping nader}\\

\haiku{- Doe echter van jouw,.}{zijde niets voordat hij het}{onderwerp aanroert}\\

\haiku{Immer voorheen was.}{ze de ziel en de glorie}{onzer festijnen}\\

\haiku{de zacht stralende.}{genegenheden van haar}{diepste vrouwlijkheid}\\

\haiku{{\textquoteright} {\textquoteleft}Ik verwacht thans nog,.}{geen antwoord van je op die}{vraag mijn lieveling}\\

\haiku{Dat zal u den kop.}{verpletteren en gij hem}{den hiel verwonden}\\

\haiku{- Prinses Faroena's.}{gezondheid liet dat minder}{toe den laatsten tijd}\\

\haiku{Mijn vader in een:}{zijner psalmen spreekt geheel}{dienovereenkomstig}\\

\haiku{Plotseling richtte,:}{ze zich half op sperde de}{oogen wijd open en riep}\\

\haiku{Dus duizend vrouwen,!}{die samen een groot geheim}{moeten bewaren}\\

\haiku{{\textquoteright} vroeg de koning, in.}{zijn intiemste vertrek en}{samenzijn gestoord}\\

\haiku{Snel kleedden zich de,.}{twee vrouwen beraadslagend}{wat ze doen zouden}\\

\haiku{Ze is in de angst,{\textquoteright}.}{der wanhoop naar Tyrus gevlucht}{antwoordde Ba\"ana}\\

\haiku{De lampen waren,,.}{met krep omfloersd behalve}{\'e\'en boven het bed}\\

\haiku{Toen trad de koning.}{binnen in zwarten siemlah}{met zilver omboord}\\

\haiku{Telkens voelde ze,:}{haar gedachten doorbroken}{door die andere}\\

\haiku{{\textquotedblright} - Daarin wandelen{\textquoteright}......}{we thans en gij hebt uw woord}{niet gebroken}\\

\haiku{Hem in Sjilo te......}{gaan opzoeken zou wat te}{veel belangstelling}\\

\haiku{- Men zou aan misdaad,.}{kunnen denken maar dat werd}{door niets bevestigd}\\

\haiku{Ze hadden altijd,.}{wel gezegd dat die geen goed}{hier was komen doen}\\

\haiku{Voetstappen op het.}{terras meldden de komst van}{Zaboed en Boeni}\\

\haiku{De pittige geur.}{der gelende bladeren}{deed den ruiter goed}\\

\haiku{In Anatooth had hij.}{sinds lang reeds zijn schuldig hoofd}{ter ruste gelegd}\\

\haiku{De dalen werden,,.}{wijder de hoogten lager}{het landschap vlakker}\\

\haiku{{\textquoteleft}laat ik thans eerst de,.}{boodschap overbrengen die mij}{herwaarts deed komen}\\

\haiku{{\textquoteright} {\textquoteleft}Maar de boodschap, die:}{Machazioth er aan}{had toe te voegen}\\

\haiku{{\textquoteright} {\textquoteleft}Dat meende hij toch,.}{te hebben gedaan maar het}{heeft h\`em overheerscht}\\

\haiku{{\textquoteleft}Ach, ik moet al uw:}{bezwaren toestemmen en}{toch zegt mijn hart mij}\\

\haiku{O Thamaar, die ure,......}{dat ze heengingen en ik}{alleen achter bleef}\\

\haiku{Trouwe vriend van mijn,.}{kort geluk Jahw\`e heeft u}{van mij genomen}\\

\haiku{Zijn ademhaling ging,.}{regelmatig maar met een}{zweem van vermoeidheid}\\

\haiku{{\textquoteright} {\textquoteleft}Vermoeidheid, gevoel.}{van zwakheid of onwelzijn}{kende ik nauwlijks}\\

\haiku{Als alleen de slang,,.}{de groote dierzondares in}{haar vleesch opwekt}\\

\haiku{{\textquoteleft}Mijn vorst sprak,{\textquoteright} Zaboed, {\textquoteleft}.}{op gedempte toonik breng}{u een doodstijding}\\

\haiku{- Na den rouw over haar,.}{maagdom van ruim dertig thans}{den weduwlijken}\\

\haiku{de hope van het......}{ontkiemde zaad ging ook hier}{den oogst te boven}\\

\haiku{Wat  heeft Jahw\`e,.}{aan schimmen dat is aan een}{verscheurde schepping}\\

\haiku{Wat heeft het nieuwe,?}{Paradijs aan schimmen wat}{het nieuw Jeroesjaleem}\\

\haiku{{\textquoteleft}Omdat ik Jahw\`e,.}{geduriglijk voor mij stel}{is mijn hart verblijd}\\

\haiku{ik kan, Sjalomo.}{te herinneren aan het}{jongste verleden}\\

\haiku{{\textquoteright} {\textquoteleft}Mij daar gelaten,.......}{en mijn opvolgers maar ook}{om mijns vaders wil}\\

\haiku{{\textquoteright} {\textquoteleft}Daarom is ons feest.}{een feest der dankbaarheid aan}{Jahw\`e bovenal}\\

\haiku{God verplettert den,,.}{harden kop van wie daarheen}{gaan trots op hun schuld}\\

\haiku{{\textquoteright} {\textquoteleft}Ook op den weg der.}{schaduwen vond uw hoogmoed}{geen bevrediging}\\

\haiku{{\textquoteright} {\textquoteleft}Die drongen in de.}{richting der verwaarloozing van}{mijn schaduwhuwlijk}\\

\haiku{o, Dat alles was,......}{haar geen nieuwe beleving}{slechts herinnering}\\

\haiku{o Maak in het oog!}{uwer vrienden uw liefste niet}{tot een vermomde}\\

\haiku{Een hoofdstel van goud.}{laat ik maken met beng'lende}{zilveren klokjes}\\

\haiku{Leunend op den arm,.}{van haar grooten vriend zocht ze}{haar zitplaats weer op}\\

\haiku{Fluiten met nu en.}{dan trompetten en cymbels}{namen het lied over}\\

\haiku{Trompet en cymbel.}{vervingen even melodie}{en begeleiding}\\

\haiku{Waar is uw liefste,?}{heengegaan o gij schoonste}{onder de vrouwen}\\

\haiku{Haar oogen staan wijd en.}{strak op den wagen met het}{koningspaar gericht}\\

\haiku{De koning wilde.}{zijn werkvertrek als zaal des}{doods zien ingericht}\\

\haiku{Wat heeft het nieuwe,!}{Paradijs aan schimmen wat}{het nieuw Jeroesjaleem}\\

\haiku{{\textquoteright} {\textquoteleft}Nog was het dat bij.}{de komst der koningin van}{Aethiopia en Koesh}\\

\haiku{{\textquoteright} {\textquoteleft}Op uw vraag naar mijn,.}{schuld zult ge in deze rol}{geen antwoord vinden}\\

\haiku{{\textquoteright} {\textquoteleft}Geniet het aanzijn,.}{daarom dankbaar doe wel en}{verwacht Gods gericht}\\

\haiku{{\textquoteright} {\textquoteleft}Boven het leven.}{heerschen de ordeningen}{en raadslagen Gods}\\

\haiku{Zoo was gestalte.}{en rusting van den Vorst van}{het heir des Hemels}\\

\haiku{God niet liefhebben?}{boven alles en mijzelf}{boven mijn naaste}\\

\haiku{Maar er is iets meers.}{dan zelfverontschuldiging}{van het zondaarshart}\\

\haiku{Want Gij hebt geen lust,;}{tot offerande anders}{zou ik ze geven}\\

\haiku{Keer weder, mijn ziel,.}{tot uwe rust omdat Jahw\`e}{u heeft welgedaan}\\

\haiku{De sneeuwtoppen van.}{den Libanoon waren in}{wolken verborgen}\\

\haiku{- E\'en binnen opende.}{de zware kleine deur op}{Eshmoenazaar's geklop}\\

\haiku{{\textquoteright} {\textquoteleft}Dat heb je me nog,,:}{eens gevraagd lieveling en}{toen antwoordde ik}\\

\haiku{- - - - - - - - - Keer weder, mijn ziel,,.}{tot uw rust omdat Jahw\`e}{u heeft welgedaan}\\

\haiku{het duister van mijn.}{zonde en onrecht en den}{glans van Jahw\`es recht}\\

\haiku{terwijl tranen mijn,.}{oogen verduisterden voor mijn}{God op de knie\"en}\\

\haiku{Looft Jahw\`e, mijne,,.}{ziel en al wat in mij is}{zijn heiligen Naam}\\

\haiku{Enkele dagen.}{na zijn genezing riep de}{koning hem tot zich}\\

\haiku{juist is onderkend.}{en haar verwerkelijking}{tijdig voorkomen}\\

\haiku{{\textquoteright} {\textquoteleft}o Vertel ook dat,{\textquoteright}.}{fluisterde ze en vleide}{zich aan zijn schouder}\\

\haiku{{\textquoteleft}Hij naderde met,.}{korte doffe geluiden}{tot recht boven ons}\\

\haiku{En gij zult mij een.}{priesterlijk koninkrijk en}{een heilig volk zijn}\\

\haiku{Ik uwer vijanden,.}{vijand zijn en benauwen}{die u benauwen}\\

\haiku{Mosj\`e als zondaar sluit.}{voor zich den ingang af tot}{den eersten hemel}\\

\haiku{- koningin Abisjag.}{was verzameld tot haar volk}{in het schimmenrijk}\\

\haiku{- En die vreugde duurt,!}{voort nu het opwast tot het}{evenbeeld zijns vaders}\\

\haiku{Wij verdrevenen.......}{uit dat oord der zaligheid}{om onze zonde}\\

\haiku{Opeens riep hij uit:}{met iets van geestverrukking}{in houding en oogen}\\

\haiku{Hij zonk neer op zijn,,.}{zetel doodelijk bleek het hoofd op}{de borst gebogen}\\

\haiku{Ik zal bestendig,.}{bij U zijn want Gij hebt mijn}{rechterhand gevat}\\

\haiku{Toen werd een snik het,.}{teeken dat de band tusschen}{ziel en lichaam brak}\\

\haiku{Bleek en roerloos, zooals,.}{hij gisteren den laatsten}{adem gaf lag hij neer}\\

\haiku{Het Auteurschap van,.}{het boek De Prediker dat}{aanvangt op pag. 497}\\

\haiku{Ze zijn, zooals ieder,.}{kenner weet talloos in de}{Ridderpo\"ezie}\\

\haiku{17Ongeveer f 3000,.}{maar van de koopkracht weten}{we ongeveer niets}\\

\subsection{Uit: De Egyptische}

\haiku{iride\"een, blauwe.}{en witte crocussen en}{fel roode tulpen}\\

\haiku{Ik moest van den zoon,.}{worden meer dan ik eens van}{den vader was}\\

\haiku{maar om na strijdlooze,.}{overwinning door zaligheid}{opnieuw te sterven}\\

\haiku{Een zacht geklop op '.}{de vleugeldeuren vans}{konings werkkamer}\\

\haiku{Daar willen we ons,.}{juublend verheugen uw liefde}{den wijn overprijzen}\\

\haiku{Een hoofdstel van goud.}{laat ik maken met beng'lende}{zilveren klokjes}\\

\haiku{Profetie zei mij,.}{opnieuw dat mijn profetie}{zich ging vervullen}\\

\haiku{ik liefheb, o zeg,...}{hem dat mijn liefde is tot}{waanzin geworden}\\

\haiku{Zacht en zangerig,:}{begon hij te lezen wat}{nieuw was toegevoegd}\\

\haiku{Wat zullen we doen,,?}{als die dag komt dat men zal}{vragen naar haar}\\

\haiku{Blozend beklom Achia.}{het troonbordes en ging rechts}{achter den troon staan}\\

\haiku{{\textquoteright} {\textquoteleft}Wat U heden werd,,.}{voorgedragen was een lied}{een huwelijkslied}\\

\haiku{Het feest was gesteld,.}{in de week die aanvangt met}{den eersten Nisaan}\\

\haiku{{\textquoteright} {\textquoteleft}Dat voorgevoelen.}{wordt uitgedrukt in het lied}{onzer koningen}\\

\haiku{Daar willen w' ons,!}{juublend verheugen uw liefde}{den wijn overprijzen}\\

\haiku{O, maak in het oog.}{uwer vrienden uw liefste niet}{tot een vermomde}\\

\haiku{Een hoofdstel van goud.}{laat ik maken met beng'lende}{zilveren klokjes}\\

\haiku{De laatste gele.}{zonnestralen fonkelden}{op zijn paarlenkroon}\\

\haiku{Toen koor en orkest,.}{zwegen barstte de jubel}{der toeschouwers los}\\

\haiku{Wie is 't die op, -:}{ons als het morgenrood schouwt}{dan lager en teer}\\

\haiku{Ik wil dien palmboom,,:}{beklimmen hoog tot ik die}{trossen kan grijpen}\\

\haiku{omring den nabi.}{met kussens en dek hem toe}{tegen allen tocht}\\

\haiku{- Hoe drinken mijn hart.}{en zinnen den zoelen adem}{der westewinden}\\

\haiku{Ik heb de laatste,.}{jaren veel met uw vader}{gesproken Zaboed}\\

\haiku{De lijfwachten, reeds,.}{opgestegen hadden zich}{in rotten geschaard}\\

\haiku{Dan lijdt de pharao.}{zijn koninklijken gast naast}{zich op den wagen}\\

\haiku{{\textquoteleft}Zijn westen,{\textquoteright} zongen, {\textquoteleft}.}{de priesters van Tanisis}{een tempel van Amon}\\

\haiku{heel een wijk, waar ze.}{hun leven naar wijs en wensch}{kunnen inrichten}\\

\haiku{Een wit, geplisseerd.}{byssusgewaad omsluit de}{gevulde leden}\\

\haiku{En in haar klauwen;}{klemde ze het teeken der}{koningsheerschappij}\\

\haiku{Osiris is in.}{een graf en hij is heerscher}{in het doodenrijk}\\

\haiku{Gij gaat op, gaat op,.}{en straalt en straalt gekroond als}{koning der goden}\\

\haiku{Gij zonnen voor de,.}{menschen die de duisternis}{uwer landen verdrijft}\\

\haiku{Gij hebt gestalten,.}{als onze god R\^e die aan}{den hemel opgaat}\\

\haiku{- Niet, voorzoover dat in.}{den tijd van zijn beelddrager}{slechts enkele zijn}\\

\haiku{{\textquoteright} {\textquoteleft}En van wie hebben?}{de Lybi\"ers blauwe oogen}{en blonde haren}\\

\haiku{{\textquoteright} {\textquoteleft}Ze komen van over,.}{de groote groene zee zeggen}{onze leermeesters}\\

\haiku{Op het ruime plein.}{voor het koninklijk paleis}{zou ze plaats hebben}\\

\haiku{- Allen vielen op.}{hun kni\"een en bogen het}{gelaat tot den vloer}\\

\haiku{De kapel stond op,.}{de bark van Amon cederhout}{met goud overtrokken}\\

\haiku{Ook het Middenrijk.}{heeft zijn grooten op den troon}{van Horus gekend}\\

\haiku{Eindelijk komt weer.}{een horizontale gang}{uit de klimmende}\\

\haiku{de stroom sleepte den.}{edelen vorst van de Delta}{in een wreeden dood}\\

\haiku{Het grootste deel van.}{de bevolking der vlakte}{van Ibdou speelt mee}\\

\haiku{Bij toerbeurt worden.}{de bewoners der vlakte}{er bij ingedeeld}\\

\haiku{Ze dragen hem als!}{hun heer naar een zuivere}{plaats in zijn tempel}\\

\haiku{{\textquoteleft}Nog slechts korten tijd,.}{en gij zult zien wat het huis}{der goden doen zal}\\

\haiku{Waarom de dood voor?}{Osiris en niet voor de}{andere goden}\\

\haiku{En dan de Ka, die...,?}{het gewilde leven zou}{zijn maar waar is hij}\\

\haiku{{\textquoteright} {\textquoteleft}Wat we er zullen,,,.}{zien hebt gij o koning ook}{in Tanis aanschouwd}\\

\haiku{Het heilige meer.}{in hoogopgaand geboomte}{en bloemen gevat}\\

\haiku{{\textquoteright} {\textquoteleft}Daar bemerkte hij, '.}{dats vijands oostvleugel}{nog slecht versterkt was}\\

\haiku{Beken-Chons volgde,;}{hem toen Sjalomo en de}{zijnen \'e\'en voor \'e\'en}\\

\haiku{De neus verweerd tot,.}{op het been de lippen staan}{vernufteloos open}\\

\haiku{De terugreis langs,,!}{een langen rijken reisweg}{is een nieuwe reis}\\

\haiku{{\textquoteright} {\textquoteleft}Gij meent uw beider?}{gave tot het dichten van}{schoone liederen}\\

\haiku{Zij wil de daden.}{van haar wijzen gemaal niet}{veroordeelen}\\

\haiku{Wie was de grootste,?}{pharao die op den troon van}{Cheem heeft gezeten}\\

\haiku{{\textquoteleft}Mij heugen omtrent}{ons bezoek aan Noet-Amon}{vier oogenblikken}\\

\haiku{- Hier in mijn paleis,.}{teruggekeerd ontbreken}{ze mij nog immer}\\

\haiku{{\textquoteleft}Den Edomiet zult gij,.}{voor geen gruwel houden want}{hij is uw broeder}\\

\haiku{- De zaaitijd is steeds,.}{droog geweest sinds gij den troon}{uws vaders besteegt}\\

\haiku{Hij sloeg onstuimig:}{een bekken aan en tot den}{verschijnenden slaaf}\\

\haiku{Van hun zaad zal er.}{geen in de vergadering}{Jahw\`e's komen}\\

\haiku{Een poos heerschte,}{er zwijgen in het kleine}{vertrek tot Zaboed}\\

\haiku{Het was de elfde,.}{der zevende maand een uur}{voor zonsondergang}\\

\haiku{- Sjalomo stond thans, -.}{met zijn kemel aan het hoofd}{van den stoet alleen}\\

\haiku{Jahwe verheffe.}{zijn aangezicht over u en}{geve u vrede}\\

\haiku{Naar vertrekken met...}{verregelde deuren en}{gesloten vensters}\\

\haiku{{\textquoteright} {\textquoteleft}Tusschen die wensch en!}{onze baarden bestaat toch}{zeker geen verband}\\

\haiku{Hoor, o Dochter en;}{neig uw oor Vergeet uw volk}{en uws vaders huis}\\

\haiku{Hij sloeg de schalmei,.}{aan die aan den paleismuur}{in haar smeewerk hing}\\

\haiku{De zegeningen:}{van Jehoeda en Joseef}{vlochten zich dooreen}\\

\haiku{O, moge mijn heer.}{de koning hem daarin nog}{eenmaal overtreffen}\\

\haiku{Benaja en ik.}{zijn uw eenig overgebleven}{oude dienaren}\\

\haiku{De bergen sprongen,.}{als rammen de heuvelen}{als lammeren}\\

\haiku{Ook langs des jongen.}{konings baard vielen paarlen}{op zijn zijden kleed}\\

\haiku{{\textquoteleft}En mijn lieve nicht,}{te oordeelen naar wat uw}{bruidegom ons zoo}\\

\haiku{{\textquoteleft}Zoo waarlijk Jahwe,}{leeft de bouw van dat altaar}{en dat huis onzes}\\

\haiku{Bindt met touwen het.}{offerdier tot vlak bij de}{hoornen des altaars}\\

\haiku{een reukvaas om uw, -.}{koningszaal met zoeten geur}{te vullen zie toch}\\

\haiku{- Eerst kruiste de weg;}{een nog drooge winterbedding}{na langzaam dalen}\\

\haiku{Looft Jahwe, want Hij,!}{is goed want eeuwig duurt zijn}{goedertierenheid}\\

\haiku{In een hymne op,,:}{Ramses II eveneens uit zijn}{tijd treffen we aan}\\

\section{Jan Apon}

\subsection{Uit: De roode anjelier (onder ps. Max Dupont)}

\haiku{Hij heeft een klein maar.}{goed gevormd figuur en is}{zorgvuldig gekleed}\\

\haiku{Zijn kleeding is gewoon,,,.}{niet erg netjes maar ook niet}{slordig doodgewoon}\\

\haiku{Het is de bloem van.}{een in elkaar geknepen}{roode anjelier}\\

\haiku{{\textquoteright}        Hoofdstuk III   .}{C.D. Even later komt Elly}{de kamer binnen}\\

\haiku{Het zal ongeveer.}{halftwaalf zijn geweest toen ik}{op mijn kamer kwam}\\

\haiku{{\textquoteleft}Maar vertelt u me,,!}{eens u voelde zich niet erg}{wel zei u immers}\\

\haiku{Vermeer bedoel ik,.}{in deze kamer en de}{deur stond op een kier}\\

\haiku{Toen hij dus op zijn,.}{herhaald kloppen niets hoorde}{ging hij naar binnen}\\

\haiku{Ik....{\textquoteright} {\textquoteleft}Hoe wist je, dat,{\textquoteright}.}{het precies kwart over tien was}{onderbreekt Reggie}\\

\haiku{Je had Casanova.}{vergeten en wilde nog}{wat lezen in bed}\\

\haiku{{\textquoteright} {\textquoteleft}En waarom heb je?}{ons dat allemaal nu niet}{dadelijk verteld}\\

\haiku{Plotseling draait hij.}{zijn hoofd weer in haar richting}{en kijkt haar strak aan}\\

\haiku{Ik geloof dat ik.}{dan de meeste kans heb om}{haar thuis te treffen}\\

\haiku{De man in kwestie.}{moet daar ongeveer een half}{uur hebben gestaan}\\

\haiku{{\textquoteright} {\textquoteleft}Nou, dat heb je dan,{\textquoteright}.}{maar weer schrander ingezien}{prijs ik sarcastisch}\\

\haiku{Hij stokt, loopt vlug en,.}{geruischloos naar de deur die}{hij met een ruk opent}\\

\haiku{Hij heeft er blijkbaar,.}{niet veel zin in maar durft niet}{goed te weigeren}\\

\haiku{hier is het restje.}{van de sigaret die hij}{zooeven heeft gerookt}\\

\haiku{Hij houdt plotseling.}{op en staart in gedachten}{naar de schemerlamp}\\

\haiku{{\textquoteleft}Neen, ik weet niet waar.}{ze heen is en ook niet hoe}{laat ze terugkomt}\\

\haiku{{\textquoteright} vraag ik kregelig,.}{want door zijn praatjes ben ik}{den tel kwijtgeraakt}\\

\haiku{{\textquoteright} Hij strijkt zich met de:}{hand over het voorhoofd en gaat}{pathetisch verder}\\

\haiku{{\textquoteleft}Komt u even binnen.}{alstublieft en doet u de}{deur achter u dicht}\\

\haiku{Die stok kunt u wel,}{weer terugzetten waar u}{hem hebt gevonden}\\

\haiku{{\textquoteright} Na deze woorden,,.}{stapt hij door ons gevolgd de}{tuinkamer binnen}\\

\haiku{Lucien staat op, gaat.}{naar het buffet en schenkt haar}{een glas water in}\\

\haiku{Het is zoo stil, dat,,.}{ik Lucien die naast mij staat}{gejaagd hoor ademen}\\

\haiku{{\textquoteleft}Als u me nog iets,,}{te vragen hebt doet u het}{dan alstublieft nu}\\

\haiku{Ik ben in tweestrijd.}{of ik al dan niet naar mijn}{slaapkamer zal gaan}\\

\haiku{Frans Ferguson moet.}{naar mijn schatting ongeveer}{dertig jaar oud zijn}\\

\haiku{Na het diner heb.}{ik me verkleed om naar de}{Apollolaan te gaan}\\

\haiku{Hij zoekt zorgvuldig.}{een sigaret uit en knipt}{zijn aansteker aan}\\

\haiku{Zeker, het is heel,!}{interessant d\`at wilde}{u zeker zeggen}\\

\haiku{Het had er meer van!}{of hij naar motgaten zocht}{of iets dergelijks}\\

\haiku{{\textquoteright} {\textquoteleft}Weet je misschien welk '?}{deel van de jas hij int}{bijzonder bekeek}\\

\haiku{Ferguson is net!}{zoo onschuldig als hier dat}{kanariepietje}\\

\haiku{{\textquoteleft}Ik heb persoonlijk.}{met den kellner gesproken}{die ze bediend heeft}\\

\haiku{Volgens hem zijn ze.}{stellig niet voor halfelf uit}{Zandvoort vertrokken}\\

\haiku{{\textquoteleft}En, voel je je nog?}{steeds op je gemak in dit}{sinistere huis}\\

\haiku{{\textquoteleft}Enfin, met behulp.}{van den notaris zal ik}{het wel klaar spelen}\\

\haiku{{\textquoteleft}Heeft u ook altijd?}{zoo'n verschrikkelijken dorst}{als het zoo warm is}\\

\haiku{logeeren om geld te,.}{leenen zijn portefeuille te}{laten verliezen}\\

\haiku{{\textquoteleft}Gelukkig dan maar,,!}{dat u haar hier verloren}{hebt meneer Vermeer}\\

\haiku{Je verregaande.}{onbeschaamdheid gaat alle}{perken te buiten}\\

\haiku{{\textquoteleft}Doe me een plezier,.}{blijf vanmiddag thuis en geef}{je oogen flink den kost}\\

\haiku{{\textquoteleft}Wat ter wereld,{\textquoteright} vraag, {\textquoteleft}?}{ik me afis beter dan}{een koude douche}\\

\haiku{Om u de waarheid,.}{te zeggen ik verwacht hem}{ook met ongeduld}\\

\haiku{{\textquoteright} {\textquoteleft}Dan is het verder.}{het beste dat u alles}{maar aan ons overlaat}\\

\haiku{Daarom, als u zich,....{\textquoteright} {\textquoteleft}?}{niet een beetje in acht neemt}{danIs Mona ziek}\\

\haiku{Als ik klaar ben met,.}{mijn verhaal blijft hij zwijgend}{voor zich uit staren}\\

\haiku{Ze geven me een.}{onbehaaglijk gevoel van}{naderend onheil}\\

\haiku{Het is bijna kwart.}{voor negen als Dorothee de}{kamer binnen komt}\\

\haiku{Ik herinner me,.}{trouwens heel goed dat ze het}{vanmorgen nog droeg}\\

\haiku{Ik zelf heb haar, zij,.}{het dan ook bij toeval dien}{middag gevonden}\\

\haiku{Het is reeds kwart voor.}{elf en Rudolf is dus al}{drie kwartier over tijd}\\

\haiku{{\textquoteleft}Ik zal even met je,.}{mee naar boven gaan dan kan}{ik je verbinden}\\

\haiku{Daarna ontspannen.}{zijn trekken zich en komt hij}{langzaam naar me toe}\\

\haiku{{\textquoteright} {\textquoteleft}All right,{\textquoteright} brom ik en.}{sta op om de deur achter}{hem op slot te doen}\\

\haiku{{\textquoteright} {\textquoteleft}Toen,{\textquoteright} vervolgt Reggie, {\textquoteleft},:}{toen deed ik voorloopig}{niets dat wil zeggen}\\

\haiku{Daardoor groeiden m'n.}{vermoedens omtrent Lucien}{tot zekerheid aan}\\

\haiku{Zooals je weet, had hij.}{op den avond van den moord bij}{een vriend gedineerd}\\

\haiku{aanvankelijk had.}{hij niet de bedoeling om}{haar te gebruiken}\\

\section{Frank Martinus Arion}

\subsection{Uit: Afscheid van de koningin}

\haiku{De koningin was.}{er ook de laatste keer dat}{ik in Songo was}\\

\haiku{Ze is tweemaal zo,.}{groot als Nederland met 3}{miljoen inwoners}\\

\haiku{Of pernods, want!}{Gaston Senyo Wawili}{was zeer francofiel}\\

\haiku{De laatste resten.}{van mijn jeugd achterhaalden}{mij in die balzaal}\\

\haiku{Dat was het enige.}{wat ik voor de koningin}{niet kon opbrengen}\\

\haiku{Ze waren blijkbaar.}{andere reacties op}{hun testvraag gewend}\\

\haiku{Maar het was nu ook!}{een belangrijke zaak die}{aan de orde was}\\

\haiku{Ik ben journalist!}{zoals u ziet en ik wil}{mijn krant opbellen}\\

\haiku{{\textquoteleft}Ik geef er meer om,,}{zo gezellig hier met u}{te zitten praten}\\

\haiku{Maar ze was voor een!}{studente zoals mij voor}{ogen stond te zwijgzaam}\\

\haiku{Als je per se iets,?!}{wilt drinken kunnen we een}{flesje meenemen}\\

\haiku{{\textquoteleft}Maar waarom wil je,?}{met me pr\'aten je weet toch}{al dat ik je mag}\\

\haiku{Ze werkte op haar.}{rug en ze verwachtte van}{mij een bijdrage}\\

\haiku{Ik had voldoende.}{geld om het een tijdlang uit}{te kunnen zingen}\\

\haiku{Door mijn studie in.}{de economie ontdekte}{ik dat allemaal}\\

\haiku{Ik ben misschien te.}{jong om jou helemaal te}{kunnen begrijpen}\\

\haiku{Anders zou voor haar.}{alleen maar vaststaan dat ik}{haar als vrouw afwees}\\

\haiku{Ze sloeg me hard op:}{m'n dij met haar vlakke hand}{en zei kwasi boos}\\

\haiku{Een handig ding ook.}{voor plaatsen waar je niet mag}{fotograferen}\\

\haiku{{\textquoteright} Hij onderzoekt het.}{effect van zijn woorden eerst}{even in zijn spiegel}\\

\haiku{Dan zou de dood van?}{die anderen eenvoudig}{een ongeluk zijn}\\

\haiku{Ga maar eens kijken!}{in de St. Pieter als de}{paus de mis opdraagt}\\

\haiku{ik heb nog meer van{\textquoteright}.}{hetzelfde als jullie ook}{durven opkomen}\\

\haiku{Maar Ali lijkt me in.}{ieder geval een echte}{mensenverkoper}\\

\haiku{Interessant hoe,?!}{de dingen soms kunnen}{samenvallen he}\\

\haiku{We wilden onze.}{honeymoon in april \`en in}{Parijs doorbrengen}\\

\haiku{{\textquoteleft}I never knew the.}{charms of spring Never met it}{face to face}\\

\haiku{Anders hadden we{\textquoteright} {\textquoteleft}.}{toch-In Amsterdam is}{het ook niet alles}\\

\haiku{Maar er waren wel,?}{niet zoveel zwarten daar toen}{als nu waarschijnlijk}\\

\haiku{En slim ontwijken.}{zoals hij trouwens ook doet}{als het moeilijk wordt}\\

\haiku{{\textquoteright} {\textquoteleft}Interessant,{\textquoteright} zegt, {\textquoteleft}:}{hij weer duidelijk verrast}{heel interessant}\\

\haiku{Zo'n klein land... zoveel,.}{grote ondernemingen}{zoveel partijen}\\

\haiku{Hij zet daarna zijn.}{bril op en onderzoekt het}{allemal opnieuw}\\

\haiku{Op tennisgebied.}{presteren ze ook wel wat}{met hun Tom Okker}\\

\haiku{die bars in te gaan,,.}{als u er bent maar u moet}{w\`el voorzichtig zijn}\\

\haiku{Zo zelfs, dat ik een.}{soort ambassadeur van mijn}{land ben geworden}\\

\haiku{Dit hele gesprek,.}{met je geeft me veel stof tot}{nadenken weet je}\\

\haiku{Ik heb dat wel meer,.}{horen beweren maar het}{is volstrekt onzin}\\

\haiku{Misschien wil ze ook.}{gewoon een keer een zwarte}{man goed verwennen}\\

\haiku{Zeg maar, dat ik de.}{situatie toch wel wat}{humoristisch vind}\\

\haiku{Ze probeert opnieuw,.}{te lezen maar legt het boek}{gauw weer op haar schoot}\\

\haiku{Met irre\"ele.}{getallen zoals dat in}{de wiskunde gaat}\\

\haiku{Niet helemaal zwart,.}{zoals ik mezelf voor het}{gemak wel beschrijf}\\

\haiku{6 Direct na ons.}{vertrek krijgen we deze}{keer een warme lunch}\\

\haiku{Ik zeg wanhopig!}{tegen mezelf dat de vrouw}{hier naast me dom is}\\

\haiku{Daarom was ik zo.}{verbaasd toen u me vroeg of}{ik B\'elgische was}\\

\haiku{En die Bobbejan,.}{gaat de rooien verslaan de}{Indianen dus}\\

\haiku{Het wordt allemaal.}{door anderen uitgemaakt}{v\'o\'or je geboorte}\\

\haiku{{\textquoteleft}Ja,{\textquoteright} zeg ik, {\textquoteleft}en in.}{Nederland ken ik iemand}{die Du Plessis heet}\\

\haiku{Want misschien wil ik,{\textquoteright}.}{er inderdaad zelf een keer}{heen zegt ze lachend}\\

\haiku{{\textquoteleft}Neen, blijft u maar bij,.}{het raam zitten dan kan ik}{ook naar buiten zien}\\

\haiku{{\textquoteleft}Ik doe dus hard m'n,.}{best om z\'o uit te komen}{dat begrijpt u wel}\\

\haiku{Het is logisch aan,.}{te nemen dat hij dat niet}{zo leuk zal vinden}\\

\haiku{Balthasar Vorster.}{zich voor zijn overtuiging in}{zeer slecht gezelschap}\\

\haiku{Ambitieuze.}{mensen zijn inderdaad vaak}{opportunistisch}\\

\haiku{{\textquoteleft}Moeder is al zo,{\textquoteright}, {\textquoteleft}.}{oud zeggen zeen ze is}{ook zo ziekelijk}\\

\haiku{N\'og niet tenminste,.}{daarom draai ik voorlopig}{nog om Nederland}\\

\haiku{In deze uren ben.}{ik een beetje een ander}{mens aan het worden}\\

\haiku{Dat is alles wat.}{ik weet en dat ik niemand}{aan land mag laten}\\

\haiku{Ik vraag me af, wat!}{ze in siaieesnaam daar}{hopen te vinden}\\

\haiku{{\textquoteright} Ik wijs met mijn hoofd.}{naar de soldaten v\'o\'or het}{stationsgebouw}\\

\haiku{En kijk, ze halen,.}{ook de koffers er uit dat}{is straks niet gezegd}\\

\haiku{Terwijl we naar het,.}{stationsgebouw lopen}{herhaal ik de vraag}\\

\haiku{{\textquoteright} {\textquoteleft}Wawili's dood was, -?}{dus niet zomaar een moord maar}{het duidelijke}\\

\haiku{Nota bene een.}{vrouw die zich zeer inspande}{voor de mensen hier}\\

\haiku{En nu hebben die.}{Ieren haar in  opdracht}{van Bakari vermoord}\\

\haiku{The Barrel of a, {\textquoteleft}.}{Guninterventie is geen}{pl{\'\i}cht van de Fransen}\\

\haiku{Waarom hebt u hem?}{niet gezegd dat ik zo graag}{naar Tamina wil}\\

\haiku{Als ik besluit me}{van het een en ander te}{gaan vergewissen}\\

\haiku{Deze hier die de,.}{leiding heeft ging meteen voor}{hem in de houding}\\

\haiku{Als ik terug ben,.}{staat mevrouw Jobert ook naar}{het lijk te kijken}\\

\haiku{U woonde niet bij,?}{uw dochter maar in Hotel}{Tamina nietwaar}\\

\haiku{Ik dacht eigenlijk.}{dat ze u en mevrouw hier}{ook geraakt hadden}\\

\haiku{En met de wens voor.}{een happy landing kondigt}{hij het diner aan}\\

\haiku{{\textquoteright} {\textquoteleft}Ja, het laatst werkte;}{u met de vrouwen op de}{markt of voor de fao}\\

\haiku{Nieuw Nederland, een,.}{g\'oede krant en verder van}{alles zo'n beetje}\\

\haiku{Mevrouw Prior lacht.}{hardop en genietend van}{onze verbazing}\\

\haiku{Vlak nadat ik weg.}{was uit Holland waren er}{nog demonstraties}\\

\haiku{Dat is het gebied,.}{tussen Enkhuizen en Hoorn}{waar mijn broer nu zit}\\

\haiku{Eerst nog met een muur,,.}{dus half om half maar daarna}{helemaal van glas}\\

\haiku{Op een hoek van de,.}{straat in de Transvaalbuurt dat}{is Amsterdam-Oost}\\

\haiku{Je zegt maar dat die,!}{orchidee\"en morgenvroeg}{gebracht worden hoor}\\

\haiku{E\'en man speciaal.}{voor de bestellingen en}{twee in de winkel}\\

\haiku{Het zaken doen wordt.}{er niet gemakkelijker}{op tegenwoordig}\\

\haiku{Ja, dat is de meer.}{moderne manier zoals}{u het haar nu heeft}\\

\haiku{Ik verwissel ook.}{het cassettetje van mijn}{kleine apparaat}\\

\haiku{De katholieke.}{bond van landarbeiders was}{altijd zo langzaam}\\

\haiku{Die jongen was nog!}{donkerder dan u. Maar een}{aardige jongen}\\

\haiku{Als hij kwam en ik,:}{stond nog in de winkel dan}{kwam hij naar me toe}\\

\haiku{De mensen uit het.}{noorden en oosten komen}{die vis hier kopen}\\

\haiku{Ik zeg tegen m'n,:}{dochter nadat ik een twee}{maanden bij haar ben}\\

\haiku{Het is zo dat het.}{vuil bij hen soms in weken}{niet wordt opgehaald}\\

\haiku{Daarom wilde ik.}{nu eens te voet en op mijn}{gemak alles zien}\\

\haiku{Of de bestaande.}{situatie zelfs probeert}{te verdedigen}\\

\haiku{Vooral als het even.}{mis gaat met de oogst is er}{grote mis\`ere}\\

\haiku{Ook om Brigitte,.}{Bardot die dus eigenlijk}{schapenhoedster is}\\

\haiku{{\textquoteright} zegt mevrouw Prior, -! -.}{terwijl ik de foto veel}{te lang bestudeer}\\

\haiku{Ze denkt diep na, om,.}{daarna te herhalen dat}{het dat echt niet was}\\

\haiku{Ze had ook gehoopt.}{bij mijn dochter nederlands}{te kunnen leren}\\

\haiku{Enfin, Gadizha.}{heeft toch wel een paar woorden}{van me opgepikt}\\

\haiku{Ik zeg terwijl ik:}{een nieuwe cassette in}{de recorder doe}\\

\haiku{Daar moest hij zichzelf,.}{z'n vrouw en twee kinderen}{van onderhouden}\\

\haiku{Want de mensen in!}{Nanik\'e gingen allemaal}{mee verzamelen}\\

\haiku{Ik ging gewoon uit.}{van wat orchidee\"en in}{Holland opbrengen}\\

\haiku{Als mijn man er nog,,.}{maar was dacht ik soms want die}{wist daar alles van}\\

\haiku{Maar ik zag niet in!}{dat ik welke economie}{dan ook ontwrichtte}\\

\haiku{Ik was al zo lang,.}{in Hotel Tamina dat}{ik m'n gang mocht gaan}\\

\haiku{Je hebt in een van,.}{die straten de winkel van}{een Syri\"er Omar}\\

\haiku{En sommige van.}{die jonge mensen van het}{vrijwilligerskorps}\\

\haiku{{\textquoteleft}Die mensen doen dat,.}{allemaal uit vrije wil dat}{is heel wat anders}\\

\haiku{Naomi's kin is rond.) {\textquoteleft}.}{en haar neus kortAllemaal}{opportunisme}\\

\haiku{Na die aanslag op.}{Wawili zijn links en rechts}{mensen opgepakt}\\

\haiku{Er wordt gezegd dat -{\textquoteright} {\textquoteleft},.}{een deel van het legerJa}{dat is de l\'andmacht}\\

\haiku{Ik geeft haar een hand.}{om haar te bedanken voor}{het interview}\\

\haiku{Bij de staatsgreep in,!}{Chili had je ook zo een}{die maar zijn gang ging}\\

\haiku{{\textquoteright} {\textquoteleft}Mijn ervaring geldt.}{vooral de ontvangende}{kant van de tafel}\\

\haiku{En ik wilde je;}{het een en ander rustig}{kunnen uitleggen}\\

\haiku{Daarom heb ik je.}{ook niet gewaarschuwd dat die}{kolonel er was}\\

\haiku{dat mevrouw Prior.}{hoe dan ook gevaar liep in}{Hotel Tamina}\\

\haiku{{\textquoteright} {\textquoteleft}Ach kom nou, Sesa,{\textquoteright} {\textquoteleft},.}{nu ga j{\'\i}j te ver.Neen als}{ze t\'och vrienden zijn}\\

\haiku{Ik heb haar beloofd.}{haar over enkele dagen}{weer op te zoeken}\\

\haiku{Naomi weigert de.}{slaapkamer te nemen als}{ik haar die aanbied}\\

\haiku{Al vraag ik me af:}{of hun hele geloof niet}{\'e\'en groot excuus is}\\

\haiku{Toen dacht ik, deze!}{zwarte man is toch echt een}{grote gentleman}\\

\haiku{Ik zou op den duur.}{toch met een Zuidafrikaan zijn}{getrouwd waarschijnlijk}\\

\haiku{En ik heb toch nooit.}{iets gedaan om het met haar}{weer goed te maken}\\

\haiku{Ik heb moeite om,}{te verbergen wat ik weet}{maar de zekerheid}\\

\haiku{Ze wentelt haar hoofd,,.}{met de ogen dicht woest heen en}{weer op het kussen}\\

\haiku{Ze zegt, dat ze toch.}{niet denkt lang in Nederland}{te willen blijven}\\

\haiku{Ik wilde toen nog.}{eerst wachten totdat ik een}{tijd was weggeweest}\\

\haiku{Bijna zo beheerst,,.}{als Jozef Maria zo heb}{ik haar behandeld}\\

\haiku{{\textquoteleft}Ik ben ook al oud.}{en wijs genoeg om ervoor}{te kunnen z\'orgen}\\

\haiku{Het geeft je dus toch?}{een beetje voldoening dat}{ik niet terugga}\\

\haiku{{\textquoteright} {\textquoteleft}En omdat ik jouw,{\textquoteright}, {\textquoteleft}.}{moed bewonder zeg ikniet}{uit zwakte alleen}\\

\haiku{Ik vertel haar van.}{Colombia waar ik ook een}{tijd gezeten heb}\\

\haiku{{\textquoteright} Naomi springt naakt uit.}{bed en haalt de recorder}{uit de werkkamer}\\

\section{Diederik van Assenede}

\subsection{Uit: Floris en Blancefloer}

\haiku{Ze zouden liever.}{dood zijn dan lang van elkaar}{gescheiden te zijn}\\

\haiku{Maar ze had er niets.}{van gemerkt dat er zo over}{haar gesproken werd}\\

\haiku{op het graf waren,.}{lange buizen gemaakt waar}{de wind doorheen blies}\\

\haiku{Er werd ook een boom -!}{geplant zo vindt men er niet}{\'e\'en in heel het land}\\

\haiku{Floris vond het graf,:}{heel mooi hij zag de letters}{en las wat er stond}\\

\haiku{{\textquoteleft}Floris, m'n lieve,!}{kind wat heb je een dwaze}{liefde gekoesterd}\\

\haiku{Hij zal zijn verdriet,.}{helemaal vergeten u}{zult het spoedig zien}\\

\haiku{En, heer, ik vraag u}{en mijn moeder bovendien}{me te vertellen}\\

\haiku{Maar Floris was er.}{blij om dat ze nog leefde}{en niet gedood was}\\

\haiku{Toen de maaltijd was,.}{bereid werden de grote}{tafels opgezet}\\

\haiku{Mijn liefde voor haar,.}{houdt me zo in haar ban dat}{ik ben gaan zwerven}\\

\haiku{Hij is gemaakt van,.}{rood marmer en heel mooi rond}{op een rond voetstuk}\\

\haiku{Zodra er een bloem,,.}{valt of geplukt wordt groeit er}{weer een nieuwe aan}\\

\haiku{U kunt begrijpen,,!}{hoe bang Floris de kroonprins}{van Spanje wel was}\\

\haiku{Hij is mijn troost en,!}{toeverlaat heel mijn geluk}{is in zijn handen}\\

\haiku{Wanneer ze iemand}{goedgezind is geweest en}{hem heeft toegestaan}\\

\haiku{Ondertussen was.}{haar vriendin Clarijs snel naar}{de zuil gelopen}\\

\haiku{Je kunt geen levend.}{schepsel bedenken of het}{was daar afgebeeld}\\

\haiku{de veelgeprezen,.}{Blancefloer die hierboven}{in mijn toren woont}\\

\haiku{Ik trok het zwaard, ze.}{werden wakker en smeekten}{me om genade}\\

\haiku{Hoewel de emir zeer,.}{vertoornd was vond hij het zelf}{ook verschrikkelijk}\\

\haiku{De beeldhouwwerken;}{op het graf dragen deze}{bloemen in de hand}\\

\haiku{mijn alderliefste,.}{die ick oyt aensach Adieu het}{moet ghescheyden zijn}\\

\haiku{W.P. Gerritsen \& A.G. (),.}{van Mellered. Van Aiol}{tot de Zwaanridder}\\

\section{T. Avany}

\subsection{Uit: Dolly de danseres}

\haiku{Veronderstel je,}{ook maar \'e\'en moment dat het}{je zou gelukken}\\

\haiku{Hoe heb je het toch,,{\textquoteright}}{klaargespeeld je zoo vlug te}{verkleeden Dolly}\\

\haiku{{\textquoteleft}U heeft ons eenige.}{momenten van het grootste}{genot geschonken}\\

\haiku{Hoe zou ik mij met,?}{U kunnen meten ik ben}{toch geen danseres}\\

\haiku{Zij hield even op als,.}{verwachtte zij een antwoord}{doch hij bleef zwijgen}\\

\haiku{Na afloop van den.}{maaltijd ging het geheele}{gezelschap uiteen}\\

\haiku{ik had je beloofd.}{je een kijkje achter de}{schermen te gunnen}\\

\haiku{Ik ga even achter,;}{dit scherm staan dan zie je mij}{straks verkleed voor je}\\

\haiku{{\textquoteright}, riep Edith verbaasd uit,.}{te zeer verbluft om nog iets}{anders te zeggen}\\

\haiku{Je leek in dat hemd,,.}{vooral door die lichtstralen}{bijna geheel naakt}\\

\haiku{ik kan het denkbeeld,.}{niet van mij afzetten dat}{hij een masker draagt}\\

\haiku{{\textquoteleft}Wat heeft Clarcke,?}{U eigenlijk misdaan dat}{U hem steeds ontwijkt}\\

\haiku{haar woorden hadden.}{blijkbaar niet geheel hun}{uitwerking gemist}\\

\haiku{Hij wilde het doen,.}{voorkomen alsof zij juist}{de trap opkwamen}\\

\haiku{{\textquoteleft}U gevoelt, geloof,,.}{ik meer voor paardensport als}{ik mij niet vergis}\\

\haiku{{\textquoteright} Op dat oogenblik,.}{verscheen een der bedienden}{die zoekend rondkeek}\\

\haiku{Was je reeds daar, toen?}{Mr. Harold Wright met Mr. Dunn}{de kamer verliet}\\

\haiku{Die viel natuurlijk,{\textquoteright}.}{buiten het verbod voegde}{hij er terloops bij}\\

\haiku{{\textquoteright} vroeg hij den heer, die.}{reeds toebereidselen voor}{zijn vertrek maakte}\\

\haiku{Hij schoof een stoel bij.}{de tafel en nam zelf op}{een andere plaats}\\

\haiku{De tweede sigaar.}{stak U aan met het laatste}{puntje der eerste}\\

\haiku{De detective,.}{trad nu binnen waar Edith hem}{reeds tegemoettrad}\\

\haiku{{\textquoteright} Zij opende de deur.}{verder en noodigde hem uit}{binnen te treden}\\

\haiku{Verruimd haalde zij,.}{adem toen de deur zich achter}{hem had gesloten}\\

\haiku{{\textquoteleft}Wacht,{\textquoteright} vervolgde zij,, {\textquoteleft}.}{haar tweede kous uittrekkend}{ik zal eens even zien}\\

\haiku{{\textquoteright} Zij keek haar vriendin,.}{die haar nog steeds omkneld hield}{ook thans nog niet aan}\\

\haiku{Vera lachte eveneens,.}{doch het scheen haar niet geheel}{van harte te gaan}\\

\haiku{{\textquoteright} {\textquoteleft}Persoonlijk heb ik,{\textquoteright}.}{U echter nimmer ontmoet}{merkte Dolly op}\\

\haiku{Dat is immers een,?}{der dagen waarop ik zal}{moeten optreden}\\

\haiku{voor de pauze het,.}{symphonie-orkest na de}{pauze Miss Forest}\\

\haiku{{\textquoteleft}Is er misschien nog,?}{iets anders waarmede ik}{U van dienst kan zijn}\\

\haiku{Hoe dicht onder mijn,{\textquoteright}.}{bereik heb ik haar gehad}{knarsetandde hij}\\

\haiku{Toen ik het kantoor,.}{verliet sprak ik hem nog even}{in de wachtkamer}\\

\haiku{{\textquoteleft}Kom hier eens naast mij,.}{zitten want ik heb iets met}{je te bespreken}\\

\haiku{Sinds ik jou heb leeren,,}{kennen kan er geen ander}{voor mij meer bestaan}\\

\haiku{Bovendien, als je,, -}{je werk goed verricht zal je}{na je harde taak}\\

\haiku{{\textquoteright} Wright, die inzag, dat,.}{hij een domheid begaan had}{knikte levendig}\\

\haiku{{\textquoteright}, riep hij opgewekt,.}{uit met uitgestoken hand}{op haar toetredend}\\

\haiku{Ik moet nog vijf maal.}{optreden en vertrek dan}{naar Baltimore}\\

\haiku{of ik daarna naar,.}{New-York terug zal}{keeren weet ik nog niet}\\

\haiku{Wright fluisterde haar,.}{iets in het oor waarop Vera}{toestemmend knikte}\\

\haiku{Langzaam wandelde,.}{hij verder twijfel was bij}{hem opgekomen}\\

\haiku{Heb je overigens?}{nooit iets ten nadeele van Miss}{Forest opgemerkt}\\

\haiku{Dan is mij ook haar.}{houding tegen over Dunn steeds}{een raadsel geweest}\\

\haiku{{\textquoteright} Harrison volgde,.}{den ander in de gang waar}{Wright hem staande hield}\\

\haiku{Plotseling zag hij:}{den door hem te volgen weg}{duidelijk voor zich}\\

\haiku{Wonderlijk genoeg.}{bleek deze geen letsel te}{hebben bekomen}\\

\haiku{{\textquoteright} Voor hij zelf wist, wat,:}{dit eigenlijk beteekende}{had hij geantwoord}\\

\haiku{Haar mantel liet zij.}{met een losse beweging}{achter zich glijden}\\

\haiku{{\textquoteright} Dolly slaakte een.}{zucht en haar gezicht toonde}{duidelijk onrust}\\

\haiku{{\textquoteleft}Althans niet dat gij.}{in staat zoudt zijn iemand leed}{te berokkenen}\\

\haiku{Inspecteur Maxwell,,.}{die het onderzoek leidt is}{van haar schuld overtuigd}\\

\haiku{Zooals U wellicht weet,,.}{is het mij niet om geld te}{doen Mr. Harrison}\\

\haiku{Men heeft mij verteld,,.}{dat het Miss Forest was die}{Mr. Wright vermoordde}\\

\haiku{Punt 1 en punt 2,.}{hingen samen terwijl punt}{3 op zichzelf stond}\\

\haiku{Er is druk werk voor,,{\textquoteright}.}{vanavond Gladys lichtte Jack}{de jonge vrouw in}\\

\haiku{Doch daar kan het niet,.}{van zijn want dan zou zij ook}{ziek zijn geworden}\\

\haiku{{\textquoteright} Harrison voelde.}{zich  een siddering door}{het lichaam loopen}\\

\haiku{Hij, die het briefje,.}{met het teeken trekt zal de}{gelukkige zijn}\\

\haiku{{\textquoteleft}Blijf nog wat liggen,,.}{Miss Forest totdat U zich}{wat sterker gevoelt}\\

\haiku{Jack reikte haar het.}{glas aan en noodigde haar uit}{het te ledigen}\\

\haiku{Uw verklaringen,.}{zullen aantoonen of ik}{daarin gelijk heb}\\

\haiku{Het was toen, dat ik,.}{op het denkbeeld kwam mij naar}{U te begeven}\\

\haiku{Zijn blik scheen tot in.}{het diepst van haar ziel te}{willen doordringen}\\

\haiku{{\textquoteright} {\textquoteleft}Aha, ik geloof, dat,{\textquoteright}.}{Harrison's kansen nog zoo slecht}{niet staan schertste Jack}\\

\haiku{Ik voel, dat gij een,.}{persoon zijt die hiervan geen}{misbruik zal maken}\\

\haiku{{\textquoteleft}Hoevelen in haar?}{plaats zouden niet moreel ten}{gronde gegaan zijn}\\

\haiku{Hij liet zich op een.}{knie vallen en drukte haar}{hand aan zijn lippen}\\

\haiku{Op luidruchtige;}{wijze onderhielden zij}{zich met elkander}\\

\haiku{{\textquoteright} klonk het thans helder,.}{zoodat het door het geheele}{vertrek hoorbaar was}\\

\haiku{Zij wachtte even en,.}{keek naar den sheik die haar als}{versteend aanstaarde}\\

\haiku{Ik zal dan ook niet.}{de moeite nemen daar thans}{op te antwoorden}\\

\haiku{Als gij mijn verhaal,.}{gehoord hebt zult gij het mij}{moeten toegeven}\\

\haiku{{\textquoteleft}Verspil Uw woorden,,.}{niet Mr. Harrison het is}{noodelooze moeite bij haar}\\

\haiku{De buitenpartij.}{bij Wright stelde mij voor tal}{van nieuwe feiten}\\

\haiku{Speciaal gij, Miss,.}{Forest interesseerdet}{mij in hooge mate}\\

\haiku{{\textquoteright} {\textquoteleft}Het dient nergens toe,,.}{op een dergelijken toon}{voort te gaan Mr. Wright}\\

\haiku{Miss Forest zocht U,.}{op en deelde U mede}{wat er gebeurd was}\\

\haiku{Het was een val die.}{ik voor hem had opgezet}{en hij liep er in}\\

\haiku{Ik heb nog een en.}{ander onder vier oogen met}{U te bespreken}\\
