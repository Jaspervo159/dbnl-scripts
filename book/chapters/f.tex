\chapter[12 auteurs, 1228 haiku's]{twaalf auteurs, twaalfhonderdachtentwintig haiku's}

\section{Johan Fabricius}

\subsection{Uit: Leeuwen hongeren in Napels}

\haiku{de branding hief het;}{op en slingerde het weer}{in z'n koers terug}\\

\haiku{Mangia-tutto;}{wilde men heelemaal niet}{meer laten heengaan}\\

\haiku{na haar zes panters;}{had zij ook hem reeds onder}{haar ban gekregen}\\

\haiku{het bewijs daarvan.}{vormde de halfleege tent van}{den Zaterdagavond}\\

\haiku{Het terrein was nu,.}{tenminste halfweegs droog dank}{zij zijn ingrijpen}\\

\haiku{{\textquoteleft}En al die wagens,?}{dan die er elken dag maar}{komen aanrijden}\\

\haiku{Dadelijk  zou.}{de ellende van voren}{af aan beginnen}\\

\haiku{Een ander circus.}{zou er toevallig verlet}{om moeten hebben}\\

\haiku{En durft geen enkel...?}{circus het meer aan wat baat}{Saul dan zijn contract}\\

\haiku{{\textquoteright} Gottfried Sturm knikte,.}{langzaam alsof hij het zelf}{niet meer geloofde}\\

\haiku{Toen bleek eensklaps, dat.}{de Sardijnsche slager niet}{wilde teekenen}\\

\haiku{Zoo ging met dezen;}{gezamelijken inkoop}{geruime tijd heen}\\

\haiku{het scheen hem zelf toe.}{alsof hij al jaren lang}{met hen verkeerde}\\

\haiku{zij vertelden hem,;}{van moeilijke dagen die}{achter hen lagen}\\

\haiku{tegen het einde.}{van het leeuwennummer liep}{alles voorbeeldig}\\

\haiku{Ook aan het meisje.}{in de loge dacht hij daar}{tusschen door wel weer}\\

\haiku{Bedrukt, verward zocht.}{Gottfried Sturm nog weer naar een}{eervollen aftocht}\\

\haiku{- Ja... en hijzelf kwam.}{er dan natuurlijk niet meet}{voor in aanmerking}\\

\haiku{Rambaldo fronste.}{het voorhoofd en dacht snel en}{ingespannen na}\\

\haiku{Toen keek hij in zijn,.}{actentasch boordevol van}{circusgeheimen}\\

\haiku{Een weck was zeven.}{dagen en elke dag kon}{nog redding brengen}\\

\haiku{Hij was de gansche.}{zaak plotseling een weinig}{moede geworden}\\

\haiku{{\textquoteleft}Als u al optreedt,,!}{signor Saul maakt u het dan}{tenminste bekend}\\

\haiku{Deze gaf het weer,;}{aan Grazia door die juist met}{Gi-gi speelde}\\

\haiku{Bij het afscheid keek.}{zij hem nog eenmaal aan met}{haar Sphynxenblik}\\

\haiku{maar zijn leeuwen hier,.}{laten weghalen dat was}{nog weer wat anders}\\

\haiku{{\textquoteright} Rambaldo voelde.}{zich teleurgesteld en wist}{zelf niet goed waarom}\\

\haiku{En wij  hoeven,.}{alleen maar te zorgen dat}{ze hun thrill krijgen}\\

\haiku{{\textquoteright} Hij lachte zelf om.}{den goeden mop en sloeg Saul}{gul op den schouder}\\

\haiku{{\textquoteleft}Nou, en wat wou die,?}{vent nou van me die niet de}{burgemeester was}\\

\haiku{Levend vee en ook,.}{vleesch als ze aan boord een}{koelkamer hebben}\\

\haiku{Vroeger gaf mama,}{minder om zulke dingen}{maar sinds ze signor}\\

\haiku{Waarop Jeffries hem.}{van den tooverdrank met opium}{er in vertelde}\\

\haiku{{\textquoteleft}Voor den donder, neem,!}{me dan toch ook zoo zonder}{m'n zeeleeuwen}\\

\haiku{allen klopten hem.}{troostend op den schouder en}{vulden zijn glas weer}\\

\subsection{Uit: Venetiaans avontuur}

\haiku{Dat wil zeggen, in,,:}{de annonce waarop hij}{zich aanmeldde stond}\\

\haiku{{\textquoteright} De jongeman ziet,.}{hem afwachtend aan of er}{nog meer zal volgen}\\

\haiku{{\textquoteleft}Ons gesprek, mijnheer....}{is als een illustratie}{bij mijn artikel}\\

\haiku{{\textquoteleft}Hoe lang nog zal de;}{machtwaan van de werkgever}{kunnen voortwoeden}\\

\haiku{Het is stilzwijgend,.}{begrepen in het maandloon}{dat wij ontvangen}\\

\haiku{{\textquoteleft}Ik zelf duid het vers,...}{nog anders dan het hier is}{aangewend mijnheer}\\

\haiku{hij staat toch nog maar,.}{aan de overkant en tussen}{hen is het leven}\\

\haiku{{\textquoteleft}Daar moet en zal een,{\textquoteright},.}{eind aan komen zegt Stefan}{Kleingeld de jongste}\\

\haiku{misschien ook is de.}{lucht vandaag bijzonder zwaar}{van geuren en groei}\\

\haiku{{\textquoteleft}Mogen we voor dit?}{alles ook zo iets als een}{verklaring horen}\\

\haiku{Als ik dan niet slaag,.}{weet ik niet hoe het nog met}{me eindigen zal}\\

\haiku{Ik heb er nu geen,{\textquoteright}.}{aardigheid meer aan antwoordt}{Pepi onhandig}\\

\haiku{ze knipogen tegen,}{de meisjes die verwonderd}{om de hoek komen}\\

\haiku{{\textquoteleft}Hij moet zaterdag,{\textquoteright}.}{toch z'n vakantie hebben}{zeggen de pakkers}\\

\haiku{Een conducteur, die,,...}{als Walther juist wegdoezelt}{het biljet wil zien}\\

\haiku{Wie meer bagage,.}{heeft dan hij dragen kan valt}{hun weerloos ten prooi}\\

\haiku{Dan ontpopt zich de;}{hotelportier echter als}{redder in de nood}\\

\haiku{thans weer schijnt hem het:}{allerbelangrijkste wat}{hij zou kunnen doen}\\

\haiku{Niemand hier in de.}{Trattoria della Rosa}{interesseert het}\\

\haiku{het is namelijk.}{net alsof hij die vroeger}{al eens gezien heeft}\\

\haiku{Misschien heeft bij het!}{morgen al niet meer nodig}{om te rekenen}\\

\haiku{Zijn taille komt er,.}{goed in uit en zijn schouders}{lijken er fors in}\\

\haiku{maar zo'n jan- en!}{allemansfoto had ik}{van u niet verwacht}\\

\haiku{{\textquoteright} Lachend vat zij haar,:}{vriendin onder de arm en}{zegt haar meetrekkend}\\

\haiku{het komt vanuit de,;}{diepte alsof het uit het}{plaveisel opsteeg}\\

\haiku{Het lokt hem weinig.}{meer om naar de piazza}{terug te keren}\\

\haiku{- Het zou een flink gat,,...}{in zijn reisbeurs slaan maar wie}{niet waagt wie niet wint}\\

\haiku{Zijn avondkleding zal,...!}{hem de moed schenken op haar}{af te gaan vervloekt}\\

\haiku{- \'als je er een neemt,,?}{zal het toch alleen maar van}{de vrouw zijn niet waar}\\

\haiku{Zeg nu maar, dat u,!}{komt dan sturen we Bill om}{op u te wachten}\\

\haiku{Ze noemen hem in,:}{volgorde hun namen en}{Walther herhaalt ze}\\

\haiku{dat zou veel te saai!}{zijn als het altijd maar op}{een kus uitdraaide}\\

\haiku{Zoals wij zeggen,.}{zo is het spel en zo is}{het altijd geweest}\\

\haiku{{\textquoteright} {\textquoteleft}Dad is vanavond niet,{\textquoteright}.}{uitgenodigd stellen de}{meisjes hem gerust}\\

\haiku{Het komt hem echter,.}{verdacht voor dat zij hem zo}{lang laten wachten}\\

\haiku{{\textquoteleft}Ik ben ook verliefd,{\textquoteright}.}{bezweert hij en rilt erbij}{over zijn ganse lijf}\\

\haiku{Hij zal er trots op,:}{zijn dat hij een dichter in}{zijn familie krijgt}\\

\haiku{En juist mij gaf je...}{geen arm toen we daar zo door}{de stad rondliepen}\\

\haiku{Zij veegt ze vluchtig,:}{weg en troost zonder er zelf}{in te geloven}\\

\haiku{Heeft hij werkelijk,?}{ernstig gemeend op een van}{hen verliefd te zijn}\\

\haiku{Wat later buitelt}{hij in de branding tussen}{een aantal bruine}\\

\haiku{Intussen vindt hij {\textquoteleft}!}{de nieuwe grap uit om op}{elke vraag metwau}\\

\haiku{{\textquoteright} Dan zet hij zich voor.}{het kleine tafeltje en}{begint te schrijven}\\

\haiku{Enige tijd zit hij.}{zonder aan iets te denken}{en drinkt zijn koffie}\\

\haiku{Hardop herhaalt hij,.}{nog weer eens dat hij immers}{niet meer terug kan}\\

\haiku{hij weer vervalt als {\textquoteleft}{\textquoteright} {\textquoteleft}{\textquoteright}.}{hij ookDear wegwist en er}{Dearest van maakt}\\

\haiku{{\textquoteright} Daarna beveelt hij.}{zich nog voor een volgende}{dans aan en eclipseert}\\

\haiku{{\textquoteright} roept Walther en klapt,.}{passieloos in de handen}{zoals allen doen}\\

\haiku{Lilo beleeft van.}{deze gestolen dans ook}{al geen genoegen}\\

\haiku{{\textquoteright} zegt het meisje in.}{een volkomen Duits en wijst}{blozend op Walther}\\

\haiku{dan moet ze zich op.}{de lippen bijten om het}{niet uit te proesten}\\

\haiku{{\textquoteright} vraagt ze als Bibi.}{zich vlak voor haar voeten in}{het zand rondwentelt}\\

\haiku{Toen stond voor een man;}{met durf en lust tot avontuur}{de wereld nog open}\\

\haiku{Die gids vertelt het...?}{daar net aan zijn horde 70}{pas in de lengte}\\

\haiku{50 maal 25 maal 15...}{meter zou een kubieke}{ruimte geven van}\\

\haiku{Zo, daarom is de:}{kerel dus al de hele}{dag achter hem aan}\\

\haiku{daar leeft ze in 't,,;}{verborgen als een kleine}{arme studente}\\

\haiku{- Hoe laat arriveert,?}{de postkoets uit D. als ik}{u verzoeken mag}\\

\haiku{{\textquoteleft}Ik zal me op de.}{een of andere wijze}{wel zien te redden}\\

\haiku{hij wendt zich naar haar.}{om en doet een poging om}{haar toe te lachen}\\

\haiku{zij is moe en wil):}{ook nog een brief schrijven vraagt}{zij hem onverwachts}\\

\haiku{{\textquoteright} {\textquoteleft}Maar wat drommel, het,!}{gaat er toch niet om of er}{op hen gepast wordt}\\

\haiku{hij zelf denkt er niet.}{aan om zich voor Bibi in}{het touw te spannen}\\

\haiku{Hij is zo lief, kijk,;}{dan en hij wil zo graag mooie}{schilderijen zien}\\

\haiku{{\textquoteleft}U weet toch wel ma,...}{dat ik niet veel geef om al}{die oorlogsdingen}\\

\haiku{Als dit niet zo zou,...}{zijn ware het in hoge}{mate opwindend}\\

\haiku{{\textquoteleft}Maar natuurlijk, zelfs...{\textquoteright}}{dan zou u in 1918 nog maar}{een kind zijn geweest}\\

\haiku{Als zij samen bij,:}{ma terugkeren zijn de}{rollen verwisseld}\\

\haiku{Zij doet het aardig,,.}{en kinderlijk zo dat het}{eerder vleit dan kwetst}\\

\haiku{... zet morgen iets op...}{haar hoofd wat nog nooit op een}{hoofd gezeten heeft}\\

\haiku{{\textquoteright} Ma en Miep kijken;}{naar de diva en trachten}{dit te verwerken}\\

\haiku{Let op en schreeuw niet, -.}{ik ga je in je eigen}{belang opereren}\\

\haiku{De anderen zijn,{\textquoteright}.}{naar de Lido gegaan deelt}{Walther hem mede}\\

\haiku{{\textquoteleft}Als jij er me brengt, ',{\textquoteright}.}{zal ik wel zeggen of het}{t goeie is zegt hij}\\

\haiku{Het is onbillijk.}{om daar dan toch een oordeel}{over te verlangen}\\

\haiku{Zou je werkelijk,?}{durven volhouden dat dat}{alles je niets zegt}\\

\haiku{Zij houdt ermee op,;}{en zegt dat zij te vermoeid}{is om te dansen}\\

\haiku{Nu komt Doug als een;}{vluchtige donkere schim}{uit de deuropening}\\

\haiku{Hij trekt zijn zakdoek.}{te voorschijn en veegt er zich}{het voorhoofd mee schoon}\\

\haiku{hij had helemaal.}{niet anders kunnen springen}{dan hij gedaan heeft}\\

\haiku{hij moet zich haasten,.}{daar zij zich reeds vlak onder}{de wal bevinden}\\

\haiku{Bitter ontnuchterd,;}{wacht hij af wat er verder}{nog gebeuren zal}\\

\haiku{Als hij zijn ogen weer,;}{opent valt hem op hoe duister}{het geworden is}\\

\haiku{Marcolina ligt.}{met het bovenlijf over de}{tafel te snikken}\\

\haiku{Voor Marcolina,;}{was het te hopen dat dit}{maar gauw gebeurde}\\

\haiku{daglicht naar zee doen,.}{afdrijven en de vissen}{zorgden voor de rest}\\

\haiku{alsof hij gedoemd,}{is deze handen aan zijn}{lijf mee te dragen}\\

\haiku{Zwijgt hij tegenover,.}{de waard dan heeft hij hier nog}{twee dagen krediet}\\

\haiku{Vannacht moet ze als!}{een kwade meid onder mijn}{dak zijn weggevlucht}\\

\haiku{Ze bevallen hem.}{geen van beiden nu hij ze}{in het gelaat ziet}\\

\haiku{Hij verlangt naar ogen,;}{die in \'e\'en blik doorzien wat}{hij geleden heeft}\\

\haiku{Hij schijnt te denken,!}{dat hij hier voor zijn plezier}{op de uitkijk staat}\\

\haiku{Hij heeft nu genoeg.}{van dit plein en van deze}{mensen om hem heen}\\

\haiku{Kan hij geen stuk van?}{zich zelf verkopen om bij}{Peggy te komen}\\

\haiku{Dat zou mooi zijn als.}{de meneer hem zelf in het}{mandje kon leggen}\\

\haiku{Het hoofd natuurlijk.}{weer in diep gepeinzen naar}{de grond gebogen}\\

\haiku{En jij zult lachen,,.}{om dingen die mij hebben}{doen schreien Peggy}\\

\section{Louis Ferron}

\subsection{Uit: Turkenvespers}

\haiku{{\textquoteright} {\textquoteleft}De stoplichten niet,{\textquoteright}.}{meegerekend probeerde}{ik hem te troosten}\\

\haiku{Heb je eigenlijk?}{wel een idee waarom juffrouw}{Kamenow kaarsen brandt}\\

\haiku{Hun braaksel was zo.}{groen als de wouden die de}{akkers omzoomden}\\

\haiku{Hij legde zijn hand.}{op haar knie en knikte haar}{bemoedigend toe}\\

\haiku{Zes vrouwen had hij,.}{ge\"exploiteerd vijf had hij}{er doodgeranseld}\\

\haiku{Ik dacht, welke gek,.}{heeft mij eens verwekt daar in}{de Dorotheeengasse}\\

\haiku{Ze waren arm maar.}{jong en dat leek een reden}{om vrolijk te zijn}\\

\haiku{Maar omdat ik toen,.}{nog niet wist wat een sfinx was}{vergat ik het weer}\\

\haiku{Hammen vlogen uit,,.}{het keukenraam vergieten}{deksels en pannen}\\

\haiku{Het bloed spoot uit zijn,}{oren zijn neus en zijn mond en}{nog vraag ik me af}\\

\haiku{Ook mijn stiefvader.}{werd op de grond gesmeten}{en vastgebonden}\\

\haiku{Iemand vloekte toen.}{hij met zijn voet achter een}{wortel bleef haken}\\

\haiku{Ik had alleen nog.}{maar de nacht en de sterren}{om naar te kijken}\\

\haiku{Waarom moest ik juist,,?}{toen piekerend in die boom}{aan die film denken}\\

\haiku{{\textquoteright} De man blies een pluis.}{van zijn mouw en nam me nog}{eens aandachtig op}\\

\haiku{Nee, het was echt geen.}{inbeelding dat ik overal}{verval bespeurde}\\

\haiku{{\textquoteright} {\textquoteleft}Voor een kunstenaar,.}{hebt u wel een zeldzaam hard}{gemoed jonge vriend}\\

\haiku{Wat deugt er niet aan,?}{mij dat mensen van vlees en}{bloed mij ontwijken}\\

\haiku{{\textquoteright} {\textquoteleft}Ach,{\textquoteright} zei ik, {\textquoteleft}het zijn.}{de besten niet die op het}{uiterlijk afgaan}\\

\haiku{{\textquoteleft}Let op, er komt een.}{dag dat ze niet langer om}{me zullen lachen}\\

\haiku{Ik leerde de geur.}{van patchouli onderscheiden}{van die van muskus}\\

\haiku{Ik dacht, vliegend fort,.}{of Lancaster dat zou er}{een mooie naam voor zijn}\\

\haiku{Ik kon de vlieger.}{duidelijk zien zitten in}{zijn glazen koepel}\\

\haiku{{\textquoteleft}Alma,{\textquoteright} zei ik de,}{volgende dag toen Korngold}{naar de opera was}\\

\haiku{Eens zou een minnaar.}{haar vereeuwigen in een}{onthutsend portret}\\

\haiku{Ze sloeg haar armen,.}{om mijn middel haar handen}{streelden mijn liezen}\\

\haiku{{\textquoteright} Of, {\textquoteleft}jammer genoeg.}{is de techniek nog niet ver}{genoeg gevorderd}\\

\haiku{Als ik hoestte, klonk.}{mijn hoest pas als ik weer op}{adem was gekomen}\\

\haiku{Verliefd wordt men op.}{vrouwen wier voorkomen men}{niet beschrijven kan}\\

\haiku{{\textquoteleft}Dit had ik kunnen,{\textquoteright},.}{weten zei de stem die de}{stem van Korngold was}\\

\haiku{Een Zigeunerkind.}{was ik en het zou nog slecht}{met me aflopen}\\

\haiku{En ik vroeg me af.}{welke vorm die gedachten}{konden aannemen}\\

\haiku{Die jongeman joeg.}{zijn centen er al net zo}{vlot door als zijn neef}\\

\haiku{Wat heeft nog waarde,?}{in deze tijden wat mag}{nog een naam hebben}\\

\haiku{Tot mijn verbazing.}{rook zijn adem niet naar drank maar}{naar inkt en papier}\\

\haiku{Wat zouden de  ?}{meest geheime wensen van}{deze Freiherr zijn}\\

\haiku{{\textquoteleft}Goed,{\textquoteright} vervolgde Sayn, {\textquoteleft}.}{dan zult u het ook met me}{eens zijn dat \ensuremath{\sum}ipij=1}\\

\haiku{{\textquoteright} {\textquoteleft}O, zeker,{\textquoteright} viel ik, {\textquoteleft},.}{Sayn bijheel onschuldig geen}{onvertogen woord}\\

\haiku{wat liefde was, dan.}{zouden ze zich toch ook}{niet zo vergooien}\\

\haiku{{\textquoteright} {\textquoteleft}O, je houdt je van,,}{den domme Kaspar je weet}{het allemaal best.}\\

\haiku{Zo werkt dat systeem.}{dat iedereen op de plaats}{houdt waar hij thuis hoort}\\

\haiku{Ik had me al te.}{zeer laten meeslepen door}{mijn vooroordelen}\\

\haiku{Ik houd zo van je,,.}{Kaspar waarom wil je dat}{toch niet begrijpen}\\

\haiku{Daarna Wein, Weib und,,.}{Gesang Donauwellen het}{daverde maar door}\\

\haiku{{\textquoteright} {\textquoteleft}Zo wil de keizer,.}{het nu eenmaal hij wil dat}{we plezier hebben}\\

\haiku{Die opmerking scheen.}{zijn vermoedens alleen maar}{te bevestigen}\\

\haiku{Wij hebben altijd.}{ons best gedaan voor hen die}{over ons zijn gesteld}\\

\haiku{{\textquoteright} Buiten bulderden.}{de kanonnen en ik zocht}{naar zwavelstokken}\\

\haiku{{\textquoteright} En mijn stem dempend, {\textquoteleft}?}{Denkt u dat we het daar nog}{mee redden zullen}\\

\haiku{{\textquoteright} {\textquoteleft}Nee, mijnheer, maar ik.}{zou het met de nagel van}{mijn duim kunnen doen}\\

\haiku{{\textquoteright} Hij spuwde op de,.}{grond gorde zijn mand weer om}{en liep van ons weg}\\

\haiku{De hongerende.}{bevolking schraapte kalk en}{gips van de muren}\\

\haiku{Ik kon daar uren over.}{peinzen op mijn kamer en}{ik moest dat ook wel}\\

\haiku{En welke artsen?}{sprongen het slordigst met de}{hygi\"ene om}\\

\haiku{{\textquoteright} {\textquoteleft}En ik heb last van.}{huidverkleuringen en mijn}{tanden vallen uit}\\

\haiku{\v{C}elinek liet een.}{onderzoek instellen dat}{niets opleverde}\\

\haiku{Een vertwijfelde.}{dokter \v{C}elinek stormde}{de jachtzaal binnen}\\

\haiku{Met het zwellen van,.}{haar buik groeide haar gezag}{in de omgeving}\\

\haiku{De hygi\"ene.}{in het hospitaal nam er}{echter niet door toe}\\

\haiku{{\textquoteright} {\textquoteleft}Als wij het hier voor,.}{het zeggen hebben krijg je}{misschien nog wel meer}\\

\haiku{{\textquoteright} {\textquoteleft}Kunst, kunst aan mijn broek,.}{niets dan geiligheid waar jij}{mee loopt te leuren}\\

\haiku{Met juffrouw Kamenow.}{had ik nog nooit iets gehad}{en zij was zwanger}\\

\haiku{{\textquoteleft}Hij ademt,{\textquoteright} zei ze op, {\textquoteleft},.}{haar buik kloppendhij trappelt}{hij roept zijn moeder}\\

\haiku{{\textquoteleft}Je moet hem met U,.}{aanspreken hij is niet de}{eerste de beste}\\

\haiku{Dat was de Habsburgse.}{horoscopie waaraan ik}{onderworpen was}\\

\haiku{Vol afschuw stootte.}{ik hem van me af en hij}{viel zwaar op de grond}\\

\haiku{wie doorkneed was in.}{de anti-logica die}{hier ontwikkeld werd}\\

\haiku{Eynhuf had niet het.}{flauwste vermoeden van wat}{er in mij omging}\\

\haiku{Mackensen nam op.}{en begon een lang gesprek}{in een vreemde taal}\\

\haiku{De situatie.}{was omgeklapt en Kunz was}{zichzelf gebleven}\\

\haiku{Hij is op hun hand.}{en bazelt daarom maar wat}{over negermuziek}\\

\haiku{{\textquoteright} {\textquoteleft}Ach, excellentie,...}{als ik u dat allemaal}{moet gaan uitleggen}\\

\haiku{En wat hadden Kunz?}{er mee te maken en de}{Edler von Eynhuf}\\

\haiku{{\textquoteright} {\textquoteleft}Jou staat op zijn best,{\textquoteright}.}{de dood door verzakking te}{wachten sneerde ik}\\

\haiku{De zwart gelakte.}{marskramersmand op zijn rug}{hinderde hem zeer}\\

\haiku{Daarna moesten we in.}{groepjes van twee de lijken}{de kerk in dragen}\\

\haiku{De driften komen,{\textquoteright}, {\textquoteleft}.}{zoals ze gaan zei hijals}{een dief in de nacht}\\

\haiku{De oude knikte,.}{goedkeurend maar ik deed of}{ik het niet merkte}\\

\haiku{Zoon van een prins van{\textquoteright}.}{Baden schreven de boeken}{over mijn naamgever}\\

\haiku{Hij had zijn leven.}{in handen gelegd van de}{verdoemde machten}\\

\haiku{Het paradijs kon,.}{alleen werkers gebruiken}{geen pati\"enten}\\

\haiku{Ik dacht, in dit kind.}{moet alles gebeuren wat}{mij onthouden is}\\

\haiku{Dat liet hij, zei hij,.}{aan anderen over die daar}{in geschoold waren}\\

\haiku{{\textquoteright} Ik kuste het kind.}{op de fontanel en viel}{in een diepe slaap}\\

\haiku{Als men op zijn hoofd.}{ging staan kon men zien hoe de}{prins naar zijn hart greep}\\

\haiku{als thuis en ik zie.}{waarachtig niet in wat daar}{om te lachen valt}\\

\haiku{Tussen de spijlen.}{van de kroon stonden mensen}{met verrekijkers}\\

\haiku{Met zijn vlakke hand.}{klopte de man het ritme}{op de tafel mee}\\

\haiku{De man die Kaspar.}{heette knikte en zei dat}{ze voort moesten maken}\\

\section{Juul Filliaert}

\subsection{Uit: Jan Bart}

\haiku{Zijn hoog achterdek,,.}{dat hem in den rug beschermt}{wordt weggeschoten}\\

\haiku{Eens binnenloopen ',,,.}{int Boomstraatje ja ja}{dat zou Roosje wel}\\

\haiku{Keesje loopt aan.}{zijn hand en Jantje zit op}{den arm van moeder}\\

\haiku{van dekgevechten;}{met bijlen en pistolen en}{zwaarden en messen}\\

\haiku{De plunjezak wordt,.}{toegestropt in den hoek van}{de kamer geplaatst}\\

\haiku{Jan ziet moeder, van,.}{op den deurdorpel met de}{\'e\'ene hand wuiven}\\

\haiku{Iedere prooi was,.}{hem welkom als hij ze maar}{bemeesteren kon}\\

\haiku{Dit derde deel werd.}{in een aantal gelijke}{bedragen gesplitst}\\

\haiku{Een jaartje oorlog.}{tegen den Engelschman}{kan hem geen kwaad doen}\\

\haiku{- Waarom niet, antwoordt -.}{Jan Top. De overvaart eischt}{kunde en kennis}\\

\haiku{Wij ondervinden,,.}{dat allen best. Uw vader}{ondervond dat ook}\\

\haiku{Geen wilde vaart, want.}{het weder is helder en}{de wind is slapjes}\\

\haiku{Het smaldeel dat aan,.}{bakboord afzwaaide stevent}{naar de Midwaybocht}\\

\haiku{in zijn plaats had ik '.}{zoowel de Lords alst grauw te}{roosteren gelegd}\\

\haiku{Hij bezit ook een,.}{vierde aandeel in een mooie}{boot de Sint Michiel}\\

\haiku{Jan heeft het kleine.}{bootje dadelijk in de}{gaten gekregen}\\

\haiku{Het laken mag toch?}{niet heelemaal langs eenen kant}{getrokken worden}\\

\haiku{Als we persoonlijk,.}{ons zelf blijven dan gaat niet}{alles verloren}\\

\haiku{Langs den neus en op,.}{den rug van anderen haalt}{hij zijn oogst binnen}\\

\haiku{Jan Bart krijgt deze.}{gevaarlijke opdrachtjes}{voor zijn rekening}\\

\haiku{Hij was daarbij heel.}{anders dan de gewone}{zeeman aangelegd}\\

\haiku{Wat verdienste kon?}{er ook gemaakt worden aan}{boord van zoo'n prulding}\\

\haiku{Deze wendt het hoofd,.}{om doet de manschap teeken}{nog wat te wachten}\\

\haiku{begint er daar reeds '.}{een aant einde van de}{tafel te zingen}\\

\haiku{Waardin Martien Van,,.}{den Broucke vrouw Gouthiere}{glanst van voldaanheid}\\

\haiku{De wezens gloeien.}{hoogrood en glimmen boven}{de blauwe baais uit}\\

\haiku{In koortsachtige.}{gejaagdheid worden alle}{zeilen geheschen}\\

\haiku{Een lading Spaansche,.}{wijn die op een Engelsche}{smack werd veroverd}\\

\haiku{Wanneer ze, bijna,.}{boord aan boord zijn schiet Jan een}{laatste salvo af}\\

\haiku{Lassijn staat aan 't,,,.}{roer om met zijn buit Koning}{David te volgen}\\

\haiku{In het dekgevecht,,.}{dat een uur aanduurt weert}{hij zich als een leeuw}\\

\haiku{Daar waren nog de.}{bezoeken op de Groenplaats}{en in de Boomstraat}\\

\haiku{Haast geen tijd om met.}{Nicole een luchtje in}{de stad te scheppen}\\

\haiku{Met de roeibooten '.}{komen de schippers aan boord}{vant kaperschip}\\

\haiku{Zoodoende halen,.}{ze de geleden schade}{wellicht ruimschoots in}\\

\haiku{De killigheid van.}{de zieltjesweek druilt reeds over}{pleinen en haven}\\

\haiku{Wakte en natte '.}{dringen langs zijn broekbeurzen}{tot opt bloote vel}\\

\haiku{Mannen, in Godsnaam, '.}{t Laatste dat ge voor uw}{kapitein kunt doen}\\

\haiku{- Ik voel me alleen,,.}{gelukkig tevreden en}{voldaan Nicole}\\

\haiku{- Als men verloofd is,.}{dan stapt men fier en trotsch aan}{den arm van den man}\\

\haiku{- Maar, gij moet dan ook!}{begrijpen dat uw vrouw toch}{meer is dan uw schip}\\

\haiku{Zijn samenwerking.}{met de kapiteinmaats is}{winstgevend geweest}\\

\haiku{Er is pinkeling,.}{van licht dat wemelt in de}{plooien van de zee}\\

\haiku{De Neptunus is.}{ook al omgezwenkt en lost}{een tweede slavo}\\

\haiku{De twee mannen staan.}{tegenover elkander als}{razende honden}\\

\haiku{Ze gelijken twee,.}{kampers die een bloedige}{veete beslechten}\\

\haiku{- Stand houden, gilt en,, '.}{hijgt Cuyper die stuiptrekkend}{wentelt opt dek}\\

\haiku{Zij, immers, moeten,.}{vaart zee en oorlog niet meer}{leeren of leeren kennen}\\

\haiku{Ik heb ze van over.}{de tafel gehaald om haar}{een zoen te geven}\\

\haiku{En ten slotte nog,;}{twee en dertig zakken geld}{250 stuks gouden munt}\\

\haiku{De andere zijn?}{dan ook bij den kapitein}{terecht gekomen}\\

\haiku{Daar was wel goud op,.}{De Pelikaan maar toch geen}{zulke geschenken}\\

\haiku{Inmiddels wordt zijn.}{eerste klas fregat Mats in}{gereedheid gebracht}\\

\haiku{Wat zegt u, van zoo'n,,?}{kapitein als Bart meneer}{de havenbaljuw}\\

\haiku{de taal in eere.}{en aanzien te houden en}{te doen voortleven}\\

\haiku{Hij voelde zich een.}{soort hoofdman zonder bezit}{en zonder werkveld}\\

\haiku{Vauban,  trouwens,.}{bevond zich een beetje in}{het geval van Bart}\\

\haiku{Reeder spelen in,,.}{de politiek kan nooit geen}{kwaad vooral nu niet}\\

\haiku{Hij ziet toch wel, dat '?}{Duinkerke de stapelplaats}{vant Noorden wordt}\\

\haiku{Jammer dat Jan het,!}{vertikt baas en kapitein}{aan wal te spelen}\\

\haiku{Jan verdient het, want.}{hij is de uitblinker in}{de bloedverwantschap}\\

\haiku{Onder de kristen.}{slaven zijn verschillende}{stamgenooten}\\

\haiku{De nagalm zindert.}{weg in een vreugdetrilling}{of in smartgevoel}\\

\haiku{In de woonkamer.}{had hij de meesters angstig}{om bescheid verzocht}\\

\haiku{Dood had hij kunnen,.}{zaaien dood overwinnen lag}{buiten zijn bereik}\\

\haiku{zeemansrantsoen op,.}{zijn Vlaamsch dertig centimes}{per dag en per kop}\\

\haiku{In het groote huis op.}{de Groenplaats wil en kan hij}{niet meer verblijven}\\

\haiku{hij heeft oog en oor.}{open voor alle verlangens}{en alle wenschen}\\

\haiku{Twee dagen daarna,,.}{brengt Jan te Duinkerke zijn}{eersten buit binnen}\\

\haiku{Eigenlijke storm,.}{is  het nog niet wel de}{voorbode ervan}\\

\haiku{Bang is hij niet, maar.}{zoo'n wiegedans heeft hij nog}{niet medegemaakt}\\

\haiku{Met 't klaren van.}{den morgen verspringt de wind}{in een ander gat}\\

\haiku{Aan den grooten mast.}{laat Jan het signaal van den}{terugtocht hijschen}\\

\haiku{Patoulet had de.}{zaken op grootscheepsche}{wijze aangepakt}\\

\haiku{Maria liep toen, als,.}{driejarig kindje nog in}{haar eerste rokske}\\

\haiku{Een alledaagsch en.}{doodgewoon voorval in het}{boordleven aan wal}\\

\haiku{Van nu voortaan werd.}{ze losser en vrijer in}{haar doen en laten}\\

\haiku{men wandelt niet steeds,.}{met zijn heele zeemanshart}{de kaai op en af}\\

\haiku{Waarom was uw blik,,,?...}{z\'oo z\'oo droomerig-koel}{zoo hard dezen avond}\\

\haiku{De strooming van '.}{t water helpt den roeier}{in zijn zware taak}\\

\haiku{- Het vel van den beer.}{niet verkoopen vooraleer}{aan land te wezen}\\

\haiku{De boorden van zijn.}{hemd plooit hij tot boven de}{ellebogen op}\\

\haiku{Een oogenblik taakt,.}{ze het watervlak duikelt}{dan weg in de zee}\\

\haiku{Indien ik hier aan ',.}{t stuur moet blijven zitten}{dan val ik in slaap}\\

\haiku{Beweging zal me,.}{niet alleen wakker houden}{maar tevens deugd doen}\\

\haiku{Ze drinken rhum om,.}{zich op te monteren om}{wakker te blijven}\\

\haiku{Ze moeten wakker,.}{blijven willen ze in hun}{poging gelukken}\\

\haiku{Zestien dagen lang.}{bijna ter plaatse blijven}{draaien en laveeren}\\

\haiku{{\textquoteleft}Herinner U wie,.}{de oude Tromp De Ruyter}{en Duquesne waren}\\

\haiku{Waarom negeeren ze?}{ons werk en dwarsboomen ze}{onze inzichten}\\

\haiku{De drank heeft hem op,.}{dezen tocht ten slotte er}{onder gekregen}\\

\haiku{Maar dit ongeduld.}{van den minister werkt me}{op de zenuwen}\\

\haiku{, vliegt naar de kaai en.}{doet zijn schepen tusschen de}{staketsels meeren}\\

\haiku{In de bureelen.}{van von Heine is het een}{standje van belahg}\\

\haiku{Hij toont zich trouwens.}{waardig van het vertrouwen}{dat men in hem stelt}\\

\haiku{Maar Jan krijgt ook een {\textquoteleft}{\textquoteright},:}{plunderkommissaris aan}{boord die hem dwars zit}\\

\haiku{Ze gedragen zich,.}{echter niet als prinsen wel}{als aftroggelaars}\\

\haiku{Samuel Bernard... -.}{worstelt tegen stroom op Niet}{te verwonderen}\\

\haiku{Alleen de genster '.}{is nog noodig omt vuur aan}{de lont te brengen}\\

\haiku{- Een flinke grog, vrouw,.}{om dat beestje van onder}{mijn huid te jagen}\\

\haiku{Wanneer hij 't hoofd,.}{wil oprichten krijgt hij als}{een slag in den nek}\\

\haiku{Hij hoest nu en dan,.}{hard en droog en koud zweet klamt}{onder zijn oksels}\\

\haiku{Maar desondanks blijft,.}{zijn denken helder nu is}{zelfs merkbaar verscherpt}\\

\haiku{Achter het schip stroelt.}{en schuimt het witte kielzog}{van de laatste vaart}\\

\haiku{de Fransche werken,,;}{komt Jan Bart als zeeheld op}{het voorplan te staan}\\

\haiku{Wat men over Jan Bart,.}{wilde behouden werd op}{het voetstuk geplaatst}\\

\haiku{Als hij stierf was dit.}{stuk Vlaamsch land reeds vijftig jaar}{onder Fransch beheer}\\

\haiku{De bindselriemen.}{zitten in de roeimikken}{om houtsleet te sparen}\\

\haiku{In een stroommonding,.}{ook driepikkels waar men in}{nood aanleggen kan}\\

\haiku{Iemand die op zijn:}{stoel zit te wrikkellen van}{ongeduld enz. Zootje}\\

\subsection{Uit: Tijl's oog op den puinhoop}

\haiku{In den bak gedraaid,,.}{worden om de waarheid te}{zeggen d\`at kan niet}\\

\haiku{Mijn gebuur noemt dat,.}{z\'o\'o maar een eigenlijke}{spelonk is het niet}\\

\haiku{Dit stuk muur, eindigt, '.}{afgerond int profiel}{van een menschenhoofd}\\

\haiku{De zon schittert er,.}{door den heelen dag brandt er}{door bij avondzinken}\\

\haiku{Waar kunnen we nog!}{beter zijn   Dan in ons}{moeders keuken}\\

\haiku{Vooraleer wij de,;}{stad zouden binnentrekken}{bleef de voerman staan}\\

\haiku{We zullen spoedig.}{gaan zien indien ze nog niet}{ingenomen zijn}\\

\haiku{{\textquoteright} - {\textquoteleft}Maar we huizen nu, {\textquotedblleft}{\textquotedblright} '.}{wel in een spelonk in een}{abri ent gaat ook}\\

\haiku{Wat het gemis aan!}{barakken wel duizendmaal}{vergoelijken kon}\\

\haiku{Ik vraag een barak,.}{eerst en meest omwille van}{de kleine dutsen}\\

\haiku{Moet ik u zeggen?}{dat wij als van de hand Gods}{waren geslagen}\\

\haiku{, daar is geen nood voor,.}{Nele de menschen zouden}{het toch niet gelooven}\\

\haiku{Toch nooit een koffer,?}{of zoo iets waarin een schat}{kon verborgen zijn}\\

\haiku{Had mijn spade dien,!}{kop afgestekt wie weet wat}{er zou gebeurd zijn}\\

\haiku{Een onder hen, was,,}{zoo goed toch eens mede te}{komen om te zien}\\

\haiku{De kinders mochten.}{niet buiten piepen of we}{moesten hen achterna}\\

\haiku{De obussen hadden.}{die speelzieke kinderen}{als gefascineerd}\\

\haiku{In klas vernam de. '}{meester de oorzaak van het}{te laat-komen}\\

\haiku{Die blijven slechts een.}{oogenblik truntelen en}{toeteren verder}\\

\haiku{N\`u voelen wij ook!}{den polsslag van het leven}{door de wereld gaan}\\

\haiku{Als hij mij gewaar,.}{wordt schudt hij de veeren en}{wipt de ruimte in}\\

\haiku{De zucht naar roem en;}{naar macht leidt naar verdrukking}{en onderdrukking}\\

\haiku{Zijn glinsterende.}{oogen pinkelen achter de}{glazen van zijn bril}\\

\haiku{verdwijnt dan ergens,.}{achter een stuk muur als in}{een grondeloozen put}\\

\haiku{Dat ondervonden.}{wij weldra persoonlijk met}{dat ander konijn}\\

\haiku{we weten het al,.}{dertig jaar dat een gendarm}{geen genade kent}\\

\haiku{Ze stond er nog niet,.}{half of men ondervond dat}{het misloopen was}\\

\haiku{In den ondergang:}{van een heel gewest had ze}{haar roem gehandhaafd}\\

\haiku{Want opruiming en:}{heropbouw moeten hierheen}{veel vreemd volk lokken}\\

\haiku{Dit Fransch kerkhof is.}{een begankenisplaats van}{belang geworden}\\

\haiku{Het oorlogsgeweld.}{kwam uit het Noord-Oosten}{en uit het Oosten}\\

\haiku{Het eenige dat in,.}{deze stervende stad niet}{stierf was het kerkhof}\\

\haiku{Dat wil echter niet.}{zeggen dat wij den winter}{niet hebben gevoeld}\\

\haiku{Het geluk kwam ons.}{te gemoet en we zijn het}{niet voorbijgegaan}\\

\haiku{Ons onbekommerd,.}{vrij en los bestaan heeft een}{einde genomen}\\

\haiku{Er heerschte, al,:}{dien tijd in heel de stad slechts}{\'e\'en bekommering}\\

\haiku{Het wordt te allen.}{kant een gewedijver om}{ter eerst en ter meest}\\

\haiku{niest onder  de.}{geweldige prikkeling}{en hervat de taak}\\

\haiku{{\textquoteleft}Een naaste maal, tracht!}{dat karweitje te schikken}{buiten de werkuren}\\

\haiku{Pas had ik de deur:}{geopend of Nele was}{er al met de vraag}\\

\section{Emiel Fleerackers}

\subsection{Uit: Baveloo-Boetjes}

\haiku{{\textquoteright} zei Baveloo, {\textquoteleft}dat '...,?}{s telepathie Zijn mijn}{schoenen gelapt Boetjes}\\

\haiku{en een van hen, de,...}{bugel spitst zijn ooren en}{verneemt het geheim}\\

\haiku{Boetjes zat en zocht een,, -.}{derde woord ten minste een}{tweede maar hij zweeg}\\

\haiku{of ten minste naar, '!}{een tweede woordt woord van}{de redelijkheid}\\

\haiku{Maar zij had al meer.}{gekwakt dan een gewone}{kwakkel op zes jaar}\\

\haiku{sinds die kwakkel bij,.}{u hangt werkt dat kwaksel mij}{op de zenuwen}\\

\haiku{{\textquoteright} - {\textquoteleft}Enfin, d\`at heeft hij,.}{door mijn raam gesmeten vlak}{op een schilderij}\\

\haiku{Boetjes kwam dien avond op '.}{t studeerkamertje van}{den Heer Kanunnik}\\

\haiku{{\textquoteright}... - {\textquoteleft}Ja 't is schoon{\textquoteright} - zei.}{moeder Boetjes en slikte een}{pilleke binnen}\\

\haiku{{\textquoteright} - - {\textquoteleft}Neen, Mr Pastoor{\textquoteright} - zei.}{moeder Boetjes en keek of er}{een zot in huis stond}\\

\haiku{maar hij behoorde,: - {\textquoteleft},?}{plots tot de milddadigen}{enEen sigaar Boetjes}\\

\haiku{{\textquoteright} - {\textquoteleft}Dat kan waarschijnlijk '.}{nog heel goed vant bloed zijn}{van Karel den Groote}\\

\haiku{Maar als 't op den,...}{mesthoop stond en kraaide dan}{viel het omverre}\\

\haiku{{\textquoteright} zei Boetjes, {\textquoteleft}ge zult er,...:}{misschien mee lachen maar laat}{me rechtuit zeggen}\\

\haiku{{\textquoteright} - vroeg Sander, die twee. - {\textquoteleft},!}{uren lang gezwegen hadEen}{beetje geduld zoon}\\

\haiku{{\textquoteright} -        Boetjes vertelt van '... - {\textquoteleft} ',, -!}{ne muilezelDats heel goed}{gezegd Boetjes h\'e\'el goed}\\

\haiku{{\textquoteleft}Ga weg, snauwde dan,,,.}{Geert ga weg leelijkerd en}{kijk naar uw eigen}\\

\haiku{{\textquoteleft}En kijk nu zelf maar,,!}{Po-hotje of die geit ook}{maar \'e\'en gebrek heeft}\\

\haiku{En hij begaait mijn!... '',?}{wasch nogt Kan anders nie}{meer kapot zeker}\\

\haiku{{\textquoteright} - {\textquoteleft}Maar Mijnheer Pastoor{\textquoteright}, {\textquoteleft}?}{verweet moeder Boetjeshoe durft}{ge zoo iets zeggen}\\

\haiku{- {\textquoteleft}En toch, Boetjes{\textquoteright} hernam, {\textquoteleft}, ',.}{hij toen luiden tocht moet}{hem z\'o\'o zitten Boetjes}\\

\haiku{- {\textquoteleft}Gij denkt altijd, Boetjes{\textquoteright}, {\textquoteleft}...{\textquoteright}}{verweet hijdat ik u in}{de doeken wil doen}\\

\haiku{En voor de derde,,:}{maal en de derde maal heel}{driftig snauwde Boetjes}\\

\haiku{{\textquoteright} vroeg Baveloo 's.}{anderdaags en wreef zijn twee}{handen over malkaar}\\

\haiku{Dit is de vraag, Mr,,,:}{Pastoor n\`amelijk en te}{w\'eten en aldus}\\

\haiku{Te voet, om toch maar.}{bij uw meester te zijn en}{te kunnen broeien}\\

\haiku{Begaat ge zelf een,!}{groote dommigheid dat noemt ge}{een groote slimmigheid}\\

\haiku{Maar moeder Boetjes, als ',:}{ne man nam het op voor de}{eer van haar huis en}\\

\haiku{Maar in de kerk moet,...}{ge niet knikken tegen mij}{maar tegen O.L. Heer}\\

\haiku{Spreek met moeder daar ',,,...}{ns over alleen stillekes}{zonder koleire}\\

\haiku{Ze zouden toch van,, -!}{armoe thuis komen dacht ze}{Boetjes zeker en vast}\\

\haiku{Bij de ouders staat!}{het kinderspel maar al te}{dikwijls van geen tel}\\

\haiku{Ik zie Sander daar...}{yo-yo spelen en item zoo}{Mieke yo-yo maar}\\

\haiku{en die andere,,!}{yo-yo daar van Mieke dat}{was voor u moeder}\\

\haiku{Als ge niet wordt zooals,...}{kinderen zult ge nooit in}{den hemel komen}\\

\subsection{Uit: Brieven van Nonkel Pastoor}

\haiku{n.l. dat gij erom.}{beschaamd zijt geweest dat ze}{lachten met uw naam}\\

\haiku{Nu zal ik dat heel;}{in den korte en in den}{nauwe moeten doen}\\

\haiku{ge moet 'ne mensch geen,.}{toenamen geven vooral}{niet aan Pater Adams}\\

\haiku{En mag ik u eens?}{eventjes met een dilemma}{op uwen kop tikken}\\

\haiku{Denkt ge soms dat het,:}{plezant is voor mij als heel}{de parochie zegt}\\

\haiku{ne zekere Kant,!}{beweerde dat er geen tijd}{bestaat de sloeber}\\

\haiku{Wat zouden wij in ',?...}{s Hemelsnaam gaan doen als}{we geen tijd hadden}\\

\haiku{'t Slimste dat een}{gewoon mensch al doet met die}{twee instrumenten}\\

\haiku{een nagel trekken,.}{met de trektang dien hij met}{den hamer scheef sloeg}\\

\haiku{Een broek tusschen-in,, -.}{half-weg knoesel en}{knie dat is geen broek}\\

\haiku{maar Bertje, daar zijn,.}{dingen in de wereld die}{zoo simpel niet zijn}\\

\haiku{en hij verdient de...{\textquoteright} ' ';}{straf Ent schoonste vant}{geval is dit nog}\\

\haiku{en ge weet niet waar.}{het op aanloopt noch van waar}{het gekomen is}\\

\haiku{Is de huidige?}{jeugd van den dag van vandaag}{z\'o\'o teergevoelig}\\

\haiku{de jury stond er,.}{naast om te tellen en de}{parochie rondom}\\

\haiku{Na de 9e telloor;}{viel Tistje voorover met zijn}{gezicht in zijn 10e}\\

\haiku{{\textquoteright} - Of begon hij te ',?}{weenen alsne jongen die}{Nepos niet vertaald krijgt}\\

\haiku{'t Evangelie van.}{de arme weduwe met}{de twee penningskes}\\

\haiku{maar zoo juist is de...:}{perekwatie van de pastors}{binnengevloeid en}\\

\haiku{Sommige menschen,;}{zijn altijd bij de laatsten}{d\'oen wat ze willen}\\

\haiku{Maar in de tweede ',...}{helft vant leste bedrijf}{daar haperde wat}\\

\haiku{{\textquoteright} - En 'k heb altijd.}{ondervonden dat hij min}{of meer gelijk heeft}\\

\haiku{en 3o) als ik zooals,!}{gij nog zestien jaar oud was}{dan deed ik het wel}\\

\haiku{Ton is de man, die;}{thuis zit met dikke schijven}{en warme voeten}\\

\haiku{want de puntjes, die,...}{ge er mij in meedeelt zijn}{nog al van belang}\\

\subsection{Uit: Kijkkast}

\haiku{t Was de vader, -.}{zelf van den kleinen doode}{Petrus Nellemans}\\

\haiku{Binkske kwam voorbij -:}{het huis en al met eens zijn}{blij gemoed viel weg}\\

\haiku{en hoe juist bijtijds,,!}{op den boord van den afgrond}{gelukkig en rap}\\

\haiku{- En 't zou, den 23n, {\textquoteleft}{\textquoteright}.}{een elf-ure-lijk zijn}{in splendoribus}\\

\haiku{Een klein klokske kroop.}{van bedeesdheid weg achter}{Sint-Gummarus}\\

\haiku{En zoo, mijn bronzen,!}{predikers vergeet nooit uw}{plicht van predikers}\\

\haiku{eens voor henzelf, eens,;}{voor elk hun Pastoor eens voor}{elk hun parochie}\\

\haiku{Tegenover hem staat.}{Graaf Haditz in kersrood van}{begeerte naar macht}\\

\haiku{het lijk van hem die,...}{was Keizer Frans wordt naar den}{grafkelder gevoerd}\\

\haiku{Dien ken ik niet...{\textquoteright} De,,.}{Ceremoniemeester een}{derde maal klopt aan}\\

\haiku{De Schout floot, en twee,.}{honden vlogen op uit het}{struiksel naar Jan toe}\\

\haiku{{\textquoteleft}Ja, moederke, er,.}{aan moet hij toch want de Wet}{heeft lange armen}\\

\haiku{Ouwe-Jobbie stond, ':}{op en de rechterhand hoog}{bovent hoofd}\\

\haiku{- zoo stonden mij de.}{haijren te berghe aan}{mijnen lijve}\\

\haiku{Waer waerdij doen?}{ick de fundamenten der}{aerde leijde}\\

\haiku{{\textquoteright} - ~ Ouwe-Jobbie {\textquoteleft}{\textquoteright},...}{begon tescharren met de}{lakens te spelen}\\

\haiku{{\textquoteright} - Fiks viel opeens de, ',.}{Duitscher sloeg de rechterhand}{aant hoofd groette}\\

\haiku{{\textquoteright} zei Maarten, nu zelf.}{zoo verbaasd dat hij al Fransch}{begon te spreken}\\

\haiku{Hij was versleten '.}{nu van den ouderdom en}{t ruige leven}\\

\haiku{Mijn bullekarke..., '.}{en mijn gareel ik peins dat}{s hoop en alles}\\

\haiku{Maar Toghrai zelf, al,.}{reed hij niet op Basra ging}{Basra inhalen}\\

\haiku{- Basra vloog, - het vloog,,!}{als een pijl als een pijl als}{een snorrende pijl}\\

\haiku{En de Kadi liet,.}{hem nazoeken want hij had}{het zoo gezworen}\\

\haiku{Iedereen wilde,,.}{kost wat kost het dreigende}{gevaar afweren}\\

\haiku{E\'en oogenblik, \'e\'en,...! '}{h\'e\'el kort oogenblikje is}{er niets gebeurd niets}\\

\haiku{Wel, nu ik het ding,,, ', '.}{bepeins Pater ge zijt zelf}{int kleinne Paus}\\

\haiku{En morgen vroeg zal '.}{t weer uitvallen dat ik}{Schamel Binkske ben}\\

\haiku{(en hij snikte nu)!}{en de tranen droppelden}{langs zijn baard Binkske}\\

\haiku{De Pater knikte {\textquoteleft}{\textquoteright}:}{zoo iets vangemeenschap der}{heiligen en vroeg}\\

\haiku{t Kruisbeeld dat de.}{Heer den kleine bewaren}{en geleiden zou}\\

\haiku{Mijnheer Gerardus..........}{X   X straatje   X}{Belgium}\\

\haiku{Wat moest hij nog de!}{liefelijke schoonte leeren}{van de heiligheid}\\

\haiku{{\textquoteleft}En ze bekeken,, '...}{malkander Lieven halfdood}{ent spook heel kalm}\\

\haiku{Waar ergens groeit het?}{mastenhout zoo wild en knoest}{als in de Kempen}\\

\haiku{{\textquoteright} lachte de Keizer,:}{en hij keerde zich tot den}{stijven gouden heer}\\

\haiku{Al dadelijk moest.}{Patsken opgekleed voor zijn}{nieuwe bediening}\\

\haiku{en de \'e\'ene te,,;}{groot en de andere te}{klein en elk wist iets}\\

\haiku{{\textquoteright} - - {\textquoteleft}Daar is 'ne schout op '...}{elk dorp enne pastor op}{elke parochie}\\

\haiku{De Nederlanden.}{met hun nijverheid laat ik}{over aan Snotje Vandom}\\

\haiku{{\textquoteright} En de Keizer was:}{verstandig genoeg om de}{slotsom te trekken}\\

\haiku{Eerst spreken, aap, of!}{ik tast met mijn roeike naar}{uw ribbekast}\\

\haiku{Vanplan, met mijnen {\textquoteleft}{\textquoteright} -;}{van gij zijt de slimste van}{heel mijn kazerne}\\

\haiku{en al wat schoon is}{en lief en heilig en al}{wat een ideaal draagt}\\

\subsection{Uit: Kronijken}

\haiku{en al even stijf, langs,.}{zijn lange gezicht hingen}{lange snorpinnen}\\

\haiku{En hij deed den arts.}{en den generaal teeken}{dat ze heen mochten}\\

\haiku{{\textquoteleft}'t Es ol fiertig'}{jaor da me moeder den}{trottoir bij Menhier}\\

\haiku{eten we den baljuw, ' '!}{op ens morgens hebben}{wene zwaren kop}\\

\haiku{En Bernwald smeekte:}{opnieuw met vrees in de stem}{en op het gelaat}\\

\haiku{{\textquoteleft}Zalig zijn ze die...{\textquoteright} - {\textquoteleft}?}{Wat mogen die woorden te}{bedieden hebben}\\

\haiku{Dat we gebocheld,?}{zijn als we recht-op staan in}{den dienst des Heeren}\\

\haiku{en altijd zoo juist,, '.}{juist precies hairjuist nevens}{nen echten vloek af}\\

\haiku{De Pastor had hem:}{bij zijn vertrek zoo de les}{gespeld en gezegd}\\

\haiku{bewaar nu toch, in,;}{Gods naam uw ziele zuiver}{zoo lang ge maar kunt}\\

\haiku{de duivel hitste...}{de dazen al woedender}{tegen Maarten op}\\

\haiku{{\textquoteright} sprak de Heilige... {\textquoteleft} '!}{Opt oogenblik dat de}{donder u doodsloeg}\\

\haiku{maar een kind, dat men,!}{overlaat aan zijn wil doet zijn}{moeder schande aan}\\

\haiku{{\textquoteright} - {\textquoteleft}Zie, Mijnheer Pastoor,...}{stel geen vragen en ge zult}{geen leugens hooren}\\

\haiku{{\textquoteleft}En de ouders, die,!}{den Bode lezen mogen}{er een les in leeren}\\

\haiku{een paraplu is,...}{toch maar een vod een vod met}{een stokjen in}\\

\haiku{Toen knikte Pater, -...}{Rector tot afscheid ging heen}{en liet mij alleen}\\

\haiku{dat is nog iets, wat '...}{ik hun meen te vragen in}{t leste oordeel}\\

\haiku{en 'k weet niet uit '!}{wat voor een wolkt ding op}{me neerdonderde}\\

\haiku{En ik vertel hem ';}{int kort wat ik u zelf}{vandaag vertelde}\\

\haiku{gij, voed me met uw,!}{stillen vrede Beata}{Soror Paupertas}\\

\haiku{de legende nl..}{die gaat over den oorsprong van}{den Konijnenberg}\\

\haiku{Maar haring - haring, -.}{simpelweg zonder pekel}{is sympathieker}\\

\haiku{Soms vraagt wel eens een,:}{ouwe mensch die zijn eeuw niet}{meer kan bijblijven}\\

\haiku{Daarbij, 't heele... - {\textquoteleft}:}{Itinerarium krijgt ge}{nietAntifone}\\

\haiku{En ze praatten als,...}{broers en zusters gezellig}{en gemoedelijk}\\

\haiku{Maar Haroen bezag.}{het al met sombere oogen}{en hangende lip}\\

\haiku{'t Is een domme,.}{visch dien ze tweemaal vangen}{met denzelfden pier}\\

\haiku{{\textquoteright} Van hand tot hand gaan,.}{de teerlingen ratelen}{over het tafelblad}\\

\haiku{Spontijn keert het hoofd,: - {\textquoteleft},...}{stoot de deur aan en fluistert}{Hoog bezoek vrienden}\\

\haiku{de eenen zwegen uit,,.}{eerbied de anderen niet}{wetend wat te doen}\\

\haiku{{\textquoteright} Prins Jan komt op met,,,:}{een knaap een twaalf jaar oud stelt}{hem vlak v\'o\'or Bon roept}\\

\haiku{Serclaes op met een,,: - {\textquoteleft}!}{knaap elf jaar oud stelt hem vlak}{v\'o\'or BonNummer twee}\\

\haiku{- {\textquoteleft}Prins Jan,{\textquoteright} spreekt Vaartje nu, {\textquoteleft},'?...}{weer beleefduw vader ligt}{op sterven nie waar}\\

\haiku{Nu staat Bon hoog-op,,:}{de teenen zwaait met zijn armen}{als vleugelen kraait}\\

\haiku{En hij kreeg per kans,.}{Aldegonde te snappen}{trok hem bij de mouw}\\

\haiku{Hier, bij mijn voeten,...}{dartelt en spartelt muze}{Clio Serafientje}\\

\haiku{Geef mij een nagel,.}{en ik hang het op tegen}{den wand van mijn huis}\\

\haiku{En ik sta erbij,,...}{als Pontius-Pilatus}{en wasch mijn handen}\\

\haiku{Dat zijn \`uwe neefjes,...!...,?}{Jan Verantwoordelijkheid}{Krieuwelt er niets Jan}\\

\haiku{{\textquoteright} Bullarius buigt -}{even om te bedieden dat}{het zoo gebeuren}\\

\subsection{Uit: Opinies van Proke Plebs}

\haiku{en ze zouden nog,.}{niet op me willen spuwen}{als ik in brand stond}\\

\haiku{En wie de waarheid,...}{liefheeft eet het bitter brood}{van de minderheid}\\

\haiku{Als 'ne jager schiet,:}{met een tweeloop dan denkt hij}{altijd halveling}\\

\haiku{en ze komen te -!}{paard en gekroond op ons af}{en zegevieren}\\

\haiku{schaf God af, en geen.}{museum is groot genoeg}{voor uw afgoden}\\

\haiku{t k\`an... driemaal op,... '}{de tienduizend tweemaal op}{de twintigduizend}\\

\haiku{En Janeke Bots...}{zelf bekent en staaft al die}{getuigenissen}\\

\haiku{En zoo stapte 't.}{beestje waar de molenaar}{het hebben wilde}\\

\haiku{opheffen tot den,.}{hoogsten top van wetenschap}{welstand en plezier}\\

\haiku{en mee moet ge, en,.}{eeuwig zijt ge ten minste}{langs \'e\'enen kant}\\

\haiku{met al hun zalfjes,;}{en pottekes we zijn kaal}{en we blijven kaal}\\

\haiku{daar is een tijd van,;}{te spreken daar is een tijd}{van te  zwijgen}\\

\haiku{want, zeggen ze, wij.}{hebben de ster gezien en}{we zijn gekomen}\\

\haiku{niet met herders en,;}{schapewachters maar bankiers}{en pelsen mantels}\\

\haiku{want inderdaad, gij;}{hadt uw nikkeltjes slechter}{kunnen gebruiken}\\

\haiku{en dat geluk moet,;}{hem van buiten komen want}{binnen zit het niet}\\

\haiku{en als God het hun,.}{toelaat dan kunnen ze met}{ons komen spreken}\\

\haiku{en alle wegen,;}{loopen naar Gheel als ge zelf}{liever naar Gheel loopt}\\

\haiku{en alles hangt er,...}{van af langs waar toe het paard}{met zijn kop staat}\\

\haiku{Tegen den Paus en '...}{t Vatikaan is alles}{geprobeerd geweest}\\

\haiku{\'en tatati \'en,;}{tatata kwaad en koleirig}{en opgewonden}\\

\haiku{{\textquoteleft}Waarom gaat de Paus,?}{altijd zoo zitten met zijn}{twee vingers omhoog}\\

\haiku{{\textquoteright} - - {\textquoteleft}En het Kruis in de{\textquoteright}, -.}{plaats stellen duwt er Petrus}{bij simpel-weg}\\

\haiku{ge wilt en laat ze ';}{verzorgen door den besten}{veearts vant vak}\\

\haiku{dat kindeke is;}{de ware president van}{den Volkerenbond}\\

\haiku{of denkt ge soms dat?...}{ik mijn eigen kinderen}{niet meer kan tellen}\\

\haiku{de kindekes die,,...}{moesten zijn Van den Bemde en}{die niet geweest zijn}\\

\haiku{{\textquotedblleft}Ik ben te arm{\textquoteright}... en,.}{de schoonste rijkdom dat zijn}{de kinderzielen}\\

\haiku{En grooter zot leeft, - -!}{er niet dan een redelijk}{mensch die z\'ot wil zijn}\\

\haiku{Kunstmest... ~ (pooze) ~ , '.}{Als ik niet getrouwd wask}{ging op de markt staan}\\

\haiku{Maar 'ne clown in de ';}{cirk speelt muziek opne gam}{van zeven flesschen}\\

\haiku{beleefdheid staat ook.}{al slecht ge-hippotekeerd}{den dag van vandaag}\\

\haiku{Maar dat 's alweer,, '...}{voorbij Van den Bemde en}{t onweer is over}\\

\subsection{Uit: Proke vertelt...}

\haiku{Het Deezeke schudt -:}{zijn beddeken uit maar de}{groote menschen klagen}\\

\haiku{de boeren waren,...}{fier dat ze zoo de pluimkes}{op hun klak kregen}\\

\haiku{Rozeke had wel ' ';}{geen schoone stem ent leed}{aanne korten adem}\\

\haiku{Ik geloove in...:}{God den Vader enz. met de}{drie groetenissen}\\

\haiku{Al wat ik zeggen,, ', ';}{kan valt mis ent valt te}{min naark vreeze}\\

\haiku{Maar opeens schoot hij,:}{in zijn koleire en zijn}{fraksken uit en riep}\\

\haiku{{\textquoteright} - brulde Potje, en;}{ditmaal had de Kater geen}{tijd om te poeffen}\\

\haiku{Maar Potje zweeg nog,.}{en hij kon zijn ooren en}{zijn oogen niet gelooven}\\

\haiku{Op 'ne morgen plots,,,!}{daar lag vlak v\'o\'or hen op de}{heuvelen Rome}\\

\haiku{En den godslieven;}{dag door waschten ze maar}{en ze sjauwelden}\\

\haiku{kwam de Wijze Prins,;}{in de parochie met een}{troon en soldaten}\\

\haiku{Al die hun kemel, '!}{verliezen loopen vant}{gras naar de biezen}\\

\haiku{{\textquoteleft}'ne Goeie kemel is ', '.}{t want zijn vier pooten staan}{hem ondert lijf}\\

\haiku{Al die hun kemel, '!}{verliezen loopen vant}{gras naar de biezen}\\

\haiku{Al die hun kemel, '!}{verliezen loopen vant}{gras naar de biezen}\\

\haiku{Al die hun kemel, '!}{verliezen loopen vant}{gras naar de biezen}\\

\haiku{En wat die man niet,.}{wist dat was de moeite niet}{waard om te weten}\\

\haiku{en met 'ne manken.}{kemel geraakt ge verder}{dan zonder kemel}\\

\haiku{Al die hun hemel, '!}{verliezen loopen vant}{gras naar de biezen}\\

\haiku{{\textquoteright} gilde moederke, {\textquoteleft},,!}{Jan luister toch naar mijn raad}{en blijf bij moeder}\\

\haiku{En Jan valt bots het,...}{Kutteke kapot en plat}{in de koekepan}\\

\haiku{Die smid was altijd;}{een boezemvriend geweest van}{Godfried van Boeljon}\\

\haiku{en 't eerste van!}{al stapt Herenthals recht}{naar Jeruzalem}\\

\haiku{- Vooruit nu al wat... - -!}{Janssens heet en zoo voort en}{zoo voort en zoo voort}\\

\haiku{{\textquoteleft}Maar Majesteit, wat?}{houdt ge toch z\'o\'o van boeken}{en veelweterij}\\

\haiku{- {\textquoteleft}Kwestie of Koning!}{Baltazar de Koningsstar}{zal willen volgen}\\

\haiku{De starre schoot een,,}{straal op mij een engel zong}{me een melodij}\\

\haiku{En hij ging ook wal ',;}{klaver scheren int veld}{goeie vette klaver}\\

\haiku{- Hij stond op, stapte,,:}{bij en klaar staande in den}{maneschijn hij sprak}\\

\haiku{- {\textquoteleft}Dan is 't goed{\textquoteright} - zei.}{de veekoopman en liet ze}{begaan met den ezel}\\

\haiku{En toen de intocht,:}{over was toen streelde O.L. Heer}{het ezeltje en zei}\\

\haiku{en zei zoo, dat, wie, ';}{het wapen gebruikt doort}{wapen zal vergaan}\\

\haiku{zuchtte de pastoor,!}{wat hebben ze schoon over den}{vrede gesproken}\\

\section{Joos Florquin}

\subsection{Uit: Lente van het hart. Brieven van Tijl aan Neleke}

\haiku{waren de schakel, -.}{in de redeneering en}{zuchtte toen eens diep}\\

\haiku{Hij sprak geen woord, maar.}{ik zag aa nzijn glimlach dat}{hij gewonnen was}\\

\haiku{Eerst en vooral werd.}{het kacheltje opgesteld}{in ons hoofdkwartier}\\

\haiku{heet het jochie dat.}{zich in ons huis een vaste}{plaats heeft verworven}\\

\haiku{Maar ik wil me door.}{Mercator niet van mijn doel}{laten afleiden}\\

\haiku{Ik wil je al mijn.}{pech van de laatste week niet}{verder uitbeelden}\\

\haiku{het mooiste daarvan.}{is nu wel dat het vijgen}{na Paschen zijn}\\

\haiku{Een sentimenteel.}{tochtje kon dat nu precies}{niet genoemd worden}\\

\haiku{{\textquoteleft}Of als hij komt, of,.}{als hij scheidt heeft de oude}{Maart zijn gif bereid}\\

\haiku{{\textquoteleft}Na de vasten komt,{\textquoteright}.}{Paschen maar het was toch}{maar een schrale troost}\\

\haiku{{\textquoteleft}Zoo het de veertig,.}{martelaars vindt blijft veertig}{dagen weer en wind}\\

\haiku{{\textquoteleft}Als 't helder is,.}{op Jozefsdag een goed jaar}{men verwachten mag}\\

\haiku{{\textquoteleft}De Maartsche zon en,.}{de Aprilsche wind schendt er zoo}{menig koningskind}\\

\haiku{Ik kan je nu mijn.}{gedroomd verzoek van daar straks}{blijde herhalen}\\

\haiku{Zullen grenzen of?}{andere complicaties}{ons tegenhouden}\\

\haiku{Onze gevleide.}{glimlach had waarschijnlijk iets}{weg van een grimas}\\

\haiku{Een vriendelijke.}{waard verwelkomde ons met}{een breeden glimlach}\\

\haiku{Zoo waren we in.}{een wip uit het bed en kon}{de dag beginnen}\\

\haiku{Ik zei dat hij met.}{den graaf toch over politiek}{had kunnen spreken}\\

\haiku{is je vroolijkheid,.}{dat ze alle sombere}{gedachten ontkent}\\

\section{N.E. Fonteyne}

\subsection{Uit: Kinderjaren}

\haiku{Zaken sedert een.}{eeuw opgehoopt die ons voor}{weken rijk maakten}\\

\haiku{om het mysterie;}{dat met elke aardschop ons}{voor oogen gegooid werd}\\

\haiku{een felle klaarte,;}{alleenheerschend tusschen de}{andere schijnsels}\\

\haiku{Ze heeft ons volk te.}{veel ontnomen en niets in}{de plaats gegeven}\\

\haiku{Of gedruktheid om.}{een grijze lucht of om een}{plotse eenzaamheid}\\

\haiku{En bij ieder versch:}{geopend graf werd het een}{heele vertelling}\\

\haiku{In die dagen ben:}{ik honderde malen naar}{moeder geloopen}\\

\haiku{in mijn voorstelling.}{besloot haar dood meteen mijn}{eigen verdwijning}\\

\haiku{En van geen vreemde.}{heb ik ooit meer gehouden}{als toen van mijn hond}\\

\haiku{Slechts een enkele.}{dag teekent zich heller in}{die groote grijsheid uit}\\

\haiku{ik hoe kieskeurig.}{die kinderboekerij was}{bijeengevallen}\\

\haiku{Die dolven we nu,,:}{stofferig vergeeld uit een}{vergeten kast op}\\

\haiku{Ze geeft niet zoozeer de,.}{schoonheid als wel een droom een}{honger naar schoonheid}\\

\haiku{Het wordt voor zulk een.}{kunstenaar moeilijk om de}{natuur te vinden}\\

\haiku{Lente en Herfst die.}{voor het kind nog maar alleen}{te doorvoelen zijn}\\

\haiku{Spraken we thuis over,;}{klerikalisme dan trok}{vader de neus op}\\

\haiku{omdat ze nooit de.}{omvang van de gevreesde}{smart bereiken kan}\\

\haiku{Voor haar open venster {\textquoteleft}{\textquoteright}.}{speelde de pianiste}{God save the Queen}\\

\haiku{En nu ging het met.}{de hoop en het vertrouwen}{elke dag bergaf}\\

\haiku{wat hadden we een!}{plezier toen we eindelijk}{ook mochten vluchten}\\

\haiku{Een woelige stroom.}{die niet eens meer in de nacht}{onderbroken werd}\\

\haiku{Voor een huis werd een,.}{zwijn geslacht en elders hing}{een koe op de kaak}\\

\haiku{rechtop, vier aan vier,.}{gebonden tot boven de}{knie\"en in het zand}\\

\haiku{En dagen hadden.}{we werk met het afloopen}{der nieuwe lagers}\\

\haiku{Dagen naeen trok de:}{heele veestapel van het}{Noordvrije door het dorp}\\

\haiku{Aan ieder huis hangt.}{voor de eerste maal sedert}{vier jaar weer een vlag}\\

\haiku{wij hebben te zeer.}{voor hun erbarmelijke}{lafheden geboet}\\

\haiku{Zoeken we niet over?}{de vrouw naar een houvast in}{een schimmenwereld}\\

\haiku{{\textquoteright} Of zijn de menschen?}{voor die dingen werkelijk}{zoo onverschillig}\\

\section{Ellen Forest}

\subsection{Uit: Aleid}

\haiku{{\textquoteleft}Oh, heelemaal niet -.}{op een dag als vandaag k\`an}{haast niets te vroeg zijn}\\

\haiku{{\textquoteright} Mevrouw vermeed het,.}{woord kellner omdat ze zoo}{fel anti-Duitsch was}\\

\haiku{{\textquoteleft}Hoe heerlijk niet, om!}{je diaphragma zoo geheel te}{kunnen ontspannen}\\

\haiku{Ik k\`an niet in 't,,....}{gerij loopen opzitten}{pootjes geven}\\

\haiku{Marijke nam haar:}{later even apart en zei met}{tranen in haar stem}\\

\haiku{E\'en van de tantes.}{kwam even aanloopen en nam}{haar mee de stad in}\\

\haiku{Met het aanhooren.}{van dit ritornello kocht}{ze een uur vrijheid}\\

\haiku{Ik zal hem alleen.}{melden dat u iets zoekt en}{u introduceeren}\\

\haiku{Dan schudde ze zich.}{inwendig en begon om}{zich heen te kijken}\\

\haiku{Nu jaag ik u naar,{\textquoteright},:}{bed zei hij toen iemand in}{het voorbijgaan zei}\\

\haiku{Daaronder stroomden,. '}{traag en moeizaam gedachten}{verbonden met Rein}\\

\haiku{Hij had in alle -.}{stilte willen trouwen in}{zijn daagsche pakje}\\

\haiku{Ze had t\`och wel iets,.}{van hem begrepen dat gaf}{haar eenige vreugde}\\

\haiku{Er zit meer energie.}{in een Londonsche season}{dan in een oorlog}\\

\haiku{Ja - d\`at is het groote,.}{raadsel dat ik nog bezig}{ben op te lossen}\\

\haiku{Moeder denkt dat we.}{nooit mogen weigeren als}{we gevraagd worden}\\

\haiku{Aan dien plicht viel niet -.}{te tornen dien bepaalde}{zelfs Mr. Li Tiang niet}\\

\haiku{Aleid van Doornhagen{\textquoteright}, {\textquoteleft}{\textquoteright}.}{zei ze tot zichzelfmake}{the very best of it}\\

\haiku{{\textquoteright} {\textquoteleft}Ja Elsy, stoor me maar,....}{niet behalve als mevrouw}{mocht telefoneeren}\\

\haiku{Ik ben geschrokken.}{van de photo van den tuin}{die u me stuurde}\\

\haiku{totaal anders dan.}{het beeld wat ik van ons huis}{meegenomen heb}\\

\haiku{Haar grootste afkeer - {\textquoteleft}{\textquoteright}.}{geldt de Japanners die ze}{lie-Europeans noemt}\\

\haiku{Ik kom,  om eens.}{echt met een Europeesch}{meisje te praten}\\

\haiku{Ze zat nog steeds op,.}{den divan haar hand weer slap}{hangend langs den kant}\\

\haiku{niets beduidde Aleid.}{of hij den strijd gewonnen}{of verloren had}\\

\haiku{Ze moest nu alle.}{romantisch gedaas van Oost}{en West vergeten}\\

\haiku{Om niet achteruit,.}{te gaan moet China bij het}{Westen aankloppen}\\

\haiku{Er bestond voor mij.}{een wereld en achter die}{wereld een leegte}\\

\haiku{Daar was eigenlijk.}{geen woord in dien brief dat op}{direct gevaar wees}\\

\haiku{Ze stond tegenover.}{een natuurverschijnsel van}{ongewonen aard}\\

\haiku{Ze draaide om dat.}{verlangen maar durfde het}{geen naam te geven}\\

\haiku{Ze was dolblij met,.}{haar vondst hoewel die vreugde}{niet onvermengd was}\\

\haiku{De eentonige.}{sing-song van den dominee}{ontroerde haar nooit}\\

\haiku{Ge staat nu op steenen -.}{van den Eiffel ge loopt op}{grind uit den Donau}\\

\haiku{{\textquoteright} vroeg Ruth, {\textquoteleft}luchthartig,,?}{oppervlakkig lachend om}{alles en n\`og wat}\\

\haiku{Lieve kind, je bent.}{zoo jong en je gaat een heel}{leven tegemoet}\\

\haiku{Zoolang je jong bent,.}{is het leven d\'a\'ar om er}{op te antwoorden}\\

\haiku{{\textquoteright} {\textquoteleft}We zijn beste maatjes,,.}{maar heel oppervlakkig en}{dat is maar goed ook}\\

\haiku{Dat zeggen alleen -.}{menschen in wie de weerstand}{voor goed gedoofd is}\\

\haiku{{\textquoteright} Er klonk wrevel uit,.}{grootvaders stem iets dat ze}{van hem niet kende}\\

\haiku{{\textquoteleft}Als je dat denkt, mijn,}{kind hebben we er ons maar}{bij neer te leggen}\\

\haiku{Je kon een vrouw, van,?}{tante Arda's leeftijd dit}{toch niet weigeren}\\

\haiku{Hier begon hij, wat,,.}{hij voor zichzelf noemde zijn}{nieuw eenzaam leven}\\

\haiku{De wereld is zoo,,{\textquoteright}.}{slecht niet als zij er uit ziet}{zei Otto Vorendonk}\\

\haiku{Dit is alleen het,.}{bewijs voor mijn kameraad}{dat u betaald hebt}\\

\haiku{{\textquoteleft}Als de jeugd met haar?}{enthousiasme het nu}{eens in handen nam}\\

\haiku{Van zoo'n avond bleef niets.}{over dan een beetje weemoed}{over de mislukking}\\

\haiku{Als hij daar stond en}{de menschen gevangen hield}{in zijn stillen blik}\\

\haiku{Ze viel zich zelf op.}{als anders dan de meeste}{Europeanen}\\

\haiku{Z\'o\'o zag ze ook Rein,,.}{een man die naar den hemel}{had willen reiken}\\

\haiku{Dat hij weggegaan,,.}{was bewees dat zij beiden}{zich vergist hadden}\\

\haiku{hij had alleen haar.}{hand gegrepen en haar even}{heel diep aangezien}\\

\haiku{Een Chineesche vrouw,:}{mag geen emotie toonen maar}{haar jongen lachte}\\

\haiku{{\textquoteright} - Met den rug naar het.}{venster stond zij en las de}{woorden nog eens over}\\

\haiku{Bleek die waarde te,.}{klein dan zouden zij elkaar}{niet meer moeten zien}\\

\haiku{Toen zij eenmaal op.}{het balkon stond was alle}{twijfel verdwenen}\\

\haiku{Toch wist ze dat ze,,.}{geen kwaad gedaan had noch zij}{noch de anderen}\\

\haiku{Ze wilde niet in.}{de armen van Chung liggen}{en aan Rein denken}\\

\haiku{Vandaag voor het eerst.}{voel ik onze belangen}{als tegenstrijdig}\\

\haiku{Morgen kan alles,.}{uit zijn maar het kan even goed}{nog eeuwen duren}\\

\haiku{Dit is ook weer een {\textquotedblleft}{\textquotedblright},.}{van onze grootedrawbacks dat}{kasten-systeem}\\

\haiku{Die had er maar een,,.}{paar die op hem aasden maar}{die aasden zoo goed}\\

\haiku{En doet hij het niet,.}{dan is hij aan handen en}{voeten gebonden}\\

\haiku{{\textquoteright} En toen in een voor,:}{hem vreemde vervoering liet}{hij zich gaan en zei}\\

\haiku{En langzaam begon.}{de vastheid van zijn besluit}{te verminderen}\\

\haiku{Het was een brief van,.}{zijn moeder waarin deze}{schreef over de meisjes}\\

\haiku{Dat wil Wellington,.}{Koo ook en dat wil ik ook}{en velen met mij}\\

\haiku{{\textquoteright} Hij antwoordde niet,:}{dadelijk maar zei na een}{oogenblik wachten}\\

\haiku{Ze begreep hem weer,.}{niet maar ze voelde dat hij}{iets moois bedoelde}\\

\haiku{{\textquoteleft}Stellig kleurling of!}{halfbloed tot in het derde}{en vierde geslacht}\\

\haiku{Omzichtig schoof hij.}{tusschen de anderen door}{tot hij naast haar stond}\\

\haiku{Nu was een gevoel.}{van spanning het eenige dat}{ze registreerde}\\

\haiku{Hij liep in dien avond -.}{als in een wolk van goudstof}{recht op zijn doel af}\\

\haiku{Dan zouden ze aan.}{zichzelf en hun stemmingen}{overgelaten zijn}\\

\subsection{Uit: Passiebloemen (onder ps. Lucy d'Audretsch)}

\haiku{{\textquotedblright} dan dat de wereld}{voor je in aanbidding ligt}{en je eigen ziel}\\

\haiku{{\textquoteright} Moe van de emotie,.}{en van het waken sliep ze}{gauw in en droomde}\\

\haiku{{\textquoteright} {\textquoteleft}Och God, 't is niets,, '.}{heelemaal nietsk weet niet}{wat je van me wilt}\\

\haiku{Begrijp jij niet, dat,?}{als jij daar zoo zit dat ik}{dan niet weg kan gaan}\\

\haiku{telkens was je er... -,}{weer en dacht ik aan je en}{zag ik je voor me}\\

\haiku{die maanden waarin '... '...}{ze wachtten opt examen}{t allerlaatste}\\

\haiku{Waar dacht ze aan, toe! ',...}{t Zou wel gaan Mies wilde}{nu ook haar best doen}\\

\haiku{Nu, toen is hij een...{\textquoteright} {\textquoteleft}, '...}{eindje omgegaanOch ja}{t zal wel zoo zijn}\\

\haiku{ze waren ook z\'oo,.}{gek verliefd zij twee\"en en}{zoo altijd samen}\\

\haiku{Hij gaf geen antwoord, '...}{slierde zich inn grooten}{stoel en nam een krant}\\

\haiku{H\`e ja, je lieve -,...}{armen om mijn hals h\'eerlijk}{dat zachte kussen}\\

\haiku{God, zeg, zoo bij jou,...,...{\textquoteright}.}{te kunnen blijven nog \'een}{minuutje h\`e II}\\

\haiku{Vlugge, schuchtere,.}{blikken die afgleden voor}{zij ze goed merkte}\\

\haiku{Neen, stervende - dat ',.}{s beter dan had ze meer}{reden tot blijven}\\

\haiku{Eigenlijk vind ik '...{\textquoteright}}{t ook wel heerlijk dat je}{zoo'n dom vrouwtje bent}\\

\haiku{Maar als jij denkt dat...}{je goeddoet met z\'oo gekleed}{bij armen te gaan}\\

\haiku{Raar, dat zij twee\"en '...,...}{nooit overt kindje spraken}{n\'een zelfs niet voor Nic}\\

\haiku{Och, neen, 't was nu,...}{al weer weg die behoefte}{om alleen te zijn}\\

\haiku{... en t\`och, h\'eel diep - w{\`\i}st,...}{ze wel dat hij niet d\`at was}{wat ze gehoopt had}\\

\haiku{Eigenlijk had Nic,......}{haar pas geleerd wat passie}{was tellen nou toch}\\

\haiku{en in haar was ze, ' '... {\textquoteleft}}{toch wel bang bang voort \'een}{als voort ander}\\

\haiku{nu was hij w\`eg... en '...}{zij hadt gewild en nu}{had ze niets te doen}\\

\haiku{ze had hem toch wel,...,,...}{lief maar ellendig dat leege}{niet weten w\`at doen}\\

\haiku{en straks... uit 't bad - '...}{met ander linnen enn}{andere japon}\\

\haiku{die hoek nog - niet in '..., '...}{n vuile spiegel zien wacht}{even nogr haar los}\\

\haiku{De twee voor mij uit -'...}{h\'eel innig kussen elkaar}{voor Homerus beeld}\\

\haiku{En nu zijn ze weg - '...}{w\`eg int gewoel van de}{Parijsche straten}\\

\haiku{{\textquoteright} ~ {\textquoteleft}Lieve, 'k zou......{\textquoteright} {\textquoteleft}}{wat goed willen doen wat goed}{aan die menschjes}\\

\haiku{Ik ben vandaag naar ' - '...}{t kerkhof geweestt is}{Allerheiligen}\\

\haiku{Die twee zijn dood, Odin....}{en wij bezochten hen op}{Allerheiligen}\\

\section{Fritz Francken}

\subsection{Uit: Aan het vinkentouw}

\haiku{Voorzeker ook niet, (!}{uit leedvermaakde vlaag kon}{zijn vrouw overvallen}\\

\haiku{Met tranen in de.}{oogen vertelde mama wat}{haar overkomen was}\\

\haiku{Het was beslist een,,.}{aardig man die direkteur}{betrekkelijk jong}\\

\haiku{Als Calkoen met hem -,}{uitreed naar Holland en dat}{gebeurde dikwijls}\\

\haiku{Hij sprak z\'oo keurig,,!}{wist van alles iets \`af en}{van dat iets \`alles}\\

\haiku{Die Cornelissen... -...!}{is in Holland geweest en}{Och die sukkelaar}\\

\haiku{nog zoo kwaad niet... - Graaf...!}{een vaart en de Schelde mondt}{uit te Zeebrugge}\\

\haiku{Ge weet, bromde de,,!}{aspirant-Minister}{de gazetten he}\\

\haiku{Als hij zich spoedde,.}{haalde Hagedoorn nog den}{trein van tien v\'o\'or eenen}\\

\haiku{Hij verhaastte den.}{pas. Eindelijk ontwaarde}{hij het station}\\

\haiku{Hij heeft me gevraagd,.}{u te verzoeken zoolang}{op hem te wachten}\\

\haiku{Hij scheen beslist in.}{staat om iemand onverhoeds}{een lik te geven}\\

\haiku{Wou hij misschien even?}{de appartementen in}{oogenschouw nemen}\\

\subsection{Uit: De blijde kruisvaart}

\haiku{Wij werden op 't.}{oefenplein gedrild dat we}{draaiden als doppen}\\

\haiku{Den volgenden dag, ',.}{s middags kuierden we}{te Ieperen rond}\\

\haiku{Den 19en Oktober, ',.}{s middags gingen we in}{Le Havre aan wal}\\

\haiku{si nous n'avions,!}{pas eu la Belgique nous}{serions fichus}\\

\haiku{t Is een stadje,,.}{aan zee als een nest in de}{vallei gedoken}\\

\haiku{Oude wijvekens.}{verwelkomden de troepen}{met tandeloozen mond}\\

\haiku{En 't sneeuwde zoo.......}{nijg dat dat de peerden er}{niet niet door konden}\\

\haiku{U wil ik het wel,,...}{vertellen sergeant u}{kunt me begrijpen}\\

\haiku{In den kaos van.}{de frontdrukte verloor ik}{Seppe uit het oog}\\

\haiku{De boer kwam in de,:}{schuur een riek op den schouder}{en zei goedzakkig}\\

\haiku{zoolang d'r leven,,.}{is is er hoop beweerde}{Vodden-en-Beenen}\\

\haiku{Altijd koest blijven,... -!}{geduldig afwachten Met}{de vrinketten d'rop}\\

\haiku{Mismoedig brak de,.}{Spion de harde korst at}{met lange tanden}\\

\haiku{Goedschiks leenden de '.}{jongens mekaart beste}{stuk uit hun kleerkast}\\

\haiku{k Ben eens kurieus.}{of hij mijn kommissie zal}{bol gewerkt hebben}\\

\haiku{- 't Is al elf uur,, '.}{en de Spion is er nog}{niet zeit Zwierken}\\

\haiku{En op een schoonen '.}{dag verscheen hij terug in}{t Schipperskwartier}\\

\haiku{De panden van hun.}{verbalemonden kapoot}{flapten in den wind}\\

\haiku{Met wanhopigen,.}{blik zocht hij grabbelde die}{der gesneuvelden}\\

\haiku{'t Was een echte:}{verlossing toen de dokter}{den zesden dag zei}\\

\haiku{Rustig labeurde,.}{hij voort plaatste de zakjes en}{damde elke laag}\\

\haiku{Door de gaten van.}{de schietkanteelen loerden}{ze naar den vijand}\\

\haiku{De marsch van 't.}{regiment werd ingeluid}{door den Hondendief}\\

\haiku{Broer, ik schaam me en,.}{verwensch den dokter die me}{nog niet weerkeeren laat}\\

\haiku{In 't grauwe der,;}{schemering hingen ze te}{dansen de muggen}\\

\haiku{De Rik, 't Zwierken,,, '.}{de Voddetromp Drupneust}{Stropke mochten mee}\\

\haiku{Maar ze lazen een.}{akte van berouw na die}{onzinnige daad}\\

\haiku{- Als 't zoo voort gaat,.}{dan komen we nog bij den}{keunink te biechten}\\

\haiku{Wie geld verdiende,,}{noodigde broederlijk de maats}{uit bezocht met hen}\\

\haiku{- Bruggemans, geef me,!}{den hamer dat ik z'nen}{kop v\`astnagel}\\

\haiku{- Wacht sukkelaar, tot ',.}{we int Schipperskwartier}{komen zei Drupneus}\\

\haiku{Me dunkt dat ik dien.}{rakker nog zien ravotten}{heb op de Veemarkt}\\

\section{E. Franquinet}

\subsection{Uit: Maskeraad}

\haiku{{\textquoteleft}Kom{\textquoteright} zag er, {\textquoteleft}iech weet,,.....}{daste zwiege kins es et}{t'rop aon kump en dan}\\

\haiku{veur aon w\`erreke,.}{te dinke iers nog e flink}{st\"ok weijer te goon}\\

\haiku{men gedachtes, die.}{op tat momint wel erreg}{verward waore}\\

\haiku{Veer st\'onte dao wie,.}{geslage zeker wel e}{paar menute laank}\\

\haiku{veziet aon Pastoer,}{Bertels boe iech gistere}{den hielen aovend}\\

\haiku{sjoen st\"ok preuf en k\"a\"ort.}{in ze gehiel en in zen}{apaarte motieve}\\

\haiku{{\textquoteright} Et \'onweer leet nao,.}{dee slaag neet aof ieder woort}{et nog erreger}\\

\haiku{De bedeleer WAT,}{z\`ekste van deen erreme}{n\'onk Zjeraar dee dao}\\

\haiku{Euver de loch, die,.}{de maan van ziech aofgaof}{zal iech mer zwiege}\\

\section{Robert Franquinet}

\subsection{Uit: Drijfzand}

\haiku{Ze heeft een van de.}{meest opwindende ruggen}{die ik gezien heb}\\

\haiku{Een zware slag met.}{een stalen cylinder doet}{mijn kaakbeen barsten}\\

\haiku{De aftapper is.}{niet goed op de elektrische}{stroom aangesloten}\\

\haiku{De pennen dringen.}{langzaam binnen tussen de}{nagel en het bot}\\

\haiku{[II] Veel zoekt naar een.}{logische samenhang in}{mijn herinnering}\\

\haiku{Je hoeft er niet meer.}{op de juiste wijze te}{relativeren}\\

\haiku{Nu ik Clara van.}{kortbij heb gezien denk ik}{niet aan dat alles}\\

\haiku{Ik ben nog geen twee.}{dagen in de bungalow}{of ik spreek met haar}\\

\haiku{{\textquoteleft}U hebt een figuur.}{om een klassiek beeldhouwer}{mee te verrukken}\\

\haiku{De avond sluit tegen.}{de duinen als met luiken}{van beslagen zink}\\

\haiku{Al het andere.}{immers belemmert me}{direkt te schrijven}\\

\haiku{Mijn handen glijden.}{tussen je benen terwijl}{je de trap op gaat}\\

\haiku{Je familie is.}{in rep en roer want de man}{is wereldberoemd}\\

\haiku{Ze zegt dit laatste.}{met  een klemtoon die een}{beetje spottend is}\\

\haiku{Sedert jaren heeft.}{geen enkele dokter daar}{iets aan kunnen doen}\\

\haiku{Als je er levend,...}{aankomt vertel hen dan waar}{ik ben met het lijk}\\

\haiku{Ik keer me af van.}{Dalan die zijn hoofd in zijn}{armen houdt en schreit}\\

\haiku{We komen in een.}{residenti\"ele wijk}{met veel prikkeldraad}\\

\haiku{{\textquoteright} Ik begin haar uit.}{te kleden met een schroom of}{ik het nog nooit deed}\\

\haiku{Ik voel de spieren,.}{van haar benen haar buik en}{haar armen spannen}\\

\haiku{Ze doet haar ogen open.}{en begint in wanorde}{dingen te zeggen}\\

\haiku{Achter me is een.}{wagen uit een kleine steeg}{komen aanrijden}\\

\haiku{Mijn mensen brengen...}{u tot in de kleine straat}{achter de studio}\\

\haiku{Ik voel me niet thuis.}{in een leven waarin geen}{dingen gebeuren}\\

\haiku{Ik wil ook niet dat.}{je aan het leibandje van}{mijn gevoelens loopt}\\

\haiku{Ik heb de pest aan.}{jaloezie die ongegrond}{en kwaadaardig is}\\

\haiku{Hij strekt zijn benen}{languit en ik zie dat zijn}{broek verfomfraaid is}\\

\haiku{Een soldatenknoop.}{zit met een brede veter}{vast tegen haar keel}\\

\haiku{Zijn bleke pishuid?}{betasten als deeg van haar}{eigen substantie}\\

\haiku{{\textquoteright} {\textquoteleft}Laat je kapsel in.}{ieder geval natuurlijk}{en zonder linten}\\

\haiku{Ik vind die grote.}{kraag van zwart fluweel wel mooi}{op dat bleke blauw}\\

\haiku{Er zit een stukje.}{touw in  de klep van de}{smalle brievenbus}\\

\haiku{Zacht rochelend komt.}{langs de kapotte tong nog}{even adem naar buiten}\\

\haiku{o, nee, Marc, dat niet,,}{niet nu op deze trap wacht}{ik kom er van af}\\

\haiku{Al geiler wordend:}{fluisterde ze met hete}{lippen in mijn oor}\\

\haiku{Ze zegt dat zachtjes,,.}{niet hatelijk maar wel op}{een toon van verwijt}\\

\haiku{De Lardys hebben.}{me gevraagd om hen na mijn}{werk hier te treffen}\\

\haiku{En daarbij verzorgd,{\textquoteright}.}{als een prins zegt de dame}{nu gemoedelijk}\\

\haiku{De genotskrampen.}{wringen zich als vuurmessen}{door heel mijn lichaam}\\

\haiku{Zij kijkt de kamer.}{rond alsof ze niet meer heel}{goed weet waar ze is}\\

\haiku{Je wilt zeggen dat.}{je de transformatie als}{een totaalbeeld ziet}\\

\haiku{De slordige man.}{in de deuropening kijkt me}{achterdochtig aan}\\

\haiku{Ik kom weer in het.}{daglicht en zie haar op de}{stoep aan de overkant}\\

\haiku{Ze wentelt zich om.}{en gaat met het gezicht in}{de kussens liggen}\\

\haiku{Er stopt een auto.}{voor het huis in de stille}{straat met de tuintjes}\\

\haiku{Ik houd er niet van.}{dat anderen zich met mijn}{zaken bemoeien}\\

\haiku{Ze keek naar boven.}{en heeft mij ongetwijfeld}{aan het raam gezien}\\

\haiku{ik remmen verder,}{in de straat maar voor dat ik}{aan het tuinhek ben}\\

\haiku{Een vrouw rent uit een,:}{huis aan de overkant met de}{handen in de lucht}\\

\haiku{Een kind is voor mijn,.}{wagen gevallen maar ik}{heb hem niet geraakt}\\

\haiku{Het kan niet meer uit,.}{mijn nekmerg het kan niet meer}{van het netvlies af}\\

\haiku{Hij had honger want}{hij vertelde dat hij al}{acht dagen leefde}\\

\haiku{Ik schreef ook niet om.}{gefilmd te worden en nu}{is het toch gebeurd}\\

\haiku{{\textquoteright} Hij is ontstemd en.}{wordt het nog meer wanneer ik}{zijn aanbod weiger}\\

\haiku{Het provoceert een.}{druk op mijn ingewanden}{die me naar de w.c}\\

\haiku{En heb ik met mijn?}{achterhoofd de tegels van}{het toilet geraakt}\\

\haiku{Ik huil van wanhoop.}{en lig uren op mijn rug in}{de nacht te staren}\\

\haiku{{\textquoteright} {\textquoteleft}Fally,{\textquoteright} zegt ze, {\textquoteleft}erg.}{knap en intelligent maar}{zo links als de pest}\\

\haiku{Ik ben nog nooit een.}{intelligent journalist}{tegengekomen}\\

\haiku{Je hoeft dat niet te.}{zijn omdat de vrouwen je}{vanzelf graag mogen}\\

\haiku{Ze laat zich door Bunk.}{met zijn fluwelen neus op}{haar wang liefkozen}\\

\haiku{tussen haakjes, met...{\textquoteright} {\textquoteleft}}{het voorbeeld dat het kind thuis}{altijd gehad heeft}\\

\haiku{Je gaat een beetje,,{\textquoteright}.}{te ver Georges merkte}{Veronica op}\\

\haiku{Met mijn mond in haar.}{kut en mijn vingers in haar}{anus of omgekeerd}\\

\haiku{Wanneer we er uit:}{komen legt ze een hand op}{mijn schouder en zegt}\\

\haiku{loop je de kans dat '.}{ers ochtends een zwarte}{schorpioen in zit}\\

\haiku{Mij kop wordt onder.}{een kraan gehouden waaraan}{een waterzak zit}\\

\haiku{Ik voel het ritme.}{van mijn hartslag verhevigd}{slaan achter de ogen}\\

\haiku{dan word ik dronken...}{en als ik dronken ben kan}{ik niet schrijven}\\

\haiku{Zijn ogen rollen en.}{hij draait met zijn heupen als}{een buikdanseres}\\

\haiku{{\textquoteleft}Ze is zo warm dat.}{ze je aan het spreken zal}{brengen van waanzin}\\

\haiku{Boutan heeft de wacht,{\textquoteright}, {\textquoteleft}.}{zegt ze bedeesdik kan wel}{een uurtje blijven}\\

\haiku{Ze gaat tegen een,.}{muur staan met beide armen}{over de borst gekruist}\\

\subsection{Uit: Ghislaine la Bruy\`ere en ik}

\haiku{Zij was uitgestrekt,.}{met haar kleine buik tegen}{den grond gaan liggen}\\

\haiku{Ik verzocht haar een.}{middag in de stad om een}{rendez-vous}\\

\haiku{Den volgenden dag.}{sprak ik met Claude over de}{groote desillusie}\\

\haiku{Hij haalde een klein:}{verzenboek van Paul Geraldi}{uit zijn zak en zei}\\

\haiku{ik moest Ghislaine...,...}{ergens ontmoeten en dat}{geschiedde aldus}\\

\haiku{doch wie wilde plots!?}{dat uwe vrees en uw vreugde}{geen grenzen kende}\\

\haiku{Zij beheerschen u,;}{en het lokaas in beider}{handen is Adja}\\

\subsection{Uit: Marat, de marskramer}

\haiku{Niets of niemand mocht.}{hem aan zijn vorstelijke}{taak herinneren}\\

\haiku{Hij bleef bij de deur,.}{staan die achter hem door de}{knecht gesloten werd}\\

\haiku{Misschien werd wel juist.}{daarom zijn gevoel voor haar}{nog aangewakkerd}\\

\haiku{Duizenden mensen,.}{op straat die niets deden dan}{praten en zingen}\\

\haiku{Marat heeft zich in.}{het vest der zo misprezen}{meesters gestoken}\\

\haiku{Zijn voeten, waren.}{gehuld in het vilt van een}{oud soldatenvest}\\

\haiku{Even later werden.}{uit de hooikar enkele}{wapens geladen}\\

\haiku{Inmiddels stonden.}{de karpers in vuurvaste}{schotels in de oven}\\

\haiku{Mirabeau had zijn.}{stoutmoedige frase reeds}{tienmalen herhaald}\\

\haiku{Monarchisten en.}{Revolutionnairen}{zongen en juichten}\\

\haiku{{\textquoteright}... {\textquoteleft}Maar het gaat  er,.}{om dat men de jongens slaafs}{leert gehoorzamen}\\

\haiku{In de Assemblee.}{Nationale gingen}{stemmen op voor hem}\\

\haiku{Op de rand van zijn.}{dagblad schreef hij elke dag}{zijn aanmerkingen}\\

\haiku{{\textquoteleft}Ik had nochtans wat,...}{moeite om hem kapot te}{krijgen de bandiet}\\

\haiku{{\textquoteleft}Aan de lantaarn met,!}{de priesters die de eed niet}{willen afleggen}\\

\haiku{wel de traditie,.}{doch niet de Waarheid maakte}{er ons aan gehecht}\\

\haiku{De postmeester van.}{Chanitrix brengt vleesbouillon}{voor de kinderen}\\

\haiku{Een onbekende:}{nadert het rijtuig en roept}{met halfluide stem}\\

\haiku{vrienden wachten ons....}{nog anderhalf uur en wij}{worden afgelost}\\

\haiku{{\textquoteleft}Wij zullen wachten,{\textquoteright}.}{totdat het morgenlicht in}{de lucht is zucht hij}\\

\haiku{Hij wil emigreren!}{als de Cond\'e en d'Artois met}{het goud van Frankrijk}\\

\haiku{Maar het leek of de.}{stad Parijs met dit alles}{niets te maken had}\\

\haiku{Er kwamen nu af;}{en toe patrouilles aan bij}{het huis van Danton}\\

\haiku{De negentiende.}{Augustus schrijft Marat zijn}{groot requisitoir}\\

\haiku{In een wolk van stof.}{dringt deze duistere vloed}{tot aan de Abbaye}\\

\haiku{Een grote blonde:}{jongen werkte zich uit hun}{midden los en riep}\\

\haiku{Hij had bemerkt dat.}{zijn woorden geen rook waren}{geweest voor Marat}\\

\haiku{Barbaroux voelde!}{nu nog een vlaag van woede}{in hem opkomen}\\

\haiku{Een ogenblik leek het,.}{hem dat iemand zich in het}{trappenhuis bewoog}\\

\haiku{{\textquoteleft}Ik heb een boodschap,{\textquoteright}.}{voor citoyen Barbaroux}{herhaalde de stem}\\

\haiku{Morgenvroeg kun je.}{in de rue de la Perle}{je honger stillen}\\

\haiku{{\textquoteleft}Dit is de vrucht van.}{mijn grenzeloze liefde}{voor de Republiek}\\

\haiku{Met een tedere.}{streling gleed nu zijn hand over}{de flank van het dier}\\

\haiku{{\textquoteleft}Ik sterf zonder schuld,!}{aan de misdaden waarvan}{ik  wordt beticht}\\

\haiku{De vogeldiertjes,}{fluiten in de bessenstruik}{kom je nog niet aan}\\

\haiku{{\textquoteleft}Ik vind dat je me.}{toch eindelijk wel eens je}{voornaam noemen mag}\\

\haiku{{\textquoteright} {\textquoteleft}Wat ik er in draag,,!...}{is niet alleen voor vandaag}{het is voor altijd}\\

\haiku{Hij keek Barr\`ere:}{met een spottende blik}{aan en zij opeens}\\

\haiku{Dacht je soms, dat het,?}{volk vergeten zou wat je}{voor hen gedaan hebt}\\

\haiku{Dacht je soms dat de!}{loopjongens mijn krant uit hun}{mouw kunnen schudden}\\

\haiku{Maar het leek meer een!}{samenraapsel van Rouxisten}{dan Royalisten}\\

\haiku{{\textquoteleft}Kijk eens naar mij, mooi,,...}{duifje of ben je op weg}{naar je biechtvader}\\

\haiku{Toen viel Charlotte's.}{oog op een kleine dolk met}{een ivoren handstuk}\\

\haiku{Zij komt weer in haar.}{hotel en gaat zich in haar}{kamer opsluiten}\\

\haiku{Hij liet haar een kop.}{koffie naar boven brengen}{met een krakeling}\\

\haiku{Zij weet nu, dat ze.}{dagen en nachten lang zo}{zou kunnen wachten}\\

\subsection{Uit: Mijn hart zal niet vrezen}

\haiku{{\textquoteleft}Al legert zich ook,.}{tegen mij een krijgsmacht mijn}{hart zal niet vrezen}\\

\haiku{Maar in zijn toogzak....}{greep zijn hand instinctief naar}{de portefeuille}\\

\haiku{Blonken er een paar?}{sterren door het vreemdsoortig}{gewei van de boom}\\

\haiku{Hij stond lichtelijk.}{in een heup geknikt als de}{Venus van Milo}\\

\haiku{glou{\textquoteright}, deed hij met zijn,.}{keel alsof hij er een fles}{in liet uitlopen}\\

\haiku{Celeste was toen..}{twaalf jaar en wij maakten het}{voor haar in orde}\\

\haiku{De kostelijke.}{rechtvaardigheid richt zich door}{het schepsel tot God}\\

\haiku{je kinderen te....}{verwaarlozen om in de}{stad rond te hangen}\\

\haiku{Er werd een oude.}{bonbondoos uitgehaald met}{vergeelde foto's}\\

\haiku{Servaas was verbaasd.}{en verzekerde dat het}{geen sou kosten zou}\\

\haiku{Ik zal  pogen.}{hen een beetje gelukkig}{te leren leven}\\

\haiku{Misschien kan ik zelf....}{wel bij een van mijn vrienden}{daarvoor aankloppen}\\

\haiku{Maar de nooddruft heeft.}{van geslacht op geslacht de}{wegen verduisterd}\\

\haiku{Neen, hier werd hij de!}{ontzaglijke taak van het}{priesterschap gewaar}\\

\haiku{Hij versnelde zijn.}{passen en voelde het bloed}{door zijn aderen slaan}\\

\haiku{Motors van tanks en,.}{vlammenwerpers die nog nooit}{gebruikt zijn geweest}\\

\haiku{als men buiten de,!}{Kerk is en men twijfelt dan}{is men buit voor God}\\

\haiku{Het waren allen.}{dezelfde mensen in dat}{wonderlijke licht}\\

\haiku{Maar nu klonk het zo,.}{onrustbarend dat hij naar}{het tuinpoortje liep}\\

\haiku{Bij het venster stond,.}{Moron naar het bed van de}{priester te staren}\\

\haiku{{\textquoteleft}Wat wil je, kindje,.}{je kunt een dode geen nieuw}{leven inblazen}\\

\haiku{haal mijn broertje uit,..{\textquoteright} {\textquoteleft}....}{het graf en de goede man}{deed hetAls je blieft}\\

\haiku{Het was of zij de....}{kans kreeg om haar handen rond}{zijn keel te snoeren}\\

\haiku{{\textquoteleft}Is citroenwater?}{goedkoper zonder suiker}{dan op de prijslijst}\\

\haiku{Dat is een ramp voor{\textquoteright},.}{de caf\'e-houders vandaag}{zei hij stotterend}\\

\haiku{Het enige wat me,,.}{er in aanstaat is dat hij}{van angst barsten zal}\\

\haiku{Het betekende;}{voor de arbeiders ook het}{summum van weelde}\\

\haiku{Het voorval met de.}{Duitse motorfiets stond haar}{nog fris voor de geest}\\

\haiku{{\textquoteright} {\textquoteleft}De dokter heeft een..{\textquoteright} {\textquoteleft}}{rapport opgemaakt over een}{ernstige kwetsuur}\\

\haiku{Feitelijk voelde.}{hij reeds dat zijn komst hier iets}{belachelijks was}\\

\haiku{Het ontwapende.}{hem en beledigde hem}{tegelijkertijd}\\

\haiku{Hetgeen hen bindt, dat.}{zijn Uw verhoudingen tot}{hun eigen wereld}\\

\haiku{Het begrip dat zij,....}{zich van U vormen is dat}{Gij geschapen hebt}\\

\haiku{Ik moet u weer op{\textquoteright}, -.}{uw kleinheid van geest wijzen}{antwoordde Servaas}\\

\haiku{Soms verdroeg zij de.}{stilte van het landhuis niet}{meer en vluchtte weg}\\

\haiku{Temidden van de,.}{vijandig geworden hoon}{wandelt de priesteh}\\

\haiku{Tientallen malen.}{had ze iets bedacht om met}{hem alleen te zijn}\\

\haiku{Er was veel vis, maar.}{hij smaakte naar het vuil van}{de stadsriolen}\\

\haiku{In ieder geval{\textquoteright}.}{wel als hij een glas onder}{zijn neus heeft gehad}\\

\haiku{Op alle wijzen.}{die de mens veredelen}{en zijn hart openen}\\

\haiku{Dat het dieptepunt.}{van elk probleem reeds lang door}{Kristus is ontknoopt}\\

\haiku{{\textquoteleft}Zalig de armen,.}{van geest want hen behoort het}{rijk der hemelen}\\

\haiku{Boven het lage.}{bahut hing een stilleven}{met witte eenden}\\

\haiku{Toen begon hij met:}{een gebroken stemgeluid}{langzaam te spreken}\\

\haiku{{\textquoteleft}Als je zin hebt, mijn,.}{klein konijntje dan kun je}{met mij mee-eten}\\

\haiku{Het was alsof hij.}{in de uiterste hoek werd}{teruggeslagen}\\

\haiku{In die toestand vond,.}{C\'eleste hem die kijken kwam}{waar haar vader bleef}\\

\haiku{Tussen acht en elf '.}{uurs avonds had de wind de}{grond droog geblazen}\\

\haiku{Intussen was het.}{lijk van de priester in de}{gang blijven liggen}\\

\haiku{{\textquoteleft}Ik bedoel...., werp eens,....}{een oogje in die richting}{je kunt nooit weten}\\

\subsection{Uit: Spiegelgruis}

\haiku{Zij is niet bedroefd,.}{maar zij is gegrepen door}{iets onheilspellends}\\

\haiku{Entre ta chair et,.}{la mienne un r\^eve}{brulant m'a poss\'ed\'ee}\\

\haiku{Mon corps tout entier, -.}{s'est livr\'e \`a tes l\`evres}{infatigables H\'elas}\\

\haiku{Zijn lichaamskracht geldt,}{die van twee jonge vrouwen}{totdat zij beiden}\\

\haiku{In dit ogenblik wist.}{zich voor het eerst mijn telling}{der minuten uit}\\

\haiku{De ontelbaren,.}{voor wie de dood een zoete}{verlossing zou zijn}\\

\haiku{Niemand denkt meer aan,.}{het brood dat hij vanochtend}{niet gekregen heeft}\\

\haiku{Mijn hoofd  wordt als.}{het ware door een nevel}{van goud licht omstroomd}\\

\haiku{Dan staat hij wijdbeens.}{recht en tussen zijn knie\"en}{door zie ik de dood}\\

\haiku{Hij krijgt een blauwe.}{linnen werkbroek en een trui}{met mottengaten}\\

\haiku{Rustige avonden.}{vliegen met merkwaardige}{flitsen door mijn hoofd}\\

\haiku{In de duisternis.}{en door het slijk zoek ik de}{weg naar de Rijksbaan}\\

\haiku{se tirent les poils!}{des fesses pour se faire}{des cure-dents}\\

\haiku{De toren van het.}{mergelkerkje steekt boven}{de heuvelrug uit}\\

\haiku{- Het is bij twee uren,,.}{heren ik heb hier niets meer}{aan toe te voegen}\\

\haiku{Maar Tigre schijnt,:}{niets meer aux s\'erieux te}{nemen hij zingt}\\

\haiku{Ik heet Koenraad, zegt,.}{hij ik moet u de groeten}{doen van Tigre}\\

\haiku{Het landhuis van mijn.}{vader en het dochtertje}{van de notaris}\\

\haiku{Ik betrap mij er,.}{op dat ik haar als een dwaas}{zit te bestaren}\\

\haiku{Tegen middernacht.}{kwam ik ziek en dronken de}{slaapkamer binnen}\\

\haiku{Het zijn zij, die op,.}{het juiste ogenblik slaan waar}{zij te slaan hebben}\\

\haiku{Bij C\'eline is ',.}{het steedst onverwachte}{dat je overweldigt}\\

\haiku{Over een half uur ben,...}{ik aan de Pont Neuf waar we}{de metro nemen}\\

\haiku{Bij de uitgang ziet,.}{Tigre dat er een trein}{is aangekomen}\\

\haiku{Zij trok hem tegen.}{zich aan en nestelde haar}{gelaat in zijn nek}\\

\haiku{Zij zijn de plaatsen.}{waar de natuur de mens nog}{zijn crediet verschaft}\\

\haiku{Wat er verpulvert,;}{keert terug tot aarde en}{damp tot vruchtbaarheid}\\

\haiku{De anarchie doet.}{hiermede haar intrede}{in het cultuurbeeld}\\

\haiku{Doch dan weer, haar hand:}{over de zijne schuivend en}{met gedempte stem}\\

\haiku{Zij maakt een einde.}{aan al deze oprechte}{en geveinsde ernst}\\

\haiku{De belastingen,,!}{de flikflooierijen de}{stijgende waanzin}\\

\haiku{Maar sinds maanden is.}{er geen weekheid naar mijn keel}{geweld als deze}\\

\haiku{Het lijkt mij plots of.}{ik van de ene wereld in}{de andere leef}\\

\haiku{Een ontzettende,.}{gemeenzaamheid waarvan de}{tragiek mij aangrijpt}\\

\haiku{Haar handen liggen.}{gevouwen in het smalle}{vlak van haar bekken}\\

\haiku{neen C\'eline, ik;}{verwarm me in het zachte}{vuur van je schoonheid}\\

\haiku{Ik kruip op mijn buik,.}{door het sparrebos greppel}{na greppel zoekend}\\

\haiku{Voordat ik de tijd,}{heb om vier figuren te}{ontwaren vallen}\\

\haiku{Dan hoort hij knallen,,.}{achter zich onder op de}{weg langs het water}\\

\subsection{Uit: Uitdagend spel}

\haiku{Al wie zich tegen,.}{iets verzet heeft er nog niet}{mee afgerekend}\\

\haiku{Een mens wordt altijd.}{meer bepaald door zijn actie}{dan door zijn denken}\\

\haiku{De vrouw bekeek hem.}{even en ging toen door met het}{wassen van de vaat}\\

\haiku{Onder het slechte.}{licht zag Charat zijn vet zwart}{krullend haar glanzen}\\

\haiku{Maar hij bleef op zijn.}{hoede om elke plotse}{slag te ontwijken}\\

\haiku{Hij herinnerde:}{zich wat er gebeurd was en}{voegde eraan toe}\\

\haiku{Maar ik zou ook graag...}{weten of zijn handen er}{verzorgd uitzagen}\\

\haiku{Synthese is maar.}{een voorbijgaande ziekte}{van het intellect}\\

\haiku{{\textquoteright} zei Serge somber, {\textquoteleft}.}{ik voel er niets meer voor om}{nog te antwoorden}\\

\haiku{{\textquoteright} {\textquoteleft}Tegenwoordig zijn,.}{de omgangsvormen vrijer}{dan vroeger Daddy}\\

\haiku{Hij stond klaar om de.}{rechterarm van de Afrikaan}{uit het lid te slaan}\\

\haiku{Die eenzaamheid was.}{iets uit pijn en begeerte}{samengevlochten}\\

\haiku{IX Charat lag naast,.}{een stapel boeken op de}{grond lang uitgestrekt}\\

\haiku{... langgerekt en met.}{een pijnlijke klemtoon om}{nostalgisch te zijn}\\

\haiku{{\textquoteright} terwijl hij in zijn.}{barkast kijken ging wat hij}{hun aan kon bieden}\\

\haiku{Voordat hij de slag,.}{kon afweren dreunde het}{al  in zijn hoofd}\\

\haiku{wie er verder voor.}{jullie komst nog bij Perez}{aanwezig waren}\\

\haiku{{\textquoteright} De man sloop langs de.}{achterdeur naar buiten als}{een geslagen hond}\\

\haiku{De Huid nam de hoorn'.}{op van het toestel dat op}{Perez tafel stond}\\

\haiku{Als een in elkaar.}{gezakt ding ging hij in zijn}{bureaustoel zitten}\\

\haiku{Een diep en duister.}{heimwee doorspoelde hem als}{een vaag zwart water}\\

\haiku{Het gebeurde dat.}{er ogenblikken waren dat}{zij hem verraste}\\

\haiku{dat verticale.}{gevoel van de feniks die}{uit zijn as herrijst}\\

\haiku{Nergens vond hij wat,.}{hij zocht wat hij meende te}{hebben uitgedrukt}\\

\haiku{{\textquoteleft}Ik begrijp niet wat,.}{ik heb uit te staan met uw}{geheimen mijnheer}\\

\haiku{{\textquoteright} Zij bleef bij de deur,:}{van het atelier staan alsof}{zij zichzelf afvroeg}\\

\haiku{Hij ging naar haar toe.}{en liet zijn hand vriendelijk}{over haar hoofd glijden}\\

\haiku{{\textquoteright} {\textquoteleft}Maja,{\textquoteright} vroeg Charat, {\textquoteleft}.}{onverwachtnoem me eens iets}{waar je wel van houdt}\\

\haiku{{\textquoteleft}Soms hield ik met een,.}{grenzeloos gevoel van hem}{soms haatte ik hem}\\

\haiku{Hij aarzelde, maar:}{sprak toen zacht een zin uit die}{in hem opwelde}\\

\haiku{{\textquoteleft}Ik ben op weg naar,{\textquoteright}.}{het Paradijs ik heb mijn}{wandelschoenen aan}\\

\haiku{Serge zag dat hij.}{de oren spitste om niets van het}{gesprek te missen}\\

\haiku{Zij zag een bijna.}{trieste zinnelijkheid in}{zijn vragende blik}\\

\haiku{De werkelijkheid.}{balde zich te zamen in}{dit snikkend lichaam}\\

\haiku{Hij overwoog reeds op.}{welke wijze en met wie}{hij zijn slag zou slaan}\\

\haiku{De gedachte aan.}{een benzinepomp leek hem}{niet onuitvoerbaar}\\

\haiku{Twee agenten slopen.}{in gebukte houding naar}{de Byzantijn toe}\\

\section{E.L. Franken}

\subsection{Uit: Staatsgeheim}

\haiku{{\textquoteright} {\textquoteleft}Ik arme blijf dus,{\textquoteright}.}{met mijn koude dronk alleen}{zuchtte Mijnsbergen}\\

\haiku{{\textquoteleft}Voor mijn part, maar als.}{er wat gebeurt is het op}{jouw verantwoording}\\

\haiku{{\textquoteleft}Denk je dat ze me?}{alleen in het gezelschap}{van die geesten laat}\\

\haiku{Als u zich verder}{nog zoo aanstelt dwingt u mij}{u Knock-Out te slaan}\\

\haiku{Op de andere.}{kant van de straat stonden de}{parkeerende autos}\\

\haiku{Nou wees dan maar blij.}{dat je van avond de eer hebt}{me te leeren kennen}\\

\haiku{Mijn God hoe lang is.}{het geleden dat ik een}{druppel gezien heb}\\

\haiku{{\textquoteright} {\textquoteleft}Die deur ligt aan de.}{achterkant van het huis en}{komt in de tuin uit}\\

\haiku{Op de drempel stond {\textquoteleft}{\textquoteright},.}{met zijnheilig glimlachje}{Cornelis Baron}\\

\haiku{Vorst verstond de kunst.}{de pati\"enten op hun}{gemak te stellen}\\

\haiku{{\textquoteright} Nog lang nadat zij.}{de kamer verlaten had}{dacht hij over haar na}\\

\haiku{ofschoon ik geloof,.}{dat wij ook dien dader nog}{wel zullen pakken}\\

\haiku{een feit is het, dat.}{hij steeds in de buurt van de}{zaak te vinden was}\\

\haiku{{\textquoteright} Verveeld nam de heer.}{uit de portefeuille een}{diplomatentasch}\\

\haiku{Daar buiten in mijn.}{zeilboot heb ik de tijd om}{te leeren berusten}\\

\haiku{{\textquoteright} Een oogenblik dacht,.}{Mary na maar dan knikte}{zij hem alweer toe}\\

\haiku{Daar heb ik mij over,;}{verheugd omdat het onze}{werkkracht verhoogde}\\

\haiku{{\textquoteright} Het bellen van de.}{telefoon sneed den dokter}{ieder antwoord af}\\

\haiku{Toen voelde zij hoe.}{de handen van den man over}{haar lichaam gleden}\\

\haiku{Hij deed het licht uit,.}{sloot de deur af en begaf}{zich naar zijn woning}\\

\haiku{Met lichtveerende.}{stappen sprong hij de trap op}{en opende de deur}\\

\haiku{Haastig scheurde hij.}{de enveloppe open en}{ontvouwde de brief}\\

\haiku{{\textquoteright} Vorst vroeg zich af hoe.}{de brief op zijn schrijftafel}{kon zijn gekomen}\\

\haiku{De spelers spraken - -;}{deels uit afgunst en nijd over}{haar achteloosheid}\\

\haiku{Ik had nog een boek,.}{van U dat wilde ik U}{even terug brengen}\\

\haiku{{\textquoteright} En toen Baron haar,:}{zonder haar te begrijpen}{aankeek meende zij}\\

\haiku{Hij werd aan de kant,;}{geslingerd zijn rechterschoen}{was opengereten}\\

\haiku{Met gedempte stem.}{sprak Grudel in het mondstuk}{van de telefoon}\\

\haiku{De heeren stegen.}{uit en naderden zonder}{te spreken de deur}\\

\haiku{{\textquoteleft}Zoo'n keuken zou men,.}{hier zeker niet verwachten}{evenmin als de rest}\\

\haiku{Maar de moord op Dr.;}{Vorst zou ook uit een ander}{motief denkbaar zijn}\\

\haiku{Voorzichtig, latje,.}{voor latje liet Smit zich naar}{beneden glijden}\\

\haiku{Ik kon niets zien, maar,.}{toch had ik het gevoel dat}{hij er nog moest zijn}\\

\haiku{Als dit antieke!}{kamermeisje nu maar eens}{wat spraakzamer was}\\

\haiku{Het schoot hem opeens,.}{te binnen dat hij met haar}{had afgesproken}\\

\haiku{{\textquoteright} Een klopje op de.}{deur sneed het antwoord van den}{Hoofdinspecteur af}\\

\haiku{Hier kreeg hij voor de.}{tweede keer deze morgen}{een teleurstelling}\\

\haiku{{\textquoteright} {\textquoteleft}Kom, zet nu eens je,.}{detective-oogen op het}{is werkelijk niets}\\

\haiku{En al zou het tot -,.}{een fiasco komen ik}{ben jong inspecteur}\\

\haiku{{\textquoteright} Nijman lachte en.}{ook Smit kon zijn gevoelens}{niet onderdrukken}\\

\haiku{Stuk voor stuk nam Smit.}{ze eruit en liet ze door}{zijn vingers glijden}\\

\haiku{U zult dus Uw dienst.}{bij de Bank ongehinderd}{kunnen voortzetten}\\

\haiku{Veel meer zou het mij.}{interesseeren den heer}{Grudel te spreken}\\

\haiku{De Hoofdinspecteur.}{nam het wapen in de hand}{en onderzocht het}\\

\subsection{Uit: De vulpendetective}

\haiku{{\textquoteleft}Uw vrouw zal zonder,.}{twijfel vinden dat U zeer}{juist heeft gehandeld}\\

\haiku{Met oorverdoovend.}{lawaai zette de wagen}{zich in beweging}\\

\haiku{GEVONDEN: tusschen.}{Haarlem en Amsterdam LEEREN}{DAMESHANDTASCHJE}\\

\haiku{{\textquoteright} Th\'er\`ese Grenier.}{ging in een fauteuil naast de}{schrijftafel zitten}\\

\haiku{Of had U verwacht,?}{dat ik er onmiddellijk}{in zou toestemmen}\\

\haiku{{\textquoteright} Met belangstelling.}{sloeg zij de uitwerking van}{haar woorden gade}\\

\haiku{Alle kleur was uit.}{zijn gezicht geweken en}{zijn handen beefden}\\

\haiku{Onbeweeglijk zat.}{hij aan zijn schrijftafel en}{staarde voor zich uit}\\

\haiku{Hij stak zijn pijp weer,.}{aan maar legde haar na een}{paar trekken weer weg}\\

\haiku{Waarom bracht hij je,?}{niet naar de boot als hij je}{wilde laten gaan}\\

\haiku{{\textquoteright} {\textquoteleft}Het is de eenige,{\textquoteright}.}{belooning die ik vraag zei}{Baron glimlachend}\\

\haiku{{\textquoteleft}Gelooft U niet, dat,?}{het beter zou zijn wanneer}{U mij vertrouwde}\\

\haiku{Tusschen rijdende.}{taxi's en fietsers door liep hij}{naar de Kalverstraat}\\

\haiku{Dat  is nog vroeg,.}{genoeg om het vliegtuig naar}{Londen te halen}\\

\haiku{Ik dacht al, dat U,.}{mij heelemaal vergeten}{had mijnheer Baron}\\

\haiku{Ik zal U in de,.}{gelegenheid stellen er}{over na te denken}\\

\haiku{De dokter heeft een,.}{betere opinie over mij}{dan U inspecteur}\\

\haiku{{\textquoteright} Sadie was naast haar.}{gaan staan en had den arm om}{haar schouder gelegd}\\

\haiku{- De boodschap, die ik,.}{hem nu ga brengen is wel}{een rijksdaalder waard}\\

\haiku{Toen zij het schmink van,:}{haar gezicht verwijderde}{zei ze plotseling}\\

\haiku{{\textquoteright} vroeg Bronsdijk en hij,.}{zag hoe zij bewonderend}{naar het sieraad keek}\\

\haiku{Hij zou misschien wel,.}{in staat zijn geweld tegen}{U te gebruiken}\\

\haiku{{\textquoteright} {\textquoteleft}Dan moest je weten,.}{dat hij een dame ook als}{dame behandelt}\\

\haiku{Maar ik kan je w\`el,.}{zeggen dat ik niet graag in}{zijn schoenen zou staan}\\

\haiku{Bemoei je liever.}{met je eigen zaken en}{laat Bronsdijk met rust}\\

\haiku{Heb je wel eens van,?}{een tijger gehoord die goed}{voor een gazel was}\\

\haiku{{\textquoteright} {\textquoteleft}Al tien dagen lang.}{stuurt hij mij elken avond de}{prachtigste bloemen}\\

\haiku{Hij bewonderde.}{Th\'er\`ese en had eens veel}{van haar gehouden}\\

\haiku{Het was niet moeilijk,:}{voor haar zijn gedachten te}{raden en zij vroeg}\\

\haiku{Maar de boekhouder.}{was onverrichter zake}{teruggekomen}\\

\haiku{Wanneer U teekent,.}{blijft dit papier rustig in}{mijn safe liggen}\\

\haiku{Zoodra Bronsdijk,.}{alleen was begon hij heen}{en weer te loopen}\\

\haiku{Hij drukte op de.}{bel en dadelijk kwam de}{boekhouder binnen}\\

\haiku{heeft mijn chef er over,,?}{gesproken dat hij van plan}{is mij te ontslaan}\\

\haiku{Zij sloot de oogen, haar:}{gezicht was verwrongen van}{smart en zij kreunde}\\

\haiku{- Middeldorp moet haar.}{morgen vroeg om acht uur uit}{haar woning halen}\\

\haiku{{\textquoteright} Intusschen was dr..}{de Jong met het onderzoek}{gereed gekomen}\\

\haiku{- Toen ik de misdaad,.}{ontdekte probeerde ik}{te telefoneeren}\\

\haiku{{\textquoteright} Zij keek hem aan en,.}{uit haar oogen sprak een wanhoop}{die Smit ontroerde}\\

\haiku{Ik kreeg af en toe,,.}{den indruk dat zij het niet}{goed vond dat ik kwam}\\

\haiku{{\textquoteleft}Inspecteur, - is het -, -?}{absoluut noodzakelijk}{dat mijn man hier blijft}\\

\haiku{Ik hoorde, wat er,.}{gebeurd was en wilde haar}{niet alleen laten}\\

\haiku{{\textquoteright} Zij probeerde den.}{knop weer aan de paraplu}{te bevestigen}\\

\haiku{Ik moet U een paar,.}{vragen stellen die juffrouw}{Russell betreffen}\\

\haiku{Niemand weet, waar zij,.}{is want zij heeft geen boodschap}{achter gelaten}\\

\haiku{Als U van mijn schuld,,.}{overtuigd bent is het Uw zaak}{het te bewijzen}\\

\haiku{, vindt U het goed, dat?}{ik hier een paar woorden met}{mijnheer Baron spreek}\\

\haiku{{\textquoteright} {\textquoteleft}Inspecteur, weest U.}{niet zoo geheimzinnig in}{Uw uitdrukkingen}\\

\haiku{{\textquoteright} {\textquoteleft}Het is niet zoo gek,,,{\textquoteright}.}{als U denkt Sadie zei hij}{nauwelijks hoorbaar}\\

\haiku{Als U zegt, dat ik,,}{Amelie er mee help zal ik}{er niet naar toe gaan}\\

\haiku{Maar dikwijls, als zij,:}{met hem samen was en ik}{haar blik zag dacht ik}\\

\haiku{Ik telegrafeer,.}{nu naar de jongens dat de}{zaak in orde is}\\

\haiku{En als je nog wilt - -.}{wat je mij toen hebt gevraagd}{vandaag zeg ik ja}\\

\haiku{De hoofdinspecteur.}{stond op en ging haar een paar}{passen tegemoet}\\

\haiku{Maar dat komt zeker,,.}{door de thrillers die men leest}{en ook door de film}\\

\haiku{Maar zij kon toch niet,.}{gelooven dat Joe deze daad}{zou hebben begaan}\\

\haiku{Als ik zeg elf uur,,.}{dan bedoel ik ook elf uur}{geen minuut later}\\

\haiku{{\textquoteright} Met zijn linker hand.}{nam bij het blad papier aan}{en gaf het van Dam}\\

\haiku{Dank U. - U ziet er.}{eigenlijk veel te netjes}{uit voor een speurhond}\\

\haiku{Ik kon die rommel,,.}{die men mij voorzette niet}{naar binnen krijgen}\\

\haiku{{\textquoteright} Van Dam was bij hem.}{komen staan en nam een van}{de patronen op}\\

\haiku{{\textquoteright} {\textquoteleft}Het was een domheid,.}{haar in de dickey-seat}{te laten zitten}\\

\haiku{{\textquoteright} {\textquoteleft}Je wist, dat ik al.}{die papieren zoolang bij}{mij had gestoken}\\

\haiku{Tot nu toe hebben,{\textquoteright}.}{wij nog niets van hem gehoord}{antwoordde van Dam}\\

\haiku{{\textquoteleft}Ik begrijp er niets,.}{van dat het antwoord uit Ault}{nog niet binnen is}\\

\haiku{{\textquoteleft}Mocht ik over een uur,,.}{niet terug zijn dan weet U}{wat U te doen staat}\\

\haiku{Het leek hem beter,.}{zonder licht de gang van het}{huis te betreden}\\

\haiku{{\textquoteleft}Hallo, inspecteur -!}{U wordt door de hemel hier}{naar toe gezonden}\\

\haiku{Zoo onhebbelijk:}{en scherp mogelijk snauwde}{hij tegen Baron}\\

\haiku{{\textquoteleft}Ik geloof, dat ik.}{mij rustig aan Uw zorgen}{kan toevertrouwen}\\

\haiku{De wagen kwam van;}{de Weteringschans en scheen}{vaart te minderen}\\

\haiku{Hij draaide het licht -.}{op en keek in vijf op hem}{gerichte brownings}\\

\haiku{Ik krijg mijn ontslag - -.}{mijn carri\`ere alles}{is afgeloopen}\\

\subsection{Uit: Wie heeft de admiraal gewurgd?}

\haiku{De kapitein liet.}{den inspecteur en Clive}{Harrow binnengaan}\\

\haiku{Deze heer heeft zijn.}{dekstoel tot ongeveer vijf}{uur niet verlaten}\\

\haiku{{\textquoteright} Zwijgend, zonder hem,.}{aan te zien legde zij haar}{hand op de zijne}\\

\haiku{Ik hoop dat u het,.}{mij niet kwalijk neemt dat ik}{daar niet op antwoord}\\

\haiku{{\textquoteright} {\textquoteleft}Dan zullen wij u,.}{naar het station brengen}{als u het goed vindt}\\

\haiku{Misschien was van Soest.}{op reis en zijn bediende}{daarom met verlof}\\

\haiku{Hoofdinspecteur, de.}{bediende van mijnheer van}{Soest is gekomen}\\

\haiku{Daar zullen wij het.}{op het politiebureau}{nog eens over hebben}\\

\haiku{Tegen half negen,}{ging hij naar de hal maar hij}{zag den man niet dien}\\

\haiku{{\textquoteleft}Het doet mij net zoo,,}{veel pleizier als u en ik}{geloof dat dit komt}\\

\haiku{Zij was een paar jaar,,.}{jonger dan Jeanne slank}{blond en zeer sportief}\\

\haiku{{\textquoteright} {\textquoteleft}Ik geloof, dat u,.}{het meent wanneer u ons met}{de zon vergelijkt}\\

\haiku{En als zij het is,.}{noodigt u haar dan uit bij ons}{te komen zitten}\\

\haiku{Ik heb mij dikwijls,.}{afgevraagd of wij elkaar}{nog eens zouden zien}\\

\haiku{Maar vergeet u niet,.}{dat ik ook nu nog hiertoe}{zou kunnen overgaan}\\

\haiku{dat,{\textquoteright} bromde Wolters,:}{binnensmonds en zich tot van}{Dam wendend zei hij}\\

\haiku{{\textquoteleft}Wat weet u van de,?}{menschen die bij mijnheer van}{Soest over huis kwamen}\\

\haiku{Dan begaf hij zich.}{naar den hoofdcommissaris}{voor een bespreking}\\

\haiku{Smit reed vroeger als.}{anders naar zijn woning in}{de Euterpestraat}\\

\haiku{{\textquoteright} {\textquoteleft}Een vriendin van hem.}{is voor een korten tijd naar}{den Haag gekomen}\\

\haiku{Dit intermezzo.}{won voor hem door Mary's vraag}{nog aan beteekenis}\\

\haiku{{\textquoteleft}Corry, mijnheer hier,.}{vraagt hoelang mijnheer Wolters}{nu al bij ons woont}\\

\haiku{Een paar tafeltjes.}{van Harrow verwijderd zag}{zij Riemsdijk zitten}\\

\haiku{{\textquoteleft}Mister Murphy Trast -,{\textquoteright}.}{juffrouw Jeanne Morrees}{stelde Riemsdijk voor}\\

\haiku{{\textquoteright} {\textquoteleft}Wel - ik heb hem nog.}{nooit zoo gunstig over iemand}{hooren oordeelen}\\

\haiku{Uw vader schijnen.}{de tropen beter te zijn}{bekomen dan mij}\\

\haiku{{\textquoteright} Smit ging het kantoor.}{binnen en zag drie heeren}{tegenover zich staan}\\

\haiku{{\textquoteright} vroeg hij met zachte,.}{stem toen hij met Smit in de}{hal was aangeland}\\

\haiku{Het gaat hier om meer.}{dan de belangen van een}{schietvereeniging}\\

\haiku{Harrow en Sewell.}{zaten in diepe fauteuils}{tegenover elkaar}\\

\haiku{Hij heeft zich - zoover ons -.}{bekend is in ons land aan}{niets schuldig gemaakt}\\

\haiku{{\textquoteright} {\textquoteleft}O, dan is alles,{\textquoteright}.}{te begrijpen antwoordde}{Harrow glimlachend}\\

\haiku{Het heeft er den schijn,.}{van dat jouw loopbaan vooruit}{was afgebakend}\\

\haiku{{\textquoteright} Met een bitteren.}{glimlach had hij de laatste}{woorden gesproken}\\

\haiku{Eindelijk was de,.}{dag gekomen waarop hij}{met smart had gewacht}\\

\haiku{Ik herinner mij,,;}{dat het juist acht uur sloeg toen}{mijnheer Ruissaard kwam}\\

\haiku{Hij keerde naar de.}{bibliotheek terug en}{trad aan het venster}\\

\haiku{Voorzichtig koos hij,:}{zijn  woorden toen hij met}{gedempte stem zei}\\

\haiku{Als een gloeiende.}{bal verzonk de zon aan den}{horizon in zee}\\

\haiku{{\textquoteleft}Ik geloof,{\textquoteright} begon, {\textquoteleft}.}{Ruissaard ten slottedat wij}{gauw klaar zullen zijn}\\

\haiku{{\textquoteright} {\textquoteleft}Ja, waarschijnlijk een,{\textquoteright}.}{dubbelganger zei Ruissaard}{schouderophalend}\\

\haiku{{\textquoteleft}Ik geloof, dokter,,{\textquoteright}.}{dat dit spoedig noodig zal zijn}{antwoordde Smit}\\

\haiku{Hij had tijd genoeg,.}{en gaf zijn chauffeur opdracht}{langzaam te rijden}\\

\haiku{Ik verwijt mijzelf,.}{dikwijls dat ik u veel te}{dikwijls uw zin geef}\\

\haiku{Hij zag duidelijk,.}{dat zij steeds onrustiger}{begon te worden}\\

\haiku{Je mag er niet aan - -?}{te gronde gaan Jeanne}{hoor je wat ik zeg}\\

\haiku{Vanuit een kleine.}{kamer voerde een smalle}{trap naar den zolder}\\

\haiku{{\textquoteright} Langzaam ging de man,.}{zitten zonder Smit met zijn}{oogen los te laten}\\

\haiku{Vertelt u mij dus,.}{zoo kort mogelijk wat zich}{hier heeft afgespeeld}\\

\haiku{Alles, wat de man,.}{naast haar uitsprak waren haar}{eigen gedachten}\\

\haiku{Hij voelde een stoot,.}{en juist toen hij viel trof hem}{een felle lichtschijn}\\

\haiku{{\textquoteleft}Ik weet, inspecteur,.}{dat u mij aan Engeland}{zult uitleveren}\\

\haiku{Pas toen het te laat,.}{was heb ik de geheele}{waarheid vernomen}\\

\haiku{Hoofdinspecteur Smit:}{ziet in deze kwestie twee}{mogelijkheden}\\

\section{Kester Freriks, A.F.Th. van der Heijden, Oek de Jong, Frans Kellendonk, Nicolaas Matsier, Doeschka Meijsing en Geerten Meijsing}

\subsection{Uit: Over God}

\haiku{dat geluk bestaat,.}{uit schijn met een offer moet}{je het afdwingen}\\

\haiku{Ik leg mijn hoofd op.}{haar schouder en neurie een}{liedje in haar oor}\\

\haiku{Ik ben benieuwd of.}{al die mensen nog op het}{plein zijn verzameld}\\

\haiku{Met het naderen.}{van de engelen groeide}{zijn begeerte nog}\\

\haiku{Albert begon de.}{graankorrels van de eerste}{heuvel te tellen}\\

\haiku{{\textquoteleft}U vraagt mij wat het,?}{voortreffelijkste voor de}{mens is majesteit}\\

\haiku{Je houdt onbewust.}{rekening met zijn bestaan}{of je doet het niet}\\

\haiku{De Zoon herrees uit,,.}{het graf en verloste ons}{aldus van de dood}\\

\haiku{De Zoon predikte,:}{de liefde maar het was een}{bepaald soort liefde}\\

\haiku{De Here Here.}{deed mij alras verlangen}{naar verzonkenheid}\\

\haiku{Te ontdekken uit?}{welke ritmes evenwicht en}{gemoedsrust ontstaan}\\

\haiku{Om de Melkweg te.}{beschrijven vergelijk je}{hem met een sluier}\\

\haiku{Het liefst sloop hij door.}{de benauwde kruipgangen}{onder het gebouw}\\

\haiku{Niemand heeft me nog.}{duidelijk kunnen maken}{wat geloven is}\\

\haiku{{\textquoteleft}Het is mij bang om,,;}{u mijn broeder Jonathan}{gij waart mij zeer lief}\\

\haiku{Of deden we het,?}{zelf geschapen naar Zijn beeld}{en gelijkenis}\\

\haiku{Niet was het om de '.}{fraaie tonen ent geluid}{dat ik ontlokte}\\
