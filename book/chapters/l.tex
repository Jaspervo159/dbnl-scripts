\chapter[16 auteurs, 3167 haiku's]{zestien auteurs, drieduizendhonderdzevenenzestig haiku's}

\section{L.H.J. Lamberts Hurrelbrinck}

\subsection{Uit: Het beulsjong}

\haiku{beroerd - lamzalig -;}{toch moesten ze er heen voor hun}{eigen ponteneur}\\

\haiku{{\textquoteright} {\textquoteleft}Allo dan, ik zal, -}{mijn best doen al was het maar}{daarvoor alleen}\\

\haiku{{\textquoteright} {\textquoteleft}Dat wil niet zeggen,.}{dat we het dezen keer ook}{niet zullen hebben}\\

\haiku{Visioenen van, -,.}{smaad en verachting van roem}{en eer toen de man}\\

\haiku{{\textquoteright} {\textquoteleft}Justement, 't doet,}{mij pleizier dat je dat zoo}{goed onthouden hebt}\\

\haiku{Dat - dat nooit meer... maar..., -...}{wat dan wat dan nondediu}{weer d'r van door gaan}\\

\haiku{{\textquoteright} {\textquoteleft}Neen, er ontbreekt niks{\textquoteright},, {\textquoteleft},}{als Wetzels terugkeertgeen}{rooie duit contrarie}\\

\haiku{{\textquoteright} {\textquoteleft}Ja effectief, dat,...}{weet ik maar daarom hoeft het}{toch niet uit te zijn}\\

\haiku{koude windvlagen,.}{welke golven over den zwart}{geregenden grond}\\

\haiku{{\textquoteright} Dan weer zich wendend:}{tot de nog aanwezigen}{met tergenden lach}\\

\haiku{hij had hem  zoo -...}{valsch aangekeken toen hij}{dat zei of Kouwen}\\

\haiku{alles om hem heen;}{overtrokken met gordijn van}{nevelachtig waas}\\

\haiku{{\textquoteright} {\textquoteleft}O dat went wel als - -}{je eenigen tijd hier geweest}{bent kom nou maar mee}\\

\haiku{{\textquoteright} Een woedeblik van;}{Peter naar dien heen en weer}{zwaaienden dronkaard}\\

\haiku{{\textquoteright} {\textquoteleft}Werken Harieke,,.}{net als alle jongens als}{ze van school af zijn}\\

\haiku{Schuchter, verlegen.}{neemt Tineke plaats op den}{nog ledigen stoel}\\

\haiku{-Harieke had}{gelogen en Peter had}{ook gelogen zou}\\

\haiku{hij hun niet, en weer.}{een behoedzaam sluipen door}{de struiken naar huis}\\

\haiku{{\textquoteright} {\textquoteleft}Goed, doen jelui dat,.}{en laat me dan weten wat}{afgesproken is}\\

\haiku{maar dat andere...,.}{neen Tineke dat mag ik}{niet van je vergen}\\

\haiku{hij verdient het voor - '.}{wat hij gepresteerd heeftt}{was mirabilant}\\

\haiku{{\textquoteright} {\textquoteleft}Kan jij hem helpen{\textquoteright},.}{aan een andere manier}{met felle bitsheid}\\

\haiku{{\textquoteleft}Vindt u niet, dat hij,,{\textquoteright}.}{mooi geschoten heeft vader}{na geruime poos}\\

\haiku{{\textquoteleft}Neen, dat is niet waar,{\textquoteright},.}{dat is niet waar met heftig}{vlugge ratelstem}\\

\haiku{{\textquoteright} {\textquoteleft}Ik zal gaan, vader,,;}{sebiet maar dat zeg ik u}{en onthoud het goed}\\

\haiku{Jij zult Hari niet,?}{het erf afdonderen als}{hij hier kwam voor mij}\\

\haiku{een nevelwaas voor; ', '...}{zijn oogent draaitt wentelt}{alles om hem heen}\\

\haiku{Bartels, Hubertus,...}{Bartels gonst en raast het in}{zijn brandenden kop}\\

\haiku{{\textquoteright} {\textquoteleft}Neen, neen{\textquoteright}, met driftig - ',.}{afwerend gebaart is}{niks zeg ik je toch}\\

\haiku{zij zou hem vragen.}{waarom en hij zou weer niet}{kunnen antwoorden}\\

\haiku{Hazen, konijnen,.}{die bengelen aan om het}{lijf vastgesnoerd touw}\\

\haiku{Eindelijk niet meer,.}{zichtbaar die twee voor hen die}{gebleven zijn}\\

\haiku{Zoo den geheelen - -,,;}{dag zou hij ziek zijn Jezus}{Maria Jozef nog}\\

\haiku{als ze serieus.}{ziek was zou Peter hem wel}{gewaarschuwd hebben}\\

\haiku{{\textquoteright} {\textquoteleft}Ja, 't is maar voor,.}{een paar minuten dan kom}{ik het weer halen}\\

\haiku{{\textquoteleft}Wat is er met hem{\textquoteright},.}{gebeurd tot Hari geknield}{aan zijne zijde}\\

\haiku{ik heb je alleen.}{maar gevraagd of je die plaats}{zoudt kunnen wijzen}\\

\haiku{{\textquoteright} {\textquoteleft}Hari, Hari{\textquoteright}, als.}{de gerechtsdienaar de hand}{op zijn schouder legt}\\

\haiku{Ik beloof het u.{\textquoteright}.}{Zoo is de dienaar Gods bij}{Bartels gekomen}\\

\haiku{ik zal haar terstond -,?}{schrijven is er nog iets wat}{ik voor je doen kan}\\

\haiku{{\textquoteright} {\textquoteleft}Zoo absoluut heeft,,.}{hij dat niet gezegd maar wel}{dat de kans bestaat}\\

\haiku{{\textquoteright} Niet spreken - en zij.}{wil hem juist verzorgen om}{hem te doen spreken}\\

\haiku{{\textquoteleft}We wenschen Bartels{\textquoteright},.}{te spreken de oudste der}{heeren tot Peter}\\

\haiku{zij hooren niet het;}{rammelend gepraat van hun}{medereizigers}\\

\haiku{{\textquoteleft}Hoeber{\textquoteright}, terwijl Bertha.}{vat de slap neerhangende}{hand in de hare}\\

\haiku{een beefschrikken, als}{een der knechten hem nadert}{om hem te vragen}\\

\haiku{{\textquoteright} {\textquoteleft}Dat het Hari niet,.}{was die hebben ze dan ook}{al losgelaten}\\

\haiku{Nieuwe angst in hem -!}{zouden ze toch nog daar zijn}{om hem te halen}\\

\haiku{{\textquoteright} {\textquoteleft}Stil nou kinders, stil,.}{nou niet zoo'n spektakel bij}{zoo'n zwaren zieke}\\

\haiku{{\textquoteright} {\textquoteleft}Wat wil je daarmee,{\textquoteright}.}{zeggen Bertus terwijl zij}{hem verbaasd aanstaart}\\

\haiku{{\textquoteright} {\textquoteleft}Als u dat zoo braaf -?}{en christelijk vindt waarom}{doet u het dan niet}\\

\haiku{handen, die drukken,,.}{handen der mannen vrouwen}{die kussen vrouwen}\\

\subsection{Uit: De heks van Heinsbroek}

\haiku{Ongeveer een jaar;}{na hun huwelijk werd hun}{een zoon geboren}\\

\haiku{Een luide juichkreet,,;}{een hartelijk galmende}{lach dien hij uitstoot}\\

\haiku{hij had ze willen,;}{bespieden van nabij hun}{geheimen kennen}\\

\haiku{enkele dagen;}{later was deze zelf in}{het dorp gekomen}\\

\haiku{'t gat in den grond.}{was verborgen achter dik}{begroeid kreupelhout}\\

\haiku{Wie zou hem kunnen,?}{opvolgen wie zou zijn plaats}{kunnen innemen}\\

\haiku{zonder meitskes, da,{\textquoteright}.}{h\"ob ich gaar oet gein spasz in}{antwoordt zijn buurman}\\

\haiku{niets anders dan het;}{krakend geritsel over de}{roode plavuizen}\\

\haiku{Enkelen, die nog;}{trachten te dansen op de}{waggelende beenen}\\

\haiku{er zijn enkele;}{nieuwe huizen gebouwd in}{de kom van het dorp}\\

\haiku{'t Zou werkelijk,.}{niet kwaad zijn als Wessels haar}{eens bij zich riep}\\

\haiku{Zoo bent u daar{\textquoteright}, vangt,, {\textquoteleft}}{hij aan in het Duitsch om haar}{ter hulpe te zijn}\\

\haiku{dat was ook geen werk,.}{voor u want u schijnt mij toe}{eene geleerde vrouw}\\

\haiku{vooral tegen de;}{koorts had ik een uitstekend}{recept gevonden}\\

\haiku{stijf, bewegingloos,.}{staat ze voor hem starend hem}{aan met holle oogen}\\

\haiku{een vaag twijfelen,.}{dat is geslopen in dat}{kinderlijk gemoed}\\

\haiku{{\textquoteright} {\textquoteleft}O Jeui, o Jeui, o,?}{Jeui waat zal er dan van mien}{erme vrouw w\`ere}\\

\haiku{{\textquoteleft}Stil Marieke, stil,;}{ich wil neet da's te dao nog}{oits euver  spriks}\\

\haiku{ich zal et seffens{\textquoteright}:}{in mien gebeibook ligke}{en dan tot Wessels}\\

\haiku{Hij zal de eerste,;}{weken niet kunnen zien heeft}{de dokter gezegd}\\

\haiku{Tevergeefsch heeft;}{Ubachs zijn kameraden dat}{alles gewezen}\\

\haiku{hij drukt haar hand weer,,.}{innig hartstochtelijk steeds}{dank stamelend}\\

\haiku{Marieke{\textquoteright} heeft hij:}{haar toegevoegd met ietwat}{haperende stem}\\

\haiku{Hij houdt plotseling, '.}{zijn woorden in terwijlt}{bloed naar zijn hoofd stijgt}\\

\haiku{een wraak, die leed zou,,.}{doen die zou pijnigen wreed}{verschrikkelijk wreed}\\

\haiku{ze aanschouwt Wessels,;}{vroolijk lachend in blijde}{opgewondenheid}\\

\haiku{{\textquoteleft}Waat h\"ob ich gehoird,{\textquoteright},:}{Marieke roept hij uit de}{verte reeds haar toe}\\

\haiku{Toen ineens weer een,.}{ontzettende angst die mijn}{lach deed verstommen}\\

\haiku{Achter een boom heb,,.}{ik gewacht loerend dat er}{iemand zou komen}\\

\haiku{Mijn broer heeft mij het;}{mij toekomend deel voor de}{voeten geworpen}\\

\haiku{waarom dat je nu.}{eerst erkent de moeder van}{Marieke te zijn}\\

\haiku{zou ik op mijn beurt,?}{nu ook iets mogen weten}{wat ik niet begrijp}\\

\haiku{{\textquoteright} {\textquoteleft}Dat kan ik je niet,,.}{zeggen mijn kind dat moet je}{vader zelf vragen}\\

\haiku{Dan heet zie mich ouch,.}{gelag veur eur deur zie heet}{mich verlaote}\\

\haiku{{\textquoteright} Spoedig is ze weer,.}{terug in de hut die ze}{straks heeft verlaten}\\

\haiku{ontzet aanschouwt hij,;}{dat gevaar dat hem omringt}{van alle zijden}\\

\haiku{Een waanzinnige,}{drift in hem als hij ziet die}{blijdschap als hij hoort}\\

\haiku{nooit zouden zij zich;}{meer neerzetten op die bank}{vlak voor zijn schuilplaats}\\

\haiku{dat woest dansen in;}{de lucht van het aan het koord}{bengelend lichaam}\\

\haiku{wij kunnen hem hier,{\textquoteright}.}{niet laten liggen een der}{gerechtsdienaars}\\

\haiku{{\textquoteleft}U moet er ook uit,,{\textquoteright}.}{burgemeester voegt een der}{gendarmen hem toe}\\

\haiku{hij huivert, als hij,.}{de beide vrouwen ontwaart}{gebukt over een bed}\\

\haiku{zij wil hooren uit,;}{zijn mond of haar kind in het}{leven zal blijven}\\

\haiku{{\textquoteleft}'t waor toch zien,,{\textquoteright}.}{keend al waor er ouch slech}{gewes stottert hij}\\

\haiku{{\textquoteright} {\textquoteleft}Als ik dat gedaan -?}{had wat zou je dan wel van}{mij gedacht hebben}\\

\subsection{Uit: Limburgiana}

\haiku{Ich h\"ob dich genog;}{in de gate gehauwe}{um dat te weite}\\

\haiku{m\`e noe wil ich gei,,.}{ruizie ich goon mit ich h\"ob}{ouch doors es e peerd}\\

\haiku{even draait het den kop,.}{om terwijl het mij aanstaart}{met vreesschuwe oogen}\\

\haiku{{\textquoteright} {\textquoteleft}Negen glazer beer,,?}{Madame verdaolt9}{geer uch neet dao m\`et}\\

\haiku{zoo gesmaden man,:}{dan lijdt het bijna immer}{eensluidend antwoord}\\

\haiku{allebei kunnen.}{ze hun op dit oogenblik}{gestolen worden}\\

\haiku{van allebei is..{\textquoteright}!}{ooze keel neuchter gebleve}{Vervloekte kerels}\\

\haiku{De Hemel zij dank,,,.}{zij is het niet een oude}{man die binnentreedt}\\

\haiku{geer zeet toch geine?}{badraof47 al maakt geer e}{bitteke ambras48}\\

\haiku{{\textquoteleft}'n raar taol, dat,.}{Hollandsch m\`et niks anders es}{fransche weurd er in}\\

\haiku{hij zal derhalve.}{slechts hebben te tellen en}{te teekenen}\\

\haiku{{\textquoteright} {\textquoteleft}Maar verklaar mij dan ',.}{toch ins Hemelsnaam waar}{je man is vrouwtje}\\

\haiku{Alles op zijn plaats,,....}{ja alles terwijl z'n oogen}{vorschend rondstaren}\\

\haiku{Zie zoo, eindelijk,:}{als overal onzichtbaar is}{gemaakt het bewijs}\\

\haiku{In Roosdaal troont de {\textquoteleft}.}{RederijkerskamerIn}{Liefde Bloeiende}\\

\haiku{Toen eindelijk de,;}{plechtige dag maar tevens}{de dag der vreugde}\\

\haiku{wat is dat veur 'n,?}{negerij woe de hoeser}{gein nommers h\"obbe}\\

\haiku{{\textquoteleft}Tr\"uk{\textquoteright} dondert deze,.}{terwijl hij de getrokken}{sabel hoog zwaait}\\

\haiku{Hei, juffrouw, bezeet}{uch dat mer ins good en kint}{geer casueel ouch}\\

\haiku{Zoo steeds in hunne;}{gedachten een beurtelings}{hopen en vreezen}\\

\haiku{de muzikanten;}{brengen de instrumenten}{aan hunne lippen}\\

\haiku{vreugderazend hollen zij}{weg naar alle kanten om}{uit te schetteren}\\

\haiku{veer hauwe ze toen,;}{nog neet die van noe die mer}{altied door scheete}\\

\haiku{in de bevende,.}{perkamenthanden steeds een}{dikken ruwen stok}\\

\haiku{hij heeft er gebracht,;}{z'n enkele povere}{meubeltjes z'n bed}\\

\haiku{we zullen eens zien:}{of je voor de Rechtbank ook}{niets weet -denk d'raan}\\

\haiku{{\textquoteright} {\textquoteleft}De pastoor heeft hier,.}{niets te maken laat die er}{asjeblieft buiten}\\

\haiku{{\textquoteleft}'t Zal veur dich nog{\textquoteright},.}{al schaoi zien zegt een hunner}{tot den kastelein}\\

\haiku{die vrem prei\"en et -,{\textquoteright}.}{us komme opvrete noets}{noets van z'n leve}\\

\haiku{roode haren, die.}{stijf opstaan op het lage}{bollende voorhoofd}\\

\haiku{Hij ziet kinderen;}{van zijn leeftijd met elkaar}{spelen en stoeien}\\

\haiku{de menschen met voor.}{het meerendeel ons geheel}{vreemde gezichten}\\

\haiku{e paar jaor es,}{dich alles verruniweerd}{is dan kinste d'n}\\

\haiku{Van af dat moment,;}{een geheel zich geven zich}{wijden aan dat kind}\\

\haiku{{\textquoteleft}Lieske, ene tourn\'ee{\textquoteright},.}{veur mich roept hij nu zelf tot}{de herbergierster}\\

\haiku{{\textquoteleft}ongelukkige,.}{k\`el ene zoeplap gewore}{allein oet chagrijn}\\

\haiku{{\textquoteright} Een even vragende,.}{blik tot haar man beantwoord}{met korten hoofdknik}\\

\section{Lambrecht Lambrechts}

\subsection{Uit: Het wingewest}

\haiku{want een schoolmeester,....}{krijgt maar elfhonderd frank}{om te beginnen}\\

\haiku{Jasperken is het,.}{eigenlijk die mij op de}{beenen geholpen heeft}\\

\haiku{Ofwel, zie, hij zou!}{er eene moeten opleiden}{naar zijn eigen hand}\\

\haiku{- Een der talen, die.}{het minst gesproken worden}{op het wereldrond}\\

\haiku{Ge zoudt voorwaar niet,!}{zeggen dat hij in een leemen}{huis geboren is}\\

\haiku{- Ik wed dat hij zelfs!}{deken benoemd zal worden}{met den eersten keer}\\

\haiku{zijn broer leest de mis,.}{bij den graaf van Sonnebeek}{die senator is}\\

\haiku{Zij kan v\'o\'orkomen,,,?...}{waar het noodig is de menschen}{te woord staan wa blieft}\\

\haiku{Daarmee zouden de.}{jongens van de straat af en}{de herberg uit zijn}\\

\haiku{Die van den bakker,,;}{zijn niet veel zegt gij en gij}{hebt wellicht gelijk}\\

\haiku{Vrouwen dulden geen,.}{verdeeldheid van gezag heb}{ik altijd gehoord}\\

\haiku{{\textquoteright} zonder te weten,.}{of een rups werkelijk kwaad}{k\`on zijn ja of neen}\\

\haiku{- Zoo lang ze niet met,.}{een anderen getrouwd is}{bestaat er nog kans}\\

\haiku{- Worden ze bij u?}{misschien verkocht tegen drie}{ellen voor een frank}\\

\haiku{Hoe meer kwaad zij over,.}{Jasper hoorde zeggen hoe}{liever zij het had}\\

\haiku{In dergelijke.}{bezigheidjes vond hij een}{roerend genoegen}\\

\haiku{Zij kende dagen,.}{waarop zij buitengewoon}{lekker gemutst was}\\

\haiku{En hennen, ja, ook...}{een tiental hennen zouden}{we kunnen houden}\\

\haiku{En een mooien haan, ' {\textquoteleft}!}{die er tusschen loopt ens}{morgensKukeluuk}\\

\haiku{Tot op zijn derde,.}{vel haddet gij hem moeten}{invetten Alexander}\\

\haiku{{\textquoteright} riep ik onlangs tot,.}{\'e\'en van de drie ik weet niet}{meer welke het was}\\

\haiku{die zal thuis wel niet.}{veel meer dan aardappelen}{en droog brood vinden}\\

\haiku{Fier-gelukkig.}{liet Jasper het mooie beeld van}{hand tot hand rondgaan}\\

\haiku{Een onzer eerste:}{nationale helden}{was een Limburger}\\

\haiku{De mooie mode kwam.}{het getal der uitvoerders}{nog verdubbelen}\\

\haiku{De paarden die de,:}{haver verdienen moeten}{ze ditmaal krijgen}\\

\haiku{- Zulk een vrouw bestaat,!}{in Helseghem niet misschien}{in heel Limburg niet}\\

\haiku{wil dien brief eens even, -.}{doorloopen het werk van een}{concurrent wellicht}\\

\haiku{- Heer Minister, ik,.}{kan de zaak dadelijk klaar}{spinnen zoo gij wilt}\\

\haiku{- Als dat z\`oo is, dan.}{kent de meester wellicht meer}{armoe dan weelde}\\

\haiku{Vergeleken met.}{de bakkersdochters is het}{een geleerde vrouw}\\

\haiku{Die is den heelen!}{dag bezig met jachthonden}{en watersneppen}\\

\haiku{een halfdonkere,;}{cirk waarin een juffrouw over}{een ijzerdraad liep}\\

\haiku{Toen het donker was,}{geworden sneed hij eenige}{bloeiende takken}\\

\haiku{- Ge gelooft toch niet,,?}{kerels dat ge mij iets op}{de mouw zult speten}\\

\haiku{{\textquoteleft}Aan den hoogsten boom!}{van Helseghem moesten ze dien}{schavuit opknoopen}\\

\haiku{Het rende heen en.}{weer in de kamer gelijk}{een das in het krijt}\\

\haiku{Neen, verwacht niet, dat,.}{een ernstig mensch uw taktiek}{goed zal keuren man}\\

\haiku{Te Paschen had}{zij hem verstooten en den}{heelen zomer door}\\

\haiku{En gij, gij zijt maar,,.}{een mensch een zwakke zondaar}{een arme stumperd}\\

\haiku{Gij eet uw hart op,?}{van verdriet en gij wilt het}{mij niet zeggen he}\\

\haiku{En als hij binnen,:}{kwam noodde hij niet altijd}{meer als te voren}\\

\haiku{- Waarom heb ik die?}{vreemde dingen eigenlijk}{allemaal geleerd}\\

\haiku{Maar de verwer was,.}{klaar-nuchter en daarbij}{rap als een weerlicht}\\

\haiku{- De hoovaardigen,.}{zullen vernederd worden}{leert het Evangelie}\\

\haiku{Zondag, na het lof,.}{moet gij beiden eens in de}{pastorij komen}\\

\haiku{Maar dat waagden de.}{schuchtere Limburgers dan}{toch weer niet te doen}\\

\haiku{Ook heeft de Vlaming.}{geen het minste gevoel van}{eigenwaarde meer}\\

\haiku{- En zoo zal het na.}{tien jaren ook gaan in de}{Limburgsche Kempen}\\

\section{Olaf J. de Landell}

\subsection{Uit: De appels bloeien}

\haiku{zijn schuwe, blauwe,.}{ogen de kleine mond en het}{smalle gezichtje}\\

\haiku{Waarop Agneta,.}{zei heel best z\`elf het kind te}{kunnen opvoeden}\\

\haiku{- ~ Achter het huis.}{Wynendael ligt een prachtig}{rosariumpje}\\

\haiku{Ze sloot de deur even, '.}{nadrukkelijk als zem}{had geopend}\\

\haiku{Toen zij de trap af,,.}{liep was ze recht en haar ogen}{keken opgewekt}\\

\haiku{Dit was de eerste,.}{maal dat Coen voor publiek was}{opgetreden}\\

\haiku{Ze vonden er Coen,.}{met schitterende ogen en}{gloeiende wangen}\\

\haiku{Ze had een patroon,.}{gekocht in de stad en wol}{in twee kleuren blauw}\\

\haiku{Hij liet me 'n heel,,!}{lang lint zien van bloemen en}{die waren van glas}\\

\haiku{{\textquoteright} informeerde Coen,.}{daar hij behoefte hieraan}{scheen te gevoelen}\\

\haiku{Tante Agneta.}{wendde haar gelaat naar de}{tuin en belde Braam}\\

\haiku{{\textquoteleft}Ik heb geen vader,{\textquoteright},.}{en moeder meer vervolgde}{Coen en zuchtte diep}\\

\haiku{Omda'k nie deftig,,{\textquoteright}.}{bin zeit me moeder lichtte}{de Lange scherp toe}\\

\haiku{Lientje zuchtte. {\textquoteleft}Tot,{\textquoteright}.}{elke mogelijke prijs}{zei Agneta hard}\\

\haiku{{\textquoteright} Hij overdacht met een,.}{wee gevoel of hij hier lang}{zou kunnen blijven}\\

\haiku{Coen zat rechtop in,.}{bed met warrige krullen}{en een bleek gezicht}\\

\haiku{Niemand had ooit de.}{freules zulke dingen in}{hun gezicht gezegd}\\

\haiku{Beneden, onder,:}{de schemerlamp zei Alexander}{tegen zijn ouders}\\

\haiku{En in 't vervolg.}{ga je niet meer om met die}{proleet van Gaalders}\\

\haiku{Wat deden zij daar,?}{ook op een uur dat niemand}{hen daar verwachtte}\\

\haiku{Doch Coen zag hem met,.}{zulke vlammende ogen aan}{dat Braam terugweek}\\

\haiku{Ze hadden bijna,.}{even lang op het Huis gewoond}{Agneta en Braam}\\

\haiku{Ze waren fragiel,.}{en uit de tijd zoals ze}{daar binnen gleden}\\

\haiku{Wij hebben hem dat,'...}{tweede afscheid niet bespaard}{na zijn ouders dood}\\

\haiku{de klank bleef hangen {\textquotedblleft},,{\textquotedblright}.}{in een blauw bordje metOost}{West thuis best erop}\\

\haiku{Hij vroeg beteuterd,.}{of die vader niet op een}{stoel kon plaatsnemen}\\

\haiku{{\textquoteright} murmelde ze, {\textquoteleft}is,... -{\textquoteright};}{dat die Verbrinke die Coen}{Agneta kuchte}\\

\haiku{Lientje stak de brief,.}{kalm in haar tasje en deed}{er het zwijgen toe}\\

\haiku{Als ik later groot,...,,}{ben mag er niemand weggaan}{die ik aardig vind}\\

\haiku{s Middags kochten.}{zij cadeautjes voor meneer}{Alexander z'n ouders}\\

\haiku{- Alexander, voorzichtig,.}{de deur sluitend wendde zich}{zuchtend naar de trap}\\

\haiku{De Lange had ook,.}{een vuurpijl afgeschoten}{naar eigen zeggen}\\

\haiku{Coen had de geestkracht,.}{niet om een verpletterend}{zwijgen te ontgaan}\\

\haiku{{\textquoteright} zei Klop, en legde.}{liefkozend zijn hand op de}{glanzende zijkant}\\

\haiku{Drie huizen verder '.}{was die morgen een man van}{t dak gedonderd}\\

\haiku{De Lange zat recht,.}{tegenover hem en kauwde}{op een grassprietje}\\

\haiku{Al die tijd hadden -.}{zij verschillende dingen}{gedacht en gezien}\\

\haiku{Meneer Alexander kon,.}{zo hartelijk lachen met}{het hoofd achterover}\\

\haiku{{\textquoteleft}Zouwe die tuntels?!}{van jou nou nog altijd nie}{groot genog weze}\\

\haiku{Dit alles in streng,;}{geheim zoals alles in}{het dorp placht te gaan}\\

\haiku{Dat veroorzaakte.}{een nadenkend zwijgen bij}{de tegenpartij}\\

\haiku{Ten slotte zat Coen,;}{op een dun-potig stoeltje}{wit-met-goud}\\

\haiku{{\textquoteleft}We zijn er nog te...,{\textquoteright}.}{jong voor en wilde daarmee}{zijn afschuw sussen}\\

\haiku{Nee, Coen meende wel.}{zijn ganse leven alleen}{te zullen blijven}\\

\haiku{{\textquoteright} zei boven haar de,.}{stem die eensklaps zo z\'e\'er op}{de hare geleek}\\

\haiku{dat ik 'm van u -?...}{k\'o\'op en d'r nog een m\`ens van}{probeer te maken}\\

\haiku{Maartje tolde terug.}{in haar slaafse toewijding}{aan de familie}\\

\haiku{{\textquoteleft}Alles,{\textquoteright} fluisterde,.}{ze en begon haastig de}{scherven te ruimen}\\

\haiku{- ~ In de loop van.}{de namiddag werd er aan}{de voordeur gescheld}\\

\haiku{Woarom h\`e' je ze '??!}{nie stijf gevloekt en de oge}{uitr kop gesch\`opt}\\

\haiku{Op meneer Alexander,;}{die zo-maar trouwt terwijl}{ik er niets van weet}\\

\haiku{En ik dacht erover,...{\textquoteright} {\textquoteleft}?!}{dat ik zelf misschien nooit zou}{trouwenWoarom niet}\\

\haiku{een hoofd groter dan,.}{Kaarsberg jong en zich bewust}{van zijn lichaamskracht}\\

\haiku{Zij liepen hard door,.}{het weiland en sprongen hoog}{en v\`er over slootjes}\\

\haiku{{\textquoteleft}Ook met de handen.}{{\`\i}n mijn zakken ben ik een}{Borgh van Wynendael}\\

\haiku{Hadden ze dan niet,?}{steeds het goede gedaan voor}{deze jongen}\\

\haiku{{\textquoteleft}En d\`enk erom, neef,{\textquoteright}, {\textquoteleft}.}{Barend drong Agneta aan}{n{\'\i}\'et per trein terug}\\

\haiku{Van zijn gastvrouw kreeg.}{hij een haastige handdruk}{met een  glimlach}\\

\haiku{{\textquoteleft}We zullen toch maar,,{\textquoteright}.}{pogen iets beters voor je}{te vinden zei hij}\\

\haiku{In stilte hoopte,:}{hij dat Peun niet aan dat pak}{mee-betaalde}\\

\haiku{Nicht Ida wuifde nog,.}{even doch de tram moest wachten}{op een verkeerslicht}\\

\haiku{De stenen van dit!}{huis laten elkander los}{van liefdeloosheid}\\

\haiku{En al die tijd heb,!}{ik gewerkt en gestudeerd}{om niet te sterven}\\

\haiku{Gelukkig begon,.}{hij toen weer over geesten met}{bloedende monden}\\

\haiku{Een stuk jeugd viel van -.}{hem af en de Lange moest}{daar even van stilstaan}\\

\haiku{Bij ons ligt brood nooit,{\textquoteright}, {\textquoteleft} -}{op de grond zei hijen wij}{gooien er niet mee}\\

\haiku{Onbegrijpelijk,.}{dat hij daarnet over de dood}{had lopen soezen}\\

\haiku{{\textquoteleft}Mar Jezus, jonker, ',{\textquoteright}.}{t is bij tiene zei de}{politie-agent}\\

\haiku{Elk in zijn branche,.}{zei vrolijk dat het wel in}{orde zou komen}\\

\haiku{{\textquoteleft}Ik bin d'r kepot,{\textquoteright}.}{van vervolgde de Lange}{met een lage stem}\\

\haiku{{\textquoteleft}Maar ik kan niet zo, '.}{erg lang want ik m\'o\'et nogn}{massa werk leren}\\

\haiku{Daar had hij altijd,,.}{geweten wat hij wilde}{zeggen en zou doen}\\

\haiku{langs slingerende,.}{paadjes achter in de tuin}{waar het hout dicht was}\\

\haiku{als kind had hij er.}{niet mogen spelen omdat}{er zoveel glas was}\\

\haiku{de sfeer in huis was.}{er een van plechtig bezoek}{en ongewoonte}\\

\haiku{Het was misschien wel,.}{gek zo op stel en sprong het}{huis te verlaten}\\

\haiku{hel, purper-roze,,.}{met schichten van het witste licht}{dat er kan bestaan}\\

\haiku{Het ontbijt verliep,.}{zo snel dat Coen geen tijd vond}{om na te denken}\\

\haiku{{\textquoteright} Zijn vriend bekeek hem:}{afkeurend en gaf als enig}{antwoord te kennen}\\

\haiku{{\textquoteright} Toen klommen ze uit.}{de wagen en gingen aan}{de wegkant zitten}\\

\haiku{{\textquoteleft}Dat heb ik toch m'n,{\textquoteright}.}{hele leven gedaan gaf}{Coen vriendelijk toe}\\

\haiku{Er tjilpte ergens,.}{een schelle vogel-stem}{en Coen glimlachte}\\

\haiku{de mond gaf niet af!).}{en zonk schokkend in het open}{portier op de mat}\\

\haiku{{\textquoteleft}Als u maar niet zo,,,}{boos was zou ik vragen of}{u met me wilt eten}\\

\haiku{Hij had eigenlijk.}{alleen maar de warme toon}{van haar stem gehoord}\\

\haiku{{\textquoteright} Maar daarna bedacht,.}{hij hoeveel kostbaar eten ze}{hem had afgestaan}\\

\haiku{Ik zou 't prettig,{\textquoteright}.}{vinden zei Verbrinke met}{zijn warmste glimlach}\\

\haiku{een vrouwenlach en,.}{een babbelend kwetterend}{kinderstemmetje}\\

\haiku{{\textquoteleft}Nee,{\textquoteright} zei de Lange, {\textquoteleft}{\textquoteright}!}{botik k\`en dat. Toen moest Coen}{d\`at nog vertellen}\\

\haiku{Hoe kan een mens zo!}{veel ervaringen in zo'n}{korte tijd opdoen}\\

\haiku{Hoewel hij de pijn.}{in haar blik niet zou hebben}{kunnen verdragen}\\

\haiku{Elisa en Alexander;}{keken haar met grote ogen}{aan en glimlachten}\\

\haiku{Hij schokte op, en,.}{ze had durven wedden dat}{hij alweer rood werd}\\

\haiku{Ze hielden zulke, '.}{gesprekken samen als Coen}{s morgens opstond}\\

\haiku{Na een h\'e\'el zachte,,:}{mislukte klap van hem zei}{de bokser nog eens}\\

\haiku{maar hij zweeg, en kon,.}{niet voorkomen dat hij n\`og}{eens hevig bloosde}\\

\haiku{{\textquoteright} vroeg Coen, zijns ondanks.}{geboeid door de ernst van de}{Lange z'n gezicht}\\

\haiku{Toen hij ongeveer,:}{acht maanden les had zei de}{danser op een keer}\\

\haiku{Neef Barend tikte {\textquoteleft}...{\textquoteright}.}{aan zijn hoedrand en zeiD\`aggg}{en stapte voorbij}\\

\haiku{Hij had zich alles.}{duidelijk door Verbrinke}{laten uitleggen}\\

\haiku{Nu blijft het boven,{\textquoteright},.}{nog even na-klinken dacht}{Coen met bonzend hart}\\

\haiku{{\textquoteleft}Maar dat is toch geen!}{excuus voor dergelijke}{nalatigheden}\\

\haiku{Hij keek de kring rond,.}{zoals hem nooit tevoren}{vergund was geweest}\\

\haiku{Zij liet het huis zien.}{met de onpersoonlijkheid}{van een rondleidster}\\

\haiku{{\textquoteleft}Die oude mensen,}{moeten een geweldige}{staat hebben gevoerd}\\

\haiku{Maar dat besefte,.}{Coen niet en het zou hem ook}{niet hebben geraakt}\\

\haiku{Iedereen kon n\'u,.}{begrijpen dat hij Mona}{alleen bedoelde}\\

\haiku{{\textquoteright} Hij glimlachte ook,.}{maar deed geen moeite om het}{vriendelijk te doen}\\

\haiku{Maar dat was niet het,,.}{antwoord wat iemand jou had}{mogen geven toen}\\

\haiku{- Dit gebeurde de,.}{zesde dag dat Coen in de}{revue danste}\\

\haiku{{\textquoteleft}Omdat ik het heb,.}{met die arme boeren die}{teveel betaalden}\\

\haiku{{\textquoteleft}Mooi is belangrijk,,{\textquoteright}.}{in het lelijke leven}{leerde Mona hem}\\

\haiku{En het brugje over,.}{de gracht dat tante Lientje}{had laten leggen}\\

\haiku{De vier hoeken van.}{de lijst droegen het wapen}{Borgh van Wynendael}\\

\subsection{Uit: Ave Eva}

\haiku{En Eva begreep, hoe.}{grof hij zou zijn tegen de}{eerstvolgende klant}\\

\haiku{{\textquoteleft}Ja,{\textquoteright} stemde Sally,.}{in vol medeleven met}{Eva's medeleven}\\

\haiku{{\textquoteleft}Zeg, Eef, denk erom,,.}{dat je tegenover Derk de}{brutale speelt hoor}\\

\haiku{Ze had geen tijd om.}{te lachen of te huilen}{of adem te halen}\\

\haiku{Het is eigenlijk,.}{voor niemand een kunst op die}{manier mooi te zijn}\\

\haiku{Er klonk flirt in zijn,.}{stem wat zij tot elke prijs}{wilde omzeilen}\\

\haiku{Het was lang niet zo,.}{moeilijk als ze gedacht had}{met hem te praten}\\

\haiku{En als ik denk, dat,,.}{het genoeg is dan scheur ik}{me los dat het knalt}\\

\haiku{Toen begreep Eva pas,,.}{dat hij niet om haar woorden}{lachte maar om h\'a\'ar}\\

\haiku{'t Was wel weer een,,!}{leugen maar ach aan een boom}{zo volgeladen}\\

\haiku{{\textquoteright} en daar moest Derk weer,.}{zo ontzettend om lachen}{dat hij haar losliet}\\

\haiku{{\textquoteright} {\textquoteleft}En laat een kamer.}{in orde maken voor dit}{schattige meisje}\\

\haiku{Ze stond gewoonweg,?}{in brand van het blozen want}{wat was dat voor taal}\\

\haiku{Ze zag zichzelf al.}{onbeheerst liggen ronken}{tegen Derks arm}\\

\haiku{Zowel de vader,:}{als de moeder lachte en}{haar verloofde zei}\\

\haiku{{\textquoteright} bekende mevrouw,.}{Van Hellenduyn met iets van}{eerbied in haar stem}\\

\haiku{En morgen was het,...}{zaterdag daar kon Eva dus}{niets tegen hebben}\\

\haiku{Hij zuchtte en sloeg.}{een arm om haar middel en}{trok haar naar zich toe}\\

\haiku{Maar Eva kon niet met,.}{zekerheid vaststellen dat}{hij had gelachen}\\

\haiku{Ze vertelde hem.}{van haar vader en tante}{Gien en het kantoor}\\

\haiku{De trekpot deed zijn,.}{plicht met choquant geluid in}{de zware stilte}\\

\haiku{Derk zette haar de.}{volgende ochtend in de}{stad af bij Sally}\\

\haiku{{\textquoteleft}En nou is 'ie al,{\textquoteright}.}{veertig jaar met haar getrouwd}{besloot Derk bedrukt}\\

\haiku{En te weinig geld.}{gaf hij slechts als zij teveel}{uitgegeven had}\\

\haiku{De hit, die liep te,.}{slenteren schoot ervan in}{een haastige stap}\\

\haiku{Hoeveel?{\textquoteright} {\textquoteleft}Nou,{\textquoteright} - Eva dacht;}{aan Amanda P\`eche en aan}{Monsieur Fran\c{c}ais}\\

\haiku{{\textquoteleft}Lach maar gerust, - dacht,?!}{je dat iedereen jou hier}{au s\'erieux neemt}\\

\haiku{En ze voelde zich.}{doodongelukkig in haar}{kant en juwelen}\\

\haiku{{\textquoteright} Z'n haar had alle,.}{brillantine verzaakt en}{golfde om zijn hoofd}\\

\haiku{Het werd tijd, dat ze.}{het pad der dame trachtte}{terug te vinden}\\

\haiku{Was het niet, om \'uit,?}{te stappen en desnoods naar}{de stad te lopen}\\

\haiku{Op dergelijke,.}{lichtzinnige taal wist ze}{niets te antwoorden}\\

\haiku{Ze stonden in de,.}{hal en haar begeleider}{knipte het licht aan}\\

\haiku{Maar dat andere - - -.}{meisje en Derks gedrag}{dat stak haar het meest}\\

\haiku{{\textquoteleft}Ik heb een heerlijk,{\textquoteright}.}{ontbijt voor je trachtte Derk}{haar te verlokken}\\

\haiku{{\textquoteright} {\textquoteleft}Och joa, ieder wat,{\textquoteright}.}{antwoordde Gert P. met \'e\'en}{dichtgeknepen oog}\\

\haiku{Was het wel beleefd,,?}{om zo te schateren toen}{ze over de fuif sprak}\\

\haiku{Daar was gelukkig.}{afleiding genoeg om wat}{op adem te komen}\\

\haiku{Pa bok wist van niets,.}{meer dan van horenstoten}{en mummelde woest}\\

\haiku{Eva dacht, dat hij het.}{meest ge\"ergerd was over zijn}{eigen warme kop}\\

\haiku{Er zou wel iets niet,.}{in orde zijn dat er geen}{receptie van kwam}\\

\haiku{{\textquoteright} En daar moest pa Van.}{Hellenduyn zijn wenkbrauwen}{weer van rimpelen}\\

\haiku{Verschil in leeftijd,{\textquoteright}.}{misschien dacht mevrouw Van der}{Dam pessimistisch}\\

\haiku{{\textquoteright} kon niet verhoeden,.}{dat mevrouw Van Hellenduyn}{licht gaf van vreugde}\\

\haiku{Op 't laatst kon je '.}{wel voort hele leven}{gekke ogen zetten}\\

\haiku{Het was duidelijk,.}{dat zijn trots hierbij niet op}{stal kon blijven staan}\\

\haiku{En de verliefde,.}{roos die het tuinpoortje zo}{lyrisch omhelsde}\\

\haiku{Had hij niet gezegd, {\textquoteleft}{\textquoteright}?}{van dinsdagnacht en had zij}{nietja geantwoord}\\

\haiku{{\textquoteright} zei Derk voor zich heen,.}{en striemde het rulle zand}{met een korenaar}\\

\haiku{En Aatje Broek vond '.}{t meisje van meneer Derk}{een fijne dame}\\

\haiku{{\textquoteright} {\textquoteleft}'t Lijkt me heerlijk,,{\textquoteright},.}{als er niemand komt overwoog}{Eva veel te verhit}\\

\haiku{{\textquotedblright}{\textquoteright} En een vrachtrijder,,:}{overtuigende schakel met}{de stad had beweerd}\\

\haiku{Dat was hoog nodig,.}{want de oude Eva groeide}{door de nieuwe heen}\\

\haiku{{\textquoteright} wuifde mevrouwtje,.}{Van Hellenduyn vanaf de}{marsepeinen stoep}\\

\haiku{De auto schoot met,.}{zo'n vaart weg dat tante's hoofd}{bijna achterbleef}\\

\haiku{Het was helemaal,!}{niet aardig want er kon toch}{best iets gebeuren}\\

\haiku{{\textquoteright} gilde tante aan,,.}{Derks oor met priemende}{vijandige ogen}\\

\haiku{Ze werd zolang bij,.}{Sally gedeponeerd die}{druk zat te typen}\\

\haiku{Sally vroeg niets en,.}{zei weinig want haar nieuwe}{artikel moest af}\\

\haiku{Zij beheerste zich voor, {\textquoteleft}}{\'e\'enmaal niet en gaf een}{klap onder zijn hand.}\\

\haiku{Ja, Derk had haar maat,.}{onthouden van die eerste}{ring met amethysten}\\

\haiku{Lang zal ze leven{\textquoteright},.}{hoewel het orgel een aria}{van Tosca draaide}\\

\haiku{Eerst vader, toen de ', - - {\textquoteleft}!}{baas vant kantoor enDat}{vervloekte geklets}\\

\haiku{Toen zag ze Derks,,:}{wit verbeten gezicht en}{restaureerde weer}\\

\haiku{{\textquotedblleft}Ga nou eindelijk,.}{eens mee naar huis nu heb je}{het niet te druk meer}\\

\haiku{12 {\textquoteleft}Eefje,{\textquoteright} zei Derk,, {\textquoteleft}}{plechtig toen ze in de tuin}{waren aangeland}\\

\haiku{Zij probeerde met ',.}{m te praten zonder dat}{de baas het merkte}\\

\haiku{Ze wilde het graag,.}{vergeten maar kon aan niets}{anders meer denken}\\

\haiku{ze moest er haar neus,.}{van snuiten want alles had}{zo mooi kunnen zijn}\\

\haiku{{\textquoteright} Eva had het gevoel,,.}{of ze op haar hoofd stond en}{kon geen woord spreken}\\

\haiku{En wie weet, hoeveel!}{er nog met achttienhonderd}{gulden te doen viel}\\

\haiku{Als ik je met iets,, '.}{kan helpen dan weet je dat}{k voor je klaar sta}\\

\haiku{{\textquoteleft}Vader is over de,.}{kop geweest en dat had ik}{moeten begrijpen}\\

\haiku{Waarom gaat een man, '?}{houden van juist die vrouw die}{m niet wil hebben}\\

\haiku{Ze stond op en even.}{staarde ze peinzend over Eva's}{hoofd heen het raam uit}\\

\haiku{En op dat ogenblik.}{werd ze pas verblind door de}{gloed van het geluk}\\

\haiku{Daarom wist ik toen,,.}{opeens dat je geschikt was}{om Derk te helpen}\\

\haiku{En begon maar vast.}{de haakjes van Eva's grijze}{ruit los te maken}\\

\haiku{{\textquoteleft}Het is jammer, dat,{\textquoteright}.}{je zo weinig met hem op}{hebt zei Sally zacht}\\

\haiku{En Sally had de,...}{brief waarin Derk zijn liefde}{voor Eva bekende}\\

\haiku{Pas loater is, '...}{de dokter gekommes}{morgens om drie uur}\\

\haiku{Die richtte zich op.}{en duwde het baretje}{in de juiste stand}\\

\haiku{Ze begroette Eva,.}{achteloos alsof deze}{niet was weggeweest}\\

\haiku{en Aatje Broek had '.}{rimmetiek en haar dochter}{n zere vinger}\\

\haiku{{\textquoteleft}Zijn jullie daar nog,,?}{monsters winterwortels en}{zure augurken}\\

\subsection{Uit: Blonde Martijn}

\haiku{Ik kreeg een kleur en.}{probeerde een andere}{kant op te kijken}\\

\haiku{Zijn blik rustte kalm {\textquoteleft}?}{en vriendelijk op me.Wat}{is er aan de hand}\\

\haiku{{\textquoteleft}Dit is gedaan op.}{de dag dat Nellie in haar}{wasbeurt werd gestoord}\\

\haiku{Want o, wat weet ik -.}{nog goed dat verhaal het {\`\i}s}{Nellie van der Grijp}\\

\haiku{God, wat heb ik veel,.}{van hem gehouden in die}{schaarse ogenblikken}\\

\haiku{Hij moet een hele.}{tijd doodstil op de zolder}{hebben gezeten}\\

\haiku{een appelige.}{rijpheid was met het blote}{oog kenbaar aan haar}\\

\haiku{De werklui hebben;}{het dakbeschot nu bijna}{geheel verwijderd}\\

\haiku{En opeens heeft hij -.}{er eentje gepakt fel en}{raak om haar middel}\\

\haiku{{\textquoteleft}Eigenlijk heb jij,...{\textquoteright}}{me ook gekust toen je die}{lippen aanraakte}\\

\haiku{Maar ja, ik was toen,.}{pas zestien en Martijn was}{al een-en-twintig}\\

\haiku{-        5 Zijn tante,,.}{mevrouw Van Haysmaal heb ik}{pas later ontmoet}\\

\haiku{Toen kwam mijn vader,.}{net binnen niets wetend van}{enig ernstig gesprek}\\

\haiku{Iets heel liefs en toch,.}{flink en sterk waarmee je kon}{stoeien en praten}\\

\haiku{{\textquoteright} Dat Martijns pink het,.}{nadeed maakte de zaak slechts}{begrijpelijker}\\

\haiku{- Hij liep het huis uit,.}{klom op z'n fiets en hijgde}{naar de politie}\\

\haiku{als vrouwen onder.}{elkaar kon je de details}{beter bespreken}\\

\haiku{Hij had zijn woede.}{gelaafd met de aanblik van}{moeder en dochter}\\

\haiku{De zwijgzaamheid zonk.}{met steeds zwaarder weefsel op}{Alice Bronneberg}\\

\haiku{Die gewoon naar het.}{buffet ging en er in een}{lade rommelde}\\

\haiku{Hij schoofde lade.}{dicht en wendde zijn ogen naar}{Alice Bronneberg}\\

\haiku{Later trof ik mijn,.}{moeder terwijl ze stond te}{telefoneren}\\

\haiku{je bent een stukje,!}{stille lente waar ik van}{zou kunnen janken}\\

\haiku{Angelique was:}{toen op zijn schoot gaan zitten}{en had geantwoord}\\

\haiku{Dat mocht niet van mijn -.}{moeder ik drentelde dus}{niet te dicht bij huis}\\

\haiku{{\textquoteleft}Hij houdt zo veel van,{\textquoteright}.}{Emilie van Wijdevelden}{fluisterde Agnietje}\\

\haiku{Van de werkster, die,;}{dikke Marie moest helpen}{tweemaal in de week}\\

\haiku{{\textquoteleft}Martijn,{\textquoteright} zei ik toen, {\textquoteleft},?...}{eindelijk als een ventje}{houdt van Emilie h\`e}\\

\haiku{Hoewel ik toch zelf.}{zo ten diepste was geboeid}{door Agnietje Weisse}\\

\haiku{Je hebt gedacht dat;}{je een uurwerk was om de}{tijd aan te wijzen}\\

\haiku{Wij weten ook, dat,...}{dit een vliegtuig is en dat}{er een mens in zit}\\

\haiku{Wellicht was dat de,.}{grief waarom het stuk naar de}{zolder was verhuisd}\\

\haiku{{\textquoteleft}Ja, \`of je weet wat,;}{hij gedaan heeft en dan moet}{je me waarschuwen}\\

\haiku{En als je denkt dat,.}{ik het niet begrijp moet je}{het me uitleggen}\\

\haiku{Hij zuchtte ervan,.}{en haastte zich om het huis}{heen naar het balkon}\\

\haiku{De notaris zou.}{inlichtingen inwinnen}{over een avondcursus}\\

\haiku{Dezelfde brede, -.}{soepele mond iets verhard}{door ervaringen}\\

\haiku{Hij blikte van haar,.}{weg over de Meent waar paarden}{liepen te grazen}\\

\haiku{{\textquoteright} want ik vond het maar,.}{een bedompt hokje daar aan}{het eind van de gang}\\

\haiku{Ze konden erop.}{zweren blonde Martijn te}{hebben aangepakt}\\

\haiku{En nu zou hij de.}{stilte gaan vieren in de}{Blauwe Kamer}\\

\haiku{{\textquoteright} Achter mij hoorde,.}{ik Bart lachen ver boven}{het radiootje uit}\\

\haiku{waarom sloeg ze niet,?}{een arm om me heen en vroeg}{gewoon naar Martijn}\\

\haiku{Op een avond waren.}{er twee vriendinnen bij mijn}{moeder op bezoek}\\

\haiku{Het drong zuur door hout.}{en kalk en vond mij in m'n}{eigen kamertje}\\

\haiku{{\textquoteleft}Dat is h{\'\i}j,{\textquoteright} zei ik,.}{in een wrange poging om}{iets te doorbreken}\\

\haiku{Ze legde haar wang.}{tegen de mijne en hield}{me tegen zich aan}\\

\haiku{En dan komen er.}{weer andere auto's die}{het puin ophalen}\\

\haiku{{\textquoteright} Of een laatste wuif.}{van een lichaamsdeel dat het}{niet wil afleren}\\

\haiku{U hebt waarschijnlijk,{\textquoteright}.}{nooit een vriend verloren door}{de dood zei ik dan}\\

\subsection{Uit: De dief stelen}

\haiku{{\textquoteright} {\textquoteleft}Lafbek,{\textquoteright} gromde Pol, {\textquoteleft}!}{om zo'n kind met die goudsbloem}{te laten trouwen}\\

\haiku{{\textquoteleft}We mogen in geen,,}{geval vergeten dat de}{goede God ons ziet}\\

\haiku{{\textquoteleft}Ze moet veel gekker,{\textquoteright}, {\textquoteleft}!}{doen dacht Polanders wordt ze}{zenuwpati\"ent}\\

\haiku{Hij keek er kies bij,.}{op z'n bestelling alsof}{hij deze oplas}\\

\haiku{Morgane boog zich:}{naar Pol en vroeg met weer die}{duidelijke stem}\\

\haiku{En een salade..., -....}{Italienne en eh une}{fromage vari\'ee}\\

\haiku{Maar ja, ze was zo,....}{dom telkens die garenklos}{te laten vallen}\\

\haiku{zodat hij als een!}{vlieg langs de muur naar boven}{zou kunnen lopen}\\

\haiku{{\textquoteleft}Alors,{\textquoteright} Toine haalde, {\textquoteleft} -.}{diep ademkus de Madonna}{je mag niet vallen}\\

\haiku{Bah! - Morgane trok:}{het licht uit met het gouden}{koord boven haar bed}\\

\haiku{{\textquoteleft}Ja, velen zien mij,,}{voor een dichter aan maar ik}{ben meer een opener}\\

\haiku{Ik heb u trouwens,.}{vanavond al verteld dat ik}{niet dicht als ik werk}\\

\haiku{De gasten schreden.}{op droombenen  aan en}{groetten Pol beleefd}\\

\haiku{Morgane trachtte.}{hem kennelijk in een val}{te laten lopen}\\

\haiku{Nee, niet die witte -,.}{de gouden waar je straks mee}{in je handen stond}\\

\haiku{{\textquoteleft}Als de inbreker,.}{succes had gehad zou dit}{zijn winst zijn geweest}\\

\haiku{met een schatrijke,,.}{vrouw die nu al bedacht wat}{hij moest aantrekken}\\

\haiku{Ido liet zijn ogen uit {\textquoteleft}?}{de azuren lucht neerdalen}{op Pol.Wat zeg je}\\

\haiku{Vlot, vormelijk, met.}{rollende zinnen en een}{ouwelijk handje}\\

\haiku{{\textquoteleft}Ik wou graag, dat je,.}{met me meeging papa en}{mama afhalen}\\

\haiku{{\textquoteright} Deze opmerking.}{lichtte de ontmoeting in}{de trein wel schel uit}\\

\haiku{Ik zal haar zeggen,,.}{dat ze de buitendeur los}{moet laten vannacht}\\

\haiku{Moest ze het eigen.}{bloed niet loslaten aan de}{Middellandse zee}\\

\haiku{Een proleet, zeg, die!...}{op straat ligt met allerlei}{goedkope slungels}\\

\haiku{Elke gast kon met.}{de directie een ander}{code vaststellen}\\

\haiku{Ik wil, dat jullie,{\textquoteright}.}{verdomd goed op me passen}{zei Pol vermanend}\\

\haiku{{\textquoteright} en hij legde de.}{hevig murmelende hoorn}{tevreden terug}\\

\haiku{En heeft de sterke?}{dame je voldoende met}{rust kunnen laten}\\

\haiku{Toen er negentien,;}{schelpjes in haar zak zaten}{was de zon onder}\\

\haiku{Morgane was uit -.}{wandelen met Van Aadel wat}{haar geen plezier deed}\\

\haiku{{\textquoteright} converseerde Pol. {\textquoteleft}, -!}{Kijk straks bruiste de regen}{en w\`eg is alles}\\

\haiku{Je moet je veilig...}{stellen voor je ouders en}{hun beslissingen}\\

\haiku{Ido was geslagen,.}{weggegaan en Pol had een}{gedicht geschreven}\\

\haiku{{\textquoteleft}Heb je een boompje,?..}{kunnen vinden waarvan de}{vruchten rijp waren}\\

\haiku{Een gevoel of je,?...}{in bloei staat als een van die}{struiken buiten h\`e}\\

\haiku{Ja, oppassen dat....}{niemand anders hem bereikt}{met een beter bod}\\

\haiku{De titel - nee, dat?...}{ik uw dochter dat boekje}{ten geschenke geef}\\

\haiku{{\textquoteleft}Een vrouw gaat voorbij,,{\textquoteright}.}{en haar mantel is van puur}{goud vertelde hij}\\

\haiku{Hij grijpt de vrouw aan.}{en ontneemt haar het witte}{kleed met de parels}\\

\haiku{En ze m\'o\'est hem in -.}{zijn beweging volgen het}{gelukte haar ook}\\

\haiku{Ze vroeg zich met angst,.}{af wat d\`at nu weer had te}{betekenen}\\

\haiku{Er was dus iemand,.}{die hem in de gaten hield}{en nut van hem had}\\

\haiku{{\textquoteright} Hij mocht niet geheel -.}{en al loochenen dat kon}{gevaarlijk blijken}\\

\haiku{Ido had hem een paar,:}{keer strak aangekeken en}{\'e\'enmaal gezegd}\\

\haiku{De nabije lantaarn.}{stond in een omarming van}{oleander-groen}\\

\haiku{Hij stapte ferm door,.}{perken en gazons tot hij}{onder de vlieg stond}\\

\haiku{Lieve kind,{\textquoteright} zei Pol, {\textquoteleft}}{met een grinnik die alle}{ernst moest ontberen}\\

\haiku{{\textquoteright} Hij bekeek het stuk.}{met welbehagen en gaf}{het haar toen terug}\\

\haiku{{\textquoteright} zei ze nadenkend, {\textquoteleft},.}{vooral als we weten dat}{d\`at waarheid behelst}\\

\haiku{Meneer Merlin is!...}{gezond en sterk en zo lief}{voor alle mensen}\\

\haiku{Ik kom tegenover.}{de scherven van een derde}{huwelijk te staan}\\

\haiku{{\textquoteleft}Ik zal helemaal,{\textquoteright}.}{opnieuw moeten beginnen}{ging hij peinzend voort}\\

\haiku{Ze wankelde - - of - - - -:}{ze even zou spelen met hem}{Dat kon ze ook niet}\\

\haiku{Dat heb ik in jou -.}{gevonden en daar moet ik}{mijn deel van hebben}\\

\haiku{{\textquoteleft}Erger,{\textquoteright} antwoordde, {\textquoteleft}.}{Polin het gevang zit je}{met kameraden}\\

\haiku{Ze sloeg haar armen,.}{om zijn hals en kuste hem}{aandachtig terug}\\

\haiku{Pol, geef monsieur,,.}{mijn pas dan kan hij zien dat}{ik de waarheid spreek}\\

\haiku{Ja, - ze moesten koffers - -.}{pakken en afrekenen}{ze gingen naar huis}\\

\haiku{Je mag die stukken,.}{hebben als je in vrede}{met ons kunt leven}\\

\haiku{- zo is het mij ook,{\textquoteright}.}{niet opgevallen hoonde}{Venens sinister}\\

\haiku{Morgane voelde;}{een vreselijke spanning}{in haar geest komen}\\

\haiku{Ze keek nog eens om.}{en gierhuilde weer met de}{handen voor de ogen}\\

\haiku{{\textquoteleft}Met brandtrapjes en.}{afgesloten vensters en}{elektrische seinen}\\

\haiku{Vooral, nu er een.}{wettige echtgenote}{bij is gekomen}\\

\haiku{{\textquoteright} wikte Pol. {\textquoteleft}Maar je.}{vader zal alles vatten}{wat hij kan krijgen}\\

\subsection{Uit: In het hol van de tamme leeuw}

\haiku{Daarom was ze niets.}{minder romantisch in zijn}{leven gekomen}\\

\haiku{{\textquoteleft}Als u eens een keer,.}{n{\'\i}\'et langs kwam zou ik beslist}{ook op \'u wachten}\\

\haiku{- Ze moest zich haasten,,.}{om Paps te zeggen dat ze}{met Bob zou trouwen}\\

\haiku{{\textquoteright} {\textquoteleft}O, gos, nee, da's waar,{\textquoteright}.}{verdoezelde meneer zijn}{onvoorzichtigheid}\\

\haiku{Trouwens, je hebt die,...{\textquoteright}}{auto nog geen jaar als ik}{me goed herinner}\\

\haiku{Maar niets maakte hem,.}{ooit zo giftig als smelten}{door anders toedoen}\\

\haiku{meneer Van der Spa,;}{starend in een toekomst vol}{blijde gezichten}\\

\haiku{{\textquoteright} {\textquoteleft}Ik wou, dat ik een,{\textquoteright}.}{auto van hem kreeg zuchtte}{Liza uitgeput}\\

\haiku{Dus had Hetty niet:}{kunnen vertrekken zonder}{een opdracht aan Kees}\\

\haiku{{\textquoteright} zei meneer Van der,.}{Spa en brandde zich flitsend}{aan het belknopje}\\

\haiku{Misschien was hij in.}{zijn prilste dagen een goed}{jongetje geweest}\\

\haiku{{\textquoteleft}Ik vind die nieuwe.}{tuinjongen op \'e\'en na de}{grootste idioot hier}\\

\haiku{En mams heeft ze nooit,.}{gekregen dus die kan er}{niet over oordelen}\\

\haiku{er zijn er die een -,.}{man prachtig vindt er zijn er}{die hij niet kan zien}\\

\haiku{Maar elke man heeft,.}{er \'e\'en waar zijn hand zonder}{bedenken naar grijpt}\\

\haiku{Dat ziet alleen de,{\textquoteright},.}{buitenstaander antwoordde}{Bob haar aankijkend}\\

\haiku{{\textquoteleft}Kind,{\textquoteright} zei hij, {\textquoteleft}jou kan,.}{ik mijn zegen geven want}{dat maakt niet veel uit}\\

\haiku{Zijn gelaatstint klom.}{van tomaat via aardbei en}{radijs tot biet op}\\

\haiku{Liza richtte zich.}{op in haar huilstoel en hing}{haar ogen aan de deur}\\

\haiku{{\textquoteleft}Als je belooft dat,.}{ik met Bob mag trouwen zal}{ik je ophijsen}\\

\haiku{wat hij wellicht had,.}{ge\"erfd van de dame naar}{wie hij was genoemd}\\

\haiku{Het gaat me zo aan,{\textquoteright}.}{mijn hart van Kees zei ze met}{extra veel hoofdpijn}\\

\haiku{en hij wreef zich in.}{de handen tot er een lucht}{van geschroeid vlees hing}\\

\haiku{Van Dalen stond toen,.}{ook op en vroeg verlof om}{zijn hond te fluiten}\\

\haiku{Dat gedoe tussen,{\textquoteright}.}{jou en mijn zoon verklaarde}{de trotse vader}\\

\haiku{Een schaap zou sluik haar,.}{krijgen van schaamte als het}{z\'o moest mekkeren}\\

\haiku{en dat d{\`\i}t dus was,.}{hoe mensen een kapitaal}{konden verdienen}\\

\haiku{En of er wel eens,??}{was ingebroken in dat}{schatrijke gedoe}\\

\haiku{{\textquoteleft}Ik wil er meer van,,!}{weten Joris en op}{zeer korte termijn}\\

\haiku{{\textquoteleft}Het is \`opgebruikt,{\textquoteright}.}{constateerde meneer Van}{der Spa lamzakkig}\\

\haiku{Toekomst maakte je,.}{met jiu jitsu of met een}{aardappelmesje}\\

\haiku{Dat betekent dat,.}{ze \`of iets in haar schild voert}{\`of is afgeleid}\\

\haiku{{\textquoteleft}Ik denk, dat hij zijn,.}{schade inhaalt van een paar}{dagen tegelijk}\\

\haiku{Dat is de wulpse,...!}{bloeddorstige grimas van}{een bezetene}\\

\haiku{\`of ze heeft er n{\'\i}\'et,.}{mee te maken en dat is}{n\`og penibeler}\\

\haiku{Lotje bleek weinig.}{tijd te hebben voor verder}{omslachtig gesprek}\\

\haiku{{\textquoteright} wilde hij haastig,.}{weten kennelijk bezwaard}{door het kusverbod}\\

\haiku{Je ziet eruit als -,.}{een pistache en dat ligt}{niet aan jou dit keer}\\

\haiku{Zou juffrouw Parels,...?}{niet denken dat ze daarin}{onherkenbaar was}\\

\haiku{{\textquoteright} vertelde ze haar, {\textquoteleft}!}{moeder zeer stralenden daar}{kan ik niet meer uit}\\

\haiku{Als ze praatte, klom.}{er een onbekend beest in}{haar strot op en neer}\\

\haiku{Dit was het laatste,,.}{wat ze had verwacht van haar}{prilste jaren af}\\

\haiku{Eigenlijk kwam hij.}{veel te snel bij het hek van}{de grote villa}\\

\haiku{Ze zouden toch niet...?}{allemaal zijn gevlucht voor}{een ontzettendheid}\\

\haiku{{\textquoteleft}Ik weet zelf niet, waar, -{\textquoteright} {\textquoteleft}}{ik soms mijn gedachten heb}{neemt u me dus niet}\\

\haiku{Lieve God, wat zat!}{hij onwrikbaar in de greep}{van de vrolijkheid}\\

\haiku{De gastheer wist nog.}{altijd niet of het te veel}{of te weinig was}\\

\haiku{Zo moest Eva ook zijn,.}{begonnen toen ze wakker}{werd uit het scheppen}\\

\haiku{De temperatuur;}{in de woning was hoger}{dan het bouwsel zelf}\\

\haiku{en dat laatste zou.}{een ongetwijfeld droeve}{ervaring worden}\\

\haiku{Hij loog gitzwart dat - -...}{hij zat te werken en dat}{hij niets te kort kwam}\\

\haiku{{\textquoteleft}Zou uw moeder geen...?}{belangstelling hebben voor}{een nieuwe wagen}\\

\haiku{{\textquoteright} sputterde meneer,.}{Van der Spa krachteloos en}{derhalve bozer}\\

\haiku{V\'o\'orda't'ie zich loeiend,!}{van zonde in de hel gooit}{met al die wijve}\\

\haiku{{\textquoteright} lalde de laffe,.}{schurk en hees zich min of meer}{aan haar stoel overeind}\\

\haiku{En zonder dat hij,!}{het wist deed hij oefening}{no. 4 volmaakt goed}\\

\subsection{Uit: De kant\'elen k\`antelen}

\haiku{alsof er tijd en,.}{ruimte in het gesprek was}{om te antwoorden}\\

\haiku{En wij zullen dan,.}{de tijd nemen tezamen}{iets te gebruiken}\\

\haiku{{\textquoteleft}Anisette voor de...,,.}{dames en brandewijn voor}{de heren Simon}\\

\haiku{{\textquoteright} Waarna de lieve,:}{stralend antwoordde dat dit}{geenszins hinderde}\\

\haiku{Het juffertje, met,:}{een halve rev\'erence}{nam haar glas en zei}\\

\haiku{Hij is nog zo jong -!...}{dan doen wij allemaal wel}{eens dwaze dingen}\\

\haiku{De heer Bertels zond,.}{een brief welke opwindend}{dreigde te zijn}\\

\haiku{, zoals mij door de.}{heer Ter Wamel van Heuvell}{is aangeraden}\\

\haiku{Hij had getoond rijp.}{te zijn voor de veiligheid}{van het huwelijk}\\

\haiku{Beschaafd, zeer juist van,.}{woordkeus en vol van de tact}{die ware adel is}\\

\haiku{Ja, ze voelde zich,.}{rood worden en glimlachte}{tot krimpens toe}\\

\haiku{{\textquoteleft}Zij is gekwetst en,{\textquoteright}.}{verlangt toch naar haar kind dacht}{Emilia van Heuvell}\\

\haiku{{\textquoteleft}Lieve Allaer,{\textquoteright} schreef, {\textquoteleft}?}{zijhoe gaat het toch met Jou}{en Je echtgenoote}\\

\haiku{Bijna geruisloos,.}{slipte zij de kamer uit}{een smalle gang in}\\

\haiku{{\textquoteleft}Als iemand tracht over,!}{te zwemmen of te varen}{dan schiet je subiet}\\

\haiku{{\textquoteright} Eindelijk had ze,.}{doel getroffen ze zag het}{aan de oude ogen}\\

\haiku{Hij had een gevoel,.}{alsof zijn eten eensklaps niet}{meer wilde zakken}\\

\haiku{Uw keuken is puik,,{\textquoteright}.}{lieve jongen voegde hij}{er nadenkend bij}\\

\haiku{Ik bezweer u, dat.}{mijn man en ik niets van uw}{nood hebben bevroed}\\

\haiku{De lopers werden -.}{toen al gelegd alles ging}{met kostbare haast}\\

\haiku{De Herengracht wist,.}{toen dat daar de familie}{Sytz kwam te wonen}\\

\haiku{Ergens moest toch een,?...}{brug van menselijkheid zijn}{in dit mysterie}\\

\haiku{Ach! - kon Betje Sytz,!?}{werkelijk zo menselijk}{zijn en zo geestig}\\

\haiku{Een blonde kerel,,.}{met geweldige schouders}{en ogen als een arend}\\

\haiku{de kleintjes grienden,.}{hem te hard hij zat maar met}{gebalde vuisten}\\

\haiku{Va is al zo lang,' '!}{bij meneer Carel da we}{t motten vieren}\\

\haiku{{\textquoteright} En Rinus knikboog,.}{en struikelde verzaligd}{het kantoortje uit}\\

\haiku{een blijmoedig, stil.}{ventje met een dunne stem}{en een dunne hals}\\

\haiku{Die nacht lag meneer,.}{Mathijs wakker zoals wel}{vaker gebeurde}\\

\haiku{{\textquoteleft}Wat een geluk, dat -!}{j{\'\i}j me dat kunt zeggen \`en}{dat ik het inzie}\\

\haiku{Ze stond op, en zocht.}{tussen de banken door haar}{pad naar het midden}\\

\haiku{De bakkersvrouw vroeg.}{of ze verkering had met}{die zwarte jongen}\\

\haiku{Doch een paar dagen.}{later bracht de postillon}{een brief voor haar mee}\\

\haiku{Nu was de verte,.}{een losse wolk drijvend op}{onbekende wind}\\

\haiku{Met naar kroezig haar,.}{terwijl golvende vlechten}{werden bezongen}\\

\haiku{De zusters huwden,...}{eveneens enkelen met een}{z\'e\'er goede partij}\\

\haiku{een man geweest van,?}{klein karakter zodat hij}{zich liet ompraten}\\

\haiku{{\textquoteright} Waarover Gerrit, nog,.}{altijd te goeder trouw weer}{schaterlachte}\\

\haiku{Ja, het leek of ze,.}{zich strekte en hoger zat}{dan de anderen}\\

\haiku{Maar ja, Graddus zei:}{dus op een dag tegen een}{kwitantieloper}\\

\haiku{Nu begreep hij, hoe.}{gemakkelijk Graddus met}{anders goed omsprong}\\

\haiku{{\textquoteright} zeiden de mensen, {\textquoteleft}!...}{in de kleine stadGraddus}{loopt met een handkar}\\

\haiku{Een enkele man,?!}{vroeg zich luidop af of hij}{soms met vodden liep}\\

\haiku{Nee, beste lezer,,.}{denk nu niet dat Graddus de}{honderdduizend won}\\

\haiku{{\textquoteright} Op dat ogenblik kwam,.}{bedoelde heer naar buiten}{vriendelijk lachend}\\

\haiku{Op de Postweg hield.}{naast hem een reusachtige}{luxe-auto stil}\\

\haiku{{\textquoteleft}Trouwens - wie zegt u, - -...,?...!}{dat ik inderd\'a\'ad niet mal}{doe op zo'n toneel}\\

\haiku{Hij schreef er na haar,.}{weggaan een rapport over voor}{het procesverbaal}\\

\haiku{Was haar prestatie?...}{eventueel w\`erkelijk}{zoveel drukte waard}\\

\haiku{De directeur van:}{de Radiant Dancing Group}{zei met een grinnik}\\

\haiku{Daarna begonnen -.}{haar armen te bewegen}{het lichaam trilde}\\

\haiku{De armen waren.}{onwerkelijk blank in hun}{snelle gebaren}\\

\haiku{Nog wist het publiek,.}{niet of deze val  ook}{opzet was geweest}\\

\haiku{{\textquoteleft}Hoe weet u nou, dat -?}{die dame naar de bank gaat}{om geld te halen}\\

\haiku{Het stemde mevrouw.}{Thijmens op een prettige}{manier weemoedig}\\

\haiku{{\textquoteleft}Ik moet daar dienst doen,.}{en heb geholpen bij het}{onderzoek en zo}\\

\haiku{{\textquoteleft}Goed, en degene,,,.}{die haar heeft vermoord wist dat}{ze geld in huis had}\\

\haiku{Maar bij mij heeft hij...{\textquoteright}}{nooit het voordeurslot hoeven}{te repareren}\\

\subsection{Uit: Het klooster van de lichtgroene paters}

\haiku{De bomen weken.}{en toonden een veld van even}{hoge afmeting}\\

\haiku{Het jonge paar werd;}{een middelbaar paar met een}{zoon en een dochter}\\

\haiku{Het geluk van de.}{voorouders moet hem door het}{hoofd hebben gespeeld}\\

\haiku{Een drankje om nooit.}{te vergeten en nimmer}{weer te nemen}\\

\haiku{Tommy was door een.}{kwaadaardig stuk rolpens de}{kamer opgezegd}\\

\haiku{God, zeg niet zulke,,}{onzin ik zou immers nog}{even blijven leven}\\

\haiku{We zijn toch niet dol,!}{dat we teruggaan naar die}{suffe kamer-troep}\\

\haiku{{\textquoteleft}Welke opdracht heeft?...{\textquoteright}.}{die man van wie gehad Hij}{was alweer terug}\\

\haiku{{\textquoteright} vroeg de meneer, nu.}{toch definitief over z'n}{hele ziel bevreemd}\\

\haiku{{\textquoteleft}Maar u voelde zich....}{unaniem meer aangetrokken}{tot dit matte groen}\\

\haiku{Tommy richtte een.}{paar schroeiende ogen naar de}{vorige spreker}\\

\haiku{Daar werd het gelach,.}{zo veelvuldig dat niemand}{meer verstaanbaar was}\\

\haiku{{\textquoteleft}Of iemand bemint,,}{hem en heeft Hans een geschenk}{willen aanbieden}\\

\haiku{Maar ze toeterde:}{bliksemscherp uit een hoek van}{haar verknepen mond}\\

\haiku{God helpe je, als!}{je in de dorpsgemeenschap}{wordt opgenomen}\\

\haiku{{\textquoteright} vroeg Tommy, en hief '.}{een rood gezicht op metn}{paar zielige ogen}\\

\haiku{Gijs stond stil bij een.}{gemakkelijke crapaud}{van bleek rood fluweel}\\

\haiku{Gijbertje was een.}{mooi meidje met felrood haar}{en donkere ogen}\\

\haiku{Dat haar haren een.}{gouden licht afstraalden in}{de luikende avond}\\

\haiku{De woordkeus van de.}{verteller was naief en}{ongecompliceerd}\\

\haiku{Nee, meneer Joan.}{greep in z'n  zak en gaf}{de boer een gulden}\\

\haiku{Meneer Joan werd,.}{een wildebras die geen wijf}{met rust kon laten}\\

\haiku{Ach, het werd een slaatje;}{van zure komkommer met}{uitjes en sardientjes}\\

\haiku{{\textquoteleft}Daar moet dan rode,{\textquoteright}.}{wijn bij worden geserveerd}{bestelde Tommy}\\

\haiku{Sommige mensen.}{ondergaan het leven zo}{bizonder simpel}\\

\haiku{Ze trokken hem aan,,}{slappe armen overeind maar}{hij boog overal uit}\\

\haiku{Die dure stoelen,{\textquoteright}.}{zijn van Kareltje kefte}{Tommy waarschuwend}\\

\haiku{Ze renden de deur,.}{uit en stoven weg in de}{wagens rechts en links}\\

\haiku{Het liep afgetobd,;}{en zonder enige moed het}{liet de kop hangen}\\

\haiku{Terwijl hij hen toch.}{ietwat had gesticht met zijn}{pianogetingel}\\

\haiku{{\textquoteleft}Het is jammer, dat,{\textquoteright}.}{ze zo van ons schrok klaagde}{Rogier zoetsappig}\\

\haiku{{\textquoteleft}En as julle geen,?}{poaters ben woarom draoge}{julle dan rokke}\\

\haiku{{\textquoteleft}Niks nie,{\textquoteright} weerde ze, '.}{met toch nogn losse moer}{in het lachcentrum}\\

\haiku{En kijk, al praatten -!}{die kerels nou volslagen}{zot w\`erk kreeg ze}\\

\haiku{{\textquoteleft}Maar dit is dan ook,,?}{de laatste kip die hier wordt}{geintroduceerd niet}\\

\haiku{Miquel was overeind.}{en stak zijn hoofd om de hoek}{van de kamerdeur}\\

\haiku{Rinus zette het.}{laatste theekopje neer en}{applaudisseerde}\\

\haiku{...{\textquoteright} Ukje hijgde en.}{blafte kleine geluidjes}{langs de buitendeur}\\

\haiku{want wie garandeert?}{dat niet ook de liefdoener}{je leven begeert}\\

\haiku{dat ik de stam die - -,?}{avond fuifde op eh zeg wat}{{\`\i}s dit voor gerecht}\\

\haiku{Nou ja, dat wisten,.}{ze wel maar als hij gepest}{werd was dat logisch}\\

\haiku{{\textquoteright} Miquel grinnikte.}{vaak en knikte terwijl hij}{z'n blik afwendde}\\

\haiku{{\textquoteleft}Nee, aan mijn servet,{\textquoteright}.}{kleeft geen lippenrood voegde}{hij er peinzend bij}\\

\haiku{Toen de wagen was,.}{weggereden liep Miquel}{langzaam naar boven}\\

\haiku{met visioenen.}{of God tegen hem met de}{vingers knipte}\\

\haiku{Die smoking is ook.}{echt niet het goedkoopste stuk}{van de uitverkoop}\\

\haiku{Zijn handen bleven,.}{langs z'n lichaam toen hij boog}{en zich afwendde}\\

\haiku{Rinus praatte half.}{grinnikend gedempt en zweeg}{opeens als verrast}\\

\haiku{In de hal vroeg hij.}{het telefoonboek en keek}{er een beetje in}\\

\haiku{{\textquoteright} Miquel haalde de.}{schouders op en glimlachte}{half naar Angelo}\\

\haiku{Tommy keek van de.}{een naar de ander en trok}{de wenkbrauwen op}\\

\haiku{Hij bukte zich toch.}{en begon de zaken in}{het tasje te doen}\\

\haiku{Ze maakten een kooi.}{van een flinke lap gaas uit}{de Welkomwinkel}\\

\haiku{{\textquoteright} Maar Angelo stond,:}{bewegingloos naar de maan}{te kijken en zei}\\

\haiku{Op kantoor was een,,{\textquoteright}.}{vent waar ik al lang de pest}{aan had ging Hans door}\\

\haiku{Hij zat altijd te,.}{lachen en hij had {\'\i}\'ets wat}{me irriteerde}\\

\haiku{vooral Rogier en.}{Tommy hadden zoiets nog}{onlangs meegemaakt}\\

\haiku{{\textquoteright} {\textquoteleft}Nee,{\textquoteright} gaf Hans toe, {\textquoteleft}maar...}{iedereen kon begrijpen}{wat hij bedoelde}\\

\haiku{{\textquoteleft}Hans,{\textquoteright} bitste hij, {\textquoteleft}als je,!}{nog \'e\'en woord zegt sla ik je}{persoonlijk tot moes}\\

\haiku{{\textquoteleft}Mijn meisje heeft twee,,}{jaar ziek gelegen en van}{de winter stierf ze}\\

\haiku{{\textquoteright} {\textquoteleft}Dat is al maanden,{\textquoteright},.}{geleden zei Tommy half}{tegen Angelo}\\

\haiku{{\textquoteright} Angelo begreep.}{dat hij ook niet bij hem had}{kunnen aankloppen}\\

\haiku{Dwing me niet, strenger,.}{maatregelen te nemen}{want ik sta voor niets}\\

\haiku{De veldwachter liet.}{zich op de knie\"en zinken}{in het autolicht}\\

\haiku{In de stad was een.}{huis van hem vrij gekomen}{door een sterfgeval}\\

\subsection{Uit: Koninklijke omnibus}

\haiku{wat had de jongen,!}{zijn verlegen trotse hart}{toen voelen krimpen}\\

\haiku{Te ver naar voren,.}{zodat hij telkens tegen}{de treden schopte}\\

\haiku{In de bovengang,,:}{langs holle stapgeluiden}{zei de jonge vorst}\\

\haiku{Terwijl hij sprak, zag.}{hij in de ogen van Dov\`ec}{schrik en verrassing}\\

\haiku{Doch hij speelde met,.}{overgave mee en voelde}{zich heel gelukkig}\\

\haiku{{\textquoteleft}Als ik de eerste,?}{keer al niet de tijd neem wat}{moet ik dan later}\\

\haiku{{\textquoteleft}Hij ontdoet zich van.}{zijn tooi en strekt zich uit op}{zijn leger en rust}\\

\haiku{{\textquoteright} Hij vlijde handig.}{een brede hermelijnen}{kraag om de schouders}\\

\haiku{{\textquoteright} Langs het vorstelijk,:}{oor fluisterde Kudja aan}{de kraag frunnikend}\\

\haiku{Hij klopte Adalbert.}{op de schouder en wees naar}{de prachtige kroon}\\

\haiku{{\textquoteleft}Je bent helemaal,.}{geen familie van me en}{je opa ken ik niet}\\

\haiku{Ze maken rovers...{\textquoteright}.}{van ons Adalbert keek hem met}{brandende ogen aan}\\

\haiku{{\textquoteleft}Alleredelste,{\textquoteright}, {\textquoteleft} - -!}{hakkelde hiju kunt zich}{hier niet uit praten}\\

\haiku{{\textquoteleft}U schijnt de ernst van,.}{het gebeurde niet te zien}{Alleredelste}\\

\haiku{Hij liet zich languit.}{vallen en keek tussen het}{hoge bloeisel door}\\

\haiku{Zijdelings stond op.}{het veld een jongen van iets}{oudere leeftijd}\\

\haiku{De jongen wendde.}{zich geheel naar hem om en}{bekeek hem opnieuw}\\

\haiku{{\textquoteright} {\textquoteleft}Precies,{\textquoteright} zei Karalj,.}{en trok z'n benen onder}{zich om op te staan}\\

\haiku{Toen hij de jongens,.}{met de bloemen zag kreeg hij}{een schok door z'n lijf}\\

\haiku{Dov\`ec zette zijn,:}{glas voorzichtig neer en zei}{zacht maar zeer stellig}\\

\haiku{Het moet nu vier jaar,...}{geleden zijn hij was dus}{ongeveer zestien}\\

\haiku{{\textquoteleft}Ze zijn op raad van - -{\textquoteright} {\textquoteleft}!}{de koetsier uitgestegen}{bij het moerasN\'e\'e}\\

\haiku{Maar u sluit zich van,,.}{mij af met uw cynisme}{en dat is jammer}\\

\haiku{De beste raadsman.}{leek Adalbert altijd nog het}{vriendje uit zijn jeugd}\\

\haiku{Hij lachte en sprong,.}{over de stenen heen van de}{weg af het veld in}\\

\haiku{Altijd was er om.}{Dj\'ura Gw\'ano een sfeer van haat}{en onrust geweest}\\

\haiku{Pas toen Gw\'ano de,.}{zaal had verlaten barstte}{er een tumult los}\\

\haiku{{\textquoteleft}Die is nou blij, maar '.}{over een poosje heeftie weer}{iets anders nodig}\\

\haiku{Maar die avond kwamen,:}{Dov\`ec en Bartenstein bij}{Adalbert en zeiden}\\

\haiku{- ze hadden het als.}{goede soldaten stellig}{zelf kunnen vinden}\\

\haiku{Onverwacht sprak de -.}{priester zijn stem viel als een}{galm in het zwijgen}\\

\haiku{Die jongen stond weer,.}{net zo star als hij in het}{begin had gedaan}\\

\haiku{Maar er was iets in,.}{het snuiven van het dier dat}{hem waakzaam maakte}\\

\haiku{{\textquoteleft}Als ik gewoon in,!}{Jarb\'ogic was gebleven}{zou hij nog leven}\\

\haiku{de tafel kraakte.}{met enorm geraas onder hun}{handen in elkaar}\\

\haiku{{\textquoteleft}Maar als je 's nachts,.}{kouwe voeten hebt laat ze}{je mooi bibberen}\\

\haiku{Hij richtte het hoofd}{een beetje waanwijs op en}{keek uit het venster}\\

\haiku{{\textquoteleft}En ik acht u een -.}{betrouwbaar man en die zijn}{schaars in de wereld}\\

\haiku{Hij deed een greep in.}{zijn zak en bood haar plat op}{zijn hand de solid}\\

\haiku{{\textquoteleft}Er zijn altijd zo,{\textquoteright}.}{veel vraagstukken tegelijk}{zei hij voorzichtig}\\

\haiku{En het drong pas die,.}{avond laat tot hem door wat ze}{kon hebben bedoeld}\\

\haiku{Dat heeft mij na het.}{gesprek met mevrouw Kurgic}{een beetje bevreemd}\\

\haiku{en dit standpunt had.}{nog nimmer enige Vorst zo}{simplistisch verwoord}\\

\haiku{Dit waren woorden.}{die Adalbert een prik in zijn}{hersenpan gaven}\\

\haiku{Ik geef het je niet,.}{in handen want men zou naar}{ons kunnen kijken}\\

\haiku{Dat is veel erger.}{dan boze koningen en}{zure prinsessen}\\

\haiku{{\textquoteleft}Wie beseft er, wat,?}{het wil zeggen een kroon op}{je hoofd te krijgen}\\

\haiku{Niet alleen had ik,...{\textquoteright}}{zo veel begrip niet verwacht}{zo ver van mijn land}\\

\haiku{U bent reeds te lang,...}{opstandig geweest om te}{kunnen vertrouwen}\\

\haiku{Een gans apparaat;}{van ceremoniekenners}{kwam in beweging}\\

\haiku{Hij heeft er toch goed,,}{aan gedaan al die wijven}{eruit te trappen}\\

\haiku{Om drie minuten.}{over half elf remde de trein}{piepend en hijgend}\\

\haiku{Dat moet u niet doen,{\textquoteright}, {\textquoteleft}.}{verzocht Adalbert haastigwij}{zijn incognito}\\

\haiku{van het onderhoud.}{met Gw\'ano had hij alleen}{vermoedens gehad}\\

\haiku{Volgens Adalbert moest,.}{er arnika op en dat}{deed Karalj dus}\\

\haiku{Baron Bartenstein.}{hoorde dit alles met een}{brede glimlach aan}\\

\haiku{{\textquotedblleft}Wij verlenen het -{\textquotedblright} {\textquotedblleft}}{agr\'ement U zoudt eventueel}{kunnen doen schrijven}\\

\haiku{{\textquoteright} De baron boog en.}{liet zich uitgeleide doen}{door de knecht van dienst}\\

\haiku{Je bent je eigen -{\textquoteright} {\textquoteleft},{\textquoteright}.}{L\'a\'at me nou zei Adalbert met}{een wit schuimgezicht}\\

\haiku{{\textquoteright} en ontsloot schielijk,:}{de stroom van woorden welke}{hij had klaarliggen}\\

\haiku{Toen ontmoette zijn -.}{blik die van de jongeman}{en Bartenstein zweeg}\\

\haiku{het water spatte.}{als twee dunne vlerken langs}{de smalle boeg op}\\

\haiku{Het was wonderlijk -.}{alsof hij een jong diertje}{in bescherming nam}\\

\haiku{Drie van de paarden.}{knikten met hun hoofd terwijl}{het vierde brieste}\\

\haiku{De laatste deur was.}{van binnen afsluitbaar en}{aldus deden zij}\\

\haiku{Zij gingen speurend,;}{nu langs alle laden en}{kasten en kisten}\\

\haiku{Maar nu blonk achter,.}{elke gedachte die ene}{heel lieve glimlach}\\

\haiku{{\textquoteleft}En d\`an moeten we,,{\textquoteright}.}{uitvissen waar die baron}{woont vulde hij aan}\\

\haiku{Terwijl de Fijne,:}{het netnummer draaide van}{Grifsdorp sprak de Boom}\\

\haiku{Hij vingerde aan,;}{een microfoon die gillen}{uitzond als een big}\\

\haiku{{\textquoteleft}Verknepen knoedels,,!}{drekdoerakken rotsmerige}{stinkstommelingen}\\

\haiku{maar in dienst had hij.}{het niet verder geschopt dan}{rekruut en cachot}\\

\haiku{n theemuts op je,{\textquoteright},.}{hoofd antwoordde de Fijne}{wat niet aardig was}\\

\haiku{{\textquoteleft}Als de zaak morgen,,{\textquoteright};}{mislukt is het jouw schuld zei}{hij gemakkelijk}\\

\haiku{{\textquoteright} Hier moest de Fijne -.}{toch om glimlachen wellicht}{vleide hem het beeld}\\

\haiku{Pingel deed zijn jack -.}{uit en een sjiek colbert aan}{de broek droeg hij reeds}\\

\haiku{{\textquoteleft}Ik wou toch \`erg graag,{\textquoteright}.}{effe de vlag probere}{zei Jules bedeesd}\\

\haiku{{\textquoteleft}Ik mot toch wete,,....}{hoe of'tie beweegt als ik}{mot saluere}\\

\haiku{{\textquoteright} {\textquoteleft}Dora wordt ervoor,{\textquoteright}.}{betaald wees de hertog haar}{hooghartig terecht}\\

\haiku{En waarom meteen!}{van die onbeheerste grote}{trossen van alles}\\

\haiku{En terwijl ik wil,....}{doorlope zie ik dat de}{pin d'r half uitsteekt}\\

\haiku{De post is geweest,.}{en die smeris komt over zes}{minuten terug}\\

\haiku{{\textquoteleft}Ach,{\textquoteright} sprak de eerste, {\textquoteleft},....}{blas\'edat is een grapje}{van Jules Thera}\\

\haiku{{\textquoteright} {\textquoteleft}Nee, merci,{\textquoteright} ontkwam, {\textquoteleft}....}{de ouderedat draag ik}{altijd zelf bij me}\\

\haiku{Wat deed ze ook met....}{zo'n onzinnig accent vlak}{onder haar steekneus}\\

\haiku{De koffie stond haar,.}{aan de boorden van haar ziel}{dat was duidelijk}\\

\haiku{De Fijne leek wat -,.}{te horen hij boog zich uit}{zijn bank en riep iets}\\

\haiku{Hij gebaarde naar -.}{de schommelbank zij konden}{samen gaan zitten}\\

\haiku{{\textquoteleft}Dat beleef ik maar,!...{\textquoteright}}{\'e\'enmaal en ik heb het}{totaal niet verwacht}\\

\haiku{Zeg dat me moeder,,!...}{ziek is dat ik met jullie}{mee moet bed\`enk iets}\\

\haiku{Hij ging dus open doen,,.}{en was totaal niet verbaasd}{politie te zien}\\

\haiku{Hij hoorde Dora,,.}{beneden achter een deur}{jankerig praten}\\

\haiku{De complicatie.}{had hem een rake klap op}{het hoofd gegeven}\\

\haiku{Het kwam niet te pas,.}{opeens in je eentje de}{tuin in te lopen}\\

\haiku{hij, de Fijne, was,!....}{zonder het zelf te willen}{uit zijn huid geglipt}\\

\haiku{Hij voelde edele.}{tranen prikken achter zijn}{schurke-oogleden}\\

\haiku{Waarom hebben ze,?}{dat nog niet uitgevonden}{en die atoombom wel}\\

\haiku{Zijn sterke kluiven '.}{wendden de hertog alsof}{hij aant spit hing}\\

\haiku{U kunt het er met -.}{water en zeep afhalen}{aan u de keuze}\\

\haiku{De politie knijpt!...}{me in alle uiteinden}{van m'n zenuwen}\\

\haiku{En bijna nog v\'o\'or,:}{zij was uitgesproken bood}{de hofdame aan}\\

\haiku{Vergeet niet, dat we,.}{niet weten w\`at er in die}{juwelenkist zit}\\

\haiku{D'r is d'r n\`og een,,{\textquoteright}.}{die koffie mot hebbe zei}{de Boom dromerig}\\

\haiku{Amad\'e trok zijn das recht.}{en de baron zat weer te}{vingeroefenen}\\

\haiku{Jij hebt nog geen deur,{\textquoteright}.}{opengedaan beschuldigde}{Pingel korzelig}\\

\haiku{Het was een oude,.}{heer die half buigend een hoed}{van zijn hoofd plukte}\\

\haiku{Nou, dag mevrouw, dag,!}{schat magikwelzeggen}{dag lekkere kluit}\\

\haiku{- {\textquoteleft}ik heb panne, ziet,.}{u en ik m\'o\'et een luchtpomp}{hebben en een schaar}\\

\haiku{De barones sloot,.}{de deur goed en leunde even}{tegen het paneel}\\

\haiku{{\textquoteright} kreet ze, en keek toch,.}{weer even om naar de tijger}{die nog steeds niet kwam}\\

\haiku{{\textquoteleft}Ja, maar we kunnen,!...{\textquoteright}}{natuurlijk ook wel iets van}{jou verwachten Door}\\

\haiku{{\textquotedblright} toen hadden een paar....}{dames en heren ze bont}{en blauw geslagen}\\

\haiku{Ze bracht enkele.}{haren ter plaatse waar zij}{hoorden in de taart}\\

\haiku{En hoeveel hadden,?}{jullie gedacht Dora dan}{later te geven}\\

\haiku{{\textquoteleft}Ik begrijp het niet,{\textquoteright}.}{meer sliste hij nederig}{en teleurgesteld}\\

\haiku{En die hofdame.}{vertrouwde ze nog minder}{dan Roje Gerrit}\\

\haiku{Gelukkig had de -.}{inspecteur hem gezien hij}{kon niet ver komen}\\

\haiku{{\textquoteleft}Ik geloof dat mijn,{\textquoteright}.}{ring in de terrine is}{gevallen zei ze}\\

\haiku{De prinses verloor,:}{haar voorname kalmte niet}{en zei vriendelijk}\\

\haiku{De Fijne richtte.}{zijn ogen in een dringende}{smeekbede naar haar}\\

\haiku{Pingel had zich op.}{een knie laten vallen en}{greep Fijne's benen}\\

\haiku{Ze had geen oog van -.}{hem af kunnen houden hij}{zag d'ruit als vijftien}\\

\haiku{De baron veegde,.}{over zijn voorhoofd dat vochtig}{was van de warmte}\\

\haiku{Een grote tafel,.}{stond in het midden omringd}{van hoge stoelen}\\

\haiku{Het was vreselijk -.}{prinses Eline de la Tour}{Olmberg schreide}\\

\haiku{Hij poetste over zijn.}{gezicht met gesloten ogen}{en haalde diep adem}\\

\haiku{{\textquoteleft}Ik heb nauwkeurig.}{de portretten van prinses}{Eline bekeken}\\

\haiku{Hij trok het bankje.}{onder zijn zitspieren en}{opeens speelde hij}\\

\haiku{{\textquoteleft}En dan gaan we met,.}{mekaar naar een verdomd goed}{eethuisje vanavond}\\

\haiku{Toen begon hij te,;}{lopen in zijn tennispak}{veerkrachtig en snel}\\

\haiku{Nou geef ik een schop,!}{tegen de tafel zodat}{de theepot kantelt}\\

\haiku{Waren er dan toch?...}{schepselen die over haar en}{haar vader praatten}\\

\haiku{Tussen al deze:}{individuutjes was Jaapje}{nooit opgevallen}\\

\haiku{Hij legde zijn arm.}{om veel vrouwelijke en}{manlijke schouders}\\

\haiku{Tante Sanna had.}{de resten van haar glimlach}{teruggevonden}\\

\haiku{{\textquoteleft}Je kunt altijd bij,{\textquoteright}.}{ons aankloppen om goede}{raad vulde ze aan}\\

\haiku{Alsof iemand haar.}{werkelijk au s\'erieux}{had genomen}\\

\haiku{{\textquoteleft}God hebbe zijn ziel,{\textquoteright}.}{en stemme hem gelukkig}{antwoordde Agniet}\\

\haiku{zo'n bezonken stem,!}{als ze aan het werk was met}{een behandeling}\\

\haiku{zij waren over het.}{nichtje dat eensklaps alleen}{was komen te staan}\\

\haiku{Dat maatschappelijk:}{werk had haar op andere}{gedachten gebracht}\\

\haiku{Zo  wakker en.}{actief was ze de hele}{dag nog niet geweest}\\

\haiku{hij had nog kuiltjes,!..}{in zijn wangen ook en wat}{een prachtige mond}\\

\haiku{Iedereen spreekt over.}{geld en niemand herkent de}{waarde van vriendschap}\\

\haiku{zij eten gevaarlijk.}{voedsel dat hun karakter}{en geest ondermijnt}\\

\haiku{Het was alsof ze.}{in een schitterend belicht}{panorama keek}\\

\haiku{Maar haar hoofd begon,:}{te gloeien en om een hoek}{dacht ze in paniek}\\

\haiku{{\textquoteleft}een schaakspeler die.}{geen enkele partij ten}{einde wil spelen}\\

\haiku{En Agniet, op geen,.}{geluid voorbereid voelde}{dat ze een kleur kreeg}\\

\haiku{Het ding hing daar al -,.}{minstens tien jaar ze wist niet}{waar het vandaan kwam}\\

\haiku{Dat wist ze op dit,.}{moment aan de scherpte van}{haar teleurstelling}\\

\haiku{{\textquoteright} Maar hij greep, zacht en,.}{ferm haar enkel en nam het}{schoentje van haar voet}\\

\haiku{- -{\textquoteright} Met verbijstering,.}{bemerkte Agniet dat ze}{uit haar japon gleed}\\

\haiku{En dat, besefte,:}{ze was de heiligheid van}{het wedervaren}\\

\haiku{Loom sloot ze later,.}{de voordeur af liep de trap}{op naar haar kamer}\\

\haiku{- ~ Ze bedacht, dat,,.}{ze eigenlijk niet wist welk}{werk Idris deed en waar}\\

\haiku{Ze lag weer in het.}{duister en de kille hand}{knelde haar hart dicht}\\

\haiku{{\textquoteleft}Ik heb daarbij hard -...,}{moeten werken ik heb een}{examen afgelegd}\\

\haiku{Ach, dat was ook niet,.}{erg geweest want ze had veel}{van hem gehouden}\\

\haiku{Waarom?{\textquoteright} {\textquoteleft}Omdat jouw,{\textquoteright}.}{land te ver van het mijne}{ligt zei Agniet zacht}\\

\haiku{Van Idris hoorde zij,.}{in deze dagen niets en}{dat begreep ze wel}\\

\haiku{Mijn man heeft zijn naam.}{genoteerd en naar deze}{heer ge{\"\i}nformeerd}\\

\haiku{Wij hebben thans de.}{naam opgekregen van een}{keurige dame}\\

\haiku{Er was weinig strijd.}{geweest bij de uitroeping}{van de nieuwe vorst}\\

\haiku{Ze moest geen enkel,.}{bijvoeglijk naamwoord overslaan}{als het Idris betrof}\\

\haiku{{\textquoteright} Agniet voelde zich.}{als een geigerteller die}{positief ontmoet}\\

\haiku{de expressie van,.}{Agniets gelaat in dat het}{hartverwarmend was}\\

\haiku{{\textquoteleft}Zou ons land daar een, - -?}{ambassade hebben of}{eh een legatie}\\

\haiku{Stel, dat Idris w\`el met, -....}{die Emier heeft gesproken maar}{inderdaad niet niet}\\

\haiku{{\textquoteleft}Al neem ik niet aan,,.}{dat hij nog een harem heeft}{hedentendage}\\

\haiku{Het was lang niet zo,.}{heerlijk als ze had gedacht}{om te vertellen}\\

\haiku{{\textquoteleft}Bleekblauw, zilver, heel,.}{blank goud in de tint van je}{haar met iets van zwart}\\

\haiku{Geen grote kunst, geen,.}{olie geen antiquiteiten}{of oude steden}\\

\haiku{{\textquoteleft}Ik verbeeld het me,,{\textquoteright}.}{net als de meeste mensen}{antwoordde Agniet}\\

\haiku{{\textquoteleft}Ik ken iemand uit,.}{zijn familie die zal me}{introduceren}\\

\haiku{Agniet volgde een,.}{donker meisje naar een hal}{waar ze moest wachten}\\

\haiku{Anderen bogen.}{in het stof en brachten hun}{hoofden ter aarde}\\

\haiku{De mensen keken.}{allemaal alsof ze een}{revuenummer was}\\

\haiku{Ze zou natuurlijk.}{worden ondergebracht in}{het vrouwen verblijf}\\

\haiku{een zeer donker, slank.}{meisje met ravenzwart haar}{in een zware wrong}\\

\haiku{Ze vroeg zich vaag af,.}{wat haar programma voor die}{avond zou mogen zijn}\\

\haiku{Zij besefte goed,,.}{het voedsel te eten wat Idris}{dus gewend moest zijn}\\

\haiku{{\textquoteleft}Maar Allah weet, wat -.}{er gaat gebeuren de Emier}{moet het afwachten}\\

\haiku{De bevestiging.}{van haar vermoeden liet niet}{lang op zich wachten}\\

\haiku{- Toen besefte ze,.}{dat ze met het boek voor de}{Emier in haar hand zat}\\

\haiku{Zijn ogen richtten zich -.}{weer naar haar en daar had ze}{geen weerstand tegen}\\

\haiku{De orchidee - of,,.}{welke bloem ook zit niet aan}{de buitenkant Agggniet}\\

\haiku{Hij trok haar tegen,.}{zich aan onweerstaanbaar van}{omzichtige kracht}\\

\haiku{{\textquoteright} en wachtend op de,:}{traktatie voegde hij zacht}{aan zijn verhaal toe}\\

\haiku{Een wachter buiten.}{een van de voorhangen gaf}{hij een kort bevel}\\

\haiku{Uit de verte had -}{nog onbekommerd gelach}{geklonken v\'o\'or hen}\\

\haiku{Die poincettia in -?...}{Holland was zo roodbladig}{geweest wat w{\`\i}st ze}\\

\haiku{Er trok een fijne.}{rimpeling van glimlach langs}{Djamilas kaken}\\

\haiku{En toen begon een:}{zeer naarstig onderricht in}{allerlei woorden}\\

\haiku{De hoofddoek welke:}{buiten werd gedragen met}{een rolband van koord}\\

\haiku{{\textquoteleft}Dit is koorts - ik ben,.}{besmet met iets waarvoor geen}{inenting bestond}\\

\haiku{Terwijl ze van de,}{spiegel weg naar buiten keek}{zag ze door de tuin}\\

\haiku{{\textquoteright} Djamila wendde.}{haar ogen naar haar en schudde}{vriendelijk het hoofd}\\

\haiku{{\textquoteright} Daarmee ontnam hij.}{aan het samenzijn een al}{te formele toon}\\

\haiku{bladen vol bekers,.}{binnen die voor de gasten}{werden neergezet}\\

\haiku{Terwijl ze naar hem,:}{luisterde zag Agniet de}{hand van de dienaar}\\

\haiku{Zijn lichaam boog als.}{een brug hol \`opkrommend in}{gruwelijk lijden}\\

\haiku{Ze schaamde zich en.}{meende liefde en geluk}{te hebben verspeeld}\\

\haiku{Agniet vroeg zich af, -.}{waar de dode dienaar was}{gebracht en zijn vrouw}\\

\subsection{Uit: Kroelen met de kroon}

\haiku{Nou geef ik een schop,!}{tegen de tafel zodat}{de theepot kantelt}\\

\haiku{Waren er dan toch?...}{schepselen die over haar en}{haar vader praatten}\\

\haiku{Tussen al deze:}{individuutjes was Jaapje}{nooit opgevallen}\\

\haiku{Hij legde zijn arm.}{om veel vrouwelijke en}{manlijke schouders}\\

\haiku{Tante Sanna had.}{de resten van haar glimlach}{teruggevonden}\\

\haiku{{\textquoteleft}Je kunt altijd bij,{\textquoteright}.}{ons aankloppen om goede}{raad vulde ze aan}\\

\haiku{Alsof iemand haar.}{werkelijk au s\'erieux}{had genomen}\\

\haiku{'God hebbe zijn ziel,{\textquoteright}.}{en stemme hem gelukkig}{antwoordde Agniet}\\

\haiku{zo'n bezonken stem,!}{als ze aan het werk was met}{een behandeling}\\

\haiku{zij waren over het.}{nichtje dat eensklaps alleen}{was komen te staan}\\

\haiku{Dat maatschappelijk:}{werk had haar op andere}{gedachten gebracht}\\

\haiku{Zo  wakker en.}{actief was ze de hele}{dag nog niet geweest}\\

\haiku{hij had nog kuiltjes,!..}{in zijn wangen ook en wat}{een prachtige mond}\\

\haiku{Iedereen spreekt over.}{geld en niemand herkent de}{waarde van vriendschap}\\

\haiku{zij eten gevaarlijk.}{voedsel dat hun karakter}{en geest ondermijnt}\\

\haiku{Het was alsof ze.}{in een schitterend belicht}{panorama keek}\\

\haiku{Maar haar hoofd begon,:}{te gloeien en om een hoek}{dacht ze in paniek}\\

\haiku{{\textquoteleft}een schaakspeler die.}{geen enkele partij ten}{einde wil spelen}\\

\haiku{En Agniet, op geen,.}{geluid voorbereid voelde}{dat ze een kleur kreeg}\\

\haiku{Het ding hing daar al -,.}{minstens tien jaar ze wist niet}{waar het vandaan kwam}\\

\haiku{Dat wist ze op dit,.}{moment aan de scherpte van}{haar teleurstelling}\\

\haiku{{\textquoteright} Maar hij greep, zacht en,.}{ferm haar enkel en nam het}{schoentje van haar voet}\\

\haiku{- -{\textquoteright} Met verbijstering,.}{bemerkte Agniet dat ze}{uit haar japon gleed}\\

\haiku{En dat, besefte,:}{ze was de heiligheid van}{het wedervaren}\\

\haiku{Loom sloot ze later,.}{de voordeur af liep de trap}{op naar haar kamer}\\

\haiku{- ~ Ze bedacht, dat,,.}{ze eigenlijk niet wist welk}{werk Idris deed en waar}\\

\haiku{Ze lag weer in het.}{duister en de kille hand}{knelde haar hart dicht}\\

\haiku{{\textquoteleft}Ik heb daarbij hard -...,}{moeten werken ik heb een}{examen afgelegd}\\

\haiku{Ach, dat was ook niet,.}{erg geweest want ze had veel}{van hem gehouden}\\

\haiku{Waarom?{\textquoteright} {\textquoteleft}Omdat jouw,{\textquoteright}.}{land te ver van het mijne}{ligt zei Agniet zacht}\\

\haiku{Van Idris hoorde zij,.}{in deze dagen niets en}{dat begreep ze wel}\\

\haiku{Mijn man heeft zijn naam.}{genoteerd en naar deze}{heer ge{\"\i}nformeerd}\\

\haiku{Wij hebben thans de.}{naam opgekregen van een}{keurige dame}\\

\haiku{Er was weinig strijd.}{geweest bij de uitroeping}{van de nieuwe vorst}\\

\haiku{Ze moest geen enkel,.}{bijvoeglijk naamwoord overslaan}{als het Idris betrof}\\

\haiku{{\textquoteright} Agniet voelde zich.}{als een geigerteller die}{positief ontmoet}\\

\haiku{de expressie van,.}{Agniets gelaat in dat het}{hartverwarmend was}\\

\haiku{{\textquoteleft}Zou ons land daar een, - -?}{ambassade hebben of}{eh een legatie}\\

\haiku{Stel, dat Idris w\`el met, -....}{die Emier heeft gesproken maar}{inderdaad niet niet}\\

\haiku{{\textquoteleft}Al neem ik niet aan,,.}{dat hij nog een harem heeft}{hedentendage}\\

\haiku{Het was lang niet zo,.}{heerlijk als ze had gedacht}{om te vertellen}\\

\haiku{{\textquoteleft}Bleekblauw, zilver, heel,.}{blank goud in de tint van je}{haar met iets van zwart}\\

\haiku{Geen grote kunst, geen,.}{olie geen antiquiteiten}{of oude steden}\\

\haiku{{\textquoteleft}Ik verbeeld het me,,{\textquoteright}.}{net als de meeste mensen}{antwoordde Agniet}\\

\haiku{{\textquoteleft}Ik ken iemand uit,.}{zijn familie die zal me}{introduceren}\\

\haiku{Agniet volgde een,.}{donker meisje naar een hal}{waar ze moest wachten}\\

\haiku{Anderen bogen.}{in het stof en brachten hun}{hoofden ter aarde}\\

\haiku{De mensen keken.}{allemaal alsof ze een}{revuenummer was}\\

\haiku{Ze zou natuurlijk.}{worden ondergebracht in}{het vrouwenverblijf}\\

\haiku{een zeer donker, slank.}{meisje met ravenzwart haar}{in een zware wrong}\\

\haiku{Ze vroeg zich vaag af,.}{wat haar programma voor die}{avond zou mogen zijn}\\

\haiku{Zij besefte goed,,.}{het voedsel te eten wat Idris}{dus gewend moest zijn}\\

\haiku{{\textquoteleft}Maar Allah weet, wat -.}{er gaat gebeuren de Emier}{moet het afwachten}\\

\haiku{De bevestiging.}{van haar vermoeden liet niet}{lang op zich wachten}\\

\haiku{- Toen besefte ze,.}{dat ze met het boek voor de}{Emier in haar hand zat}\\

\haiku{Zijn ogen richtten zich -.}{weer naar haar en daar had ze}{geen weerstand tegen}\\

\haiku{De orchidee - of,,.}{welke bloem ook zit niet aan}{de buitenkant Agggniet}\\

\haiku{Hij trok haar tegen,.}{zich aan onweerstaanbaar van}{omzichtige kracht}\\

\haiku{{\textquoteright} en wachtend op de,:}{traktatie voegde hij zacht}{aan zijn verhaal toe}\\

\haiku{Een wachter buiten.}{een van de voorhangen gaf}{hij een kort bevel}\\

\haiku{Uit de verte had -}{nog onbekommerd gelach}{geklonken v\'o\'or hen}\\

\haiku{Die poincettia in -?...}{Holland was zo roodbladig}{geweest wat w{\`\i}st ze}\\

\haiku{Er trok een fijne.}{rimpeling van glimlach langs}{Djamilas kaken}\\

\haiku{En toen begon een:}{zeer naarstig onderricht in}{allerlei woorden}\\

\haiku{De hoofddoek welke:}{buiten werd gedragen met}{een rolband van koord}\\

\haiku{{\textquoteleft}Dit is koorts - ik ben,.}{besmet met iets waarvoor geen}{inenting bestond}\\

\haiku{Je blik is dauw over -.}{mijn eenzaamheid en ik kan}{slechts je ogen strelen}\\

\haiku{Terwijl ze van de,}{spiegel weg naar buiten keek}{zag ze door de tuin}\\

\haiku{{\textquoteright} Djamila wendde.}{haar ogen naar haar en schudde}{vriendelijk het hoofd}\\

\haiku{{\textquoteright} Daarmee ontnam hij.}{aan het samenzijn een al}{te formele toon}\\

\haiku{bladen vol bekers,.}{binnen die voor de gasten}{werden neergezet}\\

\haiku{Terwijl ze naar hem,:}{luisterde zag Agniet de}{hand van de dienaar}\\

\haiku{{\textquoteleft}Als er niets is, ben!}{ik voor altijd het malle}{mens van de slokjes}\\

\haiku{Zijn lichaam boog als.}{een brug hol \`opkrommend in}{gruwelijk lijden}\\

\haiku{Ze schaamde zich en.}{meende liefde en geluk}{te hebben verspeeld}\\

\haiku{Agniet vroeg zich af, -.}{waar de dode dienaar was}{gebracht en zijn vrouw}\\

\subsection{Uit: Maneschijn over uw hart}

\haiku{Eigenlijk leek het,:}{wel of hij vleugels achter}{zijn schouders voelde}\\

\haiku{De wandelaar bleef,.}{uit beleefdheid stil staan en}{wist niets te zeggen}\\

\haiku{En als hij baadde,.}{lag er een boekje met een}{potlood naast zijn kuip}\\

\haiku{Het verlangen in.}{zijn blik was te zwaar voor de}{Eerste Minister}\\

\haiku{De ochtend daarna.}{ging de Eerste Minister}{niet naar zijn bureau}\\

\haiku{Men zweeg, zoals dat.}{behoort onder ministers}{van goede komaf}\\

\haiku{Het uitmuntende;}{en voortreffelijke lag}{in onze snelheid}\\

\haiku{{\textquoteright} zeiden ze, {\textquoteleft}je komt!}{op Pasen misschien wel in}{een valstrik terecht}\\

\haiku{{\textquoteright} Zo kan een mens z'n.}{leven afhangen van \'e\'en}{onbedacht half uur}\\

\haiku{Op een nacht werd de.}{vrouw wakker omdat ze haar}{man hoorde zingen}\\

\haiku{{\textquoteright} Want iedereen kan,.}{begrijpen dat zoiets geen}{pas geeft in de nacht}\\

\haiku{En daar kon Onze.}{Lieve Heer op zijn beurt niets}{tegen bedenken}\\

\haiku{De dokter zette,.}{zijn bril op en verzocht de}{heer diep te zuchten}\\

\haiku{s Middags was de,.}{oven afgekoeld en er was}{aldoor niets gebeurd}\\

\haiku{{\textquoteleft}Beseft u alleen,,.}{reeds het feit dat niemand zich}{verbaast als \'u praat}\\

\haiku{Doch naast zijn bed stond,,.}{een radio die heel zachte}{mooie muziek speelde}\\

\haiku{Doch Jan Luchtbel viel {\textquoteleft},!}{voor hem op zijn knie\"en en}{zeiHier ben ik God}\\

\haiku{Zend mij asjeblieft,;}{terug naar de aarde m\`et}{armen en benen}\\

\haiku{En de koning lag,.}{op zijn purperen bed zo}{zwak als een leeg hemd}\\

\haiku{En de kroonprins stond.}{bij het lijk van zijn vader}{en keek naar de ring}\\

\haiku{{\textquoteleft}Laten we onze,{\textquoteright}.}{zaken bepleiten stelde}{de oudste neef voor}\\

\haiku{Maar zij is een vrouw,!}{ze kan niet vergeleken}{worden bij een man}\\

\haiku{Allen zongen een,:}{vreemd lied dat heerlijk en toch}{melancholiek klonk}\\

\haiku{{\textquoteright} zei het meisje, {\textquoteleft}want!}{je hebt me zo ontzaglijk}{gelukkig gemaakt}\\

\haiku{Hij onderzocht de,.}{tong van de pati\"ent en}{voelde hem de pols}\\

\haiku{Ja, dat was voor een;}{man des geloofs heel moeilijk}{te beantwoorden}\\

\haiku{Laat het even over aan,.}{de Schepper die u en uw}{klacht heeft geschapen}\\

\haiku{{\textquoteright} Nou, dat hadden er,.}{m\'e\'er geweten dus daarin}{stond zij niet alleen}\\

\haiku{Hij liep steeds voort, en.}{om hem heen zonk de stilte}{der verlatenheid}\\

\haiku{Hij kon het weten,.}{want hij had een oogje op}{een van de dochters}\\

\haiku{Hij wenste Pirre,:}{tot Ho-ho te bekeren}{en zei derhalve}\\

\haiku{Hij zei midden in {\textquoteleft}{\textquoteright}, {\textquoteleft}{\textquoteright};}{de geschiedenisTot ziens}{engoedemorgen}\\

\haiku{Doch de goede man,.}{was zo geschokt dat hij hun}{niet kon antwoorden}\\

\haiku{Want waar gebrek wordt,.}{geleden daar teert de mens}{in gezondheid weg}\\

\haiku{{\textquoteleft}Ik had het zo graag,,{\textquoteright}.}{willen voleinden tijdens}{mijn leven zei hij}\\

\haiku{{\textquoteleft}Ik heb een appel, -!}{met een wurm erin en ik}{h\'o\'ud zo van wurmen}\\

\haiku{Doch wanneer \`allen,.}{hun zorgen verliezen slinkt}{de dankbaarheid snel}\\

\haiku{{\textquoteleft}Je hebt hun alles,.}{ontnomen wat de hemel}{hun had beschoren}\\

\haiku{Zo naderde de,.}{dag dat de Keizer voor het}{eerst zou afdalen}\\

\haiku{Iedereen vermocht,.}{te begrijpen dat hem de}{dood slechts kon wachten}\\

\haiku{Want het was stil - veel,.}{stiller dan het ooit in een}{leeg bedehuis is}\\

\haiku{Nu was de pastoor (}{niet zo-maar de eerste}{de beste priester}\\

\haiku{Maar het Kruisbeeld hielp,.}{niets en de vragen werden}{heel goed beantwoord}\\

\haiku{Het was hem aan te,.}{zien dat hij in gedachten}{haastig iets telde}\\

\haiku{Niettemin was de.}{liefdadigheid gebaat met}{haar tien miljoen tun}\\

\haiku{{\textquoteleft}Je bent uit de hel -.}{opgeborreld en daar kun}{jij \'o\'ok niks aan doen}\\

\haiku{Het was ontzettend,.}{enerverend we wisten geen}{naam te bedenken}\\

\haiku{{\textquoteleft}Ik word dadelijk,.}{begraven en dat gebeurt}{niet iedere dag}\\

\haiku{- Maar een paar dagen.}{later vond ze de eerste}{cursus in de bus}\\

\haiku{Zo begon dat, en,.}{zo ging het door tot zij er}{niet meer buiten kon}\\

\haiku{Nu zal iedereen,!}{weten dat ik werkelijk}{een heilige ben}\\

\haiku{{\textquoteright} De oude pater.}{voelde zijn ziel vollopen}{met medelijden}\\

\haiku{Hij was de eerste,.}{die haar sedert lange tijd}{bij haar naam noemde}\\

\haiku{Daar kwam het larfje,.}{weer en het hielp ook deze}{zieke naar boven}\\

\haiku{Want ook boven het,.}{water is een hogere}{macht die grenzen legt}\\

\haiku{God schiep de mens, en.}{Hij schept hem nog aldoor en}{Hij zal hem scheppen}\\

\haiku{Zijn voeten heten {\textquoteleft}{\textquoteright} {\textquoteleft}{\textquoteright} {\textquoteleft}{\textquoteright}.}{Vader enMoeder en zijn}{hoofd heetNageslacht}\\

\haiku{De ruiten waren,.}{stuk de gordijnen hingen}{vergaan op de grond}\\

\haiku{{\textquoteright} hoorde men de Pang.}{in zijn troonkamertje voor}{zich heen bulderen}\\

\haiku{Een feit is, dat de.}{Pang op statige benen}{de Kamers doorschreed}\\

\haiku{Hij troonde me mee,.}{naar een lunchroom voor een kop}{koffie met slagroom}\\

\haiku{Anderen keken,.}{en glimlachten ook het leek}{wel besmettelijk}\\

\subsection{Uit: Met liefde en respect. Deel 1: Het devies}

\haiku{als een donkere,.}{holte waarin iemand een}{sleutel omdraaide}\\

\haiku{Weer voelde Didier:}{enige hoop in zijn diepste}{verlangens doemen}\\

\haiku{een pijl, gonzend en.}{haarscherp afgeschoten op}{het verliefde hart}\\

\haiku{En observeerde, -.}{hoe zij rechter-op kwam te}{zitten ze kuchte}\\

\haiku{{\textquoteleft}Wij moeten erop -,{\textquoteright}.}{achten dat het op peil blijft}{voltooide hij week}\\

\haiku{Soms dacht hij in z'n,}{eentje te weten wat voor}{een lief dom gansje}\\

\haiku{Mary bepeinsde;}{intussen dat Egelsbergh haar}{heel anders vasthield}\\

\haiku{{\textquoteleft}Ja,{\textquoteright} beaamde hij, {\textquoteleft}.}{peinzenddat zou ik aardig}{hebben gevonden}\\

\haiku{Mary leunde heel.}{voorzichtig naast hem tegen}{het gelakte hout}\\

\haiku{gebaad als ze werd.}{door het stortbad van pastoors}{diepzinnigheden}\\

\haiku{Mary was nog zo -.}{jong Antoine was ook nog}{maar vierentwintig}\\

\haiku{Het land veerde op.}{en ontwaakte voorzichtig}{tot bloei-poging}\\

\haiku{naar den hemel ging!}{wou ze zo g\`ere mevrouw}{gesproken hebben}\\

\haiku{Haar handen waren,;}{klaar met troosten breien en}{eten-bereiden}\\

\haiku{{\textquoteright} {\textquoteleft}Zeker niet, als het,{\textquoteright}.}{een biechtgeheim is stemde}{Antoine vroom in}\\

\haiku{Zijn vrouw blikte snel.}{en wantrouwend van onder}{haar dwaas chasseurtje}\\

\haiku{Maar ze kreeg een schoon,{\textquoteright}.}{jungsken vertelde meneer}{pastoor gedempt}\\

\haiku{Vreet God en Maria,,!}{op en verbrand m\`en d\`e ik}{niet meer bestoai}\\

\haiku{Dan mocht de Vrouwe.}{wel een prachtige poffer}{tentoonstellen}\\

\haiku{{\textquoteleft}Moar ge krijgt 'et ok,{\textquoteright}, {\textquoteleft} '!}{nie had Sjef geantwoordik}{hout lekker zelf}\\

\haiku{opeens zag men in,.}{dat hier de juiste woorden}{waren gebezigd}\\

\haiku{Mevrouw van het Huis -.}{stapte sjust van heuren fiets}{zij was hier ook klant}\\

\haiku{{\textquoteleft}En dan zetten we,{\textquoteright}, {\textquoteleft}:}{d'r nen bordje bij vulde}{ze aanwaarop stoat}\\

\haiku{In stilte besloot,:}{ze de eventu\'ele zoon naar haar}{vader te noemen}\\

\haiku{{\textquoteright} Mary verkilde.}{en mompelde haperend}{dat mijnheer uit was}\\

\haiku{Maar Mary vond het,.}{dienstig nu van onderwerp}{te veranderen}\\

\haiku{{\textquoteleft}Voorzover ik dit,.}{huis waardeer heb ik deze}{ruimte het liefste}\\

\haiku{En meteen gaf hij,.}{zo'n verschrikkelijke gil}{dat hun oren tuitten}\\

\haiku{Een doodstille roep.}{die alles zou overklinken}{als Het Ogenblik kwam}\\

\haiku{Het sluike stappen,.}{van de dragers de kist die}{zich maar liet voeren}\\

\haiku{En ze schaamde zich -.}{ze sloot de ogen om zichzelf}{niet te beseffen}\\

\haiku{{\textquoteleft}Wat ben ik blij, dat,{\textquoteright}.}{ik dit eens mag zien zei ze}{zeer  hartgrondig}\\

\haiku{Aan Leentje maakte.}{zij haar wens kenbaar Mevrouw}{te mogen spreken}\\

\haiku{Ach ja, ach ja, de.}{bezoekster  moest daarvan}{heen en weer wiegen}\\

\haiku{en met dat doel had.}{zij tamelijk lang in het}{klooster verbleven}\\

\haiku{Juffrouw Van Toossen,{\textquoteright}}{komt om mij te waarschuwen}{tegen Miet Lintjen}\\

\haiku{En die afgang was:}{eveneens statiger gewenst}{dan geworden}\\

\haiku{Ze legde haar hand,:}{even op de andere arm}{van Miet en zei zacht}\\

\haiku{dit alles in de.}{dagen van winterfeest op}{haar pad te vinden}\\

\haiku{{\textquoteleft}Ja, dat geloof ik, -,,,...{\textquoteright}}{wel ja Joujou of Kiki}{of Trala of Foeifoei}\\

\haiku{{\textquoteright} Nee, ze voelde zich.}{onder de schijn van volmaakt}{geluk heel alleen}\\

\haiku{En Janus had zich -?}{onhandig gevoeld wat moest}{je daarop zeggen}\\

\haiku{In de winkel werd.}{druk gesproken over Nilles}{van den Bollebek}\\

\haiku{{\textquoteright} lispelde een vrouw,.}{die haar mond niet langer kon}{weerhouden van spraak}\\

\haiku{En Miet glom alsof.}{haar ziele gepoetst brons was}{uit engelhanden}\\

\haiku{Een beven zonk door,,}{haar leden alsof ze met}{iets gruwelijks sprak}\\

\haiku{Het hinderde niet -.}{of er iemand in huis was}{of niet ze praatte}\\

\haiku{Nee, dat dee ze niet -.}{ze trok de benen onder}{het lijf en stond recht}\\

\haiku{- Antoine zal je,.}{snel berichten en alles}{in orde maken}\\

\haiku{Ze liep heel langzaam,.}{naar het terras en zocht een}{papier en potlood}\\

\haiku{{\textquoteright} Maar vriendin Marie.}{Antoinette heette ook}{niet naar de duivel}\\

\haiku{Het Huis beleefde.}{dagen van glorie en de}{gloed van jong leven}\\

\haiku{Wist een vrouw ooit, waar...?}{haar echtgenoot de gelden}{vandaan toverde}\\

\haiku{En dan stondt ge in,}{het keurige vertrek waar}{onder glimstolpen}\\

\haiku{{\textquoteleft}Want d\`e huis was toch,!}{van Egelsbergh en d'r kan wel}{fortuin in zitten}\\

\haiku{Soms dacht Antoine,,.}{herhaling te zien alsof}{het een spel betrof}\\

\haiku{{\textquoteleft}Ik dacht dat je in,...{\textquoteright},.}{Itali\"e zat of ergens Ja}{dat deed Claire ook}\\

\haiku{Toine voelde zich!}{onprettig alert tegenover}{zijn eigen vrouw}\\

\haiku{Wat kon er dus nog?}{voor roerends tot stand komen}{in hun lieve kerk}\\

\haiku{En die lach had hij.}{nu om de lippen van het}{mevrouwtje gezien}\\

\haiku{En ik doe het nou,...{\textquoteright} {\textquoteleft},!}{ook pastoorM'n goeie vriend wat}{bent ge zwaarwichtig}\\

\haiku{Zelfs de goede wil.}{van een zieleherder is}{niet altijd genoeg}\\

\haiku{Een Kerstnacht-mis...?}{mee echt levende mensen}{in het stalleken}\\

\haiku{Sinterklaas kwam uit.}{Spanje en voer na enig gul}{gedrag weer terug}\\

\haiku{het smadelijke.}{grinniken en w\`egkijken}{van dorpsgenoten}\\

\haiku{{\textquoteleft}Ik denk, dat ik maar!}{eens naar de commissaris}{van politie ga}\\

\haiku{Maar in het bos had.}{een grote jongen haar het}{geld afgenomen}\\

\haiku{- {\textquoteleft}Waarom moeten wij?}{medelijden hebben met}{die armoedzaaiers}\\

\haiku{en men vroeg elkaar,?}{wat de buitenwacht daarmee}{ook van node had}\\

\haiku{{\textquoteleft}Zoudt ge d'r niks voor, '?}{voelen opt Gavenoord}{te komen wonen}\\

\haiku{Ze oogde hem z\'o,.}{recht in zijn hart dat hij er}{bijna van verschoot}\\

\haiku{{\textquoteright} en in zijn werkkiel;}{en pillow-broek stapte}{Ruur in naast mijnheer}\\

\haiku{{\textquoteleft}Moet ik me door een,?}{vrouw laten voorschrijven hoe}{ik moet handelen}\\

\haiku{Het was verfrissend -, -.}{iemand die haar nodig had}{die haar een taak bood}\\

\haiku{{\textquoteleft}De hekken van mijn,,{\textquoteright}.}{hart zijn d\`e nie kaploan}{zei ze vriendelijk}\\

\haiku{{\textquoteright} {\textquoteleft}Ach,{\textquoteright} antwoordde haar, {\textquoteleft}!}{man over-volwassenwat}{h\`elpt je best doen nou}\\

\haiku{En intussen wist.}{ze dat kindje groeiende}{binnen haar lichaam}\\

\haiku{Och, mevrouw, - 'n mens,{\textquoteright}.}{hee wel es zurge zei vrouw}{Van Mosse klankloos}\\

\haiku{Hij bleek toch in de;}{kerk al een ietwat schrille}{figuur te worden}\\

\haiku{Het lieve ouwe.}{mens had een jurkje gebreid}{voor de luiermand}\\

\haiku{Bij het vuur, in een;}{soort lila tent omdat ze}{zo omvangrijk werd}\\

\haiku{Amadeetje was nu -!}{ruim twee-en-een-half het zou}{zo allerliefst zijn}\\

\haiku{De oude Koning,.}{boog het eerst en vertelde}{van hun verre tocht}\\

\haiku{Vrouw Besonder was,.}{de ganse avond kalm geweest}{zoals meestal}\\

\haiku{Dat zij zo totaal,.}{ontkleed waren shockeerde}{haar een kort moment}\\

\haiku{{\textquoteright} Maar Toine lachte.}{met zijn al te heldere}{ogen op haar gericht}\\

\haiku{Antoine zweeg, raar,. {\textquoteleft}?}{op zijn hoede als voor een}{verkeerd seinClairtje}\\

\haiku{{\textquoteright} {\textquoteleft}O,{\textquoteright} antwoordde Sjef, {\textquoteleft},.}{gauw dan mar want me klaante}{stoan de wachte}\\

\haiku{Hij was zo totaal,.}{niet de beklaagde dat ze}{allemaal zwegen}\\

\haiku{maar ze d\`acht aan die - - {\textquoteleft}?}{twee ranke lichamenBlijf}{je altijd bij me}\\

\haiku{Een vriendelijke,.}{mannenstem informeerde}{of mijnheer thuis was}\\

\haiku{En de boerin die,,:}{had gesproken wendde zich}{tot mevrouw en vroeg}\\

\haiku{De Egelsberghs volgden.}{in een geleende wagen}{van goede vrienden}\\

\haiku{Toen de wagen voor,.}{het bordes stopte schrok de}{echtgenoot wakker}\\

\haiku{En nog nasnuivend,;}{stapte ze uit bed en ging}{naar de linnenkast}\\

\haiku{hij was geweest op,.}{het fr\^ele blonde meisje}{Mary van Genthen}\\

\haiku{Tussen twee wee\"en,.}{bedacht ze dat hij verloofd}{was en sprak daarover}\\

\haiku{de dorpsjongen wist,!}{een holle boom waarin je}{je hand kon steken}\\

\haiku{Omdat het bloed ons -.}{bindt aan leven en aan dood}{we blijven soamen}\\

\haiku{en oorhangers als - -... {\textquoteleft}}{van een koningin en een}{armband en een speld}\\

\haiku{Samen begonnen.}{ze hem te wassen en het}{bloed af te weken}\\

\haiku{Johan reed de wagen,.}{tot vlak voor de kerk waarvan}{de deuren openstonden}\\

\haiku{maar tegelijk was:}{er een vaag gevoel in haar}{van teleurstelling}\\

\haiku{Mary besloot, dat,;}{Johan alleen naar huis moest de}{auto wegbrengen}\\

\haiku{Wat wist zij ook van?}{woede en van Ter Tuynen}{Egelsberghse speelsheid}\\

\haiku{Maar Classen had hem,.}{meegetroond en Mary had}{hem binnengehaald}\\

\haiku{en daarbij was zij.}{met een voet in een molshoop}{te land gekomen}\\

\haiku{Ach, als ze het slechts (...)}{had gez\`egdze had het niet}{hoeven te menen}\\

\haiku{Z{\'\i}j was begonnen,.}{vrouw Van Mosse haar stukske}{land af te nemen}\\

\haiku{een levensgroot zicht:}{op huwelijksgeluk in}{de rijke wereld}\\

\haiku{Johan, als je tijd hebt,?}{wil je dan eens kijken naar}{de kinderwagen}\\

\haiku{{\textquoteleft}En waarom hedde?}{gij d\`e dan loater nie}{tegen m\`en gezegd}\\

\haiku{Antoine moest weg -, -;}{ze wist niet waarheen zij kreeg}{de auto met Johan}\\

\haiku{{\textquoteleft}Ik dreig ginnen mens,{\textquoteright}, {\textquoteleft} {\textquotedblleft}{\textquotedblright},.}{antwoordde de Muntik zeg}{heu tege m'n perd}\\

\haiku{Het wapen glipte,.}{Bollebek uit de hand de}{man zelf wankelde}\\

\haiku{Barntje werd namens.}{de pastoor thuisgebracht door}{een grote jongen}\\

\haiku{En hij hield pas op,.}{toen Noud hem een dubbeltje}{beloofde \`en gaf}\\

\haiku{Ze wankelde op.}{tastende voeten naar de}{deur en opende die}\\

\haiku{zodat men toch als, {\textquoteleft}{\textquoteright}.}{een soort gezin tezamen}{bleefonder vrienden}\\

\haiku{En ver achter haar.}{leunde Sjef Castel tegen}{een pilaar en keek}\\

\haiku{Haar glimlach droeg een -.}{vernis van schreien haar ogen}{speelden niet meer mee}\\

\haiku{{\textquoteright} maar terwijl ze sprak, -!}{zag ze een schrik over zijn ogen}{zwemen snel weer weg}\\

\haiku{{\textquoteleft}Je bent een prachtig,!}{wijf ik  vind je na elk}{kind mooier worden}\\

\haiku{{\textquoteleft}Ik met een dikke,,!}{buik en zwangerschapsvlekken}{en luiers tellen}\\

\haiku{Maar hij greep haar hand,,.}{en kuste die terwijl hij}{haar stevig vasthield}\\

\haiku{{\textquoteleft}Ja, maar als ik zo,!}{dicht mogelijk bij je blijf}{word je weer zwanger}\\

\haiku{Ik worstel ook niet,{\textquoteright}.}{spiernaakt met hem bracht Mary}{zeer lief naar voren}\\

\haiku{Aan de telefoon -.}{kwam een lieve stem dat was}{de vrouw des huizes}\\

\haiku{As d'r engelkes,!}{z\`en dan hebbe ze genoeg}{aander toak te doen}\\

\haiku{Hij nam het epistel.}{in handen en liet er zijn}{vingers over tasten}\\

\haiku{en zoons en dochters.}{hielden zich gereed om de}{moeder bij te staan}\\

\haiku{En iedere boer,!}{kon van nabij zien van welk}{edel ras zij waren}\\

\haiku{Harry poogde, de -.}{rem te grijpen maar dat zou}{ook gevaarlijk zijn}\\

\haiku{{\textquoteright} en hij zag even iets.}{als begin van een glimlach}{om haar mond plooien}\\

\haiku{Hij had allerlei -.}{zinnen op de lippen en}{sprak er niet een uit}\\

\haiku{{\textquoteleft}Hij moest jou maar es,{\textquoteright}.}{in je nakie zien snibde}{de jonge moeder}\\

\haiku{De afzender had,.}{alleen adres vermeld en dat}{was haar onbekend}\\

\haiku{Niet alleen in de;}{kerk was enkele malen}{voor hen gebeden}\\

\haiku{{\textquoteleft}Binnen enkele!}{jaren rijden de kleine}{burgers in auto's}\\

\haiku{{\textquoteright} en dan knelde hij,.}{haar in zijn armen dat ze}{geen adem kon vangen}\\

\haiku{Maar was de wijze,...?}{statige priester ook niet}{van groter waarde}\\

\haiku{{\textquoteleft}O ja, zegt u maar,!}{dat wij er om ongeveer}{tien uur zullen zijn}\\

\haiku{Mary huiverde.}{en was blij dat Antoine}{haar begeleidde}\\

\haiku{maar met het oog op.}{mijn wettige zoon kan ik}{haar kind niet echten}\\

\haiku{Als goed arts had hij,.}{meer oog voor haar verdriet dan}{voor het eigene}\\

\haiku{Voorzichtig nam hij,.}{de vracht over en legde die}{naast zijn eigen plaats}\\

\subsection{Uit: Met liefde en respect. Deel 2: Maskeravond}

\haiku{De buslijn loopt via.}{Den Deun door Woenselsven naar}{Rogunen en voort}\\

\haiku{Toen dat allemaal,...}{gebeurde waren wij ook}{in Woenselsven}\\

\haiku{Zijn rokken zijn te}{lang en zijn kop te vuil en}{zijn stem snijd mijn deur}\\

\haiku{Hij had een zwembad,.}{geopend in 1928 en een}{fiets-school in 1932}\\

\haiku{{\textquoteleft}Ik hoop, dat die naam,{\textquoteright}.}{jullie allen geluk zal}{brengen zei Mevrouw}\\

\haiku{En toen dit meneer,:}{pastoor ter ore kwam knikte}{hij beheerst en zei}\\

\haiku{Hij zei gewoon dat.}{ze niet mee kon doen daar ze}{het type niet was}\\

\haiku{Antoine was een,,!}{aardige vlotte kerel}{ze hield veel van hem}\\

\haiku{IK GELOOF NIET   .}{en meer onderaan stond zijn}{bloedeigen naam}\\

\haiku{Misschiens moest hij nou,.}{wel veur de radio komen}{om te proaten}\\

\haiku{Mijnheer pastoor De.}{Wett was een man van uiterst}{strenge begrippen}\\

\haiku{En Ceeske voelde.}{zich gebrand en gerafeld}{tussen deze twee}\\

\haiku{{\textquoteleft}Voel je zelf niet, dat?...}{het ongehoorzaamheid preekt}{en de geest zwak maakt}\\

\haiku{En van Antoines.}{kant kende ze eigenlijk}{alleen zijn vader}\\

\haiku{Hij vestigde een.}{paar geduldige ogen op}{de brutale vent}\\

\haiku{Hij wentelde zich,.}{in eigen rouw en trok daar}{het gezin in mee}\\

\haiku{hij zoende -) en ze -...}{leunde met haar hoofd tegen}{zijn borst \'e{\'\i}ndelijk}\\

\haiku{De zoon keek naar de.}{tassen en vandaar naar de}{liggende vader}\\

\haiku{Hij zweeg, terwijl zij.}{de tafel dekte en de}{borden neerzette}\\

\haiku{In de gang rook het.}{naar slechte tabak en het}{was er smoezelig}\\

\haiku{{\textquoteright} En van zijn hoge:}{toren uit repliceerde}{haar fiets-leraar}\\

\haiku{Twee dagen later;}{dwarrelde er een bonte}{postkaart in de bus}\\

\haiku{En wat was er veel,.}{voor een Egelsbergh-zoon}{om nu te denken}\\

\haiku{Snotterend verrees.}{moeder Van Drimmelen en}{voegde zich bij haar}\\

\haiku{Ze schudden mekaar.}{stevig de hand. En elk ging}{weer zijn eigen weg}\\

\haiku{Het was een kille,,.}{dag nevelig en kaal een}{afwijzend ogenblik}\\

\haiku{{\textquoteleft}Goeiendag,{\textquoteright} zei de,,.}{deken glimlachte knikte}{en schoof zijn stoel bij}\\

\haiku{{\textquoteright} herhaalde Deetje,.}{hardop en bezichtigde}{Sjef nadrukkelijk}\\

\haiku{Hij knielde bij haar,.}{neer sloeg zijn arm om haar hals}{en vroeg wat er was}\\

\haiku{Laat die nacht, sliep ze -.}{eindelijk in diep-weg}{voor enkele uren}\\

\haiku{Mary dacht, dat ze!}{later nooit weer marsmuziek}{zou kunnen horen}\\

\haiku{Mary had op de -.}{radio moeten letten het}{maakte haar doodziek}\\

\haiku{Het was haar, of er.}{een groot licht in de kamer}{begon te schijnen}\\

\haiku{Als de koningin,!}{zou worden doodgeschoten}{hadden we niets meer}\\

\haiku{Ach - we voelen ons,{\textquoteright}.}{zeker niet \`on-veilig}{antwoordde Toine}\\

\haiku{Hij sloeg de armen, ().}{om haar heen waar Noud bij stond}{die wendde zich af}\\

\haiku{Geen zwezerik, geen,,;}{ossetong geen nierstuk geen}{biefstuk van de haas}\\

\haiku{Dat was een bijnaam.}{geweest voor een jongen die}{dikwijls dronken was}\\

\haiku{De distributie,.}{was merkbaar doch er was van}{alles nog genoeg}\\

\haiku{aan het slot van de.}{dienst leken zij er geheel}{bij te behoren}\\

\haiku{Ze sprak met velen.}{van hen en wenste allen}{een zalig Kerstfeest}\\

\haiku{Toen het meisje hem, -}{aandiende kreeg Evelien de}{prikkels over de rug}\\

\haiku{Toen dit bericht het,.}{Huis bereikte keek Mary}{haar echtgenoot aan}\\

\haiku{hij mocht een beetje,.}{spelen in de huiskamer}{en moest vroeg naar bed}\\

\haiku{{\textquoteleft}Het wordt nu zaak, heel,.}{goed uit te kijken wie wij}{in ons huis halen}\\

\haiku{Hij liep moeilijk, zijn.}{gezicht was vervallen en}{hij had een blauw oog}\\

\haiku{Hij was dan toch een -.}{juweel al had ze dat nooit}{achter hem gezocht}\\

\haiku{Toen werd de priester -.}{witgloeiend wat natuurlijk}{toch zwak van hem was}\\

\haiku{Hij wou er verder -!...}{niets van weten een mens zou}{d'r ziek van worden}\\

\haiku{En waarom daar, als?...}{je zulke brede muren}{had in elk vertrek}\\

\haiku{Meteen liet hij zich,.}{over haar heen vallen en sloeg}{zijn armen om haar}\\

\haiku{Alleen Mevrouw wist, -.}{waar die vandaan kwamen het}{was een zoet geheim}\\

\haiku{Kortbesloten stond,.}{Mary op klaar om boos te}{worden op wie ook}\\

\haiku{hij zette alles,.}{neer in de kamer en sloot}{de gordijnen}\\

\haiku{Het wonder, dat zo'n,!... {\textquoteleft}}{grote meneer hem begr\'e\'ep}{en meteen maar hielp}\\

\haiku{Zijn  biechtvenster,,.}{uitzicht naar vergiffenis}{was toegevallen}\\

\haiku{De eerste gast op,.}{de Woens was papa Egelsbergh}{die geld kwam lenen}\\

\haiku{Hij was zo geplet,.}{dat hij zijn thee dronk en zich}{geweldig brandde}\\

\haiku{De vilten deur door,,.}{een poortje van latwerk een}{donkere trap af}\\

\haiku{Ze gleed bijna uit - -.}{ze moest beter kijken de}{trap was slecht verlicht}\\

\haiku{Hun stemmen galmden.}{bol en onbeheerst tegen}{het lage gewelf}\\

\haiku{En het lachende,:}{jongenskopje dat om de}{hoek van de deur keek}\\

\haiku{Daar wachtte hem een.}{goeiendag en vragen naar}{zijn welbevinden}\\

\haiku{Het leed geen twijfel.}{of de pastoor verdiende}{dit brood zeer moeizaam}\\

\haiku{Gr\"ucklich,{\textquoteright} herhaalde -.}{hij enkele malen en}{trok haar naar zich toe}\\

\haiku{Zij was volslagen.}{overstuur thuisgekomen met}{gescheurde kleren}\\

\haiku{En kijk nu, wie zij:}{om de hoek van de Lange}{Kruisstraat tegenkwam}\\

\haiku{Hoe vaak hadden de...?}{struiken om dat bankje heen}{gebloeid en gegeurd}\\

\haiku{Opeens dook er een.}{veelvuldige schaduw op}{uit het dorre blad}\\

\haiku{- Enkele mensen.}{vergaten de spertijd en}{vlogen hun huis uit}\\

\haiku{Ze hurkte neer en,, {\textquoteleft}!}{zamelde natte hete}{scherven roependAu}\\

\haiku{Mevrouw Mary ging -.}{ook na het eten niet weg zij}{bleef hen geleiden}\\

\haiku{en over die woorden.}{heeft juffrouw Marie Calchoen}{werkelijk geschreid}\\

\haiku{Het zoontje Aartje mocht.}{even binnenkomen om de}{gast te begroeten}\\

\haiku{{\textquoteleft}Wij hebben genoeg,.}{en van andere grond smaakt}{het soms lekkerder}\\

\haiku{En toen opeens werd, {\textquoteleft}!!}{haar peinzen verflard door een}{wilde stem dieHALT}\\

\haiku{Traag, druppelsgewijs.}{vormde zich een vermeld feit}{uit zijn gestamel}\\

\haiku{en voelde zich zo,.}{vreselijk arm dat ze er}{bijna ziek van werd}\\

\haiku{Hoe kwam het, dat zo,?...}{velen wisten waarheen hun}{schreden zich richtten}\\

\haiku{En toen een kreet - z\'o,.}{afgrijselijk dat allen}{naar boven stormden}\\

\haiku{Vader, moeder en.}{broertjes en zusje vlogen}{de kamer binnen}\\

\haiku{{\textquoteright} De gast knikte en {\textquoteleft}{\textquoteright}.}{glimlachte en herhaalde}{memorerendRuur}\\

\haiku{Hij kon zo prachtig!}{controleren of de stof}{echt begrepen was}\\

\haiku{De jongen was nu,.}{bijna veertien jaar oud en}{fors voor zijn leeftijd}\\

\haiku{Ze werd angstig, en,.}{als ze z\'e\'er nerveus was keek}{ze in de spiegel}\\

\haiku{En elk uur kwam het -.}{geraas nog nader het schoof}{bijna in hun huid}\\

\haiku{{\textquoteleft}Ik heb geen kleren - - -...{\textquoteright}}{en geen geld ik weet niet wat}{er zal gebeuren}\\

\haiku{Ze liep de kamer,.}{uit naar de achterdeur om}{die te ontsluiten}\\

\haiku{Berooid was ze, en -.}{verlaten straatarm in haar}{diepste gevoelens}\\

\haiku{De volgende dag!}{kwam de mare dat Breda}{eveneens was bevrijd}\\

\haiku{Zo had Mary dus.}{twee Duitse jongens gemeld}{bij de bevrijders}\\

\haiku{Nou ja, zij waren,{\textquoteright}.}{niet altijd onze vrienden}{vulde Toine aan}\\

\haiku{Zij kusten Amad\'e Auf,.}{Wiedersehen als echte}{Duitse kinderen}\\

\haiku{Mevrouw bleef binnen,.}{de deur die nog door de knecht}{werd open gehouden}\\

\haiku{Dieje Barnt was ne,,.}{vuil ventje moar joa zo'n keind}{wist toch van niks nie}\\

\haiku{Een schaduw van iets.}{onaangenaams somberde}{over Mary's denken}\\

\haiku{12 Het leven in.}{Nederland leek spoedig weer}{normaal te worden}\\

\haiku{De winkel stond vol,.}{mensen toen Bollebek daar}{binnenkwam dwalen}\\

\haiku{Hij praatte gewoon,.}{verder mee Piet en nam een}{geducht pak vlees mee}\\

\haiku{Zo vriendelijk en,.}{blank alsof er nimmer een}{conflict was geweest}\\

\haiku{Daar drentelde zij -,;}{dus mevrouw Mary in haar}{eigen stille tuin}\\

\haiku{Ik hoop dat je er,{\textquoteright}.}{nog eens over wilt nadenken}{drong zijn moeder aan}\\

\haiku{Het zakje viel op,.}{de grond hij had zijn vingers}{vol stroperigheid}\\

\haiku{Gewond en geknakt,.}{trok de troep weer naar Spanje}{en vandaar noordwaarts}\\

\haiku{Ze had Toine nog.}{nooit z\'o duidelijk en zo}{plat horen praten}\\

\haiku{De eigen teelt bleek,.}{niet goed dat jaar de zomer}{was te warm geweest}\\

\haiku{Sommige dames,.}{of heren negen zeer diep}{al voortwandelend}\\

\haiku{Het viel Mary op;}{dat meisjes meer wuifden dan}{de jonge kerels}\\

\haiku{Oma had flossig bleek,.}{haar gekregen met grote}{lokken wit daardoor}\\

\haiku{De blijdschap was zo,.}{onverbloemd dat Mary weer}{geheel opveerde}\\

\haiku{En ze beloofde,.}{hemelsblank de groeten te}{zullen overbrengen}\\

\haiku{{\textquoteright} kefte Wine - en.}{zij zetten hun aarzeling}{om in halve draf}\\

\haiku{Johan en Amad\'e waren.}{daarneven en brachten haar}{omzichtig overeind}\\

\haiku{{\textquoteleft}Gistelbergen,{\textquoteright} had,.}{hij gezegd en hij had zijn}{hand uitgestoken}\\

\haiku{Zo'n veurnoame mens had,.}{de Munt nie in z'ne buurt}{en d\`e kon \^ok nie}\\

\haiku{Maar in gesprekken.}{bemerkte je nooit iets van}{bezorgdheid of smart}\\

\haiku{Maar naast hem zat Barnt,,.}{ruiger van uiterlijk een}{beetje onrustig}\\

\haiku{Mary wist dat hij.}{zelf ook nooit een ander zou}{hebben geholpen}\\

\haiku{Tabak en te lang.}{gedragen ondergoed en}{roestige spijkers}\\

\haiku{Dat eten is al gek -?}{maar waar wou je die ouwe}{man van betalen}\\

\haiku{De ouwe heer had.}{dus gepeinsd over begraven}{van zijn geldwaarden}\\

\haiku{{\textquoteright} En toen zijn vader,}{zich na een uurtje nogmaals}{absenteerde liep}\\

\haiku{Nu was het vrede,.}{en nog was de wereld vol}{snikken en tranen}\\

\haiku{{\textquoteright} Babette lachte -!}{vrolijk maar het klonk zo ver}{en los van het bed}\\

\haiku{{\textquoteright} Babette belde.}{met een ijzerdraad-dun}{handje de zuster}\\

\haiku{Die middag brandde;}{ze haar kindervleugels aan}{de kunstmatigheid}\\

\haiku{Kleine, ontkleurde,...}{schim die zo dapper op dat}{bed had gelegen}\\

\haiku{Uit het huis kwam vaag -,.}{gerucht van spreken een deur}{die opende en sloot}\\

\haiku{Jan,{\textquoteright} zei Mary, {\textquoteleft}ik;}{ben gekomen om afscheid}{van haar te nemen}\\

\haiku{Dat was bijna op -.}{de hoek aan het begin van}{de Gevloekte Weg}\\

\haiku{maar een uur later:}{stond er op het bord met nog}{veel oranjer letters}\\

\haiku{Dit is vreselijk,,{\textquoteright},.}{voor een vader stemde ze}{toe met zachte stem}\\

\haiku{Maar ze wist - aan het -.}{wriemelen van zijn handen}{dat hij was geraakt}\\

\haiku{Hij zonk en verloor -.}{zijn laatste expressie hij}{werd een leeg lichaam}\\

\haiku{Nu en dan opende.}{ze de ogen en dacht aldoor}{te hebben gewaakt}\\

\haiku{nie zien d\`e de twee...,!}{families vechten om ne}{heilige Mis nee}\\

\haiku{De dode moest op.}{een schone eigen wagen}{worden gereden}\\

\haiku{dat \'e\'en de eerste.}{moest zijn in het middenpad}{van Sinte Maria}\\

\haiku{En met een lichte.}{hoofdnijging liep hij naar het}{Wit Engelpad}\\

\haiku{Op de bioscoop van:}{wijlen Sjef Castel stond nu}{met gouden letters}\\

\haiku{Hij was maar \'e\'en jaar -.}{ouder dan Toontje van de}{chauffeur die was tien}\\

\haiku{Alle dieren in,.}{de stal waren geluidloos}{hoewel er geen sliep}\\

\haiku{Wat m\'o\'est zo'n vrouw nou,...}{allenig in huis en dan}{achter in de tuin}\\

\haiku{Alleen Amad\'e keek blij,.}{met glimmende ogen en een}{aardige grinnik}\\

\haiku{Bij navraag bleek dat.}{Amad\'e erover had gejubeld}{tegen zijn broer Barnt}\\

\haiku{wat verschrikkelijk - - -{\textquoteright}.}{hoe k\`an dat Doch hij sloeg dicht}{op haar kille blik}\\

\haiku{Het konijn wipte.}{lekker naar binnen alsof}{hij had geoefend}\\

\haiku{en d\`e knijn kennik...{\textquoteright},!}{nie En ach bijna had hij}{Buikje vergeten}\\

\haiku{Ze meende bij al...}{deze dieren iets liefs in}{de ogen te speuren}\\

\haiku{Die mevrouw ontzet,,:}{dat begrijpt u. Maar hij blijft}{kalm en toont haar aan}\\

\haiku{Mary had zich snel -.}{willen afwenden maar iets}{had haar weerhouden}\\

\haiku{en knipte af, en -...}{rolde snel het volgende}{beeldvlak voor haastig}\\

\haiku{Ze hebben te veel,!}{gezien en gehoord en zelf}{aan de kant gestaan}\\

\haiku{Ze sprak over haar bruur, '...}{die ze altijn bietje}{gek hai gevonden}\\

\haiku{Veel bomen in het.}{bos bezweken en zwiepten}{omver als latten}\\

\haiku{Toine vertelde.}{zo breedvoerig als vleiend}{bleef voor de dochter}\\

\haiku{Binnen twee weken!}{sturen ze je terug als}{een geplukte kip}\\

\haiku{Daarna het ontbijt,.}{dat in volkomen stilte}{moest worden gebruikt}\\

\haiku{ik heb kinderen.}{van jouw zaad gedragen en}{ter wereld gebracht}\\

\haiku{een enkele groep,...?}{argelozen om over de}{grens te geraken}\\

\haiku{Maar wellicht zou zij.}{binnenkort een heel echte}{lady worden}\\

\haiku{Zij k\`on niet denken.}{aan een demonstratie van}{huwelijksgeluk}\\

\haiku{In de Bijbel staan.}{alleraardigste dingen}{over dat onderwerp}\\

\haiku{En om haar hals had.}{ze een gouden rozenkrans}{met paarse stenen}\\

\haiku{Ach, zij was ook zo,!...}{broos en lichtvoetig als zij}{door de gangen ging}\\

\haiku{Mary probeerde.}{altijd enige vreugde te}{brengen aan dit graf}\\

\haiku{Ik denk ook vaak aan,{\textquoteright}.}{dat arme kleine jonkje}{antwoordde Mary}\\

\haiku{En Mary vroeg niet,.}{verder zij vertrouwde haar}{prachtige dochter}\\

\haiku{Het begrijpen van.}{de situatie zonk als}{waanzin op Mary}\\

\haiku{Moar hoe d\`e ze dan?...}{tot dizze bepoaling}{had kunnen komme}\\

\haiku{Toen knielde meneer.}{pastoor neer en hij vouwde}{de haand en hij bad}\\

\haiku{Freer begreep tot diep,;}{in zijn vezels wat dat te}{betekenen had}\\

\haiku{Want ja, verbeeldje '!...}{dat Corrie doar beneen}{sluksken te kort kwam}\\

\haiku{Dat w\`as ze ook - maar...}{de werkelijkheid smaakte}{anders dan de droom}\\

\haiku{Antoine breidde.}{zijn armen uit en Derk deed}{een stap naar voren}\\

\haiku{Mary spitste de blik,.}{erheen maar sloeg meteen haar}{ogen neer als betrapt}\\

\haiku{Derk bracht hen naar een.}{grote kamer met blanke}{houten meubelen}\\

\haiku{{\textquoteright} Toine zat met een.}{vreemd scherpe expressie naar}{zijn kind te kijken}\\

\haiku{In 't voorbijgaan.}{zag Mary een paar keren}{de negerjongen}\\

\haiku{slank, met een tragisch,.}{masker alsof hij net uit}{een huilbui verrees}\\

\haiku{maar door geldgebrek.}{was hij gaan poseren voor}{naaktfotografie}\\

\haiku{Zij begreep nu dat;}{ze nooit de moeder van Aartje}{had kunnen worden}\\

\haiku{Toine begon steeds.}{dringender te klagen over}{alle verliezen}\\

\haiku{Daarnaast pijnigde.}{haar de vergeefse reis naar}{Denemarken}\\

\haiku{{\textquoteright}... en {\textquoteleft}Lisabeth, o,!,!...}{moar m'n h\'emel ge het oew}{voeten nie geveegd}\\

\haiku{{\textquoteleft}Moar, zuster Porta,!...}{w\`e bende slonzig mee oew}{ermen en bene}\\

\haiku{Kom, stoat op en doe,,!}{dees over want dit is lillek}{rauw gedoan heur}\\

\haiku{{\textquoteright} Andere zusters.}{kwamen toegelopen en}{stonden geschokt stil}\\

\haiku{daar stond, met haar kap,.}{dwars over het hoofd en zonder}{ijver in haar ogen}\\

\haiku{{\textquoteleft}Maar hij is nog geen, -!}{twintig hij woont daar maar hij}{verdient te veel geld}\\

\haiku{Ze wandelde met,;}{een boerin vertrouwelijk}{naar elkaar geneigd}\\

\haiku{Johan wipte van zijn,,.}{plaats liep om de wagen heen}{hield het portier open}\\

\haiku{Mevrouw Egelsbergh kreeg.}{haar zoon en haar juwelen}{mee terug naar huis}\\

\haiku{wat was ze geroerd,...}{geweest toen hij voor het eerst}{bij hen mocht branden}\\

\haiku{Ze kon toch ook wel?}{een paar goede kinderen}{hebben voortgebracht}\\

\haiku{Zo zaten zij te.}{zamen voor een kop koffie}{in de eetkamer}\\

\haiku{maar hielden alert oog.}{op gebalde vuisten en}{vlammende blikken}\\

\haiku{Hij sprak steeds sneller,.}{om niet in de rede te}{worden gevallen}\\

\haiku{Met zeventien jaar.}{modellen gestolen en}{adressen verraden}\\

\haiku{Hij was spierwit en.}{keek zijn bloedverwanten aan}{met brandende ogen}\\

\subsection{Uit: De mooiste verhalen}

\haiku{Dat is niet duur, voor,.}{zoveel historie als daar}{ligt opgetast}\\

\haiku{En dan mot 't snoer,,.}{in die kast ligge Wullem}{bij de ring van Hoen}\\

\haiku{{\textquotedblright} zei ze, {\textquotedblleft}want moe is...{\textquotedblright} -,?}{vannacht weggegaan Hoe vindt}{u zo-iets meneer}\\

\haiku{De onderwijzer.}{zesde klas keek het lijstje}{van de rollen na}\\

\haiku{Ze was ook dom, en.}{buiten de klas maakte ze}{geen uitzondering}\\

\haiku{Ik kleurde en dacht,.}{n\`og veel meer terwijl ik mijn}{mouw weer neerstroopte}\\

\haiku{{\textquoteleft}Noblesse oblige,{\textquoteright},.}{want we leerden al Frans en}{verstonden dit reeds}\\

\haiku{En we hebben nooit.}{weer iets geks van Elvira}{gehoord of gezien}\\

\haiku{Nee, het hondje had -,.}{haar hart maar toch niet zo als}{dat van Enze}\\

\haiku{Maar wie moesten ze nog? '.}{vragens Nachts luisterden}{ze tot in hun slaap}\\

\haiku{{\textquoteleft}Al zal ik d'r aan,{\textquoteright}, {\textquoteleft}!}{dood gaan zei hijik rij de}{hele weg terug}\\

\haiku{Ze kwam haastig naar,:}{binnen want ze begreep zijn}{hulpeloosheid niet}\\

\haiku{Gert heeft er wakker,.}{van gelegen want hij was}{opeens geen mens meer}\\

\haiku{Om dit waanbeeld te,:}{temmen heeft hij op een dag}{tegen Jans gezegd}\\

\haiku{Nou ja,{\textquoteright} antwoordde, {\textquoteleft},!}{Ganterdat moet je geheim}{houden vrouw Blokker}\\

\haiku{Hij kon nog geen drie.}{minuten in het water}{hebben gelegen}\\

\haiku{E\'en drager drukte,.}{op zijn borst om het water}{eruit te pressen}\\

\haiku{{\textquoteleft}Heb niet het hart, die,{\textquoteright}.}{ketting ook maar \'e\'en dag af}{te leggen zei Gert}\\

\haiku{Daar was ze ook even,.}{beteuterd van en ze wist}{zo gauw geen antwoord}\\

\haiku{En - misschien was het,...}{ook alleen medelijden}{wat haar zo diep trof}\\

\haiku{en ik ben zo bang,{\textquoteleft} -}{dat je n zult denken dat}{ik medelijden}\\

\haiku{De dokters zeggen,,}{dat ik de eerste tijd niet}{zal kunnen lopen}\\

\haiku{Kees waardeerde zijn,.}{vrouwtje bizonder en zij}{schatte hem ook hoog}\\

\haiku{Zo kwam hun laatste,.}{avond aan de C\^ote d'Azur}{en Hetty was stil}\\

\haiku{Ze waren altijd,.}{netjes gekleed en beleefd}{tot in de puntjes}\\

\haiku{maar Gijbert kwam met,.}{een waterpistool en schoot}{de vent in zijn ogen}\\

\haiku{Op een dag - hij was -.}{toen al zevenentwintig}{werd hij heel anders}\\

\haiku{Die borstelige -.}{wenkbrauwen en de rechte}{mond hij kende ze}\\

\haiku{Duidelijk zicht op,.}{het orkest een aardig punt}{in de zaalruimte}\\

\haiku{Bella was enige.}{tijd een veelgeziene gast}{bij hen thuis geweest}\\

\haiku{Ja, zo had hij het,.}{als kind genoemd wanneer hij}{naar die handen keek}\\

\haiku{En ze legde een.}{eerbiedige hand op haar}{eigen naveltje}\\

\haiku{Ze begon zo zacht,:}{te praten en opeens leek}{alles ijl aan haar}\\

\haiku{Ze voelde alleen,.}{heel sterk dat grote mensen}{schrokken van doodgaan}\\

\haiku{Ze kon er 's avonds,.}{bijna niet van slapen en}{kreeg een nachtlichtje}\\

\haiku{Bij de school stoeiden.}{de kinderen joelend en}{schreeuwend door elkaar}\\

\haiku{De knechten wisten:}{het hem niet te melden en}{de vrouwen schreiden}\\

\haiku{Zo kwam hij op een,.}{dag terug te keren naar}{de keizer Karel}\\

\haiku{{\textquoteleft}Zal ik vergaan van?}{smartelijke dorst naar de}{meren van haar ogen}\\

\haiku{Ik heb palmwijn voor,{\textquoteright}.}{je gesneden vleide de}{zon zijn beminde}\\

\haiku{Een enkele maal.}{echter werd haar verlangen}{sterker dan haar angst}\\

\haiku{{\textquoteright} En dan zou de zon,.}{weer weten dat de maan hem}{niet had vergeten}\\

\haiku{Er was geen wanklank,.}{of zorg tussen ons dan dat}{zijn bezit klein was}\\

\haiku{Haar levenloze.}{ogen waren wijd-open van}{onduldbaar lijden}\\

\haiku{Als de dood haar   -...}{had betoverd zodat zij}{hem niet meer liefhad}\\

\haiku{En in Kwangsi zou,.}{hij zijn waren verkopen}{en goud ontvangen}\\

\haiku{Hij kon van niemand,.}{anders  meer dromen dan}{van deze schone}\\

\haiku{{\textquoteright} Wellicht had Uma hem,;}{willen zeggen dat zij h\`em}{toch had gekozen}\\

\haiku{{\textquoteleft}Weet jij nog, wat je,?}{wilde overdenken toen ik}{was weggezonden}\\

\haiku{Het was een soort dolk,,.}{in vreemde golven gesmeed}{met een scherpe punt}\\

\haiku{Op een dag zag hij;}{een bouwmeester de stand van}{pilaren meten}\\

\haiku{Ze verlangden eten,.}{en waren zeer vriendelijk}{voor de jonge vrouw}\\

\haiku{De jongelingen.}{aten geconfijte vruchten}{en zij dronken wijn}\\

\haiku{Dan word je prachtig,}{donkerrood en je bevat}{vitamine C.}\\

\haiku{De kolonel doet -:}{zelf open en dat is een klap}{op Van Bommels hart}\\

\haiku{{\textquoteleft}Het is Serafien,{\textquoteright}.}{verbeterde een van de}{andere boeren}\\

\haiku{En nu is het laatst,.}{gebeurd dat Annabartje naar}{de bushalte liep}\\

\haiku{Hij vloog scherend laag.}{over  de grond en ging op}{het hekje zitten}\\

\haiku{Maar van het raam uit,.}{kon ze zien dat de merel}{van de beschuit at}\\

\haiku{En toen sloegen de,.}{twee klokken allebei vijf}{want zo laat was het}\\

\haiku{Het parelhoen moest.}{deze inzichten winnen}{in zijn eenzaamheid}\\

\haiku{{\textquoteleft}Vriend,{\textquoteright} zei ze stil, {\textquoteleft}ik, -}{weet niet wat er is gebeurd}{ver-weg klonk}\\

\haiku{Hij schaamde zich, dat.}{zo'n doodgewone plant een}{hekel aan hem had}\\

\haiku{Hij hoorde er iets,.}{vervelends in alsof dat}{een wonder zou zijn}\\

\haiku{Ze zouden me v\'o\'or.}{zonsondergang de ware}{wijsheid doen kiezen}\\

\haiku{Hij liep op stille.}{voeten tempelwaarts met zijn}{kostbare besluit}\\

\haiku{Op 't laatst zult u.}{niet meer geloven in pijn}{en slapeloosheid}\\

\haiku{{\textquoteright} Toontje zuchtte weer,.}{hij leek het gelach niet te}{hebben gewaardeerd}\\

\haiku{Ik meende aarde,,.}{scherven kralen en schatten}{te horen lachen}\\

\haiku{{\textquoteright} Maar reeds bukte hij,.}{nogmaals en wroette kort en}{fel in de kluiten}\\

\haiku{Er was duidelijk,:}{verschil in kleur en glazuur}{hij toonde het ons}\\

\haiku{Hij keek naar de agent,.}{die rookte en oplettend}{naar de lamp staarde}\\

\haiku{Ik heb er nooit pijn,.}{van gehad behalve de}{laatste twee dagen}\\

\haiku{Hun huwelijk was.}{nu niet bepaald een vorm van}{rose harmonie}\\

\haiku{Kees ging veel naar de.}{soci\"eteit en Betty}{had negen clubjes}\\

\haiku{{\textquoteleft}Ik zal een doosje,.}{uit mijn tas halen dan kun}{je hem meenemen}\\

\haiku{Daar was hij dus zelf,.}{met een stekende vierde}{rib als verklaring}\\

\haiku{{\textquoteright} Hij probeerde te,.}{glimlachen maar zijn vierde}{rib stak te hevig}\\

\haiku{er was meer wit in,.}{een schrille glimmer in zijn}{goedig zwart gezicht}\\

\haiku{{\textquoteleft}Het is kennelijk,{\textquoteright}.}{een intoxicatie stelde}{de professor vast}\\

\haiku{Langs de weg zitten,.}{was ook geen genoegen daar}{groeiden brandnetels}\\

\haiku{Mar die van jou, die,,{\textquoteright}}{het alles opgespaard toen}{jij most komme bitste}\\

\haiku{De inspecteur wou,.}{nog niet maar kon een grinnik}{ook niet beheersen}\\

\haiku{maar de kundigheid.}{van zijn huisdokter blijkt daar}{niet minder om}\\

\haiku{De pan stond treurig,.}{verlaten terzijde het}{bord was leeg en koud}\\

\haiku{Het was vreselijk -,?...}{zou hij dan toch een beetje}{geschokt zijn door iets}\\

\haiku{{\textquoteright} informeerde ze.}{nonchalant en schudde haar}{veren nog es op}\\

\haiku{Toen had hij haar te.}{pakken en beet haar met \'e\'en}{hap de kop af}\\

\haiku{Hij had haar toch tot,;}{het laatste toe gegeven}{wat ze verwachtte}\\

\haiku{Afscheid nemen van,.}{opa en oma vermaningen}{en groeten innen}\\

\haiku{Mamma bromde dat,.}{het een vieze lap was van}{een onbekende}\\

\haiku{{\textquoteright} zegt ze, {\textquoteleft}je weet 'r ',...!}{toch marn krui{\"\i}g broodje}{van te bakken h\`e}\\

\haiku{{\textquoteleft}Tante Merlina,{\textquoteright}, {\textquoteleft}.}{wees ik haar kies terechteen}{heks is geen dame}\\

\haiku{{\textquoteleft}Wat ben je toch een,!...}{afschuwelijk mens je bent}{precies je moeder}\\

\subsection{Uit: De porselein tafel}

\haiku{Ik ben blij, dat er.}{zoveel vrouwen in onze}{familie waren}\\

\haiku{De nieuwe woning,.}{lag buiten de Waterpoort}{aan de Lemsterweg}\\

\haiku{Alberdien moest het,.}{hoofd koel houden daar Speyer}{lang niet goedkoop bleek}\\

\haiku{Van het eens grote.}{gezin waren slechts twee broers}{en een zuster over}\\

\haiku{want Orne had de,.}{doek om zijn buik gewonden}{tussen broek en jas}\\

\haiku{zij was bevreesd voor,.}{verwijdering juist nu ze}{het derde kind droeg}\\

\haiku{Zij riep het kind bij,,.}{zich en vroeg hem ernstig wat}{hij daar gedaan had}\\

\haiku{Daar kon Orne maar,.}{nauwelijks om lachen want}{de mop was niet fijn}\\

\haiku{Was het leven dan?}{niets anders dan geboren}{worden en sterven}\\

\haiku{Hij vond haar om half.}{elf met kloppende hoofdpijn}{in de woonkamer}\\

\haiku{Nicht Pietje geleek.}{eensklaps een olielichtje op}{de laatste druppels}\\

\haiku{Op de weg naar huis.}{werd het rijtuigje omgonsd}{door de oostenwind}\\

\haiku{Deze, welke zij,;}{niet meer nodig had vond ze}{in een hoedendoos}\\

\haiku{En drie kommetjes,,.}{van groen net erwtensoep met}{draken buitenop}\\

\haiku{Er waren zelfs tot.}{in Leeuwarden verhalen}{over doorgedrongen}\\

\haiku{De avond verviel tot,;}{nacht de duisternis sloot haar}{hand om de ramen}\\

\haiku{Ze had het vreemde, - -...}{gevoel dit verdiend bijna}{verwacht te hebben}\\

\haiku{idem 1938 - De laars op- - -.}{de nek 19391944 Jozef duikt}{1946 De knopenman}\\

\haiku{TONNY VAN DER HORST,.}{Ik schreef een Kerstverhaal voor}{jonge meisjes 1946}\\

\subsection{Uit: Probleem in Aerdenberg}

\haiku{{\textquoteleft}Ja,{\textquoteright} zei Terry, die,.}{sedert hij detective}{is recht door zee gaat}\\

\haiku{Zelfs niet \'e\'en, want dan.}{zou Netty Brand dadelijk}{gealarmeerd zijn}\\

\haiku{Pas op, dat je kop,{\textquoteright}.}{d,'r niet af valt waarschuwde}{Terry sarcastisch}\\

\haiku{Ik heb mijn leven,{\textquoteright}.}{ingedeeld in kwartieren}{legde Terry uit}\\

\haiku{Wij werden volgens.}{afspraak bekend gemaakt als}{vrienden van Henri}\\

\haiku{Doch deze haalde,.}{de schouders op en wees op}{de commissaris}\\

\haiku{Maar is er dan geen?}{sprake geweest van n\`og een}{ruzie of zoiets}\\

\haiku{Toen hij weg was, keek,:}{Burgheem ons met een vage}{glimlach aan en zei}\\

\haiku{We beloofden het.}{woordelijk namens haar over}{te zullen brengen}\\

\haiku{Hij keek eens naar ons,:}{via het spiegeltje bij de}{voorruit en lachte}\\

\haiku{{\textquoteleft}Maar uw spreektrant zal.}{deze gevoelens zeker}{niet versterkt hebben}\\

\haiku{Hij voelde aan de,....}{toetsen en streek over de kast}{klopte op het hout}\\

\haiku{Hij tilde de sprei,.}{op en bezag het gladde}{lak met een loupe}\\

\haiku{{\textquoteright} {\textquoteleft}En -,{\textquoteright} hernam Terry, {\textquoteleft},....{\textquoteright} {\textquoteleft},.}{kiesexcuseer de vraag Gert}{Tot uw dienst meneer}\\

\haiku{En die keer dasse?!}{zo'n herrie kreeg met meneer}{Martijn over die pook}\\

\haiku{Veel erger, dan ik.}{van haar verwacht had en het}{verbaasde me zeer}\\

\haiku{{\textquoteright} vroeg ze glimlachend,.}{terwijl ze haar voet op de}{eerste tree zette}\\

\haiku{Het klonk spookachtig,,.}{en ik schaam me niet om te}{zeggen dat ik schrok}\\

\haiku{{\textquoteright} {\textquoteleft}Als ik het wist, Han,,{\textquoteright}.}{zou ik het je graag zeggen}{gaf hij ten antwoord}\\

\haiku{Dan had het in mijn.... {\textquoteleft}}{eigen verschijning w\`el zo}{aardig kunnen zijn}\\

\haiku{Het galmde kolkend,.}{langs de holle gangen in}{daverende echo's}\\

\haiku{Haalde achter de:}{rug van de sidderende}{vrouw zijn schouders op}\\

\haiku{Ik kwam naast hem staan:}{en luisterde vol spanning}{naar zijn conclusie}\\

\haiku{{\textquoteleft}Stel, dat Booner een klein, -{\textquoteright}.}{teder plekje heeft in haar}{hart Terry lachte}\\

\haiku{Want dat verwachtte,?}{je zeker toen je over iets}{griezeligs praatte}\\

\haiku{Ik heb rechercheurs,!}{gekend die bijna net zo}{slim waren als ik}\\

\haiku{Ik zag, dat Terry,:}{de vuisten balde maar hij}{zei beminnelijk}\\

\haiku{Om dan zodoende,?....}{aan Gert te verraden dat}{ze van Henri hield}\\

\haiku{Ik zou in staat zijn,.}{het je te vertellen van}{louter opwinding}\\

\haiku{en toen is 'ie naar....,....}{meneer Martijn gegaan om}{voorschot te vrage}\\

\haiku{{\textquoteright} {\textquoteleft}Dat is niet in een,{\textquoteright}.}{vloek en een zucht uitgelegd}{antwoordde Terry}\\

\haiku{Ik kan niet zeggen,.}{of het haat was of woede}{of vernedering}\\

\haiku{Van mij zul je je,.}{geld moeten afwachten tot}{je eraan toe bent}\\

\haiku{Menters stond op en.}{ging naar buiten om een paar}{agenten te roepen}\\

\haiku{{\textquoteleft}Ben je met meneer,?}{Martijn in de kamer van}{zijn broer geweest Gert}\\

\haiku{{\textquoteleft}Nou moet u zich niets,.}{laten vertellen door de}{dorpsmensen meneer}\\

\haiku{{\textquoteleft}En dan zul je juist....}{buiten de gemeente een}{dwarsweg links vinden}\\

\haiku{Je herkent 'm niet,.}{dus kijk hem ook niet te erg}{aan in het donker}\\

\haiku{Waarschijnlijk zal hij,.}{je willen binden en dat}{mag je niet toestaan}\\

\haiku{om me verslag uit,.}{te brengen en dan ga je}{naar dat afspraakje}\\

\haiku{De lange man naast.}{me maakte een beweging}{van verbazing}\\

\haiku{{\textquoteleft}Het had maar weinig,!....}{gescheeld of ik had je niet}{weer levend gezien}\\

\haiku{In plaats daarvan keek,.}{hij naar de grauwe hemel}{buiten en kuchte}\\

\haiku{{\textquoteleft}Ik wou u alleen,.}{maar even plagen om u wat}{op te monteren}\\

\haiku{Die ruzie met m'n....}{broer zat me eerlijk gezegd}{nog zo in het hoofd}\\

\haiku{Ik mocht hem graag om.}{de manier waarop hij zijn}{onkunde beleed}\\

\haiku{{\textquoteleft}En draait die nou ook, '?!}{de bak in omdat ik wat}{vanm verteld heb}\\

\haiku{Wie weet, hoe hij ons!}{en een gesprek met ons van}{zijn kant schilderde}\\

\haiku{Als ze wisten, hoe, -,}{verdacht ze zichzelf maken}{dat is juffrouw Booner}\\

\haiku{- Ik wou, dat je je,.}{eigen gezicht zo nu en}{dan eens kon zien Han}\\

\haiku{- Hier zat ik nou, met,....}{veel werk zodat ik me niet}{kon wijden aan kunst}\\

\haiku{Ik stak op mijn dooie:}{gemak een sigaret aan}{en informeerde}\\

\haiku{Het was duidelijk,.}{dat de twee Van der Lindens}{lucht voor haar waren}\\

\haiku{- Toen boog Terry het,,.}{hoofd alsof hij nadacht hoe}{hij moest beginnen}\\

\haiku{Terry ging in de,.}{gang waar we hem met Crommer}{hoorden fluisteren}\\

\haiku{Het was precies de,....}{wond die Martijn van Doff aan}{de slaap had gehad}\\

\haiku{Met een auto is '.}{zes nachts weer in het dorp}{hier teruggekeerd}\\

\haiku{Het was natuurlijk,.}{de overzetting waarbij ik}{hem geholpen had}\\

\haiku{Op het ogenblik dat,:}{hij de eerste hand drukte}{zei Arlette koel}\\

\haiku{{\textquoteright} De jonge vrouw kon.}{een helle triomf in haar}{ogen niet beheersen}\\

\haiku{{\textquoteleft}Zie je wel, dat je?!}{geen schoon ondergoed hoefde}{te laten komen}\\

\haiku{{\textquoteleft}Wat ben je toch een,{\textquoteright}.}{afschuwelijk ondier zei}{Terry afkeurend}\\

\haiku{Hij keek sprakeloos.}{en gekrenkt voor zich uit en}{trok zijn lippen in}\\

\subsection{Uit: Spiegel aan de wand}

\haiku{Om half acht is hij.}{de vrouwen voorbijgegaan}{met een korte groet}\\

\haiku{tot  de ouwe.}{Berrends op een ochtend stierf}{aan hartverlamming}\\

\haiku{Frank ontdekt er steeds,.}{weer een diepte in waarin}{hij betoverd staart}\\

\haiku{Voor een man is er,.}{toch altijd nog het werk dat}{hem interesseert}\\

\haiku{De tweede dag heeft {\textquoteleft}{\textquoteright},.}{zeverrek gezegd midden}{in de rose zaal}\\

\haiku{Er is geen mooier,,.}{huwelijk denkbaar zelfs nu}{nog dan het hunne}\\

\haiku{Ze dacht, dat hij maar...,?}{\'e\'en bewonderaarster had}{Pardon madame}\\

\haiku{De zaak zal haar niet, '....}{meer kunnen  gebruiken}{alst zichtbaar wordt}\\

\haiku{{\textquoteleft}Ik ben toch - {\textquoteright} ze zweeg, {\textquoteleft},{\textquoteright}.}{een fatsoenlijke vrouw had}{ze willen zeggen}\\

\haiku{Het lijkt wel, of er.}{een snelle trilling door haar}{hele lichaam gaat}\\

\haiku{Fr\"oken Hannsen.}{lacht en neemt een nieuw blaadje}{complexion paper}\\

\haiku{Ik voor mij vind het..{\textquoteright} {\textquoteleft},?}{geen aanbeveling voor de}{zaakPardon mevrouw}\\

\haiku{Iedereen weet, dat {\textquoteleft}{\textquoteright}.}{hetBella Monica geen}{filialen heeft}\\

\haiku{Haar japon zit van;}{achteren net zo volmaakt}{als van  voren}\\

\haiku{En zo zet Helen.}{Carfew haar snoepzuchtige}{man op dieet}\\

\haiku{Juffrouw Liza bijt.}{op haar lip en vervloekt in}{stilte Mr. Carfew}\\

\haiku{Haar gezicht straalt zo,.}{op dat de afkeuring in}{Frank's ogen wegsmelt}\\

\haiku{Om die v\`ent, die haar.}{de glorie van het nieuwtje}{zo doodleuk ontneemt}\\

\haiku{Knikt, voor het oog der,.}{wereld en schrijdt regelrecht}{naar cabine 11}\\

\haiku{- Er moet natuurlijk.}{een antwoord komen van de}{ondergeschikte}\\

\haiku{Ik heb haar een paar.}{dagen geleden gezien}{met een jongeman}\\

\haiku{Mademoiselle:}{dient eigenlijk alleen maar}{als chaperonne}\\

\haiku{{\textquoteleft}Alle kleuren zijn,{\textquoteright}.}{bij uw gelaat en type}{aangepast zegt Frank}\\

\haiku{{\textquoteleft}Levendig en diep,.}{zonder onaangenaam in}{het oog te lopen}\\

\haiku{de kleur is alleen.}{een sterkere nuance}{van die van de lait}\\

\haiku{- O, ja! 't Maakt je.}{oren aan het gloeien en je}{hart aan het kloppen}\\

\haiku{Hij hoeft haar nooit iets,.}{te vertellen en zij zegt}{ook bijna nooit wat}\\

\haiku{Maar het is een lang.}{slank meisje met blond haar en}{een ernstig gezicht}\\

\haiku{De rouge ligt als.}{een los poeierig waas op}{de jukbeenderen}\\

\haiku{Maar dan zou hij toch.}{niet zo snel zijn hart aan een}{ander verliezen}\\

\haiku{Onnozel houdt ze.}{het dampende kopje bij}{haar bevende mond}\\

\haiku{Dag juffrouw,{\textquoteright} zegt ze,.}{vriendelijk en toch niet al}{te gemoedelijk}\\

\haiku{Dag mevrouw,{\textquoteright} antwoordt:}{het meisje en grijpt meteen}{de huis telefoon}\\

\haiku{Dan zinkt ze met een.}{sidderende zucht neer in}{de behandelsfoel}\\

\haiku{{\textquoteright} En als je 't zelf,?}{hoort moet je haast lachen want}{dat k\`an immers niet}\\

\haiku{op een gegeven:}{ogenblik legde hij zijn hand}{over de hare heen}\\

\haiku{Het overhemd was van,.}{zo dunne zijde dat het}{de huid zelf geleek}\\

\haiku{Of was ze alleen?}{maar niet zo geraffineerd}{als de anderen}\\

\haiku{Als hij zijn gezicht,,,.}{dichtbij het hare opzij}{wendt herkent Bob hem}\\

\haiku{Mimi's stem klinkt hoog.}{en haar lachjes zijn een snoer}{kristallen kralen}\\

\haiku{Is er niet een tram,?}{of een bus of iets waarmee}{hij naar Ina kan gaan}\\

\haiku{ze vindt het zelf niet,.}{aangenaam want ze is er}{niet rustig onder}\\

\haiku{- Frank bedenkt, dat hij.}{niets van haar leven buiten}{het instituut weet}\\

\haiku{Is het geen teken,?}{dat hij ontevreden is}{over haar uiterlijk}\\

\haiku{Ze kijkt aldoor in.}{de diepte van haar rijkdom}{en dan duizelt ze}\\

\haiku{Juffrouw Diller neemt.}{een abonnement voor twintig}{behandelingen}\\

\haiku{dan was die heer toch.}{nooit een ondergeschikte}{van een der dames}\\

\haiku{Het pijnlijkste vindt,.}{de Barones echter dat}{Jo toen kribbig werd}\\

\haiku{De vrolijkheid op.}{Johanna's gezicht stolt tot}{een kille hardheid}\\

\haiku{Achter Johanna.}{schrijft de assistente snel}{het rapport even bij}\\

\haiku{Het doet er niet toe,,.}{hoe ze hem zal begroeten}{of helemaal niet}\\

\haiku{Voor de huidvoeding, {\textquoteleft}{\textquoteright}:}{koos ze net als elke vrouw}{allespecials}\\

\haiku{Als zodanig is;}{Maria's dood-zijn ook}{onbegrijpelijk}\\

\haiku{Ze vindt het opeens,.}{grof van zichzelf om zo vroeg}{gekomen te zijn}\\

\haiku{{\textquoteright} De blonde en de:}{bruine schateren en de}{eerste informeert}\\

\haiku{Ze komt uit een land,;}{waar de vrouw minstens evenveel}{waard is als een man}\\

\haiku{{\textquoteleft}Wie iets niet begrijpt,.}{of ergens over spreken wil}{moet bij mij komen}\\

\haiku{Meneer Geerts is de.}{oprichter van de zaak en}{ook de bezitter}\\

\haiku{De assistente.}{laat zich er dan ook niet door}{van de wijs brengen}\\

\haiku{{\textquoteleft}Ik heb nog nooit een,.}{assistente ontmoet die}{zittend kon werken}\\

\haiku{{\textquoteright} {\textquoteleft}Oh,{\textquoteright} miss Jean wiegt op, {\textquoteleft}?}{haar kruk van bevreemdingmaakt}{that a verskil}\\

\haiku{Hierop slaat Frank zijn:}{ogen neer en zijn stem breekt het}{vragend zwijgen open}\\

\haiku{{\textquoteleft}En dat ik me niet,....}{vergis bemerk ik uit de}{woorden van die vrouw}\\

\haiku{En juist omdat ze,,.}{nu doorv\'o\'elt wat hij meent kan}{ze hem niet troosten}\\

\haiku{{\textquoteleft}En als je je plaats,.}{verlaat dan kijk {\`\i}k je niet}{meer aan Frank Berrends}\\

\haiku{Zij neeg heel even het.}{hoofd en richtte het toen wel}{tweemaal zo hoog op}\\

\haiku{De mensen zijn wreed,....}{en wellicht ziet het er erg}{onsmakelijk uit}\\

\haiku{we mogen zelf bij, {\textquotedblleft}{\textquotedblright} {\textquotedblleft}{\textquotedblright}.}{alles beslissen of het}{ja ofnee zal zijn}\\

\haiku{Dan komt er eensklaps.}{een grote geldkrapte in}{zijn zakenleven}\\

\haiku{{\textquoteright} {\textquoteleft}De handel behoort,{\textquoteright}.}{een schaakspel te zijn zegt Geerts}{met trillende stem}\\

\haiku{{\textquoteright} {\textquoteleft}Meestal,{\textquoteright} antwoordt,.}{Frank onbewogen en schenkt}{een borrel in}\\

\haiku{Dan is er nog een,:}{wonderlijk huis dat vele}{aanbouwsels vertoont}\\

\haiku{of alleen maar het:}{plotselinge ervaren}{van de romantiek}\\

\subsection{Uit: Ter ere van}

\haiku{Vort, voortrennen - we,!}{moeten nog negentienmaal}{het dorp rond vandaag}\\

\haiku{{\textquoteleft}Zoek dan tenminste,!...}{een mooi medailletje uit}{of draag iets \`anders}\\

\haiku{Hij slurpte beschaafd.}{zijn koffie en keek over de}{rand van het kopje}\\

\haiku{- Ze schudden Ron om,.}{beurten de hand terwijl ze}{naar de deur gingen}\\

\haiku{Als d'r es 'n wijf,.}{opkomt knikt ze mee d'ren}{kop en goait weer}\\

\haiku{Pak dus je koffer,!}{ik kom oe vanmiddag om}{drie ure hoalen}\\

\haiku{{\textquoteright} zo voldaan, alsof.}{hij ze allemaal met de}{hand had gevangen}\\

\haiku{en twee dagen na:}{de begrafenis zei Huub}{tegen zijn moeder}\\

\haiku{Zo kwam hij op het.}{idee dansles te gaan nemen}{in het naaste dorp}\\

\haiku{Spanjaarts richtte zijn:}{onrustige oogjes op}{hem en vroeg lijnrecht}\\

\haiku{Langs alle kanten.}{werden matten geklopt en}{ramen gewassen}\\

\haiku{Zijn ogen waren in.}{de haven van de nis tot}{stilstand gekomen}\\

\haiku{De betovering;}{omtrok hem met een wand van}{stille melodiek}\\

\haiku{{\textquoteleft}Ach, denkte, d\`e'k?}{nie meer kan spitten omdat}{ik zeventig ben}\\

\haiku{moar het wordt toch tijd,, ' '!}{vind ik datr n\'o\'u esn}{Gemeenschapshuis komt}\\

\haiku{Hij groette met de,:}{anderen doch \'e\'en zuster}{wendde zich tot hem}\\

\haiku{{\textquoteleft}Excuseert u, dat,{\textquoteright},.}{ik eens even de benen strek}{zei hij en stond op}\\

\haiku{{\textquoteleft}Het spijt me, dat Rieks,{\textquoteright}.}{van Brugge ziek is bracht Ron}{bezorgd naar voren}\\

\haiku{Maar toen had hij stil,.}{gestaan om te bedenken}{waar hij moest strooien}\\

\haiku{Meneer Pastoor,{\textquoteright} zei, {\textquoteleft} -,?...}{hij bevendwoar is oew oew}{barmhartigheid man}\\

\haiku{As die hier nou es -,?}{verstrooid wou zijn w\`e zoude}{dan zeggen pastoor}\\

\haiku{Pastoor had ze hoog,.}{geroemd en God geprezen}{voor zulk schoon schepwerk}\\

\haiku{Maar Huub was hem recht.}{blijven aankijken met zijn}{allerzwartste ogen}\\

\haiku{Slechts \'e\'en jongen die,.}{hij niet eerder had gezien}{was ongedurig}\\

\haiku{Buiten deinde het;}{bolle wezen van Spanjaarts}{voorbij het venster}\\

\haiku{Diejen knol kan nog -}{genen wind loaten of}{hij zegt Weesgegroet}\\

\haiku{Hier is oew koffie,,{\textquoteright}, {\textquoteleft} '?}{Spanjaarts zei Annet zorgzaam}{wilden kuuksken}\\

\haiku{{\textquoteleft}Ik geloof, dat God, '!}{h\'e\'el gelukkig was de dag}{datie jou maakte}\\

\haiku{Nu meende Ron hem.}{met tact zijn verwondering}{te mogen tonen}\\

\haiku{En die kleine van -,!}{Monders d\`e was nen mesken}{van zes nen nichtjen}\\

\haiku{Pietje Monders mocht.}{niet vaster in de zadel}{zitten dan hij}\\

\haiku{En Jef ontzag zich,.}{niet het Mariaspel weer naar}{voren te schuiven}\\

\haiku{Dat was nu eenmaal,.}{een soort kreet van hem waarmee}{hij altijd goed zat}\\

\haiku{Daar wil ik liggen,,}{zodat mijn oog net zo hoog}{is als de bloemen}\\

\haiku{{\textquoteright} en ze vertelde.}{de belevenis aan Huub}{en de kinderen}\\

\haiku{Ron had een gevoel,,.}{of hij stikte terwijl hij}{zijn hand uitstrekte}\\

\haiku{Groter was Rieks in -.}{zijn beheerstheid kleuriger}{Martien in zijn wil}\\

\haiku{De machine sleet,.}{bij de dag ongewend en}{verroest als ze was}\\

\haiku{{\textquoteright} Ron vestigde hun,.}{aandacht op de prachtige}{grote veldbloemen}\\

\haiku{En zo kwam Ron weer,:}{te dwalen die middag langs}{Akkerweg en Stroopad}\\

\haiku{t is gauw zo ver.{\textquoteright} {\textquoteleft}{\textquoteright}.}{Maar bijna niemand kent zijn}{rol barstte Ron uit}\\

\haiku{{\textquoteright} zei ze bleekjes, want.}{daaraan had ze heel nare}{herinneringen}\\

\haiku{Gadverju, kerel,,!}{ge bent toch nen vent en geen}{gedoopte rochel}\\

\haiku{En ze houden oe,!}{allemoal teugen ge}{kumt nie oan den kop}\\

\haiku{De mooie vrouw, die je -.}{ongetwijfeld bent als je}{magerder zou zijn}\\

\haiku{Ben de Weyn was daar,.}{lid van hij wilde graag te}{paard opkomen}\\

\haiku{Met koesterende,:}{opwinding met triomf sprak}{Spanjaarts de woorden}\\

\haiku{{\textquoteright} {\textquoteleft}Daar geeft Rennevoirt,{\textquoteright}.}{nog geen enkel bewijs van}{repliceerde Ron}\\

\haiku{Hij klopte haar op.}{enig klinkend lichaamsdeel en}{was Ron vergeten}\\

\haiku{Want op de laatste.}{brief naar Amerika was geen}{antwoord meer gevolgd}\\

\haiku{{\textquoteleft}Het is z\'o dringend,{\textquoteright}, {\textquoteleft}!}{zei meneer Pastoorhet wordt}{bekant wijwater}\\

\haiku{en in de verte,;}{zong hun de klank van een zeis}{toe die gescherpt werd}\\

\haiku{{\textquoteright} schreeuwde Huub, langs het.}{schaterend gelach van de}{beide anderen}\\

\haiku{wat wist je weinig!...}{op wie je kon bouwen van}{ogenblik tot ogenblik}\\

\haiku{Tafels en stoelen.}{neerzetten en een spiegel}{in de grimeerhoek}\\

\haiku{Alle ketenen,,.}{de ringen en hangers de}{kronen en gespen}\\

\haiku{Ze zag zijn handen.}{voorzichtig borststukken en}{kappen gladstrijken}\\

\haiku{En gij, Ron, hedde?!}{\`alle gedachten achter}{oe geloaten}\\

\haiku{De akkers lagen.}{grijs en verlaten achter}{het mooie bouwwerkje}\\

\haiku{Het comit\'e zocht.}{mekander en wist van uur}{op uur geen uitkomst}\\

\haiku{Om zeven uur was.}{het betraand droog boven een}{versopte aarde}\\

\haiku{Maar het bleef droog en.}{het pleintje voor de kapel}{hernam zijn aanzicht}\\

\haiku{Hoe warm en direct,!}{waren ze gebleken en}{hoe raadselachtig}\\

\haiku{Buiten legde hij,.}{zijn hand tegen de muur waar}{de Dood had geleund}\\

\haiku{Pas toen hij het in,.}{de hand hield zag hij dat het}{een telegram was}\\

\haiku{Natuurlijk was hij -.}{woedend er zou altijd wel}{een aanleiding zijn}\\

\subsection{Uit: Versprongen ster}

\haiku{Er was geen regen,.}{meer alleen een aandachtig}{auditorium}\\

\haiku{Ze sprong beslist niet;}{zo technisch en zo mooi als}{haar lotgenote}\\

\haiku{De baas kon dat ook,,:}{zo vragen en als je dan}{een klacht had zei hij}\\

\haiku{{\textquoteright} {\textquoteleft}Nee,{\textquoteright} zei Betsie, met, {\textquoteleft}!}{haar tanden tegen het glas}{dat begrijpt u niet}\\

\haiku{{\textquoteleft}Ik word binnenkort,.}{vijftig dus een baan zult u}{mij niet aanbieden}\\

\haiku{Het is een tikje,...}{vergezocht en daardoor niet}{onverwacht genoeg}\\

\haiku{{\textquoteleft}Kom nou,{\textquoteright} zei hij, en, {\textquoteleft},!...}{streelde haar hulpeloze}{handkom nou Betsie}\\

\haiku{Zijn wangen waren,.}{hoogrood hij keek Betsie aan}{met ontzette ogen}\\

\haiku{Hij bleek opeens veel,.}{minder onaantastbaar dan}{hij ooit was geweest}\\

\haiku{Er hoeft maar een reep,.}{chocola in te zitten}{of een zakdoekje}\\

\haiku{En vervolgens bleek:}{de bakker een grote doos}{bezorgd te hebben}\\

\haiku{Ze had nu juist die,!}{dag zo veel gedachten die}{niet aardig waren}\\

\haiku{Daarom kon je juist,.}{zo gemakkelijk gek doen}{en eens wat zeggen}\\

\haiku{Ze herinnerde,.}{zich hoe ze voor het eerst naar}{een baan was gestapt}\\

\haiku{Of wilde de vrouw?}{van de directeur kennis}{maken met Betsie}\\

\haiku{Nog drie tellen, en.}{Alie zou de bevreemding een}{kop thee aanbieden}\\

\haiku{Maar die kanarie - -...}{lieve Cor en Aal die moet}{toch nog even wachten}\\

\haiku{Een jongeman in.}{een groen satijnen broek stond}{midden in de zaal}\\

\haiku{Betsie struikelde.}{door de stoot in haar rug op}{het gezelschap toe}\\

\haiku{Ze beschouwen je,.}{als een excentrieke ster}{die mij intiem kent}\\

\haiku{Stephans moest spelen,;}{was een donkere vrouw van}{een jaar of vijftig}\\

\haiku{Ze stond daar, en wist,.}{niet eens wat ze eigenlijk}{precies gezegd had}\\

\haiku{{\textquoteleft}dan zal ik er een,.}{canap\'e van maken met}{de rug naar je toe}\\

\haiku{Betsie van de Pen.}{werd langzaam maar heel zeker}{Betty Vandepen}\\

\haiku{Ik heb 's morgens,...}{altijd een basstem dat is}{nog uit mijn diensttijd}\\

\haiku{Hemel, ja, een mens!...}{kan toch niet altijd rijk en}{deftig geweest zijn}\\

\haiku{naar haar kleedkamer,,.}{gooide de bontcape af}{trok de japon los}\\

\haiku{Veel sneller dan ze,.}{gedacht had  stond ze weer}{op het podium}\\

\haiku{Slechts langzaam drong het,.}{tot Betsie door dat zij de}{ouwe werkster was}\\

\haiku{Loretta was na.}{de premi\`ere niet meer}{terug gekomen}\\

\haiku{Je ziet er uit naar,{\textquoteright}.}{geldzorgen constateerde}{Betsie moederlijk}\\

\haiku{{\textquoteleft}Toen ik je voor het -{\textquoteright} {\textquoteleft}-,{\textquoteright}.}{eerst zag in de Kiekenstraat}{vulde Betsie aan}\\

\haiku{Hij deed een snelle,.}{stap hogerop en stond \'e\'en}{trede onder haar}\\

\haiku{Ze durfde zich te,.}{laten gelden en was veel}{zelfverzekerder}\\

\haiku{Alie voorzag verdriet;}{en ontnuchtering voor haar}{jongere zusje}\\

\haiku{{\textquoteright} Betsie stond stil en,.}{vroeg zich ijlings af wat hij}{gehoord kon hebben}\\

\haiku{Ze zat opeens in,.}{een luchtledig en begreep}{de mopjes maar half}\\

\haiku{{\textquoteleft}Dat hij nog met haar,!}{wilde praten na alles}{wat hij van haar wist}\\

\haiku{{\textquoteright} Een jonge kerel}{met een fluwelen jasje}{en hoog opgekamd}\\

\haiku{Overigens was het,.}{geen plezier de kamer te}{delen met Lyra}\\

\haiku{Toen keek ze in de.}{al te twinkelende ogen}{van haar directeur}\\

\haiku{Ze had eens zo'n soort,.}{mop gelezen maar wist die}{toch niet precies meer}\\

\haiku{{\textquoteleft}Voor de vrouw, die zo...{\textquoteright},.}{prachtig op de vleugel springt}{stond er op het kaartje}\\

\haiku{Ze boog stralend en,.}{glimlachend naar de zaal die}{klapte en stampte}\\

\haiku{{\textquoteright} zei het jongetje,.}{lichtelijk verbaasd dat zo'n}{mevrouw d\`at niet wist}\\

\haiku{Wat wist je weinig!}{van je medemensen en}{van hun gevoelens}\\

\haiku{{\textquoteright} Dat werd gehoord door,.}{een kwieke oudere heer}{die voor hen  liep}\\

\haiku{{\textquoteleft}Ik zou geen moment,.}{geloofd hebben dat je zo'n}{feestpet kon dragen}\\

\haiku{Ze keken op een,.}{vreemde verwachtingsvolle}{manier naar Betsie}\\

\haiku{{\textquoteleft}Mijn zuster had eens,,{\textquoteright}.}{een strijkijzer dat onder}{stroom stond zei ze toen}\\

\haiku{{\textquoteright} {\textquoteleft}O,{\textquoteright} stelde de man, {\textquoteleft}!}{haar geweldig ongerust}{het is z\'o gebeurd}\\

\haiku{Ze stond erop met,.}{een wijdopen mond en een been}{als een voorhamer}\\

\haiku{En nog was ze zo,.}{vermusicald dat ze er}{een ballet in zag}\\

\haiku{Ze liep naar boven,.}{en zette haar koffer met}{een bons op de grond}\\

\haiku{Het ergste was, dat,.}{ze niet wist wat ze doen moest}{toen het eten op was}\\

\haiku{Maar wie gaat er nu,?}{ook voor het raam zitten als}{ze zo bekend is}\\

\haiku{Op die manier moet,!...}{iedereen wel denken dat}{je in de film speelt}\\

\haiku{{\textquoteleft}Maar dat mag voor hen,.}{geen  aanleiding zijn je}{te belemmeren}\\

\haiku{{\textquoteleft}Ik breng u hier het,.}{meest spierwitte schaap thuis dat}{u zich denken kunt}\\

\haiku{Het was n\'a\'ar, dat ze!...}{zo met bedrog een geschenk}{lekker moest maken}\\

\haiku{{\textquoteleft}Ik heb laatst iemand,...}{ontmoet die zei dat hij haar}{kende van vroeger}\\

\haiku{Nu komt het,{\textquoteright} dacht ze,.}{wraakbelust en volgde hem}{zonder enig gerucht}\\

\haiku{In de gang maakte,:}{hij hoorbare voetstappen}{en praatte met haar}\\

\haiku{Een misselijke,.}{vraag die Betsie het bloed naar}{de hersenen dreef}\\

\haiku{de mensen in de.}{zaal kenden en ontzagen}{mekander te zeer}\\

\haiku{{\textquoteleft}Ik heb dat arme,{\textquoteright}.}{kind een glaasje tonicum}{gegeven zei hij}\\

\haiku{Vreemd, vond Betsie, dat,...}{hij zo'n geveinsde indruk}{maakte nu en dan}\\

\haiku{Ze hoorde toch wel,}{bij deze troep ze hoorde}{er inderdaad bij}\\

\haiku{Nog daargelaten -:.}{of ze het k\'on ze werd er}{niet voor geroepen}\\

\haiku{Wat hadden zij toch,!}{een malle simpele kijk}{op deze dingen}\\

\haiku{{\textquoteright} informeerde Alie,.}{later gebiologeerd}{door de gedachte}\\

\haiku{{\textquoteright} En Sandor, over de,.}{toetsen gebogen begon}{weer te spelen}\\

\haiku{Je springt in elk stuk,,{\textquoteright}, {\textquoteleft}!}{dat je speelt vertelde Thea}{al word je honderd}\\

\haiku{Het publiek is er,,!}{dol op en \'e\'en sprong is toch}{ook niets zeg nou zelf}\\

\haiku{En toen zei Betsie,:}{eindelijk zacht wat ze al}{lang op het hart had}\\

\haiku{De mensen in de,.}{zaal hadden even gelachen}{terwijl ze opkwam}\\

\haiku{Doch Robbens zag met,.}{grote scherpte dat Lyra}{de dialoog speelde}\\

\haiku{Hij greep haar bij de,.}{arm en duwde haar in haar}{eigen kleedkamer}\\

\haiku{{\textquoteleft}Dat is het bij mij...{\textquoteright}.}{vanavond niet De andere}{drie zeiden geen woord}\\

\haiku{{\textquoteleft}Ik zou nooit willen,!}{spelen als het moest op kracht}{van de alcohol}\\

\haiku{Zou ze nu nog een?...}{verlopen actrice op}{sterk water worden}\\

\haiku{Ze zei het niet luid,,.}{ze gilde het niet zoals}{was afgesproken}\\

\haiku{{\textquoteright} zei Robbens zorgzaam, {\textquoteleft},.}{ik wacht buiten en breng je}{met de wagen thuis}\\

\haiku{Ze moest sportief  ,.}{kunnen zijn en op tijd aan}{de kant kunnen gaan}\\

\haiku{De taxi reed haar in.}{zwierige bochten naar het}{kantoor van Robbens}\\

\subsection{Uit: Winterverhalen}

\haiku{{\textquoteleft}Ik eh - ben niet de,{\textquoteright}.}{jongen die duizendjes vindt}{zei hij eenvoudig}\\

\haiku{je sluit het raam dat,.}{je hebt opengemaakt en je}{gaat weg door de deur}\\

\haiku{Een straatlantaarn en:}{de beroemde maan uit het}{liedje hielpen hem}\\

\haiku{Toen schoof hij alle.}{brandbare zaken uit de}{buurt van de brandkast}\\

\haiku{{\textquoteright} {\textquoteleft}Ach ja,{\textquoteright} antwoordde, {\textquoteleft}!}{Jan beschaafdmaar ik wilde}{niet herkenbaar zijn}\\

\haiku{Hij rimpelde er.}{zijn wenkbrauwtjes van en zag}{er zorgelijk uit}\\

\haiku{Morgen mag ik met,{\textquoteright}.}{vader en moeder naar de}{kerk vertelde Cor}\\

\haiku{Terwijl moeder zei,...!}{dat hij al lang naar Spanje}{was teruggegaan}\\

\haiku{{\textquoteleft}Soms,{\textquoteright} zei Cor, die het.}{mysterie toch ook nog niet}{geheel doorgrondde}\\

\haiku{Het eten dat nog op,.}{tafel kwam was niet vet meer}{en krap gemeten}\\

\haiku{In de maatschappij,,.}{die een groot uurwerk was mocht}{je niets zelf willen}\\

\haiku{Was ze maar machtig,...:}{of tenminste verstandig}{dat ze zeggen kon}\\

\haiku{Iedereen in de,.}{buurt wist dat Monseigneur het}{nooit zelf mocht weten}\\

\haiku{De mensen om hem,.}{heen waren ook geroerd zij}{bogen en knielden}\\

\haiku{Vader zat met de,.}{benen op een andere}{stoel en wou niet uit}\\

\haiku{Hij zweeg, en trachtte.}{het verhaal voor het opstel}{te achterhalen}\\

\haiku{Hij vroeg nog zachter,:}{met veel te veel geduld voor}{zo'n klein mannetje}\\

\haiku{En hij wist al, wat,:.}{hij zou vragen en vooral}{wat hij zou z\`eggen}\\

\haiku{De geestelijken:}{hoorden haar in gedachten}{allemaal zeggen}\\

\haiku{Pietje Schiplijn bleef.}{al die tijd gevangen in}{het visioen staan}\\

\haiku{In de eerste plaats,,.}{hoop ik dat gij gezond moogt}{zijn en gelukkig}\\

\haiku{Wij waren in de.}{greep van het grootst ontzag dat}{ik ooit heb gevoeld}\\

\haiku{Hoewel ze net zo,.}{scherp keek als nu verzachtten}{haar trekken zich even}\\

\haiku{Die laatste dagen,.}{waren zo schoon als elk uur}{hier in de hemel}\\

\haiku{De kleine jonkvrouw,.}{sjorde haar kleed op en bond}{het met dat goudkoord}\\

\haiku{{\textquoteright} De woorden riepen.}{een vaag medelijden in}{de jongen wakker}\\

\haiku{Hij was vergezeld,.}{van zes knechten waarvan twee}{koffers meevoerden}\\

\haiku{Daarna ging het in,,.}{fiere gestrekte draf de}{laatste paar mijlen}\\

\haiku{Hij verraadde haar,.}{in halve woorden wat de}{gezant gemeld had}\\

\haiku{{\textquoteright} Dit allerliefste...?}{kinderfiguurtje met de}{Maria-ogen}\\

\haiku{Indien hij gedurfd,.}{had zou hij haar kinnetje}{hebben opgelicht}\\

\haiku{Gij zijt van verre,{\textquoteright}.}{tegen die pilaar gebotst}{zei de jongeman}\\

\haiku{al zou een kamer.}{voor hem en mij tezamen}{mij eenzaam lijken}\\

\haiku{Sweder van Urssen,.}{dronk en begon luider en}{luider te spreken}\\

\haiku{{\textquoteleft}Het viel me woord op,...!}{woord mee dat hij mijn dochter}{vermocht te volgen}\\

\haiku{En reeds was daar aan:}{zijn deur die vervelende}{oude biechtvader}\\

\haiku{Ze leken beiden,.}{in gedachten verdiept zo}{zwijgzaam waren zij}\\

\haiku{De jonkvrouwe kwam.}{als een levend geworden}{spooksel de zaal in}\\

\haiku{Dan vraag ik mij af,,}{waarom zo veel soldaten}{u vergezellen}\\

\haiku{Iets in zijn woorden.}{bracht het reisdoel terug in}{Rychilda's gedachten}\\

\haiku{Ik heb u beproefd,,.}{op kracht op beheersing op}{dapperheid en zelfdwang}\\

\haiku{{\textquoteright} polste Rychilda,.}{alsof ze zulks anders n{\'\i}\'et}{zou hebben gedaan}\\

\haiku{De dag kwam, waarop:}{de dochters van de hertog}{van Gordon zeiden}\\

\haiku{Wat weten wij van?}{liefdes kracht en zwakte in}{het hart van een vrouw}\\

\haiku{en daartussen lag.}{de droom van vlakke zee en}{wazige kusten}\\

\haiku{Ja, hij reed hoog te,.}{paard als in de dagen toen}{hij kapitein was}\\

\haiku{want de edelman aan.}{het hoofd van de troep had een}{bezwegen haast}\\

\haiku{{\textquoteleft}Mijn voedsel is uw,.}{voedsel mijn bezittingen}{zijn geheel de uwe}\\

\haiku{En de stille zaal.}{aan haar voeten was als een}{schaal voor haar tranen}\\

\haiku{de vrouwen bij het,.}{braadspit de oude heerser}{in zijn hoge stoel}\\

\haiku{Het was de vreemdste,.}{Kerstmis welke iemand zich}{had kunnen denken}\\

\haiku{maar waarom heb je?}{die fonkelende witte}{eruit gelaten}\\

\haiku{Hij keek telkens naar,.}{zijn vrouw die met stroeve ernst}{haar kousen heelde}\\

\haiku{hij leek tussen hen,.}{te staan en evenzeer zijn adem}{in te houden}\\

\haiku{Aan de andere.}{kant van de stille straat stond}{haar vriendin Lida}\\

\haiku{{\textquoteright} Met een nuance:}{van ontevredenheid in}{haar stem zei Lida}\\

\haiku{en daarnaast te arm.}{om in beschaafde kringen}{zijn plaats te vinden}\\

\haiku{{\textquoteleft}Wat ben je toch een,!}{voorbeeld van ijver als er}{iemand naar je kijkt}\\

\haiku{Sander was toen al.}{aan zijn vijfde tekening}{voor de kleine zaal}\\

\haiku{Bij nacht werk ik het,,...{\textquoteright}}{beste dan is het stil dan}{zijn de dromen los}\\

\haiku{Hij keek uit het raam.}{en strekte werktuigelijk}{zijn pijnlijke rug}\\

\haiku{Om dat gedicht te,...?}{mogen zeggen  tussen}{al die pretmakers}\\

\haiku{Dat had hij immers...!}{van het eerste ogenblik af}{kunnen verwachten}\\

\haiku{Nee, stoor hem nu niet!}{met je flauwe gepraat over}{koffie en pastei}\\

\haiku{Hij kwam nader, en.}{stond een beetje verlegen}{stil bij de tafels}\\

\haiku{Zij hadden op een.}{verjaardag zitten praten}{over Kerst-sagen}\\

\haiku{In een donkere,,.}{stinkende koeiestal bij}{het vak van de stier}\\

\haiku{{\textquoteleft}Nee,{\textquoteright} zei de stier, met.}{weer die rare grimas van}{lippen en wangen}\\

\haiku{Zelfs de man in de.}{opname-wagen sprak over}{goedkope mopjes}\\

\haiku{Alles wat restte.}{was een geluidsband met een}{brommerige stem}\\

\haiku{Hun zoontje Piet, een,.}{jongen van twaalf jaar was al}{een tijdje niet goed}\\

\haiku{alsof een Kerstboom,!}{het uithield in regen en}{storm en desnoods sneeuw}\\

\haiku{Op het Dweelsepad {\textquoteleft}}{schudde meneer pastoor de}{notaris de hand.}\\

\haiku{{\textquoteleft}Weet je wat, Kors,{\textquoteright} zei, {\textquoteleft},.}{hij danik kom eens terug}{als je alleen bent}\\

\haiku{Aan zijn praat was te,.}{merken dat hij weinig werd}{tegengesproken}\\

\haiku{Verlejen jaar heeft...{\textquoteright}}{hij Kerstliedjes gezongen}{voor zieke mensen}\\

\haiku{{\textquoteright} Zijn woorden leken.}{vleugels te krijgen tegen}{de koele hoogte}\\

\haiku{Dirk heeft gelijk,{\textquoteright} zei, {\textquoteleft},}{Theuns aarzelendals ik nou}{die boom geef komen}\\

\haiku{Daarna wendde hij,.}{zich tot Dirk en legde een}{hand op zijn schouder}\\

\haiku{Nee, hij kwam niet voor,;}{Kors die al met een brede}{grijns de stoel bijschoof}\\

\haiku{- Zo kwam Jaantje bij.}{Pietje van Rien en Mien in}{de kamer te staan}\\

\haiku{We hebben boven,...{\textquoteright} {\textquoteleft},}{nog zo'n dunne koperen}{bak van ome KreelJa}\\

\haiku{Iets in het duister,.}{gaf hem inspiratie haar}{hand vast te pakken}\\

\haiku{Vrouw Martje Theuns zei, dat.}{ze naar de stad wilde gaan}{om engelenhaar}\\

\haiku{Op het gebroken.}{hekje van Miebartjes tuin zat}{een van Barnds duiven}\\

\haiku{Daarachter lag het,.}{wijde veld met een stukje}{van het Dweelsepad}\\

\haiku{{\textquoteleft}Dat heeft notaris,,{\textquoteright};}{al gedaan toen we jouw brood}{gingen eten zei ze}\\

\haiku{Twee dagen later.}{werd er in de avond geklopt}{bij de notaris}\\

\haiku{De avond viel, en spon.}{de boom met al zijn sier in}{een rust van schaduw}\\

\haiku{het was net zoals:}{mevrouw Marie tegen de}{dokter had gezegd}\\

\haiku{{\textquoteleft}O,{\textquoteright} zei hij ademloos, {\textquoteleft}!}{tegen zijn oudersalles}{is vanavond \'even mooi}\\

\haiku{Hij celebreert in,.}{grafse stilte de Mis en}{zingt zijn litanie}\\

\haiku{Men kan, wanneer men,;}{verloren is beter voor}{de dag uit varen}\\

\haiku{Op de middenbank.}{zaten twee oude mannen}{in rode mantels}\\

\haiku{Terwijl hij toch zo'n.}{welbesteed leven achter}{zich had liggen}\\

\haiku{{\textquoteleft}Ik ben de Goede,{\textquoteright},.}{Bedoeling zei hij als om}{zich voor te stellen}\\

\haiku{Mij is niemand te,.}{machtig en ik zal u de}{reden verklaren}\\

\haiku{{\textquoteleft}Doorzoek mijn huis dan,{\textquoteright}, {\textquoteleft}.}{raadde Sylvester hemwant}{ik hecht daaraan niet}\\

\haiku{{\textquoteleft}Waarom de mens faalt,{\textquoteright}, {\textquoteleft}.}{antwoordde zedat is al}{zo dikwijls gezegd}\\

\haiku{V\'o\'or hem lag dat strand,.}{en hij had zo juist zijn boot}{aan wal getrokken}\\

\haiku{En hij dacht in een:}{plotseling verlangen naar}{stilte en puurheid}\\

\section{Ferdinand Langen}

\subsection{Uit: In pyama}

\haiku{Toen liet hij zijn hand.}{wippen voor zijn open mond en}{begon te kauwen}\\

\haiku{Een flauwe smaak kwam,}{in mijn mond en ik voelde}{dat ik hoofdpijn kreeg}\\

\haiku{In de tijd tussen.}{twee schokken luisterde hij}{naar den dominee}\\

\haiku{Eigenlijk schaamde,.}{ik mij voor hem maar ik wist}{niet precies waarom}\\

\haiku{Die is flauw{\textquoteright}, zei de, {\textquoteleft}}{Weesneem maar mee naar huis en}{leg zout op z'n pens.}\\

\haiku{Ik lachte hem uit.}{en zei dat hij nauwlijks tot}{de helft zou komen}\\

\haiku{Aan de gevolgen,.}{durfde ik niet denken die}{moesten vreselijk zijn}\\

\haiku{Er gebeurde niets,.}{maar ik durfde niet weer over}{de rand te kijken}\\

\haiku{Het teer was taai en.}{ik rolde het in mijn hand}{tot een balletje}\\

\haiku{Ik herinnerde.}{mij alleen nog duidelijk}{een voldaan plezier}\\

\haiku{en nou jou, om die,!}{kat en zijn huishoudster is}{nog wel zijn zuster}\\

\haiku{Maar ik voelde het,.}{wel kreeg blauwe plekken en}{huilde stil van pijn}\\

\haiku{Ik probeerde de,.}{klauw te grijpen maar ik greep}{Fijt en hij was nat}\\

\haiku{De uitgestrekte.}{houten arm bleef mij tergend}{in de rug steken}\\

\haiku{Bij de kerk wilde,.}{ik het hek in draaien maar}{Fijt hield me tegen}\\

\haiku{Des avonds zat ik met.}{de vreemde jongens in een}{weiland aan de gracht}\\

\haiku{Ik draafde door het.}{weiland tot het water me}{in de schoenen stond}\\

\haiku{Waarschijnlijk hoopte.}{hij op die manier iets van}{Hein los te krijgen}\\

\haiku{{\textquoteleft}Ja, ja{\textquoteright}, viel Evert mij,.}{dadelijk bij terwijl hij}{schuchter naar Hein keek}\\

\haiku{In de winter die,.}{volgde zocht ik opnieuw het}{gezelschap van Evert}\\

\haiku{Zij was veel zwaarder:}{dan ik en het resultaat}{bleef altijd gelijk}\\

\haiku{Van hun gesprekken,}{begreep ik de helft niet toch}{wonden ze me op}\\

\haiku{Nu begrijp ik, dat.}{dat niet helemaal buiten}{zijn eigen wil was}\\

\haiku{We liepen zwijgend.}{in de karresporen van}{het zandweggetje}\\

\haiku{Waarom, dacht ik, zijn.}{er zoveel mannen door hen}{verleid geworden}\\

\haiku{Haar benen en haar,,.}{armen haar neus en haar oren}{maar vooral haar mond}\\

\haiku{{\textquoteleft}Wat is er met jou,?}{Memmeling heb je zo slecht}{geslapen vannacht}\\

\haiku{Ik was naakt, alleen.}{over mijn ene schouder vielen}{haar blonde haren}\\

\haiku{Toen het begon te,.}{regenen leerde ik in}{kroegen te schuilen}\\

\haiku{{\textquoteleft}Ze zal toch nog  ,{\textquoteright}.}{wel een staartje hebben dat}{zeemeerminnetje}\\

\haiku{Ze nam het  in.}{haar handen en streelde er}{over met haar vingers}\\

\haiku{Je vader het wel,{\textquoteright}.}{geld maar die verdomt het \'o\'ok}{om af te schuiven}\\

\haiku{nu vermoord ik hem,.}{hoorde zij het bekende}{fluitje weer op straat}\\

\haiku{Op de trappen van.}{het stadhuis kreeg ze ruzie}{met haar schoonmoeder}\\

\haiku{Hij kon dat moppig.}{overdrijven van Cornelis}{niet erg waarderen}\\

\haiku{Wat ons bond was geen,.}{vriendschap meer het was van mijn}{kant een obsessie}\\

\haiku{Hij weet weer waar hij,,....}{is en hij kan nog roken}{Goddank roken}\\

\haiku{De bloemen waren,.}{nu zo dicht bij dat ik ze}{meende te ruiken}\\

\haiku{Ik glimlach daar nu,.}{om maar ik ben eigenlijk}{nog niets veranderd}\\

\haiku{Ik kan dat rustig,.}{zeggen nu ik alleen ben}{in deze kamer}\\

\haiku{Bovendien ziet hij.}{alles in de verte als}{door beslagen ogen}\\

\haiku{Hij tekent met zijn.}{pen kleine figuurtjes op}{zijn bolle nagels}\\

\haiku{Zullen de bellen?}{tussen zijn vingers hem nog}{kunnen verleiden}\\

\haiku{Maar Vita haalt haar.}{schouders op en opnieuw raakt}{haar lichaam zijn arm}\\

\haiku{Hij denkt de dagen.}{door en des nachts is hij te}{moe om te slapen}\\

\haiku{Het personeel roept.}{hij bijeen en hij speecht als}{een feestredenaar}\\

\haiku{Ze moeten het goed,,:}{begrijpen het is niet erg}{belangrijk alleen}\\

\haiku{De ramen van de.}{huizen knipogen hen toe en}{de bomen wuiven}\\

\haiku{Mijn tijd komt, want nog.}{altijd geloof ik dat het}{goede beloond wordt}\\

\section{Jef Last}

\subsection{Uit: Partij remise}

\haiku{Hun lachen smoort als.}{het piraatje tusschen de nat}{geworden vingers}\\

\haiku{De \'e\'ene  hoop.}{van iederen dag kleurt hun}{wangen en voorhoofd}\\

\haiku{Langs de dijken ligt.}{de levende guirlande}{van jonge menschen}\\

\haiku{Nog later in den.}{nacht komen de pinose}{jongens de straat op}\\

\haiku{Met fellen slag jaagt,.}{de spoel heen en weer heen en}{weer door den inslag}\\

\haiku{Zoo staan ze, wanneer.}{ze een standje krijgen of}{ontslagen worden}\\

\haiku{Dan treedt Sternheim op:}{om een acte uit Goethe's}{Faust voor te dragen}\\

\haiku{Is het wonder dat?}{Gerrit tevreden is over}{den gang van zaken}\\

\haiku{Wie klaar is, mag de{\textquoteright}.}{sommen van gisteren in}{het net gaan schrijven}\\

\haiku{, wij zullen er ons{\textquoteright} (,).}{bij neer gelegen hebben}{fout zegt de meester}\\

\haiku{Argumenten en.}{scheldwoorden stuiten af op}{een koppig zwijgen}\\

\haiku{{\textquoteleft}Ach{\textquoteright}, zegt hij, {\textquoteleft}een haal{\textquoteright}.}{uit de mok kreeg ik thuis ook}{wel eens van mijn oue}\\

\haiku{Een enkele ligt.}{met zijn laarzen aan reeds op}{de stroozak te slapen}\\

\haiku{Een korte kreet wordt.}{zoo snel gesmoord alsof ze}{niet had geklonken}\\

\haiku{Ja, domin\'e, we{\textquoteright}.}{spraken net over de krijgstucht}{bij de marine}\\

\haiku{{\textquoteleft}U bent nog te kort,,.}{op de vloot domin\'e U}{hebt nog idealen}\\

\haiku{Brieven zijn aan boord.}{van een opleidingsschip geen}{priv\'e bezitting}\\

\haiku{{\textquoteright} roept ze, {\textquoteleft}ja, thuis is, '{\textquoteright}.}{ie twee trappen op maar en}{overt portaaltje}\\

\haiku{Bij de Neptuun bar,:}{brandt een gloeiende lichtbaak}{in blauw rood en geel}\\

\haiku{jaagt het bloed door zijn,}{lichaam naar zijn kop toe in}{twee groote teugen zuipt}\\

\haiku{{\textquoteleft}Verdomme{\textquoteright}, fluistert, {\textquoteleft},{\textquoteright}.}{hijals ik maar slof als ze}{maar niet zuur is}\\

\haiku{Een anderen keer.}{heeft hij haar meegenomen}{naar het Gebouw toe}\\

\haiku{- Dokter Noppes, - De, -, -.}{oue va\`ar Kapitein Slikjas}{Luitenant Roosje}\\

\haiku{Voorzichtig, zachtjes,.}{om niet te storen brengt zijn}{vrouw de thee binnen}\\

\haiku{Een volle heen, een -,,.}{leege terug f\"ordern f\"ordern}{jakkert de steiger}\\

\haiku{{\textquoteleft}Wat ik zeggen wou{\textquoteright},, {\textquoteleft}}{vervolgt van Garderenvoor}{mij is de wereld}\\

\haiku{Maar de mensch is juist.}{vermoeid geworden van het}{eeuwen rechtop staan}\\

\haiku{Grijze lucht, grauwe.}{golven en de regen die}{in zijn gezicht striemt}\\

\haiku{Ineens herwint de.}{schipper zijn bezinning en}{breekt de verstarring}\\

\haiku{Op dat oogenblik.}{is de mijn geen drie meter}{meer van het schip af}\\

\haiku{Willem hier, liep nog.}{geen drie maanden geleden}{op z'n tandvleesch}\\

\haiku{Nou heeft ie een paar,.}{nieuwe moli\`eres een}{das en een dasspeld}\\

\haiku{Maar  dan met de,.}{mannetjes in de hand met}{de vuist op tafel}\\

\haiku{Troelstra herinnert zich:}{een tafelgesprek tegen}{David en Heine}\\

\haiku{De Twerskaja is,.}{zwart van demonstranten de}{Soljanka vloeit over}\\

\haiku{{\textquoteright} Op het hout van zijn.}{brits drukt fuselier Hannes}{nijdig een luis dood}\\

\haiku{{\textquoteleft}Een arrebeier{\textquoteright},, {\textquoteleft},.}{zegt Hannesis nou een keer}{geen mensch dat weet je}\\

\haiku{Hoe meer herrie d'r{\textquoteright},, {\textquoteleft}}{komt merkt de mijnheer bedaard}{ophoe moeilijker}\\

\haiku{Niet afwachten en.}{protesteeren en er het}{beste van maken}\\

\haiku{Slingerend in den.}{wind strooit een bleeke booglamp z'n}{licht op de spijlen}\\

\haiku{{\textquoteright} Jaantje klampt zich aan,,.}{Gerard die z'n arm in een}{verband draagt  vast}\\

\haiku{Alsof 't van een!}{legertje en een vloot als}{de onze afhing}\\

\haiku{Maar in '16 is 't{\textquoteright}.}{ernst geworden en verdomd}{bloedige ernst ook}\\

\haiku{In Tjimahi een '.}{reuzen vergaderings}{nachts op de renbaan}\\

\haiku{Die dienstweigering.}{op de Regentes is te}{vroeg gekomen}\\

\haiku{honderd meiden die.}{ik beter gekend heb dan}{m'n eigen zuster}\\

\haiku{Als ze met Gerard,.}{uitging moest hij een boordje}{om en een dop op}\\

\haiku{Je kunt probeeren weg,.}{te loopen dan krijg je de}{kogel in je rug}\\

\haiku{Zie je hoe ze olie?}{krijgen voor hun motor en}{geld voor hun manschap}\\

\haiku{Natuurlijk zijn er.}{andere  factoren}{ook bij gekomen}\\

\haiku{Met de winsten uit.}{Indi\"e ging de illusie}{samen verloren}\\

\subsection{Uit: Van een jongen die een man werd}

\haiku{Het merkwaardige,.}{was dat ik mij in beide}{kringen thuis voelde}\\

\haiku{nergens zoo goed als;}{stil in mijn kamer met}{mijn boeken om me}\\

\haiku{een pak slaag, als het,.}{eenige wat hij er aan doen}{kon voor remedie}\\

\haiku{Hij, sinds twee jaar dood,?}{voor zichzelf leefde nog in}{hun herinnering}\\

\haiku{Een breede bundel,}{glimmende rails straalde door}{een dal het land in}\\

\haiku{Weer trokken ze met.}{breede trossels jongens des}{Zondags den weg langs}\\

\haiku{{\textquoteleft}kom laten we gaan,{\textquoteright}.}{zwemmen en dan naar huis de}{boer wacht met melken}\\

\haiku{Aan de gootsteen hielp.}{een meisje haar moeder de}{vaten te spoelen}\\

\haiku{Frenske, die aan zoo'n,.}{woordenvloed niet gewoon was}{werd er beduusd van}\\

\haiku{Wat wou nou zoo'n vent!}{hem vermanen die zelf niets}{eens naar de kerk ging}\\

\haiku{Hij zag er uit of.}{hij ons met zijn sabel in}{stukjes wou hakken}\\

\haiku{Er was tusschen hun.}{twee\"en nog nooit een woord over}{liefde gesproken}\\

\haiku{De bitterheid drong.}{zijn hart binnen en keerde}{zich tegen hemzelf}\\

\haiku{Eerst verloren ze.}{hun werk en dan zette men}{hen uit hun huisje}\\

\haiku{Eigenlijk was de.}{geestelijkheid door dezen}{aanval overvallen}\\

\haiku{Frenske liet zich door,.}{een stroom meestuwen een}{andere straat in}\\

\haiku{Het was hardbruin met.}{aan den kanten leelijke}{houten barakken}\\

\haiku{Boven hen in de.}{hooge boomen zat ergens een}{vogel te zingen}\\

\haiku{Dan kom ik uit de}{donkere mijn En hoor bij}{vagen schemerschijn}\\

\haiku{Het was of van de.}{beantwoording dezer vraag}{zijn leven afhing}\\

\haiku{de meesten het in.}{d'r hart als je met kunst je}{brood moet verdienen}\\

\haiku{Voor de kroegen 's}{avonds en voor de meiden had}{hij geen centen meer}\\

\haiku{, smokkelen dorst hij,.}{ook niet meer en het werk bleef}{altijd even rottig}\\

\haiku{Aan het fornuis was.}{moeder nog bezig met de}{pannen en schotels}\\

\haiku{{\textquoteleft}Heilige Maria,,.}{vol van genade gij weet}{dat ik haar liefheb}\\

\haiku{Je kon dat met den.}{besten wil van de wereld}{niet geluw noemen}\\

\haiku{Was hij dan niet in,?}{de mijn geweest toen dat groote}{ongeluk plaats vond}\\

\haiku{Frenske was te veel.}{arbeider om dit ooit te}{kunnen begrijpen}\\

\haiku{Aan den anderen.}{kant van het plein hoorden ze}{hem nog even schelden}\\

\haiku{mocht je rondkruipen,.}{over het land dat je met hun}{strond bemest had}\\

\haiku{Het kon geen toeval,.}{zijn dat ze hier al dien tijd}{op hem gewacht had}\\

\haiku{Om een inbeelding,,?}{een gril het terugschrikken}{voor een nieuwigheid}\\

\section{Gheraert Leeu}

\subsection{Uit: Dialogus Creaturarum dat is Twispraec der creaturen}

\haiku{Dit geldt ook voor de.}{mens als hij zich bij tijd en}{wijle niet ontspant}\\

\haiku{De dialoog wordt in:}{de inhoudsopgave als}{volgt aangekondigd}\\

\haiku{op de initiaal.}{geeft de regelhoogte van}{de initiaal aan}\\

\haiku{als die heylighe}{leerraer sinte thomas van}{aquinen | bescrijft}\\

\haiku{|  [30] dat xxxi}{van mandragora ende}{venus dat ons leert}\\

\haiku{dat wij an | een.}{anders qualick varen ons}{spieghelen sellen}\\

\haiku{ons maeteliken}{ende saetliken te re}{| gieren |}\\

\haiku{vier Van eenen ionghen [].}{bock die alte groten goec}{|10 kelaer was}\\

\haiku{Die maen antwoorden.}{Ghanck van mi want ick di}{niet | lief en heb}\\

\haiku{dye haer seluen:}{seer began te verheffen}{| ende seyde}\\

\haiku{ende hoy eeten [}{als een os ende leuen}{tijden sel- |}\\

\haiku{tot dat die seuen}{iaren | volbracht waren}{In welken seuen}\\

\haiku{Ist dattu yegens mi}{vechten wils so en bistuniet}{|  wel bedacht}\\

\haiku{Ghelijck als een [].}{|25 drachticheet cleyne}{dinghen tot groot brenghet}\\

\haiku{Du selste weder}{op een ander tijt | mit}{hem werden verblijt}\\

\haiku{mer si dede an}{hoor wa | penen ende}{began te strijden}\\

\haiku{Quade harders en.}{hebben gheen sorghe van}{horen sca | pen}\\

\haiku{Die zee antwoorden.}{Siluer ende goudt}{en is niet met my}\\

\haiku{mit boesheyt ende.}{houerdicheyt te | ghen}{horen ouersten}\\

\haiku{int vader leuen}{dat een broeder | vraechde}{enen ouden vader}\\

\haiku{Dat neghentiende[]}{dyalogus |          15}{TOt 134 dat gout quam}\\

\haiku{dat dye vrouwe van}{eenre duyf die van bouen}{| quam ende nam}\\

\haiku{gheliken noch mit {\textparagraph}}{hem veel hanteren Want die}{wi | se man seyt}\\

\haiku{veel luden spraken [].}{wort gheuraghet waer |25 om}{hy dat dede}\\

\haiku{beloghen ende .}{bedraghen doe seyde hy}{totten keyser |}\\

\haiku{om dijnre lieften []}{ont- |10 fanghen heb}{die mogen spreeken}\\

\haiku{Wi bidden di dattu}{dy van die werlt wechdoeste}{ende dattu dy |}\\

\haiku{LAet 177 ons soecken.}{den besochten meester van}{| medicinen}\\

\haiku{lijfs ontgaen mochten.}{dat si dan hoor son- |}{den souden biechten}\\

\haiku{Ende doe si te}{lande ghecomen waren}{| spraken si hoor}\\

\haiku{also | verdreef.}{mandragora dat onreyn}{wijf ende seyde}\\

\haiku{goeden wil dye hi}{had om te clooster te gaen}{| ende segghen}\\

\haiku{bistu mi seer waert}{| Daer om wil ick mi mit}{di verenighen}\\

\haiku{want alsoe sal hi.}{iv oeck doen ist dat ghy |}{v niet en wachtet}\\

\haiku{Augustinus dat}{god den danckbarighe ghe}{| gheuen hadde}\\

\haiku{machtighen mensche}{op dattu | niet en valles}{in sinen handen}\\

\haiku{Mer dat hondekijn}{spoelde myt sinen heer dat}{hy oeck | voeden}\\

\haiku{bekenden dattie}{voghel astur steruen most}{en woude hi |}\\

\haiku{Hi antwoorden Ia}{ick hebber noch wel drieDie}{vader | seyde}\\

\haiku{dat vleys crighen dat.}{hi heeft Hier om seyde hi}{totten ra | uen}\\

\haiku{Daer om salstu my}{gheloeuen ende |}{betrouwen Coom daer}\\

\haiku{ende leyden dat.}{pack of | ende die ezel}{ontfinck sijn loon}\\

\haiku{alle die werlt o}{| uer Daer om seyt sinte}{Ian in sijn epistel}\\

\haiku{seyt is een voghel}{vanden | gheslachte des}{ghiers wit van verwen}\\

\haiku{Ende om dat si.}{gheen goet beghin en |}{hadde inder ioecht}\\

\haiku{an die merct om te.}{verkopen | seyde tot}{sinen ghesellen}\\

\haiku{wanneer die wijse}{gheslagen is wat sal hi}{doen | Hi antwoort}\\

\haiku{alle die stocken.}{om ende veriaghede}{alsoe die bijen}\\

\haiku{dat sijns va- | [].}{5 ders leeringhe goet ende}{profitelick was}\\

\haiku{Hi leit in sijn nest.}{een | smaragd teghen die}{vernijnde dieren}\\

\haiku{recht of si op een}{muer ghe | staen hadden}{Dese beesten sijn}\\

\haiku{Dyalogus xc}{|          SAtirus 352 is}{een mensch die hoornen}\\

\haiku{hem beyden om te {\textparagraph}}{vliegen | ende van daen}{te come Ende}\\

\haiku{Het is beter den}{| bosen prior of te}{setten ende enen}\\

\haiku{Ic ben selue wijs}{ende verstandel ghenoech}{Ick en wil van v}\\

\haiku{366 |          EEnhoorn}{367 is een dyer dat een}{hoorn in sijn voerhoeft}\\

\haiku{perikel was in}{die zee ende hy sat op}{enen groten stoel |}\\

\haiku{oec die qualen der}{zielen gepurgeert mitten}{siecten ende |}\\

\haiku{hem den hals ende.}{seyde Qualiken 384 |}{heeft hi wreaek ghedaen}\\

\haiku{quam ende nam dat [}{sackelkijn mytten goude}{ende liep daer |}\\

\haiku{oec som | wijlen.}{gheschieden als dat beest een}{goet stuck van hem was}\\

\haiku{ende dan dicwijl}{gheen van allen krijghen en}{mochte | alsoe}\\

\haiku{Paulus Inden staet [].}{daer ghi ge |25 ropen}{sijt daer bliuet in}\\

\haiku{wi sijn ghesont ende.}{en besighen nym- |}{mermeer medicijn}\\

\haiku{hoe langhe sal men}{| silencium houden Die}{oude antwoorden}\\

\haiku{Maeckt hier of meester}{een leringhe | te rijm}{dat vwe konst daer wt}\\

\haiku{dat een velthen was}{die veel iongher kuyken had}{die si seer nau |}\\

\haiku{rusten in hoer hol []}{ende als si drie dagen}{geslapen |10}\\

\haiku{ver- | droech hi}{oeck al goetelick Van den}{keyser augustus}\\

\haiku{een ander geweest []}{haddeGaet haes- |20}{teliken van hier}\\

\haiku{Het is beter |}{dat wi hem doden dan dat}{wi ons seluen}\\

\haiku{wil ic di leren}{in die clergi dattu ghe-}{| leert mogeste}\\

\haiku{acker ghebouwet}{had ende hadde daer |}{in ghesaeyt vlasschen saet}\\

\haiku{huden en heeft hi.}{niet mo- | ghen ontgaen}{die scutten des doots}\\

\haiku{Die oghen plaghen}{mit ghenoechten scoon dinghen}{| te sien ende}\\

\haiku{wes ontfanck is}{sonde wes gheboerte is}{| onsalicheyt}\\

\haiku{dat is Twispraec der,,;}{creaturen Gheraert Leeu}{Gouda 4 april 1481}\\

\haiku{39a7 (boek begint op),,,,,,,;}{a2r b8-i8 k6 l8-m8 n6}{nn8-z8 {\cyryhcrs}6 96}\\

\haiku{400H initiaal, 3,.}{regels hoog geen zichtbare}{representant}\\

\subsection{Uit: Dye hystorien ende fabulen van Esopus}

\haiku{In zijn Latijnse ().}{editie1486 komen deze}{anekdotes wel voor}\\

\haiku{Na de Latijnse.}{vita volgt het leven van}{Esopus in het Duits}\\

\haiku{Het papier bevat,.}{veel gaatjes met name langs}{de binnenrand}\\

\haiku{een cleyne tijt daer}{hij hem tusschen onscul}{| dighen mochte}\\

\haiku{Soe coempt die ghene [}{die de sorghe hadde}{vanden hee- |}\\

\haiku{alzoe ghij siet dat.}{ick ben die minste en |}{de die cranckste}\\

\haiku{| Mer nae haren [].}{goetduncken gauen sij hem |}{5 den meesten last}\\

\haiku{Ick biddu dat ghij}{mijne woir | den int quaet}{niet nemen en wilt}\\

\haiku{Ende als esopus}{dese ant | woerde was}{hoorende begonste}\\

\haiku{dach als xanctus.}{hem baden | wilde met}{sijnen studenten}\\

\haiku{ghenomen hadde}{Ende als die | voeten}{ghenoech ghesoden}\\

\haiku{In dien daghe |}{des ordeels als die menschen}{verrisen sullen}\\

\haiku{| den welcken.}{ghij die costelijcke}{spyse gheseyndt hebt}\\

\haiku{Nu siet ghij wel dat}{v wijf die wech is ghegaen}{v niet lief en |}\\

\haiku{Mijn heerschap doet dy.}{bidden dat ghij coemt eten v}{myddachmael met | hem}\\

\haiku{hoe dattet sij Ick.}{wil vollen mynen buyck}{ende altijt eten}\\

\haiku{ende als dese}{tafelen aldus gheset}{| ende bereydt}\\

\haiku{fabule ende}{hystorie die welcke}{vertelt van twee rauen}\\

\haiku{Gaet en- | de}{deylt den scadt van goude die}{ghij gheuonden hebt}\\

\haiku{dat die honden den}{woluen ghegheuen ende}{gheleuert souden}\\

\haiku{dat ghij my wijset die}{ste- | de daer ghij den}{voirghenoemden toorn}\\

\haiku{ghelykerwijs als}{die merye peerden die ghi hebt}{la | ten comen}\\

\haiku{so namen sij eenen}{gulden cop vten tempel van}{appollijn haren}\\

\haiku{Ende badt sijnen}{meester dat hij eens mocht |}{die steden besien}\\

\haiku{hem dit gonnende}{so hebben sij hem op eenen}{wae | ghen gheset}\\

\haiku{dat hi tot ghenen}{daghen eenich broot | van hem}{en hadde ghehadt}\\

\haiku{Daer nae als die teue.}{haer cleyne honden ghe-}{| worpen hadde}\\

\haiku{Ende daerom.}{| en is haer gheselscap}{noch goet noch vruchtbaer}\\

\haiku{Waer af dat esopus.}{vertelt eene aldusdane}{| ghe fabule}\\

\haiku{Dat die ghene die [].}{altijt quaet doet |15 die}{wijle dat hij vroom}\\

\haiku{seg- | [15]}{| ghende totten afgod}{iupiter aldus}\\

\haiku{des doots want het is}{veel beter in sekerheyt}{groffelijc | ken}\\

\haiku{soe vergheue ict.}{v Mer alsoe langhe als}{ick | sal leuen}\\

\haiku{| Die dinghen die.}{ouermits Crachte ende}{vreese ghelooft sijn}\\

\haiku{|    HEt was een}{wolf die op eenre tijt vant}{een hooft van enen man}\\

\haiku{aldusdanighen}{sentencie ende sprack |}{totten wolf aldus}\\

\haiku{Die vijfthienste {\textsection}}{fabule is vanden wolf}{ende vanden hont}\\

\haiku{Hier om en is die}{ghene niet goet die twee |}{contrary heeren}\\

\haiku{ende dan sal ic}{| doen dat ghene dat v}{belieuen sal}\\

\haiku{want ist sake dat}{die | ossen dryuers oft}{onse meester dy}\\

\haiku{want ghij een Coninc.}{sijt so ist alte samen}{tot uwen ghebode}\\

\haiku{v eten oft hij sal}{v brenghen | op die merct}{om te vercopen}\\

\haiku{| sijn oft eyghen}{te wesen ouermits dat}{hij wrake van eenen}\\

\haiku{Ende daer om en}{salmen inden flatterers}{gheene | gheloue}\\

\haiku{Ha arme sotten.}{ghi en weet niet waer om dat}{| ick voerlope}\\

\haiku{mer hier is twerck}{der waerheyt gheble |}{ken ende gheschiet}\\

\haiku{ym] [e] een zweert in[]}{sinen wech ghevondene}{hebbe Ende |}\\

\haiku{die nochtans zeer sot[] {\textsection} ()}{is ende gheen wijsheyt en}{doet |         3586}\\

\haiku{Soe bidde ic v.}{dat | ghij my hier inden}{wech niet eten en wilt}\\

\haiku{want hi was in sijn}{hollekijn bi tlogijs vanden}{leeuwe daer hi |}\\

\haiku{wel ghespijset werden}{Ende sprac | tot desen}{tween scapen aldus}\\

\haiku{die scapen | [25]}{staen op beyde die hoecken}{vanden kamp Ende}\\

\haiku{een lam Ende doe}{dijne naersticheyt omme}{my dat te nemen}\\

\haiku{die wolf en soude.}{onse lammeren niet wech}{ghedraghen hebben}\\

\haiku{waer af dat ic dy []}{zeer dancke Ende die |}{20 wolf seyde hem}\\

\haiku{alle tgheent dat groen}{ende drooch is | Ende}{die derde seyde}\\

\haiku{ende alle |}{die telgheren Ende die}{rechter sprack tot hem}\\

\haiku{sal moghen alst noot []}{|10 is Ende die wolf}{antwoerdede hem}\\

\haiku{Ic wilt gheerne doen}{ende bins wel te vreden}{| Ende mettien}\\

\haiku{ende dat scaep}{saghen vercleet wesende}{mittet vel vanden}\\

\haiku{Het is die mensche []}{Ende hij seyde weder}{om |25 tot hem}\\

\haiku{want ic niet voerder}{gaen en mach Ende die man}{seyde | tot hem}\\

\haiku{Ende mettien so}{begonste hem die man op}{sijn hooft te smijten}\\

\haiku{Ghelijck alst blijct}{by deser nauolghender}{fabulen |}\\

\haiku{5] helpen wilt soe.}{sullen wi wel ter stont wt}{desen put comen}\\

\haiku{wat ghi doet dat doet{\textsection} ()}{wijslick ende aensiet dat}{eynde |          101}\\

\haiku{god aenbedende.}{was op dat hi hem vele}{goets verleenen wilde}\\

\haiku{hem geuende so.}{groten slach mitten hoofde}{teghens die muer}\\

\haiku{Ende als hi sach []}{datse niet dansen en |}{10 wilden soe werdt}\\

\haiku{hoofde ende wt []}{sinen baerde om dat}{hij |5 haer so}\\

\haiku{daeromme so wie.}{den anderen be- |}{rispen oft leeren wil}\\

\haiku{meynende dathet.}{eenen leeuwe geweest ware}{ende liepen wech}\\

\haiku{die welcke mal.}{| canderen ghemoetten}{op een watere}\\

\haiku{hoe sijdi al |}{dus dwaes van v seluen}{aldus te prijsen}\\

\haiku{fabule is van.}{eenen visscher ende | van}{een cleyn visschelkijn}\\

\haiku{ende te dijnen}{passe moghen teten so}{ic my betrouwe}\\

\haiku{ende groot profijt.}{dat hi van te voren te}{hebbene | plach}\\

\haiku{wachtu lieue sone.}{dat die mire niet wi |}{ser en sij dan ghi}\\

\haiku{zo laet dan voir mij.}{comen alle de vrou |}{wen van uwen huyse}\\

\haiku{Die lantman seide}{Daeromme | hebbe ic}{v gheuangen om}\\

\haiku{ne dat seker is[] {\textsection} ()}{voir tghene dat onseker}{is |         20156}\\

\haiku{Als sij ghegeten []}{hadden quam nedius ten}{|10 aenganghe}\\

\haiku{alle die ghene}{diemen hem aenbrochte die}{ontsinnich waren}\\

\haiku{Henderson, Arnold, '',:}{ClaytonHaving Fun with the}{Moralities in}\\

\haiku{Dit is ook de druk.}{die Hecker in zijn editie}{heeft getranscribeerd}\\

\subsection{Uit: Het ongelukkige leven van Esopus}

\haiku{En wat het ergste,.}{was hij was stom zodat hij}{geen woord kon spreken}\\

\haiku{Wie goed doet, mag dus.}{goede hoop op God hebben}{beloond te worden}\\

\haiku{{\textquoteleft}Maar ik heb een knecht,.}{die niet mooi maar wel van de}{juiste leeftijd is}\\

\haiku{Volgens ons hebt u.}{hem meegebracht om ons voor}{de gek te houden}\\

\haiku{Mijn echtgenote}{is zo verwaand dat ze het}{niet zou verdragen}\\

\haiku{{\textquoteleft}Heb je niet in de?}{gaten dat ik meer van haar}{houd dan van mijzelf}\\

\haiku{{\textquoteright} Vervolgens wendde,:}{Xanctus zich wederom}{tot Esopus en zei}\\

\haiku{{\textquoteleft}Ga naar de markt en.}{koop het allerbeste eten}{dat je kunt vinden}\\

\haiku{Toen de studenten,:}{het gerecht zagen spraken}{zij tot Xanctus}\\

\haiku{{\textquoteright} {\textquoteleft}Breng ons het derde,{\textquoteright}.}{gerecht beval Xanctus}{Esopus even later}\\

\haiku{Waarna Esopus als.}{derde gerecht wederom}{tongen opdiende}\\

\haiku{Kom daarom vandaag,.}{opnieuw bij mij dan zullen}{we wat anders eten}\\

\haiku{{\textquoteleft}Vrouw, doe water in.}{het hekken en was deze}{pelgrim de voeten}\\

\haiku{De steen die u daar,.}{ziet liggen lag eerst in het}{begin van het bad}\\

\haiku{{\textquoteleft}Niet u geeft mij de,.}{schat maar degene die hem}{hier neergelegd heeft}\\

\haiku{{\textquoteleft}Omdat de letters,{\textquoteright}, {\textquoteleft}:}{antwoordde Esopusons dat}{duidelijk maken}\\

\haiku{Weet je dan niet dat!}{de god die wij vereren}{er net zo uitziet}\\

\haiku{Nadat ze lekker,:}{gegeten hadden sprak de}{kikker tot de rat}\\

\haiku{{\textquoteleft}De armen moeten.}{door de rijken niet bespot}{of versmaad worden}\\

\haiku{Hij voegt daarom de:}{gecursiveerde woorden}{toe aan Macho's tekst}\\

\section{Aart van der Leeuw}

\subsection{Uit: Ik en mijn speelman}

\haiku{Wij schertsten dit weg,.}{of we een lastig insect}{van ons afsloegen}\\

\haiku{Spoedig verloor ik.}{hem uit het gezicht in het}{warnet der straten}\\

\haiku{Ook hij haalde de,.}{schouders op en keerde zich}{kalm tot zijn klanten}\\

\haiku{{\textquoteright} Mijn speelman ligt te,,;}{ijlen dacht ik zeker is}{de koorts gestegen}\\

\haiku{Toen ik de oogen weer,.}{opende lag mijn hoofd in den}{schoot van een meisje}\\

\haiku{Intusschen is het,.}{tijd geworden om aan den}{uitgang te denken}\\

\haiku{Het was al laat in,.}{den nacht geworden voor ik}{ter ruste kon gaan}\\

\haiku{Behagelijk schik.}{ik mij terecht tusschen de}{geurige halmen}\\

\haiku{Waar het schild uithing van,.}{een goudgele krakeling}{stapte ik binnen}\\

\haiku{{\textquoteright} De boer dankte en,}{draafde op zijn dunne beenen}{naar den kant heen dien}\\

\haiku{Valentijn keek een,.}{beetje bedroefd naar den kluif}{dien ik hem aanbood}\\

\haiku{Mijn makker diept een.}{dik geknopten stok op uit}{wat ouden rommel}\\

\haiku{{\textquoteleft}Valentijn,{\textquoteright} zeg ik, {\textquoteleft}.}{speel ons een liedje en noodig}{den nacht tot den dans}\\

\haiku{We buigen voorover,.}{en met de handen hollen}{we een grafkuil uit}\\

\haiku{Dadelijk daarop}{echter hooren wij achter}{ons in de herberg}\\

\haiku{Zoo ten minste werd,.}{er gemompeld en sedert}{bleef die plaats geschuwd}\\

\haiku{Mijn hart klopt luid als,}{van een kleinen jongen en}{waarlijk voel ik mij}\\

\haiku{{\textquoteright} Dan fluistert ze met,:}{bevende lippen terwijl}{ze mij bang aanziet}\\

\haiku{Ik moet uit zien te,;}{vinden waar hij zijn hoofdpijn}{uit ligt te slapen}\\

\haiku{Nieuwsgierig sloop ik,.}{naderbij en maakte een}{slip los van het pak}\\

\haiku{{\textquoteright} Dezelfde kamer,,.}{op het meer uitkomend werd}{mij aangewezen}\\

\haiku{Ik sprong van den stroozak,,.}{dankte hem en zwoer hem zijn}{hulp te vergelden}\\

\haiku{{\textquoteright} Van verwondering.}{rolde hij met ton en al}{ondersteboven}\\

\haiku{hij zoo ruimschoots had,.}{genoten hem nog door het}{brein gemoesseerd heeft}\\

\haiku{Ik wilde hem naar:}{aanleiding er van een vraag}{stellen en zeide}\\

\haiku{{\textquoteright} riep hij ge\"ergerd,.}{en haalde medelijdend}{de schouders op}\\

\haiku{{\textquoteleft}Aha, die Mathilde,,.}{d'Almonde die feeks en dat}{dekselsche manwijf}\\

\haiku{Een stilte, en dan,.}{de trapleer die met zwaren}{stap wordt bestegen}\\

\haiku{{\textquoteright} {\textquoteleft}En {\'\i}k zat juist aan;}{een anderen held uit de}{oudheid te denken}\\

\haiku{Toch verwijt me mijn,.}{geweten geen oogenblik}{dat ik bedrog pleeg}\\

\haiku{Uit straf er voor werd}{ik dadelijk daarop aan}{de mouw getrokken}\\

\haiku{{\textquoteright} Verscheen mijnheer de.}{Pomponne en maakte een}{deftige buiging}\\

\haiku{De dorpsklerk leest nu,;}{het stuk voor en als hij er}{onder gezet heeft}\\

\subsection{Uit: De kleine Rudolf}

\haiku{En nu klettert een,,}{glazen deur open je stommelt}{een paar trapjes op}\\

\haiku{Dat de kussenkast.}{ze in zijn oud-eiken}{binnenste berge}\\

\haiku{je afstamt, en dat,.}{je een man bent die in zijn}{jeugd gestudeerd heeft}\\

\haiku{Op een morgen stormt.}{Koba met een hoogrood hoofd}{mijn kamer binnen}\\

\haiku{Neen, waarlijk, slechts goed,.}{over doden en oom Jakob}{stierf niet lang daarna}\\

\haiku{En dan plotseling,.}{bruist een stormvlaag nader doet}{de ruiten trillen}\\

\haiku{Verschrikt sta ik stil,.}{buiten terwijl ik langzaam}{tot bezinning kom}\\

\haiku{Daar zijn er op mijn,,.}{hoedrand gevallen op mijn}{knie\"en mijn handen}\\

\haiku{{\textquoteleft}Ach veel te weten,{\textquoteright}, {\textquoteleft}.}{zucht ze mismoedigik ben}{vroeg van huis gegaan}\\

\haiku{Ernstig inspecteert!}{hij de spijzen en snort als}{een oud spinnewiel}\\

\haiku{Natuurlijk, dat hij,.}{zwijgt erover dat hij me de}{weg heeft gewezen}\\

\haiku{Roomwit is Clara,.}{en met een waasje mosgroen}{om de natte muil}\\

\haiku{dat als een hamer.}{op zijn aambeeld in mijn borst}{begint te kloppen}\\

\haiku{Een wildvreemd meisje,,,}{vraagt je op bezoek neen zeg}{je onmogelijk}\\

\haiku{Het blijkt een nauwe,.}{zitplaats als we er ons in}{hebben gewrongen}\\

\haiku{Maar de ander heeft,.}{al naar mijn hand gegrepen}{krachtig schudt hij die}\\

\haiku{Tussen ons en het.}{doelwit spreidt zich zij{\"\i}g het}{groene gazon uit}\\

\haiku{Ik zal beginnen,,.}{natuurlijk met het schiettuig}{dat het kleinste is}\\

\haiku{Nu schrik je bijna,;}{van de ruk waarmee ze de}{hand naar het oor brengt}\\

\haiku{En wat blijft me dan,.}{verder nog over nu alles}{uitgesproken is}\\

\haiku{Ik begin met de,:}{aanhef nadat ik een blad}{glad gestreken heb}\\

\haiku{En ik, die ze me.}{altijd als onnoemelijk}{rijk heb voorgesteld}\\

\haiku{Als ze me uitlaat,,;}{glimlacht ze tegen me god}{weet waarom dankbaar}\\

\haiku{{\textquoteright} En nu haak je de.}{koperen sloten van de}{statenbijbel open}\\

\haiku{Bij het weggaan merk.}{ik dat ze een hoogrode}{kleur heeft gekregen}\\

\haiku{Ademloos laat ik bij;}{een straathoek mijn hartklop tot}{bedaren komen}\\

\haiku{Terwijl de wereld:}{van je vrijheid vlak bij je}{wenkt door de ruiten}\\

\haiku{Op een ongelijnd:}{blad schrijf ik hem uit in mijn}{fraaiste schoonschrift}\\

\haiku{Ik beloof haar een,.}{hulp bij de arbeid die {\'\i}k}{zal bekostigen}\\

\haiku{toch kunnen er in.}{de kortst mogelijke tijd}{met je gebeuren}\\

\haiku{Dat ze om de schat,.}{moet denken waarvan ze de}{draagster mag wezen}\\

\haiku{{\textquoteright} en ik zie Martha.}{met een beweging van schrik}{achteruit wijken}\\

\haiku{Nu wil ze, dat ik.}{zelf het onderwerp van het}{gesprek zal worden}\\

\haiku{{\textquoteright} Door het aanzetten.}{van de motor gaat me haar}{antwoord verloren}\\

\haiku{En dan breekt hij in,.}{een stille steeg de stenen}{los en gaat graven}\\

\haiku{Terwijl ze in de,,,}{houding blijven staan kijk ik}{neer op ze ja n\'u}\\

\haiku{Goed, ja, uitstekend,.}{en binnen twee jaar had ik}{mijn examen gedaan}\\

\haiku{{\textquoteright} {\textquoteleft}Goed,{\textquoteright} zeg ik, \'e\'en woord,.}{slechts maar met het brandmerk van}{mijn eerste leugen}\\

\haiku{Mogelijk dat ik.}{daarvoor op een emmer zou}{hebben te klimmen}\\

\haiku{Het is alles een,,.}{beeld slechts nevel geschetst in}{een paar wolkstrepen}\\

\section{Jan de Liefde}

\subsection{Uit: Uit drie landen}

\haiku{hoe spoedig kan men....{\textquoteright}}{mij berooven van mijn jeugdig}{leven en dan zou}\\

\haiku{De Lollards waren '.}{overt algemeen een stil}{en lijdzaam volkje}\\

\haiku{maar ging hij met list.}{en overleg te werk om zich}{den weg te banen}\\

\haiku{Misschien is er op.}{een of andere manier}{wel iets aan te doen}\\

\haiku{Wel verbazend, welk,!}{een zware ijzeren deur}{is dat kapitein}\\

\haiku{Nu draagt hij de kroon,.}{der heerlijkheid en niemand}{kan hem die ontrooven}\\

\haiku{Daarbij kwam, dat wij.}{volstrekt niet wisten wat er}{van ons worden zou}\\

\haiku{Na haar kwamen er,.}{nog andere vrouwen die}{hetzelfde deden}\\

\haiku{{\textquoteright} En nu verhaalde.}{hij den moord in al zijne}{bijzonderheden}\\

\haiku{De opzichter nam,.}{hiermede genoegen want}{hij kon niet anders}\\

\haiku{In plaats van banken;}{waren er eenige planken}{op blokken gelegd}\\

\haiku{Daarom drong hij er.}{nu des te meer op aan dat}{de boom vallen moest}\\

\haiku{{\textquotedblright} - Zijn smeeken hielp hem,.}{niets hij werd gedood en in}{stukken gehouwen}\\

\section{P. van Limburg Brouwer}

\subsection{Uit: Romantische werken. Deel 2. Diophanes}

\haiku{Hij  worstelde,,,,,.}{liep jaagde schoot zwom evenals}{al zijne makkers}\\

\haiku{Men doet dat bij ons, -;}{zoo niet en vooral men doet}{het niet te Sparta}\\

\haiku{Dit tooneel is  voor.}{het vervolg belangrijker}{dan gij misschien denkt}\\

\haiku{- Wel zeg mij dan eens,,?}{jonge vriend wat keurt gij het}{beste voor den mensch}\\

\haiku{Ongetwijfeld, het.}{zijn de stappen van iemand}{die haastig voortgaat}\\

\haiku{Hoe weet ik dan of,,?}{dit een meisje was of een}{godin of een nimf}\\

\haiku{Hoe 't ook zij, de.}{uitkomst bewees dat ik mij}{niet bedrogen had}\\

\haiku{Zoodra zij mij,.}{zag wees zij mij den weg dien}{ik te volgen had}\\

\haiku{Ik bracht er deze}{woorden zoo schielijk uit dat}{zij niet in staat was}\\

\haiku{Eene godin was zij,.}{niet maar toch ook geene vrouw zooals}{andere vrouwen}\\

\haiku{doch ik bemerkte.}{aldra dat hij zich uit de}{voeten gemaakt had}\\

\haiku{Zij, gij weet wel, heeft,.}{er genoeg maar die zitten}{altijd achter slot}\\

\haiku{in een woord, hij had, -.}{een zware kou gevat zoodat}{hij niet kon uitgaan}\\

\haiku{Dat ik de lier van,,.}{Timotheus en de Scias47}{moest zien begrijpt zich}\\

\haiku{Het heeft ook zijne,....}{lasten eene jonge vrouw zoowel}{als zijne lusten}\\

\haiku{Pedonomus,  ,!}{hier een uwer kweekelingen die}{uit het gelid loopt}\\

\haiku{Maar nu die man, dacht,!}{ik alweder verliefd op}{de schoone Gorgo}\\

\haiku{Leon had het mij,;}{niet verzocht maar hij had mij}{toch zijn nood geklaagd}\\

\haiku{en zou een jongen?}{er dan zijne grillen niet}{aan opofferen}\\

\haiku{- Ja, zie nu eens, ik.}{wed dat gij nog niet eens weet}{wat hier gebeurd is}\\

\haiku{- Maar wat, vroeg ik nu,?}{heeft u bewogen u aan}{mij over te geven}\\

\haiku{- Wat, zeide hij op,.}{een vrij hoogen toon wat anders}{dan hunne ontrouw}\\

\haiku{Gij zoudf ons van een, ....}{onberekenbaren dienst}{kunnen zijn en zelf}\\

\haiku{Hoe dit zij, er was.}{niet anders op dan mij in}{mijn lot te schikken}\\

\haiku{Jongenlief, dat gaat.}{zoo gemakkelijk niet in}{die nauwe straten}\\

\haiku{Er behoefde hier.}{naar geen balsem of stlengis}{gezocht te worden}\\

\haiku{Ik moet echter tot,}{mijne schande bekennen}{dat ik zoodra}\\

\haiku{- Een slaaf, ja, maar dat.}{zegt hier wat anders dan in}{andere steden}\\

\haiku{Hier was het in den.}{beginne nog moeilijker}{gehoor te krijgen}\\

\haiku{maar gij moet nog een ().}{bewijs van den apographeus}{controleur hebben}\\

\haiku{Evenwel er waren,;}{er daar behoefde ik niet}{aan te twijfelen}\\

\haiku{Daar is die oude,.}{zaniker weer hoorde}{ik er een zeggen}\\

\haiku{Eensklaps, als door schrik,,:}{verbijsterd vloog zij op en}{vroeg mij hijgende}\\

\haiku{- Maar bij Hercules,,.}{zeide Polycles nog eens}{wacht hen gerust af}\\

\haiku{Bij Hercules, ik.}{geloof dat Plutus er zelf}{in gekropen is}\\

\haiku{- Zoo, ik dacht dat hier.}{in Athene allen gelijk}{waren voor de wet}\\

\haiku{- Ja, en waarom niet,,.}{ten minste om kwaad te doen}{hernam ik schielijk}\\

\haiku{Hij zal natuurlijk;}{de welwillendheid zijner}{moei niet versmaden}\\

\haiku{Gij verwondert u,;}{zeker daarover ten hoogste}{wijze Demeas}\\

\haiku{Als gij hem spraakt, zoudt.}{gij nooit zeggen dat hij zulk}{een groot wijsgeer is}\\

\haiku{Voor den ingang der.}{Academie stond een altaar}{aan Eros geheiligd}\\

\haiku{Maar is niet Plato's?}{wijsbegeerte geheel op}{de Liefde gegrond}\\

\haiku{Ja, wie weet of gij,,}{nog niet zooals zij zult zeggen}{dat ik wel dwaas ben}\\

\haiku{die heeren vliegen,.}{over de rivieren of zij}{ooievaars waren}\\

\haiku{Maar (en ik ben het)}{aan mij zelven verplicht er}{dit bij te voegen}\\

\haiku{Ik had, ja, mij niet.}{zoo onvoorbereid daarheen}{moeten begeven}\\

\haiku{mij ging alleen het,.}{resultaat aan en dat werd}{ik spoedig gewaar}\\

\haiku{Gelukkig verstond.}{ik dien blik en had ik tijd}{mij te herstellen}\\

\haiku{en een vreemdeling.}{wil ik altijd gaarne eene}{inlichting geven}\\

\haiku{de Eleusinia te willen,.}{bijwonen127 en mij zelfs te}{laten inwijden}\\

\haiku{(ik wilde hier iets,).}{anders zeggen maar ik had}{er den moed niet toe}\\

\haiku{Honderdmaal dacht ik:}{om het gezegde van den}{wijzen Aristippus}\\

\haiku{Honderdmaal stond ik}{op het punt tot hem te gaan}{en hem te smeeken}\\

\haiku{Verwonderd over dit,,,:}{gezicht neem ik den brief scheur}{het koord los en zie}\\

\haiku{Ik vlieg de kamer,,.}{uit roep de huishoudster vraag}{haar naar Lagisca}\\

\haiku{Uit vrije keuze heb.}{ik u gekozen boven}{honderd anderen}\\

\haiku{- Elpinice, die, -.}{gij reeds vroeger op Creta}{bemind hebt schreef zij}\\

\haiku{- Vergeef, vergeef mij,,}{zeide hij bedenk dat ik}{om zoo te zeggen}\\

\haiku{Ik herken er u.}{aan en zou niet anders van}{u verwacht hebben}\\

\haiku{- Dat zal moeielijk zijn,,,.}{Lamprias zeide ik want}{ik heb geen huis meer}\\

\haiku{De wet der natuur.}{wil dat de sterkere meer}{heeft dan de zwakke}\\

\haiku{Zijn doel is vrijheid,.}{en geluk zooals dat van den}{onrechtvaardige}\\

\haiku{- Lampis is er nog,,.}{niet zeide mijn geleider}{maar hij komt zeker}\\

\haiku{- De stem die u dat,,.}{zeide sprak nu Athenagoras}{was die der godheid}\\

\haiku{Mij dacht, eene vrouw als.}{Elpinice kon toch zoo}{onbekend niet zijn}\\

\haiku{gij hadt mij gezegd,.}{waar gij woondet en ik ben}{nooit bij u geweest}\\

\haiku{Neen, Diophanes, stel,.}{u gerust gij hebt u niets}{te verwijten}\\

\haiku{, zoo weinig met u,;}{bekend wel gewacht hebben}{u te waarschuwen}\\

\haiku{hervatte hij met.}{eene zachte stem en bijna}{bedeesde houding}\\

\haiku{Vraag mij nu niet hoe;}{de rechtvaardigheid eene kunst}{kan genoemd worden}\\

\haiku{20Er is op Creta,,.}{zegt Plutarchus een beeld van}{Zeus zonder ooren}\\

\subsection{Uit: Romantische werken. Deel 1. Een leesgezelschap te Diepenbeek. Een ezel. Eenig speelgoed}

\haiku{het verbond, reeds daar,.}{gemaakt te hernieuwen en}{te bevestigen}\\

\haiku{Ik, voor mij, zou het;}{dwaasheid vinden nu reeds aan}{eene plaats te denken}\\

\haiku{maar, al gaat het in,.}{Utrecht of in den Haag beter}{dat helpt ons hier niet}\\

\haiku{De boeken, die ter,;}{tafel moeten komen zijn}{mij al gezonden}\\

\haiku{En, ik weet niet of;}{gijlieden allen zulke}{sterke bollen hebt}\\

\haiku{Als ik in uw plaats,,.}{was baas Hartman dan gaf ik}{hem het roer maar over}\\

\haiku{Ik laat nu daar dat:}{de drie-eenheid er met geen}{woord in vermeld wordt}\\

\haiku{In het tweede deel,}{werd nu de zaak omgekeerd}{en aangetoond welk}\\

\haiku{- Wie uwer, zeide hij,?}{onder anderen zou nog}{kunnen aarzelen}\\

\haiku{om malkanderen,}{te verdragen in liefde}{en te behouden}\\

\haiku{Het groote onderscheid,.}{was of zij wakker waren}{dan of zij sliepen}\\

\haiku{- Hoe dat, vader, vroeg,?}{nu de burgemeesterske}{was het dan niet goed}\\

\haiku{des kapiteins hand,:}{hartelijk en zeide zijn}{lach versmorende}\\

\haiku{- Met uw permissie,,.}{hernam de kapitein dat}{heb ik niet gezegd}\\

\haiku{Ik zeg maar dat die,.}{leer God niet verheerlijkt maar}{hem oneer aandoet}\\

\haiku{Wij willen daarom.}{het lieve meisje echter}{niet veroordeelen}\\

\haiku{Ik heb u zelfs nog,.}{verscheiden vragen te doen}{voor wij zoover zijn}\\

\haiku{Vreest gij niet dat gij.}{uzelven van het koninkrijk}{Gods zult uitsluiten}\\

\haiku{- Dus het uitkiezen.}{en het niet uitkiezen hangt}{alleen van God af}\\

\haiku{Eerst hebt gij alleen.}{gezegd dat gij het niet met}{Hellenbroek eens waart}\\

\haiku{Ik wil nu niet eens {\textquoteleft}{\textquoteright}.}{vragen of dat woordgodsdienst}{hier wel gepast is}\\

\haiku{En nu, vrienden, nu,,.}{zooals ik straks zeide eens over}{een anderen boeg}\\

\haiku{Schielijk nam nu de:}{chirurgijn-diaken het}{boekje en zeide}\\

\haiku{{\textquoteright} Dat gaat best. Geef mij ', ';}{t boekske maar mee ik zal}{t voor u wel doen}\\

\haiku{In hoeverre die,.}{conclusie juist was zullen}{wij nu daarlaten}\\

\haiku{, maar daarom is het.}{niet minder zooals ik u daar}{voorgelezen heb}\\

\haiku{- Wat niet mogelijk,!}{is bij de menschen dat is}{mogelijk bij God}\\

\haiku{Nu, hernam de heer,}{Van Groenendaal het doet mij}{genoegen te zien}\\

\haiku{voor zooveel gij dit,}{aan de armen zult gedaan}{hebben voor zooveel}\\

\haiku{- Ja, wel een booswicht,,.}{hervatte Willem en een}{listige booswicht}\\

\haiku{Gij begrijpt dat ik;}{in het eerst als verpletterd}{was van verbazing}\\

\haiku{En wat lezen wij?}{in den tweeden brief aan de}{Thessalonicensen}\\

\haiku{en op dat drietal.}{prijkte bovenaan de naam}{van Jacobus Klos}\\

\haiku{Was dit alles niet?}{meer dan het gevolg van Gods}{eeuwig raadsbesluit}\\

\haiku{Evenwel, zoo verblind,.}{was hij niet of hij begon}{het zelf in te zien}\\

\haiku{- Hoe komt het toch dat?}{die Klos bovenaan op het}{drietal bij u staat}\\

\haiku{Dat zal een ander;}{leventje zijn dan met dien}{dominee Wilbrink}\\

\haiku{hij was vriendelijk,.}{genoeg maar ik kon het zoo}{niet met hem vinden}\\

\haiku{Hij begreep dus ook.}{eens een anderen toon te}{moeten aannemen}\\

\haiku{Alle schrift is van,,;}{God ingegeven dat staat}{er gij hebt gelijk}\\

\haiku{wie zou in hare?)}{omstandigheden anders}{gehandeld hebben}\\

\haiku{Hartman en Kootje,.}{gaan naar Amerika met de}{afgescheidenen}\\

\haiku{Ik heb u eenige.}{gewichtige tijdingen}{mede te deelen}\\

\haiku{Ik ontving hem, om,;}{u de waarheid te zeggen}{niet heel vriendelijk}\\

\haiku{Doch wees voor Willem.}{niet bevreesd die zal den Heer}{niet ontrouw worden}\\

\haiku{Het was haar wel niet.}{ongewoon den kapitein}{te hooren lachen}\\

\haiku{Want als ik Cesar,.}{was dan zou ik mijzelven}{wel kunnen helpen}\\

\haiku{Van welk een geest denkt,,?}{gij dat Caligula de}{Cesar bezield is}\\

\haiku{Gij vraagt immers niet ', '.}{naart geen hij mag maar naar}{t geen hij kan doen}\\

\haiku{Palestra kwam om.}{het noodige voor den maaltijd}{gereed te zetten}\\

\haiku{Geduld dus, lieve,;}{jongen bedenk dat gij het}{om mijnentwil doet}\\

\haiku{Maar weldra begon.}{ik ernstig over mijn eigen}{lot na te denken}\\

\haiku{Hij onderzocht ook.}{nauwkeurig naar de plaats van}{mijne geboorte}\\

\haiku{{\textquoteright} De oude man vroeg,.}{nu of ik wel mak was en}{geen kribbebijter}\\

\haiku{Dit gedeelte van.}{mijne geschiedenis is}{wezenlijk tragisch}\\

\haiku{De geheele buurt,.}{wordt bijeengeroepen om}{het wonder te zien}\\

\haiku{- een vernieuwd en nog.}{luider gelach was al het}{antwoord dat ik kreeg}\\

\haiku{- Doch goed, ik zal mij,,!}{zelven wel helpen wacht maar}{wreede Palestra}\\

\haiku{Palestra deed mij, -.}{beloften en kuste mij}{en ik bleef een ezel}\\

\haiku{Vrienden, om zijne,.}{genietingen mede te}{deelen heeft hij niet}\\

\haiku{De eerste liefde,,.}{is nog zoo gij het noemt de}{liefde der onschuld}\\

\haiku{Wat zeg ik, het is.}{niet eens noodig het geprezen}{werk zelf te lezen}\\

\subsection{Uit: Romantische werken. Deel 3. Charicles en Euphorion. Grillus}

\haiku{want hij zou het voor.}{geen geld uit zijne handen}{gegeven hebben}\\

\haiku{Euphorion dit,:}{bemerkende zeide op}{een lachenden toon}\\

\haiku{de getrouwe, de,.}{deelnemende vriend hij was}{de gelukkige}\\

\haiku{In \'e\'en woord, lieve,.}{Charicles de Platonist}{werd een Cynicus}\\

\haiku{Het klamme zweet brak.}{Hermotimus eensklaps bij}{deze woorden uit}\\

\haiku{Polydorus nog het.}{best geschikt om de zaken}{te vereffenen}\\

\haiku{Lyde trippelde,.}{naar het bestje toe en bracht}{haar in mijn vertrek}\\

\haiku{Charicles vond bij,:}{Polydorus hetgeen hij lang}{vergeefs gezocht had}\\

\haiku{Zij nam het stoute,;}{besluit het haren vader}{bekend te maken}\\

\haiku{- Zoo, zoo, dan neemt gij!}{maar vast de eerste klasse}{voor u zelve in}\\

\haiku{Op deze wijze;}{nam het gesprek eene eenigszins}{andere wending}\\

\haiku{Uwe ziel, het beste,.}{van uw aanzijn behoort niet}{tot deze wereld}\\

\haiku{Of gij bij deze,.}{ruiling wint of verliest kan}{ik niet bepalen}\\

\haiku{maar zoo ik u zeg,;}{gij behoeft niet te vragen}{of gij dat doen moogt}\\

\haiku{Sosandra begon met.}{eene geveinsde koelheid en}{onverschilligheid}\\

\haiku{Met dat al droeg men,.}{nauwkeurig zorg zijn geduld}{niet uit te putten}\\

\haiku{Het genot, dat hij,.}{als het hoogste goed stelde}{had hem bedrogen}\\

\haiku{- Ziedaar dan het loon,!}{der liefde ziedaar de prijs}{mijner weldaden}\\

\haiku{Wat zij op weg nog,;}{verder gesproken hebben}{is niet zoo bekend}\\

\haiku{, en door Charicles.}{lang gevreesde oogenblik}{was eindelijk daar}\\

\haiku{- Zult gij eene slavin?}{tegen hare meesteres}{laten getuigen}\\

\haiku{Gij hebt, zeide hij,, -.}{met moeite voortgaande mij}{het leven gered}\\

\haiku{Maar was het dan een,?}{zoo noodzakelijk een zoo}{verdienstelijk werk}\\

\haiku{Charicles maakte,:}{van deze gelegenheid}{gebruik en zeide}\\

\haiku{- Die echtgenoot is,.}{Charicles van Athene de}{zoon van Glaucias}\\

\haiku{Wij hebben beiden.}{een tegenovergestelden}{weg ingeslagen}\\

\haiku{Het gezegde dient.}{alleen om u  tot de}{orde te roepen}\\

\haiku{Hebben wij er geen,.}{eer van wij hebben ook geen}{schande te vreezen}\\

\haiku{Het komt zelfs niet eens.}{te pas over het een of het}{ander te denken}\\

\haiku{52De twee eerste,,.}{beroemde dichteressen}{zijn genoeg bekend}\\

\section{Manuel van Loggem}

\subsection{Uit: Insecten in plastic}

\haiku{De kleuren waren,.}{hard en scherp net tegen de}{werkelijkheid aan}\\

\haiku{{\textquoteleft}Je had niet zo gauw.}{je reserves van weerstand}{moeten uitputten}\\

\haiku{Het was alles zo.}{onzinnig en eigenlijk}{onbegrijpelijk}\\

\haiku{Dit schilderij is{\textquoteright}.}{mooi en wilde daarmee mijn}{weerstand verdoven}\\

\haiku{Als ik maar wist dat.}{het donker was buiten zou}{me dat rust geven}\\

\haiku{Het is zelfstandig.}{geworden en ook Richard}{is onderhorig}\\

\haiku{Ik vind het mooi,{\textquoteright} zei.}{de jongen en ik wist dat}{hij het  meende}\\

\haiku{Het komt toch alleen.}{aan op de ontroeringen}{die een kunstwerk wekt}\\

\haiku{{\textquoteright} Hij had een droge.}{stem en sprak of hij me een}{lesje wou geven}\\

\haiku{Als je het ziet dan.}{voel je dat een moeder haar}{kinderen liefheeft}\\

\haiku{Wat bedoelde hij?}{er mee dat het de laatste}{keer zou kunnen zijn}\\

\haiku{Wie  die ander.}{was kon ik me achteraf}{nooit herinneren}\\

\haiku{Dat ik de tijd niet.}{kon indelen vond ik niet}{meer zo hinderlijk}\\

\haiku{Soms betastte ik}{mijn gezicht om mij ervan}{bewust te worden}\\

\haiku{Dageraad 1946 - Het -.}{volksgezicht 1948 De tocht van}{de dronken man 1950}\\

\haiku{Het werk van Gerrit -.}{Achterberg 1950 Inleiding}{tot het toneel 1951}\\

\section{Harie Loontjens}

\subsection{Uit: Wieker Lui}

\haiku{Zou dit misschien een}{reminiscentie zijn aan}{een oud volledig}\\

\haiku{Meint d'r mesjiens of 't}{plezerig is veur miech um}{vaan oet de winkel}\\

\haiku{Iech snapde neet gaw ', '.....}{gen\'og watr meinde meh toen}{zagr eur water}\\

\haiku{Ze m\^os strak toch eve,.}{denao touw goon zoe gaw es}{Tonia toes waor}\\

\haiku{Jao, meh zoe'ne nuije, '.}{dee w\`et alweer get miejer}{esnen aandere}\\

\haiku{'n maog vaan e '.}{paar hoezer weijer m\`etne}{verbranden errem}\\

\haiku{Meh noe d'r toch heij,.}{zeet noe hoof iech ouch neet nao}{Uuch touw te komme}\\

\haiku{Meh ze wis ouch wie ' '.}{t oetzaog in aander}{tije vaant jaor}\\

\haiku{Ze zaog altied '.}{oet wiet iewig leve}{en heel h\"a\"or good hum\"or}\\

\haiku{Nein, es 'r 't k\^os, '.}{veerdig kriege d\'a\'ann hoes}{heij in de straot}\\

\haiku{Eder hoeshawwe.}{kraog zie kruus en noe ouch}{mesjiens weer dat vaan h\"a\"om}\\

\haiku{{\textquoteleft}Geer.... gere s\^okker?.... '.}{h\`et Waor goddaank gei}{resep wat koed k\^os}\\

\haiku{rege en zoe get '.}{kaw boeste besj eemp veur bis veur}{t kaw te neume}\\

\haiku{had 't zellef es '}{kraol gez\'onge en nog}{waort ein vaan}\\

\haiku{Quand le J\'esus venait,.....}{au monde Le bon sauveur}{au barbe blonde}\\

\haiku{'n Sjoen m\`es hadde,.}{ze gelierd z\`esstummig m\`et}{de jonges debij}\\

\haiku{Eigenaordig.}{tot zoe wienig Mastreechse}{keersleedjes waore}\\

\haiku{Dee had al koffie.}{gehad m\`et de zengers en}{z'n stumpkes al op}\\

\haiku{boe ze gans allein.}{st\'onge m\`et um hun niks es}{blaanke witheid}\\

\haiku{Dat waor veur d'n}{iersten erreme mins dee}{sanderendaogs}\\

\haiku{En noe kraog eder - -.}{nog altied vaan oonder de}{serv\`et e st\"okske}\\

\haiku{De winter kump en,....}{deft us beve De winter}{kump en dao is noed}\\

\haiku{{\textquoteright} meh de res had 't:}{leedsje euvergenome}{en z\'ong weijer}\\

\haiku{Z'n sjotel broed zat ' ' '.}{r neer en op zien tiene}{leepr naot leech}\\

\haiku{Zien femilie woort}{gezegend en gaof v\"a\"ol}{vaan h\"a\"or kinder aon}\\

\haiku{Meh nein, daan zouwe....}{ze toch ouch tegen  eur}{vrijaasj zien gewees}\\

\haiku{Me geit 'ne jonge '....}{geistelik aofhole en}{d\'at ist veurnaomste}\\

\haiku{De gouwen tieger{\textquoteright} ' '.}{waort daonaon}{dr\"okde vaan belang}\\

\haiku{Eder muzikant en}{zenger k\^os dao z'n b\"ongskes}{inwissele veur}\\

\haiku{In 383 verplaatste hij,.}{zijn zetel naar Maastricht waar}{hij in 384 stierf}\\

\haiku{Aanvankelijk heeft.}{het gestaan in de Oude}{Minderbroederskerk}\\

\haiku{bij Roermond bevindt.}{zich een bron welke genoemd}{wordt naar Sint Servaas}\\

\section{Jac. van Looy}

\subsection{Uit: Feesten}

\haiku{Met hollige oogen:}{lag juffrouw Broense in de}{kamer te kijken}\\

\haiku{och, juffrouw verlost,.}{U me van een koppie ze}{benne met suiker}\\

\haiku{Hij knikte naar den,:}{bedste\^e-hoek vroeg met zijn}{korte beveelstem}\\

\haiku{{\textquoteleft}Leukerd, jij begrijpt,,?}{me geef jij me eens de hand}{hoe oud b\`e-je}\\

\haiku{jou trommel is ook,,{\textquoteright}.}{nog niet geborsten nog lang}{niet zei oome Piet}\\

\haiku{{\textquoteright} De Bruid, verheerlijkt,,:}{stijf op haar stoel groette toen}{het leven uit was}\\

\haiku{De kachel was ook, ', '?}{me\^e verhuisdt was warmpies}{genogt wast niet}\\

\haiku{je kan nooit weten,{\textquoteright}....}{met die presenten gaf zij}{zich zelve antwoord}\\

\haiku{wij weten er hier,{\textquoteright}.}{wel raad me\^e snaakte oome}{Willem goedhartig}\\

\haiku{{\textquoteright} {\textquoteleft}Een goed geloof en,.}{een kurke ziel dan drijft een}{mensch altijd boven}\\

\haiku{O, die draaiende,;}{liedjes de kamer begon}{er van te draaien}\\

\haiku{{\textquoteleft}da's mooi gezeid,{\textquoteright} gaf:}{hij dadelijk toe en me\^e}{zong de kamer}\\

\haiku{Soms had ze geloofd.}{dat de lage zolder kwam}{ne\^erdreunen op haar hoofd}\\

\haiku{Ze hield de hand aan, {\textquoteleft}.}{haar voorhoofddie warmte gaat}{nou we\^er beginnen}\\

\haiku{maar h\`un zou dat ten.}{eeuwigen dage kwalijk}{worden genomen}\\

\haiku{{\textquoteleft}Als ie opstaat, valt,{\textquoteright},;}{ie kletste oome Willem}{zijn bro\^er bedoelend}\\

\haiku{och, och, ze wist niet,,,....}{waar te kijken het werd haar}{paars voor de oogen foei}\\

\haiku{Onder de starre.}{lamp verhieven de bazen}{zich met boller oogen}\\

\haiku{hij at met smaak en,.}{een appel in zijn zak voor}{Grietje dat had hij}\\

\haiku{de gladde flesschen,;}{kwamen want iedereen had}{er aardigheid in}\\

\haiku{Weelsen werd v\`o\`or door.}{Willem geholpen aan het}{zoeken naar zijn jas}\\

\haiku{{\textquoteright} terugzakte langs}{den zeepmast en zoo smachtend}{was blijven kijken}\\

\haiku{de gouvernante,,.}{gehoorzaam kwam waarschuwend}{een eindje aan}\\

\haiku{mannen met sikken,;}{en uit den nek geschoren}{haar oude zeelui}\\

\haiku{een sieraad van een:}{meid heeft ze zoo mooi lang zwart}{als een gitana}\\

\haiku{Toch zijn er ook wel;}{straten waar het tot diep-in}{ellendig zwart is}\\

\haiku{koekenbakkers en;}{de komenijs die glazen}{vol kokinjes heeft}\\

\haiku{Maar er zijn er nog,:}{wakker genoeg hoor hoe ze}{leven maken}\\

\haiku{van uit de bedste\^e,}{zijn geeuwen zacht klaagde en}{we\^er wat later kwam}\\

\haiku{iedereen was toch,;}{zoo goed voor hem de klantjes}{vergaten hem niet}\\

\haiku{dat doet goed van een, '.}{vreemde te ervarent}{geeft wel een dag steun}\\

\haiku{Ja, de zon zou de.}{jongen branden tusschen de}{koren-velden}\\

\haiku{{\textquoteright} Niet op zijn gemak ',:}{voor dat gezicht int bed}{meende hij verder}\\

\haiku{En Juffrouw Weelsen.}{was toch wel blij dat Broense}{er al geweest was}\\

\haiku{Zij waren als met.}{blauwe verf besmeerd en in}{doorsopte kle\^eren}\\

\haiku{En hij had geen duur,.}{in zich gehad alles moest}{nog worden geboekt}\\

\haiku{Of ze die passen?}{nog maakte en de handen}{hield boven het hoofd}\\

\haiku{{\textquoteleft}hoor es, moe,{\textquoteright} ook die:}{haar hand en allebei de}{handen wippend zoo}\\

\haiku{Hoe ze eindelijk.}{morgen voor de melange}{zouden klaar komen}\\

\haiku{je weet geen raad, ik '.}{houd het er voort loopt nog}{eens falikant uit}\\

\haiku{{\textquoteleft}Hoor me nou zoo'n schaap,{\textquoteright},:}{an mopperde Antoon en}{Geertrui vervolgde}\\

\haiku{dan zegt-ie {\textquotedblleft}bijen{\textquotedblright}:,:}{en zeg je dan ik zie er}{geen een dan zegt hij}\\

\haiku{{\textquoteright} Gestookt onder het:}{gevlei harer stem drongen}{de woorden aan}\\

\haiku{{\textquoteright} {\textquoteleft}Neen, zoo ver kwam het,,.}{niet eens ze wou niet met hem}{uitgaan dat was het}\\

\haiku{{\textquoteleft}Weet je wat, Antoon,{\textquoteright}, {\textquoteleft};}{zei Weelsen toenweet je wat}{je nu eens doen moest}\\

\haiku{Hoe lang was het niet;}{geleden dat hij er het}{laatst op had gespeeld}\\

\haiku{Andere jaren;}{had de boer nog wel eens een}{maat me\^egegeven}\\

\haiku{het houtje belegd,.}{met een hardgruizige laag}{te voorschijn haalde}\\

\haiku{Een felle schichting,:}{lichtte door de oogen van den}{jongen toen hij zei}\\

\haiku{Thijs,{\textquoteright} riep hij hem toe, {\textquoteleft}?}{toen hij onder bereik was}{heb je nog drinken}\\

\haiku{of er hier katten,,.}{hadden liggen stoeien hei}{zoo'n plok van geweld}\\

\haiku{Toen had 't Wimme.}{geleken dat zijn lot zou}{gaan  beteren}\\

\haiku{{\textquoteright}... Tusschen de acht en,?}{negen honderd gulden wil}{je daar goed voor zijn}\\

\haiku{doffer oogelden.}{de bloempjes weg onder de}{vlamming van de zeis}\\

\haiku{Gisteren had ik ',}{een oogenblik dat ikm}{zag maar daarna zag}\\

\haiku{een boterwarmer, '.}{glom aan de andre kant}{evenver vant brood}\\

\haiku{{\textquoteright} {\textquoteleft}Nu, opsnijen kan,,{\textquoteright}, {\textquoteleft}}{je dat moet worden gezegd}{grunnikte Roota we\^er}\\

\haiku{ik ben nog niet au,}{bout de mon latin dat moet}{u toch nog hooren}\\

\haiku{hij bemoeide zich;}{met het modelletje dat}{stijf naar hem opzag}\\

\haiku{je maakt dat ik me}{daar ineens zie zitten op}{het tabouretje}\\

\haiku{Het was nu lang niet;}{meer zoo vinnig en ze had}{al zeven stuivers}\\

\haiku{Het jongetje kwam {\textquoteleft}!}{met zijn vader de winkel}{uit enadie heeren}\\

\haiku{Dan keek zij uit haar.}{krullen op in eens of wou}{ze wat van haar}\\

\haiku{Soms kon ze door het.}{raampje een heer met glimhoed}{rechtop zien zitten}\\

\haiku{de diender klom er,.}{op den drempel en had zijn}{touwen op zijn rug}\\

\haiku{Moeder bleef lang uit;}{en zij had meer verdiend dan}{moeder had verdiend}\\

\haiku{Ze knoopte nu de,;}{doek los van haar lenden ruig}{en wit bespikkeld}\\

\haiku{Ze klopte met de,,.}{doek verscheien malen deed}{hem om toen anders}\\

\haiku{De smeltende hars;}{der vuurmakers smookte uit}{de kachel kieren}\\

\haiku{{\textquoteleft}Wat,{\textquoteright} grauwde we\^er de, {\textquoteleft},.}{vrouw in eensheb jij op je}{kop te krauwen zeg}\\

\haiku{Hij bromde iets van {\textquoteleft}{\textquoteright};}{te laat en werd al drukker}{van bewegingen}\\

\haiku{{\textquoteleft}Je kunt toch nooit eens,{\textquoteright},, {\textquoteleft}'}{praten vond ze terwijl ze}{haar hoed recht zette}\\

\haiku{Dat was je eerst een,.}{toer geweest mannen maken}{zulke groote stappen}\\

\haiku{Onder de rouw van;}{de binnenboog zagen zij}{de opene poort in}\\

\haiku{Had zij wel ooit zoo'n.}{zuivere boog gezien en}{in zoo een hemel}\\

\subsection{Uit: Jaapje}

\haiku{hij droeg een brief in;}{zijn hand en een driekanten}{steek had 			 hij op}\\

\haiku{Jaapje 			 hield zijn.}{hand in zijn zak en klemde}{het harde kluitje}\\

\haiku{als een vlieg was hij,.}{opgegaan koud en licht door}{de 			 warme lucht}\\

\haiku{Jaapje rolde zich.}{om als een hoopje en}{van het licht we\^er af}\\

\haiku{zijn hart trommelde.}{in 			 zijn lijf en toen werd}{het vreeselijk stil}\\

\haiku{Heer, waarom verstoot?}{Gij mijn ziel en verbergt Uw}{aanschijn voor 			 mij}\\

\haiku{{\textquoteright} zei Nico barsch.}{en greep hem vast bij de split}{van zijn borstrok}\\

\haiku{Maar nu dacht hij 		  ,;}{daaraan niet er leefde in}{Jaapje wat anders}\\

\haiku{Hij voelde dat zij,}{heel dicht bij de tobbe kwam}{staan onderwijl}\\

\haiku{{\textquoteright} Jaapje zag uit het;}{bladerige boek felle}{kleurtjes fladderen}\\

\haiku{Hij keek oplettend,.}{naar de mannekoppen}{we\^erszijds de kapel}\\

\haiku{{\textquoteleft}Zoo, kom je maar es,{\textquoteright},.}{kijken bromde Rijs of kwam}{het uit de 		  grond}\\

\haiku{Jaapje had armen;}{zien liggen van buizen en}{pijpen van broeken}\\

\haiku{die niet door 'n goeie,.}{bril kijkt kijkt licht door de bril}{van een 		  ander}\\

\haiku{De meisjes zaten,.}{op de voorste 		  bank de}{jongens achteraan}\\

\haiku{in 't midden was,.}{een stralende zon daar}{glom een nummer in}\\

\haiku{{\textquoteleft}Wanneer ik groot ben,{\textquoteright}, {\textquoteleft}.}{troostte hij zich zelfword ik}{ook 		  generaal}\\

\haiku{{\textquoteleft}Net zoo blauw als de,{\textquoteright}, {\textquoteleft}.}{rest zei Doorik geloof het}{ten minste 		  wel}\\

\haiku{{\textquoteleft}Groomoe,{\textquoteright} zei Jaapje, {\textquoteleft}?}{danmag ik Door tot voor de}{poort 		  wegbrengen}\\

\haiku{wij moeten altijd,'.}{wederstaan den 		  Booze t}{aller uur en tijd}\\

\haiku{de ziel heeft het 		  ,.}{altijd warm ze is van de}{duvel bezeten}\\

\haiku{Laat hij mij maar voor,;}{de heeren roepen ik}{zal mijn woord wel doen}\\

\haiku{Dan zette groomoe.}{haar bril op en dan zat ie}{er altijd 		  in}\\

\haiku{stak zij haar hand door.}{de leuning van haar stoel en}{hield de zijne vast}\\

\haiku{Jaapje hoorde hem.}{schuifelen over de 		  sneeuw}{en eventjes hoesten}\\

\haiku{{\textquoteright} {\textquoteleft}Da\`arvoor heb ik het,{\textquoteright}.}{niet gekregen van Rudolf}{antwoordde Jaapje}\\

\haiku{hij groeide veel te,,;}{hard 		  zei moeder er was}{geen bijhouen aan}\\

\haiku{hij werd uit de bank.}{geroepen en moest in}{de eetzaal komen}\\

\haiku{Ze zaten altijd.}{aan je lijf te wriemelen}{als het niet noodig was}\\

\haiku{Jaapje snufte en.}{deed zijn boezelaar van}{achteren zelf vast}\\

\haiku{{\textquoteleft}vraagt u maar gerust,{\textquoteright}.}{of ik wat krijg maar 		  bleef}{heel stijf staan kijken}\\

\haiku{Ze gingen er bij.}{zitten 		  en spalkten hun}{oogen vastberaden}\\

\haiku{ze was we\^er op het, {\textquoteleft},}{hoopje zand gaan kloppen}{de rijstebrijberg}\\

\haiku{Onmiddellijk was;}{Jaapje opgekomen en}{liep hard naar de pomp}\\

\haiku{Nico was de poort, {\textquoteleft}{\textquoteright}.}{gauw ingegaan omdat hij}{koetsier wou worden}\\

\haiku{uit 		  de hand te,.}{eten een sneedje grof en een}{sneedje roggebrood}\\

\haiku{{\textquoteleft}Zal je eens goed je,{\textquoteright}, {\textquoteleft}.}{oogen uitwasschen zei Doorze}{zijn we\^er erg 		  zwart}\\

\haiku{je 		  wordt al te,.}{groot voor een talhout maar je}{verdiende het wel}\\

\haiku{Jaapje pinkte en.}{lachte vergevingsgezind}{naar de kant van Koos}\\

\haiku{{\textquoteleft}Ziet onze lieve?}{Heer dan door de jassen van}{de soldaten heen}\\

\haiku{Je grootvader en {\textquotedblleft}{\textquotedblright}.}{ik waren getrouwd onder}{delamme koning}\\

\haiku{toen klikte we\^er de,.}{deurtjes buiten 		  dicht de}{eene na den ander}\\

\haiku{er lag een knoop in,}{met een wilde zwijnekop}{er op zijn eenig}\\

\haiku{Deijlius was,.}{zelf   gekomen had een}{rond 		  hoedje op}\\

\haiku{en ten leste had.}{Kareltje ook een nieuwe}{pet 		  gekregen}\\

\haiku{{\textquoteright} Jaapje kreeg een 		  ,.}{bangig wezen daar had hij}{nog nooit aan gedacht}\\

\haiku{Je hoorde almaar:}{loopen boven je hoofd en}{toen Koos had gezegd}\\

\haiku{de hoek van inval;}{is gelijk aan den hoek}{van terugkaatsing}\\

\haiku{Ze hadden w\^eer,:}{een poosje zitten visschen}{toen de meester zei}\\

\haiku{Wat kiest gij schuw uw,:}{pad  of van de boer die}{tot zijn paard zei}\\

\haiku{Doch nu het kermis,.}{was geworden las hij niet}{meer in zijn prijsje}\\

\haiku{Er waren altijd {\textquoteleft}{\textquoteright}.}{krieken en 		  peertjes ook}{met rooie wangetjes}\\

\haiku{Er stond niet erg veel,;}{in omdat het zulke groote}{drukletters had}\\

\haiku{Hij hield zijn adem in,,}{en fronsde er schuin tegen}{aan hij zag het goed}\\

\haiku{Telkens was het of,.}{hij op wou zitten 		  dan}{viel hij slap we\^er ne\^er}\\

\haiku{Er liep een vrouw met;}{wafels op een 		  blaadje}{tusschen de banken}\\

\haiku{hij nam het van zijn.}{pruikebol en zwaaide}{naar al de menschen}\\

\haiku{{\textquoteright} vroeg de loodgieter,.}{kijkend 		  naar twee schrammen}{op Jaapjes gezicht}\\

\haiku{Wanneer wij hem straks,}{zullen toevertrouwen aan}{den schoot der aarde}\\

\haiku{{\textquoteright} had tante gevraagd, {\textquoteleft},}{hoe heb je het gewaagd we}{houen het nooit}\\

\haiku{voor de ruitjes 		  ,.}{waren hekjes soms dat ze}{niet zouen breken}\\

\haiku{Toen zag hij in een}{groote glazen kast een heer die}{zat te schrijven}\\

\haiku{{\textquoteright} zei de jongen {\textquoteleft}as,;}{je dat niet heb gezien heb}{je ook niks gezien}\\

\subsection{Uit: Jaap}

\haiku{man die de {\textquoteleft}platen{\textquoteright}.}{had in zijn handen als de}{moeder een schoon hemd}\\

\haiku{{\textquoteright} {\textquoteleft}Het nieuwe is hem,{\textquoteright}}{te machtig zei groomoe en}{le{\^\i} haar hand op}\\

\haiku{En zoo ging het om,.}{en om van de eene dag in}{de 			 andere}\\

\haiku{{\textquoteleft}Nou, domkop dan,{\textquoteright} riep, {\textquoteleft},{\textquoteright} {\textquoteleft}}{Dolfhou maar je groote snavel}{we kommen 			 al.}\\

\haiku{Door gaf een zoen aan.}{groomoe en aan Koos en toen}{er een aan 			 hem}\\

\haiku{{\textquoteright} {\textquoteleft}Met alle soorten,{\textquoteright}.}{van genoegen zei Koos en}{hupte op de been}\\

\haiku{Hendrik zat 			 Jaap.}{recht aan te kijken met zijn}{bruin-blauwe oogen}\\

\haiku{{\textquoteright} {\textquoteleft}Groomoe,{\textquoteright} liet Jaap we\^er, {\textquoteleft}?}{hoorenwanneer krijgt Door nou}{een 			 koppie thee}\\

\haiku{Groomoe le{\^\i} haar hand.}{naar Jaap's blauwe 			 arm en}{liet die daar rusten}\\

\haiku{{\textquoteright} antwoordde Jaap en.}{leunde zich 			 meteen los}{tegen zijn stoel aan}\\

\haiku{{\textquoteleft}Ze zijn op 't laatst,}{niet meer te tellen al die}{tikken op je}\\

\haiku{Piet Pollee raapte.}{een driekant brokje op}{en ging er van eten}\\

\haiku{{\textquoteright} vroeg hij, {\textquoteleft}dat is mijn,;}{bibliotheek dat bennen}{mijn klassieken}\\

\haiku{de 			 linkerhoek.}{van zijn mond gaan lachen en}{toen de rechterhoek}\\

\haiku{ik zou haar gauw haar,{\textquoteright}.}{luier geven raadde}{groomoe zacht naar Door}\\

\haiku{{\textquoteleft}Maak voort,{\textquoteright} bromde Jaap.}{naar het paard en loende schuin}{en naar 			 boven}\\

\haiku{{\textquoteleft}Kijk es aan,{\textquoteright} zei van, {\textquoteleft},;}{Essennou zie dan of ik}{geen gelijk heb}\\

\haiku{De tree\"en en de.}{slagijzers waren gedofzwart}{en de lantarens}\\

\haiku{{\textquoteleft}'t Is Woensdag, je!}{moet vanmiddag naar de}{catechesatie}\\

\haiku{die van de jonge.}{baas was wat dikker gedeukt}{van 			 onderen}\\

\haiku{{\textquoteleft}Ik zal het zeggen,{\textquoteright}, {\textquoteleft}.}{zei Koosmet een le\^ertje om}{aan te 			 dragen}\\

\haiku{stijf op 			 een leeuw.}{en met je voeten in de}{beugels als je kon}\\

\haiku{De palen staken,.}{uit de keien op er lei}{maar \'een 			 kei bij}\\

\haiku{Koenraad zal 			 ze;}{dan na schafttijd voor me in}{de stopverf zetten}\\

\haiku{De looper was een,;}{hompige kei aan een kant}{glad geslepen}\\

\haiku{rol dan eerst 			 de,.}{stopverf door de rauwe olie}{dan houdt ze beter}\\

\haiku{'t Was toch wel een,;}{mooi vak 			 schilderen en}{glazen-maken}\\

\haiku{{\textquoteleft}Gauw,{\textquoteright} zei de witte, {\textquoteleft}, '.}{ventbezorg hem heett is}{meer dan je 			 weet}\\

\haiku{Het was 			 Psalm 1, '.}{waarvan hijt eerste vers}{van 			 buiten kon}\\

\haiku{Jaap was soms benauwd,.}{door het 			 orgel om wat}{er uit kon komen}\\

\haiku{zijn kwasten spoelen.}{in terpentijn tot er niets}{geen olie meer in was}\\

\haiku{Toen bromden ze vlak,.}{naast elka\^ar alsof ze}{oneenigheid hadden}\\

\haiku{hij deed het nochtans,.}{naarstig mede tot 			 er}{de rook van af vloog}\\

\haiku{Ze hadden Jaap eens.}{opgetild en toen had hij}{het ook 			 gezien}\\

\haiku{Bertus was heel kort.}{geknipt en dikker door de}{lucht 			 geworden}\\

\haiku{zoo schoten zij het.}{jaar uit en alles wat er}{in was 			 gebeurd}\\

\haiku{maar Jaap was bij zijn.}{kopje blijven zitten en}{had geschud van neen}\\

\haiku{Alles wat hij meer:}{nog leerde liet Jaap als}{achter in de school}\\

\haiku{achter hem sprak een.}{schorre stem en siste wat}{over 			 zijn schouder}\\

\haiku{of allemaal in, {\textquoteleft}{\textquoteright}.}{sn\'el gevolg na een wat de}{kaskade heette}\\

\haiku{wanneer de lommerds.}{maar eens spreken konden en}{als het dan woei}\\

\haiku{Gezichten bleven,?}{goor en handen want waarin}{moest je je wasschen}\\

\haiku{Vrijdags hooren kwam.}{of er niet een paar voor}{Zondag waren noodig}\\

\haiku{al reed je op 			 {\textquoteleft}{\textquoteright}.}{kunstschaatsen allicht mooier}{op de korte baan}\\

\haiku{ze sloeg haar schaatsen.}{dadelijk uit 			 en reed}{terug naar Gerard}\\

\subsection{Uit: Jacob}

\haiku{Zijn vader werkte,;}{aan de spoor daar konden zij}{hem niet gebruiken}\\

\haiku{Hij was toen erg 			 , ....}{verschrokken maar je wou toch}{ook wel eens spelen}\\

\haiku{{\textquoteleft}hij h\'et hem 			 we\^er{\textquoteright}.}{en wist dan dat pikeur de}{Boer de winner was}\\

\haiku{Oom keek meestal;}{blijmoedig en soms met open}{lippen naar zijn zoon}\\

\haiku{Meinardus liep almaar,;}{vroolijk te praten wist veel}{te vertellen}\\

\haiku{tante sprak altijd.}{of ze de tijd 			 had en}{lachte onverwacht}\\

\haiku{Af en toe 			 drong;}{het laag geratel van een}{rijtuig naar boven}\\

\haiku{Dan hoorde je hun,.}{lachen door het huis soms}{tot in de keuken}\\

\haiku{Koos was er ook, had,.}{kringen onder haar oogen was}{erg nieuwsgierig}\\

\haiku{{\textquoteright} Jakob voelde dat,.}{er toch niets aan te doen was}{hij hield zijn mond}\\

\haiku{Nu 			 hij niet al, '}{de lessen aan de avondschool}{meer volgde zat hij}\\

\haiku{Jaantje was zoo kwiek,.}{als een jongen zou er om}{willen vechten}\\

\haiku{Dat weten we wel,{\textquoteright}, {\textquoteleft},.}{praatte Doorkom drink nou je}{kopje eerst 			 le\^eg}\\

\haiku{Koos droogde haar neus,.}{nam we\^er haar kopje 			 bij}{het schoteltje op}\\

\haiku{Ze waren voor het.}{hek en in de schijn 			 van}{de tuinlantaren}\\

\haiku{hij lichtte zijn pet,.}{krabde even zijn voorhoofd}{en spoelde nog eens}\\

\haiku{{\textquoteright} stemde Jakob grif.}{we\^er toe en met den platsten}{toon van de 			 straat}\\

\haiku{Het feest van meester;}{Boudewijnse was reeds een}{poosje geleden}\\

\haiku{Ik heb mijn bril niet,{\textquoteright}, {\textquoteleft};}{bij me zei de baaszing het}{me maar es 			 voor}\\

\haiku{{\textquoteleft}ik zelf speelde niet,.}{l\'ang met 			 poppen ging al}{gauw in betrekking}\\

\haiku{Zoo zijn we,{\textquoteright} had Door,, {\textquoteleft}}{toen geantwoord schenkend de}{kopjes we\^er vol}\\

\haiku{Geesteraag, hij en;}{ik zijn 			 ongeveer van}{dezelfde leeftijd}\\

\haiku{Het stelde een heer,,.}{en een dame voor 			 pas}{getrouwd dat zag je}\\

\haiku{{\textquoteright} De baas voerde elk,.}{we\^er mede nu zou het pas}{goed beginnen}\\

\haiku{Al liep iemand weg,.}{moest je nog niet denken dat}{hij je gelijk gaf}\\

\haiku{, zit goed vast, 			 maar.}{ik zou toch niet graag een paard}{van u willen zijn}\\

\haiku{Hij nam zijn zakdoek.}{uit zijn zak en wreef 			 zijn}{hand terdege schoon}\\

\haiku{Gras was er niet en,.}{geen mos maar wel sprieten scherp}{voor het gevoel}\\

\haiku{wolken maken 			 .}{het duin tot golvend zand of}{tot \'e\'ene groote kuil}\\

\haiku{Soms vloog er over je,.}{hoofd een vogel dat wel een}{zeearend zijn 			 kon}\\

\haiku{het genie is een, {\textquoteleft}{\textquoteright}.}{lang geduld stond inHelp u}{zelf te lezen}\\

\haiku{De wind en de zee;}{moest 			 je uit elkander}{weten te houden}\\

\haiku{het 			 zal zooveel.}{niet verschillen en kan toch}{evengoed zoo bestaan}\\

\haiku{{\textquoteright}  De komiek zong.}{het eene Parijsche liedje}{na het andere}\\

\haiku{Hij was toen door de.}{stilste straatjes gegaan naar}{het spel van 			 Basch}\\

\haiku{dikwijls als je keek,.}{bleek het alwe\^er anders dan}{de vorige maal}\\

\haiku{Jakob dacht aan het:}{Wilhelmus en aan de}{Geuzenliederen}\\

\haiku{Tinus kwam er nooit,.}{omdat hij te ver van de}{winkel woonde}\\

\haiku{K. Kommer eischt,.}{geduld en moed L. Luiheid}{leidt tot tegenspoed}\\

\haiku{Q. Quasi-deugd,.}{is slijm en slijk R. Rijkdom}{maakt niet altoos rijk}\\

\haiku{We zijn ten slotte,{\textquoteright}.}{allemaal weezen zei Baas}{met zijn oogen 			 ne\^er}\\

\haiku{Een jongen bracht 't.}{karbiesje we\^erom Dat zij}{verloren had}\\

\subsection{Uit: Nieuw proza}

\haiku{{\textquoteleft}Het wordt tijd dat ik,{\textquoteright}.}{mijn conterfeitsel maken}{laat dacht hij weder}\\

\haiku{{\textquoteright} {\textquoteleft}Niet denken doet men,{\textquoteright}.}{altijd genoeg meende we\^er}{zoetjes Ambroise}\\

\haiku{kom laten wij het,!}{nog eens samen hebben nog}{eens een duetje}\\

\haiku{het tintelde hem,.}{tegen het kleurde er als}{in geslepen glas}\\

\haiku{Het was maar goed dat.}{mijnheer van Oudentijd niets}{van dit alles zag}\\

\haiku{{\textquoteright} {\textquoteleft}Je oogen bennen goed,{\textquoteright}, {\textquoteleft}.}{oordeelde Marretjeen}{ik weet je drinkt niet}\\

\haiku{het stond er toch reeds,.}{in het rijtje huizen toen}{ik nog heel jong was}\\

\haiku{Het varen ging nooit,,.}{recht altijd in zwenkende}{baan het water trok}\\

\haiku{{\textquoteright} scherp keek ze naar over,;}{het bermpje naar het grotje}{waar ze had gestrooid}\\

\haiku{ze liet haar boekje.}{zinken en plaatste haar voeten}{ordelijk nevenseen}\\

\haiku{E. daarentegen.}{zie ik altijd dadelijk}{als in de vlakte}\\

\haiku{Hij had in Chigi {\textquoteleft}{\textquoteright}.}{het meeste te zeggen en}{stelde nogal eens}\\

\haiku{Ik weet niet juist meer,.}{wat ik neuriede het was}{iets van Beethoven}\\

\haiku{Mei-regen maakt,,}{dat ik grooter word Grooter}{word Ender dat wensch}\\

\haiku{achter hem stond de.}{geest van Zorolla in zijn}{geruite pakje}\\

\haiku{Niet alleen zijn vrouw,;}{aanbad hem ieder die hem}{kende hield van hem}\\

\haiku{Het was een gulle,,;}{ouwerwetsche zeeman een}{aangespoeld matroos}\\

\haiku{{\textquoteright} vroeg de kerel, {\textquoteleft}naar,?}{het wijfje kijken dat zoo}{aorig lullen kan}\\

\haiku{Hij was toen we\^er gaan,.}{dwalen hij had zijn messen}{in zijn pelerien}\\

\haiku{er schoven telkens;}{levende tronies achter}{elkander voorbij}\\

\haiku{De knaap rolt om op,....}{het voetpad de agent liet zijn}{blik op hem wegen}\\

\haiku{Wanneer zal eens een?....}{alliance hollandaise}{worden opgericht}\\

\haiku{{\textquoteright} hielp mede zijn vrouw, {\textquoteleft}.}{herinnerenlag alles}{wijd onder de sneeuw}\\

\haiku{hij zal wel we\^er zijn,,.}{beste beentje v\'o\'or zetten}{weet ik mijn broeder}\\

\haiku{{\textquoteright} {\textquoteleft}Jullie moesten hem over,{\textquoteright}.}{zien te houden vond tante}{Van Hoevelaken}\\

\haiku{ieder hield zich bij.}{zijn bordje of onderhield}{zich met zijn naaste}\\

\haiku{De doeken daar dit, '.}{kind in leit Ist purper}{van zijn majesteit}\\

\haiku{Van Hoevelaken;}{zat met een der hulstblaadjes}{voor hem te spelen}\\

\haiku{{\textquoteright} liet ingehouden.}{straf oom Vervieren over heel}{de tafel hooren}\\

\haiku{{\textquoteleft}En een enkel geel,{\textquoteright}.}{gordijntje vulde mevrouw}{Stavoren aan}\\

\haiku{het is mij niet recht.}{duidelijk waar Maria met}{het kindje op rust}\\

\haiku{{\textquoteright} Vervieren richtte.}{zich achterover en pufte}{zijn rook naar boven}\\

\haiku{Mevrouw Vervieren.}{lachte het heele klavier}{van haar tanden bloot}\\

\haiku{wees er van zeker,....}{Rembrandt zou er ook niet om}{gelachen hebben}\\

\haiku{hij leek we\^er in zijn,;}{zwak getast de voorliefde}{voor de novelle}\\

\haiku{Tante Agnes kneep,.}{haar eene oog deed alsof ze}{zoog op iets heel fijns}\\

\haiku{{\textquoteleft}En u krijgt een warm,{\textquoteright}.}{kruikje vleide Dora aan}{haar andere zij}\\

\haiku{{\textquoteleft}Wat een gezellig,{\textquoteright},.}{voorwerp zei die kijkend scherp}{naar den kandelaar}\\

\haiku{Zijn uitzicht was dan,.}{streng en week hij leefde er}{geheel in mede}\\

\haiku{hebt Gij U eere,....}{ontzegd Werdt Gij in stroo en}{in doeken gelegd}\\

\haiku{Landoue zat stil te.}{glunderen en grunnikte}{naar het schermutsel}\\

\subsection{Uit: Proza}

\haiku{Maar op het plein San.}{Marco vierde de sneeuw feest}{in de groote stilte}\\

\haiku{Van toen aan was het.}{plein niet langer verlaten}{en ongeschonden}\\

\haiku{Hij sloeg zich met de.}{handen tegen de schouders}{om warm te worden}\\

\haiku{- {\textquoteleft}Zit stil, ni\~na,{\textquoteright} zei,.}{de schilder zenuwachtig}{haastig arbeidend}\\

\haiku{- {\textquoteleft}Dood, heertje,{\textquoteright} zei ze, {\textquoteleft}.}{verleden jaar gestorven}{aan de cholera}\\

\haiku{- {\textquoteleft}Es tonto,{\textquoteright} zei de, {\textquoteleft},,.}{deernehet doet hem geen pijn}{heertje es tonto}\\

\haiku{{\textquoteright} zei de jonge man.}{tot zijn onbewegelijk}{rookenden buurman}\\

\haiku{Beelden gehaald van,.}{ver uit verre jaren en}{uit verre streken}\\

\haiku{{\textquotedblright} en staande, met \'een,,.}{teug dronk hij het glas le\^eg dat}{de vrouw hem inschonk}\\

\haiku{Na een klein poosje.}{kwam de stem van baas van D.}{achter zijn rug om}\\

\haiku{het silhouetje.}{van het buitenhek stond er}{donker tegen uit}\\

\haiku{Dat waren altijd.}{oogenblikken geweest van}{bizonder genot}\\

\haiku{want laag achter hem,.}{rees de klare maan zuiver}{in den nacht stijgend}\\

\haiku{In het smettelooze;}{maanlicht was alles z\'o\'o ijl}{en onstoffelijk}\\

\haiku{{\textquoteright} - {\textquoteleft}En de begeerte,{\textquoteright}.}{is eeuwig prevelde de}{afgekeerde bloem}\\

\haiku{en daar flikkerde;}{het lichtende karkas van}{een gebouw omhoog}\\

\haiku{Als een straatvlam een,}{rij in het gezicht sloeg kon}{hij zien hoe verlept}\\

\haiku{En nu vol van drank,.}{en pret klotsten en dansten}{ze hier den nacht uit}\\

\haiku{Ja, dat was groot, dat,.}{was groot dat stond vast in den}{dag en in den nacht}\\

\haiku{van dien dwaas die meer.}{wijsheid spreekt dan tien wijzen}{van den kouwen grond}\\

\haiku{De dorre adem die,.}{uit de vlakte zwoegde hing}{in de houten kast}\\

\haiku{De felle zon sloeg.}{van het land op en brandde}{ne\^er op den wagen}\\

\haiku{'t was als de rest, '.}{t zocht zijn pleizier en haar}{eigen goed leven}\\

\haiku{Dood, dood, gevoelloos,....}{voor koud en naar weder en}{voor mijn roepen doof}\\

\haiku{{\textquoteright} Zijn hals en krop was ' ';}{ondert vertellen aan}{t zwellen gegaan}\\

\haiku{en toen ben ik naar........}{zijn bed gegaan bij ons in}{de wagen ziet u}\\

\haiku{met schrik om het hart,.}{met kloppende keel stond ik}{het aan te staren}\\

\haiku{Had ik niet een klein?}{donker geslinger gezien}{naar het open buiten}\\

\haiku{En even daarna kwam:}{de stem der padrona hun}{weggaan nagalmen}\\

\haiku{Hij vond dezen met,.}{zijn rug tegen de tafel}{zeker in zijn hoek}\\

\haiku{En de comedor;}{was geheel stil geworden}{van menschengepraat}\\

\haiku{Hij schuurde met zijn.}{hoofd langs den muur om een goed}{plaatsje te vinden}\\

\haiku{laat, zich mooi maakt voor,.}{haar liefste voor hem haar schoon}{vermenigvuldigt}\\

\haiku{De snoer was maar half,}{afgewonden de dobber}{schommelde dichter}\\

\haiku{Onder geruisch.}{zit zij van serafijnen}{die haar overkroonen}\\

\haiku{vlak, vermindert in.}{het minst niet de kloekheid van}{het volgroeide blad}\\

\haiku{Die zon en zomer, ':}{te beminnen leeren ent}{treuren gehad}\\

\haiku{Er bewoog zich in.}{alle opzichten veel jongs}{om dat jonge boek}\\

\haiku{Is dit zich als even '?}{verraden een schade voor}{t boek-geheel}\\

\subsection{Uit: Reizen}

\haiku{Kom, heer gemaal, al.}{deze dingen moeten niet}{dus worden overpeinsd}\\

\haiku{De zaak is, waarde,{\textquoteright}, {\textquoteleft}}{compeer meesmuilde hijdat}{ge er zelf razend}\\

\haiku{als neef-lief laat.}{voor de Bar blijft plakken met}{Tanger's fine fleur}\\

\haiku{Chopin: speelde....}{hij en fantaseerde dat}{het liep als water}\\

\haiku{honderdmaal liep ik.}{dit blanke straatje en dat}{ik dit pas oplet}\\

\haiku{{\textquoteright} Toen voor een groote plaat,:}{bleef hij aandachtig staan het}{onderschrift beturend}\\

\haiku{we gaan hetzelfde,, {\textquoteleft}{\textquoteright}.}{weggetje waarin de stoet}{verdwijnt inGekken}\\

\haiku{Hij is toch mooi wit,.}{ze heeft zoo'n aardig kuifje}{en zulke goeie oogen}\\

\haiku{Naar Tetuaan werd '...}{het wel twaalf uur enk was}{nog heelemaal frisch}\\

\haiku{ik geloof dat ik...}{nog liever op een pak zit}{dan in een boxsaddle}\\

\haiku{Rijden was heerlijk,....}{je werdt een sterker man daar}{bovenop zoo'n beest}\\

\haiku{{\textquoteleft}Hoe was ook we\^er de,{\textquoteright}...}{naam van die rivier zocht hij}{in zijn gedachten}\\

\haiku{die intonatie,{\textquoteright},.}{sprak Emilia beduusd toen zij}{waren in hun tent}\\

\haiku{Hasj stil en waardig,.}{reikte nog wat eieren}{warm uit den ketel}\\

\haiku{het is hetzelfde.}{gevoel dat ik als jongen}{op een schommel kreeg}\\

\haiku{plotseling brandde,....}{het zonvuur raasde over den}{bodem en verzwond}\\

\haiku{zij trokken onder....}{de groote triomfpoort door en}{waren toen thuis}\\

\haiku{{\textquoteright} {\textquoteleft}Dat is dus het eind,.}{van het complot als onze}{vriendin het noemde}\\

\haiku{zoo was het zeker.}{geworden dat zij Laraisj}{zouden bezoeken}\\

\haiku{op elke wang en.}{op het voorhoofd iets als een}{sterretje droegen}\\

\haiku{{\textquoteleft}Lekoes, Lekoes,{\textquoteright} klapte er,.}{de naam van uit het af en}{aan loopende volk}\\

\haiku{De Lekoes rolde zijn.}{gele golven te midden}{van zanderijen}\\

\haiku{ze we\^er plannen en,}{keek ze maar gewoon naar haar}{minnaar misschien ook}\\

\haiku{{\textquoteright} Om den heuvelzoom.}{fladderden vale vlinders}{en zetten zich ne\^er}\\

\haiku{je wordt er dronken,{\textquoteright};}{van riep Emilia uit en ze}{bukte en bukte}\\

\haiku{{\textquoteright} Koud tinkelden reeds,:}{sterren hij hoorde verdwaasd}{het schreeuwen van Hasj}\\

\haiku{koppen van kerels,,.}{gehurkt op den vloer gloeiden}{boven den drempel}\\

\haiku{{\textquoteright} Emilia hield een nieuw.}{schrijfboek op haar schoot en was}{druk met Mohammed}\\

\haiku{Maar door den lagen.}{stand der zon gelukte het}{nu niet zoo spoedig}\\

\haiku{{\textquoteright} lachte gelukkig,.}{zijn ellendig hoofd toen hij}{den zonprik voelde}\\

\haiku{Hij is zeker eens,{\textquoteright}.}{gebeten op het Zocco}{meende Theobald}\\

\haiku{Voor de drassige;}{sleuring daar hield nu ook de}{andere stoet halt}\\

\haiku{Vlakbij hoorden zij;}{de bange stemmetjes van}{de moorsche vrouwen}\\

\haiku{schrik maakt blind als toorn,,...}{de oogen gaan inwaarts zien iets}{anders dan er is}\\

\haiku{Evangeline wou}{rusten en in eens gaf toen}{Roosevelt bevel}\\

\haiku{De drijver stampte.}{zijn voet tegen den grond om}{te zien of het hield}\\

\haiku{een  reikte een '.}{sachet tot aan zijn neus en}{snooft parfum}\\

\haiku{zijn voorhoofd echter;}{was veel nietiger en zijn}{uitkijk blauw en struw}\\

\haiku{af en toe klaagt de.}{roep van een blindeman door}{de rulle ruimte}\\

\haiku{Een van de jonge,{\textquoteright}.}{generatie lachte Miel}{toen hij afscheid nam}\\

\haiku{Zeg aan uw meester,{\textquoteright}:}{antwoordde mijn gezel op}{we\^er zulk een boodschap}\\

\haiku{Er is van Fez nog,, {\textquoteleft}{\textquoteright}.}{niets gemaakt zei de gezant}{mij hoorend dat ikschreef}\\

\haiku{bang en siddrend beestje, ', ..............}{O watn verbijstering}{in je borstje}\\

\subsection{Uit: De wonderlijke avonturen van Zebedeus}

\haiku{De reisverhalen,;}{die wij u hebben te doen}{zijn de peine waard}\\

\haiku{de zwaarste spijzen,,.}{de vruchten onzer wijsheid}{ze verteren licht}\\

\haiku{de vogel zal zich,,}{nimmermeer verblijen Mijn}{kooi wordt le\^eg van zang}\\

\haiku{(betracht zich aldus)}{andermaal maar nu naar de}{andere zijde}\\

\haiku{voor je hoofd de proef.}{op de som en een gloedje}{onder de borstkuil}\\

\haiku{Wat moet ik dan doen?}{om je we\^er te brengen in}{je knolletuin}\\

\haiku{(houdt de handen nog.)}{wat vooruit gestoken en}{haalt ze dan snel in}\\

\haiku{geen eerlijk middel.}{te klein  om te komen}{tot een groot einde}\\

\haiku{jammervollen  ,;}{zij wier lippen pas lachen}{in den vasten dood}\\

\haiku{Vier, vijf, achter de....}{zevende daar groeien de}{dikste bramen}\\

\haiku{dit is het ware,}{reuzen-voedsel wat ik}{zeer wel ken laat ik}\\

\haiku{droomen, nat, voedsel,.}{en zonnelicht de aarde}{heeft het overvloedig}\\

\haiku{- Het is de rechte,,,,?}{tijd zei daarop kortaf de}{eerste dichter w\`aar}\\

\haiku{Koekoek, koekoeke,...}{Liefde-gelok maar geen}{pake en moeke}\\

\haiku{sommigen noemen...}{het regen en anderen}{noemen het tranen}\\

\haiku{ik bewonder, ik,,.}{bewonder vergeef wend uw}{aangezicht niet af}\\

\haiku{le kerk Vindt je hier...,......}{werk Hemeltjes-blauw}{in het stroo O O}\\

\haiku{Veel vragen zijn even}{jong als vele vraagsters en}{vele antwoorden}\\

\haiku{ben ik toch niet, dat,...}{er de grond van wankt dat er}{de grond van davert}\\

\haiku{Hoor, hoor, fluisterde,.}{Zebedeus ze hebben}{het over de grasjes}\\

\haiku{hij keek, of keek hij.}{naar de achterdeelen van}{een denkbeeldig span}\\

\haiku{En zijn ze, na een,, '?}{tijdje zoo gedroogd dan gaat}{ge ist niet zoo}\\

\haiku{dat is omdat het,,...}{kil is omdat het kil is}{huiverde de reus}\\

\haiku{Velen uwer was het,,}{toen niet duidelijk wat dit}{zeggen wou en wij}\\

\haiku{Wat is uw meening?}{daaromtrent en welke houdt}{gij voor de beste}\\

\haiku{Zoek maar niet langer,,.}{viel Ruigrok in de rede}{zij leit aan den weg}\\

\haiku{Hij tuurde naar het;}{slijpende geglibber en}{wist niet waar te zien}\\

\haiku{wat jong is, verheft,,;}{zich gereede wat oud wordt}{allicht zich verdiept}\\

\haiku{De grootste schrijder}{onder ons doet ten leste}{uit geen anderen}\\

\haiku{En rond den pegel.}{Onder den regel Schimmert}{het rood en blauw}\\

\haiku{Eensklaps de pegel.}{Onder den regel Stort naar}{bene\^e met een krak}\\

\haiku{{\textquoteright} - {\textquoteleft}Uit ouwe kennis,{\textquoteright},.}{zei ze naar Zebedeus}{toen de dienaar ging}\\

\haiku{{\textquoteright} had zich de stem van ', {\textquoteleft}}{den gastheer gemengd int}{gesprekhoe hooger}\\

\haiku{{\textquoteright} niesde plotseling, {\textquoteleft}.}{Zebedeusik ben het}{geheel met u eens}\\

\haiku{Maar deze w\`as van,.}{de zee de vertrouweling}{der groote elementen}\\

\haiku{Door het raam verscheen,.}{nog het park in den eindloozen}{droevigen schemer}\\

\haiku{{\textquoteright} Tourniput tilde.}{zijn knie\"en tot zijn kin en}{schaterde het uit}\\

\haiku{Ik geloof dat 'k,,...?}{even maar Zou kunnen slapen}{nu Vindt je het naar}\\

\haiku{Er wies geen bloem, er;}{groeiden stakkelstruiken Ter}{we\^erszij van den weg}\\

\haiku{zoo vertrouwelijk,}{meestal onderhield hij}{ons dan uit den schat}\\

\subsection{Uit: De wonderlijke avonturen van Zebedeus}

\haiku{Wanneer hij spot als,}{nu en ongevoelig zich}{toont verbeurt hij juist}\\

\haiku{Wanneer Narango's,?}{gordel is geslonken naar}{wien verlangt hij dan}\\

\haiku{hij was, dat abele;}{lieden Afstandelijk hem}{groeten met ontzag}\\

\haiku{Toen werd zijn gansche,;}{leven De gansche wereld}{werd die maagd voor hem}\\

\haiku{) Hoor, ze dwarrelen.}{als kapelletjes tusschen}{de stammen telkens}\\

\haiku{Aardige onzin.}{verbergt vaak meer klaarte dan}{onaardige zin}\\

\haiku{Als een plaatse der.}{zoete bijeenkomst staat mijn}{hart altijd open}\\

\haiku{een hinde is toch,.}{geen wolf geen wezen dat haat}{inboezemt of angst}\\

\haiku{Die zijn geweer soms {\textquoteleft}{\textquoteright},.}{hem een zorg Bij moeder de}{vrouw in bed verborg}\\

\haiku{{\textquoteleft}Maak me zoo'n ding,{\textquoteright} zei,, ';}{Jan heel straf Wanneer het deugt}{koop ik jet af}\\

\haiku{Konden ter kerk zij,...}{Zondag rijden Net zoo goed}{als de grootste hans}\\

\haiku{Hij trok zijn geldbeurs, '.}{naar het licht Een mollevel}{metn veter dicht}\\

\haiku{Zij had haar jak nog ',.}{onderr rok En trijpten}{slof om wollen sok}\\

\haiku{{\textquoteleft}Als of 't gesmeerd,{\textquoteright},.}{was meende Bram En daarop}{staan moest een oorlam}\\

\haiku{'t Lemoen lag met;}{zijn spoorstok vast En aan de}{haken wel gepast}\\

\haiku{Af van de deur en '.}{kwam al gauw Schuddend de bel}{aann eindje touw}\\

\haiku{{\textquoteleft}Nee, die is goed,{\textquoteright} riep, {\textquoteleft},!}{moeke LijsHeere me God}{de knul is niet wijs}\\

\haiku{Een meisje achter, {\textquoteleft}{\textquoteright}.}{een spiegelglas Zei dat het}{de arke Noach's was}\\

\haiku{En toen de zon naar ',:}{t Westen zonk In roode}{schijnsels alles blonk}\\

\haiku{'t Kwam wel terecht,,.}{Floor schonk op pof En morgen}{rekenden zij of}\\

\haiku{Geen kunst heeft het nog;}{kunnen stellen buiten wat}{men noemt de natuur}\\

\haiku{Men heette hem een,:}{aristocraat dan lachte hij}{weemoedig en zei}\\

\haiku{Daar wordt gescheld en.}{een jonkman komt op om zijn}{dienst aan te bieden}\\

\haiku{Ik geef het U, zooals:}{ik het heb gevonden in}{mijn ooms papieren}\\

\haiku{Vertel me wat je,,;}{wilt praat mij over je meester}{praat mij over je zelf}\\

\haiku{Wat ik wenschte,,.}{te weten R\^evard is de}{waarde der wrake}\\

\haiku{{\textquoteright} {\textquoteleft}Maar, R\^evard,{\textquoteright} uitte, {\textquoteleft}?}{Zebedeushoe komt gij}{aan die gedachte}\\

\haiku{Gij hebt haar met uw,}{Lorrijnsche redenen zelf}{den weg gewezen}\\

\haiku{{\textquoteleft}Wilt u mij langzaam?}{en duidelijk de woorden}{nog eens herhalen}\\

\haiku{hij haalde zijn adem.}{fluitend in en a\^emde dien}{fluitend w\^eer uit}\\

\haiku{m\'a-ar... toen de baard......}{er eens was fluit-te ik}{van vo-ren-af-aan}\\

\haiku{Geen mensch weet dat, ik...,...}{fl\'uit geen mensch komt meer naar mij}{luisteren ik fluit}\\

\haiku{om-dat u......}{mijn neus niet hebt u d\`enkt het}{zal sn\`orken worden}\\

\haiku{{\textquoteright} {\textquoteleft}Doet het mij maar eens,{\textquoteright},.}{na gromde de man gebelgd}{of hij beleedigd was}\\

\haiku{Op vele vragen.}{geeft mijn oom mij nog altijd}{het beste antwoord}\\

\haiku{Er kwamen er nog,...}{meer Naast mij zeeg een juffrouw}{op het stoeltje ne\^er}\\

\haiku{Laten wij eens in,.}{de groene kamer kijken}{als het u belieft}\\

\haiku{{\textquoteright} Dit zeggende had.}{Philippus weder naar zijn}{meester omgezien}\\

\haiku{Voldaan, wanneer, al,...}{wist ik niet om wat Er iets}{in mij ontbloeide}\\

\haiku{Het kwam zoo voor een,.}{glimploos vlak te staan Als oud}{ijs in de maan}\\

\haiku{Wij hebben mannen,,;}{noodig zegt doctor Swellius}{mannen van de daad}\\

\subsection{Uit: De wonderlijke avonturen van Zebedeus}

\haiku{Wij zaten er als.}{kleinen aan wie een vol bord}{kersen werd beloofd}\\

\haiku{Ik zag hem, bromde,.}{R\^evard laatstelijk met een}{bloem in het knoopsgat}\\

\haiku{Zebedeus, de,.}{blaadjes ter hand zit weer in}{vragende houding}\\

\haiku{op 't oud Atheensche,:}{voorrecht Wijl ze is van mij}{beschik ik over haar}\\

\haiku{Demetrius vindt,;}{het niet Die wil niet zien wat}{ieder ander ziet}\\

\haiku{Een stevig stuk werk,,.}{dat verzeker ik je en}{een verrukkelijk}\\

\haiku{De duif vervolgt den,.}{griffioen de hinde Spoedt}{zich ter tijgervangst}\\

\haiku{Al heeft hij Hermia,,,,;}{lief Heer liefheeft o Toch heeft}{ze alleen u lief}\\

\haiku{{\textquoteright} {\textquoteleft}Gij ziet met deze ',{\textquoteright}.}{drank int lijf bromt R\^evard}{met zijn basstem we\^er}\\

\haiku{Zij keert zich op de,:}{rustbank maakt zich klein als een}{diertje en fluistert}\\

\haiku{Nee, doe er twee bij,.}{laat het geschreven worden}{in acht en acht}\\

\haiku{{\textquoteleft}Antropologen,,{\textquoteright}, {\textquoteleft}.}{mevrouw zegt R\^evardzijn geen}{antropophagen}\\

\haiku{Dan gaan twee er plots,;}{\'een vrijen Dat moet tot een}{spelletje leien}\\

\haiku{hang me niet aan, Of.}{ik zal je als een slang mijn}{lijf afschudden}\\

\haiku{Maar hij verjoeg mij,}{schimpend en hij dreigde Te}{slaan te schoppen mij}\\

\haiku{Hij kwam er recht mee:}{aangeloopen achter uit}{de sneeuwjacht en zei}\\

\haiku{mijn koningin, uw,.}{hand Betreed de slaapvloer hier}{naar wiegetrant}\\

\haiku{Mij dunkt, ik zie dit,.}{al met loenschen blik Als elk}{ding dubbel schijnt}\\

\haiku{Ik kon die oude,.}{fabels Die kinderlijke}{sprookjes nooit gelooven}\\

\haiku{Hoe te verschalken,?}{Den lakschen tijd anders als}{door wat vreugde}\\

\haiku{'k vertelde reeds,.}{mijn lief Van Herkules mijn}{glorieus verwant}\\

\haiku{Ik heb order hier.}{gekregen Achter de deur}{het stof te vegen}\\

\haiku{{\textquoteleft}Ja,{\textquoteright} zuchtte ze uit, {\textquoteleft};}{ze dansten bij de kom waar}{de koe komt drinken}\\

\haiku{{\textquoteleft}O,{\textquoteright} kwam de stem van, {\textquoteleft}.}{Dorinde binnen sprekener}{was geen sleutelgat}\\

\haiku{{\textquoteleft}Vraag mij dat nu niet,{\textquoteright}:}{bromde Zebedeus en}{vervolgde luider}\\

\haiku{{\textquoteright} {\textquoteleft}Denk er eens over na,{\textquoteright}.}{antwoordde Zebedeus}{op de oude toon}\\

\haiku{De gastvrouw had hun.}{nadering bespeurd en trad}{rustig naar voren}\\

\haiku{hij zag de gastvrouw.}{afkijken ook en ging met}{Dorinde terzijde}\\

\haiku{{\textquoteright} antwoordde de heer;}{met een open afgrijzen naar}{zijn zegsman starend}\\

\haiku{ze heeft haar buis aan'.}{en haar hoedje op en rookt}{uit Manus pijpje}\\

\haiku{tuinders, R\^evard, zijn.}{soms de grootste stoorders van}{het werk der bijen}\\

\haiku{{\textquoteright} meende moeder Mie,.}{aanloopend omdat zij zich}{even verwijderd had}\\

\haiku{het geloofde aan.}{den boom en dat was wat de}{kinderen boeide}\\

\haiku{{\textquoteleft}Ik heb het hem niet,{\textquoteright}.}{gevraagd antwoordde verstrooid}{Zebedeus}\\

\haiku{{\textquoteleft}Het ging niet langer,,{\textquoteright}.}{niet l\'anger herhaalde het}{beeld in de spiegel}\\

\haiku{{\textquoteright} Mevrouw Popotte.}{leek opstond de reiniging}{der kooi vergeten}\\

\haiku{Ik verheelde u.}{niet in welk een toestand zich}{de boedel bevond}\\

\haiku{Hij is in de lucht,,;}{van ons in onze wolken}{verduisteringen}\\

\section{Louise B.B.}

\subsection{Uit: Janneke de pionierster}

\haiku{Nog dien ochtend had,:}{vader vol trots me in de}{wangen geknepen}\\

\haiku{Met gebogen hoofd,,;}{fronsende wenkbrauwen stond}{ik te luisteren}\\

\haiku{En nu klopte mijn,:}{hart zoo bevangen toen ik}{zachtjes mompelde}\\

\haiku{de gebeden der,....}{stervenden opdreunend zooals}{ik later vernam}\\

\haiku{Sidin dengar, hoor, {\textquoteleft},!}{apa-apa!{\textquoteright}24Wat heb-je}{gehoord spreek vrij uit}\\

\haiku{Als je in Holland,}{een dinertafel ziet dan}{vallen de bloemen}\\

\haiku{Ik zag ons zitten,,;}{moeder en ik over elkaar}{Henk tusschen ons in}\\

\haiku{dat ik die droge!}{harde rijst aanzie voor een}{smedigen pudding}\\

\haiku{En weet je wat me?}{nu op het oogenblik het}{meest interesseert}\\

\haiku{{\textquoteleft}Kom, Henk, nu, voor wij {\textquotedblleft}{\textquotedblright},,?}{naarkooi gaan ons gewone}{halfdekje slaan h\`e}\\

\haiku{ik ben z\'o\'o blij, z\'o\'o,!}{innig dankbaar dat ik je}{gevolgd ben hierheen}\\

\haiku{Vrouwen gevoelen!}{en  denken nu eenmaal}{anders dan mannen}\\

\haiku{{\textquoteleft}Ik dank u voor uw,,;}{bitter maar stellig gezond}{drankje dokter Spaan}\\

\haiku{Als u soms eenige....?}{kisten blikjes en wijn van}{mij wilt overnemen}\\

\haiku{{\textquoteright} Onwillekeurig,.}{in mijn groote vreugde liet ik}{mijn handen rusten}\\

\haiku{Ik riep hem luide,.}{toe en na wat geschuifel}{ontsloot hij de deur}\\

\haiku{{\textquoteleft}Wat zijn jullie druk,,!}{ik ben niets nieuwsgierig laat}{mij nu maar met rust}\\

\haiku{Ik knikte, het was,.}{waar Johnstone alleen hield}{kippen op zijn erf}\\

\haiku{Men wil niet weten,!}{voor de rechter- wat de}{linkerhand wegschenkt}\\

\haiku{de respectieve {\textquoteleft}{\textquoteright}!}{paddenstoelen-cultuur}{in de woningen}\\

\haiku{Van Offenberg, dat.}{is zware concurrentie}{die u mij aandoet}\\

\haiku{Ik ook, ik ben heel,,....}{trotsch tegen haar geweest dat}{weet ik wel enne}\\

\haiku{Weldra stond ik in.}{de achtergalerij van}{Johnstone's woning}\\

\haiku{Eindelijk klopte:}{ik haar tot afscheid nog eens}{op den schouder}\\

\haiku{En zoodra zij,.}{zaten bood Henk sigaren}{aan schonk  ik thee}\\

\haiku{Nu toost ik op uw:}{lang verblijf alhier met dit}{geurig kopje thee}\\

\haiku{Wil-je dadelijk,,?}{gaan over veertien dagen met}{de volgende boot}\\

\haiku{{\textquoteright} Des avonds kwamen, als,.}{naar gewoonte de heeren}{op het theeuurtje}\\

\haiku{Mijn heer en meester.}{met zijn dankbaar kiekgezicht}{drentelde mij na}\\

\haiku{allen, ieder voor, {\textquoteleft}{\textquoteright}!}{zich had om diegrroote eer}{willen smeeken}\\

\haiku{{\textquoteright} Spaan sloeg zijn armen,:}{om mij heen tilde me op}{en droeg me naar bed}\\

\haiku{Sidin begon ook!}{met zoo raar te beven en}{zich flauw te voelen}\\

\haiku{Toean heeft mij zoo!}{gezegd niet te vergeten}{u in te geven}\\

\haiku{Er klonk angstige....}{droefheid uit die wanhopig}{berustende stem}\\

\haiku{niet meer was en heel.}{goed een behoorlijk toilet}{had kunnen maken}\\

\haiku{maar dat weet je wel,,!}{kind ik moet hier blijven tot}{mijn contract om is}\\

\haiku{hier te blijven, maar.}{juist de groote apathie waarvoor}{dokter Spaan bang is}\\

\haiku{en nu wilden zij.}{het je zoo gemakkelijk}{mogelijk maken}\\

\haiku{Geen wonder dat je,!}{het niet te boven komen}{zou mijn eenige schat}\\

\section{Virginie Loveling}

\subsection{Uit: Bina}

\haiku{Zijn boerderij was ':}{de grootste van het dorp en}{t omliggende}\\

\haiku{Lietje en Merie.}{geleken noch op Bina}{noch op elkander}\\

\haiku{{\textquoteleft}Er bestaat maar een.}{vrouwmensch op de wereld voor}{my en gij zijt het}\\

\haiku{maar gij, rijke boer,.}{zult toch uw dochter in haar}{hemd niet laten gaan}\\

\haiku{dat is te vetten,,.}{elk  in een kleine ren}{in een donker kot}\\

\haiku{{\textquoteright} zei Deodaat, haar.}{minzaam van onderen op}{in de oogen ziende}\\

\haiku{Deodaat, hare,.}{verschijning beloerend had}{op de ruit getikt}\\

\haiku{wat die zich eens in,.}{het hoofd had gezet was er}{niet uit te krijgen}\\

\haiku{Ge kunt ze immers,,{\textquoteright}.}{kwijt zijn zoodra ge wilt}{verweet haar Merie}\\

\haiku{Het duurde eenige;}{dagen eerdat de dokter}{was gerustgesteld}\\

\haiku{Van gansche dagen,.}{sprak Bina schier geen woord meer}{stroef in zich gekeerd}\\

\haiku{Bina zoo werkzaam,,!}{zoo waakzaam scheen voor alles}{onverschillig thans}\\

\haiku{Zoodra het vet,}{in de pan siste schepte}{zij een lepel deeg}\\

\haiku{{\textquoteright} Wat Merie zei, toen,.}{ze kort daarop binnenkwam}{was te bevroeden}\\

\haiku{Wij kunnen het toch,,?}{niet helpen nietwaar mijnheer}{de onderpastoor}\\

\haiku{Bina was weder,.}{ter been doch tot den rang van}{slonsmeid afgedaald}\\

\haiku{Ik kom u vragen?....}{of Bina mij ten langen}{laatste hebben wil}\\

\haiku{{\textquoteright} was het eenige wat,.}{haar verbazing uitbracht toen}{ze hem ontwaarde}\\

\haiku{{\textquoteright} zei hij gebiedend,,.}{als boosaardig in zijn smart}{en het geschiedde}\\

\haiku{Bij het vloerstroo bleef,.}{hij staan als ontdekte hij}{voor het eerst iets nieuws}\\

\haiku{Was het niet genoeg, {\textquoteleft}?}{dat baasken van Dorpeden}{penning werd gejond}\\

\haiku{daar een, nog recht, zijn;}{rosse flarden-armen in}{gebed openhoudend}\\

\haiku{{\textquoteright} {\textquoteleft}Eene belofte, gij,?}{zult mij niet bespieden niet}{volgen in mijn tocht}\\

\haiku{Dien had hij reeds vast,.}{terwijl hij omzichtig de}{trappen afdaalde}\\

\haiku{De daders waren....}{tot dusverre aan elke}{hinderlaag ontsnapt}\\

\haiku{Niet straffeloos had.}{Jasper het fijne polsje in}{zijne hand gedrukt}\\

\haiku{Zijn ergernis en....}{zijne verootmoediging}{waren grenzeloos}\\

\haiku{zijne schreden hem,.}{van zelf naar de kade waar}{hij vroeger woonde}\\

\haiku{Hij wreef de hand over,.}{zijn voorhoofd waar het koude}{zweet op parelde}\\

\haiku{{\textquoteright} En zoo werd het nieuws,.}{van deur tot deur van mond tot}{mond overgeleverd}\\

\haiku{Waren zij allen,?}{te beklagen zoo diep als}{hun ellende scheen}\\

\haiku{Hij sprak een persoon;}{of twee aan met zijn holle}{schooiersbetoning}\\

\haiku{En hij vertelde.}{de geschiedenis van den}{brief en het erfdeel}\\

\haiku{{\textquoteleft}Onze vader die,{\textquoteright}:}{in de hemelen zijt aan}{de deuren en zei}\\

\haiku{Hij betrok het dan.}{ook mits zesmaandelijksche}{betaling voorop}\\

\haiku{{\textquoteright} {\textquoteleft}Die wat krijgt moet wat,{\textquoteright},.}{geven antwoordde Peetje}{als een hondgeblaf}\\

\haiku{{\textquoteleft}Mijd u,{\textquoteright} zei de meid,.}{tegen hem met een emmer}{kolen aankomend}\\

\haiku{{\textquoteright} Peetje ging naar den,,.}{vlaskoopman deed zijn beklaag}{hevig uitvallend}\\

\haiku{{\textquoteright} {\textquoteleft}Och Heere, mijn geld,,,!}{mijn schoon geld alles kwijt en}{het zoo noodig hebben}\\

\haiku{ik zou gaan, nu eens?}{stappen in deze dan in}{gene richting deed}\\

\haiku{Lag de stad v\'o\'or mij,?...}{lag ze aan mijn rechterhand}{of aan mijn linker}\\

\haiku{{\textquoteright} vroeg ik mij af, en.}{toen trok ik doelloos verder}{met haastigen stap}\\

\haiku{{\textquoteleft}Bellotje,{\textquoteright} zei ik,,.}{schuchter als een schooljongen}{die iets heeft misdaan}\\

\haiku{Een kopje met melk, '.}{stond onaangeroerd naast haar}{opt vensterbord}\\

\haiku{Kalmte is in mijn,.}{gemoed gekomen kalmte}{en melancolie}\\

\haiku{{\textquoteleft}Mejuffrouw, ik kan.}{u niet genoeg danken voor}{uw gedienstigheid}\\

\haiku{Ik keek door 't raam,;}{zag boeren en boerinnen}{trekken naar de kerk}\\

\haiku{Du gr\"unst nicht nur in,.}{Sommerzeit Im Winter auch}{wen's friert und schneit}\\

\haiku{Dat struikelen in!}{de bramen op den barm van}{genen diepen weg}\\

\haiku{Ik wendde 't hoofd.}{ter zijde en worstelde}{om los te komen}\\

\haiku{Wel zag ze dat ik,.}{bleek was en gedrukt dat ik}{geen eetlust meer had}\\

\haiku{De boomen staan  ,.}{nog naakt maar lentewalmen}{hangen in de lucht}\\

\haiku{20 Maart 19... ~ Welk?}{een goddelijke hand heeft}{mij daarheen gestuurd}\\

\haiku{{\textquoteleft}Moeder heeft nooit weer.}{het portret van Cecile}{willen bekijken}\\

\haiku{Omdat ik het zoo,{\textquoteright}}{schoon vind stamelde ik met}{het beeld in de hand.}\\

\haiku{{\textquoteright} {\textquoteleft}Of al kwam het zelfs,{\textquoteright}.}{niet levend ter wereld zei}{ze godsdienstschendend}\\

\haiku{Ofschoon op 't feit,:}{betrapt trachtte hij zich te}{verontschuldigen}\\

\haiku{Zij wachtte op de,.}{diligence die destijds}{nog de spoor verving}\\

\haiku{In het vervolg werd.}{nooit een woord meer tusschen hen}{over die zaak gerept}\\

\haiku{Hij trad juist binnen,,.}{hij had staan luisteren hij}{had alles gehoord}\\

\haiku{Hij sleepte zich tot.}{aan de voordeur en trok den}{ondergrendel uit}\\

\haiku{'t Was ook niet noodig,.}{iets er bij te voegen het}{feit sprak voor zich zelf}\\

\haiku{Dit jaar was 't heel.}{den morgen een geratel}{en gerij geweest}\\

\haiku{De stoet kwam langs den.}{Steenput voorbij in zijne}{omreis rond het dorp}\\

\subsection{Uit: Erfelijk belast}

\haiku{ge zoudt mij zeker?}{in een muit willen houden}{gelijk een vogel}\\

\haiku{Madame D'Haeyer.}{zat uit te kijken aan haar}{bovenvenstertje}\\

\haiku{Mietje donderde.}{los op haar en ik spaarde}{haar ook geen verwijt}\\

\haiku{Zij had Colette,,,:}{vroeger nog gezien ging bij}{haar gedrukt en sprak}\\

\haiku{{\textquoteright} {\textquoteleft}Goed,{\textquoteright} sprak het kind met.}{sombere schaduwen van}{wanhoop in het oog}\\

\haiku{Het was te zien, dat.}{het er bedrijvig toeging}{en geen werk ontbrak}\\

\haiku{Berenice had.}{nooit andere hoven dan}{stadshoven gezien}\\

\haiku{{\textquoteleft}Tante, wie was die ',?}{kleine int groen met het}{anker in de hand}\\

\haiku{Welnu, dat duurt reeds,,,....}{laat zien wel vijftien jaren}{zij wachten naar haar}\\

\haiku{Hoe innig lief, steeds,.}{hopend dat de beterschap}{ernstig zou wezen}\\

\haiku{dien wierp zij eerst los,.}{over zich toen waagde ze het}{hem aan te trekken}\\

\haiku{Te Vroden had hij.}{geen leeftijdsgenooten}{van zijn geestespeil}\\

\haiku{de dagen waren,,,,.}{kort koud mistig donker ten}{beste dat het ging}\\

\haiku{Otto wist het wel.}{en herhaalde het dikwijls}{genoeg aan zich zelf}\\

\haiku{oom Eed schudde het,.}{hoofd en zei rechtuit dat hij}{het niet begeerde}\\

\haiku{Colette zuchtte.}{reeds met het vooruitzicht van}{een dronken thuiskomst}\\

\haiku{Tante, weet ge wat,.}{breng Berenice eens mee}{als ge terugkomt}\\

\haiku{Ondanks dat kwamen.}{de klanten na de vroegmis}{en v\'o\'or de hoogmis}\\

\haiku{Er was een hanglamp,,.}{en toen die brandde werd het}{eerst recht genotrijk}\\

\haiku{{\textquoteright} {\textquoteleft}Zeker niet, zeker, '.}{niet doch spreken wij niet meer}{overt gebeurde}\\

\haiku{Ik u wel, ik ging, '.}{u te gemoet toenk u}{hierin zag vluchten}\\

\haiku{Berenice mocht,.}{het niet voort vertellen was}{haar aanbevolen}\\

\haiku{Wat was het toch, dat?}{haar zoo drukte op dien tot}{vreugd bestemden avond}\\

\haiku{{\textquoteright} {\textquoteleft}Mijn eigen hart rijdt,,}{op een karreken16 als ik}{dat allemaal hoor}\\

\haiku{{\textquoteleft}Clette, er mag toch?}{zeker wel een druppel17 af}{op zulk een einde}\\

\haiku{wij hebben reeds te,{\textquoteright}:}{veel tijd verloren en hij}{nam een boek ter hand}\\

\haiku{Toen zei hij in eens,:}{den loop van inwendige}{gedachten uitend}\\

\haiku{{\textquoteleft}Het is een schande,}{zulke kinderen op de}{wereld te brengen}\\

\haiku{En hij dacht aan de,.}{tormenten der hel waarvan}{de priesters spraken}\\

\haiku{Zij hoorde hem aan,,}{reeds half verloomd en telkens}{was het haar weder}\\

\haiku{{\textquoteright} en zij nam al haar,.}{krachten te zamen stond op}{en wankelde heen}\\

\haiku{Aan 't avondmaal werd.}{hij tot de werkelijkheid}{teruggeroepen}\\

\haiku{Dat geestbesliste:}{in hem breidde zich tot het}{stoffelijke uit}\\

\haiku{Tante Colette,.}{trad binnen ze waarschuwde}{dat het etenstijd was}\\

\haiku{hij gaat ten onder,!}{van gebrek van uitputting}{en vermoeienis}\\

\haiku{{\textquoteright} en hij strekte de, {\textquoteleft}?}{armen uit boven zijn hoofd}{wilt ge nu heengaan}\\

\subsection{Uit: Een revolverschot}

\haiku{Hij bracht een handvol,.}{kaartjes een goeden wensch en}{een plakalmanak}\\

\haiku{Men moet overal een {\textquotedblleft}{\textquotedblright},{\textquoteright}.}{harden pakken op zulke}{dagen zei de een}\\

\haiku{Zij nam 't al eens, '.}{op den arm zij mindet}{eindelijk met drift}\\

\haiku{Toen Georgine,.}{elf jaar telde was Marie}{er twee en twintig}\\

\haiku{De kleine{\textquoteright}, aldus,.}{noemde haar Marie en hun}{vader zei het na}\\

\haiku{hij scheen overal een.}{zonneglans van opwekking}{mede te brengen}\\

\haiku{Altijd zeker is,.}{het dat hij het goed van zich}{wist af te schudden}\\

\haiku{Ik heb haar gezien,,,.}{een schoone doode men zou}{zeggen dat ze slaapt}\\

\haiku{En waar eertijds, en,:}{gedurende zoovele}{jaren de plaat met}\\

\haiku{Luc Hancq en de:}{twee raadsleden-jonkmans}{deden hetzelfde}\\

\haiku{{\textquoteright} met haar vingeren.}{woelde zij werktuiglijk de}{mulle aarde om}\\

\haiku{hij wist het immers,,...}{wel hij moest het weten dat}{zij veel van hem hield}\\

\haiku{Waarom liet hij zich?}{van den eersten den besten}{de loef afsteken}\\

\haiku{Zij gingen rond op ':}{t enge ruim en keken}{naar alle winden}\\

\haiku{{\textquoteleft}Marie, dat is niet,{\textquoteright},.}{wel gedaan sprak hij ontsteld}{zichtbaar beangstigd}\\

\haiku{{\textquoteleft}Uw beenen zijn jonger,,{\textquoteright}.}{dan de mijne ga zelve}{zei ze eenigszins scherp}\\

\haiku{Zij was donkergroen,,.}{aan fluweel gelijk niet of}{weinig vertreden}\\

\haiku{{\textquoteright} herhaalde Marie,, {\textquoteleft} '?}{verslagenwaarom hebt ge}{t mij niet gezegd}\\

\haiku{{\textquoteleft}Mijnheer Luc Hancq,, '?}{heeft niet gewild dat ik u}{riep verstaat get}\\

\haiku{Georgine scheen,.}{zoo onverschillig zoo niet}{vijandig gestemd}\\

\haiku{Het regende thans,.}{bijna bestendig zoodat ze}{zich binnen hielden}\\

\haiku{De beide zusters.}{troostten hem in de maat van}{het mogelijke}\\

\haiku{Zou Luc misschien een?}{voorwendsel zoeken om te}{Vroden te blijven}\\

\haiku{Hij kwam niet dien avond '.}{daar er repetitie van}{t muziekkorps was}\\

\haiku{ginder, dan alleen!}{met het gruwelijk geheim}{bekend te wezen}\\

\haiku{Sprakeloos had ze,.}{het hoofd geschud dat ze er}{niet bij gaan wilde}\\

\haiku{{\textquoteright} zei ze wrevelig, {\textquoteleft},,.}{doe uw werk ik zal bellen}{als ik u noodig heb}\\

\haiku{Stasius werd door.}{den vrederechter van het}{kanton ondervraagd}\\

\subsection{Uit: De twistappel}

\haiku{Die broeders zouden ',.}{t saam bewonen er hun}{leven eindigen}\\

\haiku{Weldra was al het,;}{grof werk gedaan de metsers}{en de dienders weg}\\

\haiku{{\textquoteleft}Indien alzoo een!}{schepselken eens eeuwig in}{de hel moest branden}\\

\haiku{en mijnheer die niet, '!}{thuis ist zal al aan mij}{geweten worden}\\

\haiku{{\textquoteright} Zij ijlde vooraan,.}{toomloos heen gedreven als}{een waanzinnige}\\

\haiku{{\textquoteright} Was 't eerste, wat;}{ze aan Kathelijntje met}{doffe stemme vroeg}\\

\haiku{haar glimlach was een,.}{zoen elke handeling een}{streeling voor het kind}\\

\haiku{zij hinkte pijnlijk,.}{eene hand steunzoekend in}{de zijde houdend}\\

\haiku{maar nu gebeurt dat,.}{allemaal niet meer aldus}{het geloof is weg}\\

\haiku{{\textquoteright} gewichtig knikte, {\textquoteleft},,.}{Petruskiekens konijnen}{en duiven juffrouw}\\

\haiku{Aan het hek staan was, '.}{streng verboden door mijnheer}{zij waagdet niet}\\

\haiku{{\textquoteright} antwoordde mevrouw, {\textquoteleft},.}{Duquennehij heet Gaspard}{hij behoudt zijn naam}\\

\haiku{Gaspard is een van, '.}{de Drie Koningen ik heb}{t u reeds verklaard}\\

\haiku{Fernande bedacht.,..}{zich verstandiglijk het voor}{en tegen wikkend}\\

\haiku{Dit dacht hij nu en.}{dankbaarheid borrelde weer}{boven in zijn hart}\\

\haiku{Thans brak voor hem en:}{voor Fernande een tijdperk}{aan van groot geluk}\\

\haiku{Doch ze wisten 't,:}{niet en velen moesten wel hun}{meening deelen want}\\

\haiku{{\textquoteright} antwoordde deze,.}{geringschattend zich tot de}{anderen richtend}\\

\haiku{Niets, niets, het was een,{\textquoteright}.}{grap aldus wilde Petrus}{het verbeteren}\\

\haiku{zij was zijn eigen.}{moeder niet en had zich als}{deze aangesteld}\\

\haiku{Op zijne kamer;}{gesloten heel den dag te}{water en te brood}\\

\haiku{En Fernande stond:}{pal onder dien stortvloed van}{beschuldigingen}\\

\haiku{{\textquoteleft}Heere, verleen mij!}{de kracht om die beproeving}{te doorworstelen}\\

\haiku{Diep kwetste 't haar doch.}{zij was veel te fier om het}{te laten merken}\\

\haiku{de zekerheid en....}{de bevestiging van het}{reeds lang gevreesde}\\

\haiku{In zijne hand houdt....}{hij nog enkele bladen}{van een reisgids vast}\\

\haiku{Zij tilt hem overeind,:}{in hare armen drukt zijn}{hoofd aan hare borst}\\

\haiku{Hem met hare hand,.}{opgetild hem aan haar hart}{geaaid als een kind}\\

\haiku{een stofje, een atoom,,,....}{onzichtbaar iets wegblaasbaar}{weggeblazen reeds}\\

\haiku{haar bloote borst was met,}{zeven zwaarden doorstoken}{en plots begreep hij}\\

\haiku{hij was thans aan zijn {\textemdash}?}{laatste examen altijd het}{recht verdedigen}\\

\section{Lambert Rijckxz Lustigh}

\subsection{Uit: 'Kroniek I' van Lambert Rijckxz Lustigh (1656-1727)}

\haiku{Gevangenneming-;}{en executie Dick Jansen}{Spilt151153~1711}\\

\haiku{, en dat hij oock,}{wel staande Houde en noch}{met eenen dierbaren}\\

\haiku{daar mede hare}{keleren tongen wascht}{ende op den 16}\\

\haiku{van hem sijn Laaste:}{drie schoone koijen af}{op den 26 decemb}\\

\haiku{hebbe u besten}{gedaan om uwe koijen}{sieck te maken dogh}\\

\haiku{1714 doen sterft van klaas:}{meijnsen sijn Laaste koe}{Op den 12 febr}\\

\haiku{Ik meende dat Ik ',}{een deuselingh int hooft}{kreegh daarom soo gingh}\\

\haiku{de grietenijen}{terwetardeel en beijde}{de dongalen staan}\\

\haiku{alles wat uijt de,:}{voorz vaart sal proveneren}{ook geven haar Ed}\\

\haiku{van Ebbe Willemsz}{koij voor seven stuvers aen}{rottekruijt gekogt}\\

\haiku{1721 van amsterdam}{in koetsen tot weesp wierden}{gevoert om aldaer}\\

\haiku{1722 wiert gedaan,}{een waar en waragtighe}{beschrijvinge uijt}\\

\section{Christiaan Creemers, Jos. Habets, Maria Luyten en A. Nieuwenhuizen}

\subsection{Uit: Kronijk uit het klooster Maria-Wijngaard te Weert, 1442-1587. Eene bijdrage tot de voorgaande kronijk, op het jaar 1566. Een vijftal stukken betrekkelijk de Hervorming te Weert 1583-1584}

\haiku{wij meynden dat de;}{kerck op ons hooft soude}{gevallen hebben}\\

\haiku{drij stonden gebaert;}{op de spincamer tusschen}{ider pilaer \'e\'ene}\\

\haiku{daerom is sij;}{verhoort en de gevangen}{sijn vrijgelaeten}\\

\haiku{waeren wij in seer,.}{grooten noot en lijden niet}{wetende wat doen}\\

\haiku{\'e\'en pont boter twee,.}{stuijver min \'e\'en oort \'e\'en ton}{haringh aght gulden}\\

\haiku{en strackx naer den oogst.}{begonsten alle dingen}{seer dier te worden}\\

\haiku{sij beraede haer;}{met twee of drij  van de}{oudste susteren}\\

\haiku{wij hadden nogh soo.}{grooten schaede niet in het}{gewas des ackers}\\

\haiku{Eenige hebben hun;}{verhangen en eenige}{op wege geweest}\\

\haiku{Precies doen de clock.}{vier uuren sloegh wiert het light en}{men sagh het niet meer}\\

\haiku{Hij zou voortaan zijn.}{gebed aan de kerk-deur}{mogen verrichten}\\

\haiku{39Het handschrift van den.}{Heer Habets heeft Buren in}{plaats van Beijeren}\\
