\chapter[2 auteurs, 232 haiku's]{twee auteurs, tweehonderdtweeëndertig haiku's}

\section{Ucee}

\subsection{Uit: De moordende hand}

\haiku{Noodgedwongen moest.}{hij dus het einde van den}{wedstrijd afwachten}\\

\haiku{Ik neem uw opdracht}{gaarne aan en begin met}{u te verzoeken}\\

\haiku{op je kan worden!}{en daarvan profiteer je}{maar al te dikwijls}\\

\haiku{{\textquoteleft}Kees van Berg, ik daag.}{je nu uit tot een boksmatch}{van tien ronden}\\

\haiku{Dat is natuurlijk,{\textquoteright},, {\textquoteleft}}{Lefsky die er een eind aan}{gaat maken dacht Kees}\\

\haiku{Het resultaat was,,.}{natuurlijk dat de Rooie zijn}{tegenstander sloeg}\\

\haiku{Mejuffrouw, ik vraag{\textquoteright},, {\textquoteleft}}{u duizendmaal excuus riep}{hij zeer ernstig uit}\\

\haiku{{\textquoteleft}Neen, mijnheer, het zijn,!}{mijn Hollandsche vrienden let}{u maar eens goed op}\\

\haiku{{\textquoteright} {\textquoteleft}Zoo zie je, Kees, dat{\textquoteright}.}{men met zijn vermoedens zeer}{voorzichtig moet zijn}\\

\haiku{{\textquoteright} {\textquoteleft}Maar mijnheer Verdoorn,?}{herkent u den Indischen}{Sherlock Holmes niet}\\

\haiku{Het duurde nog wel,.}{een half uur voordat Kees zijn}{bewustzijn herkreeg}\\

\haiku{Had je het mij maar,.}{verteld toen ik dat reptiel}{in mijn armen had}\\

\haiku{{\textquoteleft}Brandhorst, ben je tot,,?}{wederdienst bevrijd pardon}{ik bedoel bereid}\\

\haiku{En pas op voor zijn,,{\textquoteright}.}{wraak jonge dame die zal}{niet voor de poes zijn}\\

\haiku{Als ik mijn hanen,}{niet binnen vijf minuten}{terugkrijg dan jaag}\\

\haiku{{\textquoteleft}Het spijt mij, waarde,{\textquoteright}.}{vriend maar ik weiger beslist}{je neer te schieten}\\

\haiku{Deze zag rood van,}{woede terwijl hij den hem}{volgenden katjoeng}\\

\haiku{Ziska bevond zich,.}{in de voorgalerij toen}{de Ford binnenreed}\\

\haiku{Ik had een vrouw, zooals.}{er slechts \'e\'en in de honderd}{jaar geboren wordt}\\

\haiku{, naar een naburig.}{boschje begaf om daar een}{man te ontmoeten}\\

\haiku{{\textquoteleft}Als het zoover is, dan!}{krijgt de Rooie weer het noodige}{van mij te hooren}\\

\haiku{Met een kamertje!}{in de bijgebouwen zijn}{zij reeds tevreden}\\

\haiku{{\textquoteright} {\textquoteleft}Maar, natuurlijk Leo,,!}{je doet maar precies wat jou}{het beste voorkomt}\\

\haiku{{\textquoteright} {\textquoteleft}Dus je denkt, dat......{\textquoteright} {\textquoteleft}Neen,,.}{denken doe ik het niet want}{ik weet het zeker}\\

\haiku{Zoodra u iets,.}{ongewoons merkt noteert u}{dit onmiddellijk}\\

\haiku{Dan begrijp ik niet,,.}{waarom u nog niet overtuigd}{is mijnheer Hansen}\\

\haiku{{\textquoteleft}Maar, lieve meid, heb?}{je zoo weinig vertrouwen}{in je aanstaande}\\

\haiku{Stel je eens voor, dat!}{hij Othelloachtige}{oprispingen krijgt}\\

\haiku{Ik vrees, dat de Raad!}{van Justitie er gaarne}{meer van wil weten}\\

\haiku{Ondanks zijn flegma.}{liep het den Engelschman}{ijskoud over den rug}\\

\haiku{Men moet werkelijk!}{krankzinnig zijn om al dien}{onzin te slikken}\\

\haiku{{\textquoteright} {\textquoteleft}Zeg, Kees, ik ben ook,!}{een volbloed ben ik dan ook}{een boerepummel}\\

\haiku{Als alle Indo's,...............{\textquoteright} {\textquoteleft}}{zoo zijn als dat tweetalEn}{niet te vergeten}\\

\haiku{Twintig minuten.}{later bereikten zij de}{bekende badplaats}\\

\haiku{{\textquoteleft}Als er iets is, dat,.}{ik niet verdragen kan dan}{is het leedvermaak}\\

\haiku{Vooruit, Leo, laten!}{wij dat  reptiel op een}{paar steenen trakteeren}\\

\haiku{Die kwam zich bij mij.}{melden om zijn wraak op u}{beiden te koelen}\\

\haiku{Hij liep zooeven vlak!}{langs je en kon je toen met}{gemak neerschieten}\\

\subsection{Uit: Het spookhuis van Tandjong-Priok}

\haiku{{\textquoteright} {\textquoteleft}Wat een vraag, mijnheer,:}{Brandhorst maar als u het toch}{zoo graag weten wilt}\\

\haiku{{\textquoteright} {\textquoteleft}Het is gewoonweg{\textquoteright},.}{wonderbaarlijk riep de Jong}{bewonderend uit}\\

\haiku{{\textquoteright} {\textquoteleft}Gepeddeld wordt er,{\textquoteright},.}{wel maar naar huis gaan wij nog}{niet antwoordde Leo}\\

\haiku{{\textquoteright} {\textquoteleft}En daarna had ik......}{den Chef van de recherche}{gefeliciteerd}\\

\haiku{{\textquoteright} {\textquoteleft}Heel vriendelijk van{\textquoteright},, {\textquoteleft}?}{u antwoordde Leomaar is}{daar zooveel haast bij}\\

\haiku{{\textquoteleft}Wees maar blij, dat ik,}{niet meega want anders zou}{je daar op Zandvoort}\\

\haiku{Het was net, alsof.}{dit eene woord de schellen van}{zijn oogen deed vallen}\\

\haiku{Dames en heeren,,,.}{mijn dank voor den eh nectar}{en het ambrozijn}\\

\haiku{Zeg, Fientje, wat zou?}{je zeggen van een tochtje}{met een motorboot}\\

\haiku{{\textquoteright} {\textquoteleft}Dan zullen zij bot,{\textquoteright}.}{vangen dat kan ik je op}{een briefje geven}\\

\haiku{Eindelijk dan toch{\textquoteright},.}{heb ik je op een verzuim}{betrapt juichte Kees}\\

\haiku{Nu nog mooier, wat[]?}{moet ik in den nacht met}{een kiektoestel doen}\\

\haiku{Daarna vertelde,[]:}{hij dat het ook op het ker}{khof erg spookte}\\

\haiku{naar het kerkhof om.}{voorloopig het terrein}{te verkennen}\\

\haiku{Ik moet bekennen,;}{dat ik mij niet erg op mijn}{gemak gevoelde}\\

\haiku{Daarna schudde hij {\textquoteleft}}{de beide kameraden}{hartelijk de hand.}\\

\haiku{{\textquoteleft}Enfin, als het toch,.}{niet anders kan dan zal ik}{maar van wal steken}\\

\haiku{Leo tuurde scherp langs,.}{den weg welken zij zoo juist}{afgelegd hadden}\\

\haiku{Als ik dien vent ooit,{\textquoteright}.}{te pakken krijg dan zal ik}{hem eens mores leeren}\\

\haiku{Loop jij intusschen,,{\textquoteright}.}{kalm door ik zal wel fluiten}{als ik je noodig heb}\\

\haiku{Plotseling voelde,.}{Leo dat Kees hem krampachtig}{in den schouder greep}\\

\haiku{{\textquoteleft}Stel je gerust, Kees,,.}{het skelet is al weg maak}{je dus niet angstig}\\

\haiku{De schurk, de schavuit,!}{om mij op zoo'n manier bij}{den neus te nemen}\\

\haiku{{\textquoteleft}Grijp nooit het handje{\textquoteright}.}{van een jongedame in}{het volle daglicht}\\

\haiku{wij zoo vrij zijn het.}{zoogenaamde spookhuis eens}{te inspecteeren}\\

\haiku{Leo vertelde zijn,.}{makker wat hij met de Jong}{had afgesproken}\\

\haiku{Ik ben weliswaar,!}{knock-down geweest maar nog}{lang niet down-hearted}\\

\haiku{Alleen hebben wij{\textquoteright}.}{nog een appeltje met dien}{Ursus te schillen}\\

\haiku{{\textquoteleft}Laten wij Fientje.}{vragen hier een dag of acht}{te komen logeeren}\\

\haiku{Jij matigt je dus.}{mir nichts dir nichts de rechten}{van een censor aan}\\

\haiku{In spanning keken;}{de aanwezigen naar het}{brandende pakket}\\

\haiku{Op dit oogenblik.}{passeerde de hoofdagent met}{den gevangene}\\

\haiku{Juist wilde hij zich,.}{snel verwijderen toen Kees}{hem in den weg trad}\\

\haiku{Ik heb voor het eerst.}{van mijn leven tevergeefs}{op mijn kracht gebouwd}\\

\haiku{Heb jij, terwijl wij,?}{aan het schrijven waren niets}{ongewoons gemerkt}\\

\haiku{allemachtig, nu,!}{gaat mij een licht op de odeur}{heeft hen verraden}\\

\haiku{{\textquoteleft}Zeg, Leo, waarom heb,}{je niet op hem geschoten}{toen hij zoo tartend}\\

\haiku{Zwaar moest bedoelde!}{autoriteit dan ook voor}{zijn misstap boeten}\\

\haiku{Tot mijn spijt mag ik{\textquoteright},.}{mij hierover niet uitlaten}{antwoordde Brandhorst}\\

\haiku{En jij, Ziska, krijgt,.}{een man dien duizenden je}{zullen benijden}\\

\section{Bob den Uyl}

\subsection{Uit: De ontwikkeling van een woede}

\haiku{Steeds was hij bezig.}{ontslag te nemen om het}{te gaan proberen}\\

\haiku{Misschien heb ik wel,.}{eens iets in die geest gezegd}{maar ik zeg zoveel}\\

\haiku{Op zich zelf de meest.}{originele tekst die er}{te bedenken valt}\\

\haiku{Huilen, zeggen dat,.}{ik gelijk had dat ze het}{ook niet kon helpen}\\

\haiku{Na een tijdje ging,,.}{ze gekalmeerd weg en dat}{was het dan dacht ik}\\

\haiku{Kwam mijn bezoek niet,.}{gelegen dan zouden we}{wel weer verder zien}\\

\haiku{waarschijnlijk had ook.}{zij het leeglopen van het}{huis geobserveerd}\\

\haiku{wel weet ik nog dat.}{het op een Zuidzee-eiland}{gesitueerd was}\\

\haiku{Dat jij kijkt juist op.}{het ogenblik dat er wat met}{dat object gebeurt}\\

\haiku{Ben ik niet altijd.}{even aardig tegen dieren}{en oude mensen}\\

\haiku{Ik raap hem op, en.}{het blijkt een band te zijn van}{zo'n halve meter}\\

\haiku{Veel vertrouwen in;}{zijn navigatiekunsten}{heb ik daarom niet}\\

\haiku{En als zo vaak klopt,;}{het zelfs zo dat ik me een}{beetje bekocht voel}\\

\haiku{Een puts water over,.}{zijn hoofd zou helpen maar hoe}{daaraan te komen}\\

\haiku{Ik was er nog een;}{beetje te jong voor al vond}{ik het wel spannend}\\

\haiku{dat waardeerde ik,.}{al kwam je alleen maar om}{boeken te lenen}\\

\haiku{En toen was alles,.}{in \'e\'en klap afgelopen}{weer door mijn lachen}\\

\haiku{Ik neem afscheid, zeg}{hem dat ik over een dag of}{twee nog eens langs kom}\\

\subsection{Uit: Quatro primi}

\haiku{Mijn inboedel was.}{beperkt en praktisch in zijn}{geheel nieuw gekocht}\\

\haiku{Hij was zelfs nog zo.}{gewillig me tweeduizend}{gulden te lenen}\\

\haiku{Met sigaren en.}{steekpenningen had ik hem}{toch binnen een week}\\

\haiku{Als we uitgeput.}{raakten namen we een paar}{weken vakantie}\\

\haiku{Wat nu te doen met,?}{mijn rijkdom hoe zou ik mijn}{leven inrichten}\\

\haiku{Alles bijeen had.}{ik het heel wel naar mijn zin}{in de warme zon}\\

\haiku{De dochter van \'e\'en.}{van de topfiguren uit}{de streek ging huwen}\\

\haiku{In de avond belde,.}{er al iemand op wanneer}{ik nou precies ging}\\

\haiku{zo snel mogelijk,.}{naar een expert die in de}{arm was genomen}\\

\haiku{Snel stapte ik langs.}{de huizen in de richting}{van de verkeersweg}\\

\haiku{Er zat ongeveer.}{een millimeter ruimte}{tussen de twee hulsjes}\\

\haiku{Bij mijn binnenkomst.}{draaide hij zich om en keek}{me afwachtend aan}\\

\haiku{Vlug overlegde ik.}{hoe ik mijn wens duidelijk}{zou kunnen maken}\\

\haiku{Ik zie de sport niet.}{van het op en neer kruisen}{op een plas water}\\

\haiku{Maar, ofschoon ik van,.}{goede wille ben ik vind}{er geen vreugde in}\\

\haiku{Ik zeg dan ferm en.}{vrijmoedig dat hij volgens}{mij ongelijk heeft}\\

\haiku{Natuurlijk is de,.}{bedoeling dat ik meega}{dat begrijp ik wel}\\

\haiku{Ze moet toch op de.}{lange duur haar eigen stem}{niet meer herkennen}\\

\haiku{Het water spoelt over.}{mijn gezicht en ik krijg een}{slok zout naar binnen}\\

\haiku{Ik hoor door een waas,,?}{iemand een woord zeggen ben}{ik het is zij het}\\

\haiku{We blijken in een,.}{duinpan te liggen midden}{tussen de struiken}\\

\haiku{Het dringt langzaam tot.}{me door dat er dorens in}{mijn benen prikken}\\

\haiku{Ik hoop dat Riet haar,.}{mond zal houden ik heb geen}{zin wat te zeggen}\\

\haiku{Riet heeft haar badpak.}{aangetrokken en is klaar}{om terug te gaan}\\

\haiku{Mijn huid is lichtrood,.}{gebrand ik wil in ieder}{geval uit de zon}\\

\haiku{Ik verlang naar de,.}{straten van de stad stenen}{onder mijn voeten}\\

\haiku{vind ik dat, daar moet.}{je echt dronken voor zijn om}{er op te komen}\\

\haiku{De conducteur geeft.}{mij een schalkse knipoog die}{ik niet retourneer}\\

\haiku{Ik zie heel goed in}{dat het feit niet meer van haar}{te houden reden}\\

\haiku{Ik zie hoe de man.}{probeert Riet dichter tegen}{zich aan te drukken}\\

\haiku{Vlug, zegt Riet, ga met,.}{me dansen anders komt die}{vent me weer vragen}\\

\haiku{Ik zou me bevrijd,.}{moeten voelen maar daar is}{niets van te merken}\\

\haiku{Je moet gewoon doen,}{had de dokter gezegd doe}{zoveel mogelijk}\\

\haiku{Lekker hard rijden,.}{de weg schiet onder je door}{en dan de botsing}\\

\haiku{Touw geknapt, hoorde,.}{je later en je kwam met}{je hoofd op de grond}\\

\haiku{Door een schok kan dit,.}{spoor uitgewist worden zo}{was het ongeveer}\\

\haiku{Kool hoort een vrouw au,,.}{roepen het mannetje grijnst}{wordt dan weer ernstig}\\

\haiku{Klinkert is weer op.}{zijn tree gaan zitten met zijn}{hoofd in zijn handen}\\

\haiku{{\textquoteleft}Je bent oud,{\textquoteright} zegt hij, {\textquoteleft} {\textquotedblleft}}{als iemand dus tegen jou}{zegtoude meneer}\\

\haiku{{\textquoteleft}Grandioos zeg, geen,,!}{belediging hahaha}{kostelijk enorm zeg}\\

\haiku{{\textquoteright} Boven aan de trap.}{sist de vrouw weer met haar tong}{tussen de tanden}\\

\haiku{Op haar en op de.}{buren ben ik al jaren}{uitgedomineerd}\\

\haiku{{\textquoteright} De vrouw antwoordt niet.}{en klemt haar lippen samen}{tot een smalle streep}\\

\haiku{Maar toen gingen er.}{doden vallen en dat vond}{ik verschrikkelijk}\\

\haiku{{\textquoteleft}Ik wil wel helpen,{\textquoteright}, {\textquoteleft}?}{zegt zemaar denk je dat het}{wat uit zal maken}\\

\haiku{Klinkert schrikt op van,.}{het geluid kijkt snel omhoog}{en ziet het gevaar}\\

\haiku{{\textquoteright} {\textquoteleft}Wat,{\textquoteright} zegt Klinkert, {\textquoteleft}god,,?}{Eef waarom heb je me dan}{niet even gewaarschuwd}\\

\haiku{Misschien een kleine?}{geestelijke afwijking}{of iets dergelijks}\\

\haiku{En, dat hebben ze,.}{ook allemaal gemeen ze}{zakken te ver in}\\

\haiku{Ik had anders de,{\textquoteright}, {\textquoteleft}.}{indruk zegt Kooldat dat niet}{de eerste keer was}\\

\haiku{Het is de wereld.}{die wij onder elkaar de}{vierde schil noemen}\\

\haiku{{\textquoteright} {\textquoteleft}Ja, we komen er,{\textquoteright}:}{aan roept Klinkert vrolijk en}{zich tot Kool wendend}\\

\haiku{{\textquoteright} Kool wringt zich volgens.}{de aanwijzingen op de}{aangegeven plaats}\\

\haiku{Op aanwijzingen.}{van Klinkert wordt het tempo}{langzaam opgevoerd}\\

\haiku{Hij wankelt en moet.}{met zijn hand steun zoeken aan}{de rand van het bed}\\

\haiku{Als iemand hier het.}{recht heeft zich te beklagen}{ben ik het toch wel}\\

\haiku{{\textquoteright} {\textquoteleft}Ja,{\textquoteright} zegt Kool, {\textquoteleft}ik ga,.}{nu naar huis goeiendag en}{bedankt voor alles}\\

\haiku{Hij heeft het eens in.}{een tijdschrift zien staan en hij}{was er verrukt over}\\

\haiku{En neem de manier!}{waarop hij me dwong met hem}{naar boven te gaan}\\

\haiku{De angst komt terug,.}{met moeite weet hij deze}{te onderdrukken}\\

\haiku{{\textquoteright} {\textquoteleft}Natuurlijk,{\textquoteright} zegt Kool, {\textquoteleft}.}{ik geloof alleen niet dat}{dat je opzet was}\\

\haiku{{\textquoteleft}Klinkert,{\textquoteright} zegt hij dan, {\textquoteleft}!}{geef me toch een verklaring}{waar ik wat aan heb}\\

\haiku{Jongen jongen,{\textquoteright} kreunt, {\textquoteleft},?}{Klinkertwat heb je gedaan}{wat heb je gedaan}\\

\haiku{Achteraf bezien.}{is deze middag niet slecht}{voor mij verlopen}\\

\haiku{Neuri\"end loop je.}{naar de keuken en  drinkt}{daar een glas water}\\

\haiku{In New York heb ik.}{me uiteindelijk bij zo'n}{groep aangesloten}\\

\haiku{Elke dag feesten,.}{met meiden en drank zoveel}{als je hebben wou}\\

\haiku{Hij geeft de baas een.}{teken de glazen van de}{ronde te vullen}\\

\haiku{Niemand geloofde,.}{dat ze gooiden de hoorn neer}{of lachten me uit}\\

\haiku{Ik huur een auto,.}{neem een gunstige plaats in}{en wacht meneer af}\\

\haiku{Soms moet ik zelfs een.}{opdracht weigeren omdat}{ik het te druk heb}\\

\haiku{Waarom loop je dan?}{niet weg als je te lang naar}{je zin moet wachten}\\

\haiku{Je berust, je staat,,.}{voor de deur in regen en}{wind wetend wachtend}\\

\haiku{Hij houdt zijn vreugde,.}{in een lachend gezicht zou}{nu geen pas geven}\\

\haiku{Hij gaat in zijn bank.}{zitten en kijkt verslagen}{naar de man voor hem}\\

\haiku{ik meed eveneens de.}{gruwelkelder waar weinig}{te gruwelen viel}\\

\haiku{Ik heb me altijd.}{afgevraagd wat andere}{mensen daar aantrekt}\\

\haiku{Die klanken trekken.}{me aan op een manier die}{ik zelf niet begrijp}\\

\haiku{Want denk niet, zei hij,.}{dat het een aangename}{gewaarwording is}\\

\haiku{Geleidelijk had.}{hij ook een zekere macht}{over mij gekregen}\\

\haiku{{\textquoteleft}U bent de enige.}{met wie De Galande op}{het ogenblik omgaat}\\

\haiku{We praatten nog wat,.}{besloten elkaar bij de}{voornaam te noemen}\\

\haiku{Vervelend voor een,.}{beginnend onderzoeker}{maar niets aan te doen}\\

\haiku{Toen ik dat zag kreeg,}{ik ook geweldige trek}{maar al mijn sloffen}\\

\haiku{Bovendien is het,.}{landschap heuvelachtig met}{hier en daar kloven}\\

\haiku{Achter zijn woorden.}{proef ik die fatale hang}{naar roem en glorie}\\

\haiku{Uren zaten we daar,.}{schipbreukelingen op een}{onbewoond eiland}\\

\haiku{Maar wat er ook mocht,.}{naderen ik zou er niet}{beter van worden}\\

\haiku{In haar ogen kon ik,.}{ook mijn gezicht zien ik keek}{in mijn eigen ogen}\\

\haiku{Ik voelde traag en,.}{week de angst opkomen een}{lauw dier in je hoofd}\\

\haiku{Als er dan wat tijd,.}{overheen is gegaan vergeet}{ik het zelf ook weer}\\

\haiku{Het opknappen van.}{de cellen moet van tijd tot}{tijd toch gebeuren}\\

\haiku{Mijn naam, die hem toch,.}{bekend moet zijn heb ik hem}{nooit horen noemen}\\

\haiku{Toch krijg ik niet de.}{indruk dat hij werkelijk}{onvriendelijk is}\\

\haiku{Ook denk ik wel eens.}{dat hij verlegen is met}{de situatie}\\

\haiku{Ik zou niet weten.}{wat te beginnen als ik}{er uit zou moeten}\\

\haiku{Dit ogenblik kan net.}{zo goed twee maanden of twee}{jaar geleden zijn}\\

\haiku{Het was een schande.}{dat de leiding van het huis}{hierin niet voorzag}\\

\haiku{Dan gaat hij terug.}{naar mijn cel en legt schone}{lakens op mijn bed}\\

\haiku{Het was vervelend,.}{het onderhoud was wel wat}{kort uitgevallen}\\

\haiku{Het raam staat nog steeds,.}{open zie ik zelfs dat hebben}{ze niet dichtgedaan}\\

\haiku{In heldere rust.}{keerde ik naar mijn lichaam}{en de pijn terug}\\

\haiku{Maar ik wil nog wel,.}{een uurtje blijven liggen}{op die oude boot}\\

\haiku{Dat we geen hopen.}{kinderen verwekten was}{onbegrijpelijk}\\

\haiku{Zwijgend keek ik haar,.}{na ook haar achterzijde}{was de moeite waard}\\

\haiku{Thuis pakte ik het.}{telefoonboek en zocht de}{bovenste naam op}\\

\haiku{Het was zaak om, zo,.}{lang het nog voor de wind ging}{de tijd te rekken}\\

\haiku{Ik besloot daar nog,.}{even mee te wachten het zou}{mee kunnen vallen}\\

\haiku{hier was de man zelfs,.}{niet te zien die stond zeker}{de vaat te wassen}\\

\haiku{Na een half uur hief.}{ik het hoofd op om eens om}{me heen te kijken}\\

\haiku{Zij vermoeden bij;}{hem geheime reserves}{die zij niet kennen}\\

\haiku{Hij staat op en gaat.}{naar de toiletruimte aan}{het eind van de gang}\\

\haiku{Om vijf uur haast hij,.}{zich naar huis zijn kin diep in}{zijn sjaal gedoken}\\

\haiku{Even later komt ze.}{er weer uit en houdt de deur}{uitnodigend open}\\

\haiku{{\textquoteleft}Dag meneer Jaichek,,.}{mijn naam is Voogt waarmee kan}{ik u van dienst zijn}\\

\haiku{{\textquoteleft}Meneer Jaichek,{\textquoteright} zegt, {\textquoteleft},}{hijlaten we even niet over}{dat ontslag praten}\\

\haiku{dit ter tafel te,.}{brengen zelfs al zou hij het}{van plan zijn geweest}\\

\haiku{Hij gaat zelfs zo ver;}{de oude heer lachend op}{de schouder te slaan}\\

\haiku{Steeds was hij bezig.}{ontslag te nemen om het}{te gaan proberen}\\

\haiku{Misschien heb ik wel,.}{eens iets in die geest gezegd}{maar ik zeg zoveel}\\

\haiku{Op zich zelf de meest.}{originele tekst die er}{te bedenken valt}\\

\haiku{Huilen, zeggen dat,.}{ik gelijk had dat ze het}{ook niet kon helpen}\\

\haiku{Na een tijdje ging,,.}{ze gekalmeerd weg en dat}{was het dan dacht ik}\\

\haiku{Kwam mijn bezoek niet,.}{gelegen dan zouden we}{wel weer verder zien}\\

\haiku{waarschijnlijk had ook.}{zij het leeglopen van het}{huis geobserveerd}\\

\haiku{wel weet ik nog dat.}{het op een Zuidzee-eiland}{gesitueerd was}\\

\haiku{Dat jij kijkt juist op.}{het ogenblik dat er wat met}{dat object gebeurt}\\

\haiku{Ben ik niet altijd.}{even aardig tegen dieren}{en oude mensen}\\

\haiku{Ik raap hem op, en.}{het blijkt een band te zijn van}{zo'n halve meter}\\

\haiku{Veel vertrouwen in;}{zijn navigatiekunsten}{heb ik daarom niet}\\

\haiku{En als zo vaak klopt,;}{het zelfs zo dat ik me een}{beetje bekocht voel}\\

\haiku{Een puts water over,.}{zijn hoofd zou helpen maar hoe}{daaraan te komen}\\

\haiku{Ik was er nog een;}{beetje te jong voor al vond}{ik het wel spannend}\\

\haiku{dat waardeerde ik,.}{al kwam je alleen maar om}{boeken te lenen}\\

\haiku{En toen was alles,.}{in \'e\'en klap afgelopen}{weer door mijn lachen}\\

\haiku{Ik neem afscheid, zeg}{hem dat ik over een dag of}{twee nog eens langs kom}\\
