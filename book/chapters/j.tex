\chapter[5 auteurs, 426 haiku's]{vijf auteurs, vierhonderdzesentwintig haiku's}

\section{A.M. de Jong}

\subsection{Uit: Frank van Wezels roemruchte jaren \& Notities van een landstormman}

\haiku{dat hij met zoveel.}{inspanning om zich heen had}{weten te krijgen}\\

\haiku{Zoiets moest je aan,.}{den lijve ervaren eer}{je er oog voor kreeg}\\

\haiku{'t Is al zo lang,.}{geleje da'k geen uniform}{om me donder had}\\

\haiku{{\textquoteright} {\textquoteleft}Make ze nooit meer,?}{behoorlijke soldaten}{van geloof jij wel}\\

\haiku{{\textquoteleft}O{\textquoteright}, vroeg Rengers, de, {\textquoteleft}?}{studentnoemen ze dat in}{het leger een bed}\\

\haiku{Zij staken de borst,:}{vooruit trokken de schouders}{vierkant en hoonden}\\

\haiku{{\textquoteleft}Waarom zijn we toch?}{zo gek om dat allemaal}{maar goedsmoeds te doen}\\

\haiku{Ik kan me toch niet?}{as een kleine jongen uit}{laten kankeren}\\

\haiku{en als het eenmaal, '.}{gezegd was wist jet voor}{je hele leven}\\

\haiku{ze schoven immers!}{alleen maar zover ze zelf}{geschoven werden}\\

\haiku{Och{\textquoteright}, antwoordde Frank, {\textquoteleft}'.}{t is eigenlijk toch een}{komieke wereld}\\

\haiku{Je bent een sekreet{\textquoteright},:}{grauwde Van Wezep en zijn}{vriend voegde er bij}\\

\haiku{{\textquoteright} Frank keek nadenkend.}{naar de slapende kleine}{jongen in de wieg}\\

\haiku{As je dienst weigert,.}{verzet je je tegen de}{bestaande orde}\\

\haiku{Langzaam wendde hij,.}{zijn ogen naar hun gezicht keek}{ze een voor een aan}\\

\haiku{{\textquoteleft}Wie weet waar we zelf. '.}{nog toe koment Is een}{treurige wereld}\\

\haiku{Nou goed de kolf in.}{je schouder drukken en niet}{bang zijn voor het schot}\\

\haiku{Dus salueerde,.}{hij maakte rechtsomkeert en}{ging het bureau af}\\

\haiku{{\textquoteright} En even later kwam,}{hij met een somber gezicht}{de kamer weer in}\\

\haiku{Hun tong lag verdroogd,.}{in hun mond en zij waren}{hongerig en moe}\\

\haiku{om uit te houden,.}{in de andere is geen}{redding mogelijk}\\

\haiku{een officier wordt,.}{geboren op de dag dat}{men hem be\"edigt}\\

\haiku{Ze vervloekten de.}{hele wereld in de zwartst}{denkbare termen}\\

\haiku{Zaterdagsmorgens.}{zouden ze met de eerste}{trein naar huis reizen}\\

\haiku{Het gezicht van de.}{kompagnieskommandant werd}{bijna menselijk}\\

\haiku{Hij dacht ook aan de.}{aanstaande promotie en}{wist zich verloren}\\

\haiku{Hij was nog steeds op,,.}{z'n hoede vertrouwde de}{zaak maar half en zweeg}\\

\haiku{Met een beetje goeie '.}{wil zullen wet onder}{mekaar wel vinden}\\

\haiku{hun leven, hun werk,,.}{hun vak hun denkbeelden op}{allerlei gebied}\\

\haiku{{\textquoteright} De schaduw van een.}{glimlach gleed heerlijk over de}{bruine gezichten}\\

\haiku{En ze spraken af,.}{dat ze doodeenvoudig niet}{mee zouden zingen}\\

\haiku{Dit strekte niet om.}{de woede der landstormers}{te doen bedaren}\\

\haiku{Jij dacht, dat ik je?}{nou elke week een spoorpas}{zou moeten geven}\\

\haiku{{\textquoteleft}Mag ik misschien ook,',?}{weten luit waarom u me}{hebt laten roepen}\\

\haiku{Het was de eerste.}{schrede op het pad naar de}{rang van officier}\\

\haiku{{\textquoteleft}Ga d'r eerst es as{\textquoteright},.}{een fatsoenlijk mens bijstaan}{gromde de stuurman}\\

\haiku{Wel, De Groot, as je,.}{niet boksen kunt wordt de zaak}{veel eenvoudiger}\\

\haiku{{\textquoteright} De overrompelden.}{keken stom verbaasd naar de}{vreemde snoeshanen}\\

\haiku{Maar hij voelde nog.}{meer voor een open brief aan die}{deftige burgers}\\

\haiku{Alleen het rooie blad:}{kwam er mee en dadelijk}{was Leiden in last}\\

\haiku{Jullie hadden geen.}{klachten mogen hebben over}{je inkwartiering}\\

\haiku{Maar verdomd lelijk '...?}{blijftet Is dat een bakkes}{voor een militair}\\

\haiku{Ik vind de stukjes{\textquoteright},.}{prachtig geschreven deelde}{Barten rustig mee}\\

\haiku{Hij greep onverwijld:}{zijn kwartiermuts van de bank}{en zei opgelucht}\\

\haiku{{\textquoteright} {\textquoteleft}Zeker, kolonel,,.}{maar ik geef er de voorkeur}{aan dat niet te doen}\\

\haiku{Hij verdomde het.}{nu zeker om een direkt}{antwoord te geven}\\

\haiku{Toen hij terug op,}{de kamer kwam wisten de}{kameraads het al.}\\

\haiku{je gauw ziek wordt, laat '.}{ik je van avond nog naart}{hospitaal brengen}\\

\haiku{Zouden de cellen?}{in de gevangenissen}{ook onverwarmd zijn}\\

\haiku{Hij kon precies langs,.}{de brits schuurde dan al met}{z'n arm langs de muur}\\

\haiku{Hij voelde zijn pols,,.}{lei de hand op zijn hart keek}{op z'n horloge}\\

\haiku{De redaktie zou.}{iemand sturen om alle}{bijzonderheden}\\

\haiku{Het toneeltje met.}{het Kind stond te vlak vooraan}{in zijn bewustzijn}\\

\haiku{Bijna allemaal.}{stamgasten van provoost en}{politiekamer}\\

\haiku{Ben je een haartje?}{belazerd om je de boel}{zo aan te trekken}\\

\haiku{Ik ben Tolman, neem,.}{me niet kwalijk da'k me niet}{eerst voorgesteld heb}\\

\haiku{{\textquoteright} Hij lachte dreunend,.}{en Frank lachte mee maar het}{ging niet van harte}\\

\haiku{k heb helemaal,.}{geen ethisch bloed of zo net zo}{min as principes}\\

\haiku{Waarachtig, kerel, ' '.}{t kan je opt laatst geen}{bal meer verrotten}\\

\haiku{Jij denkt te veel{\textquoteright}, zei, {\textquoteleft}.}{hij wijsgerigen v\'e\'el te}{veel aan anderen}\\

\haiku{En Frank van Wezels.}{hart was donker van weemoed}{en sombere haat}\\

\haiku{En nou leek het hem,.}{een onherbergzaam oord een}{echt verbanningsoord}\\

\haiku{Een eerste en een.}{tweede luitenant stonden}{bij hem te praten}\\

\haiku{Hij sloeg z'n hakken,:}{tegen mekaar richtte zich}{tot de kapitein}\\

\haiku{Jij zoekt een kwartier,.}{voor deze man waar ie een}{kamer alleen heeft}\\

\haiku{{\textquoteright} {\textquoteleft}Jawel, kaptein, 'k{\textquoteright},.}{zal m'n best doen beloofde}{de soldaatschrijver}\\

\haiku{Een man, vrouw en drie.}{kinderen zaten om de}{tafel en aten pap}\\

\haiku{{\textquoteleft}Ik geloof, dat ik ',?}{et niet slecht getroffen heb}{met de kompie wat}\\

\haiku{{\textquoteright} {\textquoteleft}Ja... nog al{\textquoteright}... {\textquoteleft}'k Weet.......}{er alles van uit de krant}{en uit de stukken}\\

\haiku{voor een maffie in,!}{de week onderhou ik je}{hele zwikkie hoor}\\

\haiku{{\textquoteleft}En donder nou m'n,, '!}{bureau af vaandrig en hou}{et je voor gezegd}\\

\haiku{{\textquoteleft}Ik laat me door geen,.}{ene brigges wegboksen al}{heit ie drie baarden}\\

\haiku{In plaats van een cent:}{schoof hij een dubbeltje naar}{het midden en zei}\\

\haiku{Z{\'\i}j zouen zich zo'n.}{buitenkansje niet hebben}{laten ontglippen}\\

\haiku{Maar... eh, majoor... eh...:}{kan ik van middag mijn pas}{niet krijgen alvast}\\

\haiku{Och, majoor, ik kan.}{me eigenlijk overal zo'n}{beetje in schikken}\\

\haiku{Je hebt natuurlijk,.}{gelijk dat je je kiezen}{niet van mekaar doet}\\

\haiku{Daarna zaten ze,.}{tot middernacht bij mekaar}{rookten en praatten}\\

\haiku{{\textquoteright} De stakker met zijn.}{pijnlijke benen keek hem}{vol afgrijzen aan}\\

\haiku{Dat was toch voor het,.}{rijk veel voordeliger en}{voor de man beter}\\

\haiku{{\textquoteleft}Kan je je misschien?}{zo'n geval uit de laatste}{tijd herinneren}\\

\haiku{{\textquoteleft}Na de zware dag.}{van gisteren kunnen ze}{dit niet meemaken}\\

\haiku{Toen ze buiten het,:}{dorp waren en de bossen}{in trokken vroeg Frank}\\

\haiku{{\textquoteleft}Op 't ogenblik is.}{Staak bezig de luitenant}{Dijkstra weg te pesten}\\

\haiku{Zijn hele gezicht}{straalde van voldoening en}{hij had een papier}\\

\haiku{Ze liepen samen:}{de dorpsstraat in en Frank viel}{met de deur in huis}\\

\haiku{Maar de laatste straf{\textquoteright},.}{is drie maanden geleden}{herinnerde Frank}\\

\haiku{{\textquoteright} Even zweeg Frank om z'n.}{antwoord beter tot z'n recht}{te laten komen}\\

\haiku{de mening van de,.}{mensen die zo'n geval in}{de kranten lezen}\\

\haiku{Ik zal er nog es.}{over denken en Toon es bij}{me laten komen}\\

\haiku{Frank was benieuwd, of.}{de boer hun niet een hapje}{zou laten mee eten}\\

\haiku{Mensen, die gozer '.}{is zo vies bij int open}{snijen van koppen}\\

\haiku{Frank was juist op het,.}{kompagniesbureau toen Jan}{binnengebracht werd}\\

\haiku{{\textquoteleft}Die hebben ze van,.}{me gestolen de vuile}{keleiradieven}\\

\haiku{En hij vergat om.}{te keren en is sedert}{dien nooit meer gezien}\\

\haiku{Maar een paar weken.}{geleden was ie tegen}{de lamp gelopen}\\

\haiku{Op de binnenplaats.}{waren de mannen bezig}{piepers te jassen}\\

\haiku{\'als dat tenminste...}{het eind was en hij nog niet}{tot erger verviel}\\

\haiku{{\textquoteleft}Ik neem \'o\'ok niet meer,.}{verantwoordelijkheid dan}{nodig is majoor}\\

\haiku{Er werd zes weken.}{militaire hechtenis}{tegen hem ge\"eist}\\

\haiku{De man was getrouwd,.}{had drie kinderen en een}{ziekelijke vrouw}\\

\haiku{dat ze behoorden,...}{bij het leven van eeuwen}{die voorbij waren}\\

\haiku{Dan kon er toch ook?}{geen sprake zijn van een zwaar}{gevecht in Limburg}\\

\haiku{{\textquoteright} {\textquoteleft}In zeker opzicht...{\textquoteright},.}{misschien gaf de luitenant}{met tegenzin toe}\\

\haiku{{\textquoteright} {\textquoteleft}Mag ik misschien ook,,?}{zien meneerde president}{welke twee dat zijn}\\

\haiku{Maar ze waren hier.}{op neutraal terrein en hij}{moest de pil slikken}\\

\haiku{De opleiding was.}{op het terrein gekomen}{en stond te rusten}\\

\haiku{{\textquoteleft}Maar wat heb je dan?}{te zeggen op wat ik naar}{voren gebracht heb}\\

\haiku{Ze moesten er dus maar.}{blijven en deelnemen aan}{de verdediging}\\

\haiku{toen wendde hij zich:}{tot de hospitaalsoldaat}{en zei lakoniek}\\

\haiku{En dat onder de!}{strenge kritische blikken}{van de kolonel}\\

\haiku{{\textquoteright} Franks gezicht werd \'e\'en.}{grote demonstratie van}{schrik en afgrijzen}\\

\haiku{{\textquoteleft}En nou gaan we es{\textquoteright},.}{een ronde over de kamer}{doen grinnikte Frank}\\

\haiku{Dat noemt ie rustig,!}{as-t-ie je pas op de}{bon geslingerd heit}\\

\haiku{{\textquoteright} vroeg de sergeant.}{van de week met een volmaakt}{onschuldig gezicht}\\

\haiku{Maar de wereld is.}{nou eenmaal ondankbaar en}{niet vlug van begrip}\\

\haiku{Nou kwam het er op.}{aan een smoes te vinden om}{verlof te krijgen}\\

\haiku{Maar al luidden de,.}{berichten ongunstig het}{doodsbericht bleef uit}\\

\haiku{{\textquoteright} {\textquoteleft}Maar je zult wel een{\textquoteright},.}{fatsoenlijk pensioentje}{krijgen troostte Frank}\\

\haiku{Maar de vent hield z'n.}{poot scheef en schinkelde als}{een kreupele rond}\\

\haiku{{\textquoteright}... {\textquoteleft}Laat ie de pokke!}{krijge met ze sjokkela}{en z'n sigaren}\\

\haiku{En begrepen niet,.}{eens hoe akelig doorzichtig}{hun spelletje was}\\

\haiku{H\'et moment was het.}{ontvangen van het werkpak}{en de kwartiermuts}\\

\haiku{dubbele gouden.}{streep en gouden kroon op de}{linkerbovenarm}\\

\haiku{De hitte was om,,,.}{ons overal overal en het}{was onverdraaglijk}\\

\haiku{geef me wat terug,.}{voor m'n goeie zorgen en ze}{maakt je officier}\\

\haiku{En toch - toch waren.}{we nauwelijks een maand lang}{bij elkaar geweest}\\

\haiku{Het was ergens op,,.}{een groot zandig terrein waar}{men aan athletiek doet}\\

\haiku{Ik heb nog nooit een.}{pijp zo wetenschappelijk}{langzaam zien stoppen}\\

\haiku{Ik zal dromen, dat.}{ik de kunst versta ook zo}{de lijn te trekken}\\

\haiku{'k Had de moord in,?}{en ik verrekte van de}{pijn maar wat doe je}\\

\haiku{Daar zou ik staaltjes,.}{van kunnen openbaren die}{opzien verwekten}\\

\haiku{Alles gaat vrij wel,.}{je ser vet zelfs valt nog al}{zo onhandig niet}\\

\haiku{Maar in eens heb je,.}{een stuk aardappel in je}{mond dat je niet smaakt}\\

\haiku{De schijtlijster het '.}{et lef niet om d'r mee voor}{de draad te komme}\\

\haiku{Daarmee waren \'en',?}{post \'en luit naar de weerga}{zou je zo zeggen}\\

\haiku{De nood was op 't,!}{hoogst gestegen maar nu was}{ook redding nabij}\\

\haiku{En toen verklaarde'.}{de luit dat zijn afdeling}{het gewonnen had}\\

\haiku{{\textquoteright} En als bij afspraak:}{stijgt uit de gelederen}{het gezang op}\\

\haiku{Het spijt me, dat dit,.}{tegen de krijgstucht is maar}{de levenstucht eist het}\\

\haiku{'t Ziet er nogal.}{vies uit en helemaal niet}{begerenswaardig}\\

\haiku{de eerste aanloop.}{tot de onthulling van m'n}{griezelig geheim}\\

\haiku{{\textquoteleft}Nee{\textquoteright}, zei hij koppig, {\textquoteleft},,,:}{nee k'praal daar m\`o je nou niet}{van praten ik zeg}\\

\haiku{Dientengevolge {\textquoteleft}{\textquoteright},.}{klaagt hij nooitover het eten of}{wat daarvoor doorgaat}\\

\haiku{is 's avonds terug '.}{gekomen van verlof in}{plaats vans middags}\\

\haiku{Hoe hij plotseling,:}{was gereduceerd tot nul}{tot minder dan nul}\\

\haiku{Dus spreekt het vanzelf,.}{dat je dagen te voren}{je komst aankondigt}\\

\haiku{Vaandrig zouden ze,.}{worden dat is zoiets als}{leerling-luitenant}\\

\haiku{Daar was er een bij,,.}{die woonde als burger dan}{vroeger in Den Haag}\\

\haiku{(Bij een dikte van!)}{1.3 cm namelijk kan de}{oorlog niet doorgaan}\\

\haiku{dat weet ik wel, maar,!}{wat duivel dan moeten ze}{maar niet gebeuren}\\

\haiku{Nou zullen ze hard...}{gaan werken en met plezier}{officier worden}\\

\haiku{De pijn lag op  .}{zijn verwrongen gezicht en}{hij steunde hoorbaar}\\

\haiku{Ik ontsloot de deur,.}{en kwam in een gang waar het}{nog heel donker was}\\

\haiku{De korporaal van.}{de week haalde de dekens}{en strozakken weg}\\

\haiku{En dat straf is om,.}{te verbeteren niet om}{te verbitteren}\\

\haiku{De kommandant van:}{de tegenpartij wenkt naar}{de onze en roept}\\

\haiku{Maar na een minuut.}{of wat komt de lenigheid}{een beetje terug}\\

\haiku{om mooie cijfertjes,.}{te halen want dat wil ik}{van jou ook altijd}\\

\haiku{Hier en daar, eenzaam,.}{ligt een stukje vlees in een}{wasblik te dromen}\\

\section{Max de Jong en Hans van Straten}

\subsection{Uit: Ik ben een echt genie. De briefwisseling van Max de Jong en Hans van Straten 1942-1951}

\haiku{Naar die juffrouw in ().}{Den Haag2 ga ik nietwillen}{ze thuis ook al niet}\\

\haiku{Misschien kan dit als.}{het af is een klein boekje}{op zichzelf worden}\\

\haiku{Ik zou je wel een,}{ex. willen sturen maar jij}{leest toch geen proza?35}\\

\haiku{een opeenhoping (...,...,... ).}{van korte bijzinnetjes}{wanneer als wanneer}\\

\haiku{Theosophie,?}{in zijn geheel moet dat niet}{zijn in haar geheel}\\

\haiku{Uit de la van een.}{kast kwam een schets tevoorschijn}{van een toneelstuk}\\

\haiku{Het is trouwens niet.}{eens nodig dit partijtje}{nog uit te spelen}\\

\haiku{Ik dacht natuurlijk, '.}{ook dat het niks was omdat}{hij int Fries schrijft}\\

\haiku{Ik stel Lehmann hoog,,.}{dat weet je maar Wadman Staat}{nu minstens even hoog}\\

\haiku{redactie heb ik!}{niet zoveel te vertellen}{als de vorige}\\

\haiku{{\textquoteleft}Wanneer het denken,.}{rondtolt en het hart breekt roept}{de mens om een god}\\

\haiku{Dat Neeltje je Heet.}{van de Naald slecht vond bewijst}{haar nuchter oordeel}\\

\haiku{D\`er Mouw moet erbij,.}{maar hij publiceerde niet}{in De Beweging}\\

\haiku{Zijn smaak is van een.}{hoogst boekenkasterige}{po\"ezie\"erigheid}\\

\haiku{We kunnen hier wel,.}{enkele concessies doen}{maar toch niet teveel}\\

\haiku{Engelman, Jet Holst,,.}{Campert en Van der Graft meer}{kunnen er niet bij}\\

\haiku{Zes uur per dag, en.}{als je ingewerkt bent kun}{je ook thuis werken}\\

\haiku{We zijn nu toch weer,.}{met De Neve bezig maar}{die wil pas in Sept}\\

\haiku{Hans, Ik d\'enk er niet, -.}{aan om over de kop te slaan}{maar ik z\'al meedoen}\\

\haiku{In de bl\'oemlezing,.}{moet Du Perron vanzelf w\'el}{dat kan jij dan doen}\\

\haiku{Adriaan van der Veen,,.}{heeft opgemerkt dat ik op}{Jezus lijk jasses}\\

\haiku{Ook met het oog op!}{de uitgave ervan is}{dat noodzakelijk}\\

\haiku{Hoe doen we nou met,?}{die bloemlezing maken we}{die nou af of niet}\\

\haiku{Overigens heb je.}{nog aardig wat boeken staan}{om te verpatsen}\\

\haiku{Zelfs alles-lezers.}{als Vermeulen en Schwithal}{lezen dat niet meer}\\

\haiku{Max        Amsterdam, [].}{ca. 13 januari 1948}{briefkaart Beste Hans}\\

\haiku{De anderen gaan,.}{hun gang daarom wil het ACC}{ook eens zijn gang gaan}\\

\haiku{van Roger Martin,.}{du Gard de geschiedenis}{van een bekering}\\

\haiku{Vroeger leek zoiets,.}{nog wel wat toen gebruikten}{ze er nieuw hout voor}\\

\haiku{Zeg nou zelf, moet je}{om van dat soort dingen zo}{lekker te worden}\\

\haiku{Documenteer je,?}{evenwel even of hoe maken}{we dat anders uit}\\

\haiku{Het bewijst weer eens, ().}{dat goed schrijven een kwestie}{van geldis tijd is}\\

\haiku{Daarbij schijnt de NVSH.}{jaarlijks een behoorlijke}{duit over te houden}\\

\haiku{Hoe staat het met die?}{fusie van Podium en}{Libertinage}\\

\haiku{Je moet eens op die.}{nieuwe Soc. Bew. letten van}{Frits Kief en Jef Last}\\

\haiku{1949 [briefkaart] Beste,.}{Max ~ Dinsdagmiddag kom}{ik naar Amsterdam}\\

\haiku{al dat verwerpen}{van waar je je nog niet mee}{bezig gehouden}\\

\haiku{de Jong en Rudie.}{van Lier worden er alleen}{belangrijker door}\\

\haiku{Het is maar een klein,.}{boekje en je kunt overslaan}{wat je niet bevalt}\\

\haiku{Ik denk maar dat ik,.}{fascist word dat komt straks weer}{erg in de mode}\\

\haiku{Schrijf me even hoe of,.}{wat dan weet ik weer waar ik}{beginnen moet}\\

\haiku{Dat je Liefde en.}{Goudvissen wilt gaan lezen}{is verdienstelijk}\\

\haiku{{\textquoteleft}Ik bracht de bloemen,.}{aan een andere vrouw die}{ik ook wel lief vond}\\

\haiku{10, 115, 117, 118, 141,,,,,, (-):}{145 153 168 169 171 Reydon}{Hermannus18961943}\\

\haiku{29[onder de brief heeft]:}{MdJ geschreven Rodenko}{stottert en is links}\\

\haiku{89Zie de noten.}{bij de brief MdJ van ca. 18}{januari 1947}\\

\section{Pieter Joossen}

\subsection{Uit: De kroniek van Pieter Joossen Altijt Recht Hout}

\haiku{Voort ist water van,}{de Nieuhaven geloopen}{duer de strate}\\

\haiku{En tot dien eynde}{hadde hy gesonden twee}{vaendelen Waelen}\\

\section{Sjouke Joustra}

\subsection{Uit: Vertrouw nooit een zeeloods}

\haiku{{\textquoteright} Op korte afstand.}{van Scheltema stopte hij}{en draaide zich om}\\

\haiku{{\textquoteright} De genoegzame.}{trek verdween van het gezicht}{van de Germaan}\\

\haiku{De kapitein liep.}{echter op de wandkast toe}{en deed de deur open}\\

\haiku{Hier de sleepboot voor,,?}{de Mitsu Maru wat is}{de bedoeling loods}\\

\haiku{We zwaaien zeker?}{met de kop in de richting}{van Cittershaven}\\

\haiku{De vierde sleepboot,{\textquoteright}.}{maakt op de haven vast klonk}{door de marifoon}\\

\haiku{Kunnen jullie bij?}{het loodswezen nu niet eens}{een planning maken}\\

\haiku{naar wat later de.}{invasie van Normandi\"e}{zou blijken te zijn}\\

\haiku{Frans Naerebout werd.}{op 30 augustus 1748 in}{Veere geboren}\\

\haiku{Naerebout geniet.}{slechts korte tijd van deze}{late waardering}\\

\haiku{Rondom het schip lag.}{niets anders dan een grauwe}{wollige massa}\\

\haiku{{\textquoteleft}Dat kan niet, of er,{\textquoteright}.}{moet nog volk beneden zijn}{mompelde Jacob}\\

\haiku{wat moeten wij doen,?}{nu de anderen het schip}{verlaten hebben}\\

\haiku{Doelbewust hield het.}{patrouillevaartuig op de}{Scheldemond aan}\\

\haiku{Vlagerige wind,,.}{uit het noordwesten kracht 8}{Bft zee onstuimig}\\

\haiku{{\textquotedblleft}Rivier,{\textquotedblright} antwoordde.}{ik schuchter en beledigd}{om het woord leerling}\\

\haiku{Het komt niet meer voor:}{dat men van jongsaf aan met het}{besef heeft geleefd}\\

\haiku{Het was stikdonker.}{en zware zee\"en beukten}{over beide schepen}\\

\haiku{Tegen de ochtend.}{ruimde de wind en ook werd}{de zee handzamer}\\

\haiku{{\textquoteleft}Veel schepen zullen,,.}{hier niet varen stuurman maar}{je kunt nooit weten}\\

\haiku{Volgens mij is zijn.}{positie nog ten minste}{een half uur te gaan}\\

\haiku{{\textquoteright} Het was de langste.}{zin die Jacob hem ooit had}{horen uitspreken}\\

\haiku{Jan besloot zich in.}{te laten schrijven voor de}{cursus eerste rang}\\

\haiku{Natuurlijk lagen.}{in de oorlog de lintjes}{voor het oprapen}\\

\haiku{It is time to}{go now Haul away your anchor}{Haul away your anchor}\\

\haiku{Ik kreeg orders om.}{honderd ton aardappels in}{Hansweert te laden}\\

\haiku{Hij was er zeker.}{van dat het de Congoboot}{voor hem zou worden}\\

\haiku{Hij knoopte zijn jas.}{dicht en verdween zonder groet}{naar de Roeierswacht}\\

\haiku{Nog voor de week ten.}{einde was bleek de staking}{verlopen te zijn}\\

\haiku{De rozenkrans was.}{nu van de ochtendjas naar}{zijn broekzak verhuisd}\\

\haiku{Langzaam maar zeker.}{naderde de sleep nu de}{roergang tot de sluis}\\

\haiku{Yes I told you you,.}{are a very good pilot an}{excellent pilot}\\

\section{Tjibbe Joustra}

\subsection{Uit: Superpop}

\haiku{Zo langzamerhand.}{zijn er vandaag alleen maar}{andere kanten}\\

\haiku{Mijn moeder zegt dat.}{ze water genoeg gezien}{heeft van de zomer}\\

\haiku{Wij staken de straat.}{over om ons bij de lui naast}{de stok te voegen}\\

\haiku{Ze springt erin als.}{ik en komt met een vlugge}{zwemslag op me af}\\

\haiku{Ze proberen over,.}{hun schaduwen te springen}{wat niet lukken wil}\\

\haiku{We komen langs een.}{tot monument verklaarde}{openbare belcel}\\

\haiku{We pakken vlug zijn.}{zak en goed op en zwaaien}{er uitbundig mee}\\

\haiku{We troffen het wel.}{en trokken ons terug in}{een kleine ruimte}\\

\haiku{Wanneer ze langs de.}{wagen van de sr komt is}{viervoet verdwenen}\\

\haiku{Het verkeer golft in.}{een onafgebroken stroom}{langs het drietal heen}\\

\haiku{Ik geef hem een waai,.}{voor zijn kop wil hem een waai}{voor zijn kop geven}\\

\haiku{Mar en ik uit het.}{vorige moment over naar}{dit nieuwe ogenblik}\\

\haiku{Rol buigt zich over de.}{rand van de kade waar een}{vlot tegenaan drijft}\\

\haiku{Kris rammelt met het.}{opgepakte in haar hand.}{Gooi dan als je durft}\\

\haiku{De wagen is paars,.}{de overal van de jongen}{die er uitstapt ook}\\

\haiku{Mar gaat eens kijken,.}{terwijl de jongen de deur}{voor haar openhoudt}\\

\haiku{) Dank je wel, zeg ik,.}{wij zijn hier niet gekomen}{om weer weg te gaan}\\

\haiku{En zo pakt hij de.}{doos op en weet de deur van}{de zaak te vinden}\\

\haiku{En ja, voor niets gaat.}{de zon hier zeker als een}{jojo op en neer}\\

\haiku{Hoe uitgestrekt kan.}{een plek zijn om de plaats heen}{waar je naar toe wilt}\\

\haiku{Uitkijken naar wat.}{ons verder brengt zal hier niet}{makkelijk worden}\\

\haiku{Mar probeert ondanks.}{de zichtbeperking om ons}{heen te kijken}\\

\haiku{Een wegglijdende.}{spiegelende schim aan de}{rand van gedachten}\\

\haiku{Nou dank je hoor, wat,.}{ben jij knap zeg zeggen we}{tegen het meisje}\\

\haiku{Ontzettend aardig,,,.}{zeg maar we moeten weg zie}{je we zijn al laat}\\

\haiku{De fig klikt om de,}{hoek van de deurpost het licht}{te voorschijn onthult}\\

\haiku{Mar, wil ik zeggen,,.}{draai mij om maar ook Mar is}{nergens meer te zien}\\

\haiku{Ik ontdek dat de.}{mist opgelost is en de}{avond nacht geworden}\\

\haiku{Klauwt haar vingers, grist.}{met een razendsnelle graai}{de maan uit de lucht}\\

\haiku{Dichte nevel hangt.}{boven de vlakte en er}{is niets meer te zien}\\

\haiku{Doordat onze vriend,.}{ophoudt met trappen want de}{weg gaat hier omhoog}\\

\haiku{Mar slaat een arm om,.}{mijn schouders ik sla een arm}{om de jongen heen}\\

\haiku{Onderwijl kunnen ().}{enige zaken onsgeestes}{oog gaan passeren}\\

\haiku{Mar fluistert in mijn,.}{oor ik kijk naar mijn glas wat}{inderdaad leeg is}\\

\haiku{Voordat het meisje,,.}{op de laadvloer klautert kijkt}{ze even opzij lacht}\\

\haiku{De laatste spitst zijn.}{lippen als hij denkt dat het}{meisje het niet ziet}\\

\haiku{Das gooit dat wat hij.}{gevonden heeft weg ergens}{tussen de struiken}\\

\haiku{Mar weet haar kamer,.}{te vinden het slot op de}{tast open te krijgen}\\

\haiku{Ze drukt zich langzaam.}{omhoog tegen de gladde}{groenbemoste schors}\\

\haiku{In een tel zijn de.}{takken op dit deel van het}{grasveld opgeruimd}\\

\haiku{Leeg, op de schipster,.}{na en zonder rechts zie je}{niet wat je links ziet}\\

\haiku{De deining verzwakt,.}{tot een nauwelijks merkbaar}{dobberen tot niets}\\

\haiku{De jongen draait zich,.}{om gooit de kaart terug in}{de bak van de fiets}\\

\haiku{Misschien een kans om.}{van een ogenblik een ander}{moment te maken}\\

\haiku{Verwonderd kijkt de,.}{fietser om zich heen wie weet}{dat hij op straat zit}\\

\haiku{Kijk nog eens goed, het.}{is duidelijk dat dit daar}{heel iets anders is}\\

\haiku{Opgewonden staan.}{we met ons vijven op het}{dak om het blik heen}\\

\haiku{In de ramen aan.}{mijn kant wordt de wereld aan}{mijn kant weerspiegeld}\\

\haiku{Neuri\"end loop ik,.}{over mijn dak met mijn voeten}{spullen verschuivend}\\

\haiku{wat me geschikt lijkt,.}{om iets in neer te leggen}{in op te bergen}\\

\haiku{Met een klap die over.}{het water weg stuitert slaat}{daarop het raam dicht}\\

\haiku{Ik gris rondom me,.}{heen allerlei rommel van}{het dek ga gooien}\\

\haiku{Ik zou in lachen.}{willen uitbarsten maar laat}{dit achterwege}\\

\haiku{Ik zwaai met de top,.}{in mijn hand heen en weer wat}{zal ik besluiten}\\

\haiku{Hee, hoor ik Jansen,.}{naast mij zeggen ga je iets}{voor ons opvoeren}\\

\haiku{Nee, Jonna roert met.}{de staaf van ijzer in de}{inhoud van het blik}\\

\haiku{Doggo kom op, steek.}{je struisvogelpolitiek}{eens door dit open gat}\\

\haiku{Maar nee hoor, het komt,.}{wel dieper te liggen maar}{blijft keurig drijven}\\

\haiku{het minder moeilijk,,.}{springen van het dak op het}{dek elk aan een kant}\\

\haiku{Ik, Doggo, strijk met.}{mijn hand door mijn haar of over}{een stugge borstel}\\

\haiku{En de handmeester,.}{die ik er uithaal papier}{is geduldig}\\

\haiku{Ik ontdoe mij van,.}{mijn meesteres stop deze}{terug in de tas}\\

\haiku{Ik kijk nog net op.}{tijd op om te zien dat het}{op mij gericht was}\\

\haiku{Ratta staart naar het,.}{raakvlak van zijn voeten met}{de straat ik staar mee}\\

\haiku{Bijvoorbeeld in een.}{verhaal dat bijna niet na}{te vertellen is}\\

\haiku{Ik houd mijn glas op.}{ooghoogte en kijk er door}{naar een stad in glas}\\

\haiku{Achter mij zegt een.}{reiziger in vertaalde}{taal goed meisje}\\

\haiku{Mijn meisje pakt de,.}{bril het meisje zet mijn bril}{op onder bijval}\\

\haiku{Op de rand van de.}{kade staat een deinende}{figurantenbrij}\\

\haiku{Ik doe of ik zweet,,}{het zweet van mijn voorhoofd veeg}{verschuif met mijn voet}\\

\haiku{Doggo, luister, dat,}{is om de vaargeul uit te}{diepen hoor je dat.}\\

\haiku{De boot drijft traag rond,.}{onder een wolk van woorden}{tot  ze dwars ligt}\\

\haiku{Een verklaring brengt.}{je trouwens zelden dichter}{bij de oplossing}\\

\haiku{Met een boog gooi ik,,.}{de toeter over mijn schouder}{weg de gracht in}\\

\haiku{Jonna probeert de.}{opengesprongen parasol}{weer dicht te vouwen}\\

\haiku{Ik steek over nadat,.}{alles voorbijgegaan is}{en beklim de brug}\\

\haiku{Hoe komt het toch dat.}{sommige dingen soms wel}{breken en soms niet}\\

\haiku{Weg er mee, dit  ,.}{maakt mij dorstig of het doet}{mij mijn dorst voelen}\\

\haiku{Ik veer overeind, doe,.}{mijn ogen open knipper tegen}{het felle zonlicht}\\

\haiku{Hij zet het achter.}{de parasol en de palm}{rechtop op het strand}\\

\haiku{In mijn ongeduld.}{duw ik hem bijna voor een}{langs rollende tram}\\

\haiku{Links naast mij ligt een,.}{gitaar op de straatstenen}{rechts een racefiets}\\

\haiku{Iemand loopt met haar.}{reiszak tegen de gitaar}{in mijn linkerhand}\\

\haiku{Vind ik het niet vreemd,.}{dat ik hem had en dat hij}{paste de sleutel}\\

\haiku{Zo'n gitaar op je.}{rug heeft eigenlijk toch wel}{iets compleets vind ik}\\

\haiku{Ze klinken na in,,.}{de gitaar net als de bel}{van de tram exact zo}\\

\haiku{Bereik het punt van.}{de straat waar ik deze zal}{moeten oversteken}\\

\haiku{Ik draai mij om om,.}{ergens naar te gaan zoeken}{wat weet ik nog niet}\\

\haiku{Met schorre keel schreeuw.}{ik de tonen van deze}{improvisatie}\\

\haiku{hoe zit het met de,.}{uitgestrektheid strekt het mee}{of strekt het tegen}\\

\haiku{Een eind voor mij uit.}{is iets te zien wat opeens}{weer verdwenen is}\\

\haiku{Mijn lippen sluiten,.}{zich gespannen kijk ik naar}{een luchtspiegeling}\\

\haiku{Ik knijp mijn ogen dicht.}{om beter naar Majjo}{te kunnen kijken}\\

\haiku{ja, daar, ja die kant,,}{zie je dat meisje zonder}{die gitaarkoffer}\\

\haiku{De welving van haar,.}{buik verlokkend priemen van}{vuurrode tepels}\\

\haiku{Een eind verder loop.}{ik likkend aan in de zon}{en uit de drukte}\\

\haiku{Voordat het beeld met.}{hem mee rent probeert ze aan}{mijn staart te trekken}\\

\haiku{Maak een vaag gebaar.}{terug dat ergens bij het}{begin blijft steken}\\

\haiku{Nou, nee, laat maar, maar.}{heb je toevallig geen kist}{om op te zitten}\\

\haiku{ik heb er trouwens,,.}{alleen maar even op gestaan}{niks gepoetst plaatstaal}\\

\haiku{Ik krabbel overeind,.}{laat een wind en vind een plek}{om op te schijten}\\

\haiku{Ik rek mij uit en.}{voel met mijn handen aan de}{palmboom op mijn hoofd}\\

\haiku{Okee goed jij wint,  ,.}{brengt hij moeizaam uit maar mag}{ik nog even voelen}\\

\haiku{Oh schone met je,.}{palm je zoete lippen en}{je volle tieten}\\

\haiku{Vanuit het water,,}{gespartel en geproest schijt}{stik verdomme blub}\\

\haiku{Onderzoekend kijkt,,.}{ze nog steeds in gebukte}{houding om zich heen}\\

\haiku{Wat jij daar hebt is,}{geen voorwerp maar alleen maar}{iets wat er op lijkt}\\

\haiku{De warme, hijgend.}{uitgeblazen voelbare}{adem van Fatala}\\

\haiku{Fanata met je.}{afdalende wielen en}{je zekere zit}\\

\haiku{De weg vrijgemaakt,.}{voor als de drie meisjes nog}{terugmoeten straks}\\

\haiku{Van de hellingkant.}{van de brug komen nu twee}{meisjes aanfietsen}\\

\haiku{Trams, auto's, fietsers.}{en voetgangers trekken op}{naar de binnenstad}\\

\haiku{Totala grinnikt,.}{Fatala beziet de stad}{en de lucht zo blauw}\\

\haiku{Totala ziet het.}{weerspiegeld in de glazen}{van haar zonnebril}\\

\haiku{Uit het portiek waar.}{ze nu langslopen komt een}{meisje naar buiten}\\

\haiku{Nou, ik ga naar Mar,.}{onze eigen avonturen}{beleven weet je}\\

\haiku{Ja nee, een goed plan,,,.}{hoor maar eh ik heb ook een}{idee zegt Labarbra}\\

\haiku{Weet je dat ik het,.}{al een keer uitgeprobeerd}{heb zegt Labarbra}\\

\haiku{Een van de wielen.}{blijft steken in de railgoot}{van een kadekraan}\\

\haiku{Labarbra duwt haar.}{handen in de zakken van}{haar fantasiejurk}\\

\haiku{Meisje rijdt rondjes,,.}{straatorkest aan de overkant}{van de straat auto's}\\

\haiku{Ze haalt het mes licht,.}{drukkend even heen en weer trekt}{het dan iets terug}\\

\haiku{Volg mij, want wij gaan,.}{opstijgen roept hij tegen}{het stadslawaai in}\\

\haiku{Wat kan hij in zijn.}{schild voeren onder de rest}{van de sluier}\\

\haiku{Weet je, wanneer ze,.}{maar vroeg genoeg met vliegen}{beginnen wie weet}\\

\haiku{Onnono tikt de.}{vermaker van zich af die}{achteruit tuimelt}\\

\haiku{Jammer alleen voor,.}{haar dat het water leeg}{is zegt Fatala}\\

\haiku{De vraag is nu, denkt,:}{Fanata welke boodschap}{hier het best overkomt}\\

\haiku{Gek dat Totala.}{al zo ver is dat ik hem}{nergens meer zien kan}\\

\haiku{Bij het uitspreken.}{van deze woorden heeft zij}{haar ogen gesloten}\\

\haiku{Hij wil haar op de.}{zich dwaas terugtrekkende}{figuren wijzen}\\

\haiku{Fatala wil naar,.}{het heft van haar mes grijpen}{maar ziet er vanaf}\\

\haiku{je boft nog dat je.}{hier niet net zo vaak staat als}{dat ik hier langskom}\\

\haiku{Wacht, onderbreek me,.}{niet er staat nog meer op gaat}{Totala verder}\\

\haiku{Onnono barst in,.}{schateren uit verslikt zich}{in een hap gras}\\

\haiku{Animee geeft hem een,.}{duw zodat hij in de berm}{zal terechtkomen}\\

\haiku{Animee wijst met haar.}{vinger grillige wegen}{over het kaartoppervlak}\\

\haiku{Het meisje buigt zich,.}{nog wat verder naar buiten}{lacht naar beneden}\\

\haiku{Het ding zeilt een eind,.}{door de lucht komt ergens net}{buiten beeld terecht}\\

\haiku{Animee bukt zich om.}{haar op de grond liggende}{hamer te pakken}\\

\haiku{Ze houdt haar adem in,.}{en luistert hoort Fatala}{een stem fluisteren}\\

\haiku{Vanuit de richting.}{van de stad komt niets naar hun}{eiland toe varen}\\

\haiku{Plotseling blijft ze,.}{roerloos liggen laat haar hoofd}{op de grond rusten}\\

\haiku{Fatala springt op,,.}{laat haar ogen over haar lichaam}{over het eiland gaan}\\

\haiku{Hee Onnono, een,.}{nieuw gezichtspunt roept hij naar}{de uitzichteter}\\

\haiku{zijn schouders op, maakt,.}{een gebaar of hij er iets}{overheen gooit wegwerpt}\\

\haiku{Logisch, want als het.}{andersom zou zijn zou ik}{andersom zitten}\\

\haiku{Hoe ziet vanaf de.}{overkant iemand er uit die}{naar de overkant kijkt}\\

\haiku{Fanata doet een,.}{stap naar voren schopt er met}{haar voet tegenaan}\\

\haiku{scandeert Onnono,.}{zijn handen als een roeper}{om zijn mond houdend}\\

\haiku{En, zegt Fanata,.}{wil je je liefdesverdriet}{soms gaan begraven}\\

\haiku{Juist, knikt Onnono.}{en probeert gevaarlijk te}{doen op de dakpunt}\\

\haiku{Onnono schudt zijn,,.}{hoofd flarden tekst wolken op}{dwarrelen omlaag}\\

\haiku{zich heenkijkend of.}{er iets is om zich koelte}{mee toe te wuiven}\\

\haiku{Totala kijkt  .}{naar Fatala's gezicht van}{onderen gezien}\\

\haiku{Zij knippert met haar.}{ogen tegen het felle licht}{van de nabije zon}\\

\haiku{De ontmoeting, denkt,.}{Animee evaluatie voor}{meer dan een persoon}\\

\haiku{Aanval vanuit de,,.}{lege ruimten de lege}{tijd denkt Fatala}\\

\haiku{Bloed, denkt Animee, ik.}{zou natuurlijk ook wat voor}{haar kunnen pissen}\\

\haiku{Het licht beweegt in.}{haar ogen mee op het ritme}{van haar benen}\\

\haiku{De meisjes raken.}{elkaar tegelijk met hun}{vingertoppen aan}\\

\haiku{Fatala loopt om.}{het pak stenen heen naar de}{kant van het water}\\

\haiku{Haar haren vegen.}{bij het bewegen van haar}{hoofd over de vloer stof}\\

\haiku{De jongen houdt met.}{een aangedikt gebaar zijn}{hand achter zijn oor}\\

\haiku{Zonder dat hij op.}{het idee is gekomen naar}{boven te kijken}\\

\haiku{De samenloop van,.}{uitgesproken gedachten}{zegt Labarbra's stem}\\

\haiku{Daarachter de straat,.}{vol auto's fietsen en trams}{die het plein verdeelt}\\

\haiku{Onnono legt zijn,.}{handen in zijn nek spant de}{spieren van zijn borst}\\

\haiku{Samen tikken zij.}{een ingewikkeld ritme}{op de straatstenen}\\

\haiku{Zonder grip op dat,.}{wat gebeurd is alles wat}{nog moet gebeuren}\\

\haiku{Omdat ik wel zie,.}{dat je haast hebt hoor zegt hij}{tegen Fanata}\\

\haiku{Sorry maar weet jij.}{misschien de weg naar boven}{of naar beneden}\\

\haiku{dat men heeft ziet er.}{alleen daarom zo al een}{stuk draaglijker uit}\\

\haiku{Allicht ja, logisch,.}{dat je hier naar beneden}{kijkt waar anders naar}\\

\haiku{Wiens hamer's steel om.}{alles in het straatzand op}{te tekenen}\\

\haiku{voorbijgaat zonder,.}{daartoe uitgenodigd te}{zijn gaat hij verder}\\

\haiku{Wie zet zo'n ding ook.}{op de gang zonder er zelf}{naast te gaan liggen}\\

\haiku{Zij weet nog net te.}{voorkomen dat de klanken}{haar mond verlaten}\\

\haiku{Labarbra gaat naast.}{Animee voor het grachtkant raam}{van het caf\'e staan}\\

\haiku{Toen zij fluisterde.}{maar jij mijn eindeloze}{tong van lik en taal}\\

\haiku{Wat die zijn dus, die,.}{inwoners in het genot}{van alle rechten}\\

\haiku{Wie of wat bepaalt.}{het groeien van de ruimte}{tussen de muren}\\

\haiku{Langs de randen loopt.}{een galerij die aan de}{buitenkant open is}\\

\haiku{His feet sinking away,.}{in the burning hot loose sand}{Yessimo toils on}\\

\haiku{He stoops to pick up,.}{the book but it lies a few}{meters further on}\\

\haiku{The question is,,}{Yessimo says as if lost}{in thought where're}\\
