\chapter[7 auteurs, 612 haiku's]{zeven auteurs, zeshonderdtwaalf haiku's}

\section{Karel van den Oever}

\subsection{Uit: Het inwendig leven van Paul}

\haiku{Het begin dezer.}{werkdadige liefde had}{voor Paul geen einde}\\

\haiku{de lucht Delftsch-blauw,.}{Beider soort  zichtbaarheid}{was bijna identiek}\\

\haiku{Vo\'or dat Paul de kerk,.}{intrad had hij reeds den smaak}{harer heiligheid}\\

\haiku{de dag nadien las {\textquoteleft}{\textquoteright};}{hijDe ware Wijnstok van}{Bonaventura}\\

\haiku{{\textquoteleft}Und eine neue Welt{\textquoteright}:}{entspring auf Gottes Wort zag}{Paul de kleine Haydn}\\

\haiku{Paul overdacht hoe het}{aardsch leven \'een ren was naar}{God en hoe ijdel}\\

\haiku{bloedkegels hingen;}{stijf uit de doornen-kroon}{langs zijn aangezicht}\\

\haiku{De diepte hief haar{\textquoteright}.}{hand op herinnerde zich}{Paul een psalm-vers}\\

\haiku{als voorheen zag hij;}{een boter-gele wolk}{rooken voor de zon}\\

\haiku{De heide lag er,.}{gedroomd en onwezenlijk}{bijna stoffeloos}\\

\haiku{de {\textquoteleft}Ark des Verbonds{\textquoteright}, {\textquoteleft}{\textquoteright}, {\textquoteleft}{\textquoteright}:}{deGeestelijke Roos de}{Toren van David}\\

\haiku{Zijn eucharistisch}{verlangen was zoo fel dat}{hij bereidwillig}\\

\haiku{hij vreesde ook de {\textquoteleft}{\textquoteright}.}{schaduwen der buien die}{verkoeling brachten}\\

\haiku{een kleine inzet,.}{op het eeuwig Leven een}{hervormde voor-spijs}\\

\haiku{Paul, die nu schuilde,:}{onder een portiek zegde}{luid-op den psalmtekst}\\

\haiku{De menschen moesten het,,.}{Vagevuur begeeren meende}{Paul en niet vreezen}\\

\subsection{Uit: Kempische vertelsels}

\haiku{me dunkt zelfs dat ik}{heur goudklaar taaiken tot hier}{in mijn ooren hoor}\\

\haiku{mutseken sloeg er.}{eentonig op en neerewaarts}{in een zot bedrijf}\\

\haiku{{\textquoteright} en Peere sufte, den,...}{denkenden kop schuddend naar}{zijnen leegen stoel weer}\\

\haiku{Daar was een vrouwken ';}{dat spon Giele giele gon}{zoo hard datt kon}\\

\haiku{{\textquoteleft}Eh, vrouwken, schei toch!}{eens uit en laat astemblieft}{toch dat licht branden}\\

\haiku{{\textquoteright} zei hij eindelijk {\textquoteleft}',... '}{t zijn altijd wel zware}{patatten vrouwken}\\

\haiku{toen kwam de morgen...}{op in den Oosten met een}{stille klaarte}\\

\haiku{dempigen mist die.}{alle geluid en doening}{heimelik verdook}\\

\haiku{blauwen kiel die dan.}{omhoog en neersloeg lijk een}{klepperende vlag}\\

\section{Frans van Oldenburg Ermke}

\subsection{Uit: Limburg aan de galg. De legende van de Bokkerijders en de geschiedenis van hun lot}

\haiku{Om hem werden de.}{kraaien en raven niet in}{hun maaltijd gestoord}\\

\haiku{Doch wie arm is, heeft.}{de plicht het te blijven tot}{de dood erop volgt}\\

\haiku{En de koning is,.}{steeds een vr\'e\'emde koning de}{vrouw een vr\'e\'emde vrouw}\\

\haiku{Waarom stond je ook,?!}{niet op toen de hond bij de}{buren zo blafte}\\

\haiku{Een kermis is er,...}{niets bij en de marskramers}{doen goede zaken}\\

\haiku{Doch spoedig daarna.}{al was er weer en b\'eter}{werk aan de winkel}\\

\haiku{En zij die de wacht,.}{hielden wachtten gespannen}{op het eerste alarm}\\

\haiku{Met een stevig stuk -!}{hout had zekere Andries}{je moest hem kennen}\\

\haiku{naar Arret L\"utgens -}{in Bank bij Kohlscheid opbrengst}{drie rijksdaalders}\\

\haiku{Wie het in slechte,.}{tijden niet slecht gaat moet op}{alles verdacht zijn}\\

\haiku{En er is geen dief,.}{zo groot of hij acht de drost}{een n\`og grotere}\\

\haiku{Want je kunt nooit z\'o,!}{arm zijn of door zo'n oorlog}{word je nog armer}\\

\haiku{Maar er is altijd:}{het zekere weten dat}{eens het einde komt}\\

\haiku{Het was zijn eerste,.}{glaasje maar zijn laatste zou}{het nog lang niet zijn}\\

\haiku{\'o\'ok afgetrokken,.}{bekeerd en beschaamd en als}{betere mensen}\\

\haiku{Hij zwoegt op akkers.}{waar het graan wast dat voor hem}{niet wordt gemalen}\\

\haiku{in het zwert neffens;}{hem eenen tweeden met eenen boek}{in zijne handen}\\

\haiku{een list sedertdien.}{door menighe filou met}{goet gevolgh herhaeld}\\

\haiku{Die werd gepakt, op,.}{23 januari 1751 te}{Landsraad-Gulpen}\\

\subsection{Uit: Pastoor Pius Paerel. Een geschiedenis zonder moraal}

\haiku{Oud-pastoor.}{Hendriks is altijd een goed}{predikant geweest}\\

\haiku{Ik droeg den koster.}{op om een briefje bij je}{in de bus te doen}\\

\haiku{En aan het hart van.}{het volk zelf legde hij zijn}{oor te luisteren}\\

\haiku{Adriaan van Harte:}{glimlachte en antwoordde}{alleen maar met een}\\

\haiku{De gastheer lachte.}{dreunend en stak een vinger}{dreigend naar hem op}\\

\haiku{Zij scheen het drinken.}{van koffie zoo'n groote zonde}{nog niet te vinden}\\

\haiku{- Het doet den mensch goed,,! - {\textquoteleft},!}{te hooren dat de mensch slecht}{isO die menschen}\\

\haiku{{\textquoteright} vroeg Joachim nu, door.}{zooveel vroolijkheid aan het}{twijfelen gebracht}\\

\haiku{Het was inderdaad:}{een oogenblik geweest om}{nooit te vergeten}\\

\haiku{{\textquoteleft}Nog \'e\'en maand les bij!}{z\'o\'o'n man en het ventje is}{voorgoed bedorven}\\

\haiku{{\textquoteleft}Toen ik nog zong in....}{het Cecilia-koor}{te Rammelrade}\\

\haiku{Jongens, wat er ook,...!}{gebeure we blijven bij}{elkaar en vr{\`\i}enden}\\

\haiku{Maar zijn verloofde.}{was  inderd\'a\'ad van z\'e\'er}{goede familie}\\

\haiku{Voor een retourtje!}{van zeven gulden vijftig}{heb ik mijn kerk vol}\\

\haiku{{\textquoteleft}Dat zou niet gek zijn,{\textquoteright}, {\textquoteleft}.}{antwoordde Joachimal was}{het ook de \'e\'erste keer}\\

\haiku{Hij droomde van steenen.}{engelen en een hemel}{van verschoten zij}\\

\haiku{{\textquoteleft}'t Is wel geen  ,{\textquoteright}, {\textquoteleft}.}{Afrika lachte Joachimmaar}{toch bijna Parijs}\\

\haiku{Maar boven zijn hoofd.}{dreunde het geronk van den}{slapenden koster}\\

\haiku{Ze hadden nachtdienst,.}{gehad maar wilden deze}{kans niet verzuimen}\\

\haiku{En  zelfs van de.}{bruikbare kon men niets met}{zekerheid zeggen}\\

\haiku{{\textquoteright} Het was duidelijk,.}{dat hij dit alles niet voor}{het eerst vertelde}\\

\haiku{{\textquoteleft}Het lijkt meer op een,{\textquoteright}.}{luchtschip met een strijkje aan}{boord grapte Joachim}\\

\haiku{{\textquoteright} vloekte Adriaan, die.}{over Goedeltje's dekschild was}{komen struikelen}\\

\haiku{Het werd zomer en.}{op het oksaal hoopten stof}{en modder zich op}\\

\haiku{Hij laat den heeren.}{verzoeken toch vooral niet}{op hem te wachten}\\

\haiku{Maar gelukkig riep {\textquoteleft}!}{Pastoor Pius Paerel}{zelf zoo hardHoera}\\

\section{Alfons Olterdissen}

\subsection{Uit: Prozawerken in Maastrichtsch dialect}

\haiku{{\textquoteleft}Van stad en lui veur{\textquoteleft}.}{50 jaor I. Algemeen}{voorkomen der stad}\\

\haiku{vestingm\"or, wie mien {\textquoteleft}{\textquoteright}.}{grameer in h\"a\"orecrinoline}{m\`et nege reipe}\\

\haiku{Es te slachter de,}{boer op trok um bieste te}{koupe dan k\'oste z'em}\\

\haiku{En de waor st\'ont.}{minder in de vitrien en}{mie in de winkel}\\

\haiku{Me k\'os ze kriege.}{van de Vuelta Abajo}{aon tien sent te tien}\\

\haiku{Alloh, in eeder.}{geval heele ze leve}{in de brouwerij}\\

\haiku{Iers kraoge ze vrij}{en k\'oste ze speule op te}{groete speulplaots}\\

\haiku{Kwaom heer em, neet hoole,}{dan marsjeerde de v\`elder m\`et}{ze slachoffer aon}\\

\haiku{Dan bleek wel, of te.}{kuipster kontent k\'os zien of}{tatse rouwkoup had}\\

\haiku{Euveral zit er.}{m\`et z'n doeme-n-aon}{en heer k\"op toch niks}\\

\haiku{Pa Bonk waos ene,.}{struise kerel mer zoe lui}{es er groet waos}\\

\haiku{te bloete hakke,.}{wie twie oliekeuk bove z'n}{klompe oetkomme}\\

\haiku{de zwingel en m\`et.}{twie kamprajer woort te boel in}{beweging gebrach}\\

\haiku{Ze pombde diech neet,.}{allein de wiesheid in ze}{slooge ze debinne}\\

\haiku{ze m\`et ze kuijke.}{aon enen andere spieker}{gehange h\"obbe}\\

\haiku{Lam gen\'og, daste toch.}{nog tao rekene m\'os en}{zinne ontlede}\\

\haiku{als een ei 3.5 cent,.}{kost hoeveel verdient dan wel}{die kip in een jaar}\\

\haiku{zoe vas opein, tot.}{zene mond achter de punt}{van z'n neus zaot}\\

\haiku{m\`et Sinterklaos ',.}{n humme ene steve en}{ene peperkooke maan}\\

\haiku{Wel klobde ziech te,}{leef kinder nao zien doed mie}{onderein met m\`et}\\

\haiku{{\textquoteright} M\`et t'n {\textquoteleft}timbre{\textquoteright} van:}{enen twieden tenor liet ze}{noe vollege}\\

\haiku{{\textquoteright}, doog Graadsje, dee noe,.}{vas euvertuig waor dat}{niemes em zaog}\\

\haiku{e sjeep meister woort.}{en tot et tege d'n ierste}{pileer aon Wiek stoot}\\

\haiku{Van d'n Eker en de.}{Zooj is allewijl in de}{stad niks mie te zien}\\

\haiku{totse heij zeker.}{van e raozetig sjaop}{hadde gevrete}\\

\haiku{Is et einmaol,;}{geslach dan is et aal}{good wat t'raon is}\\

\haiku{te goon wandele.}{of nao de stad gesjik um}{kommissies te doen}\\

\haiku{ze achteroet en,.}{es ze nate veuj kriege}{kriege ze de pups}\\

\haiku{Veur de varjaassie,.}{zeet heer dan ouch wel ins tot}{et enen olifant is}\\

\haiku{{\textquoteleft}dao moot iech ouch get{\textquoteright},:}{van h\"obbe mer wie de vrouw}{hiel kordaot zag}\\

\haiku{iefer aon de gaank, '.}{um toch op te ierste balsn}{gooj figuur te sloon}\\

\haiku{Op v\"a\"ol plaotse {\textquoteleft}{\textquoteright},}{waor dansmeziek ofspeul wie}{ze zagte en m\`et}\\

\haiku{Ouch patattefrits,.}{lever en peerdsvleis woorte}{gerikkemandeerd}\\

\haiku{V\"a\"ol lui g\'onge ziech}{sm\"orreges e kruiske hoole}{en et leve g\'ong}\\

\haiku{{\textquoteright} Noe woort et ouch tied.}{veur Greteke en ze kwaom}{sjouw aongesloeperd}\\

\haiku{- {\textquoteleft}Nein, iech mein, h\"obder{\textquoteright},.}{ze al verkoch klonk et weer}{van et z\"olderke}\\

\haiku{Binne kwaome de.}{drei ongel\"oksveugel m\`et en}{m\`et tot bedare}\\

\haiku{H\"a\"ore maan zaoliger}{waor enen ierste fitseleer}{gewees en dee had}\\

\haiku{d'n ingaank van et {\textquoteleft}}{str\"a\"otsje ene paol en dao}{leunde noe Zjann\`et}\\

\haiku{Koelek had er evels,}{zene mond opegedoon of}{Janneske st\'ont veur}\\

\haiku{H\"a\"ostig bont heer de.}{lanteerie aon de kaord}{en leet ze zakke}\\

\haiku{ins in zenen droum,,.}{want es heer sleep k\'os er de}{r\"os nog neet vinde}\\

\haiku{Et {\textquoteleft}meitske{\textquoteright} keek h\"a\"or}{onger\"os euver h\"a\"ore br\`el}{aon en zeker neet}\\

\haiku{Ik had u willen,{\textquoteright}.}{verrassen want ik ben naar}{Maastricht overgeplaatst}\\

\haiku{Ene reuzeketel.}{woort op et vuur gezat en}{de kookerij beg\'os}\\

\haiku{veel em ouch nog wel,}{ins in tot heer onder z'n}{vreuger kinnesse}\\

\haiku{Et waor in et.}{jaor 1868 tot Mastreech zou}{ontmanteld weurde}\\

\haiku{Dan had heer altied.}{ene blouwen humperok aon}{en ene sjollek veur}\\

\haiku{medamme em al. '}{t\"osse hun twieje-n-in}{nao z'n meer gebroch}\\

\haiku{Ze waore toen}{good bevrund gewees en dooge}{ziech geere onderein}\\

\haiku{{\textquoteright}, waarsjouwde nog te,.}{jong vrouw van Tienes mer et}{waor al te laat}\\

\haiku{Es et evels op lol,.}{make aonkwaom leete zij}{ziech ouch neet lompe}\\

\haiku{er neet te deep nao,.}{um de invoudige reije}{tot heer et neet k\'os}\\

\haiku{{\textquoteright}, klonk tao opins ene,.}{pistolsjeut door de spin of}{e kanon aofg\'ong}\\

\haiku{Ze prouzde wie 'n,.}{kat die ze mosterd aon h\"a\"or}{neus hadde gesmeerd}\\

\haiku{Noe arriveerde.}{de deputaassies oet te}{ander d\"orrepe}\\

\haiku{ene barmeter en.}{die van Eeksterbos m\`et e}{paar houte vaze}\\

\haiku{Es er dan ouch al,.}{ins get oet et loed g\'ong k\'os}{er toch gei koed mie}\\

\haiku{Noe begreep er de:}{belangst\`ellende vraog}{van d'n andere}\\

\haiku{andere sloog ze.}{in de gawwigheid ene}{knien in zene nak}\\

\haiku{d'n onderein, um.}{ziech evekes te verpoeze}{van et dr\"ok get\'offel}\\

\haiku{{\textquoteright} - {\textquoteleft}Dan mooste miech ouch,.}{mer medein de sj\`elderije}{ophange Stiene}\\

\haiku{Iech waor ouch aon '.}{t witte wie noe en dao}{loerde dee sjijns op}\\

\haiku{Dao zant iech in de.}{sjerrever en m\`et waor}{de keuningin langs}\\

\haiku{{\textquoteleft}Vivat St. Jean{\textquoteright} (Ene;}{sjieke s\'okkerbekker deit}{alles op ze Frans}\\

\haiku{Wikske sjreef al ':}{mer z\"ochtenteeren 3 en}{ondervroog wijer}\\

\haiku{De {\textquoteleft}sc\`ene{\textquoteright}, die ziech,.}{tao zou aofspeule k\'oste ze}{ziech wel indinke}\\

\haiku{um h\"a\"or gemond ins,:}{te luchte of ze beg\'os}{tege de dames}\\

\haiku{Dao woort geklop en '.}{medam spoojde ziech naon}{ander t\"a\"ofelke}\\

\haiku{- {\textquoteleft}Trijnsje, iech m\'os 'n,?}{nuij kindermeitske h\"obbe}{h\"obste eint veur miech}\\

\haiku{zij neet ete en van.}{enen andere z'n drei sent}{nog profeteere}\\

\haiku{m\`et ene prachtige.}{sprunk te vinster oet nao et}{Leliestr\"a\"otsje in}\\

\haiku{D'n hiemel g\'ong d'n}{hook um van de Wolfstraot}{en op et kruuspunt}\\

\haiku{Noe moot me bij de,.}{Waole komme da's toch}{kokende m\`ellek}\\

\haiku{D'n eine voesslaag.}{nao d'n andere kwaom neer}{op kop en sjouwers}\\

\haiku{{\textquoteright} Teun verdween achter. '}{de deure van de buroo}{onder et Stadhoes}\\

\haiku{Dee wouw mer altied,.}{opgen\'omme weurde anders}{heel heer die moul neet}\\

\haiku{Iech verdrej et{\textquoteright}, zag, {\textquoteleft},.}{zeiech staon neet mie op}{doeg tiech et noe mer}\\

\haiku{Iech laot miech toch{\textquoteright}.}{noe al neet sjendeleere door}{m'n eige kinder}\\

\haiku{Et waos neet um,.}{te geluive wie goojekoup}{tao alles waor}\\

\haiku{- {\textquoteleft}Sagen sie maal{\textquoteright}, vroog, {\textquoteleft}?}{erhaben zie de kleine}{joenge niech gezeen}\\

\haiku{W\"omke beg\'os te,}{begriepe tot alles oet}{waor tot te meeg}\\

\haiku{{\textquoteright}, vroog er aon eine, '.}{dee em m\`etn lanteerie in}{ze geziech luugde}\\

\haiku{Mie es z\`es, zeve,.}{jaor waor er neet mer}{bij-der-hand en}\\

\haiku{- {\textquoteleft}Iech moot em evels aon{\textquoteright},.}{d'n aptieker laote}{zien deilde Greet m\`et}\\

\haiku{begaeden), ziech - =,:}{overdadig eten en drinken}{begrepder~~uit}\\

\haiku{Kempesch Kauf),kenaar (,.:}{de)-het kanaal van Maastricht}{naar Luikkenkee-(Fr}\\

\haiku{Krack),krammerin (den.:}{uitgang in uit te spreken}{als in het Fr.)-(Fr}\\

\haiku{versierde tak op,.}{de bekapping van een in}{aanbouw zijnd huis 4}\\

\haiku{Reefel),reigere-rillen,, (.}{remmelke-rammelaarreng}{mv. v. raank)-1}\\

\section{Adriaan van Oordt}

\subsection{Uit: Irmenlo. Deel 1}

\haiku{Alkwert voelde in zich,.}{een nieuw leven een leven}{van geheime smart}\\

\haiku{De oude Goden,.}{lieten er geen vrouwen toe}{geen lijfeigenen}\\

\haiku{{\textquoteleft}De kerkheer was boos,.}{omdat de nieuwe bisschop}{ontevreden is}\\

\haiku{Ze antwoordde niet, ',}{maar bracht de handen aant}{hoofd met een blik dien}\\

\haiku{Alkwert bevestigde,.}{met enkele woorden met}{goedige knikken}\\

\haiku{Hij hield er niet van,.}{de vrouw na te gaan in haar}{werken en weven}\\

\haiku{Weifelend haalde.}{hij een buidel te voorschijn}{en bood hem Alkwert aan}\\

\haiku{Ze reikte Warnef '.}{de hand en boogt hoofd in}{een weerlooze houding}\\

\haiku{En toen ze in de,;}{schaduw zijner liefde bleef}{hunkeren sprak hij}\\

\haiku{En ziende, dat hij,.}{haar blikken ontweek liet ze}{zich terugvallen}\\

\haiku{De zijnen wisten.}{andere wegen door de}{heilige wouden}\\

\haiku{Den strijdhamer zou.}{hij over hun vette nekken}{zwaaien als Donar}\\

\haiku{Naast hem gezeten,.}{scheelde ze telkens naar den}{blikkerenden hoop}\\

\haiku{en dan den bijvang,:}{verlatend aan de hoeven}{der buren klopten}\\

\haiku{Winkhorst, gevolgd door, '.}{een sleep mannen en vrouwen}{rende doort dorp}\\

\haiku{En toen ze vanavond,....}{niet terugkeerde heb ik}{gezocht en gezocht}\\

\haiku{Zijn haren vielen,.}{verward vuil vergrijsd over zijn}{voorhoofd en schouders}\\

\haiku{Zich oprichtend om,.}{te zien zag hij kersen als}{bloeddroppels vallen}\\

\haiku{{\textquoteright} Zacht spreidde hij de,.}{armen uit Alkwert aanziende}{met vochtige oogen}\\

\haiku{Hij had de handen,.}{uitgestrekt om de gunst van}{God af te dwingen}\\

\subsection{Uit: Irmenlo. Deel 2}

\haiku{En beneden hem.}{waren de oldermans in}{hun zetels van zand}\\

\haiku{Hij stortte zich neer, '.}{t gelaat naar de aarde}{en bad en dankte}\\

\haiku{Thuis gekomen, vond.}{hij Marfa met een ontsteld}{gelaat aan den haard}\\

\haiku{Ze kon niet den eenen.}{dag God en den anderen}{duivels erkennen}\\

\haiku{Hij hoorde haar niet,,:}{en tot zich zelven komend}{kreet hij smartelijk}\\

\haiku{Ze bouwden hem een,.}{hoeve statig genoeg voor}{een Wodanpriester}\\

\haiku{en bleef, de lange,.}{ledematen intrekkend}{hardnekkig zwijgen}\\

\haiku{en daar ze telkens ',.}{t hoofd schudde trachtte hij}{haar te overreden}\\

\haiku{En de wijkwasten. '.}{kwijldent Bloed regende}{over de gewaden}\\

\haiku{Zwijgend voegde hij,}{zich bij de Irmenlo\"ers}{die den vermoeiden}\\

\haiku{{\textquoteright} De lippen in een:}{wreed spotachtig beven stiet}{ze hem af en sprak}\\

\haiku{Zijn woorden brokten,.}{van elkander af terwijl}{zijn borst zich bewoog}\\

\haiku{Eeuwig draadde de,,.}{regen eeuwig als de tijd}{sinds Woonfred weg was}\\

\haiku{Alles klopte aan,,,.}{haar lijf aan haar hoofd aan haar}{hals aan haar polsen}\\

\haiku{Misschien zat hij in '.}{de hoeve tegent licht}{van den haard geleund}\\

\haiku{{\textquoteleft}Gonda, kom, 't stormt, ',,.}{alsoft in mijn hoofd ook}{stormt kom moeder wacht}\\

\haiku{t hooren van die.}{ongewoon verzoete stem}{een wellustbeving}\\

\haiku{Hij wilde wel, 't,.}{lijf door het tentezeil heen}{uitdagend schreeuwen}\\

\haiku{Zijn sterke smart werd?}{overstoft door het ootmoedig}{wachten en op wat}\\

\subsection{Uit: Warhold}

\haiku{{\textquoteright} Daar de anderen,:}{duister voor zich zagen stond}{hij op en schreeuwde}\\

\haiku{ik afschaffing van,,.}{de schande aan de kerk dus}{ook aan mij gedaan}\\

\haiku{Haar gelaat in de,,:}{handen liet ze zich voorover}{gaan al stamelend}\\

\haiku{en zijn handen in,,.}{elkander peinsde hij hoe}{hij zich redden zou}\\

\haiku{Hij rilde, en met.}{een koude stem klaagde hij}{zichzelf aan bij God}\\

\haiku{Hij gaf zich over aan,.}{de hooge gedachten die zijn}{leven vervulden}\\

\haiku{Hij rilde - toen hij.}{een schaduwgestalte door}{de hamei zag gaan}\\

\haiku{Hij dacht niet over wat.}{hij had misdaan en wat er}{verder te doen stond}\\

\haiku{Uw gelaat is als.}{zon in het moer en uw gang}{als der honden vlucht}\\

\haiku{'t Was hem, alsof.}{door haar woorden een slag in}{hem te vallen kwam}\\

\haiku{Zij had al zoo veel}{maren gehoord van nood daar}{binnen en van nood}\\

\haiku{Maar dan werd een groot.}{gekal van stemmen en een}{kraakgestoot gehoord}\\

\haiku{en taal noch teeken,.}{noch minder een maere werd}{van hem vernomen}\\

\haiku{Zijn armen schokten,:}{en het speeksel schuimde uit}{zijn mond toen hij riep}\\

\haiku{{\textquoteleft}Wij zijn gewoon, ons.}{zelf te genezen en ons}{zelf te wapenen}\\

\haiku{Maar niet willende,,.}{denken zweepte hij zich voort}{nog sneller loopend}\\

\haiku{Ik leefde zwaar aan,,.}{de aarde te loom om mijn}{handen te heffen}\\

\haiku{Maar zoetjes aan werd,.}{hij vervuld van een warmte}{die hem beven deed}\\

\haiku{Toen deze Warhold,:}{herkende schoot hij in een}{schaterlach en riep}\\

\haiku{Met stijve lippen,.}{bevroeg hij Janne hoe dit}{zoo geworden was}\\

\haiku{Maar ik zal u een.}{maal doen bereiden en dan}{alles verhalen}\\

\haiku{Maar steeds zijn handen,.}{om de hare lengde hij}{zich tegen haar aan}\\

\haiku{Zij liepen terug,.}{en begonnen weer de vaart}{nog eens en nog eens}\\

\haiku{weifelen deed en.}{gaandeweg de dingen van}{zijn blikken leidde}\\

\haiku{Hij kon zijn gevoel,.}{zijn blikken en zijn schreden}{niet meer beheerschen}\\

\section{G.A. van Oorschot}

\subsection{Uit: Twee vorstinnen en een vorst (onder ps. R.J. Peskens)}

\haiku{Ik hoefde niet te,.}{kijken want ik wist hoeveel}{streepjes er stonden}\\

\haiku{Ik weet nog hoe ik.}{liep te klappertanden van}{kou en ellende}\\

\haiku{Onder de schoorsteen.}{lagen forse houtblokken}{hoog opgestapeld}\\

\haiku{Op tafel stond een.}{volle schaal met bolussen}{en krentebollen}\\

\haiku{En sindsdien begint.}{voor mij elke kerstdag met}{scherven op de muur}\\

\haiku{Hij wou geen kolen,,.}{geven zei ik terwijl ik}{moeilijk overeind kwam}\\

\haiku{ontnam een van de,.}{knechten zijn schop en vulde}{de beide emmers}\\

\haiku{Ik richtte mij op.}{en zag haar hele gezicht}{overstroomd met tranen}\\

\haiku{Ga eens even op zij,.}{zei moeder en duwde me}{de kamer weer in}\\

\haiku{Omdat de agent zo.}{beleefd en verlegen deed}{werd ik nog banger}\\

\haiku{Ik heb geen vriendjes,.}{opgezocht maar ben naar mijn}{grootmoeder gegaan}\\

\haiku{Vader trok aan het.}{touw de voordeur open en kwam}{ook naar beneden}\\

\haiku{Daardoor wist ik dat.}{de afbetaling een vol}{jaar had geduurd}\\

\haiku{Ik zag dat alle.}{kinderen hun boeken en}{schriften al hadden}\\

\haiku{En wat die boeken.}{betreft moet je maar even naar}{de concierge gaan}\\

\haiku{Wat heb je een rooie,,.}{kop zei moeder toen ik de}{keuken binnenkwam}\\

\haiku{Ik ben niet naar mijn,.}{klas teruggegaan maar de}{school uitgelopen}\\

\haiku{Ik wilde eerst naar.}{mijn grootmoeder gaan omdat}{ik veel van haar hield}\\

\haiku{Ze pakte mij bij.}{de linkerpols en sleepte}{mij achter zich aan}\\

\haiku{Daarna werd de deur.}{op een kier geopend en}{gevraagd wie er was}\\

\haiku{Boezeroen heeft een,.}{trui joelde het die ochtend}{door de kleedkamer}\\

\haiku{Ik weet niet wat je.}{begonnen bent hem naar die}{rotschool te sturen}\\

\haiku{Van mijn gespaarde.}{zondagscenten kocht ik mij}{zelf stiekum een zwembroek}\\

\haiku{Het strand was in twee,.}{helften verdeeld een kleine}{en een grote helft}\\

\haiku{Geen van de jongens.}{merkte op dat ik niet mee}{het water inging}\\

\haiku{alsof ik een steen,.}{wegwierp in de richting van}{de leraar gooide}\\

\haiku{Zelfs per ongeluk.}{heb ik nooit een bal door de}{korf kunnen gooien}\\

\haiku{Toen ik op straat kwam,.}{zei ze dat ik niet altijd}{zo treuzelen moest}\\

\haiku{Piet Zeeman aaide.}{over mijn haar en daar werd ik}{erg verdrietig van}\\

\haiku{meelopend konden.}{ze elkaar gelukkig nog}{net een hand geven}\\

\haiku{Links en rechts sloeg ze.}{me om de oren en joeg me}{voor straf naar bed}\\

\haiku{Dan laten we het,.}{licht nog maar even uit voegde}{hij er aan toe}\\

\haiku{Toen ze de kamer.}{binnen kwam wierp ze het met}{een smak op tafel}\\

\haiku{Maar die zaten nog.}{helemaal aangekleed op}{de rand van hun bed}\\

\haiku{Die eerste weken '.}{werd ers nachts om toerbeurt}{bij vader gewaakt}\\

\haiku{Al was je honderd,, '}{keer de burgemeester zei}{ze je komt niet bij}\\

\haiku{Mijn vader las de,.}{kranten debatteerde weer}{met zijn bezoekers}\\

\haiku{Na een minuut of.}{tien riep vader of moeder}{even komen wilde}\\

\haiku{Bij het naar bed gaan.}{legde ik mijn broek onder}{mijn hoofdkussen}\\

\haiku{Op de hoogste bank.}{kon je met je handen het}{tentzeil aanraken}\\

\haiku{Heb je nooit ergens,.}{voorgoed willen blijven vroeg}{moeder plotseling}\\

\haiku{waar zou moeder nou.}{wezen en wanneer zou ze}{nou terugkomen}\\

\haiku{Eerst gaf vader geen.}{antwoord en er ontstond een}{moedeloos zwijgen}\\

\haiku{In het begin vroeg.}{hij haar nog wel of ze de}{brief niet lezen moest}\\

\haiku{Het enige waar ze.}{met hartstocht aan deelnam was}{het toneelspelen}\\

\haiku{En ze vroeg of er.}{een of andere pestkop}{was die hem dwars zat}\\

\haiku{Ik heb gezegd, zei,.}{hij plechtig na een poos}{gewacht te hebben}\\

\haiku{En toen heb je met,.}{je zakdoek het zweet van je}{kop geveegd zei ik}\\

\haiku{hoe mevrouw Coster}{de aardrijkskundestok op}{mijn rug kapot sloeg}\\

\haiku{Als je het liever,.}{niet vertelt zei ik ineens}{vol medelijden}\\

\haiku{Ach, lachte hij, dat.}{zeg ik natuurlijk maar bij}{wijze van spreken}\\

\haiku{Je begrijpt hoe ik.}{schrok en totaal van de kaart}{was van dit antwoord}\\

\haiku{En hoe lang heeft dat,.}{met elkaar geduurd in de}{voorkamer vroeg ik}\\

\haiku{Ze haalde de lijst.}{er af en hing de foto}{boven het buffet}\\

\haiku{VII Mijn vader werd.}{op hoge leeftijd naar het}{ziekenhuis gebracht}\\

\haiku{Twee dagen voor zijn.}{dood lagen de brieven nog}{in de linnenkast}\\

\haiku{Ik ben humeurig.}{omdat ik de hele dag}{niets heb uitgevoerd}\\

\haiku{Ik zou letterlijk.}{kunnen opschrijven wat er}{gezegd gaat worden}\\

\haiku{Ze zet haar knie\"en,.}{naast elkaar trekt haar rok en}{onderrok omhoog}\\

\haiku{Dat kan wel wezen,.}{zegt moeder maar ik heb er}{geen eigenschap aan}\\

\haiku{Tw\'e\'e en tachtig zegt.}{vader op een toon alsof}{hij hem de baas is}\\

\haiku{De burgemeester.}{vraagt of er veel zieken zijn}{in de gemeente}\\

\haiku{Ze richt plotseling.}{haar hoofd op en zegt dat ze}{geen zin in eten heeft}\\

\haiku{Een dunne witte.}{kaars is vastgebonden aan}{een stukje boomschors}\\

\haiku{Zal ik het grote.}{licht nou ook maar een beetje}{aandoen vraag ik haar}\\

\haiku{Vader helpt moeder.}{naar bed en zet daarna de}{televisie aan}\\

\haiku{Maar ik begrijp best.}{dat je morgen vroeg weer op}{je werk moet wezen}\\

\haiku{Mijn moeder heeft een.}{paar maanden geleden een}{attaque gehad}\\

\haiku{Hij wist echt niet hoe,.}{het kwam maar de hele stad}{was op de hoogte}\\

\haiku{En de huisdokter.}{die ik erover aansprak vindt}{dat eigenlijk ook}\\

\haiku{De lijkdienaren.}{stonden opgesteld met de}{kist op hun schouders}\\

\haiku{Ach, zei hij, ik heb.}{de bloemen nog geen water}{gegeven vandaag}\\

\haiku{We moeten haar een,.}{proces aandoen had een van}{de broers gezegd}\\

\haiku{Hij trok de schuif weer.}{helemaal open en vulde}{wat antraciet bij}\\

\haiku{Ik doe mijn broek en.}{overhemd uit en ga aan het}{tafeltje zitten}\\

\haiku{En hij verzuchtte.}{dat een mens het helaas niet}{voor het zeggen heeft}\\

\haiku{Mij komt de zondag,.}{in verband met mijn werk het}{beste uit zei hij}\\

\haiku{Mijn zuster Ka heeft,.}{eens gezegd dat onze broer}{gemakkelijk leeft}\\

\haiku{En langer dan een.}{halve dag moeten we er}{niet aan besteden}\\

\haiku{Ja, zei tante Cor,.}{ik hoop niet dat jullie het}{mij kwalijk nemen}\\

\haiku{Hij trok zijn jasje.}{uit als een werkman die voor}{een groot karwei staat}\\

\haiku{Zit-ie klem die,.}{deur vroeg hij en duwde me}{een eindje op zij}\\

\haiku{O marianne,,.}{O proletaren Aan de}{strijders Morgenrood}\\

\haiku{Ze zat recht op in,.}{haar bed met haar handen om}{de spijlen geklemd}\\

\haiku{Toen ze helemaal.}{niet  reageerde ben}{ik maar weggegaan}\\

\haiku{doe net of je me,.}{omhelst en druk dan even je}{vingers om mijn hals}\\

\haiku{En daar op tafel.}{vindt u een bakje met wat}{kleine spulletjes}\\

\haiku{Ik zal enkel dat,.}{gouden broche en haar ring}{meenemen zei ik}\\

\haiku{Ik wou het haar toch,.}{liever zelf zeggen want ik}{hield toen ook van h\`a\`ar}\\

\section{Jo Otten}

\subsection{Uit: Bed en wereld}

\haiku{{\textquoteleft}Ik ben even langs de.}{werkster geweest en heb haar}{een tientje gebracht}\\

\haiku{Hij vertelt verder;}{hoe hij Ann kort daarna voor}{goed had verloren}\\

\haiku{Hij weet het niet, hij}{weet alleen dat hij haar zou}{hebben weggehaald}\\

\haiku{Zo verdiept was hij.}{in zijn lectuur dat hij mij}{niet heeft opgemerkt}\\

\haiku{Zeker, zeker, had,.}{hij dat kunnen doen maar hij}{heeft het niet gedaan}\\

\haiku{Ik werd bedwelmd door,.}{de rook de zweetlucht en de}{goedkope parfums}\\

\haiku{Maar terugkomend.}{kreeg ik lust om bij een van}{haar binnen te gaan}\\

\haiku{Voor een gebarsten.}{kapspiegel installeerde}{zich Little Esther}\\

\haiku{Ik ging het trapje,.}{af dat van haar kamertje}{naar de gang leidde}\\

\haiku{Langs ons heen is een;}{va-et-vient van negers}{en negerinnen}\\

\haiku{Ik wil geen vrouwen,,.}{ik wil slapen geef mij toch}{wat om te slapen}\\

\haiku{We komen op een...}{kamer met een groot bed en}{alweer met spiegels}\\

\haiku{Nee, nee, dat zal zij,.}{zeker niet doen dat zal zij}{toch zeker niet doen}\\

\haiku{zij vindt het zelf veel.}{te aangenaam dat mijn blik}{over haar lichaam strijkt}\\

\haiku{Zij hebben bressen.}{geschoten en andere}{harten vrij gemaakt}\\

\haiku{de rest vreet zich in.}{het lichaam en blijft voor het}{daglicht verborgen}\\

\haiku{Nee, er was niemand,.}{alleen maar die oude vrouw}{en die oude man}\\

\haiku{Ik wil niet eenzaam,.}{sterven laat ik nu maar dood}{blijven in mijn bed}\\

\haiku{Ik kom een zaal met,,}{mensen binnen ik loop naar}{voren doe of ik}\\

\haiku{je portret aan de...}{muur van de slaapkamer was}{het enige wat bleef}\\

\haiku{Luisteren naar de...,,...}{klok staren in de vlam niet}{naar rechts niet naar links}\\

\haiku{Ik moet op tijd zijn,,;}{ik moet het voorbeeld geven}{ik heb nog een uur}\\

\haiku{geen tijd, geen klok, geen,...}{zwarte wijzers die sluipen}{over een wijzerplaat}\\

\haiku{slaperig kijkt haar,.}{hoofd boven het gestikte}{zijden overdek uit}\\

\haiku{Voor het naar bed gaan,.}{bidt zij haar knie\"en worden}{koud op de stenen}\\

\haiku{zij is de vrouw van,.}{een vossenkweker niet van}{een warmoezenier}\\

\subsection{Uit: Muizen en demonen}

\haiku{zij houdt z\'o\'oveel van,.}{hem m\'e\'er dan ik ooit van hem}{zal kunnen houden}\\

\haiku{Na den dood van mijn.}{vader bleef ik geruimen}{tijd bij mijn moeder}\\

\haiku{W\'e\'ervinden, mag}{ik van weervinden spreken}{nu hij z\'o\'o dichtbij}\\

\haiku{Altijd wist hij mij,.}{moed in te spreken nooit liet}{hij zich temeer slaan}\\

\haiku{d\'a\'arom  moest hij, om,.}{zijn leven te redden in}{de diepte springen}\\

\haiku{zijn gestamelde.}{explicaties zijn nooit tot}{mij doorgedrongen}\\

\haiku{Tenslotte bood zij,;}{geen weerstand meer alles deed}{zij wat hij wilde}\\

\haiku{Dat {\textquoteleft}eigen werk{\textquoteright} ging:}{een voorname plaats in mijn}{leven innemen}\\

\haiku{Iederen morgen}{stond ik op met nieuwen moed}{en meer dan vroeger}\\

\haiku{Ik dacht alleen maar,.}{aan Marceline die hulp}{en  zorg noodig had}\\

\haiku{Ik was overtuigd dat;}{ik toch altijd dezelfde}{voor haar zou blijven}\\

\haiku{Ik trachtte hem een,;}{beeld te geven van de angst}{waarin wij leefden}\\

\haiku{Steeds was ik beangst,.}{voor ons kind dat het eenige}{zou moeten blijven}\\

\haiku{Het was een leven,.}{in een paradijs dat wreed}{zou worden verstoord}\\

\haiku{Het speelde vaak door,.}{mijn hoofd in de nachten dat}{ik slapeloos lag}\\

\haiku{Nooit was zij liever.}{dan in dien tijd toen zij mij}{z\'o\'o vermagerd zag}\\

\haiku{koud en afwerend,.}{stond zij tegenover mij ik}{wist niet wat te doen}\\

\haiku{ik zag haar oogen en,.}{die oogen waren als altijd}{trouw en toegewijd}\\

\haiku{Een beetje schuldig:}{zal hij behoedzaam binnen}{komen en vragen}\\

\haiku{Wanneer de bloemen,,.}{bloeien denkt zij kan ook de}{liefde niet sterven}\\

\haiku{Wij vonden elkaar,...}{in een omarming die niet}{wilde eindigen}\\

\haiku{Zij ging terstond naar.}{de kleine keuken om wat}{eten klaar te maken}\\

\section{J. van Oudshoorn}

\subsection{Uit: Achter groene horren}

\haiku{Thans echter werden.}{van den nacht de schaduwen}{niet uiteen gevaagd}\\

\haiku{In allerhande.}{vormen stonden er ook reeds}{gevlochten manden}\\

\haiku{Nu lag zij er door.}{straten en huizenblokken}{in toom gehouden}\\

\haiku{Een drabbige sloot,,.}{met begin van iets als een}{sluisje dwars er door}\\

\haiku{Waarom juist nu dat?}{andere zich tusschen hen}{beiden had gesteld}\\

\haiku{Een lekkertje, zij,,,,.}{met Verwaaien en die en}{die daar in die keet}\\

\haiku{Tien jaar lang geduld,.}{te moeten hebben enkel}{om iets te worden}\\

\haiku{Ten slotte uit de,,;}{bank gesleurd struikelend nog}{aan den muur gekwakt}\\

\haiku{Minstens drie, en zoo:}{mocht hij er wel drie dagen}{minstens over denken}\\

\haiku{liet den ander het.}{laatste stuk door de stad een}{eind voor zich uit gaan}\\

\haiku{Maar toen reed hij al.}{achterwaarts om voldoende}{aanloop te krijgen}\\

\haiku{{\textquoteright} stond er enkel nog.}{met reuzenletters in het}{midden der heining}\\

\haiku{En tot geen prijs ter.}{wereld er met wie dan ook}{ooit over begonnen}\\

\haiku{Dat begreep hij niet,,.}{hij die zich reeds ongestraft}{voorschot had verschaft}\\

\haiku{Wat getemperd licht,.}{ook zacht over afgeronde}{lage meubeltjes}\\

\haiku{Voorschot, genomen....}{weer op zoo'n voorschot nog in}{het verschiet Elastiek}\\

\haiku{Voor een tragedie.}{ontbrak het aan mede-}{en tegen-spelers}\\

\haiku{Vreemde gezichten,.}{maar niet \'e\'en van die enkel}{dom-gezonde}\\

\haiku{Hier werd, met kalfsvel,.}{en trompetten een hooger}{leven ingeluid}\\

\haiku{En toch, iets reins en.}{veiligs was zooeven uit zijn}{leven heen gegaan}\\

\haiku{Dat haar zulk verschil.}{in postuur op den duur niet}{onuitstaanbaar werd}\\

\haiku{Langer dan een maand.}{was hij hier door centrale}{verwarming verwend}\\

\haiku{Had daarbij de chef?}{toch niet de hand vluchtig aan}{zijn schouder gebracht}\\

\haiku{Want hoe natuurlijk,.}{hoe bevrijdend werkte thans}{daartegen de straat}\\

\haiku{Door de harmonie,.}{van het landschap die eens tot}{teekenen noopte}\\

\haiku{Al was hij er niet,...}{meer op in gegaan liet het}{hem toch geen rust}\\

\haiku{Maar die mocht hij, na,.}{aankomst daar declareeren}{en bleef zoo safe}\\

\haiku{En toch, niet enkel -}{in gedachten hij merkte}{het met voldoening}\\

\subsection{Uit: Bezwaarlijk verblijf}

\haiku{De handtasch trouwens.}{bevatte het hoognoodige}{voor zoo'n eerste nacht}\\

\haiku{Intusschen had hier.}{en daar een der andere}{eters plaats genomen}\\

\haiku{geen liflafjes bij.}{een half fleschje azijn als}{in dat warenhuis}\\

\haiku{Dit vooral om den,,}{ander te toonen hem eens}{te laten voelen}\\

\haiku{Vroegen daar beide?}{riolen niet even gretig}{om volle aandacht}\\

\haiku{Van haar salaris.}{bij de kapel alleen kon}{zij niet rond komen}\\

\haiku{Was zij, liever dan,.}{nadeel te riskeeren weer haar}{eigen weg gegaan}\\

\haiku{Op een leugen meer.}{of minder scheen het daarbij}{niet aan te komen}\\

\haiku{Aan crediet voor die.}{laatste dagen kon dit slechts}{ten goede komen}\\

\haiku{Het d\'efil\'e werd.}{door het algemeen gedrang}{telkens weer gestremd}\\

\haiku{Nog geen voertuig of.}{voetganger was intusschen}{voorbij gekomen}\\

\haiku{Dan bevond hij zich.}{in zoo'n overvolle coup\'e}{met houten banken}\\

\haiku{Zijn ademhaling even,.}{deed stokken als in een te}{eng bedompt vertrek}\\

\haiku{Terwijl geld toch het.}{aller-onontbeerlijkste}{hier beneden was}\\

\haiku{Van haar salaris.}{bij de kapel alleen kon}{zij niet rondkomen}\\

\haiku{Was zij, liever dan,.}{nadeel te riskeeren weer haar}{eigen weg gegaan}\\

\haiku{Want dat die hem een,.}{beentje gelicht had daarvan}{ging evenmin iets af}\\

\haiku{Wel jammer, nu hij...}{zich hier wat op dreef en thuis}{begon te voelen}\\

\haiku{Hij had het gevoel.}{te moeten wijken voor het}{neefje van de chef}\\

\haiku{Ik heb me - als de -.}{doctorandus19 van iets}{moeten vrijmaken}\\

\haiku{wanneer je in de,.}{put zit doe dan juist hetgeen}{je het moeilijkst schijnt}\\

\haiku{Van Oudshoorn schreef het.}{tussen begin oktober}{en eind december}\\

\haiku{Ich war also ganz,}{ruhig aber in Antwerpen}{konnte ich ihn}\\

\haiku{Schreibe auch sehr oft.}{und sei herzlich gek\"usst von}{deinem trauen Mann Jan}\\

\haiku{Tobias en de,,:}{dood Deventer Dagblad 25}{november 1925 W.}\\

\haiku{Pinksteren, Weekblad,:}{voor Rotterdam 4 mei 1929}{Israel Querido}\\

\haiku{Pinksteren, Dagblad,:}{van Rotterdam 27 juni}{1929 Henri Borel}\\

\haiku{Pinksteren, De Vrije,,,-:}{Bladen vi december 1929}{blz. 406407  N.N.}\\

\haiku{Holland's Welvaren,,:}{De Groene Amsterdammer}{19 juli 1930 N.N.}\\

\haiku{De Java Bode, []:}{21 februari 1931 N.N.}{Dr. P.H. Ritter Jr}\\

\haiku{Hendrik Hagenaars,,:}{Hoek De Nieuwe Courant 20}{oktober 1949 N.N.}\\

\haiku{Van Oudshoorn bekroond,,:}{Algemeen Handelsblad 9}{augustus 1950 N.N.}\\

\haiku{Schrijven uit noodzaak,, []:}{Trouw zaterdag 4 mei 1968}{Verz. W.I. W.A.M. de Moor}\\

\haiku{De eerste druk van.}{het handboek vermeldt immers}{het juiste tijdstip}\\

\haiku{64Waarover Feijlbrief,.}{met zijn neef gesproken heeft}{is onduidelijk}\\

\subsection{Uit: De fantast}

\haiku{Daar de vrouw hem niet,:}{meer weg joeg versterkte hem}{dit in het besef}\\

\haiku{- De bedoeling was, '.}{s dichters werkkamer te}{laten verbouwen}\\

\haiku{Alleen opzij van,.}{een groepje mensen die daar}{op de tram wachtten}\\

\haiku{{\textquoteright} {\textquoteleft}Het vreemdste lijkt toch,.}{wel dat je ook kunt dromen}{ni\`et te dromen}\\

\haiku{{\textquoteright} {\textquoteleft}Vroeger dachten ze,.}{dat de aarde zo plat als}{een pannekoek was}\\

\haiku{Van geluk naar de,.}{godsdienst en geharrewar}{was nog slechts een stap}\\

\haiku{{\textquoteleft}Maar, maar heeft U dan?}{niet genoeg aan wat ik U}{zelf reeds gezegd heb}\\

\haiku{Hij had niets tegen,.}{Dani\"el al was het ook}{een raar stuk schoonzoon}\\

\haiku{Jawel, al zet U,.}{ook grote ogen op zo was}{het en niet anders}\\

\haiku{Zou het niet goed zijn,?}{wanneer je er eens voor een}{paar weken uitkwam}\\

\haiku{Van een buitenlands.}{paspoort was dan ook nog steeds}{geen werk gemaakt}\\

\haiku{Trouwens, Doortje hield.}{alweer een nieuw karweitje}{voor hem in petto}\\

\haiku{De geruchten over....}{een nieuwe oorlog werden}{steeds hardnekkiger}\\

\subsection{Uit: In memoriam}

\haiku{Hoe warmer het werd,.}{hoe meer ter Laan zich op dat}{verlof verheugd had}\\

\haiku{Het mag zijn, dat de.}{dikke h\^otelhouder het}{een en ander dacht}\\

\haiku{Toen dat ook zonder,:}{zichtbare uitwerking bleef}{haar eenig antwoord een}\\

\haiku{Niettegenstaande,.}{herhaald opbellen meldde}{er zich niemand meer}\\

\haiku{Daartoe was de les,,.}{naar beide zijden al te}{gevoelig geweest}\\

\haiku{Zijn innerlijke.}{verwarring schrijnde tot het}{ondragelijke}\\

\haiku{Ze zaten in een.}{ruime rieten bank in een}{hoek der veranda}\\

\haiku{Van de boulevard.}{beneden viel geen voetstap}{meer te vernemen}\\

\haiku{Een elkander even.}{aanzien leek thans intiemer}{dan een omhelzing}\\

\haiku{zijn terughouding,,.}{zijn wantrouwen waren nog}{eenmaal doorbroken}\\

\haiku{Met opgeslagen.}{jaskraag bevond de student}{zich weder op straat}\\

\haiku{Even leek het hem, als.}{was hij sinds jaren niet meer}{in die club geweest}\\

\haiku{Daarbij wist hij te {\textquoteleft}{\textquoteright}.}{goed deZeester niet meer te}{willen verlaten}\\

\subsection{Uit: Louteringen}

\haiku{Fagin, de roode,.}{jood voor zijne rechters en}{ter dood veroordeeld}\\

\haiku{De dood en Fagin's.}{schrikkelijke aantijging}{tegen het leven}\\

\haiku{Zelfs met het oor aan.}{het sleutelgat kon hij nog}{niets bepaalds verstaan}\\

\haiku{Tot beneden de:}{trapdeur open gerukt werd en}{Cato's vreugde galm}\\

\haiku{Maar wanneer zij naakt,.}{geweest was had zij de deur}{niet toegeworpen}\\

\haiku{Pas in de winkel.}{hoorde hij het algemeen}{gelach losbreken}\\

\haiku{bij hetgeen hem den.}{daarop volgenden laatsten}{kermisdag overkwam}\\

\haiku{Tegen Simon bleek.}{hij dien eersten middag al}{niet opgewassen}\\

\haiku{O, o, nog slechts twee.}{nuttelooze dagen scheidden}{hem van zijn vertrek}\\

\haiku{Het was niet meer te,.}{loochenen dat hij op den}{verkeerden weg was}\\

\haiku{En wat kon het hem.}{schelen dat hij wederom}{een les verzuimde}\\

\haiku{Hij haalde diep adem.}{en keerde zich met een schok}{weder naar de klas}\\

\haiku{Hij wist precies hoe.}{het binnen die andere}{jongens was gesteld}\\

\haiku{Ja, wanneer hij haar.}{voor zich alleen gehad had}{in zijn kamertje}\\

\haiku{Plots voelde hij zich.}{koortsig eenzaam en durfde}{niet meer op te zien}\\

\haiku{Alleen met schrijven.}{was het hem na dien eersten}{keer niet meer gelukt}\\

\haiku{En toch wanneer zijn.}{moeder een tweede sleutel}{van het kastje had}\\

\haiku{Vind je hier soms niet.}{veel meer dan jij noodig hebt en}{bij vertrouwde lui}\\

\haiku{Toch had deze hem.}{nog het adres van Moleschot}{medegegeven}\\

\haiku{Eindelijk alleen,.}{maar de kelner met zijn oog}{voor het sleutelgat}\\

\haiku{Na afloop van den.}{dag stond zijn gelaat nog scheef}{van het huichelen}\\

\haiku{Welk een geluk, dacht,.}{hij dat er nimmer halfheid}{in mijn leven was}\\

\haiku{Slechts hier en daar werd.}{aan het lage water een}{lantaarn ontstoken}\\

\haiku{Hij achtte het even,...}{natuurlijk om keizer te}{zijn als een ander}\\

\haiku{Hij kon weenen van,.}{geluk maar weemoed drong niet}{binnen tot dien rust}\\

\haiku{De natuurlijke.}{mensch leek spoorloos in hem te}{zijn ondergegaan}\\

\haiku{Het breken van de.}{zee werd er door het geroesch}{der menschen overstemd}\\

\haiku{Zijn aschblond haar hing.}{in een vreemde schuine punt}{over het leege voorhoofd}\\

\haiku{Om zich een houding.}{te geven frommelde hij}{het programma open}\\

\haiku{De hemel stond nog.}{in klaren weerschijn van de}{ondergaande maan}\\

\haiku{Wel stond hij nog in.}{de kinderschoenen van zijn}{geestelijk streven}\\

\haiku{Nog nimmer had hij.}{de gevaren van dezen}{omgang onderkend}\\

\haiku{Er was daar ergens.}{een verwaarloosd archief in}{orde te brengen}\\

\haiku{Jammer, dat hij met.}{haar nooit eens iets dergelijks}{ondernomen had}\\

\haiku{Hij ontwaarde het.}{jonge model als door een}{vreemde hand geteekend}\\

\haiku{Slechts voor de vensters}{was een schuine ruimte vrij}{gebleven tusschen}\\

\haiku{Maar hij moest toch weer.}{eens terloops de juffrouw haar}{meening uitlokken}\\

\haiku{Ook wist hij waarom.}{hij alle verkeer met hen}{had afgebroken}\\

\haiku{Zich in te beelden.}{dat hij zelf een dier kleine}{zwarte menschen was}\\

\haiku{Het is de nevel,.}{die in het bewustzijn zelf}{verhelderen moet}\\

\haiku{Verbergen was het,.}{onbewuste beginnen}{leugen het einde}\\

\haiku{Maar zou dit ijdel?}{verbeeldingswerk dan nooit een}{einde meer nemen}\\

\haiku{Zoo moest ook Paula.}{er hebben uitgezien toen}{zij negen jaar was}\\

\subsection{Uit: Pinksteren}

\haiku{Onwillig keerde.}{zich de andere jongen}{naar het drietal om}\\

\haiku{Zij kende dat reeds.}{tot vervelens toe en het}{bleef steeds hetzelfde}\\

\haiku{Arie, met opzet  ,.}{natuurlijk was een heel eind}{achter gebleven}\\

\haiku{Een doordringende.}{reuk van broeiend hooi steeg van}{alle kanten op}\\

\haiku{Want geen twijfel, of....}{er kwam een onweer van je}{welste opzetten}\\

\haiku{Het hardnekkigste.}{echter had haar die jonge}{violist vervolgd}\\

\haiku{zooals die sinds lang dien?}{laten avond tusschen hen was}{wedergekeerd}\\

\haiku{Maar toen werd het toch...}{een zich haasten om niet te}{vroeg te wezen}\\

\haiku{Arie den volgenden.}{Zondagmiddag nog heel wat}{moeten slikken}\\

\haiku{Zoo met zijn vieren.}{kon het natuurlijk tot geen}{bespreking komen}\\

\haiku{Plotseling had zij.}{het daar op het grasveldje}{benauwd gekregen}\\

\haiku{Het bleef dus, voor dien,.}{avond tenminste bij enkel}{groot-doenerij}\\

\haiku{Een heete bloedgolf,.}{steeg Arie alles verdoovend}{naar de hersenen}\\

\haiku{De hotelhouder.}{wist zoo'n nauwlettendheid ten}{zeerste te waardeeren}\\

\haiku{Want nu lag Marie,.}{in bed zij het dan van de}{kamer afgewend}\\

\haiku{Neen, dan was het maar.}{beter zwijgend alle schuld}{op zich te nemen}\\

\haiku{Maar geen twijfel meer,.}{woord voor woord had hij Marie's}{gestamel verstaan}\\

\haiku{Maar wel zeker, en,.}{wat d\`at betreft Arie had er}{al over nagedacht}\\

\haiku{Ook keek hij niet op,.}{toen Marie dan eindelijk}{de stilte verbrak}\\

\subsection{Uit: Tobias en de dood}

\haiku{Dat hem aanlokte.}{en tegelijkertijd met}{afkeer vervulde}\\

\haiku{Eenmaal zoo dicht in,.}{de buurt had hij toch even goed}{naar huis kunnen gaan}\\

\haiku{Vol zelfbewustheid.}{streek Tobias zijn lange}{zwarte snorren op}\\

\haiku{Men maakte er zich.}{in dit gezelschap hoogstens}{belachelijk door}\\

\haiku{Hoe dan ook, met die.}{bende daarachter had hij}{voor goed gebroken}\\

\haiku{Thans liet hij een paar {\textquoteleft}{\textquoteright}.}{nieuwe platen draaien om}{zein te studeeren}\\

\haiku{Met haar hoed op en,.}{haar mantel aan want zij was}{er slechts op bezoek}\\

\haiku{Met de handen in.}{de schoot bleef deze roerloos}{aan tafel zitten}\\

\haiku{Tobias lachte.}{tot hem de tranen over de}{wangen biggelden}\\

\haiku{O! o!, daartegen.}{zou de Roode het beslist}{moeten afleggen}\\

\haiku{Daarin week hij nu.}{eenmaal geen duimbreed van zijn}{beginselen af}\\

\haiku{Tot hem een stem aan.}{de telefoon in die rust}{was komen storen}\\

\haiku{Ja het ontbrak er,.}{nog maar aan dat zij bij de}{gramophoon ging zingen}\\

\haiku{Dat was al jaren,.}{geleden maar nu begreep}{hij het toch op eens}\\

\haiku{Zonder te drinken.}{had Tobias het glas weer}{van zich afgezet}\\

\haiku{Alles goed en wel,?}{maar waarom rekenschap te}{willen afleggen}\\

\haiku{gruwelijk zat te,.}{vervelen hardnekkig haar}{eigen gang gegaan}\\

\haiku{Maar aan het verstand.}{van Tobias twijfelde}{zij daarom niet meer}\\

\haiku{In de biljartzaal;}{waren eenige vreemden zeer}{luidruchtig bezig}\\

\haiku{Als zoo dikwijls had.}{Tobias haar discreete}{kloppen niet gehoord}\\

\haiku{X. Na langen tijd.}{komt Tobias weder in}{beter gezelschap}\\

\haiku{In dit verband moest.}{hij dikwijls aan zijn eerste}{huwelijk denken}\\

\haiku{Peet vernam, dat de.}{dokter zelf een bekende}{stille drinker was}\\

\haiku{De betreffende.}{vertrok nog haastiger dan}{zij gekomen was}\\

\haiku{Kitty, Tobias',.}{verloofde geeft blijken van}{groote zelfstandigheid}\\

\haiku{Verduiveld, hoe had....}{hij dat walgelijke tuig}{kunnen vergeten}\\

\haiku{zooals die zich tijdens.}{hun alleen zijn nog nimmer}{kenbaar had gemaakt}\\

\haiku{Er scheen nog iets van.}{den ouden zee-officier}{in hem te leven}\\

\haiku{Zulke voorschriften.}{waren even aanmatigend}{als belachelijk}\\

\haiku{Het speet Tobias,.}{nu dat er besloten was}{om op te breken}\\

\haiku{{\textquoteleft}Tobias is moei{\textquoteright}, {\textquoteleft}{\textquoteright}.}{zeide zijn schoonzustermen}{kan het aan hem zien}\\

\haiku{Dat was, om er niet,.}{meer van te zeggen toch wel}{hoogst eigenaardig}\\

\haiku{Om kort te gaan, de,!}{aangifte was gedaan de}{aanklacht ingediend}\\

\haiku{Wie of er dan van?}{afwist en daarvan misbruik}{trachtte te maken}\\

\haiku{Maar nu kwam toch de.}{beurt aan Tobias om groote}{oogen op te zetten}\\

\haiku{Hier moest Irma nog,}{meer naar het licht gaan staan en}{hield Tobias haar}\\

\haiku{Blijkbaar had thans de;}{tegenpartij twee volle}{uren op hem gewacht}\\

\haiku{Zoo behoefde hij,.}{Irma niet te roepen of}{om haar te bellen}\\

\haiku{{\textquoteright} herhaalde hij dus,, {\textquoteleft}?}{als had hij Irma niet goed}{verstaanBij de hand}\\

\haiku{Maar, wat hij van den,.}{aanvang af vermoed had was}{waarheid geworden}\\

\haiku{De toespraak van den.}{geestelijke wilde maar}{geen einde nemen}\\

\haiku{Goede hemel, op}{de vergevensgezindheid}{van een heiligen}\\

\haiku{Maar genoeg hiervan,.}{Tobias wilde vooral}{geen spelbreker zijn}\\

\subsection{Uit: Verhalen}

\haiku{Nachtgeest IN deze.}{buurten voelde de geest zich}{onvrij en bedrukt}\\

\haiku{Ja, zoodra die.}{ongedurige geest de}{overhand behield}\\

\haiku{Wanneer hij dus op!}{den duur niet aan zich zelf als}{koning gelooven kon}\\

\haiku{{\textquoteleft}Ha, ha, ziet daar ons,{\textquoteright}.}{volk vervolgde de eenzaam}{overgeblevene}\\

\haiku{Hij lachte er om.}{tot hem de tranen uit de}{zwarte oogen rolden}\\

\haiku{Dat stond als zware.}{bedreiging tegen den zoo}{zachten avond-val}\\

\haiku{Maar is dan ieder!}{gevoel voor recht bereids in}{jelui verstorven}\\

\haiku{Soms werd het even stil.}{maar zonder overgang tot het}{bevrijdend afscheid}\\

\haiku{Hij had gedaan wat,......}{anderen deden getracht}{zich aan te passen}\\

\haiku{Was hij soms toen in?}{gelijkmatigheid en rust}{zich zelve geweest}\\

\haiku{Hoe stellig wist hij!}{thans wat deze oorlog voor}{zijn bewustzijn was}\\

\haiku{Een gezondere;}{buiten-kleur begon bij}{haar op te komen}\\

\haiku{Haar te verlaten,.}{heette opnieuw alles op}{het spel te zetten}\\

\haiku{Overal om hen was.}{gedempt gepraat als in een}{besloten ruimte}\\

\haiku{Ik moest mij in een.}{dicht gewoel van haastige}{menschen bevinden}\\

\haiku{Een lauwbedwelmend.}{mengsel van slechte parfums}{vervulde de zaal}\\

\haiku{Nog zelden had ik.}{mij van deze omgeving}{zoozeer vervreemd gevoeld}\\

\haiku{Mijn vader kon het.}{late feestmaal blijkbaar al}{evenmin verdragen}\\

\haiku{Zou hij haar opnieuw?}{bereid vinden van meet af}{aan te beginnen}\\

\haiku{{\textquoteleft}Maar nee, dat kon toch,.}{niet zoo vol als het kleine}{ding met vlooien zat}\\

\haiku{Het was de herberg,.}{enkele schreden van de}{hoeve verwijderd}\\

\haiku{Want dit hier was iets,.}{nieuws waaraan zij bereids geen}{deel meer nemen kon}\\

\haiku{Van den hevigen.}{regenval was bijna niets}{meer te bemerken}\\

\haiku{De naakte aarde.}{om de hut was hier en daar}{nog zwart en drassig}\\

\haiku{Toch kon hij er niet.}{toe besluiten met Nelly}{naar binnen te gaan}\\

\haiku{Maar vervloekt, daar moest.}{iemand met zijn pooten aan}{gezeten hebben}\\

\haiku{Door Verkade op,.}{sokken begroet steeg het}{gemeente-hoofd uit}\\

\haiku{Achteruit tredend.}{kwam hij tusschen een heer en}{een dame te staan}\\

\haiku{Waren de laatste?}{jaren daarginds niet immer}{zonder haar geweest}\\

\haiku{{\textquoteright} Johanna volgde,.}{hem in het portaaltje nog}{immer fluisterend}\\

\haiku{Van Lier, met wien hij.}{een jaar lang in de groote stad}{had samengewoond}\\

\haiku{Ook de voorstelling.}{in de zaal gebeurde zoo}{als iets bijkomstigs}\\

\haiku{Hij wist dat hare.}{gedachten immer om hem}{toefden en ook thans}\\

\haiku{Hoe zou hij alleen?}{nog de kracht vinden met dat}{verval te breken}\\

\haiku{En schier onbewust.}{had zijn verlangen hem aan}{zee teruggevoerd}\\

\haiku{Bijna had hij nog.}{ruzie met den man van de}{juffrouw gekregen}\\

\haiku{Wanneer hij den trein!}{miste en ditzelfde nog}{eens moest beginnen}\\

\haiku{Of ja, hij zou van.}{het station terugkeeren}{en haar verrassen}\\

\subsection{Uit: Willem Mertens' levensspiegel}

\haiku{Het besef eener.}{lichamelijke ikheid}{sprak bijna niet aan}\\

\haiku{Nog noodde, met een,.}{drieste hoofdbeweging zij}{hem na te komen}\\

\haiku{In den aanblik van.}{het bewegelijk stadsbeeld}{kwam iets ongekends}\\

\haiku{Het was de doffe,.}{nadreun van een doffen slag}{die alles velde}\\

\haiku{Hij was een dwaas te,.}{meenen dat zij nu eveneens}{naar hem verlangde}\\

\haiku{hij plots de moeder,,, {\textquoteleft},{\textquoteright}.}{die hem zacht werendgekke}{vreemde jongen zei}\\

\haiku{Het was voorbij en.}{weldra zou een ander hier}{zijn plaats vervullen}\\

\haiku{Toen borg hij het met.}{een schouderophalen in}{zijn portefeuille}\\

\haiku{Na het middageten.}{had hij de laatste weken}{meestal hoofdpijn}\\

\haiku{naar binnen en  .}{liet het orgel spelen om}{den schijn te redden}\\

\haiku{Ze zaten in de.}{warme huiskamer in den}{zachten lampeschijn}\\

\haiku{{\textquoteleft}Ja, en als jelui,.}{me niet helpen kom ik in}{de gevangenis}\\

\haiku{{\textquoteleft}Wel donders{\textquoteright} dacht hij {\textquoteleft}?}{hebben de dingen zich dan}{op hun kop gesteld}\\

\haiku{Toen rekte hij zich,,.}{uit en zocht zich binnensmonds}{pratend een schoon boord}\\

\haiku{Geen sterveling in,.}{geen enkelen tijd die het}{niet beweenen zou}\\

\haiku{Hij lag in bed met.}{de brandende lamp op het}{nachtkastje naast hem}\\

\haiku{Dat had alleen de,.}{tante die alle brieven}{zelf naar de post bracht}\\

\haiku{Het was er als een,.}{kippenhok waarin men een}{steen geworpen had}\\

\haiku{Hij snikte in het}{duister van radeloozen angst}{en de oogen puilden}\\
