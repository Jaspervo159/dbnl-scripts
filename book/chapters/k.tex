\chapter[19 auteurs, 1415 haiku's]{negentien auteurs, veertienhonderdvijftien haiku's}

\section{Mathias Kemp}

\subsection{Uit: De felle novene}

\haiku{Pastoor Grompers was,.}{een beste kerel maar wat}{van de oude school}\\

\haiku{- Hij bedoelt het zoo,.}{kwaad niet haastte zich pater}{Herman te sussen}\\

\haiku{Ja, dat kon Grompers,.}{niet ontkennen doch hij bleef}{niettemin critisch}\\

\haiku{wij studeeren na het.}{seminarie in biechtstoel}{en bij huisbezoek}\\

\haiku{Het voorbeeld van Sint,,.}{Paulus die eenmaal Saulus}{was voor oogen houden}\\

\haiku{Eigenlijk gold hij.}{zoowat als het bedorven kind}{der communauteit}\\

\haiku{liet Ernestine,.}{Woltmakers die toevallig}{met ritmeester Jhr}\\

\haiku{Voorop het bestuur,;}{der Broederschap met zijn nieuw}{processievaandel}\\

\haiku{Het werd een dreigend.}{opdringen rondom Debrit}{en zijn trawanten}\\

\haiku{Waarom studeeren, zich?}{blootstellen aan proefjaren}{en tegenslagen}\\

\haiku{Je leeft nu in een.}{sfeer van heiligheid die ik}{je benijden moet}\\

\haiku{haar laatste wenschen,,.}{zijn mij nu ik alles weet}{heilig geworden}\\

\haiku{- Wat mij betreft, kunt,;}{U het tot morgen laten}{meende P. Thomas}\\

\haiku{{\textquoteright} MEER en meer raakte.}{Torenen in beroering}{door de Novene}\\

\haiku{Waarom P. Herman?}{baron Richelle met een}{bezoek lastig viel}\\

\haiku{hard zou loopen, ze.}{is namelijk een zuster}{van graaf Lahnenstein}\\

\haiku{Ik tusschen elf en - -.}{twaalf het uur van ebbe aan}{den biechtstoel er heen}\\

\haiku{Ze zond ons uit om,.}{te bedelen anders kan}{ik het niet noemen}\\

\haiku{Ik telde voor oud,!}{vuil en ik was toch een frisch}{en aardig meisje}\\

\haiku{Toen onze oudste,,.}{kwam Lowieke werd het een}{paar maanden beter}\\

\haiku{- Vertel nu eens wat...... ',.}{van jezelf en neem eensn}{snuifje dat doet goed}\\

\haiku{- Menschen van uw slag,.}{zien we hier maar zelden viel}{Nelia spinnig uit}\\

\haiku{Men ging ook wel eens,.}{na wie zich nog niet in de}{kerk vertoond hadden}\\

\haiku{Er ketste iets, met,.}{hol en droog geluid op den}{steenen vloer van de kerk}\\

\haiku{Daar rondom drongen,,.}{de menschen stom van schrik in}{afgrijzen terug}\\

\haiku{- Breng Guus Pollissen, '.}{ook nog op den goeden weg}{dan zijnt er twee}\\

\haiku{- Men weet nooit waar het,.}{zaad terecht komt  troostte}{zich zijn confrater}\\

\haiku{Hoe die bij het volk,.}{ingeslagen is zult U}{wel gemerkt hebben}\\

\haiku{- Ge kunt gerust eens,.}{inschenken moedigde de}{Conciliant aan}\\

\haiku{ze verzekerde,.}{dat hij een spion van de}{belastingen was}\\

\haiku{De bewoners van.}{het reuzenkrot hadden hun}{verblijf zelfs versierd}\\

\haiku{Een goed loon in den,, '.}{zomer geen zin voor sparen}{s winters gebrek}\\

\haiku{Dat heftige wijf.}{heeft niet zoo'n ongelijk met}{haar verbittering}\\

\haiku{Het mag U wel niet,.}{verwonderen dat de droom}{mij interesseert}\\

\haiku{Met een hoopvol en.}{dankbaar gemoed ben ik naar}{Brussel getogen}\\

\haiku{Er gebeurde iets,.}{in mijn kop of een bol vuur}{erin ontplofte}\\

\haiku{Noch Marie noch haar.}{moeder hadden hem ooit als}{arts geconsulteerd}\\

\haiku{De eigenlijke:}{aard van dat schepsel leek hem}{niet twijfelachtig}\\

\haiku{daarmee wilde de.}{vreemde persoonlijkheid niets}{te maken hebben}\\

\haiku{P. Medardus kon:}{niet nalaten even in dien}{hoop te snuffelen}\\

\haiku{Een fijne flesch {\textquoteleft}Mumm,{\textquoteright}.}{cordon rouge verhoogde}{de intimiteit}\\

\haiku{Ze dansten af en.}{toe in een klein zaaltje met}{eigen strijkorkest}\\

\haiku{Even beet wel een  :}{vluchtig wantrouwen in de}{verhitte koppen}\\

\subsection{Uit: Sterren, musschen en ratels}

\haiku{Zegt het spreekwoord niet,?}{dat alleen het ongeluk}{geschiedenis schrijft}\\

\haiku{Niemand kende, als,}{hij de vorming en den aard}{van den zoo rijken}\\

\haiku{Remerswael ligt in.}{de goede jaargetijden}{paradijselijk}\\

\haiku{met de drie kleine '.}{siroopstokerijen ging}{t ook al niet meer}\\

\haiku{En hij, de vinder,?}{wat zouden de gevolgen}{voor hem zelf wezen}\\

\haiku{- En ik geloof aan,.}{geen eeuwigen dood stemde}{een opzichter in}\\

\haiku{Montluce had van.}{verre deze tragedie}{der geesten gevolgd}\\

\haiku{Hij zat uren aan uren,.}{in zijn werkkamer star in}{de verten turend}\\

\haiku{Maar de pastoor sprong.}{bij het hooren dier oude}{galmen ontzet op}\\

\haiku{Toen hij plots oversloeg,.}{in godslasteringen viel}{een stilte rond hem}\\

\haiku{De vrouwen deinsden,.}{angstig de mannen slopen}{ruimer naar voren}\\

\subsection{Uit: Vallende vogels}

\haiku{La belle blonde{\textquoteright},.}{nam de hooge heeren in den}{tuin van een caf\'e}\\

\haiku{toen ze het twisten.}{der jonge meisjes hoorden}{treuzelden ze wat}\\

\haiku{Het gaat op leven,.}{en dood kreet de heesche stem}{van een blokbreker}\\

\haiku{Van dezen mensch hield,,.}{ze nu opeens met heel haar}{hart met heel haar wil}\\

\haiku{Al haar vroegere,,,;}{sentimenten leken vaag}{flauw verzinsels zwak}\\

\haiku{zeker, ze was mooi,,,.}{en zelfs met smaak hoewel zeer}{eenvoudig gekleed}\\

\haiku{Als kind zag ze in.}{Achiel de verwezenlijking}{van dat andere}\\

\haiku{Hij schreef het in een.}{plotsen roes en voelde het}{als een bevrijding}\\

\haiku{Hij kende Wanroth;}{genoeg om zich met reden}{te verontrusten}\\

\haiku{dat ik ze voor niks!}{beter meer aanzie dan die}{mamsellen uit Luik}\\

\haiku{Ze merkten dat ze.}{raak schoten en zetten hun}{wraakneming voort}\\

\haiku{- Laten we liever,.}{wat gaan wandelen ik houd}{het binnen niet uit}\\

\haiku{- En ik, onnoozele,......}{die je vertrouwde die je}{idealiseerde}\\

\haiku{een uitbarsting van.}{redelooze drift had ze hem}{niet in staat geacht}\\

\haiku{Wat dat beteekende,.}{wist ze zelf niet precies je}{zag een hollen weg}\\

\haiku{Voor ze zoover was, greep.}{hij haar om het midden en}{trok haar naar zich toe}\\

\haiku{Het viel Sanne nu,.}{op dat ze dit erg zwaar en}{overdreven deden}\\

\haiku{- Die fotos nam ik,,.}{zelf hakkelde hij en ik}{liet ze vergrooten}\\

\haiku{Willy hield zich goed,.}{al steeg een verscheurende}{woede in hem op}\\

\haiku{Daarom wilde ze,.}{niet meer van hem houden niet}{meer aan hem denken}\\

\haiku{Alles leek haar dan,,.}{onverschillig doelloos grauw}{en onbelangrijk}\\

\haiku{Inderdaad had een.}{diepe desillusie zijn}{karakter misvormd}\\

\haiku{- Met je dronken kop.}{vertelde je me wat je}{me nuchter verzweeg}\\

\haiku{'s Middags Stegen,,.}{beiden de parachute}{op den rug omhoog}\\

\haiku{Op de kade wat,}{wenkende en wuivende}{menschen op het schip}\\

\haiku{Maar een boeketje.}{bloemen voor Sanne wilde}{hij niet vergeten}\\

\haiku{Onder hysterisch.}{gelach van Conny viel de}{debutante flauw}\\

\haiku{mompelde hij, na.}{de voorgeschiedenis te}{hebben vernomen}\\

\haiku{Het werd een totaal,.}{ineenzinken van alle}{hoop energie en trots}\\

\haiku{Ze steeg met Willy,.}{op boven een landschap met}{spitse bergtoppen}\\

\haiku{- Sanneke, vergeef......... -}{me vergeef me brulde hij}{als een zinnelooze}\\

\haiku{Willy was toch geen,.}{slecht mensch geweest had altijd}{hoog en edel gewild}\\

\haiku{De gewonde ging,.}{achteruit maar kon toch nog}{bezoek ontvangen}\\

\haiku{Sanneke vergleed,,...}{in een eindelooze witte}{zalige stilte}\\

\haiku{Voor Demerrel zou,.}{ze een paar prachtige groote}{afdrukken maken}\\

\section{Pierre Kemp}

\subsection{Uit: Limburgs sagenboek}

\haiku{Dit scheen een teken,.}{van boven te zijn om op}{die plaats te graven}\\

\haiku{Deze vluchtten nu.}{naar Maastricht en verhaalden}{daar het gebeurde}\\

\haiku{De plaats waar deze,;}{gebouwd werd was toen nog een}{waterloze streek}\\

\haiku{het welcke hy.}{volmaeckt heeft door synen Apostel}{den H. Jacobus}\\

\haiku{Verhaal nu hetgeen,!}{u is geschied tot meerder}{glorie van Zijn naam}\\

\haiku{Niemand had die struik.}{geplant en hij bloeit er nog}{tot heden toe voort}\\

\haiku{evenmin konden ook.}{nu de einden aan elkaar}{bevestigd worden}\\

\haiku{Die plek had precies.}{de vorm van de grondslag van}{een kapelletje}\\

\haiku{Weer beproefde hij.}{van wal te steken en weer}{lukte het hem niet}\\

\haiku{Gedurende de;}{H. Mis overviel een diepe}{slaap de hertogin}\\

\haiku{{\textquoteleft}Oh, nu hebben wij.}{toch eindelijk Margreetje}{teruggevonden}\\

\haiku{Zij voeren verder.}{en verder tot zij aan de}{stad Athene kwamen}\\

\haiku{Zij werd veroordeeld.}{om de volgende morgen}{verbrand te worden}\\

\haiku{Het was een wreed en,.}{hardvochtig man die niets gaf}{om God noch gebod}\\

\haiku{Het bleek een monnik,.}{te wezen die hier blijkbaar}{wilde overnachten}\\

\haiku{{\textquoteleft}Ik was juist van plan,?}{m'n ziel te verdobbelen}{wat dunkt u ervan}\\

\haiku{Hij was dapperder:}{dan zijn naam en een buurvrouw}{van hem riep hem toe}\\

\haiku{{\textquoteright} riep de visboer, blij.}{verrast en wilde zich door}{de rijen dringen}\\

\haiku{Als ik je eens een,.}{raad mag geven laat het dan}{zo en zo maken}\\

\haiku{{\textquoteright} In hun angst holden.}{zij tussen de menigte}{door naar de prior}\\

\haiku{Daar vertelden zij.}{het gebeurde en toonden}{de bebloede draad}\\

\haiku{Dat is natuurlijk.}{honderden en honderden}{jaren geleden}\\

\haiku{Maar in de huizen.}{op de Heerestraat is nooit meer}{iemand gaan wonen}\\

\haiku{Maar de koning, die {\textquoteleft}{\textquoteright},.}{links te paard stijgt zal te Mook}{vluchten over de brug}\\

\haiku{Hij wilde het niet,.}{geloven hij had het te}{duidelijk gehoord}\\

\haiku{Bij het herenhuis,.}{van Tielens gekomen schreed}{zij daar de stoep op}\\

\haiku{Zij zochten nog toen.}{het al twaalf uur sloeg op de}{kerkklok van Gulpen}\\

\haiku{Maar toen kwam daar een,:}{zwarte vlugge heer aan en}{zei tegen de knecht}\\

\haiku{Onderwijl beval.}{de heer de aap de gast eens}{goed te bedienen}\\

\haiku{Hadt gij ook dat nog,!}{nagelaten dan waart gij}{in zijn macht geweest}\\

\haiku{De man biechtte de.}{tweede maal en later nog}{eens tot acht maal toe}\\

\haiku{De man werd na dat.}{uur beter en beter en}{herstelde spoedig}\\

\haiku{Dit duurde tot er,.}{een steen werd gekapt die juist}{in het gat paste}\\

\haiku{{\textquoteleft}Ga naar de molen!}{en maal de zak koren die}{is aangekomen}\\

\haiku{Hij gaf hem de kom.}{met zaad en gebood hem de}{korrels te tellen}\\

\haiku{Want zo aanstonds moet,!}{ik mij wreken al is het}{op mijn beste vriend}\\

\haiku{Terzelfdertijd kwam:}{de bezetene weer tot}{bezinning en riep}\\

\haiku{moorden, diefstallen.}{en branden waren aan de}{orde van de dag}\\

\haiku{Dit had zo enige.}{jaren geduurd en de graaf}{was ten einde raad}\\

\haiku{Die mensen waren.}{zeer beangstigd en spoedden}{zich biddend naar huis}\\

\haiku{Ouden van dagen, '.}{waarschuwen us nachts nooit}{door de Pas te gaan}\\

\haiku{{\textquoteright} {\textquoteleft}'t Is geen ekster, ',{\textquoteright}, {\textquoteleft}!}{t is een raaf snauwde de}{koningwerk maar voort}\\

\haiku{Geen bergkant was hem,,.}{te steil geen water te breed}{geen moeras te diep}\\

\haiku{Het was winter, de.}{sneeuw lag nog al hoog en het}{had flink gevroren}\\

\haiku{Zo kwamen zij op,.}{een plaats die de schijnwerker}{nog nooit had gezien}\\

\haiku{Het spook nam de steen.}{weg en nu vertoonden zich}{twee potten met geld}\\

\haiku{Bij een volgende;}{bevalling bleef de man thuis}{bij zijn vrouw achter}\\

\haiku{Wel hoorden beiden,:}{de stem van de berggeest die}{hen spottend toeriep}\\

\haiku{Het naderde het.}{bed en hield de soldaat het}{licht onder de ogen}\\

\haiku{Dat was hem welkom,.}{want hij had grote honger}{na dat ronddwalen}\\

\haiku{{\textquoteright} De gedaante liet.}{hem echter geen tijd om nog}{verder te praten}\\

\haiku{Vroeger zou op die.}{plaats alle avonden een licht}{hebben gebrand}\\

\haiku{De jonge heks zou,.}{al gauw merken hoe lelijk}{zij zich had vergist}\\

\haiku{het wijf lag daar met.}{stevige hoefijzers aan}{handen en voeten}\\

\haiku{Op een donkere - -}{avond het was winter begaf}{hij zich dan met}\\

\haiku{Hij wierp het kamrad.}{op de mestvaalt en maakte}{zich uit de voeten}\\

\haiku{{\textquoteleft}Zie je nu, dat het ' ....,:}{t kamrad van is zoals}{ik je gezegd heb}\\

\haiku{Hij wilde dus iets.}{verzinnen om de kat in}{de oven te werpen}\\

\haiku{{\textquoteright} riep de vader boos.}{en zette de appel op}{de schoorsteenmantel}\\

\haiku{Hij zag ook dat het;}{dier moeite genoeg deed om}{vooruit te komen}\\

\haiku{En toen hij dan ook,:}{zijn nood klaagde aan die man}{antwoordde deze}\\

\haiku{Het meisje bracht de.}{boodschap over aan mijnheer en}{deze liet haar gaan}\\

\haiku{na die tijd zijn geen.}{ongelukken meer in de}{brouwerij gebeurd}\\

\haiku{Je hebt mijn dochter!}{met een kwade hand geraakt}{en ze doen kwijnen}\\

\haiku{{\textquoteleft}Wat zijt gij toch een,!}{gelukkige vrouw die zo}{een goede man hebt}\\

\haiku{{\textquoteright} {\textquoteleft}Zeg dat wel,{\textquoteright} meende,.}{de vrouw gevleid dat haar man}{zo geprezen werd}\\

\haiku{Die heb ik dat ook!}{geraden en nu zijn ze}{maar wat gelukkig}\\

\haiku{De pastoor keek eens.}{om en bad toen luider in}{het  Latijn voort}\\

\haiku{Toen zij weg waren,.}{nam de jonkman ook de pot}{en zalfde zich ook}\\

\haiku{Onderwijl sloop de.}{knecht stilletjes naderbij}{en nam de zeef weg}\\

\haiku{Heel de berg zat vol.}{katten en er kwamen er}{nog maar altijd bij}\\

\haiku{Hij wilde weten.}{waar hij aan toe was en ging}{weer naar beneden}\\

\haiku{Dat wijf kan ook nog,{\textquoteright}.}{wel wat anders dan brood eten}{antwoordde de man}\\

\haiku{Zij wisten geen raad.}{ertegen en de plaag werd}{met de dag erger}\\

\haiku{{\textquoteright} Op deze wijze,.}{zanikte zij zolang de}{mannen er waren}\\

\haiku{Die man was dood en.}{het was zijn doodshemd dat zij}{in de  hand hield}\\

\haiku{{\textquoteleft}Haaf water, haaf m\`elk,,}{iech h\"ob get krie gemete}{En iech h\"ob m'n ziel}\\

\haiku{Wel vond men enige.}{dagen later zijn lijk dat}{boven kwam drijven}\\

\haiku{Die dag was het een,,.}{weer dat men er geen hond door}{joeg laat staan een mens}\\

\haiku{{\textquoteleft}Gij moet de vrouw weer.}{leggen op de plaats waar zij}{is dood gebleven}\\

\haiku{Toen zij 's morgens, '.}{wakker werd lag haar man weer}{naast haar int bed}\\

\haiku{{\textquoteleft}Het kan gaan, gelijk,!}{het wil maar morgennacht schiet}{ik het veulen dood}\\

\haiku{Het werd hem nu wel,}{wat bang te moede maar hij}{besloot toe te zien}\\

\haiku{Zo bleef hij wachten,.}{in zijn duistere schuilhoek}{tot klokslag een uur}\\

\haiku{Ze hoorden de kat;}{uit de verte aankomen}{of over de daken}\\

\haiku{De mensen van de.}{hoeve wisten niet wat zij}{ervan denken moesten}\\

\haiku{Zo werd de knecht dan.}{ook stilaan ervan verdacht}{een weerwolf te zijn}\\

\haiku{want de kogel vloog.}{over de bomen in plaats van}{de hond te treffen}\\

\haiku{De pastoor gaf het,;}{gevraagde doch liet zich over}{de zaak zelf niet uit}\\

\haiku{De stem van boven.}{Te Schaesberg zaten enige}{mannen te kaarten}\\

\haiku{Onderweg werd hij.}{telkens lastig gevallen}{door een zwarte hond}\\

\haiku{Een jager was in.}{het Fazantenbos onder}{Oh\'e en Laak op jacht}\\

\haiku{De dronkemannen,.}{wilden hem wel eens zien dat}{was juist iets voor hen}\\

\haiku{Zo kwam hij over de.}{weg van Mari\"enwaard naar}{het Limmerlerbroek}\\

\haiku{{\textquoteright} lachte de lange.}{juffrouw nog en toen was ze}{ineens verdwenen}\\

\haiku{{\textquoteright} Eindelijk liet de.}{veerman zich toch bepraten}{en zette hem over}\\

\haiku{Ook vertoonde de,.}{kanunnik zich soms zijn hoofd}{in de hand dragend}\\

\haiku{wanneer zij het nog,.}{eens zag moest ze het vragen}{wat het begeerde}\\

\haiku{De eerste pater,.}{die je tegenkomt daar moet}{je goed op letten}\\

\haiku{Die overste zal je,.}{aannemen dan kun je voor}{priester studeren}\\

\haiku{Buiten de kom van.}{het dorp waren twee maaiers}{in het hooi werkzaam}\\

\haiku{maar het was of zij.}{versteend waren en iets hen}{op de plaats vasthield}\\

\haiku{hij kon evenwel niet, {\textquoteleft}{\textquoteright}.}{merken dat hij het dier ook}{maard\`at verwondde}\\

\haiku{En nu bekende.}{hij dat hij zijn ziel aan de}{duivel had verkocht}\\

\haiku{{\textquoteleft}Ik zie wel, dat je!}{erg bedrukt bent en dat het}{je niet naar wens gaat}\\

\haiku{{\textquoteleft}God zegene u,{\textquoteright}.}{mocht de duivel zich van haar}{ziel meester maken}\\

\haiku{{\textquoteright} Met deze boodschap,,.}{door de bankier overgebracht}{kon de duivel gaan}\\

\haiku{{\textquoteright} verliet de boze,.}{hem onder het breken van}{vele boomtakken}\\

\haiku{Verontwaardigd wees;}{mevrouw het misdadige}{voorstel van de hand}\\

\haiku{Zij wilde in dat.}{spookhuis geen dag meer blijven}{en zei haar dienst op}\\

\haiku{Nu heb ik geen rust,!}{meer zelfs niet in het graf en}{in de eeuwigheid}\\

\haiku{Het was volle maan,.}{toen hij ging en buiten zo}{helder als de dag}\\

\haiku{Weer hernamen zij.}{hun geweldige arbeid}{van voren af aan}\\

\section{Paul Kenis}

\subsection{Uit: F\^etes galantes}

\haiku{Zij was geen {\textquoteleft}fille{\textquoteright},;}{galante niet een vrouw uit}{de lichte wereld}\\

\haiku{Ondanks den fellen.}{wind bleef het zooals altijd erg}{druk op de groote brug}\\

\haiku{{\textquoteright} Met moeite baanden;}{de twee wandelaars zich door}{dat alles een weg}\\

\haiku{Zij was eene {\textquoteleft}fille{\textquoteright},,.}{eene meid van lichte zeden}{van de armste soort}\\

\haiku{de lange bruine {\textquoteleft}{\textquoteright}}{lokken ongepoeierd lijk}{eenincroyable}\\

\haiku{op de grenzen der.}{heide stierf alle geluid}{van de wereld weg}\\

\haiku{Ook als Fabre;}{haar aanspreekt kijkt ze daarom}{niet verwonderd op}\\

\haiku{De waterkruik staat;}{on-aangeraakt op den rand}{van het bronbekken}\\

\haiku{Over het met gekleurd:}{zand bestrooid paadje stapten}{beiden nevens een}\\

\haiku{een zuur gezicht te.}{trekken tegen de heeren}{die lief wilden zijn}\\

\haiku{Over al de dingen:}{begon de avond zijn grijzig}{webbe te spinnen}\\

\haiku{Ten volle was dan.}{ook de danseres op haar}{aanbidder verliefd}\\

\haiku{het oude regiem.}{tegen de omwenteling}{zou verdedigen}\\

\haiku{Maar mag ik nu ook}{weten wat er verder met}{mij gebeuren zal}\\

\haiku{de heer Cazotte.}{had ons allen nog den angst}{op het lijf gejaagd}\\

\haiku{in het gevang waar.}{hij het verband van zijne}{wonden zou rukken}\\

\haiku{het verhaal mijner;}{wederwaardigheden zou}{u slechts vervelen}\\

\haiku{toren stevenden.}{die den ingang van de}{haven verdedigt}\\

\subsection{Uit: Historische verhalen}

\haiku{En de Heer heeft mij,.}{gezegend want mijn arbeid}{gedijt voor dit land}\\

\haiku{Zoo vermochten die,;}{heeren van de wet het de}{rust te herstellen}\\

\haiku{ook had hij nog het,.}{schootsvel voor waarmede hij}{te arbeiden placht}\\

\haiku{De magistraat bleek.}{niet bij machte er paal en}{perk aan te stellen}\\

\haiku{Mijnheer van Egmont,;}{was in een draagstoel met twee}{muilezels bespannen}\\

\haiku{Hier en daar vonkte.}{de bleeke zon een schittering}{in al dat metaal}\\

\haiku{De Cellebroeders.}{kwamen aan om de lijken}{van de galg te doen}\\

\haiku{moeizaam leunend op,.}{den arm van een metgezel}{strompelde hij voort}\\

\haiku{Maanden lang had de;}{late Winter alles blank}{gelegd en verstard}\\

\haiku{Toen hadden luwe;}{regens de sneeuw doen smelten}{en den grond doorweekt}\\

\haiku{bijtenden hoogmoed,.}{en knagenden nijd die hem}{het hart wegvraten}\\

\haiku{koortsgloed vlamde in,.}{de blikken die deemoedig}{ten gronde keken}\\

\haiku{Buiten verblindde.}{hen weer het licht van de nu}{reeds schuin staande zon}\\

\haiku{Met de kruisbroeders;}{was ook veel vreemd volk de stad}{binnengekomen}\\

\haiku{een koffer werd van.}{de tweede verdieping in}{de straat leeggeschud}\\

\haiku{V Jonker Wenzel;}{trof schikkingen om den burcht}{te overmeesteren}\\

\haiku{Clio, de muze van,.}{de geschiedenis moge}{het mij vergeven}\\

\haiku{De moeilijkheden.}{van mijn onderneming heb}{ik niet onderschat}\\

\haiku{Te meer daar ik met.}{al mijn plannen zoo weinig}{blijk op te schieten}\\

\haiku{Elk tuintje sluimert.}{als in een bocht van de snel}{vlietende rivier}\\

\haiku{Wat verder voert een,;}{vlonder over het water naar}{de dennenbosschen}\\

\haiku{Waarom had ik aan:}{mijn hospita eenvoudig}{niet de vraag gesteld}\\

\subsection{Uit: De kleine Mademoiselle C\'erisette}

\haiku{Het eerste wat ik '}{nu deed als iks morgends}{vroeg ontwaakte was}\\

\haiku{, dagen lang hadden;}{wij er nieuwe schoonheden}{kunnen genieten}\\

\haiku{keeltjes trilden bij,.}{het me\^eneurie\"en van het}{lied borstjes golfden}\\

\subsection{Uit: De roman van een jeugd. Een ondergang in Parijs}

\haiku{Eigenlijk had hij:}{ook wel gemeend de vrienden}{wat te overbluffen}\\

\haiku{hij voelde hoe 't}{hem te benauwd zou worden}{op zijne kamer}\\

\haiku{eventjes kwam het in {\textquoteleft}{\textquoteright}.}{hem op dat deweltschmerz ook}{ingebeeld kon zijn}\\

\haiku{Zoo stellig had hij;}{vader beloofd er ditmaal}{door te geraken}\\

\haiku{hij trok de laden:}{open waaruit hij \'e\'en voor \'e\'en}{zijne zaken kreeg}\\

\haiku{den briefschrijven die.}{hen thuis van zijn voornemen}{moest verwittigen}\\

\haiku{Tusschen het gewoel:}{heen drongen verkoopers van}{velerlei dingen}\\

\haiku{Zou hij niet op eene?}{bank wat uitrusten of in}{een koffiehuis gaan}\\

\haiku{eerst moest hij het wat.}{gewend worden en met de}{stad kennis maken}\\

\haiku{De kade lag koel;}{en rustig in het lommer}{van oude boomen}\\

\haiku{het oude {\textquoteleft}quartier{\textquoteright}:}{du Temple met historisch}{klinkende namen}\\

\haiku{slechts mogelijk was...}{zich in een blad of tijdschrift}{bekend te maken}\\

\haiku{een paar keeren ging hij:}{nog en dan was er een kort}{briefje gekomen}\\

\haiku{Waarschijnlijk zou hij:}{vandaag den bestuurder niet}{meer kunnen spreken}\\

\haiku{Begoochelingen,,!}{ach die waren zoo lang en}{zoo verre voorbij}\\

\haiku{Zoo kon hij taalman;}{worden in een hotel of}{een groot magazijn}\\

\haiku{Zelfs in armoede;}{was het vrije leven heerlijk}{in deze groote stad}\\

\haiku{- Neen, illuzies wel,;}{niet meer die hadden reeds te}{dikwijls bedrogen}\\

\haiku{Vroeger had Vincent,.}{dat alles niet opgemerkt}{nu leerde de nood}\\

\haiku{Hoe kwam je er toch,;}{op zoo maar heereknecht}{te willen worden}\\

\haiku{hij zou twintig frank;}{per week ontvangen en in}{huis eten en slapen}\\

\haiku{nu bemerkte hij.}{dat de zon over de huizen}{begon te nijgen}\\

\haiku{zijn kennis had hem.}{in de steek gelaten en}{zou niet meer komen}\\

\haiku{Na 't eten gingen.}{zij heen en namen afscheid}{op den boulevard}\\

\haiku{soms schoven over den;}{vloer de slepende passen}{van een dansend paar}\\

\haiku{daarbij ze hadden;}{gezegd dat hij tweemaal per}{week kon terug keeren}\\

\haiku{Zonder overtuiging;}{stemde Vincent met heel die}{redeneering in}\\

\haiku{Alvorens toe te:}{geven stelde hij echter}{zijne voorwaarden}\\

\haiku{de vijf frank van het,;}{cachet de opbrengst van het}{verkochte uurwerk}\\

\haiku{Dan had Hettner zijn:}{beschermeling een paar frank}{in de hand gestopt}\\

\haiku{toen de andere.}{bleef aandringen wees hij hem}{af met stug gebaar}\\

\haiku{anderen deden,.}{het wel die het minder noodig}{hadden dan hij zelf}\\

\haiku{een eerste groep ging,;}{vroeg in den avond uit en bleef}{slechts tot tien elf uur}\\

\haiku{Soms schalde in de;}{verte het gezang van een}{huiswaarts keerende groep}\\

\haiku{wel had hij niet veel,.}{verdiend maar genoeg om zijn}{honger te stillen}\\

\haiku{Vincent keek rond of;}{hij niet eene geschikte plaats}{om te slapen vond}\\

\haiku{de {\textquoteleft}crocheteurs{\textquoteright} en;}{andere arme stakkerds}{hadden weer arbeid}\\

\haiku{De rij volgend ging,}{Vincent een smal gangetje}{door naar eene groote zaal}\\

\haiku{ondanks de lucht en.}{ondraaglijke hitte kon}{je veilig rusten}\\

\haiku{De zaal was zoo vol.}{dat er niet \'e\'en enkele}{meer bij zou kunnen}\\

\haiku{De {\textquoteleft}hotels{\textquoteright} hadden:}{allen hetzelfde vuile}{schurftige uitzicht}\\

\haiku{en zoo je tien cent;}{opleg betaalde kreeg je}{zuivere lakens}\\

\haiku{ook het supplement.}{voor reine lakens wilde}{hij wel uitgeven}\\

\haiku{maar zijn walg om \'e\'en.}{bed met twee te deelen kon}{hij niet overwinnen}\\

\haiku{De eerste nacht, dien,:}{Vincent hier doorbracht leek hem}{bijzonder akelig}\\

\haiku{daarbij, je kon 't:}{in die vuile atmosfeer}{toch niet uithouden}\\

\haiku{Ook Vincent kreeg zijn,;}{part zoodat hij twee dagen lang}{kaas bij zijn brood had}\\

\haiku{Een andermaal kwam;}{l'Asticot aangedragen}{met een zak noten}\\

\haiku{zij gingen altijd,;}{bij paren met denzelfden}{afgemeten stap}\\

\haiku{iederen dag gaan...}{schrijvers weg en komen er}{nieuwe in de plaats}\\

\haiku{Met muffen reuk sloeg.}{het stof neer en weer kwam er}{wat verademing}\\

\haiku{Als zijn wrokkige:}{menschenhaat smolt weg voor de}{warmte van dien blik}\\

\haiku{Slechts eventjes wou hij}{haar terug zien en laten}{weten hoe dankbaar}\\

\haiku{Voor den geringen:}{prijs van vijftien cent had hij}{een overvloedig maal}\\

\haiku{hij wou er niet aan,.}{denken en schreef maar voort maar}{altijd voort adressen}\\

\haiku{Zij verlangden slechts.}{iedereen even ellendig}{als zich zelf te zien}\\

\haiku{hij wilde niet meer,.}{aan dat alles  denken}{het was te pijnlijk}\\

\haiku{Anderen dosten:}{hun bedienden uit in een}{bont vastenavondpak}\\

\haiku{zes potloodkrabbels:}{en vier penseelvegen en}{daar stond het landschap}\\

\haiku{De meesten waren {\textquoteleft}{\textquoteright},:}{l\^acheurs die je onmiddellijk}{in den steek lieten}\\

\haiku{ze zouden je van;}{niets hebben ingelicht en}{niets voor je gedaan}\\

\haiku{hij was erg bij de.}{hand en zijne papieren}{waren in orde}\\

\haiku{Voor de eerste maal,;}{weer sedert den langen tijd}{dacht hij aan de vrouw}\\

\haiku{n\'ecessit\'e faict,.}{gens mesprendre et faim le}{loup saillir du bois}\\

\haiku{hoe hij zes maand had.}{gekregen voor diefstal in}{een automobiel}\\

\haiku{de welstand van een.}{ander was als een hoon op}{eigen ellende}\\

\haiku{gisteren avond nog;}{had zij een klant gehad die}{niet betalen wou}\\

\haiku{Zoo kwam de deur voor;}{een oogenblik vrij en was}{hij buiten gesneld}\\

\haiku{Hij was hongerig,.}{en vermoeid de kou beet hem}{in het aangezicht}\\

\haiku{Een ander had er;}{van gesproken om naar het}{Zuiden te trekken}\\

\section{M.J.H. Kessels}

\subsection{Uit: Der Koehp va Hehle in de sjlag va Waterloo}

\haiku{zit ee aoth versjrummeld,{\textquoteright}.}{menke noh sjatting wiet in}{de nuhgentig joar}\\

\haiku{Ich zaan, da gao ich ' '.}{t iesjte evvelt Niehs}{noch adieh zage}\\

\haiku{adieh Koehp, osse,.}{leeve Hergot bewaart dich}{ich zal dervuur beehne}\\

\haiku{Der ruksjtrank doog}{mich geweldig pieng en de}{rubbe krakde mich}\\

\haiku{Noew wosj heh mich der '.}{ruk duchtig aaf en vreefm}{in mit veerevet}\\

\haiku{{\textquoteleft}Sjmak het, Koehp{\textquoteright}, vroog ich, {\textquoteleft}{\textquoteright},, {\textquoteleft}{\textquoteright}.}{of dat antwoordde hehnoch}{behter wie bookeskook}\\

\haiku{Die sjpas hei der,.}{motte zieh vier sjlooge ze neer}{wie de wil kanieng}\\

\haiku{{\textquoteright} -  {\textquoteleft}Dan ginne tied{\textquoteright},{\textquoteleft}:}{verlore zaan ich tegen}{der Bam   Ich zaan}\\

\haiku{Ze zooge het dan ooch.}{in en leepe wie echte}{winkhong dervan durch}\\

\haiku{maar de plank brik aaf.}{en ich val weer oppen ruk}{tusje de verke}\\

\haiku{Zouwe 't flits de{\textquoteright} ().}{Pruuse zieh of der Kroehsjel}{generaal Grouchy}\\

\haiku{dat niks mieh doh woar ().}{wie de ouw Garde onger}{der FrijanFriant}\\

\haiku{- {\textquoteleft}Drei joar a ee sjtuk ' '.}{is ze neet oehtt bed en}{t lieje gewehs}\\

\section{Mensje van Keulen}

\subsection{Uit: Van lieverlede}

\haiku{In ieder geval '.}{gaatt niet over als ik me}{steeds moet verkleden}\\

\haiku{Toen Coby nog thuis,.}{woonde werd er beneden}{ook wel gezongen}\\

\haiku{Ze keek toe hoe de.}{vrouw haar moeder op de rug}{begon te kloppen}\\

\haiku{Mevrouw Beijer kreeg.}{haar boterham niet op en}{klaagde over haar maag}\\

\haiku{ik heb je daar nog,.}{nooit gezien of je daar geen}{zin in zou hebben}\\

\haiku{De kleine stond te.}{dreinen en in paniek aan}{haar rok te trekken}\\

\haiku{{\textquoteright} In de woning van.}{haar oudste dochter begon}{ze er opnieuw over}\\

\haiku{{\textquoteright} Coby lachte en.}{negeerde Dannie die een}{lelijk gezicht trok}\\

\haiku{Dan is Dannie dat,{\textquoteright}, {\textquoteleft}.}{ook zei mevrouw Beijerwant}{die heeft ook vrienden}\\

\haiku{{\textquoteright} Dromerig staarde.}{Coby voor zich uit en toen}{schudde ze haar hoofd}\\

\haiku{Ze legde een hand.}{over haar ogen en hoorde haar}{moeder wegsloffen}\\

\haiku{De panty was hard.}{aan de teenstukken en moest}{nodig gewassen}\\

\haiku{Over hem wisten ze,.}{niet zoveel behalve dat}{ie erg laat thuiskwam}\\

\haiku{{\textquoteright} {\textquoteleft}Als je me dat zou.}{gunnen mag je me wel het}{dubbele geven}\\

\haiku{En jullie waren,}{jullie waren vreselijk}{onaardig voor me.}\\

\haiku{Denk eraan dat opa.}{vanmiddag komt om paardjes}{voor je te bakken}\\

\haiku{Haar handen lagen,,.}{gebald als twee geplukte}{duifjes op de sjaal}\\

\haiku{{\textquoteright} zei mevrouw Beijer:}{met een gesmoorde stem die}{piepend eindigde}\\

\haiku{De mis heeft veel geld, ',,.}{gekostt was in de kerk}{sjiek met een loper}\\

\haiku{Van die dingen die.}{ze in flessen afsteken}{kan ik hier niets zien}\\

\haiku{{\textquoteright} Hanna hield de fles.}{boven het glas tot er geen}{druppel meer uitkwam}\\

\haiku{Buiten klonk er nu,.}{en dan nog een knal de fik}{lag na te smeulen}\\

\haiku{{\textquoteright} Hanna ging zitten.}{en duwde haar hand in het}{biddende gezicht}\\

\haiku{Hanna legde het.}{slechts van een naam voorziene}{poststuk voor haar neer}\\

\haiku{of ie nou mee-,.}{of tegenvalt je bent een}{ervaring rijker}\\

\haiku{Zijn vrouw negerend, doch,.}{de voordeur openlatend liep}{hij het portiek uit}\\

\haiku{Ik doe het licht uit,,.}{dacht ze zodat ie denkt dat}{er niemand thuis is}\\

\haiku{Ze eindigt altijd.}{met zielig en spijtig doen}{als zeuren niet helpt}\\

\haiku{Hij heeft me wel eens...}{in de steek gelaten en}{dan komt het slechte}\\

\haiku{Er bewoog iets rechts,.}{onder in de hoek ze ging}{op haar tenen staan}\\

\haiku{Blijven hangen aan.}{die stomme leuning van dat}{stomme rotportiek}\\

\haiku{{\textquoteleft}Hij gelooft nooit dat '.}{jij met dat ouwe lijkn}{dagje op stap gaat}\\

\haiku{{\textquoteright} {\textquoteleft}Welja, begin jij, '.}{ook nog maar eens iedereen}{neemtt voor hem op}\\

\haiku{{\textquoteright} Hanna stopte de.}{bestelling in haar zak en}{tilde de tas op}\\

\haiku{Hij is haar ontrouw,{\textquoteright}.}{geweest zei mevrouw Beijer}{en stak haar hand uit}\\

\haiku{De poten waren,.}{geknakt de rugleuning lag}{voor de commode}\\

\haiku{{\textquoteleft}Ik heb de hele,{\textquoteright}.}{dag nog geen honger gehad}{zei mevrouw Beijer}\\

\haiku{Twee platte, smalle.}{latjes zaten als pleisters}{tegen de deurpost}\\

\haiku{Hanna voelde haar.}{maag zwellen en nog leek haar}{dorst niet te lessen}\\

\haiku{En niet alleen 'n,.}{flesje limonade doe}{er wat stevigs in}\\

\haiku{Ik heb 'n hekel.}{aan mensen die er trots op}{zijn dat ze sparen}\\

\haiku{{\textquoteright} Hij duwde de deur.}{verder open en zette zijn}{voet op de drempel}\\

\haiku{Ze schraapte er de.}{schimmel uit en zette de}{deur naar de tuin open}\\

\haiku{Ze keek naar de korst.}{in haar hand en stond op om}{hem weg te gooien}\\

\haiku{Toen mevrouw Beijer,.}{bij bewustzijn kwam zag ze}{dat het donker was}\\

\section{Jan J. Klant}

\subsection{Uit: De geboorte van Jan Klaassen}

\haiku{Want soms viel er een.}{rukwind op de Dam en trok}{aan de gordijnen}\\

\haiku{De heer Duivel liet.}{mij rustig uithuilen en}{stak een sigaar op}\\

\haiku{Welk een zegening,.}{dit blijde uitstromen van}{het watercloset}\\

\haiku{{\textquoteleft}Je zoent niet als een,{\textquoteright}, {\textquoteleft}.}{kantoorman zei het meisje}{maar als een dichter}\\

\haiku{Een ogenblik nog,{\textquoteright} zei, {\textquoteleft},?}{hijjuffrouw hoe laat begint}{mijn vergadering}\\

\haiku{{\textquoteleft}Zeg Sat, ik zou je,,.}{v\'o\'or we aankomen graag nog}{even willen spreken}\\

\haiku{Katrijn had deze.}{foto in de rand van de}{spiegel gestoken}\\

\haiku{s Nachts greep ik naar,.}{de deurknop Waar wanhoop streed}{om door te breken}\\

\haiku{De secretaris.}{zat aan zijn bureau en mijn}{chef stond achter hem}\\

\haiku{{\textquoteright} {\textquoteleft}Integendeel,{\textquoteright} zei, {\textquoteleft}.}{de secretaris snelwij}{zijn zeer tevreden}\\

\haiku{{\textquoteright} riep ze, stampvoetend, {\textquoteleft}}{van woedejullie dichters}{gaan altijd te ver.}\\

\haiku{Ik ging achter het,.}{voor mij bestemde bureau}{zitten bij het raam}\\

\haiku{Voortdurend ruist er,,}{hier water dacht ik en nu}{ontdek ik het pas.}\\

\haiku{Mijnheer Duivel kwam.}{terug en nodigde mij}{uit hem te volgen}\\

\haiku{Velenzijn er die,.}{verdorstten Mil sloot zij niet}{haar deuren dicht}\\

\haiku{Is het een wonder?}{dat men zich ergert aan hun}{gemaskeerd misbaar}\\

\haiku{{\textquoteright} Toen ik weer in het,.}{portaal stond hoorde ik ze}{bulderend lachen}\\

\haiku{Tenslotte stopte,.}{het toestel toen mijn hoofd juist}{het plafond raakte}\\

\haiku{Hij ging voor mij op,.}{zijn bureau zitten met zijn}{armen over elkaar}\\

\haiku{Wat vond hun holle?}{zoekende blik in deze}{doodse wildernis}\\

\haiku{Ik nam de pion,.}{niet maar richtte mijn loper}{op zijn zwakste punt}\\

\haiku{Ze zijn nog banger,.}{dan ik want zij durven zich}{zelfs niet bewegen}\\

\section{Jos Kleinjans}

\subsection{Uit: Het acces van Meijel}

\haiku{{\textquoteleft}Ik kan u alleen.}{maar complimenteren met}{uw discipline}\\

\haiku{Toen hij de kurk weer,;}{op de fles sloeg zag hij de}{commandant staren}\\

\haiku{Plotseling sprong een.}{van de jongere jongens}{voor hen op het pad}\\

\haiku{De man verviel steeds.}{met overdreven ernst in zijn}{commandantenrol}\\

\haiku{Hij spoog iets in de,.}{lap keek er even naar en borg}{de zakdoek vlug weg}\\

\haiku{{\textquoteright} De vrouw sloeg betrapt.}{haar schort voor haar gezicht en}{vluchtte de stal in}\\

\haiku{Latour stelde zich - -:}{en stond zich niet anders toe}{slechts drie kenmerken}\\

\haiku{Als je een paar maal {\textquotedblleft}{\textquotedblright},.}{het woordvolk herhaalt lijkt het}{niet meer te bestaan}\\

\haiku{{\textquoteleft}Het tijdpotlood is.}{een bijzonder nuttige}{en slimme vinding}\\

\haiku{Het geratel van.}{de mitrailleurs kwam over de}{spoordijk dichterbij}\\

\haiku{Hij richtte zich half.}{op en rukte de Colt uit}{zijn okselholster}\\

\haiku{De jongen met de:}{hamer greep voor de tweede}{maal doeltreffend in}\\

\haiku{De officieren:}{van het Militair Gezag}{volgden op de voet}\\

\haiku{Het valt dus wel mee.}{met mijn kennis ontrent uw}{activiteiten}\\

\haiku{Besloten werd om.}{RVV-groepen selectief}{te bewapenen}\\

\haiku{Over de top van de.}{organisatie heb ik}{geen informatie}\\

\haiku{Ik accepteer dit.}{rapport en daarmee is uw}{opdracht ten einde}\\

\haiku{Morgen meldt u zich.}{bij mij en ik deel u uw}{nieuwe taken mee}\\

\haiku{{\textquoteright} Schuurman knikte en.}{krabbelde een paar regels}{op een stuk papier}\\

\haiku{{\textquoteright} Het sarcasme in,.}{haar blik ontging hem niet maar}{hij negeerde het}\\

\haiku{{\textquoteright} Latour haalde diep.}{adem na de provocatie}{en schudde zijn hoofd}\\

\haiku{Maar daarom hoeven,?}{we hun systeem nog niet te}{omhelzen nietwaar}\\

\haiku{en nog vroeger, een.}{sergeant-majoor die}{stamrozen kweekte}\\

\haiku{{\textquoteleft}Ik ben bang dat ik,.}{niet helemaal begrijp wat}{u bedoelt majoor}\\

\haiku{{\textquoteright} Latour ontweek nu.}{de autoriteit in de}{stem van de majoor}\\

\haiku{{\textquoteright} Schuurman donderde.}{een gebalde vuist op het}{blad van zijn bureau}\\

\haiku{De enige dekking,.}{die ze hadden werd gevormd}{door hun vrachtwagen}\\

\haiku{Latour duwde haar.}{met geweld verder open en}{glipte de gang in}\\

\haiku{{\textquoteright} vroeg Latour en liep.}{zonder haar antwoord af te}{wachten de trap op}\\

\haiku{Latour pakte het.}{lichaam bij de schouder vast}{en trok het terug}\\

\haiku{Haal jij die even op,,.}{maar maak voort want wij moeten}{Serv\'e wegbrengen}\\

\haiku{Een geelgroene fluim.}{bleef op het eikehout van}{een trede liggen}\\

\haiku{het vergeten zijn,.}{we moeten Serv\'e naar het}{hospitaal brengen}\\

\haiku{{\textquoteleft}Nu geloven de.}{mensen misschien nog dat de}{LO echt heeft bestaan}\\

\haiku{{\textquoteright} Latour nam zelf een,.}{slok hield ondertussen de}{man bij diens arm vast}\\

\haiku{{\textquoteleft}En ik ben Little,{\textquoteright}.}{John zei de reus en sloeg}{zichzelf op de borst}\\

\haiku{Jij bent net zo'n stom,{\textquoteright}.}{Frans wijf antwoordde Gleason}{triomfantelijk}\\

\haiku{Jij brengt je schrijfsels.}{bij allerlei idioten}{onder de aandacht}\\

\haiku{{\textquoteright} {\textquoteleft}Ik twijfel er niet,.}{aan dat u uw promotie}{verdiend heeft majoor}\\

\haiku{Ook Latour had het,.}{koud ondanks de gevoerde}{parka die hij droeg}\\

\haiku{Leroy frunnikte.}{een moment aan de leren}{kinband van zijn helm}\\

\haiku{{\textquoteleft}Als de luitenant,.}{mij wil verexcuseren}{ik moet even pissen}\\

\haiku{De deur zwaaide wijd,.}{open nog voordat Latour had}{kunnen aankloppen}\\

\haiku{Hij kon eind vijftig,.}{zijn maar met gemak voor tien}{jaar ouder doorgaan}\\

\haiku{Het was al avond en,.}{donker in de hut van de}{Bisschop widdege}\\

\haiku{De man stond zonder,.}{iets te zeggen op en ging}{voor de gang in}\\

\haiku{enig begrip voor de.}{abnormaliteit hiervan}{bereikte haar niet}\\

\haiku{De regen van de'.}{afgelopen weken heeft}{nie veel goed gedaan}\\

\haiku{Blijf nie' achter, mijn.}{sporen verdwijnen heel wat}{vlotter als gij peinst}\\

\haiku{Ze schenen hem niet.}{te horen en gingen hun}{eigen hut binnen}\\

\haiku{{\textquoteleft}Eh... wat is... eh...{\textquoteright} {\textquoteleft}De,,{\textquoteright}.}{deur luitenant herhaalde}{de man geduldig}\\

\haiku{Op het zand lagen,,.}{in de vorm van een ruit vier}{dode konijnen}\\

\haiku{{\textquoteright} {\textquoteleft}Graag,{\textquoteright} zei Latour en.}{voelde zich belachelijk}{in zijn gretigheid}\\

\haiku{Wil men overleven,.}{dan moet straf koste wat kost}{vermeden worden}\\

\haiku{{\textquoteleft}We zijn er, dat is,{\textquoteright}.}{een gegeven antwoordde}{hij na een ogenblik}\\

\haiku{Ik wilde immers.}{revanche voor mijn verlies}{van de schoolmeester}\\

\haiku{De morgen van de.}{achtentwintigste was het}{weer niet verbeterd}\\

\haiku{D'n Pie  had een.}{stengun die hij ook leek te}{kunnen gebruiken}\\

\haiku{D'n Pie sloeg Wijngaards:}{op diens schouder om hem te}{bedanken en zei}\\

\haiku{totdat u mij die.}{papieren geeft waar ik}{voor gekomen ben}\\

\haiku{Toen liep hij op het,,.}{stilliggende lichaam zijn}{eigen lichaam toe}\\

\section{Johannes Kneppelhout}

\subsection{Uit: Studentenschetsen. Deel 1. Teksten (onder ps. Klikspaan)}

\haiku{s bekannt, Und ' '.}{wo ihrs packt da ists}{interessant}\\

\haiku{- O gulden vrijheid,!}{der Studentenwereld ik}{zal u nooit kennen}\\

\haiku{hij moet getuige:}{zijn van een gesprek waarin}{wreedelijk voorkomt}\\

\haiku{Is hij een Stoicus?}{die zich in de lijdzaamheid}{zoekt te oefenen}\\

\haiku{Drie weken daarna.}{bragt mij het toeval voor de}{derde maal bij hem}\\

\haiku{Gelukkig hij wiens!}{Aeskulaap de vrienden van}{zijne sponde weert}\\

\haiku{men komt zijn vriend geen,.}{gezelschap houden men komt}{koffijhuis houden}\\

\haiku{- zoo, ben jij daar nog? -.}{krijgt Hendrik een stoel bij het}{vuur en vat eene pijp}\\

\haiku{- Kr... - maar ik zou geen -.}{politiek aanroeren en}{andere dassen}\\

\haiku{Maar die menschen hier,.}{souperen nog comme au}{temps de nos p\`eres}\\

\haiku{Hij plaatst zich boven,,,.}{buiten ja tegenover de}{Studentenwereld}\\

\haiku{hij slaat zich de borst.}{kampot om iets te vinden}{dat er naar gelijkt}\\

\haiku{Op de zee van het;}{leven laat hij de hulk van}{den doctor zweven}\\

\haiku{Alleen omdat het.}{eene Dissertatie is zal}{men het niet inzien}\\

\haiku{Daar komt onverwachts:}{eene vervaarlijke stem van}{de achterkamer}\\

\haiku{- Toen werd papa, tot,;}{het uiterste gedreven}{woedend en razend}\\

\haiku{monumenten, voor?}{niemand toegankelijk dan}{die latijn verstaat}\\

\haiku{Het boek zag er uit,.}{als de dief zelf overal met}{smetten en scheuren}\\

\haiku{Ik vraag verschooning,.}{wij hebben slechts den naam met}{elkander gemeen}\\

\haiku{De hinderpalen,.}{waar hij niet over kan mogen}{links blijven liggen}\\

\haiku{Wie uwer schaamt zich zulk,?}{eene armhartige vleitaal}{niet mijne vrienden}\\

\haiku{Nu haast de fleemkous '.}{zich zijne makkers opt}{tapijt te brengen}\\

\haiku{Neen, 't is daar nog,.}{te vroeg voor bovendien is}{dit thans het doel niet}\\

\haiku{Het is een krakeel,,,.}{een oproer eene vischmarkt een}{bordeel van klanken}\\

\haiku{- Och, die gemeene,.}{Theologant ik weet zelf}{niet meer hoe hij heet}\\

\haiku{T is lichte maan.}{en op de sneeuw onderscheidt}{men gemakkelijk}\\

\haiku{En zij gehaat als,!}{trappenschuren Steeds zij zijn}{buidel zonder geld}\\

\haiku{Men vreest bij ons geen,.}{witte mouwen Wij smijten}{uit al wat ons knelt}\\

\haiku{de kerel is vast.}{een half uur te laat in de}{wereld gekomen}\\

\haiku{dat iemand joolig.}{en luchtig maakt zoodra}{hij er binnen komt}\\

\haiku{'T zij 'k avondrood,!}{of morgen zie Ik drink mijn}{glas Crambamboeli}\\

\haiku{Men bedenke slechts,!}{dat tijd hier niet staat voor dag}{of week maar voor maand}\\

\haiku{, zou het geen voortgang,.}{hebben gehad daar kunt gij}{verzekerd van zijn}\\

\haiku{, was het om vele.}{redenen zaak dat over dag}{vermeden werden}\\

\haiku{Flanor antwoordde met.}{een vreesselijken vuistslag}{op een Leidschen neus}\\

\haiku{Ge hebt immers zelf.}{gezien hoe de Leidenaars}{mij met steenen smeten}\\

\haiku{- You see how these,,,.}{fellows drink and smoke and}{roar replied Mr. Pickwick}\\

\haiku{Fest gemauert in.}{der Erden Steht die Form aus}{Lehm gebrannt}\\

\haiku{bij voorkeur neemt men,.}{C die reeds zoo dikwijls zijn}{hoofd heeft gestooten}\\

\haiku{s bekannt, Und ' '.}{wo ihrs packt da ists}{interessant}\\

\haiku{Naauwelijks bij de,.}{Stads-Gehoorzaal daar}{moest het er op los}\\

\haiku{En de vreemdeling:}{wijst er op met den vinger}{en zegt met deernis}\\

\haiku{Minerva, die op,:}{het Academiegebouw prijkt}{zijne schutsgodin}\\

\haiku{Gusje van Eijkens.}{beeldtenis teekent zich in}{de opening der deur}\\

\haiku{fluistert de deugniet.}{op plegtigen toon en den}{vinger voor den mond}\\

\haiku{dan liever zijne:}{schamelheid openlijk bekend}{en ronduit gezegd}\\

\haiku{- Ik zou gaarne mijn... -?}{testimonium hebben}{Van welk collegie}\\

\haiku{- Wat hebben we er!}{voor je ingezeten op}{sommige plaatsen}\\

\haiku{- Mijnheer heeft belet -:}{of Mijnheer schreeuwt zelf uit al}{zijne waardigheid}\\

\haiku{Maar wien van deze,?}{beide nu zal het gelden}{Lisse of ten Deyl}\\

\haiku{luidkeels uit, die het.}{naadje van de kous nog maar}{niet juist vinden kan}\\

\haiku{- Dat 's eene vervloekt.}{gemeene hatelijkheid}{op de jongelui}\\

\haiku{Maar dat tot aan uw,!}{dood mijn naam in uw gemoed}{Toch blijve wonen}\\

\haiku{Met een zoet lijntje.}{trachtte men hem weder naar}{binnen te krijgen}\\

\haiku{- Nog al! - En hoeveel?}{partijen worden er wel}{op een jaar gespeeld}\\

\haiku{zoekt hem vooral in.}{de laatste vaderlandsche}{gebeurtenissen}\\

\haiku{Vergunt uwen schrijver!}{een voorbeeld en wilt het hem}{ten goede houden}\\

\haiku{Dit alles weet ik,}{van alles wat gij mij daar}{tegenwerpt loochen}\\

\haiku{- En ben jij dan zoo?}{pedant van te denken dat}{je hier bent geweest}\\

\haiku{een glas jenever!}{aan mijnheer Bivalva voor}{mijne rekening}\\

\haiku{Van daar de twist, doch -!}{hij zal morgen wel met een}{bittertje of neen}\\

\haiku{Of het Studentje ', -!...}{t hoorde maar om er een}{ui op te zeggen}\\

\haiku{Ewoud brandt zich bij het;}{aansteken van zijn cigaar}{eene blaar op den neus}\\

\haiku{Velen overweldigt.}{de slaap en zitten stom als}{Egyptische beelden}\\

\haiku{De dageraad breekt.}{aan en nog duurt de nacht voor}{de feestvierders voort}\\

\haiku{Gusje van Yken met;}{de zijnen gaat ontbijten}{aan het Haagsche Schouw}\\

\haiku{Praat zoo tegen je,.}{sletten maar niet tegen een}{ordentelijk mensch}\\

\haiku{De een begraaft zich;}{onder boeken en sluit zich}{op tusschen muren}\\

\haiku{ik heb mijn tijd al}{staan te verbiljarden in}{den Paauw en ten zes}\\

\haiku{- meer dan gewone.}{veerkracht en liefde voor het}{Genootschap vereischt}\\

\haiku{Hoe ongelukkig -?}{dat door eene zekere hoe}{zal ik het noemen}\\

\haiku{hoe menigmaal men}{dezelfde phrase telkens}{op hare hielen}\\

\haiku{als er toch niet aan,!}{te doen is bedank dan ten}{minste in verzen}\\

\haiku{Dit herstelde hem.}{van zijne huivering en}{versterkt hief hij aan}\\

\haiku{Niemand onzer of,;}{hij heeft nog veel te leeren veel}{te verbeteren}\\

\haiku{Op elk bord lag \'e\'en;}{gebakken aardappel en}{\'e\'en gebraden ui}\\

\haiku{eene vete tusschen.}{de Buitengewone en}{Werkende Leden}\\

\haiku{want, indien iemand,}{hij was zwaar en moeijelijk}{ter  sprake.90}\\

\haiku{Verder, midden uit,;}{dien zwarten hoop verheft zich}{een boven allen}\\

\haiku{Wij bevelen ons,,.}{voor het vervolg in hare}{welwillendheid aan}\\

\haiku{Hier  niet meer de,;}{ruwste vuist die den schepter}{der heerschappij zwaait}\\

\haiku{Het is hier vooral,;}{het zedelijke overwigt}{dat gehuldigd wordt}\\

\haiku{Mevrouw Iburg kon men.}{slechts een weinig overdrijving}{te laste leggen}\\

\haiku{het zijn mannen en,;}{vrouwen die om den broode}{rollen opzeggen}\\

\haiku{Er moest een band zijn:}{tusschen den Hoogleeraar}{en den Muzenzoon}\\

\haiku{Dat nu deze soort!}{van Afleggers allen maar}{Bivalva's waren}\\

\haiku{waar den jongeling;}{de toegang steeds openstaat tot}{gezellig verkeer}\\

\haiku{Spoedig zongen het,,;}{tien spoedig honderd spoedig}{duizend straatjongens}\\

\haiku{{\textquoteright} Het gezelschap rukt,,.}{niet zonder gevolg de poort}{van Arnhem binnen}\\

\haiku{Men bevindt zich, naar,,.}{huis keerende voor een meer dat}{het Wielermeer heet}\\

\haiku{denk je, dat ik het,,!}{niet vervloek om voor jullie}{pleizier jan domie}\\

\haiku{zij zelve werken,;}{mede tot de betoovering}{die u overmeestert}\\

\haiku{Alzoo is ook de,,.}{tijd als zoodanig aan de}{muzijk vijandig}\\

\haiku{il appelle \`a;}{lui tous les enfans pauvres}{qui ont de la voix}\\

\haiku{zij doet stappen als,!}{een dragonder en wat zit}{zij dik in het vet}\\

\haiku{Gelieft mij slechts te.}{volgen bij het eindigen}{van het paardenspel}\\

\haiku{De wetenschap, het,,.}{gekozen vak dit zij de}{pit het middenpunt}\\

\haiku{je viendrai retremper!}{mon \^ame \`a l'abreuvoir de}{la gourmandise}\\

\haiku{Je t'assure.}{qu'il n'est pas facile de}{se faire juste}\\

\haiku{het ontbreekt mij aan.}{moed om de redoutables}{te negligeren}\\

\haiku{het is alles vorm,,,,:}{alles geest altijd lagchen}{stoeijen malligheid}\\

\haiku{il m\'eriterait.}{d'\^etre clou\'e \`a ter Gou ou}{\`a Egmondbuiten}\\

\haiku{Ah \c{c}a, mon cher, la.}{cloche sonne midi et}{mon caf\'e m'attend}\\

\haiku{Met regt mogt hij de,.}{verzekering geven dat}{hij niet dronken was}\\

\haiku{Me voil\`a \`a vingt-trois,.}{et tu n'es encore que}{vig\'esimaire}\\

\haiku{Une chaise de nuit,;}{et un gros bonnet fourr\'e}{voil\`a ce qu'il me faut}\\

\haiku{Il y avait plus de.}{trente estomacs au}{d{\^\i}ner doctoral}\\

\haiku{Mes saluts \`a tous.}{les amis et crois-moi}{quem nosti}\\

\haiku{De Dissertatie,.}{wordt gedrukt weldra zal de}{Promotie volgen}\\

\haiku{Mon cher ami,    Il;}{y a fort longtemps que je}{ne t'aper\c{c}ois plus}\\

\haiku{Loisir doublement.}{m\'erit\'e apr\`es tant de mois}{de piochage}\\

\haiku{roept Pluyx vrij luid, en.}{twintig stemmen herhalen}{dat heerlijke woord}\\

\haiku{- Och, zwijg toch, vriend, 't.}{is immers maar om onze}{dubbeltjes te doen}\\

\haiku{Ook is de soort van;}{geestigheid in het Verhaal}{hoogst oorspronkelijk}\\

\haiku{Het tweede couplet.}{van het eerste versje kan}{er naauwelijks door}\\

\haiku{De overbrenger had.}{er den oorspronkelijken}{geest uitgesneden}\\

\haiku{De twee volgende.}{coupletten zijn evenzeer fraai}{en vloeijend vertaald}\\

\haiku{- waart ge ellendig,,,;}{beroerd geesteloos ontbloot}{van alle talent}\\

\haiku{Geen liever is er, '.}{ooit ontmoet Geen trouwer ooit}{int minnen}\\

\haiku{Men moet zeggen, dat.}{we wonder wel aan onze}{bestemming voldoen}\\

\haiku{Gij, Redacteurs, zijt,,,.}{niet slecht niet onbeschoft niet}{lomp niet hatelijk}\\

\haiku{ik zou in ue.'s plaats;}{nooit den moed gehad hebben}{zoo iets te schrijven}\\

\haiku{De wetenschap, het,,.}{gekozen vak dit zij de}{pit het middenpunt}\\

\haiku{het is alles vorm,,,,:}{alles geest altijd lagchen}{stoeijen malligheid}\\

\haiku{Men heeft mij gezegd...:}{dat P mij gisteren is}{komen opzoeken}\\

\haiku{Deze vervloekte,!}{furore heeft mijn pen stomp}{gemaakt \'e\'en moment}\\

\haiku{ik snijd er een punt,,!}{aan een scherpe punt en wat}{voor een scherpe punt}\\

\haiku{Met regt mogt hij de,.}{verzekering geven dat}{hij niet dronken was}\\

\haiku{Voor een zin die niet,.}{van Bossuet afkomstig is}{is hij lang genoeg}\\

\haiku{Weer een jaar dat we.}{aan onze verzameling}{kunnen toevoegen}\\

\haiku{Is dandy-achtig?}{gedrag als het jouwe voor}{mij nog weggelegd}\\

\haiku{Ik zou willen dat.}{jij er een vertaling in}{versvorm van maakte}\\

\haiku{Naar uw stranden Hef ';}{k iedre avondstond mijn}{zegenende handen}\\

\haiku{De Dissertatie,.}{wordt gedrukt weldra zal de}{Promotie volgen}\\

\haiku{zo zul je zeggen, {\textquoteleft}.}{wat je alleen maar vanuit}{Parijs kunt schrijven}\\

\subsection{Uit: Studentenschetsen. Deel 2. Commentaar (onder ps. Klikspaan)}

\haiku{En de vreemdeling:}{wijst er op met den vinger}{en zegt met deernis}\\

\haiku{Op het omslag van, {\textquoteleft}{\textquoteright},:}{de tweede aflevering}{Wuftheid staat vermeld}\\

\haiku{Toen kwam ook het plan.}{op de afleveringen}{te illustreren}\\

\haiku{De student Schrijver,.}{tekenaar en lithograaf}{werkten nauw samen}\\

\haiku{Over de prent bij {\textquoteleft}Flanor{\textquoteright}:}{bijvoorbeeld schreef Ver Huell}{aan Kneppelhout}\\

\haiku{f 2,50 (Nieuwsblad voor,):}{den boekhandel 14 juni}{1860  Oplage}\\

\haiku{f 1,90 (gebonden) (,);}{Nieuwsblad voor den boekhandel}{4 oktober 1872}\\

\haiku{19 december 1873 (,);}{Nieuwsblad voor den boekhandel}{19 december 1873}\\

\haiku{Exemplaren daarvan.}{zijn tot op heden echter}{niet aangetroffen}\\

\haiku{het overgrote deel.}{van de lesuren werd besteed}{aan Latijn en Grieks}\\

\haiku{caf\'e-biljart,, ().}{aan de Nieuwe Rijn wijk 7}{nr. 27nu nr. 20}\\

\haiku{(Blok en Martin, De,-):;}{Senaatskamer p. 23}{Simplex eenvoudig}\\

\haiku{{\textquoteleft}violen{\textquoteright} - voor de).}{betaling van het gelag}{laten zorgen}\\

\haiku{\'e\'en voet telt twaalf duim.}{en is ongeveer dertig}{centimeter}\\

\haiku{de intocht van Jan ():}{van Beieren in Leiden}{1420   154ongewacht}\\

\haiku{Van Zonneveld, {\textquoteleft}Het{\textquoteright},):}{Leiden van Piet Paaltjens}{p. 15~210Waalboer}\\

\haiku{(Van Zonneveld, De,-):}{Romantische Club p. 81}{128~342den Burg}\\

\haiku{(Brom, Omkeer in 't,):}{studenteleven p. 63}{371novicius}\\

\haiku{- Kr... - maar ik zou geen -:}{politiek aanroeren en}{andere dassen}\\

\haiku{{\textquotedblleft}Kon ik maar een broek,!}{krijgen zoals jij ze in}{de Typen teekent}\\

\haiku{Organiek Besluit,):}{van 2 augustus 1815 art.}{84~258oosterling}\\

\haiku{Organiek Besluit,).}{van 2 augustus 1815 art.}{148 en 156       26}\\

\haiku{Organiek Besluit,):}{van 2 augustus 1815 art.}{56~389jura}\\

\haiku{(Schneider en Hemels,,) [...]:}{De Nederlandse krant p.}{150~16slaat een kout}\\

\haiku{Een nieuw treurspel over:}{de grootvorst ontlokte aan}{Beets de verzuchting}\\

\haiku{Organiek Besluit,) ':}{van 2 augustus 1815 art.}{205~467leit af}\\

\haiku{de Leidse kermis.}{duurde van hemelvaartsdag}{tot Pinksteren}\\

\haiku{want hetgeen ik wil,,,.}{dat doe ik niet maar hetgeen}{ik haat dat doe ik}\\

\haiku{er was in die tijd.}{geen hospes of hospita}{met deze naam}\\

\haiku{waarschijnlijk is het ().}{de voornaamVincent van de}{caf\'ebediende}\\

\haiku{wellicht is dat de.}{reden dat hij ruggelings}{is afgebeeld}\\

\haiku{Het was een dispuut {\textquoteleft}{\textquoteright},.}{voor deNieuwe letteren}{opgericht rond 1839}\\

\haiku{volksliedje op de {\textquoteleft}{\textquoteright},:}{wijs vanLes \'etudiants met}{als derde regel}\\

\haiku{21-22En de pet - Regt -, -:}{coquet Op \'e\'en oor Zwiert de}{breede straten door}\\

\haiku{{\textquoteleft}Daar ziet hij - maar met, (?), [...]{\textquoteright}.}{ernstige oogen/'t Hoofd met}{een Oreool omkranst}\\

\haiku{(Gids voor Leiden en,-):}{omstreken p. viii en}{7778~80Houri}\\

\haiku{Lady Macbeth draagt,, {\textquoteleft}{\textquoteright}.}{een kaars haar ogen zijn openbut}{their sense are shut}\\

\haiku{(Vademecum voor,-):}{den student p. 122123}{529kardinaal Puff}\\

\haiku{928-929dat zijne:}{pomp wel verstopt zou wezen}{van de haarpruiken}\\

\haiku{Het duel ({\textquoteleft}Mensur{\textquoteright});}{was een onder- ~deel}{van de erecode}\\

\haiku{Volgens Bilderdijk:}{wist Beyling welk gruwelijk}{einde hem wachtte}\\

\haiku{(Wetboek van het Strafregt,)-...}{p. 189 en 287~13781392Het}{doel zal toch wel zijn}\\

\haiku{de Leidse kermis.}{duurde van hemelvaartsdag}{tot Pinksteren}\\

\haiku{Klikspaan lijkt hier het:}{omgekeerde te zeggen}{van wat hij bedoelt}\\

\haiku{(De La Bruy\`ere,,):}{Oeuvres compl\`etes p. 478}{38wezenlijkheid}\\

\haiku{beslaat een deel van.}{de provincies Zuid- en}{Noord-Holland}\\

\haiku{9-10waar de vetste...:}{melk vloeiten de geurigste}{kaas bereid worden}\\

\haiku{der over elkander,:}{geslagene armen der}{duimpjesdraaijerij}\\

\haiku{met hun familie.}{wandelen en daar lelijk}{mee inzitten}\\

\haiku{289-298Wij sluiten in...}{geen kamermurenZoolang}{er wijn in flesschen}\\

\haiku{het zonder afspraak.}{door elkaar ondervragen}{van de studenten}\\

\haiku{eigenlijk een huis,.}{loods of schuur waar bokkingen}{gerookt worden}\\

\haiku{Organiek Besluit,):}{van 2 augustus 1815 art.}{87~23reizen}\\

\haiku{{\textquoteleft}Alle examina{\textquoteright}.}{zonder onderscheid moeten}{een vol uur duren}\\

\haiku{verwijzing naar de.}{mythe over de bruiloft van}{Peleus en Thetis}\\

\haiku{In deze schets speelt;}{het Academiegebouw een}{belangrijke rol}\\

\haiku{Openingswoord van de,.}{promovendus volgens een}{vaste formule}\\

\haiku{702-703als een god van:}{Homerus weggescholen}{in eene wolk van dauw}\\

\haiku{hij was gevestigd,, ().}{op de Koepoortsgracht wijk 2}{nr. 97nu nr. 34}\\

\haiku{verwijzing naar de:}{slotregel van het elfde}{couplet van psalm 118}\\

\haiku{1126-1127En een onzer:}{meest bekende geleerden}{het woord hybridisch}\\

\haiku{de verteller laadt;}{zijn pistool door een kogel}{in de loop te doen}\\

\haiku{Op 4 november.}{1839 was het opgevoerd in}{de Leidse schouwburg}\\

\haiku{1865-1866Want huwlijksheil:}{en vadervreugd/Boeit vaster}{dan een droom der jeugd}\\

\haiku{Vanaf 1839 werd de.}{redactie gekozen uit}{leden van het lsc}\\

\haiku{de daar genoemde)-:}{wet niet teruggevonden}{401402onder het zeil}\\

\haiku{(Lunsingh Scheurleer e.a.,,,-):}{Het Rapenburg dl. 1 p.}{271272~968vaak}\\

\haiku{Zij besloten de.}{dag met een groot feest op de}{soci\"eteit}\\

\haiku{Hij maakte reizen.}{door heel Europa en naar}{het Midden-Oosten}\\

\haiku{officier van de.}{met lansen bewapende}{cavalerie}\\

\haiku{Organiek Besluit,):}{van 2 augustus 1815 art.}{109 en 111~1490atoom}\\

\haiku{op 1 juni 1841.}{bracht koning Willem ii een}{bezoek aan Leiden}\\

\haiku{Chr.), het beroemde.}{boek over de retorica}{van Cicero}\\

\haiku{(Gids voor Leiden en,-).}{omstreken p. viii en}{p. 7778~      89}\\

\haiku{den tol, vreze, dien,,.}{gij de vreze eer die gij}{de eer schuldig zijt}\\

\haiku{Kevers sterven kort.}{na de bevruchting en het}{eieren leggen}\\

\haiku{Vluchtte na de val,.}{van Athene naar Perzi\"e waar}{hij werd vermoord}\\

\haiku{Chr.) bouwde voort op.}{de fundamenten die zijn}{vader had gelegd}\\

\haiku{respectievelijk.}{vergiffenis schenken en}{vergiftigen}\\

\haiku{[Namen der leden] [].}{volgens de orde waarin}{zij zittenZegel}\\

\haiku{de grootste beker,.}{de kleinere beker en}{de kleine beker}\\

\haiku{mottoQu'il est grand,...}{qu'il est beau de se dire}{\`a soi-m\^eme}\\

\haiku{Het accent lag nu.}{meer op het voordragen van}{werk van anderen}\\

\haiku{vanaf 1829 student),;}{theologie thesaurier}{of penningmeester}\\

\haiku{De ondergang der (-).}{eerste wareld18091810 van}{Willem Bilderdijk}\\

\haiku{De geschiedenis;}{van het gezelschap is goed}{gedocumenteerd}\\

\haiku{{\textquoteleft}violen{\textquoteright} - voor de).}{betaling van het gelag}{laten zorgen}\\

\haiku{144-145de almagt,:}{der kunst die steden bouwde}{en tijgers temde}\\

\haiku{uitbreiding van de {\textquoteleft}{\textquoteright} ().}{verwensingloop naar de maan}{en pluk starren}\\

\haiku{bepaalde pijpen,;}{bestonden uit een losse}{kop steel en mondstuk}\\

\haiku{Programma eerste:}{invitatieconcert sc}{286verdieping}\\

\haiku{354-356Beethoven [...] [...] [...],,,:}{Weber Rameau Mozart Gl\"uck}{Spohr Cherubini}\\

\haiku{De lezing {\textquoteleft}hunne{\textquoteright}.}{is ontleend aan de derde}{en vierde druk}\\

\haiku{verwijst mogelijk.}{naar discussies voorafgaand}{aan de dies van 1841}\\

\haiku{verkorte vorm van {\textquoteleft}{\textquoteright}, {\textquoteleft}{\textquoteright}.}{zetteden de verleden}{tijd vanzetten}\\

\haiku{verkorte vorm van {\textquoteleft}{\textquoteright}, {\textquoteleft}{\textquoteright}.}{zetteden de verleden}{tijd vanzetten}\\

\haiku{(Worp, Geschiedenis,,;}{van het drama en van het}{tooneel dl. 2 p. 377}\\

\haiku{Het stuk werd op 31.}{mei 1838 voor de eerste maal}{in Leiden vertoond}\\

\haiku{in mei 1843 trad hij.}{op ter gelegenheid van}{de Leidse kermis}\\

\haiku{Hij was de zoon van,.}{Ward Bingley een van de grootste}{acteurs van zijn tijd}\\

\haiku{bijlage 875 en):}{876~722ontstond er vrij}{wat minder nachtrumoer}\\

\haiku{De uitdrukking gaat (;}{terug op een passage}{in Tartuffe1669}\\

\haiku{eigenlijk een huis,.}{loods of schuur waar bokkingen}{gerookt worden}\\

\haiku{(Stokvis, De wording,-):}{van modern Den Haag p. 245}{246~397aanspraken}\\

\haiku{Organiek Besluit,):}{van 2 augustus 1815 art.}{104~436koesten}\\

\haiku{(Usener, {\textquoteleft}Maatschappij {\textquotedblleft}{\textquotedblright}{\textquoteright},;}{ijzergieterijDe prins}{van Oranje p. 389}\\

\haiku{Frans auteur (1806-1866),.}{van romans toneelstukken}{en feuilletons}\\

\haiku{het risico dat.}{hij zou worden nagevolgd}{was dus gering}\\

\haiku{Citaat uit Les chants ().}{du cr\'epuscule xiv1835}{van Victor Hugo}\\

\haiku{Zij mogen aan de [...]{\textquoteright}.}{huizen der professoren}{gehouden worden}\\

\haiku{verhaal van hetgeen,.}{elke dag voorvalt vooral}{van reizigers}\\

\haiku{Kopie\"en van het.}{beeld waren toen al over heel}{Europa verspreid}\\

\haiku{Voor het tweede deel.}{van de omschrijving is geen}{bron gevonden}\\

\haiku{Gerrit de Clercq (-),.}{18211857 vanaf 1839 student}{rechten te Leiden}\\

\haiku{n'interdis pas \`a:}{ma cendre les caveaux}{de Saint-Denis}\\

\haiku{Naar uw stranden Hef ';}{k iedre avondstond mijn}{zegenende handen}\\

\haiku{ook hier combineert;}{Klikspaan een wel en een niet}{bestaande senaat}\\

\haiku{Digesta, p. 860),:}{222Ornatissime quaenam}{fuerunt ultima}\\

\haiku{(Anoniem, Aballino,,):}{de groote bandiet p. 61}{7eigenaardigheid}\\

\haiku{[Anoniem], {\textquoteleft}Goethe en{\textquoteright}.}{eenige zijner beroemdste}{tijdgenooten}\\

\haiku{Dyserinck, J.,.}{Het studentenleven in}{de literatuur}\\

\haiku{2 dln. Paris, z.j.,, {\textquoteleft}.}{Heemskerk G.De bloem van de}{Leydsche academie}\\

\haiku{Gedenkboek van het.}{Collegium classicum}{cui nomen M.F. cond}\\

\haiku{Revue de Paris,, (),-.}{seconde \'edition dl.}{vi1834 p. 101116}\\

\haiku{Twee en veertigste.}{jaarboek van het genootschap}{Amstelodamum}\\

\haiku{Lochem, 1970 Koppen,,.}{C.A.J. van De geuzen van de}{negentiende eeuw}\\

\haiku{Nijland, J.A., Leven (-).}{en werken van Jacobus}{Bellamy17571786}\\

\haiku{Schulze, F./P. Ssymank,}{Das Deutsche Studententum}{van den aeltesten}\\

\haiku{Een zedekundig.}{tafereel uit het begin}{der vijftiende eeuw}\\

\haiku{2e vermeerderde, ',,.}{dr. z.p. z.j. Vissert Hooft}{H.P. De student Beets}\\

\haiku{88, 212, 360 Alexander,:}{Nikolajevitsj Grootvorst}{van Rusland   i}\\

\haiku{617 ii: 30, 345, 347,,,():}{424 544 Clercq Mathurin}{Joseph le   i}\\

\haiku{128, 141 Brieven over:}{den aard en de strekking van}{hooger onderwijs}\\

\haiku{664 ii: 175, 269, 415,,,,, (:}{504 525 557 573 Scriblerus}{Martinuszie ook}\\

\haiku{54Mededeling ().}{op het omslag van Leven}{v bis1 april 1842}\\

\haiku{uba 2350 h 9-10,.}{102Leidsch Dagblad 11 en 15 april}{1868 en 16 mei 1868}\\

\section{J.M.W. Knipscheer}

\subsection{Uit: De blauwe draak}

\haiku{Blijkbaar zie jij kans,.}{om nog iets van deze zaak}{terecht te brengen}\\

\haiku{Denkt u er w\`el om {\textquoteleft}{\textquoteright}!}{dat het woordGenade voor}{ons niet bestaat}\\

\haiku{En ze toonde mij,.}{het stofblik volgeladen}{met doode muggen}\\

\haiku{Al mijn spieren en.}{zenuwen waren nu op}{het hoogst gespannen}\\

\haiku{- Over een drietal uren,.}{zal het mij geoorloofd zijn}{u toe te laten}\\

\haiku{gezicht ging ik weer {\textquoteleft}{\textquoteright}.}{zitten enverdiepte mij}{weer in mijn lectuur}\\

\haiku{Ik volgde ze, bij, '.}{mezelf concludeerend datt}{wel eens mis kon zijn}\\

\haiku{In dat geval liet,!}{die andere car het niet}{op zich zitten hoor}\\

\haiku{Ik zei niets tegen.}{de anderen en wachtte}{nog maar eens even af}\\

\haiku{En ik vertelde.}{mijn wederwaardigheden}{op den terugweg}\\

\haiku{op je, omdat je!}{zoo'n stevig pantser tegen}{zijn aanvallen hebt}\\

\haiku{meestal komen;}{de hoofdpersonen zich al}{uit zichzelf melden}\\

\haiku{Het pistool gleed in.}{mijn zak en ik wendde mij}{nu naar het bureau}\\

\haiku{Zonder meer haastte.}{ik mij naar de woning van}{den betrokkene}\\

\haiku{barstte hij \'even los,.}{zoodat alle omstanders ons}{ineens aankeken}\\

\section{anoniem}

\subsection{Uit: 'Het dagverhaal van een onbekende. Een Gouds dagboek uit het jaar 1788 en later'}

\haiku{{\textbullet} De {\textquoteleft}Waarschouwing{\textquoteright} van--.}{07021788 staat ook in het}{Publicatieboek}\\

\haiku{de leede van de ',;}{oraniesocitijd voort Harthuijs}{meede gewapent}\\

\haiku{97Koestraat, thans is ().}{dit de Markthet deel tussen}{Groenendaal en Hoogstraat}\\

\haiku{de tegenstanders),.}{van elkaar te scheiden uit}{elkaar te houden}\\

\haiku{105Voluit is de.}{naam van dit regiment Bosc}{de la Calmette}\\

\section{Marie Koenen}

\subsection{Uit: Het hofke}

\haiku{{\textquoteright} Willem had vluchtig.}{omgezien naar Sanderkens}{kunstenmakerij}\\

\haiku{Maar niet lang liet hij,.}{zijn zachte zegenende hand}{op Milia's hoofd}\\

\haiku{{\textquoteright} 't Was Willem, die, '.}{daar voor haar stond voort eerst}{na die \`eenen avond}\\

\haiku{Vaag schemerden er.}{de muren van het Hofke}{tusschen de boomen}\\

\haiku{{\textquoteright} Grave was met zijn.}{sjees stedewaarts geweest voor}{den paardenhandel}\\

\haiku{{\textquoteleft}Terstegen, hoe oud,.}{is die dochter van jou ik}{bedoel je voorkind}\\

\haiku{Hij geeft den kleinen.}{buurman een gemoedelijk}{tikje op het hoofd}\\

\haiku{En toch is het de,.}{vergiffenis verzoenend}{en vereffenend}\\

\haiku{{\textquoteright} {\textquoteleft}Wel dan, ik zal h\`aar!}{de deur van het Hofke niet}{gesloten houden}\\

\haiku{{\textquoteleft}Hij slaapt{\textquoteright}, beduidt hij,.}{zijn kameraad wenkend en}{wijzend met den blik}\\

\haiku{wat bleek, wat ernstig,.}{en stil en schijnt te droomen}{van verre dingen}\\

\haiku{Er vlamt iets op in,.}{zijn gedachten zijn hoofd wordt}{warmer en warmer}\\

\haiku{Waarom zouden wij?}{samen geen nieuwen tijd op}{het Hofke brengen}\\

\haiku{Zij bestond alleen,.}{al het overige verzonk}{in nevels voor hem}\\

\haiku{En nu die Thielde,!}{met haar aanstellerij en}{haar brutale oogen}\\

\haiku{{\textquoteleft}Wie had gedacht, dat?}{de dingen nog zoo'n goeden}{keer zouden nemen}\\

\haiku{Op het Hofke zat;}{Milia dien middag voor het}{wijd-open venster}\\

\haiku{En had ze ooit zooals,:}{nu kunnen begrijpen wat}{het beduiden wil}\\

\haiku{Hij had verwijten,.}{verwacht uitbarstingen van}{verdriet bij Milia}\\

\haiku{dat de beesten ziek,,;}{en dood ook al in den stal}{werden gevonden}\\

\haiku{Daar ging hem dan ten!}{laatste het Hofke toch nog}{in handen vallen}\\

\haiku{Hij durfde er niet,}{over spreken dat zijn vader}{hem gezonden had}\\

\haiku{{\textquoteleft}Het is niet om mij,{\textquoteright},.}{het is om de kinderen}{zei hij nog zachter}\\

\haiku{Ze heeft nooit zoo op.}{haar moeder geleken als}{in dit oogenblik}\\

\haiku{Het huis van Sander.}{en Thielde lag in dezen}{tijd reeds verlaten}\\

\subsection{Uit: De korrel in de voor}

\haiku{Onderwijl was op.}{Garversberg de heele buurt}{al in opschudding}\\

\haiku{{\textquoteright} Plonia joeg Paulus.}{en den oudste van Ruiters}{haar zoon achterna}\\

\haiku{Zwarte Marjan, om, ';}{met geen tang aan te pakken}{enr drie jongens}\\

\haiku{{\textquoteleft}Niet toegapen, maar,{\textquoteright}.}{bidden allemaal samen}{beheerde Drikus}\\

\haiku{Zelfs de belhamels.}{van Zwarte Marjan trokken}{de pet van het hoofd}\\

\haiku{een dochter van de.}{baronnen van Laag Case}{met den koetsierszoon}\\

\haiku{Dat wist hij toch wel, '?}{waart bij den rentmeester}{van den baron was}\\

\haiku{Alzoo - bij den heer.}{notaris zeggen juist zooals}{bij de anderen}\\

\haiku{Waar was dan ook z'n?}{goede moed gebleven van}{midden op den dag}\\

\haiku{Daar stond hij opeens, '.}{recht met een schok omt van}{zich af te schudden}\\

\haiku{{\textquoteright} Zoo pleitte Nelis,,.}{voor zichzelf barmhartig z'n}{doen met den hond goed}\\

\haiku{- Zou Nelis Broens soms, '?}{niet weten hoet behoort}{op een heerenhof}\\

\haiku{al zoo vroeg met de,...}{melk in het voorhuis juist als}{zij naar de kerk ging}\\

\haiku{Daar zullen we dan,,!}{om bidden moeder om zoo'n}{bouwknecht uit duizend}\\

\haiku{De beslissing hoeft.}{immers niet vandaag nog te}{worden genomen}\\

\haiku{Na Amen en kruisteeken ':}{was het eerste woord vant}{Rosalien voor h\'em}\\

\haiku{Ze had zijn naam zelfs.}{nog niet voor haar eigen en}{niet voor God genoemd}\\

\haiku{'k Zou je zoo graag '.}{bij de Zusters van Overdael}{int klooster zien}\\

\haiku{Niks hoef je te doen,,:}{dan er te vragen wat wij}{er wilden vragen}\\

\haiku{{\textquoteright} Dat was al, wat hij '.}{aant Rosalien terug}{had te boodschappen}\\

\haiku{{\textquoteright} {\textquoteleft}Maar, - je hebt voor ons?}{je plaats bij de Zusters toch}{niet opgegeven}\\

\haiku{{\textquoteright} Wel, zie, - dacht Nelis -.}{of die Leonardus ook}{oogen in z'n kop heeft}\\

\haiku{En echt blij was ze,:}{haar met een gerust hart te}{kunnen antwoorden}\\

\haiku{Een woord te weinig,,.}{kind is dikwijls nog erger}{dan een woord te veel}\\

\haiku{dat het heele dorp...}{blij mocht zijn met zoo'n aanwinst}{als Leonardus}\\

\haiku{Vandaag klonk hem haar.}{hartewoord als een lach en}{een snik tegelijk}\\

\haiku{En toch - ze kon er.}{zich hoegenaamd niks goeds meer}{van voorspiegelen}\\

\haiku{{\textquoteleft}Wat eenmaal den stoot,...}{heeft gekregen bergaf blijft}{naar omlaag rollen}\\

\haiku{Zooals Leonardus.}{dat zoo goed onthouden had}{uit het Evangelie}\\

\haiku{Nelis stond, eer hij ',.}{t wist aan de schuurdeur naar}{Peereneer uit te kijken}\\

\haiku{{\textquoteleft}Dat ziet er hier nu!}{toch anders uit dan omtrent}{Allerheiligen}\\

\haiku{Almaar berg-op,,.}{berg-af als een blind paard}{in den tredmolen}\\

\haiku{{\textquoteleft}Mariajoos nog toe, '!}{alst hem maar niet in den}{kop is geslagen}\\

\haiku{{\textquoteright} Opgewonden sloeg,.}{ze zich door den oploop heen}{recht op den bergrand aan}\\

\haiku{Den varkensketel.}{vullen met koolbladen en}{stronken en afval}\\

\haiku{Had die stiekemerd '?}{vann Neliske daar den}{wind van gekregen}\\

\haiku{Hij kwam recht en dronk,,,.}{gehoorzaam zwolg zoo vlug als}{eenigszins mogelijk}\\

\haiku{Met den rug van haar.}{hand streek ze zich schichtig de}{tranen uit de oogen}\\

\haiku{Leonardus stond.}{hem midden op den zolder}{al op te wachten}\\

\haiku{{\textquoteright} - Al zou hij er ook.}{alleen in de uiterste}{noodzaak aan roeren}\\

\haiku{Als ge ze nou nog...}{met u twee\"en naar boven}{zoudt willen dragen}\\

\haiku{{\textquoteleft}Franciscus, ga, en,,.}{bouw Mijn kerk weer op die in}{puin valt zooals ge ziet}\\

\haiku{Hij sloeg het dek op,,.}{en daar stond moeder Plonia}{ook met een lantaarn}\\

\haiku{nou is de weg naar.}{Garversberg voortaan niet meer}{te ver voor die twee}\\

\haiku{het zei, toch wees hij.}{met een los gebaar den kant}{van Laag Case uit}\\

\haiku{{\textquoteright} Het eigenlijke,:}{van de Derde Orde dat}{wist Nelis nu wel}\\

\haiku{Mijnheer pastoor bij,...}{ons zal er mettertijd wel}{een weg op weten}\\

\haiku{{\textquoteleft}Geef, goede God, dat...{\textquoteright}}{het tweede weer een zoon mag}{zijn Den volgenden}\\

\haiku{Heil en zegen bracht,.}{ze mee beweerden Plonia}{en de buurvrouwen}\\

\haiku{- {\textquoteleft}Vijf stappen vaneen,,...}{en niet eens te kunnen gaan}{zien of ze goedligt}\\

\haiku{En zelfs Ferdinand,,.}{stond het span als een kenner}{te bewonderen}\\

\haiku{Ze probeerde 't,.}{wel vooral ook om hem wat}{op te vroolijken}\\

\haiku{{\textquoteright} {\textquoteleft}Als we Nelis maar '?}{vastns om de vigilant}{naar Overdael stuurden}\\

\haiku{Maar daarmee ook - tot -!}{den laatsten halven cent voor}{Bella geblazen}\\

\haiku{Nelis was al den,,...}{inrij uit handen in de}{zakken blik ver weg}\\

\haiku{Want wat zou 'k daar, '?}{kunnen zeggen zonder dat}{gijt me voorzegt}\\

\haiku{Bij de bocht klom-ie,, -.}{den hoogen kant op om den weg}{te overzien naar links}\\

\haiku{Mijn jagershoed met.}{het fazantenpluimke zou}{hem beter kleeden}\\

\haiku{Ons den oudste nou?}{al en voorgoed heelemaal}{afhandig maken}\\

\haiku{- {\textquoteleft}Hij kijkt je naar de, '.}{oogen en ziet meteen oft}{meenens is of niet}\\

\haiku{Maaien... {\textquoteleft}Achter mij,{\textquoteright}.}{blijven gebood Plonia den}{ganschen morgen door}\\

\haiku{Ze was zelfs nog nooit.}{zoo zwijgzaam geweest als van}{dit oogenblik af}\\

\haiku{onmogelijk van '.}{Truuke te kunnen houden als}{n man van z'n vrouw}\\

\haiku{s Maandags nog voor ':}{den middag vernam hijt}{al van z'n moeder}\\

\haiku{{\textquoteright} En v\'o\'or den eten, in,:}{de volle achterkeuken}{Truuke zelf al direct}\\

\haiku{{\textquoteright} - Iets anders wilde.}{er niet in z'n gedachten}{opleven voor haar}\\

\haiku{Dat k\'on Nelis toch, '.}{den schoft z'n verachting in}{t gezicht blazen}\\

\haiku{Over het kozijn heen - -!}{zag hij een bleeke schim in den}{sterrenschemer Truuke}\\

\haiku{- Zonde voor God zou ',.}{t zijn hem in z'n roeping}{tegen te werken}\\

\haiku{- enkel en alleen,.}{op kattekwaad gespitst geen}{moment opletten}\\

\haiku{Te zien, hoe er niets...}{was dan volop reden tot}{overgroote dankbaarheid}\\

\haiku{Uit den hoofddoek keek '.}{r gezicht ouwelijk bleek}{en spits geworden}\\

\haiku{{\textquoteright} {\textquoteleft}Nog geen vier maanden,{\textquoteright} '.}{meer kniktet Rosalien}{haar veelzeggend toe}\\

\haiku{of die genegen..}{is zijn part van Garverskamp}{aan ons over te doen}\\

\haiku{Dat had Gradus hem '.}{bijt eerste woord evengoed}{willen toetellen}\\

\haiku{{\textquoteright} - {\textquoteleft}Vraag dat maar 'ns op,!}{den Bulthoek waar Nelis Broens}{goed genoeg voor is}\\

\haiku{God in den Hemel:}{en dien naasten voorspreker}{van hem daarboven}\\

\haiku{Het derde drietal,:}{in den stoet ze dienen meer}{tot opluistering}\\

\haiku{Onder de kanten,:}{doopdekens uit worden ze}{te voorschijn gebracht}\\

\haiku{- Tot eindelijk ook,...}{die donkere oogen opengaan}{groot en verwonderd}\\

\haiku{Toch weet Nelis wel,.}{hoe hij siddert en beeft voor}{dien woesten Colla}\\

\haiku{{\textquoteright} zegt 't Reeke, en strekt.}{z'n rechterhandje al uit}{om hem te aaien}\\

\haiku{de tortels van Sint,{\textquoteright}.}{Siskus en hij wijst omhoog}{het hellingbosch in}\\

\haiku{Verleden zomer.}{eerst is het andere er}{tusschengekomen}\\

\haiku{Al wekenlang zie ' '.}{k uit om jouns alleen}{te kunnen spreken}\\

\haiku{Als ze vast maar aan '..}{t wankelen zijn in hun}{vertrouwen op mij}\\

\haiku{- {\textquoteleft}Loop hard, Trinette.}{vragen er stevig een nat}{verband om te doen}\\

\haiku{{\textquoteleft}Ik kom v\'o\'or den avond ',?}{ns even hooren wat er toch}{aan de hand is hier}\\

\haiku{{\textquoteleft}Je doet of hij wel...}{de grootste misdaad van de}{wereld heeft begaan}\\

\haiku{Maar aldoor bleef 'k,.}{hopen dat het ware toch}{nogwel komen zou}\\

\haiku{ten opzichte van;}{Colla meer en meer gestijfd}{in z'n tegenzin}\\

\haiku{Met Ferdinand daar.}{en Trinette van u als}{tusschenpersonen}\\

\haiku{{\textquoteleft}Laat mij nou eerst 'ns,{\textquoteright},.}{uitspreken Wevers kalmeert}{hem de bezoekster}\\

\haiku{ik heb ditonderwerp.}{tot nog toe maar liever niet}{willen aanroeren}\\

\haiku{geen stroobreed zal ze.}{het geluk van haar heerke}{in den weg leggen}\\

\haiku{Haar eigen portret!}{in een breloque voor z'n}{horlogeketting}\\

\haiku{{\textquoteright} {\textquoteleft}Die Zwaan doet opgeld!}{tegenwoordig voor de twee}{boezemvriendinnen}\\

\haiku{En 't Rosalien,:}{is echt opgelucht dat ze}{er uit kan brengen}\\

\haiku{Al zal hij hun wel, -!}{de ooren van het hoofd eten}{zoo'n reus in z'n groei}\\

\haiku{Je weet nogwel. - {\textquoteleft}Naar,{\textquoteright} {\textquoteleft} '! - -?}{moeder dacht ikoft komt}{er van En dan zij}\\

\haiku{Als de meester ze,}{maar weer allemaal rondom}{zich ziet de zijnen.\ensuremath{\therefore}}\\

\subsection{Uit: De moeder}

\haiku{Ja, ze kan met een.}{vrij en gerust hart op haar}{leven terugzien}\\

\haiku{Tila begon mee.}{te naaien en het werk vloog}{hun uit de handen}\\

\haiku{{\textquoteright} {\textquoteleft}En Manders zou bij{\textquoteright}.....}{den bovenmeester Kroes een}{goed woord kunnen doen}\\

\haiku{Door 't eerste raam,,.}{dat van den winkel schijnt wat}{flauw licht den weg over}\\

\haiku{Moeder Severiens,.}{talmt verwonderd opgeschrikt}{uit haar gemijmer}\\

\haiku{De winkeldeur staat,,.}{open en achter den winkel}{de keukendeur ook}\\

\haiku{{\textquoteright} Er flitst een booze vlam '.}{uit Tila's oogen en het bloed}{slaat haar naart hoofd}\\

\haiku{{\textquoteleft}Ge moet beginnen,.}{met morgen naar Manders te}{gaan tusschen schooltijd}\\

\haiku{Nu met Paschen.}{was er over dat vergane}{loof een dunne rijm}\\

\haiku{Tila heeft al haar.}{gebeden gepreveld en}{toch niet gebeden}\\

\haiku{Zie, nergens is de.}{Geul zoo vreedzaam en gedwee}{als hier in Vlake}\\

\haiku{Of ze aldoor d'r.}{uitval van dien Aprilavond}{wilde goedmaken}\\

\haiku{Ik heb nu gezien - '.}{hoe dat kann nevel van}{goud hing door het dal}\\

\haiku{D'r tanden blinken.}{tusschen de in trotschen lach}{trillende lippen}\\

\haiku{Nooit heeft ze zoo scherp.}{alles gehoord en gezien}{als dezen middag}\\

\haiku{Moeder Severiens '.}{bukt naarn klaproos tusschen}{de schriele halmen}\\

\haiku{Maar losjes wippend:}{op z'n voetzolen lacht hij}{ineens luchthartig}\\

\haiku{{\textquoteright} {\textquoteleft}Leonie{\textquoteright} maant d'r, {\textquoteleft}, '.}{moederkom liever binnen}{t wordt onze tijd}\\

\haiku{Tot Sint-Jan{\textquoteright} mompelt,.}{Leonie hem toe terwijl}{ze de deur uitgaan}\\

\haiku{{\textquoteright} {\textquoteleft}Die heb ik ook{\textquoteright} poogt, '.}{het meisje te lachen maar}{t is als een snik}\\

\haiku{{\textquoteleft}'k Moet gaan{\textquoteright} ineens.}{verlegen trekt Treeske zich}{terug naar de deur}\\

\haiku{{\textquoteleft}Jongen, nu raad 'ns,?}{wie er is geweest wie daar}{op den stond wegging}\\

\haiku{{\textquoteright} hervat ze bijna,.}{verwijtend nu ze de lamp}{heeft aangestoken}\\

\haiku{{\textquoteleft}Ge moet met Louis naar.}{Berghof om uw huwelijk}{bekend te maken}\\

\haiku{Hier, je krijgt op den '.}{koop toen suikerboon om}{op te sabbelen}\\

\haiku{{\textquoteleft}Is 't waar - hebt gij,?}{dat allemaal klaargemaakt}{madame Curvers}\\

\haiku{'t Zingt in hun hart,,.}{maar ze spreken niet droomen}{maar en luisteren}\\

\haiku{In de gang en in.}{de voorzaal gaat het roezen}{en rumoeren voort}\\

\haiku{Jules' moeder legt.}{haar warme hand op die van}{Treeske in haar schoot}\\

\haiku{En Jules zoekt en ',.}{zoekt naart rechte woord nu}{weet niet wat en hoe}\\

\haiku{{\textquoteleft}Goddank{\textquoteright} verademt, '.}{moeder Severiens als ze}{opt plein komen}\\

\haiku{Zij stappen snel en.}{schuw het donker in van de}{Daelhoverstraat}\\

\haiku{Dat hij zich nu in,.}{Godsnaam toch sterk houdt zorgt haar}{alles te zeggen}\\

\haiku{t Gesprek gaat voort,.}{deint telkens tusschen gescherts}{en bespiegeling}\\

\haiku{Hier in de keuken ',.}{konden wet hooren of}{we er bij waren}\\

\haiku{Ze zijn op den spronk.}{en treden haastig op de}{twee wachtenden toe}\\

\haiku{t Lijkt Jules of.}{ze zich schreiend aan z'n hart}{wil komen bergen}\\

\haiku{{\textquoteright}.... Heeft hij die woorden?}{van het Hooglied  op z'n}{viool gezongen}\\

\haiku{Moeder Severiens'.}{gedachten zwerven weg in}{een zonnigen droom}\\

\haiku{t Ergste is dat.}{wij daardoor op schrikkelijk}{hooge lasten zitten}\\

\haiku{{\textquoteright} {\textquoteleft}Ge hadt het aan uw.}{moeder eerder en anders}{dienen te zeggen}\\

\haiku{En toch zal ze voor!}{de zooveelste maal er maar}{weer overheen moeten}\\

\haiku{Altijd dat studeeren,,.}{en sterk is hij nooit geweest}{nooit als anderen}\\

\haiku{hij zit verscholen,.}{de voeten langs de lage}{oeverglooiing neer}\\

\haiku{Nu eerst voelt ze hoe,.}{koud en bevend z'n handen}{zijn hoe klam z'n kleeren}\\

\haiku{Natuurlijk dat hij.}{in z'n koortsdroom juist over dit}{alles bezig was}\\

\haiku{Vaal-blauw komt het}{vroegste schemeren van den}{nieuwen dag over hem}\\

\haiku{{\textquoteright} Moeder Severiens,.}{weet niet waarom ook haar spraak}{ineens zoo beklemt}\\

\haiku{Nu de andere,,:}{haar aanziet wachtend op het}{eindwoord zegt ze stroef}\\

\haiku{Heel z'n lichaam is.}{in siddering en door z'n}{hoofd warrelt een storm}\\

\haiku{Z'n laatste maanden.}{in Vlake sprak Jules geen}{woord meer over Treeske}\\

\haiku{Is't de eenzaamheid?}{die haar in korte maanden}{zoo heeft veranderd}\\

\haiku{Er liggen 'n paar,.}{dorre olmblaren die ze}{met den voet wegschuift}\\

\haiku{{\textquoteright} Louis z'n oogen en z'n.}{mond blijven gesperd in een}{verdwaasden glimlach}\\

\haiku{Neen, Jules zal haar!...}{niet heelemaal als een oud}{mensch terugvinden}\\

\haiku{Haastig schuift ze 't.}{takje tusschen twee pakken}{wol in de muurkast}\\

\haiku{Ze ziet nu eerst dat.}{d'r moeder met starende}{oogen zit te schreien}\\

\haiku{En waarom niet trotsch,....}{op haar jongen dien de ernst}{uit de oogen donkert}\\

\haiku{Maar bij z'n zware....}{stem schudt moeder Severiens}{afwerend het hoofd}\\

\haiku{Dat zal haar goeddoen{\textquoteright},.}{begint dan ineens tegen}{Dolfke te spelen}\\

\haiku{Maar Treeske heeft met:}{sluippassen een stoel gehaald}{en prevelt haar toe}\\

\haiku{{\textquoteleft}Waarom zou ik 't, '?}{niet goedvinden als zijt}{zoo graag zou hebben}\\

\haiku{Moeder Severiens.}{ligt stil met opgewend hoofd}{en gesloten oogen}\\

\subsection{Uit: Wassend graan}

\haiku{{\textquoteright} Tot haar gelukkig:}{groote Maria en Lucia in}{duo te hulp komen}\\

\haiku{{\textquoteright} z'n schoonmoeder kan ':}{niet laten ook een duit in}{t zakje te doen}\\

\haiku{We hebben 't hier,.}{waarachtig best naar onzen}{zin moeder Wevers}\\

\haiku{Haar {\textquoteleft}Heerke{\textquoteright} is er,.}{voor in de wieg gelegd dat}{weet z'n moeder wel}\\

\haiku{Frans verschrikt van de.}{schrille jaloezie in de}{stem van z'n meisje}\\

\haiku{ze spelt hem dit als ',.}{t ware voor en hij moet}{het wel herhalen}\\

\haiku{{\textquoteright} - {\textquoteleft}Vraag dat liever maar ',{\textquoteright}.}{ns aan Frans wijst Anneke}{haar vinnig terecht}\\

\haiku{{\textquoteleft}Moest ik dien braven?}{heerenknol hier soms van dorst}{laten versmachten}\\

\haiku{Aanvankelijk had:}{z'n aanstaande schoonmoeder}{daar veel op tegen}\\

\haiku{Lucia monter als, '.}{nog van-haar-leven niet met}{n kleur van plezier}\\

\haiku{Je hebt 't nou zelf,{\textquoteright},.}{bijgewoond begint hij z'n}{stem nog niet meester}\\

\haiku{{\textquoteleft}Wie zou dien deugniet,?}{in toom moeten houden als}{ik er niet meer was}\\

\haiku{{\textquoteright} {\textquoteleft}Dus in elk geval '.}{is hij daar nogn jaarlang}{goed opgeborgen}\\

\haiku{{\textquoteleft}Ik weet wel, 't is, -.}{niks voor jou de dochter van}{den burgemeester}\\

\haiku{{\textquoteright} Plotseling stelt hij,!}{zich ook schrap met even veel recht}{toch zeker als zij}\\

\haiku{{\textquoteright} 't Ontbreekt Frans aan.}{kalmte en kracht om op haar}{woorden in te gaan}\\

\haiku{{\textquoteright} Om hem te sparen, ' ':}{wilt Rosalient hem}{niet laten hooren}\\

\haiku{- Maar dat kan ik bij.}{jou immers heelemaal niet}{veronderstellen}\\

\haiku{{\textquoteleft}En graag of niet, u.}{zult mij moeten dulden als}{meester op den Hof}\\

\haiku{Allesbehalve, '!}{een parmante schalk maarn}{wijsneus in persoon}\\

\haiku{'t Lijkt haar eerder.}{of er een eeuwigheid ligt}{tusschen toen en thans}\\

\haiku{t Rosalien weet,.}{genoeg dat er in het dorp}{om gelachen wordt}\\

\haiku{{\textquoteleft}Is me dat nou 'n,!}{partuur voor ons Anneke}{die jonge Wevers}\\

\haiku{Als ik je zeg, dat! '}{de ware Jozef al op}{haar te wachten staat}\\

\haiku{Jong als Frans is, en.}{vol illusies als hij was}{over z'n Anneke}\\

\haiku{- Maar nou wordt het dan,.}{toch meer dan tijd er mee voor}{den dag te komen}\\

\haiku{- Voor 't eerst van haar.}{leven op Zondag niet naar}{de Heilige Mis}\\

\haiku{Maar ja, dezen keer '.}{ist natuurlijk om dat}{lied van de bruiloft}\\

\haiku{Ondanks z'n negen.}{kinderen heeft hij dat nooit}{mogen beleven}\\

\haiku{{\textquoteright} {\textquoteleft}Alsof ik ookmaar '!}{n halfuur met dien vlegel}{overweg zou kunnen}\\

\haiku{{\textquoteright} {\textquoteleft}Willen is kunnen,, -?}{Frans dat hoeft je moeder je}{toch niet meer te leeren}\\

\haiku{Deze Cecile.}{Steeg scheen nou opeens Rita}{te hebben ontdekt}\\

\haiku{{\textquoteleft}Al van negen uur,.}{op weg winkelwaar voor me}{halen in Overdael}\\

\haiku{Al zou dat er nog, '.}{van komen alst zoo moest}{aanhouden met mij}\\

\haiku{Van dezen middag,,{\textquoteright}.}{te beginnen als je wilt}{zet Gregoire door}\\

\haiku{{\textquoteright} - Waarin heer Frans dan.}{niets dan wrange jaloezie}{meende te hooren}\\

\haiku{{\textquoteleft}Kom zeg, bemoeien.}{jullie je liever alleen}{maar met de poppen}\\

\haiku{Gejaagd streek ze het,.}{voorhuis in terwijl hij haar}{hardop uitlachte}\\

\haiku{Geen wonder dus, dat '.}{zet hoe langer hoe meer}{bij elkaar zochten}\\

\haiku{Wezenlijk, ik ben.}{niet van kraakporcelein zooals}{Anneke Reinders}\\

\haiku{{\textquoteright} Wat ze feitelijk,,}{bedoelde begreep Frans niet}{goed maar meer en meer}\\

\haiku{Allebei hadden.}{ze den adem ingehouden}{om te luisteren}\\

\haiku{hoe bitter weinig.}{uithoudingsvermogen hij}{eigenlijk nog had}\\

\haiku{Ga  'ns naar h\`em, -.}{toe laat dat uw eerste gang}{zijn op eigen beenen}\\

\haiku{Want immers juist iets,:}{voor Anneke hem zoo fijn}{te laten verstaan}\\

\haiku{En als jij er nou,... ':}{ook bijkomt zoomaar vanzelf}{t zou zoo goed zijn}\\

\haiku{opeens alles weer,...}{in orde en op z'n ouds}{ook met Trinette}\\

\haiku{Je kunt niet gelooven, '.}{watn aantrekkingskracht zoo'n}{kind in de wieg heeft}\\

\haiku{M'n geweten heeft!}{me nou lang genoeg gekweld}{over die eereschuld}\\

\haiku{{\textquoteright} - {\textquoteleft}Allemaal goed en, '!}{wel maar veels te veel ineens}{voorn eersten gang}\\

\haiku{Verwijten zal ze, '.}{hem niets enkel hem maar op}{n afstand houden}\\

\haiku{{\textquoteleft}respect{\textquoteright}, en {\textquoteleft}hoe jij '{\textquoteright}...}{voor ons allemaal t\`ochn}{vader bent geweest}\\

\haiku{{\textquoteleft}Als nonk Nelis niet,,!}{naar hem komt komt hij naar nonk}{Nelis die van ons}\\

\haiku{Immers keer na keer:}{werd hem dat opgelegd in}{den loop der jaren}\\

\haiku{- aan een moeder de?}{eerste tijding brengen van}{den dood van haar kind}\\

\haiku{Er na greep ze z'n... {\textquotedblleft},{\textquotedblright}, {\textquotedblleft}}{handen en riep hem bij}{z'n naamFrans riep ze}\\

\haiku{{\textquoteleft}Neen, Nelis... laat mij -.}{maar ik heb meer aan hem goed}{te maken dan jij}\\

\haiku{Ik heb hem eerst ook '.}{allesbehalven goed}{hart toegedragen}\\

\haiku{om het kribje, om, - ':}{het zingen zonderdat ze}{t zelf besefte}\\

\haiku{Aarzelend bij den,...}{drempel kon ze niet anders}{dan een kruis maken}\\

\haiku{{\textquoteleft}En mag 'k dan v\'o\'or?}{m'n vertrek nog \'e\'en keer bij}{u terugkomen}\\

\haiku{{\textquoteright} {\textquoteleft}Alle respect voor,{\textquoteright} '.}{uw courage kniktet}{Rosalien hem toe}\\

\haiku{Dank zij ook de  , '!}{goede verzorging diek}{allewijl geniet}\\

\haiku{'t Rosalien was,.}{bekans al opweg maar toen}{bedacht ze zich toch}\\

\haiku{Al kenden we dien -.}{al sindslang te veel kwam er}{zich tusschen stellen}\\

\haiku{Enkel en alleen '.}{dan tochmaar omdatt hem}{zelf haast overmand had}\\

\haiku{vroeger had hij wel:}{meermalen voor korter of}{langer tijd geloofd}\\

\haiku{Dat crucefix zal,, -:}{er hangen eer iemand er}{erg in heeft ja juist}\\

\haiku{Dienzelfden avond nog,:}{opeens de heele keuken}{vol van z'n geluk}\\

\haiku{{\textquoteright} {\textquoteleft}Maar dan zullen we '!}{toch eerst Jeskens nader}{moeten leeren kennen}\\

\haiku{Nelis uit, zonder '?}{er tegen hemn woord van}{gerept te hebben}\\

\haiku{{\textquoteleft}Was jij maar zoo ver,.}{dat je de teugels hier in}{handen kon nemen}\\

\haiku{{\textquoteleft}Zeg aan Nelis,{\textquoteright} al.}{eveneens telkens iets anders}{dat geen uitstel leed}\\

\haiku{t Boterde niet!}{bijster tusschen vader en}{zoon Alexander Doree}\\

\haiku{Die kwestie zou hij.}{weten te regelen tot}{aller voldoening}\\

\haiku{Andr\'e Ruiters zal '.}{t zoolang als pachter voor}{z'n neef beheeren}\\

\haiku{Maar ik, \'o\'ok niet van,:}{gisteren zooals je weet zei}{zoo langs m'n neus weg}\\

\haiku{Want bij dat groote nieuws:}{van Garverskamp kon ze zich}{niet meer goedhouden}\\

\haiku{En hoe Moeder hier?}{opeens zonder iemand zou}{komen te zitten}\\

\haiku{Dadelijk na de,,{\textquoteright}.}{Vasten in alle stilte}{beslist Trinette}\\

\subsection{Uit: Wat was en werd. Verhalen uit Limburgs legende en historie}

\haiku{De gunst en ook de.}{manschappen der grooten zijn}{hem onontbeerlijk}\\

\haiku{roep 't maar van de,.}{daken af dat heel de stad}{alles hoort en weet}\\

\haiku{{\textquoteleft}Ja, ik versta 't - ' -....}{nu ik verstat hij is}{bijna weggeruimd}\\

\haiku{Karel, haar zoon, z'n....}{vaders opvolger worden}{in Austrasi\"e}\\

\haiku{{\textquoteleft}Laat ons dat alles,.}{eens flink maar rustig onder}{de oogen zien vrouwe}\\

\haiku{Lambertus begrijpt,...}{dat en heft de hand om haar}{te zegenen}\\

\haiku{Eerst op den drempel,:}{der voorhal spreekt Hubertus}{die hem uitgeleidt}\\

\haiku{{\textquoteleft}Aangewezen zijn,.}{we den een aan den ander}{dat is duidelijk}\\

\haiku{wat ik redden en}{richten kan voor Christus en}{Zijn rijk met dit mij}\\

\haiku{{\textquoteleft}Wiens naam gezegend{\textquoteright},.}{zij met dien van Lambertus}{bidt Karel Martel}\\

\haiku{Gaat het niet rechtstreeks,.}{dan maar langs een omweg zijn}{vertrouwen winnen}\\

\haiku{ik werd veertien en,.}{zestien en had aldoor nog}{aan mijn droom genoeg}\\

\haiku{Dat is de Vita{\textquoteright},,.}{Sancti Servatii weifelt}{hij niet begrijpend}\\

\haiku{{\textquoteright} {\textquoteleft}Dan hoef je nooit je,.}{zelf te verwijten het niet}{beproefd te hebben}\\

\haiku{De sleutelbos aan.}{zijn leeren gordel klinkelt bij}{iederen voetstap}\\

\haiku{, komen we je niet?}{al te onverwacht uit je}{studie opjagen}\\

\haiku{Hij nestelde zich.}{onder haar mantel en liet}{haar arm niet meer los}\\

\haiku{{\textquoteright} Richardis ziet den:}{ruigen knoestigen man naast}{haar nadenkend aan}\\

\haiku{{\textquoteright} {\textquoteleft}Waarom een weerstand,,?}{die ons het afscheid pijnlijk}{laat worden Gerhard}\\

\haiku{{\textquoteright} {\textquoteleft}Lofwaardig is de{\textquoteright},.}{goede meening antwoordt de}{monnik bedachtzaam}\\

\haiku{{\textquoteright} {\textquoteleft}Gerhard, hoe zou een?}{moeder de liefde van haar}{kind kunnen weerstaan}\\

\haiku{{\textquoteright} Chris Vaesen is nooit.}{verlegen geweest en lacht}{mee met de twee}\\

\haiku{En als Chris trotsch knikt,.}{wisselen de twee een blik}{van verstandhouding}\\

\haiku{{\textquoteright} {\textquoteleft}Wel, ik heb m'n woord,....}{gegeven aan den Prins wiens}{vlag ik gevolgd ben}\\

\haiku{De Prins van Oranje,.}{is daar binnen en wil je}{zelf zien en spreken}\\

\haiku{Totdat de weergalm,.}{opeens wegzinkt omdat de}{ruimte zich verwijdt}\\

\haiku{Hij kan alleen nog,.}{maar bang wachten dat de boot}{ergens zal landen}\\

\haiku{Door de schuld van hem,,, -!}{dwaashoofd luchthart opsnijer}{door zijn schuld alleen}\\

\haiku{God vergeve hem,.}{wat hij zich zelf nooit meer zal}{kunnen vergeven}\\

\section{R.A. Kollewijn}

\subsection{Uit: Verweghe en zijn vrouw (onder ps. C.P. Brandt van Doorne)}

\haiku{Kompromitteerde,!}{hij er z'n schoonzuster mee}{dat r\'a\'akte hem niet}\\

\haiku{'t Was de vijfde '.}{maal dat hijt verhaal nu}{deed op die middag}\\

\haiku{Want hij was nu zich,,.}{zelf niet meer hij was Brikhof}{de postdirekteur}\\

\haiku{Merkwaardig dat ze.}{zo fris en bevallig bleef}{onder dat werken}\\

\haiku{Kwam het d\'a\'arvandaan,,,?}{dat hij kalm was niet opsprong}{in drift of wanhoop}\\

\haiku{hij begon iets te.}{voelen van schaamte over z'n}{argwaan en bitsheid}\\

\haiku{Je denkt nu zo,  ,.}{maar als ik het deed zou je}{er spijt van hebben}\\

\haiku{Hij reikte haar koel,.}{en stijf de hand keerde zich}{om en ging heen}\\

\haiku{Er was daarbij geen.}{sprake van opzettelik}{anders zich voordoen}\\

\haiku{Toen ze boven de,.}{twintig kwam dacht ze niet dat}{ze ooit zou trouwen}\\

\haiku{Ze kwam nu en dan,.}{bij Verweghe aan huis als}{vriendin van Marie}\\

\haiku{Natuurlik was het,.}{niet alles z\'o als zij het}{zou hebben gewenst}\\

\haiku{maar erger was 't,.}{dat nu leugen lag in haar}{woorden en blikken}\\

\haiku{Was hij gekomen?}{om te vertellen dat hij}{haar raad had gevolgd}\\

\haiku{Hij onderzocht haar - -.}{z\'e\'er tegen haar zin en vond}{niets verontrustends}\\

\haiku{Maar d\`an was de zaak -....}{ook uit gesteld dat-ie}{ooit had bestaan}\\

\haiku{En hij voelde aan, -!}{z'n bezorgdheid z'n angst die}{toch \`ongegrond was}\\

\haiku{opgetrokken haar, '....}{voeten alst water haar}{stoel bereikte}\\

\haiku{Niet zoals laatst, in,.}{een half wanhopige bui}{maar na kalm overleg}\\

\haiku{Het scheen dat men voor.}{z'n schuchtere avances niet}{onverschillig was}\\

\haiku{En eindelik het.}{wanhopig besluit om hem}{alles te zeggen}\\

\section{Gerrit Komrij}

\subsection{Uit: Heremijntijd. Exercities en ketelmuziek}

\haiku{En altijd laten,.}{ze me in de steek wanneer}{ik een pen vasthoud}\\

\haiku{Maar zodra ik 'n,.}{pen in mijn hand houd ben ik}{niet meer te stuiten}\\

\haiku{Ze heeft meer oog voor.}{functionaliteit dan}{voor menselijkheid}\\

\haiku{Er is behoefte.}{aan ontwerpers die de}{klerken trotseren}\\

\haiku{Zinloos gingen ze.}{verder waar de anderen}{gebleven waren}\\

\haiku{In werkelijkheid,.}{is er maar \'e\'en vrouw en dat}{is al erg genoeg}\\

\haiku{z\'o verguld mee dat {\textquoteleft}{\textquoteright}, ' {\textquoteleft}{\textquoteright}.}{hij opsex niets tegen heeft}{dat hijtspeels noemt}\\

\haiku{Ik heb het gevoel,.}{dat ik iets goddelijks ja}{God in hun ogen zie}\\

\haiku{De tweede is een:}{literatuur waarmee ik}{nogal verguld ben}\\

\haiku{Over de vetplanten?}{die de onderste helft in}{beslag nemen heen}\\

\haiku{Snufferds van oudjes '.}{kijken om kwart voor achts}{avonds televisie}\\

\haiku{Zij bloeit op waar zij.}{wordt vertroeteld door zielen}{waar de rek in zit}\\

\haiku{Je kan immers \'o\'ok {\textquoteleft}{\textquoteright}.}{niets vertellen door onzin}{aaneen te breien}\\

\haiku{Elke dichter houdt.}{zijn eigen po\"ezie voor}{de enige ware}\\

\haiku{honderd om hem te,.}{begaan en honderd om er}{achter te komen}\\

\haiku{{\textquoteleft}Neem een emmer en.}{zet die tegen de muur of}{een andere wand}\\

\haiku{Hij hield ditmaal geen,.}{papiermand of emmer vast}{maar een tabouret}\\

\haiku{Ik weet niet hoe het,.}{zo kwam maar er wilde geen}{evenwicht meer komen}\\

\haiku{Maar mijn liefde voor.}{solitaire spelletjes}{was nog niet geblust}\\

\haiku{Vaag staat me nog bij.}{dat het  beest er niet echt}{florissant uitzag}\\

\haiku{Nu gebeurt er iets,.}{heel tragisch als we er goed}{over nadenken}\\

\haiku{Ik heb ze meteen ' '.}{bijt begin de stuipen}{opt lijf gejaagd}\\

\section{Dirk Ayelt Kooiman}

\subsection{Uit: De grote stilte}

\haiku{Ik ben scheefgegroeid - -.}{dat blijkt maar weet niet eens waar}{het begonnen is}\\

\haiku{een arrogante -.}{klootzak vond wat ook niet te}{verwaarlozen was}\\

\haiku{Zelfs h\`em viel het op}{dat men vergeleken met}{wat hij gewend was}\\

\haiku{Hij vroeg zich af of.}{het initiatief van haar}{kant gekomen was}\\

\haiku{En wie zwijgt stemt niet,.}{alleen toe hij wordt ook wel}{de wijste genoemd}\\

\haiku{Waarop ze gevraagd.}{had of hij nou ook met al}{die vriendinnen sliep}\\

\haiku{{\textquoteleft}Weet je wat die naam...?}{betekent in het land waar}{ik net vandaan kom}\\

\haiku{Er waren dingen.}{die je bij nader inzien}{namelijk niet zei}\\

\haiku{(Pauze.) Haar koffer.}{kregen we een paar weken}{later opgestuurd}\\

\haiku{Er zaten nog wat,,.}{kleren in beschimmeld de}{rest was gestolen}\\

\haiku{- Het was alsof er.}{een deur werd opengegooid en}{weer dichtgeslagen}\\

\haiku{Hij boog zich naar haar,.}{over voelde zich ijzig kalm}{en vastberaden}\\

\haiku{{\textquoteright} Ze aarzelde, en.}{even zag het er naar uit dat}{ze zou gaan gillen}\\

\haiku{[7] Op de dag af.}{een week later gebeurde}{er het volgende}\\

\haiku{Ze vertrouwde me.}{toe dat ze op het punt stond}{jarig te worden}\\

\haiku{Ze omhelsde me,.}{op mijn beurt waarbij ze op}{haar tenen moest staan}\\

\subsection{Uit: Een romance}

\haiku{{\textquoteleft}Wat was er gebeurd,?}{en hoe gebeurde het wat}{zou de uitkomst zijn}\\

\haiku{Geen zuchtje wind dreef.}{door de wijdgeopende}{vensters naar binnen}\\

\haiku{Want die vogel, die,,...}{neerschoot op de aarde af}{naar de put immers}\\

\haiku{ze draaide me om,.}{mijn as zodat ik in haar}{lengterichting hing}\\

\haiku{Daar kon u best eens,,.}{gelijk in hebben meneer}{antwoordde ik vroom}\\

\haiku{Ik vegeteerde -.}{en die luxe kon ik me nog}{permitteren ook}\\

\haiku{Wat was er gebeurd,?}{en hoe gebeurde het wat}{zou de uitkomst zijn}\\

\haiku{- Zo, daar is ie dan! -,.}{Aardig van je om me op}{te halen Bomdal}\\

\haiku{Ik hoorde hem al:}{fluisterend achter zijn hand}{verduidelijken}\\

\haiku{Ik beklopte het:}{koetswerk als betrof het de}{flank van een renpaard}\\

\haiku{Een giechelige,!}{dithyrambe van glas naar}{urinoir wat een tijd}\\

\haiku{logeerpartijtjes,.}{in de vakantie een jaar}{of tien gelede}\\

\haiku{In zo'n koekblik, u,{\textquoteright}.}{weet wel zo'n afgeroste}{militaire kist}\\

\haiku{Het glas verhief zich,,.}{ademnood de tranen liepen}{me over de wangen}\\

\haiku{Mijn bewustzijn {\textquoteleft}toont{\textquoteright} -!}{me namelijk mijzelf want}{dat is wat het doet}\\

\haiku{ik heb ze in mijn,.}{macht ze zijn volledig aan}{mij onderworpen}\\

\haiku{Alles goed en wel,,?}{helemaal mee eens maar wat}{moet ik daar n\'u mee}\\

\haiku{Dat schudden om te...?}{zetten in een knikken het}{te onderbreken}\\

\haiku{Hij veerde op uit -.}{zijn stoel het leek wel of de}{zitting in brand stond}\\

\haiku{Door een jubelend,:}{koortje begeleid had ik}{hem de hand gereikt}\\

\haiku{Ik keek de mensen,.}{langs die zich tot een massa}{hadden verenigd}\\

\haiku{En dat het zo snel,.}{gebeuren zou kon ik nog}{minder vermoeden}\\

\haiku{Een enorme spiegel,.}{was het zoals men die ziet}{in modehuizen}\\

\haiku{We dansten echter,,.}{al heel opeens zonder dat}{ik er bij nadacht}\\

\haiku{op en neer, heen en,.}{weer synchroon als de stokken}{van een strijkorkest}\\

\haiku{Dat ze een blunder.}{hadden begaan door mij zo}{te onderschatten}\\

\haiku{Ik liep naar een van,.}{de auto's toe betastte}{de benzinedop}\\

\haiku{Ze schijnt hem af te,.}{wijzen te oordelen naar}{de intonatie}\\

\haiku{Ik sluit de deur van.}{mijn kamer en wacht tot hij}{is teruggekeerd}\\

\haiku{Bomdal amuseert zich,.}{wel onnodig me over hem}{zorgen te maken}\\

\haiku{Bovendien, hij is.}{het die me zijn vaarwater}{heeft binnengeloodst}\\

\haiku{Aan weerzijden van,.}{het pad een web van dunne}{naakte worteltjes}\\

\haiku{De schedel van een,.}{konijn gesteriliseerd}{door zon en regen}\\

\haiku{Nu kan ik, dat wil,.}{ik best bekennen niet zo}{erg best  vangen}\\

\haiku{Ik richt mijn blik op,.}{een witte beweeglijke}{stip in de verte}\\

\haiku{Vaag drong het tot me.}{door dat hij af en toe in}{zichzelf mompelde}\\

\haiku{Een angstaanjagend,.}{verschijnsel temeer omdat}{het ook de stem gold}\\

\haiku{Ongeloofwaardig:,,!}{wij op een zandverstuiving}{op een heideveld}\\

\haiku{De bossen in het.}{verschiet verdwenen achter}{een loodkleurig waas}\\

\haiku{Koppig bleef hij staan.}{en vroeg met dikke tong wat}{de bedoeling was}\\

\haiku{De bliksemslagen.}{worden luider en volgen}{elkaar sneller op}\\

\haiku{Merkwaardig... - Maar is?}{het niet gevaarlijk om in}{zo'n schuur te zitten}\\

\haiku{daar ligt - met Martha -.}{en dat hij best mag zien dat}{zij hier ligt met mij}\\

\haiku{Ik sta erbuiten,!}{ik wil er verder niets mee}{te maken hebben}\\

\haiku{- Ja, wie zou er nu...!}{een wesp in jouw glas werpen}{Zijzelf natuurlijk}\\

\haiku{Ik hoop dat ik je.}{niet deprimeer op deze}{feestelijke avond}\\

\haiku{herhaalde Bomdal,.}{nerveus trommelend op de}{leuning van zijn stoel}\\

\haiku{Had hij me misschien?}{iets gevraagd en moest ik hem}{nu antwoord geven}\\

\haiku{Haar andere arm.}{had ze om de schouder van}{Martha geslagen}\\

\haiku{- Zeg Martha, zou je?}{misschien een kop koffie voor}{me kunnen maken}\\

\haiku{Beter  te koud,.}{dan te warm daar verweken}{de hersens maar van}\\

\haiku{Joyce streek met een.}{bevochtigde wijsvinger}{langs haar wenkbrauwen}\\

\haiku{Of was het een vonk?}{van verontwaardiging die}{in hem opgloeide}\\

\haiku{Plotseling drong de.}{betekenis ervan ten}{volle tot me door}\\

\haiku{We schuifelen in,.}{de richting van de auto}{alleen en samen}\\

\haiku{En daarbij was het.).}{gebleven Maar nu zat zij}{naar mij toegewend}\\

\haiku{En wij zouden op.}{afstand van elkaar staan en}{elkaar opnemen}\\

\haiku{Wellicht zouden haar... -?}{borsten me in verwarring}{brengen Maar zijzelf}\\

\haiku{Om niets te doen, niets,.}{te laten ontstaan niets te}{laten gebeuren}\\

\haiku{Maar, drong tot me door,.}{dat zou niet van het minste}{belang geweest zijn}\\

\haiku{Een zwaar wolkendek,,,.}{struiken rondom een hek}{daar stond de auto}\\

\haiku{Van haar gezicht kon.}{ik me later nooit meer een}{voorstelling maken}\\

\haiku{En in het bordeel:}{met de duizend trappen en}{de duizend deuren}\\

\subsection{Uit: De vertellingen van een verloren dag}

\haiku{Uit weinig m\'e\'er valt.}{op te maken hoezeer hij}{zich voelt aangedaan}\\

\haiku{Het doffe gevoel:}{van meewarigheid wanneer}{zijn moeder uitroept}\\

\haiku{Omdat ze het zelf.}{niet durven te doen gaan ze}{naar hun buurtcaf\'e}\\

\haiku{dat blijkbaar ook zij -.}{bestaat en bovendien in}{mijn gezelschap is}\\

\haiku{Dat is gratie, maar.}{niet een gratie omwille}{van het behagen}\\

\haiku{Doen alsof het ook.}{voor mij de gewoonste zaak}{van de wereld is}\\

\haiku{Een blik achter zich,.}{alsof daar geheugensteun}{te verwachten is}\\

\haiku{- Het gaat hem, wanneer,.}{we hem geloven moeten}{zelden voor de wind}\\

\haiku{hij duidt aan met zijn),,:}{hand eminentie stond ik op}{een schip en ik riep}\\

\haiku{{\textquoteleft}En de terugweg,.}{is korter dan de heenweg}{valt me altijd op}\\

\haiku{Er klinkt verbazing,.}{in hun stem verbazing over}{zoveel avonturen}\\

\haiku{Maar dat was wel een.}{beetje hinderlijk wanneer}{er al iemand was}\\

\haiku{Wie zegt me dat iets?}{alsvolgt zal gaan omdat het}{altijd al zo ging}\\

\haiku{Hij knikt gevleid, maar {\textquoteleft}{\textquoteright}.}{houdt het voorlopig toch maar}{liever opsignor}\\

\haiku{Hij was bijvoorbeeld.}{dodelijk nieuwsgierig naar}{mijn toekomstplannen}\\

\haiku{Het patent komt op,}{jouw naam te staan dat zal ik}{meteen regelen}\\

\haiku{Kompaan staat kreunend,.}{op en rekt zich uit zijn haar}{glanzend van de dauw}\\

\haiku{En het moet gezegd,.}{dat karweitje is in een}{ommezien geklaard}\\

\haiku{Hij pakt de cognacfles,:}{giet het laatste bodempje}{naar binnen en zegt}\\

\haiku{Waar de dimensie?}{die aan het verslag ontbreekt}{vandaan getoverd}\\

\haiku{Wat te beginnen:}{met het gevoel dat hij op}{dit moment ervaart}\\

\haiku{Zeker, zo was het,.}{onmiskenbaar er is een}{gesloten verband}\\

\haiku{Ik trek haar zonder,:}{uitstel aan de arm en vraag}{iets in de geest van}\\

\haiku{Nu! - En dat dan in.}{tegenstelling tot wat er}{volgde op die avond}\\

\haiku{{\textquoteright} heeft hij gevraagd, een,.}{beetje afwezig zonder}{haar aan te kijken}\\

\haiku{Nooit, nooit zou ik haar,.}{weer terugzien het leven}{had geen waarde meer}\\

\haiku{Het zonlicht op mijn.}{haren is van een zon die}{nooit meer schijnen zal}\\

\haiku{Het rosarium,.}{alleen nog herkenbaar aan}{de tegelpaadjes}\\

\haiku{En mijn hand op zoek.}{naar welvingen die ze nog}{niet te bieden had}\\

\haiku{{\textquoteleft}Wanneer we onze}{bagage nou eens later}{op gingen halen}\\

\haiku{Een gammel houten.}{bruggetje voert ons over een}{snelstromende beek}\\

\haiku{ieder woord, ieder.}{gebaar geeft uitdrukking aan}{die situatie}\\

\haiku{{\textquoteright} - Die vraag moet me wel.}{op de lippen bestorven}{hebben gelegen}\\

\haiku{Dat ging door tot er:}{een nooit  te voorspellen}{eindstreep werd bereikt}\\

\haiku{wanneer het nu zou.}{regenen zou de damp van}{het plaveisel slaan}\\

\haiku{die scherpgepunte,,.}{glasscherven zo dichtbij zo}{binnen handbereik}\\

\haiku{Hij is begonnen,.}{bij de A en inderdaad}{in slaap gevallen}\\

\haiku{Aan het tafeltje.}{tegenover het zijne zit}{een verliefd paartje}\\

\haiku{Zweet op zijn voorhoofd -.}{zelfs om het af te vegen}{ontbreekt hem de kracht}\\

\haiku{{\textquoteleft}Neem me niet kwalijk,.}{dat ik u stoor maar ik zit}{met een probleempje}\\

\haiku{Informeert hoe laat.}{de ochtendpostbestelling}{gewoonlijk plaatsvindt}\\

\haiku{Naar welke van de,?}{twee zou uw voorkeur uitgaan}{als ik vragen mag}\\

\haiku{Ze passeerden het,.}{Casino de schijnwerpers}{diffuus door de sneeuw}\\

\haiku{Voetje voor voetje volgde}{Merkuur door een dikke laag}{krakende  sneeuw}\\

\haiku{Jazeker, een man,,.}{en twee meisjes de ene rood}{de andere blond}\\

\haiku{k Gooi je er hier,.}{uit want verderop is het}{eenrichtingsverkeer}\\

\haiku{Wanneer de werkster.}{geweest is kan ik een dag}{lang niets meer vinden}\\

\haiku{Zeno's kooi wordt eens.}{in de week verschoond en van}{vers zand voorzien}\\

\haiku{{\textquoteright} {\textquoteleft}Door ze geregeld.}{te laten reinigen en}{ze te borstelen}\\

\section{Anton Koolhaas}

\subsection{Uit: De laatste goendroen}

\haiku{De eerste keer dat,;}{ze dit probeerde werd het}{Bladroes wel wat te bar}\\

\haiku{Ze maakte een zacht.}{jankend geluid en deed dat}{toen nog een paar keer}\\

\haiku{En voor zover dat:}{voor andersdenkenden niet}{zo'n probleem zou zijn}\\

\haiku{Hij zou alleen zijn;}{eigen weg door het leven}{gevonden hebben}\\

\haiku{Jonas antwoordde.}{dat hij dat maar al te graag}{voor hem wilde doen}\\

\haiku{dat veel bruin bevat.}{buitengewoon moeilijk zijn}{te ontdekken}\\

\haiku{Dat moet U zeker{\textquoteright},, {\textquoteleft}}{doen riep de heer R\"utl\"u van}{het departement}\\

\haiku{{\textquoteleft}Inderdaad dit kon{\textquoteright}.}{wel eens een vondst van enorme}{betekenis zijn}\\

\haiku{Misschien deze ene,}{nacht nadat hij vandaag zou}{hebben besloten}\\

\haiku{Het was prof Ruddell}{die de impasse doorbrak}{door voor te stellen}\\

\haiku{Zojuist had je het.}{over die ene gehoorzenuw}{van  de sprinkhaan}\\

\haiku{Nou, ik peins er niet{\textquoteright},.}{over riep Strutt en hij trok zijn}{handschoenen weer uit}\\

\haiku{ik u met mijn hand -.}{op het hart dan dat het hier}{zo'n saai rotgat is}\\

\haiku{Even voortvarend als,.}{Ruddell binnengekomen}{was verdween hij weer}\\

\haiku{alleen kalmte is{\textquoteright} {\textquoteleft}{\textquoteright}.}{hier vereist en zei nog een}{keerMijne heren}\\

\haiku{{\textquoteleft}Ja ik weet niet of{\textquoteright},, {\textquoteleft}}{het iets van waarde is zei}{Jonas verlegen}\\

\haiku{{\textquoteleft}Wel, wel{\textquoteright}, riep Ruddell, {\textquoteleft},{\textquoteright}.}{en dr Laeet zeiDront je bent}{onverbeterlijk}\\

\haiku{{\textquoteleft}Maar ze hebben het,{\textquoteright},.}{niet opgeschreven daar zegt}{Fenja riep Jonas}\\

\haiku{{\textquoteleft}Wat onze kleine{\textquoteright}.}{geleerde hier gesteld heeft}{vraagt om dat beraad}\\

\haiku{Professor Ruddell.}{wist eigenlijk ook niet meer}{wat hem te doen stond}\\

\haiku{Eigenlijk konden.}{er niet zo goed drie mensen}{in het kamertje}\\

\haiku{Maar ze deden of}{het iets heel gewoons was en}{dat verbaasde me.}\\

\haiku{{\textquoteleft}En ik geloofde,{\textquoteright},.}{het meteen wat ze zeiden}{mompelde R\"utl\"u}\\

\haiku{Maar aangezien hij.}{van haar weggelopen was}{deerde hem dat niet}\\

\haiku{Het dient vastgesteld {\textquoteleft}{\textquoteright}.}{dat het woordbedreigen hem}{eigenlijk niets zei}\\

\haiku{Een enkele keer,.}{als er niet gevoetbald werd}{vergezeld van Jomp}\\

\haiku{Misschien kan ik daar{\textquoteright}.}{dan iemand anders nog eens}{een plezier mee doen}\\

\haiku{Er was een gesprek.}{gaande tussen zijn vader}{en zijn grootvader}\\

\haiku{Daarna verklaarde,.}{hij dat hij helemaal niets}{rook laat staan voorjaar}\\

\haiku{Maar waar had hij de?}{vleugels nog die zorgden voor}{de goede afloop}\\

\haiku{Dat je zijn blik nu.}{uitgesproken treurig kon}{noemen is niet zo}\\

\haiku{Maar toch wel zo hoog.}{dat de beek te breed was om}{overheen te springen}\\

\haiku{Toen dacht hij echter:}{aan wat hem vlak daarvoor door}{het hoofd was gegaan}\\

\haiku{Bij elkaar is dit.}{wel wat je aftakeling}{zou kunnen noemen}\\

\haiku{De avond daarna was.}{hij te moe om een derde}{bed weg te graven}\\

\haiku{De zwarte grond was,.}{onder opzij en boven}{Bladroes en achter hem}\\

\haiku{Bladroes moet veranderd,,.}{zijn na voorbereiding en}{inkeer in een plant}\\

\haiku{{\textquoteright}, fluisterde Hiske, {\textquoteleft}.}{en ineens stak ik toen mijn}{handen in die plant}\\

\haiku{En daarom schrok ik.}{zo ontzettend en holde}{ik naar julie toe}\\

\subsection{Uit: Vanwege een tere huid}

\haiku{Anton Koolhaas}{Vanwege een tere huid}{Colofon}\\

\haiku{bang worden terwijl.}{er niets te zien is waar je}{bang voor hoeft te zijn}\\

\haiku{Wel goed dat ik je{\textquoteright},.}{nu eindelijk eens buiten}{tref zei Jokke zacht}\\

\haiku{Een zinnetje om.}{op te krijgen om in het}{Duits te vertalen}\\

\haiku{Jokke knikte - {\textquoteleft}Maar.}{stap dan meteen op je fiets}{als ik er aankom}\\

\haiku{Het is de groei{\textquoteright}, zei, {\textquoteleft}{\textquoteright}.}{zijn moederdaar tobben al}{die kinderen mee}\\

\haiku{Hij was verliefd op,.}{Takkie mateloos verliefd}{er in verdronken}\\

\haiku{dacht hij vaak en hij.}{huiverde dan en wou dat}{hij er van af was}\\

\haiku{Daar zal Hij het niet,.}{druk mee hebben want er zijn}{er nog maar een paar}\\

\haiku{{\textquoteright} Jokke was daar niet.}{vatbaar voor en voelde er}{zich door beledigd}\\

\haiku{Jezus meid, met je,,.}{kouwe drukte dacht Jokke}{net of ik dat weet}\\

\haiku{Gisteren hadden.}{de sproeten op haar armen}{hem toch gehinderd}\\

\haiku{Hij maakte zich dan.}{wel een voorstelling van hoe}{dat dan zou toegaan}\\

\haiku{de Koning had het,.}{in de gaten dat Takkie}{eindelijk beet had}\\

\haiku{De kano's waren;}{te huur bij een zwembad aan}{de rand van de stad}\\

\haiku{{\textquoteleft}Meervallen, je weet;}{wel die vissen die zo zwart}{zijn met die snorren}\\

\haiku{Ze weten trouwens.}{evenmin dat ze zelf practisch}{zijn uitgestorven}\\

\haiku{De ooms en tantes.}{hadden lijven gehad in}{zwarte omhulsels}\\

\haiku{Op een gegeven.}{ogenblik valt het je op dat}{je geen pijn meer hebt}\\

\haiku{zo kind, zo zijn we,.}{toch zekerlijk allemaal}{gelukkig vandaag}\\

\haiku{{\textquoteleft}Nee, nou moet je het{\textquoteright},.}{niet nog mooier maken dan}{het al was riep Cor}\\

\haiku{Ik zal er snel in ().}{voorzienhij plakt de zegels}{er op stond er dan}\\

\haiku{Hier zijn ze opnieuw,{\textquoteright}.}{Adriaan en ik bedank u}{voor uw waakzaamheid}\\

\haiku{Cor en Annepiet,.}{trouwens ook maar die kregen}{wat ze verdienden}\\

\haiku{Uit verdriet{\textquoteright}, jammert,:}{ze heel zacht maar ze zou het}{uit willen gillen}\\

\haiku{Natuurlijk denkt ze.}{aan de volgende dag en}{wat ze zullen doen}\\

\haiku{En na enige niet,.}{te overleven momenten}{waren ze er af}\\

\haiku{Daarna zullen ze.}{het gesprek voortzetten over}{dokter de Koning}\\

\haiku{Zijn ogen vormen een.}{neutrale materie in}{zijn kleurloze kop}\\

\haiku{Takkie was al heel;}{wat keren in en weer uit}{de wherry gestapt}\\

\haiku{Zo lang gewacht, tot.}{ze er eenvoudig te moe}{door geworden was}\\

\haiku{Mogelijk hebben;}{ze een ander niet gebruikt}{terrein gevonden}\\

\haiku{Van jongetjes en.}{meisjes vragen we niet waar}{ze ge bleven zijn}\\

\section{Pieter Korthuys}

\subsection{Uit: Menschen in malaise}

\haiku{s Avonds na het eten,.}{toen wij samen waren kwam}{het hooge woord eruit}\\

\haiku{Eerst zat Va nog wat,.}{in de krant te lezen maar}{toen kwam het eruit}\\

\haiku{Nog een week of twee,.}{kon hij blijven maar dan was}{het afgeloopen}\\

\haiku{{\textquoteright} {\textquoteleft}Nou, ik geloof, dat.}{ik dan nog liever in een}{achterbuurt hier zit}\\

\haiku{{\textquoteleft}Wij kunnen morgen,,,.}{eens kijken h\'e Flosje of}{wij iets goeds vinden}\\

\haiku{Overmorgen is het,.}{al de 28ste dan hebben wij}{nog maar twee dagen}\\

\haiku{{\textquoteright} {\textquoteleft}Lottie nemen wij,}{mee die kan bij mij in het}{mandje achterop}\\

\haiku{het was deze week,.}{een late boot een van de}{oudere typen}\\

\haiku{Maar breek er je nek,!}{niet als je morgen naar je}{bezit komt kijken}\\

\haiku{Een doeltje, dat wij,.}{samen nastreven om jouw}{ouders te helpen}\\

\haiku{Van tweehoog hoort hij.}{de sliffers van pantoffels}{omlaag klepperen}\\

\haiku{Z\'o\'o gemakkelijk.}{als zij het voorstelde zou}{het zeker niet zijn}\\

\haiku{Maar ze zijn dan toch,}{motte verhuize om ons}{te kenne helpe}\\

\haiku{Hij wil later ook,.}{naar Indi\"e zien te komen}{als zijn oudste broer}\\

\haiku{{\textquoteright} Een gelukkige.}{blik van verstandhouding is}{haar eenige antwoord}\\

\haiku{Maar van Nel weet ze,.}{dat je tegenwoordig z\'o\'o}{billijk kunt koopen}\\

\haiku{Flos eet dan ook vijf.}{dagen van de week alleen}{met de kinderen}\\

\haiku{Die vent denkt zeker,.}{in haar een goede klant te}{hebben gevonden}\\

\haiku{Hij licht zijn slappe,:}{vilthoed en vraagt haar of ze}{niets kan gebruiken}\\

\haiku{Zijn haren zijn nat,.}{hij was zich klaarblijkelijk}{aan het mandi\"en}\\

\haiku{{\textquoteleft}En zelf heb ik in!}{geen maand een sigaar gerookt}{uit bezuiniging}\\

\haiku{{\textquoteleft}Het is acht uur, ik!}{word nog wel geholpen als}{ik dadelijk ga}\\

\haiku{{\textquoteleft}'t Is Meijers,{\textquoteright} zegt,:}{Frans en ter verklaring voor}{Rudolf en Hanny}\\

\haiku{Frans zoekt een geschikt.}{woord om zijn afgebroken}{zin te hervatten}\\

\haiku{En ook wil hij zijn.}{vrouw helpen met de kamer}{aan kant te brengen}\\

\haiku{De blauwe rook trekt.}{in breede banen het half}{openstaande raam door}\\

\haiku{Ze ziet het weer voor,.}{zich die blanke pracht van de}{bloeiende bongerds}\\

\haiku{Eigenlijk heeft zij,'.}{er nu reeds wroeging over als}{ze Frans gezicht ziet}\\

\haiku{En Meijers is ook,!}{geen vent om zich nu eens voor}{je in te spannen}\\

\haiku{Dan durven ze zelfs.}{geen honderd gulden in de}{maand te weigeren}\\

\haiku{{\textquoteright} vraagt hij verwijtend, {\textquoteleft},!}{anders kwam je er altijd}{mee bij me vroeger}\\

\haiku{Hij is verdwenen.}{in de ontelbaarheid van}{het inlandsche volk}\\

\haiku{Nou, in vredesnaam,{\textquoteright}.}{dan maar zoo'n suikerdrankje}{doet hij wanhopig}\\

\haiku{tusschen de dikke.}{wolkengroepen die de zon}{reeds verduisteren}\\

\haiku{Ze gevoelt zich hier,.}{als met Frans en ze hebben}{nog geen kinderen}\\

\haiku{Zwijgend zitten ze,.}{naast elkaar die drukkende}{rust overweldigt je}\\

\haiku{Het begint harder.}{te waaien en Flos trekt haar}{sjaal onder de keel}\\

\haiku{Zoo opgewekt en!}{moedig als nu heeft ze zich}{haast nog nooit gevoeld}\\

\haiku{En Lottie, achter,,.}{Moesje bedelt om er ook}{te mogen zitten}\\

\haiku{Frans kijkt elken avond,.}{de kranten nauwkeurig na}{maar er is niet veel}\\

\haiku{Enfin, het zou wel,.}{droog worden en anders de}{regenjas maar aan}\\

\haiku{Je kunt toch niet voor?}{ieder wissewasje naar}{den dokter loopen}\\

\haiku{{\textquoteright} Aan den anderen,}{kant komt wat spottend de vraag}{dat het dan blijkhaar}\\

\haiku{Op het platje is, ',.}{het om dezen tijd \'e\'en uur}{s middags te warm}\\

\haiku{Lottie heeft hem al,.}{ontdekt Dickie heeft aandacht}{voor iets op den grond}\\

\haiku{{\textquoteleft}Maar ik wilde nog.}{minstens een maand hier blijven}{uitzien naar een baan}\\

\haiku{Met een sprong zijn haar.}{gedachten dan weer bij haar}{eigen belangen}\\

\haiku{Een volgende brief,,.}{met de landmail logenstraft}{haar luchthartigheid}\\

\haiku{{\textquoteright} {\textquoteleft}Nooit, dat weet je wel,,!}{hoe blij ik ben die rijkdom}{is niet te meten}\\

\haiku{Jij bent in dien tijd,...}{voor mij de vrouw geworden}{die ik moest hebben}\\

\section{Alfred Kossmann}

\subsection{Uit: De moord op Arend Zwigt}

\haiku{Met wortel en tak.}{zou hij die vertedering}{uit moeten roeien}\\

\haiku{Ik kots van dat huis,,.}{van die schijnheiligheid van}{die smeerlapperij}\\

\haiku{{\textquoteright} vroeg ze verveeld maar, {\textquoteleft},?}{dringendeen glaasje likeur}{een glaasje klare}\\

\haiku{Ik heb honger{\textquoteright} zei, {\textquoteleft}.}{hijwe moeten eens iets te}{eten zien te krijgen}\\

\haiku{Hij begreep dat het.}{verdacht zou zijn om op dit}{voorstel in te gaan}\\

\haiku{{\textquoteright} Simon vroeg het op,.}{een onpersoonlijke maar}{dreigende manier}\\

\haiku{als het warm is moet,.}{je liever niet eten dat is}{goed voor de lijn ook}\\

\haiku{{\textquoteleft}Als de heren niet}{betalen kunnen er twee}{dingen gebeuren}\\

\haiku{En in die valse.}{atmosfeer begon Arend als}{een kind te huilen}\\

\haiku{{\textquoteright} zei Simon, {\textquoteleft}de deur,,.}{opendoen het geld stelen het}{is onherstelbaar}\\

\haiku{Hij wist niet waar hij.}{aan toe was en zocht naar een}{gevoel dat overheerste}\\

\haiku{Waarom had hij zo,?}{slecht gewerkt op school terwijl}{hij het beter kon}\\

\haiku{Bij toeval of bij;}{gratie van instinct kon het}{wonder gebeuren}\\

\haiku{Het wonder was er,.}{wel maar het leek nog niet het}{beslissende}\\

\haiku{Zij keerde zich om,.}{keek rond door de nu lege}{winkel en ging weg}\\

\haiku{{\textquoteleft}Het was mijn vader{\textquoteright}, {\textquoteleft},.}{zei hijik ben blij dat hij}{ons niet heeft gezien}\\

\haiku{Hij voelde hoe zijn,,.}{vlees waar zij het aanraakte}{begon te leven}\\

\haiku{Het lome onheil,,.}{van de atmosfeer smachtend}{naar wind deed hem niets}\\

\haiku{ik heb de wereld,.}{van binnenuit veroverd}{ik kan nu alles}\\

\haiku{Hij stond op en liep,.}{verder overwegend of hij}{maar zou gaan slapen}\\

\haiku{Hij snikte hardop.}{en wenste dat hij zich om}{zou durven draaien}\\

\haiku{Het liefst had hij zijn.}{hoofd in zijn handen gelegd}{en was doodgegaan}\\

\haiku{De tellen die hij,,.}{niet meer uitsprak tikten weg}{verprutst verloren}\\

\haiku{De wereld ligt in,.}{scherven en ik moet er een}{eenheid van maken}\\

\haiku{{\textquoteright} voegde hij tot zijn.}{schaamte nog bijna smekend}{aan zijn betoog toe}\\

\haiku{je denkt zeker dat,.}{ik gek ben je denkt zeker}{dat ik het niet doe}\\

\haiku{Zij keek hem kalm aan,.}{maar in haar vage antwoord}{klonk enige dreiging}\\

\section{Wim Kuipers}

\subsection{Uit: Platbook 4. Fitsprovins}

\haiku{Wie veer aankaome.}{biej de sjtart woort dat geveul}{allein mer sjterker}\\

\haiku{Fietse, wie vrouwluuj, '.}{mit tesse vol w\`ek poor en}{erpel aant stuur}\\

\haiku{Vaals, dan is 't nie '.}{mie\"er asn tepelke}{op de werreldbol}\\

\haiku{op \'os sportfietse.}{waor we de spatborde}{vanaf han gesloop}\\

\haiku{de wolke... alles......}{zwart de waeg smaal gehoebeld}{miene fiets rammelt}\\

\haiku{'t Waas al laat in '.}{t sezoen en de blajer}{vele van de buim}\\

\haiku{Inne va h\"on hauw '.}{de mie\"etste kans umt}{renne tse winne}\\

\haiku{Wool d'r Piet ing kans,.}{maache moe\"et he\"e inne fiets}{mit versjnellinge han}\\

\haiku{Sjouwer a sjouwer '.}{varete zet letste}{sjtuk noa d'r knietsjroam}\\

\haiku{Wie veer ouch achter,.}{h\"a\"om aan karde veer krege}{geine maeter good}\\

\haiku{Vreuger kwaam ich dich...,?}{nog al ins teenge mer duis}{te t'r niks mie\"e aan}\\

\haiku{Van doe aaf waor,.}{ich verkoch aan dees vorme}{van sjport m\`et aafzeen}\\

\haiku{'t Irriteert mich.}{dat bie de euverheid sjport}{is blieve ligke}\\

\section{Albert Kuyle}

\subsection{Uit: De bries}

\haiku{Men had haar hooren,.}{gillen aan het keldergat}{maar meer wist men niet}\\

\haiku{Fedor Fedorowitz.}{was de laatste naam die op}{de rol werd gezet}\\

\haiku{Fedor is bij de.}{acht die met een sloep den weg}{hakken naar het land}\\

\haiku{Hij voelt hoe elke.}{beweging van de bek in}{de spieren doortrilt}\\

\haiku{Als hij de oude.}{man goedendag gezegd heeft}{zal hij haar kussen}\\

\haiku{haar haren hangen.}{in een zware wrong op een}{nieuw geruit kleedje}\\

\haiku{{\textquoteleft}we gaan trouwen, wees,.}{niet koppig geef een brief mee}{aan de veekooper}\\

\haiku{En hoe meer hij er,.}{over nadacht hoe meer hij dacht}{dat zij gelijk had}\\

\haiku{Toen ze uitging was.}{het omdat de kamer te}{klein was voor haar droom}\\

\subsection{Uit: Jonas}

\haiku{Als een kleine pijn.}{begint achter in zijn hoofd}{een angst te groeien}\\

\haiku{Na ieder uur een,.}{top gewonnen na ieder}{uur een dal gedaald}\\

\haiku{Nooit meer water dan.}{de bron zich verzamelde}{in haar kleine kom}\\

\haiku{Dit is geen zwerver,.}{zooals er zoovelen aan de}{havenkaden zijn}\\

\haiku{Het is een rond en,.}{blauw luchtledig waarin het}{schip gewichtloos zweeft}\\

\haiku{De kapitein keert,.}{het gezicht naar boven als}{hij de beker draait}\\

\haiku{De zee\"en waarop{\textquoteright}.}{ik vluchtte en de aarde}{die ik ontvluchtte}\\

\haiku{Zijn barmhartigheid.}{is als de morgenwolk en}{als de morgendauw}\\

\haiku{Hij luistert naar de:}{muziek die hem beneden}{en ter weerzij is}\\

\haiku{Nog veertig dagen.}{en Niniveh zal omver}{geworpen worden}\\

\haiku{Die het water zijn,.}{schatten ontnamen het vuur}{dreef ze nu landwaarts}\\

\haiku{Of zou het waar zijn,?}{dat hij den smid bij diens vrouw}{had toegenomen}\\

\haiku{De roeden worden.}{geheven in de huizen}{van de buitenbuurt}\\

\haiku{Jonas kent hun reuk,:}{uit de duistere holten}{van het visschenlijf}\\

\haiku{Laat dan af van uw.}{boosheid en weest den Koning}{een gewillig volk}\\

\haiku{Zij staan scherp tegen.}{den witten weg die naar de}{tempeltrappen voert}\\

\haiku{In bloed smoort gij uw,.}{vruchtbaarheid in bloed zal de}{Heer u verstikken}\\

\haiku{wat baat brengt mij het,?}{l\'even waar ik de w\'o\'orden}{heb om te dienen}\\

\haiku{staat eerbiedig stil,.}{dat het woord van uw wijsheid}{hem verkwikken zal}\\

\haiku{allen is hij, en.}{Jahwe legde de wereld}{in zijne handen}\\

\haiku{Wie bewaarde het?}{graan van den overvloed voor de}{dagen vol honger}\\

\haiku{Wie sprak wijsheid door?}{zijnen mond en gaf acht op}{den vrede des rijks}\\

\haiku{Een oogwenk staat hij.}{onbeweeglijk temidden}{van zijn knielend volk}\\

\haiku{De Koning heeft de:}{ketens genomen van die}{geketend waren}\\

\haiku{Bitter in den mond,.}{smaakt Jahwe en als alsem}{zijn Zijne daden}\\

\haiku{Het einde wenkt, en.}{hij buigt zich van verlangen}{over naar dat einde}\\

\haiku{No\'e ziet om zich heen,.}{als was hij kinderen en}{aarde vergeten}\\

\haiku{Dan schrijdt hij uit, naar,.}{buiten waar de zon laag over}{de akkers wentelt}\\
