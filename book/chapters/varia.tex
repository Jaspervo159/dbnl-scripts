\chapter[4 auteurs, 1584 haiku's]{vier auteurs, vijftienhonderdvierentachtig haiku's}

\section{Ad Interim}

\subsection{Uit: Ad Interim. Jaargang 1}

\haiku{ik worde gedwee,.}{is water noch zand deze}{oever der zee}\\

\haiku{{\textquoteright} 4  Zo zingen, '.}{wij ten slot ent is een}{lofzang van de dag}\\

\haiku{De wellust van te.}{leven Houdt om uw oog haar}{vochten mist gespreid}\\

\haiku{, blijven we even staan,.}{verdronken in de glanzen}{van elkanders ogen}\\

\haiku{Mijn hart is er zo,.}{weinig bij Als kleedt zij zich}{niet uit voor mij}\\

\haiku{En wie zich aankleedt,.}{bij het raam Die kent haar niet}{meer bij de naam}\\

\haiku{Daar woekerde het,.}{laagste leven Of wat de}{schijn ervan bezit}\\

\haiku{Zij drong ontbindend,.}{in ons leven Als splijtzwam}{dreef z'ons van elkaar}\\

\haiku{Op de jongste  ,.}{Jij bent de jongste maar mijn}{stoutste ongerief}\\

\haiku{De laatste keer was}{hij van ver komen loopen}{om mij te vragen}\\

\haiku{Jij bent waarschijnlijk,.}{de eenige mensch die mij niet}{teleurgesteld heeft}\\

\haiku{Hij besefte niet,.}{dat hij voorgoed moest blijven}{of niet weer komen}\\

\haiku{Soms moet hij de les.}{duizendmaal leeren alvorens}{de overgave komt}\\

\haiku{Een weggeblazen.}{rozenblad Drijft haastig op}{het zwalpend nat}\\

\haiku{En ach, zo dicht bij.}{Charon's veer Wagen zich geen}{verliefden meer}\\

\haiku{Toen het bed, als een,.}{wielewaal Stond te zingen}{op zijn spiraal}\\

\haiku{Geen bekommernis}{om wat ik deed of wat}{ik heb gelaten}\\

\haiku{Hij lijkt, al is zijn,,.}{houding fier Niet op een musch}{maar op een dier}\\

\haiku{laag en drukken zwaar.}{en ratten tieren in de}{kelderkluizen}\\

\haiku{Onschuld en schoonheid,,.}{eerstelingen zijn wereld's}{vijanden sindsdien}\\

\haiku{De stad ging over in.}{eigendom van een met haar}{begonnen plan}\\

\haiku{Vandaag hebben wij.}{een goede greep gedaan in}{de appelbomen}\\

\haiku{de metaphoor ligt in;}{levende lijve op die}{en die breedtegraad}\\

\haiku{Zodat ik b.v. naar:}{een herbergier zou kunnen}{lopen en zeggen}\\

\haiku{Ik vlood de bergen '.}{af tot hier mijt water}{stuit van de rivier}\\

\subsection{Uit: Ad Interim. Jaargang 2}

\haiku{Soms ziet men er een.}{zijn lege bierglas naar de}{tapkast schuiven}\\

\haiku{De bosschen van het}{vaderland de bloemen in}{de lenteweiden}\\

\haiku{De zes of zeven.}{ouwe kereltjes lachen}{openlijk en honend}\\

\haiku{Wie een zoon heeft, zooals,.}{ik weet dat hij van zijn zoon}{toch het meeste houdt}\\

\haiku{Daarna was het weer.}{stil en ik keerde mij op}{mijn andere zij}\\

\haiku{Toen hij slap, oogen dicht,,.}{op den divan lag wist ik}{dat hij niet dood was}\\

\haiku{Hij lijkt op een koord,;}{waar de danser langs gaat in}{de trillende lucht}\\

\haiku{Strijdvaardig, wakker,,?}{gelijk de wind Ach waarom}{u te overijlen}\\

\haiku{Het fonklend water,.}{danste als zij Doch lichter}{was hun vroolijkheid}\\

\haiku{er roepen grauwe,......}{wulpen en berken staan er}{wit en ongeteld}\\

\haiku{Nu is de vrijheid'!}{uitgebroken een nieuwe}{lent die niemand stuit}\\

\haiku{Er is een liefde,,,.}{niet te blusschen voor man noch}{vrouw en zonder tijd}\\

\haiku{mijn hoofd zat voller;}{dan mij lief was met cijfers}{en verhoudingen}\\

\haiku{{\textquoteleft}Marie,{\textquoteright} zei hij - {\textquoteleft}had, -{\textquoteright}}{je niet moeten doen heb ik}{niet aan je verdiend}\\

\haiku{Wat er geweest is,?}{weet hij niet meer een hond die}{hem heeft doen schrikken}\\

\haiku{Twee dagen later,:}{zegt Frederik tegen hem}{bedaard en rustig}\\

\haiku{Van haar hoorde hij {\textquoteleft}{\textquoteright}.}{desmokkelarij met Slau's}{geboortedatum}\\

\haiku{{\textquoteright} Onder ons gesprek.}{kwamen Ter Braak en Adriaan}{van der Veen binnen}\\

\haiku{{\textquoteleft}Jullie denkt, dat het,,.}{me spijt weg te gaan maar ik}{ben blij dat ik ga}\\

\haiku{Hoe k\'on ik dat niet,?}{eerder weten niet beter}{zien in vroeger tijd}\\

\haiku{, een volksbewustzijn,,.}{dat alleen nog maar zwart-wit}{kent vrijheid of dood}\\

\haiku{Verbeter de wet,,.}{de maatschappij dan wordt ook}{de mensch wel beter}\\

\haiku{Het spel dat er met,.}{hem gespeeld wordt bevredigt}{hem echter maar half}\\

\haiku{hier kwam genas aan,}{U o Amsterdam en al}{wie van dit leven}\\

\subsection{Uit: Ad Interim. Jaargang 3}

\haiku{Het is natuurlijk,,.}{oliedom zal men zeggen zo}{te redeneren}\\

\haiku{Daar werd een zangstuk.}{van Wagner opgevoerd in}{de Franse versie}\\

\haiku{Anderen meenden.}{weer dat hij een verborgen}{liefde koesterde}\\

\haiku{soms als een ster die,.}{flonkert door wolkenscheur breekt ge}{wat u omhing}\\

\haiku{Hij neemt de fluit, en,.}{blaast en blaast Dat het een lust}{was om te hooren}\\

\haiku{Alleen als slager,,.}{onderlegd Heb ik begeerd}{fluitist te wezen}\\

\haiku{Neen, zij zaten er,.}{nog zij voelden stijf en hard}{tusschen de vingers}\\

\haiku{dat de Staten macht,;}{hadden en meer macht zouden}{krijgen binnenkort}\\

\haiku{Grootendeels waren;}{zij trouwens uit andere}{dorpen afkomstig}\\

\haiku{Maar ook hij kende,.}{geen angst en nog wel een uur}{bleef hij zoo zitten}\\

\haiku{Wij stonden huid aan,.}{huid met geschrokken beenen en}{vlijmende keelen}\\

\haiku{Wij eten om zes uur,{\textquoteright}.}{precies het is nu vijf voor}{zessen zei de vrouw}\\

\haiku{Nee, nee, nee, dat ging,,...}{niet dat ging onmogelijk}{ze zouden denken}\\

\haiku{Het was de eerste,.}{keer dat ze rechtstreeks het woord}{tot Loulou richtte}\\

\haiku{rondom mij bloeit het,.}{licht En stort in d'avond uit}{zijn zegeningen}\\

\haiku{Wie houdt van dorp of:}{stad op aarde heeft daarom}{meer lief dan hij ziet}\\

\haiku{wat zij zich wel niet.}{Kan droomen dan als roos van}{hartsverlangen}\\

\haiku{Mij is 't om het,,,.}{even Noemt gij het \'uw of mijn}{ons aller leven}\\

\haiku{er is een vlijmscherp.}{onderscheid tussen elk ding}{in het heelal}\\

\haiku{Wat het buiten zich, -}{zag als fantoom Was in hem}{als toekomst mij staat}\\

\haiku{Een kreet door den nacht,?}{desolaat Of was het een}{vlaag van den wind}\\

\haiku{Moede en ziek en,.}{oud Maar nog niet verslagen}{Wacht ik op uw woord}\\

\haiku{Als gij naadren zoudt,.}{Werd alsnog verhoord Wat ik}{niet kan vragen}\\

\haiku{Hij kon dat moppig.}{overdrijven van Cornelis}{niet erg waarderen}\\

\haiku{Hij knoert 'r oever{\textquoteright},,:}{schaterde oom T. nog maar}{plotseling weer kwaad}\\

\haiku{De avonden die ik,.}{niet bij Cornelis of oom}{T. doorbracht las ik}\\

\haiku{Wat ons bond was geen,.}{vriendschap meer het was van mijn}{kant een obsessie}\\

\haiku{Hij verbreekt het, zooals,}{een kind gedaan zou hebben}{door aan te bieden}\\

\haiku{En ook het leven,,.}{kan soms als het zin heeft wel}{sprookjes vertellen}\\

\haiku{Er is geen toovenaar noodig,.}{die met \'e\'en zwaai van zijn staf}{alles anders maakt}\\

\haiku{Daar net, toen hij met.}{zijn rug naar haar toe stond had}{hij niet gelachen}\\

\haiku{{\textquoteright} zei ze, toen ze voor,.}{hem bleef staan hijgend en met}{een piek voor haar ogen}\\

\haiku{En dan de derde,,.}{klas en de vierde met de}{grote kinderen}\\

\haiku{Als een elastiek dat,.}{je bij de einden houdt en}{rekt en rekt en rekt}\\

\haiku{Toch bleef die portier,;}{mij dwars zitten toen ik in}{de wachtkamer was}\\

\haiku{- Als u ouder wordt,,.}{zei hij met een glimlach zult}{u hem beter zien}\\

\haiku{Het wil de aarde.}{groeten met hart en voetstap}{klinkend in zijn lied}\\

\haiku{Ik zelf was \`op na -.}{den langen middag en ik}{was slechts toeschouwer}\\

\haiku{{\textquoteright} Onder dit alles:}{heeft Paap in De Spectator}{laten afdrukken}\\

\haiku{Dat begin lag al.}{betrekkelijk vooraan in}{zijn schrijversbestaan}\\

\haiku{{\textquoteleft}Uw satire is, ().}{amusant pittig endit is}{hoofdzaak ze is waar{\textquoteright}.56}\\

\haiku{hoe heet het over de:}{geweigerde bijdragen}{moet zijn toegegaan}\\

\haiku{Naar de inhoud van.}{de motie zal men echter}{tevergeefs zoeken}\\

\haiku{Daarenboven dreef.}{Paaps aard hem als vanzelf in}{de oppositie}\\

\haiku{Het onderwerp noch,.}{de omvang was bijzonder}{de t\'o\'on echter wel}\\

\haiku{Paaps relaties met;}{de Douwes Dekkers brachten}{hem ook voordeel aan}\\

\haiku{de kleuren zijn juist,.}{iets te hel de toon juist iets}{te nadrukkelijk}\\

\haiku{Na zijn verblijf in.}{Duitschland kwam Paap weer in}{Amsterdam terug}\\

\haiku{Zijn werkzaamheden:}{beperkten zich zuiver tot}{vermogensbeheer}\\

\haiku{En dan ten slotte.}{moet men hem vermelden als}{het paard van Troje}\\

\haiku{Ach, dat zijn gouden,.}{droomen Uit een voorbijen}{half vergeten tijd}\\

\haiku{{\textquoteleft}vergeven zijn uw,,{\textquoteright}.}{zonden Vrouwe omdat ge}{veel hebt liefgehad}\\

\haiku{Nu haar oogen niet meer,:}{voor mij bestonden bestond}{zijzelf ook niet meer}\\

\haiku{En ik heb toch niets,.}{dan de liefde der menschen}{verlangd snikte zij}\\

\haiku{Gedreven in mijn.}{voortgang beklom ik moeizaam}{trede na trede}\\

\haiku{De angst en de haat,,.}{die ik eens heb gekend zijn}{van mij geweken}\\

\haiku{alleen het rood is,,!}{man wees dan een zuster doch}{wees wat je kan}\\

\haiku{laat mij dan maar de,!}{Donau zijn opdat de Theiss}{kan binnenstroomen}\\

\haiku{Hoe kan het zijn dat,,?}{wat jou koelte brengt mij tot}{het merg verzengt}\\

\haiku{Zij liep nu door de,.}{volksbuurt die tusschen haar en}{het station lag}\\

\haiku{zij waren alle,.}{drie precies op tijd niet te}{laat en niet te vroeg}\\

\haiku{De muur wies hoog, ik.}{kan niet langer blikken in}{uw lokkend oog}\\

\haiku{Zoo is het, dat ik,.}{niet meer zie de deernen schoon}{fraai van gewrichten}\\

\haiku{Boven haar schoorsteen,,.}{stijgt rook van pek Netel en}{doorn distel en brem}\\

\haiku{Uit verveling en.}{oude weerzin wordt het er}{bijna gezongen}\\

\haiku{Ik blijf hier slapen!}{of laat me dragen weg door}{andre knapen}\\

\haiku{en blonde treintje.}{weten het proza van den}{middag met zijn eten}\\

\haiku{- wie garandeerde,?}{me dat hij geen verdere}{afwijkingen had}\\

\haiku{Het is alleen maar,.}{de pijn daardoor scheiden mijn}{traanklieren vocht af}\\

\haiku{{\textquoteleft}Het spijt me meer dan.}{ik zeggen kan dat je me}{geen gezelschap houdt}\\

\haiku{Er was weinig groot,.}{nieuws in die dagen en de}{beurt was aan Polen}\\

\haiku{Ik lag op een paar;}{planken in een hoek van een}{betonnen kubus}\\

\haiku{Ik was in mijn cel, -;}{volkomen ge{\"\i}soleerd}{buitengewoon stil}\\

\haiku{ik probeerde er,.}{vormpjes van te maken maar}{alles stortte in}\\

\haiku{{\textquoteright} {\textquoteleft}Duivels waren het{\textquoteright},.}{zei ik met een machtelooze}{verontwaardiging}\\

\haiku{Maar hoe ik 't ook -.}{bekijk zout kan ik van mijn}{leven niet meer zien}\\

\haiku{Is niet reeds mijn naam, '?}{te veel dit seizoen vant}{nameloze blad}\\

\haiku{Het belangrijkste.}{stuk van die rivieren ligt}{in het buitenland}\\

\haiku{Het masker van den,:}{tijd kan niemand afleggen}{of anders gezegd}\\

\haiku{De werkelijkheid......}{overmeesteren door haar te}{vereenvoudigen}\\

\haiku{En soms scheen het, dat.}{dit centrum zijn bindende}{kracht geheel verloor}\\

\haiku{Eindelijk werd hij.}{moe en hij wachtte tot zijn}{moeder terugkwam}\\

\haiku{Er volgde haar iets,,.}{een vage gestalte maar}{ze merkte het niet}\\

\haiku{Hij was er amper,.}{of Geurtje kwam kijken of}{hij er al in lag}\\

\haiku{De tuinman draaide.}{de handkar om en zette}{zich in beweging}\\

\haiku{Niets is er van 't '.}{stoffig experiment Aan}{t beeld gebleven}\\

\haiku{misschien, want {\textquoteleft}Huisjes{\textquoteright},,.}{van Kaarten haar vijfde is}{zeker niet minder}\\

\haiku{Maar nu was het, als.}{waneer je je handen met}{sneeuw hebt gewasschen}\\

\haiku{Hij meent het lied te.}{horen dat eens het droompaard}{mende op zijn reis}\\

\haiku{Ik heb u niet zo.}{lief gehad als ik u had}{moeten lief hebben}\\

\haiku{Dan alleen nog maar,:}{uw gezicht zoals zij het}{hebben doen worden}\\

\haiku{Wij echter weten.}{geen toverspreuk om onze}{dromen te bannen}\\

\haiku{Hij was de nazaat.}{van een oud doch onverzwakt}{geslacht van strijders}\\

\haiku{Hoe vol de wereld -.}{om hem heen was hij had het}{nimmer ervaren}\\

\haiku{Hij sprong als een schicht,.}{over het lage hek Fanje}{rechtstreeks naar de keel}\\

\haiku{de bloemen geurden,;}{reeds bitter de nazomer}{was al gekomen}\\

\haiku{Dat soort van held was,....}{Frits en hij zat vol kleine}{wrok om klein onrecht}\\

\subsection{Uit: Ad Interim. Jaargang 4}

\haiku{[Ad interim, 1947,]}{nummer 1 De dikke en}{de dunne muze}\\

\haiku{Daar komt nu een stroom (?}{los van geschriftenzijn het}{nog wel geschriften}\\

\haiku{Men blijft er gezond,,.}{bij de maag blijft graag het hart}{verliefd en dorstig}\\

\haiku{Waarom het in het,;}{bijzonder vaandrigs moesten zijn}{is mij niet bekend}\\

\haiku{je hebt me dingen.}{over het vliegfeest verteld die}{me onjuist lijken}\\

\haiku{Men luistert, het oor:}{aan de grond en wordt langzaam}{maar zeker gewis}\\

\haiku{Ik liep langzaam en.}{ik spande mij in of ik}{zware zakken droeg}\\

\haiku{{\textquoteleft}Ze heb zich wat te{\textquoteright},:}{goed gedaan en de mevrouw}{beaamde nog eens}\\

\haiku{De lijkwagen reed.}{juist den hoek van het parkje om}{toen wij instapten}\\

\haiku{Het begon reeds te.}{schemeren en voor zessen}{moesten wij weer thuis zijn}\\

\haiku{Maar ik achtte hem,.}{er des te meer om dat hij}{geen officier was}\\

\haiku{Het dak, van roode,,.}{broze pannen brokkelde}{naar boven toe af}\\

\haiku{Het water klotste.}{zacht tegen de onderste}{treden van de straat}\\

\haiku{Ik wilde niet dat}{zij daar naar keek en er even}{angstig van worden}\\

\haiku{Ja vele, vele{\textquoteright},, {\textquoteleft}.}{zei ze\'e\'en wil mij koopen}{als mij zoo niet geef}\\

\haiku{Maar ik lette niet.}{op haar en sloeg haar met mijn}{sabel een hand af}\\

\haiku{(Mary Carmichael).}{af Ik zou wel wenschen dat}{de maand Voorbij waar}\\

\haiku{Mij dunkt hij zoude,,,.}{koning moeten zijn En God}{weet ik tevreden}\\

\haiku{Dan vluchtte ze weg,.}{over de kleine bosweide}{het kreupelhout in}\\

\haiku{Gerrit Achterberg}{Extemporeetje ~ in}{vino veritas}\\

\haiku{Maar geld of geen geld,.}{er moet heel wat gebeuren}{als me dat niet lukt}\\

\haiku{Toch impliceert de.}{dynamiek van den geest de}{discontinuiteit}\\

\haiku{Ook in dezen tijd.}{merken wij soortgelijke}{verschijnselen op}\\

\haiku{Hoe kan de critiek?}{anders zijn dan de geest van}{de letterkunde}\\

\haiku{Ik sta daar met de.}{lucht te spreken en weet niet}{wat er achter staat}\\

\haiku{In de deuropening.}{draait hij zich een halve slag}{om en legt aan}\\

\haiku{De bruid gaat nu met:}{de afgeschoten roos op}{haar handen verder}\\

\haiku{Een lichte ruk aan.}{het gordijn en geen spier licht}{kan meer naar buiten}\\

\haiku{Op het parket ligt,.}{hij half gekromd tussen de}{brokstukken marmer}\\

\haiku{Maar na een poosje,.}{ga ik toch bemerken dat}{het heel warmpjes is}\\

\haiku{'t Is alles een,,......}{pot nat of ik nu vroeg of}{laat of laat of vroeg}\\

\haiku{Frans Coenen was het {\textquoteleft}{\textquoteright}.}{tegendeel van wat meneen}{geliefd schrijver noemt}\\

\haiku{Het heeft alles van.}{een practische leefregel}{en het is dat ook}\\

\haiku{het zwemmen in het:}{gevoel en het spelen met}{de doodsgedachte}\\

\haiku{{\textquoteleft}nog maar eentje en{\textquoteright},......}{dan niet meer en zoo voort tot}{de trommel leeg was}\\

\haiku{{\textquoteleft}il faut juger les'{\textquoteright}.}{collections d apr\`es leur}{collectionneurs}\\

\haiku{weinig meer dan een.}{naam en een aanleiding tot}{wat chauvinisme}\\

\haiku{laat hen nog maar wat.}{genieten en laat ons niet}{sikkeneurig zijn}\\

\haiku{Wij zagen hem, en.}{het was verkwikkend voor het}{Vaderlandsche Hart}\\

\haiku{van dat kleine Ik,,,?}{dat in zich zelf besloten}{geen kosmos meer kent}\\

\haiku{En nog veel vroeger -?}{al wist zij getuigt daarvan}{niet haar boek Heleen}\\

\haiku{Zondagsrust (roman,,);}{gevolgd door Bezwaarlijke}{Liefde novelle}\\

\haiku{De witte duiven,,';}{trekkebekken Warm in de}{warmte van de Lent}\\

\haiku{De philister{\textquoteright}, met.}{de naam des docerenden}{doctors erbij}\\

\haiku{en in December.}{1898 verscheen er een vijftal}{in De Nieuwe Gids}\\

\haiku{De kamerwanden,.}{zullen niets omsluiten Dat}{warm en levend is}\\

\haiku{De kamerwanden,.}{zullen niets omsluiten Dat}{warm en levend is}\\

\haiku{- Weet je, mijn beste,}{ik heb je zo\"even toen}{meneer binnen kwam}\\

\haiku{Hij heeft zijn beenen om}{zijn nek heen gevouwen en}{draait rond op \'e\'en hand.}\\

\haiku{De vorige maal:}{was hij er ook en ik heb}{gezegd tegen Fred}\\

\haiku{Blijft hij een nacht over,.}{dan zitten we urenlang in}{de hoek van een bar}\\

\haiku{{\textquoteleft}Het is gek,{\textquoteright} zegt hij, {\textquoteleft}.}{nog nooit heb ik mij zoo thuis}{gevoeld als bij jou}\\

\haiku{{\textquoteright} Mevrouw Vroom liet zich.}{met een zucht van aandoening}{weer neer in haar stoel}\\

\haiku{Want kostbaar is des;}{dichters uur Als de Engel}{van den Dood verschijnt}\\

\haiku{Mijn lief, gij hoort geen,,;}{zingen meer Geen zucht geen kreet}{kan u bereiken}\\

\haiku{Ik leef in u een:}{ander leven Dan ooit het}{lot mij heeft bereid}\\

\haiku{En in uw wezen,;}{wordt verweven Al wat mijn}{hart dorstend belijdt}\\

\haiku{er was er geen, die.}{niet verging van angst of die}{mij niet ontvlood}\\

\haiku{Als het elf uur is,:}{schuift de majoor zijn stoel naar}{achteren en zegt}\\

\haiku{Morgen, als het goed,,.}{weer is een eindje om naar}{de Abdij misschien}\\

\haiku{Kort daarna, op haar, '.}{vrijen dag kwam zijs avonds}{met hem laat naar huis}\\

\haiku{Maar wat is hij nu,,;}{mooi want hij staat in het licht}{dat hem betooverd heeft}\\

\haiku{Wist jij toen dat een?}{lied zo moe kon worden om}{wat wij verzwegen}\\

\haiku{De vader sprak, in,:}{dezelfde beurs tastend tot}{de onthutste vrouw}\\

\haiku{aan de eeuw van het,.}{wiegetouw het slaapliedje}{en de sluimerrol}\\

\haiku{Ik volstond ermee.}{je ten overstaan van haar een}{deugniet te noemen}\\

\haiku{Wat ben ik jong, dacht,,}{ik ik ben absoluut geen}{mannelijk ridder}\\

\haiku{Vreeselijk, zei ik,.}{hardop tegen mezelf wat}{ben ik nog een kind}\\

\haiku{Neem mij echter niet,,{\textquoteright}.}{kwalijk ch\'eri als ik daar}{een beetje om lach}\\

\haiku{voor de wegen het,.}{landelijk aspect kregen}{waar ik naar haakte}\\

\haiku{Onder mijn alweer.}{verlangenden blik viel zij}{in een diepen slaap}\\

\haiku{Ik weet alleen, dat.}{ik wakker was en mij in}{het praalbed bevond}\\

\haiku{Maar plotseling zag.}{ik scherper toe terwijl ik}{me over haar heen boog}\\

\haiku{Naast mij in het hooge,.}{statige praalbed lag een}{verschrompelde vrouw}\\

\haiku{Het hoofd leek uit een,;}{vochtige groezelige}{knolraap gesneden}\\

\haiku{het was bultig en,.}{bruingeel hier en daar staken}{lange sprieten uit}\\

\haiku{Ik voelde echter,.}{dat ik den aanblik niet lang}{meer kon verdragen}\\

\haiku{Gezichten kwamen,;}{\'op zich dringen al dicher}{in ontruste slaap}\\

\haiku{toen elk, vol walging,.}{hem verliet bleef ik me bij}{het lijk verbazen}\\

\haiku{Ik had dat lichaam,,;}{willen stelen te zien wat}{verder zich voltrok}\\

\haiku{Dit mengde zich in;}{la\^atre dromen met Frieslands}{wateren dooreen}\\

\haiku{zo'n dubbel landschap,,.}{was er geen domein om niet}{doorheen te komen}\\

\haiku{Men moet van dichters,;}{wel iets weten om alle}{verzen te verstaan}\\

\haiku{de kennis van het,,.}{goed en kwaad wil n\`og als vrucht}{niet zijn gegeten}\\

\haiku{wie sluipt van moeder,,.}{naar den zoon die weet precies}{wat plank er kraakt}\\

\haiku{Ik keer weer naar mijn.}{stoel terug en blader in}{de album foto's}\\

\haiku{Zij is pittig, ter,.}{zake vol verrassingen}{en zeer natuurlijk}\\

\haiku{pessimisme \`en.}{een bepaalde mate van}{blijder levenskijk}\\

\haiku{De dichter Verwey,;}{bijdrage tot het verstaan}{van zijn po\"ezie}\\

\haiku{verrassend dan in.}{zoover als er ter wereld nog}{iets verrassend is}\\

\haiku{Dit is eigenlijk,}{een zeldzaam gevoel want hoe}{vaak twijfelt men niet}\\

\haiku{De moeder en het,,.}{dochtertje herkent men vaak}{soms apart soms samen}\\

\haiku{Tot ik Caldwell las.}{heb ik gedacht dat hij me}{wilde choqueren}\\

\haiku{The Pocket Book of,.}{Popular Verse edited}{bij Ted Malone}\\

\haiku{Doch neen, ik zou het.}{gras voor de voeten van den}{heer J. wegmaaien}\\

\haiku{Wij vervingen het;}{visioen voor logische}{gevolgtrekkingen}\\

\haiku{maar de axioma's.}{van Euclides kunnen niet}{bewezen worden}\\

\haiku{In dat kamp leidden.}{wij het tamelijk grijze}{gijzelaarsbestaan}\\

\haiku{maar bovendien helpt.}{dit feit de recensent over}{zijn eerste schroom heen}\\

\haiku{Geen Nederlander,;}{die daarover niet een duit in}{het zakje wil doen}\\

\haiku{Meester Rembrandt, N.V. {\textquoteleft}{\textquoteright},.}{Uitgevers-Maatschappij}{Kosmos Amsterdam}\\

\haiku{zij betreuren een.}{beangstigend gemis aan}{oorspronkelijkheid}\\

\haiku{Het gemiddelde,,}{is hoog inderdaad doch kent}{wezenlijke kunst}\\

\haiku{Alleen al daarom.}{heeft Van Leeuwen met deze}{publicatie m.i}\\

\haiku{Siodmak doet niet.}{onder voor Hitchcock in de}{thriller-klasse}\\

\haiku{Het meisje laat een:}{stok tegen de spijlen van}{een hek ratelen}\\

\haiku{Schilderijen van.}{Milly van Duivenboden}{en Jos Vi\"ester}\\

\haiku{Een uniek overzicht, dat.}{men niet moet verzuimen te}{gaan genieten}\\

\haiku{En dat mevrouw Kloos.}{uiteraard het best van ons}{allen heeft gekend}\\

\haiku{Aldus ongeveer.}{is het ook met het boek van}{mevrouw Kloos gesteld}\\

\haiku{{\textquoteright}, omdat je toch wat,.}{zeggen moet zo nu en dan}{en luisteren weer}\\

\haiku{Dat ik hier en daar, -}{een vraagteken plaatste doet aan}{deze lof niets af}\\

\subsection{Uit: Ad Interim. Jaargang 5}

\haiku{Er is geen reiken,.}{naar de morgen waarbinnen}{gij verloren zijt}\\

\haiku{Iemand rent kermend.}{de  trap op en bonkt bij}{Mossel op de deur}\\

\haiku{{\textquoteleft}Gij moogt den doolhof,{\textquoteright}.}{nog niet uit Gij hebt nog niet}{genoeg gemind}\\

\haiku{Ik weet niet of ik}{nooit iets met bladeren te}{maken heb gehad}\\

\haiku{Doch desondanks heb,.}{ik nooit armoede gekend}{ook in mijn jeugd niet}\\

\haiku{Als ganzen liepen,.}{zij achter elkaar naar de}{deur die op slot was}\\

\haiku{{\textquoteleft}Het huis zult u wel,.}{krijgen maakt U zich daar maar}{niet ongerust over}\\

\haiku{Ze stond op en nam.}{uit een kast een boek in een}{groen verschoten kaft}\\

\haiku{Zij bladerde het.}{door en legde haar vinger}{op een bladzijde}\\

\haiku{Ik antwoordde niet,,,.}{maar ging hem voor twee trappen}{op naar de zolder}\\

\haiku{Ik bewaarde mijn,.}{nuchterheid ook in deze}{omstandigheden}\\

\haiku{prima adressen, een.}{jarenlange ervaring}{en lage prijzen}\\

\haiku{Natuurlijk wist hij,}{van dat huis waarvan wisten}{deze mensen niet}\\

\haiku{Toen ik bij haar was,.}{hield zij even met het kloppen}{op en keek mij aan}\\

\haiku{De torenvalk scheert,}{over veld en sloten aan en}{waar hij even wiekelt}\\

\haiku{{\textquoteleft}uit pi\"eteit voor{\textquoteright}.}{den overledene zoals}{ze het motief noemt}\\

\haiku{Nog zijn wij slechts tot:}{spel bij machte en niets duurt}{korter dan een lach}\\

\haiku{Doch loop niet sneller,}{door dit veld en wijk niet uit}{voor tere bloemen}\\

\haiku{de vogel valt, hoe,...}{ver hij zweefde wij sterven}{iedere dag}\\

\haiku{Hij maakte weer zijn,.}{wijde armgebaren zijn}{neus snoof weer luid}\\

\haiku{Ik herkende het,.}{woord ik had het onlangs bij}{Nietzsche gevonden}\\

\haiku{Ach ja, iedereen, '.}{ziet het maar niemand zegt het}{m in z'n gezicht}\\

\haiku{Er moest een bijna.}{ondragelijke wanhoop}{in haar leven zijn}\\

\haiku{Ik heb het gevoel,.}{dat de wereld volkomen}{vastgelopen is}\\

\haiku{Opeens greep hij mij.}{bij de schouder en keek mij}{met felle ogen aan}\\

\haiku{Ik weet... ik weet het... -,.}{zelf niet Als m'n vrouw het maar}{begreep ging hij voort}\\

\haiku{Ik liep zelf naar de,.}{huisdeur door een wonderlijk}{voorgevoel bezield}\\

\haiku{Wij hingen bijna.}{achterover tegen de wind}{om niet te rennen}\\

\haiku{Had hij Oudejaar,?}{gevierd of kwam hij pas thuis}{van een spoedgeval}\\

\haiku{'t Is misdadig.}{dat ze hun licht onder een}{korenmaat zetten}\\

\haiku{En in den Haag weet, '!}{men het en werkt me tegen}{int nasporen}\\

\haiku{Mainz 6 Juni 70,.}{Geachte heer Tersteeg}{Dank voor de ontv}\\

\haiku{Mainz 6 Juni 70,.}{Geachte heer Tersteeg}{Zooeven bragt ik br}\\

\haiku{Om bloed dat jaagt naar,:}{zijn verderven De nacht zingt}{ijler dan een riet}\\

\haiku{geen vogel, geen wolk,.}{geen dans van muggen aan den}{rossigen einder}\\

\haiku{Een doordringende.}{klank van metaal ruist in de}{mond van de morgen}\\

\haiku{Het wijfje knaagde.}{hardnekkig aan de wortel}{van de rozenstruik}\\

\haiku{Soms weet men mij voor,:}{iets te winnen Soms weet ik}{wat men van mij wil}\\

\haiku{{\textquoteright}, waarop zij zich met.}{een berustende grijns op}{een stoel liet vallen}\\

\haiku{Terwijl Eddie zich,:}{naar voren drong werd er van}{boven geroepen}\\

\haiku{Toen hij overlas wat,.}{hij geschreven had kon hij}{zijn oogen niet gelooven}\\

\haiku{Ik weet, dat het nu,.}{beginnen gaat maar het kan}{mij niets verdommen}\\

\haiku{Over het grondloos Niet}{zal hij verheerlijkt drijven}{in het onstilbaar}\\

\haiku{Zij sloegen met een.}{hakbijl en platte messen}{op een houten blok}\\

\haiku{Drie maal belde ik.}{ook aan en eveneens drie maal}{kreeg ik geen gehoor}\\

\haiku{Nou, dat is wel wat,!}{laat want zij is al meer dan}{een jaar gescheiden}\\

\haiku{Iedereen weet dat,.}{zij gescheiden is ook al}{zegt zij zelf van niet}\\

\haiku{{\textquoteright} Zij rookte nerveus,.}{ik had haar trouwens nog nooit}{eerder zien roken}\\

\haiku{het kind lag op 't,.}{aanbeeld te slapen moe van}{z'n manespel}\\

\haiku{Daarna keerde ik.}{aandachtig terug naar de}{meer besloten kom}\\

\haiku{Maar laat h\`en slape',,,:}{en m{\`\i}j Heer in dees landen}{Een wijl nog rouwen}\\

\haiku{en dat zijn blik hier.}{derft Het zicht op streken waar}{hij straks ontwaakt}\\

\haiku{door kloven zag men ', '}{t dal In ijle verte}{ent gele duin}\\

\haiku{Zeer sterk komt dit tot.}{uiting op het gebied van}{de literatuur}\\

\haiku{Wat men wel te zien:}{kreeg was een vernieuwing in}{verticale zin}\\

\haiku{nu richtte zij zich,.}{op de eigen wereld de}{kosmos van het Ik}\\

\haiku{Kwaad is de wind en,,......}{bitter de zee en grijs is}{de hemel grijs grijs}\\

\haiku{This is the day his.}{grief will be remembered}{By all who grieve}\\

\haiku{But there's deer in,.}{the orchard My apples are}{yours for the picking}\\

\haiku{En in de kiezels.}{van het Heilig water klonk}{de sabbath zacht}\\

\haiku{Daar zit ik met mijn.}{hand voor mijn hoofd en denk aan}{wat ik heb geloofd}\\

\haiku{De waarde van het.}{beeldhouwen berust op de}{moeite die het kost}\\

\haiku{ze staan al achter '.}{opt balcon Dat langzaam}{in de bocht verdwijnt}\\

\haiku{Niets anders rest hun.}{dan de troost van woorden naakt}{en zinneloos}\\

\haiku{{\textquoteright} {\textquoteleft}Best{\textquoteright}, zei mijn vader, {\textquoteleft},.}{opgewektheel best. Kijk ik}{heb mijn handen vrij}\\

\haiku{{\textquoteleft}Zij zullen het niet{\textquoteright},, {\textquoteleft}}{slecht hebben op deze reis}{zei iemand naast mij}\\

\haiku{Het middelste van.}{die vertrekken had hij tot}{eetkamer bestemd}\\

\haiku{Ik liep de gang door.}{naar de keuken en vandaar}{naar de voorkamer}\\

\haiku{Nadat hij dat voor,.}{de zesde maal had gezegd}{kwam hij niet terug}\\

\haiku{Ik herinnerde.}{mij Gerard Lutmer op dat}{oogenblik heel goed}\\

\haiku{Jij bent het geweest,.}{tegen wie ik mijn eerste}{leugen heb gezegd}\\

\haiku{Arnold was geheel,.}{anders dat zag zij alleen}{reeds aan hun handen}\\

\haiku{Hij zuchtte diep maar.}{een scheurende pijn deed hem}{in elkaar krimpen}\\

\haiku{hij kon nog juist zijn,.}{schoenen uitschoppen toen gleed}{hij in diepen slaap}\\

\haiku{De dokter trad naar.}{voren en legde zijn hand}{op Martyns schouder}\\

\haiku{Hij wilde een klein;}{schip maken en nu was hij}{bezig aan den romp}\\

\haiku{De koude er in.}{voelde aan als messcherpe}{kanten en hoeken}\\

\haiku{Ik knip je de oren...}{van het hoofd en lepel je}{ogen op een bordje}\\

\haiku{Harry zou hem half...}{vermoorden als hij er nu}{vanavond niet heen ging}\\

\haiku{De litteratuur.}{zal dat vermoedelijk niet}{veel kunnen schelen}\\

\haiku{Zijn gedichten en;}{essays bracht hij onder in}{een eigen domein}\\

\haiku{Over zijn leven en.}{publicaties enkele}{korte notities}\\

\haiku{Ieder van hen en.}{elk van hun idee\"en moeten}{wij verafschuwen}\\

\haiku{{\textquoteleft}Neem aan, dat deze:}{onbekende soldaat u}{zou willen vragen}\\

\haiku{Als men daaruit een,:}{conclusie trekken mag is}{het toch wel deze}\\

\haiku{{\textquoteleft}Wij trekken uit de /:}{aardsche schatten een drang die}{immer zich vergroot}\\

\haiku{Men vraagt zich af, wat.}{van deze film eigenlijk}{de bedoeling is}\\

\haiku{Ik moest denken aan:}{een strophe uit een zijner}{sterkste gedichten}\\

\haiku{Een vraag, die als een.}{kwellend teken tussen hem}{en mij in bleef staan}\\

\haiku{Dat het zich tot het,.}{einde geboeid lezen laat}{is onverkort lof}\\

\haiku{sympathiek door een.}{volkomen gebrek aan ernst}{en prekerigheid}\\

\haiku{Die winde dwarrel...}{eers en swaai dat die stowwe}{helder blink en draai}\\

\haiku{sy gang was van die,.}{wye westewind sy bruine}{o\"e blink en bly}\\

\haiku{{\textquoteleft}Die hoed had ik 's,{\textquoteright};}{middags gekocht omdat ik}{me treurig voelde}\\

\haiku{Voor het charmante.}{stofomslag van G. Douwe}{een apart compliment}\\

\haiku{Het is werk om veel,.}{op te slaan en dus om in}{zijn kast te hebben}\\

\haiku{de terugtocht van;}{twee kameraden uit de}{hel rond Duinkerken}\\

\haiku{In dit tweeledig:}{debuut schijnt een dominee}{aan het woord te zijn}\\

\subsection{Uit: Ad Interim. Jaargang 6}

\haiku{Want het gaat hier om.}{het belangrijk onderscheid}{tussen ginds en hier}\\

\haiku{de aandacht voor het,,.}{detail werkelijkheidszin}{gevoel voor humor}\\

\haiku{de merel stort zich.}{met een kreet vol wildheid in}{de voorjaarsvlagen}\\

\haiku{Ik voelde dat ook.}{haar handen klam waren van}{mist die regen werd}\\

\haiku{En daarna greep haar.}{hand feilloos een deurknop en}{deed ze een deur open}\\

\haiku{Luister, zei ze, en}{ze drukte me daarbij in}{de enorme stapel}\\

\haiku{{\textquoteleft}als de inhoud van,{\textquoteright};}{een zachte ziel waarop te}{lang de leeddrop viel}\\

\haiku{ik houd veel van kunst,,.}{maar alleen van oude niet}{van de moderne}\\

\haiku{En dat was waarlijk,.}{lang niet altijd het geval}{denk maar aan Rembrandt}\\

\haiku{Dan bedenk ik wel '.}{n strik voor balsturige}{demagogen}\\

\haiku{Periclidas, die!}{namens geheel Sparta om}{hulp kwam vragen}\\

\haiku{Naast de vrije pachters '.}{staan de mannen vant vrije}{beroep in de stad}\\

\haiku{Het is alsof het.}{donkere zaakje er van}{uitgebeten is}\\

\haiku{Zo onverhoeds was,.}{het krachtige licht dat ik}{eerst de fles niet zag}\\

\haiku{Maar zijn zichtbare.}{trots dwong hem toch ons voor te}{gaan naar de kelder}\\

\haiku{Ik rustte eerst, toen.}{hij zijn geheim volledig}{had prijs gegeven}\\

\haiku{Het was niet de lak,.}{maar mijn buikstreek die het had}{moeten ontgelden}\\

\haiku{Ook mijn scheutje olie.}{glibberde dus knetterend}{over het pannetje}\\

\haiku{Ik liet hem door de.}{buffels regelrecht naar een}{verfspuiter slepen}\\

\haiku{{\textquoteright} {\textquoteleft}De volgende week,{\textquoteright}, {\textquoteleft}.}{zei hijzult u eenmaal per}{dag gelucht worden}\\

\haiku{{\textquoteleft}Het spijt mij wel,{\textquoteright} zei, {\textquoteleft}.}{hijmaar dan moet ik eerst eens}{beter gaan vragen}\\

\haiku{{\textquoteright} zei ik, {\textquoteleft}als wij nu,.}{deze kroeg in gaan dan moet}{je goed opletten}\\

\haiku{En wij hebben het?}{toch niet over wissewasjes}{of vind jij soms wel}\\

\haiku{{\textquoteright} zei hij, {\textquoteleft}ja, laten.}{wij maar zeggen dat het een}{figuurtje voorstelt}\\

\haiku{Maar ik zou er nog.}{graag een beetje zonneschijn}{in willen hebben}\\

\haiku{Wanneer ik schrijf, zei,.}{een romancier dan ben ik}{het zelf niet die schrijf}\\

\haiku{Aan mijn dochter  :}{Bert Voeten  Als men je}{later zal zeggen}\\

\haiku{Twee tegen een en.}{die ander staat er toch maar}{alleen tegenover}\\

\haiku{Toch liever doodziek.}{op een eenzaam eiland dan}{Marge naast mijn bed}\\

\haiku{al had het al in.}{de ochtendbladen gestaan}{nog zonder foto's}\\

\haiku{Nee, Woensdag was het,,.}{waar blijft de tijd alweer drie}{dagen geleden}\\

\haiku{En ik huilen maar,.}{gewoon de ergste nervous}{breakdown sinds Daddy}\\

\haiku{Hij boezemde mij,.}{geen angst in maar weerzin en}{zwijgende wanhoop}\\

\haiku{Tegenover haar zat,}{een heel dikke man van wie}{ik van onderaf}\\

\haiku{De lange hals lag.}{bij beiden verdord tussen}{gestrekte pezen}\\

\haiku{{\textquoteleft}Nonsens,{\textquoteright} weerlegde,.}{kort haar overbuur die tot nog}{toe gezwegen had}\\

\haiku{Heb even geduld, mijn.}{tong moet losraken  van}{mijn gehemelte}\\

\haiku{{\textquoteright} vroeg ze even later.}{toen ze hem bezig zag toch}{een krans te vlechten}\\

\haiku{Wanneer ik dacht aan,:}{de oproep kon ik maar \'e\'en}{verklaring vinden}\\

\haiku{ik zou alleen maar,,.}{nogmaals de toevalligheid}{kunnen vaststellen}\\

\haiku{{\textquoteright} vroeg ik een kellner,.}{die juist een volgeladen}{blad op zijn arm nam}\\

\haiku{{\textquoteright} En kennelijk blij,,.}{van mij verlost te zijn schoot}{hij zijn hokje in}\\

\haiku{Na een sluimering.}{van een kwartier was het tijd}{om uit bed te gaan}\\

\haiku{Trouwens, het is niet,,?}{eens echt haar het is een soort}{schimmel begrijpt u}\\

\haiku{{\textquoteleft}Een ding nog, Bertrand,,!}{kom doe der wat an gooi even}{mijn lijk uit het raam}\\

\haiku{{\textquoteleft}Reinig ons van het,{\textquoteright}.}{kwaad Als de Tabor staat uw}{rechtvaardigheid}\\

\haiku{Hij is geboren.}{met een rijke hoeveelheid}{mogelijkheden}\\

\haiku{Gewiss, ihm geben.}{auch die Jahre Die rechte}{Richtung seiner Kraft}\\

\haiku{Der Unfall lauert.}{an der Seite Und st\"urzt ihn}{in den Arm der Qual}\\

\haiku{- heette het - {\textquoteleft}ist f\"ur.}{die Deutsche Literatur}{ein unersetzlicher}\\

\haiku{Ik zie hem zitten.}{en ik verheug mij over zijn}{tegenwoordigheid}\\

\haiku{Een enkele keer {\textquoteleft}}{veegde hij zijn mond af met}{de rug van zijn hand.}\\

\haiku{{\textquoteright} {\textquoteleft}Het is gek,{\textquoteright} zei hij, {\textquoteleft},.}{maar als jij gaat zoeken dan}{ga ik met je mee}\\

\haiku{{\textquoteright} Ik tastte met mijn.}{ene hand in mijn zak en gaf}{hem een dubbeltje}\\

\haiku{De muzikant bleek:}{inderdaad  verdwenen}{te zijn en ik zei}\\

\haiku{Ik keek mijn buurman,.}{aan die aldoor aandachtig}{naar buiten staarde}\\

\haiku{{\textquoteright} {\textquoteleft}Verrotte appels,{\textquoteright}, {\textquoteleft}.}{zei mijn buurmandie man heeft}{ons rotzo verkocht}\\

\haiku{{\textquoteright} {\textquoteleft}Nee,{\textquoteright} zei mijn buurman, {\textquoteleft}.}{maar daarginder staat iemand}{naar ons te kijken}\\

\haiku{Ook dan bedreigt hem,.}{nog de tijd als iedere}{mens als ieder ding}\\

\haiku{wordt met afkeer niet,;}{vervuld Over ons lot omdat}{wij zwaar misdreven}\\

\haiku{Ik heb mijn moed en,;}{macht verloren mijn vrienden}{en mijn vroolijkheid}\\

\haiku{En zo gij woorden,:}{samenbrengt Niet \'e\'en verdient}{er uw misprijzen}\\

\haiku{de oude vrouwen}{Onder de hemel grauw en}{laag Op het kerkhof}\\

\haiku{Een gepantserde.}{automobiel brengt mij naar}{de gevangenis}\\

\haiku{Zij heeft een blinkend.}{gepoetst zilveren kruisje}{op haar borst hangen}\\

\haiku{Het eerste half uur.}{liep ik rond met een vreemd licht}{gevoel in het hoofd}\\

\haiku{Want zij zijn bezig,.}{de taal te vermoorden waar}{ik nu nog van houd}\\

\haiku{Ik weet, dat ik niet.}{anders praat dan U. Ook U}{bent ontevreden}\\

\haiku{wat hij me schreef ben...}{ik blijkbaar toch een beetje}{van mijn stuk geraakt}\\

\haiku{de uniformpet komt.}{hem niet meer toe en is een}{belachelijk ding}\\

\haiku{Gadegeslagen.}{door zijn zoontje wordt hij op}{heterdaad betrapt}\\

\haiku{Zelfs wanneer man en,.}{kind niet in het beeldvlak zijn}{duurt de beklemming}\\

\section{anoniem}

\subsection{Uit: Van Brabant die excellente cronike}

\haiku{iaren so dachten.}{die goede lieden haer graf}{te verbeteren}\\

\haiku{So antwoorde si}{van binnen ontfunct sijnde}{vande vuere}\\

\haiku{Si ordineerden}{nochtan de tijt wanneer}{men die brulocht houden}\\

\haiku{iaren out sijnde.}{die iongers in arbeyde}{ginck te bouen}\\

\haiku{So dachte hi ooc.}{haer graf te versoeken met}{groter deuocien}\\

\haiku{hem met der caken,.}{ende trecten opt droghe}{twelck si dede}\\

\haiku{so gaf si alle.}{haer goet den armen menschen}{om de minne gods}\\

\haiku{mit abstinentien}{ende een milde gheuer}{van aelmoessenen}\\

\haiku{hy visiteren.}{die plaetsen der heyligen}{in allen landen}\\

\haiku{JNden tijden van}{Godefroot van bullion}{so leefde sinte}\\

\haiku{als hi wt sijnre.}{moeders lichaem geboren}{was Als hi .xvij}\\

\haiku{Dit aenhorende,}{so bleef hi daer ende die}{abdisse ende}\\

\haiku{opten lesten dach.}{van Junio op eenen vrydach}{na dat hi .xxvi}\\

\haiku{Maer die voorseyde}{broeder Aernout sprack hem}{vriendelicken toe}\\

\haiku{Daer na vraechde si,,}{haer na hare vrienden som}{leuende som doot}\\

\haiku{Christi seer sieck so}{datmen haer doot verwachte}{Doe wert daer ghemaect}\\

\haiku{drie iaer lanc so dat}{hi leggen noch slapen en}{conde Ende so}\\

\haiku{Ende so gheuielt}{op een auontstonde dat}{hi denckende}\\

\haiku{wert int cloostere}{te Ewyers in wals Brabant}{Mer daer te voren}\\

\haiku{so vraechde hi den}{man oft hi die heylige}{vrouwe gesien had}\\

\haiku{Als die man ontrent}{een vierendeel van eender}{milen gegaen was}\\

\haiku{Op die selue stont}{soe wert die boose knechte}{diese belogen}\\

\haiku{hi stichte in dat.}{lant en grote stadt die hi}{noomde Sicambrien}\\

\haiku{Vanden genen die}{van hem achtersprake deden}{plach hi te seggen}\\

\haiku{Hi liet na hem twee,,.}{sonen deen heet Brabon dander}{Eneas dat sijn .ij}\\

\haiku{seer ende seyde,}{Jst dat dese dingen van}{god zijn so bid ic}\\

\haiku{Alchuinus leerde.}{coninc karel dat meeste}{deel sijnre consten}\\

\haiku{Ende karel gaf}{wt een ghebot dat alle}{man hem bereyden}\\

\haiku{Ende die gheen die}{ongelouich blijuen wilden}{die werden ghedoot}\\

\haiku{ic ben van vanden}{geslachte der Fransoysen Die}{ruese seide}\\

\haiku{Ende met groten}{gheroepe ende ghecrijsch}{van beyden siden}\\

\haiku{waren gesalft van.}{haren vrienden met mirre}{ende balseme}\\

\haiku{Want op dyen dach}{dat wi te Vyenne van}{malcander scheyden}\\

\haiku{drie iaer gheschiet sijn,.}{te weten dat de sonne}{ende mane .vij}\\

\haiku{Aldus heeft god dan}{tlant van Brabant hoochlic}{verciert met desen}\\

\haiku{vander hemelscher.}{Jerarchyen die sinte}{Dyonijs maecte}\\

\haiku{si gingen hem aen.}{inden name gods ende}{versloegen dit dier}\\

\haiku{Huge vader van.}{Huge capet maerschalc van}{al Vrancrijcke}\\

\haiku{Oede die namaels}{nam te manne Lambrecht graue}{van Bruessel daer}\\

\haiku{gheheten metten.}{baerde des grauen broeder}{van Henegouwe}\\

\haiku{voorseit ~ Jtem in.}{dese tijt werden gedoot}{tot Jherusalem .xi}\\

\haiku{iammerlic verdruct,}{Dies hi grote compassie}{hadde ende oec}\\

\haiku{om te winnen dat}{heylich lant Ende als si}{tot Colen quamen}\\

\haiku{Mer God sandt ouer}{hem al sulcke anxte}{datse die mueren}\\

\haiku{Doe scheyden si haer}{heyr van malcander om te}{bat te passeren}\\

\haiku{Daer na vonden si}{een riuier daer vele}{hem doot droncken}\\

\haiku{voorscreuen want hi}{wiste wel datter niet meer}{volcx en quam wt}\\

\haiku{Als dander kersten}{sagen ende vernamen}{dat die kersten int}\\

\haiku{onsen ende plach.}{niet te hebben dan een mael}{ende een palster}\\

\haiku{Ende Guillelmus.}{van Parijs die ordene}{vanden Agustinen}\\

\haiku{als Benedictus.}{Expugnat  Domine}{in virtute tua}\\

\haiku{Die vierde is de,.}{graue van Vlaenderen dye draecht}{des conincx swaert}\\

\haiku{dat nv Loreynen,.}{heet ende besat also}{beide die landen}\\

\haiku{ia si sloten voor}{hem ende voor sijn moeder}{die poorten vander}\\

\haiku{na so stelden hem.}{al Vlaenderein in handen}{vanden coninck}\\

\haiku{wederstonden sijn,.}{tyrantscap want die coninc dat}{niet en beterde}\\

\haiku{iaren geregneert}{hadt so sterf hi vanden steen}{ter Vueren Jnt}\\

\haiku{So ontboot coninc}{Philips hertoge Jan dat}{hi den voorseyden}\\

\haiku{Ende en quame.}{hi daer niet so souden si}{de stadt behouden}\\

\haiku{So track coninc Eduwaert}{self totten  keyser te}{Couelens Ende}\\

\haiku{iare quam een nieu}{mare totten hertoge}{datter in Sauoyen}\\

\haiku{M. mannen oft meer,.}{rouende al dat hi in}{sijn ghemoete vant}\\

\haiku{ten lesten den strijt.}{verloren daer verslagen}{bleuen omtrent .xl}\\

\haiku{hi sinen bode}{marten tot Perpiniaen aen}{coninc Segemont}\\

\haiku{so quaden raet dat.}{si noch haer dochter niet en}{mochten verhoort sijn}\\

\haiku{oft water datmen}{den gront daer niet van vinden}{en conde so diep}\\

\haiku{oordens, die dat voort}{in allen hoecken ende}{landen des werelts}\\

\haiku{Jan bont cancellier}{van brabant van hertoghe}{Philippus ontset}\\

\haiku{Ende also wert}{den twist ende oorloge}{ghecesseert ende}\\

\haiku{grote scade int.}{lant van ludick wt ende}{inne rijdende}\\

\haiku{Kaerle graue van.}{Charloys hertoge Philips}{soon van Bourgondien}\\

\haiku{ordinerende}{een schone processie om}{den prince met}\\

\haiku{graue van vlaenderen}{ende den ingesetene}{der stede van ghent}\\

\haiku{hem omme metten.}{sinen ende versloegen}{alle die van Ghent}\\

\haiku{sinen pays metten.}{hertoge Eduwaert ende bleef}{voort aen getrouwe}\\

\haiku{Jan van koesteyn die.}{welck een speciael vrient was}{des heren van Croy}\\

\haiku{hi van stonden aen,}{weder te scepe ende}{hier en binnen starf}\\

\haiku{battalien   Den.}{eersten leyde die graue van}{Simpol met sijn .iij}\\

\haiku{Ende als hi dat.}{bisdom in groten rusten}{ende vreden .v}\\

\haiku{wouden namen si.}{op te branden die molen}{van Montenaken}\\

\haiku{sien, ende binnen}{desen tijden worden hem}{ministreert ende}\\

\haiku{Een vre te voren}{dat die Keiserlijcke}{Maiesteyt wt rijden}\\

\haiku{soude, so reet heer.}{Jan van baden aertsbisscop van}{trier wter stadt met .CCCC}\\

\haiku{Daer na quamen twee}{edelen wt oostenrijck}{met bloten armen}\\

\haiku{Daer werden soveel.}{tenten ende husen op}{gestelt datter .iiij}\\

\haiku{so crachtlijken dat}{si van noots wegen mosten}{achter rugge keeren}\\

\haiku{ende commissie}{namen die ambassaten}{ouer te dragen}\\

\haiku{So heeft si dan ten}{lesten haer consent ende}{wille ghegeuen}\\

\haiku{is een die bouen,}{alle de anderen seer}{schoon is ende alst}\\

\haiku{ende als hi dan,.}{verslagen is so hebdi}{den slach gewonnen}\\

\haiku{lancie spranck in, {\textparagraph}}{vij stucken ende reedt doen}{wter banen na thof}\\

\haiku{hoeck wert heel vanden,.}{turcken verwoest dat mijn}{herte seer swaer is}\\

\haiku{strijt voor tkersten bloet,.}{datter ghestort was ende}{die Fransoysen weecken}\\

\haiku{biden heere van}{Beuere Ende die Arien}{ouer gaf ontfinc}\\

\haiku{Te weten heere}{Jan van Lannoy abt sinte}{Bertijns cancellier}\\

\haiku{het wert met wijsheyt}{beslicht datter so groten quaet}{niet en geschiede}\\

\haiku{Doe quam de coninc,}{wt ende vergaf dat hem}{misdaen was ende}\\

\haiku{het wert gemaect dat}{dye van Vlaenderen ende}{dye Roomsche coninc}\\

\haiku{wert beuonden dat}{die procuratie vanden}{here van Polem}\\

\haiku{dit gesciede op,.}{onser vrouwen lichtmis auont}{int iaer .xvC. ende.xij}\\

\haiku{Hoe die coninc van {\textparagraph}}{Vrancrijck oorlochde op}{die Venechianen}\\

\haiku{soude te geschien,}{inden heyligen Roomschen}{rijcx saken daer}\\

\haiku{grotelijc verpeynt}{was Jtem by rade vanden}{keyser Maximiliaen}\\

\haiku{twee heeren dye een.}{steecspel beroepen hadden}{So quam opten .xi}\\

\haiku{De hertoge van.}{Lunenborch vader van mijn}{vrouwe van Gelre}\\

\haiku{maenden daer buten,.}{gehouden hadden daer hi}{wel in was .xlv}\\

\haiku{dat vreeslick schieten.}{dat Ferdinandus dede}{heeft hijt binnen viij}\\

\haiku{van des conincx.}{pagien opt turcx ende}{ander schoon peerden}\\

\haiku{Van binnen waren.}{si verdruct ende verheert}{vanden ghelderschen}\\

\haiku{Jnden seluen}{tijden waren twee vaenkens}{knechten van sbisscopen}\\

\haiku{Aldus bleef des bisscops}{volck al stille tot des}{anderen morgens}\\

\haiku{dit nauolgende}{Jnden eersten hebben si}{gheordineert .xvC.}\\

\haiku{M. peerden ende,.}{.xxM. voetknechten ende C.}{sacken met coren}\\

\haiku{ende quamen ooc}{inne met haer comende}{in ordenen Eerst}\\

\haiku{[Sinte Begge,,]}{hertogin zie hiernaast}{Sinte berlendis}\\

\subsection{Uit: Het boeck vanden pelgherym}

\haiku{pelgrim slaapt met hand;}{onder wang op een helling}{links van een beekje}\\

\haiku{elckerlijc leeren}{wat wech men sculdich is te}{gaen of te laten}\\

\haiku{Dit is licht te doen}{want daer en is geen so rijc}{hy en is wel arm}\\

\haiku{maendenin gheweest}{hadde sonder dat ic ye}{daer wt quam Ende}\\

\haiku{niet ter werelt thans.}{my so van node is als}{vwe bystandicheyt}\\

\haiku{Want ic seg di dat}{hier meer kinderkijns lijden}{dan oude luyden}\\

\haiku{Dit is het eerste}{passagie na ierusalem}{van allengoeden}\\

\haiku{wt eender toerne.}{ende was een maecht ende}{ghinc tot hemluyden}\\

\haiku{Ende dan daer na.}{vinstu hem ouerhorich}{ende niet obedient}\\

\haiku{v gheopenbaert}{ende beleden te zijn}{Besiet nu of ghij}\\

\haiku{gij my zeer misdaen}{hebt ende tegen de loep}{mijnder gewoenten}\\

\haiku{die wel soe suldy}{claerlic bekennen v}{mestersse ende}\\

\haiku{int gros want also}{int gros te biechten en is}{niet anders danmen}\\

\haiku{v verwonderen}{mach ende daer bij zijdijt}{schuldich te weten}\\

\haiku{wt wil doen datmen}{den bessemdaerwaerts keert}{want het waer schande}\\

\haiku{een die geheten}{is nemia dats poorte van}{de vuylichede}\\

\haiku{de x geset is}{Daer is by te verstaen dat}{ic ben geheeten}\\

\haiku{Sapiencia of wijsheit.}{Al datmen visieren mocht}{dat conste zij doen}\\

\haiku{dat die gene die}{dit wambaeys an heeft hi doet}{sijn profijt ende}\\

\haiku{siet hyer daer in.}{dat sinte benedictus}{zweert stac doe hijt droech}\\

\haiku{Dijn secreten als}{hijse weet sal hy seggen}{den philistienen}\\

\haiku{Als ic my aldus.}{sach ontladen soe vloech ic}{terstont in de lucht}\\

\haiku{hem hebben Ende}{dan als hy soe tondere}{ghedaen soude zijn}\\

\haiku{mede du en zijs.}{de eerste pelgrym niet die}{hier gecomen heeft}\\

\haiku{En wil niet meenen.}{dat ic gaen sal weegen die}{men laecken mach}\\

\haiku{Het is de selfde}{stoc die de kaerl in zijn}{hant had hier voeren}\\

\haiku{ende nochtan.}{soe en waen ic niet dat ic}{dan steruen sal}\\

\haiku{men kent die lieden}{niet bi haren clederen}{noch den goeden wijn}\\

\haiku{goeden mensch dan de}{natuerlicste dief die}{ye gheboren was}\\

\haiku{hoe qualic dat my.}{quam In desen bosch daer ic}{of genoept hebbe}\\

\haiku{ende waert also.}{dat icken dienen moeste bi}{anxte vander doot}\\

\haiku{die neerkijkt op de[]}{kerk en het schaakbord\ensuremath{>}}{55va Doe dede}\\

\haiku{dien heb ic een edel:}{instrument ghegeuen om}{keerels werc te doene}\\

\haiku{Ic wachte hem al.}{zijn groot goet zijn gout ende}{zijn siluere}\\

\haiku{daer en is gheene.}{vervulte an noch gheene}{genoechlichede}\\

\haiku{In marcten gaet si.}{die luden vijlende voer}{elckerlijke}\\

\haiku{haeck ende dese}{figuere thoent dat ic}{bin abdisse mair}\\

\haiku{si is mi seere}{nootsakelic om dat ic}{te doene hebbe}\\

\haiku{dies ghelijke daer.}{ick meest ghels zie darwaerts treck}{ic mijne tonge}\\

\haiku{ende noch sulstuut}{bet weten alstu suls horen}{van mijnen monde}\\

\haiku{wat sulstu segghen.}{ketijt nv bistu comen}{by dijnen eynde}\\

\haiku{my vertroest van.}{mijnre droefheden ende}{beschermt vander doot}\\

\haiku{nv sie ic wel dat.}{ghy noch myne siele niet}{vergheten en hebt}\\

\haiku{anden coninck.}{ende settene weder}{inden rechten wech}\\

\haiku{want tot dy comick.}{om te sine ghenesen}{van mijne quetsuere}\\

\haiku{wat peynsende te.}{wat inden ic die hage}{ghelaten hadde}\\

\haiku{Nu coemt voirt ende.}{gheeft mi dine male By}{lode seyde ic}\\

\haiku{dat ware al te.}{starc te doene luden van}{putertieren sinne}\\

\haiku{hebstu eenen [74rb] soe.}{heb ic wel gheuaren want ic}{salre op smeden}\\

\haiku{want te meer schaemten.}{dat een man heeft te meer vint}{hi persecucien}\\

\haiku{ende ghebieden}{hem allen dat si di hier}{in sijn onderdaen}\\

\haiku{want nye mensche en}{setter voet hi en hadder}{te vorengesent}\\

\haiku{si leyden my te}{bedde ende bonden my}{daer in seggende}\\

\haiku{82ra] ricordia om.}{die te treckene wt der}{ketiuicheden}\\

\haiku{Nu segt seide die}{queene maer haest di want elder}{heb ic te doene}\\

\haiku{45 De laatste zes.}{regels van deze kolom}{zijn niet bedrukt}\\

\haiku{De rechterkolom.}{van dit folium is dus}{deels leeg gebleven}\\

\subsection{Uit: Cronyk van Sint Aagten Convent. Een oude kloosterkroniek uit de 15-17e eeuw}

\haiku{Wij zien dit alles.}{weliswaar vanuit een zeer}{beperkt gezichtspunt}\\

\haiku{De Geuzen zijn met.}{behulp van hun vrienden in}{de stad gekomen}\\

\haiku{Zou deze tijd - toch.}{een van de belangrijkste}{uit onze Vaderl}\\

\haiku{De houte geute.}{van de Keucken verbrande}{geheel buyten af}\\

\haiku{Men gaf hem half meer,.}{dan Baden ende Baden}{half meer dan David}\\

\haiku{*~         In dit jaar wierd '.}{de eerste steen geleyt van}{t nieuwe Ziekhuys}\\

\haiku{Dan brande niet meer.}{door de gratie Godts dan de}{cap van de Tooren}\\

\haiku{Daar was soo grooten,.}{elende van honger dat men}{niet schryven en kan}\\

\haiku{Daar was groot gerugt,.}{in de Stadt alsoo de Geusen}{voor de Stadt kwamen}\\

\haiku{Syn Broeder hadt ons,.}{ontboden dat wy hem niet}{inlaaten souden}\\

\haiku{waar op den Heere '.}{Vicariust selve}{heeft geconsenteert}\\

\haiku{we vermeld in een {\textquoteleft}{\textquoteright}.}{Amersfoortse kroniek van een}{naamloze schrijver}\\

\haiku{Zie ook onder 1531 {\textquoteleft}{\textquoteright}.}{waar over ditvoor de traelje}{gaan gesproken wordt.1426St}\\

\haiku{Nu heeft iedere,.}{bisschop een rechtstitel een}{rechtsgebied nodig}\\

\haiku{Hiervan kunnen we.}{ons op twee manieren een}{denkbeeld vormen}\\

\haiku{Theoretisch is,.}{dit ook nu nog mogelijk}{zelfs met een paus}\\

\haiku{het vruchtgebruik van '.}{het goed hebben en geen macht}{overt goed zelf}\\

\haiku{b. Welke functies?}{van de zusters worden in}{de kroniek genoemd}\\

\haiku{Wat betekent het {\textquoteleft}{\textquoteright},?}{woordontboden als je het}{zinsverband nagaat}\\

\haiku{Men kan op de kaart.}{precies nagaan hoe de brand}{over de stad heen sloeg}\\

\subsection{Uit: Handschrift Hattem C 5}

\haiku{Ist zake dat hij}{zijn rike   voeght ende}{regiert na de wet}\\

\haiku{Zij worden daer wel}{ontfanghen ende zij}{hebben daer goet}\\

\haiku{wort een co-}{ninc ontzien ende bedwingt}{zijn vianden}\\

\haiku{v machtich zijn   {\textparagraph}}{te cranckene Ooc en}{zuldi niet uwee}\\

\haiku{wair omme dat de}{daghen cort ende lanc}{worden Sij kende}\\

\haiku{hoe vele heeren}{hebben   alzoo fenijn}{ghedroncken}\\

\haiku{Nu merct wederomme.}{met wat   zaicken dat de}{mensche magher wort}\\

\haiku{regiert dat vier {\textparagraph} Voort}{alle boomen ende}{planten die groyen}\\

\haiku{Sommeghe doen ooc}{wel de warach-  ticheit}{voor ooghen comen}\\

\haiku{Den eenen en zuldi}{niet beghiften bouen}{den anderen}\\

\haiku{zeere beroert / hij}{en wiste niet wat   dit}{bedieden mochte}\\

\haiku{hebbe ende dat.}{jc dine wet ghehouden}{heb-  be}\\

\haiku{heeft want dicwil}{merctmen bijden bode de}{wijsheit vanden}\\

\haiku{moet   ghij hebben}{bij v wijze meesters in}{astro-  nomien}\\

\haiku{ventositates /}{a meridie quia ille}{sunt immunde}\\

\haiku{werden335 {\textparagraph}  Men sal}{nemen herts hoern ghepul-}{uereert Semis}\\

\haiku{jn houden olie van}{rosen mit een   lettel}{mastic Ende ist}\\

\haiku{mit borne ende /}{mit zeeme in enen stoep}{borrens sal-}\\

\haiku{Dit coemt somtijt}{van bloede Ende somtijt}{soe comet wt}\\

\haiku{be-  uerscul}{ende olium ende die}{galle vanden}\\

\haiku{men houden wil die}{galle vanden stier   soe}{salmense houden}\\

\haiku{xi marc ende of}{1 onse  ende es}{des copers een marc}\\

\haiku{hij lan-  gher}{blanchieren moet Ende soe}{hij dan lichter}\\

\haiku{M549Eister rasis550}{seit dat bloet te   laten}{verlicht die oghen}\\

\haiku{seluen arm Mer}{gheuoeltmen  weede off}{prekelinghe an}\\

\haiku{int vier iij ofte}{iiij korlen van mirre}{Ende dus sultu}\\

\haiku{vanden philo-}{sophen ende die tot}{den dienst behoren}\\

\haiku{de / of den ghenen}{die van quaden humoren}{ghe-  zwollen}\\

\haiku{het derde water}{is alsoe sterck datmer}{stael in soode}\\

\haiku{dulce lac mixtum}{cum farina ordij}{et impositis}\\

\haiku{dit boeck is vol vol728 /}{van gheproefde medi-}{cine ende}\\

\haiku{vanden oghe met}{deser curen  V765An die}{ander ziecheide766}\\

\haiku{hem wezen drie of}{iiij   ende alsoe van}{anderen dinghen}\\

\haiku{coemt vter armenien}{van ouer zee Ende}{es cout ende}\\

\haiku{purgiert   hete}{colere vander maghen}{ende vander}\\

\haiku{die ghescuert}{sijn ende in die culbalch}{gheuallen dair}\\

\haiku{heeft ghelue verwe}{Ende es hol   ende}{breect lichtelike}\\

\haiku{terut Ende  het}{saet in vlaemsche Spo-}{rie Ende es}\\

\haiku{socht de zweere}{vander artetike die}{van hetten coemt}\\

\haiku{dat es om te    {\textparagraph}}{wachtene van quaet doene}{Men  mach gheuen}\\

\haiku{{\textparagraph} Gheminget mit}{aysile ende ghesmeert op}{die stede daert}\\

\haiku{ghesoden ende}{ghe-  plaestert op het}{hooft doet het hair}\\

\haiku{Dat es roy dat}{op het water vliet Ende}{es cout   ende}\\

\haiku{Het   sop vander}{scortsen van roode wilghen}{es goet jnde}\\

\haiku{verteert   coude}{ja ghemenghet mit pepere}{S1549olatrum1550}\\

\haiku{es goet gheplaestert{\textparagraph}}{op dropen    Glas smout1600}{S1601aginum vitri1602}\\

\haiku{Dat es dolke die}{wast   in coerne ende}{meest in vorten}\\

\haiku{hij sal merken of}{dat   bloet zwert vte springet}{of coemt   Ende}\\

\haiku{Of het   coemt van}{medecinen diemen te}{vele   gheeft Off}\\

\haiku{lichame doere}{off in ene stede ende}{dan die drope}\\

\haiku{dair of Nochtan}{salmense smeren mit}{arragon off}\\

\haiku{Een ander1798  Men}{sal nemen borax ende}{peper ende}\\

\haiku{ende doet dat    {\textparagraph}}{gat droghen Ende jnden}{mont ghehouden}\\

\haiku{gate op dat    {\textparagraph}}{si ghesuuert sijn van hairren}{quaden  gronde}\\

\haiku{Op dien dach dat sijt}{nuchteren drincken}{die quaden adem}\\

\haiku{het conforteert de {\textparagraph}}{maghe ende alle}{de leede Ende}\\

\haiku{pen alrehande}{onghemake alse}{kan-  kere}\\

\haiku{water datmen}{maket van eenen cruude ende}{van anderen}\\

\haiku{wateren ende}{doetse al te hope}{van elken euen}\\

\haiku{die steden vanden /}{puusten  die sij hebben}{sijn ghelue ende}\\

\haiku{die tweedeel ende /}{dat salmen temperen}{mit aysine off}\\

\haiku{Ende eist dat hijt}{etet soe en was die hont}{niet ver-  woet}\\

\haiku{handen {\textparagraph} Die   ooc}{tysike sijn nutten van}{dezen sape het}\\

\haiku{vin blancus2333  et}{estouppes de chen-}{neue bien moullies}\\

\haiku{veurres mer-}{ueille Car nul reme-}{de nest milleur}\\

\haiku{onche dencens}{bien puluersiee et}{demy onche}\\

\haiku{de ventre con-}{tinuellement plus}{que  es aultres}\\

\haiku{Vissche wt de}{riuieren in stede}{dair   steenen leggen}\\

\haiku{Ende van ouden}{zoghen Ende vanden}{osse Ende}\\

\haiku{samenen die in}{enen cloc   ouer tfier daer}{wt sal-  3007}\\

\haiku{alrehande}{drope diese dair mede}{saluet Ende}\\

\haiku{smorghens nuch-    /}{teren drinct al cout ende}{des auonts all}\\

\haiku{doer die natheit der}{eerden dat hi nv is}{ghe-  wonnen}\\

\haiku{lichaem heeft die}{drincke de olye vander}{fyolen off}\\

\haiku{de ghe-  nen}{die ghewont js al   te}{hant gaet dat yser}\\

\haiku{Als dit js ghe-}{daen soe salmen nemen}{paerden}\\

\haiku{ghenen  diese {\textparagraph}{\textparagraph}}{dicke nutten  Teghens}{het buuc euel3241}\\

\haiku{vander betonien}{ende bijn-  dense}{vp sijn voerhooft}\\

\haiku{te   poluer}{ende nvttense   des}{wijnters alsmen}\\

\haiku{des ghenen die de}{zucht   heuet ende niet}{slapen   en mach}\\

\haiku{Men sal stoten een}{crwt dat heet henne}{kers ende nemen}\\

\haiku{Regels 23-27 zijn.}{verticaal geschreven in}{de linkermarge}\\

\haiku{511HOemen sal maken.}{licht dat altoes dueren}{sal Rood onderstreept}\\

\haiku{imitatie van resp. ().}{r. 39 en 38laatste woord}{niet te traceren}\\

\haiku{332714 Tekstfragment in (,):}{linkermargegebruiksspoor}{deels afgesneden}\\

\subsection{Uit: De historie vanden vier heemskinderen}

\haiku{alle sijn goet   .}{dat hi hem of sijn vrienden}{genomen had}\\

\haiku{Daer wort ont-  .}{boden voer den coninc Alaert}{ende Maeldegijs}\\

\haiku{Ende als vrou Aye,}{vernam dat   Aymijn quam}{ginc hem te gemoet}\\

\haiku{Aldus had Aymijn.}{iiii kin-  der dat}{hi niet en wist}\\

\haiku{Mer gi moet met mi.}{te hove eer dat gi}{op heidenis vaert}\\

\haiku{Het waer scande wiet,.}{hoerde of sage heren}{of   joncfrouwen}\\

\haiku{hare ban-  .}{nieren ontwonden so si}{eerlicste conden}\\

\haiku{Ic sal proeven in.}{corter tijt of Reynout mijn}{maech is of   niet}\\

\haiku{{\textquoteleft}Neve, gevet mi,;}{dat   paert daer gi op}{sidt men prijstet seer}\\

\haiku{Doe quam clacht voir den}{coninc dat sijn coc was doot}{gesla-  gen.}\\

\haiku{Als de   coninc,:}{dit hoerde seide hi met}{toernigen moede}\\

\haiku{Het is een schoon goet,.}{ghi moget u staet   daer}{eerlic op dragen}\\

\haiku{{\textquoteleft}Heer coninc, ic en.}{spele om so cos-}{teliken pant niet}\\

\haiku{Ic scaec u ende,{\textquoteright}}{mat u met enen rock seide}{Adelaert ende313}\\

\haiku{Ic seg u, de u,.}{gaf desen raet dat hem u}{leven verdroet}\\

\haiku{Aldus reden si,.}{den coninc tegen dair hi}{mit sijn volc quam}\\

\haiku{{\textquoteleft}Heer coninc, hier is,,.}{Bartram   mijn sone}{ic heb hem seer lief}\\

\haiku{Of hy tegen u,?}{yet mesdede soude}{ic dat ontgelden}\\

\haiku{Si gaven den   ,.}{coninc haren scat dat hijt}{bewaren soude}\\

\haiku{U dochter neme.}{ic gaerne ende die}{roetse   mede}\\

\haiku{lant ende grote,.}{eere gedaen daer hi seer}{toernich om was}\\

\haiku{Ende466  sijn si,.}{doot of verdaen God wil haer}{sielen ontfermen}\\

\haiku{hi   is een die.}{starcste vander werelt}{ende den koensten}\\

\haiku{Hi sal ons geven.}{scats genoech ende maken}{ons rijke luden}\\

\haiku{{\textquoteright} Roelant ghinc in de:}{sale totten joncfrouwen}{ende seide}\\

\haiku{{\textquoteleft}Myn rouwe de577  ,.}{ic te hants int herte heb}{is onseggelic}\\

\haiku{{\textquoteright} Mit dat Maeldegijs,,:}{de   woerden sprac bleef Reinout}{staende seggende}\\

\haiku{{\textquoteleft}Doet dese598  heuc,.}{over u hernas an dat ment}{hernas niet en sie}\\

\haiku{xx pont, ende doe,}{ic in dit bosch quam so quam}{mi te gemoet}\\

\haiku{{\textquoteleft}Si leven noch, mer,}{si leg-  gen in}{groot verdriet ende}\\

\haiku{Doe verstont   Reinout.}{wel dattet Maeldeghijs}{om tbeste dede}\\

\haiku{Nochtans doch-616  .}{tet Reinout vreemde dat zyn oem}{dese woirden sprac}\\

\haiku{De cop was oec so.}{groot datmens niet veel so groot}{gesien en had}\\

\haiku{Ende dair is noch,,.}{een hiet Maeldegijs ende}{doet toveri}\\

\haiku{{\textquoteleft}God loens u,   edel,.}{heer coninc ic en mach hier}{niet langer letten}\\

\haiku{{\textquoteleft}Heer coninc, des en,}{acht ic niet mer gi moget}{u wel scamen}\\

\haiku{{\textquoteleft}Wije sal de gene?}{wesen die   dese drie}{heren hangen sal}\\

\haiku{Ende als hij den,:}{coninc   ghegruet had}{seyde de bode}\\

\haiku{{\textquoteleft}Roelant neve, siet,.}{ginder Ogier met sijn volc al}{scone mannen}\\

\haiku{Is dit wair,{\textquoteright} seide, {\textquoteleft}.}{Reinoutso wil   ic selve}{tot Parijs varen}\\

\haiku{{\textquoteleft}Ic rade u wel.}{dat ghijer vaert   ende}{u broeders mede}\\

\haiku{{\textquoteright} Reinout antwoerde met:}{soeten woerden in Bertaens}{ende   seide}\\

\haiku{{\textquoteright} {\textquoteleft}Ya ic, oem,{\textquoteright} seide, {\textquoteleft},.}{Reinoutdanc heb God ende}{ghi oem Maeldegijs}\\

\haiku{Ic bid u, raet uwen;}{neve dat hi mijn crone}{wedergeve}\\

\haiku{u leveren mijn}{swager Reinout met sijn broeders.{\textquoteright}828}{Coninc Karel}\\

\haiku{{\textquoteright} Dus reden si soe.}{lange dat si quamen bi}{Mon-  talbaen}\\

\haiku{Coninc Yewijn is.}{tot u gecomen om}{te sien wat gi doet}\\

\haiku{Ic bid864  Gode.}{dat hi ons beware voer}{scade of scande}\\

\haiku{Ende hi sloech mit.}{alle sijn volc crachtelic}{op die heren}\\

\haiku{Ende885  eer ic,.}{op comen conde was mi}{mijn swaert benomen}\\

\haiku{was met sijn volc, ghinc}{Maeldegijs ende blies den}{horen ende}\\

\haiku{bidden an Reinout dat.}{hi haer ver-  gave}{sinen evelen moet}\\

\haiku{{\textquoteright} Mettien trat937  .}{Gontier voert ende ontfinc}{den hantscoe van Ogier}\\

\haiku{ende beleiden.}{dat cloester Beurepaer}{sterckelijc}\\

\haiku{Als Rei-  nout}{van sijn vrou dese woerden}{hoerde ende haer}\\

\haiku{Want ist dat ghien   ,.}{doden wilt ic sallen u}{mit crachte nemen}\\

\haiku{metten swaerde,.}{of ten si dat gi mi met}{Beyert971  ontvliet}\\

\haiku{{\textquoteleft}Roelant neve, gi,:}{sijt ijmmer mijn bloet ic986}{bid u vriendelic}\\

\haiku{wout u ghelieven}{dat gi mi helpen wout in}{mijnre   eeren}\\

\haiku{{\textquoteleft}Vaert met mi, neve,.}{Ridsaert want   gi en moget}{u niet verweren}\\

\haiku{{\textquoteleft}Wie sal so coe-?}{nen man wesen die mijn}{broeder hangen sel}\\

\haiku{{\textquoteleft}Ripe, nu doet met,.}{mij dat u   gelieft het}{vergae mi soet mach}\\

\haiku{Mit dat Ridsairt dit,:}{sach ver-  blijde hi}{hem ende seide}\\

\haiku{{\textquoteleft}Gi heren,   ic.}{wil u Gode bevelen}{ende sijn Moeder}\\

\haiku{{\textquoteright} Met dese woer-.}{den scheiden de heren}{van malcander}\\

\haiku{{\textquoteleft}Edel here coninc,}{wildi   noch die soene}{ontfaen die u boet}\\

\haiku{Ic sal u goede.}{borge setten dat   ic}{sal bliven gevaen}\\

\haiku{Doe ghinc Maeldegijs:}{voer tconincx bedde}{staen ende seide}\\

\haiku{Als hi dit gedaen,}{had ontboet hi heimelic}{een snijer ende}\\

\haiku{ic bid u, lieve,.}{here dat gi Gode}{voer mi wilt bidden}\\

\haiku{dat hi geploct had.}{ende dranc water daer}{toe uuter fonteyne}\\

\haiku{Daer1124*~          vandt hi.}{scepinge ende voer int}{lant van Slavonien}\\

\haiku{of si die stede.}{tegen die kersten houden}{wouden of niet}\\

\haiku{hem nyemaer dat die.}{Turcken Jherusalem}{gewonnen hadden}\\

\haiku{{\textquoteright} Desen raet dochte}{den ker-  stenen goet}{ende deelden hair}\\

\haiku{Aldus wonnen die.}{kersten in Jherusalem met1160}{Gods hulpe}\\

\haiku{hi wilde selven}{voer hem allen gevangen}{bliven ende}\\

\haiku{Aldus geleyde.}{men Reinout met gro-1162  ter}{eren te scepe}\\

\haiku{Ende dese   :}{was veel tijt bijden heren}{ende hi seide}\\

\haiku{{\textquoteleft}Heer coninc, wi doen}{maken een kerc ende}{u neve Reinout quam}\\

\haiku{Het woord {\textquoteleft}vertaling{\textquoteright}:}{is in verband met de Reinolt}{eigenlijk misplaatst}\\

\haiku{En tot slot hebben;}{we de Historie vanden}{vier Heemskinderen}\\

\haiku{We zullen hier van.}{doen hebben met een merk van}{de opdrachtgever}\\

\haiku{Ook Maeldegijs zelf.}{belandt op een kwade dag}{in Karels kerker}\\

\haiku{Beyaert trapt de stenen,.}{kapot zwemt naar de oever}{en loopt naar Reinout toe}\\

\haiku{Deze laatste vecht.}{tegen de heidenen in}{het Heilige Land}\\

\haiku{Dankzij Maeldegijs.}{kan Yewijn zich bijtijds uit}{de voeten maken}\\

\haiku{Vooral de drieste,,.}{Reinout die een aardje naar zijn}{vaartje heeft bevalt hem}\\

\haiku{De vorm van deze.}{teksten hield nauw verband met}{de overdracht ervan}\\

\haiku{In de vijftiende:}{eeuw vond een belangrijke}{verandering plaats}\\

\haiku{Goods ghevonden is;}{en Een suverlijck crans}{van dusent rosen}\\

\haiku{vier Heemskinderen.}{kon hij zich van inkomsten}{verzekerd weten}\\

\haiku{Achterin is de:}{oude approbatie uit}{1619 opgenomen}\\

\haiku{Bij al zijn acties.}{maakt hij handig gebruik van}{zijn mensenkennis}\\

\haiku{Op afbeeldingen.}{is hij te zien met zijn hoofd}{in zijn handen}\\

\haiku{Deze zalfde de,.}{koning gebruikmakend van}{de heilige olie}\\

\haiku{Waarschijnlijk is er.}{al in de tiende eeuw een}{kasteel gebouwd}\\

\haiku{14105alst wel blikelic....}{was an Elegast dattet}{niet en geschiede}\\

\haiku{Na Jezus' dood nam (-).}{hij Maria bij zich in huis}{Johannes 19:2527}\\

\haiku{14545dat Onse God....}{in Bethleem geboren was}{oetmoedich wesen}\\

\haiku{Ter herinnering.}{hieraan is de strijdkreet van}{de Fransen Monjoie}\\

\haiku{De verwijzing naar.}{Napels komt hier min of meer}{uit de lucht vallen}\\

\haiku{De taal van K is.}{een mengsel van Nederlands}{en Ripuarisch}\\

\haiku{Amerijn van Nerboen.}{\ensuremath{<} Die een is geheten}{aymijn van nerboen}\\

\haiku{gestelt - 57 van zijn.}{volcke verloren \ensuremath{<}}{van zijn volcke}\\

\haiku{B. Besamusca, (),!}{F. Brandsma en D. van der}{Poelred. Hoort wonder}\\

\haiku{der uitgave van],.}{1878 Jacob van Maerlant}{Spiegel historiael}\\

\haiku{Tijdschrift voor boek- (),-.}{en bibliotheekwezen}{51907 p. 135}\\

\haiku{3 dln. Overdiep, G.S. (),.}{ed. De Historie van den}{vier Heemskinderen}\\

\haiku{M. Tersteeg en P.E.L. (),.}{Verkuylred. Ic ga daer ic}{hebbe te doene}\\

\haiku{B. Besamusca, (),!}{F. Brandsma en D. van der}{Poelred. Hoort wonder}\\

\haiku{Zo komt, naast hielt, de,.}{vorm helt een paar maal voor en}{naast sal de vorm sel}\\

\haiku{zo luisterrijk als:}{maar mogelijk was 221yegen}{trecken den genen}\\

\haiku{verbranden omdat:}{zij de moeder was van hem}{die 337dorperheit}\\

\haiku{prachtig straalden en...:}{schitterden 635hem ende zijn}{geselle sopten}\\

\haiku{zal u binnenkort:}{een plaats in een rijk klooster}{geven 689ducaten}\\

\haiku{voegde graaf Amerijn,,;}{de zoon van Aernout van}{Benlant zich bij hem}\\

\haiku{zodat ze daarna:}{nooit meer om soldij zullen}{vragen 921secreet}\\

\haiku{omdat hij erop}{vertrouwde dat Maeldegijs}{te hulp zou komen}\\

\subsection{Uit: Kroniek van het Sint-Elisabethsconvent te Huissen (1667-1752 en 1782-1801)}

\haiku{die weijgerde, '.}{het selve soo dat die in}{t klooster bleef}\\

\haiku{Gelijck men wel}{dencken kan wat droefheijt}{en  benauwtheijt}\\

\haiku{Op dien tijt quam ook,.}{Heer Petrus Codde en de}{neef van Sijn Hooghw}\\

\haiku{En syn Hooghwde.}{Petrus Codde was eenigen}{dagen voor haar Eerw}\\

\haiku{En hebben in 't.}{begin het Iubile\`e niet}{afgekondight}\\

\haiku{Hr van Beest wel mocht.}{op den dagh der professie}{gepreedickt hebbe}\\

\haiku{In dit jaer 1718 heeft}{het convent gekoft het land}{genaemt De Lange}\\

\haiku{En elk heeft in de}{volgende jaar eenigh gelt daer}{voor ontfangen}\\

\haiku{mater Francisca,,.}{Staats ons niet tegen geweest}{maer geholpen heeft}\\

\haiku{mater was, diende.}{voor diaken en deede een}{treffelyk sermoen}\\

\haiku{En ook zijn twee van '.}{onse kneghten int}{bouwhuis gestorven}\\

\haiku{De rivieren zijn.}{toegevroren en met dick}{ys beset geweest}\\

\haiku{En aen ons convent.}{meede door den heer richter}{Buno geinsinieert}\\

\haiku{Ook is dit iaer een.}{nieuwe galerye van houd in}{patershof geset}\\

\haiku{De redens die het}{convent bewogen hebben}{om te verkoopen}\\

\haiku{goethartig en ruym,'.}{toegevende genoeg soo}{dat ten waer d Eerw}\\

\haiku{Waerom sy ook.}{hebben aengehouden en}{endtlyck syn Eerw}\\

\haiku{En in de stadt den,.}{geheele nagt gelogeert}{met schrijven overbragt}\\

\haiku{De verkiesing.}{gong dan voort op zijn tijdt op}{den 13 november}\\

\haiku{Gong in de saal en,,.}{vorder in de gang waar sij}{hoorde dat glas viel}\\

\haiku{dat men vreesde hij.}{de duur aan stukke soude}{geslaagen hebben}\\

\haiku{t Waar al te lang,.}{ales te melden dat in die}{tijt is omgegaan}\\

\subsection{Uit: De kroniek van het St. Geertruiklooster te 's-Hertogenbosch}

\haiku{Later heeft iemand.}{anders dit gedeelte van}{de tekst doorgehaald}\\

\haiku{b. G.C.M. van Dijck,,-, ().}{De Bossche Optimaten}{13181973Tilburg 1973}\\

\haiku{men sijt, dat dese;}{woninge is geweest het}{hoff van den hertogh}\\

\haiku{den oirlog tussen;}{vrouw Johanna ende den}{hertoch van Gelder}\\

\haiku{ende den hertog;}{vervolgde sijnen soone}{seer om te vangen}\\

\haiku{daermede die;}{neringe uijt onse stadt}{seer getogen wordt}\\

\haiku{sag men hier ten Bosch,}{een vreeslijcke comeet}{ende daernaer}\\

\haiku{daer waeren oock,;}{jonge susterkens dat noch}{clijn kinder waeren}\\

\haiku{ende daer bleeff van.}{de partijen niemant doodt}{dan den hoogschout}\\

\haiku{herwaerts, waernaer;}{hij de geestelijcke}{seer swaer heeft geschat}\\

\haiku{ende is des 26;}{mert van de quetsuren tot}{Brugge overleden}\\

\haiku{de Spanssche pocken,.}{die men noemde Sint Jobspassie}{ofte siecte}\\

\haiku{Eodem anno is.}{gefondeert het Rabauwhuijs}{tot Oisterwijck}\\

\haiku{{\textquoteleft}Mijn susterken, mijn.}{heer vader en sall u int}{clooster niet laeten}\\

\haiku{ende quaemen op.}{den Ham ende namen het}{huijs te Kessel in}\\

\haiku{branden insgelijxs;}{deselve dorpen met de}{kerck tot Geldrop}\\

\haiku{vinde hij daer geen,{\textquoteright}.}{gebreck soo comt dan en segt}{wat mijn leth}\\

\haiku{ende daervan.}{is collator den prior}{vant Fraterhuijs}\\

\haiku{In den jaere 1516.}{regneerde hier een siecte}{genaemt plerensis}\\

\haiku{Het was een specie,.}{van de pest waervan veele}{menschen storven}\\

\haiku{Eodem anno is}{de stadtsmuer gelijt van den}{Vuchterendijck}\\

\haiku{item den coster een;}{stuijver en een schoon broot van}{een halve stuijver}\\

\haiku{Item dijlt men in de.}{vasten broot ende een groote}{quantitijt haring}\\

\haiku{ende op dese;}{dagvaert wiert voor vrouw Magriet}{versocht een bede}\\

\haiku{ende daervan.}{is collator den naesten}{van den bloeden}\\

\haiku{nochtans raede haer,}{sommige schepenen dat}{sij maer geloven}\\

\haiku{het ging een tijt lang;}{in dese stadt ofte in}{een slach was geweest}\\

\haiku{Eodem dito de;}{gemijnte dit hoorende sijn}{seer ontstelt geweest}\\

\haiku{den gulden van de,;}{colveniers die men segt van}{Sint Cristophorus}\\

\haiku{de Hagenaers niet;}{beter wetende als dat}{sij vrinden waeren}\\

\haiku{wert priester ende.}{wederom rector ende}{beneficiaet}\\

\haiku{tegen dien tijt wiert;}{een balie gemact op}{den Wintmolenberg}\\

\haiku{{\textquoteleft}Godt almachtig, wilt{\textquoteright};}{mijn niet meer opleggen als}{ick verdragen magh}\\

\haiku{ende doen hebben;}{se hem opt schavot noch den}{cop afgehouwen}\\

\haiku{ende doen troc den;}{prins met sijn volck over de}{Locht naer Antwerpen}\\

\haiku{ende bleeff daer twee.}{dagen liggen sonder iet}{uijt te rigten}\\

\haiku{Item ider mergen is,.}{600 roijen groot de roeij}{als boven 14 voet}\\

\haiku{hetwelck De Kock.}{wel vijftienhondert gulden}{gekost hadde}\\

\haiku{Eodem anno was.}{binnen Den Bos groot gebreck}{van folio 136v brant}\\

\haiku{ende doen volgde:}{noch een wagen daer opsaet}{den soth seggende}\\

\haiku{op eenen niuwen raet}{van twalf personen die men}{noemde den Bloet Raet}\\

\haiku{onder haer armen.}{dicke goude ketenen}{hadde hangen}\\

\haiku{naer die galge die;}{op de Merct stont ende hem}{daer af gehangen}\\

\haiku{een teecken soude}{geven van sijn ontschult voor}{het volck ende}\\

\haiku{ende met dit hoog;}{water dreeff aff de lange}{brugge tot Orten}\\

\haiku{ende daervan.}{is collator een van sijn}{naeste vrinden}\\

\haiku{In den eersten de}{maete van de hooftstadt van}{s'Hertogen Bos}\\

\haiku{ende sij hebben}{gereetschap gemact om}{haer te pijnigen}\\

\haiku{Eodem anno den}{10 januarij begon}{het te doijen}\\

\haiku{ende sijn dien avont;}{gijselaers gestelt om te}{parlementeren}\\

\haiku{de aenslach met.}{haer misluct was waeren niet}{wel te vreden}\\

\haiku{gestelt worden opt.}{feijt ende exercitie}{van de religie}\\

\haiku{Int selve jaer den.}{20 januarij heeft den}{prince folio 232r}\\

\haiku{wordt alle jaer op}{den eersten julij bij den}{bisschop van Den Bos}\\

\haiku{Godt die Heere sij.}{met u. Uijt Mastrigt den 23}{februari 1580}\\

\haiku{ende daer van is.}{collator den deecken van}{Hilvarenbeeck}\\

\haiku{de soldaten en;}{Bobadilla dit siende}{schieten daer op toe}\\

\haiku{dit dessijn ende.}{gevolg van dien is geweest}{op dese manier}\\

\haiku{ende sij moste;}{alle gebogen sitten}{ende niet recht op}\\

\haiku{den prins dit hoorende,;}{soo heeft hij de aenvangers}{laeten aftrecken}\\

\haiku{gesteecken, soo dat;}{de kerck lichter flamme}{begon te branden}\\

\haiku{want den cardinael.}{hadde groote devotie tot}{dien hijligen}\\

\haiku{Eodem anno op.}{den 8 februarij sijn}{opt hoog- blz. 22}\\

\haiku{de welcke den;}{winter sneuw ende vorst haer}{op het lijff gestiert heeft}\\

\haiku{ofte iet anders;}{aen te tasten tot naedeell}{van dese landen}\\

\haiku{Den 3 dito is.}{den hertog van Brabant tot}{Weerdt gecomen}\\

\haiku{Eodem anno den}{27 mert sanderendaegs naer}{Paesschen isser}\\

\haiku{Eodem anno is.}{de corps du grade aen de}{Vuchterpoort gemackt}\\

\haiku{als hij een brigaert}{vaendragers int gesigt creeg}{van de welcke}\\

\haiku{Eodem dito is.}{een paert geruijlt tegen}{een hoendereije}\\

\haiku{het welck raer om,.}{te sien was want waeren geenen}{arbijt gewoon}\\

\haiku{toe oock ander.}{consent ofte paspoort te}{derven verwerven}\\

\haiku{Gedaen int leger ';}{voors-Hertogenbosch den}{14 september 1629}\\

\haiku{ende het licham.}{op een rath geseth op de}{Vuchterhijde}\\

\haiku{ende daegs daer aen.}{dede hij de revu over}{het garnisoen}\\

\haiku{In den jaere 1685,.}{is alhier een schoon niuw}{vischmerckt gemackt}\\

\haiku{ende achter een.}{niuwe galderije met een}{blauw stene trap gelijt}\\

\haiku{want sij hadden in;}{Vrancrijck in de stadt Aras}{gevangen geseten}\\

\haiku{{\textquoteleft}Ende stont als een.}{bepiste paep tot spoth}{van alle menschen}\\

\haiku{ende sommige.}{grutters van vooren vant stadthuijs}{comende blz. 461}\\

\haiku{\ensuremath{<}de tekst van dit}{paskwil ontbreekt\ensuremath{>} ~ Den}{4 october sijn}\\

\haiku{A.R.M. Mommers, Brabant (;}{van Generaliteitsland}{tot gewestdl. II}\\

\subsection{Uit: Medische en technische Middelnederlandse recepten}

\haiku{Priebsch, R., Deutsche (,).}{Handschriften in Engeland}{Erlangen 1901}\\

\haiku{naast het vergulden (), {\textquoteleft} ...}{van metalen524~		 of}{het maken vanliesten}\\

\haiku{er is een brede.}{marge aan alle zijden}{van het perkament}\\

\haiku{Zie ook hieronder:}{bij de bespreking van B.M.}{Ms. Sloane 345}\\

\haiku{O tu corporis ...}{curam gerens per omnia}{auxiliari}\\

\haiku{De verzameling:}{wordt voorafgegaan door de}{Latijnse titel}\\

\haiku{Dit is een reeks van.}{27 vragen en antwoorden}{in het Latijn}\\

\haiku{algemeen 298~		,,,.}{301~		 van een ziek dier 299}{verse 300}\\

\haiku{Men kan er eveneens {\textquoteleft}{\textquoteright} ().}{diaquillon maturativum}{op smeren1194}\\

\haiku{Men wrijft de vorm in ().}{met kalkwater opdat hij}{hard worde596}\\

\haiku{is de persoon in, ().}{kwestie ziek dan schuimt de olie}{anders niet14}\\

\haiku{is de persoon ziek,.}{dan zal het bloed in het ei}{branden anders niet}\\

\haiku{de ene made eet.}{de andere op tot er}{nog maar een overblijft}\\

\haiku{Ende van desen,}{vorseiden cruden neemt een}{hant vulle ende}\\

\haiku{es hi besiect, die.}{olie sal scumen gelyc dat}{een beer doet die uecht}\\

\haiku{Scorse van bongi}{ghepuluert droegt quade}{gate in veden}\\

\haiku{te puluere}{neemt elx euen vele bi}{ghewichte ende}\\

\haiku{daghen ende doet,.}{inde oghen dickent hi}{sal ghenesen}\\

\haiku{de mag\ensuremath{<}h\ensuremath{>}[],}{e ende verduwet wel}{ende dien therte}\\

\haiku{Dit is van elken}{crude sonderlinke dat}{men doet in wonden}\\

\haiku{sincse, hi sel,.}{steruen ende vlietse}{hi sal nesen}\\

\haiku{*~         {\textparagraph} Alrehande:}{melc is van couder ende124}{droghe natuer}\\

\haiku{Die gaghel is cout.}{inden eersten graet ende}{droghe inden .ij}\\

\haiku{{\textparagraph} Men sel nemen oly}{die van rosen is ghemaect}{ende saluen}\\

\haiku{*~         {\textparagraph} Nem agrimonie:}{ende wriifse met gheten}{melc op die stede}\\

\haiku{nem die blade van.}{agrimonie ende legghes}{op die wonde}\\

\haiku{ende leggent myt.}{den pynseel hoe groet of hoe}{cleyn dat ghy wylt}\\

\haiku{zuarte of ander,.}{alre hande verue so}{salmen nemen .ij}\\

\haiku{een luttel ende,.}{dan so settet af ende}{dan ys bereyt}\\

\haiku{*~         Recipe eyn}{steen daer wisselars gout op}{pruuen ende wrift dar}\\

\haiku{Ende doet dat feel,.}{daer yn ende laetse een}{dach dar yn liggen}\\

\haiku{Ende dan suldi (.}{se weel  wtreecken ende}{makent dan alstfol}\\

\haiku{Dan neem[t] dat vel.}{ende naijet toe ende laet}{een gat an die .ij}\\

\haiku{loet brisilien houts.}{ende is brisilien seer}{goet so neemt bi .iii}\\

\haiku{veel schoenmakers219 zwart.}{als des voirseit is ende}{rorent met een stock}\\

\haiku{Ende laet dat te.}{gader sieden met water}{ende vrine ana231}\\

\haiku{of also lange.}{dat dat fermilioen so}{swart sy als een coel}\\

\haiku{*~         Aldus salmen.}{tegen molken toueren}{remedium doen}\\

\haiku{Noch zinziber aen.}{stucken gesneden ende}{schoen ghemact .vij}\\

\haiku{Meen maeght ock gieten,.}{in wormen ende maken}{daer af watmen wil}\\

\haiku{{\textparagraph} Mede machmen dy[].}{rose weruen myt wat}{werue datmen wil}\\

\haiku{laten staen tot dat,}{dy honich doer hoer lyef}{gethogen is ende}\\

\haiku{Ende yn vele,.}{boke hetet aqua vite dat}{is twerstaen des leuens}\\

\haiku{Dyt water ys seer.}{laxatiff ende nuchteren}{gedroncken .xiij}\\

\haiku{Ende alsmen dy,.}{bladeren ontwee trect soe}{yrst al haer achtich}\\

\haiku{Ende genest noli.}{me tangere eyn plaester}{daer af op geleit}\\

\haiku{*~         Recipe dy}{blommen vander kleyner}{matelyuen ende}\\

\haiku{Ende doet scheiden.}{grote groue winden ende}{maket guden adem}\\

\haiku{alle wonden nye}{ende olde ende doet}{oeck alle saken}\\

\haiku{hebstu dy oghe,.}{appel behalden du salts}{gesont werden}\\

\haiku{hase noten dat;}{syet yn een pan ende}{latet enen nacht staen}\\

\haiku{Nemt agremonia,,,,}{bugghel roet coel mede dat}{doet yn dry pynten}\\

\haiku{die melck drinct sauents,.}{ende dat water morgens}{law dyt doit .xviij}\\

\haiku{{\textparagraph} Tegen dy koeck dy.}{van den euel bleuen is}{onder dy ribben}\\

\haiku{Tegen dy swaerte.}{blomme dy nyt genoch}{gesworen is}\\

\haiku{Nemt salue dy wan}{sauelboem gemact is}{ende salft dy daer}\\

\haiku{*~         Recipe olie.}{van comillen ende olie}{van dille elx .vj}\\

\haiku{desen dranc is tot.}{gesteken wonden dy nyt}{willen etteren}\\

\haiku{Ende dan salmer.}{mede werken in deser}{manyren daert .j}\\

\haiku{*~         spumidusIs,}{dat bloet schumende so is}{di sucht in dy borst}\\

\haiku{Ook met De Vr. 301,.}{doch hier wordt nog toegevoegd}{dat de wijnruit .ix}\\

\haiku{Er is zogoed als:}{volledige overeenkomst}{met De Vr. 370}\\

\haiku{Hetzelfde middel.}{komt ook voor in 863~		 en}{in MGR 1003 en 1175}\\

\haiku{wilts lijnzaets voor wit.}{linzaets en scaperroet voor}{hamelin roete}\\

\haiku{Een gedeelte van,.}{dit nummer komt ook voor in}{MGR 1057 1257 en 1258}\\

\haiku{Een gedeelte van.}{dit nummer vindt men eveneens}{terug in MGR 1262}\\

\haiku{51{\textquoteright}, Niederdeutsche (),-,-.}{Mitteilungen X1954 5}{22 vooral 810}\\

\haiku{Er is een grote,,.}{gelijkenis met 336}{441~		 en MGR 1231}\\

\haiku{deze gegevens.}{vindt men daar terug in het}{volgende nummer}\\

\haiku{G. Keil, R. Rudolf, (.),.}{etc.eds Fachliteratur}{des Mittelalters}\\

\haiku{Er is grotere.}{overeenkomst met 223~		 en}{667~		-668}\\

\haiku{Men vgl. MGR 930, de,,.}{tweede helft van MGR 954 1353}{De Vr. 57 en 346}\\

\haiku{Nauw verwant is dit.}{recept met het voorlaatste}{middel uit MGR 876}\\

\haiku{laet coelen, dan maect,,;}{tappe vinghers lanc ende steec}{in hu fondament}\\

\haiku{De bedoeling van.}{dit merkwaardig recept is}{niet zeer duidelijk}\\

\haiku{Item is dat bloet hert,.}{ende zwart dat bloet is to}{langhe geholden}\\

\haiku{738 ypericon, 669,,,.}{724 yrindina 749 ysopy}{690 yuorie}\\

\haiku{688, 689 Lat. huislook, ().}{donderbladSempervivum}{tectorum L.}\\

\haiku{197 grote - ende.}{middele ende clene}{zie confeli}\\

\haiku{1) 4, 38, 63, 240,,,,,,,,.}{241 253 315 327 341 344 347}{355 enz. zie dic}\\

\haiku{689 de geboorte.}{van Maria wordt gevierd op}{8 september}\\

\haiku{5 andoorn, witte ().}{malroveMarrubium}{vulgave L.}\\

\haiku{is eyn krut ende.}{vasset op hogen steden}{zie hierboven}\\

\haiku{Voor de bereiding,.}{en de eigenschappen zie}{Grabadin 286}\\

\haiku{Ende alsmen dy,{\textquoteright}.}{bladeren ontwee trect soe}{yst al haer achtich}\\

\haiku{704 cleyne - wolfsmelk,.}{een van de zeer talrijke}{Euphorbiasoorten}\\

\haiku{261, 262, 263, 266, 267,, ().}{268 weit tarweTriticum}{vulgare L.}\\

\haiku{6R. Forbes, Studies (,).}{in Ancient Technology}{Leiden 1971 Vol. III}\\

\haiku{r verbeterd uit.}{t. 218Na verwen staat van}{bresilien doorstreept}\\

\haiku{282v verbeterd uit.}{w. 283Na saluie volgt aqua}{per omnia doorstreept}\\

\haiku{v verbeterd uit.}{w en u uit w. 300Na}{gut volgt in het Hs}\\

\subsection{Uit: Memorye}

\haiku{H.J. ende was een}{hennepklopper ende den}{anderen ghenaemt}\\

\haiku{Maer mijn heren die.}{schepens en wilden hem dat}{niet concenteren}\\

\subsection{Uit: Middelnederlandse geneeskundige recepten}

\haiku{Willems, J.F., {\textquoteleft}Brokken{\textquoteright},.}{uit een oud Geneesboek der}{XIVe eeuw Belg. Mus}\\

\haiku{in het papieren;}{gedeelte vari\"erend}{van 143/195 {\texttimes} 200/243 mm}\\

\haiku{in het bezit van.}{een medicijnmeester die}{Jan van Utrecht heette}\\

\haiku{nie in nye was (814), () (), ().}{mit739~		 en mer740}{vuel745}\\

\haiku{De samenstelling.}{der katernen is niet meer}{te achterhalen}\\

\haiku{Een {\textquoteleft}stritorie{\textquoteright} maakt ().}{men met gemalen wierook}{en eiwit233}\\

\haiku{as van look, atrament () {\textquoteleft}{\textquoteright} ().}{en bonen568~		 of wit}{honds quade575}\\

\haiku{is het gezwel in, ();}{de lies dan binde men het}{boven de voet278}\\

\haiku{Tegen {\textquoteleft}harteevel{\textquoteright} {\textquoteleft}{\textquoteright} ().}{drinke menpollioen met}{honig in wijn496}\\

\haiku{Gezwellen geneest ();}{men door te gorgelen met}{hete wijn1113}\\

\haiku{Om {\textquoteleft}veel naturen{\textquoteright} {\textquoteleft}{\textquoteright} ();}{te hebben drinke men sap}{vankamillen180}\\

\haiku{zaad van {\textquoteleft}jusquiami{\textquoteright},, {\textquoteleft}{\textquoteright} {\textquoteleft}{\textquoteright} ();}{atrament zaad vancaerden}{en vandolke511}\\

\haiku{Een beknopte leer.}{van de urinoscopie wordt}{gegeven in 1112}\\

\haiku{Als {\textquoteleft}vreemd{\textquoteright} voorwerp kan:}{men ook de beensplinters in}{wonden beschouwen}\\

\haiku{bereiding (353~		) ():}{en gebruik354~		 tot 358}{.aqua petralis}\\

\haiku{bereiding (359~		) ():}{en gebruik360~		 tot 364}{aqua yrundinea}\\

\haiku{Bijzonder talrijk:}{zijn de pleisters die men op}{wonden kan leggen}\\

\haiku{men netelmoes met {\textquoteleft}{\textquoteright} {\textquoteleft}{\textquoteright},:}{zout en pulver vanalune}{ofcalmijn of nog}\\

\haiku{boven de regel, {\textbackslash} /;}{toegevoegd werd staat in de}{uitgave tussen}\\

\haiku{Nemt erdbesie cruud.}{metten wortelen ende}{metten bladeren}\\

\haiku{lepel vol nauonts.}{ende nuchtens tote dat}{ghi sijt ghenesen}\\

\haiku{lepel vol een nacht.}{ende smorghens suldi sien}{dien steen ghesmolten204}\\

\haiku{ende doet dicke,.}{in die noselocken het}{sal betren}\\

\haiku{hem hebben goeden.}{lust tetene ende te}{drinckene}\\

\haiku{Jeghen menisoen.}{van bloede ende ieghen}{alle menisoen}\\

\haiku{Omme yser oft houdt,.}{scicht oft doren dat es int}{tlijf wt te doene}\\

\haiku{bosse ende als,.}{ghijt doet vp die wonde soe}{legghet vp vlasse}\\

\haiku{mortier ende doet}{in een scotele238 ende}{doet vte die239 polen}\\

\haiku{ende wriuet wel,.}{cleene ende doet in bossen}{dats poplioen}\\

\haiku{{\textparagraph} Sied werc ende soudt.}{in een lettel aysins ende}{bindet vp tled}\\

\haiku{mortier ende dan [.}{salmene temperen in}{sterken wittenfol}\\

\haiku{linen cleet ende,.}{drinckent nuchteren ghi}{sult ghenesen}\\

\haiku{pater noster mach,.}{segghen dan latet coelen}{ende dan neemt .j}\\

\haiku{Ende die pusoen,.}{nemt vp die daghe hi sal}{steruen voer .xv}\\

\haiku{daghe lanc elcs,.}{daghes sperma rute hi}{worde binnen .ix}\\

\haiku{Oec es goet tote.}{desen drancke die crop}{vanden biuoete}\\

\haiku{Looc ghesoden in.}{fonteyne verdrijft smerte}{ende ghezwel}\\

\haiku{die in frenesien,,.}{sijn drincke {\textbackslash}men/368 salre}{bi genesen}\\

\haiku{Ende dan salment.}{nemen ende netten int}{witte vanden eye}\\

\haiku{Nem zaet van genste,,.}{ende tempert met wine}{ende ghef hem .j}\\

\haiku{Nem papple, ende,.}{steenvaren ende nutte}{dat te samen}\\

\haiku{*~         Stoete die gherste,.}{alsoe langhe dat die dop}{aff gaet dan nem .iij}\\

\haiku{Item j aer die comt.}{tuschen die vinger ende}{is j lastich seer}\\

\haiku{j vir, Ende als,}{sy corten so corten sy}{alle drie weeken}\\

\haiku{77r] Nem eltena.}{ende roden coel ende}{cipres appel}\\

\haiku{Item bakelaer daer,.}{maectmen of oly van beyen}{senwer oly ist}\\

\haiku{te gader also [.}{want dicke wordt ende dat}{leg{\textbackslash}g/he op eenfol}\\

\haiku{eest van hitten of,.}{van walghene hi gheneest}{of van bloede}\\

\haiku{ende bestrijct de[][.}{gate altoes in tuusche}{n deken]668}\\

\haiku{*~         Item coppen spin,.}{gheleit op enen wonde die}{en svellet niet}\\

\haiku{ende teghen wint.}{ende teghen steecten die}{ter herten gaen}\\

\haiku{*~         Item sali mit.}{water ghesoden is goet}{teghen die borst}\\

\haiku{Item clare vrine,.}{in den vate mit siechten}{die is dodelic}\\

\haiku{Salmen nuchteren, [].}{eten ende daer goeden wijn}{opdrincken}\\

\haiku{*~         Sijn cout in den;}{eersten grade ende nat}{inden anderen}\\

\haiku{Is couder ende;}{natter van naturen dan}{die ghesouten is}\\

\haiku{si verdriuet die, [.}{bose winde des menschen}{is goet ghetenfol}\\

\haiku{hi laten in die.}{luchter hant ende inden}{vorderen voet}\\

\haiku{Naar welk boek in {\textquoteleft}des{\textquoteright},.}{boecs leeringhe verwezen wordt}{is mij niet bekend}\\

\haiku{In deze laatste.}{toepassing is het middel}{duidelijk magisch}\\

\haiku{nem saet van esschen,;}{ende van anise ende}{minghet te gadere}\\

\haiku{Ende werdt yement(),;}{ghequetstt dat hi bloet hi}{salder of sterven}\\

\haiku{Vgl. de passage.}{die door R. Foncke wordt}{aangehaald uit Hs}\\

\haiku{De woorden {\textquoteleft}ieghen{\textquoteright}.}{quaet ghespaente komen in}{de tekst tweemaal voor}\\

\haiku{698.-699 Deze twee.}{nummers maken deel uit van}{hetzelfde geheel}\\

\haiku{De betekenis.}{van rooesel en gestoft is}{mij niet duidelijk}\\

\haiku{Uit de kontekst blijkt.}{in alle geval dat hier}{een plant bedoeld wordt}\\

\haiku{Cleyne stot unde.}{an nese pustet vordrift}{dat blot der nese}\\

\haiku{Dat sap is ock g\r{u}t,.}{an de oghen droft deme}{de oghen ser sint}\\

\haiku{Mit wine soden}{unde dat hovet mede}{gedwagen vordrift}\\

\haiku{Wat tussen haakjes,.}{staat in de bovenstaande}{tekst komt in de Mnl}\\

\haiku{Du scult eme beden.}{dat he sic warme decke}{so wan he svetet}\\

\haiku{so wel he denne,;}{drinken so sint eme alle}{sine aderen open}\\

\haiku{Senep unde etik is,.}{eme bose alle versche}{spise is eme gut}\\

\haiku{De drovich is de,.}{drinke stedes merc sap dat}{maket ene vro}\\

\haiku{und och dat dode,}{kynt mit erer bitterheit}{is se g\r{u}t weder}\\

\haiku{Poleie ende.}{wegebrede de salt du}{hebben gerede}\\

\haiku{174 arguel es,.}{effel van wine droesem}{bezinksel van wijn}\\

\haiku{1335 wortelen van () ().}{bietBeta vulgaris L.}{ofB. cicla L.}\\

\haiku{Een van de vele.}{soorten van de geslachten}{Carduus of Cirsium}\\

\haiku{267, 269, 270, 328, 351,,,,,,,,.}{393 494 566 568 569 570 571}{575 zie festele}\\

\haiku{daz man nicht pald sein, ...{\textquoteright}.}{geleichen mag gefinden daz}{also heilsam sey}\\

\haiku{- roberti 256, 317 ().}{robertskruidGeranium}{robertianum L.}\\

\haiku{603 de bladeren ().}{van mispelboomMespillus}{germanica L.}\\

\haiku{some die plecken,{\textquoteright} (.}{sijn wit some root ende}{som sijn si swertAnt}\\

\haiku{In de Herbarys {\textquoteleft}.}{leest menPetrocilium}{dats petercelle}\\

\haiku{492, 734, 765, 890, 937,.}{stranguria d.i. langzaam en}{pijnlijk urineren}\\

\haiku{839 rood operment, ook.}{wel rood arsenicum of}{sandracha genoemd}\\

\haiku{886, 1293 bloesem van ().}{roggeCentaurea cyanus}{L. of roggemeel}\\

\haiku{Cf. Van Leersum en, (,).}{verder R. Crawford The King's}{EvilOxford 1911}\\

\haiku{2) 878, 887, 889, 893,,,,,,,,,, ().}{910 959 960 992 1056 1062 1087}{1088 1120 1370 kookhet}\\

\haiku{382 uit het verband.}{krijgt men de indruk dat een}{plant bedoeld zou zijn}\\

\haiku{741, 746 de fijt, een.}{pijnlijke verzwering aan}{de vingerwortel}\\

\haiku{866, 872 oogziekte,.}{vliesje op het oog dat doet}{denken aan een vlek}\\

\haiku{68E.C. Van Leersum, De {\textquoteleft}{\textquoteright} (,).}{Cyrurgie van Meester Jan}{YpermanLeiden 1912}\\

\haiku{recepten zijn de:}{volgende uitgaven en}{studies van belang}\\

\haiku{d. Wissenschaften () (),-,,.}{Wien XLII1863 110200 en Fr.}{Wilhelm O.c. no XXV}\\

\haiku{152c en d zijn de.}{1e en 2e kolom van de}{verso-zijde}\\

\haiku{er is een vlek op,.}{deze plaats die het woord zeer}{onduidelijk maakt}\\

\haiku{Sprachforschung LXXXI (1958),).}{38 nota wordt gesteund door}{muusore in de Mnl}\\

\subsection{Uit: De Middelburgsche avanturier. Of het leven van een burger persoon}

\haiku{Toen wy t'huis kwamen,:}{vonden wy onze Ouders}{by elkanderen}\\

\haiku{Zy kregen 'er wel,.}{haast een die omtrent achtien}{jaren bereikte}\\

\haiku{Jannetje, dus wierd,.}{dat Meisje genaamt was eene}{volmaakte schoonheid}\\

\haiku{Dagelyks hield ik,;}{by myne Moeder aan om}{haar te laten gaan}\\

\haiku{Van een volmaakt schoon,;}{en jong Meisje bemind te}{worden is iets groots}\\

\haiku{\^o Onbezonnen,!}{Jeugd hoe driftig loopt gy in}{u eigen bederf}\\

\haiku{zoude zyn, besloot,.}{hy myn  noodlot aan den}{tyd overtelaten}\\

\haiku{dat de vreugd zo groot,.}{binnen Scheepsboord was als op}{de beste kermis}\\

\haiku{het ontbrak my aan,}{geen geld en door middel van}{het zelve konde}\\

\haiku{Osmans vreugde was,.}{overbodig toen hy ons zo}{wel te vreden vond}\\

\haiku{Ik smeet myn pistool,,.}{ter neder greep een Sabel}{en sprong mede over}\\

\haiku{met haar zou men wel.}{een zieke Antonetta}{kunnen ontberen}\\

\haiku{Nu wierden myne,}{oogen eerst geopend en ik}{zag wat godloos stuk}\\

\haiku{In de Kloosters in;}{het tegendeel heerscht de}{overvloed en de pracht}\\

\haiku{maakte kruis op kruis,.}{en zag onophoudelyk}{naar myne voeten}\\

\haiku{{\textquoteleft}Als Moeder lief'er zelf, ';}{spikkel op heeft wil iker}{niet meer van spreken}\\

\haiku{zoude hebben, waar;}{in zy myn verstand tot de}{wolken verhefte}\\

\haiku{\^o Ysselyke,!}{minnenyd hoe deerlyk weet gy}{de mensch te grieven}\\

\haiku{zeker gy zult hem,,.}{trouwen of ondervinden}{dat ik Landryk ben}\\

\haiku{Niet te hoog, riep zy,,.}{in toorn ontstoken niet te}{hoog Verrader}\\

\haiku{Gaarne had ik die;}{Dame myne dankbaarheid}{mondeling betuigt}\\

\haiku{derhalven geef u,.}{aan my over en ons beider}{fortuin zal een zyn}\\

\haiku{hy wezentlyk, toen '}{hyer slegts een jaar en zes}{maanden geweest was}\\

\haiku{In weinig tyds had....}{ik myne Plantagie en}{die van Juffrouw P}\\

\subsection{Uit: Ongelukkige levensbeschrijving van een Amsterdammer}

\haiku{Ik wilde, dat gij,{\textquoteright}, {\textquoteleft}}{was als ik zegt hij ergens}{tegen zijn zuster}\\

\haiku{Dus richtte hij zijn.}{aandacht op het platteland}{in de omgeving}\\

\haiku{Ik had al  een}{jaar of twee op het smeeden}{geweest en ik docht}\\

\haiku{En ik verhaalde.}{het hele geval zoals}{het was toegegaan}\\

\haiku{Ge kunt begrijpen.}{of zo een officier ook}{een goede reis doet}\\

\haiku{Hij is nu een groot,.}{jaar bij mij geweest maar wat}{schrijft hij nu al mooi}\\

\haiku{Op de aspot, daar,.}{braaf vuur onder was zodat}{het schielijk hiet wierd}\\

\haiku{Zij moest de melk weg.}{gooien en wierd niet gewaar}{wie het gedaan had}\\

\haiku{Maar ik kreeg op het {\textquoteleft}{\textquoteright}.}{lest van de jongens de naam}{vanketelscheiter}\\

\haiku{Wij hadden allang.}{een overslag gemaakt43 om het}{wijf te betrekken}\\

\haiku{{\textquoteright} {\textquoteleft}Neen vader,{\textquoteright} zei ik,.}{en meteen stong ik op en}{ging naar mijn winkel}\\

\haiku{Mijn baas zal het ook,.}{wel gezien hebben maar niet}{hebben willen zien}\\

\haiku{En toen gingen ze,.}{aan het ramenassen63 van}{kelken en flessen}\\

\haiku{Nu, de kapitein,:}{kwam en het compliment dat}{hij ons maakte was}\\

\haiku{{\textquoteright} De Heer ging weg, en.}{hij vroeg mij aanstonds waar of}{ik ze krijgen kon}\\

\haiku{{\textquoteright} De Beurs ging uit en.}{de Heer kwam en vroeg of hij}{ze had gekregen}\\

\haiku{Om mijn vader een.}{genoegen te doen moest ik}{weer een baas hebben}\\

\haiku{Waar ik vandaan ben,,.}{wie mijn ouders zijn en wie}{mijn famielje is}\\

\haiku{Als ik niets meer kon,.}{verkopen dan ging ik truk}{of biljard spelen}\\

\haiku{{\textquoteleft}Wel,{\textquoteright} was zijn antwoord, {\textquoteleft}.}{de joden weten hem veel}{geld te bezorgen}\\

\haiku{{\textquoteright} Er waren ook vier.}{Heren die bij malkander}{schenen te horen}\\

\haiku{Ik heb de brieven;}{dikwijls gelezen die gij}{haar geschreven hebt}\\

\haiku{Ze hebben nu meer.}{geld dan ze ooit voor deze}{bezeten hebben}\\

\haiku{Kom, ik zal je daar.}{brengen waar ze u dat wel}{zullen verleren}\\

\haiku{maar dan niet in de,.}{jordaene die in Asia leyt}{maar te Amsterdam}\\

\haiku{Somtijds kreeg ik wel,.}{voor twee schellingen slaag maar}{daar gaf ik niet om}\\

\haiku{Zij die op krukken,.}{gingen als zij bij de weg}{waren maar hier niet}\\

\haiku{{\textquoteright} Meteen sche\"en de.}{kaartspelers uit en gingen}{wij allen naar bed}\\

\haiku{Ik rookte en zei,.}{dat ik vertrok en kwam om}{afscheid te nemen}\\

\haiku{Het ging weer over, en,}{ik vernam dat die mij die}{klap gegeven had}\\

\haiku{Ik was verbazend, '.}{leerzaam en daarom ging ik}{mede aant roer}\\

\haiku{Pas op, de dienders.}{van het Zeerecht zullen je}{bij je gat krijgen}\\

\haiku{{\textquoteright} {\textquoteleft}Ja, verdomd,{\textquoteright} viel die, {\textquoteleft}}{kaagschippersknecht hem in de}{redejou vader}\\

\haiku{Ik liet mij niet lang,.}{uitnodigen maar kreeg hem}{snel in mijn vlodders}\\

\haiku{Nu,{\textquoteright} zei hij, {\textquoteleft}laat het,.}{niet weer gebeuren of er}{zal wat opzitten}\\

\haiku{Wij kwamen dan aan,.}{de wal en wierden door een}{ieder verwelkomd}\\

\haiku{Zij weten daar ook,.}{wel dat het geld de God van}{deze wereld is}\\

\haiku{{\textquoteright} De zilversmid kwam,.}{te sterven en twee dagen}{later de snijer}\\

\haiku{{\textquoteleft}Wel,{\textquoteright} zei hij, {\textquoteleft}heb je?}{geen liefhebberij om in}{dit land te blijven}\\

\haiku{Wij waren pas een,.}{week of zes op zee of wij}{kregen een zieke}\\

\haiku{En nog het mooiste,.}{er van is men ziet ze zeer}{zelden baas worden}\\

\haiku{die maaltijd heeft je,?}{zo veel  gekost maar wie}{waren je gasten}\\

\haiku{Ik begon al wat,,,:}{koeler te worden maar toen}{zij dat zag zei ze}\\

\haiku{{\textquoteright} {\textquoteleft}Welnu dan, dewijl,.}{gij de man zijt zal ik je}{mijn komst uitleggen}\\

\haiku{Alle weken drie,.}{gulden op afrekening}{tot het betaald is}\\

\haiku{In plaats dat ik een.}{Sara had gekregen was}{het een Xantippe}\\

\haiku{Hij had kinderen,.}{noch famielje zodat hij}{allenig woonde}\\

\haiku{Nu zie ik wat een.}{liefde voor je famielje}{er in je huisvest}\\

\haiku{Ik denk dat het komt.}{doordat mijn hele natuur}{er geen trek in heeft}\\

\haiku{en dat ze mij, in,.}{plaats van schuld te bekennen}{nog lelijk uitschold}\\

\haiku{En altijd als gij,.}{de deur uit bent al was het}{maar om een boodschap}\\

\haiku{Was het er maar \'e\'en,.}{ik zou zeggen dat het uit}{liefde mocht wezen}\\

\haiku{Ik zei hem niet te,.}{geloven dat zij die dans}{zoude ontspringen}\\

\haiku{Maar toen een ieder,.}{hoorde dat ik alles kocht}{kreeg ik braaf nering}\\

\haiku{Al werkte ik mij,.}{dood de baas zou zeggen dat}{ik te weinig deed}\\

\haiku{{\textquoteleft}Ik heb je geval,{\textquoteright}, {\textquoteleft}.}{gehoord zeide hijik durf}{niet meer te komen}\\

\haiku{Maar nu begrijp ik.}{het best. Toch benne zij ook}{wel eens bedrogen}\\

\haiku{{\textquotedblleft}Mannen, als gij geld,.}{wilt hebben dan moet gij mij}{laten dagvaarden}\\

\haiku{Laat ik je zeggen,.}{dat onze neef het juist en}{billijk heeft gedaan}\\

\haiku{Want of men goed of,.}{kwaad doet men ondervindt dat}{toch niet voor zijn dood}\\

\haiku{Zodat ik de smaak.}{begon te krijgen van een}{vergenoegd leven}\\

\haiku{{\textquoteleft}Volgens mijn gebruik,.}{goed maar ik heb niet de eer}{mijn Heer te kennen}\\

\haiku{Hij had een slaapbaas',.}{dochter tot vrouw die aardig}{wat duiten meebracht}\\

\haiku{Het wijf ging uit de,.}{kamer en toen wilde de}{vent mij gebruiken}\\

\haiku{Op 't lest wierd ik.}{de brutaalste en leepste}{hoer van de wereld}\\

\haiku{Ik bracht het geld waar,.}{het wezen moest en kreeg mijn}{schuit met goed weerom}\\

\haiku{37kruishoeren = {\textquoteleft}{\textquoteright}.}{waarschijnlijk 18e eeuwse term}{voortippelaarsters}\\

\haiku{{\textquoteleft}Klootjes-volck{\textquoteright} (.)}{van de vesten of uijt de}{slopjesBredero}\\

\haiku{Bicker Raije b.v. had;}{verscheidene vrienden die}{er lid van waren}\\

\haiku{Zij verzoeken dan.}{om een vrijwillige gift}{voor de huisarmen}\\

\haiku{Straks gaan we naar ien,}{Waefelhuis daer we ongs}{vryheid zellen zien}\\

\subsection{Uit: Reis van Jan van Mandeville}

\haiku{Reis van Jan van}{Mandeville N.A. Cramer}{Colofon}\\

\haiku{krieketeren.}{ofte 			 kersselteren}{oft prumelteren}\\

\haiku{fonteyne der houe.}{ende putte borne der}{leuender watre}\\

\haiku{Jtem neuen akon,.}{loopt een cleyne ruuiere}{die be{\textbackslash}leon  heet}\\

\haiku{Want men mach tlant niet}{winnen om die grote}{verscheit ende daer}\\

\haiku{Dese grendere.}{of greniere sijn nv al}{vol serpenten}\\

\haiku{Dese stat heeten,:}{die moneke besebeel}{dat is te segghen}\\

\haiku{Ende daer om en.}{ontsien si niet den 			 soudaen}{noch ander princhen}\\

\haiku{Jn dese stat van.}{barsabee woonde abraham die}{patriarke}\\

\haiku{Bij ebron is die berch,.}{van mambre van wien tdal heeft}{sinen name}\\

\haiku{dat onse here}{starf anden cruce ende}{doe begonste hi}\\

\haiku{sine quaetheit moet.}{dalen op sine 			 cop}{van sinen hoofde}\\

\haiku{Jherusalem is int}{conincrike van jude}{ende omme}\\

\haiku{Ende daer neuen,.}{in die mure es die}{stede daer die .iiij}\\

\haiku{dierbaer ghesteynte.}{ende een busse van}{jaspide met .vij}\\

\haiku{waerlic dese.}{stat is heylich ende ic}{en wistes niet}\\

\haiku{jc kenne ende.}{weet dat die 			 here v}{sal dit lant gheuen}\\

\haiku{sliep bi hem ende,}{lach ende wan aen hem twee}{zonen die hieten}\\

\haiku{fonteynen, die daer.}{sijn ende gheheten sijn}{jor ende 			 dan}\\

\haiku{Ende dit sijn haer,:}{name also si heten}{bouen ghescreuen}\\

\haiku{ofte ander dinc,}{dat wel ruuct ende inden}{wyerooc so biechten}\\

\haiku{Vandaer gaetmen te,}{beruth daer sinte jorijs}{den draec door 			 stac}\\

\haiku{Ende weet, datter.}{vele meer in dat lant vriest}{dan in 			 dit lant}\\

\haiku{Want hi plach desen}{goeden man gherne horen}{te predicken}\\

\haiku{Na dien dat jc v}{hier voren ghesproken hebbe}{vanden heileghen}\\

\haiku{Ende dese 			 .}{was wiser in sijn wenschen}{dan die coninc was}\\

\haiku{Daer omtrent en wast,.}{ghenen wijn 			 noch froyt ten}{si alte luttel}\\

\haiku{Also hi dede}{ende brachte met hem een}{vanden plancken}\\

\haiku{een keytiuich abijt}{wijt ende cort ende die}{mouwen daer of sijn}\\

\haiku{tnoorden leit hem op.}{die slincke zide van}{sinen aensichte}\\

\haiku{Die coninc 			 van.}{desen eylande ende}{van thyathana}\\

\haiku{gheheylicht moghe,.}{werden van desen dinghen}{die niet en doghen}\\

\haiku{Want also jc v,}{voren hebbe gheseit die}{helleft vanden}\\

\haiku{en ontsiedi niet,.}{die die aerde in niet}{heuet ghehanghen}\\

\haiku{Ende daer na die,.}{meesters van astronomien}{segghen dat 			 vijc}\\

\haiku{Ende dye lieden.}{die daer wonen die hebben}{al honts 			 hoefden}\\

\haiku{Ende alsi enen,}{coninc kiesen so gheuen}{si hem desen}\\

\haiku{die mure heeft ij}{milen ommegaens ende}{binnen desen}\\

\haiku{ende vele 			 ,.}{ordinancien maecte die}{si heten ysatan}\\

\haiku{Deus in celo et;}{can super terram eius}{fortitudo}\\

\haiku{Daer 			 na maken,.}{si den nacht so datmen enen}{steke niet en siet}\\

\haiku{des 			 keysers jn.}{eenre haluer dachuaert}{na ter rechter hant}\\

\haiku{groter steden 			 ,.}{sonder dander cleyne die}{zonder ghetal zijn}\\

\haiku{Ende ghemeenlijc,}{riden si sonder sporen}{mer si draghen}\\

\haiku{Ende dan alt lant.}{sent hem prosente in dien}{daghe meer dan .c}\\

\haiku{Ende ten oosten,.}{waert steet een wildernisse}{die is wel 			 .c}\\

\haiku{dachuaerden lanc}{ende die beste stat van}{dien lande 			 is}\\

\haiku{hem westwaert toter,}{riuieren van phison}{die een 			 vanden}\\

\haiku{Doe 			 vielen die}{kerstene op hare knien}{ende dan daden}\\

\haiku{der 			 naturen.}{in sijnre hoocheit ende}{in sijnre glorien}\\

\haiku{dit hadden gheseit,}{op die betrouwenisse}{gods wi 			 daden}\\

\haiku{ghe{\textbackslash}pijnt gode te,}{dienen so is die placke}{al vergaen ende}\\

\haiku{seg{\textbackslash}ghen, datse}{dat vier sal reynighen van}{allen 			 vlecken}\\

\haiku{In dit 			 lant sijn}{twee wintere ende twe}{somere ende}\\

\haiku{vijf gherechten te}{gader ende si brenghen}{dese gherechten}\\

\subsection{Uit: Robrecht de Duyvel}

\haiku{In r. 7-8 wordt.}{de stad genoemd waar hertog}{Oubeert open hof houdt}\\

\haiku{Ende als dese,.}{Rolle ghedoept wert so wert}{hi ghenoemt Robbrecht}\\

\haiku{Tenslotte maakt de.}{engel zich bekend en wordt}{Robert weer koning}\\

\haiku{{\textquoteright} Welcke woerden,}{ghehoort sijnde van den}{heren so stont daer}\\

\haiku{moghes ende die,.}{ghi    oec wel verwerven}{sult dat weet ic wel}\\

\haiku{Sie rie-    pen,}{alle tot Robrechten dat}{hy afliete mer}\\

\haiku{{\textquoteleft}Mijn lieve sone,.}{ic bidde u    dat ghi}{mi mijn hoot171 afslaet}\\

\haiku{{\textquoteright} Ende alle    :}{de ander rovers seyden}{met eender stemmen}\\

\haiku{Ic heb mijn leven.}{qualijck    beleyt205}{ende overgebracht206}\\

\haiku{Doen Robrecht hem al238,:}{ghebiecht had seyde die}{heremijt tot hem}\\

\haiku{mochte met also.}{luttel penitencien die}{hi doen soude}\\

\haiku{Ende des morgens.}{vroech stont hi op ende}{ghinc te Rome waert}\\

\haiku{verweert nu u lijf,.}{te-    ghen mi want ghi}{moet van my sterven}\\

\haiku{daerom so moet{\textquoteright}.}{ghi geloont wer-    den}{van uwen wercken}\\

\haiku{Ende dairna was}{hi verheven ende}{gheeert van dengenen}\\

\haiku{In veel legenden.}{en exempelen vinden we}{daar voorbeelden van}\\

\haiku{Jan van Beverley,,.}{Mariken Malegijs en}{vele anderen}\\

\haiku{In het algemeen:}{zijn van belang voor sprookjes}{en -motieven}\\

\subsection{Uit: De zeven wijze mannen van Rome}

\haiku{{\textquoteleft}Vraag wat je wenst, want,.}{ik zal je niets weigeren}{wat je ook maar vraagt}\\

\haiku{{\textquoteright} {\textquoteleft}Als het is zoals,.}{u zegt dan wil ik om een}{kleine gunst vragen}\\

\haiku{{\textquoteright} Maar de zoon boog slechts.}{het hoofd naar zijn vader en}{antwoordde hem niet}\\

\haiku{{\textquoteright} Toen zij de brief had.}{gelezen verscheurde ze}{hem met haar tanden}\\

\haiku{{\textquoteright} De keizer was in.}{de grote zaal toen hij zijn}{vrouw hoorde roepen}\\

\haiku{{\textquoteleft}O, goede meester,.}{haast u naar het paleis en}{bevrijd uw leerling}\\

\haiku{{\textquoteright} {\textquoteleft}O, heer keizer, ik.}{verdien het anders door u}{begroet te worden}\\

\haiku{Helaas, wat moet ik,?}{nu doen nu ik mijn enige}{kind verloren heb}\\

\haiku{Ik ben u dankbaar.}{dat u hem op mijn verzoek}{vandaag hebt gespaard}\\

\haiku{{\textquoteright} {\textquoteleft}Heer, u weet dat zich.}{bij de deur van ons huis een}{waterput bevindt}\\

\haiku{{\textquoteright} Terwijl hij haar zo.}{toesprak ging de maan onder}{en werd het duister}\\

\haiku{{\textquoteleft}Nou ja, heer, als het.}{er dan zo voor staat moet ik}{mezelf verdrinken}\\

\haiku{En ik wens dat mijn.}{lichaam wordt begraven in}{de Sint-Pieterskerk}\\

\haiku{Toen de ridder het,:}{geluid hoorde riep hij met}{jammerende stem}\\

\haiku{De ridder  stond:}{nu bij de put en huilde}{bittere tranen}\\

\haiku{{\textquoteleft}Ach, beste man, het.}{is geen goed teken dat u}{op dit uur hier staat}\\

\haiku{Hij vertelde de,:}{keizer wat hem overkomen}{was die daarop zei}\\

\haiku{{\textquoteright} De meester gaf zijn.}{paard de sporen en spoedde}{zich naar het paleis}\\

\haiku{Bovendien heeft hij.}{ons door de hele stad een}{slechte naam bezorgd}\\

\haiku{Hij brak meteen zijn.}{speer in drie\"en en trok naar}{het Heilige Land}\\

\haiku{{\textquoteleft}O, goede meesters,.}{de afgelopen nacht heb}{ik een droom gehad}\\

\haiku{{\textquoteleft}Neem een schep en graaf.}{op de plek waar u denkt dat}{de bron zich bevindt}\\

\haiku{{\textquoteright} Nadat ze de raad,.}{van haar moeder had gehoord}{ging de vrouw naar huis}\\

\haiku{Omdat hij het niet,.}{tijdig dichtte verloor haar}{gezicht alle kleur}\\

\haiku{{\textquoteleft}Nooit heb ik zulke.}{betrouwbare en wijze}{waarzeggers ontmoet}\\

\haiku{{\textquoteright} {\textquoteleft}De toren met de.}{beelden is uw lichaam met}{de vijf zintuigen}\\

\haiku{Uw zoon heeft het bij.}{ons namelijk niet zo slecht}{gehad als u denkt}\\

\haiku{Toen  Galienus,.}{bij de koning kwam werd hij}{eervol ontvangen}\\

\haiku{{\textquoteright} {\textquoteleft}Wie anders zou de?}{vader van het kind zijn dan}{mijn heer de koning}\\

\haiku{{\textquoteright} {\textquoteleft}Als u dat doet, zult.}{u een grote beloning}{van mij ontvangen}\\

\haiku{een welgevormde,.}{mooie vrouw die deze nacht op}{mijn schoot kan slapen}\\

\haiku{{\textquoteright} In de avond bracht de.}{rentmeester zijn vrouw naar het}{bed van de koning}\\

\haiku{{\textquoteright} De volgende dag.}{bestookte de koning de}{stad met veel geweld}\\

\haiku{De eerste ridder.}{kwam zonder aarzeling en}{klopte aan de deur}\\

\haiku{{\textquoteleft}Wee mij, wee mij, hij.}{is opnieuw opgestaan en}{teruggekomen}\\

\haiku{Terwijl hij zo bij,:}{het vuur stond kwam de vechter}{naar hem toe en zei}\\

\haiku{Meteen toen zij het,.}{hoorde werd ze gegrepen}{door liefde voor hem}\\

\haiku{De vrouw merkte dit.}{en schreef hem een brief die ze}{naar beneden wierp}\\

\haiku{Daarna ging hij naar.}{de koningin en groette}{haar met veel achting}\\

\haiku{Onderricht haar goed.}{zodat ze zich u altijd}{zal herinneren}\\

\haiku{Toen de moordenaar.}{werd gevangen verloor hij}{twee boventanden}\\

\haiku{In die tijden was.}{er een keizer te Rome}{die Titus heette}\\

\haiku{De keizer zag dit,:}{en stond op uit zijn zetel}{kuste hem en zei}\\

\haiku{Hij groette haar met.}{veel eerbied en zette het}{gerecht voor haar neer}\\

\haiku{{\textquoteleft}O, Alexander, waar heeft?}{Lodewijk dit kostbare}{kleed kunnen kopen}\\

\haiku{Als ik tegen hem.}{in het strijdperk kom zal ik}{zeker verliezen}\\

\haiku{Maar als je met mijn,.}{bruid naar bed gaat zorg dan dat}{je mij daar trouw bent}\\

\haiku{Daarom zult u mij.}{voortaan nog geliefder en}{waardevoller zijn}\\

\haiku{{\textquoteright} Alexander nam afscheid.}{van de keizer en reisde}{terug naar zijn rijk}\\

\haiku{De voedsters liepen}{snel naar het vertrek waar de}{kinderen waren}\\

\haiku{{\textquoteright} {\textquoteleft}O, heer koning,{\textquoteright} zei, {\textquoteleft}?}{de ridderwaarom vraagt u}{dit over onze zoon}\\

\haiku{Mijn verheffing strekt.}{u tot eer en is u tot}{een eeuwig voordeel}\\

\haiku{Waar Boccaccio zijn,.}{verhalen vandaan haalde}{weten we dus niet}\\

\haiku{Wie waren nu de?}{lezers of toehoorders van}{de Zeven wijzen}\\

\haiku{De vertaling wil.}{ten volle recht doen aan de}{inhoud van de tekst}\\

\subsection{Uit: Het boek van Sidrac}

\haiku{Opmerkingen bij.}{enige eigennamen in}{het Boek van Sidrac}\\

\haiku{en vervolgens, dat.}{de Sidrac hoogstwaarschijnlijk}{v\'o\'or 1325 vertaald is}\\

\haiku{De mogelijkheid;}{onder c sluit natuurlijk}{a of b niet uit}\\

\haiku{Jacobs I bl. 345 {\textsection}),.:}{en 110 waarnaast soms vormen}{met \={o} in open lgr}\\

\haiku{hem darme mensche}{nederen voir den riken.171Eest}{sonde datmen eet}\\

\haiku{die natuere}{smenschen tsamen alsi gaet}{uten live ende}\\

\haiku{Sidrac ende dat}{selve boec ginc na coninc}{Bottus doot vanden}\\

\haiku{Ghi heren ghi weet.}{wel dat ic ben die meeste}{coninc van Orienten}\\

\haiku{mijn here, dien een}{ingel brachte van Gode}{den heylegen man}\\

\haiku{Sidrac sach op te:}{hemele weert ende sprac}{dese bedinge}\\

\haiku{Du heefs gheloeft aen.}{Sidracx seggen ende an}{sine toverie}\\

\haiku{dat firmament heeft,.}{ij lichte dats de sonne}{entie    mane}\\

\haiku{anderwerf was hy,.}{ongehor-  sam want}{hy brac tgebod Goids}\\

\haiku{een gepens, soe en.}{haddi    niet geweest een}{gherecht rechtere}\\

\haiku{deen sterft omme    .}{datti volleeft heeft den tijt}{die hem God sette}\\

\haiku{dat vier trect weder.}{ter sonnen die sine}{natuere es}\\

\haiku{hy ghejugeert498 wert,}{na sine verdiente}{wilt de here sijns}\\

\haiku{Die ghene die meer}{eten ende drincken dan}{sy    behoeven}\\

\haiku{Darumme dat see}{nicht ghelyk en sijn Gode}{darumme moten}\\

\haiku{dat moet den beesten.}{macht gheven  scholen see}{deghe hebben}\\

\haiku{sal hem houden    ;}{in goeder stede ende}{in goeden poente}\\

\haiku{Maer es de   579}{lichame580 dan sterc ende}{vroet ende heefti}\\

\haiku{Maer lichtelike,;}{mach hijs telivereert}{werden wille hy}\\

\haiku{Want die duvel mach}{niet gesaedt werden van}{quaet doene ende}\\

\haiku{niemen en es diet,.}{woude want hem souts zere}{ver-  drieten}\\

\haiku{Ende oec souden.}{vrouwen dair af    ontfaen}{ende kint dragen}\\

\haiku{Hoe maecte God die?}{werelt ende waer    ane}{onthout se hare}\\

\haiku{Hets een ander lant.}{daert altoes doncker es}{alse de    nacht}\\

\haiku{ende die goede [],}{werken werct hy en    mach}{engheen quaetdoen maer}\\

\haiku{anderen ende}{vele mere    ende}{sterkere ende}\\

\haiku{Ende sy lijt die.}{XII tekene ende in}{elc blijft sy ene maent}\\

\haiku{dat vierde omme.}{dat sy niet en hebben te}{verliesene}\\

\haiku{Ende    dits de.}{goede ghenuechte met}{wiven te sine}\\

\haiku{Maer die anders doet.}{hy en aenbeet    Gode}{niet788 gherechte}\\

\haiku{sinen sceppere,}{ende hem selven want hy}{ende sijn wijf}\\

\haiku{ende alsoe van,.}{beesten van vogelen}{ende van visschen}\\

\haiku{Ende en haddi}{al dit niet    gheweten}{soe en haddi gheen}\\

\haiku{Ende des en sal.}{een    twint niet gaen op den}{ghenen diene doot}\\

\haiku{Ende daer omme.}{maecti die nacht omme der}{rasten wille}\\

\haiku{Het sal comen een:}{prophete    die seggen}{sal uten monde Goids}\\

\haiku{lichame ende.}{gadert alle die southeit}{in die blase}\\

\haiku{Dit soude dy met}{rechte herde    leet sijn}{ende du soutster}\\

\haiku{Aldus eest vanden;}{kinde alst sterft    in den}{buuc siere moeder}\\

\haiku{Dat lieflijcste}{aensien dat es dat een te}{hemele waert}\\

\haiku{Salmen scuwen den:}{rijs ende sine rijsheit?1017}{Sidrac antwort}\\

\haiku{Mesdoen die liede?}{iet alsi doemen die}{liede ter doot}\\

\haiku{ende het berret.}{altoes ende en    gheeft}{ghene claerheit}\\

\haiku{[Ende daer omme}{dat sy niet en keerden ten}{ghewaregen}\\

\haiku{meerre] bliscapen,;}{want hare bliscap en}{es noch niet volmaect}\\

\haiku{1114 want dat let es /.}{ghemaect na die maniere}{vander blasen}\\

\haiku{Vier dingen sijn daer().}{de zielen sceet van-}{den lichame}\\

\haiku{maer die rijcheit der.}{zielen gaet boven    al}{dander rijcheit}\\

\haiku{Die rike sijn meer.}{ayse ende die arme}{sijn    meest seker}\\

\haiku{Sal hem een belgen?}{des dat hem een    ander}{toent quaet gelaet}\\

\haiku{ende daer omme.}{sijn wy alle vremde in}{dese werelt}\\

\haiku{Hoe comt dat die uten?}{westen vroeder    sijn dan}{ander liede}\\

\haiku{1172 Mair/ hadde God,}{gewilt hy hadt altoes dach}{ghemaect    ende}\\

\haiku{dat sy sijn    soe.}{berren sy alse een vier}{in grooter pinen}\\

\haiku{Ende alsi sien;}{datmen sonde doet soe sijn}{sy herde droeve}\\

\haiku{Dierste    es dat de,;}{hane ghecroent es dander}{datti sporen heeft}\\

\haiku{Die Goids sone sal?}{hy enich huus    hebben op}{eerterike}\\

\haiku{Ende die en sal;}{anders niet ontfaen dan}{1241 broot allene}\\

\haiku{gheen oir-  boer.}{doen ende selen weder}{achter weert keren}\\

\haiku{staen die doode, die.}{oirconde dragen selen}{van dien datti seit}\\

\haiku{Want die ghene die}{voeden sal den lichame}{des Goids soens dat sal}\\

\haiku{Twee oghen, twee oren,,.}{dat sijn si vier Dat sijn der}{herten messelgier}\\

\haiku{Anglicus XII, 1.}{citeert een dergelijke}{meedeling van Isidorus}\\

\haiku{onwellecome (),.}{96 r. 12 K bijvorm van}{onwillecome}\\

\haiku{vadheit (194 r. 36),,.}{bijvorm van vaddicheit =}{vadsigheid luiheid}\\

\haiku{ghedaen hebben 39 - -:}{den l. groot 40 gheselscap}{tieghen bl. 50 K}\\

\haiku{et pour ce que nous.}{sceussions quil nous a fait}{a sa semblance}\\

\haiku{1-2 no g. - eist.}{dat alemoesene of}{z. 3 g. no qu}\\

\haiku{heeft sine woenst a. - - (....}{up die aerde die st.}{5 aerdeblijft}\\

\haiku{) aerden l. bi.}{h. ende 5 also die}{man metten wive}\\

\haiku{et ne lui chault quel:}{conseil il lui donne soit}{a son prouffit J}\\

\haiku{1 verhoghet-.}{34 honden achter te}{zamen vaster d.a.b}\\

\haiku{geduert in elk ().}{II Sd.i. 2 1/2 iaer soe}{es van groter macht}\\

\haiku{- v. ere ander g. - -.}{nadien dat 31 tpoint g. es}{32 hebben an e.u}\\

\haiku{) 35 of - helighen -.}{noch 36 creature ne}{weet noch ne mach w.tg}\\

\haiku{7 voor ziele e.- - - (....}{voor lijf 89 anebeden}{anebedenreinre}\\

\haiku{nieute 32 ne woont - -.}{dat daer lieden woenden s.}{gezonder 33 s.w}\\

\haiku{daer sy sonder niet ():}{sijn en machzo ongeveer}{ook Kln bl. 137 K}\\

\haiku{a la cervelle:}{et aux yeulx et par toute}{la teste 28 Haag}\\

\haiku{want sonder sy in,.}{mochte man neit sin der smit}{is herre van a.c}\\

\haiku{Du sult - staerc 5.}{dat quade ghepeins ende}{onnutte suls v.e}\\

\haiku{werpen 6 o. voet.}{e. danne sultu ontgaen}{den wille v.d.qu.d}\\

\haiku{3-4 geb. wort een.}{kint ende d.m. si in P.}{van naturen d.w}\\

\haiku{) 6 soudser omme}{danne harde zere sijn}{te vernoye omme}\\

\haiku{) - edel 15 drouch - dat het - -.}{hemel 16 precieusen}{met s.l. 17 d.s}\\

\haiku{so worden si g.- -.}{3132 ghewerken vonden}{sijn eist g. eist qu}\\

\haiku{aerbeit dat h. (.).}{28 ontfermet so zweet e.}{alstdan om 29 zw}\\

\haiku{men coomt tote}{3 dine herberghe m.}{dine mesnieden}\\

\haiku{1 scijnt - hooft in - -.}{e. 2 stont worden 3 d.}{toe e. d. van u.w.m.e}\\

\haiku{) sine gramscepe als}{hi in scaden es of in}{verliese 26 maer}\\

\haiku{ghemict te doene - - ().}{33 dueghet peinsen h. dueghetvraag}{356 ontbreekt in K}\\

\haiku{pour la grant doubte:}{et pour la grant crainte quilz}{auront deulx 14 Fr. druk}\\

\haiku{17 my wyl 18 stadt}{van Andwerpen 19 ick dit}{boeck transferierde}\\

\haiku{men bekinnen de.}{goede entie quade die}{harde I ro kol}\\

\haiku{omme gaf hi den}{menssce sin te verstane}{datti den duvel}\\

\haiku{ende alse god.}{woude so waest dach alse}{wi dagelijx sien}\\

\haiku{ende alse hem.}{de donkereit vergaet so}{comti tote ons}\\

\haiku{Want al datter in.}{comt moet se rumen C.XCV}{De coninc vraget}\\

\haiku{Alse de man siet}{dwijf ende hi se dan mint}{ende dwyf den man}\\

\haiku{Het boek van Sidrac:}{Noten 1Zie o.m. K. de}{Flou en E. Gaillard}\\

\haiku{Entstehung und,).}{Ausbreitung der Alchemie}{Berlin 1919 bl. 294}\\

\haiku{1ro, zoals ik die,.;}{hier laat volgen heb ik naar}{het hs gecopieerd}\\

\haiku{Maar prof. dr. W. de}{Vreese was zo vriendelijk}{mij mee te delen}\\

\haiku{Et convoitise.}{si est fille a envie}{car deli descent}\\

\haiku{Zie het opstel van,:}{Eringa in TMNL XLIV bl. 100}{e.v. 1379A.J.H. Charignon}\\

\subsection{Uit: Exempel van een Soudaensdochter}

\haiku{hert begonste al}{getogen te worden}{tot zijnre lieften}\\

\haiku{Des naesten nachts so}{quam hi weder cloppen}{noch so vraechde si}\\

\haiku{u te                     houden}{op minen coste Iesus}{scheyde doe weder}\\

\haiku{si bleef daer voer die}{poorte                     sitten ende}{wachte na haer lief}\\

\haiku{Ende hi                     brocht}{met hem een grote schaer van}{eerbaren gesin25}\\

\haiku{van Duyse te Gent.}{zoo vriendelijk was mij te}{verschaffen}\\

\haiku{Wie mach den maker,.}{der bloemen zijn Mocht ick hem}{eens ghewinnen}\\

\haiku{Mijn alderliefste?}{Ionghelinck schoon Van waer}{komt ghy gheganghen}\\

\haiku{Zeght my schoon maghet wat,?}{ghy begheert Waerom}{soo meught ghy weene}\\

\haiku{Schoon maghet u Lief en,,}{is hier niet Ick heb hem niet}{vernomen Voorwaer}\\

\haiku{Hy is ghekleedt met,:}{blauw lazuer Om beset}{met gulde sterren}\\

\haiku{Al waer hy wt des.}{Hemels Throon Hy en mocht niet}{beter wesen}\\

\haiku{Mijn Lief is alle,.}{eere wel weert Die ick soo}{seere beminne}\\

\haiku{Doet op, doet op, en,.}{latet mij in Mijn lief die}{is hier binnen}\\

\haiku{Sijn wangen sijn soo,.}{schoon geroozet Gelijck een}{roode roose}\\

\haiku{Al in een kersten,;}{leven Hij dedese in}{een cloosterkijn}\\

\haiku{Ter vergelijking:}{laat ik hier het begin van}{het lied volgen}\\

\haiku{Zij zag hem zoet en, ():}{vriendlijk aanEn neeg tot op}{der aarde En sprak}\\

\haiku{De Poortier, die zyn,', '?}{Stem verhief Ja Vader zeid}{hy dats myn Lief}\\

\haiku{E. Antwerpen, Jan, ().}{van Ghelen z.j.tweede helft}{der                             16de eeuw}\\

\haiku{veste/                            inden.}{witten Hasen- || wint}{bi Jan van Ghelen}\\

\haiku{dochter soe ben ic}{dan doot met u. ende ic}{in                     sal niet meer}\\

\haiku{ende dit dochte}{haer herde                     goet sijnde}{ende sij pijnde}\\

\section{Henricus Pomerius en anoniem}

\subsection{Uit: Ridderboek}

\haiku{U hebt beloofd Hem:}{te dienen toen men bij uw}{doop in uw naam sprak}\\

\haiku{Nu zal ik u de.}{hoedanigheid van beide}{heren beschrijven}\\

\haiku{{\textquoteleft}O schone stad van,.}{Jeruzalem wat heb je}{een schone bouwer}\\

\haiku{{\textquoteleft}Heer, mijn hart - dat is -.}{mijn ziel is ongeduldig}{tot het in U rust}\\

\haiku{Maar de dienstmaagd had.}{alleen aandacht voor wat haar}{zinnen prikkelde}\\

\haiku{nu moet de ridder.}{haar volgen in plaats van dat}{zij hem gehoorzaamt}\\

\haiku{Niet op aarde maar;}{in de zalige hemel}{is haar bestemming}\\

\haiku{die heeft immers ook.}{geen schuld aan de dronkenschap}{van de gulzigaard}\\

\haiku{Ik dank de koning.}{dat wij over deze wapens}{mogen beschikken}\\

\haiku{Maar dit leed is voor,:}{het hart een genot zoals}{Augustinus zegt}\\

\haiku{Zijn dienstmaagd begon}{zich te bekreunen samen}{met de koningin}\\

\haiku{{\textquoteleft}Het verwondert mij.}{dat u uw zoon priester of}{monnik wilt maken}\\

\haiku{{\textquoteleft}Wie u andere,.}{raad geeft is een bedrieger}{die u moet mijden}\\

\haiku{{\textquoteleft}Indien u opziet,.}{tegen de arbeid denk dan}{aan de beloning}\\

\haiku{Deze mensen staan.}{tussen de soldaten en}{de hoge heren}\\

\haiku{Het vierde gebod.}{is dat een mens zijn vader}{en moeder moet eren}\\

\haiku{Men moet zich hierover,.}{niet verbazen noch trachten}{dit te doorgronden}\\

\haiku{Hierna volgen de.}{tien geboden maar die heb}{ik reeds behandeld}\\

\haiku{Zalig zijn zij die.}{een zuiver hart hebben want}{zij zullen God zien}\\

\haiku{Deze dochter is.}{de eigenwaan die voortkomt}{uit zelfbehagen}\\

\haiku{Zij weerstreeft alle.}{deugden en vooral verzet}{zij zich tegen God}\\

\haiku{Hij draagt dure wol,,,,.}{zijde parels edelstenen}{goud en zilver}\\

\haiku{Dan ontstaan hebzucht,,.}{wellust verlekkerdheid en}{onrechtvaardigheid}\\

\haiku{Kom, zie eens wat een,.}{weelderige tombe wat}{een kostbare zerk}\\

\haiku{Doch, al zwijgen de,.}{dwaze kinderen hierover}{de profeet zwijgt niet}\\

\haiku{U moet echter nog.}{wel weten van twee soorten}{driftige mensen}\\

\haiku{Volgens de boeken.}{heeft de gulzigaard van zijn}{buik zijn god gemaakt}\\

\haiku{Bij de een ligt het;}{voedsel zo zwaar op de maag}{dat hij in slaap valt}\\

\haiku{Want het schaadt zowel.}{de ziel als het lichaam en}{ook de zintuigen}\\

\haiku{Mijn goede vriend, ik.}{kan met recht over het onheil}{van dit kwaad spreken}\\

\haiku{Zodra zij met het,.}{gif van de wellust besmet}{zijn worden zij loops}\\

\haiku{De een verlaat zijn,.}{klooster de ander verlaat}{vader en moeder}\\

\haiku{Geen rijkdom, eer, vrees,,.}{liefde pijn of moeite kan}{hen tegenhouden}\\

\haiku{voor zover het in,.}{mijn vermogen ligt zal ik}{de waarheid schrijven}\\

\haiku{hij wil alles mooi,.}{en goed hebben maar zelf is}{hij geen van beide}\\

\haiku{Zij moeten het na,.}{dit leven duur bekopen}{maar ook op aarde}\\

\haiku{{\textquoteleft}Zonder de woorden.}{mijn en dijn zouden alle}{mensen vreedzaam zijn}\\

\haiku{Het woordje mijn kleeft.}{aan alles zo vast dat men}{het niet meer los krijgt}\\

\haiku{Maar de ander zond.}{een vette os die veel meer}{waard was dan het schaap}\\

\haiku{En zo verloor de.}{eerste man zijn recht dat door}{mijn was omgekeerd}\\

\haiku{Dan was zijn recht hem,.}{niet ontgaan want mijn zou hem}{hebben bijgestaan}\\

\haiku{{\textquoteright} Zij zullen wonen.}{waar nu de valse keizer}{Nero verblijf houdt}\\

\haiku{De jonkheer meent dat.}{het goed en eervol is om}{ridder te worden}\\

\haiku{Maar helaas, deze,.}{noodzaak groeit meer en meer want}{hebzucht houdt niet op}\\

\haiku{Wee degenen die!}{verantwoordelijk zijn voor}{deze wantoestand}\\

\haiku{{\textquoteleft}Valse rentmeester,.}{geef mij verantwoording van}{je rentmeesterschap}\\

\haiku{Maar wie in leven,;}{wil blijven moet de last der}{accijnzen dragen}\\

\haiku{Maar helaas, het is,.}{beter niet gedaan dan al}{doende God vertoornd}\\

\haiku{{\textquoteleft}Het is een schande.}{dat de zon een luie christen}{in zijn bed beschijnt}\\

\haiku{De Heer is allen.}{nabij die Hem aanroepen}{in waarachtigheid}\\

\section{anoniem en anoniem}

\subsection{Uit: Walewein, de neef van koning Arthur}

\haiku{Maar dat zal spoedig,.}{gewroken worden want ik}{wil met u strijden}\\

\haiku{De witte ridder.}{viel dood neer voor de voeten}{van de mooie jonkvrouw}\\

\haiku{Daarna liet hij een.}{zeer fraai overkleed brengen dat}{Walewein aantrok}\\

\haiku{Walewein stak zijn.}{speer met geweldige kracht}{op de draak kapot}\\

\haiku{Hij moest zich dus wel,.}{verdedigen anders zou}{hij gedood worden}\\

\haiku{{\textquoteright} De hofmaarschalk liet.}{Walewein liggen en reed}{op Gringalet  af}\\

\haiku{De koning gaf hem.}{de beste wapenrusting}{die hij kon vinden}\\

\haiku{Bovendien vraag ik.}{u dringend hen niet al te}{zwaar te verwonden}\\

\haiku{hij gaf er een zo,.}{van langs dat deze hem snel}{om genade bad}\\

\haiku{Zodra ze binnen,;}{waren brachten knapen hun}{paarden naar de stal}\\

\haiku{Men sloot de poort en.}{men haalde zonder iets te}{zeggen de brug op}\\

\haiku{Op dat moment was.}{het eten klaar en men zette}{zich aan de maaltijd}\\

\haiku{Drauwedon zelf werd.}{uitverkoren om het ten}{uitvoer te brengen}\\

\haiku{{\textquoteleft}Heer, als u de mis,.}{wilt bijwonen zal ik u}{naar de kerk brengen}\\

\haiku{wilt u het tegen?}{mij opnemen of maakt u}{liever rechtsomkeert}\\

\haiku{Ze nam Walewein.}{bij de hand en maakte een}{bad voor hem gereed}\\

\haiku{Walewein lag in.}{een fraai en zacht bed en sliep}{een gat in de dag}\\

\haiku{Iedereen verliet.}{het strijdperk en begaf zich}{naar zijn logement}\\

\haiku{De paarden die ik,.}{vandaag veroverd heb schenk}{ik u bij voorbaat}\\

\haiku{{\textquoteleft}Uit naam van God heet,!}{ik u welkom gij bloem van}{alle ridderschap}\\

\haiku{Zonder gekheid, daar.}{heb ik al lange tijd mijn}{zinnen op gezet}\\

\haiku{Het verhaal vertelt.}{ons nu over een ridder die}{Moriaan heette}\\

\haiku{mijn goederen, mijn.}{erfenis en mijn leen had}{ik erdoor gejaagd}\\

\haiku{Het is immers het.}{wreedste beest waarover men ooit}{heeft horen lezen}\\

\haiku{Ik druk u op het - -.}{hart om hen zo mogelijk}{hierheen te brengen}\\

\haiku{Zeg me welke weg.}{u wilt op gaan en welke}{u voor mij overlaat}\\

\haiku{een met een rode.}{wapenrusting en een met}{Arthurs wapenteken}\\

\haiku{Laten we niet zo.}{ver uit elkaar gaan dat we}{er spijt van krijgen}\\

\haiku{Hij mishandelde,.}{haar zwaar omdat zij niet met}{hem mee wilde gaan}\\

\haiku{De jonkvrouw had een.}{groen kleed aan dat op vele}{plaatsen gescheurd was}\\

\haiku{hij wendde zijn paard;}{en liet het de diepte van}{de rivier peilen}\\

\haiku{De angst zal u om,.}{het hart slaan of ik schiet er}{mijn leven bij in}\\

\haiku{Volgens mij zult u.}{vandaag nog inzien dat u}{zich misdragen hebt}\\

\haiku{Een beroep op God.}{of zijn naastenliefde kon}{hem niet vermurwen}\\

\haiku{Ze waren allen.}{zeer verheugd dat ze veilig}{en wel bij hen was}\\

\haiku{Op de brug over de.}{slotgracht stond de kasteelheer}{met een groot gevolg}\\

\haiku{hij was immers de.}{vader van de ridder die}{dood op de baar lag}\\

\haiku{hij had zijn wapens;}{afgelegd en had alleen}{maar zijn kleren aan}\\

\haiku{In dat geval zou:}{her en der schande over hem}{gesproken worden}\\

\haiku{Zodra we buiten,.}{zijn moet de deur van de zaal}{gesloten worden}\\

\haiku{{\textquoteright} Aldus werden zijn.}{gedachten voortdurend heen}{en weer geslingerd}\\

\haiku{Immers, nergens in.}{de verre omtrek zult u}{onderdak vinden}\\

\haiku{Dat is echter een.}{lang verhaal waarmee ik u}{niet wil vermoeien}\\

\haiku{{\textquoteright} Heer Walewein stond.}{direct op en haastte zich}{naar het  raampje}\\

\haiku{Omdat hij hen op,;}{het spoor was was hij voor dag}{en dauw vertrokken}\\

\haiku{Bij deze goede.}{spijzen en drank vergaten}{ze even hun zorgen}\\

\haiku{U kunt namelijk.}{de weg all\'e\'en niet tot het}{einde afleggen}\\

\haiku{De vriendschap tussen.}{hen beiden zou niet door hem}{worden opgezegd}\\

\haiku{Ik ben daarom bang.}{dat we nooit aan de overkant}{zullen geraken}\\

\haiku{Dit is de duivel,.}{niet en evenmin is hij ooit}{in de hel geweest}\\

\haiku{Maar komt u binnen.}{en laten we mijn oom en}{mijn broer opzoeken}\\

\haiku{elke trede staat,.}{voor een dag een  week of}{wellicht wel een maand}\\

\haiku{hij stootte  zijn,.}{zwaard diep in de keel van het}{beest tot het hart toe}\\

\haiku{Als ik weet dat u,.}{dat doet kan ik met een}{gerust hart sterven}\\

\haiku{hij zelf was ontsnapt.}{en de koning van Ierland}{had hij in zijn macht}\\

\haiku{Nu zult u horen.}{wat er met de koning van}{Ierland gebeurde}\\

\haiku{Keie bijt hem toe dat:}{hij maar achter die rode}{ridder moet aangaan}\\

\haiku{Het tweede voorbeeld.}{betreft het motief van het}{geschonden gastrecht}\\
