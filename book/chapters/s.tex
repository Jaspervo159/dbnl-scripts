\chapter[33 auteurs, 7666 haiku's]{drieëndertig auteurs, zevenduizendzeshonderdzesenzestig haiku's}

\section{Maurits Sabbe}

\subsection{Uit: 't Pastorken van Schaerdycke}

\haiku{- Zij hebben mij ook,!}{allemaal helpers troosten}{mijn broers en zusters}\\

\haiku{Jan Potagie, die, ';}{den domme speelt maar het in}{t geheel niet is}\\

\haiku{- Ge hoort het, kind, dat,.}{moeten ze hebben en dat}{kan ik niet geven}\\

\haiku{Verlegen trok ze.}{haar tengere hand terug}{en zocht heen te gaan}\\

\haiku{Dat kind kan voor u... -... '}{geen kwaad doen En voor u kan}{dat kind geen goed doen}\\

\haiku{Hij voelde zich als,.}{een die wacht houden moest bij}{een dierbaren schat}\\

\haiku{een preekheerentaart,;}{met merg van ossen krenten}{en rozewater}\\

\haiku{Vlugger dan hij 't, '.}{verwacht had was de knop aan}{t bloeien gegaan}\\

\haiku{'t Was Jozijnken, '. -?}{die int geval een dol}{plezier had Waa-aar}\\

\haiku{Dankbaar zijn om 't,,... -?}{goede ja en niet te veel}{droomen Niet droomen}\\

\haiku{De ontgoocheling,.}{komt van de werkelijkheid}{niet van de droomen}\\

\haiku{Op knechts en meiden.}{kon ze niet rekenen om}{haar te vervangen}\\

\haiku{- Zoo is mijn broer, de,.}{jonkheer pensionaris}{juffrouw Jozijnken}\\

\haiku{Adhemar was al!}{smoorlijk verslingerd op die}{listige dante}\\

\haiku{Agla\"e greep naar de,.}{bel doch eensklaps kreeg ze een}{anderen inval}\\

\haiku{Fiete besteedde.}{haar laatsten zorgen aan den}{opschik van den tuin}\\

\haiku{En zoo kwamen ze:}{weer tot de bekommering}{van al hun dagen}\\

\haiku{- Ik ga, Fiete, ik......}{ga maar ik ben als van de}{hand Gods geslagen}\\

\haiku{het was, Jozijnken,.}{hier te zoeken zonder een}{aanwijzing of wenk}\\

\haiku{- Juist, daar gaan vele, '!}{schavuiten bij omdat ze}{ert veiligst zijn}\\

\haiku{s Pastorkens hart.}{schreide van weedom en zijn}{oogen waren vochtig}\\

\haiku{- Ja, en nu heb ik,,,?}{helaas mijn vrijheid verbeurd}{God weet voor hoe lang}\\

\haiku{De officier bleef.}{een lange poos bedremmeld}{en besluiteloos}\\

\haiku{Hij kon toch zoo maar.}{niet ineens het pastorken}{laten aanhouden}\\

\haiku{Dat ging ook niet, zij.}{was al de patronesse}{van de schrijnwerkers}\\

\haiku{Weet gij, eerwaarde,,?}{dat ons Jozijnken weg is}{spoorloos verloren}\\

\haiku{Met zijn gepeinzen.}{en gevoelens verlangde}{hij alleen te zijn}\\

\haiku{Eerst toen hij zelf wou,.}{gaan besloot de maarte hem}{te gehoorzamen}\\

\haiku{'t Was Coddebiers,.}{die hen in den donkeren}{avond kwam verrassen}\\

\haiku{- Fiete, zei toen de,!}{oude man we moeten aan}{onzen gast denken}\\

\haiku{Bij 't krieken van.}{den morgen schokkelde de}{wagen naar de stad}\\

\haiku{drong 't pastorken.}{aan om een einde aan die}{kwelling te maken}\\

\haiku{- Aan wie berouw heeft,.}{en weer goed wil leven schenkt}{Hij zijn genade}\\

\haiku{wees lijdzamig en;}{laat uw barmhartigheid zijn}{op alle vleesch}\\

\haiku{De waardinne houdt... -,;}{me gevangen Laat dit aan}{mij over Jozijnken}\\

\haiku{'t Duurde niet lang.}{of de uitvaartstemming was}{er heelemaal uit}\\

\haiku{Wat jammer, dat bij,!}{hier niet is hij zou wel goed}{meegedaan hebben}\\

\haiku{Wat zou ze hier nog,?}{te doen hebben indien hij}{haar ontnomen werd}\\

\haiku{- Weet ge, Coddebiers,?}{wat Jozijnken doen zal als}{ik er niet meer ben}\\

\haiku{Alles cirkelde,;}{om hem heen de sneeuwvelden}{en de hemelboog}\\

\section{Annie Salomons}

\subsection{Uit: Daadlooze droomen (onder ps. Ada Gerlo)}

\haiku{kan ik me jou nog.}{niet goed voorstellen tusschen}{al dat ethisch gedoe}\\

\haiku{Ze lachte, en die.}{lach streek de harde lijnen}{weg van haar gezicht}\\

\haiku{iederen dag bij ';}{t ontbijt wordt daar in den}{bijbel gelezen}\\

\haiku{Hij is goed voor me,,.}{geweest elken dag al die}{veertig jaren lang}\\

\haiku{{\textquoteright} verzwaarde ze weer, {\textquoteleft} ' {\textquotedblleft}{\textquotedblright}, ' {\textquotedblleft}{\textquotedblright}.}{dat ze welsja zegt als}{tneen moest wezen}\\

\haiku{Hij kwam dichter naast,;}{haar rijden en stak  zijn}{arm door den hare}\\

\haiku{Dat is eigenlijk,.}{de eenige plaats ter wereld}{waar ik me thuis voel}\\

\haiku{- toen ik ineens h\'a\'ar.}{zag staan in de rij met de}{andere menschen}\\

\haiku{haar mond trilde aan,;}{den eenen hoek en ze liep toch}{met de boekentasch}\\

\haiku{{\textquoteleft}Dat zullen wij nu,.}{ons heele leven blijven}{koningskinderen}\\

\haiku{Ze had tegenover;}{hem immers geen gedachten}{en meeningen meer}\\

\haiku{Nu zag je toch maar,!}{eens hoe ver je met zulke}{neuswijzigheid kwam}\\

\haiku{{\textquoteleft}Nou, maar toen ik met ',,.}{m zou gaan trouwen zei ik}{toch \'o\'ok niet waarom}\\

\haiku{Nu 't eenmaal zoover,.}{was had hij dan maar moed voor}{twee moeten hebben}\\

\haiku{en ik hoor hier, bij.}{de wijsheid en de sjawls en}{de rheumatiek}\\

\haiku{ze wist meteen, dat,.}{dat nu niet meer noodig was maar}{het roerde haar niet}\\

\haiku{knoeide zelf in een.}{laboratorium met}{allerlei proeven}\\

\haiku{En Amy was voor mij {\textquotedblleft}{\textquotedblright},...}{het meisje geweest van m'n}{jongensjaren af}\\

\haiku{ik voelde, dat ik,;}{nu wel sterven wou en ook}{wel eeuwig leven}\\

\haiku{Het was een angst voor,;}{ieder uur dat ik haar niet}{onder m'n oogen had}\\

\haiku{En toen heeft hij het,.}{opengemaakt en toen zat er}{een wollen das in}\\

\haiku{{\textquoteleft}Ik heb me toch niet,...}{op hem geabonneerd toen}{ik met hem trouwde}\\

\haiku{{\textquoteright} {\textquoteleft}Ja,{\textquoteright} zei ze vaag, en.}{de tranen raakten tot de}{randen van haar oogen}\\

\haiku{in den doodstillen.}{nacht bleven hun stappen nog}{een heel eind hoorbaar}\\

\haiku{{\textquoteleft}Laten we 't ons;}{allemaal zoo makkelijk}{mogelijk maken}\\

\haiku{Ze zal mij niet noodig;}{hebben gehad om in den}{hemel te komen}\\

\haiku{En zoo wachtend en.}{werkend was ze achter in}{de dertig geraakt}\\

\haiku{we aten lekker en,.}{volop en we zaten goed}{in onze spullen}\\

\haiku{alles wat er over,,.}{was werd aan haar besteed en}{m\'e\'er dan wat over was}\\

\haiku{en dat ze op een...,...;}{zomeravond samen en dat}{hij toen ineens zei}\\

\haiku{{\textquoteright} {\textquoteleft}O, maar dit w\`as toch,{\textquoteright}.}{heerlijk en romantisch vroeg}{de backfisch dringend}\\

\haiku{dat zie ik wel aan,}{uw oogen en die tram en het}{warme bed van straks}\\

\haiku{t voor jou zal zijn... ';}{t Zal  voor jou \'o\'ok wel}{de liefde wezen}\\

\haiku{Wat de eene mensch den,,.}{ander daarin aandoen kan}{onwaardig fnuikend}\\

\haiku{Ze wilde het zoo, ';}{duidelijk uiteen zetten}{waaromt niet m\'ocht}\\

\haiku{Een beetje verslapt...}{natuurlijk ieder mensch in}{daaglijkschen omgang}\\

\haiku{niet ziende bewoog;}{ze haar handen onzeker}{over het theegerei}\\

\haiku{De laatste jaren.}{van zijn leven bracht hij in}{het buitenland door}\\

\haiku{zonder behoefte,;}{om te bespiegelen om}{in verband te zien}\\

\haiku{naaktstudies kan je,,.}{dat noemen nou je kan er}{alles op studeeren}\\

\haiku{{\textquoteright} {\textquoteleft}Misschien ben je nog,{\textquoteright};}{te jong en te onstuimig}{zei hij voorzichtig}\\

\haiku{U begrijpt toch wel,.}{dat ik zoo iets niet in m'n}{kamer wil hebben}\\

\haiku{hij converseerde;}{met een chimpans\'e en een}{ziekelijken orang}\\

\haiku{de zon scheen er vol,.}{naar binnen en het raam was}{wijd open geschoven}\\

\haiku{{\textquoteleft}Kijk 's, ik vind het,;}{gezellig dat ik je ben}{tegengekomen}\\

\haiku{{\textquoteleft}Als jij met 'm uit, '.}{gaat zou ikm ook wel eens}{willen leeren kennen}\\

\haiku{en nu zeurde hij,.}{tien minuten over iets dat}{er niet op aan kwam}\\

\section{Anna de Savornin Lohman}

\subsection{Uit: Om de eere Gods}

\haiku{Dat oogenblik, het!}{had immers beslist over heel}{zijn jonge leven}\\

\haiku{{\textquoteleft}Ik hoop heel spoedig.}{mijn opwachting bij mevrouw}{te komen maken}\\

\haiku{- Mevrouw heeft het dus '.}{op h\'a\'ar geweten als zij}{t goede doel schaadt}\\

\haiku{- En ik vind het z\'o\'o,,.}{veel prettiger dat je nu}{met h\'a\'ar verkoopt Geert}\\

\haiku{Of misschien waren,;}{het \'o\'ok wel tranen die in}{haar oogen opwelden}\\

\haiku{Ver weg had hij het;}{socialistisch standpunt}{van zich gestooten}\\

\haiku{- - {\textquoteleft}Ja, Geert heeft al h\'e\'el!}{wat mee te beredderen}{in het huishouden}\\

\haiku{{\textquoteright} Nu eindelijk kon.}{dominee Pieter van der}{Grijp zijn hart luchten}\\

\haiku{- maar hij houdt je voor, -.}{den lap dat doet hij jullie}{meisjes allemaal}\\

\haiku{indien ze huilde}{was het omdat ze nu stond}{v\'o\'or de beslissing}\\

\haiku{{\textquoteleft}Christendom in de{\textquoteright}.}{praktijk van het leven steeds}{meer in te werken}\\

\haiku{dat zijn niet bijtijds.}{v\'o\'or de stemming thuis kunnen}{zijn aankondigde}\\

\haiku{hoe prettig ze het.}{vond nu met De Ekenhuize}{kennis te maken}\\

\haiku{van hem wilde ik, -,}{je juist verteilen dat hij}{min wordt bepaald min.}\\

\haiku{Zij deinsde vuurrood,;}{terug toen zij Govert-Jaap en}{Geerte zag zitten}\\

\haiku{{\textquoteright} {\textquoteleft}Mijn vrouw, juffrouw van,{\textquoteright}, -.}{der Grijp van de pastorie}{stelde Govert-Jaap voor}\\

\haiku{- Arme Dina liet.}{haar schaaltje bijna vallen}{van verlegenheid}\\

\haiku{- Sjoerd echter was nu,.}{al z\'o\'o ver gegaan Lizzy}{hoopte al z\'o\'o lang}\\

\haiku{Mama heeft het ook,.}{niets goed gevonden dat ik}{hem heb afgeschaft}\\

\haiku{- Ze stond op, wierp 't.}{telegram in een laadje}{van haar bureautje}\\

\haiku{de naijver der.}{leelijke op de mooie was}{daarbij te voelbaar}\\

\haiku{Maar iedereen noemt.}{hem immers zoo als er van}{hem gesproken wordt}\\

\haiku{Zeg toch aan meneer,{\textquoteright}, - -,,...}{dat we w\`achten beval Geert}{nerveus aan den knecht}\\

\haiku{het is 'n quaestie;}{van leven of dood voor dat}{Militaire Thuis}\\

\haiku{Maar ze had Sjoerd naast....}{zich aan tafel en hij was}{allerliefst vandaag}\\

\haiku{En daarbij keek hij -.}{brutaal langs haar figuur zoodat}{ze opnieuw bloosde}\\

\haiku{Hij moest nu voortaan, -.}{heel dikwijls komen net als}{vroeger met haar zijn}\\

\haiku{{\textquoteright} - - Toen, toen de deur zich,,:}{achter den knecht sloot zag hij}{zijn vrouw aan en vroeg}\\

\haiku{'t Was nu 'n veel.}{veiliger spelletje voor}{hemzelf dan vroeger}\\

\haiku{{\textquoteright} - Met hare moeder.}{had ze er een ernstige}{explicatie over}\\

\haiku{van de eerste kreeg ',.}{jet kapitaal van de}{laatsten de stemmen}\\

\haiku{Net goed{\textquoteright}, - zei ze met - {\textquoteleft} ' {\textquoteright} - - {\textquoteleft}?}{Schadenfreudeh\'a\'ar gun ik}{tWie gun je wat}\\

\haiku{weet je dan van niets, - -,!}{nog en jij die toch achter}{de schermen kunt zien}\\

\haiku{{\textquoteleft}O Verhaeghen, - voor,{\textquoteright}.}{dien ben ik niet bang haastte}{zij zich te zeggen}\\

\haiku{{\textquoteright} - - Maar haar klaagwoorden:}{wekten in hem een weerklank}{van verbittering}\\

\haiku{zij had ontvangen!}{van de oprechte vroomheid}{van Engelsch leven}\\

\haiku{- Ze verheugde zich,.}{op de mooie kleertjes die ze}{voor haar zou koopen}\\

\haiku{En dan kwam ze laat,,}{thuis met een opgewonden}{gelaat soms ook met}\\

\haiku{- - Want Govert-Jaap was o\'ok.}{in zijn huwelijksleven}{zoo gelukkig nu}\\

\haiku{Dat ze veel uit was.}{nam hij haar niet meer kwalijk}{zooals in het begin}\\

\haiku{- - U weet dat mevrouw?}{nu naar meneer en mevrouw}{Eduma de Witt is}\\

\haiku{- Je kunt er toch niet.}{den heelen avond naar zitten}{kijken zonder meer}\\

\haiku{{\textquoteright} - Eduma de Witt kneep.}{even zijn oogen dicht om het zich}{te herinneren}\\

\haiku{Govert-Jaap ontdooide.}{van het lieve gebroken}{kindergestamel}\\

\haiku{- -{\textquoteleft}D\'o\'or wien heb je?}{zoo ge{\"\i}nfluenceerd op}{de benoemingen}\\

\haiku{{\textquoteright} - - Gerty-ook zei,;}{woedend dat zij zich nergens}{meer zou vertoonen}\\

\haiku{want ik moet met hem.}{samen optreden in een}{tableau-vivant}\\

\haiku{{\textquoteleft}Tegen wie denk je,?}{dat je spreekt tegen je vrouw}{of tegen je meid}\\

\haiku{Het was of alles.}{tegen hem samenspande}{in den laatsten tijd}\\

\haiku{{\textquoteright} ... informeerde een, ':}{van de anderen metn}{hoopvol gezicht van}\\

\haiku{Hij wist het zoo goed:}{wat zijn streng geloof in dat}{opzicht leeraarde}\\

\haiku{- {\textquoteleft}Nu geeft God mij mijn,.}{\'e\'enen grooten wensch die nog}{van de aarde was}\\

\haiku{vanaf dien morgen,.}{waarop hij den anoniemen}{brief haar had getoond}\\

\haiku{- van je positie,,, - -.}{en je geld en je naam hield}{ik maar van jou niet}\\

\haiku{En dat wist je wel - -...{\textquoteright} {\textquoteleft}{\textquoteright}, -, - {\textquoteleft},.}{wist je welNeen zei hijdat}{wist ik niet Geerte}\\

\section{Nine van der Schaaf}

\subsection{Uit: Friesch dorpsleven uit een vorige tijd}

\haiku{Ik waarschuwde haar ',.}{v\'o\'ort nieuwe huis gebouwd}{werd maar zij lachte}\\

\haiku{Haar vader's vreemde.}{uitingen van de laatste}{tijd troffen haar hart}\\

\haiku{Doch zijn mond beefde.}{en hij kon zijn tranen niet}{geheel weerhouden}\\

\haiku{{\textquoteright} Natuurlijk ging hij, -?}{uit waarom zou hij op de}{Zondag thuisblijven}\\

\haiku{{\textquoteleft}Ik geloof niet dat, -}{Jelmer bij mij past en bij jou}{past hij nog minder}\\

\haiku{Hij had overnacht in,;}{de stad en was nu op de}{terugweg naar huis}\\

\haiku{vandaag en merkte.}{op dat ze te veel thuis zat}{in de laatste tijd}\\

\haiku{Doch deze keer leek ',.}{t haar moeilijker ze wist}{zelf niet recht waarom}\\

\haiku{Nauwelijks zag zij.}{hem in de donker en hij}{verdween geruischloos}\\

\haiku{Hij was opgeruimd;}{van aard en treurde niet om}{een meisje zooals Heerk}\\

\haiku{Er ontstond eenige;}{opgewondenheid toen het}{ding klaar bleek te zijn}\\

\haiku{Germen moest verteld '.}{hebben dat hij plan had heen}{te gaan uitt dorp}\\

\haiku{{\textquoteleft}Omdat ik ze niet{\textquoteright}.}{goed versta en zij lachen}{mij uit als ik praat}\\

\haiku{Heerk hoorde dat zij;}{gedeeltelijk sprak als de}{menschen in Manswerd}\\

\haiku{{\textquoteleft}Je ziet haar in lang,,{\textquoteright}.}{niet weer Heerk ze gaat morgen}{terug naar Manswerd}\\

\haiku{Hij gaf Iefke een,,}{hand wenschte haar goede}{reis zoo opgewekt}\\

\haiku{Wiebe en Jikke ', '.}{zaten opt dek daart}{nog weinig koel was}\\

\haiku{zij wenschte niet.}{in nauwer betrekking te}{komen met Leida}\\

\haiku{Toen ze zich voelden,.}{opgemerkt knikten ze en}{hervatten haar werk}\\

\haiku{Toen Iede thuiskwam ',.}{wast gesprek uit doch zij}{stond nog op het dek}\\

\haiku{HIJ kwam tegen de.}{winter thuis en terug in}{zijn oude leven}\\

\haiku{In deze tijd ving.}{hij aan in zijn landstaal}{verzen te schrijven}\\

\haiku{Enkele vrouwen.}{zag men wel in het deurgat}{van haar woning staan}\\

\haiku{Moeder is er boos, -{\textquoteright},,.}{om en ik ook zei hij half}{ernstig half in scherts}\\

\haiku{Hij hoorde de wind '.}{niet en door het raampje zag}{hijt hemelsblauw}\\

\haiku{Oene keek nu ook.}{meer in andere richting}{terwijl hij boomde}\\

\haiku{Hij voer voorbij de,.}{hut van oude Ulbe die}{nog wel slapen zou}\\

\haiku{- Het zou toch nog wel,.}{wat duren eer het zoover kwam}{troostte hem Oene}\\

\haiku{En zou er van zijn?}{geld iets overblijven als hij}{het hier instak}\\

\haiku{Doch zijn bezit was,.}{zoo weinig slechts enkele}{honderden guldens}\\

\haiku{Ook Heerk had thans voor,.}{hem iets dat hem vroeger nooit}{was opgevallen}\\

\haiku{Nee, van de dochter '.}{wist men geen kwaad ent was}{aan haar dat Heerk dacht}\\

\haiku{Maar de twee mannen.}{hadden daar ondervinding}{van en klaagden niet}\\

\haiku{Het water voor de.}{koffie ging overkoken en}{siste in het vuur}\\

\haiku{En uit de verte,;}{naderde een oude boer}{stooterig van gang}\\

\haiku{{\textquoteleft}Als de tram komt dan!}{gaan we alle dagen naar}{de school in Manswerd}\\

\haiku{Zij naderden in,:}{flinke vaart het dorp dat zich}{scheen uit te breiden}\\

\subsection{Uit: Heerk Walling}

\haiku{Ik waarschuwde haar ',.}{v\'o\'ort nieuwe huis gebouwd}{werd maar zij lachte}\\

\haiku{Haar vader's vreemde.}{uitingen van de laatste}{tijd troffen haar hart}\\

\haiku{Doch zijn mond beefde.}{en hij kon zijn tranen niet}{geheel weerhouden}\\

\haiku{{\textquoteright} Natuurlijk ging hij, -?}{uit waarom zou hij op de}{Zondag thuisblijven}\\

\haiku{{\textquoteleft}Ik geloof niet dat, -}{Jelmer bij mij past en bij jou}{past hij nog minder}\\

\haiku{Hij is zoo kwaad niet{\textquoteright},}{zei Oene en Harmke die}{niet meer antwoorden}\\

\haiku{Hij had overnacht in;}{de stad en was nu op de}{terugweg naar huis}\\

\haiku{vandaag en merkte.}{op dat ze te veel thuis zat}{in de laatste tijd}\\

\haiku{Doch deze keer leek ',.}{t haar moeilijker ze wist}{zelf niet recht waarom}\\

\haiku{Nauwelijks zag zij.}{hem in het donker en hij}{verdween geruischloos}\\

\haiku{Hij was opgeruimd,;}{van aard en treurde niet om}{een meisje zooals Heerk}\\

\haiku{Braaf dat je komt, want{\textquoteright},.}{je kunt mij juist even helpen}{antwoordde Germen}\\

\haiku{Er ontstond eenige;}{opgewondenheid toen het}{ding klaar bleek te zijn}\\

\haiku{Germen moest verteld '.}{hebben dat hij plan had heen}{te gaan uitt dorp}\\

\haiku{{\textquoteleft}Omdat ik ze niet.}{goed versta en zij lachen}{mij uit als ik praat}\\

\haiku{{\textquoteright} Heerk hoorde dat zij;}{gedeeltelijk sprak als de}{menschen in Manswerd}\\

\haiku{{\textquoteleft}Je ziet haar in lang,,.}{niet weer Heerk ze gaat morgen}{terug naar Manswerd}\\

\haiku{Hij gaf Iefke een,,}{hand wenschte haar goede}{reis zoo opgewekt}\\

\haiku{Wiebe en Jikke ', '.}{zaten opt dek daart}{nog weinig koel was}\\

\haiku{zij wenschte niet.}{in nauwer betrekking te}{komen met Leida}\\

\haiku{Toen ze zich voelden,.}{opgemerkt knikten ze en}{hervatten haar werk}\\

\haiku{HIJ kwam tegen de.}{winter thuis en terug in}{zijn oude leven}\\

\haiku{In deze tijd ving.}{hij aan in zijn landstaal}{verzen te schrijven}\\

\haiku{Enkele vrouwen.}{zag men wel in het deurgat}{van haar woning staan}\\

\haiku{Moeder is er boos, -{\textquoteright},,.}{om en ik ook zei hij half}{ernstig half in scherts}\\

\haiku{Hij sprak Anne in,.}{de volgende lente toen}{zij  trouwen ging}\\

\haiku{Hij hoorde de wind '.}{niet en door het raampje zag}{hijt hemelblauw}\\

\haiku{Hij beschouwde en,;}{herkende ze schoon ze ver}{verwijderd waren}\\

\haiku{- Het zou toch nog wel,.}{wat duren eer het zoover kwam}{troostte hem Oene}\\

\haiku{Doch zijn bezit was,.}{zoo weinig slechts enkele}{honderden guldens}\\

\haiku{Ook Heerk had thans voor,.}{hem iets dat hem vroeger nooit}{was opgevallen}\\

\haiku{Nee, van de dochter '.}{wist men geen kwaad ent was}{aan haar dat Heerk dacht}\\

\haiku{Maar de twee mannen.}{hadden daar ondervinding}{van en klaagden niet}\\

\haiku{Het water voor de.}{koffie ging overkoken en}{siste in het vuur}\\

\haiku{En uit de verte,;}{naderde een oude boer}{stooterig van gang}\\

\haiku{Maar er waren ook.}{oude eigenschappen hem}{blijven aankleven}\\

\haiku{{\textquoteleft}Als de tram komt dan!}{gaan we alle dagen naar}{de school in Manswerd}\\

\haiku{Zij naderden in,:}{flinke vaart het dorp dat zich}{scheen uit te breiden}\\

\section{Jeanne van Schaik-Willing}

\subsection{Uit: Witte veren}

\haiku{dat hij negentien,}{was had hij zich aangepast}{aan de maatschappij}\\

\haiku{Ten aanzien van dit.}{laatste geheim wist hij niet}{van marchanderen}\\

\haiku{Ze was het enige.}{heel jonge meisje onder}{de aanwezigen}\\

\haiku{Zonder antwoord af,.}{te wachten barstte ze los}{in een schorren lach}\\

\haiku{Daarginds is je stoel,{\textquoteright},.}{Minny zei ze en wees naar}{een stoel verder weg}\\

\haiku{{\textquoteleft}Geen gekheid,{\textquoteright} zei hij, {\textquoteleft},.}{Geen gekheid morgen gaan wij}{twee\"en wandelen}\\

\haiku{Ook Mia was, na de,.}{nachtrust de opwinding van}{den avond te boven}\\

\haiku{Maar Mia roerde het,.}{chapiter niet aan ze kon}{het niet aanroeren}\\

\haiku{H\'a\'ar voeten zouden.}{bruin en stevig zijn als van}{een gezond jong dier}\\

\haiku{Ach, een droppel bloed,.}{verscheen aan haar vinger ze}{stak hem in haar mond}\\

\haiku{Hoogmoed, hoogmoed, ja,.}{waar hij doorheen moest om straks}{de straf te smaken}\\

\haiku{De kreet der meeuwen,.}{kwam nu uit een wereld die}{te wijd voor haar was}\\

\haiku{Zelfs trok zijn ooglid,.}{even omhoog zij zag zijn oog}{dicht bij het hare}\\

\haiku{Was niet juist zij de?}{verleidster met onschuld als}{verleidingsmiddel}\\

\haiku{je ziet me wel weer{\textquoteright}.}{eens terug in een wolk van}{parfum naar buiten}\\

\haiku{Zij, Graminska van.}{zich zelf was een goeie vriendin}{van mij uit Mexico}\\

\haiku{En dat wie den dood.}{niet riskeert het leven niet}{waard is te leven}\\

\haiku{Zo, nu laat ik je.}{even alleen om met elkaar}{kennis te maken}\\

\haiku{Deze liet op haar.}{beurt het document in de}{eigen tas glijden}\\

\haiku{Het vreemde was, dat,}{ze over den achternaam niet}{nadacht wat ze zocht}\\

\haiku{Het enige antwoord.}{w-as een hartstochtelijk}{schudden van het hoofd}\\

\haiku{{\textquoteright} Ambroise keerde:}{zich met het gezicht naar den}{muur en fluisterde}\\

\haiku{{\textquoteright} lachte madame, {\textquoteleft}.}{Gaillardjij bent en je blijft}{onverbeterlijk}\\

\haiku{Zelfs vermoedde zij,.}{dat hij minder ziek  was}{dan hij voorwendde}\\

\haiku{Marie wilde het,.}{ook niet langer negeren}{ze was zeer ontroerd}\\

\haiku{{\textquoteleft}Niet d\'a\'arom,{\textquoteright} zei hij nog.}{eens binnensmonds en wendde}{beiden zijn rug toe}\\

\haiku{Rembrandt's hunkeraar.}{rees overeind en werd man van}{de wereld uit 1947}\\

\haiku{Dat wil niet zeggen,.}{dat hij cynisch met deze}{vriendinnen omging}\\

\haiku{{\textquoteleft}Maar kijk toch eens, wat,!}{een schone krullekens het}{lijkt wel zuiver goud}\\

\haiku{Met een deel van haar.}{moeders versterf nam Mia den}{inventaris over}\\

\haiku{Dan trachtte zij zich.}{uit te putten in kleine}{vriendelijkheden}\\

\haiku{{\textquoteleft}Weet je wel, dat dit,?}{een prachtig dingetje is}{in \'e\'en woord prachtig}\\

\haiku{Dit was de enige.}{bijdrage tot het gesprek}{van Chamotte}\\

\haiku{De deuren naar een,.}{tuintje stonden open daarin}{bloeiden seringen}\\

\section{Margo Scharten-Antink}

\subsection{Uit: Catherine}

\haiku{achter zich schuurden,,.}{zijn handpalmen steunzoekend}{tegen den schachtmuur}\\

\haiku{Hij had dadelijk,.}{wel geroken dat er wat}{te verdienen viel}\\

\haiku{Nou was het alleen,;}{nog de vraag maar hoe het er}{van binnen uitzag}\\

\haiku{t Was een rare,,;}{troep een raar zoodje waar die}{kapel aan hoorde}\\

\haiku{en hij had drommels,...}{goed begrepen dat het niet}{van de droogte kwam}\\

\haiku{{\textquoteright} dacht dan de man weer,:}{maar hij kon toch zijn oogen niet}{van haar afhouden}\\

\haiku{En ik kan zooveel,,!}{jongens krijgen als ik wil}{maar ik wil ze niet}\\

\haiku{Ik wil het nog eens,;}{op mijn gemak bekijken}{een kwartiertje maar}\\

\haiku{Ze zijn allemaal,?}{zot op die kapel daar heb}{je geen begrip van}\\

\haiku{hun gaan, samen, naar,;}{boven in de warmte waar}{hij over gevloekt had}\\

\haiku{Een gevoel van groote.}{ellende was eensklaps op}{haar neergedonkerd}\\

\haiku{stilletjes, stikem......}{gingen ze een andermans}{bidhuis nabouwen}\\

\haiku{die kapel was ook,.}{maar larie dat had ze dien}{morgen wel geleerd}\\

\haiku{En de kapel, die,?...}{kapel waar ze altijd zoo}{trotsch op was geweest}\\

\haiku{Ze gaf wel niets om,......}{dien jongen ze had nooit wat}{om hem gegeven}\\

\haiku{hoe de beeldjesman,,......}{zelf bij hen bij de Daene's}{was komen kijken}\\

\haiku{{\textquoteright} Wat konden al die?}{schreeuwende menschen in hun}{huis haar nu schelen}\\

\haiku{Reeds draafden haar de;}{rappe voeten terug in}{instinctieve vlucht}\\

\haiku{niets van buiten af.}{liet ze als werkelijkheid}{tot zich doordringen}\\

\haiku{De week was al op,, -.}{de helft het werd Donderdag}{de man kwam nog niet}\\

\haiku{een komplot, negen,;}{kerels en haar grootmoeder}{samen tegen \'e\'en}\\

\haiku{ze waren wel geen......}{femelaars en de duivel}{kon ze niet schelen}\\

\haiku{t Wordt morgen vier,}{weken dat ik die sleutel}{voor je gappen moest}\\

\haiku{door de zand-rulte.}{ging het den karreweg af}{naar de kapel toe}\\

\haiku{{\textquoteright} Catherine had.}{het stemgeluid van Lambert}{den voerman herkend}\\

\haiku{zij verzon zelfs niet.}{bij welk dorp zij wel zoo dicht}{genaderd kon zijn}\\

\haiku{Later nog kwam zij,;}{door een streek van steenschachten}{die zij niet kende}\\

\haiku{Een oogenblik dacht.}{het kind om er heen te gaan}{en eten te vragen}\\

\haiku{Toen, met de leege schaal,,:}{op schoot zittend wat loom-warm}{en voldaan dacht ze}\\

\haiku{Meteen draafde ze,.}{al den grintweg af die naar}{de heirbaan leidde}\\

\haiku{Zij ging weer naar den,,.}{voorkant wou d\'a\'ar wachten of}{er ook een uit kwam}\\

\haiku{De kerels, in een,.}{kring om hem heen waren nu}{meester van het licht}\\

\haiku{Barsch voorbijgaand,,.}{wat grom-vloekend lieten}{zij het kind met rust}\\

\haiku{Later zwaar-plofte...}{beneden het rondstappen}{van den bultenaar}\\

\haiku{Halsstarrig lag zij;}{maar met het gezicht naar den}{bedstee-wand gekeerd}\\

\haiku{in twee sprongen was,,.}{ze bij de deur gluur-tuurde}{door een kier angstbleek}\\

\haiku{Zij luister-loerde,,.}{naar een aanduiding raadde}{ze lokte ze uit}\\

\haiku{Ze wist nu, dat het......}{schilderij in de kapel}{van Waramme hing}\\

\haiku{beefhandend liet ze.}{zich zoo maar het vaatwerk uit}{de vingers glippen}\\

\haiku{{\textquoteright} Gansch niet overrast zag,,;}{zij hem aan zij wist wel dat}{dit nu komen moest}\\

\haiku{zich afjakkerend,......}{voor haar kinders geranseld}{misschien bovendien}\\

\section{Carel Scharten en Margo Scharten-Antink}

\subsection{Uit: De nar uit Maremmen. Deel 1: Massano}

\haiku{Carel Scharten en,.}{Margo Scharten-Antink De}{nar uit Maremmen}\\

\haiku{Samen stonden zij.}{bij de deur-opening naar}{het groote dakterras}\\

\haiku{Lorenzo keek niet,.}{on vermaakt maar Ottavio}{trok de schouders op}\\

\haiku{Je moest trouwens zelf{\textquoteright},.}{ook naar Florence komen}{viel Lorenzo bij}\\

\haiku{'k Zou er z\'o\'o een,.}{voor je kunnen huren een}{vorstelijk atelier}\\

\haiku{En daar zou hij niet,,,....}{op gesteld zijn omdat na}{twaalf jaar hij Pimpia}\\

\haiku{Ik voel n\`og meer voor!}{de nieuwe vriendschap van de}{Rooien en Rome}\\

\haiku{altijd nog wel een {\textquotedblleft}{\textquotedblright}....}{of andereMadonna}{van het Kattegat}\\

\haiku{{\textquoteright} {\textquoteleft}Een andere maal,{\textquoteright},;}{amico mio viel Renato}{hem in de rede}\\

\haiku{De man achter haar,.}{trok zijn schouders op draaide}{zich weer naar zijn plaats}\\

\haiku{In zijn stompen muil,,;}{bleek-roze en nat spalkten}{kwaad de neusgaten}\\

\haiku{Hij stak zijn schetsboek,.}{weg wilde het koord om zijn}{middel los knoopen}\\

\haiku{{\textquoteleft}Maar u heeft toch niet {\textquotedblleft},{\textquotedblright}?}{gel\'o\'ofd in dien nieuwenChristus}{Leider en Richter}\\

\haiku{{\textquoteleft}Hoeveel merken zijn?}{er vandaag wel gezet op}{uw mercatura}\\

\haiku{De markies zette,,.}{zijn lorgnet op bezag het}{aandachtig knikte}\\

\haiku{Hij zat tegenover,.}{mij nog dichter dan u nu}{tegenover mij zit}\\

\haiku{David was door de {\textquotedblleft}{\textquotedblright};}{bevolking d\'a\'ar het eerstde}{heilige genoemd}\\

\haiku{David, ja, die werd,,.}{blindelings gevolgd en al}{wat hij deed was goed}\\

\haiku{Maar het was, of het.}{dezelfde geschriften niet}{meer voor mij waren}\\

\haiku{Zijn gastheer had het.}{gevoeld en streek wat beschaamd}{door zijn witten baard}\\

\haiku{voor het vervangen,.}{der geheime biecht door de}{openlijke minder}\\

\haiku{was het niet, of die,;}{op dit oogenblik z\`ag wat}{hij niet had gezien}\\

\haiku{Toch was ik toen al,.}{zesentwintig en sinds drie}{jaar in Florence}\\

\haiku{Ik holde terug.}{om ze tegen te komen}{aan de wegwending}\\

\haiku{in mijn oog zijn zijn....}{jongeren nog veel mooier}{geweest dan hij zelf}\\

\haiku{in het kerkje van....}{San Giorgio een vijfluik van}{Andrea di Niccol\`o}\\

\haiku{Maar wat is dat, pro,?}{fessore de Madonna}{van het Kattegat}\\

\haiku{Uw portret heb ik{\textquoteright},.}{gezien in Florence ging}{Sergio vertellen}\\

\haiku{{\textquoteright} Een uur lang zaten.}{de oude schilder en de}{jonge te werken}\\

\haiku{- had van den eersten:}{dag af de bewondering}{van Sergio gewekt}\\

\haiku{{\textquoteright} {\textquoteleft}Nou{\textquoteright}, zei Renato, {\textquoteleft},,:}{wie weet wat ik doe  maar}{op \'e\'en voorwaarde}\\

\haiku{{\textquoteright} dacht Renato, die - {\textquoteleft}?}{Mastropieri niet kende}{zit hem d\'a\'ar de kneep}\\

\haiku{en de politie.}{hield zich aan de functie van}{verkeers-agent}\\

\haiku{En de paarden zelf,,,.}{mooie rasbeesten die liepen}{dat het een lust was}\\

\haiku{Het handgeklap en;}{het getrappel ratelde}{tien minuten lang}\\

\haiku{dien middag, op den,....}{terugweg zouden ze er}{nog eens door moeten}\\

\haiku{Maar Renato zag,.}{wel hoe begeerig hij naar}{het krabbeltje was}\\

\haiku{Met welk een sierlijk!}{gemak maakte die jongen}{zich van hem  los}\\

\haiku{Hij zette zich voor.}{een klein caf\'e en dronk er}{een zwarte koffie}\\

\subsection{Uit: De nar uit Maremmen. Deel 2: Florence, de drie blinden}

\haiku{Carel Scharten en,.}{Margo Scharten-Antink De}{nar uit Maremmen}\\

\haiku{Hij zit sinds een week...!}{op wat hij zijn atelier noemt}{een ouwe toren}\\

\haiku{Wat een zotte vent, -,...}{en je moest erkennen dat}{ze gelijk hadden}\\

\haiku{Don Pompeo zou wel,.}{zeggen dat de logica}{ver te zoeken viel}\\

\haiku{{\textquoteright} Een diepe zucht van,,.}{verluchting onder naast hem}{ontroerde zijn hart}\\

\haiku{{\textquoteright} Met groote, ernstige,.}{oogen keek Silvio om alles}{goed te begrijpen}\\

\haiku{Renato, die het,:}{kind meende te begrijpen}{waarschuwde zachtjes}\\

\haiku{En toen hij de deur,...}{opendeed zat kwispelstaartend}{Brisc om het hoekje}\\

\haiku{Er hing een fijne,,.}{zoete geur van de witte}{wassige bloesems}\\

\haiku{'t Geneerde hem,,.}{die pet die zich met hem scheen}{bezig te houden}\\

\haiku{dan, hartelijk en,.}{gezellig deed hij een soort}{familierelaas}\\

\haiku{vroeg Niccolini,.}{toen zij weer stonden in de}{deur van het atelier}\\

\haiku{{\textquoteright} Tegelijkertijd.}{werd hij zich bewust van een}{zekere schaamte}\\

\haiku{Ik zeg alleen, dat;}{hij geen quitantie's geeft en}{geen boeken laat zien}\\

\haiku{En voor jou, voor ons,!}{alle drie was het een kans}{op verkoopen meer}\\

\haiku{{\textquoteleft}O, als je er met...{\textquoteright};}{Ottavio over spreken moet}{hoonde Lorenzo}\\

\haiku{Je hebt behoefte,,}{om je uit te spreken zeg}{je en je vertrouwt}\\

\haiku{{\textquoteright} {\textquoteleft}Vooruit Sandro, en,{\textquoteright}.}{waardeer je goed gesternte}{schertste de markies}\\

\haiku{de natuurlijke,...}{de onvermijdelijke}{strijd om het bestaan}\\

\haiku{Ik zie ons nog, na,.}{lange dagen van onrust}{als er zoo'n kaart kwam}\\

\haiku{Een uur later riep:}{soms mijn vader plotseling}{met een heete stem}\\

\haiku{{\textquoteleft}en wij in ons huis,{\textquoteright}.}{met centrale verwarming}{zei ze vol schaamte}\\

\haiku{{\textquoteright} - Dat is zeker, de!}{oorlog heeft onze zielen}{niet weinig gestaald}\\

\haiku{Twee jaar nadien, toen,...}{ik achttien was geworden}{ben ik ook gegaan}\\

\haiku{Intusschen ontsloeg!}{Nitti de deserteurs uit}{de gevangenis}\\

\haiku{{\textquoteright} {\textquoteleft}Over me zoon,{\textquoteright} deed Pia,.}{gesloten met een blik vol}{argwaan naar Sandro}\\

\haiku{Doch middelerwijl.}{hoorde hij zachte stappen}{de zaal verlaten}\\

\haiku{Het schilderijtje,,:}{dat zij had meegebracht bleek}{een klein stilleven}\\

\haiku{Nog niet zoo heel lang,,...}{geleden is geloof ik}{zijn vrouw gestorven}\\

\haiku{Ottavio ging naar.}{voren en zond het jonge}{mensch op de taxi-jacht}\\

\haiku{t personeel was,...}{naar bed en de menschen zelf}{waren in de stad}\\

\haiku{Nooit, dacht zij, had zij.}{haar vader zoo begrijpend}{en lief gevonden}\\

\haiku{eigenlijk alleen.}{om een afspraak te maken}{voor Gian Carlo}\\

\haiku{Bloeide die jongen?}{daar niet zelf als een bloem van}{blakende jonkheid}\\

\haiku{{\textquoteleft}Kom mee, kom mee,{\textquoteright} drong,.}{Renato terwijl hij hun}{voorging met de kaars}\\

\haiku{{\textquoteright} riep hij plotseling.}{met een luide stem door de}{nachtstille kamer}\\

\haiku{Kon het dan niet \'e\'en,,?}{uur zijn geweest of half twee}{dat ze hoorde slaan}\\

\haiku{Want Pia, dat wist hij,,}{uit de zes maanden dat ze}{nu bij hem diende}\\

\haiku{- Nee... zei de man, - dat,...}{beetje gerochel dat was}{al van tien jaar her}\\

\haiku{een man met altijd,,...... {\textquoteleft}}{bronchitis een vrouw die nooit}{thuis is en Gino}\\

\haiku{'t Is al beroerd,...}{genoeg dat je vader den}{jongen schuldig denkt}\\

\haiku{Hij voelde zich als,.}{verlamd na de opwinding}{der laatste dagen}\\

\haiku{Zelfs de marchese.}{Niccolini kwam met een}{dergelijk voorstel}\\

\haiku{en dat hier al zijn!}{uitgangetjes met Silvio}{had opgeluisterd}\\

\haiku{Gian Carlo kwam,.}{hem halen in den auto}{zorgzaam als altijd}\\

\haiku{Een boerenmeisje,;}{opende het hooge hek door twee}{steenen leeuwen bewaakt}\\

\haiku{{\textquoteright} En Renato zag,:}{in dat \'e\'en offer te groot}{kan zijn voor een mensch}\\

\haiku{de twee voornaamste;}{theaters van Florence}{waren gesloten}\\

\haiku{De opera, dat was,.}{het wat hij in Massano}{altijd had gemist}\\

\haiku{een afwerende.}{vijandelijkheid lag er}{over heel haar wezen}\\

\haiku{Ja, natuurlijk, ze....}{zullen op \`ons wel weer de}{verdenking gooien}\\

\haiku{{\textquoteleft}Als u me soms ook,!}{niet meer vertrouwt ga ik net}{zoo lief direct heen}\\

\haiku{Ze greep Renato's:}{beide handen en met haar}{stem nog vol tranen}\\

\haiku{{\textquoteright} riep Renato, {\textquoteleft}heeft?}{hij dat gestolen goed bij}{jullie gevonden}\\

\haiku{'t k\`an toch alles,{\textquoteright}, {\textquoteleft} '....}{zoo wezen drong Pia opnieuw}{mijn man zegtt ook}\\

\haiku{Hier was dan toch \'e\'en....}{heete vlam uit het Rijk van}{de Liefde verdwaald}\\

\haiku{De agent, die Gino....}{Gori herkend zou hebben}{op de motorfiets}\\

\haiku{{\textquoteleft}Zie morgen maar eens,, -....}{wat je doen wilt ik kan niet}{voor elven hier zijn}\\

\subsection{Uit: De nar uit Maremmen. Deel 3: Naar de eeuwige stad}

\haiku{Carel Scharten en,.}{Margo Scharten-Antink De}{nar uit Maremmen}\\

\haiku{De geelgrijze kuif}{van Sor Agostino trilde}{driftig tusschen}\\

\haiku{{\textquoteright} Renato moest even:}{de wenkbrauwen fronsen om}{zich voor te stellen}\\

\haiku{een harde trek lag,.}{er om dien mond zelfs als die}{mond van liefde sprak}\\

\haiku{{\textquoteright} vroeg een man met een, {\textquoteleft}....}{doodsbleek gezichtik kon ze}{niet onderscheiden}\\

\haiku{{\textquoteright} {\textquoteleft}Twee legers in een,{\textquoteright}, {\textquoteleft}....}{land zei domp een anderhoe}{moet dat afloopen}\\

\haiku{De regen kwam bij.}{stroomen neer en gudste en}{droop langs de ruitjes}\\

\haiku{{\textquoteleft}weet u nog die keer?}{van de mercatura op}{de Villa Magna}\\

\haiku{{\textquoteleft}Als er eten is voor,,!}{negen is er ook eten voor}{veertien zei de vrek}\\

\haiku{er bleek v\'o\'or vijf uur.}{in den morgen geen trein naar}{Florence te zijn}\\

\haiku{Bij elk portier stond,.}{een zwarthemd op schildwacht het}{geweer in den arm}\\

\haiku{de koning deed hem,.}{naar Rome ontbieden om}{hem te raadplegen}\\

\haiku{Zij staken schuin het,.}{plein over waar het druk was van}{de Dinsdagsche markt}\\

\haiku{En opnieuw had hij,.}{een ingeving die hem stil}{maakte van binnen}\\

\haiku{'t Is al zooveel,!}{jaren geleden dat ik}{het laatst geskied heb}\\

\haiku{Wij dachten, dat hij,,.}{h\'e\'erschen zou rechtvaardig maar}{ongenadig streng}\\

\haiku{H\'e\'el interessant,,,!}{zeggen ze moderne kunst}{zooals je hier nooit ziet}\\

\haiku{Een antwoord, dat hij,.}{had willen geven vaagde}{weg van zijn lippen}\\

\haiku{Verscheidene uren,,.}{dezen nacht had hij erover}{wakker gelegen}\\

\haiku{zij behandelen,,;}{hem slecht den nietsnut die hun}{genadebrood eet}\\

\haiku{zwijgt tegen mij, zijn,,....}{verdediger tegen den}{rechter tegen U}\\

\haiku{Renato voelde.}{zich als gedreven naar de}{woning van zijn zoon}\\

\haiku{{\textquoteleft}Het proces is weer,{\textquoteright}.}{verdaagd tot over drie weken}{zei ze gelaten}\\

\haiku{- Was zij het, die zoo?}{plotseling zijn gedachten}{verhelderen deed}\\

\haiku{{\textquoteleft}Kijk eens, Aurora,{\textquoteright}, {\textquoteleft},}{zei hijje zegt dat Silvio}{geen band meer tusschen}\\

\haiku{{\textquoteright} Aurora keek hem.}{met een wonderlijken blik}{van overwinning aan}\\

\haiku{Daar had hij een heel,....}{regiment voor zich staan en}{hij had niet gezien}\\

\haiku{Plotseling ging zijn.}{blik naar het portret aan den}{bespinragden wand}\\

\haiku{{\textquotedblleft}Weest niet bang voor mij,,....}{eerwaarde Senatoren}{omdat ik paardrijd}\\

\haiku{Maar de jeugd is een,...}{goddelijk kwaad waarvan men}{elken dag geneest}\\

\haiku{{\textquoteright} Renato zat met.}{gefronste wenkbrauwen voor}{zich heen te kijken}\\

\haiku{Met een mat lachje;}{had hij het vergrootglas in}{ontvangst genomen}\\

\haiku{Vele weken bleef.}{Renato opgesloten}{binnen zijn arbeid}\\

\haiku{en.... u en ik, als,....}{we niet vertellen wat we}{niet kwijt willen zijn}\\

\haiku{{\textquoteleft}verzwijgen is soms;}{een bewijs van waardigheid}{en zelfbeheersching}\\

\haiku{doch bij zoov\'e\'el al had;}{de ijdele man zich}{moeten neerleggen}\\

\haiku{Groepen menschen, langs,.}{de Arnokade scholen}{bangelijk bijeen}\\

\haiku{Uit een groep naar huis,,:}{haastenden die hij langs kwam}{ving Renato op}\\

\haiku{Maar de geestdrift van....}{zijn jongetje liet hem toch}{niet onverschillig}\\

\haiku{{\textquoteright} {\textquoteleft}Vraag dat maar eens aan,....}{een appeltje dat al ver}{den winter in is}\\

\haiku{Doet u maar netjes!}{uw nieuwe zwarte pak van}{onze bruiloft aan}\\

\haiku{Ik wil, dat ons volk.}{weer een landbouwend volk wordt}{bij uitnemendheid}\\

\haiku{Groote werken werden,.}{alom ondernomen op}{velerlei gebied}\\

\haiku{Een vrede lag over,;}{deze trekken gestreken}{die even droevig scheen}\\

\haiku{Doch de mond was nog,.}{altijd niet de mond die hem}{bevredigen kon}\\

\haiku{K\`on die Liefde niet?}{uitgesproken worden met}{menschelijken mond}\\

\haiku{{\textquoteleft}En ik dacht nog wel,!}{dat er geen mysterie was}{in deze trekken}\\

\haiku{- waarom was hij voor?}{deze gelegenheid zoo}{ijselijk precies}\\

\haiku{En aan den einder,,.}{als op een heuveltroon het}{ivoorgele Fiesole}\\

\subsection{Uit: 't Geluk hangt als een druiventros}

\haiku{Jawel, jawel,{\textquoteright} had, {\textquoteleft};}{Angelo gezegddat's}{braaf uitgerekend}\\

\haiku{{\textquoteright} {\textquoteleft}De Melli's staan geen,{\textquoteright}.}{zoon af voor een veermansdienst}{zei Filippo fel}\\

\haiku{Nu vermengde zich,;}{de vocht die opsteeg met de}{nevels die daalden}\\

\haiku{{\textquoteleft}Waarom hebben zij, '?}{mij maar niet even geroepen}{int voorbijgaan}\\

\haiku{En toen de overgang,...;}{dan voorgoed vast stond was hij}{meteen maar hertrouwd}\\

\haiku{En nou kwam hij hem...}{nog zijn zoon afhalen voor}{een veermanspostje}\\

\haiku{Palmira was tot;}{deze vertrouwelijkheid}{nooit toegelaten}\\

\haiku{{\textquoteleft}Bifoli is toch......}{in zijn ongelijk laat hij}{ze binnen vragen}\\

\haiku{Waarom konden ze?...}{hem en zijn twee jongens hier}{niet met rust laten}\\

\haiku{III Dien avond werd het.}{een luidruchtiger maaltijd}{nog dan gewoonlijk}\\

\haiku{een getal, waar, bij,;}{het lezen in de courant}{zijn aandacht op viel}\\

\haiku{het nummer van de,;}{locomotief die zijn trein}{naar Florence trok}\\

\haiku{- Wat w\`as Florence!}{toch heerlijk op zoo'n mooien}{December-dag}\\

\haiku{ik heb honger... en,,?}{de Cavaliere die mijn}{gast is zeker ook}\\

\haiku{{\textquoteleft}Ja, nat\'u\'urlijk was,...}{het noodig dat de jongste van}{Rovai veerman werd}\\

\haiku{tienduizend lire,......}{een ronde som die overschrijdt}{je dan ook niet meer}\\

\haiku{van de vierhonderd:}{lire had hij nu al meer}{dan de helft besteed}\\

\haiku{hij had toch voor h\`en,!}{\'o\'ok zijn best gedaan en aan}{alle drie gedacht}\\

\haiku{{\textquoteright} Voor geen geld zou hij;}{zijn ouden vriend over Grassi}{hebben gesproken}\\

\haiku{Goede reis, goede,{\textquoteright};}{reis knikkebolde Carlo}{een beetje onthutst}\\

\haiku{Hij zou toch niet op?}{een dadelijk offer van}{hun kant aandringen}\\

\haiku{Hun eigen landwijn,;}{was onverbeterlijk in}{heel de streek vermaard}\\

\haiku{het eenige dat haar...}{nog een bevrijding was uit}{dit valsche leven}\\

\haiku{eindelijk moest ook.}{de Casa Rovai van de}{hand worden gedaan}\\

\haiku{hij ergerde zich;}{aan het gemis aan goeden}{toon en aan doorzicht}\\

\haiku{De maan stond als een;}{goud-witte schijf aan den}{beslagen hemel}\\

\haiku{Langs en boven en.}{voor hem wemelden de wit}{betroste twijgen}\\

\haiku{{\textquoteleft}het vogeltje, dat,.}{in den spiegel kijkt{\textquoteright}15 zoo heet}{het in den volksmond}\\

\haiku{Domenico en.}{Guido bekwamen niet van}{hun verwondering}\\

\haiku{Door dit gevaar had,!}{hij dat lieve fiere kind}{moeten heenbrengen}\\

\haiku{Maar een oogenblik,:}{later was hij toch gevleid}{als Angelo prees}\\

\haiku{- Was dat de jongste, -?...}{Signorina Sassetti}{of was zij het niet}\\

\haiku{Toch had de Conte...}{bij zijn vertrek 20 L. aan}{de meid gegeven}\\

\haiku{Nee, nee, om Godswil,...}{geen verplichtingen meer aan}{dien ellendeling}\\

\haiku{Was zij het niet, die?}{ook nu weer de hypotheek}{wilde verhoeden}\\

\haiku{{\textquoteleft}In geen acht jaar is;}{Filippo over den drempel}{van mijn huis geweest}\\

\haiku{Ga je beneden?}{niet even Virginia en de}{kinderen groeten}\\

\haiku{{\textquoteright} Francesca, bij,:}{diergelijke uitvallen}{temperde met schrik}\\

\haiku{Toch woog hem nog meer,.}{het geld dat zij van Aldo}{genomen hadden}\\

\haiku{Maar toen zij hem den,... {\textquoteleft}}{vierden dag \'o\'ok nog misten}{gingen ze zoeken}\\

\haiku{Eindelijk vonden,,;}{ze hem achter het vierde}{vat in den kelder}\\

\haiku{Tallooze dingen moesten,.}{er gebeuren over en door}{en ondanks elkaar}\\

\haiku{Op de terugvaart,,:}{van halfweg het water riep}{zonnig Silvano}\\

\haiku{En de kinderen,,;}{den ganschen middag waren}{er niet weg te slaan}\\

\haiku{tersluiks hadden zij:}{er geplukt en de korrels}{gepeld en geproefd}\\

\haiku{In den namiddag;}{kwam ook de Signorina}{Giselda kijken}\\

\haiku{Ze hadden wel een,,}{mud graan verloren wel twee}{mudden misschien z\'o\'o}\\

\haiku{Maar hoe dan ook, het '.}{ging hem opt oogenblik}{niet naar den vleesche}\\

\haiku{Eerst, zoometeen, een;}{kinawijntje met ijs en}{spuitwater drinken}\\

\haiku{maar vlekkerig rood.}{zag zijn gelaatskleur en zijn}{oogen stonden troebel}\\

\haiku{Omdat zij van d\'at,!}{geld het zekerst was dat hij}{het zou herstellen}\\

\haiku{Filippo lei zijn,.}{pijpje naast zich neer sloot de}{handen om zijn knie}\\

\haiku{Het hardsteenen perron.}{voor de hoofddeur blaakte wit}{in de morgenzon}\\

\haiku{die sloofde zich uit,...?}{om van wijn en graan dat geld}{te halen en dan}\\

\haiku{in de schaduwbocht.}{der bijgebouwen toefde}{zij besluiteloos}\\

\haiku{Zij ging zitten op,.}{het bankje dat achter de}{dichte deurhelft stond}\\

\haiku{Zij lette weinig,.}{op het meisje merkte niets}{bizonders aan haar}\\

\haiku{{\textquoteright} {\textquoteleft}Geen wonder, dat hij,{\textquoteright};}{nog niet uitgeslapen is}{zei Emilia bitter}\\

\haiku{En opeens, als met,.}{een slag van stilte was het}{hagelen gedaan}\\

\haiku{Voor 't oogenblik}{waren de vijgen en de}{late perziken}\\

\haiku{- Zondag ging zij naar,,...}{den Incontro biechten en}{bidden boete doen}\\

\haiku{Liet Guido met een...}{enkel woord zijn komst op dien}{dag voorbereiden}\\

\haiku{En van den vroegen:}{morgen af doorvorschten}{Ubaldo's oogen het land}\\

\haiku{Hij heeft het lesje,,!}{vergeten dat hij van thuis}{meekreeg de meelzak}\\

\haiku{Vader heeft hem een...}{paar maanden geleden in}{Florence gezien}\\

\haiku{maar ik weet zeker,,!}{als het er op aankwam klom}{ik weer uren ver mee}\\

\haiku{Leonetta, den,:}{laatsten tijd had zich gewend}{aan de gedachte}\\

\haiku{{\textquoteright} {\textquoteleft}Zoo,{\textquoteright} zei Filippo, {\textquoteleft}?}{heeft de brutale hond het}{dan toch gewonnen}\\

\haiku{Ik zou nog tienmaal...}{meer zaakjes opknappen dan}{nou al allemaal}\\

\haiku{Paolina zou...}{tante Ortenzia wel even}{gezelschap houden}\\

\haiku{{\textquoteleft}Waarachtig, Signor,,...{\textquoteright} {\textquoteleft}}{Grassi ik mag doodvallen}{zoowaar als ik hier sta}\\

\haiku{{\textquoteright} - hij gromde van nijd - {\textquoteleft};}{tienduizend lire met de}{loopende rente}\\

\haiku{Hoe dat den 15den,?}{December moest gaan met dien}{mislukten wijnoogst}\\

\haiku{Een oogenblik had;}{Angelo een argwanend}{verkennenden blik}\\

\haiku{Zij zag den pastoor...}{dwars door zijn groentelandjes}{hollen naar de kerk}\\

\haiku{{\textquoteright} zei Domenico, {\textquoteleft}'...{\textquoteright}}{stil voor zich heent is ook}{\`onze Nella Dan}\\

\haiku{Hij stond met een kop,,.}{vuurrood van toorn en strekte}{zijn arm naar de deur}\\

\haiku{Zeker, het leven,...}{zal je hier moeilijk vallen}{de eerste jaren}\\

\haiku{En v\'o\'or het bed, vast,}{op haar stoel gekampeerd of}{zij den ganschen dag}\\

\haiku{{\textquoteleft}Zulke prachten van,{\textquoteright}, - {\textquoteleft}...!}{oogen mompelde hij voor zich}{heenzoo'n prachtig kind}\\

\haiku{{\textquoteright} Emilia verstond maar,.}{gebrekkig Engelsch sprak het}{nog gebrekkiger}\\

\haiku{Filippo, onder,.}{den geesel van haar hoon had}{het hoofd gebogen}\\

\haiku{{\textquoteleft}Laat hem dien tijd dan,.}{middendoor deelen en twee}{dagen hier komen}\\

\haiku{En dadelijk was.}{Carolina overeind en naast}{haar S'or Filippo}\\

\subsection{Uit: Typen en curiositeiten uit Itali\"e}

\haiku{een stil tournooi van,.}{kleurige droomen zonder}{botsing noch zege}\\

\haiku{Alle roccolo's;}{uit den omtrek werden in}{gereedheid gebracht}\\

\haiku{De kleine, zwarte;}{koormuts stond hem achter op}{het scherpe voorhoofd}\\

\haiku{Hij was volstrekt niet,....}{van plan dezen morgen al}{aan den slag te gaan}\\

\haiku{Don Matteo boog zich.}{voor het andere luikje}{en spiedde evenzoo}\\

\haiku{De vader wil geen,.}{gevolg aan de zaak geven}{want hij vreest weerwraak}\\

\haiku{Er was een huisknecht,.}{in livrei en Pepino}{droeg witte kokskleeren}\\

\haiku{De aanval van het.}{meisje duurde maar kort en}{liet geen sporen na}\\

\haiku{Je hebt al gauw mijn,{\textquoteright}.}{veeren niet meer noodig lachte}{soms de Contessa}\\

\haiku{Dien middag zou het.}{jonge paar naar hun huisje}{aldaar vertrekken}\\

\haiku{de wind bespeelde,.}{zijn blauw boezeroen open op}{de bekroesde borst}\\

\haiku{er boven, als een,.}{dreiging fronste het woeste}{berg-voorhoofd}\\

\haiku{- Ja, hij was getrouwd,....}{geweest en voor een jaar was}{zijn vrouw gestorven}\\

\haiku{{\textquoteleft}La Fine di un{\textquoteright}, -.}{Miracolo het Einde}{van een Mirakel}\\

\haiku{Wat hadden die drie,}{menschen een verdriet gehad}{en hoe hulpeloos}\\

\haiku{Herinnerde zich,?}{Clorinda niet dat verhaal van}{Papa uit zijn jeugd}\\

\haiku{Was het dit zoele?}{schemeruur met dat alles}{heiligende licht}\\

\haiku{{\textquoteright} En hij schreef nog iets,.}{voor dat de pijn wellicht wat}{verzachten zou}\\

\haiku{Want, eene zondares,!}{was zij geweest en op zoo}{velerlei wijze}\\

\haiku{{\textquoteright} zeide eindelijk,:}{haar ijle stem en het was}{bijna of zij zong}\\

\haiku{Hij bloost rond zijn oogen,,:}{die schroomend dieper worden}{en over zijn voorhoofd}\\

\haiku{nu smolt de laatste.}{zweem voor de lieflijkheid van}{dezen ouderdom}\\

\haiku{Den volgenden dag:}{waren de berichten al}{even weinig gunstig}\\

\haiku{Zijn handen, uit de,.}{roze hemdsmouwen lagen}{weerloos op het dek}\\

\haiku{De dokter vond hem.}{minder goed en had alle}{bezoek verboden}\\

\haiku{{\textquoteleft}Maar U kunt het zoo,{\textquoteright}, {\textquoteleft}.}{niet hebben zei hijvraagt U}{Ottavino eens}\\

\haiku{De rood-blonde....}{golven stroomden dieper neer}{om rooder wangen}\\

\haiku{{\textquoteright} {\textquoteleft}Si....{\textquoteright}, fluisterde een;}{bedeesd stemmetje tusschen}{de gouden haren}\\

\haiku{een bloemlezing van,.}{Italiaansche dichters daar}{was hij erg blij mee}\\

\haiku{Op het vroolijke;}{zonnepleintje schoolde het}{halve dorp bijeen}\\

\haiku{maar hij had nu den.}{drang zijner ijverige}{gedachten elders}\\

\haiku{) vier meter \'e\'en en,,....}{twintig verdeeld in caissons}{metende deze}\\

\haiku{Dan worden de twee!}{kapiteelen weer precies}{als de andere}\\

\haiku{de jonge madam ',!....}{r tuinhoed lei weer in het}{gras slap van de dauw}\\

\haiku{De oude juffrouw,,;}{deed zelf open sjaaltje om het}{hoofd tuinhoed daarop}\\

\haiku{Nog eenmaal gaf hij:}{zijn heelen brief van voor twee}{maanden ten beste}\\

\haiku{de Olijfberg, met een{\textquoteleft},!}{bekend klooster op den top.}{Davvero cara}\\

\section{Arthur van Schendel}

\subsection{Uit: De berg van droomen}

\haiku{hij had het besluit,.}{genomen zij had gezegd}{dat zij mee zou gaan}\\

\haiku{Wat hij zocht was veel.}{schooner en zijn vader zou blij}{zijn als hij het zag}\\

\haiku{Maar het is al laat,.}{en als ze niet komt weet ik}{niet wat ik doen zal}\\

\haiku{Het touwwerk piepte,.}{groote golven bruisten alom}{in de duisternis}\\

\haiku{Misschien is het lang,.}{geleden misschien ook is}{het zooeven gebeurd}\\

\haiku{{\textquoteleft}Dat zijn de man die.}{bakken kan en zijn makker}{de menscheneter}\\

\haiku{Zij waren omtrent:}{honderd schreden gegaan toen}{Puikebest zeide}\\

\haiku{Toen hij gedaan had:}{stak hij het boekje weer in}{zijn zak en zeide}\\

\haiku{De hond kroop onder.}{het bed en leide zijn hoofd}{op zijn voorpooten}\\

\haiku{Zij zingt in den Hooge,,{\textquoteright}, {\textquoteleft}.}{de prinses zeide zijlaat}{ons in de laan gaan}\\

\haiku{{\textquoteright} riep hij onder het.}{schateren naar den mond van}{den knaap wijzende}\\

\haiku{Daarom weet ik zoo.}{goed wat het is als je een}{verkeerden naam hebt}\\

\haiku{, en Denkmar zou het.}{niet aardig vinden als je}{hem anders noemde}\\

\haiku{{\textquoteright} Toen liep Puikebest,.}{hard naar het rozenpoortje}{gevolgd door Kaka}\\

\haiku{Zij kwamen in een,.}{veld van blauwe bloemen het}{gansche veld was blauw}\\

\haiku{{\textquoteright} Er klonk een gil en}{een lach en toen verdween er}{iets snel uit den boom}\\

\haiku{{\textquoteright} Zij lachte zachtkens.}{of er een bron murmelde}{in de nabijheid}\\

\haiku{Budde zat daar met,.}{zijn hoofd in zijn handen een}{oud kaboutertje}\\

\haiku{Toen stonden zij voor.}{den toren van Vreugde de}{Hooge en het paleis}\\

\haiku{Zie, gij strijdt tegen,.}{onze vrienden niet tegen}{onze vijanden}\\

\haiku{{\textquoteleft}Vreemdeling, Abel is,.}{uw naam slechts uw naam kennen}{wij en anders niet}\\

\haiku{Abel, gij zijt groot, er.}{is niemand die zooals gij nog}{nooit heeft gesproken}\\

\haiku{En de stilte klinkt.}{en de zon hoort mijn roep en}{de wereld ontwaakt}\\

\haiku{Alban het witte.}{paard spitste zijn ooren en de}{ruiter hief zijn hoofd}\\

\haiku{Tobias kraaide,.}{zoo plotseling en zoo luid}{dat allen schrikten}\\

\haiku{{\textquoteright} Maar Tobias met:}{het hart van zonlicht verhief}{zijn schallende stem}\\

\haiku{Wij vogels reizen,{\textquoteright},.}{veel sprak Peregrijn de eend}{recht voor zich ziende}\\

\haiku{Zij stonden rondom,.}{den zeeroover ongeduldig}{vragend wat er was}\\

\haiku{Hij was zoo moede.}{dat hij op den peluw bij}{den nachtwacht neerzeeg}\\

\haiku{Eindelijk legde.}{Corinna zachtkens haar}{hand op zijn schouder}\\

\haiku{{\textquoteleft}Het is dwaasheid weg,.}{te loopen als je nooit iets}{gedaan hebt zooals ik}\\

\haiku{Er ging een heele.}{tijd voorbij zonder dat wij}{iets van oom hoorden}\\

\haiku{Een nederig man.}{met een takkenbos op zijn}{rug volgde haar}\\

\haiku{{\textquoteright} Reinbern dacht aan de,.}{prinses waar en wanneer hij}{haar weder zou zien}\\

\haiku{Maar jullie zijn toch}{domme ganzen dat je niet}{eerder verteld hebt}\\

\haiku{{\textquoteright} {\textquoteleft}Vertel het, Morgan,.}{als je goed nieuws hebt geven}{wij je een meloen}\\

\haiku{En zoo werkten we.}{bij ploegen den heelen nacht}{door tot het dag werd}\\

\haiku{Dat moet  je eens,.}{komen zien het akkertje}{staat heelemaal groen}\\

\haiku{Toen zweeg de klok, zooals.}{de wind zwijgt hoog in het woud}{op een lentedag}\\

\haiku{Maar dezen nacht was,.}{zij in slaap gevallen dat}{was nog nooit gebeurd}\\

\haiku{{\textquoteleft}Ik heb gehoord van.}{het allerliefste kind dat}{ooit geboren is}\\

\haiku{De Koning reed voor,.}{dicht langs de stroomende beek}{starend en denkend}\\

\haiku{Hoog is de hemel,!}{zoo hoog als de hemel is}{de stem van mijn hart}\\

\haiku{het lachen dat zoo.}{groot is als heel de wereld}{en nooit kan vergaan}\\

\haiku{En zij klapten in.}{de handen en dansten met}{elkaar heel den dag}\\

\haiku{Maar hij lachte en.}{maakte het mooie geluid dat}{hij verzonnen had}\\

\haiku{X.   Zij zien het.}{meisje dat altijd vlucht en}{zoeken in het woud}\\

\haiku{{\textquoteright} {\textquoteleft}Neen, haar immers niet,{\textquoteright},,.}{lispelde een andere}{een blonde bedeesd}\\

\haiku{Dan komt zij om te.}{troosten over wat fee\"en en}{nimfen niet hebben}\\

\haiku{En zij lachten, maar.}{hij hoorde slechts haar stem en}{zij slechts de zijne}\\

\haiku{Zij zat daar, klein en,,.}{lief met haar hand op haar knie}{en ze zag mij aan}\\

\haiku{van wat je zelf bent,.}{geweest te weenen om de}{schoonheid van alles}\\

\haiku{Maar ik wist wel dat.}{hij weerzou komen op het}{veld onder den berg}\\

\haiku{De wereld werd groot,.}{en was geheel van hem die}{zoo goed is zoo lief}\\

\haiku{ik wil niet anders,, -?}{dan haar vinden haar alleen}{is zij de prinses}\\

\haiku{je wezen naar de.}{altijd nieuwe beelden die}{de wolken maken}\\

\haiku{Schoon was de rechte,.}{diepte zijner oogen schoon de}{hoogheid van zijn hoofd}\\

\haiku{De vreugde werd mij,}{zoo lief als mijn vader maar}{hem vond ik nergens}\\

\haiku{De prinses heb je,}{nooit gezien de wondere}{Psyche ken je niet}\\

\haiku{Haar, haar zelf had hij.}{vergeten toen hij dacht dat}{alles eender was}\\

\haiku{Het gefluister drong.}{diep in hem en daar in de}{diepte trilde iets}\\

\haiku{En toen zij aan den,.}{oever zaten zag hij zijn}{beeld hoe schoon hij was}\\

\haiku{{\textquoteright} Maar de oude vrouw.}{suste met haar vinger en}{knikte Reinbern toe}\\

\haiku{Ik weet nu dat je,.}{ook naar de prinses zoekt dat}{had ik vergeten}\\

\haiku{Daar, die donkere,.}{poort door misschien weet zij waar}{wij heen moeten gaan}\\

\haiku{Zij zetten zich daar,,.}{neder lang uit hun hoofden}{achterover geleund}\\

\haiku{ik verlang wel naar,,....}{brood ik heb honger maar ik}{zal het niet vragen}\\

\haiku{Toen zag hij in het.}{midden der zaal een blankheid}{waar Psyche in stond}\\

\haiku{{\textquoteleft}Laten we elkaar,.}{dan vasthouden dan kunnen}{wij niet verdwalen}\\

\haiku{Om de prinses te.}{zoeken hebben wij immers}{ook oogen voor den nacht}\\

\haiku{Wij weten zelfs niet,.}{waarom wij zoeken moeten}{waarom zij heen ging}\\

\haiku{En hij geloofde.}{niet dat het een heilige}{was die voor hem stond}\\

\haiku{En Reinbern begreep.}{op eens hoe goed en groot de}{heele wereld was}\\

\haiku{Blauw licht glansde daar,}{waar zij stond en het gelaat}{van Regel met wien}\\

\haiku{Maar zij hoorden haar,:}{en ook de antwoorden van}{Regel en Denkmar}\\

\haiku{Rein was verbaasd, want.}{hij had gedacht ook de elf}{bij zich te houden}\\

\haiku{Ik heb het, bij den,,.}{Bitsan het spotspook omdat hij}{grapjes kan maken}\\

\haiku{Toen hij nog zeer klein,.}{was had hij een zusje maar}{zij was heengegaan}\\

\haiku{Er ging een koele,,.}{wind geurig van teer van nieuw}{hout en van vruchten}\\

\haiku{daar is hetzelfde,{\textquoteright}, {\textquoteleft}}{als het allermooiste hier}{zeide het meisje}\\

\haiku{Ga nu wat slapen,,.}{jongen dan zal ik je van}{middag wel roepen}\\

\haiku{Eindelijk hoorden.}{zij weer beweging in het}{kamertje boven}\\

\subsection{Uit: Drie Hollandse romans}

\haiku{Hij keek beiden aan,.}{en hij wilde iets vragen}{maar hij wist niet wat}\\

\haiku{Dan bleef hij alleen.}{maar kijken naar de kuil die}{vol was gelopen}\\

\haiku{Toen hij hoorde dat}{zij met de postbode over}{het ijs zouden gaan}\\

\haiku{En hij luisterde,.}{zoals de grootmoeder las}{hij zag het voor zich}\\

\haiku{Maarten hoorde het,.}{en bad hij bleef nu staan tot}{het schieten ophield}\\

\haiku{Bij het scheiden kreeg*.}{Maarten menige}{klop op de schouder}\\

\haiku{men zag hem met de,.}{lange benen haastig gaan}{turend over het ijs}\\

\haiku{En elke dag had,.}{hij meer gehoord en eerder}{dan iemand anders}\\

\haiku{Telkens vroeg zij iets,.}{met haar kinderstem dan kwam}{haar adem aan zijn wang}\\

\haiku{Hij wist niet hoe hij,.}{het zeggen moest hij keek haar}{aan en zij wachtte}\\

\haiku{Dan liep hij nog een,.}{eind de dijk op de witte}{maan stond nevelig}\\

\haiku{Maarten lag wakker.}{lang nadat de torenklok}{twaalf had geslagen}\\

\haiku{En dat moest men maar.}{verdragen en werk zoeken}{voor het brood alleen}\\

\haiku{Hij schepte ieder,.}{de aardappelen op het}{bord zij aten zwijgend}\\

\haiku{Het is niets, vrouw, de.}{Heer slaat daar minder acht op}{dan op jou koffie}\\

\haiku{Op een morgen dat}{Rossaart aan de andere}{oever wachtte zag}\\

\haiku{De commandant was.}{een bejaarde kapitein}{van de schutterij}\\

\haiku{Op de hoek kwam hij,.}{zijn broer Hendrikus tegen}{die voor hem staan bleef}\\

\haiku{Bij de plecht stond de,.}{kolenpot het aardewerk}{hing er aan spijkers}\\

\haiku{De wijnkoper had.}{haar in huis genomen en}{zocht een dienst voor haar}\\

\haiku{Om godswil, jongen,*?}{heb ik jullie daarvoor dan}{geholpen}\\

\haiku{Ik hoop alles goeds,}{voor je daarginds maar kijk niet}{neer op je vrienden}\\

\haiku{hout, spijkers, verf gaf.}{Seebel hem in ruil voor zijn}{aandeel in de tjalk}\\

\haiku{daar staat de dood voor,,,.}{mij goed ik geef mij over er}{is niets aan te doen}\\

\haiku{Hij boog het hoofd en.}{staarde door de hor naar de}{nevel over de gracht}\\

\haiku{Hij staarde naar het,.}{ijs op de ruit hij dacht en}{schudde soms zijn hoofd}\\

\haiku{Hoe verder de schuit.}{voer zo meer wendde Marie}{het hoofd naar achter}\\

\haiku{Van toen aan merkten,.}{zij een verschil hoewel zij}{het niet beseften}\\

\haiku{Man, zeide zij, ik.}{zal erom huilen dat je}{alleen moet varen}\\

\haiku{- De steeg was nauw, hij;}{bukte laag om door de hor}{naar de lucht te zien}\\

\haiku{'t Is anders een,.}{hele tijd vijfenveertig}{jaar alleen te zijn}\\

\haiku{Hadden ze maar ja.}{gezegd toen ik je bij me}{in huis wou nemen}\\

\haiku{Geld was er genoeg,.}{je was heemraad geworden}{en allang dijkgraaf}\\

\haiku{er zullen er nog,,.}{heel wat verdrinken ik weet}{het zeker zei je}\\

\haiku{nee, dat geld leg ik,.}{opzij dan heeft hij wat meer}{als hij bij me komt}\\

\haiku{hij vroeg Rossaart of.}{zij haar samen wat met de}{post zouden zenden}\\

\haiku{Jij bent de enige,.}{van wie ik het zie maar er}{zijn ook anderen}\\

\haiku{Je wordt stijf in je,,,}{rug zeg je morgen kan je}{toch niet meer varen}\\

\haiku{Hij zat rechtop met,.}{haar hand in de zijne maar}{hij kon niet spreken}\\

\haiku{Het water klotste.}{tegen het boord en de schuit}{trok aan de touwen}\\

\haiku{Een man riep telkens:}{wanneer hij de kruiwagen}{grond had uitgestort}\\

\haiku{vroeg hij, het is toch.}{niet de eerste keer dat je}{in Hurwenen komt}\\

\haiku{een mens haast niets meer,.}{te kosten en zij nam geen}{geld meer van hem aan}\\

\haiku{Doe je plicht, dacht hij,,.}{dan en vraag niet het zal wel}{gegeven worden}\\

\haiku{, zij werd oud en hem.}{werd het soms te veel altijd}{alleen te varen}\\

\haiku{De hond, die aan zijn,.}{voeten stond schudde zich de}{sneeuw van de haren}\\

\haiku{Gelukkig dat het,,.}{er is ik heb angst gehad}{waarom weet ik niet}\\

\haiku{Maar voor het jaar ten.}{einde ging begon hij zich}{te verontrusten}\\

\haiku{Hij had met alle.}{schuldeisers gesproken en}{zich met hen verstaan}\\

\haiku{Gerbrand had noch de.}{toon noch de betekenis}{hiervan begrepen}\\

\haiku{Eens, toen zij opstond,:}{om naar boven te gaan keek}{hij op en zeide}\\

\haiku{Het huilen ging voort,.}{het was te horen in de}{andere winkels}\\

\haiku{en dat maakte hem.}{zo blij dat hij het hard in}{de armen drukte}\\

\haiku{Frans hief het hoog op.}{naar de bloesems zodat het}{met de ogen knipte}\\

\haiku{Ach broer, zou het niet?}{beter zijn hem met zachtheid}{te behandelen}\\

\haiku{tot hij het weer was.}{die sloeg en de zwakkeren}{de knikkers afnam}\\

\haiku{klagen dat Floris.}{haar kind geknepen had of}{de kleren gescheurd}\\

\haiku{Jongen, vroeg hij, weet?}{je niet dat een dief in de}{gevangenis komt}\\

\haiku{Hij sloeg de zijne,,.}{neer zijn lippen trilden maar}{hij kon niets zeggen}\\

\haiku{Toen Agnete even,:}{bij hem kwam staan om iets te}{vragen zeide hij}\\

\haiku{Dit moet hij voor de,.}{ogen houden en ons voorbeeld}{van rechte zeden}\\

\haiku{de meeste waren,.}{van rode steen maar ook die}{verschillend van kleur}\\

\haiku{En hun moeder keek,.}{ze altijd lachend aan trots}{en vol vertrouwen}\\

\haiku{Daar heeft zij toch ook,,.}{schuld aan zeide Stien en dat}{begreep Floris niet}\\

\haiku{Hij nam ze mee op,}{grote wandelingen tot}{voorbij Bennebroek}\\

\haiku{Dan hosten zij arm,,:}{aan arm schreeuwend tegen de}{mensen joelend van}\\

\haiku{Hij ging iedere.}{dag met de jongens tot de}{laatste zaterdag}\\

\haiku{Hij woelde, hij kon,.}{niet slapen buiten zongen}{nog kermisgangers}\\

\haiku{In de straat sprak zij,,:}{niet maar voorbij de brug waar}{niemand ging vroeg zij}\\

\haiku{Ik weet wat u voor*,.}{mij  ~          gedaan hebt ik}{ben u erg dankbaar}\\

\haiku{Zijn oom sloeg maar even,:}{de ogen op zette zijn bril}{recht en antwoordde}\\

\haiku{Het waren kleine,.}{hoge tonen klagend door}{de witte ruimte}\\

\haiku{Je neemt evengoed een.}{stuk koek uit de kast en dat}{is toch geen diefstal}\\

\haiku{Hij lag wakker, hij.}{besefte niet eens hoe de}{gedachten kwamen}\\

\haiku{Maar dan vroeg Jansje,.}{naar de oom in Hoorn of hij}{oud geworden was}\\

\haiku{En vergeet het niet,.}{een mens hoeft geen kwaad te doen}{als hij het niet wil}\\

\haiku{Maar over vijf, zes jaar.}{zou er geen vlek meer op de}{naam van Floris zijn}\\

\haiku{Zij zagen Werendonk,.}{die wees naar een barst in de}{muur boven het raam}\\

\haiku{Nee, daar kunnen wij,.}{niets aan doen er steekt meer kwaad}{in dan wij weten}\\

\haiku{En plotseling zweeg,,.}{zij met de hand voor de ogen}{of zij zich bedwong}\\

\haiku{Als je je verstand,.}{verliest ga dan naar je huis}{en denk erover na}\\

\haiku{Meer dan twintig jaar,.}{is zij hier geweest al voor}{je geboren was}\\

\haiku{En als stelen niet,.}{genoeg is dan zal ik nog}{wel wat anders doen}\\

\haiku{Werendonk rees, groot stond.}{hij voor Floris die week en}{de stoel liet vallen}\\

\haiku{Toen hij erheen ging.}{de eerste morgen voelde}{hij een verlichting}\\

\haiku{iedere avond hoor.}{ik dat gefluit en anders}{is het hier zo stil}\\

\haiku{Elke avond wanneer.}{hij langskwam stak zij het hoofd}{even uit het venster}\\

\haiku{de moeder was kort.}{zoals Wijntje en droeg een}{muts met keelbanden}\\

\haiku{Maar met volharding,,.}{in het geloof dacht hij wordt}{de ziel behouden}\\

\haiku{Kort en goed, wij zijn}{gekomen om tegen je}{vader te zeggen}\\

\haiku{Alleen het zingen.}{van de treurige liedjes}{kon zij niet laten}\\

\haiku{Kom, zeide zij, help,.}{mij maar liever wat stuk is}{wordt wel weer gemaakt}\\

\haiku{Zeg mij eens, wat heb?}{je op het hart dat je zo}{ongedurig bent}\\

\haiku{zou als er op zijn.}{kantoor een dief geweest was}{die niet gestraft was}\\

\haiku{Om elf uur vond hij.}{zijn broer aan de tafel met}{de Bijbel voor zich}\\

\haiku{Maar Werendonk wilde.}{geen ander in de winkel}{of bij hem aan bed}\\

\haiku{Zij was het die hem.}{zijn eten en melk moest brengen}{en de kamer doen}\\

\haiku{Frans stond op en gaf*,,.}{hem een  ~          hand zijn mond}{trok hij zeide niets}\\

\haiku{Wat zingt die Stien toch,,,?}{zeide hij honderduit waar}{heeft zij dat geleerd}\\

\haiku{Jongen, wij doen voor,.}{je wat wij kunnen maar er}{hindert je nog iets}\\

\haiku{Eens nam hij hem 's}{avonds mee naar de Bavo om}{hem te laten zien}\\

\haiku{En al voelde hij,.}{de moeheid hij stelde het}{uit naar huis te gaan}\\

\haiku{Hij zou navragen.}{wanneer er een boot vertrok}{en wat het kostte}\\

\haiku{Zij staarde naar hem,.}{zij zag hoe wit zijn gezicht}{was in de schemer}\\

\haiku{Alleen om weg te.}{komen uit dat huis had hij}{het geld genomen}\\

\haiku{Je zal toch niet zo,,}{dom zijn hem weer in huis te}{nemen zeide zij}\\

\haiku{als hij gek is laat.}{hem dan oppakken en doe}{hem in Meerenberg}\\

\haiku{Hij wachtte in het,.}{lamplicht aarzelend of hij}{naar hem toe zou gaan}\\

\haiku{Je moet niet denken,.}{dat ik om het brood kom maar}{ik moest in huis zijn}\\

\haiku{Zij stond vlugger op,.}{dan Stien zij was het die een}{stuk brood voor hem sneed}\\

\haiku{Dan komt pas de rust.}{in huis waar je recht op hebt}{met de oude dag}\\

\haiku{Die hem goed kenden.}{merkten dat de gedachten}{hem bezighielden}\\

\haiku{Waar hij keek zag hij,.}{de gezichten groter bleek}{met de ogen donker}\\

\haiku{En hij zocht aan de.}{muren naar een spoor waar de}{bliksem was gegaan}\\

\haiku{Hij zat te suffen,,.}{met hoofdpijn zoals altijd}{op zulke dagen}\\

\haiku{Toen hij weer naar de.}{stad kon gaan reisde juffrouw}{Amalia met hem mee}\\

\haiku{Naar de kerk wilde,.}{hij niet hij had gezegd dat}{men thuis kon bidden}\\

\haiku{Op een avond zette}{zij haar stoel dicht naast hem en}{terwijl zij sprak hield}\\

\haiku{Heiltje richtte zich,:}{op met een kreet van schrik zij}{werd bleek en zij riep}\\

\haiku{De redenen, vrouw,.}{die zal je zien zodra je}{verstand verlicht wordt}\\

\haiku{Nu heeft je man een.}{plicht te doen en die plicht heb}{je mee te dragen}\\

\haiku{Dat zij een hekel.}{aan hem hadden hoefde niet}{gezegd te worden}\\

\haiku{Zij keek naar Heiltje,,.}{die met de hals gestrekt haar}{zat aan te staren}\\

\haiku{Plotseling rees er.}{luide twist tussen man en}{vrouw en zij liep weg}\\

\haiku{Kleed je eens netjes,.}{aan en kom mee wij hebben}{met je te spreken}\\

\haiku{Het zal er dus van,.}{moeten komen al gaat het}{mij tegen het hart}\\

\haiku{Zij dankte Blok dat:}{hij haar weer geholpen had}{en bij zijn woorden}\\

\haiku{Dan kan ik bijna.}{zeker zijn dat het verkeerd}{gaat met allebei}\\

\haiku{Hij was moeilijk te,.}{bedaren maar eindelijk}{kon hij het zeggen}\\

\haiku{Zij moesten bekrimpen.}{en voor de winter hoopte}{hij op de bakteelt}\\

\haiku{Er  *~          kon een,.}{slechte tijd verwacht worden}{met krapheid in huis}\\

\haiku{Dikwijls was het stil.}{in zijn hoofd wanneer hij daar}{de druk niet voelde}\\

\haiku{Veel onkruid bleef er,.}{die zomer weg de aarde}{zag er schoner uit}\\

\haiku{Toch, eerlijker dan.}{zij het bedoelde in het}{gebed kon het niet}\\

\haiku{moest hij een deel van?}{zijn kracht nutteloos laten}{in de voldaanheid}\\

\haiku{Ze kijken alleen,.}{maar ze bespieden wat er}{in mijn hoofd omgaat}\\

\haiku{Het moet, daar kan je,.}{zeker van zijn ik zie dat}{het geschreven staat}\\

\haiku{De man heeft niets te.}{doen als piekeren en daar}{wordt het hoofd moe van}\\

\haiku{Toen het beter ging.}{was hij nog moeilijker van}{zijn plaats te krijgen}\\

\haiku{Ik dacht anders dat,.}{degene die beproefd wordt}{ervan verbetert}\\

\haiku{Waarom dan moest ik?}{ondankbaar worden voor wat}{mij gegeven werd}\\

\haiku{Als je denkt dat het,,.}{moet zeide hij kan ik je}{niet tegenhouden}\\

\haiku{Dan was zij niet zo.}{dom geweest haar zoon naar de}{slachtbank te sturen}\\

\haiku{Misschien wachten ze.}{daar al op me. Als mijn broer}{mij maar niet loslaat}\\

\haiku{Het kind vroeg toen of.}{haar vader niet even bij haar}{kon komen zitten}\\

\haiku{We pikken wat, we,,.}{zoeken wat we vinden wat}{we vliegen verder}\\

\haiku{Het was te dwaas te.}{denken dat de plaats daar iets}{mee te maken had}\\

\haiku{Ik was nog een kind,}{toen men zei dat het in zou}{storten zo oud is}\\

\haiku{Bid zoveel je wil,.}{maar waar de duivel woont zal}{het je niet helpen}\\

\haiku{Sofie ging er wel,.}{heen om hem toe te spreken}{maar het hielp niet veel}\\

\haiku{Wel was het moeilijk.}{zich te bedwingen en de}{deur voorbij te gaan}\\

\haiku{Blok kwam dichter bij.}{hem staan en merkte dat hij}{niet gedronken had}\\

\haiku{Met de dikke stok,,:}{die hij nu had langs de weg}{wandelend dacht hij}\\

\haiku{De knecht, menende,.}{dat zij vochten greep hem aan}{en rukte hem los}\\

\haiku{Ze denken dat ik.}{nog officier ben en in}{het hospitaal lig}\\

\haiku{, want behalve de.}{zieke en de werkvrouw was}{er niemand overdag}\\

\haiku{kleine ijssleden.}{die met prikstokken worden}{voortbewogen}\\

\haiku{de uitreiking van.}{onder andere voedsel}{aan de armen}\\

\haiku{omdat namelijk ().}{de Belgische Opstand1830}{was uitgebroken}\\

\haiku{de schipper op de...:}{beurtschuit.105die de wonderen}{en wetenschap had}\\

\haiku{en ga ten huize.}{uws broeders niet op den dag}{van uwen tegenspoed}\\

\haiku{het vierde gebod,;}{dat de heiliging van de}{zondagsrust voorschrijft}\\

\haiku{Voor het verwarmen.}{van de bak werd meestal}{paardenmest gebruikt}\\

\haiku{Vanaf 1870 was dat.}{gesticht gevestigd op de}{Stadhouderskade}\\

\haiku{de puntige stok.}{waarmee jonge gewassen}{gepoot worden}\\

\haiku{De betekenis.}{van deze passage is}{onduidelijk}\\

\haiku{Er wordt gespit en,.}{geschoffeld meel wordt gezeefd}{of turf verhandeld}\\

\haiku{Driemaal bezoekt hij.}{de Grote Kerk en tart hij}{God hem te straffen}\\

\haiku{Zijn geloof in en.}{angst voor een God der wrake}{is hij voorgoed kwijt}\\

\haiku{{\textquoteleft}Die gezindheid zou {\textquotedblleft}{\textquotedblright}{\textquoteright}.}{u ook kunnen vinden in}{Een zwerver verdwaald}\\

\haiku{Dan komt pas de rust.}{in huis waar je recht op hebt}{met de oude dag}\\

\haiku{Dat had hij gedaan,,{\textquoteright} ().}{niet anders daarom voer hij}{nu alleenp. 151}\\

\haiku{Aan het begin van:}{hoofdstuk vi van Een Hollands}{drama lezen we}\\

\haiku{Sporadisch begint.}{de directe rede met}{een kleine letter}\\

\haiku{het handschrift van De:}{waterman heeft op deze}{plaats namelijk}\\

\haiku{Driemaal wordt dit, steeds, {\textquoteleft}{\textquoteright}.}{in de directe rede}{geschreven alsjou}\\

\haiku{Het is niets, vrouw, de.}{Heer slaat daar minder acht op}{dan op jou koffie}\\

\subsection{Uit: Fratilamur}

\haiku{Hoewel ik hem niet.}{gezegd had waar ik woonde}{verwachtte ik hem}\\

\haiku{Ik was elf jaar, een,.}{kind dat weinig kreeg voor den}{mond noch voor het hart}\\

\haiku{Ik voelde dat ik.}{grooter was en voor eeuwig}{bevrijd van een band}\\

\haiku{Het was zomer, stil,.}{alleen ging soms achter mij}{een locomotief}\\

\haiku{Zij had alles dat,.}{verblinden kan want alles}{scheen wat het niet was}\\

\haiku{Brandt voor het onrecht,.}{bidt om genade voor wie}{je je vrienden noemt}\\

\haiku{Roep en smeek tot er.}{zijn die je alles opendoen}{behalve hun ziel}\\

\haiku{Ik verliet het dorp.}{en dwaalde in een stille}{stad een zomer lang}\\

\haiku{Je zou zelf je pijp.}{gemaakt hebben en in den}{kring kunnen zitten}\\

\haiku{Zooals een kind dat zijn.}{zin niet gehad heeft bleef ik}{turen over de zee}\\

\haiku{De stad wemelde.}{van menschen die er voor hun}{genoegen kwamen}\\

\haiku{Ik antwoordde dat,.}{ik dit was hoewel niet in}{Holland geboren}\\

\haiku{dit oogenblik, van,.}{hem  voor mij alleen heeft}{mij opgeheven}\\

\subsection{Uit: Jan Compagnie}

\haiku{Jan de Brasser was.}{Amsterdammer omdat hij}{hier had leeren loopen}\\

\haiku{De koster lachte,.}{waarop Ras kwaad werd en hem}{voor huichelaar schold}\\

\haiku{Op hem lag de taak.}{den ouderdom van moeder}{en oom te steunen}\\

\haiku{Daar hij liever in.}{een warm land wilde wonen}{koos hij voor Indi\"e}\\

\haiku{De directeur prees.}{hem en voorspelde dat hij}{het ver zou brengen}\\

\haiku{De kapitein had.}{geen anderen bij stand dan}{van den lanspassaat}\\

\haiku{In Jacatra had.}{hij zich onderscheiden door}{moed en bekwaamheid}\\

\haiku{Maartensz teekende.}{het briefje en ontving het}{geld voor het vervoer}\\

\haiku{De kwartiermeester.}{liet de kisten stouwen met}{nog twee andere}\\

\haiku{Bij Andries in huis,.}{zeide hij dat hij vriend was}{maar ook korporaal}\\

\haiku{Een uur later zei.}{hij terloops dat Manilla}{niet ver weg kon zijn}\\

\haiku{De eenige die Ai,,.}{kon helpen een klein Engelsch}{schip voer voor hen weg}\\

\haiku{Maar een anderen,,.}{keer bij het herhaald verzoek}{beraadslaagden zij}\\

\haiku{De droppels van den.}{regen spatten van den grond}{op of het bloed was}\\

\haiku{Ai wonen noch op.}{een andere plaats waar de}{Compagnie gebood}\\

\haiku{Hij las de brieven.}{van huis die alle zegen}{en welvaart meldden}\\

\haiku{met bewogen stem,.}{noemde hij hem een deugdzaam}{een edelmoedig heer}\\

\haiku{Allen klonken en.}{Fonseca noodigde allen}{voor het bruiloftsmaal}\\

\haiku{En als het werk goed.}{gedaan werd achtte hij niet}{of het langzaam ging}\\

\haiku{Hij had zooveel zwart,.}{volk vervoerd dat hij ze voor}{enkel koopwaar hield}\\

\haiku{De vreugde onder.}{het pakken en redderen}{duurde korten tijd}\\

\haiku{Toen scheepte hij zich.}{met hen in op het jacht dat}{naar Ambon vertrok}\\

\haiku{Water hadden zij,.}{uit de geulen hout voor het}{drogen uit het bosch}\\

\haiku{Yen Pon de winst in,.}{goud doen wegen drie kistjes}{elk van dertig pond}\\

\haiku{Van de raden van.}{Indi\"e had er maar een wat}{geld overgehouden}\\

\haiku{Hij had geen tijd voor.}{onbenulligheden en}{verliet het kantoor}\\

\haiku{Er was veel vermaak,.}{met de vischnetten de tafels}{stonden ruim voorzien}\\

\haiku{Ai doen vervoeren.}{om nootmuskaat te plukken}{voor de Compagnie}\\

\haiku{De klokken waren.}{maar zwak te hooren in het}{geknal en gejuich}\\

\haiku{zij bestemd waren,.}{gaf verlof ze dienzelfden}{dag uit te laden}\\

\subsection{Uit: Jeugdherinneringen}

\haiku{De andere was,,.}{een jongen groter dan ik}{Bantoet heette hij}\\

\haiku{Er zou nog meer te.}{vertellen zijn uit die tijd}{in het Florapark}\\

\haiku{Eens is het gebeurd,.}{dat ik twee dagen niets heb}{gehad ook geen brood}\\

\haiku{Na school wachtte ik.}{haar op en ik vroeg of ik}{haar tas mocht dragen}\\

\haiku{Eens zei ze dat ze.}{thuis verboden hadden dat}{ik met haar meeliep}\\

\haiku{Voor jongens van een.}{bekende familie had}{hij een gunstig oog}\\

\haiku{Het geld kreeg ik van,.}{de tante Quentin over wie}{ik straks vertel}\\

\haiku{Het was op een dag.}{een verrassing toen Johan zei}{dat  hij ook schreef}\\

\haiku{Ik had haar weinig.}{gekend en er was nooit meer}{over haar gesproken}\\

\haiku{Dat deed ik, ik bracht '.}{hems zaterdags zijn tien}{gulden salaris}\\

\haiku{Het heeft ook niet lang,,.}{geduurd hij stierf in april even}{vijfentwintig jaar}\\

\haiku{En kijk eens hier, zei,,.}{hij nog dit kasboek klopt ook}{niet telt u maar op}\\

\haiku{Ik heb haar pas uit.}{het oog verloren toen ik}{naar Engeland ging}\\

\haiku{Maar ook ons bedrag.}{van tweehonderd gulden was}{een meevaller}\\

\subsection{Uit: Verhalen}

\haiku{Ik werd vriendelijk}{jegens het paard en vond het}{aangenaam te zien}\\

\haiku{Nogmaals faalde ik;}{den vogel te grijpen en}{haalde nu mijn boog}\\

\haiku{De honden stonden.}{woedend te blaffen naar den}{vogel die heenging}\\

\haiku{en van den tijd waar;}{wij in leven dat die vol}{geheimenis is}\\

\haiku{het water in den;}{stroom daar beneden had een}{goedhartigen klank}\\

\haiku{Wanneer Alan kwam zag;}{hij telkens hoe zij gegroeid}{en veranderd was}\\

\haiku{Ik hoor het gaarne.}{want hun geluid is muziek}{bij mijn gemijmer}\\

\haiku{men vond ze ook een.}{enkel keer bij het kruis dat}{aan den afgrond staat}\\

\haiku{Dienzelfden middag,.}{verliet zij het gehucht naar}{het zuiden gaande}\\

\haiku{Toen ging er geen dag.}{zonder tranen en troost en}{lange gepeinzen}\\

\haiku{zij zag haar moeder.}{die altoos ernstig naast haar}{ging op de tochten}\\

\haiku{Aldebrand die de.}{kreten hoorde trad buiten}{en knipte zijn oogen}\\

\haiku{Toen hij zijn dochter}{zag viel menig man in het}{gedrang ter aarde}\\

\haiku{De veldheer trad in,.}{de zaal gevolgd door zijn zoon}{en een toortsdrager}\\

\haiku{Slechts het gelaat aan.}{het kruis waar het lampje voor}{brandde zag haar aan}\\

\haiku{Met een kreet trad hij,.}{nader maar snel rees zij en}{vluchtte de zaal uit}\\

\haiku{De jonkman wees en,:}{hief zijn armen groot waren}{zijn luide woorden}\\

\haiku{Nu zou gewis de.}{vijand wijken en het}{land gelukkig zijn}\\

\haiku{In den dageraad,,}{fonkelend van dauw zag zij}{bij het ontwaken}\\

\haiku{De weg steeg en die.}{laatste inspanning werd te}{zwaar voor de Vlaamsche}\\

\haiku{een overste kon niet,.}{gemist worden zij zelve}{moest aanvoerder zijn}\\

\haiku{of het de haat was.}{die het ongeluk bracht of}{de straf des hemels}\\

\haiku{Toen wisten beiden.}{dat Ermonne een van haar}{zoons verliezen moest}\\

\subsection{Uit: Verzameld werk. Deel 1}

\haiku{De vrouwen gingen.}{meer dan ooit ter kerke voor}{hun mannen bidden}\\

\haiku{Waarom kunnen de?}{honden blaffen en janken}{en kunt gij het niet}\\

\haiku{Ga en bid op de.}{berg Kalvari\"e voor uw heil}{en bid voor uw kind}\\

\haiku{Het onderwijs der;}{beide zoons was toevertrouwd}{aan de kapelaan}\\

\haiku{, zijn deugdzaam zwaard kon.}{hij schier niet meer tillen met}{de lome armen}\\

\haiku{wat mijn geluk kan,?}{worden behoort een ander}{mag ik het nemen}\\

\haiku{{\textquoteright} {\textquoteleft}In 't graf, jongen,,,.}{gaat hij nog trager maar hij}{staat niet stil goddank}\\

\haiku{De lafhartigheid.}{Ermgarde te krenken was}{verre van hem}\\

\haiku{Gij zult toch wel eens '?}{een vrouw ontmoet hebben die}{gijt liefste hadt}\\

\haiku{{\textquoteright} En zij maakte een.}{gebaar of zij zijn stoel tot}{zich trekken wilde}\\

\haiku{Het regende nog;}{toen hij die morgen voor de}{hut van Karo kwam}\\

\haiku{De kapelaan zocht,;}{hem in milder stemming te}{brengen doch vergeefs}\\

\haiku{toen keek de edelvrouw.}{om en zette zich weer kalm}{aan het borduurwerk}\\

\haiku{Sta, sta op, zeg ik,,...!}{u vlied van hier zoek of ge}{uw graf kunt vinden}\\

\haiku{{\textquoteright} Hand in hand zochten,.}{zij het pad voorzichtig in}{het duister tastend}\\

\haiku{Hij werd grauw van iets,.}{angstigs dreigends dat over hem}{neerkwam als een stolp}\\

\haiku{{\textquoteleft}Gauw, gauw, Drogon! 't,,.}{Volk komt de monniken zij}{willen u doden}\\

\haiku{het water in de;}{stroom daar beneden had een}{goedhartige klank}\\

\haiku{Wanneer Alan kwam zag;}{hij telkens hoe zij gegroeid}{en veranderd was}\\

\haiku{dit leven was niet:}{te dulden en thans was ik}{er diep van overtuigd}\\

\haiku{Nogmaals faalde ik;}{de vogel te grijpen en}{haalde nu mijn boog}\\

\haiku{De honden stonden.}{woedend te blaffen naar de}{vogel die heenging}\\

\haiku{en van de tijd waar;}{wij in leven dat die vol}{geheimenis is}\\

\haiku{Helaas, waarom wil?}{je niet je hele leven}{bij mij doorbrengen}\\

\haiku{Maar ik vond je lief;}{voor je mij gezegd had dat}{je me beminde}\\

\haiku{Dat ledige en.}{die gevoelloosheid zouden}{mij gans niet lijken}\\

\haiku{men heeft me hier kort;}{geleden poortresse van}{dit klooster gemaakt}\\

\haiku{De wind, die zostraks door,.}{de schoorsteen gierde was van}{lieverlee bedaard}\\

\haiku{Zij bedachten, dat.}{het laat was en dat zij naar}{de stad moesten keren}\\

\haiku{En Rogier zat weer.}{alleen in de vensterbank}{met gebogen hoofd}\\

\haiku{De straten waren,.}{leeg maar Carolus voorop}{was op zijn hoede}\\

\haiku{hij had zich nooit om.}{de mensen bekommerd en}{nooit verdriet gehad}\\

\haiku{Het lot moet zijn loop,.}{hebben en het is een dwaas}{die het wil weren}\\

\haiku{Eindelijk zag hij:}{op met een klare gloed in}{de ogen en zeide}\\

\haiku{Eens toen hij die weg}{besteeg bemerkte hij dat}{het zeer donker werd}\\

\haiku{Een poosje gingen;}{zij zonder spreken voort met}{gebogen hoofden}\\

\haiku{Hij boog er zich even,.}{over knikte dan tot Mevena met}{goedhartige blik}\\

\haiku{Mevena bezon zich, dat;}{de monnik heen zou gaan nu}{zij bij Rogier was}\\

\haiku{dan richtte hij zich:}{tot zijn volle grootte op}{en voegde erbij}\\

\haiku{de een was een man,}{bedrogen door zijn vrouw de}{ander een moeder}\\

\haiku{Lugina klopte.}{hem vertrouwelijk op de}{schouder en ging heen}\\

\haiku{Hij begreep, dat zij.}{alles gehoord had wat hij}{zopas vertelde}\\

\haiku{{\textquoteleft}Rogier mocht van de,.}{keizer niet heengaan daarom}{wacht ik hier op hem}\\

\haiku{Hij ontwaakte door;}{een hevige slagregen}{en stond verschrikt op}\\

\haiku{Zij stonden zwijgend,.}{op de weg in de schemer}{der hoge bomen}\\

\haiku{hij staan, glurend door.}{de bladeren over de weg}{naar rechts en naar links}\\

\haiku{weer sloeg de ene, de,.}{zware klok de kleinere}{herhaalde de galm}\\

\haiku{Toen haar vrees verzwond;}{zuchtte zij en bemerkte}{hoe hij haar aanzag}\\

\haiku{Het water spoelde,.}{tegen het roer het ijzer}{piepte geregeld}\\

\haiku{Omtrent de middag,.}{trad Seffe binnen de schelm}{was ietwat dronken}\\

\haiku{Er was iets dat hem,.}{verlicht deed ademen hij had}{met de stad gedaan}\\

\haiku{Daar zat Maluse,.}{in het licht van het venster}{de hond lag er ook}\\

\haiku{Tamalone, zijn,;}{hoofd heffend antwoordde dat}{hij Polein zocht}\\

\haiku{Hij keerde zich om.}{uit het licht en liep weer door}{de gang naar de straat}\\

\haiku{Toen ook Simon aan,}{boord was riep Meron Joseph}{zijn bevelen uit}\\

\haiku{Ik hoor het gaarne.}{want hun geluid is muziek}{bij mijn gemijmer}\\

\haiku{Die grijsaard moet een,.}{tovenaar geweest zijn als}{hij niet erger was}\\

\haiku{Diezelfde middag,.}{verliet zij het gehucht naar}{het zuiden gaande}\\

\haiku{Toen ging er geen dag.}{zonder tranen en troost en}{lange gepeinzen}\\

\haiku{De pest, de kwaal wier,.}{naam zacht wordt uitgesproken}{heerste toen in Stratford}\\

\haiku{Dan renden allen,.}{naar huis zoals de liefste}{naar de liefste gaat}\\

\haiku{hoe de kauwen het;}{gevaar vergeten wanneer}{het graan wordt gemaaid}\\

\haiku{Zijn moeder bracht het;}{apostellepeltje voor}{haar eerste kleinkind}\\

\haiku{Als men niet zijn kan,.}{zoals zijn ideaal moet men}{zijn zoals men is}\\

\haiku{Het heerlijkst schouwspel.}{dat ooit verbeelding schiep zou}{daar gebeuren}\\

\haiku{De binnenplaats van:}{de herberg waar zij vroeger}{speelden gaf het plan}\\

\haiku{Zij waren beiden,.}{opgewekte mannen vol}{hoop en goede moed}\\

\haiku{Hij kende thans een.}{inniger wellust dan die}{van louter vormen}\\

\haiku{Dat zag Jacobus:}{duidelijk toen hij een paar}{jaar later zeide}\\

\haiku{Maar hij was te lomp,,,.}{te stug en zeiden zijn}{vrienden te lelijk}\\

\haiku{Will behoefde rust,,}{zij hadden het lang gezien}{het zou hem goed doen}\\

\haiku{Berowne, die hij,.}{als zichzelf had gekend was}{thans een vreemdeling}\\

\haiku{toen dwingelanden;}{arm werden en geringen}{rechten verkregen}\\

\haiku{Vlak achter het huis,:}{dicht bij de open venstertjes}{lagen de bedden}\\

\haiku{- So man and man should{\textquoteright},.}{be antwoordde zij die aan}{het hof had geleefd}\\

\haiku{these are flowers,}{Of middle summer and I}{think they are given}\\

\haiku{New Place echter.}{en het huis in Henley Street}{bleven ongedeerd}\\

\haiku{Een zoeker was hij,,,.}{een minnaar een dichter en}{hij schreef een treurspel}\\

\haiku{En toch, heerlijk dat!}{er zoveel verschil is ook}{waar zielen spreken}\\

\haiku{Waarlijk groot is het,.}{te strijden als het moet zelfs}{om een nietigheid}\\

\haiku{Een ander, zoals,:}{Kent draagt duldzaam de smarten}{die zij bed\'elen}\\

\haiku{knaves, thieves, and,;}{treachers by spherical}{predominance}\\

\haiku{Nooit daarvoren en;}{nooit daarna heeft hij zulke}{monsters voortgebracht}\\

\haiku{Dit is de kleding.}{volgens het borstbeeld boven}{het graf in de kerk}\\

\haiku{De Berg van dromen,}{I De knaap het meisje en}{Peter gaan op reis}\\

\haiku{hij had het besluit,.}{genomen zij had gezegd}{dat zij mee zou gaan}\\

\haiku{Maar het is al laat,.}{en als ze niet komt weet ik}{niet wat ik doen zal}\\

\haiku{Misschien is het lang,.}{geleden misschien is het}{zo\"even gebeurd}\\

\haiku{Toen blies hij op zijn.}{hoorn en  vroeg met grote}{stem hoe zij heetten}\\

\haiku{Zij waren omtrent:}{honderd schreden gegaan toen}{Puikebest zeide}\\

\haiku{Toen hij gedaan had:}{stak hij het boekje weer in}{zijn zak en zeide}\\

\haiku{{\textquoteright} riep hij onder het.}{schateren naar de mond van}{de knaap wijzende}\\

\haiku{, en Denkmar zou het.}{niet aardig vinden als je}{hem anders noemde}\\

\haiku{{\textquoteright} Toen liep Puikebest,.}{hard naar het rozenpoortje}{gevolgd door Kaka}\\

\haiku{Zij kwamen in een,.}{veld van blauwe bloemen het}{ganse veld was blauw}\\

\haiku{{\textquoteright} Er klonk een gil en}{een lach en toen verdween er}{iets snel uit de boom}\\

\haiku{{\textquoteright} Zij lachte zachtkens.}{of er een bron murmelde}{in de nabijheid}\\

\haiku{Budde zat daar met,.}{zijn hoofd in zijn handen een}{oud kaboutertje}\\

\haiku{zie, gij strijdt tegen,.}{onze vrienden niet tegen}{onze vijanden}\\

\haiku{{\textquoteleft}Vreemdeling, Abel is,.}{uw naam slechts uw naam kennen}{wij en anders niet}\\

\haiku{En de stilte klinkt.}{en de zon hoort mijn roep en}{de wereld ontwaakt}\\

\haiku{Tobias kraaide,.}{zo plotseling en zo luid}{dat allen schrikten}\\

\haiku{{\textquoteright} Maar Tobias met:}{het hart van zonlicht verhief}{zijn schallende stem}\\

\haiku{Wij vogels reizen,{\textquoteright},.}{veel sprak Peregrijn de eend}{recht voor zich ziende}\\

\haiku{{\textquoteright} Zij stonden op en,.}{gingen hun schreden klonken}{in de grote zaal}\\

\haiku{Hij was zo moede.}{dat hij op de peluw bij}{de nachtwacht neerzeeg}\\

\haiku{Eindelijk legde.}{Corinna zachtkens haar}{hand op zijn schouder}\\

\haiku{dat moest je niet weer,.}{doen want nu heb ik nog}{maar \'e\'en gouden ring}\\

\haiku{Er ging een hele.}{tijd voorbij zonder dat wij}{iets van oom hoorden}\\

\haiku{{\textquoteright} Reinbern dacht aan de,.}{Prinses waar en wanneer hij}{haar weder zou zien}\\

\haiku{Maar jullie zijn toch}{domme ganzen dat je niet}{eerder verteld hebt}\\

\haiku{{\textquoteright} {\textquoteleft}Vertel het, Morgan,.}{als je goed nieuws hebt geven}{wij je een meloen}\\

\haiku{En zo werkten we.}{bij ploegen de hele nacht}{door tot het dag werd}\\

\haiku{Ik werd wakker - maar,,.}{in mijn droom want ik had maar}{gedroomd dat ik sliep}\\

\haiku{Maar deze nacht was,.}{zij in slaap gevallen dat}{was nog nooit gebeurd}\\

\haiku{{\textquoteleft}Ik heb gehoord van.}{het allerliefste kind dat}{ooit geboren is}\\

\haiku{Hoog is de hemel,!}{zo hoog als de hemel is}{de stem van mijn hart}\\

\haiku{het lachen dat zo.}{groot is als heel de wereld}{en nooit kan vergaan}\\

\haiku{Maar hij lachte en.}{maakte het mooie geluid dat}{hij verzonnen had}\\

\haiku{{\textquoteright} {\textquoteleft}Neen, haar immers niet,{\textquoteright},,.}{lispelde een andere}{een blonde bedeesd}\\

\haiku{Dan komt zij om te.}{troosten over wat fee\"en en}{nimfen niet hebben}\\

\haiku{En zij lachten, maar.}{hij hoorde slechts haar stem en}{zij slechts de zijne}\\

\haiku{Zij zat daar, klein en,,.}{lief met haar hand op haar knie}{en ze zag mij aan}\\

\haiku{van wat je zelf bent,.}{geweest te wenen om de}{schoonheid van alles}\\

\haiku{Maar ik wist wel dat.}{hij weer zou komen op het}{veld onder de berg}\\

\haiku{De wereld werd groot,.}{en was geheel van hem die}{zo goed is zo lief}\\

\haiku{ik wil niet anders, -?}{dan haar vinden haar alleen}{is zij de Prinses}\\

\haiku{je wezen naar de.}{altijd nieuwe beelden die}{de wolken maken}\\

\haiku{Schoon was de rechte,.}{diepte zijner ogen schoon de}{hoogheid van zijn hoofd}\\

\haiku{Sibylle en {\textquoteleft},!}{de schone JongelingNiet}{haar beeld maar haar zelf}\\

\haiku{Haar, haar zelf had hij.}{vergeten toen hij dacht dat}{alles eender was}\\

\haiku{Het gefluister drong.}{diep in hem en daar in de}{diepte trilde iets}\\

\haiku{En toen zij aan de,.}{oever zaten zag hij zijn}{beeld hoe schoon hij was}\\

\haiku{{\textquoteright} Maar de oude vrouw.}{suste met haar vinger en}{knikte Reinbern toe}\\

\haiku{Ik weet nu dat je,.}{ook naar de Prinses zoekt dat}{had ik vergeten}\\

\haiku{Daar, die donkere,.}{poort door misschien weet zij waar}{wij heen moeten gaan}\\

\haiku{Zij zetten zich daar,,.}{neder languit hun hoofden}{achterover geleund}\\

\haiku{ik verlang wel naar,,...}{brood ik heb honger maar ik}{zal het niet vragen}\\

\haiku{{\textquoteleft}Laten we elkaar,.}{dan vasthouden dan kunnen}{wij niet verdwalen}\\

\haiku{Om de Prinses te.}{zoeken hebben wij immers}{ook ogen voor de nacht}\\

\haiku{Wij weten zelfs niet,.}{waarom wij zoeken moeten}{waarom zij heenging}\\

\haiku{En hij geloofde.}{niet dat het een heilige}{was die voor hem stond}\\

\haiku{En Reinbern begreep.}{opeens hoe goed en groot de}{hele wereld was}\\

\haiku{Maar zij hoorden haar,:}{en ook de antwoorden van}{Regel en Denkmar}\\

\haiku{Rein was verbaasd, want.}{hij had gedacht ook de elf}{bij zich te houden}\\

\haiku{Ik heb het, bij de,,.}{Bitsan het spotspook omdat hij}{grapjes kan maken}\\

\haiku{Hoe kon hij verder,.}{hoe kon hij hoger gaan dan}{de top van de Berg}\\

\haiku{En de Koning sprak,:}{dat klonk als windgeruis in}{warme zomeravond}\\

\haiku{Toen hij nog zeer klein,.}{was had hij een zusje maar}{zij was heengegaan}\\

\haiku{Er ging een koele,,.}{wind geurig van teer van nieuw}{hout en van vruchten}\\

\haiku{daar is hetzelfde,{\textquoteright}, {\textquoteleft}}{als het allermooiste hier}{zeide het meisje}\\

\haiku{Eindelijk hoorden.}{zij weer beweging in het}{kamertje boven}\\

\haiku{Hun wijzen menen;}{dat de mens geen taak heeft dan}{de smart te ontgaan}\\

\haiku{geen mens te doden,;}{slechts onwetenden achten}{het naastenbloed niet}\\

\haiku{Zij reden verder,.}{om in Bethel te vernachten}{dichter bij de stad}\\

\haiku{Wat in mij is dringt,,.}{zeide zij breng mij waar ik}{neder kan liggen}\\

\haiku{De dwingeland van;}{Zion had met het zwaard zijn}{zonen omgebracht}\\

\haiku{IV In Nazareth.}{woonden zij in de stilte}{beneden de berg}\\

\haiku{De krekels zongen,.}{aan alle kant de sterren}{schitterden boven}\\

\haiku{Achter de leerschool,,;}{was een gaarde van vijgen}{granaten amandels}\\

\haiku{maar toen Judas van,?}{Golan voor de joden vocht}{waar was Micha\"el}\\

\haiku{De Jetser, dat is,.}{de kwade neiging in u}{dat is de boosheid}\\

\haiku{Zo stond Jezus in:}{hun midden en zag alles}{aan hoe zij dachten}\\

\haiku{Aan de grootheid die,.}{zijn makker meende dacht hij}{niet zij was te klein}\\

\haiku{De hemel straalde,:}{van nieuwe klaarheid de wind}{zong van bevrijding}\\

\haiku{En het moede lijf.}{lag neder en groot verrees}{de zon over zijn slaap}\\

\haiku{de Eeuwige uw.}{God zult gij aanbidden en}{hem alleen dienen}\\

\haiku{En Jezus keerde.}{naar de woning der ouders}{en zat in stilte}\\

\haiku{De nacht was warm en,.}{schoon de sterren schitterden}{toen hij binnentrad}\\

\haiku{Meester, mijn slaaf die.}{mij zeer dierbaar is ligt thuis}{in zware pijnen}\\

\haiku{Zie toch en help mij,.}{opdat ik niet zal moeten}{bedelen om brood}\\

\haiku{Alleen het hart zal,.}{hem zien en het reine hart}{is boven alles}\\

\haiku{Maar wie u op de,;}{rechterwang slaat keer hem ook}{de andere toe}\\

\haiku{En uw Vader, die,.}{in het verborgen ziet zal}{het u vergelden}\\

\haiku{En uw Vader, die,.}{in het verborgen ziet zal}{het u vergelden}\\

\haiku{Wie van u kan door?}{bezorgd te zijn \'e\'en el tot}{zijn lengte toedoen}\\

\haiku{En Jezus ging voort.}{in hun midden tot aan de}{oever van het meer}\\

\haiku{Laat ons overvaren.}{naar de eenzamen aan de}{andere oever}\\

\haiku{Verlangde zij de?}{vreugde van het geven niet}{meer dan andere}\\

\haiku{En niet \'e\'en musje.}{valt op de aarde zonder}{de wil uws Vaders}\\

\haiku{Wanneer zij u in,;}{de ene stad vervolgen vliedt}{naar de andere}\\

\haiku{Toen gij een kind waart,,.}{toen hadt gij al de wereld}{lief tot uzelve toe}\\

\haiku{hij ging voort tot zijn.}{getrouwen en zat aan de}{oever en leerde}\\

\haiku{Alle plant die mijn.}{Vader niet geplant heeft zal}{uitgeroeid worden}\\

\haiku{En zij kwam voor hem,:}{en viel neder en zij hield}{aan en jammerde}\\

\haiku{de vogels waren.}{daar roerig en stegen in}{zwermen uit het riet}\\

\haiku{Nu dan daar het heil,.}{was en hier de poort gebouwd}{zo moest zij openen}\\

\haiku{Zij zagen enkel,,.}{het licht maar Jezus niet en}{het licht was over hen}\\

\haiku{En hij leerde hun}{aangaande de voorzegging}{dat eerst Elia komen}\\

\haiku{Een jonge man daar,,:}{in zijn binnenste geroerd}{trad voor hem en vroeg}\\

\haiku{Ook een man van de,.}{tempel kwam daar langs en zag}{hem en ging voorbij}\\

\haiku{Die voorbijgingen,:}{geboden hem te zwijgen}{maar luider riep hij}\\

\haiku{Ja, zeg ik u, een.}{iegelijk die heeft die zal}{gegeven worden}\\

\haiku{Heer, trekt gij u dat?}{niet aan dat mijn zuster mij}{alleen laat dienen}\\

\haiku{Doch Martha deed stil,.}{haar werk daar het dienen haar}{liefste vreugde was}\\

\haiku{Heer, wij weten niet,?}{waar gij heen gaat hoe zullen}{wij de weg kennen}\\

\haiku{Maar in het midden:}{der olijfgaarde stond Jezus}{weder stil en sprak}\\

\haiku{Een stem riep de naam,.}{van Jezus een andere}{begon te klagen}\\

\haiku{Anderen heeft hij,.}{verlost maar  zich zelf kan}{hij niet verlossen}\\

\haiku{Maar de kinderen.}{groeiden en schreiden gelijk}{de ouders schreiden}\\

\haiku{Tijdschriftpublikatie,,,,-.}{in Nederland deel II 1904}{jg. 56 blz. 369375}\\

\subsection{Uit: Verzameld werk. Deel 2}

\haiku{zij zag haar moeder.}{die altoos ernstig naast haar}{ging op de tochten}\\

\haiku{Aldebrand die de.}{kreten hoorde trad buiten}{en knipte zijn ogen}\\

\haiku{Toen hij zijn dochter}{zag viel menig man in het}{gedrang ter aarde}\\

\haiku{De veldheer trad in,.}{de zaal gevolgd door zijn zoon}{en een toortsdrager}\\

\haiku{Slechts het gelaat aan.}{het kruis waar het lampje voor}{brandde zag haar aan}\\

\haiku{Met een kreet trad hij,.}{nader maar snel rees zij en}{vluchtte de zaal uit}\\

\haiku{Eens twistten zij, en.}{sedert zag men de broeders}{zelden te zamen}\\

\haiku{De jonkman wees en,:}{hief zijn armen groot waren}{zijn luide woorden}\\

\haiku{Evor zweeg en legde,.}{een hand op zijn schouder goed}{en zwaar en ernstig}\\

\haiku{In de dageraad,,}{fonkelend van dauw zag zij}{bij het ontwaken}\\

\haiku{De weg steeg en die.}{laatste inspanning werd te}{zwaar voor de Vlaamse}\\

\haiku{een overste kon niet,.}{gemist worden zij zelve}{moest aanvoerder zijn}\\

\haiku{Dat was het waarvoor,.}{Cintia uitging gisteren}{en vanmiddag weer}\\

\haiku{Ik kan u nog in.}{vertrouwen zeggen dat gij}{hier vergeefs komt}\\

\haiku{je boete doet en}{je op laat sluiten in het}{penitentenhuis}\\

\haiku{Nog eenmaal, Mira,.}{kom ik vragen of je het}{overwogen hebt}\\

\haiku{Maar de broeders van;}{San Marco namen het als}{een welkom wapen}\\

\haiku{Het huis behoort je,.}{als je wilt de stad zal je}{eerbiedigen}\\

\haiku{Borso helpt misschien,,.}{wie weet hij is de neef van}{Bongardo en zijn vriend}\\

\haiku{Of zou het zijn dat?}{er voor ieder vuur maar \'e\'en}{hand is die het dooft}\\

\haiku{zou het zijn dat wie,?}{het hier niet  vindt het toch}{eenmaal vinden moet}\\

\haiku{Zij is de moeder.}{die ik verloren had en}{liever dan gij weet}\\

\haiku{zoals dat meisje.}{en laat mij twintig woorden}{met haar spreken}\\

\haiku{mira Ja, dat wil,.}{ik en laat mij dan de rust}{die ik nodig heb}\\

\haiku{ciprian De dwaasheid,,.}{die vrouwen maken heer is}{licht als kinderspel}\\

\haiku{serafina Ja,,.}{ik begrijp het zij was schoon}{en daarom ging je}\\

\haiku{Als het waar mocht zijn,.}{geloof niet dat ik haar heb}{teruggebracht}\\

\haiku{Tegen zulk gerucht.}{kan men niet beter doen dan}{het uit de weg gaan}\\

\haiku{Mijn naam zal hem dan.}{waarborg zijn dat die vrouw geen}{last veroorzaakt}\\

\haiku{Het misverstand van,.}{uw verondersteld geval}{begint al schijnt het}\\

\haiku{Ik heb nu weinig,.}{tijd maar ik liep even binnen}{om te waarschuwen}\\

\haiku{De hoop van zoveel,!}{jaren  op een oprecht}{bestuur een vrije staat}\\

\haiku{Vele nachten zijn:}{er geweest en ik heb dat}{zelf ook veel gevraagd}\\

\haiku{zolang ik een taak.}{heb in de wereld wil ik}{je niet bij mij zien}\\

\haiku{En ik, ben ik zo?}{ver dat ik niet meer zag wie}{Ruffini is}\\

\haiku{Als je hier bent en,}{ik daarginder of als je}{gaat en ik blijf hier}\\

\haiku{zo wanneer ik weg,.}{ben ik zal je stem horen}{en rustig zijn}\\

\haiku{Hij is uitgegaan,,.}{naar de Palazzo als ik}{goed gezien heb}\\

\haiku{Je kunt niet met mij.}{samenwonen voor je mij}{weer recht kunt aanzien}\\

\haiku{En jij, dochter, hebt.}{een strengere hand dan de}{mijne nodig}\\

\haiku{Het vuur dat ons heeft.}{aangevat geeft het geluk}{zonder einde}\\

\haiku{borso Kom terug,,.}{met mij de avond valt ik zal}{je rustplaats zoeken}\\

\haiku{Hoor mij, spreek, moet ik:}{blijven of zullen wij doen}{wat dit teken zegt}\\

\haiku{Wij moeten het zijn,}{zie toch de dwaasheid anders}{te willen dan zo}\\

\haiku{Het is de oude,.}{vraag waarom de een meer dan}{een ander krijgt}\\

\haiku{valdarno Mijn dochter,,.}{natuurlijk alleen van mijn}{dochter spreekt men}\\

\haiku{Kom, Rossi is in,.}{mijn kamer laten wij dit}{met hem bespreken}\\

\haiku{Had die vervloekte,.}{Bongardo niet bestaan het zou}{niet gebeurd zijn}\\

\haiku{Zij klaagt niet, zij spreekt,.}{niet van pijn maar zij ligt heel}{de nacht met open ogen}\\

\haiku{Zeg haar dat ik er,.}{ben zij zal mij dadelijk}{willen horen}\\

\haiku{ciprian Alleen dat.}{het Bongardo goed gaat en dat}{nieuws is goed voor haar}\\

\haiku{Als ik hier vandaan, '.}{moet ga ik in het leger}{t is eender waar}\\

\haiku{Alleen ken ik de.}{naam niet van de plaats waar hij}{gesproken heeft}\\

\haiku{Wie weet een plaats waar,?}{wij kunnen leven wie weet}{de mooiste plaats}\\

\haiku{Ik heb dit nodig,.}{maak de gordel vast en leg}{die zoom wat hoger}\\

\haiku{Die ziek geweest is,,.}{ja en die mij veel zorgen}{heeft gegeven}\\

\haiku{rossi 't Is laat,.}{na een nacht van rust kun je}{beter oordelen}\\

\haiku{Want wat nu van mij, '.}{gevraagd wordt heeft geen waarde}{t is niet te koop}\\

\haiku{Mijn vader mag het,.}{niet weten hij zou denken}{dat ik verdriet had}\\

\haiku{Het zal je kennen,.}{het zal opengaan voor wie het}{zo beschermde}\\

\haiku{God en mij welkom,,.}{sprak Marke tot hem gebied}{over al het mijne}\\

\haiku{bevend hield zij de.}{sluier om het hoofd om de}{blos te verbergen}\\

\haiku{Zie, de dag gaat ten,,?}{avond waar zal ik mij bergen}{waar vind ik toevlucht}\\

\haiku{Neen vriend, neem deze,.}{harp laat horen wat men bij}{u te lande zingt}\\

\haiku{Haastig ging zij, zij,.}{nam de scherf en vond dat zij}{paste op het zwaard}\\

\haiku{Ach moeder, dit is.}{de moordenaar Tristan}{die uw broeder sloeg}\\

\haiku{Toen hij haar klagen.}{hoorde trad hij er binnen}{om haar te troosten}\\

\haiku{Want heeft hij ook geen,.}{deugden tenminste ware}{ik hem lief geweest}\\

\haiku{Daar in de grauwe.}{dag verrees de grimmige}{burcht van Tintagel}\\

\haiku{Schone, zo begon,.}{hij het doet mij leed dat ik}{van u scheiden moet}\\

\haiku{Isolde schreed zuchtend,.}{en treurig voort treurig ging}{Tristan zijns weegs}\\

\haiku{De zorg is mijn en.}{aller vrouwen lot wanneer}{zij verlaten zijn}\\

\haiku{Tristan zit en,.}{denkt aan de vrouwe die zo}{nabij denkt en wacht}\\

\haiku{Maar aanzien kan hij,.}{die wreedheid niet snel rijdt hij}{met zijn heren voort}\\

\haiku{Ook droegen zij de.}{schoonste kleding die ooit man}{en vrouw kan passen}\\

\haiku{En op een morgen.}{drong zijn jachtstoet ver tot in}{het land van Moro{\"\i}s}\\

\haiku{Tristan, er zijn,.}{duizend duizend vragen die}{ik niet vragen mag}\\

\haiku{Hij gehoorzaamde,.}{de roep hij droeg zijn vrouwe}{in zijn arm aan land}\\

\haiku{Edele gast, ik zweer.}{dat ik de duisternis uit}{uw ogen bannen kan}\\

\haiku{dat een zwaard, dat een?}{mensenkind in het leven}{liet onedel is}\\

\haiku{Geduld, mijn gade,.}{wanneer zo een spreekt is het}{in reine waarheid}\\

\haiku{Maar wat zal ik u?}{spreken van het binnenste}{van onze woning}\\

\haiku{Voort met de nar, zegt,,,.}{de koning doet zijn bevel}{voort voort met de nar}\\

\haiku{En Tristan zal.}{haar voeren naar zijn zaal in}{de dageraad}\\

\haiku{Vrouwe, sprak hij, ik,.}{moet mij nederleggen want}{het gif brandt mijn bloed}\\

\haiku{Houd uw bark zeilree,.}{voor morgen bij de dag voer}{mij snel over de zee}\\

\haiku{En duidelijker}{zichzelve en elkander}{ziende ontwaarden}\\

\haiku{Over het kwaad waarvan}{ik Venturi zou kunnen}{beschuldigen moet}\\

\haiku{Haar arm echter hield.}{hij vastgeklemd en hij kneep}{die tot zij schreeuwde}\\

\haiku{Dus moest zij wachten,.}{hoe het lot beschikken zou}{zij had besloten}\\

\haiku{zijn liefde was drong.}{hij aan dat hij doen zou wat}{hem geboden werd}\\

\haiku{Haastig ontkleedde.}{zij zich en borg het gewaad}{in de koffer weg}\\

\haiku{Met Nannina.}{kijk ik iedere dag naar}{de weg of je komt}\\

\haiku{En de volgende.}{dag was het alleen Corso}{die hen onderhield}\\

\haiku{De morgen daarna.}{werd het troostbehoevend hart}{weder diep beroerd}\\

\haiku{daarboven was een,}{schijnsel in het venster de}{stem van het kind klonk}\\

\haiku{Ik vrees dat het niet,.}{meer kan het is zo erg wat}{ik gebroken heb}\\

\haiku{Zij mijmerde en.}{begreep niet dat zij dit niet}{eerder had gekend}\\

\haiku{Voor mij had deze.}{dag het einde kunnen zijn}{van mijn aardse hoop}\\

\haiku{{\textquoteright} {\textquoteleft}Landro, ik heb veel,.}{gedwaald er zijn veel tranen}{geweest  door mij}\\

\haiku{aangezien ging zij.}{achteruit om te zoeken}{waar hun schaduw lag}\\

\haiku{zij nam haar teder,.}{tussen de vingers want zij}{kende de geur nog}\\

\haiku{De gestalte van;}{haar vader durfde zij niet}{meer te naderen}\\

\haiku{En soms zuchtte zij.}{en wist dat zij aan iets van}{een droom had gedacht}\\

\haiku{in haar ziel trilde.}{de schrik nog na van die flits}{der werkelijkheid}\\

\haiku{In de ernst van haar.}{afzondering nam zij met}{scherpe zinnen waar}\\

\haiku{Die avond verscheen zij,.}{ook aan het groot eremaal niet}{daar zij hoofdpijn had}\\

\haiku{Met ongeduld zat.}{zij aan de maaltijden het}{einde te wachten}\\

\haiku{zij voelde zich of.}{zij bevrijd was van een last}{die haar bedrukt had}\\

\haiku{Een luwte had haar,.}{opgenomen zij wist dat}{zij hoog zou stijgen}\\

\haiku{Hij had geschertst over,:}{het hart der vrouwen en zij}{had zacht geantwoord}\\

\haiku{Zij bloosde zoals,.}{zo vaak een vrouw bloost zonder}{te weten waarom}\\

\haiku{De andere zweeg.}{en keek eerst de kamer rond}{en dan naar buiten}\\

\haiku{In de avond hoorde.}{zij de deur van de kamer}{achter zich sluiten}\\

\haiku{Eerst begreep zij het,.}{niet zij dacht voor de moeiten}{die zij te doen had}\\

\haiku{Maar een schuchtere.}{stem fluisterde of zij die}{brief wel zenden mocht}\\

\haiku{Bij het peinzen hoe.}{zij gaan moest dacht zij ook aan}{mensen in de stad}\\

\haiku{Florine kwam soms,.}{in de winkel soms in de}{kamer daarachter}\\

\haiku{Maar wie komt voor iets.}{dat hier niet te vinden is}{geef ik mijn vriendschap}\\

\haiku{Bij de roep van een.}{mens in het donker strekte}{zij de handen uit}\\

\haiku{Ik heb een dochter.}{die haar man verloor toen hij}{krijgsgevangen was}\\

\haiku{Toen zij ontwaakte.}{zag zij hem voor zich staan in}{de opening der deur}\\

\haiku{Terwijl zij haar ogen}{wijd openhield voor het gulden}{licht van de hemel}\\

\haiku{Zij liet hem haar aan,.}{zijn borst drukken zij durfde}{nog niet te spreken}\\

\haiku{Zij schudde haar hoofd,,.}{het moest koud zijn daarbinnen}{en zij schreide niet}\\

\haiku{Genevi\`eve,,.}{haar dochter zou zeker nog}{in het klooster zijn}\\

\haiku{De rest van die brief.}{kon Rose-Ang\'elique}{eerst later lezen}\\

\haiku{toen en toen heb ik?}{aan je gedacht zonder dat}{mijn hart werd geroerd}\\

\haiku{Ach, ik kan hiervan.}{niet spreken omdat geen mens}{dit ooit kan verstaan}\\

\haiku{Ik verlang naar dit.}{ogenblik je te vinden in}{je liefste vreugde}\\

\haiku{zij bogen over hem.}{neder aan de ene en de}{andere zijde}\\

\haiku{Het geluid van een.}{vogel deed hen opzien naar}{de klare hemel}\\

\haiku{Maar eerst wilde ik.}{danken en de naaste kerk}{was San Stefano}\\

\haiku{Toen stonden allen,.}{op want zij moesten aan de dag}{van morgen denken}\\

\haiku{Mijn borst was vol, ik.}{kon niet nog meer geuren van}{de nacht omarmen}\\

\haiku{Of is het dat de?}{zucht des overvloeds verstaan en}{verhoord moet worden}\\

\haiku{Al Zagal, dat zij geen,.}{wens kon hebben want zij was}{hoog boven de stad}\\

\haiku{Zuiver is uw oog,,,.}{zuiver uw hart zuiver zal}{uw naam zijn Safija}\\

\haiku{zoude, omdat hij;}{aan haar ogen zag dat haar ziel}{gezond en goed was}\\

\haiku{De jongelieden.}{met fluiten en trompetten}{begeleidden hem}\\

\haiku{Die eigen dag vroeg.}{zij de abb\'e te komen}{om haar te horen}\\

\haiku{Innig drukte zij,.}{het aan haar borst biddend voor}{het jonge zieltje}\\

\haiku{In Mechtilt was de.}{overspelige geboren}{lang voor zij het wist}\\

\haiku{zuchtend dat zij niet.}{kon zolang het lichaam van}{overwinning juichte}\\

\haiku{Hij troostte haar dat}{de genade zekerlijk}{voor haar verschijnen}\\

\haiku{Zij vreesde niet meer,.}{zij wachtte duldzaam in haar}{biddende aandacht}\\

\haiku{En toen zij de ogen,}{weder opende begreep zij}{de rampzaligheid}\\

\haiku{En op een dag scheen,.}{het haar of het verwachte}{onheil nederviel}\\

\haiku{De mensen zagen,,;}{haar aan verwonderd over haar}{bleekheid meende zij}\\

\haiku{In de dageraad.}{keerde Pietro terug met}{schitterende ogen}\\

\haiku{Dan ging zij bij de}{Bargello om te vragen}{met welke rechter}\\

\haiku{Op de markt ontving,:}{zij de eerste wonderlijk}{bedachte hulde}\\

\haiku{Zij dankte God, zij,.}{had geen smart want alles droeg}{zij in lijdzaamheid}\\

\haiku{Andere stadjes;}{zijn die van burgers die al}{in rijkdom rusten}\\

\haiku{Het handwerk wordt nog;}{in het gebied van Genua}{lonend beoefend}\\

\haiku{van waar komt al dat,?}{goed van waar al die kleren}{van het bruidsgeschenk}\\

\haiku{Wat de jonkman zegt,.}{heeft hij in het woud van een}{kluizenaar gehoord}\\

\haiku{N.B.: het betreft hier.}{wellicht een bewerking van}{een Duitse editie}\\

\subsection{Uit: Verzameld werk. Deel 3}

\haiku{Hij houdt, ondanks zijn,.}{jaren rustig de gouden}{band aan zijn vinger}\\

\haiku{Perzik en amandel,,.}{die rokjes dragen wensen}{glimlach en buiging}\\

\haiku{De dadelpalm lacht,.}{maar luiert en verdient zijn}{naam niet ten volle}\\

\haiku{als hij spreken moet,}{ook al ware er niemand}{om te luisteren}\\

\haiku{Het licht schittert door,}{vocht voor zijn ogen maar hij}{weet niet of het komt}\\

\haiku{Velen ontvangen,.}{in hun roes het bericht dat}{de boot weer vertrekt}\\

\haiku{De eerlijkheid van.}{het hart komt als daglicht van}{de hand die arbeidt}\\

\haiku{Een kus heeft voor hem,.}{maar \'e\'en naam die terzelfder}{tijd een daad beduidt}\\

\haiku{Terstond zweeft het weer,.}{weg omdat het in die halm}{geen behagen vindt}\\

\haiku{Het is de hemel.}{zelf die hun in het vreemde}{land geschonken wordt}\\

\haiku{Hier wiegelen de,.}{palmen in het gulden azuur}{de bloemen geuren}\\

\haiku{Als het hart de trek,:}{naar het verre oord gevoeld}{heeft vraagt het verstand}\\

\haiku{In een warm land brengt.}{de avond ook de koelte die}{bij de rust behoort}\\

\haiku{En drie andere:}{zijn er waarin een geluid}{uit de diepten klinkt}\\

\haiku{Die eerste bundel,,.}{Les po\`emes saturniens werd}{weinig opgemerkt}\\

\haiku{Hij heeft verteld hoe.}{het begonnen was in een}{radeloze smart}\\

\haiku{Parijs, zijn vrouw, zijn,.}{moeder de po\"ezie moesten}{verdedigd worden}\\

\haiku{Maar het kwam uit de,.}{drift van avonturiersbloed niet}{uit leergierigheid}\\

\haiku{Het kind was misschien;}{zeven jaar toen hij ruimer}{dan de meesters zag}\\

\haiku{Van ijdelheid of.}{zelfvoldaanheid heeft niemand}{hem ooit beschuldigd}\\

\haiku{Zij vergaf, vele,.}{jaren daarna maar zij heeft}{hem nooit weergezien}\\

\haiku{{\textquoteright} Rimbaud werd te sterk.}{door het bloed gedreven om}{te kunnen troosten}\\

\haiku{Een politieagent.}{bemerkte de vluchtende}{en de vervolger}\\

\haiku{En hij ging naar de,,.}{kerk iedere zondag des}{morgens en des avonds}\\

\haiku{Rustig vervolgde.}{hij dit eerste jaar van zijn}{vergeefse arbeid}\\

\haiku{De hemel dreef een.}{spel dat hem uit de diepte}{der kelk deed drinken}\\

\haiku{het loon voor zijn pen,,;}{reeds jaren oud kon men met}{stuivers rekenen}\\

\haiku{in de mannentijd;}{gloeide de versmachting naar}{het goddelijk heil}\\

\haiku{Hoezeer was hij een!}{christen geweest met goede}{wil en duldzaamheid}\\

\haiku{Zijn moeder nam haar.}{altijd met dezelfde hand}{waaraan de ring blonk}\\

\haiku{dan ving hij een woord,.}{op dat hij tot dusverre}{van knechts had gehoord}\\

\haiku{Hij voelde een blos.}{op zijn wangen komen bij}{deze ontdekking}\\

\haiku{Had Beatrice,,?}{de rouw afleggend zijn raad}{niet te vroeg gevolgd}\\

\haiku{de gelijkenis}{derhalve kon niet getoond}{worden door hetgeen}\\

\haiku{Merona echter.}{reed snel en zijn dienaar zag}{hem met vreugde aan}\\

\haiku{dat gij meer in mijn.}{gedachten zijt dan ik in}{een brief kan zeggen}\\

\haiku{Vroeg in de morgen.}{reden zij uit de stad tot}{een verlaten veld}\\

\haiku{zoude hetgeen hij.}{op zijn reizen gezien en}{ondervonden had}\\

\haiku{Had hij niet een adem?}{gevoeld of de lente uit}{het zuiden waaide}\\

\haiku{Veronica riep.}{hem eens tot zich op een bank}{in de vensternis}\\

\haiku{Het is goed gezegd,,.}{sprak Ercole mijn trouw is}{inderdaad verdacht}\\

\haiku{Merona ontving.}{het bevel naar Ferrara}{terug te keren}\\

\haiku{hij had Filippo.}{zien binnensluipen en hem}{door een reet bespied}\\

\haiku{Doch de klaarheid zag.}{hij iedere morgen en}{avond na het gebed}\\

\haiku{Zijn moeder plaatste hij.}{voor het venster opdat zij}{hem na kon wuiven}\\

\haiku{Renaldo Maria,.}{een edel man had hij afscheid}{van haar genomen}\\

\haiku{De bode vertrekt,.}{nog voor de middag omdat}{de dagen korten}\\

\haiku{En hij begreep welk.}{geliefd beeld een ander in}{zijn gedachten droeg}\\

\haiku{hij meende dat zij.}{hem dankbaar was en daarom}{niet durfde spreken}\\

\haiku{Ga, zoek, vind spoedig,.}{en kom terug want ook uw}{land heeft u nodig}\\

\haiku{Merona trok door.}{verscheidene gewesten}{der eedgenootschap}\\

\haiku{Maar ik heb ook het,,,.}{geluk gehad de liefde}{mijn vriend de liefde}\\

\haiku{Merona zag dat:}{er een verandering over}{hem was gekomen}\\

\haiku{Maar ik geloof dat.}{het portret het enige is}{dat mij hier nog bindt}\\

\haiku{Een traan viel, maar hij.}{dankte de hemel voor haar}{en hem te zamen}\\

\haiku{de juffers van het.}{hof hadden in een klooster}{toevlucht genomen}\\

\haiku{De degen hield hij,.}{in de hand zeggend dat die}{hem geschonken was}\\

\haiku{Een ieder, mijn vorst,.}{kent zijn plicht zo hij naar zijn}{geweten luistert}\\

\haiku{Het waren lange,.}{dagen waarin de zon scheen}{door geen wolk gestoord}\\

\haiku{Sedert mijn zuster.}{haar zoon verloor hield zij de}{handen te zamen}\\

\haiku{Hoewel ik hem niet.}{gezegd had waar ik woonde}{verwachtte ik hem}\\

\haiku{Ik was elf jaar, een,.}{kind dat weinig kreeg voor de}{mond noch voor het hart}\\

\haiku{Ik voelde dat ik.}{groter was en voor eeuwig}{bevrijd van een band}\\

\haiku{Het was zomer, stil,.}{alleen ging soms achter mij}{een locomotief}\\

\haiku{Zij had alles dat,.}{verblinden kan want alles}{scheen wat het niet was}\\

\haiku{Brand voor het onrecht,.}{bid om genade voor wie}{je je vrienden noemt}\\

\haiku{Roep en smeek tot er.}{zijn die je alles opendoen}{behalve hun ziel}\\

\haiku{in een kinderkreet;}{herkende ik hoe ik zelf}{had willen grijpen}\\

\haiku{Ik verliet het dorp.}{en dwaalde in een stille}{stad een zomer lang}\\

\haiku{Je zou zelf je pijp.}{gemaakt hebben en in de}{kring kunnen zitten}\\

\haiku{De stad wemelde.}{van mensen die er voor hun}{genoegen kwamen}\\

\haiku{ik heb ook gezocht,,}{wanneer Amsterdam sliep naar}{ogen die mij helpen}\\

\haiku{Ik antwoordde dat,.}{ik dit was hoewel niet in}{Holland geboren}\\

\haiku{ik moet betoverd,.}{geweest zijn zodat geen oog}{mij kon waarnemen}\\

\haiku{Het was een wonder,:}{er kon in heel de stad geen}{mooier jongen zijn}\\

\haiku{Maar de vriendin zwoer.}{met een eed het niet verder}{te zullen zeggen}\\

\haiku{Het is of een geest.}{op een zekere plek moet}{wonen en er heerst}\\

\haiku{Wanneer de klok voor:}{noen begon te luiden en}{Gelsomino zeide}\\

\haiku{Men zegt dat elk van.}{die verloren maagden een}{goede huisvrouw werd}\\

\haiku{En Dianora kon.}{die clarissezuster zelfs}{geen antwoord geven}\\

\haiku{Ook al de vrienden,.}{volgden onder elkander}{reeds op wraak zinnend}\\

\haiku{Cecco antwoordde.}{dat hij haar kende en dat}{zij welvarend was}\\

\haiku{{\textquoteright} De stem zweeg en zij.}{hoorden zware voetstappen}{naar de keuken gaan}\\

\haiku{Achter het huis was.}{een tuintje met laurieren}{en klimop begroeid}\\

\haiku{Zij was te zwak om,,.}{zich op te richten zij riep}{maar niemand hoorde}\\

\haiku{Hier klopte zij en.}{stond tot haar voeten als steen}{waren op de grond}\\

\haiku{Zij werd moedeloos,.}{zij had geen traan meer en wrong}{de handen niet meer}\\

\haiku{En waarschijnlijk ook,.}{de nimf niet want zij zweeg en}{maakte geen geruis}\\

\haiku{In de drukte der;}{volgende dagen sleet de}{nieuwheid van het dek}\\

\haiku{Zeker, je hebt voor,.}{mij gedaan behoorlijk naar}{je plicht maar niet meer}\\

\haiku{Daar hij langer thuis,.}{kon blijven viel het afscheid}{zwaarder toen het kwam}\\

\haiku{De derde stuurman.}{werd niet weer aangenomen}{en Bos kwam terug}\\

\haiku{Maar Brouwer meende.}{dat er ander werk te doen}{was dan een wedstrijd}\\

\haiku{Hij nam Pot mee en.}{aan de deur verzocht hij hem}{binnen te treden}\\

\haiku{Zijn vrouw lag op bed,,.}{hij kon haar niets zeggen want}{zij kende hem niet}\\

\haiku{Gedurende twee.}{maanden zat Wilkens alleen}{in zijn woonkamer}\\

\haiku{als de kapitein.}{hem de hand wilde geven}{was elk woord te veel}\\

\haiku{Hij sprong toe, zette.}{hem rustig terzijde en}{vierde zelf de lijn}\\

\haiku{In waarheid hadden.}{het schip en Brouwer er het}{meeste voordeel van}\\

\haiku{Hij had afkeer van;}{het smokkelen en achtte}{het oneerlijkheid}\\

\haiku{Brouwer wist in zijn,.}{hart dat hij het vinden zou}{waar het ook zijn mocht}\\

\haiku{Hij was het toen die.}{beval en zelfs de stuurman}{deed wat hij schreeuwde}\\

\haiku{een Hollander op.}{jaren zou wel iemand zijn}{die het werk verstond}\\

\haiku{bovendien zou hij.}{als tweede stuurman varen}{en Meeuw als bootsman}\\

\haiku{Brouwer deed mee bij,.}{het werk en kreeg zijn deel dat}{hij zelf bewaarde}\\

\haiku{Plotseling zagen.}{de mannen die toehoorden}{hem veranderen}\\

\haiku{Hij bleef staan en vroeg.}{Brouwer hoeveel hij voor het}{schip had geboden}\\

\haiku{Brouwer wist dat het,,,.}{slecht wantslag was drieduims nieuw}{in Iquique maar bruin}\\

\haiku{Toen hij zelf voet aan}{dek zette schudde hij de}{drie mannen de hand.}\\

\haiku{En hij hoorde het,.}{knarsend gedreun in zijn borst}{hij sloeg achterover}\\

\haiku{Every bleef alleen,,,.}{hij keek iedere dag uit}{een maand lang en meer}\\

\haiku{Veel verhalen van}{wat er vroeger geweest was}{kenden zij evenmin}\\

\haiku{De bomen en de,.}{wortelen gaven spijs de}{watervallen drank}\\

\haiku{En daar alles van,.}{de liefde geluk was werd}{er niets verborgen}\\

\haiku{Ook had het kind er,.}{geen nadeel van het was het}{kind van de moeder}\\

\haiku{het bosje waar de.}{twee groene stenen lagen}{in de beek was tap\`u}\\

\haiku{Een oude vrouw kwam.}{aan boord met een banaanblad}{en een varkentje}\\

\haiku{Hij liet tarwe, gerst,,,.}{haver rijst voor hen zaaien}{uien en groenten}\\

\haiku{Kuisheid is onder.}{vrouwen van een zekere}{soort schier onbekend}\\

\haiku{Churchill, Burkitt en.}{Mills zijn handen stevig op}{de rug gebonden}\\

\haiku{Anderen hielden.}{de musketten gericht op}{de sloep beneden}\\

\haiku{Even na middernacht.}{riep de man aan het roer dat}{hij branding hoorde}\\

\haiku{Er stond een ruwe,.}{zee er moest gehoosd worden}{door wie nog konden}\\

\haiku{Zij wachtten tot de.}{morgen en bemerkten toen}{dat het bewoond was}\\

\haiku{Kort daarna kwamen.}{er nog twee mannen zich op}{Pitcairn vestigen}\\

\haiku{Uit ver gelegen.}{dorpen kwamen de mensen}{er ter bedevaart}\\

\haiku{Hierheen trok Lot met,.}{zijn vrouw en zijn kinderen}{zijn knechts en zijn vee}\\

\haiku{Sta op, trek door het,.}{land van noord tot zuid het zal}{u toebehoren}\\

\haiku{Zie de sterren van,.}{de hemel even talrijk zal}{uw nageslacht zijn}\\

\haiku{Ook noemde hij de,,:}{naam die zij hem geven moest}{Isma\"el dat is}\\

\haiku{Vandaag of morgen.}{bespringt mij een troep wolven}{en word ik verscheurd}\\

\haiku{De vluchtelingen,.}{konden niet ver zijn want het}{vee had pas gegraasd}\\

\haiku{Jahweh verlost zijn.}{volk en voert ons naar het land}{aan Abraham beloofd}\\

\haiku{Jozua zag de macht.}{van zijn god en Jahweh koos}{hem tot zijn strijder}\\

\haiku{Daarna klom Mozes.}{op de berg Nebo en hij}{keerde niet terug}\\

\haiku{En onder de scherts.}{die zij maakten gaf  hij}{hun een raadsel op}\\

\haiku{En Simson, die in,.}{slaap was gevallen rukte}{de doek van de muur}\\

\haiku{Hij kon niet meer, maar.}{hij voelde dat het haar weer}{groeide op zijn hoofd}\\

\haiku{Na de plechtigheid.}{nam hij zijn staf en reisde}{naar Rama terug}\\

\haiku{Iets verder lag de.}{bevelhebber en rondom}{sliepen krijgslieden}\\

\haiku{Nu weet ik dat gij,,.}{de grotere koning van}{Isra\"el zult zijn}\\

\haiku{Argeloos zijt gij,,,}{sprak David uw vader weet}{dat gij mijn vriend zijt}\\

\haiku{aan u zal hij dus.}{zeker niet zeggen dat hij}{kwaad tegen mij wil}\\

\haiku{Maar als hij toornig.}{wordt is het zeker dat hij}{mij vervolgen wil}\\

\haiku{Ik zal op deze.}{plek komen met mijn knecht en}{een pijl afschieten}\\

\haiku{Hij weigerde aan.}{de feestmaaltijd te zitten}{en verliet de zaal}\\

\haiku{Des nachts stierf haar kind.}{omdat zij er in de slaap}{op was gaan liggen}\\

\haiku{Toen ik hem nog voor}{het dag was aan mijn borst nam}{om hem te zogen}\\

\haiku{Gelukkig het volk!}{dat door zijn god met zulk een}{vorst gezegend is}\\

\haiku{Hij spotte met de,.}{Almachtige maar in zijn}{hart vreesde hij hem}\\

\haiku{Maar ontneem hem eens,.}{uw weldaden gij zult zien}{hoe hij u verzaakt}\\

\haiku{Ziet, ik heb maanden,;}{van ellende gekregen}{nachten zonder slaap}\\

\haiku{behandel mij niet,.}{als een schuldige gij weet}{dat ik het niet ben}\\

\haiku{Ik zit op as en,,.}{vuilnis zelf niet anders dan}{stof weggeworpen}\\

\haiku{Hij  vroeg om een:}{stift en een tablet van was}{en hij schreef daarop}\\

\haiku{Jozef zat in de,.}{donkere hoek Maria hield}{haar kind aan de borst}\\

\haiku{Wie van u, die twee,?}{stuks kleren had heeft er hem}{een van gegeven}\\

\haiku{Toch was dat uw plicht,.}{en gij weet het het is u}{van kindsbeen geleerd}\\

\haiku{Maar hij weigerde,.}{hij greep hem en bracht hem naar}{de gevangenis}\\

\haiku{De drie vielen op.}{de knie\"en in vrees voor het}{wonderbaarlijke}\\

\haiku{Hij verliet dat land.}{en reisde terug naar het}{huis van zijn vader}\\

\haiku{Dus beslisten de;}{Romeinen af te wachten}{wat er gebeurde}\\

\haiku{De poort stond open, de.}{zon scheen in de eenzaamheid}{van de binnenhof}\\

\haiku{De priesters en de.}{schriftgeleerden leren u}{de wet van Mozes}\\

\haiku{Maar doet niet zoals,,.}{zij doen want zij spreken wel}{maar handelen niet}\\

\haiku{Hij had met zondaars,.}{verkeerd met verdoemden aan}{de dis gezeten}\\

\haiku{Tijdschriftpublikatie,,,:}{in De Hollandsche Revue}{1929 jg. 34 als volgt}\\

\haiku{Het Westerdok{\textquoteright} van,.}{J. Plaat en verklaring van}{scheepstermen Amsterdam,1975}\\

\subsection{Uit: Verzameld werk. Deel 4}

\haiku{De koster lachte,.}{waarop Ras kwaad werd en hem}{voor huichelaar schold}\\

\haiku{Op hem lag de taak.}{de ouderdom van moeder}{en oom te steunen}\\

\haiku{Daar hij liever in.}{een warm land wilde wonen}{koos hij voor Indi\"e}\\

\haiku{De directeur prees.}{hem en voorspelde dat hij}{het ver zou brengen}\\

\haiku{Wat hun veroorloofd,.}{was mochten zij houden}{de rest moest aan wal}\\

\haiku{De kapitein had.}{geen andere bijstand dan}{van de lanspassaat}\\

\haiku{In Jacatra had.}{hij zich onderscheiden door}{moed en bekwaamheid}\\

\haiku{Maartensz tekende.}{het briefje en ontving het}{geld voor het vervoer}\\

\haiku{De kwartiermeester.}{liet de kisten stouwen met}{nog twee andere}\\

\haiku{Zij wisten ook  .}{dat niet allen Nederland}{terug zouden zien}\\

\haiku{Bij Andries in huis,.}{zeide hij dat hij vriend was}{maar ook korporaal}\\

\haiku{Een uur later zei.}{hij terloops dat Manilla}{niet ver weg kon zijn}\\

\haiku{Naar foelie kwam meer,,.}{vraag ook naar noten gebruikt}{bij alle spijzen}\\

\haiku{De enige die Ai,,.}{kon helpen een klein Engels}{schip voer voor hen weg}\\

\haiku{Maar een andere,,.}{keer bij het herhaald verzoek}{beraadslaagden zij}\\

\haiku{De droppels van de.}{regen spatten van de grond}{op of het bloed was}\\

\haiku{Ai wonen noch op.}{een andere plaats waar de}{Compagnie gebood}\\

\haiku{Hij las de brieven.}{van huis die alle zegen}{en welvaart meldden}\\

\haiku{met bewogen stem,.}{noemde hij hem een deugdzaam}{een edelmoedig heer}\\

\haiku{En als het werk goed.}{gedaan werd achtte hij niet}{of het langzaam ging}\\

\haiku{Hij had zo veel zwart,.}{volk vervoerd dat hij ze voor}{enkel koopwaar hield}\\

\haiku{Het gevolg van de.}{belediging moest haar man}{Antonio dragen}\\

\haiku{bomen doormidden,.}{gescheurd huizen ingedrukt}{en stukgeslagen}\\

\haiku{De vreugde onder.}{het pakken en redderen}{duurde korte tijd}\\

\haiku{De Brasser anders,,.}{dan vroeger slordiger}{onverschilliger}\\

\haiku{Toen scheepte hij zich.}{met hen in op het jacht dat}{naar Ambon vertrok}\\

\haiku{Water hadden zij,.}{uit de geulen hout voor het}{drogen uit het bos}\\

\haiku{Yen Pon de winst in,.}{goud doen wegen drie kistjes}{elk van dertig pond}\\

\haiku{Van de raden van.}{Indi\"e had er maar een wat}{geld overgehouden}\\

\haiku{Hij had geen rijd voor.}{onbenulligheden en}{verliet het kantoor}\\

\haiku{De kisten werden,.}{niet geopend zij gingen}{in het grote schip}\\

\haiku{Ai doen vervoeren.}{om nootmuskaat te plukken}{voor de Compagnie}\\

\haiku{De klokken waren.}{maar zwak te horen in het}{geknal en gejuich}\\

\haiku{Hij keek beiden aan,.}{en hij wilde iets vragen}{maar hij wist niet wat}\\

\haiku{Dan bleef hij alleen.}{maar kijken naar de kuil die}{vol was gelopen}\\

\haiku{Toen hij hoorde dat}{zij met de postbode over}{het ijs zouden gaan}\\

\haiku{En hij luisterde,.}{zoals de grootmoeder las}{hij zag het voor zich}\\

\haiku{Maarten hoorde het,.}{en bad hij bleef nu staan tot}{het schieten ophield}\\

\haiku{En elke dag had,.}{hij meer gehoord en eerder}{dan iemand anders}\\

\haiku{Telkens vroeg zij iets,.}{met haar kinderstem dan kwam}{haar adem aan zijn wang}\\

\haiku{Hij wist niet hoe hij,.}{het zeggen moest hij keek haar}{aan en zij wachtte}\\

\haiku{Dan liep hij nog een,.}{eind de dijk op de witte}{maan stond nevelig}\\

\haiku{Maarten lag wakker.}{lang nadat de torenklok}{twaalf had geslagen}\\

\haiku{Het is niets, vrouw, de.}{Heer slaat daar minder acht op}{dan op jouw koffie}\\

\haiku{Op een morgen dat}{Rossaart aan de andere}{oever wachtte zag}\\

\haiku{De commandant was.}{een bejaarde kapitein}{van de schutterij}\\

\haiku{Op de hoek kwam hij,.}{zijn broer Hendrikus tegen}{die voor hem staan bleef}\\

\haiku{Bij de plecht stond de,.}{kolenpot het aardewerk}{hing er aan spijkers}\\

\haiku{Velen wisten van:}{hen te vertellen en hun}{faam verergerde}\\

\haiku{zij namen stenen.}{mee en hakmessen om de}{schuit te vernielen}\\

\haiku{De wijnkoper had.}{haar in huis genomen en}{zocht een dienst voor haar}\\

\haiku{Ik hoop alles goeds,}{voor je daarginds maar kijk niet}{neer op je vrienden}\\

\haiku{hout, spijkers, verf gaf.}{Seebel hem in ruil voor zijn}{aandeel in de tjalk}\\

\haiku{Maar nu je toch hier.}{bent zal ik je zeggen wat}{wij van je denken}\\

\haiku{daar staat de dood voor,,,.}{mij goed ik geef mij over er}{is niets aan te doen}\\

\haiku{Hij boog het hoofd en.}{staarde door de hor naar de}{nevel over de gracht}\\

\haiku{Hij staarde naar het,.}{ijs op de ruit hij dacht en}{schudde soms zijn hoofd}\\

\haiku{Van toen aan merkten,.}{zij een verschil hoewel zij}{het niet beseften}\\

\haiku{Man, zeide zij, ik.}{zal erom huilen dat je}{alleen moet varen}\\

\haiku{- De steeg was nauw, hij;}{bukte laag om door de hor}{naar de lucht te zien}\\

\haiku{'t Is anders een,.}{hele tijd vijfenveertig}{jaar alleen te zijn}\\

\haiku{Hadden ze maar ja.}{gezegd toen ik je bij me}{in huis wou nemen}\\

\haiku{Geld was er genoeg,.}{je was heemraad geworden}{en allang dijkgraaf}\\

\haiku{er zullen er nog,,.}{heel wat verdrinken ik weet}{het zeker zei je}\\

\haiku{neen, dat geld leg ik,.}{opzij dan heeft hij wat meer}{als hij bij me komt}\\

\haiku{hij vroeg Rossaart of.}{zij haar samen wat met de}{post zouden zenden}\\

\haiku{Jij bent de enige,.}{van wie ik het zie maar er}{zijn ook anderen}\\

\haiku{Je wordt stijf in je,,,}{rug zeg je morgen kan je}{toch niet meer varen}\\

\haiku{Hij zat rechtop met,.}{haar hand in de zijne maar}{hij kon niet spreken}\\

\haiku{Het water klotste.}{tegen het boord en de schuit}{trok aan de touwen}\\

\haiku{vroeg hij, het is toch.}{niet de eerste keer dat je}{in Hurwenen komt}\\

\haiku{een mens haast niets meer,.}{te kosten en zij nam geen}{geld meer van hem aan}\\

\haiku{Doe je plicht, dacht hij,,.}{dan en vraag niet het zal wel}{gegeven worden}\\

\haiku{, zij werd oud en hem.}{werd het soms te veel altijd}{alleen te varen}\\

\haiku{De hond, die aan zijn,.}{voeten stond schudde zich de}{sneeuw van de haren}\\

\haiku{In gedachten schold.}{hij op zijn vrouw dat zij niet}{meer had klaargezet}\\

\haiku{Haast u, Achroem, geef.}{het lichaam uw gunsten en}{laat het snel vergaan}\\

\haiku{Zijn spraak was verward,.}{hij zuchtte en zijn stem klonk}{als een zwakke klacht}\\

\haiku{Elphin Bach gaf hem.}{de ketel en wilde er}{geen geld voor hebben}\\

\haiku{Hij leerde en deed,.}{zijn taak maar hij dacht aan een}{bloem en aan een oog}\\

\haiku{heb ik aanschouwd, ik,}{had de keuze tussen meer}{dan honderd allen}\\

\haiku{Geen pot te koken,,.}{geen stofte vegen daar zijn}{deerns en knechten voor}\\

\haiku{De maat loopt over, zei,.}{de bakker en hij schopte}{Matthes de deur uit}\\

\haiku{De jongen spiedde.}{en luisterde in alle}{hoekjes en gaatjes}\\

\haiku{Opeens was het veld,.}{er niet maar zij zag twee ogen}{die haar aankeken}\\

\haiku{Dit was vergif voor,.}{zijn idealisme een snel}{werkend bovendien}\\

\haiku{Toen zij hem vroeg of.}{hij haar herkende bleef hij}{het antwoord schuldig}\\

\haiku{In zijn huis zag zij,,.}{veel ruiten veel geld meer dan}{zij ooit gezien had}\\

\haiku{Een achterlijke.}{sukkel mocht al blij zijn als}{iemand hem aankeek}\\

\haiku{Engeltje, zeide,.}{zij met grommende stem je}{hebt mij slecht gediend}\\

\haiku{Het geneurie zwol,.}{aan tot gedreun want heel de}{menigte deed mee}\\

\haiku{Wel zag hij dat hun.}{bewegingen allengs van}{aard veranderden}\\

\haiku{Vier dagen lag hij,.}{te woelen en te krabben}{zuchtend en vloekend}\\

\haiku{Het ergste was dat,.}{hij het geloof verloor hoe}{weinig dat nog was}\\

\haiku{Opgewekt of hij,.}{de jeugd weer had besteeg de}{sukkelaar de berg}\\

\haiku{Maar zo eenzaam zat.}{die visser dat hij het niet}{durfde aan te zien}\\

\haiku{, opgevrolijkt met.}{doedelzak en klarinet}{van kermisgasten}\\

\haiku{toevertrouw dient gij.}{toch behoorlijk in de echt}{verbonden te zijn}\\

\haiku{Zij spande zich des.}{te meer in om te tonen}{dat zij het wel kon}\\

\haiku{Soms tikte zij mee,.}{eerst met een duwtje in de}{keel en dan de tik}\\

\haiku{Daar denk ik al zo,,.}{lang aan antwoordde zij ik}{wou dat het maar kwam}\\

\haiku{zag ik aan de vorm.}{van uw hoofd waarmee ik u}{een pleizier kon doen}\\

\haiku{Een hoofd vol dikke,,.}{krullen rood als gepolijst}{koper stak eruit}\\

\haiku{Hij zond de kassier.}{om Mik twee rijksdaalders per}{avond aan te bieden}\\

\haiku{In de winkels kon.}{men tekeningen kopen}{die hem voorstelden}\\

\haiku{Hij stond met de rug.}{naar nummer acht gekeerd toen}{hij daar iets hoorde}\\

\haiku{En plotseling hield,.}{het op er kwam toen een rood}{licht over de bomen}\\

\haiku{Zij kleedde zich, deed.}{de grendels af en ging naar}{de kant van het ven}\\

\haiku{Snel stak de wind op,.}{de zee\"en sloegen schuimend}{tegen de boorden}\\

\haiku{De tranen werden.}{parelen en de mond sloot}{tot eeuwige rust}\\

\haiku{Uw hals is, dunkt mij,.}{iets minder gevuld dan voor}{een jaar of twintig}\\

\haiku{Mevrouw sprak met haar.}{man en met haar dochter over}{haar beschermeling}\\

\haiku{Hij was in Londen.}{geboren en getogen}{en hij heette Tim}\\

\haiku{Plotseling voelde,}{hij dat hij daar getrokken}{werd plotseling spron}\\

\haiku{In Ecbatane.}{droeg iedere straatveger}{een gouden ster}\\

\haiku{Er bestaan omtrent.}{de betekenis van dit}{woord misverstanden}\\

\haiku{Voor een andere.}{deur stond zij stil omdat haar}{borst zo zwaar bewoog}\\

\haiku{De rijke burgers.}{zagen haar niet op het feest}{met veel flambouwen}\\

\haiku{Zij klapte in de,:}{handen zij kuste hem op}{de wang en zij riep}\\

\haiku{Dank je, zeide hij,.}{zo help je na je leven}{nog de andere}\\

\haiku{De leverancier.}{had slechts in te vullen wat}{hij geleverd had}\\

\haiku{Zelfs de domste begreep.}{dat hij niet meer mocht kopen}{dan hij nodig had}\\

\haiku{Het was feest in huis,.}{de kanarie zat schel te}{fluiten in zijn kooi}\\

\haiku{Ook Prisca kon geen,.}{brief aan hem zenden hoewel}{zij er vaak een schreef}\\

\haiku{s morgens uit en?}{waar zou zij anders gaan dan}{naar de rivierkant}\\

\haiku{Thuis sprak zij hardop,.}{tot de kanarie die dan}{zweeg en haar aankeek}\\

\haiku{In het struikgewas,}{dat aan de heirweg grensde}{strekte hij zich uit}\\

\haiku{Voor de dochters der:}{rijken tikte hij aan de}{hoed met de woorden}\\

\haiku{Hij stond stil en hij,.}{merkte hoe zij naar hem keek}{of zij iets wenste}\\

\haiku{Vrouw, je hebt mij goed,,.}{verzorgd zeide hij ik zal}{je wel belonen}\\

\haiku{Maar de weg naar de.}{stad was lang en snel kon zij}{met de last niet gaan}\\

\haiku{Maar wanneer hij zijn.}{degen gordde zag hij dat}{haar ogen strak werden}\\

\haiku{Maar binnen een}{jaar nadat hij zijn vader}{was opgevolgd had}\\

\haiku{Wie ik ben gaat je,.}{geen sikkepit aan noem mij}{maar Zo-en-zo}\\

\haiku{Zie je die pot met?}{dat barstje erin waar het}{water doorsijpelt}\\

\haiku{Maar weldra kregen.}{zij bevel een wakend oog}{op hen te houden}\\

\haiku{De luitenant liet.}{de geweren aan rotten}{zetten en ging mee}\\

\haiku{Nu hij zowat een.}{jaar gereisd had was Jorden}{al veel veranderd}\\

\haiku{Jorden tuurde heel.}{de dag en heel de nacht of}{hij het land zou zien}\\

\haiku{Vele dagen liep:}{hij met geschroeide voeten}{en hij vroeg zichzelf}\\

\haiku{In Koerdistan kwam.}{hij waar een man hem het graf}{van zijn ouders wees}\\

\haiku{Lang voer hij tussen.}{de hemel en het water}{naar de horizon}\\

\haiku{Goed, zei Jorden, ik,.}{zal twee minuten wachten}{dan reis ik verder}\\

\haiku{Men herinnere.}{zich de toestand waarin de}{wereld verkeerde}\\

\haiku{De kleermaker, de,.}{zilversmid de kantwerksters}{kregen veel te doen}\\

\haiku{omdat immers zijn.}{ouders hem alles gaven}{wat hij behoefde}\\

\haiku{En toen Daan naar huis.}{ging volgde het dier met de}{neus aan zijn hielen}\\

\haiku{Hier heb je het geld,,,.}{mijn vriend en als het op is}{kom je maar terug}\\

\haiku{En toen de monden.}{gesloten bleven keerden}{de vissen terug}\\

\haiku{Aan het eind van een.}{steil pad bleef hij staan voor een}{hoge bronzen deur}\\

\haiku{Andere wijzen.}{denken anders en weten}{het misschien beter}\\

\haiku{Beste vrouw, het heeft.}{geen nut jezelf te kwellen}{met zulke vragen}\\

\haiku{In het voorportaal.}{naast de zuilen steeg de rook}{uit offervaten}\\

\haiku{Ik heb altijd mijn,.}{brood gehad genoeg om geen}{honger te hebben}\\

\haiku{Maar ik verdiende.}{niet genoeg om mijn schoenen}{te laten lappen}\\

\haiku{Wij zullen het de,.}{oude vragen die weet het}{bij ondervinding}\\

\haiku{Gelukkig dat het,,.}{er is ik heb angst gehad}{waarom weet ik niet}\\

\haiku{Maar voor het jaar ten.}{einde ging begon hij zich}{te verontrusten}\\

\haiku{Hij had met alle.}{schuldeisers gesproken en}{zich met hen verstaan}\\

\haiku{Gerbrand had noch de.}{toon noch de betekenis}{hiervan begrepen}\\

\haiku{Ik doe  wel mee,,.}{antwoordde Diderik maar}{niet boven mijn kracht}\\

\haiku{Eens, toen zij opstond,:}{om naar boven te gaan keek}{hij op en zeide}\\

\haiku{Het huilen ging voort,.}{het was te horen in de}{andere winkels}\\

\haiku{en dat maakte hem.}{zo blij dat hij het hard in}{de armen drukte}\\

\haiku{Frans hief het hoog op.}{naar de bloesems zodat het}{met de ogen knipte}\\

\haiku{Ach broer, zou het niet?}{beter zijn hem met zachtheid}{te behandelen}\\

\haiku{tot hij het weer was.}{die sloeg en de zwakkeren}{de knikkers afnam}\\

\haiku{klagen dat Floris.}{haar kind geknepen had of}{de kleren gescheurd}\\

\haiku{Jongen, vroeg hij, weet?}{je niet dat een dief in de}{gevangenis komt}\\

\haiku{Hij sloeg de zijne,,.}{neer zijn lippen trilden maar}{hij kon niets zeggen}\\

\haiku{Toen Agnete even,:}{bij hem kwam staan om iets te}{vragen zeide hij}\\

\haiku{Dit moet hij voor de,.}{ogen houden en ons voorbeeld}{van rechte zeden}\\

\haiku{En hun moeder keek,.}{ze altijd lachend aan trots}{en vol vertrouwen}\\

\haiku{Daar heeft zij toch ook,,.}{schuld aan zeide Stien en dat}{begreep Floris niet}\\

\haiku{Voor de deur zelf  ,.}{gleed hij uit viel van de stoep}{en brak de enkel}\\

\haiku{Hij nam ze mee op,}{grote wandelingen tot}{voorbij Bennebroek}\\

\haiku{Dan hosten zij arm,,:}{aan arm schreeuwend tegen de}{mensen joelend van}\\

\haiku{Hij ging iedere.}{dag met de jongens tot de}{laatste zaterdag}\\

\haiku{Hij woelde, hij kon,.}{niet slapen buiten zongen}{nog kermisgangers}\\

\haiku{In de straat sprak zij,,:}{niet maar voorbij de brug waar}{niemand ging vroeg zij}\\

\haiku{Zijn oom sloeg maar even,:}{de ogen op zette zijn bril}{recht en antwoordde}\\

\haiku{Het waren kleine,.}{hoge tonen klagend door}{de witte ruimte}\\

\haiku{Je neemt evengoed een.}{stuk koek uit de kast en dat}{is toch geen diefstal}\\

\haiku{Hij lag wakker, hij.}{besefte niet eens hoe de}{gedachten kwamen}\\

\haiku{Dan kan ik in dit,,.}{huis niet blijven  dacht hij}{het wordt hier te erg}\\

\haiku{Maar dan vroeg Jansje,.}{naar de oom in Hoorn of hij}{oud geworden was}\\

\haiku{En vergeet het niet,.}{een mens hoeft geen kwaad te doen}{als hij het niet wil}\\

\haiku{Maar over vijf, zes jaar.}{zou er geen vlek meer op de}{naam van Floris zijn}\\

\haiku{Neen, daar kunnen wij,.}{niets aan doen er steekt meer kwaad}{in dan wij weten}\\

\haiku{En plotseling zweeg,,.}{zij met de hand voor de ogen}{of zij zich bedwong}\\

\haiku{Als je je verstand,.}{verliest ga dan naar je huis}{en denk erover na}\\

\haiku{Meer dan twintig jaar,.}{is zij hier geweest al voor}{je geboren was}\\

\haiku{En als stelen niet,.}{genoeg is dan zal ik nog}{wel wat anders doen}\\

\haiku{Werendonk rees, groot stond.}{hij voor Floris die week en}{de stoel liet vallen}\\

\haiku{Dan baden zij te.}{zamen en Floris ging met}{vochtige ogen heen}\\

\haiku{Toen hij erheen ging.}{de eerste morgen voelde}{hij een verlichting}\\

\haiku{iedere avond hoor.}{ik dat gefluit en anders}{is het hier zo stil}\\

\haiku{Elke avond wanneer.}{hij langskwam stak zij het hoofd}{even uit het venster}\\

\haiku{de moeder was kort.}{zoals Wijntje en droeg een}{muts met keelbanden}\\

\haiku{Je mag niet weggaan,,.}{fluisterde zij weer anders}{komt er ongeluk}\\

\haiku{Maar met volharding,,.}{in het geloof dacht hij wordt}{de ziel behouden}\\

\haiku{Kort en goed, wij zijn}{gekomen om tegen je}{vader te zeggen}\\

\haiku{Alleen het zingen.}{van de treurige liedjes}{kon zij niet laten}\\

\haiku{Kom, zeide zij, help,.}{mij maar liever wat stuk is}{wordt wel weer gemaakt}\\

\haiku{Zeg mij eens, wat heb?}{je op het hart dat je zo}{ongedurig bent}\\

\haiku{zou als er op zijn.}{kantoor een dief geweest was}{die niet gestraft was}\\

\haiku{Om elf uur vond hij.}{zijn broer aan de tafel met}{de bijbel voor zich}\\

\haiku{Maar Werendonk wilde.}{geen ander in de winkel}{of bij hem aan bed}\\

\haiku{Zij was het die hem.}{zijn eten en melk moest brengen}{en de kamer doen}\\

\haiku{Wat zingt die Stien toch,,,?}{zeide hij honderduit waar}{heeft zij dat geleerd}\\

\haiku{Jongen, wij doen voor,.}{je wat wij kunnen maar er}{hindert je nog iets}\\

\haiku{Eens nam hij hem 's}{avonds mee naar de Bavo om}{hem te laten zien}\\

\haiku{En al voelde hij,.}{de moeheid hij stelde het}{uit naar huis te gaan}\\

\haiku{Hij zou navragen.}{wanneer er een boot vertrok}{en wat het kostte}\\

\haiku{Zij staarde naar hem,.}{zij zag hoe wit zijn gezicht}{was in de schemer}\\

\haiku{Alleen om weg te.}{komen uit dat huis had hij}{het geld genomen}\\

\haiku{Je zal toch niet zo,,}{dom zijn hem weer in huis te}{nemen zeide zij}\\

\haiku{als hij gek is laat.}{hem dan oppakken en doe}{hem in Meerenberg}\\

\haiku{Hij wachtte in het,.}{lamplicht aarzelend of hij}{naar hem toe zou gaan}\\

\haiku{Je moet niet denken,.}{dat ik om het brood kom maar}{ik moest in huis zijn}\\

\haiku{Zij stond vlugger op,.}{dan Stien zij was het die een}{stuk brood voor hem sneed}\\

\haiku{Dan komt pas de rust.}{in huis waar je recht op hebt}{met de oude dag}\\

\haiku{Die hem goed kenden.}{merkten dat de gedachten}{hem bezighielden}\\

\haiku{Waar hij keek zag hij,.}{de gezichten groter bleek}{met de ogen donker}\\

\haiku{op Hollands papier,,.}{bij Nijgh en Van Ditmar nv}{te Rotterdam 1934}\\

\subsection{Uit: Verzameld werk. Deel 5}

\haiku{Hij wilde er meer,.}{van weten hij wilde er}{alles van kennen}\\

\haiku{Kaap Bojador was.}{de eerste grote naam op}{de nieuwe weg}\\

\haiku{De golven stonden,.}{daar te hoog een notedop}{kon er niet varen}\\

\haiku{De afstand tussen.}{ontwerp en uitvoering werd}{langzaam overwonnen}\\

\haiku{De nieuwe weg naar.}{het Oosten was al een eind}{verder gevonden}\\

\haiku{De Portugezen,.}{kwamen met een zinnebeeld}{gehaat en gevreesd}\\

\haiku{En het volk vindt een.}{bestaan met het vlechten van}{strooien hoeden}\\

\haiku{Columbus stelde.}{voorwaarden die de vorsten}{voetstoots verwierpen}\\

\haiku{de waardigheid van;}{onderkoning over al de}{te vinden landen}\\

\haiku{Maar toen hij het land.}{eenmaal gevonden had heerste}{de winzucht vooraan}\\

\haiku{Nombre de Dios,.}{was nu op zijn hoede daar}{had hij geen kans}\\

\haiku{Het was het begin.}{der vestigingen waar de}{Engelse vlag woei}\\

\haiku{Maar ongestoord van.}{de vroegere bezitters}{regeerde hij niet}\\

\haiku{Zeildoek was er niet.}{te krijgen in kleden van}{voldoende lengte}\\

\haiku{Altijd wees zij er,.}{de opziener \'e\'en aan wiens}{naam dan gemerkt werd}\\

\haiku{Hij vestigde zich.}{in Londen en heette daar}{de rijkste emigr\'e}\\

\haiku{Aan boord zag men een.}{menigte inboorlingen}{het kamp omringen}\\

\haiku{Gedurende de,.}{nacht brandde er een vuur waar}{een schildwacht bij stond}\\

\haiku{Hiermede was het.}{roemrijkste avontuur van zijn}{leven ge\"eindigd}\\

\haiku{Een konijn is wel.}{het ergste dat iemand op}{het schip kan brengen}\\

\haiku{Alweer mijn schuld, het.}{is maar beter je nergens}{mee te bemoeien}\\

\haiku{'t Is niets, man, was,.}{het antwoord een stakkerd die}{naar huis gebracht wordt}\\

\haiku{Gaan jullie nu maar,.}{gauw naar binnen hier is het}{geld voor  de nacht}\\

\haiku{De halzen rekten,.}{over andermans schouders het}{woord goudtientje klonk}\\

\haiku{Maar nu ondervond.}{hij de moeilijkheden van}{de vertroeteling}\\

\haiku{Een fraai handschrift werd,.}{het genoemd hoewel niet veel}{om op te bogen}\\

\haiku{Ja, dat is zo, kreeg,.}{hij ten antwoord een ieder}{doet het naar zijn aard}\\

\haiku{Het was er \'e\'en meer.}{in het koor van toezieners}{en bemoeials}\\

\haiku{De schoonzoon, gaf Blauw, '.}{ten antwoord was het bests}{avonds thuis te vinden}\\

\haiku{Zij wist dat hij op.}{dit kind meer gesteld was dan}{op de andere}\\

\haiku{hoe zij zich daarvan,.}{vrijwaren moest tevens van}{de krenking der eer}\\

\haiku{Toen hij zweeg was het,.}{stil alleen een wagen op}{de brug te horen}\\

\haiku{Zij durfde niet te,.}{vragen maar zij wachtte op}{een gelegenheid}\\

\haiku{Wat in de hand is.}{moet eruit en de lege}{hand wil weer vol zijn}\\

\haiku{Zij was het echter.}{niet die zich dwingen moest om}{de maat te houden}\\

\haiku{En dacht je dat het,?}{in de graanhandel beter}{ging met die prijzen}\\

\haiku{Misschien hebt u een,.}{erfenis te wachten maar}{dat is geen waarborg}\\

\haiku{Ik wil je anders.}{wel een handje helpen als}{je er wat op weet}\\

\haiku{, kan het altijd nog}{wat beter en als het je}{aangeboden wordt}\\

\haiku{En van die nacht af.}{hebben wij onze opkomst}{aan hem te danken}\\

\haiku{Geesje zat daar veel}{op de kamer en wanneer}{zij beneden kwam}\\

\haiku{Tegen u kan ik,.}{het wel zeggen dat ik mijn}{hart voor ze vasthoud}\\

\haiku{En mijnheer gelooft.}{dat het alles waar is wat}{ze hem voorpraten}\\

\haiku{En geld verdienen,.}{met papieren dat leert hij}{zo op dat kantoor}\\

\haiku{Niemand zag er hem,.}{op rijden niemand wist waar}{hij het bewaarde}\\

\haiku{Ja, zeide hij, het,.}{is waar maar het ging zonder}{dat ik het merkte}\\

\haiku{Ze noemen hem de,,.}{rijke man misschien uit spot}{en dat is hij ook}\\

\haiku{Maar kind, zei hij, lig?}{je jezelf te plagen met}{zulke gedachten}\\

\haiku{En zij zweeg, zij keek,.}{met een lachje maar haar hoofd}{bleef ermee bezig}\\

\haiku{Zij hield haar de arm.}{om de hals zoals zij deed}{toen zij nog klein was}\\

\haiku{Ik heb u ook niet.}{meer gezien nadat uw kind}{u ontvallen is}\\

\haiku{Ach daar weet ik van,.}{mee te praten ik zit hier}{al zo lang alleen}\\

\haiku{Er is berouw van,,.}{mijnheer meer dan ik met die}{pijnen zeggen kan}\\

\haiku{Er was hun, zeide,.}{hij door de weldadigheid}{al zoveel ontgaan}\\

\haiku{Een zekere som.}{mocht hem pas na vijf jaren}{worden uitgekeerd}\\

\haiku{Toen werd er druk aan.}{de deur gescheld en Tonia}{was ook niet jong meer}\\

\haiku{En als u mij een,.}{pleizier wil doen laat mij uw}{geld dan bewaren}\\

\haiku{En nu die man zich.}{aan de drank overgeeft is hij}{tot alles in staat}\\

\haiku{Mijnheer, zei hij, ik,}{dacht dat u thuis zat want ik}{zag daarnet mijnheer}\\

\haiku{Plotseling vielen,.}{er klappen Kompaan zag niet}{wie ermee begon}\\

\haiku{De boekjes lopen,,.}{op zeg je en we hebben}{niets om te geven}\\

\haiku{Toen hij gegeten.}{had hoorde hij iets in het}{kamertje boven}\\

\haiku{Hij deed de handen.}{voor de ogen en legde het}{hoofd op de tafel}\\

\haiku{Maartje, hoe heb je het,?}{kunnen doen jij de enige}{die niet gevraagd hebt}\\

\haiku{Mijnheer liet het door,.}{Maartje houden maar in haar kast}{was niets gevonden}\\

\haiku{Het is beter bij,.}{jou bewaard anders brengt het}{maar verleiding voort}\\

\haiku{Alleen wat zilver.}{heb ik behouden voor de}{onverwachte gast}\\

\haiku{Onder een lantaarn.}{toonde hij hun een handvol}{nieuwe dubbeltjes}\\

\haiku{Maar waarom blijft u?}{dan geloven wat uw zoon}{u heeft voorgepraat}\\

\haiku{uw eigendom en.}{zolang mijnheer Engelbertus}{rondloopt blijft dat zo}\\

\haiku{Ik kon helderder.}{zien dan menig ander van}{mijn tijdgenoten}\\

\haiku{En hij zocht aan de.}{muren naar een spoor waar de}{bliksem was gegaan}\\

\haiku{Hij zat te suffen,,.}{met hoofdpijn zoals altijd}{op zulke dagen}\\

\haiku{Van de jongens in.}{de klas wist hij de namen}{en anders weinig}\\

\haiku{Toen hij weer naar de.}{stad kon gaan reisde juffrouw}{Amalia met hem mee}\\

\haiku{Naar de kerk wilde,.}{hij niet hij had gezegd dat}{men thuis kon bidden}\\

\haiku{Op een avond zette}{zij haar stoel dicht naast hem en}{terwijl zij sprak hield}\\

\haiku{Heiltje richtte zich,:}{op met een kreet van schrik zij}{werd bleek en zij riep}\\

\haiku{De redenen, vrouw,.}{die zal je zien zodra je}{verstand verlicht wordt}\\

\haiku{Nu heeft je man een.}{plicht te doen en die plicht heb}{je mee te dragen}\\

\haiku{Dat zij een hekel.}{aan hem hadden hoefde niet}{gezegd te worden}\\

\haiku{Plotseling rees er.}{luide twist tussen man en}{vrouw en zij liep weg}\\

\haiku{Kleed je eens netjes,.}{aan en kom mee wij hebben}{met je te spreken}\\

\haiku{Het zal er dus van,.}{moeten komen al gaat het}{mij tegen het hart}\\

\haiku{Zij dankte Blok dat:}{hij haar weer geholpen had}{en bij zijn woorden}\\

\haiku{Dan kan ik bijna.}{zeker zijn dat het verkeerd}{gaat met allebei}\\

\haiku{Hij was moeilijk te,.}{bedaren maar eindelijk}{kon hij het zeggen}\\

\haiku{Zij moesten bekrimpen.}{en voor de winter hoopte}{hij op de bakteelt}\\

\haiku{Dikwijls was het stil.}{in zijn hoofd wanneer hij daar}{de druk niet voelde}\\

\haiku{Veel onkruid bleef er,.}{die zomer weg de aarde}{zag er schoner uit}\\

\haiku{moest hij een deel van?}{zijn kracht nutteloos laten}{in de voldaanheid}\\

\haiku{Ze kijken alleen,.}{maar ze bespieden wat er}{in mijn hoofd omgaat}\\

\haiku{Het moet, daar kan je,.}{zeker van zijn ik zie dat}{het geschreven staat}\\

\haiku{De man heeft niets te.}{doen als piekeren en daar}{wordt het hoofd moe van}\\

\haiku{Toen het beter ging.}{was hij nog moeilijker van}{zijn plaats te krijgen}\\

\haiku{Ik dacht anders dat,.}{degene die beproefd wordt}{ervan verbetert}\\

\haiku{Waarom dan moest ik?}{ondankbaar worden voor wat}{mij gegeven werd}\\

\haiku{Leentje lachte even:}{en zij liet  haar hoge}{stemmetje horen}\\

\haiku{Als je denkt dat het,,.}{moet zeide hij kan ik je}{niet tegenhouden}\\

\haiku{Dan was zij niet zo.}{dom geweest haar zoon naar de}{slachtbank te sturen}\\

\haiku{Misschien wachten ze.}{daar al op me. Als mijn broer}{mij maar niet loslaat}\\

\haiku{Het kind vroeg toen of.}{haar vader niet even bij haar}{kon komen zitten}\\

\haiku{We pikken wat, we,,.}{zoeken wat we vinden wat}{we vliegen verder}\\

\haiku{Het was te dwaas te.}{denken dat de plaats daar iets}{mee te maken had}\\

\haiku{Ik was nog een kind,}{toen men zei dat het in zou}{storten zo oud is}\\

\haiku{Bid zoveel je wil,.}{maar waar de duivel woont zal}{het je niet helpen}\\

\haiku{Sofie ging er wel,.}{heen om hem toe te spreken}{maar het hielp niet veel}\\

\haiku{Wel was het moeilijk.}{zich te bedwingen en de}{deur voorbij te gaan}\\

\haiku{Blok kwam dichter bij.}{hem staan en merkte dat hij}{niet gedronken had}\\

\haiku{Met de dikke stok,,:}{die hij nu had langs de weg}{wandelend dacht hij}\\

\haiku{En zo komt er ook.}{een eind aan de ellende}{die ik gezien heb}\\

\haiku{De knecht, menende,.}{dat zij vochten greep hem aan}{en rukte hem los}\\

\haiku{, want behalve de.}{zieke en de werkvrouw was}{er niemand overdag}\\

\haiku{Een ieder, zelfs de,.}{gildeknecht sliep in een bed}{en at van een bord}\\

\haiku{Het was nu eenmaal.}{zo dat voor zorgeloosheid}{betaald moest worden}\\

\haiku{Na een verkenning.}{van de straten komen de}{pleinen aan de beurt}\\

\haiku{Had Dante misschien?}{nog een andere reden}{voor zijn misnoegen}\\

\haiku{Met de cynici.}{en de sto{\"\i}cijnen was}{het erger gesteld}\\

\haiku{Hij bleef de hele,,.}{nacht wakker maar zij zeide}{niets zij snurkte slechts}\\

\haiku{Zij kenden meer dan,.}{de verschillen van blad en}{hout van schors en sap}\\

\haiku{Vroeger danste men.}{om de eerste meidoorn die}{in volle bloei stond}\\

\haiku{Alleen omdat er.}{po\"ezie was met iets dat}{al lang voorbij is}\\

\haiku{Waag het niet een glad.}{gazon te maken voor gij}{kennis daarvan hebt}\\

\haiku{Vrees niet voor het kind.}{want er is in je land een}{kruid voor gewassen}\\

\haiku{steek, tussen april en,;}{augustus uw haak in de}{mond van de kikvors}\\

\haiku{Ook bij velen der.}{latere zeeschuimers was}{de vroomheid een trek}\\

\haiku{Zij vlogen uit, zij:}{vonden Pili zwemmend op}{de zee en hij sprak}\\

\haiku{Neen, zeide zij, hoe.}{kon ik zo verdwalen dat}{ik hun letsel deed}\\

\haiku{Gag\^atiomes, git,,,.}{hetzij dof hetzij glanzend}{verdrijft demonen}\\

\haiku{waarom, gedachten.}{zonder einde en erger}{dan al het geween}\\

\haiku{Als de jonge maan,,.}{helder glanst dacht hij is er}{mooi weer in aantocht}\\

\haiku{Wij hoorden mijnheer.}{praten met iets erger dan}{wij denken kunnen}\\

\haiku{En samen was het.}{toen twee dagen veel geven}{en nog meer vragen}\\

\haiku{De stille witte:}{man verscheen uit de dag en}{knielend zeide hij}\\

\haiku{Hij hield zijn hand naar,.}{het licht van de wolken hij}{wist niet wie hij was}\\

\haiku{Uw hart ruikt bitter,,.}{uw kleur is groen gij zijt oud}{van het zoeken}\\

\haiku{Het oog, zeide een,.}{gedachte neemt de kleuren}{van de planten waar}\\

\haiku{Als de kleur verdwijnt.}{in het zwart van de nacht ziet}{het geen vormen meer}\\

\haiku{Toen zij naast hem stond.}{ging er een koelte en de}{bladeren trilden}\\

\haiku{de zomer was lang.}{voorbij en de duisternis}{hield al te lang aan}\\

\haiku{De wachters stonden,.}{arm aan arm de hoorns bliezen}{naar alle streken}\\

\haiku{Dan moet het een nieuw,,.}{lied zijn dacht hij maar hij kon}{niets nieuws verzinnen}\\

\haiku{Misschien was zij geen,.}{muze maar wat zij wel was}{kon hij niet raden}\\

\haiku{Jonge man, zeide,,,}{zij als je van de liefde}{weten wilt volg mij}\\

\haiku{Kom mee, zeide zij.}{en ik ging achter haar of}{ik geen benen had}\\

\haiku{Wist ik dan niet dat?}{de lach en de dwaasheid maar}{een ogenblik duren}\\

\haiku{die het verlangen?}{heeft geschud en de rimpels}{weer glad zal strijken}\\

\haiku{Is het straf dat ik?}{in mijn tijd van pijn begeerd}{heb en genomen}\\

\haiku{Neen, ik beef, maar ik.}{zal niet vrezen zodra ik}{weer geroepen word}\\

\haiku{Het is om gek te,,.}{worden ja daar is het ook}{zeker voor bedoeld}\\

\haiku{Alleen een rare.}{knoop in je hoofd dat je er}{niet van spreken kan}\\

\haiku{Het was al mooi dat.}{men zich kindergezichtjes}{herinneren kon}\\

\haiku{bromde de man die.}{gisteren geld verloren}{had en wakker lag}\\

\haiku{Maar zij moest de ogen.}{sluiten toen opeens zo veel}{muziek haar omving}\\

\haiku{Wat blijft er van de?}{dag van heden anders dan}{vragen zonder eind}\\

\haiku{dan een hart waar de?}{tranen vloeien tot morgen}{en de dag daarna}\\

\haiku{Het kan vreemd lopen.}{in de wereld en het lot}{heeft rare grillen}\\

\haiku{Ik moet erkennen.}{dat zij mij van het eerste}{ogenblik bekoorde}\\

\haiku{De wereld is een -.}{dansfeest en wie niet danst een}{domoor un tonto}\\

\haiku{Maar we hadden ook.}{al gauw van elkaar ontdekt}{dat we graag dansen}\\

\haiku{Het noodlot had mij.}{tussen die Deursting rechts en die}{Jonas links gezet}\\

\haiku{Toen hij hoorde dat,.}{er straks bal zou zijn werd hij}{opeens levendig}\\

\haiku{Toen vroeg hij weer of,.}{ik het menuet kon daar}{hield hij zoveel van}\\

\haiku{Het gekste is wat die.}{kleine Marion zich in}{het hoofd heeft gehaald}\\

\haiku{Zij begreep niets van,.}{de maten zij dacht dat zij}{het nooit zou leren}\\

\haiku{Gelukkige tijd,,.}{die kindertijd als men niet}{weet met wie men danst}\\

\haiku{Met de derde weet.}{ik nog steeds niet wat voor vlees}{ik in de kuip heb}\\

\haiku{Ja, mijnheer, - op een.}{toon dat de anderen hun}{lachen verbergen}\\

\haiku{Alle stoelen uit.}{de eetkamer werden naar}{het salon gehaald}\\

\haiku{Het waren bijna.}{allemaal meisjes die zich}{aanmeldden als lid}\\

\haiku{Des te beter, zei,.}{Walewijn dan neem ik die}{twee kamers achter}\\

\haiku{En Petronel is,?}{toch eigenlijk van mij geen}{familie meer wel}\\

\haiku{Petronel en ik,.}{zaten in de veranda}{we hoorden alles}\\

\haiku{Zo, dacht ik, voor zo.}{gemeen had ik Petronel}{toch niet aangezien}\\

\haiku{Ik ben haar vader,.}{het spreekt dus vanzelf dat ik}{voor haar blijf zorgen}\\

\haiku{Ik had bedacht dat}{het beter was geen gesprek}{met hem te voeren}\\

\haiku{op mijn kamer en}{het eenvoudigste was hem}{zonder omwegen}\\

\haiku{Het eerste, ik zei,,.}{het al is de dans die hem}{helemaal vervult}\\

\haiku{Bij het dansen van.}{de bolero schijnt fractuur}{niet zeldzaam te zijn}\\

\haiku{Ziedaar, mijnheer en,.}{mevrouw alles wat ik u}{kan mededelen}\\

\haiku{Het ging mij aan het.}{hart dat ik haar alle}{hoop ontnemen moest}\\

\haiku{Het is lastig, zei,.}{Frans maar een Ringelinck}{geeft nooit de moed op}\\

\haiku{Een ieder zoekt het,,.}{geluk dat spreekt vanzelf en}{zoveel mogelijk}\\

\haiku{Dit bevestigde.}{ons vermoeden dat het een}{jonggehuwd paar was}\\

\haiku{En zij keek rond naar,,.}{de bloemen en zij keek hem}{aan zonder een woord}\\

\haiku{Als men de dames.}{kleedt ziet men gauw het verschil}{van de karakters}\\

\haiku{Toen de koorts afnam.}{vertrouwde zij mij veel van}{haar geheimen toe}\\

\haiku{Op een morgen vond,.}{ik haar wakker met een kleur}{en blinkende ogen}\\

\haiku{Samen lukte het.}{ons mademoiselle weer}{naar bed te krijgen}\\

\haiku{En dan tilde hij:}{de slippen van zijn jacquet}{weer hoog op en zei}\\

\haiku{Zo iets te zien is.}{misschien mooier dan het te}{ondervinden}\\

\haiku{Ook deze typen.}{waren niet nieuw voor wie in}{de streek bekend is}\\

\haiku{Spoedig had ik ook.}{reden mijn verblijf hier niet}{langer te rekken}\\

\haiku{Toen hij om halfacht:}{in de morgen mij mijn thee}{bracht zei  Woode}\\

\haiku{De comtesse scheen,.}{hij niet te kennen terwijl}{ik toch beter wist}\\

\haiku{Toen ging zij terug,.}{met een gezicht of zij die}{vent verafschuwde}\\

\haiku{Stil, fluisterde hij.}{en hij voerde mij bij de}{arm naar het venster}\\

\haiku{Uw partner zal er.}{ongetwijfeld waardig en}{sierlijk mee dansen}\\

\haiku{omdat ik ze zelf.}{voor een jaar of drie aan haar}{vader had verkocht}\\

\haiku{Nathalie kwam ook,:}{nog binnen en ook mijn vrouw}{die mij uitlachte}\\

\haiku{Marion, zei ik,}{ik heb je verzocht vanavond}{bij mij te komen}\\

\haiku{Ja, ik weet dat u}{daar ook van overtuigd bent en}{dat verwondert me.}\\

\haiku{zo zeker weet ik.}{dat ze voor Daniel nooit de}{liefde heeft gekend}\\

\haiku{Het was gedaan voor.}{ik het wist en ik had er}{een gloeiend hoofd van}\\

\haiku{En Mr. Sedge dacht.}{ook dat het pijn doet als het}{ritme zo diep zit}\\

\haiku{Toen die kwam zei hij.}{alleen dat mevrouw het heel}{rustig moest hebben}\\

\haiku{Bij de eerste maat.}{al was het of ik van mijn}{stoel moest opspringen}\\

\haiku{Lewis bij ze thuis,.}{en kocht een paar dingen die}{ze goed betaalde}\\

\haiku{Nu, die heer kwam dan.}{vertellen dat ze niets te}{verwachten hadden}\\

\haiku{En zo weinig idee}{van  geld hadden ze dat}{ze niet eens vroegen}\\

\haiku{Zij haalde er de:}{schouders over op en tegen}{mij zei ze later}\\

\haiku{Ik kreeg er drie pond.}{op en daar konden ze een}{paar weken mee door}\\

\haiku{Op die morgen met,,}{dikke mist toen Moralis}{was gestorven om}\\

\haiku{Ik heb dit rustig.}{landschap voor mijn venster en}{doe niets dan vragen}\\

\haiku{Zij duwde mij van.}{zich af omdat zij dacht dat}{ik uit de maat was}\\

\haiku{Wij hebben elkaar,.}{meer gezien voor het eerst al}{jaren geleden}\\

\haiku{Ik zag dat hij het.}{gezicht gewend hield naar het}{hek en nog omkeek}\\

\haiku{omdat wij meenden.}{een ander te horen die}{in de verte floot}\\

\subsection{Uit: Verzameld werk. Deel 6}

\haiku{Kom nu maar naar mijn.}{woning mee en geleid de}{jongskens veilig thuis}\\

\haiku{En Filips maakt daar.}{grapjes over en dan krijgen}{ze weer gekibbel}\\

\haiku{Het is dat geplaag,.}{met zijn broer maar ik zal er}{niet meer van zeggen}\\

\haiku{Moll moet niets van haar,?}{hebben maar hoe is ze met}{je eigen meisjes}\\

\haiku{Je kon wel gelijk.}{hebben dat Filips wat al}{te zorgeloos is}\\

\haiku{En ik zit dikwijls.}{te denken hoe het met mijn}{kinderen zal gaan}\\

\haiku{Het was een lied van.}{vele strofen en de zang}{werd allengs vaster}\\

\haiku{Bijna iedere.}{morgen ook kwam hij ergens}{de tuinbaas tegen}\\

\haiku{Guldelingh wilde.}{nog meer van de vrolijkheid}{van zijn dochter zien}\\

\haiku{De taal heb ik niet,.}{verstaan  maar ik heb het}{niettemin gehoord}\\

\haiku{Geen dag dat zij niet,.}{voor ze leest en soms nog een}{hoog liedeken zingt}\\

\haiku{Dit zou alles niets.}{zijn als ge zijn gezicht niet}{had waargenomen}\\

\haiku{Meen niet dat iemand.}{mij naderen kan zonder}{gehoord te worden}\\

\haiku{Daar heb je het, denkt,,.}{De Kroon jaloezie en jij}{denkt dat zeker ook}\\

\haiku{En dat idee heeft hij,,.}{niet van nu pas neen hij loopt}{er al lang mee rond}\\

\haiku{En het is soms in.}{het holle van de nacht dat}{je ze hoort weggaan}\\

\haiku{Het was wel niet de,.}{lieveling maar dit zou toch}{te erg geweest zijn}\\

\haiku{Praten en nog eens,.}{praten dat ze het haar toch}{niet lastig maken}\\

\haiku{Je mag van geluk.}{spreken dat je kind zo kalm}{en rechtgeaard is}\\

\haiku{Dat is ze, zei De,.}{Kroon en we hebben haar nooit}{veel hoeven leren}\\

\haiku{Hij meent dat er nu.}{ook wel iets zal zijn dat het}{daglicht niet mag zien}\\

\haiku{Allemaal omdat.}{hij eigenaardig is en}{van niemand geliefd}\\

\haiku{Hij heeft niets geen kwaad,,.}{in de zin mevrouw daar kan}{u gerust op zijn}\\

\haiku{De koorts houdt te lang.}{aan en het kan zijn dat er}{iets anders bijkomt}\\

\haiku{Wat konden jonge?}{jongens te zoeken hebben}{bij die oude heks}\\

\haiku{En het schijnt dat hij.}{hier alleen komt omdat hij}{met haar spreken wil}\\

\haiku{Maar ge kunt zelf iets.}{waarnemen dat  we niet}{vertrouwen moeten}\\

\haiku{Zij kijken naar de.}{zuidkant en dat op de dag}{van het nieuwe jaar}\\

\haiku{Ik heb het je meer,.}{dan eens verboden dat je}{dicht bij het huis komt}\\

\haiku{Hij weet het, vraag het,.}{hem wie er in de nacht naar}{hem gekeken heeft}\\

\haiku{Vroeger konden we,.}{het wel vinden al had hij}{het achter de mouw}\\

\haiku{nu willen ze weer,,.}{niet ze worden ook zo suf}{altijd over kleren}\\

\haiku{Als je klein bent is,}{het beginsel altijd zoet}{zijn en gehoorzaam}\\

\haiku{Bastiaan klopte.}{ze een voor een op de nek}{en ze volgden hen}\\

\haiku{Wat daarbuiten wacht,,}{daar weten we niet van maar}{kijk eens rond hoe mooi}\\

\haiku{Ik denk dat ik het.}{nog eens waag bij het oude}{mens van de overkant}\\

\haiku{Wat denken jullie,?}{zou ik er niet eens met hem}{over moeten praten}\\

\haiku{Melodidio, zei hij,.}{bij zichzelf daar moet ik de}{baas eens naar vragen}\\

\haiku{Er is nog meer, er,.}{zijn nog dieper dingen maar}{die begrijpt u niet}\\

\haiku{al rondom deze.}{plaats vindt ge er immers geen}{enkele groeien}\\

\haiku{Maar of Blankendaal,.}{erbij gebaat is dat is}{een andere vraag}\\

\haiku{En dat is het wat,.}{mevrouw ook al begrijpt ze}{heeft het zelf gezegd}\\

\haiku{Je vraagt of ik dan,.}{mijn oordeel niet gebruik maar}{dat doe ik immers}\\

\haiku{Kijk nog maar goed, de.}{dag is nabij dat die daar}{grote mensen zijn}\\

\haiku{De grond lag vertrapt,,.}{gedeukt bezaaid met stukken}{en splinters wit hout}\\

\haiku{De halve maan stond.}{zo tussen schaapjes dat het}{wel vast zal blijven}\\

\haiku{door de schuld van  ,.}{het voorgeslacht of van de}{sterren weet ik het}\\

\haiku{Maar ter andere.}{is de oorzaak van zijn komst}{mij niet gevallig}\\

\haiku{Wat hij ermee deed,.}{weet ik niet maar Jacob wou}{het hem afnemen}\\

\haiku{'t Is aardig van,.}{een jonge man als hij het}{mooie ervan begrijpt}\\

\haiku{En dan opeens die,.}{geluiden om je eraan}{te herinneren}\\

\haiku{, maar dat zwijgen en.}{de eenzaamheid zoeken zie}{ik toch te dikwijls}\\

\haiku{Altijd voelt hij zich,,.}{treurig zegt hij dat hij wou}{dat hij huilen kon}\\

\haiku{Zo rood als de hel,,?}{zeide er een waarom moet}{hij het juist daar doen}\\

\haiku{Je schrikt er soms van.}{zoals die man hem als zijn}{schaduw achtervolgt}\\

\haiku{Ik ben er zeker,,.}{van zeide hij want Fideel}{heeft het zelf gezien}\\

\haiku{waren we hier niet,.}{bijtijds geweest want Fideel}{gaf dadelijk alarm}\\

\haiku{Maar ik geloof niet.}{dat het goed is hem zonder}{toezicht te laten}\\

\haiku{En hij kon het niet,.}{verdragen daarom is hij}{van hier weggegaan}\\

\haiku{Daar moet Platen dan,.}{op letten het zou beter}{zijn als ze thuisblijft}\\

\haiku{Dat de jongeheer,,.}{geen kwaad in de zin heeft ja}{dat geloof ik wel}\\

\haiku{Toen hoorde ik een,,.}{schreeuw mevrouw gelijk of een}{mens gestoken was}\\

\haiku{Toen hij een tweede:}{stoel had neergezet en zij}{zaten sprak De Kroon}\\

\haiku{Want hij was niet de.}{enige die een verbazend}{avontuur beleefde}\\

\haiku{Hij voelde zich wat,.}{doof en suf met een neiging}{om in te slapen}\\

\haiku{Dat is voor een week.}{genoeg als de andere}{er niet van horen}\\

\haiku{Ik beken, sprak de,.}{worm tot zichzelf dat ik die}{dingen niet begrijp}\\

\haiku{Aan wat heb ik het?}{dan te danken dat ik weer}{in de aarde kruip}\\

\haiku{Misschien kom ik nooit.}{te weten wat de oorzaak}{van die dingen is}\\

\haiku{De koetsier zat op,.}{de bok te slapen het paard}{stond diep gebogen}\\

\haiku{De lantaarns stonden,.}{scheef vlak bij de huizen de}{lichten flikkerden}\\

\haiku{Hij klopte op de,.}{deur drie slagen die elk een}{vonnis beduidden}\\

\haiku{Binnen verstomde,.}{het de waard kwam aan het luik}{om wat te liegen}\\

\haiku{Papa daalde de.}{treden van de stoep af en}{kwam naast het meisje}\\

\haiku{Meisje, kijk ze nog,,.}{eens allemaal aan meisje}{je moet kiezen gaan}\\

\haiku{Hoe buitengewoon,}{hij geweest moet zijn begrijpt}{men als men bedenkt}\\

\haiku{Achter de keuken.}{hoorde hij het fris gebruis}{van een waterval}\\

\haiku{Indien ik niet meer.}{verlies vind ik de nieuwe}{kracht niet voor beter}\\

\haiku{Zonder een woord te}{spreken vatten zij Krijn bij}{de schouders en daar}\\

\haiku{Bovendien, je weet,.}{het de vergelding is in}{de hand van de Heer}\\

\haiku{De trein vertrok nog.}{voor de dageraad en de}{wreker reisde mee}\\

\haiku{Bovendien, zoals.}{je nu bent zou je te veel}{zijn voor de landheer}\\

\haiku{Dit verwijt zond hij,,}{hun terug zeggend dat zij}{niet beter deden}\\

\haiku{Geef mij meer goud dan.}{alle koningen en ik}{breng mijn dochter hier}\\

\haiku{Haar geheimen kent,.}{men nooit hoeveel men ook van}{haar houden mag}\\

\haiku{Hebt gij nooit van de?}{ware God gehoord die ons}{geopenbaard is}\\

\haiku{Drinken als Puymys,:}{een vaste uitdrukking werd}{op de wijze van}\\

\haiku{Als je mij voor een,.}{kwartje melk geeft zal ik mij}{niet prettig voelen}\\

\haiku{Sta je eindelijk,.}{op hoorde Lemmerts zeggen}{toen hij wakker werd}\\

\haiku{'t Is gek, zeide:}{Jonas aan het ontbijt en}{Lemmerts herhaalde}\\

\haiku{Beiden kregen zij.}{van ergernis een groene}{tint op het gezicht}\\

\haiku{Zeker kind, dat heb,,.}{je goed gekozen braaf is}{hij ook welgesteld}\\

\haiku{Ik kan toch met mijn,?}{gewoonte van Anton niet}{zo maar breken wel}\\

\haiku{wanneer ik alles.}{zou krijgen waar ik recht op}{meende te hebben}\\

\haiku{Zo is het, zuchtte,.}{Japperotte dat heb ik}{altijd begrepen}\\

\haiku{Hij wees Jansen een,:}{stoel hij bladerde in het}{schrijfboek en hij sprak}\\

\haiku{In andere tij.}{den heeft men wel gemeend dat}{het ding de naam was}\\

\haiku{U kan mij niet eens.}{zeggen of het een jongen}{of een meisje is}\\

\haiku{Toen gaven al die.}{gapers elkaar de arm en}{dansten om hem heen}\\

\haiku{Hij vond dat haar stem,.}{veel zachter klonk hoewel ze}{toch niet fluisterde}\\

\haiku{Juffrouw Das kon het,.}{niet nadoen ze kreeg er een}{traan van in het oog}\\

\haiku{En oom Hendrik zei.}{iets van de opera en ze}{lachten allemaal}\\

\haiku{Het is zo stil in}{de bomen en de kikkers}{mogen eens horen}\\

\haiku{Wie zegt je dat het?}{niet schadelijk kan zijn ze}{te leren kennen}\\

\haiku{Ik zeg dat, omdat,}{ik erover heb nagedacht}{of het heus zo erg}\\

\haiku{Maar waarom zouden?}{we over de mensen praten}{op zo'n stille avond}\\

\haiku{voor de maan opkomt,.}{daar achter de bomen van}{de nieuwe buren}\\

\haiku{Nu je eenmaal hebt}{toegegeven dat we van}{de mensen houden}\\

\haiku{Als hij maar niet zo.}{lelijk was had ik misschien}{toch nog ja gezegd}\\

\haiku{Wat de liefde is,,.}{ging hij voort ik heb het eens}{horen beschrijven}\\

\haiku{Je schaamt je telkens,.}{over een kleur die je krijgt al}{heb je niets gedaan}\\

\haiku{Schei maar uit, zeide,.}{zij ongeduldig ik heb}{het al begrepen}\\

\haiku{Ze is heel lief, maar.}{het zou niet helemaal mijn}{keuze zijn geweest}\\

\haiku{Er worden er acht.}{gebeten en er komen}{er toch maar twee in}\\

\haiku{Meer hoorde Klaartje,.}{niet van het gesprek want zij}{ging de kamer uit}\\

\haiku{Je hoeft het niet te,,.}{geloven hoor want het is}{maar een fabeltje}\\

\haiku{Toen ik zijn kamer.}{binnenkwam wist ik al dat}{ik hier niet moest zijn}\\

\haiku{Ik moet zeggen dat.}{hij een welwillende man}{was en heel bekwaam}\\

\haiku{En hij keek naar de:}{madelief en hij keek Jan}{aan en hij zeide}\\

\haiku{Nu zou ik weleens.}{van Abram willen weten wat}{voor raad hij Jan geeft}\\

\haiku{Ik moet zeggen dat,.}{je het verstandig inziet}{sprak mijnheer Oberon}\\

\haiku{Als het veel is heb,}{je raad van niemand nodig}{als het weinig is}\\

\haiku{en het is waar dat.}{ik er niet van aangetast}{zou willen worden}\\

\haiku{Als die altijd zijn.}{raad gevolgd had zou het slecht}{met hem gegaan zijn}\\

\haiku{U kan mij van niets,.}{overtuigen behalve van}{uw onbeschaamdheid}\\

\haiku{Mevrouw Oberon stond,.}{op een stoel in de handen}{klappend van pleizier}\\

\haiku{de laatste maanden,.}{kon alleen door zwijgen mijn}{verachting blijken}\\

\haiku{Eens, nadat de deur,:}{weer was toegedaan viel zij}{uit in gefluister}\\

\haiku{Juffrouw Das, aan de,.}{hand de blauwe lampion}{stond al aan het hek}\\

\haiku{In het  midden.}{dansten tot voorbeeld mijnheer}{Oberon en mevrouw}\\

\haiku{Er werd weer gedanst,.}{maar mevrouw Oberon was met}{Fientje weggegaan}\\

\haiku{Ik denk dat wij voor.}{ze moeten doen wat wij voor}{de ouders deden}\\

\haiku{Dat had mij die heks,,.}{gedaan dacht ik en ik kon}{ervan genezen}\\

\haiku{Eerst ontdekte je.}{dat Klaartje koel was en niet}{genoeg van je hield}\\

\haiku{Roep Dina eens om.}{de kandelaar en kom hier}{tussen ons zitten}\\

\haiku{Wat de liefde is,,.}{dat weet je niet dus je moet}{er maar naar raden}\\

\haiku{Wel, zeide mijnheer.}{Oberon met een stem of hij}{begon te zingen}\\

\haiku{Blijf die je bent, het,.}{is genoeg dat jij van hem}{houdt en verwacht niets}\\

\haiku{Hij zat weleens met}{verwondering naar haar te}{kijken als mevrouw}\\

\haiku{Je kijkt me aan of,?}{ik wartaal spreek is het zo}{onbegrijpelijk}\\

\haiku{Klara was stil en.}{zij merkte wel dat mevrouw}{dikwijls naar haar keek}\\

\haiku{Maar mogen wij nog?}{iets voor ze verwachten van}{de dorst naar kennis}\\

\haiku{Vertel mij niet dat,.}{je het niet hebt je kan het}{vinden of maken}\\

\haiku{En zij heeft een drang,.}{naar innigheid maar daar staart}{zij nog in donker}\\

\haiku{Eindelijk zijn we,:}{van het spook verlost zeide}{Ida en Toon zeide}\\

\haiku{haar gewoonte was.}{en toen zij voor hen stond had}{zij een rode kleur}\\

\haiku{Boel, die de rest van}{zijn dagen zal schelden op}{zijn landgenoten}\\

\haiku{Ja, wellicht was zij,,.}{in geen kleding grootgebracht}{wat overdreven preuts}\\

\haiku{Er brak een karaf,,.}{er viel een klap de eerste}{in dit restaurant}\\

\haiku{Het is waar, er zijn.}{er waarvan dit nauwelijks}{te begrijpen is}\\

\haiku{En hij sprak zo zacht,.}{dat zij hem soms moesten vragen}{het te herhalen}\\

\haiku{Aan de deur groette,.}{die man maar mijnheer Faustus}{groette niet terug}\\

\haiku{Vrolijk in troepjes.}{trokken zij van de ene naar}{de andere streek}\\

\haiku{En op een morgen.}{ontdekte hij een berg die}{uit de zee verrees}\\

\haiku{Toen hij met Hassan:}{voor de tent kwam riep hij en}{zeide tot de vrouw}\\

\haiku{Toen hij nog een kind.}{was merkten zijn ouders iets}{vreemds op in zijn ogen}\\

\haiku{De toestanden in.}{het land Assassini\"e waren}{onbegrijpelijk}\\

\haiku{En altijd kwam hij.}{onverwacht al was men nog}{zo op zijn hoede}\\

\haiku{Er zijn wel honderd.}{namen voor en niemand weet}{er het rechte van}\\

\haiku{En ik was niet de.}{enige die de verscholen}{romantiek begreep}\\

\haiku{al was het alleen.}{maar omdat er een woord voor}{zijn persoon ontbreekt}\\

\haiku{Wie ik ben weet ik,.}{nog niet dus heb ik mijn naam}{nog niet gevonden}\\

\haiku{Hij dook neder in,.}{de houding van het roofdier}{dat zijn prooi beloert}\\

\haiku{Sta recht, man, zeg wat.}{de begeerte is en ik}{zal het je geven}\\

\haiku{En de menigte,,:}{danste zong waanzinnig met}{\'e\'en enkele kreet}\\

\haiku{Zij hield het verdriet,:}{voor zich maar eens had zij toch}{iets uitgelaten}\\

\haiku{En erger was dat.}{de eenzaamheid van koude}{vergezeld moest zijn}\\

\haiku{Het is niet de angst,;}{van het schuldig geweten}{dat de straf verwacht}\\

\haiku{niet de angst van de,;}{weerloze die zich omringd}{waant van gevaren}\\

\haiku{De gelegenheid.}{werd mij hier geboden om}{haar te overtuigen}\\

\haiku{Het is ook erger,.}{een gevaar te vermoeden}{dat onzichtbaar blijft}\\

\haiku{Nadat hij mij weer.}{lang had laten wachten deed}{de knecht de deur open}\\

\haiku{Hij had redenen.}{genoeg om zich volkomen}{veilig te voelen}\\

\haiku{Men keek ernaar met,.}{ontzag maar geen mens die ze}{aan zou raken}\\

\haiku{Hierin lag nog een.}{reden waarom zij elkaar}{soms niet begrepen}\\

\haiku{Omdat ze moeten.}{en ook alweer omdat ze}{er schik in hebben}\\

\haiku{Waar ik je heen breng,.}{is het altijd mooi dat kan}{ik je beloven}\\

\haiku{Zij huilde niet, maar.}{zij keek hem recht in de ogen}{om medelijden}\\

\haiku{Daar zat een kleine,.}{vogel in grijs met een blauw}{stipje op de borst}\\

\haiku{Zij hield het kooitje,:}{voor haar mond zij fluisterde}{tegen de vogel}\\

\haiku{Deze zucht, vrienden,.}{was zo onvergankelijk}{als het leven}\\

\haiku{En wat schitterde?}{er aan haar vinger en wat}{blonk er aan haar hals}\\

\haiku{'k Vaar voor zulke '.}{kleinigheden u naart}{gindse dorp niet heen}\\

\haiku{Lieve schone, zei,.}{hij na dat kusje komen}{dagen vol geluk}\\

\haiku{Lieve schipper, zei,,}{ze op dat mandje is mijn}{tante zo gesteld}\\

\haiku{Ik zat aan de wand,.}{de draden vielen aan mijn}{voeten en ik sprak}\\

\haiku{Van mijn zes zusters.}{is er een ongetrouwd en}{daarbij tevreden}\\

\haiku{Maar wie nieuwsgierig}{is kan zelf onderzoeken}{hoevele termen}\\

\haiku{Ze hebben het van,.}{hun pa want oom Godfried is}{ook een leugenaar}\\

\haiku{Daar het mooi weer was.}{gingen wij langzaam toen ik}{Alethea naar huis bracht}\\

\haiku{En dat gezegde,,?}{van Slingewiel vroeg ik wat}{betekende dat}\\

\haiku{Je vergist je zelfs,.}{drie keer twee met Socrates}{en \'e\'en met Willem}\\

\haiku{Ik heb niets op je,.}{te zeggen we zijn juist heel}{tevreden over je}\\

\haiku{Toen juffrouw Verkijk:}{vroeg of wij het ook niet guur}{vonden zeide hij}\\

\haiku{Die vent heeft je stuk,,.}{ook gelezen voegde hij}{erbij pas maar op}\\

\haiku{met  een brief in,.}{de hand die hij voor hem op}{de tafel legde}\\

\haiku{Dat had ik trouwens,.}{op zijn leeftijd evenmin heb}{ik misschien nog niet}\\

\haiku{- Dat zij het begreep,,}{natuurlijk begreep kon ik}{aannemen daar zij}\\

\haiku{Zeker, antwoordde,.}{ik je bent even scherpzinnig}{als juffrouw Verkijk}\\

\haiku{De anderen zijn.}{degenen die het geld uit}{hun laden missen}\\

\haiku{Houd daar dus maar je,.}{mond over anders kost het mij}{ook nog mijn baantje}\\

\haiku{De knecht gaf ons de.}{hoeden terug en opende}{weer langzaam de deur}\\

\haiku{Hij ging toch omdat,,.}{zeide hij hij zich aan de}{afspraak moest houden}\\

\haiku{Willem ook besloot.}{te verhuizen en zocht een}{kleinere woning}\\

\haiku{Duizend gulden, ik.}{zou niet weten waar ik het}{vandaan moest halen}\\

\haiku{Ons komt het oordeel,.}{niet toe wij hebben alleen}{de plicht wel te doen}\\

\haiku{hetzelfde zag wat.}{hij mij eens gewezen had}{op een lentenacht}\\

\haiku{Doebel zit bij hem.}{in de schuld en daarvoor heeft}{hij dat geld gebruikt}\\

\haiku{Ik had zelf een plan.}{dat zich in de loop van dit}{voorjaar had gevormd}\\

\haiku{Moet men geloven?}{aan geesten gebonden aan}{een plek der aarde}\\

\haiku{Willem kwam in de.}{deur en keek met een glimlach}{naar de hemel}\\

\haiku{Daar zit meer achter,.}{dan u denkt wij zullen zien}{of dat zomaar kan}\\

\haiku{Je weet er niets van,,.}{zei ze je deugt ook niet voor}{de samenleving}\\

\haiku{Hij leeft nu voor niets.}{anders dan voor wat er in}{de gedachten is}\\

\haiku{Van haar ogen begrijp,,}{ik niets al houdt zij ze nog}{zo lang voor mij open}\\

\haiku{Kijk dan maar eens naar,,.}{die neef van je zeide ze}{en die professor}\\

\haiku{Ik begrijp niet dat.}{Willem die groene vent bij}{zich in huis ontvangt}\\

\haiku{Het is een slecht mens,,.}{gaf hij toe daar heb ik mij}{lelijk in vergist}\\

\haiku{dat zijn vrienden hem.}{een voor een verlaten en}{hij alleen zal staan}\\

\haiku{Zeker, hij is er.}{nog een uit de grote hoop}{die mij verrast heeft}\\

\haiku{Ik mocht vooral de,:}{hoop niet verliezen Alethea}{had ook al gezegd}\\

\haiku{Ja, zegt die plant, er.}{zijn er maar een paar die het}{voornaamste weten}\\

\haiku{Nu ik ouder ben.}{spreek ik zelfs meer dan vroeger}{in de verbeelding}\\

\haiku{- Was het maar waar, zei,.}{Alethea dan zou iedereen}{ook wel deugdzaam zijn}\\

\haiku{van Hein kreeg ik een.}{tekening van de kroon van}{de Westertoren}\\

\haiku{Wie mij via de post.}{twee spotprenten op Willem}{toezond weet ik niet}\\

\haiku{Bijna iedere,.}{dag ging ik naar Haarlem soms}{overnachtte ik er}\\

\haiku{Het kwam mij voor dat.}{hij met een verbijsterde}{geest gesproken had}\\

\haiku{Maar terwijl ik het.}{zeide was het mij of ik}{iets voelde breken}\\

\haiku{Dus je ziet, zeide,,.}{ik al is hij ziek hij leeft}{toch met alles mee}\\

\haiku{Haar ogen tonen dat}{er tranen geweest zijn die}{zij vergeten wil}\\

\haiku{Ach, mijn ouders zijn,.}{er immers ook niet meer en}{zo veel anderen}\\

\haiku{{\textquoteleft}De godloochenaar{\textquoteright},,,,.}{in Groot Nederland 1934 jg.}{32 dl. II blz. 393}\\

\subsection{Uit: De waterman}

\haiku{Hij ging vlug naar het,.}{lichaam toe hij stampte er}{op met zijn geweer}\\

\haiku{zij gezichten en.}{Klaas Tiel deed hem na zooals hij}{uit den bijbel las}\\

\haiku{Hij keek beiden aan,.}{en hij wilde iets vragen}{maar hij wist niet wat}\\

\haiku{Dan bleef hij alleen.}{maar kijken naar den kuil die}{vol was geloopen}\\

\haiku{Toen hij hoorde dat}{zij met den postbode over}{het ijs zouden gaan}\\

\haiku{Maarten hoorde het,.}{en bad hij bleef nu staan tot}{het schieten ophield}\\

\haiku{En elken dag had,.}{hij meer gehoord en eerder}{dan iemand anders}\\

\haiku{Telkens vroeg zij iets,.}{met haar kinderstem dan kwam}{haar adem aan zijn wang}\\

\haiku{Hij wist niet hoe hij,.}{het zeggen moest hij keek haar}{aan en zij wachtte}\\

\haiku{Dan liep hij nog een,.}{eind den dijk op de witte}{maan stond nevelig}\\

\haiku{Maarten lag wakker.}{lang nadat de torenklok}{twaalf had geslagen}\\

\haiku{En dat moest men maar.}{verdragen en werk zoeken}{voor het brood alleen}\\

\haiku{Hij schepte ieder,.}{de aardappelen op het}{bord zij aten zwijgend}\\

\haiku{Het mijne is het,,:}{uwe dat is onze manier}{zooals geschreven staat}\\

\haiku{Het is niets, vrouw, de.}{Heer slaat daar minder acht op}{dan op jou koffie}\\

\haiku{Op een morgen dat}{Rossaart aan den anderen}{oever wachtte zag}\\

\haiku{De commandant was.}{een bejaarde kapitein}{van de schutterij}\\

\haiku{Op den hoek kwam hij,.}{zijn broer Hendrikus tegen}{die voor hem staan bleef}\\

\haiku{Bij de plecht stond de,.}{kolenpot het aardewerk}{hing er aan spijkers}\\

\haiku{Velen wisten van:}{hen te vertellen en hun}{faam verergerde}\\

\haiku{De wijnkooper had.}{haar in huis genomen en}{zocht een dienst voor haar}\\

\haiku{Hoewel hij toen den}{vollen wind had maakte het}{woelende water}\\

\haiku{hout, spijkers, verf gaf.}{Seebel hem in ruil voor zijn}{aandeel in de tjalk}\\

\haiku{Maar nu je toch hier.}{bent zal ik je zeggen wat}{wij van je denken}\\

\haiku{daar staat de dood voor,,,.}{mij goed ik geef mij over er}{is niets aan te doen}\\

\haiku{Hij boog het hoofd en.}{staarde door de hor naar den}{nevel over de gracht}\\

\haiku{Hij staarde naar het,.}{ijs op de ruit hij dacht en}{schudde soms zijn hoofd}\\

\haiku{Hoe verder de schuit.}{voer zoo meer wendde Marie}{het hoofd naar achter}\\

\haiku{Van toen aan merkten,.}{zij een verschil hoewel zij}{het niet beseften}\\

\haiku{Man, zeide zij, ik.}{zal er om huilen dat je}{alleen moet varen}\\

\haiku{De steeg was nauw, hij;}{bukte laag om door de hor}{naar de lucht te zien}\\

\haiku{Hadden ze maar ja.}{gezegd toen ik je bij me}{in huis wou nemen}\\

\haiku{Geld was er genoeg,.}{je was heemraad geworden}{en allang dijkgraaf}\\

\haiku{er zullen er nog,,.}{heel wat verdrinken ik weet}{het zeker zei je}\\

\haiku{neen, dat geld leg ik,.}{op zij dan heeft hij wat meer}{als hij bij me komt}\\

\haiku{hij vroeg Rossaart of.}{zij haar samen wat met de}{post zouden zenden}\\

\haiku{Ik lees nog altijd,,:}{hetzelfde boek zooals je ziet}{en \'e\'en van twee\"en}\\

\haiku{Je wordt stijf in je,,,}{rug zeg je morgen kan je}{toch niet meer varen}\\

\haiku{Hij zat recht op met,.}{haar hand in de zijne maar}{hij kon niet spreken}\\

\haiku{Het water klotste.}{tegen het boord en de schuit}{trok aan de touwen}\\

\haiku{Een man riep telkens:}{wanneer hij den kruiwagen}{grond had uitgestort}\\

\haiku{vroeg hij, het is toch.}{niet de eerste keer dat je}{in Hurwenen komt}\\

\haiku{een mensch haast niets meer,.}{te kosten en zij nam geen}{geld meer van hem aan}\\

\haiku{Doe je plicht, dacht hij,,.}{dan en vraag niet het zal wel}{gegeven worden}\\

\haiku{De hond, die aan zijn,.}{voeten stond schudde zich de}{sneeuw van de haren}\\

\subsection{Uit: De wereld een dansfeest}

\haiku{Het kan vreemd loopen.}{in de wereld en het lot}{heeft rare grillen}\\

\haiku{E\'en herinner ik,,:}{mij in het Spaansch dat zij}{voor mij vertaalde}\\

\haiku{De wereld is een -.}{dansfeest en wie niet danst een}{domoor un tonto}\\

\haiku{Maar we hadden ook.}{al gauw van elkaar ontdekt}{dat we graag dansen}\\

\haiku{Marion krijgt er:}{altijd een hooge kleur van en}{aan het eind zingt ze}\\

\haiku{Het noodlot had mij.}{tusschen dien Deursting rechts en dien}{Jonas links gezet}\\

\haiku{Toen hij hoorde dat,.}{er straks bal zou zijn werd hij}{opeens levendig}\\

\haiku{Toen vroeg hij weer of,.}{ik het menuet kon daar}{hield hij zooveel van}\\

\haiku{Jonas daar alleen,.}{op een stoel zag zitten net}{of hij hoofdpijn had}\\

\haiku{Het gekste is wat die.}{kleine Marion zich in}{het hoofd heeft gehaald}\\

\haiku{Gelukkige tijd,,.}{die kindertijd als men niet}{weet met wien men danst}\\

\haiku{Ja, mijnheer, - op een.}{toon dat de anderen hun}{lachen verbergen}\\

\haiku{Alle stoelen uit.}{de eetkamer werden naar}{het salon gehaald}\\

\haiku{Het waren bijna.}{allemaal meisjes die zich}{aanmeldden als lid}\\

\haiku{En de achternaam.}{moest zooiets geweest zijn als}{sterrewichelaar}\\

\haiku{Hier wandelden wij,.}{ook genietend van het groen}{en de zoele lucht}\\

\haiku{Des te beter, zei,.}{Walewijn dan neem ik die}{twee kamers achter}\\

\haiku{En Petronel is,?}{toch eigenlijk van mij geen}{familie meer wel}\\

\haiku{De lucht was bedekt,.}{maar licht zooals wanneer de maan}{achter wolken schijnt}\\

\haiku{Ik ben haar vader,.}{het spreekt dus vanzelf dat ik}{voor haar blijf zorgen}\\

\haiku{Ik had bedacht dat}{het beter was geen gesprek}{met hem te voeren}\\

\haiku{op mijn kamer en}{het eenvoudigste was hem}{zonder omwegen}\\

\haiku{Het eerste, ik zei,,.}{het al is de dans die hem}{heelemaal vervult}\\

\haiku{Bij het dansen van.}{de bolero schijnt fractuur}{niet zeldzaam te zijn}\\

\haiku{Ziedaar, mijnheer en,.}{mevrouw alles wat ik u}{kan mededeelen}\\

\haiku{Het is lastig, zei,.}{Frans maar een Ringelinck}{geeft nooit den moed op}\\

\haiku{Een ieder zoekt het,,.}{geluk dat spreekt van zelf en}{zooveel mogelijk}\\

\haiku{Dit bevestigde.}{ons vermoeden dat het een}{jonggehuwd paar was}\\

\haiku{En zij keek rond naar,,.}{de bloemen en zij keek hem}{aan zonder een woord}\\

\haiku{Als men de dames.}{kleedt ziet men gauw het verschil}{van de karakters}\\

\haiku{Toen de koorts afnam.}{vertrouwde zij mij veel van}{haar geheimen toe}\\

\haiku{Op een morgen vond,.}{ik haar wakker met een kleur}{en blinkende oogen}\\

\haiku{Samen lukte het.}{ons mademoiselle weer}{naar bed te krijgen}\\

\haiku{En dan tilde hij:}{de slippen van zijn jacquet}{weer hoog op en zei}\\

\haiku{Zooiets te zien is.}{misschien mooier dan het te}{ondervinden}\\

\haiku{Ook deze typen.}{waren niet nieuw voor wie in}{de streek bekend is}\\

\haiku{Spoedig had ik ook.}{reden mijn verblijf hier niet}{langer te rekken}\\

\haiku{De comtesse scheen,.}{hij niet te kennen terwijl}{ik toch beter wist}\\

\haiku{De g\'erant haalde.}{eenige keeren de schouders op}{en keek ge\"ergerd}\\

\haiku{Toen ging zij terug,.}{met een gezicht of zij dien}{vent verafschuwde}\\

\haiku{Stil, fluisterde hij.}{en hij voerde mij bij den}{arm naar het venster}\\

\haiku{Uw partner zal er.}{ongetwijfeld waardig en}{sierlijk mee dansen}\\

\haiku{omdat ik ze zelf.}{voor een jaar of drie aan haar}{vader had verkocht}\\

\haiku{Nathalie kwam ook,:}{nog binnen en ook mijn vrouw}{die mij uitlachte}\\

\haiku{De laatstgenoemde,}{had hem dikwijls ontmoet maar}{haar geheugen was}\\

\haiku{Marion, zei ik,}{ik heb je verzocht vanavond}{bij mij te komen}\\

\haiku{Ja, ik weet dat u}{daar ook van overtuigd bent en}{dat verwondert me.}\\

\haiku{Daarna deden ze.}{weer een mode-dans en}{nog een anderen}\\

\haiku{Het was gedaan voor.}{ik het wist en ik had er}{een gloeiend hoofd van}\\

\haiku{En Mr Sedge dacht.}{ook dat het pijn doet als het}{rhythme zoo diep zit}\\

\haiku{Toen die kwam zei hij.}{alleen dat mevrouw het heel}{rustig moest hebben}\\

\haiku{Bij de eerste maat.}{al was het of ik van mijn}{stoel moest opspringen}\\

\haiku{Nu, die heer kwam dan.}{vertellen dat ze niets te}{verwachten hadden}\\

\haiku{Zij haalde er de:}{schouders over op en tegen}{mij zei ze later}\\

\haiku{Ik kreeg er drie pond.}{op en daar konden ze een}{paar weken mee door}\\

\haiku{Zij duwde mij van.}{zich af omdat zij dacht dat}{ik uit de maat was}\\

\haiku{Had ik haar toen de?}{juiste maat moeten leeren en}{niet weg laten gaan}\\

\haiku{Wij hebben elkaar,.}{meer gezien voor het eerst al}{jaren geleden}\\

\haiku{ik langer gekend,.}{had dan iemand anders ook}{eenige portretten}\\

\haiku{Ik zag dat hij het.}{gezicht gewend hield naar het}{hek en nog omkeek}\\

\haiku{omdat wij meenden.}{een ander te hooren die}{in de verte floot}\\

\subsection{Uit: Een zwerver verdwaald}

\haiku{weer sloeg de eene, de,.}{zware klok de kleinere}{herhaalde den galm}\\

\haiku{De stad was ontwaakt,,.}{in water en dageraad}{zonder geraas}\\

\haiku{Toen haar vrees verzwond;}{zuchtte zij en bemerkte}{hoe hij haar aanzag}\\

\haiku{Hij zat verwonderd,.}{bij  het vuur bedrukt door}{het onheil in huis}\\

\haiku{Het water spoelde,.}{tegen het roer het ijzer}{piepte geregeld}\\

\haiku{Omtrent den middag,.}{trad Seffe binnen de schelm}{was ietwat dronken}\\

\haiku{Er was iets dat hem,.}{verlicht deed ademen hij had}{met de stad gedaan}\\

\haiku{Daar zat Maluse,.}{in het licht van het venster}{de hond lag er ook}\\

\haiku{Hij keerde zich om.}{uit het licht en liep weer door}{de gang naar de straat}\\

\haiku{Toen ook Simon aan,}{boord was riep Meron Joseph}{zijn bevelen uit}\\

\section{W.F. Scheepsma}

\subsection{Uit: De Limburgse sermoenen (ca. 1300). De oudste preken in het Nederlands}

\haiku{We duiden het in (}{het vervolg aan met het siglum}{hbijlage i}\\

\haiku{Het staat wel vast dat;}{Bernard van Clairvaux in de}{volkstaal heeft gepreekt}\\

\haiku{56, een lang traktaat.}{waarin wordt gespeculeerd}{over de Triniteit}\\

\haiku{De zogenoemde {\textquoteleft}{\textquoteright},.}{Vorderspiegel met een deel}{van de tekst van Rd}\\

\haiku{Veel meer persoonlijks,}{komen we over hem niet te}{weten al laat hij}\\

\haiku{Dat zijn aantallen.}{waarbij alleen Keulen in}{de buurt kan komen}\\

\haiku{Het grootste deel van.}{het register is door de}{teksthand geschreven}\\

\haiku{(h, opschrift)  Dits.}{hoe wi in gode bliven}{ende god in ons}\\

\haiku{Het is de moeite}{waard om te onderzoeken}{op welke manier}\\

\haiku{19, 26), eerst in het:}{Middelhoogduits en dan pas}{in het Latijn}\\

\haiku{Ez  sint och niht.}{die hindir rede spulgint}{unde virkerer}\\

\haiku{de stroom, de beide,.}{oevers de boom des levens}{en de twaalf vruchten}\\

\haiku{Daarna volgen de, {\textquoteleft}{\textquoteright}:}{conventsbroeders met al even}{fraaietelling names}\\

\haiku{de vriendin staat voor.}{de heilige ziel en de}{vriend voor onze Heer}\\

\haiku{[...]  Hi ginc Jhesus.}{in den tempel under die}{Juden ende sprach}\\

\haiku{de eerste maakt de,,.}{drank de tweede tapt hem en}{de derde schenkt hem}\\

\haiku{{\textquoteleft}Zalig degene,{\textquoteright}.}{die altijd vreest te vallen}{want hij staat vast}\\

\haiku{Een dergelijke.}{bescheidenheidsverklaring}{is allerminst uniek}\\

\haiku{43 (waartoe dus ook).}{de broeders uit handschrift h}{kunnen behoren}\\

\haiku{45 onderscheidt drie.}{manieren waarop Christus}{tot de mensen spreekt}\\

\haiku{Derste es dasse,.}{hoge ligt ende es oec}{dar ombe seker}\\

\haiku{de bruidegom staat.}{voor Christus en de bruid voor}{de heilige ziel}\\

\haiku{45 kunnen immers ( {\textsection}).}{nauwelijks anders worden}{opgevatzie 2.11}\\

\haiku{Dat is aanzienlijk,.}{later dan in het Frans het}{Duits of het Engels}\\

\haiku{De meningen over.}{de toepassing van deze}{fraaie tekst verschillen}\\

\haiku{Die lugene hat,.}{ein unseliche dochter}{die heizet smeichen}\\

\haiku{in de dertiende.}{eeuw zo  geliefd dat ze}{overal verschijnen}\\

\haiku{Deze passage;}{werd door een middeleeuwse}{hand gecorrigeerd}\\

\haiku{komt vermoedelijk:}{voor op de boekenlijst uit}{Rooklooster.916  b}\\

\haiku{komt vermoedelijk:}{voor op de boekenlijst uit}{Rooklooster.917  c}\\

\haiku{39 en Brief 1, die.}{overigens slechts drie losse}{zinsneden omvat}\\

\haiku{Het zou naderhand}{min of meer integraal in}{omloop zijn gebracht.951}\\

\haiku{We spreken nu in {\textquoteleft}{\textquoteright}.}{navolging van Oliver van}{Luikse psalters}\\

\haiku{het coumnandement;}{op basis waarvan li fin}{amant moet handelen}\\

\haiku{zie verder 316, 11,,,,,,,).}{en 13 321 9 16 20 24}{en 27 en 322 7}\\

\haiku{Untersuchungen zur.}{Geschichte der Metapher vom}{Herzen als Kloster}\\

\haiku{Publications.}{de l'Institut d'\'Etudes}{M\'edi\'evales}\\

\haiku{H\"aring, N. (ed.), {\textquoteleft}Der,:}{Literaturkatalog}{von Affligem in}\\

\haiku{Hollander, August, \&, {\textquoteleft}{\textquoteright}.}{den Ulrich Schmid Het Luikse}{Leven van Jezus}\\

\haiku{Publications.}{de l'Institut d'\'Etudes}{M\'edi\'evales}\\

\haiku{Publications.}{de l'Institut d'\'Etudes}{M\'edi\'evales}\\

\haiku{Te verschijnen in (),.}{P. Broomans e.a.red. 1}{have heard about you}\\

\haiku{Sinclair, Keith Val,.}{French devotional texts}{of the Middle Ages}\\

\haiku{Publications.}{de l'Institut d'\'Etudes}{M\'edi\'evales}\\

\haiku{Vankenne, A., (vert.),.}{Vie de Marie d'Oignies par}{Jacques de Vitry}\\

\haiku{Pierre Roch\'e \& Guy (),.}{Lubrichonred. Le Moyen}{Age et la Bible}\\

\haiku{28, Dit sprict van xii,,,.}{dogeden die ane Gode sin}{93 210 222 269 Ls}\\

\haiku{4 (vgl. Kern 1895, 215,-),.}{1116 en in diens Duitse}{tegenhanger Rd}\\

\haiku{Maagdendries kan ook.}{niet de eerste bezitter}{van h zijn geweest}\\

\haiku{vgl. de r. 10-12,,.}{waar a wordt weergegeven}{dat van br\r{u}der spreekt}\\

\haiku{230Over de datering.}{van het schrift van het eerste}{gedeelte van hs}\\

\haiku{Volker Honemann ().}{M\"unster bereidt een editie}{van deze tekst voor}\\

\haiku{oudere versies,.}{van de Vita Lutgardi}{waaronder een Ofr}\\

\haiku{372Verwijzingen {\textsection}.}{naar beeldmateriaal van}{h in 1.1 n. 67}\\

\haiku{60 hadden we het.}{palmboomtraktaat verwacht op}{de plaats tussen Ls}\\

\haiku{425Een afbeelding van,.}{de opening f. 4v-5r in}{Hamburger 1990 afb}\\

\haiku{437De gehele,,-,;}{interpolatie in Kern}{1895 464 20466 14}\\

\haiku{36 (Kern 1895, 510, 20-) (,-,,-,,-).}{25 en 38528 2226 531}{1524 531 2729}\\

\haiku{De Bruin 1970a, 268, 15--,,-.}{18 en Corpus Gysseling}{113 633 1519}\\

\haiku{ook Reynaert 1975, 239-.}{246 beschouwt de Parijse}{tekst als zelfstandig}\\

\haiku{op de p. 26-32.}{worden e en f (= het}{laatste stuk van Ls}\\

\haiku{Reypens 1964, vooral ( {\textsection}),,.}{cap. 4deels aangehaald in}{1.5 269 275 en 276}\\

\haiku{Moltzer 1875, 536-537 (,-), ().}{vgl. Zacher 1842a 347348 en}{Schneider 1987b 6053v}\\

\haiku{785Schmidtke 1982, (-).}{nr. 9p. 3435 is het}{tweede deel van Rd}\\

\haiku{935Zie Van Mierlo,-.}{1929 en ook Van der Zeyde}{1934 128129 en 160}\\

\haiku{1017h leest hier vol. Kern,,;}{1895 465 11 herkent hier geen}{kopiistenfout}\\

\haiku{1025Schweitzer 1997, 192-193 (),- ()- ().}{n.a.v. vraag 25 208209vraag 73}{en 212213vraag 102}\\

\section{Bert Schierbeek}

\subsection{Uit: De andere namen}

\haiku{ik ga even aan de}{berm van de weg liggen want}{je kunt nooit weten}\\

\haiku{was noords Loyola}{maar de smart drong ravijren}{diep in mijn lichaam}\\

\haiku{heet hoofd om de muur,}{van het weer soms loopt de film}{mis zei het meisje}\\

\haiku{en ik dacht moest ik,}{haar voorbeeld zijn en ik zei}{het haar maar zij zei}\\

\haiku{zij lagen bloot op}{de tafel en knikten de}{aanwezigen toe}\\

\haiku{dat is de zucht van}{de gestorven sultanes}{in het Alcazar}\\

\haiku{maar ik vluchtte weg}{van haar want haar lach las in}{mij vele ziekten}\\

\haiku{het mooie dier tekent}{filmische waanbeelden op}{de huid van het kind}\\

\haiku{het astronomisch}{zicht zou verloren zijn maar}{men weet veel tudo}\\

\haiku{wij hebben nog zo}{weinig gemaakt op aarde}{dat stof werd en beeld}\\

\haiku{het samengaan der}{tegenstellingen lag niet}{in de doctrines}\\

\haiku{een man uit de stad}{leeft vegetatief van het}{land en zijn buren}\\

\haiku{dan zie ik de trams}{vol bedelaars langskomen}{binnen een geel licht}\\

\subsection{Uit: Verzameld werk. Deel 2. Het boek ik. De andere namen. De derde persoon}

\haiku{Wij gaan daar dieper;}{op in bij Bert Schierbeek en}{het onbegrensde}\\

\haiku{Schierbeek's werk sluit aan;}{bij het Dada{\"\i}sme en}{Surrealisme}\\

\haiku{o droesem... want IK,...}{GOD die recht sta in mijn ziel}{en het verdriet ken}\\

\haiku{je bent het bord, het...}{eten en het borduursel en}{alles doe je zelf}\\

\haiku{alle kalkputten:}{heeft hij gezien en zijn hoofd}{geschud en gezegd}\\

\haiku{en het vlees dat hard...}{is en helemaal niet zwak}{en niet wachten kan}\\

\haiku{In mij leeft het volk...}{het onmondige en het}{kent geen genade}\\

\haiku{een wit boetekleed,...}{past u en mij hoog aan de}{hals dicht gebonden}\\

\haiku{Hij leefde hierdoor.}{vijf minuten langer dan}{de bedoeling was}\\

\haiku{- Je zoekt het te ver,.}{lieve Lilith dat was van}{voor het paradijs}\\

\haiku{De Grote Dag is!!}{gekomen O Lawd have}{thy mercy upon us}\\

\haiku{God gaat terug naar,?}{zijn hemel maar mijn hemel}{mijn Ik mij waarheen}\\

\haiku{het hart was hard het}{godvergeten fatsoen o}{ons fatsoen o god}\\

\haiku{Ik denk wel dat ik.}{eens de Vestaalse maagd zal}{worden die ik lees}\\

\haiku{ik l\'a\'at u allen,}{ingaan ik wil u allen}{vrouw zijn en mij zelf}\\

\haiku{ik weet dat ik de}{oeverdieren zal zijn die}{hun kroost vermalen}\\

\haiku{want het goed is van...}{het kwaad losgeslagen en}{zij bestaan niet meer}\\

\haiku{We liggen tussen.}{de bene het uur nu een}{eeuw te verzweten}\\

\haiku{omdat hij dood zou '...}{zijn van het vaderschap en}{t gebeur der ziel}\\

\haiku{Zullen mijn vingers?}{mijn hand de papavers zijn}{en rood van het zon}\\

\haiku{o, liefste de grof,}{van het graf van mijn hand in}{de mond o liefste}\\

\haiku{ik heb ze op hun,...}{sodemieter gegeven}{allemaal schrijft hij}\\

\haiku{O, ik Dolle Schuld...}{zal de anjer het lied van}{dit leven zingen}\\

\haiku{de voorgang staat open...}{en de deur ook nou is het}{niet zo benauwd meer}\\

\haiku{zij verscheen hem 't}{gezicht over de lakens van}{help ons de weiden}\\

\haiku{ik heb mijn hele}{leven veel gereisd en al}{is het ook oorlog}\\

\haiku{de mensen lopen}{geheel langs je zij en gaan}{heen om te treden}\\

\haiku{en ik alleen in......}{dit huis hij heeft nooit van een}{ander gehouden}\\

\haiku{goolgraag was geen slecht......}{mens de messen zo stomp en}{bang voor een dooie vis}\\

\haiku{want allen was mij...}{toen veel in de vege me}{monis van mijn lijf}\\

\haiku{je stuurt urine naar,...}{de kikkers ze drinken het}{op en worden geel}\\

\haiku{ik drink het uit jou...}{uit alles wat je van jou}{hebt en fijn de pijn}\\

\haiku{Laat je man Van Dijk.}{de liefde verzekeren}{die jij voor hem voelt}\\

\haiku{want 't was nog geen}{oorlog zeiden ze want ze}{zouden zeiden ze}\\

\haiku{de verschrikking zie...}{je en wat er aan messen}{uit de stenen steekt}\\

\haiku{zu h\"angt in allen...}{w\"andern des jugendlebens}{zum traufiel trocken}\\

\haiku{naar buiten want de}{sleutels passen niet van het}{geluk dat open wil}\\

\haiku{wie liefheeft zal de}{teller en noemer van de}{namen van liefde}\\

\haiku{haben wir gar nicht,,,,...}{gewollt wir wollten du sollst}{du kannst du musst}\\

\haiku{wir k\"onnten ja......}{gar nichts gewusst haben da}{wir nicht wollten wir}\\

\haiku{zijn hoogmis in de}{urnen ons zelf tot de uren}{vertekel gegaan}\\

\haiku{mij is de bokaal}{van de balgen der liefde}{de ofri gaan doen}\\

\haiku{de dood legt lange}{voeten in mijn woord nu het}{heeft zich uitgekleed}\\

\haiku{wat de distel in}{de rode bloem van het bloed}{schrijft de verschrikking}\\

\haiku{- De 6e druk is een.}{fotografische herdruk}{van de 1e druk}\\

\haiku{ik ga even aan de}{berm van de weg liggen want}{je kunt nooit weten}\\

\haiku{dat in mij staat een}{afschuwelijk en heerlijk}{beeld een wereldbeeld}\\

\haiku{en ik dacht moest ik,}{haar voorbeeld zijn en ik zei}{het haar maar zij zei}\\

\haiku{zij lagen bloot op}{de tafel en knikten de}{aanwezigen toe}\\

\haiku{dat is de zucht van}{de gestorven sultanes}{in het Alcazar}\\

\haiku{maar ik vluchtte weg}{van haar want haar lach las in}{mij vele ziekten}\\

\haiku{dat is long zei zij}{zacht na drie weken vond men}{het lijk van de zoon}\\

\haiku{het mooie dier tekent}{filmische waanbeelden op}{de huid van het kind}\\

\haiku{het astronomisch}{zicht zou verloren zijn maar}{men weet veel tudo}\\

\haiku{wij hebben nog zo}{weinig gemaakt op aarde}{dat stof werd en beeld}\\

\haiku{het samengaan der}{tegenstellingen lag niet}{in de doctrines}\\

\haiku{een man uit de stad}{leeft vegetatief van het}{land en zijn buren}\\

\haiku{dan zie ik de trams}{vol bedelaars langskomen}{binnen een geel licht}\\

\haiku{- De 5e druk is een.}{fotografische herdruk}{van de 1e druk}\\

\haiku{ik ben koning en}{dienaar en het offerdier}{nadert in die tijd}\\

\haiku{zij ziet de vrouwen}{die wakker liggen in het}{dorp van hun leegte}\\

\haiku{en in alles was}{verweg zo de gedachte}{wel naar binnen niet}\\

\haiku{wat is water, zegt}{iemand het schip drijft er op}{het vaart Hendrik gooit}\\

\haiku{het spelen en niet}{weten de vader leeft tot}{wat de moeder was}\\

\haiku{dit is analogie}{en verbondenheid met die}{leven en sterven}\\

\haiku{de moeder houdt het}{mooiste kind van de wereld}{tegen de ramen}\\

\haiku{leert de juffrouw de}{meisjes de kruissteek in de}{hoofden der jongens}\\

\haiku{ik ben een kind in}{uw omkleding sprak hij door}{de huid van dit huis}\\

\haiku{hij vormt zijn mond tot:}{sacrale staat en leest met}{magie in zijn stem}\\

\haiku{omdat we 't goed}{menen eet je wortels want}{wortels zijn gezond}\\

\haiku{als je de slaaf van,}{de wereld en zijn buurman}{wil zijn schreeuwt het kind}\\

\haiku{die honger hebben}{we allemaal maar tussen}{de honger en mij}\\

\haiku{de man valt neer met}{een vloek in het gebed van}{zijn bloed het leven}\\

\haiku{een klooster binnen}{het klooster halen zij de}{beelden en zetten}\\

\haiku{hij koopt die jas en,}{hij groeit er nooit meer uit met}{die jas zei een vriend}\\

\haiku{dit land is nat soms}{als de missouri maar de}{dijken zijn beter}\\

\haiku{en zo stond het te}{vroeg stil dan kan men het niet}{meer op gang brengen}\\

\haiku{loopt loeiend het land}{in en waarschuwt de mensen}{dat het zal komen}\\

\haiku{hier zijn wij zoals}{wij gaan over u ~ in de}{ontferming van wat}\\

\haiku{De 3e druk is een.}{fotografische herdruk}{van de 1e druk}\\

\section{Anda Schippers}

\subsection{Uit: De kikker die zichzelf opblies en andere Middeleeuwse fabels}

\haiku{De aap zag hoe de.}{vos Reinaart gezegend was}{met een lange staart}\\

\haiku{Beschouw je jezelf,.}{als aanzienlijk dan meet en}{oordeel je jezelf}\\

\haiku{{\textquoteleft}O zuster, ik raad, - -!}{je aan kom snel gezond en}{wel van die troon af}\\

\haiku{Bovendien moet je.}{alle kosten betalen}{die hier zijn gemaakt}\\

\haiku{Ik voel me erg slecht{\textquoteright},.}{dan is er geen hoop dat hij}{zijn ziekte overleeft}\\

\haiku{Dit hoorde de vos,.}{die de werking van het kruid}{verbena kende}\\

\haiku{Dankzij deze drie.}{zaken heb ik zo lang en}{zo gezond geleefd}\\

\haiku{Daarom is het goed.}{om een slechte gewoonte}{direct te weerstaan}\\

\haiku{Waarover Aesopus de,.}{volgende fabel vertelt}{over twee ratten}\\

\haiku{De zeug en de wolf.}{De vierde fabel gaat over}{de zeug en de wolf}\\

\haiku{Want de boer maakte.}{de ossen los van de ploeg}{en dreef ze naar huis}\\

\haiku{en degene die,.}{het had zwoer evenzo dat hij}{het niet had gepakt}\\

\haiku{Toevallig is het:}{personage Aesopus daar}{een goed voorbeeld van}\\

\haiku{Wolff en A. Deken.}{Historie van mejuffrouw}{Cornelia Wildschut}\\

\section{Frans Schleiden}

\subsection{Uit: Hazegerf}

\haiku{Kaplaon komt mit ':}{t Fes9 De vrouw Peltjes komt}{i geloope en reep}\\

\haiku{Der Pitter zaat n\"uks,:}{e bezoog der Joep ins es}{wente zage wool}\\

\haiku{{\textquoteleft}E kamp os verdaat,.}{v\"or e kiekt graat of wente}{kaffie sjmoekelet}\\

\haiku{e kratset twei drei.}{maol i gen \`e\"ed en sjprong}{op der ozze aa}\\

\haiku{'t Loewet ovvesklok.}{en der Erwinus zonk i}{ge veld oppen knee}\\

\haiku{Diej moe\"ete de.}{munneke gidder ovvend mit}{n\`eme n\`o gen kirk}\\

\haiku{Der broor Andreas.}{mit der Johannes w\`ore}{va ter Munnekef d\`o}\\

\haiku{{\textquoteright} 'ne Mond dern\`o w\`or '.}{der nonk Macheel i ge veld}{ant mie\"ene}\\

\haiku{{\textquoteright} Der Mertens w\`or al:}{pruttelent\`ere76 n\`o gen}{wei i gegange}\\

\haiku{En wurkelich, de.}{dreide naat w\`or de pieng nit}{mie\"e oet te houwe}\\

\haiku{e zow gevelles,.}{koame der vadder hei pieng v\"or}{gek te w\`e\"ede}\\

\haiku{De hoezer zunt dan.}{frisj wiej geweisje en}{ze blinke wiej nuj}\\

\haiku{E paar jong jonge.}{kompte mit br\`e\"er96 vol}{vlaam oet ge bakkes97}\\

\haiku{{\textquoteright} Der lie\"erer heel ze '}{i gen heng en leet zem}{zie\"e en sjroevvet ze}\\

\haiku{en dat geet nie mie\"e -.}{ich bi vunf en achsig oet}{gene braokmond139}\\

\haiku{t Is es went et.}{kie\"em va wied uvver ge veld}{en lants gen hage}\\

\haiku{{\textquoteleft}Komt sjtraks nao de,{\textquoteright}.}{hoe\"emis bei mich op de}{pasterei alle veer}\\

\haiku{Doew sjleep e ruhig ',,:}{i. Ent murges wie der}{dokter komt zaat d\`e}\\

\haiku{Noe z\`enete zich:}{in alle hoezer de luuj}{en b\`enete ze}\\

\haiku{{\textquoteleft}D\`e,{\textquoteright} zaat-e ze, {\textquoteleft}dat,.}{is der Kl\"os d\`e waor op}{w\`e\`eg nao gen kirk}\\

\haiku{Doew sjtonge ze.}{in der sjtal en dao waor}{ie\"elend en ermood}\\

\haiku{Der wink sjn\`e\"et 'm.}{kaod lants gen oe\"ere}{en uvver zie gezich}\\

\haiku{{\textquoteleft}Simeon,{\textquoteright} zaat ich, {\textquoteleft},..... '....}{jong dink ins nao watste deest}{Alderhilligste}\\

\subsection{Uit: De hillige vaggen durp}

\haiku{E minnig medje '-.}{sjtongt murges 67}{maol v\"or des sjpeegel}\\

\haiku{Ze zitte iggen.}{hoes bei der \`oavend21 mit de}{kap oppene kop}\\

\haiku{es wiej oet der tied,.}{va Don Quichotte d\`e gong}{och mit ezu get oet}\\

\haiku{De vrouw Groonesjild komp '.}{aaterm. Ze sjtonge}{en wronge de heng}\\

\haiku{'t Is sjtil in.}{der operatioenszaal en}{ginge zet e w\`oad}\\

\haiku{hedelfienger en.}{loene en kling welsjkere64}{mit de haffele65}\\

\haiku{ich zie\"en 't durp, ',...}{en der t\`oan van de kirkt}{loewt de mis is oet}\\

\haiku{E sjtong al ins,,,.}{op en gong op krukke twei}{drei meter nit mie\"e}\\

\haiku{Wat is 't sjtil, '.}{iggen durp nog sjtiller}{wiej int sjpitaal}\\

\haiku{Ezu gonge de daag.}{verbei en der Macheel blef}{zitte in d\`e sjtool}\\

\haiku{Der Guillijom mus.}{karoe\"ete hakke en}{hui g\`oa kie\"ere}\\

\haiku{De kleier zunt nat}{bis oppen hoet en ze ruuke}{wiej der sjtank va}\\

\haiku{Ozze Herregod '.}{hat ging vruid ant kapotsjl\`oa}{va velder en weie}\\

\haiku{all\`o 't is nog ', '... '.}{nit watt zie\"e mot m\`et}{geet v\"orne sjoester}\\

\haiku{Oppen hoof waor ' '.}{t klaor wie uvver daag en}{roe\"ed wiejn hil98}\\

\haiku{Dat hof mer inge, '.}{te zie\"e da wete zet}{sjnak allenui}\\

\haiku{E beft en razelt, ' ' '...}{en d\`o kumptn wil gloed en}{n vlam inm op}\\

\haiku{Went d\`er Goddes Zoon,,.}{zut dan zat dat dis sjting}{broe\"ed w\`e\"ede}\\

\haiku{Der wink sjpelt'n um ',.}{der kop en int hemp dat}{los hingt op de bros}\\

\haiku{De jonge sjtunt 'n '!}{owweblik en kiekke zich}{aat Is ged\`oa}\\

\haiku{E let ze oppen.}{linker hand en sjmie\"et118}{ze dik mit botter}\\

\haiku{M\`e van noe aa wos}{der Macheel datte-n-'t}{Thriske zel\`eve nit}\\

\haiku{- Went 't get wuur uvver ', '.}{t veld uvvert k\`oon en der}{terf en de havver}\\

\haiku{Dat is 'n wet, diej.}{ezu aod is wiej de welt en}{ummer blieve zal}\\

\haiku{De madam en 't '. '.}{Truudje sjtunt int d\"a\"oresjpan}{t Heisjt opgepaast}\\

\haiku{De Hakkeret is '.}{mit vant sjunste erf van de}{ganse sjtrie\"ek}\\

\haiku{Der Macheel zoot dao:}{mit ene glans iggen owwe}{en der Groonesjild zaat}\\

\haiku{Doe veele i sjlaop i ' '.}{zie geluk en int good}{vant mondelit}\\

\haiku{Der sjlimste vaggen.}{durp AGGENE beuisj wont der}{Mertens mit zieng vrouw}\\

\haiku{Pastoer hat mit der.}{Mertens nog mie\"e leed wiej mit}{gans Lutterendal}\\

\haiku{{\textquoteright} Noe zoot der Mertens '.}{iggene kaffie\"e beit}{Macheelkes Truudje}\\

\haiku{E minnigmaol}{is dat w\`e\`er oet gen dil}{van onder op lants}\\

\haiku{Aate de weie, woe 't,.}{veld aavingt huu\"ete get wat klaagt}{en kriesjt wiej e kink}\\

\haiku{{\textquoteright} Dao zitteter aate,.}{iggen kirk nog mie\"e die nit}{kommeneseere}\\

\haiku{Dao kleft blood an 't?}{sjabeleer en an de hand.}{Och dat nit woe\"er}\\

\haiku{Doe dees dieng hand op.}{en vervuls alle l\`eve}{mit dienge z\`ege}\\

\haiku{Noe dunt ze zich de '.}{ring aa en dan ist good}{en zunt ze contint}\\

\haiku{Noe wiej 't ged\`oa,,.}{is mingt der kuster e wuur}{nit mie\"e iggen kirk}\\

\haiku{Op 'n oethaot,.}{kun d\`er wir k\`oame went der}{sjnie\"e gesjmolten is}\\

\haiku{{\textquoteleft}mer landsem, v\`er hent,{\textquoteright}.}{noe tied doe kries huuj havver}{ezu v\"a\"ol wieste wils}\\

\haiku{De luuj gleuven nog ',.}{nit datt oethaat is bis}{dat de liester zingt}\\

\haiku{{\textquoteright} Bis noe waor 't.}{es went ze te hoop mer ing}{zie\"el heie gehad}\\

\haiku{Alles wat groe\"et,...!...}{en hoe\"eg is is mit angs}{verbonge ene berg}\\

\haiku{In de wei sjtunt ze.}{te kiekke mit groe\"ete}{glazere owwe}\\

\haiku{E wilt ene boer van ',....}{m make der litste en}{der grutste Groonesjild}\\

\haiku{Da zalle zage,:}{wente v\`e\"edig is mit}{diej va Ie\"epe}\\

\haiku{Herregod iggen,{\textquoteright}.}{himmel laot mich sjtil}{en geduldig zie\"e}\\

\haiku{Pastoe\"er ka gee...}{wo\`ad mie\"e oetbringe en dao}{in ins ene sjlaag}\\

\haiku{Dao is 'n welt, diej '.}{de sjtad nit kint en dat is}{diej vant jonk vie\"e}\\

\haiku{E paar minute,.}{dan is och de krao mit ene}{sjrei e-weg}\\

\haiku{E wit och dat dat, ' '.}{e bewies is dattn}{interessie\"et}\\

\haiku{{\textquoteright} Der Mertens hei de.}{vreier ging twientig jaor}{laote ko\`ame}\\

\haiku{Noe geet 't op en '.}{aaf uvvern heen en probeere}{ze get te versjto\`a}\\

\haiku{De ouwere hent.}{de beuk al onder gen erm}{en zunt e-weg}\\

\haiku{V\"a\"or mees te lane}{hoft me ging sjl\`e\`eg toe te}{do\`a en da hat me}\\

\haiku{Diej zonge \`eve hel ' '.}{naot oksaal truk en ezu}{gongt hin en weer}\\

\haiku{In der aavank ersjaffet.}{God himmel en \`e\`ed en e}{maket alles good}\\

\haiku{zate... en e sjprook:}{uvver tied en ivvigheed en}{uvver de zung en zaat}\\

\haiku{Zoste dao kunne,,?}{pr\`edige Barthelomee in}{de Sint Servaos}\\

\haiku{{\textquoteleft}al kost 't os twei,{\textquoteright}.}{doezend gulde v\`er zulle}{dat offer bringe}\\

\haiku{{\textquoteright} H\`e\"er Bussjep,, '!}{went d\`er ins wust wat inn}{zie\"el kan umgo\`a}\\

\haiku{Noe is 't 'n koo,.}{diej miskoft dan e p\`e\"ed}{dat vervangen is}\\

\haiku{Ich zal 't uch wal,....}{laote weete wienie\"e}{dat d\`er kaome mot}\\

\haiku{Bei 't Mieke komt....}{ene r\"a\"o\"ek de deur oet ene}{ouwe-luujsr\"a\"o\"ek}\\

\haiku{En dan {\textquoteleft}voes{\textquoteright}, {\textquoteleft}broeng{\textquoteright} en...,.}{dan der Mertens ene duuvel}{andesj geet it nit}\\

\haiku{Namens de ganse,...}{pfaar danke v\`er uch bezondesj}{v\"a\"or die twei woe\"e zunt}\\

\section{Annie M.G. Schmidt}

\subsection{Uit: In Holland staat mijn huis}

\haiku{er zijn er bij die.}{glooien en er zijn er bij}{die zelfs rechtop staan}\\

\haiku{Maar wanneer men in,.}{Nederland wil blijven wordt}{de keus beperkter}\\

\haiku{Goeie help, dacht ik, het?}{zal toch niet een gothische}{cathedraal worden}\\

\haiku{Toen de architect,,:}{mij vroeg hoe ik de keuken}{wou hebben zei ik}\\

\haiku{Pakje wasmiddel,!!}{in kastje dat er uit schiet}{en weer terugschiet}\\

\haiku{Zo'n oprechte spijt,.}{dat we het maar niet liever}{zelf hebben gedaan}\\

\haiku{Maar dat is altijd, '.}{zo zeiden de mensen die}{t weten kunnen}\\

\haiku{Men is oud voor het,,,,.}{klaar is oud grijs afgeleefd}{moe en afgeknapt}\\

\haiku{Ik betrapte me,:}{er op dat ik hele uren}{achtereen uitriep}\\

\haiku{Ik boog me uit het.}{raam en daar in de diepte}{stond een kruidenier}\\

\haiku{Het waren Linda.}{en Klaas en natuurlijk moesten}{ze eerst alles zien}\\

\haiku{Welnee, ik zal er.}{nog wat kranten opleggen}{en nog wat houtjes}\\

\haiku{Ik weet niet of u ',...}{t weet maar er zit iets aan}{u vast van achter}\\

\haiku{En of wij ons niet:}{veel beter een boek hadden}{kunnen aanschaffen}\\

\haiku{Het spijt me, meneer,.}{ik heb me bedacht en ik}{doe het toch maar niet}\\

\haiku{Hij kronkelt zich nu.}{op onze piano en}{ik laat het maar zo}\\

\haiku{Het is nutteloos.}{om zich tegen het noodlot}{te verzetten}\\

\haiku{Het is zelfs zo, dat;}{het begrip huis verweven}{is met haar erotiek}\\

\haiku{Maar gek, juist voor die.}{vrouw is er in het hele}{huis geen eigen plaats}\\

\haiku{En toch zit ik hier,.}{nu al van half tien tot half}{een met m'n bril op}\\

\haiku{Hij maakte een stuk.}{lood vast aan een touw en liet}{het langs de deur neer}\\

\haiku{Zie de maan schijnt door,....}{de bomen makkes met u}{wild gestaak zingt hij}\\

\haiku{Misschien, en daar komt,.}{het moederhart weer misschien}{wordt hij iets heel hoogs}\\

\haiku{Precies zoals laatst,:}{toen hij naast me dribbelde}{en ik aldoor riep}\\

\haiku{Hij ligt zo vreemd te.}{hijgen en verder is hij}{zo stil en zo slap}\\

\haiku{Maar ik zie al dat,.}{dit onzin is want hij is}{niet in het minst ziek}\\

\haiku{Mensenmeisjes zijn,!}{tegenwoordig veel vlotter}{als je dat maar weet}\\

\haiku{Natuurlijk wil ik....}{in dat comit\'e zitten}{zei ik opgewekt}\\

\haiku{We kwamen  in.}{de tuin en zagen de kat}{met iets geels spelen}\\

\haiku{Toen zag ik dat een.}{van de koeien een stuk touw}{om zijn horens had}\\

\haiku{Eentje met een snor.}{en een dikke buik met een}{rode band er over}\\

\haiku{Ze keek ons kwijnend.}{aan en trok wellustig haar}{nagels in en uit}\\

\haiku{Je doet maar, zegt haar.}{man en gaat een beetje over}{de schutting lopen}\\

\haiku{Maar hij voelde dat,.}{hij iets verkeerds gezegd had}{iets dat men niet zegt}\\

\haiku{De zuilen waren,.}{geschilderd op een doek maar}{ze leken heel echt}\\

\haiku{Nou, wat zei ik, o,.}{ja dat hele gezin is}{een beetje overstuur}\\

\haiku{Pas op, maar ze heeft.}{zo'n dorst en slaat het achter}{elkaar naar binnen}\\

\haiku{en toen stond ik te,...}{strijken aan zo'n strijkplank maar}{het was geen strijkplank}\\

\haiku{Een stofzuiger, een,,:}{hele fraaie die alles kan}{behalve kraaien}\\

\haiku{Zou je de trap en,.}{het portaal willen doen vroeg}{ik onderdanig}\\

\haiku{Zou ik misschien   '?}{zelf een beetje achterlijk}{aant worden zijn}\\

\haiku{Maar ze is nooit meer.}{terug gekomen om het}{me te vertellen}\\

\haiku{Denk er aan, mevrouw,.}{ik heb niet graag dat u mijn}{kleren stiekem draagt}\\

\haiku{Hij zegt nou is het,,.}{uit zegt ie en hij gaat me}{vermoorden zegt ie}\\

\section{Jacques Schreurs}

\subsection{Uit: Het godsbewijs van dokter Chantrain}

\haiku{Een gril, beweren;}{de nuchtere geesten met}{tante Simone}\\

\haiku{Er is niets zo erg.}{als wanneer die aangetast}{en bedorven zijn}\\

\haiku{Niemand ter wereld,;}{is dan ook zo hol zo leeg}{als de leugenaar}\\

\haiku{Soms schijnt het huis even,.}{te wankelen slagzij te}{maken als een schip}\\

\haiku{dat er een grote,;}{zeer grote verscheidenheid}{was in de eenheid}\\

\haiku{Dat er dan toch nog,.}{iemand was die haar doorschouwd}{had had haar ontsteld}\\

\haiku{zich in gezelschap:}{van  heren gedragen}{dan onder elkaar}\\

\haiku{maar haar noodlot is.}{dat zij er slechts zelden haar}{nut mee weet te doen}\\

\haiku{En er is niemand,.}{om mij te helpen indien}{nog hulp zou baten}\\

\haiku{De vingertoppen....}{van haar man worden al groen}{onder de nagels}\\

\haiku{een verre, vage,.}{en schielijk voorbijgaande}{verschietende God}\\

\haiku{Zij was in vele,.}{dingen onderwezen maar}{in de godsdienst niet}\\

\haiku{Het werken was heel,;}{zijn leven zijn gezondheid}{geweest zijn enige}\\

\haiku{maar alles ontzinkt....}{haar en haar handen tasten}{in het ledige}\\

\haiku{Wij allen zijn in:}{de hand des Heren en het}{is alleen de vraag}\\

\haiku{Het boek, dat hij dan,:}{altijd opnieuw weer openslaat}{spreekt hem niet meer aan}\\

\haiku{Goed, maar wie zegt u,,}{dat de God Die gij aanroept}{niet even misselijk}\\

\haiku{Hij ziet niet altijd.}{wat hij ziet en hij hoort niet}{altijd wat hij hoort}\\

\haiku{Hij is nog slechts aan.}{het begin van zijn lange}{en moeilijke weg}\\

\haiku{E\'en voor \'e\'en kan hij,;}{ze nog voor zijn geest halen}{deze kinderen}\\

\haiku{Het klooster is arm.}{en de vergoeding voor de}{inwoner gering}\\

\haiku{de geschiedenis.}{is daar om er overvloedig}{van te getuigen}\\

\haiku{Hij is opgestaan.}{van zijn bidstoel waarop hij}{neergeworpen lag}\\

\haiku{Item zijn sombrero.}{en hij plaatst hem op het hoofd}{en is reisvaardig}\\

\haiku{de priester in een.}{inrichting voor zielszieken}{onder te brengen}\\

\haiku{Het tweede geval.}{in zeer korte tijd dat me}{onder de ogen komt}\\

\haiku{Een vriend aan een vriend,.}{was de priester hem in de}{rede gevallen}\\

\haiku{Meer dan gebeden;}{zagen wij hem storten over}{zijn gestorven vriend}\\

\haiku{Andere woorden.}{willen niet levend worden}{op hare lippen}\\

\haiku{De schaduw die over;}{zijn gelaat gelegen heeft}{is opgetrokken}\\

\subsection{Uit: Kleine vertellingen}

\haiku{En ook deze zat.}{op zijn beurt over hem met de}{handen in het haar}\\

\haiku{om de schouder lei.}{en zij neuri\"end samen}{hun weg vervolgden}\\

\haiku{Maar ook aan deze.}{geluksstaat kwam helaas een}{vroegtijdig einde}\\

\haiku{Hij bond zijn paard aan.}{een boom en zette zich op}{een steen langs de weg}\\

\haiku{hoe potsierlijk me,,.}{de pluimen die je me op}{de muts steekt tooien}\\

\haiku{En daar zouden zij,,;}{nu nog staan vermoed ik als}{ik niet beter wist}\\

\haiku{{\textquoteleft}maar een roeping, een,,!}{hogere zending dat is}{wat daar hoort wat toe}\\

\haiku{En toen hij alle,:}{ongemakken uit de weg}{had geruimd zei Axel}\\

\haiku{De dame nam Axel.}{weer van de groentevrouw over}{en ging naar haar huis}\\

\haiku{dat wij het, als wij,!}{z\'o voort blijven gaan v\'e\'el te}{ver zullen brengen}\\

\haiku{de geleerde, de,.}{supergeleerde zeker}{de uitzondering}\\

\haiku{schudden de buren;}{met de kop en maakten dat}{ze binnen kwamen}\\

\subsection{Uit: Kroniek eener parochie. Deel 1. De kraai op den kruisbalk}

\haiku{zij vertrouwen hem.}{niet en durven hem nochtans}{niet te wantrouwen}\\

\haiku{Geen dank, eerwaarde,.}{want wij zijn al blij genoeg}{het te m\'ogen doen}\\

\haiku{Een paar grootere:}{boeren met paarden en een}{stal rundvee dachten}\\

\haiku{hoorde ik een stem.}{mij duidelijk vragen en}{ik schrok van haar klank}\\

\haiku{- Ik wist toch wel dat.}{u van die uitnoodiging geen}{gebruik zou maken}\\

\haiku{ik gun het haar, en.}{daar alles mee gezegd wat}{men maar zeggen kan}\\

\haiku{In het priesterkoor;}{brandt de Godslamp voor het}{Allerheiligste}\\

\haiku{De pastoor heeft er.}{op gewacht en zich nog niet}{te bed begeven}\\

\haiku{omdat zij zich te;}{deftig voelde om naast vrouw}{Bonte te zitten}\\

\haiku{Zij staken den kop.}{in de veeren en deden}{alsof zij sliepen}\\

\haiku{Toen ik hem vragend,.}{aanzag vertelde hij mij}{dat hij trouwen ging}\\

\haiku{Ook in den omgang,}{met den jongen was hij schaars}{in woorden die \'als}\\

\haiku{en de jongen is.}{er toch al erg genoeg aan}{toe zonder moeder}\\

\haiku{die dan dikwijls een.}{zekerder uitsluitsel gaf}{dan de gasten zelf}\\

\haiku{Deze wildernis.}{zou in een lusthof kunnen}{herschapen worden}\\

\haiku{hartstochten, deugden,.}{en ondeugden waardoor zij}{gedreven werden}\\

\haiku{Een winterkoning,,.}{niet grooter dan een duim had}{er zijn holletje}\\

\haiku{doch evengoed had ik,.}{hooi kunnen nemen dat nog}{goedkooper is}\\

\haiku{overwonnen door den;}{slaap rusten haar hartstochten}{en vijandschappen}\\

\haiku{aanvankelijk met.}{die kalme tevredenheid}{die haar eigen is}\\

\haiku{Doch het is Miete,!}{alleen niet waarmee wij te}{maken hebben vrouw}\\

\haiku{en ik vertrek op,.}{staanden voet vervolgde hij}{stampend met den voet}\\

\haiku{riep vrouw Briels, als ik;}{eraan denk wat wij  met}{jou gehad hebben}\\

\haiku{doch de freule werd.}{met den dag dan ook meer en}{meer onhandelbaar}\\

\haiku{Doch de schande hing.}{dreigend boven zijn bloed en}{zijn gansche geslacht}\\

\haiku{een wild gericht hangt,.}{boven de huizen dat dan}{plotseling afbreekt}\\

\haiku{kinderen leunen.}{lachend en roepend uit de}{vensters der gevels}\\

\haiku{Kort na den dood van.}{vrouw Van der Schoor al was het}{verkeerd gaan loopen}\\

\haiku{Zoozeer was Bertus niet,.}{het Panhuis vergroeid dat aan}{hem niet geroerd werd}\\

\haiku{alleen de gekke.}{Bert droeg zijn borst wellicht n\'og}{hooger dan anders}\\

\haiku{Doch de kinderen.}{hadden alle aandacht voor}{het vierde gebod}\\

\haiku{En wat is er nu,,.}{van uw dienst eerwaarde biecht}{u nu maar eens op}\\

\haiku{- Toch zult u er niet.}{in slagen mij bij u mijn}{biecht te doen spreken}\\

\haiku{Als ik hem zeg, dat,.}{hij een vernufteling is}{lacht hij grandioos}\\

\haiku{Het contact met zijn.}{parochie heeft hij echter}{nog niet gevonden}\\

\haiku{Paulus Lumens lag.}{zeker reeds in een diepen}{en tevreden slaap}\\

\haiku{Als ik morgen door,;}{het dorp ga zijn de kleine}{huizen gesloten}\\

\haiku{dat klein goed vraagt meer}{dan ik hen geven kan en}{honger en gebrek}\\

\haiku{ik zal er danig.}{over waken moeten en ook}{over de anderen}\\

\haiku{Hij moest het zelf nu;}{maar weten of hij er werk}{van maakte of niet}\\

\haiku{doch als zijn vader,.}{hem ook in den steek liet zijn}{moeder deed het niet}\\

\haiku{Een man loopt met een;}{brandmerk en weet er zich niet}{van te bevrijden}\\

\haiku{Zij heeft een pleister,.}{gelegd op zijn wang die niet}{ophoudt te bloeden}\\

\haiku{De menschen hebben.}{een goed hart en de ruwen}{soms nog het beste}\\

\haiku{Wat moet een mensch op?}{zijn voorbijgang nog meer dan}{groote verwachtingen}\\

\haiku{en de wateren.}{der beproeving komen hem}{tot de lippen}\\

\haiku{ik geloof niet dat.}{mij ooit iemand op mijn plicht}{heeft moeten wijzen}\\

\haiku{brouwer, je gaat er,.}{kapot aan maar ik zal je}{helpen als je wil}\\

\haiku{Louis zijn deel ging bij.}{Van der Schoor in het bedrijf}{met nog wat erbij}\\

\haiku{Het was allemaal,.}{te verbijsterend voor twee}{kleine grijze oogen}\\

\haiku{Laat ons met rozen,.}{kronen voor ons hoofd zich diep}{naar het aschkruis buigt}\\

\haiku{ook in zijn jonge,;}{jaren niet als hij wel eens}{een glas teveel had}\\

\haiku{Hij herinnerde,.}{zich een anderen nacht niet}{lang geleden nog}\\

\haiku{het is de liefde,!}{die alles bederft tot haar}{eigen vruchten toe}\\

\haiku{Niets! - Denk er wel aan,,...}{Lumens ik laat je vandaag}{niet los vooraleer}\\

\haiku{Ook om een waarheid,.}{te verduisteren die \`al}{te duidelijk is}\\

\haiku{- Je zult me moeten,:}{toegeven dat je beeldspraak}{bepaald verward is}\\

\haiku{Hoe weinig is er,!}{noodig om een koning hoeveel}{om een mensch te zijn}\\

\haiku{de figuur van de.}{Mater Dolorosa raakt}{nergens uit ons oog}\\

\haiku{Ik wacht op vader,.}{antwoordde de jongen en}{slenterde verder}\\

\haiku{Ieder zijn eigen;}{droom is hem weer verschenen}{in nieuwen luister}\\

\haiku{zijn morgenbrood in.}{de hand vermorzeld en aan}{de paarden gevoerd}\\

\haiku{Enkele drinkers.}{vegen van verbazing met}{hun mouw langs hun mond}\\

\haiku{Marie-Cathrien.}{heeft het geld geteld en weer}{in de tasch gedaan}\\

\haiku{Mijnheer Bongaerts zingt,;}{nog immer in de kerk maar}{lang zoo hard niet meer}\\

\haiku{Ook wanneer zij zich:}{niet inspant ziet men een hoogen}{blos op haar wangen}\\

\haiku{Van een weduwman.}{met kinderen mag men niet}{alles verwachten}\\

\haiku{den meester is veel.}{eer bewezen en zijn vrouw}{heeft daarin gedeeld}\\

\haiku{Aan de sc\`enes, die,.}{zijn moeder gemaakt had heeft}{hij zich niet gestoord}\\

\haiku{Anderen vinden,;}{het heel natuurlijk dat een}{spin in haar web zit}\\

\haiku{Maar wat je gehad,.}{hebt dat heb je en nemen}{ze je niet meer af}\\

\haiku{Daar mag geen geweld,.}{aan gedaan worden evenmin}{als aan zijn geloof}\\

\haiku{Wij zijn, allemaal,,.}{zondaars eerwaarde maar het}{ligt er maar aan hoe}\\

\haiku{Het doet mij deugd u,.}{dit alles eens te mogen}{zeggen eerwaarde}\\

\haiku{uit ellendigheid.}{en omdat ook ik niet recht}{in mijn schoenen sta}\\

\haiku{En Johannes Den,,?}{Hertog dan tante Dora}{wat heeft hem bezield}\\

\haiku{houdt dien Hollander,,.}{in de gaten Bert want die}{bederft je de pap}\\

\haiku{Dit echter was voor.}{mijnheer Erik Odekerke de}{groote verrassing niet}\\

\haiku{de bisschop heeft ze;}{als geldend aanvaard en dat}{is genoeg voor mij}\\

\haiku{Dit is mijn eenige,.}{rechtvaardiging voor Hem die}{mij oordeelen zal}\\

\haiku{Toen de pastoor zijn.}{bezoekers uitliet stond de}{hemel vol sterren}\\

\haiku{Godsgevangen en,.}{duivelsgeplaagd als bij ons}{de menschen zeggen}\\

\haiku{Het spijt me wel, maar.}{voor u valt hier voorshands nog}{niets te verdienen}\\

\haiku{Doch die hangt af van.}{den wil en gij weet hoe het}{daarmee gesteld is}\\

\haiku{Na den maaltijd nam.}{baron Isidoor den kapelaan}{mee naar het salon}\\

\haiku{De personen die:}{er een rol in spelen zijn}{achtereenvolgens}\\

\haiku{veeren, bladeren.}{en bloemen die opstuiven}{in de platanen}\\

\haiku{Ik heb veel voor jou,!}{gebeden en ook voor u}{mijnheer Den Hertog}\\

\subsection{Uit: Kroniek eener parochie. Deel 2. De mensch en zijn schaduw}

\haiku{Eerst de keuken dan,,?}{maar want van de maag moet je}{het toch hebben niet}\\

\haiku{Niets dan miserie,?}{had je van je jongens te}{verwachten of niet}\\

\haiku{En als de korsten.}{hen steken moet de vader}{het maar ontgelden}\\

\haiku{Nicolaas, je moet;}{alles nu ook niet erger}{maken dan het is}\\

\haiku{heeft mijnheer Lumens,?}{gestameld met wanhoop in}{de oogen wat nu Erik}\\

\haiku{Gelukkig maar dat,!}{wij alles van te voren}{niet weten pastoor}\\

\haiku{Laat het dan ook voor,,!}{jou een probleem zijn maar help}{het oplossen Erik}\\

\haiku{Harten waren voor,.}{hem opengegaan oogen hadden}{hem toegeblonken}\\

\haiku{Doch als de heele,?}{keuken voor mijn oogen begint}{te draaien pastoor}\\

\haiku{Die zou een beetje,;}{aardig en stijlvol moeten}{zijn en gezellig}\\

\haiku{De pret was soms zoo}{groot dat hun vader zich een}{paar maal genoodzaakt}\\

\haiku{Stommerik, viel zijn,!}{vader tegen hem uit doe}{dat ding van je kop}\\

\haiku{was het antwoord en.}{ten einde raad trok zij er}{mee naar de keuken}\\

\haiku{als er over ieder,.}{stroospier gestruikeld moet worden}{wordt liet huis een hel}\\

\haiku{Toen deze binnen.}{het bidden hoorde bleef hij}{even staan luisteren}\\

\haiku{Moe is het woord niet,,.}{eerwaarde maar ik voel me}{diep ongelukkig}\\

\haiku{Moeder, zei hij voor,}{hij de deur uitging ik hoop}{dat de zegen dien}\\

\haiku{wel moest hij haar tot,,}{haar eer nageven dat zij}{hoe ziek zij reeds was}\\

\haiku{Je hebt me de pap,;}{in den mond gegeven zei}{mijnheer Odekerke}\\

\haiku{doch wanneer hij op.}{zwart zaad zat was er geen huis}{met hem te houden}\\

\haiku{Maar die heeft dan ook:}{m\'e\'er dan Lambert en Peter}{zijn uiterlijk mee}\\

\haiku{Wien het juk dat ze}{om eigen bestwil op zich}{genomen hebben}\\

\haiku{Wil u nu nog meer,.}{van me weten dan moet u}{maar spreken Heeroom}\\

\haiku{bij Weenink over den,;}{vloer geweest omdat ik hen}{niet wou passeeren}\\

\haiku{Daar is niemand die.}{beter weet hoe hij me klein}{moet krijgen dan jij}\\

\haiku{Met dat bescheid had.}{hij haar de voordeur voor de}{neus dichtgesmeten}\\

\haiku{Maar toen hij vroeg wat,.}{hij voor ons doen kon wist ik}{het niet te zeggen}\\

\haiku{Nu wel, nu heb je}{me veel te verbergen en}{nu belieg je me.}\\

\haiku{Laat die hunnen draai,.}{maar eens krijgen dan wordt de}{wereld nog te klein}\\

\haiku{Er is reeds gezegd.}{dat er genoeg geblaft wordt}{onder de dieven}\\

\haiku{Dan moet je ze maar...!}{eens in de herbergen zien}{en de cinema}\\

\haiku{Die komt het verste in,.}{de wereld voegde zij er}{uit zichzelf aan toe}\\

\haiku{De vrijgestelden,!}{schijnen het bij jou verbruid}{te hebben jongen}\\

\haiku{Custers knikt dat hij,.}{het gehoord heeft en ook de}{anderen knikken}\\

\haiku{was hij kort van draad,!}{maar d\`at had hij haar toch wel}{even kunnen zeggen}\\

\haiku{Wil je soms zeggen?}{dat ik niet hard genoeg voor}{jullie gezwoegd heb}\\

\haiku{Louis was er al vroeg,.}{bij daarna Dorus en daarna}{Lambert en Peter}\\

\haiku{wat eigenlijk het.}{middelpunt van haar wereld}{en verlangens is}\\

\haiku{Met antipathie,,.}{zooals mij verweten wordt heeft}{dat niets te maken}\\

\haiku{Het had allemaal,.}{anders kunnen loopen en}{ook beter misschien}\\

\haiku{Het is altijd de,.}{l\'a\'atste stap die Nicolaas}{Bonte moeite kost}\\

\haiku{Als hij in het dorp.}{komt zijn de verkiezingen}{reeds in vollen gang}\\

\haiku{Thuis reikt zijn vrouw hem,.}{een brief van Karel juist met}{de post gekomen}\\

\haiku{Voor het overige.}{gaat hij zich in woorden n\`och}{in drank te buiten}\\

\haiku{Zijn vader zet de,!}{bloemetjes eens buiten dat}{mocht wel voor een keer}\\

\haiku{Het spijt me, moeder,,!}{dat ik het zeggen moet u}{maakt u zelf iets wijs}\\

\haiku{Dat is niet mooi van,,!}{u gezegd moeder dat is}{zelfs leelijk en grof}\\

\haiku{Het zou de eerste!}{keer niet zijn dat jij je over}{je ouders schaamde}\\

\haiku{Heeft u me zelf dan,?}{niet gezegd dat het ver met}{ons gekomen is}\\

\haiku{Ik wil niets, ik geef.}{je den raad te doen wat je}{moeder van je vraagt}\\

\haiku{In ieder geval.}{moogt u weten dat ik uw}{raad niet zal volgen}\\

\haiku{evengoed als je te,.}{preeken mis te lezen err}{biecht te hooren hebt}\\

\haiku{Je mag gerust een.}{boon zijn als je me nu eerst}{maar eens praten laat}\\

\haiku{Hoemeer jij pastoor,;}{tegen me speelt hoemeer hij}{me te pakken neemt}\\

\haiku{Eerlijk, dat kun jij.}{voor God en je geweten}{niet verantwoorden}\\

\haiku{Een mensch kan er oud,.}{en gebrekkig in worden}{tot den dood vermoeid}\\

\haiku{Zij komt van bij Dorus.}{met den halsdoek aan den mond}{tegen den wind in}\\

\haiku{Hij is een echte,,.}{broer van Louis zegt zijn moeder}{soms maar hij is trouw}\\

\haiku{Jacob heeft zeker.}{de bel niet horen overgaan}{toen zij binnenkwam}\\

\haiku{Hij wil weten wat.}{er gebeurd is en vraagt het}{hen zonder verlet}\\

\haiku{Nicolaas Bonte,.}{maakt korte metten dat is}{men van hem gewoon}\\

\haiku{Zij zeggen dat zij.}{op het ziekenhuis ligt en}{dat zal dan wel zijn}\\

\haiku{Van mijn kant heb ik.}{niets zoozeer verlangd dan u die}{rust te vergallen}\\

\haiku{Indien zij er ligt.}{zal zij wel nergens beter}{kunnen zijn dan daar}\\

\haiku{Telkens blijft deze;}{ondanks zijn voornemens in}{zijn aanloop steken}\\

\haiku{En dat juist omdat.}{de priester de dienaar van}{den Gekruiste is}\\

\haiku{en moeten wij zelf?}{niet door ervaring wijzer}{worden over onszelf}\\

\haiku{En weten alle!}{menschen wat u zich in den}{kop heeft  gehaald}\\

\haiku{Als dat zoo met je}{door blijft gaan spring je nog eens}{de vuilniskar op}\\

\haiku{Nicolaas Bonte.}{is een van diegenen die}{nooit capituleeren}\\

\haiku{Er zijn goede en,.}{kwade engelen goede}{en kwade driften}\\

\haiku{tobben was het en.}{krom liggen om de touwtjes}{aaneen te krijgen}\\

\haiku{Het is de tweede,}{keer in een maand tijds nu al}{dat me dit overkomt}\\

\haiku{Je moest wijzer zijn,,.}{Bonte na alles wat je}{achter den rug hebt}\\

\haiku{m\`ere Canisia.}{had hem reeds gezegd wat er}{van hem verwacht werd}\\

\haiku{XXIV Sedert mijnheer}{Lumens den dood in de oogen}{van zijn kapelaan}\\

\haiku{heeft pastoor Lumens.}{geantwoord met een geloof}{dat bergen verzet}\\

\haiku{Zwart tegen zwart, heeft,,!}{Coenraad gezegd vuil tegen}{vuil wij zullen zien}\\

\haiku{een parochie kan.}{niet meer gestraft zijn dan door}{het gemis daaraan}\\

\haiku{Als de klok het uur;}{slaat knielt de parochie voor}{het tabernakel}\\

\subsection{Uit: Kroniek eener parochie. Deel 3. De weg zijner zonen}

\haiku{Allen zijn eenzaam;}{die het geloof aan zichzelf}{verloren hebben}\\

\haiku{Heb je nog iets van,?}{je vader gehoord waar die}{zich ophoudt of zoo}\\

\haiku{Reden te meer voor.}{de menschen om elkander}{er door te helpen}\\

\haiku{Zulke menschen zijn.}{er helaas en ze zullen}{er altijd blijven}\\

\haiku{Het is genoeg om;}{het hart van mijnheer Lumens}{al warm te maken}\\

\haiku{hij legt de hand op.}{het hoofd van het kind en treedt}{de keuken binnen}\\

\haiku{Op \'e\'en kostganger!}{meer of minder kwam het bij}{Van der Schoor niet aan}\\

\haiku{Spreek er me niet van,;}{zegt hij wanneer iemand hem}{met raad wil helpen}\\

\haiku{onder de snikken.}{der oude Geertrui was zij}{de deur uitgegaan}\\

\haiku{gedane zaken.}{nemen geen keer en berouw}{komt na de zonde}\\

\haiku{Hij had alleen maar.}{niet geweten waar hij de}{kracht vandaan haalde}\\

\haiku{Severinus van.}{der Schoor constateert het met}{spijt en achterdocht}\\

\haiku{je doet afstand of,,.}{niet je hebt den moed of je}{hebt hem niet zegt hij}\\

\haiku{zei Peter Tobben.}{opeens en liet verwonderd}{den hamer zakken}\\

\haiku{Sedert je ziek bent.}{kan men ze den evenveel van}{het lijf aflezen}\\

\haiku{Om den dooien dood,.}{niet maar je moet het er dan}{ook niet naar maken}\\

\haiku{Een brievengaarder.}{zonder duim is ook maar een}{steel zonder lepel}\\

\haiku{en zijn oogen gaan op,.}{en af nu eens wijd open en}{dan weer langzaam toe}\\

\haiku{Reinout Eussen was een:}{eind in zijn richting de hei}{op gegaan en had}\\

\haiku{het past niet daarover,.}{te oordeelen zeker den}{buitenstaander niet}\\

\haiku{Den meesten tijd weet.}{je van voren niet dat je}{van achteren leeft}\\

\haiku{Het kan Dorus Bonte,.}{bliksems veel schelen dat het}{met Nico misloopt}\\

\haiku{De lange Peter,}{staat voor niets vooral niet als}{het er op aankomt}\\

\haiku{Dat is meer dan u,!}{van een redelijk schepsel}{vragen kunt baron}\\

\haiku{dat ze nog niet te!}{genieten zou zijn als ze}{op brandewijn stond}\\

\haiku{hij woont zijn voor zijn,.}{invallen te dom voor zijn}{grappen te nuchter}\\

\haiku{Daarom verlangt de.}{brievengaarder naar niets zoo}{zeer als naar zijn dienst}\\

\haiku{Neen, van de vrouwen!}{moet men Peter Tobben niets}{nieuws meer vertellen}\\

\haiku{en wie er kwaad over}{spreken zijn de ergsten als}{het er op aankomt}\\

\haiku{Buiten staat dokter.}{Liebaert zijn motor aan te}{trappen die weigert}\\

\haiku{de grootste misschien.}{die de Mijn tot nog toe op}{haar geweten heeft}\\

\haiku{Nico weet nog van,.}{niets die is met een wagen}{naar het buitenland}\\

\haiku{Wat een mensch tusschen!}{opstaan en slapengaan al}{niet overkomen kan}\\

\haiku{Peter Tobben heeft!}{er niet voor niets de winkels}{voor afgeloopen}\\

\haiku{Lambert gaat met hen.}{naar buiten maar niet verder}{dan tot het hekje}\\

\haiku{De navraag die men;}{naar hem gedaan had was op}{niets uitgeloopen}\\

\haiku{Zelf kan hij zich niet,.}{herinneren dat hij er}{ooit een gestort heeft}\\

\haiku{Misschien ga ik naar,,;}{Amerika zei Lambert droog}{en misschien ook niet}\\

\haiku{Even daarna luwde}{het getij en toen in een}{stilte vol snikken}\\

\haiku{de dunne vingers,}{had hij het voornamelijk}{tegen een man dien}\\

\haiku{riep Klabbers spontaan.}{met een slag van de platte}{handen op tafel}\\

\haiku{Hals over kop liet hij}{de vertering bij zijn glas}{achter en alsof}\\

\haiku{Over het hoe, maakte;}{hij zich voorloopig de}{grootste zorgen niet}\\

\haiku{Of als Nico maar!}{eens eerder geweten}{had wat hij nu wist}\\

\haiku{Het lanterfanten.}{was tenminste gedaan en}{daar was hij blij om}\\

\haiku{zij was maar blij, dat.}{Peter  op stukken na}{z\'o\'o niet gelukt was}\\

\haiku{dat zijn offer met!}{dat van Reinout Eussen niet te}{vergelijken viel}\\

\haiku{Tegen Maria die,:}{hem vroeg wat er nu aan de}{hand was zei Lambert}\\

\haiku{er valt niet mee te;}{lachen met wat Lena over}{het hoofd is gegaan}\\

\haiku{heel je leven op!}{water en brood zou beter}{te verdragen zijn}\\

\haiku{Maar de reden, die,;}{hij haar gevraagd had had ze}{hem nooit gegeven}\\

\haiku{vraagt Severinus.}{den mijnportier kregel wat}{er van zijn dienst is}\\

\haiku{een die zoo met lijf,.}{en ziel in het bedrijf zou}{opgaan zeker niet}\\

\haiku{en daarop barsten.}{dan de goden los in een}{algemeen gelach}\\

\haiku{Nee, d\'at vergeet ze,.}{nooit daar kent Lambert Bonte}{zijn schoonzuster voor}\\

\haiku{God en het eigen.}{geweten maken dat uit}{en niet de menschen}\\

\haiku{Maria van Dorus komt.}{met de pan melk onder den}{voorschoot de Kamp af}\\

\haiku{om den tijd hoeft hij,.}{het niet te laten dien heeft}{hij meer dan genoeg}\\

\haiku{ze rijden en het.}{laat Lambert al hoe langer}{hoe meer koud waarheen}\\

\haiku{Lieske had haar in.}{haar vellenjas de deur wel}{uit kunnen kijken}\\

\haiku{Lambert telt het geld:}{tot den laatsten cent op de}{tafel uit en zegt}\\

\haiku{Die geschiedenis;}{met oom Lambert is weer een}{zorg en verdriet apart}\\

\haiku{Een goede vrouw is,;}{een lot uit de loterij}{\'e\'en op de duizend}\\

\haiku{Indien ze allen,!}{als Maria van Dorus waren}{dan \`a la bonheur}\\

\haiku{Dat heb ik voor heel,.}{mijn leven getuigt Lambert}{met heiligen ernst}\\

\haiku{Oom Lambert loopt niet}{meer met de kin op de borst}{en zijn beteuterd}\\

\haiku{Je moet niet lachen,,!}{Willem Bidlot en jij ook}{niet meesterbrouwer}\\

\haiku{twee handhavers der.}{orde staan met de armen}{gekruist bij de deur}\\

\haiku{Laat me met rust, hebt,;}{ge mij gezegd doch daar zijt}{ge niet mee gered}\\

\haiku{zijt - over God spreek ik -.}{niet eens en herinner u}{uw goede dagen}\\

\haiku{En dat voor de paar?}{magere jaren die ge}{nog te leven hebt}\\

\haiku{niet om mijnheer van -}{den Brande een hand boven}{het hoofd te houden}\\

\haiku{Daar vindt Lambert den,.}{oudste te voorzichtig voor}{veel te verstandig}\\

\haiku{Een mensch geneest niet.}{van een ziekte om er weer}{in te hervallen}\\

\haiku{Zouden wij elkaar?}{na zooveel jaren niet eerst}{eens een hand geven}\\

\haiku{En wie zegt jou, dat?}{we het verleden niet met}{rust kunnen laten}\\

\haiku{En wat hebben wij?}{eigenlijk te verbergen}{wat niet openbaar is}\\

\haiku{Je zou de eerste,.}{niet zijn die last kon krijgen}{van die bekoring}\\

\haiku{Die had zwaar voor zijn.}{domheden geboet en die}{boette misschien nog}\\

\haiku{De menschen hadden,.}{er dus het hunne weer van}{gemaakt dacht Lambert}\\

\haiku{Geen van de broeders!}{zou zich voortaan de vingers}{nog aan hem branden}\\

\haiku{Zou je niet naar de,?}{Lindeboom terug willen}{komen Louis Bonte}\\

\haiku{Ik evenwel weet niet,;}{wat daar zoo maar pardoes op}{te antwoorden baas}\\

\haiku{Louis Bonte ging niet,;}{terug naar Canada werd}{algemeen verteld}\\

\subsection{Uit: Mijn moeder Elisabeth}

\haiku{alhoewel ik niet.}{kan zeggen dat ze er nooit}{naar gestaan hebben}\\

\haiku{De boeken die ik;}{gelezen heb willen me}{anders doen gelooven}\\

\haiku{welken reuk zou ik,:}{moeilijk kunnen zeggen maar}{geen aangenamen}\\

\haiku{m\'e\'er dan het builtje -;}{met loon dat dan toch eten en}{leven beteekende}\\

\haiku{Als de vooruitgang?}{van het menschelijk geslacht}{ermee gediend was}\\

\haiku{want van den kelder.}{tot den zolder kwam er in}{zijn huis niets te kort}\\

\haiku{al schoot zij dan ook.}{in den noodigen eerbied voor}{hem nimmer te kort}\\

\haiku{Der Joehan was naar,;}{het dorp naar de Schapens An}{om een pond reuzel}\\

\haiku{Dan wendde hij zich,.}{tot mijne moeder die stil}{over haar naaiwerk zat}\\

\haiku{en die is tot nog,....}{toe door niemand verbeterd}{geworden meen ik}\\

\haiku{Doch geen onzer heeft.}{zich ooit over zijn Heergod te}{beklagen gehad}\\

\haiku{de twee mannen met.}{elkander op en neer naar}{de Mijn te zien gaan}\\

\haiku{Maar het kwam dan toch.}{nog altijd in zijn geheel}{bij moeder terecht}\\

\haiku{Omdat andere;}{mannen dat w\`el doen als ze}{boos op hun vrouw zijn}\\

\haiku{der Goore Joep doet}{het altijd als hij dronken}{is en dan loopen}\\

\haiku{Daar zijn dingen waar,!}{een vrouwmensch nu eenmaal geen}{verstand van heeft vrouw}\\

\haiku{Hoelang reeds gingen?}{vader en moeder niet meer}{samen op en af}\\

\haiku{Hoeveel keeren denk je?}{dan dat ik haar nog naar Aken}{zal moeten brengen}\\

\haiku{Ik kom niet buiten '}{de deur dan oms Zondags}{naar de kerk te gaan}\\

\haiku{en het is zomer,;}{en winter altijd naar de}{vroegmis dat weet je}\\

\haiku{De krant bleef open op.}{tafel liggen en niemand}{sprak verder een woord}\\

\haiku{Bij allebei ben,,....}{ik welkom maar ik denk dat}{ik in het dorp blijf}\\

\haiku{Dat hij nog bidt, en,;}{dat zij die bidden nimmer}{verloren gaan ja}\\

\haiku{doch men verdacht hem,}{ervan er een kunstgreep bij}{van pas te brengen}\\

\haiku{zijn middelen uit.}{nooddruft en veelal schrijnend}{onrecht geboren}\\

\haiku{, hadden we reeds lang.}{geweten en hem ook wel}{eens laten voelen}\\

\haiku{Wat die brief echter;}{behelsde ben ik nooit te}{weten gekomen}\\

\haiku{Maar de hand die hij,;}{me daarbij aan den schouder}{legde woog me zwaar}\\

\haiku{dat hij haar wel naar.}{den trein zou brengen in de}{koets van het Stetsje}\\

\haiku{zei der Hannesso\"e,;}{die daar meteen zijn lepel}{op tafel gooide}\\

\haiku{maar z\'o\'o ook is het!}{al bitter genoeg wat je}{me te slikken geeft}\\

\haiku{De kinderen aan,,?}{huis te hechten is ook wat}{waard dacht ik Jozef}\\

\haiku{Je kunt eruit leeren,!}{wat erin staat niet minder}{en niet meer dan ik}\\

\haiku{Was der Joehan daar?}{onderhand oud genoeg voor}{geworden of niet}\\

\haiku{al zou men hem uit.}{de hel moeten halen of}{met goud beleggen}\\

\haiku{ging der oom Joehan.}{met een dreigend kijken in}{mijn richting verder}\\

\haiku{bol moeten zijn die.}{hem dat alsnog duidelijk}{zou kunnen maken}\\

\haiku{Laat de menschen niet;}{meenen dat we iemand naar}{de o ogen kijken}\\

\haiku{mijn moeder kosten.}{zou haar heelemaal als van}{ons te aanvaarden}\\

\haiku{doch ik had er een.}{gevoelen van waarvoor ik}{geen verklaring wist}\\

\haiku{want die jongen liep.}{toch reeds met het merkteeken van}{Satan in zijn vel}\\

\haiku{Ik stelde me dan}{ook voor dat ik zonder een}{spier te vertrekken}\\

\haiku{Ik moest het hoofd maar,;}{niet dieper laten zinken}{dan noodig was schreef ze}\\

\haiku{Dat heb je er nu,!}{van je had beter kunnen}{blijven waar je was}\\

\haiku{een ervaring, die.}{de hoefsmid Lucassen reeds}{lang had opgedaan}\\

\haiku{Hij had rijkelijk;}{den leeftijd om een meisje}{aardig te vinden}\\

\haiku{er hun rekenschap!}{van afleggen zou zij in}{ieder geval niet}\\

\haiku{Op zekeren dag -:}{ik weet nog goed dat het op}{St. Hubertus was}\\

\haiku{En hij heeft dat lang,.}{volgehouden langer dan}{mijn vader lief was}\\

\haiku{Doch daarmee loopen!}{wij de geschiedenis weer}{een goed eind vooruit}\\

\haiku{niet z\'o\'o bezopen,!}{natuurlijk maar toch alsof}{Atjeh gevallen was}\\

\haiku{Of meende ik soms?}{dat de menschen compassie}{met me moesten hebben}\\

\haiku{Dat was de tweede;}{maal geweest dat zij me ter}{wereld had gebracht}\\

\haiku{Ik moest weer zeggen.}{dat ik gelukkig was weer}{bij moeder te zijn}\\

\haiku{handen, die bijna.}{met alle werkelijkheid}{hadden afgedaan}\\

\haiku{dat die ook met God.}{en met eere tot hunne}{bestemming kwamen}\\

\haiku{Ik stamelde dat.}{men het ook niet erger moest}{maken dan het was}\\

\haiku{En wie er een hoofd!}{voor gekregen had moest dat}{maar leeren gebruiken}\\

\haiku{zoover al dat jij je!}{dochter nu spoedig naar het}{altaar ziet leiden}\\

\haiku{echter zonder de.}{Voorzienigheid ooit uit het}{oog te verliezen}\\

\haiku{Want wanneer de zon.}{eenmaal onder was werden}{de avonden weer kil}\\

\haiku{En daar kwam dan niet,;}{zelden nog schelden tieren}{en dronkenschap bij}\\

\haiku{Die wordt ook met den,;}{dag al gemakkelijker}{vond die Annebil}\\

\haiku{En daar tusschen de;}{nachtwacht met de kanarie}{in de lantaren}\\

\haiku{Als hij dan maar aan,:}{de afspraak dacht drukte der}{Klaus hem op het hart}\\

\haiku{met nog weinig olie.}{in mijn scharnieren en met}{een beetje hoogen rug}\\

\haiku{Wel grepen een paar;}{posteerende Duitschers me}{al gauw bij den kraag}\\

\haiku{En samen hebben.}{wij dan de koeien van de}{zusters gemolken}\\

\haiku{Vandaag alleen heb;}{ik er meer gemaakt dan mijn}{heele leven lang}\\

\section{Emile Seipgens}

\subsection{Uit: De kapelaan van Bardelo}

\haiku{{\textquoteleft}Maar... maar... maar...{\textquoteright} uitte,.}{hij nog zonder te weten}{wat hij zeggen zou}\\

\haiku{Zijn enige troost was;}{het menigvuldig bezoek}{van zijn collega's}\\

\haiku{Gij zijt een steenrots,!}{en op deze steenrots zal}{ik mijn kerk bouwen}\\

\haiku{{\textquoteright} Dirk begreep er niets.}{van hoe de pastoor opeens}{aan een steenrots kwam}\\

\haiku{Toen de karavaan ',.}{s avonds te K. aankwam was}{Peterke doodziek}\\

\haiku{De volgende dag.}{verlangde Arnold Peter}{alleen te spreken}\\

\haiku{de knecht en de meid.}{knielden in een andere}{hoek van de kamer}\\

\haiku{{\textquoteright} Doch Peter zag hem.}{opgetogen aan en hief}{de hand ten hemel}\\

\haiku{Peter Grubbeler.}{mocht terecht een toonbeeld van}{priesterdeugd heten}\\

\haiku{Zijn priesterwijding.}{is inderdaad enkele}{maanden uitgesteld}\\

\haiku{Achter de namen:}{van 3 laatstgenoemden komt}{de opmerking voor}\\

\haiku{Van oktober 1856.}{tot maart 1857 heeft hij dus op}{dat adres  gewoond}\\

\haiku{Deze getuigen {\textquoteleft}{\textquoteright}.}{warende naburen van}{de overledene}\\

\haiku{Misschien moeten wij.}{het antwoord zoeken bij het}{type Seipgens zelf}\\

\haiku{Causerie over {\textquoteleft}De{\textquoteright} ().}{kapelaan van Bardelo}{niet gepubliceerd}\\

\haiku{Of liberaal is,:}{ook iemand die b.v. over eene}{preek wel eens oordeelt}\\

\haiku{Z\'o\'o is 't op 't - '.}{platteland in de stad is}{t eenigszins anders}\\

\section{Kees Simhoffer}

\subsection{Uit: Een geile gifkikker}

\haiku{{\textquoteleft}U mag hem gerust,,}{proberen als u wel uw}{schoenen wilt uitdoen}\\

\haiku{Hij kijkt me aan en.}{pakt een sigaret uit een}{metalen koker}\\

\haiku{{\textquoteleft}Ik zie het al. In.}{het weekend een beetje te}{lang buiten geweest}\\

\haiku{Of schaft u zich een.}{Philifoonsmell Klank-}{en Reukorgel aan}\\

\haiku{Tenzij wij hun slecht.}{bekomen en ze ons nog}{\'e\'en keer uitkotsen}\\

\haiku{Ik help haar uit de.}{vrieskast en sla de rijp van}{haar fluwelen jurk}\\

\haiku{Ze denken gewoon.}{dat we zoals altijd aan}{het ontbijt zitten}\\

\haiku{Ze doet het knipje.}{van de badkamerdeur en}{komt fris naar buiten}\\

\haiku{Zonde van al die.}{tandpasta met aktieve}{superfluorol}\\

\haiku{{\textquoteright} {\textquoteleft}Mijn moeder vindt dat.}{pletsspuitbus sneller klient dan}{peliwrijfwas boent}\\

\haiku{De erwten rollen.}{met honderden tegelijk}{over de kale grond}\\

\haiku{gelukkig dat ie...}{eindelijk verstandig is}{en naar de dokter}\\

\haiku{Scheidt een giftige,.}{zeer besmettelijke stof}{af in paringstijd}\\

\haiku{{\textquoteright} {\textquoteleft}Ik zou maar gauw een,{\textquoteright}.}{bad nemen als ik jou was}{antwoordt de moeder}\\

\haiku{Even later lig ik.}{languit in bad en blaas het}{schuim van mijn handen}\\

\haiku{{\textquoteright} Dan draait ze zich om.}{en loopt een lange gang in}{achter het paneel}\\

\haiku{Ik duik onder het.}{bed door en ren op de deur}{in het paneel af}\\

\haiku{We zullen eens een.}{keertje niet onderdoen voor}{de geschiedenis}\\

\haiku{Later pas ga je,.}{begrijpen wat je veel te}{jong nog moest leren}\\

\haiku{De jonge mensen.}{staan al dertig jaar in de}{rij voor een woning}\\

\haiku{En als de Turk griep,.}{heeft ontsmetten we niet meer}{meteen zijn kamer}\\

\haiku{{\textquoteright} {\textquoteleft}Laat maar,{\textquoteright} zeg ik en}{hij loopt terug naar de bar}{met een gezicht van}\\

\haiku{Daar begint ie met.}{beverige handen zijn}{broek los te maken}\\

\haiku{Ik heb er allang,.}{geen haakje meer op alleen}{in verband met Joop}\\

\haiku{Als ik een tijdje,.}{later het caf\'e uitkom}{regent het nog steeds}\\

\haiku{Zoals dokters een:}{kankerpati\"ent naar huis}{sturen en zeggen}\\

\haiku{Ik herken haar pas.}{als de trein de donkere}{overkapping uit is}\\

\haiku{{\textquoteleft}Uw tante is net,...}{van dat paard gevallen ik}{geloof dat ze dood}\\

\haiku{tante voelt zich niet,.}{zo goed vanmorgen tante}{heeft weer wat gebloed}\\

\haiku{{\textquoteleft}Sorry, ook zonder.}{pokkenbriefje kan een mens}{de klere krijgen}\\

\haiku{We rijden weg naar,,.}{het zuiden de stad uit over}{de autosnelweg}\\

\haiku{Dit uitzicht wordt u{\textquoteright}}{aangeboden door Poly}{Chemic Company}\\

\haiku{Ik steek de straat over.}{en kom even later in een}{drukke winkelstraat}\\

\haiku{Er is niks aan de.}{hand. Iedereen loopt weer druk}{boodschappen te doen}\\

\haiku{In Rex draait nog steeds,.}{blanke  onschuld ook al}{regent het niet meer}\\

\haiku{Maar als ze een doek,.}{over me heen willen leggen}{gaat de telefoon}\\

\haiku{{\textquoteright} roept de chauffeur van,.}{de volgende vrachtauto}{hangend uit zijn raam}\\

\haiku{(Conventie van de.}{Verenigde Naties van}{9 december 1948}\\

\haiku{{\textquoteright} ~ Terwijl ze de,}{woorden van Dylan zingt vraag}{ik me verbaasd af}\\

\haiku{{\textquoteleft}H\`e leuk dat u toch,}{nog tijd gevonden hebt om}{te komen kijken}\\

\haiku{{\textquoteright} zegt een dikke man,.}{tegen ons het vet glimmend}{in zijn mondhoeken}\\

\haiku{{\textquoteright} schreeuw ik en duik hem,}{na maar ik kom terecht in}{een leeg donker gat}\\

\haiku{{\textquoteright} Ze staart angstig naar.}{de twee lege spoelen die}{langzaam ronddraaien}\\

\haiku{En de vierde keer.}{had u weer bloemen in de}{vaas en geen sleutel}\\

\haiku{Ik zit immers al.}{gevangen in het web van}{een andere spin}\\

\haiku{{\textquoteright} In paniek ren ik,.}{weg zo snel mijn benen me}{dragen kunnen}\\

\haiku{Men is het echter.}{nog niet eens over de vorm van}{de biljarttafel}\\

\section{Jozef Simons}

\subsection{Uit: Eer Vlaanderen vergaat}

\haiku{En vandaag was het.}{de inhuldiging van de}{nieuwe kasteelheer}\\

\haiku{{\textquoteright} En om het over een,:}{andere boeg te gooien}{zei hij wat later}\\

\haiku{het is uw eigen!}{vader die mij dat vroeger}{heeft ingeblazen}\\

\haiku{Zul je hier liever?}{wonen dan te Brussel of}{te Blankenberge}\\

\haiku{Boven de tuinhaag:}{zagen ze het hoofd van een}{fietser voortglijden}\\

\haiku{Mocht weer het uur voor,?}{Vlaanderen slaan hoevelen}{hebben de ogen open}\\

\haiku{Ik ben altijd veel.}{te goed geweest dan dat je}{mij dat zou aandoen}\\

\haiku{En hoe schandelijk ',!}{gingen ze te werk mett}{vrouwvolk jong en oud}\\

\haiku{hij was toch geen kind -.}{meer en achter de IJzer}{te gaan meestrijden}\\

\haiku{zou papa evenwel.}{liever hebben dan dat hij}{naar het leger trok}\\

\haiku{Met wat een hartstocht.}{volgde hij het streven van}{de activisten}\\

\haiku{{\textquoteright} - ze moesten de paarden.}{gaan roskammen en wrijven}{en haver geven}\\

\haiku{Sedert 1830 begon.}{de lijdensgeschiedenis}{van het Vlaamse volk}\\

\haiku{Wat er niet genoeg,.}{uit sprak was zelfstandigheid}{en daadvaardigheid}\\

\haiku{{\textquoteleft}En gij, Fons, zoudt gij?}{ook voor Vlaanderen door een}{vuur willen lopen}\\

\haiku{{\textquoteright} De brancardier ging,:}{met een beursje rond en toen}{hij telde vond hij}\\

\haiku{De tocht vorderde,,.}{langzaam over Bulskamp Nieuwe}{Herberg Hoogstade}\\

\haiku{Om halfnegen kwam.}{plots het bevel dat ze moesten}{tegen-schieten}\\

\haiku{{\textquoteright} {\textquoteleft}Ja, maar soms heb ik,.}{toch heimwee naar het voetvolk}{naar de piotten}\\

\haiku{In die richting mag,.}{op mij gerekend worden}{ik houd me bereid}\\

\haiku{- Marieke .... 'k Dacht!}{toch dat ik Marieke had}{zien binnenkomen}\\

\haiku{Dat onze Staat het,:}{recht had ons bloed en leven}{te vragen dit is}\\

\haiku{daaruit sprak immers.}{een geest bereid tot opstand}{tegen het onrecht}\\

\haiku{IV De maand maart met,.}{haar lichtere dagen bracht}{meer gevechtsdrukte}\\

\haiku{{\textquoteright} zei de wachtmeester,.}{met tien vingers tegelijk}{in zijn haar krabbend}\\

\haiku{de twee andere.}{stukken moesten ook achteruit}{in reparatie}\\

\haiku{Een vliegtuig toerde.}{boven de haven waar de}{lichtjes aanploften}\\

\haiku{{\textquoteright} {\textquoteleft}Vier uur door... en zijt,!}{maar br\^a content ge krijgt straks}{allemaal rijstpap}\\

\haiku{Florimond was er.}{beschaamd over dat hij zichzelf}{niet meer meester bleef}\\

\haiku{Clara vernam dan.}{ook wat meer over Florimond}{en zijn familie}\\

\haiku{Trouwens, ik geloof.}{niet dat Borms koel en klaar ziet}{waar hij naartoe gaat}\\

\haiku{{\textquotedblleft}Es ist unsere,.}{verdammte Pflicht Oesterreich}{treu zu verraten}\\

\haiku{{\textquoteright} {\textquoteleft}Jan,{\textquoteright} zei Florimond, {\textquoteleft}:}{herinner u het woord van}{onze kapelaan}\\

\haiku{Het plan staat me klaar,?}{voor de geest maar wie helpt mij}{aan de middelen}\\

\haiku{Onopgemerkt sloop,.}{hij weer de winkel door over}{de straat naar het strand}\\

\haiku{En hoe de mannen!}{leute hadden met de schrik}{van de gendarmen}\\

\haiku{Of slechts, omdat ze?}{bij Depla geen gehoor}{hadden gevonden}\\

\haiku{Niets dan gas - tot de!}{Fritzen ervoor bedanken}{en het aftrappen}\\

\haiku{Kef,{\textquoteright} deed een kleine, '.}{vijfenzeventiger die}{t aanvalsuur aangaf}\\

\haiku{In een holle weg.}{werden de stukken tegen}{de haag getrokken}\\

\haiku{Gaarne offer ik,.}{mijn leven op aan God voor}{vrijheid en voor recht}\\

\haiku{{\textquoteright} zei, met zijn zware,,.}{basstem de Antichrist de}{wenkbrauwen fronsend}\\

\haiku{geen klare klank van.}{hogere idealen een}{weerklank vinden zou}\\

\haiku{{\textquoteleft}Maar gaat toch niet naar,...}{Ramskapelle daar schieten}{ze met schrapnellen}\\

\haiku{De mensen hadden.}{al lang moeite om zich stil}{en koest te houden}\\

\haiku{En de pastoor stond!}{de laatste tijd bij het volk}{slecht aangeschreven}\\

\haiku{Jan en Clara moesten.}{aanhoudend uitwijken voor}{bellende fietsers}\\

\haiku{{\textquoteright} Opeens greep Jan haar,:}{arm met meer tederheid en}{zei zo heel innig}\\

\haiku{De kom van 't dorp,.}{was als een bobbelende}{ziedende ketel}\\

\subsection{Uit: In Itali\"e}

\haiku{Fientje, Bert noch de.}{President wisten goed hoe}{zich aan te stellen}\\

\haiku{Neen, het bestuur der...}{bedevaart zal niet in fout}{gevonden worden}\\

\haiku{Slingerend klimmen.}{de witte krijtwegen naar}{het land van Savoie}\\

\haiku{Ge ziet ze stappen,.}{met welgemeten tred zooals}{ze doen rond hun land}\\

\haiku{De President wil.}{weten voor hoeveel volk de}{kerk wel ruimte biedt}\\

\haiku{Op de Cavourbrug}{voorbij het gerechtshof zien}{wij den  schoonen}\\

\haiku{V\'o\'or een albergo,.}{waar een strijkje speelt komt een}{hoektafeltje leeg}\\

\haiku{Maar rond middernacht:}{viel hij uit zijn bed met groot}{lawaai al kreunend}\\

\haiku{zeg ik tegen den,.}{chauffeur die terzelfder tijd}{cicerone speelt}\\

\haiku{een modern klooster,,.}{een ander uit de jaren}{1052 de kerk van Ste}\\

\haiku{In het Vaticaan:}{is ondergebracht gansch het}{bestuur der H. Kerk}\\

\haiku{{\textquoteleft}Welkom, dierbare,...}{kinderen uit het diepste}{van ons vaderhart}\\

\haiku{De kerker is in:}{de rots gehouwen en heeft}{twee verdiepingen}\\

\haiku{Al de gazetten.}{geven het conterfeitsel}{van Mussolini}\\

\haiku{Vlaamsche boeren het.}{gekir en geklapwiek der}{St. Marcusvogels}\\

\haiku{De wolken drijven.}{en de zon doet moeite om}{een spleet te vinden}\\

\haiku{De Pilatusberg:}{bij Luzern is met een hoed}{van wolken bedekt}\\

\haiku{De musschen sjirpen.}{en de merels fluiten in}{den frisschen morgen}\\

\haiku{Het bezoek aan den.}{Gletschergarten laten we aan}{de liefhebbers over}\\

\haiku{na nog een beetje, - -!}{dommelen stoppen wij laat}{kijken te Aarlen}\\

\haiku{Namen - Hoeilaart, het {\textquoteleft}{\textquoteright} - -:}{glazen dorp Groenendaal en}{kort na zeven uur}\\

\subsection{Uit: In Spanje}

\haiku{Drie heeren lezen,.}{hun dagblad twee juffrouwen}{een romannetje}\\

\haiku{Heeroom draait zich op,,:}{zijn gemak in den hoek neemt}{nog een snuifke zegt}\\

\haiku{De dru{\"\i}den met... -...}{gouden sikkel Kijk eens daar}{in dien hollen weg}\\

\haiku{Ik hoor de dame.}{achter den toog keuvelen}{met haar dochtertje}\\

\haiku{In mijn hofje stond.}{een mooi pereboompje dat}{vijftien peren droeg}\\

\haiku{Mannen, vrouwen en, '.}{kinderent krioelt er}{bedrijvig dooreen}\\

\haiku{Als 't niet op tijd,.}{regent is een boer op \'e\'en}{jaar gansch ten onder}\\

\haiku{nu eerst beginnen,,.}{de stadslantarens voor zooveel}{er zijn te branden}\\

\haiku{Twee misdienaars, vlug,.}{als jonge katten houden}{er boksoefening}\\

\haiku{we akkoord treffen.}{voor negen pesetas met}{recht op drie uur tijd}\\

\haiku{slechts \'e\'en beuk, doch twee:}{afsluitingen verdeelen}{haar in drie deelen}\\

\haiku{Al de landbouwers.}{trachten zooveel mogelijk}{grond bij te huren}\\

\haiku{Wij rijden over een,.}{hoogvlakte uitheuvelend}{op al de kimmen}\\

\haiku{Misschien is het ras '?}{van dieven en moordenaars}{aant verkwijnen}\\

\haiku{Stilaan zijn we met.}{onze twee Spanjaards dikke}{vrienden geworden}\\

\haiku{hij was in zijn geest ',.}{aant spinnewebben dat}{was hem aan te zien}\\

\haiku{- Och ja, zei ze en,:}{ze kreeg een kleur dit vergat}{ik u te zeggen}\\

\haiku{De Padre doet ons,.}{uitgeleide tot aan neen}{tot in de statie}\\

\haiku{{\textquoteright} En de Zuidersche,;}{zon laait daarover en doet al}{de kleuren vlammen}\\

\haiku{'t Jongetje vangt.}{ze behendig en bijt er}{in met vollen mond}\\

\haiku{Na het avondmaal nog.}{even gaan slenteren in het}{drukke stadsgewoel}\\

\haiku{Een wriemeling als.}{van kleurige kabouters}{op Walpurgisnacht}\\

\haiku{Sierlijk slaan ze met:}{de panden van hun mantels}{van diverse kleur}\\

\haiku{De espadas buigen,.}{voor den President allen}{ontblooten het hoofd}\\

\haiku{Op zijn knie\"en kruipt... -! -!}{de matador er op af}{Dat is geen sport Awoert}\\

\haiku{Aan den omdraai de:}{kleine misdienaar gevolgd}{door een hulpkoster}\\

\haiku{Heeroom wordt in zijn.}{les onderbroken door het}{stoppen van den trein}\\

\haiku{Goudsmeed- en:}{brokaatwerk voor een waarde}{van millioenen}\\

\haiku{- Hier ook hebben de;}{boeren tijdens den oorlog}{veel geld gewonnen}\\

\haiku{Zij waakt over de stad,.}{zij strekt haar machtigen arm}{uit over gansch Aragon}\\

\haiku{hij stak een schoonen ' '.}{cent op zak en kwam bijt}{volk int gevlei}\\

\haiku{'s Avonds te zes uur.}{plechtige vergadering}{in den stadsschouwburg}\\

\haiku{'s Morgens in de,;}{kerk van den Pilar een pracht}{van een processie}\\

\haiku{Te half twee, in het,.}{hotel Lac een banket voor}{zestig genoodigden}\\

\haiku{Laat ik nog even, voor,:}{de laatste maal het menu}{hier overschrijven}\\

\haiku{Oom Jan zit in zijn:}{hoekje in zijn brieventesch}{te  snuisteren}\\

\subsection{Uit: De laatste flesch}

\haiku{{\textquoteright} Kapelaan Deckers.}{at zijn sigaar half op van}{opgewondenheid}\\

\haiku{Mijnheer pastoor is.}{ook zeer tevreden over zijn}{nieuwen kapelaan}\\

\haiku{{\textquoteright} De knie\"en van den.}{kapelaan knikten en een}{floers schoof voor zijn oogen}\\

\haiku{{\textquoteright} - {\textquoteleft}Wie zal er mij met,?}{het scheermes schoon maken als}{ik gestorven ben}\\

\haiku{{\textquoteright} - {\textquoteleft}Ik scheer U nu al,,.}{zoolang mijnheer pastoor dat}{moet dan ook maar gaan}\\

\haiku{Eulalie schoot hem: - {\textquoteleft}?}{voor met de vraagEn wat gaan}{we hem voederen}\\

\haiku{{\textquoteright} - {\textquoteleft}Maar ziet eens wat een,,.}{lange snuit wat een grove}{graat wat een hoogen rug}\\

\haiku{En hoe duchtiger,.}{Meneerke schrobde hoe}{luider Kees gilde}\\

\haiku{'s Dinsdags kwam de - - {\textquoteleft}{\textquoteright}.}{slachter en glorieloos lot}{kapte Kees infrut}\\

\haiku{De pastoor werd oud,.}{leed aan een maagkwaal en kon}{geen wijn meer rieken}\\

\haiku{loontje komt om zijn,.}{boontje hij heeft er vroeger}{genoeg geroken}\\

\haiku{Dat behoort tot de,,.}{gewoonte de zede de}{mores van de streek}\\

\haiku{geen enkele der.}{gewone       slapers kon}{zijn uiltje vangen}\\

\haiku{En een gezellig:}{koutje geslagen over wat}{de dag zooal meebracht}\\

\haiku{Ja, Tintelenteen...}{had zin in duiven en was}{een fijne kenner}\\

\haiku{Hij springt los over een:}{koppel veulens en vliegt op}{mij toe als een zot}\\

\haiku{De wind zwiepte dat.}{de boomtakken kraakten en}{de kruinen zoefden}\\

\haiku{Opeens zag ik de:}{voorsten gebaren maken}{met armen en beenen}\\

\haiku{En zij voegt de daad,.}{bij het woord smeedt de praktijk}{aan de theorie}\\

\haiku{{\textquoteright} - {\textquoteleft}Ch\'eri, ik dank je,...}{voor de attentie je bent}{nog steeds dezelfde}\\

\section{J.R.W. Sinninghe}

\subsection{Uit: Limburgsch sagenboek}

\haiku{Allen, die daarvan,.}{wisten zijn echter gesmoord}{in de moerassen}\\

\haiku{Zij dacht alleen nog;}{maar aan die kist met geld en}{aan niets anders meer}\\

\haiku{Maar de koning die,.}{links te paard stijgt zal te Mook}{vluchten over de brug}\\

\haiku{Ten laatste werden}{ze in den nacht overvallen}{en kwamen allen}\\

\haiku{Als hij ze eenmaal,.}{op Horn had nam hij de rest}{voor zijn rekening}\\

\haiku{de hooge muren en.}{onder de breed overwelfde}{bogen der keukens}\\

\haiku{{\textquoteleft}bij de Maas zal hij,.}{stil moeten houden hij is}{te Ool nog niet over}\\

\haiku{Ik kreeg van iemand,:}{zeshonderd gulden en ik}{dacht al bij mezelf}\\

\haiku{1) H. Welters in,-,,,.}{Limburgsche Legenden I.}{121122 127 132 229}\\

\haiku{Schippers, die op de,.}{Noordzee varen hebben het}{spookschip vaak ontmoet}\\

\haiku{{\textquoteleft}So leget mi in,,.}{die coetse blank Opdat ic}{ruste ic ben crank}\\

\haiku{{\textquoteleft}O grond, rijt op, 'k.}{wil in din schoot Bi Reynout}{wesen in der doot}\\

\haiku{Uit dank schonk Keizer.}{Karel het verblijf van zijn}{dochter aan den Huyn}\\

\haiku{Die Jan was verliefd,,}{op Hanna de dochter van}{zijn baas maar zij wou}\\

\haiku{Een generaal, hoog,.}{te paard gezeten met zijn}{ruiters achter hem}\\

\haiku{Een boer uit Wijk zei,.}{tot Alfred Harou dat steenen}{groeien als planten}\\

\haiku{De knecht hield echter,.}{zijn onschuld vol en zeide}{dat het niet waar was}\\

\haiku{Van Einighausen,,:}{een gehucht ten oosten van}{Limbricht verluidt het}\\

\haiku{Sindsdien dragen de {\textquoteleft}{\textquoteright}.}{Venlonaren den bijnaam}{Wannenvliegers}\\

\haiku{het klokkengebrom.}{Doet ijslijk de loopende}{landlieden zuchten}\\

\haiku{Weer probeerde hij,.}{van wal te steken en weer}{lukte het hem niet}\\

\haiku{Zij wist hem zelfs te,.}{bepraten het beeldje naar}{een kerk te brengen}\\

\haiku{Ineens brak ergens,,.}{de stilte en het dier zag}{op even verwonderd}\\

\haiku{Thans kon het niet lang,....}{meer duren of het schaap zou}{worden verslonden}\\

\haiku{Het trok, met al zijn,,.}{macht en zoowaar het rukte het}{paaltje uit den grond}\\

\haiku{Die plek had precies.}{den vorm van den grondslag van}{een kapelletje}\\

\haiku{Gedurende vier.}{en zeventig jaar had hij}{zijn bisdom bestuurd}\\

\haiku{De arbeiders, die.}{hij daar bezig vond in den}{wijnberg verdreef hij}\\

\haiku{Deze vluchtten nu.}{naar Maastricht en verhaalden}{daar het gebeurde}\\

\haiku{{\textquoteright} 1) ~ * * * ~ Voor men.}{op reis ging dronk men steeds de}{St. Geertenminne}\\

\haiku{{\textquoteright} In hun angst holden.}{zij tusschen de menigte}{door naar den prior}\\

\haiku{Daar vertelden zij.}{het gebeurde en toonden}{den bebloeden draad}\\

\haiku{Geen mensch had echter,.}{eenig letsel bekomen dank}{zij Sint Gerlach}\\

\haiku{Het bleek een monnik,.}{te wezen die hier blijkbaar}{wilde overnachten}\\

\haiku{{\textquoteleft}Ik was juist van plan,?}{mijn ziel te verdobbelen}{wat dunkt u ervan}\\

\haiku{Daarop ging hij weer,.}{naar zijn vrienden terug en}{speelde en dronk voort}\\

\haiku{{\textquoteright} Na deze woorden. '}{reed de duivel hem weer naar}{zijn molen terug}\\

\haiku{s Morgens vond de.}{vrouw haar man halfdood voor den}{molen  liggen}\\

\haiku{{\textquoteleft}Ga naar den molen,!}{en maal den zak koren die}{is aangekomen}\\

\haiku{{\textquoteleft}Het zou u straks te,!}{lastig zijn om de paarden}{te doen stilhouden}\\

\haiku{Zij aten er niet van,,.}{want niemand wist wat het was}{of van waar het kwam}\\

\haiku{Geen bergkant was hem,,.}{te steil geen water te breed}{geen moeras te diep}\\

\haiku{De reus - vertelt men -.}{in Maastricht woonde op een}{berg met zijn dochter}\\

\haiku{{\textquoteright} {\textquoteleft}'t Is geen kraai, 't,{\textquoteright}, {\textquoteleft}!}{is een spreeuw antwoordde de}{koningwerk maar door}\\

\haiku{De man met den haak.}{wordt door alle Limburgsche}{kinderen gevreesd}\\

\haiku{De grond was warm door.}{het sleepen van den vurigen}{ketting van Satan}\\

\haiku{Zij, die ziek werden,;}{en de koorts kregen stuurde}{hij zonder hulp weg}\\

\haiku{Dorren, Het Kasteel,-,.}{van Valkenburg blz. 4748}{en Pierre Kemp id}\\

\haiku{{\textquoteleft}Als ge hem dezen,.}{nacht nog eens ziet vraag hem dan}{wat hij hebben moet}\\

\haiku{Maar noch haar broers, noch.}{haar zusters bekommerden}{zich om haar bede}\\

\haiku{- maar niet voor lang, want.}{zij keerden spoedig weer en}{vielen opnieuw aan}\\

\haiku{Te Venlo kon men:}{de blinde ronde op de}{wallen ontmoeten}\\

\haiku{{\textquoteleft}Ik ging dan weerom{\textquoteright} - - {\textquoteleft};}{zoo vertelt hijden bosch in}{als niemand mij zag}\\

\haiku{Ge kunt denken, dat '.}{iks Zondags nooit meer naar}{het bosch ben geweest}\\

\haiku{{\textquoteright} Eindelijk liet de.}{veerman zich toch bepraten}{en zette hem over}\\

\haiku{Het heette, dat het.}{vurig ros met hem in den}{grond verzonken was}\\

\haiku{Al in de verte.}{hoorden ze de kat over de}{daken naderen}\\

\haiku{De vriendin maakte,.}{eerst bezwaren maar zij liet}{zich toch overhalen}\\

\haiku{De jonge heks zou,.}{al gauw merken hoe leelijk}{zij zich had vergist}\\

\haiku{H\`e haw \`evel gaar gein,}{macht meer euver de heksen}{en vloag allein weer}\\

\haiku{Wij komen ons hier,{\textquoteright}, {\textquoteleft}!}{amuseeren was het antwoord}{wat gaat jou dat aan}\\

\haiku{Doch eer de dag om,:}{was kwam het bericht dat hij}{verongelukt was}\\

\haiku{daar tolde de haas.}{met het volle schot hagel}{in zijn kop omver}\\

\haiku{Te Gronsveld waren,.}{het krekels die door een heks}{gezonden waren}\\

\haiku{{\textquoteright} De pastoor gaf den.}{moed niet verloren en de}{terugreis begon}\\

\haiku{Maar kaartspelen op,!}{Zondag onder de Hoogmis}{dat is wat anders}\\

\haiku{2) A.F. van Beurden,,,.}{in Limburgs Jaarboek XX blz.}{183 en XXV blz. 88}\\

\haiku{Dadelijk begreep,.}{hij dat hij met een weerwolf}{te doen had gehad}\\

\haiku{De man, niet mis, nam '.}{zijn mes en stak stillekens}{t beest in den buik}\\

\haiku{Opeens, met klokke,.}{twaalf viel het wolfsvel uit den}{schoorsteen op het vuur}\\

\haiku{Onderwijl sloop de.}{knecht stilletjes naderbij}{en nam de zeef weg}\\

\haiku{{\textquoteright} Nu, dat had eens een:}{mensch afgeluisterd en die}{dacht in zijn eigen}\\

\haiku{{\textquoteleft}Ge moet niet grijnen,.}{ik zal u de kinderen}{wel terug geven}\\

\haiku{Wat was de koning,:}{blij maar ook wat was hij kwaad}{op de drie zusters}\\

\haiku{'t Waren Onze,.}{Lieve Heer en Sint Pieter}{maar Jan wist dat niet}\\

\haiku{In den kelder wou ',:}{het spook Jan aant graven}{zetten maar Jan zei}\\

\haiku{Maar ieder maal dat,.}{Jan terugkwam blies het spook}{opnieuw zijn lamp uit}\\

\haiku{(blz. 283) Sinninghe,,,.}{Overijsselsch S. blz. 59 vlg.}{Idem Noord-Brab}\\

\haiku{Bijdragen tot de ();}{geschiedenis der hoofdbank}{ClimmenMaastricht 1906}\\

\haiku{34Dit kunststukje '.}{staat ook op naam van een knecht}{vant kasteel Horn}\\

\subsection{Uit: Verhalen uit het land der bokkenrijders en der Teuten}

\haiku{Op een morgen was!}{de ruitersman weg en de}{prinses was ook weg}\\

\haiku{Anderen zetten.}{wel eten voor ze klaar en daar}{werkten ze dan voor}\\

\haiku{Het graven van dat:}{kanaal werd opgedragen}{aan twee aannemers}\\

\haiku{Opeens zag hij in.}{de verte een vuurgloed die}{recht op hem toekwam}\\

\haiku{Wie hijn ein paar trij,:}{neive waas huurde hijn den}{honk inens zegge}\\

\haiku{{\textquoteleft}dao moot iech ouch get{\textquoteright},:}{van hobbe mer wie de vrouw}{hiel kordaot zag}\\

\haiku{Omstreeks 1772 kwam een,,.}{dronkaard ijselijk vloekend}{zijn kamer binnen}\\

\haiku{{\textquoteleft}soodaenige dingen ()}{als dezelve oculomet}{eigen ogen gesien}\\

\haiku{Het getuig hing aan.}{de disselboom en het paard}{stond er doornat naast}\\

\haiku{In mijn tijd nog had, '.}{je in Maasniel vrouw Giesbers dat}{wasn ouwe vrouw}\\

\haiku{Iedereen wist met.}{zijn klompen aan dat dit het}{werk van die heks was}\\

\haiku{Twie\"e j\'ongmen zogan.}{eum saoves es drie\"e op}{ein soets aaf loupe}\\

\haiku{Sindsdien heeft men nooit}{meer iets vernomen van het}{duivelsboekje.136}\\

\haiku{M'n moeder had maar.}{een broer en die ging altijd}{met die paarden om}\\

\haiku{Hij was acht meter.}{hoog en had een middellijn}{van twintig meter}\\

\haiku{Dou verennigden sich,:}{de wisen van den N\"ustad}{en do sagt einen}\\

\haiku{De rovers gingen}{hem dadelijk achterna}{en ze doorzochten}\\

\haiku{Hollandse Pier zat.}{met een knie op de wilg want}{meer plaats was er niet}\\

\haiku{Gij moet leggen ten}{kronen voor het houter en}{ten voor het broken}\\

\haiku{Maar al de knechts, de ().}{gendarmen en de boyszoons}{zaten bij het geld}\\

\haiku{Op 'n keer was hij.}{weer daar en ze gongen naar}{de bijen kijken}\\

\haiku{Dat was 'ne grote.}{boer en daar hadden ze pas}{het verken geslacht}\\

\haiku{Met dat worstelen.}{had ze gezien dat het die}{van de Schans waren}\\

\haiku{{\textquoteleft}Ja maar{\textquoteright}, zei de baas, {\textquoteleft},...{\textquoteright} {\textquoteleft}{\textquoteright},.}{zeg er maar niks van wantNiks}{daarvan zei de maagd}\\

\haiku{Het geld lig in de,.}{kelder onger den ungerste}{onger eine stein}\\

\haiku{Dat is een goeie{\textquoteright}, zei, {\textquoteleft}.}{de kosterzo lust ik er}{wel elke dag een}\\

\section{J. Slauerhoff}

\subsection{Uit: Het lente-eiland en andere verhalen}

\haiku{Maar de karakters.}{zullen verwrongen zijn door}{den groei van het hout}\\

\haiku{{\textquoteleft}Zie mijn voeten, ze,.}{zijn zo klein ze nemen geen}{plaats op het dek in}\\

\haiku{En mijn lichaam, zo,.}{slank als een lelieblad ik}{kan het opvouwen}\\

\haiku{Als de wind door de,.}{pijnen toog hoorde hij het}{koor der zusteren}\\

\haiku{Hij dronk geen wijn meer.}{en kocht den edelsten inkt en}{het beste papier}\\

\haiku{voortdurend gingen.}{bliksemschichten heen en weer}{tussen hen en hem}\\

\haiku{Een mandarijn is.}{daar en verlangt dit vertrek}{om te overnachten}\\

\haiku{{\textquoteright} {\textquoteleft}Ik ben van mijn post.}{ontheven en naar het hof}{teruggeroepen}\\

\haiku{{\textquoteleft}In welk land kent men?}{een moorddadige kracht toe}{aan de po\"ezie}\\

\haiku{In het Westen is.}{er wel een land waar men haar}{jeugdbederver noemt}\\

\haiku{Ik ben geen profeet,,,.}{meer geen po\"eet meer ja geen}{geletterde meer}\\

\haiku{Ik daarentegen.}{ga naar Tsjung Tsjow om een}{vrouw te ontvluchten}\\

\haiku{Zij kan borduren,.}{en luitspelen bloemen en}{vogels schilderen}\\

\haiku{Zij eiste van mij.}{volledige overgave}{van lichaam en ziel}\\

\haiku{even zwaar is het, een.}{gestorven kind weer in het}{leven te roepen}\\

\haiku{Neen toch, er was iets,;}{in waar al het latere}{werk niet bij haalde}\\

\haiku{Maar nooit waagden zij,.}{het rechtstreeks iets tegen hem}{te ondernemen}\\

\haiku{Opeens liet zij haar,.}{houding varen strekte de}{armen naar hem uit}\\

\subsection{Uit: Schuim en asch}

\haiku{Sinds jaren hangt zijn,:}{linkerhand neer machteloos}{ter aarde wijzend}\\

\haiku{{\textquoteright} Hassein ging naar huis,.}{vervolgd door den schimp van den}{woedenden moefti}\\

\haiku{De eerste droeg een.}{groenen tulband met  de}{gouden halve maan}\\

\haiku{Met de opbrengst had.}{hij geen drie dagen gage}{kunnen betalen}\\

\haiku{Kasem Hassein deed thans:}{datgene waarmee hij had}{moeten beginnen}\\

\haiku{Mijn heupwond schrijnde,.}{mij ook meende ik jicht te}{voelen opkruipen}\\

\haiku{Ik belde vergeefs.}{en eindelijk sleepte ik}{mij tot de tafel}\\

\haiku{Ik was volkomen,.}{gedachteloos leed niet en}{zwijg dus over dien tijd}\\

\haiku{Vol wantrouwen bood.}{ik hem een glas wijn en vroeg}{wat hij verlangde}\\

\haiku{Ik maakte mij boos,,.}{maar de monnik lachte en}{boog en trok mij mee}\\

\haiku{{\textquoteleft}Ik wil hier blijven,,.}{een boot hebben zeilen en}{visschen in de baai}\\

\haiku{Het scheepje slingert,.}{hoewel de zee niet woelig}{is en de wind luw}\\

\haiku{Direct van Itali\"e,.}{hierheen gekomen zou ik}{verheugd zijn geweest}\\

\haiku{{\textquoteleft}Om te luisteren.}{ben ik van het einde der}{wereld gekomen}\\

\haiku{{\textquoteright} {\textquoteleft}Het schip ligt sturdy,?}{over hoe kan een losse kist}{dan zoo te keer gaan}\\

\haiku{Bevelen vielen,.}{niet te geven het schip dreef}{op Gods genade}\\

\haiku{Deze beval hem.}{kort en ruw dien rommel uit}{de poort te smijten}\\

\haiku{Als we onthoofde,?}{lijken aanbrengen wat staat}{ons dan te wachten}\\

\haiku{Tast het hout niet aan,,.}{maakt den vloer niet glad de booze}{geesten niet wakker}\\

\haiku{Nu zou de Nyborg,,.}{als het aan hem lag zeilschip}{blijven tot zijn eind}\\

\haiku{Soden wilde wel,}{maar het vooruitzicht leek hem}{onbereikbaar ver.}\\

\haiku{{\textquoteright} {\textquoteleft}Je begrijpt toch wel.}{dat ik met deze mist niet}{door de riffen ga}\\

\haiku{Maar hij lag op den,,,}{vloer een matroos schudde hem}{trok hem naar buiten}\\

\haiku{Geld had ik niet, het.}{biljet naar Bordeaux was nog}{mijn eenigst aardsch bezit}\\

\haiku{Het was mij alsof.}{ik mij in dit land nog met}{je samen voelde}\\

\haiku{Had een monnik hier?}{bij jou met zijn geloof zijn}{pij laten liggen}\\

\haiku{Het eten werd haastig.}{en kleumend in de kille}{messroom gegeten}\\

\haiku{Misschien was je wel,.}{je naam vergeten dacht je}{dat ik dronken was}\\

\haiku{Hij ziet me, staat op,,,.}{drukt mij op een stoel schuift een}{glas voor mij ik drink}\\

\haiku{{\textquoteright} Ibsen bedoelde.}{de quarantainevlag die}{aan den voormast hing}\\

\haiku{{\textquoteleft}Laat de deur open, ik!}{moet er straks zijn en wil niet}{in de stank zitten}\\

\haiku{Met kleurenblindheid.}{kon hij zich als dokter niet}{verontschuldigen}\\

\haiku{{\textquoteright} Werkelijk, Bruce;}{kon bijna niet uit de boot}{en op weg komen}\\

\haiku{Talman laadde hem.}{in den eersten draagstoel dien}{zij tegenkwamen}\\

\haiku{Als je zoo de pest,?}{aan schepen hebt waarom zit}{je er dan nog op}\\

\haiku{{\textquoteright} {\textquoteleft}Zoo zie je, dat de,{\textquoteright}.}{wal niets dan ellende geeft}{triomfeerde Young}\\

\haiku{Hij was congestieus, '.}{eenige malens jaars liet}{hij zich aderlaten}\\

\haiku{Neem een bad, droog je,,.}{af schiet een kimono aan}{vijf minuten werk}\\

\haiku{Waarom mocht zelfs de?}{nacht niet in vredesnaam stil}{en leeg en zwart zijn}\\

\haiku{Zij schoven vlug hun,,.}{bank in sloegen de bladzij}{op wezen elkaar}\\

\subsection{Uit: Het verboden rijk}

\haiku{Het hoofd verlangde.}{toegang tot den goeverneur}{Antonio Farria}\\

\haiku{Met zijn helm schepte.}{hij wat water uit den poel}{en koelde zijn hoofd}\\

\haiku{In Malakka zou.}{men ons spottend ontvangen}{en triomfeeren}\\

\haiku{Hij bouwde een paar -.}{forten en loodsen kerken}{kwamen er vanzelf}\\

\haiku{Macao lag halfweg -.}{Malakka Japan aan een}{beschutte reede}\\

\haiku{Telkens kwamen nog.}{een paar menschen of een paar}{vaten de plank over}\\

\haiku{Er werd weinig meer,,.}{gesproken de vader leed}{maar klaagde niet meer}\\

\haiku{De krekels gingen.}{te keer alsof zij levend}{geroosterd werden}\\

\haiku{{\textquoteright} De dominikaan.}{verdween en liet hem hijgend}{en vloekend achter}\\

\haiku{De vader ging op.}{in het leed van den tot zoon}{gewenschten minnaar}\\

\haiku{Hij zat alleen voor,.}{zijn middagdisch daarna zond}{hij om zijn dochter}\\

\haiku{Moedeloos liet zij.}{zich neerglijden tegen de}{harde vensterbank}\\

\haiku{hij zonk achterover,.}{en de wijn vloeide over zijn}{laarzen op den vloer}\\

\haiku{III Behoedzaam sloop.}{Campos de trappen op en}{stond stil voor de deur}\\

\haiku{Maar de sporen van:}{den zwaren last waren niet}{te verwijderen}\\

\haiku{{\textquoteleft}Mislukt, de vogel,.}{gevlogen ik bijna in}{de kooi gevangen}\\

\haiku{Hij leidde hem eerst,:}{om het lijk liet hem toen los}{en zeide kortaf}\\

\haiku{Vroeger brachten de.}{geminachte ontdekkers}{hulde aan het hof}\\

\haiku{De toren Belem.}{zag hij aan voor zijn vader}{die daar maar staan bleef}\\

\haiku{II Den anderen,.}{morgen vroeg was hij weer aan}{dek de zee was leeg}\\

\haiku{het schip zonder hem.}{te laten vertrekken en}{aan wal te springen}\\

\haiku{Een honderd meter,,}{misschien drong ik erin door}{verder kon ik niet}\\

\haiku{Het was mijn eigen.}{gedaante gezien in een}{verweerde spiegel}\\

\haiku{Het duurde lang, maar.}{Pilar kende ook een kruid}{dat den slaap verdreef}\\

\haiku{Toch draalde zij nog,.}{opeens was er een groote rust}{over haar gekomen}\\

\haiku{Maar voor het klooster.}{keerde zij om en liep de}{Chineesche wijk in}\\

\haiku{Die hoorde Pilar,:}{ook maar na een paar dagen}{wist zij wat het was}\\

\haiku{zij verlangde er,.}{geen liefde voor zij wilde}{onaangeroerd zijn}\\

\haiku{Beiden bezaten.}{niet de gevoelens die haar}{konden ontroeren}\\

\haiku{de beide vleugels,.}{brandden het middenstuk was}{nog onaangetast}\\

\haiku{De regeering had de;}{toegangen tot het klooster}{laten bezetten}\\

\haiku{De vrouw - wonderlijk! -;}{voelde niet dat ik leefde}{in het schimmenrijk}\\

\haiku{{\textquoteright} Heel lang leunde ik.}{tegen een stam en kwam laat}{in den nacht naar huis}\\

\haiku{een violist dien.}{ik in mijn goeden tijd in}{Brighton had gehoord}\\

\haiku{Hij ging beschaamd en.}{levensmoe met het gezicht}{naar den muur liggen}\\

\haiku{Eindelijk drong het,.}{door toen vroeg hij waar hij dat}{aan te danken had}\\

\haiku{{\textquoteleft}En gij, beken ook,.}{dan hebben we alles wat}{we weten moeten}\\

\haiku{Wilde men hem door?}{goede behandeling tot}{verraad bewegen}\\

\haiku{Maar een blad van de.}{Hesperiden-tuin was}{gekreukt en bevlekt}\\

\haiku{Metelho stak het hoofd,.}{buiten de draagstoel kwam toen}{heelemaal overeind}\\

\haiku{Gisteravond bij het,.}{donker wierp ik deze steen}{aan deze doek uit}\\

\haiku{Ik stond stil voor het,.}{steenen tuinhuis te rusten er}{was wat koelte daar}\\

\haiku{Ik wist het niet en,....}{wilde het niet weten want}{als ik dat ook wist}\\

\haiku{En toen kwam het ook.}{als ik op wacht zat met de}{seinkap om mijn hoofd}\\

\haiku{Het loopen was ik,:}{ontwend ik was geworden}{als de anderen}\\

\haiku{Zij waken over de,.}{schepen zooals anderen over}{het heil der zielen}\\

\haiku{Nu stieten wij op,.}{elkaar als de wagons van}{een geremde trein}\\

\haiku{Het was zoo hoog en.}{de donkergladde steenen van}{den ingang noodden}\\

\haiku{{\textquoteleft}Als je meer van die,.}{stukken hebt zal ik ze wel}{voor je wisselen}\\

\haiku{Hij bleef stilstaan, het,,.}{musket dat hij als knots wou}{gebruiken in rust}\\

\haiku{Zij zonk haast op de.}{knie\"en toen zij zag wat uit}{hem geworden was}\\

\haiku{hij daarbuiten was,.}{en in het donker zat hield}{hij op te bestaan}\\

\haiku{waarom ik hier toch.}{was en wat dat allemaal}{te beteekenen had}\\

\haiku{Ik dacht over mijn angst,,:}{van vroeger ik verbaasde}{mij ik vroeg mij af}\\

\section{Conny Sluysmans}

\subsection{Uit: Nog kans op de hemel}

\haiku{Hij stelde me voor.}{eerst ergens te gaan eten en}{ik knikte gretig}\\

\haiku{Het was waar, ik had.}{Hem nooit opzettelijk aan}{het kruis geslagen}\\

\haiku{Als je morgen om.}{zes uur langs de garage}{komt kun je hem zien}\\

\haiku{Voor mijn gevoel staan.}{alle koosnaampjes daarbij}{ver in de schaduw}\\

\haiku{Het was een van de.}{episodes die mijn leven}{hadden getekend}\\

\haiku{dank zij mijn eigen!}{veroveringstactiek ten}{opzichte van hem}\\

\haiku{Dat was dus dat. - Kom?}{je vanmiddag een beetje}{bij me op bezoek}\\

\haiku{Het hele orkest.}{van mijn gevoelens zette}{gelijktijdig in}\\

\haiku{dat zij het bestaan.}{van die andere Sonja}{niet meer nodig vond}\\

\haiku{Die nacht wilde ik,,.}{meer nog dan de vorige}{maal de zijne zijn}\\

\haiku{Waarom vindt u het?}{zo moeilijk die promotie}{te accepteren}\\

\haiku{Mijn auto is snel.}{genoeg om je heel vaak te}{komen bezoeken}\\

\haiku{Je kunt toch niet aan,.}{het leven ontkomen had}{mijn moeder gezegd}\\

\haiku{Niet als aan iemand.}{met wie ik in de toekomst}{te doen zou hebben}\\

\haiku{Ik heb een schat van.}{een vader en een zusje}{waar ik veel van hou}\\

\haiku{Wij stonden daar maar.}{tegenover elkaar en wij}{zeiden heel lang niets}\\

\haiku{- je weet best dat het,}{niet om dat lopen gaat maar}{ik kom niet No\"el}\\

\haiku{- je weet best dat het,}{niet om dat lopen gaat maar}{ik kom niet No\"el}\\

\haiku{Het was een tasten.}{naar de kwalificatie}{van onze liefde}\\

\haiku{Ik was geboren....}{onder het sterrebeeld van}{de Leeuw Nee No\"el}\\

\haiku{Jij bent een vrouw voor.}{\'e\'en man en naar die man zul}{je blijven zoeken}\\

\haiku{Goed, ik had vrouwen.}{gekend die plooibaar waren}{tot het uiterste}\\

\haiku{Tot zes uur wist ik,.}{mij te beheersen toen was}{het proces voorbij}\\

\haiku{- Dus No\"el is toch,! -?}{teruggekomen zei ze}{Teruggekomen}\\

\haiku{N\'og was er in mij.}{het absurde geloof in}{zijn medeleven}\\

\haiku{Hij was niet weg, ik.}{had immers geweten dat}{het een leugen was}\\

\haiku{Ik sloot zorgvuldig.}{het portier van mijn auto}{en ging naar hem toe}\\

\haiku{- Natuurlijk, zei hij,.}{en die wereld kun je ook}{niet veranderen}\\

\section{Alie Smeding}

\subsection{Uit: Achter het anker}

\haiku{Ze schortte de rok.}{wat op en plaatste de voeten}{op het zettelbord}\\

\haiku{{\textquoteright} Kruiselings spande,.}{ze de armen over de borst}{lachte uitdagend}\\

\haiku{{\textquoteright} Een gemelijke,.}{glimlach trok om zijn mond en}{zijn stem klonk stroever}\\

\haiku{Steven Roos en zijn,.}{vrouw zaten breed en donker}{vlak voor de tafel}\\

\haiku{Bouk zou ook 'n best, -,,}{wijfje worden zacht en zoo}{zacht en j\'a schuwer}\\

\haiku{{\textquoteright} Beverig kromp hij,...}{in duwde de handen voor}{de oogen en kreunde}\\

\haiku{Een lach wipte door.}{zijn oogen en hij streek de tong}{langs de onderlip}\\

\haiku{{\textquoteleft}Nee Bart, 'k wil je,{\textquoteright}, - {\textquoteleft}!}{z\'o\'o niet mee hebben driftte}{hijdat verl\`ap ik}\\

\haiku{Oh god - zoo'n m\'o\'oie meid,{\textquoteright},.}{het kreunde door zijn kop hij}{neep de  handen}\\

\haiku{er van wezen, 't, '.}{was menschelijk zooals Ate de}{Leeuw ook wels zei}\\

\haiku{Ve'-dikkie, je.}{adem kon je vlak onder de}{neus zien bevriezen}\\

\haiku{{\textquoteright} Rillig en vervreemd.}{stond hij even later in de}{schemerige gang}\\

\haiku{Tjeerd begon verward.}{en haastig zijn relaas over}{hun wedervaren}\\

\haiku{{\textquoteleft}Allooh jong',{\textquoteright} lachte, - {\textquoteleft} ',!}{ze naar hemnou eerst maars}{flink toetasten hoor}\\

\haiku{,{\textquoteright} joolde ze, met be{\^\i}.}{haar handen woelde ze in}{zijn ruige kuifhaar}\\

\haiku{Bleek en plechtig stond.}{de witte maannacht over het}{wijde watervlak}\\

\haiku{{\textquoteright} Suf bleef hij een poos,.}{voor zich uitstaren dan kwamen}{zijn gedachten weer}\\

\haiku{{\textquoteright} Smartelijk keek hij.}{voor zich neer op de vredig}{ruischende zee}\\

\haiku{Pafferig stond haar.}{goorwitte kop onder de}{scheefgezakte hoed}\\

\haiku{Och God, nou - die ik,?}{bedoel die h\^et ze achter}{de elleboog h\`e}\\

\haiku{Heerke, dat hoofd van, -!}{haar dat deed nou toch altijd}{zoo'n  pijn z\'o\'o'n p{\'\i}jn}\\

\haiku{zweeg hij en schamig.}{glipte zijn blik weg uit haar}{oogen toen ze opzag}\\

\haiku{{\textquoteright} Rap praatte hij het,,.}{beweeglijk zijn magere}{wangen donker-rood}\\

\haiku{Jij ben nog 's met ', ',?}{Eefke \^an de rinkelweest}{opn kermis h\`e}\\

\haiku{'n Slimme duvel,,.}{die meid nou maar die neem je}{niet gauw bij de neus}\\

\haiku{{\textquoteleft}Veeg 't af,{\textquoteright} zei hij, - {\textquoteleft}, '.}{je moet niet zoo bijtent}{wordt heelegaar blauw}\\

\haiku{wel, maar daar moet je ',,?}{t nou niet meer over hebben}{Vader begrijp je}\\

\haiku{{\textquoteleft}Wacht 's, nou waren,,...}{ze h\'a\'ast onder Marken h\`e}{daar ginter dat vuur}\\

\haiku{Moet er ook nog 'n ' '?}{eeuwigheid wezen enn}{hel enn duivel}\\

\haiku{zich, een wasemdrop}{spatte hard-tikkend}{van het zoldertje}\\

\haiku{{\textquoteright} vertelde ze, - {\textquoteleft}Van,,,:}{Maarle de Koster weet je}{wel die heb gezegd}\\

\haiku{{\textquoteleft}Ja, ik kom toch 's '!}{effe bij je kijken of}{jet knapjes heb}\\

\haiku{Nou was-t-ie er,...}{af en nou moest ie maar zoo}{gauw mogelijk weg}\\

\haiku{{\textquoteleft}Werken maar, altijd,{\textquoteright},...}{werken hunkerde het in}{hem zong het in hem}\\

\subsection{Uit: Als een bes in een hofje}

\haiku{haar oogen gloeiden nog,.}{van het mijmeren haar oogen}{tintelden diep-in}\\

\haiku{{\textquoteright} In haar hart bonsde,.}{de nijd groot en wild stond het}{verzet in haar op}\\

\haiku{{\textquoteright} De vreugde zette,.}{een beklemming in haar borst}{zij ademde hortend}\\

\haiku{de volle liefde,.}{wezen tenminste als je}{man Nol Franken was}\\

\haiku{au fond heb ik wel {\textquotedblleft}{\textquotedblright}.}{zoo ongeveer wat uidee}{gelieft te noemen}\\

\haiku{{\textquoteleft}Cor,{\textquoteright} met een zachte,.}{bekommering in de stem}{noemde hij haar naam}\\

\haiku{Het was of alle.}{geluiden een oogenblik}{plat bleven liggen}\\

\haiku{{\textquoteright} Maar toen zij zich op,.}{woorden bezon vluchtten haar}{heete gedachten}\\

\haiku{NOG EEN SMALLE STREEP.}{donker zon-goud door de}{zoele zomerlucht}\\

\haiku{{\textquoteleft}Zij was geen flirt, dus,,}{w\`at ze toonde dat was wel}{waar dat meende ze}\\

\haiku{{\textquoteleft}Die aangebeden{\textquoteright},, {\textquoteleft}, '!}{pipa ironiseerde hij}{jat is prachtig}\\

\haiku{Moeder woont nu bij,,.}{ze in d\`at bood hij zelfs aan}{hij is dol op Cor}\\

\haiku{Hij stak glimlachend,.}{een nieuwe sigaret aan}{wachtte op antwoord}\\

\subsection{Uit: Bruggenbouwers}

\haiku{En ver weg gonzen.}{en snorren de drukpersen}{en zetmachines}\\

\haiku{Vaak genoeg is hij,...}{dagen lang weg als hij weer}{zoo'n prutsbaantje heeft}\\

\haiku{Maar weet je wat ik?, {\textquotedblleft}}{jammer vind er is niet veel}{ingekomen op}\\

\haiku{Ik heb nog van de.}{antwoorden gemaakt wat er}{van te maken was}\\

\haiku{En als ze geen van -...}{allen meer te zien zijn dan}{ziet Kaatin ze nog}\\

\haiku{Met kleine oogen tuurt.}{hij in de rook en mompelt}{wat voor zich uit}\\

\haiku{Anne-Cris en.}{Cobie Savrij zitten naast}{elkaar aan tafel}\\

\haiku{{\textquoteleft}Zie je Solwerda, {\textquotedblleft}}{dat vinden de lui mooi dat}{je er zelf bent voor}\\

\haiku{{\textquoteright} Taco kan alleen...}{maar glimlachend over Onno}{Krabbeel heenkijken}\\

\haiku{{\textquoteright} Terloops kijkt hij naar.}{de mannen en vrouwen die}{hem passeeren}\\

\haiku{{\textquoteleft}Voil\`a{\textquoteright}, denkt hij, {\textquoteleft}maar zoo,...?}{hebben Crijna en ik nooit}{gekeken en n\'u}\\

\haiku{En zijn Vader zat.}{in de werkplaats achter de}{groote klokkenwinkel}\\

\haiku{Hij gluurde door een.}{klein dik glaasje dat hij vlak}{voor zijn eene oog hield}\\

\haiku{Solwerda heeft dat.}{nog zoo uitgeplozen in}{zijn hoofdartikel}\\

\haiku{Ze willen alleen.}{maar hun eigen gedachten}{in de krant hebben}\\

\haiku{Een krot!, jouw mensch, die,:}{nieuwe mensch  van jou hij}{woont in een villa}\\

\haiku{Hier is het huisje,,.}{van Lizelotte Buun een}{klepdeur een zwart raam}\\

\haiku{{\textquoteright} Er hangt nog rijp aan.}{de boomen en de takken}{lijken van zilver}\\

\haiku{En het portlandhuis.}{op het Staalborchplein heeft een}{bedeesd voorkomen}\\

\haiku{De vorige keer,,.}{waarachtig waar Solwerda}{toen was het te rood}\\

\haiku{dat zijn kaken een.}{rammelende beweging}{maken als hij praat}\\

\haiku{Vanmiddag afscheid,...?}{van mijnheer Bergsma op de}{H.B.S. zal ik d\`at maar}\\

\haiku{{\textquoteleft}Dus als u dat goed,?,,?}{vindt meneer komt het u dan}{gelegen meneer}\\

\haiku{In lang heb ik z\'oo{\textquoteright},, {\textquoteleft}.}{niet de Avondster gezien denkt}{Tacoin lang niet}\\

\haiku{Het is Taco of.}{hij dronken geweest is en}{nu weer nuchter wordt}\\

\haiku{Vroeger tuurde ze,.}{hem zoo lang mogelijk na}{zoo lang mogelijk}\\

\haiku{overal duisternis,,.}{sparreschimmen nachtkoelte}{en vaag geruisch}\\

\haiku{Hij is maar gauw in,...}{zijn jas geschoten die jas}{hangt wijd om hem heen}\\

\haiku{{\textquoteleft}Ja{\textquoteright}, denkt hij zonder, {\textquoteleft} '?}{overgangen ik zou immers}{s naar Moeder toe}\\

\haiku{Ze heeft toch nog kleur,.}{op haar wangen en ze heeft}{uitgeruste oogen}\\

\haiku{{\textquoteleft}Ik heb nog aan de.}{deur geluisterd met me oor}{op het sleutelgat}\\

\haiku{Ze kijkt naar hem op,,}{een eenvoudig goedmoedig}{eerlijk gezichtje}\\

\haiku{{\textquoteleft}Toen je indertijd{\textquoteright},,}{bij mij kwam Crijna wendt haar}{oogen niet van hem af}\\

\haiku{Hij leunt nog even met.}{zijn hoofd tegen de deurpost}{aan en sluit zijn oogen}\\

\haiku{{\textquoteleft}Is er copij voor?}{de fuljeton om in het}{voren te zetten}\\

\haiku{{\textquoteright} Bos wil toch lachen,,.}{wil hartelijk lachen het}{klinkt zoo armoedig}\\

\haiku{Soo net ging u vriend,?,.}{ook voorbij sag u hem niet}{meneer Altenstadt}\\

\haiku{Je neemt wel wat te.}{veel en te eenzijdig over}{uit die N.S.B.-lectuur}\\

\haiku{{\textquoteright}, vraagt hij op de man, {\textquoteleft}?,...?}{afwat hebben die menschen}{wat wordt er verteld}\\

\haiku{Op het oogenblik}{bent u bezig mij ergens}{in te betrekken}\\

\haiku{Altenstadt die belt,?}{mij immers telkens op met}{een verdraaide stem}\\

\haiku{Waren die menschen?}{in de gemeenteraad nu}{zoo gauw uitgepraat}\\

\haiku{{\textquoteright} Van tijd tot tijd zal.}{ze onder de maaltijd iets}{dergelijks zeggen}\\

\haiku{het zoo echt als de?}{auteurs een dominee in}{de maling nemen}\\

\haiku{op de bovenste.}{of onderste tree van een}{wenteltrap desnoods}\\

\haiku{Hij kijkt het avondblad.}{door zonder er veel van in}{zich op te nemen}\\

\haiku{Hij legt de courant}{neer en omvat zacht met wijd}{gespreide handen}\\

\haiku{En de nikkelen.}{stang van de leeslamp voor hem}{krimpt in en zet uit}\\

\haiku{Maar dat is niet het.}{allervreemdste wat hem op}{weg naar huis overkomt}\\

\haiku{Ze kwam ergens uit,...}{de schaduw vandaan uit de}{schaduw van een muur}\\

\haiku{Heeft Anne-Cris?}{je naar me toegestuurd om}{dat uit te visschen}\\

\haiku{{\textquoteright}, zegt hij bij zichzelf, {\textquoteleft} -?}{ik ik heb dat  toch wel}{eerder meegemaakt}\\

\haiku{{\textquoteleft}O ja, nu kijk je,?,.}{gepast-verdraagzaam}{h\`e net als Wedzieg}\\

\haiku{Wat wordt ze ook al,?,,!}{anders h\`e Anne-Cris}{verb\'azingwekkend}\\

\haiku{Hij zit ook voorover,,.}{er hangt een sliert haar voor zijn}{oogen zijn oogen gloeien}\\

\haiku{, was dit wat, dat   -...?}{je dat je in je stille}{tijd gekregen hebt}\\

\haiku{Ik heb een die bij,,.}{me bleef alleen gelaten}{al ging ik niet weg}\\

\haiku{Duizenden menschen -...}{slachten duizenden menschen}{af in deze tijd}\\

\haiku{En h\`eb je ooit wat,?}{geweigerd in die tijd t\'oen}{ze je nog wat vroeg}\\

\haiku{Zoo zag ik het ook -.}{zoo zal alles gebeuren}{wat je gedacht hebt}\\

\haiku{Omdat{\textquoteright}, raadt Imkje, {\textquoteleft}.}{Mareeshij de vrouw van die}{vriend ambieerde}\\

\haiku{En hij leert de  !}{kinderen niet eens waarom}{ze gelooven moeten}\\

\haiku{Dan is het weer of...}{hij zichzelf verdedigen}{moet ergens tegen}\\

\haiku{Taco neemt zijn pijp:}{een oogenblik uit zijn mond}{en zegt in zichzelf}\\

\haiku{{\textquoteleft}Als Anne-Cris,.}{vroeger maar terug gepraat}{had op wat ik zei}\\

\haiku{Ze ontwijkt Taco's,... {\textquoteleft}}{blik en kijkt toch naar hem kijkt}{toch telkens naar hem}\\

\haiku{Ze zitten ook in.}{de goudbruine salon bij}{burgemeester Heinz}\\

\haiku{Hij ziet de zachte,:}{winter daarbij de winter}{buiten de stadspoort}\\

\haiku{{\textquoteleft}Ik zal haar wakker - -...{\textquoteright}}{kijken ik wil praten wil}{het uit-praten}\\

\haiku{Er is zelfs wat zwoels.}{in het fijne geknister}{van haar pyama}\\

\haiku{Ik heb een die bij,.}{mij bleef alleen gelaten}{al ging ik niet weg}\\

\haiku{Later is het of:}{hij repeteert wat hij in}{Grensted gelezen heeft}\\

\haiku{God-god, al die...,?}{jaren wat is er met ons}{wat is er met jou}\\

\haiku{Hoe lang denk je hier,, -?}{nog mee door te gaan met die}{houding met alles}\\

\haiku{Dan is het of hij,.}{haar slaan wil of hij zich in}{woede op haar werpt}\\

\haiku{J{\'\i}j mocht haar nooit erg -..{\textquoteright}}{idioot zooals ze beslag lei}{op Anne-Cris}\\

\haiku{Ze trekt een effen.}{gezicht en speelt argeloos}{met haar servet-ring}\\

\haiku{{\textquoteleft}Als ik iets in dat{\textquoteright},, {\textquoteleft}.}{opzicht weet zegt hij wrangdan}{ligt dat niet aan jou}\\

\haiku{{\textquoteleft}Het moet allemaal, -.}{nog heel anders worden met}{ons heel heel anders}\\

\haiku{Ze loopt zoetjes voort.}{in de gang zonder nog naar}{hem om te kijken}\\

\haiku{Maar we kunnen die...}{foto's ook afzonderlijk}{verkrijgbaar stellen}\\

\haiku{{\textquoteright} Bos maakt ook nog een,.}{praatje voor de schijn neemt hij}{een paar clich\'e's mee}\\

\haiku{{\textquoteright} Maar dan merkt hij toch,.}{ook dat Jurgen bij hem staat}{en dringender praat}\\

\haiku{En ik kan het werk,.}{van Jozefien minstens zoo}{goed doen als zij zelf}\\

\haiku{Daarom, als u er,.}{voor voelt u kunt gerust wat}{laten vallen ook}\\

\haiku{{\textquoteright} En Taco zou de.}{jongen graag een eind van zich}{af willen duwen}\\

\haiku{{\textquoteright} Cobie kijkt op om.}{te zien of Taco er ook}{om glimlachen moet}\\

\haiku{Je schaadt mij in mijn,.}{goede naam en dat zal je}{merken in je zaak}\\

\haiku{Dat is altijd zoo,:}{als hij alleen achterblijft}{hij kent dat zoo goed}\\

\haiku{En de grijns van het.}{Satzuma-aapje zou hij}{stuk willen gooien}\\

\haiku{Och, nou ja, het moest,.}{er toch van komen het ligt}{er nu eenmaal toe}\\

\haiku{{\textquoteleft}Je moet me nou maar,,.}{alleen laten Cobie laat}{me nou maar alleen}\\

\haiku{, het doet er nou toch,!}{niks meer toe het is nou toch}{voor alles te laat}\\

\haiku{Dat Marees weg is -,.}{dat Imkje dit nu heeft dat}{drukt mij als een schuld}\\

\haiku{Ik heb me er vaak,}{in verdiept wat je toch deed}{het licht brandde hier}\\

\haiku{Nu ziet ze er uit:}{of iets of iemand haar heel}{erg beschadigd heeft}\\

\haiku{Ze bijt zoo heftig,.}{in haar lip of ze haar lip}{er afbijten wil}\\

\haiku{{\textquoteright} Wedzieg glimlacht op.}{zijn rustige langzame}{boeren-manier}\\

\haiku{{\textquoteleft}Geen vraag mocht ik daar!}{op die houseparty in}{het openbaar stellen}\\

\haiku{v\'oor de vrede moest,.}{zijn d\`an de menschen van de}{volkomen liefde}\\

\haiku{Hier en daar, ver het,,...}{land in staat een verkleumd huis}{een kreupele boom}\\

\haiku{{\textquoteleft}Is het toeval dat...?}{die man uit Stritz over Marees}{begon te zwammen}\\

\haiku{Uit de verte kijkt,,.}{hij ook door schemer en mist}{heen naar Crijna's huis}\\

\haiku{{\textquoteright} Wat verwonderd merkt,.}{hij dat hij weer in zijn stoel}{bij de kachel zit}\\

\haiku{Ze bedenkt zich, haar,.}{oogen gaan wijder open rare}{opengescheurde oogen}\\

\haiku{Ja, maar ik had soms,...}{dag aan dag hoofdpijn ik viel}{herhaaldelijk flauw}\\

\haiku{Je liefkoosde ze,.}{haast toen je ze in het vloei}{onder je arm hield}\\

\haiku{Ze wist de naam van,.}{die dame wist allerlei}{bizonderheden}\\

\haiku{Cobie en Weigel -.}{dat waren allebei mijn}{troeven tegen jou}\\

\haiku{Nou begint het weer{\textquoteright},, {\textquoteleft} -...}{hijgt zedat benauwde dat}{was vanmiddag ook}\\

\haiku{Ze tast naar iets dat -.}{vlak bij haar gezicht was en}{het is er niet meer}\\

\haiku{Dan breng ik het naar {\textquotedblleft}{\textquotedblright},.}{je collega vanBosch en}{Ven die neemt het wel}\\

\haiku{En een vreemd gevoel,.}{bekruipt hem iets van warmte}{en verteedering}\\

\haiku{Is er... is er wat...}{is er iemand werkelijk}{iets aan gelegen}\\

\haiku{Als ze weg is, blijft.}{hij nog een heele poos op}{de waterpoort staan}\\

\haiku{Maar Gijs Bard trekt zich.}{daar voor het oogenblik geen}{syllabe van aan}\\

\haiku{Hij lijkt weer min of.}{meer op die oue montere}{Gijs Bard van vroeger}\\

\haiku{{\textquoteleft}Vader{\textquoteright}, zegt Us, {\textquoteleft}we!}{hebben een wegenkaartje}{gevonden voor je}\\

\haiku{{\textquoteleft}Heeft Anne-Cris{\textquoteright},.}{altijd de hand in bepaalt}{Taco heimelijk}\\

\haiku{, met al die menschen,?}{over de heele wereld die}{openstaan voor Oxford}\\

\haiku{{\textquoteright} Maar op een morgen,,.}{voor kantoortijd hoort hij haast}{kregel Kaatin aan}\\

\haiku{Kaatin kijkt glunder,.}{op hem neer onder de rand}{van zijn deukhoed uit}\\

\haiku{{\textquoteright}, vraagt hij zich verbluft, {\textquoteleft}?}{afhoe ter wereld krijgt Look}{d\`at nou voor elkaar}\\

\haiku{Moeder is... nou - het, -}{kan niks schelen maar het is}{toch zoo niet zoo bar}\\

\haiku{En dat alles bij{\textquoteright},, {\textquoteleft}.}{elkaar denkt Tacois dus}{een houseparty}\\

\haiku{{\textquoteleft}Ik wou alleen maar - {\textquotedblleft}{\textquotedblright},!}{beste kerel tegen je}{zeggen anders niet}\\

\haiku{De meeste luitjes -.}{zijn tot de nok toe vol van}{hun eigen zonden}\\

\haiku{{\textquoteleft}Je kan het immers?}{niet doorgronden met je goeie}{normale verstand}\\

\haiku{{\textquoteleft}Zoo Louwtje{\textquoteright}, zegt hij, {\textquoteleft}?}{heb je het nog al naar je}{zin tegenwoordig}\\

\haiku{En soms is het, of,,... {\textquoteleft}}{een geluid daar als op de}{teenen heen en weer sluipt}\\

\haiku{Ja, misschien of we '...}{weers samen op reis gaan}{in de vacantie}\\

\haiku{{\textquoteleft}Alle duivels, op.}{dat verl\`angen kom je nou}{ook telkens terug}\\

\haiku{Ik voelde mij toch -:}{nog altijd stukken beter}{dan Anne-Cris}\\

\haiku{Hij knaagt een beetje.}{op zijn onderlip en hij}{loopt niet al te vlug}\\

\haiku{{\textquoteleft}Veel te erg, het is -{\textquoteright},.}{veel te erg wat aardig haar}{stem blijft er mat bij}\\

\haiku{Want die groote sterke.}{aandrift is Taco nu te}{machtig geworden}\\

\subsection{Uit: De domineesvrouw van Blankenstein}

\haiku{{\textquoteleft}En het hoefde niet{\textquoteright},, {\textquoteleft}.}{denkt Djoeke vaaghet was niet}{eens noodzakelijk}\\

\haiku{Er gaat een lange,}{schrale vrouw langs haar heen een}{gehavend streng-oud}\\

\haiku{Je sleepte mij dwars,.}{door de woede heen je nam}{mij mee naar de wraak}\\

\haiku{{\textquoteleft}Het heerlijk ambt{\textquoteright}, zei,, {\textquoteleft}.}{hij stil als uit de verte}{het ambt der ambten}\\

\haiku{Djoeke slaat rechts af,,.}{een laantje in daar is het}{witte pension}\\

\haiku{Heeft ooit iemand hier,?}{het uitzicht bewonderd en}{de stilte geroemd}\\

\haiku{Een oogenblik...{\textquoteright} {\textquoteleft}Voor{\textquoteright},, {\textquoteleft}!}{mij overdrijft Djoekewas het}{een menschen-leeftijd}\\

\haiku{Djoeke kent dat van,,.}{hem als hij lang gezwegen}{heeft  praat hij zoo}\\

\haiku{Zinnend op de een.}{of andere plicht komt zij}{de huiskamer in}\\

\haiku{Misschien zit er meer,,.}{goeds in die jongen dan in}{jou of in mij Veen}\\

\haiku{Drie paden loopen,.}{er op het plein uit Aage}{kiest het middelste}\\

\haiku{{\textquoteleft}Kind, ik ben hier nu,,...}{pas dit is de eerste dag}{niet zoo doordrijven}\\

\haiku{Djoeke tuurt naar dit,.}{doffe en verdoofde zij}{tuurt op een raadsel}\\

\haiku{{\textquoteright} {\textquoteleft}Kom{\textquoteright}, zegt Aage dan, {\textquoteleft},.}{van vlakbijwe moeten naar}{huis het is etenstijd}\\

\haiku{Het is of elke.}{bloem met een uitgestrekte}{hals het pad afkijkt}\\

\haiku{Hij schiep het bosch en.}{strooide een blonde schemer}{over de paden heen}\\

\haiku{{\textquoteleft}Ben je zelf ook niet...?}{door de goedheid gegrepen}{en overeind gezet}\\

\haiku{{\textquoteleft}Voor Ties{\textquoteright}, denkt zij, zij,.}{schikt de bloemen zorgzaam een}{mooie boeket wordt het}\\

\haiku{Wacht nog een beetje,,.}{man de duivel maakt je al}{een mooie strik gereed}\\

\haiku{Over een klein stukje -, -.}{van de wereld een scherf en}{meer niet trekt een grijns}\\

\haiku{{\textquoteleft}Weet je nog, dat wij,?}{hier samen hebben zitten}{schreien op een keer}\\

\haiku{Hij veegt stilletjes,.}{zijn oogen af hij doet zijn best}{om te glimlachen}\\

\haiku{En Aage legt de.}{handen vast en dwingend op}{de vuisten van Ties}\\

\haiku{Bedremmeld ligt Ties,.}{daar in zijn bedstee en zwak}{pinkt hij met de oogen}\\

\haiku{{\textquoteright} {\textquoteleft}Excuseer mij even{\textquoteright},,.}{fluistert zij tegen Djoeke}{en loopt vlug vooruit}\\

\haiku{Zij begint daar op.}{het boschpad al een gesprek}{met Aage over Thea}\\

\haiku{Neen, Heile heeft geen,.}{zin om te praten hoog kijkt}{Heile over haar heen}\\

\haiku{{\textquoteright}, grijnst dat geringe,.}{en het verheft zich het rijst}{hoog boven haar uit}\\

\haiku{Hard loopt ze op huis,,.}{toe het valt niet mee het is}{een heel eind naar huis}\\

\haiku{De bleekheid van zijn.}{wangen lijkt zich ook uit te}{spreiden over zijn oogen}\\

\haiku{{\textquoteright} Hij strijkt met de rug -.}{van zijn hand over de oogen of}{hij tijd wil winnen}\\

\haiku{Titels van boeken,.}{opgeven een enkele}{maal iets uitleggen}\\

\haiku{{\textquoteright} {\textquoteleft}Dat moest een man nooit{\textquoteright},, {\textquoteleft},!}{doen keurt Aage afeen vrouw}{ook niet natuurlijk}\\

\haiku{De rimpels in zijn,.}{gezicht spannen zich over een}{zorg heen een droefheid}\\

\haiku{{\textquoteleft}Je hamer en je,.}{spijkers deugen er niet voor}{Djoeke Veenema}\\

\haiku{Dit is van alles{\textquoteright},, {\textquoteleft}.}{het mooiste valt Djoeke in}{het gaan naar de kerk}\\

\haiku{Hij ziet een paar groote,...}{verschrikte vrouwenoogen oogen}{vol pijn en liefde}\\

\haiku{{\textquoteleft}Heer{\textquoteright}, bidt zij met open, {\textquoteleft}.}{oogenlaat ons hart zoo wijd als}{de aarde worden}\\

\haiku{, alles wuift aan haar,,,.}{zij knikt en zij wuift en als}{zij zit wuift zij nog}\\

\haiku{Hij heft de handen, -.}{op en spreekt het votum uit}{een gebed blijft hij}\\

\haiku{Men hoort het suizen.}{van de zomerwind langs de}{hooge spitse ramen}\\

\haiku{{\textquoteleft}Alle drie zijn zij{\textquoteright},, {\textquoteleft}.}{al lang dood zegt hijen zij}{leven toch nog zoo}\\

\haiku{Soms kneep ik mijn oogen:}{stijf dicht en dan zei ik een}{paar maal achtereen}\\

\haiku{{\textquoteleft}Mijn Moeder die is,,.}{vroeg getrouwd Aage die is}{lang eenzaam geweest}\\

\haiku{{\textquoteleft}Zou je dan eerst niet,?}{die blauwe boon op tafel}{leggen Jan Hendrik}\\

\haiku{Jan Hendrik was \'een.}{van zijn beste aardigste}{catechisanten}\\

\haiku{Hij zwijgt al weer, er,!}{flikkert iets door hem heen er}{gaat hem een licht op}\\

\haiku{De Heere God knielt,:}{op de aarde neer en schept}{het madeliefje}\\

\haiku{Wel mensch, ze stond me,!}{toch zoo klaar voor oogen in mijn}{binnenst schreide ik}\\

\haiku{Ze wil... wil pijn doen,.}{omdat het van binnen zoo}{schreien kan bij haar}\\

\haiku{En wat zal ik in,?}{de kerk vinden dat ik in}{het bosch niet aantref}\\

\haiku{Wietze-zelf zit.}{onder de breede schaduw}{van zijn eikeboom}\\

\haiku{Hij glimlacht toch al -.}{weer maar het is een glimlach}{met wat strengs er in}\\

\haiku{{\textquoteleft}Luister, het mooiste,,.}{wijf van de heele omtrek}{dat ben jij Elsie}\\

\haiku{{\textquoteleft}Zeg Elsie, bidt ons?}{kind wel iedere avond}{voor zij slapen gaat}\\

\haiku{De stille zwarte.}{dennen en de bleeke sterren}{verdwijnen ineens}\\

\haiku{Jenneke van Heist,.}{laat het gordijn zakken en}{steekt het lampje op}\\

\haiku{zij steken, zij haalt}{zich Aage's blik voor de geest}{en Aage's glimlach}\\

\haiku{ik aanstonds laten{\textquoteright},, {\textquoteleft}.}{zien bepaalt hij zakelijk}{als Vader er is}\\

\haiku{{\textquoteleft}Oh-ho{\textquoteright}, Heile hoest, {\textquoteleft}!,!}{van de lachwat {\`\i}s het weer}{mooi wat is het mooi}\\

\haiku{Hij legt zijn hand open,,.}{voor Rein neer een betuiging}{is dat een vraag ook}\\

\haiku{of ik stel me voor,....}{dat we praten over dat we}{ergens over praten}\\

\haiku{{\textquoteright} Aage drukt zijn wang.}{tegen het gladde bruine}{kindergezicht aan}\\

\haiku{{\textquoteright} Aanhalig klinkt dat.,.}{Zij graven klimop-planten}{uit zij drinken thee}\\

\haiku{{\textquoteright} Een paar simpele.}{woorden van Thea trekken haar}{weer in de kamer}\\

\haiku{Op jou{\textquoteright}, bekent Thea, {\textquoteleft}.}{luchtigis de notaris}{bijzonder gesteld}\\

\haiku{{\textquoteright}, biecht ze luchtig, {\textquoteleft}och,,.}{ja ik heb van die buien}{dan ga ik flirten}\\

\haiku{{\textquoteright} Koud wordt ze daarbij,.}{in haar gezicht dat koude}{grijpt haar bij de keel}\\

\haiku{maar smartelijke,.}{dingen een schaamtevolle}{biecht leggen zij af}\\

\haiku{* * * De week rust uit in:}{de heldere stilte van}{de Zaterdagavond}\\

\haiku{{\textquoteleft}Djoeke{\textquoteright}, fluistert hij, {\textquoteleft}...{\textquoteright},?}{Djoeke Meer verstaat zij niet}{maar hoe klinkt dat toch}\\

\haiku{Nu, zoeken, dat is,...}{een vermoeiend werk en zwaar}{als het voor niets is}\\

\haiku{Zij staan bijvoorbeeld,.}{aan een golvend korenveld}{de vrouw en haar man}\\

\haiku{{\textquoteleft}Schrale aren, de boer,.}{is te zuinig geweest met}{zijn kalk zijn phosphor}\\

\haiku{{\textquoteright} De notaris kijkt.}{Djoeke aan of hij van haar}{een antwoord verwacht}\\

\haiku{Zij merkt nu pas dat,.}{ze doorgeloopen zijn de}{notaris en zij}\\

\haiku{Zij wikkelt de stof,.}{van het karton en spreidt die}{uit over de handen}\\

\haiku{Langs de enkele,:}{huizen op het plein loopen}{maar een paar menschen}\\

\haiku{Op een zonderling.}{afwerende manier maakt}{zij het pakje open}\\

\haiku{{\textquoteright} {\textquoteleft}En morgen samen{\textquoteright},, {\textquoteleft}.}{in de kerk valt Djoeke nog}{insamen bidden}\\

\haiku{{\textquoteright} Als er huisjes in,.}{het zicht komen ademt zij weer}{gemakkelijker}\\

\haiku{Zij wrijft zich in de,.}{handen er is nog wel meer}{dat zij weten wil}\\

\haiku{Daar is dag dien God:}{een onsterfelijke ziel}{gegeven  heeft}\\

\haiku{Als Djoeke er aan,.}{terugdenkt is het of zij}{wegzinkt in een droom}\\

\haiku{Zij staat met de rug,!}{naar Aage toe zij weet toch}{precies wat hij doet}\\

\haiku{Hij denkt aan  een,.}{fout van Tjisse hij ziet een}{tekort van zichzelf}\\

\haiku{tegemoetkomend,!}{hartelijk v\'aderlijk zou}{men kunnen zeggen}\\

\haiku{Nu wil zij heengaan,,,!}{neen zij luistert en kijkt toe}{zij blijft toch nog staan}\\

\haiku{die pepert het de,.}{menschen w\`el in kerken vol}{volk trekt hij dan ook}\\

\haiku{Je bent de oudste,,!}{man van het dorp Tjisse je}{weet wat ik bedoel}\\

\haiku{{\textquoteright} {\textquoteleft}Die heeft{\textquoteright}, haalt Tjisse, {\textquoteleft}.}{dan nog mompelend aanzal}{gegeven worden}\\

\haiku{Tjisse wordt bang, maar,}{hij haat zijn bangheid zooals hij}{ook zijn beschaamd heet}\\

\haiku{je zult dit willen,:}{vergeten maar je zult het}{moeten onthouden}\\

\haiku{Zij hadden hem al,!}{vergeten zij hadden hem}{de rug toegedraaid}\\

\haiku{{\textquoteleft}Kijk naar deze fraai{\textquoteright}.}{gevormde hand. En men kijkt}{onwillekeurig}\\

\haiku{{\textquoteleft}Zoo?, eigenaardig{\textquoteright},.}{en dan luistert hij naar iets}{dat niet gezegd wordt}\\

\haiku{Er gaat ook iets langs,.}{haar heen in die vraag dat niet}{gekend wil worden}\\

\haiku{Zij kijkt schuchter op,.}{zij wil het zoo zachtzinnig}{mogelijk zeggen}\\

\haiku{Stoppelig geel haar,,.}{hebben zij hel-blauwe}{oogen roode wangen}\\

\haiku{{\textquoteleft}Kan jij nu nog naar,?}{slechte voorbeelden kijken}{Djoeke Veenema}\\

\haiku{De vrouwen loopen -!}{op schoenen met hooge hakken}{en het zijn dames}\\

\haiku{De zonde zit daar.}{aan de overkant en kijkt haar}{recht in het gezicht}\\

\haiku{Het is waar{\textquoteright}, beseft, {\textquoteleft} -...{\textquoteright}}{ze bang-verbaasddat is}{er ook nog dat ook}\\

\haiku{En hij kijkt ook op -:}{hij oogt naar buiten en het}{uitzicht bekoort hem}\\

\haiku{{\textquoteright} Maar dan is het of.}{er een pijn-plooi in Aage's}{vage glimlach komt}\\

\haiku{{\textquoteleft}Als Thea hier nu zat{\textquoteright},, {\textquoteleft}...}{werpt ze op in zichzelfof}{Mieneke Eiber}\\

\haiku{Als men boven op,.}{die berg staat is het of men}{God aanraken mag}\\

\haiku{Hij maakt met zijn twee.}{groote vuisten een uitbundig}{gebaar in de lucht}\\

\haiku{{\textquoteright} {\textquoteleft}Net zoo goed als ik,{\textquoteright},.}{jou te woord sta Reinbeek zegt}{Aage zonderling}\\

\haiku{{\textquoteright} Er suist iets voorbij,,,}{iets grilligs is dat het komt}{terug ook daar suist}\\

\haiku{Roode randen had{\textquoteright},, {\textquoteleft}.}{Aage om de oogen denkt zij}{en wat zag hij wit}\\

\haiku{{\textquoteleft}Ja{\textquoteright}, stemt zij toe, {\textquoteleft}heel...{\textquoteright}.}{erg Maar dan voelt men dat zij}{aan iets anders denkt}\\

\haiku{{\textquoteright}, de gramschap van de.}{Kosteres lijkt plotseling}{te verergeren}\\

\haiku{{\textquoteleft}Hij loopt al een maand,.}{lang met stof op de hoed met}{een knoop van de jas}\\

\haiku{Djoeke praat, zij discht,.}{ware verhalen op zij}{bedenkt verhalen}\\

\haiku{Djoeke wacht tot de,.}{toonladder ten einde is}{dan komt zij binnen}\\

\haiku{{\textquoteleft}Als vrouwen een kind,.}{verwachten schijnen zij die}{trek veel te hebben}\\

\haiku{{\textquoteleft}Goed{\textquoteright}, bewilligt hij, {\textquoteleft}.}{een dezer dagen kom ik}{gewoon aanloopen}\\

\haiku{De kamerdeur staat,,.}{aan en gaat nu langzaam open}{Aage komt binnen}\\

\haiku{En Aage luistert,.}{met een trek van inspanning}{hij antwoordt verstrooid}\\

\haiku{Och ja, een mensch hoort,.}{en ziet altijd precies wat}{zijn vrees hem ingeeft}\\

\haiku{Zij legt haar hand op,.}{zijn schouder en ziet dat zijn}{voorhoofd vochtig is}\\

\haiku{Maar zijn oogen trekken,.}{haar dichterbij hij drukt het}{gezicht in haar haar}\\

\haiku{Zij maakt licht, sluit de.}{gordijnen en wil moe op}{een stoel gaan zitten}\\

\haiku{Neen, zij zou naar bed,,.}{gaan zij beloofde het nu}{moet ze het ook doen}\\

\haiku{{\textquoteright} Ja, maar haar hand draalt,,.}{haar hand zweeft om de knop heen}{en beroert die niet}\\

\haiku{{\textquoteright} Zij probeert er niet,.}{naar te luisteren maar dat}{moet zij opgeven}\\

\haiku{Ik keek Maria aan, -.}{ik keek naar het leven ik}{hoorde Gods hart slaan}\\

\haiku{{\textquoteleft}Het regent niet meer{\textquoteright},, {\textquoteleft}?,...{\textquoteright}}{merkt Aage ineenshoor je}{het regent niet meer}\\

\haiku{Djoeke kan haar oogen,.}{weer openen het laken weer}{wat terugschuiven}\\

\haiku{Maar Djoeke loopt nog,.}{naar vrouw Wulk toe daar in het}{donker van haar bed}\\

\haiku{HET KAN WEZEN DAT,}{ER IEMAND BIJ AAGE IS}{HET IS MOGELIJK}\\

\haiku{Nu gaan wij aanstonds{\textquoteright},, {\textquoteleft}.}{naar Maria's graf denkt zij}{naar Maria's graf}\\

\haiku{als Aage het niet,!}{over Maria heeft hoort zij toch}{dat hij over haar praat}\\

\haiku{Ze duikt een beetje,.}{ineen ze duikt enkel maar}{een beetje ineen}\\

\haiku{Daar liepen wij vaak '....}{s avonds Ook in de tijd toen}{zij Rein verwachtte}\\

\haiku{De zomer is niet -.}{dood de zomer is naar een}{ander land gegaan}\\

\haiku{Wietze zit onder.}{het gele Boeddhaatje in een}{krakende leunstoel}\\

\haiku{Het is of Wietze,.}{hem beet pakt bij de jasmouw}{dat is toch niet zoo}\\

\haiku{{\textquoteleft}Twaalf gulden in de,.}{week komt er ook nog bij twaalf}{gulden in de week}\\

\haiku{{\textquoteleft}Ik ben nu{\textquoteright}, haspelt, {\textquoteleft}.}{hij met zijn lachende mond}{een rijke mijnheer}\\

\haiku{Jetske Zwart loopt af,.}{en aan Gerreke van Driel}{steekt een bezoek af}\\

\haiku{Nu, {\`\i}k voor mij, {\`\i}k.}{geef mij niet bij voorkeur over}{aan oude droomen}\\

\haiku{hij klimt de trap op,,.}{hij gaat zijn kamer in hij}{loopt in gedachten}\\

\haiku{{\textquoteleft}Neen{\textquoteright}, zegt Djoeke, {\textquoteleft}men,...?}{kan niet weten wat het is}{wie zal het zeggen}\\

\haiku{{\textquoteleft}Dag{\textquoteright}, hij geeft haar een,, {\textquoteleft}{\textquoteright},.}{hand hij buigt zich wat naar haar}{toed\`ag zegt hij weer}\\

\haiku{{\textquoteleft}Ja-ja, dat zal ik,...{\textquoteright},.}{doen dank je dank je wel zij}{fluistert dat bijna}\\

\haiku{{\textquoteright} Maar dan plotseling.}{trekt er wat doezeligs van}{haar gedachten weg}\\

\haiku{Aage heeft blauwe,.}{plekken in het gezicht hij}{ziet er verkleumd uit}\\

\haiku{Wij glijden altijd,.}{met een heele bende de}{hooge brug af bij school}\\

\haiku{Hij vouwt de handen,.}{ineen over de knie hij drukt}{de kin op de borst}\\

\haiku{Een eekhoorntje uit.}{het dorp zorgde al-vast}{voor hazelnoten}\\

\haiku{* * * ~ Dikwijls wendt het,.}{leven zich van Djoeke af}{in deze dagen}\\

\haiku{Nou, Moeder is toch,?}{z\'oo niet dat ze haar eigen}{daar verstoppen zal}\\

\haiku{Maar Mieneke hoort -.}{dat niet zij luistert naar de}{stap van haar Moeder}\\

\haiku{Ze kijkt de kamer,,.}{in haar oogen gloeien blauw vuur}{is er in die oogen}\\

\haiku{Mieneke's handen,:}{knellen niet meer Mieneke's}{oogen staren niet meer}\\

\haiku{Dat ik niet meer zoo,.}{moe zal zijn dat ik nu niet}{meer zoo hoesten moet}\\

\haiku{Afgetrokken gaat,.}{zij door de bleeke dag laat in}{de nacht slaapt zij in}\\

\haiku{{\textquoteright} Dan nadert God haar,.}{ook in een pijn de pijn trekt}{schrijnend door haar borst}\\

\haiku{{\textquoteright}, trekt het schemerig, {\textquoteleft}?}{door haar heenheb ik er nooit}{eerder aan gedacht}\\

\haiku{Japke vischt een paar.}{hazelnoten op uit een}{zak onder haar jurk}\\

\haiku{{\textquoteleft}Als Mevrouw het merkt{\textquoteright},, {\textquoteleft} -.}{voorziet zijgaat ze voort weg}{ben ik weer alleen}\\

\haiku{{\textquoteright} Weer kijkt Japke op.}{met die vreemde helderheid}{in haar zwarte oogen}\\

\haiku{Van wie zijn ze toch{\textquoteright},, {\textquoteleft}?,?}{soest Djoekedie voetsporen}{welk doel hebben zij}\\

\haiku{de groote bladen, als,, {\textquotedblleft}}{ze hem d\'aar willen mij best}{maarDe Kandelaar}\\

\haiku{{\textquoteright} Murman heft een groote,.}{harige vuist op en wijst}{er mee naar Aage}\\

\haiku{Gods uitgestrekte,:}{handen en een glans valt over}{haar gedachten heen}\\

\haiku{Nu staat zij op een,.}{steenen hoogte en de steden}{kijken naar haar om}\\

\haiku{vrouwen met een ring,.}{in de neus mannen met een}{vacht om de heupen}\\

\haiku{Toen hebben wij toch -,:}{om de dokter gestuurd nou}{een zware ziekte}\\

\haiku{{\textquoteright} Riek krabbelt in de,.}{stugge bakkebaardjes plukt}{aan de wenkbrauwen}\\

\haiku{Wit staan de fijne,,.}{dennen daar wit de eiken}{en wit is het dorp}\\

\haiku{Rein springt daverend,.}{de trap af en hij praat met}{een hooge schelle stem}\\

\haiku{Djoeke, help je de,...?}{touwen losmaken vouw jij}{de papieren op}\\

\haiku{{\textquoteleft}Christus is met ons{\textquoteright},, {\textquoteleft}.........{\textquoteright}.}{prevelt zeChristus is met}{ons De schrik laat af}\\

\haiku{Maar Eiber rekt de,,...}{hals beweegt de lippen komt}{een stap naderbij}\\

\haiku{Hoe zoo...?, wij spelen,.}{hier open kaart wij nemen hier}{geen blad voor de mond}\\

\haiku{En een oogenblik.}{is het of Djoeke wegzinkt}{in een  diepte}\\

\haiku{De lichtschijven  ,,.}{wentelen weer rond maar ze}{zijn dunner vager}\\

\haiku{Zij stelt vragen op,.}{in haar gedachten zij is}{bang-nieuwsgierig}\\

\haiku{{\textquoteleft}Ja{\textquoteright}, prevelt zij, {\textquoteleft}wij...{\textquoteright}:}{moeten naar huis Droomerig}{tuurt ze voor zich uit}\\

\haiku{En Rein kijkt zwijgend,.}{naar haar gebogen schouders}{haar gebogen hoofd}\\

\haiku{In haar borstwering,.}{van sjaals doeken en dekens}{kijkt Djoeke er naar}\\

\haiku{De rijkdom van de.}{geheele wereld ligt er}{in opgesloten}\\

\haiku{Haar zachte stappen,.}{sterven weg in het huis in}{de sneeuw daarbuiten}\\

\haiku{Achter een ringmuur,.}{van sterren loopt Djoeke en}{kijkt naar het leven}\\

\haiku{Een kleine lamp brandt,,...}{een blauwsteenen kan glanst het}{voorhoofd van een man}\\

\haiku{En zal je onder,?}{dit alles de vreugden van}{een Moeder voelen}\\

\haiku{God, dan werd het zoo,.}{guur en kil in mij of mijn}{hart in de tocht stond}\\

\haiku{Zon-goud wiegelt,.}{tak-schaduwen spelen}{op het vensterglas}\\

\haiku{{\textquoteright} En zij luistert naar,.}{de tik van het klokje een}{stem op het landpad}\\

\haiku{Zij verdwijnen nu...{\textquoteright},,.}{begrijpt zij en zij pinkt of}{zij slaperig is}\\

\haiku{wij prijzen U.{\textquoteright} Als,,.}{een kind loopt zij door luchtig}{onregelmatig}\\

\haiku{{\textquoteleft}Zij moest ziek worden{\textquoteright},, {\textquoteleft}.}{beseft hijopdat ik mijn}{angst zou begrijpen}\\

\haiku{Er is een sterke,.}{strenge eerbied in zijn hart}{een stil blank ontzag}\\

\haiku{Zijn oogleden zijn.}{blank en glanzend of ze een}{gebed verbergen}\\

\subsection{Uit: Grillige schaduwen}

\haiku{Bikkelglad praat wat.}{stennerig en ze zucht diep}{en nadrukkelijk}\\

\haiku{{\textquoteright} Bikkelglad zucht, en,.}{even in de stilte peinst ze}{dan vertelt ze weer}\\

\haiku{Even duurt dat, dan praat,,,.}{hij weer door bijnagewoon}{bezonken langzaam}\\

\haiku{Ook zoo'n stille nacht... ',!}{enn maantje en de klok}{op bij-twaalven}\\

\haiku{Heer in de hemel,,,.........?}{dan ook de buitelaar wie}{was hij wie was hij}\\

\haiku{'n Gewoonte had, ', '...}{hij opt rooie kamertje}{t opkamertje}\\

\haiku{En toen... toen \`al die,:}{dagen er na soesde hij}{er gedurig over}\\

\haiku{{\textquoteright} 'n Beetje benard,,.}{en dof ook beklemder van}{adem praat hij dan door}\\

\haiku{Nee, g\'een mensch weet w\`at ',... ',...}{t is en wi\'e ent is}{meer gezien hu-u}\\

\haiku{Klokkeloodje{\textquoteright} me.}{schokkend van schrik het witte}{spitse gezicht toe}\\

\haiku{De borrel maakte, '.}{z'n tong los die maaktet}{booze in hem gaande}\\

\haiku{Elikster  zat in,,.}{de voorste rij hij lachte}{luid hij spotte luid}\\

\haiku{En die priester met,.}{z'n witte kazuifel die}{jaagt mij niet van je}\\

\haiku{Toen ze de deur open ' ':}{deed vielt lampje ent}{glas uit haar handen}\\

\haiku{Op die manier, zie, '.}{je ben je dan zoo'n beetje}{n gesjochte knaap}\\

\haiku{En ook gelijk hield.}{hij zoo ies-of-wat tusschen}{z'n duim en vinger}\\

\haiku{{\textquoteleft}Ja, ja, maats, ik hoor,,,,!}{je wel ik k\`om maats wacht nog}{maar effe ik kom}\\

\haiku{{\textquoteleft}'n mensch blijft gauw stil,{\textquoteright}, '!}{staan voor wat glinstert jaak en}{dat sluit asn bus}\\

\haiku{Goeie-grutjen,,?}{man bij zoo iets prakkezeer}{je ommers geeneens}\\

\subsection{Uit: Harlekijntje}

\haiku{Maar het hindert niet -,.}{ze doen dat enkel voor de}{grap de toovermannen}\\

\haiku{Er loopt daar - als een -.}{torentje van donkerheid}{een klein dik vrouwtje}\\

\haiku{De Groene Parkiet{\textquoteright},.}{niet uit de caf\'e-deur}{maar uit het poortje}\\

\haiku{Maar haar kin bibbert.}{een beetje en in haar eene}{wang is een kuiltje}\\

\haiku{Dat ben' stakkers met.}{een aangestoken plekkie}{in hullie hersens}\\

\haiku{En een vraag die er ',.}{als meer geweest is klopt}{opnieuw bij hem aan}\\

\haiku{Hij kan de woorden,.}{die achter Moeder's antwoord}{langs gaan ook verstaan}\\

\haiku{- kan die lachen... ziet!}{er net uit of hij ook wel}{kuiltjerol1 kan doen}\\

\haiku{Hij betast het  ,.}{randje van zijn rechteroor}{hij bevoelt een zorg}\\

\haiku{Een enkel woord vangt,:}{hij maar op een woord dat zoo}{hard als een duw is}\\

\haiku{Het steekt hem erger,.}{dan ooit dat zij geen huis met}{een voordeur hebben}\\

\haiku{Op de nagel van:}{zijn middelvinger heeft hij}{drie witte stippen}\\

\haiku{Aan een beek die naar,.}{klaver en margrieten ruikt}{lesschen zij hun dorst}\\

\haiku{{\textquoteleft}Je harlekijntje{\textquoteright},, {\textquoteleft}?}{vraagt hij ineensis dat nou}{al weer afgedankt}\\

\haiku{We hebben de wind{\textquoteright},, {\textquoteleft} -.}{tegen hijgt Vlooienbeetloopt}{zwaar in de wind op}\\

\haiku{{\textquoteleft}Weet j{\'\i}j waarom ze -,?}{Gibbetje Vonk Gibbetje}{Vonk noemen buurman}\\

\haiku{Waarom ben jij van?,?}{vleesch en bloed en niet van}{porcelein en melk}\\

\haiku{Daantje gaat midden,.}{in de straat staan en kijkt naar}{het dak van hun huis}\\

\haiku{Men kan niet uit het.}{hoofd opnoemen wat er daar}{op de akkers staat}\\

\haiku{{\textquoteleft}Kinderen hebben{\textquoteright},.}{het maar gemakkelijk zegt}{ze en dat kraakt weer}\\

\haiku{Zijn adem moet telkens.}{over een bergje klauteren}{binnen in z'n keel}\\

\haiku{En de stilte en.}{de zon koestert hen en de}{schaduw dekt hen toe}\\

\haiku{Dat h\^otel heeft een.}{deur die op een pennetje}{in de rondte draait}\\

\haiku{ik wil er toch heen,.}{het is hier in geen jaren}{op het dorp geweest}\\

\haiku{En ze stak ze in.}{de grond en ze maakte ze}{in de boomen vast}\\

\haiku{De klompenmaker,:}{hakt hout en Vrouw Grom wascht}{erwten in een pan}\\

\haiku{Stijf knijpt hij de oogen,.}{toe en bromt iets tegen de}{voering van zijn pet}\\

\haiku{Krom aan \'e\'en kant loopt,.}{Garmen en zijn beenen zwikken}{door in de knie\"en}\\

\haiku{{\textquoteleft}Ja broer{\textquoteright}, grinnikt hij, {\textquoteleft},?}{met een ernstig gezichteen}{raar brillehuis h\`e}\\

\haiku{Hij knijpt zich in de.}{dijen van de pret en trekt}{aan Crissie's wortel}\\

\haiku{Bovenmeester is -.}{nooit heelemaal uit school ook}{als hij uit school is}\\

\haiku{{\textquoteleft}Zie je Immetje?,?,?}{Groen niet en Sybrecht en is}{er geen-een van Isa}\\

\haiku{En elke keer als,.}{hij een huilschreeuw geeft grijpt hij}{naar zijn achterste}\\

\haiku{Hij wil een voornaam,.}{handgebaar maken zwierig}{zijn muts afnemen}\\

\haiku{{\textquoteright} En Harlekijn haalt, -...}{gauw de vinger uit de mond}{en hijscht hijscht}\\

\haiku{En alles in de,.}{wankele uitkijktoren}{beweegt trilt en schudt}\\

\haiku{De gedachten gaan,.}{met een zwenkende vogel}{mee hoog de lucht in}\\

\haiku{Maar zijn Vader en:}{Moeder zullen telkens-weer}{tegen hem zeggen}\\

\haiku{Daantje knijpt tusschen.}{duim en vinger een tuitje}{in zijn onderlip}\\

\haiku{{\textquoteleft}Ik zou nog wel graag{\textquoteright},, {\textquoteleft}.}{een broer willen hebben houdt}{Daantje aanwel gr\'aag}\\

\haiku{En ik - ik ben dan,,...}{een deftige mijnheer zie}{je met een wit vest}\\

\haiku{{\textquoteright} Hij bekijkt nog 's.}{de kapittelstokjes en}{het schotsche haarlint}\\

\haiku{{\textquoteright} Maar Moeder zegt niets,,.}{zij stoot Opoe aan fluistert iets}{en kijkt naar Daantje}\\

\haiku{Maar dan ineens is,.}{Teetje Schep er Teetje Schep als de}{zonneschijn-zelf}\\

\haiku{{\textquoteright} Uit hun ooghoekjes.}{kijken de monkelende}{groote menschen naar hem}\\

\haiku{{\textquoteleft}Maar nou{\textquoteright}, bedisselt, {\textquoteleft},,.}{Opadoen we het ook eerst v\'oor}{het brood eten kom Vrouw}\\

\haiku{Heb je nou ooit{\textquoteright}, lacht, {\textquoteleft}.}{ze opgetogenn\`et iets}{wat ik erg noodig had}\\

\haiku{- En later kijken,.}{ze recht voor zich uit en ze}{praten onderdrukt}\\

\haiku{Maar waarom moet Oom?}{Herre zich dan toch ineens}{naar hem omdraaien}\\

\haiku{{\textquoteright} Verlangende oogen.}{krijgt Daantje daarbij en een}{verlangende mond}\\

\haiku{Beschroomd denkt hij aan,.}{de musch terug en begint}{haastig te praten}\\

\haiku{Hij weet plotseling,,.}{dat hij onder het bidden}{gluren zal naar God}\\

\haiku{En Daantje moet zijn.}{hand in een vuistje tegen}{zijn lippen drukken}\\

\haiku{{\textquoteleft}Zie je wel{\textquoteright}, mijmert, {\textquoteleft},,.}{hijdat land waar ik was voor}{me Moeder me kreeg}\\

\haiku{En hij  wil er.}{toch mee naar het lichte land}{waar men zweven kan}\\

\haiku{En de Koning zegt,:}{in de donker dreunende}{stem van het orgel}\\

\haiku{En hij gaat dan wel,.}{behoorlijk recht-op zitten}{maar hij trekt een lip}\\

\haiku{Hij ziet ineens weer.}{het plagerige gezicht}{van Tante Celien}\\

\haiku{En Roel zet gauw het.}{napje met centen neer en}{flapt weer op zijn stoel}\\

\haiku{Maar Daantje hoeft niet,.}{in zijn boekje te kijken}{h{\'\i}j weet het z\'oo wel}\\

\haiku{En Daantje's handen.}{klemmen zich krampachtig om}{de stoelzitting vast}\\

\haiku{Juffrouw gichelt, en.}{ze is opeens weer Juffrouw}{van de Operette}\\

\haiku{Zij gaan over de brug,,.}{zij komen langs de kerk zij}{loopen de straat in}\\

\haiku{Het moet maar...{\textquoteright} En het,.}{kruis valt tegen het huis aan}{dat geen voordeur heeft}\\

\haiku{Nadenkend schuift Isa.}{haar wijsvinger onder haar}{kapittelstokjes}\\

\haiku{{\textquoteright} {\textquoteleft}Haal {\`\i}k me serviesie{\textquoteright},, {\textquoteleft}.}{verzoet Isavraag {\`\i}k een cent}{voor wonderballen}\\

\haiku{Ze begint ineens:}{een Miranda-versje}{op te zeggen}\\

\haiku{Isa het op, ze kan,.}{het zoo goed dat het haast niet}{meer te verstaan is}\\

\haiku{Opa heb een boomgaard,.}{met wel vijfhonderd boomen}{dan is die ook rijk}\\

\haiku{Een maand geleden{\textquoteright},, {\textquoteleft}.}{was dat al een beetje valt}{hem inmet die ster}\\

\haiku{{\textquoteleft}Stil, stil{\textquoteright}, knipperen,.}{zijn oogen tegen Katinka}{die doorloopen wil}\\

\haiku{Wat spijtigs springt naar,.}{voren in zijn oogen en duikt}{dadelijk weer weg}\\

\haiku{De teenen staan alle.}{tien op het randje van de}{groenhouten tafel}\\

\haiku{{\textquoteright} Grootvader's stoel geeft,.}{een krak net of Grootvader}{ineens zwaarder wordt}\\

\haiku{{\textquoteright} - - - - - - - - - Grootvader slaat weer.}{op de leuning van zijn stoel}{en hij hapt naar adem}\\

\haiku{En Katinka moet.}{nu opeens aan een lied van}{Zondagsschool denken}\\

\haiku{Dat verhelderde:}{in Gibbetje's gezicht blijft}{Brikkelebrit bij}\\

\haiku{Daantje Diddes heeft,.}{in het leven bereikt wat}{hij bereiken wou}\\

\haiku{Het beddescherm waar.}{Jonkvrouw Maleen achter staat}{is niet hoog genoeg}\\

\haiku{En zij moeten net.}{doen of ze heelemaal niet}{weten waar hij is}\\

\haiku{Het gonst en zoemt daar.}{of er tweeduizend vliegen}{dooreen dwarrelen}\\

\haiku{Maar dan is alles,,.}{als bij tooverslag veranderd}{juist als in een droom}\\

\haiku{En haastig wordt een.}{bloemetjes-gordijn}{opzij getrokken}\\

\haiku{Toen Teetje Schep in haar,.}{doodshemd op haar bed lag was}{die stilte er ook}\\

\haiku{{\textquoteleft}Ik ben ook nog maar{\textquoteright},, {\textquoteleft}...}{negen jaar zegt hij tegen}{zijn angstnegen jaar}\\

\haiku{{\textquoteleft}Het is waar{\textquoteright}, geeft hij, {\textquoteleft},}{toeik h\`eb het vergeten}{maar ik kom wel weer}\\

\haiku{het is of iemand,.}{hem bij de hand neemt hij kan}{niet anders loopen}\\

\haiku{{\textquoteright}, een pijn keert zich om,, {\textquoteleft}?}{midden in de gedachten}{ni\'et w\'ezenlijk}\\

\haiku{{\textquoteright}, herinnert Daantje.}{haar met een stevige tik}{op zijn eigen wang}\\

\haiku{Voor een nachtegaal -.}{ben je niet knap genoeg een}{uil zal je worden}\\

\haiku{Hij merkt dat hij bar,.}{veel schik kan hebben zonder}{zich te verroeren}\\

\haiku{{\textquoteright} {\textquoteleft}Een likeurglas{\textquoteright}, haalt,,}{Vader uit en hij doet of}{hij grilt van afschuw}\\

\haiku{{\textquoteleft}Wat zei Dokter nou?,?}{allegaar over mij en wat}{is dat met die zon}\\

\haiku{, denk jullie er om?,}{dat ik dan geen korsies krijg}{van de hittigheid}\\

\haiku{Zij kijken elkaar,.}{aan het is of er licht uit}{hun voorhoofden komt}\\

\haiku{{\textquoteright} Onaannemelijk,.}{lijkt dat Daantje niet al is}{er toch wat raars bij}\\

\haiku{En de knoppen van:}{de madelieven lijken}{op bakerkindjes}\\

\haiku{Terloops wil hij er,,.}{een paar van plukken Daantje}{en hij vergeet het}\\

\haiku{Opa moet zijn hoofd diep,.}{voorover buigen het antwoord}{komt als een ademtocht}\\

\haiku{Dan zakt hij weer weg.}{in een stralende witte}{diepte en wiegelt}\\

\haiku{Hij bijt er bij op,.}{een zakdoekpunt hij wringt zijn}{zakdoek om en om}\\

\haiku{Hij is er toch wel.}{een beetje trotsch op dat hij}{het gegeven heeft}\\

\haiku{{\textquoteright}, krijscht hij, {\textquoteleft}snij me,,!}{buik open haal het er uit haal}{de dorens er uit}\\

\haiku{{\textquoteright} Onbeholpen strijkt,.}{hij langs haar oogleden er}{zijn daar geen tranen}\\

\subsection{Uit: De ijzeren greep}

\haiku{{\textquoteright} Hij draait aan de knop,.}{van een laag deurtje peutert}{aan een sleutelgat}\\

\haiku{Maar zijn Moeder schijnt,.}{dat niet te begrijpen ze}{zet hem op de vloer}\\

\haiku{En Bielke ziet de:}{eerste verschijnselen van}{de samenleving}\\

\haiku{De zonneschijn trekt,.}{weg het gaat regenen en}{de deur moet dicht}\\

\haiku{Hij schraapt er met de.}{schoenzolen over heen en sleept}{ze om beurten mee}\\

\haiku{Hij zou er graag bij,.}{opklimmen maar het mag niet}{en het kan ook niet}\\

\haiku{En Grootmoeder duwt.}{Bielke een stukje vooruit}{bij de begroeting}\\

\haiku{En zijn makker kruipt.}{ook al weer door het gat van}{de ligusterhaag}\\

\haiku{Tot de hoek dan maar...{\textquoteright}.}{Maar de tonnetjesman wil}{er niets van weten}\\

\haiku{Maar haar zoon Driek, die.}{is jonger dan Vader en}{gedurig  ziek}\\

\haiku{Eerst tikte ze met:}{de punt van het mes op de}{koek en mummelde}\\

\haiku{Hij gluurde door de,.}{bedgordijnen en was eerst}{verbaasd en toen blij}\\

\haiku{De koperen bak.}{van de olielamp glinsterde}{als een klok van goud}\\

\haiku{Hij had zijn beenen al.}{gauw over de beddeplank en}{stak zijn armen uit}\\

\haiku{Toen kreeg hij opnieuw.}{een andere kiel met een}{horlogezakje}\\

\haiku{Nu zou de witte -.}{tafel met de hartjes weer}{komen Sinterklaas}\\

\haiku{haast zoo doorschijnend.}{als de glazen kast-engltjes}{in de voorkamer}\\

\haiku{{\textquoteleft}Joppe's Grootmoeder.}{Tonia is aardiger dan}{zijn Grootmoeder Brecht}\\

\haiku{Het is altijd een.}{dienst-vertelling of een}{spokenvertelling}\\

\haiku{{\textquoteleft}Die malle Freerk Kret{\textquoteright},, {\textquoteleft}?}{zeggen zewie doet er nou}{\'een kip bij een haan}\\

\haiku{Zijn Vader heeft een.}{erge snee in zijn hand en}{geen lap er om heen}\\

\haiku{{\textquoteleft}Sterven, dat is niet,,.}{het ergste maar van je kind}{weg moeten je kind}\\

\haiku{{\textquoteright} {\textquoteleft}Drinken{\textquoteright}, bedenkt hij, {\textquoteleft}...{\textquoteright} {\textquoteleft}{\textquoteright},.}{schuwtheeThee herhaalt Moeder}{en zij wil opstaan}\\

\haiku{{\textquoteleft}Moeder{\textquoteright}, hij omvat.}{haar rokken en drukt zijn hoofd}{vast tegen haar schoot}\\

\haiku{Als ik er niet ben{\textquoteright},, {\textquoteleft}....}{fluistert ze wonderlijkben}{ik er toch even goed}\\

\haiku{{\textquoteleft}Kom me jongen{\textquoteright}, zegt, {\textquoteleft}.}{hij gehaastvannacht mag je}{bij Joppe slapen}\\

\haiku{Hij draait zich fel om,.}{wil Vader's hand pakken en}{Vader is al weg}\\

\haiku{hij wil zich van zijn,.}{stoel laten glijden wil zijn}{Vader naloopen}\\

\haiku{En om de stemmen.}{komt al dichter het vreemde}{zware van de avond}\\

\haiku{Boven op zolder.}{wordt hij van de kilte weer}{een beetje wakker}\\

\haiku{hij bleek en beschroomd.}{op de binnenplaats en in}{de grijs-steenen gang}\\

\haiku{daar zijn gezichten,,,.}{van te maken een maan een}{hansworst een kerel}\\

\haiku{worst krijgen,  van...}{Soling de bakker moppen}{en pepernoten}\\

\haiku{{\textquoteright} - - - - - - - - Strak kijkt hij daarbij}{naar de sterren op en hij}{houdt de lampion}\\

\haiku{De Heer verheffe.}{Zijn aangezicht over u en}{geve u vrede}\\

\haiku{En voor de villa.}{Eusebia bloeien de}{paarse seringen}\\

\haiku{En Arjen Kappel,,.}{de veldwachter dat is geen}{gemakkelijke}\\

\haiku{En Hint's vrouw die was.}{ook telkens voor het bord van}{het Gemeentehuis}\\

\haiku{We hebben immers?}{onze plichten tegenover}{de samenleving}\\

\haiku{{\textquoteleft}Dat moest toch bij mij{\textquoteright},, {\textquoteleft}}{aan huis niet gebeuren speelt}{hij plotseling op}\\

\haiku{Grootmoeder breit ook,,.}{de heele dag en Moeder}{elk vrij oogenblik}\\

\haiku{{\textquoteright} En Micha\"el's.}{kwieke snorre-lip gaat}{bol vooruit  staan}\\

\haiku{Ze laadt haar huis af,.}{of het een schip is waar ze}{mee wegvaren kan}\\

\haiku{zoo dicht bij of hij.}{met die neus-van-hem het}{pak toedekken wil}\\

\haiku{Ze strijkt het haar glad,,.}{doet haar doek netjes ze heeft}{geen pruttellip meer}\\

\haiku{En alle menschen.}{hebben strakke gezichten}{en gespannen oogen}\\

\haiku{{\textquoteleft}Waar gaan we heen?, en?}{wat komen er nou nog meer}{voor erge dingen}\\

\haiku{{\textquoteright} Het is heel goed te.}{merken dat Moeder met haar}{zwijgzaamheid antwoordt}\\

\haiku{En hij is nog niet:}{oud genoeg om meewarig}{te kunnen zeggen}\\

\haiku{Het vroege licht is.}{zoo wonderbaar zacht of er}{maneschijn in is}\\

\haiku{De raamruitjes zijn}{vurige vierkantjes en}{de deuren glanzen}\\

\haiku{Het is nu ook of:}{hij door een vergrootglas naar}{de toekomst kan zien}\\

\haiku{rookstralen met,,,.}{kogels er in het siste}{knetterde kraakte}\\

\haiku{En Vader keert zich.}{zoo heftig snel om of hij}{boos uitvallen wil}\\

\haiku{wat belachelijks,,.}{nu is iedereen het en}{het is niet erg meer}\\

\haiku{De vrouwe-muts,.}{die v\'oor Freerk Kret opduikt is}{van Hesseltje Stoop}\\

\haiku{Hesseltje heeft al.}{in geen twee maanden bericht}{van Koertje gehad}\\

\haiku{Hij trekt jolig de,.}{wenkbrauwen op gluurt jolig}{naar de zoldering}\\

\haiku{Tusschen haar lange.}{zwarte oogharen is het}{kleurloos als regen}\\

\haiku{Ze roert toch langzaam,.}{regelmatig in de brij}{precies zooals het moet}\\

\haiku{door de winter is,.}{het toch niet het vorige}{jaar was het anders}\\

\haiku{{\textquoteleft}Thomasken{\textquoteright}, fleemt ze, {\textquoteleft}.}{buiten adem en jagerig}{ik kom u halen}\\

\haiku{{\textquoteleft}Als je de zakken,.}{vol gestopt hebt eerst maar het}{zaagsel opvegen}\\

\haiku{Ze krijgt een erg hooge,,.}{buik Nustancia het is al ver}{heen met het kindje}\\

\haiku{Vandaag het bestek,,.}{gezien en de teekening}{eerste klas spul man}\\

\haiku{{\textquoteright} {\textquoteleft}Ja{\textquoteright}, knikt Bielke, hij,.}{gaat er verder niet op in}{hij houdt zijn stuk vast}\\

\haiku{We hebben het toch,?,,?}{goed niet waar zeg jongske we}{hebben het toch goed}\\

\haiku{Het is ook zoo maar:}{met een vloek en een zucht in}{elkaar geslagen}\\

\haiku{Ze kijken eerst wie,.}{er in de werkplaats is dan}{praten ze oog wat}\\

\haiku{Ze praten er over.}{zooals ze over aambeien en}{negen-oogen praten}\\

\haiku{dat is een heel ding,.}{maar de coulissen moet je}{ook niet uitpoetsen}\\

\haiku{Met een hoovaardig.}{gezicht denkt Bielke daar nog}{altijd aan terug}\\

\haiku{{\textquoteleft}Zaterdagavond{\textquoteright}, staat, {\textquoteleft}.}{hij zichzelf toezoo'n doos van}{honderd twintig stuks}\\

\haiku{{\textquoteleft}Ja, warm{\textquoteright}, Angelia.}{schuift een ringetje op en}{neer aan haar vinger}\\

\haiku{twee ramen, een deur.}{en een tuintje  zoo groot}{als een tafelblad}\\

\haiku{{\textquoteright} De achterkant van.}{het nieuwe huis staat nu naar}{De Fonteintjes toe}\\

\haiku{het zou de oude '.}{baas goed doen als hij dit nog}{s beleven kon}\\

\haiku{Met een overdreven.}{rechte rug loopt Titia het}{keldertrapje af}\\

\haiku{Maar Donnardus vindt.}{het niet de moeite waard er}{op te antwoorden}\\

\haiku{Maar je bergplaats - je,.}{garage nou weer en dan}{hier in De Klinkert}\\

\haiku{Eigenlijk moesten wij,.}{in onze situatie}{uit zoo'n straat weggaan}\\

\haiku{{\textquoteright} Hij weet ook best wat.}{vrouw Pannes bedoelde met}{de zeemleeren filter}\\

\haiku{We weten het wel,,.}{Anna jij blijft het liefst op}{je eigen terrein}\\

\haiku{{\textquoteright} Maar als hij in de,.}{half verstijfde aardappels}{prikt vraagt hij toch weer}\\

\haiku{{\textquoteright} Hij steekt de handen,.}{diep in de jaszakken trekt}{de schouders wat op}\\

\haiku{Vijf guldens flitsen,.}{langs Bielke's gedachten hij}{lacht zacht in zichzelf}\\

\haiku{{\textquoteright} Ineens ziet hij zijn.}{Moeder's beangstigende}{oogen en proest het uit}\\

\haiku{Een deur flapt open en,.}{toe met een stug zuiggeluid}{een scharnier kreunt dof}\\

\haiku{Zoo'n stoffel als ik?,...!}{die van toeten noch blazen}{af-weet nee kind}\\

\haiku{Vanmorgen bericht,.}{gehad acht en veertig stuks}{en zes lessenaars}\\

\haiku{Als je de cene,.}{meid meer geeft dan de ander}{maak je schele oogen}\\

\haiku{Het schoot me zoo te,.}{binnen nou ligt die wensch ook}{altijd bovenaan}\\

\haiku{Het is... is of je -...}{mijn hart tegen de keien}{slingert dat hart hier}\\

\haiku{{\textquoteright} Hij omhelst haar, en.}{het is of hij haar niet in}{de armen heeft}\\

\haiku{Het was nou ook wel.}{zoo ver gekomen dat ik}{hem best missen kon}\\

\haiku{{\textquoteleft}Ik heb er lang op,,...}{gewacht baas lang  verwacht}{en toch verkregen}\\

\haiku{loonsverlaging en,.}{bekrimping van arbeidskracht}{al wat de klok slaat}\\

\haiku{Seerp liet alles,...}{zoo prachtig inrichten had}{er zoo'n pleizier in}\\

\haiku{Er stappen vreemde,.}{menschen uit menschen die hem}{onverschillig zijn}\\

\haiku{Ja, het zou wel erg.}{ondoordacht wezen als we}{nu gingen trouwen}\\

\haiku{En er zijn een paar,.}{heel lieve menschen hier daar}{heeft Tante veel aan}\\

\haiku{{\textquoteright} {\textquoteleft}Ik leer voor kapster{\textquoteright},, {\textquoteleft}}{valt Tripke in v\'oor er een}{stilte kan komen}\\

\haiku{Bielke loopt of hij,,.}{vlucht met lange stappen het}{hoofd in de schouders}\\

\haiku{Ze schrijft meer dan eerst{\textquoteright},, {\textquoteleft}.}{tracht hij nog te sussenveel}{meer en geregeld}\\

\haiku{Het orchestrion.}{achter haar rettelt en bonkt}{als een stoomhamer}\\

\haiku{{\textquoteright} Hij verzet geen voet,.}{plukt venijnig hard aan de}{voering van zijn pet}\\

\haiku{{\textquoteleft}Ik zal wel me best,,.}{doen dat ik wat krijg ik zal}{van alles probeeren}\\

\haiku{En we hebben hier,,,...{\textquoteright}}{zoo fijn gewerkt h\`e Seerp}{en zoo lang en nou}\\

\haiku{En dat is verdraaid,,.}{moeilijk me jongen dat zal}{ons tegenvallen}\\

\haiku{En weer schrikt Bielke,.}{op in zijn bed het is al}{ver in de ochtend}\\

\haiku{{\textquoteright} Dan sluit hij de oogen '?}{nog weers. Waarom zal hij}{de oogen niet sluiten}\\

\haiku{Onzinnig-lang staart.}{Bielke naar een kamerhoek}{waar niets te zien is}\\

\haiku{Maar ze mopperen,,.}{niet Vader en Moeder ze}{kijken alleen maar}\\

\haiku{De zak met kolen.}{ligt als een looden last op zijn}{ontvelde schouders}\\

\haiku{{\textquoteleft}We wisten niet waar,?,.}{je bleef zie je we hebben}{vaak uitgekeken}\\

\haiku{Titia loopt en kijkt -.}{of het leven ver achter}{haar ligt ze is oud}\\

\haiku{Hij schuift zich ook weer,,,...}{tusschen de toeschouwers in}{wuift nog even dringt stompt}\\

\haiku{ik was op zoek naar, -,.}{werk en en het was weer mis}{zoo'n lamme stemming}\\

\haiku{Bielke weet ineens.}{dat het een sjouwerman uit}{De Meerendonck is}\\

\haiku{Toch een goed ding en,:}{die werkeloozen-uit keering ook}{ze mogen zeggen}\\

\haiku{Daar vraagt de tijd nog{\textquoteright},, {\textquoteleft}...{\textquoteright} {\textquoteleft},{\textquoteright},}{al naar valt hij uitwat je}{k\`anNou ja enfin}\\

\haiku{Ik snap niet dat jij...}{er niks van wist en bij je}{thuis en in de straat}\\

\haiku{Even later staat hij.}{op het Wolvenbruggetje}{bij zijn kornuiten}\\

\haiku{Het zijn altijd de -.}{menschen die het verkeerd doen}{altijd de menschen}\\

\haiku{Hij tuurt en hij ziet'.}{de verweerde liefheid in}{Goitske Dubies oogen}\\

\haiku{Ik heb nog centen,,...}{over anders krijg ik ze wel}{ergens neem ik ze}\\

\subsection{Uit: Ik verwacht het geluk}

\haiku{En dacht jij dat dit,?}{het voor-portaal van de hel}{was snoetebakkes}\\

\haiku{Ze draagt een emmer.}{met dweilwater en buigt wat}{door in de schouders}\\

\haiku{{\textquoteright} {\textquoteleft}Op naailes was ze,{\textquoteright},.}{leuk pleit Stijn toch nog en keert}{zich met een ruk om}\\

\haiku{{\textquoteleft}Kan het haast niet zien,{\textquoteright}, {\textquoteleft}.}{verontschuldigt Pietahet}{is zoo schemerig}\\

\haiku{Kors die wou altijd.}{gelijk hebben en dagen}{lang kon hij koppen}\\

\haiku{De nieuwe jurk staat,,.}{haar goed zij is er slanker}{mee volwassener}\\

\haiku{jij verre man, daar,?}{ergens in die groote wereld}{dat ik je noodig heb}\\

\haiku{{\textquoteright}, zeggen ze tegen, {\textquoteleft}?,?}{Pietawaar bleef je zoo lang}{wat moest je nog doen}\\

\haiku{{\textquoteleft}Kijk mijn mooie gebit '{\textquoteright}.}{s. Stijn Mets stond daar haar hoofd}{bij te  schudden}\\

\haiku{Je moet natuurlijk.}{eerst meerderjarig wezen}{of je moet trouwen}\\

\haiku{{\textquoteleft}Iedereen heeft geen.}{tante Koos die de boel zoo}{lang voor je opbergt}\\

\haiku{Wat een volmaakte,,?}{stilte ijver en orde}{h\`e juffrouw Siebje}\\

\haiku{Oue mijnheer Baruut, de,.}{antiquair schuifel-sloft}{dicht langs de huizen}\\

\haiku{De verte ligt wit.}{onder een roode kerf in}{de hellende lucht}\\

\haiku{Vader's zagerig.}{praten hindert haar deze}{keer bovenmate}\\

\haiku{In de verte zal,.}{het licht zijn en vriendelijk}{en lente-achtig}\\

\haiku{Ze luistert er naar.}{zooals een mensch luistert die door}{slaap bevangen is}\\

\haiku{En dan staat ze daar.}{of ze beet gegrepen wordt}{en niet verder kan}\\

\haiku{dat zal je zien, dan.}{magge we vast allemaal}{kastanjes rapen}\\

\haiku{H\`e - rare,{\textquoteright} moppert,.}{Wina Levina maar ze}{is niet onvoldaan}\\

\haiku{De zetels van Buk.}{en Dibbe wrijft ze zonder}{geweldpleging af}\\

\haiku{De zonnestralen,.}{flitsen als fonteinen de}{boomkruinen branden}\\

\haiku{Hij bracht de mand met.}{witlof naar binnen en keek}{toevallig omhoog}\\

\haiku{De stemmen van de.}{jongetjes dreinen aan een}{opgeschoven raam}\\

\haiku{{\textquoteright} En Vader heft met.}{een moede verbazing de}{handen ten hemel}\\

\haiku{{\textquoteleft}Wagentje, waar moet?,,?}{je toch wezen stappen waar}{gaan jullie naar toe}\\

\haiku{Daar was ze toen nog,.}{verwonderd over nu begrijpt}{ze er alles van}\\

\haiku{En je bent te laat,.}{uit school gekomen dat moet}{niet meer gebeuren}\\

\haiku{{\textquoteright} Later zat ze als.}{een stijf klein pijn-frommeltje}{onder aan de trap}\\

\haiku{Eenmaal gaan ze er '.}{s zomers met hun allen}{een dagje naar toe}\\

\haiku{En ik zelf, tja, neem,.}{mijn nou ik bestee geen rooie}{duit \^an me kanis}\\

\haiku{{\textquoteleft}zag ik u net toen.}{u dat leegstaande huis van}{Brammers voorbijging}\\

\haiku{En hoe licht ken de,?}{pliksem niet inslaan door soo'n}{glasen dingetje}\\

\haiku{Pieta kijkt er met.}{kinderlijke oogen naar en}{loopt weifelend voort}\\

\haiku{{\textquoteleft}Je kan nou maar kort,{\textquoteright}, {\textquoteleft},.}{bij Jet blijven valt haar in}{een uurtje jammer}\\

\haiku{Hij trommelt in het.}{voorbijgaan hardhandig op}{Pieta's achterhoofd}\\

\haiku{Buiten de fletse.}{lichtkringen van de lantaarns}{loert verlatenheid}\\

\haiku{E\'en voor \'een gaan ze:}{het kabinetje met de}{witte beeldjes in}\\

\haiku{En ze glimlachen,,.}{nu nog alle vijf maar met}{puntige lippen}\\

\haiku{{\textquoteright} En de boombl\^aren.}{en de paden zijn zoo rood}{of er vuur in brandt}\\

\haiku{De onstuimige.}{dikke druppels kletteren}{als hagelkorrels}\\

\haiku{De regen wordt als.}{met volle nappen tegen}{de ruiten geplenst}\\

\haiku{{\textquoteright} Er staat dan toch ook.}{wel een stugge pijn-vouw}{om haar glimlach heen}\\

\haiku{als je hier vandaan.}{komt is het daar heelemaal}{een spook-salon}\\

\haiku{En opnieuw is het.}{Pieta of haar een spiegel}{voorgehouden wordt}\\

\haiku{De kinderen in:}{de weezenbank fleuren er}{heelemaal van op}\\

\haiku{ze hebben goddank,.}{iets te doen zij galmen zoo}{hard mogelijk mee}\\

\haiku{Wina Levina.}{is altijd het eerst aan het}{eind van een regel}\\

\haiku{Over de zinnen en.}{woorden die ze opvangt denkt}{Pieta nog wel na}\\

\haiku{{\textquoteright} Ordelijk stappen:}{ze weer terug door de bleeke}{koele zonneschijn}\\

\haiku{{\textquoteleft}Geloof jij dat het?}{Kerstkindje wat wonders wil}{doen als je het vraagt}\\

\haiku{Oh - ik.... die weeflamp in,{\textquoteright}, {\textquoteleft} -?}{de uitstalling stamelt ze}{de de prijs er van}\\

\haiku{Het is of ze van.}{haar hoofd tot haar voeten \'een}{vurige blos wordt}\\

\haiku{Bijna vinnig drukt.}{ze haar ronde zachte kin}{op haar mantelkraag}\\

\haiku{Hans Wietzel legt over.}{de leuning van zijn fauteuil}{zijn hand op haar arm}\\

\haiku{'s Nachts droomt ze dat.}{ze sterft en een vriesbloem op}{de vensterruit wordt}\\

\haiku{En Pieta pinkt en.}{spert de oogen open als een die}{plotseling ontwaakt}\\

\haiku{de rug tegen de....}{eene zij-leuning en de beenen}{over de andere}\\

\haiku{Hij vraagt het maar zoo,.}{luchtig-weg er is toch}{wat straks in zijn lach}\\

\haiku{Het wordt een gesprek.}{zonder overgangen en met}{vele gapingen}\\

\haiku{Maar Hans Wietzel en,,....}{Pieta die dicht naast elkaar}{zitten ontgaat dat}\\

\haiku{Hij neemt haar ook nog,.}{mee naar het opkamertje}{rechts van de winkel}\\

\haiku{Ik zal hem nog 's,,.}{van Vader vertellen op}{een keer het komt wel}\\

\haiku{Wat een kostbare,?}{dingen in die winkel bij}{Pa Baruut h\`e lieverd}\\

\haiku{{\textquoteleft}Ja, laten we dan,{\textquoteright}.}{maar alledrie schichtig spiedt}{ze naar een zaalhoek}\\

\haiku{Want Moeder maakt een.}{woest maai-gebaar dicht bij}{zijn dunne krullen}\\

\haiku{{\textquoteleft}Je zou ook nog een,,,....}{klok krijgen eendvogeltje}{eekhorentje muis}\\

\haiku{{\textquoteleft}Stijn de tobber zag,?}{grauw van zorg en wat zouen}{de joggies denken}\\

\haiku{En met een harde,:}{afwerende verbazing}{in haar stem polst ze}\\

\haiku{{\textquoteright} En dan proest ze het,.}{zelf uit omdat het zoo dom}{en onzinnig is}\\

\haiku{Hans Wietzel heeft het.}{haar op hun wandelingen}{meer dan eens verteld}\\

\haiku{Pieta luistert er,}{verrast naar en ze glimlacht}{er kinderlijk om}\\

\haiku{{\textquoteleft}Hans, lijkt het jou ook,?}{niet vreemd dat het pas vandaag}{was dat we trouwden}\\

\haiku{{\textquoteleft}Hier zijn geen menschen,,,!}{kind alleen de aarde de}{zon en de stilte}\\

\haiku{Op de terugreis,,:}{als ze weer naar de aarde}{toezakken tobt ze}\\

\haiku{En midden onder.}{de dienst scheurt de duimnaad van}{haar eene handschoen uit}\\

\haiku{Nu is ze alleen.}{maar Pieta Arsting uit het}{weeshuis en niets meer}\\

\haiku{Vannacht dacht ik  : -?}{nog zijn we nou ook langs langs}{ravijnen gegaan}\\

\haiku{Pieta kan nu niet.}{zoo dicht bij hem komen als}{ze wel zou willen}\\

\haiku{Ze wuift zoo vurig.}{dat ze haast twee bloempotten}{uit het venster stoot}\\

\haiku{Hier loop  ik, en -,.}{dit is mijn huis ons huis het}{huis van ons drietjes}\\

\haiku{Er hangen tulen.}{vleugels over haar mouwen en}{haar rug lijkt van gips}\\

\haiku{Bij ons,{\textquoteright} schertst hij, {\textquoteleft}is.}{de liefde vannacht om twaalf}{uur afgeloopen}\\

\haiku{{\textquoteright} Ze zakt op een stoel,,.}{neer de armen gekruist de}{kin op haar halskuil}\\

\haiku{{\textquoteright} {\textquoteleft}Ph-ph,{\textquoteright} doet Wina, {\textquoteleft}....}{Levina minachtendde}{Keizer van Sina}\\

\haiku{{\textquoteleft}Hans Wietzel vertelt -.}{me ook niks nooit de dingen}{waar het op aankomt}\\

\haiku{Hij heeft het naar zijn,.}{zin gehad vandaag dat is}{duidelijk genoeg}\\

\haiku{Breed-uit of het,.}{een dekje is legt hij het}{jurkje over haar schoot}\\

\haiku{{\textquoteright} Dan vindt zij ook aan,.}{de knop van een stoelleuning}{mevrouw's taschje}\\

\haiku{de beugel is blond,.}{en golverig de beurs rond}{en frambooskleurig}\\

\haiku{{\textquoteright}, ginnegapt Pieta, {\textquoteleft}!}{bedektelijkdat het mag}{van het reglement}\\

\haiku{{\textquoteleft}Dat is de oude,{\textquoteright}.}{smaak en ze krijgt nijpplooien}{aan de mondhoeken}\\

\haiku{{\textquoteleft}Mal dat ik die muts, - -.}{maak mal mal nooit is er}{iets zotters vertoond}\\

\haiku{{\textquoteright} Een slaperige.}{bejaarde vredigheid hangt}{in de huiskamer}\\

\haiku{{\textquoteright} Ze glimlacht maar met,,}{een stukje lip en trekt de}{wenkbrauwen hoog op}\\

\haiku{{\textquoteleft}I am fond of you,,,?}{Jackie but of course you}{knew that did not you}\\

\haiku{- wat ben ik toch ook,.}{al ontwikkeld dat ik het}{allemaal zoo weet}\\

\haiku{Och stoeteltje, het,!}{gaat niet om die clubs het gaat}{om de connecties}\\

\haiku{Ik heb maar gedaan,,....}{of ik niks merkte ik wou}{me goed houen ik}\\

\haiku{Je goed houen,{\textquoteright} flitst, {\textquoteleft},....{\textquoteright}}{het nog door haar heenom Pa}{om die oue stakker}\\

\haiku{Zou je mij herkend,?,,?}{hebben Geer d\'adelijk op}{het eerste gezicht}\\

\haiku{Dan - dan liever  ,,,.}{niets dit was voor een ander}{nee dan liever niets}\\

\haiku{{\textquoteleft}Kan  best wezen,,,,?}{Geer maar hier duld ik het niet}{hi\'er niet begrepen}\\

\haiku{, Pa is er alleen,,.}{maar we hebben stamppot met}{worst Hollandsche kost}\\

\haiku{Het moet zoover komen,.}{dat hij op de knie\"en om}{vergiffenis vraagt}\\

\haiku{Je zei dingen die.}{ik begreep en waarover ik}{nog niet praten kon}\\

\haiku{En het tweede hart,....}{klopt het tweede hart leeft vlak}{bij haar eigen hart}\\

\haiku{te vluchten naar - naar,?}{andere menschen naar een}{plek waar het stil is}\\

\haiku{wie er is, tuimelt.}{Pa bijna van het rechte}{smalle trapje af}\\

\haiku{{\textquoteright} Links en rechts kijkt hij,}{de ruischende stegen}{en zijstraatjes in}\\

\haiku{Hij wil zijn arm van.}{haar wegnemen en zij grijpt}{angstig zijn hand beet}\\

\subsection{Uit: In de witte stilte}

\haiku{Zorgvuldig borg ze,.}{het emmertje op in de}{koele keukenkast}\\

\haiku{{\textquoteleft}As 't nou daomee,{\textquoteright}, {\textquoteleft}.}{ophold nam Barta zich voor}{z\^a-'k gras snieden}\\

\haiku{Maor alles mut, '.}{zoo komm'nt steet in de}{Openbaoringen}\\

\haiku{We zatten mit de.}{kop boven op die olde}{praom van Vok Krieger}\\

\haiku{{\textquoteleft}'s Aovens te veuren ', '...}{had ik hem nog watbrochts}{morgens lag hij stief}\\

\haiku{{\textquoteright} Met een slip  van.}{haar doek friemelde ze over}{haar natte wangen}\\

\haiku{{\textquoteleft}Domeneer kon wel,.}{slechter want dat zal dan bie}{Riek Stoffers wezen}\\

\haiku{wat gij de minste,?}{van Mijn broederen doet dat}{hebt ge Mij gedaan}\\

\haiku{Ze ging dicht langs een,...}{diepe zwarte sloot en bleef}{er telkens bij stil}\\

\haiku{Maar Barta ving er,.}{niets van op ze had geen oog}{van de tafel af}\\

\haiku{, en in Dominee's.}{envelop was een briefje}{van tien gesloten}\\

\haiku{Wel verdulleme,{\textquoteright}, {\textquoteleft} ' '...}{kwam hij losis mie datn}{toon enn taol}\\

\haiku{{\textquoteright} {\textquoteleft}Och,{\textquoteright} wou Jentien dan, {\textquoteleft}?}{nog bangelijk uitstellen}{wacht tot morgenvruug}\\

\haiku{{\textquoteleft}Oe Grovaoder,}{en Gromoeder kan ie toch}{wel alles zeggen}\\

\haiku{{\textquoteleft}Veurzichtig vrouwe,{\textquoteright}, {\textquoteleft},...}{maande Reulefpas toch op}{laot mi\'en liever}\\

\haiku{Toen  hij bij de,.}{deur was deed hij of hem nog}{wat te binnen viel}\\

\haiku{We hebt altied nog ',!}{n rekening met oe te}{vereffen'n Reulef}\\

\subsection{Uit: Liefde}

\haiku{{\textquoteleft}Goed zoo{\textquoteright}, maar hij zoent.}{haar en zijn zoen glijdt als een}{veertje langs haar wang}\\

\haiku{Een kabouter staat:}{daar en draagt een muts met een}{belletje en zegt}\\

\haiku{Ze springt ergens over.}{heen en ligt weer op haar bed}{in het kamertje}\\

\haiku{En Lied moet hard op,.}{haar pink bijten want ze wil}{er naar luisteren}\\

\haiku{, en Lied zit bij haar,.}{Vader in de bank tusschen}{twee groote menschen in}\\

\haiku{Hanne neemt haar op.}{de arm en laat haar in het}{spiegeltje kijken}\\

\haiku{Ze kijkt ook door de.}{open deur in de kamer en}{er is daar geen vrouw}\\

\haiku{Hij draagt haar H{\'\i}j vindt.}{haar \'ook niet te groot en te}{zwaar om te dragen}\\

\haiku{Maar er moet nog wat -,,.}{bij een cijfer ze weet het}{haast alleen maar haast}\\

\haiku{Hij doet mal zijn hoofd,.}{heen en weer hij doet mal zijn}{schouders heen en weer}\\

\haiku{{\textquoteright} En ze mag zoo maar.}{met haar schoenen en al op}{Moeder's schoot zitten}\\

\haiku{Het paadje is wit, -.}{en glad ze glijen Lied wint}{het van Moeder}\\

\haiku{En Vader loopt ook,!}{om het hardst met Moeder maar}{Moeder wint het niet}\\

\haiku{En Vader pakt haar.}{op en zoent haar Hand in hand}{gaan ze het huis in}\\

\haiku{En ze heeft zwarte:}{handen en er zit stof in}{haar haar en ze zegt}\\

\haiku{De takkenbezem,.}{staat in een andere hoek}{en het stinkt naar roet}\\

\haiku{En Oom Luuk kijkt nog,}{altijd zoo erg en het is}{akelig dat hij zoo}\\

\haiku{De klaprozen hier,,.}{kent ze ook en ze kent ook}{de korenbloemen}\\

\haiku{En geen mensch zegt er,}{wat er is ook geen mensch}{om wat te zeggen}\\

\haiku{{\textquoteright} En ze pakt haar knieen {\textquoteleft}{\textquoteright},.}{beet en haar bak met boonen}{Mien kiend zegt Hanne}\\

\haiku{Ze schuift dichter bij,.}{Hanne ze legt haar armen}{op Hanne's  schoot}\\

\haiku{Een beest ritselt in,!}{de bladeren een eekhoorn}{misschien of een rat}\\

\haiku{Ze zitten achter.}{hun borden en er is al}{een beetje ruzie}\\

\haiku{Gister was ze toch,,.}{nog maar een klein kind vandaag}{niet vandaag niet meer}\\

\haiku{haartjes heeft ze aan.}{haar oogen en lange gouen}{haren om haar hoofd}\\

\haiku{Het prikt zoo in haar,,.}{oogen het prikt heel erg ze huilt}{toch alleen maar droog}\\

\haiku{Een leege zwarte boom.}{kijkt boos door de ruiten naar}{het kleine rooie vuur}\\

\haiku{Ze staat stil en loopt,.}{zoetjes verder ze loopt of}{ze in de kerk is}\\

\haiku{Vader moet hooren -?}{hoe goed ze zingen kan Hoort}{hij het wel Vader}\\

\haiku{Tante Belin windt.}{garen op de spoelen van}{de naaimachine}\\

\haiku{Ze kan dan nog maar,.}{met \'een oog kijken en het}{is toch wel genoeg}\\

\haiku{En als ze bij de,.}{paal met de duiventil is}{roept Hanne nog wat}\\

\haiku{Ze gaan die zwarte}{poort door en ze komen in}{een donkere gang}\\

\haiku{{\textquoteleft}Ja!, en dan m\'oet {\`\i}k!}{precies op tijd thuis wezen}{om jou open te doen}\\

\haiku{Lied loopt haar na in.}{de gang en de voordeur slaat}{vlak voor haar neus dicht}\\

\haiku{En Lied luistert naar.}{de wielen van die wagen}{tot ze niets meer hoort}\\

\haiku{Dan kijkt ze om in,.}{de gang en alles is nog}{kouer en stiller}\\

\haiku{En dan eet ze haar}{twee sneden bruin brood op en}{kijkt zoo naar Vader}\\

\haiku{Dat is niet tegen,.}{de reus Goliath dat is}{tegen Oom Reinhold}\\

\haiku{Ze houdt de klink van,.}{de deur stijf vast die klink is}{de klink van h\'aar huis}\\

\haiku{{\textquoteleft}Ik heb geen-eens een,,{\textquoteright}}{mantel \^an Moeder ik heb}{geen-eens me mantel}\\

\haiku{{\textquoteleft}W\`at Oom Reinhold?, w{\'\i}j,!}{hebben geen O\'om R\'einhold in}{de familie hoor}\\

\haiku{En ze luistert naar.}{zijn stappen tot het tikken}{zijn en nog langer}\\

\haiku{{\textquoteright} En ze kijkt naar de.}{bikkel en ziet niet goed dat}{het een bikkel is}\\

\haiku{Het is nacht en het,}{vlammetje van  de kaars}{trilt of het bang is}\\

\haiku{En ze trekt Lied haar.}{Zondagsche jurk aan en haar}{Zondagsche mantel}\\

\haiku{Dat zei hij toe', me - -.}{Vader me Vader die keek}{naar mij k\'eek naar mij}\\

\haiku{Ze zet boos-hard een}{steenen ketel met wijn op de}{kachel en boos-hard}\\

\haiku{En Oom's eene wang moet.}{er nog meer van glimmen dat}{Tante hem zoo prijst}\\

\haiku{En de lieve Heer.}{uit Anderst die is niet in}{de Noorder-kerk}\\

\haiku{Dat komt omdat ze.}{naar de schotsche juffrouw keek}{en niet naar de straat}\\

\haiku{De schotsche juffrouw:}{die knipoogde tegen Oom}{Louis  en ze riep}\\

\haiku{Vreemd zijn groote menschen.}{Lied kijkt naar Oom en Anne}{of ze haast niet durft}\\

\haiku{En ze hoort meteen.}{dat Oom de klant uitlaat en}{daar is ze blij om}\\

\haiku{En Lied blijft er haast {\textquoteleft},{\textquoteright},}{weer van stil staanKom schiet op}{zegt Oom achterom}\\

\haiku{En de doosjes met...{\textquoteright}}{vloeibare schoensmeer en de}{zak-spiegeltjes}\\

\haiku{Ze lacht meteen en.}{ze duwt het poesebontje}{nog meer achteruit}\\

\haiku{of hij het lekker}{vindt om Tante Alwine}{een arm te geven}\\

\haiku{Verleeen jaar was me, -...}{Vader er nog me Vader}{en en me Moeder}\\

\haiku{En in het bed zit.}{ze met opgetrokken knieen}{en wacht op Tante}\\

\haiku{{\textquoteright} Maar van allerlei.}{knal-geluiden in de}{straat wordt ze wakker}\\

\haiku{en langzaam-aan wordt -}{ze koud de zaal is warm en}{ze heeft heete thee}\\

\haiku{Haar hand gaat open en.}{dicht z\'oo of ze de tip van}{een jurk stijf vastgrijpt}\\

\haiku{Hard drukt ze haar knieen.}{tegen de onderkant van}{het bankvakje aan}\\

\haiku{{\textquoteleft}Louis, schei nou uit met,.}{dat eeuwige fluiten van}{je ik heb hoofdpijn}\\

\haiku{Ni\'et fluiten zeg,,}{ik en ni\'et rooken hier}{ik kan het nou niet}\\

\haiku{{\textquoteright} En ze praat dan over}{een heele hoop dingen die}{Lied niet vatten kan}\\

\haiku{En hij doet gauw die.}{brief open en leest alles en}{hij wordt zoo alleen}\\

\haiku{Het is dan toch of, -}{Vader zijn hoofd over haar schudt}{daar in de verte}\\

\haiku{En ze vergeet haast,}{de kruisbessen op te eten}{die ze krijgt Bekkie}\\

\haiku{En ze is  boos {\textquoteleft}{\textquoteright},, {\textquoteleft}!}{Is me dat uitblijven zegt}{Tanteeen schande}\\

\haiku{{\textquoteright} Lied hoeft dan enkel,.}{maar haar kin op haar borst te}{drukken anders niet}\\

\haiku{En na het standje.}{geeft Tante haar toch ook nog}{een blaadje papier}\\

\haiku{half-luid leest.}{ze hem en luid-op En dan}{kijkt ze Tante aan}\\

\haiku{Het is nou net of '.}{Vader nogs weer bij haar}{vandaan gegaan is}\\

\haiku{Er is met Kerst geen,.}{kaart gekomen van Vader}{geen brief en geen kaart}\\

\haiku{Ze knikt als Oom dan, {\textquoteleft}.}{zegt wat het is en het haar}{laat lezenJa Oom}\\

\haiku{En dan ineens ziet.}{ze ook weer een heele hoop}{van Oom en Tante}\\

\haiku{hier is vast een beuk.}{en die daar dat zal wel een}{eikenboom wezen}\\

\haiku{{\textquoteright} En zijn gezicht is.}{z\'oo vroolijk dat Tante er}{mopperig van wordt}\\

\haiku{En dan vergeet ze}{heelemaal dat Lied ook nog}{in de kamer is}\\

\haiku{Hij maakt een dikke {\textquoteleft}}{kin op zijn boord en zegt een}{paar vreemde woorden}\\

\haiku{{\textquoteright}, zegt Oom, hij neemt zijn,.}{hoed in zijn hand net of hij}{het op\'eens te warm krijgt}\\

\haiku{N\'ee Liedia Ulen, zei...{\textquoteright}}{Oom En Sjeuke die moest haar}{toen wel loslaten}\\

\haiku{Bekkie had eerst ook, {\textquoteleft}}{geglimlacht en later was}{ze giftig geweest}\\

\haiku{En hoe lang kan het,...?}{dan nog duren eer het zoo}{gaat als bij mij thuis}\\

\haiku{{\textquoteleft}Dat merk ik dan toch{\textquoteright}.}{allemaal Stil loopt ze wat}{later door de gang}\\

\haiku{Ze sjokt een beetje,.}{de tasch met boeken wordt al}{zwaarder in haar arm}\\

\haiku{{\textquoteright} Oom Louis en Tante.}{Alwine zijn nog niet thuis}{als ze terugkomt}\\

\haiku{Het donker wacht haar.}{zwart en geheimzinnig op}{in de bovengang}\\

\haiku{En ze doet of ze,...}{de krant inkijkt slaat nog een}{reken-schrift open}\\

\haiku{Ze draait zoo maar wat,.}{rond tuurt naar de oue gele}{prentjes aan de wand}\\

\haiku{Als het warmer wordt,}{heeft ze ook weer veel aan te}{merken op de stad}\\

\haiku{De huizen leunen}{met grijze avondmuren in}{de maanschemer weg}\\

\haiku{Er valt een beetje,,,...}{maanlicht over heen dat maanlicht}{leeft het ademt het trilt}\\

\haiku{Vader neuriet een,.}{oud kerk-vers een vers uit}{de tijd van Anderst}\\

\haiku{Nou stil maar{\textquoteright}, prevelt.}{ze Haar slappe rug trekt nog}{een beetje krommer}\\

\haiku{Als je jong  kijkt,.}{dan ben je toch al op weg}{om oud te worden}\\

\haiku{En op een avond komt.}{er ook weer een groote mand met}{cineraria's}\\

\haiku{Hij denkt er h\'eel wat{\textquoteright}.}{van En Lied weet niet wat ze}{daarop zeggen moet}\\

\haiku{En we moeten nou.}{ook n\`og later thuis  zien}{te komen dan hij}\\

\haiku{Geen firma-adres er,.}{op voor eventueele navraag}{en dit kaartje er bij}\\

\haiku{Ze moet toch altijd,}{wel haar handen uitstrekken}{naar de lieve Heer}\\

\haiku{Franske zal misschien.}{al de een of andere}{betrekking hebben}\\

\haiku{{\textquoteright}, vraagt Oom, {\textquoteleft}en zullen '?}{we vanavond alle dries}{naar de bioscoop gaan}\\

\haiku{Ze hoeft ook niet meer -.}{de stad in te gaan om Oom}{jaloersch te maken}\\

\haiku{En nou ben ik toch,.}{benieuwd wat voor menschen er}{komen opdagen}\\

\haiku{Bij een winkelraam}{staat ze stil en tuurt naar een}{pot met anemonen}\\

\haiku{{\textquoteleft}Laat de meid 's een,{\textquoteright}}{kop extra-sterke koffie}{brengen Alwine}\\

\haiku{Je hebt me Vader?}{ook gekend Bedoelde je}{daarstraks me Moeder}\\

\haiku{Er moeten toch een,.}{hoop dingen zijn waar z{\'\i}j ook}{over uit praten wil}\\

\haiku{Het is dan ook of:}{ze alles van zichzelf naar}{God kan opheffen}\\

\haiku{{\textquoteright} En ze moet meteen}{diep blozen en ze gaat gauw}{een stap achteruit}\\

\haiku{{\textquoteright}, denkt Lied, {\textquoteleft}oh - h\'emel...{\textquoteright},:}{Als Prosper dan weer naar zijn werk}{toe is zegt Tante}\\

\haiku{Het is de moeite...}{niet waard om  er nog \'een}{keer naar te kijken}\\

\haiku{{\textquoteleft}Het is toevallig{\textquoteright},, {\textquoteleft}.}{zeggen ze nu ook weermaar}{het komt net zoo uit}\\

\haiku{Hij stompt nu ook met.}{een stok op het ledikant}{Lied wil al opstaan}\\

\haiku{Och, zoo alleen{\textquoteright}, zucht {\textquoteleft},?}{zeHeeft Tante wat te doen}{dat {\`\i}k niet mag zien}\\

\haiku{Ik ben toch altijd,.}{te vroeg aan het raam om naar}{je uit te kijken}\\

\haiku{En Prosper lacht er om,.}{maar Valentijn Brunt lacht niet}{Valentijn Brunt zegt}\\

\haiku{Daar schrikt ze nou weer.}{erg van Ze fronst en ze wil}{er niet op ingaan}\\

\haiku{Op de hoek steekt hij}{zijn hand op Lied wil wuiven}{en hij is al weg}\\

\haiku{{\textquoteright} Hij glimlacht of hij, {\textquoteleft}{\textquoteright},}{voor de spiegel staat hij lonkt}{evenSchat zegt hij lauw}\\

\haiku{{\textquoteleft}Z\`eg nou dat hij best,,...{\textquoteright}}{een wandeling kan doen als}{hij wil z\`eg het toch}\\

\haiku{Ze is nu wel zoover.}{dat ze altijd goed weet waar}{ze mee bezig is}\\

\haiku{Eerst trekt hij zijn jas,,.}{uit eerst zet hij zijn hoed af}{dan komt hij binnen}\\

\haiku{Ik schrok...{\textquoteright}, en dan is.}{er iets in haar gezicht of}{ze gestoken wordt}\\

\haiku{Maar Valentijn Brunt...}{kijkt naar hem om of hij een}{hekel aan hem heeft}\\

\haiku{{\textquoteright} Zijn oogen worden klein {\textquoteleft}?,?,?}{van verteederingSmaakt het}{ja b\`en ik wel lief}\\

\haiku{Ze zegt ook {\textquoteleft}Ieder -?}{mensch heeft toch wel recht op een}{beetje geluk niet}\\

\haiku{Oom Louis...{\textquoteright} Maar ze wil {\textquoteleft}}{toch wel goedig-instemmend}{tegen Prosper knikken}\\

\haiku{{\textquoteleft}Als ik 's uit mijn, {\textquotedblleft}}{humeur durfde te wezen}{dan zei Oma Krunzel}\\

\haiku{naar die eenzame:}{buitengesloten avond en}{ze zegt in zichzelf}\\

\haiku{Maar ze strijkt toch even,.}{goed over haar haar ze trekt toch}{haar halskraagje recht}\\

\haiku{Het is of hij iets, {\textquoteleft}}{zegt waar ze eigenlijk niet}{naar luisteren moest}\\

\haiku{{\textquoteright}, soest ze Nu hangt die}{rare slaapnevel weer over}{haar gedachten heen}\\

\haiku{Het is een heele}{tijd stil Maar dat vinden ze}{allebei wel goed}\\

\haiku{En als ze dan bij,.}{hem is kan ze enkel maar}{naar hem luisteren}\\

\haiku{Want die moet ik dan,}{toch wel hebben het is nog}{een heele afstand}\\

\haiku{Valentijn Brunt komt,.}{en is stil en kijkt als uit}{de verte naar Lied}\\

\haiku{{\textquoteright} En ze is er een,}{beetje jaloersch op omdat}{ze zelf zoo warm is}\\

\haiku{En ze zakt zoo wit.}{en slap achterover of ze}{bewusteloos wordt}\\

\haiku{Hij gooit zich ook weer -.}{woedend om en om in zijn}{bed het kind krijscht}\\

\haiku{En hij zal vast nog{\textquoteright}.}{wel ergens een poos blijven}{zitten Elk woord schrijnt}\\

\haiku{Ze wil in haar lip}{bijten en laat dat. D\'aar geeft}{ze geen antwoord op}\\

\haiku{Misschien herinner.}{jij je er nog wel meer van}{dan ik weten kan}\\

\haiku{Je Moeder verviel,}{van kwaad tot erger toen ze}{al te ver heen was}\\

\haiku{{\textquoteleft}Al wat u vroeger,,?}{tegen Oom Louis zei dat gaf}{toch ook niks Tante}\\

\haiku{Je ziet er uit of.}{er nog maar de helft van je}{overgebleven is}\\

\haiku{die heeft me gezegd,}{wat er met Moeder is en}{hoe het daar ginder}\\

\haiku{Er is al lang een,.}{andere facturiste}{een knap Jodenkind}\\

\haiku{{\textquoteleft}Als mijn gezicht hem,...?}{obstinaat maakt dan moet ik}{toch anders kijken}\\

\haiku{Het is net of ik -?}{een heele poos weg geweest}{ben waar was ik dan}\\

\haiku{Hij keert zijn hoofdje,.}{naar zijn Moeder toe en hij}{kijkt ook lang naar haar}\\

\haiku{Maar als ze aan haar,.}{Moeder denkt krijgt ze zelfs een}{blos in haar voorhoofd}\\

\haiku{Z\'oo keek ik ook naar?}{Moeder en Zwisters Onthoud}{h{\'\i}j dit nou ook al}\\

\haiku{Je hebt er toch niets?}{op tegen dat ik naar die}{voetbal-match ga}\\

\haiku{Vandaag ziet ze er,,.}{knap uit zoo uitgerust haar}{oogen zijn zoo helder}\\

\haiku{En ze lacht met leege}{lippen tegen hem en brengt}{hem naar zijn bedje}\\

\haiku{Hij legt zijn schaap op,,.}{haar schoot zijn leege doosjes zijn}{leege garenklossen}\\

\haiku{Hij komt ineens op,,}{haar toe hij legt zijn handen}{zwaar op haar schouders}\\

\haiku{{\textquoteleft}Singe{\textquoteright} En ze zingt:}{van de drie Koningen met}{de ster En ze zingt}\\

\haiku{Maar ze glimlacht in, {\textquoteleft}}{zichzelf ze glimlacht over een}{pijn heen en ze zegt}\\

\subsection{Uit: Menschen uit een stil stadje}

\haiku{of die niet genog '.}{had an zoo'n presentje uit}{t sterfhuis as um}\\

\haiku{Verschrikt soesde ie...}{dat uit in de broeiwarmte}{van z'n hooge bedstee}\\

\haiku{een in 't donker ', ', '...}{z'n weg zocht naarm metn}{dreigingn vast plan}\\

\haiku{hoog-overbuischte...}{zeiltjes zaten d'r zonder}{schutvan-schaduw}\\

\haiku{n geld dat 't vr\`at,.}{en alles zonder baat geen}{ziertje beterschap}\\

\haiku{Hij schudde bedrukt...}{z'n ou\"e kop om de flarden}{net aan de steilen}\\

\haiku{{\textquoteleft}Kijk me dat 'r 's,...}{an haast heelemaal d'r uit}{en geen visch vanzelf}\\

\haiku{{\textquoteright} En dan ineens om ',.}{t nare geprik achter}{z'n oogen rauw lachend}\\

\haiku{{\textquoteright} {\textquoteleft}Tater..,{\textquoteright} rolde Sien's ', '.}{wulpsche stemr door hard van}{n tartende lach}\\

\haiku{'n Sikkie had ie, '.}{enkel en niks asn slap}{geite-sikkie}\\

\haiku{Geen flauw benul van.}{z'n fijn-proeverige}{artistiekiteit}\\

\haiku{{\textquoteleft}Ik poes 'm daaijeluk, ',.}{maar ik zeg je nog d'rs}{l\`ak an je nachtwacht}\\

\haiku{{\textquoteleft}En de eiere, ' '...?}{benne ook duur wou je nog}{rs zegge niet}\\

\haiku{Wat zel dat weer 'n,!}{kopere luchie worre weer}{gloeie och heer de oue}\\

\haiku{De kerel loopt met.}{veters en stukkies stinkzeep}{en pakkies naalde}\\

\haiku{Jij ben toch niet die?}{Meheer die bij me zuster}{heb weze vrage}\\

\haiku{Wil je 'n lekker,?}{schoteltje rolpens van me}{keurige rolpens}\\

\haiku{{\textquoteright} Deurtjes kraakten open '...}{enn zacht gesmoezel van}{vraaggeluid stak op}\\

\haiku{{\textquoteright} {\textquoteleft}O-o-o... me doosie{\textquoteright},...}{met mussies snotterde Gon}{want d'r was niet veel}\\

\haiku{- Kee lachte schorrig, '...}{uit in de stilten w\'el}{erg bitter lachie}\\

\haiku{D'r m\`ot toch wat op.}{te vinde weze om d'r}{achter te komme}\\

\haiku{Geen mensch doet 'r mooi.}{achter de scherreme en}{bij z'n eigen thuis}\\

\haiku{{\textquoteleft}D'r buite voor 't}{raam gluurde ferachtegies}{Oome Dries na d'r}\\

\haiku{En raak gepakt is......... '}{dat molentje d'r op dat}{hoogtentje zoo \"e}\\

\haiku{Maar de glimlach van. '.}{de zon was toch andersn}{Vecansieochtend}\\

\haiku{{\textquoteleft}Ja... stil nou 's Hein, '.}{je benne weer heelemaal}{doort dolle heen}\\

\haiku{Maar nou ineens wier ',... '...{\textquoteleft}}{t treurig lam treurign}{wijsie dat ze kon}\\

\haiku{{\textquoteright} {\textquoteleft}Ja-a{\textquoteright}, koersten Ceesie'.}{Randers plezierige oogies}{Fenne's richting uit}\\

\haiku{{\textquoteright} Peet's stem verkromp tot '...}{n steunend geluid en z'n}{gezicht trok valer}\\

\haiku{zoo van alles nog '...}{s opgehaald en ik d'r}{nog heelemaal in}\\

\haiku{Maar nou die wissel...,...?}{driehonderd pop waar most dat}{nou vandaan kome}\\

\haiku{Dat had d'r toe al... '}{dadeluk de heetigheid}{in d'r bloed gejaagd}\\

\haiku{{\textquoteright} - Had ze wezenluk '.}{niet eens geweten datn}{mensch dat ook nog had}\\

\haiku{Ze had d'r nou toch ',...}{ook al zoo dikwuls gevraagd}{opn koppie Jans}\\

\haiku{K\'o\'ope mensch, b\`e-je, ' '.}{niet wijsn cedeautje}{vann goeie kennis}\\

\haiku{En dan had ie ook ' '.}{nog wat gezeid vann ring}{inn varkenssnuit}\\

\haiku{wat beware voor ' '.}{n \`ander en wekenlang}{zelf opn drogie}\\

\haiku{later zalle de:}{moeders an d'r kindere}{kenne vertelle}\\

\haiku{och heer, allemaal...{\textquoteright}}{hebbe ze d'r bloei en d'r}{pronk en jij het niks}\\

\haiku{aj-je maar berouw, ',}{het Koos d'r komtt maar op}{an berouw mot je}\\

\haiku{Mooie oogen had ie of ',...}{de lieve Heer jer deur}{ankeek zukke oogen}\\

\haiku{as je ja meene, '.}{je hadt zeker met je}{zenuwpies te kwaad}\\

\haiku{Koert Huibers bij de ',...}{koffie en v\'o\'ort slape}{Koert voor en Koert na}\\

\haiku{{\textquoteleft}O mensch hou op, ik...}{doen d'r haast wat in me broek}{van \'o ha-ha-ha}\\

\haiku{gosterdankie, di\'e, '...?}{bij oome Dries op bed zou}{j{\'\i}jm dat gunne}\\

\haiku{Overal kwam ie 'r,}{dan mee te hulp z\'o\'o dat je}{je eige voelde}\\

\haiku{as vrouw zijnde voor ',,}{n heelemaal nakende}{buurman z{\'\i}j niet hoor}\\

\haiku{En dan die met die,.}{pulk haar op z'n knikker d\^a's}{dan vezelf Simson}\\

\haiku{{\textquoteright} Dries staarde in z'n.}{peinzen naar de meiden d'r}{vlugge werkdoening}\\

\haiku{Elk 'n riks voor de...{\textquoteright},.}{kermis schoof oome Dries die}{naar d'r-lui toe}\\

\haiku{Z'n voet tastte dan ',.}{onmiddelijk naarn steen}{maar ie bedacht zich}\\

\haiku{En onder de lamp '...}{s avens zullie saampies}{an d'r boterham}\\

\haiku{As 'k 'n vrouw over, '.}{de vloer neem mott me vrouw}{maar meteen weze}\\

\haiku{mollig ding, doen je ', '?}{t muisie doen jet dan}{venavend hee}\\

\haiku{h\'a\'ast...{\textquoteright} Als ze op z'n, '.}{deelneming wachtte had ze}{n teleurstelling}\\

\haiku{'k Vind Stiena n\'a\'ar..., '.}{o n{\`\i}ks geen moeite jongen}{k lees gr\'a\'ag luid op}\\

\haiku{'t Lischgoud aan de.}{slootjeskant wiegelde met}{vredig geruisch}\\

\haiku{of 'n reppende,...}{vlieg met overal de frischheid}{en maagdelukheid}\\

\haiku{verspille wil... nou,...?}{h\'o\'orde ie d'r altemet}{niet z\`at genoeg van}\\

\haiku{Als 'n man met 'n '.}{somber gezicht stond de lucht}{bovent water}\\

\haiku{Even door de strakke.}{gespannenheid van z'n ou\"e}{kop lichtte glimlach}\\

\haiku{zwaar boemde dan weer, '.}{met barstig gerinkel van}{ruitjest raam neer}\\

\haiku{Bij 't venster met ',.}{t uitkijkie op straat bleef}{ze omtreuzelen}\\

\haiku{{\textquoteleft}Wat hej-je toch,?}{allemaal van zilvere}{tressies malle meid}\\

\haiku{{\textquoteleft}ie was ommers bij...?}{de k\`olledokter en hier}{in z'n spr\'e\'ekkamer}\\

\haiku{{\textquoteleft}'k Gaan na me dood,{\textquoteright},}{krimmeneelde Toon zat ie}{te snotwolleve}\\

\haiku{{\textquoteleft}'k Zel ze effe,{\textquoteright}.}{bijlichte met de lange}{zes zei Klaas Krone}\\

\haiku{Nog voor d'r eigen '?}{n schijn van geluk ophieuw}{of voor de mensche}\\

\haiku{Ja z\'o\'o noeme de ',...}{menschet maar di\'e denke}{zoo'n beetje over d\`at}\\

\haiku{Nou, en Jans had d'r ' '}{welrs van te vore}{verteld dat ie zoo}\\

\haiku{'t Wordt je ook zoo ',?}{voorgehou\"e bijt trouwe}{dat weet je ommers}\\

\haiku{Te vol{\textquoteright} had Em d'r ' ' {\textquoteleft}...}{s gezeid en later nog}{d'rsVeel te bont}\\

\haiku{{\textquoteright} Lui stood ie op van,.}{de  divan rekkerig}{in z'n beklemming}\\

\haiku{Ach heer ja... zoo is '... ' '....}{t ent is je asn}{last opleid trouwe}\\

\haiku{{\textquoteright} Trui's denken hortte '.}{daar enn afschuw besloop}{d'r om wat ze dacht}\\

\haiku{{\textquoteleft}Ze had maar aldeur '...}{zoo'n rare droogte in d'r}{keel enn zeerte}\\

\haiku{nou de boel zoo de... '...}{hoogte in gaat wat g\'ale en}{n tros menielje}\\

\haiku{niks geen wijfie om, '...}{d'r's te knuffele om d'r}{s gr\'a\'ag te zoene}\\

\haiku{{\textquoteleft}'k Vin Gon ook niks,...}{voor j\'o\'u z{\'\i}j zoo'n sukkeltje}{en jij zoo'n flinke}\\

\haiku{Je bent gezond en...}{flink en je bent in staat je}{brood te verdiene}\\

\haiku{Hij neep z'n handen.}{tot logge vuisten en z'n}{stoel kraakte verdacht}\\

\haiku{D'r uit met 'n niet,,,'.}{ach ja je mot maar denke}{t gebeurt meer}\\

\subsection{Uit: Een menschenhart}

\haiku{Zijn Vader houdt hem.}{juist zoo vast dat hij Gabe's}{kant moet uitkijken}\\

\haiku{Obbe die ligt daar -.}{als een vertrapt dier in de}{modder en huilt huilt}\\

\haiku{{\textquoteleft}O{\textquoteright}, zei Roelien, {\textquoteleft}dat,.}{is uit pure wangunst dat}{de jongens dat doen}\\

\haiku{En koster van Tijn.}{rookt een neuswarmertje en}{klopt oue boeken uit}\\

\haiku{In een kapotte.}{teenen wieg bij haar ligt een dik}{bol zuigkind dat slaapt}\\

\haiku{Hij moet pinken en.}{hij wordt rood en hij kijkt een}{andere kant uit}\\

\haiku{Het orgel jammert.}{en het zingen jammert en}{de preek jammert ook}\\

\haiku{Iets in Gabe wil,...}{er bij zijn iets anders in}{hem wil ook weer niet}\\

\haiku{Gabe die mag soms ',.}{wel in de zaal komens}{avonds en soms ook niet}\\

\haiku{Met een zucht kijkt hij.}{dan maar weer naar de mannen}{om in het caf\'e}\\

\haiku{{\textquoteleft}Als ik jou niet had,,?}{lieverd dan was het ommers}{niks met me gedaan}\\

\haiku{Er zijn ook stukjes,:}{bosch bij sparren en dennen}{en eindjes bongerd}\\

\haiku{Heertje springt overeind.}{en hij schiet het huis in en}{komt ni\'et terug}\\

\haiku{Hij komt die dag nog.}{heel te Tenderloo en in}{het Hunteler bosch}\\

\haiku{En de haan van de,.}{Oosterkerk dat is een mooi}{endje bliksemlicht}\\

\haiku{{\textquoteright}, Gabe lacht met zijn.}{mond wijd-open en zijn oogen}{haast heelemaal toe}\\

\haiku{En wat ze op de,:}{boodschappen toe krijgen dat}{eten ze samen op}\\

\haiku{Er rijen wagens, -}{af en aan er gaan jongens}{voorbij op fietsen}\\

\haiku{Hij kijkt van opzij,.}{naar Aaike en ze loopen}{samen weer verder}\\

\haiku{Hij zet zijn hoedje,.}{op om het nog een keer af}{te kunnen nemen}\\

\haiku{Het maakt hem ineens,.}{klam en onrustig en een}{beetje kregel ook}\\

\haiku{Ze knijpt hem haast, ze,.}{kneedt hem zoo'n beetje ze wrijft}{hard over zijn armen}\\

\haiku{{\textquoteright} Hij schuift de ring van,.}{zijn centenbeurs en haalt er}{een dubbeltje uit}\\

\haiku{Ik had haar te kort.}{terug gegeven op de}{anijs die ze haalde}\\

\haiku{{\textquoteleft}Wat een kerelsgek{\textquoteright},, {\textquoteleft}.}{vit Gabeze geeft om}{iedere kerel}\\

\haiku{'s zomers hard werk.}{in de turf en in Holland}{maaien en hooien}\\

\haiku{Ze ruiken aan de,.}{sneeuw op een muur-richel ze}{likken er ook aan}\\

\haiku{Zware beenen krijgen.}{ze ineens en ze worden}{bloedrood allebei}\\

\haiku{{\textquoteleft}Ik had jou toch n\'ooit,,.}{alleen gelaten Aaike}{als dat gebeurd was}\\

\haiku{Dat de visschen niet,?}{stikken onder dat dichte}{ijs begrijp jij dat}\\

\haiku{H\`e, als wij twee\"en '{\textquoteright},, {\textquoteleft}.}{daars woonden zucht Aaike}{beslistwij samen}\\

\haiku{En vaak zeggen de.}{menschen ook een hoop dingen}{die ze niet meenen}\\

\haiku{Ze kijken elkaar,}{aan en ze lachen zoo maar}{en ze staan zoo maar}\\

\haiku{En Gabe heeft weer,.}{die zeere plek midden in}{zijn borst zit die plek}\\

\haiku{Maar in zijn oogen zit,.}{wat dat is nog scherper dan}{de punt van een naald}\\

\haiku{Hij kijkt toch of hij.}{Gabe graag een draai om zijn}{ooren zou geven}\\

\haiku{Haar bloote borst is weer.}{als een appeltje boven}{haar kleine handen}\\

\haiku{En hij komt met een,.}{paar groote stappen op Gabe}{af met springstappen}\\

\haiku{Maar Roelien zit in,.}{het caf\'e bij Johannes}{en de anderen}\\

\haiku{{\textquoteleft}Geef die kerel op,,.}{zijn mieter lieve God geef}{hem op zijn mieter}\\

\haiku{Ze weet dat hij rilt.}{en klappertandt en dat hij}{natte wangen heeft}\\

\haiku{{\textquoteleft}En dan moeten we,,?}{ook netjes wezen Gabe}{ben j{\'\i}j pas verschoond}\\

\haiku{{\textquoteleft}Jij was er ook bij,.}{toen die vent me achterover}{boog en weg smeet}\\

\haiku{Ze slaat de klep van,:}{haar hengselmand open en laat}{zien wat er in is}\\

\haiku{In Tenderloo is,.}{het bar-stil nog stiller}{dan te Alkerleik}\\

\haiku{Het piepende hek,:}{maken ze maar niet open ze}{klimmen er over heen}\\

\haiku{En ze geeft me een.}{zoen en ze zeit weer dat ik}{haar rechterhand ben}\\

\haiku{een zwarte zure,!}{kribbige deur een deur van}{enkel maar planken}\\

\haiku{Ze zitten daar net,,.}{te eten grauwe erten en}{spek en aardappels}\\

\haiku{En hij snuift zoo, dat,.}{zijn neus er bol van staat en}{dan ademt hij diep-uit}\\

\haiku{Dat jullie ons wel,.}{erg zouen zoeken en dat}{het toch wel naar was}\\

\haiku{Ze weten wel dat,.}{ze rijen maar ze weten}{niet precies waar}\\

\haiku{Ja, wat zit er al,,?}{niet in een mensch me jongen}{in een menschenhart}\\

\haiku{{\textquoteright} Maar in Weierlei.}{krijgt hij toch weer een naar droog}{gevoel in zijn keel}\\

\haiku{hij kijkt wel door zes.}{en dertig muren heen in}{de naaste toekomst}\\

\haiku{Het is al weer een,.}{aardig tijdje geleden}{dat hij van school kwam}\\

\haiku{{\textquoteright} Hij moet er n\'ou zelf.}{ook om grinniken als hij}{er aan terug denkt}\\

\haiku{En je ken het nooit,.}{weten hoe ver j{\'\i}j het nog}{brengt in de wereld}\\

\haiku{Als ze dan maar niet,!}{een van die zussen bij haar}{heeft en geen breiwerk}\\

\haiku{{\textquoteleft}Dat heb ik in de,,,}{drukte vergeten Sien och}{ik heb altijd zoo}\\

\haiku{Ze doet haar geld in.}{haar portemonnee en pakt}{de zak met gort op}\\

\haiku{Ze zeggen ook weer -.}{flauwe aardigheidjes over}{vrouwen de kerels}\\

\haiku{Ik dacht dat je eerst?}{je zussen en je broertjes}{naar bed brengen moest}\\

\haiku{zaaien in een tuin, -.}{\`onze tuin \`onze tuin al}{is hij nog zoo klein}\\

\haiku{Ze koopen warme,.}{krentebollen en spreken}{wat af over morgen}\\

\haiku{{\textquoteright} Het nauwe van die,.}{dag laat hem dan heelemaal}{los ook van binnen}\\

\haiku{Ja, een hart van hem,,.}{alleen een menschenhart en}{hij heeft het n\`og niet}\\

\haiku{En ze zijn over de.}{ophaalbrug gegaan zonder}{er bij te denken}\\

\haiku{muur- en kruiskruid.}{en herderstaschje en}{nieuwe grasscheuten}\\

\haiku{Ze schuift haar handen,.}{verder  over zijn rug ze}{staat dichter bij hem}\\

\haiku{{\textquoteright} En er is nog meer.}{glans om haar voorhoofd en nog}{meer glans om haar oogen}\\

\haiku{Aaike die neemt het,:}{voorjaar mee in haar oogen in}{de reuk van haar kleeren}\\

\haiku{{\textquoteleft}Ik zing om Aaike,,!}{ik zing om het voorjaar en}{nergens anders om}\\

\haiku{{\textquoteright} De zon slaat uit zijn.}{oogen en uit zijn lach en uit}{zijn rood sterk gezicht}\\

\haiku{acht gulden heeft hij {\textquoteleft}?}{al.Kan ik nou nog niet bij}{je an huis komen}\\

\haiku{De bloemen hebben,...}{gekleurde vleugeltjes en}{de wolken witte}\\

\haiku{En om tien uur 's,,:}{avonds of misschien al om kwart}{v\'oor tien dan zegt hij}\\

\haiku{{\textquoteright} Hij loopt zacht over die,:}{dikke rulle grasranden}{daar in Berkenhart}\\

\haiku{De kamperfoelie.}{en de vlier fonkelen of}{ze van groen glas zijn}\\

\haiku{{\textquoteright} Hij wacht maar - en al.}{wachtend begint hij weer over}{zichzelf te praten}\\

\haiku{Ze neemt zijn hand en.}{drukt er aan de binnenkant}{een lange zoen op}\\

\haiku{{\textquoteright} En Johannes die,,.}{strijkt hem over zijn haar ja die}{strijkt hem over zijn haar}\\

\haiku{Hij drukt zijn vingers.}{tegen zijn mond aan en kijkt}{omhoog in de lucht}\\

\haiku{Hij praat heesch, hij.}{staat daar als een bedelaar}{die om een cent vraagt}\\

\haiku{Want we moeten er,.}{schot achter zetten zooveel}{tijd is er niet meer}\\

\haiku{Nou - n\'ou ben jij ook,,, -,.}{haast me Vader Brunt zeg Brunt}{d\`ag tot vanavond Brunt}\\

\haiku{Van het caf\'e hier -.}{ken ik kennen ook geen twee}{gezinnen leven}\\

\haiku{{\textquoteright} En Gabe die valt,.}{tegen hem aan of hij hem}{omver wil loopen}\\

\haiku{{\textquoteright} Hij snikt er nog bij,,.}{hij moet na-snikken als een}{kind die Johannes}\\

\haiku{Aaike knielt d\'aar ook,.}{neer onder de zegenende}{handen van Jezus}\\

\haiku{{\textquoteright} En Gabe is er:}{dan opeens niet te groot voor}{om te mompelen}\\

\haiku{Aaike en Gabe.}{knielen op het mos tusschen}{de boomwortels in}\\

\haiku{Heel Alkerleik is,.}{midden in de zomer grauw}{en  glimmerig}\\

\haiku{Hier was je een keer,,,'.}{op een avond Aaike toe had}{ik je weggebracht}\\

\haiku{{\textquoteright} Ja, en wat is het,?}{waar het dan op aan komt in}{dat heele leven}\\

\haiku{{\textquoteright} Johannes' haar lijkt,.}{witter te worden omdat}{zijn kop rooier wordt}\\

\haiku{Gabe heeft geen erg,.}{in Roelien hij heeft haast nooit}{meer erg in Roelien}\\

\haiku{{\textquoteleft}Zit ze daar dan niet?,?}{met een schram op haar wang en}{een plek op haar kin}\\

\haiku{Hij maakt zoo-maar,.}{een keelgeluid het klinkt een}{beetje minachtend}\\

\haiku{Ik laat je niet los,...{\textquoteright} {\textquoteleft},,.}{voor jeOh ja Gabe laat}{me ook maar niet los}\\

\haiku{{\textquoteright} En dan praat  hij,,.}{ineens schor en hard ja hij}{praat ook schor en hard}\\

\haiku{Aaike, lieve God,?}{dat is toch immers de slag}{van mijn eigen hart}\\

\haiku{Een keer toen hij in,.}{een spiegeltje zijn lachend}{gezicht zag schrok hij}\\

\haiku{Gezien heeft hij het,,.}{toch wel ja gezien heeft hij}{dat alles toch wel}\\

\haiku{Hij knijpt zijn handen.}{stijf om een scherp dingetje}{heen in zijn jaszak}\\

\haiku{Maar hij draagt een fijn.}{opknapperspak en hij rookt}{een fijne sigaar}\\

\haiku{En haar bovenlip.}{heeft ze haast heelemaal naar}{binnen gebeten}\\

\haiku{vanavond is het of.}{hij na een lange zwerftocht}{thuis gekomen is}\\

\haiku{hij drukt zijn knie\"en.}{hard tegen de tafelrand}{aan onder het blad}\\

\haiku{Ze licht het deksel,.}{op van die pan zonder in}{die pan te kijken}\\

\haiku{Groot staat de diepe.}{kom-van-de-lucht over}{alle dingen heen}\\

\haiku{Hij weet het, hij heeft.}{het gezien toen ze onder}{de trouw-eik waren}\\

\haiku{Hij zit op de bok.}{onder de huif maar stil voor}{zich uit te turen}\\

\haiku{{\textquoteright} Sander schuift zijn pijp.}{van zijn eene mondhoek naar zijn}{andere mondhoek}\\

\haiku{{\textquoteleft}Perzik{\textquoteright}, mompelt hij -.}{dan meteen net of hij wat}{goed te maken heeft}\\

\haiku{En makkelijker.}{en mooier dat kennen we}{ook niet betalen}\\

\haiku{Het kon toch nooit wat,.}{raars worden nooit iets om er}{over te grinniken}\\

\haiku{Hij moet denken dat.}{hij zijn kin heen en weer wrijft}{in haar  haartjes}\\

\haiku{En Juffrouw Gees is.}{al voor drie-vierde part}{van de aarde weg}\\

\haiku{{\textquoteleft}Ja{\textquoteright}, denkt Gabe, {\textquoteleft}hier,,...{\textquoteright}}{zitten we nou de baas en}{ik twee oue kerels}\\

\haiku{Ze hebben het over,.}{een nieuwe Fongers over een}{fietstocht naar Friesland}\\

\haiku{De oue schutting wordt,.}{daar weggebroken er komt}{een betonnen muur}\\

\haiku{Hij staat daar tot het,.}{donker wordt daar boven tot}{het venster dicht gaat}\\

\haiku{Moeder, bid je wel,,'...?}{ooit zeg Moeder denk je wel}{an onz lieve Heer}\\

\haiku{Er komt nou altijd.}{een beklemming over hem als}{hij op huis toeloopt}\\

\haiku{En Aaike, die houdt.}{nou nog altijd mijn leven}{in het rechte spoor}\\

\haiku{Haar gezicht is als,.}{een knoet gekorven hout het}{is al strak en dood}\\

\haiku{Alleen menschen die,.}{ergens mee zitten hebben}{zoo'n vluchtende blik}\\

\haiku{De wind omvat hem,,.}{beweegt zich als ademend legt}{zich tegen hem aan}\\

\haiku{Wat later als de,.}{mannen weg zijn gaat hij naar}{boven naar Roelien}\\

\haiku{En \`als het haar eind,.}{zou verhaasten dan zou het}{haar pijn verkorten}\\

\haiku{Maar Sander is nog,.}{niet zoover dat hij zeggen kan}{wat hij zeggen moet}\\

\haiku{Dat was niet wat van,.}{een man en een vrouw dat was}{wat van twee menschen}\\

\haiku{Iemand gooit op een.}{dag een paar centen in de}{muziekautomaat}\\

\haiku{Gabe hakt hout op.}{de binnenplaats en hij zet}{daar al zijn kracht op}\\

\haiku{En dan ineens - waar? -.}{vandaan en hoe zegt Aaike}{iets in de verte}\\

\haiku{Jentje heeft haar kind,.}{al een poos een mooi stevig}{voorlijk kind is het}\\

\haiku{Je kan nooit weten, -.}{waar het goed voor is wacht nog}{wat dat is alles}\\

\haiku{{\textquoteright} Johannes krijgt een,,.}{hoestbui hij heeft adem-nood}{hij komt adem te kort}\\

\haiku{Hij zit niet in de,.}{deur aan de straatweg hij zit}{in de vogelknip}\\

\haiku{{\textquoteleft}Nee, ik had - wijzer{\textquoteright},, {\textquoteleft}.}{moeten wezen denkt hijik}{had niet moeten gaan}\\

\haiku{En Gabe weet niets.}{terug te zeggen  en}{niets terug te doen}\\

\haiku{Hij betast haar en,}{kreunt hij neemt haar op en draagt}{haar naar het licht toe}\\

\haiku{Ineen geknepen.}{zit ze in het holst van de}{nacht in de keuken}\\

\subsection{Uit: Naakte waarheid}

\haiku{Al dat gedaas over,.}{de moderne jeugd het is}{gewoonweg absurd}\\

\haiku{Mama begrijpt niets,.}{als het haar niet met ronde}{woorden gezegd wordt}\\

\haiku{Fijntjes glimlacht ze,,.}{wiegelt met haar eene been en}{houdt het hoofd wat scheef}\\

\haiku{Ze heeft zin om te,.}{joedelen om een schop in}{de lucht te geven}\\

\haiku{Mijnheer Bax staat nog,.}{even in de deur hij praat met}{een grommel-stem}\\

\haiku{Ja gut... is nooit  ,.}{over gesproken misschien als}{ik van school af ben}\\

\haiku{{\textquoteright}, vorscht hij door, {\textquoteleft}vin'?}{je het hier in Wensveld ook}{zoo'n bezopen boel}\\

\haiku{{\textquoteleft}Maar ik ga toch niet{\textquoteright},, {\textquoteleft}....}{zegt ze in zichzelfik ga}{niet met die nieuwe}\\

\haiku{{\textquoteright} Voor tijdverdrijf laat.}{ze haar lippen spelen met}{een rilgeluidje}\\

\haiku{{\textquoteright}, Liz maakt een schimpend, {\textquoteleft},.}{keelgeluidverlakkerij}{k\`an natuurlijk niet}\\

\haiku{{\textquoteright} Hij rukt de knoopen,.}{los hij wil haar vasthouden}{onder haar mantel}\\

\haiku{Op elke plek hier.}{in de laan zou een misdaad}{kunnen gebeuren}\\

\haiku{Het roode teeken{\textquoteright} {\textquoteleft}{\textquoteright}.}{denken en aanDe moord in}{de Kastanjelaan}\\

\haiku{Maar wat kan er toch?}{zoo stinken in lekkere}{bittere cacao}\\

\haiku{Liz kan er soms van.}{alles uitflappen en men}{kan er niets aan doen}\\

\haiku{{\textquoteleft}Ja, voorkomend moet,.}{ze zijn van Mama ook als}{ze er niets van meent}\\

\haiku{{\textquoteright} Even houdt hij nog zijn,.}{rimpelig grijnsmaskertje}{voor dan zakt het weg}\\

\haiku{Daar heb je het nou{\textquoteright},, {\textquoteleft}.}{weer denkt zealle jongens}{willen die kant uit}\\

\haiku{{\textquoteleft}Niet doen{\textquoteright}, mompelt ze.}{met een vakerige stem}{tegen zijn handen}\\

\haiku{{\textquoteleft}Het eten wordt koud{\textquoteright}, zegt, {\textquoteleft},....}{Liz tegen mijnheer Oscar's}{handenz\`eg het \'eten}\\

\haiku{Breede schouders heeft,.}{hij en zijn handen zijn veel}{te sterk voor zijn vak}\\

\haiku{{\textquoteleft}Dag{\textquoteright}, ze duwt hem haar,, {\textquoteleft}...!}{groet toe ze is een en al}{aandrangz\`eg d\`a-\`ag}\\

\haiku{{\textquoteright} Och ja, Coby moet.}{zien hoe aardig het toegaat}{bij hen aan tafel}\\

\haiku{{\textquoteleft}Ja, net als u, h\`e?,,?}{Papa wij zijn erfelijk}{belast niet Papa}\\

\haiku{{\textquoteright} Liz vindt dat alles.}{te onbenullig om er}{op te antwoorden}\\

\haiku{Duco vat het best. {\textquoteleft},{\textquoteright},, {\textquoteleft},?}{Oh \`al zestien grinnikt hij}{heele leeftijd h\`e}\\

\haiku{Ze drukt haar smalle.}{kinderlijke handen in}{vuistjes op de borst}\\

\haiku{En ze is zoo klein.}{in haar gevoel en zoo klein}{in werkelijkheid}\\

\haiku{En ze sluipt als een,,.}{poes niets gooit ze om nergens}{stoot ze tegen aan}\\

\haiku{{\textquoteleft}Stond je daar naar de,?,?}{lichtjes te kijken h\`e stond}{je daar te droomen}\\

\haiku{Het is dus van het.}{grootste belang welk vak zij}{uitoefenen zal}\\

\haiku{{\textquoteright} Liz ademt beklemd, en.}{haar handen worden vochtig}{aan de binnenkant}\\

\haiku{Mama laat zich ook.}{vaak genoeg uithooren door}{Tante Petertje}\\

\haiku{{\textquoteright} Keurig spreidt Mama.}{Richie's fijne Kasjmirdoek}{over de divan uit}\\

\haiku{Bijna slordig droogt,}{ze het marmeren blad af}{en gehaast plaatst ze}\\

\haiku{{\textquoteleft}Zeg 's, ik geloof,?}{dat jij alles van mij staat}{af te gluren niet}\\

\haiku{Maar het spijt Mama.}{nu dat ze haar handen niets}{te doen kan geven}\\

\haiku{Maar haar gedachten,.}{geven haar diep van binnen}{een kneep telkens weer}\\

\haiku{{\textquoteright} En ze wil dan nog,.}{veel scherper uitvaren maar}{daar komt ze niet toe}\\

\haiku{Oma gaf mij een en,.}{ander Oma had nu eenmaal}{zoo iets als voorkeur}\\

\haiku{Veel woorden zal ik,.}{niet aan je verspillen d\`at}{in de eerste plaats}\\

\haiku{{\textquoteright} En nu doet Pa als,.}{Duco hij schuift in zijn stoel}{langzaam dichterbij}\\

\haiku{{\textquoteleft}W\`el...{\textquoteright}, haalt ze uit met}{een donzig stemmetje en}{haar vingertoppen}\\

\haiku{{\textquoteright} En Liz kijkt naar hem.}{om of ze hem op staande}{voet vermoorden wil}\\

\haiku{Moet verdorie ook.}{nog een staatje opmaken}{in het magazijn}\\

\haiku{Maar zij glimlachen.}{niet z\'oo of het van begin}{tot eind onzin is}\\

\haiku{Onze Duuc die heeft ',,...}{nogs is het niet Pa haast}{een elf gehad voor}\\

\haiku{de leeraar had,,...{\textquoteright}}{toch gezegd het was meer dan}{een tien dat hem Pa}\\

\haiku{{\textquoteleft}Coby Duker heeft{\textquoteright},, {\textquoteleft}.}{opgezegd vertelt Mama}{erg spijt het me niet}\\

\haiku{Trouwens de heeren...{\textquoteright},.}{die ik nu heb ze stokt en}{ze laat het er bij}\\

\haiku{En je hebt nog al,?}{veel last van houtworm in je}{meubels is het niet}\\

\haiku{Ik zag daar net hoe.}{je meid aan het spatten was}{met de waterkraan}\\

\haiku{Liz gaat toch maar naar.}{haar uitkijk-post terug}{aan het erkerraam}\\

\haiku{{\textquoteright}, moedert ze, {\textquoteleft}of is?}{er mogelijk nog een klein}{priv\'e verlangen}\\

\haiku{hij uitgaat, dan - dan.}{kan hij niet genoeg op mijn}{mooie dingen letten}\\

\haiku{{\textquoteright} Zij heeft haar peignoir.}{al weer opgevouwen en}{rangschikt de cadeaux}\\

\haiku{En mijnheer Oscar,,.}{reikt haar met beide handen}{alles ineens over}\\

\haiku{Zij gaat er verbluft,.}{bij zitten midden op het}{kleed naast de koffer}\\

\haiku{Liz luistert er naar.}{en trekt de schouders even op}{of ze het koud heeft}\\

\haiku{Gek dat Tante Juup.}{nog altijd niets gestuurd en}{niets geschreven heeft}\\

\haiku{Daar staat de koffer,.}{een log onpractisch ding met}{koperen hoeken}\\

\haiku{{\textquoteleft}Moeder zou mij ook.}{nog zoo graag een horloge}{gegeven hebben}\\

\haiku{{\textquoteleft}O ja - ja...{\textquoteright}, knikt ze,.}{slapjesterloops en laat haar}{knie\"en dansknikken}\\

\haiku{Je koffer mocht er '.}{nogs niet zijn en denk aan}{je tandenborstel}\\

\haiku{Men kan er niet uit,.}{wijs worden wat zij denkt wat}{zij voornemens is}\\

\haiku{{\textquoteleft}Tien uur{\textquoteright}, deelt hij ook ',, {\textquoteleft}.}{nogs mee met een stroeve}{blik op Lizbedtijd}\\

\haiku{Ze trekt, nog altijd,.}{nadenkend haar kousen uit}{en zoekt haar pon op}\\

\haiku{{\textquoteright} Papa zat onder,}{de dennetakken met de}{kunstsneeuw te flirten}\\

\haiku{Ze rillen overluid,.}{klappertanden overluid en}{stennen zoo maar wat}\\

\haiku{rozig-glad zijn ze,.}{met keurige maantjes en}{zorgvuldig gevijld}\\

\haiku{Als ze wat later,.}{zoekend opkijkt treft hen geen}{bestraffende blik}\\

\haiku{Maar als ze haar groote.}{blauwe oogen opslaat is ze}{heelemaal anders}\\

\haiku{haar Moeder is van,.}{een muis geschrokken toen ze}{in positie was}\\

\haiku{Ka Kool, tegenover,.}{haar is onbehoorlijk lang}{en leelijk en braaf}\\

\haiku{De boekentasschen -.}{staan al ingepakt bij de}{deur sinds gisteravond}\\

\haiku{Och wat!, Ilsevoort.}{hoeft enkel op mantels en}{hoeden te letten}\\

\haiku{, j{\'\i}j bent altijd veel...,!}{te bescheiden Madame}{La Polyandrie}\\

\haiku{Maar voor een raar stroef,.}{gevoel in zichzelf heeft ze}{enkel een grijns over}\\

\haiku{Een afgerichte{\textquoteright},, {\textquoteleft}.}{kever dan wijzigt zeeen}{kermis-kever}\\

\haiku{{\textquoteright} En onderhand wordt.}{ze voortgeduwd zonder dat}{iemand haar aanraakt}\\

\haiku{Zij zal een groote flesch!?}{lilas koopen of jasmin}{of violette}\\

\haiku{Roos de Wit achter,.}{hen gichelt om het een of}{ander en fluistert}\\

\haiku{En de oudste vroeg {\textquoteleft}{\textquoteright},.}{haarvoor een baantje en de}{andere vroeg B\'e}\\

\haiku{Zij weet nu ineens.}{wat ze gezocht heeft in de}{dag die v\'oor haar ligt}\\

\haiku{{\textquoteright} Verrukkelijk - en,,.}{daarna als Ot en Ed klaar}{zijn mag zij crepeeren}\\

\haiku{{\textquoteright} Een doorzichtige,.}{aansporing is dat. Och ja}{zij is zijn dochter}\\

\haiku{{\textquoteright}, hij kucht tegen iets, {\textquoteleft},?}{zenuwachtigs in de keel}{zooals jij daar zat h\`e}\\

\haiku{, zaagt hij nu in de, {\textquoteleft}.}{verkeerde richting doorheb}{je geen vacantie}\\

\haiku{{\textquoteleft}Ik wil net zoo vrij,,,.}{zijn als jij zelf Papa om}{te doen wat ik wil}\\

\haiku{Er is iets in Liz'.}{houding en stem dat haar tot}{zijns gelijke maakt}\\

\haiku{Papa neemt zijn bril.}{af en bekijkt de glazen}{aan beide kanten}\\

\haiku{{\textquoteright} Hij moet nog even zijn.}{natte oogen afvegen en}{wat nakuchen en slikken}\\

\haiku{Hij denkt ergens over,.}{na hij  heeft nog iets te}{bespreken met haar}\\

\haiku{Aanhoudend loopt zij.}{daar over door te piekeren}{in haar gedachten}\\

\haiku{En als Papa het,.}{wist zou het zijn oordeel over}{haar niet verzachten}\\

\haiku{knikt bijna oolijk,.}{en het wasachtige glijdt}{weg van haar gezicht}\\

\haiku{Emiel Kan praat... Emiel Kan.}{voert een dubbel-gesprek}{met oogen en woorden}\\

\haiku{Elk detail van zijn.}{uiterlijk neemt ze met groote}{aandacht in zich op}\\

\haiku{{\textquoteright} En Emiel Kan voert zijn.}{dubbel-gesprek weer met}{haar en met Papa}\\

\haiku{{\textquoteright} Zij kijken elkaar,.}{aan schijnbaar terloops en toch}{van heel nabij}\\

\haiku{{\textquoteleft}O ja{\textquoteright}, overdrijft ze, {\textquoteleft},.}{geniale aanleg ben}{ik mee geboren}\\

\haiku{Het is onprettig.}{warm in het werkkamertje}{en onprettig stil}\\

\haiku{Van je verstand moet{\textquoteright},.}{je het hebben zegt Liz met}{iets van een glimlach}\\

\haiku{{\textquoteright} En dat woord klinkt zoo.}{nuchter in haar mond als een}{telefoonnummer}\\

\haiku{En het wrijft zich met.}{een lustgevoel tegen de}{zware zijde aan}\\

\haiku{Drie garnituren,,.}{heeft zij nu een rose een}{wit en een lila}\\

\haiku{Ze doet haar lippen.}{van-een en spant ze glad om}{het tandvleesch heen}\\

\haiku{{\textquoteleft}I 'll make a,,.}{string of pearls out of the}{dew for you for you}\\

\haiku{{\textquoteright} Maar daarmee verdwijnt.}{het kwaadaardige gevoel}{nog niet heelemaal}\\

\haiku{{\textquoteright} {\textquoteleft}Black and white{\textquoteright} schiet.}{even later als een spin in}{een web op hen toe}\\

\haiku{Hij  transpireert,.}{hevig vloekt er in stilte}{om en glimlacht zuur}\\

\haiku{Die  zijn handig,{\textquoteright},, {\textquoteleft}...{\textquoteright} {\textquoteleft}?}{wou ik al lang fluistert ze}{doe ik zelfErg duur}\\

\haiku{Er staat een practisch:}{schrijfbureau met verwaande}{presse-papiers}\\

\haiku{Bedaard steekt ze een,.}{nieuwe sigaret aan en}{glimlacht als Papa}\\

\haiku{Dat voel je dan toch{\textquoteright},, {\textquoteleft},?}{zegt  zeof het gaat of}{je er klaar voor bent}\\

\haiku{{\textquoteright}, zijn handen glijden, {\textquoteleft}?}{mijnheer-Oscar-achtig over}{haar heendat toch wel}\\

\haiku{Ze kijkt met halve,.}{oogen over zee uit soms kijkt ze}{ook met een kwartoog}\\

\haiku{Leuk is het om die.}{ge\"etaleerde menschen}{zoo te zien zitten}\\

\haiku{Mama stilletjes,.}{staan waar ze staat en wandelt}{om het terras heen}\\

\haiku{Stom vervelend{\textquoteright}, denkt,.}{ze en geniet hevig van}{de belangstelling}\\

\haiku{Beredderig doen,.}{zij geen van allen en ook}{niet geagiteerd}\\

\haiku{En de anderen.}{wachten onwillekeurig}{of er nog iets komt}\\

\haiku{Duuc grijpt per abuis naar,.}{zijn bierpot er is nog maar}{een sliertje schuim in}\\

\haiku{Daar zal dan nog 's.}{een vakkundige aan te}{pas moeten komen}\\

\haiku{{\textquoteleft}Ik slaap daar benauwd{\textquoteright},, {\textquoteleft}...}{mokt Ducoeen kamertje}{als een stijfselkist}\\

\haiku{Door de oogen van de.}{man met de groote hoed lijken}{vonken te springen}\\

\haiku{Nuchter bekijkt ze,.}{zichzelf nuchter controleert}{ze haar gedachten}\\

\haiku{Ze zal aanstonds even,.}{aan de deur luisteren eer}{ze naar binnen gaat}\\

\haiku{Handig opent en sluit,.}{ze de deur en wipt luchtig}{de steile trap af}\\

\haiku{Heb afspraakjes bij, '.}{bosjes gemaakt maar je wilt}{wels wat anders}\\

\haiku{Ze hebben vieze.}{neuzen en afzakkende}{broekjes en kousen}\\

\haiku{{\textquoteleft}Wij houden alles - '.}{stijf vast Papa brengt het nog}{wels tot Koster}\\

\haiku{De voeten glijden.}{uit in het verschuivende}{zand van de helling}\\

\haiku{{\textquoteleft}Een goeie pot bier was{\textquoteright}.}{dat. Liz kruipt in haar mantel}{of ze het koud heeft}\\

\haiku{Het zomer-strand -.}{is sexualiteit op een}{presenteerblaadje}\\

\haiku{{\textquoteright} Wat later zanikt.}{Krillertje over een dame}{die Elise Bock heet}\\

\haiku{Sloom kijkt ze om en.}{lacht uitbundig maar met een}{raar hik-geluid}\\

\haiku{{\textquoteright} Ed en Ot staan er,,.}{gereserveerd bij droog wijs}{en Duco-achtig}\\

\haiku{Het is drukker en,,.}{jolijtiger dan anders}{nu het is voller}\\

\haiku{Zij bedrinken zich,.}{aan de muziek zij gaan er}{zich aan te buiten}\\

\haiku{De avondlucht valt frisch,.}{op haar warm gezicht op haar}{vochtige dijen}\\

\haiku{Twee weken - veertien -.}{lange dagen en nachten}{is dit nu al zoo}\\

\haiku{Hij wou op-laatst wel,.}{graag weer naar school en niet zoo}{zeer om de school zelf}\\

\haiku{{\textquoteright} Zijn stappen gaan niet.}{z\'oo ver-weg dat ze}{uitsterven kunnen}\\

\haiku{{\textquoteleft}Hou je zelf toch niet,.}{zoo krampachtig vast zielig}{burgermannetje}\\

\haiku{Daar is het altijd{\textquoteright},, {\textquoteleft}.}{mee begonnen denkt zemet}{een zoen op mijn kruin}\\

\haiku{Een lief jongetje{\textquoteright},,.}{ben je wil ze zeggen maar}{ze bezint zich nog}\\

\haiku{Hij ziet weer een klein,.}{huis met een stroodak aan de}{zonkant van een pad}\\

\haiku{{\textquoteleft}Ja{\textquoteright}, fluistert ze in, {\textquoteleft} {\textquotedblleft}{\textquotedblright} -?}{een lachen daten zoo is}{het voornaamste niet}\\

\haiku{{\textquoteleft}Daar heb je nou een -!}{vriendin voor noodig en dan nog}{wel een getrouwde}\\

\haiku{{\textquoteleft}als je daar bang voor,.}{bent zal ik mijn kamerdeur}{wel even afsluiten}\\

\haiku{En Emiel plukt aan Liz'.}{kanten halskraagje en streelt}{over haar achterhoofd}\\

\haiku{Nu zijn we aan een.}{schriftelijke cursus in}{het Duitsch begonnen}\\

\haiku{{\textquoteright} Hij bijt op zijn vuist,, -.}{wordt bleeker en komt toch}{terug elken dag}\\

\haiku{Bovendien heeft ze,.}{een beetje echte pijn een}{beetje echte angst}\\

\haiku{Ze rilt daarbij op,.}{twee manieren gekunsteld}{en ongekunsteld}\\

\haiku{{\textquoteleft}Dat was ook voor de,.}{eerste maal en wat had ze}{toen al niet beleefd}\\

\haiku{Hij schuift zijn handen.}{in zijn broekszakken en buigt}{zich lachend voorover}\\

\haiku{En de routes van.}{de trams weet ik ook nog niet}{en de nummers niet}\\

\haiku{{\textquoteright} Liz begrijpt hem best. {\textquoteleft},?,?,?}{Dank je is het goed met ze}{met Mama en Duuc}\\

\haiku{Och, bent u Lizzy's?,,?}{Oom geeft u toch uw hoed hier}{wilt u plaats nemen}\\

\haiku{{\textquoteright}, Oscar tast al in,,.}{zijn zak links rechts en diept een}{keurig pakje op}\\

\haiku{In de schemer lijkt -.}{ze een groot kind in het licht}{een gevallen vrouw}\\

\haiku{Vroeger waren de,.}{mannen trouw uit sleur nu zijn}{ze ontrouw uit sleur}\\

\haiku{En ze weet dat ze,.}{naast de dikke Lowis zit}{Lowis-de-Jood}\\

\haiku{Maar Lot Kreevelt trekt,.}{hem naast zich op de stoel die}{Herfst verlaten heeft}\\

\haiku{{\textquoteright} Gehoorzaam doet ze {\textquoteleft} -...{\textquoteright},, {\textquoteleft}.}{dat.Men moest dat niet hakkelt}{zemet iedereen}\\

\haiku{Strak kijkt ze naar de.}{dingen om haar heen en ze}{ziet ze toch niet goed}\\

\haiku{{\textquoteleft}Als God{\textquoteright}, herhaalt ze,.}{en strijkt met de kin langs de}{fijne bloembladen}\\

\haiku{Stilletjes wascht.}{ze een paar kousen uit in}{haar toilet-emmer}\\

\haiku{{\textquoteleft}W\`erken{\textquoteright}, herhaalt, {\textquoteleft}.}{Lot op een gewichtige}{toonmoet w\`erken}\\

\haiku{Maar je moet Liz vrij,.}{laten ze moet niet voelen}{dat je op haar let}\\

\haiku{{\textquoteright} {\textquoteleft}Precies, daarom{\textquoteright}, ze, {\textquoteleft}?}{knikten ben je nu nog bang}{om te verliezen}\\

\haiku{{\textquoteleft}Thee, lui?, z\`eg - th\'ee?, of,?}{thee-met-rum of rum met}{thee of rum-puur}\\

\haiku{{\textquoteleft}Lowis - Lowisje -...{\textquoteright}.}{nur eine Maar ze vergeet}{er op door te gaan}\\

\haiku{Ze staan onder een,.}{rood driekantig lampje en}{klinken ergens op}\\

\haiku{Een buitensporig.}{triumfantelijk gevoel}{geeft haar dat ineens}\\

\haiku{Ze was werkelijk,.}{een oogenblik aan het strand}{in de roosterkuil}\\

\haiku{{\textquoteright} Een oogenblik is,.}{alles duidelijker dan}{verdoezelt het weer}\\

\haiku{En Nicolette.}{bevoelt doelloos de rand van}{de roode deken}\\

\haiku{Hij lachte er om,,}{ik hoorde hem lachen dat}{verwachtte ik wel}\\

\haiku{{\textquoteleft}Laat ik hem toch in,...}{zijn gezicht spuwen laat ik}{hem weg-jouwen}\\

\haiku{{\textquoteleft}Te kunnen bidden{\textquoteright},, {\textquoteleft} -.}{denkt zete kunnen bidden}{wat vreemd moet dat zijn}\\

\haiku{Ze heeft verdriet en -.}{kan de handen uitstrekken}{God is geen leegte}\\

\haiku{Kassen duwt haar weg.}{met de eene hand en streelt haar}{met de andere}\\

\haiku{{\textquoteright}, zijn stem giert, rochelt,.}{hij zakt op  een stoel neer}{en staat ook weer op}\\

\haiku{{\textquoteleft}En omdat ik dat,, -......?}{gedaan heb gooi jij je weg}{als een een stuk vuil}\\

\haiku{Kassen Herfst was er.}{erg op gesteld dat ze een}{sluier zou dragen}\\

\haiku{Ze wordt voorgesteld,.}{en begroet maar ze onthoudt}{geen enkele naam}\\

\subsection{Uit: De ontmoetingen van Rieuwertje Brand}

\haiku{Al bijna een uur,}{lang zat Rieuwertje op zijn}{kleine manke bank}\\

\haiku{Nebekadnezer,!}{kan dat ook niet weten die}{komt op de schijn af}\\

\haiku{Kleine geluidjes}{scharrelen in het rond en}{aan die geluidjes}\\

\haiku{Duidelijk hoort hij -.}{daar de hel knetteren het}{is een goeie afschrik}\\

\haiku{want die gedachte,.}{gaat dwars door een zeerigheid}{heen binnen in hem}\\

\haiku{{\textquoteleft}Ja, dat kenne we!,,!,...}{handen thuus asjeblift gien}{malaberigheid}\\

\haiku{t begeeren is meer ',,!}{dant hebben ferachtig}{weerheid ferachtig}\\

\haiku{Maar de Knaak is breed,.}{op het vierkantige af}{en rood als baksteen}\\

\haiku{Op het Hoofd, bij de,}{Harlinger boot praat hij een}{verlept mijnheertje}\\

\haiku{Dartien stuver... dat, -!}{ken niet dan dan moet ik er}{nog op toeleggen}\\

\haiku{{\textquoteleft}Grutje-me-tut{\textquoteright},,, {\textquoteleft} '.}{smaalt hij in zijn gedachten}{watn gootwater}\\

\haiku{Maar eerst had hij 'n ' '.}{skrievenricht tot de regeering}{oft wel schikte}\\

\haiku{Affien, veul en niet......':}{genog en deer niet van maar}{toe zee de Koning}\\

\haiku{{\textquoteright}, Rieuwertje schraapt het,.}{grom van de planken en sluit}{zijn wagentje weer}\\

\haiku{{\textquoteright} {\textquoteleft}Lever me niet uut,{\textquoteright},, {\textquoteleft}.}{Heere smeekt hij nederig}{lever me niet uut}\\

\haiku{Want waar hij zichzelf,.}{terugvindt daar is het meer}{dan verschrikkelijk}\\

\haiku{Rieuwertje wil hem,}{uit de weg gaan heelemaal}{in-de-war}\\

\haiku{{\textquoteright} Hij hoort de wielen.}{van zijn wagentje over de}{klinkers ratelen}\\

\haiku{Hij ziet de jongen.}{hakken en schaven in de}{timmermanswerkplaats}\\

\haiku{De pluchen rozet...}{op haar eene pantoffel hipt}{jolig heen en weer}\\

\haiku{{\textquoteright}, vraagt hij benepen, {\textquoteleft} -?}{en en heb je gien poesie32}{eten veur mijn bewaard}\\

\haiku{{\textquoteleft}O ja, wel zeker!,!}{en dat mondjen-vol}{eten komt er vanzelf}\\

\haiku{Het is ook of hij,.}{grooter wordt het verdriet gaat}{een beetje opzij}\\

\haiku{je geven, al wat,...{\textquoteright},}{ik verdiend heb maar dat \^are}{geld onbeholpen}\\

\haiku{Met de buitenkant.}{van zijn vingers strijkt hij een}{paar maal over het kret}\\

\haiku{Heel  zacht doet hij,.}{dat net zooals hij anders over}{Koosie's haar zou strijken}\\

\haiku{Hij ziet plotseling.}{dat het plankendekje van}{zijn wagen openligt}\\

\haiku{Klauwen grijpen zijn -?!}{hart aan en hoe het zoo kan}{en wat dat toch is}\\

\haiku{het is of zijn oogen.}{op de wiegende plooien}{van haar rok hangen}\\

\haiku{En laat niemand nu}{ooit weer zeggen dat die kroeg}{van Woutjen een vuil}\\

\haiku{RIEUWERTJE MOET ZICH.}{INSPANNEN om er iets van}{te onderscheiden}\\

\haiku{Gelukkig was die,,.}{man luw waren zijn dagen}{vredig zijn nachten}\\

\haiku{Hij moet een hooge stoep,,!}{op dat is lastig zoo'n stoep}{een heele opstap}\\

\haiku{Machteloos zinkt hij,.}{voorover met zijn mond op een}{stoffig vloerkleedje}\\

\haiku{Domp hoort hij scherpe,.}{jonge stemmen vaag ziet hij}{kleine figuren}\\

\haiku{Die vrouwachtige.}{meid is Rieuwertje's dochter}{Leen-die-dient}\\

\haiku{Rederijk is die,!}{goeie vrind in de verte hoor}{dat schepsel praten}\\

\haiku{{\textquoteright} De vischreuk aan zijn.}{goed wil hem op andere}{gedachten brengen}\\

\haiku{Z\'eker... mos' je 'n '... '...}{wuufhad hebbe as iene}{asn zekere}\\

\haiku{Ja, die andere,,!}{zit er nog zoo waarachtig}{als God hij zit er}\\

\haiku{{\textquoteleft}K-kom niet in...{\textquoteright} {\textquoteleft}{\textquoteright},.}{mijnHou je waffel gebiedt}{de Cosmopoliet}\\

\haiku{{\textquoteright} En die woorden keeren,.}{zich verwonderd naar hem om}{eer zij verdwijnen}\\

\haiku{Louter bij toeval.}{komt hij in Woutjen's gele}{warme kroeg terecht}\\

\haiku{Zijn protest snerpt als:}{de krijschende schreeuw van een}{beest dat geslacht wordt}\\

\haiku{* * * ~ Schurftig zijn die,,.}{plagen rood-ontstoken}{boosaardig en wreed}\\

\haiku{Zijn hand boort een kuil,,}{in het blinde zwart hij werpt}{het donker opzij}\\

\haiku{En zoo folterend!}{als dat visioen van zijn}{delirium is}\\

\haiku{Maar het glanzende.}{warme leven tracht hem van}{zich af te schudden}\\

\haiku{Rieuwertje luistert,.}{nog maar zijn aandacht gaat een}{andere kant op}\\

\haiku{Huilie binne er,!}{nou niet huilie kenne je}{nou niet uutlache}\\

\haiku{Rieuwertje bukt zich,!}{houterig en gluurt onder}{tafel daar is niets}\\

\haiku{Rieuwertje merkt het,.}{en het is of hij van zijn}{eigen hart vervreemdt}\\

\haiku{{\textquoteleft}Gaan na' buten, jij,!,!,!}{allaah pak je biezen niet}{omklungelen hier}\\

\haiku{Hij zit ook al een,,.}{heele poos eer hij bemerkt}{dat hij niet meer loopt}\\

\haiku{{\textquoteright}, prevelt hij, {\textquoteleft}en n{\'\i}je38 -...{\textquoteright}.}{n{\'\i}je minsche En hij smakt zoo'n}{beetje als de zee}\\

\haiku{Het lijkt wel of de...}{stilte eerst een vloeibaar iets}{was en nu stolt}\\

\haiku{{\textquoteright} Als een vlaggetje,}{zwaait de tong heen en weer in}{die zwarte leege mond}\\

\haiku{n sp\'oog water... veur,,...{\textquoteright},}{je over h\^et gien levende}{ziel die verbaasd blijft}\\

\haiku{{\textquoteright} Meteen mummelt hij,...}{weer door over de donkere}{tijd en over Engel}\\

\haiku{Dat doet hij dan toch,.}{niet maar hij zucht als een mensch}{die zwaar werk verricht}\\

\haiku{Maar Rieuwertje let,.}{er niet op hij heeft het veel}{te druk met zichzelf}\\

\haiku{{\textquoteleft}Wou je dat overdag,,...?}{doen ellendelingkien op}{de klaarlichte dag}\\

\haiku{{\textquoteleft}'t Maakte h\'em niks, ',...}{meer uutt kon hem niks meer}{skelen de jongen}\\

\haiku{Als Kako omkijkt,.}{kan het best gebeuren dat}{hij hem terug jaagt}\\

\haiku{En Rieuwertje kijkt,.}{nog naar hem als hij al-lang}{niet meer te zien is}\\

\haiku{maar het is of zijn,.}{bloed zweet en zijn hart krijgt weer}{zoo'n kaduuk gevoel}\\

\haiku{gaan er af en toe.}{een paar gedachten door zijn}{sufferige hoofd}\\

\haiku{Gosse de Kiezer,.}{die z'n lichaam was an'kocht}{veur de snijkamer}\\

\haiku{maar dat dorst hij niet,,,}{deer had hij gien koerasie}{veur jisses jisses}\\

\haiku{{\textquoteright} Zoetelijk-paaiend,,...}{of hij het tegen een klein}{kind heeft vraagt hij het}\\

\haiku{Elke dag begint,.}{hij daar opnieuw mee altijd}{is het vruchteloos}\\

\haiku{Hij rammelt er mee,,...}{hij bluft er mee de spijt huilt}{boven alles uit}\\

\haiku{Hij kocht een vloerkleed,.}{voor Engel kastbekers en}{ook een theeservies}\\

\haiku{O minsch, ik ken,,!}{je niet luchten of zien ik}{spij50 op je minsch}\\

\haiku{De droom houdt hem vast,.}{en er lijkt een glans van zijn}{gezicht uit te gaan}\\

\haiku{in elk groen-blauw!}{bobbeltje kan een stukje}{pauweveer zitten}\\

\haiku{Hij weet niet waarheen.}{hij gaat en hij weet niet waar}{hij neerzit oplaatst}\\

\haiku{Zijn licht-schuwe.}{oogen klampen armzalig de}{voorbijgangers aan}\\

\haiku{, de zonderbare.}{eensgezindheid van Engel}{met de kinderen}\\

\haiku{Nog vol-op moet de,,...}{zon daar-bij-hem-thuis}{door het raam vallen}\\

\haiku{Muziek ratelt en,,...}{boemt paardenhoeven ketsen}{kleuren schitteren}\\

\haiku{Vlak voor Rieuwertje.}{staan een paar vreemdsoortige}{personages stil}\\

\haiku{hij kan loopt hij de,.}{straat op hij wil een beetje}{monterheid koopen}\\

\haiku{Hij wil ook lachen,,.}{overluid wil hij lachen maar}{dat kan hij niet meer}\\

\haiku{Daar komt de optocht,,...}{aan in zilver en goud en}{vertreedt het hondje}\\

\haiku{Le\^et maar an m{\'\i}jn over,, ' '!}{dat is m{\'\i}jn toevertrouwdt}{zeln spul worden}\\

\haiku{{\textquoteright}, protesteert Kako, {\textquoteleft} '...{\textquoteright} {\textquoteleft}}{vermoeideten geef je nog an}{n zwarvende hond}\\

\haiku{{\textquoteright} Rieuwertje knikt ten,.}{teeken dat hij het verstaat}{maar hij zegt nog niets}\\

\haiku{En elke keer als,,.}{ze daar op doorgaat gromt haar}{man en kleurt Japie}\\

\haiku{En Engel geeft een.}{zenuwachtige ruk aan}{haar bloemenhoedje}\\

\haiku{{\textquoteleft}Toe maar - toe maar, 't, ', '.}{is lekkert verwarmtt}{brandt in je botten}\\

\haiku{{\textquoteleft}'n Riksdaalder{\textquoteright}, stelt, {\textquoteleft}!}{Leen hem kordaat uit zichzelf}{voorken je kriegen}\\

\haiku{Sputterend steken.}{de kinderen-en-Engel}{de hoofden bijeen}\\

\haiku{Maar als hij zijn naam,,.}{moet zetten kan hij haast niet}{schrijven zoo beeft hij}\\

\haiku{{\textquoteleft}Tjisses...{\textquoteright}, valt het door, {\textquoteleft}......!}{hem heendie kol was  er}{altied en nooit zelf}\\

\haiku{Terloops dringt het tot,,...}{Rieuwertje door hij let er}{toch niet op verder}\\

\haiku{, bin je niet iensen...}{dronken en ken je dan toch}{even goed bidden veur}\\

\haiku{net as op die nacht...'...}{op die avend le\^et toe je}{bezeten wasse}\\

\haiku{n Minsch zeit op,...!}{z'n starfbed \^are dingen dan}{in z'n leven oue}\\

\haiku{{\textquoteright} Maar Engel wrikt zich.}{netelig achteruit in}{het verwoelde bed}\\

\haiku{Op skool{\textquoteright}, ratelt ze, {\textquoteleft},'!}{doene ze m{\'\i}jn ook pien mos}{je op skool hooren}\\

\haiku{De flakkerende.}{zon geeft een ongewisse}{blijheid aan het erf}\\

\haiku{Tja, nou, as die Leen... ' '!}{de wasch weeromstuurdet}{most tochdaan worden}\\

\haiku{{\textquoteleft}Le\^et Vader nou maar ',.}{n stre\^etje om gaan Moeder}{zel je wel helpen}\\

\haiku{{\textquoteright} En Rieuwertje grijpt...}{zijn hoed en zijn jekker van}{de stoel en hij gaat}\\

\haiku{Me-jongen{\textquoteright}, vraagt, {\textquoteleft}?}{hij afgetrokkenwillen}{we wat kuieren}\\

\haiku{Daar zit oue minschie...}{onder de kaarsen-kroon}{en tjilpt als een musch}\\

\haiku{Vadertje-God, '{\textquoteright}}{h\^et je an'nomen en je}{binne heerlijkred.}\\

\haiku{{\textquoteleft}Kako, die liep   ',...}{deer ook nog ers en kon}{temet niet veerder}\\

\haiku{En Engel rukt en.}{trekt als een waanzinnige}{om los te komen}\\

\haiku{Pas was 't kind er{\textquoteright},, {\textquoteleft}....}{toch nog denkt hij doezelig}{passies vlak bij mijn}\\

\haiku{De huizen staan als...}{dommelige gedachten}{in het  duister}\\

\haiku{Armoedigkaal voelt.}{de gangvloer aan onder zijn}{tastende voeten}\\

\haiku{Hij doet omzichtig.}{de kamerdeur open en kijkt}{schichtig naar binnen}\\

\haiku{Hij heeft zoo voor de.}{menschen in het algemeen}{niet veel aandacht meer}\\

\haiku{En Veronica.}{begint haastig wat houtjes}{op te rapen}\\

\haiku{As je skarp kieken, ' '.}{zie je haast hoe de lieve}{Heertskapen h\^et}\\

\haiku{Zwakjes moeten ze,.}{allebei glimlachen hij}{en Veronica}\\

\haiku{{\textquoteright}, vraagt Veronica, {\textquoteleft}?}{bijna fluisterendis dat}{nou wezenlijk waar}\\

\haiku{{\textquoteleft}Nog wel honderde,,...}{malen bedankt Vroon nog wel}{honderde malen}\\

\haiku{Eenmaal, is hij nu,,.}{in-het-kort Engel}{tegengekomen}\\

\haiku{Hij knikte nog, en.}{hij had even goed tegen een}{huis kunnen knikken}\\

\haiku{Wijs en vriendelijk.}{kijkt grijze oue minschie over}{haar witte breikous}\\

\haiku{{\textquoteleft}O lieve Heer... o,...}{Vadertje-God le\^et ze}{mijn niet wegsturen}\\

\haiku{in Gods naam, kom weer,,}{bij mijn of le\^et mijn bij jou}{blieven Engel toe}\\

\haiku{HART WORDT DAAR niet meer,.}{blij van in blijheid schittert}{het leven te fel}\\

\haiku{Maar die moeheid is,.}{er nu eenmaal en moet er}{al meer geweest zijn}\\

\haiku{Ze loopt van hem af,.}{en gaat zonder slag of stoot}{rugwaarts de steeg in}\\

\haiku{Voor de zijdeur van,.}{haar Moeders erf hurkt ze neer}{en prevelt en bidt}\\

\haiku{haar hangt tot op haar.}{onderworpen handen en}{bedekt haar gezicht}\\

\haiku{Hij moet er elke,.}{dag aanbellen en soms ook}{tweemaal op een dag}\\

\haiku{Ze willen mijn niet{\textquoteright},, {\textquoteleft}.}{vertelt hij aan de stilte}{ze sturen mijn weg}\\

\haiku{En hij denkt er niet,,.}{aan om naar huis te gaan hij}{heeft honger noch dorst}\\

\haiku{Hij verstaat niet wat,,...}{iedereen hoort maar het is}{goed zoo het is goed}\\

\haiku{{\textquoteleft}Ja{\textquoteright}, sust haar stem, {\textquoteleft}ja...,...{\textquoteright}.}{stil nou maar stil nou Maar er}{zijn nog meer zorgen}\\

\haiku{{\textquoteleft}Wat wordt nou alles, ' '.}{nog goedt ken nog best in}{t reine komme}\\

\haiku{{\textquoteright} En Rieuwertje kan,.}{haast niet meer praten maar zijn}{geest is nog helder}\\

\subsection{Uit: Oude kennissen}

\haiku{, zijn doezelige.}{gedachten verdoolden in}{herinneringen}\\

\haiku{Maar haar kleine oogen,,!}{gnuifden ze was blij dat ze}{er zelf niet heen moest}\\

\haiku{En Duif peinsde waar,,.}{hij heen moest die man en wie}{of het wezen kon}\\

\haiku{{\textquoteright} De vrouwen zuchtten,.}{en beklaagden Duif en twee}{haalden er een smid}\\

\haiku{Een diepe zorgplooi,:}{had hij tusschen zijn oogen en}{hij dacht telkens weer}\\

\haiku{Tjee, dat hij ook geen,...{\textquoteright}}{ander had kenne krijgen}{as juist dat vreempie}\\

\haiku{{\textquoteleft}G\^o, ook 'n leven{\textquoteright},, {\textquoteleft}'. '}{pruttelde hijn leven}{zoo op je eentje}\\

\haiku{{\textquoteright} Hij keek oolijk op.}{naar zijn vrouw en hij lei zijn}{hand op de hare}\\

\haiku{er maar niet boos om,, ',?}{worden man voor jou ist}{zoo'n bezwaar niet h\`e}\\

\haiku{, en zijn gedachten,.}{wriemelden verward dooreen}{zijn hoofd was overvol}\\

\haiku{Saar, voor aan de straat,,.}{en Brecht midden-achter en}{Aagt uit de keuken}\\

\haiku{s vragen willen... '}{offe we wel in de goeie}{spoor zitten en zoo}\\

\haiku{{\textquoteright} Achter een boschje.}{bleven ze ginnegappend}{naar hen staan kijken}\\

\haiku{Maar u bent hi\'er toch{\textquoteright},, {\textquoteleft}}{niet te Amsterdam lichtte}{hij glimlachend in}\\

\haiku{Klaas, die al deze,,.}{dingen verstond omdat hij}{gek was hijgde zwaar}\\

\haiku{Jawel, jawel, ze,,.}{is er ze moet er wezen}{ze m\'oet er wezen}\\

\haiku{{\textquoteleft}De buren moeten, '.}{niet merken datk zoo op}{de postbode wacht}\\

\haiku{De hand van de vrouw,,.}{was warm haar stem zacht en ze}{liep dicht naast hem voort}\\

\haiku{En de duisternis.}{werd ondoordringbaar in het}{zware hoofd van Klaas}\\

\haiku{{\textquoteleft}En verder niks, niks,.}{van zien olde minschen en}{niks van zien verleeje}\\

\haiku{Het ontgleed haar weer.}{in een knijperig gevoel}{van teleurstelling}\\

\haiku{{\textquoteleft}Han denkelijk toch,.}{al weg en nou moest zij d'r}{reppen van belang}\\

\haiku{Jet deed erg haar best.}{om niets te laten blijken}{van haar onthutstheid}\\

\haiku{Toen opeens, midden,.}{in haar blerrend vertellen}{onderbrak Sien haar}\\

\haiku{Abel smeet zijn werk neer,,.}{zijn hoofd was zweeterig en}{zijn vingers beefden}\\

\haiku{{\textquoteleft}Catootje{\textquoteright}, Abel groette haar,,.}{dof er was een schrik in zijn}{stem een rilling ook}\\

\haiku{{\textquoteleft}'t - 't Was toch 'n,?, ',?}{bovenneturelijke toer}{h\`en krachttoer h\`e}\\

\haiku{En het maantje was,.}{naargeestig-bleek en het}{glimlachte ni\'et}\\

\haiku{Hij greep naar de groote}{bolle knop en trok meteen}{de hand weer terug}\\

\haiku{Soezerig zocht hij,.}{meteen naar de woorden die}{hij verloren had}\\

\haiku{Om de deurhoek schoof.}{het kleine gladde hoofdje}{van het dienstmeisje}\\

\haiku{{\textquoteleft}Ga gauw naar Mijnheer....}{Steenstra van hier-naast of}{hij dadelijk komt}\\

\haiku{de koffiepot op.}{het lichtje zette en haar}{handen in de zij}\\

\haiku{Zoo'n naam nou{\textquoteright}, sufte, {\textquoteleft} '.}{hijeigenlijk zat ern}{nuchtere klank in}\\

\haiku{via de klanten... ter,?}{oore gekomen dat ik}{er wat van versta}\\

\haiku{... ik ben er blij mee......?}{dat jij als je wat geduld}{met me wil hebben}\\

\haiku{{\textquoteright} Even kreeg hij nu toch.}{weer een stroef gevoel in de}{keel en at vlug af}\\

\haiku{{\textquoteleft}'n Ingespannen{\textquoteright},, {\textquoteleft}}{dag zei ze zorgelijk en}{schuw in aarzeling}\\

\haiku{Het hitje bracht met een.}{gewichtig airtje de}{groene brief binnen}\\

\haiku{{\textquoteright} Ongenadig viel:}{ze toen ineens weer tegen}{de kinderen uit}\\

\haiku{{\textquoteleft}Mijndert, rakker, denk,}{er om as ik je nou nog}{\'een keer verbieden}\\

\haiku{{\textquoteright} Het scherpe kuchje.}{van schoonmoeder stootte een}{gat in haar gepeins}\\

\haiku{'n Kip zonder kop '!}{is nogn prefester bij}{j\'ou vergeleken}\\

\haiku{{\textquoteleft}En 't geld voor dat,!}{uitstappie  d\'aar kom je}{ook niet eerlijk \^an}\\

\haiku{Want Sieuwert zag er.}{monter en welgemoed uit}{en hij glimlachte}\\

\haiku{Hij had een felle,.}{klop in zijn keel maar zijn mond}{glimlachte rustig}\\

\haiku{{\textquoteleft}Weet je nog Annet, ','?}{zooals wet gezien hebben}{toe in Amsterdam}\\

\haiku{Eerst stapte hij heen,.}{en weer vijf schreden heen en}{vijf schreden terug}\\

\haiku{Het stadje viel zoo.}{van de eene verwondering}{in de andere}\\

\subsection{Uit: Sterke webben}

\haiku{Ze praatte d'r niet,.}{van liep dicht naast Da voort op}{de deukige weg}\\

\haiku{Heet in de wangen.}{en schuw hoorde ze naar de}{ingesmoorde schimp}\\

\haiku{{\textquoteleft}Moet je Bert de Krey - ',......{\textquoteright}}{s over hooren Bert d\^a-'s}{toch zoo'n dolle z\`eg}\\

\haiku{Mien hoorde d\`an strak '.}{en gespannen toe opt}{rappe gefluister}\\

\haiku{Hi-hi... we moeten.}{ook nog danken voor onze}{avond-boterham}\\

\haiku{Mien merkte d'r niets..., '.}{vant echode gestadig}{na in h'r peinzen}\\

\haiku{Schamig sloop ie naar,.}{h'r toe  drukte z'n wang}{tegen de hare}\\

\haiku{En tot in nek en '.}{voorhoofd begloeide Mienn}{kregele schaamte}\\

\haiku{'n angst - doortokkeld '... ' - '.}{vann vreemde vreugdt Was}{z\'o\'on oogenblik}\\

\haiku{Ja - daar vertelde:}{vrouw Bartelinck vroeger}{zoo'n mooi sprookje van}\\

\haiku{Ja - hij kende h'r,,:}{Vader niet die zou n'tuurluk}{uitvaren ie zei}\\

\haiku{Door h'r peinzen gleed '.}{n gretige verwachting}{en ze ademde snel}\\

\haiku{Ze keek 't wijfje ',.}{verwonderd int ronde}{oolijke gezicht}\\

\haiku{niet zoo, Kostertje,, ',{\textquoteright}}{haast je n\'ou ni\'ett is}{hi\'er nog gez\`ellig}\\

\haiku{{\textquoteright} En als Meurs na 'n -,}{poos w\'eer l\'eep-vorschend over}{Ubbels begon hieuw}\\

\haiku{Zoo'n boel kleuren en -, -,?}{dan zoo'n zoo'n rust over alles}{nu ook n\'u \'ook h\`e}\\

\haiku{t leek vlak-bij, ',.}{inn ruk had ze zich los}{keek verwilderd om}\\

\haiku{{\textquoteright}, morde ze, en h'r, '...}{wangen schoten vol bloed ze}{hoordet t\`och w\`el}\\

\haiku{Mien hoorde 't... en,,.}{ze huiverde dr\`alender liep}{ze m\'eer str\'ompelend}\\

\haiku{Dan weer heftiger ',... {\textquoteleft}}{had zet verlangen naar}{ruimte wind-frischte}\\

\haiku{Bew\'eer {\`\i}k toch niet? ' -..., '...?}{k Bedoel die \'eene di\'e}{eene diet dan is}\\

\haiku{{\textquoteright} - Om 't huis baste, '.}{de luwe winter-windn}{donkere neurie}\\

\haiku{Bij di\'e Keppel, bij...,?}{die \'akelige l\'e\'elukke d'r}{gingen de jongens}\\

\haiku{{\textquoteright} Blozend morde ze '.}{t en h'r oogen ontweken}{z'n gretige blik}\\

\haiku{Even toen ze langs 'm, '.}{ging wreef h'r schou\"er tegen}{t week van z'n borst}\\

\haiku{{\textquoteright} Snel draaide ze 't, '.}{hoofd en onderzoekend keek}{zem in de oogen}\\

\haiku{Je netje ligt nog,...?}{op de regenbak als je}{soms li\'ever w\`eg wil}\\

\haiku{{\textquoteright} 'n Keer toen z'n wang,.}{h'r hand beroerde voelde}{ze warme druppen}\\

\haiku{{\textquoteleft}Co was van al di\'e,..., -..!}{\`anderen de b\`este nog Co}{Bruins die zou h'r m\`an}\\

\haiku{Je worde op - op...{\textquoteright},.}{lange-lest zoo loof van dat}{opstaan zuchtte ie}\\

\haiku{{\textquoteright} H'r wangen werden,.}{donker-rood heel h'r gezicht}{voelde heet en klam}\\

\haiku{{\textquoteleft}Hij deed 't zeker,{\textquoteright}.}{z\`elf \'aarzelig dreef ze h'r}{aandacht daar van weg}\\

\haiku{Bl{\'\i}j met 't voorrecht...{\textquoteright},,.}{spotte ie z'n lach klonk zwaar}{door de stille straat}\\

\haiku{dat je... waar je - waar...? ' - '...}{je moeiluk van weg kunt h\`e}{tt Bekende}\\

\haiku{{\textquoteleft}J\'a-\'a, die n'tuurluk - '...}{n't\'uurluk ook en terdeeg de}{kat int donker}\\

\haiku{Mien ging dan stadig '.}{jachtig doort duisterst van}{de nauwe straten}\\

\haiku{Nee - n\'ee, 't h\'oefde -......, '...?}{toch niet dat ze d\`at weer d'r}{wasr toch n\`og een}\\

\subsection{Uit: Tijne van Hilletje}

\haiku{Het was of ze veel {\textquoteleft}{\textquoteright}.}{te dicht voor eenhel van een}{kachel-vuur stond}\\

\haiku{{\textquoteleft}Zoo, bin je ook al,, '...?}{op'eschote Lijs de wesk al}{int waoter}\\

\haiku{Ze hield de armen.}{kruiselings over de borst en}{lachte in zichzelf}\\

\haiku{{\textquoteleft}As je allebei, ',...}{hulfte hadde maoktet}{niks uit maor n\'ou}\\

\haiku{ik heb de wesk ook ' '}{aledaon en houtjesehakt}{en in de vruugte}\\

\haiku{Maar zij morste of.}{ze nog nooit een tas koffie}{ingeschonken had}\\

\haiku{Als Aorie praatte.}{leek het net of er in zijn}{stem wat neuriede}\\

\haiku{Ze greep Tijne zoo.}{maar de hark uit de handen}{en gooide die neer}\\

\haiku{Ze had het ook over,.}{Keesie van haar eigen had}{Aogie geen kinderen}\\

\haiku{Op het oogenblik,.}{scheen ze niet aanwezig de}{deur was pot-dicht}\\

\haiku{Ze ging een steegje,.}{door en een slop voorbij met}{een schok stond ze stil}\\

\haiku{De wind ging voor het.}{glop staan en blies er een paar}{zware zuchten in}\\

\haiku{Toen blikte ze koud:}{van ontsteldheid om naar het}{gordijn voor het bed}\\

\haiku{{\textquoteright} Een raar schor hikken,...}{zaagde door de stilte mal}{gepiep en gesnuif}\\

\haiku{Mijnt zette met een.}{knorrig keel-geluid haar}{tanden in een schar}\\

\haiku{En Mao spande haar.}{linkerhand breed over Tijne's}{witte klompen uit}\\

\haiku{een klipper met drie:}{gele roefraampjes en de}{man er op floot van}\\

\haiku{Ze hield nog altijd.}{de Heiland vast bij een tip}{van zijn witte kleed}\\

\haiku{terug'ehaold, 't, '...}{is nou zoo zwaort is}{nou weer zoo zwaor}\\

\haiku{{\textquoteleft}...en die loon ontvangt '.}{die ontvangt dat loon inn}{doorboorde buidel}\\

\haiku{{\textquoteleft}Morgen,{\textquoteright} zei ze in, {\textquoteleft}{\textquoteright}...}{haar gedachtenmorgen al.}{Maar dat lag ver weg}\\

\haiku{{\textquoteleft}En den heb ik ook ' '............}{n paor maoledocht ik}{ikke miende v\`est}\\

\haiku{Zonder groet holde,.}{ze de straat op en haar oogen}{zwommen in tranen}\\

\haiku{{\textquoteright} {\textquoteleft}Jao - jao,{\textquoteright} galmde Tao, {\textquoteleft},.}{in de wind opgoddeloos}{net as je zegge}\\

\haiku{{\textquoteleft}Zoo Kaars,{\textquoteright} groette hij, {\textquoteleft},?, '...}{los-wegwaaierig}{h\`en brieke42 lucht}\\

\haiku{{\textquoteright} Tijne moest eventjes,,.}{spieden om te zien of dat}{tegen h\'aar Tao was}\\

\haiku{Toch scheen er dan nog,.}{een geluid van het Eiland}{te komen een schreeuw}\\

\haiku{je 't vraogde,,...?}{en waorveur ook wou jij}{soms-te-met}\\

\haiku{{\textquoteright} Opeens kregen ze,.}{een stoot en ze werden van}{elkaar afgeduwd}\\

\haiku{Naar haar droeg elke,.}{scheldende golf en elke}{booze wind-haal een angst}\\

\haiku{Wat hijgden er toch?}{voor benauwdhedens in die}{diepe wilde zee}\\

\haiku{Warm was het er, en,.}{er dwaalde een geur rond die}{er niet thuis hoorde}\\

\haiku{{\textquoteright} Maar Tijne vergat,}{te antwoorden daar had ze}{die starre oogen weer}\\

\haiku{De hijgende zee.}{schopte de boot voort of het}{een stuk speelgoed was}\\

\haiku{hier heb je 't eerst, '.}{goed in de gaote hoe}{bar datt is}\\

\haiku{{\textquoteleft}Nou most ik toch er ''.}{s vraoge waor nao toe}{dat we gaone}\\

\haiku{Gehaast stapte ze,.}{voort naast Tao ze slofte en}{strompelde van haast}\\

\haiku{En aan het einde - {\textquoteleft}{\textquoteright} -.}{van de straat voorde Doele}{wachtte de stoomtram}\\

\haiku{Telkens was er wat,.}{anders maar alles spoedde}{zich even haastig voort}\\

\haiku{De groote stad kwam op,:}{ze af in de menschen die}{ze tegenkwamen}\\

\haiku{{\textquoteleft}Tao,{\textquoteright} drong zij ineens, {\textquoteleft},'?}{zeg mijn nou waor nao toe}{dat we gaone}\\

\haiku{een deftig man, {\textquoteleft}bas{\textquoteright}46,.}{gekleed en met een rug zoo}{recht als een koning}\\

\haiku{De rooie bloed-oogen.}{van Bappe keken haar aan}{uit Tao's gezicht}\\

\haiku{{\textquoteleft}Nee,{\textquoteright} zei ze hard, en,.}{schoof het bordje met de kom}{een stuk van zich af}\\

\haiku{{\textquoteright} Haar gezicht beefde,,}{en bibberde ze boog zich}{over de koffie heen}\\

\haiku{Ze keken naar het - -...}{Eiland Tao en zij of ze}{ergens naar speurden}\\

\haiku{{\textquoteleft}Keesie Kaors, je,, '.}{vrouw het vemiddag te twee}{ure de geestegeve}\\

\haiku{En, man, ze is in...}{volle verzekerdheid des}{geloofs ontslaope}\\

\haiku{zelfs Lobbetje kreet,!,...!}{mee en Aogie wat schreeuwde die}{niet en Bappe dan}\\

\haiku{{\textquoteleft}Trijntje, je mag toch -... -.}{niet niet in opstand niet al}{te te bedroefd zijn}\\

\haiku{Ze merkte niet eens.}{dat ze op haar kousen liep}{en zonder muts was}\\

\haiku{Tot ze ineens wat...... {\textquoteleft}}{miste en verschrikt stil bleef}{en staarde staarde}\\

\haiku{{\textquoteright} Gedwee stond ze niet,.}{op het was of ze er zich}{toe aansporen moest}\\

\haiku{Tijne gaf daar ook,.}{wel acht op en het had haar}{toch moeten stichten}\\

\haiku{De keukendeur werd.}{met hortjes en stooten een}{eindje open geduwd}\\

\haiku{want als hij nergens,.}{behoefte aan had werd hij}{stug en afkeerig}\\

\haiku{' Memme nog leefde,{\textquoteright}, {\textquoteleft} ''...{\textquoteright}}{schoot het Tijne te binnen}{wat wast toe aors}\\

\haiku{{\textquoteleft}Wat was er an te,?}{zien en te hoore waor}{je van lache moeste}\\

\haiku{{\textquoteleft}As 't keind, dat 'k, ',}{n\'ou draog weern maasie is}{durf ik er niet mee}\\

\haiku{tot in 't derde.}{en vierde lid dergienen}{die Mij haoten}\\

\haiku{Ging de lui-klok?,?}{niet zachter luisterden de}{kleine huizen niet}\\

\haiku{En het eerste wat,,:}{er in haar opkwam toen ze}{weer denken kon was}\\

\haiku{Goeie reis den maor,{\textquoteright}, {\textquoteleft}.}{wenschte ze hem toeen}{behouwe thuiskomst}\\

\haiku{Wat 'n waoi!, en wat ',,, '?}{n lucht ook Gaart minsch heb}{je ooit zoo'n luchtezien}\\

\haiku{{\textquoteleft}Jao-jao, dat bin,, '.}{ik mit je iens Fokkiek}{zel ook vortmaoke}\\

\haiku{Als ze aan Aorie,.}{dacht was het meteen of ze}{aan Aorie's hand liep}\\

\haiku{Een jongen had zijn:}{voorletters in het hout van}{een schutting gekrast}\\

\haiku{Schichtig-snel nam:}{ze de veranderingen}{waar in de kamer}\\

\haiku{Dat witte daor,,:}{dat waore de bedde}{die zwarte bulte}\\

\haiku{Waor zoude ik}{henengaon veur Uwen Geest}{en waor zoude}\\

\haiku{{\textquoteright} Tijne sloop door het.}{bevende huis of het hun}{eigen huis niet was}\\

\haiku{{\textquoteright} Rillend betastte,.}{ze de kachel die kachel}{was zoo koud als ijs}\\

\haiku{Je binne toch niet ',...?}{veer vann iegelijk die}{je anroept Heere}\\

\haiku{Ons heule huisie,,...}{gaot er an Heere God}{ons heule huisie}\\

\haiku{Och Heere, om Uws -,,,.}{naoms wil om Uws n\'aoms wil h\'o\'or}{mijn aomen aomen}\\

\haiku{Het huis kraakte en:}{dreunde of er timmerlui}{in doende waren}\\

\haiku{Grauw-wit was hij in,:}{zijn gezicht een streep bloed liep}{bij zijn kin beneer}\\

\haiku{Makker, je ziene -'...{\textquoteright}.}{toch kijk nao buite Tijne}{luisterde niet meer}\\

\haiku{Haar lippen werden,,.}{blauw haar koonen ook uit niets bleek}{dat ze nog ademde}\\

\haiku{De storm verzwakte.}{gaandeweg en het water}{scheen moete worden}\\

\haiku{het leek al op een,...}{begrafenis het rook daar}{al naar rouw-kleeren}\\

\haiku{Veurzichtig nou, dat,,.}{ze niet stikt laot ik je}{toch helpe minsch}\\

\haiku{Had zij die veurhien59 '?, '?,?}{nietekonne langeleje die}{iendere stemme}\\

\haiku{{\textquoteright} En het was of een:}{mes haar hart openscheurde van}{boven tot onder}\\

\haiku{{\textquoteleft}Schei daor mee uit,, '?}{Tijne-van-Hilletje}{wat baott je}\\

\haiku{De dagen van voor,:}{de storm kwamen naar haar toe}{die dagen heetten}\\

\haiku{En er kwamen geen,.}{dagen op haar toe en geen}{herinneringen}\\

\haiku{Als Aogie tegen een,.}{meerdere sprak praatte ze}{altijd onderdrukt}\\

\haiku{Maar dat stoffelijk, -.}{omhulsel daar daar moet je}{niet zoo aan denken}\\

\haiku{{\textquoteleft}Staon er toch niet, '!}{zoo suf bij je maoken}{minsch tureluursch}\\

\haiku{{\textquoteright} Er stapten menschen,.}{langs haar heen een-ieder ging}{dezelfde kant uit}\\

\haiku{De wind,{\textquoteright} kon Tijne, {\textquoteleft} '?}{ineens weer denkenwaor}{is di\'e nouebleve}\\

\haiku{En de grauwe vloer...}{werd barscher en alles om}{haar heen verkilde}\\

\haiku{Het kwam haar voor of.}{ze verder van alles en}{iedereen afzat}\\

\haiku{En ze keek opeens,.}{niet meer naar de Dominee}{ze keek naar Aorie}\\

\haiku{{\textquoteright}, Tijne gleed haast van,, {\textquoteleft}...}{haar stoel af zoo onthutste}{zebij bij baos}\\

\haiku{Ik - ik was temet,'.}{al weg toe heb juilie me}{weer terug'ehaold}\\

\haiku{Ze most op 'n aor'...{\textquoteright}.}{idee ebrocht worde En al gauw}{vond ze er iets op}\\

\haiku{Ze veegde met de,.}{rug van haar hand haar oogen af}{en keek ontsteld op}\\

\haiku{{\textquoteright}, gaf ze te raden,}{en ze drukte een koud hard}{ding in Tijne's hand.}\\

\haiku{En  n\'ou is 't,.}{de sleutel van jouw weuning}{n\'ou mag jij him houe}\\

\haiku{{\textquoteright} De gedachte trok.}{diepe pijn-plooien in haar}{pipsche gezichtje}\\

\haiku{{\textquoteright} Maar ze had het wel,.}{verstaan want al hoofdschuddend}{lachte ze er om}\\

\haiku{{\textquoteleft}Maot,{\textquoteright} smoesde ze, {\textquoteleft} '' '?}{lacherigheb jet nog}{al naot zin z\'o\'o}\\

\haiku{{\textquoteleft}Die Aogie, di\'e kon wel......}{rooie koonen hebbe en en schik}{in haor leve}\\

\haiku{{\textquoteleft}Me maosie,{\textquoteright} zei, {\textquoteleft}.}{hij in het naderenme}{\`erme kleine maad}\\

\haiku{{\textquoteleft}Gaart,{\textquoteright} zeit hij dan, {\textquoteleft}ik,.}{zel je plaoge tot je l\`este}{aodemtocht minsch}\\

\haiku{Gaart, is 't den, jij',,}{gaone rechtstreeks nao de}{hel minsch veur jou}\\

\haiku{{\textquoteleft}Maor zoo waor,...}{as ik  leef ik weut van}{gien schuld of kwaod}\\

\haiku{Het was een wonder.}{zoo gauw ze de brug over en}{de kerk voorbij was}\\

\haiku{{\textquoteright}, en ze greep achter...}{zich naar steun en deinsde een}{stap of wat terug}\\

\haiku{Tijne drukte haar,,:}{handen voor haar ooren haar}{oogen maar dat gaf niet}\\

\haiku{{\textquoteright} Oplaatst liep Mijnt er - -...}{een paar stappen achter hem}{als vergeten bij}\\

\haiku{Opzettelijk hard.}{stiet Baas Sanders toen ineens}{de keukendeur open}\\

\haiku{{\textquoteleft}Man, dat je toch zoo'n,}{truup draoge kenne in i\'en}{keer maok maor gien}\\

\haiku{{\textquoteright} En amper was hij,...}{vertrokken of ze keek al}{weer om naar Aorie}\\

\haiku{{\textquoteright}, prevelde ze, {\textquoteleft}oud, '.}{wordt die man netn keinsche}{Bappe bij tij\"en}\\

\haiku{het lezen{\textquoteright} boven.}{de breede groene  deur}{van de lichtwachter}\\

\haiku{Je binne nou net,,,...}{zoo gnap as Mijnt nee gnapper}{nog jao v\'eul gnapper}\\

\haiku{De weelderige.}{ansjovis-teelt liep ten}{laatste op zijn eind}\\

\haiku{Haar smartelijke.}{verwondering kon ze niet}{te boven komen}\\

\haiku{{\textquoteright}, mopperde ze, {\textquoteleft}'n '.}{bedoening ook dat die man}{gien-iensn vrouw het}\\

\haiku{Hoe meer haost, hoe,{\textquoteright}.}{meer teugespoed knorde ze}{verontschuldigend}\\

\haiku{hij 'evraogd had, zoo'n,...}{keind was zij toch niet dat ze}{d\`at niet zou vatte}\\

\haiku{In de reuk van het...}{zwart-gebrande pijpje}{was Bappe geweest}\\

\haiku{En toen moest ze haar.}{handen op haar mond leggen}{om niet te schreeuwen}\\

\haiku{{\textquoteleft}Nee, nao' de Baos...'.}{ken ik ook niet hiermee nao}{de Baos ook niet}\\

\haiku{Tijne lichtte de.}{klink van de deur maar de deur}{was ook gesloten}\\

\haiku{{\textquoteright} En meteen, als op,.}{pijn-vlagen kwamen haar}{gedachten terug}\\

\haiku{Het klonk of dat het,,.}{laatste was dat hij zeggen}{zou voor hij wegging}\\

\haiku{En in plaats van te,.}{vertrekken kwam hij nog een}{stap dichterbij}\\

\haiku{{\textquoteleft}En j{\'\i}j,{\textquoteright} stoof het heet, {\textquoteleft}...}{door haar heenj{\'\i}j geve de}{minsche minder den}\\

\haiku{{\textquoteleft}D\`en moet je hier t\`och,,...}{vedaon jao d\`en ken je}{hier \'o\'ok niet weuze}\\

\haiku{Nou maor gauw over,,{\textquoteright}.}{wat aors beginne binne}{nam Tijne zich voor}\\

\haiku{En hij mag den niet -...}{zoozeer mit mit uiterlijk schoon}{toebedeeld weuze}\\

\haiku{Want as je mit z{\'\i}jn ',{\textquoteright}, {\textquoteleft}}{etrouwd binne jolig lachte}{zewat zel hij aors}\\

\haiku{Warme pijnlijke.}{oogen kreeg ze er van en een}{warm pijnlijk hart ook}\\

\haiku{{\textquoteleft}Leer mijn je weg te,,{\textquoteright}, {\textquoteleft} '.}{gaon Heere bad zeik}{wilehoorzaom weuze}\\

\haiku{{\textquoteleft}Je magge er niet,,...!}{an legge te frunneke}{Aont veur de voet op}\\

\haiku{En die blijheid over:}{Aorie's vraag tuimelde er}{verwarrend doorheen}\\

\haiku{Maar ze lachte er,.}{bij en ze zag er weer zoo}{jong uit als ze was}\\

\haiku{{\textquoteright} Maar hoog in de lucht,.}{schaterde de zon en de}{wolken lachten mee}\\

\haiku{{\textquoteleft}Was ze eigelijk ' - ', '... '...?}{wel ooit eerezoendez\'oend zien je}{enan'ehaoldan'eh\'aold}\\

\haiku{Toen Aorie al-lang -.}{de hoek om was stond zij hem}{nog na te kijken}\\

\haiku{Voor de eerste maal.}{in haar leven trok ze een}{uitdagend gezicht}\\

\haiku{En daor wordt de '.}{vruchtbaorheid mee an'etast}{en tenietedaon}\\

\haiku{En - en wil ik den?}{eerst meschien die aorepels}{veur je ofspruite}\\

\haiku{Hij trok haar mee naar,,.}{een stoel ging zitten en nam}{haar op zijn knie\"en}\\

\haiku{{\textquoteleft}Maor wat moete '?}{we er ins heeren naom}{den mee uitrichte}\\

\haiku{{\textquoteright} Tijne deed moeite,.}{om het te verwerken al}{dat onverwachte}\\

\haiku{Toe maor,{\textquoteright} drong hij, {\textquoteleft} '....}{zegt mijn maor op mijn}{ken je betrouwe}\\

\haiku{En soms was het net.}{of ze een heel stuk van de}{anderen afzat}\\

\haiku{En niemand zei er,.}{wat grappigs en geen lach deed}{ze uitgeleide}\\

\haiku{Haast gretig deed ze,,.}{dat ze leunde ook zwaar haar}{hoofd hield ze voorover}\\

\haiku{De Goede Herder,.}{kwam naar haar toe hij stak zijn}{handen naar haar uit}\\

\subsection{Uit: Tusschen twee droomen}

\haiku{Ik drijf weg over een:}{effen waterspiegel en}{ben iets oneigens}\\

\haiku{Ik probeer ook mijn.}{groote broer die dood is in zijn}{lichte oogen te zien}\\

\haiku{{\textquoteleft}Ingelotje - had toch,....}{liever boontjes geweckt dan}{gedichten gemaakt}\\

\haiku{En ik maak ook een.}{sluier van maneschijn en}{wikkel mij daarin}\\

\haiku{Behoedzaam gluren,.}{ze daarbij naar mij eenigszins}{triest-verkennend}\\

\haiku{Van de school valt me,.}{altijd maar weinig op ook}{als ik er naar kijk}\\

\haiku{{\textquoteright} En een fragment uit.}{de geschiedenisles echoot}{ook nog door mijn bol}\\

\haiku{Ik druk de knokkels.}{van mijn verrukte vuisten}{stijf tegen mijn mond}\\

\haiku{Het is de heele.}{dag zoo gegaan met al mijn}{andere vakken}\\

\haiku{{\textquoteright} In de smalle hooge,,.}{huizen die ik passeer leeft}{men als op de teenen}\\

\haiku{Ben je ergens met?,?}{hem geweest ik zag jullie}{uit de Rul komen}\\

\haiku{Maar je moet niet te, -,.}{veel doen dan dan word je te}{moe dat haalt niets uit}\\

\haiku{Pas nu toch op,{\textquoteright} zegt,.}{hij verschrikt en vergeet zijn}{hand weg te nemen}\\

\haiku{En alles wordt dan.}{zoo angst-aanjagend stil}{en vreemd om mij heen}\\

\haiku{Maar het is ook of.}{een droom mijn gedachten in}{een nevelwaas zet}\\

\haiku{Nu denk je niet zelf,{\textquoteright}, {\textquoteleft}.}{veronderstel iknu w\`ordt}{er in je gedacht}\\

\haiku{{\textquoteleft}Mij is dit en dat,!}{niet gelukt tot nog toe maar}{jullie al evenmin}\\

\haiku{{\textquoteright} - Later, als Biechte,:}{en Kennisse mij oprecht}{vervelen denk ik}\\

\haiku{Of zal het Abram zijn,?}{en Sarai zijn huisvrouw en}{Hagar de dienstmaagd}\\

\haiku{{\textquoteleft}Zou je me tot de?}{bodem uitdrinken als je}{me uitdrinken kon}\\

\haiku{Mijnheer Blommers schudt {\textquoteleft} -....}{ons bijna heftig de hand.}{Collega Elmie}\\

\haiku{Als ik dan met dat - -....}{werk werk dat er eigenlijk}{niets toe doet klaar ben}\\

\haiku{Peter heeft vaak die.}{ingehouden lach-trek als}{hij over mijn werk praat}\\

\haiku{een  heerschzuchtig, -.}{in bezit nemend gebaar}{heel heel aangenaam}\\

\haiku{Vroeger ging ik met.}{Vader en Moeder naar een}{hotel aan de Rijn}\\

\haiku{En mijn man neemt zelfs.}{op een verliefde manier}{de pijp uit de mond}\\

\haiku{Hij streelt daarbij zijn....}{manhaftige knevel en}{glimlacht verteederd}\\

\haiku{dat is er een die.}{de handen op de heupen}{zet en zwierig loopt}\\

\haiku{Natuurlijk, we zijn -!}{ergens ergens tusschen Spiez}{en Interlaken}\\

\haiku{soms te ijl, soms wat,.}{afgewend toch stemmen die}{steeds naderkomen}\\

\haiku{De boomkruinen zijn.}{van doorschijnend geel-groen}{kristal in het licht}\\

\haiku{alles voor jou - zoo,?,?}{was het toch h\`e Elmie en}{jij alles voor mij}\\

\haiku{{\textquoteright} Heftig druk ik mijn.}{duim tegen de spitse punt}{van mijn sierspeld aan}\\

\haiku{Onder veel goede.}{pijn wordt er een stramme kracht}{in mij geboren}\\

\haiku{{\textquoteright} Och, dat vergeet ik,.}{al nog terwijl ik er over}{poog na te denken}\\

\haiku{We wandelen veel,.}{maar een groote wandeling maakt}{me ongeduldig}\\

\haiku{ik naar bed ga, zal -.}{ik jullie schrijven morgen}{in ieder geval}\\

\haiku{Het is of de wind,....}{de lippen vast opeen sluit}{een koude vrieswind}\\

\haiku{{\textquoteright} {\textquoteleft}En dan,{\textquoteright} vervolg ik, {\textquoteleft}?}{in stiltegen\'oeglijk over de}{pottenkast praten}\\

\haiku{Je moet je nu maar,}{schrap zetten je moet je zelf}{dan maar aanpakken}\\

\haiku{Eensklaps grijpt hij mijn.}{bovenarm beet of hij mij}{dooreen wil schudden}\\

\haiku{het is ondenkbaar.}{dat er \'een tooneeldirecteur}{bestaat die d\`at neemt}\\

\haiku{Nu heb ik niets meer -.}{over om op door te gaan nu}{heb ik niets meer over}\\

\haiku{De deur is op slot,,.}{de lamp brandt de gordijnen}{zijn dicht geschoven}\\

\haiku{En wij spreken over,.}{ons zelf maar als achter een}{gordijn van nevel}\\

\haiku{- In mijn gedachten,.}{en als mijn werk niet vlot twist}{ik vaak met Peter}\\

\haiku{Onthoud d\`at nu toch:.}{ik wil niet de achterkant}{van het huwelijk}\\

\haiku{{\textquoteright} Verlegen speel ik.}{met iets dat me toevallig}{in de handen komt}\\

\haiku{mijn leven hangt aan -.}{een zijden draad die zijden}{draad is Uw antwoord}\\

\haiku{Ik voel me iets m\'eer,.}{dan een schoolmeisje als ik}{tegenover haar zit}\\

\haiku{Sinds lang zie ik dan ':}{de dingen van mijn kamer}{weers welbewust}\\

\haiku{{\textquoteleft}Dag Peter,{\textquoteright} roep ik, {\textquoteleft},.}{als tegen een dooveik}{ga nog even uit hoor}\\

\haiku{De eene straat na de,.}{andere jacht ik door ik}{haal iedereen in}\\

\haiku{nee, deze handen,.}{heb ik lief en ook de man}{van deze handen}\\

\haiku{{\textquoteright} Mijn kin bibbert een,.}{beetje ik doe mijn best om}{rustig te praten}\\

\haiku{Wij wonen aan een.}{modderig pad en zingen}{een romantisch lied}\\

\haiku{Het jonge kwieke,.}{paardje danst over de weg ik}{zie het duidelijk}\\

\haiku{Ik zou ook wel mijn:}{hand op de schouder van de}{boer willen leggen}\\

\haiku{Ik keer terug naar.... -.}{Waterloo Er wordt lang op}{de gong geslagen}\\

\haiku{{\textquoteright} Gelaten ga ik.}{achter het roodachtige}{aan naar beneden}\\

\haiku{D\`at moet je er voor,.}{over hebben anders is het}{ook het ware niet}\\

\haiku{In de huiskamer,.}{richt de stilte zich hoog op}{als ik binnenkom}\\

\haiku{Op een dag is het.}{of ze nog maar alleen in}{het groote hooge huis is}\\

\haiku{{\textquoteleft}Later zal ik erg,.}{lief voor Peter zijn als eerst}{mijn boek maar klaar is}\\

\haiku{Het ontbijt sla ik -,,.}{over het bespaart tijd n\'ee ik}{heb er geen trek in}\\

\haiku{gewoon gesprek over.}{dagelijksche dingen kan}{ik niet meer volgen}\\

\haiku{{\textquoteright}, de gedachten in.}{zijn oogen lijken zich toch van}{mij weg te buigen}\\

\haiku{Een meisje met een,,!}{roode muts op tja-ja een}{pittig dingetje}\\

\haiku{Daar heb je dus wel....{\textquoteright} {\textquoteleft},{\textquoteright}, {\textquoteleft}.}{tijd voorPeter zeg ik moe}{het is om het werk}\\

\haiku{Ik zou op de vloer,.}{willen stampen ik zou iets}{willen neergooien}\\

\haiku{{\textquoteleft}U hebt als vrouw een -!}{zeldzame gave U kunt}{zwijgen \`en wachten}\\

\haiku{- Soms is het me of.}{ik dagen-lang niet bewust}{opgekeken heb}\\

\haiku{Ze blijft het langst van,.}{ons drie\"en op ze komt het}{eerst weer te voorschijn}\\

\haiku{Dan legt de Prins de:}{hand op Ruprecht's schouder en}{spreekt met luider stem}\\

\haiku{{\textquoteleft}Adjudant Ruprecht,.}{Reintz ik betuig U mijn groote}{tevredenheid}\\

\haiku{In een oogenblik:}{tijds tracht ik het alles in}{mij op te nemen}\\

\haiku{Thuis blikt ze naar de,.}{eenzaamheid om als naar een}{kwaadwillig schepsel}\\

\haiku{Op een avond, voor we,.}{inslapen omhels ik hem}{plotseling heftig}\\

\haiku{De tranen die te,....}{voorschijn springen lijken mijn}{oogen te verschroeien}\\

\haiku{Daarna breekt er een.}{tijd aan dat ik mij nog meer}{in mijn werk opsluit}\\

\haiku{Onderzoekend kan,.}{ik om mij heen zien spiedend}{kan ik luisteren}\\

\haiku{Door de voorspraak van.}{Burgemeester Reimertz wordt}{die straftijd verkort}\\

\haiku{Enkele dagen.}{geleden vroeg Udo Meeken}{mij voor een souper}\\

\haiku{ik zie een glimp van,.}{haar ronde zwarte oogen haar}{wachtende houding}\\

\haiku{{\textquoteright} zegt hij, {\textquoteleft}een lief oud,,.}{huis heel rustig kamers met}{uitzicht op een tuin}\\

\haiku{{\textquoteleft}Ik wil t\`och 's zien.}{hoe zoo'n goederenloods er}{van binnen uit ziet}\\

\haiku{Ik sla het boek open,,.}{een mooie heldere letter}{toch leest aangenaam}\\

\haiku{Ik druk de knoop als -!}{een kleinood tegen mijn wang}{aan Peter's jasknoop}\\

\haiku{{\textquoteright} Maar het mag niet in,.}{mij glanzen het mag alleen}{maar in mij krimpen}\\

\haiku{Ik heb ook weer wat,?,?,?}{scherps in de mond wat is het}{een takje een doorn}\\

\haiku{{\textquoteright}, prevel ik, {\textquoteleft}en - en...?,?}{moet je man ze niet moet je}{niet naar je man toe}\\

\haiku{Ik leg mijn hand open.}{op de tafel en schuif die}{open hand naar haar toe}\\

\haiku{{\textquoteright} Soms denk ik ook dat.}{ik barstjes-oogen krijg en}{zoo'n gekerfde blik}\\

\haiku{{\textquoteright} In mijn gedachten.}{zet ik Udo Meeken ook vaak}{tegenover mij neer}\\

\haiku{Er staat een potplant.}{met dikke roode bloemen}{in de vensterbank}\\

\haiku{Ik leg het vuur aan,,.}{in de schouw dek de tafel}{steek de olielamp aan}\\

\haiku{Een afgetrokken,.}{meisje-op-leeftijd bleek en}{wat te uitgerekt}\\

\haiku{{\textquoteright} ~ *** ~ Langzaam loop {\textquoteleft}{\textquoteright}.}{ik over het kronkelende}{landpad ophuis toe}\\

\haiku{{\textquoteleft}Hij is werkelijk,{\textquoteright}, {\textquoteleft}.}{bruin geworden denk ikzoo}{bruin als in mijn droom}\\

\haiku{Het zal ook wel weer ',,.}{s niet goed wezen met jou}{niet en met mij niet}\\

\haiku{{\textquoteright} Ik schud mijn hoofd en.}{ik bloos als een bakvisch en}{ik wijs hem de weg}\\

\subsection{Uit: Tusschen de menschen}

\haiku{Toevallig had hij.}{dan die dag zijn werk in het}{laantje van de ahorn}\\

\haiku{{\textquoteleft}Bleu was ie altijd,, '.}{geweest en dat was gek maar}{t wier nog erger}\\

\haiku{In de avondstilte.}{viel helder de uurslag van}{de torenklokken}\\

\haiku{{\textquoteright} {\textquoteleft}'t Is hier nou net ',{\textquoteright}, {\textquoteleft} '{\textquoteright}.}{n toom telde Door Reestvijf}{kippen enn haan}\\

\haiku{Ook al wat je hebt,{\textquoteright}, {\textquoteleft},{\textquoteright}.}{praatte hij onvlot doornet}{als ik tot nog toe}\\

\haiku{Hij kan vragen, maar '{\textquoteright}.}{dan heb zijt antwoord nog}{achter haar tandjes}\\

\haiku{{\textquoteleft}Kijk zoo niet,{\textquoteright} drong zij, {\textquoteleft}{\textquoteright}.}{nauw-hoorbaarik krijg er zoo'n}{angstig gevoel van}\\

\haiku{Netteke liet de.}{klink van het hekje achter}{zich  toevallen}\\

\haiku{Luuk keek nog toen er.}{allang niets meer van hen te}{onderscheiden viel}\\

\haiku{Doch hij had dan de,.}{jaren van een man en de}{kracht van een man}\\

\haiku{en kon er toen ook?}{een vroolijk geluid in een}{plasregen wezen}\\

\haiku{Hij wil opgeruimd.}{kijken en hij moet vechten}{tegen zijn tranen}\\

\haiku{Haar fijne rechte.}{schouders krommen als onder}{den last van een kruis}\\

\haiku{Hebben ze niet van?}{aangezicht tot aangezicht}{de Liefde gezien}\\

\haiku{{\textquoteleft}Wel ou\"etje{\textquoteright}, zei,, {\textquoteleft}?}{ie goedig en ie gaf Kees}{een stoelwat had je}\\

\haiku{De Lente in beeld...{\textquoteright},}{Guus glimlachte hij meende}{dat hij er bedaard}\\

\haiku{{\textquoteleft}Minnie, kindje, je,...}{moet er niet boos om wezen}{dat ik dat zei pas}\\

\haiku{{\textquoteleft}... hij gong en weunde,.}{bij de beek Krith die veur an}{de Jordaon is}\\

\haiku{de Heere is nog}{dezelfde machtige God}{van oudsher en gien}\\

\haiku{{\textquoteright} Zijn wrange glimlach,.}{werd breeder uitdagend stak}{hij zijn hoofd vooruit}\\

\haiku{hadden zij misschien...?}{de ster gezien die de drie}{Wijzen de weg wees}\\

\haiku{{\textquoteleft}Christus de Heere,{\textquoteright}, {\textquoteleft}.}{voltooide ze prevelend}{in de stad Daovids}\\

\haiku{Nou was er nog voor,.}{\'een keer koffie en brood en}{dan was alles op}\\

\haiku{{\textquoteleft}En wat had Jaopie... '...?}{ook iedere maol en}{telkens van nuuwsezeid}\\

\haiku{aorepels heb je ',,,.}{ezeid en dat is er meer niet}{en ook niet minder}\\

\haiku{En de andere,...?}{morgen wat denk je dat hij}{toen in zijn netten}\\

\haiku{{\textquoteright} In de verte, rood,.}{vierkant en nuchter rijst het}{Diaconie-huis op}\\

\haiku{{\textquoteleft}Ik dacht eigenlijk '...?, '...?}{als we d\`at nous samen}{h\`e inn hotel}\\

\haiku{Miekie glipte de,...}{straat op ze wuifde nog even}{en haar oogen lachten}\\

\haiku{Het grove tumult.}{van de groote stad verfijnde}{tot een ijl gerucht}\\

\haiku{Door haar tranen heen,, '...}{wou ze toch nog glimlachen}{maart ging niet best}\\

\haiku{{\textquoteright} lichtte ze koddig, {\textquoteleft}.}{inof je de kamer nog}{gedaan kreeg vandaag}\\

\haiku{- Kijken ze niet erg,,?}{vroeg ze direct riepen de}{lui je nog wat na}\\

\haiku{Maar een gewone ',,.}{straatmuzikant wast ook}{niet zie je vast niet}\\

\haiku{En ze had nog geen,.}{vijf stappen gedaan toen haar}{Moeder overeind kwam}\\

\haiku{Eerst na het derde.}{spelletje durfde Carry}{weer op te k{\`\i}jken}\\

\haiku{Met angstige oogen.}{staarde ze plotseling naar}{den langen mijnheer}\\

\haiku{{\textquoteright} Casparis zuchtte,,.}{en hij schudde het hoofd zijn}{oogen tuurden ver heen}\\

\haiku{Frits was het niet eens,...{\textquoteright}}{die snorkte dat was hij zelf}{Ongeduldig trok}\\

\haiku{{\textquoteright}, angst rilde door hem,.}{heen en beklemmende vrees}{neep zijn adem haast af}\\

\haiku{{\textquoteleft}Toe,{\textquoteright} drong ze schattig, {\textquoteleft},, '.}{wees nou geen stoute jongen}{h\`et is je tijd}\\

\haiku{alleen d\`an...{\textquoteright} Met een.}{energieke hoofdbeweging}{brak Do het  af}\\

\haiku{Hij boog zich er naar,,.}{toe gluurde er door heen zijn}{gezicht verstrakte}\\

\haiku{groene testen en,,.}{roode vergieten bekers}{pullen en poppen}\\

\haiku{{\textquoteleft}Me dochter Marie,...,.}{die-eh die ver-g-g\'o\'odt}{je gewoon Meuje}\\

\subsection{Uit: Het wazige land}

\haiku{Ineens resoluut,,.}{draaide ze zich af nam het}{valies op en ging}\\

\haiku{{\textquoteright} Een rimpel trok rond,.}{haar mond haar smalle gezicht}{leek ineens ouder}\\

\haiku{Smoezelend werd er,.}{nog gauw een veete beslist}{een stomp toegediend}\\

\haiku{Jud sloot de deur en,.}{riep Ap bij zich er was geen}{bevel in haar stem}\\

\haiku{Dan krijg je 'n mooie,.}{griffel van me eentje met}{goud papier er om}\\

\haiku{Eenzaam bleef ze een.}{oogenblik achter in het}{gesloten lokaal}\\

\haiku{Verwonderd merkte.}{Jud dat ze al een poosje}{aan het praten was}\\

\haiku{Arie die je tergde,!}{en uitlokte en je dan}{ineens alleen liet}\\

\haiku{In een teedere.}{schoone vergankelijkheid}{omving haar de herfst}\\

\haiku{De bootromp schoof hoog en.}{norsch in het blinkende}{licht van de lampen}\\

\haiku{Ze sloot de oogen en.}{drukte het hoofd tegen de}{pluchen coup\'e-wand}\\

\haiku{Eerst arriveeren...,,,.}{en dan nee geen sprake van}{weg kom je niet hoor}\\

\haiku{Je brak ons dispuut,{\textquoteright}, {\textquoteleft} '.}{schertste hijwe haddent}{juist over de liefde}\\

\haiku{Dat geldt alleen voor,{\textquoteright}, {\textquoteleft}...!}{de afhankelijke vrouw}{gaf ze toeja di\'e}\\

\haiku{Hij voelde Jud's,.}{blik en keek haar plotseling}{recht  in de oogen}\\

\haiku{{\textquoteright} Hij trok haastig een,.}{jas aan en duwde zich een}{pet over de ooren}\\

\haiku{{\textquoteleft}Ja, als je log\'e's,{\textquoteright}, {\textquoteleft} '.}{heen zijn joolde zeen je}{hebtt wat stiller}\\

\haiku{Uit de hooge eiken.}{in het park tikte nog af}{en toe een druppel}\\

\haiku{Ze ontsloot de deur.}{en groette hem vluchtig bij}{het naar binnen gaan}\\

\haiku{{\textquoteright} Met een ruk werd de.}{deur geopend en Ans stak}{haar hoofd naar binnen}\\

\haiku{{\textquoteleft}Wel nee,{\textquoteright} verwierp ze, {\textquoteleft}.}{kribbigmogelijk kun je}{Jud nog wat helpen}\\

\haiku{En dat viel geloof,,......}{ik nooit iemand op Moes niet}{en Pa niet geen mensch}\\

\haiku{Allemachtig, jij, -.}{bent zoo'n heerlijk vrouwmensch zoo'n}{zoo'n echt raspaardje}\\

\haiku{Ik geloof,{\textquoteright} stelde, {\textquoteleft}.}{ze koel vastdat je nu eerst}{ziet wat ik aan heb}\\

\haiku{Ze spreken voor hun,.}{leeftijd werkelijk keurig}{daar ben ik trotsch op}\\

\haiku{Het was of haar stem.}{in haar borst gevangen zat}{midden in een pijn}\\

\haiku{Zeg, doe geen moeite,,...{\textquoteright}}{ga nu liever terug naar}{Ans ik kom er wel}\\

\haiku{{\textquoteright} zei Toonie, hij schraapte.}{een lucifer af en trok}{de vlam in zijn pijp}\\

\haiku{Rusteloos, in haar,:}{groeiende vrees had ze hem}{telkens opgezocht}\\

\haiku{{\textquoteleft}Ik moet hem toch maar ',{\textquoteright}, {\textquoteleft}'}{n briefje schrijven overlei}{ze feln briefje}\\

\haiku{Och blijf ook maar, blijf,,,.}{maar ik ben zoo bang ou\"e hond}{ik ben zoo eenzaam}\\

\haiku{Oh ja, 't was mooi....}{en helder en er dreven}{geen cadavers in}\\

\haiku{Ruw brak ze de brief,....}{open tuurde verward op de}{enkele regels}\\

\haiku{Ze had allang iets,,....}{onrustigs en als ze je}{aankeek ik weet niet}\\

\haiku{{\textquoteright} {\textquoteleft}Morgen,{\textquoteright} zei Jud in, {\textquoteleft}.}{zich zelf en ze glimlachte}{smartelijkm\`orgen}\\

\haiku{Moes, wat onhandig.}{beverig schepte de soep}{uit de terrine}\\

\haiku{{\textquoteleft}Ik zal heusch wel,{\textquoteright}, {\textquoteleft} '....}{weggaan zei ze kleintjesdoe}{jullie dans wat}\\

\haiku{Natuurlijk w\`el zwaar,,...}{zoo'n eerste keer de stad in}{Moes durfde niet mee}\\

\haiku{{\textquoteright} De gillende lach.}{van de wijven snerpte nog}{lang achter Jud aan}\\

\haiku{Plotseling, in een,.}{heete pijn besefte ze}{haar verworpenheid}\\

\haiku{{\textquoteright} ried hij bondig, {\textquoteleft}maar.}{ons tot de zakelijke}{punten bepalen}\\

\haiku{{\textquoteleft}Oh jongen, j\`ongen,,...,}{laat haar niet zoo om God's wil}{om G\`od's wil ontferm}\\

\subsection{Uit: De zondaar}

\haiku{Maar daar moest je nooit,,.}{over praten hoor tegen geen}{mensch natuurlijk niet}\\

\haiku{Beteuterd liep hij,.}{op het raam toe en klepte}{het bovenlicht open}\\

\haiku{Er was ineens wat.}{huiverigs in de schuwe}{stilte die volgde}\\

\haiku{{\textquoteleft}'k Ben schrikkelijk,{\textquoteright}, {\textquoteleft}','?}{laat excuseerde zen}{schande vin u niet}\\

\haiku{{\textquoteleft}Ze kletst zoo,{\textquoteright} mokte, {\textquoteleft} '...}{hijen dat wast toch ook}{weer niet heelemaal}\\

\haiku{{\textquoteleft}En waaraan kon je '?}{nou zien datt al-weer ver}{in Augustus was}\\

\haiku{Anne-Marie.}{tuurde altijd verder dan}{iemand kijken kon}\\

\haiku{{\textquoteleft}'t Was dan wel 'n,....}{uurtje op z'n allerlangst}{maar toch wel aardig}\\

\haiku{{\textquoteleft}Hij zelf, hij zou zich...}{zeker aansluiten bij de}{Geheelonthouders}\\

\haiku{dat zag je toch vaak '...}{genoeg op de kermis in}{t dorp en Zondags}\\

\haiku{Suffig luisterde,.}{hij er naar en leeg staarde}{hij langs alles heen}\\

\haiku{{\textquoteleft}Nou,{\textquoteright} wou hij dan nog, {\textquoteleft}.}{nuchter overleggenze kan}{er best niet wezen}\\

\haiku{Bloo-stil liepen,}{ze het eerste stukje op}{straat dicht bij elkaar}\\

\haiku{{\textquoteleft}En nou wou Moeder,,...{\textquoteright}}{nog wel dat Jan trouwen ging}{gut dat vind ik nou}\\

\haiku{Oh gr\`a\`ag,{\textquoteright} zuchtte ze, {\textquoteleft}.}{uit een voelbare volheid}{van klachtengr\`a\`ag hoor}\\

\haiku{foetsie...{\textquoteright} De laatste,.}{hap cake smaakte hem niet}{meer hij slikte stroef}\\

\haiku{{\textquoteleft}Nou,{\textquoteright} betuigde ze,, {\textquoteleft}.}{verstikt of haar adem even weg}{wasmaar ik jou \'ook}\\

\haiku{Hij strekte zijn beenen,.}{zoo ver mogelijk uit en}{rekte zijn armen}\\

\haiku{{\textquoteright} Dirk lachte slap een,.}{beetje mee zijn mond had er}{wat onrustigs bij}\\

\haiku{Opeens moest hij toen.}{aan de boerderij denken}{en aan zijn Moeder}\\

\haiku{Dirk keerde zijn hoofd.}{met een ruk naar haar om en}{glimlachte vragend}\\

\haiku{Haastig bevoelde,.}{ze toen haar heet gezicht de}{dikke oogleden}\\

\haiku{Zijn warm gezicht wreef,.}{aaiend langs haar arm zakte}{zwaar af naar haar heup}\\

\haiku{{\textquoteleft}Nou,{\textquoteright} weifelde hij, {\textquoteleft}..{\textquoteright},,.}{nee Snel zonder dorst dronk hij}{de heete thee uit}\\

\haiku{{\textquoteright} Toen dacht hij ook weer.}{aan het wachtende werk en}{kwam gejaagd overeind}\\

\haiku{{\textquoteleft}Er komt misschien 'n,}{vacature aan die school}{op de Brouwersgracht}\\

\haiku{{\textquoteleft}Hij denkt al niet meer,.}{aan mijn getob hij is al}{alles vergeten}\\

\haiku{En dat had ze wel, ',...?}{noodig ookn boel liefs en zachts}{en verder verder}\\

\haiku{Ze verborg het woord,.}{diep in haar gedachten en}{het brak toch weer door}\\

\haiku{Vijftien golden met,{\textquoteright},.}{opslag zei ze kort kleurend}{trok ze haar hand los}\\

\haiku{{\textquoteright}, hij slikte een paar.}{maal achtereen tegen een}{dikte in zijn keel}\\

\haiku{Op 'n goeie dag ben,.}{je opgestapt nou moet je}{niet meer omkijken}\\

\haiku{Ja, hai docht maar over, -.}{z'n studie en den den mos}{je je jakes hou\"e}\\

\haiku{{\textquoteright} Hij hief zijn hoofd op,.}{het was of hij het tegen}{de wind aan drukte}\\

\haiku{En ze voelde zijn,... {\textquoteleft}}{glijdende hand zwaar en warm}{over haar borst haar heup}\\

\haiku{{\textquoteleft}Hij besliste of,.}{zij er niet was hij betrok}{er haar haast niet in}\\

\haiku{En dan nog, wat je,...{\textquoteright} {\textquoteleft}?}{voor kleeren berekende z\'o\'o'n}{bagatelDacht je}\\

\haiku{{\textquoteright}, voor het eerst had ze, {\textquoteleft}!}{weer wat joligs in haar stem}{dan vergis je je}\\

\haiku{Trouwens, z\`elf wou hij,,...{\textquoteright}}{toch ook zoo enorm graag god nou}{zoo met z'n twee\"en}\\

\haiku{Voor 'n verzetje '.}{zou hij misschiens met Jans}{Faber gaan praten}\\

\haiku{Dirk zich dieper naar,,}{haar toe hij zoende heftig}{haar heele gezicht}\\

\haiku{'t Bijt me de keel,,,?}{af wil je dat wel gelooven}{altijd bij je h\`e}\\

\haiku{En Dirk tikte al.}{tegen de paarse ruitjes}{van de keukendeur}\\

\haiku{naaktheid zou hem nou, '...}{denkelijk niet prikkelen}{niets inn stemming}\\

\haiku{,{\textquoteright} een zwaar gevoel steeg,.}{naar zijn hoofd en er kwamen}{vlekken voor zijn oogen}\\

\haiku{Nou, die jongen zou, '.}{toch wel niks gemerkt hebben}{n  groote stoetel}\\

\haiku{Enfin, als hij voor '.}{t toelatingsexamen van}{de H.B.S. maar slaagde}\\

\haiku{En de directeur:}{had hem nog in z'n eigen}{kamer geroepen}\\

\haiku{{\textquoteleft}wat ben ik blij om, '...}{je je hebtt zoo verdiend}{in al die jaren}\\

\haiku{Terloops keek hij maar,.}{naar hen en zag aan ieder}{toch wat ongewoons}\\

\haiku{De schoenmaker van,{\textquoteright}, {\textquoteleft}.}{beneden troefde Toos op}{haar beurtkomt ook niet}\\

\haiku{Van Hasselt blies een.}{mond vol rook uit en snoof}{taxeerend in de geur}\\

\haiku{{\textquoteright} {\textquoteleft}En bezeerde de,?}{man die de ruiten insloeg}{zijn handen niet erg}\\

\haiku{{\textquoteleft}Er zal over jou wat,{\textquoteright}.}{gepraat worden giste ze}{grappig-verwaand}\\

\haiku{De koelte raakte.}{zijn heete oogen aan en zijn}{gloeierig voorhoofd}\\

\haiku{Ze struikelde haast,,.}{bij zijn onverhoedsche greep}{en schoot in een lach}\\

\haiku{En stil lachte ze,.}{mee maar haar lippen voelden}{stijf en trekkerig}\\

\haiku{Ze moest Mevrouw van.}{Haaften ook nog voorstellen}{aan Mevrouw Hubbink}\\

\haiku{En dan de eerste,.}{tien minuten moest ze ook}{niets presenteeren}\\

\haiku{Enfin, ze zou wel... '}{een en ander afneuzen}{van die anderen}\\

\haiku{{\textquoteright} Dirk luisterde, of,.}{hij haar al hoorde komen}{maar alles bleef stil}\\

\haiku{Wrijf jij je gezicht,,...}{maar liever wat af met je}{zakdoek je glimt zoo}\\

\haiku{{\textquoteright}, vorschte ze, voor,, {\textquoteleft} '?}{haar doen levendigkomtt}{Woensdag gelegen}\\

\haiku{Nou doe je maar net,,}{of panje-drinken je}{dagelijksch werk is}\\

\haiku{{\textquoteleft}Branders,{\textquoteright} haalde Jans, {\textquoteleft},,...{\textquoteright}}{uitnee idioot zeg zoo als}{die zich uitsloofde}\\

\haiku{Dat malle gezwam,.}{over die Branders daar werd je}{wee om je hart van}\\

\haiku{{\textquoteleft}O ja, was waar ook,...{\textquoteright} {\textquoteleft}?}{die ziekteBen je nu weer}{heelemaal beter}\\

\haiku{Misschien nog maar 't ',{\textquoteright},}{beste omt te negeeren}{sloeg het door hem heen}\\

\haiku{Och, die soort dingen,, '...{\textquoteright}}{begrijp jij toch niet Hartsen}{jij begrijptt niet}\\

\haiku{{\textquoteleft}'t Zal me smaken,{\textquoteright},.}{zei hij goedig zijn glimlach}{mislukte toch nog}\\

\haiku{om die japonnen,{\textquoteright}, {\textquoteleft},}{te doen zei ze wat zachter}{ik wou maar zeggen}\\

\haiku{{\textquoteright} Teuterig sneed ze,.}{een boterham aan reepjes}{praatte  zeurig}\\

\haiku{want Toos vond dat zoo.}{aardig staan voor de buren}{die er op letten}\\

\haiku{Gek, datzelfde had,,.}{hij vanmorgen op weg naar}{school ook al gedacht}\\

\haiku{En hij had al 'n, '...}{paar keer gedroomd dat hijn}{jongen wurgde}\\

\haiku{{\textquotedblleft}Winde-kind,{\textquotedblright} en, ':}{stereotiep hoorde je na}{Van Looy alsn echo}\\

\haiku{En trouwens, ik had '.}{er geen idee van datt jou}{zou irriteeren}\\

\haiku{ik word 'n duvel,...}{op school en eerst mocht ik ze}{toch graag de jongens}\\

\haiku{Er zette zich een.}{trek van wrevel vast in zijn}{magerbleek gezicht}\\

\haiku{{\textquoteleft}Dwarskijker,{\textquoteright} gromde,.}{hij en zijn donkere stem}{had een warme klank}\\

\haiku{Mokkend ging hij de,.}{kamer uit en op de trap}{bleef hij telkens stil}\\

\haiku{{\textquoteright} Hij struikelde haast,.}{over een steen en schopte die}{nijdig uit de weg}\\

\haiku{Verdomme,{\textquoteright} mokte, {\textquoteleft},,.}{hij schorverdomme toe doe}{dat nou niet kerel}\\

\haiku{Hij vloekte zwaar in,.}{zijn binnenst z\'o\'o zwaar dat hij}{er van hijgen moest}\\

\haiku{t Eigenlijke,:}{begin was er nooit geweest}{dat moest nog komen}\\

\haiku{Jij hebt 't liever,{\textquoteright}.}{niet dan wel bepaalde hij}{toen onomwonden}\\

\haiku{we zaten veel te ',...{\textquoteright}}{krap om aann gezin te}{denken kinderen}\\

\haiku{Als je 't aanstonds,, '.}{druk hebt met je werk op school}{vergeet jet weer}\\

\haiku{ik voel 't onder,, '.}{m'n lessen door even goedt}{is geen bevlieging}\\

\haiku{En Toos wendde als...}{een schuchter jong-meisje}{haar hoofd van hem af}\\

\haiku{Daar had hij toen niet,,?}{op geantwoord nee wist hij}{van die dingen af}\\

\haiku{Het was hem ontgaan.}{dat Jans op zijn verweer niet}{eens geantwoord had}\\

\haiku{En de menschen die',...}{ben best tevreden met de}{schijn dat is alles}\\

\haiku{{\textquoteright} Met klein getrokken,,.}{oogen gluurde hij tegen het}{licht in de straat op}\\

\haiku{De steenen spiegelden,.}{van zon de boomen werden}{al  kaal en zwart}\\

\haiku{zijn ook nog koekjes ',:}{int buffet of als je}{wat anders wenscht}\\

\haiku{Hij hoorde meteen.}{haar zagerige adem en}{de knars in haar mond}\\

\haiku{Want wist ineens dat.}{achter die gedachte zijn}{begeerte wegschool}\\

\haiku{Di\'e moest hij vragen,,,...}{en dan draaide ze hem haar}{wang toe nou ja T\'oos}\\

\haiku{{\textquoteleft}Geef me nou 'n zoen,{\textquoteright}, {\textquoteleft} '?}{hunkerde zegeef me nou}{voor \'e\'en keern zoen}\\

\haiku{Angst spiegelde in... {\textquoteleft}}{de helle schuldbewuste}{oogen van de jongen}\\

\haiku{En het leek wel of.}{hij met de jongen alleen}{in het lokaal was}\\

\haiku{Die jongen pestte.}{net zoo lang tot je je niet}{meer in kon hou\"en}\\

\haiku{Telkens stond er een.}{lange booze stilte tusschen}{hem en een leerling}\\

\haiku{hij maakte van de,...}{stug-verwijtende oogen}{heet-bedeesde}\\

\haiku{{\textquoteleft}Ik wist 't wel, ik '...{\textquoteright}:}{wistt En het leek hem geen}{werkelijkheid meer}\\

\haiku{In elk geval, er,... '!}{was tenminste niks gebeurd}{niksn Geluk maar}\\

\haiku{Toch praatte hij, in.}{zijn baloordheid tegen haar}{of ze naast hem liep}\\

\haiku{{\textquoteright} Eerst ging hij op zijn,.}{stoel bij de haard zitten en}{toen op de divan}\\

\haiku{Vroeger zou ik daar...,{\textquoteright}.}{niet eens aan gedacht hebben}{en n\'ou ze snikte}\\

\haiku{Of zooals die man met,...{\textquoteright}.}{het lupusgezicht laatst in}{de tram Ze rilde}\\

\haiku{Hij leek het niet eens,.}{te hooren keek zoekend om}{zich heen op de grond}\\

\haiku{Hoe lang waren ze?, '...}{nou al hiert leek wel of}{er geen tijd meer was}\\

\haiku{Mit dat klaine kriel,{\textquoteright}, {\textquoteleft}}{schepte hij opken ik ook}{net zoo goed overweg}\\

\haiku{Haar mond viel open, het,.}{leek een pijn-vertrekking}{het was een glimlach}\\

\haiku{Nou nou bin je er,?, -?}{toch niet kwaad om hee dat ik}{dat ik d'r over praat}\\

\haiku{'n mot is maar 'n...}{beesie van niks en hij bait de}{fainste dingen stik}\\

\haiku{, spijt waar je niks mee -...{\textquoteright}}{goedmaken kon en dat dat}{zeere in je borst}\\

\haiku{Gos j\'ongen,{\textquoteright} haalde, {\textquoteleft} - '!}{ze uitik ik weet mit skik}{niet hoe ikt heb}\\

\haiku{Maar onmiddellijk,.}{er-op bedwong ze zich en}{groette kalmbeleefd}\\

\haiku{{\textquoteleft}'t Gewicht is er,.}{ook niet slager en ik moet}{nog m'n centen w'rom}\\

\haiku{{\textquoteright} Dirk lachte gesmoord.}{in het donkere hoekje}{van zijn elleboog}\\

\haiku{Nou, als we eerst maar,.}{weer in huis benne dan kan}{ie ons niet krijgen}\\

\haiku{{\textquoteleft}We gaan eten, hier, doe, ',.}{maar gauw alsn knappe meid}{je slabbetje voor}\\

\haiku{Toch wonderlijk dat,...}{ze niet kwaad gebleven was}{na die avond op straat}\\

\haiku{{\textquoteright} De herinnering...}{aan de sterfnacht van zijn Moeder}{vlamde door hem heen}\\

\haiku{{\textquoteright} Afkeerig keek hij.}{naar het bleeke gegnies in haar}{slap-dik gezicht}\\

\haiku{Ze praatte door, of.}{ze het tegen  iemand}{had die naast Dirk zat}\\

\haiku{Morgen moet je maar '....}{weern extra beurt hebben}{zit niks anders op}\\

\haiku{, tusschen dames die.}{allemaal gezellig met}{z'n twee\"en zaten}\\

\haiku{Wat 'n idee ook om,?}{hier langs te komen niemand}{koopt immers van je}\\

\haiku{{\textquoteright} Meteen nam ze het.}{boordevolle theekopje}{op en wou weggaan}\\

\haiku{{\textquoteright} Kregel onderbrak.}{Toos ineens het aanpraten}{van de winkelchef}\\

\haiku{{\textquoteright} Rusteloos zochten,,.}{haar oogen over en weer in de}{straat de winkels af}\\

\haiku{Ze vergat ook te {\textquoteleft}}{groeten en hield de knop van}{de deur in haar hand.}\\

\haiku{{\textquoteright} Een warm gevoel schoot,.}{in haar op een gevoel als}{een herinnering}\\

\haiku{n Tijd terug leek,,...}{dat uurtje verleden week}{mooie les boeiend wel}\\

\haiku{t neemt 'n massa, ' '.}{tijd ent kost enkel maar}{n beetje kleefstof}\\

\haiku{En Dirk knikte als,.}{een harlekijn of zijn hoofd}{aan een draadje zat}\\

\haiku{En Dirk drukte vrij.}{hartelijk Iet's kil-gladde}{hand in de glac\'e}\\

\haiku{{\textquoteright}, joeg het onthutst door, {\textquoteleft} '...}{hem heennou zie jet zelf}{en \`als je sterk was}\\

\haiku{{\textquoteright} {\textquoteleft}Ja,{\textquoteright} gaf ze snibbig, {\textquoteleft} '.}{toedan zou er weern jaar}{priv\'e opzitten}\\

\haiku{{\textquoteright} {\textquoteleft}Hil,{\textquoteright} kwam hij er dan, {\textquoteleft}?}{toch nog tegen ophoe kom}{je nou op zoo iets}\\

\haiku{{\textquoteleft}Wat zou er anders,{\textquoteright}, {\textquoteleft}?}{wezen drong hij in vreemde}{bevangenheidHil}\\

\haiku{Haast doorloopend, de,, '...}{laatste tijd soms eventjes niet}{n\'ou mett gedicht}\\

\haiku{{\textquoteright} Haar schuw-warme blik.}{ging niet hooger dan tot het}{front van zijn overhemd}\\

\haiku{{\textquoteright} Zijn trieste lachje.}{bleef hem als een krop van leed}{steken in de keel}\\

\haiku{Even was het nog of,,.}{ze schrok tegenspartelen}{wou toen bleef ze stil}\\

\haiku{{\textquoteright} {\textquoteleft}Nee,{\textquoteright} protesteerde, {\textquoteleft} ',.}{ze gegriefdik zegt niet}{om te vleien hoor}\\

\haiku{Hil, lieve kind, na....}{deze eene keer doen we doen}{we weer als eerst hoor}\\

\haiku{{\textquoteright} Lui weerde hij een,.}{langpootige vlieg af zijn}{arm viel slap terug}\\

\haiku{duinzand flonkerde,,...}{helmplanten wiegelden ze}{hoorden wind en zee}\\

\haiku{Gekke stoethaspel,.}{om zoo tegen hem te doen}{ijselijke prul}\\

\haiku{{\textquoteleft}Vast 's kijken zoo,?}{meteen hij zou toch niet iets}{gebroken hebben}\\

\haiku{{\textquoteleft}Ja, nou moest hij zoo,......{\textquoteright}}{meteen opstaan en eten en}{kijven tr\'eiteren}\\

\haiku{{\textquoteleft}En altijd,{\textquoteright} sufte, {\textquoteleft}.}{hij doorkon je alles nog}{vierkant ontkennen}\\

\haiku{je bent...{\textquoteright} {\textquoteleft}Dat,{\textquoteright} knikte, {\textquoteleft}:}{hij tegemoetkomendheb}{je straks al gezegd}\\

\haiku{{\textquoteleft}'t Zou toch niks voor,,.}{mij wezen zoo vervelend}{geen mensch die je ziet}\\

\haiku{Nou ja, of op je, '}{eentjet is merkwaardig}{wat je zoo buiten}\\

\haiku{Nou omdat je er,.}{zoo op aan tamboereert zal}{ik gaan zoo meteen}\\

\haiku{{\textquoteright} {\textquoteleft}Tut-tut,{\textquoteright} pruttelde, {\textquoteleft},.}{hij binnensmondsik knijp er}{tusschen uit strakkies}\\

\haiku{Die goed voor je zorgt,{\textquoteright}, {\textquoteleft}:}{prevelde hij nanou ik}{zal niet ontkennen}\\

\haiku{{\textquoteright} Zijn tanden sloegen,.}{op elkaar met een klikkend}{geluid hij knikte}\\

\haiku{Dat - dat doe je maar -,?}{om wat wat goed te maken}{van vroeger ni\'et}\\

\haiku{ze heeft de slanke,.}{lijn tot in de perfectie}{nou di\'e kan meedoen}\\

\haiku{Hij betaalde de,.}{conducteur en bleef op het}{achterbalkon staan}\\

\haiku{Ruw gooide hij de,.}{ramen open en zakte zwaar}{neer in zijn crapaud}\\

\haiku{{\textquoteright} En Dirk tuurde zoo,.}{afwezig voor zich uit of}{hij niets gehoord had}\\

\haiku{'n Lichte stap had,.}{dat kind zoo of ze de vloer}{haast niet aanraakte}\\

\haiku{En snel boog hij zich,...}{naar het blaadje op zijn knie}{las over fouten heen}\\

\haiku{{\textquoteright} Het gesprenkelde.}{bloemige rose van haar}{koonen werd wat dieper}\\

\haiku{Pas op,{\textquoteright} dreigde hij, {\textquoteleft} ', '.}{ik kant uit je vandaan}{kn{\'\i}jpen alst moet}\\

\haiku{{\textquoteright} {\textquoteleft}Er is ook nog wel ' ',{\textquoteright}.}{n zoetigheidje int}{buffet bood hij aan}\\

\haiku{Hij keek haar enkel,,.}{maar aan star met toch iets van}{een lach om zijn mond}\\

\haiku{Ik kon me op laatst,.}{haast niet meer inhou\"en als}{je langs me ginge}\\

\haiku{{\textquoteright} Haar handen gleden,,... {\textquoteleft}}{over hem heen omvatten hem}{in bezit nemend}\\

\haiku{{\textquoteright} Zoo zag ze er uit,.}{als een furie maar dan toch}{als een mooie furie}\\

\haiku{{\textquoteright} En meteen was het.}{of de tik van de klok haar}{oorvlies aanraakte}\\

\haiku{, had h{\'\i}j nou ook niet?}{allerstomst zitten}{soezen al die tijd}\\

\haiku{{\textquoteleft}Dit laten duren - '.}{nogn beetje dat goeie van}{hem laten duren}\\

\haiku{Hij merkte het op,.}{en zijn adem schokte of hij}{innerlijk lachte}\\

\haiku{{\textquoteright} Een verontrustend:}{gevoel van onvoldaanheid}{bleef in haar achter}\\

\haiku{En Dirk knikte wel,:}{zwoel-vermaakt maar hij zei}{stroef uit de hoogte}\\

\haiku{Met een verbeten...}{lach zag Dirk van haar naar de}{binnenkomenden}\\

\haiku{Kerel,{\textquoteright} luchtte hij, {\textquoteleft}}{eensklaps netelig-oolijk}{zijn verwondering}\\

\haiku{Gunst en dat vind ik,:}{toch heel  gewoon dat heb}{ik wel meer gedaan}\\

\haiku{Als je nu met de?}{vacantie maar niet te moe}{bent om uit te gaan}\\

\haiku{{\textquoteleft}Als ik die man op, '.}{straat tegenkom kijk ikn}{andere kant op}\\

\haiku{{\textquoteright} Haar kleine diepe.}{oogen hielden zelfs onder het}{lachen wat vinnigs}\\

\haiku{Ze zag er in haar.}{witte toiletje nog altijd}{als een meisje uit}\\

\haiku{{\textquoteright} Zijn wit smal gezicht.}{stond z\'oo strak of hij op een}{begrafenis kwam}\\

\haiku{Jij 'n cognacje,?,?}{Hubbink en Bollema wat}{voor vergift kies jij}\\

\haiku{Ik heb 't ontdekt,!}{en ik kijk al twee weken}{naar hun kaartjes uit}\\

\haiku{{\textquoteleft}Oh natuurlijk, de,.}{school en de jongens anders}{wisten ze ook niet}\\

\haiku{De pendule op.}{de schoorsteenmantel sloeg scherp}{en rap de uren af}\\

\haiku{Toos was in een dun:}{hemd-met-kantjes}{bij hem komen staan}\\

\haiku{Hij praatte - opdat -.}{Diet er niets van denken zou}{expres snauwerig}\\

\haiku{leuk dat die haar zoo,, '...}{opzocht moederlijk mensch echt}{n lieve Moeke}\\

\haiku{Ze keek van het groote.}{kil-witte naambord naar de}{norsche hardsteenen stoep}\\

\haiku{Aan alles in  .}{het huis leek iets van angst en}{pijn vast te zitten}\\

\haiku{'t Is heel moeilijk ',,.}{omt uit te leggen u}{begrijpt h\'eel moeilijk}\\

\haiku{Maar in zijn stem kwam.}{haar enkel zijn ernstige}{goedheid tegemoet}\\

\haiku{Och, ik was - ik - ik,...}{ben niet sterk dokter en zoo}{zenuwachtig ik}\\

\haiku{Ook dat dan nog maar ', - - '.}{alst moest ook dat alles}{alst baten kon}\\

\haiku{Ze hoorde er zich,:}{al over spreken tegen Dirk}{zag hem opschrikken}\\

\haiku{{\textquoteright} Het verzonk meteen.}{weer in de gewichtigheid}{van het oogenblik}\\

\haiku{Ze gunde zich niet,}{eens de tijd eerst haar hoed af}{te zetten viel zooals}\\

\haiku{Weer 's ouderwetsch...,{\textquoteright}.}{met z'n beidjes vanavond ze}{glimlachte onvast}\\

\haiku{M'n slaapmiddeltje,{\textquoteright}, {\textquoteleft}...{\textquoteright}}{ook nog hakkelde ze in}{een snikzoo'n hoofdpijn}\\

\haiku{Het was of hem een}{gloed tegemoet sloeg uit haar}{donker-verhit}\\

\haiku{{\textquoteright} Met een nijdige.}{kopruk maakte ze zich los}{van haar tobberij}\\

\haiku{Zijn plekkerig-rood,.}{gezicht leek op te zwellen}{hij zweette hevig}\\

\haiku{Ze kreunden als van,,.}{pijn hun handen knepen ze}{deden elkaar zeer}\\

\haiku{{\textquoteright} Diet kwam hard neer, ze,:}{steunde en ze lachte er}{onzinnig doorheen}\\

\haiku{Grinnikend reikte,.}{hij haar het dampende glas}{kwam naast haar zitten}\\

\haiku{Bedaarder leven......}{weer kregen die uurtjes met}{Diet weer meer waarde}\\

\haiku{{\textquoteleft}God, z'n kop en z'n,',.}{rug alles dee zeer dronken}{waren ze geweest}\\

\haiku{{\textquoteleft}Ja, de heele nacht '...}{bloot gelegen en dan met}{de deur opn kier}\\

\haiku{of zat ze misschien?}{bij de gedekte tafel}{op hem te wachten}\\

\haiku{{\textquoteright} Toen de buitendeur,.}{weer dichtviel werd de stilte}{hem toch te machtig}\\

\haiku{je z-zal toch...,?,}{wel weer je wordt natuurlijk}{weer beter h\`e Diet}\\

\haiku{Ze zou ook wel 'n, '.}{boel verdriet hebben maart}{niet laten blijken}\\

\haiku{In de keuken nam,.}{hij allerlei dingen op}{die hij niet noodig had}\\

\haiku{Zenuwachtig liep,.}{hij naar zijn slaapkamer keek}{op de pendule}\\

\haiku{, reeg zijn schoenen dicht,.}{en liep op de teenen naar Diet's}{kamertje terug}\\

\haiku{Dirk rekte zijn hals,.}{zocht gemelijk-aandachtig}{naar Moeke en Toos}\\

\haiku{Werachtig j\^o, we ',.}{hebt-er wat vaak de praat over}{had je vrouw en ik}\\

\haiku{{\textquoteleft}Ziezoo,{\textquoteright} trachtte Dirk, {\textquoteleft}.}{dan nog vriendelijk-blij}{te zeggenweer thuis}\\

\haiku{Beducht luisterde,.}{hij onder aan de trap maar}{hij kon niets verstaan}\\

\haiku{, ja, jij denkt wel erg?, '.}{humaan h\`e wel int oog}{vallend menschlievend}\\

\haiku{{\textquoteleft}Altijd moest ze je!}{er aan herinneren dat}{zij er ook nog was}\\

\haiku{{\textquoteright} Stiekeme lol kneep,.}{slap-dikke lach-plooien}{aan zijn oogen zijn mond}\\

\haiku{Hij ging er zoo in,.}{op dat hij haast zonder groet}{de kamer uitliep}\\

\haiku{Verdomd waar, al dee',:}{je nog zoo hondsch tegen haar}{di\'e speelde toch maar}\\

\haiku{{\textquoteleft}Tevreden?, 'k zit '.}{nog wel metn paar losse}{steekies \^an je vast}\\

\haiku{{\textquoteleft}Maar nou 't \^an h\'aar,.}{leit durf je maar \'een keertje}{in de acht dagen}\\

\haiku{Het was of hij haar,.}{beet pakken wou en het bleef}{maar een beweging}\\

\haiku{, en ze was er toch,...}{even fijn van gebleven geen}{haar minder lekker}\\

\haiku{{\textquoteleft}Hij moest goed voelen, '}{wat hij aan haar had wat hij}{missen zou alst}\\

\haiku{{\textquoteleft}Anders was 't ook....}{zoo licht in de kamer met}{al die zonneschijn}\\

\haiku{in je netheid ben...}{je gemeener dan die meid}{die dat wel moest doen}\\

\haiku{Nou, als je me noodig,,...}{hebt je hebt maar te bell'n}{ik ben er op slag}\\

\haiku{God, god, zooals ik, is - '. '}{er nog nooit nog nooitn vrouw}{vertrapt en verguisd}\\

\haiku{En onthutste van.}{de verbijsterde wanhoop}{in zijn grijns-op-haar}\\

\haiku{Met z'n tw\'ee\"en over,{\textquoteright}, {\textquoteleft}.}{stond hij gek te prevelen}{met z'n tw\'ee\"en \'over}\\

\haiku{Heul maar weer met Jan,,.}{loop hem achterna ga mee}{naar de directeur}\\

\haiku{Mijnheer De Rijck!}{di\'e kwam toch ook uit zichzelf}{op bezoek bij hem}\\

\haiku{niet glimlachend, maar,...}{ook niet stroef-ernstig want dat}{leek gauw op bangheid}\\

\haiku{{\textquoteright} Langzaam, met beenen zoo,.}{zwaar als lood klom hij de hooge}{blauw-steenen trap op}\\

\haiku{{\textquoteleft}En Toos die 't zoo,...{\textquoteright}}{mooi liet voorkomen hem dit}{ook nog leverde}\\

\haiku{{\textquoteright} Hij moest onderhand:}{ook aldoor schichtig letten}{op de anderen}\\

\haiku{{\textquoteleft}Is dat eigenlijk,,...?}{niet gek voor de eerste keer}{op je verjaardag}\\

\haiku{{\textquoteright} En Dirk boog zich over.}{het tafeltje heen of hij}{niet goed gehoord had}\\

\haiku{, ik maak altied in, ',...}{dat ben ik zoowend van thuus}{snieboon'n en zuurkool}\\

\haiku{Toe zeg nu 's, wat?,,, '?}{zal je nemen Samos port}{n advocaatje}\\

\haiku{{\textquoteright} En al-door was.}{er wat onderdanigs in}{zijn doen en laten}\\

\haiku{{\textquoteleft}Net of ze onder ',.}{n ban zaten ze moesten maar}{gauw aan de borrel}\\

\haiku{{\textquoteright} spiegelde ze Toos, {\textquoteleft} '...{\textquoteright}:}{voormetn heel mooi bloemstuk}{Moeke pruttelde}\\

\haiku{{\textquoteleft}Wat had jij dan ook?}{die verwenschte Moeke}{er bij te vragen}\\

\haiku{En wat moet die vent?,!}{hier kan me niet schelen wat}{hij te zeggen heeft}\\

\haiku{Hij schrok op van een,:}{geluid dichtbij en oogde}{sufverwezen rond}\\

\haiku{{\textquoteleft}Oh nou, die h\^et de... '.}{kuiten genoment heb}{ik hem ook gezeid}\\

\haiku{{\textquoteleft}Nou ja, na wat er -......}{wat er voorgevallen was}{waar je om weg moest}\\

\haiku{'t was niet eens van,...{\textquoteright}.}{jou dat gevalletje Even}{hing er een stilte}\\

\haiku{Maar Diet duwde hem,:}{verder ze hing zwaar tegen}{hem aan en smoesde}\\

\haiku{Maar 't kwam nou mooi, ' ':}{voor mekaart werd nou net}{zooalst wezen moest}\\

\section{Martha S. Bakker en Mieke B. Smits-Veldt}

\subsection{Uit: In een web van vriendschap}

\haiku{kijken die hier kwam,.}{maar we verzwegen dat het}{voor hen was gedaan}\\

\haiku{Nu, van dit naar iets,.}{anders als ik het maar niet}{te lang voor je maak}\\

\haiku{Van zo'n saaie boel als,.}{het hier nu is heeft men zijn}{leven niet gehoord}\\

\haiku{En nu krijg ik een,;}{brief uit Antwerpen van de}{vierentwintigste}\\

\haiku{de koning wil ons,:}{helpen met woorden en met}{werken dubbelop}\\

\haiku{En nu is deze;}{bode toch gebleven en}{reist pas morgen af}\\

\haiku{Song ~ Vergeet niet.}{te schrijven wanneer je denkt}{terug te komen}\\

\haiku{Song, gisteren ben!}{ik bij jouw familie op}{je kamer geweest}\\

\haiku{Onderweg zijn er.}{vijf schepen met Duinkerkers}{aan boord gekomen}\\

\haiku{We hebben alle.}{drie gehuild toen we afscheid}{van elkaar namen}\\

\haiku{Wij hebben op dit.}{moment meer hoop dan ooit voor}{zijn carri\`ere}\\

\haiku{Je zou gezonder.}{zijn als je wat botter was}{of een beetje gek}\\

\haiku{Ik verlang net zo,?}{erg als gij kunt verlangen}{maar wat moet ik doen}\\

\haiku{Hij zei dat door de.}{val van Adam een moordenaar}{daar niets aan kon doen}\\

\haiku{Kinderen die slim,.}{zijn zijn moeilijker op te}{voeden dan domoren}\\

\haiku{zij is goed, zij is,.}{uitermate zuinig ja}{zelfs meer dan zuinig}\\

\haiku{Je broer onderhoudt,.}{contact met zijn vader via}{brieven doe ook zo}\\

\haiku{Naar het besluit wat.}{gij daar zult nemen zal ik}{mij moeten schikken}\\

\haiku{Caelo receptuur.}{sidus immenso sinu}{aeternitatis}\\

\haiku{kunnen wij daar niet.}{naartoe gaan om iets aan de}{tuin te laten doen}\\

\haiku{deze diende nu;}{als trofee of bewijsstuk}{voor de overwinning}\\

\haiku{Uit deze brief kan.}{blijken dat Maria daarvoor}{nu al bang was}\\

\haiku{cordon bleus ridders ();}{van dehoge orde van}{de Heilige Geest}\\

\haiku{Met haar {\textquoteleft}zoetemelks{\textquoteright}.}{hart bleef zij altijd begaan}{met zijn wel en wee}\\

\section{August Snieders}

\subsection{Uit: Werken. Deel 10. Alleen in de wereld. Deel 1}

\haiku{In de stad ziet men.}{echter niets van alle die}{kleine wonderen}\\

\haiku{{\textquoteright} De stuursche man neemt.}{het kind bij de hand en gaat}{den winkel binnen}\\

\haiku{Immers, hij ziet de,.}{kinderen der buurt die op}{den dorpel spelen}\\

\haiku{het is met een kruis.}{bekroond en dat kruis met een}{rouwkrep omgeven}\\

\haiku{Claudine wil het,,}{zoo De man doet het ook maar}{zoo weinig galant}\\

\haiku{Hij staart door de spleet:}{der gordijntjes naar het huis}{van den overbuurman}\\

\haiku{{\textquoteright} Mijnheer Daliski klotst,.}{verder zonder op deze}{stem acht te geven}\\

\haiku{{\textquoteleft}Kom hier,{\textquoteright} roept Mijnheer,.}{Golden en de taalmeester}{verschijnt voor het bed}\\

\haiku{{\textquoteright} {\textquoteleft}Ja, dat is zij...{\textquoteright} {\textquoteleft}En.}{zij heeft wel getoond dat zij}{ons innig bemint}\\

\haiku{{\textquoteright} {\textquoteleft}Dat is waar, 't Is...}{misschien ook beter dat wij}{niet alles weten}\\

\haiku{Welnu, ik zal niet;}{verder herhalen wat die}{vreemdeling zegde}\\

\haiku{Sybrand, ik heb uw,{\textquoteright}.}{gesprek met uwe zuster niet}{beluisterd zegt zij}\\

\haiku{doch ik geef u de -!}{verzekering en onthoud}{deze woorden goed}\\

\haiku{{\textquoteleft}Doch ik geef u de -!}{verzekering en onthoud}{deze woorden goed}\\

\haiku{Op dergelijke;}{zaken heeft Sybrand nog nooit}{ernstig nagedacht}\\

\haiku{Had ik haar alleen,,.}{aangesproken zij had hulp}{moord en brand geschreeuwd}\\

\haiku{hij is echter zoo.}{onthutst dat hij zelfs geen woord}{meer tot Sybrand richt}\\

\haiku{{\textquoteright} Mijnheer Golden keert.}{zich andermaal om en schijnt}{te willen heengaan}\\

\haiku{wat de buurt van hem;}{zegt en wat hij wezenlijk}{in karakter is}\\

\haiku{integendeel 't;}{is of de onnoozele pop}{deze nog vergroot}\\

\haiku{Adriana gaat naar;}{boven en klopt op de deur}{van Mijnheer Golden}\\

\haiku{De oude man staat,.}{voor het venster met den rug}{naar de deur gekeerd}\\

\haiku{{\textquoteright} zegt de oude man,.}{nadenkend en hij blijft strak}{op aen vloer staren}\\

\haiku{De reden dat ik-:}{wil heengaan ligt niet in u}{verre van daar}\\

\haiku{Aller wezen heeft;}{eene andere uitdrukking}{dan die van vroeger}\\

\haiku{{\textquoteright} vraagt de Zuster, nu.}{het kind zonder naar haar op}{te zien voorbij snelt}\\

\haiku{Zij heeft immers recht,.}{op onze onderwerping}{achting en liefde}\\

\haiku{Zij zelve steekt den,.}{brief in den post en gaat nu}{dwars door het stadje}\\

\haiku{Als Mijnheer Golden.}{heb ik niets te zien in al}{uwe moeilijkheden}\\

\subsection{Uit: Werken. Deel 11. Alleen in de wereld. Deel 2}

\haiku{Morgen, morgen denkt,.}{hij en koortsig woelt hij op}{zijne legerste\^e}\\

\haiku{Theodora beurt.}{het meisje op en trekt haar}{zusterlijk tot zich}\\

\haiku{{\textquoteright} mompelt Adriana.}{en slaat in verwarring de}{oogen naar beneneden}\\

\haiku{{\textquoteright} vraagt ze en heft de,,.}{oogen door tranen overwolkt naar}{den jongeling op}\\

\haiku{{\textquoteright} Nu de jongeling,}{eenige minuten later}{we\^er beneden komt}\\

\haiku{Uw zoon is slechts op,.}{\'e\'en punt van u gescheiden}{en dat punt ben ik}\\

\haiku{{\textquoteleft}Ik zal u later,.}{alles zeggen later aan}{u en uwen broeder}\\

\haiku{Nu Dobs den hoek der,;}{straat omkeert hoort hij eensklaps}{zijnen naam roepen}\\

\haiku{{\textquoteright} Juist roept de wachter;}{dat de trein op het punt is}{van te vertrekken}\\

\haiku{{\textquoteright} {\textquoteleft}Daar zit er in gansch,}{de kolonie van Gheel geen}{een die zoo gek zoo}\\

\haiku{Dobs even manhaftig}{iets in de toegestoken}{hand. In den flauwen}\\

\haiku{ik wees hem de deur -,.}{en hij hij glimlachte en}{trok de schouders op}\\

\haiku{haar voorhoofd lag schier,.}{op de knie\"en gedrukt zij}{nokte en snikte}\\

\haiku{De burggraaf beurt den,.}{gevallen naam der Duolet's}{op vergeet dit niet}\\

\haiku{Eindelijk zegt hij,:}{en er is iets welwillends}{in den toon der stem}\\

\haiku{Heb geen vrees, Eenige,.}{ruiten aan stukken en de}{storm zal voorbij zijn}\\

\haiku{een denkbeeld dat zich,,.}{ik weet niet hoe verspreid had}{en veel geloof vond}\\

\haiku{hij kan het denkbeeld. '}{aan die ijselijke dood}{niet verwijderen}\\

\haiku{{\textquoteright} {\textquoteleft}Gij beloont mij slecht,.}{voor al de diensten die ik}{u bewezen heb}\\

\haiku{Ik zal u onder,}{den voet vertrappen indien}{ge mij niet weergeeft}\\

\haiku{- maar uit achting voor.}{u wilde ik u over den}{toestand inlichten}\\

\haiku{gij hebt vandaag geen,,,:}{vuur geen gloed geen vlammende}{woorden geen poezi\"e}\\

\haiku{En waarom mij zelfs!}{om mijn eigen dwaasheden}{te bekommeren}\\

\haiku{Ik wil ze in den,.}{wijn verdrinken zij drijven}{gedurig boven}\\

\haiku{indien onder dat,,....}{zwarte baarkleed nog eens iets}{klopte leefde dacht}\\

\haiku{Dobs ziet dit en werpt.}{de koord achteloos op den}{stoel v\'o\'or de bedste\^e}\\

\haiku{Er volgt eene nieuwe.}{handschudding van wege den}{hartelijken Pool}\\

\haiku{Nu gaat hij naar de,:}{deur opent deze en zegt met}{diepe ontroering}\\

\haiku{{\textquoteleft}Ik heb mij door den,;}{nijd den haat en de gramschap}{laten verblinden}\\

\haiku{ik keer terug, hoe,,!}{laat ook met het wee en het}{berouw in de ziel}\\

\haiku{De inktkoker is,.}{uitgedroogd de stalen pen}{is als weggeroest}\\

\haiku{Nu opent hij eene deur,;}{en staat voor een steenen trap die}{naar beneden leidt}\\

\haiku{het slot wijkt en eene,,.}{kleine zware ijzeren}{deur draait buitenwaarts}\\

\haiku{Bij het verlaten.}{der kapel biedt de burggraaf}{zijner vrouw den arm}\\

\subsection{Uit: Werken. Deel 49. Allerlei}

\haiku{{\textquoteleft}'t Was misschien ook}{jaren geleden dat gij}{haar niet meer bezocht}\\

\haiku{Janmaat durfde niet,:}{landwaarts ingaan omdat ik}{hem niet volgen kon}\\

\haiku{Hij zweeg en ik had,.}{den moed niet dat stilzwijgen}{te verbreken VI}\\

\haiku{O, de eischen zijn!}{nog veel omvattender voor}{den romanschrijver}\\

\haiku{Antwerpen was toen,,,! '}{nog zoo glinsterend zoo trotsch}{zoo prachtig niet neen}\\

\haiku{Wy komen u als.}{koning groeten En kennen}{u voor onzen Heer}\\

\haiku{Links van u, in het,.}{oostelijk gedeelte is}{een ronde holte}\\

\haiku{Welke pelgrim, die,}{de grot bezocht drukte in}{diepe ontroering}\\

\haiku{De aarde verkeert:}{onder den invloed van de}{hemelsche woorden}\\

\haiku{{\textquoteleft}Wie zoude waagen,.}{yets te hervatten dat in}{de vierschaar van UEd}\\

\haiku{doch de gewone,:}{bezoeker kent men aan den}{kalmen vasten stap}\\

\haiku{Niettemin beschouwt!}{men den theater als het}{toppunt van ideaal}\\

\haiku{De diplomaat kan;}{zoo wat veertig \`a twee en}{veertig jaar oud zijn}\\

\haiku{beiden gaan nu hand,.}{aan hand over den weg in de}{richting van het dorp}\\

\haiku{Baron Davis staart;}{met stillen glimlach in de}{vlammen van het vuur}\\

\haiku{Zijn naam wordt ten prooi,.}{geworpen aan het publiek}{dat niets eerbiedigt}\\

\haiku{baron Davis kan:}{er niet toe besluiten de}{vensters te openen}\\

\haiku{het getingel brengt -....}{hem integendeel Edil te}{binnen Edil die zingt}\\

\haiku{Twintig jaren is;}{het geleden dat baron}{Davis hier dwaalde}\\

\haiku{doch zij is jong en.}{blond en blijft bewegingloos}{aan den ingang staan}\\

\subsection{Uit: Werken. Deel 35. Arme Julia}

\haiku{In dezelfde straat.}{en voor een prachtig hotel}{hield het rijtuig stil}\\

\haiku{{\textquoteleft}Ik vraag u niet hoe;}{gij mijne verblijfplaats hebt}{kunnen ontdekken}\\

\haiku{Gij offert alles;}{op aan het uiterlijke}{der familie-eer}\\

\haiku{{\textquoteleft}Julia,{\textquoteright} klonk de stem, {\textquoteleft},.}{van den dokterkom dan toch}{beneden lief kind}\\

\haiku{Doch het was mijn plicht,.}{te spreken toen anderen}{hardnekkig zwegen}\\

\haiku{de fortuin der Van.}{Hoogenhuyzen weegt niet op}{tegen de zijne}\\

\haiku{vernielt, verplettert,;}{vermoordt de eer en de faam}{eener familie}\\

\haiku{er ligt eene helsche.}{nieuwsgierigheid op aller}{trekken te lezen}\\

\haiku{Later had zij, in,;}{hare droomen die vogels}{en bloemen bezield}\\

\haiku{Er zat toch een goed,!}{een nog kinderlijk hart in}{vader Benninck}\\

\haiku{Eindelijk werd de.}{deur geopend en mevrouw}{d'Arton trad binnen}\\

\haiku{{\textquoteright} {\textquoteleft}Wie....{\textquoteright} en de stem van, {\textquoteleft}?}{de schuldige vrouw beefde}{wie is uw vader}\\

\haiku{mij zoo innig aan,.}{het hart dat ik u steeds goed}{en braaf gedroomd heb}\\

\haiku{ik kan, dag en nacht,.}{om het brood te verdienen}{dat gij mij toereikt}\\

\haiku{de nieuwe dag ving,?}{aan Wat zal hij der arme}{Julia aanbrengen}\\

\haiku{De heer d'Arton zag.}{het en de koele man scheen}{er door bewogen}\\

\haiku{{\textquoteright} mompelde hij en,.}{trad het vertrek binnen waar}{mevrouw zich bevond}\\

\haiku{Zou de heer d'Arton?}{weten dat Julia Martin}{hare dochter is}\\

\haiku{Mevrouw d'Arton was.}{nog altijd aan duizenden}{gissingen ten prooi}\\

\haiku{{\textquoteright} Het meisje dacht niet.}{anders of mevrouw was in}{de hersens gekrenkt}\\

\haiku{Maar ik bid u, in,;}{naam van al wat u heilig}{is verlaat dit huis}\\

\haiku{Ja, 't was akelig, '.}{daar docht was er beter}{toch dan in dat huis}\\

\haiku{In de verte rijst.}{de toren van Saventhem}{voor haar oog  op}\\

\haiku{Hoe meer Julia het,.}{dorp naderde hoe drukker}{de weg bezocht werd}\\

\haiku{Hoe zou eene moeder!}{zich in de stem van haar kind}{kunnen vergissen}\\

\haiku{Ik ben te voet van,....}{Brussel gekomen mijne}{voeten zijn doorwond}\\

\haiku{Haar echtgenoot heeft;}{haar verlaten en zelfs om}{haar nooit we\^er te zien}\\

\haiku{{\textquoteright} De worm sloeg de deur.}{toe en het onmetelijk}{graf werd weer doodstil}\\

\haiku{Mevrouw d'Arton hield:}{den adem in en antwoordde}{op dat geklop niet}\\

\haiku{Moeder, wat heb ik,!}{u lang gezocht lang gesmacht}{naar uwe omhelzing}\\

\haiku{Ik heb geen recht om.}{uwe handelwijs van vroeger}{te onderzoeken}\\

\haiku{Straks zal de wereld.}{zich andermaal om u en}{mij bekommeren}\\

\subsection{Uit: Werken. Deel 37. Avond en morgen}

\haiku{{\textquoteright} hervatte Clara,.}{en streelde de magere}{wangen des grijsaards}\\

\haiku{Uren lang stond zij soms;}{droomend voor het venster en}{tuurde de straat in}\\

\haiku{{\textquoteleft}Gij ziet nu wat kwaad,.}{welke sleep rampen eene booze}{tong kan voortbrengen}\\

\haiku{hij voelde, dat hij -,.}{er onder bezweek maar toch}{hij torschte die}\\

\haiku{Alles scheen stil en;}{akelig in zijn doodlaken}{te zitten droomen}\\

\haiku{een traan blikkerde,.}{in zijn oog en diep zuchtend}{verliet hij het graf}\\

\haiku{{\textquoteleft}Het is uw feestdag,...{\textquoteright}.}{juffer Liva stamelde}{de jongen bedeesd}\\

\haiku{Waarop Liva we\^er,,:}{met eene zilverachtige}{stem antwoordde}\\

\haiku{Bram zat op den grond,.}{het aangezicht in de twee}{handen verborgen}\\

\haiku{Liva had geen woord;}{gesproken over de ziekte}{van Guido's vader}\\

\haiku{{\textquoteright} De fabrikant had.}{gesidderd toen het meisje}{die woorden uitsprak}\\

\haiku{Bram werd bang voor dat,.}{wezen en deinsde een paar}{schreden achteruit}\\

\haiku{Zie, Guido, ik bracht,}{Vondel's treurspelen me\^e om}{u te verzoeken}\\

\haiku{Zij werpen u van,!}{verre hunne kushandjes}{toe juffer Liva}\\

\haiku{{\textquoteright} Er lag een glans van.}{zaligheid over het gelaat}{der blinde verspreid}\\

\haiku{{\textquoteright} {\textquoteleft}Ik ben gelukkig,,,;}{mijnheer niet voor mij maar voor}{u en uwe dochter}\\

\haiku{Mijnheer, gij zeidet,:}{zooeven dat gij geene luiaards}{in uwen dienst duldet}\\

\haiku{nu, integendeel,,!}{joeg het als eene klok welke}{eene ramp verkondigt}\\

\haiku{Bram weende lang en.}{smartelijk op het graf van}{den miskenden man}\\

\haiku{maar de zon was zoo,;}{weldoende de hemel zoo}{frisch en verkwikkend}\\

\haiku{doch men had gehoopt.}{den jongeling een pijnlijk}{gevoel te sparen}\\

\haiku{het was alsof hij.}{zijn heerlijken tooverdroom reeds}{verwezenlijkt zag}\\

\haiku{Sedert de laatste,;}{maal dat wij hem zagen was}{hij veel veranderd}\\

\haiku{zij gevoelde zich.}{bloedig in haar gevoel van}{eerbaarheid gekwetst}\\

\haiku{welnu, die jongen,,.}{is mijn trouwe leidsman mijn}{eerste vriend geweest}\\

\haiku{Alvorens in te,:}{stappen wendde mevrouw zich}{tot Bram en zeide}\\

\haiku{{\textquoteleft}Maar,{\textquoteright} zeide Bram tot, {\textquoteleft},,...,!}{zich-zelvenniet dansen}{Brammetje neen neen}\\

\haiku{bleek was zijn gelaat.}{en een diepe ontroering}{had hem bevangen}\\

\haiku{doch dagen, weken,.}{en maanden verstreken en}{niemand keerde we\^er}\\

\haiku{{\textquoteright} {\textquoteleft}Laat ons aan al die,!}{wonderlijke dingen niet}{meer denken moeder}\\

\haiku{Peerke duwde het.}{weg en kwam met den schelmschen}{Reinaart voor den dag}\\

\haiku{In de verte klonk:}{eene mannenstem welke het}{zoo gekende lied}\\

\haiku{Het is hier wel wat,;}{veranderd sedert ik er}{het laatst geweest ben}\\

\haiku{{\textquoteleft}Slecht, mijn goede heer,{\textquoteright};}{zeide de oude man op}{mismoedigen toon}\\

\haiku{heel het dorp was op.}{de been om de bruid en den}{bruidegom te zien}\\

\subsection{Uit: Werken. Deel 48. Beelden uit ons leven}

\haiku{Hugo stond, eene poos,.}{stom van verwondering over}{die driftige taal}\\

\haiku{Het was hem of haar.}{naam een balsem van troost op}{zijn lijdend hart goot}\\

\haiku{Bij die woorden wierp.}{Richard een spottenden blik}{door de werkkamer}\\

\haiku{de dochter, welke,;}{Richard kende zou er aan}{mama over spreken}\\

\haiku{want nooit zag men nog.}{van zijne stukken in de}{tentoonstellingen}\\

\haiku{Hugo geleidde.}{haar naar den leuningstoel en}{deed haar ne\^erzitten}\\

\haiku{Naar wien anders dan,.}{naar u die hem altijd zoo}{liefderijk ontvangt}\\

\haiku{Zijn kleed is niet meer,;}{in de mode zijn hoed ruw}{en ongeborsteld}\\

\haiku{- Uw naam, mijnheer, heeft,,.}{daarvoor vergeef het mij nog}{geen waarborg genoeg}\\

\haiku{De koopman sloeg de.}{gewaarwordingen op Hugo's}{aangezicht gade}\\

\haiku{Hij greep den beker.}{der wanhoop vast en dronk er}{milde teugen uit}\\

\haiku{en mag eene zuster?}{de hartepijn van haren}{broeder niet stillen}\\

\haiku{Denk niet dat ik mij.}{in mijn eigen woning zal}{laten beleedigen}\\

\haiku{Ziedaar dan de bron,! -!}{dier liefde welke zij den}{rijkaard toedroeg Mensch}\\

\haiku{De werkzaamheid bracht;}{in het huis van Rodolf een}{nederig bestaan}\\

\haiku{- Miskend genie, sprak,;}{de kenner uw vaderland}{weigert u eene plaats}\\

\haiku{Gij denkt dat ik ook -,!}{kom om u een spotwoord toe}{te werpen neen Casper}\\

\haiku{zelfs die heer niet meer,.}{die beloofd had ons jaarlijks}{te komen voldoen}\\

\haiku{ik zou geene rust meer.}{hebben voor dat ik van hier}{ware weggegaan}\\

\haiku{Ik zie u altoos,....}{droevig en gij weendet straks}{zoo ongedwongen}\\

\haiku{Maar wist gij, hoe de;}{liefde tot eene moeder mij}{in de ziel brandde}\\

\haiku{toen zij voor mij bad;}{en zich slechts met geweld van}{mij liet wegscheuren}\\

\haiku{- Dat is inbeelding,,.}{Mina sprak de jongeling}{op bevenden toon}\\

\haiku{terwijl de zieke....}{Mina voor hen den God der}{lijdenden bad}\\

\haiku{maar wij, wij leven,....}{altijd altijd met nieuwe}{wonden aan het hart}\\

\haiku{Als eene wolk dreef de.}{hondengroep in golvende}{bogen hem voorbij}\\

\haiku{Zij juichte, omdat.}{zij hare gezellinnen}{zoo ver achterliet}\\

\haiku{zij begrijpt niet wat,,....}{het zegt voor eene moeder een}{kind te verliezen}\\

\haiku{zoete belooning!}{voor de bewezen vriendschap}{in het ongeluk}\\

\haiku{Lezer, keer met mij,.}{terug tot den morgen van}{den vorigen dag}\\

\haiku{Wellicht ook hadden:}{zij alle hoop op aardsche}{hulp opgegeven}\\

\haiku{- en hij bedekte,.}{met zijne handen de oogen}{in tranen gebaad}\\

\haiku{Edelaardig man, zeg,?}{mij wat is er van mijne}{zuster geworden}\\

\haiku{want men dacht het spook.}{van den hongersnood aan de}{deur te zien kloppen}\\

\haiku{Ik vond niets in de,,,!...}{wereld dat mij van deugd van}{liefde van God sprak}\\

\haiku{Ik sprong toe - met een;}{geweldigen ruk bleef mij}{dien schat in de hand}\\

\haiku{maar zij zijn thans niet,.}{daar om die groote tranen van}{Bertha's wang te wisschen}\\

\haiku{En toch was hij het,;}{wel die haar op het pad der}{ondeugd gesleurd had}\\

\haiku{want de wanhoop heeft;}{mij voortgedreven op het}{pad der vertering}\\

\haiku{Kind, ging zij voort, en;}{zag met een verbijsterden}{oogslag op het wicht}\\

\haiku{Dan kuste zij het,,.}{vaarwel totdat zij komen}{zou en drukte toe}\\

\haiku{- men zag dit zoowel aan.}{zijne houding als aan die}{van het gezelschap}\\

\haiku{Gaat, en zegt aan de,....}{justitie dat ik het ben}{die hem vermoord heb}\\

\haiku{Te vergeefs trachtte;}{hij echter die gedachten}{te verwijderen}\\

\haiku{al was zij dan ook,.}{boelinne moordenares}{van haar kind geweest}\\

\haiku{Tranen stroomden den.}{ouden man door de rimpels}{van zijne wangen}\\

\subsection{Uit: Werken. Deel 20. Burgerdeugd}

\haiku{lichtgeloovigheid....}{is dikwijls een kenmerk van}{onschuld en eenvoud}\\

\haiku{Dikwijls stond Zeno;}{op dien kring van lijden en}{smarte te staren}\\

\haiku{Wat sloeg het hart van!}{den vader dankbaar voor dat}{engelachtig kind}\\

\haiku{des nachts beven zij.}{voor het geruisch van het}{ritselende loover}\\

\haiku{als er tranen over,;}{uwe wangen vloeien bij het}{zien van het lijden}\\

\haiku{De menschheid klimt;}{slechts van trap tot trap naar het}{punt der volmaaktheid}\\

\haiku{Als ik u sprak van,:}{de liefde van twee harten}{dan riept gij mij toe}\\

\haiku{Maar wat zal het mij?}{baten dat die juichtoon der}{vernieling opstijgt}\\

\haiku{Kom, geliefd overschot,...}{van mijn aardsch geluk kom hier}{aan mijn jagend hart}\\

\haiku{geen zachter genot,!}{dan het vergelden van het}{booze door het edele}\\

\haiku{zegde zij smeekend,,,!}{ik ook heb een vader die}{misschien God weet het}\\

\haiku{- Ik heb mij boven,;}{den kleingeestigen haat der}{menschen verheven}\\

\haiku{De graaf had langen,,;}{tijd geboeid dat ijselijk}{schouwspel moeten zien}\\

\haiku{niet zoo, dat men haar;}{liefdevol aan het harte}{zou kunnen drukken}\\

\haiku{Indien er nog een.}{liefdevonk voor uwe dochter}{in het harte gloeit}\\

\haiku{- Schiet toe, indien gij,;}{hem vermoorden wilt vermoord}{dan ook uwe dochter}\\

\haiku{doch er kwam geen woord;}{van het voorgaande leven}{over zijne lippen}\\

\haiku{- en de jongeling -!}{boog de knie dan ben ik toch}{zinneloos van u}\\

\haiku{Ach, laat mij deze,,...}{niet zoeken zooals Rosa in}{het golvend water}\\

\haiku{Deze ook had een,.}{somber gevoel dat hem zwaar}{op de borst drukte}\\

\haiku{gij zijt Rosa, die,!}{uit het water opstijgt om}{mij te martelen}\\

\subsection{Uit: Werken. Deel 7. Dit sijn Snideri\"en}

\haiku{{\textquoteright} En vurig kuste,.}{zij de moeder wier haren}{reeds sneeuwwit waren}\\

\haiku{Wat vond de lieve,,!}{blinde veel deelneming veel}{vriendschap veel liefde}\\

\haiku{Hij scheen er zich in.}{te verlustigen dat ik}{hem niet herkende}\\

\haiku{hij, en het kwam mij,,.}{voor dat hij bij het heengaan}{diep was aangedaan}\\

\haiku{Hij zakt op de bank, -.}{terug zijn hoofd valt op de}{borst hij is dood}\\

\haiku{Maar zij staken wel,,....}{diep zeer diep hunne vingers}{in onze zakken}\\

\haiku{In hunne schaduw,;}{hebben allen gerust die}{ons zijn voorgegaan}\\

\haiku{maar de stallen en.}{schuren zijn ledig en het}{huis is gesloten}\\

\haiku{Ik zette mij een;}{oogenblik op de oude}{bank v\'oor het huis neer}\\

\haiku{Tegen den mast van,.}{het vaartuig leunt eene vrouw met}{een kind op den arm}\\

\haiku{Dora staat recht aan,.}{het roer en haar oog brandt op}{de beschuldigster}\\

\haiku{{\textquoteright} En Dora, die nu,:}{ook aan het gebeurde denkt}{zegt in haar eigen}\\

\haiku{{\textquoteright} {\textquoteleft}Neen,{\textquoteright} was het antwoord, {\textquoteleft},,}{neen hij is een dichter en}{volgens sommigen}\\

\haiku{Hij volgde haar met}{bangen oogslag en begreep}{wel tot wat volkje}\\

\haiku{{\textquoteright}, hare zuster, de,,?....}{zwarte Peternel te Thoir}{gehangen en zij}\\

\haiku{Kortom, Brussel was,.}{toen reeds eene stad waar men zich}{heerlijk vermaakte}\\

\haiku{{\textquoteleft}Baron,{\textquoteright} zegde de, {\textquoteleft}!}{markiesgij hebt gisteren}{gelukkig gespeeld}\\

\haiku{doch aan terugkeer.}{in de samenleving was}{niet meer te denken}\\

\haiku{De zieke wilde,;}{spreken doch hare lippen}{weigerden de spraak}\\

\haiku{{\textquoteright} ~ De notaris,,.}{bewoont in de kom van het}{dorp een groot steenen huis}\\

\haiku{Heemrik glimlacht en;}{bevestigt met een hoofdknik}{Rika's woorden}\\

\haiku{Het huis staat aan den,.}{zoogezegden dijk die dwars}{door de heide snijdt}\\

\haiku{Nu zitten de twee;}{vriendinnen op de ruwe}{bank nevens de deur}\\

\haiku{{\textquoteright} {\textquoteleft}Nu,{\textquoteright} antwoordt Frida, {\textquoteleft}{\textquoteright};}{die wel begrijpt wie ze met}{dat woordboos bedoelt}\\

\haiku{doch Frida durft zich:}{op het slibberige pad}{niet verder wagen}\\

\haiku{die Heemrik smokkelt,,:}{die Heemrik stroopt die Heemrik}{schijnt een dief te zijn}\\

\haiku{Toch sprak er eene stem,:}{in haar die haar telkens met}{vastheid deed zeggen}\\

\haiku{Er zal eene stem in,.}{zijn hart weerklinken die hem}{over het sneeuwveld roept}\\

\haiku{{\textquoteright} Nu het avond wordt, opent,:}{zij de deur met eene spleet en}{steekt het hoofd buiten}\\

\haiku{Wie duivel zou zich!}{op dat ijsbed neerleggen}{zooals de dienst gebiedt}\\

\haiku{hij is overtuigd dat.}{zij bezig is met hem eene}{booze pert te spelen}\\

\haiku{hij meent te zullen.}{vallen en moet zich aan de}{putmik vasthouden}\\

\haiku{{\textquoteright} onderbreekt Lindorp,.}{verschrikt en er overvalt hem}{waarlijk eene beving}\\

\haiku{{\textquoteright} Dat woord {\textquoteleft}trouwen{\textquoteright} jaagt.}{Rika eene huivering over}{de ledematen}\\

\haiku{Nu het kind dood is,.}{wil Rika dat Frida naar}{huis zal terugkeeren}\\

\haiku{het grasperk is met,,.}{witte bloemekens wit als}{sneeuwvlokjes bezaaid}\\

\haiku{Knecht en meid zeggen;}{dat de jeneverduivel}{er geweldig spookt}\\

\haiku{de hinnikende.}{paarden galoppeeren angstig}{over het akkerland}\\

\haiku{*** Het erfgoed is in;}{het bezit van Rika en}{Heemrik gekomen}\\

\haiku{Neen, we hadden den;}{Brander nooit een speldepunt}{in den weg gelegd}\\

\haiku{Eindelijk werd het;}{geld op de tafel van den}{veldmaarschalk geschud}\\

\haiku{Jan de pijper was,:}{een jonge maat met vlug oog}{en blonden haarbos}\\

\haiku{Doch de vrijbuiter.}{hoopte nog altijd op den}{aanbrekenden nacht}\\

\subsection{Uit: Werken. Deel 39. Fata morgana. Deel 1}

\haiku{De knaap weet het niet,:}{en het zet hem ook niet tot}{onderzoeken aan}\\

\haiku{{\textquoteright} De moeder neemt het,;}{jongste kind op den arm het}{tweede bij de hand}\\

\haiku{of bij het effen,.}{watervlak waarboven de}{reiger langzaam drijft}\\

\haiku{{\textquoteright} en hoorbaar klopt zijn.}{hart bij het naderen van}{de zwarte bende}\\

\haiku{zij houdt zich bezig -.}{met wat Parijs betreft niet}{met wat hier omgaat}\\

\haiku{{\textquoteright} De lofspraak brengt op.}{het aangezicht van Christ geen}{spier in beweging}\\

\haiku{hij komt terug als, '}{het nieuwe werkuur aanvangt}{ens Zaterdags}\\

\haiku{doch rood wordt ze, rood,:}{als de kolroos wanneer de}{jongen lachend zegt}\\

\haiku{{\textquoteright} Hooger streven de.}{droombeelden van den jongen}{kunstenaar niet}\\

\haiku{zij huldigde de,;}{leer der stoffelijkheid den}{uiterlijken vorm}\\

\haiku{De aarde - 't is -;}{altijd zij die sprak is met}{wellusten overzaaid}\\

\haiku{Theo Rigobert zal.}{u een der ontelbare}{antwoorden geven}\\

\haiku{doch, tot overmaat van,.}{ramp vonkte er liefde voor}{Doria in zijn hart}\\

\haiku{Doch hij vergat dat;}{er reeds zoovele jaren}{vervlogen waren}\\

\haiku{{\textquoteright} Op een honderdtal;}{stappen verder verlaat de}{pastoor juist de kerk}\\

\haiku{doch de hond loopt snel,,.}{en waakzaam met gespitste}{ooren rechts en links}\\

\haiku{Een pistoolschot galmt.}{en wordt twee of driemaal in}{de verte herhaald}\\

\haiku{V\'o\'or hem, in den stal,;}{staat een man van tamelijk}{zware gestalte}\\

\haiku{{\textquoteright} {\textquoteleft}Zijt ge in zoo'n groot,,,?}{huis in een waarachtig steenen}{huis geboren Wim}\\

\haiku{Om die reden had,,.}{dan ook de rakker tot nu}{toe voor mij gebukt}\\

\haiku{Bij elken dronk kwam;}{er eene soort berengebrul}{uit  zijne keel}\\

\haiku{Men had nooit iets meer.}{van hem gehoord en elkeen}{dacht dat hij dood was}\\

\haiku{{\textquoteleft}Gij zijt ook zoo wild,,{\textquoteright}}{jongenlief en oom Morris}{meent het goed met u.}\\

\haiku{Men hoorde de stem,,.}{van den bandhond den mop den}{wolfshond en den taks}\\

\haiku{Eenige minuten.}{later verscheen de jongen}{der rarekiekkas}\\

\haiku{Eindelijk hield de.}{wagen op eene binnenplaats}{van een gebouw stil}\\

\haiku{Klonken ze laag, dan;}{keek ik naar den vloer om te}{zien waar zij bleven}\\

\haiku{jammer, ze wist dat.}{ze schoon was en de moeder}{wist het nog beter}\\

\haiku{{\textquoteright} Nora door hare.}{trotsche en verachtende}{onverschilligheid}\\

\haiku{de snikken van mijn,.}{hart nabootsen tikte de}{hangklok beneden}\\

\haiku{De Fitzel's werkten,}{van vader tot zoon bij den}{instrumentmaker}\\

\haiku{Vele bekenden,,;}{en dat deed mij huiveren}{gingen mij voorbij}\\

\haiku{Meester Krok, vroeger,;}{zoo vroolijk en welwillend}{spotziek sprak geen woord}\\

\haiku{Zij, zij wist dus wat!}{er gebeurd was en wat er}{nog gebeuren zou}\\

\haiku{'t Was God geklaagd,,!}{zegden de buren een wees}{uit te plunderen}\\

\haiku{Toen ik een eind weegs,:}{van het dorp verwijderd was}{wendde ik mij om}\\

\haiku{Wij spraken over het,,.}{verre land toen het kind met}{de pop wakker werd}\\

\haiku{Het verre land kreeg.}{voor mij eene betooverende}{aantrekkelijkheid}\\

\haiku{Ik sidder nog als,}{ik denk aan dat bonzend en}{klapperend geloei}\\

\haiku{Het zou voor mij een,!}{feest zijn dien eerlijken man}{de hand te drukken}\\

\haiku{Deuren en vensters,:}{waren gesloten en op}{eenen plakbrief las men}\\

\haiku{Ik heb z\'o\'o veel, z\'o\'o,.}{lang geleden en ik zie}{nergens uitkomst meer}\\

\haiku{{\textquoteright} Eindelijk opende,.}{ik de deur der kamer waar}{Nora zich bevond}\\

\haiku{Ik bevond mij aan,.}{eene spoorwegstatie ergens}{in Zuid-Duitschland}\\

\haiku{{\textquoteright} Gedurende een;}{paar seconden heerscht er}{stilte in den stal}\\

\subsection{Uit: Werken. Deel 40. Fata morgana. Deel 2}

\haiku{Nu ook verdwijnt de;}{kalme uitdrukking op het}{wezen des grijsaards}\\

\haiku{Met afkeer stoot hij;}{het dagblad ter zijde dat}{die dagteekening meldt}\\

\haiku{zij ziet met haren.}{kalmen blik den grijsaard vlak}{in het aangezicht}\\

\haiku{De vader nadert,:}{haar neemt zacht hare hand in}{de zijne en zegt}\\

\haiku{Boven dat kleedsel;}{staat een vinnige kop met}{kortgeknipt zwart haar}\\

\haiku{hier leeft men in het.}{lieflijk Zuiden en in eene}{gebalsemde lucht}\\

\haiku{dan bezoekt hij eenen,;}{oud collega bij wien hij}{echter zelden komt}\\

\haiku{Deze is hij nooit;}{met een godsdienstig gevoel}{binnengetreden}\\

\haiku{De oude koopman,;}{durft niet meer hooger durft niet}{meer naar beneden}\\

\haiku{Mijne moeder gaf.}{het leven aan een kind en}{stierf in het gasthuis}\\

\haiku{want hij gevoelt dat.}{hij zijn leed met een tweede}{wezen dragen zal}\\

\haiku{Wanneer zal er nu?}{in den aangeduiden zin}{recht worden gedaan}\\

\haiku{De rijke koopman, {\textquoteleft}{\textquoteright};}{leeft nog eenige maandenmet}{den worm in het hart}\\

\haiku{De linde- en;}{kastanjeboomen zijn met}{dicht loof omhangen}\\

\haiku{{\textquoteleft}Hebt ge gezien, baas,?}{wat schoone lamp bij Spronk op}{de huistafel brandt}\\

\haiku{dat hij Tegel den '.}{doorn der jaloezie int}{hart heeft gestoken}\\

\haiku{{\textquoteright} zegt het meisje dat,;}{zich eenigszins voorover buigt om}{zich te doen verstaan}\\

\haiku{{\textquoteleft}Huu....uu....{\textquoteright} {\textquoteleft}Gij zijt er,{\textquoteright}, {\textquoteleft}?}{zegt de oude Baldof rijdt}{ge mee naar het dorp}\\

\haiku{Eensklaps staat hij stil,:}{met den voet op het harde}{pad stampend zegt hij}\\

\haiku{{\textquoteleft}Ze brengen Helm in {\textquoteleft}{\textquoteright},.}{triomf naarden Hanekam}{zegt ze welgemoed}\\

\haiku{daar krast en snijdt de!}{schaats over het ijs en laat een}{sneeuwig spoor achter}\\

\haiku{t is de gloed van,.}{het vuur die door het venster}{van de hoeve schijnt}\\

\haiku{Zijn aangezicht is -;}{doodsbleek doodsbleek in den rooden}{lichtglans der lantaarns}\\

\haiku{Wat hebt ge met Helm,?}{uitstaan die sedert lang}{u den rug toekeert}\\

\haiku{zij is altijd voor;}{de Spronk's toegevend en}{week als een meelzak}\\

\haiku{Hier en daar, langs de,,;}{baan staan eenige berken met}{kalkwitte schors}\\

\haiku{Xander zet zich op,:}{eene voetstoof naast Berga en}{als deze hem zegt}\\

\haiku{doch deze is te.}{diep geslagen om nog te}{kunnen genezen}\\

\haiku{een vijfde was op, -!}{het dwaalspoor en dat viel den}{vader hard zeer hard}\\

\haiku{kortom, ging aan kaai {\textquoteleft}{\textquoteright}.}{en dok door als iemand met}{een gezonden kop}\\

\haiku{Centen en kroeg zijn.}{een machtig lokaas voor den}{wereldhervormer}\\

\subsection{Uit: Werken. Deel 25. De fortuinzoekers}

\haiku{zullen wij, als de,:}{Geest des goeds den inboorling}{toe fluisteren}\\

\haiku{O beuk, gij zijt de,:}{vriend des grijsaards Die somtijds}{aan uw wortel rust}\\

\haiku{{\textquoteright} morde Huibert, nog;}{eenige schreden van den boom}{verwijderd zijnde}\\

\haiku{Brengen uwe akkers,?}{niet genoeg op als God u}{duizend voor een geeft}\\

\haiku{Evert Krans was, in het,:}{huiselijke leven door}{den dichter gevormd}\\

\haiku{Een ei dat nog niet,. ....}{is geleyd Daervan en}{dient niet veel gezeyt}\\

\haiku{{\textquoteright} {\textquoteleft}De dooden zijn dood, en!}{de levenden maken mij}{het leven bitter}\\

\haiku{{\textquoteright} De grijsaard had die.}{woorden met eene ontroerde}{stem uitgesproken}\\

\haiku{Gij hebt eene reis van.}{verscheidene weken over}{zee af te leggen}\\

\haiku{Ik zou mijne vrouw....{\textquoteright} {\textquoteleft},?}{niet geven voorJa hoe hoog}{schat gij uwe vrouw wel}\\

\haiku{Aan Heva's hand over,.}{den zandweg gaande kwam hij}{langs het huis van Evert}\\

\haiku{Werken kan ik niet,.}{meer en ik ben te oud om}{te leeren bedelen}\\

\haiku{Denk in uwen voorspoed,,.}{aan hen die in een vreemd land}{een graf gaan zoeken}\\

\haiku{{\textquoteleft}Waarom dreeft gij aan,,!}{den beukenboom uw paard zoo}{eensklaps voort Huibert}\\

\haiku{{\textquoteright} riep de Rosse, en.}{legde opnieuw de zweep over}{het snuivende paard}\\

\haiku{wat vroeger of wat,.}{later afscheid genomen}{dat is hetzelfde}\\

\haiku{hij stond met Huibert,.}{Monica en Heva op}{de kaai der Schelde}\\

\haiku{Was zijn uiterlijk,;}{gunstig ook zijn innerlijk}{strekte hem tot eer}\\

\haiku{Wat Evert betreft, hij.}{zweeg en liet de toekomst aan}{den goeden God over}\\

\haiku{Van waar kwam zij, die,?}{arme Heva en hoe was}{zij daar gekomen}\\

\haiku{Gij zijt zoo bleek, gij,.}{ziet mij zoo wonderlijk aan}{gij spreekt zoo somber}\\

\subsection{Uit: Werken. Deel 42. In 't vervallen huis}

\haiku{het publiek, zooals ik,.}{zeide bestond uit zes of}{zeven personen}\\

\haiku{vriendschap, liefde, recht,!}{en zelfs gansch het menschdom}{grootsch en edel voor ons}\\

\haiku{De verveling was;}{immers ook vroeger uit ons}{midden gebannen}\\

\haiku{doch bij de minste.}{tegenkanting nam hij zijn}{toevlucht tot den drank}\\

\haiku{maar hij is het die.}{uw levensgeluk en het}{mijne vernield heefd}\\

\haiku{Rechts van de knapen,;}{zag hij den witten schoorsteen}{van zijns moeders huis}\\

\haiku{alleen Willem was.}{nog niet in het moederlijk}{huis teruggekeerd}\\

\haiku{Ook nu we\^er kwam zij:}{haren toevlucht zoeken in}{het huis des Heeren}\\

\haiku{We gingen samen.}{te Rotterdam aan boord en}{verder naar de Oost}\\

\haiku{gij gevoelt wat ik.}{lijden moet om het verlies}{van mijn eenigen zoon}\\

\haiku{Wat volgde er een!}{onrustige nacht op dien}{avontuurvollen avond}\\

\haiku{Zijn aangezicht was,;}{akelig verwrongen zijn oog}{blonk wilder dan ooit}\\

\haiku{{\textquoteright} sprak Willem, {\textquoteleft}laat ons.}{ter kerke gaan en God voor}{zijne ziel bidden}\\

\haiku{gij blijft hier in uwen,?}{leuningstoel liggen en de}{vrienden wachten u}\\

\haiku{Die lange witte!}{kerel steekt den draak met mij}{en mijn manuscript}\\

\haiku{mijn neus moet prachtig,.}{gepurperd mijne wangen}{moeten loodblauw zijn}\\

\haiku{Men ziet het wel, Van:}{Velden was een man van den}{ouden stempel}\\

\haiku{{\textquoteright} {\textquoteleft}Zwijg gij ook, vrouw, die.}{blinder zijt dan een mol in}{uwe moederliefde}\\

\haiku{Als een bliksem vloog.}{Georges de deur uit en}{ijlde de straat op}\\

\haiku{{\textquoteright} {\textquoteleft}O, o, wij komen!}{niet in de hoedanigheid}{onzer ambachten}\\

\haiku{wat ik loopen kan,,.}{de kamer rond de trappen}{af de trappen op}\\

\haiku{Liet nu de kachel?}{een gebrom van af- of}{goedkeuring hooren}\\

\haiku{het helsch bataillon,.}{liep er eene rilling over al}{de aanwezigen}\\

\haiku{Welnu, zoodanig.}{zaten de groote kinderen}{in Het Molentje}\\

\haiku{Het gekrijt van het.}{kind maakte den vader een}{oogenblik wakker}\\

\haiku{den jongen scharesliep.}{in het oud wammes van zijn}{vader gewikkeld}\\

\haiku{dat is, men stak hem,.}{in eene enge ruimte in}{den wagen zelven}\\

\haiku{{\textquoteleft}Zie, nu heb ik spijt,!}{dat ik al uw pruimen niet}{heb opgegeten}\\

\haiku{Waarom zet gij uwen?}{korf niet ne\^er en komt gij ook}{niet in het water}\\

\haiku{Maar als het in de -!}{week spinning op den Hooiberg}{is duivekaters}\\

\haiku{En dan de tijger,!}{die met een paard in den muil}{over eenen draaiboom springt}\\

\haiku{Het manneken staat,.}{overeind met vlammend oog en}{uitgestrekten arm}\\

\haiku{en gij ook, blaadjes,,!}{die daar boven soms kwettert}{fluistert en giechelt}\\

\haiku{Hij slentert zelfs - maar;}{ook de kruipende slak komt}{waar zij wezen moet}\\

\haiku{zijne gedachten -.}{gaan verder heel verre van}{al die booze menschen}\\

\haiku{In de huizen rilt.}{en bibbert men van schrik bij}{het zien der ruiters}\\

\haiku{betalen we nu,:}{zware lasten laat ons op}{de toekomst hopen}\\

\haiku{in Gent ieverden,,;}{Willems Snellaert van Duyse}{en Ledeganck}\\

\haiku{Al wat wezenlijk,.}{Vlaamschgezind was liet zich}{op dat feest vinden}\\

\haiku{Tweemaal poogde hij,,.}{de volkspers te stichten dat}{is de weekbladpers}\\

\haiku{Het eerste nummer.}{van dit weekblad was vooral}{het werk van Gerrits}\\

\subsection{Uit: Werken. Deel 36. Het Jan-Klaassen-spel}

\haiku{{\textquoteright} zegt de stoeldraaier.}{met eene niet meer bedwongen}{verontwaardiging}\\

\haiku{ik zeg dat hij een, -}{slecht hart heeft en zie ik zou}{hem eens willen doen}\\

\haiku{Laat ons den baron.}{van Dormael en Max Franck}{nader leeren kennen}\\

\haiku{{\textquoteright} En allen bersten.}{bij die godslastering in}{schaterlachen uit}\\

\haiku{Ik wil niets van hem, '....}{weten maart is enkel}{uit eene aardigheid}\\

\haiku{{\textquoteright} {\textquoteleft}Nu ben ik toch wel,{\textquoteright}, {\textquoteleft}.}{eens nieuwgierig dacht Janof}{ze we\^er komen zal}\\

\haiku{{\textquoteleft}En eene kan goede {\textquoteleft},{\textquoteright}.}{seef.{\textquoteright}3Die schuimt als Champagne}{antwoordt het meisje}\\

\haiku{{\textquoteleft}En waarom ziet gij,?}{er uit alsof ge naar eene}{begrafenis gingt}\\

\haiku{Zijne haren zijn,;}{eenigszins verward zijn gezicht}{rood opgeblazen}\\

\haiku{deze met opzet,.}{gene gedwongen door de}{omstandigheden}\\

\haiku{Ik beken, van die;}{overdrijving ben ik eenigszins}{teruggekomen}\\

\haiku{Ziedaar eene liefde,.}{zoo als men er in onzen}{tijd geene meer aantreft}\\

\haiku{Ik zal hem zien, hem,!}{nogmaals spreken misschien zal}{hij mij beminnen}\\

\haiku{Mijnheer komt ge hier '?}{s avonds ingedrongen met}{slechte inzichten}\\

\haiku{Mijnheer Bareel rookt.}{den geurigen Varinas}{uit eene lange pijp}\\

\haiku{Tony Darenge,.}{wendt zich om want hij voelt zijn}{hoofd gloeiend worden}\\

\haiku{In zijn hart wenscht;}{hij nu eens een liedje te}{mogen afgeven}\\

\haiku{Dit belet niet dat.}{hij de fijne watersnep}{zeer smakelijk kraakt}\\

\haiku{{\textquoteleft}Ik zou uwen broeder{\textquoteright} - - {\textquoteleft}?}{Tony wordt bloedroodof is}{het uw broeder niet}\\

\haiku{Voorwaarts dus op den,.}{weg waar een dubbel profijt}{hem te wachten staat}\\

\haiku{hij gevoelt hoe ver,,.}{Mos de lijkbidder boven}{hem verheven staat}\\

\haiku{{\textquoteright} De toekomst schemert,;}{zoo schoon als de eerste glans}{van den dageraad}\\

\haiku{{\textquoteright} maar van Dobbelsteen,.}{voegt de daad bij de woorden}{en ijlt den trap op}\\

\haiku{{\textquoteright} - nu, zeg ik, is het.}{zeker dat de jonker zijn}{hoofd verloren heeft}\\

\haiku{met zoodanig man,,.}{mijnheer Max vrees ik den last}{der dankbaarheid niet}\\

\haiku{eerbied voor hare.}{ouders is geenszins eene van}{Marietta's deugden}\\

\haiku{het hoofd zakt dieper.}{naar de borst en hij voelt zijn}{oog vochtig worden}\\

\haiku{noodigde hem te eten,}{schonk hem een goed glas wijn en}{gaf zijner dochter}\\

\haiku{Er waren sinds die.}{operatie verscheidene}{weken verloopen}\\

\haiku{Gouden spreuk, welke.}{in onzen tijd maar al te}{veel vergeten wordt}\\

\subsection{Uit: Werken. Deel 19. Karakters en silhouetten}

\haiku{De neus is recht, de.}{neusvleugels zijn scherp geteekend}{en sterk beweeglijk}\\

\haiku{Maar eensklaps zaten,, {\textquoteleft}{\textquoteright}.}{er zooals hij zegdezwarte}{beesten in zijn hoofd}\\

\haiku{Dit kleine puntje!}{voltooit zoo voortreffelijk}{mijne silhouet}\\

\haiku{In zijn grijsblauw oog;}{lag duidelijk bewustheid}{van eigenwaarde}\\

\haiku{Een hoofdtrek in zijn,.}{karakter was misnoegdheid}{soms lichtgeraaktheid}\\

\haiku{in zijnen pennetwist,.}{waarin hij afbrak wat hij}{vroeger opbouwde}\\

\haiku{In zijnen stand sprak,;}{men geen Fransch of wat men z\'o\'o}{gelieft te noemen}\\

\haiku{Die gedurige.}{strijd is voor mij eene tweede}{natuur geworden}\\

\haiku{Een eerlijk kamper.}{voor mijne overtuiging ben}{ik altijd geweest}\\

\haiku{zij moeten dus niet.}{meer rood worden als zij soms}{nog aan mij denken}\\

\haiku{Aan voorstellen om,.}{hooger te vliegen heeft het}{mij niet ontbroken}\\

\haiku{maar{\textquoteright} - zooals de oude - {\textquoteleft}...}{dichter van onzen Reynaert}{zegdedoe beter}\\

\haiku{daarenboven, de.}{hulpbronnen ontbraken hem}{in zijne woonstad}\\

\haiku{kortom, of hij een,.}{onnadenkend kind of een}{booze geest is geweest}\\

\haiku{Mij had het gevaar;}{des Vaderlands diep ontroerd}{en verontwaardigd}\\

\haiku{Ziedaar eene vraag, die.}{moeilijk of liever niet kan}{worden beantwoord}\\

\haiku{In 1842, gaf de Laet:}{een zoogezegd historisch}{verhaal in het licht}\\

\haiku{Die vraag kunnen wij,:}{niet beantwoorden tenzij}{met eene tweede vraag}\\

\haiku{geene Vlamingen, geene -,!}{Walen meer broeders zonen}{van eenen vrijen grond}\\

\haiku{In denzelfden vorm,,.}{schreef de dichter in 1860 een}{vers Aan Pius IX}\\

\haiku{Voorwaar, hij is een,!}{dergenen wiens naam nooit zal}{worden uitgewischt}\\

\haiku{Die herstelling had;}{plaats zonder een druppeltje}{bloed te vergieten}\\

\haiku{, laat het ons zeggen,;}{in zijn nachthemd en zijne}{witte slaapmuts op}\\

\haiku{Dat stond gelijk aan,.}{een halven schelling zegt de}{dichter nijdig}\\

\haiku{Weliswaar kende;}{de vorst hem eene jaarwedde}{van 1800 gulden toe}\\

\haiku{Men moest slechts op de,!}{steenrots slaan om een milden}{stroom te doen vloeien}\\

\haiku{omdat zij, onder,;}{meer dan \'e\'en opzicht aan uw}{land is verwantschapt}\\

\haiku{* * * ~ Ter inleiding:}{van mijn onderwerp vang ik}{aan met te vragen}\\

\haiku{En leer niet op wat, '.}{staat te maken Alstgeen in}{eigen krachten is}\\

\haiku{hij ziet er weinig ' -;}{naar oft nationaal}{is of niet uitsteekt}\\

\haiku{Slavinne is de,.}{vrouw maar slavinne in den}{vollen zin des woords}\\

\haiku{Vlaamsche vrienden toe,:}{toen men nog huiverig was}{van het Noorden}\\

\subsection{Uit: Werken. Deel 27. Klokketonen. Deel 1}

\haiku{maar ik, het zij mij,.}{vergeven ik vooral had}{trotsche denkbeelden}\\

\haiku{ten minste een nieuw.}{leven werd uit dat schijnbaar}{doodskleed geboren}\\

\haiku{maar de arme vrouw,:}{trok hem angstig voort en ik}{hoorde haar zeggen}\\

\haiku{{\textquoteleft}Ik zoek en vind het.}{in hetgeen hier beneden}{tastbaar voor mij ligt}\\

\haiku{Een Opperwezen,;}{erkennen is het schepsel}{aan banden leggen}\\

\haiku{De mensch,{\textquoteright} hervatte, {\textquoteleft};}{ikvordert dagelijks in}{wijsheid en verstand}\\

\haiku{Nu hingen al de.}{kleedingstukken naast elkander}{aan een langen rek}\\

\haiku{Ieder liep om het, -:}{zeerst want men dacht zelfs de}{honden misschien wel}\\

\haiku{D\`at pijnigde den,;}{vader d\`at deed de moeder}{heimelijk weenen}\\

\haiku{zij was de tweede.}{moeder voor de kleinere}{zusters en broeders}\\

\haiku{Op een bepaald uur.}{van den dag komt hij altijd}{voorbij mijn venster}\\

\haiku{Eens dat gevonden,...,.}{zal men doch neen ik loop hier}{wat al te haastig}\\

\haiku{Mevrouw van Torlits,,.}{was zoo als wij zeiden eene}{statige dame}\\

\haiku{Heil en eere aan!}{de wakkere afdeeling}{van St Nicolaas}\\

\haiku{{\textquoteright} {\textquoteleft}Dan zal ik toch nooit,{\textquoteright}.}{iets afgebedeld hebben}{zeide Barend trotsch}\\

\haiku{Een legio meesters;}{had aan het bed van den graaf}{van Buren gestaan}\\

\haiku{de ondergaande.}{winterzon wierp een flauwen}{gloed over het sneeuwveld}\\

\haiku{Gij ziet, lezers, dat,,!}{ik wel zijn Mephistopheles doch}{in het goede ben}\\

\haiku{{\textquoteright} pocht de bezoeker,.}{op hoogen toon en ontsteekt eene}{nieuwe manilla}\\

\haiku{Vergeeft den aardkluit,!}{in den zomer den sneeuwbal}{in den winter}\\

\haiku{hij was hard gelijk.}{het hart van een rechter van}{den scherpen zwaarde}\\

\haiku{maar, geloof mij, ik,,.}{geachte confrater ik}{heb er nooit ontmoet}\\

\haiku{Ge ziet, ik heb in {\textquotedblleft}{\textquotedblright}.}{mijn leven ook zoo al wat}{Lavater gespeeld}\\

\haiku{en ze zag in 't,;}{rond eerst naar den hemel en}{eindelijk naar mij}\\

\haiku{Op zekeren dag -,,;}{mijn vleugels waren gescheurd}{gehavend ontkleurd}\\

\haiku{{\textquoteleft}Waarom zet men zoo'n!}{misdadigen paraplue}{niet in het tuchthuis}\\

\haiku{Gij hebt gelijk, zoo'n!}{creatuur verdient den naam}{van engel niet meer}\\

\subsection{Uit: Werken. Deel 28. Klokkentonen. Deel 2}

\haiku{En zouden dat, wat,?}{verder op dezelfde hooge}{canadassen zijn}\\

\haiku{Ik had lust om op;}{te springen en tegen het}{glas te trommelen}\\

\haiku{Ik hoorde in de:}{duisternis eene vroolijke}{stem mij toeroepen}\\

\haiku{t... ging, naar het mij,;}{toescheen in  de verre}{verte verloren}\\

\haiku{'t zijn misschien de,.}{laatste de allerlaatste}{die ik schrijven zal}\\

\haiku{Ik droomde niet meer,,.}{maar de stem heb ik zeker}{zeer zeker gehoord}\\

\haiku{Hoe kwam ik toch aan?}{de herinnering van dien}{lang gestorven vriend}\\

\haiku{Die ernstige en:}{denkende geest had echter}{eene zwakke zijde}\\

\haiku{Loop er maar eenigen.}{tijd mede en hij wordt ruim}{en gemakkelijk}\\

\haiku{Ik ben van adel{\textquotedblright}, {\textquotedblleft}Wees,{\textquotedblright}.}{edel belieg de deugden van}{uwen stamvader niet}\\

\haiku{zij wakkerde de,,.}{spelers ongelukkige}{landskinderen aan}\\

\haiku{hoe drommel, heeft men!}{mij nu toch in dat hol doen}{verloren loopen}\\

\haiku{Flora stopte nog:}{altijd kousen en zeide}{met een diepen zucht}\\

\haiku{Niet overal is Mr.,,.}{Burtel zoo spraakzaam als bij}{Griet neen gewis niet}\\

\haiku{'t geeft niets, ik zal!}{de openingsrede wel voor}{de vuist uitspreken}\\

\haiku{Doch nog altijd is:}{er een  zwarte wolk aan}{Mr. Burtel's hemel}\\

\haiku{Overigens de man:}{heeft eene voortreffelijke}{voorzorg genomen}\\

\haiku{Neen, gewis, hij wist}{zelfs niet dat hij een zoo door}{en door geleerd man}\\

\haiku{- vandaag is zij als,.}{eene kleine provinciestad}{zoo stil zoo rustig}\\

\haiku{Ik liet, niet zonder,.}{ontroering het oog op de}{arme vrouw rusten}\\

\haiku{Hoor, weet je wat, ik, ';}{volg mijn zink Wil ook de}{wijde wereld in}\\

\haiku{Een burger, van mijn, ';}{rang en stand Moet nu en dan}{eens buitent land}\\

\haiku{{\textquoteright} {\textquoteleft}Onder de vrouwen.}{is die ziekte grooter dan}{onder de mannen}\\

\haiku{De meesten komen.}{hier om hare dochters op}{de markt te brengen}\\

\haiku{Tra la la la, les,,.}{demoiselles Tra la la}{la se forment l\`a}\\

\haiku{in huis is als stil,,,,:}{stil als een graf want ik de}{jongste ben er niet}\\

\haiku{Van tijd tot tijd werpt.}{moeder een onrustigen}{blik op de huisklok}\\

\subsection{Uit: De kraaien zullen 't uitbrengen}

\haiku{Welnu, ik houd het}{er nog altijd voor dat dit}{meisje zoo schuldig}\\

\haiku{Zij heeft haast om de.}{stad te verlaten en in}{het vrije veld te zijn}\\

\haiku{De koetsier loert nog.}{altijd onder de kap der}{antieke sjees uit}\\

\haiku{En dan - 't is niet... '}{goed bij klaarlichten dag in}{het dorp te komen}\\

\haiku{maar ook tevens werpt.}{zij een oogslag op den kaal}{wordenden mantel}\\

\haiku{{\textquoteleft}Hier zult ge moeten,,.}{afstappen want mijn weg loopt}{links de uwe recht voort}\\

\haiku{zij ziet in  het.}{ronde op de steenen vloer en}{zoekt in hare kleeren}\\

\haiku{Als wraakgenot in,,,}{ons hart woelt bidt men niet en}{al zou men bidden}\\

\haiku{doch zij kan nog niet,,.}{opstaan al zou zij de plaats}{waar zij zich bevindt}\\

\haiku{- op eene kar met wat,.}{stroo naar de stad en naar de}{gevangenis bracht}\\

\haiku{Het loover houdt nog aan;}{de takken en vormt dus nog}{eene dichte gordijn}\\

\haiku{'t Is alsof ze,,...}{beiden in eenen droom voor zich}{heeft zien drijven}\\

\haiku{Tilla zegt geen woord ' -;}{vant geen er in de kerk}{geschied is geen woord}\\

\haiku{Gij hebt den aard naar,,.}{uw vader Tilla die was}{ook zoo loszinnig}\\

\haiku{Werken zal ik voor.}{u en voor mij en God zal}{het overige doen}\\

\haiku{maar ik zeg ook, dat.}{over dien diefstal het licht nog}{niet geschenen heeft}\\

\haiku{Ik geloof dat een.}{eed bij hem veel lichter weegt}{dan een zilverstuk}\\

\haiku{{\textquoteright} {\textquoteleft}Juist, want mijn vader,,:}{en dat was een scherpzinnig}{man zei altijd}\\

\haiku{Soms zou Barend den,}{molen willen doen spreken}{dien Tilla's vader}\\

\haiku{{\textquoteright} {\textquoteleft}Indien ik zeker,.}{wist dat zij onschuldig was}{ik ging er op af}\\

\haiku{toen ik in het dorp,.}{kwam wonen viel Tilla mij}{altijd in het oog}\\

\haiku{'t is de priester. '}{die het Brood des levens aan}{een stervende brengt}\\

\haiku{Nu zijn de akkers,,;}{kaal de weiden vaalgroen de}{grachten gezwollen}\\

\haiku{{\textquoteright} {\textquoteleft}Den weg op naar het,.}{dorp tot aan het binnenpad}{door het kreupelhout}\\

\haiku{{\textquoteright} {\textquoteleft}Och, uitkomen zal,!}{het toch al zouden het de}{kraaien uitbrengen}\\

\haiku{Ze zit reeds in de....}{donkere gevangenis}{onder den toren}\\

\haiku{{\textquoteright} Eensklaps verschrikt zij,,.... '}{want nu zij de oogen opheft}{staat Evert Dils voor haar}\\

\haiku{Heeft zij niet meer dan,,,?}{te veel alles zelfs eer en}{naam opgeofferd}\\

\haiku{{\textquoteright} {\textquoteleft}Gij zorgdet wel mij,}{niet te ontmoeten en gij}{hebt gelijk gehad}\\

\haiku{Het denkbeeld aan de.}{stad schiet voorbij en Evert keert}{tot het dorp terug}\\

\haiku{Niemand beweegt zich.}{bij zijn buurman en dat maakt}{hem nog woedender}\\

\haiku{Wat een dwaasheid, ook!}{zelfs maar een oogenblik aan}{een droom te gelooven}\\

\haiku{hij begrijpt iets van.}{het edel gevoel dat zijnen}{vriend handelen doet}\\

\haiku{Hare schoenen en;}{de randen des mantels zijn}{door de sneeuw bemorst}\\

\haiku{Tilla heeft de groep;}{verlaten v\'o\'or dat deze}{aan de huizen komt}\\

\haiku{De andere is -?}{een blonde gij herinnert}{het u immers nog}\\

\haiku{Gaat het hem aan, dat?}{deze of gene door eene}{ramp getroffen wordt}\\

\haiku{Gij hecht, mijn beste,,?}{Sommer aan familienaam}{aan familie-eer}\\

\haiku{{\textquoteright} {\textquoteleft}Ik beken dat die,....}{acten zeer rechtskundig zijn}{opgesteld maar toch}\\

\haiku{voor mij persoonlijk:}{ware het te wenschen dat}{de zaak zou doorgaan}\\

\haiku{{\textquoteright} {\textquoteleft}Dan heeft de oude...{\textquoteright} {\textquoteleft},...}{SommerAls gij gezegd hadt}{dan heeft mijn vader}\\

\haiku{{\textquoteleft}hij of een zijner.}{medeplichtigen heeft die}{acte vervalscht}\\

\haiku{Reeds denzelfden avond;}{meldt zich Sommer opnieuw bij}{den advokaat aan}\\

\haiku{{\textquoteright} gilt Tilla, {\textquoteleft}moeder, '.}{wij gaan weer int huis van}{den molen wonen}\\

\subsection{Uit: Werken. Deel 29. De landverrader}

\haiku{{\textquoteright} {\textquoteleft}En het worde u!}{zevenmaal zeven malen}{we\^ergegeven}\\

\haiku{De een stond op den,;}{weg die naar de woning zijns}{vaders geleidde}\\

\haiku{Het gebeurde van.}{dien avond had geen overtuiging}{in zijn hart gestort}\\

\haiku{Het was alsof hij;}{zijn brandend hart door den drank}{verkoelen wilde}\\

\haiku{hij zag de scherpe,;}{trekken niet die zich op haar}{gelaat afteekenden}\\

\haiku{Gij dreigt mij als ik;}{u spreek van eene der grootste}{machten der aarde}\\

\haiku{door vaderlandsche,....}{kreten waarvan een spotlach}{de we\^ergalm is}\\

\haiku{Daarna greep hij zoo,.}{dikwijls den wijnbeker dat}{zijn geest bedwelmde}\\

\haiku{een weinig balsem.}{op mijne gewonde ziel}{te voelen leggen}\\

\haiku{Op dit voorhoofd,{\textquoteright} sprak, {\textquoteleft}....}{Bertholdzal de blos der}{leugen niet stijgen}\\

\haiku{Dat immers was ook.}{de eenige stem die rechtstreeks}{tot zijne ziel sprak}\\

\haiku{Berthold had bij.}{het krassen der deur het hoofd}{niet  opgelicht}\\

\haiku{{\textquoteleft}Dat nooit mijn zoon zich.}{op den verrader wreke}{om mijnentwille}\\

\haiku{nog aan het hart van,!}{den armen grijsaard die geen}{eigen kind meer heeft}\\

\haiku{Daar nog klemde de;}{grijsaard het meisje aan het}{hart en mompelde}\\

\haiku{haar hart was zelden.}{of nooit in overeenstemming}{met hare lippen}\\

\haiku{Aan gene zijde,.}{des heuvels was eene plek die}{hij nog groeten moest}\\

\haiku{Ik heb op eens al.}{het verschrikkelijke van}{mijn toestand gezien}\\

\haiku{{\textquoteright} Zij zweeg, staarde op.}{het vloertapijt en scheen koud}{als een marmerbeeld}\\

\haiku{{\textquoteright} riep zij glimlachend.}{en dronk den beker tot den}{bodem toe ledig}\\

\haiku{doch alvorens te,....}{sterven kan ik haar den kop}{nog verpletteren}\\

\haiku{Zal d\`a\`ar het beeld van?}{de gemartelde Paula}{mij nog vervolgen}\\

\haiku{Zal d\`a\`ar het schrikbeeld?}{mijns vaders mij nog dreigend}{in den weg treden}\\

\subsection{Uit: Werken. Deel 46. Maria Stuart}

\haiku{duld echter dat ik,:}{ze u te binnen brenge}{en dat ik u zeg}\\

\haiku{Die leugen was noodig,:}{om het zich voorgestelde}{doel te bereiken}\\

\haiku{Zij zag gewis al,;}{de bannelingen die zij}{had uitgeplunderd}\\

\haiku{Diep was zij ontroerd, {\textquoteleft},,{\textquoteright}, {\textquoteleft}}{Staak uw kermen en weenen}{Melvil zeide zij}\\

\haiku{uwe meesteresse,....}{uwe koningin beveelt zich}{in uwe gebeden}\\

\haiku{{\textquoteright} Waarin was toch, in?}{die laatste omstandigheid}{den moed gelegen}\\

\haiku{Het vooroordeel moet.}{echter te dien tijde wel}{diep ingeworteld}\\

\haiku{De verklaringen,,;}{zijn op meer dan \'e\'en punt met}{elkander in strijd}\\

\haiku{Z\'o\'o eenzaam wonen,, '!}{zal men zeggent is om}{van te huiveren}\\

\haiku{een zoo onbepaald.}{vertrouwen heeft hij in de}{schietkunst van Donaat}\\

\haiku{De ketting, waaraan,.}{Vos vastligt ratelt over den}{dorpel van zijn hok}\\

\haiku{En daar zou, zegde,!}{de citoyen glimlachend}{geen haan naar kraaien}\\

\haiku{Gij hebt Donaat door,!}{uwe oproerige taal den}{kop op hol gebracht}\\

\haiku{Waarom is hij niet,!}{gisteren waarom over een}{uur niet weggegaan}\\

\haiku{Het meisje hoort niet;}{rechtstreeks in den boerenstand}{onzer Kempen thuis}\\

\haiku{Hij had zich willen;}{wreken op de feestkle\^eren}{der dorpelingen}\\

\haiku{Was de plechtigheid,?}{nog niet begonnen of was}{zij reeds voltrokken}\\

\haiku{Maar het doet mij zeer.}{aan het hart dat gij voor mij}{uw geluk vergeet}\\

\haiku{Gisteren waart gij,;}{een kind der weelde gij hadt}{al wat gij droomdet}\\

\haiku{Hij vernietigde,}{zeer behendig het werk van}{den ouden pastoor}\\

\haiku{met de koorts op het,}{lijf zong hij zijn lied en in}{den afgeloopen}\\

\haiku{De man zat juist in;}{den warmen leuningstoel voor}{het vuur te suffen}\\

\haiku{{\textquoteright} {\textquoteleft}Mijn zoon heeft het hart,,!}{op de rechte plaats zitten}{en is och arme}\\

\haiku{hij had de macht niet.}{meer zijne krukken door de}{sneeuw voort te slepen}\\

\haiku{de kinderstemmen,,:}{rein en helder zingen het}{lied van Tollens}\\

\haiku{{\textquoteright} {\textquoteleft}In wat betrekking?}{staat hij met het huisgezin}{des burgemeesters}\\

\haiku{{\textquoteright} Den volgenden dag}{vertelde Jasper Hompel}{mij gedeeltelijk}\\

\subsection{Uit: Werken. Deel 45. De nachtraven}

\haiku{Dat alles komt bij.}{de twee nachtraven echter}{niet in aanmerking}\\

\haiku{Zijne houding en.}{zijn gang kenmerken iets meer}{dan achteloosheid}\\

\haiku{De deftigen uit;}{zijnen stand vermijden hem}{zooveel mogelijk}\\

\haiku{Dat meisje, rijzig,:}{van gestalte kon zoo wat}{veertien jaar tellen}\\

\haiku{zoodra hij den -.}{voet in zijn eigen huis zet}{de waskaars ontsteekt}\\

\haiku{De bezoeker klopt,.}{want een bellentrekker houdt}{men er niet op na}\\

\haiku{Gorl is de type,;}{van een eerlijk gezond en}{verstandig werkman}\\

\haiku{'t is waar ook, een....}{jaar geleden ontmoette}{ik hem bij toeval}\\

\haiku{{\textquoteright} {\textquoteleft}Ja, maar dit wil niet,.}{zeggen dat ik haar gaarne}{zou willen kwijt zijn}\\

\haiku{deze speculeert,:}{evenmin op de diensten die}{het kind hem bewijst}\\

\haiku{{\textquoteright} {\textquoteleft}Neen, Gorl, zij moet er,.}{meer geweest zijn want zij danst}{als eene duivelin}\\

\haiku{De vereelte hand.}{des werkmans grijpt die van Murg}{en drukt ze dankbaar}\\

\haiku{Neen, zij heeft niets van,.}{den vader zij heeft misschien}{iets van de moeder}\\

\haiku{in een kristallen;}{vaas leunde een verlepte}{en verdorde bloem}\\

\haiku{gij hebt nog altijd.}{dezelfde bedoelingen}{als v\'o\'or eenigen tijd}\\

\haiku{De woorden van den:}{ouden heer kwamen Anna}{Eerling voor den geest}\\

\haiku{De voorspelling van.}{den ouden heer werd in haar}{meer en meer waarheid}\\

\haiku{De arme dwaze -;}{kreeg voor dons en hermelijn}{koude sneeuwvlokken}\\

\haiku{{\textquoteright} Bij het uitspreken.}{dezer woorden neemt zij het}{licht en gaat hem voor}\\

\haiku{{\textquoteleft}Hiertoe,{\textquoteright} zegt ze, {\textquoteleft}zou,;}{ik het recht hebben want uw}{naam is de mijne}\\

\haiku{{\textquoteleft}Toen ge laatst bij mij,{\textquoteright}, {\textquoteleft}}{waart hervat Gorlwilde ik}{er niet van hooren}\\

\haiku{Deze ook maakt zich,,;}{als bij ingeving gereed}{om te vertrekken}\\

\haiku{{\textquoteright} {\textquoteleft}Kom,{\textquoteright} zegt het meisje, {\textquoteleft},,.}{plotselingkom laat ons naar}{huis gaan grootvader}\\

\haiku{maar bij Nelia heeft.}{die stem zelfs nog niet in de}{verte geklonken}\\

\haiku{want zij doen haar aan.}{de gewijde kaars en den}{doodwagen denken}\\

\haiku{oprecht is hij in,;}{deze onderhandeling}{niet geweest slim wel}\\

\haiku{Nelia blikt door de;}{openstaande schuifdeur van den}{goederenwagen}\\

\haiku{{\textquoteright} {\textquoteleft}Ja wel, een bezoek,.}{dat ik u v\'o\'or jaar en dag}{had moeten brengen}\\

\haiku{Ik versta u niet...{\textquoteright}.}{en de schilder houdt altijd}{de hand aan het oor}\\

\haiku{De Goede Herder,;}{is echter geen gesticht voor}{verlorenen neen}\\

\haiku{De kinderen, die,;}{naar de kapel gaan komen}{kwetterend voorbij}\\

\haiku{{\textquoteright} en als hij wakker,.}{is en opschiet waant hij dat}{Nelia buiten staat}\\

\haiku{De tegenpartij,,;}{is als heer ofschoon eenigszins}{verwaarloosd gekleed}\\

\haiku{bij Murg brengt zij al.}{niet veel meer teweeg dan een}{schouderophalen}\\

\haiku{{\textquoteright} {\textquoteleft}Ik mediteerde,{\textquoteright},;}{zegt Murg met dien cynieken}{lach hem zoo eigen}\\

\haiku{{\textquoteright} luidt Murg's antwoord.}{en hij legt het oor tegen}{de deur en luistert}\\

\haiku{Kortom, Bronveld heeft;}{zijne slechte gewoonte}{niet vaarwel gezegd}\\

\haiku{met niemand heeft zij,;}{gesproken over de reis die}{ze gaat beginnen}\\

\haiku{Koortsachtig heft hij:}{plotseling de hand op ter}{hoogte van het hart}\\

\haiku{doch wie kon tot in?}{haar gemoed doordringen en}{haar beoordeelen}\\

\haiku{De klokketoon en.}{de beelden uit mijne jeugd}{hebben mij gered}\\

\haiku{ik herinner mij.}{toch hem een of twee malen}{gezien te hebben}\\

\haiku{{\textquoteright} {\textquoteleft}Gij zijt mij welkom,,{\textquoteright};}{nichtje liet er de goede}{dame op volgen}\\

\haiku{Ja, alles kwam het -.}{meisje voor als een droom een}{droom van den hemel}\\

\haiku{De machinist kent,:}{den binnentreder niet al}{zegt deze dan ook}\\

\haiku{Bronveld zit neer, knoopt.}{den overjas los en slaat den}{kraag naar beneden}\\

\haiku{Toen ik jongeling,,;}{was Gorl had ik lust om mij}{bezig te houden}\\

\haiku{Ik wilde de huur,...}{mijner woning opzeggen}{toen mijne dochter}\\

\haiku{haar wijzer loopt niet,;}{in wilde vaart heen en weer}{zooals de schaduwen}\\

\haiku{personen uit die,;}{wereld welke gewoon is}{grof te verteren}\\

\haiku{En dan die oude {\textquoteleft}{\textquoteright}...}{constschilder en die nog veel}{gekker apotheker}\\

\haiku{De zon verdwijnt in,.}{eene zee van ineensmeltend}{goud rood en purper}\\

\haiku{doch zij zal zich wel.}{wachten die bedreiging ten}{uitvoer te brengen}\\

\haiku{Rup wil naar buiten;}{zien en nagaan wat er zooal}{op den weg gebeurt}\\

\haiku{Nu Rup den ruiter,:}{nadert bevangt hem eensklaps}{eene zekere vrees}\\

\haiku{'t is of zij niet.}{de minste belangstelling}{voor Romald heeft}\\

\haiku{{\textquoteright} {\textquoteleft}In alle geval....}{kan dit aan de dochter niet}{verweten worden}\\

\haiku{Ik heb die neiging.}{tot afdaling in hem steeds}{moeten bevechten}\\

\haiku{doch het karakter;}{haars echtgenoots heeft zij nooit}{kunnen doorgronden}\\

\haiku{Mevrouw gaat langzaam,,;}{in gedachten verzonken}{door den bloementuin}\\

\haiku{{\textquoteleft}Romald, uwe moeder,,.}{verlangt evenals gij dat uw}{vader terugkeere}\\

\haiku{Mevrouw ontvangt de,;}{gasten in haar klein doch net}{gemeubeld salon}\\

\haiku{het mist echter den;}{waren toon om bij Nelia}{gehoor te vinden}\\

\haiku{{\textquoteright} {\textquoteleft}Ik ben, ik herhaal ',...}{t veeleer ongelukkig}{dan wel slecht geweest}\\

\haiku{Op eenige stappen:}{van het hekken staat hij stil}{en mompelt lachend}\\

\haiku{Een half uur later,;}{zit Murg in de herberg waar}{wij hem aantroffen}\\

\haiku{'t Is echter de,,.}{Bloemstraat niet die haar aantrekt}{maar wel grootvader}\\

\haiku{{\textquoteleft}Belooft ge mij de?}{geheimhouding van hetgeen}{ik ga meedeelen}\\

\haiku{Gij zijt,{\textquoteright} zoo spreekt hij, {\textquoteleft}.}{voortin uwe denkbeelden geen}{man van onzen tijd}\\

\haiku{{\textquoteright} {\textquoteleft}Indien ge mij van....}{tijd tot tijd iets over Romald}{wildet  schrijven}\\

\haiku{doch haar gemoed is,.}{op dit oogenblik week zij}{hoort haar hart kloppen}\\

\haiku{Eene eentonige,;}{weide levert niet veel spel}{op zal men zeggen}\\

\haiku{Dat onstaat om zoo.}{te zeggen onder den trap}{van zijne klompen}\\

\haiku{dat mijne moeder.}{mijnen vader vrijpleit ten}{nadeele van den uwen}\\

\haiku{'t Is of de stem:}{van Nelia andermaal aan}{Romald's oor suizelt}\\

\haiku{men lachte volop,,,?}{men mocht er cyniek zijn als}{een hond niet waar Murg}\\

\haiku{Drie of vier hoofden;}{verschijnen andermaal in}{de opening der deur}\\

\haiku{De familie van;}{Segelaer had het kasteel}{nog niet verlaten}\\

\haiku{Mevrouw valt jufvrouw,.}{Monica in de armen}{en beiden weenen}\\

\haiku{Op klompen voord eene, '.}{Excellentie verschijnen}{t is noodlottig}\\

\haiku{{\textquoteright} Een glimlach speelt over,.}{Bert's bleek wezen doch geen woord}{komt op de lippen}\\

\haiku{door eene andere,.}{stelling op te werpen het}{gesprek afleiden}\\

\haiku{Machteloos worstelt,.}{gij omdat het vergeten}{niet menschelijk is}\\

\haiku{{\textquoteright} De dokter neemt zacht.}{en vertrouwelijk hare}{hand in de zijne}\\

\subsection{Uit: Werken. Deel 26. Oranje in de Kempen}

\haiku{want daar had men een;}{paar knechten van ridder de}{Knuyt zien binnengaan}\\

\haiku{{\textquoteright} {\textquoteleft}Daarin geef ik je,}{volkomen gelijk en ik}{zal dien praatzieken}\\

\haiku{Hij herinnerde,.}{zich niet den sjacheraar ooit}{te hebben gezien}\\

\haiku{De jood moest even snel,;}{loopen als hij want de stem}{klonk immer even luid}\\

\haiku{{\textquoteleft}Nah, mijnheer Ralph, je.}{zoudt een oud man waarachtig}{den adem afnemen}\\

\haiku{dat weet menheer Ralph{\textquoteright} {\textquoteleft},,....}{het best.Nog eens verklaar u}{of voor den drommel}\\

\haiku{Wij zouden in den,}{donkeren avond en vooral}{niet op dien welken}\\

\haiku{Het vertrek, waarin,;}{wij de vrouw ontmoeten was}{goed gemeubeleerd}\\

\haiku{Eindelijk meende;}{zij gerucht te hooren aan}{de kleine straatpoort}\\

\haiku{Geduld dus, zet u,;}{ne\^er wees bedaard en vertel}{mij het gebeurde}\\

\haiku{niemand beter dan.}{hij kan den sluier van het}{geheim oplichten}\\

\haiku{{\textquoteright} Dat woord bracht een waas;}{van droefheid over het wezen}{der oude dame}\\

\haiku{{\textquoteright} riep Retha verschrikt,.}{en Elie drong zich dichter bij}{de oude dame}\\

\haiku{Ja, Retha, gij hebt;}{een trek der Midletown's}{in het aangezicht}\\

\haiku{hij trachtte rechts en.}{links inlichtingen over Eric}{Ralph te bekomen}\\

\haiku{{\textquoteleft}Kalverliefde,{\textquoteright} dacht, {\textquoteleft}!}{de kramerdie geen roode}{duit winst oplevert}\\

\haiku{{\textquoteright} {\textquoteleft}Ik ben zeker dat.}{gij mij de kleinste helft der}{som hebt gegeven}\\

\haiku{Wel man,  dan zal,!}{het nog wel noodig zijn dat ik}{u van kant helpe}\\

\haiku{Darvis, zeg toch aan {\textquotedblleft}{\textquotedblright},!}{menheer je hond dat hij me}{vleesch respicteert}\\

\haiku{{\textquoteleft}Mozes zelf zou zich,{\textquoteright}.}{aan de galg klappen als hij}{sprak hervatte hij}\\

\haiku{{\textquoteleft}En wanneer zal de?}{prinses van Oranje in de}{Vrijheid aankomen}\\

\haiku{{\textquoteright} onderbrak een klein,.}{en mager potbakkerken}{op nijdigen toon}\\

\haiku{Dat zal wel niet in,{\textquoteright}.}{deze omstandigheden}{meende de wever}\\

\haiku{{\textquoteleft}Darvis, zou dat de,?}{gevreesde ossenkooper}{met zijn rooden hond zijn}\\

\haiku{Het scheen wel, dat de.}{6 guldens 12 stuivers hem}{in den zak dansten}\\

\haiku{Het papier was geel;}{en blijkbaar een schutblad van}{een kerkboek geweest}\\

\haiku{hij sprong recht, alsof,.}{het de dood was geweest die}{bij hem aanklopte}\\

\haiku{{\textquoteleft}Als waon de galg,,.}{zen mot hekik blaose}{zoo heet thum gezeed}\\

\haiku{{\textquoteleft}Voorwaarts naar....{\textquoteright} en de.}{rest fluisterde hij in het}{oor van den vorster}\\

\haiku{Nu volgt de gilde,{\textquoteright}.}{van St.-Sebastiaan}{hervatte Quinten}\\

\haiku{De Jonggesellen;}{voerden er hunne schoonste}{muziekstukken uit}\\

\haiku{{\textquoteright} {\textquoteleft}Kom, kom, laat ons op,!}{zoo'n vroolijken dag niet al}{te streng worden Eric}\\

\haiku{{\textquoteleft}Kom,{\textquoteright} zeide Eric stil, {\textquoteleft}?....}{willen wij nog even door de}{warande loopen}\\

\haiku{{\textquoteright} De prinses had de:}{oude vrouw opgericht en}{zeide zeer ontroerd}\\

\haiku{Ik beschouwde ze,.}{als de mijne en noemde}{ze Elie en Retha}\\

\haiku{Ralph zette zich ne\^er.}{en wischte tranen uit zijn}{zoo mannelijk oog}\\

\haiku{{\textquoteleft}Laat ons daarover niet,,{\textquoteright}.}{spreken Harry zeide het}{meisje na een poos}\\

\haiku{doch in den loop des.}{avonds dacht zij meer dan eens aan}{den zoon des ballings}\\

\haiku{hij droeg een kleinen,.}{hoed met witte ve\^er en had}{hooge rijlaarzen aan}\\

\haiku{Darvis was gewis,,;}{niet daar zoo wel dan ware}{hij reeds verschenen}\\

\haiku{De lieve hemel!}{weet wat al rijkdommen daar}{verborgen zaten}\\

\haiku{Zonder twijfel wel,!}{de guldens die aan Mozes}{ontstolen waren}\\

\haiku{nog geruimen tijd.}{hoorde hij het geblaf en}{gebrul van den hond}\\

\subsection{Uit: Werken. Deel 13. De orgeldraaier}

\haiku{{\textquoteright} ging hij voort met een, {\textquoteleft}...{\textquoteright} {\textquoteleft},?}{diep ontroerde stemlaat mij}{Ha ik versta u}\\

\haiku{waarom plaag ik toch!}{mijne arme hersens met}{die gekke droomen}\\

\haiku{{\textquoteright} sprak hij plotseling, {\textquoteleft}.}{op zachter toon voortook dat}{zal ik verdragen}\\

\haiku{hij greep de hand der.}{jonkvrouw en bracht die bevend}{aan zijne lippen}\\

\haiku{Gij gelooft niet aan,!}{de verklaring dat ik uw}{vader ben kindlief}\\

\haiku{{\textquoteright} {\textquoteleft}Ik ook, ik dank den,.}{hemel dat hij mij nog eens}{in uw bijzijn brengt}\\

\haiku{Het is niet alleen,;}{het ongeluk der scheiding}{dat mij treft Willem}\\

\haiku{laat de schemering;}{in het straks ontgloeiende}{Oosten opdagen}\\

\haiku{De beschuldigde,,:}{die als neergedrukt zit is}{u niet onbekend}\\

\haiku{De beschuldigde.}{zag den rechter strak en met}{ontsteltenis aan}\\

\haiku{Ik beminde het,,.}{gedruisch der vrienden de}{feesten het gewoel}\\

\haiku{Neen, het was in geen,.}{eerlijk tweegevecht dat de}{misdaad gepleegd werd}\\

\haiku{Droeg ik de wroeging,?....}{niet met mij mee drukte ik}{haar niet aan het hart}\\

\haiku{Tienmaal heb ik dat,,;}{kermend wicht in mijne vlucht}{achtergelaten}\\

\haiku{{\textquoteleft}'t Is waar, het is,{\textquoteright}:}{uw kind zegde er eene in}{het hart van Willem}\\

\haiku{Voor een paar dagen;}{nog dacht ik alle hoop te}{moeten opgeven}\\

\haiku{zelfs eene fortuin die.}{groot en een wapenschild dat}{onbezoedeld is}\\

\haiku{Heb ik ongelijk,,?}{lezer u zoo een geluk}{toe te wenschen}\\

\subsection{Uit: Werken. Deel 44. De verstooteling. Stomme Nora. Kreupele Dorus}

\haiku{Tafereelen uit,.}{onzen Tijd I.   De Dood}{van Piko-Poko}\\

\haiku{Piko-Poko, zoo,.}{heette de aap was ook zijn}{leven en bestaan}\\

\haiku{hij sprak echter geen.}{woord en richtte het gelaat}{weer naar zijn dooden vriend}\\

\haiku{{\textquoteleft}maar ik herinner.}{mij flauw dat mijne moeder}{eene kaartlegster was}\\

\haiku{{\textquoteright} was het antwoord, en.}{er gleed een lichten glimlach}{om de bleeke lippen}\\

\haiku{Ik denk altijd dat;}{niemand zoo veel recht heeft om}{te weenen als ik}\\

\haiku{{\textquoteleft}Ik werd zinneloos,.}{van smart en lijden maar God}{was goed jegens mij}\\

\haiku{men heeft mij dat slechts.}{gezegd om mijne diepe}{smart te stillen}\\

\haiku{Met geopende:}{armen liep ik binnen en}{riep in verrukking}\\

\haiku{Gij verstaat dat woord,,.}{niet gij die nooit de vrijheid}{hebt moeten missen}\\

\haiku{doch hoe kaal de jas,;}{ook zij dien hij tot aan den}{hals heeft opgeknoopt}\\

\haiku{pas op, raak geen pijl, -!}{van zijn hoofdhaar aan of het}{zal u berouwen}\\

\haiku{Zou die jongeling,,?}{zoo dacht hij misschien aan de}{policie denken}\\

\haiku{{\textquoteright} {\textquoteleft}Ja, daar dichtbij is,}{Piko-Poko gestorven}{en ook niet ver van}\\

\haiku{met den oppersten.}{meester van het zwervende}{Bohemer-volk}\\

\haiku{Daar lag ik op de,}{straat te spartelen naakt als}{een afgesleten}\\

\haiku{- maar wat maakt ons die,.}{Wij wij kussen den kroes en}{zijn zeker van vreugd}\\

\haiku{hij gezeten had,.}{en wilde er gewis een}{wapen van maken}\\

\haiku{Kristiaan wilde -!}{hen bijstaan en beschermen}{de arme jongen}\\

\haiku{{\textquoteleft}Ga heen, gij hebt u!}{immers met onze zaken}{niet te bemoeien}\\

\haiku{Boven de tent van:}{Tamerlan leest men op een}{versleten doek}\\

\haiku{Hij sliep, en het was;}{of God hem in zijnen slaap}{een zoeten droom gaf}\\

\haiku{Vijftien jaar heb ik....}{nu reeds in armoede en}{lijden doorgesloofd}\\

\haiku{{\textquoteright} {\textquoteleft}Maar wie dan toch is?}{zoo onbarmhartig geweest}{mij aan te klagen}\\

\haiku{Hij vraagt lucht, leven,!}{en liefde en men versmoort}{den rampzalige}\\

\haiku{er lag eene blauwe.}{tint over zijne lippen en}{onder zijne oogen}\\

\haiku{{\textquoteleft}Wat voerde u op,?}{dit oogenklik in dit huis}{in deze kamer}\\

\haiku{Ik heb soms ademloos,.}{en met kloppend hart u van}{daar g\^ageslagen}\\

\haiku{Ik vraag u enkel,.}{laat dien ongelukkigen}{grijsaard in vrede}\\

\haiku{{\textquoteleft}Ik zal u aan mijn, {\textquoteleft},.}{harte kluistrenD\`a\`ar waar men}{nooit het scheiden vindt}\\

\haiku{{\textquoteleft}Wij zullen zien, of.}{zij ons het goed geluk zal}{kunnen voorspellen}\\

\haiku{want een licht purper.}{overtoog haar gelaat bij het}{hooren van dien naam}\\

\haiku{de andere had.}{wel eens willen pogen in}{onmacht te vallen}\\

\haiku{Men zegt, Zingolina,?}{dat gij in de sterren en}{in de kaarten leest}\\

\haiku{{\textquoteright} Er heerschte eene;}{gespannen verwachting in}{de vergadering}\\

\haiku{Agnes deed hem naar -!}{de keuken geleiden en}{d\`a\`ar o welk geluk}\\

\haiku{{\textquoteright} Het wijf, wier trekken,;}{niets goeds voorspelden scheen de}{vraag niet te verstaan}\\

\haiku{Hier rustte sedert!}{vijftien jaren het eenige}{dat gij achterliet}\\

\haiku{aan deze zijde,!}{staat uw vader aan gene}{zijde uw meester}\\

\haiku{Mahomet den geest,.}{gaf legde men den reus op}{eene lange tafel}\\

\haiku{ik zal, ik moet het,!}{geheim weten dat mijne}{geboorte bedekt}\\

\haiku{Met den laster en,;}{den vloek op de lippen stierf}{de schuldige man}\\

\haiku{zij had zoo gedwee}{als een lam ne\^ergeknield}{als het den kleinen}\\

\haiku{zij heeft niet alleen,}{in Kristiaan een vriend maar}{ook in den armen}\\

\haiku{dien gij nu we\^er in....}{de handen der policie}{hadt overgeleverd}\\

\haiku{{\textquoteleft}Ik geloof echter.}{niet dat gij mij eenig goed zoudt}{kunnen aanbrengen}\\

\haiku{Kristiaan gaf geen;}{acht op den gemoedstoestand}{van den rijken oom}\\

\haiku{Als Nora echter,}{den zoon van den molenaar}{den zwartlokkigen}\\

\haiku{Alles scheen stil en;}{akelig in zijn doodslaken}{te zitten droomen}\\

\haiku{Simon maakte zich.}{zekeren morgen gereed}{om te vertrekken}\\

\haiku{men iederen avond.}{voor de rust der ziele van}{de overledene}\\

\haiku{hij speurde den haas;}{en het schuchtere konijn}{op het sneeuwveld na}\\

\haiku{{\textquoteright} - en bij die woorden.}{glinsterde er een edel vuur}{in het oog des knaaps}\\

\haiku{{\textquoteleft}Maar denkt hij dan, dat?}{ik mij zoo maar straffeloos}{zou laten straffen}\\

\haiku{{\textquoteright} De dood kwam echter;}{meer dan eens op den rand der}{bedsponde zweven}\\

\haiku{Met den kogel dien,!}{gij kreupelen Dorus eens in}{het been schoot vader}\\

\subsection{Uit: Werken. Deel 38. Villa Pladelle}

\haiku{Nu heb ik haast om,.}{op het slot te komen waar}{het avondmaal mij wacht}\\

\haiku{Welnu, gij zult Hark,....}{tot leenman verheffen hem}{rijk begiftigen}\\

\haiku{Kloek en stout is hij,.}{als een echte zoon uit het}{grimmige noorden}\\

\haiku{Na eene poos in het,.}{vertrek vertoefd te hebben}{ging Karl naar buiten}\\

\haiku{Om zijnen hals hing.}{eene zware gouden ketting}{met breede schakels}\\

\haiku{aan het koningschap,,.}{zoo dacht de heerschzuchtige}{zouden bewijzen}\\

\haiku{{\textquoteleft}Hoe,{\textquoteright} zegde deze, {\textquoteleft},,?}{gij hier vrouw Ragna en}{waarom niet ginder}\\

\haiku{{\textquoteright} zegde Dirk bleek en, {\textquoteleft}?}{ontsteldzijt ge wel zeker}{van hetgeen ge zegt}\\

\haiku{{\textquoteright} {\textquoteleft}Ja,{\textquoteright} antwoordde Karl, {\textquoteleft},,,.}{en wat Dirk mijn getrouwe}{Manne zegt doet hij}\\

\haiku{{\textquoteleft}Wat verlangt ge van,?}{Hark van den heer en meester}{op Grimma herna}\\

\haiku{want de gevlochten,,.}{horde die voor deur diende}{was andermaal toe}\\

\haiku{Die naam tooverde eene.}{glinstering van hoop op de}{lippen van Simplex}\\

\haiku{{\textquoteright} De breede borst van.}{den pelsendief ging als een}{blaasbalg op en neer}\\

\haiku{De nieuwsgierigen.}{bleven op eenen afstand en}{volgden schoorvoetend}\\

\haiku{{\textquoteright} zegde de moeder,.}{zacht den arm op dien haars zoons}{latende rusten}\\

\haiku{Al de omstanders.}{waren diep bewogen en}{velen zelfs weenden}\\

\haiku{Die rijke vrouw zou,,.}{zich wreken zooals zij deed zooals}{zij reeds gedaan had}\\

\haiku{Een uur later ving.}{de grafelijke stoet de}{terugreize aan}\\

\haiku{41Vrijen droegen,.}{zilveren lijfeigenen}{koperen ringen}\\

\subsection{Uit: Werken. Deel 33. De voetbranders}

\haiku{Voor wapens hadden,.}{ze verroeste geweren}{pieken en lansen}\\

\haiku{Op eenigen afstand,;}{voor mij uit zag ik twee of}{drie groote wachtvuren}\\

\haiku{Op die hoogte zag,,.}{ik als in zegepraal over}{gansch het leger heen}\\

\haiku{In een oogenblik;}{vielen de woestaards in de}{eenzame woning}\\

\haiku{later kwam het mij.}{in al zijne schoonheid en}{reinheid voor den geest}\\

\haiku{Vier mannen droegen,.}{blootshoofds eene doodkist met een}{zwart baarkleed bedekt}\\

\haiku{Ik voelde dien avond:}{iets wat ik nooit voor mijne}{moeder gevoeld had}\\

\haiku{de soldaten en,.}{vergiel-jotineeren zooals de}{republikanen doen}\\

\haiku{We gingen op de;}{puinen en voelden of de}{steenen nog warm waren}\\

\haiku{Op den avond dat de,;}{Franschen in het dorp vielen}{keerde hij terug}\\

\haiku{{\textquoteright} stookte hij op, even.}{als men twee honden tegen}{elkander opjaagt}\\

\haiku{{\textquoteright} riep ik plotseling.}{opspringende en heel de}{bende sprong overeind}\\

\haiku{Abel, hield hem bij zich,.}{en zegde daarna dat Abel}{niet meer mocht weggaan}\\

\haiku{Hij was ernstig en.}{zijn aangezicht was bleek als}{dat van eenen doode}\\

\haiku{Maar waarom steken?...}{deze ook hun hoofd boven}{de anderen uit}\\

\haiku{Gij zijt tevreden,,?}{van de Franschen ontslagen}{te zijn niet waar Niels}\\

\haiku{die zijn vertrokken,.}{nadat ze het halve dorp}{hebben afgebrand}\\

\haiku{Maar Jan wist op slot.}{van rekening veel meer te}{vertellen dan ik}\\

\haiku{Jan, die gaarne de {\textquoteleft}}{verzekering gaf vanzoo}{zeker als tweemaal}\\

\haiku{Dat was het gevolg,,.}{beweerde men van het leenen}{op woekerintrest}\\

\haiku{- de andere was,;}{de lange  zwikzwak maar}{die was niet gekwetst}\\

\haiku{Ik zag onwillens,:}{naar den langen neus van den}{zwikzwak en dacht zoo}\\

\haiku{De kring, rondom het,;}{vuur stoof uit elkander als}{een bende musschen}\\

\haiku{Op den bok zat een,;}{gewonde wiens hoofd door eenen}{doek omzwachteld was}\\

\haiku{De arme duivel,,;}{smeekte wellicht om hulp om}{troost om lafenis}\\

\haiku{de molenaar sneed,,.}{met een groot mes brokken brood}{in den haverbak}\\

\haiku{De wagenmaker.}{hamerde zoo wat aan den}{krakenden wagen}\\

\haiku{maar alleen de bleeke.}{gekwetste wilde zijnen}{zegen ontvangen}\\

\haiku{oud en jong rees op '.}{ent smeeken der vrouwen}{mocht niet meer baten}\\

\haiku{Op een bepaald punt.}{in het dorp zullen wij ons}{weer vereenigen}\\

\haiku{en snelde daarna.}{voort om aan Roodwammes het groote}{nieuws me\^e te deelen}\\

\haiku{De zon brak met eenen.}{stroom van purper en goud door}{de grauwe wolken}\\

\haiku{De nadering van,.}{een dreigend gevaar is een}{wonderlijk gevoel}\\

\haiku{Slechts eene voorwacht had;}{een oogenblik aan de brug}{derf aanval beproefd}\\

\haiku{want deze lichtte:}{de rechterhand op en ik}{hoorde hem zeggen}\\

\haiku{zijne uniform hing,,.}{aan flarden was vuil besmeurd}{en onkennelijk}\\

\haiku{Soms hief ik het oog.}{op en ontmoette dan zijn}{toegenegen blik}\\

\haiku{het was, of wij voor.}{hem defileerden en hij}{ons kommandeerde}\\

\haiku{Na eene poos vertoefd,;}{te hebben kropen de twee}{plunderaars verder}\\

\haiku{Ik wendde nog eens, {\textquoteleft}{\textquoteright};}{het hoofd om groette met een}{luidHoerrah Roodwammes}\\

\haiku{Wat was het feest op!}{het buitengoed en in het}{naburige dorp}\\

\haiku{De voetbranders    .}{Noten 1De gemalin van}{den prins van Oranje}\\

\subsection{Uit: Werken. Deel 6. Waar is de vader?}

\haiku{Hij weet het wel, doch.}{hij heeft zich daarover tot nu}{toe niet bekommerd}\\

\haiku{maar het juiste woord.}{van dat alles is bij het}{publiek niet bekend}\\

\haiku{Als vorm mocht daar soms,;}{wel iets op af te wijzen}{vallen  gewis}\\

\haiku{{\textquoteleft}maar het verlangen,,,.}{en dit verlangen moet u}{denk ik heilig zijn}\\

\haiku{{\textquoteright} {\textquoteleft}Ik dank u, mijnheer,{\textquoteright}.}{Albert en mevrouw stak den}{jongman de hand toe}\\

\haiku{{\textquoteright} vroeg zij en dit niet.}{zonder eene trilling in de}{stem te verraden}\\

\haiku{daarenboven, de;}{omstandigheden zijn soms}{onverbiddelijk}\\

\haiku{Na een tweeden ruk;}{aan de bel deed zich binnen}{eenig gedruisch op}\\

\haiku{tegenover haar, aan,:}{den muur hing het portret van}{den overledene}\\

\haiku{Hij stond op en bleef.}{in gedachten verzonken}{voor het venster staan}\\

\haiku{En echter, als gij,,.}{hard trekt zoudt gij haar zoudt gij}{Nora doen schrikken}\\

\haiku{Och ja,{\textquoteright} hervatte,;}{Van Leefdael vroolijk ik kwam}{juist niet erg van pas}\\

\haiku{Het {\textquoteleft}neen{\textquoteright} van Mijnheer,,;}{Albert was naar zijn inzien}{onvermijdelijk}\\

\haiku{Zij beminde dien.}{jongen man uit al de kracht}{harer reine ziel}\\

\haiku{{\textquoteright} Mijnheer Albert wierp;}{een blik van meewarigheid}{op den ouden heer}\\

\haiku{Uwe rechterhand, daar,.....}{in uwen jas gestoken rust}{op ets zeer kostbaar}\\

\haiku{doch de honderd pond.}{intrest van het loopende}{jaar waren verteerd}\\

\haiku{{\textquoteleft}Ik zal toch overal.}{zeggen dat ik het kind van}{den poppenman ben}\\

\haiku{Mijnheer Van Velthem.}{was misschien nog meer ontroerd}{dan de poppenman}\\

\haiku{ik schik mij in mijn....}{lot en de goede God zal}{het overige doen}\\

\haiku{{\textquoteleft}En ik, mijnheer Van,.}{Velthem ik geef haar u. Wees}{beiden gelukkig}\\

\haiku{De oude heer stond,.}{op hij had groote moeite om}{zich goed te houden}\\

\subsection{Uit: Werken. Deel 22. Zoo werd hij rijk}

\haiku{op de helling des;}{heuvels krast de zaag door den}{ouden eikenstam}\\

\haiku{ofwel hij maait er,;}{het graan  zaait de voor-}{of najaarsvruchten}\\

\haiku{Als Hiob terugkeert,;}{zal hij wel is waar Nard en}{Narda niet vinden}\\

\haiku{Nu was de Nachtuil;}{in eenige dagen niet in}{het dal gekomen}\\

\haiku{Zoo gewapend - met, -.}{stok pistolen en mes klom Hiob}{langzaam naar boven}\\

\haiku{Toen de blauwe damp,;}{optrok was gansch de bende}{wolven verdwenen}\\

\haiku{nogmaals klopte hij.}{en legde daarna het oor}{tegen het houtwerk}\\

\haiku{- als Jean Hibou op,.}{strooptocht ging verliet hij het}{huis niet langs de deur}\\

\haiku{Het kind knikte met.}{het oog en frazelde iets}{dat niet verstaan werd}\\

\haiku{{\textquoteleft}En zijn er meer zoon?}{wagens en tenten op eene}{kermis in de stad}\\

\haiku{Bij Narda, wekte;}{dit gezicht niet de minste}{bewondering op}\\

\haiku{De mulder begreep;}{nu alles wat er op de}{kermis gebeurd was}\\

\haiku{doch ik wilde u.}{echter opmerken dat het}{vandaag Zondag is}\\

\haiku{de jongen vindt het,.}{zoo wonderlijk in dat kleed}{ter kerke te gaan}\\

\haiku{{\textquoteleft}Neen, manneke, ge!}{zult niet weten met wat troef}{ik u nog afwacht}\\

\haiku{{\textquoteright} {\textquoteleft}Maar ik die hem ken,.}{ik heb het recht dat oordeel}{over hem te vellen}\\

\haiku{{\textquoteleft}Indien hij niet over,?}{mij sprak dan toch gewis sprak}{hij over de abdij}\\

\haiku{Maar dan zal het de,?}{booze zijn die zijnen staart of}{de wagens voortsleept}\\

\haiku{{\textquoteright} {\textquoteleft}Toch zou ik al dat,.}{schoons al dat zilver en goud}{willen zien blinken}\\

\haiku{{\textquoteright} {\textquoteleft}Kom, Nard, geef mij uwe...,{\textquoteright}.}{hand terug en hand in hand}{gaan beiden weer voort}\\

\haiku{{\textquoteright} vraagt Yolande op,.}{eenen toon die als zilverklank}{in de ooren galmt}\\

\haiku{{\textquoteright} Die naam verwekt eene.}{zekere verwondering}{bij den ouden man}\\

\haiku{indien zijn zoon eens? '.}{d\`a\`ar waret Zou niet te}{verwonderen zijn}\\

\haiku{Razend van gramschap;}{ijlt mijnheer Delmon de la}{Carde de straat op}\\

\haiku{Zij was zooeven d\`a\`ar,,.}{aan het hekken en moet hier}{zijn binnengevlucht}\\

\haiku{Toen ik kind was had....}{de kerk voor mij inderdaad}{aantrekkelijkheid}\\

\haiku{het gevalt mij niet,,!}{dat gij den titel dien ik}{draag niet eerbiedigt}\\

\haiku{Dat vertraagt den tocht,.}{bij het klimmen dat verhaast}{hem bij het dalen}\\

\haiku{Dat belooft de zoon;}{van den Bloedberg en hij geeft}{daarop den handslag}\\

\haiku{{\textquoteright} Me dunkt dat ik een,;}{bro\^er heb gehad doch ik ben}{er niet zeker van}\\

\haiku{{\textquoteleft}Narda,{\textquoteright} hervat Max, {\textquoteleft}?}{een kind met donkere oogen}{en zwarte haren}\\

\haiku{In den hoek der tent}{lagen de poppen en toen}{het middernacht sloeg}\\

\haiku{doch de slaap verlamt.}{zijne grijpende vingers}{en hij ronkt weer voort}\\

\haiku{Max richt zich op, gaat.}{tot bij den slaper en tikt}{hem op den schouder}\\

\haiku{geruiten zakdoek,;}{om den hals geknoopt zoodat de}{tip op den rug hangt}\\

\haiku{Weet ge wel dat het...{\textquoteright}.}{strafbaar is en de spreker}{houdt plotseling op}\\

\haiku{Gij kunt,{\textquoteright} zegt hij, {\textquoteleft}gij,.}{kunt op den molen blijven}{wonen zooals voorheen}\\

\haiku{maar somtijds drijft die.}{bloedgierige valk van het}{kasteel boven ons}\\

\haiku{{\textquoteleft}Denkt ge, Narda, dat?}{wij daar niet samen zullen}{wonen gelijk hier}\\

\haiku{Ik zal u volgen,,.}{doch gij zult mij toelaten}{met een bepaald doel}\\

\haiku{Kalm is het buiten, '.}{storm is het int gemoed}{van den kasteelheer}\\

\haiku{Gisteren was het,,;}{de zoon toen was het Hiob nu}{is het de dochter}\\

\haiku{hij ziet zijnen heer:}{verwonderd aan en zegt min}{of meer verlegen}\\

\haiku{Na een oogenblik.}{peinzens keert hij terug en}{gaat in den bloemhof}\\

\haiku{{\textquoteleft}Ja, laat ons in de,{\textquoteright};}{taal van onzen kindertijd}{spreken hervat Hiob}\\

\haiku{Ook die menschen meent,:}{zij ooit gezien te hebben}{zoowel als dien wagen}\\

\haiku{{\textquoteright} {\textquoteleft}In hoeveel tijds zou?}{men wel te voet van hier naar}{Brussel kunnen gaan}\\

\haiku{Neen, het stadsleven,.}{bevalt hem niet zelfs niet nu}{hij weer wakker wordt}\\

\haiku{Hij zou genegen....}{zijn een huwelijk met de}{gravin aan te gaan}\\

\haiku{de lange, korte,;}{dikke en spitze glazen}{zijn le\^eg of half le\^eg}\\

\haiku{de... zit in alle;}{geval voor de gemaakte}{brokken en stukken}\\

\haiku{doch hij heeft reeds eene....}{uitzondering voor mijnheer}{Doblain de gemaakt}\\

\haiku{Wie wordt hier, in dit,?}{heiligdom door den jongen}{edelman afgewacht}\\

\haiku{{\textquoteleft}Ik kan, ik mag uw,.}{edelmoedig aanbod niet meer}{aannemen Maurits}\\

\haiku{Daar in dat kerksken,!}{is immers voor haar het nieuw}{leven begonnen}\\

\section{Jan Renier Snieders}

\subsection{Uit: De hut van Wartje Nulph}

\haiku{maar om het even, wilt?}{gij mij nu eens zeggen waar}{ik hier verzeild ben}\\

\haiku{nu ik er aan denk,,;}{ben ik blijde dat gij mij}{wakker hebt gemaakt}\\

\haiku{- Uw paard zal u dien,.}{zeggen zoodra wij in}{Turnhout aankomen}\\

\haiku{Uw reisgenoot kan,.}{gezond zijn maar hij ziet er}{mager en spits uit}\\

\haiku{- Nulph, gij kunt nog eens,.}{een groot man worden zei de}{reiziger spottend}\\

\haiku{het dier brieschte.}{en kwam met woeste sprongen}{op het tooneel gehold}\\

\haiku{Toen droomde Wartje,}{van den muilezel en deszelfs}{eigenaar van wien}\\

\haiku{Nulph gevoelde iets,;}{vreemds iets geheimzinnigs in}{zijne ziel omgaan}\\

\haiku{- Otto Richardi,;}{mompelde een der ruiters}{in het Italiaansch}\\

\haiku{onthoud dit, indien.}{gij geen kennis wilt maken}{met zijn lang rapier}\\

\haiku{het  was alsof:}{het dagend morgenlicht hem}{had toegeroepen}\\

\haiku{Zeg haar ook, dat zij}{niet vergeet te bidden voor}{het kruisbeeld voor zij}\\

\haiku{en geknield bad hij,.}{voor de kleine Bertha om brood}{voor zijne ouders}\\

\haiku{Nulph hief nog even het,.}{hoofd op en poogde door het}{venster te staren}\\

\haiku{Terwijl hij het geld}{op de tafel neertelde}{verzekerde Nulph}\\

\haiku{- Dat kind, ging Nulph voort,,...}{is verdwenen geroofd door}{een uwer edellieden}\\

\haiku{Er tintelde een.}{zonderlinge gloed in de}{oogen van den grijsaard}\\

\haiku{Er glinsterde een '.}{straal van hoop opt gelaat}{van den koolbrander}\\

\haiku{- Maar ik versta niet?}{wat hij met een geroofd kind}{zoude aanvangen}\\

\haiku{gij hebt u verzet;}{tegen de soldaten van}{Zijne Majesteit}\\

\haiku{De Pluimgraaf fronste:}{nog sterker de wenkbrauwen}{en hernam driftig}\\

\haiku{Indien hij weigert,,,.}{ziedaar neem dan mijn rapier}{en klopt het er uit}\\

\haiku{In eens zweeft er iets;}{donkers in de lucht dat steil}{naar beneden valt}\\

\haiku{- Zoo hard ik maar kan,,.}{door de velden de bosschen}{dwars over de heide}\\

\haiku{- Maar hoe komen wij,;}{aan den overkant der vest wierp}{baas Canutus op}\\

\haiku{- Bedenk, Charles, zei,;}{de prins die den grond zijner}{gedachten raadde}\\

\haiku{laat mij eerst alles,.}{eens goed afzien of de zaak}{ook wel juist uitkomt}\\

\haiku{- vroeg de Ritmeester,.}{Bacx met een stem bevend van}{verontwaardiging}\\

\haiku{doch deze, zijn hoofd,.}{terugtrekkend maakte de}{spleet nogal kleiner}\\

\haiku{ik zie, dat gij de...,;}{kleine Bertha lief hebt Welaan}{zij blijve bij u}\\

\haiku{Baas Canutus, sprak,.}{hij hard genoeg om maar even}{gehoord te worden}\\

\haiku{en hij wees op twee.}{gewapende mannen in}{den hoek der kamer}\\

\haiku{vergun mij verhaal, -...}{en wees minder gelukkig}{dan naar gewoonte}\\

\haiku{zelfs beweert men, dat '.}{t geheele dorp Ravels}{wemelt van krijgsvolk}\\

\haiku{- Geen genade voor,,;}{die benden morde Varax}{in  het heengaan}\\

\haiku{dat heet bang zijn, zoo,}{te gaan loopen wanneer de}{vijand nog een uur}\\

\haiku{- Ziet gij, Kelker, zooals,;}{ik u gezegd heb ging de}{Zegewoude voort}\\

\haiku{Het was alsof dat;}{kleine voordeel den moed der}{Staatschen bezielde}\\

\haiku{doch Nulph worstelde;}{uit al zijne macht tegen}{de twee soldaten}\\

\haiku{Intusschen was reeds;}{het gevecht tusschen de twee}{benden ge\"eindigd}\\

\haiku{De heide rookte,.}{van bloed en was bedekt met}{dooden en gekwetsten}\\

\haiku{- Heer Graaf... stotterde,,.}{Nulph die niet meer wist wat hij}{hoorde zag of deed}\\

\haiku{Eindelijk, door een,;}{wonder des Hemels heelde}{de wonde der vrouw}\\

\haiku{- Toen spreidde er zich;}{een akelige verf op haar}{vermagerd wezen}\\

\haiku{- Wat mocht den armen,!}{koolbrander zoo ontstellen}{toen hij aanklopte}\\

\subsection{Uit: Het kraaien-nest}

\haiku{Is er in de  , '?}{wereld nog een schooner stee dan}{t Kraaien-nest}\\

\haiku{het lied te hooren,.}{dat de meisjes gedurig}{opnieuw aanhieven}\\

\haiku{- Dat zal daar niet bij,,,.}{blijven riep Lootman de vuist}{dreigend opstekend}\\

\haiku{- Let op uw woorden,,.}{riep Carrero den vinger}{dreigend opstekend}\\

\haiku{neen, al moest er mijn,.}{hoofd af toch laat ik mij niet}{langer dwarsboomen}\\

\haiku{gedurig had hij,.}{de flesch in de hand en het}{glas aan de lippen}\\

\haiku{met mijne handen.}{laat ik geen kastani\"en}{uit het vuur krabben}\\

\haiku{Geurik ging met eens,.}{zoo vluggen stap naar huis als}{hij gekomen was}\\

\haiku{Juist toen hij zijn hand '... (}{uitstak en dreigend naart}{Kraaien-nest wees}\\

\haiku{De ooievaars zijn,,;}{evenals de weelderige}{boschzangers verdwenen}\\

\haiku{- Het spijt mij, niet meer,;}{dan een bed in huis  te}{hebben zei Minten}\\

\haiku{- Moedertje, vroeg hij,?}{beleefd kan ik mijn sigaar}{hier eens aansteken}\\

\haiku{maar neen, de naam van.}{dien rechtsgeleerde boezemt}{mij vertrouwen in}\\

\haiku{Wie is nu de man,?}{die zoo veel belang stelt in}{den armen Minten}\\

\haiku{Nu moet ick u ghaen, '.}{verlaten En kiesennen}{anderen stal}\\

\subsection{Uit: De lelie van 't gehucht}

\haiku{Ik ken er wel meer,,.}{die een grooten naam hebben}{hernam de eerste}\\

\haiku{hij had zwart kroeshaar,,;}{zwarte brandende oogen en}{een pikzwarten baard}\\

\haiku{- Neen, Huibert speelt niet,.}{slecht onderbrak David op}{verzoenenden toon}\\

\haiku{- De dagen zijn kort,.}{heeren antwoordde de meid}{heimelijk lachend}\\

\haiku{Waren dan de twee?}{jongelieden haar beiden}{even onverschillig}\\

\haiku{men laadde 't op,.}{karren of men droeg den zak}{op het hoofd naar huis}\\

\haiku{maar indien de wind,.}{niet wat meer opsteekt krijgen}{wij ons werk niet af}\\

\haiku{- Krampe, het is uw,,;}{graan dat wij gaan inschudden}{sprak de molenaar}\\

\haiku{ik gaf er mijn pink,,;}{van indien ik een zoon had}{zooals Ari\"e van de Schans}\\

\haiku{hij klopte zijn zoon,:}{met zelfvoldoening op den}{schouder en zeide}\\

\haiku{- Vader Krampe, gij,;}{hebt mij boer gemaakt lachte}{Mathias dankbaar}\\

\haiku{wanneer gij trouwt krijgt.}{gij van den ouder Krampe}{een kostbaar bruidstuk}\\

\haiku{- Wat zijt gij vandaag,,.}{netjes met uw sitsen kleed}{lachte de vader}\\

\haiku{men at manden vol...!}{peperkoek en maar heden}{gaat dat heel anders}\\

\haiku{Het meisje voelde,.}{den steek en bleef zwijgend door}{het venster staren}\\

\haiku{ik zal u er maar,;}{door helpen dewijl ik wel}{weet waar gij heen wilt}\\

\haiku{- Wat zijn de vrouwen?}{toch veranderd sedert den}{tijd dat ik jong was}\\

\haiku{- Ik dans niet, vader,,.}{herhaald het meisje met de}{grootste beleefdheid}\\

\haiku{Tegen het vallen}{van den avond stak de speelman}{den strijkstok tusschen}\\

\haiku{- Ik zoo min als een,.}{ander liet Ari\"e van de Schans}{er op volgen}\\

\haiku{kan ik het helpen...?}{dat gij vandaag een blauwe}{scheen hebt geloopen}\\

\haiku{maar neen, ik wil u...,.}{een kans geven ziedaar hebt}{gij uw mes terug}\\

\haiku{- Daar is mijne hand,;}{zei de hoevenaar op een}{gulhartigen toon}\\

\haiku{- Neen, morde Krampe,;}{de wereld staat mij dezen}{avond tegen mijn dank}\\

\haiku{of ben ik niet oud?}{en wijs genoeg om alleen}{naar de Schans te gaan}\\

\haiku{- Een Brabander, die,.}{zooals men zegt wel iets van een}{duivel moet hebben}\\

\haiku{riep een ambtenaar,.}{terwijl hij achter den hoek}{van de schuur wegsloop}\\

\haiku{Zie dat gaat boven,.}{mijn verstand dat niemand van}{ons u herkend heeft}\\

\haiku{om u hiervan te.}{verzekeren kwam ik u}{uit uw bed kloppen}\\

\haiku{- Indien ik nog een,.}{jaar lang kan smokkelen loop}{ik op mijn muiltjes}\\

\haiku{In mijnen tijd... - Laat,;}{de boerderij naar de galg}{loopen lachte Ari\"e}\\

\haiku{- Ach, vader, ik bid,,;}{u word toch geen smokkelaar}{smeekte het meisje}\\

\haiku{vervolgde Dwina,.}{het aangezicht achter haar}{voorschoot geborgen}\\

\haiku{- Ik dacht aan mijne,,.}{moeder Huibert antwoordde}{het meisje treurig}\\

\haiku{- Dwina, beken mij,;}{openhartig waarom gij er}{zoo treurig uitziet}\\

\haiku{Maar zeg, en misleid,?}{mij niet gelooft gij dan niet}{aan het voorgevoel}\\

\haiku{Dien avond, zit Dwina,;}{zonder het rad te draaien}{voor haar spinnewiel}\\

\haiku{Laat anderen maar.}{boeren en zich voor Koning}{Willem lam werken}\\

\haiku{- Gij ziet wel, hoe al,;}{die gekke visioenen}{uitkomen sprak hij}\\

\haiku{gij hebt mij woord, en,}{trouwt met mijn dochter al wilt}{gij morgen dien dag}\\

\haiku{baas Urkhoven zal...?}{die bloem wel openmaken ziet}{gij uw kaarten nog}\\

\haiku{Nooit had men in de;}{dorpen zooveel gesproken}{van den sluikhandel}\\

\haiku{- Vraag mij duizendmaal,, '.}{meer was het antwoord ent}{is u geschonken}\\

\haiku{Het meisje gaf een.}{akeligen gil en sloop door}{het kreupelhout heen}\\

\haiku{kom, neem gij hem bij,.}{de voeten ik leg hem het}{hoofd op mijn schouder}\\

\haiku{Met een teeken van,.}{de hand belette hem de}{zieke voort te gaan}\\

\haiku{- Waarom?... herhaalde;}{de koewachter met een slecht}{bedekten spotlach}\\

\haiku{- Ik wil en zal mijn,!}{geld en ook den verloopen}{interest hebben}\\

\haiku{Waarom kwam hij dien?}{dag bij zijn ouden vriend een}{bezoek afleggen}\\

\haiku{Neen, David, neem mij,.}{niet kwalijk maar uw dochter}{wil ik aan geen prijs}\\

\haiku{Van de Hees tot aan,!}{de poort der hel ken ik geen}{grooter schobbejak}\\

\subsection{Uit: Narda}

\haiku{Dat is het denkbeeld,;}{hetwelk zij nimmer uit haar}{hoofd kan wegdrijven}\\

\haiku{hoe klein di\`e winst ook,...}{zij zou ik toch die stuivers}{niet kunnen missen}\\

\haiku{De lijder kan elk.}{oogenblik ontwaken en}{haar zorg noodig hebben}\\

\haiku{- Nooit had de arme.}{weduwe van dit alles}{een woord geweten}\\

\haiku{er is daar bij die.}{arme menschen aan alles}{volslagen gebrek}\\

\haiku{sprak Narda de beurs,.}{toedraaiend en ging met haar}{vader de deur uit}\\

\haiku{vader, onderbrak,,...}{het meisje dat ware niet}{kiesch niet edelmoedig}\\

\haiku{kuste haar op het,.}{voorhoofd en haalde nog een}{goudstuk uit de beurs}\\

\haiku{Op de tafel stond,,:}{een groote glazen kom waarin}{de nommers lagen}\\

\haiku{hij verschrikte toen.}{hij een zware stem zijn naam}{hoorde oproepen}\\

\haiku{voor het tweede is.}{de politiekamer een}{overheerlijk middel}\\

\haiku{Hild was misschien de,;}{eenige welke geen sterken}{drank had gedronken}\\

\haiku{gij  ziet wel dat,,.}{al ben ik maar een boer ik}{toch wel den weg weet}\\

\haiku{Indien hij mijn zoon,.}{ware deed ik hem rijden}{op een ganzenpen}\\

\haiku{- Tweemaal verscheen de,;}{weduwe op het raadhuis}{luidde het antwoord}\\

\haiku{- Indien de heeren?}{mij van die vernedering}{willen vrijlaten}\\

\haiku{dacht Hild, terwijl hij.}{daar sprakeloos voor de twee}{geneesheeren stond}\\

\haiku{noch zij noch haar zoon.}{hadden de opgediende}{spijzen aangeroerd}\\

\haiku{hoe en wanneer zal,,?}{ik die geen stuiver bezit}{u dat weergeven}\\

\haiku{ik zal u aanstonds,.}{wijd en breed uitleggen hoe}{ik ben gevaren}\\

\haiku{- En wat de menschen,,.}{aangaat niemand raakt het wat}{ik met mijn geld doe}\\

\haiku{het vertrek van den.}{toekomenden dokter was}{bepaald vastgesteld}\\

\haiku{Zoo redeneerde,.}{Brinkpoel wanneer men sprak van}{Oscar Veldenus}\\

\haiku{Hild schokschouderde,.}{even en bleef verlegen naar}{het vloerzand staren}\\

\haiku{- Een glimlach van Hild.}{bewees dat de heer ditmaal}{juist had geraden}\\

\haiku{doch het oogenblik.}{was al te weinig geschikt}{tot gekscheerderij}\\

\haiku{In een der schoonste.}{straten der stad trof men een}{likeurwinkel aan}\\

\haiku{Oscar vroeg bij zich,;}{zelven of hij zich niet in}{een hotel bevond}\\

\haiku{die heeren zitten!}{hier evenals de internen}{in een muizenval}\\

\haiku{genot, Maakt mij al.}{de zegeningen Nogmaals}{weer mijn zalig lot}\\

\haiku{mij dunkt dat ik mijn!}{tegenpartij met mijn vuist}{dwars door zijn maag stiet}\\

\haiku{Neen, mijnheer, zulke,}{kinderen stoot de Alma}{Mater op wie gij}\\

\haiku{de President van.}{het Collegium ligt reeds}{in de diepste rust}\\

\haiku{Alles is doodstil,.}{in het Collegium het}{is bij middernacht}\\

\haiku{en gespaard voor zijn,.}{moeder heeft hij gespaard dat}{hij zelf gebrek lijdt}\\

\haiku{hoeveel koolzaad men,.}{dat jaar had gedorscht en}{hoe het er uitzag}\\

\haiku{neen, ik heb er twee, ',.}{ent zijn goede ooren}{zelfs van de beste}\\

\haiku{zou zijn dochter niet '?}{de rijkste boerendochter}{uitt dorp wezen}\\

\haiku{maar toen ik daareven {\textquoteleft}!}{den burgemeester met zijn}{fleemenddag Nardje}\\

\haiku{ik weet wel dat het,;}{waar is dat hij somtijds met}{mij uit de kerk gaat}\\

\haiku{De hoovaardige,,;}{mensch waar hij gaat of staat werpt}{ook alles overhoop}\\

\haiku{mijn grootste geluk.}{is u en uw kinderen}{gelukkig te zien}\\

\haiku{Het woord was niet uit:}{zijn mond of de wildkooper}{had reeds geroepen}\\

\haiku{vroeg de notaris,.}{terwijl de klerk wederom}{een pitje aanstak}\\

\haiku{blijf in huis, moeder,?}{Horbaak er is toch wel plaats}{voor twee huishoudens}\\

\haiku{sedert eenigen tijd,}{was zij verhuisd en woonde}{eenige minuten}\\

\haiku{het deed haar zoo goed,.}{dat hij haar in haar nieuwe}{woning was gevolgd}\\

\haiku{In de Kempen, in,?}{de Meierij in een of}{ander heidorp}\\

\haiku{en de meisjes en,:}{de straatjongens schaterden}{het uit en riepen}\\

\haiku{En dien zondag was, '!}{het zoo aangenaam zoo frisch}{buiten int veld}\\

\haiku{tranen zoudt gij er,.}{mee schreien verzekerde}{de burgemeester}\\

\haiku{waarom gaat u dat?}{altijd het een oor in en}{het andere uit}\\

\haiku{anderen gekwetst,.}{om de spottende woorden}{trokken den neus op}\\

\haiku{- Zoo dom toch, als de,,.}{ekster dat denkt zijn wij juist}{niet zei een ander}\\

\haiku{- Was die kegelbol,,}{zoo dachten anderen ook}{misschien de vinger}\\

\haiku{De juryleden,;}{zien het gevaar waarin hun}{ambtgenoot verkeert}\\

\haiku{- Och, dat is waar ook,,;}{hernam Veldenus de hand}{opnieuw uitstekend}\\

\haiku{het hoofd wordt er lam,,,;}{van en armen beenen maag en}{hart alles verlamd}\\

\haiku{Hild, gij die altijd,?}{braaf en spaarzaam leeft hebt toch}{wel iets in de beurs}\\

\haiku{- Ik wist niet dat gij?}{in betrekking stond met de}{dochter van Brinkpoel}\\

\haiku{Doch later zal dat;}{geheim geheel ontsluierd}{voor den dag komen}\\

\haiku{zij zag bleek als een,:}{doode en stamelde met}{neergeslagen oogen}\\

\section{Rosalie Sprooten}

\subsection{Uit: Muren van glas}

\haiku{{\textquoteright} Door het raam keek ik.}{hem na terwijl zijn geur nog}{in de ruimte hing}\\

\haiku{{\textquoteleft}Als er vandaag weer.}{geen antwoord op mijn brief komt}{dan vermoord ik hem}\\

\haiku{Ze pakte de draad.}{tussen twee vingers en liep}{ermee naar het raam}\\

\haiku{Voordat Louise de,.}{dop op de stift draaide rook}{ze er altijd aan}\\

\haiku{Ze zou hem bellen.}{en vragen of hij nog aan}{haar brief had gedacht}\\

\haiku{{\textquoteleft}Ik zal ze daarginds{\textquoteright},.}{eens mores leren schreeuwde}{ze door de kamer}\\

\haiku{Ik schatte hem van.}{mijn leeftijd en bekeek hem}{wat aandachtiger}\\

\haiku{{\textquoteright} {\textquoteleft}Ja natuurlijk, ik.}{heb pati\"enten van hem}{in mijn programma}\\

\haiku{Hanekamp werkt heel,.}{hard hij is ook aardig voor}{de pati\"enten}\\

\haiku{De teleurstelling.}{aan de andere kant van}{de lijn was voelbaar}\\

\haiku{mooi weer vandaag, ik,.}{ga de zon schilderen eerst}{een sigaretje}\\

\haiku{De goudkleurige.}{oorbellen hingen bijna}{tot op haar schouders}\\

\haiku{{\textquoteright} {\textquoteleft}Er waren te veel.}{mensen die dachten dat hij}{het goed bedoelde}\\

\haiku{{\textquoteright} {\textquoteleft}Als jij en ik een.}{hemel hebben dan hebben}{de joden die ook}\\

\haiku{De werkzaamheden.}{in overeenstemming met het}{plaatje op de deur}\\

\haiku{De chauffeur trok zo.}{snel op dat ze tegen de}{leuning werd gedrukt}\\

\haiku{Als ze haar ogen dicht,.}{deed kon ze nog zijn arm om}{haar schouders voelen}\\

\haiku{De helft van de tijd.}{vulde ik met zinloze}{voorbereidingen}\\

\haiku{Waarom wil hij niet?}{van mijzelf horen hoe de}{vork in de steel zit}\\

\haiku{In de zes jaar dat.}{ik mijn werk deed had ik dit}{nooit eerder gehoord}\\

\haiku{Met beide handen.}{pakte ze hem vast en wist}{even niets te zeggen}\\

\haiku{Er was een moment.}{geweest dat er uitzicht leek}{op verandering}\\

\haiku{Als je buiten de,.}{kliniek staat kijk je er heel}{anders tegenaan}\\

\haiku{Ik voelde me een.}{kleine zelfstandige met}{te weinig klanten}\\

\haiku{Zijn dagtaak bestond.}{grotendeels uit bijwonen}{van besprekingen}\\

\haiku{Hier, bij de vakbond,,.}{in de politiek dat was}{overal hetzelfde}\\

\haiku{Hij keek even de kring.}{rond en richtte zich tot een}{van de verplegers}\\

\haiku{Anderen waren.}{voor korte of langere}{tijd opgenomen}\\

\haiku{Toen ik afscheid van.}{Lucas had genomen ging}{ik naar mijn atelier}\\

\haiku{Louise ging in de.}{achterste bank zitten en}{volgde de werkster}\\

\haiku{Achter de pilaar,.}{waar Louise naast zat waren}{stemmen te horen}\\

\haiku{{\textquoteleft}Die blaast ook hoog van{\textquoteright},.}{de toren zei Joris}{op maandagmorgen}\\

\haiku{Ik ben het met hem.}{eens maar die neerbuigende}{toon bevalt me niet}\\

\haiku{Vreemd genoeg was ik '.}{s morgens toch blij dat ik}{weer naar mijn werk kon}\\

\haiku{{\textquoteleft}Deze kliniek is{\textquoteright}}{als een kerstboom met te veel}{ballen in de top.}\\

\haiku{Ik moest er zo hard.}{van boeren dat Frederik}{er wakker van schrok}\\

\haiku{Was het gierigheid,?}{van de boeren gebrek aan}{hart voor hun dieren}\\

\haiku{Een veertiger, goed.}{van de tongriem gesneden}{en zelfverzekerd}\\

\haiku{Had ze misschien voor?}{deze personeelschef ook}{nog iets in petto}\\

\haiku{Met enige moeite.}{wist ze aan de overkant van}{de straat te komen}\\

\haiku{Ik vertelde over.}{de tuin en het schilderij}{waar ik aan werkte}\\

\haiku{Ze raakte hem even.}{aan om hem als het ware}{in gang te zetten}\\

\haiku{{\textquoteleft}Legendarische,.}{man helemaal uit Frankrijk}{hierheen gekomen}\\

\haiku{Op een betonnen.}{verhoging liep een van de}{beren heen en weer}\\

\haiku{Er is niets vreemds aan,,.}{mij dacht ze ik zit hier heel}{gewoon met iemand}\\

\haiku{{\textquoteleft}Ik moet nu even naar}{een afspraak maar als je straks}{tijd hebt wil ik graag}\\

\haiku{Die goeie geluiden.}{kunnen toch niet van onze}{afdeling komen}\\

\haiku{In de verte lag.}{het heuvelland alsof er}{niets aan de hand was}\\

\haiku{{\textquoteright} Louise voelde de.}{stroefheid nu al en zag er}{een slecht teken in}\\

\haiku{Nam de hoorn op maar.}{besloot ogenblikkelijk het}{toch maar niet te doen}\\

\haiku{Ze parkeerde haar.}{auto z\'o dat ze zicht had}{op de hoofdingang}\\

\haiku{De maag moet tot rust{\textquoteright},.}{komen zei hij beslist en}{schreef een dieet voor}\\

\haiku{Ik was zo intens.}{moe dat ademhalen nog te}{veel energie kostte}\\

\haiku{Ik heb zowat vier.}{dagen in een tentje aan}{het strand gezeten}\\

\haiku{Waarom had ik niet?}{eerder gezien wat er met}{hem aan de hand was}\\

\haiku{De dag kwam dat ik.}{achter Frederik aan de}{tuin in wandelde}\\

\haiku{Hij was heel bang dat.}{ze hem weer terug zouden}{sturen naar zijn werk}\\

\haiku{{\textquoteright} {\textquoteleft}Ja, ja, zo is het,{\textquoteright}, {\textquoteleft}.}{zei Severijnsbij het GAK}{moet je oppassen}\\

\haiku{Nou nee, dat niet, maar?}{jij hebt toch jaren in het}{gekkenhuis gewerkt}\\

\haiku{Al vijf jaar probeer.}{ik in gesprek te komen}{met de directie}\\

\haiku{Fernand knipperde.}{met zijn ogen en vouwde zijn}{handen op de rug}\\

\haiku{Joris en Felix.}{zouden samen bezig zijn}{met speltherapie}\\

\haiku{In de schemering.}{van het trapportaal rees een}{schaduw voor me op}\\

\haiku{Af en toe pakte.}{ze zijn hand om hem door de}{drukte te loodsen}\\

\haiku{Ze schrok z\'o dat ze.}{al haar pori\"en even open}{en dicht voelde gaan}\\

\haiku{De muzikanten:}{draaiden om Fernand en haar}{heen en zongen}\\

\haiku{Waarom kon ik die?}{wurgende molen in mijn}{hoofd niet stil zetten}\\

\haiku{Felix stond plotseling.}{naast me. De afstand tussen}{ons was al voelbaar}\\

\haiku{{\textquoteleft}Je praat weer als een{\textquoteright},.}{therapeute had hij ooit}{tegen me gezegd}\\

\haiku{Bij de trappen stond.}{het vijftal alsof ze nu}{al van brons waren}\\

\haiku{De obers liepen zelfs.}{op straat met dienbladen vol}{gevulde glazen}\\

\haiku{De clown blies dapper.}{zijn partij mee maar erg van}{harte ging dat niet}\\

\haiku{Samen gleden ze.}{op de grond waar hun hoofden}{elkaar even raakten}\\

\haiku{{\textquoteleft}Ik had bijna, let,,.}{wel bijna een moord gepleegd}{tijdens de optocht}\\

\haiku{Maar dit is wel erg,?}{anekdotisch en buiten de}{orde vind je niet}\\

\haiku{Dat weet ik ook niet,.}{maar wat plotseling opwelt}{heeft betekenis}\\

\haiku{Periodiek werd.}{me namens de kliniek weer}{een bloemstuk gebracht}\\

\haiku{De tekst kende ik {\textquoteleft}{\textquoteright}.}{al.Van harte beterschap}{stond er altijd op}\\

\haiku{Haastig zette ze.}{haar fiets in het schuurtje en}{vluchtte het huis in}\\

\haiku{Ze opende af en.}{toe haar ogen maar ze vielen}{als vanzelf weer dicht}\\

\haiku{{\textquoteright} {\textquoteleft}Nee, hij is over een.}{plantenbak gestruikeld toen}{hij wilde blussen}\\

\haiku{Fernand stond niet meer.}{op de plek waar ze hem voor}{het laatst had gezien}\\

\subsection{Uit: ...De pest voor een schip}

\haiku{Ik stelde mij voor.}{hoe het zou zijn om in zo'n}{jurk te dansen}\\

\haiku{Daar bovenop werd.}{de blauwe deksel van de}{braadketel gelegd}\\

\haiku{Toen pas zag ik dat.}{er echte ogen in zaten}{die mij aankeken}\\

\haiku{Hij zweeg even om dat.}{toekomstbeeld goed tot mij door}{te laten dringen}\\

\haiku{Enkele dagen.}{later was de strijd rond mijn}{huwelijk beslecht}\\

\haiku{{\textquoteleft}Als we kinderen,,.}{krijgen laat ik ze dopen}{dat beloof ik je}\\

\haiku{Het zal de mooiste,.}{dag uit ons leven worden}{dat beloof ik je}\\

\haiku{Ik kon niet anders.}{dan hard in de riem van mijn}{schoudertas knijpen}\\

\haiku{Ik kan me niet meer.}{herinneren of hij ons}{gelukgewenst heeft}\\

\haiku{Dat moet je zien, hoe.}{ze vastberaden op hun}{doel af stevenen}\\

\haiku{Ik krabde tussen.}{haar horens en wreef over haar}{harige wangen}\\

\haiku{{\textquoteright} {\textquoteleft}Maar jullie moeten,,.}{toch iets eten Max wie weet hoe}{laat je weer iets krijgt}\\

\haiku{Ik vond het een te.}{alledaags begin voor een}{reis naar Amerika}\\

\haiku{Onze voetstappen.}{klonken hol toen we over het}{eerste dek liepen}\\

\haiku{In het lamplicht viel.}{mij de versleten groene}{kleur van de vloer op}\\

\haiku{Waarmee ik maar wil.}{zeggen dat je voorlopig}{veel alleen zult zijn}\\

\haiku{Er was veel blikgoed,,,.}{met sardientjes zalm tonijn}{haring en makreel}\\

\haiku{{\textquoteleft}Denk je er wel aan.}{dat over een half uur het eten}{op tafel moet staan}\\

\haiku{Hij sprak met vrijwel.}{niemand en liep meestal}{met gebogen hoofd}\\

\haiku{Even later was ik.}{alleen met de stuurman die}{Lars Wallin heette}\\

\haiku{Table, la vache,,,,.}{escalier fourchette mon}{fr\`ere mes fr\`eres}\\

\haiku{Zijn mooie tenorstem.}{galmde dan door de stal of}{in de weilanden}\\

\haiku{Met de hand voor mijn.}{mond spoedde ik mij naar het}{toilet en gaf over}\\

\haiku{Enkele malen.}{zorgde Max ervoor dat de}{tafel werd gedekt}\\

\haiku{De zwaar bedekte.}{lucht maakte de sfeer in de}{hut onheilspellend}\\

\haiku{Ik had er graag met.}{Wallin over gesproken maar}{het kwam er niet van}\\

\haiku{De taal leek in haar.}{uitspraak verrassend veel op}{mijn Limburgs dialect}\\

\haiku{Onder het kussen.}{had ik  een foto van}{mijn ouders gelegd}\\

\haiku{Natuurlijk zouden.}{we Chicago halen en}{verder ook nog wel}\\

\haiku{Een besmetting van,.}{tien jaar wilde vaart zoals}{hij zelf verklaarde}\\

\haiku{Een lichte schok was.}{voelbaar toen het schip even de}{kademuur raakte}\\

\haiku{Ze wilde heus wel.}{toegeven dat er ook wel}{armoede was hier}\\

\haiku{Achter ons reden.}{de Herings met twee Noren}{op de achterbank}\\

\haiku{Ik had geen enkel.}{verlangen meer om er ooit}{nog terug te gaan}\\

\haiku{Ik genoot ervan.}{als alles in de mess schoon}{en opgeruimd was}\\

\haiku{Hij maakte enige.}{dribbelpasjes om haastig}{bij mij te komen}\\

\haiku{Misschien was Fauch\'e.}{in de machinekamer}{in slaap gevallen}\\

\haiku{Gelukkig was er.}{niemand in de buurt toen ik}{op de deur klopte}\\

\haiku{De dieren vlogen.}{af en aan door de kleine}{opening in het gras}\\

\haiku{Het moet gegonsd en.}{gebromd hebben daar in de}{grond van stervensnood}\\

\haiku{{\textquoteleft}Is het jou weleens?}{opgevallen dat ik last}{van reumatiek heb}\\

\haiku{De stuurman wierp een.}{sentimentele blik op}{een van de foto's}\\

\haiku{Dat zou mij dus ook.}{kunnen gebeuren als Max}{altijd op zee bleef}\\

\haiku{De paniek over haar.}{ouderdom klemde een}{moment mijn adem af}\\

\haiku{{\textquoteright} vroeg ik toen we op.}{een avond in de taxi op weg}{naar de boot waren}\\

\haiku{Er werd gekucht, een,.}{sigaret opgestoken}{een woord gefluisterd}\\

\haiku{Ik werd droevig van.}{die stoere onhandige}{kerels om mij heen}\\

\haiku{De stuurman sloeg vroom.}{af en toe de maat met een}{wapperend handje}\\

\haiku{Een halve minuut.}{later liepen de tranen}{weer over zijn wangen}\\

\haiku{Max en ik keken.}{elkaar aan en slopen er}{behoedzaam naar toe}\\

\haiku{Ik had de indruk.}{dat hij te bang was om lang}{aan boord te blijven}\\

\haiku{Toen hij verheugd thuis,.}{kwam vond hij zijn vrouw in bed}{met een landgenoot}\\

\haiku{Dat zou tenminste.}{enige troost zijn voor het op}{de rede liggen}\\

\haiku{Weet je wat het kost?}{als je zelf de overtocht moet}{organiseren}\\

\haiku{Max sprong meteen op.}{en zocht geschrokken op de}{tast naar de lichtknop}\\

\haiku{Ik ging weer liggen.}{en luisterde gespannen}{naar de geluiden}\\

\haiku{niet want plotseling.}{draaide hij zich om en viel}{mij in de rede}\\

\haiku{Ik dacht aan de vrouw.}{in Stavanger die nu een}{baby koesterde}\\

\haiku{Geboorte en dood.}{op de boerderij deden}{mij op de vlucht slaan}\\

\haiku{Ik dacht dat er iets.}{ernstigs was gebeurd want ik}{zag dat hij beefde}\\

\haiku{Ik pakte de hand.}{van Max en omsloot ze met}{mijn beide handen}\\

\haiku{Op de bovenste.}{trede bleef hij staan en keek}{naar de kapitein}\\

\haiku{In de verte zag.}{ik een klein stipje op het}{water bewegen}\\

\haiku{de man stond die op.}{het ritme van de golven}{op en neer ging}\\

\haiku{Het is geen noodzaak,.}{dat jij ook werkt ik verdien}{meer dan genoeg hier}\\

\haiku{Het heeft me maanden.}{gekost om alles weer op}{orde te krijgen}\\

\haiku{Twee ggd-broeders.}{stapten uit en kwamen met}{een brancard aan boord}\\

\haiku{Even later kwamen.}{de broeders met de brancard}{door de smalle deur}\\

\haiku{Joviaal zwaaide.}{hij terug met een brede}{glimlach om zijn mond}\\

\haiku{{\textquoteright} Hij kuste me en.}{duwde me met bagage}{en al in de trein}\\

\haiku{Terwijl ik al op,}{de bovenste trede stond}{greep hij weer mijn hand.}\\

\haiku{Een voor een liet ik.}{de kledingstukken naast mij}{op de grond vallen}\\

\haiku{Het is het kleinste.}{en minst zeewaardige schip}{van de hele vloot}\\

\section{Albertine Steenhoff-Smulders}

\subsection{Uit: Een abdisse van Thorn}

\haiku{Dan wendde hij zich.}{met snelle beweging naar}{zijn geheimschrijver}\\

\haiku{Maar uit de bonte;}{vensters der Lambertuskerk}{straalde helder licht}\\

\haiku{z\'o\'o kan ik u mijn,;}{opdracht niet geven alsof}{ik tot een knecht sprak}\\

\haiku{Hij behandelt ons,.}{allen altijd alsof wij}{leekebroeders waren}\\

\haiku{ik weet zeker, in.}{dagen van nood zou hij uw}{trouwste vriend blijken}\\

\haiku{De Bisschop stond op:}{en stak zijn afgezant ten}{afscheid de hand toe}\\

\haiku{En ik vertrouw, dat.}{het maal naar genoegen van}{Uwe Edelheid zal zijn}\\

\haiku{Bedoelt gij onze?}{Hooge Vrouwe Mechthild of de}{Vorstin van Heynsbergh}\\

\haiku{De boomen, nog vol,;}{rossige knoppen stonden}{roerloos in de lucht}\\

\haiku{De zoetheid van den.}{avond was voor zijn rusteloos}{hart als koel water}\\

\haiku{{\textquoteleft}Neen,{\textquoteright} zeide hij, haar, {\textquoteleft};}{handen vattendegij zult}{doen wat u goeddunkt}\\

\haiku{voor gansch uw leven,.}{is dit eene beslissing die}{gij zelf moet nemen}\\

\haiku{De la Marck ving.}{hem in zijn armen op en}{ondersteunde hem}\\

\haiku{Want er staat nog meer,.}{nieuws in mijn brief dien ik u}{verder lezen zal}\\

\haiku{wanneer ge bang zijt,.}{voor booze droomen moet ge daar}{niet naar luisteren}\\

\haiku{zij had dan ook in.}{het stift den naam  van zeer}{hooghartig te zijn}\\

\haiku{{\textquoteright} {\textquoteleft}Ridder Willem heeft,?}{voor hem bij de Abdisse}{gebeden nietwaar}\\

\haiku{Ik moet oppassen,.}{en geen woord loslaten dat}{ik niet kwijt wil zijn}\\

\haiku{z\'o\'o strenge tucht als.}{onder haar beheer was er}{te Thorn nooit geweest}\\

\haiku{Heer Haeck herkende.}{den jonker van Perwez en}{riep hem bij zijn naam}\\

\haiku{Maar bij mijn patroon,!}{deze gewelddaad zal niet}{ongestraft blijven}\\

\haiku{'t Is een goed woord,.}{dat gehoorzamen lichter}{valt dan bevelen}\\

\haiku{{\textquoteright} zeide de Cleefsche:}{jonkvrouw met een blik op het}{fijne schilderwerk}\\

\haiku{Verdacht Josine?}{haar tot het huwelijk te}{hebben geholpen}\\

\haiku{straatroovers zouden zich;}{wel wachten de Vorstin van}{Thorn aan te randen}\\

\haiku{kwam zij hier, dan zou.}{zij hare gebiedster geene}{beleediging sparen}\\

\haiku{Waarom bezoekt gij,.}{niet meer de stad dat zou u}{afleiding geven}\\

\haiku{{\textquoteright} De Gravin zag met:}{haar heldere oogen den abt}{onbevangen aan}\\

\haiku{{\textquoteright} En zijn knie buigend,.}{voor den Bisschop trok hij zijn}{zwaard uit de scheede}\\

\haiku{{\textquoteleft}Zegen mij en mijn,,.}{wapen zoo ik het noodig mocht}{hebben mijn vader}\\

\haiku{{\textquoteright} Door alle buurten,.}{verspreidde zich het goede}{nieuws snel als de wind}\\

\haiku{Achter de boomen.}{was de hemel rooskleurig}{waar de zonne zonk}\\

\haiku{De meier sprong op.}{van de houten bank en ging}{den gast te gemoet}\\

\haiku{een tafel v\'o\'or de,;}{bank die zij met een doek van}{grof pellen dekte}\\

\haiku{Hier links om, de weg.}{die naar het water gaat bij}{den Plompen Toren}\\

\section{Reimond Stijns}

\subsection{Uit: Hard labeur}

\haiku{Het dier 			 sloeg met,;}{den kop sprong recht met een snel}{beenengescharrel}\\

\haiku{Gansch het dorp wist het,.}{weldra hoe kranig Mie zich}{verdedigd 			 had}\\

\haiku{Hij kwam eens tot bij,,.}{haar zei 			 hij om wil van}{het ontstichtend feit}\\

\haiku{ze tastte in den,.}{zak naar het mes dat ze nu}{altijd bij zich had}\\

\haiku{Maar wat is een mensch,,!}{die geen geld heeft of er geen}{genoeg 			 bezit}\\

\haiku{Ze staarde hem aan,,,.}{zwijgend met 			 vrees dat hij}{haar afkeuren zou}\\

\haiku{{\textquoteleft}'k Geloof, dat we ',!}{t akkoord 			 zijn maar geen}{katten in zakken}\\

\haiku{Hij had zijn lepel.}{afgelikt en lei hem nu}{neer op de tafel}\\

\haiku{Ze ging, zonder ziel,,;}{gebroken zonder wilskracht}{nog tot opstand}\\

\haiku{Soms wilde hij te,;}{huis blijven als 			 het weer}{te ijselijk was}\\

\haiku{hij mocht er voortaan,,,:}{urenlang 			 stenen weenen}{jammeren huilen}\\

\haiku{de ouders bleven ';}{aant werk op het veld of}{om de woning}\\

\haiku{Maar ja, hij was dan,.}{nog 			 schier een vreemdeling}{en kende haar niet}\\

\haiku{Ho, die verdoemde,!}{rekel kon geld verdienen}{en 			 wilde niet}\\

\haiku{ook de kousen op,.}{haar 			 beenen gevoelde hij}{maar ontdekte niets}\\

\haiku{Eens reeds had ze den,.}{doodsangst gevoeld die nu}{koud over haar neerzeeg}\\

\haiku{{\textquoteright} stotterde ze na, {\textquoteleft}{\textquoteright}.}{een poosof ik smijt mij}{in de Bundergracht}\\

\haiku{Speeltie liet hem los.}{om de deur met dreunenden}{paf toe te slaan}\\

\haiku{zijn 			 gedachten;}{zouden stilvallen bij zijn}{opgezweept pogen}\\

\haiku{over haar kort hemd 			 ,.}{droeg ze een lijveken dat}{niet toegeknoopt was}\\

\haiku{Vercleijen stak de.}{dunne wenkbrauwen op naar}{zijn klein voorhoofd}\\

\haiku{De jongens kunnen,.}{het gaan 			 zeggen als er}{verandering is}\\

\haiku{er volgde een snel.}{voetengescharrel 			 bij}{een wild gestommel}\\

\haiku{Hier is alles te,.}{krijgen wat ge maar in}{de stad vinden kunt}\\

\haiku{We zullen 			 hem;}{brengen tot aan het huis van}{Wieze-Marie}\\

\haiku{Of 			 wroette er?}{misschien minnenijd in het}{diepe van zijn ziel}\\

\haiku{zijn haar, dat Speeltie,.}{gisteren gesneden had}{was vol 			 trappen}\\

\haiku{rechts waren er maar,.}{drie en 			 daar hing niets v\'o\'or}{de donker ruiten}\\

\haiku{Roze, op haar 		  	,.}{zelfkanten sloffen ging en}{kwam geruchteloos}\\

\haiku{Ze naderde meer,.}{de 			 tafel en dacht niet}{aan heur slijkvoeten}\\

\haiku{hij lag er vroolijk.}{in teere lentefrischheid}{van blad 			 en bloem}\\

\haiku{Ze 			 voelde, dat,.}{ze nog niet wieden kon hoe}{ze ook haar best deed}\\

\haiku{Thuis had Lize nooit;}{iets anders gehad dan wat}{stroo en een kafzak}\\

\haiku{Men bracht Lize in,.}{een kamertje en de deur}{werd gesloten}\\

\haiku{Lize durfde de;}{oogen van den wegel niet}{laten afdwalen}\\

\haiku{daarna sloop ze naar,.}{de stalletjes en alles}{was 			 er ook vast}\\

\haiku{{\textquoteleft}'t Zal wel een jaar,.}{duren eer dat gij nen}{cent zult verdienen}\\

\haiku{Ge 			 hadt mij slecht,,?}{verstaan maar dat en zult ge}{nooit meer doen newaar}\\

\haiku{ze had nooit een pop,}{bezeten en in een vaag}{vizioen}\\

\haiku{{\textquoteleft}Vader moet subiet,.}{weten waarom Lize niet}{naar huis en 			 komt}\\

\haiku{Maar, 			 ge zijt, gij,,,?}{de deugniet die het arm jonk}{altijd slaag geeft he}\\

\haiku{{\textquoteright} {\textquoteleft}Lowietje, ga eens mee,{\textquoteright}.}{naar het huis van dien jongen}{verzocht Sofie}\\

\haiku{{\textquoteright} En plots groeide er;}{een onuitsprekelijke}{vrede in haar}\\

\haiku{de maan rees op, en}{er viel een zilvergroene}{schijn in brokken}\\

\haiku{Hier, op deze plaats,.}{had de doodkist gewacht met}{Wannie er in}\\

\haiku{Het avond-angelus, '.}{had geklept ent volk was}{weg van de akkers}\\

\haiku{Overal blonk het 			 ;}{krakend-rijp koren in de}{schoone morgenzon}\\

\haiku{boven klonk er een,.}{wilde kreet die verstierf in}{een gesmoord gesteen}\\

\haiku{Renieldeken lei,;}{het boek neer en haar boezem}{jaagde geweldig}\\

\haiku{de lucht was kalm, en;}{het oosten schitterde met}{stekende stralen}\\

\haiku{{\textquoteleft}G' en moet er al,{\textquoteright}.}{dat verdriet niet in maken}{zei ze meewarrig}\\

\haiku{Roze was toch heel,;}{goed geweest voor haar eer d\'at}{voorviel met Lowietje}\\

\haiku{{\textquoteleft}G' en moogt er niets, '.}{van gebaren maart schuim}{stond op haren mond}\\

\haiku{toen allen binnen,.}{waren sloop Sofie haastig}{naar het schotelhuis}\\

\haiku{Haar schreden brachten}{ze voort altijd verder over}{den zoo bekenden}\\

\haiku{, en voor de eerste.}{maal gaf Speeltie ze maar de}{helft van de som mee}\\

\haiku{'t Was altijd een,;}{leelijke tijd als de pacht}{moest betaald worden}\\

\haiku{{\textquoteright} Die stem maakte heur,,;}{laf gedwee en ze haastte}{zich door het hofgat}\\

\haiku{De jongste zoon liet;}{na elken slok wantrouwend}{de oogen rondwaren}\\

\haiku{Om vier uren schoof de,.}{lucht toe en eenige schaarsche}{druppels pletsten neer}\\

\haiku{het voeder uit het;}{wijmenboschje zou niet tot}{hooi kunnen drogen}\\

\haiku{hij had er reeds aan}{gedacht hun te verbieden}{des Zondags na den}\\

\haiku{{\textquoteright} Maar altijd voort bleef.}{dezelfde tergende stap}{op het wegelken}\\

\haiku{de aderen lagen.}{als een blauw koordennet op}{zijn bruine handen}\\

\haiku{gansch in de verte:}{klonk een enkele maal}{een brallend getier}\\

\haiku{Er lag voorzeker!}{reeds een verschrikkelijke}{som in het kistje}\\

\haiku{, en een hortende,.}{krakende donder doet de}{aarde daveren}\\

\haiku{In reusachtige;}{hallen woedt er een dreunend}{dommelen en slaan}\\

\haiku{, en wanneer ze den,;}{arm loslaten tjaffelen}{ze van rechts naar links}\\

\haiku{{\textquoteright} Zoo klabetterde '.}{de stem van Fine boven}{alt rumoer uit}\\

\haiku{{\textquoteright} en dan trachtte ze,.}{den hoek te ontwaren waar}{Mitie Speeltie zat}\\

\haiku{Peutrus, de waard, was,;}{in het schotelhuisken en}{tapte er het bier}\\

\haiku{Speeltie was niet meer,;}{benauwd maar opnieuw sterk en}{vol zelfsvertrouwen}\\

\haiku{indien Speeltie zich,;}{bezoop dan zouden ze kort}{spel met hem maken}\\

\haiku{om het broodje te.}{verdienen moet men toch iets}{door de vingers zien}\\

\haiku{Hij wendde zich om,.}{en plaatste den wijsvinger op}{de borst van zijn broer}\\

\haiku{{\textquoteleft}Wie ons belet heeft,!}{er te winnen moet nu maar}{voor ander zorgen}\\

\haiku{Het was echter geen,;}{broederlijkheid die Mitie}{en So samenbracht}\\

\haiku{{\textquoteleft}'t Mag in vergift,.}{veranderen als ik er}{mijn lippen aan steek}\\

\haiku{{\textquoteleft}Preutelt ge misschien,?}{z\'o\'o omdat ik hier tegen}{uw goesting verkeer}\\

\haiku{{\textquoteleft}En iemand, die het,.}{al lang wist heeft mij vandaag}{de oogen opengedaan}\\

\haiku{hij had nog eens So,;}{naar huis gezonden en was}{komen vensteren}\\

\haiku{{\textquoteleft}Heb ik zelf beter,,?}{eten gehad dan gij zoolang}{ik niet ziek en was}\\

\haiku{{\textquoteleft}Er valt dezen avond.}{niets meer te verteuteren}{in den Koterhaak}\\

\haiku{ik kan, en 'k zal,.}{mij stil houden zoolang men}{mij niet en bedriegt}\\

\haiku{{\textquoteleft}Ik, ik en laat mij,!}{geen smering geven en ik}{en heb geenen doek noodig}\\

\haiku{g'en hebt nu niets te,.}{doen en vandaag nog moet ge}{naar den huismeester}\\

\haiku{Van de eerste week,,.}{af dat we ginder zijn zal}{ik hen betalen}\\

\haiku{hij rekte den hals,;}{uit en stak de kin omhoog}{om asem te hebben}\\

\haiku{Het haardvuur was lang, ';}{uitgedoofd ent water}{in den pot verdampt}\\

\haiku{Gij moogt doen, wat ge,.}{wilt maar met u en kom ik}{onder zijn oogen niet}\\

\haiku{hij en doet zijnen,!}{mond niet open en hij en wil}{noch eten noch drinken}\\

\haiku{Het was wellicht reeds,.}{lang na middernacht toen ze}{nog eens wakker werd}\\

\haiku{hij keek een poosje,}{naar een delver die ver op}{het veld zijn klompen}\\

\haiku{En voor zijn dood zou! '!}{hij nog het mijne willen}{t Is al te zot}\\

\haiku{{\textquoteright} Mie zou gaarne iets,.}{meer gehoord hebben doch ze}{wachtte te vergeefs}\\

\haiku{We worden beiden, ',.}{oudt is waar doch te oud}{en zijn we nog niet}\\

\haiku{Er is veel over ons,.}{gebabbeld en we zullen}{dat doen ophouden}\\

\haiku{Maar, God lof, Bien had,!}{verzekerd dat de schelm niet}{meer genezen kon}\\

\subsection{Uit: Arme menschen}

\haiku{En Mie zou het niet,.}{meer kunnen verbergen wat}{er gebeuren moest}\\

\haiku{{\textquoteright} In onze buurt woont,;}{een ongehuwde vrouw met}{haar zoon krommen Tist}\\

\haiku{Moet ik ze den hals,?}{omwringen en zoo achter}{de grendels komen}\\

\haiku{Stipt betaalde ze,.}{en tot nu toe kende Jaak}{Gone haar val niet}\\

\haiku{Mie Gone heeft juist,.}{een rok versteld wil ander}{werk ter hand nemen}\\

\haiku{En Nelleken was,...}{zoo voorzichtig maar bij het}{oversteken der straat}\\

\haiku{{\textquoteleft}Vader, vader, laat...{\textquoteright}.}{mij hier Onverwachts duwde}{hij haar achteruit}\\

\haiku{Een deurtje ging open,,:}{een vogel kwam te voorschijn}{en tienmaal klonk het}\\

\haiku{{\textquoteright} Jaak wendde vroolijk.}{zijn stoel om en scheen te zien}{naar het horloge}\\

\haiku{de vrouw zal vader.}{en moeder verlaten om}{haar man te volgen}\\

\haiku{Waarom laat ge mij, '?}{verstaan dat ik vant werk}{mijner dochter leef}\\

\haiku{gij ontnaamt hem zijn,,,...}{gade twee kinders het licht}{zijner oogen en nu}\\

\haiku{het was koesterend,.}{warm in de plaats en helder}{keek er de zon in}\\

\haiku{Haar stem trilde, toen;}{ze hem bad zijn fleschje}{te mogen halen}\\

\haiku{En als ik weet, dat,....}{ge zoo zult handelen dan}{zal ik rustig gaan}\\

\haiku{Soms dacht ze, dat ze,}{zoo aanstonds haar vader zou}{zien binnentreden}\\

\haiku{{\textquoteleft}En de heks ging weer!}{opsteken zonder iets te}{laten  weten}\\

\haiku{'t was, omdat ze,;}{iemand anders gevonden}{had iemand met geld}\\

\haiku{niet meer wil, daar ze,, '.}{een oude een rijke heeft}{diet geld weggooit}\\

\haiku{leefde ze dan nog,!}{alleen om de aanklacht zou}{hij haar doodslaan}\\

\haiku{Er kwam een kindje,...}{en er zonk iets almachtig}{teeders in haar hart}\\

\haiku{{\textquoteright} Het goede meisje,,.}{voelde zich geroerd boog zich}{kuste dat uitschot}\\

\haiku{vol schrik keek ze de,,:}{weduwe aan trok deze}{achteruit en sprak}\\

\haiku{Dagen later was.}{het er heel droevig op het}{zolderkamertje}\\

\haiku{De armendokter,,.}{was gekomen had gezegd}{dat het niets was}\\

\haiku{Zijn fonkelend brein.}{heeft de laatste olie in het}{klein lampje verbrand}\\

\haiku{Denkt ge niet, moeder,?....}{dat vader ginder boven}{gansch anders zal zijn}\\

\haiku{En gingt ge nu voort, '.}{k zou zelfs de doodkist niet}{kunnen betalen}\\

\haiku{Mie Gone voelde,:}{een verlichting toen Wanne}{haar verlaten had}\\

\haiku{daarna was het tijd...}{voor haar om zich naar het werk}{te begeven}\\

\haiku{veertien dagen lang;}{kon ze niet terugkeeren om}{wil van haar vader}\\

\haiku{ze richtte zich naar,;}{den buitenkant rechts waar men}{de kleine begroef}\\

\haiku{daar had men het  ,;}{kuiltje gedolven niet ver}{van een treurwilgje}\\

\haiku{In een huis als het,.}{mijne mag iemand als gij}{den voet niet zetten}\\

\haiku{De weduwe van,}{Marli of ze wist in wat}{berooiden toestand}\\

\haiku{ze dacht er niet aan,,;}{dat haar borst schier ontbloot was}{doch hij zag het wel}\\

\section{Louise Stratenus}

\subsection{Uit: Een verborgen bladzijde uit het leven van Sherlock Holmes}

\haiku{Ik spreek de waarheid,.}{die zelden aangenaam is}{om aan te hooren}\\

\haiku{maar wij zullen ons.}{door geen  hinderpalen}{laten afschrikken}\\

\haiku{een brandgangetje!}{tusschen het spookhuis en het}{aangrenzend perceel}\\

\haiku{Ik voor mij zag niets,;}{dan een zeer alledaagsche}{verlaten keuken}\\

\haiku{wel stak de sleutel,.}{er van buiten op maar hij}{was niet omgedraaid}\\

\haiku{die vrouw was schrander,!}{dat zouden niet velen haar}{hebben nagedaan}\\

\haiku{{\textquoteright} {\textquoteleft}O, heel gaarne, als.}{mijne levensrust er maar}{niet door wordt verstoord}\\

\haiku{Was die kleine maar,;}{eens voor goed weg dan zou er}{nog hoop genoeg zijn}\\

\haiku{maar hij vertrok twee.}{dagen geleden en kwam}{van morgen terug}\\

\haiku{Zij hadden alles..{\textquoteright} {\textquoteleft}?}{voor anderen overEn de}{ouders zeker ook}\\

\haiku{Mijnheer Andr\'e, haar,.}{man was zoo geheel anders}{dan zijne ouders}\\

\haiku{{\textquoteleft}Welk een mooi gezicht!}{en hoe teer en bevallig}{is die gestalte}\\

\haiku{{\textquoteright} {\textquoteleft}Nu, nu, ik zou mij}{in uwe plaats de zaak maar niet}{zoo erg aantrekken}\\

\haiku{men kon zien, dat zij.}{bestemd was spoedig naar den}{hemel weer te keeren}\\

\haiku{{\textquoteright} {\textquoteleft}Ik ga een bezoek.}{afleggen bij vrienden van}{mevrouw Monkbridge}\\

\haiku{{\textquoteleft}Hebt gij nog moed mij?}{naar de familie Arundel}{te vergezellen}\\

\haiku{Nu eens meen ik dat,}{er iemand door de kamer}{sluipt dan weder is}\\

\haiku{Ik word dan ook nooit}{moede er de ontknooping}{van gade te slaan}\\

\haiku{{\textquoteleft}daarna zal ik mij.}{pas weer aan oorspronkelijk}{werk durven wagen}\\

\haiku{{\textquoteright} {\textquoteleft}Maar mijn waarde{\textquoteright}, kon.}{ik niet nalaten in het}{midden te brengen}\\

\haiku{maar op die plaats of.}{in den omtrek was geen spoor}{van hem te vinden}\\

\haiku{{\textquoteright} {\textquoteleft}En dat zal ik ook{\textquoteright},,.}{doen verklaarde Holmes mij}{lachend aanziende}\\

\haiku{Hij zal niet weinig}{verbaasd opkijken als hij}{u thans leert kennen}\\

\haiku{Men heeft de grootste.}{moeite haar een verstaanbaar}{woord te doen uiten}\\

\haiku{hij scheen van oordeel,.}{te zijn dat die hulde hem}{rechtmatig toekwam}\\

\haiku{Ik heb haar niet meer,.}{gezien nadat ik den brief}{had afgegeven}\\

\haiku{Wat hebben zij aan?}{een zilveren soeplepel}{of iets dergelijks}\\

\haiku{{\textquoteright} {\textquoteleft}Waarom gaat gij nog?}{niet eens een bezoek aan ons}{buurvrouwtje brengen}\\

\haiku{Het kan toch uw wensch,?}{niet zijn dat iemand door uw}{toedoen zou sterven}\\

\haiku{{\textquoteright} {\textquoteleft}En ik begrijp er{\textquoteright},.}{al minder en minder van}{gaf ik ten antwoord}\\

\haiku{{\textquoteright} {\textquoteleft}Voor het oogenblik,{\textquoteright},, {\textquoteleft}.}{neen antwoordde hij met kracht}{maar misschien vroeger}\\

\haiku{Die allen vragen.}{zich slechts af hoe spoedig zij}{thuis zullen komen}\\

\haiku{Ik kan er niet over.}{oordeelen voordat ik de}{wond heb onderzocht}\\

\haiku{Mabel Sandford had.}{de zachtste snaren van zijn}{gemoed doen trillen}\\

\haiku{Niet weinig verrast,:}{ons beiden tegenover zich}{te zien zeide zij}\\

\haiku{{\textquoteright} {\textquoteleft}Verder neem ik een,.}{der flinkste agenten mee die}{mij ooit bijstonden}\\

\haiku{Nergens vertoonde.}{zich een spoor door een wapen}{achtergelaten}\\

\haiku{Toch wendde ik nog.}{eene laatste poging aan om}{haar te vermurwen}\\

\haiku{Zij was zonder een{\textquoteright},.}{kreet ter aarde gevallen}{mompelde Percy}\\

\haiku{Ik begaf mij zelfs,}{dien morgen tot een beroemd}{hoogleeraar dien}\\

\haiku{Ik was nog bezig,.}{met mijn ontbijt toen Lestrade}{bij mij binnentrad}\\

\haiku{{\textquoteright} {\textquoteleft}Wij deden het om,.}{beurten wetende hoe schuw}{hij voor vreemden is}\\

\haiku{Hij gaf den door hem;}{gekozen agent een wenk en}{beiden verdwenen}\\

\section{Stijn Streuvels}

\subsection{Uit: Jantje Verdure}

\haiku{er vette blazen,.}{in opzwollen die met een}{zucht uiteen barstten}\\

\haiku{mede - het was de,;}{lucht die hem deugd deed gelijk}{water aan den visch}\\

\haiku{omdat zij er op.}{gesteld was de bakkerij}{in gang te houden}\\

\haiku{Toen waren ze drie,;}{struische jonkheden in}{den bloei van hun macht}\\

\haiku{Het ging er den dag,,,....}{door oven in oven uit zonder}{eind of ophouden}\\

\haiku{om 't geen zij zijn {\textquoteleft}{\textquoteright},.}{kwaden duivel noemde in}{bedwang te houden}\\

\haiku{of hij zelf het was,?}{ofwel Theresia die het}{kwaad in hem stookte}\\

\haiku{fleemde Theresia,,:}{en als de meid reeds op}{straat was riep ze nog}\\

\haiku{- de krekels met hun, '!}{schril gekriept was alsof}{zij hem uitlachten}\\

\haiku{Vroeger had hij nooit,.}{vermoeienis noch het eind}{van zijn macht gekend}\\

\haiku{de ellendigste, -.}{dompelaar der wereld door}{het kwaad bezeten}\\

\haiku{- Ga waar ge wilt, maar, -!}{hier bij mij niet de duivel}{moet eerst uit uw lijf}\\

\haiku{Als Theresia eens;}{iemand uitzond om af te}{spieden waar hij ging}\\

\haiku{- hij stond er bij met,.}{gebogen hoofd beschaamd als}{een verschopte hond}\\

\haiku{- Zie, vandage zou,.}{ik moeten tarwe ziften}{en ik ontzie het}\\

\haiku{schreeuwde Treze, en;}{in een vlaag van razernij}{stiet zij de deur open}\\

\haiku{w\`at zal er mij nu,,?}{overkomen vandaag en de}{volgende dagen}\\

\haiku{Hij kreeg den daver,;}{op het lijf als iemand die}{een koude wind voelt}\\

\haiku{hij zelf gaf ook niet;}{meer toe aan de vrees er te}{zullen verstijven}\\

\subsection{Uit: 'Het leven en de dood in den ast'}

\haiku{Onverwijld nu zal.}{het derde vertoon van het}{schouwspel aanvangen}\\

\haiku{Van eerst af is hij, ':}{op dreef en de draad vant}{verhaal gevonden}\\

\haiku{eer de anderen '!}{het gewaarwerden wast}{al afgeloopen}\\

\haiku{- Ik heb het toen zelfs.}{aan Polfliet noch aan Wipper}{durven vertellen}\\

\haiku{we lieten er hem,,.}{voor elk drie dat was negen}{deuntjes aflappen}\\

\haiku{Ik plaatste hem met de.}{zate op mijn hoofd en hield}{hem bij de pikkels}\\

\haiku{We waren allen '.}{omt even welgezind en}{preusch met den koop}\\

\haiku{hij houdt Fliepo bij ',,;}{t gat van zijn broek en trekt}{lijfelijk trekken}\\

\haiku{'t overige blijft.}{in dunne laag opengestrooid}{om op te drogen}\\

\haiku{Hij is er altijd,!}{zoo bang voor geweest en nu}{lijkt het niemendal}\\

\haiku{Bij 't ontwaken,.}{blikt Fliepo angstig rond als}{om iets te zoeken}\\

\subsection{Uit: Minnehandel. Deel 1}

\haiku{wendde zij nog naar,:}{moeder en weer zonder naar}{antwoord te wachten}\\

\haiku{De beurt ging dapper,,, {\textquoteleft}{\textquoteright}.}{voort Sanne de groote blonde}{meid zong vanLaura}\\

\haiku{- 'k Zou dat liedje,?}{ook willen leeren geef het mij}{om uit te schrijven}\\

\haiku{Jan Derycke moest '.}{het meisje kussen dat hij}{t meest beminde}\\

\haiku{hij voelde dat er;}{hem iets ontbrak dat hij niet}{goed kon uitbrengen}\\

\haiku{Die sone die nam,.}{die menscheit aen die bi den}{vader comen can}\\

\haiku{Die os ende die}{esel die hebbent gheweten}{mer dat dat kint is}\\

\haiku{- Vandage was ze}{met moeder alleen thuis en}{in den achternoen}\\

\haiku{Heur stap danste licht ';}{overt stroo en ze deed de}{inzen rinkelen}\\

\haiku{- Ja, dat hoort erbij,!}{we moeten den duivel soms}{een keersken luchten}\\

\haiku{En Anneke nam.}{den korf met turksch koorn en was}{m\^ee de deuren uit}\\

\haiku{Uw beminnende '.}{altijd voort leven}{lang Phara{\"\i}lde}\\

\haiku{In een wrong was ze -}{weer beneden en den brief}{droeg ze op haar hert.}\\

\haiku{- de tranen sprongen,.}{haar uit de oogen en haar keel}{was toegenepen}\\

\haiku{- Aan niemand zeggen,.}{Max mag niet weten dat ik}{het u verteld heb}\\

\haiku{- Max, Max, wat hebt gij?!}{toch aan mij gedaan dat ik}{u zoo geerne zie}\\

\haiku{- Aan Klaarken moeten ',!}{wet niet vragen dat is}{bijlange gekend}\\

\haiku{- 'K kom uit den meersch,, ', '.}{Maxt gras groeit gulzigt}{is schoon om te zien}\\

\haiku{Dan stak Elsje haar:}{hoofd van achter den muur en}{ze riep al lachend}\\

\haiku{haar wezens waren.}{blozend welgezind en de}{oogen straalden vol lust}\\

\haiku{Al de anderen '.}{kwamen bij en elk beschonk}{die hijt liefst zag}\\

\haiku{- hebt medelijden,,!}{hebt compassie hier met een}{armen blindeman}\\

\haiku{En zij was jong En!}{ik was jong En gij kunt wel}{denken hoe het gong}\\

\haiku{De kerels stormden ';}{binnen int Sterreke}{als in eigen huis}\\

\haiku{De vruchten die van,.}{weerskanten den weg stonden}{ze groeiden lijk zot}\\

\haiku{Met die belofte}{legde hij zijn moede lijf}{te rusten en voer}\\

\haiku{dacht hij zonder te....}{durven zeggen of vragen}{wat er haperde}\\

\subsection{Uit: Minnehandel. Deel 2}

\haiku{buiten dat was er, ',;}{niets op de wereld overt}{dorp over de velden}\\

\haiku{Al op \'een teeken '.}{wast bedrijf begonnen}{over heel de vlakte}\\

\haiku{Verder kwam hij langs,,....}{de hooge roggevelden de}{tarwe de beeten}\\

\haiku{in al zijn geluk,:}{hij aanzag zijn leven al}{een anderen kant}\\

\haiku{- Met die gedachten,}{opgewonden liep hij rond}{en met leede oogen}\\

\haiku{- Hoort ge 't gasten,.}{riep hij naar Max en Fons die}{kwamen bijzitten}\\

\haiku{dat hij iets wonen,!}{wist een princesse van een}{meid en pleizierig}\\

\haiku{de korte houwen.}{neerstig het kruid van tusschen}{de planten schreepten}\\

\haiku{omdat hij zulke...}{zotte dingen deed zonder}{heur te raadplegen}\\

\haiku{- Wat moet hij weten,,!}{hij moet het niet weten hij}{mag het niet weten}\\

\haiku{- voor dat zot gedacht!}{van dien peerdenkweek moeten ze}{ons al ontpachten}\\

\haiku{gaat ge elders geld, '?}{halen voor de landpacht en}{t zaad en de mest}\\

\haiku{- Met hem niet meer dan... -,.}{met een ander Dat zou ik}{niet kunnen Marie}\\

\haiku{vroeger zette men:}{uit achter een wijf lijk een}{boer achter een kalf}\\

\haiku{Pauwels, wie had er?...}{zooiets durven denken dat}{hij met zijn dochter}\\

\haiku{Max was ook wel in,,...}{aanzien ze wist het maar de}{zaken stonden nu}\\

\haiku{Hij dubde, 't ging.}{tegen zijn gemoed maar hij}{dorst het niet zeggen}\\

\haiku{We kunnen altijd '...}{den voet nevens hem zetten}{alst er opaan komt}\\

\haiku{hij liet hem geen tijd.}{en begon altijd zelf van}{de gevreesde zaak}\\

\haiku{'k ben Meijer even:}{gaan uithooren maar er is}{niets uit te krijgen}\\

\haiku{- En als 't hof nu?}{zou open komen en ik het}{zou willen pachten}\\

\haiku{Kannaert, Derycke,,;}{elk voor zich snapte  maar}{wat hij krijgen kon}\\

\haiku{Daarom zelf genoot}{hij van de voldoening om}{den goeden naam dien}\\

\haiku{Aan het hofgat vond.}{hij Peetje Mullie met zijn}{dochter staan kouten}\\

\haiku{of waarom bleef hij?}{niet thuis waar hij zoo wel en}{weeldig en vrij was}\\

\haiku{- De heeren meenen.}{misschien dat er goud in den}{grond te delven ligt}\\

\haiku{hij zag dat ze zich.}{verveelde en liever niet}{alleen met hem was}\\

\haiku{o, ik zie altijd;}{nog uw schoone zwarte oogen}{in mijn gedachten}\\

\haiku{Dan was de drukke -:}{kwettering van andere}{vogels begonnen}\\

\haiku{t moet hier nog al!}{verricht worden en seffens}{komen de gasten}\\

\haiku{Het meisje schreeuwde, '.}{scharrelde om los en sloeg}{hem int wezen}\\

\haiku{- Max ziet er goed uit,.}{vezelde Sanne Kannaert}{tegen Martje Kraaynest}\\

\haiku{, en als er zulk eene.}{in het spel komt vergeet men}{de oude liefde}\\

\haiku{Ze gingen voort bij.}{de bende waar er luide}{leute gemaakt werd}\\

\haiku{riep de burgmeester}{en hij sloeg met de opene}{hand  op den boer}\\

\haiku{En hij twijfelde.}{ook al of het werkelijk}{een ongeluk was}\\

\haiku{Ginder, tenden de;}{dreef kwam de troep op maatstap}{van den trommelslag}\\

\haiku{De dag was nog nog,...!}{niet teneinde en de nacht}{bij lange nog niet}\\

\haiku{- Laat me nog wat, dans, ',.}{maark blijf hier bij moeder}{zei ze vriendelijk}\\

\haiku{ge ziet het wel - we '}{hebben veel leute gehad}{van den zomer en}\\

\haiku{- Ik vergeet het zoo,,:}{gauw niet we dansen nog nu}{zijn we familie}\\

\haiku{dan werd Klara weg.}{gehaald en Max danste met}{een ander meisje}\\

\haiku{Ze worstelden daar,}{zwijgend en alleen tot ze}{hem heel overmeesterd}\\

\haiku{- Ja, ge hebt haar uw,!}{hof al afgestaan ze trouwt}{met Fons Derycke}\\

\haiku{- Zij kan doen wat ze, -.}{wil maar dan moet ze buiten}{op den stond ik blijf}\\

\haiku{ze ving er hier en}{daar een spreuk van op maar dat}{alles lag zoo ver}\\

\haiku{Daarbuiten, verder ';}{overt veld hing de avondlucht}{in schemerblauwte}\\

\haiku{Voor hen bestonden.}{de meisjes en al wat er}{van leute rond stond}\\

\haiku{Zie, nu danst hij wel,,!}{met Mathilde die met haar}{eerste liefde ha}\\

\haiku{- Mij overpeinzen, ik}{heb er geen haaste bij om}{nu al te trouwen}\\

\haiku{- Dat is den rechten,....}{zin als er meenste bij is}{grijpt de boer maar door}\\

\haiku{- We kunnen er niets,,.}{aan doen Klara die dingen}{hangen in de lucht}\\

\haiku{waren stand van de.}{dingen of wat er nu goed}{of slecht gebeurd was}\\

\haiku{- Ik ben blij, vrouwe ',.}{datt voorbij en gedaan}{is mompelde hij}\\

\haiku{hij wilde haar niet.}{meer genaken en ging al}{den overkant alleen}\\

\subsection{Uit: De oogst}

\haiku{Waarom hield zij de?}{armen zoo hoog en haar lijf}{zoo uitgespannen}\\

\haiku{dat Pikkaert Zondag.}{laatst gevochten had tegen}{drie felle boschkanters}\\

\haiku{Waarom deed hij niet?}{lijk Pol en Lieven en Jaak}{en zijn ander broers}\\

\haiku{alles uit zijn hoofd,.}{steken lustig leven en}{aan niemand denken}\\

\haiku{Morgen zal 't er,}{op losgaan jongen hoe meer}{we werken hoe meer}\\

\haiku{voor kermiszondag.}{waren we met ons zakken}{vol geld alweer thuis}\\

\haiku{Bij Qu\'elin, daar zal ' ';}{t een lang getij opt}{zelfde gedoen zijn}\\

\haiku{was ik dat gij mij.}{stuur zoudt bekeken hebben}{en boos zijn op mij}\\

\haiku{z'hadden met een half '.}{dozijn pikkers heelt dorp}{omvergestooten}\\

\haiku{Maar nu moest het er}{door en hij vertelde in}{korte  reken}\\

\haiku{'s Anderen daags.}{zetten de twee voorloopers}{uit met de boodschap}\\

\haiku{Wie mag dat hier al?}{bezorgen en bezeilen}{op zulk een gedoen}\\

\haiku{Hier en daar \'e\'en wreef.}{den vaak uit de oogen en keek}{vragend in de lucht}\\

\haiku{Na het eten zochten.}{zij koelte en verfrissching}{in den vischvijver}\\

\haiku{een dubbele dreef '.}{oogstkoorn die vant veld tot}{aan de hofste\^e stond}\\

\haiku{en Lida, ziet ze,?}{hem geern en verlangt ze ook}{tot hij naar huis komt}\\

\haiku{Wies zat nog altijd.}{te luisteren als Aga reeds}{lang uitgekout was}\\

\haiku{de vijf kunnen we,,,!}{hier niet blijven toe kerels}{niet hondig zijn hoor}\\

\haiku{En Aga, o, Aga, ze}{zat nu zeker nog verpaft}{te kijken naar den}\\

\haiku{vreemden jongen die.}{het zilverstuk in heuren}{schoot geworpen had}\\

\subsection{Uit: De teleurgang van de Waterhoek}

\haiku{- Warm weer... Eens dat de,.}{brug er ligt zal het toch veel}{gemak meebrengen}\\

\haiku{- 'k Heb mij laten.}{gezeggen dat hier een brug}{over de Schelde komt}\\

\haiku{bij huwelijk en.}{bij geboorte werden er}{geen vreemden geduld}\\

\haiku{dan waart ge ineens,?}{van indringers en brug al}{te zamen verlost}\\

\haiku{In Broeke's kop stond;}{het nog altijd als iets dat}{niet gebeuren k\'an}\\

\haiku{De mannen trimpten,;}{naderbij stonden schijnbaar}{onverschillig}\\

\haiku{Hij dorst zichzelf de.}{afkeer niet bekennen om}{dat lijk nog te zien}\\

\haiku{over de meers hing de,;}{mist eendikte zodat men}{geen worp ver zien kon}\\

\haiku{de streek van Ronse,.}{af en zou eerst tenden de}{week terugkeren}\\

\haiku{- Neen wijf, niemand mag,.}{het weten we mogen het}{niet voortvertellen}\\

\haiku{Wanne bleef echter.}{in twijfel en niet geneigd}{om aan te pakken}\\

\haiku{Ik gevoel geen lust.}{Lander gezelschap te gaan}{houden waar hij zit}\\

\haiku{Dat de brug er komt,:}{en de steenweg kan ons niets}{dan voordelig zijn}\\

\haiku{Werken om geld te,?}{verteren was er buiten}{dit nog iets anders}\\

\haiku{- Gooi die vent buiten,,.}{zegde hij droog weg hij komt}{onze baard smouten}\\

\haiku{Het was kwestie van....}{uit de ogen te zien en te}{laten betijen}\\

\haiku{In Thyssen en zijn:}{makkers herkenden zij hun}{eigen geaardheid}\\

\haiku{Een verdomd dingen:}{dat geen van zijn eigen zoons}{er toe geschikt scheen}\\

\haiku{- Ik zal het verdomd!}{wel uithouden tot die me}{kan opvolgen ook}\\

\haiku{Zijn stap was immer,;}{gehaast zijn blik verstrooid of}{naar binnen gekeerd}\\

\haiku{Met de personen ';}{vant hotel zelf kwam hij}{weinig in gesprek}\\

\haiku{t Overige van.}{de dag besteedde hij aan}{zijn briefwisseling}\\

\haiku{riep  hij zwetsend,.}{naar een makker d\'a\'ar zal hij}{wat tegenkomen}\\

\haiku{hij verlangde naar -;}{niets zijn gemoedsrust had hij}{volkomen bewaard}\\

\haiku{Thyssen praatte voort,.}{deed alsof hij er niets van}{wist of gevoelde}\\

\haiku{in 't water te,?}{rollen dat ze u morgen}{verzopen vinden}\\

\haiku{Doch wanneer hij zocht}{hoe het uit te voeren of}{op welke manier}\\

\haiku{Zij kwelde zich met:}{vermoedens tenemaal uit}{de lucht gegrepen}\\

\haiku{Hij keek er naar uit,}{enkel en alleen omdat}{die wandelaarster}\\

\haiku{Inwendig was hij.}{tevreden er alzo van}{af te geraken}\\

\haiku{Ziet ge, zonder zijn.}{tussenkomst was het toch langs}{die kant uitgedraaid}\\

\haiku{als ze er trek in, ':}{vond hem even aan te gluren}{wast gelopen}\\

\haiku{Nu er echter weer,.}{iets op schuit was voelde hij}{trek er bij te zijn}\\

\haiku{E\'en voor \'e\'en trok hij.}{bij arm of schouder op de}{pont en voer aan kant}\\

\haiku{- En hoe dat ge nu,?}{hier zit als die heer u thuis}{moest komen vinden}\\

\haiku{- richt er later mee,.}{uit wat ge wilt hij zal u}{de handen likken}\\

\haiku{Niet dat hij bang was -.}{voor de ontknoping het moest}{toch eens gebeuren}\\

\haiku{Soms stelde Mira:}{er een wreed genot in haar}{minnaar te plagen}\\

\haiku{haar tot beternis,.}{te brengen tot deftigheid}{aan te wakkeren}\\

\haiku{God, wie zou er ooit!}{gedacht hebben dat hem zo}{iets te wachten stond}\\

\haiku{een hinderlaag waar -.}{hij zich had laten vangen}{een oord van verderf}\\

\haiku{Hij rekende het:}{zijn moeder aan als ikzucht}{en eigenbelang}\\

\haiku{evenals de stroom van ',.}{t water in de Schelde}{gestadig vooruit}\\

\haiku{Als ge mij anders,,.}{wilt laat me dan maar en zoek}{er een van uw soort}\\

\haiku{Maurice had heel 't ',.}{bestier ent bevel moest}{overal raad geven}\\

\haiku{al de huizen met,.}{loof vaantjes en kleurige}{lanteerntjes gepint}\\

\haiku{De gedachte aan;}{Mira drukte hem weer in}{de werkelijkheid}\\

\haiku{telkens de hartstocht,}{hem overmande was het als}{baren die komen}\\

\haiku{Spikkerelle in.}{nar verkleed die muilen trok}{en grappen maakte}\\

\section{Nico van Suchtelen}

\subsection{Uit: De stille lach}

\haiku{Drie jaren voor je,;}{geboorte dacht ik dat mijn}{verlangen dood was}\\

\haiku{Neen, Agnes, ik zal.}{geen half-verdichte}{memoires schrijven}\\

\haiku{Uw gestalte was.}{mij bekend en vertrouwd als}{uit een ouden droom}\\

\haiku{En ge houdt veel van,,.}{uw oom dat zag ik ook in}{dienzelfden glimlach}\\

\haiku{In elk geval was.}{het d{\'\i}e glimlach die mij moed}{geeft u te schrijven}\\

\haiku{Mevrouw, ik zal het,;}{lied opschrijven ofschoon g{\'\i}j}{het ook moet kennen}\\

\haiku{Maar enfin, mogen;}{zij onbeantwoord blijven}{tot in eeuwigheid}\\

\haiku{U hebt ook met een;}{bijzondere aandacht op}{Anneke gelet}\\

\haiku{U hebt, o, u hebt;}{dien dag zooveel gedaan en}{vooral niet gedaan}\\

\haiku{En Jaap heeft het nog {\textquoteleft},{\textquoteright}.}{dikwijls overdat mefou dat}{zoo voor me lachte}\\

\haiku{Ik zou hem dan maar,.}{haten en mij bovendien}{schamen over mijzelf}\\

\haiku{En ten slotte, z\'o\'o ',.}{als ikt daar beschrijf zoo}{zag ik het leven}\\

\haiku{En vandaag dacht ik {\textquoteleft} '.}{er bijent moeilijkste}{om over te schrijven}\\

\haiku{Ik wilde u heel;}{iets anders vertellen dan}{wat ik ben of doe}\\

\haiku{Maar d\`an, als ik den,.}{stillen lach hoor dan begrijp}{ik ze en glimlach}\\

\haiku{Daarom, als ik de,}{menschen minacht dan doe ik}{dat alleen omdat}\\

\haiku{{\textquoteleft}zie je wel, zie je?}{wel dat menschen elkaar t\'och}{begrijpen kunnen}\\

\haiku{En nu tracteer ik.}{mijzelf op middagthee en}{mijmer zoo'n beetje}\\

\haiku{Ja, welbeschouwd geldt.}{dit alles precies evenzeer}{tegenover menschen}\\

\haiku{Ik geloof zelfs dat,,.}{het sterker gezonder wordt}{physiek en psychisch}\\

\haiku{Maar van wie of wat?}{gaat de opzettelijke}{verandering uit}\\

\haiku{Dat ik dit alles.}{beleefd heb kan ik mij haast}{niet meer voorstellen}\\

\haiku{sinds zijn terugkomst.}{uit het gesticht heeft hij haar}{nooit meer mishandeld}\\

\haiku{{\textquoteleft}Ik heb honger, ga,{\textquoteright}.}{mee naar Vrijland en houdt haar}{den Prins Regent voor}\\

\haiku{- Juffrouw Bergman had:}{op de Zondagschool aan de}{kinderen gevraagd}\\

\haiku{Voor het eerst heb ik:}{Jaap echt driftig gezien op}{een levenloos ding}\\

\haiku{Dit zijn de vragen:}{waarmee ik mij kwel als ik}{het portret aanzie}\\

\haiku{Gelukkig, want van.}{middag betrapte ik de}{ware boosdoeners}\\

\haiku{Ik stond op en in;}{het zelfde oogenblik hief}{het beest den kop op}\\

\haiku{Zij vloog zeer snel en.}{voortdurend hoorde ik een}{sterk suizend geluid}\\

\haiku{Ik wil u heusch;}{liever alleen schrijven voor}{mijn eigen plezier}\\

\haiku{of, om in den geest {\textquoteleft}{\textquoteright}:}{te schrijven van uw waarlijk}{vernuftigen vriend}\\

\haiku{{\textquoteleft}Wees voorzichtig en,.}{kom terug als je er je}{ongelukkig voelt}\\

\haiku{{\textquoteleft}Oom, vertel me eens,;}{wat van Joost Vermeer u zult}{hem wel goed kennen}\\

\haiku{En hij, blij dat ik,.}{zoo kennelijk een praatje}{verlangde ging door}\\

\haiku{en we hadden hier ';}{eensn mierenhoop van meer}{dan een meter hoog}\\

\haiku{{\textquoteleft}Ik zal er me wel,{\textquoteright}}{voor wachten een roman te}{vertellen aan \`u}\\

\haiku{Aan z'n steenen nymphen;}{en godessen in het park}{neemt geen mensch aanstoot}\\

\haiku{Ik weet zoo weinig.}{van u. Misschien bent u niet}{als de anderen}\\

\haiku{Maar v\'o\'or Anneke,.}{aan haar portie begon schoot}{hem iets te binnen}\\

\haiku{En hoe anders deed:}{hetzelfde liedje in de}{variatie}\\

\haiku{Zij kwamen juist uit,.}{de Zondagsschool thuis met erg}{brave gezichtjes}\\

\haiku{Wij konden blijkbaar.}{den juisten toon tegenover}{elkaar niet vinden}\\

\haiku{Ik kan er niet af,?}{ik kan toch de partij niet}{laten mislukken}\\

\haiku{Maar nu achteraf.}{lijkt het mij een vrij banaal}{buitenpartijtje}\\

\haiku{Misschien onder het;}{verzinnen of uitvoeren}{van een nieuwe streek}\\

\haiku{Ik had moeite om,.}{niet te laten merken dat}{ik dat alles wist}\\

\haiku{Er is niemand die{\textquoteright},,}{van mij houdt dacht zij dan en}{van dat treuren werd}\\

\haiku{Maak zulk een mantel,,.}{en als hij eenmaal klaar is}{zal hij wel komen}\\

\haiku{Maar de prins kwam niet,,.}{en als de nacht viel ging zij}{eenzaam weer huiswaarts}\\

\haiku{Weer wees hij met zijn:}{zweep wijdheen in den ronde}{en weer richtte hij}\\

\haiku{{\textquoteleft}'t Hielp me altijd{\textquoteright},, ',.}{erg voegde ze God beter}{t er nog aan toe}\\

\haiku{Alleen, als we doen,,.}{of die kunst iets bijzonders}{is zijn we ijdel}\\

\haiku{Vertrouw op mij, dan.}{zal ik zoo sterk w\'orden als}{jij denkt dat ik ben}\\

\haiku{Ik zal je alles,}{alles geven waarom je}{moe en ontberend}\\

\haiku{Och Joost, eigenlijk.}{maakte die laatste zin van}{je brief mij bedroefd}\\

\haiku{wat ik je schrijf lijkt.}{me dikwijls zoo armelijk}{bij het overlezen}\\

\haiku{hij zag afgebeeld.}{alle foltertuigen van}{des Heilands passie}\\

\haiku{*~        {\textquoteleft}Mein Freund, das grad'.}{ist Dichter-Werk Dass er sein}{Tr\"aumen deut und merk}\\

\haiku{Glaub mir, des Menschen.}{warster Wahn Wird ihm im}{Traume aufgetan}\\

\haiku{wat mijn optreden,,.}{vrees ik dikwijls vrij idioot}{en hinderlijk maakt}\\

\haiku{Strauss, Futurisme,.}{snobbisme en tientallen}{andere ismen}\\

\haiku{Hij zweefde boven.}{gletschers die nog niemand had}{kunnen beklimmen}\\

\haiku{Het benauwt me zoo,,.}{als ik voel dat iemand iets}{van me verwacht}\\

\haiku{LIESBETH, Liesbeth, neen,,!}{dat is niet te dulden dat}{is direct idioot}\\

\haiku{Je vond mijn overkomst,,}{benauwend ik zag het aan}{je verschrikt gezicht}\\

\haiku{{\textquoteleft}Ik kom terug, na, '{\textquoteright}.}{mijn lezingen en dan blijf}{ikn paar weken}\\

\haiku{je moet je rust en,.}{evenwicht weer herkrijgen en}{ik kom je helpen}\\

\haiku{Het is waar, wij zijn,.}{beiden even eenzaam elk op}{ons perronnetje}\\

\haiku{Maar ik kan het niet,,.}{helpen Joost ik voel niet dat}{ik naar jou toe moet}\\

\haiku{Het lijkt heel hard, dat,;}{zoo koel te zeggen tegen}{jouw verlangen in}\\

\haiku{Teleurgesteld was, {\textquoteleft}{\textquoteright}.}{je omdat ik begeerde}{als een ander man}\\

\haiku{Maar gisterennacht}{sloop je naast mijn bed en je}{nam mijn hoofd tusschen}\\

\haiku{Ik heb je voordracht,;}{hier bijgewoond verborgen}{achter een zuiltje}\\

\haiku{Ik heb er uren mee,!}{in mijn handen geloopen}{ik heb het gestreeld}\\

\haiku{Misschien, waarschijnlijk;}{is het dit besef dat mij}{zoo gelaten maakt}\\

\haiku{Maar als ik ooit mijn,,?}{macht over hem verlies mijn arm}{duifje wat  dan}\\

\haiku{Ze prakkiseert over, ' '.}{iets ik hoop maar niet datt}{omn jongen is}\\

\haiku{Ik vroeg het oudste.}{meisje of Liesje hun}{zusje was geweest}\\

\haiku{Wij keken alle,;}{vier zwijgend naar het lieve}{teere gezichtje}\\

\haiku{Al zing ik haar geen,.}{welkom meer Mijn grafje tooit}{zij telken keer}\\

\haiku{- ~ O schrei niet wijl,?}{ik heen moest gaan Wat kwaad heeft}{mij de Dood gedaan}\\

\haiku{Van al wat schoonst en.}{lieflijkst is Ben ik nu ziel}{en heugenis}\\

\haiku{hier rust ik aan uw.}{heilig hart even vertrouwd als}{waar ook ter wereld}\\

\haiku{Liesje, liefste, veel;}{tijd zal ik wel nooit hebben}{om je te schrijven}\\

\haiku{In den Courrier,,.}{tegelijk met mijn brief vind}{ik jouw novelle}\\

\haiku{Onmogelijk, riep,.}{hij zoo iets kan alleen een}{vrije Duitscher zeggen}\\

\haiku{je denkt zelfs niet dat,;}{ik gek ben je denkt in het}{geheel niet aan me}\\

\haiku{hoe gruwelijker,.}{zij zijn hoe gedwee\"er zij}{worden gediend}\\

\haiku{Denk aan de pers, de,.}{koningin der aarde de}{vrouwe Babylons}\\

\haiku{Geen dictatuur van,}{vechthelden en krachtpatsers}{zelfs al hadden zij}\\

\haiku{Weer dwaalde ik in ':}{het bosch ens avonds schreef ik}{deze versjes}\\

\haiku{Ik vond een weide,;}{als nooit betre\^en Konijnen}{talloos haastten heen}\\

\haiku{En bij elk wonder,:}{dat ik zag Luidde in mijn}{hart de stille lach}\\

\haiku{Je leek op Liesje,,.}{het gestorven meisje maar}{je was het toch zelf}\\

\haiku{Den eersten keer toen.}{hem werd medegedeeld dat}{hij blind zou blijven}\\

\haiku{denk eens aan, zuster,?}{hoevelen valt zooiets te}{beurt in honderd jaar}\\

\haiku{om als een bende}{amokmakers te vuur en te}{zwaard te verwoesten}\\

\section{Sonja Surink}

\subsection{Uit: De man met de vele gezichten}

\haiku{Thea stelt ook thans haar.}{gaven nog in dienst van het}{algemeen belang}\\

\haiku{Heb je d'r al wat,?}{van gehoord dat er weer eens}{eentje vermoord is}\\

\haiku{Tegelijkertijd;}{echter was zij nieuwsgierig}{naar wat hij zou doen}\\

\haiku{Aan deze voorzorg.}{zou ze weldra haar leven}{te danken hebben}\\

\haiku{Maar wat ik daar mee,,.}{te maken heb waarlijk het}{is me een raadsel}\\

\haiku{U kan er niet over,.}{oordeelen zoolang u mijn}{positie niet kent}\\

\haiku{We gaven het den.}{wat weidscheren naam van een}{Geheim Genootschap}\\

\haiku{En dat binnen twee,.}{weken waarvan er nu al}{anderhalf om zijn}\\

\haiku{{\textquoteright} {\textquoteleft}In hun oogen zal die,{\textquoteright}.}{zoo onmogelijk niet zijn}{luidde het antwoord}\\

\haiku{{\textquoteright} vroeg Thea, zonder zich.}{iets om zijn ontsteltenis}{te bekommeren}\\

\haiku{{\textquoteright} {\textquoteleft}Om hem te laten,?}{weten dat ik met u over}{hem heb gesproken}\\

\haiku{Nee, om hem te doen,,{\textquoteright}.}{gelooven dat u niet met mij}{sprak antwoordde ze}\\

\haiku{{\textquoteright} Zijn stem klonk rauw en.}{met onverholen dreiging}{een beetje hoonend}\\

\haiku{Sloterdijk eenmaal,.}{voorbij zou zelfs een schot niet}{meer worden gehoord}\\

\haiku{Ze peinsde zoo diep,.}{dat ze zich de tanden in}{de onderlip beet}\\

\haiku{waarom anders had?}{die kerel niet meteen ook}{h\'a\'ar doodgeschoten}\\

\haiku{Aan mij, ja, maar we.}{bespreken de kwestie thans}{in vergadering}\\

\haiku{ik begreep, dat de,}{donkere gestalte in}{de lucht een vrouw was}\\

\haiku{ik verwachtte meer,.}{succes in Amsterdam et}{voil\`a daar zijn we dan}\\

\haiku{Aan den anderen.}{muur is geen venster en voor}{dat eenige stond ik}\\

\haiku{{\textquoteright} {\textquoteleft}Dat is handeling,,{\textquoteright}.}{waarbij u zelf betrokken}{bent merkte Thea op}\\

\haiku{{\textquoteright} {\textquoteleft}U wilde dus nu?}{nog uw onderzoek naar du}{Maurier beginnen}\\

\haiku{Het was bij elven.}{toen zij den sleutel stak in}{haar eigen huisdeur}\\

\haiku{Zeker, dacht Thea, maar '.}{in de stad is het ook niet}{alless winters}\\

\haiku{Gedateerd op 15,.}{November was het ding nog}{geen twee weken oud}\\

\haiku{Jimmy had haar een,.}{telefoonnummer doch geen}{adres opgegeven}\\

\haiku{Uw toestand is niet,.}{te benijden dat stem ik}{natuurlijk grif toe}\\

\haiku{{\textquoteright} {\textquoteleft}Zeker, maar ik wist,.}{niet dat ik mijn leven aan}{hem te danken had}\\

\haiku{Ze begon zich moe,.}{te voelen te verlangen}{naar huis en naar rust}\\

\haiku{een als het ware,.}{idyllische plek om iemand}{van kant te maken}\\

\haiku{Veel tijd had ze er,.}{niet voor noodig want er stond maar}{weinig geschreven}\\

\haiku{Want met nummer vier.}{viel nog minder te spotten}{dan met diens meester}\\

\haiku{Om half acht wilde,.}{ze de deur uitgaan en dus}{moest ze voortmaken}\\

\haiku{Maar komt u liever,}{vergeefs aan de deur mij is}{het natuurlijk best.}\\

\haiku{De vrouw aarzelde,.}{een oogenblik nam haar van}{hoofd tot voeten op}\\

\haiku{Wie zou ooit zonder,,?}{begeleiding als vreemde}{hier boven komen}\\

\haiku{Nu voelde ze aan,.}{den houten deurknop die zich}{niet om liet draaien}\\

\haiku{Dan sprak de stem zoo,.}{vlug dat Thea de woorden niet}{kon onderscheiden}\\

\haiku{Ik vermoedde, dat;}{ze jou onder den duim had}{weten te krijgen}\\

\haiku{het geval is, in.}{een  stemming van speelschen}{overmoed te komen}\\

\haiku{Ik heb hier te doen.}{en kan u niet gebruiken}{op het oogenblik}\\

\haiku{Daarheen zou de Waard,.}{gegaan zijn als je niet op}{hem geschoten had}\\

\haiku{{\textquoteright} {\textquoteleft}Voor vanavond zal ik,{\textquoteright},.}{hem zeker hebben zei ze}{ietwat voorbarig}\\

\haiku{{\textquoteleft}Maar wel kan ik u,.}{vragen waarom u mij geen}{antwoord geven d\`urft}\\

\haiku{Denk daar niet meer aan,,.}{er staat heel wat anders te}{doen vermoedelijk}\\

\haiku{Als aanstonds die stem?}{aan de telefoon zich weer}{tot h\'a\'ar richten zou}\\

\haiku{Maar ze wist, dat ze,.}{dit keer aan niemand gezegd}{had waar ze heen ging}\\

\haiku{En ik zou het graag,,,}{nog eens doen om te kijken}{hoe je er uitziet}\\

\haiku{Omdat je Charles.}{geen belemmering in den}{weg wilde stellen}\\

\haiku{{\textquoteright} {\textquoteleft}Dat is wat te veel.}{geofferd op het altaar}{van je ijdelheid}\\

\haiku{Daar heb je misschien?}{kort voor zijn dood nog wel een}{staaltje van gezien}\\

\haiku{Je zal in de cel,,}{tijd genoeg krijgen om het}{mij eens te schrijven}\\

\subsection{Uit: Het verdwenen meisje}

\haiku{Ja, natuurlijk, als,....}{ik er eenmaal over praat ga}{ik er ook op af}\\

\haiku{Als jagers in uw,,,.}{dienst met Anna als vergeef}{mij het woord als wild}\\

\haiku{{\textquoteright} {\textquoteleft}Ja, een pas heeft ze,.}{doch natuurlijk droeg ze dien}{niet altijd bij zich}\\

\haiku{Maar dat is hier, het,{\textquoteright},}{was hier verleden Dinsdag}{zei mijnheer Barends}\\

\haiku{terwijl hij meteen.}{in een ander laatje van de}{secretaire keek}\\

\haiku{De vriendin echter.}{zou wellicht om dezen tijd}{thuis te treffen zijn}\\

\haiku{Dan zal ik je nou,,}{alleen laten opdat je}{er aan werken kunt}\\

\haiku{Thea vond het prettig,}{\'e\'en dag per week te kunnen}{uitslapen zoolang}\\

\haiku{{\textquoteright} {\textquoteleft}En nu is er iets,?}{gebeurd waar u mijn raad in}{noodig dacht te hebben}\\

\haiku{Was dit bezoek een,.}{spoor dan zou ze het zich niet}{laten ontglippen}\\

\haiku{Ze had zich overtuigd,,.}{doch ze kon niet absoluut}{zeker zijn wist ze}\\

\haiku{Ze voelde het, toen,.}{iemand achter haar langs liep}{naast haar stoel bleef staan}\\

\haiku{{\textquoteleft}Wilde je zeggen,?}{dat ook Betty een stem in}{het capittel heeft}\\

\haiku{Ze liepen in de,.}{Amstelstraat in de richting}{van het Rembrandtplein}\\

\haiku{Betty kroop vast in,.}{den wagen waarvan de kap}{neergelaten was}\\

\haiku{Ze zou hem leelijk.}{op de teenen trappen voordat}{ze er bij neerviel}\\

\haiku{Ze trachtte af te,;}{leiden wat de anderen}{in hun schild voerden}\\

\haiku{Ze haalde den haan,,.}{nogmaals over en nogmaals vijf}{zes keer na elkaar}\\

\haiku{Als ze de kans krijgt,.}{geeft ze jouw en mijn portret}{aan de politie}\\

\haiku{Hij verwijderde;}{zich van den wagen in de}{richting van Haarlem}\\

\haiku{Ze stond op het punt,,.}{in te stappen toen ze een}{nieuwen inval kreeg}\\

\haiku{Hij  moest, had hij,,.}{Betty verklaard zelf zorgen}{dat het weer thuis kwam}\\

\haiku{{\textquoteright} {\textquoteleft}Ik zeg je toch al,,.}{dat jij het ook mag doen als}{je dat liever is}\\

\haiku{Want het maakt immers,!}{niets meer uit of je mij aan}{den dijk zet of niet}\\

\haiku{Ik heb mijn tijd noodig,.}{om jullie spoedig weer te}{kunnen ontmoeten}\\

\haiku{Enfin, dat zal je.}{zelf op kantoor ook wel eens}{meegemaakt hebben}\\

\haiku{{\textquoteleft}Ik weet alleen, dat,.}{hij geen rust zal hebben voor}{dat ding terecht is}\\

\haiku{Je vader is een,{\textquoteright}.}{reuzenbaas antwoordde ze}{op Betty's vraag}\\

\haiku{Ik heb toch zeker,.}{eerst aan haar gevraagd of zij}{Anna Barends is}\\

\haiku{Ze keek recht in de,.}{oogen van een man dien ze nog}{nimmer gezien had}\\

\haiku{Om zoo te zeggen.}{zat ik in de ijskast en}{jij in de broeikas}\\

\haiku{Ze liet de armen.}{zinken en w\'e\'er leek alles}{haar een hersenschim}\\

\haiku{Zeker, ik h\`eb een,.}{pas als je het met alle}{geweld weten wil}\\

\haiku{hij ging heen en sloot.}{de deur als gewoonlijk met}{sleutel en grendel}\\

\haiku{Het geluid van het.}{overschakelen overstemde}{dat van haar starter}\\

\haiku{Plotseling, na een,.}{paar minuten gedraafd te}{hebben stond ze stil}\\

\haiku{Hij stond stellig op?}{de treeplank en gaf wellicht}{laatste instructies}\\

\haiku{Als hij er niets van,?}{wist waarom maakte hij dan}{geen enkel geluid}\\

\haiku{{\textquoteleft}Maar je mag ze geen.}{meisjes laten ontvoeren}{en banken berooven}\\

\haiku{zie je nou, had ik,.}{dat toen geweten dan had}{ik hem mat gezet}\\

\haiku{{\textquoteright} vroeg Ferdinand, toen.}{hij zijn wagen gekeerd en}{op gang gebracht had}\\

\haiku{Ik heb een sleutel,,}{gehad maar die hebben ze}{me afgenomen}\\

\haiku{Juist, had ze dan zelf,?}{aan die kerels verteld waar}{het te vinden was}\\

\haiku{Dat leek me vreemd, ik,.}{dacht niet aan misdaad ik dacht}{aan een ongeluk}\\

\haiku{Het gaat tenslotte,?}{immers toch alleen om het}{verhaaltje nietwaar}\\

\haiku{Als hij er toe over,.}{te halen is zal het een}{bom duiten kosten}\\

\haiku{We zien elkaar nog.}{wel eens weer en we doen het}{ook nog wel eens over}\\

\haiku{Je bedoelt het goed,,,.}{Katrien maar heusch ik heb}{er geen vrede bij}\\

\haiku{Die gaat met je mee,.}{naar huis kijken of je nog}{wat te drinken hebt}\\

\haiku{{\textquoteright} {\textquoteleft}Vind jij het niet het,?}{eenvoudigst bij jou thuis}{te telefoneeren}\\

\haiku{Dat het een beetje,?}{erg gauw is gegaan dat is}{toch geen overtreding}\\

\haiku{Dan was je misschien,?}{in de buurt toen ik daar met}{de politie was}\\

\haiku{Ik vermoed, dat de.}{politie ze je dan wel}{zal overhandigen}\\

\haiku{{\textquoteright} Ouders maken zich.}{vaak zonder reden over hun}{kinderen bezorgd}\\

\haiku{Ze hadden gedacht,,.}{over twee weken ondertrouw}{over een maand trouwen}\\

\haiku{Vroeger maakten we,:}{dat alles aan huis zelf of}{met  een naaister}\\

\haiku{Ze namen deel aan.}{een groot aantal inbraken}{van den laatsten tijd}\\

\haiku{En een nader adres.}{hebben ze jou natuurlijk}{niet opgegeven}\\

\haiku{Misschien kom ik het,.}{u wel brengen als u dat}{goed vindt tenminste}\\

\haiku{{\textquoteright} {\textquoteleft}En m\'o\'oi gezegd,{\textquoteright} prees,.}{Andries zijn kaarten op een}{bundeltje kloppend}\\

\section{Hieronymus Sweerts}

\subsection{Uit: De tien vermakelijkheden van het huwelijk}

\haiku{Vervrolijk je naast}{de bruiloftsgasten zoveel}{de gelegenheid}\\

\haiku{Maar (och Schipper, breng).}{de Puts ik vrees dat het een}{heel kalf zal worden}\\

\haiku{, zich niet schamen je.}{een onfatsoenlijke som}{geld af te persen}\\

\haiku{Maar mij dunkt, ik zou.}{hier van de Juffrouw wel tot}{de meid vervallen}\\

\haiku{de uitvoering van.}{de plannen zal je immers}{geld genoeg kosten}\\

\haiku{En Juffrouw Perfect,.}{heeft ze beide wel maar die}{zijn juist uitgeleend}\\

\haiku{Want het kind uit het,.}{huis te doen kan Moeder noch}{Vader van het hart}\\

\haiku{{\textquoteleft}Och, ik zal, noch kan,}{niet verlossen tenzij ik}{het af zie snijden}\\

\haiku{Nauwelijks was de.}{man te bed of de vrouw kreeg}{een ander wezen}\\

\haiku{{\textquoteright} {\textquoteleft}Och Hartje,{\textquoteright} zei hij, {\textquoteleft}.}{je weet wat ik om jouw wil}{toegelaten heb}\\

\haiku{En omdat ik al,:}{diverse malen gedreigd}{had zeiden deze}\\

\haiku{Och, hoe zoet is de!}{Huwelijkse Staat boven}{de ongetrouwde}\\

\haiku{Het een lijkt haar te,:}{stijf of te gewoon zodat}{ze bij zichzelf zegt}\\

\haiku{En dit is nog het,,.}{kleinste maar wacht overmorgen}{gaat de Baker weg}\\

\haiku{De vrouw krijgt de schuld.}{van het weinig soepele}{huwelijksleven}\\

\haiku{Of dat werkelijk,.}{zo was zullen we wel nooit}{te weten komen}\\

\haiku{Het ging er vooral.}{om het familiebezit}{veilig te stellen}\\

\section{M.H. Sz\'ekely-Lulofs}

\subsection{Uit: De andere wereld}

\haiku{Wat verlegen en.}{met zichzelf geen raad wetend}{had hij daar gestaan}\\

\haiku{Hij zat op den rand.}{van de haverkist en zij}{stond tegen hem aan}\\

\haiku{Al was hij er maar,:}{loopjongen t\`och had hij toen}{het besef gehad}\\

\haiku{Altijd afgesnauwd,,!}{altijd weggestuurd altijd}{in een hoek gedouwd}\\

\haiku{{\textquoteleft}Zie je, hoe je elk,...?}{beest kunt winnen als je}{er maar goed voor bent}\\

\haiku{Een oogenblik had,.}{hij niet geweten wat zij}{daarmee bedoelde}\\

\haiku{Ze schrok, dat hij dit.}{spelletje merkte en keek}{dadelijk voor zich}\\

\haiku{Ik ga weg!-.}{Drie dagen later was hij}{op de Prinsengracht}\\

\haiku{Veel gezegd had die,.}{niet alleen hem de hand op}{den schouder gelegd}\\

\haiku{Eigenlijk had hij!}{dat tientje van Lien nog best}{kunnen gebruiken}\\

\haiku{En dan had moeder.}{haar zakdoek weer voor haar mond}{gepropt en gehuild}\\

\haiku{Een bad, zooals hij nog......}{nooit gezien had alleen maar}{in een \'etalage}\\

\haiku{het was, dat hij geen.}{plaats had kunnen krijgen op}{een Hollandsche boot}\\

\haiku{De envelop met.}{overgeschoten plakadressen}{kwam in zijn handen}\\

\haiku{Naast Van der Steeg stond.}{plotseling de silhouet}{van de verschansing}\\

\haiku{Het anker werd al.}{opgehaald en tegelijk}{loeide de stoomfluit}\\

\haiku{Een gevoel, dat met.}{geen enkelen weemoed om}{vertrek gemengd was}\\

\haiku{hem zoo duidelijk.}{de ongewenschtheid daarvan}{had laten voelen}\\

\haiku{Hij hield het hoofd wat.}{achterover en keek als in}{peinzing voor zich uit}\\

\haiku{{\textquoteleft}die zou Onkel von...}{M\"uhlock heel dankbaar zijn voor}{zijn goede zorgen}\\

\haiku{Zijn vrouw staat niet in...?}{de passagierslijst zou hij}{weer gescheiden zijn}\\

\haiku{Van der Steeg ging op:}{het eene eind zitten en zei}{wat vriendelijker}\\

\haiku{Hoorde hem een vloek.}{zeggen en direct daarop}{een deuntje fluiten}\\

\haiku{Zijn gedruktheid sloeg,.}{over in melancholie in}{een soort vaag heimwee}\\

\haiku{En dan doortrok hem:}{ook al de uitwaseming}{van dit nieuwe land}\\

\haiku{Hij keek wat verbaasd,,.}{op toen naast hem de loome}{klanklooze stem ophield}\\

\haiku{Hij wist zich ineens.}{staan in het volle licht en}{de volle aandacht}\\

\haiku{De Chinees nam de.}{boomen op en zette een}{sukkeldrafje in}\\

\haiku{De Chinees houdt zijn.}{open handpalm bij de lantaarn}{en ziet hem stom aan}\\

\haiku{En toch was de man.}{daar boven bijna tot aan}{het dak geklommen}\\

\haiku{Hij drukte zich wat,.}{vaster in de kussens die}{zijn rug omvlijden}\\

\haiku{Aan beide zijden,,.}{van de spoorbaan rees dat op}{donker duistergroen}\\

\haiku{Dezelfde wee\"e lucht.}{als van bij het station}{omgolfde hem weer}\\

\haiku{Hij had zijn stoel wat.}{achterover gewipt en stak}{een  sigaar op}\\

\haiku{Voorloopig hebt,.}{u zelf niets te denken te}{zeggen of te doen}\\

\haiku{{\textquoteleft}Ta-b\'eeh toe-w\'a\'an......}{Toewan wordt beleefd verzocht}{binnen te komen}\\

\haiku{{\textquoteleft}O, verduiveld, dat, ',!}{is waar ook gas even een}{eindje op zij Pot}\\

\haiku{{\textquoteleft}Daar moest eigenlijk,{\textquoteright},, {\textquoteleft}}{een lamp aan hangen begrijp}{je verklaarde hij}\\

\haiku{Je gaat denken, dat,:}{er  maar \'e\'en geluk maar}{\'e\'en doel voor je is}\\

\haiku{Anders zeg je 't, '.}{maar gerust hoor dan leg ik}{t je nog eens uit}\\

\haiku{Ze wil nooit gelooven,.}{dat je alle vuil van je}{schoenen kunt schrappen}\\

\haiku{Met een zwaai draaiden,.}{ze den tuin binnen kwamen}{met een ruk tot staan}\\

\haiku{Pieter keek even op,.}{in de bolle blauwe oogen}{met hun zachten blik}\\

\haiku{Verschool zich achter:}{een gewild jovialen}{lach en zei iets van}\\

\haiku{- Ga jij maar aan je,,}{eigen werk Blom ik neem dat}{sinkeh wel verder}\\

\haiku{- Met dien knul hebben '!}{ze me nou godbetert}{in \'e\'en hut gedouwd}\\

\haiku{{\textquoteright} En nu waren er,...}{niet alleen geen kerken maar}{ook geen Zondagen}\\

\haiku{En dan was er een.}{vaag verlangen in hem naar}{zoo'n Zondagsstemming}\\

\haiku{Bij de eerste de...}{beste gelegenheid moest}{hij dien zien te loozen}\\

\haiku{- En Pasman voelde:}{deze gedachte en werd}{nog dienstvaardiger}\\

\haiku{hij komt tot in de...}{tuin en daarom heeft Pasman}{niet graag hier gewaakt}\\

\haiku{Hij vluchtte naar bed,.}{binnen de bescherming van}{het tullen gordijn}\\

\haiku{In rijen zaten,.}{zij daar gehurkt bij elke}{ploeg stond de mandoer}\\

\haiku{Hij haalt uit, geeft den.}{koelie een onverwachten}{stomp in het gezicht}\\

\haiku{Dat laffe, zwarte!}{slaventuig durft immers toch}{niets terug te doen}\\

\haiku{de schuifelende,.}{haast sluipende stappen van}{Pasman's bloote voeten}\\

\haiku{Overal, waar ze liep,,,;}{hing haar zoetig parfum een}{beetje te zoet dof}\\

\haiku{Hij redde zich in,:}{het besef dat Van der Steeg}{hem bijgebracht had}\\

\haiku{- Dat was ook beter.}{en jonge toewans dronken}{meestal weinig}\\

\haiku{Ze maakte hem te...}{schande met haar brutale}{mond en haar tinka's}\\

\haiku{Alle bitterheid,.}{van vroeger kwam naar boven}{kwam wrang in zijn mond}\\

\haiku{{\textquoteleft}Ze zeggen ...dat ze.}{het houdt met de chauffeur van}{de toewan besar}\\

\haiku{Pasman moet hem maar{\textquoteright},.}{dood maken voegde ze er}{dan vol afkeer bij}\\

\haiku{Pasman moet er een,...}{hok voor timmeren ik wil}{hem levend houden}\\

\haiku{In de donkere,,.}{natte stilte kiemde zij}{op deze weemoed}\\

\haiku{Duizend dreigingen.}{hoonlachten in het gesnerp}{van een cicade}\\

\haiku{de rooie sufferd wel}{te voorschijn als jullie baas}{en meerdere.-}\\

\haiku{- Voordat Asminah,:}{er was had hij nog wel eens}{verteld over Pasman}\\

\haiku{En voor Asminah,.}{waren Pieter's ouders broer}{en zuster Blanken}\\

\haiku{Blanken, ergens in,.}{een ver land dat ze zich niet}{eens voorstellen kon}\\

\haiku{Als ik altijd op,.}{mijn mooist ben dan ziet toewan}{nooit meer het verschil}\\

\haiku{Had Asminah niet?}{iets gezegd van nog een kast}{in de slaapkamer}\\

\haiku{Ze wisselde voor,.}{het laatst de knoopen in zijn}{schoone witte jas}\\

\haiku{In die halve maand.}{zouden ze het huis koopen}{en zich installeeren}\\

\haiku{En die lamstraal van...... {\textquoteleft}}{een Idris wat of Minah toch}{met dien vent opheeft}\\

\haiku{Had heel wat aan dek... -.}{moeten slapen Plotseling}{werd hij weer somber}\\

\haiku{En die kletspraatjes...,...}{over Dinah God mag weten}{wat hij binnen kreeg}\\

\haiku{Toch drong zich door zijn.}{wrevel heen een poging tot}{rechtvaardig blijven}\\

\haiku{{\textquoteleft}En weet je, waarom?}{ik je nou toch de heele}{maand loon uitbetaal}\\

\haiku{Ik zal ook altijd:}{van jou houden en ik vraag}{maar \'e\'en ding van je}\\

\haiku{{\textquoteright} Het juffrouwtje, dat,:}{dien dag haar slechten dag had}{snauwde venijnig}\\

\haiku{Hij ging naar binnen,.}{ging op de bank liggen en}{vouwde de krant open}\\

\haiku{Och{\textquoteright}, zei hij, het van, {\textquoteleft}?}{zich afschuivendben je daar}{nou wel zeker van}\\

\haiku{Hij dacht alleen aan,,.}{dat vreemde dat verre dat}{onwezenlijke}\\

\haiku{Ze ging hem voor en.}{toen stond hij naast haar in de}{leege logeerkamer}\\

\haiku{{\textquoteright} En ze wendde zich,.}{iets om naar een bundeltje}{dat achter haar lag}\\

\haiku{Maar in Pieter was,.}{een vreemde weeke ontroering}{geweest bij dat woord}\\

\haiku{Hij drukte het kind.}{tegen zich aan en streelde}{het over zijn kopje}\\

\haiku{Doeltje was heer en.}{meester in den tuin en in}{de bijgebouwen}\\

\haiku{doodstil, haar blik strak.}{starend op het slapende}{kindergezichtje}\\

\haiku{Ging hij dit kind hier?...}{achterlaten om terug}{te gaan naar Holland}\\

\haiku{Het was het oude,.}{gebrek aan zelfvertrouwen}{de oude ziekte}\\

\haiku{Ze voelden beiden,.}{de maatschappelijke kloof}{die tusschen hen lag}\\

\haiku{{\textquoteright} zei hij opgewekt, {\textquoteleft},!}{de meeste lui trouwen als}{ze met verlof zijn}\\

\haiku{{\textquoteright} ~ Dien avond, toen hij,.}{de leeszaal binnenkwam bleef}{hij plotseling staan}\\

\haiku{Wandversiering was.}{niet inbegrepen geweest}{bij den inboedel}\\

\haiku{Zelfs Betty... Ja, wat!}{w\'as ze dadelijk aardig}{tegen hem geweest}\\

\haiku{Hij gaf zijn hoed en.}{stok aan den toegeschoten}{boy en ging zitten}\\

\haiku{Ze zoende hem op,, {\textquoteleft}...}{zijn mond met ondeugende}{speelschheiden z\'o\'o}\\

\haiku{- Samen slapen is, -.}{ouderwetsch en proza{\"\i}sch}{had Betty gezegd}\\

\haiku{En we weten het.}{alle drie en toch zullen}{we het loochenen}\\

\haiku{{\textquoteleft}Ik ken een beetje{\textquoteright},, {\textquoteleft}}{Fransch zei Bettynog van mijn}{tijd bij madame}\\

\haiku{En het huis is ook,...}{oud en leelijk maar ik ben}{er toch geboren}\\

\haiku{{\textquoteright} Betty stond naast de.}{mannen en hoorde Van Beek}{verslag uitbrengen}\\

\haiku{Geen van beiden zei,.}{dat en toch wisten beiden}{wat de ander dacht}\\

\haiku{{\textquoteleft}Als u d\'at voor me...{\textquoteright}.}{doen wou Een dankbare blik}{schoot uit Van Beek's oogen}\\

\haiku{Die komedie, of....}{wat het dan ook was daar moest}{een eind aan komen}\\

\haiku{Verleidelijk werd,.}{de blanke boog van haar buik}{vlak voor zijn gezicht}\\

\haiku{De boy keek haar na,,.}{keek even naar haar hoed dien ze}{nog altijd op had}\\

\haiku{{\textquoteright} zei Betty peinzend, {\textquoteleft},,...{\textquoteright}}{alles wat ni\'et mag dat}{is pas \'echt prettig}\\

\haiku{dit leven, dat het,.}{bestaan was voor een ander}{ras een ander volk}\\

\haiku{Hij worstelde met,.}{de knoopjes van zijn overhemd}{met zijn boord en das}\\

\haiku{provinciaal... kan,,?}{ik helpen dat die rotvent}{niet weet waar hij staat}\\

\haiku{- Wat zouden jullie,...}{wel zeggen van Doeltje als}{jullie dat wisten}\\

\haiku{En pijnlijk voelde,:}{hij daarin dat er toch een}{kloof gebleven was}\\

\haiku{Pieter's verhaal was,.}{afgebroken hij vond zoo}{gauw het vervolg niet}\\

\haiku{Hij herinnerde,.}{zich ineens dat hij die doos}{bonbons had gekocht}\\

\haiku{Nou op je geluk,...{\textquoteright} {\textquoteleft}...}{en welkom thuis welkom in}{HollandMet je vrouw}\\

\haiku{Dat heb je nog niet,...{\textquoteright}.}{eens verteld Pieter Pieter}{deed onverschillig}\\

\haiku{van stoere werkkracht,,.}{van stroeve berusting van}{harde soberheid}\\

\haiku{Je ouders en je,!}{broer een troep armoedzaaiers}{in een achterbuurt}\\

\haiku{{\textquoteright} benijdde Betty, {\textquoteleft},{\textquoteright}.}{hemw\'at een geluk om d\'at}{te kunnen zeggen}\\

\haiku{Hij is niet heelem\'a\'al,,.}{zoo'n lammeling weet je als}{de menschen denken}\\

\haiku{- Maak er nou liever,....}{een eind aan Betty laat het}{niet te lang duren}\\

\haiku{En ze was van hem,,.}{gaan houden met een vreemde}{warme sympathie}\\

\haiku{Veenstra sloot zich bij hen,.}{aan had altijd het een of}{ander leuke plan}\\

\haiku{Het was, of ze zich,.}{niet realiseerde dat}{dit hun toekomst werd}\\

\haiku{{\textquoteright} Waarom lachten nou,.}{de anderen bij elke}{vraag van Willemse}\\

\haiku{Waarom moest hij zich,?}{dat op zijn hals halen zoo'n}{beroerd kwartiertje}\\

\haiku{Dan bespraken ze.}{de krantenberichten en}{nieuwtjes uit Indi\"e}\\

\haiku{En altijd was er,.}{boven dat werk haar stille}{zwijgende glimlach}\\

\haiku{Hij verdedigde,,.}{Betty met een woord een lach}{een schouderophaal}\\

\haiku{Het was hem, of een.}{afgrond plotseling aan zijn}{voeten opengaapte}\\

\haiku{Hij zat daar met het.}{bonnetje in zijn hand en}{keek naar het plakkaat}\\

\haiku{Van Brinkman... - Brinkman... -.}{heeft gezegd Hij scheurde de}{enveloppe open}\\

\haiku{Hij voelde, tusschen,,.}{zijn teenen het nat dat door zijn}{schoenzolen heen trok}\\

\haiku{Op hooge, wankele.}{hakken drentelden ze over}{het slechte trottoir}\\

\haiku{De vrouw morrelde.}{dan aan een sleutelgat en}{er ging een deur open}\\

\haiku{En hij zag Betty,,.}{staan voor den spiegel met de}{blauwzijden nachtpon}\\

\haiku{Af en toe rukte,.}{een windstoot aan de ruiten}{die even rammelden}\\

\haiku{nou hoefde niemand,...}{van haar te zeggen dat ze}{een slechte vrouw was}\\

\haiku{Hij trok het dek over.}{zich heen en op hetzelfde}{oogenblik sliep hij}\\

\haiku{Er was een afgrond,,.}{een kloof die nergens  meer}{te overbruggen was}\\

\haiku{Zoo landde hij in,.}{de kleine heete haven}{en reed naar de stad}\\

\haiku{{\textquoteleft}En... zal toewan niet,...{\textquoteright} {\textquoteleft},.}{erg all\'e\'en zijn zonder de}{mimNee Asminah}\\

\haiku{Onder het stof en.}{vuil was haast niets meer te zien}{van de teekening}\\

\haiku{En dan liepen ze,,.}{den langen weg die vroeger}{zoo kort was terug}\\

\haiku{Even liepen ze hard,,.}{maar de bui haalde hen in}{kletste op hen neer}\\

\haiku{Er is niemand in,,.}{het dorp die zijn naam kent maar}{dat is ook niet noodig}\\

\haiku{In de bocht lag het,.}{strand het brandend heete zand}{glanzend in het licht}\\

\haiku{Meubels waren er,,:}{niet m\'e\'er dan Asminah had}{toen hij terug kwam}\\

\haiku{En dan dwaalden zijn,.}{oogen over het papier dat ze}{weggeworpen had}\\

\haiku{Elken dag had hij.}{een sarong en een baadje}{daar weggenomen}\\

\haiku{Van Beek was ook een... -.}{goeie vriend Hij liet de krant van}{zijn knie\"en glijden}\\

\haiku{{\textquoteleft}Ik zal eens kijken.}{in de goederenloods van}{de onderneming}\\

\haiku{Hij hoorde het heen.}{en weer geloop en gesjouw}{met de manden visch}\\

\haiku{Er was ergens een,,...}{troost die op hem neerdaalde}{over hem heen stulpte}\\

\subsection{Uit: Het laatste bedrijf}

\haiku{Een kleurige en,,.}{geurige oud vertrouwde}{concrete basis}\\

\haiku{{\textquoteleft}Kunt u me misschien,?}{ook zeggen hoe lang de zaak}{van Ger\"o leeg staat}\\

\haiku{{\textquoteright} (Natuurlijk niet, - dacht).}{hij en veranderde zijn}{vraag onmiddellijk}\\

\haiku{jij zat daar maar en,....}{schilderde maar en je dacht}{het komt wel terecht}\\

\haiku{{\textquoteright} George hield de,.}{foto dichter bij het raam}{het werd al  avond}\\

\haiku{{\textquoteright} George trok met,.}{zijn schouders stak zijn handen}{in zijn broekzakken}\\

\haiku{Het werd drukkend stil,.}{in de kamer die niet aan}{den gevelkant lag}\\

\haiku{Je kunt niet eens zien,...{\textquoteright} {\textquoteleft}?}{dat we broersWaarom is hij}{teruggekomen}\\

\haiku{Dat je zo\'omaar kunt?}{overgaan van de eene branche}{in de andere}\\

\haiku{Ik heb nog nooit zoo'n!}{onpractisch en angstvallig}{mensch gezien als jij}\\

\haiku{Ze zweeg dan ook en,.}{samen stonden ze voor het}{raam in donker}\\

\haiku{Toen schikte ze met:}{luchtige gebaartjes nog}{iets aan haar kapsel}\\

\haiku{Zijn vrouw zat in een.}{klein kamertje aan tafel}{de krant te lezen}\\

\haiku{Ferri ontving hem,,, - {\textquoteleft}!}{hartelijk maar een beetje}{te druk nerveusAh}\\

\haiku{Toen hij nog jong was,.}{nog geloofde aan de Groote}{Kunst en aan zichzelf}\\

\haiku{- Ferri heeft gelijk,,,.}{jong is ze niet meer ze is}{rijp maar nog bloeiend}\\

\haiku{{\textquoteright} Ze verdween en kwam.}{terug en hielp Bella bij}{het tafeldekken}\\

\haiku{En dat niet als een,,.}{leege troostelooze weemoed maar}{als realiteit}\\

\haiku{- Ik ben toch thuis.... - dacht, -,....}{hij het was toch goed dat ik}{thuisgekomen ben}\\

\haiku{Hij liep alleen over,,.}{straat zijn overjas los zijn hoed}{juist i\'ets te scheef op}\\

\haiku{Hij had er een glas.}{wijn bovenop gedronken}{en verder gedanst}\\

\haiku{Een man voor wien het, - -.}{geld om het uit te geven}{al niet meer telde}\\

\haiku{George liep nu.}{door een plantsoentje en hij}{dacht aan zijn moeder}\\

\haiku{Hij was al niet meer.}{wat zijn vader was geweest}{en zijn grootvader}\\

\haiku{In dezen eenen nacht.}{had de wereld een ander}{aanschijn gekregen}\\

\haiku{hij was precies zo\'o,.}{ook zoo zenuwachtig en}{geschrokken en bang}\\

\haiku{Iets uit Parijs, iets.}{bijzonders in onderwerp}{of behandeling}\\

\haiku{Jij hoeft je over niets,.}{zorgen te maken je met}{niets te bemoeien}\\

\haiku{Toen drukte ze hem.}{nog vaster tegen zich aan}{en zoende zijn mond}\\

\haiku{Ineens herdacht hij,,....}{den avond van gisteren die}{kamer die menschen}\\

\haiku{{\textquoteright} {\textquoteleft}Ik moet toch wat geld....}{hebben om de nieuwe zaak}{mee te beginnen}\\

\haiku{Of, als je wilt, kun....}{je ook direct komen als}{je aangekleed bent}\\

\haiku{Ook daarbij waren.}{de woorden oplichter en}{zwendel gevallen}\\

\haiku{Ze stuurt elke maand,....}{bijna haar heele loon naar}{haar ouders Bella}\\

\haiku{Ferri luisterde.}{naar dat alles en stemde}{toe zonder een woord}\\

\haiku{Een zwak licht, dat nooit, -, -.}{meer hij wist het heelemaal}{helder zou worden}\\

\haiku{Hij was dankbaar voor,.}{dit schijnsel hij was zoo lang}{in donker geweest}\\

\haiku{Waren ze binnen,}{een half jaar niet betaald dan}{zou de winkelier}\\

\haiku{Dat suffe kind was!}{nog te stom om behoorlijk}{te telefoneeren}\\

\haiku{Ze liepen naar het,.}{restaurant waar ze Bella}{zouden ontmoeten}\\

\haiku{misschien maak je af,....}{en toe iets iets h\'e\'el moois en}{dat verkoop je niet}\\

\haiku{George keek haar.... - {\textquoteleft}}{argwanend aan Hoe wist ze}{al die dingen?-}\\

\haiku{- George is toch,.}{een heel wat knapper kerel}{dan Ferri dacht ze}\\

\haiku{Hakkelend en met:}{neergeslagen oogen begon}{hij aan een uitleg}\\

\haiku{- Iemand vond Susanne,,.}{te naakt maar een ander zei}{dat dit echt Fransch was}\\

\haiku{het was zijn tijd om.}{op te treden in Bella's}{nieuwen kunsthandel}\\

\haiku{Dan zou ik het eerst....}{jaren op de bank moeten}{zetten en god weet}\\

\haiku{Een goed schilderij,!}{dat kunstwaarde heeft is even}{goed als een effect}\\

\haiku{Gods verwarmend licht '!....}{in de duisternissen van}{s menschen jammer}\\

\haiku{En dan gaan we voor.}{een kop koffie en een glas}{fijne wijn naar Ritz}\\

\haiku{Ze liet haar groote oogen,.}{op hem rusten klaar om zijn}{blik op te vangen}\\

\haiku{Ze bleef zitten, zooals,,.}{ze zat op haar knie\"en met}{haar beenen onder zich}\\

\haiku{- Ze wiegde zich op....}{deze voorstelling van het}{eindelooze geluk}\\

\haiku{Ze wendde langzaam,.}{haar warme bruine oogen naar}{hem en glimlachte}\\

\haiku{wij zijn alledrie,,.}{op onze eigen manier}{schipbreukelingen}\\

\haiku{Ze wiegde zacht heen.}{en weer en wiegde hem mee}{in haar omarming}\\

\haiku{Kunst is wel mooi en,.}{verheven maar leven moet}{je tenslotte ook}\\

\haiku{Ze keken beiden.}{verschrikt naar de dichte deur}{en toen naar elkaar}\\

\haiku{Hij keek maar even om,,.}{over zijn schouder toen groef hij}{weer in zijn jaszak}\\

\haiku{{\textquoteright} {\textquoteleft}Niemand....{\textquoteright} George,...}{at zwijgend hij kon niet zoo}{gewoon meepraten}\\

\haiku{Kom dan tegen een,.}{uur of vier bij me misschien}{is er wat te doen}\\

\haiku{Een groot gevoel van.}{kameraadschap tegenover}{haar vervulde hem}\\

\haiku{En als je ergens,,.}{geluk ziet dan moet je het}{maar grijpen Bella}\\

\haiku{En dan te denken,!}{dat zoo'n jongen misschien nog}{een moeder heeft o\'ok}\\

\haiku{Hij zag hem al van.}{den hoek en zijn oogen knepen}{zich turend samen}\\

\haiku{Hij keek dreigend rond,,.}{en haatte de menschen de}{wereld het leven}\\

\haiku{Ik zal u nu even,.}{uitleggen wat ik bedoeld}{heb als onderwerp}\\

\haiku{{\textquoteright} riep Bella uit, {\textquoteleft}als!}{je een artiest maar als een}{artiest behandelt}\\

\haiku{Je zult zien, je maakt,.}{er iets moois van en ik zorg}{wel dat hij het neemt}\\

\haiku{{\textquoteright} vroeg hij benepen,.}{met een schuwen blik naar zijn}{pas begonnen doek}\\

\haiku{Maar een mensch schiet zich,!}{niet zoo gemakkelijk voor}{zijn kop George}\\

\haiku{Ik zou niets liever,.}{willen dan jou tevreden}{stellen George}\\

\haiku{Terwijl hij op den,.}{concierge wachtte nam hij}{de omgeving op}\\

\haiku{- Ze wonen te duur, -,.}{stelde hij vast denkend aan}{den leegen kunsthandel}\\

\haiku{Enfin dat is nu,,,!}{eenmaal zoo bij ons h\`e nu}{eens vloed dan weer eb}\\

\haiku{{\textquoteleft}Ik zou alleen eerst,.}{even willen weten wat dat}{te beteekenen heeft}\\

\haiku{{\textquoteright} George boog zijn:}{hoofd naar de lucifervlam}{tusschen zijn handen}\\

\haiku{Je vroeg me, wat ik,,,....}{bedoelde met jouw te groote}{handigheid nou kijk}\\

\haiku{Maar ik heb nog een,.}{andere plicht ook waar jij}{misschien niet van wist}\\

\haiku{Dacht je, dat ik maar?!}{altijd en altijd voort kan}{en van ijzer ben}\\

\haiku{Toe zeg, ga eens even,....}{een eindje op zij je staat}{precies in het licht}\\

\haiku{{\textquoteright} fluisterde ze heet, - {\textquoteleft}!}{en verontwaardigdik doe}{het toch ook voor h\`em}\\

\haiku{Als je dan weer gaat,.}{kijken ligt er alleen nog}{maar een plasje vocht}\\

\haiku{Hij stikte in zijn.}{woorden en haalde diep adem}{om lucht te krijgen}\\

\haiku{- {\textquoteleft}Dacht je, dat alleen?}{jij het monopolie hebt}{op kwaje buien}\\

\haiku{dus niet een van die,....}{dingen die ik alleen maar}{in commissie heb}\\

\haiku{Ze lachte schel en.}{zenuwachtig en gooide}{haar hoofd in den nek}\\

\haiku{De beide paarden,;}{lieten hun kop neerhangen}{hun knie\"en knikken}\\

\haiku{Hadden we het niet?}{veel beter met ons twee\"en}{alleen ingericht}\\

\haiku{George keek hem,:}{over zijn krant heen aan oogde}{over de straat en zei}\\

\haiku{Nu een beetje de,.}{Oostenrijksche Alpen in}{dat zou niet kwaad zijn}\\

\haiku{Ze rekenden met.}{den ober af en hielden iets}{meer dan een peng\"o over}\\

\haiku{Hij zag de wet van:}{de natuur en daarin de}{goedertierenheid}\\

\haiku{Ik geloof, dat ze.}{meer gezond verstand heeft dan}{de meeste vrouwen}\\

\haiku{L\'a\'at ze snertdingen!....}{koopen als ze dan toch het}{goede niet willen}\\

\haiku{Ik ben niet in de...}{gelegenheid gesteld om}{de kunst te dienen}\\

\haiku{als George eens....}{werkelijk niet meer om zijn}{kunst of de Kunst gaf}\\

\haiku{Nee, leven moesten ze,....}{van de anderen daarin}{had Bella gelijk}\\

\haiku{Norsch groette hij....}{hen en gaf onwillig den}{sleutel voor de lift}\\
