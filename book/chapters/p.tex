\chapter[13 auteurs, 973 haiku's]{dertien auteurs, negenhonderddrieënzeventig haiku's}

\section{Willem Paap}

\subsection{Uit: De doodsklok van het Damrak}

\haiku{Hy  zal sneller,.}{loopen opdat hy gauw aan}{de Rozengracht is}\\

\haiku{De storm, dien hy recht,;}{in het gezicht had gehad}{woei nu van terzy}\\

\haiku{Verbeeld ik my dat,?}{of ziet u er niet zoo}{goed uit als anders}\\

\haiku{Wie wat, ervaring,!}{in het leven heeft zal dat}{niet tegenspreken}\\

\haiku{dan zouden ze eerst.}{maar eens de overjassen aan}{den stander hangen}\\

\haiku{Voorheen hadden zy,.}{steeds dokter Arends gehad maar}{die was overleden}\\

\haiku{- Jy doet hier altyd;}{je best om je klantjes by}{mekaar te houden}\\

\haiku{die verdobbelt z'n,.}{geld aan de beurs maar aan z'n}{huizen doet hy niets}\\

\haiku{Kyk jy eens in die,,.}{kast en jy op de kast ik}{zelf kyk er onder}\\

\haiku{Corbelyn viel snel,,.}{op zyn boterham aan want}{hy had haast groote haast}\\

\haiku{Baby had altyd.}{over Emmy en Corry het}{moedertje gespeeld}\\

\haiku{de voorkamer, die,;}{salon werd genoemd met drie}{ramen op de gracht}\\

\haiku{Doch reeds kwam Thilde:}{achter in de gang uit de}{huiskamer en riep}\\

\haiku{- Wel beste meid, maar.}{gauw je mantel af en maar}{gauw in de kamer}\\

\haiku{Laat je bedienen, '.}{door de dames waar jes}{avonds je tyd doorbrengt}\\

\haiku{Want de naam doet er,;}{veel toe by een papier dat}{je aan de markt brengt}\\

\haiku{- Nou, hy dankte de.}{heeren voorloopig al}{wel voor de moeite}\\

\haiku{Als Schibaieffs van,.}{middag niet aan de beurs zyn}{zorg dan voor een koers}\\

\haiku{Hy sluit, gelyk men,.}{dat noemt de orders van zyn}{klanten in mekaar}\\

\haiku{De Waert enz., op de,.}{kantoren waar ze gekocht}{hebben gezegd wordt}\\

\haiku{Nu, dat was morgen,.}{verkoopen en wat er te}{kort kwam bypassen}\\

\haiku{- Want als je wilt, had,.}{hy gezegd kan je dat niet}{veel moeite kosten}\\

\haiku{je zelf zou het niet,;}{hebben gewaagd die dingen}{te laten maken}\\

\haiku{wat hy daarover zoo,,,.}{eens vertelde was gelyk}{spreekt slechts oppervlak}\\

\haiku{ik kan me dat best -.}{begrypen toen heeft hy de}{tweede genomen}\\

\haiku{het zonnetje schynt,?}{zoo lekker buiten maar als}{het u te koud is}\\

\haiku{Want 'n meisje zoo,.}{haar leven lang alleen dat}{was niet het rechte}\\

\haiku{Een jongen van een,,:}{veertien jaar kwam haar tegen}{liep dicht langs haar zei}\\

\haiku{maar je kunt toch wel,.}{zoo nu en dan doen alsof}{zy aanwezig is}\\

\haiku{- Ja, m'nheer! - En heb,?}{ik het mis of leven je}{beide ouders nog}\\

\haiku{want Alida lag een.}{verdieping hooger op de}{slaapkamer te bed}\\

\haiku{God, het is zoo gek,,.}{zei-d-i maar daar leef je}{heelemaal van op}\\

\haiku{Voor het eerst in zyn.}{leven zat hy op de weeke}{kussens der weelde}\\

\haiku{Nu, zoo vlug zullen.}{we met de bankjes toch maar}{niet om ons gooien}\\

\haiku{dat zou-d-i wel, ';}{eens willen zien dat-i}{t niet terug kreeg}\\

\haiku{Want als je doorgaat,, ';}{dat is bekend genoeg dan}{wordt hetn hartstocht}\\

\haiku{daar komt de stroom van,;}{beursheeren de Kalverstraat}{uit het Rokin af}\\

\haiku{Destyds woonde in.}{het huis van Dries Corbelyn}{de vreugde niet meer}\\

\haiku{die Unions, die nu,;}{124 staan stonden toch nog geen}{jaar geleden 190}\\

\haiku{De gongslag in de.}{rotonde verkondigde}{het aangaan der beurs}\\

\haiku{Het gemurmel wordt.}{stil daar by den hoek van de}{Santa Parulla's}\\

\haiku{zy duizelde, zy,,.}{kon niet over dien weg die daar}{zoo breed zoo hol was}\\

\haiku{En zy grepen haar.}{by den arm en gingen met}{haar terug naar huis}\\

\haiku{En dat, terwyl hy.}{het zoo benauwd had met dat}{kloppen van het hart}\\

\haiku{'t Was hem, of hy,,.}{weg moest van elke plaats waar}{hy zat waar hy stond}\\

\haiku{Pas op, niets merken,,!}{laten hier in huis houd je}{goed houd je kranig}\\

\haiku{Nou de Kimberleys,!}{niks meer waard zyn ben je naar}{de verdommenis}\\

\haiku{Waarom had zyn vrouw?}{hem aangepord om aan de}{beurs te speculeeren}\\

\haiku{En Geurtje bracht het.}{theewater en zette het}{theeblad op tafel}\\

\haiku{En Dries Corbelyn.}{liep wild heen en weer als een}{roofdier in zyn kooi}\\

\haiku{want, duivels, het had, '.}{hem toch aangegrepen hy}{wasn beetje moe}\\

\haiku{Iets vroolyker was het.}{in de woonkamer van Arie}{Zuydam en Alida}\\

\haiku{Zyn advocaat had,,,}{hem gezegd dat nu hy toch}{niets van haar hoorde}\\

\subsection{Uit: Vincent Haman}

\haiku{In der Beschr\"ankung,:}{zeigt sich der Meister en}{zei niet 			 Hegel}\\

\haiku{Kyken hoe iets in, '.}{elkander 			 zit moet hem}{n pleizier worden}\\

\haiku{- Wel, als het korter,;}{was dan zond ik het aan de}{Amstelbode}\\

\haiku{{\textquoteleft}'t Is heel knap van{\textquoteright},:}{zoo'n kind en oom 			 Gabriel}{Haman beweerde}\\

\haiku{de kranten, 			 Buckle.}{en de brochure van een}{dubbeltje}\\

\haiku{Hy heeft je stuk 			 .}{gelezen en zegt dat er}{goeie zinnen in zyn}\\

\haiku{Dit nu is zeer mooi,}{maar een auteur die niets te}{zeggen heeft dan}\\

\haiku{'k Weet immers wel,.}{dat u die niet voor me}{zou willen koopen}\\

\haiku{Met een sarrige.}{treiterlust zat hy het werk}{te concipieeren}\\

\haiku{Haar 			 vader vond ', '.}{hemn deugniet dus was het}{n flinke jongen}\\

\haiku{hy zou haar in z'n,}{proza precies volgen de}{houtsnee alleen wat}\\

\haiku{Esther lachte 			 :}{haar blymoedigen lach op}{het opene gelaat}\\

\haiku{Hy streek zich met de,.}{linkerhand over het voorhoofd}{ging zwygend 			 mee}\\

\haiku{Wacht, die brief in de,...}{zak 			 steken mama kon}{eens binnen komen}\\

\haiku{En 			 bovendien,,.}{och zeuren hoe kon ze over}{die drift zaniken}\\

\haiku{Marie zei dat er,;}{iets heel byzonders moest zyn}{w\'at wist ze 			 niet}\\

\haiku{Maar er kwam geen brief,,;}{dien wachtenden dag niet}{den volgenden niet}\\

\haiku{Zy had 'n rytuig '.}{genomen enn krans}{gebracht naar het graf}\\

\haiku{'t Was een achttal,.}{jaren geleden nu dat}{Vincent gegaan was}\\

\haiku{Een jaar later was,.}{haar moeder gestorven kort}{daarna haar vader}\\

\haiku{zoo teer als ik die,.}{tonen 			 wilde hebben}{kryg ik ze toch niet}\\

\haiku{Hoog en rank op haar.}{elegante Columbia zat}{de slanke Esther}\\

\haiku{die 's altyd zoo -;}{zwaar op de hand. Schimp nou niet}{meer zoo op Reinhold}\\

\haiku{- Kom, zei Esther, een,.}{weinig geschokkeerd laten}{we naar huis 			 gaan}\\

\haiku{Zy waren Halfweg,,}{nu lang voorby gingen den}{tol door en v\'oor}\\

\haiku{Esther moest iets doen,.}{aan de ellende die zy}{daar voor zich zag}\\

\haiku{Zy waren zwygend,.}{gegaan en ook zittende}{zwegen zy een wyl}\\

\haiku{Een notaris uit;}{Den Haag stuurde 			 wel eens}{geld per postwissel}\\

\haiku{Maar 'n paar maanden ';}{later was er 			 weern}{vonnis gekomen}\\

\haiku{{\textquoteleft}God, Puck wat wor je{\textquoteright},.}{zwaar zette 			 hem op het}{laken van het bed}\\

\haiku{Vroeger kwam Van den,,.}{Berg er veel 			 later Van}{Wheele nu Jules}\\

\haiku{In eens bleven ze,}{dan gewoonlyk weg zonder}{dat je 			 hoorde}\\

\haiku{Esther slank in haar,,.}{eenvoudig grys Marie}{kleiner wat dikjes}\\

\haiku{Wat ik je vragen,,?}{wou 			 Vincent heb je nog}{over het tydschrift gedacht}\\

\haiku{Witsen wist altyd,;}{precies het moment te}{pakken mooi van toon}\\

\haiku{'t is half tien, en.}{ze hadden hier 			 om acht}{uur zullen wezen}\\

\haiku{Wat spyt me dat, wat,...}{spyt me dat dat hy daar nu}{weer mee 			 begint}\\

\haiku{Nu gaat ook het licht;}{van dat hooge raam daar ver aan}{den overkant 			 uit}\\

\haiku{Als ik wist dat hy,...}{alleen was om aan my te}{kunnen denken}\\

\haiku{Beroerd idee, ik 			  '....}{wilt niet meer hebben Laat}{ik wat rondkyken}\\

\haiku{Vincent, onder die,;}{woorden was een weinig in}{pose gekomen}\\

\haiku{Ik hou van 'r. Zy,,.}{heeft mooie oogen zoo raar grys}{grauw zou je zeggen}\\

\haiku{Toen hy een wyle,,.}{gelezen had stond hy op}{liep in de kamer}\\

\haiku{October ging heen,.}{November kilde in de}{gangen der huizen}\\

\haiku{Esther, dien zomer,,.}{en herfst was wat nerveus zooals}{zy het noemde}\\

\haiku{zy had hem 			 niet.}{moeten ontvangen daar in}{dat schemerend uur}\\

\haiku{zei weer Marie, en,}{lei de 			 hand op den}{schouder van Esther}\\

\haiku{wat 			 zat ze nu ':}{hier byt ontbyt haar tyd}{te verbeuzelen}\\

\haiku{Dat was voor Esther,,.}{by haar schilderen 			 weer}{een heele drukte}\\

\haiku{'k 			 Heb vandaag,.}{behoorlyk gewerkt en mag}{nu wel wat praten}\\

\haiku{Daar had-i nou z\'oo,.}{op gehoopt dat-i weer}{zou gaan 			 schryven}\\

\haiku{Vincent vond het w\'at.}{prettig dat er zoo over hem}{geschreven 			 werd}\\

\haiku{Wel, dat 			 zou iets, '.}{nieuws zyn zoo eens iets uitn}{heel andere sfeer}\\

\haiku{Maar 			 nu, 't was '.}{enorm zoo veel als-i soms schreef}{opn voormiddag}\\

\haiku{Maar de man had nu '.}{werkelykn 			 heel goed}{inzicht in de zaak}\\

\haiku{Die heftigheid van, ',;}{ons van vroeger dat wast}{jonge 			 bloed oom}\\

\haiku{Wel, dat vond Vincent ', '.}{n belangryk ideen zeer}{belangryk idee}\\

\haiku{En 'n predikant, ', ';}{ook inn toga staat heel}{goedn 		  toga}\\

\haiku{- Och, je weet toch wel,.}{dat ik daar tegenwoordig}{zoo 			 niet meer kom}\\

\haiku{zou   moeten leeren,...}{zou het by hem toch ook zoo}{vlug niet meer 		  gaan}\\

\haiku{Ze moest niet denken,,;}{had-i gezeid dat het niet}{precies waar was}\\

\haiku{Zoo zaten zy daar.}{vredig by den komenden}{September-avond}\\

\haiku{'t Was wel geen slot,.}{maar het heette dan toch}{het   Vondelhuis}\\

\haiku{Pag. 170, regel 14-,.}{21 is citaat uit diens}{Huysman's L\`a-Bas}\\

\section{Gerrit Paape}

\subsection{Uit: Reize door het aapenland (onder pseudoniem J.A. Schasz)}

\haiku{Bijgekomen, duwt.}{hij haar met kracht van zich af}{en vlucht ijlings weg}\\

\haiku{Voorlopig staan de '.}{papieren van Pietert}{Hoen dus nog het sterkst}\\

\haiku{Had hij intussen?}{zijn bekomst gekregen van}{de revolutie}\\

\haiku{De reiziger leert.}{de apentaal en loopt ook op}{handen en voeten}\\

\haiku{In drie bedrijven,,,,,.}{door J.A. Schasz M.D. Utrecht bij G.T.}{van Paddenburg 1779}\\

\haiku{hoe is het mooglijk,?}{dat ik   mijn arme Paard}{kan vergeeten}\\

\haiku{en heeft het zelf wel,?}{ooit gedagten gemaakt om}{op hol te   gaan}\\

\haiku{Dan zou het er slegt,}{met hem uitzien wanneer hij}{gekreegen werd?129}\\

\haiku{- en gelust ons den,;}{Aapenstaat   dan is de staart}{volstrekt on\"ontbeerlijk}\\

\haiku{In 't eerst, werd het.}{geval wonderbaarlijk door}{een gehaspeld}\\

\haiku{\'e\'en ten hoogsten, doch.}{keurde dat van nommer}{vijf ten sterksten af}\\

\haiku{Gij moet er door uw!.}{eigen oogen van overtuigd}{worden zei hij}\\

\haiku{dagt ik, deeze zijn!}{gekomen,232  om mijn Staart}{aftehakken}\\

\haiku{Maar wij zijn hier met!}{ons zo veel wijze Mannen}{bij elkander}\\

\haiku{Het gevolg van dit, '.}{alles was   een zundvloed}{overt Aapenland}\\

\haiku{want, zei   hij, gij,.}{zegt toch dat het volmaakt met}{mijn concept strookt}\\

\haiku{Ten minsten waren:}{negen tiende deelen voor}{de afkapping}\\

\haiku{gij weeten, dat elk;}{Mensch een goede en   een}{kwaade zijde heeft}\\

\haiku{Veelen hunner:}{hadden dien gan-  schen}{nagt niet kunnen slaapen}\\

\haiku{wordende elke.}{kring geduurig honderd}{Aapen talrijker}\\

\haiku{Nommer 17  En.}{al uw Mede\"aapen zijn er}{bijna om koud}\\

\haiku{van de Reize door (.}{het Aapenland zijn aanwezig}{in KB Den Haagsign}\\

\haiku{Zie Mia I. Gerhardt,, (),-:}{Two Wayfarers UtrechtU.P.A.L. nr.}{5 1964 p. 2026}\\

\haiku{A study in the, (.).}{history of an idea3}{CambridgeMass 1948}\\

\haiku{allusie op de.}{partijnamen Orangisten}{en Patriotten}\\

\section{Jan van Panders}

\subsection{Uit: Kroniek van Alkmaar in de Bataafs-Franse tijd, 1787-1797}

\haiku{Dit mij ten vollen.}{verzeekerd door juffrouw}{Van der Slot aldaar}\\

\haiku{Verscheide van hun.}{hadden tevooren voor zwaar}{patriot gespeeld}\\

\haiku{De thermomeeter.}{teekende deze morgen}{4 graaden onder 0}\\

\haiku{1 maart In deze}{middag kwamen weder hier}{in van Purmerend}\\

\haiku{Dit dankuur werd alhier}{bij alle gezindheedens}{gehouden en schoon}\\

\haiku{(Ik heb niet kunnen.)}{goedvinden om dit request}{te teekenen}\\

\haiku{Den 28 dito heeft '.}{het klokgelui voort eerst}{opgehouden}\\

\section{Willem Pijper}

\subsection{Uit: Het papieren gevaar. Verzamelde geschriften (1917-1947) (3 delen)}

\haiku{Het is de echo van:}{wat hij in 1930 al schreef over}{Pijpers teksten}\\

\haiku{Het moet hem vele.}{maanden voorbereidingstijd}{hebben gekost}\\

\haiku{Pijpers uitspraak {\textquoteleft}de.}{muziek is een volkomen}{nutteloze zaak}\\

\haiku{In dat geval wordt:}{echte kunst geschapen met}{eeuwigheidswaarde}\\

\haiku{Toen ik kwam, had ik.}{Johan Rasch en Meyer Drukker}{als concertmeesters}\\

\haiku{men miste hier te.}{zeer contrastwerking.{\textquoteright}95 In Utrecht}{was het niet anders}\\

\haiku{Daarbij ben ik zelfs,,.}{nu nog in twijfel of ge}{vals bewust vals zijt}\\

\haiku{hij deed alsof dit.}{concert verlopen was als}{elk ander concert}\\

\haiku{de natuur van het.}{instrument wordt dan het minst}{geweld aangedaan}\\

\haiku{Daarentegen was;}{het Adagio niet gezonken}{en niet eindeloos}\\

\haiku{Men heeft toch vroeger!}{Beethoven en Berlioz ook}{moeten doorzetten}\\

\haiku{alles werd slapper -,.}{doch niet de heer Van Gilse}{treft hier alle schuld}\\

\haiku{Over de waarde der}{cadens zou nog het een en}{ander te zeggen}\\

\haiku{Maar, wat Wagenaar,.}{met zijn Berlioz bereikt is}{in \'e\'en woord volmaakt}\\

\haiku{Ik weet het ook niet -!}{ik kan er me geen noot meer}{van herinneren}\\

\haiku{Het vertolkt alle,.}{emoties het heeft accenten}{voor elk sentiment}\\

\haiku{Ik heb haar dit en,}{veel meer nog vergeven toen}{ik bemerkte welk}\\

\haiku{Wij Utrechtenaren!}{mogen onze concertzaal}{wel in ere houden}\\

\haiku{Kubbinga, Helvoirt,.}{Pel en Van der Ploeg vielen}{mij niet mee ditmaal}\\

\haiku{En ik moet helaas,}{zeggen de heer Caro is}{er niet in geslaagd}\\

\haiku{En als zodanig.}{vind ik de suite niet meer}{of minder dan zwak}\\

\haiku{En dat zij, die door}{haar natuur het volste recht}{heeft lief te hebben}\\

\haiku{Het werk bespreek ik,:}{binnenkort nader over de}{uitvoering slechts dit}\\

\haiku{Tod und Verkl\"arung'.}{Strauss Tod und Verkl\"arung}{besloot het concert}\\

\haiku{Tegen een opera:}{van Verdi valt inderdaad}{niets in te brengen}\\

\haiku{Eerste symfonie -!}{Welk een programma en}{welk een uitvoering}\\

\haiku{misschien lukte Faur\'es () -.}{kwartet even goedof bijna}{Mozart zeker niet}\\

\haiku{In hoeverre dit,.}{juist gezien is wil ik in}{het midden laten}\\

\haiku{De herenrollen,,.}{waren dunkt mij ook zo goed}{mogelijk bezet}\\

\haiku{mijn lijstje geeft niets.}{dat in dit blad niet reeds voor}{kort werd besproken}\\

\haiku{De positie van;}{dirigent van het U.S.O. is}{een zeer moeilijke}\\

\haiku{{\textquoteleft}Doe dat nu niet, die!}{stukken hebben ze al zo}{vaak in Utrecht gehoord}\\

\haiku{Als contrapuntist;}{staat Franck geheel onder}{Bachs sterke invloed}\\

\haiku{En het tweede deel,,:}{Scherzo herinnert weer aan}{een ander liedje}\\

\haiku{een kunstwerk dat een,:}{microkosmos is waarin}{alles zijn plaats heeft}\\

\haiku{het Schone en het,.}{Onschone het Sublieme}{en het Trachtende}\\

\haiku{Comme son coeur, sa,:}{religion est toute}{amour toute piti\'e}\\

\haiku{Elke techniek heeft,.}{haar tegenheden zo ook}{natuurlijk deze}\\

\haiku{het ergste wat we -:}{zouden kunnen verwachten}{en ik voeg erbij}\\

\haiku{En bij Liszts:}{Faust-wals werd het andere}{uiterste bereikt}\\

\haiku{Haar kostbare toon;}{tovert u onvermoede}{zaligheden voor}\\

\haiku{Laat men toch de o!}{niet zo dik en grof in de}{ruimte plamuren}\\

\haiku{Hun kwartetspel is.}{de grootste tovermacht die}{ik nog onderging}\\

\haiku{Van de doorvoering,.}{noch van de terugkeer valt}{veel te verhalen}\\

\haiku{Alles wat erin,.}{gebeuren gaat is reeds van}{tevoren bepaald}\\

\haiku{Dit is juist weer een,.}{theatereffect dat men}{liever zou missen}\\

\haiku{Het orkestspel heeft.}{mij ook in veel opzichten}{genot geschonken}\\

\haiku{Scherts, lezer, is dit,!}{misschien voor u en mij maar}{niet voor iedereen}\\

\haiku{Waarom ik met de,,?}{heer Z. aanvang lezer en}{niet met miss Parlow}\\

\haiku{Meesterlijk is een,:}{zeer veelzeggend woord doch het}{kan ook beduiden}\\

\haiku{De Finale lijdt.}{ook onder die barokke}{tegenstellingen}\\

\haiku{Hoe dit nu zij, opus.}{59.1 laat nooit na een diepe}{indruk te maken}\\

\haiku{Haar travestie was,.}{meer flatteus dan waarschijnlijk}{op zijn zachtst gezegd}\\

\haiku{Het zingen van dit.}{koor was thans beter dan ik}{het wel eens hoorde}\\

\haiku{Maar hij vat de zaak.}{aan op een wijze die mij}{niet de juiste lijkt}\\

\haiku{een stemming dus in ().}{kwarten met een terts in het}{middenluitstemming}\\

\haiku{het elektriseert u,;}{niet gelijk het vioolspel}{van een Fritz Kreisler}\\

\haiku{Und bedenkt man nun {\textquotedblleft}...}{dass diese Regeln in der}{Form desDu darfst nicht}\\

\haiku{Waarom volgt ook zij,?}{het  funeste gebruik}{zwaar zwaar te spelen}\\

\haiku{De zes liederen,.}{opus 68 intoneren ook}{niets anders niets nieuws}\\

\haiku{Het Interm\`ede,.}{is zeer ironisch hier en daar}{bepaald beklemmend}\\

\haiku{Cyclusconcert    ().}{5 februari 1920UD}{Tivoli   U.S.O. o.l.v}\\

\haiku{Regers werkwijze.}{voor al zijn composities}{kunnen aantonen}\\

\haiku{de heer Petri heeft.}{Bachs grandioze fuga}{uitmuntend gespeeld}\\

\haiku{Cyclusconcert654   ().}{17 februari 1920UD}{Tivoli   U.S.O. o.l.v}\\

\haiku{Zoiets is typisch'.}{voor de mentaliteit van}{Ewers bewonderaars}\\

\haiku{Zij verkonden de, {\textquoteleft}{\textquoteright}.}{winter zij groeien slechts daar}{waarVerwesung heerst}\\

\haiku{de Vreugde (Scherzo), (), ().}{de LiefdeAdagietto de}{LevensmoedRondo}\\

\haiku{Doch het aanvaarden.}{als een uiting van ziel tot}{ziel kan ik ook niet}\\

\haiku{effectbejag - is.}{helaas manifest genoeg}{in dit Requiem}\\

\haiku{Bringen Herz und Hirn -,.}{in Not  Ruhe ruhe}{meine Seele}\\

\haiku{Waar was het gaande,?}{statig-rouwende tempo}{van het eerste deel}\\

\haiku{\ensuremath{\cup} - heeft Schubert z\'o.}{genoteerd dat de tweede}{tel het accent krijgt}\\

\haiku{Het orkest had zijn,.}{slechtste en Zimmermann had}{zijn beste avond niet}\\

\haiku{Beroemd heeft het werk,.}{hem overigens niet gemaakt}{zomin als Lecocq}\\

\haiku{Das Lied vom Kummer...}{soll auflachend in die}{Seele euch klingen}\\

\haiku{Doch, houd de mensen, {\textquoteleft}}{de kunstenaars v\'o\'or dat de}{hoogste wijsheid dit}\\

\haiku{Orgelconcert    ().}{29 september 1920UD}{Buurkerk   Dirk Corn}\\

\haiku{Niet overal speelde;}{Rijnbergens reproductie}{op het hoogste plan}\\

\haiku{een royaal werk van,.}{Hans Brandenburg getiteld}{Der moderne Tanz}\\

\haiku{Dit behoorde tot().}{het hoogste dat een zanger}{es ooit kan geven}\\

\haiku{En het is helaas.}{te zelden dat men deze}{lof neerschrijven kan}\\

\haiku{Ik wenste dat Van;}{Gilse zich aan dit nabije}{voorbeeld spiegelde}\\

\haiku{Hun timbre iriseert,.}{te zelden flonkert niet in}{vele facetten}\\

\haiku{Bezitten we niet?}{evenzeer een muziek die de}{Heldenmoed prikkelt}\\

\haiku{Thans rangschikken wij.}{het onder de Muzieken die}{de Slaap oproepen}\\

\haiku{Het is het oude:}{liedje van de lyrische}{muziekbespreking}\\

\haiku{Het gaat niet zonder,,.}{een programma van acht of}{tien of twaalf nummers}\\

\haiku{Er is geen spoor van,.}{verzwakking in dit late}{werk integendeel}\\

\haiku{Charlotte Bara's.}{grootste vergissing is dat}{zij op muziek danst}\\

\haiku{En beschouwingen ', -?}{overs meesters artiestschap}{zienerschap waartoe}\\

\haiku{Het is een daad om.}{het publiek het Heden te}{doen accepteren}\\

\haiku{De reproductie.}{van dit laatste werk was zeer}{verre van volmaakt}\\

\haiku{En technisch (de twaalf).}{delen zijn alle wel zeer}{beknopt en psychisch}\\

\haiku{Kamermuziekavond - ():}{Das Rheinische Trio   18}{maart 1921UD ~ Kaun}\\

\haiku{Doch pijnlijk is het,,;}{soms en zwaar om zichzelf te}{kunnen uitspreken}\\

\haiku{Laat uw motieven.}{juist zo terugkomen als}{ze afgerold zijn}\\

\haiku{{\textquoteleft}faire du th\'e\^atre{\textquoteright} {\textquoteleft}}{gaat nooit samen metfaire}{de la musique{\textquoteright}.986}\\

\haiku{Zij zijn verplicht om:}{hun eigen wezen in hun}{muziek te zoeken}\\

\haiku{Licht in vele, naar.}{goedvinden te rangschikken}{betekenissen}\\

\haiku{En zo met bijna,.}{alle halfgoden van het}{tweede derde plan}\\

\haiku{Zie Lalo, Piern\'e, (!),,,,.}{Ravelja Bizet Chabrier}{Chausson Charpentier}\\

\haiku{Anders staat het ook.}{al niet met mijn opinie over}{zijn Sneeuwimpressie}\\

\haiku{Het kan dit, doch het.}{d\'e\'ed dit gisteravond zonder}{uitzondering niet}\\

\haiku{De Weser Zeitung {\textquoteleft}.}{roddelt wat overEen rijpe}{jonge musicus}\\

\haiku{{\textquoteleft}Gr\"uner Pfau{\textquoteright} (die wit),:}{was of op dat der land-}{en volkenkunde}\\

\haiku{modulaties der.}{stemmingen stonden open bij}{iedere maatstreep}\\

\haiku{{\textquoteleft}Als-ie er wat van?}{wist zou-d-ie ommers niet}{in de kr\'ant schrijven}\\

\haiku{Maar haal er nu de,,!}{muziek niet bij laat de kunst}{met rust ik bid u}\\

\haiku{de melancholie...}{dezer muziek bleven zelfs}{niet te vermoeden}\\

\haiku{Maar zijn Schelomo,,.}{een sterk een persoonlijk werk}{noemen mag ik niet}\\

\haiku{Het moet lokale,.}{politiek worden voor het}{interessant is}\\

\haiku{Welke gedachte:}{verstopt zich achter een zin}{als de volgende}\\

\haiku{Een omstandigheid,.}{buiten mijn wil buiten haar}{schuld waarschijnlijk ook}\\

\haiku{Ik zeg overigens -.}{niet dat ik dit een gemis}{vind integendeel}\\

\haiku{er bleven niet veel.}{landen ter wereld over die}{hij niet bezocht heeft}\\

\haiku{Tot zijn werken uit:}{de laatste jaren van zijn}{leven behoren}\\

\haiku{Zij heeft echter een,:}{niet zeer omvangrijke noch}{klankvolle mezzo}\\

\haiku{[...] Meine Symphonie,!}{wird etwas sein was die Welt}{noch nicht geh\"ort hat}\\

\haiku{Merk op dat hij \'o\'ok -:}{last had van de drie B's doch}{bij h\'em heetten ze}\\

\haiku{De heer Van Gilse.}{vermoedt uitsluitend uit vrees}{voor het Utrechtsch Dagblad}\\

\haiku{Het Utrechtse orkest () [}{woensdag 22 december}{1921Het Vaderland}\\

\haiku{Het is de mening.}{van de heer Pijper dat hij}{het niet heeft geleerd}\\

\haiku{Holland is w\'el het.}{voorbeschikte land voor de}{Begrafenissen}\\

\haiku{En ik meen dat de.}{muziek ouder rechten heeft}{dan het Commentaar}\\

\haiku{De muziek, ook hier,.}{in Holland iriseert reeds in}{andere sferen}\\

\haiku{Veel extase zal,.}{wellicht verbleekt blijken veel}{rust zou trekvoeten}\\

\haiku{Doch wat men w\'el, na,,:}{vijf minuten luisteren}{kon vaststellen is}\\

\haiku{Het deed denken aan';}{Berlioz land der muzikaal}{gelukzaligen}\\

\haiku{{\textquoteleft}Auferstehn, ja,,{\textquoteright}.}{auferstehn wirst du mein}{Staub nach kurzer Ruh}\\

\haiku{met meer Geloof (dat).}{allicht een klein heuveltje}{verzet dan Inzicht}\\

\haiku{Klanglich sch\"on{\textquoteright} (nu, nu!).}{en de houtblazers hebben}{er veel in te doen}\\

\haiku{Zulke foutjes zijn -.}{gauw gemaakt gelukkig even}{spoedig verbeterd}\\

\haiku{En dit gevaar had.}{Martine Dhont kunnen en}{moeten vermijden}\\

\haiku{Koene speelde het.}{concert beter dan ik het}{nog van hem hoorde}\\

\haiku{De muzikale:}{waarden lijken mij daarvoor}{ook hier te gering}\\

\haiku{Dit is het beste,.}{het persoonlijkste fragment}{van het Requiem}\\

\haiku{Zijn Beethoven voelt,,.}{hij niet stijf niet dogmatisch}{niet puriteins aan}\\

\haiku{Men heeft gisteravond:}{een zijner meest geslaagde}{liederen gehoord}\\

\haiku{iets anders is dan.}{het elkander te woord staan}{van twee musici}\\

\haiku{Maar de vraag die wij,:}{ons vandaag in de eerste}{plaats  stelden is}\\

\haiku{We hebben vandaag ();}{8 augustus ons eerste}{schandaaltje gehad}\\

\haiku{Waarom heeft hij dan...?}{ook geen An der sch\"onen blauen}{Donau geschreven}\\

\haiku{Het komma-punt:}{van uitgang onzer scepsis}{is het volgende}\\

\haiku{Een nocturne van.}{Chopin voor viool en}{piano eveneens}\\

\haiku{Een symfonie van.}{Bruckner voor piano \`a}{quatre mains evenzeer}\\

\haiku{Ik vraag een ogenblik...}{aandacht voor de portee der}{vier adjectieven}\\

\haiku{Frankrijk-Itali\"e,:}{Oostenrijk-Itali\"e zich}{ook voortsponnen tot}\\

\haiku{Maar zover zal het;}{met de Nationale}{Opera niet komen}\\

\haiku{De voorstelling werd,,.}{nog iets nieuws gedirigeerd}{door Rudolf Tissor}\\

\haiku{Hij blijkt een onzer.}{meest talentvolle jonge}{violisten}\\

\haiku{En ook het laatste.}{programmapunt bevatte}{levende namen}\\

\haiku{Er valt redelijk:}{vrij weinig op dit muziek}{maken te zeggen}\\

\haiku{haar inzicht zou zich.}{aangepast hebben aan de}{muziek van vandaag}\\

\haiku{te zeggen heeft) dan!}{Debussy's of Bachs Myra}{Hess stellig zou doen}\\

\haiku{De mannenrollen:}{waren beter bezet dan}{de vrouwenrollen}\\

\haiku{De tijdgenoten.}{van Moesorgski misten het}{gevoel voor zijn kunst}\\

\haiku{we gooien net zo;}{lang met het mooiste speelgoed}{tot het kapot is}\\

\haiku{Proberen - hij heeft (...).}{daarvoor te veel talentniet}{kwaadaardig bedoeld}\\

\haiku{zijn werk zal daar, hoop,.}{ik op den duur belangrijk}{genoeg voor worden}\\

\haiku{Tivoli-concert - Jan ().}{Dekker   2 november}{1922UD ~ U.S.O. o.l.v}\\

\haiku{Doch dit is wellicht.}{een atavisme waarvoor ik}{mij te schamen heb}\\

\haiku{Men kan er gewis,:}{een Wagneriaans klinkend}{stuk van maken doch}\\

\haiku{in 1922 lijken Brahms.}{en Wagner ternauwernood}{nog tijdgenoten}\\

\haiku{de composities.}{van zijn leermeesters Bart\'ok B\'ela}{en Kod\'aly Zolt\'an}\\

\haiku{Ook in de Etude.}{van Chopin waren zeer}{schone ogenblikken}\\

\haiku{Expressie volgens {\textquoteleft}{\textquoteright},:.}{mijn mening wat overdreven}{mild dus meewarig}\\

\haiku{{\textquoteright} De Telegraaf schijnt,,:}{het sinds een paar jaar z\'o te}{interpreteren}\\

\haiku{de kunstrubriek van:}{De Telegraaf voorlopig}{niet meer te lezen}\\

\haiku{Deze ambrozijn -!}{was een godenvoedsel maar}{in Beethovens tijd}\\

\haiku{Scherts, van meester X,,.}{Y en Z gecompileerd}{door Hans Pfitzner}\\

\haiku{Wat men in Holland.}{nimmer was en hopelijk}{ook nooit zal worden}\\

\haiku{voorstellen en ik,,.}{heb dat zelfs in een opera}{vaak genoeg gehoord}\\

\haiku{{\textquoteleft}Die ik de laatste (){\textquoteright},.}{keer gehoordof gespeeld heb}{was nog zo dwaas niet}\\

\haiku{Ik durf te zeggen.}{dat wij de Roussel van 1920}{hier nog niet kennen}\\

\haiku{Op die wijze is,.}{Roussels vioolsonate}{te lang geloof ik}\\

\haiku{Hinderen deed de.}{tamboerijn in abstracto}{overigens geen mens}\\

\haiku{Zo niet - poenitet).}{me peccasse monotoon}{en fantasieloos}\\

\haiku{Er waren ook wat ().}{onpreciesheden in de}{strijkerseerste deel}\\

\haiku{Bazel, of Z\"urich,,,.}{of Gen\`eve zijn dat op dit}{ogenblik althans niet}\\

\haiku{De uitspraak van het.}{koor is niet zeer zuiver en}{men zet wat traag in}\\

\haiku{Het is niet praktisch.}{vijf mensen een taak van twee}{toe te vertrouwen}\\

\haiku{Die verhouding zou.}{wel eens bij benadering}{vast te stellen zijn}\\

\haiku{ook de beeldstormer,.}{destructeur heeft soms recht op}{een herinnering}\\

\haiku{Zo hakt men bomen,.}{met een schaaf snijdt camee\"en}{met een koubeitel}\\

\haiku{Ik vind in zijn werk.}{geen kiemen van een zelfs maar}{korte eeuwigheid}\\

\haiku{Een lief dialoogje,,.}{een dito muziekje een}{dito aankleding}\\

\haiku{Ik voor mij, ik houd,.}{wel van zulk soort dubbel het}{animeert tenminste}\\

\haiku{Het is op twee na.}{de bekendste van Schuberts}{acht symfonie\"en}\\

\haiku{Das M\"archen von,.}{der sch\"onen Melusine}{van Heinrich Hofmann}\\

\haiku{Zijn muziek is op.}{dit ogenblik in Holland nog}{vrijwel onbekend}\\

\haiku{Ook in het vierde,,.}{deel staan welbewust enige}{italianismen}\\

\haiku{Chopin, Liszt),:}{honderdmaal veelzijdiger}{gefacetteerd dus}\\

\haiku{Natuurlijk klinkt Les - []:}{soir\'ees de P\'etrograde}{anders zeggenwij}\\

\haiku{En in de tweede:}{relatie overtreft Schumann}{Brahms vele malen}\\

\haiku{Die keerzijde bleef,,.}{bij Stravinsky tot \ensuremath{\pm} 1920}{erg onafgewerkt}\\

\haiku{Dit toch is het doel.}{der kunstbewerking genaamd}{psychoanalyse}\\

\haiku{Kod\'aly's werken zijn;}{hier tot dusverre bijna}{niet doorgedrongen}\\

\haiku{Het vrouwenkoor van.}{Toonkunst bevat overigens}{goed materiaal}\\

\haiku{En men kan dan zelfs}{tot conclusies komen die}{hier en daar lijnrecht}\\

\haiku{Bezinningen - en.}{te weinig elans werden tot}{autoprojecties}\\

\haiku{De reien bleven,.}{de hoogtepunten ook van}{deze voorstelling}\\

\haiku{Die eerste strofen,.}{die een verhaspelde Van}{Eeden voorstellen}\\

\haiku{En dus eigenlijk.}{al geantiquiseerd v\'o\'or}{het geschreven was}\\

\haiku{dat do\'et het leven.}{in Europa van onze}{dagen ook niet meer}\\

\haiku{En Der Abschied heet.}{het beste fragment van Das}{Lied von der Erde}\\

\haiku{Het is pijnlijker.}{hem te moeten verwerpen}{dan Strauss of Reger}\\

\haiku{Dat de Eerste een,.}{Volgelingenwerk is kan}{het stuk niet schaden}\\

\haiku{Spaanderman schijnt (zie);}{ook zijn dirigeren geen}{geboren leider}\\

\haiku{{\textquoteright}1782 ~ Ziehier een.}{gloednieuwe definitie}{van het contrapunt}\\

\haiku{de Debussynse,:}{Pastorale voor zang en}{piano van 1906}\\

\haiku{Praktisch zal men daar,.}{dus nimmer voorbeelden van}{vinden vermoed ik}\\

\haiku{Het verschijnsel op.}{zichzelf behoeft ons dus niet}{te verontrusten}\\

\haiku{die er genoeg van,}{hebben Zimmermann altijd}{links vooraan te zien}\\

\haiku{{\textquoteleft}Met hoeveel fluiten?}{wilt u het concert van Bach}{uitgevoerd hebben}\\

\haiku{{\textquoteright} Het is gelukkig.}{alleen voor de schrijver van}{dat stukje maar waar}\\

\haiku{En men kon dus de.}{muzikale factoren}{laten beslissen}\\

\haiku{Laten wij maar niet;}{op nieuwe openbaringen}{gaan zitten wachten}\\

\haiku{Dit is constructief,.}{beter dan Ferroud het}{is zelfs scholastisch}\\

\haiku{{\textquoteright} In het origineel {\textquoteleft},,.}{staatJe t'aimerai Seigneur}{d'un amour tendre}\\

\haiku{Misschien heeft hij het?}{thema bij ongeluk in}{kreeftengang gebracht}\\

\haiku{Maar er ligt weinig:}{belofte voor de toekomst}{in opgesloten}\\

\haiku{Zij schijnen daar voor:}{rituele doeleinden}{gebruikt te worden}\\

\haiku{Pas na de oorlog.}{waagde Sch\"onberg zich weer aan}{de compositie}\\

\haiku{Doch juist hier ligt het.}{verschil tussen hem en de}{beide genoemden}\\

\haiku{Zijn motivering,:}{was een muzikale de}{andere waren}\\

\haiku{De opera stelt nu.}{eenmaal andere eisen}{dan de concertzaal}\\

\haiku{zijn constructies zijn'.}{uitbreidingen van Stamitz}{en Mozarts structuur}\\

\haiku{Maar de kunstenaar,.}{van deze van onze tijd}{gedraagt zich anders}\\

\haiku{\'e\'en meesterwerk, de,;}{Sonate voor fluit alt en}{harp van Debussy}\\

\haiku{Een stuk als Saul en.}{David heeft niemand hem hier}{nog nagemaakt}\\

\haiku{Melodisch blijft het.}{wat vlak en er zijn een paar}{doffe plekken in}\\

\haiku{in 1909 bestond de.}{Pell\'eas van Debussy}{reeds bijna tien jaar}\\

\haiku{te veel wildheid nog,.}{te veel oppervlakkige}{scherts en ontroering}\\

\haiku{Want Sch\"onberg was de,;}{dogmaticus de louter}{destructieve geest}\\

\haiku{Weberns F\"unf S\"atze -.}{zijn hier voor de eerste maal}{gespeeld tweemaal zelfs}\\

\haiku{het aambeeld, de man, -.}{het beest laat zich natuurlijk}{wel analyseren}\\

\haiku{Hoe vaak brengen de?}{internationale}{virtuozen nieuws}\\

\haiku{Maar het lijkt mij dat.}{gij aan het jazzsymptoom veel te}{veel waarde toekent}\\

\haiku{Dit lijkt mij, in beeld,.}{gebracht de zaak die wij aan}{de orde stelden}\\

\haiku{Maar in 1784 had hij ().}{reeds een Pianoconcert}{in Es geschreven}\\

\haiku{Dit orkest is een.}{buitengewoon subtiel en}{willig apparaat}\\

\haiku{Maar Berlioz ervoer,.}{altijd alles anekdotisch}{buiten samenhang}\\

\haiku{Eerste en Derde,.}{symfonie Ouverture}{Coriolanus}\\

\haiku{De tijden voor de.}{Nederlandse muziek zijn}{gunstiger dan ooit}\\

\haiku{Louise Wijngaarden, ( {\textquoteleft}...}{van het Gebouwklinkt het niet}{alsde l'Institut{\textquoteright}?}\\

\haiku{twee koningen voor -?).}{\'e\'en gemenebest wie kon}{zich dat voorstellen}\\

\haiku{Paul Sanders laat zijn;}{executanten meer vrijheid}{dan Pieter Tiggers}\\

\haiku{Zowel Tierie als.}{de heer en mevrouw De Boer}{komt daarvoor lof toe}\\

\haiku{Een periode.}{waarvan Mahler een typisch}{representant was}\\

\haiku{Er wordt angstwekkend (),:}{geslagentoch heeft zij geen}{groot forte ofwel}\\

\haiku{In aanleg is er.}{zelfs literair vermogen}{bij hem aanwezig}\\

\haiku{de reis is erg duur.}{en Itali\"e garandeert ons}{geen Frankfurts comfort}\\

\haiku{Maar als opera is.}{het stuk van Gounod beter}{dan dat van Busoni}\\

\haiku{Bewijst dit nu de?}{juistheid of onjuistheid van}{Sch\"onbergs principes}\\

\haiku{De werkelijke.}{waarde van een kunstwerk wordt}{er niet door bepaald}\\

\haiku{Het hoofdthema van.}{dit rondo is bovendien}{bepaald triviaal}\\

\haiku{De wezenlijke.}{waarde van een kunstwerk wordt}{er niet door bepaald}\\

\haiku{Het hoofdthema van.}{dit Rondo is bovendien}{bepaald triviaal}\\

\haiku{Het hoofdthema van.}{dit Rondo is bovendien}{bepaald triviaal}\\

\haiku{precies even typisch.}{Hollands zijn als Das Lied von}{der Erde Chinees}\\

\haiku{Maar Saskia lijkt een}{meesterwerk naast Nico van}{der Lindens toonstuk}\\

\haiku{Henri Zagwijn (1878).}{is the most modern of this}{generation}\\

\haiku{let us hope that,,.}{music in general will}{reap the fruits}\\

\haiku{Dat kan iedereen.}{met goede oren en goede}{hersenen leren}\\

\haiku{Superieur zijn (;}{de karikaturen van}{BrahmsIntermezzo}\\

\haiku{Pastiche nr. 16,:}{bestaat niet uit noten doch}{uit een regel druks}\\

\haiku{Mettez beaucoup de,,.}{notes n'importe lesquelles}{sauf celles qu'il faut}\\

\haiku{Het is trouwens niet;}{erg om zijn wiegeliedjes}{vergeten te zijn}\\

\haiku{Voor de ter zake.}{kundige lezer is dit}{nu wel voldoende}\\

\haiku{De woorden van het,.}{liedje dat zij daarbij zingt}{zijn van Debussy}\\

\haiku{En het publiek kan,...}{over iets wat het nooit gehoord}{heeft kwalijk denken}\\

\haiku{Hij {\textquoteleft}laadt{\textquoteright} de noten ();}{vanbijvoorbeeld de Vijfde}{met zijn eigen geest}\\

\haiku{Zijn de noten van,?}{een compositie dode}{symbolen of meer}\\

\haiku{ondergeschikt aan.}{de muziek en daardoor juist}{wat het wezen moest}\\

\haiku{Waarom geschiedt in?}{dit drama alles zoals}{het geschieden moest}\\

\haiku{Hoe sterker deze,.}{drang is des te sterker ook}{de vernielingsdrang}\\

\haiku{Hij dirigeert niet,.}{zonder entrain hij stelt zich}{het klankbeeld juist voor}\\

\haiku{De orkesten zijn.}{talrijk en de concerten}{worden goed bezocht}\\

\haiku{Van die bedoeling}{is de componist echter}{teruggekomen.{\textquoteright}2376}\\

\haiku{Die ontroering kan,,:}{direct ongecontroleerd}{aan de dag treden}\\

\haiku{Instructief was in.}{dit opzicht vooral het Oud}{driekoningenlied}\\

\haiku{Bach was noch een dor,:}{theoreticus noch een}{gevoelig dichter}\\

\haiku{het mangelt hem aan,,.}{felheid aan kritische zin}{aan vitaliteit}\\

\haiku{toen Mengelberg de.}{Suite uit L'oiseau de}{feu dirigeerde}\\

\haiku{Hiermee waren wij.}{dus voor de tweede keer de}{gasten van Itali\"e}\\

\haiku{Webern heeft zich het:}{merk van Sch\"onberg-adept}{te diep ingebrand}\\

\haiku{De  langzame,;}{delen zijn zeer grof voor een}{pianomuziek}\\

\haiku{Deze opera is.}{van 1865 en er staat gewis}{al veel Bizet in}\\

\haiku{Men offert veel op,.}{aan de logica ja zelfs}{aan de scholastiek}\\

\haiku{Parijs, Berlijn en,:}{Wenen verloren veel van}{hun betekenis}\\

\haiku{hier in Amsterdam.}{behaalde het nauwelijks}{een succ\`es d'estime}\\

\haiku{Het eerste deel is,.}{goed maar de rest is bijster}{weinig overtuigend}\\

\haiku{Karl Marx, Leo\v{s} Jan\'a\v{c}ek}{Onbevredigende}{kamermuziek}\\

\haiku{Adriano Lualdi's.}{Le furie d'Arlecchino}{is een luchtledig}\\

\haiku{Belangrijk was ook ().}{het StrijktrioSerenade}{van Alexander Jemnitz}\\

\haiku{Niemand zal tegen.}{deze gedachtegang iets}{willen inbrengen}\\

\haiku{Maar dat is ook niet;}{in eerste instantie de}{taak van het publiek}\\

\haiku{het intermezzo ().}{heeft aan de helft genoegzes}{maten plus opmaat}\\

\haiku{Het is bijna een,.}{revenant van het eerste}{deel in het verkort}\\

\haiku{Men herkent soms in;}{de aanvang de melodiek}{niet onmiddellijk}\\

\haiku{Dit melodische ().}{gegeven loopt totniet tot}{en met de slot-a}\\

\haiku{wat wordt ons, zowel, {\textquoteleft}{\textquoteright}?}{hier als elders voorgespeeld}{alsnieuwe muziek}\\

\haiku{Hij sleept niet mee, hij,.}{verkondigt niet hij vleit niet}{en hij dreigt evenmin}\\

\haiku{Wij komen op de.}{betekenis van Anton}{Webern nog terug}\\

\haiku{Hierop berust ten {\textquoteleft}{\textquoteright}.}{dele de dooddoener van}{DoppersHollandsheid}\\

\haiku{Zijn lyriek ontroert,.}{ons niet zijn climaxen grijpen}{ons niet in het hart}\\

\haiku{Misschien verwondert:}{deze uitlating u uit}{mijn mond een weinig}\\

\haiku{Ik had in hoofdzaak.}{hetzelfde willen vragen}{als de heer Gomperts}\\

\haiku{Ten slotte zijn ook ({\textquoteleft}}{de varianten op het}{oude Wilhelmus}\\

\haiku{Ik ben verheugd dat,.}{Wozzeck hier komt omdat het}{een prestatie is}\\

\haiku{Want voor de kunst als.}{zodanig loopt het grote}{publiek niet warm}\\

\haiku{En vangen aan te,.}{schelden of te loven in}{hun halve wijsheid}\\

\haiku{De grote massa.}{der tijdgenoten kan dit}{niet observeren}\\

\haiku{Het is, inderdaad,,.}{het allervoornaamste maar}{het is niet genoeg}\\

\haiku{{\textquoteright} zijn wij het in het.}{gegeven geval zonder}{enig voorbehoud eens}\\

\haiku{Maar neutraliteit.}{is de meest funeste vorm}{van tegenwerking}\\

\haiku{Dit hier lijkt op een.}{hedendaagse versie van}{Das Narrenschneiden}\\

\haiku{de hoofdrolspelers...}{verzuimden tot nu toe hun}{baard te laten staan}\\

\haiku{Op twee punten moet.}{ik in het bijzonder de}{aandacht vestigen}\\

\haiku{Met de muziek heeft.}{dit alles ternauwernood}{nog iets te maken}\\

\haiku{Julius R\"ontgen,,.}{bijvoorbeeld was het type}{van de muzikant}\\

\haiku{Maar dit is niet het.}{tijdstip voor stijlkritische}{bespiegelingen}\\

\haiku{Er zijn daar zalen;}{waarin het een genot is}{te musiceren}\\

\haiku{Maar hoezeer is de,!}{stemming erdoor be{\"\i}nvloed}{de sfeer vergiftigd}\\

\haiku{Aandelen Grieg zijn;}{tegenwoordig nauwelijks}{verhandelbaar meer}\\

\haiku{De situatie.}{is dus veel ernstiger dan}{wij gevreesd hadden}\\

\haiku{Stel daartegenover:}{een soortgelijk muzikaal}{probleem uit Lulu}\\

\haiku{Op het terrein der.}{muziekpedagogie valt}{nog zeer veel te doen}\\

\haiku{De Maneto zal,,.}{concerteren op 5 6}{8 en 12 juni}\\

\haiku{een Engels (Purcell), () ().}{een HollandsVoormolen en}{een FransDelibes}\\

\haiku{Wij signaleren,.}{dit niet voor de eerste niet}{voor de tiende maal}\\

\haiku{En dan verwondert.}{men zich nog dat de gasten}{dit maal niet lusten}\\

\haiku{Dich ber\"uhrte in,...}{Welschland fremder s\"usser}{Kunst neue Verk\"undung}\\

\haiku{Dich ber\"uhrte in,...}{Welschland fremder s\"usser}{Kunst neue Verk\"undung}\\

\haiku{Hij kwam, met deze,.}{conceptie schijnbaar in de}{buurt van Sch\"onbergs school}\\

\haiku{Wat dacht je, zou daar?}{nog een artikel voor de}{Groene in zitten}\\

\haiku{Het Concertgebouw;}{vernieuwt principieel zo}{weinig mogelijk}\\

\haiku{Weinig, te weinig,.}{van die beloften is in}{vervulling gegaan}\\

\haiku{de actuele.}{situatie geeft niet veel}{hoop op de toekomst}\\

\haiku{Maar de wereld der.}{klanken blijft onberoerd door}{het oorlogsgeweld}\\

\haiku{Terugziend op het:}{festival 1946 lijkt mij dat}{meer dan mogelijk}\\

\haiku{Jardins sous la pluie (1903)}{waarin de chromatisch}{dalende baslijn}\\

\haiku{Hier in Nederland.}{wordt Schmitts muziek zelden ten}{gehore gebracht}\\

\haiku{De composities (;}{uit de jaren 1904 en 1905}{Acht Lieder opus 6}\\

\haiku{H\'aba eveneens, doch.}{bovendien verwierp hij het}{hele toonsysteem}\\

\haiku{Tot dusverre bleef.}{de beweging tot het land}{van oorsprong beperkt}\\

\haiku{Natuurlijk klinkt Les -:}{soir\'ees de P\'etrograde}{anders zeggen wij}\\

\haiku{Mettertijd zal dit (}{het geval niet meer zijnmen}{went nergens vlugger}\\

\haiku{Harmonisch is het,.}{stukje zo ge wilt a-}{of polytonaal}\\

\haiku{sluit het werk met een.}{uit \ensuremath{\beta} afgeleid ritmisch}{figuur abrupt}\\

\haiku{Ook in het vierde,,.}{deel staan welbewust enige}{italianismen}\\

\haiku{Ons Europese,.}{\'e\'en-tw\'e\'e heeft zich wel}{overleefd naar het blijkt}\\

\haiku{Het eerste stukje;}{heeft het karakter van een}{marche fun\`ebre}\\

\haiku{A en b stijgen,.}{tot een climax c verloopt}{pianissimo}\\

\haiku{Het is alsof de;}{inwijdeling zich op gaat}{heffen tot het Al}\\

\haiku{Wil men zulks t\'och doen,.}{dan wachte men tot onder}{het Broedermaal}\\

\haiku{(Voor een overzicht van,,-.}{de inhoud zie Bijlage}{3 HPG 2 922927}\\

\haiku{in het Engels met.}{kapitalen en in het}{Frans in onderkast}\\

\haiku{{\textbullet} Acht\'elik, Josef,.}{Der Naturklang als Wurzel}{aller Harmonien}\\

\haiku{Annalen van de-.}{operagezelschappen in}{Nederland 18861995}\\

\haiku{Debussy, Claude,- ().}{Correspondance 18841918}{ed. Fran\c{c}ois Lesure}\\

\haiku{Verkade, Eduard \&,.}{dr. E.F. Cartier van Dissel}{Eduard Verkade}\\

\haiku{{\textbullet} Vuillermoz, \'Emile, {\textquoteleft}{\textquoteright}.}{Le style orchestral}{de Maurice Ravel}\\

\haiku{{\textbullet} Wolff, Betje \& Aagje,.}{Deken Historie van den}{Heer Willem Leevend}\\

\haiku{100 I 430, 431, 656,,,.}{657 658 Vioolsonate}{nr. 3 in d op}\\

\haiku{La vita nuova,}{I 801 Paradiso I 507}{807 Purgatorio}\\

\haiku{82 I 662 II 64,,,,.}{235 240 Gli\`ere Reinhold}{Strijkkwartet op}\\

\haiku{I 390 Mein Herz I,:}{390 Ord-Hume Arthur W.G.}{Pianola}\\

\haiku{832 Hoek, H.G. van de,,,, (-}{organist I 354 355 685}{686 Hoer\'ee Arthur18971986}\\

\haiku{) schrijver/librettist,,, (-)}{II 67 717 747 Hol J.C.1874}{1953 musicoloog}\\

\haiku{Simon van (1849-1929),, (-)}{musicoloog II 596 602}{Mittler Franz18931970}\\

\haiku{mecenas II 789, (-)}{Winding August Hendrik1835}{1899 componist II}\\

\haiku{115Ingezonden brief, (),.}{van Ovink opgenomen in}{Van Gilse2003 413}\\

\haiku{Pijper gebruikt het.}{woord grondtoon hier wellicht in}{figuurlijke zin}\\

\haiku{281Zie voetnoot 100.}{bij de recensie van 30}{januari 1918}\\

\haiku{289Zie voetnoot 83.}{bij de recensie van 16}{januari 1918}\\

\haiku{308Zie voetnoot 125.}{bij de recensie van 21}{februari 1918}\\

\haiku{341Zie voetnoot 83.}{bij de recensie van 16}{januari 1918}\\

\haiku{{\textquoteleft}Het is de eerste.}{maal dat de heer Pijper op}{zijn stuk terugkomt}\\

\haiku{slechts zal hier en daar.}{de nodige beperking}{worden ingevoerd}\\

\haiku{18.4 en het kwartet,.}{van Ravel n\'a de pauze}{stond Dvo\v{r}\'aks kwartet op}\\

\haiku{Het citaat {\'\i}n het,,-.}{citaat is van Berlioz \`A}{travers chants 89}\\

\haiku{{\textquoteleft}L'affirmation!}{de l'impuissance y est}{pouss\'ee jusqu'au dogme}\\

\haiku{452Voor Stephan P\'artos.}{zie de recensie van 19}{januari 1919}\\

\haiku{Zie ook voetnoot 130.}{bij de recensie van 27}{februari 1926}\\

\haiku{471Zie voetnoot 22.}{bij de recensie van 23}{januari 1918}\\

\haiku{481Parafrase:}{van een regel uit Schillers}{Ode an die Freude}\\

\haiku{511Zie voetnoot 68.}{bij Pijpers recensie van}{28 november 1919}\\

\haiku{514Pijper in de.}{programmatoelichting van}{26 november 1919}\\

\haiku{{\textquotedblleft}Dieu le voit{\textquotedblright}, maar.}{ik zie het ook en anders}{dan hij zich verbeeldt}\\

\haiku{Zie de recensie.}{van 28 november 1917 en}{de voetnoot aldaar}\\

\haiku{Zie ook voetnoot 126.}{bij de recensie van 21}{februari 1918}\\

\haiku{{\textquoteright} is een regel uit ().}{Der AbschiedDas Lied von der}{Erde van Mahler}\\

\haiku{Zie verder voetnoot.}{91 bij de recensie van}{20 december 1919}\\

\haiku{de namen van Van,,,.}{Anrooy Averkamp Viotta}{Wagenaar en Zweers}\\

\haiku{Deze schrijft, als hij:}{het heeft over de kenmerken}{van Franse muziek}\\

\haiku{Zie ook voetnoot 78.}{bij de recensie van 14}{januari 1918}\\

\haiku{631Minstrels waren,.}{geen negers maar als negers}{geschminkte blanken}\\

\haiku{644Zie voetnoot 240.}{bij de recensie van 13}{januari 1919}\\

\haiku{657Zie voetnoot 199.}{bij de recensie van 12}{februari 1920}\\

\haiku{Op 19 maart en 8.}{oktober 1921 noemt Pijper}{de opera nogmaals}\\

\haiku{713Op het programma,.}{na de pauze stond Sindings}{Vioolconcert op}\\

\haiku{722Ruyneman schreef zijn ();}{lied L'absolu1919 op tekst}{van Arthur P\'etronio}\\

\haiku{Matthijs Vermeulen.}{deed er ironisch verslag van}{in De Telegraaf}\\

\haiku{809Zie voetnoot 161.}{bij de recensie van 21}{januari 1920}\\

\haiku{818Zie voetnoot 122.}{bij de recensie van 12}{februari 1918}\\

\haiku{820Zie voetnoot 161.}{bij de recensie van 21}{januari 1920}\\

\haiku{828Zie voetnoot 207.}{bij de recensie van 17}{februari 1920}\\

\haiku{837Zie voetnoot 125.}{bij de recensie van 21}{februari 1918}\\

\haiku{852Zie voetnoot 171.}{bij de recensie van 23}{januari 1920}\\

\haiku{{\textquoteright} 876Zie voetnoot 218.}{bij de recensie van 21}{februari 1920}\\

\haiku{903Zie voetnoot 109.}{bij de Muziekkroniek van}{24 december 1920}\\

\haiku{938Zie voetnoot 125.}{bij de recensie van 21}{februari 1918}\\

\haiku{944Zie voetnoot 138.}{bij de recensie van 12}{januari 1920}\\

\haiku{963Zie voetnoot 78.}{bij de recensie van 14}{januari 1918}\\

\haiku{985Pijper schept hier.}{waarschijnlijk een beetje op}{over zijn geheugen}\\

\haiku{987Zie voetnoot 205.}{bij de recensie van 14}{februari 1920}\\

\haiku{988Zie voetnoot 223.}{bij de recensie van 21}{februari 1920}\\

\haiku{Het roode lampje,,.}{signifische gepeinzen}{verschenen in 1921}\\

\haiku{Wellicht is deze ().}{Sneeuwimpressie identiek aan}{de Dansimpressie1920}\\

\haiku{1024Zie voetnoot 223.}{bij de recensie van 21}{februari 1920}\\

\haiku{{\textquotedblright} Daarop, tot angst en,,.}{schrik van mijn vrouw liep ik de}{deur uit de straat op}\\

\haiku{1054Zie voetnoot 138.}{bij de recensie van 12}{januari 1920}\\

\haiku{1079Zie voetnoot 154.}{bij de recensie van 11}{februari 1921}\\

\haiku{1082Zie voetnoot 133.}{bij de recensie van 11}{januari 1921}\\

\haiku{1087Zie voetnoot 54.}{bij de recensie van 14}{februari 1921}\\

\haiku{1149In de Camera (,:}{ObscuraDe familie}{Kegge hoofdstuk 7}\\

\haiku{Diepenbrock aanvaardt.}{de dedicatie en woont}{de uitvoering bij}\\

\haiku{Zie verder voetnoot.}{165 bij de recensie van}{13 augustus 1922}\\

\haiku{De verzameling ().}{is uitgegeven bij W.}{de Haan te Utrechtz.j.}\\

\haiku{Musikbl\"atter des,,-,.}{Anbruch jrg. 4 nr. 34}{februari 1922}\\

\haiku{1328Zie voetnoot 46.}{bij de recensie van 27}{januari 1922}\\

\haiku{1343Zie voetnoot 19.}{bij de recensie van 6}{januari 1922}\\

\haiku{Vooral de laatste.}{paragraaf van het tweede}{deel werd herschreven}\\

\haiku{1404Zie voetnoot 109.}{bij de recensie van 7}{januari 1920}\\

\haiku{Musikbl\"atter des,,-,.}{Anbruch jrg. 4 nr. 34}{februari 1922}\\

\haiku{1469Zie voetnoot 138.}{bij de recensie van 12}{januari 1920}\\

\haiku{1518Zie voetnoot 166.}{bij de recensie van 21}{januari 1920}\\

\haiku{/ Zu singen um das, /',':}{Meisterst\"uck Gewinn es}{Kunst gewinn es Gl\"uck}\\

\haiku{zie de recensie.}{van 17 oktober 1921 en}{voetnoot 252 aldaar}\\

\haiku{Nu 'k over Claire '.}{Jache spreek moetk nog even}{verder vertellen}\\

\haiku{1604Zie voetnoot 171.}{bij de recensie van 23}{januari 1920}\\

\haiku{{\textquoteright} 1610Zie voetnoot 19.}{bij de recensie van 6}{januari 1922}\\

\haiku{{\textquoteright} Hendrik Marsman, {\textquoteleft}De{\textquoteright}.}{positie van de jonge}{Hollandse schrijver}\\

\haiku{1707Tekstregel uit ().}{Der AbschiedDas Lied von der}{Erde van Mahler}\\

\haiku{{\textquoteright} 1737Pijper beschouwt.}{de Scythische suite als}{programmamuziek}\\

\haiku{Zie ook voetnoot 126.}{bij de recensie van 21}{februari 1918}\\

\haiku{1978Herbert Antcliffe, {\textquoteleft},{\textquoteright}.}{De Planeten suite voor}{orkest van Gustav Holst}\\

\haiku{1995Dit essay is.}{later opgenomen in}{De Quintencirkel}\\

\haiku{2013Bedoeld zijn de.}{componisten van d'Indy's}{Schola cantorum}\\

\haiku{2048Zie voetnoot 176.}{bij de recensie van 25}{januari 1920}\\

\haiku{nr. 103 in Es (Mit) ().}{dem Paukenwirbel en nr.}{104 in DLondon}\\

\haiku{2080Zie voetnoot 167.}{bij de recensie van 21}{februari 1927}\\

\haiku{2116Een Equale is een.}{stuk voor gelijke stemmen}{of instrumenten}\\

\haiku{[...] Sie sehen bereits, []?}{wie sehr mich diese Musik}{Carmen verbessert}\\

\haiku{Beethoven, voor vijf,.}{pianoconcerten meer}{dan een half dozijn}\\

\haiku{De voorstellingen,,.}{waren op 10 12 14 en}{16 november 1927}\\

\haiku{2327Zie voetnoot 187.}{bij het essay in het Haagsch}{Maandblad van maart 1927}\\

\haiku{2339La puerta del.}{vino is de vijftiende}{van de 24 Pr\'eludes}\\

\haiku{Ook verschenen zijn.}{prenten in De Vrijheid en}{De Ware Jacob}\\

\haiku{2346Adolf Oberlander (-).}{18451923 was een beroemd Duits}{karikaturist}\\

\haiku{{\textquoteleft}Willem Pijper... c'est.}{une gloire musicale}{de la Hollande}\\

\haiku{{\textquoteright} 2350Zie voetnoot 130.}{bij de recensie van 27}{februari 1926}\\

\haiku{2358Zie ook de open ({\textquoteleft}{\textquoteright}.}{brief van Pijper aan Rutters}{Moderne muziek}\\

\haiku{2369Zie voetnoot 96.}{bij de recensie van 24}{januari 1928}\\

\haiku{Pijper schrijft op 4:}{juni 1946 in een brief aan}{Louise Bolleman}\\

\haiku{2380Zie voetnoot 161.}{bij de recensie van 21}{januari 1920}\\

\haiku{De voorstellingen,,.}{waren op 15 17 19 en}{21 november 1928}\\

\haiku{Op deze plaats is.}{de formulering uit De}{Muziek gehandhaafd}\\

\haiku{2474Zie voetnoot 161.}{bij de recensie van 21}{januari 1920}\\

\haiku{het staat er onder {\textquoteleft},{\textquoteright}-.}{het kopjeZ\"urich juni}{1926 op p. 119131}\\

\haiku{Maar een tijd die reeds,.}{gekomen is behoeft men}{niet meer te beiden}\\

\haiku{2520Midas staat hier.}{voor de kunstrechter die zich}{belachelijk maakt}\\

\haiku{2526Dit Strijkkwartet.}{uit 1927 was opgedragen}{aan Willem Pijper}\\

\haiku{de boston is een.}{Amerikaanse versie van}{de langzame wals}\\

\haiku{2744Toen Alban Berg.}{stierf was zijn tweede opera}{Lulu onvoltooid}\\

\haiku{mit allen Mitteln.}{der vorhandenen Technik}{eine Welt aufbauen}\\

\haiku{De reeks concerten.}{waarover Pijper hier schrijft was}{daarvan de eerste}\\

\haiku{2847De Anschluss van.}{Oostenrijk bij Duitsland vond}{plaats op 12 maart 1938}\\

\haiku{] Hindemith {\textquoteleft}hoort{\textquoteright} als.}{componist slechts tonale}{mogelijkheden}\\

\haiku{zie de recensie.}{in het Rotterdamsch Nieuwsblad}{van 9 juli 1927}\\

\haiku{] Georg Schaeffner, ().}{Claude Debussy und das}{PoetischeBern 1943}\\

\haiku{2896Pijper verwijst.}{hier met een voetnoot naar de}{huidige p. 804}\\

\haiku{2917De naam van Arthur.}{Hoer\'ee is vanaf de tweede}{druk weggelaten}\\

\haiku{2983Zie de voetnoot.}{bij de recensie van 12}{januari 1920}\\

\haiku{En ook anders dan.}{in een volmaakt geslaagde}{atonale muziek}\\

\haiku{{\textquoteright} 3006Zie de voetnoot.}{bij de recensie van 15}{januari 1920}\\

\haiku{3019Zie de voetnoot.}{bij de recensie van 21}{januari 1920}\\

\haiku{3022Zie de voetnoot.}{bij de recensie van 21}{januari 1920}\\

\haiku{tevens werden er.}{muziekvoorbeelden in de}{tekst opgenomen}\\

\haiku{3036Dit is niet de.}{eerste maal dat Pijper het}{woord kiemcel gebruikt}\\

\haiku{Zijn leerling Jan van.}{Dijk voltooide het werk in}{1992 als zijn opus 839a}\\

\section{Filip de Pillecyn}

\subsection{Uit: Elizabeth}

\haiku{Het mooie, donkere.}{kind was een vroeg ontwikkeld}{meisje geworden}\\

\haiku{Met donkere blik.}{had Konrad von Marburg op}{hen neergekeken}\\

\haiku{Zij nam het kind op.}{en \'e\'en voor \'e\'en bekeken}{de monniken het}\\

\haiku{Van ver het land in,.}{kwamen de mensen naar de}{voet van de Wartburg}\\

\haiku{Reeds uren wacht zij daar.}{met een knecht die de paarden}{aan de toom rondleidt}\\

\haiku{Dezelfde morgen.}{had zij hem toegefluisterd}{dat zij zwanger was}\\

\haiku{Hij keert terug naar.}{zijn tent en strekt zich uit op}{zijn legerstede}\\

\haiku{In \'e\'en ogenblik ziet;}{zij voor zich al de vreugden}{die zij heeft gekend}\\

\haiku{De woeste vreugde;}{van de feestvierenden dringt}{tot Elizabeth door}\\

\section{Herman Pleij}

\subsection{Uit: Van schelmen en schavuiten}

\haiku{{\textquoteright} Daar gaat hij op in,.}{en hij bedriegt de boeren}{vanwege hun geld}\\

\haiku{Wanneer men deze,.}{mensen niet kent dan zijn het}{doorgaans oplichters}\\

\haiku{Daarmee stichten ze.}{onder gehuwden grote}{angst en verwarring}\\

\haiku{Hij biedt drie of vier,.}{stuiver en zo is er dan}{weer \'e\'en bedrogen}\\

\haiku{Kooplieden, die hun.}{waren duur inkopen en}{goedkoop aanbieden}\\

\haiku{Hij, die het werk van,.}{een ander overneemt maar daar}{niets van terecht brengt}\\

\haiku{Zo blijft Aernout,.}{een ware schavuit in al}{zijn doen en laten}\\

\haiku{Ik versta de taal.}{der vogels even goed als gij}{die van uw pastoor}\\

\haiku{Een dergelijke.}{orde dienen de broeders}{in ere te houden}\\

\haiku{Meldt u derhalve,.}{in deze leerschool aan en}{weest nu allen stil}\\

\haiku{Maar hoe snel hij de,.}{koop tot stand bracht zo traag was}{hij met betalen}\\

\haiku{{\textquoteright} {\textquoteleft}Mijn lieve zoon, spot,.}{niet maar denk aan God en bid}{om Zijn ontferming}\\

\haiku{Derhalve gebood:}{hij zijn makkers om achter}{te blijven en sprak}\\

\haiku{Daarentegen was.}{hij zeer bedreven in de}{kunst van het schooien}\\

\haiku{En wanneer hij dat,.}{gevonden had moest hij hun}{dat laten weten}\\

\haiku{Ze kwamen tot de,.}{slotsom dat ze \'e\'en van hen}{zouden blinddoeken}\\

\haiku{{\textquoteright} Toen kwam de boer uit,.}{zijn bed en de pastoor liet}{de lakens wassen}\\

\haiku{Dus ging hij maar op,.}{pad zonder te weten wat}{hij precies moest doen}\\

\haiku{In Antwerpen ging.}{een schoenmaker eens uit eten}{in De Berkentak}\\

\haiku{{\textquoteleft}Breng me nog een pot,.}{bier dan komt het precies uit}{op zeven stuiver}\\

\haiku{wie niet werkt is een,.}{luie oplichter die buiten}{de maatschappij hoort}\\

\section{Sybren Polet}

\subsection{Uit: De geboorte van een geest}

\haiku{Keek haar na, deurknop.}{in de hand en glimlachte}{toen ze ook omkeek}\\

\haiku{Standaert / welcke:}{sy volghen ende lollen}{op haeren wijze}\\

\haiku{hadden zich lekker,;}{gewassen voor ze hier naar}{toe gingen met zeep}\\

\haiku{Iedereen vliegt wel!}{eens met een groen takje in}{z'n bek Op niemand}\\

\haiku{- als straf of boete?}{voor zonden die mijn ouders}{eens begaan hebben}\\

\haiku{De enorme glimlach,...}{boven mij glimlach in de}{vorm van een gezicht}\\

\haiku{want men daeromme;}{in toecomende tyden}{gheen recht doen en sail}\\

\haiku{-Waarom heb je.}{nooit plastiese chirurgie}{laten toepassen}\\

\haiku{Hij moest zich nu al.}{inhouden terwijl het spel}{pas begonnen was}\\

\haiku{daarna, op 't laatst,.}{nauwelijks nog bewegend}{bijna stilliggend}\\

\haiku{misschien kan ik straks}{in een ander te weten}{komen wat voor ras}\\

\haiku{omme gebragt te ',}{werden opt pleyn alhier}{alwaer men gewoon}\\

\haiku{deze zijn weg en;}{steekt messcherp door het lichaam}{van de bezoeker}\\

\haiku{verlengt zich nog een,}{meter of zo en veert dan}{weer terug terug}\\

\haiku{Daarnaast nog wel vijf.}{andere lichtpunten en}{-ornamenten}\\

\haiku{-In zwart wit kun.}{je ze helemaal niet meer}{uit elkaar houden}\\

\haiku{-O, niks aan de ' -.}{hand.t Was gelukkig maar}{psychies krijg pillen}\\

\haiku{alleen had ik me.}{nooit gerealiseerd dat}{ze utopies waren}\\

\haiku{-Schenk mij dan maar,,...}{eens in zei Birgitta mijn}{geest is zo aards dat}\\

\haiku{woensdagmiddag weer...}{\'een van die machteloze}{schijnvertoningen}\\

\haiku{Daar komen dan flats,.}{zoals in de Bijlmer met}{een huur van f 400}\\

\haiku{Daar geraakte de, '.}{schaafbank op de straat en de}{karel spronger uit}\\

\haiku{Wat duyvel of je '!}{wel meent de beurgers en al}{t volk te villen}\\

\haiku{hoe fyn singt dien hont,!}{nou hoe zit hij nu op met}{hangende pootjes}\\

\haiku{Jij kaerel, seg, wil,?}{je aanstonds onse maats hier}{doen koomen ofte niet}\\

\haiku{mijn kleine klauwen,.}{zijn scherpe tandjes die in}{mijn nekvel bijten}\\

\haiku{De kamer waarin.}{ik binnengelaten word}{is geweldig groot}\\

\haiku{Er komt een schuwe.}{blik in zijn zwarte ogen als}{we hem aanspreken}\\

\haiku{Ook zonder pil voelt;}{hij zich geladen met een}{grote vrolijkheid}\\

\haiku{- Liegen alsof het,.}{waar is schrijven alsof het}{werkelijkheid is}\\

\haiku{Dat Doelenhotel.}{was daarom in mijn jeugd het}{toppunt van rijkdom}\\

\haiku{Hier wonen dacht hij,.}{verder lopend en pillen}{bijna overbodig}\\

\haiku{het is echter een,.}{puimsteen welke men deze}{vorm gegeven heeft}\\

\haiku{Het gevolg was dat.}{velen hun behoefte maar}{ter plaatse deden}\\

\haiku{En over drie jaar ga,,.}{je dus met pensioen-}{Ja zei hij blij toe}\\

\haiku{Dat was de enige.}{heerlijke avond die ik in}{Amsterdam doorbracht}\\

\haiku{met \'e\'en woord, 't was,.}{het yslykste schouwspel dat}{men bedenken kan}\\

\haiku{zelfs de kleuren van,...}{het hout hebben iets grijs het}{groen en het beige}\\

\haiku{-... -Soms denk ik.}{dat de mensheid meer pijn dan}{plezier heeft beleefd}\\

\haiku{Hij had al urenlang;}{met zijn offici\"ele}{gast opgetrokken}\\

\haiku{ik moet hem zien te.}{pakken te krijgen voor hij}{het terrein verlaat}\\

\haiku{En herinneren - (:),: (:), (:),:}{etengoed en openengoed en}{opnemengoed en}\\

\haiku{Had het gevoel dat,,...}{als hij maar bleef lopen er}{niets kon gebeuren}\\

\haiku{Hij kon echter met {\textquotedblleft}{\textquotedblright}.}{zijnmeesters niet overweg en}{werd steeds lastiger}\\

\haiku{om reden dat ze.}{geen slaven hadden die voor}{hen konden werken}\\

\haiku{een kort verslag van ().}{hoe het boekongeveer tot}{stand is gekomen}\\

\subsection{Uit: In de arena}

\haiku{Zeer vreemde dingen,.}{heb ik daar gezien te veel}{om te beschrijven}\\

\haiku{zoet water zonder.}{te weten of er water}{op het eiland was}\\

\haiku{Voorts kwam men overeen'.}{dat Lucretia Cornelisz}{eigendom zou zijn}\\

\haiku{Cornelisz zelf had.}{zoals gezegd Lucretia}{voor zich opge\"eist}\\

\haiku{bij zich had hij een,.}{ton water een ton met brood}{en een vaatje wijn}\\

\haiku{De afgang van de,:}{held was nu kompleet want Pelsaert}{schreef in zijn verslag}\\

\haiku{Daarna liepen we.}{op het gebouwenkompleks}{toe en luisterden}\\

\haiku{Hij liet  ons toe.}{tot de salon en belde}{om de huishoudster}\\

\haiku{De jongen in bed.}{knikte heftig van ja en}{begon te hoesten}\\

\haiku{De jongen dacht na.}{en we lieten hem verder}{maar alleen denken}\\

\haiku{Ook Perdie ontging,.}{dit uiteraard niet maar zijn}{houding verried niets}\\

\haiku{6 De treinreis naar.}{Baywater nam niet meer dan}{twintig minuten}\\

\haiku{de moeder wordt nooit,.}{echt zelfs niet als de vader}{het kind ge\"echt heeft}\\

\haiku{de deur was op slot,.}{maar de sleutel stak er aan}{de buitenkant in}\\

\haiku{Nu keek de man op,.}{fronste eerst zijn wenkbrauwen}{en glimlachte toen}\\

\haiku{mogelijk heeft hij,.}{geen motieven ik bedoel}{eigen motieven}\\

\haiku{Hij gaat schuil achter}{de werkelijkheid die hij}{konstateert of schept}\\

\haiku{hooguit spreekt het lijk.}{de waarheid via ons of via}{onze buikstem}\\

\haiku{trouwens alles was,,.}{halfzichtbaar en leek zelfs in}{het halfdonker grauw}\\

\haiku{Wel spraken vader.}{en zoon de hele winter}{niet tegen elkaar}\\

\haiku{Op een avond keerde,,.}{Ekil en met hem twaalf man niet}{naar het schip terug}\\

\haiku{De Koeren gingen.}{daarop eten en drinken en}{waren zeer vrolijk}\\

\haiku{we gaan terug naar.}{de hoeve en zeggen hem}{wat er gebeurd is}\\

\haiku{Wanneer hij kwaad werd.}{kreeg zijn gezicht harde en}{grimmige trekken}\\

\haiku{- Niet tegen u of,,.}{tegen een overmacht zei Ekil}{maar man tegen man}\\

\haiku{Nadat Ekil aan boord.}{was gegaan hesen ze het}{zeil en gingen scheep}\\

\haiku{Op zee, ja Ekil is,.}{nu op zee varend in de}{richting van IJsland}\\

\haiku{Hij sloeg werktuiglijk.}{een kruis en riep tegelijk}{in zijn hoofd Odin aan}\\

\haiku{de orewoet zich niet.}{op maar in de aarde man}{vrouw buurvrouw dochter}\\

\haiku{Op de drempel neemt,.}{de monnik de wachtenden}{in ogenschouw broedend}\\

\haiku{Ik voel dat ik het.}{leven zal verlaten v\'o\'or}{1 januari}\\

\haiku{Of ze tonen hem.}{beter dan hij is en dan}{zijn het ook schoften}\\

\haiku{Hij staat echter niet.}{het Absolute Toeval}{toe dat hem ontkent}\\

\haiku{Pericelcius keek.}{met \'e\'en oog en ik zag zijn}{gezicht verstrakken}\\

\haiku{Hij was die hij is,,.}{in alle eeuwigheid hij}{is niet die hij was}\\

\haiku{Cellinius sprak;}{zich uit voor het verfraaien}{van het lelijke}\\

\haiku{Men keek mij bevreemd,,.}{aan behalve Brusano die}{nadenkend knikte}\\

\haiku{Toch realiseert.}{de verbeelding zich misschien}{vooral via het woord}\\

\haiku{En dacht - en haar beeld:}{werd er alleen maar aardser}{en vuriger door}\\

\haiku{Wie twijfelt moet de.}{kok z'n rug krabben en z'n}{wangen opblazen}\\

\haiku{En die het meest op.}{gewone mensen lijken}{zijn het gevaarlijkst}\\

\haiku{Deze vertrok geen,.}{spier terwijl het gezelschap}{in spanning toekeek}\\

\haiku{- Ik realiseer,!}{mijzelf via een ander via}{een keizer nog wel}\\

\haiku{- Breng ons nog een vis,,!}{een meerval of morene}{het doet er niet toe}\\

\haiku{Zijn geest behoorde.}{niet meer tot het domein der}{veranderingen}\\

\section{Servatius Josef Ponten}

\subsection{Uit: Die Bockreiter}

\haiku{Und die ganze Zeit,,.}{\"uber die du bei uns wohnst}{habe ich dich lieb}\\

\haiku{In der Nacht verliert?}{Ihr einen Finger in der}{H\"ackselmaschine}\\

\haiku{{\textquoteright} fluchte der Doktor, {\textquoteleft}?}{wo zum Teufel habt Ihr denn}{den Finger gelassen}\\

\haiku{Sonderbar, sagte,!}{er f\"ur sich wie das Kind im}{Manne nicht ausstirbt}\\

\haiku{{\textquoteright} frug Frau Elisabet.}{freundlich erstaunt und streichelte}{ihr das schwarze Haar}\\

\haiku{Hahn vom Turme der,.}{Abtei wenn es auch nur ein}{blecherner Hahn war}\\

\haiku{{\textquoteleft}Ich glaube, er kann.}{sie besser auswendig als}{die Psalmen Davids}\\

\haiku{Die Blumen in den}{G\"arten waren geschlossen}{und schliefen genau}\\

\haiku{Du erschienst oben \"uber,.}{der Dachtraufe die leichten}{Leitern nachziehend}\\

\haiku{Und wenn die Sonne,?}{verl\"oscht ist z\"undet Gott}{dann den Mond nicht an}\\

\haiku{Die Nacht ist die Zeit,.}{der Diebe und Dichter der}{Gespenster und Engel}\\

\haiku{Die G\"ange schienen.}{aus den Kehlen der Winkel und}{Ecken aufzuschreien}\\

\haiku{Auch die Nonnen und!}{Heiligen haben keine}{englischen Leiber}\\

\haiku{Alle Fenster der,:}{Gasse \"offneten sich}{und man rief sich zu}\\

\haiku{Jetzt klatschte die.}{Jauche auf und verschlang}{ein ringendes Paar}\\

\haiku{Sollen sie uns noch?}{weiter unsere gr\"une}{Jugend vorwerfen}\\

\haiku{Die R\"auber stellen.}{bei Beginn der Nacht rund um}{das Dorf Wachen aus}\\

\haiku{Durch viele blinde.}{Flintensch\"usse t\"auschen sie}{eine gro{\ss}e Zahl vor}\\

\haiku{Alles klatschte,}{und tausend grobe H\"ande}{streckten sich den wei{\ss}en}\\

\haiku{in der linken Hand.}{und die R\"uckenlehne des}{Stuhles in der rechten}\\

\haiku{So geschehe mit,,{\textquoteright}.}{Euch was Rechtens ist sagte}{seufzend der Richter}\\

\haiku{{\textquoteright} {\textquoteleft}Wenn das Gericht mehr,?}{wei{\ss} als ich selbst was will es}{dann von mir wissen}\\

\haiku{Es gab nur eine,,.}{Strafe den Galgen f\"ur ihn und}{alle Bockreiter}\\

\section{Elisabeth Maria Post}

\subsection{Uit: Het land, in brieven}

\haiku{Alleenlijk heb ik;}{mij met het nazien van de}{drukproeven belast}\\

\haiku{Welk een vrolijkheid:}{spreidt deze gedachte op}{mijn toekomstig lot}\\

\haiku{zowel wanneer zij -.}{predikt als wanneer zij op}{het toneel verschijnt}\\

\haiku{welk een gunstige!}{bestiering derhalve van}{een menslievend God}\\

\haiku{Ik zag de zwakheid,!}{van mijn lichtverleid hart dat}{zo dikwijls helaas}\\

\haiku{Nu verlang ik naar,.}{de rust het is meer dan \'e\'en}{uur na middernacht}\\

\haiku{de schone vlokken!}{dalen met een statige}{gelijkheid neder}\\

\haiku{De tedere bloem, ',:}{diet veld versiert Blijft uit}{zijn handen leven}\\

\haiku{- op hoop van spoedig,;}{antwoord geef ik hem vandaag}{aan de post mede}\\

\haiku{vergenoegdheid en.}{deftigheid waren op zijn}{gelaat geschilderd}\\

\haiku{De oude man ging.}{voort met verhalen van de}{dagen zijner jeugd}\\

\haiku{uit hun vette wei,;}{mijn paarden grinneken of}{mijn schapen blaten}\\

\haiku{Haar man poogt zoveel;}{mogelijk zijn droefheid voor}{haar te verbergen}\\

\haiku{Ik ging naar een klein,;}{somber zijvertrek waar het}{lijkje geplaatst was}\\

\haiku{uit het kiempje van.}{dit verdervend stof zal een}{Engel voortkomen}\\

\haiku{De tuinman snoeide;}{de vruchtbomen wier knopjes}{dagelijks zwellen}\\

\haiku{Ik zuchtte wel eens ';}{bijt vooruitzicht van zijn}{donkere dagen}\\

\haiku{O Eufrozyne,?}{wie zou hier niet voelen dat}{de Schepper goed is}\\

\haiku{en het zijn vette,.}{droppelen die een frisse}{geur medebrengen}\\

\haiku{Waar een ijdele,!}{wereldslaaf wanhopig is}{daar juicht een Christen}\\

\haiku{als de natuur ons;}{hoger opleidt dan vervult}{zij de ziel volmaakt}\\

\haiku{en de gehele.}{natuur kreeg een vrolijker}{gedaante voor mij}\\

\haiku{U een aangenaam;}{ogenblik te verschaffen is}{mij uren moeite waard}\\

\haiku{Oostwaards zocht ik, met,;}{mijn teleskoop de stad van}{mijn Eufrozyne}\\

\haiku{- Een gevoelige:}{ziel moet zelfs dikwijls meer dan}{gewoon vrolijk zijn}\\

\haiku{dan gevoel ik de ':}{kracht vant schone gezang}{van lavater}\\

\haiku{en deze geeft ons.}{hier de natuur in dit stil}{verblijf der onschuld}\\

\haiku{hoe sterk werken zij!}{beide om mijn gehele}{ziel te verrukken}\\

\haiku{Laat ons daarom ons:}{geheel aan de leiding der}{vriendschap overgeven}\\

\haiku{en zoudt gij tot zulk?}{een prijs uw verwondering}{wel kopen willen}\\

\haiku{Zijn schoon welgevormd;}{lichaam was de woning van}{een nog schoner ziel}\\

\haiku{- Al mijn eindige!}{begeerten waren voldaan}{in mijn Melidor}\\

\haiku{en nooit dacht ik dit.}{verlies zo lang te zullen}{kunnen overleven}\\

\haiku{Gisteren deden,,;}{wij na ons avondmaal nog een}{lieve wandeling}\\

\haiku{Hoe vriendlijk zijn uw,,,!}{bleke stralen O lieve}{zachte schone Maan}\\

\haiku{Gij kort zijn bange,.}{lijdensnachten Uw blij gelaat}{verkwikt zijn ziel}\\

\haiku{Zij gebruiken schild -, ' '.}{noch wapent Hoofd is hun}{vant schreien warm}\\

\haiku{Want hoe gevreesd ons,,?}{deze stond zij hij komt toch}{zeker en wanneer}\\

\haiku{Waar vond men toch  ,.}{een genoegen dat niet met}{verdriet gemengd is}\\

\haiku{achter ons was de;}{aarde nog door de vale}{schemering bedekt}\\

\haiku{mijn nauwkeurigste:}{teke  ning zou al het}{schoon doen verdwijnen}\\

\haiku{emilia Grootser en;}{treffender gezicht levert}{de natuur niet op}\\

\haiku{- een boer zelfs staat stil,,;}{op zijn land en ziet de Zon}{aan die het koestert}\\

\haiku{mijn vriendin had mij.}{lang begerig gemaakt om}{dit te bezoeken}\\

\haiku{Opgetogen van,.}{nieuwsgierigheid hielden wij}{hier een ogenblik stil}\\

\haiku{ziet en hoort gij in?}{dit alles niet een loflied}{voor de Algoedheid}\\

\haiku{{\textquoteleft}wordt{\textquoteright} een aanzijn gaf,,;}{werd een machteloos kind lag}{in een arm verblijf}\\

\haiku{en vooral nog korts '.}{bijt bezoeken van een}{hunner ondervond}\\

\haiku{aan de andere.}{kant een onafzienbaar vak}{van korenvelden}\\

\haiku{Hoe veel vlijt, hoe veel,!}{kunstvermogen bezitten}{deze schepseltjes}\\

\haiku{Hier bedankten wij.}{onze heuse geleidster}{voor haar gul onthaal}\\

\haiku{de half gerotte.}{vensters hingen scheef in de}{verzakte sponning}\\

\haiku{{\textquoteleft}De vossen hebben,;}{holen en de vogelen}{des hemels nesten}\\

\haiku{Het smaaklijk brood dat,;}{zij ons schenken Doet ons aan}{uwe goedheid denken}\\

\haiku{Nu beginnen de;}{eerste flikkeringen der}{morgenschemering}\\

\haiku{hoe verkwikkelijk!}{zal de gezuiverde lucht}{mij tegenkomen}\\

\haiku{het een, noch ander,.}{kan ooit anders dan in mijn}{verbeelding bestaan}\\

\haiku{slaafse banden, die!}{mij meer drukken nadat ik}{de vrijheid kende}\\

\haiku{Over enige tijd is***;}{mijn moeders voornemen met}{mij naar te reizen}\\

\haiku{maar een harmonie,.}{die ik wel gevoelen doch}{niet beschrijven kan}\\

\haiku{Gij hebt de knopjes,.}{voort doen komen Gerijpt in}{warme zomerlucht}\\

\haiku{Zo ver ik haar ken,;}{heeft ook haar karakter een}{zweemsel van het uwe}\\

\haiku{Gaarne had ik hier.}{de verrijzing der maan en}{sterren afgewacht}\\

\haiku{mijn luister zal ook,;}{groeien Als mij de zon der}{eeuwigheid bestraalt}\\

\haiku{mijn jeugd onsterflijk,.}{bloeien Terwijl ook gij met}{eeuwge schoonheid praalt}\\

\haiku{- Mijn verbeelding doet,;}{mij nog de nachtegaal nog}{het duifje horen}\\

\haiku{De roos die op uw,,;}{koontjes gloort Cefize zal}{niet eeuwig bloeien}\\

\haiku{De schoonheid is een;}{bloem die sterft Als tijd en smart}{haar blaadjes krullen}\\

\haiku{wat haar luister moog,;}{bestaan Nooit heft ze uw ziel}{tot hoger orden}\\

\haiku{Zij heeft waarschijnlijk;}{op ons zeereisje koude}{op de long gevat}\\

\haiku{- de mijne kan haar,.}{op het levenspad geen troost}{meer geven ik sterf}\\

\haiku{{\textquoteright} Zij wees mij in een,:}{lade een klein doosje aan}{en voegde er bij}\\

\haiku{En laat Sofia ook,;}{uw vriendin worden als ik}{voor u niet meer ben}\\

\haiku{Schoonder dan alle.}{bloemen is mij het mos dat}{er nu rondom groeit}\\

\haiku{De seksuele;}{component maakt immers de}{liefde onzeker}\\

\section{J. Presser}

\subsection{Uit: De nacht der Girondijnen}

\haiku{Oppassen, dat ik,.}{Schiller niet vergeet ook al}{zo'n citatenkast}\\

\haiku{Denk nu niet  *~          ,.}{alleen aan dat baardje maar}{ook aan die bloedplas}\\

\haiku{ik richt me tevens,.}{tot een meelezer die over}{mijn schouder heen kijkt}\\

\haiku{{\textquoteright} {\textquoteleft}Over u. U weet toch,.}{ook waarachtig wel hoe de}{Joden ervoor staan}\\

\haiku{Je zou je toch rot,,:}{lachen zo'n meneer Acohen}{die tegen me zegt}\\

\haiku{U vond het heerlijk, - -,.}{u u bent toch niet boos nou}{u zwelgde erin}\\

\haiku{In Westerbork, waar,.}{ik met de vacantie ben}{geweest bij Vati}\\

\haiku{Maar daar is maar \'e\'en.}{weg heen en die  heb ik}{je aangewezen}\\

\haiku{ik schreef daar al iets -,,.}{over niet weer doen zegt Jacob}{maar vergeet het niet}\\

\haiku{ik heb zelf daarna,,).}{vrouwen vlak v\'o\'or de baring}{naast de ton gezet}\\

\haiku{toen ik even inhield,:}{om te braken kreeg ik van}{Adelphi zelf een trap}\\

\haiku{Komt me op een dag -,?}{de Raaf in de les je weet}{toch wie de Raaf is}\\

\haiku{Nu had ik net een.}{juweel van een onderwerp}{voor de kinderen}\\

\haiku{Want waar hij was, daar,.}{was het gezellig in een}{kamp een zeldzaamheid}\\

\haiku{dit was de eerste,;}{joodse geestelijke die}{ik ooit had ontmoet}\\

\haiku{{\textquoteleft}Heb je zo'n hekel?}{aan de Protestanten en}{de Katholieken}\\

\haiku{Ik bedoelde het,,}{uiteraard ironisch maar \`of}{dat ontging hem \`of}\\

\haiku{Toch weet ik dankzij,.}{de kampklets wel ongeveer}{wat er gebeurd is}\\

\haiku{wat hij verkondigt,;}{is de  waarheid alleen}{de toekomstige}\\

\haiku{Ik kan alleen maar,;}{zeggen dat er sindsdien niets}{meer van me over is}\\

\haiku{Ze berust, ze is,:}{heel dapper maar tevens erg}{bezorgd om haar man}\\

\haiku{Meer kon hij me ook}{al niet vertellen en ik}{helde ertoe over}\\

\section{Sientje Prijes}

\subsection{Uit: Een bewogen vrijdag op de Breestraat (onder pseudoniem Sani van Bussum)}

\haiku{Want die was er al, '.}{den vorigen dag en is}{s nachts gebleven}\\

\haiku{Eigenlijk verlangt,,,....}{hij nu juist n\`u naar die vrouw}{die daar zoo ziek ligt}\\

\haiku{Want de zieke moet,....}{immers w\`el pijn hebben wil}{er voortgang komen}\\

\haiku{Die was niet meer in,....}{te halen die was naar zijn}{pati\"enten toe}\\

\haiku{Al gappen ze me,....}{maar een goud horloge ben}{ik al gesjochten}\\

\haiku{Hij drukte zich den,....}{hoed diep in het achterhoofd}{terwijl hij nadacht}\\

\haiku{Daar had hij willen,....}{troosten ezel die hij was en}{nog wat kwaads gezegd}\\

\haiku{{\textquoteleft}Nee, ik ga niet weg,{\textquoteright}, {\textquoteleft}.}{zei hij geruststellendik}{blijf erbij zitten}\\

\haiku{Al leg je d'r goud,,....}{bij d\'a\'ar dan kan ik n\`og niks}{door me keel krijgen}\\

\haiku{{\textquoteright} vroeg ze, toen de meid,,.}{weer beneden kwam kwaad dat}{die haar zag eten}\\

\haiku{De stem van Costa Gomez....}{donderde tegen die van}{de barende op}\\

\haiku{Gekheid, hij zal eens,:}{gaan hooren zooals men bij een}{zieke gaat hooren}\\

\haiku{De smaak was Jolie,....}{vergaan het walgde hem voor}{zijn tweede glaasje}\\

\haiku{Bezorgt u d'r 'n, ' ' ',....!}{paar mooiet komtr niet op}{an watt kost daar}\\

\haiku{n ouwen man ben, ', '}{ik niet jaloersch laatm gaan}{n ouwe bok lust}\\

\haiku{Inderdaad, hij was '.}{zoo lekker dik en vetjes}{alst maar hoefde}\\

\haiku{Zijn ruggetje was,,.}{van spek zijn zachte zijen}{nekje zuiver room}\\

\section{Arij Prins}

\subsection{Uit: De heilige tocht}

\haiku{Het z.g. beschaafde,,.}{publiek tot voor kort kende}{geen schrijver Arij Prins}\\

\haiku{D., {\textquoteleft}indertijd aan.}{Dr. ten Brink zond en waar ik}{niets meer over hoorde}\\

\haiku{Er kan geen sprake -:}{zijn van moralistische}{bedoeling laat staan}\\

\haiku{ik zijn wezen te {\textquoteleft}}{karakteriseeren in}{den aanhef van mijn}\\

\haiku{Zoo ging de ridder,.}{lange voort en wel hem was}{alsof hij daalde}\\

\haiku{, en op zijn borst was,.}{zware druk door soepelen}{last die zich bewoog}\\

\haiku{De ridder somber,.}{ging daarheene en zond zijn}{knaap met fakkel weg}\\

\haiku{Wind koelde  af.}{zijn aangezicht waarvan de}{druppelen vielen}\\

\haiku{De muren, die dit,:}{droegen hoog kleurden uit in}{wisseling van steen}\\

\haiku{De Saracenen,;}{donkere groepen die om}{de vuren lagen}\\

\haiku{Prins nam toen Alfred,:}{St\"urken in zijn firma op}{die voortaan heette}\\

\subsection{Uit: Een koning}

\haiku{De koning, alleen,.}{in een stijf hoogen zetel voor}{het open vensterluik}\\

\haiku{een streep weggaande.}{zonne-kracht boven de}{steengrijze wallijn}\\

\haiku{hij maakte  groot,.}{misbaar en wenschte zijn}{gunsteling terug}\\

\haiku{Naast hem zijn slagzwaard,,.}{zoo lang als een knaap dat hij}{niet kon hanteeren}\\

\haiku{Maar Harold al naar,,.}{de stad voor-over op zijn paard}{dat laag-draafde}\\

\haiku{Om hem steigeren,.}{en slaan met armen en beenen}{naar de dreighoeven}\\

\haiku{In zijn gulzige;}{schouwen het woninghout van}{menschen zonder vuur}\\

\haiku{Toen tweemaal openslaan,.}{in grilligheid het Boek dat}{op het graf gelegd}\\

\haiku{wel twist als \'een steeds,.}{winnen en trekken dan in}{toorn breede messen}\\

\haiku{En in den witten,.}{den nacht-dag nog menig}{man zoo omgebracht}\\

\haiku{En speren oppe.}{splinters schenen met glans van}{spitsen bovenaan}\\

\haiku{Glad de weg, die steeg,,.}{en moeijelijk het gaan dat}{was met ongeduld}\\

\haiku{Vlijm-glans van daggen,,.}{dreigend hoog in vuisten en}{bloed langs wanden droop}\\

\haiku{Ook stijve voeten.}{bloot-gemaakt door die slecht}{schoeisel hadden}\\

\haiku{schoten schaarsch, en,,.}{veel te hoog met droog geluid}{als hout dat kraken}\\

\haiku{hij kwam door hoorn van,}{een handlantaarn waarachter}{was een oud gelaat}\\

\haiku{De bode echter,.}{sprak en werd tersluiks in het}{huis gelaten}\\

\haiku{Zij dwaalden op het.}{dek met zacht ge-krab als}{van een stervende}\\

\haiku{een dichtheid grijs van.}{fijn gesprenkel in het stil}{ten-avond-gaan}\\

\subsection{Uit: Uit het leven}

\haiku{Spinoza vroeg aan?}{Janus hoeveel hij er daar}{wel mee had gemold}\\

\haiku{hare oogen waren.}{vochtig en een dikke traan}{rolde langs haar wang}\\

\haiku{Zonder door iemand,.}{te zijn opgemerkt was zij}{binnengekomen}\\

\haiku{Freek hield het paard vast,.}{dat opeens zoo mak als een}{duif was geworden}\\

\haiku{Werken ging echter,.}{niet en Oliehoek zette zich}{tegen een hooiberg}\\

\haiku{{\textquoteright} De dokter haalde,.}{de schouders op en ging weer}{naar de bedstede}\\

\haiku{Zij stemde toe, en.}{tegen zes uur des avonds ging}{het tweetal van huis}\\

\haiku{Zijn hoofd dreigde te,.}{bersten en hij viel als een}{meelzak op den grond}\\

\haiku{maar opeens kreeg hij,.}{een geduchten stoot waarvan}{hij nooit meer opkwam}\\

\haiku{Twee dagen later,.}{had hij het op de borst en}{moest te huis blijven}\\

\haiku{De patroon hield hem.}{dan ook eigenlijk alleen}{uit medelijden}\\

\haiku{Des Maandags ging Jan.}{Zomer er op uit om een}{baantje te zoeken}\\

\haiku{Hap, en 't glas is, ' '.}{leeg net alsoft inn}{laars is leeggegooid}\\

\haiku{{\textquoteright} {\textquoteleft}Dat had ik al lang,.}{gedaan als jij maar niet zoo}{sullig was geweest}\\

\haiku{Hij heeft 't weer op,.}{de borst en leit te hijgen}{als een koespaard}\\

\haiku{Een enkele maal;}{kwam een der kinderen of}{Krijns naar hem kijken}\\

\haiku{De reis werd te voet,.}{gemaakt want hij had geen geld}{om per spoor te gaan}\\

\haiku{Eindelijk, na veel,.}{hijschen en trekken kregen}{zij hem op een stoel}\\

\haiku{Hare geheele.}{persoonlijkheid had iets luis}{en onverschilligs}\\

\haiku{Zij was toch ook niet,!}{geboren om onder de}{boeren te leven}\\

\haiku{Als dat nu nog eens, '.}{gebeurt dan zal ikt van}{je loon afhouden}\\

\section{Jan Pelgrum Pullen}

\subsection{Uit: Die navolginghe Christi}

\haiku{\'e\'en enkele maal ().}{Beda en Gregorius}{de Grotecap. 10}\\

\haiku{Wat was Sijn Passie,.}{Sijn steruen anders dan}{ootmoedicheijt}\\

\haiku{dat Hij dick wils den,}{menschen dit ende  dat}{beneempt hier in proefs}\\

\haiku{ende    als hij}{sus in die gelaetenheijt}{gestelt is dan sal}\\

\haiku{Den mensch en moet daer,.}{niet sijn hij moet daer sijn}{oft hij niet en waer}\\

\haiku{Dat is seker, wort,}{den mensch inden wille Godts}{geset   soe can}\\

\haiku{{\textquoteleft}daer den mensch staet in,,}{die   puerheijt daer is}{hij in Godt ende}\\

\haiku{Den   mensch is een,}{edel creatuer sijn siel}{is geschaepen nae}\\

\haiku{dat hij Godt, Sijnen,}{Heere in alles dat Hij}{hem gegheuen}\\

\haiku{Lucam, dat Hij des}{daechs leerden inden tempel}{ende des nachs}\\

\haiku{dat hij dick sijn knien.}{boechden in sijn gebedt voor}{alle menschen}\\

\haiku{Helias was een}{mensch sterffelijck als}{wij ende hij badt}\\

\haiku{Dat is seker, waer}{den mensch niet en begeert}{hem te beteren}\\

\haiku{Siet alsoe moet dat;}{ock in ons sijn ende daer}{moeten wij   sijn}\\

\haiku{bij alle menschen.}{ende in alle steeden}{afghescheijden}\\

\haiku{comen, jae, niets niet:}{en achten ende dat}{en is niet wonder}\\

\haiku{Dan ist, dat wij hier}{duer in Godt ghetoghen}{werden ende}\\

\haiku{Hoe ghij met Christo.}{altijt opghericht sult staen in}{Godt den Vader}\\

\haiku{sij, ende soe veel,.}{haer leuen doot was   soe}{veel wast een wesen}\\

\haiku{Sij hadden, dat  .}{was in gheelheijt ghewesent}{in den wesen Godts}\\

\haiku{in God bevangen,,,.}{besloten opgenomen}{van God vervuld}\\

\haiku{het doordringen van.}{het goddelijk licht in het}{menselijk gemoed}\\
