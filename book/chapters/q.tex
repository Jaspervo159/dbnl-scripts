\chapter[3 auteurs, 1151 haiku's]{drie auteurs, elfhonderdeenenvijftig haiku's}

\section{H.P.G. Quack}

\subsection{Uit: Herinneringen uit de levensjaren van Mr. H.P.G. Quack 1834-1913}

\haiku{Maar er was een groot,.}{onderscheid die schilder doet}{het niet om zichzelf}\\

\haiku{Het werken in de.}{samenleving kon zoo schoon}{zijn en is het niet}\\

\haiku{Den aard der school heb.}{ik pogen te schetsen in}{mijn opstel over Joh}\\

\haiku{Wij plaagden wreed als -}{naar gewoonte de arme}{stakkers van Zwitsers}\\

\haiku{Doch zelfs op onze.}{vergeten school kwam er nu}{als een soort van koorts}\\

\haiku{Het profetisch lied {\textquoteleft},?}{van Isa\"ac da CostaWachter wat}{is er van den Nacht}\\

\haiku{Langs twee kanten kreeg '.}{ik een nieuwen kijk opt}{leven daar buiten}\\

\haiku{Het was wederom,.}{August Zimmerman die mij}{daarop deed letten}\\

\haiku{Doch dat was slechts de.}{uiterlijke aanleiding}{van het verschijnsel}\\

\haiku{Het knellen van 't,.}{gareel was al te voelbaar}{al te hinderlijk}\\

\haiku{Dit stond voor mij vast,.}{dat ik wakker moest zijn in}{allerlei opzicht}\\

\haiku{Die aandacht ging over,.}{tot verbazing wanneer men}{hem hoorde spreken}\\

\haiku{Voor ons, leden van,,}{het Unica dier dagen werd}{hij de leider dien}\\

\haiku{Die serenade {\textquoteleft}{\textquoteright}.}{was eenpanache van ons}{bestaan dier dagen}\\

\haiku{In mijn oogen reikte.}{hij over de eeuwen he\^en de}{hand aan Martinus}\\

\haiku{Hij, die zijn leven,.}{daaraan ten offer brengt vindt}{het ware leven}\\

\haiku{Daer eick bij eick soo,.}{vrolijck groeit Hel velt vol}{soete boeckweit bloeit}\\

\haiku{De kracht en werking.}{en wisseling dier vormen}{moest worden bepaald}\\

\haiku{Ik had zelfs geen geld.}{meer om plaats te nemen voor}{een tocht naar elders}\\

\haiku{Men mocht ook bij hem.}{niet spreken over de dingen}{die men niet goed wist}\\

\haiku{Hij was in dit en {\textquoteleft}{\textquoteright}.}{elk ander opzicht de man}{van derechte lijn}\\

\haiku{Het uitzicht op het;}{Spaarne vlak tegenover de}{groote brug boeide hem}\\

\haiku{{\textquoteleft}on ne discute,{\textquoteright}.}{pas avec ses adversaires}{on les supprime}\\

\haiku{Let wel, dat hij in.}{zijn vormen de beleefdste}{man ter wereld was}\\

\haiku{Eens slechts kreeg ik het, {\textquoteleft}{\textquoteright},.}{met mijn officieren als}{corps bijna te kwaad}\\

\haiku{Met het begin van.}{September 1861 kwam ik in}{Amsterdam terug}\\

\haiku{Bij wijlen ging hij.}{hierin tot aan de grenzen}{van uitbundigheid}\\

\haiku{Met zijn vriend van Braam.}{had Vriese een groot landgoed}{aan de Ruhr gekocht}\\

\haiku{Trouwens waar waren?}{in ons land de mannen die}{daarvoor oog hadden}\\

\haiku{Toch ging alles, zij ',.}{t met eenige schokken en}{rukken nog vooruit}\\

\haiku{Zoowel met den staat als.}{met den heer baron de Hirsch}{werd onderhandeld}\\

\haiku{Hij bewonderde.}{altijd het gezegde van}{Quesnay aan den dauphin}\\

\haiku{Ik wist thans beter,}{dan vroeger welk een zware}{taak der menschheid}\\

\haiku{Zij kwamen dikwijls}{des Zondags bij ons en wij}{gingen veel malen}\\

\haiku{Ik moest daarvoor stil,,.}{in mij zelf gekeerd zonder}{ophouden werken}\\

\haiku{Mij niet gevangen.}{te geven in het gareel}{der partijschappen}\\

\haiku{economie was slechts.}{een fragment van de leer der}{sociologie}\\

\haiku{Het zal blijken dat.}{in Holland het hart nog op}{de goede plaats klopt}\\

\haiku{Vondel teekende:}{den Hollandschen staatsman met}{het \'e\'ene woord}\\

\haiku{En wij moesten het in:}{de Gids van Augustus 1866}{ternederschrijven}\\

\haiku{In de dagen van.}{mijn secretariaat had}{ik hem leeren kennen}\\

\haiku{Maar zijn werkkring aan '.}{de staatsspoorwegen boeide}{hemt allermeest}\\

\haiku{elk zijner woorden.}{en handelingen droeg een}{Hollandschen stempel}\\

\haiku{Het was een rede {\textquoteleft}.}{overTraditie en Ideaal}{in het Volksleven}\\

\haiku{De beste krachten.}{van vroeger werden als met}{lamheid geslagen}\\

\haiku{Dank zij dat middel.}{had mijn vrouw nimmer meer iets}{van die kwaal gemerkt}\\

\haiku{Men deed het reeds voor;}{de stroo-mingen in de}{lucht en in de zee}\\

\haiku{Toen \'e\'ens de stoot,.}{was gegeven zette de}{beweging zich voort}\\

\haiku{Neen, hij was de man;}{der souvereiniteit van}{het individu}\\

\haiku{Charles Edmond zelf.}{had zich in die dagen ook}{niet veel rust gegund}\\

\haiku{Ik meende op die;}{wijze mijn vaderland trouw}{te kunnen dienen}\\

\haiku{Godin had - als echo -:}{der woorden van Fourier tot}{zich zelven gezegd}\\

\haiku{een nog niet langen:}{tijd uitgekomen boek van}{Friedrich Nietzsche}\\

\haiku{{\textquoteright} - {\textquoteleft}Das der Verbrecher,.}{sich selbst richtete war sein}{h\"ochster Augenblick}\\

\haiku{In de bureaux der.}{Bank werd hij volkomen de}{meester en leider}\\

\haiku{toen hij president).}{was geworden voelde hij}{zich volkomen thuis}\\

\haiku{Want hij was in den,.}{vollen zin van het woord een}{braaf trouw en edel man}\\

\haiku{Bank zilverbons uit,.}{te geven van f1.- f}{2.50 en f 5.-}\\

\haiku{Doch de twee eerste.}{onderdeelen behielden}{toch haar groote waarde}\\

\haiku{Elk uur van den dag,,.}{was hij aan den arbeid kloek}{slim en onstuimig}\\

\haiku{In het gezicht van.}{de haven liet hij het stuur}{aan anderen over}\\

\haiku{Nooit heb ik zulk een.}{droeven gang gedaan als op}{dien tocht naar Den Haag}\\

\haiku{Het wapen van zijn.}{woorden en daden was een}{degen voor den Raad}\\

\haiku{Op het begrip der {\textquoteleft}{\textquoteright}.}{gemeenschap kon geen beroep}{meer worden gedaan}\\

\haiku{van Kretschmar voort,.}{om aan al die verlangens}{het hoofd te bieden}\\

\haiku{Ik waagde toen reeds.}{een terugblik te werpen}{op het verleden}\\

\haiku{Het bezigen van;}{staats-geweld kan slechts een}{uitzondering zijn}\\

\haiku{Dus {\textquoteleft}verder{\textquoteright} is de,.}{leus want de maatschappij heeft}{haar schepen verbrand}\\

\haiku{Zij daarentegen.}{hadden het oog op alle}{gedeclasseerden}\\

\haiku{Toch is het - volgens -.}{Ruskin niet z\'o\'o bezwaarlijk}{dit te ontdekken}\\

\haiku{Een vriendelijke.}{herinnering blijft mij bij}{van deze bonden}\\

\haiku{Dit alles werd door.}{Izoulet merkwaardig puntig}{uit\'e\'engezet}\\

\haiku{Op die wijze had {\textquoteleft}{\textquoteright}.}{hetNut gewerkt in de eeuw}{die het had beleefd}\\

\haiku{w\`el heb ik hem bij.}{uitstek gewaardeerd en nu}{en dan bewonderd}\\

\haiku{Op de zeden en.}{gewoonten van het volk moet}{worden ingewerkt}\\

\haiku{Zijn kloek v\'o\'orgaan in.}{die Kamer is algemeen}{bekend en geroemd}\\

\haiku{zij zijn tevreden;}{als zij des Zondags hun preek}{hebben gehouden}\\

\haiku{Voor 't oogenblik.}{bepaalden wij ons tot dat}{wat voor de hand lag}\\

\haiku{Over en we\^er zouden.}{wij ons hier elkander de}{hand kunnen bieden}\\

\haiku{Les d\'elicats sont,.}{malheureux Rien ne saurait}{les satisfaire}\\

\haiku{Ik durf dus, zelfs niet,.}{fluisterend die woorden op}{mijn lippen nemen}\\

\haiku{Ik wensch voortaan liefst,,,}{te zwijgen eerbiedig als}{het kan op te zien}\\

\haiku{naar de hoogte, en.}{te blijven vereeren wat}{te\^er is en heilig}\\

\haiku{Baudriliart, H.J.L.,,,.}{prof. staathuishoudkunde te}{Parijs geb. 1821 55}\\

\haiku{Broere, prof., dichter, {\textquoteleft}{\textquoteright},.}{en wijsgeer redacteur van}{De Katholiek 41}\\

\haiku{Cervantes Saavedra,,,-,,,.}{Migu\"el de Spaansch dichter}{15471616 14 134 146}\\

\haiku{Damlust{\textquoteright}, fabriek van,,,,.}{spoorwegmate-rieel te}{Utrecht 107 111 117 196}\\

\haiku{Goethe, J.W. von, de,-,,,,,.}{Duitsche dichter 17491832 30}{87 92 214 392 noot}\\

\haiku{Hirsch, M. baron de,,,,,,,,.}{bankier te Brussel 103 111}{116 117 120 121 187}\\

\haiku{Kasteele, J.C. van,,.}{de ambtenaar aan een der}{ministeries 123}\\

\haiku{, 1824-82, 34, 92 v.,,,,,,,,,,,,.}{94 95 120 128 171 184 192}{194 234 302 335 336}\\

\haiku{Vissering, S., jur.,,,,,,,,.}{prof en minister 43 87}{126 140 141 205 335}\\

\haiku{minister, 106, 107V.,,,,,,,,,,,,.}{110 112 113 114 115 119 120}{122 137 139 189 196}\\

\haiku{Zimmerman, August,,,,,,,,,,,,.}{8 9 10 13 14 16 17}{22 23 29 40 416}\\

\haiku{{\textquoteleft}De natuur wil niet.}{het uitsluitend bezit van}{een enkelen sijn}\\

\haiku{{\textquoteleft}evenwijdig loopen.}{het recht van den mensch en het}{recht van eigendom}\\

\haiku{De lezers, evenals,}{de schrijver zullen hem den}{versctmldigden}\\

\section{Em. Querido}

\subsection{Uit: Het geslacht der Santeljano's. De verweerde jaren (onder ps. Joost Mendes)}

\haiku{De stad In 1876 was.}{Dortendam de groote stad van}{het kleine Holland}\\

\haiku{Ze ademden dieper,,,;}{en forscher ze hijgden ze}{huilden ze lachten}\\

\haiku{kon moeilijk op als.}{hij zat en ging dadelijk}{zitten als hij stond}\\

\haiku{Het was als wekte.}{haar Lot telkens heel zacht tot}{het leven terug}\\

\haiku{Mordechai's gezicht.}{was onder het vertellen}{even vaal-gebleekt}\\

\haiku{Maar dan kwam na zijn;}{stil versnikten weemoed een}{diep onmachtsgevoel}\\

\haiku{het heeleparket;}{had er lol in gehad en}{naar hem gekeken}\\

\haiku{- Heb jullie 't ook,, ...}{gehoord begon Jonas}{opgewekter nu}\\

\subsection{Uit: Het geslacht der Santeljano's. Het licht dat gloorde (onder ps. Joost Mendes)}

\haiku{Aan het andere,,;}{eind ook over en weer zaten}{de meer gegoeden}\\

\haiku{Langzamerhand was.}{de straat dicht-gebouwd en}{ook de sloot gedempt}\\

\haiku{wat niet weg ging in,.}{het gezin lapte hij er}{grof voor zichzelf door}\\

\haiku{Naar zich toe had hij.}{dat gezin gehaald met een}{overstelpenden drang}\\

\haiku{In het wezenloos;}{wegzinken van Zadok prees}{hij de bedaardheid}\\

\haiku{een wilde draai{\"\i}ng.}{en verwarrende mijmer}{ging zwaar door zijn hoofd}\\

\haiku{De wegsmuigeming.}{van hun hartelooze pret had}{Daan opslag gezien}\\

\haiku{- Nee, jij niet, kleine......}{geweldenaar dank je voor}{je Nova Zembla}\\

\subsection{Uit: Het geslacht der Santeljano's. De dorrende akker kiemt (onder ps. Joost Mendes)}

\haiku{Zwaar stapelde het.}{materiaal tegen de}{jongens zich daar op}\\

\haiku{Zoo'n dag had alles.}{dan zijn blij-na{\"\i}eve}{belangstelling}\\

\haiku{En wanneer zij hen,;}{vasthadden kwamen ze niet zoo}{gemakkelijk los}\\

\haiku{hij vulde hun  ,,.}{barre norsche werkdagen}{met stil fier geluk}\\

\haiku{Die klaarde met zijn,;}{spits vernuft hun overvolle}{wazige gevoel}\\

\haiku{Dan kwamen ze tot,,.}{hem gejaagd en hunkerend}{om hulp en uitleg}\\

\haiku{Zoo groeide, warm en,;}{oprecht hun leven om dat}{van Van Collem heen}\\

\haiku{Van Collem kreeg er.}{zijn deel in alsof hij een}{Santeljano was}\\

\haiku{Ko hield vast dan en,.}{vuurde aan Daan boog zich er}{onder en droeg}\\

\haiku{- Ko lacht, hield Jonas,,.}{gemeen-opzettelijk even}{onderbrekend vast}\\

\haiku{de mannen quasi,;}{steviger ieder in zijn}{malle eigenheid}\\

\haiku{Als op een eersten,,.}{rang zoo lekker en op zijn}{gemak zat Jonas}\\

\haiku{t waren jongens......}{die durfden gepakt waren}{door nieuwe idee\"en}\\

\haiku{- Goede morgen, zei,,.}{hij zuiver beschaafdwaardig}{naar de jongens toe}\\

\subsection{Uit: Het geslacht der Santeljano's. De revolte-dagen (onder ps. Joost Mendes)}

\haiku{materieel en,.}{geestelijk gonsde het er}{al meer en sterker}\\

\haiku{Maar al meer was hun.}{maatschappelijke macht in}{de stad gestegen}\\

\haiku{die lachte breed en,.}{juichte zacht zijn oogen diep van}{glans en aandoening}\\

\haiku{secuur gedaan met.}{zijn naarbuiten geperste}{gladgestrekenheid}\\

\haiku{Niet op het gebied;}{van de uitbuiting was dit}{compromittante}\\

\subsection{Uit: Het geslacht der Santeljano's. Het wonderschone rijpen (onder ps. Joost Mendes)}

\haiku{Hij had gezien dat,}{Ko niets wist enkel leefde}{in den brandenden}\\

\haiku{Dan verstierf al het;}{kabaal van Ko's wetenschap}{en werd hij heel stil}\\

\haiku{de groote taart al 's ':}{morgens v\'o\'or achten gebracht}{metn rijmpie}\\

\haiku{Ze sprak weinig, vroeg,;}{niets luisterde enkel en}{was vol zorgzaamheid}\\

\haiku{En nooit was in haar.}{bewegen de inspanning}{der veerkracht merkbaar}\\

\haiku{Even, \'e\'en moment, greep}{Daan dan alles wat er in}{zijn ziel aanliefde}\\

\haiku{op het stadhuis was,.}{geen luchie meer het stonk nog maar}{alleen ter plaatse}\\

\haiku{Vogelendaal was.}{voor de jongens het blije oord}{van hun blanke jeugd}\\

\haiku{Hij gr\'e\'ep niet naar de, ...}{dingen die hij wilde hij}{stortte zich er op}\\

\subsection{Uit: Het geslacht der Santeljano's. 's Wasdoms volle tooi (onder ps. Joost Mendes)}

\haiku{zijn voorhoofd vluchtte}{als in verbijstering met}{een vaart naar achter}\\

\haiku{te zwaar drukkenden.}{last van al die vereering en}{aanhankelijkheid}\\

\haiku{De Dort kostte Ko,.}{halve dagen zoover liep hij}{langs de oevers door}\\

\haiku{Want het geweldig.}{torsen van de ijs-Dort}{was het groote wonder}\\

\haiku{- Goed Kootje, goed, gaf, ...... -}{Balront innig toe maar waarom}{ons niet meegevraagd}\\

\haiku{Ze lachten als hij,...}{lachte keken op slag weer}{stroef als hij het deed}\\

\haiku{Pekel en Koe strije...,... -...!}{z{\`\i}j zegge ijs is niks als}{water Waterstof}\\

\subsection{Uit: Het geslacht der Santeljano's. 's Werelds daverende wedloop. Eerste boek (onder ps. Joost Mendes)}

\haiku{Zij was er niet voor,.}{de dingen maar de dingen}{waren er voor haar}\\

\haiku{Haar destructieve,;}{desorganiseerende}{geest vrat alles aan}\\

\haiku{Toen had Myriam,,.}{haar oom Jacques Pardo op}{kantoor genomen}\\

\haiku{Als u werken wilt, ',...}{neemt u maarn schoon doekje}{d'r staan  er wel}\\

\haiku{... en ik vermoed, ik ',, - '}{weett niet ik gis maar hoor}{s menschen wegen}\\

\haiku{zij lachte zich de,,.}{tanden blinkend bloot hij bleef}{zonder krimp immuun}\\

\haiku{En met een vaart sloop,.}{hij ineens toen de kamer}{door op den haard aan}\\

\haiku{Geen levend wezen,.}{met gevoel gedachte en}{spraak verkozen ze}\\

\haiku{maar zij, lachend er,.}{op bedacht was ineens weer}{door het deurtje weg}\\

\haiku{Nu ging ze zich voor,.}{de koffie kleeden riep ze}{Ko in haar kamer}\\

\haiku{Heel het personeel,.}{de familie en de kring}{lagen voor hem neer}\\

\haiku{In haar kloek en fraai,;}{imponeerend handschrift schreef ze}{dozijnen brieven}\\

\subsection{Uit: Het geslacht der Santeljano's. 's Werelds daverende wedloop. Tweede boek (onder ps. Joost Mendes)}

\haiku{En na een poos, toen,,.}{het stond een vesting geleek}{het een citadel}\\

\haiku{Hij voelde zich een.}{door het socialisme}{direct bedreigde}\\

\haiku{- Dit alles was de.}{bezielende glans hunner}{vergaderingen}\\

\haiku{Nu vorderde het.}{algemeen aanzien van het}{groote gild met den dag}\\

\haiku{Het huis, allereerst,,.}{was nu opengezet wijder}{dan ooit te voren}\\

\subsection{Uit: Het geslacht der Santeljano's. 's Werelds daverende wedloop. Derde boek (onder ps. Joost Mendes)}

\haiku{En de rooien, ze ....}{lachten enverdubbelden}{hun heilzame hel}\\

\subsection{Uit: Het geslacht der Santeljano's. De hooge lichte kim der stilte (onder ps. Joost Mendes)}

\haiku{opheffing van De,.}{Fakkel of royement de}{zaal ingestooten}\\

\haiku{met derzelver hoog,!}{geroemden raken dialoog}{merde verdomme}\\

\haiku{Van vrijwel alles,;}{wat gebracht was had hij het}{record verbeterd}\\

\section{Isra\"el Querido}

\subsection{Uit: De Jordaan: Amsterdamsch epos. Deel 1}

\haiku{Bi j\'ei d'r d\`en noar,.... '!}{Folled\`em s\`el me doar auk}{n leindertje sa\`an}\\

\haiku{- - Sau'n kern\`ek,.. beet de.}{aalman Manus nijdig den}{kant van Karel uit}\\

\haiku{Bang door het geweld.}{keken de burgers benauwd}{uit naar politie}\\

\haiku{- Sch\`af je snurkert,.... op ' '!}{t petrauleim-f\`et}{leit nogn snei braud}\\

\haiku{huylklep.... en nau de,.....}{steine op en ploag d'r nau}{je schorremorrie}\\

\haiku{Toorn en twist waren.}{de dreigendste gedaanten}{van hun hartstochten}\\

\haiku{tr\`ep je foader nie....}{op se pink en staur s'aage}{niet in se tukkie}\\

\haiku{lachte hij weer en.}{stapte met een streel onder}{de kin van Bet weg}\\

\haiku{Ook nu weer weende,}{ze haar woorden uit terwijl}{ze toch zoo zoetjes}\\

\haiku{Hij slorpte zoetjes.}{zijn koffie in en keek weer}{peinsdiep naar den grond}\\

\haiku{Voor haar met Frans had.}{de meid net zooveel ontzag}{als voor een deurknop}\\

\haiku{Ze stonden in haar.}{insteekje en ze zag ze}{de h\'e\'ele week v\'o\'or zich}\\

\haiku{Ze jammerde haar.}{leef-ellende in alle}{bizonderheid uit}\\

\haiku{En niet \'e\'en keer in,.}{de week maar iederen dag}{slurpte hij zich zat}\\

\haiku{Ze herinnerde}{zich precies al de knusse}{babbeltjes tusschen}\\

\haiku{- Stijn zwierf nu al 's,;}{nachts op zee met de vlet de}{kwakken tegemoet}\\

\haiku{ze zou ze spuwen.}{op hun kliergezichten en}{moes van ze trappen}\\

\haiku{Stijn had Thijs beet die.}{zich telkens wou opwerken}{en op Neel smakken}\\

\haiku{Hij voelde nu zelf.}{dat hij er afschuwelijk}{gemeen moest uitzien}\\

\haiku{dat was zijn leven,.}{nu de verontrusting van}{hem was \'afgetild}\\

\haiku{Stijn stopte hem het.}{tabakszakje toe en keek}{kalm over het water}\\

\haiku{Thijs gromde woedend,,.}{wat terug maar aarzelde}{liet het zuipje los}\\

\haiku{Je zoudt ze tegen.}{hun dekschaal de eeuwigheid}{wilen intrappen}\\

\haiku{als die kwak maar zijn....}{zeiltjes wou draaien en te}{loefter oversteken}\\

\haiku{In een vuil-goor.}{onderhemd was hij blootshoofds}{de straat opgehold}\\

\haiku{Ze hadden allen.}{in de buurt gloeiend het land}{aan Dien's gierigheid}\\

\haiku{lachte Neel vadsig,.}{haar kleffe handen over haar}{hooge buik afstrijkend}\\

\haiku{Ze leek van binnen,.}{wel een doodzerk zoo drukkend}{stil was het in haar}\\

\haiku{Wil je haar beter,.}{hebben doe dan alsof je}{gelooft wat ze zegt}\\

\haiku{- Mie de Roeier kwam.}{weer van het plaatsje en de}{wasch-lijntjes}\\

\haiku{Plots schreeuwde ze naar.... -,,....}{Mie de Roeier Mie ka\`ak doar}{slentert je ma\`ad \'an}\\

\haiku{Groote bekommering.}{voor ziekte en droogte sloeg}{angst in de harten}\\

\haiku{Dronkenschap, armoe.}{en vuil besmetten al hun}{daden en woorden}\\

\haiku{Ze werden ontzien {\textquoteleft}{\textquoteright},;}{alsnette menschen door het}{gemeenste crapuul}\\

\haiku{mit achttien stuyfer.... -,}{f'rdient tie d'r achttien Nou}{en ikke seg d\`et}\\

\haiku{- Vlak voor de Wijde,.}{Gang hield een zwarte koets stil}{in de Willemstraat}\\

\haiku{Maar hij deed niets en.}{zijn schelden krijtte zich dood}{op eigen onmacht}\\

\haiku{Al dat kletsende?}{janhagel wou zijn jool nu}{in wroeging verkeeren}\\

\haiku{Een onbeschaamder.}{vrouwspersoon had hij in zijn}{leven niet ontmoet}\\

\haiku{- - Blinde Jaonus,....;}{so\`agt se luchtkesteile lichtte}{de gelagbaas in}\\

\haiku{Weer draaide Karel.}{bij en strafte den sleeper}{met hoonende woorden}\\

\haiku{Benauwde luchten.}{van peulen en bleekpoeier}{mengden zich dooreen}\\

\haiku{Nou speet het Karel.}{dat hij zijn harmonica}{niet had meegetild}\\

\haiku{Dien sproetneus zou hij.}{hebben laten hinken en}{springen als een clown}\\

\haiku{Traag slenterden, bij,.}{stoetjes achter elkaar de}{vrinden de markt op}\\

\haiku{Eindelijk was het.}{trotsche dienstmeisje door een}{bres heengedrongen}\\

\haiku{Ze lachten elkaar,.}{toe in verdekte schaamte}{overmoedig van woord}\\

\haiku{Nou moest hij wel zijn.}{brandendst begeeren naar die meid}{weg-leugenen}\\

\haiku{Een heetige koorts.}{van verlangen groeide \'a\'an}{in den menschendrom}\\

\haiku{in zijn armen, beenen,,.}{romp hoofd en handen werd de}{muziek lijn en stand}\\

\haiku{loat ik nou jo\`a '!....}{n biggetje kra\`age die}{d'r me b\`uyk uytd\`enst}\\

\haiku{De Lindengracht lei.}{bekrioeld van dansende}{meiden en kerels}\\

\haiku{het was zoo een gul.}{wereldje bij tante Nel}{uit de Goudsbloemstraat}\\

\haiku{Vlug was haar hulp, rad.}{haar tong en guitig keken}{haar dartele oogen}\\

\haiku{Maar als de drank in.}{hem gistte wist hij nooit wat}{hij eigenlijk deed}\\

\haiku{Ze verlangde toch?}{geen lust-slot en geen tuin}{met karseboomen}\\

\haiku{En plots weer met een.... -,....}{luimig lachje versleepte}{hij zijn woordjes Uch}\\

\haiku{Haar heele lichaam.}{voelde ze drijven in een}{heete vochtigheid}\\

\haiku{Moest ze zich nou weer,?}{zuur denken om het geld n\'a}{de negen dagen}\\

\haiku{- Die m\`elligha\`ad m\^o '!....}{jen Bokkebek nie an}{se geifel h\`enge}\\

\haiku{Werkeloos,.... \'eten van,!}{zijn vrouw's standje van wat zij}{bijeen had gezwoegd}\\

\haiku{Door d\'at en door die,....}{feuilletonsche boeken had}{ze Gronjee na\'ast haar}\\

\haiku{Neel, nog een beetje,.}{erger bijgeloovig dan}{anders schrok ervan}\\

\haiku{Hij baggerde door,,.}{door zonder te zien waar hij}{eigenlijk heenging}\\

\haiku{Alles deinde nu.}{op het rustige rhythme}{van haar ademhaal mee}\\

\haiku{Vreemd,.... nu beklemde.}{haar niet meer de gedachte}{aan zijn werkloosheid}\\

\haiku{Mientje luisterde,.}{nu naar een comedie niet}{zelf door Lien gezien}\\

\haiku{Mientje had den zin.}{voor plagerigen humor}{van Neeltje in zich}\\

\haiku{waspit = Jordaansch.}{meisje van de vroegere}{waskaarsenfabriek}\\

\haiku{pl\`ekje van rood.}{haar opleggen = je kop}{tot bloed ranselen}\\

\subsection{Uit: De Jordaan: Amsterdamsch epos. Deel 2: Van Nes en Zeedijk}

\haiku{Corry's aandacht was.}{alweer afgedwaald naar een}{bokking-venter}\\

\haiku{Joden Jet bleef koers,.}{houden begon weer met haar}{lokkende koopjes}\\

\haiku{Maar wat had ze een...!}{goddelijk blank velletje}{en wat een knar haar}\\

\haiku{Ze had nog nooit zoo.}{een aanhaalderigmooie meid}{uit het volk gezien}\\

\haiku{De blanke boezems.}{der vrouwen fonkelden van}{valsche juweelen}\\

\haiku{In tijden had ze.}{van zulk een mak handeltje}{niet z\'o\'o genoten}\\

\haiku{8Ze schold en keef met,.}{haar compagnon ze trapte}{op haar inkoopen}\\

\haiku{Ja toch, ook om zijn.}{snapsie vond hij het leven nog}{wel de moeite waard}\\

\haiku{Een mensch moest als de.}{wind ongezien verschijnen}{en ongezien gaan}\\

\haiku{Dat is jonkheer Van... -,,....}{Overhoof Soo edelaardige}{Dirk schertste Peet door}\\

\haiku{De wrok groefde de.}{vette rimpels nog dieper}{in zijn laag voorhoofd}\\

\haiku{Jij smakkert,... jij bint.........!}{t'r ijser ballast schot en}{tonne bij mekaar}\\

\haiku{- W\`a sou 't!... zei Jaap,......}{geraakt en haastig slikte}{hij zijn borrel ik}\\

\haiku{'n krats! - Hoeveel?... vroeg,.}{Jo haar gepoetste nagels}{zacht beademend}\\

\haiku{Ze schaterden valsch.}{en een gniepige drift vrat}{Jo's mooi gezicht op}\\

\haiku{Schuw kwam Toontje op.}{en drillend dwongen ze hem}{-te luisteren}\\

\haiku{... gierde Jo naar de.}{koppelaarster en trapte}{een hoedendoos in}\\

\haiku{Toen vertelde de.}{Rooie dat zij altijd te laat}{kwam op de b\^uhne}\\

\haiku{- De laatste van 't...!}{jaar bij de burgemeester}{op de bouleva\'ard}\\

\haiku{Ze wou, wo\'u en geen.}{enkel ondersmoord begeeren zou}{haar doen wankelen}\\

\haiku{Een mensch als zij, kon.}{nou niet maar ineen goud uit}{een hoorntje drinken}\\

\haiku{Ze had nooit in haar,.}{gansche leven zoo gevloekt}{gehuild en gesnikt}\\

\haiku{Dien wou terug en.}{stapte op de tram bij een}{lijn-driehalte}\\

\haiku{Met zijn duiven kon.}{hij veel beter omgaan dan}{met zijn kinderen}\\

\haiku{Een kluchtspel, al dat,,.}{grut dat ze g\'af terwijl ze}{zelf betalen moest}\\

\haiku{Haar ruim vierjarig.}{Leendertje was toch zoo een}{zoete lieveling}\\

\haiku{Een zware sigaar.}{bungelde losjes tusschen}{zijn kinderlippen}\\

\haiku{Annetje zat vlak.}{bij het raam inspannend een}{brok krant te lezen}\\

\haiku{Als God toch haar kind,.}{in het ongeluk stortte}{dan liever nog zo\'o}\\

\haiku{Als die meid niet van,.}{kersen en honing leefde}{dan deugde het niet}\\

\haiku{Nee, ze groef door heel.}{andere gangen in haar}{donker binnenste}\\

\haiku{Hoe had haar eigen.}{meisjeshoogmoed haar zelf niet}{geplaagd en gekweld}\\

\haiku{ze was, hoe ze den,}{slaap van zijn oogen roofde hoe}{krankzinnig verliefd}\\

\haiku{hij zou haar smeeken,...!}{all\'e\'en voor h\'em te leven}{alleen h\'e\'el alleen}\\

\haiku{duwde hij liever!}{zijn twee klavieren in het}{kokende water}\\

\haiku{Het was er stampvol.}{en de menschen zeurden rond}{hem de honderd uit}\\

\haiku{Nou zou hij nog even.}{op zijn bed blijven rekken}{en ren- gelen}\\

\haiku{Wat had hij de rooie.}{kanes van Mie nou toch vlak}{voor zijn oogen gezien}\\

\haiku{De meid had er een,}{kleur van gekregen toen ze}{het zag z\'o\'o prachtig}\\

\haiku{Manus kende geen,.}{guller hartelijker en}{meewariger vent}\\

\haiku{Waarom bleef hij bij?}{Joden Jet als hij haar toch}{zoo tegenwerkte}\\

\haiku{dat hij voor zichzelf,.}{niets te bedisselen had}{noch iets te ontzien}\\

\haiku{En Manua werd het.}{een troost te weten dat het}{toch nog geen herfst was}\\

\haiku{maar toch kon ieder.}{mensch een stemmetje geven}{als hij in nood zat}\\

\haiku{Een sterker zagend,,.}{gesnork als hoorde ze zijn}{spot was het antwoord}\\

\haiku{En weer geeuwde ze.}{vervaarlijk naar de kamer}{toe waar Manus zat}\\

\haiku{... suste Jet weer en {\textquoteleft}{\textquoteright}.}{meteen duwde ze zacht haar}{vangst de kamer in}\\

\haiku{Ze leek hem wel een.}{beetje erg beverig en}{zeer kort van begrip}\\

\haiku{Zijn heele klucht met.}{dit juffie kon hij morgen}{vergeten wezen}\\

\haiku{Corry was woedend.}{dat de Bochel haar niet eens}{had aangekeken}\\

\haiku{... herhaalde Corry,.}{giftig stampend dat ze niet}{verder kon komen}\\

\haiku{ze den vorigen.}{avond ieder voor tien gulden}{hadden afgezet}\\

\haiku{Corry leek zoo lang,,.}{als zij zag ze maar tienmaal}{mooier en blanker}\\

\haiku{Over alles ademde,.}{de losse wellustige}{bandeloos- heid}\\

\haiku{De koppelaarster {\textquoteleft}{\textquoteright},.}{wou voor de meid in hetstof}{knielen zei ze zelf}\\

\haiku{Ze beloofden haar {\textquoteleft}{\textquoteright}.}{eengoeden nacht en zooveel}{drank als ze lustte}\\

\haiku{De andere liep.}{in het grijs flanel met een}{groote bloem op zijn borst}\\

\haiku{- Zeg liefert,... hoorde,...?}{ze Gonda zeggen mag ik}{effe anschuife}\\

\haiku{- Steek jij maar je borst!...,... '!}{op joolde hij voort wantn}{kop heb je t\'och niet}\\

\haiku{De maintenee Fucht {\textquoteleft}{\textquoteright}.}{sprong plots bij en zong van de}{Happies days in Dixie}\\

\haiku{Ze begluurden hem,.}{ademloos maar waagden nimmer}{den kleinsten uitval}\\

\haiku{De neger had haar.}{opgehaakt en neergesmakt}{als een vuil pak kleeren}\\

\haiku{Het seil, mamselle,...,......}{is de Liefde en de Deugd}{en het anker}\\

\haiku{Ze knikte, maar in.}{stilte nam ze zich voor geen}{slok meer te nemen}\\

\haiku{De neger had zijn.}{heer weer doodomzichtig op}{een stoel neergezet}\\

\haiku{Joden Jet grijnsde,.}{en zong mee stomdronken van}{al de champagne}\\

\haiku{En fijner plaagde,:}{Corry terug overzeker}{van haar overwinning}\\

\haiku{En ieder moest zijn.}{rol worden toebedeeld zooals}{zij dat verlangde}\\

\haiku{Neen, ni\'et veel, nog,....}{niets meer dan afgepast kreeg}{ze geen achterdocht}\\

\haiku{as t'r \'e\'en manke '...!}{vlieg int land is gaat tie}{op jo\'u neus sitte}\\

\haiku{dat ze den vent moesten!}{rossen tot hij er half dood}{bij te krimpen lag}\\

\haiku{Nooit vroeg hij iemand.}{op zijn krot en ook nimmer}{bezocht hij een buur}\\

\haiku{Snikkend lei ze wat,.}{neer drukte hem de polsen}{en ging toen snel weg}\\

\haiku{- Soo achtbare heer,....}{spotte Karel op den knik}{van den souteneur}\\

\haiku{Een bloedstroom gulpte,.}{door het haar over de schouders}{en op de keien}\\

\haiku{Dan zouden ze een.}{potje rollen waarvan de}{lui konden roojemen}\\

\haiku{Maar die twee waren;}{van avond op nieuw avontuur de}{stad ingekuierd}\\

\haiku{- En se spinnekop!....}{sat nog op se das wist Guus}{Sand te vertellen}\\

\haiku{- W... \`at?... aarzelde,.}{Willem zijn spel-stomheden}{nog zelf niet overziend}\\

\haiku{- Je krijg nog 'n cent,...!}{fan d'r mag se morrege}{veur komme singe}\\

\haiku{Annemie was de,.}{buur van Dien en Blonde Kee}{woonde onder Nel}\\

\haiku{Hij kon geen brood meer,!}{verdienen dan moesten het maar}{de kinderen doen}\\

\haiku{Op het Oudekerksplein.}{bleef hij dralen zonder het}{zelf te beseffen}\\

\haiku{Treuzelend begon.}{ze te verhalen en haar}{stem zonk heesch in}\\

\haiku{op de lat. Cobus.}{wist precies van de kerels}{hoever hij kon gaan}\\

\haiku{Anne Looy's geelbleeke,.}{gezicht staarde weer den Dijk}{op zonder te zien}\\

\haiku{Joden Jet pakte,.}{weer in met het eeuwige}{stroohoedje op haar hoofd}\\

\haiku{Op Lou wachten,... nee,,.}{dat wierd te wisselvallig}{kon te lang duren}\\

\haiku{- Doch je d\`a 'k nie ' '?}{asn stammene\'e achter}{t vlot kan komme}\\

\haiku{- Doch je... doch je... d\`a ' ' '?}{kn salmpie l\'a uitsuige}{fann palinkie}\\

\haiku{Sterven vond Piet een,.}{ijselijkheid al hield hij}{niks van het leven}\\

\haiku{Anders zou hij wat,...!}{beleven om op die kist}{te durven zitten}\\

\haiku{Waarom de spullen?}{dan niet geborgen waren}{in kasten of zoo}\\

\haiku{Ze zou er koud van.}{hebben kunnen worden als}{ze dat al niet was}\\

\haiku{Toen ging ze, met een.}{hooge stem Karel overdreven}{hartelijk groetend}\\

\haiku{mi Moed, Beleid en... ' '!}{Trouw kompern mensch aln}{heel end achterop}\\

\haiku{Dien kreeg een hartbons.}{toen Karel met een ooiijk}{gebaar instemde}\\

\haiku{Frans Leerlap kwam met.}{zijn sleependen jichtvoet van het}{duivenplat hinken}\\

\haiku{Wat h\'ede suipt, falt,,!}{gistere in de Kuil Foor}{oud fuil Foor oud fuil}\\

\haiku{blauwkop,... lachte weer,....}{Karel en l\'a je vrou nie}{in de drek sitte}\\

\haiku{Ze wisten immers.}{niet eens wa\'ar en met wie ze}{gevochten hadden}\\

\haiku{Door een troep beesten,...!}{van kerels gehavend door}{meiden verraden}\\

\haiku{Ze vertelde dat,.}{er vijf door de politie}{waren opgepikt}\\

\haiku{Een paar uur later.}{ke\'erden de gedachten}{en plannen opnieuw}\\

\haiku{Zoo verbijsterd en.}{wild had  hij hem nog nooit}{om een deern gezien}\\

\haiku{Wa\'ar je haar roerde,.}{brandde een creatuur als}{hij zich de toppen}\\

\haiku{Zo\'o kroop hij den Dijk,}{weer op waar hij zooveel wind}{ving voor en aleer}\\

\haiku{- Soo'n warmte,... bromde,....}{hij ferbeel je je eige}{an de keerkringe}\\

\haiku{... stootte Simbad er,.}{nijdig uit die voorvoelde}{waar Manus heen wou}\\

\haiku{Ik wacht  op 'n,......}{bekende jonge h\`e die}{me in de flank falt}\\

\haiku{n lachie,... 'n rookie ',... '...}{enn babbeltje mitn}{schorpioen d'r op}\\

\haiku{Hoe zou nou zoo een?}{gouden vlinder h\'em ooit op}{de hand gaan zitten}\\

\haiku{- 't Varen was niks... ',....}{gedaann Beroerd beetje}{leven bromde hij}\\

\haiku{Zag hij nu heusch,?}{niet dat die deern geen portuur}{was voor een matroos}\\

\haiku{Hij wierd niet gauw week,.}{in het hart maar d\'at zou hem}{toch laten grienen}\\

\haiku{En 't skip nam toen '!}{n draai An de kop fan de}{handelskaai Skip \'ehooy}\\

\haiku{die vigeleerde!...}{als ze er lust in had of}{voor eigen nood moest}\\

\haiku{Wat schreeuwde, schermde,.}{of vloekte dat \'overschreeuwde}{of \'overvloekte ze}\\

\haiku{Ze deinsde voor niets,.}{terug al werd het een strijd}{op leven en dood}\\

\haiku{ze verlangde naar,.}{een handdruk een gebaar van}{mekaar-verstaan}\\

\haiku{Het bleef zacht in haar.}{roepen en kreunen naar}{181bevrediging}\\

\haiku{Even lachte ze, het,.}{bleeke mooie gelaat zacht-sarrend}{naar hem toegekeerd}\\

\haiku{Al wat danste, zat,.}{en stond rende naar den hoek}{waar gevochten werd}\\

\haiku{De blonde zou hij.}{op een andere manier}{wel inpeperen}\\

\haiku{Voor de zooveelste.}{maal dwong Karel den kerels}{bewondering af}\\

\haiku{We\'er gaf hij haar een.}{ruk dat ze nog lager voor}{zijn voeten uitzonk}\\

\haiku{mannen die leven.}{van het geld dat een vrouw met}{haar lichaam verdient}\\

\haiku{losse werklieden,.}{met een boekje van het Veem}{genieten voorkeur}\\

\subsection{Uit: De Jordaan: Amsterdamsch epos. Deel 4: Mooie Karel}

\haiku{De Jordanertjes.}{schreeuwden alsof de heele}{wijk in brand laaide}\\

\haiku{De heimelijke;}{v\'o\'orvreugd van Pinksteren moest}{in Luilak gisten}\\

\haiku{Want van die vlotschuit,,.}{7Mooie Karel hadden de}{rekels niet terug}\\

\haiku{En Alie, haar meid van,?}{zeventien lei die ook nog}{vadsig op haar stroo}\\

\haiku{Zijn laatsten spaander,.}{schoot hij uit als hij zag dat}{ze maar \'even grienden}\\

\haiku{Voor een paar dagen,...}{althans nijp-zorgen weg}{en kommer om Greet}\\

\haiku{Waarom poekelde?}{Karel eigenlijk niet met}{die kluif-gasten}\\

\haiku{je swarte kouse... '!}{en je rooje petoffels en}{m\'e\'et ope erf op}\\

\haiku{Want nou leken al,;}{zijn kracht en behendigheid}{ingetoomd geschoold}\\

\haiku{Bij Corry viel er.}{niets te redderen van haar}{intiemste wezen}\\

\haiku{Het bedrog van zijn.}{hart hield hij niet langer vol}{tegenover Corry}\\

\haiku{Mooie Karel kende.}{geen beuzelingen en geen}{onverdraagzaamheid}\\

\haiku{- Slurp je m\'e\'e, 'n slok,...?}{uit me flesch pofsak of bi je}{fol op je borsie}\\

\haiku{n natte klap op!}{je kruintje geife d\^a je}{ooge uitsplintere}\\

\haiku{- Die komp bij mijn niet,... '!}{goed dan haal ikm tug s\'o\'o}{se hart uit se lijf}\\

\haiku{- Fan je mollege... -!...}{arrempies En ikke fan}{je lekkere grens}\\

\haiku{Karel hield zijn grauw.}{petje op en bedelde}{den kring verder rond}\\

\haiku{Karel stapte weg.}{en bekommerde zich nauw}{meer om den zatlap}\\

\haiku{Thijs snurkte zwaar als,;}{een dronken bruigom in de}{duistere bedste\^e}\\

\haiku{s Morgens al om.}{klokke-vier schoot Karel}{met een schrik wakker}\\

\haiku{Maar even daarna wist;}{hij niet eens meer waarom hij}{was losgebarsten}\\

\haiku{Greet, al dieper en,.}{schrijnender van spijt gekweld}{sou maar naar bed gaan}\\

\haiku{Ze schenen compleet.}{engelen van goedheid en}{inschikkelijkheid}\\

\haiku{Ze was van boven,...}{tot onder pluim en bel. Ja}{een pantemime}\\

\haiku{Greet voelde zichzelf,,!}{zoo ernstig zoo schrikkelijk}{schrikkelijk ernstig}\\

\haiku{En waar de schat lei,.}{daar kronkelden helblauwe}{vlammen uit den grond}\\

\haiku{En de klantjes van.}{de Frans Halsstraat hielp zij schuw}{en gedachteloos}\\

\haiku{Doordringender en.}{nijpender keek de malle}{kastelein hem aan}\\

\haiku{De commissaris.}{preekte saai en gluurde loensch}{naar vader en zoon}\\

\haiku{Tweemaal per week moest.}{hij een heelen avond bij hem}{komen doorbrengen}\\

\haiku{Had hij nou maar een,...}{eindje cigaret om te}{trekken te zuigen}\\

\haiku{Hij tikte Zacht, iets,,:}{harder nog iets harder t\'ot}{hij zekerheid kreeg}\\

\haiku{Hij vreesde van de,.}{dieven en inbrekers om}{hem heen geen verraad}\\

\haiku{Zij stond te snikken:}{voor den commissaris en}{jammerde al maar}\\

\haiku{En nog was er geen!}{koopster te vangen onder}{een strooien dakje}\\

\haiku{Je laat 'r je schaats,,!...}{je hengels je danse en}{je duife foor staan}\\

\haiku{En tartend met zijn,,:}{heerlijk-buigzame hooge}{mannenstem zong hij}\\

\haiku{ferslikke se d'r '!}{eige fast mitn falsche}{guide in d'r keel}\\

\haiku{En madam Bedil,.}{was z\'elf een gifmengster als}{het er op aankwam}\\

\haiku{Greet had een brief in,;}{de  hand pas door den post}{haar schoot opgegooid}\\

\haiku{Guns, wat had ze met,!}{zoo een blok aan haar been aan}{haar jonge leven}\\

\haiku{Heel de Jordaan was,.}{op de been nog kooplustig}{in de smoorhitte}\\

\haiku{- Bij jo\'u jonke de ',?}{pietermanne achtert}{stille luikie h\`e}\\

\haiku{Meiden en kerels,.}{bulderden van het lachen}{tierden en spotten}\\

\haiku{De Goudsbloemdwarsstraat,;}{als kruispunt ving van allen}{kant het buurtrumoer op}\\

\haiku{- Kopsorg... as 't '!}{n slachterij wordt bin ik}{d'r tug s\'o\'o te piel}\\

\haiku{Stil draaide hij zich.}{om en stapte voet voor voet}{het kamertje in}\\

\haiku{Karel weerde het.}{wellustige gesmeek der}{meiden boertig af}\\

\haiku{En hij danste op,.}{zijn stoepje Greet woest in een}{warrel meetrekkend}\\

\haiku{En even, tusschen de,:}{bekoorlijke tandjes zong}{zij nazinnend}\\

\haiku{... jammerde Teun uit,,...}{bed naar Ant die stil op een}{kruk was neergesakt}\\

\haiku{Hoe lam-ellendig.}{voelde Karel zich onder}{al de scherts en boert}\\

\haiku{Gebbetjes van 'n,......}{kitsige adviseerde}{Poort langs den neus weg}\\

\haiku{Harmen, ja di\'e had;}{hem eigenlijk naar alle}{slechtigheid gejaagd}\\

\haiku{- 's Morgens kwam de.}{cipier en we hoorde de}{sleutels rinkele}\\

\haiku{Hij rook niks anders.}{dan poffertjes en wafels}{van een kermiskraam}\\

\haiku{Zoo een halfblanks vent,.}{was waard dat hem de duim in}{de oogen gedrukt wierd}\\

\haiku{Frans zag de wanhoop.}{en het kwaad geweld op zijn}{verwrongen tronie}\\

\haiku{Het bleek niets anders.}{dan een verachtelijke}{baantjesjagerij}\\

\haiku{Op een dag liet de'.}{Rechter-Commissaris}{Frans vader roepen}\\

\haiku{Hij smeekte Frans en.}{hij wroette in de pijn van}{zijn onrust en angst}\\

\haiku{Dan kon hij ieder.}{woord scherper en krenkender}{van scherts uitstooten}\\

\haiku{Die schepte nooit op,.}{deed nooit dik en speelde niet}{met een zakdoekje}\\

\haiku{Hun wandelstokjes.}{plakte hij in een hoekje}{166achter de deur}\\

\haiku{- Als we nou aan de,!....}{wieg stooten zijn we geknipt}{fluisterde Frans weer}\\

\haiku{Maar de spanning hield.}{nijpend het bezetene}{vreugdegevoel vast}\\

\haiku{Daan Blikkie zuchtte,.}{dat di\'e gedwongen kalmte}{zenuwkracht verslond}\\

\haiku{Frans leefde weer in.}{een koortsroes en onder een}{heete betoovering}\\

\haiku{- Ik wou maar d\^a 'k '...!}{jo\'u inn lijsie had kon}{ik je ophange}\\

\haiku{Plots overduizelde.}{hem weer een stoutmoedige}{geluksgedachte}\\

\haiku{as me fader 'n,}{bok slacht mit dolle kerfel}{mag je me raje}\\

\haiku{en je frommes breit '...}{r hier d'r trui ferder en}{je kooters gaan}\\

\haiku{- Hij krijgt op se jak,...,...!}{soofeul astie lust Trui foor de}{goksie maar hij lust niks}\\

\haiku{Maar telkens lachte,.}{Karel weer zijn heerlijken}{uitdagenden lach}\\

\haiku{Wat een heerlijke,.}{heupen had die meid om haar}{zoo \'op te tillen}\\

\haiku{hei-je niks mee te...!}{make en slobber alleenig}{maar je taskef\'ee}\\

\haiku{Dat had toch als een?}{menschelijk voornemen in}{hem rondgezongen}\\

\haiku{Een kille windvlaag.}{suizelde huiverig door}{de nachtstilte heen}\\

\haiku{Nou moest hij maar niet.}{zijn kousen verzolen en}{zichzelf bedriegen}\\

\haiku{Hij zou wel kunnen,!}{kuiltjes-schieten met}{knikkers van pleizier}\\

\haiku{Waarom wierd zijn loop,?}{nu een stoffig slenteren}{slof-slof-slof}\\

\haiku{Vroeger wipte hij.}{dat vogelnestje op zijn}{hand als een draaitol}\\

\haiku{Dat wisten Thomas......?...}{Slokkebier toch wel en Jan}{de Gans en Hompie}\\

\haiku{Nu schepte zij voor,.}{een iegelijk op borden}{gort-met-stroop}\\

\haiku{- Soo'n Schoppeboer... hei '!}{k nog nooit-nie in me}{klefiere gehad}\\

\haiku{Er dreigde iets om.}{hem heen in de lucht als bij}{onweer-op-komst}\\

\haiku{Karel doorzag de:}{omhulling van haar jaloersch}{spel en mompelde}\\

\haiku{De Afslager, een,.}{vriend van Karel keek hem met}{ontzette oogen na}\\

\haiku{Op de Lindengracht,,.}{tusschen zijn kameraden}{wierd Karel begekt}\\

\haiku{Moest misschien ruimte,...}{gemaakt met een opsteker}{e\^er gaf het geen lucht}\\

\haiku{Begon hij weer te?}{razen en te schimpen in}{bezeten tarten}\\

\haiku{Wat verlangde hij?}{nog tusschen de joeldrukte}{van de Noordermarkt}\\

\haiku{Terwijl hij alleen,.}{snakte naar het vreemde naar}{het onbekende}\\

\haiku{Hij schold zichzelf nog.}{een proper maagdelijntje}{in de tiejeiskraak}\\

\haiku{Het mistte zwaar en.}{grauw op den laten middag}{van de insluiping}\\

\haiku{Frans had een gevoel;}{alsof zij zich telkens op}{hem wilden werpen}\\

\haiku{heel het heldere.}{licht scheen nijpend gedoofd in}{haar beschroomde oogen}\\

\haiku{dat hij overeind moest,!}{rijzen opdat hij hem uit}{elkaar kon scheuren}\\

\haiku{de heele Jordaan;}{beefde en rilde voor zijn}{zwijgende zinning}\\

\haiku{En weer verklaarde,;}{Frans dat hij hem alles zou}{vertellen alles}\\

\haiku{Natuurlijk over jouw...}{link liefdesspel met meide}{van allerlei slag}\\

\haiku{Maar toch lei er een.}{nieuwe vraag te martelen}{op Karel's lippen}\\

\haiku{dat Corry van h\'em,.}{misschien wel hield maar Karel}{Burk alleen liefhad}\\

\haiku{Dat, d\^at vooral moest.}{Karel het allereerst zijn}{knar inhameren}\\

\haiku{Zijn ouders hadden,.}{hem schrikkelijk bejammerd}{maar zichzelf nog meer}\\

\haiku{een heel mooi vrouwtje.}{met een droomerig-zacht}{meisjesgezichtje}\\

\haiku{als een vermomden,?}{duivel in priesterrok nu}{hij niet meer inbrak}\\

\haiku{Hij lanterfantte,,,;}{hij danste speelde zong voor}{de tippelaarsters}\\

\haiku{bij alle twee, een.}{even hevig en vlijmend in}{het leven kerven}\\

\haiku{Maar Karel, tot Frans',.}{vinnige verbazing vroeg}{nauwelijks meer wat}\\

\haiku{Wat verlangde die?}{troebele zwammer toch van}{hem en met welk recht}\\

\haiku{Nou kom ik mit de...}{heele schutterij-kep\`el}{op theefesite}\\

\haiku{Burk rilde voor de.}{allergemeenste groeven}{om Poort's mondhoeken}\\

\haiku{Nou kwam Poort met de.}{bijl in de hand en joeg hem}{de inspringer uit}\\

\haiku{behallefe de... -!....}{ferstaancente Bink is-ie}{gierden de meiden}\\

\haiku{Hij was vanavond weer,.}{in een verschrikkelijke}{ellendige bui}\\

\haiku{Maar in\'e\'en sloeg zijn,.}{chagrijn \'om in een dolle}{grijnzende dreigpret}\\

\haiku{Maar in een vrijen;}{karakterdans kon Karel}{veel eruit gooien}\\

\haiku{Burk liep heen en weer,,.}{onder een schuwe schaamte}{den kop ingebukt}\\

\haiku{Karel huiverde,...:}{en toch daemonisch gromde}{hij er tegenin}\\

\haiku{Het woede-licht.}{in zijn z\'elf nog bloedende}{oogen scheen uitgebluscht}\\

\haiku{339Ze stemde in,.}{maar alleen wanneer de twee}{kerels meestapten}\\

\haiku{Het wierd over en weer,.}{spreken tot het bittertje}{erbij te pas kwam}\\

\haiku{In den donkeren.}{stal waschte Thijs het af}{en voedde het goed}\\

\haiku{Ook met den nieuwen.}{compagnon kreeg hij heibel}{en wierd het knokken}\\

\haiku{Eindelijk geheel {\textquoteleft}{\textquoteright},.}{los strompelde hij weer bij}{zijn eega binnen}\\

\haiku{Hij konkelmonkte '!...}{t gebefte gajes en}{de affekate}\\

\haiku{Het geschal scheen hem.}{onvoorziens een oogenblik}{te ontnuchteren}\\

\haiku{In Caf\'e {\textquoteleft}Marktzicht{\textquoteright}.}{joelde opgewonden de}{avond-Kerstdrukte}\\

\haiku{Uitkijk, een makker,.}{hield de politie links en}{rechts in de gaten}\\

\haiku{Greet, Alie en Sientje,.}{hadden er voor gezorgd en}{zelfs Thijs wou meedoen}\\

\haiku{Het liefelijke,.}{en argelooze er in bracht}{haar geheel van streek}\\

\haiku{Karel voelde haar.}{adem heet beven tegen zijn}{trillende handen}\\

\haiku{Het was alsof de.}{heele Jordaan meejuichte}{met hun vereendheid}\\

\haiku{In Corry was het,.}{wonderlijke gebeurd dat}{Poort niet had bevroed}\\

\haiku{nu was Corry niet...}{op Pinksteren maar op Kerst}{tot hem weergekeerd}\\

\haiku{iemand die al op.}{jeugdigen leeftijd in de}{gevangenis zat}\\

\subsection{Uit: Levensgang: roman uit de diamantwerkerswereld}

\haiku{Dan heb ik ni\'et,;}{meer te spreken over maar van}{menschen en dingen}\\

\haiku{In z'n onderbroek,,.}{was Hein zacht met ingeho\^uen}{vaart opgesprongen}\\

\haiku{Een vuurspat van.}{woest lucifers-gestrijk}{brak door haar kijfvaart}\\

\haiku{Zou je niet z\'oo  ,?}{wegloopen en den heelen}{boel laten stikken}\\

\haiku{Zijn baas zou dan \'o\'ok.}{zien dat hij die vodden niet}{langer dragen kon}\\

\haiku{- Late we'm na de, ' '!}{kantien drage mitn slok}{bier is ier op}\\

\haiku{- Had jelui nou die?}{dolle schreeuwleelik daar niet}{kenne tegehou\^e}\\

\haiku{De benauwing van, '}{voorspellingen te hooren}{weer druktem al.}\\

\haiku{En as die 'n groote,,.}{smoel zet die jood dan l\^a je}{je niet afbluffe}\\

\haiku{Dat voelde ie ook.}{veel beter passen bij z'n}{heldenkarakter}\\

\haiku{Toen begon er als '.}{vanzelf plotsn zacht-innig}{spel in z'n zieltje}\\

\haiku{Het leven thuis met ' '.}{z'n o\^uers en broers wasmn}{gruwel geworden}\\

\haiku{In kleinigheidjes,.}{had ie wel veel gejokt maar}{altijd voor zichzelf}\\

\haiku{Hein zat, zat zonder ',.}{n woord te spreken met z'n}{hand onder de kin}\\

\haiku{Smoorheet blakerde ' '.}{t kacheltje int eng}{krotje om zich heen}\\

\haiku{Woedend was Hein dat '.}{z'n moedert arme kind}{bij d'r scheldnaam riep}\\

\haiku{as 'k de fasch breng... ' '...}{soene se me rejoal}{t isn merakel}\\

\haiku{Giftig voelde ze '.}{zich worden bijt zien van}{Heins koppig volhou\^en}\\

\haiku{Bij elken nieuwen.}{opschep duwde ie z'n bord}{verder van zich af}\\

\haiku{Z'n lippen voelde,.}{ie aan kurkdroog korrelig}{van vleeschbultjes}\\

\haiku{dreigend gekijk van,.}{aanhangers nog dreigender}{van tegenstanders}\\

\haiku{in zwaar rumoer en:}{dreunend geweld werden z'n}{woorden afgehakt}\\

\haiku{Scherp-bevend',:}{klonk Veeges stem losgescheurd}{uit z'n heeschheid}\\

\haiku{sentimenteel bi, '!}{je dat ken je van mijn op}{n briefie krijge}\\

\haiku{Bakounine mot, -, -!}{je leze kon je maar fransch}{en Kropotkine}\\

\haiku{Toen, inspannend z'n,.}{wil had ie ook dat nare}{gevoel weggedacht}\\

\haiku{- Goeie heer, met wat 'n.}{vorsten-branie staan ze}{je n\`et even te woord}\\

\haiku{Maar dat onder z'n ',!}{makkerst leven zoo duf}{zoo uitgesuft was}\\

\haiku{Dan had ie pas voor.}{zich gezien brokken wei aan}{de Utrechtsche zijde}\\

\haiku{Ja, maar zooveel groote '.}{lui vant geloof hadden}{dat toch \'ook gedaan}\\

\haiku{Hij wist zelf niet of ' ' ';}{iet vann man of van}{n vrouw verlangde}\\

\haiku{- Lepper, van middag,,,?}{bin je toch af en u ook}{meneer Spauer niet}\\

\haiku{- Zes h\`arde h\`o\'ek-k\`e... '...,,......}{h\`etn heel st\`e\`entje door}{z'n d{\`\i}ks door z'n pl\'ats}\\

\haiku{- Adden\`om blijf staan, maak - '.}{me geen m\`erchel17 schreeuwde}{ier dreigend toe}\\

\haiku{- N\`oh d\`ammaar, hou u '... '!}{t dammaar \`af t\`och laatk}{me niet afzette}\\

\haiku{Lepper, Sprauer, Mopens,,, '}{Remeni Mierikstein enn}{hoop meisjes drongen}\\

\haiku{- vroeg Pronkman op den, '.}{man af in dreig-houdingn}{antwoord afwachtend}\\

\haiku{- laat je moeder je '!}{maar iedere dagn paar}{eiere klosse}\\

\haiku{Intusschen hadden.}{veel kijkers geprobeerd den}{tiltoer na te doen}\\

\haiku{Maar 't gestommel,.}{geschuif en gevloek hield aan}{zonder resultaat}\\

\haiku{Bleekman was plots naar ';}{m toegestapt om in z'n}{nota te kijken}\\

\haiku{Al de chipswerkers;}{klaagden mee en vertelden}{van hun blindkijken}\\

\haiku{Heviger draaide ',.}{en rommeldet in z'n}{keel boven z'n maag}\\

\haiku{stil bleef ie even staan,,;}{zonder kaakbeweeg met de}{kauwvracht in zijn mond}\\

\haiku{Rooier bestoof de;}{jagende kongestie z'n}{kop in bloedaandrang}\\

\haiku{moest ie zich nou door?}{dien beroerden kerel voor}{den mal laten ho\^uen}\\

\haiku{- Meneer Broos, daar is '...,... '}{n beheime35 om je te}{spreke oggen\`ebbiech}\\

\haiku{verlegen wou ie,,.}{wegkruipen de gang in om}{z'n woesten uitval}\\

\haiku{hij voelde dat ie.}{geen houvast meer zou hebben}{aan z'n eigen drift}\\

\haiku{iets dat afstootte.}{in voorname meerderheid}{en toch te\^er meeging}\\

\haiku{Eindelijk had Hein.}{weer z'n geluid losgeschord}{uit schokbedwelming}\\

\haiku{hoe zou\^en ze in hun,.}{vuistje lachen als Hols nou}{weer werd afgemaakt}\\

\haiku{Eva triumfeerde.}{in zichzelf om den scherpen}{koelen trots van Hein}\\

\haiku{Ze voelde, na zoo'n.}{uitgescheurde smart niets meer}{te moeten vragen}\\

\haiku{lief klonk haar stem weer,,}{in doordringend gevraag wat}{ie dan bedoelde}\\

\haiku{Heel stil was ik mit '}{m d'r-naa-toe gegaan}{en opgewonde}\\

\haiku{Maor je seit, d\^e.... -,, '!}{je mit juff'r E\'ef\'a\'a Ja stil}{nou daar k\`omk op}\\

\haiku{Dol verlangend was,;}{Hein in\'e\'ens te weten}{wat ze dan wel dacht}\\

\haiku{Bij Bresser was 't.}{heele personeel al drie}{weken werkeloos}\\

\haiku{Hij had iets dors, iets, '.}{stroefs gevoeld plotseling dat}{eerst niet inm was}\\

\haiku{Toch wou z'm nog even, '.}{d\`oorplagen al meende ze}{t meeste ervan}\\

\haiku{Z'n oogen traanden van,.}{geestdrift en z'n mond lachte}{breed en bezielend}\\

\haiku{- Nee juffrouw, u weet ', '...}{niet hoeveel wij naat mooie}{t hooge verlange}\\

\haiku{omdat 'k uit me.....}{zelf heelemaal verlang naar}{wat mooi en goed is}\\

\haiku{Al z'n drukte en;}{woordgespat bestierf op z'n}{lippen aan haar zij}\\

\haiku{, met strak voorbijzien, '.}{van Sak wel wetend dat ie}{m daarmee pestte}\\

\haiku{Met z'n hand stond ie, '.}{klaar en Bresser hadm nieteens}{goeiendag gezegd}\\

\haiku{'n Kristen schoonschoon ',!}{vond ien ideaal en z'n}{eigen kind dokter}\\

\haiku{Want de klub sprak van, '.}{z'n huisgezin als vann}{muzikaal wonder}\\

\haiku{hoe as 't toe-d'r!}{tijd wat \`anders was as van}{d\'aag-des-daags}\\

\haiku{ze verzuipe de, '!}{boel toch of verteret}{an de m\`eide}\\

\haiku{jij h\`et jou porsie,,!}{dubbelt en dwars gehad maar}{ik ik bin nog jong}\\

\haiku{Met  de hitte '.}{vant gesprek golfden \`op}{weer de dierkreten}\\

\haiku{'t publiek voorover.}{boog en dan in beefbeweeg}{verdween in klokzij}\\

\haiku{- en plots zich bewust,:}{wordend schorde ie nijdig}{met zwaardere stem}\\

\haiku{'n verschwartzter nar,!...}{die elkeene do\'odslage}{wil mit ze ch\`ente}\\

\haiku{Jij bent zoo, blijft zoo, '.}{al zou je voor haar doort}{vuur willen vliegen}\\

\haiku{Buiten Eva, die nog,.}{niet beneden was zat de}{heele familie}\\

\haiku{in de tragedie '.}{vann landleven dat ze}{voor zich zag bloeden}\\

\haiku{wat roerend verteld '.}{vann leven waar ze nooit}{flauw aan gedacht had}\\

\haiku{Daar zat ze weer in,;}{haar kamer met allemaal}{weelde-dingen}\\

\haiku{Op stoelen en schoot.}{had ze dan eerst den vollen}{bloemschat uitgestort}\\

\haiku{Als ie 't maar niet, ';}{dwaas vond dat verbergen en}{half schuilgaan vanr}\\

\haiku{ik wil niet hebbe ',?}{meer dat je werklui opt}{kantoor laat verst\`aan}\\

\haiku{en je weet hoe 'k ', '!}{r de pest an heb ann}{vreemd op m{\`\i}jn kantoor}\\

\haiku{je eige vader, '!}{afstaan voor zoo'n arreme}{gojn versteller}\\

\haiku{dat as je 't hart...}{h\^et w\'e\'er die kerel op m'n}{kantoor te neme}\\

\haiku{... zal 'k verrekke,!}{Eef die h\`et-je je kop}{mechogge gemaakt}\\

\haiku{Stil liet Bresser zich.}{op den stoel zakken waarop}{Eva gezeten had}\\

\haiku{Hij barstte, hij had ',,.}{r wel kunnen worgen maar}{hij kon niet k\`on niet}\\

\haiku{En hoe dol blij zal, '.}{Hols zijn als ie hoort dat ze}{eens metm mee wil}\\

\haiku{- Och Zeelt, vertel me, '...}{maar niet verdert maakt me}{zoo gloeiend driftig}\\

\haiku{anders gaat ie me...}{nog op de late avond j\'ou}{en mij anblaffe}\\

\haiku{- en weer schoffelden.}{z'n stappen weg zonder dat}{ie Hein gezien had}\\

\haiku{Vreemd-lekker had '.}{ie nou es uitgeslapen}{alsn gewoon mensch}\\

\haiku{Alles wat ie 'r,.}{wou zeggen stokte in z'n}{keel wou niet verder}\\

\haiku{Ja, ze voelde dat ';}{in dien man heelemaalt}{kind gebleven was}\\

\haiku{en ik ben juist heel...... -,...}{laat opgestaan heusch O}{nee juffrouw dat treft}\\

\haiku{dieper in te gaan '.}{opt socialisme}{aan alle kanten}\\

\haiku{haar onzedelijk.}{gewoeker en vernielen}{van menschenlevens}\\

\haiku{Waarom durfde ze ' '?}{niet gewoonn wandeling}{metm te maken}\\

\haiku{ie weer vrijer, en.}{terug leefde ie weer den}{tijd van extaze}\\

\haiku{Rozalie was even '.}{stil blijven staan en keekn}{donkere steeg in}\\

\haiku{Als j't maar uitzegt......}{met eigen temperament}{is alles gloednieuw}\\

\haiku{ik geloof dat de......}{juffrouw ziek is z'n dochter}{die in de zaak is}\\

\haiku{Maar och... z{\`\i}j wou 't....}{toch ook niet dan had z'm toch}{wel laten roepen}\\

\haiku{Banger joeg 't door'm, '.}{heviger dat ie n\`og niet}{wist hoet met'r was}\\

\haiku{En sterker schreide ' ' '.}{t inm dat ier niet}{dadelijk mocht zien}\\

\haiku{Nog pas had ie 'r,.}{gezien vol leven en nooit}{was ze ziek geweest}\\

\haiku{je weet, 'n gojsche147!}{honderd gulde is meer dan}{e jiddische148 tauzend}\\

\haiku{Toen ie de kamer, '.}{uitging zag ie onder aan}{t trapje Hein staan}\\

\haiku{dan maar eerst van zich,,...}{afschuiven dat die dooie daar}{die stille Eva was}\\

\haiku{ie Zeelt nu in z'n,,.}{warm geduldig opporren}{in z'n taai aanhou\^en}\\

\haiku{toch wou ik je in}{het kort iets verklaren dat}{verklaring eischt}\\

\haiku{Nu kan ik me al.}{levendig voorstellen wat}{jij antwoorden zult}\\

\haiku{je weet niet wat een,...}{toer het voor mij is om daar}{nog in te komen}\\

\haiku{- Leg jij je mo\^er 'n,,!}{nijfie rotsodemieter}{ik ken soo nie voort}\\

\haiku{Soo as hij 't ons, ' '!}{uitgelege h\`et kann}{kindt begrijpe}\\

\haiku{... nou most 'k enkel,?}{nog maar vier hoog afsakke}{om te fluite seg}\\

\haiku{hoe meer g\^ojjem hoe! -.}{meer de mazele driftte}{Kiehl er tegenin}\\

\haiku{verleuje weuk as '... -!...}{k hier geweus binne Leg}{niet te o\^uehoere}\\

\haiku{... ik neem 'm, as ie,!}{velooht is op slag therug}{teuge zes ghulde}\\

\haiku{Schatergelach en '.}{stemmengespot omgierde}{t worstelgroepje}\\

\haiku{in hem was die stem ', ';}{gaan klinken doort leven}{t volle leven}\\

\haiku{Want zoo hoog als ging,}{de galm-golf van hun}{opstandingslied zoo}\\

\subsection{Uit: De Jordaan: Amsterdamsch epos. Deel 3: Manus Peet}

\haiku{- Hier bifakeer ik'... '!}{m eige hier enter ik}{teget want op}\\

\haiku{Hier peleton, hier,,...!}{sta je Manus mit al je}{fisemente bens}\\

\haiku{Hij verafschuwde,.}{de hitte de schroeiende}{huizen en straten}\\

\haiku{Hij jammerde en.}{perste de ellende zijn}{menschelijk merg uit}\\

\haiku{Dat hiette met de,....}{grove bijl er inhakken}{mompelde Manus}\\

\haiku{de mensch die God wou,....}{naderen je naderde}{alleen de wormen}\\

\haiku{hij dacht aan den dood,.}{hij van binnenuit zich als}{leeggeschept voelde}\\

\haiku{wat beseffen wij,,?}{tijdelijke schepseltjes}{van het tijdelooze}\\

\haiku{deze menschen zijn,.}{al niet meer terwijl zij toch}{nog spreken en gaan}\\

\haiku{- Heil en sege in ',......}{t ouwe mompelde Peet}{zich sarcastisch toe}\\

\haiku{een mensch moest als de,.}{wind ongezien verschijnen}{en ongezien gaan}\\

\haiku{Dat was nu  zijn,.}{eerste kaars die branden zou}{zonder sprankeling}\\

\haiku{En dadelijk het.}{wreed-tartende in haar}{krenkenden spreektoon}\\

\haiku{Waterverf Manus,,.........}{alles dun spoelsel jawel}{overmorgen ziet u}\\

\haiku{Corry, vlijmend in,,;}{haar hoon haar stukscheurende}{bloedige ironie}\\

\haiku{En toch wilde hij.}{weten wat Corry van hun}{tweetjes mompelde}\\

\haiku{Wier naam bij vele,...}{was bekend Se liet sich in}{mit vuile dinge}\\

\haiku{- De lichte schaduw {\textquoteleft}{\textquoteright}.}{van hetGulden Boekske valt}{over zijn zoldertje}\\

\haiku{Zoo vurig bonsde.}{en ziedde het bloed niet meer}{in den naspruit Peet}\\

\haiku{Doch w\'o\'orden duidden,.}{aan bij rechtzinnigen en}{bij vrijzinnigen}\\

\haiku{heulegaar \`onder...... -?}{de serrekies binne se}{geschaakt Serrekies}\\

\haiku{Mit 'n daai en 'n '...}{soeslap geef je sen trap}{feur hullie kiese}\\

\haiku{- Gut, aume,... ontviel,...!}{hem plots je sit hier s\'o\'o mit}{de dooje an tafel}\\

\haiku{hij zoo een beetje.}{naar zijn buurt en drentelde}{dan kalmpjes tot huis}\\

\haiku{Nou het zoo in de,.}{praat te pas kwam mocht hij het}{bestig vertellen}\\

\haiku{Geen jongen waagde.}{zich zoo hachelijk op de}{nauwste dakgoten}\\

\haiku{De kleine zusjes.}{en broertjes beefden voor het}{gezag van Bromtol}\\

\haiku{Ook had hij al lang,.}{in de kluisgaten dat die}{drie h\'em noodig hadden}\\

\haiku{Dan stotterde Nel,:}{gebluft door het uitblijven}{van Dirkje's verweer}\\

\haiku{Turrefeglomsel ', '.}{inn testje Veur me moes}{t kwaje besje}\\

\haiku{Met een angstdriftkreet.}{was hij dien dwarreltol op}{de keel gesprongen}\\

\haiku{Zij hielden liever,.}{een flikflooier die gaapte}{dan een vent die beet}\\

\haiku{Ook de blik- en.}{gaslucht maakte hem ziek en}{sloom-onderworpen}\\

\haiku{Hij verdroeg nauw den.}{spotschimp van zijn maats om zijn}{vlammige bakkes}\\

\haiku{Jan Gouwenaar zag.}{allerlei vreemde streken}{en vreemde landen}\\

\haiku{toen naar Malaya en,,.}{Barcelona en toen naar}{Oneglia in Itali\"e}\\

\haiku{En 't Skip nam to\'e ',!}{n draai An de kop fan de}{Handelskaai Skip ehooi}\\

\haiku{Je kon die radde,.}{kerels die mastlieren toch}{nooit overk\'akelen}\\

\haiku{die balanceerden;}{zoo een mooi fonteintje op}{hun gesjochten neus}\\

\haiku{a\^ars krijg je 't op... -,!}{je graatje Olie jij maar je}{kruiskoppe jobstraan}\\

\haiku{- Van mijn spekkertje, ' '!}{n goud maffie ofn pot}{lood d'r bovenop}\\

\haiku{zoo zoetjes en zoo,...!}{fijn-beverig door den}{stillen hemel o}\\

\haiku{Toen gelastte hij.}{Jan driemaal om zijn bullen}{weer aan te trekken}\\

\haiku{Jan zonk kermend in.}{elkaar en werd weggebracht}{naar de buitencel}\\

\haiku{enne... behoorde!}{zijn stamva\^ar nog wel tot het}{Utrechtsche Kapittel}\\

\haiku{... snauwde die loods, die.}{geen aasje sjoege had van}{wat hij bedoelde}\\

\haiku{Werom komp gij nie,?}{as feur deise Bij mijn in}{mijn schape-kouw}\\

\haiku{Ook in dat muffe.}{hol zat hij weer weken aan}{weken te fronsen}\\

\haiku{Me legge de man '.}{n kwartje uit Bij Piet in}{de Kopere Tuit}\\

\haiku{- De mismaakte stelt.}{den volmaakte eischen en}{peilt de jaloezie}\\

\haiku{En andersom moest.}{zijn vrouw van en in hem dit}{precies zoo begeeren}\\

\haiku{Simon Twei-Duim,,,}{Kau de Reus Hein Uilje mit}{Na de Neus Auk Piet}\\

\haiku{En toch, het kookte,.}{en gistte in Peet gelijk}{nimmer te voren}\\

\haiku{problemen  die.}{alle toch tot daden moesten}{worden \'omgeleefd}\\

\haiku{zoodra ze de?}{onpeilbare diepten van}{muziek niet verstond}\\

\haiku{Camille hoonde.}{en schimpte openlijk in zijn}{vurige pamphletten}\\

\haiku{en nog erger van,,.}{hun lach hun vertier van hun}{verdierlijkend kroost}\\

\haiku{En eeuwig door, een.}{frissche klatering van het}{eindelooze water}\\

\haiku{Doch ook, in reine,:}{overgave en geloof vroeg}{hij zich telkens af}\\

\haiku{Alle somberte.}{en verdriet waren in\'e\'en}{uit hem weggelicht}\\

\haiku{Manus zwol van trots.}{en tegelijk kromp hij in}{van nederigheid}\\

\haiku{Zij hoorde hem weer.}{laat in den nacht spelen op}{zijn harmonika}\\

\haiku{- Als je zoo begint, ',....}{smeer ikm op slag dreigde}{Corry minachtend}\\

\haiku{Maar Manus liet zich.}{niet uit het spoor lichten door}{verdekte tweespalt}\\

\haiku{De goudvlammige.}{hoofddos krinkelde los langs}{ooren en wangen}\\

\haiku{Zijn diplomatiek.}{Mongolensnuit deed niets dan}{gluren en loenschen}\\

\haiku{Hij wou zelf zien, ho\'e,,;}{beperkt ook zelf oordelen}{ho\'e onjuist misschien}\\

\haiku{Telkens als Manus,}{van Bakoenine wat las}{dan was het alsof}\\

\haiku{Hier en daar had hij,.}{inrichtingen afgelensd}{zonder veel behaai}\\

\haiku{l\'a-je dan vast voor!}{engeltje oproepe bij}{Onseliefeheir}\\

\haiku{Het blauwe maanzaad,.}{smaakte compleet als mosterd}{zoo scherp en giftig}\\

\haiku{Nu, in het voorjaar,,.}{eind Mei begon de herrie}{hel \'op te laaien}\\

\haiku{Die werd vast majoor,!}{van de ratelwacht als hij}{zoo voortslofte}\\

\haiku{De dame legde,.}{Frans den stamboom van haar hond}{voor dien zij kwijt wou}\\

\haiku{Daarom voelde zij,.}{zich verplicht iets vriendelijks}{te moeten zeggen}\\

\haiku{Nel had er nu weer,.}{een kind bijgekregen een}{wolk van een jongen}\\

\haiku{Zij flapte er nou,.}{eenmaal alles uit wat haar}{jeukte op de tong}\\

\haiku{Nel vond het toch maar,.}{fijn dat zij zich voor het eerst}{zoo liet bedienen}\\

\haiku{- Om te griesele,....}{viel eigendunkelijke}{Mie Nat rillend bij}\\

\haiku{Frans leek een beetje.}{verbaasd dat Corry nog niet}{was thuisgekomen}\\

\haiku{... vroeg Frans plots ontzet,.}{toen hij Eenpoot zag zuigen}{op zijn linkerhand}\\

\haiku{Gisteren had hij,;}{ook nog een paar schoothondjes}{verkocht heel prijzig}\\

\haiku{Maar Bromtol zag het {\textquoteleft}{\textquoteright} {\textquoteleft}{\textquoteright}.}{dadelijk aan hetsaurtement}{vansmoeltjestrekke}\\

\haiku{'t Was natuurlijk,.}{omdat Corry thuis sprak van}{Bad-Aap in haar wrok}\\

\haiku{maar jij loopt naar de.}{Kromme Palmstraat toe en ik}{naar de Lindegracht}\\

\haiku{En dat kon je niet,,.}{wegpraten slecht en recht met}{geen duizend woorden}\\

\haiku{Zij begreep niets van.}{het georganiseerde}{arbeidersverzet}\\

\haiku{Maar nou... nou... was hij...!}{voor Frans maar een kind van het}{gezin maar een mensch}\\

\haiku{Nou, na een maand, had.}{zij geen  143voedsel meer}{voor haar eigen wicht}\\

\haiku{En daar zat Frans maar.}{\'al over te treuren en te}{kniezen in zichzelf}\\

\haiku{De deklief-hebbers.}{in den Jordaan hielden niet}{eens meer sierduiven}\\

\haiku{Plots schoot Frans in een...,.}{lach o die grapjasserij}{van Bromtolletje}\\

\haiku{Alleen die blauwe,.}{Dragonder die liet zich niet}{wegdringen door hem}\\

\haiku{Manus Peet had nog '.}{tweemaal Frans Leerlaps avonds}{aan huis opgezocht}\\

\haiku{Dan voelde hij dat;}{de geest en de vervoering}{echt in hem waren}\\

\haiku{Dit was misschien wel.}{het sterkste en tegelijk}{het zwakste standpunt}\\

\haiku{een droef beetje, dat!}{nauwelijks gemalen of}{geraspt kon worden}\\

\haiku{Kom, hij ronkte als,!}{een slagveer een seconde}{v\'o\'or het afloopen}\\

\haiku{In het gezin van.}{Frans Leerlap mocht Manus niet}{meer worden gemist}\\

\haiku{En, even bevend van,.. -!}{stem spotte Manus schijnkoel}{Schuin over je Dahlia}\\

\haiku{tusschen de ketels.}{werken en de vlampijpen}{uitvegen op schip}\\

\subsection{Uit: Menschenwee. Roman van het land. Deel 2}

\haiku{komp bai de huur t'recht.. ' '!....}{doent sellefers sel je}{n vrachie voele}\\

\haiku{Verduufeld aa's tie nie.}{twee Piets en twee Dirke veur}{sain rekening nam}\\

\haiku{voortzwoegend tot den,,.}{avond zwijgzwaar geradbraakt in}{pijning van elk lid}\\

\haiku{Witte hoeden en.}{strooien kiepen lichtten en}{blondden in de wei}\\

\haiku{- Ou\"e Gerrit voelde,.}{zich nijdig worden hoorde}{dol-klank maar half}\\

\haiku{- Hou je bek, snauwde,....}{woest-driftig Kees ik}{vroag je sinte nie}\\

\haiku{Ze voelden wel, de, ',.}{werkers datt nu ging om}{hun rust h\`un bestaan}\\

\haiku{F'rduufeld, nou gonge ',.}{de kerelsr koejeneere}{bromde ou\"e Gerrit}\\

\haiku{- Kees, Kees, bromde Piet,,,,!....}{weer die hep gain rug gain stuit}{gain kop die hep niks}\\

\haiku{Dirk en neef Hassel '.}{konden met hun karrent}{zijhek nog niet in}\\

\haiku{je laikt puur daa's,,,....}{se benne bestig loog Dirk}{om zich te redden}\\

\haiku{En van allen kant,,.}{de zwoeggezichten keken}{strakker vermoeider}\\

\haiku{De eerste gouden;}{zomerjubel van wei en}{boom was weggelicht}\\

\haiku{Z\'o\'o, elken dag bleef,}{Wimpie moederziel alleen}{huilde hij soms als}\\

\haiku{- Heen en weer liep ie,.}{dol-angstig als plots zoo vreemd}{z'n hart stil bleef staan}\\

\haiku{Om de kerels had,:}{ie rondgedrenteld ze in}{z'n tel-rhytme}\\

\haiku{Van alle kanten,.}{tegelijk krakeelde bod}{tegen elkaar \`op}\\

\haiku{- Saa'k f'rbrande aa's '' '....}{k snap w\'at waif d'r mee}{uithoale mot}\\

\haiku{Is d'r netuurlik,!}{daa's sullie d'r honde snachts}{skiete loate}\\

\haiku{m\^o je laine.. en '......}{borge beklomppe ens}{winters hai je nood}\\

\haiku{.. je heb van ochend tut............}{nacht te werke en nie \`om}{te kaike fort fort}\\

\haiku{Veel venters, bek-\`af,,.}{stemden toe in prijs lam en}{gebroken van sjouw}\\

\haiku{De kerels roerden,,.}{smakkerden en de meiden}{lachten en likten}\\

\haiku{Den wagen had Kees '.}{n eindje hooger smaller}{dijk opgereden}\\

\haiku{Z'n lijf stond in walm,.}{van zweet poriede open in}{rauwen vleeschgeur}\\

\haiku{licht dat schroeide z'n,, ';}{vleesch groef en dreunde sloeg}{en vrat int hooi}\\

\haiku{- Nog 'n vaif vorke,,.}{galmde ie terug met z'n}{rug naar Dirk gekeerd}\\

\haiku{Als was z'n oog met ''}{n zuur gif volgedrupt zoo}{vrat en brandde}\\

\haiku{Gunter stoan d'r..!}{Bolk en Hannes Skrepel en}{Piet Steinstroa en Gais}\\

\haiku{Achter 'm stonden,.}{in bloei van lichtend paars de}{vroege aardappels}\\

\haiku{hoonde Dirk den Ou\"e,.}{na die achter de deur hem}{nog wou meetronen}\\

\haiku{de Ou\"e is d'r puur,!}{tuureluurs van nou die soo feul}{kwait k\`en aa's tie wil}\\

\haiku{- D\`at veel liever, dan,, '.}{op stap op den gloeizandweg}{naar zeen uur gaans}\\

\haiku{En dan maar brommen,, '.}{en klagen luid luid om iets}{r t\`egen te doen}\\

\haiku{Wisselend in gang,.}{trokken de werkers drie maal}{\`op naar de groote stad}\\

\haiku{Soms, als ie wat beet, ';}{had weer konm de heele}{boel niet meer schelen}\\

\haiku{Nu en dan zag ie.}{z'n wijf verdwaald rondzoeken}{in de tuinderij}\\

\haiku{'n Paar dagen had,.}{ie achtereen in huis wat}{bollen gesorteerd}\\

\haiku{Vloeken k\`on ie, als.}{ie daar niet dikwijls genoeg}{vrij mocht afzakken}\\

\haiku{Kees was bedankt, mocht '.}{weer eens aanslenteren in}{t drukst van den pluk}\\

\haiku{Hun morsige bloote '.}{voeten ploeterden int}{warmzonnige gras}\\

\haiku{Maar Trijn wou niet dat,.}{Geert d'r hoofd zou breken om}{die narigheidjes}\\

\haiku{Maar Jan was z\`eker '.}{bij  Dirk minstensn paar}{keer Guurt te treffen}\\

\haiku{Wai.. wille d'r vast.. '!}{mi de haire meegoan aa't}{t in frinskap lait}\\

\haiku{De meiden vonden.}{dat de knapen zich kranig}{gehouden hadden}\\

\haiku{in stikdonker, bij;}{schaduwrood schijnsel van wat}{eenzame lantaarns}\\

\haiku{Most dus puur tug 'n...}{toeval weuse a\`as tie die}{vint dur nog antr\`of}\\

\haiku{En Wimpie had dat.}{met smeekjes bij d'r moeder}{voor haar klaar gespeeld}\\

\haiku{Donker bonkten z'n.}{schonken boven de lage}{rij pofpetten uit}\\

\haiku{gain groasje, dood,.. '!}{an die swaine trek jullie}{d'rn poar kiese}\\

\haiku{Moordhol timbreerde, ' '....}{z'n stem en vlak naastm klonk}{n andere zang}\\

\haiku{Over twee minute....}{sal de nieuwe voorstelling}{een aanvang neme}\\

\haiku{voorsiet u van een, -.}{plaats suggereerde de stem}{van de estrade}\\

\haiku{Onder 't vreten ' ';}{rimpelder kop alsn}{oudwijvenmasker}\\

\haiku{Daa't is d'r mit 'n, '.}{half uur d\`a\`an hoonden meid}{belust op emotie}\\

\haiku{'n Zangeres was.}{achter gala-chanteur naar}{voren gekropen}\\

\haiku{- Ze drentelden en,.}{draaiden tot Dirk en Willem}{er uit wankelden}\\

\haiku{menschen die elkaar.}{in ego{\"\i}stischen angststuip}{knellend verdrongen}\\

\haiku{- In\'e\'ens voelde,,.}{ie zich weer laf kruiperig}{laf klam in doodsnood}\\

\haiku{stilte die 'm deed.}{rillen en huiveren van}{al stijgender angst}\\

\haiku{Dat was d'r eerst met,;}{snijboon negentig cent de}{duizend voor fabriek}\\

\haiku{Ze schreeuwden 'm toe, ', ',.}{vant hok uitt erf dat}{ie zich bergen zou}\\

\haiku{Ou\"e Gerrit huilde,,,.}{snikte rochelde van angst}{ontzetting en drift}\\

\haiku{Als 'n rouw, ging er '.}{stomme ontzetting en stil}{geween overt land}\\

\haiku{- Hij voelde iets heel ',.}{bangs opm drukken iets ergs}{dat gebeuren mo\'est}\\

\haiku{De vraag ontshutste,.}{ou\"e Gerrit z\'o\'o dat ie sip}{voor zich bleef kijken}\\

\haiku{Loenschig kippigden,.}{z'n oogen op Hassel en zwaar}{donderde z'n stem}\\

\haiku{Plots zei Beemstra 'm iets ', - '.}{int oor en Troost lispte}{t over aan Stramme}\\

\haiku{Als ik nog meer van,..}{die klanten had zou ik zelf}{op de valreep staan}\\

\haiku{je grond is er niet,..}{slechter op geworden dat}{zal ik niet zeggen}\\

\haiku{ik seg mo\`ar.. da\`as 'n..!}{kerel die help je nie van}{de wal in de sloot}\\

\haiku{Ja, hij most de boel,.}{vergoeilijken met meelij}{met verkleineering}\\

\haiku{Enn... veur w\^a gong die?}{nou nie in hande van de}{aere netoaris}\\

\haiku{Vrouw Zeune was links,,;}{gaan staan klaar om hem op te}{vangen als ie viel}\\

\haiku{En dan nog.. aa's 't!}{donkere hokkie opestong}{t'met de loodpot}\\

\haiku{ik hep d'r puur ses....!}{pop wonne heb d'r perdoes}{op Peloone wed}\\

\haiku{op 't krotje daar, ';}{op den Duulweg \'e\'en kamer}{metn achterend}\\

\haiku{Expres heb ik mijn..}{lens late zitte zag je}{niet dat ik merkte}\\

\haiku{Dat had ie eerder,.}{moeten doen toen ie alleen}{thuis was gekommen}\\

\haiku{In laatsten stamel,:}{van angst half-verstikt}{stotterde ie uit}\\

\haiku{stapte Piet naar Dirk,, '.}{liepen ze stom naast elkaar}{n havenkroeg in}\\

\haiku{En daarna Ant, in ';}{snikkende huilkramp voort}{lijkje van Wimpie}\\

\haiku{- Hij slenterde weer, ',.}{werkeloos rond en niks kon}{m meer schelen niks}\\

\haiku{- In 't lage krot;}{knaagde weer naderende}{winter-ellende}\\

\haiku{- 'n Geweldige,.}{uitstorting van haat woede}{en wrok barstte los}\\

\subsection{Uit: Menschenwee. Roman van het land. Deel 1}

\haiku{Maar toch, \'e\'en ding was,....}{er voor hem gebleven in}{al z'n ellende}\\

\haiku{se hewwe main, ',....}{neudig al wi'k veurn sint}{p'r uur puur nog nie}\\

\haiku{Met schrik-gezicht,,.}{draaide vrouw Hassel om wild}{grijpend naar d'r schort}\\

\haiku{die z'n rekening......}{hewwe wou en veurskot}{van grondbelasting}\\

\haiku{pats, dan d'r van langs,,....}{mit d'r stevige knuiste}{dan seie se niks meer}\\

\haiku{- Bar, bin soo koud aa's, '.... - '....}{m'n paip mokten ander}{Kruip inn stuk hout}\\

\haiku{dan lager dalend:}{met spannender stilterust}{in cijferzakking}\\

\haiku{Guurt was 'n meid die,.}{alleen aan d'r zelf dacht dat}{voelde ze nog wel}\\

\haiku{wacht se mos sich nou....}{moar puur inprate da't van}{selvers betert gong}\\

\haiku{Ze groef 't in 'r, ' ',.}{hoofd metseldet inr}{geheugen met drift}\\

\haiku{Ze zat zoo lekker,.}{zoo lekker d'r kansen te}{berekenen}\\

\haiku{Niks meer noodig, voor se ',.}{aigen paar kan en de}{rest veur de venter}\\

\haiku{Dat was 't eenige ', '.}{datr staande hield enr}{verdriet verdoofde}\\

\haiku{doedelsak, je skenkt,.... -,....}{op main poote helhoak Main}{kristus wa\^a jokkes}\\

\haiku{en jullie.. jullie........ '?....}{wete d'r ook gain snars van}{weet jait moeder}\\

\haiku{St. Nikolaas was '.}{in wit gewaad neergedaald}{int stedeke}\\

\haiku{de hoogste hep sain,..... -....}{twee pond paling of ses pond}{speek'laos twoalf gooit}\\

\haiku{- Nou aa's j'r senie '....}{in hep ka je f'nacht int}{koarte-huis maffe}\\

\haiku{Dronken tronies van;}{mannen en meisjes lachten}{al \`a\`an in zuipgrijns}\\

\haiku{Een had partij voor '.}{Piet getrokkenn ander}{voor den Duinkijker}\\

\haiku{Zacht vlokte sneeuw neer,.}{wemel-schimmig op zwak}{lichtend haventje}\\

\haiku{Huiverig en koud,,.}{keek ie uit overal waar ie}{het sneeuwstraatje langs ging}\\

\haiku{Voor 'n koek- en,.}{broodwinkel bleef ie staan met}{groote gulzige oogen}\\

\haiku{Want hij zelf had 'r ', ',.}{n gruwel van vond hetn}{lam ellendig werk}\\

\haiku{Dan joeg 't in 'm,, '.....}{brandde bonsdet die lui}{die alles hadde}\\

\haiku{ikke  en de........}{hufters sou je je aige}{nie f'rmorsele}\\

\haiku{In huilsnikken barstte, '.}{ze los metr vuile schort}{tegen d'oogen gedrukt}\\

\haiku{De kinders hurkten,,,.}{bij Wim's bed saamgeklonterd}{in doodsnood spraakloos}\\

\haiku{De kinders konden,,.}{beginnen gulzigden in}{met oogen en handen}\\

\haiku{Jammerlijk vaalgroen '.}{bleekte z'n kopje int}{schuwe val-licht}\\

\haiku{Z'n vuile hansop,,.}{liet z'n beentjes uitspaken}{latjesplat recht uit}\\

\haiku{- En je moeder dan,,....}{ken die nie blaife meskien}{is d'r feur main wa}\\

\haiku{Ze had gloeienden ',.}{hekel aanm omdat ie}{van z'n vader hield}\\

\haiku{V\'o\'or 't huisje van,.}{Grint bleef ie staan trapte ie}{even tegen de deur}\\

\haiku{- Nou, ik wou moar da,,....}{se d'r sno\`ater hield driftte}{Jan Grint is me da}\\

\haiku{Klaas was woedend, hij.}{begreep niet waar de kerel}{zich mee bemoeide}\\

\haiku{zat ie gezwollen,.}{van kleine afgunst kleine}{indringerijtjes}\\

\haiku{twai volle ure en....'....}{nou gong de hailige en}{d appeteker}\\

\haiku{Z'n bleek groenige '.}{kop zag in schrik \`op naar z'n}{vader int licht}\\

\haiku{moar jullie hep sain,,}{achter de mouw zei Kees straf}{onverschillig voor}\\

\haiku{Bolk, spraakzaam, blij, dat,.}{ie wat takkebosjes te}{maken had ging door}\\

\haiku{Loomer schuurden de,,.}{takken op langs de kar bij}{kleine optrekjes}\\

\haiku{lijf naar beneden,.}{kijkend of Krelis al klaar}{was om te sjorren}\\

\haiku{komp, spotte Heelis,'!}{mo je eerst fesoenlik}{tellegrefeere}\\

\haiku{hai kon ook tug nie '....}{betoale en nou weer}{n dokter bai Piet}\\

\haiku{Op den dag kwam er,, '}{toch nooit niks van was ie bang}{dat zem snappen}\\

\haiku{Met 'n bons had ie..}{de bedsteedeurtjes achter}{zich dichtgeslagen}\\

\haiku{- Moeder, nou effe.. '?}{et vuurskutje van de}{stal hoal jait}\\

\haiku{Sou ie niet gille ',.}{aa's tiet sag aa's tie d'r}{was in sain kelder}\\

\haiku{Z'n hart bonkte en....}{wilder gloeiden z'n oogen in}{z'n grijzen glanskop}\\

\haiku{Het lampje eerst uit,,.}{dat wou ie nou eenmaal zoodat}{ie niemand zien kon}\\

\haiku{Maar heel bedaard bleef, '.}{zem loom vragend of ie}{d'r uit was geweest}\\

\haiku{Nooit kon ie slapen, '.}{of moeder moest ook liggen}{al werdt elf uur}\\

\haiku{Rams aansjokken uit,}{achterend vaagtastend in}{schuifel-pasjes}\\

\haiku{- Ik seg moar, scherpte,'........}{ze stil uit da se d'r van}{bekold is behekst}\\

\haiku{aa's je nie je lol....'....}{d'rin houwt sou je t'met j}{aige f'rsuipe}\\

\haiku{De half-open.}{slaapmondjes zuurden adem uit}{in eng ruimtetje}\\

\haiku{Enn ikke... ikke... '....}{bin protestant moart sit}{d'r dunnetjes op}\\

\haiku{Ant frommelde 'r,.}{plunje los zwaar gapend met}{zenuwgeluid}\\

\haiku{Maar nou mo' je sain..!..}{pakke hoor!nou van niks meer}{prakkeseere hoor}\\

\haiku{Dan nog liever met,,.}{sprenkels maar da gong nie meer}{nou ie niks bezat}\\

\haiku{hai mos t'r soo wa'........}{baif'rdiene ken se kossie}{bestig hoale}\\

\haiku{Met de anderen,,.}{was ie zeker in z'n sprong}{de sloot ingestapt}\\

\haiku{Breugel stapte naast,.}{Kees die z'n achterlader}{weer be-hageld had}\\

\haiku{'t zwart-stille, '.}{pad \`opstappend tegent}{donkere krot aan}\\

\haiku{Hij zou d'r 'n pats '....}{tegen d'r kop geven aa's}{se nogn woord zei}\\

\haiku{Laag groezelde 't '.}{lampje wat licht neer int}{killige vertrek}\\

\haiku{Dit jaar had ie zich '.}{nog voorn prijsje van de}{koeien afgemaakt}\\

\haiku{en al \'e\'en keer s'n, '.}{borge ansproke die niks}{meer mitm wilde}\\

\haiku{in tait van nood skil '? -....'!....}{je oarepels mitn bail}{hee Moar da k\^e soo}\\

\haiku{'t Zwaarste werk ging '.}{m licht van de hand. Stank van}{beervuil rook ie niet}\\

\haiku{nou dacht ie aan z'n,,.}{meid voor vanavond in heete}{opjoelende lol}\\

\haiku{Gisteren hadden.}{ze de eerste centen van}{spinazie gebeurd}\\

\haiku{Hoe kon ie zoo stom '.}{zijn te gelooven datt bleef}{staan wachten op hem}\\

\haiku{Onrustig joegen.}{de tuinders elkaar \`op in}{hun verborgen angst}\\

\haiku{Zoo bleef de Meiemaand '.}{rondgaan int stedeke}{en dorpjeszeeweg}\\

\haiku{Stil zat dominee, ',.}{in mijmer inn rieten}{tuinstoel aan den weg}\\

\haiku{niks had d'een voor d'a\^er, '....}{over of ze motte wete}{datt vast goed gong}\\

\haiku{En Kees was al heel.}{blij als ie zeven pop kreeg}{voor de heele week}\\

\haiku{loom gezang van de, ',.}{aarde naart lichtende}{groeiende leven}\\

\haiku{An Peters van de....}{Baanwaik h\'e'k femurge twee}{hoek oarbeie f'kocht}\\

\haiku{Hommels streepten van;}{allen kant fluweelige}{kleurtjes door de lucht}\\

\haiku{vroeg ie in-\'e\'en,,.}{door dikke vrouw negeerend die}{nog wat zeggen wou}\\

\haiku{Strak drukte ie z'n,.}{bril in neusgleufje montuur}{tegen z'n ooren}\\

\haiku{zei verbleekend, mank... '}{vrouwtje schuiner afzakkend}{r linkerschouder}\\

\haiku{- Wat, schreeuwde woedend,!.}{assistentje donder op}{wordt hier niet verkocht}\\

\haiku{maar ik zeg dat 't,,;}{nou genoeg is barstte uit}{z'n nijdige stem}\\

\haiku{hoho.... aa's d\'a' 't waif '....}{miskien bestiges inn}{gesticht sel konne}\\

\haiku{- Dokter Troost bleef 'r,,.}{stil bekijken schudde soms}{even zwak z'n log hoofd}\\

\haiku{Dokter stond nog wat,.}{te bedisselen met Guurt}{die fijntjes lachte}\\

\haiku{en ook mi' sonder.. '....}{aige oapeteek en nooit komp}{ien keer teveul}\\

\haiku{noa niks! - Moar, komp 'r,, '!}{t'met rege schreeuwde Piet}{bestig veurt raip}\\

\haiku{of onze lieve ', '?....}{Heerm zou ranselen dat}{ier bij neerviel}\\

\subsection{Uit: De oude waereld I: Het land van Zarathustra}

\haiku{Samir wachtte een,.}{blik of de Gebieder er}{iets van begeerde}\\

\haiku{All\'e\'en bij het volk.}{perst gij het zoetste sap van}{de levenskern uit}\\

\haiku{het lokkend kweelen.}{van een loofvogel achter}{doornen-struweel}\\

\haiku{Inhevige, toch:}{bedwongen bewogenheid}{ging Darius voort}\\

\haiku{En toch drukte er.}{op zijn ziel de onheilstilte}{der zee v\'o\'or den storm}\\

\haiku{Scythia, waar zij geen.}{voedsel meer vinden en geen}{uitweg meer kennen}\\

\haiku{al wat het diepe,.}{menschen-wezen aan zichzelf}{ontleent alleen blijft}\\

\haiku{Gij hebt mijn zoon door,.}{zelfmoord in bloedstortingen}{laten omkomen}\\

\haiku{- Dek je zondestem...,....}{toe en zwijg zei hij barsch}{tot den voorlezer}\\

\haiku{Stil wenkte hij den,.}{gunsteling met de schalksche}{oogen aan te vangen}\\

\haiku{V\'o\'or wij doof worden,.}{moeten wij hooren en v\'o\'or}{wij blind worden zien}\\

\haiku{Ik zeg u Koning,,.}{uw grootvizierszoon sprak stout}{maar ook laf en half}\\

\haiku{Maar Zerubbabel,.}{verlangde niets voor zichzelf}{doch veel voor zijn volk}\\

\haiku{Een juichenden drom;}{van werkmeesters vereenden}{de Hebreeuwers tesa\^am}\\

\haiku{in het Zuiden naar;}{het Idumeesche gebergte}{en Arabia Petraea}\\

\haiku{W\'e\'er reed Nadab, de,.}{grijze Susi\"er ontschroomd naast}{Koning Darius}\\

\haiku{waar hij bezweek van.}{gekweel en gevley zijner}{heidensche tortels}\\

\haiku{Zerubbabel, den,.}{peha op te trekken naar}{den heiligen grond}\\

\haiku{Tooyt u niet op als.}{een satyrhoender in}{zijne pronkveeren}\\

\haiku{Zion brandt... Zion,......}{brandt de veste verkoolt tot}{een zwarten bouwval}\\

\haiku{Want nu, n\'u mannen,.}{Isra\"els richt God ons weer}{goedgunstiglijk \'op}\\

\haiku{De leschem van Naphtali,.}{golfde zeegroen-blauwe}{purperingen toe}\\

\haiku{Raad mij broeder, bij... -...}{de heilge Fravashi van}{Zarathustra Hoor}\\

\haiku{Toch, het vuur was hem,,.}{ontstolen het vuur ontspat}{aan steen tegen steen}\\

\haiku{Firdusi wordt een {\textquoteleft}{\textquoteright}.}{gloeiend vereerder genoemd}{der oude tijden}\\

\haiku{een herschepping van.}{het leven der Oudheid in}{al zijn uitingen}\\

\haiku{Dit  feit lijkt een.}{tegenstrijdigheid en is}{het toch slechts schijnbaar}\\

\haiku{Omdat hij met een,.}{essentieel orgaan mensch}{en tijden herschept}\\

\haiku{Ik bekommer mij}{om historische feiten}{en bronnen zooveel}\\

\haiku{Jammer dat de man.}{niet op de hoogte van de}{jongste feiten is}\\

\haiku{Vgl. ook Ed. Meyer,,, {\textsection}-:}{Geschichte des Altertums}{Erster Band 117118}\\

\haiku{kennis van tempels,,...:}{priesterleven en feesten}{in dit \'e\'ene groote}\\

\haiku{Omdat deze zich.}{evenzeer in groepeeringen van}{feiten openbaren}\\

\haiku{Het verleden en.}{het tegenwoordige is}{een en hetzelfde}\\

\haiku{In den Bijbel b.v..}{geeft Cyrus de vaten uit}{Babylon terug}\\

\haiku{In zijn Lehrbuch der,;}{alten Geographie schrijft}{H. Kiepert p. 193}\\

\haiku{Nog iets anders dan.}{het minachtende woord van}{onzen prof. Hartmann}\\

\haiku{Der J\"ager ist im.}{orientalischen M\"archen}{der Unterweltsmann}\\

\haiku{Deuteronomium.}{wierd gevonden en op naam}{van Moyzes gesteld}\\

\haiku{Avant l'aube, il fait,,}{froid tout est triste l'homme}{reste inquiet}\\

\haiku{Ze liggen echter,.}{niet altijd op hooge punten}{doch ook in dalen}\\

\haiku{Bolland bleef zijn slurf.}{zwaayen en vermorzelde}{wie hem te na kwam}\\

\haiku{Ed. Meyer blijkt ook.}{nog een groot bewonderaar}{van de Achaemeniden}\\

\haiku{Hij bestrijdt Winckler,.}{die beweert dat Astyages}{geen Meed is geweest}\\

\haiku{Firdusi Zie E.,,.}{Burnouf Commentaire sur}{le Yasna p. 400}\\

\subsection{Uit: De oude waereld II: Zonsopgang}

\haiku{Richt de hoorns en zingt....}{met een schallende stem mijn}{mannen der bergen}\\

\haiku{Een misvormende.}{ontsteltenis wrong er over}{heel zijn schoon gelaat}\\

\haiku{En de vingeren.}{mijner kinderen in de}{bescherming van Thot}\\

\haiku{Op zijn rug, dezen....}{tartenden beschimper van}{Azi\"e's Majesteit}\\

\haiku{Hij, Heer van het Al,....}{voortbrengende stier onder}{de negen Goden}\\

\haiku{Eens was Aethiopi\"e,:}{vastgeklonken aan Aegypte}{nu begeerde het}\\

\haiku{Toch kroop het eerst naar,.}{hem toe Peshtudibart en de}{anderen volgden}\\

\haiku{- Groot is uw Heer,... viel.}{schijn-nederig Haman}{op zijn beurt nu in}\\

\haiku{Zij worden door de.}{pijlen onzer knaapjes in}{de vlucht gedreven}\\

\haiku{Doch zij, een stonde,.}{verdwijnen als schaduwen}{in het nevelland}\\

\haiku{- Doen wij altijd... doet.}{ook de heilige kat met}{de zwarte ooren}\\

\haiku{- O Yima, wist ik...}{hoe het zal worden aan het}{einde der tijden}\\

\haiku{Maar de booze geesten.}{hebben ze gevloekt en hun}{trony's roodgebrand}\\

\haiku{- Ik neem hem levend,...}{mee naar Susa tusschen de}{smaragdduifjens}\\

\haiku{Houdt jij ook zoo van?}{ivoor en struisveeren als die}{zwarte grijnzers hier}\\

\haiku{De papegaay zweeg,.}{en er gonsde een nare}{klemmende stilte}\\

\haiku{dat de knaap dien zij,.}{sidderend begeerde haar}{niet lief mocht hebben}\\

\haiku{Laat uw doekdrager.}{u een scheut welriekende}{specerij gunnen}\\

\haiku{hoe zie ik in het,!}{gouden licht van uwe oogen de}{waarheidsdrift fonklen}\\

\haiku{- Ja mist, nevel, mist!}{die tusschen de reten uwer}{vingers weer ontsnapt}\\

\haiku{De gevretene.}{wordt vreter en de vreter}{weer gevretene}\\

\haiku{Ik ken een Spartaan,!}{die veel liever een edelman}{is dan een edel man}\\

\haiku{Uw vruchtboomen laat.}{gij met Calybonischen}{wijn besprenkelen}\\

\haiku{- Al drie dagen leeft.}{hij gillend en hijgend op}{de paalpen gespit}\\

\haiku{Of vallen vrees en?}{verschrikking en duizeling}{ook over jou Spamitres}\\

\haiku{Hij zegt dat ik geen,;}{rijk van het Oosten maar van}{de aarde begeer}\\

\haiku{Mijn Sidonische....}{galey zal mijn troonhemel}{onbezwalkt torsen}\\

\haiku{Ieder woord dat zij....}{zeggen neemt een gedaante}{aan van een wezen}\\

\haiku{Je riekt naar zoeten,...}{narcis en sandelhout deern}{en naar saphraanbrood}\\

\haiku{- Bij den heiligen,......}{Darius ik zie dat je}{gelitteekend bent}\\

\haiku{Rond den Hellespont moet....}{de bodem dreunen van het}{bouwen der schepen}\\

\haiku{Ik deern, ben altijd.}{gelukkig met de dingen}{die ik niet begrijp}\\

\haiku{- Spamitres,... viel hevig,.......}{Xerxes uit ik wil de}{vrouw van mijn broeder}\\

\haiku{schepsel wordt nu al,!...}{vervreten door de groote}{smachtende zonde}\\

\haiku{Want weet Spamitres,... weet....}{schoon meisjen het geslacht der}{Achaemeniden vreest niets}\\

\haiku{Ze beeft, het fijne,....}{zangstertje het lokkende}{dochtertje van Maya}\\

\haiku{- Vertrek kamerling,...,....}{zei in zachte fluistering}{Xerxes en bid}\\

\haiku{{\textquoteright} Aan het slot van zijn, {\textquoteleft}{\textquoteright}:}{boekL'exotisme am\'ericain}{schrijft Gilbert Chinard}\\

\haiku{A very low door in.}{the western side serves}{as an entrance}\\

\haiku{{\textquoteleft}Verhandeling over{\textquoteright}:}{geestenzien en wat daarmee}{samenhangt beweert}\\

\haiku{Vgl. over Hrihor Chapter,:}{II The History of the}{Philistines I}\\

\haiku{Abessyni\"e noemt hij.}{een land der contrasten in}{meer dan een opzicht}\\

\haiku{F\"ur die biblische.}{Literatur ist das von}{groszer Wichtigkeit}\\

\haiku{In het Hebreeuwsch is.}{de rhythmische vaart en dreun nog}{veel geweldiger}\\

\haiku{{\textquoteright} Over {\textquoteleft}the meadow{\textquoteright}.}{in the suburd of the city}{volgt uiteenzetting}\\

\haiku{Perrot en Chipiez}{halen er Herodotus bij}{en doen uitkomen}\\

\haiku{Daneben finden.}{wir auch den Titel K\"onig}{des Nuhselandes}\\

\haiku{il explore en;}{navigateur le fleuve}{des temps r\'evolus}\\

\haiku{Deuxi\`eme Partie, {\textquoteleft}{\textquoteright}.}{Am\'elineau eenExpos\'e}{du syst\`eme geeft}\\

\haiku{{\textquoteright} 45) verbonden met.}{waardevolle opgaven}{over zijn regiment}\\

\haiku{Zie ook ibid. Zweiter,-:}{Band vooral p. 4647 en}{het gansche hoofdstuk}\\

\subsection{Uit: De oude waereld III: Morgenland}

\haiku{Vernietigd wordt het,.}{leugenwoord Het Ware zal}{vernietigen hem}\\

\haiku{Hij werpt zich het \'e\'erst,.}{op de ziel in zwakheid en}{in vreeze ademend}\\

\haiku{De wijze mint de.}{armoe en het stille grauw}{der nederigen}\\

\haiku{Haman moest op zijn.}{hoede wezen en d'ooren}{rekken tot den grond}\\

\haiku{De poort der aarde.}{omstralen zij met den glans}{hunner wetenschap}\\

\haiku{Dadelijk ontstond:}{het eerste visioen van}{het schrikbarende}\\

\haiku{Langzamerhand zag;}{Haman al helderder en}{wonderbaarlijker}\\

\haiku{\'op deed rijzen tot;}{den vorstentulband boven}{de stierenhorens}\\

\haiku{de adder kronkelt.}{zich in den kroondiadeem}{onzer koningen}\\

\haiku{Beseft ge dan n\'u?}{reeds dat gij den twintigsten}{van Thoth moet vreezen}\\

\haiku{Blijf eerst nog zingen.}{bij het vuur der Perzen en}{weerstreef uw lot niet}\\

\haiku{grooter raadsel nog.}{dan het marmer-vlammen}{van hun aardewerk}\\

\haiku{- Gelooft gij dat wij,?}{bestaan uit stof schaduw en}{ongeschapen ziel}\\

\haiku{En ik, Bes, ik laat.}{de zon lachen en jaag de}{treurigheid weg}\\

\haiku{- Omdat zij niet uw,,....}{volk maar \'u bespionneeren}{viel wrang Ruba uit}\\

\haiku{Bij het sterven van.}{den dag verkwijnde ook zijn}{drang tot handelen}\\

\haiku{Ook de natuurstreek}{van Anatoli\"e bedwelmde}{hem weer gansch en al.}\\

\haiku{Hydarnes, aan het,.}{hoofd der Onsterfelijken}{leek van louter goud}\\

\haiku{Ruba zag dat zij}{zich lieten bedienen door}{naakte slavinnen}\\

\haiku{, gelaat aan gelaat.}{nog \'e\'en tel in doodswanhoop}{het licht inhieven}\\

\haiku{Ook hij heeft niets in.}{te brengen bij den Raad der}{Amphyctionen}\\

\haiku{Droomerig, tegen,':}{Ruba in klonk Xerxes}{stem weer aarzelend}\\

\haiku{- Braaf zoo... heel braaf zoo,,....}{mijn Gebiedertje gij leert}{in de ruimte zien}\\

\haiku{Wankel niet altijd.}{naar de tragische grenzen}{der dingen terug}\\

\haiku{het is dood-uit met.}{Pan. En de vallei is heel}{diep en half duister}\\

\haiku{- Het purperkrijt op,,....}{uwe geschminkte wangen mijn}{Koning korrelt los}\\

\haiku{Neem u in acht voor!}{het bloedend gebit van het}{geranselde paard}\\

\haiku{- O Volschapene,.}{hoe zoet is uw geluid bij}{dit broze peinzen}\\

\haiku{Geweld... geweld, dat....}{is uw schuimende kracht uw}{brijzelende drift}\\

\haiku{Kermend-zacht van,':}{stem klonk het haperend van}{Xerxes lippen}\\

\haiku{Weet wel, de goden.}{van Hellas verduisteren}{de menschenbreintjes}\\

\haiku{O mijn heerlijke...}{Achaemeniden-Koning als}{gij wist hoe lekker}\\

\haiku{- De dood sluipt er op,.....}{de teenen rond zong Pan eens in}{liefde-verdriet}\\

\haiku{- Uw bode naar het,.}{orakel der Branchiden kwam}{terug zonder stem}\\

\haiku{Hoor Arche{\"\i}sche,,.}{Hera juichen haat juichen}{rauw van stem en kreet}\\

\subsection{Uit: Zegepraal}

\haiku{Dit boek is voor jou,.}{subliem kind van verbeelden}{en werkelijkheid}\\

\haiku{Maar dan, mijn geluk,,!}{mijn bevend geluk dat ik}{jou heb Florence}\\

\haiku{mij iederen avond.}{en iederen doodstillen}{nacht op mijn kamer}\\

\haiku{Florence, zie die,....?}{prachtige amfora is}{dat geen lijntooverij}\\

\haiku{Het gigantische,.}{er in komt op me \`af en}{ik stort er op neer}\\

\haiku{Ik hoorde door m'n '.}{oorent geruisch van}{oceanen vloeien}\\

\haiku{Hoe vervloekte ik!}{mijn minnarijen met die}{heerlijke vrouwen}\\

\haiku{riep ze heel zacht, zoo.}{verbergend haar schrik om mij}{met te ontstellen}\\

\haiku{ik smeekte 'r jou,,.}{te beschermen te helpen}{als ik toch dood ging}\\

\haiku{Je weet Florence.}{hoe mijn hoofd altijd vol zat}{met melodie\"en}\\

\haiku{In ieder dezer '!}{vlamtt vuur van een Carmen}{uit de gouden oogen}\\

\haiku{Zoo stom, zonder kreet.}{vergaat de vrouwendans in}{woesten zinnenwaan}\\

\haiku{Je handen wiegen,.}{nog traag boven je hoofd in}{verslappenden wuif}\\

\haiku{- Als ik iemand zie '.}{vallen word ik kwaad en schiet}{direkt inn lach}\\

\haiku{Elk vingertje zingt,.}{een gracieus zangetje}{van lijnbekoring}\\

\haiku{de maandelijksche.}{post zond die ik ook v\'o\'or m'n}{ziekte van hem kreeg}\\

\haiku{En nu moet je den.}{bitsen ineengedrongen}{kerel naast haar zien}\\

\haiku{Daarom wachtte ik,,.}{al maar wachtte ik tot ik}{wat meer zeggen kon}\\

\haiku{met toekomst-wezens,,....}{vol vrijen ademhaal groot en}{overvloedig van kracht}\\

\haiku{Dat geluk liet me,.}{nooit meer los ook in mijn n\`og}{bangere uren niet}\\

\haiku{'t Was soms of ik.}{telkens iemand de trappen}{hoorde afrollen}\\

\haiku{goud en violet.}{van weerschijn-verrukking}{en droomrige pracht}\\

\haiku{Als je hoofdje een,}{beetje zakt op je borst en}{je oogen kijken}\\

\haiku{ze wondert in mij.}{als de zachte zang van heel}{stil gemijmer}\\

\haiku{En daarom, daarom,.}{Florence is mijn snakken}{naar jou nog grooter}\\

\haiku{Nu wachtte ik m'n,.}{benauwingen af vreesde}{ik ook veel minder}\\

\haiku{Die geur beroerde.}{toen ontzettend en akelig}{m'n doodsgedachten}\\

\haiku{O Florence, hoe '.}{t lieve vrouwtje me toen}{treurig aankeek}\\

\haiku{Ik had ze in 't '.}{hartje van de stad verlangd}{opt Rembrandtsplein}\\

\haiku{Ik was overstelpt van,.}{ontroering om zooveel moois}{zooveel goddelijks}\\

\haiku{de stralende sneeuw,.}{den verblindenden schitter}{van blank-wit licht}\\

\haiku{Zijn zelfvergoding '.}{isn andere pool van}{\`ons kommunisme}\\

\haiku{m'n jonge vriend Sam,,.}{van Daalen mijn dokter vier}{jaar ouder dan ik}\\

\haiku{Ik wist niet waar ik,.}{heen moest ik alleen met m'n}{kosmischen godsdienst}\\

\haiku{wat 'n troostelooze;}{wolken drijven er van daag}{in de hemelen}\\

\haiku{Wat 'n grauwpaarse!}{daken triesten er in de}{schemerende stad}\\

\haiku{Er zijn parijsche.}{schepsels die ik nooit intiem}{zou willen  zien}\\

\haiku{Mijn verhuizing naar.}{achter gebeurde op een}{somberen herfstdag}\\

\haiku{Jou levensgeur woei '.}{op me \'a\'an als d'aroma}{vann druivenveld}\\

\haiku{Ik verdoemde de!}{engelachtige streeling}{van vrouwenmonden}\\

\haiku{De liefde als 'n, '!}{hemelsch preludium Eros}{opn goud troontje}\\

\haiku{Er is in mij zelfs.}{een bange schuchterheid om}{je te naderen}\\

\haiku{Maar 't is alles,, ' '}{\`echt waar echt want iedereen}{ziett en zegtt.}\\

\haiku{En dan te denken,,!}{dat ik uit de verte zoo}{dikwijls jaloersch was}\\

\haiku{Ik sidder bij 't, '.}{hooren van je stap bijt}{kreuken van je rok}\\

\haiku{De zonneboog kleurt.}{de hemelpoorten daar met}{wondre vuurbloemen}\\

\haiku{Ik hoor nu je lach,.}{je zilver-trillenden}{zangerigen lach}\\

\haiku{Het grootst en machtigst.}{opneemvermogen is toch}{m\`enschelijk begrensd}\\

\haiku{Ach Florence, wat.}{mij dat weer heeft doen snikken}{van ontsteltenis}\\

\haiku{Ik zag v\'o\'or me, den, '.}{blinden Milton maart scherpst}{den dooven Beethoven}\\

\haiku{Want hij durft \`alles,.}{dadelijk nadoen zonder}{aarzelen of schrik}\\

\haiku{Dan botsen ze, loopt.}{Arie ontdaan naar z'n moeder}{of mij en huilt stil}\\

\haiku{je kunt, wat je dan,.}{te pakken hebt heeft je ten}{minste goed gedaan}\\

\haiku{Iedere stem die.}{ik hoorde achter den muur}{gaf ik gestalte}\\

\haiku{Ik voelde all\'e\'en.}{menschelijke eenheid en}{zielegelijkheid}\\

\haiku{'t Vonkelde er,}{heet in de tropische zon}{en nu en dan hield}\\

\haiku{Toen hoorde ik hoe;}{hij zelf aan vreeselijke}{slapeloosheid leed}\\

\haiku{Want 't geluk van.}{het socialisme zit}{niet in de boeken}\\

\haiku{Ze hadden strakke,,.}{weeke hartstochtelijke en}{droomfijne monden}\\

\haiku{die boekenverkoop.}{heeft mij bang-verweende}{nachten gegeven}\\

\haiku{Florence, was jij, '!}{hier geweest ik zout je}{afgesmeekt hebben}\\

\haiku{Florence, heb je,,?}{me toen in die schemeruren}{niet hooren roepen}\\

\haiku{Telkens, onder nieuw,,.}{lichtreflex vergoocheling}{van buik kop en staart}\\

\haiku{Toch voelde ik me, '.}{m\`ett loopen wat beter}{en gelukkiger}\\

\haiku{Soms ijlde ik, riep, '.}{ik Zus w\`a\`art moedertje}{toch verborgen zat}\\

\haiku{Kuische vrouwtjes;}{bolderde hij de rokken}{\`op als windzeilen}\\

\haiku{Ze laten 'n brok ' '.}{vant groote leven int}{kleine leven zien}\\

\haiku{'t Leek me, of je ',.}{vant boereland kwam zoo}{frisch en geurig}\\

\haiku{'n Week lang bleef ik.}{dood voor alle andere}{dingen om mij heen}\\

\haiku{Maar ik kon bijna,.}{niet ademen zoo zwaar werkte}{de lucht op me in}\\

\haiku{Op 'n middag nam,.}{hij zich plots voor alleen met}{me op straat te gaan}\\

\haiku{Het klonk me soms heel,,,!}{geheimzinnig zacht en ver}{en zoo teer zoo teer}\\

\haiku{Tot die schepsels leek.}{hij te behooren in die}{verbitteringsuren}\\

\haiku{aan den man, die mij ' '.}{zelf weern endt leven}{ingetrokken had}\\

\haiku{Hij zelf erkende,,.}{dat ik er vreeselijker}{aan toe was dan hij}\\

\haiku{Ze stonden er door,.}{elkaar uit kweekkassen en}{wild uit de natuur}\\

\haiku{hoe ontroerde mij ',.}{t landgerucht meer nog dan}{de stem van de stad}\\

\haiku{En hoe verder ik,.}{ging hoe meer de landarbeid}{zich voor mij opdrong}\\

\haiku{Lieve, ken je de?}{voorstelling dier seringen}{plukkende schepsels}\\

\haiku{T\`ot 't dier weer in '.}{m opgetergd wordt door nood}{en woesten aandrift}\\

\haiku{'t Duurt niet lang. 't,!}{Dier woest en rauw gromt heel gauw}{\`op in die menschen}\\

\haiku{Of trappen ze hun!}{vrijster tegen den buik als}{de nachtbuit slecht was}\\

\haiku{Prachtig geveeg en,.}{lijngekras elk streepje z'n}{beduidenis toch}\\

\haiku{, want alles bijna!}{is de moeite waard apart er}{wat van te zeggen}\\

\haiku{'t Is er gesmook,,,.}{gezuip gelal gestoei en}{potsierlijk gedans}\\

\haiku{Wat 'n buurt, wat 'n ' '.}{huiver van sombertes}{winters ens avonds}\\

\haiku{Florence, in mij '.}{schreit niet de smart van \'e\'en ziel}{maar vann wereld}\\

\haiku{In mij jubelt niet ' '.}{t geluk van \'e\'en mensch maar}{vant menschdom}\\

\haiku{dat 't leven z\'o\'o,!}{groot zoo allergeweldigst}{in mij wil werken}\\

\haiku{Had ik 't voor drie,.}{maanden gehoord ik zou er}{onder bezwijmd zijn}\\

\haiku{dat er niets, niets van,.}{me bleef geen heugenis en}{geen herinneren}\\

\haiku{t Golfgeklots is '.}{t zangerige verhaal}{van dat verlangen}\\

\haiku{ik geef ze w\`at van,.}{m'n beetje en zij geven}{mij van hun beetje}\\

\haiku{Ik zorg altijd voor.}{prachtige aanwijzingen}{en introdukties}\\

\haiku{Maar je houdt je in,.}{uit vrees dat die menschen je}{toch niet begrijpen}\\

\haiku{er is 'n vreugde. '}{dan in me die me alle}{leed doet vergeten}\\

\haiku{'t leven dat de.}{heele menschheid aandraagt}{haar smart en geluk}\\

\haiku{kind was ik nu niet, '}{zwak zoo zwak dat mijn lichaam}{siddert ondert}\\

\haiku{Dacht je dat 't 'n, '?}{ongelukkige liefde}{wast nonnetje}\\

\haiku{soort waarzeggerij,?}{en al uiterlijk vertoon}{van mystiek gedoe}\\

\haiku{Je rammelt leeg, en '.}{je wordt uitgeschud alsn}{zak aardappelen}\\

\haiku{Ik weet wel dat met,.}{alles gezegd k\`an worden}{niet alles verwerkt}\\

\haiku{Want weet lieve, dat '.}{ik me niet door de smart van}{t leven laat slaan}\\

\haiku{Florence, wil je,?}{weten hoe precies hoe ik}{mij dat voorstel}\\

\haiku{En hoe schreiend slecht.}{is niet de Clotilde in}{Dokter Pascal}\\

\haiku{Dan kon hij schreien;}{van angst dat er niet alles}{tegelijk uitkon}\\

\haiku{Van de lui die hun {\textquoteleft}{\textquoteright}}{w\`erk z.g. van zich kunnen}{afzetten begrijp}\\

\haiku{In 't Hollandsch kan, ';}{je teederste dingen broos}{alsn adem vatten}\\

\haiku{dat stille happen, '....}{die droefnis om den mond bij}{t droge slikken}\\

\haiku{En ik ben blij, dat.}{ik me verzet heb tegen}{die smartaandoening}\\

\haiku{Werelden bouwen,,,.}{dat is groot werk dat werk wil}{ik doen mo\`et ik doen}\\

\haiku{Deze eeuw eischt.}{de belichaming van het}{wereldepos}\\
