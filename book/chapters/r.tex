\chapter[12 auteurs, 946 haiku's]{twaalf auteurs, negenhonderdzesenveertig haiku's}

\section{Henri Rademaekers}

\subsection{Uit: Sancta Innocentia}

\haiku{Kerdju, det waar get!}{gewaes es d'r noe eine}{drin had gezaete}\\

\haiku{Het waar flink gaon}{vrere en de windj bloos hem}{door zien dun jeske}\\

\haiku{Ware die knien dan,?}{t\'och van de Keuningin wie}{Wulmke had gezach}\\

\haiku{{\textquoteleft}det kos hem eine{\textquoteright}.}{maondj of ettelik in}{de Pollartsjtraot}\\

\haiku{Angers zoot d'r gaer,.}{in de b\"os mer vandaag waar}{d'r gei plezeer aan}\\

\haiku{In-ins ware, ',.}{ze dao mert waar nog v\"a\"ol}{te vreug zach moder}\\

\haiku{Bang det ze hem dao,}{hole waar d'r neet meer want}{vader had gezach}\\

\haiku{{\textquoteright} {\textquoteleft}Ja, mijnheer, hij heeft,{\textquoteright}.}{het w\`el gedaan zach opins}{Jeuke zie moder}\\

\haiku{Krintebreudjes en,.}{flaetjes dao z\`ol d'r hem ins}{ein vaeg aan gaeve}\\

\haiku{dzjing dzjing ging 't en.}{opins sjt\`ong veur h\"a\"or naas waat}{ze betale m\`oste}\\

\haiku{Moder sjt\`ong get te}{drentjele en ging toen nao}{eine lange gaam}\\

\haiku{Die hadde z\`onne,.}{veldjwachter auch waal gel\"os}{die Bataviere}\\

\haiku{{\textquoteright} en Jeuke, dae zich,:}{pas gebiech had durfde neet}{te lege en zach}\\

\haiku{Toen had moder hem}{eine wermen dook \`om de}{kop gedaon mer}\\

\haiku{Det had ze zellef,.}{al ins \`ongerv\`onje wie}{det ging mit zo fonds}\\

\haiku{Hae verzoop haos,......}{in dae sjtool en op ins ging}{d'r zjwoek zjwoek}\\

\haiku{Jeuke sjt\`ong t\"osse,.}{de luuj in mit vader dae}{de late sjich had}\\

\haiku{{\textquoteright} Moder bloos ins get}{sjtaof van de kraon aaf}{en bloos hem vaort}\\

\haiku{In Remunj, langs 't,.}{sjpaor dao sjt\`elde zich}{de persessie op}\\

\haiku{En later z\`ol ze, '.}{verrijze op d'n daag van}{t laatste oordeel}\\

\haiku{Jeuke had nao d'n.}{opt\`och gekeke wie dae}{bie h\"a\"or veurbie kwaam}\\

\haiku{* * * ~ Berb, de maag van,;}{meneer Pesjtoor waar ei good miens}{mit ein hert van g\`oldj}\\

\haiku{D'n Enesbejer ging '.}{mit zienen h\`ondjt veldj in}{\`om ein oer of twee}\\

\haiku{nette p\`oppekop.}{had wie die dames die in}{de zjoernaal sjt\`onge}\\

\haiku{neet meer verder k\`os '.}{zinge en midde int}{leedje bleef sjtaeke}\\

\haiku{Toen w\`ol Oswald auch}{zie vader op dezelfde}{meneer oet d'n tied}\\

\haiku{gezichter taege,.}{hem sjneej en toen k\`os d'r zich}{neet haaje van de lach}\\

\haiku{D'n deef m\`os door de.}{achterdeur van de keuke}{zeen binne gek\`omme}\\

\haiku{De res zoot veilig,.}{in de brandjkas wo d'n deef}{neet aangewaes waar}\\

\haiku{Aan Juffrouw Mieke.}{de Krekel te Zandbergen}{Limburg Nederland}\\

\section{Hilda Ram}

\subsection{Uit: Schetsen, novellen en vertellingen}

\haiku{Nu krijg ik moed en.}{wie weet word ik u nog geen}{lastige klante}\\

\haiku{5e afl. bl. 233).}{en volg. moeten wij hier ook}{terloops aanstippen}\\

\haiku{sa po\'esie montre.}{alors l'homme du m\'etier et}{non le po\"ete}\\

\haiku{Het duurde tot in,.}{1894 vooraleer Hilda Ram}{nog iets liet drukken9}\\

\haiku{Ik kampte om eenen,,,!}{blik Die thans vol hulde me}{als een zon bestraalt}\\

\haiku{Als verloofde van ';}{Johan trekt Eleonora naart}{hof van Portugal}\\

\haiku{{\textquoteright} Emmanu\"el is:}{getroffen door Eleonora's}{kinderlijke trouw}\\

\haiku{, En wanneer ze 't}{voorhoofd beurde Scheen zoo ver}{haar blik te zwerven}\\

\haiku{Het metrum is niet:}{dichterlijk genoeg en niet}{bijzonder verzorgd}\\

\haiku{Zijn zij niet op de,}{hoogte van wat hen gevraagd}{wordt zoo verzoeken}\\

\haiku{en ik moest nu een,!}{sermoentje ondergaan dat}{niet van de poes was}\\

\haiku{Zoo kon ik zeker,!}{mijnen stiel niet leeren d\'at kon}{niet blijven duren}\\

\haiku{{\textquoteright} - Nog eenige dagen.}{leefde hij en dan luidde}{de doodsklok voor hem}\\

\haiku{Eenieder begon!}{te zoeken en haast was het}{raadsel opgelost}\\

\haiku{Zooals ik gezegd heb,,.}{droeg ze Oomken opgevuld}{uit schoonheidsgevoel}\\

\haiku{Janneken hield zich ',!}{stil en als hijs morgens}{wakker werd wat vreugd}\\

\haiku{en zijn paard er bij.}{en hoopen met lekkers voor}{hem en zijn broerken}\\

\haiku{Wapper voor haar, die,:}{met zijn vinger naar haar hoofd}{wijzend haar toeriep}\\

\haiku{Maar nauwelijks was,.}{het er af of hij ging weer}{zijnen ouden gang}\\

\haiku{Lange Begge!}{zou de les gespeld hebben}{aan wie het durfde}\\

\haiku{Op Lange Wapper.}{had de arme weduwe}{niet  eens gedacht}\\

\haiku{Ze bogen hun hoofd.}{onder den bevrijdenden}{zegen des priesters}\\

\haiku{Ons Janneken moet '.}{eenen bloempot  dragen naar}{t Kindergasthuis}\\

\haiku{{\textquoteright} {\textquoteleft}Vrouwken, spreekt hij, er,.}{kwam tot mij iemand die me}{geheel overreedde}\\

\haiku{Loon het, Heere, den,!}{goeden Engel die zijne}{redding bewerkte}\\

\haiku{{\textquoteleft}Daar, vrouw, ga, koop een,!}{bloempot met mijn borrelgeld}{ik drink niet vandaag}\\

\haiku{niet om langer te,,.}{kunnen slapen neen om naar}{de vroegmis te gaan}\\

\haiku{ze is binnen nog ' ' '.}{t eene ent andere}{aant wegschikken}\\

\haiku{{\textquoteright} fleemt hij en moedigt.}{haar aan met een stootje van}{zijnen elleboog}\\

\haiku{{\textquoteright} {\textquoteleft}Ik ook niet, moeder,,.}{onderbreekt haar Toon laat ons}{maar seffens trouwen}\\

\haiku{de bekers en al '.}{t ander tafelgerief}{blinken als zilver}\\

\haiku{Afwasschen en het,.}{huis uitkeren gaan gauw als}{de jongens weg zijn}\\

\haiku{Later komt vader '.}{thuis en om 7 uren gaan de}{kleinen naart bed}\\

\haiku{{\textquoteright} En op het bewijs,,.}{dat Token gelijk heeft moet}{Toon niet lang wachten}\\

\haiku{In zijne jongens.}{weet hij iederen aanleg}{te onderscheiden}\\

\haiku{En zij vinden geene.}{woorden om er de schoonheid}{van te beschrijven}\\

\haiku{Nooit zoudt ge denken, '.}{dat zet minste uitstaans}{met vodden hebben}\\

\haiku{antwoordt vieze Trien,!}{in Token's plaats men moet nooit}{den moed opgeven}\\

\haiku{{\textquoteright} roept Toon en, met de,.}{handen in zijne haren}{loopt hij de deur uit}\\

\haiku{Weet, dat Token en,.}{Toon hun deel hadden van smart}{en angst en droefheid}\\

\haiku{Zalig degenen, '!}{diet gegeven is er}{zulke te weenen}\\

\haiku{En wil er een den,:}{slechten weg op gij zult tot}{hem gaan en zeggen}\\

\haiku{Sedert dien was hij.}{van een kleinen een grooten}{baas geworden}\\

\haiku{want die naam, Poeier,.}{en Kruit zou den drager er}{van zelf overleven}\\

\haiku{Dat {\textquoteleft}zoo{\textquoteright} klonk weinig.}{aanmoedigend en een wolk}{betrok zijn gelaat}\\

\haiku{ik zou niet gaarne,!}{hebben dat hij in mijn huis}{armen en beenen brak}\\

\haiku{Zoo zat hij eens te,,.}{wiegen te wiegen dat ik}{er wakker van werd}\\

\haiku{{\textquoteright} {\textquoteleft}Ja, ja, antwoordde,!}{Hoppert maar ik dank liever}{alles u alleen}\\

\haiku{Zij wist ook wel, als,.}{het ernstig was hoe ze haar}{vischje vangen moest}\\

\haiku{{\textquoteleft}Hang het nu niet uit,,{\textquoteright}.}{Peerken zei Mevrouw Hoppert}{met ontroerde stem}\\

\haiku{Geen kwartier verliep,,.}{of terug kwam hij met een}{politiedienaar}\\

\haiku{Peerken was niet meer,.}{te zien en er werd niet meer}{van hem gesproken}\\

\haiku{Er waren zooveel,.}{antwoorden als hoofden maar}{Otto's stem klonk luidst}\\

\haiku{{\textquoteleft}Hoort eens, meisjes, voor,.}{we beginnen wil ik u}{eene kunst laten zien}\\

\haiku{Miss Holding was van.}{haren troon gedaald om den}{oproer te dempen}\\

\haiku{Zij is de oudste.}{en levert ons meer spel dan}{de andere twee}\\

\haiku{in hare zwarte,,.}{oogen spraken bewondering}{ja verrukking uit}\\

\haiku{Eene nieuwe wereld,,.}{de wereld van den geest was}{voor haar opengegaan}\\

\haiku{want daar waren er.}{in oorlogstijd honderden}{vermoord geworden}\\

\haiku{huisde met  haar,.}{en dan nog het eenig kind van}{die dochter Fonsken}\\

\haiku{In gansch de wereld.}{ging eene rilling door alle}{eerlijke harten}\\

\haiku{zijn ooren waren.}{zoo doorschijnend en stonden}{zoo ver van zijn hoofd}\\

\haiku{Waar blijven ze uit,,?}{de vijanden als het iets}{of iemand goed gaat}\\

\haiku{Hij droeg de vlag en?}{ging vooruit en d\'at zou hij}{moeten opgeven}\\

\haiku{{\textquoteright} {\textquoteleft}Begin maar gauw te,.}{eten gij hebt een gezicht als}{een hongerlijder}\\

\haiku{Nu is hij kreupel,.}{en zit in een rolstoel maar}{hij heeft het verdiend}\\

\haiku{{\textquoteright} En weer liep hij rond.}{om zijne volgelingen}{bijeen te zoeken}\\

\haiku{Daar lag haar Fonsken.}{met het hoofd in bebloede}{doeken gewonden}\\

\haiku{Och, manneken, slaap,.}{nu wat en zwijg anders zult}{ge niet genezen}\\

\haiku{{\textquoteright} En dan legde hij}{zijne hand op de vlag en}{fluisterde zooveel}\\

\haiku{Dood van den hertog.}{van Brabant.- Geboorte}{van Prins Boudewijn}\\

\subsection{Uit: Slachtoffers voor Transvaal}

\haiku{want daar waren er.}{in oorlogstijd honderden}{vermoord geworden}\\

\haiku{In gansch de wereld.}{ging eene rilling door alle}{eerlijke harten}\\

\haiku{zijn ooren waren.}{zoo doorschijnend en stonden}{zoo ver van zijn hoofd}\\

\haiku{Waar blijven ze uit,,?}{de vijanden als het iets}{of iemand goed gaat}\\

\haiku{Hij droeg de vlag en?}{ging vooruit en d\'at zou hij}{moeten opgeven}\\

\haiku{{\textquoteright} {\textquoteleft}Begin maar gauw te,.}{eten gij hebt een gezicht als}{een hongerlijder}\\

\haiku{{\textquoteright} En ze stelde hem,.}{allerlei vragen doch hij}{kon niet antwoorden}\\

\haiku{Nu is hij kreupel,.}{en zit in een rolstoel maar}{hij heeft het verdiend}\\

\haiku{{\textquoteright} En weer liep hij rond.}{om zijne volgelingen}{bijeen te zoeken}\\

\haiku{Daar lag baar Fonsken.}{met het hoofd in bebloede}{doeken gewonden}\\

\haiku{Och, manneken, slaap,.}{nu wat en zwijg anders zult}{ge niet genezen}\\

\haiku{{\textquoteright} En dan legde hij}{zijne hand op de vlag en}{fluisterde zooveel}\\

\section{Relham}

\subsection{Uit: De man achter de schermen}

\haiku{De auto bracht ons.}{intussen naar de villa}{van mijn gastheer}\\

\haiku{- Bepaald vinden zult!}{U de juwelen en het}{geld zeker ni\`et}\\

\haiku{En soms beweer je,!}{dat ik de nieuwsgierigste}{mens op aarde ben}\\

\haiku{Een koel windje streek,.}{door het gebladerte dat}{zachtjes ritselde}\\

\haiku{Maar op het ogenblik.}{is het mij onmogelijk}{meer te vertellen}\\

\haiku{Z\'o hartelijk te,.}{lachen als ik hem slechts zeer}{zelden gezien had}\\

\haiku{Ik heb hem gevraagd.}{hier te komen en hij heeft}{het me toegezegd}\\

\haiku{Precies te twaalf uur.}{verscheen de commissaris}{met z'n inspecteurs}\\

\haiku{En ik begreep, dat,.}{ik me aangesteld had als}{een klein dom meisje}\\

\haiku{\'e\'en onvoorzichtig.}{woord kan mr. Tarani v\'e\'el}{kwaad doen en den m.a.d.s}\\

\haiku{Mocht U mij nodig,.}{hebben dan ben ik altijd}{voor U te spreken}\\

\haiku{{\textquoteright} Donald knikte met.}{een onverschillig gezicht}{en sloot even z'n ogen}\\

\haiku{{\textquoteright} {\textquoteleft}Juist, mijn jongen, dat.}{is het volgende punt van}{ons werkprogramma}\\

\haiku{Hij zweeg en Donald:}{trok de conclusie uit z'n}{mededelingen}\\

\haiku{Mijn personeel heeft - -.}{sinds \'e\'en week dus al lang v\'o\'or}{de diefstal vrij af}\\

\haiku{{\textquoteleft}Mr. Tarani, ik,}{weet op het ogenblik m\'e\'er van}{de zaak af dan U.}\\

\haiku{De zon scheen nog even, -.}{opgewekt de hemel was}{nog even vlekkeloos}\\

\haiku{{\textquoteright} {\textquoteleft}Je bedoelt, dat ik?}{er niet al te snugger uit}{zie op dit ogenblik}\\

\haiku{{\textquoteright} {\textquoteleft}Goed, dan zal ik je.}{nu een kort college in}{het schieten geven}\\

\haiku{De gemaskerde.}{onbekende herstelde}{zich verrassend gauw}\\

\haiku{Doch een gepaste.}{vermakelijkheid is wel}{aan te bevelen}\\

\haiku{) Er werden massa's.}{vruchten-ijs besteld en naar}{binnen gelepeld}\\

\haiku{De ander viel hem {\textquoteleft}!}{echter dadelijk in de}{rede met eenAha}\\

\haiku{We hadden toch naar {\textquoteleft}{\textquoteright}.}{denbaas gevraagd en waren}{hier binnengeleid}\\

\haiku{Niet in de richting,.}{van de speelzaal doch juist naar}{de andere kant}\\

\haiku{{\textquoteright} begon ik de \'e\'erste.}{van een hele rij vragen}{te formuleren}\\

\haiku{{\textquoteleft}Johny, - n\'u komt!}{pas het belangrijkste deel}{van onze speurtocht}\\

\haiku{en heb  ik al.}{mijn conclusies tot nu toe}{op los zand gebouwd}\\

\haiku{U kunt me schrijven,.}{naar Milaan of naar Rome}{poste restante}\\

\haiku{{\textquoteleft}Ja, maar{\textquoteright}, viel me toen, {\textquoteleft}!}{inover een kwartier vertrekt}{het avond-vliegtuig}\\

\haiku{Zo konden we in '.}{t voorbijgaan slechts enige}{woorden opvangen}\\

\haiku{Maar nu is er een,:}{kleinigheid die ik niet van}{me kan afzetten}\\

\haiku{{\textquoteleft}Vanavond precies om,,?}{half acht hier in de villa}{h\`e afgesproken}\\

\haiku{Ik weet alleen, dat.}{signor Corelli vreselijk}{opgewonden is}\\

\haiku{{\textquoteright} {\textquoteleft}Hoe weet je dan...{\textquoteright} {\textquoteleft}Geen,!}{vragen nu alsjeblieft ik}{moet nu rust hebben}\\

\haiku{Wij zwegen beiden - -:}{wat moest je d\'a\'arop antwoorden}{en hij vervolgde}\\

\haiku{{\textquoteright} {\textquoteleft}Dan zal ik me met.}{deze verzekering maar}{tevreden stellen}\\

\haiku{In de volgende, -:}{seconde hoorden we een}{gil dan uitroepen}\\

\haiku{{\textquoteleft}Ik zal hem direct{\textquoteright},.}{naar een ziekenhuis brengen}{zei Corelli tot mij}\\

\haiku{Resultaat van dit.}{gesprek kon ik nog niet te}{weten komen}\\

\haiku{Het gezelschap stond,.}{ook op omdat de pauze}{juist was begonnen}\\

\haiku{Wij stonden enige.}{ogenblikken roerloos op \'e\'en}{plaats en luisterden}\\

\haiku{Hij bukte zich en.}{bleef enkele seconden}{gespannen kijken}\\

\haiku{Johny{\textquoteright} - hij greep - {\textquoteleft}.}{mijn handje moet proberen}{je te beheersen}\\

\haiku{We stapten haastig.}{in en de auto vloog over}{de verlaten weg}\\

\haiku{{\textquoteright} {\textquoteleft}Geduld, Johny,!}{nog twee dagen en je zult}{ook d{\`\i}t begrijpen}\\

\haiku{{\textquoteright} vroeg hij, alsof het ' '.}{niet \'e\'en uurs nachts maar pas}{n uur of tien was}\\

\haiku{Noch dat iemand het,.}{huis binnenkwam noch dat een}{schot afgevuurd was}\\

\haiku{{\textquoteright} {\textquoteleft}Wij allen, zoals, -.}{U ons hier ziet en dan nog}{signor Sibilio}\\

\haiku{Als hij komen wil,.}{zal hij in uiterlijk 10}{minuten hier zijn}\\

\haiku{Nu zal ik mij tot.}{het aller-voornaamste}{moeten beperken}\\

\haiku{Ik lette meer op,;}{dat eigenaardige beeld}{dan op zijn woorden}\\

\haiku{Even later was de,...}{grote zwarte auto om}{de hoek verdwenen}\\

\haiku{Misschien zou het in, -?}{Amerika mogelijk zijn}{maar in Europa}\\

\haiku{Ik geloof, eerlijk,,!{\textquoteright}}{gezegd dat je me voor het}{lapje houdt Donald}\\

\haiku{{\textquoteright} {\textquoteleft}Omdat ik wou, dat -.}{U samen met ons er heen}{ging naar zijn woning}\\

\haiku{Dat was de enige,.}{keer hedenmorgen dat de}{deur geopend werd}\\

\haiku{nee, die niet, - U moet.}{rekenen dat die lange}{gang er tussen ligt}\\

\haiku{Alleen als de deur,.}{van binnen-uit geopend}{wordt blijft het alarm uit}\\

\haiku{{\textquoteleft}Sluit alle deuren,.}{af zodat niemand ons in}{de rug zal komen}\\

\haiku{Als een van de drie,!}{ook maar een vinger beweegt}{dadelijk schieten}\\

\haiku{een medewerker.}{van hem ging met den patient}{naar het ziekenhuis}\\

\haiku{Ondertussen was.}{de \`echte Corelli naar de}{villa gereden}\\

\section{M. Revis}

\subsection{Uit: 8.100.000 m{\textthreesuperior} zand}

\haiku{Helaas is het slechts,.}{om te zien dat  het \'o\'ok}{geld verdienen kan}\\

\haiku{Dan komt men er wel,.}{achter dat alles op zijn}{pooten terecht komt}\\

\haiku{{\textquoteright} Ook Kees voelde, dat.}{hij voor iets meer bestemd was}{dan polderwerker}\\

\haiku{Een rossig verlicht.}{stationnetje kwam soms}{in het duister op}\\

\haiku{{\textquoteright} De journalist vindt.}{de vergelijking met het}{kantwerk zeer geslaagd}\\

\haiku{Paarden, arbeiders,,,,:}{paardenzweet menschenzweet zand}{keien en karren}\\

\haiku{Hij zou alleen graag,,.}{kinderen gehad hebben}{of \'e\'en kind een zoon}\\

\haiku{En 1930 (dit verhaal,)!}{gaat niet verder want toen stierf}{van Dool 2020 en 8.100.000}\\

\haiku{De dragline is,.}{een prachtmachine ze schept}{op en stort 2 M3}\\

\haiku{Als je een dijk kunt,.}{leggen ga je vanzelf aan}{spoorbanen denken}\\

\haiku{En zoo groeit de weg,.}{uit den grond de nieuwe weg}{voor het snelverkeer}\\

\haiku{Maar zijn daar dan niet?}{de eerste ketenen van}{het Rotsgebergte}\\

\haiku{Morgen is er weer,,,.}{een dag grauw troosteloos weer}{een Novemberdag}\\

\haiku{Er is flauw licht in,.}{de kamer maar het bed is}{in de schemering}\\

\subsection{Uit: Kringloop. De geschiedenis van een schip}

\haiku{aan den anderen,;}{kant komt het weer te voorschijn}{versmald en verlengd}\\

\haiku{Het een ontstaat uit,.}{het ander zooals de eene mensch}{uit de andere}\\

\haiku{Hij wordt nu door de.}{teekenaars theoretisch}{in elkaar gezet}\\

\haiku{In het voorjaar van {\textquoteleft}{\textquoteright}.}{1911 wordt de kiel van dePrins}{van Oranje gelegd}\\

\haiku{Het heeft den nieuwen,,.}{nog slapenden reus gezien}{daar aan den oever}\\

\haiku{Een dag later ligt {\textquoteleft}{\textquoteright}.}{dePrins van Oranje op de}{reede van Tanger}\\

\haiku{Nevels rijzen 's.}{nachts uit zee op en drijven}{de schepen uiteen}\\

\haiku{kinabast, huiden,,,,,,,.}{kapok suiker tabak rijst}{thee koffie copra}\\

\haiku{Alleen zijn snelheid {\textquoteleft}{\textquoteright}.}{is iets grooter dan die van}{dePrins van Oranje}\\

\haiku{Op een dag met ruw.}{weer vaart de U X door de}{sluizen naar buiten}\\

\haiku{In dat geluid kan,.}{men zich niet vergissen dat}{was een kanonschot}\\

\haiku{Dat gebeurt op 13,.}{Juli een mooie zomerdag}{vol zon en wind}\\

\haiku{Ten slotte doet een.}{officier de ronde door}{het verlaten schip}\\

\haiku{Ter weerszijden van,.}{het schip is nu de breede}{blinkende rivier}\\

\haiku{De wind is Noord, de,.}{zee wordt vlakker naarmate}{de middag vordert}\\

\haiku{Een tankboot stuurt zijn {\textquoteleft}{\textquoteright}.}{lange voorschip achter de}{Prins van Oranje langs}\\

\haiku{uit den schoorsteen komt,:}{een dikke walm waarin het}{schijnt te weerlichten}\\

\haiku{De andere boot {\textquoteleft}{\textquoteright} {\textquoteleft}{\textquoteright}.}{van deMaas is dePrins van}{Oranje genaderd}\\

\subsection{Uit: Paviljoen van glas}

\haiku{Potgieter steekt met;}{hoofd en schouders boven zijn}{tijdgenoten uit}\\

\haiku{Diep in hem is een,,.}{ruimte waar geen gedachten}{zijn een gapend gat}\\

\haiku{In de statuten:}{wordt het doel der maatschappij}{aldus omschreven}\\

\haiku{Een groot orkest speelt.}{de Jubel-ouverture}{van C.M. von Weber}\\

\haiku{Gelukkig blijft het.}{Paviljoen door zijn bouwtrant}{de aandacht trekken}\\

\haiku{{\textquoteright} Mevrouw Cramp slaat het,.}{boek dicht waarin zij las en}{schudt langzaam het hoofd}\\

\haiku{Cramp kijkt opmerkzaam,.}{naar de boerderijen die}{hier en daar liggen}\\

\haiku{Soms gaan de paarden.}{van een rijtuig in volle}{draf over de straatweg}\\

\haiku{Overblijfselen van,.}{een oude muur van een poort}{en een wachttoren}\\

\haiku{Het eten is hier slecht,{\textquoteright}, {\textquoteleft}.}{merkt Cramp de tweede dag op}{en de mensen stijf}\\

\haiku{Nu, die Roderick,{\textquoteright}, {\textquoteleft},?}{gaat hij verderwie is dat}{ook weer Roderick}\\

\haiku{{\textquoteleft}Dames en heren,{\textquoteright}, {\textquoteleft}.}{roept hijdie oude man moet}{geholpen worden}\\

\haiku{Maar de Turkse vrij;}{scharen trappen de vonken}{in Bulgarije uit}\\

\haiku{Anderen vallen.}{op de knie en laten het}{hoofd langzaam zakken}\\

\haiku{Maar het gebouw staat.}{er al. Aan voortvarendheid}{mankeert het hem niet}\\

\haiku{Dat zijn karakter,.}{niet iedereen aanstaat is}{begrijpelijk}\\

\haiku{Ik bedoel dat ik.}{het niet wenselijk acht zulks}{bekend te maken}\\

\haiku{Er moet veel meer kort.}{papier in omloop zijn dan}{de balans vermeldt}\\

\haiku{Het gehoor luistert,.}{aandachtig alsof het ging}{om een reis naar Mars}\\

\haiku{Staat daar nog niet de,?}{oude Cramp doorzichtig als}{een geestverschijning}\\

\haiku{De schepping van Dr!}{d'Espina in handen van}{iemand als Mr Cramp}\\

\haiku{Gelukkig voor Het.}{Bericht doet zich plotseling}{een nieuw geval voor}\\

\haiku{De directeur wrijft.}{zich in de handen over zijn}{tactisch optreden}\\

\haiku{hoogste notering,.}{tussen 1900 en 1907 is 16}{geweest laagste 9}\\

\haiku{Intussen, dingen.}{als deze komen bij meer}{maatschappijen voor}\\

\haiku{Reeds lang is Mr Cramp.}{gewend met het Paviljoen}{te doen wat hij wil}\\

\haiku{Hij installeert een.}{maitresse op kamers in}{de Hemonylaan}\\

\haiku{Schrijf, ploerten, schreeuw je,:}{hees op vergaderingen}{schuimbek van woede}\\

\haiku{En een opvolger,.}{moet {\`\i}k aanwijzen maar voor}{U doe ik het niet}\\

\haiku{Je bent de langste,.}{tijd directeur geweest wees}{daar maar zeker van}\\

\haiku{De verbijstering,.}{maakt plaats voor misnoegen het}{misnoegen wordt haat}\\

\haiku{Eigenlijk is zijn.}{levenslange vleierij}{niets dan haat geweest}\\

\haiku{Tegen iemand die}{knoeit mag je nooit zeggen dat}{het een knoeier is}\\

\haiku{Hij heeft Romeinse,;}{munten verkocht die in Aken}{werden geslagen}\\

\haiku{{\textquoteright} vraagt Stomaski, {\textquoteleft}.}{en neemt het deksel van een}{kistdit is marmer}\\

\haiku{Na de crisis van.}{1921 is het scheepvaartverkeer}{weer toegenomen}\\

\haiku{gegadigden voor,.}{zijn project te vinden met}{betuiging van spijt}\\

\haiku{Hij houdt hen een poos,.}{aan de praat om hen geld uit}{de zak te kloppen}\\

\haiku{De mensen moeten;}{de inkomstenobligatie uit}{overtuiging nemen}\\

\haiku{Stugheid van de een,.}{die de ander nog meer op}{zijn hoede doet zijn}\\

\haiku{Aan dat soort mensen.}{heb ik alle ellende}{te danken gehad}\\

\haiku{Vooruitstrevende;}{architecten spreken van}{het Nieuwe Bouwen}\\

\haiku{Het Paviljoen is,.}{oud het gaat gebukt onder}{de last der jaren}\\

\haiku{Meermalen bezoekt.}{Mr Cramp de voorstellingen}{in de schouwburgzaal}\\

\haiku{{\textquoteright} Paraat betoont zich, {\textquoteleft}?}{ongeduldigwaar blijft dat}{oude kavalje}\\

\haiku{Zulke dingen zijn?}{toch schering en inslag in}{het zakenleven}\\

\haiku{Het publiek heeft in:}{deze jaren Mr Cramp een}{bijnaam gegeven}\\

\haiku{Hij is mensenschuw,,,.}{zegt men hij heeft kind noch kraai}{hij ontmoet niemand}\\

\haiku{Hoe kunnen deze?}{en deze obligaties nog}{te voorschijn komen}\\

\haiku{Nu heeft het publiek.}{bijna geen belang meer bij}{de gang van zaken}\\

\section{F.H. Rikken}

\subsection{Uit: Codjo, de brandstichter}

\haiku{{\textquoteright} hernam Codjo.}{als begreep hij de herkomst}{van de kippen niet}\\

\haiku{Maar zou je er zelf,,?}{niet heengaan daar jij hem toch}{beter kent dan ik}\\

\haiku{overal vervolgd en, '}{als een stuk wild opgejaagd}{mag je eersts avonds}\\

\haiku{Slaap nu wel, nu geen.}{meester de uren uwer rust meer}{telt en beknibbelt}\\

\haiku{Werktuigelijk ging,:}{Present naar haar toe waarop}{zij tot hem zeide}\\

\haiku{Waarom had hij haar?...}{niet eerst met zijn voornemen}{in kennis gesteld}\\

\haiku{{\textquoteleft}Dan zullen wij hen {\textquotedblleft}{\textquotedblright}!}{ook denSpaanschen bok laten}{voelen en goed ook}\\

\haiku{mijn jongen{\textquoteright} riep Tom, {\textquoteleft},.}{blijde uitjij weet wat een}{ouden man toekomt}\\

\haiku{{\textquoteleft}Ik voel, dat het goed,{\textquoteright}.}{doet aan mijn oude knoken}{zeide Tom vroolijk}\\

\haiku{daar kreeg je driemaal,.}{daags een soopie behalve wat}{men er nog bij nam}\\

\haiku{Het was een zware.}{arbeid op de plantages}{met waterwerken}\\

\haiku{Ik heb hem geheel,.}{in mijn macht en kan met hem}{doen wat ik verkies}\\

\haiku{{\textquoteleft}Wat zou jij kleine,,?}{nietige spin vermogen}{tegen den tijger}\\

\haiku{Je zult mij in de,!}{oogen der menschen in eere}{herstellen brrr}\\

\haiku{{\textquoteright} hernam de tijger.}{door haar tegenstreving al}{meer en meer gesard}\\

\haiku{{\textquoteright}, vroeg Codjo na.}{eenige oogenblikken van}{pijnlijk stilzwijgen}\\

\haiku{Ik wilde weten,,,.}{Afie of gij mij z\'o\'o bemint}{als ik u liefheb}\\

\haiku{Ik houd veel van mijn.}{geloof en ik zal daaraan}{nooit ontrouw worden}\\

\haiku{{\textquoteright} {\textquoteleft}Wie heeft je gezegd,?}{dat je veracht wordt en dit}{omdat je slaaf bent}\\

\haiku{De blanken hebben;}{ten koste van ons zwoegen}{rijkdom verworven}\\

\haiku{de oude wil ons,.}{jongeren gebruiken om}{voor hem te werken}\\

\haiku{Ik wilde daarom,,.}{maar dat wij wisten waar wij}{iets kunnen vinden}\\

\haiku{Zie eens, dat prachtig.}{stuk zoutevleesch hebben}{zij meegenomen}\\

\haiku{{\textquoteleft}Wil jij mij in den ()?}{nacht naar de kapoewerie}{struikgewas brengen}\\

\haiku{{\textquoteleft}Ik kwam hem vragen.}{of ik wariembo van hem}{zou kunnen krijgen}\\

\haiku{Hij herstelde zich:}{echter zoo goed mogelijk}{en vroeg brutaalweg}\\

\haiku{Ik zal jou even zoo.}{goed ls de overigen in}{mijn wraak verdelgen}\\

\haiku{{\textquoteleft}Maar nu hij er zelf,}{op uit gaat vertrouw ik hem}{weer geheel en al.}\\

\haiku{Dan zullen ook wij!}{eens gen{\'\i}eten van de vruchten}{van onzen arbeid}\\

\haiku{Zij veracht mij even,.}{diep in haar hart als ik haar}{haat in het mijne}\\

\haiku{Wil je haar dooden of?}{haar alleen een langdurig}{lijden overzenden}\\

\haiku{Dat is iets, wat ik,.}{van iemand gekregen heb}{om te verkoopen}\\

\haiku{misie kan alles,.}{koopen later zal ik het}{geld wel ontvangen}\\

\haiku{baas Willem behoeft,. '}{niet te weten hoe ik er}{aan gekomen ben}\\

\haiku{{\textquoteright} {\textquoteleft}Ach, hoe zou iemand,?}{het maken die steeds in vrees}{en angst moet leven}\\

\haiku{{\textquoteright} {\textquoteleft}Daarom heb ik het,{\textquoteright}.}{zelf maar gedaan antwoordde}{Codjo spottend}\\

\haiku{{\textquoteleft}Ik ben nu al een.}{paar maanden vrij en het gaat}{mij nog zoo slecht niet}\\

\haiku{Daar zullen wij dus,.}{nogal iets kunnen vinden}{wat ons te pas komt}\\

\haiku{Maar zou je er dan?}{tegen hebben mij all\'e\'en}{den weg te wijzen}\\

\haiku{Het wordt eindelijk,.}{eens tijd om de handen uit}{de mouw te steken}\\

\haiku{En wat geeft het ook,{\textquoteright}, {\textquoteleft},.}{ging hij voortis het hier niet}{dan ergens anders}\\

\haiku{In de keuken heb.}{ik bananen en switi}{moffo gevonden}\\

\haiku{Een oogenblik bleef}{Codjo met een helschen}{lach om de lippen}\\

\haiku{de eene trok naar den,.}{Waterkant de andere}{naar de Maagdenstraat}\\

\haiku{Zou je ze niet voor?}{mij kunnen bewaren of}{wellicht verkoopen}\\

\haiku{Dit is nu nog maar,{\textquoteright}.}{het begin voegde hij er}{overmoedig aan toe}\\

\haiku{Hij bracht als buit een.}{kalkoen en eenige stukken}{katoen mede}\\

\haiku{Welnu, ouroe:}{fajatiki no de plei}{foe teki faja81}\\

\haiku{{\textquoteleft}En zoo juist wilde!}{je mij overhalen in het}{bosch te gaan leven}\\

\haiku{{\textquoteleft}Ik ben het{\textquoteright}, zeide,.}{Present toen hij de stem van}{Betsy herkende}\\

\haiku{{\textquoteright} {\textquoteleft}Ik ga er niet heen,{\textquoteright}, {\textquoteleft}.}{zeide deze beslistals}{Present niet meedoet}\\

\haiku{Codjo blies er.}{met kracht in en weldra sloeg}{de vlam er uit op}\\

\haiku{Ik nam den koffer,,.}{op het hoofd hij was niet zwaar}{daar er niets in was}\\

\haiku{Ik schoof den koffer.}{bij de andere kisten}{en ging de trap af}\\

\haiku{Kom jij intusschen.}{maar voorloopig met mij}{mee naar het Piket}\\

\haiku{{\textquoteleft}Ik heb vanmorgen,{\textquoteright}.}{den korf half open gevonden}{hernam Frederik}\\

\haiku{{\textquoteright}, beval hij, {\textquoteleft}kijk eens,.}{hier en daar rond of de kip}{niet te vinden is}\\

\haiku{Vannacht moet ieder,.}{onzer maar zien dat hij een}{onderkomen vindt}\\

\haiku{Ik dacht bepaald, dat,.}{er onweer zou komen z\'o\'o}{benauwd vond ik het}\\

\haiku{Men wil mij voor het,.}{gerecht dagen nadat men}{mij bestolen heeft}\\

\haiku{Jammerend liep Tia.}{door het huis bij het verhaal}{van de ontdekking}\\

\haiku{{\textquoteleft}Het is hier te koud,.}{in het gras laat ons een goed}{heenkomen zoeken}\\

\haiku{Dit gezegde was,:}{dubbelzinnig daar het even}{goed kon beteekenen}\\

\haiku{Deze zeide niets;}{en gaf zelfs niet door kreten}{zijn smart te kennen}\\

\haiku{{\textquoteleft}Wat heb je dan met,?}{dat stuk bont gedaan dat ik}{je gegeven heb}\\

\haiku{{\textquoteleft}Wat heb ik je dan,?}{gedaan dat jij je ziekte}{aan mij wilt wijten}\\

\haiku{Omdat tante bang,....}{was dat je haar daarmede}{kwaad zoudt aanbrengen}\\

\haiku{De herinnering,;}{aan al het leed dat zij had}{moeten verduren}\\

\haiku{{\textquoteleft}Ik bid u, tante,,.}{vergeef mij indien ik u}{leed veroorzaakt heb}\\

\haiku{{\textquoteleft}Het is te spoedig,.}{geleden om ons veel vrees}{in te boezemen}\\

\haiku{Hij heeft dat alles,.}{geleden opdat gij niet}{verloren zoudt gaan}\\

\haiku{{\textquoteleft}Codjo{\textquoteright}, hernam,.}{zij terwijl haar eene rilling}{door de leden voer}\\

\haiku{hij was de slaaf van.}{Mary Rose Herbert en}{niet van Makenzie}\\

\section{Herman Robbers}

\subsection{Uit: De bruidstijd van Annie de Boogh}

\haiku{de sterren te zien...,}{schijnen door de zoldering}{van zijn slaapkamer}\\

\haiku{Hij passeerde al '.}{gauw het huis en keek het in}{t voorbijgaan aan}\\

\haiku{Je wilt je zeker ',....}{nog wel wat opfrisschen voor}{teten h\`e jongen}\\

\haiku{wat kwam het er op,!...}{aan tien of twaalf dagen van}{onnadenkendheid}\\

\haiku{ze gaven niets om.......}{wat voor hem het belangrijkst}{was en omgekeerd}\\

\haiku{t Waren ook zoo'n,.}{paar besten en zij hielden}{van hem om hem zelf}\\

\haiku{ze zouden er z\'o\'o,,}{passeeren maar je kondt er}{bij avond niets van zien}\\

\haiku{'t zou eigenlijk.}{heel goed voor hem zijn als hij's}{wat tegenspoed kreeg}\\

\haiku{Misschien hebben we,....}{wel tijd er even heen te gaan}{na de receptie}\\

\haiku{De moeder gaf hem;}{bijna geen gelegenheid}{naar haar te kijken}\\

\haiku{{\textquoteleft}En wat vindt-je van,{\textquoteright}, {\textquoteleft}?}{de anderen vroeg Louisvan}{Papa en Mama}\\

\haiku{Maar zoo met fixeeren....}{en naloopen was hij niet}{verder gekomen}\\

\haiku{Hij had ook al gauw}{weten op te merken wat}{voor soort van meisje}\\

\haiku{Hij had nooit gedacht,....}{dat het z\'o\'o groot kon zijn het}{verschil tusschen hen}\\

\haiku{Voor zich-zelf wou ';}{hijt niet weten dat het}{was om haar alleen}\\

\haiku{Maar ze moest het toch....}{ook wel hooren en Annie}{schaamde zich voor haar}\\

\haiku{Ze had nu uit de;}{verte kunnen hooren dat}{hij z'n rok aanhad}\\

\haiku{hij was er veel te,....}{hijgerig-onrustig voor}{liep al maar verder}\\

\haiku{Hij ging, den Maaskant,.}{volgend langs het park tot bij}{den ouden Heuvel}\\

\haiku{Dat was goed en mooi,....}{een schoone opbloei van zijn}{beste neigingen}\\

\haiku{moe van 't zitten....}{maakte elk bewegingen}{van willen opstaan}\\

\haiku{de nieuwe zou niet.}{in huis komen voor  dat}{Annie er uit was}\\

\haiku{Het bruidje was er,.}{aan gewoon aan die stemming}{en aan die standjes}\\

\haiku{Op een morgen met}{haar moeder alleen had ze}{er van gesproken}\\

\haiku{of mama dan een....}{beetje vriendelijker voor}{haar zijn wou voortaan}\\

\haiku{Het zou een al te '.}{wreede teleurstelling zijn}{alst niet zoo was}\\

\haiku{wat was dat nu stom,!...}{hij had zoo gemakkelijk}{uit kunnen blijven}\\

\haiku{Toen de bruigom weg '.}{was bleef ze een heelen tijd}{alleen aant woord}\\

\haiku{Toen kwamen ze in...,,....}{de drukke winkelbuurt het}{Boymansplein de Blaak}\\

\haiku{Ze maakte 't dus,....}{gauw af zei dat ze nog wel}{terug zou komen}\\

\haiku{Ze waren nu klaar,....}{met hun boodschappen konden}{dus wel naar huis gaan}\\

\haiku{Langzaam, een beetje....}{stroef en knarsend ging de deur}{naar achteren open}\\

\haiku{Hij had het alles...,,....}{te voren wel vermoed ja}{bijna geweten}\\

\haiku{Rillend en pijnlijk,,}{kroop hij z'n bed in maar z'n}{moeheid was zoo groot}\\

\haiku{s Middags sloot hij,.}{zich op in zijn kamer ging}{op z'n bed liggen}\\

\haiku{Het zingen was iets,.}{te laat ingevallen maar}{dat herstelde zich}\\

\haiku{als hij het los mocht...,,,!...}{maken dat het kon zwieren}{leven bewegen}\\

\haiku{Hij verkneukelde,....}{zich als een smulpaap genood}{op een fijn diner}\\

\haiku{Annie zat met haar,....}{rug naar zijn tafel gekeerd}{aan de volgende}\\

\haiku{Hij was sinds lang aan,.}{zoo weinig gewoon hij kon}{er niet meer tegen}\\

\haiku{Drinken kon hem nu....}{den roes niet geven waaraan}{hij behoefte had}\\

\haiku{Louis ging trouwen, Paul...,!}{wou nog geen tien dagen bij}{haar blijven och God}\\

\haiku{Toen hij 's middags, '.}{thuis kwam was Paul al opt}{punt van vertrekken}\\

\haiku{Die geschiedenis,!...}{van gisteren-avond dat}{was toch zoo erg niet}\\

\haiku{waarom?... Ze deed ook,....}{den mond wel open maar haar keel}{liet geen klanken door}\\

\haiku{Ze voelde zich of,....}{ze op-eens was veranderd}{herkende zich niet}\\

\haiku{hij wist het toch maar,!...}{hij had den slag beet om zoo'n}{meisje te boeien}\\

\haiku{En ze sloot het raam,....}{weer met een licht gevoel van}{teleurgesteld zijn}\\

\haiku{Maar \'e\'en tegelijk, ',....}{nement is vergif had}{de dokter gezegd}\\

\haiku{Haar hart bonkte zoo,....}{hevig op dat ze bang was}{dat het geluid gaf}\\

\haiku{Om kwart over vijven ',.}{zette zen hoed op deed}{haar manteltje om}\\

\haiku{Langzaam, voorzichtig,...,....}{draaide ze den sleutel om}{toch knarste die even}\\

\haiku{Maar een schilder is,,....}{een man en hij een jonge}{sterke jonge man}\\

\haiku{Een ander had haar,,....}{nu bezat haar nu had macht}{en rechten over haar}\\

\haiku{Zij snikte, snikte..., '....}{maart was alsof dit leed}{door tranen groeide}\\

\haiku{Mocht hij haar dan nu,:}{laten merken dat hij haar}{begeerde vragen}\\

\haiku{toch liet hij haar niet,,:}{los maar bracht z'n mond bij haar}{oor nu fluisterde}\\

\haiku{toch  wist hij nooit....}{het goddelijke zoo dicht}{te zijn genaderd}\\

\subsection{Uit: Een mannenleven. Deel 2. Op hooge golven}

\haiku{{\textquoteright} Terwijl liepen Huib.}{en Bl\'ecour met Gerbrandts mee}{naar zijn kleedkamer}\\

\haiku{Maar nog v\'o\'or ze daar,!}{waren traden hun haastig}{twee vrouwen opzij}\\

\haiku{En dan Janne en,,....}{Ruth nog nietwaar en Driesse}{en Melchior Spin}\\

\haiku{Schoonheid, kunst, beide;}{zoo volmaakt subjectieve}{begrippen trouwens}\\

\haiku{Men zat er veilig,,,;}{warm en onder elkaar in}{dit nachtelijk uur}\\

\haiku{Ik kom morgen bij,,.}{je Gerbrandts en dan zullen}{wij erover praten}\\

\haiku{Toch, zou ik zeggen,.}{moest je die menschen nu hun}{gang maar laten gaan}\\

\haiku{'is,{\textquoteright} vervolgde zij,;}{en haar gezichtje werd}{plotseling ernstig}\\

\haiku{Hoe denkt hij er nu,?}{over onze beste vriend en}{vereerde auteur}\\

\haiku{Gerbrandts lachte met,:}{verruktgroote oogen stak Huib zijn}{beide handen toe}\\

\haiku{Geen repetities,....}{vandaag dank zij n\^otre cher}{ma{\^\i}tre Huib Hoogland}\\

\haiku{je moet het wel vreemd....}{vinden dat een meisje je}{zulke dingen zegt}\\

\haiku{{\textquoteleft}Ik zal trouwens niet,....}{zeggen dat je heelemaal}{ongelijk hebt maar}\\

\haiku{{\textquoteleft}Ja,{\textquoteright} zei Janne, {\textquoteleft}ik...., '.}{heb ook een moeder gehad}{neek heb haar n\'og}\\

\haiku{Wat een leed heeft het,.}{me gedaan dat ik niet kon}{komen gisteravond}\\

\haiku{Gisterenavond is,...., '....}{hij er weer geweest en nou}{k heb dan beloofd}\\

\haiku{al zou ik niet met,....}{hem kunnen trouwen al was}{hij getrouwd desnoods}\\

\haiku{{\textquoteleft}Bedenk toch, ik ben,....}{\'e\'en-en-dertig en}{ik ben een meisje}\\

\haiku{Al scherper zag hij,....}{de sc\`enes h\'o\'orde hij den}{toon der dialogen}\\

\haiku{Het k\'on toch niet, dat,....}{het hem enkel om d\'at zou}{te doen zijn om d\'at}\\

\haiku{Hoe vreemd en ernstig, ',!}{die brandende oogen int}{strakke bleeke gezicht}\\

\haiku{En dit zonder dat.}{zij er zich een van beiden}{over verwonderden}\\

\haiku{Hij blikte zijn vrouw,.}{in het oogenzwart dat hem}{helder toeglansde}\\

\haiku{{\textquoteleft}Nou zeg, wat vinden,,?}{jelie zal ik doorlezen}{of de rest morgen}\\

\haiku{de bitterheid van,.}{haar nederlaag het scheen de}{zijne geworden}\\

\haiku{Ik dacht, om je de,.}{waarheid te zeggen dat het}{juist het zwakste was}\\

\haiku{Ze stond stil, hield nog,.}{altijd zijn hand vast zoodat ook}{hij moest blijven staan}\\

\haiku{Plotseling schreef hij,,;}{een nieuw begin en zie dit}{werd dadelijk goed}\\

\haiku{{\textquoteleft}O jij - jij - jij,{\textquoteright} kwam.}{er eindelijk schor van zijn}{trillende lippen}\\

\haiku{Maar een bittere....}{trekking van zijn onderlip}{bleef nog wijlen}\\

\haiku{Er is dan ook geen,.}{grappenmakerij bij zooals}{de vorige maal}\\

\haiku{Die amsterdamsche,.}{voorstellingen Huib sloeg er}{geen enkele over}\\

\haiku{Hij forceerde het,.}{wel maar dan wreekte dat zich}{door verergering}\\

\haiku{Huib was er zacht en, '.}{wijd ontroerd van toen zes}{avonds naar huis spoorden}\\

\haiku{En gejacht liep hij -;}{door naar de trem wat was het}{weer laat geworden}\\

\haiku{Til trouwens, bang voor,....}{scherpe woorden deed haar best}{om af te leiden}\\

\haiku{jij bent nou wel erg,;}{beroemd maar wat heb je daar}{nou eigenlijk aan}\\

\haiku{Een gesprek met de.}{Doescates had dien twijfel}{voedsel gegeven}\\

\haiku{Heb je geen partij,....}{gekozen voor dat vrouwtje}{en je kwaad gemaakt}\\

\haiku{Werk jij maar, zoek jij.... '....}{maark Heb geweldig veel}{feducie in jou}\\

\haiku{Huib zou hij heeten, ',.}{alst een jongen was had}{Co al geschreven}\\

\haiku{En hij schreef het ook,.}{dadelijk aan Co dat hij}{er zoo blij mee was}\\

\haiku{Maar nog zoo zelden.}{had hij iets van de moeder}{in Janne ontdekt}\\

\haiku{Maar ze dorst daar haast,.}{nooit naar te vragen daarop}{te zinspelen zelfs}\\

\haiku{Wie zou gelukkig?}{kunnen zijn met een vrouw die}{hem geheel begreep}\\

\haiku{En ook zijn werk voor.}{de zaken kreeg de aandacht}{niet die het noodig had}\\

\haiku{Meneer van der Kamp,....}{u als onze technische}{specialiteit}\\

\haiku{Daar had ik al zoo,{\textquoteright}.}{eenig idee van knikte Huib met}{zijn zelfden glimlach}\\

\haiku{De heeren drukten,,,....}{hem \'e\'en voor \'e\'en de hand en}{waren vertrokken}\\

\haiku{{\textquoteright} {\textquoteleft}'k Wou 'k het \'o\'ok,{\textquoteright},.}{kon zeggen bromde Huib met}{even een blik naar Til}\\

\haiku{Wat kan jou nou in.}{godsnaam de sympathie van}{het publiek schelen}\\

\haiku{Z\'o\'o alleen kan je........}{hun wat schoonheid geven wat}{troost en verheffing}\\

\haiku{Ik ben narrig en,,....}{humeurig tegenwoordig}{prikkelbaar lastig}\\

\haiku{En wendde schielijk.}{het hoofd om de betraande}{oogen te verbergen}\\

\haiku{Bedrog is volmaakt,.....}{onnoodig juist omdat we zoo}{vrij zijn allebei}\\

\haiku{{\textquoteleft}Onmogelijk,{\textquoteright} zei,, {\textquoteleft},.}{Spin rustignee-nee dat}{is onmogelijk}\\

\haiku{Ze kwam haast nooit meer.}{uit zichz\'elf en Huibs vrouw was}{daar verwonderd over}\\

\haiku{Huib mocht haar brengen,,,,?}{o ja heel graag maar dan tot}{aan de trem niet waar}\\

\haiku{Want bepaald met de, '.}{trem  wou ze gaan als hij}{t niet kwalijk nam}\\

\haiku{{\textquoteright} En ze glimlachte,.}{met een doffen blik kuste}{Janne vaarwel}\\

\haiku{trek je daar dan niet ',,....}{te veel vanan hoor en blijf}{geduldig met hem}\\

\haiku{{\textquoteright} De oude vingers {\textquoteleft},,,,.}{drukten Tils hand.Niks kindje}{niks hoor schrik maar niet}\\

\haiku{{\textquoteright} Vier weken later - -.}{Til en Huib aan haar bed stierf}{mevrouw Molano}\\

\haiku{Dol!{\textquoteright} En of hij straks,;}{even mee naar beneden ging}{een oogenblik maar}\\

\haiku{Even keek hij op zijn,,.}{horloge dacht aan thuis aan}{die hem wachtten daar}\\

\haiku{Hoe had hij het \'o\'oit,,....}{kunnen denken vroeger dat}{zulke gedachten}\\

\haiku{{\textquoteleft}Als 'k mevrouw dan....{\textquoteright} {\textquoteleft},,!}{maar niet ophoud tenminste}{O welnee welnee}\\

\haiku{Hoogland deed een stap,;}{achterwaarts maar zij bleef zich}{aan hem vastklemmen}\\

\haiku{in een seconde;}{van aarzeling had hij haar}{schouders gegrepen}\\

\haiku{{\textquoteright} Intusschen waren,.}{ze doorgeloopen zwegen}{beiden  een poos}\\

\haiku{Alsof er gevaar:}{voor iets was en hij ijlings}{terugverlangde}\\

\haiku{Een oogenblik scheen,.}{het inderdaad of Huib zich}{op hem werpen ging}\\

\haiku{Het haar schrijven zou,,.}{gemakkelijker zijn maar}{onmannelijk laf}\\

\haiku{Hij is veel ouder,,,....}{dan ik dan jij zelfs en zoo}{ziek zoo ellendig}\\

\haiku{Volkomen waar is,;}{het zeker niet zei een stem}{in zijn binnenste}\\

\haiku{Maar zich ingedacht ',.}{had hijt nog in geenen deele}{het spreken tot Til}\\

\haiku{Daar was ze dan nu,,.}{zijn herwonnen vrijheid zijn}{vrede zijn zielsrust}\\

\haiku{Het zal zoo heerlijk,,....}{zijn daar te wonen samen}{jij altijd bij me}\\

\haiku{Een vroegrijp meisje,....}{wat al  te ontijdig}{bewust en vroegrijp}\\

\haiku{Zou hij dan alles,....?}{\'e\'ens moeten verliezen}{z\'o\'o eenzaam blijven}\\

\haiku{En niets meer dat Til,,.}{en Huib scheidde vervreemdde}{dat tusschen hen stond}\\

\haiku{IJlte alleen, van,,....}{verzwegen gedachten nu}{ja dat bleef altijd}\\

\haiku{{\textquoteleft}Vergeef me, kerel,,....}{je hebt gelijk het was een}{misselijke grap}\\

\haiku{{\textquoteleft}Ik begrijp je, Mels,.}{ik begrijp je zorg en je}{verontwaardiging}\\

\haiku{als je dit maar weer,....}{eens gehad hebt dan kan je}{er wel weer tegen}\\

\subsection{Uit: Roman van een gezin. Deel 1. De gelukkige familie}

\haiku{Wel, u begrijpt toch, '!....}{dat zet volgend jaar om}{het restje kwamen}\\

\haiku{Ieder jaar - een tijd -!....}{lang zelfs iedere maand was}{de winst gestegen}\\

\haiku{Agenten beproefden.}{vergeefs de beweging er}{weer in te brengen}\\

\haiku{{\textquoteright} Voor de kinderen,;}{was het een bizondere}{dag belangwekkend}\\

\haiku{Intusschen zijn er....}{bij ons toch een stuk of wat}{binnengekomen}\\

\haiku{{\textquoteleft}Excuus vragen, denk,!}{ik d'r hangende pootjes}{laten bekijken}\\

\haiku{Je behoeft zoo bang,,,!}{niet te kijken Emmie zoo'n}{ramp is het niet hoor}\\

\haiku{{\textquoteleft}Och! 't Waren een,....}{paar van die lui die vandaag}{nog gewerkt hebben}\\

\haiku{De menschen konden......}{ook dikwijls zoo aanhouden}{dringen en liefdoen}\\

\haiku{Als mama haar dan, '?}{toch niet gelooven wou waarom}{vroeg zet haar dan}\\

\haiku{Jan zei met recht, wees, ',?....}{maar blij als jer niets mee}{te maken hebt h\`e}\\

\haiku{Jans vage {\textquoteleft}geloof{\textquoteright},.}{was het hare ook Jan sprak}{haar meeningen uit}\\

\haiku{Slecht zagen ze er,,.}{uit vreeselijk bleek en voozig}{de zetters vooral}\\

\haiku{Ze geloofde ook.}{eigenlijk niet dat Theo er}{zelf naar verlangde}\\

\haiku{Ook het andere,....}{raam werd hoog opgeschoven}{gordijnen op zij}\\

\haiku{- hij proefde vooruit.}{al het geestesgenot van}{erover te praten}\\

\haiku{heb je 't gehoord,{\textquoteright},.}{riep hem Adam toe hoogheesch}{door zijn opwinding}\\

\haiku{De Bries alleen hield,.}{zich meest wat op zij schoon hij}{lachende toekeek}\\

\haiku{En een nieuwe brief {\textquoteleft}{\textquoteright}.}{van hetlooncomit\'e dat}{niet eens werd erkend}\\

\haiku{Ze vroeg of Noortje.}{haar eigen kamer nu ook}{eens wou laten zien}\\

\haiku{Er zijn er wel, die,.}{iets toegeven willen ten}{minste in werkuren}\\

\haiku{Ru zat rechts naast haar -,;}{hij praatte zoo kregel zoo}{heftig in-eens}\\

\haiku{De vorige rees....}{als een naar visioen voor}{zijn tobbenden geest}\\

\haiku{Enfin, zoo'n jongen,.... '!...}{dat gaat wel overn Beetje}{sentimenteelig}\\

\haiku{Nieuw, jong, frisch, telkens.}{rumoeriger ontwaken}{de Maandagmorgens}\\

\haiku{Gejuich, zwaar dreunend,,.}{hoed-en-pet-gezwaai}{van alle kanten}\\

\haiku{Die vroegen of de,,....}{opslag een cent per uur ook}{daar werd gegeven}\\

\haiku{Wel vijftig, zestig, {\textquoteleft}{\textquoteright} - '.}{bleven op de keien zoo}{stond int volksblad}\\

\haiku{Want zij hadden het, '....}{nu immers ondervonden}{t hielp altijd iets}\\

\haiku{ging een oogenblik,.}{later de deur uit nog even}{een boodschap doen}\\

\haiku{soeziger zat hij.}{een beetje verzakt aan den}{fonkelenden disch}\\

\haiku{Louise Heugens en - '!}{Gonne van de Palsn paar}{contrasten die twee}\\

\haiku{'k zou ook nog best ',?,... '!}{is de lucht in willen h\`e}{t Is zoo heerlijk}\\

\haiku{Maar Jeanne ging,:}{met de anderen mee tot}{bij Baatz voor de deur}\\

\haiku{Nou bonjour,{\textquoteright} zei ze, {\textquoteleft},,,....,....}{haastigtoe Th\'e zeg vooruit}{nou Dag Gonne dag}\\

\haiku{En ze vond ook maar, -.}{goed dat het uit was nu dat}{hij maar niet meer kwam}\\

\haiku{als Jeaan nu toch...!}{eenmaal niet genoeg van hem}{hield wel natuurlijk}\\

\haiku{'t scheen wel als had.}{ze geen flauw vermoeden van}{dieper bestaan}\\

\haiku{Ook was er volstrekt,,,;}{geen reden voor Emma om}{bang te zijn vond hij}\\

\haiku{Jaloersch was Theo, en,,,.}{Ada veeleischend coquet liet}{zich duchtig gelden}\\

\haiku{Waren papa en,,....}{Theo weg na het koffiemaal}{viel er rust in huis}\\

\haiku{Die was trouwens rijk,, ' '!....}{van z'n vrouw wat kontem}{ten slotte schelen}\\

\haiku{Of - en ze bloosde - '!....}{bij die gedachte \`oft}{werd nog veel mooier}\\

\haiku{Maar Frans ging er heen,,;}{een visite maken met}{Theo en Jeanne}\\

\haiku{Wat we buitendien,.... '....}{moeten uitgeven zie je}{t Dagelijksche}\\

\haiku{waar moeten we d\'an,, ' '!}{op bezuinigen Jank}{weetet heusch niet}\\

\haiku{Hij was dan niet in!}{de wereld geschopt om voor}{vreemden te zorgen}\\

\haiku{Om in korten tijd,....}{veel bij elkaar te krijgen}{was d\'at de manier}\\

\haiku{Ongetrouwd van je, '!}{leven genietent is}{wel zoo verkieslijk}\\

\haiku{Een deftig meisje,,....}{van oude familie veel}{goede connecties}\\

\haiku{de werkelijke,,....}{trouwdag dag van receptie}{cadeaux en bloemen}\\

\haiku{Zijn houding  moest,...}{vroolijk en vriendelijk zijn}{blij en hoffelijk}\\

\haiku{De sneeuw lag, als smet,....}{van de straat op de bloemen}{die later kwamen}\\

\haiku{telkens keek ze haar,.}{bruigom aan  met plezier}{in zijn vroolijkheid}\\

\haiku{niets geen plezier meer,....}{kon hebben daarom in de}{zilveren bruiloft}\\

\haiku{den meesten scheen het;}{toch wat kras zoo dadelijk}{na het vele eten}\\

\haiku{En werkelijk, de,.}{kop werd prachtig leek sprekend}{op den ouden Croes}\\

\haiku{In h\'o\'ofdzaak was Croes,,....}{toch prachtig geslaagd en hij}{hoopte Van Oosthoff}\\

\haiku{Ze kon niet verder,,.}{komen gaf hem een hand en}{hij lachte nerveus}\\

\haiku{Ze bleven beleefd,:}{en welwillend maar ietwat}{teruggetrokken}\\

\haiku{ofschoon dat toch het,,}{natuurlijke doel van zoo'n}{studie is nie-waar}\\

\haiku{Kijk 's, als 't je....}{nou voornamelijk te doen}{is om wat meer werk}\\

\haiku{Dat wordt er dan w\'e\'er....}{een die voortdurend alleen}{op d'r kamer zit}\\

\haiku{Zijn vrienden hadden.}{al klaar gestaan met het open}{rijtuig en den krans}\\

\haiku{Ze had geen nieuwen '.}{wintermantel gekochtt}{vorige najaar}\\

\haiku{Jeanne was er,,;}{het laatste jaar ook al niet}{op vooruitgegaan}\\

\haiku{de hartelijkheid.}{van den zilverenbruilofsdag was}{gauw voorbij geweest}\\

\haiku{En letteren, hoe,?}{k\'om je aan letteren wat}{wou je dan worden}\\

\haiku{Doch dit wisten zijn,.}{vrienden alleen zijn vader}{merkte er niets van}\\

\haiku{Het wordt nou toch wel, ', '....{\textquoteright} {\textquoteleft}}{erg warm hier ik g\'a maar weer}{is geloofkNee}\\

\haiku{Als-t-ie 't erg,....}{druk heeft gehad op kantoor}{of slechte zaken}\\

\haiku{dan mag je 'n mooie ';}{nieuwe japon koopen of}{n hoed of zoo iets}\\

\haiku{Ik vind alleen, een, '...{\textquoteright} {\textquoteleft}....}{grachtehuist heeft toch nog}{veel meer cachetJa}\\

\haiku{{\textquoteright} Emma begon te,.}{huilen ze was in-eens}{heelemaal overstuur}\\

\haiku{Jan, het gaat niet met....}{enkel  twee meiden in}{zoo'n groot huis als dit}\\

\haiku{En zij zijn er nou,....}{ook eenmaal aan gewend dat}{juf van alles doet}\\

\haiku{werd hij een walging,,,}{in zich gewaar een afkeer}{niet te overwinnen}\\

\haiku{hen beiden kwam tot,:}{een praatje uitgebreider}{dan het gewone}\\

\haiku{De familie ging;}{niet naar buiten en het was}{een natte zomer}\\

\haiku{En klagen, klagen,,...}{steen en been tegen ieder}{die ze ontmoette}\\

\haiku{wat of het wel was,,...}{voor een mensch die vrouw en hoe}{of het er toeging}\\

\haiku{Zoo zou hij dan toch,!....}{iets gedaan eindelijk iets}{voortgebracht hebben}\\

\haiku{O, wat een slap, flauw,!...}{kinderachtig ventje was}{hij vroeger geweest}\\

\haiku{Ze is d'r akelig,,,....}{vandaan gekomen doodmoe}{hijgende bezweet}\\

\haiku{Theo veegde zich, met, ' '.}{z'n zijden zakdoekjet}{zweet vant voorhoofd}\\

\haiku{Ze lette haast niet,!}{op haar huisgenooten}{ze gaf niet om hen}\\

\haiku{Overal en aldoor....}{dreigde iets waaraan ze geen}{naam dorsten geven}\\

\haiku{Het sjofele mensch,;}{keek geschrokken op gaf niet}{dadelijk antwoord}\\

\haiku{Zoo waarachtig as,....}{God leef ik durf er haas niet}{over te beginnen}\\

\haiku{haar gezicht bleef naar;}{Emma gekeerd onder het}{verdere praten}\\

\haiku{Jesis, zoo'n kreng van 'n....,....}{jongen dan toch ook neemt u}{nie kwalijk mevrouw}\\

\haiku{naar haar zakdoekje,.}{wischte zich het zweetige}{gezicht ermee af}\\

\haiku{Ja, zooals ik ook al,....}{zei het is niet plezierig}{voor u en papa}\\

\haiku{Hij gaf haar een stoel.}{en stuurde een jongen om}{meneer te halen}\\

\haiku{Daar willen ze 'm,,....}{zeker niet hebben zoo'n rooie}{zoo'n socialist}\\

\haiku{{\textquoteright} {\textquoteleft}En als u 't niet,{\textquoteright}, {\textquoteleft},....}{voelt ging Theo voortnou dan zult}{u later misschien}\\

\haiku{z'n eene arm, die er,;}{ver overheen lag hief hij een}{paar malen even op}\\

\haiku{In 't ouderlijk,,.}{huis kwam Theo v\'o\'or z'n trouwen}{een paar malen nog}\\

\haiku{hij had gezegd dat;}{hij wel begreep door wie oom}{Herman was gestuurd}\\

\haiku{eigenlijk had hij;}{van jongsaf nooit bizonder veel}{met hem opgehad}\\

\haiku{- dan winnen en z'n,....}{inzet verdubbeld zien \`of}{dien zien weggraaien}\\

\haiku{Het waren papa.}{en mama die de grootste}{verliezen leden}\\

\haiku{Vijf minuten voor.}{twaalven kon Croes zich haast niet}{langer beheerschen}\\

\subsection{Uit: Roman van een gezin. Deel 2. E\'en voor \'e\'en}

\haiku{Ze zeggen immers....}{altijd dat juist mannen die}{sterk geleefd hebben}\\

\haiku{Was het trouwens niet?}{een klaargemaakte zaak waar}{hij voorgezet was}\\

\haiku{nu maar h\'e\'el erg graag ',.}{of anders veel liever in}{t geheel niet hoor}\\

\haiku{Als het kwam zou het}{de partijen ditmaal niet}{zoo onvoorbereid}\\

\haiku{Dat wil zeggen,{\textquoteright} liet, {\textquoteleft}....}{hij er scherp-fluisterend}{op volgenhij zei}\\

\haiku{Neem een rijtuig, en....,....{\textquoteright} {\textquoteleft}....}{haal laats kijken wie zijn er}{ook alzoo bijNou}\\

\haiku{Een ieder mot toch,!}{vrij wezen in z'n doen en}{laten wat weerlach}\\

\haiku{Maar daar was geld voor,,;}{noodig natuurlijk je moest wat}{kunnen inkoopen}\\

\haiku{Jammer, ellendig, '....}{jammer dat zer zooveel}{minder uitzag al}\\

\haiku{'t sprak van zelf dat;}{ze allemaal fel tegen}{het volk zouden zijn}\\

\haiku{Hij zou er niet meer,....}{in terug willen in die}{wereld van vroeger}\\

\haiku{Die toppen vooral,,,! ...}{die zijn zoo ruw en hard en}{heelemaal grauw kijk}\\

\haiku{dat hij dit over een, '....}{korten tijd n\'og eens zien zou}{en dan voort laatst}\\

\haiku{{\textquoteleft}Kom,{\textquoteright} zei hij, {\textquoteleft}ik moet,,? ....}{weer eens opstappen ga je}{misschien mee zoover Ru}\\

\haiku{Ze hadden elkaar.}{den heelen avond nog zoo goed}{niet begrepen}\\

\haiku{{\textquoteleft}Welnee, welnee! - och,!}{kom het loopt ommers met een}{sisser af misschien}\\

\haiku{Jelie blijft rustig,.}{thuis we zouden mekaar maar}{in de weg loopen}\\

\haiku{- ze scheelden dan ook...}{maar twee jaar en daar Jan de}{scherpzinnigste was}\\

\haiku{het was beroerd, ze,,?}{gaven het toe volmondig}{maar wat wou hij dan}\\

\haiku{Met bitterheid en.}{opwinding bleef Croes praten}{en zich verzetten}\\

\haiku{Hij had geen hoop meer.}{op een verbetering in}{z'n financi\"en}\\

\haiku{hij erkende dat '.}{hijt er naar gemaakt had}{in den laatsten tijd}\\

\haiku{Hij zou 't morgen,.... '}{maar in-eens zeggen dat}{hij er geweest was}\\

\haiku{Toch nog aarzelig,,,....}{langzaam wrikte Croes aan den}{deurknop duwde open}\\

\haiku{Hij sprak er over met,;}{Jeanne want die had de}{huishoudkas nu toch}\\

\haiku{Noortje kostte nu ',.}{weln boel geld maar ze ging}{dan ook het huis uit}\\

\haiku{Hij zal maar ergens,....}{in de leer moeten bij een}{timmerman of zoo}\\

\haiku{Ik heb heelemaal,...}{zoo'n verbazend duur jaar ik}{weet waarachtig niet}\\

\haiku{{\textquoteright} Maar even later, zich,:}{vermannend en een stap naar}{den knaap toetredend}\\

\haiku{{\textquoteleft}Henk, kereltje, win,,,....{\textquoteright} {\textquoteleft}?}{je nou niet zoo op toe hou}{je nou kalm ikKalm}\\

\haiku{En dan, in-eens,,.}{keerde hij zich om en liep}{weg de kamer uit}\\

\haiku{hij gaf dan bijna, '.}{nooit antwoord maar zuchtte als}{vann diep verdriet}\\

\haiku{Het A-examen had;}{ze een paar jaar geleden}{met sukses gedaan}\\

\haiku{Ook Jeanne, bij,.}{een bezoek aan haar drong er}{hartelijk op aan}\\

\haiku{'k Geloof dat jij ',{\textquoteright}.}{t nog veel meer noodig heb dan}{ik riep Gonne uit}\\

\haiku{En dat jij, die zoo'n '!....}{model vann echtgenoote}{en moeder zou zijn}\\

\haiku{Soms, \'even, gluurde zijn....}{blik van de bladzij weg en}{naar Gonne's gezicht}\\

\haiku{Wanneer de meisjes '.}{elkaar aanzagen proestten}{zet bijna uit}\\

\haiku{Anders kan die zoo.... '}{gezellig zitten praten}{en gekheid maken}\\

\haiku{{\textquoteleft}Laat ik u toch niet,,.}{storen meneer Driebeek blijft}{u rustig liggen}\\

\haiku{{\textquoteright} riep hij uit, met 'n, {\textquoteleft},,!}{armzwaainee u begrijpt me}{niet ik merk het wel}\\

\haiku{ze spraken elkaar.}{nu immers ook alle drie}{bij den voornaam aan}\\

\haiku{Je kunt met hem niet,?}{anders dan vertrouwelijk}{praten vin-je wel}\\

\haiku{Voor Gonne iets te,.}{zijn haar beter te helpen}{maken was het doel}\\

\haiku{{\textquoteright} {\textquoteleft}Heb-je dan geen....}{innerlijke behoefte}{daaraan te gelooven}\\

\haiku{afschuwelijk, vond,....}{mama zoo'n ontstemming nu}{vlak v\'o\'or de bruiloft}\\

\haiku{Wat kan 't me ook,, '!}{allemaal schelen dacht ze}{k word toch de bruid}\\

\haiku{{\textquoteright} trachtte Noortje te, {\textquoteleft}',.}{lachenna slechte inkt of}{zoo iets geloof ik}\\

\haiku{je kreeg er meelij,.}{mee zooals ze zich inspannen}{moest om mee te doen}\\

\haiku{De bruid alleen kreeg....}{een kleur en voelde tranen}{in haar oogen prikken}\\

\haiku{{\textquoteright} En ook de bode,,.}{haastig toegeschoten boog}{zich over de tafel}\\

\haiku{Nu kon ze verder,,....}{niets meer dan wachten wachten}{tot het uit zou zijn}\\

\haiku{Iedere man van....}{dertig jaar heeft wel iets in}{z'n verleden dat}\\

\haiku{'t werd tijd nu, tijd;}{om naar boven te gaan en}{zich te verkleeden}\\

\haiku{{\textquoteright} En beneden, in,,.}{de zaal tegen Croes hem er}{niet bij aankijkend}\\

\haiku{Het zal een heele....}{deun voor haar zijn om er weer}{van op te komen}\\

\haiku{{\textquoteright} zei Jan, en kuste ', {\textquoteleft}....{\textquoteright} {\textquoteleft}....}{haar zacht opt voorhoofdkom}{wees nu maar kalmJa}\\

\haiku{de tafel stond er -.}{gedekt toen ze een voor een}{beneden kwamen}\\

\haiku{Ze beseften nu, '....}{allen dat hett ergste}{was voor mama zelf}\\

\haiku{ze w\'ou er ook nooit,....}{over praten ze hield niet van}{zulke gesprekken}\\

\haiku{maar dat kon niet, want,.}{het was heel stil buiten had}{ze daar straks gemerkt}\\

\haiku{De jongen huilde,.}{blijkbaar in zijn bed kreunend}{met diepe snikken}\\

\haiku{Och ja,{\textquoteright} riep hij uit, {\textquoteleft}....}{ik weet ook zelf niet waarom}{ik zoo huilen moet}\\

\haiku{'t Was natuurlijk,,!}{waar wat iedereen hem zei}{hij hield heel wat over}\\

\haiku{Hij was niet onnoozel,....}{of krankzinnig hij wist wat}{er van komen moest}\\

\haiku{Trouwens, hij wilde,:}{het zelf eerst haast niet gelooven}{maar het was toch zoo}\\

\haiku{de woorden klankten,.}{dan in haar hoofd maar kwamen}{niet over haar lippen}\\

\haiku{Jeanne vond het, ';}{ellendigt zichzelf te}{moeten bekennen}\\

\haiku{Moet je nog al mee,....}{aankomen tegenwoordig}{zei hij in zichzelf}\\

\haiku{Anna, onthuis en,;}{verlegen bewoog zich nog}{plomper dan anders}\\

\haiku{{\textquoteright} herhaalde hij, en.}{krampte zijn fijne handen}{tot vuisten samen}\\

\haiku{Stil was ze altijd,,;}{ook aan tafel nu men was}{dat van haar gewoon}\\

\haiku{Wat m\'e\'er welvaart, wat, ...}{voorspoediger zaken ja}{dat was het eenige}\\

\haiku{{\textquoteright} De jongen praatte,.}{nog door maar Theo luisterde}{er al niet meer naar}\\

\haiku{Vlug schepte ze haar '.}{oudste de koolraap opt}{wit-steenen bordje}\\

\haiku{Het ventje keek nu {\textquoteleft}{\textquoteright}.}{en dan wat schuw en schichtig}{naar z'nmeneer op}\\

\haiku{Tegen iemand moest!}{hij zoo nu en dan toch eens}{wat kunnen luchten}\\

\haiku{Nou nacht Anna,{\textquoteright} riep, {\textquoteleft}, ',!}{ze gedempttot ziens ik kom}{wel weeris gauw hoor}\\

\haiku{O nee, het is maar....}{flikkering in de ruiten}{van die kleerenwinkel}\\

\haiku{{\textquoteleft}Denk je soms dat ik,! ...}{niet weet wat echte fijne}{galanterie is}\\

\haiku{Jij ook, Noortje,{\textquoteright} liet, {\textquoteleft},!}{hij er zachter op volgen}{z\'o\'o te kijven foei}\\

\haiku{Maar zij week schichtig,,.}{opzij haar stoel meetrekkend}{greep haar vaders arm}\\

\haiku{{\textquoteright} antwoordde hij dan,,.}{triestig-gedempt en hij}{zuchtte geluidloos}\\

\haiku{Jou ongelukkig!}{te weten en er niets aan}{te  kunnen doen}\\

\haiku{{\textquoteright} De beweging in, {\textquoteleft}!}{de zaal nam toe telkens was}{er rumoer ensst}\\

\haiku{Zij kon het nu niet,.}{meer enkel uitspattende}{woorden begreep ze}\\

\haiku{Hij had verwacht dat.}{De Bries pittiger kost zou}{gegeven hebben}\\

\haiku{Goed, goed, hij zou dat.}{laten rusten en bij de}{principes blijven}\\

\haiku{Hij begon met een.}{kalmte die na Moks gekrijsch}{bijna matheid scheen}\\

\haiku{Hoe moest het toch gaan,.}{daar in het winkeltje van}{feestartikelen}\\

\haiku{Maar kom, k\'om dan toch,! ....}{niet aan het allerergste}{denken in-eens}\\

\haiku{Elken dag kwam hij, '.}{thanss morgens tusschen elf}{en twaalf gewoonlijk}\\

\haiku{Hij draaide zich af,....}{deed alsof hij iets zocht op}{het nachttafeltje}\\

\haiku{Niemand dorst vragen, '.}{maar ze bleven Herman in}{t gelaat turen}\\

\haiku{alleen de wielen,....}{knarsten het rijtuig piepte}{of kraakte soms}\\

\section{Maurice Roelants}

\subsection{Uit: Komen en gaan}

\haiku{En zij mijn vrouw, waar?}{is er haar hart op dezen}{oogenblik aan toe}\\

\haiku{Een aftroeving vond.}{hij doelmatiger dan een}{proces-verbaal}\\

\haiku{Hij beminde mij,:}{en sloeg mij op den schouder}{terwijl hij zeide}\\

\haiku{{\textquoteright} Zij sprak eenvoudig,,.}{zonder coquetterie die}{op tegenspraak aast}\\

\haiku{Ik gooide haar een.}{bal toe en wilde haar in}{het spel betrekken}\\

\haiku{Er knalden op de.}{lucht eenige rappe schoten}{van nabije jagers}\\

\haiku{{\textquoteright} - {\textquoteleft}Zijt gij zeker dat?}{al zijn inspanning te uwer}{intentie niet was}\\

\haiku{{\textquoteright} - {\textquoteleft}Wij wel,{\textquoteright} zeide ik,.}{snel met een gedachte aan}{mijn vriend Berrewats}\\

\haiku{De dag was te warm.}{geweest voor het laat seizoen}{en het raam stond open}\\

\haiku{Haar was het die zij,.}{volgde terwijl mijn woorden}{geen weergalm wekten}\\

\haiku{- {\textquoteleft}Herinnert gij u?}{dan niet meer wat gij mij}{vroeger hebt gezegd}\\

\haiku{{\textquoteright} zei Jaak krachtig, trots.}{het eerste lallen van een}{verlammende lip}\\

\haiku{Ik keerde huiswaarts.}{met een leedwezen dat mij}{bijna zuiverde}\\

\haiku{Wij gebruikten het:}{avondeten in een zwijgzamer}{stemming dan destijds}\\

\haiku{Doch Emma nam geen:}{notitie van mijn grapje}{en zeide effen}\\

\haiku{Ik heb verleden.}{jaar zes fruitboomen in den}{tuin laten planten}\\

\haiku{Ik verzeker u,.}{dat gij van mij niets meer zult}{te vreezen hebben}\\

\haiku{Tenslotte zag ik.}{slechts uitkomst in de hulp van}{den onderpastoor}\\

\haiku{En dan, gij weet het,.}{zij heeft totnogtoe geen voet}{in de kerk gezet}\\

\haiku{{\textquoteright} Trots mezelf kwam in,.}{mijn geest een argwaan dien ik}{met afkeer verdrong}\\

\haiku{{\textquoteright} Hij was rechtgestaan}{en boven zijn soutane}{was zijn gelaat}\\

\haiku{Niettemin lei ik,.}{mijn hand in de zijne die}{droog en gloeiend was}\\

\haiku{waarom ik wreed moest,.}{zijn tegenover iemand wie}{ik gansch was onthecht}\\

\haiku{- {\textquoteleft}Ik schrijf nog vanavond,.}{aan mijn vader dat ik bij}{mijn man terugkeer}\\

\haiku{- {\textquoteleft}Toch vroeger dan wij,{\textquoteright}.}{hadden verwacht wedervoer}{ik omzichtiger}\\

\section{A. Roland Holst}

\subsection{Uit: Verzamelde werken. Deel 3. Verzameld proza. Deel 1. Deirdre en de zonen van Usnach. Het Elysisch verlangen. Tusschen vuur en maan. De afspraak. Voorteekens}

\haiku{Daden van toorn en,,.}{verraad zullen om u zijn}{o vlam van schoonheid}\\

\haiku{Zij gaven haar in,.}{de hoede van een vrouw die}{de voorspelling wist}\\

\haiku{Alleen de wind zong,,.}{er het oude vreemde lied}{dat geen woorden heeft}\\

\haiku{Lavarcham en,.}{de oude man liepen langs}{den muur geruischloos}\\

\haiku{Maar zijn hoofd neeg, zijn,.}{handen grepen in zijn borst}{en hij wankelde}\\

\haiku{{\textquoteleft}Zij, die daar komen,.}{zullen de trouwe wachters}{zijn van ons leven}\\

\haiku{Zij negen het hoofd,.}{voor de eenzame oude}{vrouw die voor hen stond}\\

\haiku{Wij weten alleen,.}{dat er veel verloren ging}{sinds zij verdwenen}\\

\haiku{Toen strekte Noisa,.}{zijn handen naar haar uit en}{wilde tot haar gaan}\\

\haiku{Maar zij weerde hem,,.}{en ging snel heen wankelend}{brekend in snikken}\\

\haiku{Zij had begrepen,;}{dat Noisa nog een ander}{leven beminde}\\

\haiku{Zij nam haar hand weg,.}{uit die van Noisa en ging}{een schrede terug}\\

\haiku{{\textquoteright} Het weinige, dat,.}{er dien morgen nog te doen}{was werd snel gedaan}\\

\haiku{En Deirdre voelde,.}{zich loopen en zij hoorde}{de stem van Fergus}\\

\haiku{Nergens meer de oogen -.}{van mijn geliefde zijn stem}{niet meer te hooren}\\

\haiku{Toen zij onder de.}{muren voorbijreden zag}{men hun gelaten}\\

\haiku{En haar brekende,:}{stem sidderde van smart en}{liefde toen zij loog}\\

\haiku{O, hoe schoon was zij,!}{in de goede jaren toen}{ik over haar waakte}\\

\haiku{Dwars door die zaal schreed,,.}{hij recht naar de harp die niet}{ophield te spelen}\\

\haiku{En rustig klonk zijn:}{stem van nabij en zonder}{dat het hoofd bewoog}\\

\haiku{In haar oogen heb ik,.}{een licht gezien dat niet meer}{is van zon of maan}\\

\haiku{Aan den voet van zijn.}{harp hadden luisterend hun}{harten gelegen}\\

\haiku{Toen hief zij het hoofd.}{en zag om naar de woeste}{honger van het vuur}\\

\haiku{Zij liep het smalle,.}{strand ten einde tot aan de}{grens van dit geweld}\\

\haiku{Zoo vaak hij om zag,,.}{was het steeds weer achter hem}{dat de muziek bleef}\\

\haiku{De bronzen voeten;}{glinsteren in de wijde}{stilte van den tijd}\\

\haiku{In sterke, zuivre;}{kleuren staat de grazige}{wei van paarden vol}\\

\haiku{Over dat leven breekt;}{de lach als de lichte zee}{op een gouden strand}\\

\haiku{de moeder zal de,.}{maagd de zoon de Koning der}{ontelbren zijn2}\\

\haiku{maar ik zie van mijn.}{wagen uit een bloemig veld}{waar hij op rijdt}\\

\haiku{Bran ziet schittert het;}{spel van de paarden der zee}{in den zomergloed}\\

\haiku{Wij zijn hier van den ';}{aanvang vrij van jaren en}{de dwang vant graf}\\

\haiku{ons vond de zonde,.}{niet en geen wien ooit de kracht}{der jeugd begaf}\\

\haiku{door dit bedrog werd.}{ouderdom en ondergang}{der ziel uw lot}\\

\haiku{van zijn stam zal een.}{korten tijd een schoon wit mensch}{op aarde gaan}\\

\haiku{geheimen zal hij,.}{aan den mensch onthullen blij}{en zonder schroom}\\

\haiku{hij zal een draak zijn,.}{in den strijd hij zal een wolf}{zijn in het woud}\\

\haiku{Doch hij wilde niet,,.}{spreken maar keek slechts naar hen}{lachend met open mond}\\

\haiku{Zoo stil werd toen zijn,.}{liefde dat hij open ging tot}{een ander leven}\\

\haiku{Daarbuiten, op den,.}{omgang hadden zij zich toen}{aan het spel gezet}\\

\haiku{Niet langer staarde;}{de vreemde tegenstander}{weg over den omtrek}\\

\haiku{Met het aanbreken:}{van den derden dag begon}{het weer om te gaan}\\

\haiku{Maar een hand werd op.}{mijn schouder gelegd en gij}{wendde uw hoofd af}\\

\haiku{Snel ontkleedde ik,,.}{mij blies de kaarsen uit en}{legde mij in bed}\\

\haiku{Zacht sloot gij de deur,,,.}{weer en gij liept langs mijn bed}{dwars door de kamer}\\

\haiku{, en hurkend bij den.}{kleinen haard hieldt gij een vlam}{onder het rooster}\\

\haiku{zulke oogen als den;}{mensch bevreemden en het kind}{naar zich toe trekken}\\

\haiku{Op dat oogenblik,}{nam ik mij alleen stellig}{voor later te doen}\\

\haiku{Eindelijk, toen het,.}{zomer was geworden kwam}{ik waar de zee is}\\

\haiku{Achter den dunnen,,.}{wand waartegen ik lag liep}{iemand op de gang}\\

\section{Richard Roland Holst}

\subsection{Uit: Overpeinzingen van een bramenzoeker}

\haiku{{\textquoteleft}En de derde{\textquoteright}, vroeg, {\textquoteleft}?}{mijn jeugdige vriendhij die}{nu gestorven is}\\

\haiku{De muziek maakte.}{mij wrevelig droefgeestig}{en toch gelukkig}\\

\haiku{hij schreef, terug zou,.}{gaan naar het huis vanwaar hij}{eens de reis begon}\\

\haiku{Hij weifelt, hij talmt,,.....}{hij blijft staan iederen dag}{opnieuw die strijd}\\

\section{Anton Roothaert}

\subsection{Uit: De vlam in de pan}

\haiku{Alleen aan zijn broek,.}{was te zien dat hij niet als}{meisje bedoeld was}\\

\haiku{Dr. Hiemstra had verder.}{den storm onbewogen over}{zijn hoofd laten gaan}\\

\haiku{Ruim tien jaar zit het.}{wagentje muurvast en er}{m\'o\'et iets gebeuren}\\

\haiku{Die sergeant was,.}{er een die nooit iets deed op}{zijn eigen houtje}\\

\haiku{Hij glimlacht, nu hij,.}{hoort dat het resultaat niet}{op zich laat wachten}\\

\haiku{Dit bataljon heeft.}{een goeden naam en hier is}{de kantine}\\

\haiku{{\textquoteleft}Denk maar niet, dat de{\textquoteright},.}{eerste klap een daalder waard}{is roept Koen hem na}\\

\haiku{Dat geeft natuurlijk.}{een onpleizierig gevoel}{van onzekerheid}\\

\haiku{Vanavond suist er een}{kwade noordooster en niet}{ver van den laatsten}\\

\haiku{Dit is het huis van,:}{mijnheer Jansen-van der}{Aa dat wil zeggen}\\

\haiku{Telkens moet hij van.}{het pad af om fietsers te}{laten passeren}\\

\haiku{Maar wat gaan we doen?}{met deze ferstoteling}{van edelen bloede}\\

\haiku{In de rustkamer,.}{heerst een ongewoon rumoer}{dat schielijk verstomt}\\

\haiku{Michel zoemde zeer {\textquotedblleft} -{\textquotedblright}:}{belangstellendM m en}{bot er boven op}\\

\haiku{{\textquoteright} {\textquoteleft}Dat klinkt inderdaad,.}{vlot maar zo had hij nog steeds}{geen platte wagens}\\

\haiku{{\textquoteright} {\textquoteleft}Ja, ik verwacht met.}{spanning het optreden van}{Michel Lanslot}\\

\haiku{Hij had al gauw een!}{groten troep om zich heen en}{lieve  hemel}\\

\haiku{Met die pis-praetjes,!}{heb ik niks te maken ad}{fundum govvedomme}\\

\haiku{{\textquoteright} {\textquoteleft}Een onderwijzer.}{komt altijd in zijn eigen}{winkeltje terecht}\\

\haiku{Het lijkt veel op ons,.}{eigen kwaaltje het is dom}{en onvoordelig}\\

\haiku{En als je zoiets,.}{ziet lijkt ons vraagstuk opeens}{veel eenvoudiger}\\

\haiku{Al twee nachten heb.}{ik \`a\`akelig gedroomd van}{lepels en vorken}\\

\haiku{Om vier uur is het -!}{timmerwerk al klaar en niet}{verder vertellen}\\

\haiku{Zij zetten den rug,.}{hol en lopen met losse}{lijven als dansers}\\

\haiku{Buitel maar met je,.}{speelgoed-vlindertje}{in de zon jongen}\\

\haiku{Als hij nu een klas,.}{van die jongens bezig ziet}{vreet hij zijn hart op}\\

\haiku{er wordt geen papier.}{vuilgemaakt en het fluimen}{is afgelopen}\\

\haiku{En waarvoor hadden?}{ze den sergeant Slotboom}{uit zijn werk gehaald}\\

\haiku{Pom-pom-pom,...}{wie heeft de suiker in de}{erwtensoep gedaan}\\

\haiku{Nu zijn baard in de,.}{waskom drijft schijnt het leven}{minder hopeloos}\\

\haiku{Waarschijnlijk zal het,:}{grootste deel B-kleding}{zijn dat wil zeggen}\\

\haiku{We waren op veel,.}{voorbereid maar zo'n uitschot}{had niemand verwacht}\\

\haiku{Hij had de jongens.}{aangespoord om  trots te}{zijn op hun uniform}\\

\haiku{Maar als hij zover,:}{is laat Ras Wenkiboe den}{instructeur vragen}\\

\haiku{In ieder geval,.}{juich ik het toe dat u niet}{langs den kant blijft staan}\\

\haiku{Schurfie zit hem op.}{enkele schreden afstand}{trouw aan te kijken}\\

\haiku{Maar op dit ogenblik.}{heeft zij een grief tegen al}{wat militair is}\\

\haiku{Net als die kale.}{luis van een ambtenaartje}{uit de Beeldenstraat}\\

\haiku{{\textquoteright} {\textquoteleft}Dat is de leider,{\textquoteright}.}{van een kring zegt Koen en buigt}{zich weer over zijn werk}\\

\haiku{Maar hun dienst zouden -, -!}{ze god hier en daar bons op}{de tafel goed doen}\\

\haiku{een huishouden met...}{niks als ruzie was voor haar}{ook zo lollig niet}\\

\haiku{En hiermee was haar.}{positie in den huize}{aanzienlijk versterkt}\\

\haiku{De compagnie gaat.}{banken sjouwen en Quinten}{heeft veel schrijverij}\\

\haiku{je vel tussen haar -.}{nagels dat je haast bleef in}{je eerste schreeuwstuip}\\

\haiku{De geschiedenis.}{van het onderwijs in het}{kwartier van Baarschot}\\

\haiku{ze konden gezien,.}{worden hoefden maar \'e\'en keer}{binnen te komen}\\

\haiku{Toch schijnt mijnheer Haak,.}{te voelen dat er niet de}{juiste stemming heerst}\\

\haiku{Toch is het verder,,.}{dan zij dachten want de}{knal laat zich wachten}\\

\haiku{O Heer, geef mij lust,...}{tot werken maar wil dien lust}{enigszins beperken}\\

\haiku{De zon blinkt op zijn,.}{groot knipmes waarmee hij den}{zesponder aansnijdt}\\

\haiku{Plons, plons, daar gaan er,.}{al twee maar zijn bliksemsnel}{weer op het droge}\\

\haiku{Oei-oei, nu gaat het,.}{verkeerd daar halverwege}{den linkervleugel}\\

\haiku{Ginds heeft Schipper een.}{walletje van een halven}{meter gevonden}\\

\haiku{Zo ver hij kan zien,,...}{zijn er geen achterblijvers}{dus geen verliezen}\\

\haiku{Brinkman, houd den troep,.}{op zijn plaats anders wordt het}{een beestenbende}\\

\haiku{Ja, dat is pech, vindt.}{Beumke en hij trekt een}{meewarig gezicht}\\

\haiku{Toch staat op de borst.}{van dezen ruwen kiel een}{adelaar geborduurd}\\

\haiku{Ja, ze zullen op '.}{t ogenblik wel alle drie}{bij den dokter zijn}\\

\haiku{O, de geest onder,...}{de jongens is uitstekend}{buitengewoon zelfs}\\

\haiku{We waren verspreid.}{opgemarcheerd en kregen}{een beestachtig vuur}\\

\haiku{Toen zag hij, dat links.}{voor ons uit de troep begon}{samen te klitten}\\

\haiku{{\textquoteright} Zij rijden door het.}{dorp en bij de oude kerk}{laten zij stoppen}\\

\haiku{Daar stond de tweede.}{met een machine-pistool}{onder den oksel}\\

\haiku{Toen kreeg de majoor.}{een idee en liet Hollandse}{signalen blazen}\\

\haiku{Gauw een patrouille.}{erheen en het resultaat}{melden aan Stuurman}\\

\haiku{Een zenuwlijder.}{zou ons hier vannacht rustig}{laten verkleumen}\\

\haiku{Drost zet voorzichtig.}{zijn bajonet op en gaat}{vooruit door de gang}\\

\haiku{weg beveelt hij luid.}{praten en dit hoeft hij geen}{tweemaal te zeggen}\\

\haiku{Die kopraal van ons vloekt,{\textquoteright}.}{een mens het geweer uit de}{vingers zegt Klinker}\\

\haiku{Op bliksemoorlog.}{is een recrutenschool dan}{ook niet ingericht}\\

\haiku{V\'o\'or je het weet, heb.}{je tijdens een aanval den}{vijand in den rug}\\

\haiku{Hij ziet duidelijk.}{de verbeten razernij}{op hun gezichten}\\

\haiku{Want Klinker ligt half.}{bedolven onder het zand}{en hij is zeer dood}\\

\haiku{Zo is er ook geen,.}{woord gerept over Karel Koen}{zelfs niet door Michel}\\

\haiku{Zij zien een mijnheer,.}{in officiersuitrusting}{gemaskerd en wel}\\

\haiku{{\textquoteright} Deze mensen zijn '.}{van lichting23 en toen was}{Vlot nog sergeant}\\

\haiku{En omdat meneer,.}{niet te overtuigen is gaat}{hij mee naar dat huis}\\

\haiku{Gelukkig hebben...}{onze wakkere mannen}{het bijtijds ontdekt}\\

\haiku{Uit een der laagste.}{hoeken drupt langzaam het bloed}{op den trottoirband}\\

\haiku{Van de bevelen,.}{begrijpt hij wel niet veel maar}{het zijn bevelen}\\

\haiku{De groeten aan den?}{reserve-kapitein}{Beumke of zo}\\

\haiku{Hij heeft het zoete,.}{gevoel dat hij hierdoor wraak}{neemt op het serpent}\\

\haiku{Wegkruipen was dus...}{het volledig bewijs van}{een slecht geweten}\\

\haiku{Het is niet nieuw, geen,.}{eigen vinding maar voor hem}{toch een ontdekking}\\

\haiku{Hij blijft boven op.}{het walletje staan en kijkt}{lachend op hen neer}\\

\haiku{Dat wil dus zeggen,?}{dat de brug eerst in handen}{van den vijand was}\\

\haiku{Neen, Beumke wist,....}{er niets van komt recht uit den}{polder Zo-zo}\\

\haiku{{\textquoteright}, gilt er een achter.}{uit den hoop en de stem slaat}{over van opwinding}\\

\haiku{En toen ik naar dat,!...}{vliegtuig tippelde zong ik}{zelfs hardop van boum}\\

\haiku{Nocker zelf spreekt van.}{een doodgeboren kindje}{met een lam handje}\\

\haiku{Dit is nog het fraaie,.}{handschrift van Karel's vader}{het lijkt wel steendruk}\\

\haiku{Hij keert haar koppig,.}{den rug toe en nu eerst ziet}{zij hoe bleek hij is}\\

\section{Felix Rutten}

\subsection{Uit: Daags veur Krismes}

\haiku{Camillo nuimde, -!}{mich eine pries en dat waar}{geine kattendrek}\\

\haiku{Wie  angesj is:}{dat noe neit bie Camillo}{et geval gewaes}\\

\subsection{Uit: Novellen}

\haiku{En auch b\`en ich, zo,.}{lankzaam aan al einen daag}{awwer gewoorde}\\

\haiku{{\textquoteright} De Prior dach nao'.}{en keek den ermen hals e}{tiedje sjwiegend aan}\\

\haiku{En de sjtein vlogen,.}{en vlogen of et b\`ontje}{veugelkes ware}\\

\haiku{{\textquoteright} Einen ougeblik.}{sjt\`ong Broeder Balderik gans}{verbawwereird}\\

\haiku{zo auch nog ummer.}{de eesjte fiool}{in h\"o\"or awd gedouns}\\

\haiku{En zo haw ich dus.}{de freule Margarethe}{lere k\`enne}\\

\haiku{H\`ei\"e ze same'?}{neit \`ens e feeske op touw}{k\`onne z\`ette}\\

\haiku{Fruit lachde einen.}{aan van kompotiees van}{zilver en kristal}\\

\haiku{M\`onter g\`ong ze mit '.}{de m\`onter bende de waeg}{op ent landj in}\\

\haiku{Auch wirkde ze 't,.}{j\`onk v\"olkske gaer in de}{handj wie ze mer koos}\\

\haiku{{\textquoteright} Taenge den aovend ' '.}{veil de deur vant hoes veur}{gouwd achterm toe}\\

\haiku{En al gaw ware,,.}{veer twee kameraote}{dr\"ok aan et moele}\\

\haiku{Camillo nuimde,.}{mich eine pries en dat waar}{geine kattendrek}\\

\haiku{En \`om noe euver,:}{et dood punt haer te k\`omme}{vraogde ich h\"o\"om}\\

\haiku{Krism\`es is dan toch,,?}{et fees van \`os herders en}{van alle veevolk}\\

\haiku{Wie angesj is dat:}{aevel neit bie Camillo}{et geval gewaes}\\

\haiku{{\textquoteleft}Dokter, zou et neit '?}{gouwd zeen es ich mesjien toch nog}{mern waek hie bleef}\\

\haiku{Hae hei dat dan toch,.}{k\`onne doon hae hei dat mesjins}{zelfs waal m\`otte doon}\\

\haiku{Bie j\`ong vrouwluuj kan.}{me dat jao nooit zo presies}{te weite k\`omme}\\

\haiku{Mer et weur eine,,;}{sjoe\"e eine lelike kael}{mit al zien marke}\\

\haiku{Moritz h\`ei mit gein, ' '....}{weurd k\`onne z\`egge wiem}{dat aant hart g\`ong}\\

\haiku{{\textquoteright} {\textquoteleft}Geer woont zo sjoon, haet,{\textquoteright}.}{Albaer mich gezag trachde}{Moritz aaf te lei\"e}\\

\haiku{Nog get korter bie,.}{en ich vuilde zien wang langs}{mien wang sjtrieke}\\

\haiku{Me gouf dat neit sjus,.}{oet mitlie\"e mer \`om de kael}{kwiet te waerde}\\

\haiku{en zo h\`olp 'r hun.}{auch nog mit et hout dat ze}{neudig hawwe}\\

\haiku{Et liek droug ein w\`onj,.}{aan de veurkop wo et k\`endj}{aan doodgeblouwd waar}\\

\haiku{dat zal mich egaal zeen{\textquoteright},, {\textquoteleft} '.}{zag de Sjoltheises veerm}{mer kwiet waerde}\\

\haiku{Doe, Reinder, wits de,{\textquoteright}.}{waeg jao nog waal daohaer de}{kamer baoven-in}\\

\haiku{En zo g\`onge veer.}{op bezuik in den tore}{van et guitje}\\

\haiku{Dan b\`en ich waal zo.}{blie dat ich daen aovend neit}{in sjlaop kan k\`omme}\\

\haiku{{\textquoteleft}K\`enjer leif, gouwd.}{aeten en dr\`enke hiltj}{lief en zeel bie-ein}\\

\haiku{t Is et erfgouwd,}{van mien k\`enjer waat ich}{in zien h\`enj l\`eg}\\

\haiku{{\textquoteright} {\textquoteleft}Vader, en ich maak '.}{mich sjterk dat ichm \`onger}{den doem zal hawte}\\

\haiku{Noe zou {\textquoteleft}nonk Sevrien{\textquoteright}, -.}{al gaw onneudig en mesjins}{waal te v\"o\"ol zeen auch}\\

\haiku{Den tied waar gek\`omme,'.}{dat er zich moos \`omzeen nao}{e nuuj besjtaon}\\

\haiku{Hubertien g\`ong \`ens,:}{onger de h\`egge t\"osje de}{netele kieke}\\

\haiku{Die v\`enj ich get,.}{bleik en die zeen mich auch get}{te kl\`ommelechtig}\\

\haiku{Et doerde noe nog,.}{weier get en dao koum ein}{kar aangeroddeld}\\

\haiku{Tilke haw de maad}{van de pastoor gevraog}{ze oet te k\`omme}\\

\haiku{Noe haw ich aevel.}{in zien pastorie neit mit}{h\"o\"om allein te doon}\\

\haiku{Ich h\"ob mich waal \`ens,.}{aafgevraog of ze ooit}{loon getrokken haet}\\

\haiku{ze deig alles om,,}{mager te waerde zag}{ze mer bleef zo get}\\

\haiku{{\textquoteright} {\textquoteleft}N\`onk, wie h\"obs doe dan,.... ?}{auch zo d\`om konne zeen zo}{gans zonger bez\`ei}\\

\haiku{veer z\`egge das doe;}{mit eine m\"orgestrein}{bievandan weggeis}\\

\haiku{Waat veer dao gekaok,.}{en gebakke h\"obbe kan}{ich neit mee z\`egge}\\

\haiku{{\textquoteright} - dan waar ze weier '.}{gans Calypso en veilm}{in sjt\`orm in de erm}\\

\haiku{Mer hae maagde dat., {\textquoteleft}{\textquoteright},.}{In wirkelikheidau fond}{lachde hae mit h\"o\"or}\\

\haiku{{\textquoteright} {\textquoteleft}Wae h\`ei dan auch zo{\textquoteright},.}{get k\`onne d\`enke zag er}{sjtil t\"osje zien tenj}\\

\haiku{Den eine zag dat;}{hae d'roet wol en zich ziene}{waeg waal zou zuike}\\

\haiku{{\textquoteright} {\textquoteleft}Ich kan papa in{\textquoteright}.}{zienen toesjtandj neit bie}{allein laote}\\

\haiku{Koos et Schaldus dan, '?}{allein laote esof}{et neit vanm h\`olj}\\

\subsection{Uit: Onder den rook der mijn}

\haiku{{\textquoteright} Zij stapte moedig '.}{int hooge kruid met hare}{tengere beentjes}\\

\haiku{Ieder ging naar de ',.}{Vespers middags wat nu}{ook niet meer gebeurt}\\

\haiku{De blauwe oogjes,.}{pinkelden schalks maar het wicht}{verroerde zich niet}\\

\haiku{Maar Willem keerde:}{zich nog eens naar de meisjes}{om en riep woedend}\\

\haiku{Een frank per dag en,.}{zweten als een os van den}{morgen tot den avond}\\

\haiku{{\textquoteright} Willem rekte zich ',}{uit overt hek  toen ze}{heenging om te zien}\\

\haiku{{\textquoteright} {\textquoteleft}Ik ben een arme,.}{man die vijf  kinderen}{heb groot te brengen}\\

\haiku{Het was een jongen.}{die op de wijze van zijn}{lied kwam aanstappen}\\

\haiku{Zijn stem bulderde,:}{de heksen tegen die stil}{hielden om hem heen}\\

\haiku{{\textquoteright} {\textquoteleft}Dan zullen we den,,.}{rozenkrans bidden Trina}{en ter ruste gaan}\\

\haiku{Mijnwerkersvolk is,.}{gevaarlijk volk had zijn groote}{makker hem geleerd}\\

\haiku{Zij bedwelmden hem.}{als de jenever die hij}{hem leerde zwelgen}\\

\haiku{Het was alsof er,.}{een zee in ruischte die}{steeg en viel en steeg}\\

\haiku{Zij wilde geluid,.}{geven in haar angst maar kon}{niets uitbrengen}\\

\haiku{Alleen de groene.}{boomen stonden als verstomd}{in al die drukte}\\

\haiku{De vogel was er,.}{af en de schutters hadden}{een nieuwen koning}\\

\haiku{Alles vloog overeind,.}{en stormde den boomgaard uit}{naar het schietterrein}\\

\haiku{Ook was er naar de,,.}{overzij van den weg geen vrij}{schoon wiegend veld meer}\\

\haiku{Nieuwe steenovens dan,!}{en nieuwe huizenreeksen}{om ze te bergen}\\

\haiku{dat Hary Gerards.}{dood geslagen was in de}{Brunssumer heide}\\

\haiku{Zijn heel gezicht was ' '.}{opgezwollen ent haar}{stond stijf vant bloed}\\

\haiku{maar Hary voelde.}{niet dat hare stem eventjes}{onzeker trilde}\\

\haiku{die kermisavond, de:}{overval waarbij de ander}{zich gewroken had}\\

\haiku{Maar ik wilde je,.}{maar eens zeggen dat d\'at zoo}{niet verder gaan kan}\\

\haiku{Het was een gedicht,.}{en Lize bedekte den}{naam van den schrijver}\\

\haiku{Ik had me moeten.}{offeren voor zijn geluk}{en zijn ommekeer}\\

\section{Jan van Ruusbroec}

\subsection{Uit: Het sieraad der geestelijke bruiloft}

\haiku{Zij geldt tot heden.}{voor de beste en er werd}{veel zorg aan besteed}\\

\haiku{Die blind blijven, en,.}{dit gebod verzuimen die}{zijn allen verdoemd}\\

\haiku{en hieruit ontspringt,:}{het andere punt en dat}{komt voort uit de ziel}\\

\haiku{En de groote nood der,.}{menschelijke natuur en}{de eer zijns Vaders}\\

\haiku{Gods barmhartigheid,.}{en onze nood Gods mildheid}{en ons verlangen}\\

\haiku{Zachtmoedigheid maakt.}{in den mensch vrede en pais}{van alle dingen}\\

\haiku{En hij zal gesierd:}{zijn en gekleed met een kleed}{in twee\"en verdeeld}\\

\haiku{Gods is eenvoudig,;}{en niet uit te spreken door}{de creaturen}\\

\haiku{deze dragen naar.}{God alle goede werken}{en alle deugden}\\

\haiku{die hoovaardig is,,.}{is niet ootmoedig en die}{behoort niet aan God}\\

\haiku{en omdat Hij in;}{ons en met ons eeuwig wil}{wonen en blijven}\\

\haiku{Wij moeten ook God.}{loven met al waarmede}{wij dat vermogen}\\

\haiku{Hoofdstuk XXVIII'}{Van de vierde wijze}{van Christus komst NU}\\

\haiku{Zij benemen den.}{grond en de oefening van}{alle innigheid}\\

\haiku{Deze ziekte is,.}{gevaarlijker dan eenige}{van de andere}\\

\haiku{en hij weet niet van,.}{God noch van zich zelven in}{de oprechte deugd}\\

\haiku{maar zij is even wijs.}{op den eersten dag als zij}{het ooit zal worden}\\

\haiku{En zij scheidt niet van,;}{God en nimmermeer doet zij}{dit op deze wijs}\\

\haiku{En daar voor eischt,,;}{noch wil hij iets dat hij niet}{aan God wil geven}\\

\haiku{Al levende sterft.}{hij en al stervende wordt}{hij weder levend}\\

\haiku{en hij gaat in God.}{met genietende neiging}{in eeuwige rust}\\

\haiku{En zij zeggen, dat,.}{zij rusten in Dengene}{dien zij niet minnen}\\

\haiku{Dan ware zij ook,;}{niet meer heilig of zalig}{dan een steen of hout}\\

\haiku{4Het woord wallen door.}{R. gebruikt leeft nog in de}{Limburgsche volkstaal}\\
