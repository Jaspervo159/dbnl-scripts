\chapter[8 auteurs, 659 haiku's]{acht auteurs, zeshonderdnegenenvijftig haiku's}

\section{Top Naeff}

\subsection{Uit: In mineur}

\haiku{{\textquoteleft}Ik ga morgen naar,,...{\textquoteright}}{Deventer altijd Zondag's}{naar de oude lui}\\

\haiku{re... {\textquoteleft}de Brinkhorst{\textquoteright}, 't...?}{buiten van Meneer de Ras}{Alting weet je dat}\\

\haiku{Dit is 't. Mooie streek...{\textquoteright}.}{h\`e De woorden verstikten}{in zijn schorre keel}\\

\haiku{Wreed was hij, wreed in ',.}{t onvermijdelijke}{wreed ook in het noodelooze}\\

\haiku{Met haar oogen groette,,.}{ze stijf recht-\'op zwaar geleund}{tegen de tafel}\\

\haiku{{\textquoteright} En zij speurde niet.}{den toon van ironie onder}{die luchtige scherts}\\

\haiku{{\textquoteleft}Nacht juffrouw Brantsberg,{\textquoteright}, {\textquoteleft}?}{zei hijkan ik u nergens}{meer mee van dienst zijn}\\

\haiku{{\textquoteright} Bruusk ging hij heen, riep,,.}{springend in het rijtuig zijn}{adres aan den koetsier}\\

\haiku{Om hem een geur van,...}{oranjebloesem rozen en}{heliotrope}\\

\haiku{De kellner bracht bier,.}{aan morste kleine plasjes}{op den vuilen grond}\\

\haiku{Misschien zou 't kind, ',.}{hem gauw volgen hij hoopte}{t zoo'n stakkerdje}\\

\haiku{Vanmiddag wacht hij.}{haar weer voor nog een kleine}{verandering}\\

\haiku{{\textquoteright} Gebogen glipt zij,, '.}{de deur uit die achter haar}{dicht slaat int slot}\\

\haiku{'s Morgens, om half,:}{zeven wekte hij haar weer}{met schorre slaapstem}\\

\haiku{Aldoor grinnikend,,,,.}{slipte Christien de deur uit}{sloeg die knallend dicht}\\

\haiku{Verder was op den,.}{ledigen langen weg geen}{mensch te bespeuren}\\

\haiku{Christien zakte op,,, '.}{de knie\"en leunde haar kin}{spits opt kozijn}\\

\haiku{Christien's hart trilde,.}{van trots toen zij daar het hek}{intrad met een heer}\\

\haiku{Zoo eindigde de.}{liefdesgeschiedenis van}{juffrouw Christien}\\

\haiku{De goejen moeten,{\textquoteright},,.}{met de kwajen lijden}{zei vinnig Christien}\\

\haiku{{\textquoteleft}Je moest toch ook nog,.}{eens gaan naar Suze en Frits}{zij drongen zoo \'a\'an}\\

\haiku{Hoe waardeerde zij,;}{dan niet te wonen in een}{groote stad vol gevaar}\\

\haiku{Zij telde... nog een......,,......}{keer stilte nog eenmaal kort}{als een snik nog een}\\

\haiku{En daarom deinst hij,.}{nu terug voor een daad die}{juist moeder verbood}\\

\haiku{Barend, op den kant,,:}{trapt forsch af met zijn voet}{tegen den bootrand}\\

\haiku{De andere mouw,,,.}{opgeblazen klaar lijkt een}{arm te omsluiten}\\

\haiku{Mien God, hoe vake ' ',...}{haktm veureholden}{hoe gevaorlijk}\\

\section{Edouard de N\`eve}

\subsection{Uit: Muziek voorop}

\haiku{{\textquoteleft}U bent zeker van,?}{de eerste lichtingen die}{op moeten komen}\\

\haiku{Hij weet niet meer wat,,.}{te zeggen en eet diep in}{gedachten verder}\\

\haiku{Ondertusschen denkt.}{hij aan het onrecht dat hij}{Francine aandoet}\\

\haiku{En als ze ziet dat:}{Jean teleurgesteld is voegt}{ze er haastig bij}\\

\haiku{Zij heeft alleen met.}{Jean te maken omdat ze}{met h\`em getrouwd is}\\

\haiku{Beroerd, eindeloos,.}{getik dat haar verhindert}{te slapen vannacht}\\

\haiku{{\textquoteleft}Je weet toch dat ik,}{niet anders kon dat ik m'n}{moeder moest zien v\'o\'or}\\

\haiku{Aan mij heeft ze   -.}{altijd een hekel gehad}{al v\'o\'or we trouwden}\\

\haiku{Daarmee schoof hij reeds.}{een eind van het uiterste}{linksche plaatsje weg}\\

\haiku{Dit lachen verbreekt,.}{de angstige spanning die}{hen omvangen houdt}\\

\haiku{Elk woord kan een klacht,.}{zijn elke klacht kan barsten}{tot een huilpartij}\\

\haiku{{\textquoteright} Francine wendt zich.}{om om niet te laten zien}{dat zij nu toch huilt}\\

\haiku{{\textquoteleft}Waarom stuurde hij,?}{hem dan niet direct aan mij}{zooals hij altijd doet}\\

\haiku{En Jean heeft er me,{\textquoteright}.}{nooit een verwijt van gemaakt}{zei Francine koel}\\

\haiku{Maar onmiddellijk.}{daarna was Francine weer}{tot vechten gereed}\\

\haiku{Maar het was misschien,....}{geen grap dat lintje van het}{Legioen van Eer}\\

\haiku{De duimschroeven aan,,,!}{zeg ik je en de poet in}{zonder pardon rang}\\

\haiku{{\textquoteright} Uit de rijen der.}{Hollandsche compagnie stapt}{een man naar voren}\\

\haiku{Hij wentelt er om,....}{en om als een scherpe boor}{in een rotte plank}\\

\haiku{Even later, als hij,.}{in zijn bed ligt kan hij niet}{in slaap  komen}\\

\haiku{Hier en daar kromde.}{een enkele vrouw zich over}{haar  veldarbeid}\\

\haiku{En zeggen dat er....}{nog kaffers zijn die blij zijn}{naar het front te gaan}\\

\haiku{En nu kan het me,,.}{niets meer schelen niets meer wat}{er met me gebeurt}\\

\haiku{Voor mijn part krijg ik '....}{n kogel zoo gauw als}{ik aan het front kom}\\

\haiku{En zij weten niet.}{wanneer zij hun bestemming}{zullen bereiken}\\

\haiku{{\textquoteright} {\textquoteleft}Ja, natuurlijk, als,....}{je de geldpest hebt zooals jij}{kun je zooiets doen}\\

\haiku{Hij had zoo graag \'o\'ok,.}{naar Parijs willen gaan maar}{hij heeft niet gedurfd}\\

\haiku{je zal zeggen hoe,}{graag ik bij je zou zijn en}{als ik gedurfd had}\\

\haiku{Terwijl Monsieur:}{Lasalle zijn auto ging}{halen vroeg L\'eonie}\\

\haiku{{\textquoteright} Francine haalde.}{haar schouders op en wist niet}{goed wat te zeggen}\\

\haiku{Goed,{\textquoteright} zei ze, {\textquoteleft}vanavond,.}{ga ik met je mee daar kun}{je op rekenen}\\

\haiku{{\textquoteright} {\textquoteleft}Daar moet je ze geen,{\textquoteright}.}{gelegenheid voor geven}{ried L\'eonie heftig}\\

\haiku{Zij weet niet hoe zij.}{haar bekentenis onder}{woorden moet brengen}\\

\haiku{Haar eigen lafheid,.}{vergetend kon zij die van}{Jean niet vergeven}\\

\haiku{Onderwijl zou zij.}{doen alsof er niets was dat}{haar verontrustte}\\

\haiku{Maar nooit sprak zij hem.}{over wat zij besloten had}{hem  te zeggen}\\

\haiku{Maar onmiddellijk.}{daarop begon hij over iets}{anders te praten}\\

\haiku{Onder tentdoek en.}{dekens kruipen ze weg en}{trachten te slapen}\\

\haiku{En als je nog \'e\'en,,....}{woord zegt dan nom de Dieu}{sla ik je bek in}\\

\haiku{Vergeefs trachtte hij.}{zich wijs te maken dat hij}{daar geen recht op had}\\

\haiku{Het was een uitkomst.}{voor hem dat die soldaat hem}{had uitgelachen}\\

\haiku{Hij was alleen maar}{ongeduldig geworden}{omdat hij bang was}\\

\haiku{De wereld was vol.}{van scharrelaars en Parijs}{in de eerste plaats}\\

\haiku{Jean zou de eerste...}{zijn van zijn regiment om}{met verlof te gaan}\\

\haiku{Verlof zouden ze,,.}{krijgen dat wisten ze maar}{niemand wist wanneer}\\

\haiku{{\textquoteright} Het leven in de.}{eerste linie was voor hen}{vol onzekerheid}\\

\haiku{Hij spreekt er met zijn,:}{reisgenooten over en}{een adjudant zegt}\\

\haiku{Denk je dat ik me....}{afvroeg waar\`om ik w\`el en de}{anderen nog niet}\\

\haiku{Bij den uitgang van.}{het station nemen ze}{afscheid van elkaar}\\

\haiku{E\'en woordje tegen.}{den chauffeur en zijn taxi brengt}{hem naar Francine}\\

\haiku{Als de Duitschers hem,.}{dan maar niet misten als hij}{er dan maar in bleef}\\

\haiku{Hij was overtuigd dat.}{Francine's huwelijk zoo}{goed als kapot was}\\

\haiku{Francine zou   '.}{nooit verliefd opn man als}{Lemerre worden}\\

\haiku{En ook omdat ik.}{niet wilde dat je minder}{over mij zou denken}\\

\haiku{Maar als het is wat,.}{je moeder me verweten}{heeft dan is het waar}\\

\haiku{Hij heeft nooit van z'n.}{moeder een edelmoedige}{opwelling gezien}\\

\haiku{Onmiddellijk na.}{de opleving is haar hoop}{weer ineengestort}\\

\haiku{{\textquoteright} Francine had maar.}{wijselijk verzwegen dat}{zij \'o\'ok zoo'n jurk had}\\

\haiku{Morgen, over 'n paar,.}{dagen ten hoogste zullen}{ze niets meer krijgen}\\

\haiku{Vijftig meter zijn.}{genomen over een lengte}{van honderd misschien}\\

\haiku{Zijn blonde haar hangt,,.}{in pieken met bloed besmeurd}{over z'n bleeke gezicht}\\

\haiku{Ze wachten op het.}{fluitsignaal dat den aanval}{moet ontketenen}\\

\haiku{En Jean zou er haar,.}{misschien komen opzoeken}{een sc\`ene maken}\\

\haiku{Dat had zij al die,.}{maanden opgespaard daar had}{zij niet aan geraakt}\\

\haiku{Hij was niet streng voor,.}{zijn soldaten hij leefde}{ook niet met hen mee}\\

\haiku{Nu wist hij niet eens.}{meer of Francine nog in}{hun woning wachtte}\\

\haiku{Indien zij niet schreef,,.}{niet antwoordde beteekende}{dat dat zij weg was}\\

\haiku{V\'o\'or hij terug gaat.}{naar het front wil hij dit ook}{\'e\'en keer meemaken}\\

\haiku{{\textquoteright} V\'o\'or hij zijn derde.}{galon krijgt ontvangt hij het}{Legioen van Eer}\\

\haiku{Ze denken dat ze.}{\'o\'ok iets voor het vaderland}{moeten over hebben}\\

\haiku{En hij weet dat die.}{rust bedenkelijk is en}{niet van langen duur}\\

\haiku{De mannen vallen,.}{om van slaap maar zij durven}{niet te gaan liggen}\\

\haiku{Alles is onze,.}{schuld van  jou en van mij}{en van den oorlog}\\

\haiku{Hij w\'e\'et dat hij geen,.}{held is dat hij niets gedaan}{heeft uit echten moed}\\

\haiku{De Russen zijn een.}{onordelijke bende}{en staken den strijd}\\

\haiku{Dat heeft zij ge\"eischt.}{omdat  zij haar vrijheid}{wilde behouden}\\

\haiku{De Italianen.}{worden teruggeslagen}{op de Piava}\\

\haiku{Zij houdt zich echter,:}{goed en zegt haar hand over haar}{voorhoofd strijkende}\\

\haiku{Zij begrijpt vaag dat.}{hij iets anders heeft gewild}{dan beroemd worden}\\

\haiku{Iedere natie.}{wil de eer der overwinning}{voor zich opeischen}\\

\section{Edmond Nicolas}

\subsection{Uit: Brocaat en boerenbont. Schering en inslag van een fabrikantenleven}

\haiku{Na een paar jaren:}{begon hij de wevers zijn}{wil op te leggen}\\

\haiku{laat hem geloven.}{dat hij afhankelijk is}{van de producent}\\

\haiku{Prosper bloosde tot in:}{zijn hals en antwoordde met}{een klein stemmetje}\\

\haiku{Enfin, hij werd door:}{Fr\`ere Canis met volle}{muziek ontvangen}\\

\haiku{{\textquoteright} Op een avond sprak Prosper.}{over deze uitingen van}{Oom Jan met Papa}\\

\haiku{{\textquoteright} Maar geleidelijk,.}{begon Prosper te betalen}{binnen zes maanden}\\

\haiku{Toen het testament,.}{geopend werd meenden ze}{het te begrijpen}\\

\haiku{En zorg nu maar eens.}{dat die order voor Funck en}{Co de deur uitkomt}\\

\haiku{onder de prijs van,.}{Leiden maar zo dat de markt}{niet bedorven werd}\\

\haiku{{\textquoteright} Dat antwoord bracht de;}{dokter zijn moeilijkheden}{weer in gedachte}\\

\haiku{Als hij nu nog aan,.}{lager wal raakte dan was}{alles in orde}\\

\haiku{Maar meneer Snakkers,,.}{had geen tijd dank U wel een}{andere keer graag}\\

\haiku{goed{\textquoteright} zei hij, {\textquoteleft}en ze.}{zullen de eerste jaren}{nog wel beter gaan}\\

\haiku{Maar voor Anna was.}{die zeventiende Mei een}{groot wonder gebeurd}\\

\haiku{Goed dus Dinsdag om,.}{zes uur dan kon ze de Mis}{zelf ook bijwonen}\\

\haiku{Prosper had zich achter,.}{het oor gekrabt en had haar}{lang aangekeken}\\

\haiku{De tuinman nam het,;}{glaasje aan en dronk het bij}{kleine teugjes uit}\\

\haiku{{\textquoteright} Langzaam liep hij met,}{Anna op het klooster toe}{en trok aan de bel.}\\

\haiku{de tweede vleugel.}{zou dan worden bestemd voor}{koetshuis en stallen}\\

\haiku{Hij hoorde van de.}{notaris dat het land erg}{vast in de hand zat}\\

\haiku{{\textquoteright} Bruysten zette zijn.}{ijzeren bril stevig op}{zijn mopsneus en keek}\\

\haiku{Ik zei gisteren?}{toch duidelijk dat ik een}{kasteel ging zetten}\\

\haiku{En toen begon de.}{wonderlijke en goede}{tijd van het convent}\\

\haiku{Het was echter geen,.}{erg resolute Fanny}{die Anna aantrof}\\

\haiku{Maar er was wel een,.}{hoge officier bij van}{de Spahi's nog wel}\\

\haiku{{\textquoteleft}Als ze nu nog \'e\'en....}{keer iets aan het plan willen}{veranderen dan}\\

\haiku{En nu komt U wel,,.}{zeggen dat het schande is}{om zo te bouwen}\\

\haiku{Het enige wat hem,:}{interesseerde  was}{of we opschoten}\\

\haiku{Dat was een kunstwerk,.}{en het is zonde dat het}{afgebroken is}\\

\haiku{{\textquoteleft}Als er nu weer een,.}{mond bij is moest je maar wat}{meer gaan verdienen}\\

\haiku{Maar nu luister eens,,,?}{mijn jongen je hebt het niet}{royaal is het wel}\\

\haiku{{\textquoteright} {\textquoteleft}Dat is te zeggen,,,.}{met een hit wel meneer Prosper}{of een mak beestje}\\

\haiku{Maar dat kon hij, zei,.}{Prosper wel informeren bij}{de franse paters}\\

\haiku{En geleidelijk:}{ontstond een gewoonte in}{het jonge gezin}\\

\haiku{En toen verstoorde,,.}{Prosper volkomen onbewust}{het prachtige feest}\\

\haiku{Nee, meneer Prosper, met,.}{alle respect voor U dat}{gaat zo niet langer}\\

\haiku{De tweede avond was,.}{er weer een volle fles en}{de derde avond weer}\\

\haiku{Maar Marius bleef.}{geen uur langer in Parijs}{dan strikt nodig was}\\

\haiku{De stoffenkoopman,.}{placht van huis tot huis te gaan}{en sloeg geen deur over}\\

\haiku{{\textquoteright} Even leek het, alsof,.}{Prosper iets tegen wou zeggen}{kwaad worden misschien}\\

\haiku{{\textquoteright} {\textquoteleft}Dat is goed en wel{\textquoteright}, {\textquoteleft}.}{zei Mariusmaar daar is}{ook mode  in}\\

\haiku{En voor je het weet,.}{heb je een assortiment}{waar geen eind aan is}\\

\haiku{de concurrentie.}{van Lyon en het Rijnland}{was trouwens te sterk}\\

\haiku{Het werd tien uur, elf,.}{uur middag en  Margot}{was zoek en bleef zoek}\\

\haiku{dat wil zeggen, zeer.}{binnen de grenzen van het}{betamelijke}\\

\haiku{we verkopen, des,.}{te beter is het voor ons}{de aandeelhouders}\\

\haiku{De combinatie:}{had \'e\'en merkwaardige}{specialiteit}\\

\haiku{zijn de mensen waar,.}{ik vanaf kom en waar jij}{ook vanaf komt Prosper}\\

\haiku{Hij betaalde zijn,.}{mensen goed hij zorgde zelfs}{voor hun oude dag}\\

\haiku{Des avonds was er op.}{het kasteel groot diner voor}{veertig personen}\\

\haiku{Een zangwedstrijd van.}{dubbelmannenkwartetten}{zou op zijn plaats zijn}\\

\haiku{Toen de duisternis,;}{langer duurde gingen de}{mensen stil naar huis}\\

\haiku{Donderdag kwam er.}{een uitvoerig telegram}{van Prosper uit Berlijn}\\

\haiku{Het pakte anders.}{uit dan Wagemans en zijn}{vrienden verwachtten}\\

\haiku{Zo betaalde dus;}{eigenlijk Prosper het loon van}{de werkelozen}\\

\haiku{Prosper voelde erg veel,.}{voor de Jezu{\"\i}eten in}{Katwijk bijvoorbeeld}\\

\haiku{eerlijk spelen{\textquoteright} zei, {\textquoteleft}.}{hij half-ernstigniet de}{jongen opstoken}\\

\haiku{Prosper richtte zich op,,.}{zover dat zijn bovenlijf}{achterover helde}\\

\haiku{Maar die rijkdommen,:}{gaven hem geen voldoening}{en hij wist waarom}\\

\haiku{Ze kunnen niet in,,.}{hun blootje gaan werken Prosper}{gebruik je hersens}\\

\haiku{{\textquoteright} {\textquoteleft}Als je gezien hebt,,.}{waar ze mee lopen dan zou}{je dat niet zeggen}\\

\haiku{Des avonds kwam Herman,,.}{uitgenodigd door Anna}{eten op het kasteel}\\

\haiku{En toen kwam er een.}{merkwaardige brief in de}{brievenbus van Prosper}\\

\haiku{Prosper was eigenlijk,,:}{een acteur een groot acteur}{die maar \'e\'en fout had}\\

\haiku{dat hij daar in de,.}{volksbond woonde maar hij was}{ongezeggelijk}\\

\haiku{Als je later denkt,.}{dat je wat bereikt hebt doe}{er dan afstand van}\\

\haiku{ANNA WONEN in.}{een pension voor oude}{dames in Brussel}\\

\subsection{Uit: De erfenis}

\haiku{Die nonnenkostschool,.}{niet meegerekend want die}{heb ik niet geteld}\\

\haiku{{\textquoteleft}Nee, dat was het niet,,.}{maar we spraken over onze}{plannen Driek en ik}\\

\haiku{{\textquoteright} {\textquoteleft}Dat had niets met de,{\textquoteright}, {\textquoteleft}}{erfenis te maken zei}{de vader waardig}\\

\haiku{Ang\`ele zette haar,:}{bril af om op afstand te}{kunnen zien en zei}\\

\haiku{Jules, geef meneer.}{Claudius iets om de keel}{te bevochtigen}\\

\haiku{Het testament, zou,.}{men kunnen zeggen bestaat}{uit twee gedeelten}\\

\haiku{Even ontstond er een.}{geprikkeld gezwatel door}{de hele kamer}\\

\haiku{{\textquoteright} Plechtig en correct.}{begon Jules de brede}{trap te beklimmen}\\

\haiku{{\textquoteleft}Als je me nu de,,.}{post brengt Jules dan zal ik}{die eerst afwerken}\\

\haiku{{\textquoteright} De stem zweeg, en het.}{leek Paca of van haar een}{antwoord verwacht werd}\\

\haiku{Die krijgt het van twee,.}{kanten van de Frenckens}{\`en van de Wevers}\\

\haiku{Ze glimlachte haar.}{openste glimlach en vroeg of}{het zo beter was}\\

\haiku{Deze Grace was,.}{een wrak een ru{\"\i}ne van}{wat ze geweest was}\\

\haiku{En wou jij nu de?}{zaak voortzetten met deze}{kieteltuinmeubels}\\

\haiku{Merkwaardig,{\textquoteright} dacht hij, {\textquoteleft}.}{wat zo'n erfenis toch een}{geluk kan brengen}\\

\haiku{{\textquoteleft}Paca heeft een baan,.}{en is voor het eerst van haar}{leven zelfstandig}\\

\haiku{{\textquoteleft}Je me demande,{\textquoteright},:}{begon hij maar de monnik}{interrumpeerde}\\

\haiku{{\textquoteleft}Als die vrouwmensen,.}{aan de gang zijn geweest kun}{je niets meer vinden}\\

\haiku{Dat moet geweest zijn,,.}{eens kijken bij de bruiloft}{van Henri W\"osten}\\

\haiku{Toen hij het voertuig}{in het oog kreeg wenkte hij}{met zijn grote hoed}\\

\haiku{Dom Willem leidde:}{heel wat herinneringen}{in met de woorden}\\

\haiku{Onze auto kan.}{U wel naar een andere}{bestemming brengen}\\

\haiku{Maar ik zal wel een.}{paar fleskes achterhouden}{voor eigen gebruik}\\

\haiku{En ik wil je wel.}{mager laten eten ook als}{je daarnaar verlangt}\\

\haiku{{\textquoteleft}Als er een oorlog,.}{komt hoeven we niet over de}{toekomst te spreken}\\

\haiku{Nog een paar jaar, en,.}{dan is Clotje een schone}{engel hopen we}\\

\haiku{Ja, ik had wel werk,.}{genoeg maar je wilt er toch}{wel eens over praten}\\

\haiku{Matthieu stond op, en.}{begon opgewonden heen}{en weer te lopen}\\

\haiku{{\textquoteleft}Omdat meneer de.}{Vries in de gaten heeft dat}{ik geen kind meer ben}\\

\haiku{Ik ben natuurlijk,.}{niet gegaan maar mijn broertje}{Jim is gaan kijken}\\

\haiku{Maar als U denkt dat,.}{dit lesje geholpen heeft}{dan vergist U zich}\\

\haiku{Het antwoord was een,,.}{beetje vaag dat hing ervan}{af zei de ander}\\

\haiku{De Vries vermande,.}{zich en kreet dat Claudius}{beledigend was}\\

\haiku{Wat ze anders zou,.}{kunnen betekenen is}{me niet duidelijk}\\

\haiku{{\textquoteright} zei hij na een paar.}{seconden ingespannen}{getuurd te hebben}\\

\haiku{{\textquoteleft}Denk eraan, Alexander,,{\textquoteright}.}{dat het een zakenbezoek}{is zei Clotje streng}\\

\haiku{Tenslotte is een.}{politiek vluchteling een}{man van karakter}\\

\haiku{Maar opeens barstte.}{Ang\`ele uit in een bijna}{hysterisch gelach}\\

\haiku{brocaten tasjes,.}{fluwelen reticules}{en dergelijke}\\

\haiku{dit was de enige.}{manier om U onder vier}{ogen iets te zeggen}\\

\haiku{Even aarzelde ze,.}{en sneed toen resoluut de}{enveloppe open}\\

\haiku{{\textquoteleft}Ik ben verbaasd over,.}{de mate van gelijk die}{je hebt Claudius}\\

\haiku{Hij was trouwens eerst.}{voorgisteren in den Haag}{geweest voor zaken}\\

\haiku{{\textquoteleft}Zo, Driek, wat zei je?}{ook weer over ontmoetingen}{op vreemde plaatsen}\\

\haiku{het luik had geen slot,.}{en zonder moeite openden}{Matthieu en Driek het}\\

\haiku{{\textquoteleft}Hij is wel een zoon,.}{van Gertrude maar hij schijnt}{toch niet te deugen}\\

\haiku{Hoge rijgschoenen,,.}{bijvoorbeeld met een nogal}{hoge brede hak}\\

\haiku{{\textquoteright} {\textquoteleft}Volgens haar was het,{\textquoteright}.}{gevaar niet voorbij lichtte}{Claudius haar in}\\

\haiku{Hoe, daar heb ik nog,.}{geen idee van maar dat mogen}{we wel aannemen}\\

\haiku{{\textquoteright} kreet Clotje opeens,.}{en ze deed een vlugge greep}{naar de paraplu}\\

\haiku{En nu begrijp je,,.}{wel meneer dat ik nu geen}{risico's meer neem}\\

\haiku{kreeg verlof van de.}{verpleegster om even met zijn}{zwager te praten}\\

\haiku{Het lijkt wel of je!}{de hele nacht niet uit de}{kleren bent geweest}\\

\haiku{Maar waarvoor in 's,?}{hemelsnaam waarvoor wou hij}{dat graf openmaken}\\

\haiku{En wat mijn train de,!}{vie betreft ik heb goddank}{mijn eigen fortuin}\\

\haiku{En aan hem dacht zij.}{het eerste toen Claudius}{sprak over sanering}\\

\haiku{{\textquoteleft}En ik stel voor, dat.}{we dit onderhoud ergens}{anders voortzetten}\\

\haiku{Hij wist dat hij hier.}{een pertinente leugen}{had neergeschreven}\\

\haiku{{\textquoteleft}Kom,{\textquoteright} zei Claudius, {\textquoteleft},,.}{we stappen op anders vat}{je nog kou meisje}\\

\haiku{{\textquoteleft}En ik,{\textquoteright} besloot Don, {\textquoteleft}!}{Carlosben de tweede zoon}{van die Gertrude}\\

\haiku{Men kon, om deze,.}{te construeren uitgaan}{van twee gedachten}\\

\haiku{Sommigen dachten.}{dat men goed oud moest worden}{om goed gek te doen}\\

\haiku{Maar hij is zover,,.}{dat hij zijn lichaam niet meer}{opmerkt naar hij zegt}\\

\haiku{{\textquoteleft}Tenslotte,{\textquoteright} zei ze, {\textquoteleft}.}{later tegen Servaasis}{geld een idioot iets}\\

\haiku{{\textquoteleft}Ik vind het wel een,,}{beetje gewaagd wat U doet}{nicht Berendina}\\

\haiku{Tenslotte verscheen,,.}{de moeder een beetje moe}{een beetje ontdaan}\\

\haiku{Ik ga iets drinken.}{dat ik sinds de watersnood}{niet meer heb geproefd}\\

\haiku{Dan gaat ze maar nu.}{en dan eens een dagje naar}{Engeland terug}\\

\haiku{Ik geloof zelfs dat {\textquoteleft}{\textquoteright}.}{zeU tegen zichzelf zegt}{als ze in bad zit}\\

\haiku{{\textquoteright} Bewonderend keek.}{Claudius zijn liegende}{echtgenote aan}\\

\haiku{Maar dat nam niet weg.}{dat Paca zich een beetje}{verwaarloosd voelde}\\

\haiku{En onze dokter.}{zegt dat het best kan komen}{door die verwantschap}\\

\haiku{{\textquoteright} Zelfs Berendina:}{Volleboezem was mild in}{haar uitlatingen}\\

\haiku{wie van jullie meent?}{me te kunnen wijzen op}{ernstige fouten}\\

\subsection{Uit: De heer van Jericho}

\haiku{Buiten dien muur ligt,.}{de hofstede waar de halfer}{van Jericho boert}\\

\haiku{Dit is goed, maar er,.}{ontbreekt iets aan dat ik niet}{kan definieeren}\\

\haiku{{\textquoteright} Nu en dan echter.}{reed de baron Aboe niet naar}{zijn oefenweide}\\

\haiku{hijzelve zeide,}{dat het zoo eenvoudig was}{dat het niet loonde}\\

\haiku{Hij kreeg de helft van.}{het land en meer dan de helft}{van de contanten}\\

\haiku{{\textquoteright} vroeg hij, toen hij met.}{baron Delsain aan een licht}{glaasje moezel zat}\\

\haiku{En de stalknecht dook:}{op uit het donker van den}{stal en riep terug}\\

\haiku{Voor U moet ik de,}{orde der Schitterende}{Schoonheid cre\"eeren}\\

\haiku{Het is waard om leed,.}{te hebben wanneer men op}{dit licht kan hopen}\\

\haiku{Ik vind het verkeerd,,{\textquoteright}.}{dat in mijn huis gewed wordt}{zeide de baron}\\

\haiku{{\textquoteleft}Ik vraag je excuus,,{\textquoteright}, {\textquoteleft}.}{Schenck zeide hijmaar ik}{liet me meeslepen}\\

\haiku{je broer was gek, je,.}{zuster was gek jij bent het}{gekste van allemaal}\\

\haiku{Het leven van den.}{jonker in Brussel kostte}{inderdaad v\'e\'el geld}\\

\haiku{De dood kan nooit zoo,.}{verschrikkelijk zijn als het}{deinen der golven}\\

\haiku{Hoe mijn geachte,.}{voorvader dat gelapt heeft}{is me een raadsel}\\

\haiku{op een anderen{\textquoteright}.}{dronk dan dit aftreksel van}{chineesche kruiden}\\

\haiku{{\textquoteright} {\textquoteleft}Je wint,{\textquoteright} zeide de,.}{baron met een geste van}{ontwapend te zijn}\\

\haiku{En haar adem bleef net,.}{zoo kalm alsof ze in den}{stal stond te vreten}\\

\haiku{Na de verloving.}{hoorden we ineens niets meer}{over publiek worden}\\

\haiku{Als er hoop is, wil,.}{ik wachten zoolang wachten}{als U  verlangt}\\

\haiku{Dan wil ik afstand,:}{doen van allerlei wat ik}{nu aangenaam vind}\\

\haiku{hij was begonnen,.}{te typen en het boek zou}{getypt worden}\\

\haiku{Ik ken de menschen,,.}{ik ken den grond en ik ben}{niet zonder invloed}\\

\haiku{De pastoor is er,}{al de misdienaars kan ik}{elken dag vangen}\\

\haiku{Het is een kwestie.}{van tijd en medewerken}{met de genade}\\

\haiku{deze man heeft een.}{kramersziel en onteert het}{notarieele ambt}\\

\haiku{{\textquoteright} Dien raad volgde de.}{jonge Heer van Jericho}{gedeeltelijk op}\\

\haiku{ze wenden aan hun,.}{steel je ziet ze groen worden}{en weer zilvergrijs}\\

\haiku{En er is bij God.}{toch niets in zijn gedrag dat}{er op zou wijzen}\\

\haiku{{\textquoteright} vroeg Cecily in, {\textquoteleft}.}{verrukte verbazingdan}{had ik toch gelijk}\\

\haiku{{\textquoteright} zei de notaris, {\textquoteleft}.}{heftigen een huiszoeking}{zal het bewijzen}\\

\haiku{{\textquoteright} Hij ging naar de stad.}{en bestelde rolluiken}{voor al zijn vensters}\\

\haiku{Hij neemt de centen,,.}{waar hij ze vindt Van man of}{vrouw van weeuw of kind}\\

\haiku{Kort daarop reed de.}{Heer van Jericho in het}{rijtuigje stadswaarts}\\

\haiku{Ik hoor zoo het een,.}{en ander als ik bij mijn}{cli\"ent\`ele kom}\\

\haiku{Deze kwam na ast:}{den stoel staan en zei met een}{opgewekte stem}\\

\haiku{Sta op, als je een,{\textquoteright}.}{man bent commandeerde de}{Heer van Jericho}\\

\haiku{{\textquoteright} Maar zonder op zijn:}{antwoord te letten ging de}{oude dame voort}\\

\haiku{De groote hoofddeur was,.}{van gebeeldhouwd eikenhout}{met brons beslagen}\\

\haiku{De markiezin is.}{een kwezel en de markies}{was een esprit fort}\\

\haiku{Men had, zoo zeide,.}{men den herder gehoond en}{met vuil geworpen}\\

\haiku{Zooiets is ons nog,{\textquoteright}.}{nooit overkomen zei ze met}{tranen in haar stem}\\

\haiku{Ge laat den Heer in.}{den hof staan en spuit hem schoon}{met de glazenspuit}\\

\haiku{Smoesjes om niet te.}{luisteren naar den pastoor}{worden niet geduld}\\

\haiku{Evenals gij, ben ik;}{sinds geruimen tijd zonder}{berichten van haar}\\

\haiku{Maar nooit vond ik het,,,.}{noodig haar om zoo te zeggen}{te reserveeren}\\

\haiku{Al die effecten.}{en aardsche schatten kunnen}{mij niets meer schelen}\\

\haiku{{\textquoteleft}Ik geloof niet dat,{\textquoteright}.}{dit in dit geval noodig is}{zeide hij halfluid}\\

\haiku{Maar alles kan niet.}{op zijn best gaan in deze}{treurige wereld}\\

\haiku{Hij heeft ons, om het,.}{zoo maar te zeggen uit de}{vuiligheid gehaald}\\

\haiku{Ik bemerkte, dat.}{Duchatel zijn hoeve zonder}{verlies had verkocht}\\

\haiku{Toen, aan het einde,.}{wilde hij niet terugkeeren}{langs hetzelfde pad}\\

\haiku{{\textquoteleft}Er zijn allerlei,.}{redenen waarom men met}{een vrouw wil trouwen}\\

\haiku{Het is tenslotte,.}{een fran\c{c}aise monsieur}{le Baron begrijpt}\\

\haiku{{\textquoteleft}We gaan Monsieur.}{Ernest Lacroix verslaan met}{zijn eigen wapens}\\

\haiku{{\textquoteright} Toen hij van het plan,.}{gehoord had wilde hij in}{elk geval meedoen}\\

\haiku{Als derde was een.}{machtig groot en dik man te}{voorschijn gekomen}\\

\haiku{Het was etenstijd en;}{men vierde in een goede}{herberg het afscheid}\\

\haiku{morgen ontwaakte,.}{de man niet toen ik op zijn}{kamerdeur klopte}\\

\haiku{Ik zette door, want.}{zulk een ontbindend lijk in}{huis stond me niet aan}\\

\haiku{Ik nam dus een pan,,.}{maakte vuur sneed spek en brak}{een paar eieren}\\

\haiku{Reeds voor den middag.}{lag het werk op de hoeven}{om Jericho stil}\\

\haiku{Toen de deur achter.}{hen beiden gesloten was}{zetten zij zich neer}\\

\haiku{{\textquoteright} Met een spring begon.}{Aboe den weg naar Jericho}{af te leggen}\\

\haiku{Dan trekt het beter,{\textquoteright},.}{door zeide Baptiste en}{hij zwaaide de zweep}\\

\haiku{{\textquoteright} zei Cecily, en.}{haar hand zocht de hand van den}{Heer van Jericho}\\

\subsection{Uit: De president}

\haiku{Hij loopt dicht langs de ':}{huizen vant Prinsenvest}{en de Loskade}\\

\haiku{ternauwernood ziet;}{hij de goederen die op}{de kade liggen}\\

\haiku{En tenslotte was.}{hij toch blijven staan om een}{aalmoes te geven}\\

\haiku{De agent kwam terug,:}{aanmerkelijk beleefder}{en eerbiediger}\\

\haiku{hij droeg 'n flambard,{\textquoteright}.}{en een wijde cape zei}{de president vlot}\\

\haiku{Maar dat was niet noodig -:}{de commissaris zei uit}{eigen beweging}\\

\haiku{In dien val streek door:}{hem heen het beeld van een wit}{en zwarten monnik}\\

\haiku{Hij sloeg met z'n hand:}{op de armleuning van den}{stoel en zei hardop}\\

\haiku{En terwijl ik zoo,:}{bedremmeld sta te kijken}{zei de dokter nog}\\

\haiku{Als een heel oude.}{herinnering bezon hij}{zich op het stigma}\\

\haiku{{\textquoteleft}Natuurlijk heb je -}{dat geld van me gekregen}{en dat je hard was}\\

\haiku{{\textquoteleft}Ik weet natuurlijk,.}{niet hoe sterk Uw gevoelens}{waren voor die vrouw}\\

\haiku{En toch wist ze wel,.}{dat het kind niet altijd}{kon blijven zwerven}\\

\haiku{En bezinnend zag.}{hij de vlakke handen van}{den zwerver voor zich}\\

\haiku{Met belangstelling.}{keek de wachtende grijsaard}{naar die voorstelling}\\

\haiku{{\textquoteleft}Een net van vragen,{\textquoteright}:}{en het beeld ontwikkelde}{zich als vanzelve}\\

\haiku{als ik \'e\'en knoop los.}{wil maken moet ik al de}{knoopen ontwarren}\\

\haiku{Ongeveer z\'o\'o als, '.}{Hugo toen was zout kind}{der vrouw nu wel zijn}\\

\haiku{Hij was echter een,;}{te hoffelijk man om dit}{te laten merken}\\

\haiku{{\textquoteright} {\textquoteleft}Ja, Dientje, dat die '.}{een veel beter hand mett}{kind heeft dan zijzelf}\\

\haiku{Ze begrijpt maar niet,{\textquoteright}.}{waar\`om U dit voor haar doet en}{hij aarzelde even}\\

\haiku{Toch moest hij z\'o\'o de.}{taak vervullen die hij op}{zich had genomen}\\

\haiku{Nella had haar hals '.}{n beetje gebogen en}{keek recht voor zich uit}\\

\haiku{{\textquoteright} Toen ze dit gedaan,.}{had bleef de vrouw verlegen}{bij de tafel staan}\\

\haiku{en tegelijk gaf.}{hem dat feit een vaag gevoel}{van onbehagen}\\

\haiku{{\textquoteright} zei de president, {\textquoteleft}.}{glimlachendeen verzetje}{doet iedereen goed}\\

\haiku{{\textquoteleft}Maar hou een volgend,.}{maal Uw trouwring maar aan dat}{voorkomt misverstand}\\

\haiku{{\textquoteleft}Nu kan ik me toch.}{niet meer herinneren waar}{ik gebleven was}\\

\haiku{{\textquoteright} Met moeite hield de:}{president zijn gekrenktheid}{uit stem en gelaat}\\

\haiku{{\textquoteright} De spreker fronste,:}{zijn wenkbrauwen en langzaam}{en moeilijk zei hij}\\

\haiku{{\textquoteleft}Maar U weet toch wel,{\textquoteright}, {\textquoteleft}.}{zei de presidentdat die}{praatjes onjuist zijn}\\

\haiku{eerst tweemaal voor de, '.}{keukenmeid toen eenmaal voor}{t kamermeisje}\\

\haiku{En nu, zooveel jaar,.}{later moest hij trachten dien}{vriend in te halen}\\

\haiku{Telkens als \'e\'en zijn.}{voet op de treeplank zette}{zwiepte de bus over}\\

\haiku{Wilde hij dan 'n?}{martelaar worden voor wat}{hij als zijn plicht zag}\\

\haiku{Als in een koortsdroom:}{gingen de laatste maanden}{aan zijn oog voorbij}\\

\haiku{Ze trokken samen,:}{op een middelpunt zoo ver}{van hem verwijderd}\\

\haiku{het zou een teeken,;}{zijn van hartelijkheid dat}{hij niet verwacht had}\\

\haiku{Het was alsof hij,.}{vreesde naar huis te gaan door}{de stad te loopen}\\

\haiku{Wij, en gij vooral,.}{mogen dit uit den grond van}{ons hart betreuren}\\

\haiku{Maar eveneens waren,,.}{er die gezondheid geld of}{liefde bezwoeren}\\

\haiku{De boer, een norsche,,.}{sombere duitendief had}{zich opgehangen}\\

\haiku{En heel misschien was.}{met een operatie iets te}{bereiken geweest}\\

\haiku{Alles wat ik voor.}{de genezing noodig had kon}{ik mij verschaffen}\\

\haiku{Het was alsof ze.}{voor mijn oogen voetje voor voetje naar}{haar graf schuifelde}\\

\haiku{ik was door het lot,,.}{van Marie gedeclasseerd}{onttroond ontheiligd}\\

\haiku{Ik wilde heel sterk,,.}{heel vast onderzoeken of}{Marie nog leefde}\\

\haiku{Dagen lang lag ik,.}{in bed en ik zweefde in}{een stille wereld}\\

\haiku{Misschien is ze dan -.}{gestorven in wanhoop God}{vervloekende}\\

\haiku{Hij vreest van tijd tot,.}{tijd dat hij de antichrist}{zal blijken te zijn}\\

\haiku{Die onderzocht de {\textquoteleft}{\textquoteright},.}{zaakConchita niet eens maar}{verscheiden malen}\\

\haiku{{\textquoteleft}Antichrist{\textquoteright} en in.}{zijn hart niets dan berouw over}{zijn nieuwsgierigheid}\\

\haiku{Rustig bad pater.}{Colango de completen}{en ging te ruste}\\

\haiku{Tenslotte vroeg hij,;}{den ouden man bij hem te}{blijven overnachten}\\

\haiku{Omdat monseigneur - -.}{drie achtereenvolgens het}{beter z\'o\'o vonden}\\

\haiku{{\textquoteleft}Als de baron tot.}{God terugkeert zijn er geen}{moeilijkheden meer}\\

\haiku{Opeens viel hij op {\textquoteleft}{\textquoteright}.}{de knie\"en en zeiJube}{benedicere}\\

\haiku{En elk dorpeling,,.}{was lid moest lid zijn moest de}{avonden bezoeken}\\

\haiku{Hij was 't die steeds {\textquoteleft}{\textquoteright};}{zijn zaal verhuren moest aan}{Kunst en Wetenschap}\\

\haiku{ze hadden allen,;}{schuld aan het kasteel van pacht}{of retributie}\\

\haiku{tenslotte zei ze.}{den dienst op en een nieuwe}{was niet te krijgen}\\

\haiku{De gebeden van '.}{t begin der Mis liepen}{al als een uurwerk}\\

\haiku{op Pinksterdag was '.}{er g\'e\'en die aans Heeren}{Lijf behoefte had}\\

\haiku{En ze vertelden:}{aan den pastoor dat ze het}{beeld moesten vernielen}\\

\haiku{Degene, die de,:}{bijl gedragen had bleef staan}{en hij zei opeens}\\

\haiku{Maar 't werd bekend,.}{en monseigneur schreef een brief}{om opheldering}\\

\haiku{Twee dagen later:}{kreeg de pastoor een briefje}{van den baron}\\

\haiku{Die eerder de bijl,.}{gedragen had was door den}{bliksem verslagen}\\

\haiku{De rentmeester droeg,, '.}{de bijl \'e\'en klap ent beeld}{zou er geweest zijn}\\

\haiku{Maar op eenmaal zag -.}{hij een licht dat uitging van}{een hoek der kamer}\\

\haiku{Hij stond op - en de ':}{armen uitgestrekt alsn}{bedelaar vroeg hij}\\

\haiku{De drie arbeiders,:}{keken elkaar aan en ze}{zeiden als in koor}\\

\haiku{Op d\'eze plek - op,.}{d\'eze plek wilde hij zijn}{liefde bekennen}\\

\haiku{het,{\textquoteright} zei de oude,.}{man en er waren klare}{tranen in zijn oogen}\\

\section{Rob Nieuwenhuys}

\subsection{Uit: Vergeelde portretten uit een Indisch familiealbum}

\haiku{De avond tevoren:}{had tante Sophie aan}{tafel al gezegd}\\

\haiku{Buiten verkleurde.}{de nacht en langzaam brak een}{nieuwe dag aan}\\

\haiku{Ik 				herkende.}{haar eigenlijk nauwelijks}{zoals ze daar lag}\\

\haiku{Ook het inschuiven.}{in de auto geschiedde}{vlug en geruisloos}\\

\haiku{Hij deed het naar 				.}{inlandse trant en snoof}{meer dan hij kuste}\\

\haiku{het vervolg is het {\textquoteleft}{\textquoteright}.}{verhaal van oom 			   Tjen over}{deooievaarsjacht}\\

\haiku{In naam van de tijd.}{kon er nu gerust over}{gesproken worden}\\

\haiku{gelukkig niet uit,.}{Batavia maar hij zou op}{zichzelf gaan wonen}\\

\haiku{Als een geest of een,?}{spook 			 waarvan ik zo vaak}{had horen spreken}\\

\haiku{Ze was eigenlijk.}{erg vriendelijk voor me en}{lachte telkens}\\

\haiku{toen ineens zoende.}{ze hem en 			 ging daarna}{vlak voor hem zitten}\\

\haiku{Soms waren hele;}{levensgeschiedenissen}{erop afgebeeld}\\

\haiku{Het moet in ieder.}{geval een merkwaardig}{gezicht zijn geweest}\\

\haiku{Een knappe vrouw, met.}{saamgeknepen lippen en}{een brede 			 mond}\\

\haiku{Bepaalde trekken,}{abstraheerde ze gewoon}{ze zette ze apart}\\

\haiku{Ze vroeg en snakte,?}{naar erkenning maar wie kon}{haar 			 die geven}\\

\haiku{Toen hoorde ze haar:}{moeder uit bed glijden}{en langzaam zeggen}\\

\haiku{Het sprak vanzelf dat.}{Midin de volgende}{dag werd uitgehoord}\\

\haiku{er was eigenlijk:}{maar \'e\'en lied waarvan 			 hij}{de wijs goed kende}\\

\haiku{Daar moeten tante.}{Sophie en oom Tjen ook}{hebben gelopen}\\

\haiku{{\textquoteleft}Zeg zeun,{\textquoteright} hoorde ik, {\textquoteleft},?}{hem tegen oom Tjen zeggen}{zeg zeun willen we}\\

\haiku{Tante Sophie,.}{naast hem in haar Japanse}{zijden kimono}\\

\haiku{Enige dagen voor.}{de bevalling viel tante}{Christien in onmacht}\\

\haiku{En zelfs als men het,.}{minimum gewicht neemt}{mocht het kind er zijn}\\

\haiku{Hij 			 glimlachte:}{opgelucht en stemde in}{met een ander plan}\\

\haiku{Maar ze wilde nooit;}{herinnerd worden aan}{sc\`enes als deze}\\

\haiku{In het voorjaar bleek.}{het gewenst dat oom Tjen nog}{\'e\'en winter bleef}\\

\haiku{Ik zie mijzelf het.}{telegram aannemen en}{aftekenen}\\

\haiku{Zelfs de kamer scheen.}{een andere 			 dan een}{ogenblik tevoren}\\

\haiku{het sprak vanzelf 			 .}{dat tante Sophie bij}{ons zou komen}\\

\haiku{Zou het zijn werking,?}{verder doen ook in ons huis}{en 			 ons gezin}\\

\haiku{Het kind was al lang{\textquoteleft}{\textquoteright},.}{overgegaan jaren en}{jaren geleden}\\

\haiku{De heer Treves gaf.}{soms ook 			 precies aan waar}{een kwaal zetelde}\\

\haiku{Rienkie leek deze.}{keer wat teruggetrokken}{en 			 verlegen}\\

\haiku{Ze zong niet alleen,,.}{maar praatte nu ook druk vlug}{en 			 geestdriftig}\\

\haiku{Wat mijzelf betreft,}{er was mij die 			 middag}{veel aan gelegen}\\

\haiku{Welnu, we slaagden.}{er soms in iets daarvan te}{realiseren}\\

\haiku{Rienkie 			 genoot.}{ervan en dat gaf mij een}{grote voldoening}\\

\haiku{Rienkie huiverde.}{toen ze in haar avondmantel}{naar de auto liep}\\

\haiku{{\textquoteleft}Kasian,{\textquoteright} hoorde.}{ik haar op een afstand al}{tegen me zeggen}\\

\haiku{Ze zetten \'e\'en 			 .}{voor \'e\'en hun instrumenten}{achter hun standaard}\\

\haiku{Bij het naar buiten.}{gaan sloeg ik mijn arm losjes}{om Rienkie heen}\\

\haiku{Het spel was alleen.}{te 			 doorzichtig om haar}{de prijs te gunnen}\\

\haiku{Ze was eenvoudig:}{in een lachstuip gevallen}{en had gezegd}\\

\haiku{Tijdens het gesprek}{zei hij dat hij een van zijn}{relaties voor mij}\\

\haiku{Heel 			 anders toch,,?{\textquoteright}:}{dan Kitty deze meisjes}{ja Ze bedoelde}\\

\haiku{Geen wonder dat de.}{oudste het nu voortdurend}{in de buik 			 had}\\

\haiku{De conferentie.}{had plaats in de kamer van}{tante Sophie}\\

\haiku{{\textquoteleft}Ja, wel d\'onker, Lien,.}{maar toch niet z\'o donker}{als deze meisjes}\\

\haiku{Nooit, n\'o\'oit waren die}{kinderen eens lief voor haar}{of deden ze iets}\\

\haiku{In deze fase.}{trad Kitty op en kwam haar}{rechten opeisen}\\

\haiku{je moet er voor een,,.}{tijd uit weg uit dit huis dat}{zal je goed doen}\\

\haiku{vierwielig rijtuig,,:}{genoemd naar de ontwerper}{Deeleman 			   desa}\\

\haiku{zacht gekookte rijst,:}{au bain-marie bereid}{njonja 			 b\u{e}sar}\\

\section{A.H. Nijhoff}

\subsection{Uit: De dagen spreken}

\haiku{Ze beschouwden je.}{als een parasiet die hun}{belasting kostte}\\

\haiku{Maar jij wist weer niet,.}{dat het lot zijn laatste poets}{nog niet gespeeld had}\\

\haiku{Gij hebt den oorlog.}{en zijn dooden en verminkten}{reeds vergeten}\\

\haiku{dat die geuzen de.}{aarde vormen waarop uw}{gouden tempel steunt}\\

\haiku{En daarom vraagt gij.}{met verbazing wat men u}{eigenlijk verwijt}\\

\haiku{Neen, z\`elfs niet als een,...}{hond want een hond verdedigt}{nog zijn meester}\\

\haiku{Velen die nimmer,.}{zonder arbeid waren zijn}{werkeloos geweest}\\

\haiku{Zeker, bezorgde,.}{ouderen de jeugd is min}{of meer verwilderd}\\

\haiku{Gij zijt weggevlucht.}{en velen hebben zelfs uw}{heengaan niet bemerkt}\\

\section{Peter J.A. Nissen}

\subsection{Uit: De akkoorden van het gemoed}

\haiku{Zij stal het hart van,- {\textquoteleft}}{Van Deyssel die aan haar in}{188485 het sonnet}\\

\haiku{Het betoog werd even}{vurig ontvangen als het}{werd uitgesproken}\\

\haiku{Ook over de eerste.}{opvoering doet een verkeerd}{jaartal de ronde}\\

\haiku{In 1883 werd Seipgens.}{benoemd tot leraar aan de}{Rijks H.B.S. te Leiden}\\

\haiku{Desalniettemin}{is het toch wel een aardig}{liefdesgedichtje.143}\\

\haiku{De Goddelyke,:}{kindervriend Door de onschuld}{ingenomen Sprak}\\

\haiku{- Gij eindelijk zet.}{de kerk niet meer beurtelings}{in gloed en tranen}\\

\haiku{O Gods naam, Gods daad,,!}{Gods Wijsheid Gods Liefde zij}{eeuwig gezegend}\\

\haiku{Hij is Heer van den,.}{tijd Hij is Heer en Meester}{van de Eeuwigheid}\\

\haiku{- dat is toch een man,;}{van fatsoen Dient netjes en}{deftig te leven}\\

\haiku{Glimlagchend, ziet zij,,:}{hem dan aan Verheugt zich om}{zijn spijt En zegt dan}\\

\haiku{Een ieder wacht van,.}{d'aanval slechts Het nog te}{geven teeken}\\

\haiku{Zelfs veegt men zich een ',.}{traan uitt oog En drinkt en}{klinkt nu weder}\\

\haiku{Emile Seipgens   , '}{Onrust  Wolken met uw}{snelle vaartk Heb}\\

\haiku{Ik kwam - doch gij werdt,....}{daadlijk bang En gij begont}{bijna te weenen}\\

\haiku{Het heel publiek, det '....}{knipt zich uigskes Mer opt}{inj applaudisseert}\\

\haiku{Men ging dan ins nao,.}{Hinsberg h\`er Dao gaof men}{ei fameus concert}\\

\haiku{Ein kalfsborst mit ein {\textquoteleft}!}{voos of zesEn meuglik oug}{daonao ein fles}\\

\haiku{Hai is van leefde,;}{half verschmacht Hai dinkt aan}{trouwe daag en nacht}\\

\haiku{Wie praoper zit,.}{dai stevelet Waat is det}{veutje klein en net}\\

\haiku{wie ein beugelp\'ort{\textquoteright} is.}{een Roermondse uitdrukking}{voor o-benen}\\

\haiku{Is eine rieke,.}{get verkaajd Dai leet mer nao}{den dokter schikke}\\

\haiku{Mer velt van oos luuj,.}{eine drin Dai mot mer zelf}{kartoesche biete}\\

\haiku{Den erme miens, de,.}{slumste droet Dai kan dich kluut}{en klompe make}\\

\haiku{De m\"oggen dansden,.}{om os haer De krekel had}{zich heis gezongen}\\

\haiku{De zon zonk ne\^er in.}{volle glans Wie in ein zee}{van golje water}\\

\haiku{{\textquoteleft}Had ich toch eer aan!}{de oer228 van m{\^\i}nen dood}{as aan uch229 gedacht}\\

\haiku{In mijn verbeelding,!}{was ze reeds een arm meisje}{dat ik redden moest}\\

\haiku{Ik gevoelde mij,.}{als gered als ontsnapt aan}{een dreigend gevaar}\\

\haiku{Renilde nam den.}{arm van Jeanne en trad}{met haar naar binnen}\\

\haiku{Des anderen daags .........................................................}{vertrok ik en heb Maasloo}{nooit weergezien}\\

\haiku{{\textquoteright} 't Was mij of een,.}{berg verplaatst werd die op mijn}{hart had gelegen}\\

\haiku{Paschen is de;}{jubel over de opstanding}{na de lijdensweek}\\

\haiku{Hoe jammer dat gij,,,!}{mijn trouwe vriend mijn leidsman}{afwezig moest zijn}\\

\haiku{Nog een korten draai -;}{en plotseling stond ik aan}{den zoom van het bosch}\\

\haiku{gestorven is en.}{hem slechts een eenig dochtertje}{heeft nagelaten}\\

\haiku{En daar ligt nu de!}{lieve vrede van mijn lief}{dorpje verbroken}\\

\haiku{Op een avond, dat ik,.}{mij door de dorpsstraat spoedde}{trad hij op mij toe}\\

\haiku{Omdat wij twee geen,.}{vrouw geen familie en geen}{kinderen hebben}\\

\haiku{Soms  was het mij,......}{onmogelijk te gelooven}{wat ik gezien had}\\

\haiku{Non vastator,Niet als,;}{dwingland~Sed salvator}{rukt hij nader}\\

\haiku{Elk vreemde scheepskiel,,!}{strijkt de vlag O Vaderland}{voor Uw gezag}\\

\haiku{L\`a, plus de trouble,,;}{plus de peine La paix y}{r\`egne pour toujours}\\

\haiku{Ig zel opstoan,:}{en noa mie vader goan en}{ig zel hem z\`egge}\\

\haiku{Nu staat Jantje nog,:}{daar met zeep en met zout Denkt}{steeds bij zichzelve}\\

\haiku{Als Sasse bij ons,,}{het graag aangenaam heeft Wel}{drommels dat hij zoo}\\

\haiku{Daartusschen staken {\textquoteleft}{\textquoteright} {\textquoteleft}{\textquoteright}.}{de velden metweit en met}{haver donker af}\\

\haiku{iedereen kende {\textquoteleft}{\textquoteright} {\textquoteleft}{\textquoteright}.}{hem in Wiel en devr\`emde}{gingen hemniks aan}\\

\haiku{Hier weigerden zijn,,.}{beenen den dienst en besloot hij}{wat rust te gunnen}\\

\haiku{De kathedraal van.}{Roermond v\'o\'or en na de brand}{van 20 mei 1892}\\

\haiku{De kathedraal van.}{Roermond v\'o\'or en na de brand}{van 20 mei 1892}\\

\haiku{Zij waren goed en,!....}{lief de bewoners van Boschdorp}{in mijn kindertijd}\\

\haiku{of wel een ander,.}{van de talrijke liedjes}{toen te Boschdorp in zwang}\\

\haiku{want toen werd ik, door,.}{een derde zusje voorgoed}{er uit verdrongen}\\

\haiku{Eindelijk meen ik,.}{dit gevonden te hebben}{en slaap rustig in}\\

\haiku{Veur zollen dig in,.}{det dupke sjt\`eken En dig}{lever zelver \`ete}\\

\haiku{Intusschen zitten {\textquoteleft}{\textquoteright}:}{moeder en ik alleen in}{onzesjtoofkamer}\\

\haiku{- En, omde's te zoo,,.....}{braaf bus kr{\^\i}gs te oug die sjoon}{knuip allemoal}\\

\haiku{Terstond word ik door;}{de spelende kornuiten}{op straat opgemerkt}\\

\haiku{Om het zeerst word,,,.....}{ik bekeken bewonderd}{en wellicht benijd}\\

\haiku{Bij den wrevel, die, ',;}{bijt zwoegen Somtijds uit}{het binnenst welt}\\

\haiku{En dan, n\'ogeens, het,,,,.}{k\'on niet neen heusch het k\'on}{niet dat van Marie}\\

\haiku{Hij kon t\'och nergens.}{aan dat vuil ontkomen dat}{over zijn ziel hing}\\

\haiku{Maar daar floot het van:}{de locomotief en de}{conducteurs riepen}\\

\haiku{Hier vlak bij, als groote,,.}{jongen daar in Roermond had}{hij vroeger geleefd}\\

\haiku{De h\^otelhouder,.}{kende hem niet meer en hield}{hem voor een vreemde}\\

\haiku{Er zouden nu toch.}{wel geen wandelaars zijn op}{de Kapellerlaan}\\

\haiku{Christus daaraan te,.}{sterven zwaar-bloedend uit}{gruwbare wonden}\\

\haiku{Kom, hij zou zich nog,...}{maar eens omdraaien dan zou}{de slaap wel komen}\\

\haiku{Het was hem of er,.}{iets in hem zou barsten en}{zijn hoofd duizelde}\\

\haiku{Negen hoofdstukken,,-.}{over Roermonds geschiedenis}{Roermond 1985 177200}\\

\haiku{Gerhard van Wessem (), () ().}{1203 De laatste Noorman891}{en Jan van Weert1635}\\

\haiku{44Exemplaar van de,.}{eerste druk in SBM van de}{tweede druk in UBN}\\

\haiku{47Een echte goede.}{biografie van Cuypers}{is er nog steeds niet}\\

\haiku{Eerste ontmoeting -{\textquoteright}, (),,-.}{Uit brieven De Beiaard 5}{1920 deel II 516}\\

\haiku{De katholieke{\textquoteright}.}{architectuurtheorie}{van Alberdingk Thijm}\\

\haiku{Bijlage tot de-,,-.}{Handelingen van 19091910}{Leiden 1910 1933}\\

\haiku{Bijlage tot de-,,-.}{Handelingen van 18961897}{Leiden 1897 120}\\

\haiku{Borel heet er Paul.}{Waerens en Sophie heet}{Corrie van Meeden}\\

\haiku{Jacob Lodewijk () (-),.}{Louis Herten18461920 koopman}{en schoolopziener}\\

\haiku{321E\'en in een zet, met.}{kracht door den ring geworpen}{bal telt zes punten}\\

\section{Nel Noordzij}

\subsection{Uit: Het kan me niet schelen}

\haiku{het was een rompje.}{met een hoofdje en op dat}{hoofdje een hoedje}\\

\haiku{ik moet er niet aan,.}{denken maar ze zou hem zelfs}{in zijn jas helpen}\\

\haiku{Natuurlijk hoor je,.}{bij ons ik zie geen verschil}{tussen jou en mij}\\

\haiku{Jenny stond het niet,.}{te denken ze fluisterde}{het de kamer in}\\

\haiku{Met spitse lippen.}{slurpte ze de gloeiende}{koffie naar binnen}\\

\haiku{{\textquoteleft}Er gebeurde iets.}{op de mannenzaal waar ik}{niets mee nodig had}\\

\haiku{Naakt was ze alleen,.}{een karkasje geschikt voor}{de vuilnisemmer}\\

\haiku{Als Vincent in de,.}{eetkamer bij de radio}{zit hoort hij me niet}\\

\haiku{Meneer had het al.}{gezien en de bende de}{bende gelaten}\\

\haiku{Die schrijverij is.}{een zoeken en tasten naar}{klaarheid met zichzelf}\\

\haiku{Als ik de kalk heb,.}{weggeveegd ga ik naar de}{Emmalaan dacht ze}\\

\haiku{van je,{\textquoteright} Volgens haar,.}{hoorde het er bij dat je}{je meisje zoende}\\

\haiku{Ze duwde hem weg,}{maar keek nieuwsgierig naar de}{foto's in zijn hand.}\\

\haiku{Had Vincent de doos?}{bonbons ook teruggelegd}{in het nachtkastje}\\

\haiku{{\textquoteright} De vrouw maakte een.}{stotend geluid met haar keel}{en blies voor zich uit}\\

\haiku{Ze hield haar ogen op,.}{dezelfde hoogte ook toen}{de tram weer doorreed}\\

\haiku{We schreven elkaar,.}{briefjes want we zaten niet}{in dezelfde klas}\\

\haiku{De blinddoek, die ze,.}{voor hadden was een voile}{maar een keiharde}\\

\haiku{Enfin, maak je niet,.}{ongerust ik heb geleerd}{me te beheersen}\\

\haiku{Misschien leek het er,,.}{op maar ik was vijftien dat}{maakt een groot verschil}\\

\haiku{Ren\'ee zal niet met.}{die vriendin naar bed willen}{vandaag aan de dag}\\

\haiku{Dus eigenlijk houdt,.}{ze alleen van zichzelf daar}{komt alles op neer}\\

\haiku{ik hou van vrouwen,?}{en meer niet of pluis je de}{oorzaak daarvan uit}\\

\haiku{Ze legde haar hoofd.}{voorover op de tafel en}{begon te brullen}\\

\haiku{In het bed in de.}{uiterste hoek van de zaal}{lag een dikke man}\\

\haiku{Zingende zagen,,.}{cirkels alles zit in dat}{hoofd en dan val je}\\

\haiku{Ik zou hem moeten,,:}{zoenen zijn dikke bange}{kop tegen me aan}\\

\haiku{{\textquoteleft}Waarom weigerde,?}{U mij gisteren Uw adres}{niet toen ik het vroeg}\\

\haiku{{\textquoteright} Jenny vertelde,.}{dat hij altijd dronken is}{ik merk er niets van}\\

\haiku{Misschien gaat Meerboom,,,.}{even kijken nee hoe kan dat}{nou die is bezig}\\

\haiku{Niet huilen, dit is,,.}{waanzin dacht ze ik heb er}{geen  reden voor}\\

\haiku{hij trok het mes er.}{uit en legde het naast het}{zijne op zijn bord}\\

\haiku{Ik deed net alsof,.}{vond het een mooie manier om}{het af te leren}\\

\haiku{(En Uw vader en),.}{Uw moeder straal flauw ben ik}{ik wou het zelf ook}\\

\haiku{Maar mam, hij verveelt,,,,.}{me zo eerst mild dan bars dan}{weer mild dan weer bars}\\

\haiku{Je moet eens een avond,.}{bij me komen ik zie je}{zelden bij mij thuis}\\

\haiku{Anderen staken.}{hun hand omhoog en gingen}{op hun tenen staan}\\

\haiku{Met haar elleboog.}{schoof ze de vrouw opzij en}{maakte de deur open}\\

\haiku{Ze boog haar gezicht.}{naar voren en deed haar mond}{wijd open voor de spijl}\\

\haiku{Ren\'ee bestelde.}{een half flesje witte wijn}{en begon te eten}\\

\haiku{{\textquoteleft}Zal ik je iets van.}{vertellen en laat het je}{een lesje wezen}\\

\haiku{{\textquoteright} Och kom, dacht Jenny,.}{in mijn eigen kamer mag}{ik niet meer praten}\\

\haiku{Het komt er niet op,.}{aan wat en voor wie je voelt}{als je maar \`echt voelt}\\

\haiku{schone lakens, een.}{grote asbak en nergens}{stof of viezigheid}\\

\haiku{{\textquoteleft}Heb jij {\textquotedblleft}Les jeux sont{\textquotedblright},?}{faits van Sartre gelezen}{of de film gezien}\\

\haiku{Ze keerde zich half.}{van Ren\'ee en Lucas af}{en keek de zaal in}\\

\haiku{Ze hoorde stemmen.}{uit de muren van lege}{lokalen bruisen}\\

\haiku{Alles wat je zegt,,.}{fraai en meestal minder}{fraai berust daarop}\\

\haiku{Zij bewogen hun.}{handen zenuwachtig langs}{elkanders lichaam}\\

\section{Cees Nooteboom}

\subsection{Uit: Een middag in Bruay}

\haiku{Aan de overkant een,.}{bloemenwinkel een Maison}{de Confiance}\\

\haiku{De andere stem,,.}{die achter in het toestel}{kabbelt rustig door}\\

\haiku{Tegen de paal van {\textquoteleft}}{een stoplicht zie ik een geel}{pamflet waarop staat}\\

\haiku{Lege, open ruiten.}{waaruit belachelijke}{gordijnen zwaaien}\\

\haiku{O liefelijke,,!}{Natie o poel van rust o}{ongestoord eiland}\\

\haiku{Hoe lang zal het nog?.}{duren eer er werkelijk}{zo gedacht wordt p.s}\\

\haiku{Definitiever.}{ziet de laatste foto van}{de serie er uit}\\

\haiku{Hij heeft een vlag bij,,!}{zich de moordenaar en wat}{draagt hij een gek pak}\\

\haiku{Eigenlijk is dat,,.}{de reden geweest voor mij}{om dit te schrijven}\\

\haiku{Verzoend met Duitsland,,.}{Algerije geregeld de}{ziekte genezen}\\

\haiku{Een kind loopt voorbij.}{met een oranje muts op en}{blaast op een toeter}\\

\haiku{Ik was elf jaar toen,.}{ik werd bevrijd en ben dus}{niet zo erg bevrijd}\\

\haiku{Dat is vlug werk, dacht -.}{hij en wachtte even met de}{hoorn op te nemen}\\

\haiku{{\textquoteright} Buylers noemde zijn {\textquoteleft}.}{bedrag en voegde erbij}{zware roebels graag}\\

\haiku{Er sterven nu in.}{onafhankelijk Kongo}{mensen van honger}\\

\haiku{Hij zegt dat hij hem,.}{nu niet af kan doen maar dat}{ik hem morgen krijg}\\

\haiku{Het zonlicht heeft de.}{halve cirkel rondgemaakt}{en is verdwenen}\\

\haiku{Het twaalfde uur van.}{de dag glijdt voorbij zonder}{dat er iets gebeurt}\\

\haiku{Hij is heel klein, en,.}{kijkt verschrikt naar het grote}{dier de menigte}\\

\haiku{Wie het nu niet weet,.}{zal het nooit weten niet door}{woorden tenminste}\\

\haiku{Een van de mannen.}{heft zijn arm hoog op en laat}{die auto stoppen}\\

\haiku{Het wordt allemaal,.}{opgeschreven en ik moet}{ondertekenen}\\

\haiku{Er staan metalen,.}{stoeltjes maar het is te koud}{om te gaan zitten}\\

\haiku{Een Duitse middag [].}{17 januari 1963 Drie}{uur in de middag}\\

\haiku{Het is ook moeilijk.}{om aan de indruk van die}{stem te ontkomen}\\

\haiku{menigte omdat}{ik aanwezig ben en help}{een hal te vullen}\\

\subsection{Uit: Een nacht in Tunesi\"e}

\haiku{Na dat halve uur}{komt er een mevrouw die zegt}{dat meneer Sadek}\\

\haiku{We blijven daar een,.}{tijdje staan hierna zullen}{we niet verder gaan}\\

\haiku{een meneer in het.}{Hollands Maandblad beweert dat}{ik gelogen heb}\\

\haiku{{\textquoteleft}Ik ben formateur,{\textquoteright}}{had hij toen geantwoord en}{zijn blik verlegen}\\

\haiku{Zo werd hij in drie {\textquoteleft}{\textquoteright}.}{maanden een negatieve}{held van deze tijd}\\

\haiku{Hij weigert naar de.}{microfoon te gaan en spreekt}{vanuit zijn koorstoel}\\

\haiku{Verderop drijven.}{ze voorbij en ik zie dat}{het water snel stroomt}\\

\haiku{{\textquoteleft}We kunnen wel even,.}{naar oma gaan daar zijn we in}{geen tien jaar geweest}\\

\haiku{Hij gaat naar Lagos, en,?}{waar ga ik naar toe en of}{ik iets wil drinken}\\

\haiku{Het is koud, voor het.}{eerst dit jaar voel ik dat het}{winter zal worden}\\

\haiku{dat er op neerkomt.}{dat die werf nog wel goed is}{voor een verkiezing}\\

\haiku{Drieduizend mensen,...}{plus zesduizend die van hen}{afhankelijk zijn}\\

\haiku{Tussen hen in staat.}{een glas cognac waar ze om}{de beurt van drinken}\\

\haiku{Maar dat kan ook een.}{kwestie zijn van simpelweg}{de prijs opdrijven}\\

\haiku{Maar niemand heeft het,.}{ooit kunnen zien en ik ga}{naar het monument}\\

\subsection{Uit: Een ochtend in Bahia}

\haiku{Terug was alles,,.}{hetzelfde dat wil zeggen}{even anders als heen}\\

\haiku{Ik weet dat het mooi,,.}{is dat het bestaat en dat}{ik er niet bij hoor}\\

\haiku{Het licht is al wit.}{genoeg om geen detail te}{laten ontsnappen}\\

\haiku{Het eerste wat ik.}{zie is de zwarte modder}{onder de huizen}\\

\haiku{De stemming is er,.}{onverdeeld vrolijk en deelt}{zich zelfs aan mij mee}\\

\haiku{Dan zullen we ons.}{er auf eigene Faust mee}{moeten uitrusten}\\

\haiku{Aan de overkant stond,.}{een heel anders geaarde}{menigte en keek}\\

\haiku{Ik laat mijn perskaart,.}{zien hij kijkt er minachtend}{naar en duwt me weg}\\

\haiku{Op de straten is.}{nog duidelijk te zien waar}{het gebrand heeft}\\

\haiku{Het asfalt is niet,.}{alleen nat het is ook erg}{onafhankelijk}\\

\haiku{Door er niet meer te,.}{zijn bestaat hij nog als de}{vraag naar wie hij was}\\

\haiku{Het wordt al donker, '.}{en buiten is hett uur}{van de paseo}\\

\haiku{De jeugd van de stad,.}{flaneert over de straten steeds}{dezelfde route}\\

\haiku{De vreemdeling loopt,.}{er tussendoor en wordt als}{vreemdeling herkend}\\

\haiku{Maar het ergste, in,,.}{die hitte zijn de zware}{leren beenstukken}\\

\haiku{Vijf keer ben ik in,.}{Portugal geweest en ik}{heb het nooit gehaald}\\

\haiku{Langs de weg karren -.}{met wielen zonder spaken}{dichte stukken hout}\\

\haiku{In een klein stadje.}{in het zuidoosten krijg ik}{pech aan mijn auto}\\

\haiku{Hoe ver is deze,,!}{kerk van laten we zeggen}{de Nederlandse}\\

\haiku{Londen heeft vandaag,.}{een zeer grijze hoed op en}{ik loop er onder}\\

\haiku{En aan het eind van,.}{Brighton de boulevard en}{daaronder de zee}\\

\haiku{In Beaune, in de,.}{Bourgogne was er geen plaats}{meer in de herberg}\\

\haiku{Een voorwereldse.}{portier achter een hoge}{bruine lessenaar}\\

\haiku{Zo ontstond, niet zo,.}{ver van Madrid de vallei}{der gevallenen}\\

\haiku{Als ik binnenkom.}{voel ik een door machines}{gemaakte koelte}\\

\haiku{Koud is het, dat is,.}{het enige terwijl buiten}{de zon verder brandt}\\

\haiku{hij vocht was beter,.}{geweest maar de farao wou}{een pyramide}\\

\subsection{Uit: De Parijse beroerte}

\haiku{Cees Nooteboom}{De Parijse beroerte}{Colofon}\\

\haiku{Herakleitos    ,!}{Leve Heraclites weg}{met Parmenides}\\

\haiku{Een enkeling spreekt.}{over een rechtse staatsgreep maar}{wordt weggelachen}\\

\haiku{Gisteren schreef ik {\textquoteleft}{\textquoteright},.}{dat hij er al niet meer was}{dat was gisteren}\\

\haiku{In een gaanderij.}{van de Sorbonne hangt een}{recept voor bommen}\\

\haiku{De ontkenning van...}{alle verlangens van de}{arbeidersklasse}\\

\haiku{Daar is het heet, ik,.}{rijd er in en er uit op}{weg naar het westen}\\

\haiku{In de dorpen zijn,}{de raadhuizen open door de}{open deuren zie ik}\\

\haiku{En wat zou je er,?}{ook mee moeten hier tussen}{al die weilanden}\\

\haiku{ga heeft Mitterrand.}{zijn treurige liedje al}{tien keer gezongen}\\
