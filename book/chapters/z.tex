\chapter[4 auteurs, 916 haiku's]{vier auteurs, negenhonderdzestien haiku's}

\section{Marie C. van Zeggelen}

\subsection{Uit: Bij het hart van Indi\"e}

\haiku{Als dit waar is, zal.}{Prins Rono Soeswito pier}{nimmer terug keeren}\\

\haiku{Even achter den vorst,.}{stonden de Radena joe}{en hare dochters}\\

\haiku{Het gelaat van den.}{Pangeran Adipati sprak}{van een groote zachtheid}\\

\haiku{De ouden wisten,.}{het wel dat had hun vereering}{gedaan voor den boom}\\

\haiku{ja in naam was het,?}{alles nog het Zijne maar}{in werkelijkheid}\\

\haiku{Alles was h\`em, want,!}{eenmaal had Ali geld van hem}{geleend eenmaal maar}\\

\haiku{Toean Allah had.}{hem geluk gegeven maar}{veel ongeluk ook}\\

\haiku{Amsin had bij een.}{ruzie in Wirio's huis}{zijn mes getrokken}\\

\haiku{Den grooten dag dien!}{Allah zegenen mocht voor}{den kleinen Simin}\\

\haiku{Niemand denkt er aan {\textquoteleft}{\textquoteright}.}{deprintah van den grooten}{heer op te volgen}\\

\haiku{In de duisternis.}{verhief zich hoog en groot de}{breede stam voor hem}\\

\haiku{{\textquoteleft}Gij zijt laat, later,.}{dan gewoonlijk de bedoek}{heeft al geslagen}\\

\haiku{{\textquoteright} Andoe richtte het,:}{hoofd op en haar oude stem}{die beefde zeide}\\

\haiku{Zij had altijd met}{verachting aan hem gedacht}{en nu deed zij weer}\\

\haiku{{\textquoteleft}Het kan niet anders,, '{\textquoteright}.}{Heert is beter dat ge}{komt als de zon zinkt}\\

\haiku{{\textquoteleft}Groote heer{\textquoteright} en zij sloeg,.....}{een rooden doek rood als een vlam}{over hoofd en schouders}\\

\haiku{De vorst weet zeker,.}{wel dat het gouvernement}{op het antwoord wacht}\\

\haiku{Heer, ze is op den,....}{weg hierheen zij zal den weg}{langs het meer nemen}\\

\haiku{{\textquoteleft}Ik groet u allen,.}{die gekomen zijt omdat}{ik geroepen heb}\\

\haiku{Zij waren aan boord,.}{van de groote Padoeakan maar}{er kwam ongeluk}\\

\haiku{Een hunner leidde.}{het dier bij den teugel om}{het te doen grazen}\\

\haiku{het straks alles te;}{vertellen aan haar man en}{zoon en vriendinnen}\\

\haiku{Tusschen haar in de.}{fijne ranke figuur van}{I Madinra}\\

\haiku{Niemand onzer kon.}{naar de heilige bron gaan}{om raad te vragen}\\

\haiku{Zij zag neer in het,:}{dal en de vreemdeling haar}{blik volgend sprak zacht}\\

\haiku{{\textquoteright} I Madinra.}{legde haar hand op het naar}{haar geheven boek}\\

\haiku{Nu rees ook Soe Ere.}{van den grond en hij beklom}{de trap der Baroega}\\

\haiku{Als roze zijde;}{glansde het tandvleesch der}{heilige dieren}\\

\haiku{Toen daalde Ali Ri,.}{Ajat Sjah de trap af gevolgd}{door zijn rijksgroeten}\\

\haiku{midden in de tot.}{een bergje gestapelde}{korrels stak een brief}\\

\haiku{Hollander - doch ook -.}{gebruikt voor menschen wit van}{vel dus uit blank ras}\\

\subsection{Uit: Onderworpenen}

\haiku{Enfin, bewaar die.}{djimats en neem den kerel}{mee naar het bivak}\\

\haiku{Maar de hand van den.}{blanke borg ze weer op in}{het grauwe papier}\\

\haiku{Iederen avond sloeg {\textquoteleft}{\textquoteright}.}{hij het uur des gebeds op}{den houtenbedoek}\\

\haiku{- {\textquoteleft}Waarom ben ik ook,?}{niet doorgeloopen waarom}{liet ik hem alleen}\\

\haiku{Daarom offerde;}{nu de oudste der koelies}{ook voor hen allen}\\

\haiku{Neen, neen, loopen maar,,,;}{ze was moe ja misschien viel}{ze wel dood neer straks}\\

\haiku{Ik merk wel dat je.}{toch eigenlijk nog te klein}{bent om te werken}\\

\haiku{en ik liep maar door,,.}{langs het bamboeboschje rechts}{af op de poort links}\\

\haiku{Er was iets in de. '}{sfeer van dezen man dat de}{beesten kalm maakte}\\

\haiku{De kettinggangers.}{spreken nooit van hun straftijd}{maar van hun diensttijd}\\

\haiku{'t Was of het op {\textquoteleft}!}{zijn gezicht stondWat komt er}{opaan wat ik aan heb}\\

\haiku{En wij verlieten...........................}{het huis en den tuin en het}{prachtige bergland}\\

\haiku{Wirio kon best het;}{toezicht op de beesten aan}{haar toevertrouwen}\\

\haiku{\'e\'ens had hij zijn,.}{jas al verkocht maar dat was}{nog niet het ergste}\\

\haiku{ze was te moe om,;}{nog harder te huilen ze}{snikte maar zacht door}\\

\section{Elisabeth Zernike}

\subsection{Uit: Bevrijding uit de jeugd}

\haiku{We doen alles zo;}{zuinig mogelijk om te}{sparen voor het huis}\\

\haiku{Titia sloeg de ogen.}{neer voor de uitdrukking van}{haar moeders gezicht}\\

\haiku{Bel nu maar voor de - -?}{koffie en ik heb taartjes}{had je dat gedacht}\\

\haiku{- een ongetrouwde,}{man is dat al genoeg om}{je de oren te doen}\\

\haiku{- Ik zou naar Parijs,, - -?}{willen ik heb mijn pas al}{aangevraagd maar Maar}\\

\haiku{U moet weten dat;}{mijn vader wijnhandelaar}{is in Chartres}\\

\haiku{Die vraag, zo dikwijls,.}{in haar opgekomen had}{ze nooit beantwoord}\\

\haiku{Kort voor de oorlog.}{was er die kleine Fransman}{geweest in Montreux}\\

\haiku{Nu belde ze en, -.}{keek de straat af vlak en recht}{het saaiste van Holland}\\

\haiku{O, ik mag zulke,.}{dingen niet zeggen jij bent}{nog zo maagdelijk}\\

\haiku{- Je weet niet hoe ik.}{je bewonder om wat je}{zoudt kunnen worden}\\

\haiku{Later heb ik daar -.}{hinder van gehad maar toen}{kon ik niet anders}\\

\haiku{- Madame, zei hij, -?}{tegen Titia hebt u een}{goede reis gehad}\\

\haiku{Ze wilde stellig,.}{het huis zien ze hoopte te}{kunnen meewerken}\\

\haiku{Hier en daar zag ze,.}{hoge zonnebloemen en}{een enkele boom}\\

\haiku{- Och, dat wist ze niet,,.}{maar zolang je leefde stond}{je bloot aan de tijd}\\

\haiku{- Ja, zei Titia, - en.}{je zoudt er je leven mee}{kunnen verknoeien}\\

\haiku{Die woorden kwamen.}{haar onvoorbedacht en ze}{zag Arthur's ontroering}\\

\haiku{Toen sloeg Arthur een arm.}{om haar schouders en kuste}{voorzichtig haar wang}\\

\haiku{Misschien - niemand had,.... -,.}{het haar gezegd maar Je blijft}{natuurlijk zei Jo}\\

\haiku{De begrafenis.}{kan niet worden uitgesteld}{in deze hitte}\\

\haiku{- Ik betuig u mijn,.}{leedwezen had madame}{Ch\'em\`ene gezegd}\\

\haiku{Vijf kinderen op, -.}{een dorp en een doktershuis}{het kon niet mooier}\\

\haiku{Wat zij zou doen was, -,.}{stof afnemen een man zag}{dat niet maar zij wel}\\

\haiku{Ze praatten daar even.}{op door en ze vertelde}{iets over haar ouders}\\

\haiku{Ze liepen het huis.}{weer in en mevrouw Honnes}{wenkte de meisjes}\\

\haiku{- Maar juffer, gekapt -.}{en gekleed en ik dorst u}{geen ontbijt brengen}\\

\haiku{Zonder groet draaide.}{ze zich om en liep langzaam}{naar de stad terug}\\

\haiku{- Neen, zei hij, ik heb, -.}{mij laten kiezen en gij}{steunt mijn zwak verstand}\\

\haiku{Keetje voelde een.}{vast verzet in zich stijgen}{en antwoordde niet}\\

\haiku{Op de terugweg,,.}{vroeg in de middag toonde}{van Blom zich spraakzaam}\\

\haiku{- Hm, zei van Blom - een -.}{blanchisseuse en noemde}{andere namen}\\

\haiku{Hij hief de hand op,,.}{met het glas een dronk gewijd}{aan de vrouw dacht hij}\\

\haiku{De oude vrouw keek:}{naar de vonkenregen en}{antwoordde langzaam}\\

\haiku{Op zijn dertigste;}{jaar is mijn vader tot hoofd}{van een school benoemd}\\

\haiku{Mijn vader had de:}{schouders opgehaald over mijn}{eerste rapporten}\\

\haiku{- Zou je liever je?}{huiswerk beneden maken}{met de anderen}\\

\haiku{Moeder had gelijk,.}{ik wist plotseling dat ik}{vaak zat te dromen}\\

\haiku{- Mijn vader heeft een,,;}{Fantin-Latour gekocht stel}{je voor een echte}\\

\haiku{- en geloof dat ik,.}{er evenzeer voor vreesde als}{ernaar verlangde}\\

\haiku{- Ik denk niet dat ik:}{zal studeren en hij trok}{de wenkbrauwen op}\\

\haiku{- En dan zonder de,?}{rozenperkjes alsof het}{het land van Dothan was}\\

\haiku{- maar ik kende toen:}{Jenny's tedere blik al}{goed en bovendien}\\

\haiku{het publiek had niets,.}{gemerkt hij rekende op}{mijn geheimhouding}\\

\haiku{- U moet weggaan, zei,.}{ik alleen haar ouders zal}{ik niet weigeren}\\

\haiku{- Kom, hij had er meer, -.}{zien vallen ze stonden wel}{weer op maar hij ging}\\

\subsection{Uit: Bruidstijd}

\haiku{Ze antwoordde niet,.}{onmiddellijk maar begon}{langzaam te blozen}\\

\haiku{Iets van die vreemde.}{drift was ook in haar en ze}{haalde schokkend adem}\\

\haiku{- Neen, dat heb ik tot,.}{nog toe gezegd maar nu ga}{ik veranderen}\\

\haiku{Dicht naast haar lag Nel,.}{die zeker gewend was aan}{zulke geluiden}\\

\haiku{Ze keek langs zichzelf,.}{omlaag de kimono hing}{tot op haar voeten}\\

\haiku{En dit zijn onze,,.}{jongens de blonde is Bas}{en de ander Henk}\\

\haiku{De jongens lachten,.}{luidkeels zodat het verhaal}{werd onderbroken}\\

\haiku{Ze bloosde bijna,.}{door de gedachte aan die}{flater en stond op}\\

\haiku{- Dag kind, had mevrouw, -.}{Moro gezegd we hopen}{je eens vaak te zien}\\

\haiku{De salon - rustig,.}{in het late licht alle}{vensters gesloten}\\

\haiku{Het was hem of hij.}{groeide en onbewogen}{neerkeek op zijn broers}\\

\haiku{- Als u haar weer ziet,, -:}{noem haar dan Nel dat is ik}{zou willen zeggen}\\

\haiku{- Het is mooi voor u,,...?}{dat begrijp ik wel het is}{had u het verwacht}\\

\haiku{En werd verontrust,.}{door dat laatste woord dat in}{hem bleef naklinken}\\

\haiku{- Zo, zei Bertha, die een,?}{pan schudde zal ik jou ook}{eens wat vertellen}\\

\haiku{Toen hij weg was, had.}{Ina een glans in haar ogen van}{verwachting en trouw}\\

\haiku{Was het niet goed wat, -?}{ze wilde omdat Doortje}{grauwe handen had}\\

\haiku{Eline tuurde in;}{het zachte schemerlicht en}{hoorde de stilte}\\

\haiku{De staande houding.}{hinderde hem en hij kon}{niet naar Ina kijken}\\

\haiku{Werd het een koop, h\'a\'ar?}{durf en wereldwijsheid voor}{zijn geld en titel}\\

\haiku{Hij zag het verband,:}{om haar vinger dacht aan de}{woorden van zijn toast}\\

\haiku{- Och, zei ze en bleef,.}{achter zijn stoel staan het is}{niet zo belangrijk}\\

\haiku{Een voorbijganger.}{keek hem aan en hij hief het}{hoofd nog wat meer op}\\

\haiku{Ze wees hem op een,.}{mand met herfstasters lila}{met een hart van geel}\\

\haiku{Ze moest weten wie,,.}{ik ben dacht hij toen ze de}{deur voor hem open hield}\\

\haiku{Mijn behanger moet.}{een voorbeeld worden van het}{gedeeltelijke}\\

\haiku{Zijn belangstelling,?}{prikkelde haar waarom sprak}{hij niet over zichzelf}\\

\haiku{Zelfs in den tijd van.}{Leo Frankenvoort had het kind}{niet zo goed gespeeld}\\

\haiku{Nog hoorde ze haar,.}{moeders stembuiging maar dat}{was nu de hare}\\

\haiku{- Maar je bent zelf nog,,.}{niet oud zei Eline en je}{moest kleuren dragen}\\

\haiku{Terwijl ze de trap,.}{afliepen voelde Ina zich}{onrustig worden}\\

\haiku{ik eiste van haar,.}{dat ze me zou volgen in}{mijn nieuwen werkkring}\\

\haiku{- Ik weet nu, dat dat -,.}{verkeerd van me was ik mag}{vragen niet eisen}\\

\haiku{Jou wil ik vragen,...}{of het je niet hinderlijk}{is wanneer ik tracht}\\

\haiku{- Neen, zei Bertha, je bent,.}{groter dan ik dat is me}{ongemakkelijk}\\

\haiku{Je zult zien, het zijn,.}{goede dwergjes ze lachen}{als ik binnenkom}\\

\haiku{Dan heb ik nog zo'n,'.}{ouwe tante die loopt met}{krante langs de straat}\\

\haiku{- Je hoeft niet mee, hoor,,.}{overal is een hoek waarop}{je kunt omdraaien}\\

\haiku{- Dan heb je nog wel,, -.}{een ogenblik zei Ina maar haar}{trekken bleven star}\\

\haiku{- Gisteravond heb je,:}{naar me gekeken alsof}{je zeggen wilde}\\

\haiku{Rika kondigde.}{het bezoek aan en sloot stil}{de deur achter zich}\\

\haiku{En nog meer had hij,.}{gezegd natuurlijk ook iets}{over geheimhouding}\\

\haiku{De eetkamer van.}{de familie Moro lag}{in het souterrain}\\

\haiku{Het is de vraag wat,,:}{wij zijn had Bas gezegd en}{Nel had verbeterd}\\

\haiku{- Ik kan niet anders,,}{klonk het een beetje schor het}{eigen innerlijk}\\

\haiku{De mens is de kroon -.}{der schepping ze stootte een}{schamper lachje uit}\\

\haiku{- en het is een maand.}{geleden dat ik het me}{heb voorgenomen}\\

\haiku{- Nou, als jelie me, -.}{nodig hebt roep je maar het}{zal tevergeefs zijn}\\

\haiku{Wij stonden stijf door.}{onze dikke kleren en}{hadden het toch koud}\\

\haiku{- Ik vind u lief, zei -,.}{Nel ik geloof niet dat veel}{moeders zo praten}\\

\haiku{- Maar de brief dien ze,.}{hem erover had geschreven}{was niet beantwoord}\\

\haiku{- Ben ik dan niet... - Sst,,.}{zei hij zulke grove taal}{mag ik niet horen}\\

\haiku{Ze keek hem aan en,.}{glimlachte om den geest die}{uit zijn trekken sprak}\\

\haiku{- We zullen goed doen.}{niet te kopen voor we het}{eens zijn geworden}\\

\haiku{- Het zou wel goed zijn,,.}{meende hij nu toch maar naar}{zijn zuster te gaan}\\

\haiku{Heleen veegde haar,.}{hand af langs haar heup voor ze}{die naar hem uitstak}\\

\haiku{Ze nam het laatste,.}{stuk brood dat voor haar lag en}{at het langzaam op}\\

\haiku{Papa was rijk - maar,.}{hij ging weg en zij zou bij}{de Moro's wonen}\\

\haiku{- Hier moeten  we, -,.}{oversteken ik geloof dat}{ik u moet leiden}\\

\haiku{Nu moest er iemand.}{komen die licht maakte en}{dezen dag aandorst}\\

\haiku{Als dit kon - als   -.}{hij haar hielp dan zou ze nog}{willen leven}\\

\haiku{Maar vijf minuten.}{later stond hij op en liep}{naar de badkamer}\\

\haiku{Het brede en toch,,.}{tere voorhoofd beheerste met}{de ogen het gezicht}\\

\haiku{ik wist wel, had zijn,.}{moeder gezegd dat hij eens}{een man zou worden}\\

\haiku{- Ja, zei Doortje - ik,.}{ben maar zo bang dat u zich}{eenzaam zult voelen}\\

\subsection{Uit: De erfenis}

\haiku{Maar je erft van hem,, -.}{je komt er bovenop ik}{ben er heel blij om}\\

\haiku{Hij was een rijke,.}{industrieel maar had haar}{nooit iets gegeven}\\

\haiku{Hoe armer ze werd,.}{hoe onrustiger en meer}{geneigd tot zwerven}\\

\haiku{ze liep langs het strand,,.}{de zeewind was zilt en guur}{de golven raasden}\\

\haiku{Ik ben maar bij je.}{binnen gekomen en neem}{je tijd in beslag}\\

\haiku{Ze is een beetje.}{ouder dan ik en was al}{eens verloofd geweest}\\

\haiku{Wat denk je dat we?}{van Ing bij ons huwelijk}{hebben gekregen}\\

\haiku{- Dat is wel goed, ze.}{mogen niet langer duren}{dan de verloving}\\

\haiku{Als kind heb ik een;}{dierenatlas gehad met}{originele text}\\

\haiku{George lachte.}{even en keek Bart van Weeze}{met schuinsen blik aan}\\

\haiku{Het dienstmeisje kwam;}{binnen met ingeschonken}{thee op een groot blad}\\

\haiku{Ze merkte dat zijn.}{aandacht alweer elders was}{en  trad terug}\\

\haiku{Toen Dora in den,.}{trein naar huis zat voelde ze}{haar grote moeheid}\\

\haiku{- Kinderlijk, hernam,.}{Truus door je vertrouwen en}{je halsstarrigheid}\\

\haiku{- Je weet, zei Truus, dat?}{ik mijn neef Huib den laatsten}{tijd weer heb ontmoet}\\

\haiku{Hij is verlegen,,.}{dacht ze omdat ik hem in}{zijn gezin betrap}\\

\haiku{- Djja, zei Nubeling,,;}{langgerekt lust tot knokken}{om het verzetje}\\

\haiku{- En het verdere?}{van de omstandigheden}{laten afhangen}\\

\haiku{- Zevenhonderdtwintig,,.}{gulden per jaar zei Viers om}{niet te kankeren}\\

\haiku{Maar dreggen is een -.}{beroerd werk je wordt er nat}{van tot in je ziel}\\

\haiku{De jongens praten, -;}{over den hengst en de merrie}{dat klinkt mannelijk}\\

\haiku{- Maar als ik daarbij,,.}{aanknoop zei George zal}{het zo lang worden}\\

\haiku{Zijn huis beviel me,.}{natuurlijk niet dus ging ik}{zelf iets ontwerpen}\\

\haiku{Het is gelukt, en:}{tegelijkertijd met een}{knauw van ellende}\\

\haiku{Ze stonden allen.}{op en weer gaf het kristal}{zijn helderen toon}\\

\subsection{Uit: De gast}

\haiku{Dan werkte hij weer.}{tot het avondeten en moeder}{boog over het fornuis}\\

\haiku{{\textquoteleft}Om mij bij te staan{\textquoteright} -.}{die woorden deden Anna}{het hoofd opheffen}\\

\haiku{Werner is hun trots,.}{maar ze laten hem gaan en}{roepen mij terug}\\

\haiku{heeft uw dorp geen plaats,?}{voor de kinderen die er}{worden geboren}\\

\haiku{Kort daarop had ze,.}{partij gekozen v\'o\'or haar}{ouders ze moest wel}\\

\haiku{In de lente lag,;}{de sneeuw voor het grijpen dan}{was er geen kunst aan}\\

\haiku{Ze liep de trap op,,.}{wilde roepen maar er kwam}{geen klank uit haar keel}\\

\haiku{Ze waren nog jong.}{en moesten hun best doen vader}{niet te vergeten}\\

\haiku{Ook van dit gesprek.}{wist ze later niet meer hoe}{het was verloopen}\\

\haiku{Langzaam liepen ze,.}{omhoog de zeere plek op}{haar schouder schrijnde}\\

\haiku{De avondlucht was koel,.}{aan haar voorhoofd de druk op}{haar slapen nam af}\\

\haiku{Er stond een stoel voor,.}{hem klaar hij trok zorgvuldig}{zijn broekspijpen op}\\

\haiku{Mevrouw schonk thee en.}{begon een gesprek over de}{schoonheid van het land}\\

\haiku{- Dat weet ik nog niet,,.}{we vergaderen er over}{we zijn vrije Zwitsers}\\

\haiku{Van de Pressalle's,.}{had ze geleerd dat het op}{de lijnen aankwam}\\

\haiku{Soms viel er wat sneeuw,;}{van een tak geruchtloos of}{met een zachten plof}\\

\haiku{Het bloed trok weg uit, -.}{haar hoofd er kwam een gevoel}{van leegte in haar}\\

\haiku{- Maar hij was naast haar.}{gekomen en ze zag den}{ernst in zijn gezicht}\\

\haiku{Het meisje voelde.}{een lang ontbeerde blijdschap}{en stak haar hand uit}\\

\haiku{Dat {\textquoteleft}Montana{\textquoteright}?}{met zijn moderne comfort}{moet ik toch eens zien}\\

\haiku{De deuren van het,.}{huis stonden open het was een}{warme voorjaarsdag}\\

\haiku{Je bent naar het huis, -,?}{wezen kijken het is veel}{verbeterd nietwaar}\\

\haiku{- en het verwacht zijn -,.}{gasten of jij er ooit zult}{wonen weet ik niet}\\

\haiku{Ze keek den weg af, -.}{naar omhoog en omlaag ze}{zag niemand buiten}\\

\haiku{Vroeger heb ik met - -.}{je mogen spelen nou dan}{en nu is het ernst}\\

\haiku{Dit keer hield Anna.}{voet bij stuk en er kwam iets}{dwingends in haar oogen}\\

\haiku{Anna voelde zich,.}{blozen ze wist hoe druk het}{in de keuken was}\\

\haiku{- Neen, hij bestudeert,.}{het dansen daar moet hij ook}{eens examen in doen}\\

\haiku{De geit mekte en,.}{liep haar tegemoet toen ze}{den stal binnenging}\\

\haiku{Na een poos dacht ze,.}{aan de kinderen die haar}{niet hadden gestoord}\\

\haiku{Anna deed haar best.}{zich niet door hun woorden te}{laten afleiden}\\

\haiku{het huis, met koelen -.}{ingang de tranen sprongen}{Anna in de oogen}\\

\haiku{Heinz was op de stoep,:}{gaan zitten Treesje stond op}{den drempel en riep}\\

\haiku{En dan wordt hij nog,, -.}{eens professor oma ik hoop}{dat je het beleeft}\\

\haiku{Nu liep Willy hem,.}{te zoeken ze wist niet of}{hij daar goed aan deed}\\

\haiku{Waar praten wij over, -.}{dacht Anna oude menschen}{zijn niet gelukkig}\\

\haiku{Kleur is ook iets, en,.}{gelukkig schijnt de maan niet}{altijd vrouw Anna}\\

\haiku{Zij knielt in mijn plaats - -,?}{of ik ben het zelf lijkt ze}{niet op mij Anna}\\

\haiku{- Dat is niet waar, zei,.}{ze maar wachtte gespannen}{op zijn wederwoord}\\

\haiku{En nog veel meer zei,.}{ze en dacht met flitsen aan}{het verstoorde werk}\\

\haiku{Vrouw Bassmann was;}{naast het fornuis gaan zitten}{en keek om zich heen}\\

\haiku{ze waren al met,.}{zijn tienen er kon geen stoel}{meer bij de tafel}\\

\haiku{- Dus, zei Willy, jij;}{loopt vanmiddag naar het dorp}{en noodigt moeder uit}\\

\haiku{Door de sneeuw was de, -.}{aarde niet donker alleen}{het licht was schaarsch}\\

\haiku{met een enkelen.}{oogopslag had ze gezien}{dat er geen ontbrak}\\

\haiku{- En de engel had,.}{zoon mooie stem zei het meisje}{dat naast Werner liep}\\

\haiku{Maria schudde het,,.}{hoofd haar neusvleugels spanden}{zich maar ze sprak niet}\\

\haiku{- Morgen ga ik naar, -.}{huis terug ging ze voort ik}{ken dit leven niet}\\

\haiku{Ge\"ente rozen, - -,.}{ja zei ze deze niet die}{op eigen stam staan}\\

\haiku{Twee dagen na den.}{droom zag ze hem met Willy}{op de bank zitten}\\

\haiku{Als ze nu maar kon.}{bedaren en denken aan}{dat wat ze wilde}\\

\haiku{dat haar verlangen,.}{niets kon winnen want dat hij}{haar zou verjagen}\\

\haiku{- En heen en terug,.}{zonder oponthoud zei de}{kok met strenge oogen}\\

\haiku{Hij legde zijn hand.}{om haar pols en telde de}{slagen van het hart}\\

\haiku{Hij nam Roosje op;}{zijn knie en las haar voor uit}{een prentenboekje}\\

\haiku{Wat later kon ze;}{duidelijk de dorpen in}{het dal zien liggen}\\

\haiku{Met onvaste stem.}{vroeg ze naar moeder Tresa}{en de kinderen}\\

\haiku{Tweemaal hield ze een -!}{praatje kon ze daarmee den}{tijd maar doen stilstaan}\\

\haiku{Nu kwam er een glans.}{in haar oogen en een guitig}{trekje om haar mond}\\

\haiku{- Vraag mij liever wat -.}{en hoorde een nadenkend}{zoemen in zijn keel}\\

\haiku{Zie je, het heeft me.}{gegrepen en ik heb het}{van me afgeschud}\\

\haiku{Ze draaide zich naar,.}{hem toe maar verborg haar hoofd}{en snikte weer}\\

\haiku{- hoe kwam ze er ook,?}{toe het graf van oma Tresa}{te willen sieren}\\

\haiku{ze was geweest toen}{ze vader verloor en wat}{zich in haar prentte}\\

\haiku{een forsche, bijna,.}{onaandoenlijke vrouw die}{het leven kende}\\

\haiku{Eigenlijk dacht ze, -.}{daar zelden meer aan het was}{zes jaar geleden}\\

\haiku{Een minuut of wat,.}{hoorde Anna hem aan toen}{stak ze haar hand uit}\\

\haiku{Elsi, het oudste,.}{dochtertje van Martha stond}{achter de toonbank}\\

\haiku{Dan riep de vraag naar.}{een volgend artikel haar}{aandacht weer terug}\\

\haiku{Gaan jelie toch naar,.}{bed ik wil eindelijk met}{vader alleen zijn}\\

\haiku{Het eerste gezin,,.}{dat ze had bezocht woonde}{aan een kweekerij}\\

\haiku{ieder bleef in de.}{duister-lichte tent van}{zijn ervaringen}\\

\haiku{Als hij omhoog liep?}{en die zemelen pop de}{waarheid vertelde}\\

\haiku{Hij had haar in lang,,}{niet gezien ze kwam niet meer}{naar omlaag maar \~als}\\

\haiku{Vreemd - zes slokken uit -.}{een glas van zeven en het}{glas was aan zijn mond}\\

\haiku{- Adieu, de Moeder -,.}{Gods bescherme u. Adieu}{prevelde Joseph}\\

\haiku{Vader Joseph sloeg;}{zijn oogen niet op terwijl hij}{kopjes waschte}\\

\subsection{Uit: De gereede glimlach}

\haiku{Het huisje waarbij,,:}{de gouden regen bloeide}{zoo laat nog heette}\\

\haiku{En de stroomende.}{regen zou heel kil zijn aan}{hun warme lijfjes}\\

\haiku{- Zie zoo jongens, nu,.}{gaan we dwars door de hei en}{wie het eerst er is}\\

\haiku{Altijd nog - zooals ze,,.}{hier lag een kind ontvangen}{met heel haar lichaam}\\

\haiku{Na een week zat hij.}{met de kinderen op het}{verandatrapje}\\

\haiku{Dan sloeg ze soms even,:}{haar handen voor haar gezicht}{verward fluisterend}\\

\haiku{Zijn onbewuste.}{eenvoud had het nieuwe in}{haar leven gebracht}\\

\haiku{Agnes wilde het,.}{onbelangrijke weten}{het bijkomstige}\\

\haiku{Bovendien heb ik.}{je laatste verloofde nog}{juist even meegemaakt}\\

\haiku{Ze zag hem langzaam,.}{zoeken rondom het huis dat}{ze gesloten had}\\

\haiku{Gisteren, toen ze,;}{naast zijn paard liep was ze toch}{jong en sterk geweest}\\

\haiku{Hoeveel dieper dan,.}{gisteren was alles nu}{hoeveel droeviger}\\

\haiku{- Ja - zei ze - in een,.}{fluister en ze boog haar hoofd}{voor het groote leven}\\

\haiku{Hij nam haar hand en,,.}{ze liepen weg snel en zacht}{met wijde stappen}\\

\haiku{Hij vertelde hoe,.}{zijn leven was geweest wat}{hij geleden had}\\

\haiku{Alleen in den nacht.}{heb ik soms smadelijk haar}{lichaam genomen}\\

\haiku{Korten tijd stond de,.}{maan dicht aan den horizont}{rood en gezwollen}\\

\haiku{De deur werd wijd voor:}{haar geopend en de man}{voegde er nog bij}\\

\haiku{III Op een avond - ze, -.}{waren toen een week bij den}{boer bleef Petja weg}\\

\haiku{En in mijn oogen ben,}{je het ook want ik heb zelf}{gezien wat je deed}\\

\haiku{- Een beetje loom, ik,.}{weet niet waarvan misschien door}{deze rust alleen}\\

\haiku{- Ik houd zooveel van,,.}{hem zei ze doorgaand op haar}{eigen gedachten}\\

\haiku{En toen zag ze weer,.}{Hanna voor zich de vrouw die}{haar vader liefhad}\\

\haiku{In ieder geval,.}{zie ik uw onrust en ik}{heb er verdriet van}\\

\haiku{Maar ik denk aan een,,.}{heel groote groene vaas waarin}{goudvisschen zwemmen}\\

\haiku{Waarom hij nu nog,?}{alle onrust verkropte}{terwijl zij toch wist}\\

\haiku{Op een middag kwam.}{haar vader de tuinkamer}{binnen waar zij zat}\\

\haiku{later misschien, dan,.}{komt het ongemerkt maar haar}{houding verstilde}\\

\haiku{Het plafondlicht in.}{haar kamer viel zonder schroom}{op alle dingen}\\

\haiku{Denk jij ook maar aan,,.}{je eigen leven meisje}{ich rate dir gut}\\

\haiku{En ze had zich nooit,.}{afgevraagd hoe er over haar}{gepraat zou worden}\\

\haiku{Wat gebeurde er,?}{dan in dit huis moest Corrie}{er vrij spel hebben}\\

\haiku{Hij liet haar polsen,.}{los en haar armen vielen}{slap langs haar lichaam}\\

\haiku{Hij verwonderde.}{zich over het snelle flitsen}{van haar gedachten}\\

\haiku{hoe hij dat haar van,.}{haar voorhoofd streek terwijl zijn}{lippen bewogen}\\

\haiku{Nu voelde ze zich,.}{bijna als een bruid die haar}{kamer binnen gaat}\\

\haiku{Ze had Wim beloofd,.}{eens naar zijn kinderen te}{komen kijken}\\

\haiku{Je bent wat jonger,,.}{dan ik maar je hebt geleefd}{en je sleept me mee}\\

\haiku{Al haar aandacht had,.}{ze noodig voor haar onbewust}{zeker handelen}\\

\haiku{Toen ze haar oogen sloot,.}{zag ze de wereld buiten}{zich heel ver en klein}\\

\haiku{Ik hoorde u naar,.}{boven gaan en heb thee voor}{u ingeschonken}\\

\haiku{Ze vermeden het,,.}{elkaar aan te zien gingen}{alle twee zitten}\\

\haiku{De zon scheen over het,.}{oude groen van de boomen}{warm en koesterend}\\

\haiku{Ze zweefde al in, -.}{de ruimte een haan kraaide}{den hemel open}\\

\haiku{Ze had haar mantel,.}{aan de kapstok gehangen}{en liep naar boven}\\

\haiku{- want zou ze later,?}{w\'e\'er haar huis kunnen uitgaan}{en hier terugkeeren}\\

\haiku{als ze gedurfd had,.}{zou ze den vader haar rug}{hebben toegekeerd}\\

\haiku{ze zag hun smalle,,;}{vermoeide gezichten schuw}{blikkend naar elkaar}\\

\haiku{Hun geest was bezig.}{met wat achter de bochten}{zou zichtbaar worden}\\

\haiku{Toen glimlachte ze,.}{en streek met haar hand over de}{ruige reisdeken}\\

\haiku{Misschien zou dat ook, -.}{gauw voorbij zijn hij wist niet}{of hij het hoopte}\\

\haiku{- om tot die hoogten,?}{te stijgen waar Elien hem niet}{had kunnen volgen}\\

\haiku{- Kleine Annie kwam.}{binnen met een springtouw in}{haar dikke vuistjes}\\

\haiku{- Dan zal ik u nog,.}{een kopje inschenken en}{we gaan vroeg naar bed}\\

\haiku{Mijn oudste jongen,.}{heeft examen gedaan een paar}{dagen geleden}\\

\haiku{Zoo moet je niet doen,, -:}{dat is niet goed terwijl ze}{vroeger gezegd had}\\

\haiku{Ze zou dus naar Aga, -.}{kunnen gaan maar ze had het}{zichzelf verboden}\\

\haiku{- en toen zuchtte ze.}{gelaten en wendde haar}{blik af van de klok}\\

\haiku{maar ze praatte niet,,,.}{veel meest luisterde ze en}{keek vooral naar Aga}\\

\haiku{Hij keek haar dan soms,,.}{ook aan met een heel vagen}{glimlach meende ze}\\

\haiku{waarvoor heeft mijn broer,?}{een candidaat als hij er}{niet tusschen uit kan}\\

\haiku{- Rinus, riep mevrouw,?}{Van Waveren waarom ben}{je zoo bescheiden}\\

\haiku{- Al die poespas, had,,:}{Te Weichel gemompeld en}{toen tegen Weversma}\\

\haiku{Och, nonsens, goed eten,,.}{zooals haar vader gedaan had}{daaraan hield ze vast}\\

\haiku{Ze wendde zich snel,, -?}{naar hem om haar oogen blonken}{in haar lach O ja}\\

\haiku{- Zitten blijven en,,}{mond houden riep Aga maar Van}{Waveren stond al.}\\

\haiku{hij merkte het niet,,,.}{en ze bleef kijken bewust}{nu en nog scherper}\\

\haiku{- Kind, ik vind het zoo,?}{heerlijk om te gaan kan je}{je dat begrijpen}\\

\haiku{Hij kon op Zondag;}{een uur langer slapen dan}{op weeksche dagen}\\

\haiku{- Mijn liefde voor haar,.}{zal zeker invloed op me}{hebben zei ze snel}\\

\haiku{Toen ze na schooltijd,.}{was thuis gekomen klopte}{Ietje bij haar aan}\\

\haiku{Terwijl ze Weversma,.}{een hand gaf voelde ze een}{verwarring in zich}\\

\haiku{- Hij praatte zoo graag, -.}{over onze liefde maar hij}{was achterdochtig}\\

\haiku{- Ik zou je verstoot,,.}{en zei hij als ik dacht dat}{je me niet lief had}\\

\haiku{- Na de inleiding;}{begon het verhaal dat ze}{las haar te boeien}\\

\haiku{- Wilde ze hem het,?}{genoegen doen een kop thee}{met hem te drinken}\\

\haiku{Ik zou ook heel graag, -.}{iets doen we moeten er nog}{maar eens over spreken}\\

\haiku{Dan huren we daar, -.}{een groot huis ik ben dol op}{een verandering}\\

\haiku{Zij had het niet zien,.}{aankomen zij was trouw en}{argeloos geweest}\\

\haiku{Ze voelde tranen,.}{naar haar oogen komen maar drong}{ze met kracht terug}\\

\haiku{bestuurslid van een.}{vereeniging tot steun aan}{gevallen meisjes}\\

\haiku{- Ik ben blij dat je,,.}{terug bent iedereen vraagt}{naar je Anton ook}\\

\haiku{- Dat weet ik niet, - ze -... -,.}{schrijft dat e Jawel dat ze}{nu al van je houdt}\\

\haiku{- En ik vergeleek,.}{haar met een grauw nachtuiltje}{dat om de lamp vloog}\\

\subsection{Uit: Het harde paradijs}

\haiku{In de slaapkamer;}{van de ouders stond ook het}{bed van de dochter}\\

\haiku{Het betekende,.}{verraad en ze had tot de}{keurbende behoord}\\

\haiku{Hij wilde trouwen,.}{er was een meisje daarginds}{dat op hem wachtte}\\

\haiku{{\textquoteright} Marguerite,.}{zag reprodukties die haar}{niet vreemd voorkwamen}\\

\haiku{Het had beter nog,,}{wat kunnen wachten en dat}{denken we gebukt}\\

\haiku{Nauwelijks, zult u,.}{zeggen maar we leven in}{een duistere tijd}\\

\haiku{Daar leek geen einde,.}{aan te kunnen komen en}{toch ging het voorbij}\\

\haiku{U hebt uw oude,}{schoolvriendin nog niet bezocht}{zij vraagt mij naar u}\\

\haiku{{\textquoteleft}Ze zal die hebben -.}{verspeeld alsof het leven}{een loterij was}\\

\haiku{Rite, die zich had,.}{opgericht zag zijn donker}{behaarde armen}\\

\haiku{De gevoelige,,.}{neusgaten van zo'n dier ach}{en de zachte oren}\\

\haiku{Niet begrijpend keek.}{ze ernaar en voelde haar}{gezicht verstrakken}\\

\haiku{Ze heeft een moeilijk.}{leven gehad of althans}{een moeilijke tijd}\\

\haiku{Je hebt wat geld op,.}{de Bank staan het zal ieder}{jaar vermeerderen}\\

\haiku{Toen Pierre zo'n jaar,:}{of veertien was geworden}{had hij eens gevraagd}\\

\haiku{Ze hoorde een stem.}{die de jongens verbood op}{dat hek te spelen}\\

\haiku{Pierrre heeft mij,,:}{toestemming gegeven dacht}{ze en bovendien}\\

\haiku{hoe goed kende ze.}{die geur en het zoevende}{geluid van de kwast}\\

\haiku{{\textquoteright} Driftig hief ze de.}{kwast op en weer spatten er}{druppels op de grond}\\

\haiku{Hoeveel inwoners,{\textquoteright}, {\textquoteleft}?}{heeft ons dorp vroeg zezouden}{het er duizend zijn}\\

\haiku{{\textquoteright} Marie keek haar aan,.}{en knipoogde haar grote}{mond trok nog wijder}\\

\haiku{{\textquoteright} Vaak dacht Rite aan;}{die woorden terug bij het}{werk op haar velden}\\

\haiku{glimlachte met \'e\'en,.}{helft van zijn gezicht waardoor}{het nog schever werd}\\

\haiku{Ze heeft een man en.}{kinderen en is de trots}{van mijn ouders}\\

\haiku{Nog kan ik zien hoe,,.}{ze ermee in haar handen}{stond onthutst verschrikt}\\

\haiku{- Nee, ze wist het niet, -?}{was alles goed als ze dien}{man haar huis verkocht}\\

\haiku{Eenmaal hief ze de:}{armen in wanhoop boven}{het hoofd en zei luid}\\

\haiku{ze wist niet waar ze,?}{was zou ze de brug over de}{Rioux Blanc vinden}\\

\haiku{- Wat betekende,?}{daarnaast haar vaste loop}{heel haar lichaamskracht}\\

\haiku{Een zware, jonge,:}{vrouw met roodachtig geverfd}{haar zei minachtend}\\

\haiku{{\textquoteleft}ze zal morgen haar -}{geit nog wel buiten brengen}{in haar werkplunje}\\

\haiku{Hij had een kort, zwart.}{jasje aan en een zwart met}{grijs gestreepte broek}\\

\haiku{Ronde stenen van,.}{verschillende grootte je}{zult het alles zien}\\

\haiku{het moest nu gauw uit,...}{zijn vandaag de klucht van een}{huwelijk en dan}\\

\haiku{Meteen nam ze die.}{over de arm en liep terug}{naar de slaapkamer}\\

\haiku{- Ze hoefde niet de.}{pluk van de hele morgen}{in \'e\'en zak te doen}\\

\haiku{De hitte sloeg haar.}{eensklaps uit en ze begreep}{te willen braken}\\

\haiku{Het linker glas was,.}{een beetje beduimeld ze}{zou het schoon vegen}\\

\haiku{Ik zal met Pi\'emont, -?}{praten hij moet er iets aan}{doen hoe ver is het}\\

\haiku{En deze vrouw, die,.}{niet eens onknap was had een}{bultenaar getrouwd}\\

\haiku{Een getrouwde vrouw,.}{moet niet werken althans niet}{in een beschaafd land}\\

\haiku{- Ze was nog strakker,.}{van lijn geworden hij zoo}{haar portret makers}\\

\haiku{{\textquoteleft}We zullen gaan,{\textquoteright} zei, {\textquoteleft}.}{ze tegen Arnolfiheb}{een ogenblik geduld}\\

\haiku{hij sloeg zijn zwager - {\textquoteleft},?}{op de schoudervijf procent}{is veel weet je dat}\\

\haiku{{\textquoteright} {\textquoteleft}Hm,{\textquoteright} zei Pierre, {\textquoteleft}dat?}{geloof ik dadelijk en}{ben jij trots op hem}\\

\haiku{Pierre draaide het,,:}{om aan de andere kant}{stond half uitgewist}\\

\haiku{{\textquoteright} {\textquoteleft}Kijken of de trap,{\textquoteright}.}{ons houdt zei de aannemer}{en liep naar boven}\\

\haiku{Dat van Trude was -,.}{jaren geleden goed}{het bleef altijd waar}\\

\haiku{Hij bedrinkt zich aan,.}{grote woorden dacht ze en}{beet op de lippen}\\

\haiku{- Het is een meisje,, -:}{dacht ze ik zal een dochter}{hebben altijd weer}\\

\haiku{{\textquoteleft}Dank je nooit meet aan,?}{tante Suzanne die goed}{voor je is geweest}\\

\haiku{Zodra het voorjaar,,.}{werd kon ze buiten staan in}{de hoek bij de schuur}\\

\haiku{en ze stond stil voor.}{de trap om het gesprek nog}{te kunnen volgen}\\

\haiku{Haar adem hokte en,.}{ze keek naar Antoine die}{tegenover haar zat}\\

\haiku{Pas op, beeldspraak is,.}{gevaarlijk en de ziel wordt}{niet met goud gemest}\\

\haiku{Ze zag planken vol,.}{staan terwijl hij verklaarde}{niets meer te hebben}\\

\haiku{Als Rite wat meer,.}{thuis kon blijven zou er al}{veel zijn gewonnen}\\

\haiku{{\textquoteleft}En wat hebt u een,,,!}{prachtig kind dat mondje dat}{neusje die houding}\\

\haiku{hij keek niet naar hen.}{om en het kind taalde niet}{naar zijn gezelschap}\\

\haiku{- Iets anders, het mocht,... {\textquoteleft}?}{misschien wel aarde zijn maar}{niet zoNiet zo week}\\

\haiku{het bleek de zoon van,,.}{madame Rubinno}{haar overbuur te zijn}\\

\haiku{{\textquoteright} Opeens liep er een.}{traan over haar wang en vloeide}{tussen haar lippen}\\

\subsection{Uit: Kieren van de nacht}

\haiku{Dat ze plezier had -,.}{in die moord tja de hele}{buurt gnuifde erover}\\

\haiku{{\textquoteleft}Geef me een pakje,{\textquoteright}, {\textquoteleft} ',?}{uit de winkel zei zedat}{s goedkoper h\`e}\\

\haiku{Was er niet een grijns?}{van verstandhouding geweest}{tussen Koos en hem}\\

\haiku{Om acht uur had hij,.}{de lamp gedoofd ontbeten}{en afgewassen}\\

\haiku{Gisteravond had hij,;}{buiten gelopen wat hij}{weer zou willen doen}\\

\haiku{{\textquoteright} vroeg hij zonder veel.}{aandacht en keek om zich heen}{naar meer toehoorders}\\

\haiku{Zo stil mogelijk,.}{liep hij verder niemand scheen}{het op te merken}\\

\haiku{Hij volgde haar door,.}{de winkel de nauwe gang}{door naar de keuken}\\

\haiku{{\textquoteright} Hij wist niet goed wat.}{hij aan die woorden had en}{bleef haar aankijken}\\

\haiku{het kon de bomen,.}{in het bos niet vragen en}{zelfs de dieren niet}\\

\haiku{Ik heb wat laten,{\textquoteright},, {\textquoteleft}}{halen zei Koos de deurknop}{alweer in de hand}\\

\haiku{Hij zag haar blozen,.}{van inspanning het dienblad}{voor zich uit dragend}\\

\haiku{Op de voorspoed van, -,.}{de meiden daar hangt veel van}{af ook voor Berrie}\\

\haiku{Koos bleef nog liggen?}{en vroeg met slapende stem}{of hij nu al ging}\\

\haiku{in de uiterste.}{hoek van het plaatsje vormden}{zij een wijde kring}\\

\haiku{{\textquoteright} {\textquoteleft}Wat - e -{\textquoteright} {\textquoteleft}Dat ze me -;}{nu niet wil kennen ze is}{met zoveel vrienden}\\

\haiku{ik slinger wel een,.}{beetje maar daar moet je je}{niets van aantrekken}\\

\haiku{Hij schoof haar het geld.}{toe over de toonbank en ving}{een blik van haar op}\\

\haiku{{\textquoteright} Het meisje liet haar.}{hand langs zijn rug glijden en}{trok hem bij de arm}\\

\haiku{Zodra ze op straat,.}{stonden nam het meisje haar}{buit onder de arm}\\

\haiku{{\textquoteright} vroeg hij, {\textquoteleft}moet je niet,?}{terug misschien ligt er weer}{iemand in die schuit}\\

\haiku{Waarvan?{\textquoteright} {\textquoteleft}Tja - dat weet,{\textquoteright}:}{ik haast niet meer waarop ze}{smalend antwoordde}\\

\haiku{{\textquoteright} Wat schichtig keek hij,.}{naar Koos die tegenover hem}{aan de tafel zat}\\

\haiku{Hoe onzeker had,.}{hij zichzelf niet gevoeld die}{dagen zonder haar}\\

\haiku{{\textquoteleft}Zie, ik ben met u.}{al de dagen totaan het}{einde der wereld}\\

\haiku{Hij zag liever een,.}{grootboek op tafel dan De}{Strijdkreet zei Pelsen}\\

\haiku{Het doet er niet toe,,;}{kan je zeggen de natuur}{stoort er zich niet an}\\

\haiku{{\textquoteleft}Loop maar met me mee,,{\textquoteright}.}{dan krijg je wat soepvlees van}{me en hij stond op}\\

\haiku{de broodwinning, - wat -}{sta je daar weer houterig}{tussen je kissies}\\

\haiku{ze dacht dat hij het, -,.}{niet hoorde hij was niet gek}{hij hoorde alles}\\

\haiku{Van jou weet ze niets,{\textquoteright}, -?}{had Dirk gezegd wat viel er}{van haar te weten}\\

\haiku{Het hoofd van de een:}{naar de ander wendend ging}{de rechercheur voort}\\

\haiku{{\textquoteleft}Dat's niet eerlijk,.}{als ze het bij jou hebben}{gekocht weet je het}\\

\haiku{Het water was grauw,.}{de meeuwen vlogen traag en}{zonder te krijsen}\\

\haiku{Als Koos dan maar sliep.}{en niet vroeg waar hij zo lang}{was gebleven}\\

\haiku{Zo was Koos geweest;}{na de geboorte van haar}{eerste dochtertje}\\

\haiku{Natuurlijk waren -?}{er schepen ver weg zou hij}{nog willen varen}\\

\haiku{{\textquoteleft}Meid,{\textquoteright} zei Pelsen, {\textquoteleft}ik.}{wil van een bordje eten en}{uit een glasie drinken}\\

\haiku{- dan moesten de Koubers,.}{thuis zijn om naar hun knecht te}{kijken die ziek lag}\\

\haiku{aan  iedere.}{beweging zag hij dat ze}{uit haar humeur was}\\

\haiku{Koos kwam binnen met.}{twee koppen koffie en wierp}{het hoofd in de nek}\\

\haiku{waarom, hij noemde.}{de prijs en zag zijn vrouw het}{kistje inpakken}\\

\haiku{Het is pleuritis,;}{de dokter zoekt het middel}{dat hem zal helpen}\\

\haiku{{\textquoteleft}Ze  is niet voor,,!}{\'e\'en gat te vangen pa houd}{haar in de smiezen}\\

\haiku{Nu zag hij Berrie,.}{in het praalbed met koortsogen}{en hete wangen}\\

\haiku{- Ze namen de lift,.}{ze stonden voor de zaaldeur}{en gingen binnen}\\

\haiku{Wel een minuut lang,:}{bleven ze in dezelfde}{houding toen zei Dirk}\\

\haiku{Hij moest dat meisje -?}{om een portret vragen maar}{had hij het beloofd}\\

\haiku{Hij trachtte daarover,.}{te denken maar voelde zich}{leeg en machteloos}\\

\haiku{Heel langzaam verschoof.}{de grote wijzer van de}{klok boven de deur}\\

\haiku{{\textquoteright} {\textquoteleft}En als je sneuvelt,?}{in een bocht wat moet er dan}{van Jantje worden}\\

\haiku{toen is hij toch nog -?}{onverwacht gestorven waar}{had hij die gehoord}\\

\haiku{{\textquoteright} {\textquoteleft}Welnee,{\textquoteright} zei ze, {\textquoteleft}geen,.}{gekheid alleen de dag van}{de begrafenis}\\

\haiku{Ze gaf hem een por,:}{met haar elleboog maar zei}{een ogenblik later}\\

\haiku{Daar geen spreker zich,.}{meer meldde dankte Dirk voor}{de belangstelling}\\

\haiku{{\textquoteleft}Wij weten het heel,.}{goed wij hebben hem vijftien}{jaar in huis gehad}\\

\haiku{Hij wou wel, en ik, -.}{heb gehoopt dat ik ook zou}{willen maar niks hoor}\\

\haiku{{\textquotedblleft}Meid, in die jaren.}{ben ik bijna gek geweest}{van lijfsverlangen}\\

\haiku{{\textquoteleft}Als ik te oud word.}{om over je heen te stappen}{zal ik het zeggen}\\

\haiku{{\textquoteleft}Het is toch beroerd,...{\textquoteright}}{genoeg geweest als je het}{me nu maar gunt dat}\\

\subsection{Uit: De loop der dingen}

\haiku{hij kon Londen met,.}{minder bereiken zelfs al}{zou hij gaan vliegen}\\

\haiku{Wel was het zomer,,.}{maar de nachten waren koel}{op het open water}\\

\haiku{Je moet nadenken,,;}{berispte hij zichzelf geen}{dwaasheden gaan doen}\\

\haiku{Je begon met het,,;}{uitwendige te zien ze}{was mooi en bloeiend}\\

\haiku{Voor de meisjes had,.}{hij niets maar dit hinderde}{hem geen oogenblik}\\

\haiku{Zonder droom bleef ze,,.}{en zonder gedachten de}{tijd bestond niet meer}\\

\haiku{Wie neemt zooiets van,,?}{mij wat je overal krijgen}{kunt en goedkooper}\\

\haiku{Op de krant schreef ze,,, -.}{groot die enkele woorden}{de inkt vloeide uit}\\

\haiku{Maar ze wilde niet,.}{naar Engeland en hij zelf}{koos ook dezen weg}\\

\haiku{om haar heen was het,.}{geraas van de stad in den}{nieuwen jongen dag}\\

\haiku{Hij zag haar zeer bleeke,.}{gezicht en het gedwongen}{staren van haar oogen}\\

\haiku{- Nou, dat ik voor een,;}{poosje weg ging en ze niet}{ongerust moesten zijn}\\

\haiku{Och, evengoed als in,.}{Holland waar hij bij vreemde}{menschen had gewoond}\\

\haiku{O, het blinken van,.}{dat water aldoor en het}{dansen van hun boot}\\

\haiku{Hij keek naar de zon,,.}{die al daalde zijn compas}{was overbodig nu}\\

\haiku{- O. Daarna gingen.}{ze beiden weer voort aan hun}{stille gedachten}\\

\haiku{- weggaan met een man,,.}{waarvan ze niet hield en die}{haar niet begeerde}\\

\haiku{van wat ondergoed,.}{rolde ze een kussen zoo}{dat hij het niet zag}\\

\haiku{De nevel trok weg,,.}{het werd lichter maar het kon}{de dag nog niet zijn}\\

\haiku{Toen ze nu het eten -}{klaar had ze kookte rijst met}{spaghetti en ham}\\

\haiku{Ze bedacht dat ze,,.}{niets had meegenomen geen}{schaar geen vingerhoed}\\

\haiku{Ik zal schrijven, ik.}{weet niet waarom ik er zoo}{lang mee heb gewacht}\\

\haiku{Maar toen zag ze den,.}{heeten blik van zijn oogen en}{ze rukte zich los}\\

\haiku{Ze deed of ze het,.}{niet merkte maar den glans hield}{ze vast in haar oogen}\\

\haiku{- Stil, zei ze zichzelf,?}{waarom zie ik alles zoo}{duidelijk vandaag}\\

\haiku{O, wel zal ze veel, ().}{kunnen bereiken heel veel}{maar ze wist niet w\`at}\\

\haiku{- een beetje te snel,.}{en onduidelijk maar hij}{kreeg toch cum laude}\\

\haiku{Hun denken dwaalde,.}{om de vrouw die moeder is}{of het worden zal}\\

\haiku{Nu zag ze Jacques,.}{ze wist niet wanneer hij was}{binnengekomen}\\

\haiku{Aan den achterkant.}{grensde het woonhuis aan de}{fabrieksgebouwen}\\

\haiku{En een eigen huis,.}{te hebben daarop heb ik}{nauwelijks gehoopt}\\

\haiku{Er was iets in haar,,.}{nuchtere starre leven}{gekomen een droom}\\

\haiku{De bleeke rozen bij.}{de veranda praalden nog}{met groote dauwdroppels}\\

\haiku{We leven hier als,.}{in het oude Egypte dacht}{ze een oogenblik}\\

\haiku{Maar nu bleef nog haar.}{blik onafgebroken op}{de dingen rondom}\\

\haiku{- en ik leef voor hem, -.}{het is niet oneervol voor}{een vrouw dit te doen}\\

\haiku{En wat mij aangaat,.}{mag ze met de kinderen}{van het dorp spelen}\\

\haiku{En voelde dat ze,.}{te ver ging zonder zichzelf}{te kunnen stuiten}\\

\haiku{Nu was ze ook weer,.}{de tevreden rustige}{Ina die hij kende}\\

\haiku{- dat Lilly weggaat,,.}{drukt ons al willen we het}{elkaar niet zeggen}\\

\haiku{- Hij praatte voort, zei,.}{haar hoe de ligging van het}{huis was en de tuin}\\

\haiku{en toen liep ze in.}{den regen langs het strand met}{George Kelwin}\\

\haiku{Dat ze hem zonder,.}{klacht had laten gaan had hij}{in haar geprezen}\\

\haiku{Zoo weinig vroeg hij,.}{van haar en zooveel had ze}{hem afgenomen}\\

\haiku{Alle gedachten.}{aan haar eigen werk trachtte}{ze ver te houden}\\

\haiku{Ze kon niet zooveel.}{bij hem zijn als den eersten}{tijd van zijn ziekte}\\

\haiku{Toen ze het atelier,.}{binnen kwam was er een hoog}{lawaai van stemmen}\\

\haiku{- Om u de waarheid,,.}{te zeggen zei ze dacht ik}{aan mijn dochtertje}\\

\haiku{Hij had iets sluws in, -.}{zijn donkeren blik toch was}{zijn houding loyaal}\\

\haiku{- U moet dat voorste,.}{haar niet zoo strak trekken het}{doet pijn om te zien}\\

\haiku{- Kom kind, het is niet,.}{goed op den laten avond zoo}{te redetwisten}\\

\haiku{Ik heb het moeder,,.}{gezegd gisteren maar ze}{geloofde het niet}\\

\haiku{Maar Amy waardeerde,;}{dit niet dorst ook nauwelijks}{een stoel te nemen}\\

\haiku{Lilly zat rechtop;}{en keurde de handigheid}{van hun bestuurder}\\

\haiku{Ze zou dit niet met, -?}{woorden aanroeren sprak Jack}{ooit over zijn liefde}\\

\haiku{Haar gezicht was bleek,.}{en vochtig met enkele}{fel-roode plekken}\\

\haiku{Eerst was het aldoor,,,.}{onzeker van September}{af tot in Maart April}\\

\haiku{- dat maakte me zoo,;}{ellendig het altijd weer}{te moeten lezen}\\

\haiku{Al gauw merkte ze,,.}{dat het Lilly zwaar viel den}{tijd af te wachten}\\

\haiku{Ze zweeg, een beetje,.}{beschaamd ze had Lilly niets}{willen verwijten}\\

\haiku{Maar onderwijl had,.}{ze soms het gevoel alsof}{zij niet mee zou gaan}\\

\haiku{- ik wil heelemaal,.}{opnieuw beginnen onder}{andere menschen}\\

\haiku{- Toen u vluchtte, was,.}{er toch ook geen andere}{vrouw die u meenam}\\

\haiku{De slaap was heel zwaar,,.}{op haar ze wilde haar oogen}{openen maar kon niet}\\

\haiku{- Ze was alleen in,.}{haar bed het licht stond groot en}{scherp in de kamer}\\

\haiku{- Ze weten daar niet,,.}{zei ze plotseling dat miss}{Lilly al weg is}\\

\haiku{zult u altijd bij?}{mij komen als u denkt dat}{ik u helpen kan}\\

\haiku{- Het is voor u het,,.}{ergste zei hij het kind richt}{er zich nog aan op}\\

\haiku{Maar ik heb hem toen,.}{alleen gelaten ik moest}{mijn eigen weg gaan}\\

\haiku{Het water moest half,.}{verzonken zijn of verdampt}{in de heete lucht}\\

\haiku{Dat jaar was als een,.}{vuur waarin de rest van haar}{leven verzengde}\\

\haiku{Hij mocht ook opstaan,,;}{en haar gezicht tusschen zijn}{handen vatten even}\\

\haiku{- dan voelde hij een,.}{gloed dien hij niet alleen uit}{zichzelf dacht te zijn}\\

\haiku{Vreemd, - hij was nu vier, -.}{jaar in Londen dat hij Ina}{nooit meer gezien had}\\

\haiku{Prachtige stof, ja: -,.}{goed voor de hooge ramen van}{een oud voornaam huis}\\

\haiku{Daar een plaat van een,:}{oud vrouwtje dat haar bijbel}{las en een pendant}\\

\haiku{Een twee drie, de jood,.}{in den pot fijn gestampt en}{de deksel erop}\\

\haiku{Haar beschermen voor,.}{haar eigen dwaasheden want}{ze was een kind}\\

\haiku{Een vrouw, in 't langs,,.}{gaan spiedde nieuwsgierig maar}{dat merkte hij niet}\\

\haiku{Ze zou moeder zijn,,.}{van drie kinderen later}{van nog meer misschien}\\

\haiku{- Ja, zei hij, misschien,,.}{niet voor mijn vrouw maar voor mij}{ik wil naar mijn werk}\\

\haiku{Hij bukte zich over,,.}{haar keek in haar donkere}{oogen die sterk glansden}\\

\haiku{Neen, hij zou haar niet,.}{deren zelfs niet even met zijn}{lippen haar raken}\\

\haiku{O, dacht ze schamper,,.}{ze zou hem niets vragen hij}{was heer en meester}\\

\haiku{- Ze liep nooit iemand,,.}{tegemoet behalve Frans}{een enkele maal}\\

\haiku{En nu leek het haar:}{meer te zijn geweest dan een}{kus uit gewoonte}\\

\haiku{eigenlijk kon hij,.}{zich niet herinneren dat}{het ooit gebeurd was}\\

\haiku{'t Was alsof er,.}{een muur weg viel door hier aan}{te denken alleen}\\

\haiku{- Doet u geen moeite,,,.}{zei Frans ik ga er alleen}{heen en tref u daar}\\

\haiku{Haar instinct zei haar,.}{dat ze geen verwondering}{mocht laten blijken}\\

\haiku{Alleen haar jongens,;}{die kon ze een Engelsche}{opvoeding geven}\\

\haiku{Ze kwam naar hem toe,;}{over het donkerkleurige}{Smyrna-tapijt}\\

\haiku{Toch prikkelde hem,:}{het aanpassingsvermogen}{van Ruth of liever}\\

\haiku{- Haastig, alsof hij,.}{geroepen was trok hij de}{deur achter zich dicht}\\

\haiku{toen voelde hij een,.}{traan opwellen zijn heete}{oogen bevochtigend}\\

\haiku{Of was haar houding,?}{bewust zoo omdat ze zijn}{woorden negeerde}\\

\haiku{den dood zoeken zou,,.}{niet moeilijk zijn alleen laf}{en verachtelijk}\\

\haiku{- Best, zei Ruth, dan kan,.}{ik me nog kleeden terwijl}{jij de auto haalt}\\

\haiku{Dat stuk van vanavond,,?}{was leuk maar het bleef wat te}{veel spel vind je niet}\\

\haiku{- Dan begin ik nu,,.}{te helpen we maken er}{nog een voor terug}\\

\haiku{- Als ik midden voor,?}{de auto ga staan zou ze}{me dan overrijden}\\

\haiku{Je moet het maar niet,.}{probeeren als moeder met dat}{kleine ventje rijdt}\\

\haiku{Het was een beetje;}{gevaarlijk om midden op}{den weg te gaan staan}\\

\haiku{- Ik dacht dat moeder,.}{ons uit school wilde halen}{en ons gemist had}\\

\haiku{- Maar mevrouw, ik ben,.. -?}{dikwijls zoo oprecht dat Dat}{u bang bent voor uzelf}\\

\haiku{Miss Conny zag hij,.}{nauwelijks en voor James}{Lomb had hij eerbied}\\

\haiku{Hij praatte nu met,.}{Lomb legde een oogenblik}{zijn hand op diens mouw}\\

\haiku{- Blijft u nog even, het.}{spijt me dat mijn vrouw u juist}{vandaag genoodigd had}\\

\haiku{Nooit wilde ze zijn,.}{geboorteland zien of zijn}{verleden kennen}\\

\haiku{Ze zou nu niet meer,.}{kunnen zeggen waarom ze}{het begonnen was}\\

\haiku{Zijzelf had nu haar, -.}{werk het werk dat de vrouw van}{een groot man doen kon}\\

\haiku{Dus waren ze in,.}{de stad gebleven en hij}{winkelde met haar}\\

\haiku{Onwillekeurig,.}{keek hij op toen hij stemmen}{hoorde bij de deur}\\

\haiku{Hij zou willen dat,,;}{hij over Ina denken kon scherp}{over haar karakter}\\

\haiku{Ruth lachte, omdat.}{hij nog altijd gebonden}{was aan de zaak}\\

\haiku{Hij at thuis, merkte.}{dat het keukenmeisje op}{hem had gerekend}\\

\haiku{Terwijl hij schreef, wist.}{hij dat hij ook dezen brief}{niet versturen zou}\\

\haiku{Mijn eigen auto,.}{is blijkbaar stuk en dus liet}{ze zich afhalen}\\

\haiku{Op dat oogenblik,.}{wist ze dat ze in lang niet}{zoo jong was geweest}\\

\haiku{vroeg hij aarzelend, - -,.}{ik herinner me niet Neen}{zei ze bevangen}\\

\haiku{- vreemd, dat hij al dien,.}{tijd had laten voorbij gaan}{zonder haar te zien}\\

\haiku{Eigenlijk kan ik;}{me niet herinneren dat}{je ziek geweest bent}\\

\haiku{Ze verdedigde,,.}{hem koos zijn partij dwong hem}{eigenlijk tot niets}\\

\haiku{Het was of een beeld,,.}{van hemzelf binnen in hem}{zich neerboog en bad}\\

\haiku{- ze reisde er vaak,,.}{heen die dagen bleef er ook}{wel eens overnachten}\\

\haiku{- Er drongen tranen,.}{naar haar oogen een enkele}{viel stil langs haar wang}\\

\haiku{- Zoo is het goed, zei,.}{ze nu ben ik van dat al}{te zware bevrijd}\\

\haiku{- ik wandel en doe,.}{boodschappen ik voel me soms}{een mondaine vrouw}\\

\haiku{Kwam het doordat ze,,?}{niet zonder eenige moeite}{haar oude taal sprak}\\

\haiku{Toen hij op straat stond,,?}{vroeg hij zich af waarom hij}{al was weggegaan}\\

\haiku{- Maar Holland is aan, -}{den overkant van het water}{dichtbij en toch ver.}\\

\haiku{Overdag zit ik als,.}{in een draaimolen waar ik}{in- en uitspring}\\

\haiku{Zijn oogen waren groot,,.}{als van een kind zijn mond was}{oud en verwrongen}\\

\haiku{Eens ging Frans op weg,.}{naar Ina een Zondagmiddag}{in Februari}\\

\haiku{Weer sloeg hij een hoek,.}{om lette scherp op of hij}{een huurauto zag}\\

\haiku{Zij hadden aan den,:}{anderen kant gewoond hij}{wist het nummer nog}\\

\haiku{Nu glimlachte ze,.}{steunde met haar hand tegen}{den rug van zijn stoel}\\

\haiku{- U wilt toch wel eens,.}{iets beleven na deze}{twee doodsche jaren}\\

\haiku{Eenmaal, kort voor hun,;}{reisje naar Frankrijk zag ze}{Frans in den schouwburg}\\

\haiku{Ze wees hem Lilly,;}{niet aan en ze geloofde}{dat hij haar niet zag}\\

\haiku{- zoo werd ook deze.}{gebeurtenis voor haar zelf}{niet heel belangrijk}\\

\subsection{Uit: De oudste zoon}

\haiku{- Ik stuur door het dorp,,,,.}{wacht ik klim wel over de bank}{schuif dan naar rechts j\^o}\\

\haiku{Jetty liep langs hem,.}{en sloeg hem met haar kleine}{hand op zijn schouder}\\

\haiku{Hij schrok van de drift,.}{die in hem opwelde beet}{zich op zijn lippen}\\

\haiku{Toch was het vrij dun,;}{van hem dat hij niet over zijn}{toekomst gedacht had}\\

\haiku{Vader deed of hij,.}{het niet hoorde en begon}{een ei te pellen}\\

\haiku{De kleintjes, Wim en,.}{Doortje stonden vaak bij hen}{en babbelden wat}\\

\haiku{- De jongen  had;}{de auto teruggebracht}{in de garage}\\

\haiku{- Later heb ik een,.}{vrouw getrouwd die om zulke}{dingen zou lachen}\\

\haiku{Herman stond veel in.}{stille aandacht achter zijn}{vaders elleboog}\\

\haiku{Op bloote voeten liep,,.}{hij den zolder over de trap}{af zijn hart bonsde}\\

\haiku{Toen huiverde hij,.}{door moeders blik zoo hard en}{minachtend keek ze}\\

\haiku{- Het is een pracht van,.}{een mansarde ik kan je}{erom benijden}\\

\haiku{vroeg Wim, het is de,.}{mooiste kleur die ik van mijn}{leven heb gezien}\\

\haiku{Als vader wegbleef,,?}{wellicht voorgoed hoe zouden}{ze dan gaan kijken}\\

\haiku{- Ik vind  het mooi,,,?}{werk maar wel vreemd erg vreemd of}{ben ik kleurenblind}\\

\haiku{Wie niet komt, beschouwt,.}{zichzelf als verloren had}{Theo Sikkesz gezegd}\\

\haiku{In Amerika zou,....}{het misschien mogelijk zijn}{maar in Amsterdam}\\

\haiku{Theo Sikkesz had een:}{cahier vol en Piet Bril wist}{maar een onderwerp}\\

\haiku{Ze leunde tegen,.}{een jas van haar vader die}{aan den kapstok hing}\\

\haiku{We zijn allemaal,,.}{ongelukkig dacht hij en}{niemand heeft de schuld}\\

\haiku{Eerst zou hij langzaam,.}{naar Hendriks loopen de zaak}{sloot om zeven uur}\\

\haiku{De hardnekkigheid,,.}{dacht hij waarmee ze me wil}{laten schilderen}\\

\haiku{hij hoopte het, maar.}{ze hadden elkaar in geen}{vijf maanden gezien}\\

\haiku{- Ja, ik verwacht nog,?}{een paar vrienden kan je straks}{wat boven brengen}\\

\haiku{Na een poosje werd.}{Jan Mastenbroek wakker en}{kroop naar den divan}\\

\haiku{- Als ik je ooit met,.}{iets van dienst kan zijn hoorde}{hij zich toevoegen}\\

\haiku{aan de verhouding?}{met vader was haar zeker}{niets meer gelegen}\\

\haiku{Hij bloosde opnieuw,.}{want hij had het zelfbeklag}{in zijn stem gehoord}\\

\haiku{- Het is niet zooveel,.}{bijzonders jullie zoudt er}{misschien om lachen}\\

\haiku{Izenburg wist meer van,.}{de chemie af dan hij en}{bood hem boeken aan}\\

\haiku{- Niet in't water,,,.}{springen hoor zei de een en}{de ander lachte}\\

\haiku{Greet stond op, maar keek,.}{Miel nog haastig even aan hij}{zag haar oogen glanzen}\\

\haiku{Haar blonde haar was,.}{kort geknipt en toch had ze}{niets jongensachtigs}\\

\haiku{Ze keek hem aan, moest,.}{hij denken met den glimlach}{van een oude vrouw}\\

\haiku{Hij had niet gevraagd,?}{of zij dansen wilde had}{ze daarop gehoopt}\\

\haiku{Dolly kwam binnen.}{en zette een schaal op de}{gedekte tafel}\\

\haiku{- Van Helsingfors naar Hamburg,.}{dat kostte me ook bijna}{mijn laatste duiten}\\

\haiku{Toen ben ik dan ook.}{dien heelen verderen dag}{een goed mensch geweest}\\

\haiku{de dokter heeft een,.}{zee'tje van me gekocht toen}{kon ik weer even voort}\\

\haiku{Als je schildert, Miel,,.}{en heel veel buiten loopt dan}{verlies je jezelf}\\

\haiku{In het duistere.}{licht van een spaarbrander zag}{hij een meisje staan}\\

\haiku{- O, zei Greet, en keek,.}{rond het is toch anders dan}{ik me voorstelde}\\

\haiku{ze was kinderlijk,.}{van gestake maar zag er}{niet ongezond uit}\\

\haiku{Hij had het toch goed,.}{gehoord maar zijn gedachten}{verdrongen elkaar}\\

\haiku{Ook speelde daar nog;}{onderdoor de gedachte}{aan Otto's bezoek}\\

\haiku{Dus liefdes-smart -?}{en daarvoor bleef haar vader}{den heelen dag thuis}\\

\haiku{Het was een grappig,.}{meisje ze had heel fijne}{sproeten op haar neus}\\

\haiku{Vanmorgen ben ik;}{in een taxi heen en terug}{naar dien man gegaan}\\

\haiku{Ik had wel heel graag,.}{een dokter geraadpleegd maar}{Tony wil het niet}\\

\haiku{- U moest toch niet te,,.}{lang wegblijven van kantoor}{zei Miel en bloosde}\\

\haiku{hier werkt mijn zusje,.}{terwijl Otto werkelijk}{niet hooghartig is}\\

\haiku{Hij streed tegen dat.}{belachelijk gevoel van}{minderwaardigheid}\\

\haiku{Miel antwoordde niet,.}{onmiddellijk en het bleef}{een oogenblik stil}\\

\haiku{We hebben elkaar,.}{gezien dat is tot nog toe}{eigenlijk alles}\\

\haiku{Hij vroeg haar zacht - hij -?}{zat tegenover haar zit je}{op heete kolen}\\

\haiku{En ik weet wat ik,,?}{kan schaatsenrijden gaat nog}{vrij goed vind je niet}\\

\haiku{en dit is dus je,,.}{kamer ja die had ik me}{wel goed voorgesteld}\\

\haiku{als hij zoo normaal,...}{doet om terug te komen}{dan zal  hij ook}\\

\haiku{Ze brak af, als was.}{ze te moedeloos om het}{alles te zeggen}\\

\haiku{Ze boeide Miel, hij.}{las bij voorbaat de woorden}{uit haar houding af}\\

\haiku{De toon doet het hem,,.}{dacht hij fooien geven is}{zoo vernederend}\\

\haiku{In de groote foyer,.}{moest hij even rondkijken voor}{hij de Ramscheid's zag}\\

\subsection{Uit: De overgave}

\haiku{- Och - ik heb Erich Weicht,.}{gekend we waren samen}{aan de Reichsanstalt}\\

\haiku{Nu zat ze weer op.}{in haar bed en leunde haar}{hoofd tegen den muur}\\

\haiku{Toen haar jongetje -,.}{stierf maar hij was nog zoo klein}{een wiegekindje}\\

\haiku{Eens dacht ik toen ook - -.}{te komen maar och en het}{was nog niet zoo noodig}\\

\haiku{- Terwijl ze nog sprak,.}{voelde ze dat haar woorden}{geen weerklank vonden}\\

\haiku{I Josien keek nog,.}{steeds naar buiten terwijl ze}{in haar hoekje zat}\\

\haiku{- vader, of Gerda,}{of Ann-Mary en}{Gerda samene}\\

\haiku{Toe, wie heeft er nog - - -,.}{iets voor het vuur papa u}{oude brieven toe}\\

\haiku{Josien stond bij den.}{pilaar van de trapleuning}{en glimlachte even}\\

\haiku{Laatst had hij zoo naar,.}{haar gekeken alsof hij}{zijn vrouw in haar zag}\\

\haiku{Ze zou de doode,.}{grijze lucht zien boven de}{bevroren velden}\\

\haiku{- Och, het drukte mij -.}{ook ik ben er tenminste}{eens uitgeloopen}\\

\haiku{En het mooie is, zei -.}{hij dat ik het nu n\'og niet}{tragisch kan vinden}\\

\haiku{- Schrik niet - zei hij, - wij -.}{zijn gezond en boog zich naar}{haar toe voor een kus}\\

\haiku{Hoe kon het ook - die?}{vrouw in Indi\"e- ze}{woonde bij een zoon}\\

\haiku{Du bist jung, und wenn,.}{das Herz es erlaubt werden}{wir neueFreunde}\\

\haiku{Maar - tracht het me te - -.}{vergeven had hij gezegd}{ik moet alleen gaan}\\

\haiku{O, ik heb altijd,,.}{veel gewild veel verlangd ik}{zal het blijven doen}\\

\haiku{Ze keken beiden,.}{naar den lichten hemel maar}{dachten niet aan zien}\\

\haiku{Hij vroeg haar en ze,.}{wees hem af want ze had nooit}{zijn liefde vermoed}\\

\haiku{ik moet nu maar voort,.}{ik moet maar zien een korten}{tijd zoo te leven}\\

\haiku{- Bertel en Franz zijn -.}{uit zijn eerste huwelijk}{hun moeder is dood}\\

\haiku{als ze even ophield,:}{met praten dan was daar zijn}{fluisterstemmet je}\\

\haiku{Zacht nam Josien zijn.}{hoofdje tusschen haar handen}{en streelde zijn haar}\\

\haiku{Ze had de laatste.}{dagen steeds sterker met Ada's}{gezin meegeleefd}\\

\haiku{- Tja - Ze praatten nog -.}{door over Ada ernstig en niet}{zonder inspanning}\\

\haiku{- Niet hier blijven, zei,.}{Franz in deze kamer is}{het een beetje suf}\\

\haiku{- Ik zie een knol, met,,,.}{een langen staart riep Franz een}{grijzen staart zoo oud}\\

\haiku{Hij was op den grond,.}{gaan liggen met zijn hoofd op}{een voetenkussen}\\

\haiku{En omdat Ada in,:}{een hoek van de kamer op}{den grond lag zei ze}\\

\haiku{De beklemming om.}{de korte disharmonie}{in Ada's huis viel weg}\\

\haiku{Als Albert kwam - dacht -,.}{ze maar het leek haar te veel}{in zoo korten tijd}\\

\haiku{Hij wilde niet den.}{indruk wekken van steeds te}{komen controleeren}\\

\haiku{Albert - riep ze - maar.}{de stilte van den tuin werd}{er niet door verstoord}\\

\haiku{En dan mag hij je,.}{toch wel eens de baas zijn als}{je het moeilijk hebt}\\

\haiku{Het huilt niet eens en.}{we laten het bijna van}{de tafel vallen}\\

\haiku{- Het kindje begon.}{te schreien en ze wiegde}{het in haar armen}\\

\haiku{vroeger was ik wel - -.}{angstig en nu sta ik er}{niet meer alleen voor}\\

\haiku{Het wil ons leeren ons -.}{aardsche leven te kennen}{en lief te hebben}\\

\haiku{- Ja - en dat doe je,,.}{wellicht ook in je hart maar}{je toont het me niet}\\

\haiku{Ze merkte wel, dat;}{hij haar het liefst van het kind}{hoorde vertellen}\\

\haiku{Het land is nog zoo, -,.}{ver weg zei ze ik kan het}{zien maar niet pakken}\\

\haiku{het stond vol bloemen.}{en het geurde naar vocht en}{vruchtbare aarde}\\

\haiku{foei, je moest naar het.}{huis van je moeder gaan en}{aan de wasch helpen}\\

\haiku{Maar hij liet het graag.}{wapperen in den wind als}{een jongen zijn kuif}\\

\haiku{Ja - mijn hart is niet - -.}{heel sterk meer maar er kloppen}{zwakkere harten}\\

\haiku{En jullie huis in,?}{Holland was dat veel grooter}{en mooier dan dit}\\

\haiku{Je kunt toch blijven,?}{om \'ons in hoe lang hebben}{we je niet gezien}\\

\haiku{Aan Alberts oogen zag,;}{ze dat hij haar kende dat}{zijn geest helder was}\\

\haiku{- nooit eerder had ze}{zoo duidelijk den val en}{den klim van het licht}\\

\subsection{Uit: De schaatsentocht}

\haiku{- Je had ineens een,.}{groot verlangen naar Harm als}{je nu aan hem dacht}\\

\haiku{je trok weer al het,.}{dek over je heen en je hart}{gaf moeder gelijk}\\

\haiku{Nu was je dertig,.}{geweest en je geloofde}{niet dat je oud was}\\

\haiku{Nu zou je alleen.}{nog naar het zingen van de}{ijzers luisteren}\\

\haiku{Tusschen Halfweg en.}{Haarlem ging ze een tentje}{op het ijs binnen}\\

\haiku{Ik laat de menschen,,.}{maar praten ik weet het zelf}{ook niet hoe ik ben}\\

\haiku{Leen had zwijgend het;}{keukentafeltje voor drie}{personen gedekt}\\

\haiku{- Och, zei je, - dat is -?}{van geen belang meer maar waar}{zal ik van leven}\\

\haiku{En daar zaten ze.}{gedrie\"en aan het kleine}{keukentafeltje}\\

\haiku{- Nou - en - vermoei je,.}{niet te veel neem een treintje}{in Lisse of zoo}\\

\haiku{Om vier uur had ze,.}{je voorgesteld thee te gaan}{drinken in Den Haag}\\

\haiku{- Nu niet meer - alles -,,....}{opgemaakt nou ja niet voor}{zichzelf hoe gaat dat}\\

\haiku{o - het gaat mij niets, -.}{aan natuurlijk maar prettig}{dat hij het kan doen}\\

\haiku{Je kon niet goed meer -?}{luisteren een baantje aan}{Dirk's eigen fabriek}\\

\haiku{dan moest ze minstens, -.}{nog veertien dagen blijven}{en dat wil ze niet}\\

\haiku{En iederen dag.}{daarvan leek moeder opnieuw}{te zijn gestorven}\\

\haiku{- Het zou gezellig, -.}{zijn jij hier in huis maar ik}{wil niet aandringen}\\

\haiku{Na moeders dood had,,.}{hij geschreven aan jou en}{Harm de twee jongsten}\\

\haiku{Promotie - een vreemd.}{woord voor een onderwijzer}{zonder hoofdacte}\\

\haiku{- Als Koos ransel moet, -.}{hebben dan zeg je het mij}{en nu geen woord meer}\\

\haiku{Dan kan hij eens naar.}{een particuliere school}{solliciteeren}\\

\haiku{- Henk is naar Utrecht - hij;}{werkt aan een artikel voor}{het Chemisch Weekblad}\\

\haiku{Leida praatte niet -.}{meer vroeger was ze ook nooit}{zoo spraakzaam geweest}\\

\haiku{Als je nu meteen,?}{kon beslissen zou je dan}{niet gelukkig zijn}\\

\haiku{De wanden en  ;}{de zoldering waren van}{lichtgeschilderd hout}\\

\haiku{de steriliteit;}{van je lichaam zou je geest}{kunnen aantasten}\\

\haiku{- Dan kruip ik er zelf,,.}{maar in want ik moet morgen}{weer werken zie je}\\

\haiku{Ik ga trouwen, - dan.}{werd je verlost van al die}{opdringerigheid}\\

\haiku{- Met ieder woord zeg, -.}{je dat je niet van me houdt}{beken het dan ook}\\

\haiku{Je sloeg de dekens -.}{op en huiverde maar je}{kon niet meer terug}\\

\haiku{De tranen sprongen -.}{in je oogen je voelde dat}{je gered was}\\

\subsection{Uit: Een sprookje}

\haiku{Jules praatte met,;}{Dutout zijn zwarte kop naar}{voren gestoken}\\

\haiku{de mond, flets en groot,.}{in goede verhouding tot}{de jukbeenderen}\\

\haiku{sneeuw hier en daar, zoo'n,.}{beetje neergestrooid zonder}{kleur-beteekenis}\\

\haiku{Ik wil het hout zien,,,.}{zei Pierre het is donker}{glimmend notenhout}\\

\haiku{Je kunt haar altijd,.}{bij me binnen laten want}{ze komt niet dikwijls}\\

\haiku{- als ik nu eens uit - -.}{een slecht gezin kom Dan zal}{ik verwonderd zijn}\\

\haiku{De muur is grauwwit,.}{en er kleeft een donkere}{klimplant tegenaan}\\

\haiku{- o, en een vlek op -,.}{de schilderij het is uw}{schuld het is spotten}\\

\haiku{Zoo hij bleef zwijgen -, -.}{goed zij sprak geen woord maar ze}{was klaar voor een lach}\\

\haiku{En als je zijlings,.}{gaat zitten komt toch die weeke}{lijn weer naar je toe}\\

\haiku{Een haan kraaide, en.}{er was een hooge jubel van}{vogels in de lucht}\\

\haiku{- Ze strekte haar arm.}{uit naar Ren\'ee en begon}{weer schel te lachen}\\

\haiku{- Hij steunde met zijn.}{armen op de  tafel}{en keek Ren\'ee aan}\\

\haiku{Vroeg in den nacht, en,.}{iederen nacht vroeger werd}{de dag geboren}\\

\haiku{Het kind had zwakke,,.}{misvormde beentjes waarop}{het niet kon loopen}\\

\haiku{- maar ik wil ook niet;}{al te veel in de maling}{worden genomen}\\

\haiku{Ze huilde, en hij - -.}{gaf haar niet den troost waarop}{ze even nog hoopte}\\

\subsection{Uit: Vriendschappen}

\haiku{vandaag proef ik het, -.}{leven zoo scherp en ik heb}{niets anders te doen}\\

\haiku{jij, als journalist, - - -;}{keurt daarin alles af maar}{de natuur Zeker}\\

\haiku{En ik mag hier van,.}{de zon houden zooals ik het}{als kind geleerd heb}\\

\haiku{Onderwijl zag ze.}{dat het in de serre vol}{en rommelig was}\\

\haiku{Stans, nog steeds in haar,.}{pyjama kwam binnen en}{liep de serre door}\\

\haiku{- Hier is het - alles -,?}{mag erin blijven het wordt}{heel mooi ziet u wel}\\

\haiku{Vroeger had het zich.}{in alle spelletjes door}{Cor laten leiden}\\

\haiku{- De duiven  gaan -?}{lager vliegen ze zullen}{toch niet moe worden}\\

\haiku{- We kunnen winkels,.}{kijken maar we hebben geen}{geld om te koopen}\\

\haiku{Leni wist niet van,.}{ophouden haar stem klonk een}{beetje zeurderig}\\

\haiku{- In Holland kunnen, -,.}{we immers niets alleen thuis}{zitten en lezen}\\

\haiku{- Och - het is niet een -.}{kwestie van benutten je}{moet ernaar leven}\\

\haiku{- Je officieel -.}{te laten fotografeeren}{zonde van het geld}\\

\haiku{de kinderen zijn,.}{in mijn huiskamer ik hoef}{toch niet weg te gaan}\\

\haiku{- Kinderen, waarom,.}{zitten we hier laten we}{ergens gaan zwemmen}\\

\haiku{- Jullie kunt praten,,.}{zei Cor eigenlijk lang voor}{dat je het noodig hebt}\\

\haiku{- Ik had Leni nog -.}{nooit zoo gezien in Indi\"e}{was ze toch anders}\\

\haiku{Blijkbaar wilde hij,.}{beneden blijven totdat}{de bezoekster ging}\\

\haiku{- En misschien, ging ze, - -}{voort als ik dadelijk naar}{u was toegegaan}\\

\haiku{- U moet het hem niet, -.}{kwalijk nemen zei ze hij}{is gezond en sterk}\\

\haiku{- vader staat voor een,.}{beslissing hij wil graag van}{de gemeente weg}\\

\haiku{- Later zal je zien -.}{dat het wel gehinderd heeft}{en dan heb je spijt}\\

\haiku{- En dan bleven ze.}{ook enkele dagen thuis}{en Cor werkte}\\

\haiku{Ze voelde dat de.}{aanblik van de sterren haar}{begrip verruimde}\\

\haiku{Cor was haar zoon - niet -.}{haar bezit hij deed wat hij}{meende dat goed was}\\

\haiku{Stel je voor dat je! - -.}{daarom trouwt en nu wil ze}{weer eens wat anders}\\

\haiku{Maar lang kon ze het -.}{niet volhouden het werd te}{eentonig en triest}\\

\haiku{- Nee, dat weet ik wel -.}{maar of er ooms en tantes}{zijn overgekomen}\\

\haiku{- Er is een neef van,.}{vader die vroeger bij oma}{in huis heeft gewoond}\\

\haiku{Johanna zag Stans;}{en Leo in een hoekje met}{elkaar fluisteren}\\

\haiku{- Waarde vrienden, zei - -}{hij op het kerkhof heb ik}{niet willen spreken}\\

\haiku{die hield haar gezicht,,.}{opgeheven het was bleek}{maar onbewogen}\\

\haiku{- Weer dacht ze aan het,.}{gezicht van Hermien en vond}{langzaam haar woorden}\\

\haiku{En wat zou het haar?}{helpen of ze Greet en haar}{moeder negeerde}\\

\haiku{tegenover dezen.}{man heeft Jeanne al haar}{wijsheid verloren}\\

\haiku{- Waarom moest hij dit -, -.}{schrijven het is schaamteloos}{dat hij dat niet voelt}\\

\haiku{Ik eisch van je,.}{dat je in de toekomst niet}{nog meer schulden maakt}\\

\haiku{Maar dat raakt me 200 -,.}{weinig wellicht is het zelfs}{beter 200 voor hem}\\

\haiku{- Stoort u me niet om - - '.}{elf uur geen kopjes koffie}{ik ga aant werk}\\

\haiku{we hebben nooit veel -.}{tijd om ons te bezinnen}{het leven gaat voort}\\

\haiku{Stans had zich een wat.}{gerekten maar luchtigen}{klaagtoon aangewend}\\

\haiku{- Eens kijken - het is -.}{nu Woensdag Zondagavond zal}{hij zijn gekomen}\\

\haiku{Kom vanavond bij mij -,.}{als moeder weer uit wil moet}{ik op Jon passen}\\

\haiku{Wat komt het zelden,, -}{voor dacht Johanna dat ik}{hen samen zie nu}\\

\haiku{Je bent te zwijgzaam, -.}{met mij praat je ook niet ik}{geloof met niemand}\\

\haiku{- Dat is geen reden.}{om de dingen niet zoo goed}{mogelijk te doen}\\

\haiku{En ik kan daar niet,.}{aan meedoen daardoor prikkel}{ik haar voortdurend}\\

\haiku{Hij tuurde voor zich,,.}{uit zijn blik was helder en}{naar buiten gericht}\\

\haiku{het kon zijn dat jij.}{ook nog naar de zuivere}{zielsverwantschap zocht}\\

\haiku{- Ga dan naar je bed,.}{zei Max. De jongen leunde}{tegen de tafel}\\

\haiku{- Dank u. - Mevrouw - Toen.}{hief ze in gedachten haar}{hoofd op naar Bart}\\

\haiku{Johanna ademde.}{diep en rook den geur van hars}{en dennenaalden}\\

\haiku{En dus, als ik geen,.}{ja en amen zeg op alles}{dan wil je scheiden}\\

\haiku{Veel later wil ik -.}{hier misschien weer wonen maar}{voorloopig niet}\\

\haiku{- De notaris woont,.}{altijd in het mooiste huis}{van het dorp zei Toos}\\

\haiku{Johanna moest een.}{glimlachje verbergen om}{die lustelooze stem}\\

\haiku{- Er was plotseling,.}{een ander licht in haar oogen}{en ze leek jonger}\\

\section{Lode Zielens}

\subsection{Uit: Het duistere bloed}

\haiku{er was iets dat zij.}{allen kenden en voor mij}{verborgen hielden}\\

\haiku{het antwoord op die,.}{eene brandende vraag kon ik}{niet onderscheppen}\\

\haiku{zij de minnares.}{van mijn vader immers en}{mijn liefste vriendin}\\

\haiku{Beelden, toestanden.}{sprongen op uit het doosje}{van het verleden}\\

\haiku{Ik voelde mij den, -.}{rijkste te rijk en wist me}{toch zoo armtierig}\\

\haiku{krachtige man wiens.}{levenslust de grenzen van}{het dorp v\`er overschreed}\\

\haiku{Zij was de eerste,,.}{aan wie ik mij verlossend}{had overgegeven}\\

\haiku{Zij lachte pijnlijk.}{uitbundig toen ik haar mijn}{vreezen vertelde}\\

\haiku{- Toen ik het epistel,.}{driemaal gelezen had liep}{ik de vensters open}\\

\haiku{Het open leven, de.}{kracht van de natuur staalden}{mijn borst en armen}\\

\haiku{En er ook alleen?}{maar tijdens het nachtelijk}{uur uitgehaald werd}\\

\haiku{Maar vooral meende.}{ik met haar te stoeien om}{John te plagen}\\

\haiku{Ik streelde haar de,,.}{wangen omvatte speels haar}{leest kittelde haar}\\

\haiku{Ik bemoeide mij.}{met Liza niet m\'e\'er dan strikt}{noodzakelijk was}\\

\haiku{Ik raadde het niet,, -:}{ik zag het  niet ik wist}{met groote zekerheid}\\

\haiku{Voor vrouwen als zij, -.}{is de keus niet moeilijk een}{verlies niet zwaar}\\

\haiku{Hij nam ze onder.}{zijn arm en duwde dan met}{vader den wagen}\\

\haiku{hier weg, - ergens heen, -,.}{w\'a\'ar deed er niet toe als ik}{maar aktief kon zijn}\\

\haiku{Ook naar mijn zin was.}{ik te koel bij de laatste}{kus op Liza's mond}\\

\haiku{Als de anderen:}{staarde ik uren naar dat eene}{punt van den einder}\\

\haiku{Ik vleide mij zeer,.}{behoedzaam tegen haar aan}{gaf zachte duwen}\\

\haiku{Heel den tijd staarden.}{we in ons zelf gekeerd en}{angstig voor ons uit}\\

\haiku{Zij deed of zij niets.}{van mijn moeilijkheden en}{konflikten merkte}\\

\haiku{Liza was er dus.}{medeplichtig aan en wist}{dat ze verkeerd deed}\\

\haiku{Ofschoon een kracht, als,.}{het ware buiten mij om}{mij naar Tine dreef}\\

\haiku{Ik neem ze op, - kus, -,.}{ze en mijn kus is een beet}{waaronder zij kreunt}\\

\haiku{Daarvoor belette.}{ik te vaak het deinen van}{haar jonge lichaam}\\

\haiku{Daarin herkende.}{ik de proletarische}{afkomst van Liza}\\

\haiku{Ze vertrouwde zich,!}{aan mij wij waren immers}{oude bekenden}\\

\haiku{En is nu nog bij,.}{mij zal vermoedelijk thans}{wel bij mij blijven}\\

\haiku{- Wat een charmant kind,,...}{die Tine van u trachtte}{haar stem te kweelen}\\

\haiku{Het scheen mij toe dat.}{zij zeer met welbehagen}{naar hem luisterde}\\

\haiku{Volgens Liza deed.}{ik weer niet genoeg voor den}{bloei van het lokaal}\\

\haiku{Hij heeft een moto.}{en komt soms midden in den}{dag even opzetten}\\

\haiku{- Hij is zeker weer,.}{dronken hoor ik Tine tot}{haar moeder zeggen}\\

\haiku{- Ik wil al mijn geld, -.}{vermaken aan Anna als}{gij het niet opeischt}\\

\subsection{Uit: Op een namiddag in september}

\haiku{Negentien ben ik,,,.}{en een vrouw rijker rijker}{armer dan vele}\\

\haiku{Hij keek mij lang en,.}{doordringend aan hij begon}{afscheid te nemen}\\

\haiku{{\textquoteright} ~ God, lieve God,:}{nu weet ik het plots met een}{koele helderheid}\\

\haiku{Hij was haastig, ik,.}{moest maar gaan  hier was toch}{niets meer te helpen}\\

\haiku{Er streek een vogel.}{neer in het struweel en zong}{een fluweelen lied}\\

\haiku{ook voor hen is hij,,}{een lafaard een verrader}{niet beseffend welk}\\

\haiku{toen ik op mijn beurt,,}{een bloem in den put wierp een}{witte anjer}\\

\haiku{Wij gingen vroeg te,.}{bed in een h\^otel vlak aan}{den stroom gelegen}\\

\haiku{Ik kwam te bed en,.}{het verwonderde mij niet}{dat ik moest schreien}\\

\haiku{Hij ontwaakte en.}{legde zijn warme hand over}{mijn vochtige oogen}\\

\haiku{Zonder antwoorden,.}{stond ik op kleedde mij aan}{en wachtte op hem}\\

\haiku{Ik verheugde mij,.}{daarover want thuis had zij het}{ellendig gehad}\\

\haiku{De vlammen van den,.}{haard verlichtten de onrust}{in mij ontstoken}\\

\haiku{Maar dit hoofd dan, zijn,,.}{hoofd ik begreep het niet ik}{begrijp het nog niet}\\

\haiku{- Moeder, hoorde ik,,?}{hem fluisteren moeder denkt}{gij dat ik niet lijd}\\

\haiku{En meteen voelde:}{ik weer wat mij van dezen}{mensch verwijderd heeft}\\

\haiku{- Ik haat u niet, zei,,,}{ik hem aankijkend ik haat}{u niet waar h\'a\'alt gij}\\

\haiku{Mijn liefde voor u,?}{is nooit verminderd hoef ik}{het u te zeggen}\\

\haiku{Aan de deur wendde,,:}{hij zich nog eens om keek mij}{aan dacht waarschijnlijk}\\

\haiku{ik zie de roode.}{kralen daarvan als kleine}{rozen schitteren}\\

\haiku{Maar van jongsaf had ik,.}{het besef dat het wel eens}{\'anders zou worden}\\

\haiku{Wacht dacht ik, later,,.}{onthoud mijn naam later zult}{gij er van hooren}\\

\haiku{Uw lichaam is als,.}{een blanke vaas waarin het}{gist van levensdorst}\\

\haiku{{\textquoteright} Tot dan toe had ik:}{slechts enkele brieven van}{vader ontvangen}\\

\haiku{ofschoon voor de wet,.}{en natuur uw kind ben ik}{toch uw kind niet meer}\\

\haiku{Na een poosje, zag.}{hij het nuttelooze van den}{strijd in en ging heen}\\

\haiku{Ik heb het hem nooit}{verteld en het gekke is}{dat ik niet besef}\\

\haiku{Al dien tijd was ik, ....}{blijven zitten ik stond nu}{recht en duizelde}\\

\section{Belle van Zuylen}

\subsection{Uit: Mijnheer Sainte Anne}

\haiku{{\textquoteleft}Maar, gegeven het,?}{feit dat u niet van lezen}{houdt wat gaat u doen}\\

\haiku{Een ogenblik later:}{slaakt mevrouw De Rieux een}{doordringende kreet}\\

\haiku{dat ik nooit iets had,.}{geleerd en dat ik zelfs niet}{had leren lezen}\\

\haiku{{\textquoteleft}Ik zal, wat voor weer,;}{het ook mag zijn iedere}{dag op les komen}\\

\haiku{Geen edelman uit de;}{nabije omtrek is ons ooit}{komen opzoeken}\\

\haiku{ik geef toe dat we,.}{er sinds de oorlog weinig}{over hebben maar toch}\\

\haiku{{\textquotedblleft}Voordat we getrouwd,;}{waren was je zachtaardig}{en inschikkelijk}\\

\haiku{Want dat is wat de,.}{vrouw over wie ik zojuist}{sprak is overkomen}\\

\haiku{Vroeger beviel hij,?}{mij waarom zou hij mij nu}{minder bevallen}\\

\haiku{Zij was als een bloem.}{waarop de stralen van de}{zon niet meer vielen}\\

\haiku{het glas valt, breekt, en.}{de wijn stroomt over mijn rok die}{helemaal wit w\'as}\\

\haiku{{\textquoteleft}Uw kinderjaren,{\textquoteright}.}{zijn overschaduwd door rampspoed}{zei Sainte Anne}\\

\haiku{Ik ben degene,.}{die daarvoor kan zorgen en}{ik zal dat ook doen}\\

\haiku{nergens onze ziel,.}{heeft geraakt sluit zijn ogen voor}{de realiteit}\\

\haiku{{\textquoteleft}Wat mij betreft,{\textquoteright} zei, {\textquoteleft}}{zijn metgezelik ben te}{oud om te leren}\\

\haiku{{\textquoteright} {\textquoteleft}U slaat mij uiterst,{\textquoteright}.}{behendig mijn wapens uit}{handen zei zijn vriend}\\

\haiku{O eerlijke en,!}{eenvoudige gastvrijheid}{wat heb ik u hoog}\\

\haiku{{\textquoteright} Toen Sainte Anne,:}{niet antwoordde ging Duval}{dicht bij Herfrey staan}\\

\haiku{{\textquoteright} {\textquoteleft}Wat zij is, kan met;}{de woorden leuk en mooi niet}{worden uitgedrukt}\\

\haiku{Trouwens, via mijn zoon.}{hebt u toch al een goed beeld}{van haar gekregen}\\

\haiku{Gehoorzaam aan die,.}{ingeving want zij is je}{een goede leidster}\\

\haiku{De koets was komen,;}{voorrijden en Tonquedec}{en zij stapten in}\\

\haiku{{\textquoteleft}Het gaat niet alleen,,;}{om leuke jurken mooie sjaals}{mutsen of een hoed}\\

\haiku{Zij wilde het niet,,:}{zeggen maar toen hij om zich}{heen keek begreep hij}\\

\haiku{Uw goede vriendin,!}{gaat trouwen en u bent op}{tijd voor de bruiloft}\\

\haiku{{\textquoteleft}Als Sainte Anne,.}{zijn cousine liefheeft zal}{ik van haar afzien}\\

\haiku{Mevrouw De Sainte.}{Anne deed hem weer opstaan}{en omhelsde hem}\\

\haiku{{\textquoteright} En zij legde de.}{hand van haar dochter in die}{van Sainte Anne}\\

\haiku{{\textquotedblleft}Wanneer de vrouw zich,:}{thuis verveelt dat de man haar}{pen en inkt verheelt}\\
