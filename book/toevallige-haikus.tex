% book example for classicthesis.sty
\RequirePackage{silence} % :-\
    \WarningFilter{scrbook}{Usage of package `titlesec'}
    \WarningFilter{titlesec}{Non standard sectioning command detected}
\documentclass[11pt,paper=a4,footinclude,headinclude]{scrbook} % KOMA-Script book

\usepackage[T1]{fontenc}
\usepackage[utf8]{inputenc}
\usepackage{amsmath}
\usepackage{amssymb}
\usepackage{wasysym}
\usepackage{lipsum}
\usepackage[linedheaders=false]{classicthesis} % ,manychapters
\usepackage[dutch]{babel}
\usepackage{needspace}
\renewcommand{\thechapter}{\alph{chapter}}
\renewcommand\thesection{\arabic{section}.}
\setcounter{secnumdepth}{1}

\renewcommand{\textthreesuperior}{$^{3}$}
\usepackage{xfrac}
\renewcommand{\textonehalf}{\sfrac{1}{2}}
\usepackage{soul}
\newcommand{\cyryhcrs}{\st{Y}}

\begin{document}
\begin{titlepage}
    %\pdfbookmark[1]{\myTitle}{titlepage}
    % if you want the titlepage to be centered, uncomment and fine-tune the line below (KOMA classes environment)
    \begin{addmargin}[-1cm]{-3cm}
    \begin{center}
        \large

        \hfill

        \vfill

        \begingroup
            \color{CTtitle}\spacedallcaps{Toevallige Haiku's} \\ \bigskip
        \endgroup
	
        \spacedlowsmallcaps{Emiel van Miltenburg}\\ \bigskip
        \small{\textit{Eerste editie, Lente 2020}}
        \vfill
        \null
    \end{center}
  \end{addmargin}
\end{titlepage}

\frontmatter
\chapter*{Voorwoord}
Naar aanleiding van eerdere experimenten van Marc van Oostendorp\footnote{Zie bijvoorbeeld ``Ik was een avondje sonnetten uit de DBNL vissen'', \textit{Neerlandistiek.nl}, 27 Februari 2020.} met de data van \textsc{dbnl},\footnote{De Digitale Bibliotheek voor de Nederlandse letteren, \url{https://www.dbnl.org}} ben ik ook eens gaan kijken wat er mogelijk is met zo'n grote verzameling aan Nederlandse literatuur. Dit is de uitkomst: een boek met honderden pagina's aan toevallige haiku's; zinnen die opgedeeld kunnen worden in drie regels van 5, 7, en 5 lettergrepen.

Het idee om toevallige haiku's te zoeken is niet origineel. Zo is er bijvoorbeeld \href{https://twitter.com/accidental575}{@accidental575}, een \textit{bot} die toevallige haiku's zoekt in berichten die op Twitter geplaatst zijn. En de website \href{https://haiku.somebullshit.net}{Reddit Haikus} geeft een overzicht van toevallige haikus op \textit{Reddit.com}. Wat maakt het dan toch interessant om nog zo'n project op te zetten?

Het archief van \textsc{dbnl} bevat momenteel zo'n 5700 Nederlandse boeken als e-book. Daar komen er ongetwijfeld nog veel meer bij. Als je iedere dag een boek zou lezen, dan doe je er met de huidige hoeveelheid boeken al meer dan 15 jaar over om alles gelezen te hebben. Ik denk niet dat ik al die boeken ooit zal kunnen lezen. Dit boek geeft dan in ieder geval een voorproefje van de ontzettend rijke verzameling van \textsc{dbnl}.

Wat mij aantrekt in deze verzameling toevallige haiku's, naast de ruwheid van de data, is de tegenstelling tussen mens en computer. Computers worden steeds vaker gebruikt voor \textit{distant reading}: het automatisch doornemen van een grote verzameling teksten, om statistieken te genereren die (hopelijk) meer inzicht geven in diezelfde teksten, en in de cultuur waarin die teksten zijn geschreven. Distant reading staat tegenover \textit{close reading}: het aandachtig doorlezen en graven naar betekenis in \'e\'en tekst, of in ieder geval een kleinere verzameling teksten. Dat kunnen computers niet. En ook in mijn eigen vakgebied, Natuurlijke Taalgeneratie, wordt er gezegd dat computers misschien wel creatieve teksten kunnen genereren, maar mensen blijven nodig om als curator op te treden, om die teksten te herkennen die ons raken. De haiku's in dit boek zijn het resultaat van een automatisch proces, waarbij de computer heeft gekeken of zinnen toevallig voldoen aan het (versimpelde) 5-7-5-patroon van een haiku. De auteurs van deze zinnen hebben ze nooit als haiku's bedoeld, en de computer begrijpt niet wat er staat. We zullen er zelf betekenis aan moeten geven.\\

\null \hfill \textit{Utrecht, April 2020}

    \tableofcontents

\newcommand{\haiku}[3]{\needspace{3\baselineskip}\noindent #1\\#2\\#3}

\mainmatter

\chapter[11 auteurs, 912 haiku's]{elf auteurs, negenhonderdtwaalf haiku's}

\section{Bertus Aafjes}

\subsection{Uit: Een laars vol rozen}

\haiku{Er kwam een kleine:}{tinteling in zijn groene}{ogen toen hij zeide}\\

\haiku{{\textquoteleft}Zou je (hij kuchte) ()?}{misschienhij kuchte weer van}{mij kunnen houden}\\

\haiku{Ik voelde hoe ik.}{bleek werd toen hij de naam van}{mijn engel uitsprak}\\

\haiku{De hete dag EEN.}{keer heeft een muilezeltje mij}{het leven gered}\\

\haiku{Vlak voor mijn ogen hing.}{een stenen kruik en daaruit}{sijpelde water}\\

\haiku{Nu sloeg hij weer een.}{zijweg in en keek even of}{ik hem gevolgd was}\\

\haiku{Honderden biezen.}{en twijgen lagen verspreid}{op de lemen vloer}\\

\haiku{{\textquoteleft}Het is niet prettig.}{een Franse stoelenmatster}{te zijn in Itali\"e}\\

\haiku{denk ik, terwijl ik.}{langs de hals van mijn engel}{naar beneden tuur}\\

\haiku{De persen zijn al{\textquoteright},.}{gesmolten zegt hij en hij}{deint met zijn vleugels}\\

\haiku{Wellicht werd hij door.}{zijn onrust naar andere}{zee\"en gedreven}\\

\haiku{Ik verhaast mijn pas.}{en win het van de langzaam}{sjokkende muilezel}\\

\haiku{Ik zet mijn ransel.}{voorzichtig op een open plaats}{tussen de tonnen}\\

\haiku{En dan gaat hij met.}{een kan naar de wijnton en}{draait de kraan open}\\

\haiku{Doch wie ook zou dit?}{bankroet van de bedelstand}{kunnen forceren}\\

\haiku{IJlings verdween zij.}{met haar melkemmer en liet}{de koe achter}\\

\haiku{H\`em vertrouwde hij.}{toe dat hij immer in dit}{park achtervolgd werd}\\

\haiku{Ik wil den lezer:}{een dezer vertalingen}{niet onthouden}\\

\haiku{Geheel de klas volgt.}{hem daverende op \'e\'en}{lettergreep lengte}\\

\haiku{Hij slaat het eerst het:}{kruis en in zijn donkere}{ogen staat te lezen}\\

\haiku{Buiten beginnen.}{ineens alle klokken van}{het dorp te luiden}\\

\subsection{Uit: De wereld is een wonder}

\haiku{Ik wil niet zeggen.}{dat een Nederlander niet}{chauvinistisch is}\\

\haiku{Oostermeer strekt zich.}{een der prachtigste tuinen}{van Nederland uit}\\

\haiku{Er ligt een jonge.}{stier van nog maar vier dagen}{tussen de planken}\\

\haiku{Groter bouwers dan.}{de Egyptenaren heeft de}{wereld nooit gekend}\\

\haiku{Volgens moderne.}{begrippen had het slechts een}{handjevol mensen}\\

\haiku{Hij had nog de man,.}{gekend die door het oog van}{God gekeken had}\\

\haiku{Het verhaal van de,.}{man die door het oog van God}{in de Sint Jan keek}\\

\haiku{Als het donker wordt.}{gaan er rode scheepslantaarns}{aan in de herberg}\\

\haiku{Zij heeft hun gezegd,.}{dat de duivel op aarde}{rond gaat en God ook}\\

\haiku{Als ratelende.}{wagens op de toppen der}{bergen springen zij}\\

\haiku{De zon en de maan.}{worden zwart en de sterren}{trekken haar glans in}\\

\haiku{Ik moet een glazen.}{pot hebben en een handvol}{groene bladeren}\\

\haiku{De volgende dag.}{gingen zij met ons mee in}{de trein naar Hoensbroek}\\

\haiku{Het was vier uur in.}{de nacht toen zijn voelhorens}{voorgoed verstarden}\\

\haiku{In de pastorie.}{van zijn ouders werkt hij als}{een bezetene}\\

\haiku{De koets springt over een,,.}{steen de wereld buitelt de}{wereld trekt weer recht}\\

\haiku{Hij maakte de fout,.}{die later zo velen na}{hem zouden maken}\\

\haiku{Ergens op de kim.}{stond een gele hooiopper}{als een negerhut}\\

\haiku{Zij verwelkomden.}{mij hartelijk alsof ik}{nooit was weggeweest}\\

\haiku{Een soort heiligheid,.}{waarvoor overigens veel te}{zeggen valt vind ik}\\

\haiku{Er staat er immers.}{een te bloeien in de tuin}{waarin ik dit schrijf}\\

\haiku{De Zenmeester stelt,.}{de vraag de Zenleerling moet}{het antwoord geven}\\

\haiku{Dat is typisch Zen,,.}{zichzelf vergeten zijn hoe}{ziek men ook is}\\

\haiku{Ik zag de tuin en.}{al mijn gevoelens waren}{op slag ontwapend}\\

\haiku{Er werd thee gebracht,.}{door een andere monnik}{dikke groene thee}\\

\haiku{Ik heb niet kunnen.}{uitmaken of de man in}{ernst sprak of schertste}\\

\haiku{Die vonden zij in.}{de persoon van kardinaal}{Agagianian}\\

\haiku{De Garde marcheert,,.}{aan hellemboswuivend met}{wapengekletter}\\

\haiku{En zij die te ziek.}{of te zwak zijn brengen hun}{stem uit op hun bed}\\

\haiku{En ineens is de.}{spreeuwenzwerm opgelost in}{de hemel als rook}\\

\section{C.S. Adama van Scheltema}

\subsection{Uit: Itali\"e}

\haiku{het kan, dan kunnen...}{wij het toch z\'eker en met}{nieuwen moed droogden}\\

\haiku{wij stonden voor de -,!}{Giovanni Bellini's zoo}{re\"eel zoo modern}\\

\haiku{En dat was de broer,}{van Gentile Bellini}{van wien wij later}\\

\haiku{{\textquoteright} - en dikwijls daarna,.}{heb ik dat woord zoo gehoord}{van verscheiden mond}\\

\haiku{En daar, na langen,....}{tijd haalt hij het weer op en}{d'er zit weer niets in}\\

\haiku{das sch\"one Florenz{\textquoteright}...}{keek hij ons verwonderd en}{beleefd spottend aan}\\

\haiku{Hij schreef links en rechts, -:}{op advertenties en na}{een maand had hij beet}\\

\haiku{de kerk, de markt, de,,, -.}{huisjes de menschjes de}{poorten de wallen}\\

\haiku{plak ze er niet op,!}{want het komt immers toch niet}{terecht uit Itali\"e}\\

\haiku{Inderdaad, hoeveel!}{hebben de Italianen}{op de Franschen voor}\\

\haiku{Het zijn alleen maar,.}{kinderen kinderen met}{een boefjesaanleg}\\

\haiku{{\textquoteright} - dan wisten wij, dat...}{het minstens een half uur lang}{zoo zou  duren}\\

\haiku{- te zitten, de jeugd.}{als bruin-rose diertjes links}{en rechts wegvluchten}\\

\haiku{mooier dan die berg,?}{alleen met de hooge muren}{van grauwe tufsteen}\\

\haiku{kijk nou daar en dat -, -!}{z\'o\'o is het geweest je ziet}{het heel duidelijk}\\

\haiku{Zie bij ons eens naar:}{een heerenhuis of grooter}{complex in afbraak}\\

\haiku{Goethe heeft er in.}{ieder geval beter den}{tijd voor genomen}\\

\haiku{Die Gothiek trouwens -,.}{curieus dat Itali\"e haar}{nooit begrepen heeft}\\

\haiku{De andere, de,....}{S. Maria in Cosmedin}{was weer heel anders}\\

\haiku{- doch dat kwam misschien.}{enkel maar omdat wij de}{beschaafdsten waren}\\

\haiku{Wat een pracht zou dat -!}{kunnen zijn en hoe droevig}{armoedig was het}\\

\haiku{Waarom moeten wij?}{altijd grijpen naar Fransche}{en Duitsche boekjes}\\

\haiku{hoe zou het oordeel,?}{over hem zijn wanneer eens niets}{dan dit was bewaard}\\

\haiku{- als het nu nog de!}{muze met den inktkoker}{was bij den dichter}\\

\haiku{Wij hadden ook een {\textquoteleft}{\textquoteright}:}{profane liefde onder}{die verzameling}\\

\haiku{Angelico zou.}{daar zeker een zoeten lach}{bij hebben geschreid}\\

\haiku{- Maar ook in het  :}{bouwen schijnt nog zooveel aan}{het oude gelijk}\\

\haiku{dat is tegen een, -!}{stijven hals begrijpt ge en}{men ziet er best in}\\

\haiku{de ontdekking van - -;}{den mensch van een  ziel van}{een menschen-ziel}\\

\haiku{En juist daarom, zou,;}{ik meenen staat hij tevens}{een toekomst nader}\\

\haiku{Met dat al voelen:}{wij nog iets ontsnappen aan}{de vergelijking}\\

\haiku{27Vosmaer noemt in {\textquoteleft}{\textquoteright}:}{zijnInwijding Rembrandt hier}{maar stoutmoedig weg}\\

\haiku{(Het Rome van den.)}{keizertijd wordt meest op 1.500.000}{inwoners geschat}\\

\haiku{S.) hat er einem,,{\textquoteright}.}{neuen Stil dem Barock den}{Boden bereitet}\\

\section{Catharina Alberdingk Thijm}

\subsection{Uit: Den harem ontvlucht. Een Turksch verhaal uit onze dagen (juni 1904-1906)}

\haiku{maar hij keerde tot,:}{de sofa terug strekte}{zich uit en hernam}\\

\haiku{Niet alleen schonk hij;}{een ruime lijfrente aan}{den ouden priester}\\

\haiku{ik begrijp meer en.}{meer wat mijn grootvader hier}{heeft moeten lijden}\\

\haiku{Ik ben bitterder,,!}{gestemd dan ooit Ella en}{ik heb er helaas}\\

\haiku{Gij hadt het gelaat.}{van den prins moeten zien bij}{onze verschijning}\\

\haiku{{\textquoteleft}Als ik dat voor mijn,.}{zieke geweest was zou hij}{gered zijn geweest}\\

\haiku{Maar vergeet niet dat:}{gij steeds omgeven zijt van}{ondergeschikten}\\

\haiku{dan zal ik altijd.}{zorgen dat je aangenaam}{gezelschap ontmoet}\\

\haiku{maar ik ben niet voor;}{niets aan den schoot eener groote}{liefde opgegroeid}\\

\haiku{ik er zorg voor, haar.}{eenige lieve kennissen}{te doen ontmoeten}\\

\haiku{Het korte leven.}{schijnt nog veel te lang te zijn}{voor trouwe vriendschap}\\

\haiku{maar als de  hoop,.}{op verlossing komt neen dat}{kan ze niet dragen}\\

\haiku{Zij hief zich met een.}{zenuwachtig lachje van}{haar rustbank overeind}\\

\haiku{Ik ben overtuigd, dat.}{hij haar omgekocht heeft om}{mij te bespieden}\\

\haiku{Wat zij echter ook,;}{mag hebben geleden zij}{droeg haar last zwijgend}\\

\haiku{Ik dank u voor al,;}{wat gij voor ons beiden deedt}{voor al wat gij zijt}\\

\haiku{, want ik ben zeker,.}{dat het uit jouw vruchtbaar brein}{is opgerezen}\\

\haiku{Onmogelijk dus.}{voor Nourya zich te laten}{verontschuldigen}\\

\haiku{Het eenige wat mij.}{kwelt is bezorgdheid omtrent}{Zeanor's toestand}\\

\haiku{{\textquoteleft}En dan zal ik geen.}{stap meer kunnen doen zonder}{bewaakt te worden}\\

\haiku{die in duigen zou,.}{zijn gevallen als je er}{over gesproken hadt}\\

\haiku{Marcelle zal je;}{beiden in alles steunen}{zooveel zij slechts kan}\\

\haiku{Zij meende een droom.}{te doorleven en liet zich}{gewillig kleeden}\\

\haiku{Voor ons kind behoef,,{\textquoteright}, {\textquoteleft}}{je niet te vreezen Djalah}{gaf hij ten antwoord}\\

\haiku{dat Allah even goed,.}{hier als elders herstel van}{krachten kan schenken}\\

\haiku{{\textquoteleft}Ahmed,{\textquoteright} smeekte zij, {\textquoteleft},}{wees niet hard voor ze als gij}{ze weer mocht vinden}\\

\haiku{men hoorde ze al,;}{van verre kijven wanneer}{iets haar mishaagde}\\

\haiku{Men verzekerde,.}{mij dat alle paspoorten}{in orde waren}\\

\haiku{Zoo bereikten zij:}{dien avond tegen zeven uur}{Bulgarije's hoofdstad}\\

\haiku{Ik ken de Turken.}{hier niet en wensch geen kennis}{met ze te maken}\\

\haiku{{\textquoteright} {\textquoteleft}Maar indien ik u,?}{mijn eerewoord gaf dat dit}{niet geschieden zal}\\

\haiku{{\textquoteleft}Waart gij niet zoo jong,.}{gij zoudt den toestand minder}{rooskleurig inzien}\\

\haiku{maar wat hij zonder,.}{plichtverzaking doen kan zal}{hij niet nalaten}\\

\haiku{Hij speelde zijn rol,{\textquoteright}, {\textquoteleft}}{waarlijk niet kwaad antwoordde}{Marcelle lachend}\\

\haiku{En, lieveling, wees;}{niet bang dat je last van mij}{zult hebben op reis}\\

\haiku{Trouwens het is geen.}{verre reis die ik van je}{krachten vergen zal}\\

\haiku{{\textquoteright} Zeanor drukte.}{de handen harer zuster}{tegen haar gelaat}\\

\haiku{{\textquoteright} fluisterde zij, {\textquoteleft}goed.}{zelfs mij wederom op het}{ziekbed te werpen}\\

\haiku{Het is de eenige,.}{weg of ik zou u dien niet}{voorgesteld hebben}\\

\haiku{{\textquoteright} Heel die eerste dag,,.}{een Zaterdag verliep zoo}{vreedzaam mogelijk}\\

\haiku{{\textquoteright} antwoordde zij, {\textquoteleft}al}{hebben zij mij nog dieper}{getroffen dan u.}\\

\haiku{Nourya, ik ben z\'o\'o.}{gelukkig deze natuur}{te hebben gezien}\\

\haiku{{\textquoteright} fluisterde zij, als.}{vreesde zij haar gedachten}{luid uit te spreken}\\

\haiku{Hoe zou ik echter?}{den moed hebben gevonden}{haar dat te zeggen}\\

\haiku{dat is het eenige.}{wat onze vlucht in mijn oogen}{kan rechtvaardigen}\\

\subsection{Uit: Koningsliefde. Het drama in Servi\"e}

\haiku{{\textquoteleft}maar zoo hij opgroeit;}{zal hij een man worden van}{zeldzame geestkracht}\\

\haiku{{\textquoteright} {\textquoteleft}Papa is zeker,,?}{vreeselijk wijs dat u hem}{alles vraagt niet waar}\\

\haiku{{\textquoteright} {\textquoteleft}Daarvan vermoedde,.}{ik niets ik zweer het u bij}{het hoofd van ons kind}\\

\haiku{De politiek van,,;}{zijn land was in meerderheid}{Oostenrijksch gezind}\\

\haiku{{\textquoteright} Overste Deludoff.}{uitte een gesmoorden vloek}{tegen den koning}\\

\haiku{Ik beloof u een,.}{troonopvolger op wien gij}{allen trotsch zult zijn}\\

\haiku{{\textquoteright} Uitgeput zonk het.}{prinsje weer op zijn kussen}{neer en sloot de oogen}\\

\haiku{Van lieverlede;}{echter geraakte hij aan}{dit alles gewoon}\\

\haiku{De orgi\"en van.}{zijn vader volgde hij met}{sombere blikken}\\

\haiku{Als ik eens koning,?}{ben komt u weer terug bij}{uw Sascha niet waar}\\

\haiku{Het voorstel werd ten.}{slotte met algemeene}{stemmen verworpen}\\

\haiku{Wat zou hij anders?}{doen als zich geheel aan de}{studie overgeven}\\

\haiku{Maar thans ben ik tot,.}{u gekomen om u niet}{weer te verlaten}\\

\haiku{Zoo menig vorst blijft,.}{ongehuwd volgt alleen de}{neiging zijns harten}\\

\haiku{{\textquoteright} {\textquoteleft}Onze afspraak is,{\textquoteright}.}{vervallen antwoordde de}{jonge vrouw kortaf}\\

\haiku{Nathalie sedert...?}{jaren geen schooner dag hebben}{gekend en Alexander}\\

\haiku{Dat is een woord, dat,,{\textquoteright}.}{ik niet licht zal vergeten}{Sascha zeide hij}\\

\haiku{Nicolesco.}{wierp zich op als den tolk van}{aller gevoelens}\\

\haiku{{\textquoteleft}Draga,{\textquoteright} zeide zij,.}{het op de borst gezonken}{hoofd opheffende}\\

\haiku{De vrouw tegenover.}{hem vergat bij dien aanblik}{dat hij koning was}\\

\haiku{Welnu, ik verklaar.}{u dat ik zulk een misdaad}{niet toelaten zal}\\

\haiku{Sascha was nog te,.}{jong om in zulk een tweestrijd}{te worden gebracht}\\

\haiku{{\textquoteleft}Ik heb het mijn plicht.}{geacht den Czaar de waarheid}{mede te deelen}\\

\haiku{{\textquoteright} Tot eenig antwoord wierp.}{hij zich aan haar voeten en}{beleed haar alles}\\

\haiku{Meer dan ooit was hij.}{besloten de roepstem te}{volgen van zijn hart}\\

\haiku{Gij zijt waanzinnig.}{ook maar een oogenblik aan}{zoo iets te denken}\\

\haiku{{\textquoteright} hernam de jonge,.}{vrouw hem een uitdagenden}{blik toewerpende}\\

\haiku{Hoe weinig kost het!}{u niet en hoeveel zal het}{voor mij beteekenen}\\

\haiku{{\textquoteleft}Ik zou mij schamen.}{het leger van zulk een goed}{soldaat te berooven}\\

\haiku{Van dat oogenblik,.}{af werd de jonge vrouw zoo}{goed als doodverklaard}\\

\subsection{Uit: Kroonprinses}

\haiku{Catharina}{Alberdingk Thijm Kroonprinses}{Colofon}\\

\haiku{{\textquoteleft}Maar hoe is het dan,?}{mogelijk dat ik het stuk}{niet gelezen heb}\\

\haiku{{\textquoteleft}Neen, Monseigneur, maar.}{wel Uwe Hoogheid noodeloos leed}{willen besparen}\\

\haiku{Zelfs het oog van een.}{tuinman kon hij thans niet op}{zich voelen rusten}\\

\haiku{Indien gij kalmer,}{waart zoudt gijzelf begrijpen}{hoe noodzakelijk}\\

\haiku{maar ik verzoek u.}{allereerst rechtvaardig te}{zijn tegenover mij}\\

\haiku{{\textquoteright} {\textquoteleft}Heeft u of moeder,?}{dan nooit begrepen waarom}{zij Vasthi haatte}\\

\haiku{{\textquoteright} {\textquoteleft}Door mij van haar te,{\textquoteright}:}{laten scheiden antwoordde}{de kroonprins bevend}\\

\haiku{Maar eerbiedig dan,;}{ook voortaan de vrouw die ik}{niet vergeten kan}\\

\haiku{maar als ik ze in,.}{mijn armen neem denk ik aan}{mama met Bertie}\\

\haiku{{\textquoteright} ~ Thesa had druk.}{bij het opstellen van dit}{epistel geholpen}\\

\haiku{zeker, en als ik,.}{het groote raam opendeed zou men}{ons misschien hooren}\\

\haiku{Hoe jammer voor u,!}{en mij dat anderen zich}{tusschen ons plaatsten}\\

\haiku{Omdat hij trouw bleef,?}{ondanks alles zou hij}{geschandvlekt wezen}\\

\haiku{Buitendien beweeg.}{ik mij vooral onder de}{gevallen vrouwen}\\

\haiku{Ik weet het wel u;}{behoort tot eene andere}{kerk dan de mijne}\\

\haiku{{\textquoteleft}Uwe houding bewijst,.}{mij hoe ver gij nog van}{Huis verwijderd zijt}\\

\haiku{{\textquoteright} Nog mistroostiger.}{dan te voren gevoelde}{zij zich thans gestemd}\\

\haiku{{\textquoteright} Elk hunner woorden '.}{drong den jongen vorst als een}{dolksteek int hart}\\

\haiku{{\textquoteright} De handen van den.}{kroonprins wrongen zich om de}{armen van zijn stoel}\\

\haiku{Ik dank u uit den.}{grond van mijn hart voor uwe daad}{van barmhartigheid}\\

\haiku{u liefhebben met}{geheel mijn  ziel en denk}{voortdurend aan u.}\\

\haiku{Bij oogenblikken,.}{is het mij alsof ik u}{aan het hart drukte}\\

\haiku{zelfs verlangde hij;}{niet dat zij eene prinses van}{den bloede zou zijn}\\

\haiku{{\textquoteleft}Het wordt dat, waar het,}{ophoudt menschelijk te zijn}{zoo althans oordeel}\\

\haiku{ik er over, en men.}{mag niet tegen zijn eigen}{geweten ingaan}\\

\haiku{{\textquoteleft}maar met u valt niet,.}{te spreken zoodra die}{vrouw in het spel komt}\\

\haiku{{\textquoteright} Den volgenden avond.}{ontving Vasthi een brief die}{haar deed verbleeken}\\

\haiku{{\textquoteright} Hij had gevoeld hoe.}{edel dat antwoord was en er}{haar om leeren achten}\\

\haiku{{\textquoteleft}Komt eens om mij heen,,{\textquoteright}:}{staan kinderen sprak zij met}{vriendelijken ernst}\\

\haiku{maar zij legde er,.}{een pathos in die hem tot}{in de ziel roerde}\\

\haiku{Ik heb zelfs nooit iets,.}{in haar opgemerkt dat mij}{onaangenaam was}\\

\haiku{Hij bleef dezelfde,,;}{voor haar altijd vol goedheid}{altijd welwillend}\\

\haiku{Door u verloor ik,;}{alles wat waarde heeft in}{de oogen eener vrouw}\\

\haiku{Die kinderen, die,;}{gij niet vergeten kunt gij}{zult ze niet weerzien}\\

\haiku{{\textquoteright} De jonge man wierp,.}{haar een somberen blik toe}{vol bedreigingen}\\

\haiku{gij zult naar liefde:}{hunkeren en betreuren}{wat gij hebt versmaad}\\

\haiku{maar zelfs de rouw kon,.}{de vete niet uitwisschen}{die men hem toedroeg}\\

\haiku{Zij voltooide haar,.}{volzin niet en waagde het}{niet hem aan te zien}\\

\haiku{maar de jonge oogen.}{die hem aanblikten zagen}{het ternauwernood}\\

\haiku{Ik had het recht niet.}{haar hier tegen wil en dank}{terug te voeren}\\

\haiku{Weenend van smart en,.}{woede kwamen zij tegen}{zulk een onrecht op}\\

\haiku{{\textquoteleft}als zij maar niet te,.}{vroeg verstaan haar niet te hard}{te veroordeelen}\\

\haiku{Later was het eene;}{verlichting toen Luta in}{het huwelijk trad}\\

\haiku{hij is heel goed, heel,,,;}{edel heel hoog Monica hij}{leeft in Duitschland}\\

\haiku{hadden zeker haar....}{vader gedwongen van de}{moeder te scheiden}\\

\haiku{{\textquoteright} {\textquoteleft}Gerust, ik heb een.}{schel bij me en wil niet dat}{ge voor me opblijft}\\

\haiku{het is vreemd dat hij,.}{eerst nu na zooveel jaren}{tegen mij optreedt}\\

\haiku{{\textquoteleft}Ik was zoo in mijn,}{lectuur verdiept dat ik doof}{schijn te zijn geweest}\\

\haiku{{\textquoteleft}maar nu gij het zegt,.}{ja daar buiten heerscht een}{zonderling rumoer}\\

\haiku{Twee of drie hunner.}{kunnen immers het woord voor}{de anderen doen}\\

\section{J.A. Alberdingk Thijm}

\subsection{Uit: Karolingsche verhalen}

\haiku{Maar de Engel, die,:}{van God gezonden was sprak}{nu tot den Koning}\\

\haiku{{\textquoteleft}Zult gij Gods gebod,,.}{in den wind slaan Koning zoo}{zijt gij verloren}\\

\haiku{Zoo zorgt hij voor zijn,.}{onderhoud waar hij rijke}{lieden kan vinden}\\

\haiku{Ik bid Gode te,!}{waken dat deze mij geen}{kwaad of oneer doe}\\

\haiku{{\textquoteleft}Dat is iemant, die.}{in dit bosch verdwaald is en}{van den weg geraakt}\\

\haiku{{\textquoteright} Toen zij elkander,.}{voorbijkwamen reden zij}{d\'oor zonder groeten}\\

\haiku{Liever zullen we -.}{vechten dan dat ik mij tot}{antwoord dwingen liet}\\

\haiku{{\textquoteleft}Spreekt eerst tot mij - dan,;}{zal ik u zeggen wat ik}{hier zoek en jage}\\

\haiku{zegt me nu, zoo 't,.}{u gelieft hoe gij in uw}{onderhoud voorziet}\\

\haiku{Is hij van zulke,?}{machte dat gij de nacht tot}{rijden moet kiezen}\\

\haiku{Elegast, die dit,.}{alles had ga\^ageslagen}{kroop er zachtkens heen}\\

\haiku{{\textquoteright} ging hij voort, {\textquoteleft}ziet hier,.}{het zadel waar ik u zoo}{even van verhaald heb}\\

\haiku{Ik was daar en had,.}{het gadegeslagen en}{kroop er zachtkens heen}\\

\haiku{Men sleepte Eggheric -:}{voort en hing hem en alle}{verraders tevens}\\

\haiku{zij duchtte, dat hij ', ', '.}{t zoude dooden waret}{dat hijt vernam}\\

\haiku{maar toen zij vijftien,.}{jaar oud waren ontwies Reinout}{Lodewijk een voet}\\

\haiku{Valt het wat zwaar en, '.}{verdrietigt is nochtans}{met eere gedaan}\\

\haiku{en, om uw eigen,:}{eer wilt mijnen magen en}{den uwen andwoord geven}\\

\haiku{liever offerde,.}{ik alles op dan dat mijn}{goed hun blijven zo\^u}\\

\haiku{Als dat Haymijn zag,,,}{vervaerde hij hem daar}{hij ter aarde lag}\\

\haiku{Haymijn gordde hem ',:}{t zwaerd en sloeg hem in}{den hals zeggende}\\

\haiku{daar sprong Reinout op, en, '.}{sprong het de lendenen aan}{stukken datet stierf}\\

\haiku{{\textquoteleft}Ik rade u, dat,,.}{gij u wapent want het Ros}{is groot fel en sterk}\\

\haiku{{\textquoteright} Met dat Haymijn die,.}{woorden tot Reinout sprak ontsloot}{men de staldeure}\\

\haiku{Vrouw Aye, dat ziende,,:}{liep haastig toe en wrong haar}{handen zeggende}\\

\haiku{Ik zal beproeven,.}{in korten tijd of Reinout mijn}{neve is of niet}\\

\haiku{{\textquoteright} Toen kwam er klachte,;}{voor den Koning dat zijn Kok}{doodgeslagen was}\\

\haiku{{\textquoteright} Na de maaltijd ging,:}{men dansen en spelen en}{daar was groote vreugde}\\

\haiku{Men schonk er den wijn.}{overvloedig in gouden en}{zilveren vaten}\\

\haiku{{\textquoteright} Haymijn ging nu tot.}{zijne Kinderen en bracht}{ze voor den Koning}\\

\haiku{... Laat ze hier komen,!}{uw Kinderen en proeven}{hun macht met den steen}\\

\haiku{{\textquoteright} sprak hij, {\textquoteleft}gij zijt niet,:}{zoo koen dat gij uw hand zoudt}{slaan aan mijnen baard}\\

\haiku{Haymijn zag naar Reinout -.}{dat hij zijn krachten aan den}{steen zoude toonen}\\

\haiku{En Lodewijk stond:}{daar met grooten nijd in het}{harte en zeide}\\

\haiku{alleen bid ik u,.}{dat gij niet meer speelt om zoo}{kostelijken pand}\\

\haiku{Ik zal blijven op ',.}{et veld en verwachten wat}{mij overkomen mag}\\

\haiku{{\textquoteleft}Dede ik dat, Heer,{\textquoteright}, {\textquoteleft}.}{Koning zeide Reinoutzoo}{ware ik een dwaas}\\

\haiku{{\textquoteright} - {\textquoteleft}Dat waar kwaad pand voor,{\textquoteright}, {\textquoteleft}!}{onzen schat zeide Adelaert}{ik nam wat beters}\\

\haiku{al had ik goud in,.}{mijne hand het werd koper}{eer het daaruit kwam}\\

\haiku{{\textquoteright} Reinout sloeg hem 'et hoofd ',:}{af en gafet zijn broeder}{Adelaert en zeide}\\

\haiku{In het kasteel van.}{Vaucloen aan de Dordone}{woonde Koning Ywein}\\

\haiku{wat raad geeft gij mij,?}{in deze dat ik mijne}{eere behoude}\\

\haiku{Naar mijn oordeel, zult,,.}{gij ze behoudens lijf en}{goed uitleveren}\\

\haiku{Nu riep Reinout door het:}{landschap velen op om tot}{de rots te komen}\\

\haiku{nu hebben wij groote,,!}{honger en dorst dus bidden}{wij om Gods wille}\\

\haiku{dat onze Heeren,,,....}{gevangen zouden zijn Ritsaert}{Writsaert Adelaert en Reinout}\\

\haiku{Koning Carel was,, {\textquoteleft}}{onverbidbaar maar zeide}{dat hij ze houden}\\

\haiku{{\textquoteright} zeide de bode, {\textquoteleft}:}{gij zijt Koning en moogt uw}{woord niet herroepen}\\

\haiku{en zoo zij 't uit ' -.}{et oog verliezen doet ze}{stokslagen geven}\\

\haiku{{\textquoteleft}Ik beveel u dit,.}{Ros op zulke straffe als}{Roelant gezeid heeft}\\

\haiku{ik vond zoo schoonen,,!}{man als gij zijt bevangen}{met zoo groote rouwe}\\

\haiku{{\textquoteleft}God loon u,{\textquoteright} en stak,;}{ze in zijn reiszak en scheen}{blijde te wezen}\\

\haiku{Madelgijs en Reinout,;}{zagen eene schure openstaan}{daar veel stroois in was}\\

\haiku{{\textquoteleft}O lieve gezel,{\textquoteright},,}{zeide hij tot Reinout dat de}{lieden het hoorden}\\

\haiku{{\textquoteright} Meteen is daar een,.}{man bij hen gekomen die}{uit de kerke kwam}\\

\haiku{Toen gaf Madelgijs -,:}{Reinout weder zijne sporen}{van goud en zeide}\\

\haiku{{\textquoteright} - {\textquoteleft}Helaas,{\textquoteright} zeide Reinout, {\textquoteleft},,:}{gij doet kwalijk oom dat gij}{den spot met mij drijft}\\

\haiku{{\textquoteleft}'t Is een Ridder,, '.}{genaamd Reinout en mag hier in}{et land niet komen}\\

\haiku{de twaalf knechten, wien, '.}{hij bevolen was hadden}{et elk aan een koord}\\

\haiku{En als zij op de,:}{baan waren zeide Koning}{Carel tot Roelant}\\

\haiku{Pelgrim rijden op, '!}{Beyaert datet aan zijn herstel}{bevorderlijk zij}\\

\haiku{De knechten, dien 'et,.}{Ros bevolen was hielden}{kwalijk de koorden}\\

\haiku{daarna de Hertog;}{van Beieren en Samsoen}{van Borgondi\"en}\\

\haiku{{\textquoteright} zeide Roelant, {\textquoteleft}wij.}{en dachten niet dat wij \'u}{hier vinden zouden}\\

\haiku{{\textquoteleft}Wat zullen wij van,?}{dezen Schildknaap zeggen dien}{Reinout verslagen heeft}\\

\haiku{in zijn heldenmoed ',;}{kan niemantet bedwingen}{noch achtervolgen}\\

\haiku{en gij haalt vele,.}{rampen over u zoo gij ze}{ter dood laat brengen}\\

\haiku{gij zet u tegen -, '.}{mij wij zullen zien wie hier}{t meeste vermag}\\

\haiku{{\textquoteright} Bij deze woorden:}{werd des Konings herte nog}{heftiger geschokt}\\

\haiku{De liefste, daar ik,!}{mijn betrouwen op stelde}{heeft mij begeven}\\

\haiku{want Koning Asises van,;}{Keulen doet u bidden dat}{gij hem hulpe zendt}\\

\haiku{{\textquoteright} Reinout nu hadde een ', ';}{verspieder ins Konings}{Hof dieet hoorde}\\

\haiku{Madelgijs ging, en.}{kocht de beste spijze die}{hij op de markt vond}\\

\haiku{Als de broeders dit,,:}{zagen lachten zij er om}{en Adelaert zeide}\\

\haiku{{\textquoteright} Aldus scheiden Reinout,.}{en Madelgijs van hem en}{reden naar Parijs}\\

\haiku{of men Reinout ergends,!}{vernam dat men hem vinge}{en tot u voerde}\\

\haiku{{\textquoteright} Toen zij aldus met,:}{hem spotteden zeide Reinout}{met zoete woorden}\\

\haiku{ware 't zoo wel, '.}{zwart als wit ik zo\^u zeggen}{datet Beyaert ware}\\

\haiku{gij zult varen te,:}{Vaucoloen en wachten daar}{Reinout en zijn broeders}\\

\haiku{Als zij buiten 'et,;}{bosch kwamen zagen zij een}{teeken aan de lucht}\\

\haiku{{\textquoteright} zeide de Vrouwe, {\textquoteleft} ':}{toendoet dan voort minst wat}{ik u zeggen zal}\\

\haiku{Maar is 't dat hij, -.}{mede rijden wil zoo gaat}{gij en uw broeders}\\

\haiku{eilaas, eilaas, ik,.}{zegge u dat mijn vader}{u verraden heeft}\\

\haiku{de Vrouwe weende!}{zeer en bad dat ze God in}{zijne hoede nam}\\

\haiku{{\textquoteright} Als de Edele Reinout, '.}{Florenberge zag werd hem}{et herte lichter}\\

\haiku{en vlieden wij - want!}{wij worden gevangen of}{moeten sneuvelen}\\

\haiku{Madelgijs trad juist:}{uit een kamer en riep tot}{den Kok en Drossaart}\\

\haiku{Dus draafden zij zoo, ';}{lang dat ze kwamen inet}{dal van Vaucoloen}\\

\haiku{{\textquoteright} - {\textquoteleft}Ik moet 'et eerst den,{\textquoteright}, {\textquoteleft};}{eed doen zeide Wouterwant}{ik aanlegger ben}\\

\haiku{Ik zeg Ogier aan, dat;}{hij verradenis gepleegd}{heeft te Vaucoloen}\\

\haiku{hij bracht daar met zich,.}{1500 man die deden wonder}{met den wapenen}\\

\haiku{{\textquoteleft}Ik en 500 mijner,,.}{mannen die van groote krachte}{zijn varen mede}\\

\haiku{hij leefde, en geen,.}{toevluchtsoord in wat land hij}{zich begeven mocht}\\

\haiku{Dat hoorde Ogier, de,:}{koene man hij sprong toornig}{vooruit op Fortsier}\\

\haiku{en stak op Ritsaert, en,,;}{hij weder op hem zoo dat}{hij Fortsier doorstak}\\

\haiku{des hadde Koning,!}{Carel groote toorn en riep zijn}{krijgsleuze Mont-joye}\\

\haiku{En Madelgijs lag ',:}{gevangen ins Konings}{tente en zeide}\\

\haiku{Maar toen de broeders,,:}{wechdraafden zag ze Koning}{Carel en zeide}\\

\haiku{{\textquoteleft}Broeder, hoe durft gij:}{dusdanige dingen ons}{te voren leggen}\\

\haiku{{\textquoteright} De Heremijt nu.}{dede zijn gebed tot den}{Almogenden God}\\

\haiku{Braes;62 daar vond hij}{schepen en voer in het land}{van den Islamme.63}\\

\haiku{Reinout hadde verstaan.}{wie tegen zijn zone den}{kamp zoude vechten}\\

\haiku{REINOUT ging tot Koning,.}{Carel en stond v\'oor hem als}{een arme pelgrim}\\

\haiku{Geen betere voor,{\textquoteright}, {\textquoteleft}.}{hem zeide de meesterdan}{Sint-Pietersman}\\

\haiku{uw neve Reinout kwam,;}{dienen de metselaars en}{niemant kende hem}\\

\haiku{{\textquoteright} zoo fluistert de knecht, '.}{Terwijl hij zich krampig aan}{t monnikskleed hecht}\\

\haiku{en als ge in dit,, ....}{bosch Alleen u verv\'eelt in}{uw luchtigen dos}\\

\haiku{Die pronkt met eens krans,;}{van robijn esmeraud En}{keurdiamanten}\\

\haiku{Met grooter eere,;}{ontving men den Koning zoo}{Heeren als Vrouwen}\\

\haiku{{\textquoteleft}Ai Heere,{\textquoteright} zegt zij, {\textquoteleft}?}{door wat oorzaak zullen wij}{ons kind verliezen}\\

\haiku{Zij dankten hem, en.}{namen oorlof en ruimden}{met blijdschap het hof}\\

\haiku{Die er onder stond,, '.}{hem dachte dat hij int}{Paradijs ware}\\

\haiku{want dat de rechte,,,.}{waarheid is dat Blancefloer}{zijne vriendin leeft}\\

\haiku{Zie echter wel toe,.}{dat gij uwen gouden beker}{niet op het spel zet}\\

\haiku{Maar zal de Emir naar -.}{recht uitspraak doen zoo zult gij}{de dood ontkomen}\\

\haiku{Om mij verliet gij.}{uw ouderlijk huis en zijt}{hiertoe gekomen}\\

\haiku{Ik zal zelf mij wraak,.}{verschaffen van den smaad die}{mij is aangedaan}\\

\haiku{om Gods en dezer,.}{Heeren wilde schenk ik u}{beide het leven}\\

\haiku{2De plaats van reg. 19.}{tot 24 houdt Dr. Jonckbloet}{voor ingeschoven}\\

\haiku{'t Is jammer, dat,:}{Dr. J.C. Matthes alleen Reinout}{op Beyaert laat zitten}\\

\section{A. Alberts}

\subsection{Uit: De vergaderzaal}

\haiku{Dek dan voor zeven,.}{en zanik niet verder zei}{de secretaris}\\

\haiku{Meneer moest naar een,.}{andere vergadering}{zei de concierge}\\

\haiku{Dag Beuzekom, zei}{meneer Dalem en hij gaf}{de ander een hand.}\\

\haiku{De oude heer wordt,,:}{kwaad staat op geeft die vent een}{rijksdaalder en zegt}\\

\haiku{Jullie hadden dat.}{smoel van die vent achter zijn}{bureau moeten zien}\\

\haiku{Kunnen de heren?}{aan de overzijde ons geen}{garantie geven}\\

\haiku{Hij keek om zich heen.}{en hij zag dat er niemand}{meer stond te wachten}\\

\haiku{Hij stak haastig de.}{rijweg over en stapte in}{de voorste wagen}\\

\haiku{Als het nodig is,.}{zal ik gevonden worden}{zei meneer Dalem}\\

\haiku{Hij draaide zich weer.}{om en liep de gang nog eens}{door tot het einde}\\

\haiku{Hij ging achter het.}{bureau zitten en legde}{de map open voor zich}\\

\haiku{Hij sloeg de deur van,.}{het caf\'e zo hard dicht dat}{de ruit rinkelde}\\

\haiku{Meneer Dalem zag.}{ze reusachtig groot tegen}{de ringdijk zitten}\\

\haiku{In het westen was.}{de lucht nu ook helemaal}{donker geworden}\\

\haiku{Hij hield zijn handen.}{op zijn buik en rolde op}{zijn rug heen en weer}\\

\haiku{Hij hield zijn hand voor.}{zijn ogen en probeerde het}{pad af te kijken}\\

\haiku{Toen hij weer voor zich.}{uit keek zag hij zijn schaduw}{over de brug vallen}\\

\haiku{Een buitengewoon,.}{verstandige opmerking}{zei meneer Dalem}\\

\haiku{Hij draaide zich om.}{naar de kant van waar hij was}{binnengekomen}\\

\haiku{En als u dat soms,:}{beter verstaat zei de vriend}{van de dikke man}\\

\haiku{Hij sneed vlees vlak langs.}{het bot af en maakte er}{dobbelsteentjes van}\\

\haiku{Hij hoorde achter.}{zich stemmen die zeiden dat}{het niet anders kon}\\

\haiku{Twee waren leeg en.}{hij zag dat ze in de vier}{anderen lagen}\\

\haiku{Je zou ze ook naar,.}{boven kunnen brengen zei}{de secretaris}\\

\section{Jo van Ammers-K\"uller}

\subsection{Uit: Mijn Amerikaansche reis}

\haiku{Die vele, en laat:}{mijn Hollandsche nuchterheid}{eerlijk bekennen}\\

\haiku{* * * ~ Armoede doet;}{zich het ergste gevoelen}{in de behuizing}\\

\haiku{Het is merkwaardig.}{hoeveel minder vermoeiend}{hier het reizen is}\\

\haiku{Hier zijn de huizen}{het hoogst en staan zij het dichtst}{op elkander}\\

\section{Seerp Anema}

\subsection{Uit: De Aetiopische}

\haiku{Voorsmaak van de rust,,.}{in Masjiaach gewaarborgd}{wilde Hij geven}\\

\haiku{Die twaalf leeuwen op.}{de zes trappen zijn de twaalf}{stammen Jisra\"eels}\\

\haiku{Het geboomte is,.}{nog te jong om die stralen}{te ondervangen}\\

\haiku{Alleen het prachtig.}{lazuur was weer uw gave}{aan de volkeren}\\

\haiku{{\textquoteright} {\textquoteleft}Wesir Senmoet, de.}{Sidonische vondeling}{wijst die hulde af}\\

\haiku{- Hij noemde zich een,.}{broze staf die een sterke}{zuil moest vervangen}\\

\haiku{Haastig wekte hij:}{een kamerdienaar in het}{rechter zijvertrek}\\

\haiku{dat der achter hen.}{neerstortende golven van}{deze zeeboezem}\\

\haiku{Zoo liet hij dan zijn.}{wagens aanspannen en nam}{zijn krijgsvolk met zich}\\

\haiku{Daar over heen spreiden.}{ruige reuzenpalmen hun}{zware schaduwen}\\

\haiku{Vaak zullen we dat,{\textquoteright}.}{op dezen tocht niet kunnen}{doen sprak de koning}\\

\haiku{{\textquoteright} {\textquoteleft}Voorzeker, mijn vorst,:}{maar zegt Asaaf ook niet in}{\'e\'en zijner psalmen}\\

\haiku{Strak en steil steeg hij.}{boven hen uit als van een}{gansch ander geslacht}\\

\haiku{En gij zult mij een.}{priesterlijk koninkrijk en}{een heilig volk zijn}\\

\haiku{{\textquoteright} {\textquoteleft}Gij zult tegen uw.}{naaste niet spreken als een}{valsche getuige}\\

\haiku{Ik uwer vijanden,.}{Vijand zijn en benauwen}{die u benauwen}\\

\haiku{Zij reizen op naar,;}{den hemel zij dalen neer}{tot de afgronden}\\

\haiku{Neemt deze woorden,.}{mee op uw reis die Jahw\`e}{voorspoedig make}\\

\haiku{Sjaoels koningschap!}{had het bewezen in een}{ontroerend drama}\\

\haiku{{\textquoteleft}Jahw\`e zij met u,{\textquoteright}.}{klonk het hem tegen uit den}{mond der komenden}\\

\haiku{Maar als gij uw jeugd,.}{en flinkheid niet kent wijt het}{uw eigen blindheid}\\

\haiku{Boden op snelle:}{kemels hadden het nieuws ten}{paleize gebracht}\\

\haiku{Een der dienstvrouwen.}{van Tafaths kleine hof}{kwam den knaap halen}\\

\haiku{Opnieuw wandelde.}{Sjalomo een wijle heen}{en weer in de zaal}\\

\haiku{De berekening;}{had in hoogte en richting}{slechts weinig gefaald}\\

\haiku{- Ik zie geen sporen......}{meer van de heilige bron}{en het heilig bosch}\\

\haiku{Alleen de tijd van.}{zes maanden moest tot op de}{helft bekort worden}\\

\haiku{Groot is Jahw\`e en,.}{hoog te loven onze God}{om zijn koningsstad}\\

\haiku{{\textquoteright} {\textquoteleft}Dat eischt een tot.}{het uiterste doorgevoerd}{belastingstelsel}\\

\haiku{{\textquoteright} Hij glimlachte en.}{dreef de rossen de Dal-}{en Hoekpoort voorbij}\\

\haiku{Ze stamden af van.}{onze goden en hebben}{groote daden gedaan}\\

\haiku{{\textquoteleft}Maar hoe heette dan,.}{uw volk want geen gerucht van}{hen drong tot ons door}\\

\haiku{Het werd gevonden.}{in een geheim vertrek in}{den burcht onzer stad}\\

\haiku{Ze vroegen zeven,.}{dagen die hun glimlachend}{werden toegestaan}\\

\haiku{machtloos poogt dan 't,,:}{ontwijken de \'e\'ene den}{ander vervolgend}\\

\haiku{De witroode pluim.}{op hun bronzen helm was hun}{herkenningsteeken}\\

\haiku{- Toen kwam men aan een,.}{vijf uur breede vlakte uit}{krijtrots bestaande}\\

\haiku{En de sluiting met,:}{de plechtige bede zacht}{en gedragen}\\

\haiku{zij brengen goud en.}{wierook en verkondigen}{den lof des Heeren}\\

\haiku{Hoe lief hij dat ook,.}{bedoelde het gelukte}{slechts gedeeltelijk}\\

\haiku{De priesterkoren.}{van Amon-Re zingen slechts}{te samen \'e\'en lied}\\

\haiku{Hier hoor ik als het......}{ware meerdere zangen}{dooreen gevlochten}\\

\haiku{{\textquoteleft}Mijn zuster,{\textquoteright} sprak de,:}{koning toen zijn blikken het}{al hadden omvat}\\

\haiku{Als Hij dat Zelf doet,.}{verschijnt Hij onder ons in}{menschengestalte}\\

\haiku{Hij komt, om Zich ten,.}{offer te geven voor wie}{zijn voeten kussen}\\

\haiku{Hier is de baan tot,.}{Jahw\`e die de zwaarden der}{cherubs vrij laten}\\

\haiku{Jahw\`es, die alles,.}{overwint wat zich tegen dit}{heil wil verzetten}\\

\haiku{God verplettert den,,.}{harden kop van wie daarheen}{gaan trots op hun schuld}\\

\haiku{Het kon niet anders,.}{of ze moest  vroeg of laat}{naar Koesj terugkeeren}\\

\haiku{{\textquoteright} {\textquoteleft}In zijn psalm noemt mijn.}{vader Malchizedeqs}{priesterschap eeuwig}\\

\haiku{Het rosse licht deed.}{het goudpoeder op pruik en}{mantel schitteren}\\

\haiku{Jahw\`e verheffe.}{zijn aangezicht over u en}{geve u vrede}\\

\haiku{Den vierden tammoez,{\textquoteright}, {\textquoteleft}.}{zegt de Palestijnlegt de}{zon haar sluier af}\\

\haiku{Is mijn lot eenig, zooals,!}{gij zeidet hoe moet dan het}{uwe worden genoemd}\\

\haiku{Als de geuren in,......}{bloem en kruid ontwaakten met}{zoete bedwelming}\\

\haiku{De weg over de brug.}{leidde achter de heesters}{rechts naar de bronnen}\\

\haiku{Rechts van de beek was,.}{een grasveld waarop het spel}{zou worden gespeeld}\\

\haiku{O Maak in het oog!}{uwer vrienden uw liefste niet}{tot een vermomde}\\

\haiku{Een hoofdstel van goud.}{laat ik maken met beng'lende}{zilveren klokjes}\\

\haiku{Fluiten met nu en.}{dan trompetten en cymbels}{namen het lied over}\\

\haiku{Trompet en cymbel.}{vervingen even melodie}{en begeleiding}\\

\haiku{aan zijn arm gaat een,,.}{bejaarde slanke vrouw de}{moeder van Abida}\\

\haiku{Waar is uw liefste,?}{heengegaan o gij schoonste}{onder de vrouwen}\\

\haiku{van de trali\"en.......}{trok hij zijn hand terug zijn}{stappen verstilden}\\

\haiku{- En, misschien \'e\'en of.}{meer der beste leerlingen}{van de Wijsheidsschool}\\

\haiku{En zou dat den roep?}{van Sjalomo's wijsheid niet}{blootstellen aan spot}\\

\haiku{{\textquoteright} En Nofernere:}{had haar moed ingesproken}{met de opmerking}\\

\haiku{Dan droeg de vorstin,,.}{na Loema's stiltewenk haar}{tweede raadsel voor}\\

\haiku{Doet de gloed haar oog,.}{druipen zoo lacht zij en haar}{traan glipt over haar wang}\\

\haiku{{\textquoteleft}Dank lieve vrienden,!}{dat gij haar hart zoo wijd voor}{mij hebt geopend}\\

\haiku{- Hij staart omhoog in:}{de schemering der vensters}{daarboven en bidt}\\

\haiku{Dr J. Ridderbos {\textquoteleft}{\textquoteright}:}{zegt inKorte Verklaring}{bij Jesaja 43:3}\\

\subsection{Uit: De Aulische}

\haiku{- En vergeet Joaab,.}{niet uws vaders moordenaar}{en die van Amasa}\\

\haiku{{\textquoteleft}Uit eerbied voor uw,:}{keus mijn vriend uit Benjamien}{doe ik dezelfde}\\

\haiku{- Al waren ze niet,.}{enkel vleierij enkel}{waarheid nog minder}\\

\haiku{Hoe zou ze voor haar:}{godin kunnen verschijnen}{zonder de woorden}\\

\haiku{{\textquoteright} {\textquoteleft}Zoo hoorde ik den.}{chakaam-vrouwenkenner}{nog nimmer spreken}\\

\haiku{- {\textquoteleft}Wat een schat,{\textquoteright} sprak ze,.}{glimlachend toen ze de deur}{weer had gesloten}\\

\haiku{Maar, hoewel ik het,!}{waardeerde wat bracht mij dat}{mijn roeping nader}\\

\haiku{- Doe echter van jouw,.}{zijde niets voordat hij het}{onderwerp aanroert}\\

\haiku{Immer voorheen was.}{ze de ziel en de glorie}{onzer festijnen}\\

\haiku{de zacht stralende.}{genegenheden van haar}{diepste vrouwlijkheid}\\

\haiku{{\textquoteright} {\textquoteleft}Ik verwacht thans nog,.}{geen antwoord van je op die}{vraag mijn lieveling}\\

\haiku{Dat zal u den kop.}{verpletteren en gij hem}{den hiel verwonden}\\

\haiku{- Prinses Faroena's.}{gezondheid liet dat minder}{toe den laatsten tijd}\\

\haiku{Mijn vader in een:}{zijner psalmen spreekt geheel}{dienovereenkomstig}\\

\haiku{Plotseling richtte,:}{ze zich half op sperde de}{oogen wijd open en riep}\\

\haiku{Dus duizend vrouwen,!}{die samen een groot geheim}{moeten bewaren}\\

\haiku{{\textquoteright} vroeg de koning, in.}{zijn intiemste vertrek en}{samenzijn gestoord}\\

\haiku{Snel kleedden zich de,.}{twee vrouwen beraadslagend}{wat ze doen zouden}\\

\haiku{Ze is in de angst,{\textquoteright}.}{der wanhoop naar Tyrus gevlucht}{antwoordde Ba\"ana}\\

\haiku{De lampen waren,,.}{met krep omfloersd behalve}{\'e\'en boven het bed}\\

\haiku{Toen trad de koning.}{binnen in zwarten siemlah}{met zilver omboord}\\

\haiku{Telkens voelde ze,:}{haar gedachten doorbroken}{door die andere}\\

\haiku{{\textquotedblright} - Daarin wandelen{\textquoteright}......}{we thans en gij hebt uw woord}{niet gebroken}\\

\haiku{Hem in Sjilo te......}{gaan opzoeken zou wat te}{veel belangstelling}\\

\haiku{- Men zou aan misdaad,.}{kunnen denken maar dat werd}{door niets bevestigd}\\

\haiku{Ze hadden altijd,.}{wel gezegd dat die geen goed}{hier was komen doen}\\

\haiku{Voetstappen op het.}{terras meldden de komst van}{Zaboed en Boeni}\\

\haiku{De pittige geur.}{der gelende bladeren}{deed den ruiter goed}\\

\haiku{In Anatooth had hij.}{sinds lang reeds zijn schuldig hoofd}{ter ruste gelegd}\\

\haiku{De dalen werden,,.}{wijder de hoogten lager}{het landschap vlakker}\\

\haiku{{\textquoteleft}laat ik thans eerst de,.}{boodschap overbrengen die mij}{herwaarts deed komen}\\

\haiku{{\textquoteright} {\textquoteleft}Maar de boodschap, die:}{Machazioth er aan}{had toe te voegen}\\

\haiku{{\textquoteright} {\textquoteleft}Dat meende hij toch,.}{te hebben gedaan maar het}{heeft h\`em overheerscht}\\

\haiku{{\textquoteleft}Ach, ik moet al uw:}{bezwaren toestemmen en}{toch zegt mijn hart mij}\\

\haiku{O Thamaar, die ure,......}{dat ze heengingen en ik}{alleen achter bleef}\\

\haiku{Trouwe vriend van mijn,.}{kort geluk Jahw\`e heeft u}{van mij genomen}\\

\haiku{Zijn ademhaling ging,.}{regelmatig maar met een}{zweem van vermoeidheid}\\

\haiku{{\textquoteright} {\textquoteleft}Vermoeidheid, gevoel.}{van zwakheid of onwelzijn}{kende ik nauwlijks}\\

\haiku{Als alleen de slang,,.}{de groote dierzondares in}{haar vleesch opwekt}\\

\haiku{{\textquoteleft}Mijn vorst sprak,{\textquoteright} Zaboed, {\textquoteleft}.}{op gedempte toonik breng}{u een doodstijding}\\

\haiku{- Na den rouw over haar,.}{maagdom van ruim dertig thans}{den weduwlijken}\\

\haiku{de hope van het......}{ontkiemde zaad ging ook hier}{den oogst te boven}\\

\haiku{Wat  heeft Jahw\`e,.}{aan schimmen dat is aan een}{verscheurde schepping}\\

\haiku{Wat heeft het nieuwe,?}{Paradijs aan schimmen wat}{het nieuw Jeroesjaleem}\\

\haiku{{\textquoteleft}Omdat ik Jahw\`e,.}{geduriglijk voor mij stel}{is mijn hart verblijd}\\

\haiku{ik kan, Sjalomo.}{te herinneren aan het}{jongste verleden}\\

\haiku{{\textquoteright} {\textquoteleft}Mij daar gelaten,.......}{en mijn opvolgers maar ook}{om mijns vaders wil}\\

\haiku{{\textquoteright} {\textquoteleft}Daarom is ons feest.}{een feest der dankbaarheid aan}{Jahw\`e bovenal}\\

\haiku{God verplettert den,,.}{harden kop van wie daarheen}{gaan trots op hun schuld}\\

\haiku{{\textquoteright} {\textquoteleft}Ook op den weg der.}{schaduwen vond uw hoogmoed}{geen bevrediging}\\

\haiku{{\textquoteright} {\textquoteleft}Die drongen in de.}{richting der verwaarloozing van}{mijn schaduwhuwlijk}\\

\haiku{o, Dat alles was,......}{haar geen nieuwe beleving}{slechts herinnering}\\

\haiku{o Maak in het oog!}{uwer vrienden uw liefste niet}{tot een vermomde}\\

\haiku{Een hoofdstel van goud.}{laat ik maken met beng'lende}{zilveren klokjes}\\

\haiku{Leunend op den arm,.}{van haar grooten vriend zocht ze}{haar zitplaats weer op}\\

\haiku{Fluiten met nu en.}{dan trompetten en cymbels}{namen het lied over}\\

\haiku{Trompet en cymbel.}{vervingen even melodie}{en begeleiding}\\

\haiku{Waar is uw liefste,?}{heengegaan o gij schoonste}{onder de vrouwen}\\

\haiku{Haar oogen staan wijd en.}{strak op den wagen met het}{koningspaar gericht}\\

\haiku{De koning wilde.}{zijn werkvertrek als zaal des}{doods zien ingericht}\\

\haiku{Wat heeft het nieuwe,!}{Paradijs aan schimmen wat}{het nieuw Jeroesjaleem}\\

\haiku{{\textquoteright} {\textquoteleft}Nog was het dat bij.}{de komst der koningin van}{Aethiopia en Koesh}\\

\haiku{{\textquoteright} {\textquoteleft}Op uw vraag naar mijn,.}{schuld zult ge in deze rol}{geen antwoord vinden}\\

\haiku{{\textquoteright} {\textquoteleft}Geniet het aanzijn,.}{daarom dankbaar doe wel en}{verwacht Gods gericht}\\

\haiku{{\textquoteright} {\textquoteleft}Boven het leven.}{heerschen de ordeningen}{en raadslagen Gods}\\

\haiku{Zoo was gestalte.}{en rusting van den Vorst van}{het heir des Hemels}\\

\haiku{God niet liefhebben?}{boven alles en mijzelf}{boven mijn naaste}\\

\haiku{Maar er is iets meers.}{dan zelfverontschuldiging}{van het zondaarshart}\\

\haiku{Want Gij hebt geen lust,;}{tot offerande anders}{zou ik ze geven}\\

\haiku{Keer weder, mijn ziel,.}{tot uwe rust omdat Jahw\`e}{u heeft welgedaan}\\

\haiku{De sneeuwtoppen van.}{den Libanoon waren in}{wolken verborgen}\\

\haiku{- E\'en binnen opende.}{de zware kleine deur op}{Eshmoenazaar's geklop}\\

\haiku{{\textquoteright} {\textquoteleft}Dat heb je me nog,,:}{eens gevraagd lieveling en}{toen antwoordde ik}\\

\haiku{- - - - - - - - - Keer weder, mijn ziel,,.}{tot uw rust omdat Jahw\`e}{u heeft welgedaan}\\

\haiku{het duister van mijn.}{zonde en onrecht en den}{glans van Jahw\`es recht}\\

\haiku{terwijl tranen mijn,.}{oogen verduisterden voor mijn}{God op de knie\"en}\\

\haiku{Looft Jahw\`e, mijne,,.}{ziel en al wat in mij is}{zijn heiligen Naam}\\

\haiku{Enkele dagen.}{na zijn genezing riep de}{koning hem tot zich}\\

\haiku{juist is onderkend.}{en haar verwerkelijking}{tijdig voorkomen}\\

\haiku{{\textquoteright} {\textquoteleft}o Vertel ook dat,{\textquoteright}.}{fluisterde ze en vleide}{zich aan zijn schouder}\\

\haiku{{\textquoteleft}Hij naderde met,.}{korte doffe geluiden}{tot recht boven ons}\\

\haiku{En gij zult mij een.}{priesterlijk koninkrijk en}{een heilig volk zijn}\\

\haiku{Ik uwer vijanden,.}{vijand zijn en benauwen}{die u benauwen}\\

\haiku{Mosj\`e als zondaar sluit.}{voor zich den ingang af tot}{den eersten hemel}\\

\haiku{- koningin Abisjag.}{was verzameld tot haar volk}{in het schimmenrijk}\\

\haiku{- En die vreugde duurt,!}{voort nu het opwast tot het}{evenbeeld zijns vaders}\\

\haiku{Wij verdrevenen.......}{uit dat oord der zaligheid}{om onze zonde}\\

\haiku{Opeens riep hij uit:}{met iets van geestverrukking}{in houding en oogen}\\

\haiku{Hij zonk neer op zijn,,.}{zetel doodelijk bleek het hoofd op}{de borst gebogen}\\

\haiku{Ik zal bestendig,.}{bij U zijn want Gij hebt mijn}{rechterhand gevat}\\

\haiku{Toen werd een snik het,.}{teeken dat de band tusschen}{ziel en lichaam brak}\\

\haiku{Bleek en roerloos, zooals,.}{hij gisteren den laatsten}{adem gaf lag hij neer}\\

\haiku{Het Auteurschap van,.}{het boek De Prediker dat}{aanvangt op pag. 497}\\

\haiku{Ze zijn, zooals ieder,.}{kenner weet talloos in de}{Ridderpo\"ezie}\\

\haiku{17Ongeveer f 3000,.}{maar van de koopkracht weten}{we ongeveer niets}\\

\subsection{Uit: De Egyptische}

\haiku{iride\"een, blauwe.}{en witte crocussen en}{fel roode tulpen}\\

\haiku{Ik moest van den zoon,.}{worden meer dan ik eens van}{den vader was}\\

\haiku{maar om na strijdlooze,.}{overwinning door zaligheid}{opnieuw te sterven}\\

\haiku{Een zacht geklop op '.}{de vleugeldeuren vans}{konings werkkamer}\\

\haiku{Daar willen we ons,.}{juublend verheugen uw liefde}{den wijn overprijzen}\\

\haiku{Een hoofdstel van goud.}{laat ik maken met beng'lende}{zilveren klokjes}\\

\haiku{Profetie zei mij,.}{opnieuw dat mijn profetie}{zich ging vervullen}\\

\haiku{ik liefheb, o zeg,...}{hem dat mijn liefde is tot}{waanzin geworden}\\

\haiku{Zacht en zangerig,:}{begon hij te lezen wat}{nieuw was toegevoegd}\\

\haiku{Wat zullen we doen,,?}{als die dag komt dat men zal}{vragen naar haar}\\

\haiku{Blozend beklom Achia.}{het troonbordes en ging rechts}{achter den troon staan}\\

\haiku{{\textquoteright} {\textquoteleft}Wat U heden werd,,.}{voorgedragen was een lied}{een huwelijkslied}\\

\haiku{Het feest was gesteld,.}{in de week die aanvangt met}{den eersten Nisaan}\\

\haiku{{\textquoteright} {\textquoteleft}Dat voorgevoelen.}{wordt uitgedrukt in het lied}{onzer koningen}\\

\haiku{Daar willen w' ons,!}{juublend verheugen uw liefde}{den wijn overprijzen}\\

\haiku{O, maak in het oog.}{uwer vrienden uw liefste niet}{tot een vermomde}\\

\haiku{Een hoofdstel van goud.}{laat ik maken met beng'lende}{zilveren klokjes}\\

\haiku{De laatste gele.}{zonnestralen fonkelden}{op zijn paarlenkroon}\\

\haiku{Toen koor en orkest,.}{zwegen barstte de jubel}{der toeschouwers los}\\

\haiku{Wie is 't die op, -:}{ons als het morgenrood schouwt}{dan lager en teer}\\

\haiku{Ik wil dien palmboom,,:}{beklimmen hoog tot ik die}{trossen kan grijpen}\\

\haiku{omring den nabi.}{met kussens en dek hem toe}{tegen allen tocht}\\

\haiku{- Hoe drinken mijn hart.}{en zinnen den zoelen adem}{der westewinden}\\

\haiku{Ik heb de laatste,.}{jaren veel met uw vader}{gesproken Zaboed}\\

\haiku{De lijfwachten, reeds,.}{opgestegen hadden zich}{in rotten geschaard}\\

\haiku{Dan lijdt de pharao.}{zijn koninklijken gast naast}{zich op den wagen}\\

\haiku{{\textquoteleft}Zijn westen,{\textquoteright} zongen, {\textquoteleft}.}{de priesters van Tanisis}{een tempel van Amon}\\

\haiku{heel een wijk, waar ze.}{hun leven naar wijs en wensch}{kunnen inrichten}\\

\haiku{Een wit, geplisseerd.}{byssusgewaad omsluit de}{gevulde leden}\\

\haiku{En in haar klauwen;}{klemde ze het teeken der}{koningsheerschappij}\\

\haiku{Osiris is in.}{een graf en hij is heerscher}{in het doodenrijk}\\

\haiku{Gij gaat op, gaat op,.}{en straalt en straalt gekroond als}{koning der goden}\\

\haiku{Gij zonnen voor de,.}{menschen die de duisternis}{uwer landen verdrijft}\\

\haiku{Gij hebt gestalten,.}{als onze god R\^e die aan}{den hemel opgaat}\\

\haiku{- Niet, voorzoover dat in.}{den tijd van zijn beelddrager}{slechts enkele zijn}\\

\haiku{{\textquoteright} {\textquoteleft}En van wie hebben?}{de Lybi\"ers blauwe oogen}{en blonde haren}\\

\haiku{{\textquoteright} {\textquoteleft}Ze komen van over,.}{de groote groene zee zeggen}{onze leermeesters}\\

\haiku{Op het ruime plein.}{voor het koninklijk paleis}{zou ze plaats hebben}\\

\haiku{- Allen vielen op.}{hun kni\"een en bogen het}{gelaat tot den vloer}\\

\haiku{De kapel stond op,.}{de bark van Amon cederhout}{met goud overtrokken}\\

\haiku{Ook het Middenrijk.}{heeft zijn grooten op den troon}{van Horus gekend}\\

\haiku{Eindelijk komt weer.}{een horizontale gang}{uit de klimmende}\\

\haiku{de stroom sleepte den.}{edelen vorst van de Delta}{in een wreeden dood}\\

\haiku{Het grootste deel van.}{de bevolking der vlakte}{van Ibdou speelt mee}\\

\haiku{Bij toerbeurt worden.}{de bewoners der vlakte}{er bij ingedeeld}\\

\haiku{Ze dragen hem als!}{hun heer naar een zuivere}{plaats in zijn tempel}\\

\haiku{{\textquoteleft}Nog slechts korten tijd,.}{en gij zult zien wat het huis}{der goden doen zal}\\

\haiku{Waarom de dood voor?}{Osiris en niet voor de}{andere goden}\\

\haiku{En dan de Ka, die...,?}{het gewilde leven zou}{zijn maar waar is hij}\\

\haiku{{\textquoteright} {\textquoteleft}Wat we er zullen,,,.}{zien hebt gij o koning ook}{in Tanis aanschouwd}\\

\haiku{Het heilige meer.}{in hoogopgaand geboomte}{en bloemen gevat}\\

\haiku{{\textquoteright} {\textquoteleft}Daar bemerkte hij, '.}{dats vijands oostvleugel}{nog slecht versterkt was}\\

\haiku{Beken-Chons volgde,;}{hem toen Sjalomo en de}{zijnen \'e\'en voor \'e\'en}\\

\haiku{De neus verweerd tot,.}{op het been de lippen staan}{vernufteloos open}\\

\haiku{De terugreis langs,,!}{een langen rijken reisweg}{is een nieuwe reis}\\

\haiku{{\textquoteright} {\textquoteleft}Gij meent uw beider?}{gave tot het dichten van}{schoone liederen}\\

\haiku{Zij wil de daden.}{van haar wijzen gemaal niet}{veroordeelen}\\

\haiku{Wie was de grootste,?}{pharao die op den troon van}{Cheem heeft gezeten}\\

\haiku{{\textquoteleft}Mij heugen omtrent}{ons bezoek aan Noet-Amon}{vier oogenblikken}\\

\haiku{- Hier in mijn paleis,.}{teruggekeerd ontbreken}{ze mij nog immer}\\

\haiku{{\textquoteleft}Den Edomiet zult gij,.}{voor geen gruwel houden want}{hij is uw broeder}\\

\haiku{- De zaaitijd is steeds,.}{droog geweest sinds gij den troon}{uws vaders besteegt}\\

\haiku{Hij sloeg onstuimig:}{een bekken aan en tot den}{verschijnenden slaaf}\\

\haiku{Van hun zaad zal er.}{geen in de vergadering}{Jahw\`e's komen}\\

\haiku{Een poos heerschte,}{er zwijgen in het kleine}{vertrek tot Zaboed}\\

\haiku{Het was de elfde,.}{der zevende maand een uur}{voor zonsondergang}\\

\haiku{- Sjalomo stond thans, -.}{met zijn kemel aan het hoofd}{van den stoet alleen}\\

\haiku{Jahwe verheffe.}{zijn aangezicht over u en}{geve u vrede}\\

\haiku{Naar vertrekken met...}{verregelde deuren en}{gesloten vensters}\\

\haiku{{\textquoteright} {\textquoteleft}Tusschen die wensch en!}{onze baarden bestaat toch}{zeker geen verband}\\

\haiku{Hoor, o Dochter en;}{neig uw oor Vergeet uw volk}{en uws vaders huis}\\

\haiku{Hij sloeg de schalmei,.}{aan die aan den paleismuur}{in haar smeewerk hing}\\

\haiku{De zegeningen:}{van Jehoeda en Joseef}{vlochten zich dooreen}\\

\haiku{O, moge mijn heer.}{de koning hem daarin nog}{eenmaal overtreffen}\\

\haiku{Benaja en ik.}{zijn uw eenig overgebleven}{oude dienaren}\\

\haiku{De bergen sprongen,.}{als rammen de heuvelen}{als lammeren}\\

\haiku{Ook langs des jongen.}{konings baard vielen paarlen}{op zijn zijden kleed}\\

\haiku{{\textquoteleft}En mijn lieve nicht,}{te oordeelen naar wat uw}{bruidegom ons zoo}\\

\haiku{{\textquoteleft}Zoo waarlijk Jahwe,}{leeft de bouw van dat altaar}{en dat huis onzes}\\

\haiku{Bindt met touwen het.}{offerdier tot vlak bij de}{hoornen des altaars}\\

\haiku{een reukvaas om uw, -.}{koningszaal met zoeten geur}{te vullen zie toch}\\

\haiku{- Eerst kruiste de weg;}{een nog drooge winterbedding}{na langzaam dalen}\\

\haiku{Looft Jahwe, want Hij,!}{is goed want eeuwig duurt zijn}{goedertierenheid}\\

\haiku{In een hymne op,,:}{Ramses II eveneens uit zijn}{tijd treffen we aan}\\

\section{Jan Apon}

\subsection{Uit: De roode anjelier (onder ps. Max Dupont)}

\haiku{Hij heeft een klein maar.}{goed gevormd figuur en is}{zorgvuldig gekleed}\\

\haiku{Zijn kleeding is gewoon,,,.}{niet erg netjes maar ook niet}{slordig doodgewoon}\\

\haiku{Het is de bloem van.}{een in elkaar geknepen}{roode anjelier}\\

\haiku{{\textquoteright}        Hoofdstuk III   .}{C.D. Even later komt Elly}{de kamer binnen}\\

\haiku{Het zal ongeveer.}{halftwaalf zijn geweest toen ik}{op mijn kamer kwam}\\

\haiku{{\textquoteleft}Maar vertelt u me,,!}{eens u voelde zich niet erg}{wel zei u immers}\\

\haiku{Vermeer bedoel ik,.}{in deze kamer en de}{deur stond op een kier}\\

\haiku{Toen hij dus op zijn,.}{herhaald kloppen niets hoorde}{ging hij naar binnen}\\

\haiku{Ik....{\textquoteright} {\textquoteleft}Hoe wist je, dat,{\textquoteright}.}{het precies kwart over tien was}{onderbreekt Reggie}\\

\haiku{Je had Casanova.}{vergeten en wilde nog}{wat lezen in bed}\\

\haiku{{\textquoteright} {\textquoteleft}En waarom heb je?}{ons dat allemaal nu niet}{dadelijk verteld}\\

\haiku{Plotseling draait hij.}{zijn hoofd weer in haar richting}{en kijkt haar strak aan}\\

\haiku{Ik geloof dat ik.}{dan de meeste kans heb om}{haar thuis te treffen}\\

\haiku{De man in kwestie.}{moet daar ongeveer een half}{uur hebben gestaan}\\

\haiku{{\textquoteright} {\textquoteleft}Nou, dat heb je dan,{\textquoteright}.}{maar weer schrander ingezien}{prijs ik sarcastisch}\\

\haiku{Hij stokt, loopt vlug en,.}{geruischloos naar de deur die}{hij met een ruk opent}\\

\haiku{Hij heeft er blijkbaar,.}{niet veel zin in maar durft niet}{goed te weigeren}\\

\haiku{hier is het restje.}{van de sigaret die hij}{zooeven heeft gerookt}\\

\haiku{Hij houdt plotseling.}{op en staart in gedachten}{naar de schemerlamp}\\

\haiku{{\textquoteleft}Neen, ik weet niet waar.}{ze heen is en ook niet hoe}{laat ze terugkomt}\\

\haiku{{\textquoteright} vraag ik kregelig,.}{want door zijn praatjes ben ik}{den tel kwijtgeraakt}\\

\haiku{{\textquoteright} Hij strijkt zich met de:}{hand over het voorhoofd en gaat}{pathetisch verder}\\

\haiku{{\textquoteleft}Komt u even binnen.}{alstublieft en doet u de}{deur achter u dicht}\\

\haiku{Die stok kunt u wel,}{weer terugzetten waar u}{hem hebt gevonden}\\

\haiku{{\textquoteright} Na deze woorden,,.}{stapt hij door ons gevolgd de}{tuinkamer binnen}\\

\haiku{Lucien staat op, gaat.}{naar het buffet en schenkt haar}{een glas water in}\\

\haiku{Het is zoo stil, dat,,.}{ik Lucien die naast mij staat}{gejaagd hoor ademen}\\

\haiku{{\textquoteleft}Als u me nog iets,,}{te vragen hebt doet u het}{dan alstublieft nu}\\

\haiku{Ik ben in tweestrijd.}{of ik al dan niet naar mijn}{slaapkamer zal gaan}\\

\haiku{Frans Ferguson moet.}{naar mijn schatting ongeveer}{dertig jaar oud zijn}\\

\haiku{Na het diner heb.}{ik me verkleed om naar de}{Apollolaan te gaan}\\

\haiku{Hij zoekt zorgvuldig.}{een sigaret uit en knipt}{zijn aansteker aan}\\

\haiku{Zeker, het is heel,!}{interessant d\`at wilde}{u zeker zeggen}\\

\haiku{Het had er meer van!}{of hij naar motgaten zocht}{of iets dergelijks}\\

\haiku{{\textquoteright} {\textquoteleft}Weet je misschien welk '?}{deel van de jas hij int}{bijzonder bekeek}\\

\haiku{Ferguson is net!}{zoo onschuldig als hier dat}{kanariepietje}\\

\haiku{{\textquoteleft}Ik heb persoonlijk.}{met den kellner gesproken}{die ze bediend heeft}\\

\haiku{Volgens hem zijn ze.}{stellig niet voor halfelf uit}{Zandvoort vertrokken}\\

\haiku{{\textquoteleft}En, voel je je nog?}{steeds op je gemak in dit}{sinistere huis}\\

\haiku{{\textquoteleft}Enfin, met behulp.}{van den notaris zal ik}{het wel klaar spelen}\\

\haiku{{\textquoteleft}Heeft u ook altijd?}{zoo'n verschrikkelijken dorst}{als het zoo warm is}\\

\haiku{logeeren om geld te,.}{leenen zijn portefeuille te}{laten verliezen}\\

\haiku{{\textquoteleft}Gelukkig dan maar,,!}{dat u haar hier verloren}{hebt meneer Vermeer}\\

\haiku{Je verregaande.}{onbeschaamdheid gaat alle}{perken te buiten}\\

\haiku{{\textquoteleft}Doe me een plezier,.}{blijf vanmiddag thuis en geef}{je oogen flink den kost}\\

\haiku{{\textquoteleft}Wat ter wereld,{\textquoteright} vraag, {\textquoteleft}?}{ik me afis beter dan}{een koude douche}\\

\haiku{Om u de waarheid,.}{te zeggen ik verwacht hem}{ook met ongeduld}\\

\haiku{{\textquoteright} {\textquoteleft}Dan is het verder.}{het beste dat u alles}{maar aan ons overlaat}\\

\haiku{Daarom, als u zich,....{\textquoteright} {\textquoteleft}?}{niet een beetje in acht neemt}{danIs Mona ziek}\\

\haiku{Als ik klaar ben met,.}{mijn verhaal blijft hij zwijgend}{voor zich uit staren}\\

\haiku{Ze geven me een.}{onbehaaglijk gevoel van}{naderend onheil}\\

\haiku{Het is bijna kwart.}{voor negen als Dorothee de}{kamer binnen komt}\\

\haiku{Ik herinner me,.}{trouwens heel goed dat ze het}{vanmorgen nog droeg}\\

\haiku{Ik zelf heb haar, zij,.}{het dan ook bij toeval dien}{middag gevonden}\\

\haiku{Het is reeds kwart voor.}{elf en Rudolf is dus al}{drie kwartier over tijd}\\

\haiku{{\textquoteleft}Ik zal even met je,.}{mee naar boven gaan dan kan}{ik je verbinden}\\

\haiku{Daarna ontspannen.}{zijn trekken zich en komt hij}{langzaam naar me toe}\\

\haiku{{\textquoteright} {\textquoteleft}All right,{\textquoteright} brom ik en.}{sta op om de deur achter}{hem op slot te doen}\\

\haiku{{\textquoteright} {\textquoteleft}Toen,{\textquoteright} vervolgt Reggie, {\textquoteleft},:}{toen deed ik voorloopig}{niets dat wil zeggen}\\

\haiku{Daardoor groeiden m'n.}{vermoedens omtrent Lucien}{tot zekerheid aan}\\

\haiku{Zooals je weet, had hij.}{op den avond van den moord bij}{een vriend gedineerd}\\

\haiku{aanvankelijk had.}{hij niet de bedoeling om}{haar te gebruiken}\\

\section{Frank Martinus Arion}

\subsection{Uit: Afscheid van de koningin}

\haiku{De koningin was.}{er ook de laatste keer dat}{ik in Songo was}\\

\haiku{Ze is tweemaal zo,.}{groot als Nederland met 3}{miljoen inwoners}\\

\haiku{Of pernods, want!}{Gaston Senyo Wawili}{was zeer francofiel}\\

\haiku{De laatste resten.}{van mijn jeugd achterhaalden}{mij in die balzaal}\\

\haiku{Dat was het enige.}{wat ik voor de koningin}{niet kon opbrengen}\\

\haiku{Ze waren blijkbaar.}{andere reacties op}{hun testvraag gewend}\\

\haiku{Maar het was nu ook!}{een belangrijke zaak die}{aan de orde was}\\

\haiku{Ik ben journalist!}{zoals u ziet en ik wil}{mijn krant opbellen}\\

\haiku{{\textquoteleft}Ik geef er meer om,,}{zo gezellig hier met u}{te zitten praten}\\

\haiku{Maar ze was voor een!}{studente zoals mij voor}{ogen stond te zwijgzaam}\\

\haiku{Als je per se iets,?!}{wilt drinken kunnen we een}{flesje meenemen}\\

\haiku{{\textquoteleft}Maar waarom wil je,?}{met me pr\'aten je weet toch}{al dat ik je mag}\\

\haiku{Ze werkte op haar.}{rug en ze verwachtte van}{mij een bijdrage}\\

\haiku{Ik had voldoende.}{geld om het een tijdlang uit}{te kunnen zingen}\\

\haiku{Door mijn studie in.}{de economie ontdekte}{ik dat allemaal}\\

\haiku{Ik ben misschien te.}{jong om jou helemaal te}{kunnen begrijpen}\\

\haiku{Anders zou voor haar.}{alleen maar vaststaan dat ik}{haar als vrouw afwees}\\

\haiku{Ze sloeg me hard op:}{m'n dij met haar vlakke hand}{en zei kwasi boos}\\

\haiku{Een handig ding ook.}{voor plaatsen waar je niet mag}{fotograferen}\\

\haiku{{\textquoteright} Hij onderzoekt het.}{effect van zijn woorden eerst}{even in zijn spiegel}\\

\haiku{Dan zou de dood van?}{die anderen eenvoudig}{een ongeluk zijn}\\

\haiku{Ga maar eens kijken!}{in de St. Pieter als de}{paus de mis opdraagt}\\

\haiku{ik heb nog meer van{\textquoteright}.}{hetzelfde als jullie ook}{durven opkomen}\\

\haiku{Maar Ali lijkt me in.}{ieder geval een echte}{mensenverkoper}\\

\haiku{Interessant hoe,?!}{de dingen soms kunnen}{samenvallen he}\\

\haiku{We wilden onze.}{honeymoon in april \`en in}{Parijs doorbrengen}\\

\haiku{{\textquoteleft}I never knew the.}{charms of spring Never met it}{face to face}\\

\haiku{Anders hadden we{\textquoteright} {\textquoteleft}.}{toch-In Amsterdam is}{het ook niet alles}\\

\haiku{Maar er waren wel,?}{niet zoveel zwarten daar toen}{als nu waarschijnlijk}\\

\haiku{En slim ontwijken.}{zoals hij trouwens ook doet}{als het moeilijk wordt}\\

\haiku{{\textquoteright} {\textquoteleft}Interessant,{\textquoteright} zegt, {\textquoteleft}:}{hij weer duidelijk verrast}{heel interessant}\\

\haiku{Zo'n klein land... zoveel,.}{grote ondernemingen}{zoveel partijen}\\

\haiku{Hij zet daarna zijn.}{bril op en onderzoekt het}{allemal opnieuw}\\

\haiku{Op tennisgebied.}{presteren ze ook wel wat}{met hun Tom Okker}\\

\haiku{die bars in te gaan,,.}{als u er bent maar u moet}{w\`el voorzichtig zijn}\\

\haiku{Zo zelfs, dat ik een.}{soort ambassadeur van mijn}{land ben geworden}\\

\haiku{Dit hele gesprek,.}{met je geeft me veel stof tot}{nadenken weet je}\\

\haiku{Ik heb dat wel meer,.}{horen beweren maar het}{is volstrekt onzin}\\

\haiku{Misschien wil ze ook.}{gewoon een keer een zwarte}{man goed verwennen}\\

\haiku{Zeg maar, dat ik de.}{situatie toch wel wat}{humoristisch vind}\\

\haiku{Ze probeert opnieuw,.}{te lezen maar legt het boek}{gauw weer op haar schoot}\\

\haiku{Met irre\"ele.}{getallen zoals dat in}{de wiskunde gaat}\\

\haiku{Niet helemaal zwart,.}{zoals ik mezelf voor het}{gemak wel beschrijf}\\

\haiku{6 Direct na ons.}{vertrek krijgen we deze}{keer een warme lunch}\\

\haiku{Ik zeg wanhopig!}{tegen mezelf dat de vrouw}{hier naast me dom is}\\

\haiku{Daarom was ik zo.}{verbaasd toen u me vroeg of}{ik B\'elgische was}\\

\haiku{En die Bobbejan,.}{gaat de rooien verslaan de}{Indianen dus}\\

\haiku{Het wordt allemaal.}{door anderen uitgemaakt}{v\'o\'or je geboorte}\\

\haiku{{\textquoteleft}Ja,{\textquoteright} zeg ik, {\textquoteleft}en in.}{Nederland ken ik iemand}{die Du Plessis heet}\\

\haiku{Want misschien wil ik,{\textquoteright}.}{er inderdaad zelf een keer}{heen zegt ze lachend}\\

\haiku{{\textquoteleft}Neen, blijft u maar bij,.}{het raam zitten dan kan ik}{ook naar buiten zien}\\

\haiku{{\textquoteleft}Ik doe dus hard m'n,.}{best om z\'o uit te komen}{dat begrijpt u wel}\\

\haiku{Het is logisch aan,.}{te nemen dat hij dat niet}{zo leuk zal vinden}\\

\haiku{Balthasar Vorster.}{zich voor zijn overtuiging in}{zeer slecht gezelschap}\\

\haiku{Ambitieuze.}{mensen zijn inderdaad vaak}{opportunistisch}\\

\haiku{{\textquoteleft}Moeder is al zo,{\textquoteright}, {\textquoteleft}.}{oud zeggen zeen ze is}{ook zo ziekelijk}\\

\haiku{N\'og niet tenminste,.}{daarom draai ik voorlopig}{nog om Nederland}\\

\haiku{In deze uren ben.}{ik een beetje een ander}{mens aan het worden}\\

\haiku{Dat is alles wat.}{ik weet en dat ik niemand}{aan land mag laten}\\

\haiku{Ik vraag me af, wat!}{ze in siaieesnaam daar}{hopen te vinden}\\

\haiku{{\textquoteright} Ik wijs met mijn hoofd.}{naar de soldaten v\'o\'or het}{stationsgebouw}\\

\haiku{En kijk, ze halen,.}{ook de koffers er uit dat}{is straks niet gezegd}\\

\haiku{Terwijl we naar het,.}{stationsgebouw lopen}{herhaal ik de vraag}\\

\haiku{{\textquoteright} {\textquoteleft}Wawili's dood was, -?}{dus niet zomaar een moord maar}{het duidelijke}\\

\haiku{Nota bene een.}{vrouw die zich zeer inspande}{voor de mensen hier}\\

\haiku{En nu hebben die.}{Ieren haar in  opdracht}{van Bakari vermoord}\\

\haiku{The Barrel of a, {\textquoteleft}.}{Guninterventie is geen}{pl{\'\i}cht van de Fransen}\\

\haiku{Waarom hebt u hem?}{niet gezegd dat ik zo graag}{naar Tamina wil}\\

\haiku{Als ik besluit me}{van het een en ander te}{gaan vergewissen}\\

\haiku{Deze hier die de,.}{leiding heeft ging meteen voor}{hem in de houding}\\

\haiku{Als ik terug ben,.}{staat mevrouw Jobert ook naar}{het lijk te kijken}\\

\haiku{U woonde niet bij,?}{uw dochter maar in Hotel}{Tamina nietwaar}\\

\haiku{Ik dacht eigenlijk.}{dat ze u en mevrouw hier}{ook geraakt hadden}\\

\haiku{En met de wens voor.}{een happy landing kondigt}{hij het diner aan}\\

\haiku{{\textquoteright} {\textquoteleft}Ja, het laatst werkte;}{u met de vrouwen op de}{markt of voor de fao}\\

\haiku{Nieuw Nederland, een,.}{g\'oede krant en verder van}{alles zo'n beetje}\\

\haiku{Mevrouw Prior lacht.}{hardop en genietend van}{onze verbazing}\\

\haiku{Vlak nadat ik weg.}{was uit Holland waren er}{nog demonstraties}\\

\haiku{Dat is het gebied,.}{tussen Enkhuizen en Hoorn}{waar mijn broer nu zit}\\

\haiku{Eerst nog met een muur,,.}{dus half om half maar daarna}{helemaal van glas}\\

\haiku{Op een hoek van de,.}{straat in de Transvaalbuurt dat}{is Amsterdam-Oost}\\

\haiku{Je zegt maar dat die,!}{orchidee\"en morgenvroeg}{gebracht worden hoor}\\

\haiku{E\'en man speciaal.}{voor de bestellingen en}{twee in de winkel}\\

\haiku{Het zaken doen wordt.}{er niet gemakkelijker}{op tegenwoordig}\\

\haiku{Ja, dat is de meer.}{moderne manier zoals}{u het haar nu heeft}\\

\haiku{Ik verwissel ook.}{het cassettetje van mijn}{kleine apparaat}\\

\haiku{De katholieke.}{bond van landarbeiders was}{altijd zo langzaam}\\

\haiku{Die jongen was nog!}{donkerder dan u. Maar een}{aardige jongen}\\

\haiku{Als hij kwam en ik,:}{stond nog in de winkel dan}{kwam hij naar me toe}\\

\haiku{De mensen uit het.}{noorden en oosten komen}{die vis hier kopen}\\

\haiku{Ik zeg tegen m'n,:}{dochter nadat ik een twee}{maanden bij haar ben}\\

\haiku{Het is zo dat het.}{vuil bij hen soms in weken}{niet wordt opgehaald}\\

\haiku{Daarom wilde ik.}{nu eens te voet en op mijn}{gemak alles zien}\\

\haiku{Of de bestaande.}{situatie zelfs probeert}{te verdedigen}\\

\haiku{Vooral als het even.}{mis gaat met de oogst is er}{grote mis\`ere}\\

\haiku{Ook om Brigitte,.}{Bardot die dus eigenlijk}{schapenhoedster is}\\

\haiku{{\textquoteright} zegt mevrouw Prior, -! -.}{terwijl ik de foto veel}{te lang bestudeer}\\

\haiku{Ze denkt diep na, om,.}{daarna te herhalen dat}{het dat echt niet was}\\

\haiku{Ze had ook gehoopt.}{bij mijn dochter nederlands}{te kunnen leren}\\

\haiku{Enfin, Gadizha.}{heeft toch wel een paar woorden}{van me opgepikt}\\

\haiku{Ik zeg terwijl ik:}{een nieuwe cassette in}{de recorder doe}\\

\haiku{Daar moest hij zichzelf,.}{z'n vrouw en twee kinderen}{van onderhouden}\\

\haiku{Want de mensen in!}{Nanik\'e gingen allemaal}{mee verzamelen}\\

\haiku{Ik ging gewoon uit.}{van wat orchidee\"en in}{Holland opbrengen}\\

\haiku{Als mijn man er nog,,.}{maar was dacht ik soms want die}{wist daar alles van}\\

\haiku{Maar ik zag niet in!}{dat ik welke economie}{dan ook ontwrichtte}\\

\haiku{Ik was al zo lang,.}{in Hotel Tamina dat}{ik m'n gang mocht gaan}\\

\haiku{Je hebt in een van,.}{die straten de winkel van}{een Syri\"er Omar}\\

\haiku{En sommige van.}{die jonge mensen van het}{vrijwilligerskorps}\\

\haiku{{\textquoteleft}Die mensen doen dat,.}{allemaal uit vrije wil dat}{is heel wat anders}\\

\haiku{Naomi's kin is rond.) {\textquoteleft}.}{en haar neus kortAllemaal}{opportunisme}\\

\haiku{Na die aanslag op.}{Wawili zijn links en rechts}{mensen opgepakt}\\

\haiku{Er wordt gezegd dat -{\textquoteright} {\textquoteleft},.}{een deel van het legerJa}{dat is de l\'andmacht}\\

\haiku{Ik geeft haar een hand.}{om haar te bedanken voor}{het interview}\\

\haiku{Bij de staatsgreep in,!}{Chili had je ook zo een}{die maar zijn gang ging}\\

\haiku{{\textquoteright} {\textquoteleft}Mijn ervaring geldt.}{vooral de ontvangende}{kant van de tafel}\\

\haiku{En ik wilde je;}{het een en ander rustig}{kunnen uitleggen}\\

\haiku{Daarom heb ik je.}{ook niet gewaarschuwd dat die}{kolonel er was}\\

\haiku{dat mevrouw Prior.}{hoe dan ook gevaar liep in}{Hotel Tamina}\\

\haiku{{\textquoteright} {\textquoteleft}Ach kom nou, Sesa,{\textquoteright} {\textquoteleft},.}{nu ga j{\'\i}j te ver.Neen als}{ze t\'och vrienden zijn}\\

\haiku{Ik heb haar beloofd.}{haar over enkele dagen}{weer op te zoeken}\\

\haiku{Naomi weigert de.}{slaapkamer te nemen als}{ik haar die aanbied}\\

\haiku{Al vraag ik me af:}{of hun hele geloof niet}{\'e\'en groot excuus is}\\

\haiku{Toen dacht ik, deze!}{zwarte man is toch echt een}{grote gentleman}\\

\haiku{Ik zou op den duur.}{toch met een Zuidafrikaan zijn}{getrouwd waarschijnlijk}\\

\haiku{En ik heb toch nooit.}{iets gedaan om het met haar}{weer goed te maken}\\

\haiku{Ik heb moeite om,}{te verbergen wat ik weet}{maar de zekerheid}\\

\haiku{Ze wentelt haar hoofd,,.}{met de ogen dicht woest heen en}{weer op het kussen}\\

\haiku{Ze zegt, dat ze toch.}{niet denkt lang in Nederland}{te willen blijven}\\

\haiku{Ik wilde toen nog.}{eerst wachten totdat ik een}{tijd was weggeweest}\\

\haiku{Bijna zo beheerst,,.}{als Jozef Maria zo heb}{ik haar behandeld}\\

\haiku{{\textquoteleft}Ik ben ook al oud.}{en wijs genoeg om ervoor}{te kunnen z\'orgen}\\

\haiku{Het geeft je dus toch?}{een beetje voldoening dat}{ik niet terugga}\\

\haiku{{\textquoteright} {\textquoteleft}En omdat ik jouw,{\textquoteright}, {\textquoteleft}.}{moed bewonder zeg ikniet}{uit zwakte alleen}\\

\haiku{Ik vertel haar van.}{Colombia waar ik ook een}{tijd gezeten heb}\\

\haiku{{\textquoteright} Naomi springt naakt uit.}{bed en haalt de recorder}{uit de werkkamer}\\

\section{Diederik van Assenede}

\subsection{Uit: Floris en Blancefloer}

\haiku{Ze zouden liever.}{dood zijn dan lang van elkaar}{gescheiden te zijn}\\

\haiku{Maar ze had er niets.}{van gemerkt dat er zo over}{haar gesproken werd}\\

\haiku{op het graf waren,.}{lange buizen gemaakt waar}{de wind doorheen blies}\\

\haiku{Er werd ook een boom -!}{geplant zo vindt men er niet}{\'e\'en in heel het land}\\

\haiku{Floris vond het graf,:}{heel mooi hij zag de letters}{en las wat er stond}\\

\haiku{{\textquoteleft}Floris, m'n lieve,!}{kind wat heb je een dwaze}{liefde gekoesterd}\\

\haiku{Hij zal zijn verdriet,.}{helemaal vergeten u}{zult het spoedig zien}\\

\haiku{En, heer, ik vraag u}{en mijn moeder bovendien}{me te vertellen}\\

\haiku{Maar Floris was er.}{blij om dat ze nog leefde}{en niet gedood was}\\

\haiku{Toen de maaltijd was,.}{bereid werden de grote}{tafels opgezet}\\

\haiku{Mijn liefde voor haar,.}{houdt me zo in haar ban dat}{ik ben gaan zwerven}\\

\haiku{Hij is gemaakt van,.}{rood marmer en heel mooi rond}{op een rond voetstuk}\\

\haiku{Zodra er een bloem,,.}{valt of geplukt wordt groeit er}{weer een nieuwe aan}\\

\haiku{U kunt begrijpen,,!}{hoe bang Floris de kroonprins}{van Spanje wel was}\\

\haiku{Hij is mijn troost en,!}{toeverlaat heel mijn geluk}{is in zijn handen}\\

\haiku{Wanneer ze iemand}{goedgezind is geweest en}{hem heeft toegestaan}\\

\haiku{Ondertussen was.}{haar vriendin Clarijs snel naar}{de zuil gelopen}\\

\haiku{Je kunt geen levend.}{schepsel bedenken of het}{was daar afgebeeld}\\

\haiku{de veelgeprezen,.}{Blancefloer die hierboven}{in mijn toren woont}\\

\haiku{Ik trok het zwaard, ze.}{werden wakker en smeekten}{me om genade}\\

\haiku{Hoewel de emir zeer,.}{vertoornd was vond hij het zelf}{ook verschrikkelijk}\\

\haiku{De beeldhouwwerken;}{op het graf dragen deze}{bloemen in de hand}\\

\haiku{mijn alderliefste,.}{die ick oyt aensach Adieu het}{moet ghescheyden zijn}\\

\haiku{W.P. Gerritsen \& A.G. (),.}{van Mellered. Van Aiol}{tot de Zwaanridder}\\

\section{T. Avany}

\subsection{Uit: Dolly de danseres}

\haiku{Veronderstel je,}{ook maar \'e\'en moment dat het}{je zou gelukken}\\

\haiku{Hoe heb je het toch,,{\textquoteright}}{klaargespeeld je zoo vlug te}{verkleeden Dolly}\\

\haiku{{\textquoteleft}U heeft ons eenige.}{momenten van het grootste}{genot geschonken}\\

\haiku{Hoe zou ik mij met,?}{U kunnen meten ik ben}{toch geen danseres}\\

\haiku{Zij hield even op als,.}{verwachtte zij een antwoord}{doch hij bleef zwijgen}\\

\haiku{Na afloop van den.}{maaltijd ging het geheele}{gezelschap uiteen}\\

\haiku{ik had je beloofd.}{je een kijkje achter de}{schermen te gunnen}\\

\haiku{Ik ga even achter,;}{dit scherm staan dan zie je mij}{straks verkleed voor je}\\

\haiku{{\textquoteright}, riep Edith verbaasd uit,.}{te zeer verbluft om nog iets}{anders te zeggen}\\

\haiku{Je leek in dat hemd,,.}{vooral door die lichtstralen}{bijna geheel naakt}\\

\haiku{ik kan het denkbeeld,.}{niet van mij afzetten dat}{hij een masker draagt}\\

\haiku{{\textquoteleft}Wat heeft Clarcke,?}{U eigenlijk misdaan dat}{U hem steeds ontwijkt}\\

\haiku{haar woorden hadden.}{blijkbaar niet geheel hun}{uitwerking gemist}\\

\haiku{Hij wilde het doen,.}{voorkomen alsof zij juist}{de trap opkwamen}\\

\haiku{{\textquoteleft}U gevoelt, geloof,,.}{ik meer voor paardensport als}{ik mij niet vergis}\\

\haiku{{\textquoteright} Op dat oogenblik,.}{verscheen een der bedienden}{die zoekend rondkeek}\\

\haiku{Was je reeds daar, toen?}{Mr. Harold Wright met Mr. Dunn}{de kamer verliet}\\

\haiku{Die viel natuurlijk,{\textquoteright}.}{buiten het verbod voegde}{hij er terloops bij}\\

\haiku{{\textquoteright} vroeg hij den heer, die.}{reeds toebereidselen voor}{zijn vertrek maakte}\\

\haiku{Hij schoof een stoel bij.}{de tafel en nam zelf op}{een andere plaats}\\

\haiku{De tweede sigaar.}{stak U aan met het laatste}{puntje der eerste}\\

\haiku{De detective,.}{trad nu binnen waar Edith hem}{reeds tegemoettrad}\\

\haiku{{\textquoteright} Zij opende de deur.}{verder en noodigde hem uit}{binnen te treden}\\

\haiku{Verruimd haalde zij,.}{adem toen de deur zich achter}{hem had gesloten}\\

\haiku{{\textquoteleft}Wacht,{\textquoteright} vervolgde zij,, {\textquoteleft}.}{haar tweede kous uittrekkend}{ik zal eens even zien}\\

\haiku{{\textquoteright} Zij keek haar vriendin,.}{die haar nog steeds omkneld hield}{ook thans nog niet aan}\\

\haiku{Vera lachte eveneens,.}{doch het scheen haar niet geheel}{van harte te gaan}\\

\haiku{{\textquoteright} {\textquoteleft}Persoonlijk heb ik,{\textquoteright}.}{U echter nimmer ontmoet}{merkte Dolly op}\\

\haiku{Dat is immers een,?}{der dagen waarop ik zal}{moeten optreden}\\

\haiku{voor de pauze het,.}{symphonie-orkest na de}{pauze Miss Forest}\\

\haiku{{\textquoteleft}Is er misschien nog,?}{iets anders waarmede ik}{U van dienst kan zijn}\\

\haiku{Hoe dicht onder mijn,{\textquoteright}.}{bereik heb ik haar gehad}{knarsetandde hij}\\

\haiku{Toen ik het kantoor,.}{verliet sprak ik hem nog even}{in de wachtkamer}\\

\haiku{{\textquoteleft}Kom hier eens naast mij,.}{zitten want ik heb iets met}{je te bespreken}\\

\haiku{Sinds ik jou heb leeren,,}{kennen kan er geen ander}{voor mij meer bestaan}\\

\haiku{Bovendien, als je,, -}{je werk goed verricht zal je}{na je harde taak}\\

\haiku{{\textquoteright} Wright, die inzag, dat,.}{hij een domheid begaan had}{knikte levendig}\\

\haiku{{\textquoteright}, riep hij opgewekt,.}{uit met uitgestoken hand}{op haar toetredend}\\

\haiku{Ik moet nog vijf maal.}{optreden en vertrek dan}{naar Baltimore}\\

\haiku{of ik daarna naar,.}{New-York terug zal}{keeren weet ik nog niet}\\

\haiku{Wright fluisterde haar,.}{iets in het oor waarop Vera}{toestemmend knikte}\\

\haiku{Langzaam wandelde,.}{hij verder twijfel was bij}{hem opgekomen}\\

\haiku{Heb je overigens?}{nooit iets ten nadeele van Miss}{Forest opgemerkt}\\

\haiku{Dan is mij ook haar.}{houding tegen over Dunn steeds}{een raadsel geweest}\\

\haiku{{\textquoteright} Harrison volgde,.}{den ander in de gang waar}{Wright hem staande hield}\\

\haiku{Plotseling zag hij:}{den door hem te volgen weg}{duidelijk voor zich}\\

\haiku{Wonderlijk genoeg.}{bleek deze geen letsel te}{hebben bekomen}\\

\haiku{{\textquoteright} Voor hij zelf wist, wat,:}{dit eigenlijk beteekende}{had hij geantwoord}\\

\haiku{Haar mantel liet zij.}{met een losse beweging}{achter zich glijden}\\

\haiku{{\textquoteright} Dolly slaakte een.}{zucht en haar gezicht toonde}{duidelijk onrust}\\

\haiku{{\textquoteleft}Althans niet dat gij.}{in staat zoudt zijn iemand leed}{te berokkenen}\\

\haiku{Inspecteur Maxwell,,.}{die het onderzoek leidt is}{van haar schuld overtuigd}\\

\haiku{Zooals U wellicht weet,,.}{is het mij niet om geld te}{doen Mr. Harrison}\\

\haiku{Men heeft mij verteld,,.}{dat het Miss Forest was die}{Mr. Wright vermoordde}\\

\haiku{Punt 1 en punt 2,.}{hingen samen terwijl punt}{3 op zichzelf stond}\\

\haiku{Er is druk werk voor,,{\textquoteright}.}{vanavond Gladys lichtte Jack}{de jonge vrouw in}\\

\haiku{Doch daar kan het niet,.}{van zijn want dan zou zij ook}{ziek zijn geworden}\\

\haiku{{\textquoteright} Harrison voelde.}{zich  een siddering door}{het lichaam loopen}\\

\haiku{Hij, die het briefje,.}{met het teeken trekt zal de}{gelukkige zijn}\\

\haiku{{\textquoteleft}Blijf nog wat liggen,,.}{Miss Forest totdat U zich}{wat sterker gevoelt}\\

\haiku{Jack reikte haar het.}{glas aan en noodigde haar uit}{het te ledigen}\\

\haiku{Uw verklaringen,.}{zullen aantoonen of ik}{daarin gelijk heb}\\

\haiku{Het was toen, dat ik,.}{op het denkbeeld kwam mij naar}{U te begeven}\\

\haiku{Zijn blik scheen tot in.}{het diepst van haar ziel te}{willen doordringen}\\

\haiku{{\textquoteright} {\textquoteleft}Aha, ik geloof, dat,{\textquoteright}.}{Harrison's kansen nog zoo slecht}{niet staan schertste Jack}\\

\haiku{Ik voel, dat gij een,.}{persoon zijt die hiervan geen}{misbruik zal maken}\\

\haiku{{\textquoteleft}Hoevelen in haar?}{plaats zouden niet moreel ten}{gronde gegaan zijn}\\

\haiku{Hij liet zich op een.}{knie vallen en drukte haar}{hand aan zijn lippen}\\

\haiku{Op luidruchtige;}{wijze onderhielden zij}{zich met elkander}\\

\haiku{{\textquoteright} klonk het thans helder,.}{zoodat het door het geheele}{vertrek hoorbaar was}\\

\haiku{Zij wachtte even en,.}{keek naar den sheik die haar als}{versteend aanstaarde}\\

\haiku{Ik zal dan ook niet.}{de moeite nemen daar thans}{op te antwoorden}\\

\haiku{Als gij mijn verhaal,.}{gehoord hebt zult gij het mij}{moeten toegeven}\\

\haiku{{\textquoteleft}Verspil Uw woorden,,.}{niet Mr. Harrison het is}{noodelooze moeite bij haar}\\

\haiku{De buitenpartij.}{bij Wright stelde mij voor tal}{van nieuwe feiten}\\

\haiku{Speciaal gij, Miss,.}{Forest interesseerdet}{mij in hooge mate}\\

\haiku{{\textquoteright} {\textquoteleft}Het dient nergens toe,,.}{op een dergelijken toon}{voort te gaan Mr. Wright}\\

\haiku{Miss Forest zocht U,.}{op en deelde U mede}{wat er gebeurd was}\\

\haiku{Het was een val die.}{ik voor hem had opgezet}{en hij liep er in}\\

\haiku{Ik heb nog een en.}{ander onder vier oogen met}{U te bespreken}\\

\chapter[52 auteurs, 9460 haiku's]{tweeënvijftig auteurs, negenduizendvierhonderdzestig haiku's}

\section{Franz de Backer}

\subsection{Uit: Longinus}

\haiku{O, daar was een God,,!}{die m\`ensch was en stierf in de}{wrangste eenzaamheid}\\

\haiku{Wat onmiddellijk.}{daarna gebeurde is me}{niet zeer duidelijk}\\

\haiku{Ik trok mijn zwaard, maar,,.}{hij bewoog niet keek me even}{aan keek naar het zwaard}\\

\haiku{Wij speculeerden,.}{over de kansen dat wij geen}{gevecht zouden zien}\\

\haiku{Die heel bleeke, stille,,.}{knaap bewusteloos maar zijn}{lippen bewogen}\\

\haiku{- {\textquoteleft}'t Is wel te zien,,.}{sergeant dat gij nog niet}{gekwetst geweest zijt}\\

\haiku{Plots, v\'o\'or ons, bij den,.}{vijand een dof gerucht van}{werkende schoppen}\\

\haiku{Ik kon lang proeven,.}{aan dien troost dien ik diep in}{mij verdoken hield}\\

\haiku{Van huis geen nieuws, mijn, -,.}{streek bezet niets dan oorlog}{oorlog v\'o\'or mijn neus}\\

\haiku{zij hadden meer kracht,,,:}{dan ik meer moed of soms meer}{lafheid maar altijd}\\

\haiku{* * * ~ Die stemming werd.}{een wijl vergeten toen ik}{zelf met verlof ging}\\

\haiku{Dites-moi o\`u,,,...}{n'en quel pays Est Flora la}{belle Romaine}\\

\haiku{- mijn hals was \'e\'en groote.}{wonde waar de kogel was}{buitengebroken}\\

\haiku{de loopgraven die.}{ik verlaten had werden}{niet aangevallen}\\

\haiku{Twee vingers van mijn,.}{linker hand ze lagen op}{mijn bloedigen buik}\\

\haiku{Wat ieder van u,.}{draagt is absoluut noodig om}{te kunnen schieten}\\

\haiku{De jongen kruipt weg,,.}{terug in de vlakte van}{dood zonder \'e\'en woord}\\

\haiku{, en het koppel dat,,}{hem belet had vertelt ook}{en met stelligheid}\\

\section{Lode Baekelmans}

\subsection{Uit: Het geheim van 'De drie snoeken'}

\haiku{We zullen zeker,.... -!}{getwee\"en moeten zitten}{dubben terwijl Zwijg}\\

\haiku{Een paar dagen bleef.}{de winkel gesloten en}{Putzeys onzichtbaar}\\

\haiku{Silberfeld was niet,.}{te vinden maar Bollekens}{liep hij op het lijf}\\

\haiku{- Maar toch slecht.... - Vergeet!}{niet dat zij de zuster van}{een Kolonel is}\\

\haiku{Intusschen had zij.}{zich gekleed en sloot zij haar}{pruik over den schedel}\\

\haiku{Een vrouw van de straat,,.}{mocht zij wel getuigen die}{geld noch eer bezat}\\

\haiku{Haar ziel verging van.}{onrust en haar gedachten}{waren in de war}\\

\haiku{zij had geen eetlust.}{en geen verlangen om op}{visite te gaan}\\

\haiku{- Maar zeg het dan toch,,.}{drong Mademoiselle aan}{ik verga van angst}\\

\haiku{Zoo zijn de menschen,,,.}{dacht Mademoiselle uit}{het oog uit het hart}\\

\haiku{In het troebel beeld.}{herkende zij de vrouw van}{het portret niet meer}\\

\haiku{Rubens was moe na.}{een langen namiddag van}{carnaval plezier}\\

\haiku{- Kom maar mee, verzocht,....}{Rubens we zijn vlak in de}{buurt van mijn atelier}\\

\haiku{'t Waren alle.}{jonge kerels met lang haar}{en vreemde hoeden}\\

\haiku{Wat mag hij van mij,.}{wel elders vertellen vroeg}{zich de schilder af}\\

\haiku{Hoe hoog voelden zij!}{zich verheven boven den}{gewonen bourgeois}\\

\haiku{De burgers keken.}{toegeeflijk de joelige}{artistenjeugd na}\\

\haiku{een meisje dat zong:}{op het Burchtplein onder de}{kale boomen}\\

\haiku{Het gebeurde wel.}{dat Mieke Serafijn met}{Rubens vergeleek}\\

\haiku{De stilte hing als.}{een eindelooze diepte in}{den wijden hemel}\\

\haiku{Nooit sprak Rubens over. '}{de ontrouwe vrouw noch over}{zijn drinkgelagen}\\

\haiku{Met den Stadhuisklerk.}{ging hij de gedachte aan}{Parijs wegspoelen}\\

\haiku{Het orkaan snokte,.}{en joelde dreef de regen}{tegen de vensters}\\

\haiku{De zeep schuimde over.}{zijn gelaat en hij kneep de}{oogen dicht van plezier}\\

\haiku{- Hundenwetter, zei,.}{ze zonder de handen uit}{den schoot te lichten}\\

\haiku{Zij ziet er nog goed,,....}{uit oordeelde de man zij}{is amper veertig}\\

\haiku{Op een dooden tak zat,.}{een verloren roodborstje}{dik in de veeren}\\

\haiku{Lui koesterde zich.}{de oude herdershond in}{de zon voor het huis}\\

\haiku{Wekelijks bracht de.}{bode met zijn vrachtwagen}{de boodschappen mee}\\

\haiku{Ik wist op voorhand.}{dat het tevergeefs was en}{toch hoopte ik nog}\\

\haiku{Ik zei immers dat....}{het een belachelijke}{geschiedenis was}\\

\haiku{hoe zijn papieren.}{reddeloos door het water}{bedorven werden}\\

\haiku{Koko had toch veel.}{smaak en een zekeren chic}{om zich te kleeden}\\

\subsection{Uit: De idealisten}

\haiku{De kapitein stond.}{juist boven de kajuit zijn}{sigaar te rooken}\\

\haiku{In den dag zag ik,.}{niets maar den volgenden avond}{herbegon het spel}\\

\haiku{Soms kwam hij dan toch.}{eens onverwacht kijken tot}{aller verrassing}\\

\haiku{Misschien... misschien... - Ik,!}{zoek naar iets anders iets om}{meer te verdienen}\\

\haiku{het was plezant met... -,.}{Hortense Dat geloof ik}{zei Max pinkoogend}\\

\haiku{Herder, ik haat geen,!}{menschen maar ik kan er ook}{geen liefhebben}\\

\haiku{Ik ben een oud man!}{en amuseer mij maar alleen}{tusschen mijn boeken}\\

\haiku{Tusschen uw vrienden!}{staan er veel die elkander}{niet kunnen luchten}\\

\haiku{Hij kloeg altijd over...}{eenzaamheid en verdorde}{tusschen de boeken}\\

\haiku{De man had een bril.}{met donkere glazen die}{zijn oogen verborgen}\\

\haiku{Toen zij het waagden}{het hoofd te wenden zagen}{zij een vrouw tusschen}\\

\haiku{Doch ditmaal hoorden. '.}{zij geen glasgerinkelt}{Gaf een opluchting}\\

\haiku{Maar als de vrouw weer.}{in de kamer kwam zat hij}{rustig te rooken}\\

\haiku{ik meende in den... -,.}{kelder nog wat te hooren}{Bepaald niets Mevrouw}\\

\haiku{Zij namen hun glas.}{en stonden aarzelend in}{de roode schaduw}\\

\haiku{de zedelijkste,;}{want zij brengt niet mede het}{slachten van dieren}\\

\haiku{altijd werken... - Ja,,... -}{onderwierp zich de vrouw de}{jongens kosten geld}\\

\haiku{Zij wuifde toen de,.}{tram afreed dan ging zij in}{het rijtuig zitten}\\

\haiku{Mieke, die den stoet,.}{had zien voorbijtrekken was}{er door aangedaan}\\

\haiku{Te bed lagen zij.}{wakker onder de spanning}{hunner zenuwen}\\

\haiku{Dat neem ik niet mee,,.}{peinsde zij nu beginnen}{wij een nieuw leven}\\

\haiku{- Ik weet het wel, gaf,.}{de Zaalwachter toe het was}{maar uit gewoonte}\\

\haiku{- Gij zijt nog een van,.}{de ouden fluisterde de}{Portier met warmte}\\

\haiku{Zij snoepte nog een,.}{stukje koek trok het licht uit}{en kroop in haar bed}\\

\haiku{Amelia fluisterde,.}{iets wat hij niet verstond want}{hij was een Duitscher}\\

\haiku{Zij stond op, deed het.}{vuur branden en zette de}{koffieketel op}\\

\haiku{Rond twaalven viel de.}{muziekdoos stil en kraamden}{de bezoekers op}\\

\haiku{ik was de eenige...}{stoker die geen jenever}{dronk en boeken kocht}\\

\haiku{Een hoeve, land en,!...}{weiden een schuur en een stal}{en wel zes peerden}\\

\haiku{En nu moet ik weg,... -!}{want ik heb nog veel te doen}{Neem nog een sigaar}\\

\haiku{Zij dronken opnieuw,.}{op de vriendschap en op de}{schoone ingeving}\\

\haiku{Boven de deur in,.}{kleur van geronnen bloed stak}{de vlaggestok uit}\\

\haiku{ik moet rondtrekken... -?}{en drinken als ik cens heb}{Gaan de zaken slecht}\\

\haiku{Het venster sloot zij,.}{draaide de lamp neer die met}{een lichtflap uitging}\\

\haiku{Zijn haren waren.}{kort geknipt en zijn klak zat}{diep op zijn voorhoofd}\\

\haiku{De rechter zei wat,.}{en hij antwoordde wat maar}{ik luisterde niet}\\

\haiku{Ik verliet de zaal.}{en zag niet verder om naar}{mijn gevangene}\\

\haiku{Juffrouw Augusta stond.}{alleen in haar winkel en}{staarde voor zich uit}\\

\haiku{Juffrouw Augusta nam,.}{een register bladerde}{er in en knikte}\\

\haiku{Mijnheer Deckers heeft...}{hem zekeren dag in zijn}{kantoor geroepen}\\

\haiku{Terwijl zijn vrouw haar,.}{kerkboek in een lade sloot}{overwoog zij luidop}\\

\haiku{- Ze krijgen een buis,, -,?}{fluisterde hij wat denkt gij}{er van Champetter}\\

\haiku{t is geuzenweer -,.}{Ja ge kunt moeilijk spreken}{in uw positie}\\

\haiku{Hij rook het gebraad.}{en de roode kooltjes en}{was zeer vergenoegd}\\

\haiku{Dan ging plots de vlag {\textquoteleft}{\textquoteright} {\textquoteleft}{\textquoteright}.}{omhoog inDe Engel en}{inDe Witte Leeuw}\\

\haiku{Wij zijn allemaal...}{poesjenellen die dwaze}{grimassen maken}\\

\haiku{- Ja, die zijn bij haar... -?}{moeder En van haar vent heeft}{ze niets meer gehoord}\\

\haiku{k laat haar... - 'k Zou,,!}{dat niet doen Stafke ge zult}{haar zoo'n verdriet doen}\\

\haiku{Zoo won Marie haar.}{Stafke weer en ging met hem}{naar het spektakel}\\

\haiku{Terwijl hij de cens,.}{opstreek fluisterde iemand}{hem wat in het oor}\\

\haiku{- Nu moeten wij toch... -,? -,.}{eventjes in de kerk gaan Ach}{waarom Kom Wagner}\\

\haiku{Aan den staart van den.}{stoet liepen van Dijck en}{Wagner naar buiten}\\

\haiku{- Neen, Edele Heer, zei.}{hoffelijk Van Dijck en}{streek door zijn haren}\\

\haiku{- En vooral vergeet.}{niet mijn complimenten aan}{Johanna te doen}\\

\haiku{Wagner was de baan,...}{op maar zij verwachtte hem}{wel wat vroeger thuis}\\

\haiku{Maar de verliezer.}{had het niet goed begrepen}{en wou herkansen}\\

\haiku{- En groenten uit den,;}{hof en ongedoopte melk}{droomde Katinka}\\

\haiku{- We gaan samen, zei,.}{de Wisselagent we kunnen}{er nog rijk worden}\\

\haiku{Een grijze poes kwam}{uit een hoek te voorschijn en}{vlijde zich neder}\\

\haiku{Maar noch de vischjes, '.}{noch de alcohol hielden}{hem int leven}\\

\subsection{Uit: Robinson}

\haiku{Daar kefte een hond.}{en speelde een schipper op}{zijn harmonika}\\

\haiku{De dag, een dag uit,.}{het Aards Paradijs gevuld}{met eten en drinken}\\

\haiku{En wanneer Fientje,...}{haar portret zou missen had}{hij het verkorven}\\

\haiku{In het vijfde jaar.}{na zijn afscheid bracht het}{toeval hem er weer}\\

\haiku{- Een goeie sloeber... - Wel,! -.}{vooruit dan met de muziek}{Adieu Robinson}\\

\haiku{En hij kon zich niet.}{onthouden een Gandhi zeer}{te bewonderen}\\

\haiku{Had Robinson een,,}{enkele keer de pest in}{wat niet vaak voorviel}\\

\haiku{Want Boy was als de.}{mensen die bij Robinson}{op bezoek kwamen}\\

\haiku{s Winters droeg hij.}{zijn muts op zijn ronde door}{het Schipperskwartier}\\

\haiku{Met zijn neef kon hij.}{beter overweg dan met de}{eigen kinderen}\\

\haiku{Eens, toen het hem te,.}{zwaar geworden was ging hij}{naar een dominee}\\

\haiku{Zij waren alleen.}{in de wereld en wisten}{niet wat aanvangen}\\

\haiku{Maar het angstgevoel.}{voor het hiernamaals kon hij}{niet kwijt geraken}\\

\haiku{Aan de binnenkant.}{van elk deksel werd een klein}{spiegeltje gehecht}\\

\haiku{Robinson sloot de.}{deur en ging wandelen in}{de bewogen stad}\\

\haiku{- Mijn affaire is,,?}{om zeep kloeg hij wat moet ik}{nu gaan aanvangen}\\

\haiku{s Anderendaags.}{ging Robinson de bouw van}{de schipbrug volgen}\\

\haiku{- Of dat ge naar een,.}{familiebegrafenis}{moest gaan vond Vrijdag}\\

\haiku{En het einde was,,.}{toch gekomen en zoals}{altijd onverwacht}\\

\haiku{In het seizoen hield,.}{Vrijdag ook op Zondag open}{althans tot \'e\'en uur}\\

\haiku{- Ge moogt wel dankbaar,.}{zijn meende de Notaris}{te moeten zeggen}\\

\haiku{s Morgens was haar:}{kamer reeds vrij en stonden}{de ramen wijd open}\\

\haiku{En verder vooruit.}{met de muziek tot aan het}{einde van de weg}\\

\haiku{De eenden zochten.}{het water op en pluisden}{onder hun veren}\\

\haiku{Een slag van het net,.}{een gelukkige slag en}{zij zat gevangen}\\

\haiku{De straat oversteken, '!}{of een lange bootreist}{was al hetzelfde}\\

\subsection{Uit: Uit grauwe nevels}

\haiku{Verder sjouwde men;}{reusachtige balen wol}{van plompe wagens}\\

\haiku{Charel had er eene:}{met de bloote hand gevat en}{neep ze den adem af}\\

\haiku{Sander stond op den,;}{drempel zag vragend naar de}{gesloten lucht op}\\

\haiku{Plots in 't licht  :}{van eenen gaslantaarn hield de}{weifelaar hem staan}\\

\haiku{de versiering moest!}{dan kant en klaar zijn en dat}{was geen kleinigheid}\\

\haiku{Wij hebben maar een, '!}{jong leven En als wij dood}{zijn ist gedaan}\\

\haiku{de deur werd achter.}{hun rug toegesloten en}{het licht neergedraaid}\\

\haiku{Het water bruiste;}{onder de wielraderen}{van den overzetboot}\\

\haiku{Lang bespiedden ze;}{den nachtwaker die op-}{en neerwandelde}\\

\haiku{Ze keken naar dien.}{vent die daar hoog stond als den}{meester van het hooi}\\

\haiku{ik wedden, dat er... '}{geen vetter beesten op de}{markt te vinden zijn}\\

\haiku{klonk het gillend en,.}{tierend uit den hoop die nu}{het hazenpad koos}\\

\section{Kamiel van Baelen}

\subsection{Uit: De oude symphonie van ons hart}

\haiku{maar de meesten zijn,.}{zieke menschen we moeten}{daar niet mee lachen}\\

\haiku{Krankzinnig is ie,.}{dan ook al staat dat niet in}{het contract vermeld}\\

\haiku{Kijkt, denkt Lou, die gaan,!}{ze nu ergens op een huis}{leggen de sullen}\\

\haiku{Maar de mis\`ere.}{loopt op drie passen achter}{het geldgebrek aan}\\

\haiku{*** ~ - Wacht even, zegt Lou,.}{ik heb nog een en ander}{te verrichten hier}\\

\haiku{Ook Magda vindt hem,,.}{nu een man d\'e man zij ziet}{zoowaar tegen hem op}\\

\haiku{*** ~ In de Golf van,;}{Gascogne overvalt hen de}{storm niet zoo zacht ook}\\

\haiku{, maar hij voelt zich zoo, '.}{ziek als een hond en werpt zich}{languit opt bed}\\

\haiku{Hij heeft echter mooi,.}{rukken en duwen die lijkt}{wel vastgenageld}\\

\haiku{Maar z'n eerste klop.}{op die gevreesde deur is}{al veel te schuchter}\\

\haiku{De acteur Vreebos:}{heeft tranen in de oogen en}{stoot z'n glas bier om}\\

\haiku{Werd de F\'e bijna,,!}{kwaad om hij had er maar \'e\'en}{geteekend labberdaan}\\

\haiku{Hij heeft alleen een,.}{huis van goud dat is te mooi}{voor een dronkelap}\\

\haiku{Maar eerst wil hij nog,.}{ergens heen de rest van z'n}{drie mille moet op}\\

\haiku{Dacht je werkelijk,?}{dat hij de bons gekregen}{had van z'n meisje}\\

\haiku{Het is stilletjes,.}{gaan motregenen maar hij}{zet z'n kraag niet op}\\

\haiku{Zie je wel, en daar,.}{ligt een boek op den grond ook}{een oude reisgids}\\

\haiku{Maar niet lang meer, hij,.}{stelt zich voor naar het klooster}{te gaan voor broeder}\\

\haiku{Een nieuw hart moet Lou,.}{hebben en een eerlijke}{kans in het leven}\\

\haiku{En onwaarschijnlijk,.}{haast zoo onwaarschijnlijk als}{de waarheid vaak is}\\

\haiku{Hij maakt het open en,.}{v\'o\'or ze rechtstaat hangt hij haar}{een parelsnoer om}\\

\haiku{blad onder F.V. V.}{Uit Het Nieuwe Geluid van}{12 Februari}\\

\haiku{Gisteren werd in.}{de Stationstraat een zwart}{hondje overreden}\\

\haiku{Had mij dan liever,,.}{doodgereden siste hij}{bleek van woede}\\

\haiku{Naar we hoorden, mag.}{hij zich in een volledig}{herstel verheugen}\\

\haiku{Tante Jet heeft om.}{gezondheidsredenen}{ontslag ingediend}\\

\haiku{Johan begrijpt er niets,.}{van maar daar wen je wel aan}{bij rijke menschen}\\

\haiku{\'e\'en naar de ruime,.}{toekomst waar ons hunkerend}{hart misschien rust vindt}\\

\haiku{Lou's hart klopt mild van,?}{ontroering maar wat moet hij}{nu zeggen of doen}\\

\haiku{Zij neemt ze van hem.}{over en laat ze keurend door}{de vingers glijden}\\

\haiku{Hij krijgt ze in geen,. -?}{geval meer terug als hij}{dat maar niet denkt Niet}\\

\haiku{- Laat me los, - zegt ze,,,.}{opeens bevelend en Lou}{ja hij doet het zoowaar}\\

\haiku{- Of had hij misschien?}{een manchester-pak aan}{en een slappen hoed}\\

\haiku{- Ach ja, zegt Lou met,.}{een droomenden glimlach z'n}{krullen is ie kwijt}\\

\haiku{- En dat is Gary,?}{Cooper zou je daar niet zoo}{verliefd op worden}\\

\haiku{De sterren aan  ,.}{den hemel loopen mee die}{hebben ook geen haast}\\

\haiku{Als het weer niet te,,.}{kwaad is doet hij wel een toer}{een doellooze zwerftocht}\\

\haiku{hij zit weer bij den,.}{Kluizenaar we zullen maar}{niet op hem wachten}\\

\haiku{*** ~ Als de slaap zich,.}{eindelijk over hem ontfermt}{krijgt hij nog geen rust}\\

\haiku{En pas op voor je,,.}{koningin jongetje hier}{kom ik op mijn paard}\\

\haiku{De ander laat zich,.}{niet bidden dat doen alleen}{kleine artisten}\\

\haiku{Hij blijft getroffen.}{staan en drukt Kloos wat vaster}{onder z'n arm}\\

\haiku{Als die af is, moet.}{Frans daar zoet blijven zitten}{tot de rest klaar komt}\\

\haiku{Niets dan een groote haard,.}{en een hangklok en zoo een}{gezellig tehuis}\\

\haiku{Maar toen hij wakker,.}{werd kwam z'n lot hem des te}{ondraaglijker voor}\\

\haiku{Maar wanneer hij haar,}{net genaderd was keerde}{zij hem den rug toe}\\

\haiku{Ik bid u. - Zeker,,;}{zal ik je helpen kind der}{menschen sprak de toovenaar}\\

\haiku{en toch is dit mijn,,.}{Land niet het Land van Geluk}{waar ik van droomde}\\

\haiku{Zij zag hem uit het.}{doksaaldeurtje komen en}{bleef verwonderd staan}\\

\haiku{Maar Arthur grijpt haar bij.}{de haren vast en trekt haar}{hoofd naar achteren}\\

\haiku{Die zit alleen, wordt:}{stilaan grijs en speelt nog wat}{herinneringen}\\

\haiku{Het is ontzettend,,.}{zoo iets praat er maar niet met}{de kinderen over}\\

\haiku{Arthur scheurt nu ook de.}{foto in vier stukken en}{werpt ze in het vuur}\\

\haiku{De nieuwe is maar,,.}{een bleekneus zoo recht van de}{boeken je weet wel}\\

\haiku{Maar we zullen goed,.}{opletten op wie het wicht}{van Leen gaat lijken}\\

\haiku{Vroeg of laat leer ik,.}{ze wel kennen ik woon hier}{pas drie of vier maand}\\

\haiku{{\textquoteright} de heele stemming.}{weg en ging met veel drukte}{en omhaal zitten}\\

\haiku{Achteraf heb ik,.}{me toch afgevraagd wat er}{wel mocht haperen}\\

\haiku{Misschien had ik haar,.}{eerder moeten opslaan dat}{kan het eenige zijn}\\

\haiku{Ik zal het op het,.}{einde van de week doen als}{blijde verrassing}\\

\haiku{Hij heeft zelfs \'e\'en van.}{z'n gevleugelde woorden}{aan me gespendeerd}\\

\haiku{En in haar blozend....}{gelaat stonden haar oogen zoo}{blauw als heerlijk blauw}\\

\haiku{Ging de bel daar niet? -,.}{Nee snakte ze overtuigd en}{huilde rustig door}\\

\haiku{ik woon hier nu haast,.}{vier of vijf maanden het wordt}{tijd dat ik ze ken}\\

\haiku{de detective,.}{en zeg tegen Emmy dat}{de prijs te laag is}\\

\haiku{Blijf dan wat, zei ik.}{haastig en wees uitnoodigend}{naar een clubzetel}\\

\haiku{Maar voor mij begint.}{nu de narigheid weer met}{een nieuw dienstmeisje}\\

\haiku{hij draagt denzelfden,.}{voornaam als Rousseau de}{would-be-wijsgeer}\\

\haiku{Zelfs loopt ze af en,.}{toe om uitleg bij me aan}{maar niet half genoeg}\\

\haiku{dat is maar goed ook,.}{want ik houd m'n hart niet eens}{onder controle}\\

\haiku{Laat het zinken in,,!}{den oceaan met een blok lood}{voer het naar de hel}\\

\haiku{Zij knikte ernstig,.}{er kwam een ongewone}{zachtheid in haar blik}\\

\haiku{Zij beloofde ten.}{slotte en scheen opgelucht}{toen ik haar uitliet}\\

\haiku{De dokters namen:}{mekaar eens goed op en men}{ging aldus te werk}\\

\haiku{Lou lacht, nerveus en,.}{superieur alsof hij}{dat wel verwachtte}\\

\haiku{Als de man gek is,;}{is het geen waanvoorstelling}{en is hij normaal}\\

\haiku{Je derde dochter.}{zou dat veel behoorlijker}{opgelost hebben}\\

\haiku{Maar de eerste de,.}{beste komt hier niet binnen}{probeer het liever}\\

\haiku{Misschien is de zon,;}{daarboven echt misschien de}{zon hierbeneden}\\

\haiku{Maar toen begon het.}{pas. De vijver kende geen}{rustigen dag meer}\\

\haiku{Intusschen steeg de.}{verwarring op den vijver}{tot haar hoogtepunt}\\

\haiku{Haar volg ik met open,;}{kelkje van Oost naar West om}{warmte en leven}\\

\haiku{Doorbuigen, knie\"en,......}{saamhouden vooroverleunen}{en het kwam nooit uit}\\

\haiku{Want het lichaam moet,,?}{flink gevoed worden niet waar}{dat is van belang}\\

\haiku{weelde van nog wat.}{late zon in een paleis}{van roode wolken}\\

\haiku{Toen ik nog kind was,.}{had ik dikwijls last met de}{lessen van godsdienst}\\

\haiku{Maar even later moet.}{Lou iets vragen en wenkt den}{professor vlakbij}\\

\haiku{*** ~ Een paar dagen.}{later is de professor}{al zeer tevreden}\\

\haiku{- Niet veel zaaks, moet de.}{professor zich hulpeloos}{verontschuldigen}\\

\haiku{Lou heeft nu minder,.}{last van z'n buren dan de}{concierge van hem}\\

\haiku{Lach me niet uit, Lou,...}{Anders het verstand hebben}{ze van hun vader}\\

\haiku{Wat een geluk voor!}{beiden dat hij niet aan haar}{blijven plakken is}\\

\haiku{- Hij weet wel dat ze,.}{nu Lena heet maar gebruikt}{dien naam met opzet}\\

\haiku{Daar een variant.}{op moet ik van avond zeker}{te pas brengen}\\

\haiku{Modern en classiek,.}{noemen ze mijn werk maar dat}{weet u natuurlijk}\\

\haiku{Lou is blij voor den,.}{kerel maar niet voldaan over}{z'n onderzoek}\\

\haiku{En ze voelt niets voor,,.}{hem geen sikkepit ze heeft}{het zelf gezegd}\\

\haiku{- Ze hebben je dus?}{al \'e\'en keer voor chantage}{de nor ingedraaid}\\

\haiku{- En heeft ze niet zoo'n,,......?}{knaap van veertien vijftien jaar}{wat heu zonderling}\\

\haiku{Hoe gemakkelijk?...}{nemen we ons persoonlijk}{geval tot maatstaf}\\

\haiku{Maar haar beschrijven,.}{kan hij niet het  is ook}{zoo lang geleden}\\

\haiku{Vroeger speelde hij,...}{wel eens aan het kerkmuurtje}{met Felicitas}\\

\haiku{Maar voor zichzelf maakt.}{hij uit dat het slot van het}{verhaal anders is}\\

\haiku{En hij nam haar tot.}{zich en ze leefden samen}{lang en gelukkig}\\

\haiku{Heeft Lou Anders een,?}{zoon in deze wereld een}{kleinen Aeneas}\\

\haiku{Dat schijnt hem  als '.}{een kwade microbe in}{t bloed te zitten}\\

\haiku{Lou kent heel weinig.}{menschen die daar een afdoend}{antwoord op weten}\\

\section{Clovis Baert}

\subsection{Uit: Het tweede leven van Wieske Veyt}

\haiku{De mane was een.}{reusachtige topaas met}{diep-gelen grond}\\

\haiku{Mijn leventje sliert.}{weg lijk een stroelke water}{in gloeiend-heet zand}\\

\haiku{Iederen keer als, '}{hij asem haalde wast hem}{lijk of hij sterken}\\

\haiku{bult en blies lijk een.}{verstopte blaasbalg in een}{half uitgegaan vuur}\\

\haiku{'t Was al. Van 't.}{leven in de wereld weet}{ik weinig of niets}\\

\haiku{het zijn liedeke!}{met de grootste overtuiging}{tot aan het einde}\\

\haiku{en hij loech ook en.}{liet zich nu stillekes op}{zijnen rug drijven}\\

\haiku{Een meiske  dat,!}{in den vijver zat dat in}{den vijver leefde}\\

\haiku{Wieske stond daar nog.}{altijd aan zijn kijkgat en}{verroerde geen spier}\\

\haiku{er is nooit een klank,.}{te weinig maar er is ook}{nooit een klank te veel}\\

\haiku{En ze waren nog ';}{maar even opt gras of ze}{lieten zich vallen}\\

\haiku{Ze gebaren, dat,.}{ze er niet van moeten van}{weten dacht Wieske}\\

\haiku{En onvermoeibaar.}{zat het bokske terbinst op}{de fluit te blazen}\\

\haiku{t was hier zake.}{van zich een beetje op den}{stok te verlaten}\\

\haiku{Zij verschoot danig,.}{want dat handeke was lijk}{een puur ijsklompke}\\

\haiku{zuiver goud viel op.}{Wieske en verguldde zijn}{wezen en zijn borst}\\

\haiku{of ze nepen eens.}{in de billen en dan ging}{het er wreed op los}\\

\haiku{'k geloof dat ik,.}{mijnen weg verloren heb}{antwoordde Wieske}\\

\haiku{Ze omhullen ons,.}{met een sluier van blanke}{zuivere zijde}\\

\haiku{De bliksemsnelle.}{vaart pakte zijnen asem en}{sneed hem bijkans af}\\

\haiku{Wieske struikelde.}{altemets en bezeerde}{zijn moede voeten}\\

\haiku{Hij ging straat-op en:}{straat-af en zag hier overal}{\'een en hetzelfde}\\

\haiku{Onder zijn weg zag.}{hij het beeld van de eene helft}{der menschenwereld}\\

\haiku{Hij hield zijn asem in}{van schrik en opeens keerde}{hij zich om en liep}\\

\haiku{zoo rap mogelijk,.}{een helling op tot op het}{hoogste van de stad}\\

\haiku{En ons Wieske werd.}{nu al ineens geplaagd door}{een priemende vraag}\\

\haiku{Zou Magdalena?}{hier nu nog rondwaren tot}{hij haar bereikt had}\\

\haiku{Was het nu zake?}{van hier te blijven of van}{weer te gaan zwerven}\\

\haiku{Het was lijk of de.}{brand voor Magdalena een}{weldaad geweest was}\\

\haiku{De stem was lijk de,;}{klank van een klok gegoten}{uit de zuiverste spijs}\\

\haiku{Het vuile, vieze.}{dier spon daar een dik net en}{zat dan te wachten}\\

\section{John Bake}

\subsection{Uit: Reisbrieven}

\haiku{Spoedig ging het weer,.}{regenen en alles was}{daar slik en plassen}\\

\haiku{Die tour is een der.}{merkwaardigste geweest die}{wij gedaan hebben}\\

\haiku{Eerst langs de rechter,,.}{Rijnoever tot Eglisau}{zeer berg op en neer}\\

\haiku{Wij vonden er ook.}{andere kennissen en}{de familie Ochsner}\\

\haiku{De aankomst is op,.}{het strand voor de herberg waar}{wij eerst dineerden}\\

\haiku{Voor de overtocht van ():}{het meer4 uur lang moesten wij}{twee schuiten hebben}\\

\haiku{Links valt een smalle ().}{stroomlintvormig van zeker}{drieduizend voet hoog}\\

\haiku{Maar het schijnt een hol,.}{te zijn hoewel het vrij groot}{en protestants is}\\

\haiku{Daarom was het mij.}{ook van belang dit nu zelf}{te onderzoeken}\\

\haiku{In 't midden is.}{een wacht te paard om voor de}{orde te zorgen}\\

\haiku{Reichmann beloofde *.}{ons te zullen waarschuwen}{als zij speelde}\\

\haiku{De komedie, in,!}{de parterre kostte ons}{8 stuiver per plaats}\\

\haiku{Veel hartelijke.}{complimenten aan allen}{die mij liefhebben}\\

\haiku{Voghera, waar wij,.}{te twee uur aankwamen is}{een slordige plaats}\\

\haiku{De bomen ieder.}{ruim zo groot als een fikse}{pereboom bij ons}\\

\haiku{Het was reeds donker,.}{toen wij terugkerende}{de stad naderden}\\

\haiku{De dag was reeds vroeg,.}{heet zelfs naar getuigenis}{van de mensen hier}\\

\haiku{Mijn laatste, lieve,,,.}{beste Anna zond ik u}{meen ik uit Turijn}\\

\haiku{s Avonds gingen wij.}{ijs gebruiken en toen met}{een kop thee naar bed}\\

\haiku{Zo veel reizen en.}{vliegen laat niets tot stille}{afzondering over}\\

\haiku{Geel en ik hebben.}{die tocht van 9 \`a 10 uur}{meest te voet gedaan}\\

\haiku{Het is bij half vijf.}{en dus bijna etenstijd aan}{de table d'h\^ote}\\

\haiku{Men is daar volmaakt,.}{goed zeer geschikt om een paar}{dagen te bijven}\\

\haiku{De Italiaanse.}{heldere lucht schijnt voor ons}{verloren te zijn}\\

\haiku{Ik schrijf u dit nog ',.}{te 9 uurs avonds zo van}{tafel komende}\\

\haiku{Hij kwam van Parijs.}{en ging nu de winter in}{Itali\"e doorbrengen}\\

\haiku{Ik neem de zin op).}{te Altdorf te gaan zien de}{finstere Aareschlucht}\\

\haiku{Het was alsof men *.}{de gehele~         aarde}{aan zijn voeten had}\\

\haiku{Te 5 uur aten wij,.}{met 4 Engelsen aan wie}{wij ons niet stoorden}\\

\haiku{De Aar vloeit erlangs.}{en zo langzaam dalende}{rijdt men de stad in}\\

\haiku{Zeker hebt gij mijn.}{complimenten aan Koos en}{Dientje overgebracht}\\

\haiku{Van Bern heb ik u,.}{mijn laatste brief gezonden}{lieve beste vrouw}\\

\haiku{Dit plaatsje ligt aan,.}{het meer van dezelfde naam}{westelijk van Bern}\\

\haiku{Deze lindeboom.}{heeft beneden een stam van}{32 voet diameter}\\

\haiku{In de zeer goede.}{en nette salon gingen}{wij ons neervlijen}\\

\haiku{En dan gaat het op,.}{Bazel waar ik morgenavond}{hoop aan te komen}\\

\haiku{{\textquoteright} In dezelfde geest:}{liet de Geelbiografe}{Hamaker zich uit}\\

\haiku{indien wij allen,?}{dat voorbeeld volgden wat wierd}{er dan van het k.b.}\\

\haiku{Van belang is, dat {\textquoteleft}{\textquoteright}.}{in Geels ogen het hier eenvrij}{verre reis betrof}\\

\haiku{De vloot stond onder (-).}{bevel van Victor-Guy}{Duperr\'e17751846}\\

\haiku{W.E. Mead, The Grand Tour,.}{in the Eighteenth Century}{Boston/New York 1914}\\

\section{Piet Bakker}

\subsection{Uit: Branding}

\haiku{Hij kefte zachtjes,.}{een paar keer de ogen brandend}{naar de zee gericht}\\

\haiku{Als er wat in  ,,.}{dreef kon je pas zien hoe hoog}{die krullers gingen}\\

\haiku{{\textquoteleft}Hoepla Pantertje,!}{dat kraigt die voile zee niet}{van den baas terug}\\

\haiku{Bloedrood stroomde de,.}{wijn in het zand toen Steven}{het vat neergooide}\\

\haiku{Nou ga je weg en....}{vanavond om acht uur kom je}{je antwoord brengen}\\

\haiku{Toen hij de duinpan,.}{afliep stond Steef al aan de}{deur en hij lachte}\\

\haiku{En hai je 'm toen....?!}{niet meteen een klap voor z'n}{harses gegeve}\\

\haiku{Ze moesten niet denken, '!}{dat zem op die manier}{weg konden pesten}\\

\haiku{Reiers hond blafte.}{er hol tegenin en kroop}{achter zijn meester}\\

\haiku{Roerloos, borst en hals,.}{\'e\'en rode vlek mouwen en}{broek donker besmeurd}\\

\haiku{Jai kan je grote,....?!}{bek tege me houwe hai}{je dat begrepe}\\

\haiku{De dorpsheid van de.}{Zandwijkse meisjes miste}{zij ten enenmale}\\

\haiku{Dat je vader een,.}{liefertje is zal je uit}{mijn mond niet horen}\\

\haiku{Eigenaardig, dat.}{ie niet met die wond naar u}{is toegelopen}\\

\haiku{Niemand zag reder,.}{Arends meer doch allen voelden}{zijn aanwezigheid}\\

\haiku{De {\textquotedblleft}IJsland{\textquotedblright} met drie....?}{duizend manden schelvis en}{twaalf honderd gemengd}\\

\haiku{Zijn ogen verrieden,.}{een boze vreugde toen hij}{zijn dochter aankeek}\\

\haiku{Zo'n bink as Steven.}{Paauwels kon je een roer in}{se pote geve}\\

\haiku{'k Zou me nog een}{beetje met de bezem op}{me flikker komme}\\

\haiku{We wazze gestrand.}{en niemand  had er wat}{van gemorreke}\\

\haiku{Iedereen heeft het,?}{recht om de smoor aan iemand}{te hebben niet waar}\\

\haiku{Alleen Ome Hannes.}{wisselde nu en dan een}{paar woorden met hem}\\

\haiku{Alle romantiek.}{van het zeedorp bewoog zich}{om de reddingboot}\\

\haiku{Als er een stranding,.}{was geweest dan preekte de}{domin\'e daarover}\\

\haiku{Dat ze hem in 't,.}{dorp voorbijliepen kon hij}{nu wel verdragen}\\

\haiku{Arends en de baron,.}{zaten niet stil daar kon je}{donder op zeggen}\\

\haiku{Maar zes walmende.}{flambouwen gingen voor de}{driftige stoet uit}\\

\haiku{{\textquoteright} Jongens werden ruw.}{bij de kraag genomen en}{in het zand gekwakt}\\

\haiku{Dan legden zij hun,.}{kleinheid af vergaten zij}{hun jaloezietjes}\\

\haiku{{\textquoteright} Ook nu liep weer het.}{vrouwvolk achter de boot en}{de bemanning aan}\\

\haiku{Daar hadden ze een,!}{degelijk mannetje aan}{gekregen da's vast}\\

\haiku{{\textquoteright} lachte Steef, nu de.}{inspanning van het roeien}{niet meer nodig was}\\

\haiku{Nog v\'o\'or Ome Hannes,.}{een bevel gaf had Steef naar}{zijn riem gegrepen}\\

\haiku{Die Raier was een.}{goeie roeier en dat is ie}{natuurlijk nog wel}\\

\haiku{Steef voelde niet veel,.}{voor zo'n uittocht maar wilde}{geen spelbreker zijn}\\

\haiku{Hij sloeg zijn glas om.}{en ging met de rug naar de}{bezoekers staan}\\

\haiku{Zonder verdriet dacht,.}{hij aan haar terug zelfs met}{een zeker verwijt}\\

\haiku{Wie hem nu wat in,.}{de weg zou leggen kon het}{goed bij hem hebben}\\

\haiku{De koddebeier.}{wilde zo opgewonden}{het dorp nog niet in}\\

\haiku{De schelle stem van.}{Klunder wekte hem uit zijn}{halve verdoving}\\

\haiku{{\textquoteleft}Hee daar, ik moet vier!}{man hebben om dit vrachie}{naar huis te kruien}\\

\haiku{{\textquoteright} Steef Pauwels keek hem,:}{met \'e\'en oog aan pakte een}{nieuwe strik en zei}\\

\haiku{Dat was al goud waard,.}{zoals die klungel daar op}{z'n platte kont zat}\\

\haiku{Daar zullen ze van,.}{afblijven al zal ik me}{d'r dood voor vechten}\\

\haiku{'k Mag blij zijn, als.}{ik die schuit in IJmuie zelf}{kan inspecteren}\\

\haiku{Over een maand of zo,.}{dan leit er weer een nieuwe}{troller in IJmuie}\\

\haiku{Was niet te vinden.}{voor een aardigheidje met}{de assurantie}\\

\haiku{Ze moeten alleen,.}{niet denken dat ik me als}{een kalf laat slachten}\\

\haiku{{\textquoteright} {\textquoteleft}Ze zal hier nog een,{\textquoteright},.}{trekpleister hebben denk ik}{grinnikte Crijnssen}\\

\haiku{Die waren van een,.}{stroper boven wien het duin}{ineengestort was}\\

\haiku{Toen hij de blote,,.}{gezwollen voet zag liet hij}{het geweer zakken}\\

\haiku{zou een van jullie.}{naer het dorp kunne fietse}{om hulp te haele}\\

\haiku{Want as ik 'm had, '!}{laete verzoipe zout wel}{moeilijk zain geweest}\\

\haiku{{\textquoteleft}Je had tegen de,?}{zolder moeten vliegen van}{de pijn weet je dat}\\

\haiku{Dat getuigen bij.}{het trouwen van Arends vond hii}{een beroerd karwei}\\

\haiku{De koddebeier.}{werd door elkaar gehusseld}{en neergesmeten}\\

\haiku{{\textquoteleft}As je \'e\'en poot over,!}{me drempel zet dan slae ik}{je de harses in}\\

\haiku{Zij brachten hier hun,.}{buit die Lauwers dan weer naar}{den grossier doorzond}\\

\haiku{Dat was toch om een.}{dodelijke ziekte op}{je lijf te krijgen}\\

\haiku{Dat moisie heb nog{\textquoteright},.}{een staertje voorspelde een}{ouwe visserman}\\

\haiku{Zijn moeder was goed,.}{voor hem geweest een vader}{had hij nooit gekend}\\

\haiku{Dokter Hagens gaf.}{Steef die dag verlof om het}{bed te verlaten}\\

\haiku{{\textquoteright} Fraukje glimlachte.}{hem vriendelijk toe en nicht}{vond het overdreven}\\

\haiku{{\textquoteright} Toen Steef die nacht op,.}{zijn harde brits lag kon hij}{de slaap niet vatte}\\

\haiku{Die man kon toch geen?}{tevreden ogenblik meer in}{z'n leven hebben}\\

\haiku{Jai mot dat zaekie.}{niet allenig opknappe}{voor ons allemael}\\

\haiku{waren zij midden.}{in een groot jachtgezelschap}{terecht gekomen}\\

\haiku{Thois was het armoe.}{en de knaine brachte een}{raiksdaelder op}\\

\haiku{De zes weken in.}{het gevang waren als een}{boze droom geweest}\\

\haiku{{\textquoteright} {\textquoteleft}Laete we er dan{\textquoteright},.}{bai gaen zitte zei Steef met}{een schamper lachje}\\

\haiku{een man tegenover,,.}{u die tot het ainde toe}{doorvecht as het mot}\\

\haiku{Toen draaide Steef zich.}{langzaam om en wilde het}{vertrek verlaten}\\

\haiku{Tegen den koster,.}{had hij gezegd dat er geen}{hulp voor nodig was}\\

\haiku{Dit was de tweede,,,.}{kist die hij laat in de avond}{in het duin begroef}\\

\haiku{Is dat verdomme?!}{nog toe een manier om me}{te late wachte}\\

\haiku{Daar had een meester,.}{op gezeten die hart voor}{zijn spulletjes had}\\

\haiku{Een tweede Grimsby,.}{een Cuxhafen zou hij van}{IJmuiden maken}\\

\haiku{Steef Paauwels schipper - '!}{op die nieuwe trawler dan}{zou jes wat zien}\\

\haiku{{\textquoteleft}As je een wijf was, ',,.}{zouk haast zeggen dat je}{er slecht uit ziet maatje}\\

\haiku{En laten ze me,!}{nou allemaal uitschelden}{dat het een aard had}\\

\haiku{Op den duur zou dit,.}{werk hem tegenstaan dat wist}{Steef met zekerheid}\\

\haiku{Heel zijn gezicht wees,.}{uit dat hij er nu wat voor}{begon te voelen}\\

\haiku{Daar zouden ze zich '!}{int dorp toch zeker een}{koliek om lachen}\\

\haiku{Wordt dat weer es een,....!}{ouwerwets daggie klaine}{doivel dat je bent}\\

\haiku{Het was waterkoud.}{en Steef rilde toen hij zijn}{trui over de kop trok}\\

\haiku{Als uitgelaten.}{jongens liepen zij naar de}{zeewering terug}\\

\haiku{In haastige drift.}{als wilde de een over de}{ander heenlopen}\\

\haiku{het schip de kop te.}{zien opheffen en in het}{golfdal dompelen}\\

\haiku{{\textquoteright} {\textquoteleft}Blaive jullie maar '{\textquoteright},.}{int zand wroete snoof}{Steef verachtelijk}\\

\haiku{{\textquoteright} Van Weerden liet de:}{bui rustig over zich heen gaan}{en antwoordde stuurs}\\

\haiku{{\textquoteright} {\textquoteleft}En ik kan maar eens{\textquoteright},.}{verzuipe zei Van Weerden}{onder het weggaan}\\

\haiku{Je moest zeker oud.}{zijn om met alles vrede}{te kunnen hebben}\\

\haiku{{\textquoteleft}Het volk, dat nu op,.}{de trawlers zit zal het wel}{minder mooi vinden}\\

\haiku{{\textquoteleft}Ik ben nu eens niet,!}{in een stemming om gekijf}{te verwekken vrind}\\

\haiku{De mannen van de -.}{vuurtoren daar boven de}{jutters beneden}\\

\haiku{Van z'n eerste reis!}{al achter de sleepboot naar}{huis in vliegend weer}\\

\haiku{De brekers spoelden.}{over het dek en namen mee}{wat niet muurvast stond}\\

\haiku{Als een witte muur.}{stond de zee secondenlang}{boven de sleepboot}\\

\haiku{Als een litteken.}{stond de vertrokken mond in}{het witte gezicht}\\

\haiku{Toen de boot omlaag.}{zakte kwam de verlamde}{schipbreukeling mee}\\

\haiku{De kiel bonkte al,.}{op het zand nog \'e\'en breker}{en ze waren er}\\

\haiku{Hij gelastte, dat.}{de boot weer op de wagen}{zou worden gebracht}\\

\haiku{bedaard zijn zwarte,.}{sigaar te roken toen de}{mannen zich meldden}\\

\haiku{Zonder aarzelen.}{liep zij op de baar toe en}{zij stak de hand uit}\\

\haiku{{\textquoteright} Vol bewondering.}{keek meneer Vermaas naar den}{ouden zeeman op}\\

\haiku{De vaste hand van.}{Hannes kon de boot haaks op}{de golven houden}\\

\haiku{De kostelijke.}{blinkende machine kon}{in zee verroesten}\\

\haiku{Het stuurhuis met de.}{kaartenkamer waren al}{lang kapot gebeukt}\\

\section{Jacobus Barnaart jr.}

\subsection{Uit: Dagverhaal van merkwaardige voorvallen}

\haiku{22.Ben ik op het,;}{Latijnsche school gekomen}{oud zijnde 11 Jaaren}\\

\haiku{Egberts Rector een.}{redenvoering gedaan over}{de waare geleerdheid}\\

\haiku{dit huis gepoogd heeft[] *.}{te bewaaren.6        17421742}{February 9}\\

\haiku{zo een helder ligt.}{van sig dat de oog en daar}{bijna van traanden}\\

\haiku{wierden terwijl zij}{besig waaren met wijn te}{sluiken61 in de kasjot62}\\

\haiku{Den 9e May zijn wij;}{met de armee93 gecampeert}{1,5 uur van Doornik}\\

\haiku{die arme zuster}{met de overige wijven}{gehoord hebbende}\\

\haiku{En Barnaart, om haar,;}{hart te streelen In schyn van}{Zephirus te speelen}\\

\haiku{37{\textquoteleft}Mijn ogen waren{\textquoteright}.}{even beschoten d.w.z. Ik was}{licht ingesluimerd}\\

\haiku{Waarschijnlijk in 1685.}{bij de herroeping van het}{Edict van Nantes}\\

\haiku{Op die plek nu nog.}{de straatnamen Prinsen-}{en Statenbolwerk}\\

\haiku{251Het onder water.}{zetten ten behoeve van}{de verdediging}\\

\section{Belcampo}

\subsection{Uit: De zwerftocht van Belcampo}

\haiku{Hij stortte al zijn.}{teleurstelling voor me uit}{en wat zeg je dan}\\

\haiku{We namen afscheid:}{en even later klonk het weer}{door de bossen van}\\

\haiku{Daarbij lieten zij.}{hem de onmogelijkste}{standen innemen}\\

\haiku{Het eten was heerlijk,.}{varkensvlees en rode kool}{met aardappelen}\\

\haiku{Maar als het enigszins,,.}{kan teken ik iedereen}{die mij eten geeft uit}\\

\haiku{Een oude boerin;}{stak het hoofd uit het raam en}{keek eerst erg lelijk}\\

\haiku{en wandelde ik.}{bij stralend weer het tweede}{vreemde land binnen}\\

\haiku{Het was daar warm en.}{ze waren juist bezig met}{oliebollenbakken}\\

\haiku{{\textquoteright} In Frankrijk beseft,;}{men pas dat van eten een kunst}{gemaakt kan worden}\\

\haiku{Haha, ik ben sterk,,.}{ik ben fors gebouwd ik kan}{wel vijf Duitsers aan}\\

\haiku{Hij was er dan ook.}{erg mee in zijn schik en gaf}{mij er vijf franc voor}\\

\haiku{Nergens praat men zo.}{gezellig als achter een}{gedekte tafel}\\

\haiku{Gelukkig ging het,.}{allemaal goed in Gap was}{de rit ten einde}\\

\haiku{Aan de weg was een,.}{punt waar ik wel een half uur}{heb v\`ergezien}\\

\haiku{De heuvels wierpen.}{naar mij toe een steeds dieper}{werdende schaduw}\\

\haiku{de koper had er}{nog nooit in gezeten en}{ik zat er al in}\\

\haiku{Zie je wel, zei de,.}{Fransman aan de Riviera is}{het altijd mooi weer}\\

\haiku{Nu was ik waar ik.}{wezen wilde en had dus}{verder geen haast meer}\\

\haiku{Het paviljoen zag;}{er uit als de werkplaats van}{een openluchtalchimist}\\

\haiku{Ik hield mijn broek aan.}{met de contanten er in}{uit vrees voor diefstal}\\

\haiku{de dames hadden.}{blijkbaar al de dood voor ogen}{en gaven het over}\\

\haiku{Ik was vol blijde,.}{verwachting van het avontuur}{dat in de lucht hing}\\

\haiku{Het beste, wat men,,.}{met schone vrouwen kan doen}{is naar ze te zien}\\

\haiku{Een er van keek mij:}{een ogenblikje aan en zei}{toen zo maar plompweg}\\

\haiku{Gelukkig werden.}{ze door mijn aanbod geboeid}{en geen een ontkwam}\\

\haiku{E\'en, die vond dat hij,.}{op minister Cavour leek}{wilde zelfs twee keer}\\

\haiku{Het hele huis stonk,,.}{ik weet niet waarnaar misschien}{wel naar oude lucht}\\

\haiku{In de spitsuren, van,,}{1 tot half 3 tekende}{ik in de caf\'es}\\

\haiku{wond zich daar erg over,,,.}{op ze willen wel maar ze}{durven niet zei hij}\\

\haiku{Vaak begonnen ze.}{dan te lachen en lieten}{zich toch tekenen}\\

\haiku{Gezicht op Napels;}{met de Vesuvius}{op de achtergrond}\\

\haiku{En die lag daar al,.}{in 1200 toen bij ons alles}{nog moest beginnen}\\

\haiku{Wie gaat er nu voor,?}{zijn plezier naar een eenzaam}{man die verdriet heeft}\\

\haiku{De graaf was verbaasd,.}{en ontdaan het was hem te}{plotseling blijkbaar}\\

\haiku{Die beeldhouwer had!!}{het lichaam van zijn vrouw uit}{het hoofd gekend}\\

\haiku{Ja, ja, ich sage,,...}{Ihnen mein lieber Herr die}{g\"ottlichen R\"omer}\\

\haiku{Bij Volterra was.}{zo een hele kerk in de}{afgrond gesukkeld}\\

\haiku{Ik heb gisteravond.}{nog lang nagedacht en weet}{nu wat je doen moet}\\

\haiku{het gesprek was niet.}{anders dan een herhaling}{van het voorgaande}\\

\haiku{gesprek hoort men steeds '.}{zeldzamer en opt laatst}{helemaal nooit meer}\\

\haiku{{\textquoteright} {\textquoteleft}Dan w\`ordt ze zeekr.}{allemoale duur dat}{gat h\`enemieterd}\\

\haiku{{\textquoteright} {\textquoteleft}St\`elt oe ees vuur, da'j}{doar zatn te kiekng en ie zang}{oe eeng detta23 k\`ats}\\

\haiku{de Wieringermeer,.}{toe't d\`en pas dreuge was emaakt}{was doar nog niks bie}\\

\haiku{Toen we bij de rand,.}{van de krater aankwamen}{was het al donker}\\

\haiku{in je huis een steeds.}{wisselende bevolking}{van wilde vissen}\\

\haiku{hij was Hongaar en.}{had zich al door de hele}{wereld geslikt}\\

\haiku{Een stationshond.}{ging vlak voor de deur staan en}{blafte mij in slaap}\\

\haiku{Toen ik wakker werd,.}{liepen de muizen om het}{hardst over de balken}\\

\haiku{Het ging gelukkig,.}{goed want de stenen gingen}{allemaal verkeerd}\\

\haiku{En diezelfde dag.}{kwam er een ontzettende}{regenbui over mij}\\

\haiku{Maar wat ik ook deed,.}{ik kon die namen er niet}{bij ze uit krijgen}\\

\haiku{Aan de wand hing een, {\textquoteleft}{\textquoteright},.}{parapluParaplu zei}{ik en wees daarnaar}\\

\haiku{Ze kwamen in een.}{kring om me heen staan en het}{regende woorden}\\

\haiku{En midden in de.}{nacht werd ik opeens wakker}{van een luid gedreun}\\

\haiku{ik had me bij het.}{laatste politiebureau}{moeten afmelden}\\

\haiku{Dit gevoelsleven,.}{wordt losgelaten het wordt}{individueel}\\

\haiku{Aan deze kant van;}{Osnabr\"uck maakten ze er}{pas een echt feest van}\\

\section{Colla Bemelmans}

\subsection{Uit: Platbook 3. Gebaorte en doe\"ed}

\haiku{Door gesprekke m\`et ' '}{dee zaogch in tot wat}{ch in m'nen doed}\\

\haiku{Waat zalle weej dich.}{noow toch misse weej meuste}{vuu\"els te flot oetein}\\

\haiku{mer doe woors d'r neet}{diene stool is laeg en oos}{bed is te groeat}\\

\haiku{Bin bliej, hae veult noow,.}{gein pien Maar hae had zoe gaer}{nog thoes wille zien}\\

\haiku{aoje boum haet now.}{ein wong wao ie\"erst dae tak}{met bloesem hong}\\

\haiku{Toch, de insigste}{vakantie die mien moder}{oe\"ets haet gehad}\\

\haiku{Ze zeen gebore, '.}{op 24 December 1928t}{zeen mien Korstkindjes}\\

\section{Hans Berghuis}

\subsection{Uit: Niet naar de maan gaan}

\haiku{Ugo Claudio was toch,.}{wel een Romein die er zijn}{mocht dacht ik woedend}\\

\haiku{Dan is hij groot en,.}{sterk dan lacht zijn mond en dan}{schitteren zijn ogen}\\

\haiku{Van daaruit kun je. '}{de slaapkamerramen van}{het  kasteel zien}\\

\haiku{s Avonds zitten de.}{sterke knapen van het dorp}{urenlang op de bank}\\

\haiku{Ik parkeerde het.}{dampende autootje aan de}{rand van het vliegveld}\\

\haiku{Hier is niets aan de.}{hand. Mark zit immers niet in}{de gevangenis}\\

\haiku{De zaken van God,...}{gaan v\'o\'or de affaires van}{de mensen ofschoon}\\

\haiku{Aarzelt zij even bij?}{exit no. 7 vanwaar je naar}{Athenai kunt vliegen}\\

\haiku{Neen, zij gaat rustig.}{verder naar de uitgang voor}{Palma-Portmany}\\

\haiku{Verleden zondag.}{zit mijn vader toevallig}{met mij in de kerk}\\

\haiku{{\textquoteright} Ik begrijp het wel,.}{maar dat kan ik tegen mijn}{vader niet zeggen}\\

\haiku{s Morgens ligt het.}{pak van mijn vader over een}{stoel in de keuken}\\

\haiku{Hij doet zijn kaken.}{evenmin van elkaar maar ik}{hoor zijn hart praten}\\

\haiku{{\textquoteleft}Maak het niet te laat,,{\textquoteright}.}{Mark zei Nora zacht toen zij}{naar haar kamer ging}\\

\haiku{Wij zijn geboren.}{tussen 1922 en 1925 en wij}{zijn de Veertigers}\\

\haiku{Ik heb zolang al.}{de wens gehad eenmaal met}{Nora te dansen}\\

\haiku{{\textquoteleft}Mevrouw, een dichter;}{is alleen als dichter in}{dit land niet veel waard}\\

\haiku{hen te verleiden.}{zich zelf volkomen aan mij}{uit te leveren}\\

\haiku{De Middellandse,.}{Zee is trouwens altijd de}{moeite waard altijd}\\

\haiku{{\textquoteleft}Materi\"ele,.}{geschiedenis heeft er niets}{mee te maken Mark}\\

\haiku{{\textquoteright}, dan zou Mark vrede,;}{met hem hebben gehad maar}{dat doet Johan G. niet}\\

\haiku{dat nog eens,{\textquoteright} zegt hij.}{alsof hij vreest dat hij mij}{verkeerd verstaan heeft}\\

\haiku{En wij weten dat,,.}{ook wel wij Westerlingen}{wij Nederlanders}\\

\haiku{Wij gaan een nieuwe,.}{tegemoet maar de nieuwe}{wereld is ook klein}\\

\haiku{maar hij gelooft in,.}{de genade als in een}{verlossing ik niet}\\

\haiku{Alleen Mon mag hem,.}{storen alleen van Mon kan}{hij alles hebben}\\

\haiku{Mijn God, s\'oms hapt hij:}{in het lokaas maar zelfs d\'at}{doet hij met opzet}\\

\haiku{vanmiddag moesten wij,.}{elkaar pijn doen maar het was}{eindelijk weer goed}\\

\haiku{Hij had mij en de.}{kinderen naar het vliegveld}{in Palma gebracht}\\

\haiku{Van Portmany kun.}{je in de winter niet per}{vliegtuig vertrekken}\\

\haiku{de rust van alle.}{mensen die het opgeven}{verder te leven}\\

\haiku{ik ben natuurlijk;}{niet zijn Beatrice maar}{zijn Assepoester}\\

\haiku{Luister eens, Ludwig,.}{ik veracht mannen die over}{hun vrouwen klagen}\\

\haiku{Ik weet het niet maar ', '.}{k voel het zok word aan}{een kruis geslagen}\\

\haiku{Je tastte niet eens.}{in het donker rond als een}{kind dat verdwaald is}\\

\haiku{Ro had vandaag een.}{paar vel schrijfpapier van zijn}{tafel genomen}\\

\haiku{Ik heb u nodig,,.}{schone baronesa ik}{heb u zeer nodig}\\

\haiku{Excellentie zal}{het mij niet kwalijk nemen}{wanneer ik opmerk}\\

\haiku{Mijn eigen zoon is.}{het slachtoffer van deze}{getuige geweest}\\

\haiku{Ter informatie.}{van Uwe Excellentie volgt}{hier de vertaling}\\

\haiku{Laat mij nu schrijven.}{over de onmogelijkheid}{van mijn terugkeer}\\

\haiku{Wellicht wenste zij,.}{het ook hoewel zij het nooit}{heeft uitgesproken}\\

\haiku{Welk een zelfbedrog,,.}{welk een menselijke waan}{mijn goede Ysbrand}\\

\haiku{Het spreekt vanzelf dat.}{jullie de oude God niet}{meer nodig hebben}\\

\haiku{Jullie baan leidt naar.}{de technische ruimte in}{de buurt van de maan}\\

\haiku{En heb geduld met,,.}{mij Pieter het is toch nog}{een moeilijk verhaal}\\

\haiku{De kleinste is een.}{lieve hangoor die wel eens}{tegen mij aankruipt}\\

\haiku{Ik moest je nog veel,.}{meer schrijven maar plotseling}{ontbreekt mij de moed}\\

\haiku{Toen viel hij uit zijn,.}{stoel hij lag gekromd op de}{vloer van de patio}\\

\haiku{Misschien kunt u mij?}{een glas whisky on the rocks}{laten serveren}\\

\haiku{Je weet wel dat ik.}{even blij ben over de reis die}{jij nu moet maken}\\

\haiku{Wat heb jij ermee?}{te maken als wij elkaar}{de nek omdraaien}\\

\haiku{Verbijsterd bleef zij.}{zitten totdat de ober haar}{de rekening bracht}\\

\haiku{Mark vond het nodig;}{om op dat uur zijn zonen}{naar bed te sturen}\\

\haiku{Mijn auto de fe,,.}{was een voorbeeld een beeldspraak}{een vergelijking}\\

\section{Anton Bergmann}

\subsection{Uit: Brigitta}

\haiku{Hij kende zijne.}{kwaal in den grond en kon er}{uren over vertellen}\\

\subsection{Uit: Twee Rijnlandsche novellen}

\haiku{Dan vereenigen.}{zich al de gasten in de}{algemeene zaal}\\

\haiku{- Op de oevers niet,.}{het kleinste huisje niet de}{nederigste hut}\\

\haiku{- 't Is voortaan geene,.}{vriendenstem meer die zijn hart}{nog treffen kan}\\

\haiku{'t Was aandoenlijk,}{en tevens belachelijk}{de wijfjes te zien}\\

\haiku{Geen woord van smaad, geene.}{schaduw van verwijt hoorde}{ik ooit uit zijn mond}\\

\haiku{Het zong altijd zoo,.}{vroolijk en lief als wij hier}{praatten en lachten}\\

\haiku{Of was het mijne,;}{hoedanigheid van vrijen}{Belg die hen aantrok}\\

\haiku{{\textquoteright} besloot Karel, met.}{zijne hand eene dreigende}{beweging makend}\\

\haiku{Hij kwam op ons toe,,.}{groette beleefd en nam den}{puntigen helm af}\\

\haiku{{\textquoteleft}Gij weet, dat ik een,.}{Frankforterin ben die geen}{Pruisen lijden kan}\\

\section{J.H. Bergmans-Beins}

\subsection{Uit: Het bloed kruipt waar het niet gaan kan. Een vertelling uit het Drentsche boerenleven}

\haiku{Een vertelling uit}{het Drentsche boerenleven}{Colofon}\\

\haiku{Wiecher Luten is {\textquoteleft}{\textquoteright} {\textquoteleft}{\textquoteright}.}{achter in denhof bezig}{dezwa te haren}\\

\haiku{In het midden is,.}{de groote dorschvloer die tot}{de groote deuren gaat}\\

\haiku{Deze {\textquoteleft}baander{\textquoteright} wordt.}{gebruikt voor het inrijden}{van hooi en koren}\\

\haiku{{\textquoteright} {\textquoteleft}Jao wal, heur,{\textquoteright} zegt het, {\textquoteleft},.}{meisjeik denk er wal um}{ik wol hen jow hoes}\\

\haiku{{\textquoteright} Bedaard stappen Harm ',.}{en Roelfien naart hooiland}{waar Wiecher hen wacht}\\

\haiku{En als Wiecher de,.}{laatste streken heeft gedaan}{komt hij hen helpen}\\

\haiku{Zij is er als een,.}{dochter in huis alleen ze}{is er niet altijd}\\

\haiku{Van dienstvolk houdt hij,,.}{niet wat met Roelfiens hulp niet}{kan wordt uitbesteed}\\

\haiku{{\textquoteright} Allen lachen om,.}{het idee dat Roelfien naar huis}{gebracht moet worden}\\

\haiku{{\textquoteright} {\textquoteleft}O, dat kan wal, wij '}{hebt de vroggen opt nij}{laand en dat laot}\\

\haiku{d'r is nog gien ien,',?}{weggaon die ik holl'n}{wilt hadd ij dan wal}\\

\haiku{Langs den wand glijdt haar.}{blik en ze voelt zich een met}{alles wat daar is}\\

\haiku{Als Wiecher terug, '.}{komt in de keuken gaat hij}{bijt vuur zitten}\\

\haiku{{\textquoteleft}As ij trouwen mient, ',.}{dan zekg ijt niet goed}{te laot mien ij}\\

\haiku{de messen afveegt.}{en in de tafella bergt}{en verder opruimt}\\

\haiku{'t Is een oud lied,,.}{dat moeder haar leerde maar}{Roelfien zingt het graag}\\

\haiku{want Roelfien was een.}{knap meisje en in haar tand}{niet onbemiddeld}\\

\haiku{{\textquoteright} {\textquoteleft}En daacht moeder dan, '?}{datt met de borrel weer}{op zien plaos kwaamp}\\

\haiku{{\textquoteright} {\textquoteleft}Now volk,{\textquoteright} zegt Heling, {\textquoteleft}, '.}{ik wil maor verloopen}{zegen mett zwien}\\

\haiku{Steeds meer komt het in,.}{haar gedachten maar ze k\`an}{het niet aannemen}\\

\haiku{Wat naar het midden,.}{zit een groep meisjes waarbij}{Roelfien en Sina}\\

\haiku{Als Wiecher naar huis,.}{gaat is Roelfien's hart lichter}{dan in langen tijd}\\

\haiku{Hij doet beslag in ',.}{t ijzer klapt het dicht en}{legt het op het vuur}\\

\haiku{Ja, h\`em hadden ze,!}{behouden maar wat hadden}{ze veel verloren}\\

\haiku{{\textquoteleft}Elk is niet eev'n '.}{kerks en elk kant ok niet}{eev'n goed waachten}\\

\haiku{Hoe was ze ineen,.}{gekrompen van angst toen ze}{haar hoorde hoesten}\\

\haiku{Elk meende, dat hij '.}{ert rechte over wist en}{elk meende verkeerd}\\

\haiku{Mij ducht, ik mus is '.}{eem hier hen en kieken hoe}{oft er heer giet}\\

\haiku{Ik miende, dat oes}{Wiecher wat bejegent was}{en dat zie dij hier}\\

\haiku{{\textquoteright} Vrouw Eling kijkt naar de,.}{leemen vloer die nog als van ouds}{in de keuken ligt}\\

\haiku{Och jao, 'n tweibak, ',.}{die magk wal kowie is}{aans ok zoo enkeld}\\

\haiku{Niet te veel, dan wordt {\textquoteleft}{\textquoteright}, '.}{hetjoegel maar een beetje}{kant wel lijden}\\

\haiku{O, o, wat zul wij,.}{nog beleev'n aal niks as}{muite en z\"orgen}\\

\haiku{Ik hadd' niks in hoes.}{doe Hillechien kwaamp en dat}{was mij min genogt}\\

\haiku{Ze heeft nog geen tijd.}{gehad om te bedenken}{wat ze zal zeggen}\\

\haiku{'t Is met 't mark.}{aal opgaon met aal die}{kovviedrinkers}\\

\haiku{Hie kun jao wal 'n,?}{wicht had hebb'n daor zie}{wat meer in zagen}\\

\haiku{{\textquoteright} {\textquoteleft}Dat kan nooit tot zien.}{geluk weez'n as hie met}{zien minderman trouwt}\\

\haiku{Hij gaat 's middags.}{naar den akker om Wiecher}{koffie te brengen}\\

\haiku{Nee, vaoder, now,.}{niet meer daorveur zin wij}{te wied hen praot}\\

\haiku{{\textquoteleft}dan huef ij ok '.}{niet te schromen urn d'rn}{enn an te maoken}\\

\haiku{Langzaam gaat hij weer ',.}{aant werk zonder naar zijn}{vader te kijken}\\

\haiku{Dat zijn vader niet,,.}{toegeven zal neen dat lijkt}{hem onmogelijk}\\

\haiku{Hij heeft Wiecher nooit, '.}{wat geweigerd al kostte}{t nog zooveel geld}\\

\haiku{En jong van darteg.}{jaor kun ij toch de wet}{niet meer veurschriev'n}\\

\haiku{Och, als Harm het goed,.}{vindt zij zal Roelfien graag als}{dochter ontvangen}\\

\haiku{Ze bukt zich en neemt.}{het koffiekannetje op}{om in te schenken}\\

\haiku{{\textquoteright} vraagt Roelfien, {\textquoteleft}ik daacht,.}{dat ij aaltied tevree en}{opgeruumd wassen}\\

\haiku{Dat kun ik niet best '.}{overgeev'n en ik kunt}{in ien dag niet doen}\\

\haiku{Maor vaoder '.}{is wat nustreg naot mark}{en dat verdr\"ot mij}\\

\haiku{Heb ij d'r met je, ' '?}{volk over praot datt met}{t haarfst wezen zul}\\

\haiku{{\textquoteright} {\textquoteleft}En denk ij, Wiecher, ', '?}{datt mienens is of zul}{t wal overbeeter'n}\\

\haiku{{\textquoteright} vraagt Roelfien, terwijl, {\textquoteleft}?}{ze haar moeder aanzietwat}{zeg moeder d'r van}\\

\haiku{{\textquoteright} {\textquoteleft}En ik,{\textquoteright} zegt Wiecher, {\textquoteleft}.}{zie niet van Roelfien of as}{zie mij hebben wil}\\

\haiku{{\textquoteright} {\textquoteleft}O, Wiecher, dat weet, '.}{ij wal daor zul ikt}{niet um overgeev'n}\\

\haiku{blieft staon, dan zoo, '}{gaauw meugliek tenminsten as}{ij hiern stee veur}\\

\haiku{Beiden bepalen, '.}{zich tot een zwijgen over wat}{hunt hoogste ligt}\\

\haiku{Schoon alles vrede,,.}{lijkt voelen alle drie dat}{er een oorlog dreigt}\\

\haiku{Hij gaat om het huis {\textquoteleft}{\textquoteright},.}{heen naar dekamer die een}{deur naar buiten heeft}\\

\haiku{{\textquoteleft}Harm hef 'n groot haart,{\textquoteright}.}{en nog stiever kop is het}{algemeen oordeel}\\

\haiku{Tegen Harm doet men,.}{dat zoo niet maar tegen haar}{zal men niet zwijgen}\\

\haiku{Hij voedt zijn toorn en,.}{praat zich steeds voor dat het recht}{aan zijn zijde is}\\

\haiku{Och,{\textquoteright} zegt Rieks, {\textquoteleft}as ij, ', '.}{dunkt datt wal kan dan mag}{ikt ok wal li\^en}\\

\haiku{{\textquoteright} {\textquoteleft}En mensk giet al zien, '.}{leven hen schoel as ijt}{maor weet'n wilt}\\

\haiku{Dan schuren ze langs.}{de palen en stooten hun}{neus in het voeder}\\

\haiku{Wiecher komt haar na,.}{met een tweede emmer die}{hij bij haar neer zet}\\

\haiku{Jammer, dat de naam.}{aan vader Rieks toekomt en}{het geen Harm kan zijn}\\

\haiku{Niet haasten, wachten.}{tot de tijd daar is en dan}{komt alles terecht}\\

\haiku{As 't ies mooi is, '}{en helder weer dan giet wal}{metn hakkenkruk}\\

\haiku{ik hum ofzee, met.}{zun gelegenheid kun hie}{wal ies besluten}\\

\haiku{Als Roelfien terug, '}{gaat komt Wiecher haar tegen}{en samen gaan ze}\\

\haiku{En nu zoo ineens,.}{vertelt Jenne haar dat er}{een kind wordt verwacht}\\

\haiku{Ze doet een kooltje.}{in de stoof en schenkt haar dan}{een kop koffie in}\\

\haiku{Verleden jaar was,.}{zijn moeder nog pas hersteld}{toen kwam er niet van}\\

\haiku{Ze houden zooveel.}{van Wiecher en gunnen het}{hem zoo van harte}\\

\haiku{as wij eerder teeg'n', '.}{d aol toorn anruepen}{dan kwamt ok weerum}\\

\haiku{Wat Wiecher verkeerd,.}{dee dat zal Haarm van Wiecher}{weer terecht brengen}\\

\haiku{En as Fennechien, '.}{komm'n wol dan wast ja}{zoo weer  terecht}\\

\haiku{Enkele dagen {\textquoteleft}{\textquoteright}.}{later wordt inRieksen hoes}{een zoon geboren}\\

\haiku{{\textquoteleft}Vaoder, moeder, ';}{wij hebtn jonge zeun en}{fiksche dikke jong}\\

\haiku{Als hij den hof is,.}{doorgegaan blijft hij even bij}{den ouden oven staan}\\

\haiku{Hij neemt zijn zakdoek.}{en veegt er hard mee over zijn}{oogen en zijn gezicht}\\

\haiku{Laot je kinner'.}{maor even zekgen wat}{dag of d anern kunt}\\

\haiku{Even wacht hij en als,,:}{hij hoort dat ze rustig ademt}{vraagt hij fluisterend}\\

\haiku{Nu neemt Annechien.}{Harm uit de wieg en legt hem}{op buurvrouw's schoot}\\

\haiku{wal wellent daor, '.}{men van zee dat zie vant}{heksenvolk wassen}\\

\haiku{De mannen zijn op.}{het land en Roelfien is in}{het tuintje bezig}\\

\haiku{{\textquoteleft}Klein Haarm{\textquoteright} met zooveel,!}{liefde verwacht met zooveel}{blijdschap ontvangen}\\

\haiku{Het verlangen om,.}{het kind van haar zoon te zien}{wordt haar te machtig}\\

\haiku{Harm is den heelen,.}{dag weg naar zijn zuster die}{wat ongesteld is}\\

\haiku{Ze doet het weer dicht '.}{zooals zet vond en haast zich}{dan naar huis terug}\\

\haiku{Mocht Fennechien nog,.}{eens komen dan is is het}{nog tijd genoeg}\\

\haiku{En as ik er dan,.}{klair met zin dan mag ien van}{je hum wal krieg'n}\\

\haiku{Wat boeten is, dat {\textquotedblleft}{\textquotedblright},{\textquoteright}.}{isspiensters gaodeng en}{dat wet elk zegt Rieks}\\

\haiku{Ik zin aal daag' nog','}{blied dat ik Wiecher hebb en}{as ik hum missen}\\

\haiku{{\textquoteleft}Het kan best, ik kan,.}{wal op Haarm passen daor}{is niks met te doen}\\

\haiku{Ze gaan samen den.}{Brink over en bij den zandweg}{gaat Sina terug}\\

\haiku{'t Lijkt Annechien,.}{wel goed toe maar de mannen}{moeten het weten}\\

\haiku{Op de {\textquoteleft}voorkist{\textquoteright} van.}{den bruidegomswagen zit}{de wasschupsneuger}\\

\haiku{Harm lacht en grijpt naar.}{moeder's oorijzer en naar}{de gouden doekspeld}\\

\haiku{{\textquoteright} zegt Roelfien, {\textquoteleft}wat kun,?}{ij daor vinnen dat van}{moeder zul weez'n}\\

\haiku{{\textquoteleft}Jao,{\textquoteright} meent Annechien, {\textquoteleft} '{\textquoteright}. '}{teeg'n twee'j maanlue kan ikt}{niet holl'm        XXXIV}\\

\haiku{{\textquoteright} vraagt Annechien, {\textquoteleft}'t,.}{leek niet zoo goed as vleden}{joar naar Wiecher zee}\\

\haiku{Och, wat zal hij het,.}{aardig vinden als zijn kind}{bij hem kan loopen}\\

\haiku{Soms bekruipt haar wel,.}{eens een gevoel van angst als}{ze Wiecher aanziet}\\

\haiku{Als het mislukt, grijpt:}{hij er met beide handen}{op en roept heel hard}\\

\haiku{Ankomm'n jaor,, '.}{as ij de boks ankriegt dan}{krieg ij okn pet}\\

\haiku{Dan verzet hij zich,.}{een beetje zoodat hij van den}{weg niet te zien is}\\

\haiku{Hoe kan iemand het, ';}{vuur nu uit laten gaan als}{t koffietijd is}\\

\haiku{Nu, zoo net voor het {\textquoteleft}{\textquoteright}.}{nieuweverbouw haalt men de}{stoet bij den bakker}\\

\haiku{hij is, dat Wiecher,.}{zou meenen dat hij daar met}{een bedoeling loopt}\\

\haiku{Langzaam valt een traan,.}{tusschen zijn vingers door maar}{hij bemerkt het niet}\\

\haiku{Daarom was het hem.}{een welkome aanleiding}{om thuis te blijven}\\

\haiku{Dat Wiecher met haar,,.}{alleen wil wezen is maar}{een praatje meent ze}\\

\haiku{Dat hij wel eens een,.}{enkele maal hoest dat heeft}{niets te beteekenen}\\

\haiku{{\textquoteleft}Nee,{\textquoteright} zegt ze, {\textquoteleft}gaot, '.}{maor is met dan za'kt}{je wal zekgen}\\

\haiku{Dan neemt Wiecher haar:}{hand tusschen zijn handen en}{zegt met heesche stem}\\

\haiku{'t Is een beetje;}{donker geworden en dit}{is Wiecher welkom}\\

\haiku{'t Is wal is 'n, '.}{dag of wat over maort}{komp ieder keer weerum}\\

\haiku{{\textquoteright} Wiecher glimlacht en.}{strijkt den kleinen jongen door}{het krullende haar}\\

\haiku{Als Roelfien na een,.}{poosje gaat kijken ligt hij}{rustig te slapen}\\

\haiku{De pogingen om,.}{Harm tot reden te brengen}{zijn te erg voor haar}\\

\haiku{{\textquoteright} Sina tracht haar de.}{duistere gedachten uit}{het hoofd te praten}\\

\haiku{Met een bezwaard hart,}{keert ze naar huis terug en}{neemt zich voor zoo gauw}\\

\haiku{Ze wil er ook niet,.}{meer over denken het maakt haar}{maar dubbel bedroefd}\\

\haiku{het zal winnen en.}{ze zonder bitterheid aan}{alles kan denken}\\

\haiku{Van de aardappels,,.}{die er op gegroeid zijn heeft}{hij niet meer geproefd}\\

\haiku{Waar ze gaat, vervolgt, '.}{het haar nooit kan zet meer}{van zich afzetten}\\

\haiku{Hij slaat zijn armpjes.}{om grootvader's hals en kust}{hem opgetogen}\\

\haiku{Een vertelling uit.}{het Drentsche boerenleven}{Noten 1slanker}\\

\section{Ger Bertholet}

\subsection{Uit: Sjweitberg (onder ps. G. Rapaille)}

\haiku{Volges Feel is dat,.}{ein volkscultureel versjeinsel}{zo\"e oud wie de welt}\\

\haiku{Ziene boum sjting '.}{in de dil ent waor}{sjtil in de dil}\\

\haiku{En gans klein sjting ':}{opt breefke in die flesj}{gesjri\"eve}\\

\haiku{Es de sjo\"el oet, '.}{is geitt natuurlik auch}{mit de lif nao heim}\\

\haiku{Ich bekiek mich 't,,}{fruit de greunte en mich geit}{durg d'r kop wae dat}\\

\haiku{Loup durg flab, de mos.}{nogal gek zi\"en om zo\"e}{deep nao te dinke}\\

\haiku{Zou me in Sjweitberg?}{zich ins good h\"obbe kinne}{laote gaon}\\

\haiku{De tant van de mam '.}{van d'r god van de luj van}{Sjweitberg ist meug}\\

\haiku{Zonger zeiver druegt.}{de aerd oet en lache}{kint nog ummer}\\

\haiku{En me moos enne.}{ki\"er dekker bukke want}{nog enne wage}\\

\haiku{Eederein dae kaom,.}{sjtumme kreeg e sjiek oranje}{vlegsjke maedje}\\

\haiku{Dae sjrie\"ewde,.}{in do\"edsno\"ed wie}{e mager verke}\\

\haiku{Mennige ki\"er.}{hat hae zich dao al d'r kop}{mit gesjto\"ete}\\

\haiku{En veer mer dinke '.}{dat dat nogt insigste}{is wat hiej nog wirkt}\\

\haiku{{\textquoteright} Zo\"e noe en dan l\"op,,.}{get nao boete m\`e de riej}{blief lank aeve lank}\\

\haiku{Dat is neet om te,,.}{lache geluif ich want de}{riej reageert neet}\\

\haiku{{\textquoteleft}Nae kink, sjtrreng,,.}{neet m\`e ze h\"obbe waal d'rr}{kop los  ocherrm}\\

\haiku{En daorom mot hae,.}{zich zo\"ev\"a\"ol meugelik}{aansjaffe vingt ze}\\

\haiku{aan 't gelle om!}{zich zo\"e good meugelik te}{kinne verkoupe}\\

\haiku{t Is ech get te.}{lank om d'r hiej lang uever}{te vert\`elle}\\

\haiku{'t Nul-nummer.}{van d'r Sji\"em van Murge}{zal gek zi\"en}\\

\haiku{{\textquoteright} {\textquoteleft}All\`e, m\`e wat dan,}{auch waor dat zal zich toch}{waal al lang gelag}\\

\haiku{Volgend jaor zal.}{ich waal huere wie dat}{aafgeloupe is}\\

\haiku{Natuurlik waere, '.}{de winkels aafgeloupe}{m\`et is angesj}\\

\haiku{Biej 't sjonste waer '.}{zitt ongel\"ok dus auch nog}{altied in de loch}\\

\haiku{Die ze opluchde.}{vuer vief dubbelzout of}{enne negerzoen}\\

\haiku{gekaoze om.}{doezende posdoeve oet}{te laote vleege}\\

\haiku{Doe mos 't i\"esj nog!}{mer ins zi\"en daste drin}{gekaoze wuers}\\

\haiku{{\textquoteright} {\textquoteleft}M\`e j\`owaal, die mam is '}{n zuster van mien mam en}{mien mam is getrouwd}\\

\haiku{Ich zi\"en die mer.}{e paar ki\"er per waek es}{ze mich get oethulp}\\

\haiku{Missjiens dat ich 't.}{auch ins mot gaon h\"obbe}{uever erotiek}\\

\haiku{{\textquoteright} Dat mot me zich, biej ',.}{n Sjweitbergse b\"oshalte dao}{laote duje}\\

\haiku{Dan begin ik toch '.}{te geluive dat geer aan}{t vlooke versjlaafd zeet}\\

\haiku{De sjnoet van de.}{Li\'es zit aeve versjtopt}{achter d'r pilaer}\\

\haiku{Bobo kriet geine.}{tied om sjeloes te waere want}{hae mot troef make}\\

\haiku{{\textquoteleft}Manou, dae moste...,,,!}{van achter pakke om dae}{witte band laok punt}\\

\haiku{Manou{\textquoteright}, vingt de Els {\textquoteleft}{\textquoteright}, {\textquoteleft}.}{opvolksgezondheidich mein}{dat die in Mestreech geit}\\

\haiku{Dae is nog te meug '.}{om te griene baove}{t unnesj\`elle}\\

\haiku{W\`etste, want mit.}{mien vrouw is gel\"okkig neet mien}{wermde gesjtorve}\\

\haiku{{\textquoteright} 't Maedje wat pis,:}{bie h\"a\"or kaom kroog gein kans om}{nog v\"a\"ol te zegke}\\

\haiku{Ze wol mer ein dink....}{en dat waor nuuj gode}{kweke ingele}\\

\haiku{M\`e hae vert\`elde.}{dat hae pas enne ouwe}{vrund mit begraave haw}\\

\haiku{{\textquoteleft}Ich weit 't.{\textquoteright} En hae:}{haolde e verfroemeld}{breefke oet zien t\`esj}\\

\haiku{Dao d'r vrund van is,?}{al twi\"e jaor allein wat}{meinste dat dat is}\\

\section{Joan Bertrand}

\subsection{Uit: Vrung va Oze Leve Hier}

\haiku{Dao green-er daeks.}{datter sjnakde uever}{de zung va vreuger}\\

\haiku{einne f\`eine,.}{wiensjmaak kietelt zieng tong verfrisjt}{zien verhiemelte}\\

\haiku{M\`e zieng weeg sjtong,:}{in et land va zon pa\^ume}{en appelziene}\\

\haiku{Dao wour aevels nieks.}{mie uever es zieng po\^usje}{zag der herbergeer}\\

\haiku{Wat hawwe ze.}{de mam al daek geplaogd}{um ei nuej breurke}\\

\haiku{Wie gaen hawwe.}{ze-em mitgenomme nao}{et noviciaat}\\

\haiku{- {\textquoteleft}Morloot hier pater{\textquoteright}, {\textquoteleft}!}{zag der vrachriejergier zeet}{mich einne sjo\`ene}\\

\haiku{{\textquoteright} Dan sjnapde hae.}{zich de kap va genne kop}{en reej op heim aa}\\

\haiku{Hae duipde-n-em.}{en wees-em eing plaatsj aan}{urges bie de Maas}\\

\haiku{Dat wour aevels neet,.}{gemekkelich al wour hae}{dan auch al zoe sjterk}\\

\haiku{Huej wour et Zondig.}{en e Goonsdig zou et Sint}{Caesilia zie}\\

\haiku{Et Pieterke wo\`ed.}{bang en leep watter laope}{kos de trappe aaf}\\

\haiku{Sjterve wour zo\`e erg,!}{neet en begrave waede}{wour auch zo\`e erg neet}\\

\haiku{Zi\`ene awwe,.}{zilvere baad jeugt taege}{et klei gezichske}\\

\haiku{onge gen erm en....}{dan leep-er dich durch gen weije}{wat der vaege kos}\\

\haiku{en in die sjtar - gier -,.}{geluift et jao neet ei klei}{wit kingerh\"andje}\\

\haiku{Dan zou der tieger.}{naeve et lemke komme}{lieke in gen weij}\\

\haiku{Eine boerejong.}{kaom aa-gelaope en}{heel de kameel vas}\\

\haiku{Der heilige Sint.}{Joezep wour jues mit ze nao}{gen durp gegange}\\

\haiku{Sint Joezep sjtong:}{Heur nog i\"esj hi\"el bezurgd}{der waeg te wieze}\\

\haiku{Gei vi\"ezelke.}{aan die kling dinger of et}{trilde van plezeer}\\

\haiku{Et Wimke wreef zich:}{van sjpas in gen heng en}{der melder fluidde}\\

\section{Anna Blaman}

\subsection{Uit: De arme student}

\haiku{Het is wel zeker}{dat de student daar nooit zou}{terechtgekomen}\\

\haiku{Armoede maakt je.}{eenzaam en hongerig in}{alle opzichten}\\

\haiku{En daarop schoof hij.}{vrijmoedig bij in hun kring}{en hief zijn pot bier}\\

\haiku{En daarna greep haar.}{hand feilloos een deurknop en}{deed ze een deur open}\\

\haiku{Luister, zei ze, en}{ze drukte me daarbij in}{de enorme stapel}\\

\subsection{Uit: Drie romans}

\haiku{ik had lang geen spijt,.}{van graad of titel die ik}{misgelopen was}\\

\haiku{Maar ik geloof dat,,.}{hij door dat te zeggen zijn}{jeugd vergeten wou}\\

\haiku{Ik kijk alleen maar,.}{naar je gezicht waarvan ik}{afscheid nemen moet}\\

\haiku{Want nauwelijks op.}{de terugtocht hield hij me}{stil en wou me kwijt}\\

\haiku{Een schemerkamer - -,.}{eerst koffiedrinken dan thee}{en sigaretten}\\

\haiku{Voor het eerst had ik,,.}{nu met een ander Jonas}{over haar gesproken}\\

\haiku{Als het te bar werd.}{laveerde ze de kant uit}{van het boertige}\\

\haiku{De morgen had ik -.}{zoek gebracht met Jonas en}{een krankzinnige}\\

\haiku{Ik wist het wel, zij,.}{had wel door dat het met Saar}{en mij niet deugde}\\

\haiku{Het deed me denken,...}{aan de dreinende ritmiek}{van regen regen}\\

\haiku{Hij had wat moeten,.}{zeggen wat anders moeten}{zeggen dan hij deed}\\

\haiku{Een prinselijke,.}{pauper een pauper met een}{prinsenziel was hij}\\

\haiku{Fortuin en liefde -.}{in uw leven zullen u}{de blonden geven}\\

\haiku{Ook Kareltje en.}{zij waren ten slotte aan}{de dijk gaan zitten}\\

\haiku{De lente bracht die;}{jongen met zijn vrouwelijk}{verholen liefde}\\

\haiku{Ze deed een vrouw in ',,.}{t bad die zich bevuild had}{een zachtzinnig mens}\\

\haiku{Maar daarna is ons.}{samenzijn ellendiger}{dan ooit tevoren}\\

\haiku{Ik hield krampachtig.}{de blijdschap in mijn labiel}{gemoed in evenwicht}\\

\haiku{{\textquoteright} {\textquoteleft}Nou, ik denk altijd.}{toch nog beter over een man}{dan over vrouwentuig}\\

\haiku{Zonder warmte in.}{zijn ogen boog hij zich naar haar}{toe en kuste haar}\\

\haiku{Hij keek haar vlug, en.}{zonder uitdrukking in zijn}{te lichte ogen aan}\\

\haiku{Jonas boog zich over.}{het portier en braakte de}{zwarte koffie uit}\\

\haiku{Het ging er nou maar,.}{om het vol te houden tot}{hij weer boven zat}\\

\haiku{Marie zat aan de,.}{grasglooiing waar ze beschut}{was tegen de wind}\\

\haiku{Ontevreden keek.}{ze naar de karreploeg die}{op het veld werkte}\\

\haiku{Hij zag hoe zij zich.}{rekte om die boven in}{de kast te zetten}\\

\haiku{{\textquoteleft}Ik had de indruk,{\textquoteright}, {\textquoteleft}.}{zei hij ingetogendat}{u iets hinderde}\\

\haiku{Misschien roken we,.}{samen een sigaret dat}{neemt de moeheid weg}\\

\haiku{Ik ging de kamer,,.}{in en streelde terwijl ik}{langs haar liep haar arm}\\

\haiku{Ik moest verdwijnen,.}{en wel onmiddellijk en}{zonder aarzeling}\\

\haiku{Ik moest het weten,,,.}{ik had haar gezien niet waar}{die zondagmorgen}\\

\haiku{Had ze zo'n honger,?}{of was haar maag nu nog niet}{helemaal op streek}\\

\haiku{Ze schonk zich nog een,.}{kopje thee in en dronk dat}{haastig dorstig leeg}\\

\haiku{Schamper haalde ze:}{de schouders op en boog zich}{naar de tafel toe}\\

\haiku{Op een gegeven.}{ogenblik trok ze de voeten}{van die stoel terug}\\

\haiku{een eenzaam man, stram,,,.}{arrogant met een lege}{hoffelijke grijns}\\

\haiku{Elke morgen nam:}{ze aan de piano haar}{oefeningen door}\\

\haiku{En bovendien, een,.}{kind het leven geven wat}{een verantwoording}\\

\haiku{Die maandag nog was,.}{hij naar een concert geweest}{een Bachrecital}\\

\haiku{My life since,.}{I loved you has been one}{prolonged agony}\\

\haiku{Op de tast greep ze.}{een nieuwe sigaret en}{zoog het vuur erin}\\

\haiku{Droomde ze wel ooit,?}{dat ze zich er daarna iets}{van herinnerde}\\

\haiku{{\textquoteright} {\textquoteleft}Ik heb,{\textquoteright} zei hij, {\textquoteleft}een,.}{grammofoon met mooie platen}{zoals Tannh\"auser}\\

\haiku{ze greep zijn hand en.}{zo bleven ze zitten toen}{het weer donker werd}\\

\haiku{De lucht was duister,.}{minder sterren waren er}{dan hij gedacht had}\\

\haiku{Maar ook aan morgen,.}{moest hij nu niet denken nu}{was hij met Marie}\\

\haiku{Heel zijn leven was.}{\'e\'en lange hunkerende}{wacht geweest op haar}\\

\haiku{Zou het ooit vriendschap?}{kunnen worden als hij hem}{nu al kwijtraakte}\\

\haiku{Tot morgen,{\textquoteright} zei hij -.}{nog en op het trambalkon}{keek hij nog even om}\\

\haiku{Ik was gekleed en.}{driftig wou ik nu op mijn}{beurt naar beneden}\\

\haiku{De zolder stond in.}{strakke binten over heel de}{diepte van het huis}\\

\haiku{Het meisje begon,.}{gejaagd te schreien daarom}{stuurden we haar weg}\\

\haiku{Zodra de bel ging,.}{haastte ik me naar de trap}{en trok de deur open}\\

\haiku{Zulk huilen maakte,.}{me machteloos ik stond daar}{maar en wist geen troost}\\

\haiku{{\textquoteright} Zij wiste met de,.}{vrije hand haar tranen weg er}{was zoveel te doen}\\

\haiku{Zachtjes trok ik de '.}{buitendeur int slot en}{draalde op de stoep}\\

\haiku{Voor koffietijd bracht.}{ik nog even mijn artikel}{aan de directeur}\\

\haiku{Hij was zuinig waar,.}{het op waarderen aankwam}{maar hij had gelijk}\\

\haiku{Mevrouw De Watter.}{keek mijn verbazing met een}{zachte glimlach aan}\\

\haiku{{\textquoteright} vroeg ze kinderlijk,.}{terwijl haarzelf de tranen}{in de ogen welden}\\

\haiku{De stappen hielden,.}{stil voor mijn deur er klopte}{iemand zachtjes aan}\\

\haiku{Op dat moment greep,}{ik hem bij de arm troonde}{hem mee en wachtte}\\

\haiku{Het regende niet,,}{meer de weg was modderig}{en onbegaanbaar}\\

\haiku{Ik wist nu ook, dat.}{hij ternauwernood nog een}{gesprek verwachtte}\\

\haiku{langs de oevers, die,.}{hun tocht vervolgden bleef het}{steekspel onbeslist}\\

\haiku{- Toos was de brede.}{straat teruggelopen en}{ze stak het plein over}\\

\haiku{Ze was nu aan het.}{aarden pad gekomen dat}{naar boven voerde}\\

\haiku{Bij jonge oogst kreeg,.}{ze een ijl angstgevoel dat}{niet onprettig was}\\

\haiku{Haar slordig haar viel,.}{voor haar ogen dreigend liep ze}{op de spiegel toe}\\

\haiku{{\textquoteleft}O neen, blijf nu toch,,{\textquoteright}.}{zitten Saartje zei mevrouw De}{Watter iets te luid}\\

\haiku{{\textquoteleft}Dat meisje,{\textquoteright} zei ik, {\textquoteleft},.}{stroefdat is de zuster van}{mijn vriend van Jonas}\\

\haiku{Haar ogen lagen nu,.}{vlak onder me ik zag de}{irissen verblauwen}\\

\haiku{Ze staarde leeg en,.}{treurig weg met vochtige}{verblauwde irissen}\\

\haiku{Ze keek haar aan en:}{ze ontmoette een paar ogen}{vol verweer en schrik}\\

\haiku{Want als dat zo niet,.}{was dan hadden we elkaar}{niet zo gevonden}\\

\haiku{Zou je lust hebben,,?}{vanavond ergens heen te gaan}{of morgen Sara}\\

\haiku{- Het ziekenhuis was.}{een door tuinen omgeven}{kloosterlijk gebouw}\\

\haiku{Op de zaal met aan,.}{weerskanten bedden zag ik}{Toos en toen pas hem}\\

\haiku{{\textquoteleft}Ik heb geen tijd, vind ',?}{jet vervelend dat ik}{je alleen laat gaan}\\

\haiku{Er daverde een,.}{trein uit het perron er kwam}{er weer een binnen}\\

\haiku{Ze trok zich los en,,:}{keek me koud afwijzend aan}{haar ogen waren grauw}\\

\haiku{In de bossen nam,.}{elk vrouwtjesdier de vlucht voor}{hem hij was geen dier}\\

\haiku{{\textquoteright} Met een ruk hield ze,,:}{toen stil ze keek me aan in}{koude haat ze zei}\\

\haiku{Ik sloeg de dekens.}{van me af en ging weer op}{m'n bedrand zitten}\\

\haiku{{\textquoteright} Maar ze kijkt me aan.}{of ze me niet herkent en}{zegt natuurlijk neen}\\

\haiku{Wat er gebeurd was,.}{was te teer om aangeroerd}{te mogen worden}\\

\haiku{{\textquoteright} Als de tram komt steekt,.}{ze waarschuwend de hand op}{wat niet nodig is}\\

\haiku{Ze zei me dat ze.}{nooit geloofd zou hebben dat}{te durven zeggen}\\

\haiku{Haar ogen waren grijs,.}{haar mond was rijp en toch weer}{zacht als van een kind}\\

\haiku{{\textquoteright} - Ze lachte even in,.}{m'n ogen maar haar hand in de}{mijne bleef passief}\\

\haiku{{\textquoteleft}Het blauwe paleis,.}{heeft me zo triest gemaakt ik}{weet niets vrolijks meer}\\

\haiku{Mijn dieptelagen;}{zijn heus van dezelfde stijl}{als mijn fa\c{c}ade}\\

\haiku{Het was de eerste.}{keer dat zij hem toestond bij}{haar aan te komen}\\

\haiku{Gisteravond, in die,.}{dancing meende hij succes}{geboekt te hebben}\\

\haiku{Stel nu dat die echo...}{eens verschald zou zijn en ik}{haar niet zou weerzien}\\

\haiku{En deze middag,,.}{fantaseerde ik wist King}{dat ze niet thuis was}\\

\haiku{En hij wist zelfs waar.}{ze heen was en hoe laat ze}{zou terugkomen}\\

\haiku{In de hand hield hij.}{de nagemaakte sleutels}{van de deur gereed}\\

\haiku{{\textquoteright} - Hij keek me lang en:}{stil aan en antwoordde toen}{bijna fluisterend}\\

\haiku{Geen wonder dat hij,.}{bang werd ik zag er uit als}{een krankzinnige}\\

\haiku{Het meningloze,,;}{meisje Annie keek even op}{en glimlachte flauw}\\

\haiku{daarom... ik zou het,.}{goed maken met hem hij zou}{h\'a\'ar nooit meer moeten}\\

\haiku{Neen, Berthe was, waar,.}{het de erotiek betrof niet}{erg prinsesselijk}\\

\haiku{Maar hadden ze ook...}{ooit gedacht dat de vorstin}{zelf met een lakei}\\

\haiku{Yolande keerde.}{zich van de etalage af}{en keek Alide aan}\\

\haiku{De palmen waren,.}{breed de vingers kort en sterk}{met sterke nagels}\\

\haiku{Dat was alsof je.}{op een ziel speelde als op}{een edel instrument}\\

\haiku{Yolande stootte,.}{uitdagend Berthe uit wat}{haar voldoening gaf}\\

\haiku{ze hebben een ziel,,?}{dat is wel zeker maar wie}{begrijpt dat overdag}\\

\haiku{Annies hand was kil.}{en Berthes hand was reddend}{in een vaste greep}\\

\haiku{{\textquoteleft}God ja,{\textquoteright} zei ze, {\textquoteleft}die,;}{heb ik nog en daarom ging}{ik maar eens lopen}\\

\haiku{Daardoor, Alide, zat,.}{ik aan die slootkant daardoor}{wist ik het niet meer}\\

\haiku{Was het een boze,,,?}{droom die Peps toe zeg dat het}{een boze droom was}\\

\haiku{Je handen hebben,.}{me gestreeld en behoed je}{schoot is kuis en koel}\\

\haiku{En dan ga ik dat,,,.}{kleuren cynisch gewaagd in}{schreeuwende verven}\\

\haiku{Toen ik eindelijk,.}{voor Mon Repos stond wist ik}{niets en dacht ik niets}\\

\haiku{Van bij het venster,.}{klonk het snuiven van huilen}{geluidloos huilen}\\

\haiku{Hij zweette als een.}{otter en hij rolde als}{een vod de trap af}\\

\haiku{Hij hield zich  groot,:}{en daarom riposteerde}{ze genadeloos}\\

\haiku{Maar Kosta bleef het druk.}{hebben met roken en keek}{haar vooral niet aan}\\

\haiku{Maar ik ben jong en.}{heb een lichaam dat zelfs nooit}{nog aan de beurt kwam}\\

\haiku{En gaf ze daarvoor,,?}{ook om dat te kunnen doen}{iets van zichzelf prijs}\\

\haiku{Daar zette hij een,.}{vast half uurtje voor als hij}{zich stond te scheren}\\

\haiku{Als 't nodig was.}{deed hij zo braaf en dweepziek}{als een heilsoldaat}\\

\haiku{Er overkomt je bij.}{haar in een korte spanne}{tijds een massa goeds}\\

\haiku{King zou toch kunnen,,.}{wachten ondanks dat briefje}{of terugkomen}\\

\haiku{Hij wou die vervloekt,.}{begerenswaardige jas}{hebben en voorgoed}\\

\haiku{Hij dacht dat hij zijn,.}{drift beheerste maar zijn stem was}{fluisterend en heet}\\

\haiku{En dat gebeurde.}{toen we de koffers pakten}{om te vertrekken}\\

\haiku{Ze deed het net zo.}{zorgzaam en zo liefdevol}{als ooit tevoren}\\

\haiku{Maar ik kon deze,.}{taak niet van haar overnemen}{dat was al te hard}\\

\haiku{Ik zou misschien zo'n,,.}{zelfde aandacht winnen dacht}{ik al pratende}\\

\haiku{{\textquoteright} zouden de tranen.}{niet te stelpen tussen mijn}{vingers door vloeien}\\

\haiku{Je schrok, zette de.}{borden uit je handen en}{brak in snikken uit}\\

\haiku{Maar waar was dat, ja,.}{in Mon Repos had ik haar al}{een keer geslagen}\\

\haiku{Alide kwijnde weg,.}{in West-Europa nu}{ik er niet meer was}\\

\haiku{Haar wang lag aan haar '.}{hand en zo keek ze de kant}{vant venster uit}\\

\haiku{Zijn hart bonsde zo.}{zwaar dat hij verwonderd was}{het niet te horen}\\

\haiku{{\textquoteright} - En ze spreidde als.}{een deken de kamerjas}{over hen beiden uit}\\

\haiku{Ze was een tors van.}{edel marmer en een duister}{bloedwarm vrouwenhoofd}\\

\haiku{Zijn spel groeit hem hier ',.}{bovent hoofd hij is zijn}{eigen spelbreker}\\

\haiku{De angst dat hij die,.}{droom niet houden kan zit hem}{al op de hielen}\\

\haiku{Juliette behoeft.}{dan ook vanzelf niet meer in}{de gevangenis}\\

\haiku{Toch, juist die avond van.}{King's avontuur was er diep in}{haar weer iets gaande}\\

\haiku{{\textquoteright} Maar daarop strekte,,.}{de hospita bezwerend}{sussend een hand uit}\\

\haiku{Maar toch herhaalde,:}{ze omdat het King maar was}{die haar gekrenkt had}\\

\haiku{Ze drukte haar borst.}{tegen me aan en hield het}{hoofd wat achterover}\\

\haiku{{\textquoteleft}Een vrouw kan kopen.}{voor zichzelf alsof ze haar}{eigen minnaar is}\\

\haiku{Bijna herfst was het,,,,.}{een vroege herfstmiddag wijd}{zonnig zorgeloos}\\

\haiku{Ik zag de schutting.}{terug van de tuin waarin}{ik als kind speelde}\\

\haiku{Terwijl het toch om,.}{hem begonnen was scheen ze}{hem glad vergeten}\\

\haiku{Ze glimlachte, ze.}{sloot de ogen en bracht zo haar}{mond op zijn gezicht}\\

\haiku{En elke keer dat.}{zij haar tanden poetste moest}{ze daaraan denken}\\

\haiku{{\textquoteleft}Maar, jij, heb jij haar,?}{nooit gemist al liet je dat}{aan mij niet merken}\\

\haiku{Ze streelde met haar,.}{sterke vingers wonderlijk}{zacht zijn wang zijn haar}\\

\haiku{- Alide nestelde.}{zich op de divan en ging}{liggen nadenken}\\

\haiku{Ze schikte een paar '.}{kussens ondert hoofd en}{staarde voor zich uit}\\

\haiku{{\textquoteright} - Ze wist het, niets kon.}{zijn visie op haar storen}{of beschadigen}\\

\haiku{En wat was er van?}{hem geworden sinds ze hem}{niet meer beschermde}\\

\haiku{Het was een rijke,.}{blik die van het lokkende}{speelzieke wijfje}\\

\haiku{Ze zagen haar het,.}{postkantoor verlaten en}{zij liep spitsroeden}\\

\haiku{Niettemin schreed ze.}{rustig en soepel voort naast}{de gebrilde Peps}\\

\haiku{Maar het klonk in haar,.}{op als hol geluid een stem}{van gene zijde}\\

\haiku{Ze bleef zichzelf, ze.}{raakte niet vereenzelvigd}{met de natuur}\\

\haiku{Hoe na{\"\i}ef om zo.}{hartstochtelijk te hechten}{aan haar instemming}\\

\haiku{Ze keek nog steeds heel.}{ernstig en voelde een soort}{innerlijke pijn}\\

\haiku{Peps kwam die kamer.}{binnen en ontdekte zijn}{Alide op dat bed}\\

\haiku{Haar minnaar vocht een,;}{gevecht tegen twee tranen}{maar in elk oog \'e\'en}\\

\haiku{Ze voelde hoe zijn.}{hand die op de hare lag}{begon te zweten}\\

\haiku{{\textquoteright} Die lag soms uren op.}{de divan met een paar ogen}{waar je bang van werd}\\

\haiku{Ik vind hem trouwens,?}{ook veranderd de laatste}{tijd merk jij dat niet}\\

\haiku{{\textquoteleft}Nou, die meiden daar,,.}{die schudden je wel uit in}{alle opzichten}\\

\haiku{Die kennen kunsten,,.}{nou daar trekken ze het merg}{mee uit je botten}\\

\haiku{{\textquoteright} - Maar toch was er nog.}{iets anders wat haar dwars zat}{en wat ze niet zei}\\

\haiku{Niet dat haar dat zo,.}{hinderde want eigenlijk}{gaf ze er niets om}\\

\haiku{Kijk, dat begreep ze,?}{van zichzelf niet was ze dus}{zo onredelijk}\\

\haiku{Die had de dag weer,,;}{in die winkel achter die}{kassa doorgebracht}\\

\haiku{Maar toen vluchtte ze,,,.}{de kamer uit de trap op}{naar haar kamertje}\\

\haiku{Ze ging naar binnen,.}{sloeg de deur achter zich dicht}{en liep naar boven}\\

\haiku{{\textquoteleft}Natuurlijk weet ik,...{\textquoteright}.}{het maar En een weigering}{in ogen en gebaar}\\

\haiku{Anne zelf, die was.}{daarbij ternauwernood van}{werkelijk belang}\\

\haiku{Zo sterk had hij zich.}{dus ge{\"\i}dentificeerd}{met zijn eenzaamheid}\\

\haiku{Hij hief alleen de.}{hand op en wreef langzaam over}{zijn vermoeide ogen}\\

\haiku{{\textquoteright} - En daarop keerde '.}{hij zich van me af en ging}{voort venster staan}\\

\haiku{Toen ik de kamer.}{uit liep hield hij het hoofd weer}{van me afgekeerd}\\

\haiku{Toen kwam de blik tot.}{hem via de brilleglazen}{en hij werd gezien}\\

\haiku{{\textquoteleft}Als ze verliest, dan...{\textquoteright}}{zal dat toch pas zijn nadat}{ze heeft gewonnen}\\

\haiku{Ze zocht daar in en.}{op dat ogenblik stond ze met}{een gekromde rug}\\

\haiku{Ze nam een flesje}{uit haar tas en daarna greep}{ze een glas water}\\

\haiku{{\textquoteleft}Wees maar niet bang dat.}{ik iets minder vriendelijks}{over haar zeggen zal}\\

\haiku{Hij is lichtzinnig,.}{en brutaal en nu is hij}{ook nog gaan drinken}\\

\haiku{{\textquoteright} - De kastelein laat,.}{zich niet van de wijs brengen}{hij kent zijn mensen}\\

\haiku{Terwijl hij daar op.}{straat loopt denkt hij nog even door}{over de kastelein}\\

\haiku{Daar zat het hijgend,,,.}{bijna berstend weer op zijn}{plaats aan de aorta}\\

\haiku{Dat praten met haar,.}{kon ook straks misschien was dat}{niet eens meer nodig}\\

\haiku{Haar ogen knipperden,.}{alsof ze bang was maar ze}{bleef roerloos liggen}\\

\haiku{Ik tastte in het,.}{donker langs de muur geen deur}{was er te vinden}\\

\haiku{Ze zaten beiden,,,.}{roerloos zij Alide dwars op}{de schoot van Berthe}\\

\haiku{Haar huid was blank, haar.}{ogen grijs en troebel en haar}{mond was breed gewelfd}\\

\haiku{Ik keek weer naar haar,.}{voorhoofd een schild des hemels}{of een schild des doods}\\

\haiku{{\textquoteleft}Ik zou iets willen,{\textquoteright}, {\textquoteleft}.}{hebben dat je gestolen}{had zei ikvan Peps}\\

\haiku{{\textquoteright} - King knikte nog een.}{keer en hield daarop het hoofd}{rouwend gebogen}\\

\haiku{De kleren geurden.}{kamferachtig en ook een}{beetje naar tabak}\\

\haiku{{\textquoteleft}Schenk jij maar koffie,,,.}{in met een likeurtje kijk}{alles staat hier klaar}\\

\haiku{Boven die ogen stond.}{de rimpel waarin King een}{teken had gezien}\\

\haiku{Ze moest natuurlijk.}{blijven bij de idioot en}{mij met rust laten}\\

\haiku{Ik zag iets aan haar,,.}{maar ik wist niet wat er was}{iets aan haar gezicht}\\

\haiku{nog \'e\'en zo'n sc\`ene,,.}{en waarom dan ook en het}{is afgelopen}\\

\haiku{{\textquoteright} - Ik leunde met mijn.}{armen op de tafel en}{boog me naar haar toe}\\

\haiku{Ja, handen spelen.}{soms een gekke rol in die}{verdomde erotiek}\\

\haiku{Dat ik dat niet deed,.}{lag aan zijn gedrag en aan}{mijn medelijden}\\

\haiku{King  smeet wat geld.}{neer voor dat ene glas en liep}{het Parthenon uit}\\

\haiku{{\textquoteleft}Of houd je niet van,,.}{me zeg het maar  eerlijk}{want dan ga ik weg}\\

\haiku{Ik lach me toch de,.}{tranen in de ogen twee of}{drie dagen daarna}\\

\haiku{Een denkbeeld had zich.}{in me vastgezet waar ik}{niet meer van loskwam}\\

\haiku{{\textquoteright} - Ik ging naar binnen.}{en sloeg onbehouwen de}{deur achter medicht}\\

\haiku{{\textquoteright} - Hij hoorde aan mijn:}{toon dat ik de spot met hem}{dreef en antwoordde}\\

\haiku{Toen zag ik dat hij.}{haar wou optillen en naar}{het hakblok brengen}\\

\haiku{En overal, door heel,,,.}{het huis knipte ze het licht}{aan overal overal}\\

\haiku{Bij God, een mantel,!}{de mantel van Juliette}{heeft ze niet gezien}\\

\haiku{Dat was geen King, en.}{dat was evenmin een Kosta die}{de moeite waard was}\\

\haiku{Die lag nog steeds te.}{sterven aan de liefde die}{ongeneeslijk wondt}\\

\haiku{{\textquoteright} - {\textquoteleft}En waarom ook weer,?}{omdat je bang was dat ik}{je niet hebben wou}\\

\haiku{dat was, als je het,,.}{nauw neemt nog w\'el zo gemeen}{jij was niet dronken}\\

\haiku{En dat gebaar was,.}{bijna al te grollig na}{al wat gebeurd was}\\

\haiku{Goddank, dacht ze, dat,...}{ik zo oud ben want hij is}{betoverend}\\

\haiku{Wat je ook doet, je.}{haalt die twee niet uit elkaar}{en jullie ook niet}\\

\haiku{Heel mijn leven van.}{de laatste maanden had tot}{dit besluit geleid}\\

\haiku{Het was alsof ze,.}{me dood had gewaand  maar}{nu hervonden had}\\

\haiku{{\textquoteright} zei ze toen we daar, {\textquoteleft};}{op die kamer kwamenals}{een zieke zwerver}\\

\haiku{Als dat dan moet, dan,.}{maar de ergste dan moet ik}{jou verloochenen}\\

\haiku{Zo kwam het dat zijn,.}{arm begon te trillen van}{ingehouden lust}\\

\haiku{{\textquoteright} zei hij met een diep, {\textquoteleft}?}{teder geluiddat ik dus}{jou verraden had}\\

\haiku{Toen wierp ze nog een.}{paar blokken op het vuur en}{liet hen weer alleen}\\

\haiku{Hoe maak ik dan nog?}{van Uit het leven van een}{speurder een roman}\\

\haiku{Ze boog het hoofd en.}{liep schuw naar de trap die naar}{de hutten voerde}\\

\haiku{Maar onmiddellijk.}{daarop liep ze resoluut}{op Virginie toe}\\

\haiku{Maar ondertussen.}{bette ze toch de ogen en}{kamde ze haar haar}\\

\haiku{maar ga ik, dan valt,.}{er misschien iets te winnen}{terug te winnen}\\

\haiku{Aller ogen keken.}{snel van Virginie weg en}{er viel een stilte}\\

\haiku{{\textquoteright} - Ze keek terzijde.}{en zag het mooie gezicht van}{Louise Riffeford}\\

\haiku{Ze botsten, elk in,.}{eigen gedachten verstrikt}{tegen elkaar op}\\

\haiku{Ze reageerde.}{niet en wachtte af wat hij}{verder zeggen zou}\\

\haiku{Niemand wist dat ze.}{daarbinnen was en geen stem}{die haar terugriep}\\

\haiku{Zo fantaseerde,.}{Louise Riffeford ver op}{de feiten vooruit}\\

\haiku{Beiden keken ze,.}{naar Louise Riffeford die}{op de drempel stond}\\

\haiku{Een kort ogenblik bleef.}{Louise Riffeford nu nog}{op de drempel staan}\\

\haiku{Ze danste met die:}{razend knappe officier}{Sterreveld en zei}\\

\haiku{Toen schreef ik alles:}{wat hij tegen me gezegd}{had op en ik dacht}\\

\haiku{Een enkele blik.}{op haar was voldoende om}{dat vast te stellen}\\

\haiku{Nu neemt zo'n jongen,.}{een meisje en neem hem dat}{dan maar eens kwalijk}\\

\haiku{Met heimelijke.}{afkeuring keek hij naar de}{paren die swingden}\\

\haiku{Eerst voelde hij de.}{neiging om op te staan en}{naar hem toe te gaan}\\

\haiku{Zijn enige succes.}{was dat de buitenwereld}{hem dat niet aanzag}\\

\haiku{{\textquoteright} - {\textquoteleft}Kijk, daar is Louise,.}{Riffeford en alsof het}{haar eerste bal is}\\

\haiku{{\textquoteleft}Als dat niet langer.}{duurt dan een gesprek heb ik}{daar niets op tegen}\\

\haiku{U moest begrijpen.}{dat ze te stom waren om}{u G.G. te maken}\\

\haiku{Overigens, het moest,.}{een visioen zijn geweest}{het kon niet anders}\\

\haiku{{\textquoteright} - Hij hield op en keek.}{weer naar het trotse gezicht}{in de portretlijst}\\

\haiku{{\textquoteleft}Het lijkt me dat ik,.}{de moed daartoe moet hebben}{wat daar ook van komt}\\

\haiku{Stormachtig was het,.}{maar van een storm kon je toch}{nog lang niet spreken}\\

\haiku{En ik heb zo goed,,.}{als nooit slaap of beter ik}{slaap zo goed als nooit}\\

\haiku{Daarop scheurde hij:}{de bladzijden zorgvuldig}{in snippers en zei}\\

\haiku{Weten is iets wat,.}{je alleen doet en dat is}{ook maar het beste}\\

\haiku{Maar later denk je}{dat er in elke muurkast}{een kerel zit die}\\

\haiku{{\textquoteright} - Ze probeerde ook,.}{nog het elektrische licht maar}{dat deed het niet meer}\\

\haiku{Hij, de Alziende,.}{keek neer op alles wat op}{aarde geschiedde}\\

\haiku{Het bewoog om haar.}{heen als een  werveling}{van zwarte sluiers}\\

\haiku{Maar dat viel niemand,.}{op God niet en de speelse}{waterdieren niet}\\

\haiku{Louise Riffeford.}{had allang begrepen dat}{ze verloren was}\\

\haiku{Hij keek neer op de.}{plaats waar de Kruisvaarder ten}{onder was gegaan}\\

\haiku{Haar moeder liet een.}{verdrietig afkeurende}{blik op haar rusten}\\

\haiku{{\textquoteleft}Egbert is hier voor,,.}{je geweest je bent net te}{laat hij is net weg}\\

\haiku{Een golf van schaamte,.}{steeg in haar op schaamte en}{verontwaardiging}\\

\haiku{De kapitein van.}{het schip was opgestaan en}{kwam haar tegemoet}\\

\subsection{Uit: Droom in oorlogstijd}

\haiku{Het is helemaal.}{niet goed als Hansie daarmee}{vriendinnetje wordt}\\

\haiku{Maar ik, ik moet dan:}{nog geloven in mezelf}{en in mijn toekomst}\\

\haiku{dat leven van mij.}{is een tantalusbeker}{en ik stik van dorst}\\

\haiku{Straks, thuis, dacht Dolf, hoor,,.}{ik dat nog en morgen nog}{en overmorgen nog}\\

\haiku{{\textquoteright} De blik onder zijn,.}{ogen werd merkwaardig donker}{of die huilen ging}\\

\haiku{En je kan je niet.}{voorstellen hoe prachtig en}{lief ik die bok vond}\\

\haiku{Toen gebeurde er.}{plotseling zoiets als een}{verlossend wonder}\\

\haiku{{\textquoteright} - Toen greep een van de.}{mannen me beet en zette}{me op zijn schouder}\\

\haiku{Ze luisterde, ze,,.}{zocht en kon het niet verstaan}{niet achterhalen}\\

\haiku{Van vijf zes kanten.}{waren de helhonden op}{haar losgebarsten}\\

\haiku{{\textquoteleft}Moeder, ik geloof.}{dat Koba daar komt met haar}{vader en moeder}\\

\haiku{Ze draagt het mee in.}{haar hart terwijl ze kachels}{doet en gangen schrobt}\\

\haiku{En er zou nog meer.}{gezegd moeten worden in}{verband met die bloem}\\

\haiku{Nu stond ze daar nog.}{steeds te lachen voor die drie}{geliefde mensen}\\

\haiku{De zin des levens, -:}{dooreengelopen Mijn hart}{wijd-open Ze zegt}\\

\haiku{Het was een keurig.}{mantelpak dat ze al voor}{het derde jaar droeg}\\

\haiku{{\textquoteright} - De vrouw bleef wachten.}{totdat een stem haar verzocht}{boven te komen}\\

\haiku{De moeder trok met:}{een gebaar van onmacht de}{schouders op en zei}\\

\haiku{Een mens wordt ook zo.}{gemakkelijk zielig en}{beklagenswaardig}\\

\haiku{En daarop knikte...}{de vrouw en gleed haar blik weer}{schichtig langs haar heen}\\

\haiku{Zijzelf bijvoorbeeld.}{had vroeger  gedacht dat}{ze studeren zou}\\

\haiku{{\textquoteright} - De moeder en haar.}{dochter stonden voor het raam}{en keken haar na}\\

\haiku{{\textquoteleft}Nu sluit ik hierbij.}{in twee novellen die ik}{allang had liggen}\\

\haiku{Ze lijken eerder.}{in opdracht van dag- of}{weekblad geschreven}\\

\subsection{Uit: Eenzaam avontuur}

\haiku{Ze zei me dat ze.}{nooit geloofd zou hebben dat}{te durven zeggen}\\

\haiku{Haar ogen waren grijs,.}{haar mond was rijp en toch weer}{zacht als van een kind}\\

\haiku{{\textquoteright} - Ze lachte even in,.}{mijn ogen maar haar hand in de}{mijne bleef passief}\\

\haiku{{\textquoteleft}Het blauwe paleis,.}{heeft me zo triest gemaakt ik}{weet niets vrolijks meer}\\

\haiku{Toen het zomer werd.}{gingen we een paar maanden}{in een bos wonen}\\

\haiku{Mijn dieptelagen;}{zijn keus van dezelfde stijl}{als mijn fa\c{c}ade}\\

\haiku{Het was de eerste.}{keer dat zij hem toestond bij}{haar aan te komen}\\

\haiku{Gisteravond, in die,.}{dancing meende hij succes}{geboekt te hebben}\\

\haiku{Stel nu dat die echo...}{eens verschald zou zijn en ik}{haar niet zou weerzien}\\

\haiku{En deze middag,,.}{fantaseerde ik wist King}{dat ze niet thuis was}\\

\haiku{En hij wist zelfs waar.}{ze heen was en hoe laat ze}{zou terugkomen}\\

\haiku{In de hand hield hij.}{de nagemaakte sleutels}{van de deur gereed}\\

\haiku{{\textquoteright} - Hij keek me lang en:}{stil aan en antwoordde toen}{bijna fluisterend}\\

\haiku{Geen wonder dat hij,.}{bang werd ik zag er uit als}{een krankzinnige}\\

\haiku{Het meningloze,,;}{meisje Annie keek even op}{en glimlachte flauw}\\

\haiku{daarom... ik zou het,.}{goed maken met hem hij zou}{h\'a\'ar nooit meer moeten}\\

\haiku{Neen, Berthe was, waar,.}{het de erotiek betrof niet}{erg prinsesselijk}\\

\haiku{Maar hadden ze ook...}{ooit gedacht dat de vorstin}{zelf met een lakei}\\

\haiku{Yolande keerde.}{zich van de \'etalage af}{en keek Alide aan}\\

\haiku{blank haar gezicht en,.}{ver haar ogen alleen die}{glimlach was dichtbij}\\

\haiku{De palmen waren,.}{breed de vingers kort en sterk}{met sterke nagels}\\

\haiku{Dat was alsof je.}{op een ziel speelde als op}{een edel instrument}\\

\haiku{En Juliette stond.}{voor de spiegel en tooide}{zich met sieraden}\\

\haiku{Yolande stootte,.}{uitdagend Berthe uit wat}{haar voldoening gaf}\\

\haiku{ze hebben een ziel,,?}{dat is wel zeker maar wie}{begrijpt dat overdag}\\

\haiku{Annie's hand was kil.}{en Berthes hand was reddend}{in een vaste greep}\\

\haiku{{\textquoteleft}God ja,{\textquoteright} zei ze, {\textquoteleft}die,;}{heb ik nog en daarom ging}{ik maar eens lopen}\\

\haiku{Daardoor, Alide, zat,.}{ik aan die slootkant daardoor}{wist ik het niet meer}\\

\haiku{Was het een boze,,,?}{droom die Peps toe zeg dat het}{een boze droom was}\\

\haiku{in 't gras alsof,.}{ik zo zou opspringen de}{fiets grijpen en gaan}\\

\haiku{Je handen hebben,.}{me gestreeld en behoed je}{schoot is kuis en koel}\\

\haiku{En dan ga ik dat,,,.}{kleuren cynisch gewaagd in}{schreeuwende verven}\\

\haiku{Van bij het venster,.}{klonk het snuiven van huilen}{geluidloos huilen}\\

\haiku{Hij zweette als een.}{otter en hij rolde als}{een vod de trap af}\\

\haiku{Maar Kosta bleef het druk.}{hebben met roken en keek}{haar vooral niet aan}\\

\haiku{Maar ik ben jong en.}{heb een lichaam dat zelfs nooit}{nog aan de beurt kwam}\\

\haiku{En gaf ze daarvoor,,?}{ook om dat te kunnen doen}{iets van zichzelf prijs}\\

\haiku{Daar zette hij een,.}{vast halfuurtje voor als hij}{zich stond te scheren}\\

\haiku{Als 't nodig was.}{deed hij zo braaf en dweepziek}{als een heilsoldaat}\\

\haiku{Er overkomt je bij.}{haar in een korte spanne}{tijds een massa goeds}\\

\haiku{King zou toch kunnen,,.}{wachten ondanks dat briefje}{of terugkomen}\\

\haiku{Hij wou die vervloekt,.}{begerenswaardige jas}{hebben en voorgoed}\\

\haiku{Hij dacht dat hij zijn,.}{drift beheerste maar zijn stem was}{fluisterend en heet}\\

\haiku{En dat gebeurde.}{toen we de koffers pakten}{om te vertrekken}\\

\haiku{Ze deed het net zo.}{zorgzaam en zo liefdevol}{als ooit tevoren}\\

\haiku{Maar ik kon deze,.}{taak niet van haar overnemen}{dat was al te hard}\\

\haiku{Ik zou misschien zo'n,,.}{zelfde aandacht winnen dacht}{ik al pratende}\\

\haiku{{\textquoteright} zouden de tranen.}{niet te stelpen tussen mijn}{vingers door vloeien}\\

\haiku{Je schrok, zette de.}{borden uit je handen en}{brak in snikken uit}\\

\haiku{Maar waar was dat, ja,.}{in Mon Repos had ik haar al}{een keer geslagen}\\

\haiku{Alide kwijnde weg,.}{in West-Europa nu}{ik er niet meer was}\\

\haiku{Haar wang lag aan haar '.}{hand en zo keek ze de kant}{vant venster uit}\\

\haiku{Zijn hart bonsde zo.}{zwaar dat hij verwonderd was}{het niet te horen}\\

\haiku{{\textquoteright} - En ze spreidde als.}{een deken de kamerjas}{over hen beiden uit}\\

\haiku{Ze was een tors van.}{edel marmer en een duister}{bloedwarm vrouwehoofd}\\

\haiku{Zijn spel groeit hem hier ',.}{bovent hoofd hij is zijn}{eigen spelbreker}\\

\haiku{De angst dat hij die,.}{droom niet houden kan zit hem}{al op de hielen}\\

\haiku{Juliette behoeft.}{dan ook vanzelf niet meer in}{de gevangenis}\\

\haiku{Toch, juist die avond van.}{King's avontuur was er diep in}{haar weer iets gaande}\\

\haiku{{\textquoteright} Maar daarop strekte,,.}{de hospita bezwerend}{sussend een hand uit}\\

\haiku{Maar toch herhaalde,;}{ze omdat het King maar was}{die haar gekrenkt had}\\

\haiku{Ze drukte haar borst.}{tegen me aan en hield het}{hoofd wat achterover}\\

\haiku{{\textquoteleft}Een vrouw kan kopen.}{voor zichzelf alsof ze haar}{eigen minnaar is}\\

\haiku{Bijna herfst was het,,,,.}{een vroege herfstmiddag wijd}{zonnig zorgeloos}\\

\haiku{Ik zag de schutting.}{terug van de tuin waarin}{ik als kind speelde}\\

\haiku{Terwijl het toch om,.}{hem begonnen was scheen ze}{hem glad vergeten}\\

\haiku{Ze glimlachte, ze.}{sloot de ogen en bracht zo haar}{mond op zijn gezicht}\\

\haiku{En elke keer dat.}{zij haar tanden poetste moest}{ze daaraan denken}\\

\haiku{{\textquoteleft}Maar jij, heb jij haar,?}{nooit gemist al liet je dat}{aan mij niet merken}\\

\haiku{Ze streelde met haar,.}{sterke vingers wonderlijk}{zacht zijn wang zijn haar}\\

\haiku{- Alide nestelde.}{zich op de divan en ging}{liggen nadenken}\\

\haiku{Ze schikte een paar '.}{kussens ondert hoofd en}{staarde voor zich uit}\\

\haiku{{\textquoteright} - Ze wist het, niets kon.}{zijn visie op haar storen}{of beschadigen}\\

\haiku{En wat was er van?}{hem geworden sinds ze hem}{niet meer beschermde}\\

\haiku{Was deze vrouw  ?}{dan meer dan wat ze in dit}{genre had ontmoet}\\

\haiku{Het was een rijke,.}{blik die van het lokkende}{speelzieke wijfje}\\

\haiku{Ze zagen haar het,.}{postkantoor verlaten en}{zij liep spitsroeden}\\

\haiku{Niettemin schreed ze.}{rustig en soepel voort naast}{de gebrilde Peps}\\

\haiku{Maar het klonk in haar,.}{op als hol geluid een stem}{van gene zijde}\\

\haiku{Hoe na{\"\i}ef om zo.}{hartstochtelijk te hechten}{aan haar instemming}\\

\haiku{Ze keek nog steeds heel.}{ernstig en voelde een soort}{innerlijke pijn}\\

\haiku{Peps kwam die kamer.}{binnen en ontdekte zijn}{Alide op dat bed}\\

\haiku{Haar minnaar vocht een,;}{gevecht tegen twee tranen}{maar in elk oog \'e\'en}\\

\haiku{Ze voelde hoe zijn.}{hand die op de hare lag}{begon te zweten}\\

\haiku{{\textquoteright} Die lag soms uren op.}{de divan met een paar ogen}{waar je bang van werd}\\

\haiku{Ik vind hem trouwens,?}{ook veranderd de laatste}{tijd merk jij dat niet}\\

\haiku{{\textquoteleft}Nou, die meiden daar,,.}{die schudden je wel uit in}{alle opzichten}\\

\haiku{Die kennen kunsten,,.}{nou daar trekken ze het merg}{mee uit je botten}\\

\haiku{{\textquoteright} - Maar toch was er nog.}{iets anders wat haar dwars zat}{en wat ze niet zei}\\

\haiku{Niet dat haar dat zo,.}{hinderde want eigenlijk}{gaf ze er niets om}\\

\haiku{Kijk, dat begreep ze,?}{van zichzelf niet was ze dus}{zo onredelijk}\\

\haiku{Die had de dag weer,,;}{in die winkel achter die}{kassa doorgebracht}\\

\haiku{Maar toen vluchtte ze,,,.}{de kamer uit de trap op}{naar haar kamertje}\\

\haiku{Ze ging naar binnen,.}{sloeg de deur achter zich dicht}{en liep naar boven}\\

\haiku{{\textquoteleft}Natuurlijk weet ik,...{\textquoteright}.}{het maar En een weigering}{in ogen en gebaar}\\

\haiku{Anne zelf, die was.}{daarbij ternauwernood van}{werkelijk belang}\\

\haiku{Zo sterk had hij zich.}{dus ge{\"\i}dentificeerd}{met zijn eenzaamheid}\\

\haiku{Hij hief alleen de.}{hand op en wreef langzaam over}{zijn vermoeide ogen}\\

\haiku{Daarna greep hij naar.}{een pakje sigaretten}{en nam daar een uit}\\

\haiku{, maar het lijken me.}{begrijpelijke feiten}{die je daar vertelt}\\

\haiku{{\textquotedblright} - En daarop keerde '.}{hij zich van me af en ging}{voort venster staan}\\

\haiku{Toen ik de kamer.}{uit liep hield hij het hoofd weer}{van me afgekeerd}\\

\haiku{Toen kwam de blik tot.}{hem via de brilleglazen}{en hij werd gezien}\\

\haiku{{\textquoteleft}Als ze verliest, dan...{\textquoteright}}{zal dat toch pas zijn nadat}{ze heeft gewonnen}\\

\haiku{ik moet de stumper.}{die hier op de divan ligt}{zien te vergeten}\\

\haiku{Ze zocht daar in en.}{op dat ogenblik stond ze met}{een gekromde rug}\\

\haiku{Ze nam een flesje}{uit haar tas en daarna greep}{ze een glas water}\\

\haiku{{\textquoteleft}Wees maar niet bang dat.}{ik iets minder vriendelijks}{over haar zeggen zal}\\

\haiku{Hij is lichtzinnig,.}{en brutaal en nu is hij}{ook nog gaan drinken}\\

\haiku{{\textquoteright} - De kastelein laat,.}{zich niet van de wijs brengen}{hij kent zijn mensen}\\

\haiku{Terwijl hij daar op.}{straat loopt denkt hij nog even door}{over de kastelein}\\

\haiku{Daar zat het hijgend,,,.}{bijna berstend weer op zijn}{plaats aan de aorta}\\

\haiku{Dat praten met haar,.}{kon ook straks misschien was dat}{niet eens meer nodig}\\

\haiku{Haar ogen knipperden,.}{alsof ze bang was maar ze}{bleef roerloos liggen}\\

\haiku{Ik tastte in het,.}{donker langs de muur geen deur}{was er te vinden}\\

\haiku{Ze zaten beiden,,,.}{roerloos zij Alide dwars op}{de schoot van Berthe}\\

\haiku{Ik keek weer naar haar,.}{voorhoofd een schild des hemels}{of een schild des doods}\\

\haiku{{\textquoteleft}Ik zou iets willen,{\textquoteright}, {\textquoteleft}.}{hebben dat je gestolen}{had zei ikvan Peps}\\

\haiku{{\textquoteright} - King knikte nog een.}{keer en hield daarop het hoofd}{rouwend gebogen}\\

\haiku{De kleren geurden.}{kamferachtig en ook een}{beetje naar tabak}\\

\haiku{{\textquoteleft}Schenk jij maar koffie,,,.}{in met een likeurtje kijk}{alles staat hier klaar}\\

\haiku{Boven die ogen stond.}{de rimpel waarin King een}{teken had gezien}\\

\haiku{Ze moest natuurlijk.}{blijven bij de idioot en}{mij met rust laten}\\

\haiku{Ik zag iets aan haar,,.}{maar ik wist niet wat er was}{iets aan haar gezicht}\\

\haiku{nog \'e\'en zo'n sc\`ene,,.}{en waarom dan ook en het}{is afgelopen}\\

\haiku{{\textquoteright} - Ik leunde met mijn.}{armen op de tafel en}{boog me naar haar toe}\\

\haiku{Ja, handen spelen.}{soms een gekke rol in die}{verdomde erotiek}\\

\haiku{Dat ik dat niet deed,.}{lag aan zijn gedrag en aan}{mijn medelijden}\\

\haiku{{\textquoteleft}Ik  kan willen,.}{wat ik ook maar wil zolang}{ik maar alleen ben}\\

\haiku{Ik lach me toch de,.}{tranen in de ogen twee of}{drie dagen daarna}\\

\haiku{Een denkbeeld had zich.}{in me vastgezet waar ik}{niet meer van los kwam}\\

\haiku{{\textquoteright} - Ik ging naar binnen.}{en sloeg onbehouwen de}{deur achter me dicht}\\

\haiku{{\textquoteright} - Hij hoorde aan mijn:}{toon dat ik de spot met hem}{dreef en antwoordde}\\

\haiku{Toen zag ik dat hij.}{haar wou optillen en naar}{het hakblok brengen}\\

\haiku{En overal, door heel,,,.}{het huis knipte ze het licht}{aan overal overal}\\

\haiku{Bij God, een mantel,!}{de mantel van Juliette}{heeft ze niet gezien}\\

\haiku{Dat was geen King, en.}{dat was evenmin een Kosta die}{de moeite waard was}\\

\haiku{En Juliette, thuis,.}{gekomen had natuurlijk}{\'o\'ok haar bed gezocht}\\

\haiku{Die lag nog steeds te.}{sterven aan de liefde die}{ongeneeslijk wondt}\\

\haiku{{\textquoteright} - {\textquoteleft}En waarom ook weer,?}{omdat je bang was dat ik}{je niet hebben wou}\\

\haiku{dat was, als je het,,.}{nauw neemt nog w\'el zo gemeen}{j{\'\i}j was niet dronken}\\

\haiku{En dat gebaar was,.}{bijna al te grollig na}{al wat gebeurd was}\\

\haiku{Goddank, dacht ze, dat,...}{ik zo oud ben want hij is}{betoverend}\\

\haiku{Wat je ook doet, je.}{haalt die twee niet uit elkaar}{en jullie ook niet}\\

\haiku{En na zes borrels.}{stond ik op en liep ik naar}{de telefooncel}\\

\haiku{Ik was in vier, vijf'.}{sprongen in  t portaal}{en trok de deur open}\\

\haiku{Heel mijn leven van.}{de laatste maanden had tot}{dit besluit geleid}\\

\haiku{Het was alsof ze,.}{me dood had gewaand maar nu}{hervonden had}\\

\haiku{{\textquoteright} zei ze toen we daar, {\textquoteleft};}{op die kamer kwamenals}{een zieke zwerver}\\

\haiku{Als dat dan moet, dan,.}{maar de ergste dan moet ik}{jou verloochenen}\\

\haiku{Zo kwam het dat zijn,.}{arm begon te trillen van}{ingehouden lust}\\

\haiku{{\textquoteright} zei hij met een diep, {\textquoteleft}?}{teder geluiddat ik dus}{jou verraden had}\\

\haiku{Toen wierp ze nog een.}{paar blokken op het vuur en}{liet hen weer alleen}\\

\subsection{Uit: Fragmentarisch. Nagelaten proza}

\haiku{een prachtige \'en.}{afschuwelijke droom over}{tuinen en holen}\\

\haiku{King moet Juliette,,.}{die van een lustmoord verdacht}{wordt ontmaskeren}\\

\haiku{{\textquoteleft}Ik heb het leven,{\textquoteright}.}{in zijn essentie ontmoet}{en beleefd denkt hij}\\

\haiku{het conflict dat zich {\textquoteleft}{\textquoteright} {\textquoteleft}{\textquoteright}.}{kan voordoen tussenmuze}{enmenselijkheid}\\

\haiku{Alleen Erica Hart,,.}{begrijpt hem sterker nog zij}{is het met hem eens}\\

\haiku{- Ze begreep dat hij,.}{dronken was en gaf hem zijn}{zin ze keek ernstig}\\

\haiku{Hij tastte nerveus.}{naar zijn zakdoek en drukte}{die tegen de mond}\\

\haiku{Niettemin, de tand,.}{moest dienen als uitgangspunt}{hoe kon hij anders}\\

\haiku{Hij ontledigde:}{zijn mond van het teveel aan}{speeksel en begon}\\

\haiku{Ze informeerde,.}{nadrukkelijk naar Adriaan}{en naar Elsa zelf}\\

\haiku{Hoe laat begint ook? -,.}{maar dat bezoekuur Om twee}{uur antwoordde ze}\\

\haiku{Ze kon desondanks.}{een kind zijn dat in hem een}{volwassene zag}\\

\haiku{Eerst dacht hij dat dit.}{een droombeeld was dat tot zijn}{bewustzijn doordrong}\\

\haiku{Lieve Emmie, ik.}{heb vandaag rustig op mijn}{kamer gezeten}\\

\haiku{{\textquoteleft}dan wacht ik maar{\textquoteright} - de:}{lezer kan niet nalaten}{zich af te vragen}\\

\haiku{Het was, ik zal het,:}{maar bekennen moeilijk om}{daar niet te denken}\\

\haiku{In het telegram,,:}{dat hij opgaf maar dat niet}{verstuurd werd schreef hij}\\

\haiku{Ik loop het dorp uit,,.}{langs kleine wegen tot ik}{bij een groot bos kom}\\

\haiku{Om u de waarheid,.}{te zeggen ik weet niet veel}{van schilderkunst af}\\

\haiku{Omdat,{\textquoteright} zei ze, {\textquoteleft}het,.}{juist zo leuk is als je voor}{iets op moet passen}\\

\haiku{{\textquoteright} {\textquoteleft}Daarom bent u wel.}{haar dochter en toch niet de}{mijne bij voorbeeld}\\

\haiku{Hij sloeg zijn benen.}{over elkaar en vouwde zijn}{handen over zijn buik}\\

\haiku{Ze zei nog veel meer -}{van zulke dingen toen ze}{we stonden juistvoor}\\

\haiku{Weineen, het moest al.}{een miljonair zijn die dit}{zou willen hebben}\\

\haiku{{\textquoteright} Hoofdstuk II [Anna]:}{Blaman De hospes tikte}{op mijn deur en riep}\\

\haiku{Zijn mening over mij.}{was dat hij me verre van}{indrukwekkend vond}\\

\haiku{Al houd ik van je,,.}{lieve Jan ik besta toch}{ook nog wel alleen}\\

\haiku{Wat daar gebeurde,.}{of nog gebeurt ze willen}{het me niet zeggen}\\

\haiku{Maar dan heeft nu dat.}{huis misschien zijn bekoring}{voor u verloren}\\

\haiku{Zelfs al had ik zo'n,.}{relatie dan zou ik het}{je nog niet zeggen}\\

\haiku{Maar toen Moeke het,.}{bereid heeft heeft zij er zelf}{niets van willen eten}\\

\haiku{Mama is dood, en.}{jij en Anne-Marie}{moeten heel stil zijn}\\

\haiku{wie het was - en ik,}{doe het expres niet mijnheer}{want een patser ben}\\

\haiku{En Victorine,.}{die denkt dat hij mij misschien}{bij zich zou houden}\\

\haiku{Ik zocht overal, ja.}{zelfs onder het bed en in}{de nog lege kast}\\

\haiku{Ik heb daar een film.}{gezien waarin een man door}{zijn vrouw werd vermoord}\\

\haiku{Ik herinnerde.}{mij dat ik ontbijt en lunch}{had overgeslagen}\\

\haiku{Hij legde rustig:}{een hand in haar gebogen}{hals waarvan hij dacht}\\

\haiku{Ze wist dat ze zich,.}{had misdragen en had toch}{niet anders gekund}\\

\haiku{De een fluisterde,}{je met een verzuurde adem}{liefdewoorden toe}\\

\haiku{En ondertussen.}{verloor hij meer en meer greep}{op Victorine}\\

\haiku{Als ik vrijdagavond,;}{laat komen zou vond ik haar}{misschien al in bed}\\

\haiku{Hoeveel bloemen denk '?}{je dat er in \'e\'en baan van}{t behang zitten}\\

\haiku{Victorine en.}{haar zuster bleken veel te}{kunnen begrijpen}\\

\haiku{- Schneider liet  me.}{de kamers zien en Willem}{kwam achter ons aan}\\

\haiku{De troosteloosheid.}{in het vertrek voorzag haar}{van een aureool}\\

\haiku{Hij was niet alleen,.}{geprikkeld maar innerlijk}{behoorlijk overstuur}\\

\haiku{Immers, de mens deelt.}{deze appreciatie}{met geen enkel dier}\\

\haiku{{\textquoteright} Haar vriendin, Annie,,:}{houdt van Hilda omdat ze}{zo menselijk is}\\

\haiku{{\textquoteleft}echt een mens, een mens{\textquoteright},.}{vulgair van echtheid mede}{dank zij haar tandpijn}\\

\haiku{Te groot zijn ook de.}{blanke tanden van Sara}{Obreen in Vrouw en vriend}\\

\haiku{Maar zelfs wanneer zij,.}{haar gebit verloor zou hij}{nog van haar houden}\\

\haiku{In de zich daarna}{ontspinnende discussie}{uit een criticus}\\

\haiku{Herinneren we, ({\textquoteleft}{\textquoteright});}{ons Christiaan die zijn hart}{aan de dood schenktSchets}\\

\haiku{Dat de kaartlegster,;}{hem de dood voorzegde raakt}{hem eigenlijk niet}\\

\haiku{Wildhaas antwoordt met:}{een ook heden ten dage}{nog modern praatje}\\

\haiku{hij lijdt onder een:}{existentieel failliet dat}{alle mensen treft}\\

\haiku{Denkend aan Stella,,:}{krijgt hij opnieuw en voor het}{laatst een hartaanval}\\

\haiku{Het is niet goed dat{\textquoteright} (,)}{de mens alleen zijblz. 148}{vgl. Genesis 2:18}\\

\haiku{En daarom krijg ik,{\textquoteright} ().}{nu die hartkwaal die helpt me}{er wel uitblz. 63}\\

\haiku{Waarom moet je die?}{twee met alle geweld op}{mekaar betrekken}\\

\haiku{Anna Blaman, aan,;}{wie het verzoek gericht was}{reageerde prompt}\\

\haiku{Anna Blaman is.}{haar hele leven zwak en}{ziekelijk geweest}\\

\haiku{Zij beschreef deze ();}{ziekteperiode in}{De verliezers1960}\\

\haiku{En als ik nu eens,?}{een stap vooruit deed wat zou}{er dan gebeuren}\\

\haiku{Het bezorgde je,.}{kippevel het dreef je de}{tranen naar de ogen}\\

\haiku{{\textquoteleft}Een reisverslag van{\textquoteright},,,-.}{Anna Blaman De Gids 135e}{jaargang blz. 6066}\\

\haiku{7Jonas' naam verwijst,.}{wellicht naar Anna Blamans}{doopnaam Johanna}\\

\haiku{9In een passage,:}{waarin Jonas Marie voor}{het eerst een kus geeft}\\

\haiku{79{\textquoteleft}Alleen heiligen{\textquoteright} (,,.}{kunnen op water lopen}{blz. 159 160 vgl. Matth}\\

\haiku{{\textquoteleft}Misschien heb je al,.}{iets gehoord maar ik ben in}{Amsterdam gestrand}\\

\haiku{100Anna Blaman over (),.}{zichzelf en anderenMijn}{eigen zelf blz. 101}\\

\subsection{Uit: De kruisvaarder en Ontmoeting met Selma}

\haiku{Ze boog het hoofd en.}{liep schuw naar de trap die naar}{de hutten voerde}\\

\haiku{Maar onmiddellijk.}{daarop liep ze resoluut}{op Virginie toe}\\

\haiku{Maar ondertussen.}{bette ze toch de ogen en}{kamde ze haar haar}\\

\haiku{maar ga ik, dan valt,.}{er misschien iets te winnen}{terug te winnen}\\

\haiku{Aller ogen keken.}{snel van Virginie weg en}{er viel een stilte}\\

\haiku{{\textquoteright} - Ze keek terzijde.}{en zag het mooie gezicht van}{Louise Riffeford}\\

\haiku{Ze botsten, elk in,.}{eigen gedachten verstrikt}{tegen elkaar op}\\

\haiku{Ze reageerde.}{niet en wachtte af wat hij}{verder zeggen zou}\\

\haiku{Niemand wist dat ze.}{daarbinnen was en geen stem}{die haar terug riep}\\

\haiku{Zo fantaseerde,.}{Louise Riffeford ver op}{de feiten vooruit}\\

\haiku{Beiden keken ze,.}{naar Louise Riffeford die}{op de drempel stond}\\

\haiku{Een kort ogenblik bleef.}{Louise Riffeford nu nog}{op de drempel staan}\\

\haiku{Ze danste met die:}{razend knappe officier}{Sterreveld en zei}\\

\haiku{Toen schreef ik alles:}{wat hij tegen me gezegd}{had op en ik dacht}\\

\haiku{Een enkele blik.}{op haar was voldoende om}{dat vast te stellen}\\

\haiku{Nu neemt zo'n jongen,.}{een meisje en neem hem dat}{dan maar eens kwalijk}\\

\haiku{Met heimelijke.}{afkeuring keek hij naar de}{paren die swingden}\\

\haiku{Eerst voelde hij de.}{neiging om op te staan en}{naar hem toe te gaan}\\

\haiku{Zijn enige succes.}{was dat de buitenwereld}{hem dat niet aanzag}\\

\haiku{{\textquoteright} - {\textquoteleft}Kijk, daar is Louise,.}{Riffeford en alsof het}{haar eerste bal is}\\

\haiku{{\textquoteleft}Als dat niet langer.}{duurt dan een gesprek heb ik}{daar niets op tegen}\\

\haiku{Overigens, het moest,.}{een visioen zijn geweest}{het kon niet anders}\\

\haiku{{\textquoteright} - Hij hield op en keek.}{weer naar het trotse gezicht}{in de portretlijst}\\

\haiku{{\textquoteleft}Het lijkt me dat ik,.}{de moed daartoe moet hebben}{wat daar ook van komt}\\

\haiku{Stormachtig was het,.}{maar van een storm kon je toch}{nog lang niet spreken}\\

\haiku{En ik heb zo goed,,.}{als nooit slaap of beter ik}{slaap zo goed als nooit}\\

\haiku{Daarop scheurde hij:}{de bladzijden zorgvuldig}{in snippers en zei}\\

\haiku{Weten is iets wat,.}{je alleen doet en dat is}{ook maar het beste}\\

\haiku{{\textquoteright} - Ze probeerde ook,.}{nog het elektrische licht maar}{dat deed het niet meer}\\

\haiku{Hij, de Alziende,.}{keek neer op alles wat op}{aarde geschiedde}\\

\haiku{Maar dat viel niemand,.}{op God niet en de speelse}{waterdieren niet}\\

\haiku{Louise Riffeford.}{had allang begrepen dat}{ze verloren was}\\

\haiku{Hij keek neer op de.}{plaats waar de Kruisvaarder ten}{onder was gegaan}\\

\haiku{Haar moeder liet een.}{verdrietig afkeurende}{blik op haar rusten}\\

\haiku{{\textquoteleft}Egbert is hier voor,,.}{je geweest je bent net te}{laat hij is net weg}\\

\haiku{{\textquoteright} {\textquoteleft}Stil maar, stil  maar,,.}{moeder nu heb ik iemand}{het leven gekost}\\

\haiku{Een golf van schaamte,.}{steeg in haar op schaamte en}{verontwaardiging}\\

\haiku{De kapitein van.}{het schip was opgestaan en}{kwam haar tegemoet}\\

\haiku{Dat ging natuurlijk.}{over nicht Marie die een breed}{pad bewandelde}\\

\haiku{een lach echter die:}{zich bevrijdde tot warme}{vreugde toen ze zei}\\

\haiku{Ze ging een nichtje:}{van de boot halen en vond}{er Jeanne Brondag}\\

\haiku{Er was beslist in.}{haar moraal iets dat daarmee}{correspondeerde}\\

\haiku{{\textquoteright} zei Jeanne zwak, {\textquoteleft}.}{dat er een vriendschap tussen}{ons geboren was}\\

\haiku{Claartje van Dort zat,,.}{tegenover haar zwijgend met}{neergeslagen ogen}\\

\haiku{Ze had niets gezegd,.}{alleen maar koud geglimlacht}{en was weggegaan}\\

\haiku{{\textquoteleft}Ik vind het toch zo,{\textquoteright}.}{heerlijk dat je bij me bent}{zei Selma spontaan}\\

\haiku{Ze wist nog niet of.}{ze dat weten kon en toch}{van Selma houden}\\

\haiku{Daar zoende hij me.}{nog alsof hij dat die nacht}{nog niet gedaan had}\\

\haiku{En ik vroeg me af.}{of ik wel wijs gedaan had}{met dit huwelijk}\\

\haiku{Ze hervond Selma,,.}{slapend vredig met een mooie}{blos op het gezicht}\\

\haiku{zoals ik me dat -...}{altijd droomde zelfs al zou}{dat niet gebeuren}\\

\subsection{Uit: Ontmoeting met Selma}

\haiku{Dat ging natuurlijk.}{over nicht Marie die een breed}{pad bewandelde}\\

\haiku{een lach echter die:}{zich bevrijdde tot warme}{vreugde toen ze zei}\\

\haiku{Ze ging een nichtje:}{van de boot halen en vond}{er Jeanne Brondag}\\

\haiku{Er was beslist in.}{haar moraal iets dat daarmee}{correspondeerde}\\

\haiku{Claartje van Dort zat,,.}{tegenover haar zwijgend met}{neergeslagen ogen}\\

\haiku{Ze had niets gezegd,.}{alleen maar koud geglimlacht}{en was weggegaan}\\

\haiku{Ze glimlachte, streek...}{met een fijne hand over mijn}{haar en noemde Erik}\\

\haiku{Ze wist nog niet of.}{ze dat weten kon en toch}{van Selma houden}\\

\haiku{Daar zoende hij me.}{nog alsof hij dat die nacht}{nog niet gedaan had}\\

\haiku{Gert is goed, die houdt,}{zo eerlijk van me maar de}{nachten hier met}\\

\haiku{En ik vroeg me af.}{of ik wel wijs gedaan had}{met dit huwelijk}\\

\haiku{Ze hervond Selma,,,.}{slapend vredig met een mooie}{blos op het gezicht}\\

\haiku{zoals ik me dat -...}{altijd droomde zelfs al zou}{dat niet gebeuren}\\

\subsection{Uit: Op leven en dood}

\haiku{Hoewel ik doodmoe,.}{was kon ik niet besluiten}{om naar bed te gaan}\\

\haiku{De nette mensen}{hadden ze waarschijnlijk op}{de grond zien vallen}\\

\haiku{Ik voelde dat ik.}{haar hart niet trefzekerder}{had kunnen raken}\\

\haiku{Nu kan het wild langs.}{komen en het arme beest}{ruikt het niet eens meer}\\

\haiku{O Marie, als ik,!}{je toch eens vond als ik je}{toch eens tegenkwam}\\

\haiku{Die waarzeggerij,.}{heeft indruk op je gemaakt}{je gelooft erin}\\

\haiku{{\textquoteleft}Pas op, zeg dat niet,.}{te hard of ze gaat het er}{nog op aanleggen}\\

\haiku{dezelfde wanhoop,?}{vind maar magisch bezworen}{in een illusie}\\

\haiku{Haar mooie bezielde.}{ogen glinsterden alsof er}{vuur in water scheen}\\

\haiku{Ik nam een taxi naar,.}{huis terug want het was al}{ver over middernacht}\\

\haiku{Het had immers maar.}{een haar gescheeld of ze was}{mijn redding geweest}\\

\haiku{De vrouwen vormden;}{toen zeer gaarne clubs waaruit}{de man geweerd werd}\\

\haiku{Ze was jaloers, maar.}{ze exalteerde zich boven}{de jaloezie uit}\\

\haiku{{\textquoteright} En toen zette ik.}{ook dat portret weer terug}{en greep het laatste}\\

\haiku{Na die kus had ik,.}{dat ook kunnen doen maar er}{was iets veranderd}\\

\haiku{{\textquoteleft}Ik zou nooit beter.}{voor hem kunnen zijn dan ik}{de laatste tijd ben}\\

\haiku{Die hand kwam zwaar op.}{haar dij terecht en gleed af}{in haar vrouweschoot}\\

\haiku{net zolang tot de...}{schone schijn eraf sprong als}{glanslak van oudroest}\\

\haiku{Die onverwachte.}{bliksemsnelle beroering}{sloeg haar door het bloed}\\

\haiku{Dus,{\textquoteright} vroeg ik, {\textquoteleft}daaruit.}{kan ik niet afleiden of}{je iets om me geeft}\\

\haiku{En ik hoef niet eens.}{naar je te kijken om te}{weten hoe dat is}\\

\haiku{Ik voel er alles,.}{voor maar er is ook het een}{en ander tegen}\\

\haiku{Ik moest uitrusten:}{en ondertussen kalm en}{nuchter nadenken}\\

\haiku{{\textquoteright} zei ze, {\textquoteleft}Paul zei 'ga,.}{hem opvrolijken ik geef}{je carte blanche}\\

\haiku{Hoe het ook zij, om:}{te beginnen zou ik me}{daar maar aan houden}\\

\haiku{Feitelijk was hij.}{het dus die haar met een lijk}{in huis opscheepte}\\

\haiku{Dat was nu wel geen,.}{boos opzet maar hij wist dat}{hij een hartkwaal had}\\

\haiku{Ik weet het wel, je}{kunt er niets aan doen dat je}{me niet kunt nemen}\\

\haiku{Of misschien was het.}{enkel maar een verschijnsel}{van volwassenheid}\\

\haiku{Ze schrok er zelf van,,.}{ze kleurde er trok een vaag}{rood over haar voorhoofd}\\

\haiku{Ik strekte dus de,:}{rug keek hen vastberaden}{tegemoet en zei}\\

\haiku{genoeg was om te,.}{begrijpen waar het om ging}{als ik maar wilde}\\

\haiku{Ik zag nog juist hoe.}{vastberaden ze de deur}{van haar kamer sloot}\\

\haiku{Dus, wat had hij er?}{dan eigenlijk op tegen}{dat hij sterven moest}\\

\haiku{Waarom moet je die?}{twee met alle geweld op}{mekaar betrekken}\\

\haiku{Ze hijgde van drift,.}{en ze was nog lang niet aan}{het eind van haar wrok}\\

\haiku{Ik zag haar nog staan,.}{voor de piano op de}{portretten wijzend}\\

\haiku{Ik kon de hoop wel,.}{opgeven ik hoefde niets}{meer te verwachten}\\

\haiku{{\textquoteleft}En nou is 't uit,,!}{verdomme nou zal ik je}{eens wat vertellen}\\

\haiku{{\textquoteleft}Lieveling,{\textquoteright} zei ze, {\textquoteleft}}{toen met een lallende tong}{als je boos wordt is}\\

\haiku{Ik wuifde haar zelfs,.}{nog na zoals je iemand}{nawuift die emigreert}\\

\haiku{Gisteren Sally,,.}{vandaag Francisca dat was}{voorlopig genoeg}\\

\haiku{In haar jeugd had ze.}{hem gevolgd tegen de zin}{van haar moeder in}\\

\haiku{In wezen had ze,}{toen al met me afgedaan}{ze sloot me buiten}\\

\haiku{{\textquoteleft}Soms denk ik erover,?}{om weer te trouwen heeft Mea}{je dat niet verteld}\\

\haiku{En daarop zei ze:}{dan toch eindelijk wat ik}{wilde uitlokken}\\

\haiku{Als jonge vrouw was,.}{ze bepaald knap geweest dat}{was nog wel te zien}\\

\haiku{Als je toch maar de!}{vrijheid hield om er wel of}{niet op in te gaan}\\

\haiku{hij stond nog altijd.}{beter tegenover mij dan}{ik tegenover hem}\\

\haiku{Ik zag ervan af:}{om hem tegen te spreken}{en zei alleen maar}\\

\haiku{Ik schrok ervan, ik {\textquoteleft}!}{kon niet zeggenmaar dat is}{helemaal niet waar}\\

\haiku{wat je me daar hebt.}{zitten vertellen weet ik}{trouwens ook allang}\\

\haiku{Waar ik wel onder.}{geleden heb dat zijn heel}{andere zaken}\\

\haiku{Net zolang totdat.}{ik begreep dat het leven}{zelf het verraad was}\\

\haiku{{\textquoteright} zei Paul Stermunt, {\textquoteleft}in.}{de Brooklynbar als je daar}{niets op tegen hebt}\\

\haiku{Er stond me nog maar,,.}{\'e\'en ding te doen meende ik}{opstaan en weggaan}\\

\haiku{Maar voordat hij me.}{achterhaald had wist ik hem}{toch te ontduiken}\\

\haiku{{\textquoteleft}Als je dat mij voor ',.}{t zeggen geeft ga dan nog}{eerst maar even zitten}\\

\haiku{Misschien kwam haar mijn.}{wanhoop alleen maar klein en}{belachelijk voor}\\

\haiku{{\textquoteright} En daarop trok ik:}{mijn handen van mijn gezicht}{weg en zei heftig}\\

\haiku{Haar mooie gezicht was.}{koud en hard om te zien en}{haar ogen stonden groot}\\

\haiku{{\textquoteright} En hierop zei ik:}{natuurlijk wat iedereen}{zou gezegd hebben}\\

\haiku{En geen man die dat,.}{had kunnen verdragen maar}{hij natuurlijk wel}\\

\haiku{Het klonk smartelijk (?).}{en verwijtendkondet gij}{niet \'e\'en uur waken}\\

\haiku{{\textquoteright} zei ik, {\textquoteleft}dat ik je,}{niet heb gehoord ik heb je}{gehoord voor zover}\\

\haiku{Paul was dood, en dat.}{viel niet weg te praten of}{weg te huilen}\\

\haiku{de zee over, het strand,,.}{langs de duinen op maar hij}{ontdekte me niet}\\

\haiku{{\textquoteright} En toen voelde ik.}{plotseling de tranen over}{mijn wangen stromen}\\

\haiku{Zij was en bleef groot,.}{in haar liefde ik was en}{bleef klein in mijn trouw}\\

\haiku{Op een gegeven.}{dag kwam ik thuis en vond ik}{haar afscheidsbriefje}\\

\haiku{Hij stond op en stak,:}{me een hand toe maar toen zei}{hij nog aarzelend}\\

\haiku{Ik zei dus tegen:}{dat jongemeisje met haar}{bezorgde gezicht}\\

\haiku{We liepen toen juist:}{langs de laatste brits en daar}{bleef ze staan en zei}\\

\haiku{{\textquoteright} Maar daar was ik nu,.}{eenmaal te laat mee dat kon}{ik niet meer vragen}\\

\haiku{{\textquoteright} bouwde ze me na, {\textquoteleft}.}{dat je die gevoelens niet}{kunt beantwoorden}\\

\haiku{Vertel het maar,{\textquoteright} zei, {\textquoteleft} '?}{ik toenwaarom heb jet}{haar niet gegeven}\\

\haiku{Het is anders wel,?}{komisch om die twee samen}{te zien vind je niet}\\

\haiku{Het beste is dat.}{we spelen dat ik jou ook}{nog nooit heb gezien}\\

\haiku{{\textquoteleft}Die vriend van me die{\textquoteright}...}{door een auto-ongeluk}{omgekomen is}\\

\haiku{Ze liep de steiger:}{langs en las de namen van}{de boten luidop}\\

\haiku{{\textquoteright} zei ik scherp, maar toen:}{herstelde ik me en zei}{gemoedelijker}\\

\haiku{Ik weet precies hoe '.}{je bent en ik weet precies}{hoe jet bedoelt}\\

\haiku{{\textquoteright} {\textquoteleft}Neen, ik zal het je,,.}{maar zeggen hij ging voor me}{stelen en daarom}\\

\haiku{{\textquoteright} En dan was het best.}{mogelijk dat ik ook niets}{anders bedoelde}\\

\haiku{Ik kan dus alleen:}{nog maar vertellen wat er}{daarna gebeurde}\\

\subsection{Uit: De verliezers}

\haiku{Zuster Vos, ze liep,,.}{tegen de veertig jaar ze}{was gezond robuust}\\

\haiku{wie weet komt er van.}{de ene op de andere}{dag de ommekeer}\\

\haiku{Ze zouden bijna,.}{vergeten dat ze zijn vrouw}{was dertig jaar lang}\\

\haiku{De hand die hij eerst.}{op de kist had gelegd bracht}{hij nu naar zijn borst}\\

\haiku{Het enige dat er;}{op dit moment te horen}{was kwam van buiten}\\

\haiku{Hij had zich van de.}{kist afgekeerd en wilde}{de gang in stormen}\\

\haiku{Het was zo, er was.}{grijs tussen het donkere}{haar aan de slapen}\\

\haiku{- Ik had, zei Driekje,.}{een vriendje dat me nooit op}{tijd terug het gaan}\\

\haiku{Ondertussen zat,.}{de vermoeide man op de}{trap onbeweeglijk}\\

\haiku{hij zou in staat zijn,.}{om te gaan smeken op z'n}{knie\"en te vallen}\\

\haiku{{\textquoteleft}O, dat is wel in,{\textquoteright}.}{orde zei meneer Das en}{dat klonk schijnheilig}\\

\haiku{Helaas, daar kwam ze, '.}{niet toe al zou zet nog}{honderd keer denken}\\

\haiku{Ik zag alleen maar,.}{haar gezicht natuurlijk maar}{daar gaat het toch om}\\

\haiku{In waarheid ben ik,.}{vooral bang om het alleen}{te doen gewoon bang}\\

\haiku{Twee hoogtepunten,.}{en daartussen een donker}{mysterieus niets}\\

\haiku{- Ze nam ze \'e\'en voor.}{\'e\'en van hem over en legde}{ze op het dressoir}\\

\haiku{Ze hoopte van niet,,.}{want hoe dan ook ze moest hem}{in de steek laten}\\

\haiku{Misschien ook in de.}{enige mens die hier alleen}{verder moest leven}\\

\haiku{zich onder haar blik.}{betrapt alsof hij dat wel}{zou hebben gedaan}\\

\haiku{Hij zou haar nog wel,.}{eens op haar nummer zetten}{als hij de kans kreeg}\\

\haiku{{\textquoteleft}Zo,{\textquoteright} zei hij, {\textquoteleft}nou zuip.}{je de kolenkit maar leeg}{als je nog meer moet}\\

\haiku{{\textquoteright} {\textquoteleft}Dat weet je best,{\textquoteright} zei}{hij en hij keek haar met z'n}{fletse blauwe ogen}\\

\haiku{je keek ze rond, naar,.}{de meubels resten uit een}{oude inboedel}\\

\haiku{Haar begrip dan? - Ik...}{heb nooit een beroep  op}{haar begrip gedaan}\\

\haiku{Hield ze geen afstand,?}{door een gebrek aan moed aan}{dieper meeleven}\\

\haiku{Je las het brief je:}{nog eens dat ze voor je op}{tafel had gelegd}\\

\haiku{Er is iets dat me,...{\textquoteright}}{nerveus heeft gemaakt dat is}{toch niets bijzonders}\\

\haiku{Ze zou natuurlijk.}{in ieder geval wachten}{tot ze terug was}\\

\haiku{Weer pakte ze de.}{badhanddoek en droogde haar}{bezwete gezicht}\\

\haiku{Honderd keer kom je.}{in een sterf huis en het doet}{je zo goed als niets}\\

\haiku{Ze hield plotseling.}{op alsof ze zichzelf het}{zwijgen oplegde}\\

\haiku{{\textquoteright} zei ze, {\textquoteleft}ik ga nog,.}{maar zelden naar huis hoe graag}{ze me daar ook zien}\\

\haiku{{\textquoteleft}Die zou ik ge\"erfd.}{hebben van een dankbare}{rijke pati\"ent}\\

\haiku{{\textquoteleft}Ik begrijp er niets,,.}{van meneer Kostiaan maar}{komt u even binnen}\\

\haiku{Want wat was dat dan,!}{ook voor gekheid zoiets hier}{te willen laten}\\

\haiku{Maar Driekje hield zich.}{op een andere manier}{met de vraag bezig}\\

\haiku{{\textquoteright} {\textquoteleft}Dus u sliep zelf ook,?}{nog niet ik heb u dus niet}{uit de slaap gehaald}\\

\haiku{die heeft zich nergens,,.}{over te beklagen maar nou}{ja vergeet het maar}\\

\haiku{Hij dacht ik denk niet,,.}{aan Lucia niet aan zuster}{Vos niet aan morgen}\\

\haiku{{\textquoteleft}Ga naar je nest,{\textquoteright} zei, {\textquoteleft}.}{ze ruwje moet nodig naar}{de dokter lopen}\\

\haiku{Zo deed je met een.}{stuk in je kraag altijd iets}{waar je spijt van kreeg}\\

\haiku{Al was het alleen.}{maar omdat ze die niet van}{u gekregen heeft}\\

\haiku{Louis is nog altijd,.}{haar zoon hij heeft er recht op}{dat te weten}\\

\haiku{De zaak is, hij is,.}{gestolen of verloren}{dat weten we niet}\\

\haiku{Het bevrijdde hem,.}{het nam dat vreselijke}{ziekzijn van hem af}\\

\haiku{Het drong beslist niet.}{tot haar door dat die ring een}{kapitaal waard was}\\

\haiku{al was 't alleen,.}{maar om dat mooie weer zo is}{het toch bijna nooit}\\

\haiku{Je kunt er zelfs met.}{een zonnebril op nog niet}{tegen in kijken}\\

\haiku{Niet dat dat er z\'o,,.}{erg op aan komt die denkt toch}{nooit fraai dat weet je}\\

\haiku{{\textquoteright} Dat klonk van z{\'\i}jn kant,.}{als een fraze maar zo was}{het evenmin bedoeld}\\

\haiku{Maar ze begon er,:}{niet onmiddellijk over ze}{keek op en ze vroeg}\\

\haiku{Hij staarde haar aan.}{alsof ze hem een schrikbeeld}{had voorgespiegeld}\\

\haiku{{\textquoteright} En terwijl ze die:}{koffie inschonk zei hij van}{achter zijn handen}\\

\haiku{{\textquoteleft}Ik dacht het wel, zo,.}{ziet u haar Godzijdank niet}{zo w\'as het ook niet}\\

\haiku{En hij kwam ook nog,.}{met iets anders aan met een}{groot rond vergrootglas}\\

\haiku{Als ze bij me zou.}{logeren zou ik me wel}{schrap moeten zetten}\\

\haiku{Hij het haar wel diep.}{doordringen in dat voorbije}{leven van Lucia}\\

\haiku{Maar ik kende haar,,.}{vergeet dat niet al had ik}{haar dan verloren}\\

\haiku{Houdt u vast aan uw,.}{eigen indruk dat is voor}{mij van enorm belang}\\

\haiku{Ze staarde hem met,.}{diepe ernst aan eigenlijk}{zonder hem te zien}\\

\haiku{Ze hadden er niet,.}{meer naar omgekeken heel}{dat gesprek niet meer}\\

\haiku{Wat zou ze niet te!}{horen krijgen als ze straks}{alles had verteld}\\

\haiku{{\textquoteleft}Ik weet niet of ik,.}{dat zo hoog moet nemen ik}{weet het eerlijk niet}\\

\haiku{Als iemand zegt dat!}{je zo toegankelijk naar}{hem geluisterd hebt}\\

\haiku{Ik vraag Bertha misschien,.}{wel wat zij ervan denkt van}{zoiets waanzinnigs}\\

\haiku{Maar nu zie ik met,.}{m'n eigen ogen wat hij je}{stuurt uit dankbaarheid}\\

\haiku{Hij was nergens op,.}{bedacht het was op heel de}{trap volkomen stil}\\

\haiku{Niet arrogant, niet,.}{afkeurend alleen maar ziek}{en smekend om hulp}\\

\haiku{En bovendien werd.}{zijn suggestie bijzonder}{gunstig ontvangen}\\

\haiku{huiveren als op,!...}{een winteravond buiten niet}{warm genoeg gekleed}\\

\haiku{Je begon dus over, '.}{Loosje en dan had jet}{plotseling over haar}\\

\haiku{{\textquoteright} En toen zag hij ook;}{plotseling wat er met die}{ogen aan de hand was}\\

\haiku{Hij wierp er zich als ',:}{t ware met volle kracht}{tegen aan hij zei}\\

\haiku{Ze zei alleen maar,:}{koel en zakelijk alsof}{er niets gezegd was}\\

\haiku{Het leek erop dat.}{hij nog nooit werkelijk naar}{haar had gekeken}\\

\haiku{Ze wou immers geen,!}{attenties ze wou toch dat}{hij aan haar naam dacht}\\

\haiku{Lucia, je bent zo.}{ontzettend ver weg als je}{zo zit te kijken}\\

\haiku{Zit je verdriet niet,.}{zo alleen te vreten maar}{praat er met me over}\\

\haiku{Zijn blik zwierf over straat,.}{er liep er niet \'e\'en die bij}{Lucia halen kon}\\

\haiku{Ha, dat was geestig,:}{als Lucia het dan wel was}{dan gold voor Lucia}\\

\haiku{In die tijd moet het:}{geweest zijn dat hem voor de}{laatste keer ontviel}\\

\haiku{Ze dacht even na, ze,.}{zocht een vergelijking maar}{kon die niet vinden}\\

\haiku{- En daarop schudde.}{hij enkel maar het hoofd en}{dacht een tijdje na}\\

\haiku{Maar komt u gerust,}{eens naarboven voor een}{kopje koffie zo}\\

\haiku{- Ze had een bordje.}{soep voor hem gemaakt en ging}{daarmee naar binnen}\\

\haiku{{\textquoteleft}Het is z\'o gedaan,{\textquoteright}.}{zei ze nog en ze zette}{vlug twee kopjes uit}\\

\haiku{Er kon hier in huis,.}{wel gestolen worden er}{kon wel brand komen}\\

\haiku{- Ik zei natuurlijk:}{alleen maar wat je in zo'n}{geval altijd zegt}\\

\haiku{Ik leef, dus ik doe,.}{en beslis en dan nog naar}{mijn beste weten}\\

\haiku{Als hij niet ziek is,,!}{zei de zuster wat heb ik}{er dan te maken}\\

\haiku{Die vriendschap gaf me,!}{nu zoveel plezier maar wat}{gaat dat nog worden}\\

\haiku{{\textquoteright} zei ze zacht, {\textquoteleft}maar ik,?}{bedoel zijn er soms dingen}{uit de kamer weg}\\

\haiku{{\textquoteleft}Geen sterveling is,!}{precies wat hij zou moeten}{wezen o God neen}\\

\haiku{En leugens, welk mens,.}{leeft er nu zonder leugens}{meneer Kostiaan}\\

\haiku{Zou u bijvoorbeeld?}{God durven haten als u}{in God geloofde}\\

\haiku{Hij keek haar niet aan,.}{toen hij dat zei hij hield de}{ogen neergeslagen}\\

\haiku{Het hield maar niet op,,.}{het bleef maar aan de gang in}{haar ze wist nog meer}\\

\haiku{U hoeft er niet over.}{te piekeren of ze wel}{of niet van u hield}\\

\haiku{- En dan rijdt ze weg,,,}{op haar fietsje met kromme}{rug en dan denk je}\\

\haiku{En als ze hem dan.}{weer zag kon ze nog wel eens}{van hem opkijken}\\

\haiku{Goed, ze had haar de,!}{mond gesnoerd maar was het niet}{om je te schamen}\\

\haiku{Dat was plagen, en.}{aan de andere kant van}{de lijn bleef het stil}\\

\haiku{Ik zie niets, dat is,.}{nu eenmaal zo maar ik heb}{wel gevoel voor sfeer}\\

\haiku{{\textquoteright} En toen nam ze dat:}{gezicht tussen haar handen}{en zei hartelijk}\\

\haiku{{\textquoteleft}Ik weet het, ik ben,!}{een onmogelijk mens ik}{zeg de gekste dingen}\\

\haiku{Kijk nu eens Bertha, was,?}{die verstandiger dan zij}{of evenwichtiger}\\

\haiku{Halverwege hief.}{ze een hand op en legde}{die op het behang}\\

\haiku{{\textquoteleft}En daarmee h\`eb je ',{\textquoteright}, {\textquoteleft}.}{t gevraagd zei Berthamaar je}{mag het wel weten}\\

\haiku{{\textquoteright} En toen greep ze haar:}{bij de schouders en keek haar}{aan en zei innig}\\

\haiku{Rustig, een beetje,.}{plechtig precies zoals die}{paren die ik zag}\\

\haiku{hij legde in 't:}{voorbijgaan een hand op haar}{schouder en vroeg luid}\\

\haiku{- Maar hij merkte het,.}{al het was de verkeerde}{met wie hij opliep}\\

\haiku{Geen stoffer komt er,.}{op de vloer geen kopje wordt}{er omgewassen}\\

\haiku{Nou, nou, als je dat...}{doet kan je zo oud worden}{als Methusalem}\\

\haiku{Toen kreeg hij ook dat.}{holle zenuwachtige}{gevoel in z'n maag}\\

\haiku{Hij deed maar niet eens.}{meer een poging om wat te}{zien of te horen}\\

\haiku{Ik weet niet wat het,.}{is terecht op de wereld}{zijn en nodig zijn}\\

\haiku{Hij had er genoeg,!...}{van gezien en begrepen}{hij wist het nu wel}\\

\haiku{Je loopt daar alsof,.}{je ergens op af gaat maar}{je doet maar alsof}\\

\haiku{Hij had er nooit bij,.}{stilgestaan maar die Bertha vond}{hij niet sympathiek}\\

\haiku{- Hij vroeg, om maar iets,:}{te zeggen om haar niet te}{laten ontsnappen}\\

\haiku{Hij zei, terwijl hij:}{z'n hand terugtrok en om}{haar schouder legde}\\

\haiku{Maar het was misschien.}{verkeerd om daarover meteen}{al te beginnen}\\

\haiku{{\textquoteright} Maar daar wilde ze,.}{beslist nog niet over praten}{al zat het haar hoog}\\

\haiku{bega je nog vaak.}{de stommiteiten die in}{je natuur liggen}\\

\haiku{Is het geen pak van?}{je hart dat we er nu over}{gesproken hebben}\\

\haiku{{\textquoteleft}Geef die roos maar hier,{\textquoteright}, {\textquoteleft}.}{zei zedie zet ik wel even}{in een glas water}\\

\haiku{Ik zei het alleen,!}{maar voorzichtig maar ik vraag}{u ten huwelijk}\\

\haiku{{\textquoteleft}Ik heb er dagen,,.}{over nagedacht geloof me}{maar ik meen het echt}\\

\haiku{Maar Kostiaan, dat,}{is ernstiger ik zweer je}{dat ik dat oprecht}\\

\haiku{{\textquoteleft}Terwijl alles wat,?}{hier staat even afschuwelijk}{is weet je dat wel}\\

\haiku{Maar door jou is het,,!}{hier goed en gezellig en}{veilig noem maar op}\\

\haiku{{\textquoteright} {\textquoteleft}Natuurlijk,{\textquoteright} zei Bertha,, {\textquoteleft} '.}{maar haar stem hield afstandals}{jet zeggen wilt}\\

\haiku{En verder,  dat,,.}{was het mooiste was die zoon}{van hem die Louis Jr}\\

\haiku{De tranen sprongen:}{haar in de ogen en ze zei}{met verstikte stem}\\

\haiku{Ik laat je niet in,!}{de steek maar ik weet het toch}{net zo min als jij}\\

\haiku{Ik moet ook wat aan,...}{de tijd overlaten de tijd}{brengt wel weer evenwicht}\\

\haiku{Hij ging, hij wierp een:}{honende boosaardige}{blik op haar en zei}\\

\haiku{Ze dronk enkel een.}{kop thee en ging vroeger dan}{gewoonlijk op pad}\\

\haiku{{\textquoteright} De zuster gaf haar:}{de injectie in een beurs}{geprikt been en zei}\\

\haiku{Hij kan me toch niet!}{een hele dag zonder eten}{en drinken laten}\\

\haiku{Toen, met alle kracht,.}{die in haar was brak ze die}{barricade weg}\\

\haiku{Het was een kwestie,.}{van seconden het was een}{wedloop met de dood}\\

\haiku{{\textquoteleft}Ik begrijp het niet,,.}{want u bent een gewone}{zuster waar of niet}\\

\haiku{Maar toch moet u het.}{zijn want altijd voel ik me}{met u verbonden}\\

\haiku{Meneer Das merkte.}{natuurlijk dat ze haar fiets}{weer buiten zette}\\

\haiku{{\textquoteleft}Ik vraag me af wie,.}{er ongelukkiger is}{Kostiaan of ik}\\

\haiku{En toen liep hij naar.}{de tafel en sloeg met een}{vuist in het gebak}\\

\subsection{Uit: Vrouw en vriend}

\haiku{ik had lang geen spijt,.}{van graad of titel die ik}{misgelopen was}\\

\haiku{Maar ik geloof dat,,.}{hij door dat te zeggen zijn}{jeugd vergeten wou}\\

\haiku{Ik kijk alleen maar,.}{naar je gezicht waarvan ik}{afscheid nemen moet}\\

\haiku{Want nauwelijks op.}{de terugtocht hield hij me}{stil en wou me kwijt}\\

\haiku{Een schemerkamer - -,.}{eerst koffiedrinken dan thee}{en sigaretten}\\

\haiku{Voor het eerst had ik,,.}{nu met een ander Jonas}{over haar gesproken}\\

\haiku{Als het te bar werd.}{laveerde ze de kant uit}{van het boertige}\\

\haiku{De morgen had ik -.}{zoek gebracht met Jonas en}{een krankzinnige}\\

\haiku{Ik wist het wel, zij,.}{had wel door dat het met Saar}{en mij niet deugde}\\

\haiku{Het deed me denken,...}{aan de dreinende ritmiek}{van regen regen}\\

\haiku{Hij had wat moeten,.}{zeggen wat anders moeten}{zeggen dan hij deed}\\

\haiku{Een prinselijke,.}{pauper een pauper met een}{prinsenziel was hij}\\

\haiku{Fortuin en liefde -.}{in uw leven zullen u}{de blonden geven}\\

\haiku{Ook Kareltje en.}{zij waren ten slotte aan}{de dijk gaan zitten}\\

\haiku{De lente bracht die;}{jongen met zijn vrouwelijk}{verholen liefde}\\

\haiku{Ze deed een vrouw in ',,.}{t bad die zich bevuild had}{een zachtzinnig mens}\\

\haiku{Maar daarna is ons.}{samenzijn ellendiger}{dan ooit te voren}\\

\haiku{Ik hield krampachtig.}{de blijdschap in mijn labiel}{gemoed in evenwicht}\\

\haiku{{\textquoteright} {\textquoteleft}Nou, ik denk altijd.}{toch nog beter over een man}{dan over vrouwetuig}\\

\haiku{{\textquoteleft}My life, since,.}{I loved you has been one}{prolonged agony}\\

\haiku{Zonder warmte in.}{zijn ogen boog hij zich naar haar}{toe en kuste haar}\\

\haiku{Hij keek haar vlug, en.}{zonder uitdrukking in zijn}{te lichte ogen aan}\\

\haiku{Jonas boog zich over.}{het portier en braakte de}{zwarte koffie uit}\\

\haiku{Het ging er nou maar,.}{om het vol te houden tot}{hij weer boven zat}\\

\haiku{Marie zat aan de,.}{grasglooiing waar ze beschut}{was tegen de wind}\\

\haiku{Ontevreden keek.}{ze naar de karreploeg die}{op het veld werkte}\\

\haiku{Hij zag hoe zij zich.}{rekte om die boven in}{de kast te zetten}\\

\haiku{{\textquoteleft}Ik had de indruk,{\textquoteright}, {\textquoteleft}.}{zei hij ingetogendat}{u iets hinderde}\\

\haiku{Misschien roken we,.}{samen een sigaret dat}{neemt de moeheid weg}\\

\haiku{Ik ging de kamer,,.}{in en streelde terwijl ik}{langs haar liep haar arm}\\

\haiku{Ik moest verdwijnen,.}{en wel onmiddellijk en}{zonder aarzeling}\\

\haiku{Ik moest het weten,,,.}{ik had haar gezien niet waar}{die zondagmorgen}\\

\haiku{Had ze zo'n honger,?}{of was haar maag nu nog niet}{helemaal op streek}\\

\haiku{Ze schonk zich nog een,.}{kopje thee in en dronk dat}{haastig dorstig leeg}\\

\haiku{Schamper haalde ze:}{de schouders op en boog zich}{naar de tafel toe}\\

\haiku{Op een gegeven.}{ogenblik trok ze de voeten}{van die stoel terug}\\

\haiku{een eenzaam man, stram,,,.}{arrogant met een lege}{hoffelijke grijns}\\

\haiku{Elke morgen nam:}{ze aan de piano haar}{oefeningen door}\\

\haiku{Geenerveerd en moe.}{schoof ze de zilvervos wat}{van haar hals terug}\\

\haiku{En bovendien, een,.}{kind het leven geven wat}{een verantwoording}\\

\haiku{Die maandag nog was,.}{hij naar een concert geweest}{een Bachrecital}\\

\haiku{My life since,.}{I loved you has been one}{prolonged agony}\\

\haiku{Op de tast greep ze.}{een nieuwe sigaret en}{zoog het vuur erin}\\

\haiku{Droomde ze wel ooit,?}{dat ze zich er daarna iets}{van herinnerde}\\

\haiku{{\textquoteright} {\textquoteleft}Ik heb,{\textquoteright} zei hij, {\textquoteleft}een,.}{grammofoon met mooie platen}{zoals Tannh\"auser}\\

\haiku{ze greep zijn hand en.}{zo bleven ze zitten toen}{het weer donker werd}\\

\haiku{De lucht was duister,.}{minder sterren waren er}{dan hij gedacht had}\\

\haiku{Maar ook aan morgen,.}{moest hij nu niet denken nu}{was hij met Marie}\\

\haiku{Heel zijn leven was.}{\'e\'en lange hunkerende}{wacht geweest op haar}\\

\haiku{Zou het ooit vriendschap?}{kunnen worden als hij hem}{nu al kwijtraakte}\\

\haiku{Tot morgen,{\textquoteright} zei hij -.}{nog en op het trambalkon}{keek hij nog even om}\\

\haiku{Ik was ondankbaar,,.}{werkelijk dat ik me dat}{niet  zeggen liet}\\

\haiku{Ik was gekleed en.}{driftig wou ik nu op mijn}{beurt naar beneden}\\

\haiku{De zolder stond in.}{strakke binten over heel de}{diepte van het huis}\\

\haiku{Het meisje begon,.}{gejaagd te schreien daarom}{stuurden we haar weg}\\

\haiku{Zodra de bel ging,.}{haastte ik me naar de trap}{en trok de deur open}\\

\haiku{Zulk huilen maakte,.}{me machteloos ik stond daar}{maar en wist geen troost}\\

\haiku{{\textquoteright} Zij wiste met de,.}{vrije hand haar tranen weg er}{was zoveel te doen}\\

\haiku{Zachtjes trok ik de '.}{buitendeur int slot en}{draalde op de stoep}\\

\haiku{Voor koffietijd bracht.}{ik nog even mijn artikel}{aan de directeur}\\

\haiku{Hij was zuinig waar,.}{het op waarderen aankwam}{maar hij had gelijk}\\

\haiku{Mevrouw de Watter.}{keek mijn verbazing met een}{zachte glimlach aan}\\

\haiku{{\textquoteright} vroeg ze kinderlijk,.}{terwijl haarzelf de tranen}{in de ogen welden}\\

\haiku{De stappen hielden,.}{stil voor mijn deur er klopte}{iemand zachtjes aan}\\

\haiku{Op dat moment greep,}{ik hem bij de arm troonde}{hem mee en wachtte}\\

\haiku{Het regende niet,,}{meer de weg was modderig}{en onbegaanbaar}\\

\haiku{Ik wist nu ook, dat.}{hij ternauwernood nog een}{gesprek verwachtte}\\

\haiku{langs de oevers, die,.}{hun tocht vervolgden bleef het}{steekspel onbeslist}\\

\haiku{- Toos was de trede.}{straat teruggelopen en}{ze stak het plein over}\\

\haiku{Ze was nu aan het.}{aarden pad gekomen dat}{naar boven voerde}\\

\haiku{Bij jonge oogst kreeg,.}{ze een ijl angstgevoel dat}{niet onprettig was}\\

\haiku{{\textquoteleft}O neen, blijf nu toch,,{\textquoteright}.}{zitten Saartje zei mevrouw de}{Watter iets te luid}\\

\haiku{Haar ogen lagen nu,.}{vlak onder me ik zag de}{irissen verblauwen}\\

\haiku{Ze staarde leeg en,.}{treurig weg met vochtige}{verblauwde irissen}\\

\haiku{Ze keek haar aan en:}{ze ontmoette een paar ogen}{vol verweer en schrik}\\

\haiku{Want als dat zo niet,.}{was dan hadden we elkaar}{niet zo gevonden}\\

\haiku{Zou je lust hebben,,?}{vanavond ergens heen te gaan}{of morgen Sara}\\

\haiku{- Het ziekenhuis was.}{een door tuinen omgeven}{kloosterlijk gebouw}\\

\haiku{Op de zaal met aan,.}{weerskanten bedden zag ik}{Toos en toen pas hem}\\

\haiku{{\textquoteleft}Ik heb geen tijd, vind ',?}{jet vervelend dat ik}{je alleen laat gaan}\\

\haiku{Er daverde een,.}{trein uit het perron er kwam}{er weer een binnen}\\

\haiku{Ze trok zich los en,,:}{keek me koud afwijzend aan}{haar ogen waren grauw}\\

\haiku{In de bossen nam,.}{elk vrouwtjesdier de vlucht voor}{hem hij was geen dier}\\

\haiku{{\textquoteright} Met een ruk hield ze,,:}{toen stil ze keek me aan in}{koude haat ze zei}\\

\haiku{Ik sloeg de dekens.}{van me af en ging weer op}{m'n bedrand zitten}\\

\haiku{{\textquoteright} Maar ze kijkt me aan.}{of ze me niet herkent en}{zegt natuurlijk neen}\\

\haiku{Mijn adem stokte of,...}{ik huilen moest moorddadig}{wild had ik haar lief}\\

\haiku{Wat er gebeurd was,.}{was te teer om aangeroerd}{te mogen worden}\\

\haiku{{\textquoteright} Als de tram komt steekt,.}{ze waarschuwend de hand op}{wat niet nodig is}\\

\section{Rein Blijstra}

\subsection{Uit: Diagnose}

\haiku{Hier, neem die kruik mee,.}{en die twee glaasjes dan volg}{ik met de sherry}\\

\haiku{Want in de eerste.}{plaats zijn Vincent en ik niet}{verliefd op elkaar}\\

\haiku{Hoe waren ze toen,.}{eigenlijk beiden vraag ik}{me nu wel eens af}\\

\haiku{Dit primitieve,,.}{gevoel je weet wel dat ons}{helemaal niet past}\\

\haiku{{\textquoteleft}Heb jij Vincent ooit,?}{laten merken dat er iets}{tussen ons bestond}\\

\haiku{'s avonds over je vak,.}{te praten al is het dan}{ook maar zijdelings}\\

\haiku{Als je arts bent moet.}{je verantwoordelijkheid}{weten te dragen}\\

\haiku{Dan meen je misschien,.}{dat je helemaal niet meer}{bij ons kunt komen}\\

\haiku{{\textquoteright} En ik haastte me.}{naar mijn kamer en verliet}{kort daarop het huis}\\

\haiku{Je wilde dus nog?}{een R\"ontgenphoto maken}{om zeker te zijn}\\

\haiku{Was het sterven je,?}{dan toch gemakkelijker}{gevallen mijn vriend}\\

\subsection{Uit: Geheim archief}

\haiku{Wraak en vergelding{\textquoteright}.}{zijn organismen in een}{onvruchtbaren schoot}\\

\haiku{Munitie kan men.}{niet van dag tot dag naar een}{andere plaats sleepen}\\

\haiku{{\textquoteright} {\textquoteleft}Ik dacht dat je iets,{\textquoteright}.}{op het spoor was de laatste}{weken of maanden}\\

\haiku{We zijn immers met......}{zijn twee\"en en als je me}{niet begrepen had}\\

\haiku{{\textquoteleft}Zoo, zoo{\textquoteright}, lachte ik, {\textquoteleft},.}{een bevel van een koning}{die reeds onttroond is}\\

\haiku{{\textquoteright} {\textquoteleft}Om het laatste dep\^ot.}{wat ik je aangewezen}{heb te ontruimen}\\

\haiku{Lluch zal zich beter.}{weten te verdedigen}{dan mijn vriend Juan}\\

\haiku{{\textquoteleft}Ik ga even kijken,{\textquoteright},.}{waar mijn zoon blijft zegt hij en}{verlaat den salon}\\

\haiku{Medicijnmannen,,...?}{priesters geneesheeren is}{dat de evolutie}\\

\haiku{{\textquoteright} {\textquoteleft}Ik weet niet, hoe ik{\textquoteright},, {\textquoteleft}.}{er toe kwam zegt Dr. Arriens}{neem me niet kwalijk}\\

\haiku{Bovendien behoort{\textquoteright}.}{de zaal hiernaast immers ook}{tot de eerste klas}\\

\haiku{Hij heeft er zelf om{\textquoteright}.}{gevraagd en je kon het hem}{nu niet weigeren}\\

\haiku{{\textquoteleft}Als we naar boven,{\textquoteright}.}{moeten kunnen we beter}{meteen wegzwemmen}\\

\haiku{Het is bijna of.}{ik een onbestemden angst}{heb voor een vreemd land}\\

\haiku{beschouwde hij het?}{in zijn hart dan toch als een}{sollicitatie}\\

\haiku{{\textquoteright} {\textquoteleft}Wij kennen deze,.}{redenen niet omdat het}{doel te ernstig is}\\

\haiku{Maar dit alles was,;}{nog zoo ver dat het als het}{ware niet bestond}\\

\haiku{Ten slotte moest men.}{de feiten misschien ook kalm}{onder de oogen zien}\\

\haiku{Zijn opmerkingen.}{waren ten minste alleen}{van technischen aard}\\

\haiku{Ten slotte, het was.}{aardig een revolutie}{voor te bereiden}\\

\haiku{Maar gelukkig, het {\textquoteleft}{\textquoteright}.}{was over dag en een vrijwel}{openbare opdracht}\\

\haiku{Elken boom, dien hij,.}{passeerde was een mijlpaal}{naar het leven}\\

\haiku{Het bleek nu, dat de.}{chauffeur door kogels in het}{hoofd was getroffen}\\

\haiku{{\textquoteleft}Het is jammer, dat{\textquoteright},.}{we geen vindersloon kunnen}{eischen overwoog ik}\\

\haiku{Ik hield de parels.}{tegen het licht om me een}{houding te geven}\\

\haiku{Het bleef griezelig.}{stil en ik hoorde nu eerst}{den wind ruischen}\\

\haiku{Bovendien kon ik?}{niet meteen weggaan want jij}{was er immers nog}\\

\haiku{Maar ik vroeg naar een,.}{revolver omdat ik me}{onveilig voelde}\\

\haiku{Misschien heb je me,,.}{zelfs gered omdat je dacht}{dat je me noodig had}\\

\haiku{Boven gekomen.}{ging hij op het bed liggen}{in haar slaapkamer}\\

\haiku{Ik heb er nog eens,.}{over nagedacht maar ik kon}{geen besluit nemen}\\

\haiku{Er was niets van te,.}{zeggen ook in vijf dagen}{kan veel gebeuren}\\

\haiku{hij krabbelde een,.}{kort briefje waarin hij zich}{verontschuldigde}\\

\haiku{zij raakten elkaars.}{leven nergens dan in den}{huiselijken kring}\\

\haiku{het was een moord, een,.}{laffe moord omdat het geen}{terechtstelling was}\\

\haiku{Geen voldoening als,.}{een dier dat zijn prooi bespringt}{noch wraakgevoelens}\\

\haiku{er toch nog wel zin,.}{in zou hebben als het niet}{zoo gevaarlijk was}\\

\haiku{Maar nu zijn het er,.}{plotseling maar vijf van de}{tien die ontsnappen}\\

\haiku{ik bedoel niet de,.}{romantiek de Carmen of}{de fatale vrouw}\\

\haiku{Nu zal ik Maurits,{\textquoteright}.}{van te voren inlichten}{dat zal hem goed doen}\\

\haiku{{\textquoteright} {\textquoteleft}Ik kom pas op het{\textquoteright},, {\textquoteleft}.}{laatste oogenblik zei hij}{vlak voor het einde}\\

\haiku{beloofd had hem de.}{hand te zullen drukken als}{hij dienst weigerde}\\

\haiku{Het verwonderde,.}{Frederik dat hij geen vaag}{gevoel van angst had}\\

\haiku{Maurits en Armand.}{konden nu elk oogenblik}{naar buiten komen}\\

\haiku{Hij gaf zijn vriend de.}{hand en bleef met zijn rug naar}{de wereld gekeerd}\\

\haiku{Zij gaven elkaar {\textquoteleft}{\textquoteright},.}{nog eens de hand.Ga nu weg}{fluisterde Maurits}\\

\haiku{Daarop keerde hij,.}{zich snel om hij kon zoo niet}{door blijven loopen}\\

\subsection{Uit: Het planetarium van Otze Otzinga}

\haiku{Een toeval, de vondst,.}{van een codeboek had hem}{op het spoor gebracht}\\

\haiku{een ladder die vlak;}{achter hem kletterend op}{het plaveisel viel}\\

\haiku{{\textquoteright} {\textquoteleft}Je zou eens met mij,{\textquoteright}.}{naar een roulette moeten}{gaan stelde hij voor}\\

\haiku{Want, zoals je zegt,.}{wij zijn luchthartig en jij}{bent verleidelijk}\\

\haiku{Hij wist niet meer wat,.}{hij wilde noch  wat zij}{precies bedoeld had}\\

\haiku{{\textquoteright} {\textquoteleft}Zij onderscheiden,{\textquoteright}.}{zich niet erg van de onze}{zei hij ontwijkend}\\

\haiku{Zo eens in de tien.}{of twintig jaar komt er hier}{wel eens eentje langs}\\

\haiku{En ik was toch van:}{de andere kant van de}{Melkweg gekomen}\\

\haiku{Hij had zich beter.}{als William Johnson}{kunnen voorstellen}\\

\haiku{Ze was bijzonder;}{aardig en ik was heel lang}{onderweg geweest}\\

\haiku{{\textquoteright} Mimi lachte wat.}{raadselachtig en streelde}{me weer over mijn haar}\\

\haiku{Het is er niet meer.}{uit te houden voor iemand}{die zelfstandig denkt}\\

\haiku{Dan ben ik burger.}{van dit land en dan kan ik}{een baantje zoeken}\\

\haiku{{\textquoteleft}Als dat voor jou iets,.}{nieuws is zijn jullie nog niet}{erg opgeschoten}\\

\haiku{{\textquoteright} {\textquoteleft}In beginsel is,{\textquoteright}.}{alles bijzonder en niets}{bijzonder zei Ralph}\\

\haiku{{\textquoteright} verzekerde Ralph.}{zo plechtig als de leugen}{hem veroorloofde}\\

\haiku{Overigens had u.}{mij kunnen vragen mij te}{legitimeren}\\

\haiku{Nu, dan  heeft uw.}{broer u wel heel slecht op de}{hoogte gehouden}\\

\haiku{En je hoeft niet bang,.}{te zijn dat je het lot van}{Kolisar zult delen}\\

\haiku{, maar Peter, die aan,.}{het stuur zat had altijd haast}{als hij chauffeerde}\\

\haiku{Alice en Peter,.}{kunnen blijven zitten als}{ze geen zin hebben}\\

\haiku{Uw kwaliteiten:}{zijn in deze tijd voor ons}{niet meer van belang}\\

\section{Robert Bloemendal}

\subsection{Uit: Diefstal in de vliegtuigfabriek}

\haiku{Ongeriefelijk,,.}{warm maar niettemin in een}{uitstekend humeur}\\

\haiku{Kort beantwoordt hij,.}{haar groet en begeeft zich naar}{het bed van no. 4325}\\

\haiku{{\textquoteleft}Maar, mijnheer{\textquoteright}, zegt zij, {\textquoteleft}.}{verbaasdal die opgaven}{heeft U daar toch reeds}\\

\haiku{Ten slotte zijn wij,?}{geheel iets anders dan de}{politie niet waar}\\

\haiku{Dat is nu wel geen,.}{schande maar bepaald prettig}{vindt hij het ook niet}\\

\haiku{Iala Reichenbach:}{hem echter te hulp en zegt}{verduidelijkend}\\

\haiku{Dreigend en donker:}{klinkt de roep van de misthoorn}{over stad en haven}\\

\haiku{Ik heb toch over haar,?}{afgrijselijke accent}{gesproken niet waar}\\

\haiku{Hij mompelt zooiets {\textquoteleft}{\textquoteright}.}{alshandschoenen en bergt de}{kwitantie weer op}\\

\haiku{Daarna bijt hij de.}{punt van zijn sigaar af en}{steekt er den brand in}\\

\haiku{{\textquoteright} {\textquoteleft}Dat is werkelijk,{\textquoteright},.}{prachtig juffrouw zegt Holl en}{verdwijnt opgeruimd}\\

\haiku{Allemachtig, ik,!}{ben het hier die vragen stelt}{en niet omgekeerd}\\

\haiku{Dus hij moet zich nog.}{ergens in de omgeving}{verborgen houden}\\

\haiku{De directeuren.}{en afdeelingchefs van de}{vliegtuigenfabriek}\\

\haiku{Zijn kleeren zijn gescheurd.}{en lang zijn achterhoofd loopt}{een dun straaltje bloed}\\

\haiku{{\textquoteleft}Ik ben Jan Wuyster.}{van het politieposthuis}{aan den Adelaarsweg}\\

\haiku{Over anderhalf uur...?}{of zou het misschien morgen}{om kwart over twaalf zijn}\\

\haiku{Steeds lager cirkelt {\textquoteleft}{\textquoteright}.}{deKoetilang boven de}{hoofdstad van Siam}\\

\haiku{Nu rukt de man zich.}{echter los en wil het op}{een loopen zetten}\\

\haiku{Grant steekt een versche:}{sigaar op terwijl mister}{Hegwood verder praat}\\

\haiku{Ongeveer twee uur.}{voorbij Alexandri\"e moet u zich}{dan gereed houden}\\

\haiku{Mijn collega doet.}{Athene aan en zal u in}{Rhodos afzetten}\\

\haiku{De machine raast.}{met vol gas over de bergreuzen}{van het Turksche rijk}\\

\haiku{En elke druppel.}{heeft een andere vorm en}{andere randen}\\

\haiku{Van zijn huishoudster,.}{heeft hij reeds vernomen wie}{zijn bezoeker is}\\

\haiku{Ik herhaal nogmaals,,...{\textquoteright}}{dat het er hoofdzakelijk}{op aankomt dat wij}\\

\haiku{Ik heb gezegd, dat,.}{U geen tijd heeft maar hij laat}{zich niet afwijzen}\\

\haiku{Hij wil zijn naam niet.}{opgeven en ook niet het}{doel van zijn bezoek}\\

\haiku{In minder dan een.}{week heeft zij de heele boel}{prachtig opgeknapt}\\

\subsection{Uit: Rapaille}

\haiku{De inspecteur klopt.}{aan en wil onmiddellijk}{naar binnen stappen}\\

\haiku{Ik zal u...{\textquoteright} {\textquoteleft}Pardon,{\textquoteright}, {\textquoteleft}}{valt de telefoonjuffrouw}{hem in de rede}\\

\haiku{Dat weet u heusch,{\textquoteright}.}{wel antwoordt de colonel}{onverbiddelijk}\\

\haiku{Wanneer Baxter weer,.}{op straat staat is hij een stuk}{wijzer geworden}\\

\haiku{{\textquoteleft}Ja, dat vind ik ook,{\textquoteright}}{geeft mijnheer Grant doodleuk toe}{en kijkt den ander}\\

\haiku{{\textquoteright} {\textquoteleft}Hm, maar een van uw?}{beambten zal haar toch wel}{eens gezien hebben}\\

\haiku{Heeft hij zich vergist,.}{dan moet hij weer van voren}{af aan beginnen}\\

\haiku{Ik heb alleen maar.}{in groote trekken mijn rapport}{kunnen uitbrengen}\\

\haiku{Ik blijf hier toch nog.}{een heelen tijd en wacht hier}{op uw terugkomst}\\

\haiku{Een gek zal het - zooals -,.}{u reeds opmerkte misschien}{doen misschien ook niet}\\

\haiku{Zij zijn op z'n minst.}{een uur voor aankomst van het}{vliegtuig op Croydon}\\

\haiku{Maar het oog van den.}{beambte aan het stuur is}{nog sneller geweest}\\

\haiku{{\textquoteright} {\textquoteleft}U maakte na de?}{plundering van uw zaak uw}{aanspraken geldend}\\

\haiku{{\textquoteright} Met deze ietwat.}{cynische woorden neemt de}{juwelier afscheid}\\

\haiku{Deze kijkt bijna:}{geamuseerd wanneer hij}{de opmerking maakt}\\

\haiku{Rex Allan steekt een.}{sigaret op en gooit haar}{even later weer weg}\\

\haiku{{\textquoteleft}Ja, het is best in,{\textquoteright}.}{orde gekomen stelt Rex}{Allan hem gerust}\\

\haiku{Hij weet echter niet,.}{dat hij Baxter vandaag niet}{terug zal zien}\\

\haiku{{\textquoteright} {\textquoteleft}Ik weet niet wie het,.}{is maar ik weet w\`el wat hij}{op zijn kerfstok heeft}\\

\haiku{{\textquoteleft}Ik ben inspecteur,{\textquoteright}.}{Baxter van Scotland Yard zegt}{hij geruststellend}\\

\haiku{Mijn klanten zien er,.}{meestal anders uit zooals}{u zult begrijpen}\\

\haiku{Maar Curleigh komt ook een.}{handje helpen en nu is}{de strijd gauw beslist}\\

\haiku{Dan richt hij zich tot:}{den inspecteur en stelt de}{verrassende vraag}\\

\haiku{De balkondeuren.}{staan wagewijd open en hij}{rent naar het balkon}\\

\haiku{Nu weten wij ten,,}{minste wat er met onzen}{chauffeur is gebeurd}\\

\haiku{Als alles goed gaat,.}{krijg je een pond en als het}{mis gaat een kogel}\\

\haiku{- Geef mij even je naam.}{en adres op en kom om tien}{uur op Scotland Yard}\\

\haiku{De man, die een kop,.}{koffie zit te drinken kijkt}{onverschillig op}\\

\haiku{{\textquoteright} {\textquoteleft}Af en toe leek het,,{\textquoteright}.}{er wel op colonel zegt}{Baxter en grinnikt}\\

\haiku{Ik zou er niet veel,.}{kunnen opnoemen die u}{dat zouden nadoen}\\

\haiku{{\textquoteright} zegt Baxter, maar eer,:}{hij verder kan gaan valt Curleigh}{hem in de rede}\\

\haiku{{\textquoteright} De deur gaat open en,.}{op den drempel staat een groote}{slanke jongeman}\\

\haiku{{\textquoteright} {\textquoteleft}Het was een idee van.}{Daisy Stenton en zij heeft}{gelijk gehad ook}\\

\haiku{Hij deed het echter.}{niet en ging rustig naast zijn}{leege safe slapen}\\

\haiku{Zult u Jim Curleigh en?}{zijn verloofde werkelijk}{ongemoeid laten}\\

\haiku{{\textquoteright} {\textquoteleft}O, bedoel je dat.{\textquoteright}.}{En dan barsten zij alle}{twee in lachen uit}\\

\haiku{Daisy hoort hem naar.}{de deur gaan en meteen klinkt}{een vroolijke stem}\\

\haiku{{\textquoteleft}Kom onmiddellijk.}{scotland yard stop zeer dringend}{stop   rex allan}\\

\haiku{{\textquoteright} Terwijl zij de doos,:}{begint open te maken zegt}{Jim Curleigh glimlachend}\\

\section{Eug\`ene de Bock}

\subsection{Uit: Hendrik Conscience. Zijn persoon en zijn werk}

\haiku{De soldaten, die,.}{het kamp verlieten voegen}{zich bij hun makkers}\\

\haiku{Met De Laet nochtans;}{gebeurt de briefwisseling}{nog steeds in het Fransch}\\

\haiku{{\textquoteright} Maar dat gaat niet, de.}{gedachten blijven achter}{in zijn hoofd steken}\\

\haiku{Die vraagt Conscience,.}{uit neemt hem mee in zijn tuin}{en verdwijnt in huis}\\

\haiku{en het meisje werd.}{nog met nydigheid tegen}{het harnas gedrukt}\\

\haiku{Hij zal voor een dag.}{zijn klompen verlaten en}{weer een artiest zijn}\\

\haiku{Hij houdt een toespraak.}{en is secretaris van}{de feestcommissie}\\

\haiku{{\textquoteright} De details waar zich '.}{s schrijvers liefde aan hecht}{zijn weinig talrijk}\\

\haiku{Hij steekt de armen,.}{naar haar uit terwijl zij poogt}{afscheid te nemen}\\

\haiku{op haer hoofd stond eene.}{zilveren kroon van zeven}{blinkende sterren}\\

\haiku{Ah, hoe schatert gy,!}{van blydschap hoe magtig slaet}{gy uwe vlerken uit}\\

\haiku{Hij slaat de armen.}{om den herdersstaf zoodat hij}{er kan op leunen}\\

\haiku{hij vertellen moest,.}{en w\`at verzwijgen om zijn}{publiek te winnen}\\

\section{Boeka}

\subsection{Uit: Een koffieopziener}

\haiku{in vluggen draf het.}{erf afreed en den weg naar}{den kampong insloeg}\\

\haiku{Ieder moet vrij zijn,,.}{dus hoe eerder je op je}{zelf woont hoe beter}\\

\haiku{Prawiro die op,.}{meer gerekend had gaf niet}{dadelijk antwoord}\\

\haiku{- Weet jij misschien, of,?}{er ook eene vrouw is die als}{kokkie kan dienen}\\

\haiku{De menschen zouden.}{niet willen en dan werd de}{toewan besar boos}\\

\haiku{Maar daar ging hem een,:}{licht op die sinjo's waren}{allen hetzelfde}\\

\haiku{Ik was nog zeer klein,.}{toen ik in huis genomen}{werd door de nonja}\\

\haiku{Door groote spaarzaamheid.}{had hij een vijftig gulden}{kunnen overleggen}\\

\haiku{Hij ging dus naar de.}{kampong en kreeg een slaapplaats}{bij den hoofdmandoer}\\

\haiku{Tegen den avond kwam,.}{de inlander terug bij}{wien zij inwoonden}\\

\haiku{Medegesleept door,.}{het gat dat zeker daarom}{zoo breed gemaakt was}\\

\haiku{Eerst moesten ze door de.}{koffietuinen en kwamen}{daarna in het bosch}\\

\haiku{- De ouders zijn naar,.}{familieleden waar een}{bruiloft gevierd wordt}\\

\haiku{Weldra zou blijken,.}{dat zijne vermoedens niet}{zonder grond waren}\\

\haiku{- Het is mijn schuld,53 maar,.}{volgens den loerah mag het}{niet dat u hier is}\\

\haiku{Was die betrekking,.}{ver af dan zou de vrouw toch}{niet mede willen}\\

\haiku{De patih vond het,.}{niet noodig Karel dit alles}{mede te deelen}\\

\haiku{Van zijn rit naar huis.}{zou Karel later niet veel}{kunnen vertellen}\\

\haiku{- Neen, maar ik weet wel,.}{dat er den laatsten tijd veel}{vee gestolen wordt}\\

\haiku{- Als sobat dat wil,.}{doen en mij zoo wil helpen}{ben ik zeer dankbaar}\\

\haiku{Steeds dwaalden deze,.}{af naar het ongeval dat}{hem getroffen had}\\

\haiku{Van groote winsten, met,.}{veeteelt te behalen was}{ook niets gekomen}\\

\haiku{Zonder te groeten.}{ging hij heen en begaf zich}{naar zijne woning}\\

\haiku{Karel wist dan ook.}{op den man ten volle te}{kunnen rekenen}\\

\haiku{] Alphabetische.}{Lijst van vreemde woorden in}{dit werk voorkomend}\\

\subsection{Uit: P\`ahkasinum}

\haiku{Instelling eener;}{behoorlijke politie}{en goede rechtspraak}\\

\haiku{Straks in het midden '.}{van den Oostmoesson wordt het}{s nachts nog kouder}\\

\haiku{Na het middagmaal,}{gingen ze slapen om eerst}{wakker te worden}\\

\haiku{Pahkasinum wist in;}{het eerste oogenblik niet}{wat te antwoorden}\\

\haiku{{\textquoteright} zei de man wenkend,.}{zich snel met het ontvangen}{geld verwijderend}\\

\haiku{Vijftig of zestig.}{gulden zal ze er wel voor}{gekregen hebben}\\

\haiku{{\textquoteright} - Dat spreekt vanzelf,{\textquoteright} vond, {\textquoteleft}?}{Pahkasinumwanneer vindt de}{politie nu iets}\\

\haiku{{\textquoteright} Pahkasinum wierp het.}{eindje van zijn strootje weg en}{stak een ander op}\\

\haiku{de hoogte af, de.}{kali door en den weg op}{naar Kondanglegi}\\

\haiku{{\textquoteleft}Ga nu slapen, en!}{heb den moed niet vanavond meer}{buiten te komen}\\

\haiku{Het werd tijd, meende,.}{ze met haar man over een en}{ander te spreken}\\

\haiku{Deze zat in de.}{emper van de loods rustig}{sirih te pruimen}\\

\haiku{- Neen, de hoofdmandoer,.}{was er maar volgens zeggen}{was mijn kind er niet}\\

\haiku{{\textquoteright} - Daghuur is er op,.}{het oogenblik niet alleen}{op de droogbakken}\\

\haiku{Zoo gezellig als.}{op Djembierit was het dan}{ook bij lange niet}\\

\haiku{Pahkasinum vond, dat:}{ze daar erg lang talmde en}{daarom riep hij luid}\\

\haiku{{\textquoteleft}Het is niet goed, dat,.}{uw zoon naar Djembierit gaat}{sta dat niet meer toe}\\

\haiku{Nog een wijle werd,.}{gepraat waarna het tweetal}{opstond en vertrok}\\

\haiku{Die dieren heb ik,.}{met mijn laatste geld betaald}{ik bezit niets meer}\\

\haiku{Op vragenden toon,,:}{bij elke zinsnede even}{wachtend sprak hij zacht}\\

\haiku{- Als er geen geschenk,?}{gegeven wordt zal er dan}{wel iemand willen}\\

\subsection{Uit: P\`ah Troeno}

\haiku{Voor de gedoegan31,.}{hield hij stil steeg af en bracht}{het paard daarbinnen}\\

\haiku{Toen heb ik bevel.}{gekregen om een kantjil}{te laten vangen}\\

\haiku{{\textquoteleft}Ik heb met Sainum.}{in de warong te Kali}{Bidji gegeten}\\

\haiku{De beide lieden.}{zeiden dan ook niets meer en}{vervolgden hun weg}\\

\haiku{indien hij slechts v\'o\'or,.}{donker uit het bosch zou zijn}{was dit voldoende}\\

\haiku{Het hoofd naar den grond,,.}{gebogen de hoed in de}{hand naderde hij}\\

\haiku{Edele figuren,!}{die hij steeds met achting zou}{blijven gedenken}\\

\haiku{Bijna onhoorbaar.}{verklaarde Troeno niet weg}{te zullen loopen}\\

\haiku{Dit bleek gelukkig,.}{niet het geval want ver had}{hij niet durven gaan}\\

\haiku{Ziezoo, die zaak was,.}{voorloopig afgedaan}{dacht de wedono}\\

\haiku{Mijn man moest vrouwen,.}{zoeken maar hij wilde niet}{en nam zijn ontslag}\\

\haiku{vroeg Troeno, die zijn.}{maal ge\"eindigd had en nu}{ook naar buiten kwam}\\

\haiku{De gastvrouw had zich.}{bescheiden uit het vertrek}{teruggetrokken}\\

\haiku{Plotseling klinken.}{buiten zware stemmen en}{hoort men gejoel}\\

\haiku{Ondertusschen werd.}{het lichter en weldra zou}{de zon opkomen}\\

\haiku{Hij vond er een paar,.}{kapala's die evenals hij}{beschutting zochten}\\

\haiku{Hier in de stad zijn.}{de menschen heel anders dan}{in mijne woonplaats}\\

\haiku{- Heeft u iets gemerkt?}{of de loerah gedobbeld}{heeft in de kotta}\\

\haiku{- Ja, de god van de,.}{Chineezen is immers een}{slang zooals ze zeggen}\\

\haiku{Doch hij zeide niets,.}{want immers hijzelf schoof den}{laatsten tijd evenveel}\\

\haiku{- Dat is mijn bamboe,:}{antwoordde Troeno en liet}{er zacht op volgen}\\

\haiku{Dan zal mijnheer niet,.}{willen betalen viel de}{mandoer verstoord uit}\\

\haiku{Negen dubbeltjes,.}{op \'e\'en dag te verdienen}{dat was nog eens mooi}\\

\haiku{- Was het die Chinees,? -.}{die je uit het hok naar de}{kamer bracht Jawel}\\

\haiku{Hij liep toen even bij.}{den warong aan om een kop}{koffie te drinken}\\

\haiku{- Dat is zoo, maar van.}{mijne kinderen laat ik}{er geen meer weggaan}\\

\haiku{Uw kind werkt daar, als,?}{u nu daarvoor geld krijgt is}{dat dan verkoopen}\\

\haiku{ik sta niet toe, dat.}{mijne kinderen bij een}{Chinees aan huis zijn}\\

\haiku{Doch wanneer, zooals thans,,.}{de koffieoogst mislukt dan}{wordt geen geld verdiend}\\

\haiku{Het was haar kind en,,.}{daar had meende zij niemand}{iets over te zeggen}\\

\haiku{hij Toenggah, waar de.}{beesten later heengevoerd}{en geslacht waren}\\

\haiku{Onmiddellijk na,.}{zijn vertrek ontspon zich een}{levendig gesprek}\\

\haiku{Na eenigen tijd kwam.}{de wedono weder te}{voorschijn en wenkte}\\

\haiku{- Neen kandjeng,.}{fluisterde het inlandsche}{Hoofd schier onhoorbaar}\\

\haiku{- Jij woont immers in,.}{Toenggah vroeg de ambtenaar}{hem in goed Javaansch}\\

\haiku{- Als het geen lieden,.}{van hier zijn zijn het stellig}{menschen uit Toenggah}\\

\haiku{Bij zijne woning,.}{gekomen zag hij Sainum}{op het erf zitten}\\

\haiku{Met zijn maal gereed,:}{wierp Troeno het leege blad in}{een hoek en vroeg zacht}\\

\haiku{- Morgenochtend moet,.}{ik naar Tjandoeredjo daar}{zal ik eens vragen}\\

\haiku{De man bleek al te,.}{weten dat Sainum door haar}{man verlaten was}\\

\haiku{Als de assistent,.}{ontslagen is betaalt hij}{natuurlijk niet meer}\\

\haiku{Hij zeide niets meer,.}{nam zijn arit en ging naar het}{huis van den loerah}\\

\haiku{Wie het gestolen.}{heeft is bekend en er is}{al een getuige}\\

\haiku{Ik heb al verklaard,.}{dat ik niet gezien had dat}{het hout vervoerd is}\\

\haiku{Djogoreso, die man blijft,.}{hier laat hem niet weggaan en}{met niemand spreken}\\

\haiku{Van dat geheele.}{verhaal van den loerah was}{natuurlijk niets waar}\\

\haiku{- Die Wariokromo?}{is gisteren immers naar}{het distrikt gebracht}\\

\haiku{Of is het misschien,?}{een welgesteld man als hij}{niet eens een huis heeft}\\

\haiku{Mandoer Kasan was reeds,.}{naar het werk zijne dochter}{vond hij alleen thuis}\\

\haiku{Zijn tweede zoon ging.}{trouwen en nu werd zijn huis}{te klein voor allen}\\

\haiku{7Offermaaltijd op.}{den avond voorafgaande aan}{den 21sten Poewasa}\\

\haiku{8Offermaaltijd op.}{den avond voorafgaande aan}{den 27sten Poewasa}\\

\section{Martinus H. Boelen}

\subsection{Uit: Jongens uit Bergrust}

\haiku{wij met z'n twee\"en,,,,.}{Joop Piet Bertus en Johan vind}{ik meer dan genoeg}\\

\haiku{{\textquoteleft}Vrienden,{\textquoteright} vervolgde,,:}{Wim trots dat z'n idee zoveel}{bijval ondervond}\\

\haiku{En als jullie even,{\textquoteright}.}{je snater houden zal ik}{het je voorlezen}\\

\haiku{Dit aantal mag niet.}{overschreden worden dan met}{aller goedvinden}\\

\haiku{Waarin de club tot,.}{de ontdekking komt dat ze}{geen voorzitter heeft}\\

\haiku{Onze ouders zijn,.}{immers veel te bang dat er}{iets met ons gebeurt}\\

\haiku{Plots klonk de stem van:}{Karel benauwd van achter}{een hogen stapel}\\

\haiku{{\textquoteright} Behoedzaam, voetje voor,,.}{voetje met ingehouden adem}{gingen ze voorwaarts}\\

\haiku{Bijna hadden ze,.}{het kamertje weer bereikt}{toen Karel bleef staan}\\

\haiku{Hier drink maar even en.}{dan zullen wij meteen je}{ouders waarschuwen}\\

\haiku{{\textquoteright} Weifelend stond de.}{Baron stil en keek vragend}{naar den inspecteur}\\

\haiku{Grond er over en we.}{hebben de mooiste schuilplaats}{die je denken kunt}\\

\haiku{Die man lachte hen, '.}{ook nog uit vond hij net of}{t zo lollig was}\\

\haiku{Ze stonden er zelf,.}{verbluft van zo mooi als de}{hut verborgen was}\\

\haiku{Ze blijven nog een,.}{tijdje zitten bomen v\'o\'or}{de vreemde opstapt}\\

\haiku{Zover als Joop wist,,.}{had alleen Wim lucifers}{dat was de afspraak}\\

\haiku{Intussen hadden.}{de andere clubleden}{zich duchtig geweerd}\\

\haiku{{\textquoteright} {\textquoteleft}Vanzelfsprekend,{\textquoteright} vond, {\textquoteleft}.}{de anderedie zijn er}{bij inbegrepen}\\

\haiku{{\textquoteleft}Kijk eens jongens, er,?}{komt rook uit de schoorsteen zou}{de hut bewoond zijn}\\

\haiku{Hier woonde in dien.}{tijd Heer Jan van Lichtenberg}{met z'n gemalin}\\

\haiku{{\textquoteright} Haastig daalden ze.}{af in de richting van de}{aangewezen cel}\\

\section{Jan van Boendale}

\subsection{Uit: Boek van de wraak Gods}

\haiku{Want zoals David,.}{ons verzekert komt daaruit}{alle wijsheid voort}\\

\haiku{{\textquotedblleft}U zult geen zeven,{\textquotedblright}.}{maal vergeven maar zeven}{maal zeventig maal}\\

\haiku{Daartoe zijn ze op.}{grond van hun heilige plicht}{altijd gebonden}\\

\haiku{Weet dat prelaten.}{eropuit zijn alles in}{handen te krijgen}\\

\haiku{van al het bezit).}{is het meeste immers in}{handen van de Kerk}\\

\haiku{De vierde is de,.}{valse eed waarmee men God}{groot verdriet aandoet}\\

\haiku{Maar dit gebeurt niet,.}{overal nochtans gebeurt het}{vaker dan goed is}\\

\haiku{U kunt er evenwel.}{staat op maken dat ik u}{de waarheid zeg}\\

\haiku{Als zijn vader in,.}{aanzien staat is dat des te}{eervoller voor hem}\\

\haiku{Naderhand lieten.}{ze u het bezit na dat}{God hun had verleend}\\

\haiku{Ik betwijfel ten.}{zeerste of landsheren in}{de hemel komen}\\

\haiku{Dat deed hij omdat.}{zij verstandig was en een}{mooi voorkomen had}\\

\haiku{Dit was de moeder,.}{van Alexander zoals ik u}{hiervoor meedeelde}\\

\haiku{De zwaarste straf van.}{alle zal in het laatste}{tijdperk geschieden}\\

\haiku{5 Over de slechtheid:}{van de wereld Onze Heer}{Jezus Christus zegt}\\

\haiku{De mannen dragen,.}{bovendien korte kleren}{tot aan hun middel}\\

\haiku{Ze zullen niets meer.}{bezitten om hun lichaam}{mee te bedekken}\\

\haiku{Toch had hij daar de.}{beschikking over nauwelijks}{\'e\'en op vijf mannen}\\

\haiku{3 Een vertelling}{over een heer die een klooster}{overlast bezorgde}\\

\haiku{{\textquoteleft}Begrijpt u niet dat?}{Ik niet graag verlies wat Ik}{zo duur moest kopen}\\

\haiku{De een is dus de.}{ander zijn broer en ik ben}{hun beider moeder}\\

\haiku{Deel III [vervolg] 13}{Over een strijd die zich voordeed}{in het land van Luik}\\

\haiku{Daar viel degene.}{die de zak op zijn hoofd droeg}{en was op slag dood}\\

\haiku{{\textquoteleft}Vrouw,{\textquoteright} sprak hij, {\textquoteleft}ik ben.}{zo geschrokken dat ik niet}{durfde te spreken}\\

\haiku{Ga onverwijld de.}{kapel binnen die daar staat}{en draag de mis op}\\

\haiku{Zo moeten alle.}{zielen gewis naar de hel}{of naar de hemel}\\

\haiku{{\textquoteright} [...] De zwaarste straf van.}{alle zal in het laatste}{tijdperk geschieden}\\

\haiku{Daaruit volgt evenwel.}{dat de versie uit 1346 n{\'\i}et}{voor hem bedoeld was}\\

\haiku{Boendale moet zich.}{dit van tevoren hebben}{gerealiseerd}\\

\haiku{Hij zegt als gevolg.}{van zijn werk en leeftijd ziek}{te zijn geworden}\\

\haiku{Dat had hij zelf  .}{welbeschouwd ook gedaan met}{de eerste versie}\\

\haiku{zijn bronnen zijn te.}{vertrouwen en het publiek}{kan h\'em vertrouwen}\\

\haiku{zie Willem v, graaf,:}{van Holland Willem ii graaf}{van Henegouwen}\\

\subsection{Uit: Lekenspiegel}

\haiku{Het firmament draait.}{van nature altijd rond}{en mag nooit stilstaan}\\

\haiku{Want het firmament.}{draait binnen \'e\'en dag en \'e\'en}{nacht helemaal rond}\\

\haiku{En als zij dan in,.}{het oosten opkomt dan ziet}{men de dageraad}\\

\haiku{God schiep de mens niet,:}{alleen uit aarde maar uit}{de vier elementen}\\

\haiku{Dan verliest hij kracht.}{en verstand die Onze Heer}{hem gegeven had}\\

\haiku{Als ze verdrogen,.}{of omvallen gaat hun ziel}{eveneens te gronde}\\

\haiku{In Hem bevinden.}{zich altijd genoegens en}{volmaakte vreugde}\\

\haiku{Kijk nu toch eens wat.}{een ellende er voortkwam}{uit Eva's kwetsbaarheid}\\

\haiku{Hoe meer men met hen,.}{omgaat hoe meer  men te}{schande wordt gemaakt}\\

\haiku{{\textquoteright} Zie, kinderen, en}{merk allemaal op welke}{grote ellende}\\

\haiku{De rook ging kaarsrecht.}{hemelwaarts als een offer}{dat God behaagde}\\

\haiku{Men kan diegenen.}{niet erger vervloeken dan}{met een lang leven}\\

\haiku{En na de kwelling,.}{op aarde branden zij in}{het eeuwige vuur}\\

\haiku{En als zij sterven.}{geeft God hun het hemelrijk}{voor hun zaligheid}\\

\haiku{voor dergelijke,.}{smerige bedorven wijn}{moet men oppassen}\\

\haiku{Daarmee verliest de,,.}{man zijn bezit zijn eer zijn}{ziel en zijn leven}\\

\haiku{ze gebruiken hem.}{met mate en ze worden}{er scherpzinnig van}\\

\haiku{Zij vaardigden het:}{volgende statuut uit voor}{het hele volk}\\

\haiku{Het vierde rijk is,.}{het rijk van Rome dat bij}{Romulus begint}\\

\haiku{Babyloni\"e was,.}{het eerste het machtigste}{en het fraaiste rijk}\\

\haiku{Het Romeinse rijk,.}{was het laatste rijk van de}{vier maar het beste}\\

\haiku{Ieder rijk dat men,.}{uiteenscheurt zal van macht en}{kracht worden beroofd}\\

\haiku{Hij was getrouwd met,.}{Lavinia de dochter van}{koning Latinus}\\

\haiku{Dat zou jammer zijn,,.}{wees daarvan overtuigd en ik}{geloof het ook niet}\\

\haiku{Hoe men het ook wendt,.}{of keert bekering moet uit}{vrije wil geschieden}\\

\haiku{Toen de wereld nog,.}{maar net bestond heeft God heel}{snel wraak genomen}\\

\haiku{Maria bleef in de,.}{tempel waar een engel haar}{dagelijks brood bracht}\\

\haiku{V\'o\'or de geboorte:}{wil een vroedvrouw meten hoe}{het met Maria is}\\

\haiku{zij vertellen aan.}{de drie koningen dat de}{ster verschenen is}\\

\haiku{Het ene brood behoort,.}{aan de ziel toe dat is het}{geestelijke brood}\\

\haiku{Deze punten zal.}{ik achtereenvolgens aan}{de orde stellen}\\

\haiku{De priester die de,.}{mis opdraagt symboliseert}{Christus aan het kruis}\\

\haiku{heiligen schreven,.}{het epistel maar God zelf spreekt}{in het evangelie}\\

\haiku{De aarde schudde:}{bij zijn passie en de zon}{liet  verstek gaan}\\

\haiku{Wereldse wijsheid.}{is niets anders dan dwaasheid}{in de ogen van God}\\

\haiku{zelfs thuis bij de open.}{haard of in een woest bos waar}{niemand het kan zien}\\

\haiku{Ondankbaarheid is.}{de grootste lompheid die op}{de aarde bestaat}\\

\haiku{zowel God als de.}{mensen u. Hieruit valt veel}{winst te behalen}\\

\haiku{U moet zich evenmin.}{zorgen maken over de dood}{of er bang voor zijn}\\

\haiku{Daarom liet  ze,.}{een bed opzetten in een}{stal ver van de haard}\\

\haiku{In het volgende.}{voorbeeld zal dat duidelijk}{gemaakt worden}\\

\haiku{Aldus kan de ziel,.}{behouden blijven zoals}{de wijzen schrijven}\\

\haiku{Want kwaadheid verlamt.}{uw verstand zodat u de}{waarheid niet meer ziet}\\

\haiku{Er was eens een non.}{die al vele jaren in}{een klooster verbleef}\\

\haiku{We meenden nog een.}{keer pret en plezier met je}{te kunnen maken}\\

\haiku{daar doe ik mezelf,.}{een genoegen mee maar ik}{verlies me in God}\\

\haiku{{\textquoteright} - {\textquoteleft}Nee,{\textquoteright} antwoordde de, {\textquoteleft}.}{jongemanik ben nu een}{ander mens dan eerst}\\

\haiku{Wanneer u bij een,:}{vreemde aan tafel zit moet}{u niet veel praten}\\

\haiku{Dat zou hen immers.}{nog meer pijn doen en kwetsen}{dan al hun rampspoed}\\

\haiku{Aldus kan iemand.}{dat beter aan als hij op}{een vreemde plaats komt}\\

\haiku{Te allen tijde.}{moet u het advies van uw}{vrienden opvolgen}\\

\haiku{Iemand die ergens,.}{om vraagt wordt makkelijk op}{zijn teentjes getrapt}\\

\haiku{Houd geheimen voor,!}{haar verborgen dat is mijns}{inziens wijsheid}\\

\haiku{Raadslieden moeten:}{vooral op de volgende}{twee zaken acht slaan}\\

\haiku{Zo verspreiden zij,.}{het werk van de duivel wat}{de duivel graag hoort}\\

\haiku{Daarom moet niemand.}{zich inlaten met zaken}{die hem niet aangaan}\\

\haiku{Eertijds vroeg iemand:}{aan een zeer geleerde klerk}{wat het beste was}\\

\haiku{Want een arme is.}{dag en nacht bezig zich in}{leven te houden}\\

\haiku{Van zo'n vrouw zal men,.}{niets dan schade schande en}{ellende hebben}\\

\haiku{Dit is een van de,.}{belangrijkste geboden}{van God Onze Heer}\\

\haiku{het zou een koning,.}{het leven kunnen kosten}{als hij dat verdient}\\

\haiku{Ad 3 De derde:}{eigenschap gaat over wijsheid}{en bescheidenheid}\\

\haiku{Maar als de heer bang,.}{is en terugwijkt verliest}{hij meteen de strijd}\\

\haiku{Want het gespuis zal,.}{zijn land verlaten omdat}{het bang voor hem is}\\

\haiku{zij hebben gezien.}{en gehoord wat een heerser}{rechtens toebehoort}\\

\haiku{De mensen zouden,.}{nog als vee leven als de}{schrijfkunst niet bestond}\\

\haiku{daarin ligt immers.}{onze zaligheid en ons}{geloof besloten}\\

\haiku{In geen enkel vak.}{kan men zo hoog klimmen als}{in de wetenschap}\\

\haiku{Daarmee boeken ze.}{gemakkelijk succes en}{wordt hun naam bekend}\\

\haiku{Tot slot wil ik u.}{vertellen welke soorten}{dichters er bestaan}\\

\haiku{De eerste is de,.}{liefde tot God de Vader}{in het hemelrijk}\\

\haiku{2 weinig spreken;}{en alleen als het zin heeft}{over goede zaken}\\

\haiku{Wie zich aan deze,.}{tien punten houdt mag een wijs}{man genoemd worden}\\

\haiku{Wie zichzelf liefheeft,.}{bemint de deugd en alles}{wat zijn ziel verheft}\\

\haiku{Men vroeg eens aan een.}{geleerde of het zonde}{is lekker te eten}\\

\haiku{Laat nu ieder voor.}{zichzelf uitmaken met wie}{hij het eens is}\\

\haiku{Wie nooit iets verkeerds,.}{deed hoeft ook nooit zijn fouten}{te verbeteren}\\

\haiku{De duivel zal dit;}{alles op sluwe wijze}{bewerkstelligen}\\

\haiku{Spitsvondigheden.}{en argumenten zullen}{geen zin meer hebben}\\

\haiku{Op vele punten.}{wijkt zijn versie af van het}{Latijnse voorbeeld}\\

\haiku{In Boek iii kan men -}{leren volgens welke door}{God gegevenregels}\\

\haiku{Deze tekst lijkt op.}{het lijf gesneden voor een}{schepenfamilie}\\

\haiku{We hebben ervoor.}{gekozen om de structuur}{intact te laten}\\

\section{Jo Boer}

\subsection{Uit: Beeld en spiegelbeeld}

\haiku{Daarom moet je er,,.}{altijd trots op zijn dat je}{Franse bent Louise}\\

\haiku{Zijn zoon had hem zijn,.}{vele en zware zonden}{vergeven zei hij}\\

\haiku{{\textquoteright} Daarbij namen haar,.}{aandachtige heldere}{ogen ook mij scherp op}\\

\haiku{Hij was bang, dat hij.}{een ge{\"\i}nteresseerden}{indruk zou maken}\\

\haiku{Plotseling zei hij -,?:}{slotsom van welken weemoed}{van welk verlangen}\\

\haiku{Maar Charles had hem,.}{buitengesloten toen in}{dien tijd met Wiesje}\\

\haiku{daar rijdt een mijnheer,.}{met een langen baard op die}{ook een priksnor heeft}\\

\haiku{Zijn felle blik gaf,....}{slechts te lezen wat nobel}{was en zieleklaar}\\

\haiku{Een schande was het,,.}{maar zij had hem lief zij dacht}{aan niets anders meer}\\

\haiku{{\textquoteright} Maar in haar hart zal,:}{een kleine bevroren plek}{zijn waardoor zij denkt}\\

\haiku{Volgende maand ben.}{ik achtentwintig en mijn}{kansen gaan voorbij}\\

\haiku{Ik wist, dat ik weer,,.}{als zo vaak tevoren de}{mindere zou zijn}\\

\haiku{Integendeel, haar.}{spoedige vertrek was nog}{mijn enige  troost}\\

\haiku{Waarom laat ik de,?}{kinderen doorgaan met dit}{wrede vreemde spel}\\

\haiku{Ik zag mijn gezicht,.}{in den spiegel als drijvend}{op een donker meer}\\

\haiku{Hij had niet van pak,.}{gewisseld zoals anders}{zijn gewoonte was}\\

\haiku{Was het mogelijk?}{om zo te lijden en toch}{nog door te leven}\\

\haiku{Daarna werd de deur,.}{van de kamer zacht doch zeer}{beslist gesloten}\\

\haiku{Ook Marceline's.}{erfdeel ging onverdeeld in}{mijn bezit over}\\

\haiku{Als je hem zijn gang,.}{laat gaan brandt hij het hele}{Les Vignobles af}\\

\haiku{En mijn moeder is.}{er net de vrouw naar om dat}{huis te bewonen}\\

\haiku{{\textquoteleft}Jij zou begrijpen,,,.}{wat ik bedoelde Louise}{als je haar kende}\\

\haiku{In geen jaren had.}{ik Maman als een deel van}{mijn leven gezien}\\

\haiku{{\textquoteright} {\textquoteleft}Misschien begrijp je,,.}{het niet Louise maar ik houd}{werkelijk van haar}\\

\haiku{Overal, waar mensen,.}{samenkomen ontlaadt zich}{een spanning van leed}\\

\haiku{Van enigen wrok was;}{toen bij Jean-Paul al geen}{sprake meer geweest}\\

\haiku{dat gespring gaf het;}{rhythme van den tijd aan en}{werd tot Gods uurwerk}\\

\haiku{Want er zijn dingen,,.}{die je niet zeggen kunt zelfs}{tegen M\'em\'e niet}\\

\haiku{Aarzelend stak hij.}{zijn handen uit naar het vuur}{om ze te warmen}\\

\haiku{Maar hij wist ook, dat;}{Charles en hij niets van haar}{te vrezen hadden}\\

\haiku{Ik moet Jean-Paul.}{Les Vignobles aanbieden}{in ruil voor Maman}\\

\haiku{Dien middag dacht ik.}{werkelijk nog niet over mijn}{verraad als verraad}\\

\haiku{Ren\'e keek naar het:}{wit weggetrokken gezicht}{van Charles en dacht}\\

\haiku{Zij zijn weer veel te.}{verlegen om naar Gis\`ele's}{partijtje te gaan}\\

\haiku{Daarom moeten we,.}{het doorzetten dat ze aan}{Ren\'e beloofd wordt}\\

\haiku{Ze hadden het over,.}{dien aardigen mijnheer die}{bij jullie logeert}\\

\haiku{Links van hem zag hij,.}{een deur waarschijnlijk de deur}{van een klerenkast}\\

\haiku{{\textquoteright} Twee pijpekrullen.}{kriebelden over zijn hand. De}{kastdeur ging weer dicht}\\

\haiku{{\textquoteright}        VIII Ren\'e.}{de Saint-Vincent lag in}{zijn bed te lezen}\\

\haiku{27 November 19.. {\textquoteleft}.}{Midden in den nacht ben ik}{wakker geworden}\\

\haiku{Zo werd ik, voor \'e\'en,.}{enkelen avond althans de}{vrouw van Jean-Paul}\\

\haiku{- zat tegenover haar.}{met zijn arm om den hals van}{de geit geslagen}\\

\haiku{{\textquoteleft}Jullie moet je gauw,.}{gaan aankleden wij rijden}{om half negen weg}\\

\haiku{{\textquoteright} {\textquoteleft}Als de sneeuw wel z\'o,,?}{hoog ligt dan kunnen we niet}{verder h\`e tante}\\

\haiku{Er kwam geen einde,.}{aan den rit waarvan een leeg}{huis het einddoel was}\\

\haiku{Dat zij andere,.}{dingen zien dan wat hier den}{laatsten tijd gebeurt}\\

\haiku{Waarschijnlijk wel, want,,.}{ook de hond beneden in}{de hal sloeg niet aan}\\

\haiku{Ik holde de trap,.}{af de donkere hal door}{naar de buitendeur}\\

\haiku{De wijnen waren,;}{uitgelopen de appels}{en peren bloeiden}\\

\haiku{Ik wilde niet, dat.}{je eerste indrukken hier}{pijnlijk zouden zijn}\\

\haiku{Dan ben ik zeker,,.}{van je van een zekerheid}{die ruist door mijn bloed}\\

\haiku{{\textquoteleft}Neen,{\textquoteright} zei hij hardop, {\textquoteleft},.}{in de stiltedit is het}{niet wat ik bedoel}\\

\haiku{En met een diepen,,:}{zucht die van heel ver scheen te}{komen dacht hij weer}\\

\haiku{Waarom hadden zijn?}{broer en hij de estate}{eigenlijk verkocht}\\

\haiku{{\textquoteleft}Vraag je helemaal,,?}{niet waarom ik gekomen}{ben Laperade}\\

\haiku{Ik zag een middel - -;}{meende ik om jullie in}{de hand te houden}\\

\subsection{Uit: De erfgenaam}

\haiku{Iets dauwigs had die,.}{blik als van ongeplukte}{donkere druiven}\\

\haiku{Achter deuren en.}{ramen klonk het getik van}{lepels in kommen}\\

\haiku{Een uil zat in de,.}{spar aan den ingang lachte}{hoonend en vloog weg}\\

\haiku{Het maakte me zoo,:}{kribbig dat mijn stem oversloeg}{van drift toen ik vroeg}\\

\haiku{Zij waren aan haar.}{gestuurd en ze was mij een}{verklaring schuldig}\\

\haiku{Maar wanneer ik haar,.}{stilte hoorde wist ik dat}{ik zwijgen moest}\\

\haiku{Een zoon is nader,,.}{dan een zuster dat zegt het}{bloed dat zegt de wet}\\

\haiku{Ik wist zelf niet meer,,?}{was het een kind was het een}{dom levenloos ding}\\

\haiku{De woorden vormden.}{zich duidelijk en klaar in}{Clementine's hoofd}\\

\haiku{De boerderij lag.}{in een gehucht hier zeven}{uur sporen vandaan}\\

\haiku{Het is het eenige.}{contact dat ik eigenlijk}{met hem gehad heb}\\

\haiku{Mijn zoon was een klein,,.}{wild dier geworden bang voor}{slaag mager en valsch}\\

\haiku{Maria Bakkersvrouw.}{had een brief gekregen van}{haar zoon uit Detroit}\\

\haiku{Ik dacht aan Maman.}{en aan een kanten kleed dat}{zij eens had geklost}\\

\haiku{En soms denk ik, dat.}{David het betreurt dat hij}{mij genomen heeft}\\

\haiku{Er is immers geen,.}{zekerheid en je denkt zelf}{dat het niet waar is}\\

\haiku{de cel en dan de,.}{angst en dan op een morgen}{een deur die open gaat}\\

\haiku{Nee, dit keer zou zij.}{zich verdedigen op haar}{eigen wijze}\\

\haiku{Zelfs het maanlicht werd,.}{verborgen door nevels die}{opstegen uit zee}\\

\haiku{Het was zeven uur '.}{s avonds en alle luiken}{waren gesloten}\\

\haiku{Je bent gemaakt om.}{te schenken en zonder mij}{kan je dat niet eens}\\

\haiku{Het is toch niet een,,?}{zondige liefde die je}{gedreven heeft wel}\\

\haiku{{\textquoteright} En David in zijn:}{onredelijke lust tot}{kwellen vervolgde}\\

\haiku{In de plooien van.}{haar rok waren haar handen}{tot vuisten gebald}\\

\haiku{Maar een kind groeide.}{op tot jong meisje en van}{jong meisje tot vrouw}\\

\haiku{Hij had den indruk,.}{dat zij hem verwachtte al}{deed zij heel verrast}\\

\haiku{verpakt een kleine.}{noodelooze en venijnige wrok}{klaar om te bijten}\\

\haiku{{\textquoteright} ~ Mijnheer pastoor.}{wilde inzicht hebben in}{dat oude drama}\\

\haiku{Ja, hij zou me daar.}{sentimenteel gaan worden}{op zijn ouden dag}\\

\haiku{De buitenwereld.}{verandert om ons heen en}{wijzigt zich niet meer}\\

\haiku{{\textquoteright} Zijn vader en zijn;}{vaders vader en verder}{het verleden in}\\

\haiku{{\textquoteright} Een jaar later had.}{hij de rouwberichten huis}{aan huis rondgebracht}\\

\haiku{{\textquoteleft}Luister, Constance,.}{wij zullen den brief openstoomen}{boven den ketel}\\

\haiku{neen, zijn moeder was,,.}{dood en trouwens hij had nooit}{een moeder gehad}\\

\haiku{De cipier zei, dat,.}{hij geschreeuwd had dat hooren}{en zien je verging}\\

\haiku{Hij zendt een foto.}{en vraagt Clementine of}{zij haar zoon herkent}\\

\haiku{{\textquoteleft}Het is toch een mooi{\textquoteright}.}{bezit zei Clementine}{hardop tot zich zelf}\\

\haiku{zekerheid omtrent.}{de identiteit van No. A}{384178 bestaat er niet}\\

\haiku{Hierop berustte.}{dan ook de verdediging}{van zijn advocaat}\\

\haiku{De  eieren{\textquoteright},:}{waren stuk antwoordde zij}{vaag en ontwijkend}\\

\haiku{{\textquoteright} {\textquoteleft}Toch heeft hij mijn oogen{\textquoteright},,.}{had David geantwoord half}{boos en half lachend}\\

\haiku{Dat is een smaragd,.}{zei het Clementientje van}{de vele beesten}\\

\haiku{Het was nog steeds de.}{vliegende duif voor de kerk}{van Bartholomeus}\\

\haiku{Het is nog uit de.}{erfenis van die arme}{tante Ursule}\\

\haiku{De ratten renden.}{over den zolder en de trap}{kraakte van het vocht}\\

\haiku{Een blauwige rook.}{vulde de kamer en deed}{Constance hoesten}\\

\haiku{Een zenuwachtig.}{plapperend vuurtje overwon}{eindelijk den rook}\\

\haiku{{\textquoteright} Constance kende.}{David vanaf den dag dat}{hij geboren werd}\\

\haiku{{\textquoteright} {\textquoteleft}Zoo zwart{\textquoteright}, zei hij, en, {\textquoteleft}.}{nam een teug brandewijnzoo}{zwart als de zonde}\\

\haiku{Zij moest iets zeggen,.}{waardoor zij allemaal naar}{haar kijken zouden}\\

\haiku{Clementine drinkt.}{te veel en Anne is bang}{voor Clementine}\\

\haiku{Een schaduw gleed over,.}{den cupido een briefkaart}{viel uit den spiegel}\\

\haiku{Het kind was wakker.}{geworden van den slag en}{begon te huilen}\\

\haiku{Een donkere vrouw.}{zal Uw ergernis wekken}{en U bedriegen}\\

\haiku{Constance legde.}{alle kaarten in een krans}{om hartevrouw}\\

\haiku{De zevende kaart,,.}{nichtje leg je omgekeerd}{op hartevrouw}\\

\haiku{Daar zat zij stil en.}{in\'e\'engedoken naast}{Simon Vrachtrijder}\\

\haiku{Je stond hem op te,.}{wachten in zijn boomgaard toen}{we hier aankwamen}\\

\haiku{en sedert dien is.}{David nooit meer als vroeger}{tegen me geweest}\\

\haiku{Zij zou nu naar huis,:}{gaan haar hoofd op zijn knie\"en}{leggen en zeggen}\\

\haiku{Wij zullen alles,.}{samen doen maar laat het weer}{goed zijn tusschen ons}\\

\haiku{Maar bij god, als die,.}{oude dood is dan sleep ik}{jou voor het gerecht}\\

\haiku{Een beetje moe, een,.}{beetje geschrokken omdat}{ze gevallen was}\\

\haiku{{\textquoteright} Een glas trilde in,,.}{de glazenkast het was zoo'n}{gek klagend geluid}\\

\haiku{Steenen boonen... dan kon.}{je gelijk wel kiezelsteenen}{op je bord scheppen}\\

\haiku{Haar gezicht is weer.}{jong geworden en het is}{net of zij glimlacht}\\

\haiku{Zij lag daar in haar,.}{zilverglans het haar in twee}{vlechten gescheiden}\\

\haiku{Maar ik probeer mij.}{te onderwerpen aan den}{wil van mijn zuster}\\

\haiku{Hij herkende het.}{schip zooals hij bijna alle}{schepen herkende}\\

\haiku{Eerst wachten tot de.}{winkels open gingen om naar}{den barbier te gaan}\\

\haiku{Ik haal rat onder,.}{stolp weg laat loopen en rat}{gaat in gaatje drie}\\

\haiku{Want meer dan hij van,.}{zijn moeder gehouden had}{hield hij van zijn zoon}\\

\haiku{Hij wist dat hij het,.}{niet doen zou want hij was de}{vader van zijn zoon}\\

\haiku{Had ook zijn broer niet?}{een stuk van zijn wijnlanden}{moeten verkoopen}\\

\haiku{De dokter was heel,,.}{tevreden ja hij gaat op}{mijn moeder lijken}\\

\haiku{Er is zoo weinig;}{toe noodig om het onkruid te}{laten woekeren}\\

\haiku{{\textquoteright} {\textquoteleft}Zij lachte op een{\textquoteright},, {\textquoteleft}}{griezelige wijze zei}{de jongen later}\\

\haiku{Er waren zeker.}{al twintig menschen in de}{sterfkamer bijeen}\\

\haiku{Ik keek om mij heen}{in dat griezelige huis}{en ik bedacht mij}\\

\haiku{Herinner je je,,?}{hoe David zei dat wij haar}{gevonden hadden}\\

\haiku{Het was niet grooter.}{dan een sinaasappelkrat}{en niet half zoo diep}\\

\haiku{Zij heeft de kasten.}{gesloten en de sleutels}{op tafel gelegd}\\

\haiku{Zij heeft haar hand in.}{de mijne gelegd en zoo}{zijn wij blijven staan}\\

\haiku{In de stad of in,.}{mijn dorp of in een ander}{land waar je maar wilt}\\

\haiku{Zij was te krom of,,.}{te recht te dik of te dun}{te jong of te oud}\\

\haiku{Er is w\'el wind, zei,.}{de boom en dreig me niet zoo}{met die paraplu}\\

\haiku{De schemering wordt,.}{dieper en dieper en toch}{is het nog geen avond}\\

\subsection{Uit: Kruis of munt}

\haiku{Bruno werkte dag.}{en nacht om orde op zijn}{zaken te stellen}\\

\haiku{Haar was het immers:}{duidelijk wat Agatha en}{Robert nog ontging}\\

\haiku{De oude vrouw had.}{haar handen gevouwen over}{haar reticule}\\

\haiku{En vreemd, helder en:}{smekend vroeg de stem van de}{vroegere Agatha}\\

\haiku{daar had helemaal.}{geen uitdrukking onder haar}{geel zijden krullen}\\

\haiku{{\textquoteright} Hij zei m{\'\i}jn eisen,.}{en niet j\'ouw eisen of zelfs}{maar \'onze eisen}\\

\haiku{Het verveelde het,.}{kind nooit dat die schoenen zo}{weinig bewogen}\\

\haiku{De letters staan weer.}{in een vlakje apart met een}{randje er omheen}\\

\haiku{Voor Jules Verne,,.}{is Jopie die niet lezen}{kan nog veel te klein}\\

\haiku{Zij gaan naar haar toe,:}{kijken haar voor de eerste}{maal werkelijk aan}\\

\haiku{Toen de juffrouw dus:}{de volgende dag aan het}{kleine meisje vroeg}\\

\haiku{{\textquoteleft}En, Jopie, kun je,?}{me vertellen hoe onze}{wereld er uitziet}\\

\haiku{{\textquoteright} Dan weer het geluid,,,.}{van blote voeten het bed}{dat piepte stilte}\\

\haiku{Maar je Pa is bij,.}{je weggelopen omdat}{je Ma een spook is}\\

\haiku{Een ding slechts hadden:}{deze uiteenlopende}{karakters gemeen}\\

\haiku{Ik krijg ze pas te,,.}{zien dacht zij spottend wanneer}{zij afgericht zijn}\\

\haiku{De portretten, die,.}{er van haar uit die tijd over}{zijn zeggen van wel}\\

\haiku{In mijn huis wordt die,{\textquoteright}.}{vuiligheid niet gerookt zei}{de oude heer bars}\\

\haiku{Ongemerkt, zonder,.}{dat zij er op betrapt kon}{worden hielp zij hem}\\

\haiku{Zij verlangde naar,.}{Bernards thuiskomst als naar iets}{vaags iets onbestemds}\\

\haiku{De dokters denken,....}{dat je kleine meisje niet}{lang te leven heeft}\\

\haiku{Zijn zoon, die op de,.}{kade stond te wuiven had}{hij afgeschreven}\\

\haiku{meegenomen had,.}{naar de bergen zag zij het}{kindje nu veel meer}\\

\haiku{Is de schorpioen?}{verantwoordelijk voor het}{vergif in zijn staart}\\

\haiku{Kan het sombere,,?}{kleine meisje daar v\'o\'or haar}{werkelijk lachen}\\

\haiku{Hij kriebelt met zijn.}{lange hengel in de hals}{van de lantaarnpaal}\\

\haiku{Het kind wist heel goed,;}{dat zij het poesje niet mee}{naar huis kon nemen}\\

\haiku{Er werd, behalve,;}{de twee meiden een kokin}{in dienst genomen}\\

\haiku{Zij boog zich over de.}{trapleuning heen en keek de}{donkere gang in}\\

\haiku{De volgende dag.}{was zij deze avondlijke}{doodsangst vergeten}\\

\haiku{{\textquoteright} Zij wist niet, dat het.}{kind geruisloos achter haar}{aan geslopen was}\\

\haiku{Als het kind er niet....}{was om een oude vrouw wat}{op te vrolijken}\\

\haiku{Johanna's trotse {\textquoteleft}.}{besluitin dit huis wordt niets}{klandestien gekocht}\\

\haiku{Een oorlog was over....}{de wereld gegaan en hier}{was niets veranderd}\\

\haiku{Jij zult me toch niet,,?}{alleen laten is het wel}{m'n lieve jongen}\\

\haiku{XVII De oorlog had;}{sommige jarenlange}{vetes uitgewist}\\

\haiku{De zusters hadden,,;}{beledigd haar handen van}{hem afgetrokken}\\

\haiku{Ook bestond er geen,;}{twijfel of het kind zou de}{naam veranderen}\\

\haiku{het fototje is al,.}{een jaar oud maar zij is nog}{niet veel veranderd}\\

\haiku{Je kunt in deze,.}{natte kleren niet weer door}{de regen jongen}\\

\haiku{{\textquoteright} Maar hij was zelf te.}{ongedurig om op haar}{antwoord te wachten}\\

\haiku{{\textquoteleft}Kijk, moet je sien, s\`eg....{\textquoteright}:}{Voorzichtig lichtte hij de}{deksel van de doos}\\

\haiku{Schuw en aandachtig:}{nam hij haar met zijn vreemde}{schuinstaande ogen op}\\

\haiku{{\textquoteleft}Ma, je kunt je je.}{Sint Nikolaasfeestje wel}{uit je hoofd zetten}\\

\haiku{De lucht was donker.}{grijs en ongemerkt begon}{het weer te sneeuwen}\\

\haiku{Zij droeg haar paarse.}{kamerjapon en haar haar}{was onopgemaakt}\\

\haiku{in haar paars wollen.}{ochtendjapon bracht zij de}{dagen peinzend door}\\

\haiku{hoe moeilijk het kind,.}{was hypernerveus en zeer}{zwak van gezondheid}\\

\haiku{Het kind had nog steeds,....}{koorts haar arme moeder kon}{niet alleen blijven}\\

\haiku{{\textquoteright} Eerst was er nog geen;}{verandering merkbaar op}{haar moeders gelaat}\\

\haiku{Zelfs nu hij dood is,,.}{gelooft U het niet alleen}{omdat Ik het zeg}\\

\haiku{Aan de kant van het,.}{Frankenslag natuurlijk maar}{toch was het angstig}\\

\haiku{En daarom had haar:}{moeder ook altijd over haar}{vader gezwegen}\\

\haiku{Zij kwetste en wondde;}{Aletta met plagerijen}{van eigen vinding}\\

\haiku{nimmer met de naam.}{van Bernard Landman of met}{wat zij over hem wist}\\

\haiku{De tafel was niet,,,.}{gedekt op de schoorsteen lag}{dreigend een open brief}\\

\haiku{Als je je driftig.}{maakte werd je lelijk en}{verried je jezelf}\\

\haiku{{\textquoteleft}En toch ben ik er,....}{van overtuigd dat je het goed}{met mijn meisje meent}\\

\haiku{Dan ben je student,....}{dan beteken je al wat}{in de maatschappij}\\

\haiku{Mama mag niet te,.}{weten komen dat ik U}{geschreven heb}\\

\haiku{Als ik je nu een,?}{hond beloof zul je dan niet}{van me weglopen}\\

\haiku{{\textquoteright} {\textquoteleft}Dat betekent dus,,.}{dat je aan dingen denkt die}{ik niet weten mag}\\

\haiku{de dochter, die met....}{haar samenwoonde is uit}{het raam gesprongen}\\

\haiku{Boog zij niet naar haar,....}{moeders wensen dan zou het}{het laatste worden}\\

\haiku{Op dat moment, elf,.}{uur in de morgen koos het}{land de derde deur}\\

\haiku{'s Morgens in bed.}{reeds had zij het geluid van}{de misthoorn gehoord}\\

\subsection{Uit: De vertroosting van het troosteloze}

\haiku{WIJ liepen op den,.}{landweg over den olijfberg mijn}{kameraad en ik}\\

\haiku{Te midden van al.}{die planten had Jozef zijn}{graf laten maken}\\

\haiku{Dikwijls op warme.}{zomeravonden zat hij daar}{voor zijn graf en dacht}\\

\haiku{Hij keek mij rustig:}{met zijn donkere oogen aan}{en zeide nogmaals}\\

\haiku{Dagenlang speelde.}{hij in de zon voor de grot}{en was gelukkig}\\

\haiku{Zij antwoordde niet,.}{maar bewoog haar hand heen en}{weer in het water}\\

\haiku{Zij zag zijn schaduw.}{zich afteekenen tegen het}{glanzen der vensters}\\

\haiku{Zij was alleen nog.}{maar een brok koude in een}{hostiele wereld}\\

\haiku{naar die gebogen,,.}{zwartsatijnen figuur die}{ons zwijgend volgde}\\

\haiku{Hij kwam van een dorp,.}{dat vlak bij Parijs ligt en}{dat Bellevue heet}\\

\haiku{Wat begrijpen wij,?}{van de nooden die ons dwingen}{tot onze daden}\\

\haiku{Ik heb de wereld.}{gezien door het beperkte}{prisma van mijn oogen}\\

\haiku{Ik zag het als een.}{soort verraad van mijzelf aan}{de buitenwereld}\\

\haiku{Maar zij moet mij toch,.}{gehoord hebben want zij keek}{op en glimlachte}\\

\haiku{all\'e\'en het kind - was.}{de wereld anders dan ik}{haar ooit gekend had}\\

\haiku{Besefte ik toen?}{al wat dit bezoek in mij}{veranderen zou}\\

\haiku{Was het de weemoed,?}{van de oude vrouw die mij}{langzaam omhulde}\\

\section{Jan L. de Boer}

\subsection{Uit: De erfdochter van de Doorwerth}

\haiku{Het oogenblik scheen,....}{nabij dat dit zijn lichten}{last zou afwerpen}\\

\haiku{Hij nam met ruwen.}{zwier zijn breedgeranden hoed}{voor haar af en boog}\\

\haiku{{\textquoteright} Met een paar woorden;}{gaf Otto opheldering}{van het gebeurde}\\

\haiku{De steden en de.}{burgerstand konden echter}{vooruitgang boeken}\\

\haiku{De geestelijke.}{opende de deur en verdween}{in zijn kamer}\\

\haiku{hoe wreed leden van.}{\'e\'en huisgezin vaak tegen}{elkaar kunnen zijn}\\

\haiku{Lucht en bladeren -.}{en grachtwater alles scheen}{met vuur overgoten}\\

\haiku{In zijn gebogen.}{houding en bewegingen}{lag iets katachtigs}\\

\haiku{In zijn blauwe oogen.}{lag de uitdrukking van den}{peinzer en dichter}\\

\haiku{{\textquoteleft}Eens toefde ik bij,!}{menschen die het zingen en}{spelen liefhadden}\\

\haiku{Voor een laatst vaarwel.}{reed hij het bergpad op naar}{Rinaldi's sterkte}\\

\haiku{Het lieve kind is....}{voor vreugde en levenslust}{geboren en toch}\\

\haiku{Behendig in het.}{gevecht en even bedreven}{in de hoofsche kunst}\\

\haiku{Daar is geen dwaling,!}{mogelijk waar de Meester}{z\`elf aan het roer staat}\\

\haiku{Wie d\`at doet, keert zich.}{van Christus en de oude}{Gemeente-idee af}\\

\haiku{Maar welke plannen?}{rijpten er thans in het brein}{van den Jezu{\"\i}et}\\

\haiku{hij onopgemerkt.}{een gesprek van Walravia}{en haar grootmoeder}\\

\haiku{Hij droeg hooge laarzen,,;}{een donkere blauwe kleeding}{en bruinen mantel}\\

\haiku{zijn vuist viel krachtig:}{op de tafel en dreigend}{luidde zijn antwoord}\\

\haiku{Ik heb ruim de helft.}{verloren van hetgeen ik}{u reeds ter leen vroeg}\\

\haiku{In de dingen van.}{het geloof toonde zij een}{strenge opvatting}\\

\haiku{Ge weet gevoelens.}{van belangen en van wat}{recht is te scheiden}\\

\haiku{{\textquoteleft}Ik mag die dame,{\textquoteright}.}{niet zei Mevrouw Van Voorst na}{een korte stilte}\\

\haiku{Er was een tijd dat,.}{zij hem ontweek ondanks zijn}{openlijke hulde}\\

\haiku{{\textquoteleft}Als u d\`at gelukt, -!}{kom dan vrij tot mij ik zal}{u niet afwijzen}\\

\haiku{zij schrok wakker uit -.}{haar gemijmer Marten Loks}{stond tegenover haar}\\

\haiku{Hoe vaak heb ik je.}{al niet afgewezen en}{steeds kom je terug}\\

\haiku{Want het geloof in;}{de leer is het begin van}{allen vooruitgang}\\

\haiku{Dat gaf u het recht.}{om in den strijd hoogere}{hulp aan te roepen}\\

\haiku{dat is ook zoo, doch.}{slechts voor geestelijk  zeer}{ver gevorderden}\\

\haiku{De Stins van Sweder '.}{heette het onverwinbaarst}{Slot int Oversticht}\\

\haiku{En later, na zijn,.}{dood is hij aan menigen}{Van Voorst verschenen}\\

\haiku{{\textquoteleft}Het moet wel heerlijk!}{zijn met zulke mannen in}{het gevaar te gaan}\\

\haiku{{\textquoteleft}En gij, een zoo sterk,?}{man verlangt niet uw kracht in}{den strijd te toonen}\\

\haiku{{\textquoteleft}Ge behoort dus tot,?}{de nieuwe secte die voor}{de weerloosheid is}\\

\haiku{{\textquoteright} hield Ravenhorst vol. {\textquoteleft},.}{Wie wapens draagt maakt er al}{spoedig gebruik van}\\

\haiku{Plechtrude stond voor '.}{t raam der keminade}{en staarde hem na}\\

\haiku{Hij had een vijand, -.}{die hem bitter had gegriefd}{Heer Jan van Hoeckelom}\\

\haiku{Toen vloeiden tranen.}{van de rozewangen van}{de schoone Kunegond}\\

\haiku{Hij reed vergramd naar,.}{Dorenweerd vroeg een gehoor}{en werd ontvangen}\\

\haiku{{\textquoteleft}Wilt ge mij zweren,?}{dat ge voor immer met dien}{Guido breken zult}\\

\haiku{Toen ik wegreed, vroeg -!}{ik terloops naar de rest n\`og}{ruim een twintigtal}\\

\haiku{De oude Otto.}{onderwees Walravia en}{Ravenhorst keek toe}\\

\haiku{{\textquoteright} vroeg Walravia toen.}{zij teruggingen naar de}{huisgenooten}\\

\haiku{Wel-is-waar ken,.}{ik den weg niet maar gij kunt}{mij dien beschrijven}\\

\haiku{De Heilige Maagd,,.}{en de Heiligen zullen}{hem hoop ik bij staan}\\

\haiku{Ten onrechte weet.}{zij Jaspers weifelende}{houding aan lafheid}\\

\haiku{De grootmoeder liet;}{haar rozenkrans langzaam door}{de vingers glijden}\\

\haiku{De groote oogen van de;}{kleine burchtvrouw werden vol}{angst op hem gericht}\\

\haiku{Met goud wil hij zich.}{na de beleediging niet meer}{tevreden stellen}\\

\haiku{Dit was het portret,.}{van Walravia's vader}{eens Ravenhorsts vriend}\\

\haiku{Het is u, of ge.}{onzen Heer in het uur der}{beproeving verlaat}\\

\haiku{maar wie zich geeft en,,.}{offert gaat niet onder doch}{komt tot hooger rang}\\

\haiku{slechts Walravia's.}{snikken verbraken de rust}{in de ridderzaal}\\

\haiku{Hij zal den uitslag,.}{z\'o\'o maken dat deze uw}{geweten niet drukt}\\

\haiku{wat een breede borst -.}{dat is een gemakkelijk}{te treffen doelwit}\\

\haiku{ook ditmaal wilde!}{hij weer een proeve van zijn}{bekwaamheid geven}\\

\haiku{{\textquoteleft}Ge hebt mijn leven,,!}{gespaard maar denk niet dat ik}{u daar dank voor weet}\\

\haiku{hij opende de deur,;}{daarvan en trad in een klein}{vierkant kamertje}\\

\haiku{{\textquoteright} {\textquoteleft}Dat is mijn vaste,:}{overtuiging want ik zal je}{eens  wat zeggen}\\

\haiku{- die zou als Edelman.}{zeker nooit tot zulk geweld}{zijn toevlucht nemen}\\

\haiku{Toen ik wakker werd,.}{scheen er niets bijzonders te}{zijn voorgevallen}\\

\haiku{{\textquoteleft}In uw kleine kast?}{ligt dus het geld en links staat}{de doos met poeder}\\

\haiku{{\textquoteright} Van Ravenhorst sprak.}{snel en zijn gelaat werd door}{smart verwrongen}\\

\haiku{Achter hem klonk de,.}{schaterende helsche lach}{van de heidin}\\

\haiku{{\textquoteright} zei hij met zijn  ,.}{zachte gevoelvolle en}{welluidende stem}\\

\haiku{Ik toefde toen op.}{een Burcht in het Kleefsche en}{ontmoette er haar}\\

\haiku{Een tweeden keer kon,.}{het toeval eens niet te uwen}{gunste zijn zoo vreest ge}\\

\haiku{hoe valsch bleek nu het,.}{beeld dat hij zich van eigen}{wezen had gevormd}\\

\haiku{Johanna wierp zich.}{hiervoor op de knie\"en en}{bad lang en innig}\\

\haiku{Het volle maanlicht.}{bescheen haar schoon gelaat en}{edele gestalte}\\

\haiku{De roodharige.}{brulde als een wild dier toen}{hij op den grond lag}\\

\haiku{Misschien wordt u dit,,{\textquoteright}.}{later wel duidelijk mijn}{kind vervolgde hij}\\

\haiku{Gij meent dat zonder.}{tusschen-middelaar te}{kunnen bereiken}\\

\haiku{Toen hij haar naar haar,:}{voertuig teruggeleidde}{zei zij in de laan}\\

\haiku{hij is ook niet zoo.}{handig op de jacht en bij}{het visschen als Daem}\\

\haiku{Zij naderde een,.}{boomgroep waarachter zich twee}{mannen bevonden}\\

\haiku{In den avond, toen Daem,.}{vertrok bracht Walravia hem}{door de oprijlaan}\\

\haiku{In het poortgebouw,.}{wachtte haar Louise die haar}{onder den arm nam}\\

\haiku{Voor ditmaal waren.}{de Mennonieten aan de}{vervolging ontsnapt}\\

\haiku{{\textquoteright} {\textquoteleft}Vergeef mij, Heer Van,?}{Hamelen dus zie ik in}{u een Humanist}\\

\haiku{Kent ge de schoone?}{geestelijke liederen}{uit die  dagen}\\

\haiku{Maar de smart verwrong.}{zijn gelaat toen zijn oog het}{hare ontmoette}\\

\haiku{Hij vertelde haar.}{in het kort de voorvallen}{van den laatsten tijd}\\

\haiku{{\textquoteright} Zoo eindigde hun.}{gesprek en teleurgesteld}{trok Ravenhorst weg}\\

\haiku{Maar sterker zou ik,!}{zijn als ik mijn stem verhief}{als Heer Van Doorwerth}\\

\haiku{Vol vertrouwen reed.}{hij daarom nu de slotpoort}{van de Doorwerth door}\\

\haiku{Het is u bekend,?}{dat uw eigen dienaar met}{moordplannen rondloopt}\\

\haiku{Hoeveel schade men,.}{op die wijze doet aan zijn}{ziel wordt niet bedacht}\\

\haiku{dat mijn houding niet{\textquoteright},.}{edel en ridderlijk is zei}{hij na een stilte}\\

\haiku{Zoo wordt Johanna....}{u tot last en ge moet dan}{wel van haar scheiden}\\

\haiku{men moet Haar boven,.}{\`alles gehoorzamen wijl}{men niet anders m\`ag}\\

\haiku{Brieven uit Arnhem,....}{een brief van den kasteleyn}{van zijn goederen}\\

\haiku{Maar in dit verband.}{heb ik nog eens nagedacht}{over Jasper en jou}\\

\haiku{En ik geloof niet.}{dat je oom v\'o\'or dien tijd reeds}{kan worden beleend}\\

\haiku{Niemand mocht ook haar,....}{ware gevoelens kennen}{vooral Jasper niet}\\

\haiku{Gisteren vroeg ik.}{nogmaals hier of men onzen}{man niet had gezien}\\

\haiku{Terwijl ik met den,.}{waard praatte lette ik niet}{op dezen krijger}\\

\haiku{Bij mijn baard, hij is,.}{de eerste die zich op zoo}{iets kan beroemen}\\

\haiku{Hier vond men dus in;}{de verdrukking bij elkaar}{een kostbaren steun}\\

\haiku{En nu naderde.}{het vreeselijke gevecht}{zijn hoogtepunt}\\

\haiku{Pirot greep de vrouw.}{bij de haren en rukte}{heur hoofd achterover}\\

\haiku{De woorden die hij,.}{daarop liet volgen waren}{niet meer te verstaan}\\

\haiku{Zij ademde nog, maar.}{haar wonden gaapten diep en}{breed in de hartstreek}\\

\haiku{{\textquoteleft}Zeg mij, vrouw,{\textquoteright} zoo drong, {\textquoteleft},....}{hij aanspreek snel de Dood heeft}{u reeds in zijn macht}\\

\haiku{Kunnen wij niet het}{hoogste bereiken als wij}{werkelijk met ernst}\\

\haiku{Waarschuw Johanna,.}{dat zij vooral niet alleen}{moet gaan wandelen}\\

\haiku{Wat had zij dit jaar?}{met al die vreugde in de}{natuur te maken}\\

\haiku{Jij, roodharige,,?}{leelijke duivel wat denk}{je wel van je zelf}\\

\haiku{Johanna sloot de,.}{oogen om zijn rood blauw verhit}{gelaat niet te zien}\\

\haiku{De levensdrang kreeg.}{echter nog een oogenblik}{de overhand in haar}\\

\haiku{{\textquoteright} Voorzichtig opende.}{zij de deur en gluurde door}{een reet naar binnen}\\

\haiku{{\textquoteleft}Hier, bij haar doode,,.}{lijf past alleen de waarheid}{de volle waarheid}\\

\haiku{Enfin - daarover valt,.}{niet te twisten je bent jong}{en onbedachtzaam}\\

\haiku{Den volgenden dag.}{moest de Jonker voor goed het}{Kasteel verlaten}\\

\haiku{Nog ademde hij, maar.}{hij had hoogstens nog een klein}{half uur te leven}\\

\haiku{{\textquoteright} Een kwartier later.}{trad de priester met ernstig}{gelaat uit de zaal}\\

\haiku{Zij drukten elkaar.}{de hand en de Jezu{\"\i}et}{ging alleen verder}\\

\haiku{Zijn zonde is groot,,.}{want wie niet v\'o\'or de Kerk is}{is tegen Christus}\\

\haiku{Mijn eer verbood het -!}{mij een ander stond immers}{tusschen jou en mij}\\

\haiku{Wij hebben samen,}{gespeeld gejaagd en gevischt}{en nooit voelde ik}\\

\haiku{Zij greep den oude.}{bij het middel en danste}{met hem in het rond}\\

\haiku{dat is van God, de.}{zuivere Liefde tot al}{wat leeft medebrengt}\\

\haiku{Men heeft, ook nog in,.}{de laatste jaren veel naar}{deze gang gezocht}\\

\haiku{Jan de Bakker of,.}{Johannes Pistorius}{Pastoor van Woerden}\\

\haiku{M.C. Nieuwbarn O.P., {\textquoteleft}Het{\textquoteright};}{Heilig Misoffer en zijn}{ceremoni\"en}\\

\subsection{Uit: Het mysterie van het Veluwehuis (onder ps. J. van Callant)}

\haiku{Bij dien arbeid had,:}{hij ook het geduld geleerd}{dat hij nu toonde}\\

\haiku{Een oogenblik nog.}{en de trein stoof snuivend en}{rammelend binnen}\\

\haiku{{\textquoteleft}Ja, ik houd van de,{\textquoteright}.}{natuur zei Winkelman na}{een korte stilte}\\

\haiku{Hij zag haar eerst niet,.}{maar weldra vond hij haar toch}{in het kreupelhout}\\

\haiku{{\textquoteright} {\textquoteleft}Dus jij bent onze,?}{naaste buurman behalve}{de heer Verhoeven}\\

\haiku{De huisknecht maakte.}{een buiging en uitte een}{soortgelijken groet}\\

\haiku{{\textquoteright} vroeg hij, zich tot de.}{gastvrouw wendende met een}{lachje van twijfel}\\

\haiku{Maar dien neef van hen,!}{met dien moet ik eens kennis}{maken als hij komt}\\

\haiku{Toen herkende hij,:}{zijn vrouw streek met de hand over}{het voorhoofd en vroeg}\\

\haiku{Trap op, trap af scheen,.}{het te schuifelen maar geen}{der treden kraakte}\\

\haiku{Er verliepen drie,.}{dagen zonder dat er iets}{bijzonders voorviel}\\

\haiku{Op den middag na.}{zijn komst ging de dokter hem}{een bezoek brengen}\\

\haiku{De wetenschap is.}{op zielkundig gebied nog}{weinig gevorderd}\\

\haiku{{\textquoteright} Winkelman uitte.}{een gesmoorden kreet en trok}{zijn handen terug}\\

\haiku{Plotseling bleef zij -!}{staan en luisterde het kon}{geen vergissing zijn}\\

\haiku{Maar zijn overspannen,.}{toestand eischte dat er}{verandering kwam}\\

\haiku{{\textquoteleft}Dat hij zijn vuist naar,!}{de beide huizen ophief}{geeft wel te denken}\\

\haiku{Uit iemands boeken,}{kan men soms belangrijke}{conclusies trekken}\\

\haiku{{\textquoteright} Hij zweeg even en keek,:}{daarop den dokter weer scherp}{aan terwijl hij vroeg}\\

\haiku{Ik zou die dan graag, ',.}{alst niet onbescheiden}{is eens willen zien}\\

\haiku{{\textquoteleft}Het is gek, dat men!}{aan zulke kleine dingen}{zooveel waarde hecht}\\

\haiku{Herinner je, dat.}{de dokter geheimhouding}{van den verkoop vroeg}\\

\haiku{{\textquoteright} {\textquoteleft}Maar de naam dan in,;}{de passagierslijst in de}{hotelboeken enz.}\\

\haiku{{\textquoteright} vroeg Robbers toen ik, {\textquoteleft}?}{gereed was met de lectuur}{wat denk je daarvan}\\

\haiku{Ik wil straks alles -.}{nog eens beter opnemen}{ook dat geraamte}\\

\haiku{In de andere:}{kast naast het raam vonden we}{evenmin iets verdachts}\\

\haiku{{\textquoteright} Robbers wilde nog,}{een vraag stellen maar op dat}{oogenblik hoorden}\\

\haiku{het gebeurde scheen.}{een diepen indruk op hem}{te hebben gemaakt}\\

\haiku{Misschien zouden wij{\textquoteright},.}{dan samen kunnen gaan zei}{de heer Tellegen}\\

\haiku{Je zult je vriend Thom?}{toch zeker nog wel een paar}{jaar levens gunnen}\\

\haiku{Ik had mij in het.}{zand uitgestrekt en Robbers}{ging naast mij zitten}\\

\haiku{op een avond ziet hij,.}{iemand en ontstelt z\'o\'o dat}{hij een flauwte krijgt}\\

\haiku{wat men sterk wenscht,.}{en verwacht gaat op den duur}{vaak in vervulling}\\

\haiku{Ik vroeg hem of hij.}{de gillende geluiden}{wel eens had gehoord}\\

\haiku{Zal jongen sturen.}{om je den weg te wijzen}{naar mijn verblijfplaats}\\

\haiku{{\textquoteright} {\textquoteleft}Nu zullen we eens.}{zien wat de heer Tellegen}{ons kan vertellen}\\

\haiku{Hij heeft zich al een.}{paar maal bewogen en schijnt}{weer bij te komen}\\

\haiku{In de Indische.}{wouden liet hij den dokter}{de diamanten zien}\\

\haiku{De dokter stond met.}{een bleek en verschrikt gelaat}{bij zijn schrijfbureau}\\

\haiku{De dokter en zijn.}{vrouw hadden krankzinnig van}{angst kunnen worden}\\

\section{Emmanuel de Bom}

\subsection{Uit: Heldere gezichten}

\haiku{Hij kwam uit een storm.}{die zijn jong gemoedsleven}{hevig had doorschokt}\\

\haiku{Hij had zijn viool.}{weer uit de kast gehaald en}{had ze weer gestemd}\\

\haiku{hij voelde geen drang,...}{naar haar broze wezen die}{ziel zonder lichaam}\\

\haiku{het toekomende -;}{maand ging zijn maar het spookte}{tusschen de regels}\\

\haiku{onder 't bed school;}{een man met een blinkend mes}{tusschen zijn tanden}\\

\haiku{hij hoorde iemand,.}{de trap van het huis afgaan}{de vrouw die hoestte}\\

\haiku{hij onderbrak b.v....}{een heel ernstig gesprek met}{een opmerking over}\\

\haiku{Zij waren omtrent;}{hetzelfde tijdstip naar de}{haven gekomen}\\

\haiku{Bij hun verschijnen.}{op de Beurs werden zij met}{ontzag aangestaard}\\

\haiku{De oogen zijn grooter, '.}{geworden en schijnen scheef}{int hoofd te staan}\\

\haiku{En de meester schijnt...{\textquoteright} - {\textquoteleft},{\textquoteright},.}{er op te wakenJa vriend}{Lodewijk zei hij}\\

\haiku{De piano hing '...}{schrijlings doort raam van de}{eerste verdieping}\\

\haiku{Rechts flankeerde haar,;}{een slanke als een degen}{zoo fiksche page}\\

\haiku{En deze laatste {\textquoteleft} '{\textquoteright};}{man wasde zoekende naar}{t onbekende}\\

\haiku{En dan moet gij, in '}{gezelschap van Vlaanderens}{grootsten schrijver als}\\

\haiku{Zij kwamen, wijl de,.}{boot een half uur te laat}{kwam even lang te vroeg}\\

\haiku{In den aanvang der;}{15e eeuw begon de zee zich}{terug te trekken}\\

\haiku{gezamenlijk uit;}{te maken een fleurigen}{bond der jongeren}\\

\haiku{Maar ik wou, dat gij.}{dien dag Dr. Jef Spijkers aan}{de taak had gezien}\\

\haiku{En wij kijken toe,...}{en voelen ons tot zwijgen}{en peinzen genoopt}\\

\haiku{{\textquoteright} Maar... aan den hemel.}{was plotseling een vreemde}{klaarte gerezen}\\

\haiku{'t jaar traant weg... 't,..}{Is een week einde een naar}{en zielig einde}\\

\subsection{Uit: Wrakken}

\haiku{is al dadelijk.}{wilskrachtige gespitstheid}{van het intellect}\\

\haiku{Al ontginnen zij,.}{zich niet zij beseffen iets}{van het eigen zijn}\\

\haiku{Maar de mensch is nog{\textquoteright}.}{niet sterk genoeg om zonder}{ideaal te leven}\\

\haiku{onze geest moet  .}{de geest van onze tijd zijn}{en niet omgekeerd}\\

\haiku{Wij kunnen onze.}{handen niet adelen aan den}{zegenenden arbeid}\\

\haiku{Zijn gezicht was plat,;}{en schier loodkleurig het had}{iets van een Mongool}\\

\haiku{Hij wou iets anders,,...}{iets dat hij nog niet gekend}{had nergens ontmoet}\\

\haiku{Het verblijdde hem,.}{alsof zij nu dichter tot}{hem genaderd was}\\

\haiku{Het lag daar vermoeid,,.}{in zijn nachtjaponnetje}{als een mo\^e bloempje}\\

\haiku{- ik verleid u, ik,...}{slorp u op ben de schuld dat}{gij verloren gaat}\\

\haiku{buiten u is de -!...}{wereld mij le\^eg zonder u}{kan ik niet ademen}\\

\haiku{en altijd zoo blij, -!}{zoo goed gezind o ge zijt}{twee deugenietjes}\\

\haiku{Zij gingen door de.}{luidruchtige straat in het}{schoone avondweder}\\

\haiku{Richard dwaalde nog.}{eenigen tijd in mijmering}{langs de straten}\\

\haiku{Zij verbleekte en '.}{t was of haar harteklop}{plotseling stil hield}\\

\haiku{Zij herinnerde.}{zich hun afspraak en het was}{een nieuwe kwelling}\\

\haiku{Op dit oogenblik;}{voelde zij een oneindig}{misprijzen voor hem}\\

\haiku{Een weemoed zonk in.}{haar bij het herdenken van}{heel dat verleden}\\

\haiku{{\textquoteright} {\textquoteleft}Ja...{\textquoteright} en hij tastte.}{in zijn zak en haalde een}{doosje te voorschijn}\\

\haiku{Hij schrikte als hij.}{zijn doodsche waskleurige}{trekken gadesloeg}\\

\haiku{En opeens dacht hij,...}{dat nu een andere in}{zijn kooi zou liggen}\\

\haiku{En nu mocht alles - '...}{opnieuw beginnen zooalst}{altijd geweest was}\\

\haiku{maar met een zucht joeg.}{zij de vernevelende}{schim ver weg van haar}\\

\haiku{{\textquoteright} - Zij vroeg hem niet eens.}{waar die verschrikkelijke}{som voor dienen moest}\\

\haiku{Zij gaf voor dat zij,.}{naar huis moest maar hij wilde}{haar niet loslaten}\\

\haiku{Hier sta ik op het {\textquotedblleft}{\textquotedblright}{\textquoteright},.}{punt eengemengd bericht te}{plegen spotte hij}\\

\haiku{De schepen vaarden.}{in blij gewemel door het}{lavende briesje}\\

\haiku{Daar stond een man aan,:}{den achtersteven die met}{een sjerpje waaide}\\

\section{I.K. Bonset}

\subsection{Uit: Het andere gezicht van I.K. Bonset}

\haiku{De engelen - ja,.}{de engelen worden aan}{het spit gebraden}\\

\haiku{{\textquoteleft}H\'e, jelui daar, kun.}{je mij geen sigaret naar}{beneden gooien}\\

\haiku{Een belangwekkend.}{stilzwijgen nam al onze}{zinnen in beslag}\\

\haiku{Elke functie die,.}{op de fysica berust}{mechaniseren}\\

\haiku{Wij nieuwe mensen.}{willen met de kunstenaars}{een strijd beginnen}\\

\haiku{Massatoestand Het,.}{is niet prettig muis te zijn}{tussen twee katers}\\

\haiku{Het nieuwe inzicht ().}{het overzinnelijk gezicht}{heft dit contrast op}\\

\haiku{Der Dadaist{\textquoteright} - aldus - {\textquoteleft};}{Raoul Hausmannerleidet}{nicht die Welt kindlich}\\

\haiku{Elk eindje touw door.}{een knoop verbonden aan een}{ander eindje touw}\\

\haiku{23Dichterlijke.}{uitdrukking als resum\'e}{van Evola's werkje}\\

\haiku{27Bloeien uit de:}{tevredenheid met onze}{eigen gebreken}\\

\section{Joachim Bontius de Waal}

\subsection{Uit: Oorspronck en opkomst der stede Alckmaar, beginnende anno DL uyt een seer oud manuscript berustende ter Liberije deser Stadt gecopieert ende vervolgt tot MDCCLX}

\haiku{Alckmaer zonder}{die kercke altemael}{ende die lieten}\\

\haiku{Den 24 april is de.}{eersten steen geleydt aen de}{oude Vrieschepoort}\\

\haiku{men soude wat in,.}{de kerck doen hetwelck}{sij soo gedaen heeft}\\

\haiku{Is de Langenstraet}{deeser stadt Alckmaer uyt}{eenderhandt verhoogt.1587In}\\

\haiku{huys opgeregt,}{buyten de Kennemerpoort}{staet op die plaets daer}\\

\haiku{Dit jaer is de craen '.}{gesteldt op het endt vant}{ooster Fnidsen}\\

\haiku{10 augusti is.}{den eersten steen geleydt aan}{de Waagtooren}\\

\haiku{nae water, maer daer}{was niet.1651Is de heerlijckheydt}{van Schagen verkogt}\\

\haiku{Men is niet bewust}{of dat ongeluck door het}{swaeren donderweer}\\

\haiku{van hier trocken sij,.}{voorts door Noordt-Hollandt daer sij}{de beelden stormden}\\

\haiku{De schaeden was in,.}{Engenlandt Vranckrijck}{en Schotlandt seer groot}\\

\section{Henri van Booven}

\subsection{Uit: Kinderleven}

\haiku{Drie poppen, alleen.}{in de stille schaduw van}{de middagkamer}\\

\haiku{V\'o\'or alles moest zijn,.}{zwarte glanzende gaafheid}{ongerept blijven}\\

\haiku{Dat waren dan tw\'e\'e,,....:}{likken \'e\'en lange streek dus}{anapaesten}\\

\haiku{{\textquoteright} In de kasten, op.}{de bovenste planken rest}{nog een en ander}\\

\haiku{Zij heeft zeker wel}{eens in de kast gekeken}{en niet geweten}\\

\haiku{Wat heb ik dan toch,}{gedaan dat die wanden mij}{hier nog altijd zoo}\\

\haiku{{\textquoteright} Een ander maal, des,.}{morgens voor de lessen werd}{er heel lang getold}\\

\haiku{Want, zonder dat ik,,.}{het hoorde plotseling daar}{tripte zij binnen}\\

\haiku{Hier moet het verjaagd,;}{met ontstoken lampen met}{ons levend geluid}\\

\haiku{Er is niets dan dat....}{in de beslotenheid van}{deze duisternis}\\

\haiku{Het zal mij nu ook,....}{nimmer meer vreemd zijn noch kan}{het teloor gaan}\\

\subsection{Uit: De scheiding}

\haiku{{\textquoteright} En hij hamerde.}{met zijn bierglas op den rand}{van een stoel voor zich}\\

\haiku{Vooruit maar Rieke,,!}{m'n jonkske in het veen ziet}{men op geen kluitje}\\

\haiku{wij verschillen van.}{de Belgen voornamelijk}{in ras en godsdienst}\\

\haiku{Maar ik zeg je dan,,{\textquoteright}.}{mijn waarde dat wij d\'a\'ar nog}{lang niet aan toe zijn}\\

\haiku{Hij had afkeer van.}{den militair in Vincent}{op dat oogenblik}\\

\haiku{Weldra zou het spel,.}{in den tuin gedaan zijn en}{moesten zij naar binnen}\\

\haiku{Hij zou, naar hij schreef,.}{waarschijnlijk niet meer naar de}{tropen teruggaan}\\

\haiku{Miel was ziek, hij leed,.}{aan ingewandsziekte en}{had dikwijls koortsen}\\

\haiku{Maar een smidsjongen,:}{en een landman met een zeis}{drongen op zeggend}\\

\haiku{{\textquoteleft}Neen, dan zou ik u,{\textquoteright}.}{gewaarschuwd hebben ik heb}{geen Duitscher gezien}\\

\haiku{Maar Marius wist,.}{zich te beheerschen en hij}{hoorde nu zijn stem}\\

\haiku{De stationschef:}{meende dat het wel van een}{ontploffing kon zijn}\\

\haiku{Zij stonden voor een,}{spoorbrug en de weinige}{reizigers staken}\\

\haiku{En met gehaaste,.}{nerveuse stappen verliet}{hij de kamer}\\

\haiku{Was zij dan in die?}{enkele jaren z\'o\'oveel}{ouder geworden}\\

\haiku{{\textquoteleft}Neen, treurig genoeg,.}{want zij trekt zich dat alles}{verschrikkelijk aan}\\

\haiku{En hij stelde zich.}{voor wat hem te doen zou staan}{als dat gebeurde}\\

\haiku{Hij deed alles wat.}{hij kon om het gezin bij}{te staan in zijn nood}\\

\haiku{Het was hun alsof.}{elk woord dat zij spraken hun}{noodlottig kon zijn}\\

\haiku{Vanaf den zolder,.}{konden zij nu zien hoe het}{brandde in de stad}\\

\haiku{hij stond de eenige.}{overgebleven burgers der}{stad Leuven waren}\\

\haiku{Er was geen klacht uit.}{de zeventig mannen van}{Leuven opgegaan}\\

\haiku{Hij vermeed het ook,.}{hem nu dadelijk over zijn}{gezin te spreken}\\

\haiku{Zij hield van Vincent,?}{maar was Marius niet van}{heel anderen aard}\\

\haiku{Wat was zij dien avond.}{van de eene stemming in de}{andere gejaagd}\\

\haiku{Het dametje stond.}{nu op uit haar stoel en kwam}{nader bij Louise}\\

\haiku{zoudt ge er tegen,?}{op zien om op zulk een kar}{een poos te rijden}\\

\haiku{niet ver van de plek,.}{waar de brug vernield lag zou}{hij hen wachten}\\

\haiku{hij kon zonder door.}{te groote vaart in de steden}{argwaan te wekken}\\

\haiku{Hoe geheel anders.}{zagen zij er uit dan de}{infanteristen}\\

\haiku{Het was waar de rails.}{bij een splitsing van den weg}{naar rechts afbogen}\\

\haiku{{\textquoteright} {\textquoteleft}Dat hebt ge gezien,{\textquoteright}.}{hier staat het stempel van de}{Duitsche Legatie}\\

\haiku{Voelde hij nu meer?}{deernis met haar dan liefde}{en genegenheid}\\

\haiku{Maar opnieuw als een:}{plotselinge pijn sloop de}{twijfel in zijn hart}\\

\haiku{Zeker, zoo waren,.}{zij die Belgische vrouwen}{zij vergaten gauw}\\

\haiku{Dat was altijd de,.}{stille rustige trots van}{haar geslacht geweest}\\

\haiku{Dat was een heete,.}{dag een Zaterdag laat in}{Augustus geweest}\\

\haiku{maar ziende, dat zij,:}{zwaar ademhaalde kwam hij bij}{haar staan en zeide}\\

\haiku{Evenwel, ondanks dien;}{voorkeur had zij ontzag voor}{een helder verstand}\\

\haiku{Aanvankelijk kon;}{hij geen berusting vinden}{bij die overtuiging}\\

\haiku{Het gebeurde een.}{Zaterdagmiddag in het}{begin van Juli}\\

\subsection{Uit: Tropenwee}

\haiku{Een negermeisje,.}{stond daar een kind van een jaar}{of vijftien leek het}\\

\haiku{Hij nam zijn vouwstoel.}{uit den corridor en klom}{er mee naar boven}\\

\haiku{Haastig stappend en '.}{armen zwaaiend waadden die}{buitent bassin}\\

\haiku{De inspecteur zat.}{midden in deze kamer}{voor zijn schrijftafel}\\

\haiku{Met geduchte en:}{vijandige zekerheid}{voelde hij het toen}\\

\haiku{Vandaag was 't niet, '.}{zoo warmt zal binnen een}{uur wel ophouden}\\

\haiku{Het geluid van de.}{stoomfluit begon te grommen}{langaange-houden}\\

\haiku{bij de kazerne.}{stonden wat struiken en een}{enkele hooge boom}\\

\haiku{Later zette hij.}{zich bij den dooden boom op een}{uitstekenden steen}\\

\haiku{Een ijzeren pier.}{stak vlak bij die loodsen het}{gele water in}\\

\haiku{In de schemering.}{keerde de witte naar de}{factorij terug}\\

\haiku{De witte en zijn.}{reisgenoot bleven alleen}{op de verandah}\\

\haiku{Ronk had zijn witten.}{hoed reeds opgezet en scheen}{klaar om uit te gaan}\\

\haiku{{\textquoteleft}Dat zou ik bijna,?}{vergeten mag ik het er}{maar van afnemen}\\

\haiku{wat meer naar voren.}{lag het stationnetje}{met schaarsche lichten}\\

\haiku{De witte gaf geen,,:}{antwoord eerst zeide toen om}{toch wat te zeggen}\\

\haiku{Nu zijn we er gauw,{\textquoteright}.}{zeide een van de mannen}{die naast hem liepen}\\

\haiku{{\textquoteright} {\textquoteleft}Ja, daar lagen we,.}{als beesten op den planken}{vloer geen eens bedden}\\

\haiku{De man die vlak voor,;}{het licht zat had een roode}{vlek op het voorhoofd}\\

\haiku{Hij luisterde, nat.}{van zweet en trillend van de}{overgroote inspanning}\\

\haiku{Enkele uren voor.}{zonsondergang was alles}{voor de tocht gereed}\\

\haiku{Enkele uren, kort,.}{voor zonsondergang was hij}{van de jacht terug}\\

\haiku{Nous avons d\'ecouvert!}{encore trois bo{\^\i}tes de}{jambon conserv\'e}\\

\haiku{Toen dat gelukt was.}{zag hij de moustiquaire}{van binnen goed na}\\

\haiku{hij snakkend naar lucht......}{neer op de vuile lakens}{en wilde slapen}\\

\haiku{Hij vraagde een en,.}{ander van het land hoe ver}{hij wel gereisd had}\\

\haiku{Comme vous \^etes p\^ale,{\textquoteright}.}{zeide Fourneau toen hij den}{witte ontwaarde}\\

\haiku{Hij beproefde de.}{geluiden van die dieren}{te onderscheiden}\\

\haiku{En hij begon den,.}{dood die nu wel heel dicht bij}{moest zijn te schuwen}\\

\haiku{As je wat noodig heb,.}{seg-ut dan an die fent}{die we hier late}\\

\haiku{Hij at alles wat.}{de negerjongen hem bracht}{en vraagde om meer}\\

\haiku{Een schriklijke dorst,.}{folterde hem toch mocht hij}{maar weinig drinken}\\

\haiku{Een neger begon,,.}{stil in den gloed kijkend een}{donker  gezang}\\

\haiku{Zijne moeder, heel.}{zelden maar had zij zich hem}{welgezind getoond}\\

\haiku{Heel moeilijk liep de.}{witte naar het station}{en ging er binnen}\\

\haiku{De zwarte haalde.}{de koffers en bracht ze in}{het getimmerte}\\

\haiku{Een schandaal van je.}{huis om je hier als een beest}{te laten liggen}\\

\haiku{{\textquoteright} De witte begreep,.}{dat dit de nieuwe chef in}{Mataddi moest zijn}\\

\haiku{Als je mijn vraagt wat ', {\textquoteleft}.}{t beste is uitviere}{en zweete enEno's}\\

\haiku{De lange, zware,.}{droeg hem voort hoog boven de}{hoofden der menschen}\\

\haiku{Maar buiten de stad,.}{bij het bosch daar dansten de}{negers  Tam-Tam}\\

\haiku{Ik had neus en keel.}{vol geronnen bloed dat mij}{bijna stikken deed}\\

\haiku{Het stond op een klein,,....}{eiland aan de Vecht was het}{geweest meende hij}\\

\haiku{De heuvels duwden....}{het bange echo-gerucht}{aarzelend weerom}\\

\haiku{In den winkel vlak.}{bij het Hollandsche huis ging}{de witte binnen}\\

\haiku{{\textquoteleft}Weet u dan niet, dat?}{die al een paar weken dood}{en begraven is}\\

\haiku{Hij was gaan zitten,.}{en keek uit over de zee het}{boek in de handen}\\

\haiku{Een slag op de gong.}{zwaaide een luid gedreun sleepend}{heen over de dingen}\\

\haiku{Onder aan de trap.}{van de pier schommelde het}{stoom-barkasje}\\

\haiku{{\textquoteright} Zij gingen naar de,.}{trap toe links daarnaast was een}{andere doorgang}\\

\haiku{Ook over het dek van.}{de tweede klasse was een}{wit zeil gespannen}\\

\haiku{Druk pratend stapten.}{de passagiers in groepjes}{naar de eetsalon}\\

\haiku{Als het twaalf uur sloeg,,.}{dan wist iedereen zou dat}{mogen gebeuren}\\

\haiku{Wat was dit vreemd en,?}{ongewoon zou hij dit wel}{ooit meer beleven}\\

\haiku{Voor hij naar zijn hut.}{ging wandelde Jules nog}{eenmaal naar achter}\\

\haiku{Half wakker dacht hij.}{er aan dat het binnen een}{paar uren dag zou zijn}\\

\haiku{Het steeg, het geweld,.}{en builde uit en zwol en}{gierde als een storm}\\

\haiku{Druppels zweet voelde,.}{hij afglijden van zijn lijf}{zijn gezicht was nat}\\

\haiku{Hij had weer lang stil,,.}{gezeten wachtend beidend}{den langzamen tijd}\\

\section{F. Bordewijk}

\subsection{Uit: Bloesemtak}

\haiku{Maar wel heb ik iets.}{dergelijks ontdekt in de}{Rotterdammersloot}\\

\haiku{Hij vertoonde het,.}{nooit lang maar toch steeds weer als}{het zijn vak betrof}\\

\haiku{De kleuren heb ik,.}{al in mijn hoofd eindigde}{hij met een glimlach}\\

\haiku{Het was half zes, en.}{het verkeer over de straatweg}{reeds iets verminderd}\\

\haiku{- Zei Termunten daar? -.}{nog wat van Hij neemt dat ook}{voor zijn rekening}\\

\haiku{Op die manier kwam.}{haar vriendschap met Leo neer op}{het hoofd van Anton}\\

\haiku{Leo had bijna een,.}{verering voor de vriendin}{al verborg ze dit}\\

\haiku{dacht hij, zittend naast,.}{zijn chauffeur nog even terug}{aan de gevelsteen}\\

\haiku{Hij herinnerde,:}{haar daaraan en vervolgde}{tot haar speciaal}\\

\haiku{ik bedoel, u zult.}{de toestand niet helemaal}{in uw macht hebben}\\

\haiku{Het is blijkbaar al,.}{weer zover en dat begint}{me te vervelen}\\

\haiku{Ik heb het tot nog,.}{toe geslikt maar ik wil nu}{weten wat er is}\\

\haiku{Je zegt maar dat het,.}{een vergissing was dat je}{vader het niet wil}\\

\haiku{Ik moet het bij haar.}{weghalen om het aan jou}{te kunnen geven}\\

\haiku{Jaja, we lossen,.}{het nu wel heel aardig op}{hier onder elkaar}\\

\haiku{Dan zijn die nauwe.}{zijstraten openingen in}{de rijen boeken}\\

\haiku{Evenwel sprak ze over.}{zijn werk nooit anders dan in}{enkele woorden}\\

\haiku{Ze begreep te zijn.}{binnengedragen in een}{tapijtenwinkel}\\

\haiku{Even later hoorde.}{ze de onmiskenbare}{klik van haar deurslot}\\

\haiku{Hij verscheen nog op;}{de dag v\'o\'or zijn dood in het}{huis op de Hooigracht}\\

\haiku{Volgens de dokter.}{zou dit verschijnsel nog wel}{een poos aanhouden}\\

\haiku{Van der Gronden reed,,;}{voorzichtig niet alleen in}{de stad ook buiten}\\

\haiku{Het luik in de kap.}{bleef wegens het branden van}{de zon gesloten}\\

\haiku{Het was buiten kil.}{en allen zaten in de}{conversatiezaal}\\

\haiku{Verder ging hij niet,.}{maar er opende zich voor haar}{een nieuw gezichtspunt}\\

\haiku{Ditmaal werden er.}{in het vestibuletje}{kussen gewisseld}\\

\haiku{- Weet je wel dat ze?}{de laatste tijd in Den Haag}{erg over je kletsen}\\

\haiku{Hij rookte niet meer,;}{maar hield het pijpje in zijn}{tandeloze mond}\\

\haiku{Toch jammer dat je.}{die beoordelingen niet}{kunt laten drukken}\\

\haiku{Dat doe ik ook niet,.}{maar ik kan toch niet alles}{laten passeren}\\

\haiku{En nog stom op de,.}{koop toe want zo propageer}{je de middelmaat}\\

\haiku{E\'en visite aan.}{dit verwarde brein was hun}{meer dan voldoende}\\

\haiku{Zeg, even groeten, daar,.}{komt meneer Nathans onderbrak}{Aurora zichzelf}\\

\haiku{- Ik mag die meneer,,.}{Nathans niet erg zei ze gedempt}{terwijl Max verdween}\\

\haiku{Aurora kwam in,,.}{de ochtend niet ver van haar}{woning Max tegen}\\

\haiku{Hij viel haar door een.}{ongewoon gelige tint}{onmiddellijk op}\\

\haiku{Deze nietige.}{vreugde ging teloor onder}{het verder klimmen}\\

\haiku{Ze kwam tot zichzelf.}{doordat ze tranen over haar}{wangen voelde}\\

\haiku{Zelfs bracht de laatste.}{conclusie een zweem van een}{glimlach om haar mond}\\

\haiku{Hij was ook altijd.}{zo zichtbaar het hoofd van zijn}{kinderloos gezin}\\

\haiku{Aurora vond het.}{beter hierover niet met Van}{Marle te spreken}\\

\haiku{Bedekt met leien.}{ging in het dak voor hem het}{symbool verloren}\\

\haiku{Het was een ochtend.}{van beginnende winter}{in allerfijnst grijs}\\

\haiku{Hij wist dat zulke.}{abnormaliteiten niet}{in haar smaak vielen}\\

\haiku{vroeg de vriendin na.}{een ogenblik en steeds in de}{grootste verwarring}\\

\haiku{Blijkbaar toen het ene.}{niet opging omdat het al}{te doorzichtig was}\\

\haiku{Hij durfde haar niet,.}{meer in de ogen te zien hij}{zou zich doodschamen}\\

\haiku{De groene tafel}{Nu Aurora meende door}{de volkomen breuk}\\

\haiku{En dat ik haar die,?}{brief teruggaf vond je toch}{ook wel goed nietwaar}\\

\haiku{Het stond in haar brief,;}{en het vereenvoudigde}{de toenadering}\\

\haiku{Een antisemiet?}{zal toch niet bij voorkeur bij}{een Jood intrekken}\\

\haiku{En ik zal je nu.}{ook precies zeggen hoe de}{vork in de steel zit}\\

\haiku{- Jacob Effra{\"\i}m,.}{vanaf dit ogenblik kennen}{wij elkaar niet meer}\\

\haiku{Waarom toch kon de?}{mens op dit punt zijn leven}{niet terugdraaien}\\

\haiku{- O neen, antwoordde,.}{Aurora haastig die mag}{u niet gebruiken}\\

\haiku{Hoop jij maar dat ze,,.}{het winnen allebei en}{dat ik het verlies}\\

\haiku{Welnu, dat deed hij,,.}{op dit moment het kon nog}{niet te laat wezen}\\

\haiku{Aurora zat met.}{de rug naar de deur aan een}{damesbureautje}\\

\haiku{Er hing een sfeer van.}{haat in het huis en deze}{sfeer werd steeds dichter}\\

\haiku{Hij keek haar aldoor,.}{strak aan met kleine enigszins}{roodbelopen ogen}\\

\haiku{- Ik heb iemand een,,.}{slag gegeven gisteren}{met een bureaulamp}\\

\haiku{Ik weet het van een.}{vriendin die eens met hem te}{maken heeft gehad}\\

\haiku{Het was gegaan bij,.}{vlagen overeenkomstig haar}{grillig karakter}\\

\haiku{Zeg Nathans, voordat je,?}{begint mag ik even juffrouw}{Monterey spreken}\\

\haiku{En u betaalt  ,.}{me vijf procent rente over}{het jaar gerekend}\\

\haiku{hij leek haar zo echt.}{een stijgende geul tussen}{muren ingeklemd}\\

\haiku{- Mijn zoon heeft met die,.}{vrouw volkomen gebroken}{zei Termunten koel}\\

\haiku{Toen kwam Fronto op:}{zijn beurt naar voren en zei}{met de grootste klem}\\

\haiku{U vertelde me.}{dat de jonge Termunten}{zo is gepousseerd}\\

\haiku{Aurora wist er;}{niets van dat haar geval toch}{nog een nasleep had}\\

\haiku{En ondanks deze.}{redeneringen bleef de}{beklemming hem bij}\\

\haiku{Hij had gevreesd voor,,.}{een onbeslapen bed zij}{elders in huis ziek}\\

\haiku{Dit verontrustte.}{Iris veel meer dan de toestand}{van de kinderen}\\

\haiku{Met hem sprak ze wel,.}{eens over zijn moeder maar hij}{gaf weinig antwoord}\\

\haiku{Deze vrouw was een.}{toonbeeld van plichtsbesef in}{zakelijke vorm}\\

\haiku{Het kon hem tot zulk.}{een foltering worden dat}{hij de straat opliep}\\

\haiku{Hij besefte zeer.}{goed wat de kinderen aan}{hem te kort kwamen}\\

\haiku{Van Marle wilde,.}{weerstreven maar de vriend trok}{hem omhoog en mee}\\

\haiku{Wie zich verplaatst naar.}{de vierde dimensie is}{ineens onzichtbaar}\\

\haiku{Maar er volgt n\'og iets,.}{en daarmee kom ik tot het}{bekende grapje}\\

\haiku{Maar daarboven, in,...}{haar bureautje lagen toch}{de drie portretten}\\

\subsection{Uit: De doopvont}

\haiku{hij hoefde haar zelfs,.}{niet te bedwingen en ook}{dat was hem bekend}\\

\haiku{Eens had hij tegen,,:}{zijn oudste halfzuster Lea}{Bearda gezegd}\\

\haiku{het was Bearda,.}{en De Bleeck en het bleef dit}{door alle jaren}\\

\haiku{De Bleeck liep langs het.}{water met zijn normale}{snelle wandelpas}\\

\haiku{De Bleeck ging uit in,.}{zijn kleine bruine wagen}{men wist niet waarheen}\\

\haiku{- De wet hanteren.}{is het tegendeel van de}{wet ondervinden}\\

\haiku{Maar tegenwoordig, -:}{het Parlement bezit geen}{mensenkenners meer}\\

\haiku{Bij een vrouw zijn de;}{zenuwtoppen doorgegroeid}{tot in haar kleren}\\

\haiku{Dat moet u hebben,.}{gezien want vrouwen kijken}{alleen naar elkaar}\\

\haiku{Het is oud nieuws, maar.}{het schijnt altijd weer nuttig}{het te herhalen}\\

\haiku{hij kon er evenwel,.}{niets aan veranderen en}{het ging hem niet aan}\\

\haiku{Gebruik toch liever,.}{die hersenkwabben voor wat}{anders Brandenburg}\\

\haiku{De organen van.}{ons perceptievermogen}{zijn altijd te laat}\\

\haiku{wat heeft de wereld?}{er aan de Duitsers nog eens}{op stang te jagen}\\

\haiku{C'est ici que tombent -.}{en ruine les merveilles}{de la cuisine}\\

\haiku{En overigens was.}{hij ook liefst afgezonderd}{met zijn gedachten}\\

\haiku{Bij weten van De.}{Bleeck was Bearda nog nooit}{op de club geweest}\\

\haiku{En trouwens, was het,,?}{een wetenschap en zo ja}{een volledige}\\

\haiku{Je ziet het, ik heb.}{me daarginds ook al voor je}{ge{\"\i}nteresseerd}\\

\haiku{- Je weet het altijd.}{zo voor te stellen dat het}{een compliment lijkt}\\

\haiku{hij kon niet enkel;}{in frak met de winter naar}{huis zijn gereden}\\

\haiku{zeker niet zichtbaar,.}{en zijn hart bleef rustig en}{evenwichtig kloppen}\\

\haiku{En de enkele.}{inwoning was mogelijk}{slechts een eerste stap}\\

\haiku{Zijn op dit punt slecht.}{geweten deed hem zich van}{alles inbeelden}\\

\haiku{ze voelde zich ook,.}{vermoeid want ze had te snel}{teruggelopen}\\

\haiku{Ze bezat, al zei,.}{ze het ronduit van zichzelf}{een goede speurneus}\\

\haiku{Ze had hem verleid,.}{als vrouw naar de zaken en}{als vrouw naar de sexe}\\

\haiku{Hij werd gewapend,.}{met een revolver en vond}{het een hele eer}\\

\haiku{De sfeer, de stijl van.}{haar ouderlijk huis had haar}{te zeer doortrokken}\\

\haiku{De middag druilde.}{in dit vertrek voort met een}{saaiheid zonder eind}\\

\haiku{Hij had dat altijd,.}{gedaan en Sara had nooit}{aanmerking gemaakt}\\

\haiku{Lea was een goede.}{huisvrouw in die zin dat ze}{perfect leiding gaf}\\

\haiku{- De laatste jaren,.}{speel ik alleen nog maar C\'esar}{Franck zei Aleida}\\

\haiku{Dat is op zichzelf,.}{ook al weer een meesterwerk}{die transpositie}\\

\haiku{Sara kwam door haar,,.}{spel tot de componist voor}{hoe kort ook maar toch}\\

\haiku{- Ja, maar ik kan er,.}{toch niet van loskomen van}{zoiets onooglijks}\\

\haiku{daardoor toch verloor.}{haar tweede leven iets van}{zijn verschrikking}\\

\haiku{- Dat is dan een streek.}{zoals gepast is onder}{boezemvriendinnen}\\

\haiku{Eigenlijk had hij.}{respect voor het geheel van}{diens directoraat}\\

\haiku{dat hij zijn fabriek,,.}{hem dierbaarder nog dan zijn}{huis niet vinden kon}\\

\haiku{Het was een stil en,.}{luchtig grapje geheel te}{eigen genoegen}\\

\haiku{Maar dit hier is voor.}{mij typisch de fin  de}{si\`ecle-sfeer}\\

\haiku{Adeldom geeft ook nu,.}{nog iets al kan je het niet}{defini\"eren}\\

\haiku{Daar ze er niets aan,.}{toevoegde moest het hier dus}{haar slaapkamer zijn}\\

\haiku{De zware rode.}{gordijnen naar de serre}{waren gesloten}\\

\haiku{kon hij daarvoor een?}{beter bewijs verlangen}{dan wat zij nu gaf}\\

\haiku{Toen zag hij zich hier,,.}{in zijn stoel zitten achter}{een barri\`ere}\\

\haiku{Dit veranderde:}{echter toen hij zijn eerste}{grote werk volbracht}\\

\haiku{Met dat al was X.}{bij alle geslotenheid}{geen zwijgzaam persoon}\\

\haiku{De ochtend van die.}{dag vielen er nog verspreid}{enkele buien}\\

\haiku{het bleef droog en was,.}{opmerkelijk zoel des avonds}{windstil bovendien}\\

\haiku{Van het gezicht wist.}{hij niets dan dat het een grauw}{vrouwengezicht was}\\

\haiku{Tot nog toe had ze,.}{hem handig ontweken thans}{zat ze in de val}\\

\haiku{- Ja, vervolgde hij,,.}{wijzend op haar omgeving}{u bent als Algol}\\

\haiku{Maar wilt u op een?}{andere manier onze}{historie kennen}\\

\haiku{Mevrouw Van Fransen,.}{kwam voorbij aan de arm van}{de legatieraad}\\

\haiku{Hoe kleiner, des te.}{eer de gevolgen van de}{oorlog te boven}\\

\haiku{Foei, foei, meneer Ake,.}{dat u zoiets tegen een}{buitenlander zegt}\\

\haiku{Nu we toch over Poe, -.}{praten als jong meisje heb}{ik hem verslonden}\\

\haiku{Een volgende keer.}{moeten wij eens zorgen voor}{een vrolijker slot}\\

\haiku{De belastingen;}{waren op zichzelf zeker}{niet onredelijk}\\

\haiku{Als er volksvrouwen.}{getroffen waren zou het}{even erg zijn geweest}\\

\haiku{- Ja, dat verklaart me,.}{wel iets al blijft er nog veel}{onbegrijpelijks}\\

\haiku{Vroonhoven reed naast - -.}{zijn kleindochter in haar of}{zijn wagen naar huis}\\

\haiku{lichamelijke.}{aftakeling hadden een}{andere oorsprong}\\

\haiku{Maar ook wilde ze;}{onder geen voorwaarde haar}{plezier bederven}\\

\haiku{Een paar jaar later.}{zag ze die weg plotseling}{in een helder licht}\\

\haiku{- 't Is een goed kreng,.}{zei mevrouw Ulius wel van}{Sara Brandenburg}\\

\haiku{Op haar naam stond voorts,.}{een stukje dat misschien niet}{enig toch zeldzaam was}\\

\haiku{Ik wacht over een week,.}{een boot en dan vallen er}{weer zaken te doen}\\

\haiku{Dat alles was nu,.}{verdwenen en Katendrecht}{werd plaatsvervanger}\\

\haiku{Met de Kersttijd zou,.}{het zoontje van daarginds naar}{hier overkomen Jack}\\

\haiku{Haar aangeboren.}{trots hield haar lichaam overeind}{en onbewogen}\\

\haiku{Toen bood de kennis,.}{Colijn een sigaar aan maar}{Colijn weigerde}\\

\haiku{- En ikzelf ook, zei,.}{mevrouw Ulius weer naar de}{theetafel gaande}\\

\haiku{Dacht je dat ik je?}{d\'a\'arvoor helemaal naar hier}{zou laten komen}\\

\haiku{Kon die Brandenburg?}{het op een procedure}{laten aankomen}\\

\haiku{Anarchisten van.}{de oude tijd maakten wel}{meer zulke bommen}\\

\haiku{moederliefde was.}{nu eenmaal iets heel anders}{dan vaderliefde}\\

\haiku{Maar hoe ze heten,.}{dat zal uw zwager u wel}{kunnen vertellen}\\

\haiku{De reserve van.}{Lea en zijn zwager op dit}{punt stelde hij hoog}\\

\haiku{hij wist alleen dat,.}{daar een vrouw woonde en had}{haar naam vergeten}\\

\haiku{Daarop besloot hij.}{aan Van der Mark het weekblad}{terug te geven}\\

\haiku{Hij had behoefte.}{aan een degelijk gesprek}{en ging naar zijn club}\\

\haiku{Maar hij antwoordde,:}{en sprak daarbij uit wat hij}{vroeger had gedacht}\\

\haiku{- Knap gezegd, Til, en,.}{voor honderd procent juist zei}{de secretaris}\\

\haiku{Die onbescheiden.}{vraag heb ik zelfs niet aan mijn}{kleindochter gesteld}\\

\haiku{zijn huwelijk kon,.}{niet mislukt heten want het}{telde voor hem niet}\\

\haiku{Dan lijkt de tijd niet.}{ver meer dat de man ophoudt}{de vrouw te boeien}\\

\haiku{Als ik spreek van de,.}{mens als jabroer bedoel ik}{de grote massa}\\

\haiku{Ik vind dat je over.}{de grote massa toch wat}{onbillijk oordeelt}\\

\haiku{En wat moest er dan,?}{in haar kind omgaan als ze}{de waarheid hoorde}\\

\haiku{een halve meter,}{lager dan het eerste en}{wie zich daar ophield}\\

\haiku{Hij bracht haar de kop,.}{koffie en ze keek op en}{dankte vriendelijk}\\

\haiku{Ik bedoel niet hier,,.}{want hier is het o-kay}{maar ik meen de buurt}\\

\haiku{En ik hoor daar van.}{de buren dat ze pas om}{tien uur thuis komen}\\

\haiku{Hij zuchtte en keek,.}{haar aan met zijn gedachten}{bij die rij kroegen}\\

\haiku{Ik had wat gespaard,,.}{h\`e en mijn compagnon had}{de rest van het geld}\\

\haiku{Ze stond nog steeds te,.}{wachten een figuurtje van}{overmatig geduld}\\

\haiku{Dus als je liever... -,...}{nog een eindje omgaat Neen}{ik ga met je mee}\\

\haiku{- Neen, zei ze, geen acht, -,.}{slaand op zijn woorden en toon}{hier niet even verder}\\

\haiku{Van der Mark deed open.}{en kondigde mevrouw en}{meneer Hartman aan}\\

\haiku{Ze heeft het zelf nooit,.}{geweten maar ze was mijn}{bloedeigen dochter}\\

\haiku{Dat deze haar man.}{ontliep en daarin slaagde}{was haar onbekend}\\

\haiku{Nauwelijks zaten.}{zij toen Sara belde dat}{ze komen wilde}\\

\haiku{- Ze zijn allerliefst,,.}{zei Lea de sneeuwklokjes in}{een vaasje schikkend}\\

\haiku{Ze was zich bewust.}{van een conventioneel}{begin van gesprek}\\

\haiku{In haar gedachten.}{hield zij zich veel bezig met}{deze familie}\\

\haiku{Neen, inderdaad, het.}{was niet waar dat er hier geen}{overeenkomst bestond}\\

\haiku{Dat doet denken aan,.}{iets onvermijdelijks en}{het klinkt veel te zwak}\\

\haiku{Dan moet je toch blij.}{zijn dat je meer gewonnen}{dan verloren hebt}\\

\haiku{De besten wonnen,,.}{hem in het geestelijke}{en daarin alleen}\\

\haiku{Ze stond droefgeestig.}{en een beetje hulpeloos}{omlaag te kijken}\\

\haiku{En als hij wakker.}{mocht worden ziet hij dat er}{wat voor hem klaar staat}\\

\haiku{Hij vertaalde voor}{zijn gehoor de tekst van niet}{meer geheel klassiek}\\

\haiku{Aan het meisje dat.}{opendeed vroeg hij dan ook niet}{of mevrouw thuis was}\\

\haiku{Een weigering zou.}{deze zuil voor zijn ogen in}{stukken doen vallen}\\

\haiku{Maar denk u dan eens:}{in dat je als moeder van}{je kind horen moet}\\

\haiku{De voorzichtigste.}{vraag zou ongewoon zijn en}{achterdocht wekken}\\

\haiku{En het was een mooie,.}{middag echt zo'n weertje waar}{een mens van opfrist}\\

\haiku{- Ze hadden moeten.}{bedenken dat wij ook met}{ons twee\"en waren}\\

\haiku{Hij vond het echter.}{van geen belang en legde}{zich op de tafel}\\

\haiku{Later op de avond,.}{werd hij onrustig kreeg een}{inspuiting en sliep}\\

\haiku{het water dat tot.}{ontzaglijke knotten leek}{samengevlochten}\\

\haiku{Hij wist deze droom,.}{niet alleen op te roepen}{doch ook te leiden}\\

\haiku{Maar hij merkt heel ver,,.}{weg waar de kade eindigt}{nog iets anders op}\\

\haiku{De tijd van spelen.}{met zijn klasgenoten op}{de fiets was voorbij}\\

\haiku{Hij was blij met haar,.}{komst en sloeg geen enkele}{van haar jours meer over}\\

\haiku{Zijn secretaris.}{bracht en haalde hem in het}{kleine wagentje}\\

\haiku{Hij had het met een.}{allernaarste spitsvondigheid}{opzij geschoven}\\

\subsection{Uit: Fantastische vertellingen. Bundel 1}

\haiku{- Dat is niks voor u,,...}{meheer dat weet uwes eve zoo}{goed als ik zellef}\\

\haiku{- Nu, stuur het me dan,.}{morgenochtend en doe er}{de kwitantie bij}\\

\haiku{Een slecht mensch zal zich.}{meer geschokt voelen als wie}{hij slecht dacht goed blijkt}\\

\haiku{Ten slotte meende.}{ik de vervulling van mijn}{wensch nabij te zijn}\\

\haiku{Tusschen een en twee ',.}{uurs nachts had mijn vrouw een}{korten hollen hoest}\\

\haiku{Toen wende ik mij.}{aan een heete kruik mede}{te nemen naar bed}\\

\haiku{Ik had plotseling.}{een hevigen afkeer van}{mijn vrouw gekregen}\\

\haiku{Het duurde dan ook.}{niet lang of ik sliep weer in}{het geheel niet meer}\\

\haiku{De doodstraf was in.}{Nederland sinds tientallen}{jaren afgeschaft}\\

\haiku{Ik kleedde mij als.}{steeds zonder veel gerucht en}{in het duister uit}\\

\haiku{Wel tien maal reeds had.}{ik mij voorgenomen op}{haar toe te kruipen}\\

\haiku{Nog een, en nog een,.}{het tikte al dadelijk}{overal rondom mij}\\

\haiku{Uit een soort doezel,.}{werd ik wakker door een zacht}{glasachtig getik}\\

\haiku{De episode met,.}{de geit bijvoorbeeld daarvan}{rook ik de leugen}\\

\haiku{Mocht dit kunstmatig,.}{zijn aangezet dan was het}{onzichtbaar gedaan}\\

\haiku{- Als je niet later,,.}{komt dan zeven uur vind je}{me nog thuis Tinny}\\

\haiku{een schitterend en,.}{schitterend werk het verhaal}{van een strafproces}\\

\haiku{Langzaam wandelden.}{wij verder in de richting}{van Aerdenhout}\\

\haiku{Op het machtige;}{bordes stonden drie deuren}{wijd-noodend open}\\

\haiku{Ik voelde mij, of.}{ik mijn noodlot tegemoet}{liep en niet weg kon}\\

\haiku{Ik zag telkens om,.}{alsof ik mij in mijn rug}{aangevallen wist}\\

\haiku{De man had geen tijd.}{gehad om het geringste}{geluid te geven}\\

\haiku{Later op den avond,}{ruik ik het onloochenbaar}{en waarschuw ik u.}\\

\haiku{Drie kleine meisjes,,.}{gearmd liepen hem langzaam}{en schuw tegemoet}\\

\haiku{Alleen stonden aan.}{den ingang twee verzakte}{baksteenen kolommen}\\

\haiku{In een winkel kocht.}{hij een broodje dat hij aan}{de toonbank opat}\\

\haiku{Toen eindigde hij:}{met de eerste woorden van}{het Onze Vader}\\

\haiku{Van de overigen.}{bleef de overgroote meerderheid}{uit nieuwsgierigheid}\\

\haiku{De vreemdeling deed,.}{een paar stappen achteruit}{en bleef toen weer staan}\\

\haiku{Men praatte, lachte,,.}{joelde en schuifelde met}{voeten en stoelen}\\

\haiku{het koele licht van.}{het noorden viel er door drie}{hooge ramen binnen}\\

\haiku{Zij keken niet op,.}{terwijl het sinistere}{kloppen weerklonk}\\

\haiku{En toch hadden wij.}{bij die gelegenheid wel}{van ons doen spreken}\\

\haiku{Leiden is niet groot,.}{maar heeft de armoede}{van een wereldstad}\\

\haiku{Hij kwam pas later,.}{op den avond tegen een uur}{of tien aanzetten}\\

\haiku{Tot diep in den nacht.}{kon hij van die dooltochten}{door de stad houden}\\

\haiku{Zwijgend, den rug naar,.}{mij toe ontkleedde Jos van}{der Haerden zich}\\

\haiku{klonk het opeens uit.}{het bed aan het andere}{einde der kamer}\\

\haiku{Het duurde lang eer:}{ik mij goed bewust durfde}{worden van dit feit}\\

\haiku{Niet ten onrechte.}{spreekt men van de schatkamers}{van het geheugen}\\

\haiku{We herinneren.}{ons niet ze ooit gezien of}{gehoord te hebben}\\

\haiku{Nog voel ik daar de,.}{rimpels de vouwen van in}{mijn vingertoppen}\\

\haiku{Dienzelfden dag werd.}{Jos van der Haerden naar}{Endegeest gebracht}\\

\haiku{Ja, jongens, dat is.}{de geschiedenis van Jos}{van der Haerden}\\

\subsection{Uit: Fantastische vertellingen. Bundel 2}

\haiku{Hoe nietig en ver,.}{lijkt mij nu die tijd en hoe}{platvloersch mijn vreugde}\\

\haiku{Men zeide mij dat.}{de vrouw zelf het kind uit het}{raam geworpen had}\\

\haiku{Ik keek neer op den,.}{slaper die veel ellende}{moest hebben beleefd}\\

\haiku{En ik zal u ook,.}{wel eens mijn naam zeggen maar}{nu liever nog niet}\\

\haiku{In een behoefte}{te ontkomen aan den looden}{last der omgeving}\\

\haiku{Een dag later kon:}{de nieuwe kostganger op}{mijn kaartje lezen}\\

\haiku{Hij stak het gaslicht.}{op en wees mij een rieten}{leunstoel bij het raam}\\

\haiku{hij was verdwenen,.}{en toen ik geen antwoord kreeg}{stapte ik binnen}\\

\haiku{Ik merkte beleefd,.}{op dat hij den jongen zoo}{juist had ontslagen}\\

\haiku{Niet alzoo evenwel,.}{in de Jodenbuurt want ik}{kwam er maar weinig}\\

\haiku{schreven ze jiddisch, -}{met hebreeuwsche schrijfletters}{tusschen twee haakjes}\\

\haiku{'s Avonds speelde hij,.}{met mijn ouders kaart en zat}{ik toe te kijken}\\

\haiku{Hij zag op, schichtig,,.}{als betrapt en trok ijlings}{den sleutel terug}\\

\haiku{Maar ach, hij is er, ().}{al zie ikdit met een blik}{op het karretje}\\

\haiku{Ik nam schutterig.}{mijn hoed af en brabbelde}{iets onverstaanbaars}\\

\haiku{En een lachje klonk.}{uit haar keel of er linnen}{vaneen werd gescheurd}\\

\haiku{In de verte zag,.}{ik een rechte streep licht en}{vernam ik geluid}\\

\haiku{Nog een tweede schot,.}{klonk maar ik was reeds naar de}{kamerdeur geijld}\\

\haiku{Neen, het nachtpitje,:}{werd ons devies en daarbij}{versta men mij w\`el}\\

\haiku{het servituut van,.}{soortgelijke uitloozing maar}{dan straalsgewijze}\\

\haiku{En ik poogde het.}{jonge vrouwtje weer in mijn}{armen te trekken}\\

\haiku{Intusschen bleek het:}{geval van Testal haar toch te}{interesseeren}\\

\haiku{- Dat zal dan, poogde.}{ik aan deze gissingen}{een eind te maken}\\

\haiku{IJl klepte de klok.}{van Hornte in onzen rug}{het uur van half tien}\\

\haiku{De Reuzenpit wou,.}{dadelijk binnendringen}{maar ik weerhield hem}\\

\haiku{'s Anderen daags;}{kwam het parket over uit de}{naburige stad}\\

\haiku{grijnsde hij, opende,.}{het portier en trok mij aan}{mijn kraag naar binnen}\\

\haiku{Laat het u genoeg.}{zijn dat ik tot het leidend}{comit\'e behoor}\\

\haiku{De methode om.}{die bergplaatsen te vinden}{was heel eenvoudig}\\

\haiku{De menschen schenen.}{mij te behooren tot het}{uitvaagsel van Praag}\\

\haiku{Ondanks mijn afkeer.}{van paarden moest ik toch dit}{dier bewonderen}\\

\haiku{En toch bleef Praag nog.}{steeds iets van zijn bekoring}{voor mij behouden}\\

\haiku{En daarmee was het,.}{uit en de Pragers gingen}{vergenoegd verder}\\

\haiku{was het laatste wat...-,,,.}{ik dacht ~  Nou nou word es}{wakker jongenlief}\\

\haiku{Kalm aan maar, meneer,.}{van de Kasteelbergen sla}{uw oogen gerust op}\\

\haiku{Bovendien hoopte,.}{ik op afleiding en zoo}{begon ik de reis}\\

\haiku{De zeden- en.}{andere politie is}{verduiveld waakzaam}\\

\haiku{Dan kan ik u nog.}{een adres noemen waar men u}{misschien helpen kan}\\

\haiku{Ik heb hem al over,}{u gesproken maar omdat}{ik niet zeker wist}\\

\haiku{Achter mijn hielen.}{sloeg Talamon de voordeur}{dicht dat het dreunde}\\

\haiku{Intusschen, het had,,.}{mij met een flesch ouden port}{weer kracht gegeven}\\

\haiku{Onderaan stootte,.}{ik tegen iets hards dat mij}{den weg versperde}\\

\subsection{Uit: Fantastische vertellingen. Bundel 3}

\haiku{Zware luiken met.}{ijzeren bouten waren}{voor de twee vensters}\\

\haiku{Ik viel plat op mijn,.}{buik maar mijn handen voor mij}{plasten in water}\\

\haiku{- Ik zal je zeggen,.}{waar je bent dat is waar je}{altijd bent geweest}\\

\haiku{De oude was in,.}{deze buurt wel bekend maar}{trok niet de aandacht}\\

\haiku{Want op het bed in,.}{het midden der kamer lag}{de oude slapend}\\

\haiku{Maar dan had hij, hij,.}{alleen ook de hand gehad}{in haar verdwijning}\\

\haiku{Ik snelde naar den,.}{tuin maar de ramen waren}{alle gesloten}\\

\haiku{Zijn tijd was omstreeks,.}{vier uur en steeds bleef hij in}{de nabijheid}\\

\haiku{Dikwijls was ik 's,.}{avonds alleen en vrij om door}{de stad te dwalen}\\

\haiku{Dit  gruwelstuk.}{stelde het vorige nog}{weer in de schaduw}\\

\haiku{Gewoonlijk waren.}{dat habitueele dronkaards}{en zwakzinnigen}\\

\haiku{Aldus eindigde.}{gelijk een droom de stilste}{dag van mijn leven}\\

\haiku{Zelfs voelde ik mij.}{in dezen tijd relatief}{kalmer dan anders}\\

\haiku{Op Oudejaar werd,.}{Londen bezocht door een mist}{den echten ditmaal}\\

\haiku{Hij liep te rillen.}{en te klappertanden en}{zijn gelaat zag blauw}\\

\haiku{In vermetelheid.}{overtrof deze daad alle}{voorafgegane}\\

\haiku{Zij volgen hier voor:}{wie iets nader komen wil}{tot de verklaring}\\

\haiku{Van zijn afkomst, naam - -.}{en leeftijd we zeiden het}{reeds weten wij niets}\\

\haiku{De schrijftrant wijzigt.}{zich geheel waar het vers tag}{der moorden aanvangt}\\

\haiku{En neen, Bob, ik wil;}{je nu geen oogenblik meer}{in spanning laten}\\

\haiku{hoe zonderling ik}{met mijn aankondiging te}{willen gaan rusten}\\

\haiku{Ze lag zoo stil dat,.}{ik dacht dat ze sliep en zelf}{dommelde ik in}\\

\haiku{Met dat al weerhield.}{mij iets den vreemdeling mijn}{rug toe te wenden}\\

\haiku{mijn stappen knerpten,.}{ik rilde en het angstzweet}{stond op mijn voorhoofd}\\

\haiku{Dat geloof je toch,,,?...}{niet van me is het wel jij}{mijn beste vriendin}\\

\haiku{Lang bleef ik ook niet.}{beheerscht door het gevoel}{van gekrenkt te zijn}\\

\haiku{Lilian had me}{toen reeds zoo in haar macht dat}{ik daar zat gelijk}\\

\haiku{Ik zal de eenige,.}{vrouw ter wereld zijn die ooit}{met een aap trouwde}\\

\haiku{{\textquoteright} Het is een gissing,,.}{meer niet maar mijn gevoel zegt}{me dat ik juist raad}\\

\haiku{, en dat wanneer ik.}{het gekund had ik het niet}{zou hebben gewild}\\

\haiku{, en als eenig antwoord.}{van dit haast stomme gedrocht}{zijn galmenden hoest}\\

\haiku{, ik nam haar in mijn.}{armen  en ik droeg haar}{naar huis als een kind}\\

\haiku{Maar het zijn alleen.}{de gebeurtenissen die}{den afstand maken}\\

\haiku{Halverwege vond.}{ik een voetpad dat recht op}{de villa aanliep}\\

\haiku{Het goed dat aan het.}{leven zijn groote waarde geeft}{en zijn eenig doel}\\

\haiku{Mijn liefde had, van,;}{mijzelf ongeweten de}{vrees doen inslapen}\\

\haiku{Aan den rand van het.}{ravijn strekten wij ons uit}{op mos en varens}\\

\haiku{Het pad voerde door,.}{een bosch bedropen met een}{fel stralend herfstgeel}\\

\haiku{Het kerkhof, voorheen,.}{bolwerk van  den dood lag}{zonder verschrikking}\\

\subsection{Uit: Rood paleis}

\haiku{Hier Tijs, Tijs heet je,,.}{ik weet het nog bliksems goed}{neem een sigaret}\\

\haiku{Wij zijn niets en we,.}{weten dat we iets moesten zijn}{dus twijfelen we}\\

\haiku{Hij had altijd een.}{kleine voldoening als hij}{zooiets had gezegd}\\

\haiku{Maar de roode stoep,.}{bleef verlaten de mannen}{waren verzwolgen}\\

\haiku{Ze lachten langs hem,.}{met mooie bijtmonden een groote}{en een kleinere}\\

\haiku{Tijs hoopte dat hij.}{bij zijn impotentie zou}{kunnen volharden}\\

\haiku{Zij moederde op.}{de vervaarlijke manier}{van een dwingeland}\\

\haiku{In de vlammen zag.}{haar gezicht schrikwekkend licht}{en ongestadig}\\

\haiku{In het licht maakte.}{zijn haar het effect van een}{kleurloos aureool}\\

\haiku{Een heelen tijd bleef.}{hij zoo in de duisternis}{met twee dotten vuur}\\

\haiku{Dat voelde hij zelf,,.}{daarvoor spande hij zich in}{daarop was hij grootsch}\\

\haiku{- Pas op, zei hij, want.}{ze liep bijna tegen de}{mand op met den hond}\\

\haiku{Aan het raam zaten,.}{ze niet dat stond niet voor een}{respectabel huis}\\

\haiku{Het duurde nogal,.}{eer Tijs zijn geld besteed had}{want hij was secuur}\\

\haiku{Het caf\'e In de.}{groote zaal van het koffiehuis}{was het warm en kil}\\

\haiku{Vanzelf kwam hij toen.}{tot het voorstel van een nieuw}{bezoek aan het huis}\\

\haiku{Rond hem, ten allen,.}{kant orgelde Rood Paleis}{zijn regenkoraal}\\

\haiku{Hij dacht vaak het rood.}{van de hoeken te zullen}{zien wegbiggelen}\\

\haiku{Blind, toch veilig liep.}{hij achter den leihond van}{zijn gedachten aan}\\

\haiku{Dan kwam wel een lach,.}{een zwijgende lach van een}{paardengebit}\\

\haiku{Ze was een groote vrouw,,.}{deze Marie van Dam een}{eind in de dertig}\\

\haiku{Het doodsbed - Eh bien,,,,.}{regardez Marie-Laure}{c'est moi madame}\\

\haiku{Tranen kwamen in,.}{zijn oogen hij vond zichzelf zoo}{verschrikkelijk goed}\\

\haiku{Een zaalzuster zag.}{hem na en dacht het woord der}{omstandigheden}\\

\haiku{Het was erg ditmaal,,.}{zag Sauger maar als altijd}{niet onherstelbaar}\\

\haiku{Aldus had God den,.}{mensch geschapen niet slechts naar}{Zichzelf uit Zichzelf}\\

\haiku{Wij traden gereed,.}{uit ons lichaam wij traden}{niet uit onzen geest}\\

\haiku{En het was vreemd, maar.}{in Henry Leroy had de}{koude bewogen}\\

\haiku{Men nam het bestaan,.}{van den gemakkelijken}{kant maar met overleg}\\

\haiku{Ze had geen wrok, het.}{hinderde alleen dat daar}{kapitaal braak lag}\\

\haiku{Op deze slappe.}{heeren hadden zij en haar}{have niet veel greep}\\

\haiku{Hij rookte een paar,,.}{sigaretten bij het vuur}{las zijn krant stond op}\\

\haiku{Onlangs nog had hij,.}{er een begraven een der}{oppervlakkigste}\\

\haiku{Hij had het weer in.}{zijn hoofd gezet den zwaren}{Eduard te tergen}\\

\haiku{De leveranciers.}{en hun personeel kwamen}{nooit aan de hoofddeur}\\

\haiku{Het mysterie van.}{den bijkelder verdiende}{benut te worden}\\

\haiku{Het gaf vertrouwen.}{dat het een gehuwd man was}{met drie kinderen}\\

\haiku{Hij zei het ronduit,.}{want hij zat om den drommel}{niet onder de plak}\\

\haiku{Hij stak vol werklust,.}{maar was te zakelijk om}{zich te overschatten}\\

\haiku{Hij kon zijn triomf,.}{niet kwijt hij liep er dien avond}{mee naar Rood Paleis}\\

\haiku{Hij dronk niet heel veel,.}{champagne maar vandaag kon}{hij er slecht tegen}\\

\haiku{De waardin had hem.}{gezegd dat ze voor zaken}{naar Marseille ging}\\

\haiku{Hij opende een raam,.}{het rijtuig liep dicht achter}{de locomotief}\\

\haiku{Dan naar de oude.}{haven en de zeemansbuurt}{achter het stadhuis}\\

\haiku{E\'en ding maar wist hij,,:}{zeker en had hij stellig}{geweten altijd}\\

\haiku{Hij vond het groene,.}{gif niet lekker en liet het}{bij een enkel glas}\\

\haiku{Keurige oude.}{heeren als hij kwamen niet}{in het strafbankje}\\

\haiku{Op zijn ergst werden.}{ze opgeborgen in een}{sanatorium}\\

\haiku{Gemakkelijk wist.}{hij al deze gedachten}{uit te schakelen}\\

\haiku{Die in het zwart had,.}{al grijzend haar maar daarom}{was ze nog niet oud}\\

\haiku{En het scheen  of.}{de meisjes onrustiger}{waren dan vroeger}\\

\haiku{Tastenbreker en.}{Leroy hield stand temidden}{van de uitbarsting}\\

\haiku{De heeren hadden,.}{afgedaan en met hen de}{spelen der heeren}\\

\haiku{Hij was ten slotte,.}{vader en had een instinct}{van bezorgdheid}\\

\haiku{Een derden keer liet,.}{hij zich niet afschepen dat}{zat niet in zijn aard}\\

\haiku{Hij voerde haar in.}{een kleine spreekkamer aan}{het eind van de gang}\\

\haiku{Het gelaat was zeer,.}{groot en zeer hevig maar heel}{anders dan vroeger}\\

\haiku{Ze ging het land uit,,.}{naar Spanje ze had nog geen}{bepaalde plannen}\\

\haiku{Als hij het lang liet.}{staan dekte het nog niet de}{kosten van opslag}\\

\haiku{Ze had daar kunnen.}{ingaan en uitgaan zonder}{te zijn opgemerkt}\\

\subsection{Uit: Verzamelde verhalen}

\haiku{het werd niet gebruikt,.}{om te slapen en er kwam}{nooit iemand voorbij}\\

\haiku{Maar toen de leeftijd.}{van het meisje bleek legde}{hij zijn potlood neer}\\

\haiku{Hij zag hem haastig.}{een jas uittrekken en door}{het vertrek gooien}\\

\haiku{{\textquoteright} {\textquoteleft}Nee,{\textquoteright} antwoordde de, {\textquoteleft}}{jonge minnaar die nu zelf}{zekerheid verkreeg}\\

\haiku{Zijn inbreuk op de.}{huiselijke gewoonten}{kon argwaan wekken}\\

\haiku{Want hij keek eerst in,,:}{de tuin dan naar zijn vriend en}{antwoordde langzaam}\\

\haiku{Maar dat kon anders,.}{worden en hij voelde zich}{verre van gerust}\\

\haiku{Hij wist niet dat het.}{in de tussentijd opnieuw}{nacht was geworden}\\

\haiku{De ogen waren nu,,.}{wijd open maar er was geen licht}{in geen enkel}\\

\haiku{Het noemen van de.}{naam Moona Wilson werd voor}{beiden een noodlot}\\

\haiku{Deze bundel hield.}{ze echter zorgvuldig voor}{Graham verborgen}\\

\haiku{Kram werd na een poos.}{afgelost door Immerzeel}{voor de  nachtdienst}\\

\haiku{Maar wel verhuisde.}{het zo gauw mogelijk naar}{een andere buurt}\\

\haiku{Hij schaamde zich haast.}{voor zijn blanke benen van}{intellectueel}\\

\haiku{Een enkele maal;}{zonk de reiziger tot de}{enkels in het zand}\\

\haiku{het was hem bij zijn,.}{nuchterheid onverklaarbaar}{haast onaangenaam}\\

\haiku{Ja, zijn zoon kreeg een,.}{mooie begrafenis dat was}{tenminste een troost}\\

\haiku{Hier was de open deur.}{naar de achterkamer met}{het lijk van zijn zoon}\\

\haiku{Halman hoopte dat,.}{niemand iets merkte maar hij}{was ronduit verbluft}\\

\haiku{Natuurlijk, dacht hij,.}{weer in zijn beperktheid dat}{valt elke vrouw op}\\

\haiku{Dat is gebeurd in,,.}{dat andere land ik weet}{niet waar en ook hier}\\

\haiku{Zijn naam wordt me niet,.}{geopenbaard ook niet in}{de initialen}\\

\haiku{Het was met een bril,?}{begonnen en waarop was}{het uitgelopen}\\

\haiku{Daar viel niet aan te,.}{twijfelen en daar lag dus}{niet het gevaar in}\\

\haiku{{\textquoteleft}Jongen,{\textquoteright} zei ze, {\textquoteleft}zou,?}{je dat nu wel doen en dan}{vlak voor je examen}\\

\haiku{De ondergrondse.}{gruwelkamer viel hem per}{saldo toch niet mee}\\

\haiku{En dan waren er.}{toch enkele grappige}{momenten geweest}\\

\haiku{De massa bleef zo,.}{ongeveer op het oude}{peil ook geestelijk}\\

\haiku{Aldus ging het ook.}{in die richting tot in het}{oneindige voort}\\

\haiku{Nu bevond meneer.}{Kars zich in de eerste laan}{van de villabuurt}\\

\haiku{Hij wilde de streek.}{na meer dan twintig jaren}{nog eens terugzien}\\

\haiku{Diende het wellicht,?}{enig vrachtvervoer dit oude}{spoorwegstation}\\

\haiku{De man stond ervoor,.}{tussen de stammen van de}{rijzige houtwal}\\

\haiku{De chef zette zich.}{na een enkele hoofdknik}{tegenover de man}\\

\haiku{En ook de chef zelf.}{moest van zijn eigen diepte}{onkundig wezen}\\

\haiku{En daaroverheen vond.}{hij het wonderlijk deze}{figuur hier te zien}\\

\haiku{Het geruis leek een.}{soort fonetisch kiekeboe}{met hem te spelen}\\

\haiku{Ik moet er weer eens.}{uit en heb een week vervroegd}{verlof genomen}\\

\haiku{Waar ik naar toe ga,.}{weet ik nog niet precies maar}{ik zalje schrijven}\\

\haiku{Hier was de natuur.}{minder gevorderd dan daar}{waar hij vandaan kwam}\\

\haiku{Tot zover  dacht.}{hij onwetend wezenlijk}{als zijn schoonvader}\\

\haiku{Hij deed het met een.}{keurigheid nauwelijks meer}{van deze aarde}\\

\haiku{De reactie op.}{deze verrassing evenwel}{was zeer ongelijk}\\

\haiku{Gerrit, allerminst,.}{van de vlugsten bleef stokstijf}{op dezelfde plaats}\\

\haiku{We hebben Vermeers.}{Meisjeskopje overgedrukt}{over Rembrandts Nachtwacht}\\

\haiku{U hebt een huisvriend.}{die u vrezen doet voor uw}{huwelijksgeluk}\\

\haiku{Mocht er dan toch iets,.}{gebeuren dan bent u er}{tenminste zelf bij}\\

\haiku{Trouwens, er waren.}{reeds bepaalde vermoedens}{bij mij gerezen}\\

\haiku{De vraag die ons toen.}{ging intrigeren was w\'a\'ar}{Willem vi stilstond}\\

\haiku{dat er van onze.}{soort een onbegrensd aantal}{in de kosmos hing}\\

\haiku{We misten iets en.}{hielden het er voor dat we}{de kleine misten}\\

\haiku{Boven zijn zinnen.}{als instrumenten van nut}{rijst hij  nooit uit}\\

\haiku{Met dat al waren.}{we onder elkaar niet meer}{geheel dezelfden}\\

\haiku{Er werd zwaarmoedig,,.}{gekeken tersluiks gezucht}{witjes gelachen}\\

\haiku{Nee, het huis was niet,.}{rechtstreeks getroffen maar de}{luchtdruk wrong het stuk}\\

\haiku{Met grote moeite.}{wrikte hij uit de zijkant}{een portefeuille}\\

\haiku{Het laatste wat hij - -:}{dacht voordat hij droomloos voor}{het eerst insliep was}\\

\haiku{{\textquoteleft}Maar ze mag niet bij.}{me slapen en ook niet}{aan het eten komen}\\

\haiku{Want binnen het jaar.}{te zullen trouwen had hij}{moeten beloven}\\

\haiku{Hij trof dan ook in.}{Zwaagwakum allerminst een}{gespannen gehoor}\\

\haiku{{\textquoteright} zei de vrouw, zonder.}{te weten hoe precies juist}{ze zich uitdrukte}\\

\haiku{Ik heb twee diners,,.}{besteld geen drie want Louron}{eet niet van de kok}\\

\haiku{Hij had opnieuw het,.}{oude huis betrokken hij}{bleef de grote man}\\

\haiku{Nu ja, ze droegen.}{het haar veel korter en met}{meer afwisseling}\\

\haiku{Wellicht zou Falker.}{zonder de klok nooit tot zijn}{daad zijn gekomen}\\

\haiku{Aan de andere.}{kant zou dat misschien toch te}{ondeugend zijn}\\

\haiku{Allemaal nonsens,.}{maar hij zou haar eens flink aan}{het schrikken maken}\\

\haiku{Me dunkt eerder dat.}{dat hier vlotter zal gaan dan}{in het vorige}\\

\haiku{Maar die stip werd met.}{razende snelheid zwarter}{en duidelijker}\\

\haiku{Zelfs lag er daardoor.}{wat extra gezelligheid}{in het vooruitzicht}\\

\haiku{Ze hoeft alleen maar.}{te letten op de deuren}{zonder naamborden}\\

\haiku{Was al dat hout er?}{de oorzaak van dat hij niet}{kon worden gepeild}\\

\haiku{Er liep dwars door dit.}{benedenhuizencomplex}{een centrale gang}\\

\haiku{Aldus bleef Ada bij,.}{de bezoeken aanwezig}{maar ook zij alleen}\\

\haiku{Particuliere.}{klachten raken het bureau}{volkshuisvesting niet}\\

\haiku{toch volbracht hij, met,.}{veel pijn zijn dagelijkse}{gang naar het bureau}\\

\haiku{Hij zorgde er wel.}{voor dat er geen getuigen}{beschikbaar waren}\\

\haiku{En middelerwijl.}{deinden de passagiers als}{een volgzaam getij}\\

\haiku{vroeger moet het toch,.}{beter zijn geweest en in}{elk geval bonter}\\

\haiku{Ze bukte naar het,:}{koffertje op de treeplank}{maar hij was haar voor}\\

\haiku{Er moesten jasknopen,.}{sneuvelen er moest kwetsbaar}{garnituur scheuren}\\

\haiku{Onderwijl had hij.}{ontdekt waarin dat bij haar}{onbepaalde school}\\

\haiku{Het gaf haar iets wreeds,.}{en dat werd versterkt door die}{beitels van tanden}\\

\haiku{Ze werd aangedrukt.}{tegen de stang die ze niet}{had losgelaten}\\

\haiku{Coentje en Mieltje.}{Coentje was nu negen en}{Mieltje nog pas zes}\\

\haiku{geen machientje en.}{zijn broer moest ze toch wel aan}{een draad voorttrekken}\\

\haiku{Ook ontwaakt hij 's,,.}{nachts \'e\'en of twee keer voor een}{half uur een kwartier}\\

\haiku{Er bestaat nergens,.}{uitwijkmogelijkheid want}{er is ook geen links}\\

\haiku{Eva mag daar niets van.}{weten en hij verdiept zich}{nu in zijn werk}\\

\haiku{Ze weten dat hun.}{gedachten parallelle}{wegen zijn gegaan}\\

\haiku{Daar stampten dof in,.}{zijn rug stappen en weer stond}{hij en wendde zich}\\

\haiku{{\textquoteright} {\textquoteleft}Dat weet ik, maar ik.}{ben een oude vriend en ik}{kom hem iets brengen}\\

\haiku{We wachten u al,.}{een halve dag en langer}{meneer Reiziger}\\

\haiku{Enfin, dit huis was,,.}{toch een hol h\`et hol en de}{uitlegging nabij}\\

\haiku{{\textquoteright} {\textquoteleft}Daar vraag je me iets,.}{waarop ik geen antwoord zou}{weten Reiziger}\\

\haiku{{\textquoteright} vroeg de reiziger,,.}{want hij kende het en er}{was schrik in zijn toon}\\

\haiku{De zender was dus.}{op de een of andere}{manier ingelicht}\\

\haiku{Mocht dit een begin,.}{zijn van wereldondergang}{welnu in godsnaam}\\

\haiku{Het was geen aanloop;}{tot een polemiek over de}{wereldondergang}\\

\haiku{{\textquoteright} {\textquoteleft}Dat zou ik van de...}{omstandigheden willen}{laten afhangen}\\

\haiku{Hij hield echter de.}{vragen kort en zo waren}{ook zijn antwoorden}\\

\haiku{Ze liet de voordeur,.}{open want ze had zich al reeds}{enigszins herwonnen}\\

\haiku{In het ziekenhuis.}{bracht dokter Kramer beiden}{naar een wachtlokaal}\\

\haiku{Mogelijk had ze {\textquoteleft}{\textquoteright}.}{na de daad ervaren dat}{ze vanhem niet hield}\\

\haiku{Haar vlug verstand trok.}{uit het verhaalde een heel}{andere slotsom}\\

\haiku{Zijn medelijden.}{met de moeder verdrong haast}{zijn eigen verdriet}\\

\haiku{Zijn positie van.}{procuratiehouder liep}{geen moment gevaar}\\

\haiku{De standseer verbood.}{het en de standseer ging de}{Kas boven alles}\\

\haiku{De eigenlijke,.}{werklokalen het sanctum}{bereikte hij niet}\\

\haiku{Zijzelf had met de,.}{kinderen gegeten en}{zou nu opdienen}\\

\haiku{Doch eensklaps keerde,,.}{zij om en liep even terug}{een winkel binnen}\\

\haiku{Hun gesprek was nu,.}{wat meer alledaags maar ook}{vertrouwelijker}\\

\haiku{Om de benauwing.}{te verdrijven stond hij op}{en rekte zich uit}\\

\haiku{Daar, eenzame,{\textquoteright} zei,.}{ze met een glimlach die hem}{gelukkig maakte}\\

\haiku{Want deze was de,.}{enige die in het dolle}{plezier niet deelde}\\

\haiku{Nog lang kon Teyne.}{hun drukke stemmen horen}{klinken in de nacht}\\

\haiku{Het is tenslotte.}{even vervelend als altijd}{ernstig te wezen}\\

\haiku{Maar uit vrees nog meer.}{te bederven ging hij er}{niet verder op door}\\

\haiku{{\textquoteleft}Ik had permissie,.}{tot de beek met u mee te}{gaan juffrouw Heding}\\

\haiku{En denk u ook niet,.}{dat ik hierin verschil van}{andere mannen}\\

\haiku{ik zie je althans,,;}{zo en ik zou wel dwaas zijn}{als ik dat verzweeg}\\

\haiku{{\textquoteright} Teyne deelde mee,.}{dat zulk een artikel in}{zijn schaapskooi ontbrak}\\

\haiku{onze twee zoons vlak,,...}{na elkaar getrouwd het huis}{uit en het land uit}\\

\haiku{Zijn honden waren.}{gauw moe van het stoeien en}{liepen in zijn spoor}\\

\haiku{Toen riep de vrouw van.}{de houtvester hen binnen}{voor de koffie}\\

\haiku{Het was de eerste,.}{die hij meemaakte als lid}{van de toneelclub}\\

\haiku{{\textquoteleft}Ze gaat met de herfst,{\textquoteright}, {\textquoteleft}.}{weg schreef hijmet haar ouders}{weer naar Utrecht terug}\\

\haiku{daarvoor stond hij te.}{spoedig kritisch tegenover}{al wat hem overkwam}\\

\haiku{het was het gezond.}{vitale verlangen van}{de man naar de vrouw}\\

\haiku{{\textquoteleft}Herinner je je,,,}{dan niet dat je eens toen we}{samen wandelden}\\

\haiku{Maar Kersti, besef,,?}{je wel wat je nu op dit}{moment voor me doet}\\

\haiku{de kleine kerel;}{plaatste het ene been telkens pal}{voor het andere}\\

\haiku{De Tombelaine;}{is alleen van het oosten}{af te naderen}\\

\haiku{Verschoor nam uit een,.}{muurkast een klein voorwerp dat}{zwaar leek te wezen}\\

\haiku{{\textquoteleft}Ik zal beginnen.}{met de grootste diepte te}{fotograferen}\\

\haiku{even een schittering,,.}{er floot iets door de lucht en}{dan was alles stil}\\

\haiku{De eerste honderd.}{meter ging de duikerklok}{zeer langzaam omlaag}\\

\haiku{Naarmate het licht.}{afnam verzocht Verschoor de}{gang te versnellen}\\

\haiku{het staaldraad wond zich.}{geluidloos van de ene klos}{na de andere}\\

\haiku{En het was voor ons:}{beiden een verademing}{toen het antwoord kwam}\\

\haiku{Met een nog vage.}{ontzetting meende ik te}{zien dat het bewoog}\\

\haiku{{\textquoteright} Opnieuw had ik met.}{een blik van wilde angst door}{de sleuf gekeken}\\

\haiku{Ik moest mij buigen,,.}{niet slechts voor mijn meester maar}{ook voor het genie}\\

\haiku{De stukken pasten -.}{in elkaar tot een lucht en}{waterdicht geheel}\\

\haiku{Toen ik bij kennis,.}{kwam was ik op dek zonder}{mijn meester en vriend}\\

\haiku{Men zag en voelde.}{hem overal onmiddellijk}{als superieur}\\

\haiku{Och kaptein, als ik,.}{aan hem terugdenk dan kan}{ik bedroefd worden}\\

\haiku{Aan de bediening.}{van de wijn schonk hij altijd}{zijn volle aandacht}\\

\haiku{Bijna struikelde,.}{ik in het open graf maar er}{was daar niemand meer}\\

\haiku{En dit voornemen,,.}{vreemd als het zijn mocht deed mij}{toch sympathiek aan}\\

\haiku{Een ogenblik als het.}{huidige moest in stilte}{worden ondergaan}\\

\haiku{Kalkemeijer zag,.}{vanuit de diepte naar haar}{op bleek en ernstig}\\

\haiku{facile princeps,.}{leerde ik indertijd op}{het gymnasium}\\

\haiku{Dwaasheid dat hij met;}{een staf tegen de rots zou}{hebben geslagen}\\

\haiku{Het schoonhouden van;}{mijn bonte winkelvoorraad}{is zijn grootste trots}\\

\haiku{{\textquoteleft}Nee maar, da's \'o\'ok 'n,{\textquoteright}.}{bak zegt hij vrij luid achter}{zijn hand tegen mij}\\

\haiku{Maar zou u denke?}{dassij nou niks beginne}{ken om die cente}\\

\haiku{Ik kan er volstrekt,.}{niet uit wijs worden hoeveel}{moeite ik me geef}\\

\haiku{Maar ook als de wil:}{zich splitst zie ik toch dat \'e\'en}{de overhand behoudt}\\

\haiku{Ik druk vaster mijn,.}{arm om het wonderkind om}{mijn eigen jongen}\\

\haiku{jullie ken me de, -...}{moord stikke en medeen}{draai ik de bak in}\\

\haiku{Mijn hemel, dat zijn!}{nu de voldoeningen van}{de man uit het volk}\\

\haiku{Nee, in zijn hand is}{een bot mes waarmede hij}{eerst onbeholpen}\\

\haiku{Twee ontzaglijke.}{zeugen kwamen er knorrend}{over een hek kijken}\\

\haiku{{\textquoteright} informeerde de,.}{bezoeker die graag enige}{kunde wou tonen}\\

\haiku{{\textquoteright} Boer Berdeur draaide.}{nadenkend de steel van zijn}{pijp rond in zijn oor}\\

\haiku{Moar hai wou en hai.}{zou met die Azaintje van De}{Joager trouwe}\\

\haiku{Ik mag d\^a vollek, ',. '}{niet ben opscheppers main}{nie kerreks genog}\\

\haiku{Moar hai blaift met z'n,,...}{pote van d'r af en nog}{es gelaik het ie}\\

\haiku{Het werd steeds geplaatst,,.}{meest in feuilletonvorm en}{gaarne gelezen}\\

\haiku{Hij worstelde zich,.}{van de droom los en ging recht}{zitten in zijn bed}\\

\haiku{Hij trok aan de deur.}{die dadelijk toegaf en}{stond nu in de stal}\\

\haiku{de justitie had.}{ongetwijfeld de zaak nog}{niet opgegeven}\\

\haiku{Het meisje werd van.}{de weg getild en op een}{hand wagen gelegd}\\

\haiku{Van het meisje weet.}{ik alleen dat ik haar op}{de grond zag liggen}\\

\haiku{Hij moest ook zien een.}{of twee politiemannen}{mede te krijgen}\\

\haiku{In een oogwenk was.}{Piquillo beneden en}{bij de anderen}\\

\haiku{Hij had, gelijk men,.}{dat noemt een verborgen snaar}{in mij doen trillen}\\

\haiku{Mocht je zijn woorden,.}{geloven dan had hij er}{als een vorst geleefd}\\

\haiku{{\textquotedblright} Ik kon in de verste.}{verte niet raden wat hij}{me had voorgezet}\\

\haiku{Nu pauzeerde hij,.}{en hij sloeg zijn ogen op ter}{hoogte van mijn maag}\\

\haiku{- maar ik wil en ik}{zal toch ook eens werkelijk}{genieten van die}\\

\haiku{En toch viel zijn woord.}{in mij als het zaaizaad in}{de akkervore}\\

\haiku{Hoe ook gehecht aan}{aardse goederen hechtte}{hij niet het minst aan}\\

\haiku{Hij kon ook in de.}{eerste kamer gaan met niets}{dan het lichtpeertje}\\

\haiku{Nadat de ander.}{was vertrokken sloot hij het}{geld in zijn bureau}\\

\haiku{welke hen om zich.}{groepeerden tot een kleine}{klas om de meester}\\

\haiku{Denk er dan om, de...}{volgende keer donder ik}{je op de keien}\\

\haiku{Het was het grote,...}{ogenblik dat woorden tot niets}{dienden maar daden}\\

\haiku{En toch kwam hij slechts}{tot het besluit de woning}{voor welker behoud}\\

\haiku{Toen ging hij terug.}{naar zijn kamer en bleef daar}{de verdere dag}\\

\haiku{3Valt chronologisch.}{tussen de eerder openbaar}{gemaakte brieven}\\

\section{Henri Borel}

\subsection{Uit: Een droom}

\haiku{Ik hoorde niet veel.}{van hem toen ik weer terug}{was in Soerabaia}\\

\haiku{Ik schijn dan toch w\`el.}{in het rijk der wonderen}{te zijn aangeland}\\

\haiku{En ik begrijp maar.}{niet hoe ik ooit iets in haar}{gezien kan hebben}\\

\haiku{Ze zijn allen zoo.}{ongedwongen en zoo lief}{en zoo hartelijk}\\

\haiku{*** 's Ochtends, dikwijls,.}{al om tien uur begint een}{somber droomen-spel}\\

\haiku{Het lijkt nu alles,,....}{eeuwen eeuwen oud en van}{een v\`er verleden}\\

\haiku{Ik staarde, staarde,,.}{maar niets was te zien dan de}{dikke grijze mist}\\

\haiku{De rug licht-grijs.}{met een smal dons van wit bont}{als rand er omheen}\\

\haiku{Die zwartende sneeuw,.}{daalt over het duistere dal}{zwaar-gedragen}\\

\haiku{Om half zeven ging '.}{ik nog even wandelen in}{t Leverlaantje}\\

\haiku{Je voelt het pure.}{leven wijd-ademend in}{je lichaam vloeien}\\

\haiku{*** We\^er een wandeling,....}{in het lievelingspaadje}{bij vallenden avond}\\

\haiku{Ik begin mij zoo....}{aan dat vroolijke vrouwtje}{Annie te hechten}\\

\haiku{Maar al dichter en,,.}{dichterbij zonder te zien}{voel ik den krater}\\

\haiku{Alles is hier zoo,,.}{ganschelijk puur en recht en}{kuisch van wezen}\\

\haiku{Het gaan is door een,,.}{vaag grauw duister onzeker}{en mysterieus}\\

\haiku{{\textquoteright} {\textquoteleft}- Neen, heusch niet, je, '.}{weet wel dat ikt van jou}{wel verdragen kan}\\

\haiku{En zal je dan niet?}{v\'e\'el ongelukkiger zijn}{dan v\'o\'or je hier kwam}\\

\haiku{En dat is onrecht,,,,.}{Ru dat is wreed wreed onrecht}{dat kan nooit goed zijn}\\

\haiku{'t Is heelemaal.}{zonder iets onreins geweest}{wat ik voor haar voel}\\

\haiku{of wel, als ik haar, {\textquoteleft}!....}{tegenkom en hoe ze dan}{melodieusD\'ag}\\

\haiku{Rein als een jonge '.}{God rees Ardjoen\r{a} omhoog}{int morgenlicht}\\

\haiku{Als ze op een hoog.}{punt is gekomen staat ze}{stil en wacht op mij}\\

\haiku{Opeens, ergens, een,,.}{open plek waar boven lichte}{blauwe hemel schijnt}\\

\haiku{Ze is een zoete,,....}{rose roos in het groen rank}{oprijzend omhoog}\\

\haiku{Maar nu is alles,....}{weer goed en gewoon en er}{is niets verloren}\\

\haiku{Wild dooreen liggen,.}{groote rotsblokken waar stemmig}{een stroompje ruischt}\\

\haiku{{\textquoteleft}Waar is nu ergens,?}{dat mooie dennelaantje waar}{je zoo over uit was}\\

\haiku{{\textquoteleft}Dank je wel, hoor, ik,....}{ben er h\'e\'el blij me\^e ik zal}{ze goed bewaren}\\

\haiku{Ze gaat w\`eg van je,,,.....{\textquoteright}....}{Rudolf w\`eg w\`eg Maar het lijkt}{te onbestaanbaar}\\

\haiku{Ik voel dat ze mijn,....}{hand drukt en de tranen nu}{langs mijn wangen gaan}\\

\subsection{Uit: Karma}

\haiku{Moeten jullie, die,?}{zooveel  verder zijt dien}{stumpert uitlachen}\\

\haiku{alles wat je in.}{den Tijd rekent is slechts toover}{en begoocheling}\\

\haiku{Ja, dit is wel de}{volheerlijkste liefde die}{bestaan kan tusschen}\\

\haiku{Esdur)  J.S. Bach,.}{stond er boven maar ik}{kon het niet gelooven}\\

\haiku{Het vogeltje rolt,.}{om het kan niet meer precies}{op zijn rug liggen}\\

\haiku{niet dat het blijft is......}{de begoocheling maar dat}{het voorbij zou gaan}\\

\haiku{altijd zal je bij,,......}{mij zijn en ik bij jou al}{weten we het niet}\\

\haiku{Zoo, dan was meneer,, '.}{niet terecht anders had ze}{niett speet haar wel}\\

\haiku{Ja, meneer, weet u.....}{daar staat nu juist een lijkie}{boven de aarde}\\

\haiku{Zij waren elkaar,.}{nu niet meer onverschillig}{zij haatten elkaar}\\

\haiku{al de mis\`ere......}{van het leven komt door het}{willen aanraken}\\

\haiku{De laatste maanden.}{dacht het Moedertje om niets}{dan om het kindje}\\

\haiku{Dokters denken nooit,.}{over het Wonder omdat ze}{veel te veel weten}\\

\haiku{Het was een pracht van,.}{een haan dien ik bij mijn twaalf}{kippen had gekocht}\\

\subsection{Uit: Vlindertje}

\haiku{{\textquoteleft}Je hadt eigenlijk,{\textquoteright}, {\textquoteleft}}{een meisje moeten worden}{zei Ellie altijd}\\

\haiku{Hij was officier.}{geworden omdat Ellie}{het zoo gewild had}\\

\haiku{Hij was dan ook niets,}{grooter dan zij geweest want}{ze was vroeg gegroeid}\\

\haiku{En Ellie was het,,.}{zachtste fijnste teederste wat}{er voor hem bestond}\\

\haiku{{\textquoteright} Ook 's winters was.}{er genoeg te doen voor een}{damemeisje als zij}\\

\haiku{Wat durfde Lize,!}{van Elsmeet een breeden}{grooten hoed dragen}\\

\haiku{Ellie hield van hem,,.}{d\'at was genoeg en daarom}{hield hij hem  hoog}\\

\haiku{{\textquoteright} {\textquoteleft}- Nu, dat zal vooreerst,!}{wel niet gebeuren reken}{d\'a\'ar maar gerust op}\\

\haiku{Denk nu zelf eens de,,,.}{Sandt of den Bergh of Waalen}{die je daar noemde}\\

\haiku{Scheveningen lag.}{in al de glorie van een}{lichten zomerdag}\\

\haiku{Het leven w\`as niet,,!}{leelijk en duister het kon}{niet het kon niet zijn}\\

\haiku{En ze vond zich dan,.}{ook nog maar zoo'n nietig kind}{bij hem den sterke}\\

\haiku{Het leven was hem,}{niets meer waard zonder dat groote}{zalige genot}\\

\haiku{Het jonge mensch moest;}{nu maar eens een geregeld}{leven gaan leiden}\\

\haiku{Dat was al een h\'e\'el,.}{geschikte oplossing vond}{de oude mevrouw}\\

\haiku{Daar werd toch nooit iets,?}{aan veranderd al was ze}{nu ge\"engageerd}\\

\haiku{Ze had hem nu juist,.}{heel erg noodig want er kwam nu}{van alles te doen}\\

\haiku{Niets was veranderd,,,.}{alles stond nog als vroeger}{even te\^er even intiem}\\

\haiku{Ik vind je dan v\'e\'el,.}{te goed voor wien ook v\'e\'el te}{goed voor het Leven}\\

\haiku{{\textquoteright} Hij was op het punt,,!}{om het uit te gillen dat}{hij het k\'ende o}\\

\haiku{Toe, geef me een arm,.}{dan gaan we heel deftig naar}{papa beneden}\\

\haiku{D\'a\'ar was hij al te,.}{ervaren voor en had hij}{te veel me\^egemaakt}\\

\haiku{Zoo heel, heel vreemd was, '.}{het voor hem wat in haar aan}{t gebeuren was}\\

\haiku{Ik heb nu juist zoo'n.}{behoefte om mijn hart eens}{aan je te luchten}\\

\haiku{Je hebt daar het recht,.}{niet toe zoolang je met haar}{ge\"engageerd bent}\\

\haiku{Ik vind dat je een.}{gemeenen streek hebt gedaan}{tegenover Ellie}\\

\haiku{Liever dan haar te.}{besmetten met het vuil van}{de vuile wereld}\\

\haiku{Ik weet heel goed, dat.}{ik geen kerel ben voor een}{meisje als Ellie}\\

\haiku{Dan is het toch in.}{elk geval beter als ze}{nu \'e\'ens wat lijdt}\\

\haiku{de dochtertjes van ....}{fatsoenlijke ouders op}{straat niet meer veilig}\\

\haiku{Een straatjongen bleef,.}{even staan en trok een leelijk}{gezicht tegen haar}\\

\haiku{stil verdwijnen in,,}{het niet in die eindelijk}{weergevonden rust}\\

\haiku{Nu was zij op het,.}{uiterste eind gekomen}{ze kon niet verder}\\

\subsection{Uit: Het zusje}

\haiku{- Ze was heusch al,.}{achttien jaar al zag ze er}{zooveel jonger uit}\\

\haiku{{\textquoteright} En ze zei het hem,,,}{na lief alsof ze met haar}{stem het woord streelde}\\

\haiku{De mooie schitteroogen,,.}{die hem aanzagen waren}{die van een groot kind}\\

\haiku{En met mijn vader,.}{gaat het niet al te best ik}{zie hem bijna nooit}\\

\haiku{Hij was al den hoek,.}{van de straat om toen ze weer}{hard kwam aanloopen}\\

\haiku{Maar \'altijd dieper,,.}{zonk haar hoofd in de zee in}{de zee de groote zee}\\

\haiku{over de sombere,,....}{daken over de duistere}{huizen en verder}\\

\haiku{Ik wil gedenken,.}{een lieve doode zoo diep}{in mij begraven}\\

\haiku{Het is nog maar een,.}{vage tocht door den schijn die}{mij van u w\'eghoudt}\\

\haiku{Ja, het zal \'opgaan,.}{het zal alles \'opgaan in}{onbevlekten staat}\\

\haiku{{\textquoteright} Maar hij was erg blij.}{dat hij haar handjes in de}{zijne kon drukken}\\

\haiku{En hij schaamde zich.}{dat hij anders om dezen}{tijd nog in bed lag}\\

\haiku{'s Avonds om negen,.}{uur wachtte hij haar weer op}{en bracht haar naar huis}\\

\haiku{En den volgenden,,....}{morgen en den volgenden}{avond en trouw z\'o\'o door}\\

\haiku{En dan 's middags!}{het schaatsenrijden op de}{Vijvers in het Bosch}\\

\haiku{Om vier uur ging hij.}{heel zachtjes de school in om}{haar te verrassen}\\

\haiku{Aan het eind, dicht bij,,.}{de kachel zat Mientje het}{hoofd diep gebogen}\\

\haiku{{\textquotedblright}{\textquoteright} {\textquoteleft}- Ja, nu begrijp ik,,{\textquoteright}.}{het nu begrijp ik het pas}{fluisterde zij zacht}\\

\haiku{Ze had het hem nooit,.}{goed kunnen zeggen waarom}{ze dat had gedaan}\\

\haiku{Terwijl zij bezig,.}{was met het zetten zou hij}{wat voor haar spelen}\\

\haiku{Een weldadige,.}{goede vertrouwelijkheid}{lag over de dingen}\\

\haiku{En een broer moet toch........}{altijd verwachten dat zijn}{zuster \'e\'ens gaat}\\

\haiku{Zou zij wel ooit meer?}{van hem kunnen houden dan}{van een grooten broer}\\

\haiku{Ik bedoel dat het,.}{zoo niet altijd kan blijven}{zooals broer en zuster}\\

\haiku{{\textquoteleft}Zeker, Paul, ik vind.}{het allemaal heel mooi zooals}{jij dat nu voorstelt}\\

\haiku{Had het hem niet h\'e\'el,?}{alleen gelaten bijna}{alles wat hij wist}\\

\haiku{Was niet het beste,?}{van alles haar simpele}{hart dat van hem hield}\\

\haiku{Zij was voor hem meer.}{een menschelijke moeder}{dan een hemelsche}\\

\haiku{{\textquoteright} {\textquoteleft}Zeg, Paul, heb je, v\'o\'or,?}{mij wel eens van een ander}{meisje gehouden}\\

\haiku{Maar ik behoef nu.}{ook niet dood te gaan en mag}{wel bij je blijven}\\

\haiku{Hun liefde bleef nu,}{zoo veilig bewaard in hun}{eigen lief geheim}\\

\haiku{Maar nu kwam het, n\'u,.}{kw\'am het hij voelde dat het}{nu beginnen zou}\\

\haiku{{\textquoteright} En ze liepen langs,.}{den Parallelweg naar den}{overstap bij het spoor}\\

\haiku{dat ik je eens zal,.}{geven en dat nu wel graag}{zou willen komen}\\

\haiku{{\textquoteleft}Je kunt toch maar \'e\'en,,.}{ding in mijn oogen lezen Paul}{m\'e\'er staat er niet in}\\

\section{A.L.G. Bosboom-Toussaint, Cd. Busken Huet en Simon Gorter}

\subsection{Uit: Drie vergeten novellen}

\haiku{ik dank u.{\textquoteright} {\textquoteleft}Dan zal.}{ik de vrijheid nemen mij}{te verwijderen}\\

\haiku{{\textquoteright} {\textquoteleft}Omdat mijnheer De.}{Choiseul hem een gezantschap}{opdroeg naar Madrid}\\

\haiku{Gij verplettert uw,!}{chignon en de guirlande}{die er tegen rust}\\

\haiku{Ieder der boeren;}{van het dorp bracht daartoe iets}{van het zijne bij}\\

\haiku{Zij ook gaf hem veel,,.}{maar als eene gift die haar met}{woeker werd beloond}\\

\haiku{want ik wist niet eens,.}{werwaarts mijne moeder mij}{met zich heenvoerde}\\

\haiku{Ik ben de Graaf de,.}{Forbin en gij zult van nu}{aan mijn naam voeren}\\

\haiku{maar verneem gij dan,,....}{uit mijn naam wie het zijn mag}{die gekomen is}\\

\haiku{Geluid en stem, zijn,;}{\'e\'enigst herkenningsmiddel}{had hem getroffen}\\

\haiku{Met het ernstige,....}{plan om er aan te voldoen}{kwam ik naar Parijs}\\

\haiku{De meeste hunner;}{bekenden hadden voor dit}{hoogere geen oog}\\

\haiku{zes dagen zal ik,.}{strijden en den zevenden}{dag ter ruste gaan}\\

\haiku{Dokter Egbert vond;}{het niet streelend alzoo te}{worden afgescheept}\\

\haiku{Julia was op nieuw.}{gaan zitten en de Bijbel}{lag op haar knie\"en}\\

\haiku{Die mijn vleesch eet,.}{en mijn bloed drinkt die blijft in}{mij en ik in hem}\\

\haiku{{\textquoteright} zoover te brengen, dat:}{de dienstmaagd onder buien}{van lachen uitroept}\\

\haiku{Dien volgenden dag.}{was er theevisite bij}{de juffrouwen Zeulig}\\

\haiku{Denk je dat ik na,?}{andermans honden kijk of}{die nat zijn of droog}\\

\haiku{De vijf minuten.}{van hare deftigheid zijn}{bovendien ook om}\\

\haiku{{\textquoteleft}we waren niet in,.}{de wereld om te praten}{maar om wat te doen}\\

\haiku{of 't heusch waar,?}{zou wezen dat de menschen}{nog een ziel hadden}\\

\haiku{- Nou Koos, - had een van -?}{de vriendinnen gezegd hoe}{heb ik het met je}\\

\haiku{ik wou, dat ze ons,.}{leerden denken leerden \'e\'en}{ding goed te weten}\\

\haiku{{\textquoteright} - zei tante, die zich, {\textquoteleft}?}{heel dom kon houden als ze}{verkooswat zou dat}\\

\haiku{De juiste sterfdag.}{van onze lieveling is}{nooit bekend geweest}\\

\section{A.L.G. Bosboom-Toussaint}

\subsection{Uit: Engelschen te Rome. Romantische \'episode uit de regering van paus Sixtus V}

\haiku{, en zoo het in de,.....}{macht eener vreemdelinge}{stond u daarvoor te}\\

\haiku{De klok van eene der.}{naburige kapellen}{sloeg de tweede ure4}\\

\haiku{Eindelijk riep hij,,:}{uit op eenen toon die bijna}{zegepralend was}\\

\haiku{en geen bezoek van;}{den geliefde stelde haar}{gekrenkt hart gerust}\\

\haiku{o! vergeef, gij weet,.}{niet wat een minnaar lijdt die}{zich verbergen moet}\\

\haiku{{\textquoteright} {\textquoteleft}Gij moet inderdaad!}{veel geleden hebben met}{zulke denkbeelden}\\

\haiku{ik hoor den tred van,.}{Signora Respanti die door}{de galerij komt}\\

\haiku{Zijnen weldoener.}{mocht hij niet opofferen}{aan zijne liefde}\\

\haiku{de handelwijze,.}{van dien jongen man had iets}{dat haar ontrustte}\\

\haiku{Van de Prelaten,.}{klimt men tot de Gezanten}{de Roomsche Prinsen}\\

\haiku{{\textquoteright} riep de jongeling,.}{zich telkens tot hoogere}{drift opwindende}\\

\haiku{Reinier, schooner was de,.}{Moedermaagd niet dan zij zich}{huichelend voordoet}\\

\haiku{maar ziet gij iets in,....}{mij dat u eenen bepaalden}{afkeer inboezemt}\\

\haiku{En gesteld nu, dat,,!}{ik het konde dat ik het}{wilde Signora}\\

\haiku{Men prees uwe goedheid,,,.}{men zeide mij dat gij jong}{waart dat gij schoon waart}\\

\haiku{met eene ernstige;}{zwijgende buiging bleef hij}{op eenen afstand staan}\\

\haiku{zijn stem schoot te kort,,.}{om hoorbaar te vragen wat}{men van hem wilde}\\

\haiku{Gelukkige, die;}{eenmaal in de Eeuwige}{Stad geademd hebt}\\

\haiku{Ik visch somtijds in,.}{het kanaaltje dat links af}{den wijngaard omgeeft}\\

\haiku{Het bootje van den.}{opzichter der tuinen is}{altijd tot mijn dienst}\\

\haiku{Gij zult het weten,,;}{dat ik diep voel hoeveel ik}{heb goed te maken}\\

\haiku{Zijne Heiligheid.}{is snel in het uitvoeren}{eener bedreiging}\\

\haiku{heden echter vond,.}{hij het geraden zich te}{laten aandienen}\\

\haiku{Dat woord, hoe zacht ook,;}{uitgesproken klonk eensklaps}{als door eene echo rond}\\

\haiku{{\textquoteright} sprak hij ernstig, {\textquoteleft}uwen.}{jongen neef zoo onverhoeds}{te verpletteren}\\

\haiku{{\textquoteright} Inmiddels trachtte.}{Paolo den vreemdeling}{bij te brengen}\\

\haiku{{\textquoteright} {\textquoteleft}De grootouders van;}{Francesco zorgden voor}{mijne opvoeding}\\

\haiku{het purper had plaats,;}{gemaakt voor het violet}{de rouwkleur der kerk}\\

\haiku{al die monden, die,:}{schijnen te willen spreken}{en die toch zwijgen}\\

\haiku{ik spiegelde mij,....}{in het blauwend kristal gij}{wenddet het niet af}\\

\haiku{mijne gloeiende....}{lippen in aanraking met}{die zachte warmte}\\

\haiku{Hij vreesde daarom,.}{dat Anna veel zoude te}{ontdekken hebben}\\

\haiku{ook lispelde zij,,.}{bijna onhoorbaar woorden}{zonder samenhang}\\

\haiku{{\textquoteleft}Ik zal niet kunnen.}{terugkeeren voor den zondag}{van Quasimodo}\\

\haiku{Men zegt, dat Zijne;}{Eminentie een wild Arabisch}{ros heeft bereden}\\

\haiku{Scipione had.}{uit voorzichtigheid dien eed}{van haar afgeperst}\\

\haiku{Nog smeek ik u uwe,.}{hand af nog bied ik u den}{steun van mijnen arm}\\

\haiku{Nog bied ik u een,;}{leven van zorgelooze rust}{en kalm genoegen}\\

\haiku{{\textquoteright} {\textquoteleft}Gij hebt dan beslist,{\textquoteright},.}{hernam hij en wischte zich}{den laatsten traan af}\\

\haiku{Het was tegelijk.}{gericht tegen den Paus en}{tegen diens zuster}\\

\haiku{hij wist gehoorzaamd,.}{te zijn zoodra men hem}{niet meer tegensprak}\\

\haiku{En daar waren er,.}{velen die er belang bij}{meenden te hebben}\\

\haiku{De Graaf bezocht dan.}{den jongen Kardinaal en}{werd goed ontvangen}\\

\haiku{En de steek van zulk?}{een insect ontrust u en}{brengt u tot wanhoop}\\

\haiku{Beschik over al het,.}{mijne handel daarin zooals}{gij goed zult vinden}\\

\haiku{Laat ons hun toonen,.}{beter te zijn dan hunne}{lage verdenking}\\

\haiku{{\textquoteright} {\textquoteleft}Ik geloof, dat wij.}{die vrouw nog vooraf zullen}{moeten gebruiken}\\

\haiku{Wat ging het haar aan,?}{dat de Heer van Rome het}{anders beschikt had}\\

\haiku{Zij bevorderde;}{gaarne al de belangen}{der Spaansche ligue}\\

\haiku{deze  wraak is,.}{niet die van den mensch zij is}{die van den duivel}\\

\haiku{{\textquoteleft}Gij vindt eenen sleutel,;}{onder de tafel die opent}{u straks dit vertrek}\\

\haiku{{\textquoteright} Na deze woorden,.}{boog hij zich alsof hij niets}{meer te zeggen had}\\

\haiku{die zich grijsaard had,.}{gemaakt naar het hart in de}{volle kracht der jeugd}\\

\haiku{Hij scheen nu zonder,.}{kunst zoo oud als hij vroeger}{had willen schijnen}\\

\haiku{Hij ziet eenen man, die,.}{met drift toesnelt misschien kan}{het een helper zijn}\\

\haiku{Het zal eene laatste,;}{samenkomst zijn een afscheid}{voor deze aarde}\\

\haiku{Castagna speelde;}{verlegen met het gouden}{kruis op zijne borst}\\

\haiku{het geldt hier niet de:}{goed- of afkeuring}{van een Conclave}\\

\haiku{zij mocht zich eenmaal,.}{luide verheffen en het}{kon dan te laat zijn}\\

\haiku{{\textquoteright} zeide hem Sixtus,.}{en leunde het hoofd tegen}{zijne trouwe borst}\\

\haiku{{\textquoteright} {\textquoteleft}Heb dan moed tegen,.}{u zelven wees uwen grooten}{bloedverwant waardig}\\

\haiku{- daar alleen is de;}{veilige wijkplaats voor een}{afgestreden hart}\\

\haiku{Dat mijn William.}{aan mij denke als aan eene}{afgestorvene}\\

\haiku{Zij had altijd stil,;}{voor zich henen geleefd en}{niemand kende haar}\\

\haiku{Hij rekende op,;}{de versche troepen die hem}{waren toegevoegd}\\

\haiku{hij moest iets anders,;}{zijn geweest dan een mensch veel}{meer of veel minder}\\

\haiku{Ik, die nog jaren,!}{van kracht aan Veneti\"e had}{te geven misschien}\\

\haiku{uwe patrici\"ers,;}{hebben geen hart en uw volk}{heeft geen geheugen}\\

\haiku{De patrici\"ers;}{hadden te rekenen met}{hunne kli\"enten}\\

\haiku{de Senaat moest hun.}{de vrijheid van Pisani}{hebben toegezegd}\\

\haiku{Het moet immers den?}{martelaar niet zien waar het}{den admiraal vraagt}\\

\haiku{de grijze held moest.}{zich weer oefenen in het}{loopen als een kind}\\

\haiku{stouter in hare,;}{eischen naarmate zij die}{zag ingewilligd}\\

\haiku{met Rome den rug,;}{toe te keeren is men nog geen}{Christen geworden}\\

\haiku{Zijne kleeding was die,.}{van zijn rang en hij had den}{degen op zijde}\\

\subsection{Uit: De vrouwen van het Leycestersche tijdvak. Deel 2}

\haiku{De eerste overtrof,.}{zelf zijne stoutste hoop als}{wij hooren zullen}\\

\haiku{zullen we van die....?}{twee verliefden met der haast}{een verloofd paar zien}\\

\haiku{{\textquoteright} {\textquoteleft}In uw geval had,{\textquoteright}.}{ik mijn schoonzoon reeds omhelsd}{sprak Kiligrew zacht}\\

\haiku{zoo gij haar ernstig,,!}{meent zoo ik haar u geve}{maak haar gelukkig}\\

\haiku{{\textquoteright} {\textquoteleft}Zoo neem dan die hand,;}{die mijne zachte Ada u}{willig geven zal}\\

\haiku{{\textquoteright} {\textquoteleft}En hebben wij dan?}{hun groeten en glimlachen}{af te bedelen}\\

\haiku{{\textquoteright} {\textquoteleft}Die reuckeloosheid ',;}{ist juist welke ik u}{te verwijten heb}\\

\haiku{Zal het van hem ook,....?}{gezegd worden dat hij tot}{de Reingoudisten hoort}\\

\haiku{Mijn gehoor zal mij,.}{misleid hebben gelijk ik}{mijne oogen wantrouw}\\

\haiku{maar ik acht, dan zal....}{hij bevel geven tot een}{heimelijken dood}\\

\haiku{{\textquoteleft}aan die zijde houdt,.}{men volgaarne mits men niet}{behoeft te geven}\\

\haiku{ik zal vaststaan in '{\textquoteright}.}{t gedenken aan u. En}{hij ging van haar heen}\\

\haiku{maar dat spreekt vanzelf,{\textquoteright}, {\textquoteleft} '?}{riepen allen uit \'e\'en mond}{wie zout durven}\\

\haiku{{\textquoteleft}ik ook zou mij wel.}{genoeg geveiligd achten}{met zijne hulpe}\\

\haiku{gij zult niet gaan, om.}{ons aan uw Engelschman}{over te leveren}\\

\haiku{allen dus dankten.}{of verontschuldigden zich}{op eenige wijze}\\

\haiku{Reeds voelde, reeds zag,;}{het Ada welhaast ook zouden}{anderen het zien}\\

\haiku{Daarna, begrijpt ge,....}{zou zijne tuigenis voor}{u niet veel gelden}\\

\haiku{Genoeg voor ons, om,:}{u te bevestigen in}{hetgeen gij vermoedt}\\

\haiku{zij voelde afschuw.}{van zich zelve onder dit}{akelige masker}\\

\haiku{In dit kisje vindt,;}{ge goud tot omkooping van}{wie gij winnen wilt}\\

\haiku{maar, terwijl hij de,:}{trap opsteeg mompelde hij}{in het Italiaansch}\\

\haiku{want gij weet niet, hoe,....;}{men dien man haat  als men}{hem eens haten kan}\\

\haiku{alleen, hoe kon zij,?}{plaats vinden en vooral hoe}{kon zij stand houden}\\

\haiku{{\textquoteright} vroeg Barneveld, die.}{verlangde op het hoofdpunt}{terug te komen}\\

\haiku{{\textquoteright} {\textquoteleft}Dat is zekerlijk,!}{minst te betreuren door u}{heer van Barneveld}\\

\haiku{{\textquoteright} {\textquoteleft}Wie kan de mate,?}{der schuld berekenen waar}{men zulk lijden ziet}\\

\haiku{{\textquoteleft}Ik heb onder eed,{\textquoteright}.}{beloofd dat niet te zeggen}{hernam de krijgsman}\\

\haiku{een edelman met twee;}{dienaren overmocht toch wel}{een paar boosdoeners}\\

\haiku{van het weerzien van....}{Fabian zou ik veel voor}{haar durven hopen}\\

\haiku{ik heb er velen,,.}{als gij weet maar toch is dit}{eene die ik niet ken}\\

\haiku{Lady Margaret,.....}{was omdat ik zweeg en hun}{niet wilde zeggen}\\

\haiku{met mij, en tot het,,.}{doel dat gij weet begeef ik}{mij met haar van hier}\\

\haiku{sinds Gideon als,.}{leeraar te Utrecht staat is}{hij half uw hart kwijt}\\

\haiku{{\textquoteleft}Sinds gij mij toch niet,,!}{ten antwoord kunt staan ga ik}{u laten Elbert}\\

\haiku{{\textquoteright} {\textquoteleft}Ik ben als altijd ',!}{aan de zijde vant recht}{heer van Barneveld}\\

\haiku{Maar mijns bedunkens.}{is het nu bovenal een}{tijd van handelen}\\

\haiku{{\textquoteleft}maar het is soms van.....}{ongemeen groot nut en vrij}{minder gevaarlijk}\\

\haiku{{\textquoteright} {\textquoteleft}Voor de uitersten,{\textquoteright}.}{hoede ons de Alwijsheid}{sprak Leoninus}\\

\haiku{want schoon ik niets kwaad,....}{van hem wete veeleer niets}{dan goeds van hem zie}\\

\haiku{nu we niet meer in,.}{den Spaanschen tijd zijn kunnen}{we de Kenau's missen}\\

\haiku{{\textquoteright} zei Gideon, {\textquoteleft}gij,.}{zelf moogt oordeelen hoe u}{dit passen zoude}\\

\haiku{heb voor 't minst het.}{geduld te luisteren en}{u te overtuigen}\\

\haiku{maar dit bewijs van,.}{aandacht scheen haar genoeg te}{zijn om voort te gaan}\\

\haiku{zij scheen zelfs het leed,,;}{waarover zij geklaagd had niet}{meer te gevoelen}\\

\haiku{maar haar trotsch gemoed.}{bleek onvatbaar voor zulke}{gewaarwordingen}\\

\haiku{hij wendde den blik.}{af van de Prinses en sprak}{bijna gebiedend}\\

\haiku{{\textquoteright} {\textquoteleft}Blijf dan, maar onder,.}{conditie gij zult mij uwe}{kwelling meedeelen}\\

\haiku{ik kan niet anders,,,.}{dan blijven waar gij zijt dan}{volgen waar gij gaat}\\

\haiku{Daartoe was echter,.}{een voorwendsel noodig en dat}{voorwendsel bestond}\\

\haiku{want men moet billijk,.}{zijn op de eerste achtte}{Modet niet het meest}\\

\haiku{Ik ben geen vriend van,{\textquoteright}.}{Nieuwenaar hernam Norrits}{barsch en spijtig}\\

\haiku{{\textquoteright} {\textquoteleft}Onder de leiding,!}{van den goeden Keurvorst van}{Keulen mevrouwe}\\

\haiku{.... ik kan toch voor een....}{dienaar der Kerke kwalijk}{mijne deur sluiten}\\

\haiku{en of hij van hen, {\textquotedblleft}.}{was diede goede cause}{wilden vorderen}\\

\haiku{Zij weet zelve niet,,.}{hoezeer zij daarmede waar}{zegt dat zij u schaadt}\\

\haiku{Aan uw vromen vriend,.}{heb ik beloofd het leven}{getroost te dragen}\\

\section{F.R. Boschvogel}

\subsection{Uit: Niet wanhopen Maria Christina}

\haiku{Zijn ogen zitten hard.}{en koel achter de glazen}{van zijn gouden bril}\\

\haiku{die jongens van de.}{chirurgijn hebben te veel}{pit achter de oren}\\

\haiku{Gedurende de.}{vijfde partij gaat het nog}{makkelijker}\\

\haiku{Omdat hij beseft.}{dat hij het tegen hem zal}{moeten afleggen}\\

\haiku{Zijn linnenhandel,.}{schijnt weer op te brengen hij}{ziet op geen gulden}\\

\haiku{Om u is het dat.}{die twee loeders v\'o\'or uw deur}{gevochten hebben}\\

\haiku{En dat hij zijn mes,.}{trok om mij te lijf te gaan}{is zeer menselijk}\\

\haiku{Het eten van Andries.}{kan ze in de hete as}{warm en gaar houden}\\

\haiku{Hij had maar dit kind,.}{en een kind dat met alle}{gaven scheen bedacht}\\

\haiku{Ik ken Andries, en.}{laat jij de afgunstige}{luidjes maar kletsen}\\

\haiku{Hij heeft al een uur,.}{wakker gelegen wachtend}{op de dageraad}\\

\haiku{Ze zei het laatst, maar.}{je kon horen hoe het haar}{ten opperste lag}\\

\haiku{Hij zit met zijn ogen.}{in de wolken als het over}{de kinderen gaat}\\

\haiku{- Zo zal het wel zijn {\textquoteleft}{\textquoteright}.}{in het vaderland van je}{nieuwe gedachten}\\

\haiku{- Zeker, Dumouriez.}{staat met een leger klaar om}{ons te bevrijden}\\

\haiku{Lieven zit met een.}{smalende snuit strak naar een}{graanzak te kijken}\\

\haiku{Als de jongens groot,.}{worden kan een mens zich aan}{alles verwachten}\\

\haiku{En er is ook iets.}{van het dovende avondrood}{dat in haar ogen gloeit}\\

\haiku{Ze zet een stap naar.}{de tafel en grijpt het boek}{uit Lievens handen}\\

\haiku{Zij is al buiten,,.}{naar de stal of de schuur of}{naar de werkwinkel}\\

\haiku{Met vier glanzende, '.}{paarden wie jet vuur uit}{de lijven kunt slaan}\\

\haiku{Je krijgt kippenvel.}{als je er aan denkt wat er}{in Frankrijk gebeurt}\\

\haiku{Het felle gestraal.}{van haar vurige blik werkt}{op zijn mondhoeken}\\

\haiku{Kom, kom, hoe durft ze?}{de zaken dadelijk zulk}{aangezicht geven}\\

\haiku{Misschien vinden ze.}{binnenkort wel een eigen}{huis op Aartrijke}\\

\haiku{Herinneringen.}{die haar anders een kelk vol}{alsem en edik zijn}\\

\haiku{- Dat kunnen praatjes,,.}{zijn praatjes uit verdachte}{clubs bijt Andries af}\\

\haiku{Zijn handen liggen...}{klaar om recht te veren en}{dit verloren brood}\\

\haiku{Maar laat hij verder.}{zijn gezag uitoefenen}{als naar gewoonte}\\

\haiku{Het leven is al,.}{kort genoeg en zij is in}{weelde opgebracht}\\

\haiku{Het is een vrouw, die.}{je kan vertrouwen als een}{eigen  moeder}\\

\haiku{Hij staat op, knielt v\'o\'or.}{de Kruislievenheer en zegt}{een kort avondgebed}\\

\haiku{Maar hij stokt in zijn,.}{pleidooi die welbespraakte}{monsieur l'abb\'e}\\

\haiku{Ze durft niets anders.}{dan even te knikken als hij}{iets tegen haar zegt}\\

\haiku{- terug meedanst in,.}{de ronde moeten ze haar}{een poos meesleuren}\\

\haiku{Moet ik hem soms geen?}{vergiffenis vragen op}{mijn blote knie\"en}\\

\haiku{Op de drempel van.}{de grote vasten gaat zij}{gewoonlijk te biecht}\\

\haiku{Er is niet veel meer.}{nodig om hem helemaal}{te doen wegblijven}\\

\haiku{Die zitten nu een.}{carmagnole te fluiten}{achter de tralies}\\

\haiku{Een jongen, die zijn.}{hoofd liet op hol brengen door}{de slechte boeken}\\

\haiku{In een smalle gang.}{vol stof en zure lucht maakt}{zij haar opwachting}\\

\haiku{- haar beurt gekomen,.}{is voelt ze de slag van haar}{hart tot in haar hoofd}\\

\haiku{Of hij het van de,.}{eerste lettergreep niet beet}{had waarover het ging}\\

\haiku{Eerst eens zien wat er... -}{bekend is over dat lieve}{kind van u. Lieven}\\

\haiku{Ze kijkt op naar het,.}{hoge venster waarin de}{halletoren staat}\\

\haiku{En ontdekt meteen.}{tal van zaakjes die moeten}{opgeknapt worden}\\

\haiku{Kom, ze zullen de.}{kommode een einde van}{de muur wegslepen}\\

\haiku{- Het was niet moeilijk,.}{om bezigheid voor hem te}{vinden kapelaan}\\

\haiku{En het huis van een.}{kunstschilder tonen ze haar}{in een stille straat}\\

\haiku{Maarten legt het boek.}{v\'o\'or haar op tafel en keert}{traag de bladen om}\\

\haiku{Ze keert gejaagd de,,}{blaadjes van een album zoekt}{naar een handwerkje}\\

\haiku{- De Pruisen en de.}{Keizerlijken hebben hun}{de oorlog verklaard}\\

\haiku{Dit dorpse bier zal.}{de hoofdman van Aartrijke}{wel zelf betalen}\\

\haiku{De lach is het van,.}{de commissaire en van}{zijn secr\'etaire}\\

\haiku{Nu eerst, omdat het.}{de eerste keer is dat je}{moeder nodig hebt}\\

\haiku{En dan ja tegen.}{het ene en neen tegen het}{andere meisje}\\

\haiku{Ze  voelt dat hij,.}{niet stevig aanpakt dat hij}{in het duister tast}\\

\haiku{Nicht Isabella houdt.}{een catechismus- en}{letterschooltje open}\\

\haiku{Maar de twee paarden.}{van seigneur Jean heb ik niet}{teruggekregen}\\

\haiku{Hoe ver de nieuwe.}{gedachten over de wereld}{zijn doorgedrongen}\\

\haiku{- Dus loop je weer met,?}{dat vreemde volk mee tegen}{ons en allen in}\\

\haiku{Here Jezus, als.}{het monsieur l'abb\'e eens}{moet ter ore komen}\\

\haiku{Na de vespers van.}{tweede Paasdag gaat Anna}{naar het molenhuis}\\

\haiku{- Ge neemt de woorden,.}{van mijn lippen ma m\`ere}{sup\'erieure}\\

\haiku{Deze jaren zijn,.}{voorbij maar het jaagt een schoon}{geluk door haar ziel}\\

\haiku{Ze wil de jente.}{meiboom niet zien wuiven op}{de halletoren}\\

\haiku{De dood die dit land,:}{alom bespringt met vier vijf}{gesels tegelijk}\\

\haiku{De heren der kerk.}{kunnen lelijke dingen}{zeggen over haar zoon}\\

\haiku{Het is of hij op,,.}{haar woorden wacht op die vraag}{die haar achtervolgt}\\

\haiku{Een nieuwe gril, een.}{korte pennetrek en het}{systeem stort ineen}\\

\haiku{Hij heeft iets van de,.}{oude hoofdman over zich want}{ze vragen hem raad}\\

\haiku{Heel het dorp spreekt er,.}{schande over dat ze Andries}{hebben meegesleept}\\

\haiku{Ze is er weinig.}{op gesteld Andries tussen}{vier ogen te spreken}\\

\haiku{Andries zit grauw en.}{stom naar de toppen van zijn}{laarzen te kijken}\\

\haiku{Zeg haar hoezeer ik.}{verlang om bij  haar te}{komen inwonen}\\

\haiku{Maarten ziet hoe haar,.}{onderlip spant hoe ze vecht}{met haar ontroering}\\

\haiku{- Een paasweertje voor,!}{de Laetare-zondag}{Maarten Engelbrecht}\\

\haiku{Tot hiertoe was hem.}{alles zo onverschillig}{als de straatstenen}\\

\haiku{Hij zou die vreemde,.}{pastoor kunnen hangen zo}{kwaad is hij op hem}\\

\haiku{V\'o\'or de pastorij,.}{kruist hij monsieur l'abb\'e}{die naar de kerk gaat}\\

\haiku{Monsieur l'abb\'e.}{zit nog afwezig op het}{slagveld te kijken}\\

\haiku{Ze zullen zoveel.}{borreltjes drinken als de}{pastoors glazen wijn}\\

\haiku{Daar zijn gelukkig -,,!}{nog anderen kom ik schenk}{nog eens in Lowie}\\

\haiku{Ze zeggen dat het,.}{een Walekop is maar dat is}{een barre leugen}\\

\haiku{Zij kan alleen met.}{monsieur l'abb\'e praten}{in het tuinhuisje}\\

\haiku{De boerin, die een,.}{invroom vrouwmens is kan het}{niet langer dulden}\\

\haiku{Deze gewiekste.}{mannen halen het sluwste}{wild uit zijn schuilhoek}\\

\haiku{PASTOOR BOUCKAERT ZIT.}{NU AL DAGEN lang alleen}{in zijn pastorij}\\

\haiku{De pastoor knikt, doet,}{zoiets als wuiven met zijn}{moede moede hand.}\\

\haiku{De kerk gonst als een.}{bijenkorf van het gezang}{en het orgelspel}\\

\haiku{De kapelaan in.}{zijn boerenpak wipt op het}{paard en rijdt vooruit}\\

\haiku{Maar die knecht, mijne - -!}{mens de rest wordt gefluisterd}{dat was de pastoor}\\

\haiku{En te Torhout zit.}{er een verborgen in een}{hokje van de schouw}\\

\haiku{En de stille mis.}{bij het vroege schemeruur}{in de opkamer}\\

\haiku{Hij gaat met Adriaan '.}{naar Leopold Lievens op}{t Kanunnikse}\\

\haiku{Hij loopt de wolf in,.}{de muil want hij weet niets van}{die gendarmen af}\\

\haiku{Toen ze rakkerden.}{op de molenwal kon hij}{deze trap niet op}\\

\haiku{een onvolkomen,,.}{jongen te week en te bleek}{te meisjesachtig}\\

\haiku{Hij gaat de weg, die.}{honderden priesters gaan in}{dit verdrukte land}\\

\haiku{Ook zij vluchten de,.}{winterse kouters over de}{hemel weet waarheen}\\

\haiku{We hebben hem met.}{een vriendelijk schotje naar}{beneden gehaald}\\

\haiku{Want zij is een vrouw.}{die ze met hun laatste louis}{zouden betalen}\\

\haiku{Luister, als je 't, '.}{meent met je gevoelens moet}{jet bewijzen}\\

\haiku{Een der vrolijksten.}{onder hen is luitenant}{de la Touraine}\\

\haiku{Zij voelt hoe ze v\'o\'or.}{haar in aanbidding liggen}{nedergebogen}\\

\haiku{Ze heeft niet veel tijd.}{om zich zorgen te maken}{over Paul-Marie}\\

\haiku{Liefst van al zou ze,.}{haar op staande voet de deur}{wijzen hard en kort}\\

\haiku{Je kan het zien hoe.}{ze dit kind aan anderen}{heeft overgelaten}\\

\haiku{- De andere is,.}{niet meer teruggekeerd de}{la Touraine wel}\\

\haiku{- Het slechte wens ik,,.}{niet Maria-Christina}{alleen het goede}\\

\haiku{Een mens is er, aan.}{wie ik hen onbevangen}{durf toevertrouwen}\\

\haiku{Vader laat je heer,.}{en meester moeder is in}{alles steengerust}\\

\haiku{De nieuwjaarders, die,:}{Maarten v\'o\'or het raam zien staan}{knikken en wuiven}\\

\haiku{De hand van joffer.}{Serafina is vaardig}{in het briefschrijven}\\

\haiku{- Dood of levend, de,.}{\'emigr\'ee moeten we hebben}{zegt de luitenant}\\

\haiku{De grote piekhond,,.}{die onder de tafel ligt}{te soezen springt recht}\\

\haiku{El\'eonore, die zich een,.}{ogenblik schuil hield achter de}{deur is al buiten}\\

\haiku{En nog kruiven de '.}{krollen hoog en weerbarstig}{haar bovent hoofd}\\

\haiku{Als de mensen het,:}{horen zullen ze het hoofd}{schudden en zuchten}\\

\haiku{Hij ziet hoe Anna.}{nog eenmaal het kind in haar}{armen neemt en zoent}\\

\subsection{Uit: Zandstuivers}

\haiku{- 't Wordt hier stille,,.}{zeiden de menschen Steven}{is er ook van door}\\

\haiku{Ze gingen eten in.}{een restaurant en huurden}{een appartement}\\

\haiku{Ik zal  den zak.}{over mijn schoere smijten en}{naar de beeten gaan}\\

\haiku{Hij stond recht, en ging.}{traag en zwaar terug van waar}{hij gekomen was}\\

\haiku{Op 't hof zaten.}{ze te vergeefs naar Steven}{en Door te wachten}\\

\haiku{Over een week moest het.}{toegewezen worden aan}{een nieuwen pachter}\\

\haiku{Wegens de lage.}{huishuur heeft zoo'n dischhuisje}{altijd veel aantrek}\\

\haiku{De pastoor was geen.}{klein beetje preutsch met zulke}{parochianen}\\

\haiku{- Zeg op, want ik ben,.}{haastig zei de brouwer een}{beetje gekitteld}\\

\haiku{'s Anderen daags.}{echter ging de brouwer reeds}{naar Steven Dagraad}\\

\haiku{Heel de parochie.}{wachtte op madame van}{den secretaris}\\

\haiku{Ze meenden dat ik,.}{het niet hoorde maar een mensch}{hoort scherp aan dien kant}\\

\haiku{Maar we zullen 't.}{beleven dat hij nog een}{echte sinjeur wordt}\\

\haiku{Blondientje was nog,.}{geen acht jaar oud toen ze ook}{eens zoek geraakt was}\\

\haiku{Hij droeg Blondientje,,,.}{zijn zusje op zijn arm als}{een groote schoone pop}\\

\haiku{Ach, en maar \'e\'en kind,. '}{hebben en er zoovele}{mee tegen komen}\\

\haiku{Met eindeloos veel.}{deernis in dien zucht voor haar}{dochter Blondientje}\\

\haiku{- Och, mijnheer Adriaan,, '.}{maar \'e\'en kind hebben hoe zou}{t anders kunnen}\\

\haiku{- Boerin Sinnaeve,, '.}{zei hij voor de belooning}{moet get niet doen}\\

\haiku{Ze zeggen dat Ons.}{Heere u beloond heeft met}{dat schoon Blondientje}\\

\haiku{'s Namiddags trok.}{ze naar het kapelletje}{op den Koudepuid}\\

\haiku{Het is de eenigste.}{keer dat Adriaantje gezoend}{werd in zijn leven}\\

\haiku{Want ze hebben bij.}{ons al mizerie genoeg}{met de varkens}\\

\haiku{Ik ben er eens tot.}{op den drempel geweest met}{mijn schoolkameraad}\\

\haiku{Ze had het gehoord:}{hoe bot en bietebauwig}{de menschen zeiden}\\

\haiku{Maar een engel uit,,.}{Gods hoogen hemel dat was ze}{zoo goed en zoo schoon}\\

\haiku{Van Ieften was het,.}{een al te vlug dalen een}{sprong in den afgrond}\\

\haiku{Jaarlijksche bolling ',!}{int Boldershof om een}{panne hoofdvleesch}\\

\haiku{Mientje had haar deel.}{gekregen en moest het er}{nu maar mee stellen}\\

\haiku{Mientje was echter.}{de laatste om ook maar \'e\'en}{klachtje te uiten}\\

\haiku{En daarom was het.}{goed en gezellig voor hem}{in den werkwinkel}\\

\haiku{De pastoor gaf haar.}{vlug de absolutie en}{het heilig oliesel}\\

\haiku{Hij kreunde zacht, toen,.}{hij opstond een groote hoop was}{in hem gebroken}\\

\haiku{Over een paar weken,.}{zijt ge hier terug troostte}{kozijn notaris}\\

\haiku{Maar wij wisten niet,.}{dat zij ooit  zoo'n lieven}{naam gedragen had}\\

\haiku{Ik heb in heel mijn.}{bewaarschooljaren nooit meer}{geschrokken als toen}\\

\haiku{De vensterluiken,.}{zijn dicht en het is angstig}{stil om dit huisje}\\

\haiku{Korte djakke j - - -!}{appeltje trappeltje a}{twee beentjes n jan}\\

\haiku{De oudste wijven '.}{vant gehucht hebben ze}{nooit weten bouwen}\\

\haiku{De inspecteur hield}{de kleuters zoo lang bezig}{tot Zuster Marie}\\

\haiku{En ons jongste heeft.}{deze week vier keeren in de}{stuipen gelegen}\\

\haiku{Achter haar krommen.}{rug staat het golvende land}{in rooden zonnegloed}\\

\haiku{De meisjes sliepen.}{in \'e\'en bed en makkerden}{uitmuntend samen}\\

\haiku{s Morgens  was.}{haar programma tot in de}{puntjes opgemaakt}\\

\haiku{Als het zoo ver was,.}{kwam de barones naar de}{Godelieveschool}\\

\haiku{Geen week nadien hield.}{het koetsje weer stil v\'o\'or de}{Godelieveschool}\\

\section{Anthony Bosman}

\subsection{Uit: Witte zwanen zwarte zwanen}

\haiku{De boeren staken.}{bun spade in de grond en}{leunden er zwaar op}\\

\haiku{Trouwens, men kan slechts.}{geluk brengen indien men}{zelf gelukkig is}\\

\haiku{Het was alsof mijn,.}{spelen geen zin meer had nu}{zij was weggegaan}\\

\haiku{- Ach, zei ze, - ik heb.}{vele namen en elke}{naam is een verhaal}\\

\haiku{We gingen een trap,}{op en dan langs een brede}{gang met veel deuren}\\

\haiku{- Maar misschien is het,.}{beter dat je alleen het}{goede van mij ziet}\\

\haiku{Het kan lang duren,,.}{of kort speelman maar eens zul}{je mij verlaten}\\

\haiku{Veel was ik in de,.}{tuin omdat ik dan het dichtst}{bij de landweg was}\\

\section{Nanne Bosma}

\subsection{Uit: De emigrant}

\haiku{{\textquoteright} Onder luid gehoon.}{van de stakende slepers}{verdween de Duitser}\\

\haiku{Ze zei het niet, ze,}{had het nooit gezegd al die}{maanden niet maar diep}\\

\haiku{Pas goed op jezelf,,{\textquoteright}, {\textquoteleft}.}{Abe fluisterde zeook als}{ik er niet meer ben}\\

\haiku{Na deze laatste.}{omhelzing haastte Aukje}{zich zwijgend van boord}\\

\haiku{Hij wilde meteen.}{het raampje openen om uit}{de trein te hangen}\\

\haiku{{\textquoteleft}Houd die jongen toch,,.}{binnen Abe zo meteen valt}{hij nog uit de trein}\\

\haiku{{\textquoteleft}We moeten maar iets,.}{doen de hele dag binnen}{zitten is ook niets}\\

\haiku{Er ratelde een,.}{rijtuig over de brug verder}{gebeurde er niets}\\

\haiku{Arbeider worden.}{was wel de zwaarste slag die}{een boer kon treffen}\\

\haiku{Na het invallen.}{van de dooi mocht Abe bij de}{Atema's thuis komen}\\

\haiku{Hij wist zeker dat.}{het er de zestiende ook}{al gelegen had}\\

\haiku{Met grote letters:}{stond er CANADA en iets}{kleiner daaronder}\\

\haiku{De menigte ging,.}{tevreden uiteen het recht}{had gezegevierd}\\

\haiku{De jongens die het,.}{paard hadden laten lopen}{waren hem gevolgd}\\

\haiku{Beppe warmde voor:}{haar kleinzoons een restje op}{van het middagmaal}\\

\haiku{De hond holde naar.}{buiten en kwam Pier vrolijk}{blaffend tegemoet}\\

\haiku{Met grote stappen.}{klom hij de smalle trap op}{naar het sloependek}\\

\haiku{Maar ja, wat wist je,?}{in die tijd van Canada}{of van Amerika}\\

\haiku{deze twee zien eten,.}{dan zouden ze zeker groen}{uitgeslagen zijn}\\

\haiku{Even later weer veel.}{geschreeuw omdat het bier te}{lauw en te oud was}\\

\haiku{Abe liep een gang door,.}{een trap af en kwam zo op}{het derdeklasdek}\\

\haiku{De mensen op de.}{twee langs varende schepen}{keken en wuifden}\\

\haiku{De eerste keer is, '.}{het wel leuk maar nu heb ik}{t wel bekeken}\\

\haiku{Ook het stampen en.}{slingeren van het schip leek}{erger dan overdag}\\

\haiku{Langzaam zakte hij,.}{weg in een toestand van half}{slapen half waken}\\

\haiku{Canada, Quebec,.}{en boot konden ze elkaar}{niet veel vertellen}\\

\haiku{{\textquoteright} Nu begreep Molly.}{er niets van en ze lachten}{er allebei om}\\

\haiku{{\textquoteleft}I am from Torquay,{\textquoteright}:}{voegde ze er aan toe en}{verduidelijkend}\\

\haiku{Het was mooi weer, soms.}{zaten ze zelfs uit de wind}{aan dek in de zon}\\

\haiku{Ze was wat bleker,}{dan anders maar verder was}{er niets aan de hand.}\\

\haiku{Molly was bleek en,.}{had rood-behuilde ogen}{ze keek Abe niet aan}\\

\haiku{Hij wist niet goed wat.}{hij er mee aan moest en deed}{een stapje opzij}\\

\haiku{{\textquoteright} Abe probeerde haar.}{wat te kalmeren en van}{zich af te houden}\\

\haiku{Molly hield op met.}{huilen en omhelsde hem}{innig en dankbaar}\\

\haiku{Hij probeerde een.}{eend te pakken te krijgen}{aan de waterkant}\\

\haiku{De hond probeerde.}{uit alle macht terug te}{zwemmen naar de kant}\\

\haiku{Voorzichtig glipt hij,.}{uit het bed zonder zijn broer}{wakker te maken}\\

\haiku{{\textquoteright} {\textquoteleft}Da's mooi jongen,{\textquoteright} zei, {\textquoteleft}.}{zeen kijk eens wat ik voor}{je verjaardag heb}\\

\haiku{Aukje trok er nog,.}{wat aan onhandig over de}{bedsteerand}\\

\haiku{Pake stak zijn hoofd.}{om de hoek en hoorde wat}{er aan de hand was}\\

\haiku{Na schooltijd moesten Jan.}{en Pier hun spullen van de}{boerderij halen}\\

\haiku{{\textquoteleft}Ga nu maar, jongen,,{\textquoteright}.}{anders missen jullie de}{trein nog zei Aukje}\\

\haiku{Het was voor het eerst.}{in hun leven dat ze in}{een auto zaten}\\

\haiku{Jopie had altijd,.}{geld op zak en Pier leerde}{hoe hij daar aan kwam}\\

\haiku{Drummondville, 8,}{juli 1913 ~ Lieve vrouw}{en kinderen}\\

\haiku{We praten niet veel,}{meestal zijn we te moe}{om veel te zeggen}\\

\haiku{In Drummondville.}{heb ik een man ontmoet die}{Emile Vassal heet}\\

\haiku{Dit is voorlopig,.}{het einde van mijn verhaal}{ik ga nu wat eten}\\

\haiku{Het schetsboek hield Pier,.}{geheim alleen mem had het}{wel eens mogen zien}\\

\haiku{Na ruim een uur kreeg.}{hij zijn eerste zuivere}{toon uit de viool}\\

\haiku{Pier had de smaak te:}{pakken en hij had nu nog}{maar \'e\'en verlangen}\\

\haiku{Voorlopig hadden.}{ze het te druk om aan de}{toekomst te denken}\\

\haiku{Moe, vuil en grimmig.}{liepen de mannen rond de}{molen en het huis}\\

\haiku{Het bleef mogelijk.}{dat de ramp het gevolg was}{van blikseminslag}\\

\haiku{Ze spanden het beest.}{voor de wagen en reden}{naar Drummondville}\\

\haiku{Daarna gingen ze.}{uiteen om elkaar wellicht}{nooit meer te zien}\\

\haiku{Bij elk station,.}{nam hij een hapje dan deed}{je er langer mee}\\

\haiku{Een brief van Aukje.}{gaf haar de volgende dag}{meer informatie}\\

\haiku{De toevoeging {\textquoteleft}and{\textquoteright},:}{sons die hij kon vertalen}{en de plaatsnamen}\\

\haiku{Daar was haast niet aan:}{te komen en bovendien}{stond er dan nog op}\\

\haiku{{\textquoteright} {\textquoteleft}Het is alweer een,.}{hele tijd geleden nog}{voor het grote feest}\\

\haiku{{\textquoteleft}En ze hebben er,,.}{hele dikke kleden op}{de vloer mem zo zacht}\\

\haiku{P\`ere en m\`ere}{Vassal zijn erg goed voor me.}{P\`ere en m\`ere}\\

\haiku{{\textquoteleft}Is 't anders niet,{\textquoteright}.}{doe de mond maar open man en}{laat maar eens zien}\\

\haiku{Het is een goede,.}{les geweest ik ga niet meer}{zo ver het bos in}\\

\haiku{{\textquoteright} Zijn moeder had hem.}{over het haar gestreken en}{hem gerustgesteld}\\

\haiku{Ze sloot de brief af.}{en schreef met duidelijke}{letters het adres}\\

\haiku{Jan werkt bij pake,,.}{hij wordt groot je zult verbaasd}{zijn als je hem ziet}\\

\haiku{Het overlijden van.}{oude mensen ligt in de}{natuur der dingen}\\

\haiku{Het duurde enige.}{tijd voor hij reageerde}{op de twee brieven}\\

\haiku{Ik ben gewoon weer '.}{in dat kantoor gaan zitten}{ens avonds buiten}\\

\haiku{Op de laatste dag.}{kwam mijn vriend naar het kantoor}{vlak voor sluitingstijd}\\

\haiku{Je zult ook leven,.}{Aukje het komt allemaal}{nog best voor elkaar}\\

\haiku{{\textquoteleft}Dat is niet genoeg,.}{voor de trein je hebt nog maar}{zeven dollar over}\\

\haiku{Abe weigerde naar,.}{een hotel te gaan wegens}{de hoge kosten}\\

\haiku{{\textquoteright} Ze wandelden door,.}{de kleurloze saaie straten}{van de drukke stad}\\

\haiku{Er scheen geen school voor,.}{de kinderen te zijn ze}{waren altijd thuis}\\

\haiku{Aukjes brief deed er.}{meer dan drie weken over om}{hem te bereiken}\\

\haiku{Hij stond voor de deur.}{van het ziekenhuis toen zijn}{vrouw naar buiten kwam}\\

\haiku{Het was op het heetst.}{van de dag dat Abe in het}{rijtuigje stapte}\\

\haiku{Ver beneden, op,.}{het binnenplaatsje slofte}{een oude man rond}\\

\haiku{Het zou zeker een.}{halve dag duren eer de}{boel opgeruimd was}\\

\haiku{Ze gingen op een.}{bank voor het hotel zitten}{en dronken koffie}\\

\haiku{Rond het middaguur.}{stoomden ze de prairie in}{op weg naar Morse}\\

\haiku{De eerste dag op.}{zijn eigen grond was Abe al}{voor dag en dauw op}\\

\haiku{Piepend en krakend.}{hield het karretje stil bij}{het hout en huisraad}\\

\haiku{{\textquoteleft}Goejemorg'n,?}{sam'n kunn'n jullie nog wat}{naberhulp gebruik'n}\\

\haiku{Het eigenlijke.}{ontginningswerk moest wachten}{tot het volgend jaar}\\

\haiku{Molly zag er in.}{een lichtblauwe jurk prachtig}{uit en ze wist het}\\

\haiku{Een al dagenlang.}{durende sneeuwstorm dwong hen}{binnen te blijven}\\

\haiku{Abe niet omdat hij,.}{met een plan rondliep dat hij}{niet kon uitvoeren}\\

\haiku{Ze vroeg hem wel tien.}{maal of hij echt van plan was}{terug te komen}\\

\haiku{{\textquoteright} De vrachtboot was een,,.}{traag oud stoomschip geladen}{met superfosfaat}\\

\haiku{Hij liep achterom,.}{om niet met nieuwsgierigen}{te hoeven praten}\\

\haiku{Hij zag veel oude.}{bekenden en vertelde}{overal zijn verhaal}\\

\haiku{Voor de tweede maal,.}{in zijn leven voer Abe weg}{nu definitief}\\

\haiku{Nee, nee, ik kom niet,,.}{uit Maryland ik heet zo}{ik woon in Detroit}\\

\section{Taecke J. Botke}

\subsection{Uit: Het revier}

\haiku{Misschien kun je je.}{voorstellen hoe feestelijk}{die ervaring was}\\

\haiku{Maar in een hete.}{barak komen de dingen}{anders te liggen}\\

\haiku{Alleen in het brein {\textquoteleft}{\textquoteright}.}{derHerrenmenschen kon zo'n}{geestrijk idee ontstaan}\\

\haiku{Misschien ligt er toch '.}{een loopbaan voor mij int}{geestelijke}\\

\haiku{Deze eiste dat.}{Eddy bij de operatie}{aanwezig zou zijn}\\

\haiku{De opkomende.}{maan verbleekte de hemel}{tot een helder blauw}\\

\haiku{De mannen die het - -;}{leven met hun aandelen}{in de zak hebben}\\

\haiku{De ene snavel is,.}{ondernemend de ander}{veeleer vermoeid}\\

\haiku{Eer de stoet zich in:}{beweging zette werd het}{dagrantsoen verstrekt}\\

\haiku{Wie er al dood was,.}{en wie nog leefde viel niet}{meer uit te maken}\\

\haiku{zo'n man die beslist.}{het leven voor iets hogers}{wilde inzetten}\\

\haiku{hij vreesde ziekten.}{zodat wij vrij waren in}{ons doen en laten}\\

\haiku{Tandartsen waren{\textquoteright}.}{hoogstwaarschijnlijk geen echte}{Akademiker}\\

\haiku{Maar het was ook het.}{enige waarin zeker niet}{voorzien kon worden}\\

\haiku{Majesteit of geen, '.}{majesteit maart moet niet}{besmettelijk zijn}\\

\haiku{Die uit de fabriek,.}{stellen zich op in rij en}{gelid evenals wij}\\

\haiku{Zijn distinctie schiep,.}{afstand maar zijn schoonheid trok}{de blikken tot zich}\\

\haiku{Een boer die kwaad wordt,.}{is een oerfenomeen zo}{iets als een onweer}\\

\haiku{Waar ging dat heen als.}{dergelijke dieven vrij}{rond konden lopen}\\

\haiku{Toen hij echter de,.}{inhoud ontdekte was hij}{geheel verbijsterd}\\

\haiku{Ook de schoorstenen.}{van het crematorium}{tekenden zich af}\\

\haiku{Dit is, zo men wil,.}{een vorm van pech hebben die}{erger had gekund}\\

\haiku{Een gouden voortand.}{verleende aan de grijns een}{naargeestig accent}\\

\haiku{Het bestaan in de.}{quarantainebarakken}{was moeilijk genoeg}\\

\haiku{Iemand gooide de.}{deur open en ons vertrekje}{liep vol met Russen}\\

\haiku{{\textquoteleft}Heb jij misschien een?}{advertentie voor werksters}{in de krant gezet}\\

\haiku{Dat zou de onze,.}{worden met Joop en mij als}{geboortehelpers}\\

\haiku{Zwak nog, maar vrijwel.}{genezen had ik mijn bed}{kunnen verlaten}\\

\haiku{een aardig gebaar.}{dat echter geen einde maakt}{aan spuitend water}\\

\section{Ina Boudier-Bakker}

\subsection{Uit: Het beloofde land}

\haiku{dat hun werk, vooral,.}{dat van Adam al het werk in}{den omtrek overtrof}\\

\haiku{Maar den tweeden dag,,.}{na Jelle's dood gebeurde voor}{Eli iets wonderlijks}\\

\haiku{Het was alles in ',...}{t begin geweest als een}{heerlijk nieuw leven}\\

\haiku{Als de avond stil was;}{en schoon kwam het volk bijeen}{onder de linden}\\

\haiku{als de sneeuwstormen,...}{woeden en de ijzige}{noordoostenwind giert}\\

\haiku{Nu komt hij, en rust,...}{in het oude kamertje}{op den ouden stoel}\\

\haiku{Van kind af samen,;}{opgevoed waren zij sterk}{aan elkaar gehecht}\\

\haiku{Toen Eli opschrikte,.}{uit zijn gedachten zag hij}{plotseling Hester}\\

\haiku{Waarom moest altijd?}{elk gevoel weer verdrongen}{worden door een nieuw}\\

\haiku{Eli hield van haar en,,.}{van Maarten om de groote rust}{die van hen uitging}\\

\haiku{En niemand wist het,:}{geheim dat hij stil en trouw}{in zijn hart besloot}\\

\haiku{* * * ~ Moeder - ik weet,,.}{nu er is iets sterker dan}{al het andere}\\

\haiku{Wat mij geen rust liet,.}{eer ik zou voelen uw hand}{op mijn moede oogen}\\

\haiku{er was een onrust,;}{in Eli die hem dreef alleen}{te zijn met Hester}\\

\haiku{Maar je bent hem niet...{\textquoteright} {\textquoteleft} - '...{\textquoteright} {\textquoteleft}}{kwijtJawel hij hindert me}{niet meer int werk}\\

\haiku{Ik heb geleefd, de - -.}{dagen en nachten door met}{dat \'e\'ene Mijn Haat}\\

\haiku{Als zij klaagden over,,}{te veel arbeid dan dacht hij}{dat de arbeid dien}\\

\haiku{... langzaam versmolt hun.}{wrok voor een vaag kinderlijk}{gevoel van schaamte}\\

\haiku{Hij heeft gelijk - hij ',{\textquoteright},.}{w\'e\'ett zei de oude Brandt}{met stralende oogen}\\

\haiku{zijn wrok tegen het - -:}{volk zijn ellende om het}{werk bl\'e\'ef alleen dit}\\

\haiku{Waarom was deze!}{vreeselijke dag in zijn}{leven gekomen}\\

\haiku{ernstig, haar blijheid,}{geschokt maar rustig bereid}{alles te dragen}\\

\haiku{Een enkelen keer,;}{zeiden de anderen iets}{dat zijn aandacht trok}\\

\haiku{zijn woorden waren...}{een openbaring  geweest}{van iets vreeselijks}\\

\haiku{De smartkreet van het,.}{moederschepsel dat zich haar}{jongen ziet ontroofd}\\

\haiku{Het is een kreet van,.}{kracht een hartstochtelijke}{stem van strijd en smart}\\

\haiku{Hij zag het, zooals hij.}{nooit te voren een werk had}{gezien en doorvoeld}\\

\haiku{De ander zag hem,.}{schuin aan met een enkelen}{snellen oogopslag}\\

\haiku{Een poos bleef hij haar -.}{zwijgend aanzien toen schudde}{hij langzaam het hoofd}\\

\haiku{{\textquoteleft}Dat begrijp jij niet -, -.}{en niemand daarom kon ik}{er niet van spreken}\\

\haiku{Hester en niemand,.}{vermoedde dit hij verborg}{het nog volkomen}\\

\haiku{Een paar keer werd hij.}{wakker en herinnerde}{zich Berends woorden}\\

\haiku{Hij wist nu wel dat.}{hij bezweken was in den}{jaren-langen strijd}\\

\haiku{{\textquoteleft}Je hoort 't, jongens,...}{Eli geeft zijn woord dan moeten}{we dat vertrouwen}\\

\haiku{Z\`elfs de angst, die de,.}{vorige weken gevuld}{had was er niet meer}\\

\haiku{{\textquoteleft}Als Hester 't niet, ', -}{weet mo\`et zet nu weten}{waarom verzwijgen}\\

\haiku{toen niemand kwam, stak ',.}{hij een sleutel int slot}{en opende de deur}\\

\haiku{een poos zag hij hem,,,;}{aan zooals hij lag zijn hoofd op}{zijn arm zwaar snurkend}\\

\haiku{Ze dacht plotseling,.}{terug hoe ze vroeger zich}{Eli voorgesteld had}\\

\haiku{{\textquoteright} {\textquoteleft}Oude man - zij ook,.}{zullen ondergaan in den}{blanken dooden vrede}\\

\haiku{Maar de brieven van.}{Foeko waren nog altijd}{niet gekomen}\\

\haiku{zal 't me later,.}{spijten ik heb om u nog}{een poos doorgezet}\\

\haiku{Het was voor 't eerst '.}{weer na dien dag dat Elis}{avonds gekomen was}\\

\haiku{Hij zag naar Adam, die,.}{daar rustig zei de harde}{heldere woorden}\\

\haiku{Het kleine lampje...}{knetterde met steeds lager}{brandend vlammetje}\\

\haiku{Het was donker, lang - '.}{donkert was moeielijk om}{hier weer te komen}\\

\haiku{Daar zijn de sterken,,.}{met vasten mond het gelaat}{doorploegd van veel leed}\\

\haiku{{\textquoteright} Eli rilde in den,.}{kouden wind een scherpe hoest}{doorschokte zijn borst}\\

\haiku{er was iets in haar,.}{zitten zoo dat hem even aan}{Hester deed denken}\\

\haiku{als u me niet hadt,...{\textquoteright}}{kunnen helpen had ik naar}{vreemden moeten gaan}\\

\haiku{En hartelijk, met,.}{hun schaarsche woorden zeiden}{ze hem te blijven}\\

\haiku{{\textquoteleft}Zie je wel - 't is ', - '...{\textquoteright}}{goed datk gegaan ben nou}{zouk te ziek zijn}\\

\haiku{het gezicht in zijn,.}{handen verborgen weende}{hij stil over Eli Bag}\\

\subsection{Uit: Een dorre plant}

\haiku{Maar daarna begon.}{het als een zacht nieuw geluk}{in hem te leven}\\

\haiku{{\textquoteright} en kreeg Willem een,.}{duw of een klap dat hij haar}{met rust moest laten}\\

\haiku{en 't liet ook niet.}{als bij Bertus een goede}{herinnering na}\\

\haiku{Telkens, als ze naar,.}{het jonge paar keek kreeg ze}{tranen in de oogen}\\

\haiku{Hij was ieder vrij,.}{oogenblik bij haar hoewel}{ze hem niet kende}\\

\haiku{Wanneer Bertus van,.}{een reis thuiskwam  fleurde}{er altijd iets op}\\

\haiku{{\textquoteright} Toen Johanna dit,:}{aan Jonas overbracht zei de}{oude man norsch}\\

\haiku{D\`at vader toch zoo;}{weinig aardigheid had in}{de kleinkinderen}\\

\haiku{{\textquoteleft}Jij moet maar met ze,.}{naar het park gaan hoor als ze}{niet thuis kunnen zijn}\\

\haiku{{\textquoteleft}Laat ze maar liever ',.}{naart park gaan daar is meer}{ruimte dan bij mij}\\

\haiku{De periode,,.}{toen hij aan niets van vroeger}{ooit dacht was voorbij}\\

\haiku{Hij begon ook te,;}{merken dat de menschen op}{hem gingen letten}\\

\haiku{Hij werd alleen kwaad,.}{als zij hem beletten wou}{buiten te zitten}\\

\haiku{{\textquoteleft}'t Lijkt geen zier op -.}{Verbruggen denk je dat ik}{die niet zou kennen}\\

\haiku{En d\`at zou ze zich,!}{laten ontnemen zonder}{hem was zij ook niets}\\

\haiku{{\textquoteleft}Zou hij gaan spreken, ' '!}{waarachtigt hadm dan}{t\'och wel aangedaan}\\

\haiku{Toen hoorde hij ook.}{Johanna slaperig moe}{naar boven sloffen}\\

\haiku{En hij wist wel, hij '.}{zou ze int voorjaar niet}{meer zien opbloeien}\\

\subsection{Uit: Kinderen}

\haiku{nee, niet naar ma nou,{\textquoteright},.}{h\'e\'elemaal naar boven naar zijn}{eigen kamertje}\\

\haiku{{\textquoteright} Vader liep met hem ',.}{door naart eind van de gang}{waar een juffrouw stond}\\

\haiku{Jip keek verlegen,,,}{op zijn lei toen moedvattend}{stak hij zijn vinger}\\

\haiku{{\textquoteright} Weifelend dwaalde.}{zijn vochtig vingertje uit}{zijn mond over zijn lei}\\

\haiku{{\textquoteleft}Kom 'r maar tusschen,{\textquoteright}, {\textquoteleft},,!}{zei hijhier kan je wel staan}{maar goed vangen hoor}\\

\haiku{{\textquoteright} Geregeld ging de,,.}{bal rond tot ook Miel er uit}{viel en toen ook Ru}\\

\haiku{{\textquoteright} Jip knikte alleen,,;}{maar uiterst voldaan zonder}{eenige jaloezie}\\

\haiku{{\textquoteleft}Kom kinderen, twee!}{aan twee in de rij gaan staan}{en dan naar binnen}\\

\haiku{In den kring, op gang,,:}{nu liepen de kinderen}{zingend schel-valsch}\\

\haiku{{\textquoteleft}'n Ander gegooid, ' '.}{in de gangk zalm maar}{weer laten meedoen}\\

\haiku{{\textquoteleft}Denk erom, anders ';}{laatk je in een heele}{week niet meespelen}\\

\haiku{{\textquoteright} {\textquoteleft}Laten we alleen...{\textquoteright}}{maar es effetjes likken}{aan de achterkant}\\

\haiku{Ze likten, een voor, '. '}{een veegdent gauw met hun}{boezelaar weer droog}\\

\haiku{Roos, met \'even-spottend,.}{neergetrokken mondje keek}{in het kacheltje}\\

\haiku{dan heb je ook geen - '....}{mazelen geh\`ad je weet}{niet eens watt is}\\

\haiku{Vooruit nou - h\`e nee,,;}{nou moet je d\`oen wat ik zeg}{ik ben de mevrouw}\\

\haiku{{\textquoteright} {\textquoteleft}Nou, toen wel honderd, -.}{jaar later toen toen vonden}{ze z'n geraamte}\\

\haiku{{\textquoteleft}'t Smaakte w\`el \`erg,{\textquoteright} - {\textquoteleft}.}{leelijk zei Aleidaze heeft}{er niks an gehad}\\

\haiku{Maar ziet u, ik loop, '.}{een beetje hard wantt mocht}{es gaan regenen}\\

\haiku{voor zij kwam, vlak na,.}{moeders dood waren Riek en}{hij veel meer samen}\\

\haiku{{\textquoteleft}och nee, ik vergis,....}{me ik ga natuurlijk niet}{uit de stad morgen}\\

\haiku{Even voorbij den hoek.}{van den Oosterweg zag hij}{de bloemisterij}\\

\haiku{Hij stond doodstil, keek,}{hardnekkig den anderen}{kant uit met geweld}\\

\haiku{Maar later, toen moes - '.}{w\`eg was to\`en wast nog ve\`el}{na\`arder geworden}\\

\haiku{Nu haar mantel en, -,....}{hoed hingen keek ze even snel}{rond nee geen juffrouw}\\

\haiku{{\textquoteleft}W\'eten jouw ouders,?}{dat jij na vieren grachtjes}{omloopt met jongens}\\

\haiku{Ze had natuurlijk,.}{b\`est gezien dat ze altijd}{met dezelfde liep}\\

\haiku{- Die gemeene Loos - - ' -}{om zoo te spionneeren maar}{ze liett t\`och niet}\\

\haiku{dat mo\`est nu eenmaal -!}{hoe konden sommige dat}{toch prettig vinden}\\

\haiku{ik heb in geen twee -.}{keer een beurt gehad ik mo\`et}{er een krijgen nou}\\

\haiku{{\textquoteright} zei hij, met een blij,.}{even opglanzen in zijn oogen}{dankbaar voor den lof}\\

\haiku{{\textquoteleft}hij zou ze aan Frits,,....}{geve die vissies hij zou}{d'r wel blij mee zijn}\\

\haiku{Een poos was er niets,.}{dan het gulzig eten en het}{klikken der vorken}\\

\haiku{Hi\`er houe, hi\`er,,!}{zeg ik stommerik k\`a je}{uit je ooge niet zien}\\

\haiku{{\textquoteright} wees Tom. {\textquoteleft}Nee, ma houdt, ',.}{er niet van datr wat op}{zit hij moet glad zijn}\\

\haiku{{\textquoteleft}Zalle toch wel niet,'.}{allemaal zoo duur zijn la}{we d'r maar ingaan}\\

\haiku{Tom merkte het, wees.}{gauw stiekem opzij in de}{mand en dan naar ha\`ar}\\

\haiku{Nu is de mand leeg,{\textquoteright},.}{zei ma alsof ze dat}{heel natuurlijk vond}\\

\haiku{{\textquoteleft}Gelukkig, dat je,!}{er nu een gezond voor in}{de plaats hebt Tommie}\\

\subsection{Uit: De klop op de deur}

\haiku{De derde gast was.}{een toast begonnen op den}{gastheer en diens vrouw}\\

\haiku{Hij moest                     stilstaan.}{en midden op straat zich een}{zoen laten geven}\\

\haiku{De lange vader.}{en het kleine kind liepen}{er stil tusschendoor}\\

\haiku{Verslonden stond ze,.}{te kijken naar den poedel}{die door hoepels sprong}\\

\haiku{Nu ja, ze wist het.}{wel dat zij nu eenmaal mooi}{gevonden                     werd}\\

\haiku{Maar moeder lachte,.}{en daar was al een heele}{gerustheid in}\\

\haiku{{\textquoteright} {\textquoteleft}Maar 't is ook niet -.}{de aard van een vrouw dat ze}{zaken doet zooals ik}\\

\haiku{Over Het Water dreef.}{de wind de hooge tonen aan}{van de Boomklok}\\

\haiku{{\textquoteleft}Zeg Marie, toe geef?}{me nu meteen even dat geld}{van de handschoenen}\\

\haiku{Pas toen ze thuis haar,}{donkere trap weer opging}{bedacht ze kregel}\\

\haiku{De handschoenen had,.}{zij weggeborgen ze niet}{durvend toonen nog}\\

\haiku{{\textquoteright} En hij dacht aan de.}{nijpende zorgen in}{hun kleinen winkel}\\

\haiku{Bij het huis van Dr.,;}{Goldeweijn zei Leentje hem}{dat mevrouw uit was}\\

\haiku{En hier op eenmaal.}{stond Karel de Roos in een}{andere wereld}\\

\haiku{Ik had een briefje -,....}{van thuis Leentje zei ik mocht}{wel naar boven gaan}\\

\haiku{Vanavond moet ik ook,.}{pianospelen daarom}{studeer ik nog}\\

\haiku{{\textquoteleft}Vijf erwtjes zaten,{\textquoteright}.}{in een peul                    heelemaal}{van buiten kende}\\

\haiku{{\textquoteright} zei ze wat snibbig,.}{dat verloren ging in haar}{goedigen                     lach}\\

\haiku{Ze werken bij mij,.}{zestig uur in de week}{of tien uur per dag}\\

\haiku{haar goed moederlijk.}{gezicht keek van Ann\`etje}{naar Annebet}\\

\haiku{Och - zij had erbij.}{gezeten en gedacht dat}{zij het niet erg vond}\\

\haiku{Maar Fransje, als de,.}{pijn                     haar even losliet kon}{zelfs hierom lachen}\\

\haiku{'t Is toch een                          -!}{beestenziekte wat zou een}{mensch daar van krijgen}\\

\haiku{hooger loon kunnen.}{eischen door anderen}{worden vervangen}\\

\haiku{En zij stroomden de,.}{lokalen uit in den avond}{een schamele troep}\\

\haiku{Slecht betaalde                     ....}{pianolessen gaven}{m\`et vreemde talen}\\

\haiku{Ann\`etje hoorde.}{en begreep ook niet alle}{bizonderheden}\\

\haiku{{\textquoteright} Zij werd z\'o\'o bleek als.}{ze niet geweest was toen ze}{haar vonnis hoorde}\\

\haiku{hoe anders zijn de,....{\textquoteright}}{Fran\c{c}aises en hoe slecht}{ook kleeden ze zich}\\

\haiku{lachen om alles,.}{wat ze niet begrijpen wat}{ze z\`elf niet weten}\\

\haiku{Waarom - omdat ik....}{de   dingen niet weet die}{ik weten moest}\\

\haiku{neen u begrijpt het,:}{verkeerd anders zoudt u het}{z\'o\'o niet zeggen}\\

\haiku{s avonds er een paar -.}{boekhouderijtjes bij dat}{helpt                     allemaal}\\

\haiku{Alles deelden zij,.}{samen maar die                     dingen}{vertelde hij niet}\\

\haiku{{\textquoteleft}Kind, mijn kleintje, als,.}{ik niet terug kom weet je}{dat ik het goed vind}\\

\haiku{de bollen bloeien,{\textquoteright}.}{zei Fransje en snoof den}{lentegeur                     in}\\

\haiku{Handen, die rustig.}{en lief de dingen om}{je heen zouden doen}\\

\haiku{- dat is me te                     , -.}{opstandig te wild zoo raar}{van gedachten soms}\\

\haiku{{\textquoteright} {\textquoteleft}Is dat nu iets voor,...}{onzen jongen die ieder}{meisje krijgen kan}\\

\haiku{Vanavond had zij}{zeker gedacht dat zij niet}{hield van Frederik}\\

\haiku{een                     stortzee had}{hem in den laatsten zwaren}{storm in het Kanaal}\\

\haiku{{\textquoteleft}En Vondel krijgt nu.}{zijn standbeeld in het Rij-}{en                     Wandelpark}\\

\haiku{Hij heeft geen moment.}{aandacht ook gehad voor een}{kunst van Rembrandt}\\

\haiku{{\textquoteleft}Dat was Alberdingk -.}{Thym die daar met Van Lennep}{stond en da\`ar Pierson}\\

\haiku{Een lang stil leven.}{op den achtergrond viel}{in deze uren weg}\\

\haiku{{\textquoteright} {\textquoteleft}Aan liefde heeft het,{\textquoteright}.}{ons nooit ontbroken zei de}{domineesche hoog}\\

\haiku{Eveline is een,.}{beest van liefdeloosheid dat}{weet je nu eenmaal}\\

\haiku{{\textquoteleft}Alles goed,{\textquoteright} dacht ze,.}{in een snik van verlichting}{vloog hem tegemoet}\\

\haiku{{\textquoteleft}Moeder, hoort u eens,....?}{was u niet bang toen u met}{vader trouwen moest}\\

\haiku{Vooral Am\'elie van.}{Dugten kon haar moederlijk}{warm                     omhelzen}\\

\haiku{Ze hadden er geen,.}{benul van hier wat dat}{alles beteekende}\\

\haiku{Haar kinderjaren -.}{die goed                     geweest waren}{zij was een mooi kind}\\

\haiku{Veel minder goed ging.}{het tusschen Ann\`etje en}{Line Bergema}\\

\haiku{Ik moest den heelen ':}{avond denken aant woord van}{Gavarini}\\

\haiku{Dokter Bergema,.}{was                     er geweest die vond}{geen direct gevaar}\\

\haiku{Hij leerde uit zijn {\textquoteleft}{\textquoteright} - {\textquoteleft}.}{hoofd HeinesDie Weber}{Das Harfenm\"adchen}\\

\haiku{En d\`at alles bracht?}{Duitschland al in                     de}{veertiger jaren}\\

\haiku{Karel sprak er niet,.}{van dat hij Goldeweijn voor}{het raam had gezien}\\

\haiku{{\textquoteleft}Guerre \`a la,{\textquoteright},:}{Prusse dat hoor je overal}{vooral het slot}\\

\haiku{Nog v\'o\'or den winter.}{dreigde al armoede}{en werkeloosheid}\\

\haiku{De groote werkgevers,,,.}{de fabrikanten bezorgd}{zetten zich                     schrap}\\

\haiku{de vernietiging,.}{die aanschrijdt onweerhoudbaar}{door den                     Elzas}\\

\haiku{Naast hem zijn zoon, de.}{tengere                     knaap die nooit}{Keizer worden zal}\\

\haiku{Voor de stad dringen,.}{hen de Pruisen op dringen}{hen binnen de stad}\\

\haiku{Ja, als je trouwde,,.}{d\`an kwam je eruit dat was}{de                     eenige weg}\\

\haiku{Een vrouw is in den.}{avond de achterdeur van het}{paleis uitgevlucht}\\

\haiku{Pannen en schoorsteenen -.}{vliegen in de Keizersgracht}{ligt een boom geploft}\\

\haiku{In het paleis van.}{Versailles zetten zich de}{Duitschers aan tafel}\\

\haiku{ik kan tegen jou.}{wel eens                     praten over de}{dingen die ik denk}\\

\haiku{Maar nu - nu worden!}{de kinderen bedreigd die}{ze                     overhielden}\\

\haiku{Na een tweedaagschen....}{slag bij Belfort zijn ze}{teruggeworpen}\\

\haiku{Ze hokken in hun,.}{vuilste nauwe straatjes hun}{slechte woningen}\\

\haiku{En zij                     zaten,.}{weer als vroeger ieder aan}{een kant van het raam}\\

\haiku{Haar beide oudste.}{dochters waren                     eveneens}{in Indi\"e getrouwd}\\

\haiku{Als ik je vertel....}{dat in Fernande                     de}{dames dansen met}\\

\haiku{Maar ze had eerder,.}{haar tong afgebeten dan}{zelfs                     maar gevischt}\\

\haiku{{\textquoteright} ~ De volgende.}{dagen lachten Frederiks}{blauwe oogen schelmsch}\\

\haiku{want de vrouw, die                     .}{in haar eigen onderhoud}{kan voorzien is vrij}\\

\haiku{Jaren woonde zij -,,.}{hier een stille oude}{eenzame juffrouw}\\

\haiku{Da\`ar zullen we, als ',.}{t God blieft                     ons lieve}{land voor bewaren}\\

\haiku{En iederen                     .}{dag wachtte zij op wat hij}{nog niet gezegd had}\\

\haiku{En                     als zij bij -!}{haar moeder kwam hoe k\`on zij}{er over beginnen}\\

\haiku{{\textquoteleft}Ik ben bang tante,.}{dat het u veel zal kosten}{uit het huis te gaan}\\

\haiku{Hij wou haar geld                     ,.}{beheeren zorgen dat zij}{niet zooveel uitgaf}\\

\haiku{Hij kon in al wat,.}{het nieuwe huis betrof niet}{tegen haar op}\\

\haiku{Voelde zich                     ruw,.}{gesleurd in harde armen}{bezeerd en gekneusd}\\

\haiku{Op de bisbilles,.}{van zijn zusters onderling}{ging hij nooit in}\\

\haiku{{\textquoteright} {\textquoteleft}'t Is mijn huis niet,{\textquoteright}, {\textquoteleft} '.}{dacht Fransje Goldeweijnen}{t wordt                     het nooit}\\

\haiku{Frederik zag hen:}{de                     stoep opstommelen}{vanuit zijn kantoor}\\

\haiku{Maar 't is toch zoo'n,,{\textquoteright}.}{kleine slaapkop dat weet}{je niet zei Stance}\\

\haiku{De                     jongen is -!}{eruit jij zal d'r ook uit}{als je niet inbindt}\\

\haiku{Maar Am\'elie hield n\`og,,.}{een boek op haar schoot als iets}{kostbaars gevangen}\\

\haiku{Al die aaa's klinken.}{als een oorlogsgehuil uit}{een                     wijde keel}\\

\haiku{Daarom is dit zoo'n.}{gezegende tijd voor}{de jonge vrouwen}\\

\haiku{Heldt, de ziel van het.}{Algemeen Nederlandsch}{Werkliedenverbond}\\

\haiku{Officier vond hij - '.}{een stom                     baantjet had}{zijn sympathie niet}\\

\haiku{Toen de zusters weg.}{waren nam Frederik}{haar in zijn armen}\\

\haiku{'t W\`as ook niet dat....}{ze oma niet helpen wou of}{niet van haar                     hield}\\

\haiku{wat hij iederen.}{dag ervaren ging als iets}{verwonderlijks}\\

\haiku{Sinds twee jaar was hij.}{mede-directeur der}{Nederlandsche Bank}\\

\haiku{Craets minachtte de,.}{vrouwen werd furieus om}{de                     arbeiders}\\

\haiku{De strijd overigens,,.}{zooals die hier gestreden}{wordt boeit me weinig}\\

\haiku{En nu ze in                     ,.}{hun program de vrouw erin}{halen meer dan ooit}\\

\haiku{{\textquoteright} Toen ze pasten keek.}{Caroline plots geboeid naar}{Louises mooien hals}\\

\haiku{{\textquoteleft}Ze wou                     ook weer.}{eens meer bij Annette en}{Frederik komen}\\

\haiku{Daar stikte ze in,!}{of ze verdronk in                     een}{walgelijken poel}\\

\haiku{{\textquoteright} riep de oude vrouw,.}{geschrikt met haar schrille stem}{De dochter keek op}\\

\haiku{wij hebben alleen.}{een ongelukkig kind om}{van te vertellen}\\

\haiku{En tenslotte week:}{alles                     onbelangrijk}{terug voor de vraag}\\

\haiku{{\textquoteleft}Z\`eg 't me niet, z\`eg', '!}{t me niet want ik word}{gek alst zoo is}\\

\haiku{Goed is                     hij als,.}{een lam maar hij laat zich niet}{op zijn kop zitten}\\

\haiku{Vreemde                     stemmen.}{spraken woorden die niet meer}{verloren gingen}\\

\haiku{Over de liefde en{\textquotedblleft}{\textquotedblright}!}{den hartstocht gillen en}{weenen op een tooneel}\\

\haiku{Ze werd zoo doodelijk bleek,,.}{dat hij schrikte dacht dat ze}{flauw                     zou vallen}\\

\haiku{Ze aaide zijn haar,.}{trok het gedachteloos in}{plukjes uit elkaar}\\

\haiku{Annette zat met,.}{de brieven in haar schoot die}{ze moeder voorlas}\\

\haiku{{\textquoteleft}Je krijgt een broertje - '.}{of een                     zusje ik zie}{t aan je moeder}\\

\haiku{Ze was weer schuw en.}{stug en keek hem letterlijk}{de kamer                     af}\\

\haiku{Francientje wilde.}{een kopje nemen en oma}{schoof er haar een toe}\\

\haiku{Kwamen de k\`erels?}{er                     niet altijd met hun}{vrachtkarren overheen}\\

\haiku{Tumult ging op, maar.}{ook hier werden de vechters}{uit elkaar gejaagd}\\

\haiku{Een kerel en een,.}{groot wijf liepen de stoep over}{bleven voor haar staan}\\

\haiku{Om h\`em hield ze zich -.}{tenslotte in verlangend}{alleen te                     zijn}\\

\haiku{Dan verlangde ze -;}{dat Philip zou komen haar}{beste kameraad}\\

\haiku{Weet u nog hoe trotsch?}{uw Pa altijd in zijn}{hart was op mevrouw}\\

\haiku{Veltman,                     mevrouw -.}{Kleine-Gartman begon}{terug te treden}\\

\haiku{Haastig, opgeschrikt, '.}{was Pieter bijt jonger}{broertje gekomen}\\

\haiku{Haar eenige zorg was -!}{geweest en hoe erg had zij}{dat al gevonden}\\

\haiku{Ze ging zitten in.}{den                     fauteuil en staarde}{den stillen tuin in}\\

\haiku{al was zij                     haast,.}{vijftig en dat beduidde}{hier in Holland oud}\\

\haiku{Heel jong had zij een;}{liaison gehad met een}{veel ouderen man}\\

\haiku{Rustig was                     hij.}{door den vreemden Hollander}{naar zijn huis vervoerd}\\

\haiku{Nu ja, nu ja, ik, '.}{weet welt is allemaal}{in liefde geschied}\\

\haiku{Ze keek hem aan, over,,.}{de tafel haar gezicht half}{spot half ergernis}\\

\haiku{Zie je, ik ben toch,.}{consequent want                     ik w{\`\i}l}{er niet naar kijken}\\

\haiku{Alleen                     Mens, de,.}{koning van de Willemstraat}{hield zijn buurt rustig}\\

\haiku{Waarom vroeg hij haar -....}{niet                     te gaan zitten of}{binnen te komen}\\

\haiku{{\textquoteright} {\textquoteleft}Eigenlijk nog niet.}{zooals ik dat bedoelde en}{ook nog altijd zoek}\\

\haiku{{\textquoteright} {\textquoteleft}Maar de eerste de,}{beste haalde je weg en}{je vergat alles}\\

\haiku{Als jij                     er niet,.}{bent komen we dadelijk}{tot malle dingen}\\

\haiku{Caroline sinds de,.}{kinderen zoo groot werden}{was van hen vervreemd}\\

\haiku{de kamer binnen,.}{waar Louise aan de tafel}{zat voor haar handwerk}\\

\haiku{Geen                     Kamerlid;}{groette het nieuwe lid of}{liet zich voorstellen}\\

\haiku{Maar Annette, met,:}{den soberen ernst van haar}{vader had gezegd}\\

\haiku{Frederik raadde ' -....}{t nooit een asyl voor dieren}{wilde oprichten}\\

\haiku{Maar 's avonds zei de,:}{oude vrouw toen zij het oom}{Pieter verteld had}\\

\haiku{Het was                     zijn met -....}{voorkeur gekozen leven}{het onbekende}\\

\haiku{geen hoogvlieger,                     .}{en zonder geestelijke}{belangstellingen}\\

\haiku{Frederik lachte.}{bij dit alles zichzelf en}{de anderen uit}\\

\haiku{Het is sterker dan -....}{vrouw en kind                    en vader}{en moeder samen}\\

\haiku{als een vrouw                     gel\'o\'ofd,.}{heeft in een man en hij gooit}{dat geloof kapot}\\

\haiku{Ze dacht:                    hoev\'e\'el.}{dingen worden er in een}{vrouw kapot gemaakt}\\

\haiku{Annette wist het,,.}{en Frederik wist het maar}{hij vroeg er nooit naar}\\

\haiku{En misschien komt hij -.}{ook heelemaal                     niet meer}{uit hij wordt te zwaar}\\

\haiku{Hij was er te                     .}{jammerlijk aan toe om hun}{lachje te zien}\\

\haiku{Wonderlijk langzaam....}{lichtte ze het deksel van}{de                     terrine}\\

\haiku{Vaag en verward in,....}{haar gloeiend                     hoofd zag ze}{Frederik zitten}\\

\haiku{{\textquoteright} Annette keek naar,,.}{hem naar zijn smalle nog zoo}{jeugdige figuur}\\

\haiku{{\textquoteright} {\textquoteleft}Dan maar niet deftig -!}{ik ben net zoo goed bij den}{slager op den hoek}\\

\haiku{Waarover dachten en....}{spraken zij ooit dan van sport}{en                     wedstrijden}\\

\haiku{Wipte dan binnen,, '.}{bij Pieter die te werken}{zats avonds laat nog}\\

\haiku{Hoe                     dikwijls ze.}{ook bij hem gezeten had}{als hij benauwd was}\\

\haiku{En Annette dacht.}{aan den morgen toen Stance}{naar Indi\"e ging}\\

\haiku{Dan doe ik 't niet, ', '.}{dan doe ikt niet Dan doe}{ikt lekker niet}\\

\haiku{En in het Paleis.}{voor Volksvlijt ging de cyclus}{der Nibelungen}\\

\haiku{zelden heeft me iets,{\textquoteright}.}{zoo wonderlijk aangedaan}{zei                     Leedebour}\\

\haiku{En telkens ook was.}{in hem de                     triomf zich}{nog vrij te weten}\\

\haiku{Hij vond het leuk door,:}{zijn dorp te loopen                     en}{gegroet te worden}\\

\haiku{manchetje streek                     .}{liefkoozend over het donkere}{hoofd aan haar knie\"en}\\

\haiku{Fritsje - ik vind -?}{ze heel lief die versjes zal}{je ze nooit wegdoen}\\

\haiku{{\textquoteright} zei hij rampzalig, {\textquoteleft}.}{maar beloof dat u niets zegt}{van de                     verzen}\\

\haiku{{\textquoteright} de onderdrukte,:}{jubel in de stem haar}{trof barstte zij uit}\\

\haiku{Het had haar dieper.}{geraakt                     dan zij zichzelf}{wilde bekennen}\\

\haiku{een                     jong meisje.}{was het tengere blonde}{vrouwtje gebleven}\\

\haiku{Nog in veel later.}{jaren zou hij zich dien blik}{herinneren}\\

\haiku{Dikwijls dat jaar zat.}{ook op het matten stoeltje}{bij De Roos Frits Craets}\\

\haiku{Daaruit blijkt hoe de.}{ziel van het volk                     er niet}{door gegrepen is}\\

\haiku{Zij wist, zij                     was:}{niet voor hem geweest wat zij}{nog voor Philip was}\\

\haiku{Maar op zijn plaats in,;}{de stalles weer                     zat hij}{wat afgetrokken}\\

\haiku{Maar ik dacht, als 't,.}{niets is waarom zou ik u}{dan laten schrikken}\\

\haiku{{\textquoteright} {\textquoteleft}Hadt je een waarborg?}{voor Philip toen je dien zijn}{eigen weg liet gaan}\\

\haiku{zij zag haar eigen,.}{leven in de baan waarin}{het gegleden was}\\

\haiku{Zij keek naar de                     ,,:}{zilverkast de bekers de}{prijzen en geeuwde}\\

\haiku{Van Frits waren het.}{brieven zonder verslag van}{eigen ervaren}\\

\haiku{{\textquoteright} dacht De Roos, de                     ,.}{teere handen volgend die}{grepen en wezen}\\

\haiku{{\textquoteleft}Karel, ik moet toch.}{eens met je praten over die}{nieuwe verzen}\\

\haiku{Alleen Frederik.}{hoorde het blijkbaar niet en}{oom Pieter                     zweeg}\\

\haiku{Ze had verwacht een,.}{ongelukkig vrouwtje een}{tragischen aanblik}\\

\haiku{{\textquoteright} {\textquoteleft}Dan is er niets aan - '...}{t is zoo leuk om naar uw}{gezicht te kijken}\\

\haiku{En niet als                     haar {\textquoteleft}{\textquoteright}.}{zuster droomde Sophie}{in ha\`ar huis overthuis}\\

\haiku{Denk aan Die Weber, -.}{Der Biberpelz van Hauptmann}{zelfs Hannele}\\

\haiku{Het heeft zoo moeten -.}{komen op deze wijze}{en in dezen vorm}\\

\haiku{{\textquoteleft}En dat het beste,.}{wat die tijd te geven heeft}{da\`aruit voort zal komen}\\

\haiku{En de wereld is '.}{aant                     veranderen}{met dien nieuwen geest}\\

\haiku{Hij was een adept                     ,,,;}{van Proudhon als journalist}{even scherp vurig raak}\\

\haiku{Doe voor mijn part een -....}{reis naar                     Egypte of klim}{in een luchtballon}\\

\haiku{Annette wachtte,.}{tot zij verder                     spreken}{zou maar zij zweeg weer}\\

\haiku{In de volgende.}{weken ging Annette meer}{naar de Hartoniussen}\\

\haiku{Was het dit, dat haar....}{plotseling terugjoeg in}{de oude baan}\\

\haiku{{\textquoteright} Hij zweeg - hij dacht, in,.}{zijn liefde voor haar helder}{dat het d\`at niet was}\\

\haiku{Langzaam boog zij zich,.}{voorover en opeens schrok de}{slapende                     op}\\

\haiku{Maar nooit kwam terug:}{waar Louise heimelijk}{elk jaar op hoopte}\\

\haiku{En zij vroeg, met haar,:}{handschoenen nog aan wat zij}{iederen dag vroeg}\\

\haiku{Het meisje leek op,.}{haar maar de moeder was vroeg}{oud en afgetobd}\\

\haiku{Kleine donkere.}{zware                     vrouwen hadden}{zijn passie gehad}\\

\haiku{{\textquoteright} Mevrouw Annette.}{zat stil voor haar toilet en}{keek in den spiegel}\\

\haiku{Zij vocht all\'e\'en om -.}{hem te                     begrijpen niet}{vreemd aan hem te staan}\\

\haiku{Maar toen hij tegen:}{Philip uit wou barsten had}{die meteen gezegd}\\

\haiku{D\`at was schoonheid, zoo -.}{goed als Catherine zoo}{goed als De Dochter}\\

\haiku{Hij wist dit zeer goed,.}{maar verwerkte het niet in}{zijn                     gevolgen}\\

\haiku{Ze st\`ond een moment,.}{keerde zich toen om en}{ging de stoep weer af}\\

\haiku{Hij liep de kamer,.}{door zooals hij gewend was zijn}{handen op zijn rug}\\

\haiku{TOEN het morgen was.}{begon de dag voor Jetje Craets}{als een                     luister}\\

\haiku{Ergens op een stoep.}{lagen twee Janhageltjes}{netjes naast elkaar}\\

\haiku{In de armen van,,,!}{een vr\'e\'emden                     man moeder}{zoo dicht bij hem b\`ah}\\

\haiku{Nee waarachtig Jet,,.}{als je getrouwd bent is me}{dat een                     leven}\\

\haiku{Zij had eens gedacht.}{ze Caroline te geven}{als die                     trouwde}\\

\haiku{de                     hoofschheid,,,.}{de begrippen van fatsoen}{van   eer van stand}\\

\haiku{Wa\`ar het vandaan kwam,,;}{w\`at het hen inblies niemand}{zou                     het zeggen}\\

\haiku{Straks zich kleeden - een -.}{diner het                     hoeveelste}{al dezen winter}\\

\haiku{Met zijn altijd toch?}{zoo heldere scherpe oogen}{verkeerd gezien}\\

\haiku{{\textquoteleft}Ja, k\"onnen wir denn -.}{gar nichts dagegen machen}{das ist doch schrecklich}\\

\haiku{{\textquoteleft}Zoo, gelukkig - we,{\textquoteright}.}{dachten je al verloren}{lachte Eug\'enie}\\

\haiku{Pieter ging den kring,:}{rond zijn scherpe oogen ieder}{gezicht monsterend}\\

\haiku{En die heeft er een.}{rijkdom   van oude en}{nieuwe                     meesters}\\

\haiku{Ze waren allen,.}{zoo jong niet meer dat wisten}{ze                     plotseling}\\

\haiku{Dat deed ze altijd, '.}{als een machine die aan}{t afloopen was}\\

\haiku{En onmerkbaar trok.}{hij zich in zijn diepste}{innerlijk terug}\\

\haiku{Ze stond even op, de,.}{cadeautjes in haar armen}{toen hij binnenkwam}\\

\haiku{{\textquoteright} Hij glimlachte, zag.}{dan plotseling hoe een traan}{langs haar gezicht gleed}\\

\haiku{hier stond, eenzaam te.}{kijken naar wat voor hem}{onbereikbaar was}\\

\haiku{Och het was ook maar.}{idee dat je buiten eerder}{zou opknappen}\\

\haiku{Moeder, vader mag.}{in die hitte niet langer}{heen en weer reizen}\\

\haiku{En even staarde ze, -.}{ernaar als betooverd sloop dan}{katzacht                     terug}\\

\haiku{{\textquoteright} vroeg ze, onverwacht.}{den ouden schersenden}{toon terugvindend}\\

\haiku{Hij reisde heen en.}{weer terwijl Frederik}{zijn vacantie nam}\\

\haiku{oma Goldeweijn waar.}{hij met                     Frans spelen ging}{in den mooien tuin}\\

\haiku{Naast hem knuffelde -.}{Mies zijn arm hij keek met zijn}{glimlach op haar neer}\\

\haiku{wat de jongen zei,.}{het was een laatste echo}{uit verganen tijd}\\

\haiku{Belangrijk waren,.}{ze plotseling geworden}{de kinderen}\\

\haiku{Een hand kroop                     naar,.}{achter tot zij belandde}{in Francines schoot}\\

\haiku{Het werd nu tijd dat.}{hij haar opvoeding wat meer}{zelf                     ter hand nam}\\

\haiku{{\textquoteright} {\textquoteleft}En vader vindt hem,{\textquoteright}.}{juist zoo leuk ontsnapte haar}{in spijtig verweer}\\

\haiku{Zij stak kinderlijk.}{haar wollen handschoentje op}{als een laatste groet}\\

\haiku{het andere, het.}{heelemaal vreemde dat}{papa in haar bracht}\\

\haiku{En dan kreeg ze het.}{gevoel of ze alles van}{hier ineens kwijt was}\\

\haiku{{\textquoteleft}Wat die rare vent,',.}{Els vader vertelde}{kon nooit veel zaaks zijn}\\

\haiku{Hij vergat den tijd.}{terwijl zij samen zaten}{uit te kijken}\\

\haiku{Jetje die binnen kwam,:}{hollen alles moest weten}{van een                     concert}\\

\haiku{{\textquoteright} Haar hoofd lag aan zijn,.}{schouder haar hand omklemde}{zijn koude vingers}\\

\haiku{Een nieuw geluid, een:.}{nieuwe gedachte door}{heel Holland vliegen}\\

\haiku{{\textquoteright} Tegen Annette:}{in haar zwaren rouw zei De}{Roos met een glimlach}\\

\haiku{Het was toch al een.}{h\'e\'ele tijd geleden dat}{oom Philip dood was}\\

\haiku{De slag van Philips.}{dood had hem zwaar getroffen}{maar niet verslagen}\\

\haiku{of hij de vreugde.}{om die                     liefde thans nog}{dieper doorproefde}\\

\haiku{Gerda zou dat ook -.}{wenschen niet n\`og eens die hel}{van samenleven}\\

\haiku{Hij was te kiesch om!}{zoo maar meteen daarover te}{kunnen spreken}\\

\haiku{hij moet zwijgen van.}{zijn moeder en praten}{van zijn grootmoeder}\\

\haiku{XIX ANNETTE nog.}{pratende kwam de trap af}{van Carolines asyl}\\

\haiku{Zij lag al op haar,.}{knie\"en en trok voorzichtig}{het dekentje weg}\\

\haiku{Dat is nog van je,{\textquoteright}.}{overgrootmoeder Goldeweijn}{zei Caroline Craets}\\

\haiku{Als er in mijn                     ,.}{tijd z\'o\'o iets geweest was dan}{had ik d\`at gewild}\\

\haiku{of niemand meer                     .}{vast en overgegeven zijn}{leven leven kon}\\

\haiku{'t Had nu eenmaal.}{vast bij hem gestaan dat zijn}{zoon zou studeeren}\\

\haiku{{\textquoteright} Hartonius liep door naar,,.}{zijn kamer smeet de deur dicht}{viel in een stoel}\\

\haiku{Het zekere, de,.}{veiligheid dat                     zalig}{overtuigde was weg}\\

\haiku{Dit kon hij toch maar -....}{niet zonder meer aanvaarden}{elkaar niet zien}\\

\haiku{{\textquoteleft}Maar 't kan hem ook....}{heelemaal niet schelen of}{Jetje hier nog ooit komt}\\

\haiku{Alleen een gevoel.}{of met die tranen zijzelf}{was uitgedord}\\

\haiku{In oorlog heb je -'.}{aan papieren geld niets is}{t eenige munt}\\

\haiku{Hij moet -{\textquoteright} haar lippen - {\textquoteleft}.}{begonnen te                     trillen}{hij moet opkomen}\\

\haiku{Maar je kunt haast geen -.}{aansluiting krijgen ik reis}{van drie uur                     af}\\

\haiku{Verbeeld je dat we.}{niet bij mekaar waren als}{er wat                     gebeurt}\\

\haiku{{\textquoteright} ~ Toen hij thuis kwam,.}{stond hij plotseling in de}{gang voor Fred Melgers}\\

\haiku{{\textquoteright} De jongen zag bleek -.}{zijn                     branie-achtige}{houding verloren}\\

\haiku{Zoo was het immers?}{geweest met Alva en}{de Watergeuzen}\\

\haiku{toen                     boog het vlak,.}{langs de Nederlandsche grens}{af Belgi\"e binnen}\\

\haiku{{\textquoteleft}Ik denk aan al de.}{moeders die nu hun zonen}{verloren hebben}\\

\haiku{Ze keek voor zich, maar,.}{ze zag zijn gezicht zoo slap}{zoo onverschillig}\\

\haiku{De duizeling door.....}{je heen dat zijn gezicht niet}{niet oprecht                     was}\\

\haiku{{\textquoteleft}hoe is er aan ons,}{rustig leven ineens}{zoo'n eind gekomen}\\

\haiku{En den volgenden:}{dag zag Amsterdam het nooit}{gekende schouwspel}\\

\haiku{{\textquoteright} Toen glimlachte Jetje,.}{uit gewoonte maar tranen}{sprongen in haar oogen}\\

\haiku{Als Mies thuiskwam vond.}{die het een pretje zoo'n paar}{uur te helpen}\\

\haiku{zijn oogen dwingend in,.}{de hare riepen haar}{plotseling terug}\\

\haiku{De Roos sloot zijn deur.}{en keek rond met een gevoel}{van verlatenhid}\\

\haiku{Zij stond moeielijk op,.}{eindelijk en hij reikte}{haar zijn hand tot steun}\\

\haiku{XXX KRAUS reed naar huis,.}{van zijn concert het tweede}{in het                     seizoen}\\

\haiku{Hier liep je nu al -....}{die kapstokken langs een voor}{een al die jurken}\\

\haiku{Dat kind - wat frisch en -.}{jong staat je aan te kijken}{of ze een spook ziet}\\

\haiku{Je staat voor vader, -;}{je gaat straks naar je eerste}{bal ja vader}\\

\haiku{Hij wist het, zag het.}{alsof zij het hem zelf nog}{eenmaal gezegd had}\\

\haiku{dit                    oogenblik,,.}{den dood van hem in haar ziel}{niet te overleven}\\

\haiku{{\textquoteleft}Een ziekte, dat zij,,...?}{zoo zuiver zoo eerlijk zoo}{exclusief liefhad}\\

\haiku{Dag en nacht liep de,.}{dijkwacht er speurend                     naar}{de zwakke plekken}\\

\haiku{Hij was                     voor 't, '.}{eerst van zijn leven moet}{liep hem over den kop}\\

\haiku{{\textquoteleft}Alsof 't haar                     .}{heusch schelen kon wat hij}{ervan dacht of vond}\\

\haiku{Nu was hij blijkbaar,.}{hersteld ging een                     kunstreis}{doen in Amerika}\\

\haiku{De jongsten zaten,,.}{erbij wat verwonderd en}{meest onverschillig}\\

\haiku{Zijn oog                     gleed naar,.}{hun korte rokken die het}{been zichtbaar lieten}\\

\haiku{En hij zag helder:}{en scherp dit opgroeiende}{vrouwengeslacht}\\

\haiku{Dit alles dacht}{Frederik en hij voelde}{met al zijn liefde}\\

\haiku{de Engelsche                     ;}{ge{\"\i}nterneerden voerden}{daar het hoogste woord}\\

\haiku{hij keek naar Betsy,,.}{die zat daar nerveus de}{vuisten verknepen}\\

\haiku{Hij lichtte even zijn,.}{hoed de oude dame neeg}{nauw merkbaar het hoofd}\\

\haiku{Als zij 's avonds in,:}{bed lagen luisterden zij}{hoorden aan de klok}\\

\haiku{ze liet zich languit, {\textquoteleft}.}{vallenper slot is Van}{Loo \'o\'ok maar een man}\\

\haiku{Ik was als de dood.}{voor die koe waar we altijd}{langs moesten naar school}\\

\haiku{Ik kon niet studeeren.}{als er koffie gemalen}{werd in de keuken}\\

\haiku{In dit oogenblik,.}{besefte hij brak wat zijn}{leven waarde gaf}\\

\haiku{ze lag bij hem op,-}{haar knieen greep naar zijn hand die}{hij wegrukte}\\

\haiku{maar zijn moeder had '....}{er altijds avonds haar muts}{op gehangen}\\

\haiku{Ik heb 't vroeger,.}{veracht idioot gevonden}{en minderwaardig}\\

\haiku{Nu benijd ik ze -.}{nu ben ik jaloersch op}{iedereen die werkt}\\

\haiku{{\textquoteleft}De oude juffrouw,{\textquoteright}.}{Leedebour                    zeiden de}{kleinkinderen Craets}\\

\haiku{Maar haar gedachten.}{gingen daaroverheen den}{eigen ouden weg}\\

\haiku{{\textquoteright} {\textquoteleft}Er is m\'e\'er in 't.}{jonge leven van dezen}{tijd dan huwelijk}\\

\haiku{{\textquoteright} {\textquoteleft}Neem me niet kwalijk,.}{ik ben maar een huismusch en}{dacht da\`ar het laatst aan}\\

\haiku{Maar aan het front - de,,,,.}{Tsaar wankel geslingerd niet}{begrijpend blijft doof}\\

\haiku{{\textquoteright} Plotseling was ze -.}{vlak bij hem haar twee handen}{om zijn arm geklemd}\\

\haiku{Het was laat toen hij.}{de deur van zijn eigen huis}{eindelijk opensloot}\\

\haiku{Ze moest lachen, zooals.}{ze sinds hun kinderjaren}{gelachen hadden}\\

\haiku{Zij stond afgewend '.}{voort raam tot zij de}{deur hoorde sluiten}\\

\haiku{Met Els wandelde,,....}{hij voor Els had hij                     tijd}{met Els praatte hij}\\

\haiku{Ik                     zal minder.}{eenzaam zijn bij vreemden dan}{in mijn eigen huis}\\

\haiku{In de Vijzelstraat.}{voor een sigarenwinkel}{dromden de menschen}\\

\haiku{staarden uit het glas.}{de luxueuse slaapkamer}{in als een geheim}\\

\haiku{{\textquoteleft}Je hebt dit keer lang -?}{gewacht met antwoorden}{schrijf je nu weer gauw}\\

\haiku{Zij zei het niemand -.}{dat zij dikwijls benauwd was}{een kramp in haar borst}\\

\haiku{Hij                     meende een,.}{ooglid te zien trillen een}{lichte ademhaling}\\

\haiku{Tot Frederik met.}{geweld het gesprek bracht op}{litteratuur}\\

\haiku{Annette hoorde:}{in hun jonge stemmen het}{leven van den tijd}\\

\haiku{Hoe makkelijk, hoe.}{snel                     stelde zijn vrouw een}{ander in zijn plaats}\\

\haiku{de kleine oude.}{hand tegen haar gloeiend}{beschreid gezicht}\\

\haiku{Je hebt nu eenmaal,{\textquoteright}.}{je kinderen niet in je}{hand prevelde hij}\\

\haiku{Och man, er is geen!}{machteloozer ding op de}{wereld dan liefde}\\

\haiku{Dat zijn moeder een.}{ander                     stelde op de}{plaats van zijn vader}\\

\haiku{De jonge Seb, een,:}{leelijken trek om zijn mond}{keek naar zijn moeder}\\

\haiku{En daarna bleef hij,.}{innerlijk eenzaam met een}{diep gemis in zich}\\

\haiku{In zijn ego{\"\i}ste,:}{egocentrische ziel was de}{eene zachte                     plek}\\

\haiku{het volk in Rusland,.}{losgebroken heeft                     de}{regeering in handen}\\

\haiku{Dit alles zagen.}{Frederik Craets en zijn}{tijdgenooten}\\

\haiku{Want ook in eigen.}{huis randde de nieuwe geest}{hen ouderen aan}\\

\haiku{Op de boot waar het,.}{stampvol was voer ze het}{donkere IJ over}\\

\haiku{Tot                     Melgers in.}{de eerste plaats als de meest}{hulpbehoevende}\\

\haiku{Ze was te oud om.}{zich over die vreemde menschen}{te vermoeien}\\

\haiku{De waarde                      van.}{een mensch is zelden voor den}{tijd waarin hij leeft}\\

\haiku{{\textquoteright} {\textquoteleft}Ja, maar jij gaat met.}{die boerentrienen om als}{met je gelijken}\\

\haiku{Toen had de ander -.}{haar stom verbaasd aangezien}{en dan gelachen}\\

\haiku{Wat v\`onden zij er '?}{dan ins hemelsnaam voor}{onfatsoenlijks in}\\

\haiku{{\textquoteright} {\textquoteleft}Ach,{\textquoteright} zei de oude, {\textquoteleft}.}{Annettedat begrijp ik}{nu zelf ook niet meer}\\

\haiku{Hij zocht het - niet zooals;}{zij dachten om zich te doen}{bewierooken}\\

\haiku{Die al wat met                     ,.}{zijn huwelijk verbonden}{was ook hebben wou}\\

\haiku{Ann\`etje hing hem,;}{aan maar jaloerscher nog zag}{Betsy dat                     Seb}\\

\haiku{Of was dit alleen....}{ongegronde angst van}{te zwaar denkenden}\\

\haiku{Zij had                     aan haar}{vaders deur geklopt en geen}{antwoord krijgend was}\\

\haiku{Hester ging den kring -,}{rond zij allen ontroerden}{omdat zij haar zoo}\\

\haiku{Hij dacht aan wat hij.}{haar                     had hooren zeggen}{tegen Ann\`etje}\\

\haiku{Het kon hem niet veel,.}{schelen wat zij                     zeiden}{de oude menschen}\\

\haiku{Maar ze speurde snel,.}{sluw zijn                     verlatenheid}{zijn hulpeloosheid}\\

\haiku{Hij                     was zich ook}{nu zeer wel bewust dat zij}{niets leek op de vrouw}\\

\haiku{Hij zag thans in een:}{diepe vermoeidheid nog slechts}{dat eene                     verschiet}\\

\haiku{, en hij                     voelde.}{haar zachte vingers bevend}{aaien over zijn hoofd}\\

\haiku{{\textquoteright} Zij nam zangles, en.}{op een avond had zij Schubert}{voor hen gezongen}\\

\haiku{Vliegtuigen ziet men,,.}{waaruit de bommen de}{granaten ploffen}\\

\haiku{Maar                     dan is er:}{plotseling aan het front zelf}{een nieuwe vijand}\\

\haiku{De lamp brandde laag -.}{zij bleef stil                     zitten om}{hem niet te storen}\\

\haiku{maar die, als alles,.}{van haar familie haar}{toch ook beklemd heeft}\\

\haiku{Hij heeft nu reeds zoo.}{dikwijls aan het sterfbed van}{een vriend                     gestaan}\\

\haiku{en Fred herinnert....}{zich de aardige brieven}{die opa hem schreef}\\

\haiku{Haar sidderende.}{handen hieven het doode}{lichaam teeder op}\\

\haiku{Het eerste wat zij:}{zag was de kamer in de}{grootste wanorde}\\

\haiku{Sloeg en vernielde,.}{wat nog de wapens de}{honger had gespaard}\\

\haiku{O ja, alles met,.}{die pijn binnenin je die}{\`al                     erger werd}\\

\haiku{{\textquoteleft}Dacht je vader, dat?}{niet iedereen ergens een}{vooze plek heeft zitten}\\

\haiku{{\textquoteright} {\textquoteleft}U is geen Craets,{\textquoteright} zei.}{hij met den eersten glimlach}{door zijn moeiten heen}\\

\haiku{hij stelt zich met het.}{leger ter                     beschikking}{der nieuwe regeering}\\

\haiku{Op Zondagmorgen}{eindelijk den Duitschers in}{hun trein overhandigd}\\

\haiku{En toen was er nog,:}{die andere beroering}{in het eigen land}\\

\haiku{{\textquoteright} Mies overeind, schreide;}{als een ongelukkig kind}{tegen Jenny aan}\\

\haiku{Net als je eens                     !}{voor je plezier gezellig}{met je hond uit was}\\

\haiku{{\textquoteright} {\textquoteleft}Neen, dat ik een ezel....}{ben vertelden mijn vrouw en}{mijn dochters me al}\\

\haiku{Hun vader is in.}{een                     gevangenkamp in}{Frankrijk gestorven}\\

\haiku{Zij zelf zweeg wat zij.}{naderen zag alsof het}{haar reeds gezegd was}\\

\haiku{{\textquoteright} Haar stem sloeg over, haar.}{handen beefden toen ze den}{rouwsluier neerstreek}\\

\haiku{{\textquoteleft}Dacht je later soms?}{je huishouden ook te doen}{met zoo'n doek om}\\

\haiku{{\textquoteright} {\textquoteleft}Ik b\`en geen vrouw als.}{oma Craets en jouw moeder en}{als moeder Hartonius}\\

\haiku{ik zou alleen je;}{vrouw kunnen worden als je}{me                     d\'a\'arin vrij liet}\\

\haiku{{\textquoteright} Het ging door haar heen,.}{d\`at ze geen man wilde die}{zich aan ha\`ar vasthield}\\

\haiku{{\textquoteright} Toen hij de Leliegracht,.}{opkwam liep Ann\`etje hem}{haast in de armen}\\

\haiku{En tusschen Hartonius.}{en                     haarzelf groeide een}{nieuwe late band}\\

\haiku{M'n kleine dot - mijn -....}{snoezig diertje ben je}{dan mijn hartedief}\\

\haiku{{\textquoteleft}Het is ook mijn deel.}{niet meer die groote kinderen}{te begrijpen}\\

\haiku{{\textquoteright} En de jonge vrouw,,.}{de oogen helder stak haar}{arm door dien van Fred}\\

\haiku{{\textquoteright} Maar alleen Mies kreeg.}{een dieper inzicht in dit}{jonge huishouden}\\

\haiku{Aan haar zoon Pieter,.}{dacht Annette al noemde}{zij zelden zijn naam}\\

\haiku{Zij zag er in haar.}{donkere japon rustig}{bevredigd uit}\\

\haiku{Bij de beroering,,:}{de verbazing die het in}{de familie bracht}\\

\haiku{Er fluistert iets, dat -....}{niet meer is dan een ademtocht}{dat lijkt op een naam}\\

\subsection{Uit: Vrouw Jacob}

\haiku{Hij was waarschijnlijk,,.}{de eenige die dit inzag}{en daarom zweeg hij}\\

\haiku{in haar eerend het.}{nieuw opgroeiend ideaal van}{het ridderwezen}\\

\haiku{Een niet te breken}{wilskracht had zij ge\"erfd van}{vader en moeder}\\

\haiku{Arkel heimelijk.}{gesteund door zijn zwager Van}{Gelre en Gulik}\\

\haiku{Was niet zijn gansche!}{regeering bemoeilijkt door de}{vervloekte Arkels}\\

\haiku{Zoo kind - dat gebroed,.}{is ten onder gebracht v\'o\'or}{het jou kwaad kan doen}\\

\haiku{Beneden is een.}{ridder met een boodschap van}{Hare Majesteit}\\

\haiku{En zij voelde zich,....}{koud worden in een gevoel}{van verlatenheid}\\

\haiku{Vrouwe Jacoba.}{zal beter passen op den}{Franschen koningstroon}\\

\haiku{En zij, aanziende,.}{deze mannen wist zich op}{eenmaal geen kind meer}\\

\haiku{Zij luisterde naar,;}{dat diep en scheurend brullen}{beklemmend geboeid}\\

\haiku{in verteedering,.}{voor het kind in vereering voor}{de ontwaakte vrouw}\\

\haiku{En ik houd van de -,.}{Henegouwers die Franschen}{ik vertrouw er geen}\\

\haiku{In den nacht lagen,,.}{zij dicht tegen elkaar in}{het groote statiebed}\\

\haiku{{\textquoteleft}De Dauphin zal niet,!}{anders in Parijs komen}{dan met Bourgondi\"e}\\

\haiku{Hij tastte naar haar {\textquoteleft} -{\textquoteright}, {\textquoteleft}.}{hand.Jaque stamelde hijlaat}{me niet meer alleen}\\

\haiku{De berichten die,.}{uit Bouchain naar Quesnoy kwamen}{luidden weinig nieuws}\\

\haiku{haar teederste vriend,.}{de verafgode vader}{ging van haar heen}\\

\haiku{{\textquoteright} Jacoba hoorde,.}{het alles aan met niet te}{doorgronden kalmte}\\

\haiku{En even klopt haar hand,,:}{de sterke gespierde den}{hals  van haar paard}\\

\haiku{hier is - ho\`e is niet -;}{te vatten een andere}{wereld geworden}\\

\haiku{De uitkomst zoeken.}{uit deze verwarring van}{wenschen en afkeeren}\\

\haiku{De Bourgondi\"er;}{hield zichzelf bescheiden uit}{den familieraad}\\

\haiku{{\textquoteleft}Als uw oom zal ik,{\textquoteright}.}{ook in Brabant wel komen}{besloot hij milder}\\

\haiku{De laatst gesproken.}{zinnen bleven hangen in}{de stilte om hen}\\

\haiku{Jacoba ontving.}{hem plechtig en gratievol}{in de ridderzaal}\\

\haiku{Ge hebt mij,{\textquoteright} sprak hij -.}{verteederd ondanks zichzelf}{om haar lieve jeugd}\\

\haiku{Naar Dordrecht begaf,.}{hij zich en de stad haalde}{hem jubelend in}\\

\haiku{Gij hebt niet genoeg.}{aanhang om den Hoeken de}{macht te ontnemen}\\

\haiku{Oh, schooner, heerlijker,!}{beteekenisvoller aanval}{was er nooit gedaan}\\

\haiku{waarop elke man.}{zich bereid houdt voor haar zijn}{leven te geven}\\

\haiku{Ze zijn niet tegen,!}{te houden de Hoeken niet}{en de onzen niet}\\

\haiku{Voor haar staat de groote,.}{gestalte van Brederode}{tot den strijd gerust}\\

\haiku{Tot het laatste toe;}{heeft Arkel gewacht met het}{sein tot den aanval}\\

\haiku{In dit helsch gedrang.}{is Brederode verder van}{Arkel afgeraakt}\\

\haiku{Door het heele land.}{was de verslagenheid der}{Kabeljauwen groot}\\

\haiku{Zij trok naar Den Haag,,.}{Vrouw Jacob en alles viel}{haar jubelend toe}\\

\haiku{Den Beier fnuiken, -!}{wa\`ar zij kon had hij haar niet}{den liefste gemoord}\\

\haiku{Reeds driemaal waren.}{de landen door den keizer}{aan vrouwen beleend}\\

\haiku{Jacoba in haar,.}{overgevoeligheid ving het}{op en huiverde}\\

\haiku{De verovering;}{van Brielle kon alleen te}{water geschieden}\\

\haiku{Niet bezielde haar.}{vreugde en hoop als bij den}{tocht naar Gorcum}\\

\haiku{de belegering.}{van de machtigste stad in}{den wereldhandel}\\

\haiku{En een  nimbus;}{omgeeft het hoofd van Dame}{Jaque die dit aandurft}\\

\haiku{Daar binnen de stad;}{ziet Jan van Beieren de}{vijandin komen}\\

\haiku{en de burgers zijn.}{welgewapend en op een}{beleg voorbereid}\\

\haiku{Maar in het leger,.}{zelf wordt nu reeds gebrek aan}{proviand gevoeld}\\

\haiku{maar er is geen toorn,.}{in hen slechts radeloosheid}{om eigen onmacht}\\

\haiku{In den nacht nog gaat.}{de bode met den brief naar}{het Brabantsche kamp}\\

\haiku{De roem van Gorcum.}{is vergeten en door al}{dezen smaad gefnuikt}\\

\haiku{Rotterdam was bij.}{verrassing den Beier in}{handen gevallen}\\

\haiku{{\textquoteleft}Zoudt ge den Beier?}{tot een onderhoud kunnen}{bewegen met ons}\\

\haiku{Met scherpen blik mat,.}{hij den man die zich zwijgend}{op een knie neerliet}\\

\haiku{{\textquoteright} {\textquoteleft}Gijlieden verraadt?}{en verkoopt uw hertogin}{achter haren rug}\\

\haiku{Alleen gelaten,,.}{bleef hij lang stil zitten het}{hooge voorhoofd gefronsd}\\

\haiku{Zij begreep wat hem?}{hier had gebracht in opdracht}{van zijn heer vader}\\

\haiku{De Beier was er -,!}{meester van erfelijk leen}{wat beteekende d\`at}\\

\haiku{Maar l\'a\'at ons tijd, laat -.}{ons even rust we k\`unnen niets}{op dit oogenblik}\\

\haiku{de vurige, licht;}{opkomende blos liep tot}{in haar blanken hals}\\

\haiku{{\textquoteright} Dien avond liet Van den.}{Berghe zich bij den hertog}{verontschuldigen}\\

\haiku{{\textquoteleft}Zij zijn ontkomen,, -....}{Heer in de verwarring men}{heeft hen niet herkend}\\

\haiku{Hij zal zich, nu hij,.}{verward en verlaten is}{aan u vastgrijpen}\\

\haiku{Dame Jaque was de -!}{sterkste geweest en ho\`e had}{zij toegeslagen}\\

\haiku{Pas op, dat ge niet,.}{verliest wat ge met zooveel}{zorg hebt opgebouwd}\\

\haiku{Een verdrietige,,.}{booze uitdrukking rimpelde}{haar jong glad gelaat}\\

\haiku{- er werd daar wegens....}{een groote overstrooming zeer}{naar zijn komst verlangd}\\

\haiku{{\textquoteleft}Wat beklaagt ge u,,?}{Genadige Vrouwe over}{gebrek aan eerbied}\\

\haiku{het woord te nemen?}{zonder daartoe verlof te}{hebben gekregen}\\

\haiku{{\textquoteleft}Er valt aan eenmaal.}{uitgesproken besluit niet}{te veranderen}\\

\haiku{{\textquoteright} Jacoba stond zoo,.}{recht dat haar lichaam scheen een}{strak gespannen veer}\\

\haiku{Naakt en uitgeschud.}{bleef zijn lijk den ganschen nacht}{op de plek liggen}\\

\haiku{Gij zijt toch immers -!}{zoo'n krijgsoverste zoo beroemd}{in het oorlogsveld}\\

\haiku{Het zal op uw hoofd,!}{neerkomen z\'o\'o dat ge u}{niet meer te keeren weet}\\

\haiku{{\textquoteleft}Bekruisig u niet,,!}{tegen mij maar tegen uzelf}{tegen uw vrienden}\\

\haiku{Mijn gemalin had}{een beteren kijk op de}{toestanden dan gij}\\

\haiku{Toen bond Tserclaes in,,.}{voer in Jacoba's schuitje}{raadde verzoening}\\

\haiku{in dit warnet van.}{vijandelijkheid wist hij}{zich niet te redden}\\

\haiku{Zij ontmoette den,;}{trouwen bezorgden blik van}{Marie van Nagel}\\

\haiku{Zij zag opeens zich:}{te Woudrichem en hoorde}{Vianen zeggen}\\

\haiku{Hij wist, dat niemand.}{er haar toe zou krijgen dit}{te bezegelen}\\

\haiku{Ge hebt Dame Jaque,{\textquoteright}.}{ook op zij willen zetten}{troefde de knaap kwaad}\\

\haiku{{\textquoteright} dacht hij verlangend - {\textquoteleft}.}{het leven zou zoo veilig}{en vroolijk wezen}\\

\haiku{Deze bracht hem de....}{berichten van den Beier}{en zijn operaties}\\

\haiku{en tegelijk wist.}{hij hem de oorzaak van al}{deze ellende}\\

\haiku{{\textquoteleft}Uw Vrouwe is in,{\textquoteright},.}{Brabant zei Tserclaes een der}{volgende dagen}\\

\haiku{een verzoening tot.}{stand te brengen tusschen u}{en uw gemalin}\\

\haiku{Hij voelde zich hier:}{in Den Bosch rampzaliger}{nog dan in Brussel}\\

\haiku{Alleen daarom al.}{was het haar een genoegen}{hem te dwarsboomen}\\

\haiku{Hoek schonk - den mannen,....}{van haar Raad een oud vriend van}{het hof haars vaders}\\

\haiku{Hij had haar in zijn,;}{ridderlijke eerlijke}{jongensziel zeer lief}\\

\haiku{Ge komt mij te Mons -.}{bezoeken met het tournooi}{op Driekoningen}\\

\haiku{Deze nacht werd wel.}{misschien de bitterste uit}{dezen ganschen tijd}\\

\haiku{De zon gaat op, een,.}{stralende lentedag een}{gelukkig voorteeken}\\

\haiku{Nu - nu eindelijk.}{voelt zij zich de bruid van Humphrey}{van Glocester}\\

\haiku{des konings waren.}{Engeland's hoogste adel en}{de vorsten bijeen}\\

\haiku{hetgeen de zwaarste.}{kerkelijke straffen zou}{brengen over hun hoofd}\\

\haiku{Scherp, honend en uit,!}{de hoogte had het meiske}{gedaan tegen h\`em}\\

\haiku{Nu n\`og voelde zij!}{zich besmeurd ooit naast den knaap}{te hebben geleefd}\\

\haiku{In dien tijd zal de.}{dood den stokouden bisschop}{toch wel eens halen}\\

\haiku{Geen genegenheid,.}{eigenbaat alleen had hen}{hem doen aanhangen}\\

\haiku{Jan van Beieren.}{viel in de stad en dempte}{in bloed elk verzet}\\

\haiku{Zij liet liever het.}{land te gronde gaan dan dat}{zij het hem gunde}\\

\haiku{En feller joeg zij,}{de Hoeken die immer in}{grooter getale}\\

\haiku{Treedt gij alzoo den!}{laatsten wil van uw grooten}{koning met voeten}\\

\haiku{De Brabander heeft!}{de gravin van Holland tot}{zijn bijzit gehad}\\

\haiku{liet zonder verweer.}{daarna den toorn van Philips}{over zich heen woeden}\\

\haiku{{\textquoteleft}Doulce chose est,....}{que mariage Je le puis}{bien par moy prouver}\\

\haiku{Jacoba opnieuw.}{verloor zich in gemijmer}{over deze Odette}\\

\haiku{Aan den blaaskaak, dien,!}{hij haatte die hem als een}{knecht had behandeld}\\

\haiku{Echter bezwoer hij,.}{Winchester te zorgen dat}{Humphrey niets ondernam}\\

\haiku{Welnu,{\textquoteright} lachte hij - {\textquoteleft}?}{dan kortademigWat kost}{mij mijn zielerust}\\

\haiku{Hij noemde vlak en.}{zonder nadruk als het doel}{van zijn reis Holland}\\

\haiku{Hij had blijkbaar ook;}{daar in den hof te lang in}{de vocht gezeten}\\

\haiku{Ge begrijpt, zwager,;}{dat ge op mijn troepen niet}{kunt rekenen meer}\\

\haiku{{\textquoteleft}Well, 't lijkt me een -.}{goed land alleszins waard het}{te veroveren}\\

\haiku{Hij boog de knie en {\textquoteleft},,}{kuste haar hand.Mijn vriend mijn}{altijd getrouwe}\\

\haiku{Sinds Albrecht's regeering.}{waren zij vijanden van}{het Beiersche Huis}\\

\haiku{{\textquoteright} Een groote beweging;}{beroerde als een golfslag}{de Nederlanden}\\

\haiku{Hij had, wat hemzelf,;}{verwonderde geen wrok meer}{tegen Jacoba}\\

\haiku{Hij dacht, hoe weinig.}{hij haar bijzijn in rust had}{kunnen genieten}\\

\haiku{Haar hand op zijn arm,,;}{gebleven verlaten en}{hulpeloos gleed af}\\

\haiku{En daar op eenmaal.}{stond weer die ellendige}{Hollandsche kwestie}\\

\haiku{Straten, stadhuizen,.}{landschappen trokken langs zijn}{schemerende oogen}\\

\haiku{Het had haar van geen,.}{belang geleken indien}{zij Humphrey slechts bezat}\\

\haiku{De heele sfeer van.}{haar kinderjaren omspon}{haar vertrouwd en warm}\\

\haiku{Maar het Engelsche - -?}{leger dat de redding moest}{brengen waar bleef het}\\

\haiku{{\textquoteright} Goud rinkelt neer in.}{zijn haastig en begeerig}{gespreide handen}\\

\haiku{hij zou tegen een '.}{slag int open veld niet meer}{opgewassen zijn}\\

\haiku{Eindelijk zal ik.}{deze heele kwestie uit}{de wereld helpen}\\

\haiku{Toen plotseling in.}{dat ondragelijk zwijgen}{rees de hertogin}\\

\haiku{Die thans naar voren,.}{drong in een plotselingen}{heftigen argwaan}\\

\haiku{De weg naar Calais,{\textquoteright}.}{bracht haar gebroken stem op}{een oogenblik uit}\\

\haiku{Dacht, gepijnigd om,.}{haar aan het glorierijke}{begin van den tocht}\\

\haiku{Het werd een  spel,,.}{waarin al wat Hoeksch was zijn}{wilde vreugde vond}\\

\haiku{Thans is er voor u,.}{slechts \'e\'en weg die u in uw}{eer zal herstellen}\\

\haiku{Gij zegt Neef,{\textquoteright} zeide, {\textquoteleft}.}{Philipsdat ge den oorlog}{meer dan moede zijt}\\

\haiku{Hij was niet tegen,.}{haar opgewassen maar de}{Bourgondi\"er wel}\\

\haiku{Nu is hij binnen,.}{de stad gekomen en klaagt}{het vorstenvolk aan}\\

\haiku{zij nemen ook de.}{hertogin-moeder weer}{op in hun midden}\\

\haiku{Uit de strakke lucht,.}{brandt de heete zon en het}{regent nog steeds niet}\\

\haiku{En haar wettige!}{man staat voor de muren en}{belegert de stad}\\

\haiku{{\textquoteright} Jacoba voelt hun,.}{afval als een mist die haar}{kil van hen afsluit}\\

\haiku{er is ook gebrek -.}{in de stad er is vrees en}{wankelmoedigheid}\\

\haiku{{\textquoteright} Maar een razende.}{vloed van stemmen verdrinkt en}{overspoelt de zijne}\\

\haiku{Donderend valt zijn,,....}{stem waarvoor alles altijd}{zweeg in het rumoer}\\

\haiku{In haar slaapvertrek,,.}{ligt zij voorover haar gelaat}{diep in het kussen}\\

\haiku{hij zal u alles,...... ......}{uitvoeriger zeggen dan}{ik het schrijven kan}\\

\haiku{En haar gelaat is,....}{zeldzaam roerend als zij hen}{een voor een aanziet}\\

\haiku{Alles in den burcht,.}{lijdt onder de hitte de}{meedoogenlooze zon}\\

\haiku{Zij merkt alles op,,.}{er ontgaat haar niets bij hen}{die nog om haar zijn}\\

\haiku{Dat volk, ho\`e is niet,....}{te begrijpen opeens aan}{de macht gekomen}\\

\haiku{Even slaat zij den blik,,.}{naar boven naar den burcht waar}{zij woonde met h\`em}\\

\haiku{Doe het mij niet aan,!}{u te zien opgeofferd}{aan mijn vijanden}\\

\haiku{Zag den avond terug,.}{toen de Beier Arkel en}{hem had doen roepen}\\

\haiku{Dit alles wist hij,.}{zeer wel en hij moest hierin}{Egmond bijvallen}\\

\haiku{haar vorstelijken,,.}{tooi waarin zij hem overtrof}{haar te ontnemen}\\

\haiku{Alsof haar geest die.}{uren ontsnappen wilde aan}{martelend denken}\\

\haiku{Zij laat een kluwen,.}{vallen het rolt tot vlak aan}{Jacoba's voeten}\\

\haiku{Weg met Jacoba,,....}{naar Rijssel voorloopig}{de sterke vesting}\\

\haiku{Zij zal in Holland,,,.}{zijn en hem aanvallen den}{roover en overwinnen}\\

\haiku{Geen paard meer onder -....}{zich het langzame rijden}{van een boerenkar}\\

\haiku{Soms ontmoeten zij,;}{ruiters die even naar binnen}{kijken in de kar}\\

\haiku{Over het water, waar,.}{de riemen plassen rijst de}{toren van Gorcum}\\

\haiku{waaruit thans een hand.}{een klein vaandel heft met de}{Beiersche kleuren}\\

\haiku{Ze heeft geslapen -,!}{ho\`e goed hoe veilig diep en}{rustig geslapen}\\

\haiku{In Zeeland brengt Van.}{Haemstede een aanzienlijk}{leger op de been}\\

\haiku{Jacoba was in,.}{Gouda gebleven had er}{haar hof gevestigd}\\

\haiku{en de eilanden.}{en groote steden onder hen}{bleven vijandig}\\

\haiku{Zij is veranderd -.}{een machtig flu{\"\i}dum schijnt}{van haar uit te gaan}\\

\haiku{de weg naar den troon,.}{van Engeland het einddoel}{van al haar streven}\\

\haiku{hoe heeft ze gedaan,?}{in Henegouwen hoe heeft}{zij dien trouw beloond}\\

\haiku{XI Het volk in de:}{Nederlanden zag wat het}{nog nooit had gezien}\\

\haiku{De Hoeksche steden,,.}{al wat zich voor Jacoba}{verklaard had lachten}\\

\haiku{{\textquoteleft}Het is de Beier,,.}{geweest hertog Jan die het}{\`al heeft bedorven}\\

\haiku{{\textquoteleft}Ik voor mij w\'e\'et, dat,.}{wie zich aan uw inzicht houdt}{veilig is bewaard}\\

\haiku{{\textquoteright} Zij stond plotseling,.}{stil of zij daadwerkelijk}{voor een afgrond stond}\\

\haiku{{\textquoteright} {\textquoteleft}Als zij met goede....}{loodsen aan boord het juiste}{oogenblik kiezen}\\

\haiku{En n\`u eindelijk,:}{gebeurde wat Marie van}{Nagel verwachtte}\\

\haiku{Op d\`at oogenblik.}{davert de lucht van beider}{kanten krijgsgeschrei}\\

\haiku{Hoeksch of Kabeljauwsch -....}{wat doet het jong is zij en}{mooi en verlaten}\\

\haiku{Eindelijk rechtte;}{Van Nagel het grijze hoofd}{en wilde spreken}\\

\haiku{{\textquoteleft}Als ge haar beter,;}{hadt doen bewaken was dit}{alles niet geschied}\\

\haiku{een  licht rood op,.}{zijn bleekt wang bewees dat de}{pijl had getroffen}\\

\haiku{Uren en uren, tot laat,.}{in den nacht soms verbleef zij}{met haar getrouwen}\\

\haiku{Grepen hun wapens.}{van den muur en telden hun}{weerbare mannen}\\

\haiku{Het was een gloed, die.}{als een ketting van vuur liep}{door de gansche streek}\\

\haiku{en optrok met een.}{grooter en sterker leger}{dan ooit te voren}\\

\haiku{En \`op rukten van.}{alle zijden Jacoba's}{troepen naar de stad}\\

\haiku{v\'o\'or Philips met zijn,.}{groote macht aankwam moest de stad}{in hun handen zijn}\\

\haiku{Maria sta ons bij -!}{het schip met den standaard van}{Vrouw Jacob in top}\\

\haiku{Vrouw Jacob is hij,.}{toegedaan met lijf en ziel}{al kent zij hem niet}\\

\haiku{Bij de Kennemer,.}{boeren buiten de stad is}{de stemming gekeerd}\\

\haiku{In den laten avond,,;}{soms alleen vond haar Marie}{starend naar buiten}\\

\haiku{{\textquoteright} Het wordt groot - het vult - -.}{de kamer de gansche stad}{het geheele land}\\

\haiku{Niet haar trots, niet haar,,, -...?}{roem haar heerlijkheid haar kracht}{haar schoonheid maar dit}\\

\haiku{Hij ziet de boeren.}{terugdeinzen onder de}{moordende pijlen}\\

\haiku{geen Kennemer of.}{Westfries mocht ooit weer wapen}{of harnas dragen}\\

\haiku{{\textquoteright} Haar vuisten gebald,.}{tegen haar witte lippen}{geperst kreunde zij}\\

\haiku{{\textquoteright} Marie die zwijgend,.}{haar leed begroef en stil trouw}{naast haar voortleefde}\\

\haiku{Jacoba lachte.}{haar vreugdeloozen lach als zij}{dit alles hoorde}\\

\haiku{Oh, gruwelijker,!}{verbijsterender dan \'e\'en}{mensch op aarde wist}\\

\haiku{En voor 't eerst zag:}{zij de onbezonnenheid}{van die vlucht over zee}\\

\haiku{En haar rechten op.}{het hertogdom Luxemburg en}{het graafschap Chimay}\\

\haiku{{\textquoteright} ~ De pen kraste,,,.}{de woorden klonken helder}{bewust scherpzinnig}\\

\haiku{En hield dien thans aan,.}{haar borst gedrukt al den tijd}{dat zij met hem sprak}\\

\haiku{Op een dag, komen:}{burgers en dringen aan bij}{den bevelhebber}\\

\haiku{Jacoba zag ze,.}{uit haar raam en klemde de}{handen in den schoot}\\

\haiku{plicht, ja w\'a\'arlijk plicht,!}{te zorgen dat zijn erfgoed}{niet verloren ging}\\

\haiku{Een tweede schrijven.}{richtte hij aan de Lords van}{den Geheimen Raad}\\

\haiku{Hij weet, dat alle.}{tegenstand u slechts prikkelt}{om vol te houden}\\

\haiku{Mijn werk - uw schoone - -.}{droom de droom van een kroon in}{Engeland heeft uit}\\

\haiku{Een jonge vrouw in -.}{wapenrusting stond voor hem}{een z\'e\'er jonge vrouw}\\

\haiku{In de voorzaal zijn,,....}{bijeen Van Kijfhoeck Van de}{Merwede Montfoort}\\

\haiku{En Gode zij dank,....}{den Engelschman heeft zij}{opzij geworpen}\\

\haiku{Zij is koud tot in,.}{haar gebeente of zij nooit}{meer warm worden zal}\\

\haiku{Philips, op vertoon,;}{van praal en macht bedacht hield}{een groote wapenschouw}\\

\haiku{En als gijlieden,;}{uw handen uitbreidt verberg}{Ik Mijn oogen voor u}\\

\haiku{Maar als de eersten,,:}{omhooggevlogen boven}{aankomen zien zij}\\

\haiku{{\textquoteright} Zij ging aan het raam -.}{vol stortte de stralende}{zomerdag over haar}\\

\haiku{Iederen dag de,.}{jachtfeesten de tournooien}{te harer eere}\\

\haiku{Borre van Doirninck,,;}{die trouw haar gevolgd was zag}{haar verlorenheid}\\

\haiku{{\textquoteleft}Ik had in Gouda,.}{moeten blijven tot de groote}{rust van den dood kwam}\\

\haiku{Nu besta ik voort,.}{zonder rust in de engte}{van een dood leven}\\

\haiku{Hij stond recht, bevend {\textquoteleft}}{over zijn gansche lichaam met}{rooddoorloopen oogen}\\

\haiku{{\textquoteright} {\textquoteleft}Die zelf - d\`at moet Uwe -.}{Genade thans wel weten}{nooit uw vijand w\`as}\\

\haiku{{\textquoteleft}Vertelt gij mij, wat.}{aan Philip's hof wordt verhaald}{van La Pucelle}\\

\haiku{{\textquoteright} {\textquoteleft}En tenslotte als -.}{heks verbrand te Rouaan nu}{een week geleden}\\

\haiku{En hij wist ook, dat.}{Glocester's ontrouw haar had}{doen capituleeren}\\

\haiku{maar voelde hij meer?}{voor Vrouw Jacob of voor het}{welzijn van het land}\\

\haiku{Dan plotseling dacht,.}{hij hoe zij door haar Hoeksche}{vrienden alles wist}\\

\haiku{Wilde zij w\`el den,,?}{opstand om haar landen maar}{niet hem als gemaal}\\

\haiku{Hoe kwam tegen haar,.}{grove trekken Jacoba's}{fijne gratie uit}\\

\haiku{Er was een groene;}{kamer met wonderlijke}{beesten aan den wand}\\

\haiku{{\textquoteright} Hij was neergeknield,,.}{en kuste haar hand. Maar recht}{weer zag hij haar aan}\\

\haiku{De verluchting na.}{den Zoen van Delft was ook in}{zijn hart lang getaand}\\

\haiku{De eindelijke.}{vergelding voor al wat zij}{hem had doen lijden}\\

\haiku{Voorgoed van dezen -.}{aardbodem doen verdwijnen}{nooit meer in zijn weg}\\

\haiku{Met zonderlinge,....}{aandacht haar bezagen de}{zwarte stille oogen}\\

\haiku{De sterken balden,.}{de vuisten verlegden hun}{hoop naar beter tijd}\\

\haiku{Nu had zij ook Van.}{Borselen's leven gewaagd}{en vernietigd}\\

\haiku{Ho\`e, dat laat ik aan,.}{God en zijn geweten over}{te beoordeelen}\\

\haiku{Hij bracht belangrijk,{\textquoteright}.}{nieuws hernam zij plotseling}{met iets aanvallends}\\

\haiku{Men houdt hier niet van,{\textquoteright}.}{de Engelschen zei de Zeeuw}{Van Borselen hard}\\

\haiku{De helle hemel.}{met zijn blindend zomerlicht}{stortte op haar toe}\\

\haiku{Als werkelijk de ',?}{Engelschen int zicht zijn}{zullen zij vechten}\\

\haiku{Weet hij, of in 't?}{geheim de heks niet met hen}{gemeene zaak maakt}\\

\haiku{Met elken vriend gaat.}{heen voorgoed een moment uit}{het voorbije leven}\\

\section{Louis de Bourbon}

\subsection{Uit: Twaalf maal Azi\"e}

\haiku{Zij liep een spoorweg,.}{over sloeg rechtsaf en kwam op}{een groote boulevard}\\

\haiku{Er was geen verkeer.}{op de straat en ook in de}{huizen was het stil}\\

\haiku{- Dan weet ge misschien,?}{wel waar ik een betrekking}{zou kunnen vinden}\\

\haiku{- Zoo, zei de man en,.}{boog nu heelemaal naar haar}{over laat dat eens zien}\\

\haiku{Toen hij daar aankwam,.}{stond de Arabier juist op het}{punt te vertrekken}\\

\haiku{Loop hard, opdat hij.}{niet de gelegenheid heeft}{ze uit te geven}\\

\haiku{En bij Sitih had,.}{ieder gelijke rechten}{die betalen kon}\\

\haiku{- O, wat griezelig,,,!}{zei de dame die naast mij}{zat wat vreeselijk}\\

\haiku{Wij stonden nu op,.}{het bordes de dame hing}{nog steeds aan mijn arm}\\

\haiku{De Annamieten.}{werden kopschuw en trokken}{terug op Chi-hoa}\\

\haiku{Het was hard noodig de.}{noordelijke gebieden}{te consolideeren}\\

\haiku{Dupuis vertrok over,.}{zee naar Hai-Phong vandaar}{over land naar Hanoi}\\

\haiku{Zijn mooi zwart haar was.}{grijzend en op sommige}{plaatsen bijna wit}\\

\haiku{Maar om een leven.}{op te bouwen moest hij haar}{opnieuw verlaten}\\

\haiku{Hij dronk een tweeden.}{beker wijn en ging door de}{open deur naar buiten}\\

\haiku{Een onvoorzichtig,.}{marinier werd getroffen}{en stierf huilende}\\

\haiku{Hij had in Pha-ung.}{een schrander en toegewijd}{helper gevonden}\\

\haiku{Het is geen angst, maar,.}{onrust die zich onder de}{bezetting verspreidt}\\

\haiku{De vanen steken.}{scherp af tegen de wolken}{aan den horizon}\\

\haiku{Tegen de muur zat,,.}{een oude Chinees alleen}{en leek te slapen}\\

\haiku{Hij keek mij aan met,.}{een wreeden grijns en lachte}{spottend uitdagend}\\

\haiku{But look here, this,,.}{is a saphire Sir a}{real saphire}\\

\haiku{Nog twee seconden,,,.}{dacht Arthur als ik mijn sabel}{heb ben ik gered}\\

\haiku{- Kijk, daar loopt Sitih,,.}{fluisterde hij de vrouw die}{sakit hati is}\\

\haiku{en dat ik altijd.}{dezelfde gebleven was}{die ik geweest was}\\

\haiku{Ik stapte in de,.}{auto en wij reden weg}{ditmaal bergafwaarts}\\

\haiku{Doch dan deed hij het,,.}{ook geducht hij liet zich dan}{volloopen boordenvol}\\

\haiku{Wie hem de meeste,.}{in zijn bakje terugbracht}{mocht er \'e\'en opeten}\\

\haiku{Soms ook vertelde,.}{hij hun verhalen in zijn}{eenvoudig Maleisch}\\

\haiku{Ank van de overkant.}{is ook nog altijd vrij en}{praat veel over je}\\

\haiku{- U niets te vreezen,,,.}{zei hij ik U laten zien}{U moet weten}\\

\haiku{Daarbinnen lag, stijf,,.}{opgerold een glanzende}{grijze vilthoed}\\

\subsection{Uit: Vier onbekenden}

\haiku{Wat lijkt dat nu weer,.}{lang geleden het lijkt een}{heel andere tijd}\\

\haiku{Een erg mooie jonge.}{vrouw en die wordt dan verliefd}{op den luitenant}\\

\haiku{Hij liep mijn kant uit,.}{en ik ging naar hem toe om}{hem voor te lichten}\\

\haiku{Eventjes maar, toen ging.}{hij een loge binnen en}{ik weer op mijn plaats}\\

\haiku{Ik kwam nooit voor half,.}{\'e\'en thuis als moeder en broer}{al in bed lagen}\\

\haiku{Heel dicht lag ik naast,,.}{hem zijn adem ging over mijn bloote}{schouders als hij sprak}\\

\haiku{- Hoe kan het vandaag,,,.}{zijn dwaze maagd zei hij de}{dag is ten einde}\\

\haiku{Dat moest prachtig zijn.}{en ik had Arthur toch ook in}{den winter ontmoet}\\

\haiku{Ik geloof, dat ik.}{zoo een heele poos heb staan}{denken en droomen}\\

\haiku{Ik sloot opnieuw mijn.}{oogen en begon mij alles}{te herinneren}\\

\haiku{Het schijnt zoo, dat de.}{meesten zich om deze vraag}{niet bekommeren}\\

\haiku{Ze werkt immers zelf '.}{op een fabriek ens avonds}{is ze in City}\\

\haiku{Maar ik zou hem wel,.}{kunnen helpen als ie een}{keer m'n zwager was}\\

\haiku{Misschien ook wel is,.}{ze bang dat het gaan zal als}{de vorige keer}\\

\haiku{Hij werd een beetje,.}{geplaagd in de barak maar}{dat ging al gauw over}\\

\haiku{- Ik heb gehoord, dat,,.}{je rijk bent geweest zei ik}{zoo langs mijn neus weg}\\

\haiku{- Maar hoe is het dan,,.}{met jou vroeg ik je bent toch}{zelf arm geworden}\\

\haiku{Hij was trouwens voor,.}{elk werk ongeschikt dat wist}{hij uit ervaring}\\

\haiku{Ja, ik moet eerlijk,.}{bekennen dat ik er een}{beetje trotsch op was}\\

\haiku{De zon was al een,.}{poosje op in de boomen}{zongen de vogels}\\

\section{Menno ter Braak}

\subsection{Uit: D\'emasqu\'e der schoonheid}

\haiku{zijn bewonderaars.}{zijn de \'e\'endags-vrienden}{van de politiek}\\

\haiku{Omdat zij nog niet?}{gearriveerd waren op}{hun zestiende jaar}\\

\haiku{Hij zal den puber;}{niet zonder meer toelaten}{in zijn gemeenschap}\\

\haiku{het gaat tusschen de.}{schoonheid als soepsnuiverij}{en als bevrijding}\\

\haiku{Heeft God de wereld?}{ook op zulk een krukkige}{manier geschapen}\\

\haiku{8 Er is nog een;}{derde mogelijkheid in}{den strijd met den vorm}\\

\haiku{dat hun aesthetiek,?}{naar afval riekt wie zal er}{zich over verbazen}\\

\haiku{De kunst is thans door,;}{de natuur heengegaan zij}{is niet olympisch meer}\\

\haiku{in 't proza staan;}{zij in een klaar licht of als}{in een schemering}\\

\subsection{Uit: Dr. Dumay verliest...}

\haiku{Hij kende dat, het.}{onberedeneerbare}{verradersgevoel}\\

\haiku{{\textquoteleft}Zeg, waarde heer, zou?}{je nu eindelijk het woord}{eens willen nemen}\\

\haiku{en in de handen,.}{van Jean Wood zag hij zijn hoed}{tasch en wandelstok}\\

\haiku{Onafhankelijk,,...}{daarvan en van alles wat}{anderen aangaat}\\

\haiku{Nu zullen Victor,.}{en ik nooit meer zoo kunnen}{vechten als vroeger}\\

\haiku{hij wist niet, wat hij,,,...}{deed hem overkwam iets een daad}{een zoo-maar-iets}\\

\haiku{Ik weet heel zeker,.}{dat ik er eigenlijk geen}{behoefte aan heb}\\

\haiku{Hebt u tijdens mijn?}{absentie de Germania}{verder behandeld}\\

\haiku{Idee\"en zonder flair;}{behooren waarschijnlijk bij}{mijn constitutie}\\

\haiku{Het was Dumay niet,.}{volkomen helder waarom}{hij zelf gegaan was}\\

\haiku{Hij wist, dat hij op;}{het kerkhof nauwelijks om}{Jean Wood gerouwd had}\\

\haiku{Maar zegt u nou zelf,,!}{mijnheer Donner wat heb je}{dan nog aan je hond}\\

\haiku{{\textquoteleft}Mag ik de heeren,{\textquoteright}.}{even aan elkaar voorstellen}{zei hij onhandig}\\

\haiku{Zij liepen een paar.}{honderd meter onder een}{pijnlijk stilzwijgen}\\

\haiku{Later op den avond,:}{kwamen de vrienden waarover}{Max gesproken had}\\

\haiku{Het individu,:}{scheen samen te spannen met}{Souzie of liever}\\

\haiku{Wees maar blij, dat je.}{met vaderlijke zorgen}{niets te maken hebt}\\

\haiku{we laten op dat.}{punt onze log\'e's altijd}{maar aan hun lot over}\\

\haiku{{\textquoteright} {\textquoteleft}Als de zaken er,,.}{naar staan is er geen reden}{om nog te wachten}\\

\haiku{Een oude dame.}{raapte er een paar op en}{gaf ze haar terug}\\

\haiku{Dus, gegeven het,.}{aantal getrouwde mannen}{t\`och geen gewoon mensch}\\

\haiku{u zoo vriendelijk,?}{zijn even uit te kijken waar}{u uw voeten zet}\\

\haiku{Tegelijkertijd.}{onmoetten zijn oogen weer die}{van den dronken boer}\\

\haiku{En plotseling vond,.}{hij zichzelf liggen op een}{vreemden zak stompend}\\

\haiku{Het was duidelijk,.}{aan hem te zien dat hij zich}{zat te ergeren}\\

\haiku{Hij was een goede,,.}{brave man geweest al had}{hij ook zijn fouten}\\

\haiku{Dat kwam er van, als...}{het geloof er niet meer was}{om halt te roepen}\\

\haiku{Toe, ik heb nog niets,.}{van je gehoord hoe je het}{gehad hebt en zoo}\\

\haiku{{\textquoteright} {\textquoteleft}Ik zeg u toch, dat,!}{ik genoeg van hem heb dat}{ik hem niet meer wil}\\

\haiku{maar in het volgend.}{oogenblik had hij haar al}{weer opgevroolijkt}\\

\haiku{Zij bleven voor de.}{photo's staan en besloten}{de film te gaan zien}\\

\haiku{de titel van de.}{film was hij aan de kassa}{al weer vergeten}\\

\haiku{En hij weet, dat hij,.}{het liegt dat ik het eerlijk}{met hem gemeend heb}\\

\haiku{Soms is een dom woord.}{van juffrouw van der Wall voor}{mij een zaligheid}\\

\haiku{Alle gebouwen;}{lagen koel en scherp in het}{maanlicht te baden}\\

\haiku{Welneen, \`als hij het,.}{geweten had was hij het}{zeker vergeten}\\

\haiku{Kunt \`u hem nu niet?}{eens een duwtje geven in}{de goede richting}\\

\haiku{U had hem moeten,!}{zien toen hij voorzitter was}{van zijn H.B.S.-club}\\

\haiku{Nel had een massa,;}{kennissen die onderling}{veel avondjes hielden}\\

\haiku{Nu dan: ik heb me,:}{verloofd met iemand van wie}{je nooit gehoord hebt}\\

\haiku{Jawel, jullie doen,!}{maar tegenwoordig jullie}{denkt maar aan jezelf}\\

\haiku{De sportiviteit;}{veerde met Dumay trede}{na trede verder}\\

\haiku{Haar gezicht was wat,.}{opgezet zij hijgde van}{het trappen klimmen}\\

\haiku{{\textquoteleft}U moet hem redden,...,.}{juffrouw als u hem niet redt}{is hij verloren}\\

\haiku{Met mij spot hij toch,,}{maar ik ben niets voor hem van}{mij neemt hij niets aan}\\

\haiku{dit had toch best nog,...}{kunnen wachten meubels zijn}{in \'e\'en dag gekocht}\\

\haiku{{\textquoteright} zei hij grijnzend in):}{de richting van Dumay dan}{goedkoop uit kon zijn}\\

\haiku{Begrijp jij, dat er,?}{menschen zijn die in zoo'n}{bed willen slapen}\\

\haiku{Overigens, hij doet,,...{\textquoteright} {\textquoteleft}?}{in kikkers dat is zijn vak}{dusIs hij getrouwd}\\

\haiku{Een tasch viel met een,.}{smak op den grond papieren}{vlogen links en rechts}\\

\haiku{{\textquoteleft}Maar... wat praten we,,?}{toch allemaal we gaan toch}{trouwen wij twee\"en}\\

\haiku{Zonder veel aandacht.}{bladerde zij in \'e\'en der}{nieuwe aanwinsten}\\

\haiku{de baas groette hem,.}{vriendelijk hij tikte aan}{zijn hoed en reed weg}\\

\haiku{daarvoor komt zij op,,.}{en dat is mijn schuld dat is}{niet af te wasschen}\\

\haiku{als hij maar praat en!}{als Lydia maar theeschenkt en}{de baby verzorgt}\\

\haiku{De adem van den tuin;}{kwam door de groote openstaande}{deuren naar binnen}\\

\haiku{De baby, schoot hem,?}{door het hoofd er zal toch niets}{met de baby zijn}\\

\haiku{Je bent juist op tijd,.}{gekomen ik moet er met}{iemand over praten}\\

\haiku{Ik hoop van niet en.}{kom zonder tegenbericht}{morgenavond bij je}\\

\haiku{Hij betrapte zich,.}{op een sensatie die op}{teleurstelling leek}\\

\haiku{Met een paar sprongen.}{was Dumay boven en in}{Karin's slaapkamer}\\

\haiku{Wil je dat op je...?}{geweten hebben dat ik}{me om jou doodschiet}\\

\haiku{{\textquoteright} zei hij bevelend, {\textquoteleft},,.}{je weet dat ik niet meer te}{zeggen heb Karin}\\

\haiku{Maar ze is verliefd,...!}{op je dat ouwe mensch die}{vogelverschrikker}\\

\haiku{Maar even nog drong zich:}{tusschen de nevelen door}{een vraag naar voren}\\

\haiku{zij wist niet, of zij;}{Dumay geluk ging wenschen}{met zijn huwelijk}\\

\subsection{Uit: Hampton Court}

\haiku{Op dat oogenblik,.}{wist hij plotseling dat hij}{het park niet zou zien}\\

\haiku{Naast hem lagen een.}{Engelschman en zijn vrouw}{op hun dekstoelen}\\

\haiku{Toen hij dien avond naar,.}{boven ging had de slaap hem}{al bijna overmand}\\

\haiku{Andreas liep er;}{onwillekeurig heen en}{betastte den stam}\\

\haiku{Je leeft naast een vrouw,.}{alsof Hendrik VIII er}{geen zes had gehad}\\

\haiku{Dank, dank, hoesten en,!}{morgenschemering dank voor}{het pad naar den slaap}\\

\haiku{hij manoeuvreerde,.}{handig met zijn valies tot}{hij vlak bij haar was}\\

\haiku{Aan de trams, die over,;}{het stationsplein reden}{hingen vlaggetjes}\\

\haiku{De trambestuurder,.}{hernam zijn plaats er kwam weer}{schot in het verkeer}\\

\haiku{Ja, het was alles,;}{hetzelfde maar het had nu}{geen medelijden}\\

\haiku{De verteedering;}{verloste hem eensklaps uit}{zijn afwezigheid}\\

\haiku{Ze is weg en ik,.}{zou eigenlijk niet wenschen}{dat zij terugkwam}\\

\haiku{Onverwachts toch nog;}{stormden een paar hossende}{jongens langs hem heen}\\

\haiku{Diederik's juffrouw,.}{scheen aan den rol te zijn er}{werd niet opengedaan}\\

\haiku{Diederik, die hem,.}{tegemoet was gekomen}{zag hem verbaasd aan}\\

\haiku{Dat was waar ook, er,.}{was nog iemand dat was hem}{door het hoofd gegaan}\\

\haiku{{\textquoteleft}Ja, dat je het met,.}{mijn bewering niet eens zou}{zijn wist ik ook wel}\\

\haiku{Maar nu remde hem.}{de aanwezigheid van van}{Haaften hinderlijk}\\

\haiku{Hij boog diep voor haar,.}{zonder zich te storen aan}{hem uit de bioscoop}\\

\haiku{Trouwens, je hoorde,!}{zelf die flauwe kul die hij}{verkocht over dat Delftsch}\\

\haiku{Ja, klets maar jongen,,!}{van Haaften is er immers}{niet nu durf je wel}\\

\haiku{Toen trilde ergens.}{uit het niets een verlossend}{denkbeeld op hem toe}\\

\haiku{en daarop stond (zag?),.}{hij goed een soldatenpet}{studentikoos scheef}\\

\haiku{Toen drong het tot hem,.}{door dat het stil was en dat}{hij hier moest praten}\\

\haiku{Ik voor mij, ik ben.}{deze houten klaas dankbaar}{voor zijn openbaring}\\

\haiku{Hij leefde nog in,.}{een andere wereld die}{van het luisteren}\\

\haiku{en weer zweefde zijn:}{stem tusschen boosaardigheid}{en genegenheid}\\

\haiku{Daar is niets aan te,.}{doen en daarom moet ik me}{wel zoo uitdrukken}\\

\haiku{Dat de menschen hem,!}{ongevoelig en cynisch}{noemden wat zou dat}\\

\haiku{Daarmee had hij dus,.}{geleefd daarmee gingen dus}{ook veel menschen dood}\\

\haiku{Het is geen eerbied,,.}{het is geen slaafschheid het}{is nu eenmaal zoo}\\

\haiku{In dit opzicht, dacht,,,.}{hij ben ik het niet met hem}{eens goddank goddank}\\

\haiku{d{\`\i}t effect heeft d\`at,.}{ten doel d{\`\i}e stoot moet leiden}{tot d{\`\i}e positie}\\

\haiku{maar een hypocriet,!}{in den gewonen zin was}{hij niet volstrekt niet}\\

\haiku{Had zij eigenlijk,?....}{wel eens verteld op welke}{etage zij werkte}\\

\haiku{{\textquoteright} Maffie glimlachte,.}{zonder den glimlach van den}{klant op te vangen}\\

\haiku{Maffie bracht bloemen,,.}{mee verplaatste kleinigheden}{deed huisvrouwelijk}\\

\haiku{zij dronken ergens,,;}{koffie en ergens thee zij}{aten samen ergens}\\

\haiku{Hij trof van Haaften.}{later op den avond in het}{gewone kroegje}\\

\haiku{Dat is nu alles,.}{goed en wel maar je geeft geen}{antwoord op mijn vraag}\\

\haiku{Hij moest tegen wil,.}{en dank grijnzen terwijl hij}{de straatdeur dichttrok}\\

\haiku{laten we liever!}{ons entree tot de eerste}{acte uitstellen}\\

\haiku{Hij is mijn vriend, ja,,,;}{zeker dacht hij niet meer mijn}{patroon zooals vroeger}\\

\haiku{vergeefs, de lange,!}{vingers trillen als zij een}{kopje aanpakken}\\

\haiku{men zag een sober,.}{decor opgebouwd uit niet}{veel meer dan planken}\\

\haiku{de courtisane;}{scheen voor de nieuwe idee\"en}{te moeten wijken}\\

\haiku{Van Rees was zeker,,.}{een innemende man een}{aristocraat jawel}\\

\haiku{Iets heet al even gauw,.}{cynisch als symbolisch op}{een bepaald niveau}\\

\section{Karel Broeckaert}

\subsection{Uit: De sysse-panne: Borgers in den estamin\'e}

\haiku{Verboden wordt het.}{geestelik habijt in het}{openbaar te dragen}\\

\haiku{alors commenc\`erent.}{la spoliation et}{le vandalisme}\\

\haiku{{\textquoteleft}Even stellig is de:}{Gazette fran\c{c}aise bij}{de proklamatie}\\

\haiku{A.J.        Inleiding...}{I De Vlamingen komen}{altijd achteraan}\\

\haiku{Zijn optreden valt.}{samen met de ingang van}{een nieuw tijdsgewricht}\\

\haiku{- gy zoed niet gelooven!}{dat n\^en Bedelaer meer kan}{doen als n\^en Bankier}\\

\haiku{Zanne doet Mie op.}{staen geeft-ze leujens}{om kaffe t'haelen}\\

\haiku{aen hangt, om gelyk,.}{Gysken zegt de Menschen van}{hem schuw te maeken}\\

\haiku{Gysken. 't Es imand.}{van myn kennisse die my}{hier in gesleept heeft}\\

\haiku{Rentier kan worden,,;}{maer tog als hy het gaede}{slaet het helpt altyd}\\

\haiku{- De Kloosters doen meer ';}{dienst aent Publiek aes de}{Parochie Kerken}\\

\haiku{wy verfoyen in;}{onze Sermoenen alles}{wat de ryke doen}\\

\haiku{Wat deugden winnen?...}{zy by het verliezen van}{hunne Religie}\\

\haiku{Zyn zy geldzugtig,;}{zy verzuymen geenen middel}{om'er te krygen}\\

\haiku{want het Huwelyk.}{en is niet anders dan eene}{plegtige hoerery}\\

\haiku{naer verlangde, en,.}{noch lange van sprak alsze}{gepasseert waeren}\\

\haiku{en uweu Vader die, '.}{het in het hymelyk ziet}{zalt u loonen}\\

\haiku{(Pabula tyrannorum \&) ':}{sunt Plebs Rusticit zyn}{oude spreekwoorden}\\

\haiku{Boeren boet - Boeren;}{zyn loeren en schelmen uyt}{de natuere}\\

\haiku{Neen Koninginne;}{zeyd den Koning m\^e zullen'er}{Ossen m\'e koopen}\\

\haiku{'K en h\^ek eventwel.}{ze leven niet gezien of}{den Regen kwampter}\\

\haiku{Het vraegt eene sterke}{en overtuygende Penne}{om te bewyzen}\\

\haiku{Ploeg heeft twee Knegten.}{moet hebben en verscheyde}{Daghuermannen}\\

\haiku{5o Dat het klaer is,,}{wanneer den Grond niet alles}{opbrengt dat hy kan}\\

\haiku{Dat 'er maer een slag,.}{van Maeten Gewigten en}{van Wetten zyn moet}\\

\haiku{Ze keunen 't er,!}{veuren doen wat zegde vyf}{honderd Millioen}\\

\haiku{Het register van.}{de Burgerlike Stand werd}{in 1796 ingevoerd}\\

\haiku{het bestuur van het.}{huis werd die dag toevertrouwd}{aan het jongste kind}\\

\haiku{een pas geboren.}{kind in het kerkregister}{laten inschrijven}\\

\haiku{zig ten leur laten,.}{stellen zich met de vinger}{laten nawijzen}\\

\haiku{116De oude en -.}{de nieuwe Wetten31 Julie}{13 Thermidor IVBlz}\\

\section{Johan W. Broedelet}

\subsection{Uit: Hofstad. Deel 1}

\haiku{Z'n gang had dan ook ',.}{t zwevende van een die}{de aarde nauw voelt}\\

\haiku{Alleen moest-ie eens ',.}{n ander dasje koopen}{want dat rafelde}\\

\haiku{ik heelemaal niets{\textquoteright},.}{van hebben draaide zich om}{en wilde weggaan}\\

\haiku{Dat colporteeren.}{met advertenties is me}{niet meegevallen}\\

\haiku{De viool zette '.}{metn zucht z'n instrument}{weer onder z'n kin}\\

\haiku{Ze was bezig, 'r,.}{bloemen te begieten waar}{ze groote zorg voor had}\\

\haiku{Bij zulk 'n rijkdom '.}{van gemoed paste slechtsn}{eerbiedvol zwijgen}\\

\haiku{{\textquoteright} De barones van ' '.}{Liktum Priktum vroegt met}{r gewonen spot}\\

\haiku{Daarop keerde de,.}{meneer zich in eens om liep}{naar den vijver toe}\\

\haiku{{\textquoteright}, wees-ie meteen, '.}{op de twee damesn goed}{vrachtje voor terug}\\

\haiku{Er zich nu verder,.}{niet mee bemoeiend stapten}{de vriendinnen in}\\

\haiku{Vagebond keerde, ',.}{metn gezicht of-ie nooit}{anders gedaan had}\\

\haiku{Nou, versta jij, jij, ',.}{zorgtt komt weer heel of ik}{hou van jou loon af}\\

\haiku{Kaatje schreef met  'n, ' '.}{gemakkelijkheid datt}{r zelf verbaasde}\\

\haiku{Als z\'o\'o de entre\'e,?}{tot den salon was hoe moest}{die z\`elf dan wel zijn}\\

\haiku{O, ze zou 't nog,!}{uitschreeuwen van ontzetting}{als dat niet ophield}\\

\haiku{Of was dat juist 'n,.... {\textquoteleft},,,!}{teeken dat zeO mama}{mama z\`eg toch wat}\\

\haiku{En ze snelde naar,,.}{mama terug die de oogen}{flauw opsloeg steunend}\\

\haiku{De schoonheid van 't '.}{Herteveld trofr toch}{telkens weer opnieuw}\\

\haiku{Ik adoreer 't. 't,.}{Heeft me gisteravond he\`el he\`el}{gelukkig gemaakt}\\

\haiku{Denk niet, dat ik u,,.}{om die gewone vunse}{aardsche liefde smeek}\\

\haiku{{\textquoteright} woelde 't in den,.}{opgezetten rooien kop}{van den zweep-ridder}\\

\haiku{Als 't nu niet uit, '.}{was met die aardigheden}{riep-ien agent aan}\\

\haiku{{\textquoteright} Weg was Amie weer en '.}{Vagebond tuurde stil over}{den kop vant paard}\\

\haiku{Z'n houding was in.}{alle omstandigheden}{des levens correct}\\

\haiku{De w\`aarlijk hooge adel.}{bleef natuurlijk voor immer}{voor hem gesloten}\\

\haiku{Ja, niet alleen 't!}{m\`anlijk geslacht takelde}{af met de jaren}\\

\haiku{Struggle slenterde, '.}{terug bekeek zich van op}{zij inn spiegel}\\

\haiku{'t Was 'n mond vol.,.}{Vervelend dat dat goedje}{altijd zoo kleefde}\\

\haiku{Even wist ze nergens,,,.}{van soezend op gevaar af}{in slaap te knikken}\\

\haiku{Maar als-ie 'r nog, ' '?}{beminde opr ouden}{dag omr hand vroeg}\\

\haiku{Ook, de alt mocht dan, '.}{wezen wie ze wilde die}{toon stondm niet aan}\\

\haiku{Et \`a pr\'esent vous,....}{vous permittez des blagues}{des plaisanteries}\\

\haiku{De heerlijkheden;}{der wereld liet-ie kalm langs}{z'n mouw afzakken}\\

\haiku{Je phantasie krijgt,,,, '.}{de ruimte breekt uit zwelt}{barst wordtn volheid}\\

\haiku{De aster stak-ie ',.}{int knoopsgat Amie dicht aan}{z'n oor te hebben}\\

\haiku{O, kon ze 'm maar,,!}{heel even zien n\`u wat zou ze}{er niet om geven}\\

\haiku{Daarop was-ie nog,.}{even aan haar kamers geweest}{die ze op slot hield}\\

\haiku{Z'n ziel toog er in,,.}{huppelde mee dat-ie}{zichzelf niet meer was}\\

\haiku{Dan schoof ze ter zij,,.}{voor de danseuse noble}{die aan de beurt was}\\

\haiku{Geen haat gevoelde,.}{ze langer voor die Lydia}{twee loges verder}\\

\haiku{Onder haar leiding ' {\textquoteleft}{\textquoteright}.}{wast huisKlingeling zeer}{in bloei gestegen}\\

\haiku{En getinkel drong.}{door en eerst niet geziene}{menschen doken op}\\

\haiku{Ik geloof juist, dat '.}{hij int algemeen te}{copieus dineert}\\

\haiku{En de knecht van den '.}{baron van Priktum heeftr}{juist teruggebracht}\\

\haiku{O, ik beh\`oef die!}{bedwelming na mijn arbeid}{van zooveel dagen}\\

\haiku{Gelukkig echter.}{had Jacobi er zich nog}{altijd uit gered}\\

\haiku{'t Begon voor 'm,.}{te draaien dat-ie haast}{tegen den grond sloeg}\\

\haiku{Met de opening van.}{z'n salon moest-ie toch niet}{te lang meer wachten}\\

\haiku{t Beste dan ook!}{was nog niet goed genoeg voor}{d\'ezen cli\"ent}\\

\haiku{Vooral 'n vijf, zes ', '.}{kinderen warent die}{r aandacht vroegen}\\

\haiku{Laat ons daarom de,,.}{uurtjes die ik vandaag vrij}{heb daar doorbrengen}\\

\haiku{{\textquoteright} {\textquoteleft}Leo{\textquoteright} antwoordde 't,,.}{manneke dat nu maar wou}{dat dat koekje kwam}\\

\haiku{'t Liep mis met 'm. {\textquoteleft}{\textquoteright} '.}{DieKleine Courant vergde}{te veel vanr Wim}\\

\haiku{Hij was net op, toen ', '.}{z'n juffrouwm zei dat er}{iemand voorm was}\\

\haiku{Vagebond echter, ',.}{vans zangers gepeinzen}{niet wetend hield aan}\\

\haiku{Mam\`a, 't levendste!}{bewijs van de verfijning}{der van Hogenloo's}\\

\haiku{En zonder 'n woord,, '.}{meer draaide-ie zich om ging}{n eigen kant uit}\\

\haiku{Van de honderd pop.}{van vanmorgen had hij er}{nog vijf en twintig}\\

\haiku{Zonder haar zou z'n '.}{levent gewone zijn}{als van iedereen}\\

\haiku{Montrose nam z'n, '.}{sigaar weer op deedn paar}{geduchte trekken}\\

\haiku{Als ieder van te,,,!}{voren wist hoe-ie later}{zou worden nu nu}\\

\haiku{H\'el\`ene, zich in 'r,.}{nachttoilet te steken liep}{de spiegelkast langs}\\

\haiku{Bij kaarsen-schijn, ', ',.}{dan opt hooge bed wast}{dat hij haar bezat}\\

\haiku{Alleen, in mannen,.}{woelden zwarter dingen dan}{men vaak vermoedde}\\

\subsection{Uit: Hofstad. Deel 2}

\haiku{Doch Lydia, die zich, ' '.}{verveelde weemoedigde}{t meer mett uur}\\

\haiku{n Druk legde zich ',.}{opr waaraan ze zich te}{ontwentelen had}\\

\haiku{Waarlijk, ze hechtte ', ' '.}{zich aanm zoum niet graag}{missen inr huis}\\

\haiku{Ach, in Hofstad had. '!}{ieder zoo z'n vermaakjes}{n Stad van pleizier}\\

\haiku{Dat krioelde en.}{kringde en slinger-schoof}{zonder ophouden}\\

\haiku{'k heb je zoo goed.}{als nooit iets in allen ernst}{hooren beweren}\\

\haiku{Wat Eveline ook, ', '.}{zeim te weerhoudent}{bleef zonder gevolg}\\

\haiku{'t Mensch spartelde ', '.}{onderm alsof zem}{nog ontglippen ging}\\

\haiku{Op 'n soort standaard,, '.}{koper gehelmd vlotten}{costuumgedeelte}\\

\haiku{in 'm. Tien vingers.}{zette-ie verticalig}{op z'n schrijfbureau}\\

\haiku{En hij zette zich,.}{de handen stevig aan de}{leuning van z'n stoel}\\

\haiku{Ja, als-ie er over ', '!}{n piano beschikte}{zout nog w\`at zijn}\\

\haiku{De golven harer, '.}{verontwaardiging sloegen}{over datt kookte}\\

\haiku{Vanmiddag was-ie.}{andermaal gerefuseerd}{op de Oude Gracht}\\

\haiku{Intusschen, dit nam, ' '.}{niet weg datt geheeln}{zaak bleef tusschen h\`en}\\

\haiku{En hij ging door voor,.}{Titri's amant wat-ie in}{waarheid toch niet was}\\

\haiku{O, neen, daar kwam h{\`\i}j,,.}{de dokter van de  chic}{zoo goed als nooit in}\\

\haiku{Ach ja, als je in '!}{n nacht-societeit ook al}{ging critiseeren}\\

\haiku{O, 't kostte 'm.}{altijd nog veel meer dan-ie}{goed betalen kon}\\

\haiku{Maar kom, geheel voor!}{spek en boonen zat die er}{toch zeker niet bij}\\

\haiku{Hoe kwam-ie daarbij,, '?}{hij diet anders nooit over}{pati\"enten had}\\

\haiku{- daar, vanuit 't graf, '.}{om zoo te zeggen nogn}{zeer slimmen zet deed}\\

\haiku{In dienst moest alles,.}{bij tijds zijn vroeg naar bed en}{vroeg uit de veeren}\\

\haiku{H{\`\i}j, die zooveel voor,.}{de jongens deed de ziel van}{Purpurae Amori}\\

\haiku{Hij moest oppassen,, '.}{voelde-ie oft ging den}{gezochten kant uit}\\

\haiku{En z'n bezoeken.}{aan de Berghuisen werden}{steeds geregelder}\\

\haiku{{\textquoteright} Zelfs al stapte-ie -!}{over alles heen hij voelde}{er zich toe in staat}\\

\haiku{Wie echter zou 'n?}{gravin van Montrose de}{toegang weigeren}\\

\haiku{Hij zou er misschien.}{meer uit afleiden dan-ie}{noodig had te weten}\\

\haiku{Hemel, die stijve!}{Keutellanders ook met hun}{enge begrippen}\\

\haiku{Olga, 'r hoed af -.}{dien waagde ze er van avond}{in godsnaam maar aan}\\

\haiku{Kwam er weer niets van? ' '.}{t Leekm toch zoo'n groote ernst}{bij dien Amerikaan}\\

\haiku{Maar, o ja, voor die,,!}{dansen straks welke ze had}{helpen instudeeren}\\

\haiku{Ze voelde, dat 'r,.}{iets gebeuren ging dat ze}{niet wilde missen}\\

\haiku{We\`et  je 't nu,,,?}{mijn vlam mijn zomer-brand}{mijn lente-braam}\\

\haiku{Niet w\`eer eens mocht '.}{ze de kans loopen vann}{kortstondig geluk}\\

\haiku{er was niets tusschen.}{mevrouw de barones en}{meneer Vagebond}\\

\haiku{'t Was heel vroeg, nog,.}{donker toen Kaatje de deur der}{Dim's achter zich sloot}\\

\haiku{Als ze weer wachtte ',.}{tott volledig dag werd}{kwam er toch niets van}\\

\haiku{Dim, die maar 'n rol, '.}{gespeeld had naar bleek trachtte}{r deur te forceeren}\\

\haiku{Waarvoor haalde-ie ', ' ' {\textquoteleft}{\textquoteright}?}{r toch aan alst niet om}{tgewone was}\\

\haiku{George zat aan ', '.}{t andere eind van de}{kamer opn stoel}\\

\haiku{Ach, ja, wie begreep ',?}{m w\`el den al zieliger}{George Kapel}\\

\haiku{Hij heeft alleen maar '{\textquoteright}.}{n zieltje zei ze nog met}{verklaarbaren spot}\\

\haiku{De laatste kende,.}{ze enkel van gezicht den}{graaf wel van vroeger}\\

\haiku{Hoe kwam echter de?}{baron van Priktum tusschen}{dat troepje verzeild}\\

\haiku{Ik ben nog altijd,,.}{je bezit maar jij behoort}{mij ook voll\`edig}\\

\haiku{Hij verzette er, '.}{zich niet tegen keek zelfsn}{beetje toegevend}\\

\haiku{Ze escorteerden ' ',.}{alst waret drietal}{dat daar vooruit liep}\\

\haiku{{\textquoteright} Echtgenooten,.}{behoorden te weten wat}{elk hunner toekwam}\\

\haiku{Maar gaandeweg ging '.}{ze toch wat aandachtiger}{naarm luisteren}\\

\haiku{De roemruchtige,.}{schrijver kraakte geweldig}{z'n oogen sloten zich}\\

\haiku{En ze zweeg maar, want:}{luid werd ze overstemd door de}{al dringender kreet}\\

\haiku{Doch onverstoorbaar '.}{gingt mannetje in z'n}{hemd met dansen voort}\\

\haiku{C'est doux comme le.}{coeur de toutes ces jeunes}{filles adorables}\\

\haiku{In z'n vreugd om 't ' -,, '!}{eindigen vant seizoen}{ach Godt werd tijd}\\

\haiku{Hij voelde, hoe-ie,.}{weer omhoog ging zichzelf niet}{langer kon sturen}\\

\haiku{De tijdschriften en ' '.}{t publiek zaten met de}{handen int haar}\\

\section{Pol Brounts}

\subsection{Uit: Mien leef lui}

\haiku{'t Had get eweg vaan ':}{nen oonderwijzer dee aon}{Harieke vraog}\\

\haiku{Ouch al umtot ze.}{neet zoe good k\'os dinke wie}{d'n aptieker zelf}\\

\haiku{{\textquoteright} Wie ze 's aoves}{m\`et hunnen twieje bijein}{zaote kraog}\\

\haiku{Mesjiens had heer tev\"a\"ol.}{elend gezeen en gehuurd um}{n\'e\'et kontent te zien}\\

\haiku{Binne pakde iech.}{de kaart um ins te kieke}{wie tot iech zouw rije}\\

\haiku{Es me aajd woord en,,.}{me had gei geld mie mezjieu}{daan heel alles op}\\

\haiku{Heer beloerde miech.}{vaan bove tot oonder veur}{zoeveer es dat g\'ong}\\

\haiku{kleier zoonder ei.}{woord te z\`egke en maakde}{tot ze weg kwaome}\\

\haiku{{\textquoteright} {\textquoteleft}Veer mage,{\textquoteright} zag Leio, {\textquoteleft}.}{langsaameigelek gaaroet}{neet aon h\"a\"om koume}\\

\haiku{Es kr\"a\"olke h\"ob {\textquotedblleft}...}{iech altied mote zinge}{In Paradisum}\\

\haiku{En oonder 't dook {\textquoteleft}{\textquoteright},.}{m\`etI had a dream druimde}{ampa allermins}\\

\haiku{veendelkes in ze, ' {\textquoteleft}}{wegelke de hermenie}{sp\"a\"olde nogne kier}\\

\haiku{Dat huurste wel ins, '.}{d\`ekser tot ze daan ininsn}{opleving kriege}\\

\haiku{In eder geval 'nen,.}{interlektuweel wie Zjang}{dat altied neumde}\\

\haiku{Dat lielek mins in.}{de twiede rij zaog heer}{nog altied zitte}\\

\haiku{Dao kwaom dat lielek, '.}{st\`el m\`et t\"ossen hun int}{printsje vaan Rafael}\\

\haiku{Wee zouw tege die?}{lui gezag h\"obbe tot ze}{dat zoe m\'oste z\`egke}\\

\haiku{Die zwarte m\`es h\"ob,.}{iech zoe d\`eks gezoonge die}{kin iech vaan boete}\\

\haiku{Wat bleef dao nog veur,?}{ier m\`et in te l\`egke m\`et}{zoe'n begraffenis}\\

\haiku{Daan kin me nao de '.}{hoegm\`es koume en dao maak}{iechn st\`el m\`es vaan}\\

\haiku{Neet sjus wie heer 't,.}{ziech had veurgest\`eld meh gans}{anders es anders}\\

\haiku{En toen beg\'os de.}{begraffenis nog ins good}{op dreef te koume}\\

\haiku{Dao verklaore,{\textquoteright}.}{ze alles m\`et allewijl}{zag d'n andere}\\

\haiku{Koffie euver h\"a\"om,,}{eweg euver ze good pak z'n}{botram m\`et gelei}\\

\haiku{Sjoen netuur, prachteg,, '.}{weer lekker werm en zoe en}{ne sjoene camping}\\

\haiku{Ze had toen op h\"a\"or ',.}{batsn dikke roef sjijf wie}{e riestevl\"a\"ojke}\\

\haiku{Good, ze koume bij,:}{d'n dokter dee bekiek ziech}{dat geval en zeet}\\

\haiku{{\textquoteright} Heer duit ins hei op,,,:}{die bats knip ins dao geer kint}{dat wel heh zoe vaan}\\

\haiku{En deen Hollender.}{m\'oste ze dat allemaol}{oetl\`egke vaanzelf}\\

\haiku{Dat waor wie d'n.}{dirrekteur belde of Jean}{effe k\'os koume}\\

\haiku{Heuge waor nog.}{neet dao en heer had ouch niks}{laote hure}\\

\haiku{En zuuste tot dee?}{mins m\`et ziene linkervoot}{achter die ploej zit}\\

\haiku{Este veur ederein,...}{get origineels w\`els h\"obbe}{m\`et e leuk versje}\\

\haiku{Heer wis zeker tot!}{heer veur deen heilige ge{\'\i}n}{surpries had gemaak}\\

\haiku{Dao waoren 'rs, '.}{gen\'og die dat k\'oste gebruke}{al bracht niks op}\\

\haiku{N\'og e gel\"ok waor,.}{tot heer die ruimde gans k\'os}{gebruke ouch nog}\\

\haiku{Meh noe...{\textquoteright} {\textquoteleft}Noe goon veer '!}{op de Vriethof eve opn}{terreske zitte}\\

\haiku{Ze h\'ong wied euver '.}{de leuning en keek naot}{water oonder h\"a\"or}\\

\haiku{H\"a\"oren aosem rook nao.}{kauwgom en oonder zien hand}{voolt heer h\"a\"or sjouwer}\\

\haiku{Heer waor zeker,.}{devaan tot ze sleep meh heer}{keek zelfs neet nao h\"a\"or}\\

\haiku{Tegeneuver h\"a\"om.}{beg\'os heer de contoure vaan}{de hoezer te zien}\\

\haiku{'t Jakkere, 't, ', ', '...}{mopperet vloket}{lachet keke}\\

\haiku{{\textquoteright} zag ze altied, {\textquoteleft}en,,.}{noe wie iech aajd weur w\`el iech}{ins lekker niks doen}\\

\haiku{Zoe get veur op te,......{\textquoteright} {\textquoteleft}?!}{fietse meh daan aanders eh}{Veur op te fietse}\\

\haiku{Z\`egk miech mer wienie ' '.}{tott uuch oetkump en iech}{maakt in orde}\\

\haiku{Menier Hellergers ' '.}{woordne gere gezene}{gas int krinkske}\\

\haiku{Edere kier es heer.}{in de k\`erk kaom m\'os heer}{weer denao kieke}\\

\haiku{{\textquoteright} Noe meint geer mesjiens tot.}{Antonius neet zoe d\`eks}{in die k\`erk kwaom}\\

\haiku{d'n heilege zien?}{tot de lui hei speciaol}{tot h\"a\"om koume beie}\\

\haiku{Meh es mien vrouw dat... '?}{oets ter oere kump Kinstet}{diech veurst\`elle}\\

\haiku{Beter tot heer 't.}{zeet es eine dee bij ze}{volle verstand is}\\

\haiku{Es iech zeker wis ',.}{tot zijt neet zouw hure}{daan heel iech m'ch st\`el}\\

\haiku{{\textquoteright} {\textquoteleft}Luuster ins hei,{\textquoteright} zag, {\textquoteleft}.}{menier noe hel-opiech haw}{neet vaan die smoesjes}\\

\haiku{{\textquoteright} {\textquoteleft}Iech geluif tot iech,{\textquoteright}.}{h\"a\"om h\"ob aosemde de jong}{in mevrouw h\"a\"or oer}\\

\haiku{{\textquoteright} Ze g\'ongen achter.}{h\"a\"om stoon en loerden euver}{z'ne wielvinger}\\

\haiku{Blaajer vlogen in de,,...}{runde tek braoke struuk}{woorte plat getrooje}\\

\haiku{{\textquoteleft}Iech weit 't neet... Iech...{\textquoteright},.}{meinde vaan wel zag ze get}{understebove}\\

\haiku{En tot 't ouch neet '.}{umn verkrachting g\'ong k\'os}{heer ouch direk zien}\\

\haiku{{\textquoteleft}Neet wijer loupe,,{\textquoteright}.}{neet korterbij koume zag}{heer autoritair}\\

\haiku{Dao hingk eine in,{\textquoteright}.}{de boum constateerde de}{wandeleer neuchter}\\

\haiku{Godmieljaar, wie die,.}{ineens opst\'ong dach ik tot}{ik wat krijge moes}\\

\haiku{Ze zagte ziech wel, {\textquoteleft},,{\textquoteright},.}{goojendaagmorge middag}{zoe meh wijer niks}\\

\haiku{Wie zij eindelek k\'os,.}{opstoon en nao boete}{g\'ong waor heer eweg}\\

\haiku{Ze had h\"a\"or m\`et nao '.}{binne genome en h\"a\"or}{opre sjoet gepak}\\

\haiku{Boete beg\'os 't.}{noe ech te regene en}{hel te wejje}\\

\haiku{H\"ob geer mesjiens get aon,?}{eur veuj tot geer zoe langsaam}{en umziechteg lop}\\

\haiku{Dee perfesser daan,.}{heh dee vert\`elde tot veer}{gans verkierd loupe}\\

\haiku{Dus noe h\"ob iech hei ' '.}{nen have kilo en dao}{nen have kilo}\\

\haiku{Heer kretsde ziech ins '}{m\`ett potloed op z'ne}{kop en bedach ziech}\\

\haiku{Meh... wiev\"a\"ol gram is, ',...?}{noe en noe kumpt wiev\"a\"ol}{gram is noe e{\'\i}n oons}\\

\haiku{{\textquoteleft}Wee zelf niks deit, moot.}{neet meine tot zij alles}{wel zal ranzjere}\\

\haiku{{\textquoteleft}Huurt ins, geer kint toch,?}{dat aajd vaan bij miech in de}{straot vaan Berens}\\

\haiku{{\textquoteleft}Iech h\"ob 't uuch toch,{\textquoteright}.}{gezag constateerde die}{vaan Janse kontent}\\

\subsection{Uit: Zal iech uuch ins get vert\`elle...}

\haiku{Jao, iech moot wel bij,,.}{d'n dokter zien veur mie bein}{meh iech h\"ob d'n tied}\\

\haiku{Doog ze miech ope in!}{h\"a\"ore nachjepon of zoe get.}{Noe vraog iech uuch}\\

\haiku{Heer staok ziene.}{kop weer nao binne en keek}{h\"a\"or trouwherteg aon}\\

\haiku{Ze had h\"a\"ore peignoir.}{aongetrokke en stoond angsteg}{nao h\"a\"om te kieke}\\

\haiku{Mesjiens bin iech wel de '{\textquoteright},.}{verkierde kant  aont}{op loupe dach heer}\\

\haiku{Iech blijf hei zitte, '{\textquoteright}.}{tot ze m\"orge opkump daan}{weet iecht zeker}\\

\haiku{En tot 't dao 'ne.}{gekkeboel waor en tot}{ze dao beer droonke}\\

\haiku{kinder, tot die gen\'og}{hadde aon get water en}{get zand of modder}\\

\haiku{{\textquoteleft}Ze begraove.}{de lui allewijl in hun}{zoondags pekske}\\

\haiku{Heer droonk ze glaas leeg,,, ':}{stoond op puunde Tiny gaof}{Maxn hand en zag}\\

\haiku{Bij de Chineze ',.}{m\'o\'oste sjus rupse naot ete}{Dat is dao beleef}\\

\haiku{De juffrouw zal toch,?}{ouch wel ins rupse of e}{puupke laote}\\

\haiku{M\`et ein hand in de.}{tes vaan ziene regejas}{leep heer nao de deur}\\

\haiku{{\textquoteright} Heer keek W\"ollems}{sjalleks aon en beg\'os}{toen te keke vaan}\\

\haiku{Heer lag e st\"okske {\textquoteleft}?}{broed op zien oetgestoke}{hand.Zeetd'r h\"a\"or nog}\\

\haiku{Meh d\'a\'an, en dat is,{\textquoteright}.}{zoe daan kin iech die alles}{liere wat iech w\`el}\\

\haiku{Die had iech oet e.}{n\`es gehaold wat oet de boum}{waor gevalle}\\

\haiku{In 't begin veel,.}{miech dat tege huur dat kin}{iech uuch verklappe}\\

\haiku{Allein jaomer tot.}{me begraffenisse neet}{zoe in de hand heet}\\

\haiku{t Geb\"a\"orde op ' ',.}{n nach naonen aovend wie}{de meiste nachte}\\

\haiku{{\textquoteleft}Jeh, ze zal ouch wel{\textquoteright},.}{klem begonne zien zag heer}{get ge{\"\i}rriteerd}\\

\haiku{M\`et al deen ambras.}{zouwe ze dat nog haos}{vergete h\"obbe}\\

\haiku{{\textquoteright}        Nuits {\textquoteleft}Noe valle,{\textquoteright},.}{d'ch toch d'n sjeun oet este}{dat huurs zag de vrouw}\\

\haiku{M\`et mier es doezend.}{doeje in Zuid-Afrika en}{vief daog rotweer}\\

\haiku{Iech waor te zier{\textquoteright}, '.}{verdeep in mie book zagr}{get veroontsj\"oldegend}\\

\haiku{Wie de zakemaan ' '.}{h\"a\"or zaog kraog heerne}{kop wiene pieper}\\

\haiku{Meh iech daon gein,.}{oug mie touw es die aon de}{geng geit nondepie}\\

\haiku{Op de gezoondheid.}{vaan mevrouw Dezems vaan de}{sjeettent neve miech}\\

\haiku{En... Geer zouwt ouch nao{\textquoteright},.}{de gemeinte kinne goon}{bedach ze obbins}\\

\haiku{{\textquoteleft}Zouwe veer effe '?}{n tas koffie goon drinke}{dao op dat terras}\\

\haiku{{\textquoteright} Noe stoond ze haaf op,.}{oet h\"a\"ore stool meh m\`et ein hand}{heel heer h\"a\"or tege}\\

\haiku{t Zouw h\"a\"om neet zoe.}{verbaas h\"obbe es ze hel}{waor goon keke}\\

\haiku{De juffrouw woord e,:}{bitteke raos in h\"a\"or}{geziech meh ze zag}\\

\haiku{'nen Awwere maan,.}{dee zien jong vrouw content m\'os}{zien te hawwe}\\

\haiku{Mesjiens es iech links ouch ', '.}{n tes had tot iechm daan}{toch rechs leet zitte}\\

\haiku{Wie Zjo tr\"okkaom aon ',.}{t t\"a\"ofelke keek de maan}{h\"a\"om oonzeker aon}\\

\haiku{Heer droonk zie glaas leeg.}{en best\`elde e rundsje}{veur de taofel}\\

\haiku{'r Waor mager '.}{gewoorde enr waor}{gans oet zienen doen}\\

\haiku{{\textquoteleft}Hiere, toen... toen g\'ong 'r... '...}{nao de kis lufde z'ne}{poet op en deegt}\\

\haiku{{\textquoteleft}Jeh, es d'r miech neet, '...... '{\textquoteright}.}{koelek numpt eht heet}{wel get eweg vaan uuch}\\

\haiku{ze allemaol:}{en Gabri\"el kaom nao v\"a\"ore}{en zag hiel serjeus}\\

\haiku{Ze g\'ong h\"a\"om veur nao ', '.}{n kamer boe ze h\"a\"omne}{lere fauteuil wees}\\

\haiku{'t Waor daobij. '.}{vochteg en k\`ellegt Zouw}{vas goon regene}\\

\haiku{Nondepie, die had!}{heer netuurlek in de b\"os}{laote ligke}\\

\haiku{hunne fort waor, '}{hadde zet toch veerdeg}{gesp\"a\"old veur Sjarel}\\

\section{Johan Brouwer}

\subsection{Uit: De schatten van Medina-Sidonia (onder ps. Maarten van de Moer)}

\haiku{Het eenige wat me.}{interesseert is de mensch}{en zijn problemen}\\

\haiku{Huichelachtige,.}{wereld die niet in zijn rust}{wil gestoord worden}\\

\haiku{Zij willen thuis dat,.}{ik rechten afstudeer en}{het staat me tegen}\\

\haiku{Morgen kom ik wel,.}{even aan dan praten wij nog}{wel over je studie}\\

\haiku{En zonder dat ik,.}{het goed besefte begon}{ik stil te schreien}\\

\haiku{Zooals het geval met,.}{die snuifdoos en andere}{kleine gevallen}\\

\haiku{U weet wel, die groote,,....}{student hoe heet hij ook weer}{iedereen zegt Zeus}\\

\haiku{Mijn besluit zou hun,.}{verdriet doen maar ik kon niet}{anders handelen}\\

\haiku{Dat moet dat plein zijn,.}{waar in de Julidagen}{zoo gevochten is}\\

\haiku{Stel je voor dat zij.}{nu ineens Barcelona}{gingen bombardeeren}\\

\haiku{Drie matrozen en,.}{een onderofficier met}{een heel jong vrouwtje}\\

\haiku{Op het perron in.}{Tarragona was het een}{drukte van belang}\\

\haiku{Een vreemdeling, een,.}{indringer temidden van}{een gezin in rouw}\\

\haiku{Het was een jonge,,.}{man een arbeider blijkbaar}{in blauwe overall}\\

\haiku{Zij hebben beiden,.}{hun einde gevonden zooals}{zij dat wenschten}\\

\haiku{Er gaan er te veel,.}{naar Madrid alleen om hun}{eindje te vinden}\\

\haiku{De wapens waren.}{gekomen en er waren}{auto's beschikbaar}\\

\haiku{De Aragonees nam.}{hem op en legde hem aan}{den kant van den weg}\\

\haiku{Twee lijken heb ik.}{half opgericht om te zien}{of ik het zelf was}\\

\haiku{Deze linie schijnt.}{al op enkele plaatsen}{verbroken te zijn}\\

\haiku{Paco trekt me mee,.}{een trap af een station}{van de metro in}\\

\haiku{Kogels kletteren.}{tegen de steenen of maken}{kleine pluimpjes stof}\\

\haiku{Ik ben zoo moe, dat.}{ik mijn geweer nauwelijks}{meer kan vasthouden}\\

\haiku{De schilderijen.}{schenen zeer gevoelig voor}{die verandering}\\

\haiku{De San Francisco{\textquoteright}.}{is onkwetsbaar had ik reeds}{vaak hooren zeggen}\\

\haiku{De fiches die ik,.}{invulde legde ik op}{groote tinnen schalen}\\

\haiku{Die Oudejaarsavond.}{was een kortstondige vlucht}{in de illusie}\\

\haiku{Voorop liep een man.}{met een leeren windjack aan en}{een zwart mutsje op}\\

\haiku{Bovendien was er,.}{nog een zeer groote som ook in}{gouden dubloenen}\\

\haiku{Ik zag haar ook, een,.}{enkele maal zonder dat}{mij dat verschrikte}\\

\haiku{Dit probleem scheen hem.}{meer te interesseeren}{dan mijn proefneming}\\

\haiku{Zij gaf mij echter,.}{de hand zonder dat zij mij}{scheen te herkennen}\\

\haiku{dat ik wel Russisch,.}{kende maar redenen had}{dit te verzwijgen}\\

\haiku{Zelfs verscheidene.}{leden van de Alianza}{zaten gevangen}\\

\haiku{Er werd gesproken.}{van een dreigenden opstand}{in Barcelona}\\

\haiku{Die psychiater.}{had al in weken niets van}{zich laten hooren}\\

\haiku{Schmidt, Iwanow, of een,.}{van de anderen die weet}{dat ik alleen ben}\\

\haiku{- Ik geloof, dat er.}{een middel is om je van}{Iwanow te bevrijden}\\

\haiku{De psychiater.}{ontving me in het bureau}{van de inspectie}\\

\haiku{Ik heb ze zelf niet,.}{gezien ik heb er alleen}{over hooren spreken}\\

\haiku{- Hoeveel tijd zal er?}{noodig zijn voor het invullen}{van alle stukken}\\

\haiku{De jongen nam een.}{steentje en raakte de geit}{boven op den kop}\\

\haiku{Nadrukkelijk zei.}{ik hem dat die ontdekking}{geheim moest blijven}\\

\haiku{- Neen, er staat een groote,.}{steen voor maar daarachter moet}{nog een ruimte zijn}\\

\haiku{Wij kwamen in een,.}{vertrek dat iets kleiner was}{dan het andere}\\

\haiku{Zij voelden zich fier,,,.}{zij voelden zich soldaten}{strijders krijgslieden}\\

\haiku{Waar kwam ik met dien?}{doode vandaan en waar ging}{ik er mee naar toe}\\

\haiku{Ik wist slechts dat ik,.}{daar aankwam en anderen}{mijn last overnamen}\\

\haiku{Zijn lichaam scheen te,.}{ontspannen zijn hoofd zonk nog}{zwaarder op mijn arm}\\

\haiku{Terwijl ik wegzonk.}{werden groote stukken steen over}{mij heen geworpen}\\

\haiku{Zij hebben de zorg.}{voor de bibliotheek en}{de kerksieraden}\\

\haiku{Dertien steeds zwakker.}{wordende lichten waren}{dertien koningen}\\

\haiku{Zes rijen van vier.}{boven elkaar en nog twee}{boven den ingang}\\

\haiku{In wezen zal het.}{dan gedaan zijn met Spanje's}{zelfstandig bestaan}\\

\haiku{- Ik geloof, dat diep.}{in zijn hart niemand in zijn}{eigen dood gelooft}\\

\haiku{Het gevaar was een.}{wezenlijk element van het}{bestaan geworden}\\

\haiku{Men ranselde hem,.}{gruwelijk af en men nam}{het crucifix weg}\\

\haiku{Het Escuriaal{\textquoteright}.}{is het uitspansel van den}{geest van Philips II}\\

\haiku{Ik herinnerde.}{hem aan ons voorgenomen}{onderzoek aldaar}\\

\haiku{Het oorlogsgeweld,.}{kan hier slechts een dichterlijk}{een episch motief zijn}\\

\haiku{De veelvuldige.}{bombardementen waren}{als natuurrampen}\\

\haiku{Deze opening bleek.}{groot genoeg te zijn om een}{man door te laten}\\

\haiku{Er moest echter een;}{gemakkelijker toegang}{tot den Toren zijn}\\

\haiku{Zij onderzochten.}{op hun beurt de muren en}{den grond nauwkeurig}\\

\haiku{- Ik had een holte,.}{onder den grond verwacht zei}{de psychiater}\\

\haiku{Wij moeten probeeren.}{of deze steen ook zulk een}{mechaniek bezit}\\

\haiku{- Het wordt tijd dat nu.}{eens een rustig man hier de}{spa in den grond steekt}\\

\haiku{Het trof ons dat de.}{steen zelf en de aarde er}{om heen vochtig was}\\

\haiku{De zerk kwam een paar,.}{centimeter omhoog en}{brak toen midden door}\\

\haiku{Wij zijn reeds buiten.}{die magische sfeer van het}{gevaar gekomen}\\

\haiku{Mijn militaire.}{dienst was in Nederland een}{geheim gebleven}\\

\haiku{Ik zie de grenzen.}{niet tusschen mijn droomen en}{de werkelijkheid}\\

\section{Carry van Bruggen}

\subsection{Uit: Avontuurtjes}

\haiku{zoo hoog over heen en.}{jaagt het als in stukken naar}{alle kanten weg}\\

\haiku{en ze verbaast zich.}{dat ze er anders nooit aan}{denkt en het nooit voelt}\\

\haiku{hij zal toch niet met....?}{dien wild-vreemde over hun}{kwitantie spreken}\\

\haiku{maar in de keuken,....}{is nog warmte is nog lucht}{van eten en koffie}\\

\haiku{dat doet de oude,....}{als het vacantie is maar}{deze weet van niets}\\

\haiku{De dokter is nog,}{niet thuis gekomen ze weet}{het al voor de meid}\\

\haiku{Ze wil het beven,....}{van haar mond bedwingen want}{ze moet het zeggen}\\

\haiku{anders zou het haar....}{wel uit de verte tegen}{het schrikding helpen}\\

\haiku{Naar wat je zoo hoort,:}{om je heen bestaan er zes}{soorten van menschen}\\

\haiku{Ze kijkt gauw weer voor,,.}{zich uit want Kleij komt langs het}{pad haar achter-op}\\

\haiku{rillingen ritsen....}{over haar hoofd en haar oogen zijn}{ineens vol tranen}\\

\haiku{Zij hebben niet zoo -;}{heel goed geluisterd en niet}{zoo heel veel verstaan}\\

\haiku{Ze sluipt op haar teenen,,,....}{wrikt de grendels voorzichtig}{hun kokertjes in}\\

\haiku{vlak daarnaast woont Brons,,}{hij is een vriendelijke}{oude weduwnaar}\\

\haiku{Ze merkt ineens haar,.}{eigen loopen het loopen}{wordt er anders door}\\

\haiku{Nu moeten ze even,.}{zitten ze loopend op te}{eten zou zonde zijn}\\

\haiku{Je ziet de menschen,....}{al aan het dek je hoort de}{boot stampen en stoomen}\\

\haiku{alles aan hem wordt....}{elke seconde grooter}{en duidelijker}\\

\haiku{Het is louter voor,,!}{de grap dat Vader zoo praat}{dat Vader zoo doet}\\

\haiku{Ze is de deur uit}{en naar de kermis gedraafd}{en nooit leek de weg}\\

\haiku{en het lijkt wel of,....}{hij nu echter lacht maar dat}{kan natuurlijk niet}\\

\haiku{hij is niet kwaad en,....}{niet norsch alleen maar ruig}{en donker en oud}\\

\haiku{gelukkig voor die,,.}{lange bleeke vrouw want hij is}{natuurlijk haar man}\\

\haiku{ze opent ze weer, en.}{rondom isde volheid van}{het lachende licht}\\

\haiku{ze zoeken elk en,.}{vinden ze een voor een elk}{op een vreemde plek}\\

\haiku{Ze zitten een heel,....}{eind hooger dan de straat dat}{is alleen maar hier}\\

\haiku{hij is de eerste....}{en eenige die ze ooit in}{het speelveld vonden}\\

\haiku{je hoort ze niet, nu,.}{liggen ze op de hoop die}{al gevallen was}\\

\haiku{Men kijkt niet meer, naar,.}{Aap of Beer Maar alleen wij}{genieten de eer}\\

\haiku{en ook {\textquoteleft}Neen, mis{\textquoteright} en!}{dan even wachten en dan zingt}{de heele zaal mee}\\

\haiku{acht Want ze wisten,.}{van hun baboe dat je daar}{altijd krijgt wat toe}\\

\haiku{Maar moeder heeft hem {\textquoteleft}.}{al zoo lang en noemt hemmijn}{akker en mijn ploeg}\\

\haiku{en zij stond midden....}{in de kamer en keek hem}{aan terwijl hij las}\\

\haiku{acht naadjes boven,!}{je teekje en als het mooi}{is zelfs wel eens tien}\\

\haiku{De jongens staan er,....}{nog en lijken bijna in}{het riet verzonken}\\

\haiku{de witte streep breekt,....}{uit zijn bruin gezicht dat was}{zooeven strak en dicht}\\

\haiku{de mannen en de,....}{zonen hun praten kaatst in}{het kale licht}\\

\haiku{die boer dacht Heilbron,....!}{te bedotten maar Heilbron}{was dien boer te slim}\\

\haiku{, voelt zacht, doet zoete.}{schommeling door je leden}{gaan en in je keel}\\

\haiku{Neen, het jaar is kaal,....}{en zwart want Simchas-touro}{was de laatste bloem}\\

\haiku{van hoog uit valt de,}{stijve straal die is als een}{ijzen staaf en boort}\\

\haiku{Er was een orgel,,....}{er kwam een bedelman en}{moeder zette thee}\\

\haiku{Het duurt nooit lang, maar,.}{het moet zijn eigen beloop}{hebben als een gaap}\\

\haiku{Dat vroeg hij en ze,.}{moest er zelf om lachen zoo}{grappig als het klonk}\\

\haiku{en als het niet lukt,........}{dan kun je toch nog altijd}{doen alsof het lukt}\\

\haiku{Neen, ze zat roerloos,,}{neer ze gaf geen geluid want}{ze wil dat hij denkt}\\

\haiku{lukt het niet, krijg je,,.}{straf is dat voorbij denk je}{er ook niet meer aan}\\

\haiku{Zij-zelf hebben....}{thuis het eene kistje van Oom}{Elie nog bijna vol}\\

\haiku{En Sabbathisten,:}{gedenken den Sabbath want}{er staat geschreven}\\

\haiku{schemerig schommelt,.}{zijn gebogen lijf zijn stem}{komt van onder op}\\

\haiku{grijs het water, het:}{steigertje en levend als}{van stekelbaarsjes}\\

\haiku{Je zou bang kunnen,....}{zijn maar niet met Vader en}{niet met den schipper}\\

\haiku{en bijna wist ze....}{wat er dreef in het water}{en wat er zoo rook}\\

\haiku{ze hoeft zich niet zoo,....}{bang te maken zij is toch}{zelf niet op het schip}\\

\haiku{Het schip heeft er dus,!}{zeker niets mee te maken}{maar w\`el zij-zelf}\\

\haiku{het mag wel.... alleen,.}{er zijn zoo maar in de week}{geen appels in huis}\\

\haiku{hij vergeet haar, laat,....}{haar in den steek is met zijn}{vriendjes gaan spelen}\\

\haiku{je zusje van de,....}{trappen te duwen ze had}{wel dood kunnen zijn}\\

\haiku{je loopt erin, je,}{kunt ertegen vechten je}{hebt licht en menschen}\\

\haiku{{\textquoteright} en iets in je wordt....}{dadelijk glad en warm naar}{omlaag gestreken}\\

\haiku{wiekt enkel nog maar,....}{op schemerige grijze}{vlerken om haar heen}\\

\haiku{Of ze denken, dat,.}{Vader wel graag wil maar niet}{kan schaatsenrijden}\\

\haiku{waarom, want je noemt,....}{het rijden maar je kan het}{ook loopen noemen}\\

\haiku{H\`e.... kan nog iemand?}{anders zoo als Vader in}{de handen klappen}\\

\haiku{en van dien kant af.}{heb jullie nog nooit  in}{den trein gezeten}\\

\haiku{Maar liever niet, ze.}{staat er liever naast en houdt}{haar hand om den rand}\\

\haiku{je loopt en voelt geen,,....}{straat je zolen slapen zelf}{slaap je bijna ook}\\

\haiku{het ijs verlaten,,.}{dichtbij en ver de menschen}{allemaal naar huis}\\

\subsection{Uit: Een coquette vrouw}

\haiku{Ze was immers niet -!}{ziek en ze had hem zelf niet}{gevraagd te komen}\\

\haiku{Iedereen hier is.}{bedaard en beschaafd en heeft}{goede manieren}\\

\haiku{Het tegen deel van.}{alles wat ik mij voorstel}{dat een man moet zijn}\\

\haiku{Ik weet ook niet of -.}{u het mij zeggen mag maar}{u zegt het mij wel}\\

\haiku{En dat wist u dus,?}{allemaal precies toen u}{hier de trap op kwam}\\

\haiku{Ze wist allang van -,.}{niet en toch verwachtte zij}{het steeds weer opnieuw}\\

\haiku{- wat een kracht, wat een.}{hoogheid beduidde zooveel}{zelfverzekerdheid}\\

\haiku{Hij kuste haar weer,,,.}{maar ze verzette zich half}{in ernst half schertsend}\\

\haiku{ik zoo weinig voor.}{familie en heelemaal}{niets voor portretten}\\

\haiku{Sinds de geboorte;}{van het kind kwam Geerte weer}{vaker dan vroeger}\\

\haiku{Egbert trommelde,.}{met een vouwbeen op tafel}{glimlachte en zweeg}\\

\haiku{{\textquoteright} ijverde Geerte, {\textquoteleft}.}{er komen heel aardige}{en geschikte lui}\\

\haiku{Vooral als ze al,.}{haar vriendjes hier vindt om haar}{het hof te maken}\\

\haiku{Nu is het gauw uw,{\textquoteright}, {\textquoteleft}?}{beurt zei hijmag ik hier op}{u blijven wachten}\\

\haiku{Ze kon ook nu niet.}{plotseling weigeren met}{hem aan te zitten}\\

\haiku{{\textquoteleft}Ze is overspannen,{\textquoteright}, {\textquoteleft}.}{vond dieze zou eens een poos}{naar buiten moeten}\\

\haiku{natuurlijk vond, dat.}{ze elkaars bijzijn niet meer}{verdragen konden}\\

\haiku{wie kon die vreemde,?}{wezen met zijn beschaafde}{welluidende stem}\\

\haiku{Het is heelemaal.}{geen deugd of edelaardigheid}{of iets van dien aard}\\

\haiku{{\textquoteleft}{\"\i}k kan anders niet,.}{zeggen dat die waschbaas}{van jou mij bevalt}\\

\haiku{Ze was dien middag.}{bij Paul geweest en hij had}{den tijd vergeten}\\

\haiku{En ik heb je maar -,}{\'e\'en ding te zeggen dan kan}{je daarna razen}\\

\haiku{het echte geluk.}{van die anderen en hun}{eigen schijngeluk}\\

\haiku{Maar ik zal in elk,.}{geval wel zorgen dat je}{wat fatsoenlijks krijgt}\\

\haiku{Als ik een man was,,,.}{zou ik vrouwen geloof ik}{heel aardig vinden}\\

\haiku{Alle liefde en.}{alle vertrouwelijkheid}{waren immers \'e\'en}\\

\haiku{Van je-zelf, en,.}{van de heele wereld het}{geheele heelal}\\

\haiku{je kan wel zooveel{\textquoteright},.}{willen of een dergelijk}{onzinnig antwoord}\\

\haiku{{\textquoteright} riep Ina, bevend van,, {\textquoteleft}?}{drift met trillende lippen}{kan die nog huilen}\\

\haiku{Ze zouden haar niet,.}{langer zwak ze zouden haar}{niet in tranen zien}\\

\haiku{{\textquoteright} De andere vrouw.}{boog zich naar Paul over en trok}{hem even aan het oor}\\

\haiku{In het begin van:}{het jaar hadden ze nieuwe}{buren gekregen}\\

\haiku{Wij hebben meer dan,,,!}{het gemiddelde Egbert}{veel meer geloof ik}\\

\haiku{Jij zag den jongen,.}{niet jij zag alleen jezelf}{en wat je noodig had}\\

\haiku{{\textquoteright} {\textquoteleft}Het is heerlijk, te{\textquoteright} -,:}{spreken met weinig woorden}{zei Ina zacht en dan}\\

\haiku{Ik wil een thuis -, ik,.}{wil een vrouw ik wil met rust}{gelaten worden}\\

\haiku{{\textquoteleft}Ja -, Charley sprak over {\textquotedblleft}{\textquotedblright},.}{tweestroomingen in ons}{zooals ze het voelde}\\

\haiku{Ze zouden hem zelf -,.}{terugbrengen hij zal zoo}{meteen wel komen}\\

\haiku{{\textquoteright} zei Josefine.}{met een kleur van pijnlijke}{verontwaardiging}\\

\haiku{Kijk -, ik neem het voor,.}{Otto mee het zal hem wel}{interesseeren}\\

\haiku{En daarom doen mij,.}{die dingen zoo'n pijn die ze}{nu van je zeggen}\\

\haiku{Vertel eens even, weet,?}{Annie dat je naar mij toe}{bent en dat ik kom}\\

\haiku{Het gaat mij niet aan -,.}{en het interesseert mij}{eigenlijk ook niet}\\

\haiku{En de dames d\'a\'ar.}{verlangen blijkbaar heel erg}{naar uw terugkomst}\\

\haiku{Zoo kon het nog uren,,.}{duren den geheelen dag}{den ganschen nacht door}\\

\haiku{Wandelen -, alleen,,?}{in de mist in de kilte}{en daarna naar huis}\\

\haiku{{\textquoteright} vroeg Ina, {\textquoteleft}ik dacht een,.}{oogenblik aan wat anders}{ik verstond je niet}\\

\haiku{Die groote blonde vrouw,,?}{weet je wel die je kerstavond}{bij mij ontmoet hebt}\\

\haiku{Arnold heeft plaatsen,.}{voor een nieuwe operette}{dus wij zijn niet thuis}\\

\haiku{Waar had ze zich in -,?}{vastgewerkt hoe moest ze er}{zich weer uitredden}\\

\haiku{{\textquoteright} {\textquoteleft}Vertel eerst eens wat,{\textquoteright} {\textquoteleft}.}{van je zelf verzocht Charley}{die reis komt straks wel}\\

\haiku{{\textquoteright} {\textquoteleft}Van mezelf,{\textquoteright} zei Ina,, {\textquoteleft}}{mat en schouderophalend}{ze keek Charley aan}\\

\haiku{Hij plukte alleen.}{bloemen om ze iemand te}{kunnen aanbieden}\\

\haiku{de formule zoo,.}{zou hem niet erg aanstaan maar}{het is zijn praktijk}\\

\haiku{dit was het einde,.}{van haar laatsten droom van haar}{jongste illusie}\\

\haiku{Het woord klonk dof -, de.}{gedachte zette haar hart}{niet langer in gloed}\\

\subsection{Uit: Eva}

\haiku{Je hebt gewacht, je.}{hebt er een plaats in jezelf}{voor open gehouden}\\

\haiku{Ik zit nu op een,,.}{toren er is geen vuil dat}{zoo hoog spatten kan}\\

\haiku{Je hebt het ook met -,.}{wie het niet verdienen met}{den boozen padrone}\\

\haiku{een beetje beschaamd,.....}{een beetje verslagen en}{een beetje getroost}\\

\haiku{En zou er nu zoo,....}{iets bestaan als een zaad waar}{alles in klaar ligt}\\

\haiku{en je ziet,.... blauwe,....,....}{hemel blauw water gele}{bloemen bloeiend gras}\\

\haiku{In glinsterkringen.}{en zilverig schuim staat het}{kortgesneden riet}\\

\haiku{En de eerste dag,,!}{dien ze langs hun oogen voorbij}{zagen gaan zij zelf}\\

\haiku{Achter ze is de,,....}{lange donkere kap als}{een drukkende hand}\\

\haiku{Elken dag weer maakt,.... '}{haar de stad tot het zijne}{neemt haar in bezit}\\

\haiku{links en rechts een snoer.}{van gouden tientjes boven}{den donkeren stroom}\\

\haiku{Ik kan al niet eens,....}{goed tegen gymnastiekles}{om de commando's}\\

\haiku{Dus dat je je niet,....}{wou laten onderzoeken}{dat begrijp ik best}\\

\haiku{Reuk, laat mij los, plaag -,.}{mij niet om een naam ik kan}{hem je niet geven}\\

\haiku{o, maar je mag zoo,.}{niet kijken zoo onder je}{wenkbrauwen omhoog}\\

\haiku{later werd het een,....}{walgelijk woord  omdat}{er altijd iets van}\\

\haiku{tusschen buurman Bol,,.}{en buurman Bruin daar woont ze}{daar wordt ze verwacht}\\

\haiku{Maar kun je helpen {\textquoteleft}{\textquoteright}.}{dat je weet watperoe}{oe-rewoe beduidt}\\

\haiku{Ze zit achteruit,.}{in haar stoel laat het lamplicht}{in haar oogen bijten}\\

\haiku{Neen, van dat eene Hoofd,, {\textquoteleft}}{dien aardigen heer die zoo}{nadrukkelijk zei}\\

\haiku{hij wil voor alle,....}{examens later werken waar}{David nu voor werkt}\\

\haiku{Het is, als kwamen,.}{ze van een verre reis als}{waren ze vermoeid}\\

\haiku{Vergeet het nu even,,.}{leg het van u af dat u}{Grootvaders zoon bent}\\

\haiku{wrevelig knarsen.}{zijn zware voeten over den}{zanderigen vloer}\\

\haiku{Was dit staan over een?}{kom met rozen gebogen}{dan weer zoo iets geks}\\

\haiku{hij zei iets tegen,,....}{Selien hij lachte gooide}{zijn hoofd achterover}\\

\haiku{Even staat Joppe tot,.}{stilte beschaamd maar dat even}{is alweer voorbij}\\

\haiku{de oude, felle....}{oogen boren zoo recht en zoo}{diep de zijne in}\\

\haiku{een zee, zwart-grauw,.}{uit een lichtende toren}{met melk overgoten}\\

\haiku{dat ze je gevraagd,.}{heeft bij haar en Dora te}{komen inwonen}\\

\haiku{dat hij met Joop op,....}{schoot in het kamertje heeft}{gezeten dat hij}\\

\haiku{Het schijnt dat hij thuis,,:}{op niets meer aan niemand meer}{antwoord geeft hij zegt}\\

\haiku{Het is alsof je....}{een man den brand zag steken}{in zijn eigen huis}\\

\haiku{Ze zeggen alles,.}{verschillend ze bedoelen}{alles verschillend}\\

\haiku{daar stond Herman in,,.}{de deur met zijn guitaar met}{zijn nieuwe liedjes}\\

\haiku{{\textquoteright} Ernestien, zeg het,,.}{nu maar vraag het nu maar dan}{is het geleden}\\

\haiku{Leun roerloos tegen,........}{het regenraam sta stil in}{de stilte en wacht}\\

\haiku{het is ongepast,....}{en het is je geluk dat}{je maar alleen bent}\\

\haiku{{\textquoteright} Het laatste dat ze,,:}{Ernestien heeft geschreven}{was op een briefkaart}\\

\haiku{Ebner, arme man,,,....}{eenzame man oude man}{kom naar mijn lippen}\\

\haiku{boven hun hoofden....}{kwam Hugo zijn klas in en}{stemde zijn viool}\\

\haiku{met een enkelen,,....}{lach een handwuif achterom}{een blinkenden blik}\\

\haiku{d{\`\i}t prijsgeven van,,....}{jezelf is het grootste het}{mooiste het hoogste}\\

\haiku{En 's avonds kwam hij,....}{je halen en je praatte}{samen in het park}\\

\haiku{wijd-open word je....}{getrokken en er stijgen}{geruischen op}\\

\haiku{dit deeltje Hauptmann....,....}{en kijk hij houdt zooveel van}{zoute bolletjes}\\

\haiku{het blaasje richtte,,........}{zich op het zette uit het}{zwol en sloeg uiteen}\\

\haiku{ze kennen alleen,,....}{samen het eene de eene en}{dat kent hen allen}\\

\haiku{Het moet zoo geweest,,.}{zijn of jij waart hier nu niet}{met mij in mijn schoot}\\

\haiku{Krankzinnig zijn we,,,....}{immers wij menschen met den}{dood overal overal}\\

\haiku{David was na die.}{ziekte van voorverleden}{jaar nooit meer gezond}\\

\haiku{hier een man, die een....}{klein meisje vermoordt voor haar}{gouden belletjes}\\

\haiku{{\textquoteleft}Pas maar op, dat je{\textquoteright}.}{niet met paraplu en al}{de hoogte in gaat}\\

\haiku{{\textquoteright} {\textquoteleft}Nu ga je dan ook,,.}{allebei voor straf onder}{den mantel vandaan}\\

\haiku{Het sterft je af -, maar,.}{zoo is het niet je laat het}{achter op je weg}\\

\haiku{Zij zijn daarvoor nog....}{niet eens dicht genoeg bij het}{witte vuur geweest}\\

\haiku{Daar staat Heleen -, ze.}{heeft mij al gezien en ze}{weet dat ik Eva ben}\\

\haiku{maar eens hebben we....}{een eruit gevischt en mee}{naar huis genomen}\\

\haiku{Toen leek het ons ver,,....}{maar het is niet zoo ver we}{gaan de haven langs}\\

\haiku{Hij heeft op David,....}{geparasiteerd hij heeft}{hem uitgezogen}\\

\haiku{En dat David als....?}{dichter begaafd was heb jij}{dat ooit geweten}\\

\haiku{En David sprak het,,.}{ook niet tegen uit trots uit}{onverschilligheid}\\

\haiku{het staat buiten de.}{intellectueele en de}{ethische problemen}\\

\haiku{Ze wilde mij den,.}{brief in mijn handen geven}{maar ze bedacht zich}\\

\haiku{Ook de drift naar den,,,!}{Plicht de inhoudlooze doellooze}{redelooze plichtdrift}\\

\haiku{Omdat het ons zoo, -!}{aangrijpt waar het ons aanraakt}{als het onszelf raakt}\\

\haiku{Wonderlijk bestaan,,.}{waarin juist dit verzwegen}{wordt vergeten wordt}\\

\haiku{Maar laat het zoo zijn,.}{laat het kunnen dat de een}{voor den ander boet}\\

\haiku{Jaap en Ben deinsden,....}{ervoor terug en toen was}{Eddy geboren}\\

\haiku{Want eerst ga ik nu....}{naar Vader en Moeder en}{straks ga ik naar huis}\\

\haiku{Dat we er trotsch op,?}{zijn dat we ons haast schamen}{als het anders is}\\

\haiku{als het hunne in {\textquoteleft},{\textquoteright}....}{de spanning van hetHerder}{laat je schaapjes gaan}\\

\haiku{Ik liep zoo veilig,,:}{ik liep vlak achter Eddy}{en Claartje ik dacht}\\

\haiku{{\textquoteleft}Ik sta ervan te -, -,}{kijken ik hield jullie voor}{gelukkig getrouwd}\\

\haiku{{\textquoteleft}Kunnen we niet eens,,....?}{samen een heelen morgen}{een heelen middag}\\

\haiku{Misschien zoo ver zelfs,}{nog niet ik ben zoo snel van}{daar naar hier komen}\\

\haiku{Maar menschen dienen.}{t\`och behoorlijk aan elkaar}{te zijn voorgesteld}\\

\haiku{Je loopt sinds lang de,.}{kermistent voorbij je lacht}{om de muizenval}\\

\haiku{{\textquoteright} {\textquoteleft}De allergrootste,,.}{dwaasheid op die eene na op}{die andere na}\\

\haiku{maar in mijn begrip,.}{kon ik het toch niet binden}{nooit kreeg het een zin}\\

\subsection{Uit: Goenong-Djatti}

\haiku{Goenong-Djatti ().}{Een Indische roman door}{Carry van Bruggen}\\

\haiku{{\textquoteright} {\textquoteleft}Ja, zeker{\textquoteright} zei De, {\textquoteleft}.}{Klerk met overtuigingzeker}{ben ik gelukkig}\\

\haiku{Maar wie is d'r ten....}{slotte verantwoordelijk}{voor de directie}\\

\haiku{en me daar, nou ja, '....}{laatk zeggen amuseerde}{op plantersmanier}\\

\haiku{En dan voor alle, '.}{gerustheid de wachters}{maars waarschuwen}\\

\haiku{Amelie droeg een wit.}{japonnetje met weidsche}{lichtroode strikken}\\

\haiku{{\textquoteleft}dat je geheimen,}{ni\`et bewaard worden daar}{zorg je zelf wel voor}\\

\haiku{Kijk, haar tjelana.......}{kleeft haar om de beentjes laat}{nonnie nu zoet zijn}\\

\haiku{{\textquoteright} 'n Ommezientje ' '.}{wast dan stil en daarna}{t kind weer terug}\\

\haiku{Goenong-Djatti ().}{Een Indische roman door}{Carry van Bruggen}\\

\haiku{nu liep het tegen.}{den avond en nog waren ze}{niet geheel gereed}\\

\haiku{En ze had juist zoo,,...;}{graag gehad dat hij nu voor}{Charlotte  voor\'al}\\

\haiku{'t Nonnaatje-zelf, '.}{was er ook ontsteld van met}{n zweem van schaamte}\\

\haiku{{\textquoteright} Enkelen, die de,, '.}{vraag gehoord hadden wachtten}{benieuwdt antwoord}\\

\haiku{{\textquoteright} De bediende bood,.}{gekookte tongen Nelly's}{delicatesse}\\

\haiku{En daarna reden.}{met traag en droog grintgeknerp}{de andere voor}\\

\haiku{Vader Hans geeuwde.}{en Charlotte was al naar}{haar kamer gegaan}\\

\haiku{Zij-zelf was toen,,.}{ondanks haar vermoeidheid ook}{wakker gebleven}\\

\haiku{Ze dorst 't maar aan,, '.}{de oude dame en dat}{in zoon hitte}\\

\haiku{{\textquoteright} 't Nonnaatje zweeg, met,.}{een schouder-schok en er}{was even een pauze}\\

\haiku{{\textquoteright} {\textquoteleft}Ja, kind, vraag liever, -.}{wie er n{\`\i}et waren van de}{lui uit de buurt dan}\\

\haiku{En dan Hemming en,....}{Rutgers en De Leur en de}{dokter en Van Twist}\\

\haiku{En Wiesje kwam, van,}{den badkamerkant met een}{te midden van al}\\

\haiku{Amelie moest nou niet '.}{doen oft heelemaal uit}{de lucht kwam vallen}\\

\haiku{Hij wandelde door,.}{hooge zalen met portretten}{tegen de muren}\\

\haiku{Ze nam hem bij de,.}{hand en ze wandelden langs}{het water samen}\\

\haiku{toen scheurde hij 't,.}{vodje in kleine stukjes}{en strooide die rond}\\

\haiku{Goenong-Djatti ().}{Een Indische roman door}{Carry van Bruggen}\\

\haiku{En mevrouw Baarslag, van {\textquotedblleft}{\textquotedblright}....}{Kalipoeti heeft me al}{zoo dikwijls gevraagd}\\

\haiku{De Klerk had zich, als,.}{de anderen neergelegd}{om wat te slapen}\\

\haiku{ze was bij mevrouw.}{Van Houweningen maar heel}{koeltjes ontvangen}\\

\haiku{Hij sprak er nog niet,.}{over er was een andere}{oplossing denkbaar}\\

\haiku{Die hem 't geschikst,.}{leken noteerde hij even}{in z'n zakboekje}\\

\haiku{Meer had ze dan ook,}{niet te vorderen ze had}{vooruit geweten}\\

\haiku{Kolff tuurde weer neer,.}{naar z'n sigaar meer broze}{asch dan tabak nu}\\

\haiku{Wies was van morgen,.}{vroeg al weg uit spelen bij}{de meisjes Schaarbeek}\\

\haiku{Kijk nou 's naar de {\textquoteleft}{\textquoteright}, '.... '}{rubriekStadnieuws d\`at is dan}{tocht voornaamste}\\

\haiku{De krees waren nog,.}{niet neer maar heel gauw zou dat}{toch wel noodig worden}\\

\haiku{{\textquoteright} {\textquoteleft}H\`e nee, Amelie{\textquoteright} zei, {\textquoteleft}.}{Nel even booszeg niet z\`ulke}{verdachtmakingen}\\

\haiku{Al h\'e\'el vroeg gegaan, '.}{om voort ontbijt terug}{te kunnen wezen}\\

\haiku{{\textquoteright} {\textquoteleft}Ja, ik kan er niet,,.}{mee dwepen met dat erge}{blonde dat witte}\\

\haiku{en een aardig paar, '.}{zij een snoezig vrouwtje en}{hijn knappe man}\\

\haiku{{\textquoteright} {\textquoteleft}Dat kun je denken{\textquoteright},, {\textquoteleft}.}{spotte Ameliena zoo een}{hopelooze liefde}\\

\haiku{hij laat zich nogal, '.}{w\`el leiden maar je moetm}{zoo alles zeggen}\\

\haiku{Ik geloof, dat zoo.}{een man een ellendigen}{werkkring moet hebben}\\

\haiku{Nu ga ik heusch. '.}{t Zal buiten toch wel al}{om te stikken zijn}\\

\haiku{al naar dat we een '.}{geschikte hut vinden op}{n prettige boot}\\

\haiku{Die Charlotte van.... {\textquotedblleft}{\textquotedblright}?}{der Hoef blijft ze nog lang op}{Goenong-Djatti}\\

\haiku{En dan verbaasde.}{ze zich ook ineenen over haar}{eigen aarzeling}\\

\haiku{{\textquoteright} {\textquoteleft}O{\textquoteright} zei die, koeltjes, {\textquoteleft}.}{om mij hoef je niets te doen}{en niets te laten}\\

\haiku{Goenong-Djatti ().}{Een Indische roman door}{Carry van Bruggen}\\

\haiku{{\textquoteright} Charlotte bloosde.}{en verwarde zich in een}{ontwijkend antwoord}\\

\haiku{Ja, je weet, je voelt,{\textquoteright}}{wel dat ik me erg tot je}{aangetrokken voel}\\

\haiku{Maar van Singapore... weet....}{je wat d\`a\`ar m'n amusantste}{indruk geweest is}\\

\haiku{{\textquoteright} zei hij, {\textquoteleft}heerej\'e,,.}{wat een zieken ineenen wat}{een ontsteltenis}\\

\haiku{'k Wou alleen even ',.}{naart hospitaal de lui}{weten er van niets}\\

\haiku{Ik was vroeg-op.}{vanmorgen en alles komt}{nu ook tegelijk}\\

\haiku{Ze had zoo tekeer,.}{gegaan toen de oppassers}{d'r kwamen halen}\\

\haiku{een aangename,}{afwisseling als je zelf}{leefde van geluk}\\

\haiku{iedereen was nu}{even zenuwachtig en}{gespitst op gerucht}\\

\haiku{ze fluisterden nu...}{in de bijgebouwen over}{wat er was geschied}\\

\haiku{Dan is-ie ook wel,.}{uitgeput en kun je zien}{ho\`e die erin zit}\\

\haiku{en hij begreep, dat,.}{er iets erger moest zijn iets}{re\"eelers vooral}\\

\subsection{Uit: Heleen: een vroege winter}

\haiku{Van het huis-zelf.}{en de oude meubels hield}{ze bijkans evenzeer}\\

\haiku{Bijkans elken dag.}{stond Heleen daar en zag er}{dezelfde boeken}\\

\haiku{zoolang zij maar in,,.}{huis was konden ze loeren}{zooveel ze wilden}\\

\haiku{HELEEN werd twaalf jaar}{oud en kende van het woord}{vertrouwelijkheid}\\

\haiku{Heleen wist het niet,.}{draaide ongedurig in}{haar bank en zuchtte}\\

\haiku{Ze wist het niet, doch.}{immer dringender klopte}{de vraag bij haar aan}\\

\haiku{Ze hadden het lang,.}{geweten doch gingen het}{nu pas beseffen}\\

\haiku{Maar waarom dan.... en..........}{waaruit dan de \'e\'en zoo en}{de ander anders}\\

\haiku{Een enkele plek.}{in haar was door overgroei tot}{ouderdom gerijpt}\\

\haiku{Want waarom zou ze?}{van nature beter dan}{anderen wezen}\\

\haiku{een eeuwige wet.}{in de eindeloosheid der}{openbaringswijzen}\\

\haiku{- haar innerlijke,}{zorgen namen haar geheel}{en al in beslag}\\

\haiku{Wat ze toen zag, was,.}{de cirkel waarbinnen ze}{voortaan dolen zou}\\

\haiku{dien anderen dag.}{was tot den avond toe alreeds}{bezet en bestemd}\\

\haiku{Haar gelatenheid,.}{leerde haar dat ze dit in}{waarheid geloofde}\\

\haiku{Damp had zich onder,.}{de binten vergaard stil en}{zwaar als witte rook}\\

\haiku{Heleen bedwong haar.}{verwarring en dorst er niet}{verder op doorgaan}\\

\haiku{Wat was ze, wat had,?}{ze wat beteekende ze meer}{dan volslagen niets}\\

\haiku{Ze sloot half de oogen.}{en hief in den lichtschijn het}{gezicht naar hem op}\\

\haiku{Bij het ontwaken,.}{bevond ze dat ze bang was}{en trilde van leed}\\

\haiku{Heleen overdacht wat.}{haar vriend haar van zijn eigen}{leven had verteld}\\

\haiku{Nooit te voren had.}{Heleen zoo schrikkelijk haar}{eenzaamheid beseft}\\

\haiku{O, zij en haar vriend,.}{hoe verschilden ze nog naar}{hun groei en maaksel}\\

\haiku{Toch rondde hij zijn.}{vingers onder haar kin en}{kuste haar vluchtig}\\

\haiku{Verlangen had haar,,;}{zoolang zij het meester bleef}{op sluwheid gespitst}\\

\haiku{{\textquoteright} Heleen las dien brief.}{en liet hem daarna uit}{haar handen vallen}\\

\subsection{Uit: Het huisje aan de sloot}

\haiku{{\textquoteright} Glad-zwart is de,....}{vloer van het paleis roerloos}{staan de pilaren}\\

\haiku{hij zijn schuit naar den,....}{overwal doet er hetzelfde}{en keert weer terug}\\

\haiku{In breede, malsche.}{plooien wijkt aan weerszijden}{het water terug}\\

\haiku{half-feesten!}{zijn bijna nog prettiger}{dan heele feesten}\\

\haiku{Zeker, juist omdat,.}{het zoo prachtig brandde moest}{het het eerste uit}\\

\haiku{ze wil den draad nu,....}{haastig binnenpalmen maar}{het hoeft al niet meer}\\

\haiku{Ze weet wel beter,.}{ze heeft opzettelijk de}{groote domheid gezegd}\\

\haiku{De kachel moet wel,}{geweldig branden wamt in}{de smalte tusschen}\\

\haiku{Dat kan vader niet,.}{zijn vader rookt geen pijp en}{vader schatert niet}\\

\haiku{{\textquoteright} Oom Zeelik houdt even,,....}{op zijn oogen zijn klein hun oogen}{zijn groot ze denken}\\

\haiku{want het leven is.}{nu \'e\'en en al trillend en}{tintelend geluk}\\

\haiku{En zijn voeten staan!}{precies zoo scheef naar buiten}{gedraaid als anders}\\

\haiku{- en ze maakte de.}{kistjes open en keek alles}{na wat er in zat}\\

\haiku{Mijmerend kijkt ze....}{uit de hoogste ruiten naar}{de witte wolken}\\

\haiku{nu is 't maar weer,.}{voorgoed achter den rug de}{lente komt eraan}\\

\haiku{Plat tegen den grond,,.}{bijna nog zonder steel zooals}{altijd die eersten}\\

\haiku{Maar dat klonk nu net.}{alsof ze heel iets anders}{had willen zeggen}\\

\haiku{Van zijn tiende jaar.}{af is hij wees en woont bij}{zijn grootmoeder in}\\

\haiku{Hij leende geld aan -!}{den Keizer van Rusland want}{z\'o\'o rijk was hij wel}\\

\haiku{Ze zouden zoo graag,.}{willen dat ze David nog}{eens tegenkwamen}\\

\haiku{Eindelijk is dan....}{toch de klare waarheid wijd}{voor ze opengegaan}\\

\haiku{of lag hij dieper?}{in het water dan hij in}{de aarde zal zijn}\\

\haiku{maar het woord doet met,....}{haar als de reuk het verschrikt}{haar en het lokt haar}\\

\haiku{Hij bewijst Zijne.}{Gerechtigheid aan hen die}{slapen in het stof}\\

\haiku{Niets vriendelijk kijkt...}{hij nu meer wat maakt haar dat}{allemaal benauwd}\\

\haiku{de twee pilaartjes,..}{in volle werking grappig}{om naar te kijken}\\

\haiku{Zoo klein is ze toch,,.}{niet dat hij haar niet meer zag}{vlak onder de lamp}\\

\haiku{in den laten avond,.}{zoodat ze van slaap niet staan en}{niet kijken konden}\\

\haiku{Snoek - dat is net als,!}{bij hen vader en moeder}{schelen ook tien jaar}\\

\haiku{Het is dol aardig,!}{als de groote menschen overhoop}{liggen met elkaar}\\

\haiku{Ze moeten ineens..,.}{allebei lachen ja maar}{om h\'e\'el wat anders}\\

\haiku{{\textquoteleft}Ja maar, vader, als..!}{het toch regent en met ons}{mooie Sjabbesgoed aan}\\

\haiku{Ze staan allemaal.}{om mijnheer Hamel heen en}{kijken naar hem op}\\

\haiku{en dan komt het nog,!}{uit dat ze zoo'n heele plas}{hebben gemorst}\\

\haiku{Het eerste uur is,.}{voorbij de meester deelt nu}{de leesboekjes rond}\\

\haiku{Nu ze dat bedenkt,,.}{kan ze niet meer huilen kan}{ze niet meer boos zijn}\\

\haiku{Ja, ze doen het, ze,.}{doen het hun stoelen schoven}{ze al achteruit}\\

\haiku{En er blonk iets dat,.}{zon ving h\'e\'el fel en dat iets}{was aan den Keizer}\\

\haiku{Helpt het, als je hooi?}{heenspreidt over je voeten om}{ze af te koelen}\\

\haiku{.. is haar broertje dan?}{een arme jongen en zijn}{zij arme-lui}\\

\haiku{Aan hun dokter stuurt.}{Vader het geld in een brief}{als het jaar om is}\\

\haiku{Ze kreeg bijna de '..}{deur int gezicht zoo vlak}{als ze er achter}\\

\haiku{helpt ze moeder in,.}{de kamer dan hoor je haar}{zingen vlak-bij}\\

\haiku{Het huisje aan de.}{sloot    Aantekeningen}{1Joodsche gemeente}\\

\subsection{Uit: In de schaduw van kinderleven}

\haiku{{\textquoteleft}'k Zal Bram en Wietje,...}{wel meenemen dan ben-u}{meteen van ze af}\\

\haiku{{\textquoteright} {\textquoteleft}Nou, gane jullie,{\textquoteright}, '.}{nou maar ongeduldigde}{Geertr thee slurpend}\\

\haiku{Ze waren nu in, '.}{wijde woelige groep om}{m heen komen staan}\\

\haiku{Dromerig klonk hun.}{gemurmel door de kille}{stilte van de gang}\\

\haiku{{\textquoteleft}Geef alles maar hier,{\textquoteright},.}{zei de moeder wachtend met}{uitgestrekte arm}\\

\haiku{'t Kind was immers.}{niet verkouden en ze had}{ook de tering niet}\\

\haiku{{\textquoteright} {\textquoteleft}Nee, vader,{\textquoteright} zeien,,... {\textquoteleft}'.}{de kinderen verlegen}{t is anstons tijd}\\

\haiku{'t Was toch wel n\`et,, '...}{of meester Bom naar hem keek}{dachtt jongetje}\\

\haiku{Want de hoofdingang.}{v\'o\'or was voor de meesters en}{juffrouwen alleen}\\

\haiku{dat w\`as geen klikken....}{vond-ie en vechten mocht-ie}{niet voor z'n vader}\\

\haiku{Maar dat duurde kort,:}{en plichtmatig ving hij aan}{z'n stem schor-hokkend}\\

\haiku{{\textquoteleft}'t Is waarachtig ' '...}{ofkn jongetje van}{de bewaarschool ben}\\

\haiku{II Op 't ijs, de,.}{winter tevoren hadden}{ze mekaar ontmoet}\\

\haiku{{\textquoteleft}Wel nee,{\textquoteright} weerlegde ',... {\textquoteleft}.}{t kind ernstighelemaal}{niet en Th\'e ook niet}\\

\haiku{Hij studeert maar wat... ' '...}{hardk Hebm al in geen}{drie dagen gezien}\\

\haiku{en nu meenden ze.}{allebei niet buiten de}{ander te kunnen}\\

\haiku{Maar z'n brood vonden.}{ze lekker en allemaal}{kochten ze bij hem}\\

\haiku{Hij moest blij zijn, dat,.}{ze h\`em met vrede lieten}{hem begunstigden}\\

\haiku{De vrouw vertoonde,.}{zich nooit de man heel zelden}{meer buiten de deur}\\

\haiku{peren en noten...{\textquoteright}...}{En schalks-nieuwsgierig}{tegen z'n moeder}\\

\haiku{Hij verstond niet w\`at,.}{ze zei hoorde alleen de}{klank van de woorden}\\

\haiku{{\textquoteright}... Duidelijk door de...}{wind hoorden we zacht kloppen}{en zacht roepen}\\

\haiku{Daar ginds was een dorp... '......}{uitgemoord omk weet niet}{wat om niks misschien}\\

\haiku{Maar woede vlamde,:}{in Brams ogen en de handen}{saamknijpend riep hij}\\

\haiku{Veel hebben we hier,.}{niet ergens anders hebben}{we j\`a niemendal}\\

\haiku{Waar was hij no\`u weer......}{heengelopen misschien een}{eind met de Rus mee}\\

\subsection{Uit: Een Indisch huwelijk}

\haiku{Wist hij niet -, alleen {\textquoteleft}{\textquoteright}, {\textquoteleft}{\textquoteright}.}{dat hijiets wilde omdat}{zij nuniets wilde}\\

\haiku{Het effen, oude -}{stemmetje meldde hem dat}{de njonja ziek was}\\

\haiku{Ja, er was in dit.}{alles toch wel iets om een}{vrouw te bekoren}\\

\haiku{Hoog stond de maan, den.}{langen weg terug zou hij}{alleen moeten gaan}\\

\haiku{En samen gingen.}{ze de roodgelooperde}{bordestreden af}\\

\haiku{Mocht het ook verdriet,?}{heeten wat hij gevoeld had}{na den dood van Ruysch}\\

\haiku{t was al om 't {\textquoteleft}{\textquoteright}.}{even hol en dom geweest als}{zijn laatsteavontuur}\\

\haiku{{\textquoteleft}Ga je met ons mee,{\textquoteright},, {\textquoteleft}}{d\'ejeuneeren Feenstra vroeg ineens}{echter de dokter}\\

\subsection{Uit: In de schaduw (van kinderleven)}

\haiku{{\textquoteleft}'k Zal Bram en Wietje,....}{wel meenemen dan ben-u}{meteen van ze af}\\

\haiku{N\`et kom-je uit  ,,....}{et gewone mot je daar}{weer heen na dat hok}\\

\haiku{{\textquoteright} {\textquoteleft}Nou, gane jullie,{\textquoteright}, '.}{nou maar ongeduldigde}{Geertr thee slurpend}\\

\haiku{Ze waren nu in, '.}{wijden woeligen groep om}{m heen komen staan}\\

\haiku{En hij woonde 'n.... '....}{heel eind ver had nou nieteensn}{boterham gehad}\\

\haiku{Droomerig klonk hun.}{gemurmel door de kille}{stilte van de gang}\\

\haiku{{\textquoteleft}Geef alles maar hier,{\textquoteright},.}{zei de moeder wachtend met}{uitgestrekten arm}\\

\haiku{Ze streek 'r witte, '....}{schort glad veegde de vochte}{handen aann doek}\\

\haiku{Z\`onder appele{\textquoteright}.... {\textquoteleft},....}{zei de moederweet-je}{wat Schooner dervoor vroeg}\\

\haiku{'t Kind was immers.}{niet verkouden en ze had}{ook de tering niet}\\

\haiku{{\textquoteright} {\textquoteleft}Nee, vader{\textquoteright} zeien,,.... {\textquoteleft}'.}{de kinderen verlegen}{t is anstons tijd}\\

\haiku{'t Was toch wel n\`et,, '....}{of meester Bom naar hem keek}{dachtt jongetje}\\

\haiku{Want de hoofdingang.}{v\'o\'or was voor de meesters en}{juffrouwen alleen}\\

\haiku{dat w\`as geen klikken.....}{vond-ie en vechten mocht-ie}{niet voor z'n vader}\\

\haiku{dan zou er  wel '....}{stellig iets heel vreeselijks}{metm gebeuren}\\

\haiku{Maar dat duurde kort,:}{en plichtmatig ving hij aan}{z'n stem schor-hokkend}\\

\haiku{'t Is waarachtig ' '.....}{ofkn jongetje van}{de bewaarschool ben}\\

\haiku{{\textquoteright} {\textquoteleft}H\`e ja{\textquoteright} zei Juul, en....}{ze stak haar arm door die van}{de oude vrouw}\\

\haiku{en nu meenden ze.}{allebei niet buiten den}{ander te kunnen}\\

\haiku{Jezes, die was in '....}{de Paaschvacantie ook al aan}{t zasen gegaan}\\

\haiku{{\textquotedblleft}As-je nou eerst eris ',{\textquotedblright}, {\textquotedblleft}}{n lamp ansteekt lijsde}{Gerritdan kanne}\\

\haiku{In hevigen angst, '....}{voor ongeluk gildet}{ouwe menschje}\\

\haiku{slepend gleden ze,.}{naar beneden op de snee}{vochtig en frisch wit}\\

\haiku{Maar z'n brood vonden.}{ze lekker en allemaal}{kochten ze bij hem}\\

\haiku{Hij moest blij zijn, dat,.}{ze h\`em met vrede lieten}{hem begunstigden}\\

\haiku{De vrouw vertoonde,.}{zich nooit de man heel zelden}{meer buiten de deur}\\

\haiku{Hij verstond niet w\`at,.}{ze zei hoorde alleen den}{klank van de woorden}\\

\haiku{'t Was een Duitsche,....}{meid meegekomen met die}{menschen naar Rusland}\\

\haiku{Veel hebben we hier,.}{niet ergens anders hebben}{we j\`a niemendal}\\

\haiku{Moesten ze terug naar, '....}{Rotterdam dan werdt weer}{krimpen in een krot}\\

\subsection{Uit: Het Joodje}

\haiku{en die zet me apart.}{op een bank en geen een wil}{er met me spelen}\\

\haiku{Ben dorst niet opzien,.}{maar de bom sprong heel anders}{dan hij had verwacht}\\

\haiku{{\textquoteright} riep Ben, toen hij den,.}{brief gelezen had rood van}{verontwaardiging}\\

\haiku{Zoo vertelde ze,.}{oom-en-tante terwijl ze}{voor den spiegel stond}\\

\haiku{En meer en meer steeg.}{zijn angst dat iemand tot hem}{het woord zou richten}\\

\haiku{{\textquoteright} zei ze, aarzelend,.}{half-verlegen en}{half-brutaal}\\

\haiku{zoo maar voetstoots op,.}{te geven had de oude}{Nachbar toegestemd}\\

\haiku{{\textquoteright} {\textquoteleft}Natuurlijk, maar zou,?}{het dan wel geschikt zijn dat}{we vanavond komen}\\

\haiku{, ik wil zeggen, dat?}{we het waardeeren zouden als}{je in de club kwam}\\

\haiku{de zaak is immers,,.}{van de baan Ben wil niet in}{de club daarmee uit}\\

\haiku{{\textquoteright} {\textquoteleft}Ze komt niet, omdat.}{ik haar heb gezegd dat ze}{wel weg kon blijven}\\

\haiku{hij had alles van,.}{me kunnen krijgen wat z'n}{hart maar had begeerd}\\

\haiku{Het viel hem dus zwaar,,,.}{genoeg maar alles moest om}{harentwil beproefd}\\

\haiku{En hij overlegde,,,}{dat hij zijn werk wat boeken}{die hij kon veinzen}\\

\subsection{Uit: Maneschijn met koek en Al om een suiker balletje}

\haiku{Ze staan allemaal.}{om mijnheer Hamel heen en}{kijken naar hem op}\\

\haiku{en dan komt het nog,!}{uit dat ze zoo'n heele plas}{hebben gemorst}\\

\subsection{Uit: 'n Badreisje in de tropen}

\haiku{Ik meende, dat je....{\textquoteright} {\textquoteleft},{\textquoteright},.}{Heelemaal niet viel Gerda}{in opgewonden}\\

\haiku{'t Dek was geruimd.}{en dat gaf haar dadelijk}{een prettig gevoel}\\

\haiku{was dan ook meteen.}{klaar wakker en met een sprong}{uit haar stoel overeind}\\

\haiku{Nu reden ze de.}{heele stad door tot aan den}{voet van den heuvel}\\

\haiku{Stil en koel stond aan.}{weerszijden van den weg het}{roerlooze schaduwbosch}\\

\haiku{O.... wat was mevrouw,....}{toch begonnen heen te gaan}{en haar te brengen}\\

\haiku{Gerda begreep, dat....,,?}{ze licht zou moeten vragen}{maar hoe maar aan wie}\\

\haiku{Zij wist geen weg in, {\textquoteleft}{\textquoteright}.}{de wereld maar hij wist geen}{weg in degoedang}\\

\haiku{Die Gerda, net een,.}{kind dat moest en dat zou nou}{van huis en op reis}\\

\haiku{als Sarian, z'n,, '.}{kleine jongen sterven ging}{zout haar schuld zijn}\\

\haiku{Nee, de cholera,;}{zat nergens die kwam rechtstreeks}{van toean-Allah}\\

\haiku{En daarna, frisch in,.}{haar versche koele kleeren kwam}{ze weer naar buiten}\\

\haiku{Daar zag ze Riboe,, '.}{den krani van haar man al}{t erf opkomen}\\

\haiku{Ze liet even staan de.}{beide mannen en liep de}{buurvrouw tegemoet}\\

\haiku{Achter uit den tuin,,;}{kwam nu ook Soemon de}{koetsier aanloopen}\\

\haiku{Nj\`a{\textquoteright}, antwoordde  ,.}{zacht de jongen als teeken}{dat hij had verstaan}\\

\haiku{Sa{\"\i}na keek hem,.}{na haar kindergezichtje}{boos en d\'edaigneus}\\

\haiku{'t heden was als '....}{gisteren en morgen zou}{t als heden zijn}\\

\haiku{Daar stonden ze stil,.}{in de klare maan statig}{en onwezenlijk}\\

\haiku{Maar wat zei mevrouw,....}{van zulk een vrouw van zulk een}{ondankbaar wezen}\\

\haiku{Uit z'n borst klom de,}{donkere toorn-grom}{om den honger dien}\\

\haiku{Hij was 'n groot en, '.}{sterk beest met forsche armen}{enn stoeren kop}\\

\haiku{Oeri, die niet op de....}{klok kon kijken en bang was}{voor de telefoon}\\

\haiku{iedereen wist toch....}{dat hij geen enkele vrouw}{met rust kon laten}\\

\subsection{Uit: Om de kinderen}

\haiku{Ja, Jeantje was,.}{altijd een gemakkelijk}{volgzaam kind geweest}\\

\haiku{Hij wist het zelf wel,.}{en het hinderde hem al}{liet hij niets merken}\\

\haiku{Een enkele maal,.}{wij sympathiseerden nooit}{zoo heel bijzonder}\\

\haiku{Dat uitkijken naar,,.}{de post die slapeloosheid}{die zenuwbuien}\\

\haiku{Dien keer, dat we in....}{het begin dien Beyerman}{hier te eten hadden}\\

\haiku{Altijd zal er iets....}{in mij  voor hem pleiten}{en hem vrijspreken}\\

\haiku{Of er ergens daar -.}{een hondje blafte zoo keek}{hij even mijn kant op}\\

\haiku{En toch.... ik kan nog,.}{altijd niet gelooven dat ik}{hem zoo miskende}\\

\haiku{Wil je nu heusch,?}{niet blijven terwijl mama}{Van der Wal er is}\\

\haiku{{\textquoteleft}Maar als ik nu wat,.}{aan hem vroeg zou het van zijn}{moeder afmoeten}\\

\haiku{En dan zou ze nu,,?}{op haar ouden dag zich nog}{bekrimpen moeten}\\

\haiku{Emilie zal je de, -}{boeken terugsturen die}{ze nog van je heeft}\\

\haiku{{\textquoteright} Heen en weer stampend,:}{in de kamer begon hij}{luidkeels te zingen}\\

\haiku{{\textquoteright} Elk met een dik pak.}{chocolade in de hand}{stonden ze v\'o\'or haar}\\

\haiku{Den laatsten keer kon....}{je het in November nog}{zien aan mijn gezicht}\\

\haiku{introduc\'ees van.}{de gasten misschien en de}{zoon van den gastheer}\\

\haiku{Ja, nu vloog de drift -.}{weer in haar op en daarom}{wilde ze slapen}\\

\haiku{dat was de blijde,.}{zekerheid geweest waaraan}{ze zich staande hield}\\

\haiku{{\textquoteleft}Emilie, die in haar.}{eentje zit te snoepen van}{een verboden boek}\\

\haiku{Het was nu toch wel;}{volslagen pikdonker om}{haar heen geworden}\\

\haiku{De aanmatiging,.}{de verwaandheid van dat kind}{werd wel grenzeloos}\\

\haiku{{\textquoteleft}Fijnbesnaard{\textquoteright} - ja, 't.}{was aardig uitgedrukt van}{dominee Lette}\\

\haiku{Ver van hem af, zou.}{ze zijn heugenis uit haar}{leven verdrijven}\\

\haiku{wat zou hij, wat zou,....}{Ard Hettema zeggen als}{hij dat van haar wist}\\

\haiku{hij, dien ze zich niet?}{anders dan vlekkeloos kon}{en wilde denken}\\

\haiku{Ce portrait, tout beau,....}{que ce soit ne vaut pas un}{baiser du mod\`ele}\\

\haiku{{\textquoteright} Het boekerige,.}{woord misklonk maar Margo vond}{zoo gauw geen ander}\\

\haiku{Anders was ze toen,.}{juist blijven komen toen ze}{het merkte van Ard}\\

\haiku{Het staat allemaal,....}{zoo ver van mij af als een}{andere wereld}\\

\haiku{En hij zou dan toch.}{wel eenigen invloed hebben}{op moeder en zoon}\\

\haiku{Je vroeg je af hoe.}{die dingen dadelijk zoo}{werden overgebracht}\\

\haiku{{\textquoteright} Ditmaal wachtte hij,,.}{weer wel antwoord dicht bij de}{tafel zacht hijgend}\\

\haiku{Plas merkte ditmaal.}{nog minder dan anders hun}{binnenkomen op}\\

\haiku{{\textquoteright} Zoo onnatuurlijk,.}{kalm was nu weer zijn stem dat}{Plas er van opkeek}\\

\haiku{{\textquoteright} {\textquoteleft}Niets, ze wilde het.}{hem niet zeggen en ik mocht}{het hem niet zeggen}\\

\haiku{Ik ben de vriend van,....}{haar vader en ik beschouw}{mijzelf zoolang hij}\\

\haiku{het is volgens uw,.}{eigen beginselen dat}{ze trouwen moeten}\\

\haiku{Vanavond nog zou hij.}{kalm en vaderlijk met Frans}{over alles spreken}\\

\haiku{En toch, hij kon zich.}{geen blijvend geluk denken}{uit hun huwelijk}\\

\haiku{neen, het was toch wel,.}{altijd een verrukking zoo}{je macht te voelen}\\

\haiku{En nog zocht Frans de.}{schuld van zichzelven af op}{haar te wentelen}\\

\haiku{Ik zei, dat ik bij.}{hem de resultaten niet}{erg schitterend vond}\\

\haiku{{\textquoteright} {\textquoteleft}Maar hoe kom je er,?}{nu zoo plotseling bij nu}{we bijna thuis zijn}\\

\haiku{Henriet had ze reeds,.}{gewonnen Margo zou ze}{gewonnen hebben}\\

\haiku{Dokter Van 't Hoff.}{en ik hopen over een paar}{maanden te trouwen}\\

\haiku{Het was als werd haar,}{Leidsche leven door de vraag}{voor haar oogen gebeurd}\\

\haiku{{\textquoteright} {\textquoteleft}'t Lijkt allemaal,.}{al zoo lang geleden er}{is zooveel gebeurd}\\

\haiku{En Margo vooral,!}{was altijd zoo sceptisch na}{haar ervaringen}\\

\haiku{Toen die blik van het -,, -?}{kind spot minachting wrevel}{wat was het geweest}\\

\haiku{{\textquoteleft}Ik kan het, wat je,.}{divorce betreft volstrekt}{niet met je eens zijn}\\

\haiku{Ze zal nog wel eens....}{vaker en nog wel eens veel}{erger bedroefd zijn}\\

\haiku{Het is zoo goed dat....}{we het maar vooraf weten}{en op ons nemen}\\

\haiku{Hij schijnt haar zelf les.}{in allerlei dingen te}{hebben gegeven}\\

\haiku{Een man had zoo zijn.... {\textquoteleft}}{ijdelheidjes en De Wit}{was burgemeester}\\

\haiku{{\textquoteright} gooide Maddy's,.}{vader er uit een beetje}{grof-schertsend}\\

\haiku{Waarom mocht hij nu,,?}{niet dien eenen dag hier bij haar}{zijn en haar troosten}\\

\haiku{een besluit nemen,?}{en namen ook anderen}{zoo hun besluiten}\\

\haiku{Tot den toren ga '.}{ik mee en dant fietspad}{over de beek terug}\\

\haiku{Maar zelfcritiek zou.}{toch Jettie's redding wezen}{op den langen duur}\\

\haiku{Het eerste wat die.}{man denkt is natuurlijk dat}{hij duelleeren moet}\\

\haiku{Net als bij dien man,....?}{in het boek maar gelukkig}{niet in die mate}\\

\haiku{Hun blikken schampten,....}{langs elkaar heen iets dat zich}{schoof tusschen hun oogen}\\

\haiku{Alleen Oma praatte, -?}{onbevangen zag niets of}{hield ze zich maar zoo}\\

\haiku{Stijve Hollandsche....!}{schoonmama en nuchtere}{Hollandsche jongen}\\

\haiku{Hij was niet jong, moest,}{de vijftiger jaren al}{beklommen hebben}\\

\haiku{{\textquoteright} {\textquoteleft}Clich\'e,{\textquoteright} smaalde Frans,, {\textquoteleft}.}{schouderschokkendgeen aasje}{oorspronkelijkheid}\\

\haiku{Het was een tijd vol -....{\textquoteright}}{emoties en niet allemaal}{prettige emoties}\\

\haiku{Haar eigen dochter.}{is volwassen en nog vond}{ze het ongepast}\\

\haiku{\'e\'en woord en hij had,,.}{haar wankelend een tweede}{en overgegeven}\\

\haiku{Opnieuw begon, in,.}{wilden doelloozen ijver een}{eend te snateren}\\

\subsection{Uit: Prometheus}

\haiku{{\textquoteleft}We kunnen niet zien{\textquoteright} -,:.}{zegt men maar ook we kunnen}{niet onderscheiden}\\

\haiku{Zien we geen contrast,.}{dan onderscheiden we niet}{dan zien we dus niets}\\

\haiku{Denken we aan de {\textquoteleft}{\textquoteright}:}{curieuse vraag in de}{Tweede Hippias}\\

\haiku{Aan dien waan houdt hij,.}{zich staande deze is de}{steun van zijn zwakheid}\\

\haiku{Huiselijk gezegd,.}{men kan geen twee ruggen uit}{\'e\'en varken snijden}\\

\haiku{Rust toch is alleen.}{bij verblinding en bij de}{hoogste helderheid}\\

\haiku{onbegrensd,  naar,;}{alle kanten vervloeiend}{zelf-opheffend}\\

\haiku{Het eindpunt zal dus.}{steeds  tegelijkertijd}{weer uitgangspunt zijn}\\

\haiku{hij anderen niet.}{den weg wijst naar bronnen van}{eigen onderzoek}\\

\haiku{Gij hebt gehoord, dat - -{\textquoteright}}{tot de Ouden gezegd is}{maar ik zeg U enz.}\\

\haiku{de gedachte dat {\textquoteleft}{\textquoteright},.}{hijafbrekend werk doet zou}{hem ondraaglijk zijn}\\

\haiku{de (zinnelijke).}{liefde in de eerste plaats}{en daarmee de vrouw}\\

\haiku{Altijd looft de mensch.}{instinctief datgene wat}{zichzelf gelijk blijft}\\

\haiku{Ook die {\textquoteleft}men{\textquoteright} zijn we, {\textquoteleft}{\textquoteright}.}{allen zelfs eencontented}{pessimist als Shaw}\\

\haiku{Deze voorkeur heeft,.}{ook nog een anderen een}{positiever kant}\\

\haiku{Hij hield zich aan de,!}{Schrift maar wilde vooral niet}{redeloos heeten}\\

\haiku{Wie was armer dan,?}{de schrijver van Gil Blas Alain}{Ren\'e le Sage}\\

\haiku{Want het wezen der.}{dingen ontgaat hem door zijn}{egocentrischen aard}\\

\haiku{{\textquoteleft}le grand Cond\'e{\textquoteright}!) aan het -.}{wankelen bracht er is een}{innerlijk verschil}\\

\haiku{Gods recht op alle.}{dingen berust op zijn macht}{over alle dingen}\\

\haiku{, {\textquoteleft}zullen allicht meer,}{twisten rijzen dan tusschen}{meesters en slaven}\\

\haiku{Na Lodewijk XIV.}{is er in diens stijl geen groot}{heerscher meer geweest}\\

\haiku{ist das Wort todt, so.}{k\"onnen es keine Riesen}{aufrecht erhalten}\\

\haiku{Het gaat hem op het:}{allerbest zooals het Faust met}{den Aardgeest ging}\\

\haiku{in de Eenheid is.}{alle zelfherkenning reeds}{zelfvernietiging}\\

\haiku{Ook Lessing heeft de.}{officieele geleerdheid}{niet malsch behandeld}\\

\haiku{Maar een nieuw licht schijnt,.}{nu in en uit de lampen}{die menschen heeten}\\

\haiku{hij ziet {\textquoteleft}liefde{\textquoteright} als;}{een daarvan en zelfs niet steeds}{de belangrijkste}\\

\haiku{Zoo was het reeds, in,.}{beginsel gedurende}{de Renaissance}\\

\haiku{een zwakke en een,.}{sterke een falende en}{een overwinnende}\\

\haiku{Zoo spiegelt alle:}{litteratuur van den tijd}{hetzelfde streven}\\

\haiku{{\textquoteleft}Ist sein Hand wider,.}{jederman wird jedermans}{Hand sein wider ihn}\\

\haiku{Dit levert hier een,.}{intellectueel verschil}{geen zedelijk op}\\

\haiku{een noodzakelijk.}{kwaad tot het schoone oogmerk}{der Verzoening is}\\

\haiku{{\textquoteright} de Vicomte zich {\textquoteleft}.}{tot veelbeteekenend motto}{koos voor zijnTh\'eorie}\\

\haiku{ihn in Demut zu,,.}{beerben und viel zu schwach um}{ihm es gleich zu thun}\\

\haiku{Voor Lorenzo heeft.}{het leven voortaan geen zin}{en geen inhoud meer}\\

\haiku{La r\'ecompense,.}{est si grosse qu'elle les}{rend presque courageux}\\

\haiku{c'\'etait peut-\^etre.}{un p\`ere de famille}{qui mourait de faim}\\

\haiku{De redeneering.}{van alle dogmatici}{in alle tijden}\\

\haiku{Als dien na{\"\i}even -.}{schellinkjes-klant z\'o\'o}{ziet hij Prometheus}\\

\haiku{{\textquoteleft}Je ne leur ai pas,.}{donn\'e la pens\'ee car je}{suis un Dieu bon}\\

\haiku{elle pousse au{\textquoteright} -,:}{crime maar er volgt een raad}{een vertroosting op}\\

\subsection{Uit: Seideravond}

\haiku{over het bittere,,.}{over het offer en over het}{ongezuurde brood}\\

\haiku{n lolletje... 'n......}{relletje om de ouwe}{smous te treiteren}\\

\haiku{(rukt ineens de deur,)?}{open schreeuwend met schorre stem}{wat moeten jullie}\\

\haiku{jij deed het al toen,,.}{je tien jaar was niewaar in}{je grootvaders huis}\\

\haiku{Hier is het ei...  (,,).}{neemt het in de hand beschouwt}{het stem weemoedig}\\

\haiku{Je moet sterk zijn om,.}{daar tusschen te leven en}{toch vroom te blijven}\\

\haiku{Waarin is deze?}{nacht verschillend van alle}{andere nachten}\\

\subsection{Uit: Tirol}

\haiku{we hebben in een}{oudtirools Gasthof onze}{intrek genomen}\\

\haiku{er binnen te gaan.}{als bijvoorbeeld te leren}{shimmy-dansen}\\

\haiku{We zaten op de.}{bank onder een machtige}{beuk en we wachtten}\\

\haiku{keek hij rond of zich.}{ergens op zijn pad ook een}{jood dorst vertonen}\\

\haiku{Ten leste mocht het,,,...}{dan lukken maar vandaag och}{arme lukt het niet}\\

\haiku{Als hij nu, in zijn,.}{eigen belang maar niet al}{te eenvoudig is}\\

\haiku{Het is Maleis, en,.}{ik sprak het met sombere}{droeve stemklank uit}\\

\haiku{Thans loer ik op mijn.}{taalgevoelige vriend als}{een spin in het web}\\

\haiku{De muzikanten.}{zijn met hun hele boeltje}{naar buiten verhuisd}\\

\haiku{Ook wij zijn door de,.}{oude enge poort maar weer}{hierheen gekomen}\\

\haiku{Maar het zwijgende.}{jonge meisje is in geen}{geval zijn zuster}\\

\haiku{Plotseling zie ik...}{het als de bovenloop van}{een grote rivier}\\

\haiku{In de ene schaal het,...}{pasgenotene Fulpmes}{met zijn gletsjerschoon}\\

\subsection{Uit: Van een kind}

\haiku{Die konden zich nu,!}{heerlijk verbeelden dat zij}{in Parijs waren}\\

\haiku{De hele zomer.}{door bleven die vazen en}{manden daar staan}\\

\haiku{Soms had een knappe.}{tuinman er een jaartal of}{letters in gevormd}\\

\haiku{Wat had ze zo trots,?}{en zelfgenoegzaam wat had}{ze zo knap te zijn}\\

\haiku{En in afwachting.}{werden ze plotseling stil}{en keken haar aan}\\

\haiku{Verslagen lag die:}{jongenstrots en alles dat}{ermee samenging}\\

\haiku{die kwam om Bart nu,!}{het te laat was die kwam nu}{nog haar verdringen}\\

\haiku{ze voelde haar bloed,.}{bonzen door haar lijf maar ze}{hield zich parmantig}\\

\haiku{- dan kregen ze ook -}{koffie met beschuitjes en}{tegen Sinterklaas}\\

\haiku{Buiten zag ze hem.}{en Maup als Indianen}{dansen en grijnzen}\\

\haiku{Ook Leentje, vuurrood,,.}{naast de stoel dorst niet opzien}{zich niet verroeren}\\

\haiku{{\textquoteleft}Laat Abraham toch eerst.}{rustig gaan zitten en een}{kop koffie drinken}\\

\subsection{Uit: De vergelding (onder ps. Justine Abbing)}

\haiku{Even later liep ze}{dan door de winkelstraten}{en verbaasde zich}\\

\haiku{Moet alle zeilen.}{bijzetten om niet in haar}{klauwtjes te vallen}\\

\haiku{De bruuske toon van.}{Verkerks ondervraging bracht}{hem in verwarring}\\

\haiku{hij dacht anders den....}{laatsten tijd weinig meer aan}{Marietje de Geus}\\

\haiku{soms in het gevoel....}{dat hij wel nemen mocht wat}{hem gegeven werd}\\

\haiku{en dan, Jaap, dan ben....}{ik ineens zoo bang dat hij}{niet terugkeeren zal}\\

\haiku{En vind jij nu niet....,....}{in zoo'n vertoon van ja hoe}{zal ik het noemen}\\

\haiku{ik zeg het verkeerd,,....}{ik ben wezenlijk alles}{waarvoor hij mij houdt}\\

\haiku{Maar niet altijd kon,,....}{hij zich verdiepend in haar}{zoo rustig blijven}\\

\haiku{want welke man was?}{nu niet als jong student een}{beetje wild geweest}\\

\haiku{maar ze zou zich nooit.}{zoo dwaas en onvoorzichtig}{hebben gedragen}\\

\haiku{nu lag ze erin.}{en mocht ze met de heete}{kruik naar boven gaan}\\

\haiku{Alleen opnieuw in,....}{donker sloeg Hetty de oogen}{open en luisterde}\\

\haiku{ze hoorde nicht naar,.}{het jassenkamertje gaan}{waar het toestel hing}\\

\haiku{te weten of zij,!}{zeker van zichzelf was van}{haar gevoel voor hem}\\

\haiku{{\textquoteright} {\textquoteleft}Niet wat je hem hebt,....}{geschreven niet wat je nu}{verder denkt te doen}\\

\haiku{Laag over de hei woei,,....}{die wind met geluiden als}{snikken lang en droef}\\

\haiku{dan kunnen we het,.}{hem laten weten dat je}{hem liever niet ziet}\\

\haiku{Waarom zou ze zijn,?}{belangstelling afwijzen}{hem niet willen zien}\\

\haiku{of trok, nu al, Jaaps....}{geest van haar weg en liet haar}{wezenlijk alleen}\\

\haiku{nog begrijp ik het,}{niet en hij gaf me zijn woord}{dat hij gezien had}\\

\haiku{Stel je voor, hij gaat.}{weg van onze school omdat}{hij professor wordt}\\

\haiku{Maar ziet u, ze heeft,.}{w\`el kort haar korte krullen}{net als een jongen}\\

\haiku{{\textquoteleft}Ik dacht wel, dat het,!}{iets was waar je niet mee voor}{den dag kon komen}\\

\haiku{{\textquoteright} Hij lachte, een kort,,.}{stootend lachje sloeg met zijn}{krukstok in het zand}\\

\haiku{daarvan \'e\'en, die we,....}{toch niet houden die nu al}{naam maakt met zijn werk}\\

\haiku{ze zal ook wel eens....}{een kwartiertje naar Bolland}{geluisterd hebben}\\

\haiku{We hopen dan na.}{de groote vacantie nader}{kennis te maken}\\

\haiku{Ineens werd het haar,.}{te erg midden in een zin}{liet ze zich steken}\\

\haiku{het was haar, bij wat,!}{het schonk toch wel heel duur te}{staan gekomen ook}\\

\haiku{En.... en op eenmaal....}{sprongen haar gedachten meer}{dan tien jaar  over}\\

\haiku{en het wat aardig,....}{een gezellig dineetje voor}{hem ervan maken}\\

\subsection{Uit: Verhalend proza}

\haiku{'t Jongste broertje,.}{keek haar donker aan stompte}{haar tegen den arm}\\

\haiku{Jaren daarna stierf.}{de vader en had hem geld}{en zaak gelaten}\\

\haiku{Hier links moesten ze die,.}{straat in die prachtige straat}{met die mooie huizen}\\

\haiku{{\textquoteright} {\textquoteleft}N\'e\'e, ze zullen n{\'\i}et,{\textquoteright}.}{invallen zei met hoonenden}{nadruk de vader}\\

\haiku{Niemand zou aan hem,.}{kunnen zien dat hij geen}{schoolgeld betaalde}\\

\haiku{Ik wou je ook nog...{\textquoteright} ',,, {\textquoteleft}'}{zeggent meisje zweeg even}{zei dan snel en zacht}\\

\haiku{t was echt akelig,,.}{echt valsch zooals de directeur}{tegen je praatte}\\

\haiku{{\textquoteright} Jozef keek verschrikt.}{op en beurtelings zijn broer}{en zijn vader aan}\\

\haiku{De meisjes waren.}{op een wenk van haar moeder}{al naar bed gegaan}\\

\haiku{{\textquoteright} Met de anderen.}{verheugde Daantje zich op}{den komenden avond}\\

\haiku{Jozef was nog niet,;}{aan tafel moeder had zijn}{bord eten toegedekt}\\

\haiku{Esther groette met.}{een nog hooger kleur en het}{kind groette terug}\\

\haiku{Van moeder hield ze,,.}{wel maar om wat d{\'\i}e zei gaf}{ze heelemaal niet}\\

\haiku{En nu weer naar de,.}{vijfde met een prijs en de}{jongste in z'n klas}\\

\haiku{Maar de moeder lag.}{daar met gesloten oogen of}{ze niets had gehoord}\\

\haiku{de oude heer was.}{ongetwijfeld de rijkste}{en dat bleef hoofdzaak}\\

\haiku{Leuk zou dat wezen....}{als het tenminste blonde}{kinderen waren}\\

\haiku{De drie sloegen hun.}{jaskragen op en liepen}{onverstoord verder}\\

\haiku{Zoo zal het nou op,,!}{die nieuwe school als ik er}{kom ook wel weer gaan}\\

\haiku{Dan zouden ze toch!}{stellig niet meer Jodin en}{smaus durven schelden}\\

\haiku{al was het voelen,,,}{meer dan beseffen dat zij}{hoe broos haar bestaan}\\

\haiku{En zij alleen hier,,,...}{in huis moeder dood Esther}{weg Dani\"el weg}\\

\haiku{Voor de rest ging hij,?}{z'n gang maar hoefde vader}{daarvan te weten}\\

\haiku{Dani\"el richtte.}{z'n pijnlijke hoofd op en}{keek naar het meisje}\\

\haiku{Een souper van den,,,.}{kok van Wertheimer die w\'el}{fijn maar zoo duur was}\\

\haiku{Nathans is geen fijne,.}{kok Nathans is een gewone}{koosjere bakker}\\

\haiku{Schrille en rauwe '.}{roepen klonken uit boven}{t confuus rumoer}\\

\haiku{David luierde.}{in een hoek en las in z'n}{hemdsmouwen de krant}\\

\haiku{In waarachtigheid,,. '}{hij had gedacht dat vader}{toen gek ging worden}\\

\haiku{n Paar dagen was,.}{hij gebleven toen moest hij}{terug naar z'n werk}\\

\haiku{En sinds leidde ze.}{het gewone leven van}{zoovele vrouwen}\\

\haiku{Ze verlangde nu,;}{hevig naar het land dat ze}{tegemoetgingen}\\

\haiku{Van Gulik zou daar,,.}{den eersten tijd althans een}{rustkuur doormaken}\\

\haiku{Ze was ook blij, dat,.}{Jozef trouwen ging voor hem}{en voor Rebecca}\\

\haiku{Vader scheen nog maar,.}{voor \'e\'en ding te leven voor}{zijn geloof alleen}\\

\haiku{Hij had zich verhard}{en hij verhardde zich meer}{en meer en alles}\\

\haiku{Ze sprak over hem, vond,.}{Roosje alsof ze met hem}{was getrouwd geweest}\\

\haiku{belangstellend kwam.}{haar van z'n vroolijke oogen}{de blik tegemoet}\\

\haiku{Dat kind deed nog wel,.}{haar plicht dat kind zat nog wel}{vast in het geloof}\\

\haiku{Dan zou ze haar brood,.}{hebben en geen zorgen dan}{zou ze veilig zijn}\\

\haiku{Het leven had haar,.}{nog niets gegeven dit leek}{haar te veel opeens}\\

\haiku{Hij had dankbaarheid,.}{verwacht verblijding om wat}{er besloten was}\\

\haiku{Dien brief legde hij.}{den volgenden dag op Heins}{kamer en vertrok}\\

\haiku{In die momenten,}{voelde hij w\'elbewust welk}{een schrikkelijk spel}\\

\haiku{Debora stapte.}{achter de toonbank vandaan}{en haar in den weg}\\

\haiku{Maar toen het meisje,.}{eenmaal verdwenen was was}{gauw haar toorn gezakt}\\

\haiku{Vast van plan was ze,,?}{het geweest maar goed beschouwd}{wat had ze voor r\'echt}\\

\haiku{- en Roos trouwde, al,?}{had-ze geen cent al was}{ze een Jiddekind}\\

\haiku{Nee... nee... nee, ze zou,.}{zich nergens mee bemoeien}{niet hier en niet daar}\\

\haiku{En... w\'at zei\"en ze,,?}{tegen mekaar wat deden}{ze samen die twee}\\

\haiku{Dani\"el had hij,;}{weggestuurd in zijn huis geen}{Gods-lasteraar}\\

\haiku{Z'n heele leven,.}{was hij trouw ter sjoel gegaan}{nou kon hij niet meer}\\

\haiku{{\textquoteleft}Ja..., er zal toch met?}{je vader over gesproken}{dienen te worden}\\

\haiku{misschien slaat vader......{\textquoteright} {\textquoteleft}}{mij of jaagt mij uit het huis}{Als je vader je}\\

\haiku{Hij wist, dat het uur.}{gekomen was en dat hij}{eenzaam sterven zou}\\

\haiku{Rudi en Roosje,.}{ontwaarden hem z\'o\'o bij hun}{binnentreden}\\

\haiku{{\textquoteright} Glad-zwart is de,...}{vloer van het paleis roerloos}{staan de pilaren}\\

\haiku{hij zijn schuit naar den,...}{overwal doet er hetzelfde}{en keert weer terug}\\

\haiku{In breede, malsche.}{plooien wijkt aan weerszijden}{het water terug}\\

\haiku{half-feesten!}{zijn bijna nog prettiger}{dan heele feesten}\\

\haiku{De lichtjes tellen -,}{met de avonden op elken}{middag giet vader}\\

\haiku{Zeker, juist omdat,.}{het zoo prachtig brandde moest}{het het eerste uit}\\

\haiku{Ze weet wel beter,.}{ze heeft opzettelijk de}{groote domheid gezegd}\\

\haiku{De kachel moet wel,}{geweldig branden want in}{de smalte tusschen}\\

\haiku{Dat kan vader niet,.}{zijn vader rookt geen pijp en}{vader schatert niet}\\

\haiku{want het leven is.}{nu \'e\'en en al trillend en}{tintelend geluk}\\

\haiku{Nu is ze achter,.}{die deur nu is ze in de}{kerke-kamer}\\

\haiku{erover gebogen...,,...}{kijkend wijzend als lazen}{ze elkaar iets voor}\\

\haiku{En zijn voeten staan!}{precies zoo scheef naar buiten}{gedraaid als anders}\\

\haiku{- en ze maakte de.}{kistjes open en keek alles}{na wat er in zat}\\

\haiku{Die zegt niets..., maar haar,.}{oogen worden groot haar mond gaat}{open van verbazing}\\

\haiku{nu is 't maar weer,.}{voorgoed achter den rug de}{lente komt eraan}\\

\haiku{Plat tegen den grond,,.}{bijna nog zonder steel zooals}{altijd die eersten}\\

\haiku{Maar dat klonk nu net.}{alsof ze heel iets anders}{had willen zeggen}\\

\haiku{zoolang, dat het haar,.}{soms wel eens lijkt alsof ze}{het maar had gedroomd}\\

\haiku{Van zijn tiende jaar.}{af is hij wees en woont bij}{zijn grootmoeder in}\\

\haiku{Hij leende geld aan -!}{den Keizer van Rusland want}{z\'o\'o rijk was hij wel}\\

\haiku{Ze zouden zoo graag,.}{willen dat ze David nog}{eens tegenkwamen}\\

\haiku{Eindelijk is dan...}{toch de klare waarheid wijd}{voor ze opengegaan}\\

\haiku{maar het woord doet met,...}{haar als de reuk het verschrikt}{haar en het lokt haar}\\

\haiku{Hij bewijst Zijne.}{Gerechtigheid aan hen die}{slapen in het stof}\\

\haiku{Niets vriendelijk kijkt....}{hij nu meer wat maakt haar dat}{allemaal benauwd}\\

\haiku{de twee pilaartjes,...}{in volle werking grappig}{om naar te kijken}\\

\haiku{Zoo klein is ze toch,,.}{niet dat hij haar niet meer zag}{vlak onder de lamp}\\

\haiku{in den laten avond,.}{zoodat ze van slaap niet staan en}{niet kijken konden}\\

\haiku{Snoek - dat is net als,!}{bij hen vader en moeder}{schelen ook tien jaar}\\

\haiku{Met die zou 't wel,.}{goed afloopen als ze geen}{Joum Kippour meer hield}\\

\haiku{Hij heet Salomon.}{en wie hem Sal of Sallie}{noemt op school krijgt straf}\\

\haiku{Het is dol aardig,!}{als de groote menschen overhoop}{liggen met elkaar}\\

\haiku{Ze moeten ineens...,.}{allebei lachen ja maar}{om h\'e\'el wat anders}\\

\haiku{{\textquoteleft}Ja maar, vader, als...!}{het toch regent en met ons}{mooie Sjabbosgoed aan}\\

\haiku{Ze staan allemaal.}{om mijnheer Hamel heen en}{kijken naar hem op}\\

\haiku{en dan komt het nog,!}{uit dat ze zoo'n heele plas}{hebben gemorst}\\

\haiku{Het eerste uur is,.}{voorbij de meester deelt nu}{de leesboekjes rond}\\

\haiku{Nu ze dat bedenkt,,.}{kan ze niet meer huilen kan}{ze niet meer boos zijn}\\

\haiku{Ja, ze doen het, ze,.}{doen het hun stoelen schoven}{ze al achteruit}\\

\haiku{en juist v\'o\'ordat ze,,}{opnieuw warm over haar heele}{gezicht haar kijken}\\

\haiku{{\textquoteright}        *~         [20] Hooi voor {\textquoteleft}!}{warme voetenVader heeft}{het toch zelf gezegd}\\

\haiku{En er blonk iets dat,.}{zon ving h\'e\'el fel en dat iets}{was aan den Keizer}\\

\haiku{Helpt het, als je hooi?}{heenspreidt over je voeten om}{ze af te koelen}\\

\haiku{... is haar broertje dan?}{een arme jongen en zijn}{zij arme-lui}\\

\haiku{Binnenkomen en '.}{brood-eten moet ze en dan}{naart Joodsche school}\\

\haiku{Aan hun dokter stuurt.}{vader het geld in een brief}{als het jaar om is}\\

\haiku{het is Harm Blok, en!}{er zat bloed aan zijn handen}{en bloed in zijn haar}\\

\haiku{helpt ze moeder in,.}{de kamer dan hoor je haar}{zingen vlak-bij}\\

\haiku{Je hebt gewacht, je.}{hebt er een plaats in jezelf}{voor open gehouden}\\

\haiku{Ik zit nu op een,,.}{toren er is geen vuil dat}{zoo hoog spatten kan}\\

\haiku{Je hebt het ook met -,.}{wie het niet verdienen met}{den boozen padrone}\\

\haiku{In glinsterkringen.}{en zilverig schuim staat het}{kortgesneden riet}\\

\haiku{{\textquoteright} {\textquoteleft}Ja, luisteren wekt...{\textquoteright} {\textquoteleft}.}{de geluidenGeluid dat}{er misschien niet is}\\

\haiku{{\textquoteright} En de eerste dag,,!}{dien ze langs hun oogen voorbij}{zagen gaan zij zelf}\\

\haiku{Achter ze is de,,...}{lange donkere kap als}{een drukkende hand}\\

\haiku{Elken dag weer maakt,... '}{haar de stad tot het zijne}{neemt haar in bezit}\\

\haiku{links en rechts een snoer.}{van gouden tientjes boven}{den donkeren stroom}\\

\haiku{Ik kan al niet eens,...}{goed tegen gymnastiekles}{om de commando's}\\

\haiku{Dus dat je je niet,...}{wou laten onderzoeken}{dat begrijp ik best}\\

\haiku{Reuk, laat mij los, plaag -,.}{mij niet om een naam ik kan}{hem je niet geven}\\

\haiku{tusschen buurman Bol,,.}{en buurman Bruin daar woont ze}{daar wordt ze verwacht}\\

\haiku{Maar kun je helpen {\textquoteleft}{\textquoteright}?}{dat je weet watperoe}{oe-rewoe beduidt}\\

\haiku{Ze zit achteruit,.}{in haar stoel laat het lamplicht}{in haar oogen bijten}\\

\haiku{Neen, van dat eene Hoofd,, {\textquoteleft}}{dien aardigen heer die zoo}{nadrukkelijk zei}\\

\haiku{Het is, als kwamen,.}{ze van een verre reis als}{waren ze vermoeid}\\

\haiku{Vergeet het nu even,,.}{leg het van u af dat u}{Grootvaders zoon bent}\\

\haiku{midden in den nacht,...}{midden in de zee en z\'o\'o}{is ook dit van hier}\\

\haiku{wrevelig knarsen.}{zijn zware voeten over den}{zanderigen vloer}\\

\haiku{Was dit staan over een?}{kom met rozen gebogen}{dan weer zoo iets geks}\\

\haiku{hij zei iets tegen,,...}{Selien hij lachte gooide}{zijn hoofd achterover}\\

\haiku{Joppe, ik moet t\'och...... {\textquoteleft}}{om je lachen lachen met}{tranen in mijn oogen}\\

\haiku{of denk je soms dat......}{de heele buurt het niet weet}{goddelooze jongen}\\

\haiku{Even staat Joppe tot,.}{stilte beschaamd maar dat even}{is alweer voorbij}\\

\haiku{Binnen in mijn Lijf,.}{draag ik mijn Geweten ben}{ik mijn geweten}\\

\haiku{dat ze je gevraagd,.}{heeft bij haar en Dora te}{komen inwonen}\\

\haiku{Het schijnt dat hij thuis,,:}{op niets meer aan niemand meer}{antwoord geeft hij zegt}\\

\haiku{Kan Arjen Brand het?}{helpen dat Bauk hem aan den}{haak heeft geslagen}\\

\haiku{Ze zeggen alles,.}{verschillend ze bedoelen}{alles verschillend}\\

\haiku{daar stond Herman in,,.}{de deur met zijn guitaar met}{zijn nieuwe liedjes}\\

\haiku{Leun roerloos tegen,......}{het regenraam sta stil in}{de stilte en wacht}\\

\haiku{het is ongepast,...}{en het is je geluk dat}{je maar alleen bent}\\

\haiku{{\textquoteright} Het laatste dat ze,,:}{Ernestien heeft geschreven}{was op een briefkaart}\\

\haiku{Je hebt mij altijd......?}{tegen haar verdedigd laat}{je mij nu alleen}\\

\haiku{Dan zeg ik u, dat,...{\textquoteright} {\textquoteleft}...}{de man die dat schreefEn zelf}{geen haar beter is}\\

\haiku{Willy... Rebecca...,,...{\textquoteright} {\textquoteleft}...}{neem een kopje neem dit het}{is het slapsteDank je}\\

\haiku{dit prijsgeven van,,...}{jezelf is het grootste het}{mooiste het hoogste}\\

\haiku{O... ga jij nu ook,......}{al weg Ben en ik wilde}{je juist vertellen}\\

\haiku{naar kalenders met.}{jaren en dagen moeten}{ze kunnen vluchten}\\

\haiku{En 's avonds kwam hij,...}{je halen en je praatte}{samen in het park}\\

\haiku{het blaasje richtte,,......}{zich op het zette uit het}{zwolen sloeg uiteen}\\

\haiku{Het moet zoo geweest,,.}{zijn of jij waart hier nu niet}{met mij in mijn schoot}\\

\haiku{David was na die.}{ziekte van voorverleden}{jaar nooit meer gezond}\\

\haiku{hier een man, die een...}{klein meisje vermoordt voor haar}{gouden belletjes}\\

\haiku{{\textquoteleft}Pas maar op, dat je.}{niet met paraplu en al}{de hoogte in gaat}\\

\haiku{Nu ga je dan ook,,.}{allebei voor straf onder}{den mantel vandaan}\\

\haiku{Het sterft je af -, maar,.}{zoo is het niet je laat het}{achter op je weg}\\

\haiku{* Zij zijn daarvoor nog...}{niet eens dicht genoeg bij het}{witte vuur geweest}\\

\haiku{Daar staat Heleen -, ze.}{heeft mij al gezien en ze}{weet dat ik Eva ben}\\

\haiku{En dat David als...?}{dichter begaafd was heb jij}{dat ooit geweten}\\

\haiku{En David sprak het,,.}{ook niet tegen uit trots uit}{onverschilligheid}\\

\haiku{het staat buiten de.}{intellectueele en de}{ethische problemen}\\

\haiku{Ze wilde mij den,.}{brief in mijn handen geven}{maar ze bedacht zich}\\

\haiku{Omdat het ons zoo, -!}{aangrijpt waar het ons aanraakt}{als het onszelf raakt}\\

\haiku{Wonderlijk bestaan,,.}{waarin juist dit verzwegen}{wordt vergeten wordt}\\

\haiku{Maar laat het zoo zijn,.}{laat het kunnen dat de een}{voor den ander boet}\\

\haiku{Jaap en Ben deinsden,...}{ervoor terug en toen was}{Eddy geboren}\\

\haiku{Want eerst ga ik nu...}{naar vader en moeder en}{straks ga ik naar huis}\\

\haiku{Dat we er trotsch op,?}{zijn dat we ons haast schamen}{als het anders is}\\

\haiku{als het hunne in {\textquoteleft},...{\textquoteright}}{de spanning van hetHerder}{laat je schaapjes gaan}\\

\haiku{Ik liep zoo veilig,,:}{ik liep vlak achter Eddy}{en Claartje ik dacht}\\

\haiku{{\textquoteleft}Ik sta ervan te -, -,}{kijken ik hield jullie voor}{gelukkig getrouwd}\\

\haiku{{\textquoteleft}Kunnen we niet eens,,...?}{samen een heelen morgen}{een heelen middag}\\

\haiku{Misschien zoo ver zelfs,}{nog niet ik ben zoo snel van}{daar naar hier komen}\\

\haiku{Maar menschen dienen.}{t\'och behoorlijk aan elkaar}{te zijn voorgesteld}\\

\haiku{Je loopt sinds lang de,.}{kermistent voorbij je lacht}{om de muizenval}\\

\haiku{{\textquoteright} {\textquoteleft}De allergrootste,,.}{dwaasheid op die eene na op}{die andere na}\\

\haiku{Uit de starende,,}{sterren viel het woord mij toe}{maar dat was later}\\

\haiku{maar in mijn begrip,.}{kon ik het toch niet binden}{nooit kreeg het een zin}\\

\haiku{benaming voor de:}{eerste fietsmodellen.402l'Alsace}{et la Lorraine}\\

\haiku{de aangehaalde:}{dichtregel komt uit een van}{de sonnetten.514Ellen}\\

\haiku{alleen zijn zoon Cham,,.}{die zijn dronken vader had}{bespot werd vervloekt}\\

\haiku{6:4, die ook op het.}{rolletje perkament in}{de mezoezo staat}\\

\haiku{{\textquoteright} verklaarde ze in.}{1915 in een interview met}{Andr\'e de Ridder}\\

\haiku{Ook haar huwelijk}{met de sterk van zijn eigen}{gelijk overtuigde}\\

\subsection{Uit: De verlatene}

\haiku{'t Jongste broertje,.}{keek haar donker aan stompte}{haar tegen den arm}\\

\haiku{Jaren daarna stierf.}{de vader en had hem geld}{en zaak gelaten}\\

\haiku{Hijgend rende hij,,,.}{zonder omzien niet hoorend wat}{ze hem najouwden}\\

\haiku{{\textquoteright} Jozef keek verschrikt.}{op en beurtelings zijn broer}{en zijn vader aan}\\

\haiku{De meisjes waren.}{op een wenk van haar moeder}{al naar bed gegaan}\\

\haiku{{\textquoteright} Met de anderen.}{verheugde Daantje zich op}{den komenden avond}\\

\haiku{Jozef was nog niet,;}{aan tafel moeder had zijn}{bord eten toegedekt}\\

\haiku{Esther groette met.}{een nog hooger kleur en het}{kind groette terug}\\

\haiku{tot laat 's avonds zat.}{ze Jozefs uitzetje}{te beredderen}\\

\haiku{Van moeder hield ze,,.}{wel maar om wat di\`e zei gaf}{ze heelemaal niet}\\

\haiku{Daniel merkte z'n,.}{Joodsch accentje dat hij}{thuis niet had gehad}\\

\haiku{En nu weer naar de,.}{vijfde met een prijs en de}{jongste in z'n klas}\\

\haiku{Maar de moeder lag.}{daar met gesloten oogen of}{ze niets had gehoord}\\

\haiku{de oude heer was.}{ongetwijfeld de rijkste}{en dat bleef hoofdzaak}\\

\haiku{De drie sloegen hun.}{jaskragen op en liepen}{onverstoord verder}\\

\haiku{Zou\"en ze je dat,,....}{nou \`erg kwalijk nemen je}{familie en de}\\

\haiku{Zoo zal het nou op,,!}{die nieuwe school als ik er}{kom ook wel weer gaan}\\

\haiku{Dan zouden ze toch!}{stellig niet meer Jodin en}{smaus durven schelden}\\

\haiku{al was het voelen,,,}{meer dan beseffen dat zij}{hoe broos haar bestaan}\\

\haiku{Ik ga nou met Moos....}{praten en morgenochtend}{dan spreek ik met jou}\\

\haiku{Voor de rest ging hij,?}{z'n gang maar hoefde vader}{daarvan te weten}\\

\haiku{Een souper van den,,,.}{kok van Wertheimer die w\`el}{fijn maar zoo duur was}\\

\haiku{Nathans is geen fijne,.}{kok Nathans is een gewone}{koosjere bakker}\\

\haiku{Schrille en rauwe '.}{roepen klonken uit boven}{t confuus rumoer}\\

\haiku{David luierde.}{in een hoek en las in z'n}{hemdsmouwen de krant}\\

\haiku{Is het zijn schuld, dat?}{z'n broer en z'n zuster}{slecht zijn geworden}\\

\haiku{In waarachtigheid,,. '}{hij had gedacht dat vader}{toen gek ging worden}\\

\haiku{n Paar dagen was,.}{hij gebleven toen moest hij}{terug naar z'n werk}\\

\haiku{En sinds leidde ze.}{het gewone leven van}{zoovele vrouwen}\\

\haiku{Van Gulik zou daar,,.}{den eersten tijd althans een}{rustkuur doormaken}\\

\haiku{Ze was ook blij, dat,,}{Jozef trouwen ging voor hem}{en voor Rebecca}\\

\haiku{Vader scheen nog maar,.}{voor \'e\'en ding te leven voor}{zijn geloof alleen}\\

\haiku{Hij had zich verhard}{en hij verhardde zich meer}{en meer en alles}\\

\haiku{Ze sprak over hem, vond,.}{Roosje alsof ze met hem}{was getrouwd geweest}\\

\haiku{belangstellend kwam.}{haar van z'n vroolijke oogen}{de blik tegemoet}\\

\haiku{Dat kind deed nog wel,.}{haar plicht dat kind zat nog wel}{vast in het geloof}\\

\haiku{Dan zou ze haar brood,.}{hebben en geen zorgen dan}{zou ze veilig zijn}\\

\haiku{Het leven had haar,.}{nog niets gegeven dit leek}{haar te veel opeens}\\

\haiku{Hij had dankbaarheid,.}{verwacht verblijding om wat}{er besloten was}\\

\haiku{Dien brief legde hij.}{den volgenden dag op Heins}{kamer en vertrok}\\

\haiku{In die momenten,}{voelde hij w\`elbewust welk}{een schrikkelijk spel}\\

\haiku{Daniel voelde zich,}{een volslagen vreemde in}{dien familiekring}\\

\haiku{Debora stapte.}{achter de toonbank vandaan}{en haar in den weg}\\

\haiku{hou je in, hier op..,....}{straat ga mee naar mijn huis en}{vertel mij alles}\\

\haiku{Maar toen het meisje,.}{eenmaal verdwenen was was}{gauw haar toorn gezakt}\\

\haiku{Vast van plan was ze,,?}{het geweest maar goed beschouwd}{wat had ze voor r\`echt}\\

\haiku{- en Roos trouwde, al,?}{had-ze geen cent al was}{ze een Jiddekind}\\

\haiku{Nee.... nee.... nee, ze zou,.}{zich nergens mee bemoeien}{niet hier en niet daar}\\

\haiku{En.... w\`at zei\"en ze,,?}{tegen mekaar wat deden}{ze samen die twee}\\

\haiku{Nee, nee, Debora,....}{was een vreemde Debora}{had nergens mee noodig}\\

\haiku{Z'n heele leven,.}{was hij trouw ter Sjoel gegaan}{nou kon hij niet meer}\\

\haiku{{\textquoteleft}Ja...., er zal toch met?}{je vader over gesproken}{dienen te worden}\\

\haiku{Hij wist, dat het uur.}{gekomen was en dat hij}{eenzaam sterven zou}\\

\haiku{Rudi en Roosje,.}{ontwaarden hem z\'o\'o bij hun}{binnentreden}\\

\subsection{Uit: Het verspeelde leven (onder ps. Justine Abbing)}

\haiku{Ze droomde niet meer,,.}{vooruit het heden had haar}{weer ze liep vlugger}\\

\haiku{maar waarom niet, toen,?}{tante de eerste maal vroeg}{een smoesje bedacht}\\

\haiku{Een zuchtje wind deed,;}{het klimop lispen verre}{auto's toeterden}\\

\haiku{wat had hij dan dat?}{mensch in haar graf nog voor te}{trekken boven haar}\\

\haiku{Die pias in zijn,.}{zijden kuitenbroek zooals Wim}{hem altijd noemde}\\

\haiku{dan praatten ze over.}{politiek en hij wist het}{meest te vertellen}\\

\haiku{Ik vind wel eens dat '....}{hijt een beetje in de}{kleinigheden zoekt}\\

\haiku{{\textquoteright} Haar wangen bleven,.}{rood terwijl ze Ida naar de}{deur begeleidde}\\

\haiku{was iets van echte,.}{warmte en hartelijkheid}{nu hij haar groette}\\

\haiku{{\textquoteright} noodde de vader, {\textquoteleft} '?}{en tot zijn vrouw lachendKan}{t niet op moeder}\\

\haiku{nu zijn denken het, '.}{jongste meisje beroerde}{wast weer vlak bij}\\

\haiku{Tante liet vragen.....}{of Ied vannacht hier slapen}{mocht enkel vannacht}\\

\haiku{de eerste dag dat,....}{haar nichtje in de stad was}{haar bij wildvreemden}\\

\haiku{daar kan dedokter,....}{of de zuster mee meten}{hoe ziek of je bent}\\

\haiku{{\textquoteleft}'t Staat nog al in '!}{t gedeelte dat je voor}{je A2 moet kennen}\\

\haiku{{\textquoteleft}Maar nu gaan we naar,,!}{het dorp je weet wel bij Spoel}{en daar eten we ijs}\\

\haiku{- en dan nog boeken,!}{te lezen buiten voorschrift}{dokters-boeken}\\

\haiku{Wat zag je er toch,.}{best uit in dien tijd en wat}{kon je vroolijk zijn}\\

\haiku{Ze leek op mevrouw,.}{maar ze moest een heel ander}{karakter hebben}\\

\haiku{U zoudt eens zien, wat.}{hij zoo'n meisje tot haar recht}{zou laten komen}\\

\haiku{Zullen we haar dan,?}{eens op de koffie vragen}{een dezer dagen}\\

\haiku{Ze heeft wel eens een,.}{vrijen dag dien kan ze dan}{bij ons doorbrengen}\\

\haiku{Jawel, en ze ligt.}{voor de verandering weer}{eens met Coen overhoop}\\

\haiku{wat laat je toch die.}{kinderen zich blij maken}{met een doode musch}\\

\haiku{En waarom zou hij '?}{t niet uit fatsoenlijkheid}{gelaten hebben}\\

\haiku{Och neen{\textquoteright} hortte het, {\textquoteleft}....}{eindelijk over Ida's lippen}{zeg maar aan je ma}\\

\haiku{toch.... als ze niet zou,.}{hebben geweten wat ze}{gesproken hadden}\\

\haiku{{\textquoteright} Marie was blijven,,.}{staan verbluft haar verhaal nog}{lang niet ten einde}\\

\haiku{Ik geloof niet dat....}{Roomsch of Protestant er iets}{mee te maken heeft}\\

\haiku{Ze houdt van Ida, ze,.}{heeft alles voor haar over dat}{zegt Ied altijd zelf}\\

\haiku{Ze praatte meer tot,.}{zichzelf dan tot Ida die dan}{ook geen antwoord gaf}\\

\haiku{{\textquoteright} Marie raapte haar,.}{lorgnet uit het zand het eene}{glas was gebroken}\\

\haiku{en eigenlijk ook, {\textquoteleft}{\textquoteright} {\textquoteleft}.}{omdat een dame het had}{aangelegd metpa}\\

\haiku{Zie je, Brand is niet,.}{kwaad maar hij houdt wel wat heel}{erg van zijn gemak}\\

\haiku{{\textquoteright} {\textquoteleft}Ze behandelen ', '.}{ze overt algemeen toch}{goed zouk zeggen}\\

\haiku{O, hoe zou dat in,?}{elkaar zitten hoe mocht dat}{allemaal wel zijn}\\

\haiku{Niemendal dan een,....}{brief van een man die vroeger}{op je heeft getrapt}\\

\haiku{of het barsten zal,....}{zoo ellendig-kapot}{voel ik me ineens}\\

\subsection{Uit: De vier jaargetijden}

\haiku{Want al wist je wel,.}{dat ze naderden dit kwam}{toch weer onverwacht}\\

\haiku{Z\'o\'o eten zij het nooit,,,}{maar ze lachen niet want ze}{begrijpen het wel}\\

\haiku{Uit het booze land is, ...}{hij voortgestooten en is}{naar hier gekomen}\\

\haiku{Daar zwelt hij nu weer ...}{naar haar wangen en maakt haar}{mond tot stikkens vol}\\

\haiku{je spookt zoo benauwd}{en zoo rusteloos rond of}{mijn mond veel te klein}\\

\haiku{de wind doet buiten ...}{de boomen fluisteren en}{ritselend zuchten}\\

\haiku{Ze gaat naast ze en ...}{luistert en ziet haar schoenen}{onder zich stappen}\\

\haiku{En zulke zullen,,!}{er zijn en meer dan \'e\'en voor}{ieder op het feest}\\

\haiku{ik zal dit en dat,, ...}{ik ga hierheen en daarheen}{ik doe zoo en zoo}\\

\haiku{Nu is het al zoo,,}{dicht bij dat het wolkend stof}{tot haar lippen komt}\\

\haiku{Hij zal zijn paard niet, ...}{slaan dat kan ook best het mooie}{wagentje trekken}\\

\haiku{O neen, zoo niet, niet ...}{iets dat zoo dicht is en zoo}{afsluit als een deur}\\

\haiku{{\textquoteleft}Meteen schreeuw werd de,.}{zeeman wakker hij had zijn}{eigen graf herkend}\\

\haiku{het roepen van de, ...}{mannen die ze lossen en}{tot vlotten vormen}\\

\haiku{En daar komt het weer, ...}{aan en zwelt als een knop en}{die breekt voor haar open}\\

\haiku{hoe houd je het uit ...}{elkaar in de volheid van}{zoo'n overvollen dag}\\

\haiku{Ze staan nu samen ...}{voor het portretje tegen}{den donkeren muur}\\

\haiku{-, je voelt immers dat,}{Grootvader het heel goed merkt}{en mijnheer De Beer}\\

\haiku{Een woord schrijf je op, ...}{als je bang bent dat het je}{weer ontglippen zal}\\

\haiku{En als Jaap hem toch,!}{missen kan zou ze hem zelfs}{graag willen houden}\\

\haiku{De avond was nabij ...,}{en het huis straalde lamplicht}{uit al zijn kamers}\\

\haiku{Maar hij liet het niet.}{blijken en staat nu half met}{den rug naar haar toe}\\

\haiku{de menschen denken ....}{dat ze loopen kunnen dat}{loopen geen kunst is}\\

\haiku{Er wordt niet openlijk, ...}{gefluisterd niet rechtstreeks met}{vingers gewezen}\\

\haiku{Hij staat voor haar, hij ...}{reikt lang en breed boven haar}{en buiten haar uit}\\

\section{Kees van Bruggen}

\subsection{Uit: Als ge niet.... dan!}

\haiku{Vreemd bleef de jonge.}{vrijwilliger zijn vijand}{aan zitten kijken}\\

\haiku{Hij keek eens, en de.}{ander antwoordde met een}{pijnlijken glimlach}\\

\haiku{Als 't er aanzat '.}{namen wet er in de}{faubourg ook goed van}\\

\haiku{Die schiet je een gat,.}{in hun jas of je spit ze}{aan je bajonet}\\

\haiku{De soldaten die,.}{hier lagen waren niet van}{zijn eigen partij}\\

\haiku{Een zwarte ziekte.}{leek het levend gewas te}{hebben aangeteerd}\\

\haiku{- Het zal drie dagen,.}{geleden zijn hervatte}{de Engelschman}\\

\haiku{nu zouden woord en.}{begrip ook nimmer meer uit}{zijn hoofd verdwijnen}\\

\haiku{- Weet ik 't! - Misschien ',.}{ist mij ook zoo gegaan}{peinsde de ander}\\

\haiku{Ik meende ook te, {\textquoteleft}{\textquoteright}...}{zeggen dat  wat wij zoo}{beschaving noemen}\\

\haiku{- Mogelijk niet heel,,:}{en al wanneer we den draad}{volgen Denk eens na}\\

\haiku{Zou een generaal,?}{ons het v{\'\i}nden van deze}{plek verbeteren}\\

\haiku{Ze heeft mij alleen!}{geroepen toen ze mij noodig}{had om te moorden}\\

\haiku{De meester op school.}{deed z'n best een klassekind}{van me te maken}\\

\haiku{Ik mocht m'n meid niet.}{verdedigen tegen den}{agent die haar sarde}\\

\haiku{Ik moet - 't is om -!}{te stikken van den lach ik}{moet voor h\`a\`ar vechten}\\

\haiku{Kennelijk was de.}{vijfde man de aanvoerder}{der vier soldaten}\\

\haiku{Maar zijn verlamde,}{tong kon niet zeggen wat hem}{ontroerde en zoo}\\

\haiku{Door zijn trawanten,,:}{gevolgd ging de graaf op zoek}{aldoor roepende}\\

\haiku{Dat eene tergende, -...}{beest voor zijn oogen dat beest het}{wilde iets van hem}\\

\haiku{Hun werk en hun aard,.}{was vredig zij leefden een}{zachtzinnig leven}\\

\haiku{- waarin verschilden?...}{de nieuwe bewindvoerders}{van de vroegere}\\

\haiku{- Heeft de wereld \`ons...?}{of hebben wij de wereld}{teruggevonden}\\

\haiku{Men hoorde van den,.}{Franschman een woord dat veel}{op een vloek geleek}\\

\haiku{Om ons heen bleef hij;}{cirkelen bij al onze}{voorbereidingen}\\

\haiku{Mijn reisdagboek staat.}{vol met de merkwaardigste}{aanteekeningen}\\

\haiku{Zij drukte zijn hand,,:}{die over het bed hing en ging}{innemend verder}\\

\haiku{Aan onzen kant was.}{het inderdaad zooals we bij}{hen onderstelden}\\

\haiku{Geschreeuw... kreten... en,,...}{de lichamen gigantisch}{dansten in den mist}\\

\haiku{Een figuur was, toen,.}{zij dit zeide tusschen de}{tenten omgegaan}\\

\haiku{het geschenk harer.}{lieftalligheid onder hen}{allen verdeelde}\\

\haiku{Als vier wielen een,,.}{wagen zoo droegen zij den}{Graaf hunnen meester}\\

\haiku{Hij voelde iets van,.}{een kneep in de keel alsof}{hij zou gaan snikken}\\

\haiku{Hij, sinds de zieke,.}{genezen was voelde geen}{lust tot den arbeid}\\

\haiku{In zijn zelfverwijt.}{verachtte en benijdde}{hij hen tegelijk}\\

\haiku{Toch bleef er iets als,.}{verbazing in hem dat het}{ook hier gebeurde}\\

\haiku{Tegen onzen wil?}{u klinken in het staal of}{binden aan touwen}\\

\haiku{Want iedere straf,.}{die wij verzinnen voor u}{keert zich tegen \`ons}\\

\haiku{Schreeuwend met telkens,.}{overslaande stem eischte}{hij gerechtigheid}\\

\haiku{Groot en blond was hij,.}{zijn hoofd stond achterover van}{uitdagende jeugd}\\

\haiku{V\'o\'or het vertrek zou.}{ik echter nog Een spannend}{avontuur beleven}\\

\haiku{in hoeverre had?}{men hen te beschouwen als}{krijgsgevangenen}\\

\haiku{mondaine Franschen;}{met bonte slipdassen en}{gespleten baarden}\\

\haiku{men begeleidde.}{de sprekers met steenworpen}{naar het station}\\

\haiku{Achter de vaandels.}{van den Keizer barst een heet}{volk de grenzen over}\\

\haiku{Men zag een hunner:}{herhaalde malen met een}{ruk van zijn groot hoofd}\\

\haiku{Op een overheidlooze,.}{kolonie intusschen was}{niet gerekend}\\

\haiku{Hij had geleerd te, ().}{dulden en het woorddat men}{z\`elf sprak was \'o\'ok iets}\\

\haiku{Zou er iemand reeds,?}{wakende zijn met wien zij}{aanknoopen konden}\\

\haiku{Voor de eerste maal;}{in z'n leven nederig}{van binnen werd hij}\\

\haiku{De vergadering;}{vond veel pleizier in deze}{bedaarde leukheid}\\

\haiku{De gevolgen zijn ';}{voor u. Maar de kring was reeds}{aant verloopen}\\

\haiku{Vooral de Fransche,,.}{gezant een causeur had wil}{van zijn episode}\\

\haiku{De vingers plaatsend,,:}{tegen elkaar bedenkt hij}{zich en hij vervolgt}\\

\haiku{Ieder onzer - let! -...}{op krijgt twintig gepelde}{amandelen v\'o\'or zich}\\

\haiku{men had den feestmast;}{op het groene middenveld}{niet omvergehaald}\\

\haiku{Zijn lichaam, in de,.}{vaart gestuit stuipte over den}{grond en bleef stil}\\

\subsection{Uit: Fontana Marina}

\haiku{Het blonde meisje,.}{volgde glimlachend om de}{grote lieve man}\\

\haiku{Er  zou dan wel {\textquoteleft}{\textquoteright},.}{een juffrouwin het spel zijn}{wilden zij zeggen}\\

\haiku{In die zin was er,.}{Toet's aanvaarding van de man}{die haar Man zou zijn}\\

\haiku{een diepe voor, daar.}{stegen te weerskant wallen}{naar een dunne lucht}\\

\haiku{{\textquoteleft}Als er nu, Arend{\textquoteright} - zij - {\textquoteleft}...}{had zijn arm gegrepeneen}{grote vinger kwam}\\

\haiku{Er was nu niets meer.}{dan een trap van in de rots}{geplante scherven}\\

\haiku{Als een paard schraapte.}{hij de teelaarde van de}{harde ondergrond}\\

\haiku{Gedwee volgde zij.}{haar man door het poortje naar}{de weg terug}\\

\haiku{Tussen die korrels,.}{drong het water daarbinnen}{bleef het machteloos}\\

\haiku{Daar staan al twee of,...?}{drie passagiers die verder}{willen Zie je ze}\\

\haiku{Nu, toen waren wij.}{onverwacht genodigd bij}{een kikkerbruiloft}\\

\haiku{Zo lang zij samen.}{waren had zij hem nimmer}{reden gegeven}\\

\haiku{Evengoed kon ieder,,.}{ander vrouw of vriend hem met}{tegenspel dienen}\\

\haiku{{\textquoteright}, je hand beklopt haast,}{zonder dat je het weet zijn}{schouder dan voel je}\\

\haiku{Sinds trok zij met hem,,,.}{op ze reisden wandelden}{deelden hun armoe}\\

\haiku{Een naam hoorde er,.}{niet bij maar alles zag hij}{duidelijk terug}\\

\haiku{Wat zal ik 'n schat '!}{vann ouwe dame van}{vijfenvijftig zijn}\\

\haiku{{\textquoteright} De lange kerel.}{sprong dat zijn kop stiet tegen}{de zwarte hemelkruin}\\

\haiku{Arend vond het genoeg,.}{hij had geen zin zo met dat}{zotshoofd door te gaan}\\

\haiku{Hij zou straks eensklaps.}{uit de struiken opduiken}{en haar daar zien staan}\\

\haiku{Teleurgesteld rees,.}{hij omhoog hersjorde zijn vracht}{op  zijn bochel}\\

\haiku{Jaap Sparkels ezeltje.}{ijlbenig in het licht en}{droeg het schilderij}\\

\haiku{Ze moeten er zijn,.}{anders blijft alles los in}{de leegte hangen}\\

\haiku{{\textquoteleft}Toet zal vanmiddag,{\textquoteright}.}{voor me poseren bedacht}{hij zonder overgang}\\

\haiku{Chateaubriant had,{\textquoteright}:}{gelijk voer Arend zonder zijn}{spot te horen voort}\\

\haiku{Van het ogenblik dat.}{je weggaat blijft er een gat}{waar je geweest bent}\\

\haiku{{\textquoteright} zei Jaap geduldig, {\textquoteleft}.}{maar laat me niet afleiden}{van mijn onderwerp}\\

\haiku{Ik kan de koffie.}{van vanmiddag opwarmen}{met wat geitemelk}\\

\haiku{Het werd zo stil in,.}{de kamer of een gordijn}{was dichtgeschoven}\\

\haiku{{\textquoteright} {\textquoteleft}Ik zou er eens over,{\textquoteright}.}{moeten denken behield Jaap}{zich zijn mening voor}\\

\haiku{{\textquoteleft}Blijf me overigens;}{met de grootvaders van ons}{bedrijf van het lijf}\\

\haiku{Daarom vrijden de.}{Schuckenscheuer Schiefsalheims}{alleen bij kaarslicht}\\

\haiku{Neem dus in godsnaam,,.}{de blonde Lorelei die}{er nu toch is mee}\\

\haiku{De levende muis {\textquotedblleft}{\textquotedblright},.}{houdt de katin form zoals}{de sportslui zeggen}\\

\haiku{Hij leek nog verder,.}{te willen gaan maar stuitte}{plotseling zijn vaart}\\

\haiku{{\textquoteleft}Wat heb jij vandaag,?}{voor de onsterfelijkheid}{gedaan Don Carlos}\\

\haiku{Er wordt druk gedicht,,.}{en veel geschilderd dat wel}{als u dat bedoelt}\\

\haiku{Het tegendeel is,.}{waar die heren hebben het}{alleen over zichzelf}\\

\haiku{{\textquoteright} De dichter richtte.}{zijn duikerskop verwonderd}{naar alle kanten}\\

\haiku{Er zal niets anders,,.}{op zitten dichter dan op}{jezelf gaan wonen}\\

\haiku{Buiten in het licht.}{hing aan een dode twijg een}{vergeten amandel}\\

\haiku{Een amandel van het,?}{vorige jaar of kon het}{alreeds zijn van dit}\\

\haiku{Ze zou hem zeker,.}{te eten vragen al keek de}{man hem de deur uit}\\

\haiku{De hemel, of wie,,.}{daar aansprakelijk voor is}{bewaar me ervoor}\\

\haiku{Je moet de schuld nooit,.}{bij jezelf zoeken leerde}{de politieker}\\

\haiku{Banale wijsheid...?}{van een dubbeltje de tien}{wie zei dat ook weer}\\

\haiku{Je bent op mij, je,,,,.}{vrouw je kind je voetveeg je}{niets ben je verliefd}\\

\haiku{{\textquoteright} Zij tolde als een.}{mannequin op de spil van}{haar lange benen}\\

\haiku{{\textquoteright} {\textquoteleft}Zodat iemand in;}{een geestelijk vacuum niet}{zou kunnen denken}\\

\haiku{De beek ging voort te,.}{huppelen van de rotsen}{dat was haar bedrijf}\\

\haiku{{\textquoteright} {\textquoteleft}O, ons geweten,{\textquoteright}.}{staat erbuiten wist de man}{naast haar onverstoord}\\

\haiku{De man  boven.}{stond als een boomstomp in de}{wit-hete middag}\\

\haiku{Dag en nacht zwoegt hij.}{in de tredmolen zijner}{minderwaardigheid}\\

\haiku{Aandachtig was zij,.}{bezig met de primus haar}{eeuwige vijand}\\

\haiku{Het heeft geen zin het.}{bijzondere geval te}{veralgemenen}\\

\haiku{{\textquoteright} riep zij verbijsterd, {\textquoteleft},...{\textquoteright} {\textquoteleft}?}{Ugo de DuitserWat is er}{nu alweer met h\`em}\\

\haiku{Balorig werkte,.}{hij zich door zijn examens heen}{studeerde niet af}\\

\haiku{Zij glimlachte in,.}{herinnering maar de glans}{gleed van haar kaken}\\

\haiku{Arend Hobbe was de.}{eerste die woorden nodig}{oordeelde en vond}\\

\haiku{Het groepje mensen.}{stond daar als een disparaat}{Cooksgezelschap}\\

\haiku{Wat mij betreft, mijn,.}{postuur leent er zich weinig}{toe stel het je voor}\\

\haiku{{\textquoteright} hield Jaap querulant.}{vol. Zijn ogen stonden scherp in}{hun smalle kassen}\\

\haiku{Maar het ging niet om,.}{Hortense het ging om dat}{Blauwe Vrouwtje hier}\\

\haiku{Nu jullie eenmaal,.}{hier zijn verlang ik dat wij}{als vrienden leven}\\

\haiku{{\textquoteright} {\textquoteleft}Kom hier zitten, Toet,{\textquoteright},.}{nodigde Jaap begerig}{naar haar nabijheid}\\

\haiku{De eerwaarde heeft,.}{het goede met ons voor laat}{ons dat waarderen}\\

\haiku{Alleen de schilder,.}{zag het hij werd bedroefd en}{woedend tegelijk}\\

\haiku{Werd zij getrokken?}{of duwde een geheime}{drang haar van Arend weg}\\

\haiku{Op ieder ander,.}{uur misschien want je bent mooi}{en ik vind je lief}\\

\haiku{{\textquoteleft}Hortense was niet,,,?}{eens een sprookje en jij Toet}{bent jaloers van haar}\\

\haiku{God schiep de mensen,.}{en de kippen daar was geen}{tussenkomen aan}\\

\haiku{De Rector zelf met.}{zijn zwarte kapje stond voor}{het ijzeren hek}\\

\haiku{Nu wil je zeker,{\textquoteright}, {\textquoteleft}?}{giste hij achterdochtig}{ook weten waarheen}\\

\haiku{En wijselijk ook,.}{zij zouden genoodzaakt zijn}{naar hem te zoeken}\\

\haiku{Zeker betekent,{\textquoteright}.}{het niet dat jij te veel was}{verzekerde hij}\\

\haiku{{\textquoteright} smeekte het meisje,.}{een lieve glimlach schetsend}{die niet gelukte}\\

\haiku{Niets gedaan hier, in,.}{een natuur die alleen om}{de natuur gaf}\\

\haiku{Ik ben gebleven.}{die ik was op de dag toen}{ik ben gekomen}\\

\subsection{Uit: De freule}

\haiku{Door het floers van het,.}{witte tulle-gordijn}{verscheen de vreemde}\\

\haiku{Daar ginds... u kunt ze,...}{juist zien achter de gazons}{bij de oranjerie}\\

\haiku{En weer zweeg hij, naar '.}{t scheen gedompeld in zijn}{landschapbeschouwing}\\

\haiku{Uit haar reserve,,:}{die haar als kunstmatig bleef}{hinderen vroeg ze}\\

\haiku{Hij moet schilder zijn,,,,.}{maar hij heeft niets bij zich geen}{verf geen doeken niets}\\

\haiku{En in een woede,.}{die zij niet meer bedwong hief}{zij haar parasol}\\

\haiku{En weder, met zijn,:}{zekeren glimlach sprak hij}{haar gedachte uit}\\

\haiku{Ik leef ze, en waar,.}{ik tegenstand vind kom ik}{voor mijn rechten op}\\

\haiku{{\textquoteright} Maar dit zou wel het,.}{ergst zijn wat hij kon doen dat}{voelde hij ook wel}\\

\haiku{Nadenkend stond hij,.}{er mee sloeg het dan peinzend}{en gewichtig om}\\

\haiku{Mag ik mijn huis  ,?}{bewonen dat veel te groot}{is voor mij alleen}\\

\haiku{De dingen waren,.}{zoo en het was goed dat de}{dingen zoo waren}\\

\haiku{Wanneer jij nu ik, -?}{was en ik was bijvoorbeeld}{jou wat zou je dan}\\

\haiku{Want uit haar eigen:}{wezen begon zij opnieuw}{grenzen te bouwen}\\

\haiku{Hij zou zich wonden,.}{hij zou zijn overwinnenden}{glimlach verliezen}\\

\subsection{Uit: De geschiedenis van het huis. Een verhaal van vele avonturen}

\haiku{een zilveren bal,.}{in een schiettent en die danst}{op een fonteintje}\\

\haiku{De Architect ging.}{over tot een verhandeling}{over Ruimte-kunst}\\

\haiku{Men fluisterde dat'.}{de koning zich bedronk aan}{Samos zoeten wijn}\\

\haiku{Ook heeft hij reeds vaak,.}{met den architect gewerkt}{men begrijpt elkaar}\\

\haiku{Dat was een heele,.}{geschiedenis men kon nooit}{weten wie er kwam}\\

\haiku{riep de aannemer, '.}{een timmerman toe geef me}{s even je duimstok}\\

\haiku{Die hadden die groote! -,!}{scheur in haar tricot gezien}{Allemaal jongens}\\

\haiku{Het leek of ze haast,.}{hadden dupe te worden}{van het misverstand}\\

\haiku{Het was dan dat de.}{schoorsteen niet op de goede}{plek gemetseld was}\\

\haiku{Er is chromaat en.}{cadmium en siena en}{curcuma en oker}\\

\haiku{Want ook hij had zijn,,.}{meeningen al stond hem niet}{vrij ze te zeggen}\\

\haiku{Ze moeten wat, ze, {\textquoteleft},!}{kijken naar het neveltje}{ze zeggenwel wel}\\

\haiku{Aan den storm aan den,,.}{maneschijn aan den weg heen}{aan den we terug}\\

\haiku{Het loont niet, nog iets.}{omvangrijks aan te vangen}{en dat moet ook niet}\\

\haiku{Zou hem met honden.}{en geweren van het erf}{willen verjagen}\\

\haiku{Iets wat niemand, op,.}{straffe van er niet geweest}{te zijn mag missen}\\

\haiku{We komen weleens.}{terug als er kolen voor}{het proefstoken zijn}\\

\haiku{Misschien waren ze,.}{weleens kraamheer dan weten}{ze er alles van}\\

\haiku{Hij blafte als een.}{razende en sprong met vier}{pooten tegelijk}\\

\haiku{Het is dezelfde,.}{mijnheer alleen rekent hij}{een hooger tarief}\\

\haiku{De regisseur komt.}{alleen om te vertellen}{dat hij er niet is}\\

\haiku{De Bouwheer begon.}{zijn werk met alleenspraken}{te begeleiden}\\

\haiku{suste Jules, die.}{als wijsgeer afkeerig was}{van sterke woorden}\\

\haiku{De huisjes waren,.}{kleine rookfabriekjes ze}{brandden van binnen}\\

\haiku{- Nou, ik had nog wel,,, '}{voor een dag gehad maar het}{gaat een vaartje weet je}\\

\haiku{Maar de keuken is,.}{een werkplaats een museum}{en een boekerij}\\

\haiku{Die twee verstonden.}{elkander en elk woord was}{verder overbodig}\\

\haiku{hier zwaaide  een,.}{deur daar ginds kon het venster}{onmogelijk open}\\

\haiku{De stucadoor kwam,.}{om ze bij te strijken toen}{was de dag ten eind}\\

\haiku{Partijen en hun.}{vertegenwoordigers in}{rechten gingen heen}\\

\haiku{Dat hij dus den heer....}{Kantonrechter verzocht te}{willen bepalen}\\

\haiku{Hier krimpt een plank, daar....}{wringt zich een draadnagel als}{een pier naar buiten}\\

\haiku{Zij beleven de.}{disilluzie hunner}{laatste illuzie}\\

\subsection{Uit: Koning Adam}

\haiku{Heen en weder, heen,.}{en weder volgde hem ginds}{de glanzende schim}\\

\haiku{V De wereld om,.}{Adam was groot ontzaglijk vol}{vragen geworden}\\

\haiku{Hij scheen zijn evenwicht,,,.}{te hebben herwonnen en}{wijs vroeg zij hem niet}\\

\haiku{Hij kwam naar binnen,,.}{haar streelen over het klamme}{gemartelde hoofd}\\

\haiku{Den morgen bij het,.}{rijzen van het licht stond hij}{uit zijn leger op}\\

\haiku{In het zoekende:}{voortzwerven bezon hij de}{mogelijkheden}\\

\haiku{Het was haar nooit naar,.}{den zin te maken altijd}{grommen en brommen}\\

\haiku{Voorzeker had hij,.}{alle kipjes even lief en}{zij beminden hem}\\

\haiku{of de linde of,.}{de beuk die hun bloeisel in}{het blad verbergen}\\

\haiku{En uit het water}{glipten te allen kant de}{zilveren vischjes}\\

\haiku{Hoor, de wind staat stil,......!}{in de boomen een fluister}{gaat er van geluk}\\

\haiku{Haar verarmd lichaam,,.}{zij wist het ontstak hem niet}{meer in verrukking}\\

\haiku{daar zit alweer een, -?}{zusje meer wie zou me dat}{hebben gegeven}\\

\haiku{Zij gedroegen zich,.}{als meesters vroegen niet eerst}{en zeiden geen dank}\\

\haiku{{\textquoteleft}Weet je dan niet dat?}{het nest uit louter zulke}{balken gemaakt is}\\

\haiku{Wie zag het licht, wie,?}{de bergen wie de wouden}{vol geheimenis}\\

\haiku{Het gevaarlijke.}{wezen erkennen en voor}{de practijk negeeren}\\

\haiku{in Eva zou het niet.}{opkomen zich zelfstandig}{te laten gelden}\\

\haiku{Nog dieper viel de,.}{eenzaamheid om hem heen een}{regen van stilte}\\

\haiku{En vaster nam hij,.}{zich voor zijn exempel tot een}{wet te stellen}\\

\haiku{Zoo geviel het dat.}{alleen ouderen Adam's leer}{begrijpen konden}\\

\haiku{stammen over den stroom,.}{gesloten door lenige}{rieten verbonden}\\

\haiku{Zooals de arbeid, werd,;}{alras de jacht een doel om}{zichzelf nagestreefd}\\

\haiku{Heil de helden die!}{voluit gedijen kunnen}{naar het zonnelicht}\\

\haiku{Ik begrijp niets van,}{uw bedoeling en zie niets}{van uw heerlijkheid}\\

\haiku{{\textquoteright} zei Eefje nog eens,.}{en staarde in de leegte}{haar vraag achterna}\\

\haiku{Hij wenschte zeer.}{ernstig ziek te schijnen om}{te worden beklaagd}\\

\haiku{{\textquoteright} Nog eens, ijzig van,,.}{ontzetting gebood Adam hen}{alleen te laten}\\

\haiku{Uw oog, ten hemel,.}{gericht geeft de blinde voet}{aan de sprenkels sprijs}\\

\haiku{Men scheurde hem de,.}{kroon van het hoofd men wrijtte}{tegen zijn gezag}\\

\haiku{Wie het paradijs,.}{bezitten leven in vrees}{het te verliezen}\\

\haiku{Herinnert ge u,?}{Adam's wetten die van hemzelf}{hun oorsprong namen}\\

\haiku{- Wie nu haast heeft moet,,.}{weg gaan maar wie luisteren}{wil moet gaan zitten}\\

\haiku{Als het te koud is,.}{of te vuil of de bodem}{is droog geloopen}\\

\haiku{hij leeft, en het is.}{een hard lot de schuldenaar}{van een slaaf te zijn}\\

\haiku{- Het gezag doet er,.}{verkeerd aan die menschen hun}{gang te laten gaan}\\

\subsection{Uit: Het leven van Joost Welgemoed}

\haiku{De reep chocola,,.}{die bij zijn boterham was}{stak hij naar haar uit}\\

\haiku{Ze zou zeker niet.}{naar z'n raar hoedje kijken}{en wel naar het lint}\\

\haiku{Jooske meende dat.}{ze in dien snellen draai naar}{hem gekeken had}\\

\haiku{Hij meende alles,.}{te begrijpen al wist hij}{niet met woorden wat}\\

\haiku{Dat het scheen samen {\textquoteleft},,!}{te hangen met de plaat van}{ach vader niet meer}\\

\haiku{Joost was bang van die,.}{havelooze jongens en ook}{van dronken menschen}\\

\haiku{op het prentje greep:}{een groote vuist Klein Duimpje beet}{en hief hem hoog op}\\

\haiku{Joost vond de Duitschers,.}{schurken die in hem en zijn}{bedrog geloofden}\\

\haiku{Coba, de oudste,,:}{die het huishouden deed vond}{noodig Joost te zeggen}\\

\haiku{Den volgenden dag.}{kon hij bezonnen nog eens}{alles afspinnen}\\

\haiku{Weken later kwam.}{de kwitantie en Coba}{was razend geweest}\\

\haiku{Eens had hij vader {\textquoteleft}{\textquoteright}.}{naar de beteekenis van dat}{woordschande gevraagd}\\

\haiku{Hij zelf voelde zich.}{daar een beetje armoedig}{en verschoven bij}\\

\haiku{{\textquoteright} {\textquoteleft}Een schandaal om je,{\textquoteright}.}{te vertoonen voor alle}{menschen meende oom}\\

\haiku{Het leek of ook hij.}{behoefte had de schuld van}{zich af te praten}\\

\haiku{Niemand zou kunnen.}{zeggen dat de familie}{niet het hare deed}\\

\haiku{Jooske kon ook weer,.}{wat zeggen nogeens verder}{tasten met een vraag}\\

\haiku{Joost, zoo naar tante,.}{ziende uit zijn stilte werd}{daarvan iets gewaar}\\

\haiku{Joost had een tijd, een,.}{zeer langen tijd niet meer aan}{zijn vader gedacht}\\

\haiku{Daarom heb ik je, -?}{de eer aangedaan je op}{te merken vat je}\\

\haiku{{\textquoteleft}Als je zoet bent, mag{\textquoteright}.}{je een dans van me. Joost had}{geen dansen geleerd}\\

\haiku{Mijnheer Zandock, de,.}{eerste correspondent kwam}{even meelachen}\\

\haiku{Dan stelde hij zich.}{voor dat Lund en Boldering}{w\`el zouden durven}\\

\haiku{Joost, om zijn antwoord,.}{verlegen zag plotseling}{de situatie}\\

\haiku{Liever hief  hij.}{haar beeltenis in het licht}{zijner aanbidding}\\

\haiku{Smartelijk bedacht.}{hij weleens dat hij alle}{kosten alleen droeg}\\

\haiku{Daar hoort een heele.}{dosis energie en kennis}{en ervaring toe}\\

\haiku{{\textquoteright} Dreigend zag oom Joost,.}{aan toch niet zoo tevreden}{met zijn oratie}\\

\haiku{Handel is de kunst,.}{winst te maken om spoedig}{rijk te worden}\\

\haiku{De afnemer mag}{niet weten voor welken prijs}{zijn leverancier}\\

\haiku{Voor die alleen is.}{het leven de moeite van}{het leven waard}\\

\haiku{Het  gedurig.}{wachtend op den uitkijk staan}{sloopte zijn krachten}\\

\haiku{Oom zou er toch niet?}{over denken hem schriftelijk}{antwoord te geven}\\

\haiku{{\textquoteright} Dadelijk werd het,.}{een relletje waar Joost zich}{haastig uit redde}\\

\haiku{hij zocht zijn coup\'e,:}{terug waar een familie}{was komen zitten}\\

\haiku{Het vreemde land ving.}{zijn bestaan niet eerst aan met}{Joost Welgemoeds komst}\\

\haiku{In de breede straat,}{gingen weinig menschen meer}{of misschien dachten}\\

\haiku{Zijn ziel schaamde zich.}{en zijn mond plooide zich tot}{een schamper verweer}\\

\haiku{Gaf hij opnieuw zijn,?}{bestaan aan een waanbeeld een}{vorm zijner wenschen}\\

\haiku{Als een verloren,.}{iemand treuzelde hij nog}{een paar straten om}\\

\haiku{Het wonder van zijn.}{terugkeer verscheen levend}{voor zijn verbeelding}\\

\haiku{Hij zag zichzelf in,;}{dit tafreel vergevend in}{zijn grootmoedigheid}\\

\haiku{{\textquoteright} Een goed gehumeurd.}{spotje klaterde na in}{zijn lachende keel}\\

\haiku{Hij legde de hand.}{op den koperen knop om}{de deur te sluiten}\\

\haiku{De huisknecht poetste.}{in zwijgenden ijver aan}{zijn schoenen verder}\\

\haiku{{\textquoteright} 'n Fregat van 'n,.}{vrouw  en alles correct}{van beide kanten}\\

\haiku{Voor het eerst was het.}{wijde huis zijner ziel zijn}{eigen eigendom}\\

\haiku{Het ego{\"\i}sme  .}{zou van de menschen vallen}{als een leelijk kleed}\\

\haiku{Dan sprong Joost omhoog,,!}{uit zijn stoel hij speelde den}{flinkerd den frisscherd}\\

\haiku{Maar het gelaten.}{antwoord op zijn bezorgde}{vragen ontkende}\\

\haiku{In ieder geval,.}{zou het in orde komen}{stelde hij gerust}\\

\haiku{Zij demonstreerden.}{in optochten als cijfers}{in een optelsom}\\

\haiku{Je bent met ons niet.}{gelukkig omdat je thuis}{niet gelukkig bent}\\

\haiku{Ontmoedigd, zette, -.}{Joost den kleine neer hij moest}{maar naar z'n bedje}\\

\haiku{Hij kreeg een aanhang;}{van lachers en negatief}{gerichte geesten}\\

\haiku{Ze konden voor zijn.}{part een zaaltje huren en}{daar samenspannen}\\

\haiku{Den langen lieven.}{dag praatten zij erover en}{kwamen tot geen eind}\\

\haiku{Moe, verschuchterd, bleef, -.}{hij in zijn honk opgejaagd}{beest in zijn leger}\\

\haiku{{\textquoteright} Glimlachend zette.}{de vreemde zich over Joost's}{vijandigheid heen}\\

\haiku{Wat verlangt iemand,?}{terug die zich de moeite}{geeft mij te helpen}\\

\haiku{Tot mijn spijt,{\textquoteright} zeide, {\textquoteleft}.}{hijkan ik u niet te veel}{inlichting geven}\\

\haiku{Ze mogen den muur.}{niet meer zien waarbinnen zij}{opgesloten zijn}\\

\haiku{Alles zagen zij,.}{en toch waren die menschen}{niet  gelukkig}\\

\haiku{Maar ook dit kon hun,.}{honger niet verzadigen}{hun dorst niet lesschen}\\

\haiku{Wanneer zij kozen,!}{hoevele malen kozen}{zij het onrechte}\\

\haiku{In het bosch hieuwen,}{zij de opene plek van de}{gevelde stammen}\\

\haiku{Doch velen hunner;}{verdroegen op den duur die}{sterke dosis niet}\\

\haiku{John Hurst grijnsde,:}{gooide in de kranten den}{satanischen spot}\\

\haiku{ter  beurze stond {\textquoteleft}{\textquoteright}.}{het fondsJoost Welgemoed niet}{ongunstig bekend}\\

\haiku{Hij zou de zaal doen,,.}{sidderen van zijn kracht zijn}{verachting zijn hoon}\\

\haiku{De handen, in de,;}{mofzakken van zijn duffel}{grepen en wrongen}\\

\haiku{Daar strompelden ze...}{met geweld omhoog op hun}{stampende pooten}\\

\haiku{De kranten maakten,.}{zijn rekening op trokken}{het passief saldo}\\

\subsection{Uit: Tweestroomenland}

\haiku{Verder gaande op,,:}{mijn tocht dien middag had ik}{het veilig gevoel}\\

\haiku{ik had mijn schrijfblok,.}{op de tafel voor mij de}{vulpen afgeschroefd}\\

\haiku{zich honderdvoudig,, -.}{ik raakte den tel kwijt was}{niet gereed jij ging}\\

\haiku{Ik was waarachtig,,,,}{mooi interessant ik was}{zelfs zeide mij een}\\

\haiku{Weldra volgden dan.}{ook lauwe thee en verdacht}{besmeerde crackers}\\

\haiku{Hoe ik het had, wat,,.}{ik deed hoe ik voelde wat}{mijn plannen waren}\\

\haiku{hij bewondert dat,.}{volle rijke blond het doet}{hem zinnelijk aan}\\

\haiku{Je lipje verborg,,}{de teleurstelling niet maar}{het was toch feest}\\

\haiku{Boven de dunne, -}{ijzeren doorgang die geen}{poort had men liep daar}\\

\haiku{verbeterde jij,.}{een beetje weterig want}{het was namiddag}\\

\haiku{Meteen begrijpend,.}{hoe wreed je was begon je}{zwaar toe te lichten}\\

\haiku{Het werd een heerlijk.}{samenzijn en van Olga}{spraken wij niet meer}\\

\haiku{Zonder opstand, maar.}{ook zonder verlangen denk}{ik daaraan terug}\\

\haiku{Zeker, sportief, frisch,,.}{jong en vroolijk was ze een}{levenslustig kind}\\

\haiku{- Doe ik toch! - Tante () '?}{ze zei Tante vind je mijn}{sweater nietn snoes}\\

\haiku{In onze warme -?}{huiskamer staat hun kasje}{zijn ze niet zeldzaam}\\

\haiku{Misschien, al haastte,.}{ik mij zou ik haar alreeds}{gestorven vinden}\\

\haiku{Zijn geheugen was -?}{verzwakt dat van de oude}{doktersdame ook}\\

\haiku{Hij had gedaan, wat,.}{hij kon naar beide kanten}{was hij verantwoord}\\

\subsection{Uit: De verlaten man}

\haiku{Men vond daarin te,,.}{loven men vond erin te}{laken zooals dat gaat}\\

\haiku{Ook niet toen... ook niet.}{toen d\`at had opgehouden}{mij verdriet te doen}\\

\haiku{Gerard van Overen.}{kwam mij de handen drukken}{als krentebollen}\\

\haiku{Vroeger, komt het mij, {\textquoteleft}{\textquoteright}.}{voor zou dateen pijnlijke}{vraag zijn geweest}\\

\haiku{Neen, eigenlijk bent.}{u er de man niet naar om}{ouderwetsch te zijn}\\

\haiku{Alle intellect.}{is naar het onderst van de}{ruggegraat gezakt}\\

\haiku{- Ik moet u mijn van,.}{Goghs laten zien u gaat}{door voor een kenner}\\

\haiku{Muzikaal, ze zingt,,.}{dertig jaar zal ze zijn en}{niet onvermogend}\\

\haiku{Ik ben al een half,.}{uur met mijn ziel bij jou wie}{weet waar ik straks ben}\\

\haiku{de narigheid zit,, {\textquoteleft}{\textquoteright}, -:}{geloof ik ingeachte}{zonder franje dan}\\

\haiku{Dank u. Zij weet haar,.}{houding niet hamert maar wat}{los op de toetsen}\\

\haiku{- spring ik uit mijn stoel,.}{treed met deftige stappen}{Blok's kamer binnen}\\

\haiku{En ik weet ook, dat.}{ik niet meer zooveel lust heb}{naar Itali\"e te gaan}\\

\haiku{Kan ik heengaan en:}{het meisje bij mij laten}{komen en zeggen}\\

\haiku{Thans kijk ik kalm daar.}{op toe als een geleerde}{op zijn preparaat}\\

\haiku{wij nemen er nog,.}{een onder gelijken die}{den rookbak vullen}\\

\haiku{iemand op wien elk,}{land trotsch zou mogen zijn hij}{is een man met wien}\\

\haiku{Mijn groot, hoog venster.}{staat vlak voor de hooge gele}{boomen van den tuin}\\

\haiku{Jeanne's aandacht.}{blijft zich scherpen daar in de}{buurt van het portret}\\

\haiku{- Ik kan begrijpen,.}{dat je er tegen op ziet}{die te verlaten}\\

\haiku{Wat wilde dit sterk,,,,!}{zijn zoo doelvast zoo vrij zoo}{geleerd zoo zeker}\\

\haiku{En een gedachte,,:}{volgend waar ik geen deel in}{heb zegt zij ineens}\\

\haiku{nu heb ik toch mijn...}{hand terug genomen en}{zij bemerkt het niet}\\

\haiku{De trap af, zwijgend,.}{geef ik haar tot de straatdeur}{mijn geleide}\\

\haiku{Mijn oude drift om,,.}{te vermooien idealen}{te zien is niet dood}\\

\haiku{Maar, lieve jongen,,.}{ik heb geen soldo meer geen}{soldo zeg ik je}\\

\haiku{Bij Maarten trok ik.}{gummi handschoentjes aan voor}{al het grove werk}\\

\haiku{Zoo is zij naar haar,.}{bestemming zoo ging het mij}{naar mijn bestemming}\\

\haiku{indien zij keerde,,!?}{op dezen oogenblik hoe}{zwak zou ik weer zijn}\\

\haiku{zij staan groot boven.}{de menschen in haar lange}{rijzige lijven}\\

\haiku{Zeer langen tijd zijn.}{alle kantoorgeluiden}{wat zij moeten zijn}\\

\haiku{- Philine, als je,,,.}{zoo doet niets zegt niets verklaart}{ga ik de deur uit}\\

\haiku{De handen wrijvend,,.}{dat ze zingen dribbelt hij}{op z'n dikke beenen}\\

\haiku{Smaragdgroen schuimen.}{de kruinen in het kobalt}{der hemeldiepten}\\

\haiku{Het lijkt of allen,.}{hun plek gekozen hebben}{maar de plek koos hen}\\

\haiku{Bovendien - neen, zij,.}{is niet van het soort dat men}{behoeft te ontzien}\\

\haiku{Tja, ik zie het wel,,,.}{hij wil uitdagen vechten}{zijn kracht beproeven}\\

\haiku{- Waarom, Tonio? - Om,.}{t\`och zegt Tonio en keert zich}{onhebbelijk af}\\

\haiku{Haar opbellen, een,,?}{briefje schrijven bezoeken}{haar noodigen bij mij}\\

\haiku{Verlang ik toch weer?}{naar de bekoring harer}{gezamenlijkheid}\\

\haiku{Weer plooit haar mond zich,.}{dun en zuinig als aan \'e\'en}{kant wreed getrokken}\\

\haiku{Hoop die geen kans op, -.}{verwezenlijking heeft maar}{eenmaal dan toch hoop}\\

\section{Max de Bruin, Eug\`ene Coehorst, Paul C.H. van der Goor, Jan Notten en Lou Spronck}

\subsection{Uit: Mosalect. Bloemlezing uit de Limburgse dialectliteratuur}

\haiku{welke criteria,?}{leggen we aan wat nemen}{we op en wat niet}\\

\haiku{Het sloog krek drie oere, ':}{in Staiveswji\`ert wie}{t vriek\`or aafmersjeerde}\\

\haiku{En es toe deze ' '.}{kie\"er neet int prezong}{k\"oms weit icht neet}\\

\haiku{Waat is hie te doon,,?}{vroog ter det geer den hemel}{zwoa geweldj aan dootj}\\

\haiku{Dao zoot van alles....}{innet veldj en de b\"os t\`ot}{waerw\"olf toe}\\

\haiku{E\"es leef zeen en dan,.}{temptere Drek tr\"okpakke}{wat ze e\"es gaeve}\\

\haiku{Leefde is get van,,!}{twe\"e luuj Neet van eine is}{leefde Van twe\"e}\\

\haiku{Leefde is get van,,.}{twe\"e luuj Neet van eine van}{twe\"e is leefde}\\

\haiku{Zwart den auto, 't, '.}{kleed ovver zien kist de minse}{diem w\`egbrochte}\\

\haiku{{\textquoteright} En d'r Pierre zag, '.}{gans bedrufd datt met d'r}{pap neet ez\`o good g\`ong}\\

\haiku{zie werrek oach waal '. '}{ns aangesjoa\"ete of}{zats Vriedes zow}\\

\haiku{de begreffenis}{zie\"e um ellef oe\"er want}{um tie\"en oe\"er}\\

\haiku{En went mich ouwesj...}{zage ku\"e- me dat}{h\"on kinger sjturve}\\

\haiku{De heemkier van d'r}{verloare zoe\"en}{Heerlens  He\"e hool}\\

\haiku{En wie op l\'ochte}{k\'olke dreef blinjelings zie}{verlange wir durch}\\

\haiku{Op ins doa huer...}{iech jet hinger miech roespere}{en sjnoespere}\\

\haiku{Opins doa huer...}{iech jet hinger miech roespere}{en sjnoespere}\\

\haiku{'nne remmel \'onger ' '.}{nnen erm enn sjmeel teussen}{\'ongerlup en sjn\'or}\\

\haiku{De muskes*~		 duun.}{d'r moier zinge en de}{schorsteen \^anders hule}\\

\haiku{alweer ei gedich.}{in ech en \'onverzawte}{Limburger dialek}\\

\haiku{ze ginge l\^egge '...}{En de j\^onges v\"osjde in}{t Gelei ~ Mer}\\

\haiku{{\textquoteright} {\textquoteleft}Auwat, iech bin doch ',.}{mar alling mit der man en}{t kink wat sjlie\"eft}\\

\haiku{{\textquoteleft}Dat is va wie iech;}{dizze n\'ommedieg plantse}{ha oes-jesjtaapelt}\\

\haiku{En wie iech zaat - iech -}{how al allerhank jedoa}{det iech d\'at nit mie\"e}\\

\haiku{e kriet de zieng{\textquoteright} en '.}{j\'ong noan kirch i en zats}{ziech in de sjtalles}\\

\haiku{{\textquoteleft}Das ist der Protest;}{des Himmels f\"ur den Tyrannen}{der Chorknaben}\\

\haiku{sjuvet-e jans}{jemuutliech de dr\"o\"aet woa-e}{knoake en hals uvver}\\

\haiku{Jedemfals vroaret,,}{der pap \'os wie vier heem koame}{of der ze\"eje}\\

\haiku{h\"uet em et kink r\"oft!}{em Bevreij mich en sjpie\"el}{mit mich voe\"egel}\\

\haiku{t Sjuntse voong iech ' '.}{t sjtuk woa me op de naat}{ant waade is}\\

\haiku{Doatussje durch.}{is nog ing verloting mit}{sie\"er sjun prieze}\\

\haiku{op de weasjdr\"oa\"ed,!}{Jraad boaver mie hemme}{vilt mienne blik}\\

\haiku{Soe eine gooije}{sul moot waul meijne dat het}{Mastreegs get slegts is}\\

\haiku{H\"obstoe ins tie koeraazje,.}{Dan krijg tien kakedoe Ouch}{fris get op zien oere}\\

\haiku{Wee admireert neet,?}{hun braaun aaugen Hun sniewit}{velke wie satien}\\

\haiku{Aaug vindt me ter van,}{alle soorten En heet me}{ze wie dat me wilt}\\

\haiku{Te willen speulen;}{een presentie En onder}{d'aaugen van de maan}\\

\haiku{Van naau de mes wat,}{veul te goon En liever een}{de kerk te zitten}\\

\haiku{Anneke Scheuffers;}{is gestorven En met heur}{aaug tin aauwen tied}\\

\haiku{Ig bin altied vol,}{helgen iever En dikwils}{heub ig nog min daaug}\\

\haiku{De luy zien nou en,;}{daan zou aaurdig En t is aaug}{deks neet umme zus}\\

\haiku{De bron wou me zich;}{een kaan zuvren Wie voul en}{zwart tat men aug is}\\

\haiku{Veel mig toch ins ein,;}{out tin hiemel Ze kwaaum mig}{excellent te pas}\\

\haiku{Ig moes in huiske;}{bouten heubben Modest en}{propel gemeubleerd}\\

\haiku{Dou heubst te veul en;}{ig te min. D'ambitie zou}{mig aug neet kwellen}\\

\haiku{De witte greemsels;}{bleven likgen En reurden}{zich neet een de heuk}\\

\haiku{De baaum wet neet wee,;}{dat em plantde Et blaad wet}{neet wou dat et weyt}\\

\haiku{Us altied haauwen,.}{by et Good Op tat ver stil}{en zaaulig sterven}\\

\haiku{m\`et de eterneel*~		 :}{strikkous en heer sjreef in}{groete l\`etters}\\

\haiku{d'n ingaank van et {\textquoteleft}}{str\"a\"otsje ene paol en dao}{leunde noe Zjann\`et}\\

\haiku{Koelek had er evels,}{zene mond opegedoon of}{Janneske st\'ont veur}\\

\haiku{{\textquoteleft}zoene lieleke......{\textquoteright}, {\textquoteleft}{\textquoteright}.}{lieleke ze k\'os gei woord}{lielek gen\'og vinde}\\

\haiku{als een ei 3,5 cent,.}{kost hoeveel verdient dan wel}{die kip in een jaar}\\

\haiku{zoe vas opein, tot.}{zene mond achter de punt}{van z'n neus zaot}\\

\haiku{andere sloog ze.}{in de gawwigheid ene}{knien in zene nak}\\

\haiku{d'n onderein, um.}{ziech evekes te verpoeze}{van et dr\"ok get\'offel}\\

\haiku{heij z\`ette, daan '.}{g\'ong er kallek hoole en bleef}{miechn haaf oor eweg}\\

\haiku{{\textquoteright} - {\textquoteleft}Dan mooste miech ouch,.}{mer medein de sj\`elderije}{ophange Stiene}\\

\haiku{Iech waor ouch aon '.}{t witte wie noe en dao}{loerde dee sjijns op}\\

\haiku{Versjeije politieke,,,;}{haone Die zien ocherrem}{ouch mer maone}\\

\haiku{Dan geit 'ne sjt\'ok*~		 , '. '}{door aal wat leef Ent hart}{van blij emotie beef}\\

\haiku{Oraanje leech stijg;}{langs de gevel en sjittert}{in de hoegste roet}\\

\haiku{'n Ierste blaad luut los;}{en kantelt en r\"os veurgood}{op kawwe stein}\\

\haiku{Heer kraog middesijn,.}{en m\^os goon ligke en moch}{gei beer mie drinke}\\

\haiku{sjoen st\"ok preuf en k\"a\"ort.}{in ze gehiel en in zen}{apaarte motieve}\\

\haiku{{\textquoteright} Et \'onweer leet nao,.}{dee slaag neet aof ieder woort}{et nog erreger}\\

\haiku{Heer blaosde en '.}{t leve Dat greuide in}{bieste en plante}\\

\haiku{Meh es iech miech neet '.}{t'rin vergis Daan waort}{veur alle twie}\\

\haiku{{\textquoteright} infermeerde Jaan, ' '.}{ne keerel wiene boum en veur}{ter duuvel neet bang}\\

\haiku{er waor taamelek.}{zwoer en kleddere veel em}{neet toe erreg m\`et}\\

\haiku{Toen b\'okde-n-er.}{zech en voolt er tot er de}{kop in zen hand heel}\\

\haiku{Wie toen 'n {\textquoteleft}uigske{\textquoteright}...,...?}{ene glimlach oet deeg springe}{Wat toen geb\"a\"orde}\\

\haiku{zoe versjrikkelik es,}{mennigein dee nao d'n oos}{is goon vare neet}\\

\haiku{de s\`okkerbekker, ' '}{vaan de stad dee doort gans}{Zuid-Limburgs land}\\

\haiku{Dao passeerde v'r {\textquoteleft}'{\textquoteright},}{t Drifke dat altied eve}{levend ze water}\\

\haiku{De vroului pakde}{hun kedokes oet en de}{kerels droonke ziech}\\

\haiku{{\textquoteright} en inpessant leep, '.}{heer de deur oet umt keend}{te goon aongeve}\\

\haiku{Heij beg\^os Marie:}{weer haos obbenuits te}{kriete wie ze zag}\\

\haiku{Zen start t\"osse zen,.}{pu gedreve Sl\"op heer nao}{hoes tou met belump}\\

\haiku{Iech zaot al in!}{de twiede klas en Mia g\'ong}{al nao de veerde}\\

\haiku{Iech weet neet wat m\`et '.}{m'ch geb\"a\"ord is en iech h\"ob}{t noets gewete}\\

\haiku{Wie mie pijn tot 't.}{deeg wie beter tot iech m'ch}{beg\'os te veule}\\

\haiku{{\textquoteright} Iech keek verweze.}{op oet mie book en deeg of}{iech vaan Sint Jaan kaom}\\

\haiku{{\textquoteright} Iech doog de sjaw d'r, ':}{op en wis neet wat iecht}{ergste zouw vinde}\\

\haiku{Alle sjtel haw hae,.}{vol vieje zitte en de}{weije leepe vol deere}\\

\haiku{Dat is 't biste,.}{oonder famille allein}{kirremis hawte}\\

\haiku{Hae dach, 't zunt jong.}{luuj en hae deeg dat altied}{nao de kirremis}\\

\haiku{Och, doe moos weete, ', '.}{waet vieje haat kin ooch}{t vel verwachte}\\

\haiku{Job 'ns \'a reegel '.}{kristeliere enm d'r}{kee\"etel sjoore}\\

\haiku{Dae vergit neet, dae '.}{haett neet oonder de sjoon}{gesjrie\"eve}\\

\haiku{Slivvenhier dae sjat ',*.}{dich opn hawf oons dae knoept}{sjuus op d'r cent}\\

\haiku{Doe waors jao sjterk,,.}{wiej twieje sjterk wiej mosterd}{doe reets al biejeen}\\

\haiku{Doe wits toch, Job, dao:}{zunt driej dinger diej neet zunt}{te besjrieve}\\

\haiku{Is 't 'ne rieke, ' ',}{of istne erreme}{Dae in de hil k\"omp}\\

\haiku{Doew sjarde de aw ' ':}{kloek get drek int loo\"ek}{en zong zet sjlot}\\

\haiku{Ja, wae zaet 't. D'r.}{weend doog nog neet dae van d'r}{Waalekaant k\"omp}\\

\haiku{Kaakele is niks,,.}{kaakele kint eekereen}{mae eijer legke}\\

\haiku{haw naogesjikt es.}{dae zich te boete g\'ong en}{neit loestere w\'ol}\\

\haiku{Dao waore de.}{luuj oet den \'omtrek in de}{Krisnach haer komme}\\

\haiku{{\textquoteright} Sevrien kos niks meei.}{z\^egge en pakde Drikus}{mit einen erm vast}\\

\haiku{Hae sjtamelde get {\textquoteleft}?}{t\`egen Sevrien vanWat mot}{ich mit dat dink doon}\\

\haiku{in 't sjwart, mager,.}{en bleik koum d'r oet en ging}{dao nao gen hoes in}\\

\haiku{En d'r Hoeb\`e\`er ' '.}{kiektns \"a\"overn sjowwer}{g'n sjtraot aaf}\\

\haiku{Es Sliekkop miens waas, '...}{gewaes z\`ow haet inne}{politiek zeker}\\

\haiku{Sliekkop zitj oppe ',.}{hoogste plaats vant eilandj}{mit dieke klaagkael}\\

\haiku{Morgevreug gaon{\textquoteright},.}{v'r hae str\`ek zien vol pens oet}{en geit slaope}\\

\haiku{Oetgevlooktj,*}{zatte zich inne vaarleis}{en wajje}\\

\haiku{Ae zaat kwaolik*:}{of ein zwelped~		 kroop doort}{st\"ortgaat en zawgt}\\

\haiku{En gaeftj mich n\'ow,{\textquoteright}.}{eur zie\"el s\'onger van eur}{schatte te preuve}\\

\haiku{ae m\`et de vlegel,.}{det bi-j edere slaag ein}{krej oet de lougt veel}\\

\haiku{Kos-ter waal.}{hondert kriege En waor}{toch no\`ets kontent}\\

\haiku{Hae keek nog ins om, '.}{schaterde vant lachen}{en sneei zijn deur in}\\

\haiku{Veer sproken neet, veer,...}{zongen neet Veer hadden dao}{toe ouch gein raeijen}\\

\haiku{Toen w\`ol Oswald auch}{zie vader op dezelfde}{meneer oet d'n tied}\\

\haiku{En haaj doe de moel,{\textquoteright}*.}{doe b\"os nog te zeer eine}{gaaplaepel}\\

\haiku{Waat sjpeelt ei kiendj d\"ok, ', '.}{door de kop Zo ist braaf}{zo ist sjtout}\\

\haiku{{\textquoteright} Rrrroebedoebs, '.}{de trap op aan de k\`erk}{sjuut weern wej}\\

\haiku{*~		 taege, en noe ' '!}{maakde mich det uulskuuke van}{ne Wulmne sjlaag}\\

\haiku{- Allaa j\'onges, 'ne!}{sjteevige gedr\'onke op}{de goojen oetsjlaag}\\

\haiku{Hae zelf waar 't meis.}{in de wolke en sjpr\'ong wie}{ei j\'onk veule r\'ondj}\\

\haiku{En h\"obs-toe nog?}{waal ins geloesterd nao Roussels}{veerde symfonie}\\

\haiku{Alling de vrouw van ',.}{d'r Joe\"ehan witt ma die}{sjwiegt wie inne dup}\\

\haiku{{\textquoteright} {\textquoteleft}D'r gantse daag in ',?}{d'r winkel vant Lieske}{woa kals doe u\"uver}\\

\haiku{{\textquoteleft}Ich be Pi\`ejr}{va Fien Dabbs oet Geb\"osjelke en}{kom vraoge of ich}\\

\haiku{Waal veel ich d\`ek i,,.}{sjlaop meh dat kaom omtot Mam}{zoea dru\"ag veurb\`ede}\\

\haiku{En alles waat \`os.}{nog naobie waar Woord lankzaam}{vraemd en bleik en vaag}\\

\haiku{{\textquoteright} - dan waar ze weier '.}{gans Calypso en veilm}{in sjt\`orm in de erm}\\

\haiku{Mer hae maagde dat., {\textquoteleft}{\textquoteright},.}{In wirkelikheidau fond}{lachde hae mit h\"o\"or}\\

\haiku{en noe zuut me auch:}{al weier dat et is wie}{et gez\`ekde zaet}\\

\haiku{God bewaar, es et,.}{mer gein \`onwaer geef ee dat}{ze weier dao zeen}\\

\haiku{'t Haet niks eweg van,.}{Nikela zeliger God}{gaef h'm d'n hemel}\\

\haiku{nog riepe proeme.}{gegaef en noots h\"ob ich mit}{mien naober get ghad}\\

\haiku{De Richter waor.}{eine Mastreichtenaer en hoelj}{ens gaer van ein grap}\\

\haiku{H.D.        Mieng sj\"onste.}{sjiech   Spekholzerheids D'r}{Sjtaat tuutet tsing vuur zes}\\

\haiku{Hae waar al ens nao,.}{B\`ellevend gefietst maar dao}{waar niks te vinge}\\

\haiku{In de kortste tied.}{ware weej in Kastele}{beej de pastorie}\\

\haiku{{\textquoteright} - Eur houegmood steeg ten,, '}{top Es geer o minst gel\"ok}{haatj om te melje}\\

\haiku{{\textquoteright}, zag Berb, {\textquoteleft}kuns ies um '}{de wesj dru\"eg te kriege}{est de gansen}\\

\haiku{Harie vertrok en ',}{oft zoa moes zeen trof hee}{in de sjtraot}\\

\haiku{De beugel waor,.}{verros de gum van den dop}{bienao vergange}\\

\haiku{Obb\`ens geit de deur ':}{aop en kump de knech vant}{Kraneveld binne}\\

\haiku{Ens op ennen daag.}{had d'r zich eemus z\'on paar}{wanne weggepak}\\

\haiku{Mien Meelkop        Storm,;}{Venloos  zwart wu\"ert}{de l\'och kaolzwart}\\

\haiku{De z\^on leet nog in.}{de wol en de waereld zuut}{oet wie beschummeld}\\

\haiku{{\textquoteleft}Waat is d'r now weer{\textquoteright}, '.}{vru\"eg de vrouw en geit zich}{naev'nm zitte}\\

\haiku{Weer wiegelde de.}{wage met eine nieje}{zenuwschoek wiejer}\\

\haiku{{\textquoteright} De deur die al half, '.}{toe waar knalde met enne}{slaag int slaot}\\

\haiku{Petran was ennen,.}{ieverigen boer den zien}{land goed verz\"orgde}\\

\haiku{, mer Petran verstond, {\textquoteleft}{\textquoteright}.}{mer hallef wat ze zaei en}{riep d\"orrum merjao}\\

\haiku{Op \`ens draejde, '.}{de weg zich en jaowel door}{h\'addet kruuspunt}\\

\haiku{Petran was van d'n}{ie\"ene kant b\'ang en van}{d'n \^andere kant}\\

\haiku{Hej kreeg d'r zin ien, '.}{want ie haj enen dru\"ege}{k\`el vant snurke}\\

\haiku{\`en vur dat ie 't,.}{eiges ien de gate haj}{haj ie ze al \`op}\\

\haiku{{\textquoteleft}Stopt er \`owen b\`ok ien,{\textquoteright}, {\textquoteleft}.}{Pier zei Toe\"enik zal um}{ow goed betale}\\

\haiku{Der Pitter zaat n\"uks,:}{e bezoog der Joep ins es}{wente zage wool}\\

\haiku{{\textquoteleft}E kamp os verdaat,.}{v\"or e kiekt graat of wente}{kaffie sjmoekelet}\\

\haiku{Da zalle zage,:}{wente v\`e\"edig is mit}{diej va Ie\"epe}\\

\haiku{Herregod iggen,{\textquoteright}.}{himmel laot mich sjtil}{en geduldig zie\"e}\\

\haiku{Pastoe\"er ka gee...}{wo\`ad mie\"e oetbringe en dao}{in ins ene sjlaag}\\

\haiku{Noe geet 't op en '.}{aaf uvvern heen en probeere}{ze get te versjto\`a}\\

\haiku{V\"a\"or mees te lane}{hoft me ging sjl\`e\`eg toe te}{do\`a en da hat me}\\

\haiku{gank e-weg, en,.}{dan i zie hats nit twiefelt}{d\`e wet verhoe\"ed}\\

\haiku{Diej zonge \`eve hel ' '.}{naot oksaal truk en ezu}{gongt hin en weer}\\

\haiku{In der aavank ersjaffet.}{God himmel en \`e\`ed en e}{maket alles good}\\

\haiku{Zoste dao kunne,,?}{pr\`edige Barthelomee in}{de Sint Servaos}\\

\haiku{{\textquoteleft}al kost 't os twei,{\textquoteright}.}{doezend gulde v\`er zulle}{dat offer bringe}\\

\haiku{{\textquoteright} H\`e\"er Bussjep,, '!}{went d\`er ins wust wat inn}{zie\"el kan umgo\`a}\\

\haiku{Noe is 't 'n koo,.}{diej miskoft dan e p\`e\"ed}{dat vervangen is}\\

\haiku{de drukdje, 't, '.}{zinge de mes en alt}{licht in de vol kerk}\\

\haiku{Zoea sjuufse wie 'ne,}{sji\"em van i\"elenj langs de}{straot Woea d'anger}\\

\haiku{Et Wimke wreef zich:}{van sjpas in gen heng en}{der melder fluidde}\\

\haiku{handschrift in bezit, ();}{van Gilles Jaspars Gronsveld}{Gilles Jaspars1937}\\

\haiku{Z.p., z.j. Blz. 16-18 () (-);}{Spelling herzien door R. Geurts}{Jos Wetzels18971967}\\

\haiku{dikke ijzeren ():}{staaf om gaten te maken}{in de grond sjt\"odig}\\

\section{Johan de Brune (de Jonge)}

\subsection{Uit: Wetsteen der vernuften}

\haiku{Want het geeft hun een,.}{kalm zelfvertrouwen dat hun}{vorming vervolmaakt}\\

\haiku{Mensen die in het}{spel volleerd dachten te zijn}{overwon hij voor ze}\\

\haiku{Het inwendige,,.}{van apen zegt hij is gelijk}{aan dat van mensen}\\

\haiku{Vonnis dat men bij.}{Papon leest tegen hen die}{aan die kwaal lijden}\\

\haiku{Elke zonde die,,.}{een mens begaat zegt hij is}{buiten het lichaam}\\

\haiku{Maar in plaats van te:}{antwoorden zei hij tegen}{zijn opdrachtgeefster}\\

\haiku{En hij is bang voor.}{blauwe plekken onder de}{druk van zijn handen}\\

\haiku{Gekscherend vroeg ik?}{of iemand niet verliefd zou}{worden op zo'n beeld}\\

\haiku{Ziedaar lezer, ik.}{eindig een beetje anders}{dan ik gedacht had}\\

\haiku{Voor droomuitleggers.}{betekenen parels niets}{anders dan tranen}\\

\haiku{Michelangelo.}{vroeg de man hoe hij in zijn}{onderhoud voorzag}\\

\haiku{Geestig verhaal over.}{Demetrius Cynicus}{en een dansmeester}\\

\haiku{Van wateren die.}{in steen veranderen wat}{erin wordt gegooid}\\

\haiku{De Brune stijgt in.}{zijn dichtwerk niet boven het}{gemiddelde uit}\\

\haiku{Het proza van Jan;}{de Brune is vertaald in}{modern Nederlands}\\

\section{M.J. Brusse}

\subsection{Uit: Boefje}

\haiku{Dat zag je wel meer, - - '...}{dacht ze zacht voor zich zelf dat}{t dan slecht afliep}\\

\haiku{{\textquoteright} - {\textquoteleft}Och, werom nou, 'k...{\textquoteright} {\textquoteleft} '!}{doe je nou toch ommers niks}{D'r isn meheer}\\

\haiku{Z'n breede schouders;}{stonden wat naar voren als}{om zwaar te duwen}\\

\haiku{- Moe, geef Lientje 'n,...}{happie suiker ze het weer}{pijn in d'r kiesie}\\

\haiku{{\textquoteright} - zei die klabbak - {\textquoteleft}bij ',.}{je moer int secreet daar}{kan je ze scheppe}\\

\haiku{daar heije de meester!}{en da's die smeris die me}{laast het opgebrocht}\\

\haiku{{\textquoteright} - {\textquoteleft}Nou, 'k denk, ik ga, '.}{an de stroomtram hange dan}{benk er gauwer}\\

\haiku{Nou, d'r sting net 'n,:}{man  voor en toe had Jan}{zoo leukweg gezeid}\\

\haiku{Als die ouwe je,:}{pakte sloeg ie je raak op}{je tweede gezicht}\\

\haiku{Doch ie dat ie 't?}{fernuisie nogeris weer}{voor d'r zou poese}\\

\haiku{Maar Pukkie mos ook...}{altijd met bossies houtjes}{voor z'n moeder loope}\\

\haiku{{\textquoteleft}k\`o, snotneus, ga na, ';}{je moeder en vraag omn}{kouwe aardappel}\\

\haiku{- Fort, wat let me, of '...{\textquoteright}}{k schop je na Schiedam met}{je schele klosooge}\\

\haiku{- Maar 'n oogenblik,}{later belde nie weer want}{ie had stilletjes}\\

\haiku{{\textquoteleft}Gewerkt... de heele...!}{nacht bij de boere gewerkt}{en c\`ente verdiend}\\

\haiku{Ze moeder zei nie',...}{veel maar Onze Lieve Heer}{hoorde d'r bromme}\\

\haiku{Als 'n kind dat bang,;}{is stond ie klein dicht tegen}{z'n vader aan}\\

\haiku{Daar hadden ze uit ';}{een rijk huisn dienstmeid zien}{gaan om een boodschap}\\

\haiku{{\textquoteright} en hij was zoetjes ',...}{t portaal door de kamer}{binnengeslopen}\\

\haiku{en op z'n gezicht '.}{n zweem van narigheid t\`och}{onbillijk vinden}\\

\haiku{{\textquoteright} - Jawel, dat dorst dat;}{lieve Jantje tegen ze}{m\'oeder te zegge}\\

\haiku{s Middags was tie.}{effe na bove komme}{schreeuwe om ze brood}\\

\haiku{En toen 't rookte, ':}{tikte hij technisch metn}{vinger an z'n pet}\\

\haiku{hij 's van de trap...{\textquoteright} {\textquoteleft};}{gevalle en toe moeder}{kwam kijkeJawel}\\

\haiku{Want de st\`ad, met al,...}{die verleiding is de hel}{voor zoo'n degener\'e}\\

\haiku{lekkere dikke ',:}{scholletjes voorm en dan}{mot Lientje zegge}\\

\haiku{Nou, zij kon ook wee;}{van de zenuwe worde}{as ze d'r an docht}\\

\haiku{Hij zal zoo in z'n, ',...}{knolle wezet kind as}{ie z'n vader ziet}\\

\haiku{Even keek ie wat d\`at,,.}{nou weer zijn zou schuwe oogen}{dadelijk weer neer}\\

\haiku{En die jonge waar...}{we mee zitte is nog veel}{gemeender dan ik}\\

\haiku{Die andere vent, ', '.}{waark mee zit het gegriend}{toe iet hoorde}\\

\haiku{{\textquoteright} Och heire chut, d\`at;}{was nou altaad de r\`amp van}{d'r leife cheweist}\\

\haiku{ba meheir Dussert,, '.}{en \'alles four moe hour chein}{cintje voorm self}\\

\haiku{dan sting ie in 't, '...}{pikke donker zouwen ze}{opm afkomme}\\

\haiku{maar de Heilige,,.}{Maagd die kon alles hadde}{de manne gezeid}\\

\haiku{En dat vliegie op...}{de plank voor ze bankie mog}{ie nou niet vange}\\

\haiku{Toen Boefje uit de, '.}{gevangenis kwam wast}{een gesl\'agen kind}\\

\haiku{En 't meiske sloeg ', ':}{r armpjes om z'n hals en}{trokm naar zich toe}\\

\haiku{dat je zoo'n kind nou,,;}{toch maar most late gaan h\`e}{as ouwer zijnde}\\

\haiku{Ja, 't was de goeie,.}{trein hoefden v\'o\'or Rosendaal}{niet over te stappen}\\

\haiku{Ik heb ze tot nu - '.}{toe zelf les gegevenk}{heb mijn hoofdacte}\\

\haiku{Hij wist nog best den,?}{weg in  Rotterdam maar}{of ie werom wou}\\

\haiku{Of ik nog wel wist?}{van toe z'n heele boeltje}{was weggereje}\\

\haiku{Maar toe heb ik nog... '!}{v\'e\'el fainder geslape in}{n eerste  klas}\\

\haiku{{\textquoteright} Zoo, en of ie w\'e\'er, '?}{zoo zou beginnen als ie}{hier uitt huis kwam}\\

\haiku{zeg moeder gedag,!}{en vader en Lientje en}{Sientje en Mientje}\\

\subsection{Uit: Landlooperij}

\haiku{Vier dage heb 'k ' '.}{plat opn stoel enn stoof}{thuis motte zitte}\\

\haiku{k Had die kras \`an,,...... '}{kanne vliege zoo'n sloeber}{die zei da'k vet was}\\

\haiku{{\textquoteright} {\textquoteleft}De kaptein kwam an,,...}{boord viel over z'n eige beene}{zoo zat as tie was}\\

\haiku{En dat maakte mij.}{ook al niet monter om den}{tocht te beginnen}\\

\haiku{{\textquoteright} - proestte Toon uit, sloeg,...}{op z'n knie\"en en sprong m\`et}{van de leut overeind}\\

\haiku{'t Uitvaagsel krijg... ', '...}{je mee op die nachtboott}{Laagstet minste}\\

\haiku{In 't Nieuwediep......}{woont nog zoo'n stuk vrijer van}{mijn zeit die blonde}\\

\haiku{Maar half uitgekleed, '...}{lag hij nu ook op de bank}{en sliep alsn roover}\\

\haiku{zoo groezelbleek en.}{verflenst in die beslapen}{sjofele plunjes}\\

\haiku{Boog moeizaam z'n rug,.}{en waschte z'n gezicht}{met zwakke veegjes}\\

\haiku{{\textquoteleft}'t Is koffie, want.}{nou hebbe me meteen de}{grooste schooiers beet}\\

\haiku{{\textquoteleft}Och{\textquoteright} - galmde Hannes - {\textquoteleft} ' ';}{wat zak je daars van}{zegge me jonge}\\

\haiku{{\textquoteleft}Die ouwe het nou ' - -;}{alt vierde wijf onder}{z'n tande hoor Toon}\\

\haiku{daar drinke me 'n ',.}{slokkie en me  eten}{stukkie hoor Hannes}\\

\haiku{{\textquoteleft}Ja maar{\textquoteright} - suste de - {\textquoteleft} '.}{Mottigeje kan wels}{te driftig zijn maat}\\

\haiku{En er kwam 'n zwerm.}{groote bonten aanvliegen op}{den wind voor ons uit}\\

\haiku{As je 't nou hebt ',!}{overn vangst dan had ik er}{eentje vanochtend}\\

\haiku{Maar hij is niet te...{\textquoteright} {\textquoteleft}{\textquoteright} -;}{bezeileKappe dan maar}{galmt Hannes lijzig}\\

\haiku{{\textquoteright} - hitste gejaagd de - {\textquoteleft}...}{Mottigeze benne nog}{aldeur onder schot}\\

\haiku{{\textquotedblleft}Kom, jonges, zeg ik, '.}{nou je nog uitgesloofd voor}{n oogenblikkie}\\

\haiku{Nou zit je in 't ',;}{volst vant wild en nou}{hei je geen geweer}\\

\haiku{As je je oogen in,}{d'rlui dienst verspeeld had dan}{schopten de heeren}\\

\haiku{Zoo jaag jij ze op,,, '...}{zie je en as ik ze dan}{hoor legk ze neer}\\

\haiku{zie je, dan komt die,....}{duimspeling onder z'n kin}{deur en dan zit ie}\\

\haiku{Als Dirk nu strikken,.}{gaat planten dan stopt ie die}{tusschen z'n body}\\

\haiku{Maar toch \'o\'ok vond ik ';}{t wel gewichtig nu m\'e\'e}{te doen aan de jacht}\\

\haiku{Da's stroope, dus Piet en, '.}{Dirk die benne d'r bij zoo}{vast asn lessie}\\

\haiku{Want 'n veldwachter ';}{zien ik opn afstand van}{hier wel na Tessel}\\

\haiku{Toch hebbe me 's;}{eens uit zoo'n gat effetjes}{acht ramme gevierd}\\

\haiku{{\textquoteleft}Afnokke, maat, 't, '!}{is vijf uur in de morge}{me gevet op}\\

\haiku{Toe was ie weer blauw,,}{toe ie de kroeg uitgegooid}{wier en dan kon ie}\\

\haiku{een zak, z\'o\'o groot, dat.}{onze heele jol er wel}{in kon verdwijnen}\\

\haiku{En in een herberg.}{komen ze dra allemaal}{samen om te schooven}\\

\haiku{Ze slapen met vloed,.}{en visschen met eb zoolang}{de haringvangst duurt}\\

\haiku{En hunkerig liep:}{ie langs den steiger met de}{boot mee te schreeuwen}\\

\haiku{as zij dan snorke, '...}{vanaved het de Mottige}{t rijk weer alleen}\\

\haiku{Zoodra we weer,.}{zaten ging de Mottige}{met z'n verhaal voort}\\

\haiku{alles staat op, ze '...}{schreeuwen de kelners omm}{eruit te gooien}\\

\haiku{As 'k rijk was, keek ',}{k geen jenever meer \`an}{want dan bl\'e\'ef ik \`als}\\

\haiku{dus zij houden er,.}{zich buiten en laten de}{klungels maar prutsen}\\

\haiku{b\`e je nooi mal, want ',.}{zietn ander mijn drave}{dan holt ie me v\'o\'or}\\

\haiku{d\`an spring ik er af, '.}{en ga ast mot tot me}{strot onder water}\\

\haiku{Maar 'k zeg tege... ',.}{moeder de vrouw \'o\'okn poetje}{om van te bikke}\\

\haiku{Wat hier en gunder - '!}{al b\`enk nou in me \'e\'en}{en t\`achetigste jaar}\\

\haiku{{\textquoteright} - Die komt voor, en met ':}{t p'rtaalste snoet van de}{wereld zegt Leen toen}\\

\haiku{De justitie hieuw... ',?}{Neus daar all\'e\'enig nou voor}{t Kon wel zijn h\`e}\\

\haiku{Want de wind over zee,.}{is hun adem de golfslag h\`un}{heftige polsslag}\\

\haiku{Is nooit meer 'n zucht,...}{van gehoord geen krummel van}{an komme spoele}\\

\haiku{Als ze niets meer op.}{de wereld hebben om zich}{aan vast te houden}\\

\haiku{Dan zou 'k ook... dan ' '!...}{woukt an m'n eigen}{body ervaren}\\

\haiku{D\'a\'arom zat ie met zoo'n;}{ernstigen ijver al die}{dingen te wrijven}\\

\haiku{{\textquoteright} - gierde nie uit van, '.}{de pret en hij sloeg op z'n}{dij datt kletste}\\

\haiku{Van begeerte om, '.}{dat \`al dikker te maken}{staptek vlug door}\\

\haiku{Heel teer nam ie 'n.}{harmonikaatje tusschen z'n}{reuzige handen}\\

\haiku{Hoe kom ik d'r an,?}{as niemand wat neemt van me}{armoeiige koopwaar}\\

\haiku{of 'k jaag je de...{\textquoteright} '}{honde achter je gat Maar}{daar dichtbij kwam juist}\\

\haiku{{\textquoteright} - zei Toon hongerig,.}{toen ie weer zoo afgescheept}{werd door de boerin}\\

\haiku{want da 's nou niet,,?}{smakelijk wel waar jullie}{al van gehapt hebt}\\

\haiku{Maar in de dorpen,.}{heb je de winkels en dus}{verkoop je niet veel}\\

\haiku{maar nou heb 'k 'n}{jonge genome om mee}{de boer op te gaan}\\

\haiku{Genaved,{\textquoteright} - {\textquoteleft}manne{\textquoteright} - {\textquoteleft}{\textquoteright} - ', ' '.}{goeie kwamt loom terug in}{n zucht enn geeuw}\\

\haiku{daar kan je d'r 'n '}{veeg boter op krijge en}{n asempie kaas as}\\

\haiku{Maar moeder zei, met ':}{r schrille ruzie-stem}{liefjes geknepen}\\

\haiku{{\textquoteleft}Maar 'k zeg u, als,:}{den mensch kijkt in den spiegel}{dan kijkt hij eruit}\\

\haiku{{\textquoteright} {\textquoteleft}Dat gaat nogal{\textquoteright} - zei, ',.}{ik want nu dachtk dat ie}{gek was geworden}\\

\haiku{n wage bed\`ekt ',.}{metr koopwaar en  dat}{staat arremoeiig}\\

\haiku{of als die dronken?...}{jongen in een razende}{bui wil gaan vechten}\\

\haiku{De liedjeszanger ' '.}{was toch weln aardige}{kwast vann kerel}\\

\haiku{Hij keek mij triest aan,,:}{en terwijl ie somber z'n}{hoofd schudde zei ie}\\

\haiku{Buiten de slaapstee ' '.}{stond overt steegjen klaar}{gouden ochtend}\\

\haiku{Die welvarende;}{stallen slokten natuurlijk}{allen kooplust op}\\

\haiku{alsof ie plan had.}{onzen boel voor niemendal}{weg te gaan geven}\\

\haiku{daar opeens maar te '.}{staan improviseeren met}{n galmende stem}\\

\haiku{Want overal in de '.}{huizen scheent avondlicht al}{door de gordijnen}\\

\haiku{{\textquoteright} En ik antwoordde, ';}{alsof ikn kapitaal}{had te commandeeren}\\

\haiku{{\textquoteright} {\textquoteleft}Nou{\textquoteright} - zei de buurman - {\textquoteleft} '.}{erg geheimzinnigdan zal}{ik jes helpe}\\

\haiku{De kostbaas, 't hoofd,.}{in z'n handen op tafel}{wachtte ons slapend}\\

\haiku{En w\'a\'ar je je wendt,;}{raak je de lauwe kleeren van}{andere slapers}\\

\haiku{Maar ook zijn oogen zag,.}{ik langzaam verdraaien strak}{starend naar mij heen}\\

\haiku{Ik had 't raam al,.}{lang open en m'n valiesje}{er onder gezet}\\

\haiku{Waarom hangt ie dan ',?}{daar aant hout En waarom}{uit ie dan dien kreet}\\

\haiku{{\textquoteright} vroeg, 'n eind van mij, '.}{vandaann ouwelijke}{stem onderdanig}\\

\haiku{Maer na dertien jaer...}{boeten is mijn schaemte toch}{vrij wel verdwenen}\\

\haiku{'k Had toen nog nooit,.}{jenever geproefd en mijn}{vrouw nog veel minder}\\

\haiku{Dat maekte haer...}{jaloersch en toen is ze \'o\'ok}{dapper begonnen}\\

\subsection{Uit: Het rosse leven en sterven van de Zandstraat}

\haiku{Zaten er van die, ':}{opgeschoten werkjoggies}{tusschen dan wast}\\

\haiku{Dat zijn soms menschen, ':}{waar u of een andert}{niet van denken zou}\\

\haiku{Die er over blijven,;}{tobben met wrange vlagen}{van melancholie}\\

\haiku{Heel ding hoor, om daar.}{dan altijd zelf je jatten}{bij thuis te houden}\\

\haiku{water, voor nog 'n.}{paar centen heb je een lik}{groene zeep erbij}\\

\haiku{En zoo niet, dan den, ':}{boer maar weer op om buiten}{n slagje te slaan}\\

\haiku{Maar je wil toch ook.}{niet altijd aan je zaken}{worden herinnerd}\\

\haiku{En den anderen.}{nacht werd Dirk z'n deel in de}{Schavensteeg verzet}\\

\haiku{Changeerde direct,,.}{van hand in hand langs heelsters}{en opkoopers heen}\\

\haiku{dat opstootje daar,!}{da's nu toch net om den zoon}{van den Zot gedaan}\\

\haiku{Dronken kerel wordt,;}{deur uitgesmeten dat de}{ruiten rinkinken}\\

\haiku{We hadden dien avond;}{alweer lang genoeg in den}{Polder geloopen}\\

\haiku{{\textquoteright} De meid had een paar,}{aardige oogen en ze keek}{er den rechercheur}\\

\haiku{'t glipte zoo uit, '...,,?}{t lid hing er bij Ja h\`e}{hoe je zoo zijn kan}\\

\haiku{Ze is al nege, -,... '}{jaar met September geweest}{zoo'n lekkere meid}\\

\haiku{krijg de kel\'era{\textquoteright}, '.}{en zoo as dat dan ging as}{Chrisn brom in had}\\

\haiku{Maar ze dacht, dat ie... ';}{lag te stervent koude}{zweet op z'n gezicht}\\

\haiku{Nog maar twee jaar met,{\textquoteright} -.}{April dan komt ie weer los zei}{ze nadenkelijk}\\

\haiku{t Is verdorie...{\textquoteright} {\textquoteleft},,!}{temet half eenOch meid maak}{niet zoo'n kapsoones}\\

\haiku{Maar 's \'e\'en keer, dat ',':}{ie \`erg veel smul inr had}{toe had ie gezeid}\\

\haiku{en gezongen voor,...}{die gesjefte jonges van}{d'r die daar zitten}\\

\haiku{- Direct hoor ik 'n,,:}{raam open schuiven en Toosje}{die naar buiten roept}\\

\haiku{{\textquotedblleft}Bu'vrouw, bu'vrouw, kijk '?}{s effe wie daar voor de}{deur beneden staat}\\

\haiku{Dan zal je zien en.}{ondervinden Dat jij de}{Polder nie meer ken}\\

\haiku{Ze gaan de Zandstraat ' '}{netjes makent Wordtn}{kermenadebuurt}\\

\haiku{Bij Nielsen ken je.}{nie meer dansen Bij Charley}{zijn geen meisies meer}\\

\section{Cornelis de Bruyn}

\subsection{Uit: Reizen van Cornelis de Bruyn door de vermaardste deelen van Klein Asia, de eylanden Scio, Rhodus, Cyprus, Metelino, Stanchio, \&c., mitsgaders de voornaamste steden van Aegypten, Syrien en Palestina}

\haiku{Alle lofdichten,,.}{ook de twee Latijnse zijn}{dus weggelaten}\\

\haiku{s\c{c}avoir No. 16, No.,,,,, \&.}{21 No. 22B No. 22C No. 33}{No.37 No. 41 No. 187}\\

\haiku{Deze zwierf twee maal,.}{vijf jaren gene twee maal}{twee maal vijf jaren}\\

\haiku{Ce voyageur \'etoit,{\textquoteright}.}{un seigneur Allemand qui}{parcouroit lItalie}\\

\haiku{[BA/43] De genoemde:}{plaatsen in Europa zijn}{respectievelijk}\\

\haiku{Maar van wie deze,.}{anderhalve dichtregel}{is staat niet vermeld}\\

\haiku{[BA/49] Stambol zou een [][][].}{verbastering zijn vanCon}{stantinopolis}\\

\haiku{Cf. Gaspar (1998) 366,.}{en de erbij horende}{aantekening 36}\\

\haiku{Jean-Bernard, {\textquoteleft}.}{de VaivreAutour du grand}{si\`ege de 1480}\\

\haiku{Description de{\textquoteright}:}{Rhodes \`a la fin du XVe}{si\`ecle in}\\

\haiku{[BA/128] Deze obelisk.}{is omgevallen bij de}{aardbeving van 1301}\\

\haiku{{\textquoteright} [BA/169] Bedoeld is de ( {\textendash}).}{Franse classicus Claude}{Saumaise1588 1653}\\

\haiku{Ook voor hem was hun.}{winterverblijf tot dan toe}{een raadsel geweest}\\

\haiku{Zie Cornelis de {\textquoteleft},.}{Bruyn\={u}n Yakin-Dogu}{gezisi blz. 39}\\

\haiku{Ook in andere.}{woorden verandert vanaf}{boek II de spelling}\\

\haiku{II komt ze nog maar (),}{sporadisch voor9x swaar vs.}{25x zwaar in boek III}\\

\haiku{Opmerkelijk is:}{de veranderde spelling}{van een werkwoordsvorm}\\

\haiku{Boek IV vertoonde].}{geen enkele vindplaats met}{de klankwaarde s=[z}\\

\haiku{Hier scheidde van ons {\textquoteleft}{\textquoteright},.}{het schipNassau verordend}{na Scanderona}\\

\haiku{De jacht is hier een.}{dagelykze bezigheid}{en staat een yder vry}\\

\haiku{De mouwen van het.}{hembde zyn byzonder groot}{en met kant bewerkt}\\

\haiku{Doorgaans is dezen,.}{tulband van witte roode}{of geele zyde}\\

\haiku{verwyderd is van.}{de plaats daar die stad eertyds}{gelegen was}\\

\haiku{Dit woord van kiosk {\textquoteright} {\textquoteleft}{\textquoteright}.}{betekent int Turkscheen}{overdekte galdery}\\

\haiku{Mogelyk is het.}{Endimion en Diana}{die hem komt vinden}\\

\haiku{waaruit een yder zo.}{veel tapt als hy tot deze}{wassching van doen heeft}\\

\haiku{Fraaye gestalte der,}{Turken enz. Kleeding der Turksche}{mannen en vrouwen}\\

\haiku{Want hoe langer en,.}{fraayer van baard hoe een man}{meer in achting is}\\

\haiku{Gemeenlyk kan men.}{ze allerwegen by de}{Grieken bekomen}\\

\haiku{Doch hierover schiet my {\textquoteleft}.}{yts te binnentgeen ik hier}{tusschen moet voegen}\\

\haiku{toe schieten en valt {\textquoteleft}.}{onverhoeds in het nettgeen}{achter de schuit legt}\\

\haiku{yder die daar passeert,.}{moet geven onderhouden}{konnen worden}\\

\haiku{{\textquoteright}t Is een soort van,.}{zeevisch doch behoeft voor de}{baars niet te wyken}\\

\haiku{Aanmerkelyke.}{byzonderheden wegens}{de kameleons}\\

\haiku{Boek II   	Griekse.}{eilanden en Egypte}{XXXIIe Hoofdstuk}\\

\haiku{Ook geloof ik niet.}{dat er slechter zeelieden}{in de wereld zyn}\\

\haiku{Ik bemerkte wel}{dat het geschiedde omdat}{wy niet zouden zien}\\

\haiku{Tegenwoordig is,.}{het toegemetseld doch de}{plaats noch kennelyk}\\

\haiku{De Grieken roemen.}{dat zy noch een arm van dien}{heilig bewaaren}\\

\haiku{Ook is er in yder.}{een opening welke doorgang}{geeft na een kamer}\\

\haiku{welken naam zy ook.}{tot op den huidigen dag}{heeft behouden}\\

\haiku{{\textquoteright} Verschil tusschen de}{heer Thevenot en de heer}{Melton nopende}\\

\haiku{Ik hielp het op zyn.}{gezondheid eeten en vond}{het uitsteekend goed}\\

\haiku{Zy voerde de naam,.}{van Porta Superba of}{de Prachtige Poort}\\

\haiku{Deze vermaarde,.}{Vliet of Beek Kedron ziet men}{tegenwoordig droog}\\

\haiku{In deze zoude.}{hy zyne klaagliederen}{gemaakt hebben}\\

\haiku{Ziet, den engel des,:}{Heeren verschynt Joseph in}{den droom zeggende}\\

\haiku{5:1, alwaar de man,:}{Gods propheterende van}{den Messias zegd}\\

\haiku{Van hier heeft men een.}{schoon gezicht op Jerusalem}{en Bethlehem}\\

\haiku{dat men derhalven;}{geen voet tot een zo kwaaden}{misbruyk moest geven}\\

\haiku{dingen, achteden.}{zich ge\"eerd door dezelve}{te mogen wegneemen}\\

\haiku{Op denzelven, aan,}{het eynde van de zee legd}{het zo berugte}\\

\haiku{1193 Ter zyde, in,.}{een kleyne kapel ziet men}{drie altaaren}\\

\haiku{Beschryving van de.}{Galileesche Zee en de}{stad Tiberias}\\

\haiku{haar water is zoet,,.}{heel goed om te drinken en}{uitsteekend visryk}\\

\haiku{Ongemaklyke.}{ontmoeting van den auteur}{met den Aga van Tyrus}\\

\haiku{Ongemaklyke}{ontmoeting van den Autheur}{met den Aga van Tyrus}\\

\haiku{1268 Was een man van;}{vermaakelyken omgang en}{vrolyk van humeur}\\

\haiku{men zeyde dat zelf {\textendash}.}{1344 Zenobia een Joodin geweest}{is ontleend hadden}\\

\haiku{daar konnen er, als,.}{men maatig rekend niet min}{dan 560 geweest zyn}\\

\haiku{een gang liep, kruyswys {\textquoteright},.}{doort geheele gebouw}{regt in het midden}\\

\haiku{En het melden van.}{Decapolis maakt my nog}{meer verbystert}\\

\haiku{De datum daarvan,, {\textquoteright}:}{554 wyst aant begin van}{deze rekening}\\

\haiku{het beest daar ik op,,.}{zat nederzette bleeven}{er tien of twaalf dood}\\

\haiku{{\textquoteright}t Overige der.}{kleding is op de wyze}{der Turkinnen}\\

\haiku{Met den avond naamen wy.}{onze verblyfplaats aan een}{loopend water}\\

\haiku{Huys van Cajaphas258Huys.}{van Annas258Huys van Salomon259Huys}{van Lazarusibid}\\

\haiku{20Bijvoorbeeld bij,}{Nicolaas Heinsius Den}{vermakelyken}\\

\haiku{s\c{c}avoir No. 16, No.,,,,, \&.}{21 No. 22B No. 22C No. 33}{No.37 No. 41 No. 187}\\

\haiku{550linkshandigen.}{551Zie ook hofdstuk IX}{over dit onderwerp}\\

\haiku{718Het is duidelijk.}{dat De Bruyn hier de letters}{verkeerd geplaatst heeft}\\

\haiku{829Zie folio 219.}{voor de Bruyns toelichting op}{deze geschenken}\\

\haiku{1115En van die lampen.}{zijn er enkele op de}{afbeelding te zien}\\

\haiku{1226iets 1227terzijde}{stellen 1228uitgegeven}{1229voor het geval}\\

\haiku{1503zelf 1504heilige.}{persoon 1505in elk geval}{1506Zie folio 27}\\

\haiku{1598toen wij dichter ()}{bij de wind gingen zeilen}{om bij te draaien}\\

\section{Boudewijn B\"uch}

\subsection{Uit: Links!}

\haiku{De enige die na,.}{afloop vragen stelde was}{Militaire Jaap}\\

\haiku{Ik zou die knul maar.}{beetgrijpen voordat ze de}{politie bellen}\\

\haiku{Na verloop van tijd.}{kwamen we hem steeds minder}{tegen in de stad}\\

\haiku{De deur stond, zoals,.}{altijd open en blijkbaar was}{iedereen naar bed}\\

\haiku{Dat duurde maar kort,:}{want na een minuut zuchtte}{hij en sprak plechtig}\\

\haiku{We scholden Saul uit {\textquoteleft}{\textquoteright} {\textquoteleft}{\textquoteright}.}{voorkapitalistenhond}{enheterovriend}\\

\haiku{{\textquoteleft}Saul verhuist morgen.}{naar een kutkamertje in}{een sombere steeg}\\

\haiku{Even later kwam hij,.}{terug met een pick-up een}{elpee en twee boxen}\\

\haiku{Ik werd weer ernstig,:}{keek in mijn eigen exemplaar}{en merkte trots op}\\

\haiku{Sirius gaf het.}{miezerige kereltje}{een dreun op zijn neus}\\

\haiku{Een paar minuten.}{later zaten we in een}{politiebusje}\\

\haiku{Een aantrekkelijk.}{agentje met pukkeltjes sloot}{ons in en lachte}\\

\haiku{Hij kwam de kamer.}{in en klapte de deksel}{van zijn pick-up open}\\

\haiku{{\textquoteleft}Je krijgt op deze,!}{manier toch de lekkerste}{klap lucht binnen Boud}\\

\haiku{{\textquoteleft}Wat bedoel je met {\textquotedblleft}{\textquotedblright},?}{de Stones enpolitiek}{bezig zijn Polly}\\

\haiku{{\textquoteleft}Sier, geile pieper,.}{van me je krijgt die jongen}{\'echt niet uit de broek}\\

\haiku{{\textquoteleft}Jongen, als je er,.}{helemaal aan verslingerd}{bent zeg het me dan}\\

\haiku{Sirius las in.}{een Aula-pocket over iets}{sociologisch}\\

\haiku{{\textquoteright} {\textquoteleft}De stencils tegen.}{de opvoering van The boys}{in the band zijn klaar}\\

\haiku{Daarom moeten we...}{actie voeren tegen dat}{toneelstuk van die}\\

\haiku{Toen Karola binnen,:}{was gekomen stelde ze}{de bos blond haar voor}\\

\haiku{we waren jaloers.}{op haar dat ze zo'n man de}{hare mocht noemen}\\

\haiku{De ellende gaat,{\textquoteright}.}{voor het werkvolk ook door op}{kerstavond brieste Dolf}\\

\haiku{* dolf had ons dan snel,.}{ingepalmd maar lideke}{kon er ook wat van}\\

\haiku{{\textquoteright} Dolf antwoordde, zeer,.}{tegen zijn gewoonte niet}{op wat Polly zei}\\

\haiku{En zo stonden er.}{nog enkele tientallen}{punten op de lijst}\\

\haiku{Dolf kwam binnen, hij:}{keek mij nauwelijks aan en}{filosofeerde}\\

\haiku{Het licht ging aan en.}{Dolf stond op de drempel met}{een vuurrode kop}\\

\haiku{Een woord dat al lang.}{tot de spreektaal behoort in}{New York en Frisco}\\

\haiku{{\textquoteright} {\textquoteleft}Dat is geen wie, maar.}{de alternatieve naam}{voor San Francisco}\\

\haiku{Hij stond op en zei {\textquoteleft}{\textquoteright}.}{dat hijzo stoned als een}{afwasteiltje was}\\

\haiku{Hij vroeg zich af of.}{een kunstboom niet beter was}{dan een natuurhoorn}\\

\haiku{Nee, Polly, ik ben,.}{er niet geweest maar lees wel}{de Peking Review}\\

\haiku{{\textquoteright} Lideke ging met.}{de theekan rond en vond dat}{we maar moesten stemmen}\\

\haiku{{\textquoteright} Dolf was gaan zitten.}{en schoof zenuwachtig op}{zijn stoel heen en weer}\\

\haiku{Als jij zegt dat Mao,.}{dat geschreven heeft dan zal}{het heus wel zo zijn}\\

\haiku{Ik keek Dolf in de.}{ogen en zag dat h{\'\i}j dat zelfs}{niet geloofde}\\

\haiku{In Schimmert ken ik,.}{een groot leegstaand klooster dat}{we kunnen kraken}\\

\haiku{{\textquoteright} {\textquoteleft}Schimmel in Schimmert,{\textquoteright}.}{joelde Sirius die zijn}{derde joint bouwde}\\

\haiku{We hoorden iemand.}{heel uit de verte door een}{lange gang klossen}\\

\haiku{Dja's konijn heeft de.}{reis hierheen waarschijnlijk niet}{kunnen verdragen}\\

\haiku{Dolf was inmiddels.}{ook gearriveerd en stond}{wat streng te kijken}\\

\haiku{Ik haalde mijn hand.}{door Dja's pagekopje en}{keek in het doosje}\\

\haiku{We hebben Dja een.}{plechtigheid beloofd en die}{zal hij krijgen ook}\\

\haiku{Dolf begon met het.}{voorlezen van een gedicht}{van Voorzitter Mao}\\

\haiku{Polly kauwde op.}{een grasspriet en Lideke}{krabde in haar kruis}\\

\haiku{Volgens mij hebben,,{\textquoteright},.}{we platjes Dolf riep ze zich}{omdraaiend naar Dolf}\\

\haiku{{\textquoteright} Bono keek naar mij:}{en ik hoefde er niet eens}{over na te denken}\\

\haiku{Toen werd alles zwart.}{en stierf alle geluid en}{de laatste kleur weg}\\

\haiku{De zonnehitte.}{deed het zweet uit mijn lange}{haren weggutsen}\\

\haiku{Ik haalde het en.}{Sirius en ik gingen}{ermee aan de gang}\\

\haiku{Toen Bono met een, - {\textquoteleft}!}{laatste vermoeide kreet hij}{schreeuwdeAgnus Dei}\\

\haiku{Wat had plotseling {\textquoteleft}{\textquoteright}?}{een plenaire discussie}{te betekenen}\\

\haiku{Om elf uur betrad,,.}{ik benieuwd maar ook bang de}{centrale ruimte}\\

\haiku{Zag je zijn gezicht?}{toen ik dat fantastische}{stuk van Mao voorlas}\\

\haiku{{\textquoteright} {\textquoteleft}Hier, hier,{\textquoteright} riep ik en.}{gooide de tekening van}{Dja weer op de grond}\\

\haiku{dat je in die drie.}{maanden  met niemand spreekt}{en niet thuis zult eten}\\

\haiku{Ik heb geen zin meer.}{om die klootzakken waar dan}{ook te ontmoeten}\\

\haiku{Dat jongetje zal -.}{zonder mij verder moeten}{dacht ik wijselijk}\\

\haiku{Behalve dan dat [...].}{hij samen schijnt te wonen}{met een rechts wijf in}\\

\haiku{dat weten onze.}{Chinese vrienden heel goed}{op prijs te stellen}\\

\haiku{Je zal zien dat we.}{een ereplekje krijgen in}{de receptiezaal}\\

\haiku{Rare snijbonen,,;}{lijken die Chinezen mij}{maar zo'n cultuur h\`e}\\

\subsection{Uit: Links!}

\haiku{De enige die na,.}{afloop vragen stelde was}{Militaire Jaap}\\

\haiku{Ik zou die knul maar.}{beetgrijpen voordat ze de}{politie bellen}\\

\haiku{* ~ dat we niet meer.}{in de mensa konden eten}{was voor Saul het ergst}\\

\haiku{Na verloop van tijd.}{kwamen we hem steeds minder}{tegen in de stad}\\

\haiku{De deur stond, zoals,.}{altijd open en blijkbaar was}{iedereen naar bed}\\

\haiku{Dat duurde maar kort,:}{want na een minuut zuchtte}{hij en sprak plechtig}\\

\haiku{We scholden Saul uit {\textquoteleft}{\textquoteright} {\textquoteleft}{\textquoteright}.}{voorkapitalistenhond}{enheterovriend}\\

\haiku{{\textquoteleft}Saul verhuist morgen.}{naar een kutkamertje in}{een sombere steeg}\\

\haiku{Even later kwam hij,.}{terug met een pick-up een}{elpee en twee boxen}\\

\haiku{Ik werd weer ernstig,:}{keek in mijn eigen exemplaar}{en merkte trots op}\\

\haiku{Sirius gaf het.}{miezerige kereltje}{een dreun op zijn neus}\\

\haiku{Een paar minuten.}{later zaten we in een}{politiebusje}\\

\haiku{Een aantrekkelijk.}{agentje met pukkeltjes sloot}{ons in en lachte}\\

\haiku{{\textquoteright} brulde ergens aan.}{het einde van de gang een}{politieagent}\\

\haiku{Hij kwam de kamer.}{in en klapte de deksel}{van zijn pick-up open}\\

\haiku{{\textquoteleft}Je krijgt op deze,!}{manier toch de lekkerste}{klap lucht binnen Boud}\\

\haiku{{\textquoteleft}Wat bedoel je met {\textquotedblleft}{\textquotedblright},?}{de Stones enpolitiek}{bezig zijn Polly}\\

\haiku{{\textquoteleft}Sier, geile pieper,.}{van me je krijgt die jongen}{\'echt niet uit de broek}\\

\haiku{{\textquoteleft}Jongen, als je er,.}{helemaal aan verslingerd}{bent zeg het me dan}\\

\haiku{Sirius las in.}{een Aula-pocket over iets}{sociologisch}\\

\haiku{* ~ een dokter die.}{uit zijn mond riekte stond over}{mij heen gebogen}\\

\haiku{{\textquoteright} {\textquoteleft}De stencils tegen.}{de opvoering van The boys}{in the band zijn klaar}\\

\haiku{Daarom moeten we...}{actie voeren tegen dat}{toneelstuk van die}\\

\haiku{Toen Karola binnen,:}{was gekomen stelde ze}{de bos blond haar voor}\\

\haiku{we waren jaloers.}{op haar dat ze zo'n man de}{hare mocht noemen}\\

\haiku{De ellende gaat,{\textquoteright}.}{voor het werkvolk ook door op}{kerstavond brieste Dolf}\\

\haiku{{\textquoteright} Dolf antwoordde, zeer,.}{tegen zijn gewoonte niet}{op wat Polly zei}\\

\haiku{En zo stonden er.}{nog enkele tientallen}{punten op de lijst}\\

\haiku{Dolf kwam binnen, hij:}{keek mij nauwelijks aan en}{filosofeerde}\\

\haiku{Het licht ging aan en.}{Dolf stond op de drempel met}{een vuurrode kop}\\

\haiku{Een woord dat al lang.}{tot de spreektaal behoort in}{New York en Frisco}\\

\haiku{{\textquoteright} {\textquoteleft}Dat is geen wie, maar.}{de alternatieve naam}{voor San Francisco}\\

\haiku{Hij stond op en zei {\textquoteleft}{\textquoteright}.}{dat hijzo stoned als een}{afwasteiltje was}\\

\haiku{Hij vroeg zich af of.}{een kunstboom niet beter was}{dan een natuurhoorn}\\

\haiku{Nee, Polly, ik ben,.}{er niet geweest maar lees wel}{de Peking Review}\\

\haiku{{\textquoteright} Lideke ging met.}{de theekan rond en vond dat}{we maar moesten stemmen}\\

\haiku{{\textquoteright} Dolf was gaan zitten.}{en schoof zenuwachtig op}{zijn stoel heen en weer}\\

\haiku{Als jij zegt dat Mao,.}{dat geschreven heeft dan zal}{het heus wel zo zijn}\\

\haiku{Ik keek Dolf in de.}{ogen en zag dat h{\'\i}j dat zelfs}{niet geloofde}\\

\haiku{In Schimmert ken ik,.}{een groot leegstaand klooster dat}{we kunnen kraken}\\

\haiku{{\textquoteright} {\textquoteleft}Schimmel in Schimmert,{\textquoteright}.}{joelde Sirius die zijn}{derde joint bouwde}\\

\haiku{We hoorden iemand.}{heel uit de verte door een}{lange gang klossen}\\

\haiku{Dja's konijn heeft de.}{reis hierheen waarschijnlijk niet}{kunnen verdragen}\\

\haiku{Dolf was inmiddels.}{ook gearriveerd en stond}{wat streng te kijken}\\

\haiku{Ik haalde mijn hand.}{door Dja's pagekopje en}{keek in het doosje}\\

\haiku{We hebben Dja een.}{plechtigheid beloofd en die}{zal hij krijgen ook}\\

\haiku{Dolf begon met het.}{voorlezen van een gedicht}{van Voorzitter Mao}\\

\haiku{Polly kauwde op.}{een grasspriet en Lideke}{krabde in haar kruis}\\

\haiku{Volgens mij hebben,,{\textquoteright},.}{we platjes Dolf riep ze zich}{omdraaiend naar Dolf}\\

\haiku{{\textquoteright} Bono keek naar mij:}{en ik hoefde er niet eens}{over na te denken}\\

\haiku{Toen werd alles zwart.}{en stierf alle geluid en}{de laatste kleur weg}\\

\haiku{De zonnehitte.}{deed het zweet uit mijn lange}{haren weggutsen}\\

\haiku{Ik haalde het en.}{Sirius en ik gingen}{ermee aan de gang}\\

\haiku{Toen Bono met een, - {\textquoteleft}!}{laatste vermoeide kreet hij}{schreeuwdeAgnus Dei}\\

\haiku{Wat had plotseling {\textquoteleft}{\textquoteright}?}{een plenaire discussie}{te betekenen}\\

\haiku{Om elf uur betrad,,.}{ik benieuwd maar ook bang de}{centrale ruimte}\\

\haiku{Zag je zijn gezicht?}{toen ik dat fantastische}{stuk van Mao voorlas}\\

\haiku{{\textquoteright} {\textquoteleft}Hier, hier,{\textquoteright} riep ik en.}{gooide de tekening van}{Dja weer op de grond}\\

\haiku{dat je in die drie.}{maanden  met niemand spreekt}{en niet thuis zult eten}\\

\haiku{Ik heb geen zin meer.}{om die klootzakken waar dan}{ook te ontmoeten}\\

\haiku{Dat jongetje zal -.}{zonder mij verder moeten}{dacht ik wijselijk}\\

\haiku{Behalve dan dat [...].}{hij samen schijnt te wonen}{met een rechts wijf in}\\

\haiku{dat weten onze.}{Chinese vrienden heel goed}{op prijs te stellen}\\

\haiku{Je zal zien dat we.}{een ereplekje krijgen in}{de receptiezaal}\\

\haiku{Rare snijbonen,,;}{lijken die Chinezen mij}{maar zo'n cultuur h\`e}\\

\subsection{Uit: De rekening}

\haiku{{\textquoteright} Mijn moeder poetste:}{het zilver op een oude}{krant en zei zachtjes}\\

\haiku{{\textquoteright} Over de rest van de.}{inboedel werd nog een jaar}{geprocedeerd}\\

\haiku{{\textquoteright} vroeg de chef die ook,.}{een stofjas droeg voorzien van}{een extra biesje}\\

\haiku{{\textquoteright} lachte een jongen.}{met een benig gezicht en}{een tatoeage}\\

\haiku{{\textquoteleft}Slordig werk heb je,?}{geleverd moet ik daar een}{riks voor betalen}\\

\haiku{Die hebben we bij,.}{Pimmetje laten opslaan}{midden in de nacht}\\

\haiku{Morgen ben ik er,,{\textquoteright}.}{ook nog Lomar sprak Hetty}{gemaakt sensueel}\\

\haiku{maandenlang stond ze.}{mij voor ogen als ik in de}{loods aan het werk was}\\

\haiku{Ik begrijp niet dat.}{die man niet af en toe een}{graai in de kas doet}\\

\haiku{Voor moeder is het,.}{natuurlijk ook leuk komt die}{er ook nog eens uit}\\

\haiku{Kijk, jij gaat straks op.}{een school les geven en dan}{verdien je goud geld}\\

\haiku{Op een keer zei ze -, -:}{grappig bedoeld maar toch ook}{een beetje gemeend}\\

\haiku{{\textquoteright} {\textquoteleft}Dat is maar goed ook,.}{want anders flikkerde ik}{ze er direct uit}\\

\haiku{{\textquoteright} {\textquoteleft}Ik schrijf wel een brief.}{en nu gaan we het over die}{verslaving hebben}\\

\haiku{Ik heb er dankzij.}{mijn ouders in mijn jeugd al}{genoeg meegemaakt}\\

\haiku{{\textquoteright} {\textquoteleft}'t Is beter, voor,;}{de nodige gespreksrust}{gelooft u mij maar}\\

\haiku{{\textquoteright} {\textquoteleft}Wat kan voor u het?}{belang zijn om de naam van}{die kat te weten}\\

\haiku{Soms fietste ik wel.}{vijf keer per week naar het huis}{aan de stille laan}\\

\haiku{ik zal 'ns kijken,}{hoe we dat met die centen}{op kunnen lossen}\\

\haiku{Ga op zoek naar de,:}{schuldvraag pijnig je kop af}{naar de oplossing}\\

\haiku{Geld moet er zijn - dat -.}{weet ik ook wel maar het moet}{onzichtbaar blijven}\\

\haiku{Beneden begon.}{er iemand anders op de}{deur te beuken}\\

\haiku{Als je goed wakker,.}{bent ga je weer neuken en}{dan opnieuw slapen}\\

\haiku{Dana vroeg op een,:}{morgen de zon stond achter}{de gordijnen klaar}\\

\haiku{Soms zag ik Liphorst.}{een jaar lang niet en hoorde}{ik ook niets van hem}\\

\haiku{{\textquoteleft}Ik heb alleen een.}{tas met wat kleren en een}{koffer vol boeken}\\

\haiku{Een wijf om neer te,,.}{sabelen zeg maar niets ik}{praat wel even met haar}\\

\haiku{We hebben hier een.}{chic huis en dat zou ik graag}{zo willen houden}\\

\haiku{Nou ja, je bent de:}{eigenaresse van een}{chic huis dus je zegt}\\

\haiku{Het duurt nu al vijf.}{jaar en er lijkt nooit meer een}{eind aan te komen}\\

\haiku{Zodoende was het.}{aan mijn deur een komen en}{gaan van geldeisers}\\

\haiku{{\textquoteright} klinkt het opeens op.}{tussen angstaanjagende}{vogelgeluiden}\\

\haiku{De middenstand heeft.}{een klein beetje de neiging}{westers te denken}\\

\haiku{{\textquoteright} Na van de schrik te,:}{zijn bekomen antwoordde}{ik zenuwachtig}\\

\haiku{Goudstaart had een aardig,.}{gezicht was klein en had niets}{van een advocaat}\\

\haiku{{\textquoteright} {\textquoteleft}Dat zou natuurlijk.}{het eerste zijn geweest dat}{hij had moeten doen}\\

\haiku{In geen geval wil.}{ik meer dan drie ton voor een}{woning betalen}\\

\haiku{Hij onderhield het.}{contact nog even door middel}{van aanmaningen}\\

\haiku{ik heb geen zin om}{al die muntjes tussen de}{lakens uit te gaan}\\

\haiku{{\textquoteright} {\textquoteleft}Geef die jongen toch,}{gewoon een gulden en laat}{hem die bus houden}\\

\haiku{Mijn moeder wees naar:}{een kleine jongen en een}{nog kleiner meisje}\\

\haiku{Jugend in D. 1916- -.}{1937 krullerige letters}{met inkt geschreven}\\

\haiku{Werd mijn vader door?}{zijn toenmalige vrouw of}{verloofde gekiekt}\\

\haiku{Tussen mijn vader (:}{en mij was er ogenschijnlijk}{weinigmijn moeder}\\

\haiku{Mary werd in 1940,.}{geboren acht jaar voordat}{ik ter wereld kwam}\\

\haiku{De keren dat hij,.}{naar Nieuw-Zeeland kwam was}{hij ook niet zuinig}\\

\haiku{Je moet beloven.}{dat je ze nooit aan iemand}{anders laat lezen}\\

\haiku{Ik ken je moeder,.}{niet maar ik heb natuurlijk}{vaak over haar gehoord}\\

\haiku{Ik dacht plotseling.}{terug aan de jaren op}{de lagere school}\\

\haiku{je melk mag drinken,.}{krijgen jullie van Hare}{Majesteit cadeau}\\

\haiku{Toen ik besloten,:}{had dat ik D. voorgoed zou}{verlaten zei ze}\\

\haiku{Hij was beter af.}{geweest wanneer hij boven}{D. was neergehaald}\\

\haiku{op tien mei 1945 schreef.}{hij al zijn eerste brief om}{schadevergoeding}\\

\haiku{Dat is het enige:}{waar hij na de oorlog mee}{bezig is geweest}\\

\haiku{Weet je nog toen die?}{oorlogsmisdadigers vrij}{werden gelaten}\\

\subsection{Uit: De rekening}

\haiku{{\textquoteright} Mijn moeder poetste:}{het zilver op een oude}{krant en zei zachtjes}\\

\haiku{{\textquoteright} Over de rest van de.}{inboedel werd nog een jaar}{geprocedeerd}\\

\haiku{{\textquoteright} vroeg de chef die ook,.}{een stofjas droeg voorzien van}{een extra biesje}\\

\haiku{{\textquoteright} lachte een jongen.}{met een benig gezicht en}{een tatoeage}\\

\haiku{{\textquoteleft}Slordig werk heb je,?}{geleverd moet ik daar een}{riks voor betalen}\\

\haiku{Die hebben we bij,.}{Pimmetje laten opslaan}{midden in de nacht}\\

\haiku{Morgen ben ik er,,{\textquoteright}.}{ook nog Lomar sprak Hetty}{gemaakt sensueel}\\

\haiku{maandenlang stond ze.}{mij voor ogen als ik in de}{loods aan het werk was}\\

\haiku{Ik begrijp niet dat.}{die man niet af en toe een}{graai in de kas doet}\\

\haiku{Voor moeder is het,.}{natuurlijk ook leuk komt die}{er ook nog eens uit}\\

\haiku{Kijk, jij gaat straks op.}{een school les geven en dan}{verdien je goud geld}\\

\haiku{Op een keer zei ze -, -:}{grappig bedoeld maar toch ook}{een beetje gemeend}\\

\haiku{{\textquoteright} {\textquoteleft}Dat is maar goed ook,.}{want anders flikkerde ik}{ze er direct uit}\\

\haiku{{\textquoteright} {\textquoteleft}Ik schrijf wel een brief.}{en nu gaan we het over die}{verslaving hebben}\\

\haiku{Ik heb er dankzij.}{mijn ouders in mijn jeugd al}{genoeg meegemaakt}\\

\haiku{{\textquoteright} {\textquoteleft}'t Is beter, voor,;}{de nodige gespreksrust}{gelooft u mij maar}\\

\haiku{{\textquoteright} {\textquoteleft}Wat kan voor u het?}{belang zijn om de naam van}{die kat te weten}\\

\haiku{Soms fietste ik wel.}{vijf keer per week naar het huis}{aan de stille laan}\\

\haiku{ik zal 'ns kijken,}{hoe we dat met die centen}{op kunnen lossen}\\

\haiku{Ga op zoek naar de,:}{schuldvraag pijnig je kop af}{naar de oplossing}\\

\haiku{Geld moet er zijn - dat -.}{weet ik ook wel maar het moet}{onzichtbaar blijven}\\

\haiku{Beneden begon.}{er iemand anders op de}{deur te beuken}\\

\haiku{Als je goed wakker,.}{bent ga je weer neuken en}{dan opnieuw slapen}\\

\haiku{Dana vroeg op een,:}{morgen de zon stond achter}{de gordijnen klaar}\\

\haiku{Soms zag ik Liphorst.}{een jaar lang niet en hoorde}{ik ook niets van hem}\\

\haiku{{\textquoteleft}Ik heb alleen een.}{tas met wat kleren en een}{koffer vol boeken}\\

\haiku{Een wijf om neer te,,.}{sabelen zeg maar niets ik}{praat wel even met haar}\\

\haiku{We hebben hier een.}{chic huis en dat zou ik graag}{zo willen houden}\\

\haiku{Nou ja, je bent de:}{eigenaresse van een}{chic huis dus je zegt}\\

\haiku{Het duurt nu al vijf.}{jaar en er lijkt nooit meer een}{eind aan te komen}\\

\haiku{Zodoende was het.}{aan mijn deur een komen en}{gaan van geldeisers}\\

\haiku{{\textquoteright} klinkt het opeens op.}{tussen angstaanjagende}{vogelgeluiden}\\

\haiku{De middenstand heeft.}{een klein beetje de neiging}{westers te denken}\\

\haiku{{\textquoteright} Na van de schrik te,:}{zijn bekomen antwoordde}{ik zenuwachtig}\\

\haiku{Goudstaart had een aardig,.}{gezicht was klein en had niets}{van een advocaat}\\

\haiku{{\textquoteright} {\textquoteleft}Dat zou natuurlijk.}{het eerste zijn geweest dat}{hij had moeten doen}\\

\haiku{In geen geval wil.}{ik meer dan drie ton voor een}{woning betalen}\\

\haiku{Hij onderhield het.}{contact nog even door middel}{van aanmaningen}\\

\haiku{ik heb geen zin om}{al die muntjes tussen de}{lakens uit te gaan}\\

\haiku{{\textquoteright} {\textquoteleft}Geef die jongen toch,}{gewoon een gulden en laat}{hem die bus houden}\\

\haiku{Mijn moeder wees naar:}{een kleine jongen en een}{nog kleiner meisje}\\

\haiku{Jugend in D. 1916- -.}{1937 krullerige letters}{met inkt geschreven}\\

\haiku{Werd mijn vader door?}{zijn toenmalige vrouw of}{verloofde gekiekt}\\

\haiku{Tussen mijn vader (:}{en mij was er ogenschijnlijk}{weinigmijn moeder}\\

\haiku{Mary werd in 1940,.}{geboren acht jaar voordat}{ik ter wereld kwam}\\

\haiku{De keren dat hij,.}{naar Nieuw-Zeeland kwam was}{hij ook niet zuinig}\\

\haiku{Je moet beloven.}{dat je ze nooit aan iemand}{anders laat lezen}\\

\haiku{Ik ken je moeder,.}{niet maar ik heb natuurlijk}{vaak over haar gehoord}\\

\haiku{Ik dacht plotseling.}{terug aan de jaren op}{de lagere school}\\

\haiku{je melk mag drinken,.}{krijgen jullie van Hare}{Majesteit cadeau}\\

\haiku{Toen ik besloten,:}{had dat ik D. voorgoed zou}{verlaten zei ze}\\

\haiku{Hij was beter af.}{geweest wanneer hij boven}{D. was neergehaald}\\

\haiku{op tien mei 1945 schreef.}{hij al zijn eerste brief om}{schadevergoeding}\\

\haiku{Dat is het enige:}{waar hij na de oorlog mee}{bezig is geweest}\\

\haiku{Weet je nog toen die?}{oorlogsmisdadigers vrij}{werden gelaten}\\

\section{Thomas Fran\c{c}ois Burgers}

\subsection{Uit: Dorp in het onderveld}

\haiku{Geen behoeften wekt,;}{hij op hij die daar volstrekt}{geen behoefte toont}\\

\haiku{Wat moet het toch zwaar!}{zijn om de rente van angst}{en vrees te trekken}\\

\haiku{de een altijd vlug,.}{en rad de ander altoos}{langzaam en bedaard}\\

\haiku{Kom, sta eens hier, en,.}{zie die lange magere}{kerel hoe hij knikt}\\

\haiku{{\textquoteleft}Neef Willem,{\textquoteright} zei een, {\textquoteleft}.}{ou tantehier is een mooi}{pak kinderkleertjes}\\

\haiku{{\textquoteleft}Oom Willem, hier is,.}{nu voor uzelf ook wat een mooie}{ouderwetse broek}\\

\haiku{De bediende thuis.}{en de kinderen krijgen}{strengere orders}\\

\haiku{Hoe ik eruitzie,,}{weet ik niet maar als ik de}{gemeente overzie}\\

\haiku{Neem, ik bid u, het.}{mij ook niet kwalijk dat ik}{rondkijk in de kerk}\\

\haiku{De kerk is lang niet.}{zo vol als gisteren en}{de leraar ook niet}\\

\haiku{neef Hendrik springt van.}{zijn tafel op en vergeet}{bijna te danken}\\

\haiku{Ik weet nu alles,,.}{maar wees gerust uw geheim}{is bij mij veilig}\\

\haiku{hem die ander dag,{\textquoteright}.}{verbrand het met die warme}{lood bracht Mietje in}\\

\haiku{anders zou ik na.}{die water kijk en dan kon}{Dina kogels giet}\\

\haiku{Hij slooft zich af en.}{offert zichzelf werkelijk}{op voor zijn gasten}\\

\haiku{Hij wendde zich naar,.}{de hertog en fluisterde}{hem iets in het oor}\\

\haiku{Maar wacht maar, ik zal,{\textquoteright}.}{de boel inpeperen zo}{ging de spreker voort}\\

\haiku{allen, behalve,.}{oom Hendrik waren die dag}{slaven in dat huis}\\

\haiku{Een menigte van,,.}{allerlei kleur en geur ook}{stond er te kijken}\\

\haiku{Koffie en thee, wijn.}{en bier werden in ruime}{mate rondgediend}\\

\haiku{Zo had oom Hendrik;}{de danslust op een plaats met}{een slag uitgeblust}\\

\haiku{{\textquoteleft}Ja, zo hebt jij ons,.}{mos van dag beet gehad voel}{nou ook hoe het smaak}\\

\haiku{Neef Nols had reeds een,,,.}{bok Paul ook het werd mijn tijd}{en ik drong voorwaarts}\\

\haiku{De bok liep als had,.}{hij geen kwetsuur Tempest als}{had hij geen ruiter}\\

\haiku{{\textquoteleft}Maar Pietje, jij het,!}{gloo sommer van dag een dooie}{bok opgetel neh}\\

\haiku{Vat zomaar met de,,{\textquoteright}, {\textquoteleft}.}{hand kerel riep er eenen}{gebruik jou jachtmes}\\

\haiku{Geen sloot te breed, geen {\textquoteleft}{\textquoteright}.}{polleno te hoog voor de}{moedige paarden}\\

\haiku{de priester bericht.}{moet doen met het verzoek zijn}{kindje te dopen}\\

\haiku{Hij weet, er is geen.}{uitweg en beproeft ook niet}{eens die te zoeken}\\

\haiku{{\textquoteleft}Wat hebben wij nog.}{getuigenis van node?{\textquoteright}o}{vraagt de andere}\\

\haiku{tata betekent,.}{bij de kleurlingen vader}{outata grootvader}\\

\haiku{kos mijn arme ziel}{verlangt zeer naar voedsel van}{morre vanmorgen}\\

\haiku{Mededeling van,.}{het Meertens Instituut d.d.}{12 december 2003}\\

\haiku{jij hebt geloof ik,!}{vandaag zomaar een dode}{bok opgepakt niet}\\

\section{Andreas Burnier}

\subsection{Uit: Een tevreden lach}

\haiku{Zij ontdekte er,:}{nooit iets nieuws maar een machtig}{beeld welde er op}\\

\haiku{Geachte heer, ik,.}{ben tijdelijk in Parijs}{maar ken hier niemand}\\

\haiku{{\textquoteright} Simone begreep.}{niet waarom de uitgever}{haar dit vertelde}\\

\haiku{{\textquoteleft}Het is een meisje,,,.}{ja hoor het is een meisje}{dat kan je goed zien}\\

\haiku{waar ik op de een.}{of andere manier}{mee te maken had}\\

\haiku{Ze is heel geschikt.}{en ze schijnt zich voor je te}{interesseren}\\

\haiku{De hele hete.}{zomer repeteerde ik}{de schoolwiskunde}\\

\haiku{{\textquoteleft}Wij zijn van de g.g.d.{\textquoteright},.}{zei de verpleegster langzaam}{en nadrukkelijk}\\

\haiku{Ik kreeg een spuitje.}{en ontwaakte veel later}{op een grote zaal}\\

\haiku{Van nu af was er}{in de buitenwereld niets}{meer te verwachten}\\

\haiku{Er ontstond tussen:}{ons wat je een schijncontact}{zou kunnen noemen}\\

\haiku{Of eigenlijk over:}{de samenhang van de drie}{polariteiten}\\

\haiku{homoseksueel,,,.}{kunstenaar misdadiger}{in zich verenigt}\\

\haiku{Zij nam het kaartje aan,,.}{zocht overal maar kon de}{boeken niet vinden}\\

\haiku{We zullen op de,,{\textquoteright}.}{motor naar Helmond rijden}{Simone zei hij}\\

\haiku{{\textquoteleft}Het was de eerste{\textquoteright}....}{keer voor mij en sloot heel zacht}{de deur achter zich}\\

\haiku{Het was een rooster.}{onder in de muur waar het}{lichtschijnsel uitkwam}\\

\haiku{Ze dronken rode '.}{wijn ens nachts kwam Rainer}{nog drie keer bij haar}\\

\haiku{Nou ken je nog zo,.}{eenzaam wezen dat heb je}{dan maar te dragen}\\

\haiku{Je moet je eigen,.}{nou eenmaal accepteren}{zoals je bent h\`e}\\

\haiku{Haar vader was voor.}{de oorlog houthakker in}{Tsjecho-Slowakije}\\

\haiku{Maar wat weten zij?}{daar van onze verstarring}{en ontlediging}\\

\haiku{Als het niet anders,:}{kan dan in Holland dan zal}{ik er doorheen gaan}\\

\haiku{We konden moeilijk.}{weer weigeren en jij hebt}{nu net geen nachtdienst}\\

\section{Cyriel Buysse}

\subsection{Uit: Het 'ezelken', wat niet vergeten was}

\haiku{een pastoor in haar -.}{familie te bezitten}{aan hem was volbracht}\\

\haiku{Juffrouw Constance.}{was een oude vrijster van}{reeds bij de veertig}\\

\haiku{en daarop volgde,:}{een lange lange optocht}{andere vrouwen}\\

\haiku{{\textquoteleft}Geen vrijage, noch,,!}{binnenshuis noch buitenshuis}{of dadelijk weg}\\

\haiku{- K'n weet ik da niet,.}{antwoordde C\'eline dan}{ook heel natuurlijk}\\

\haiku{Eerst vreesde zij nog, -!}{met welken angst van twijfel}{en onzekerheid}\\

\haiku{loat ons hou\^en 't '.}{geen da w'h\^en en kontent zijn}{datt zeu wel goat}\\

\haiku{riep juffrouw Toria,.}{geschokt met grooten mond en}{uitgezette oogen}\\

\haiku{'t es doarover ', '.}{dak ou kome spreken}{beefdet Ezelken}\\

\haiku{Wie weet ook of het,?}{om haar geld alleen was dat}{hij haar gevraagd had}\\

\haiku{Tok tok tok, hoorde.}{zij de meid aan haar broeder's}{slaapkamer tikken}\\

\haiku{Er was een korte,,.}{poos volkomen doodsche als}{versteende stilte}\\

\haiku{C\'eline's gezicht,.}{en manieren bevielen}{haar niet dien middag}\\

\haiku{H\'e-je nou gezien ''?}{wa veurn schandoal da}{g in ou huis h\^et}\\

\haiku{'k h\`e heur gezeid!}{da ze morgen uchtijnk mee}{pak en zak wig moet}\\

\haiku{zij riep ten slotte;}{haar meid naar binnen en deed}{haar licht aansteken}\\

\haiku{- O, iefer Toria,,!}{iefer Toria wa zij-je}{gij toch broave}\\

\haiku{Zij sloot den brief in,.}{het couvert en Aamlie bracht}{hem naar de bus}\\

\haiku{- Ge meug gerust zijn,,.}{ieffreiwe antwoordde de}{meid reeds in de gang}\\

\haiku{ik u door Ivo uwe.}{koffers met alles er in}{wat u toebehoort}\\

\haiku{het spotgelach der,.}{grappenmakers die zich daar}{verscholen hadden}\\

\haiku{'t Puipken stond even,}{onthutst maar v\'o\'or hij er meer}{van kon vertellen}\\

\haiku{Langzamerhand was.}{het Ezelken aan den toestand}{gewend geraakt}\\

\haiku{Juffer Toria, en,.}{ook het Ezelken waren maar}{half gerustgesteld}\\

\haiku{Het scheelde weinig ' ' -!}{oft glas stortte uits}{Ezelkens hand. Watte}\\

\haiku{'t Ezelken had het, ',....}{alst ware instinktief}{voelen aankomen}\\

\haiku{Wat moest er van haar,,?}{worden waar moest ze heen als}{juffer Toria stierf}\\

\haiku{Mirza moest het eerst,.}{bediend worden die was de}{ongeduldigste}\\

\haiku{- Ik, da huis keupen,, ' ';}{wa peist-ekn h\`e}{doar gien geld veuren}\\

\haiku{'t Was of ze twee,.}{huizen bewoonden een aan}{elken kant der straat}\\

\haiku{- 'K zal de koster,;}{loate roepen zuchtte}{nog eens het Ezelken}\\

\haiku{Kan 't hij nou euk '! '}{al nie verdroagen datn}{kind zijne stiel liert}\\

\haiku{- Pas moar op da ge ',.}{niet te lankn wacht zei streng}{de geestelijke}\\

\haiku{En hij vertrok, den,}{koster verwittigend dat}{hij den volgenden}\\

\subsection{Uit: Het ezelken}

\haiku{Juffrouw Constance.}{was een oude vrijster van}{reeds bij de veertig}\\

\haiku{{\textquoteleft}Geen vrijage, noch,,!}{binnenshuis noch buitenshuis}{of dadelijk weg}\\

\haiku{- K'n weet ik da niet,.}{antwoordde C\'eline dan}{ook heel natuurlijk}\\

\haiku{Eerst vreesde zij nog, -!}{met welken angst van twijfel}{en onzekerheid}\\

\haiku{riep juffrouw Toria,.}{geschokt met grooten mond en}{uitpuilende oogen}\\

\haiku{'t es doarover, '.}{da ik ou kome spreken}{beefdet Ezelken}\\

\haiku{Wie weet ook of het,?}{om haar geld alleen was dat}{hij haar gevraagd had}\\

\haiku{Tok tok tok, hoorde.}{zij de meid aan haar broeder's}{slaapkamer tikken}\\

\haiku{Er was een korte,,.}{poos volkomen doodsche als}{versteende stilte}\\

\haiku{C\'eline's gezicht,.}{en manieren bevielen}{haar niet dien middag}\\

\haiku{H\'e-je nou gezien ''?}{wa veurn schandoal da}{g in ou huis h\^et}\\

\haiku{'k h\`e heur gezeid!}{da ze morgen uchtijnk mee}{pak en zak wig moet}\\

\haiku{zij riep ten slotte;}{haar meid naar binnen en deed}{haar licht aansteken}\\

\haiku{Zij sloot den brief in,.}{het couvert en Aamlie bracht}{hem naar de bus}\\

\haiku{- Ge meug gerust zijn,,.}{ieffreiwe antwoordde de}{meid reeds in de gang}\\

\haiku{ik u door Ivo uwe.}{koffers met alles er in}{wat u toebehoort}\\

\haiku{het spotgelach der,.}{grappenmakers die zich daar}{verscholen hadden}\\

\haiku{'t Puipken stond even,}{onthutst maar v\'o\'or hij er meer}{van kon vertellen}\\

\haiku{Juffer Toria, en,.}{ook het Ezelken waren maar}{half gerustgesteld}\\

\haiku{Het scheelde weinig ' ' -!}{oft glas stortte uits}{Ezelkens hand. Watte}\\

\haiku{Wat moest er van haar,,?}{worden waar moest ze heen als}{juffer Toria stierf}\\

\haiku{Op den drempel van,.}{de gang verscheen de koster}{die hem gevolgd had}\\

\haiku{Mirza moest het eerst,.}{bediend worden die was de}{ongeduldigste}\\

\haiku{- Ik, da huis keupen,, ' ';}{wa peist-ekn h\`e}{doar gien geld veuren}\\

\haiku{- 'K zal de koster,;}{loate roepen zuchtte}{nog eens het Ezelken}\\

\haiku{Kan 't hij nou euk, '! '}{al nie verdroagen datn}{kind zijne stiel liert}\\

\haiku{- Pas moar op da ge ',.}{niet te langn wacht zei streng}{de geestelijke}\\

\haiku{En hij vertrok, den,}{koster verwittigend dat}{hij den volgenden}\\

\subsection{Uit: Het leven van Rozeke van Dalen}

\haiku{Alfons trok de deur '.}{opt nachtslot en stak den}{sleutel in zijn zak}\\

\haiku{{\textquoteright} vroeg opnieuw de stem,.}{norsch-wantrouwend en nu}{heelemaal wakker}\\

\haiku{{\textquoteleft}'k Ben 't ik, boas,{\textquoteright}.}{Van Doalen antwoordde}{Alfons eindelijk}\\

\haiku{zij moesten ook maar eens ';}{om \'e\'en uurs nachts opstaan}{en mee gaan slijten}\\

\haiku{de groote waakhonden.}{blaften in het gerinkel}{van hun kettingen}\\

\haiku{De boer volgde, met.}{onder iederen arm een}{groote flesch jenever}\\

\haiku{Wat dachten ze wel?}{met hun lanterfanten en}{hun gekheid maken}\\

\haiku{Het was stikwarm '.}{in huis ent zweet brak uit}{op de gezichten}\\

\haiku{{\textquoteleft}dag mejonkvreiw en,{\textquoteright}.}{gezelschap en gingen druk}{voort met hun arbeid}\\

\haiku{De duisternis was}{bijna gansch gevallen en}{het onweer trok af}\\

\haiku{zweir mij dat er nie '!}{anders gebeurdn es en}{dat de sloeber liegt}\\

\haiku{{\textquoteleft}Vreiw Urzela van,}{de Weghe verkloart-e}{gij toe te stemmen}\\

\haiku{Het jong begijntje '.}{scheen haar vragend iets int}{oor te fluisteren}\\

\haiku{Ge goat 'n taske '}{seekelou drijnken enn}{boterkoeksken eten}\\

\haiku{Hij glimlachte zoet.}{en nam streelend hare hand}{onder de tafel}\\

\haiku{{\textquoteleft}La mee Rozeke.}{in de wijnkels en ik mee}{Fons ievers elders}\\

\haiku{men wist zelfs niet wie ';}{hij was en of hij opt}{kasteel vertoefd had}\\

\haiku{Zij was gelukkig,.}{door en met Alfons en dat}{maakte alles goed}\\

\haiku{{\textquoteright} verschrikte 't jong,.}{begijntje de handen in}{elkaar geslagen}\\

\haiku{Langzaam en triestig '.}{schudde hijt hoofd en week}{terug naar de deur}\\

\haiku{{\textquoteleft}De loatste coupon, '.}{es vervallen van van doag}{af meugtem knippen}\\

\haiku{Rozeke dankte,.}{maar wist nu verder geen woord}{meer te zeggen}\\

\haiku{Ook de baron, haar,,,.}{vader zag er bekommerd}{somber triestig uit}\\

\haiku{Zij kwamen binnen,:}{terwijl Smul het paard bij den}{stal ging uitspannen}\\

\haiku{{\textquoteright} orakelde hij ruw,.}{met een rechten blik op Dons}{uit de krib komend}\\

\haiku{eerst als een heel fijn,,;}{kleurloos stuifmeel nauw zichtbaar}{in de grijze lucht}\\

\haiku{leupt gij er achter, ';}{aangezien da ge toch nie}{mier verstandn h\^et}\\

\haiku{n speuk, dat rechte!}{lijk ne pijl uit nen bogen}{noar den bosch toe leupt}\\

\haiku{Het vroor en al de ';}{sterren tintelden ins}{hemels donkerblauw}\\

\haiku{Papa en mama.}{dachten dat ik hem op reis}{wel zou vergeten}\\

\haiku{{\textquoteright} riep Rozeke, hoe.}{langer hoe dieper door het}{voorstel afgeschrikt}\\

\haiku{{\textquoteright} {\textquoteleft}Ge meug gerust zijn,,{\textquoteright} '.}{bezinne antwoorddet}{Geluw Meuleken}\\

\haiku{{\textquoteright} zuchtte Rozeke,.}{met de hand het bonzen van}{haar hart bedwingend}\\

\haiku{Een aarzelende.}{voetstap bleef stil-schuivend}{op den drempel staan}\\

\haiku{Gelukkig was het.}{viertal nu reeds weer in druk}{gepraat en gezwets}\\

\haiku{Toen volgden snel, in,:}{rijke bonte kleurschakeering}{al de anderen}\\

\haiku{'t Was nu of nooit.}{het oogenblik om het hem}{te overhandigen}\\

\haiku{En zij waagde de:}{vraag die haar boven alles}{interesseerde}\\

\haiku{of hij voelde pijn.}{in de zij als iemand die}{te hard gerend heeft}\\

\haiku{Blijf maar zitten, blijf,{\textquoteright};}{maar zitten riep dringend de}{jonge barones}\\

\haiku{{\textquoteright} klaagde zij, {\textquoteleft}'k Geef,}{ik hem alles woar da zijn}{herte noar lust moar}\\

\haiku{{\textquoteright} {\textquoteleft}Gij moogt hem vooral,.}{niet laten werken nog van}{heel de zomer niet}\\

\haiku{Reeds lag het vroege;}{lentewerk dringend op den}{akker te wachten}\\

\haiku{en zij wilde ook.}{de min met het wagentje}{doen binnenkomen}\\

\haiku{En hoe moest het op?}{de boerderij ook gaan als}{hij eenmaal weg was}\\

\haiku{Om negen uur kwam '.}{een rijtuig vant kasteel}{Alfons afhalen}\\

\haiku{Maar het verveelde.}{haar toch ook en zij ging er}{een eind aan maken}\\

\haiku{{\textquoteright} riep zij nog, met het.}{Geluw Meuleken naast het}{rijtuig meehollend}\\

\haiku{Het was niet vreemd voor,,.}{haar zij was niet bang het scheen}{haar zoo natuurlijk}\\

\haiku{Smul en Vaprijsken,,.}{gingen er rechts en links als}{wakers naast zitten}\\

\haiku{Haw\`el joa, 't es '!}{precies doarmee datt}{uitgekomen es}\\

\haiku{{\textquoteright} {\textquoteleft}'K 'n weet 't nie,, ';}{bezinne moar iederien}{int dorp zegt het}\\

\haiku{{\textquoteright} Zij schrikte van 't.}{idee alleen dat zij hem zoo}{iets  vragen zou}\\

\haiku{{\textquoteright} En v\'o\'or ze den tijd.}{had nog een woord te spreken}{was hij de deur uit}\\

\haiku{{\textquoteright} {\textquoteleft}'K en weet 't nie,, '.}{bezinne hijn ziet er}{toch moar oardig uit}\\

\haiku{Zij huilde niet, maar.}{de oogen flikkerden vreemd in}{haar doodsbleek gelaat}\\

\haiku{Gij zijt een schurk en.}{uw plaats is niet hier maar in}{de gevangenis}\\

\haiku{- Dit is de eerste,.}{en de allerlaatste keer}{dat ik u waarschuw}\\

\haiku{{\textquoteleft}Het arme beest krijgt,{\textquoteright}.}{in mijn plaats de schoppen en}{de slagen dacht zij}\\

\haiku{alleen de vrees voor.}{andere ongelukken}{beangstigde haar}\\

\haiku{Doch haar hart sloeg kalm.}{en gelijkmatig en zij}{voelde geen emotie}\\

\haiku{Met strak-stukken.}{blik van niet-begrijpen}{staarde zij hem aan}\\

\haiku{En zij ging naar het,,.}{venster bij de wieg waarin}{haar jongste kind lag}\\

\haiku{Maar onverrichter.}{zake keerden zij naar de}{boerderij terug}\\

\haiku{En 't leven ging,,;}{opnieuw zijn tragen stillen}{dagelijkschen gang}\\

\haiku{en wanneer zij hem;}{nog zag was hij bijna als}{een vreemde voor haar}\\

\subsection{Uit: 'n Leeuw van Vlaanderen}

\haiku{Hij waardeerde juist:}{in zijn broeder datgene}{wat hem zelf ontbrak}\\

\haiku{Neen, waarachtig, nooit!}{had ik u met dien baard en}{dat lorgnet herkend}\\

\haiku{s Winters in zijn,;}{huis laat ik maar zeggen zijn}{paleis te Brussel}\\

\haiku{- Wat kunnen mij die!}{lekkere diners en die}{sigaren schelen}\\

\haiku{Bizonder knap, al!}{hebben zij tot nog toe zeer}{weinig gepresteerd}\\

\haiku{doolhof der kleine.}{kronkelige straatjes van}{het oude Brussel}\\

\haiku{Zij waren laat, de;}{anderen zouden reeds op}{hen zitten wachten}\\

\haiku{Dat was tot nu toe ';}{zijn  gedragslijn in}{t leven geweest}\\

\haiku{De eerste rijen,,:}{zeer goed gezeten maakten}{het zich gezellig}\\

\haiku{in om het even welk.}{gezelschap waar de menschen}{ook Vlaamsch verstonden}\\

\haiku{murmelde hij op,.}{zijn beurt met vrome stem en}{glinsterende oogen}\\

\haiku{een groot geheim van...}{onbewuste liefde was}{in hen geboren}\\

\haiku{Haast nergens konden.}{zij een zaal huren om er}{meeting te houden}\\

\haiku{Hij lag te bed, zwak,,,.}{bleek triestig zoowel moreel als}{lichamelijk ziek}\\

\haiku{Want hij dacht er in '!...}{t geheel niet aan den strijd}{nu op te geven}\\

\haiku{liet Desgen\^ets zich,.}{onwillekeurig als een}{angstkreet ontvallen}\\

\haiku{- Gij denkt toch niet, hoop,?}{ik dat ik persoonlijk er}{eenige schuld aan heb}\\

\haiku{Er was geen weerstand:}{mogelijk tegen een zoo}{brutalen aanval}\\

\haiku{Ja, d\`at was wel wat.}{hij zoolang had willen en}{niet durven zeggen}\\

\haiku{Er was geen woord aan,!}{toe te voegen geen woord aan}{te veranderen}\\

\haiku{Als ik zou denken.}{dat je te weinig vraagt zal}{ik je meer geven}\\

\haiku{Na een oogenblik,.}{was hij terug met een blik}{vol hout en steenkool}\\

\haiku{En een misnoegde,.}{bijna minachtende trek}{kwam op zijn gezicht}\\

\haiku{Waarom toch moest nu?}{weer die nare storing in}{zijn leven komen}\\

\subsection{Uit: Het recht van de sterkste}

\haiku{- Maria, laat ze maar,,.}{gaan sprak hij wij zullen ze}{wel achterhalen}\\

\haiku{- Theofiel, jongen,,.}{goa gij naor ou bedde}{dat zal beter zijn}\\

\haiku{En hij ging onder,.}{de sperrekens die rond het}{kapelletje staan}\\

\haiku{Die onverwachte.}{handelwijze liet Balduk}{stom van verbazing}\\

\haiku{riepen haar Witte.}{Manse en de andere}{vrouwen achterna}\\

\haiku{Het water, zwart, lag,.}{onbeweeglijk omringd van}{donker struikgewas}\\

\haiku{zij was reeds veertien,.}{dagen over tijd thans kon ze}{niet meer twijfelen}\\

\haiku{Hij had iets woest, iets,;}{overweldigends dat haar haast}{schrik inboezemde}\\

\haiku{Hij achtervolgde,,.}{haar hoog van gestalte met}{gesloten vuisten}\\

\haiku{Stom van schrik kreeg zij.}{die scheldwoorden als een slag}{in het aangezicht}\\

\haiku{Zij was als van een,.}{ander geslacht als van een}{ander bloed voor hem}\\

\haiku{Witte Manse en.}{haar moeder hadden hem op}{de drempel gevolgd}\\

\haiku{Donder de Beul, Klod.}{de Vos en Smuik Vertriest}{waren de eersten}\\

\haiku{Boef Verwilst drukte.}{met een vloek zijn spijt uit dat}{hij niet mocht schieten}\\

\haiku{Hij stak het plat, scherp.}{uiteinde van zijn hefboom}{eronder en hief}\\

\haiku{En terstond, de daad,.}{bij de woorden voegend liet}{hij zich neerplonsen}\\

\haiku{Zij werden allen,,.}{doch op zeer ongelijke}{manier veroordeeld}\\

\haiku{waarvan zouden zij,,?}{de oude moeder en het}{kind voortaan leven}\\

\haiku{Hij bleef nog enige,.}{stonden sprakeloos de blik}{op haar gevestigd}\\

\haiku{Als door een slag in '.}{t aangezicht kreeg zij het}{bewustzijn terug}\\

\haiku{Ontroerd, geschokt, met,.}{tranen in de ogen knielden}{de bezoekers neer}\\

\haiku{Allen keken op.}{en staarden snuffelend en}{zoekend om zich heen}\\

\subsection{Uit: De roman van den schaatsenrijder}

\haiku{Even buiten 't dorp,,.}{op korten afstand van ons}{huis lag de Lusthof}\\

\haiku{Het kraakt, er komen,.}{sterren in maar het schijnt toch}{te kunnen dragen}\\

\haiku{Dat alles reden.}{wij voortdurend langs en wij}{zagen dat alles}\\

\haiku{Het ijs lag er steeds;}{onbetrouwbaar en had er}{een vuilgele kleur}\\

\haiku{en z\'o\'o reuzesterk,.}{en taai was hij dat hij ons}{niet zelden overwon}\\

\haiku{Wij waren banger}{voor Guus dan voor zijn hond op}{het ijs en haastig}\\

\haiku{Er was een Peetse,,:}{Kins een Bruuntje Geelewie en er}{waren drie broeders}\\

\haiku{Ik keek en hoorde.}{dat alles aan met stillen}{weemoed en emotie}\\

\haiku{hij leek z\'o\'o sprekend,:}{dat ik naar hem toe ging en}{op den man af vroeg}\\

\haiku{Iedereen, oud of,,,.}{jong man of vrouw van klein tot}{groot was bang voor hem}\\

\haiku{- Stien Smijters h\^et de '!}{boantjes op de wal van}{t Oarmhuis vermeurd}\\

\haiku{En eigenlijk wist ':}{ik nooit precies wat mij wel}{t aangenaamst was}\\

\haiku{Er bleef mij trouwens.}{nog ruim voldoende liefde}{en illuzie over}\\

\haiku{En zoo kwam ik, als, ';}{altijd aant heerlijke}{Meylegem-Zuid}\\

\haiku{ik was als van de;}{aarde opgetild en op}{wieken gedragen}\\

\haiku{zij schoven verder ',,;}{overt ijs teeder omarmd}{amoureus-fluisterend}\\

\haiku{Even voorbij de bocht,.}{keek ik eens om en zag dat}{ze mij naoogden}\\

\haiku{Ik voelde dat ik,,.}{indruk had gemaakt ja dat}{ik overwonnen had}\\

\haiku{De Groote Schilder en:}{de Groote Musicus merkten}{het ook en schertsten}\\

\haiku{{\textquoteright} ~ *** ~ Zoo ging ik}{vele dagen wandelen}{en bewonderde}\\

\haiku{Ik verademde.}{alsof ik van een zware}{dreiging werd bevrijd}\\

\haiku{En dat men die toch,,!}{hebben moest en veel om daar}{te kunnen leven}\\

\haiku{en die houden er,.}{ook wel de vroolijkheid in}{wees dat maar zeker}\\

\haiku{Ik haastte mij, ik;}{hijgde en zwoegde door de}{glinsterende sneeuw}\\

\haiku{Ik had het prettig,.}{gevoel dat ik daar een van}{de besten zou zijn}\\

\haiku{- Neen, pardon, zoo niet,,.}{het linkerbeen naar achter}{professeerde ik}\\

\haiku{zij genoot, zij was....!}{tevreden en gelukkig}{gelukkig door mij}\\

\haiku{Het brandde op mijn,.}{lippen om het te vragen}{maar ik durfde niet}\\

\haiku{Het stroomde door mijn;}{heele lichaam heen als een}{electrische trilling}\\

\haiku{den moed hebben mij?}{voor altijd in het vreemde}{land te vestigen}\\

\haiku{Een mensch die stevig;}{gegeten heeft is dikwijls}{log en loom en zwaar}\\

\haiku{Ik liet de dames,.}{voor en trad in een salon}{door Papa gevolgd}\\

\haiku{Er kwamen tranen,....}{in mijn oogen die langzaam over}{mijn wangen vloeiden}\\

\haiku{Dan zeiden weer de.}{oogen wat de mond nog niet had}{durven uitdrukken}\\

\haiku{{\textquoteright} Dat waren uit hun.}{schaal gehaalde en in melk}{gekookte oesters}\\

\haiku{Welke jonge mooie?}{vrouw is niet gelukkig in}{een mode-winkel}\\

\haiku{En Maud beaamde,,.}{door een zwijgend hoofdgeknik}{haar tante's woorden}\\

\haiku{Een ma{\^\i}tre d'h\^otel.}{kwam naar mij toe en bood mij}{stil de wijnkaart aan}\\

\haiku{samen zes dollar '!}{voorn maaltijd die zoowat een}{zestig cent waard is}\\

\haiku{En af en toe was:}{het alsof de slapende}{reus even ontwaakte}\\

\haiku{Als ik mij repte.}{kon ik de electrische van}{half drie nog halen}\\

\haiku{een paar jongelui;}{die lachend pret maakten en}{wat dronken schenen}\\

\subsection{Uit: Tantes}

\haiku{Van Rysselberghe \& /, / [].}{Rombaut C.A.J van Dishoeck Gent}{Bussum z.j.1924}\\

\haiku{Het was een knappe,,.}{man met donker haar mooie snor}{en sprekende oogen}\\

\haiku{Je wist immers wel!}{waar ze te vinden waren}{als je ze noodig had}\\

\haiku{Hij had het moeten...!}{weten weten hoe en wat}{zij voor hem voelde}\\

\haiku{Hij zou eens eventjes;}{met een paar vrienden meegaan}{naar een koffiehuis}\\

\haiku{Een oogenblik had,,.}{Zijne Heiligheid ook naar}{hem Max gekeken}\\

\haiku{Meneer Dufour was.}{al aan de deur en hielp zijn}{zusters uitstijgen}\\

\haiku{Allen, even gestoord,.}{en bijna schrikkend keken}{met verbazing op}\\

\haiku{- Een eer die lastig,.}{is om te dragen meende}{tante Clemence}\\

\haiku{het leek wel of een;}{ongekende ramp over haar}{was neergekomen}\\

\haiku{Oe-Oe, op den grond,.}{Impikoko op een stoel}{en aten met hem mee}\\

\haiku{- Pas moar op da g' '!}{ou nietn verongelukt}{mee al da rijen}\\

\haiku{- Gee moar hier, Manse, ',.}{k zal gauwe genezen}{zijn glimlachte hij}\\

\haiku{riep hij verrast, als.}{naar gewoonte Fransch en Vlaamsch}{door elkaar mengend}\\

\haiku{Wat ons betreft,... wij.}{willen er absoluut niets}{gemeens mee hebben}\\

\haiku{- Wij kunnen hem toch!}{niet beletten hier te paard}{voorbij te rijden}\\

\haiku{zij kregen zelven;}{koffiebezoek van een paar}{dames uit haar dorp}\\

\haiku{voorloopig viel,.}{er niet te aarzelen maar}{te gehoorzamen}\\

\haiku{Het kwam hem voor of.}{een paar lui uit de buurt hem}{spottend nakeken}\\

\haiku{Hij duwde een van.}{zijn vensterluiken open en}{staarde in den nacht}\\

\haiku{Hij voelde 't nu,.}{ineens heel diep het hart vol}{wroeging en verwijt}\\

\haiku{zelve die hem de.}{gevoelens ingaf en de}{woorden dicteerde}\\

\haiku{Clement dus, zei Max,:}{glimlachend met als tweeden}{naam dien van papa}\\

\haiku{De doopsplechtigheid ',.}{int kleine stadje was}{iets buitengewoons}\\

\haiku{Zij zei hem niet, dat.}{zij bij meneer de pastoor}{was te biecht geweest}\\

\haiku{Vroeger, onder zijn,.}{blik sloeg ze dadelijk haar}{oogen neer en kleurde}\\

\haiku{Zij bekwamen er, ';}{niet van zij vielen alst}{ware uit de lucht}\\

\haiku{Max blikte koel in '.}{t onbestemde en streek}{zijn baard naar voren}\\

\haiku{Max deed of hij dat,;}{zeer betreurde hoewel hij}{het begrijpen kon}\\

\haiku{Max ging open doen en:}{zijn dienstmeisje stond voor hem}{en zei fluisterend}\\

\haiku{Zelfs de oude meid;}{Eemlie was met een aardig}{sommetje bedacht}\\

\haiku{zijn huis, zijn bedrijf,,!}{zijn goede gedienstigen}{zijn trouwe dieren}\\

\haiku{Hij zag haar roerloos, ',.}{staan int zwart gekleed met}{een krijtwit gezicht}\\

\haiku{De koetsier tikte.}{zijn paarden en ratelend}{reed het rijtuig weg}\\

\haiku{Zij liepen door een.}{lange gang en hielden bij}{de laatste deur stil}\\

\haiku{Het nonnetje boog '.}{even naart sleutelgat en}{scheen te luisteren}\\

\haiku{Een vraag brandde op,.}{Clara's lippen die zij haast}{niet durfde stellen}\\

\subsection{Uit: Verslagen over den gemeenteraad van Nevele}

\haiku{(Hij haalt zijne beurs):}{uit en geeft hem eenen cens}{H. MEGANCK}\\

\haiku{Ah Meneer L\'eonce, '';}{k en zou e k ik daar}{nie achterloopen}\\

\haiku{Ei, Meneer den Bron, ' ' '?}{est voort kortste of}{voort langste}\\

\haiku{(in eenen lach schietend),, '.}{Ah ja jat es daarom}{dat da voorskomt}\\

\subsection{Uit: Verzameld werk. Deel 1}

\haiku{We kunnen het dan:}{ook met De Cock ten volle}{eens zijn waar hij schrijft}\\

\haiku{Op 28 augustus,,.}{1897 wordt een zoontje Ren\'e}{Cyriel geboren}\\

\haiku{Niet dat we menen.}{dat het werk van Buysse een}{pleidooi nodig heeft}\\

\haiku{De beide jonkmans,,.}{koutend en schertsend stapten}{nevens de meisjes}\\

\haiku{- Theofiel, jongen,,.}{goa gij naor ou bedde}{dat zal beter zijn}\\

\haiku{En hij ging onder,.}{de sperrekes die rond het}{kapelletje staan}\\

\haiku{Die onverwachte.}{handelwijze liet Balduk}{stom van verbazing}\\

\haiku{riepen haar Witte.}{Manse en de andere}{vrouwen achterna}\\

\haiku{Het water, zwart, lag,.}{onbeweeglijk omringd van}{donker struikgewas}\\

\haiku{zij was reeds veertien,.}{dagen over tijd thans kon ze}{niet meer twijfelen}\\

\haiku{- Enfin, ik peins het,:}{toch verbeterde zij haar}{eerste gezegde}\\

\haiku{Om te zien of mijn. '! '!}{slinger daar goed voor wast}{Zal gaant Zal gaan}\\

\haiku{Hij had iets woests, iets,;}{overweldigends dat haar haast}{schrik inboezemde}\\

\haiku{Hij achtervolgde,,.}{haar hoog van gestalte met}{gesloten vuisten}\\

\haiku{Stom van schrik kreeg zij.}{die scheldwoorden als een slag}{in het aangezicht}\\

\haiku{Ongetwijfeld was.}{hij er op dat ogenblik reeds}{mee op weg naar Gent}\\

\haiku{Ditmaal geraakten.}{zij nog ongedeerd uit de}{klauwen van de Wet}\\

\haiku{Zij was als van een,.}{ander geslacht als van een}{ander bloed voor hem}\\

\haiku{Ze staat daar weeral! -,.}{riep Manse soms eensklaps de}{voordeur opentrekkend}\\

\haiku{Witte Manse en.}{haar moeder hadden hem op}{de drempel gevolgd}\\

\haiku{Donder de Beul, Klod.}{de Vos en Smuik Vertriest}{waren de eersten}\\

\haiku{Boef Verwilst drukte.}{met een vloek zijn spijt uit dat}{hij niet mocht schieten}\\

\haiku{Hij stak het plat, scherp.}{uiteinde van zijn hefboom}{eronder en hief}\\

\haiku{En terstond, de daad,.}{bij de woorden voegend het}{hij zich neerplonsen}\\

\haiku{bladeren, takken,.}{kruiden en stukken hout in}{zijn vaart meeslepend}\\

\haiku{Zij werden allen,,.}{doch op zeer ongelijke}{manier veroordeeld}\\

\haiku{waarvan zouden zij,,?}{de oude moeder en het}{kind voortaan leven}\\

\haiku{Hij bleef nog enige,.}{stonden sprakeloos de blik}{op haar gevestigd}\\

\haiku{Als door een slag in '.}{t aangezicht kreeg zij het}{bewustzijn terug}\\

\haiku{Ontroerd, geschokt, met,.}{tranen in de ogen knielden}{de bezoekers neer}\\

\haiku{Allen keken op.}{en staarden snuffelend en}{zoekend om zich heen}\\

\haiku{Een bloedkleurige,:}{vlam brandde beneveld in}{de rook de kreten}\\

\haiku{De verbrande plek.}{in het bed werd met een wit}{linnen doek bedekt}\\

\haiku{Een straal schoot uit zijn,.}{oog een glimlach van geluk}{kwam op zijn lippen}\\

\haiku{In een oogwenk stond,.}{het rijpaard gezadeld had}{hij zijn sporen aan}\\

\haiku{En Gilbert was ten;}{slotte gelukkig over de}{eerste uitkomsten}\\

\haiku{Terstond hielden de.}{gesprekken op en allen}{werden zeer ernstig}\\

\haiku{De jongelieden,,.}{zich neerzettend bestelden}{ieder een glas bier}\\

\haiku{Was hun tijdschrift daar,?}{wellicht niet gekomen dat}{niemand ervan sprak}\\

\haiku{Een glimlach was op;}{de lippen van enkele}{leden verschenen}\\

\haiku{Gilbert hoorde en.}{staarde hem aan met een}{strelend genoegen}\\

\haiku{{\textquoteleft}Bonjour ma{\^\i}t' De Roo{\textquoteright},,.}{il lui tendait son journal}{le Petit Flamand}\\

\haiku{Langzaam, steeds zonder,,.}{een woord zonder een gebaar}{volgde mijnheer haar}\\

\haiku{Wat kon het hem thans?}{nog schelen of zijn feest goed}{dan slecht gelukte}\\

\haiku{Een duizeling van;}{geluk bedwelmde zijn geest}{bij die gedachte}\\

\haiku{Hij herlas nog eens,.}{zijn brief vond hem goed en sloot}{hem in zijn omslag}\\

\haiku{zij voelt er zich ten;}{hoogste door vereerd en dankt}{u uit dien hoofde}\\

\haiku{Die dochter, naar het,.}{schijnt is van een zekere}{schoonheid niet ontbloot}\\

\haiku{de hand. Zijn zending,.}{was volbracht de jongeman}{wilde vertrekken}\\

\haiku{dat Ir\`ene verloofd.}{was en ging trouwen met haar}{neef Jozef De Moor}\\

\haiku{De datum voor de:}{echtverbintenis was}{nog niet vast bepaald}\\

\haiku{zij vernederen;}{de harten die zij zouden}{moeten verheffen}\\

\haiku{zij zijn de schuld, dat!}{het bloed van de mensen met}{beken heeft gestroomd}\\

\haiku{V\'o\'or hem  strekte,;}{zich een grasplein uit omzoomd}{met zwarte lovers}\\

\haiku{Bevend hief Gilbert.}{zijn bijna opgebrande}{lucifer omhoog}\\

\haiku{Doch hij moest, er viel,,...}{niet te aarzelen en hij}{ijlde hij ijlde}\\

\haiku{Drieghe deed op de.}{drempel zijn klompen uit en}{zij traden binnen}\\

\haiku{Maar nogmaals slaakte.}{zij een wilde kreet en viel}{terug in onmacht}\\

\haiku{Het schrikbeeld van 't,;}{gebeurde achtervolgde}{obsedeerde hem}\\

\haiku{Lauwereijnssens riep,.}{het artikel Drie op het}{woonhuis van Gilbert}\\

\haiku{anders krijgen wij, '!}{er vandaag niet mee gedaan}{ent moet gedaan}\\

\haiku{'t Was of eensklaps.}{een sluier van v\'o\'or Gilberts}{ogen werd getrokken}\\

\haiku{Hij ging, hij dwaalde.}{op goed geluk af zonder}{te weten waarheen}\\

\haiku{rampzaliger dan,,?}{ooit dodelijk getroffen}{om er te sterven}\\

\haiku{Bij plaatsen waren;}{de wegen er door onkruid}{en bramen bedekt}\\

\haiku{Ik ben z\'o zwak, z\'o,,}{ontroerd z\'o ziek dat ik u}{om zo te zeggen}\\

\haiku{dat zijn minnares,;}{ook wegliep met haar schreiend}{kind in de armen}\\

\haiku{Hij liep de spoorbaan,.}{over doorkruiste haastig het}{verlaten stadspark}\\

\haiku{De letters dansten;}{hem v\'o\'or zijn ogen en hij kon}{ook niet stilzitten}\\

\haiku{- O spreek tot mij, help,?}{me door uw raad zeg me wat}{mij nog te doen staat}\\

\haiku{Toen lispelde zij,,:}{eindelijk met een doffe}{troosteloze stem}\\

\haiku{En voor ons beiden,,:}{zal het genoegen zoals}{immer dubbel zijn}\\

\haiku{Och neen, het is geen,.}{kwaadspreken want ik heb ze}{in de grond wel lief}\\

\haiku{Tot morgen om drie, ',?}{uur dus aant station}{niet waar Ren\'e}\\

\haiku{dat het verleden,,;}{het lief verleden z\'o kalm}{en zo gelukkig}\\

\haiku{Helaas, tante was.}{nu dood en haar kinderen}{hadden zich verspreid}\\

\haiku{Daar keerde hij zich.}{om en staarde weer in de}{richting van de stad}\\

\haiku{Van beide rampen,,.}{bijna onvermijdbaar was}{dit nog de zachtste}\\

\haiku{Hij het haar los, en,.}{met een soort van schrik keken}{zij elkander aan}\\

\haiku{Maar plotseling hield:}{zij op en staarde hem aan}{met ogen  vol schrik}\\

\haiku{Vol onrust stak hij.}{de brief opzij en opende}{die van Raymonde}\\

\haiku{Zij kreeg opnieuw haar.}{onbeweeglijke houding}{van zieltogende}\\

\haiku{in de grijze lucht,.}{dreven loom en traag benden}{krassende raven}\\

\haiku{- Deze morgen moet.}{de overledene in haar}{kist gelegd worden}\\

\haiku{De hinderpaal die,;}{tussen beiden oprees was}{eensklaps verdwenen}\\

\haiku{aan alle kanten.}{om hem heen was het lijden}{en vertwijfeling}\\

\haiku{Nu begrijpt ge, dat?}{ik het recht verbeurd heb nog}{aan u te denken}\\

\haiku{elk ogenblik worden,.}{er onrechtvaardigheden}{misdaden gepleegd}\\

\haiku{zou d\'a\'ar toch waarlijk?}{nog een absolutie voor}{de zondaar liggen}\\

\haiku{Lange tijd had de.}{edelman van dergelijk plan}{niet willen horen}\\

\haiku{- Nathalie, zult gij,?}{alles wel sluiten en vuur}{en licht uitdoven}\\

\haiku{Zij draaide de lamp, '.}{een weinig lager deed haar}{deur int nachtslot}\\

\haiku{Buiten had de wind '.}{zich als het ware vant}{kasteel verwijderd}\\

\haiku{De dikke aderen;}{van haar kloeke hals zwollen}{onheilspellend op}\\

\haiku{Doch zij alleen vond:}{zijn handelwijze verkeerd}{en onnatuurlijk}\\

\haiku{Had ze gedurfd, ze;}{zou er aan haar tante van}{gesproken hebben}\\

\haiku{Hij zit daar pruilend;}{in een hoek met een boek dat}{hij zelfs niet meer leest}\\

\haiku{Zenobie bukte 't,;}{hoofd vuurrood van schaamte en}{tot wenens ontroerd}\\

\haiku{Focho was blijven,.}{staan verdween hij achter de}{donkere sparren}\\

\haiku{een tijdlang in het.}{huisje van de tuinman geen}{voet meer te zetten}\\

\haiku{Doch haar toestand, in,;}{plaats van te verbeteren}{verslechtte zichtbaar}\\

\haiku{- Nochtans gij hebt nu,,}{uw nichtje mejuffrouw de}{Rorick  bij u.}\\

\haiku{Zij heeft over haar mooie;}{trekken een zweem van ietwat}{treurige zachtheid}\\

\haiku{Doch niet in 't minst.}{voelde de oude jonkvrouw}{zich gerustgesteld}\\

\haiku{En daar Nina, meer,:}{en meer ontroerd en verbaasd}{zich excuseerde}\\

\haiku{Zij bleef daar staan en,;}{draalde bijna hopend hem}{te zien verschijnen}\\

\haiku{Zij riep Focho, die,,.}{vooruitrende bij zich en}{sloeg de zijlaan in}\\

\haiku{Zij schrikte er haast,}{van het kwam haar voor alsof}{hij daar ineens v\'o\'or}\\

\haiku{Ik zweer u dat ik.}{eerlijk en rechtschapen met}{u zal handelen}\\

\haiku{Intussen kunt ge,.}{mij voortdurend schrijven ik}{zal u antwoorden}\\

\haiku{Men zegt soms dat het ',;}{t Noodlot is dat over ons}{levenslot beschikt}\\

\haiku{Verdelg de jonge,;}{nog zo tengere bloempjes}{van de bomen niet}\\

\haiku{En 't is of de:}{natuur haar angstig gebed}{wilde verhoren}\\

\haiku{het Vermogen van...,!}{de Liefde O hoe innig}{voelt ze dit ook nu}\\

\haiku{Eerbiedig gingen.}{de dokter en Marie een}{weinig opzij staan}\\

\haiku{Hij wilde spreken,,,.}{vloeken schreeuwen en kon geen}{woord meer uitbrengen}\\

\haiku{- Nee, maar was ik er,,?}{niet gekomen het zou toch}{w\'el gebeurd zijn h\`e}\\

\haiku{- Waar? - In de keuken,.}{terwijl Pier-Cies bij u}{op de akker was}\\

\haiku{En nu was hij z\'o,.}{aan haar gewend dat hij haar}{niet meer missen kon}\\

\haiku{hij staakte het, en,.}{ging nu naar de hoogmis die}{Pol ook bijwoonde}\\

\haiku{Eens, toen ze van de,,:}{markt kwam had hij haar gekruist}{op een smal eenzaam}\\

\haiku{Ondanks al zijn aplomb,.}{keek de bezoeker enigszins}{verbauwereerd op}\\

\haiku{En, om het pijnlijk:}{gesprek in een andere}{wending te brengen}\\

\haiku{En ineens, bij dat,,.}{zicht ging er in hem een groot}{vreselijk licht op}\\

\haiku{Werktuiglijk, met een,.}{ruwe stoot duwde hij die}{open en trad binnen}\\

\haiku{En toch was er iets.}{onheilspellends in die slechts}{schijnbare vrede}\\

\haiku{'s Nachts vooral leed.}{hij nog heviger onder}{die vreemde kwelling}\\

\haiku{Men zou het er zich ':}{allergezelligst maken}{int warme stro}\\

\haiku{En langzaam buigt hij, '...}{zich de rechterhand int}{donker uitgestrekt}\\

\haiku{En in zijn woestheid,:}{schreeuwt hij woorden waarvan zij}{de zin niet begrijpt}\\

\haiku{Het spel is eerlijk,.}{zijn gang gegaan dat hebben}{wij allen gezien}\\

\haiku{Al wat ik weet is,;}{dat die schelm uit de pot een}{frank gestolen heeft}\\

\haiku{nu was zijn naam toch.}{voor altijd in het dorp met}{hoon en spot bedekt}\\

\haiku{Wat had zijn neef hem,?}{toch misdaan dat hij hem zo}{vreselijk haatte}\\

\haiku{Trouwens, hoe kon de,?}{jongeling het weten dat}{ze de zijne was}\\

\haiku{En in het dorp zelf, ',.}{kwam hij nooit meer zelfs niets}{zondags voor de mis}\\

\haiku{Hij sloot moedwillig;}{zijn ogen en zijn oren voor het}{akelig tafereel}\\

\haiku{Het landschap, om hem,:}{heen strekte zich heerlijk uit}{in zijn lentepracht}\\

\haiku{En even dacht hij aan,,.}{Rosa heel even maar zonder}{kwellend verlangen}\\

\haiku{- Dat komt er niet op,,.}{aan ik weet het toch klonk het}{besliste antwoord}\\

\haiku{Hij waardeerde juist:}{in zijn broeder datgene}{wat hemzelf ontbrak}\\

\haiku{Neen, waarachtig, nooit!}{had ik u met die baard en}{dat lorgnet herkend}\\

\haiku{- Wat kunnen mij die!}{lekkere diners en die}{sigaren schelen}\\

\haiku{Bijzonder knap, al!}{hebben zij tot nog toe zeer}{weinig gepresteerd}\\

\haiku{Zij waren laat, de;}{anderen zouden reeds op}{hen zitten wachten}\\

\haiku{Er is in u geen!}{schim van bewustzijn van uw}{hogere ikheid}\\

\haiku{De eerste rijen,,:}{zeer goed gezeten maakten}{het zich gezellig}\\

\haiku{murmelde hij op,.}{zijn beurt met vrome stem en}{glinsterende ogen}\\

\haiku{Werktuiglijk keek hij.}{op en zijn blik ontmoette}{die van Ghislaine}\\

\haiku{een groot geheim van...}{onbewuste liefde was}{in hen geboren}\\

\haiku{Hun rijtuig hield vlak,,;}{tegenover de kerk midden}{op de dorpsplaats stil}\\

\haiku{Want hij dacht er in '!...}{t geheel niet aan de strijd}{nu op te geven}\\

\haiku{liet Desgen\^ets zich,.}{onwillekeurig als een}{angstkreet ontvallen}\\

\haiku{- Gij denkt toch niet, hoop,?}{ik dat ik persoonlijk er}{enige schuld aan heb}\\

\haiku{Er was geen weerstand:}{mogelijk tegen een zo}{brutale aanval}\\

\haiku{Ja, d\'at was wel wat.}{hij zolang had willen en}{niet durven zeggen}\\

\haiku{Er was geen woord aan,!}{toe te voegen geen woord aan}{te veranderen}\\

\haiku{XVIII Twee dagen.}{gingen voorbij zonder dat}{er iets gebeurde}\\

\haiku{Als ik zou denken.}{dat je te weinig vraagt zal}{ik je meer geven}\\

\haiku{En een misnoegde,.}{bijna minachtende trek}{kwam op zijn gezicht}\\

\haiku{Ik eet zo weinig, '.}{mogelijk vlees ik hebn}{gruwel aan vleeseten}\\

\haiku{Waarom toch moest nu?}{weer die nare storing in}{zijn leven komen}\\

\haiku{- Kijk daar eens naar, sprak,.}{hij haar voor het beeld van de}{Gioconda brengend}\\

\haiku{En in 't bos zelf:}{ontmoette men slechts zeldzaam}{enkele wezens}\\

\haiku{De een, Cosaque, was een,,;}{mooie grote wit-en-bruin}{gevlekte barzoi}\\

\haiku{'t Is toch ook zo '!}{ontzettend rijk van kleur in}{t volle daglicht}\\

\haiku{en ik... zal mama... '.}{inviteren en misschien}{n paar vriendinnen}\\

\haiku{Florence trad met,.}{Maxime en Paul de trappen}{af hen tegemoet}\\

\haiku{- Maxime, jij kent ze,,?}{allen niet waar zal jij de}{voorstellingen doen}\\

\haiku{- een hansworst als die,.}{jonge snob een gansje als}{die dikke Elise}\\

\haiku{- O, ik had haar niet,!}{mogen laten gaan of ik}{had mee moeten gaan}\\

\haiku{Hij zou haar nu maar,.}{niet naar Oostende volgen}{dat zou zwakheid zijn}\\

\haiku{elle qui est si,!}{jeune  et si jolie}{et si distingu\'ee}\\

\haiku{en samen gingen.}{zij door bossen en lanen}{naar Far-West terug}\\

\haiku{Hij was echter vast.}{besloten haar met koelheid}{te bejegenen}\\

\haiku{- Zoals je wilt, beet,.}{zij eindelijk kortaf met}{sissende lippen}\\

\haiku{Hij k\'on het zo niet,.}{langer uithouden er moest}{een eind aan komen}\\

\haiku{- Liever de dood, ja,!}{liever de dood dan zulk een}{ellendig leven}\\

\haiku{Werktuiglijk, versterkt,}{door zijn besluit liep hij steeds}{verder langs het meer}\\

\haiku{Je zou ze haast zo.}{spoedig hebben dan als je}{ze zelf gaat halen}\\

\haiku{Soms  hield hij strak,;}{de hand aan zijn kin als in}{gespannen denken}\\

\haiku{Ik vind zelfs dat je '.}{r lang over gedaan hebt v\'o\'or}{het te ontdekken}\\

\haiku{Zij kan het enkel,.}{doen uit vrije wil indien haar}{aard er haar toe noopt}\\

\haiku{een aantal vrouwen;}{kunnen hebben in plaats van}{maar \'e\'en enkele}\\

\haiku{Enkele heren.}{bedienden zich en staken}{sigaretten op}\\

\haiku{Alfred hield zich of;}{dat vlugge besluit hem niet}{eens verwonderde}\\

\haiku{Hij stapte met Cosaque ',.}{int bootje en roeide}{naar het eilandje}\\

\haiku{En hij knelde zijn.}{vuisten ineen en sloot zijn}{tanden op elkaar}\\

\haiku{Zijn ogen strakstaarden,.}{erop als in bedwelming}{op cijfers van vuur}\\

\haiku{Ik dacht dat je bij,.}{de Loebmullers of bij de}{Berlaimonts was}\\

\haiku{- Madame, as 't, '...}{ou belieftk zoe zeu geirn}{iets van ou weten}\\

\haiku{Moar enfin, eefer;}{de Cuup\`ere es meschien heur}{hoemoaksterigge}\\

\haiku{want weet je, mooi heb, - '.}{ik haar nooit kunnen vinden}{zon modepop}\\

\haiku{Jouw naam, onze naam;}{moet aan de publieke spot}{onttrokken worden}\\

\haiku{Haar ziel, haar liefde,,;}{dat wist hij wel was al lang}{voor hem verloren}\\

\haiku{- De hoofdzaak is, dat '.}{alles nu gaat zoals we}{t hebben willen}\\

\haiku{Oneindig was de.}{stilte en de eenzaamheid}{die hem omringde}\\

\haiku{Waarom liet ze hem,?}{niet met rust nu toch alles}{tussen hen dood was}\\

\haiku{Hij had geen zin meer '.}{haar terug te nemen en}{daarmee wast uit}\\

\haiku{dacht hij, als ik haar?}{plotseling om de hoek van}{een straat ontmoette}\\

\haiku{Ik zie er zeker,.}{heel heel anders uit dan toen}{ik nog je vrouw was}\\

\haiku{Hoe vreemd en vals die,,.}{woorden klonken begreep hij}{niet voelde hij niet}\\

\haiku{het licht er als 't.}{ware van gouden glans over}{de bloemenvlakte}\\

\haiku{XXXIII En zachtjes...}{aan gaf hij zich aan de macht}{der bekoring over}\\

\haiku{Tussen tien en elf....}{zou hij bij Florence zijn}{Verzameld werk}\\

\subsection{Uit: Verzameld werk. Deel 2}

\haiku{De Strijd, de eerste,, \&,.}{uitgave Rotterdam Nijgh}{Van Ditmar 1918}\\

\haiku{- 'k Ben 't ik, boas,.}{Van Doalen antwoordde}{Alfons eindelijk}\\

\haiku{praatte hij met luid,;}{galmende stem alsof hij}{op de akker was}\\

\haiku{In het duistere;}{van de  nacht kon hij niets}{van haar gezicht zien}\\

\haiku{zij moesten ook maar eens ';}{om \'e\'en uurs nachts opstaan}{en mee gaan slijten}\\

\haiku{En alles om hen:}{heen kreeg nu ook meer en meer}{vaste vorm en kleur}\\

\haiku{Wat dachten ze wel?}{met hun lanterfanten en}{hun gekheid maken}\\

\haiku{Het was stikwarm '.}{in huis ent zweet brak uit}{op de gezichten}\\

\haiku{De klokkeslag van,:}{de lange trage uren was}{zo gauw geslagen}\\

\haiku{{\textquoteleft}dag mejonkvreiw en{\textquoteright},.}{gezelschap en gingen druk}{voort met hun arbeid}\\

\haiku{zweir mij dat er nie '!}{anders gebeurdn es en}{dat de sloeber liegt}\\

\haiku{Het jong begijntje '.}{scheen haar vragend iets int}{oor te fluisteren}\\

\haiku{- La mee Rozeke.}{in de wijnkels en ik mee}{Fons ievers elders}\\

\haiku{dat die jongen b'ron}{d'r euk bij woare en dat}{hij ons van den oavond}\\

\haiku{men wist zelfs niet wie ';}{hij was en of hij opt}{kasteel vertoefd had}\\

\haiku{En bij uw ouders,.}{Leo en Marie Dezen avond}{het avondmaal zult eten}\\

\haiku{Zij was gelukkig,.}{door en met Alfons en dat}{maakte alles goed}\\

\haiku{Zij haalde uit de,;}{eetkast twee grote witte}{koppen en een bord}\\

\haiku{verschrikte 't jong,.}{begijntje de handen in}{elkaar geslagen}\\

\haiku{Langzaam en triestig '.}{schudde hijt hoofd en week}{terug naar de deur}\\

\haiku{Hij vouwde een van,:}{de stukken open wees op het}{couponsblad en zei}\\

\haiku{- De loatste coupon, '.}{es vervallen van vandoag}{af meugtem knippen}\\

\haiku{Ook de baron, haar,,,.}{vader zag er bekommerd}{somber triestig uit}\\

\haiku{Zij kwamen binnen,:}{terwijl Smul het paard bij de}{stal ging uitspannen}\\

\haiku{orakelde hij ruw,.}{met een rechte blik op Dons}{uit de krib komend}\\

\haiku{eerst als een heel fijn,,;}{kleurloos stuifmeel nauw zichtbaar}{in de grijze lucht}\\

\haiku{leupt gij er achter, ';}{aangezien da ge toch nie}{mier verstandn h\^et}\\

\haiku{Het vroor en al de ';}{sterren tintelden ins}{hemels donkerblauw}\\

\haiku{Papa en mama.}{dachten dat ik hem op reis}{wel zou vergeten}\\

\haiku{riep Rozeke, hoe.}{langer hoe dieper door het}{voorstel afgeschrikt}\\

\haiku{- Ge meug gerust zijn,, '.}{bezinne antwoorddet}{Geluw Meuleken}\\

\haiku{zuchtte Rozeke,.}{met de hand het bonzen van}{haar hart bedwingend}\\

\haiku{- Hahaha!... 't Spijt ' '!}{mij dak nie liever op}{ou gewedn h\`e}\\

\haiku{Gelukkig was het.}{viertal nu reeds weer in druk}{gepraat en gezwets}\\

\haiku{Wat werd het eensklaps,}{stil in Rozekes leven}{na al de drukte}\\

\haiku{En zij waagde de:}{vraag die haar boven alles}{interesseerde}\\

\haiku{of hij voelde pijn.}{in de zij als iemand die}{te hard gerend heeft}\\

\haiku{- Blijf maar zitten, blijf,;}{maar zitten riep dringend de}{jonge barones}\\

\haiku{- Gij moogt hem vooral,.}{niet laten werken nog van}{heel de zomer niet}\\

\haiku{Reeds lag het vroege;}{lentewerk dringend op de}{akker te wachten}\\

\haiku{En toch... sommige:}{dingen kon noch mocht zij zo}{niet blijven dulden}\\

\haiku{en zij wilde ook.}{de min met het wagentje}{doen binnenkomen}\\

\haiku{En hoe moest het op?}{de boerderij ook gaan als}{hij eenmaal weg was}\\

\haiku{Om negen uur kwam '.}{een rijtuig vant kasteel}{Alf ons afhalen}\\

\haiku{, zijn beide ogen dof, '.}{en doods nut gelaat haast}{zonder uitdrukking}\\

\haiku{Maar het verveelde.}{haar toch ook en zij ging er}{een eind aan maken}\\

\haiku{riep zij nog, met het.}{Geluw Meuleken naast het}{rijtuig meehollend}\\

\haiku{Het was niet vreemd voor,,.}{haar zij was niet bang het scheen}{haar zo natuurlijk}\\

\haiku{Smul en Vaprijsken,,.}{gingen er rechts en links als}{wakers naast zitten}\\

\haiku{- Haw\`el joa, 't es '!}{precies doarmee datt}{uitgekomen es}\\

\haiku{Zij stond met hoge;}{kleur te beven en wist niet}{meer wat te zeggen}\\

\haiku{En v\'o\'or ze de tijd.}{had nog een woord te spreken}{was hij de deur uit}\\

\haiku{- 'k En weet 't nie,, '.}{bezinne hijn ziet er}{toch moar oardig uit}\\

\haiku{Zij huilde niet, maar.}{de ogen flikkerden vreemd in}{haar doodsbleek gelaat}\\

\haiku{- Gij zijt een schurk en.}{uw plaats is niet hier maar in}{de gevangenis}\\

\haiku{- Dit is de eerste,.}{en de allerlaatste keer}{dat ik u waarschuw}\\

\haiku{- Het arme beest krijgt,.}{in mijn plaats de schoppen en}{de slagen dacht zij}\\

\haiku{alleen de vrees voor.}{andere ongelukken}{beangstigde haar}\\

\haiku{Doch haar hart sloeg kalm.}{en gelijkmatig en zij}{voelde geen emotie}\\

\haiku{Met strak-stugge.}{blik van niet-begrijpen}{staarde zij hem aan}\\

\haiku{En zij ging naar het,,.}{venster bij de wieg waarin}{haar jongste kind lag}\\

\haiku{En 't leven ging,,;}{opnieuw zijn trage stille}{dagelijkse gang}\\

\haiku{en wanneer zij hem;}{nog zag was hij bijna als}{een vreemde voor haar}\\

\haiku{Nonkelken was rijk.}{en had nooit anders dan voor}{zijn plezier geleefd}\\

\haiku{het leek hem dat hij ';}{eensklaps zou genezen zijn}{zodra hijt had}\\

\haiku{hij wist nog niet waar,;}{de bouwlanden lagen waar}{de huizen stonden}\\

\haiku{- Doet er mee lijk of,,.}{ge wilt veur mij es alles}{goed zei meneer Vit\`al}\\

\haiku{ging plotseling een.}{zware stem aan de verste hoek}{van de tafel op}\\

\haiku{Da z' heur nonkel ', '.}{nien ha ze zoe moeten}{inn kleuster goan}\\

\haiku{Meneer Vit\`al was tot.}{nog toe in geen enkele}{dorpsherberg geweest}\\

\haiku{Het was te dom, hij,.}{had er eensklaps genoeg van}{van die stompe lui}\\

\haiku{De flinke tocht door;}{de avondkoelte had hem weer}{eetlust gegeven}\\

\haiku{Het meisje schudde,:}{langzaam ietwat verlegen}{glimlachend het hoofd}\\

\haiku{Hij was nu weer goed.}{en gelukkig gestemd en}{dacht een grapje uit}\\

\haiku{Hij boog zich over 't;}{tafeltje waaraan zij met}{hun vieren zaten}\\

\haiku{en haar uiterlijk '.}{was hem opt eerste zicht}{niet meegevallen}\\

\haiku{en ditmaal had ze.}{op hem een heel andere}{impressie gemaakt}\\

\haiku{Hij proefde er even,.}{van trok een zuur gezicht en}{schoof het dan opzij}\\

\haiku{hij voelde zelf een;}{soort gezelligheid om daar}{nog wat te blijven}\\

\haiku{Maar ook al proefde, ';}{hij er slechts even vant werd}{op de duur toch veel}\\

\haiku{Maar hij zag aan haar}{verbouwereerd gezicht dat}{ze zijn honende}\\

\haiku{had hij dan niet meer '?}{het recht te leven zoals}{t hem behaagde}\\

\haiku{Zou ze misschien van...}{de gelegenheid gebruik}{willen maken om}\\

\haiku{Meneer Vit\`al hield zich '.}{of hijt niet hoorde en}{liet maar volop gaan}\\

\haiku{Waarom nu weer dat!}{wilde en gevaarlijke}{snorren door de nacht}\\

\haiku{De Reu loerde even,;}{wantrouwig om naar Netje}{die in huis verdween}\\

\haiku{Meneer Vit\`al fronste.}{de wenkbrauwen en staarde}{peinzend v\'o\'or zich uit}\\

\haiku{- Hij kende niemand.}{hier die met hem voelen en}{genieten kon}\\

\haiku{Toen zag hij ook de,;}{enkele korte regels}{van het fijn geschrift}\\

\haiku{Kom moar mee lijk of}{ge zijt en zend iemand noar}{huis om te zeggen}\\

\haiku{zich boos gebarend.}{en haar hand uitslaand als om}{een klap te geven}\\

\haiku{herhaalde Taghon,.}{met nadruk de ogen rond en}{strak van overtuiging}\\

\haiku{'t was of nu ook;}{een vijand h\'a\'ar van hem zou}{kunnen wegvoeren}\\

\haiku{Ook vader Peutrus ':}{en moeder Lie waren heel}{int zwart gekleed}\\

\haiku{Al dat lawaai en.}{gewoel overweldigde haar}{en gaf haar hoofdpijn}\\

\haiku{Een zoete wraak zou ',.}{t voor hem zijn indien hij}{daarin slagen kon}\\

\haiku{en v\'o\'or 't naar bed,.}{gaan leidde hij haar even rond}{in al de kamers}\\

\haiku{- 'k Zal tegen ten!}{halver ien were thuis zijn}{om te dineren}\\

\haiku{- Toe, Nathelie, leupt;}{al gauwe noar ou keuken}{en schept de soep uit}\\

\haiku{Dat zou ook niet staan,.}{ginds in die achterhoek van}{De Groene Linde}\\

\haiku{- O, en ik die toch,.}{zeu geirne sampoande}{drijnke klaagde zij}\\

\haiku{De bontjes van 't;}{barontje waren rood en die}{van meneer Vit\`al groen}\\

\haiku{gilde plotseling.}{meneer Vit\`al met van woede}{uitpuilende ogen}\\

\haiku{Hij hoorde, als een, ';}{vaag echot geschreeuw van de}{menigte ginds ver}\\

\haiku{Het deed hem goed haar,.}{zo te zien haar dicht en trouw}{bij zich te voelen}\\

\haiku{As ik euk moar {\textquoteleft}'t{\textquoteright} ',.}{bolleken nien krijge}{gelijk Nonkelken}\\

\haiku{Was dat leven voor '?}{t meisje t\'och te saai en}{te eenzaam geweest}\\

\haiku{Marguerite.}{scheen geheel weer tot kalmte}{en rust gekomen}\\

\haiku{- 't Is de schande,,!}{de oneer de ondergang}{van de familie}\\

\haiku{Meer dan zulke en.}{dergelijke was er uit}{hem niet te krijgen}\\

\haiku{- Nog niet, antwoordde,.}{stil mevrouw Dudemaine even}{naar hem omkijkend}\\

\haiku{- Dwing niet, dwing niet, denk ',.}{aant verleden snikte}{mevrouw bleek van angst}\\

\haiku{Laten wij hem toch.}{eerst en vooral gezond en}{sterk zien te maken}\\

\haiku{- Kom mee, en papa,,!}{ook er is iets gebeurd en}{ik moet het weten}\\

\haiku{- Dat is er een van!}{de bediening of van de}{boerderij geweest}\\

\haiku{- Doe ou wa klieren,,.}{aan M\'edard en kom mee}{noar de boerderije}\\

\haiku{- O, M\'edard, as 't, ',,!}{u blieft ast u blieft helpt}{hem blijf toch bij hem}\\

\haiku{riep de man hees van,.}{schrik terwijl hij zijn geweer}{haastig terugtrok}\\

\haiku{En zij liep zelf met,.}{het glas naar de kelder haar}{moeder achterna}\\

\haiku{'t Gemoed van de ';}{mensen was in stemming met}{t omgevende}\\

\haiku{En langzaam liep hij,, '.}{verder vaag teleurgesteld}{naart dorpje toe}\\

\haiku{zij merkten, en met!}{welk een zoet gestreelde hoop}{en vertedering}\\

\haiku{dat kraken van het, '!}{ijs wat boordet door haar}{benen en haar hart}\\

\haiku{- 'k Wou maar dat 't,,.}{uit was en er sneeuw kwam om}{te kunnen arren}\\

\haiku{Hij was boos, zowel,;}{als Elsa die een hoge}{kleur kreeg van emotie}\\

\haiku{Wat wilde hij toch?}{eigenlijk en wat moest ze}{met hem aanvangen}\\

\haiku{Die avond, toen hij aan,.}{de buitenrand verscheen was}{zij er niet te zien}\\

\haiku{Op 't Landjuweel,,.}{galmde luid-vermanend}{de eerste etensbel}\\

\haiku{een man uit het dorp,.}{een jonge boer of knecht uit}{de omtrek binnen}\\

\haiku{maar het gebeurde:}{ook dat vreemdelingen zich}{daar even ophielden}\\

\haiku{- 't Is goed, M\'edard,,.}{jij hoeft niet verder mee te}{komen zei mevrouw}\\

\haiku{de Wet zou er zich,!}{mee bemoeien want het kind}{was minderjarig}\\

\haiku{Maar de nood dwong, er;}{moest onmiddellijk iets op}{gevonden worden}\\

\haiku{Zijn ogen weifelden,.}{richtten zich even strak en als}{bedwelmd ten gronde}\\

\haiku{Haar hart joeg sneller,.}{een weke emotie glansde}{in baar bleke ogen}\\

\haiku{Iets woelde in haar,,.}{scherp-kwellend diep van}{onrust en van angst}\\

\haiku{Wel voelde ze soms:}{af en toe een kort verwijt}{in zich opkomen}\\

\haiku{En ofschoon tegen, '.}{haar zin had zet hem niet}{kunnen beletten}\\

\haiku{Mevrouw Dudemaine.}{poogde er hem ook niet meer}{van af te houden}\\

\haiku{Zij zwegen beiden. '}{om naar het loeien van de}{wind te luisteren}\\

\haiku{riep hij, eensklaps haar.}{nijdig wegduwend en zelf}{de deur openrukkend}\\

\haiku{- Ala toe, moeder, kom, '!}{binnen en doe de deure}{toet es hier koud}\\

\haiku{Hij keek eens vluchtig,;}{naar de deur of hij daar ook}{soms iemand hoorde}\\

\haiku{een pastoor in haar -.}{familie te bezitten}{aan hem was volbracht}\\

\haiku{Juffrouw Constance.}{was een oude vrijster van}{reeds bij de veertig}\\

\haiku{- Geen vrijage, noch,,!}{binnenshuis noch buitenshuis}{of dadelijk weg}\\

\haiku{Dat was erg, z\'o erg, '.}{dat zet haast niet kon noch}{wilde geloven}\\

\haiku{riep juffrouw Toria,.}{geschokt met grote mond en}{uitpuilende ogen}\\

\haiku{'t es doarover ', '.}{dak ou kome spreken}{beefdet Ezelken}\\

\haiku{Wie weet ook of het,?}{om haar geld alleen was dat}{hij haar gevraagd had}\\

\haiku{Tok tok tok, hoorde.}{zij de meid aan haar broeders}{slaapkamer tikken}\\

\haiku{zij greep ineens de...}{knop van de keukendeur en}{duwde die ruw open}\\

\haiku{Er was een korte,,.}{poos volkomen doodse als}{versteende stilte}\\

\haiku{C\'elines gezicht,.}{en manieren bevielen}{haar niet die middag}\\

\haiku{H\'e-je nou gezien ''!}{wa veurn schandoal da}{g in ou huis h\^et}\\

\haiku{zij riep ten slotte;}{haar meid naar binnen en deed}{haar licht aansteken}\\

\haiku{Zij sloot de brief in,.}{het couvert en Aamlie bracht}{hem naar de bus}\\

\haiku{- Ge meug gerust zijn,,.}{ieffreiwe antwoordde de}{meid reeds in de gang}\\

\haiku{ik u door Ivo uwe.}{koffers met alles er in}{wat u toebehoort}\\

\haiku{Het Puipken, trouwens,;}{voelde zich daar ook blijkbaar}{niets op zijn gemak}\\

\haiku{'t Puipken stond even,}{onthutst maar v\'o\'or hij er meer}{van kon vertellen}\\

\haiku{Juffer Toria, en,.}{ook het Ezelken waren maar}{half gerustgesteld}\\

\haiku{Het scheelde weinig ' '.}{oft glas stortte uit de}{hand vant Ezelken}\\

\haiku{Wat moest er van haar,,?}{worden waar moest ze heen als}{juffer Toria stierf}\\

\haiku{Op de drempel van,.}{de gang verscheen de koster}{die hem gevolgd had}\\

\haiku{'t Was verwacht en '.}{alst ware reeds gebeurd}{v\'o\'or het gebeurd was}\\

\haiku{Mirza moest het eerst,.}{bediend worden die was de}{ongeduldigste}\\

\haiku{- Ik, da huis keupen,, ' ';}{wa peist-ekn h\`e}{doar gien geld veuren}\\

\haiku{- 'k Zal de koster,;}{loate roepen zuchtte}{nog eens het Ezelken}\\

\haiku{Kan 't hij nou euk, '! '}{al nie verdroagen datn}{kind zijne stiel liert}\\

\haiku{- Pas moar op da ge ',.}{niet te lankn wacht zei streng}{de geestelijke}\\

\haiku{De oude vrek zat.}{met uitgepuilde ogen op}{zijn stoel te schudden}\\

\haiku{- 'k Zal 't ulder, ' ',;}{teugenk zalt ulder}{teugen snikte Ivo}\\

\haiku{De hevige schok;}{had Guustje plotseling van}{zijn kwaal genezen}\\

\haiku{Een dikke schoof stro,.}{waarmee men de hond in zijn}{hok vastgepropt had}\\

\haiku{Toen liepen zij even,.}{rondom hun stallen of ook}{d\'a\'ar soms onraad was}\\

\haiku{- Haw\`el, bezinne, ' ', ' ':}{os get nien weetk}{zalt ou zeggen}\\

\haiku{- Ha, wilt-e nou,',:}{ne kier wa weten onz Mrie}{riep krijsend de vrouw}\\

\haiku{- 't Es goed, zei de,.}{politieman kortaf zijn}{boekje dichtklakkend}\\

\haiku{Machinaal streek hij.}{met de hand over zijn wang en}{keek naar zijn vingers}\\

\haiku{Eerst werd voorspeld dat;}{geen fatsoenlijk mens er nog}{de voet zou zetten}\\

\haiku{- Mrie, 'k ko... ko... 'k!}{kom ou vroagen of da ge}{mee mij wilt treiwen}\\

\haiku{Wat 'n emotie, toen!}{dat span voor de eerste maal}{op het hoevetje kwam}\\

\haiku{Guustje zag het en '.}{het beet hem even als een scherp}{venijn int hart}\\

\haiku{- 'k Begin al woarm,, {\textquoteleft}{\textquoteright}.}{te krijgen zei P\'e\'elzie haar}{katte loshakend}\\

\haiku{- Ieffreiwe, geef mij,.}{euk ne kier da boeksken dat}{doar op de toafel ligt}\\

\haiku{- Joa joa 't, 't es,.}{nou al firm k\^ewd beaamde}{Guustje bibberend}\\

\haiku{Maar eensklaps werd hun.}{aandacht door iets anders in}{beslag genomen}\\

\haiku{'k H\`e k\^ewwe, ',.}{k h\`e doanig k\^ewwe}{bibbertandde Ivo}\\

\haiku{{\textquoteleft}Heb je 't nu goed!}{begrepen of moet ik het}{nog eens herhalen}\\

\haiku{Er waren tal van,;}{acrobaten die blijkbaar een}{vertoning gaven}\\

\haiku{de orang-oetang, ook,;}{helemaal een mens die slechts}{de spraak te kort had}\\

\haiku{Als grote, witte.}{schepen dreven die wolken}{in de blauwe lucht}\\

\haiku{Elvire zou als;}{een verwend kindje haar zin}{krijgen met Fonske}\\

\haiku{Eerst had de moeder,,.}{zo onverhoeds gepakt wel}{enige aarzeling}\\

\haiku{En hij lachte heel.}{hard om zijn buitengewoon}{geestig gezegde}\\

\haiku{Hij hoopte wel, dat.}{hij er twee of drie van de}{hand zou kunnen doen}\\

\haiku{- Al was ie-hij de, ',.}{Paus wen kennen hem niet}{zei Sylvain smalend}\\

\haiku{hij begon ook in:}{andere kunstuitingen}{belang te stellen}\\

\haiku{hij huurde het per,;}{brief van de baron aan wie}{het toebehoorde}\\

\haiku{Hij haatte hem, had,.}{hem kunnen slaan hem van haar}{kunnen wegrukken}\\

\haiku{Zij waren aan zijn ',.}{huisje ent speet hem dat}{zij er reeds waren}\\

\haiku{Fonske wou, ondanks,.}{Van Belleghems aandringen}{niet binnenkomen}\\

\haiku{en, zoals ik in,.}{die tijd schreef zou ik thans niet}{meer k\'unnen schrijven}\\

\haiku{Boerke was een kort,,.}{dik ventje van om en bij}{de vijfenzestig}\\

\haiku{Zij keken allen,,;}{op verrast alsof zij hem}{pas nu ontwaarden}\\

\haiku{en Tibi, van zijn,:}{kant was nog geweldiger}{dan bij het heengaan}\\

\haiku{het erf bleef stil en,;}{leeg en leeg en stil lag ook}{de weg daarbuiten}\\

\haiku{Maar de ochtend leek,.}{dan zo lang dat er geen eind}{aan scheen te komen}\\

\haiku{Het gewoon kleindorpse,;}{alledaagse leven scheen}{ver en vergeten}\\

\haiku{Wat was het land schoon!}{met de alom rijpende}{oogsten en vruchten}\\

\haiku{Tibi volgde hem,,.}{met trage schreden en bleef}{staan waar Boerke stond}\\

\haiku{Die alledaagse.}{bejegening ontstemde}{Boerke vagelijk}\\

\haiku{Waarom hij nog kwam,,.}{ja dat kon hij zelf ook zo}{precies niet zeggen}\\

\haiku{Boerke schudde 't.}{hoofd en keek beteuterd naar}{het knappe meisje}\\

\haiku{Maar voor hun rust was}{plaats en ruimte nodig en}{reeds zo ontelbaar}\\

\haiku{Moar hij 'n kan mee '.}{heur nie treiwen omdat hij}{gien fortuunn h\'et}\\

\haiku{vroeg ze nog, ziende.}{dat hij zijn pijp aanstak en}{naar de voordeur ging}\\

\haiku{Machinaal deed hij.}{zijn klompen uit en nam die}{in de linkerhand}\\

\haiku{Een klein lampje was.}{aangestoken en brandde}{op een tafeltje}\\

\haiku{Meteen sloeg hij, van,.}{terzijde aandachtig de}{gezichten gade}\\

\haiku{Maar zelfs in een rijk.}{huwelijk had hij tot nog}{toe geen zin gehad}\\

\haiku{De levenloze,.}{dingen schenen te rusten}{evenals de mensen}\\

\haiku{menier den dokteur,, '.}{kreet Meerken alsof haart}{ergste werd gevraagd}\\

\haiku{Dat was toch nog een,,.}{afleiding iets eigens iets}{uit het verleden}\\

\haiku{Zij wilden weten,:}{w\'eten en Roomnietje drong}{opgewonden aan}\\

\haiku{Het leek wel of hij '.}{zelf int geheel niet meer}{wist wat hij wenste}\\

\haiku{Kom, Sefrien, help mij ', ''!}{n beetsen dak z in}{mijn oarms kan pakken}\\

\haiku{Zij kenden hen door ' '}{en door ent was ermee}{te leven zoals}\\

\haiku{de gasten liepen.}{langzaam langs de ramen en}{keken naar binnen}\\

\haiku{- As ik iets veur ou,, '!}{kan doen gelijk watte ge}{n moet moar spreken}\\

\haiku{- Zie je wel, klaagde, -!}{Reinilde we zijn al veel}{te lank gebleven}\\

\haiku{Het drong hem als een.}{prop de keel dicht en hij kreeg}{tranen in zijn ogen}\\

\haiku{Maar Keijsder wou.}{zijn paard niet alleen laten}{staan en bedankte}\\

\haiku{Daar stond ginds ver de.}{molen te draaien dichtbij}{zijn vroeger gehucht}\\

\haiku{Hij ademde diep en.}{voelde nieuwe kracht door heel}{zijn lichaam stromen}\\

\haiku{De knecht was een reeds;}{bejaarde man en deed hem}{aan Sefrien denken}\\

\haiku{Hij glimlachte bij:}{het idee en meteen was zijn}{besluit genomen}\\

\haiku{- D'r komen nog al,.}{dikkels brieven van Oscar}{berichtte Roze}\\

\haiku{Hij schoof zijn stoel dicht.}{bij de hare en sloeg zijn}{arm om haar middel}\\

\haiku{Hij liet haar los, en,.}{beiden keken naar elkaar}{met sprekende ogen}\\

\haiku{Weer steeg de gloedgolf.}{over Florimonds gelaat en}{bleekte langzaam weg}\\

\haiku{Wie het waagde zijn;}{vader te mishandelen}{was zijn leven kwijt}\\

\haiku{hij sloot zijn raam en,.}{holde de trappen af om}{dat bij te wonen}\\

\haiku{anderen namen,.}{lachend wat mee zonder hun}{beurs uit te halen}\\

\haiku{hij liep, hij liep maar, '.}{door alst ware door het}{noodlot voortgejaagd}\\

\haiku{Hij zag van verre ',;}{t huisje staan van Dikke}{Roze ongedeerd}\\

\subsection{Uit: Verzameld werk. Deel 3}

\haiku{werken waarin hij [}{vooral herinneringen}{ophaalt of verwerkt}\\

\haiku{{\textquoteleft}De 29ste juli de.}{begrafenis bijgewoond}{van Cyriel Buysse}\\

\haiku{Even buiten 't dorp,,.}{op korte afstand van ons}{huis lag de Lusthof}\\

\haiku{Het kraakt, er komen,.}{sterren in maar het schijnt toch}{te kunnen dragen}\\

\haiku{Dat alles reden.}{wij voortdurend langs en wij}{zagen dat alles}\\

\haiku{Het ijs lag er steeds;}{onbetrouwbaar en had er}{een vuilgele kleur}\\

\haiku{en z\'o reuzesterk,.}{en taai was hij dat hij ons}{niet zelden overwon}\\

\haiku{Wij waren banger}{voor Guus dan voor zijn hond op}{het ijs en haastig}\\

\haiku{Er was een Peetse,,:}{Kins een Bruuntje Geelewie en er}{waren drie broeders}\\

\haiku{Ik keek en hoorde.}{dat alles aan met stille}{weemoed en emotie}\\

\haiku{hij leek z\'o sprekend,:}{dat ik naar hem toe ging en}{op de man af vroeg}\\

\haiku{Iedereen, oud of,,,.}{jong man of vrouw van klein tot}{groot was bang voor hem}\\

\haiku{En eigenlijk wist ':}{ik nooit precies wat mij wel}{t aangenaamst was}\\

\haiku{Eigenlijk voelde,.}{ik mij daar nooit zoals ik}{was of wezen wou}\\

\haiku{Er bleef mij trouwens.}{nog ruim voldoende liefde}{en illusie over}\\

\haiku{zo zwart alsof het.}{open water was en niemand}{durfde er overheen}\\

\haiku{En zo kwam ik, als, ';}{altijd aant heerlijke}{Meilegem-Zuid}\\

\haiku{ik was als van de;}{aarde opgetild en op}{wieken gedragen}\\

\haiku{zij schoven verder ',,;}{overt ijs teder omarmd}{amoureus-fluisterend}\\

\haiku{Ik wist dat zij het, ',;}{was ik hoordet aan haar}{stem dat zij het was}\\

\haiku{Even voorbij de bocht,.}{keek ik eens om en zag dat}{ze mij naoogden}\\

\haiku{Ik voelde dat ik,,.}{indruk had gemaakt ja dat}{ik overwonnen had}\\

\haiku{'t Was ook te gek,.}{zoals die steedse meisjes}{zich daar aanstelden}\\

\haiku{Ik schreide goed en,;}{diep uit daar in de stilte}{en de eenzaamheid}\\

\haiku{Ik verademde.}{alsof ik van een zware}{dreiging werd bevrijd}\\

\haiku{Ik bestelde een.}{tweede pousje en stak een}{grote sigaar op}\\

\haiku{En dat men die toch,,!}{hebben moest en veel om daar}{te kunnen leven}\\

\haiku{en die houden er,.}{ook wel de vrolijkheid in}{wees dat maar zeker}\\

\haiku{Ik haastte mij, ik;}{hijgde en zwoegde door de}{glinsterende sneeuw}\\

\haiku{Ik had het prettig,.}{gevoel dat ik daar een van}{de besten zou zijn}\\

\haiku{- Neen, pardon, zo niet,,.}{het linkerbeen naar achter}{professeerde ik}\\

\haiku{zij genoot, zij was...!}{tevreden en gelukkig}{gelukkig door mij}\\

\haiku{Het brandde op mijn,.}{lippen om het te vragen}{maar ik durfde niet}\\

\haiku{Wat zal er van mij,,?}{worden dacht ik meteen als}{het eens niet meer vriest}\\

\haiku{Het stroomde door mijn;}{hele lichaam heen als een}{elektrische trilling}\\

\haiku{'t Was stralend mooi,.}{weer nog mooier dan al de}{vorige dagen}\\

\haiku{Een mens die stevig;}{gegeten heeft is dikwijls}{log en loom en zwaar}\\

\haiku{Zij vergezelden,.}{mij om van mijn verrukking}{mee te genieten}\\

\haiku{Ik liet de dames,.}{voor en trad in een salon}{door Papa gevolgd}\\

\haiku{Er kwamen tranen,...}{in mijn ogen die langzaam over}{mijn wangen vloeiden}\\

\haiku{Dan zeiden weer de.}{ogen wat de mond nog niet had}{durven uitdrukken}\\

\haiku{{\textquoteright} Dat waren uit hun.}{schaal gehaalde en in melk}{gekookte oesters}\\

\haiku{Welke jonge mooie?}{vrouw is niet gelukkig in}{een modewinkel}\\

\haiku{En Maud beaamde,,.}{door een zwijgend hoofdgeknik}{haar tantes woorden}\\

\haiku{Een ma{\^\i}tre d'h\^otel.}{kwam naar mij toe en bood mij}{stil de wijnkaart aan}\\

\haiku{En af en toe was:}{het alsof de slapende}{reus even ontwaakte}\\

\haiku{Vraag excuus, smeek om, '!}{genade ofk sla je}{de dikke kop in}\\

\haiku{Als ik mij repte.}{kon ik de elektrische van}{halfdrie nog halen}\\

\haiku{een paar jongelui;}{die lachend pret maakten en}{wat dronken schenen}\\

\haiku{De halve nacht heeft,.}{ze gehuild na het diner}{bij Delmonico}\\

\haiku{Het troebleerde hem,;}{gaf hem soms iets rusteloos}{en ontevredens}\\

\haiku{maar zij zouden 't:}{wel gelaten hebben hem}{ooit weg te sturen}\\

\haiku{Men zag hem telkens;}{weer door sleutelgaten en}{door reetjes loeren}\\

\haiku{en dan was hij wel.}{eens de geestigste man van}{de ganse fabriek}\\

\haiku{Het kloppen op zijn,;}{bochel was een gewone}{dagelijkse grap}\\

\haiku{Zij zat daar als een,.}{mooie bloem of luxeplant als iets}{dat er niet hoorde}\\

\haiku{Bruteijn had er nu.}{eenmaal genoeg van en was}{niet meer te vangen}\\

\haiku{Soms was het lief en,.}{vriendelijk maar ook wel eens}{bezorgd en gedrukt}\\

\haiku{Het meest geraden;}{was in elk geval daar niet}{te lang te toeven}\\

\haiku{Het kwam al gauw tot;}{wederzijdse scheldwoorden}{en dreigementen}\\

\haiku{Het enige wat hij:}{er blijvend bij gewonnen}{had was zijn bijnaam}\\

\haiku{schreeuwde Guustje met, {\textquoteleft}{\textquoteright}.}{een stem die men tot in het}{stampkot horen kon}\\

\haiku{Eleken, het tweede,.}{meisje diende met stille}{bewegingen op}\\

\haiku{De vrouwen droegen,;}{zonhoeden die hun gezicht}{en hals beschutten}\\

\haiku{Men kende ze, die!}{zondagsgangen van Berzeel}{naar zijn eigen dorp}\\

\haiku{- Ze zoen ons beter ',.}{elkn keiw kieken geven}{mee sloa meende hij}\\

\haiku{En plotseling sloeg,.}{hij hem neer zo hard als hij}{kon in Pierkens nek}\\

\haiku{Hij was groot en zwaar,.}{met een dikke hangsnor en}{geverfde haren}\\

\haiku{Iek beveel u nog,.}{eens uit te scheid herhaalde}{de burgemeester}\\

\haiku{De Witte{\textquoteright} kreeg een.}{tranencrisis alsof ze}{heel zou wegsmelten}\\

\haiku{- Hij zoe wek willen, ',.}{moar hijn mag nie van zijn}{moeder snikte zij}\\

\haiku{riep Feelken ieder,,;}{ogenblik uit louter dolle}{uitgelatenheid}\\

\haiku{Zij spraken nooit van, {\textquoteleft}{\textquoteright}.}{een liter altijd van een}{kilo jenever}\\

\haiku{Meneer De Beule,,.}{het hart vol wrok vermeed met}{zijn zoon te spreken}\\

\haiku{Nu was dat alles,.}{dood zoals zijn vreugde met}{haar weggaan dood was}\\

\haiku{Hij had het even over,,.}{zaken op een bezorgde}{chagrijnige toon}\\

\haiku{En dat troostend en,.}{sterkend gevoel bleef hem bij}{enkele dagen}\\

\haiku{Zou hij dan nooit die?}{ellendige kruk in de}{duisternis vinden}\\

\haiku{Daarbinnen in het.}{huisje was het plotseling}{doodstil geworden}\\

\haiku{Toen ging eensklaps een,,.}{stem op een vrouwenstem die}{ietwat angstig vroeg}\\

\haiku{Hij stond daar nu en.}{wist eensklaps niet meer wat te}{doen of te zeggen}\\

\haiku{Hij wenkte stil de}{moeder bij zich en stopte}{het haar in de hand.}\\

\haiku{En zou de zoon hem?}{bij de keel niet grijpen en}{hem buiten smijten}\\

\haiku{- Ge weet dat toch wel,,.}{antwoordde zij stil met een}{blos de ogen neerslaand}\\

\haiku{en Leo liet zijn  , {\textquoteleft}!}{wilde schreeuw horen zijn luid}{bulderendOajo\'aek}\\

\haiku{{\textquoteright} dat wellicht tot in.}{het woonhuis door meneer De}{Beule werd gehoord}\\

\haiku{Of hij ooit met haar;}{zou trouwen bleef geheel in}{het onzekere}\\

\haiku{Vader had al niet.}{veel in te brengen thuis en}{Meries nog minder}\\

\haiku{Daarbuiten lag de,,,.}{sneeuw de kilheid de onrust}{de onzekerheid}\\

\haiku{Maar eensklaps toornden.}{toch even de woede en de}{opstand in hem los}\\

\haiku{Het geluid van zijn.}{stem overheerste het gedonder}{van de heibalken}\\

\haiku{Zou ze wellicht reeds '?}{iets weten en zouden ze}{t daarover hebben}\\

\haiku{Als hij nu maar geen, '.}{mensen ontmoette dan zou}{hijt wel halen}\\

\haiku{Kaboel, die ook mee,.}{wou kreeg de deur tegen zijn}{neus en piepte even}\\

\haiku{streelde Lisatje, met.}{vertedering de zachte}{wangetjes aaiend}\\

\haiku{- 't Denkt mij dat 't,,.}{nog koel es buiten meende}{madam De Beule}\\

\haiku{Hij hoorde ook zijn.}{vader en zijn moeder loom}{de trap opklimmen}\\

\haiku{Aan tafel sprak hij;}{geen woord en keek zijn zoon geen}{enkele keer aan}\\

\haiku{Of wist meneer De?}{Beule soms dat hij weer bij}{Siednie was geweest}\\

\haiku{Zijn lange pijp hing,,;}{er tussen twee spijkers in}{de schoorsteenmantel}\\

\haiku{- Ha moar, meniere, '!}{k kom hier die ijzere}{roe were brengen}\\

\haiku{Dit avontuur liet in;}{hem een wrange desem van}{verbittering na}\\

\haiku{en Pierken stelde:}{voor dat een deputatie}{van drie arbeiders}\\

\haiku{er kwam leven in.}{de loom-gedrukte groep}{en de ogen blonken}\\

\haiku{riep Pierken eensklaps...}{met een soort van wilde trots}{op zijn borst kloppend}\\

\haiku{We willen meinschen,!}{worden meniere en gien}{lastdieren mier zijn}\\

\haiku{Maar Eleken zei nooit,.}{veel mengde zich liefst in geen}{verwikkelingen}\\

\haiku{Zij ging weer naar de;}{haverkist toe en vulde}{ditmaal gul de maat}\\

\haiku{Om halfzeven kwam,,.}{als elke dag meneer De}{Beule beneden}\\

\haiku{- Ha, om mijn wirk te,, '.}{doene meniere schrikte}{t meisje hevig}\\

\haiku{En hij verliet de '.}{stal om aant gesprek een}{eind te maken}\\

\haiku{Het drietal stond met;}{de rug naar hem toe en had}{hem niet zien komen}\\

\haiku{Joa joa, med\'am, past '!}{op os g'in d'handen vant}{slecht vreiwevolk zit}\\

\haiku{De mannen hadden;}{gegroet zonder een ogenblik}{hun werk te staken}\\

\haiku{tot aan de kirke ' '!}{willen kruipen ast nie}{gebeurdn woare}\\

\haiku{en hij kwam strak v\'o\'or:}{Sefietje staan en begon}{te bromneuri\"en}\\

\haiku{Het speet hun niet dat.}{Pier en Feelken nu uit de}{fabriek wegbleven}\\

\haiku{Was alles weer in,?}{orde nu of moest er nog}{over gepraat worden}\\

\haiku{Een laddertje stond,;}{naast hem langswaar hij blijkbaar}{opgeklommen was}\\

\haiku{en zijn gezicht leek,,.}{zwart met uithangende tong}{alsof hij walgde}\\

\haiku{Hij keek om zich heen,.}{als gehinderd door de te}{drukke omgeving}\\

\haiku{- Mais... nous n'avons plus,.}{rien \`a faire ici meende}{de burgemeester}\\

\haiku{- Est-ce que,?}{vous n'avez besoin de rien}{monsieur le cur\'e}\\

\haiku{- Wacht totdat menier,.}{de p\'aster wig es antwoordde}{meneer De Beule}\\

\haiku{- Ge zilt loater,.}{allemoal mijn veurbeeld}{volgen zei Pierken}\\

\haiku{en keek haar neef  ,.}{strak in de ogen aan terwijl}{zij hem de hand gaf}\\

\haiku{Het was een knappe,,.}{man met donker haar mooie snor}{en sprekende ogen}\\

\haiku{Raymond spaarde zijn;}{vriend de schimpscheuten en}{de verwijten niet}\\

\haiku{Hij had het moeten...!}{weten weten hoe en wat}{zij voor hem voelde}\\

\haiku{Hij zou eens eventjes;}{met een paar vrienden meegaan}{naar een koffiehuis}\\

\haiku{Zij hadden nog niet,.}{alle hoop verloren maar}{zij vreesden zozeer}\\

\haiku{Zo'n magnifieke,!}{stad en die drukte en dat}{vrolijke leven}\\

\haiku{Meneer Dufour was.}{al aan de deur en hielp zijn}{zusters uitstijgen}\\

\haiku{herhaalde tante,.}{Estelle haar handen vroom}{in elkander slaand}\\

\haiku{Allen, even gestoord,.}{en bijna schrikkend keken}{met verbazing op}\\

\haiku{- Een eer die lastig,.}{is om te dragen meende}{tante Clemence}\\

\haiku{het leek wel of een;}{ongekende ramp over hen}{was neergekomen}\\

\haiku{Oe-Oe, op de grond,,.}{Impikoko op een stoel}{en aten met hem mee}\\

\haiku{- Pas moar op da g' '!}{ou nietn verongelukt}{mee al da rijen}\\

\haiku{- Gee moar hier, Manse, ',.}{k zal gauwe genezen}{zijn glimlachte hij}\\

\haiku{riep hij verrast, als.}{naar gewoonte Frans en Vlaams}{door elkaar mengend}\\

\haiku{Wat ons betreft,... wij.}{willen er absoluut niets}{gemeens mee hebben}\\

\haiku{- Wij kunnen hem toch!}{niet beletten hier te paard}{voorbij te rijden}\\

\haiku{Het kwam hem voor of.}{een paar lui uit de buurt hem}{spottend nakeken}\\

\haiku{Hij duwde een van.}{zijn vensterluiken open en}{staarde in de nacht}\\

\haiku{Hij voelde 't nu,.}{ineens heel diep het hart vol}{wroeging en verwijt}\\

\haiku{Clement dus, zei Max -:}{glimlachend met als tweede}{naam die van papa}\\

\haiku{De doopplechtigheid, ',.}{int kleine stadje was}{iets buitengewoons}\\

\haiku{Zij zei hem niet, dat.}{zij bij meneer de pastoor}{was te biecht geweest}\\

\haiku{Vroeger, onder zijn,.}{blik sloeg ze dadelijk haar}{ogen neer en kleurde}\\

\haiku{Zij bekwamen er, ';}{niet van zij vielen alst}{ware uit de lucht}\\

\haiku{Max blikte koel in '.}{t onbekende en streek}{zijn baard naar voren}\\

\haiku{Max deed of hij dat,;}{zeer betreurde hoewel hij}{het begrijpen kon}\\

\haiku{Max ging opendoen en:}{zijn dienstmeisje stond voor hem}{en zei fluisterend}\\

\haiku{Zelfs de oude meid;}{Eemlie was met een aardig}{sommetje bedacht}\\

\haiku{zijn huis, zijn bedrijf,,!}{zijn goede gedienstigen}{zijn trouwe dieren}\\

\haiku{Hij zag haar roerloos, ',.}{staan int zwart gekleed met}{een krijtwit gezicht}\\

\haiku{zo gaan we kalmpjes...,...}{aan naar huis en Raymond blijft}{bij ons ja zeker}\\

\haiku{De koetsier tikte.}{zijn paarden en ratelend}{reed het rijtuig weg}\\

\haiku{Zij liepen door een.}{lange gang en hielden bij}{de laatste deur stil}\\

\haiku{Het nonnetje boog '.}{even naart sleutelgat en}{scheen te luisteren}\\

\haiku{- Ja, ze slaapt, z'is nog, ';}{wat moe zeit nonnetje}{zich weer oprichtend}\\

\haiku{Een vraag brandde op,.}{Clara's lippen die zij haast}{niet durfde stellen}\\

\haiku{Wellicht had ze soms,.}{andere verlangens doch}{zij uitte die niet}\\

\haiku{Zij was in 't zwart.}{gekleed en had haar hoed met}{roze bloemen op}\\

\haiku{Toen zij enkele,}{minuten had gelopen}{kwam haar bij een bocht}\\

\haiku{'k Zal mee iene, ';}{van de twie\"e treiwen as ik}{nie verdern kan}\\

\haiku{'t Was toch iets uit,;}{haar eigen leven dat zo}{wegstierf en verdween}\\

\haiku{riep Ulekens moeder, -!}{opgewonden we moen dat}{toch uek ne kier zien}\\

\haiku{en nu eerst voelde}{Uleken zo duidelijk de}{afstand en hoe wijs}\\

\haiku{Zonder een woord, met,.}{een kort gebaar sloeg ze ruw}{zijn hand van zich af}\\

\haiku{riep Broosp\`er, toen hij aan,.}{zijn hek was en ook zijn vrouw}{wenste goeden avond}\\

\haiku{Nu was 't de beurt.}{van Uleken om een plotse}{vuurkleur te krijgen}\\

\haiku{Als gij ooit iets te,.}{beleggen hebt vergeet niet}{naar hem toe te gaan}\\

\haiku{antwoordde Uleken, - ' '!}{zeer beslistkn goa in}{gien slechte huizen}\\

\haiku{Stanus bleef, zoals ',.}{t betaamde bij zijn vrouw}{de thuiswacht houden}\\

\haiku{het ingepropte.}{wicht dan in haar armen op}{en neer deed dansen}\\

\haiku{maar Uleken, evenmin, {\textquoteleft}{\textquoteright};}{als Natsen scheen haarsoorte}{te kunnen vinden}\\

\haiku{Uleken wist nu wel,;}{heel vast en zeker dat zij}{niet meer trouwen zou}\\

\haiku{was er toch ook een:}{andere troost in Ulekens}{leven gekomen}\\

\haiku{Eerst was het meer uit,.}{gevoel van plicht dat zij er}{zich mee bemoeide}\\

\haiku{Fielemiene en.}{Stanus bekeken elkaar}{en hun ogen straalden}\\

\haiku{'t Was donker in.}{de slaapkamer en nog steeds}{kwam er geen antwoord}\\

\haiku{- Als huefd van de ''.}{gemiente est zijn plicht}{ulder t helpen}\\

\haiku{De kleine zakte.}{op zijn stoel en sloeg met zijn}{vork op de tafel}\\

\haiku{- We zijn op wig noar,.}{de sampitter of hij ons}{wilt helpen zoeken}\\

\haiku{vroeg ze enkel, op,.}{nederige toon als om}{hem te vermurwen}\\

\haiku{Gedurende een.}{paar seconden waren angst}{en smart vergeten}\\

\haiku{het was alsof ze.}{daar zaten te bidden om}{een onzichtbaar lijk}\\

\haiku{Haar stem klonk schor en.}{diep in de stilte bij het}{vlammengeknetter}\\

\haiku{Er werd buiten aan.}{de deur geklopt en meteen}{waren zij binnen}\\

\haiku{Zij aarzelde om.}{te gaan zitten en bleef maar}{liever wachtend staan}\\

\haiku{- Ach joa, natuurlijk,;}{ge zij gij veel te jonk om}{mevreiwe te zijn}\\

\haiku{Zij liep in zichzelf '.}{te glimlachen en dacht weer}{aant verleden}\\

\haiku{Uleken was trouwens;}{bereid al de onkosten}{op zich te nemen}\\

\haiku{Als een vechthond vloog ' '}{Allewies Cesar opt}{lijf ent ogenblik}\\

\haiku{Maar ondertussen.}{was er iets nieuws in Ulekens}{leven gekomen}\\

\haiku{- 't Es het ginter ',!}{al verre lijk hier int}{kluester tante}\\

\haiku{maar een hoongelach.}{weerklonk en de kroegdeur werd}{hard dichtgeslagen}\\

\haiku{- Meschien zal onze,!?}{lieven Hier ons wijzen wat}{da we moeten doen}\\

\haiku{Van alles, wat zij;}{daar en op haar eigen dorp}{destijds geleerd had}\\

\haiku{vroeg ze, zo gewoon,.}{mogelijk doend zodra het}{meisje binnen was}\\

\haiku{Peis ne kier wat dat!}{de meinschen doarvan zoen}{keunen zeggen}\\

\haiku{Hij noemde haar ook,.}{al tante of hij reeds van}{de familie was}\\

\haiku{maar er werkte een,,...}{geheime fatale kracht}{over haar die haar dwong}\\

\haiku{dat hij niet gezeid ' ' '!}{n hee datk hem int}{Frans moe aanspreken}\\

\haiku{Telkens had hij 't,,.}{daarover de laatste tijd als}{hij in verlof kwam}\\

\haiku{Alles was eensklaps.}{zo heel anders en zoveel}{gemakkelijker}\\

\haiku{Met Lauwereyns had '.}{Allewies kennis gemaakt}{int regiment}\\

\haiku{Ook meneer Santiel,:}{hijgde hoorbaar maar dat was}{van de inspanning}\\

\haiku{Het nieuwe leven,.}{deinde om hen heen zonder}{hen aan te roeren}\\

\haiku{Doch hij had bijna;}{meer moeite met de vrouw dan}{met de woesteling}\\

\haiku{het Kasteel {\textquoteleft}Uputin{\textquoteright},;}{van de oude baron van}{Houren d'Uputin}\\

\haiku{Lowiezeken zou.}{ondankbaar zijn geweest over}{wat ook te klagen}\\

\haiku{Zij knikte van ja, '.}{t schaamterood gezicht achter}{de beide handen}\\

\haiku{Zij zaten daar als,.}{stugge rechters die een streng}{vonnis gaan vellen}\\

\haiku{'t 'n Zoe toch moar'.}{ou plicht zijn os g heur in}{heur ier herstelde}\\

\haiku{- Al kwam hij om mee,!}{heur te treiwen n\'og schupte}{ik hem aan de deur}\\

\haiku{Moeder Dort\'e zei,;}{dat ze toch nog liever een}{jongen had gezien}\\

\haiku{Als ze bij haar was,.}{keek ze voortdurend naar haar}{mooie ringenhanden}\\

\haiku{fluisterde, met een,.}{wantrouwige blik naar de}{deur neef Ga\"etan}\\

\haiku{De auto reed weg,.}{onder een verward tumult}{van de menigte}\\

\haiku{Meneer Aamid\'e stond op.}{zijn bordes te trillen en}{bijna te schreien}\\

\haiku{- Veronderstel, zei, -!}{hij dat de vijand ons hier}{t\'och komt verrassen}\\

\haiku{De man in de boom,.}{liet zich neerglijden vlug als}{een aap langs de stam}\\

\haiku{Wat hadden zij toch?}{misdaan om zulk een wrede}{ramp te verdienen}\\

\haiku{Leonard stapte.}{de akker op en ging met}{de mensen praten}\\

\haiku{maar, wat hen betrof, ',.}{t had slechter oneindig}{veel slechter gekund}\\

\haiku{De dokter slikte.}{haastig een kort lachje in}{en stond meteen op}\\

\haiku{maar wie dat deed werd.}{veracht en voor later met}{wraakneming bedreigd}\\

\haiku{Zulmatje was het.}{vertroetelde kind van de}{grenswacht geworden}\\

\haiku{herhaalde Tieste.}{met bevende vingers het}{bankje verstoppend}\\

\haiku{Hij reed er zo mee,,.}{door voor de aardigheid om}{eens te proberen}\\

\haiku{Zo ging dat. En zo.}{vervloog nog eens de droeve}{oorlogswinter}\\

\haiku{Het bleek, dat hij een.}{oude schoolmakker en vriend}{was van de dokter}\\

\haiku{klonk de rustige.}{stem van Leonard in de}{benauwde stilte}\\

\haiku{Hij trok zijn kar, die,, ' {\textquoteleft}{\textquoteright}.}{leeg was opzij bleef even v\'o\'or}{tKasteelken staan}\\

\haiku{Lowiezeken had:}{een wild gebaar van schrik en}{Dort\'e riep woedend}\\

\haiku{Tiestes antwoord ging.}{in het gewoel en in het}{gedrang verloren}\\

\haiku{Trillend lei Zulma.}{haar hand in de zijne en}{drukte die knellend}\\

\haiku{Nors nam zij het boek,.}{op en bladerde wat als}{in zwijgend protest}\\

\haiku{- Ze wilde zelve, ' '.}{noar ou toe komen moark}{h\`et heur belet}\\

\haiku{- Lowiezeken, zei, -.}{Jeannette ge zit doar}{nog mee ou schoens aan}\\

\haiku{Moeder Dort\'e kwam '.}{aanbellen ent ogenblik}{daarna ook Guustje}\\

\haiku{Nu, moeder, moet ik '.}{u het een ent ander}{van mij vertellen}\\

\haiku{XII - Lowieze, wil '?}{ik heur ne kier schrijven os}{ge gij nien wilt}\\

\haiku{Was de wereld dan,}{z\'o veranderd dat het niet}{meer als een schande}\\

\haiku{Maar Zulma weet gij:}{wat voor een gedacht dat ik}{altijd gehad heb}\\

\haiku{Maar de deftige.}{chauffeur reageerde in}{het geheel niet meer}\\

\haiku{zei hij op zijn beurt,.}{met uitgestrekte hand. Even}{aarzelde Zulma}\\

\haiku{vermoedelijk was.}{Leonard nog ergens in}{het dorp gebleven}\\

\haiku{Eerder last kon hij.}{van zijn jongere broer en}{zuster ginds krijgen}\\

\haiku{Hij scheen zijn zuster,.}{zwijgend om een beslissend}{antwoord te vragen}\\

\haiku{hij was drijfnat en, ':}{beslijkt maart scheen hem niets}{te kunnen deren}\\

\haiku{antwoordde Peetsen,.}{zonder aarzelen met de}{diepste overtuiging}\\

\haiku{Moeder schreide niet,.}{maar zij zag er zo bleekjes}{en zo mager uit}\\

\haiku{IV Somber strekte.}{nu de oceaan zijn woeste}{eindeloosheid uit}\\

\haiku{Het bijna zwarte.}{water deed denken aan een}{woelend berglandschap}\\

\haiku{men had geen kracht meer,.}{om nog iets te doen om nog}{aan iets te denken}\\

\haiku{Treinen reden op,,;}{en af over een soort dijk vrij}{hoog boven de straat}\\

\haiku{Ivan keek zijn broeder.}{aan of hij door hem voor de}{gek gehouden werd}\\

\haiku{- 'k Kenne doar ne.}{goeje restaurant woar da}{we goan dineren}\\

\haiku{Alvorens binnen:}{te treden nam Oculi hen}{nog eens kritisch op}\\

\haiku{zij klommen, langs een,.}{vrij sombere trap naar de}{derde verdieping}\\

\haiku{Alles verzwond tot.}{ijlheid in haar geest en zij}{sliep zoetjes in}\\

\haiku{Ze kregen meer dan.}{genoeg om er een volle}{dag van te leven}\\

\haiku{pochte Oculi en.}{hij wenkte de neger om}{af te rekenen}\\

\haiku{andere kofiers.}{kopen en die lelijke}{kisten verbranden}\\

\haiku{Clotilde vond het.}{akelig om aan te zien en}{aan te  horen}\\

\haiku{Zij voelde zich ook,.}{veiliger dat zij daar met}{hun twee\"en lagen}\\

\haiku{Ge verstoat toch wel '.}{da ze da niet buem veur buem}{n kosten uitdoen}\\

\haiku{De trein reed over een,.}{lange metalen brug die}{donderend dreunde}\\

\haiku{- d'r 'n es hier al.}{nie veel gelegenheid om}{Vloams te klappen}\\

\haiku{Hij vertelde van,:}{Ierland zoals hij het in}{zijn jeugd gekend had}\\

\haiku{En ook Oculi en,;}{Dzjeurens hoewel zij zo}{vervelend spraken}\\

\haiku{Zij sloot haar ogen en.}{vouwde de handen samen}{in een vroom gebed}\\

\haiku{Oculi, die in zijn ',:}{bord aant slurpen was keek}{met verbazing op}\\

\haiku{- Hoe vinde dat! vroeg,,.}{hem Maria levendig met}{schitterende ogen}\\

\haiku{Franklin keerde zich half.}{terzijde om zijn sigaar}{weer aan te steken}\\

\haiku{Een schei-dreunende.}{muziek barstte eensklaps los}{en het scherm ging op}\\

\haiku{- Ien van die keirels!}{die vandoage rijke}{zijn en morgen oarm}\\

\haiku{en onmiddellijk.}{plooide Mr. Watsons gezicht zich}{tot stemmige ernst}\\

\haiku{Een jonge dame,.}{mende heel alleen in het}{rijtuig gezeten}\\

\haiku{hij wist niet wat te.}{doen met al die lege uren}{van de lange dag}\\

\haiku{Er was daar wel geen,}{reden voor maar dat verschil}{van tijd stemde hen}\\

\haiku{het trilde in zijn.}{lichaam niet wanneer hij er}{zijn voet op drukte}\\

\haiku{Ik heb twee schoone.}{duivenjongens gekregen}{van Peetse Speybroeck}\\

\haiku{Hij had vijf dagen,.}{vrij die zij in Chicago}{zouden doorbrengen}\\

\haiku{Maria schaamde zich.}{en dadelijk straalde weer}{haar gezonde lach}\\

\haiku{Ge doe mij denken '.}{aann kennisse van ons}{die Dzjeurens hiet}\\

\haiku{Zij wilde wel heel.}{graag met haar mooie wagen in}{de stad eens geuren}\\

\haiku{Nu reeds was hij van!}{de derde wereld in de}{tweede overgestapt}\\

\haiku{en de weg verliet.}{de stroom en slingerde de}{steile hoogte op}\\

\haiku{De weg slingerde,,,.}{kronkelde met bochten die}{steeds scherper werden}\\

\haiku{Hij had de vrouw van!}{zijn baas genomen en hij}{zou haar nog nemen}\\

\haiku{- Kom binnen, jongen,,.}{kom binnen en ging hem zelf}{voor in het huisje}\\

\haiku{en haar ogen peilden,.}{hem zo dat hij met een kleur}{de blik afwendde}\\

\haiku{hun leven scheen er,.}{op de maat van de huizen}{die zij bewoonden}\\

\haiku{XIX Mr. Keane.}{kwam zijn vrouw in New York aan}{de boot afhalen}\\

\haiku{Oculi glunderde ':}{ent eerste wat hij zijn}{broeder toeriep was}\\

\haiku{De waarde van de!}{grond is alweer met veertig}{percent gestegen}\\

\haiku{en wij... Dear, ik moet!}{er onmiddellijk een nog}{mooiere hebben}\\

\haiku{hedde da nou nog?}{oeit van ou leven g'huerd}{van luelijkheid}\\

\haiku{Hij vond de schikking.}{nogal billijk en hij zei}{het aan zijn zuster}\\

\haiku{Op een gegeven.}{ogenblik moesten de tramcars voor}{het gedrang stoppen}\\

\haiku{- Ge keunt er nou mee! -?...}{treiwen en ou leven lang}{gelukkig zijn Nou}\\

\haiku{Zijn armen deden;}{pijn van verlangen om haar}{te omstrengelen}\\

\haiku{Gladys stond op en.}{reikte Moeder en Peetsen}{de hand tot afscheid}\\

\haiku{En eensklaps, zonder,.}{schijnbare reden barstte}{zij in tranen uit}\\

\haiku{doch haar huwelijk;}{met Mr. Keane had dat}{alles verbroken}\\

\haiku{en na een poos, als,,:}{in een droom hoorde hij h\'a\'ar}{stem die antwoordde}\\

\haiku{Geloof mij, Ivan, treur,.}{niet om mij ge zoudt met mij}{niet gelukkig zijn}\\

\haiku{Zo kwamen er zes.}{eentonige gelijke}{dagen in de week}\\

\haiku{allen voelden in.}{peinzend stilzwijgen een groot}{geheim om zich heen}\\

\haiku{Well het eenige dat}{ik hier somtijds beter vind}{dan in Amerika}\\

\haiku{Ik  denk dat gij.}{van uwe kant ook al meer dan}{genoefd hebt van oud}\\

\haiku{ik zal al doen wat.}{ik kan om op denzelfden}{boot te zitten}\\

\haiku{Lisatjes moeder was,.}{sluw Nonkel Justien was slim}{en flink voortvarend}\\

\haiku{Hij gaf een duwtje,.}{in de zij van Lisatje die}{ook eventjes lachte}\\

\haiku{Neen, hij verkocht niets,,.}{nog geen vierkante meter}{zolang hij leefde}\\

\haiku{Toen vroeg hij, zijn olijk:}{glinsterende ogen lachend}{op Ivan gevestigd}\\

\haiku{Hij aarzelde, hij,.}{twijfelde hij voelde zich}{zo diep ellendig}\\

\haiku{{\textquoteright} Geen liefelijke,!}{zachtheid geen nederige}{onderworpenheid}\\

\haiku{- Jaja, zei Smith, - maar,.}{gij hebt Luis niet gekend de}{Mexicaanse koetsier}\\

\haiku{Er broeide reeds lang.}{iets in haar en nu zou zij}{haar hart eens luchten}\\

\haiku{zueveel meugelijk.}{money moaken en de}{reste komt vanzelf}\\

\haiku{Ivan voelde iets als.}{een stille vijandigheid}{groeien om zich heen}\\

\haiku{Dat is toch iets, niet,!}{waar voor eenen jongen van zoo}{goede familie}\\

\haiku{Ivan schelde haastig.}{om een steward en leidde}{Peetsen in de kooi}\\

\haiku{Ivan trachtte hem aan ',.}{t verstand te brengen hoe}{dat in elkaar zat}\\

\haiku{- Dag, Maria! groette,,.}{hij heel gewoon alsof hij}{haar dagelijks zag}\\

\haiku{- 'k 'n Gelueve '!}{niet dak d\'a zoe keunen}{geweune worden}\\

\haiku{'t Leek alles nu,.}{zo vreemd zo ver wat eenmaal}{zo diep geleefd had}\\

\haiku{Toen kwam er een brief,,,:}{een korte een formele}{van Clotilde}\\

\haiku{Wel, lieve moeder}{en broeders en familie}{hiermede eindig}\\

\haiku{Naar Grover keek zij,.}{verlegen alsof hij er}{niet bij behoorde}\\

\haiku{zei Franklin glimlachend -!}{en drukte haar de hand. Ge}{zij welgekomen}\\

\haiku{Maria had voor hem.}{een diep bord gevraagd en twee}{rauwe eieren}\\

\haiku{- Wat 'n verschil met!}{Blue Springs om nog niet eens van}{New York te spreken}\\

\haiku{- U hebt b.v. geen idee,,.}{wat die kleine kerel daar}{Grover al niet weet}\\

\haiku{n Zoe nie keune,!}{zeggen wat dat er van de}{die geworden es}\\

\haiku{- Van 's gelijke,,' '.}{jongen en ziet da g ou}{geld nien verliest}\\

\haiku{en zij, Maria, kreeg.}{last van haar ogen en moest een}{sterke bril dragen}\\

\haiku{Aldus kwam Grover!}{naar het geboorteland van}{zijn ouders terug}\\

\haiku{Maria schreef, dat ze,;}{zo heel h\'e\'el weinig nieuws meer}{te vertellen had}\\

\haiku{maar 't feit dat hij {\textquoteleft}{\textquoteright}.}{nietneen zei mocht eigenlijk}{ja betekenen}\\

\haiku{Velde trok een vreemd.}{gezicht en begon achter}{zijn oor te krabben}\\

\haiku{- En ziet da ze nie '!}{wign goat achter d'ieste}{moand proboassie}\\

\haiku{En de nachtegaal,!}{zong maar aanhoudend om er}{dol van te worden}\\

\haiku{Vastberaden trad;}{hij haar tegemoet in het}{smalle gangetje}\\

\haiku{- Joa moar 'k 'n laa\"e, ' '! '!}{nietkn laa\"e niett Es}{huel serieus}\\

\haiku{Da moeder da moest,!}{weten ze zoe heur in heur}{kist ornme kieren}\\

\haiku{Zij gilden allen,.}{om het hardst alsof zij op}{de akker stonden}\\

\haiku{- In mijnen tijd was,!}{wirken troef ik spreek ik hier}{van nudig wirke}\\

\haiku{bij Marina was;}{het tot het vorige jaar}{ook aldus geweest}\\

\haiku{- 't Es koart veur den,,.}{achten hernam hij als een}{klacht van ongeduld}\\

\haiku{- Moeder, doe de die\"e.}{doar ne kier ophouwe mee}{heur zoagerije}\\

\haiku{ze weunt zij moar op, '.}{mijn koamer zen es zij van}{mijn familie niet}\\

\haiku{zei hij, de arme.}{man met bezorgde liefde}{in de ogen kijkend}\\

\haiku{verpoest,   charbons;}{et cokes ~ Links was er}{een Vlaamse Kamer}\\

\haiku{- Et quand aurons-nous?}{aussi une fois le plaisir}{de vous voir chez nous}\\

\haiku{De laatste stralen.}{van de lentezon schoten}{hun goud om hen heen}\\

\haiku{- Ha joa moar, 'k 'n,.}{ben ik doar nie op geklied}{menier Florimond}\\

\haiku{Hij laat haar los, doet,}{haar nog eens vast beloven}{dat ze komen zal}\\

\haiku{de hoge kruinen.}{van het stil-verlaten}{Park werden inktzwart}\\

\haiku{Prenez du coup au, '.}{moins deux meill da ge ern}{tijdje goe mee zijt}\\

\haiku{blomme woater te, '.}{geve aan de venster van}{n twiede stoassie}\\

\haiku{en wat dat er doar ' '!}{bui te gebeurtn trekke}{k ik mij nie aan}\\

\haiku{Plotseling keerde.}{de schaduw zich naar hem om}{en bleef palstil staan}\\

\haiku{- Ha, 'k 'n doe ekik,......!}{nie mier als Julien en nog}{vele andere}\\

\haiku{- Haw\`el joa, bel moar, ' ',.}{kn k\'an nie mier hijgde}{madame Verpoest}\\

\haiku{Sloapt er nen nacht,.}{op en zeg mij morge da}{ge wijs zult worde}\\

\haiku{- Wilde gij ne kier,',!}{zwijge joa g en ziere}{noar uw bedde goan}\\

\haiku{t Zijn leugens? - 'k ', '!}{Weinste datt woar woare}{datt leugens zijn}\\

\haiku{Allen nu keken,.}{hem strak aan merkten dat hij}{aangeschoten was}\\

\haiku{Ouais, ouais, c'est,.}{parce qu'il deviendrait trop tard}{savez-vous}\\

\haiku{Niemand kende haar,;}{niemand wist wie ze was noch}{waar ze vandaan kwam}\\

\haiku{Dadelijk vormde,,.}{zich onder haar voeten een}{sijpelend plasje}\\

\haiku{hij herleefde, vrij;}{van de zo lang en zo hard}{knellende banden}\\

\haiku{En hij trok er ruw,,.}{van onder als iedere}{avond naar zijn lief toe}\\

\haiku{- Hij wirkt bij Carels, '!}{op den atelier hij wintn}{schune daghure}\\

\haiku{Op het busseltje:}{van Euzekes kind stond er}{in drukletters}\\

\haiku{En in de loop van ',:}{t gesprek vroeg Euzeke}{schuchter-aarzelend}\\

\haiku{Deze, reeds gewend,.}{aan onbezorgde kost stond}{in verlegenheid}\\

\haiku{- Nuunt, nuunt mee hem op,,.}{stroate zei ze als wrang}{besluit in haarzelf}\\

\haiku{- Toe, toe, da zoe te, ' ';}{verre lien en uukkn}{zoek ekik da niede}\\

\haiku{Les fun\'erailles,,.}{auront lieu schreef Florimond}{steeds hardop lezend}\\

\haiku{met het papier in,.}{de hand de geest op zoek reeds}{naar een uitvluchtsel}\\

\haiku{Niet begrijpend, waar,:}{hij heen wilde liet ze zich}{vangen in de strik}\\

\haiku{XV Middelerwijl...}{kwam Florimond geregeld}{weer bij Euzeke}\\

\haiku{een dame, die met;}{een volgeladen blikken}{korf van de markt kwam}\\

\haiku{- Nous d\'esirons voir,.}{votre maison qui est \`a}{louer madame}\\

\haiku{Je ne regarde.}{pas \`a un franc pour avoir des}{gens pas crapuleux}\\

\haiku{Et puis, sales qu'ils,,!}{\'etaient n\'egligents qu'ils}{\'etaient les autres}\\

\haiku{- Al voir, madame,,,.}{al voir mossieur is het}{vriendelijk antwoord}\\

\haiku{J'ai presque tout compris,,.}{en devinant le reste}{naturellement}\\

\haiku{- Joa, joa, ge zul ne,,.}{kier hure wa da ze zegt}{meende Marina}\\

\haiku{Madame M\'edard,.}{stond te wachten in fluweel}{en bont ondanks april}\\

\haiku{sprak hij bewogen,.}{tot zijn moeder toen de deur}{achter hen dicht was}\\

\haiku{Zij hadden haar zo!...}{kortstondig en zo innig}{gelukkig gemaakt}\\

\haiku{- C'est vous qui avez l'air,.}{bien portante hernam ze}{tegen Aurore}\\

\haiku{Het was madame,:}{Verpoest met een opgewekt}{gelaat als altijd}\\

\haiku{antwoordde hij met.}{een komische buiging voor}{madame Verpoest}\\

\haiku{En Paulke volgde.}{weer het gevaarte als een}{schuitje op sleeptouw}\\

\haiku{propaganda met.}{de mond en met de pen was}{gemaakt geworden}\\

\haiku{- Papa zorgdege,.}{altijd veur de cro\^ute zei}{madame Verpoest}\\

\subsection{Uit: Verzameld werk. Deel 4}

\haiku{Buysse hanteert dit {\textquoteleft}{\textquoteright}:}{ik-perspectief van}{deconte vrij vaak}\\

\haiku{Schrik, uit Elsevier's,,,-;}{Ge{\"\i}llustreerd Maandschrift d.}{IV 1892 p. 488495}\\

\haiku{De gelukkige,,,,-;}{tijding uit Eigen Haard 1895}{nr. 27 p. 417421}\\

\haiku{Zonder de minste,.}{tegenstand te bieden liet}{hij zich meeleiden}\\

\haiku{een gesmoorde zucht}{ontglipte zijn beklemde}{borst en zijn armen}\\

\haiku{Zij was ook lang en,;}{slank van gestalte zoals}{hij doch iets kleiner}\\

\haiku{- zij zijn zeer braaf en ';}{komen hiers winters schier}{elke avond kaarten}\\

\haiku{Hij zal zich na de ';}{vespers aant kaarten wat}{vergeten hebben}\\

\haiku{'t Was zonderling,.}{hij scheen verheugd en ook toch}{enigszins verlegen}\\

\haiku{Maar zij ontsnapte:}{hem opnieuw en ging midden}{in de keuken staan}\\

\haiku{Elk droeg zijn lijn en '}{had het een oft ander}{pakje in de hand.}\\

\haiku{Ik zag mijn vangst met.}{onbeschrijflijke hoogmoed}{en ontroering aan}\\

\haiku{- Hoe! kreet ik verbaasd.}{rechtstaand en met geweld mijn}{uurwerk uittrekkend}\\

\haiku{ik verontwaardigd,.}{tot de garde die achter}{mij de deur toesloeg}\\

\haiku{Die betrouwt zich op;}{de vast bepaalde tijd en}{nut op zijn gemak}\\

\haiku{{\textquoteright} 't Is te zeggen,,,.}{huis verblijfplaats  toevlucht}{voor de reiziger}\\

\haiku{Ik werd eensklaps gans, ',.}{wakker ik richttet hoofd}{zette mij overeind}\\

\haiku{Ik keerde mij tot,.}{het jong meisje om dat mijn}{arm verlaten had}\\

\haiku{Als jachthonden, met,;}{het aangezicht tegen de}{grond volgden zij dit}\\

\haiku{Het vrouwvolk had zich,,;}{in de kamer achter de}{kleerkast verscholen}\\

\haiku{zijn handen, ontvleesd;}{en knokkelig als klauwen}{waren zwart en vuil}\\

\haiku{- Zie, mensen, dat weet,, -.}{ik niet sprak hij en dat zijn}{ook niet mijn zaken}\\

\haiku{riep hij met luider,,.}{stem tot mijn buur de eerste}{getuige knipogend}\\

\haiku{De notaris, zeer,;}{ernstig was begonnen zijn}{pen af te vegen}\\

\haiku{Eenieder sprak die, '.}{woorden na sloeg een kruis en}{richttet hoofd op}\\

\haiku{Langzaam als het dof,.}{getrappel van een kudde}{begon de aftocht}\\

\haiku{Doch de notaris.}{had reeds zijn zwaveldoosje}{te voorschijn gehaald}\\

\haiku{Gampelaarken 't,.}{Was in december bij het}{vallen van de avond}\\

\haiku{In diep stilzwijgen.}{had men het einde van dit}{verhaal aangehoord}\\

\haiku{juist stond ik voor de.}{woning van Beer Roetjes32}{de vogelman}\\

\haiku{Een reusachtige,,;}{kom nog halfvol pap bevond}{zich in het midden}\\

\haiku{Een diep misprijzen,.}{blonk in zijn blik een spotlach}{kwam op zijn lippen}\\

\haiku{- Hij is er bij, zeg,!}{ik u en wij moeten en}{zullen ze vinden}\\

\haiku{Wij trokken ons een.}{weinig achteruit nevens}{het tweede schuitje}\\

\haiku{- Komaan, zeg ik, laat.}{ons een wandelingetje}{maken op het dek}\\

\haiku{'t gerucht liep rond.}{dat men misschien dezelfde}{avond zou aanlanden}\\

\haiku{Weldra bevonden.}{wij ons als te midden een}{oceaan van lichten}\\

\haiku{Op een gegeven.}{ogenblik ontstond er daar een}{soort van worsteling}\\

\haiku{Ik had gedaan met,.}{eten rees van mijn zitplaats op}{en ging eens zien}\\

\haiku{- Zo niet, zend ik u.}{voor het overige van de}{reis in steerage37}\\

\haiku{Een schielijke, schier;}{plechtige stilte schorste}{de gesprekken op}\\

\haiku{Dringend wendde ik {\textquoteleft}{\textquoteright}.}{een uiterste poging bij}{decollector aan}\\

\haiku{Hun stappen klonken {\textquoteleft}{\textquoteright};}{ongelijk en hol over de}{planken van depier}\\

\haiku{Zijn makkers volgden,,.}{hem schielijk bedaard zijn aap}{sprong van zijn schouders}\\

\haiku{Ditmaal werd zijn borst.}{niet te zwak gevonden voor}{de soldatendienst}\\

\haiku{Werktuiglijk staat Miel.}{op en neemt ook zijn pakje}{dat op een stoel ligt}\\

\haiku{En moeder, doodsbleek,,,:}{met vergrote angstige}{ogen murmelt ohaar beurt}\\

\haiku{Het is zo vet en,,?}{wel gevoed als zijn meester}{zelf nietwaar Fido}\\

\haiku{Mul begreep, dat hij.}{in zulk een staat v\'o\'or zijn neef}{niet kon verschijnen}\\

\haiku{hij trok er binnen,.}{om wat uit te rusten en}{zich te verfrissen}\\

\haiku{Een dof gefluister,.}{greep plaats zij trokken enige}{stappen terzijde}\\

\haiku{Toen zag Perseijn in,:}{de maneschijn een der twee}{mannen op de rug}\\

\haiku{En hij legde Mul,,.}{die pots begon te beven}{de akte voor ogen}\\

\haiku{hij sprong naar die stem,;}{hij verworgde ze in de}{keel van de woestaard}\\

\haiku{Gij wilt ons daar iets!}{op de mouw spelden dat toch}{wat al te kras is}\\

\haiku{- 't Is tot mevrouw,?}{Montfort dat ik de eer heb}{het woord te richten}\\

\haiku{Maar zij begreep de '.}{betekenis van die blik}{en buktet hoofd}\\

\haiku{En op een avond was:}{het als een wind van schrik die}{woei over de hoeve}\\

\haiku{- Vader Slock, riep hij, -,,.}{geloof wel dat ik ernstig}{heel heel ernstig spreek}\\

\haiku{het is mijn plicht hun!}{het geluk te geven dat}{in mijn bereik staat}\\

\haiku{Zij rees v\'o\'or haar op,,.}{ellendig en wenend met}{haar kind op de arm}\\

\haiku{En zij slaakte een,;}{kreet van verrukking toen zij}{haar gans gekleed zag}\\

\haiku{en schertsend klopte '.}{de plaagzieke dokter nog}{eens opt buikje}\\

\haiku{Na het nagerecht.}{stonden de dames op en}{verlieten de zaal}\\

\haiku{- Adela{\"\i}de, nou;}{weet ik wat dat ze in het}{dorp tegen ons h\^en}\\

\haiku{En, spijtig met het,,:}{hoofd schuddend nam hij een vers}{blad papier zeggend}\\

\haiku{Men had voor hem geen;}{toornige blikken of geen}{hoongelach meer over}\\

\haiku{De secretaris,,.}{hevig ontsteld was opnieuw}{zeer bleek geworden}\\

\haiku{dat zij begreep wat.}{in hem omging en dat zij}{er gestreeld door was}\\

\haiku{De oude moeder.}{slaat wanhopig haar ogen en}{handen ten hemel}\\

\haiku{- En deze, hier, het,?}{kind van de meesters wat moet}{er daarvan worden}\\

\haiku{Opnieuw neemt zij de, '...}{vlucht naart diepste van de}{dennenbossen}\\

\haiku{Het kindje zwijgt en,....}{zuigt de ogen van de jonge}{moeder vallen toe}\\

\haiku{schreeuwde hij, verbaasd,.}{en woedend zich in volle}{lengte oprichtend}\\

\haiku{je sau...aurai bien....}{vous prot\'eger coontre}{cette canaille}\\

\haiku{Met nadruk drong hij.}{erop aan dat de prinsen}{t\'och zouden blijven}\\

\haiku{Dit bracht de heren.}{tot het bewustzijn van hun}{deftigheid terug}\\

\haiku{- De littekens zijn, ',?}{nog zichtbaar op mijn dijen}{ist niet waar vrouw}\\

\haiku{meneer Spittael en,,}{zijn twee magere gele}{juffrouwen meester}\\

\haiku{Maai dunkt nochtans dat.}{ik niets gezeid heb da nie}{gehoord mag worde}\\

\haiku{oed, het ogenblik is,.}{gekomen sprak Massijn met}{een stokkende stem}\\

\haiku{Meester De Vreught had.}{zijn zakdoek uitgehaald en}{snoot zich luidruchtig}\\

\haiku{Doch ik kwam terstond;}{tot het bewustzijn mijner}{deftigheid terug}\\

\haiku{Toch had hij zich, wat,;}{dat betreft de laatste tijd}{in acht genomen}\\

\haiku{Zonder de blaker,.}{los te laten stak hij de}{sleutel in het gat}\\

\haiku{Er was heel weinig,.}{kans dat hij langs die weg kon}{geholpen worden}\\

\haiku{Hij vond niets dan zijn,.}{sigarenkoker met nog}{drie havana's erin}\\

\haiku{D\'a\'arover alleen moest hij,.}{steeds blijven heersen steeds vrij}{blijven beschikken}\\

\haiku{Deze was een zeer,.}{vermogend man vrij ruw en}{stroef van uiterlijk}\\

\haiku{En toen haar man, die,:}{dat wel \'al te overdreven}{vond haar uitlachte}\\

\haiku{D\'a\'ar staat hij, d\'a\'ar brengt!...}{hij hun de grote liefde}{en het groot geluk}\\

\haiku{De lege flessen;}{worden vanuit de tent in}{het water gegooid}\\

\haiku{Hij stond op toen hij,:}{ons zag en zodra hij de}{gordel ontwaarde}\\

\haiku{- Ik zocht hem overal,,.}{ik kon niet denken waar hij}{toch gebleven was}\\

\haiku{Na een week liet ik ' '.}{hems ochtends ens avonds}{een paar uurtjes los}\\

\haiku{hij profiteert als.}{zij van de rijkdom en de}{overvloed der Natuur}\\

\haiku{men zou een weinig,...}{gaan bedelen dat was ook}{vroeger reeds gebeurd}\\

\haiku{Hij had de kracht niet,.}{meer om op te staan om zich}{nog te bewegen}\\

\haiku{hij woonde daar, met,;}{de geest afschuwelijke}{taferelen bij}\\

\haiku{- Bah joa 't, da goa,:}{nogal moar gisteren was}{de vangst toch beter}\\

\haiku{Ik zal  hem twee.}{vijffrankstukken geven en}{wij zullen eens zien}\\

\haiku{En nu weergalmt, door,,:}{het gejouw heen een kreet van}{haat alom herhaald}\\

\haiku{{\textquoteright} 't Is inderdaad.}{daarheen dat zich de stoet met}{rasse schreden wendt}\\

\haiku{Dan stijgt me als een.}{walg van verachting en van}{afkeer in de keel}\\

\haiku{De kikkers Om twee,,...}{uur na zijn dutje was de}{barbier vertrokken}\\

\haiku{Zij houden van wat,.}{hun groot schijnt van al wat ruim}{en overtollig is}\\

\haiku{Even vielen zijn ogen,.}{heel en al dicht en zijn hart}{hield op met kloppen}\\

\haiku{De groffe houten;}{lepel beefde sterker in}{het klein bruin handje}\\

\haiku{Hij zou nu naar de,;}{hoeve weer terugkeren}{gelijk zijn broeders}\\

\haiku{De knapen kropen;}{uit het water en kleedden}{zich haastig weer aan}\\

\haiku{In geen enkel was,.}{er nog een eendje alle}{waren bedorven}\\

\haiku{G'en peist toch zeker ' '?}{niet uitnen hoan enn}{oande te kwieken}\\

\haiku{Zij wierpen verse.}{kruimeltjes brood en verse}{brokjes aardappel}\\

\haiku{Toen deed de wachter:}{zich geweld aan om hem toch}{te kunnen zeggen}\\

\haiku{Elke morgen, zijn,}{wagen geladen tot aan}{het dekzeil verliet}\\

\haiku{dat zelfde ogenblik.}{verliet Merci\'e het huisje}{van de twee oudjes}\\

\haiku{In het geschokte,.}{brein van Merci\'e daagde een}{licht een gedachte}\\

\haiku{Al mier of 'n joar?}{dus woaren ze van gedacht}{mij wig te zenden}\\

\haiku{het waren zijn schrik:}{en zijn gruwel zelf die hem}{werkelijk stuwden}\\

\haiku{Ik volgde hem, kwam,.}{hijgend op een portaal waar}{hij mij reeds wachtte}\\

\haiku{In elk geval was '.}{t toch de moeite weird de}{kanse te woagen}\\

\haiku{De krieken, minder,,...!}{overvloedig raakten er nog}{door maar de pruimen}\\

\haiku{Dukske bloedde en, ':}{huilde vervaarlijk maart}{was nog niet gedaan}\\

\haiku{hoe kwam het toch, dat,!}{die steeds los liep terwijl hij}{steeds gebonden lag}\\

\haiku{Het ene deel viel op,;}{de grond met het gerinkel}{van een sleutelbos}\\

\haiku{Hij stapte rechtstreeks.}{binnen en dronk twee borrels}{aan de schenktafel}\\

\haiku{hij trok het huis van,,.}{Rosten Tjeef de verklikker}{de vijand voorbij}\\

\haiku{Die toestand in het.}{huisgezin kon echter zo}{niet blijven duren}\\

\haiku{Thans stond zij in 't.}{smal gangetje tussen de}{twee eerste bedden}\\

\haiku{Neen, zij had niet met;}{voorbedachten rade een}{kindermoord beraamd}\\

\haiku{den bueze geest,.}{van den duvel die altijd}{op de meinschen loert}\\

\haiku{je hield je stevig ':}{met de hand aant lijstwerk}{van de deur en zei}\\

\haiku{Hoe vreselijk als!}{iemand hem daar bij klaren}{dag zag aanschellen}\\

\haiku{En eensklaps zag hij,,.}{hem komen links uit een der}{deuren van de gang}\\

\haiku{We willen wij niets,}{weet van cur\'e of van kirk en}{doarom vallen}\\

\haiku{zijn geluk was zo,}{intens groot dat het hem haast}{natuurlijk voorkwam}\\

\haiku{De pastoor kwam op '.}{t geluid in de gang en}{naderde de deur}\\

\haiku{Alleen de pastoor,,.}{met zijn absolutie kon}{ze nog verlossen}\\

\haiku{En midden in de!}{nacht stond hij op en holde}{hij de velden in}\\

\haiku{Als een pak viel hij,;}{op zijn bed dadelijk in}{een loodzware slaap}\\

\haiku{Een groot ge eelte.}{van de dag bleef hij aldus}{stom-roerloos liggen}\\

\haiku{{\textquoteright} Natuurlijk kon hij.}{niet in gewijde aarde}{begraven worden}\\

\haiku{De lucht was zwoel, met,.}{laag-drijvende wolken}{broeiend van onweer}\\

\haiku{De vrouw ging naar het.}{achterhuis en nam er uit}{een hoek twee spaden}\\

\haiku{Hij, weer aan zijn werk,.}{scheen zich om haar niet langer}{te bekommeren}\\

\haiku{as 't nog iene, '!}{kier moest gebeuren dan es}{t veur altijd uit}\\

\haiku{Elk ogenblik kwamen;}{haar tranen in de ogen en}{hikken in de keel}\\

\haiku{maar hij staalde zich,.}{met wilskracht en antwoordde}{trots-minachtend}\\

\haiku{zij hadden geen aren,,:}{meer om te binden en met}{het wilde gejuich}\\

\haiku{En snikkend liep zij.}{haar muts opzetten en haar}{schoenen aantrekken}\\

\haiku{En ze zag dingen,...}{die ze niet meer kende die}{ze niet meer begreep}\\

\haiku{Permentier, veinzend,.}{dit niet te zien had haastig}{zijn deur gesloten}\\

\haiku{riep plotseling de,.}{president op dreigende}{toon met boze ogen}\\

\haiku{- G'n luept uek in,...?}{de bosschen niet g'n goat er}{gien heit85 roapen}\\

\haiku{Gedurende nen'.}{halve menuut zilt g hem}{op ou\"e kogel89 h\^en}\\

\haiku{hij aarzelde een,, '.}{ogenblik de ogen fonkelend}{t gehoor gespitst}\\

\haiku{de pijn als van een,.}{hagelslag die hem dwars door}{het hoofd zou boren}\\

\haiku{Wijl ze nu toch moest,.}{trouwen dan zo graag met hem}{als met een ander}\\

\haiku{Haar afgematte.}{kinderen konden weldra}{niet langer volgen}\\

\haiku{Dan valt hij in het '.}{oog en dooft ert laatste}{stipje leven uit}\\

\haiku{Hij was op 't punt,.}{er een brok van te vragen}{als een bedelaar}\\

\haiku{Hij had maar even de, {\textquoteleft}... '!}{klink op te tillen en met}{eenBrr watn weer}\\

\haiku{Dit gezicht, in plaats,.}{van hem te paaien maakte}{hem nog woedender}\\

\haiku{{\textquoteleft}Ik heb mijn moeder,,,!}{mijn oude goede brave}{moeder geslagen}\\

\haiku{Wat die barbaar, die,!}{bandiet niet gedurfd had dat}{had hij gedaan}\\

\haiku{Buiten stierven de.}{geluiden een voor een tot}{doodse stilte weg}\\

\haiku{Het koord in de hand.}{laat hij de lantaren naar}{beneden glijden}\\

\haiku{{\textquoteleft}Vreiwken, es 't '?}{mee ou\"e wille datt lijk}{uit den huize goat}\\

\haiku{Reinildeke bij,.}{haar broeder Leontientje}{bij haar schoonzuster}\\

\haiku{daarentegen scheen.}{er zachte rust en kalmte}{overal te heersen}\\

\haiku{En zij dankte de;}{hemel dat nu toch geen ramp}{meer op hen neerviel}\\

\haiku{- Ala toe, toe, g'n moet,.}{hier nie gegeneerd zijn drong}{de keukenmeid aan}\\

\haiku{Doodstil bleven zij,.}{alle drie met star op hem}{gevestigde ogen}\\

\haiku{Pover, ontsteld en,.}{bang hield zich schuil achter zijn}{kleingeruit raampje}\\

\haiku{Daar kroop hij in weg,.}{als een doodgejaagd beest in}{zijn laatste schuilplaats}\\

\haiku{Hij nam hem mee naar,.}{buiten en spande hem aan}{v\'o\'or zijn kruiwagen}\\

\haiku{Zij was nog van de,.}{oude oude tijd en was}{het steeds gebleven}\\

\haiku{En in de lange:}{winteravonden zaten zij}{zwijgend te werken}\\

\haiku{Leef nou liever wat, '.}{op ou gemakk zal ik}{veur ou wel wirken}\\

\haiku{- 'k H\`e al gezien ',,.}{datt goed es w'h\^en al da}{w'h\^en moeten sprak zij}\\

\haiku{- Joa, joa w', joa joa',,,!}{w zueveel of da g'r wilt}{en goeje zulle}\\

\haiku{Zij stemden allen:}{in die schikking toe en Ivo}{zelf voegde erbij}\\

\haiku{Iedere plak werd.}{boven op een dikke snee}{roggebrood gelegd}\\

\haiku{Wit blonk de vette '.}{plak met strepen rood erdoor}{opt zwarte brood}\\

\haiku{Hiere, ontfirmt ou over!}{de ziele van Coleta}{van den Bossche}\\

\haiku{{\textquoteleft}dewelke van de ' '?}{vier zoek nou pakken as}{k te kiezen h\^a}\\

\haiku{- Ha joa moar, kijk, da,?}{zijn dijngen die zoen keune}{gebeuren e-woar}\\

\haiku{- Dag menier Val\`ere,,.}{dag Sietje groette hij met}{brabbelende stem}\\

\haiku{Dan trok hij weer naar,.}{binnen en dicht flapten de}{metalen luikjes}\\

\haiku{Zij konden des te.}{vrijer en gezelliger}{met Sietje omgaan}\\

\haiku{Maar {\textquoteleft}la Zeun\`esse{\textquoteright} was.}{meer dan eens wreedaardig in}{haar flauwe grappen}\\

\haiku{- Ha moar 'k 'n weet, '.}{ik niet azue lijkn soorte}{van geroaktheid147}\\

\haiku{- Nie nie, nou nog niet,,.}{we zillen loater zien}{zei meneer Val\`ere}\\

\haiku{- Kijk zie, doar es den!}{toekomstigen nieuwen boas}{mee zijn schuenvoader}\\

\haiku{- Ha doar van achter,,.}{tegen de muur Philemon}{berichtte Sietje}\\

\haiku{verweet hij de smid,.}{gans bleek wordend van moeilijk}{ingehouden toorn}\\

\haiku{- Joa joa, we goan ze '!}{moar loate vliegen da}{zet hueren}\\

\haiku{- Nie nien 't, 't 'n, ' '!}{kan zeker nie zijn ent}{n z\'al nie zijn uek}\\

\haiku{Terwijl zorgde de.}{vrouw voor het huishouden en}{voor de kinderen}\\

\haiku{- Joa z', bezinne, '.}{ze stoan doar buiten op}{t hof te wachten}\\

\haiku{zei Liza naar het.}{onbeweeglijk-liggend}{konijntje wijzend}\\

\haiku{Binus vertelde,:}{een andere historie}{ook zeer wonderlijk}\\

\haiku{{\textquoteleft}Zoe we nie iest ne?}{kier moeten informeren}{of da ze nog leeft}\\

\haiku{die zij geofferd!}{had voor de genezing van}{moeders konijntjes}\\

\haiku{Felhoen ging in zijn, ',.}{zak legde vijf cent int}{schaaltje en voelde}\\

\haiku{Zie, ze kijken nog,.}{ne kier omme antwoordde}{hij geruststellend}\\

\haiku{Het was een ander,;}{type dan zijn vader dan}{zijn broers en zusters}\\

\haiku{Maar de zondag hield,.}{hij voor zich over wilde hij}{volkomen vrij zijn}\\

\haiku{- Joa ik,... 'k goa ne ', '.}{kiern beetse wandelen}{meet schuen were}\\

\haiku{En zenuwachtig '.}{tril-krabden zijn handen}{overt grauwe schort}\\

\haiku{Zenuwachtig hief.}{de geestelijke de klink}{op en was buiten}\\

\haiku{Een trapje met de,,.}{voet verwijdert hem en weer}{jankt hij heel eventjes}\\

\haiku{en 't laatste van:}{de vier huisjes stond bijna}{altijd gesloten}\\

\haiku{De heer ging achter.}{aan de wagen en hielp nu}{ook de dames uit}\\

\haiku{De fijne geur die.}{van hen uitging vulde heel}{het klein kamertje}\\

\haiku{- 'k H\^en toch al iets,, - '.}{zei hijk h\`e miene vel\'o}{kunne verkoapen}\\

\haiku{En strakker staren.}{alle ogen in de richting}{van het station}\\

\haiku{Reeds op een afstand,,:}{achter de heesters hoorde}{ik het hees gebrul}\\

\haiku{Ala, toe, kom hier, roept,.}{aanmoedigend de meid die}{zich heeft omgekeerd}\\

\haiku{Een dronkaard, die wat,.}{te veel drukte maakt wordt ruw}{opzijgestoten}\\

\haiku{steeds talrijker dringt.}{de menigte op naar de}{stroelende tonnen}\\

\haiku{Weldra zitten de.}{twee oude sukkelige}{goedaards aan tafel}\\

\haiku{Dan vertelt hem de,, '.}{zoon door zijn lachen heen hoe}{oft gegaan is}\\

\haiku{ne pak zien veurbij ' '.}{goan enk miende woarachtig}{datt veur ons was}\\

\haiku{De lauwe frisheid.}{van een vroege lentedag}{over het stille dorp}\\

\haiku{Soms is het niet te {\textquoteleft}{\textquoteright} {\textquoteleft}{\textquoteright}.}{onderscheiden wieer uit}{en wieer in is}\\

\haiku{Zijn makkers staken.}{het gevecht en gans bebloed}{tillen zij hem op}\\

\haiku{Een voldoende huis, ',.}{al wast ook maar een krot}{hadden zij allen}\\

\haiku{G' h\`et huizen lijk.}{kastielen en geld zueveel}{of da ge wilt}\\

\haiku{Hij liep, hij liep maar,.}{altijd verder door zonder}{te weten waarheen}\\

\haiku{Hij was al oud, toen,.}{ik hem leerde kennen wel}{bij de zeventig}\\

\haiku{Die vuile Fie, een, '! -?}{bloem van schoonheid hoe wast}{mogelijk En hij}\\

\haiku{Coralie vroeg of.}{ze zo wel allen samen}{over straat zouden gaan}\\

\haiku{Zij rukt en grabbelt,,.}{gillend van pijn en valt weer}{lamzieltogend neer}\\

\haiku{Belzemien haalde.}{zijn koperen snuifdoos uit}{en nam een snuifje}\\

\haiku{- Alles zou ervan;}{afhangen hoe het verder}{met Tante verliep}\\

\haiku{en ook, natuurlijk, '.}{van de tijd diet nichtje}{hier besteden mocht}\\

\haiku{Cord\'ula kon nog eens.}{gevaarlijk worden in haar}{onweersbuien}\\

\haiku{Doar 'n es niets in '!}{de weireld da zue dwoas es}{ofn mannemeins}\\

\haiku{Met hoge schouders,,,}{als drie schuldigen dropen}{de drie broeders af}\\

\haiku{- Moeten es dwang, doar ',.}{n es niets aan te doene}{sprak ze berustend}\\

\haiku{Maar eensklaps jankte,,.}{hij en wipte als onder}{een zweepklap half op}\\

\haiku{Zij stak haar fris en,:}{vrolijk-blozend blonde}{kopje uit zij riep}\\

\haiku{Van om de beurt met.}{Leontientje uit te gaan}{was niets gekomen}\\

\haiku{ja... Tante... die was!}{eigenlijk de gehele}{oorzaak van alles}\\

\haiku{'k 'n Wee niet hoe'!}{da g ulder op ulder}{hof nog tuegen durft}\\

\haiku{Zij glimlachte en,,.}{stak hem zonder wrok de hand}{ter verzoening toe}\\

\haiku{Hij begost nog ne.}{kier of twie\"e te weerluchten}{en te donderen}\\

\haiku{Informeer moar goed '.}{en blijf zue lange wig of}{datt nuedig es}\\

\haiku{en van dat ogenblik,.}{verlieten zij elkaar niet}{meer de hele dag}\\

\haiku{Doet er mee wat da.}{ge wilt en neemt er veuren}{da ge krijgen keunt}\\

\haiku{Men bleef niet te lang,;}{vooral niet's zondags wanneer}{Veel-Hoar het druk had}\\

\haiku{Daar hielden zij even.}{stil en schaterden er hun}{dolle pret wild uit}\\

\haiku{Je keek hem aan, je,:}{vroeg hem iets je praatte met}{hem een hele poos}\\

\haiku{Hij leek mij kleiner.}{en dikker dan ik hem in}{leven gekend had}\\

\haiku{Paffe!... doar neemt hij '...! -?...?}{zijne luep en sprijngt in}{t woater En en}\\

\haiku{- Haw\`el, hij lag-ie ',.}{hij int woater herhaalt}{deze doodgewoon}\\

\haiku{zijn diepliggende,;}{grijze ogen fonkelden in}{zijn steenrood gezicht}\\

\haiku{- Weet-e nog wel,,?}{Jan da ge mij ne kier ou}{strued verkocht h\`et}\\

\haiku{- Zij zelf voelden zich,.}{nu vaag belachelijk met}{al hun raar gefeest}\\

\haiku{De kerels waagden,:}{krasse toespelingen doch}{ook al vruchteloos}\\

\haiku{De boer getuigde.}{dat de jongen dikwijls in}{die schuur ging slapen}\\

\haiku{Zij lieten Jules,.}{los maar hun overtuiging stond}{onomstootbaar vast}\\

\haiku{Eensklaps en plechtig,!}{doodse stilte in plaats van}{het woelend rumoer}\\

\haiku{vader, een guitig,;}{gezicht altijd vrolijk en}{zeer jong van hart nog}\\

\haiku{Nu en dan bracht ik,;}{een bezoek op het kasteel}{waar vrienden woonden}\\

\subsection{Uit: Verzameld werk. Deel 5}

\haiku{Die blijde losheid.}{friste en knapte meester}{Gevers heerlijk op}\\

\haiku{Maar de zuster was;}{geheel ontredderd en wist}{hoegenaamd geen raad}\\

\haiku{eenmaal van wal zou, ',;}{het vanzelft zij goed of}{kwaad wel verder gaan}\\

\haiku{de geestelijke;}{had opgekeken en hem}{dadelijk herkend}\\

\haiku{viel hem de priester,.}{toornrood met fonkelende}{ogen in de rede}\\

\haiku{Zijn hersens waren,,}{leeg zijn wil was gebroken}{en reeds stamelde}\\

\haiku{maar meester Gevers,,,;}{was sluw noch slim noch strijder}{noch opportunist}\\

\haiku{verbaasde zich de.}{dikke vrouw met in elkaar}{geslagen handen}\\

\haiku{De meester beefde,,,.}{knikte stotterde wist niet}{wat te antwoorden}\\

\haiku{Mevrouw Speliers was;}{eventjes bij de meid in de}{keuken geroepen}\\

\haiku{Ik vraag u of kij?}{heb de brood gerefuseer}{aan meester Gevers}\\

\haiku{kreet de meester als,.}{waanzinnig begrijpend wat}{de oude ging doen}\\

\haiku{en middelerwijl;}{nam zijn verlangen en zijn}{liefde aldoor toe}\\

\haiku{En haar open schaar stond.}{als een bijtende snavel}{naar hem toegeprikt}\\

\haiku{Enkel op de dag...}{van de begrafenis ben}{ik er nog geweest}\\

\haiku{Maar Free was er de.}{kerel niet naar om zich te}{laten dwarsbomen}\\

\haiku{, en het overlijden.}{van zijn vader had zij hem}{niet kunnen melden}\\

\haiku{Doch de oude vrouw.}{was van vermoeienis en}{emotie uitgeput}\\

\haiku{Zij is weg, terug,.}{naar haar ouders en wij gaan}{wettelijk scheiden}\\

\haiku{Blijkbaar vragen ze,.}{weer naar de weg en die lui}{begrijpen hen niet}\\

\haiku{'t Is maandagavond,.}{de repetitie-avond}{van de dorpsmuziek}\\

\haiku{Het eerste antwoord:}{was van trage hoofdschudding}{en een afwijzend}\\

\haiku{Had hij nog ouders,,,?}{broeders zusters andere}{familieleden}\\

\haiku{Was dat {\textquoteleft}'t woar hier,{\textquoteright}...?}{woar doar geweest waarvan hij}{elke winter sprak}\\

\haiku{Ik weet een huisje,,...}{staan in Vlaanderen waar ik}{zou willen leven}\\

\haiku{men had hem maar voor,:}{de grap zo genoemd en het}{kon hem niet schelen}\\

\haiku{Het regende niet,.}{meer maar een gure wind blies}{snerpend om zijn oren}\\

\haiku{en hij ging hun door.}{de opengebleven deur met}{vlugge schreden voor}\\

\haiku{Zij beschouwden haar.}{nog als een kind dat buiten}{hun levenskring stond}\\

\haiku{De dikke vrouw kwam,,;}{soms maar o zo zelden in}{de omwaskamer}\\

\haiku{en ook de oevers:}{en de hemel trillen van}{schitterend leven}\\

\haiku{hoe graag zou ik hem,...}{geld willen geven om vlees}{veel vlees te kopen}\\

\haiku{Met alle kracht en,.}{aandacht is hij in zijn werk}{in zijn plicht verdiept}\\

\haiku{Wel neen, zij weten,?}{het niet meer maar wat komt het}{er nu ook op aan}\\

\haiku{- Hij is naar zijn land,,.}{terug antwoordde de baas}{blijkbaar gegeneerd}\\

\haiku{en daarbinnen, in,:}{de Rosbach was het als een}{pandemonium}\\

\haiku{Ja ja, dat lekker,!...}{bier eenmaal als men eraan}{was gewend geraakt}\\

\haiku{- Meneer, er zijn er}{daar al twee en ze zeggen}{dat ze hier komen}\\

\haiku{Ook de tweede man.}{kwam binnen en deed de deur}{achter zich  toe}\\

\haiku{Hij strompelde naar.}{zijn leunstoel en liet er zich}{zwaar in neervallen}\\

\haiku{Weer ging de deur open.}{en de zoon met de knappe}{meid traden binnen}\\

\haiku{Dadelijk deelde '.}{meneer Bollekens hunt}{gewichtig nieuws mee}\\

\haiku{In zijn jeugd moest hij,.}{zeker een flinke knappe}{kerel zijn geweest}\\

\haiku{en nu eens een die;}{men mooi vond en dan weer een}{die men lelijk vond}\\

\haiku{Wat kunnen ons de,?}{Duitsers de Fransen en de}{Engelsen schelen}\\

\haiku{Tuttuttut... dat zijn,!}{dingen die wij niet kennen}{die wij niet hebben}\\

\haiku{'t Was een nogal,.}{knappe blonde meid van een}{dertigtal jaren}\\

\haiku{Ik voelde, dat het.}{mens aan de praat wou en week}{langzaam achteruit}\\

\haiku{Maar ik zag dat de.}{vrouw hem een duw gaf en hij}{keek op en bleef staan}\\

\haiku{Ik nam mijn hoed af.}{voor Fietriene en wenste}{hun beiden geluk}\\

\haiku{hij liep gebogen.}{doch met forse wil tegen}{de windbuien in}\\

\haiku{meneer Cathoen en ';}{Fietriene dicht bij elkaar}{int mulle zand}\\

\haiku{Ik merkte slechts dat.}{zij kleiner en magerder}{was dan haar zuster}\\

\haiku{Hij sloot de deur en.}{draaide tweemaal de sleutel}{in het nachtslot om}\\

\haiku{De generaal, zijn,.}{wenkbrauwen fronsend bromde}{toornig in zichzelf}\\

\haiku{De generaal en.}{de twee dames spraken een}{hele poos geen woord}\\

\haiku{Het zwart en rood van.}{zijn uniform schitterde vaag}{in de duisternis}\\

\haiku{zijn uitspraak had een,.}{vreemde klank evenals die van}{het kleine meisje}\\

\haiku{Het waren eensklaps,.}{vrienden of zij elkander}{reeds jaren kenden}\\

\haiku{De beide dames.}{hadden zich omgekeerd en}{herkenden hem ook}\\

\haiku{Anderen wenkten.}{naar de meisjes en zonden}{dikke klapzoenen}\\

\haiku{ik ben de Dood en:}{mijn vraatzuchtige woede}{is afgrijselijk}\\

\haiku{Soms is het of er;}{duizenden en duizenden}{zwermden en zoemden}\\

\haiku{en Lovergem was slechts!}{anderhalf uur verwijderd}{van hun eigen dorp}\\

\haiku{Wat nu te Lovergem,:}{gebeurde zou straks ook te}{Bavel gebeuren}\\

\haiku{{\textquoteright} en weg waren ze,,:}{met rijwielen wagens en}{karren of te voet}\\

\haiku{Zelfs de oorlog had;}{hem zijn vermoeiend bedrijf}{niet doen  staken}\\

\haiku{het was de vlucht, de,!}{wilde uitzinnige vlucht}{op leven en dood}\\

\haiku{zij hoorden Vosken;}{aan met open mond en van schrik}{uitgezette ogen}\\

\haiku{Mijn officieren,.}{waren ernstig doch geenszins}{moedeloos gestemd}\\

\haiku{Soms hoorden wij vaag.}{geluid en gestommel in}{de donkere nacht}\\

\haiku{en men strekte de.}{vermoeide benen naar de}{rode vlammen uit}\\

\haiku{De doodsmare was,,;}{rondgestrooid niemand wist hoe}{niemand wist door wie}\\

\haiku{Angstig-gejaagd.}{schoolden zij samen op de}{dorpsplaats rond de kerk}\\

\haiku{De schoonzuster trok ',.}{t venster open keek in de}{schaars verlichte straat}\\

\haiku{en meteen bralde:}{hij onsamenhangende}{verwensingen uit}\\

\haiku{En feitelijk bleef:}{er maar \'e\'en slachtoffer in}{de gehele zaak}\\

\haiku{Met nieuwjaar werd hij.}{afgedankt en op een klein}{pensioen gesteld}\\

\haiku{Hun haat tegen het;}{nieuw vervoermiddel bleef stug}{en onveranderd}\\

\haiku{Zij had het Barontje, ';}{gehuwd om de titel en}{ook omt fortuin}\\

\haiku{anderen slechts heel,;}{weinig omdat ze te zeer}{gegeneerd waren}\\

\haiku{Meneer de pastoor.}{greep naar zijn beker en hief}{hem in de hoogte}\\

\haiku{meneer Fran\c{c}ois te '.}{voet ent Barontje in zijn}{mooie automobiel}\\

\haiku{Heliodoor stond '.}{int mooi gedeelte van}{de Grote Dorpsstraat}\\

\haiku{Pierke zei niets, maar.}{glimlachte en zijn sluwe}{oogjes tintelden}\\

\haiku{- Pakke moar, zei mijn '.}{oom Heliodoor hemt}{kopje aanreikend}\\

\haiku{Toen vulde hij de.}{lege mand met keien en}{keerde weer naar huis}\\

\haiku{- We zoen meniere,.}{wille spreken zei Pierke}{stil-neerslachtig}\\

\haiku{In \'e\'en lange lijn,.}{van de Zwitserse grens tot}{aan de Hollandse}\\

\haiku{Was de vijand daar,?}{reeds of waren het nog steeds}{meer vluchtelingen}\\

\haiku{En meteen, omdat,.}{hij zo rustig was voelden}{zij zichzelf veilig}\\

\haiku{De bosman blies zijn.}{lantaarntje uit en lachte}{in zijn zwarte baard}\\

\haiku{Zij hoorden niets dan.}{hun eigen geluid en dat}{werkte kalmerend}\\

\haiku{Meneer D\'esir\'e.}{opende zelf de flessen en}{vulde de glazen}\\

\haiku{wie niet gauw genoeg,.}{uit de weg ging werd gewoon}{omvergelopen}\\

\haiku{Men wees hem Het huis,.}{van Commercie waarin hij}{als een schicht verdween}\\

\haiku{Zij staakten hun spel.}{en keken de twee heren}{met verbazing aan}\\

\haiku{- D\'a\'ar, juist achter de,,.}{kerk de Vette Os een heel}{goed hotelletje}\\

\haiku{Os zag zitten een.}{wit couvert in de hoogte}{heen en weer zwaaide}\\

\haiku{Ik 'n mag\`e daar.}{niet op peinzen of ik zou}{beginnen schreemen}\\

\haiku{Nu dat is wel dat.}{gij in Oland zijt en wij zijn}{daar toch zoo blij om}\\

\haiku{Als het daar nu maar,.}{bij mocht blijven dan waren}{zij reeds half getroost}\\

\haiku{Ik aarzelde om.}{naar hen toe te gaan en om}{hen aan te spreken}\\

\haiku{Dat konden zij mij.}{eerst niet in duidelijke}{woorden vertellen}\\

\haiku{Heel zelden was er.}{een op jeugdige leeftijd}{ten grave gebracht}\\

\haiku{Er bloeiden altijd, ':}{mooie ouderwetse bloemen}{langst geveltje}\\

\haiku{Daarbuiten op de,.}{landweg stonden mensen in}{de lucht te kijken}\\

\haiku{Hij had de indruk,.}{dat hij aan een ontzettend}{gevaar was ontsnapt}\\

\haiku{Misschien zou God zich,,.}{over hun lot erbarmen hen}{helpen hen redden}\\

\haiku{de doodsangst stolde.}{hen in stijve en stomme}{onbeweeglijkheid}\\

\haiku{De steun, die hen zo,.}{dikwijls gespaard en gered}{had was er niet meer}\\

\haiku{'t Geen da ze bij '.}{ou nien nemen komen}{ze bij mij pakken}\\

\haiku{- Joa, Fielemiene, ';}{k zoe malgr\'e menier de}{p\'aster moete spreken}\\

\haiku{Meneer de pastoor.}{stond zwart en roerloos naast de}{kist en sprak geen woord}\\

\haiku{- 'k Gelueve da,.}{ze van den nacht goan deure}{breken zei Boerke}\\

\haiku{Het roggebrood, dat,;}{ze thans aten was niet slechter}{dan v\'o\'or de oorlog}\\

\haiku{- 'n Beetse lieger,, ' ' ',.}{Pierkn ziet nie mier}{klaagde de kleine}\\

\haiku{maar gij, Irma, moet,,,,...}{weg en gij ook Elodie en}{gij vooral Emma}\\

\haiku{zij betaalden goed,;}{evenveel en soms nog meer dan}{mevrouw de gravin}\\

\haiku{al hun gedachten,;}{al hun verlangens en hun}{hoop vlogen erheen}\\

\haiku{Toen bleef het bekje '.}{eindelijk wijd open staan en}{t oog verstarde}\\

\haiku{- We willen ne stien.}{veur Alineke geven in}{de nieuwe kirke}\\

\haiku{Zij togen naar 't.}{Kasteelken en vroegen om}{meneer te spreken}\\

\haiku{Wat was dat stil en,!}{verlaten zo'n groot dorp in}{nachtelijke rust}\\

\haiku{een troepje koeien,;}{rustig grazend als grote}{bloemen op het groen}\\

\haiku{Zij baden, de ogen,.}{strak op het grote roze}{gebouw gevestigd}\\

\haiku{veertien dorpen op,!}{en af asjeblief alles}{te voet heen en weer}\\

\haiku{Na een week was de;}{laatste zorg over Alinekes}{toestand verdwenen}\\

\haiku{Op hun uiterst best,,.}{gekleed als voor een kermis}{togen zij erheen}\\

\haiku{Dan golden daar nog {\textquoteleft}{\textquoteright};}{slechts de wetten en de geest}{van derepubliek}\\

\haiku{Hij es op 't land,,.}{antwoordde de moeder met}{inspanning wrijvend}\\

\haiku{- Ha, moeder, doar 'n...}{zijn nie veel gemienten in}{Vloanderen die}\\

\haiku{Buigend onder het.}{lage boogdeurtje trad ik}{achter hem binnen}\\

\haiku{de {\textquoteleft}burgemeester{\textquoteright} '!}{en de baas uit de Speurgaal}{t laatst van allen}\\

\haiku{Vandaar die blauwe,!}{plekken die men op het lijk}{geconstateerd had}\\

\haiku{Daar rustte dus Karl,,.}{de vijand die slecht geweest}{was voor de mensen}\\

\haiku{alleen het water...}{wist en zou voor eeuwig zijn}{geheim bewaren}\\

\haiku{Zij loeiden even, en,.}{ontlastten zich rustig de}{kop naar hem gekeerd}\\

\haiku{hij herhaalde het,,:}{schreeuwend snikkend jammerend}{op alle tonen}\\

\haiku{De burgemeester.}{trok zich met wethouder en}{veldwachter terug}\\

\haiku{'t zal misschien nog...!}{wel terecht komen gelijk}{de koe van Dons}\\

\haiku{vroeg ik na een poos,.}{om de benauwd wordende}{stilte te breken}\\

\haiku{Zij drongen aan, dat;}{ik even zou binnenkomen}{en iets gebruiken}\\

\haiku{Het was een mooi en,.}{vrij uitgestrekt meer geheel}{omringd door bossen}\\

\haiku{Hij is een ietsje.}{aan de drank verslaafd en drinkt}{altijd jenever}\\

\haiku{drinkt, ietwat morsend,.}{zijn borrel leeg en bestelt}{er een tweede}\\

\haiku{Hij keek naar iets, in,;}{de eerste rij stalles daar}{schuin beneden ons}\\

\haiku{Mijn illustere, '}{vriend sprak geen woord maart kwam}{mij voor alsof zijn}\\

\haiku{Instinctmatig, met,.}{een lichte huivering trok}{ik mij even terug}\\

\haiku{of mij te laten,}{aandienen de stoeptreden op}{en de hall binnen}\\

\haiku{en zij sloeg haar ogen,.}{neer terwijl een kleur even op}{haar wangen gloeide}\\

\haiku{Ik ken Marie reeds.}{zoveel jaren en heb ze}{in lang niet gezien}\\

\haiku{zo staan geweren, '.}{van soldaten in rust op}{t exercitieveld}\\

\haiku{'t is te zeggen;}{dag van schele hoofdpijn en}{van katterigheid}\\

\haiku{Daar kwam de Duitse.}{horde met kletterende}{hoeven aangestapt}\\

\haiku{maar nu hadden de:}{mensen niets geen pret meer in}{hun rare doening}\\

\haiku{De meeste boeren.}{en bewoners waren op}{hun erf gebleven}\\

\haiku{Op de ransel was,,}{een keteltje gebonden}{dat zij losmaakten}\\

\haiku{- Hot water, some,.}{hot water herhaalde de}{man ongeduldig}\\

\haiku{Terwijl hij sprak keek,.}{hij naar de omstaanders en}{ook naar de boeren}\\

\haiku{Toen wenkte hij de.}{boeren bij zich en deed hen}{de tombe vullen}\\

\haiku{Drie vreemde namen,,.}{stonden met zwarte letters}{erop geschilderd}\\

\haiku{riepen eensklaps de,.}{kerels onzacht het volk van}{de stoep wegduwend}\\

\haiku{Hij had een hok, vlak.}{v\'o\'or het huis en werd daar soms}{aan vastgeketend}\\

\haiku{wij waren daar reeds, '.}{z\'o aan gewend dat wet}{haast niet meer merkten}\\

\haiku{t Verleden kwam '.}{alst ware in golven}{naar hem toegestroomd}\\

\haiku{O, die kleren, die,!}{gezellige civiele}{kleren van weleer}\\

\haiku{Zij wrongen hem de.}{duimschroeven op de polsen}{en namen hem mee}\\

\haiku{te weten of hij,,!}{al dan niet tot kolonel}{zou promoveren}\\

\haiku{Met een pennetrek.}{konden zij hem voordragen}{en doen benoemen}\\

\haiku{Hij zuchtte kreunend.}{en zijn zwakke ogen keken}{zoekend om hem heen}\\

\haiku{En vol zachte zorg.}{drukte zij hem weer in de}{kussens achterover}\\

\haiku{En hij wil ook graag,.}{uw foto hebben als gij}{er een kunt missen}\\

\haiku{Toen werd hij eensklaps,,.}{wakker midden in de nacht}{en hoorde niets meer}\\

\haiku{De huizen waren,.}{er wit met rood pannendak}{en groene luiken}\\

\haiku{Hij zat te zingen}{op een heestertakje dat}{begon te groenen}\\

\haiku{ze noteerde en.}{zoals ook de moderne}{Theuriet ze opschreef}\\

\haiku{Ze komen zij hier,.}{binnen drijnken nen dreupel}{of twie\"e en zijn wig}\\

\haiku{'k Moak hem 't.}{beste eten geried dat op}{de weireld bestoat}\\

\haiku{Hij stond roerloos en ' '.}{keek int verschiet naar de}{kerk ent kasteel}\\

\haiku{Wij komen op een,.}{kruispunt waar ik even  naar}{de weg moet vragen}\\

\haiku{Ik denk dat het nu;}{wel langzaam aan tijd wordt om}{terug te keren}\\

\haiku{- Moar h\`et-e gij, ',?}{ulder nie gezeid datt}{wel woar es Gaston}\\

\haiku{Ge zoedt mij plezier,,.}{doen meniere mee hem nie}{mier mee te nemen}\\

\haiku{Zodra hij mij zag, {\textquoteleft}{\textquoteright}.}{komen begon dedikke}{luid te jubelen}\\

\haiku{- 't'n Ziet er giene,.}{gemakkelijke keirel}{uit merkte ik op}\\

\haiku{- Hein, monsieur le,?...}{vicomte qu'est-ce que}{vous dites de ca}\\

\haiku{Hun bedrijvige}{vingers rondden de bruine}{schillen als krullen}\\

\haiku{ik goa uek tegen,.}{den gevel zitten percies}{gelijk de meinschen}\\

\haiku{Een klokje luidde ',;}{ergens int verschiet heel}{dromerig en ver}\\

\haiku{vroeg hem grijnzend een.}{oude kerel met gouden}{kraag en grijze baard}\\

\haiku{Es da nou nie stom '?}{da ge mee d'ander meinschen}{nie gevluchtn zijt}\\

\haiku{V\'o\'or mij stond een man,,!}{een reus zoals ik er nog}{nooit een gezien had}\\

\haiku{vroeg mij, in vrij goed,,:}{Frans met een stem die zacht maar}{diep als koper klonk}\\

\haiku{Heren beginnen.}{altijd met elkander in}{een bar te brengen}\\

\haiku{Hij verslond haar met,,;}{zijn ogen hij snoof haar op}{hij smulde van haar}\\

\haiku{Wij hadden toch een.}{halve afspraak om nog eens}{samen uit te gaan}\\

\haiku{Wat 'n idee om daar!}{alleen te komen zitten}{in die luxeloge}\\

\haiku{de pauze viel in.}{en een aantal toeschouwers}{verlieten hun plaats}\\

\haiku{Zij stegen in de.}{wagen en Vien nam weer zijn}{plaats in achter hen}\\

\haiku{Je rustte uit, met;}{de ellebogen op het}{vieze tafeltje}\\

\haiku{De voeten waren,.}{plat en klein opvallend klein}{voor zulk zwaar lichaam}\\

\haiku{Ik zag hem zijn keus,,.}{doen betalen met een pak}{naar buiten komen}\\

\haiku{- Past op, sloeber, os!}{g'azue noar mij blijf kijken}{en mij blijf volgen}\\

\haiku{- Ha! 't'n ziet er nog,!}{giene gemakkelijken}{uit dienen dikke}\\

\haiku{een zoon, die klein was,,.}{als hijzelf en een dochter}{die een bochel had}\\

\haiku{- Ja of neen, wilt gij?}{ons de plaats aanwijzen waar}{uw voorraad meel ligt}\\

\haiku{Maar weer namen de ';}{gendarmen daar niet int}{minst notitie van}\\

\haiku{En weer galmde de:}{lang-gerekte kreet als}{een hanengekraai}\\

\haiku{Uzubup\'u kwam voor en.}{leidde de bezoeker in}{het spreekkamertje}\\

\haiku{Hij dacht daarover na.}{met sardonische glimlach}{en stekende ogen}\\

\haiku{Met een soort moedwil.}{keerde B\'er\'enice zich}{tot haar vader om}\\

\haiku{Toch voelden zij wel,.}{de grote plotselinge}{leegte om zich heen}\\

\haiku{de auto was al...}{voorbij en zij zagen het}{kruisje niet meer}\\

\haiku{Zijn vrouw is lang en,.}{mager met ingevallen}{borst en een scheel oog}\\

\haiku{hij heeft zijn flesje,;}{en zijn pijpje hij geniet}{en is gelukkig}\\

\haiku{Marcel keek naar het;}{huis in wording en schudde}{ook wel eens het hoofd}\\

\haiku{Jan-Sies wendde:}{zich vriendelijk tot Marcel}{en vroeg glimlachend}\\

\haiku{Maar hij moest wel iets;}{verzinnen om zijn bezoek}{te rechtvaardigen}\\

\haiku{- Ha, da zoe toch 'n!}{schande zijn dat da azue hiel}{de nacht moest duren}\\

\haiku{Het hanengeschrei.}{klonk verzwakt achter de tuin}{van meneer Alexander}\\

\haiku{Ik zag de oude -;}{schoolmeester De Brave al}{zoveel jaren dood}\\

\haiku{In die tijd kenden;}{wij geen onderscheid tussen}{mooi en lelijk Frans}\\

\haiku{ook zij was ervan,.}{overtuigd dat ik mijn straf ten}{volle had verdiend}\\

\haiku{ging de vrouw, die hem,.}{voorbij het raam zien komen}{had hem tegemoet}\\

\haiku{- Pijn 'n h\`e 'k niet,, ' '.}{menier Ag\'ust moarkn zie}{hoast nie  mier}\\

\haiku{- 'k 'n H\`e doar nog,,.}{niet op gepeisd Celestien}{antwoordde zij kil}\\

\haiku{Met diepe smeking.}{in zijn lodderige ogen}{staarde hij haar aan}\\

\haiku{{\textquoteleft}drijnkt ou iest 'n stik' '.}{in ou kroage os g}{azue nien durft goan}\\

\haiku{Octavie schudde.}{het hoofd en maakte grote}{armbewegingen}\\

\haiku{Hij haalde rustig.}{zijn koker te voorschijn en}{stak een sigaar op}\\

\haiku{Da Celestien hem!}{nou en dan komt zat drijnken}{in ou vroeger huis}\\

\haiku{Zij sprak erover met.}{Celestien en die nam het}{ook vrij kwalijk op}\\

\haiku{en zij spoedde zich,.}{op straat om te horen hoe}{de zaken stonden}\\

\haiku{riep ze gans ontdaan,;}{tot de oude tante die}{bij haar inwoonde}\\

\haiku{gilde Hortense,,.}{sidderend van toorn toen ze}{weer bij Tante was}\\

\haiku{'k zoe uek nog ne, '.}{kier willen dansen h\^en op}{n airken meziek}\\

\haiku{en as ik het doe ',;}{t es wel te wille van}{mevreiwe zulle}\\

\haiku{Zij hadden stille.}{pret om het zuur gedoe van}{de beide vrouwen}\\

\haiku{- Afijn, Hortense, as ' '....}{get liever nien h\`et}{hernam Anneke}\\

\haiku{Kamiel is er ten;}{zeerste op gesteld om mee}{te blijven spelen}\\

\haiku{Hij is z\'o lang, dat;}{hij meestal onder het}{hemd wordt weggestopt}\\

\haiku{Als hij u aankijkt,.}{weet gij eigenlijk nooit of}{hij u wel aankijkt}\\

\haiku{Talrijke manden;}{vol vis stonden er in een}{kring om het pleintje}\\

\haiku{h\`et-e gij geld? '!}{om te betoalent Es}{twie en tseventig fran}\\

\haiku{ik zou einden ver,.}{gelopen zijn om er niet}{meer langs te komen}\\

\haiku{En hij haastte zich,.}{uit het huis de bankjes in}{zijn broekzak stoppend}\\

\haiku{En bulderend sloeg.}{hij weer met beide handen}{op zijn knie\"en}\\

\haiku{riep Vertriest, de,.}{huisschilder die gekleed stond}{als om uit te gaan}\\

\haiku{Hij gaf vijf frank fooi,.}{aan de bediende die ten}{diepste dankte}\\

\haiku{riep eensklaps Daenens,.}{die tot nog toe de mond haast}{niet geopend had}\\

\haiku{- Ala, kom binnen en'.}{drijnk ne pot kaffee ier da}{g aan ou wirk goat}\\

\haiku{vloekte Allewies, '.}{die ookt beledigend}{gedoe gezien had}\\

\haiku{Hij woonde daar heel,.}{alleen met een oude knecht}{en een oude meid}\\

\haiku{Men zette haar even,.}{neer op het gras onder de}{bloeiende kruinen}\\

\haiku{De mannen namen.}{de kist weer op en tilden}{haar op hun schouders}\\

\haiku{Eigenlijk was het ' '.}{leven wel steedst zelfde}{opt platteland}\\

\haiku{Men stond vroeg op, men;}{gebruikte het ontbijt en}{ging naar de akker}\\

\haiku{boter, eieren,....}{ham en ook geregeld wat}{geld laten zenden}\\

\haiku{Zoals vanzelf spreekt,.}{verviel hij daarbij wel eens}{in herhalingen}\\

\haiku{Hij lachte mij uit,;}{omdat ik per rijwiel tot}{daar was gekomen}\\

\haiku{Men kon wel merken,;}{dat hij gaandeweg geler}{en magerder werd}\\

\haiku{Het was om zo te.}{zeggen een sociale}{betrekking voor hem}\\

\haiku{Zij mogen alles, '!}{hebben als ze mij maart}{leven laten}\\

\haiku{En de uitdrukking,:}{waarmee hij ons nakeek zal}{ik nooit vergeten}\\

\haiku{Alleen haar mond was.}{tandeloos geworden en}{zij liep gebogen}\\

\haiku{Hij zette hun 't;}{mes op de keel en eiste}{hun verborgen geld}\\

\haiku{Soarlewie schudde.}{het hoofd en liet moedeloos}{de armen zakken}\\

\haiku{- Mais non, zei hij, - c'est;}{la premi\`ere fois que}{je passe par ici}\\

\haiku{Meneer Alfred zag,.}{zeer bleek met een pijnlijke}{grijns om de lippen}\\

\haiku{Hij kwam binnen in.}{de Schone Warande om}{afscheid te nemen}\\

\haiku{Hij had het zo graag,;}{willen weten hij had het}{haar willen vragen}\\

\haiku{Een glimlach kwam over.}{zijn lippen en even praatte}{hij gewoon met haar}\\

\haiku{Geen mens meer in de.}{omtrek en geen geluid meer}{in de stille nacht}\\

\haiku{De dokter kon er;}{zo grappig en smakelijk}{zitten vertellen}\\

\haiku{wie Anzelieksken;}{kon doen lachen had succes}{en was tevreden}\\

\haiku{Hij lei zijn vette.}{poot tussen haar vingers en}{drukte die zwijgend}\\

\haiku{ook zij was gans in ',;}{t zwart gekleed in zwarte}{zij met git erover}\\

\haiku{Zij zag eruit als,,!}{een dame waarachtig als}{een echte dame}\\

\haiku{Zij droeg ook lange,.}{gouden oorbellen en een}{diamanten broche}\\

\haiku{Pruttelend draaide.}{zij de lichten uit en ging}{hem voor naar boven}\\

\haiku{maar gehoorzamen,,.}{zou hij en werken zou hij}{ook nog als een knecht}\\

\haiku{Tegen de oude {\textquoteleft}{\textquoteright}.}{stam van deSperre hing een}{klein kapelletje}\\

\haiku{Het was iets nieuws, iets,!}{onverwachts en onbekends}{iets hartstochtwekkends}\\

\haiku{joelden de bengels.}{in dolle pret  over Bruno's}{grappig gezegde}\\

\haiku{hijgde de dikke,.}{vrouw naar het ontdaan gelaat}{van haar man starogend}\\

\haiku{Hij nam het glas in.}{de hand en hield het naar het}{schaarse daglicht toe}\\

\haiku{en nog zachter, als,:}{schromend en zich schamend met}{bevende lippen}\\

\haiku{Hij liep gebogen,,.}{met hoge schouders tegen}{de gure wind in}\\

\haiku{Dat was wat hij het.}{meest ontbeerde en waaraan}{hij niet wennen kon}\\

\haiku{Micus kreeg een vreemd '.}{gevoel over zich en zijn hart}{ging aant jagen}\\

\haiku{Zij zat daar nog een,.}{ogenblik gedrochtelijk als}{een gruwelmonster}\\

\haiku{De kleine met zijn ';}{dikke snor en kikkerogen}{zat int midden}\\

\haiku{Hij keek dan niet naar,;}{het kind waarmee hij anders}{zo gaarne speelde}\\

\haiku{Andr\'e drukte haar.}{nauw tegen zich aan en sloeg}{de lok naar achter}\\

\haiku{Het was acht uur en.}{hij wist dat Marcela v\'o\'or}{donker moest thuis zijn}\\

\haiku{Zijn stem was eensklaps.}{week geworden en tranen}{stonden in zijn ogen}\\

\haiku{- Mematsjen, geef mij!}{nog nen dreupel en geeft er}{menier uek ienen}\\

\haiku{zijn vrouw, geboren,,.}{Bokkaert-Van Imme deed}{open Het mij binnen}\\

\haiku{Zijn wangen waren.}{ietwat ingevallen en}{zijn ogen stonden dof}\\

\haiku{Alleen zijn neus bleef.}{gloeien als een tomaat in}{zijn verlept gezicht}\\

\haiku{en hij weet heel goed,.}{dat hij niet mag en dat het}{hem verboden is}\\

\haiku{Men droomt er weg in;}{po\"ezie bij het gezang}{van de vogelen}\\

\haiku{Ik open de deur en,.}{als een kogel vliegt hij de}{trap af naar buiten}\\

\haiku{en vragen rijzen,...}{in mij op die ik niet kan}{beantwoorden}\\

\haiku{Het is zeer zeker;}{op zichzelf een feit van geen}{of weinig belang}\\

\haiku{- Ik niet hoor, ik heb...!}{er net zoveel leed over als}{u. Mysterie}\\

\haiku{De hoofden buigen,.}{een vage triestigheid komt}{over de gezichten}\\

\haiku{Ik dacht, dat broertje.}{er wellicht de weinige}{geur had afgelikt}\\

\haiku{- Wel, ik meen, dat ik!}{mij de teleurstelling niet}{erg zou aantrekken}\\

\haiku{Wij schaterden z\'o,.}{dat moeder bijna angstig}{voor haar raam kwam staan}\\

\haiku{Had ik maar gedurfd,.}{dan zou ik nog m\'e\'er op zijn}{bord gelegd hebben}\\

\haiku{'k 'n kwam moar ne, '.}{kier informeren hoe dat}{t mee Luusken es}\\

\haiku{Mijn vader was een,.}{bezadigd man  die van}{maat en orde hield}\\

\haiku{Stormen, oproeren,;}{oorlogen hadden over en}{om hem heen gewoed}\\

\haiku{Teleurgesteld ging '.}{ik er rondom heen en kwam}{terug opt plein}\\

\haiku{Ook een twintigtal!}{blaffende en brullende}{honden te water}\\

\haiku{En wreed blikte hij ',,.}{int ronde uitdagend}{met zijn schele ogen}\\

\haiku{geen wrede vuurflitsen.}{branden meer dreigend uit de}{grijze golven op}\\

\haiku{De chef van de wacht.}{stond op en keek reikhalzend}{over de hoofden heen}\\

\haiku{Een hele poos stond,.}{hij roerloos-luisterend stil}{om het te horen}\\

\haiku{Peet, door de woestheid,;}{van zijn aanval meegesleept}{viel boven op hem}\\

\haiku{Men hoeft het niet te.}{vragen of zij vers uit de}{loopgraven komen}\\

\haiku{Er zijn veel vrouwen.}{uit het volk en kinderen}{in die menigte}\\

\haiku{Die angst en droefheid.}{hangen voortdurend over ons}{gehele leven}\\

\haiku{'t Was tijdens de,,.}{noenstond tussen een en twee}{anders hun rustuur}\\

\haiku{Het ongeloof, de.}{slechtheid van de wereld is}{de schuld van alles}\\

\haiku{Zij patrouilleren,.}{te paard in kleine troepjes}{alom over het land}\\

\haiku{die kiekens liepen,!}{huele doagen zat zue}{zat as meinschen}\\

\haiku{D\'a\'ar, in die hoek, stond.}{de lijkwagen en zij moesten}{er rakelings langs}\\

\haiku{s Avonds v\'o\'or elke.}{terechtstelling had er een}{repetitie plaats}\\

\haiku{Het floddert tegen,.}{een van de ramen of het}{wou binnenkomen}\\

\haiku{Waarom is de Mens ';}{de natuurlijke vijand}{vant meesje}\\

\haiku{zij lopen nog eens, ',;}{hier en daar alst ware}{doelloos door elkaar}\\

\haiku{Ik zag het gaan in ',,...}{t felle zonnelicht steeds}{verder steeds verder}\\

\haiku{Daar stond ik weer met:}{mijn vraag aan de oever van}{het klotsend water}\\

\haiku{Eensklaps... wat kan een!}{beest toch raar doen en wat mag}{het soms bezielen}\\

\haiku{Dat was een van de.}{lievelingsgerechten van}{hun overleden vriend}\\

\haiku{Maar zijn stuur was hij.}{kwijt en dat zou hem wellicht}{duur komen te staan}\\

\haiku{Ik ga terug naar.}{de kleedkamer en kom weer}{gewoon te voorschijn}\\

\haiku{Maar hij bukt zich en.}{ik vlieg  door mijn eigen}{vaart tegen de grond}\\

\haiku{se lan\c{c}aient sur.}{la proie et l'avalaient}{avec gloutonnerie}\\

\haiku{Elle \'etait loin la!}{joie insouciante de}{ses jeunes ann\'ees}\\

\haiku{Affaiss\'ee entre les,,.}{bras d'Eiso Janke pleurait}{toujours abondamment}\\

\haiku{Betje l'aper\c{c}ut;}{mais fit semblant de ne pas}{le reconna{\^\i}tre}\\

\haiku{puis, avec un cri de,;}{jubilation elle}{se pr\'ecipita}\\

\haiku{Mais il marche, tu,!}{verras nous les battrons au}{prochain voyage}\\

\haiku{{\textquoteright} Il s'interrompit,,;}{brusquement la regarda en}{face dans les yeux}\\

\haiku{prosp\`ere et le,;}{114 allait revenir il}{devait revenir}\\

\haiku{Le chemin le plus,,:}{court pour rentrer chez lui \'etait}{d'obliquer \`a droite}\\

\haiku{Une seule chose:}{en lui \'etait pr\'ecise et}{toute puissante}\\

\haiku{Un \'elan de folle.}{jalousie et de rage}{le bouleversa}\\

\haiku{L'un, l'ouvrier, passait.}{inaper\c{c}u et on le}{laissait tranquille}\\

\haiku{se cramponnaient,.}{les uns aux autres pour ne}{pas \^etre balay\'es}\\

\subsection{Uit: Verzameld werk. Deel 6}

\haiku{Men kan het zonder{\textquoteright}.}{aarzeling vergelijken}{met deze stukken}\\

\haiku{Deze titel werd.}{geschrapt en vervangen door}{Familiedrama}\\

\haiku{Levensschetsen en.}{Portretten bijeengebracht}{door Mr. J. Kalff jr}\\

\haiku{Levensschetsen en.);}{Portretten bijeengebracht}{door Mr. J. Kalff jr}\\

\haiku{Wat staat dat frivool,,!}{op die oude deftige}{statige bomen}\\

\haiku{De zon komt door de.}{grijze wolken piepen en}{lonkt mij lachend toe}\\

\haiku{'t Gebeurde daar,.}{ergens in Vlaanderen op}{mijn geboortedorp}\\

\haiku{En voor de tweede.}{maal bleef meneer de pastoor}{hopeloos steken}\\

\haiku{De vader neemt het.}{in zijn armen op en kust}{het als waanzinnig}\\

\haiku{Ik trek aan het raam,:}{om hem weg te jagen maar}{het is reeds te laat}\\

\haiku{Voor de tweede maal.}{wordt hij naar een andere}{windrichting gekeerd}\\

\haiku{hij is geen profeet,,;}{zegt hij zelf en de mensen}{weten dat ook wel}\\

\haiku{Wat lijkt de oude,!}{grijze molen ijl en fijn}{in dat etherisch licht}\\

\haiku{Een ouderwetse,,;}{rammelende koets maar toch}{een luxe-en-erekoets}\\

\haiku{en tussen al dat.}{groen vonkt en spat het goud van}{de bloeiende brem}\\

\haiku{Het waait, maar hoe de,.}{wind ook zit geen tochtje kan}{mij daar hinderen}\\

\haiku{en de aarde, die,.}{dorst heeft krijgt te veel en kan}{alles niet slikken}\\

\haiku{Trouwens, het valse,,.}{het onechte grijnst u al}{dadelijk tegen}\\

\haiku{- Joa moar, Van Rompu,?}{h\`et-e gij die muerd}{nie zien gebeuren}\\

\haiku{Ik heb geleerd te,.}{doen gelijk het oud wijs paard}{van mijn molenaar}\\

\haiku{In \'e\'en adem stormde.}{ik de heuvel op en kwam}{ook buiten adem aan}\\

\haiku{Hoe het mij bekend,,.}{was dat hij zo heette kan}{ik niet verklaren}\\

\haiku{Alles is grijs, en,.}{ook guur en kil als op een}{novemberdag}\\

\haiku{Er ligt nog zulk een, '!}{schone rijke toekomst voor}{ons int verschiet}\\

\haiku{Het hoge koren.}{staat al opeens vol rode}{en blauwe bloemen}\\

\haiku{die hebben genoeg,;}{aan hun eigen stralende}{blozende schoonheid}\\

\haiku{Hier en daar is er;}{een wit hoofdje onder al}{die blauwe hoofdjes}\\

\haiku{vlak naast mij een zeer,;}{verliefd en jeugdig paartje}{dat heel veel pret had}\\

\haiku{'t Was in de buurt,.}{van Tripoli tijdens de}{zware gevechten}\\

\haiku{men wil en moet nog ':}{steeds en nogmaals de klank van}{t wonder horen}\\

\haiku{Wat is het geluid!}{van een sikkel gans anders}{dan dat van een zeis}\\

\haiku{Het ganse land trilt '.}{alst ware en wemelt}{van bedrijvigheid}\\

\haiku{Wat staan ze mooi en!}{rijk en warm goudbruin daar in}{de zon te gloeien}\\

\haiku{Toen sloot zich weer het,.}{grijs gordijn en eensklaps werd}{het avond triestig avond}\\

\haiku{Wat horen ze, sinds,!}{eeuwen al onveranderd}{bij onze volksaard}\\

\haiku{Ik heb hem aan de.}{trein gebracht en ben alleen}{weer thuis gekomen}\\

\haiku{ben ik genoodzaakt...}{voor uw zo gewaardeerde}{uitnodiging}\\

\haiku{daar verkondigt hij,.}{zelf een leer zodat ik niet}{hoef te antwoorden}\\

\haiku{De glorie van de.}{hoge bomen scheen hen gans}{te overweldigen}\\

\haiku{Eeuwigdurend zijn.}{ze en onveranderd in}{de loop der tijden}\\

\haiku{Mijn oude molen.}{heeft nieuw doek over twee van zijn}{wieken gekregen}\\

\haiku{Zij waren vuil en '}{vies en lelijk en int}{voorbijgaan roken}\\

\haiku{De molen heeft zijn.}{twee andere zeilen ook}{vernieuwd gekregen}\\

\haiku{Is het niet alsof?}{je levend in je donker}{graf werd neergelegd}\\

\haiku{{\textquoteleft}Mon \^ame est triste{\textquoteright}... {\textquoteleft}{\textquoteright} {\textquoteleft}{\textquoteright}.}{15 oktoberMon \^ame}{is niet meertriste}\\

\haiku{In het kerkje speelt;}{het orgel en weergalmen}{plechtige stemmen}\\

\haiku{- Omdat de mussen!...}{er anders piepen dan in}{ons eigen dorp}\\

\haiku{Nooit zou ik verwacht,.}{hebben in die wereld zo}{iets te aanschouwen}\\

\haiku{Het ogenblik daarna,.}{is alles stil alsof er}{niets meer gebeurde}\\

\haiku{Of is dit nu het?}{eind van alles en daagt er}{zelfs geen morgen meer}\\

\haiku{Ik moet hem nog eens,;}{van dichtbij aanschouwen hij}{leeft intens vanavond}\\

\haiku{de ganse natuur,,.}{schreit haar rouw haar droefheid}{haar ellende uit}\\

\haiku{Soms zie ik, in die,;}{nieuwe wereld mensen die}{ik hier gekend heb}\\

\haiku{de lansier draait en,.}{draait om er wee en razend}{van te worden}\\

\haiku{en er  was een,,.}{witte stenen molen met}{helrode wieken}\\

\haiku{De plek waar Sint Elooi,.}{eenmaal gestaan had kon hij}{niet terugvinden}\\

\haiku{Rechts lag de Rode,,,.}{Berg verder op de Franse}{grens de Zwarte Berg}\\

\haiku{Ik stond in de stad.}{Ieperen voor ik er mij}{rekenschap van gaf}\\

\haiku{Daar komt iets aan, in,...}{het verschiet over de grijze}{naaktheid van de weg}\\

\haiku{En zij valt uit in,.}{een vloed van verwijten die}{niet te stelpen zijn}\\

\haiku{Dat haar boerderij,.}{aan stukken geschoten is}{daarin berust ze}\\

\haiku{een draaitje aan de,.}{slinger en weg is hij naar}{betere oorden}\\

\haiku{even krijgen wij een, (,!}{mooie zachte grintwego welk}{een verademing}\\

\haiku{Waren we maar door!}{die ellendige trompet}{niet opgehouden}\\

\haiku{Weten die domme?}{lui in Bouillon dan nog niet}{wat sandwiches zijn}\\

\haiku{We hadden 't ook!}{nog wel gedeeltelijk uit}{zuinigheid gedaan}\\

\haiku{Je hebt het gevoel;}{of ze nu voortdurend weer}{zullen platlopen}\\

\haiku{De wegen zelf zijn,,!}{er helaas in Frankrijk niet}{op vooruitgegaan}\\

\haiku{Er is werkelijk.}{t\'e veel wijn en t\'e weinig}{water in die streek}\\

\haiku{Het was alsof 't,.}{alleen bestond boven en}{buiten alles om}\\

\haiku{De bomen en de.}{heesters om ons heen waren}{gitzwart geworden}\\

\haiku{Je suis venu ici,,.}{avec monsieur il y a}{cinq ans en auto}\\

\haiku{Spelende dames,,. '}{vooral winnende dames}{moet je niet storen}\\

\haiku{Wat klinkt die muziek,...!}{verleidend en meeslepend}{als de bank betaalt}\\

\haiku{Zij gaan even in een,.}{hoekje zitten ledigen}{hun beurs en tellen}\\

\haiku{Wat lijkt het klein, klein,,!}{armzalig klein van zo}{hoog en van zo ver}\\

\haiku{en, tot een man, die:}{juist toevallig op zijn fiets}{ons tegemoet kwam}\\

\haiku{Die ru{\"\i}neren.}{de weg en men heeft geen tijd}{hem te herstellen}\\

\haiku{- C'est bien beau quand on,,,.}{est l\`a-haut verzekerde}{kalmbewust de man}\\

\haiku{Wij reden... Maar 't;}{was een dag van tegenspoed}{en van vergissing}\\

\haiku{riepen mijn dames,.}{tot de eerste man die zij}{konden aanklampen}\\

\haiku{klonk het verbaasde,.}{antwoord net als op de brug}{van de Bidassoa}\\

\haiku{Nog v\'o\'or mijn dames,.}{daartoe bevel geven houd}{ik de wagen stil}\\

\haiku{Mensen en dieren,.}{begrijpen elkander nog}{niet of niet meer}\\

\haiku{- We zitten hier als,.}{schipbreukelingen zuchtte}{\'e\'en van mijn dames}\\

\haiku{Ik zie nauwelijks,.}{nog mijn weg een weg van slijk}{en modderplassen}\\

\haiku{Het orgel zweeg, de,.}{stemmen zwegen de kaarsen}{werden uitgedoofd}\\

\haiku{Er was de lange,, ';}{blonde weg lijnrecht tot in}{t onzichtbare}\\

\haiku{'t Zijn als hopen:}{en bundels linten van de}{prachtigste kleuren}\\

\haiku{In 't geheel niet,,;}{maar zodra ik champagne}{zie moet ik zingen}\\

\haiku{Zij springt, en meteen,!}{struikelt en valt zij een kort}{gilletje slakend}\\

\haiku{Die is het welke '.}{plotseling weer een schrik over}{t gezelschap jaagt}\\

\haiku{roep ik, verbaasd dat.}{le Dieu met geen enkel}{woord zijn hond beknort}\\

\haiku{En nu nog wel in ',!}{t gure najaar in een}{open automobiel}\\

\haiku{roept Maeterlinck,.}{met een gebaar van wanhoop}{vluchtig omkijkend}\\

\haiku{in Mont\'elimar;}{wordt even opgehouden om}{nougat te kopen}\\

\haiku{Hond en kat schijnen;}{elkaar met diepe aandacht}{te bestuderen}\\

\haiku{Zijn goed wordt gedroogd,;}{hij krijgt een ander pijpje}{en verse tabak}\\

\haiku{Hij komt nu weldra,.}{uit in Groot-Nederland}{en later in boek}\\

\haiku{{\textquoteleft}Nu mag, om het even:}{welke kolossaalsterke}{bruut mij aanranden}\\

\haiku{In welke plaats zal?...}{ik hem voor de zevende}{maal bezoeken}\\

\haiku{Zij hebben op het;}{slagveld de kogels om hun}{hoofd horen suizen}\\

\haiku{I done my small,.}{letter because y ave}{headache}\\

\haiku{De vormen van het,:}{grote schip tekenen zich}{af in het vage}\\

\haiku{Zal hij straks, als het,?}{schip uitvaart toespringen en}{de moorddaad plegen}\\

\haiku{tot eensklaps uit het:}{mistige grauw-grijs een}{vage schim opdoemt}\\

\haiku{O, wat zijn er veel,,!}{ouden bij en zwakken en}{half gebrekkigen}\\

\haiku{Het was een knappe,.}{jonge vrouw in de volle}{kracht van haar leven}\\

\haiku{De kleine krijgt een.}{soort van crisis en laat zich}{op de grond vallen}\\

\haiku{Met een angstgil neemt;}{ma hem in haar armen op}{en vlucht ermee weg}\\

\haiku{meten, die sinds een!}{eeuw op militair gebied}{werd afgelegd}\\

\haiku{en zei mij, dat het.}{streng verboden was enig licht}{te laten schijnen}\\

\haiku{Jean d'Aire, Jacques.}{et Pierre de Wissant en}{de drie anderen}\\

\haiku{Ik wou nog wel graag,,.}{verder maar begrijp dat het}{onmogelijk is}\\

\haiku{En spelend met zijn...}{stokje wenst hij ons verder}{goede reis toe}\\

\haiku{een stal, of schuur, of,.}{loods of boerenwoning die}{daar vroeger niet stond}\\

\haiku{En ook hier valt het,.}{mij op dat uiterlijk haast}{niets veranderd schijnt}\\

\haiku{Zo ziet men soms de,.}{straten van een stad in heel}{vroege morgenuren}\\

\haiku{- Diksmuide, zei hij,.}{langzaam de verrekijker}{aan zijn ogen zettend}\\

\haiku{De ruime schuur, de,,.}{hooizolders het wagenhok}{alles zit propvol}\\

\haiku{{\textquoteleft}'t Is vreemd{\textquoteright}, zegt hij, {\textquoteleft}!}{eindelijkzou er vandaag}{dan niets gebeuren}\\

\haiku{Zo reden wij een,.}{lange poos door absoluut}{verlaten oorden}\\

\haiku{- Gij zijt helemaal,,!}{uit de weg kapitein ge}{rijdt recht op D. af}\\

\haiku{Maar dat alles ging;}{zonder de minste uiting}{van smart of droefheid}\\

\haiku{Hij trachtte ons van,;}{het plan af te brengen ons}{te ontmoedigen}\\

\haiku{Ik weet niet hoeveel;}{de bevolking van Lens op}{dit ogenblik bedraagt}\\

\haiku{En wat 'n luxe in!}{een land dat heet verarmd en}{uitgeput te zijn}\\

\haiku{De Filosoof keert:}{zijn mistroostig gezicht naar}{ons om en antwoordt}\\

\haiku{Het sterke leven.}{van Frankrijk ligt begraven}{op zijn slagvelden}\\

\haiku{{\textquoteright} antwoordde mij een.}{Franse vrouw aan wie ik de}{opmerking maakte}\\

\haiku{Hoeveel honderden;}{karrewielen hebben over}{hem heen gereden}\\

\haiku{Hij stapte binnen, '.}{buigend ondert deurtje}{en wij volgden hem}\\

\haiku{Het meisje ging met,.}{vlugge vaste schreden naar}{de schenktafel toe}\\

\haiku{Gaston loopt langs de:}{wielen rond en al spoedig}{verneem ik zijn stem}\\

\haiku{De nieuwe band ligt.}{op en wij dalen naar het}{lieve stadje af}\\

\haiku{Hun debiet bestaat.}{uitsluitend te Parijs en}{in het buitenland}\\

\haiku{De grijze, leien.}{daakjes doezelen weg in}{geheimzinnigheid}\\

\haiku{Nogal wat kijkers,.}{op de drempels maar weinig}{of geen hoon en spot}\\

\haiku{Ik zie alleen de '.}{witte rookpluim bovent}{stationsgebouw}\\

\haiku{Wat is een auto,,,?}{ook de beste de duurste}{de meest volmaakte}\\

\haiku{- Gaston, da schijnen,.}{mij hier nogal geschikte}{meinschen zei  ik}\\

\haiku{Was het het vuur dat,...?}{mij bedwelmde of was het}{de geest van het huis}\\

\haiku{Geef ik wel aan de?}{rechte personen die ons}{geholpen hebben}\\

\haiku{Er hing een riempje,;}{te klepperen ergens in}{het bovenste bed}\\

\haiku{ook niet zo druk en.}{toch geeft het meer de indruk}{van een wereldstad}\\

\haiku{Zij schenen het land,.}{alleen meer kracht te schenken}{zijn woestheid delend}\\

\haiku{men is hier niet in,.}{een Spaans maar in een Engels}{landelijk hotel}\\

\haiku{Morgen weer tennis,.}{golf en tea en eten van de}{kok uit Engeland}\\

\haiku{Wat schijnen ze klein,,!}{op een afstand onder de}{geweldige rots}\\

\haiku{Omdat wij alleen!}{wensen weg te jagen en}{nooit zelf te wijken}\\

\haiku{Witte gestalten,.}{knielen biddend neer met het}{hoofd tegen de grond}\\

\haiku{{\textquoteright} met rollende r's:}{en buitengewoon-sterke}{klemtoon op de i}\\

\haiku{alleen het zwijgend.}{protest van hun totale}{onverschilligheid}\\

\haiku{De lange halzen,.}{bewegen zij vragend als}{zoekend heen en weer}\\

\haiku{- Dat zijn de bergen!}{en de sneeuwtoppen van de}{Atlas-keten}\\

\haiku{Wij komen v\'o\'or een,;}{hoge muur met prachtige}{Moorse ingangpoort}\\

\haiku{Beneden ons lag.}{de uitgestrekte stad in}{haar palmentuinen}\\

\haiku{Een luisterrijke;}{praal hier te midden van al}{dit vreemde bestaan}\\

\haiku{Een maanlandschap is!}{iets doods en dit leefde zo}{aangrijpend-intens}\\

\haiku{{\textquoteright} zei hij niet zonder,.}{trots alsof het iets was dat}{hem toebehoorde}\\

\haiku{Wat is dat alles!}{nog recent en wat lijkt het}{toch al lang voorbij}\\

\haiku{zeiden mijn dames,.}{opgetogen denkend mij}{daarmee te troosten}\\

\haiku{Een plannenmaker,, ':}{een illusionist een}{dromer int groot}\\

\haiku{Dat is iets, voer hij, -,.}{voort iets dat vanzelf spreekt maar}{lang nog niet alles}\\

\haiku{Er bestaat wel een!}{Trans-Siberian en}{een Trans-African}\\

\haiku{Zijn benen waren,.}{ietwat krom wat aan zijn gang}{iets waggelends gaf}\\

\haiku{Waarom bleef ik niet,?}{thuis waar ik het zo lekker}{en gezellig had}\\

\haiku{Ik wip uit mijn bed,,;}{trek de jaloezie\"en op}{gooi open de vensters}\\

\haiku{{\textquoteright} Il faut marcher, Et',:}{quand on veut fair des \'epates}{C'est peau d'z\'ebi}\\

\haiku{Toch lijken zij in.}{niets op wat wij bij ons van}{dat soort gewend zijn}\\

\haiku{Hun kleren hebben,.}{geen waarde zij zijn verslaafd}{aan drank noch tabak}\\

\haiku{Eenmaal geven is.}{meteen je rust verbeuren}{voor de hele reis}\\

\haiku{De man, die zulk een,!}{mooie klemtoon op dit laatste}{woord legt spreekt helaas}\\

\haiku{Kleiner worden de,.}{figuren op de oever}{die vaarwel juichen}\\

\haiku{Nu zit ik in de,.}{Zwitserse trein omringd door}{Zwitserse mensen}\\

\haiku{Hoededoos is het.}{meisje dat de hoed van de}{dame naar huis brengt}\\

\haiku{Het is de triomf?}{en de glorificatie}{van Hoededoos}\\

\haiku{het baantje v\'o\'or haar,.}{schoon opdat zij toch aan het}{doel zou geraken}\\

\haiku{Maar de thee zal dat.}{alles bij de anderen}{ook weer goedmaken}\\

\haiku{Jammer, dat de lucht, '.}{zo grijs is anders ware}{t schouwspel prachtig}\\

\haiku{Men hoort het aan de,.}{toon waarop zij de Franse}{woorden uitspreken}\\

\haiku{Om de strijdenden.}{van de niet-strijdenden}{te onderscheiden}\\

\haiku{Zoverre zijn we,,, '!}{helaas nog niet ook niet in}{t vrije Engeland}\\

\haiku{Of dit nu altijd {\textquoteleft}{\textquoteright}.}{evensympathiek aandoet is}{een andere zaak}\\

\haiku{Je gaat b.v. in een.}{winkel en bestelt er een}{of ander voorwerp}\\

\haiku{En ik keek rond of.}{ik nergens in de buurt een}{policeman zag}\\

\haiku{Dat treurig schouwspel.}{heeft het genoegen van mijn}{wandeling vergald}\\

\haiku{Terstond is daar een,:}{suppoost en zodra mijn vriend}{weer te voorschijn komt}\\

\haiku{Maeterlinck kocht,;}{het legde er nog een paar}{honderd duizend bij}\\

\haiku{Meestal mensen,.}{die nog maar ternauwernood}{wandelen kunnen}\\

\haiku{De ganse nacht, tot,.}{in de ochtenduren duurt de}{kruitverspilling voort}\\

\haiku{Hij nodigt ons uit;}{om zijn kerkje van binnen}{te bezichtigen}\\

\haiku{Nog smaller dan de,.}{eerste en afgronden om}{van te duizelen}\\

\haiku{k H\`e ik doar wa,.}{konijneten weest trekken}{langs de kanten}\\

\haiku{masco Joa ik, boas,.}{Van Poamel da es ienen veur}{ou os g'hem wilt}\\

\haiku{veur die troebels die.}{doar nou were zijn mee da}{wirkvolk in Gent}\\

\haiku{'t es 's morgens}{toch zuedoanig vroeg op te}{zijn en we lagen}\\

\haiku{Ge zij gulder ons.}{miesters en we moete}{wij g'huerzoamen}\\

\haiku{de baron Zie, Van,.}{Paemel ik zal kaart op}{tafel met u speel}\\

\haiku{de barones  (),,?}{heengaande Allons alzo}{verbleef  niewaar}\\

\haiku{En vergeet niet dat.}{ik Romanie maandag op}{de kasteel verwacht}\\

\haiku{de brigadier  ().}{tot de gendarme Blijf gij}{hier de wacht hou\^en}\\

\haiku{'k Weinsche dat 't ' '.}{leugens woarent gien da}{k hier zegge}\\

\haiku{moar d'r zijn nog}{ander en misschien beter}{hofstees te vinden}\\

\haiku{(Van Paemel schudt),?...}{krachtig het hoofd Wacht ne kier}{woar woaren we dan}\\

\haiku{zoo gelukki zijn}{en zien dat ik gelijk had}{van te handelen}\\

\haiku{masco 'k Kwam ik '.}{ne kier vroagen hoe datt}{goat mee Desir\'e}\\

\haiku{Allo, 't es goed,, '.}{leg ze doar moart zal ze}{wel iemand opeten}\\

\haiku{masco Wa da 'k,...}{goa doen om van te leven}{Menier de Paster}\\

\haiku{Hij wilde malgr\'e,.}{h\^en dat-e gij uek bij}{hem kwam Moeder}\\

\haiku{Met gedempte en) '?}{geheimzinnige stem Es}{t er glen belet}\\

\haiku{losgeloaten.}{worden die ulderen tijd}{uitgedoan h\^en}\\

\haiku{We 'n stoan nie,!}{ingeschreven op ulder}{boeken zeggen ze}\\

\haiku{die poait ou mee;}{wa zoete woorden uit de}{catechissemus}\\

\haiku{- Komt met nijdige ')!}{spotlach weer int midden}{van de keuken O}\\

\haiku{'k H\`e 't hem wel, ' ';}{gevroagd moar hijn hee}{t nie willen geen}\\

\haiku{Ge zilt mij hier iest,.}{al geen wa da g'h\`et tot de}{loaste cens}\\

\haiku{(Rosten Tjeef maakt een) '?}{gebaar van schrik Est gien}{woar da ge da peist}\\

\haiku{Peisde gij misschien '?}{da ze zij nien weet wat}{dat er gebeurd es}\\

\haiku{loat ons iest nog ',.}{n dreupelke pakken da}{geeft koeraze}\\

\haiku{Eerste toneel jan '}{en zulma Bij het opgaan}{vant gordijn zit}\\

\haiku{Alle menuten'.}{kan d ien of d'andere}{binnen komen}\\

\haiku{Boas, 't en es doar, '.}{nie da we sloapent}{es hier in de koamer}\\

\haiku{'k Wachtte tot dat, '.}{hij boven was en tons gijnk}{k uek noar boven}\\

\haiku{Hij 'n zal mij nie, '.}{antwoorden hijn zal noar}{mij nie kijken}\\

\haiku{of dat hij druemt... ' '!...}{t Wasn gerucht om d'r}{schouw van te worden}\\

\haiku{'n stikske van den.}{buikschotel mee eirdappels}{en saveu\"en}\\

\haiku{de buurvrouw Goa gij,... '.}{moar Zulmatjek komme}{seffens achter}\\

\haiku{Eindelijk neemt hij.}{een bord van tafel en gaat}{ermee om eten}\\

\haiku{cloet  (met holle,)?}{doffe stem H\`et-e nie}{wat te drijnken}\\

\haiku{(Beiden achtergrond)(,;}{af  de notaris}{links op rookt een pijp}\\

\haiku{stien  (als boven) ',.}{n Beetse te veel natte}{Menier de Juge}\\

\haiku{Uleken droogt met een.}{handdoek glazen af achter}{de schenktafel}\\

\haiku{(Uleken lacht hardop)?}{de notaris Woar es}{de juge dan}\\

\haiku{Dat 'n kan hem nie, '.}{schelen os de stamenee}{moar goedn marcheert}\\

\haiku{Ge moet weten, dat.}{die jonge juffer mij een}{weinig intrigeert}\\

\haiku{de kanterik Joa,,,.}{joa joa Menier den Docteur}{hij was zeker stout}\\

\haiku{de wrijver Menier, ' ';}{den Docteurt was azuen}{koartje noar den tienen}\\

\haiku{We zaten wij mee ',:}{twie toafels aant koarten in}{mijn hirbirge}\\

\haiku{vrouw roetjes  (met)!}{ten hemel opgeheven}{handen Ha jongens}\\

\haiku{vrouw roetjes  (naar)!?}{Grondnagel wijzend Wulder}{mee hem overienkomen}\\

\haiku{packal  (woedend)! '!}{Ocht es mee al die pruts}{van da Westvloams}\\

\haiku{(Muijshondt verschuift zich,.}{derwijze dat hij maar half}{op de stoel neerzit}\\

\haiku{Ik vraag niets beter.}{dan alles ten beste te}{zien eindigen}\\

\haiku{We zillen d'r nog,.}{ne kier goed over peizen en}{dan besluiten}\\

\haiku{'k Ben koperoal ',}{vant ieste rezement}{Doar ben ik kontent}\\

\haiku{stoute threse  (,)!}{woedend de vuisten op haar}{heupen Verdome}\\

\haiku{vrouw beert  (snelt naar,)!}{Maria gevolgd door Lisatje}{Verdomde slonse}\\

\haiku{Enkele mannen.}{en vrouwen vluchten rechts en}{links in de huizen}\\

\haiku{(Hij speelt en weldra)'}{zingen allen mee  k}{Ben koperoal van}\\

\haiku{lisatje  (kijkt naar) '.}{het raam Datn zoe mij nie}{verwonderen}\\

\haiku{(kijkt met gespannen '),?...}{aandacht doort raam Moar wie}{es dat doar dien hiere}\\

\haiku{Aftrekkend geraas)(;}{in de straat  slimke snoeck}{opgewonden}\\

\haiku{slimke snoeck  (komt),!}{dreigend op Reus af Veur ou}{niet nondedzju}\\

\haiku{As w'hem nie 'n h\^an ' ' '.}{tn zoe mijt leven}{nie mier weird zijn}\\

\haiku{(Gelach)  lisatje() ',.}{met het glas bij Beert As}{t ou blieft voader}\\

\haiku{(Gelach)  lisatje(),, '.}{schenkend Joa joak ken}{ou toeren wel}\\

\haiku{nen ambachtsman En '!}{niemandn weet er wa dat}{ik kan Moar habil}\\

\haiku{(barst plotseling in),!}{woedetranen uit en da}{we sakerdzju}\\

\haiku{(tot Witte Manse),, '.}{Ala toe Manske  geef gij}{onsn pijntje}\\

\haiku{d'r zaten gister.}{achternoene86 wel twintig}{honden achter}\\

\haiku{donder de beul, klod()!}{de vos en smuik vertriest}{tegelijk O}\\

\haiku{da ze zij ons nie ',.}{amoaln veracht hier in}{de Zijstroate}\\

\haiku{We zillen d'r van.}{donderdag oavend af al}{wa goan vangen}\\

\haiku{mijn huefd gerust, ' '.}{ofk verzeker ou dat}{t schief zal zitten}\\

\haiku{Hij sloat heur dued!... ()}{Allen vliegen naar de deur}{Vierde bedrijf}\\

\haiku{Aan beide kanten,.}{van de deur een gendarme}{die er de wacht houdt}\\

\haiku{reus balduk Keunt ge ' ',?}{t ons uek nien beetse}{zouten brugadier}\\

\haiku{reus balduk  (tot),,.}{Witte Manse Kom Manske}{kom gij bij mij}\\

\haiku{(Beert en zijn vrouw met)}{nog enkele anderen}{hollend rechts af}\\

\haiku{as hij Maria nog, '}{pebliek mishandelde moar}{in huis est toch}\\

\haiku{'k H\`e ou altijd, '}{zue geirn gezien ent zoe}{mij toch zue spijten}\\

\haiku{We zoen doar zue schuen,.}{en zue gelukkig geweund}{h\^en in Frankrijk}\\

\haiku{slimke snoeck  (tot)'?}{Reus Wa kan ou da schelen}{woar da z hangt}\\

\haiku{reus balduk  (komt)!}{met gebalde vuisten op}{Slimke Snoeck af O}\\

\haiku{Moeder, midden in,.}{de kamer naait iets vast aan}{Vaders binnenzak}\\

\haiku{hee...  guust  (met)...}{een blik op Philomene}{En mee spoaren}\\

\haiku{Buufstikken zillen,.}{we ginter eten zuevele}{of da we willen}\\

\haiku{We komen ulder,.}{nog ne kier bezoeken ier}{da ge wiggoat}\\

\haiku{(tot Rozeken) En,,...}{gij jongedochter goat er}{noartoe om te}\\

\haiku{pastoor  (doet haar),.}{weer zitten Blijf moar zitten}{gruetmoeder}\\

\haiku{Moeder neemt het hem.}{ruw uit de armen en geeft}{het aan Feelken}\\

\haiku{De burgemeester,.}{geeft hem vuur en hij steekt aan}{genoeglijk smakkend}\\

\haiku{Waar...,  moeder Och, '.}{Hiere schiedt er toch uitt es}{veel te triestig}\\

\haiku{moeder  (droevig),,:}{Ha moar ieffreiwe we zijn}{wij katholieken}\\

\haiku{ik zilde en da'.}{oarm schoapken die noar d}{helle zoe goan}\\

\haiku{n Kieken woar dat '!}{er nog gien halve vlere}{van gesneenn was}\\

\haiku{binst da we wulder.}{hier van den honger liggen}{te craveren}\\

\haiku{moeder  (huilend),,?}{Och Hiere voader wa goan ze}{toch mee hem doen}\\

\haiku{Gruetmoeder es.}{schouw van ou en ik voele}{mij toch zue ziek}\\

\haiku{en zue lank as da'.}{g hier zijt zal ik schuene}{veur ou zurgen}\\

\haiku{Meent gij misschien dat '?}{ikn wet voor u apart kan}{laten maken}\\

\haiku{Frankrijk staat hoog in,,;}{mijn waardering h\'e\'el hoog daar}{kom ik rond voor uit}\\

\haiku{(Af)  else  (),,...}{rent naar de deur O maar dat}{moet ik toch ook}\\

\haiku{Dat wij elkaar in!}{zulke omstandigheden}{moeten terugzien}\\

\haiku{Ons werk nu, mijn werk,.}{hier zal voortaan zijn dat weer}{goed te maken}\\

\haiku{jan bron Of mijn vrouw?}{en mijn dochter hier wel in}{veiligheid zijn}\\

\haiku{Denkt aan hen die op.}{de slagvelden voor onze}{vrijheid sterven}\\

\haiku{jan bron  (met een);}{vuistslag op de tafel Maar}{ik wil niet slapen}\\

\haiku{de mensen zouden.}{die kerels verscheurd hebben}{hadden ze gedurfd}\\

\haiku{eerste hollandse( ').}{soldaat  dadelijk bij}{t hek Paspoort}\\

\haiku{Eerste Belgische.}{soldaat patrouilleert lustig}{rokend heen en weer}\\

\haiku{eerste belgische()...!}{soldaat  verwonderd Tiens}{en ge spreekt Olands}\\

\haiku{Geheimzinnig) Ik.}{denk dat het een verbannen}{activist is}\\

\haiku{eerste belgische ',...}{soldaat Dat esn ander}{geval meniere}\\

\haiku{van veerdeghem()?}{tot Tweede Belgische}{soldaat Ook een}\\

\haiku{eerste hollandse() '!}{soldaat  verontwaardigd}{Een gek ist}\\

\haiku{In de achtergrond.}{een bordes met toegang tot}{een zonnige tuin}\\

\haiku{ik... Op dit ogenblik ':}{hoort men achtert toneel}{de stem van Dora}\\

\haiku{Dora zal er niet...}{tegen kunnen en ze zal}{niet toegeven}\\

\haiku{Veertiende toneel,()...}{castro dora  dora}{hijgend Vader}\\

\haiku{dat mag ik wel, na...!!}{alles wat u mij war u}{ons hebt aangedaan}\\

\haiku{om je de waarheid... '...}{te zeggen ik het liever}{n potje bier}\\

\haiku{nou wordt ie spinnig,...}{nou staat ie te trippele}{van kwajigheid}\\

\haiku{(terugkerend) maar,...}{je belooft me Raaks dat je}{niet aan Castro zegt}\\

\haiku{raaks O\'ok 'n vraag... as...}{iemand d'r uitziet als de}{dood van ieperen}\\

\haiku{Je moet 's nachts es ', '!}{n grokkie nemen dan slaap}{je alsn marmot}\\

\haiku{daar is Teunisse......!}{met de tuinstoelen en of}{uwe effies komt}\\

\haiku{w\'at heb je met die......}{brief voor meneer Daal gedaan}{die ik je laatst gaf}\\

\haiku{mevrouw  (die al,)}{die tijd als een standbeeld is}{blijven staan langzaam}\\

\haiku{D'r is maar \'e\'en man...,}{die hier helpen kan en dat}{is dokter Coertens}\\

\haiku{ik ken 't niet meer...()!!}{met gebalde vuisten}{Allemachtig}\\

\haiku{Castro is voor zijn)}{schildersezel gaan staan om niet}{gezien te worden}\\

\haiku{en toch ben ik 't... ', '...}{Zie ik eruit alsn dief}{en toch ben ikt}\\

\haiku{U wordt getroffen '......}{int liefste wat u hebt}{in uw kinderen}\\

\haiku{Misschien heb ik d\'a\'arom!}{wel zo veel gewerkt in m'n}{leven om die angst}\\

\haiku{zoveel dingen... die '... '}{men verschrikkelijke}{angst hebben bezorgd}\\

\haiku{Vijftiende toneel,()?!}{mevrouw castro  mevrouw}{jubelend Ja}\\

\haiku{Dan wendt zij het hoofd - -:}{weer terug pauze weer naar}{de tafel ziende}\\

\haiku{Sientje  (Wenkt haar).}{bij haar te komen Sientje}{komt naderbij}\\

\haiku{Juffrouw Elvire......,...}{slaapt toch niet meer hier wat}{mevrouw Nee dokter}\\

\haiku{elvire Kom an....}{Doortje wat doen die bloemen}{je nu voor kwaad}\\

\haiku{Jacques, in al de,......}{tijd dat je er niet was ben}{je bij me geweest}\\

\haiku{Dat moet het toch wel............?}{zijn Wat wat dacht jij dat er}{tussen ons stond}\\

\haiku{dan... nog es op 't,............?}{orgel zoals vroeger wil}{je dat voor me doen}\\

\haiku{blijft staan, kijkt als om.}{hulp werktuiglijk om en ziet}{Coertens in de deur}\\

\haiku{coertens Kom tot je...(,)}{zelf  castro  hem van}{zich afwerpend wild}\\

\haiku{castro  (laat haar)!}{ineens los en op Coertens}{afgaande Zwijg jij}\\

\haiku{op 'n stoel vallend)...}{Twintig jaar lang heb ik met}{dat kwaad gevochten}\\

\haiku{Existe-t-il dans?}{le monde vingt personnes}{qui la connaissent}\\

\haiku{Maeterlinck steekt.}{de vinger in de hoogte}{en fluit een deuntje}\\

\haiku{een drama waarin!}{werkelijkheid en ideaal}{\'e\'en zullen worden}\\

\haiku{Gedurende twee.}{volle jaren leed hij er}{honger en gebrek}\\

\haiku{{\textquoteright}, zegt de arme vrouw, {\textquoteleft},,{\textquoteright}.}{toch niet bewaar hem goed en}{wacht maar je zult zien}\\

\haiku{en meteen stelt hij,.}{hem voor aan Lacroix die het}{werk zal uitgeven}\\

\haiku{Gedurende acht.}{volle maanden werkt hij aan}{zijn reusachtig plan}\\

\haiku{maar  telkens toch '.}{dooft de grijze massat}{heerlijk licht weer uit}\\

\haiku{Die {\textquoteleft}charme{\textquoteright} werkte,:}{z\'o algemeen en z\'o sterk}{dat om het even wie}\\

\haiku{We'n keunen wij toch,.}{moar huel slecht roapen}{missen denkt het mij}\\

\haiku{De wandeling in ' '?}{t stadspark oft bezoek}{bij Westinghouse}\\

\haiku{mond en kin - naar rechts,,}{of links trekken terwijl hij}{sprak al naar gelang}\\

\haiku{Naast de stuurman zit,.}{een kerel gewapend met}{een enorme spreekbuis}\\

\haiku{Hun beleefdheid, hun.}{generositeit waren}{er geen des harten}\\

\haiku{Veel vreemdelingen,, '}{zei ik dies zomers bij}{ons buiten kwamen}\\

\haiku{De gevoelens, die,.}{zij ondergaat staan op haar}{gezicht te lezen}\\

\haiku{{\textquoteleft}Populierenhout{\textquoteright}:}{is goed lepelhout en wacht}{dan even en vervolgt}\\

\haiku{Het meisje was het,;}{kleinste het liefste en het}{blondste van de drie}\\

\haiku{Eens was hij, hoewel,.}{zwaar verkouden per rijwiel}{naar Gent gereden}\\

\haiku{Ik moet alles goed,.}{herdenken want het is zo}{plotseling gegaan}\\

\haiku{De slippen van zijn {\textquoteleft}{\textquoteright}.}{caban woeien lichtkens rechts}{en links van hem af}\\

\haiku{gisterenavond op, {\textquoteleft}{\textquoteright};}{zijn rijwiel met zijncaban}{en zijn rond hoedje}\\

\subsection{Uit: Verzameld werk. Deel 7}

\haiku{Maar de huldiging,,.}{in Antwerpen was hoe dan}{ook toch doorgegaan}\\

\haiku{Over de volgende:}{twee dramatische schetsen}{kunnen we kort zijn}\\

\haiku{De verhollandsing}{van het oorspronkelijk in}{dialect geschreven}\\

\haiku{De Nederlandsche.}{letterkunde in Belgi\"e}{sedert 1830 door Edw}\\

\haiku{Jean Tousseul, in Groot,,,,-;}{Nederland jg. XVIII 1920}{d. I p. 228235}\\

\haiku{Want laten we 't:}{in godsnaam niet vergeten}{of niet loochenen}\\

\haiku{Deze rubriek is '.}{en blijft dan ook het lijfstuk}{vant courantje}\\

\haiku{Voor eigenlijke.}{liefde van haar man is zij}{minder verlegen}\\

\haiku{Wat lijkt de Vlaamse,!}{daarentegen doorgaans flink}{vrolijk en gezond}\\

\haiku{'t Is om er bij, ':}{te huilen ent is om}{er voor te bidden}\\

\haiku{Het ene volk amuseert.}{zich veel meer en veel beter}{dan het andere}\\

\haiku{Wij, kinderen van,.}{de tegenwoordige tijd}{hebben niets misdaan}\\

\haiku{En ook de grote, - -:}{overledene Gezelle}{werd niet vergeten}\\

\haiku{'s Nachts trekken zij.}{erop uit en roven wat}{ze krijgen kunnen}\\

\haiku{{\textquoteright} De lezer zelf schudt:}{droevig het hoofd en murmelt}{met benepen hart}\\

\haiku{Beide gezinnen,;}{zijn onder elkaar bevriend}{meer zelfs dan bevriend}\\

\haiku{hij zat al nevens '!}{Nardje op den bok ent}{gespan was in gang}\\

\haiku{Zegene u de,.}{Alderhoogste want de navond}{is nabij komt bij}\\

\haiku{Zal Zola zich bij '?}{t vonnis neerleggen en}{de zaak opgeven}\\

\haiku{De wind, die zoo hoog.}{voorbijtrok had misschien over}{de heide gewaaid}\\

\haiku{Hij wil het mooie, het, '.}{rijke het schitterende}{vant uiterlijk}\\

\haiku{Hare voorpooten}{trokken soms aan een draadje}{en haar koppeken}\\

\haiku{De taal van Herman.}{Teirlinck lijkt ook heel sterk}{op die van Streuvels}\\

\haiku{Hij sprak eerst in het '.}{Vlaams en vertaalde daarna}{zijn toast int Frans}\\

\haiku{Ongelooflijk sterk;}{zijn haar sympathie\"en en}{antipathie\"en}\\

\haiku{maar het kan ook best,.}{gebeuren dat het heel iets}{anders betekent}\\

\haiku{Ik heb het hier reeds,:}{meer gezegd en ik moet het}{nog eens herhalen}\\

\haiku{Minnehandel door,,.}{Stijn Streuvels ~ 2 dln L.J.}{Veen Amsterdam}\\

\haiku{Hij duwt het weg maar ',.}{t komt terug hoe langer}{hoe hardnekkiger}\\

\haiku{- We kunnen er niets,,.}{aan doen Klara die dingen}{hangen in de lucht}\\

\haiku{te Rotterdam, en ().}{Vlaamsche BoekhandelLeo J.}{Krijn te Brussel}\\

\haiku{C'est son b\^aton qui}{repart le premier et fait}{le premier pas. Il}\\

\haiku{Het heet dat ze doof,.}{is maar alleen voor wat ze}{niet graag horen wil}\\

\haiku{Ook niet in Frankrijk.}{en in de grote steden}{van het buitenland}\\

\haiku{Het eerste zag het;}{licht met een voorrede van}{Emile Verhaeren}\\

\haiku{Ook wij weten niet.}{waar en hoe wij ons laatste}{lied zullen zingen}\\

\haiku{De vruchten van de.}{Natuur kunnen tijdelijk}{vernietigd worden}\\

\haiku{Hij is ook z\'o groot,.}{en hij staat z\'o hoog dat men}{hem maar moet lezen}\\

\haiku{Kort v\'o\'or het einde.}{ging Maeterlinck hem daar}{nog even opzoeken}\\

\haiku{Hij drukte 't niet,;}{verder in woorden uit wat}{je daar al niet kon}\\

\haiku{Wij kijken nog naar ',,;}{t Oosten wel zwaar bedroefd}{maar niet wanhopend}\\

\haiku{Jacques Muraille,,,.}{steenbikker is getrouwd met}{Marie een werkvrouw}\\

\haiku{Zijn leven is van.}{een doodse verlatenheid}{en melancholie}\\

\haiku{Zelden heb ik een.}{werk van groter en dieper}{emotie gelezen}\\

\haiku{Il restait un peu:.}{de soupe \`a l'h\^otesse}{je le lui offris}\\

\haiku{Dit geeft haar boek iets,.}{zeer bekorends van warme}{schone zuiverheid}\\

\haiku{daarvoor was ook de.}{omvang van het werk niet breed}{genoeg opgezet}\\

\haiku{Hij riep ons allen,;}{om zich heen wij moesten kijken}{en bewonderen}\\

\haiku{Het was een feest van,.}{elke dag een jubelen}{van ieder ogenblik}\\

\haiku{voor mij l\'e\'eft hij, als,!}{Vlaanderen zelf en zal hij}{blijven leven}\\

\haiku{Doch sprekender dan.}{alle beweringen zijn}{enkele cijfers}\\

\haiku{Hij vliegt u ijzig;}{in het aangezicht en poogt}{u te verblinden}\\

\haiku{Het volk wantrouwt hen.}{en men kan het volk niet gans}{ongelijk geven}\\

\haiku{Het heeft bij hem veel;}{meer weg van een houding dan}{van een overtuiging}\\

\haiku{Laat ik nu maar eens.}{heel oprecht en desnoods heel}{onbescheiden zijn}\\

\haiku{De drie anderen;}{kon ik slechts op de rug of}{van terzijde zien}\\

\haiku{De lijkstoet en de.}{rijtuigen staan onder de}{hoge bomen stil}\\

\haiku{en na het wonder:}{leek nu alles zo gewoon}{en onbelangrijk}\\

\haiku{en daar was ik aan.}{mijn tweede weddenschap van}{vijfduizend frank}\\

\haiku{Per auto komt men,.}{in plaatsen en streken waar}{men anders niet komt}\\

\haiku{Laten wij samen,.}{hopen dat die dag spoedig}{zal aanbreken}\\

\haiku{Een groot rumoer kwam.}{aangedreund van verre in}{de donkere nacht}\\

\haiku{Slechte berichten,.}{geloven zij niet willen}{ze niet geloven}\\

\haiku{ongelukkig door;}{wat zij onherroepelijk}{verloren hebben}\\

\haiku{Ik hoef er niets aan,:}{toe te voegen ik mag er}{veel uit weglaten}\\

\haiku{Het merk {\textquoteleft}Volkenbond{\textquoteright}...}{Gisteren is de man naar}{mij toegekomen}\\

\haiku{Ik heb het van 't.}{begin tot het einde met}{aandacht gelezen}\\

\haiku{Daar schoof ik een groen:}{gordijn weg en hij zag de}{twee ruime vakken}\\

\haiku{De zogenaamde '.}{vrede kwam en ik keerde}{int land terug}\\

\haiku{een grote bom viel.}{neer en de brug vloog in een}{rookwolk uit elkaar}\\

\haiku{En wat zij zijn op, '.}{h\'un dorp zijn huns gelijken}{overt hele land}\\

\haiku{en Duitsland met een.}{deel van Vlaanderen aan de}{andere zijde}\\

\haiku{Aan het lijden van,.}{de dieren dachten zij geen}{ogenblik die mensen}\\

\haiku{blijven doorvechten?}{of vrij en vreedzaam weer naar}{huis toe mogen gaan}\\

\haiku{Op wie wacht ze, en, {\textquoteleft}{\textquoteright} {\textquoteleft}{\textquoteright}?}{wie moet daar vandaan komen}{eenhij of eenzij}\\

\haiku{- De oorlog verslindt.}{elke dag duizenden en}{duizenden mannen}\\

\haiku{De Belg zijn spotlach,,.}{zijn stille schimp ontnemen}{is onmogelijk}\\

\haiku{en aan de verre,.}{einder nog een boerderij}{en dat is alles}\\

\haiku{D\'at was het wat hij!}{van zover en met zoveel}{zorg had meegebracht}\\

\haiku{Je zit er lang niet.}{onaardig en je eet er}{bepaald heel lekker}\\

\haiku{Tot haar verbazing.}{stond hij zelf reeds kant en klaar}{om te vertrekken}\\

\haiku{- Hebben ze onze!}{soep reeds uitgeschept v\'o\'or wij}{aan tafel zaten}\\

\haiku{- Wat ik ervan denk...... -...?}{herhaalde hij heel langzaam}{wat ik ervan denk}\\

\haiku{'t Diner was fijn,,.}{die avond en de wijnkelder}{werd aangesproken}\\

\haiku{Thans behoort men in.}{beschaafd gezelschap over grint}{en zand te praten}\\

\haiku{er was geen plaats voor,!}{hen omdat er niet genoeg}{geld voor hen was}\\

\haiku{Het roodborstje zingt;}{een melancholisch-kort}{vooisje en vliegt weg}\\

\haiku{Ik lachte ook, en,.}{grijnsde en wachtte wat nu}{verder komen zou}\\

\haiku{Zij voelen niets,  , {\textquoteleft}.}{naar zij beweren voor het}{begripvaderland}\\

\haiku{Zij zullen er niet,;}{binnenkomen nog in geen}{jaren en jaren}\\

\haiku{Deze keert zich om:}{en stelt de vraag aan een van}{zijn kameraden}\\

\haiku{Mieux on conna{\^\i}t la,{\textquoteright};}{vie plus on aime son chien}{zegt een Frans spreekwoord}\\

\haiku{Wat Deulin schreef is,.}{trouwens zo weinig zo veel}{te weinig bekend}\\

\haiku{Fran\c{c}oise was een,.}{vrome vrouw in de vrees des}{Heren opgevoed}\\

\haiku{Gillette, onder,.}{Belzebuths invloed gehuwd}{was niet gelukkig}\\

\haiku{{\textquoteleft}Ik ben niet rijk, maar.}{wil toch ook iets doen voor die}{ongelukkigen}\\

\haiku{Eenieder snakt naar.}{vrede en nagenoeg de}{ganse wereld vecht}\\

\haiku{u, vrouwen zoudt er.}{glimlachend mee trippelen}{en koketteren}\\

\haiku{Een dezer dagen.}{ontving ik het bezoek van}{een aardig jongmens}\\

\haiku{Het staat zo ver van.}{ons af en toch voelen wij}{het zo heel dichtbij}\\

\haiku{wat een geluk, dat!}{we die akelige dingen}{nu nog hebben}\\

\haiku{Er werd waarachtig;}{nog een derde maal tegen}{die open deur geklopt}\\

\haiku{- Maar, meneer, wat ik,!}{schrijf is toch de waarheid de}{zuivere waarheid}\\

\haiku{- Dat komt van buiten,, -,;}{zei zij van de schaduw uit}{die mooie hoge boom}\\

\haiku{De {\textquoteleft}zangeres{\textquoteright} ging ';}{echter eenigszins bedremmeld}{naart tooneel terug}\\

\haiku{Die zal er ook wel,!}{voor te vinden zijn of de}{tooneel-criticus}\\

\haiku{en in welk zalig?}{land van weelde en vrede}{is het dan gebeurd}\\

\haiku{Maar dat lijkt alles,,.}{nu zo ver zo heel h\'e\'el ver}{en lang geleden}\\

\haiku{- Voor wat moeten wij?}{onze kinderen na de}{oorlog opleiden}\\

\haiku{daar zijn de knoeiers;}{die heimelijk handel met}{de vijand drijven}\\

\haiku{veel en lekker eten,,;}{veel en lekker drinken veel}{en lekker roken}\\

\haiku{- Een zilverbon van,.}{fl. 2.50 herhaalde ik met}{vaste overtuiging}\\

\haiku{de stad gonst in de,.}{verte alsof ze zich van}{mij verwijderd had}\\

\haiku{- ik was juist van plan.}{te trouwen en mij ergens}{vast te vestigen}\\

\haiku{geef ze mij mee in,.}{een pakje in mijn doodkist}{zei Guerliche}\\

\haiku{Zij vinden er, als,.}{ik het zo mag uitdrukken}{geen waar voor hun geld}\\

\haiku{kerstdag komen ze,.}{vast en zeker bij elkaar}{vermaande hij nog}\\

\haiku{en ik begreep niet.}{goed waar hij zijn borstels wel}{vandaan kon halen}\\

\haiku{Er staan al heel wat.}{namen op en het aantal}{neemt dagelijks toe}\\

\haiku{- Trek er spoedig een,,.}{van aan amice mitsgaders}{andere laarzen}\\

\haiku{later, l\'ater, als,.}{ik zijn werk zijn plicht geheel}{volbracht zal hebben}\\

\haiku{zij konden er niets '.}{mee uitvoeren en hebben}{t hem gelaten}\\

\haiku{Goddank dat ze zijn!}{afgebeulde paardje hem}{gelaten hebben}\\

\haiku{Hij aarzelt even, drijft.}{zijn paard met ploeg de richting}{van de steenweg uit}\\

\haiku{- Gereed zijn, zulle,.}{of anders wordt ge met de}{gendarmen gehaald}\\

\haiku{Wij reden samen,,.}{de vreemdeling en ik per}{fiets door Gelderland}\\

\haiku{- Hoe komt het toch, vroeg, -?}{hij dat men in Holland geen}{omelet kan maken}\\

\haiku{Zeg haar dat zij er;}{precies dezelfde schotels}{mee moet bereiden}\\

\haiku{Het was Kerstavond en.}{een opgewekte stemming}{leefde in het dorp}\\

\haiku{Het was of er een.}{lamheid in mijn geest en in}{mijn hand en ogen kwam}\\

\haiku{- Pardon, meneer, maar.}{ik kan niet werken als er}{iemand naast mij staat}\\

\haiku{Ik meende dat dit,,}{weerzien ondanks alles wat}{gebeurd was voor mij}\\

\haiku{Ik vroeg met enige.}{verwondering of dat de}{nieuwe pasmunt was}\\

\haiku{De achterblijvers,;}{mopperen wat het geeft een}{beetje jaloezie}\\

\haiku{Zij leven nu al,,.}{zolang al zoveel jaren}{z\'onder te werken}\\

\haiku{En meteen gaf hij.}{bevel mijn molenaar de}{boeien om te slaan}\\

\haiku{zei mijn molenaar,.}{toen ze zowat twee uur lang}{gelopen hadden}\\

\haiku{dat zo'n stumperd de}{eindeloze tocht van ruim}{vijf uren die wij v\'o\'or}\\

\haiku{Ik spring van mijn plaats,,.}{op grijp naar mijn revolver}{ruk de voordeur open}\\

\haiku{Niet alleen hebben;}{zij al het vee en al de}{paarden meegesleept}\\

\haiku{Het is een oude,.}{boer met gebogen schouders}{en grijze haren}\\

\haiku{Ik vlucht weg, op mijn,,.}{rijwiel geel van slijk in de}{lichte manenacht}\\

\haiku{Hij maakt een gebaar,;}{om te beduiden dat het}{afgelopen is}\\

\haiku{Zij kunnen maar niet,,}{begrijpen dat hij nog leeft}{en soms zijn ze bang}\\

\haiku{Zij houdt zich heel, h\'e\'el,;}{stil om zijn weldadige}{rust niet te storen}\\

\haiku{Geluidloos komen,;}{zij binnen vragen eventjes}{of alles goed is}\\

\haiku{en daarop vertrekt;}{het jonge meisje zoals}{zij gekomen is}\\

\haiku{Zij werkten aan de,.}{wegen onder toezicht van}{Franse soldaten}\\

\haiku{Ik hoor dat ze zich -!}{ginds vervelen anders wel}{een heel goed teken}\\

\haiku{uitstekend!) en dat.}{ze er allemaal meer dan}{genoeg van hebben}\\

\haiku{Terstond, gehoorzaam,.}{van aard als ik ben rijs ik}{voorzichtig overeind}\\

\haiku{Waar zitten ze nu,,,?}{die helden die bevende}{blatende schapen}\\

\haiku{Zij keken over 't,;}{muurtje heen strak en stijf als}{waren zij van ijs}\\

\haiku{Ik haal er slechts een,.}{drietal aan die mij tot een}{obsessie worden}\\

\haiku{- Nog niet zo zeer door,;}{de oorlog sprak eindelijk}{de burgemeester}\\

\haiku{- Hier zie, meniere,,!}{vlak achter mijn hof tussen}{die twee wilgen doar}\\

\haiku{Dat alles had ik!}{reeds opgegeten en moest}{ik nog eens opeten}\\

\haiku{zoals de oude;}{Aristophanes het eeuwen}{geleden hoorde}\\

\haiku{zei een werkman, die,,.}{glimlachend in hemdsmouwen}{in zijn deurgat stond}\\

\haiku{Hij zou verbaasd zijn,.}{als hij zien kon hoe ze zich}{ontwikkeld hebben}\\

\haiku{La marraine, son,;}{poupon sur le bras pleure}{\`a chaudes larmes}\\

\haiku{Il n'y avait pas de,,:}{malheur pas de sang pas de}{mort dans sa maison}\\

\haiku{Quant au fond m\^eme,.}{de mon article je n'ai}{riep \`a y changer}\\

\haiku{l'a{\^\i}n\'ee, Rosalie,,}{si merveilleusement dou\'ee}{morte au d\'ebut}\\

\haiku{extr\^eme dans ses.}{affections comme dans}{ses antipathies}\\

\haiku{Et partout dans les:}{beaux p\^aturages se meuvent}{les riches troupeaux}\\

\haiku{Je ne savais et,,.}{quelquefois je souffrais}{de ne pas savoir}\\

\haiku{Elle est blanche.}{avec une plinthe noire et}{un toit de chaume}\\

\haiku{C'est vraiment par amour.}{pour la France qu'on y va}{et qu'on y donne}\\

\haiku{{\textquoteleft}D\`es que j'arrive,.}{en Flandre j'enrichis mon}{vocabulaire}\\

\haiku{Il \'eprouvait de la.}{peine \`a s'assimiler}{les vrais sons flamands}\\

\haiku{Je lui montrais mon.}{pays et lui m'apprenait \`a}{conna{\^\i}tre le sien}\\

\haiku{Overigens, hij dacht,.}{er ook niet aan daarover zijn}{beklag te maken}\\

\haiku{Hij maakte van een,:}{seconde rust gebruik om}{schreiend te roepen}\\

\haiku{Ik alleen had aan.}{al die uitgelatenheid}{geen deel genomen}\\

\haiku{- Een beetje vlugger,,.}{dan koetsier wij brengen den}{stoet in verwarring}\\

\haiku{Hij glimlachte even,,.}{en zijn glimlach liet zeer mooie}{witte tanden zien}\\

\haiku{- Als ik terugkom.}{zal de kudde reeds klaar staan}{om te vertrekken}\\

\haiku{Zwijgend, bedwelmd door,.}{rampgevoelens heb ik haar}{we\^er thuis gebracht}\\

\haiku{Zij was te goed, te.}{zacht voor deze wereld van}{smart en ellende}\\

\haiku{Ik begreep dat ik,.}{niet k\'on uitscheiden omdat}{ik nog niet dood was}\\

\haiku{De Schilder liep maar;}{aldoor babbelend met het}{brunetje vooruit}\\

\haiku{Hij sliep dien laatsten,.}{nacht in gelukzalige}{vergetelheid}\\

\haiku{De menigte, die,.}{haar prooi zag ontsnappen drong}{brieschend om ons heen}\\

\haiku{En ik liep verder,.}{en was al spoedig weer de}{brem vergeten}\\

\haiku{In een hoek lag een,.}{grauw linnen zakje flink met}{een touw dichtgesnoerd}\\

\haiku{Hij was een beetje,.}{schrokkig maar verder had}{hij geen gebreken}\\

\haiku{De meiden sloegen;}{de armen ten hemel en}{huilden wanhopig}\\

\haiku{Zij vertrokken op.}{een vroegen ochtend en von}{Varken was er bij}\\

\haiku{Ik kan niet bouwen.}{op den wankelen bodem}{eener vergissing}\\

\haiku{Ge zie wel, e-woar, ' '!}{datt nie neudign was}{van azeu te vloeken}\\

\haiku{halsstarrig v\'o\'or de.}{auto uit en is daar niet}{vandaan te krijgen}\\

\haiku{Haar zoon, - het kalf - is,.}{dat niet minder maar op een}{andere wijze}\\

\haiku{Het kind lag in 't ',.}{zand naastt rechter voorwiel}{onbewegelijk}\\

\haiku{Eigenlijk waren {\textquoteleft}{\textquoteright}.}{we allebeien d\'efaut}{ik zoowel als Andr\'e}\\

\haiku{Wij reden dus, op,.}{een prachtigen herfstdag van}{Yonkers naar New York}\\

\haiku{En nu, kindlief, heb.}{ik honger en wou wel heel}{graag iets gebruiken}\\

\haiku{- En krijg ik nu ook?}{zoo'n wagentje met vier of}{zes kaboutertjes}\\

\haiku{Een droom Helder en...}{duidelijk heb ik mijn droom}{gezien en doorleefd}\\

\haiku{Concreet en compleet:}{had hij mijn toekomstige}{psyche voorgevoeld}\\

\haiku{Meheus O neen neen ', '...}{t Meneere de Notaris}{t en zal nie zijn}\\

\haiku{Monsieur L\'eonce,.}{ces deux mots signifient la}{m\^eme chose}\\

\haiku{Dias Vinden ze daar,?}{geen koolmijnen in da land}{Meneer L\'eonce}\\

\haiku{Mr L\'eonce Mulle.}{De Terschueren ~ Drie}{uren zijn geslagen}\\

\haiku{Mulle Meneers de,.}{raadsheer nui ik verklaar ui}{de s\'eance open}\\

\haiku{Ik zou b.v. dijnk dat,......}{het niet waar contrarie van}{aan Vervijn daar ui}\\

\haiku{nous devons faire,.}{attention \`a Dias}{savez-vous}\\

\haiku{Bosschaert Ja maar alij,,...}{ten es maar est da Vreeze}{sekretaris}\\

\haiku{Sinds gij ze niet meer,.}{kunt gade slaan houden zij}{niet op met drinken}\\

\haiku{Schielijk glijdt de schel.}{hem uit de hand en valt op}{de zoldering}\\

\haiku{Lucy, Berthe en()!}{Laurence  geestdriftig}{toespringend Zero}\\

\haiku{Beiden aanschouwen),...}{elkander glimlachend Miss}{Jane mijn vriend}\\

\haiku{(Fernand staat recht) (tot),?}{hare dochters Maar niet te}{geweldig niet waar}\\

\haiku{Mevrouw Vandame(),.}{de hand openend Aan u}{Mijnheer Robert}\\

\haiku{Waer ic mij wend, waer.}{ic mij keer Ghij sijt alleen}{in mijn ghedachte}\\

\haiku{Mijnheer Lansing Neen,,.}{Mijnheer Fernand wij hebben}{hem niet gezien}\\

\haiku{ik ken ze, ik heb}{ze overwogen en doorgrond}{en reken ze als}\\

\haiku{'K heb reeds te veel.}{gezeid en zal er geen woord}{meer bijvoegen}\\

\haiku{Georges, Mijnheer.}{Lansing en Fernand ijlen}{naar de ingangdeur}\\

\haiku{Allen staan recht en,,.}{komen behalve Fernand}{op den voorgrond}\\

\haiku{Tot Haegen) Pa, we.}{gaan nu dat boodschapje doen}{bij de modiste}\\

\haiku{Als mijn schoonzoon zijt.}{ge solidair met mij en}{met ons allen}\\

\haiku{Het is hoogst zeldzaam.}{dat de kunst zooveel opbrengt}{als de handel}\\

\haiku{Papin-Dupont(),,.}{vriendelijk glimlachend}{Ja ja natuurlijk}\\

\haiku{(drukt zijne hand. Tot,)...}{Germaine die opgestaan}{is Germaineke}\\

\haiku{Plechtig) En indien?...}{ik u thans eens zegde dat}{ik haar bemin}\\

\haiku{Germaine O, dat.}{akelig geheim zou ik toch}{zoo gaarne kennen}\\

\haiku{Hij wacht,... hij wacht dat.}{gij hem de toelating geeft}{hier te komen}\\

\haiku{Hij is wijzer dan,.}{gij hij begrijpt wel dat ik}{zal vertrekken}\\

\haiku{Gisteren, tegen '.}{den avond heeft ie hier weln}{uur rondgeloerd}\\

\haiku{Daar net sloeg ie al '.}{weern haas kapot in de}{jacht van den baron}\\

\haiku{(woedend tot Labeer),!}{die weer naar de achterdeur}{wil Wel verdomme}\\

\haiku{Zij slaan een kruis en,.}{bidden even in stilte met}{gevouwen handen}\\

\haiku{Elodie  (kijkt om,,)?}{half norsch half glimlachend}{Wat heb je daar}\\

\haiku{Labeer  (lachend,,) '!}{tot Buck met den haas in de}{hand Zoone gannef}\\

\haiku{al zie ik het graag,.}{zelf er een pakken daar mot}{ik niks van hebbe}\\

\haiku{Elodie  (smeekend), '.}{O Frans doe het nou ins}{Hemelsnaam nooit meer}\\

\haiku{(Hij toont van verre) ' ',.}{den haas aan Labeer Kijkr}{s na vader}\\

\haiku{Toen dien ouwe met:}{zijn grijze bakkebaarde}{tegen Vader riep}\\

\haiku{Labeer loopt eensklaps '.}{naar de deur om die int}{nachtslot te draaien}\\

\haiku{en in z'n brief heet.}{ie an Elodie beloofd dat}{ie nie meer zal stroope}\\

\haiku{Een oogenblik staat.}{hij roerloos naar regen en}{wind te luisteren}\\

\haiku{tegen m'n zin ben -!}{ik eens verhuisd en het is}{m'n verandering}\\

\haiku{en ik zal nog 's.}{probeere of er voor ons geen}{recht te krijgen is}\\

\haiku{en ik zal met 'm......}{wegloopen ook zoo gauw we}{de kans maar schoon zien}\\

\haiku{'k ben er zeker.}{van dat we van jou niks te}{vreeze hebbe}\\

\haiku{Daar heb ik ook al,.}{zoo wat van gehoord meneer}{de notaris}\\

\haiku{Labeer probeert om,.}{op te staan maar zakt bevend}{op zijn stoel terug}\\

\haiku{Ge hebt 'm toch wel.}{hoore zegge dat ie naar}{de laatste trein moest}\\

\haiku{Je hebt ongelijk,, {\textquoteleft}{\textquoteright}.}{Cora datjuffrouw is niet}{om aan te hooren}\\

\haiku{Twee, drie stemmen  (,),?}{tegelijk verontwaardigd}{Hoe zo meneer}\\

\haiku{Zijn jullie zulke,?}{ego{\"\i}sten die alleen aan}{zichzelf denken}\\

\haiku{Je teekent dus die.}{aardige familie voor}{vier kaarten op}\\

\haiku{de costumes van;}{mijn oudste dochter en mijn}{zoon worden gehuurd}\\

\haiku{Barrois Woont U 't,?}{gansche jaar buiten meneer}{Van Raveschoot}\\

\haiku{Dat mensch heeft iets... iets,...}{griezeligs over zich iets waar}{je van huivert}\\

\haiku{Van Maanen  (kijkt schuw.}{Lien Van Thoorn na die in den}{achtergrond verdwijnt}\\

\haiku{Cora  (snikkend) ',... '.}{Ik kant niet helpen ik}{kant niet helpen}\\

\haiku{Angstig dringt Cora.}{zich nog dieper in den hoek}{van de veranda}\\

\haiku{Ik doe d'r mijn best,.}{voor maar Cora moet toch eerst}{nog ja zeggen}\\

\haiku{Arthur  (boos) Zeg, ben,,?}{jij mal Van Maanen of hou je}{me voor de gek}\\

\haiku{Wat is dat daar in,,.}{de keuken die nare lucht}{dat stoomgesis}\\

\haiku{Daarachter, rechts en,.}{links de oude baron en}{baron Maurice}\\

\haiku{Ik weet alleen dat.}{hij te Brussel is en naar}{hier komen z\`al}\\

\haiku{Barones Maurice,?}{Tiens-toi bien tranquille}{n'est-ce pas}\\

\haiku{u mijn vriend meneer.}{Franklin Van Paemel uit Blue}{Springs voorstellen}\\

\haiku{binnen, meneer Van}{Paemel en zet u.}{Oude baron Wees}\\

\haiku{Want anders laten.}{ze wel heel graag de mannen}{voor haar werken}\\

\haiku{Weet ge wel, dat ik?}{hier herhaaldelijk overheen}{gevlogen ben}\\

\haiku{ik geloof dat wij.}{voorloopig voldoende}{afgesproken zijn}\\

\haiku{Oude barones,.}{Wacht nog een beetje meneer}{Van Paemel}\\

\haiku{Van Gele Gedacht ' ',.}{n es eigentlijkt woord}{niet meneer den B'ron}\\

\haiku{We woaren amoal, ',!}{wrie schouw van hem wantt was}{nen dullen zille}\\

\haiku{Oude barones()...}{drukt Franklin de hand Meneer}{Van Paemel}\\

\haiku{Barones Maurice,.}{Nog een taske thee meneer}{Van Paemel}\\

\haiku{Oude baron Vous,...}{savez \c{c}a n'a pas beaucoup}{d'importance}\\

\haiku{Oude barones(),?}{schrikkend Qu'est-ce qui}{te prend Maurice}\\

\haiku{Denkt aan uw naam en...!}{maatschappelijken stand en}{aan de mijne}\\

\haiku{Een tijd geleden,?}{heb ik u hier luchtfotos}{laten zien niet waar}\\

\haiku{Ik ken weinig of,.}{geen Fransch maar toch begrijp ik}{die twee woorden wel}\\

\haiku{Bij ons hoeven ze '.}{zichs nachts niet in bosschen}{te verschuilen}\\

\haiku{Zij valt snikkend met.}{gevouwen handen v\'o\'or Franklin}{op haar knie\"en}\\

\haiku{Ik vraag u, meneer,,,.}{wat gij hier doet bij mijn vrouw}{bij mijn dochter}\\

\chapter[22 auteurs, 5931 haiku's]{tweeëntwintig auteurs, vijfduizendnegenhonderdeenendertig haiku's}

\section{M.H. van Campen}

\subsection{Uit: Bikoerim}

\haiku{Zij kreeg nu in haar:}{vermoeienis-doezig}{hoofd gedachten van}\\

\haiku{{\textquoteleft}Nou bent u de vrouw,.}{met de twee hoofden u kunt}{op de kermis gaan}\\

\haiku{hij zat nog goed, maar '...}{daar had jet weer met die}{plooien in de rok}\\

\haiku{En ze trad op haar, ', '.}{moeder toe zoender sloeg}{r arm om haar heen}\\

\haiku{Nu zit-ie goed,{\textquoteright}.}{zei ze zich weer oprichtend}{en nog even kijkend}\\

\haiku{En ze wendde zich,.}{om streek in wachting met de}{hand over de tafel}\\

\haiku{Tante, terwijl zij,:}{opstond en naar de tafel}{trad zei plagerig}\\

\haiku{{\textquoteleft}Och mevro\`uw, hoe kan,...}{u nou zoo'n gekheid prate}{ja we zijn op reis}\\

\haiku{{\textquoteright} En zij bleef haar 'n.}{seconde in dreigende}{afwachting aanzien}\\

\haiku{{\textquoteleft}O, da's goed, U zorgt,...}{er wel voor dat we vanavond}{kunnen afreizen}\\

\haiku{{\textquoteleft}Betsy was een heel,...}{klein meisje ze was erg stout}{en zoo'n dr{\`\i}ftkopj\`e}\\

\haiku{heel voorzichtig, dat... '... '}{je niet wakker zou worde}{enk hiel me stil}\\

\haiku{In mijn zwijgen kent...}{Gij mijn vragen en ik blijf}{aan U verbonden}\\

\haiku{God kan toch om de...}{smart van een mensch zijn eigen}{werk niet verstoren}\\

\haiku{Wat is de vreugde ', '...}{van alt mooie de liefheid}{van alt teere}\\

\haiku{n\^o, zoolang zal me..., '.}{dalles dure die brosj die}{neemt nogn sof in}\\

\haiku{{\textquoteright} {\textquoteleft}O,{\textquoteright} zei de oude,.}{vrouw en ze keek star voor zich}{uit in gedachten}\\

\haiku{hij... met z'n linksheid..., '...}{in zoo'n huis waart bepaald}{erg vroolijk toeging}\\

\haiku{{\textquoteleft}S't, s't,{\textquoteright} fluisterde,.}{mevrouw angstig den vinger}{op den mond leggend}\\

\haiku{precies...{\textquoteright} En ze moest,...}{even ophouden ze kon niet}{verder van lachen}\\

\haiku{gij zult \`alle uwe:}{levensdagen den tocht uit}{Egypte herdenken}\\

\haiku{Toch ben 'k trotsch op,...}{Z{\`\i}jn grootheid en ik b\`en niet}{arm ik b\`en niet arm}\\

\haiku{paaienden toon van,}{overdreven-geuite}{blijheid waarme\^e men}\\

\haiku{{\textquoteleft}God heeft toch alles,...}{ten beste gekeerd denkt u}{d'r maar niet meer an}\\

\haiku{Toen hief de vader:}{het smeeklied voor den herbouw}{van den tempel aan}\\

\haiku{dan heb ik weer te.}{veel gedronke en dan ben}{ik weer uit geweest}\\

\haiku{{\textquoteleft}Maar hoe moet dat nu,{\textquoteright}, {\textquoteleft}...?}{eigenlijk vroeg Roelofsen}{twee partijen of}\\

\haiku{staat dat niet ergens,?}{de wolf en het lam zullen}{te zamen grazen}\\

\haiku{{\textquoteleft}Neen 't is verkeerd,, '!}{te vroeg wij zouden gekraakt}{worden alsn noot}\\

\haiku{Stomdronken had-ie,...}{zich over het tapijt gerold}{had overgegeven}\\

\haiku{{\textquoteleft}Wilt uwes dan effen, ',}{anneme d'r isn brief}{voor u uwes ma zegt}\\

\haiku{{\textquoteright} {\textquoteleft}Och, dat geeft niets... 't.}{zal daar wel bedaren als}{de lucht wat optrekt}\\

\haiku{{\textquoteleft}O, ma, da's nou niet,.}{aardig u weet toch heel goed}{hoe ze van u houdt}\\

\haiku{{\textquoteright} {\textquoteleft}God, nee jonge, dat ', ',.}{zegk niet krijgm maar uit}{de kast daar staat-ie}\\

\haiku{{\textquoteright} {\textquoteleft}Nee, nee, ik met m'n!}{onprecieze handen in}{jou precieze kast}\\

\haiku{{\textquoteleft}Weet je nou dat ik, '!}{je h\'a\'at je niet kan uitstaan}{opt oogenblik}\\

\haiku{'t was zoo'n heerlijk}{onderwerp om d'r over te}{philosopheeren}\\

\haiku{Toch gingen zij er,, ' '.}{door op een drafjet was}{maarn klein eindje}\\

\haiku{Iets vreemds moest zich in,...}{hem hebben genesteld zich}{hebben uitgedijd}\\

\haiku{{\textquoteright} {\textquoteleft}Nee, nee, Harmse, blijf,,.}{maar ik kom boven ik heb}{iemand meegebracht}\\

\haiku{dat was 't begin......}{scheen den weg te wijzen naar}{dat onbekende}\\

\haiku{{\textquoteleft}Nee, meheer, dat zal,.}{niet gaan kan de j\'uffer niet}{na binne late}\\

\haiku{zou late blikseme,,.}{de heele keet as-ie om}{elf uur geen geld had}\\

\haiku{{\textquoteleft}Neen... bonsoir neef{\textquoteright} en.}{verdween achter de tochtdeur}{van den corridor}\\

\haiku{Even bleef hij staan, als,,:}{versuft trippelde dan de}{stoep af mompelend}\\

\haiku{Toen je van jongen,, '}{man begon te worden dacht}{k wat verdienste}\\

\haiku{Iemand, die z\'o\'o iets,...}{zonder noodzaak dee mo\'est toch}{wel krankzinnig zijn}\\

\haiku{{\textquoteleft}Nee... zoo laat 'k je...'...}{niet gaan kom hier dat  k}{je gezicht kan zien}\\

\haiku{{\textquoteleft}Maar 'k d\'ank u, 'k, '...}{d\'ank uk kan u daar nooit}{genoeg voor danken}\\

\haiku{Jaw\'el, hij wou h\'em '...}{wel eris opn winkel zien}{zitten an d\'at goed}\\

\haiku{hij sprak, dicht bij zich,,.}{te hebben ging hij naast Jaap}{loopen greep diens arm}\\

\haiku{En hij dacht nu aan,:}{de vergadering die hij}{straks zou bijwonen}\\

\haiku{Die vergadering... '}{van avond kon-ie nou ook wel}{missen als kiespijn}\\

\haiku{Jaap teuterde nog,.}{even niet kunnend besluiten}{naar binnen te gaan}\\

\haiku{Ze zou 't 'm daar}{wel an z'n verstand zien te}{brengen as Sem}\\

\haiku{En dan moet je niet '...}{zoo alle woorden opn}{goudschaaltje legge}\\

\haiku{En nou kon-ie 't...}{ook juist beter dan morge}{as-ie d'r voor stond}\\

\haiku{Nou speet 't 'm wel,,... '}{hij had ze zoo graag verteld}{dat-ie werk had}\\

\haiku{Nou... Weet je wat, laat,,}{die kist dan maar staan k'm hier}{ken je me helpe}\\

\haiku{Siesoo, geef nou m'r, '.}{de verf dan salle wem}{effe schildere}\\

\haiku{Je bint nog soo jong,, ' '...}{sal je segge maark mot}{t je toch s\`egge}\\

\haiku{En dan vraag je maar, '.}{meteen of je uit ken snije}{t is toch op slag}\\

\haiku{Och nee... dat k\'on-ie, ',...}{niet doen zij was \'o\'okn oud}{mensch zoo goed als hij}\\

\haiku{net of 'n m\`ensch je - '...}{wat gedaan h\^et toen had-ie}{n soort wrok in zich}\\

\haiku{Hij had de arme '... '...}{omhoog geslage inn}{stuip inn wildheid}\\

\haiku{Och 't verveelt me,.}{daarover te spreken daar weet}{je toch nog niks van}\\

\haiku{Dat lag voor hem als,.}{iets langs iets blinkends waaraan}{geen eind was te zien}\\

\haiku{{\textquoteleft}Wel nee jonge, dat '... '...{\textquoteright} {\textquoteleft}}{wasn Portugeesche jood}{n baronHad-ie}\\

\haiku{net of 't arme.}{schaap daar wat an  doen kon}{en dan ging ie weg}\\

\haiku{as je ergens zoo,}{lang gewees bint en je word}{dan zoo behandeld}\\

\haiku{{\textquoteright} {\textquoteleft}Nou, dat hindert toch...,{\textquoteright} '.}{niks juffrouw toe poogde hij}{r over te halen}\\

\haiku{Hield ze 'm nog voor ', '.}{n kind daar straks had ze nog}{zoo metm gepraat}\\

\haiku{och heere-j\'e in ' '.}{t werkelijke leven}{gaatt anders toe}\\

\haiku{en nou worre we......}{an de dijk geset sie nou}{maar hoe je d'r komt}\\

\section{Jan Campert en Ben van Eysselsteijn}

\subsection{Uit: Het Chineesche mysterie}

\haiku{{\textquoteleft}Geef mij 'n oorlam,{\textquoteright},;}{commandeerde het Jantje}{dat een Jaantje was}\\

\haiku{Hij vormde met de.}{beide oude heeren een}{merkwaardig contrast}\\

\haiku{{\textquoteleft}Ga werken, man of,,!}{meld je bij het armbestuur}{ingerukt marsch}\\

\haiku{{\textquoteright} {\textquoteleft}Maar u houdt maar vol,,{\textquoteright}.}{dat ik zorgen heb trachtte}{Veraart te schertsen}\\

\haiku{Hij zette den stoel.}{overeind en liep door naar de}{geopende deur}\\

\haiku{{\textquoteright} {\textquoteleft}Precies wat ik dacht,{\textquoteright}, {\textquoteleft}:}{zei dr. van Buren peinzend}{precies wat ik dacht}\\

\haiku{hij is niet op de,.}{Witte niet thuis en noch bij}{een van u beiden}\\

\haiku{Het is beter om,{\textquoteright}.}{alles te zeggen ging kalm}{van Buren verder}\\

\haiku{{\textquoteright} Oversteeg lichtte zijn,.}{hoed en wilde verdwijnen}{rechts de Javastraat in}\\

\haiku{{\textquoteright} {\textquoteleft}Da's allemaal mooi,{\textquoteright}, {\textquoteleft}:}{bromde de overstemaar het}{is zooals je zelf zegt}\\

\haiku{{\textquoteright} {\textquoteleft}Was er niets waar je,,,?}{aandacht op viel op de gang}{op de trap op straat}\\

\haiku{Je verbeeld je toch?}{soms niet dat je een halve}{Sherlock Holmes bent}\\

\haiku{Zooeven heb ik me,.}{ook al erover verbaasd dat}{je daar op inging}\\

\haiku{Even plechtig als hij,.}{gekomen was verdween de}{correcte dienaar}\\

\haiku{de vlammende haard;}{met breede clubfauteuils en}{het rooktafeltje}\\

\haiku{Geef mij het kalme.}{Voorhout en den Vijverberg}{met z'n meeuwen maar}\\

\haiku{{\textquoteright} De bediende - een -.}{Chinees gluurde even naar de}{auto en verdween}\\

\haiku{Hoeng Tsi Lang ging hen.}{voor de trap op en liet hen}{in een ruim vertrek}\\

\haiku{De garage kwam.}{op een ruime binnenplaats}{uit met tal van boxes}\\

\haiku{Hij keek op toen hij.}{hen hoorde aankomen en}{zag hen vragend aan}\\

\haiku{Misschien willen de?}{heeren even binnen komen}{en telefoneeren}\\

\haiku{We gaan nu naar de.}{politie en verzoeken}{haar medewerking}\\

\haiku{{\textquoteright} knalde een kort en.}{droog schot uit het dakraam van}{het boardinghouse}\\

\haiku{Met een paar passen.}{was hij de kamer door en}{achter het buffet}\\

\haiku{{\textquoteleft}Ik maak mijn excuus{\textquoteright},.}{voor dit binnendringen kwam}{Veraart hoffelijk}\\

\haiku{Dr. van Buren hield.}{op met trommelen en wreef}{nerveus zijn handen}\\

\haiku{De heeren willen?}{hun getuigenis wel even}{onderteekenen}\\

\haiku{Dr. van Buren stak,.}{een sigaar op maar Overste}{Mensing bedankte}\\

\haiku{{\textquoteleft}Nee, beste kerel, '.}{schei noues uit met al die}{fraaie beleefdheden}\\

\haiku{Overste Mensing liep.}{nerveus heen en weer als een}{gekooide tijger}\\

\haiku{meneer,{\textquoteright} antwoordde.}{Max en verwijderde zich}{naar de leestafel}\\

\haiku{In dien tusschentijd,}{bel ik den commissaris}{van politie op}\\

\haiku{{\textquoteright} De commissaris.}{had diepe rimpels in zijn}{voorhoofd gekregen}\\

\haiku{{\textquoteright} {\textquoteleft}Zeker,{\textquoteright} antwoordde, {\textquoteleft}?}{van Buren gehaastkunnen}{wij u van dienst zijn}\\

\haiku{{\textquoteright} De commissaris.}{deed het verhaal rustig en}{zonder eenig vertoon}\\

\haiku{{\textquoteright} {\textquoteleft}Dat is het voor ons,,{\textquoteright}.}{allemaal Dr. van Buren}{klonk koel het antwoord}\\

\haiku{Kunnen we met z'n?}{allen niet \'e\'en zoo'n Chinees}{bij zijn staart pakken}\\

\haiku{v\'o\'or dien tijd was het....}{niet anders geweest dan een}{eenvoudige moord}\\

\haiku{Informeer naar die, ', '.}{auto gam achterna}{zorg dat jem krijgt}\\

\haiku{Vooral omdat zij.}{niet in conditie zijn om}{op ijs te landen}\\

\haiku{Hij zweeg, boog zich tot.}{zijn Haagschen collega over}{en fluisterde iets}\\

\haiku{commissaris gaat,.}{u      Wie een beweging}{maakt schiet ik neer}\\

\haiku{Dit was bittere,....}{ernst \'e\'en beweging en het}{was afgeloopen}\\

\haiku{Het heeft ook geen zin.}{middelen te beramen}{om te ontkomen}\\

\haiku{{\textquoteleft}Met \'e\'en revolver....}{had ik u allen niet in}{bedwang gehouden}\\

\haiku{lange dubbele,.}{rijen waar zij over rechte}{hoofdstraten vlogen}\\

\haiku{zij knipoogden als.}{uilen in het felle licht}{van de schijnwerpers}\\

\haiku{Je hebt eieren.}{voor je geld gekozen en}{geen lawaai gemaakt}\\

\haiku{Je zult wel gelooven,.}{dat we dat niet doen voor een}{vaatje Schiedammer}\\

\haiku{Even om den hoek stond.}{een donkere gesloten}{auto aan den kant}\\

\haiku{Als hij niet oppast,;}{slaat hij met zijn sloep en z'n}{koffer te pletter}\\

\haiku{Hij stak zijn hand uit,,.}{naar de jonge mooie vrouw die}{een lichten kreet gaf}\\

\haiku{Hij snelde op de.}{jonge vrouw toe en ving haar}{in zijn armen op}\\

\haiku{Zacht legde hij haar.}{neer en steunde knielend haar}{hoofd in zijn armen}\\

\haiku{Hij liep kaarsrecht zijn,}{kamer op en neer veegde}{zijn brilleglazen}\\

\haiku{{\textquoteright} Veraart lachte en:}{ging naast Paula H\"ulshoff op}{den divan zitten}\\

\haiku{Den laatsten keer dat,:}{ik weer in den Haag kwam trof}{mij een verrassing}\\

\haiku{zijn eerste vraag was.}{naar vader's bagage en}{naar zijn handkoffer}\\

\haiku{{\textquoteright} dacht ik en ik bad.}{vurig dat hij nog tijdig}{arriveeren zou}\\

\haiku{Ik bad in stilte.}{nog vuriger dat nu Dr.}{Li toch komen mocht}\\

\haiku{Een heer wachtte op.}{mijn kamer en verzocht mij}{dringend mee te gaan}\\

\haiku{Zij kon mij dus niet.}{op de hoogte brengen van}{de documenten}\\

\haiku{Een Chinees met een.}{auto heeft onzen vriend Mr.}{Veraart opgelicht}\\

\section{Jan Campert}

\subsection{Uit: Die in het donker...}

\haiku{Er is wel eens een,.}{tijd geweest in zijn leven}{dat hij dat w\`el deed}\\

\haiku{E\'en keer kreeg hij een,.}{baantje in een sportwinkel}{voor drie  maanden}\\

\haiku{- Tot ziens, heeft ook Joost.}{Verheijde gezegd en is}{zijn dag begonnen}\\

\haiku{Met mosterd en dan.}{haalt Truus eigenhandig het}{velletje er af}\\

\haiku{Ze heeft krullende,.}{bruine haren en draagt een}{helder-witte schort}\\

\haiku{Hij gaat zitten in,.}{een der fauteuils en glimlacht}{voor het \'e\'erst sinds lang}\\

\haiku{Maar het is nog vroeg.}{in het borrel-uur en er}{zijn nog geen heeren}\\

\haiku{Reuze fideel, we '.}{drinken samenn borrel}{en nog een borrel}\\

\haiku{Joost Verheijde en.}{Toontje M\"uller hebben De}{Kakatoe verlaten}\\

\haiku{Een tram raast voorbij... -,,...}{Daar moet ik mee oppassen}{denkt Joost met die trams}\\

\haiku{Een oogenblik schuilt,.}{hij in een portiek steekt er}{een sigaret op}\\

\haiku{Hij leunt tegen den,.}{kouden muur even strijkt zijn hand}{langs de kille steenen}\\

\haiku{Ieder op eigen.}{gelegenheid naar Oome}{Daan was de afspraak}\\

\haiku{De vrouw verlaat de,.}{kamer slaat de deur met een}{slag achter zich dicht}\\

\haiku{Joost gaat zitten aan.}{de met een rood pluche kleed}{overdekte tafel}\\

\haiku{- Vooruit, steek weg, voegt,.}{Toontje M\"uller hem toe dan}{gaan we naar voren}\\

\haiku{Het is niet alleen.}{de wetenschap dat Zwarte}{Lizzy op hem wacht}\\

\haiku{- Och, zegt hij vaag, ik... -?}{weet niet Je ken toch bij mij}{blijve tot morge}\\

\haiku{- Da's m'n vaste, legt,.}{ze hem uit die komt altijd}{\`e\`en avond in de week}\\

\haiku{- Verrek jij, antwoordt,.}{Marie maar gooit evengoed het}{pilsje achterover}\\

\haiku{Maar as ik jou was!}{zou ik dat niessie~		 in}{de gaten houden}\\

\haiku{Eens in het jaar gaan,}{ik d'r altijd opzoeke}{ze weet niet beter}\\

\haiku{Dan staat ze op, hangt.}{haar mantel en hoed weg in}{het zijkabinet}\\

\haiku{Ik zou best wille',.}{eindigt de vrouw en haar oogen}{zien hem vragend aan}\\

\haiku{Hij bladert in de.}{papieren met notities}{die voor hem liggen}\\

\haiku{Hij zou best Mabel.}{nog kunnen afhalen als}{het werk gedaan was}\\

\haiku{Hij verneemt dat de.}{voorstelling om kwart over elf}{is afgeloopen}\\

\haiku{- Ik zie dat u toch,.}{niet zoo verkeerd ingelicht}{bent commissaris}\\

\haiku{Maar des middags draalt.}{het licht langer en het ijs}{in de grachten smelt}\\

\haiku{En De Koorddanser.}{die hem kent van vroeger geeft}{dan ook geen antwoord}\\

\haiku{Je luistert naar Buys,.}{Colebrander zonder te}{hooren wat hij smoest}\\

\haiku{De Koorddanser maakt?}{een onverschillig gebaar}{met zijn hand. Morgen}\\

\haiku{Als Toontje het nu...}{niet te laat maakt hier zal hij}{haar straks nog treffen}\\

\haiku{Oh, niet om het een,.}{of ander want Sjan\`etje is}{met Bill tevreden}\\

\haiku{Hij luistert naar een;}{lezing die over Daventry}{wordt uitgezonden}\\

\haiku{Als Sjan\`etje zich bij.}{een clubje voegde daalde}{de stemming zichtbaar}\\

\haiku{Daar heeft ze het bij.}{de hand en morgen komen}{ze toch om de huur}\\

\haiku{- 't Is m'n brood, zegt,.}{De Stille kortaf als zij}{er soms op zinspeelt}\\

\haiku{als het iets of  , -...}{iemand was die je te lijf}{kon gaan verlo\`or je}\\

\haiku{Je weet niet waar het,}{vandaan komt en je weet niet}{eens ho\`e het komt maar}\\

\haiku{Nee, denkt de vrouw, hij,.}{heb eigenlijk gelijk ook}{da's nergens voor noodig}\\

\haiku{- Voor de politie.}{wist U zich toch heel wat meer}{te herinneren}\\

\haiku{Niets op haar gezicht,.}{niets in haar oogen verraadt wat}{er in haar omgaat}\\

\haiku{Langzaam, voorzichtig,.}{beetje bij beetje wordt het}{net dichtgetrokken}\\

\haiku{W{\`\i}j vinden het dooden;}{van een mede-mensch een}{der grootste zonden}\\

\haiku{Huizen en water,...}{lichte nevel en zon op}{grauwe daken}\\

\haiku{Ook al moet je maar.}{zien hoe je er morgen en}{overmorgen weer komt}\\

\haiku{As je vroeger 'n '.}{klant had kwam die nog weles}{royaal over de brug}\\

\haiku{Je wordt op een avond:}{aangesproke en dan denk}{je bij je eigen}\\

\haiku{En dan die boeren.}{van tegenwoordig brengen}{alles naar de bank}\\

\haiku{Over 'n paar uur zal.}{hij terugkomen alsof}{er niets gebeurd is}\\

\haiku{Het valt hem mee, je.}{loopt niet elken dag tegen}{negen meier aan}\\

\haiku{- Da\`ag, zegt de vrouw en,.}{wrijft met haar hoofd tegen zijn}{schouder je bent vroeg}\\

\haiku{Dien middag achter.}{in Juni was de borrel}{zeer geanimeerd}\\

\haiku{Ze zijn alle drie.}{in een zeer plezierige}{stemming  geraakt}\\

\haiku{W\`el, in De Kakatoe!}{is het dien avond achter in}{Juni gezellig}\\

\haiku{- Lekkere jonge',...}{ben jij zegt de vrouw en dringt}{zich tegen hem aan}\\

\haiku{Maar Joost Verheijde -,!}{had hem w\`el in De Kakatoe}{gezien en goed goed}\\

\haiku{De dagen gaan, het.}{water ruischt en lang zijn}{de zachte nachten}\\

\haiku{Om de vreugde die.}{hij beleeft aan deze vrouw}{en aan haar liefde}\\

\haiku{En ze ziet zichzelf.}{al op het bal masqu\'e als}{Zeeuwsch boerinnetje}\\

\haiku{De vrouw die leefde.}{zonder problemen en die}{zoo gelukkig was}\\

\haiku{Er is iets, denkt De,... -?}{Stille er is iets met haar}{Is er iets gebeurd}\\

\haiku{Hij ziet alleen het}{doodelijk-witte gelaat daar voor}{hem op het kussen}\\

\haiku{Ze zal met een paar.}{andere vrouwen Zwarte}{Lizzy afleggen}\\

\haiku{Dat heeft de boerin.}{De Stille verteld en De}{Stille vond het goed}\\

\haiku{- Die, zegt De Stille,'.}{dan onverschillig die zal}{wel wijzer weze}\\

\haiku{Een jongen als De,.}{Stille en dan zonder meid}{dat is niks gedaan}\\

\haiku{Hij aarzelde even.}{toen Greet hem voorstelde bij}{haar in te trekken}\\

\haiku{Kwart over twaalf kan de,.}{man weer binnen zijn kan De}{Stille verdwijnen}\\

\haiku{Tusschen het water.}{en hem ligt vijftig meter}{sneeuw-overdekt terrein}\\

\haiku{maar hij ziet alleen,}{het doodelijk-witte gelaat daar}{voor hem in de sneeuw}\\

\subsection{Uit: Slordig beheer}

\haiku{Helen reisde 's.}{morgens vroeg met denzelfden}{trein als ik terug}\\

\haiku{Voorzichtig opende,.}{ik de deur stond in een vrij}{ruim en licht vertrek}\\

\haiku{Als Anka dien toon.}{laat hooren is er geen kruid}{tegen gewassen}\\

\haiku{Maar dat kan ook wel... -?}{aan mij liggen Moeten we}{niet eens wat gaan eten}\\

\haiku{We zijn al binnen,.}{enkele oogenblikken}{verblind door het licht}\\

\haiku{Ook kon men immers.}{nooit weten of de gezant}{je niet opmerkte}\\

\haiku{Ook de verhouding.}{met haar moeder zou dan niet}{geleden hebben}\\

\haiku{Hoe het zij, er kwam.}{een amoureuze inslag in}{onze relatie}\\

\haiku{Op den duur werd het,.}{vervelend maar je went er}{tenslotte wel aan}\\

\haiku{Ik zag dat ook nu.}{niets of niemand hem van zijn}{plan zou afbrengen}\\

\haiku{{\textquoteright} Voor het overige}{was hij redelijker dan}{ik hem ooit  had}\\

\haiku{Waarom heb ik ooit?}{de moeite genomen om}{dit op te schrijven}\\

\haiku{Eindelijk val ik,,...}{op mijn bed neer stort in een}{looden droomloozen slaap}\\

\subsection{Uit: Wier}

\haiku{Het kind zit op de.}{vloer te spelen met een paar}{blokken en een stoof}\\

\haiku{Even nog wacht Tanne.}{totdat zij zekerheid heeft}{dat het kind slaapt}\\

\haiku{Traag wendt zij het hoofd:}{terzijde en half over de}{schouder heen vraagt zij}\\

\haiku{Voor het eerst die nacht,,.}{lacht Tanne Ingelse een}{kleine korte lacht}\\

\haiku{Aarzelend welhaast.}{beroeren haar vingers het}{zachte kinderhaar}\\

\haiku{Arjaan onderbreekt,.}{even zijn werk haalt minachtend}{snuivend zijn neus op}\\

\haiku{Er ligt een tjalk met,,.}{stenen die gelost worden}{bij het Grote Hoofd}\\

\haiku{Daarbij is hij lid.}{van de kerkeraad en van}{de gemeenteraad}\\

\haiku{Een  klein domein.}{dat hij verdedigen zou}{tot het uiterste}\\

\haiku{De enige, die er,.}{wat aan doen kan is Cysouw}{en die verdomt het}\\

\haiku{En de grond zou, met,.}{God's hulp goed voor hem zijn en}{de pacht opbrengen}\\

\haiku{En als zij elkaar.}{al eens ontmoetten dan had}{Wanne altijd haast}\\

\haiku{Maar dan zal Jaap hem.}{zeker zien en de honden}{op hem aanhitsen}\\

\haiku{Maar Gekke Floris,...}{knikt heftig van neen klemt de}{lippen op elkaar}\\

\haiku{Hij ziet een zwarte.}{vlek afsteken tegen het}{lichte duinzand}\\

\haiku{Tanne keek soms uit.}{die ogen van d'r alsof er}{niets gelegen kwam}\\

\haiku{Gabe zet zijn fiets.}{in een stalling en slentert}{wat door de straten}\\

\haiku{Hij weet eigenlijk.}{niet goed wat hij nu met die}{uren beginnen moet}\\

\haiku{Dan zat hij trots naast}{vader en klakte met zijn}{tong net als vader}\\

\haiku{De kastelein schijnt.}{zo'n  beetje te dutten}{achter de toonbank}\\

\haiku{En hij stelt er ook,.}{niet zo heel veel belang}{in eerlijk gezegd}\\

\haiku{Lou van Zakke en.}{Loes zijn in een fluisterend}{gesprek gewikkeld}\\

\haiku{De kleuren van helm,,,.}{grassen duindoorn distels en}{zand vloeien ineen}\\

\haiku{Zij neemt het kind van,.}{hem over dat hij behoedzaam}{in haar armen legt}\\

\haiku{Gekke Floris zit.}{op de bank voor het huis van}{Tanne Ingelse}\\

\haiku{Die heeft altijd d'r.}{eigen wil gehad en d'r}{eigen zin gedaan}\\

\haiku{Lou broeit hoe hij over.}{wat hij vanmiddag ontdekt}{heeft kan beginnen}\\

\haiku{Zonder dat zij het.}{zelf merken zijn hun woorden}{luider geworden}\\

\haiku{Al vertellen zijn.}{woorden dan ook precies hoe}{alles gebeurde}\\

\haiku{Vroeger zou hij niet.}{zo licht met iemand als Lou}{op stap zijn gegaan}\\

\haiku{En dan gaene me' '.}{morrege noges mee die}{van Verdeene uut}\\

\haiku{Tussen een uit stad '.}{en een vant dorp is het}{nog nooit goed gegaan}\\

\haiku{Al moet je de winst,.}{dan ook delen dit heeft toch}{ook zijn voordelen}\\

\haiku{Als het zo te pas}{komt en burgemeester Van}{Ryssel kennend zal}\\

\haiku{De boer zet zijn bril.}{op en slaat de bijbel open}{bij de leeswijzer}\\

\haiku{Zij zal een pronte,.}{boerin zijn op Nooit Gedacht}{denkt hij bij zichzelf}\\

\haiku{Belangrijker is.}{of de mens wandelt in de}{vreze des Heren}\\

\haiku{Ze hebben op het.}{dorp altijd medelijden}{gehad met Sanne}\\

\haiku{Die is eigenlijk,,.}{zo zeggen ze op het dorp}{van hetzelfde slag}\\

\haiku{Zie je, zeiden die,.}{op het dorp dan dat kind wordt}{noe al eenzelvig}\\

\haiku{In breed en machtig.}{zwieren tegen de hemel}{Lein kijkt de zot aan}\\

\haiku{- Jezis-Maria,,'...}{fluistert ze in haar angst daer}{komme rampen van}\\

\haiku{- Den '\'ond, zegt ze zacht,, '...}{hardnekkig en vol van haat}{die verdoemdenond}\\

\haiku{Soms denkt Kee wel eens.}{dat zij dokter maar zijn zin}{had moeten geven}\\

\haiku{Er brandt nog licht in,.}{de gelagkamer maar de}{deur is al op slot}\\

\haiku{naer je wijf en je, '?}{huus waerom zit je dan}{ier in het Waepen}\\

\haiku{Ge hebt er meer zorg,.}{en verdriet van dan aarigheid}{denkt hij bij zichzelf}\\

\haiku{Als hij een avond in.}{stad wil blijven dan  blijft}{hij een avond in stad}\\

\haiku{Want tenslotte, d'r.}{zijn er niet velen op het}{dorp met zo'n toestel}\\

\haiku{En Lou is er de:}{man niet naar om tegen wie}{maar wil te zeggen}\\

\haiku{Dat had ook  zo.}{zijn praktiese kant als ge}{met z'n twee\"en zijt}\\

\haiku{Want waar bleef hij zo.}{gauw met een vrachtje van het}{een of het ander}\\

\haiku{Daar had hij dan wel,?}{niet aan gedacht maar waarom}{zou hij niet meegaan}\\

\haiku{Aan de andere.}{kant der duinen heeft men er}{minder hinder van}\\

\haiku{Een korte strekke,.}{verderop ligt de pan die}{hij op het oog heeft}\\

\haiku{Vandaag zal Wanne.}{Cysouw met die oudste van}{Hubrechtse trouwen}\\

\haiku{En op het dorp zijn,.}{er maar weinigen die het}{geen goed stel achten}\\

\haiku{Voor Lena, de meid,.}{van de Olmenhoeve zijn}{het grote dagen}\\

\haiku{Tegen Kees, de knecht,.}{toen ze samen  op de}{deel bezig waren}\\

\haiku{Zij vertelt het niet,.}{alleen aan de boer maar aan}{al de anderen}\\

\haiku{Er gingen hele.}{dagen voorbij dat haar beeld}{niet in hem opkwam}\\

\haiku{Ergens in een der.}{hoeken staat een fust bier om}{de dorst te verslaan}\\

\haiku{hij knikt maar met zijn,.}{rode verhitte kop van}{ja en schudt van neen}\\

\haiku{Als er alleen de,.}{boeren waren dan zou het}{zo'n vaart niet lopen}\\

\haiku{Hij steunt met zijn hand,.}{op een stoel zijn ogen staan hard}{en vol dreigement}\\

\haiku{Die van Aagtekerke.}{en het andere knechtvolk}{dringen naar voren}\\

\haiku{Achter hen, midden,.}{op straat probeert Lein Lap zich}{staande te houden}\\

\haiku{Arjaan kan van die.}{slimmigheden hebben als}{een volwassene}\\

\haiku{Andere malen,.}{is het ver in de nacht dat}{hij aan komt zetten}\\

\haiku{Die is, hoe lang zij,.}{ook op het dorp woont haar jeugd}{nog niet vergeten}\\

\haiku{En zo\"even nog}{toen hij zijn hof verliet wist}{hij niet precies h\'oe}\\

\haiku{Zo dadelijk kan.}{zij nu haar bedrijvigheid}{niet terugvinden}\\

\haiku{Na jaren kwam soms.}{een vreemdeling de weg naar}{het kerkhof vragen}\\

\haiku{Hier en daar in het.}{dorp neemt een vrouw de blinden}{van de vensters weg}\\

\haiku{Zij zit rechtop op.}{de met leer overtrokken bank}{die langs de muur staat}\\

\haiku{Daar behoeft ge niet,.}{aan te raken want wijken}{doen zij geen duimbreed}\\

\haiku{Zij likt aan duim en.}{wijsvinger en telt het geld}{voor de mannen uit}\\

\haiku{- Hei-hei, valt Gabe,}{haar in de rede en ik}{doch dat je toch na\^er}\\

\haiku{Waarachtig, z\'o heeft.}{Marie nog maar liever dat}{die twee opkrassen}\\

\haiku{- Tj\`essis, zegt Marie,'.}{met veel luidruchtigheid wat}{binne jullie stil}\\

\haiku{advocaat uit haar:}{glas en zegt zo argeloos}{als een vrouw dat kan}\\

\haiku{Dat komt wel later,.}{tegen de tijd dat ge een}{mens eerst missen gaat}\\

\haiku{Tegen zes uur in.}{de morgen sterft de boer van}{de Olmenhoeve}\\

\haiku{- ... van zwa\^er eiken en... -?}{met zuiveren beslag Met}{zuiveren beslag}\\

\haiku{Maar op een dag als.}{vandaag ligt het niet alleen}{aan de jaren}\\

\haiku{Er zal temet nog,.}{meer komen vallen daar kunt}{ge gerust op zijn}\\

\haiku{Die laat nog na zijn.}{dood goed merken dat hij een}{man van gewicht is}\\

\haiku{Die slaapt zijn laatste,.}{slaap onder een hemel grauw}{en dik van sneeuw}\\

\haiku{- Laat 'es zien, Vader,,...}{zegt Marie liefjes wat ge}{aan te bieden hebt}\\

\haiku{Daarmee is ze sinds.}{enige maanden bij Gabe}{niet aan een goed adres}\\

\haiku{Voordat de ander,,.}{de list doorziet springt hij naar}{voren grijpt hem beet}\\

\haiku{Zij gaat ermee naar,.}{de pomp en reinigt het geeft}{het hem dan terug}\\

\haiku{Maar moeizaam tast hij,,.}{de weg verder wankelt even}{herstelt zijn evenwicht}\\

\haiku{Het geluid van de.}{boei voor de kust loeit dof en}{gonzend in hun oren}\\

\haiku{- 't Zal morgen 'n'',...}{goeie dag wizze om ons land}{te spitte Ga\^obe}\\

\section{Bernard Canter}

\subsection{Uit: Kalverstraat}

\haiku{Eduard moest dokter.}{worden omdat het deftig}{was dokter te zijn}\\

\haiku{Twee duizend gulden.}{bruidschat had ze van den vader}{mede gekregen}\\

\haiku{Dat genot moet ik.}{nu waardeeren als een geluk}{zegt rebbe Zadik}\\

\haiku{Toen was ze op een.}{dag zoek geraakt en sedert}{stond Mantua alleen}\\

\haiku{Mantua was alleen.}{met zijn schilderskistje naar}{Vincennes gereisd}\\

\haiku{Hij schilderde met.}{een zeker overleg en met}{een zekeren dwang}\\

\haiku{Daar in 't midden,,,,.}{tusschen twee kolommen stond}{strak rechtop een vrouw}\\

\haiku{{\textquoteright} {\textquoteleft}Nou 'k zou maar met ', '.}{m oppassent Lijkt me}{toch al zoo'n rare}\\

\haiku{Wat anders kon ze,.}{niet zingen omdat ze dat}{nog niet geleerd had}\\

\haiku{De moeder had haar '.}{dadelijk een slag int}{gelaat gegeven}\\

\haiku{Halma keek even naar:}{de wonde in den nek en}{dadelijk lachend}\\

\haiku{wij zijn allemaal ..,}{menschen en Dani\"el m\`ot}{men veel vergeven}\\

\haiku{Men krijgt crediet en - '...}{men geeft crediett een reikt}{het ander de hand}\\

\haiku{{\textquoteright} Er bleven telkens.}{menschen buiten voor de groote}{glazen ruiten staan}\\

\haiku{Zijn huis was immers...{\textquoteright} {\textquoteleft}...{\textquoteright} {\textquoteleft} .....}{vaders huisDag vaderDag}{David ben je daar}\\

\haiku{Weet u wat met een......}{g\`assene weggaat minstens}{driehonderd gulden}\\

\haiku{Het deed hem z\'e\'er in ',.}{t hart dat hij den ouden}{man moest afschepen}\\

\haiku{{\textquoteright} {\textquoteleft}Nou... vooruit dan maar...{\textquoteright}, {\textquoteleft},}{Souget klom haastig de trap}{opEen bestdoener}\\

\haiku{Ze wierp de bundel,}{bloemen in den linkerarm}{stond stil op den weg}\\

\haiku{Als hij zag, dat de,.}{bloemen goed gingen had hij}{w\'eer een ander plan}\\

\haiku{{\textquoteright} Maar in haar hart had.}{ze pret in dat boenen en}{zeepen en schrobben}\\

\haiku{En later, toen je...}{bij ons kwam en toen je het}{marmer schilderde}\\

\haiku{daareven heb ik de,...}{Moedermaagd voor mij gezien}{die jou beschermde}\\

\haiku{Hij ademde vol en,.}{krachtig met een gevoel van}{stage zaligheid}\\

\haiku{Zij greep hem om zijn .....! ..!...}{arm vast keek voor zich uit met}{een verrukt o o}\\

\haiku{{\textquoteleft}Omdat het anders,.}{is alsof er iets in je}{gezicht gebluscht wordt}\\

\haiku{{\textquoteleft}Waarom ben je van, '?}{middag niet gebleven toen}{ik  jet zei}\\

\haiku{Hij begon er over,.}{te denkenvan betrekking}{te veranderen}\\

\haiku{Hij wou niet meer met...{\textquoteright}}{hun wissewassen lastig}{gevallen worden}\\

\haiku{Suzanna liep hem,.}{angstig tegemoet vroeg of}{hij ziek geweest was}\\

\haiku{Hoe zou het dan met...{\textquoteright} {\textquoteleft}!}{den eerlijken handel gaan}{Eerlijke handel}\\

\haiku{Hij stapte op de,.}{twee toe rukte Mantua met}{kracht van Treesje weg}\\

\haiku{O, juist toen dat vuil,...}{tegen zijn hoed aankwam had}{zij zijn oogen gezien}\\

\haiku{Z\'o\'o moesten de oogen van,...}{Jezus geweest zijn toen hij}{voor Pilatus stond}\\

\haiku{Als hij nu eens niet.}{dat accept van Dietrich en}{Cohn betaald had}\\

\haiku{D\`at mocht hij, David,!}{de Leeuw zich wel veroorlooven}{voor al zijn zorgen}\\

\haiku{{\textquoteleft}Vader, ik kom nog...{\textquoteright} {\textquoteleft}...}{eens spreken over de bruiloft}{Spreek er niet meer over}\\

\haiku{Toen kon het niet en...{\textquoteright} {\textquoteleft}?}{nu kan het welEn hoeveel}{kan je dan geven}\\

\haiku{Alle arbeid schoof,.}{hij van zijn hals hij moest voor}{de bruiloft zorgen}\\

\haiku{t Is mooi geweest, ', '...}{t Is mooi geweestt Is}{bliksems mooi geweest}\\

\haiku{Op het bevel van,,...,,...}{\'e\'en tw\'e\'e drie hokus pokus}{pons mijnheer Goudsmit}\\

\haiku{Maar als 't vandaag, '...}{niet was gebeurd wast mij}{morgen overkomen}\\

\haiku{Hebben ze ook met?}{honger thuis gezeten met}{vrouw en kinderen}\\

\haiku{Ze zaten stil in ',.}{t wiebelende rijtuig}{alle drie vermoeid}\\

\haiku{Sommigen zeiden,.}{dat Nauman h\'e\'el rijk was en}{eigen huizen had}\\

\haiku{Hij was taai... hield zich.}{langer boven water dan}{Nauman gedacht had}\\

\haiku{k Heb wat maagzout...{\textquoteright}.}{ingenomen David had}{hem eens aangezien}\\

\haiku{Harde uren had hij, '.}{doorgebrachts nachts wakend}{in zijn kantoortje}\\

\haiku{Zij zou weldra voor.}{een beslissend oogenblik}{in haar leven staan}\\

\haiku{Hij had zich in zijn.}{hart week en tot schreiens toe}{bewogen gevoeld}\\

\haiku{{\textquoteright} Maar hij kon voor 's.}{avonds niet weg uit zijn zaak en}{zoo ontging zij hem}\\

\haiku{Franscli sprak ze en.}{Duitsch als water en zij zong}{als een chanteuse}\\

\haiku{Ondergeschikt moest,,.}{zij worden zij de dochter}{van David de Leeuw}\\

\haiku{Je zei, dat het hier,,!}{goedkoop was maar dat is veel}{te duur veel te duur}\\

\haiku{Straks zullen wij de.}{mooie beelden uit het Grieksche}{bloeitijdperk gaan zien}\\

\haiku{Dat anderen, dat...?}{vreemden zoo maar jou lichaam}{zouden kunnen zien}\\

\haiku{Voelde hij niet, dat?}{dat een beleediging was voor}{hemzelf en voor haar}\\

\haiku{Dan kan ik uw werk, {\textquoteright}.}{w\`el trachten plaatsen had \'e\'en}{kunstkooper gezegd}\\

\haiku{Je mag mij op je,...}{knie\"en danken dat ik je}{d\`at laat verdienen}\\

\haiku{Als 't heelemaal,.}{uit de lucht gegrepen was}{had je er niets aan}\\

\haiku{Hij hoorde dicht bij....}{zich spreken in een vreemde}{taal luisterde toe}\\

\haiku{Een lange, slanke,,...}{man met een grijzende baard}{sprak met een meisje}\\

\haiku{Jongen, jongen, dat,,...{\textquoteright} {\textquoteleft}...}{duurt lang hoor langDen langsten}{tijd heeft het geduurd}\\

\haiku{hem doodsbleek en met.}{behuilde oqgen in den}{winkel te gemoet}\\

\haiku{Mantua Fresco was.}{in de stad geweest met dat}{sjieksie van Vlissingen}\\

\haiku{Als jij je niet wat,,...}{tammer houdt opvreetster gooi}{ik jou de straat op}\\

\haiku{Hij hoorde haar aan,,...}{keek naar haar mooie gestalte}{haar vollen boezem}\\

\haiku{Hij overlegde, was,.}{besluiteloos keek nogmaals}{naar heur gestalte}\\

\haiku{Nu, zeg nou nog 'ris,.}{dat er geen rechtvaardigheid}{in de wereld is}\\

\haiku{Als men 'm nog kwaad,...}{an had gezien had-ie mij}{de zaak uitgezet}\\

\haiku{Zoolang die twee daar,,.}{boven jammerden had hij}{zijn huis gemeden}\\

\haiku{En alle week komt,}{de dokter driemaal zoodat ze}{bij mij niet zoo gauw}\\

\haiku{Toen ze hem zagen,,.}{schenen beiden te schrikken}{weken voor hem uit}\\

\haiku{een paar handen vol.}{van en toen duwde-ie er}{een paar in me mond}\\

\haiku{Altijd die vuile.}{bloedboel rakkere en die}{stinkrommel schrobben}\\

\haiku{op de lange duur.}{zat je de heelen dag je}{maar te vervelen}\\

\haiku{Wil je wel gelooven, '.}{datk nog net zeventien}{centen in huis heb}\\

\haiku{En drink 'n slokkie...{\textquoteright} {\textquoteleft}............}{Een blous een blauwe blous met}{opslagen fijn hoor}\\

\haiku{En wanneer ze ook, '.}{herkend werden dan nog kon}{t hem niets bommen}\\

\haiku{As je een talhout,...}{op m'n brood leit eet ik het}{op voor zoete koek}\\

\haiku{Voor {\textquoteleft}Isra\"el en{\textquoteright}.}{Oranje had-ie hem zoowaar een}{tientje gegeven}\\

\haiku{{\textquoteright} {\textquoteleft}Is 't u bekend, '.}{waar de heer Ricardi op}{t oogenblik is}\\

\haiku{Hij spiegelde zich,.}{in de ruit van een winkel}{die hij voorbijliep}\\

\haiku{en mevrouw De Leeuw,.}{God geve u dat geluk}{bij al uw dochters}\\

\haiku{Wacht... ze zullen zien...}{dat er nog rechtvaardigheid}{in de wereld is}\\

\haiku{{\textquoteright} riep hij, door den toon,.}{den bezoeker uitnoodigend}{binnen te komen}\\

\haiku{Een vuil stukkie hout,.}{waar men negen jaar op krast}{om w\`at te worden}\\

\haiku{Ik heb hier dikwijls,.}{door de straat geloopen dat}{ik gewanhoopt heb}\\

\haiku{Blz. 105 regel 1 {\textquoteleft}{\textquoteright}.}{te lezen achterverteld}{en het woord hij}\\

\subsection{Uit: Twee weken bedelaar}

\haiku{Ik trek een oude,.}{grijze wollen trui aan aan}{den hals uitgescheurd}\\

\haiku{Daarover gaat een vest,,.}{voor een veel dikker persoon}{dan ik ben gemaakt}\\

\haiku{En meteen rinkelt,.}{ze de deur dicht dat de tocht}{koel langs mij heen waait}\\

\haiku{Je bent duur vanavond...{\textquoteright} {\textquoteleft}',,{\textquoteright}.}{t Is f\"ur morgen f fr\"uh}{smeekte ik heesch}\\

\haiku{V\'o\'or 't bed, links in.}{den hoek stond een stoel en ik}{ging mij ontkleeden}\\

\haiku{In het bed van de.}{klagende vrouwestem kwam}{nu ook beweging}\\

\haiku{Hij had zich pas op '.}{t plaatsje gewasschen en}{zijn haar was nog nat}\\

\haiku{En net op die stoel,...}{die ik de vorige week}{heb laten matten}\\

\haiku{Ik nam den cent aan, '.}{stopte hem int kleine}{zakje van mijn jas}\\

\haiku{Aan den weg stond een,.}{bedelaar met een houten}{been een \`echt gebrek}\\

\haiku{Wat dorre bruine,.}{bladeren dreven niet weg}{zoo windstil was het}\\

\haiku{'k Nam mijn kruk en,.}{mijn postpapier strompelde}{weer naar hen terug}\\

\haiku{Vijf f, f, f, fel,...,}{voor een st stuiver met}{v vijf \`e \`e \`e}\\

\haiku{Opende een kleine, '.}{portemonnai\'e gaf het kind}{wat int handje}\\

\haiku{En 'k mot mijn vrouw.}{van de week nog twee kwartjes}{sturen voor de huur}\\

\haiku{Ja, terwijl ik hem,.}{aanzag herinnerde ik}{mij zijn beschrijving}\\

\haiku{Hij nam de doos aan.}{een band traag om den arm en}{ging sluik de deur uit}\\

\haiku{De gelagkamer,.}{was wazig van den rook die}{uit de keuken kwam}\\

\haiku{{\textquoteleft}Ja, 't is mij soms.}{net of die heele boel van}{binnen samentrekt}\\

\haiku{Nou, wij hebben er,.}{met z'n drie\"en een maal aan}{gehad een fijn maal}\\

\haiku{Als-ie 's avonds op,.}{de kamer ligt verpest-ie}{de heele kamer}\\

\haiku{Doch zij verweten.}{elkaar de meest intieme}{bijzonderheden}\\

\haiku{En ik zeg je, as ',.}{jem nog is anraakt trap}{ik je uit mekaar}\\

\haiku{Dan ga je rechtdoor.}{en dan rechts af en dan weer}{recht door langs de tram}\\

\haiku{De juffrouw bleef even,...}{talmen kierde de deur een}{beetje verder open}\\

\haiku{{\textquoteleft}Een cent goedkooper,{\textquoteright} '.}{zeik nogmaals zeer ernstig}{en nadrukkelijk}\\

\haiku{{\textquoteleft}Mijn vader geeft 'r,{\textquoteright}.}{een beetje boter op d'r}{brood lachte het kind}\\

\haiku{{\textquoteright} {\textquoteleft}Ja moeder, genoeg,{\textquoteright} ',.}{mompeldet knaapje nog}{altijd kruiperig}\\

\haiku{{\textquoteright} Jantje keek mij met.}{zijn fletse maar eerlijke}{oogen verwonderd aan}\\

\haiku{Ik ging heel dicht naar,...}{de golven tot ze even mijn}{schoenen bespoelden}\\

\haiku{Maar zij verkwijnen,,.}{langzaam als een plant die te}{weinig water krijgt}\\

\haiku{Kom man, ze poese '.}{ze zelfs morgens om een}{fooi uit te spare}\\

\haiku{Ze het polsies man '.}{asn pijpesteel en ze hoest}{as een simpanzee}\\

\haiku{Nou, en ik telkens,.}{vijf kwartjes voor me loopie}{dat rekende \`an}\\

\haiku{hoor,{\textquoteright} en toen vloekt-ie.}{me uit voor stommerd en de}{smerigste vloeken}\\

\haiku{Je hebt 'm ze toch...{\textquoteright} {\textquoteleft},.}{uit zijn handen genomen}{Nou dat begrijp je}\\

\haiku{Ieder gaat maar naar...{\textquoteright} {\textquoteleft},?}{den IndiesmanNou en wat}{hindert het ze dan}\\

\haiku{Maar deze zocht op,.}{zijn beurt een slachtoffer dat}{Piet natuurlijk werd}\\

\haiku{Maar ik ken 'r dan,,.}{honderden die God danken}{as de kou wegblijft}\\

\haiku{an d'r lijf...{\textquoteright} {\textquoteleft}Nou, die '...{\textquoteright} {\textquoteleft}!}{kenne naart Toevlucht gaan}{Of naar meneertje}\\

\haiku{Daar ben ik voor in ',}{doodsgevaar geweest en daar}{heb ikt gal door}\\

\haiku{Na de soep werden.}{borden met aardappelen}{en vet rondgedeeld}\\

\haiku{Zoodanig en op.}{deze wijze spraken de}{lieden tot twaalf uur}\\

\haiku{{\textquoteleft}Geef me nou toch de,. ' '.}{emmer baask Zalt straks}{zelf wel schoon maken}\\

\haiku{Je hebt je voor den....}{dnder hier in mijn huis niet}{zoo in te zeepen}\\

\haiku{Wie in mijn huis of,.}{in mijn bedden d\`at vindt heeft}{het zelf meegebracht}\\

\haiku{{\textquoteleft}Dank-je Gasman en '.}{hier is een gulden voort}{schoene-poetse}\\

\haiku{As jij d'r onder, '...{\textquoteright} {\textquoteleft}}{leit wil ikr nog wel als}{weduwvrouw hebben}\\

\haiku{{\textquoteright} En hij laat mij een, '.}{portefeuille zien z\'o\'o hoog}{vant bankpapier}\\

\haiku{De menschen kenne,.}{je niet bijhouwe die je}{wat wille geve}\\

\haiku{Vier duiten in 't '.}{zakje en vier duite op}{straat ann arm mensch}\\

\haiku{hier heeft u uw geld,,}{terug as je meent dat ik}{loopen kan Loopen}\\

\haiku{Het vrouwtje was den.}{hoek al om en een eindje}{verder op de brug}\\

\haiku{As u gelooft, dat,.}{ik goed loopen ken moet u}{uw geld maar houden}\\

\haiku{{\textquoteright} Laat ons niet alle,.}{reglementeering alle}{orde verwerpen}\\

\haiku{Ja, Den Haag was mooi,,.}{was warm was innig op dien}{heerlijken herfstdag}\\

\haiku{'k Mot nou ook maar, '.}{hopen datk weer een goeie}{boer onderweg tref}\\

\haiku{'k Had mijn verstand,.}{motte gebruiken toen ik}{bij den ouwe was}\\

\haiku{As de huisbaas komt, '.}{en z'n cente ligge klaar}{dan ist toch goed}\\

\haiku{Hoeveel hadden die,?}{menschen ontvangen hoeveel}{verdiend op \'e\'en dag}\\

\haiku{Vind morgen weer drie,...}{gekken die je een kwartje}{geven voor zoo'n pop}\\

\haiku{Daarom trok ik een,:}{pijnlijk gezicht zette het}{kopje neer en zei}\\

\haiku{{\textquoteright} Zij had het kwartje.}{aangenomen en mij een}{bed aangewezen}\\

\haiku{Ik sloeg de dekens,.}{op stak een lucifer aan}{en bekeek het bed}\\

\haiku{'t Kind pakte de,.}{pop niet aan maar scheeloogde}{schuw naar zijn moeder}\\

\haiku{{\textquoteright} Het knaapje nam de.}{pop aan en ging met pop en}{kous naar zijn moeder}\\

\haiku{{\textquoteleft}Waarom huil je weer,...{\textquoteright} {\textquoteleft},....}{gedrochtIkke ikke heb}{niet gegegeten}\\

\haiku{Zij droeg een fluweel,.}{jaquette een zwart lustren rok}{op breede heupen}\\

\haiku{Hij nam het \'etui aan,,.}{haalde het horloge er}{uit deed de kast open}\\

\haiku{Zij waren moe, schor,,.}{riepen machinaal maar zij}{riepen nog altijd}\\

\haiku{k Had vandaag nog.}{niks gehad dan een stuk oud}{brood van een dienstmeid}\\

\haiku{'k Mot twaalf stuivers,.}{thuis brengen anders laten}{ze me er niet in}\\

\haiku{De knecht is ziek en.}{daarom moet ik vanavond voor}{alles zelf zorgen}\\

\haiku{{\textquoteright} En hij wees op een,:}{agent in uniform die ook heel}{gemoedelijk zei}\\

\haiku{heilig gevoel, dat.}{der persoonlijke vrijheid}{van denken aanrandt}\\

\haiku{{\textquoteleft}Daar mannetje, als,.}{je er dat in hebt zul je}{geen trek meer hebben}\\

\haiku{Ik ben vanmorgen...{\textquoteright} {\textquoteleft}!}{door een agent uit mijn hotel}{gehaaldZijn hotel}\\

\haiku{De cither-speler.}{was er niet meer en ik heb}{hem niet meer gezien}\\

\haiku{Een der mannen stond.}{plotseling op en hield de}{lamp met de hand stil}\\

\haiku{De kachel brandt, zooals.}{je ziet en wij hebben geen}{cent van iemand noodig}\\

\haiku{Deze nacht, besloot,.}{ik zou mijn laatste van mijn}{bedelaarsreis zijn}\\

\haiku{Ik ging nu voor 't,.}{raam staan liet een pop dansen}{voor de kinderen}\\

\haiku{Je hebt meer kans mij,......,...}{te zien dan die dnderhnd}{die je vanmorgen}\\

\haiku{Ik verliet het huis,.}{ging dineeren en tegen half}{elf kwam ik terug}\\

\section{Willem Capel}

\subsection{Uit: 'Gl\"uck auf,' kompeltje! Roman uit het mijnwerkersleven in Nederland}

\haiku{Dat ging zoowat altijd,,,;}{door uur na uur dag en nacht}{winter en zomer}\\

\haiku{Toen Kompeltje aan {\textquoteleft}{\textquoteright},.}{de mijnMaurits dacht schrok hij}{op uit zijn gepeins}\\

\haiku{Daar zaten Vader ';}{en Moeder ook zoos avonds}{plannen te smeden}\\

\haiku{Toch wist hij heel goed,,.}{hoe hij het zeggen moest maar}{hij durfde het niet}\\

\haiku{Hij hoorde zijn vrouw;}{in het achterkamertje}{tegen Sjef bezig}\\

\haiku{In het kleedlokaal,.}{trok Vader al zijn kleeren uit}{zelfs zijn ondergoed}\\

\haiku{De lift zakte weer.}{wat en Vader hoorde het}{debat boven zich}\\

\haiku{Geen enkel accoord is,:}{gelijk maar overal wordt op}{de minuut gewerkt}\\

\haiku{Langzaam stond Vader,;}{op een lichte kreet kon hij}{niet onderdrukken}\\

\haiku{De lucht ziet nu mooi.}{blauw en de koude is niet}{zoo hevig  meer}\\

\haiku{{\textquoteleft}Sla je goed af,{\textquoteright} maant, {\textquoteleft}.}{Vaderdat die rotzooi niet}{mee naar binnen komt}\\

\haiku{het kon best zijn, dat.}{Stephan over een week of wat al}{weer aan het werk mocht}\\

\haiku{hij zag haar rose,;}{directoirtje waarover zij}{gauw haar rokje trok}\\

\haiku{{\textquoteleft}Ik kan nu niet naar,,}{Lutterade met die sneeuw ik}{ga maar na Nieuwjaar}\\

\haiku{de rest lag als een.}{zwarte pyramide in}{het witte landschap}\\

\haiku{Razend snel ging het.}{nu door de straatjes van de}{kolonie naar huis}\\

\haiku{Hij gleed weer met Thea,.}{de helling af voelde zich}{weer heel dicht bij haar}\\

\haiku{Dan kun je van je,.}{kinderen afhangen of}{naar het armbestuur}\\

\haiku{Van de 37 duizend,.}{mijnwerkers zijn er maar 13.500}{georganiseerd}\\

\haiku{De reprimande,.}{die hij in ontvangst heeft te}{nemen is niet malsch}\\

\haiku{De steengang kan nu.}{weer een eindje verder door}{getrokken worden}\\

\haiku{Het puin zal worden.}{opgeruimd en Vader zal}{de bouwen zetten}\\

\haiku{Vader helpt even mee,.}{dan kan hij eerder aan het}{stutten beginnen}\\

\haiku{{\textquoteright} luidde de eerste:}{vraag en het antwoord las de}{pastoor langzaam voor}\\

\haiku{Deze knikte slechts:}{en Kompeltje nam Sjeng eens}{van terzijde op}\\

\haiku{Het was een glad, rond,.}{stuk steen z\'o\'o mooi glad alsof}{het geslepen was}\\

\haiku{Wat is goed beschouwd?}{het werkelijke leven}{van een mijnwerker}\\

\haiku{Vandaag of morgen.}{loopt ie tegen een meid aan}{en trekt het huis uit}\\

\haiku{{\textquoteleft}Ik heb spijt meester,.}{dat ik niet eerder naar U}{toegekomen ben}\\

\haiku{Kompeltje loog, dat,.}{ie zoo weer terug was hij}{moest even naar de grens}\\

\haiku{'t Was bladstil en.}{de rook uit de hooge schoorsteenen}{kroop loodrecht omhoog}\\

\haiku{Achter dat hek was,.}{nu Duitschland maar de grens}{viel Thea erg tegen}\\

\haiku{{\textquoteleft}Een mooi hoofd en glad,.}{geschoren kin Daar steekt des}{mannes schoonheid in}\\

\haiku{Zwaans had Kompeltje,.}{een stille wenk gegeven}{en deze begon}\\

\haiku{Een meidenbaantje,,.}{is het een vak voor iemand}{met een bult barst nou}\\

\haiku{Aan die jas, zat nog,.}{een soort van kap die je over}{je kop kon trekken}\\

\haiku{dat had je er nou,.}{van als je een leerling uit}{dezelfde plaats had}\\

\haiku{Die muizen komen;}{met de takkenbossen en}{het hout de mijn in}\\

\haiku{Hij komt nu bij een, {\textquoteleft}}{verlaten veldgedeelte}{een zoogenaamde}\\

\haiku{En nu, als eerste,.}{Souren die in een ander}{vak gaat Kompeltje}\\

\haiku{Maar waarom mag ik,?}{de mijn dan niet in als U}{het zoo'n mooi vak vindt}\\

\haiku{Het bosch is dan in,.}{eens heel anders een nieuw dak}{en een nieuw vloerkleed}\\

\haiku{{\textquoteleft}Zeg Souren, kun jij?}{je nou nooit losmaken van}{die vervloekte mijn}\\

\haiku{De woorden van Geurts;}{den sleeper kwamen hem weer}{in de gedachten}\\

\haiku{Ik moest maar goed leeren,,.}{zei hij misschien dat het nog}{wel eens te pas kwam}\\

\haiku{Van de mijnwerkers,?}{heb je al een andere}{indruk is het niet}\\

\haiku{{\textquoteright} {\textquoteleft}Maar beste jongen,,!}{je verveelt me absoluut}{niet integendeel}\\

\haiku{Ge kunt toch wel stiekum,,}{een kruis slaan dat niemand het}{ziet als je bang bent}\\

\haiku{Zoo'n wagentje werd.}{dan gekiept en de kolen}{vielen op een zeef}\\

\haiku{Wel wist hij, dat hij,.}{veel last van zijn zenuwen}{had de laatste tijd}\\

\haiku{Maar de jongen was.}{dolgelukkig en daar was}{Vader ook blij om}\\

\haiku{En dan had je nog,:}{die idiote Janus die}{als eenig antwoord wist}\\

\haiku{Bij de uitvaart zelf.}{had half Terwinselen in}{de kerk gezeten}\\

\haiku{Die kerel had nu.}{werkelijk van alles bij}{de mijn meegemaakt}\\

\haiku{En ja, daar stapte,.}{Thea uit er was echter geen}{Kompeltje te zien}\\

\haiku{Het was nu tien over,,.}{tien nog een kwartier dan kon}{Vader binnen zijn}\\

\haiku{een sterke lier stond,.}{opgesteld die de trein naar}{boven zou trekken}\\

\haiku{Daar ging de sleep al,.}{als door een onzichtbare}{kracht opgetrokken}\\

\haiku{{\textquoteleft}Der Kasper{\textquoteright} blijft van brood,).}{en koffie af maar eet nu}{muizen voor zijn straf}\\

\haiku{Hij zou...... ja, hij zou,.}{van alles ze moesten het maar}{aan hem overlaten}\\

\haiku{Wagen na wagen.}{kwam aanrollen en de stoet}{werd hier opgesteld}\\

\haiku{Maar ze aarzelt en,:}{herhaalt alsof ze zich eerst}{nog bezinnen moet}\\

\haiku{Alleen, er was geen,.}{steenkool de jongens werkten}{alleen maar in steen}\\

\haiku{Die Stephan was totaal.}{krankzinnig en werd nog steeds}{in Venray verpleegd}\\

\haiku{{\textquoteleft}Alles drukt op den{\textquoteright},;}{werkman het was een oud maar}{een waar gezegde}\\

\haiku{Verstrooid gaf hij zijn.}{contr\^ole-penning af en}{daalde nu zelf}\\

\haiku{Hij leunde tegen.}{de wand en had wel kunnen}{janken van de pijn}\\

\haiku{Het viel hem tegen,,.}{hij verlangde naar boven}{naar de frissche lucht}\\

\haiku{Dat is een ijzer,{\textquoteright}.}{van zoo lang en zoo dik zoowat}{wees een der mannen}\\

\haiku{Hij rukte zich uit,,.}{zijn verdooving los niet naar}{huis nog niet naar huis}\\

\haiku{Maar het was toch een,.}{schrik voor Moeder geweest toen}{hij zoo voor haar stond}\\

\haiku{Zijn lange Duitsche}{pijp staat nog steeds in de hoek}{bij de linnenkast}\\

\section{S. Carmiggelt}

\subsection{Uit: Allemaal onzin}

\haiku{{\textquoteright} vroeg mijn zoontje, die.}{de woorden allemaal niet}{zo precies weet}\\

\haiku{Hij bemerkte mijn.}{honger naar contact en}{lachte zindelijk}\\

\haiku{Ik                     lig in bed.}{en droom dat een steenbok me}{in het borstbeen bijt}\\

\haiku{dat ik hem nu,                      {\textquoteleft} -!}{onder de uitroepWat gij}{beledigt mijn st\'am}\\

\haiku{We namen elkaar,.}{op als twee worstelaars voor}{de wedstrijd begint}\\

\haiku{Opeens ging de deur.}{open en trad de oude heer}{Kortlever binnen}\\

\haiku{{\textquoteright} riep Kozels verschrikt,,.}{als vreesde hij dat ik mij}{zou                     verdrinken}\\

\haiku{We konden ons dus,.}{tenminste ophangen}{als het tegenviel}\\

\haiku{We zaten er                     , ', '.}{ongemakkelijk maars}{lands wijss lands eer}\\

\haiku{{\textquoteleft}Z\'o, knulletje,{\textquoteright} zeg?}{ik tegen hem en                     raai}{eens wat hij antwoordt}\\

\haiku{Het raadsel van de.}{naamloze voorbijganger}{kwam ter tafel}\\

\haiku{In ieder geval,{\textquoteright}.}{is uw oom een gourmet zei}{Annie inschenkend}\\

\haiku{{\textquoteright} riep de buurman, uit, {\textquoteleft},.}{het raam wijzenddaar met die}{blauwe jas                     aan}\\

\haiku{Koos kijkt nu bepaald,:}{op de staart getrapt maar het}{stemmetje klaagt}\\

\haiku{Ik kan er                     niets, -...}{aan doen maar hij st\'a\'at er weer}{voluit huilend niu}\\

\haiku{{\textquoteleft}Ik kom later nog,,.}{wel eens kerel als je vrouw}{er is of                     zo}\\

\haiku{Ik strompelde naar:}{die helse bel en hoorde}{een meneer roepen}\\

\haiku{Dan wordt het duister.}{over zijn                     minuscule}{problematiek}\\

\haiku{{\textquoteright} roept de bestuurder,,{\textquoteleft}}{die de voornamen bepaald}{uit de mouw schudt}\\

\haiku{Ik verloor opeens.}{mijn linkerschoen en moest er}{even naar                     zoeken}\\

\haiku{Ergens bij de deur}{zag ik vaag hoe hij rukte}{aan de                     ketting}\\

\haiku{wat hij zei sloeg, naast,.}{deze haaibaai als een tang}{op een                     varken}\\

\subsection{Uit: Allemaal onzin}

\haiku{{\textquoteright} vroeg mijn zoontje, die.}{de woorden allemaal niet}{zo precies weet}\\

\haiku{Hij bemerkte mijn.}{honger naar contact en}{lachte zindelijk}\\

\haiku{{\textquoteright} vroeg mijn zoontje, toen.}{we in de schemergrijze}{zijstraat                     liepen}\\

\haiku{Ik                     Hg in bed.}{en droom dat een steenbok me}{in het borstbeen bijt}\\

\haiku{dat ik hem nu,                      {\textquoteleft} -!}{onder de uitroepWat gij}{beledigt mijn st\'am}\\

\haiku{{\textquoteleft}Die jongen                     moet{\textquoteright} {\textquoteleft},{\textquoteright}:}{opstaan ofH\'e vlegel en}{mijn tante zei paars}\\

\haiku{We namen elkaar,.}{op als twee worstelaars voor}{de wedstrijd begint}\\

\haiku{De lesuren werden.}{aan het uitwisselen van}{verhalen besteed}\\

\haiku{Had-ie maar                     niet,.}{moeten duwen toen ik zijn}{stoep bezemde}\\

\haiku{Opeens ging de deur.}{open en trad de oude heer}{Kortlever binnen}\\

\haiku{{\textquoteright} riep Kozels verschrikt,,.}{als vreesde hij dat ik mij}{zou                     verhangen}\\

\haiku{We konden ons dus,.}{tenminste ophangen}{als het tegenviel}\\

\haiku{{\textquoteleft}Z\'o knulletje{\textquoteright}, zeg?}{ik tegen hem en raai}{eens wat hij antwoordt}\\

\haiku{{\textquoteleft}Verdraaid{\textquoteright}, riep hij, {\textquoteleft}dat ',.}{zitm in de krachtfabriek}{als ik                     goed zie}\\

\haiku{Het raadsel van de.}{naamloze voorbijganger}{kwam ter tafel}\\

\haiku{In ieder geval{\textquoteright},.}{is uw oom een gourmet zei}{Annie inschenkend}\\

\haiku{{\textquoteright} riep de buurman, uit, {\textquoteleft},.}{het raam wijzenddaar met die}{blauwe jas                     aan}\\

\haiku{Koos kijkt nu bepaald,:}{op de staart getrapt maar het}{stemmetje klaagt}\\

\haiku{Ik kan er                     niets, -...}{aan doen maar hij st\'a\'at er weer}{voluit huilend nu}\\

\haiku{{\textquoteleft}Ik kom later nog,,.}{wel eens kerel als je vrouw}{er is of                     zo}\\

\haiku{ze duwden er                     ,.}{tegen aan de andere}{kant huilend van angst}\\

\haiku{Ik strompelde naar:}{die helse bel en hoorde}{een meneer roepen}\\

\haiku{Dan wordt het duister.}{over zijn                     minuscule}{problematiek}\\

\haiku{we hadden net zo.}{goed een huwelijksdatum}{kunnen vaststellen}\\

\haiku{Ik verloor opeens.}{mijn linkerschoen                         en moest}{er even naar zoeken}\\

\haiku{wat hij zei sloeg, naast,.}{deze haaibaai als een tang}{op een                     varken}\\

\haiku{En ge moet goed                     ,.}{begrijpen dat ik er geen}{geld in steken wil}\\

\subsection{Uit: Drie in een}

\haiku{Vreemde kostgangers.}{I Het antiekwinkeltje}{keek uit op een gracht}\\

\haiku{Maar als ik nou 't,.}{lirium heb dan beweegt}{dat                         mannetje}\\

\haiku{{\textquoteright} {\textquoteleft}Hij is gesneuveld,{\textquoteright}.}{bij de slag in de Java}{Zee antwoordde hij}\\

\haiku{Op een dag was zij.}{tenminste verdwenen en}{keerde nimmer weer}\\

\haiku{Gisteren, op het,.}{station van die wijze}{stad zag ik haar werk}\\

\haiku{De keukendeur week:{\textquoteleft}.}{en daar stond die goede Piet}{en zei                    Hallo}\\

\haiku{{\textquoteleft}Ja, daar moest                     je.}{op school altijd aan vragen}{als je iets kwijt was}\\

\haiku{Maar misschien vraagt ze.}{het nog steeds stilletjes aan}{Antonius}\\

\haiku{Toen hij de                     kroeg,:}{binnentrad zei een oude}{man aan de tapkast}\\

\haiku{Hij hield van moppen.}{vertellen en mensen voor}{de gek houden}\\

\haiku{{\textquoteright} Tante Fiet deed dat,.}{onmiddellijk                     gewoon}{d\'o\'orconverserend}\\

\haiku{Ik condoleer u,}{met het grote verlies dat}{u heeft getroffen}\\

\haiku{{\textquoteright} Onder het praten.}{was ze teruggelopen}{naar de huiskamer}\\

\haiku{Ze wonen weer in,.}{hetzelfde huis waar ze in}{1940 uitgevlucht zijn}\\

\haiku{Dat is een kwestie,{\textquoteright}.}{van appreciatie zei}{het rinkelmens koel}\\

\haiku{{\textquoteright} Zij legde het stuk.}{op tafel en schroefde de}{dop van haar vulpen}\\

\haiku{antieke borden -.}{waren                         het bij mij dan}{lachten ze je uit}\\

\haiku{Maar op een middag,,.}{om drie                     minuten voor}{twee stond ik voor school}\\

\haiku{Er stond al een man-, -.}{kort breed en grijzend en}{hij rookte een pijp}\\

\haiku{{\textquotedblleft}Een, twee, drie in                     ,,.}{Godsnaam zo vlak onder de}{kust dat doe ik niet}\\

\haiku{Toen Hij was niet eens,.}{zo ingrijpend veranderd}{in al die jaren}\\

\haiku{Hij liet zijn ogen wat,.}{groter worden en schudde}{het hoofd meewarig}\\

\haiku{Hij stak met de                     :}{peuk een nieuwe sigaret}{op en sprak hardop}\\

\haiku{Ik denk altijd, ik.}{hoop dat ze het later}{voor mij net zo doen}\\

\haiku{Nou weet je, wij die,.}{met handel op                     straat staan}{wij worden getrapt}\\

\haiku{De trap naar boven.}{kon oom alleen bereiken}{via de winkel}\\

\haiku{Ze waren lang en.}{in een handschrift dat zich}{haastig voortrepte}\\

\haiku{Er stond geen aanhef.}{boven en ze vielen met}{de deur in huis}\\

\haiku{{\textquoteleft}Omdat-ie met,.}{iedereen ruzie maakte}{had-ie maar \'e\'en vrind}\\

\haiku{Geen moeite, hoor. 't,.}{Lukt prima want ik heb}{hier een goed stekkie}\\

\haiku{{\textquoteright} {\textquoteleft}Eh... wacht even, lieve,.}{schat ik moet de radio wat}{zachter zetten}\\

\haiku{'t Was of Jeroen.}{Bosch zich                     even had bediend}{van een kinderhand}\\

\haiku{{\textquoteleft}Wie gedoemd is te,.}{verdrinken verdrinkt in}{een lepel water}\\

\haiku{op hem hebben we,.}{ons verkeken tenzij hij}{rijp is voor ontslag}\\

\haiku{ik zat namelijk.}{in een helikopter en}{vloog                         er overheen}\\

\haiku{Hij maakt 'n                         {\textquoteleft}nou{\textquoteright}:}{ja gebaar en zegt in zijn}{zorgvuldig Engels}\\

\haiku{{\textquoteright} Naast zijn bureau in.}{het redactievertrek hing}{een groot stuk karton}\\

\haiku{De helft, maar dan een,.}{klein stukkie                         meer een half}{vingertje zowat}\\

\haiku{Over een jongetje.}{dat aan het zwerven ging met}{een bedelaar}\\

\haiku{II Tegen zessen,,.}{stapte ik op het Damrak}{in de tram naar huis}\\

\haiku{{\textquoteright} Ook het weglaten.}{van de lidwoorden hoorde}{helemaal bij hem}\\

\haiku{Toen hij                     alles,.}{had doorgeslikt ademde hij}{zwaar door de neus}\\

\haiku{Je krijgt er wel het,.}{lendewater                     van maar}{je blijft bij de tijd}\\

\haiku{Die vrouw probeerde, '.}{hem af te                     weren maar}{ze konm niet baas}\\

\haiku{Als ze jou op straat,.}{door je porum steken loopt}{iedereen                     door}\\

\haiku{Zo om 'n uur of.}{tien rommel ik een beetje}{aan                     m'n ontbijt}\\

\haiku{Dat wil zeggen - ik,.}{heb iets met een vrouwtje}{van eenendertig}\\

\haiku{Dat het ook minder.}{stijlvol kan ervoeren wij}{onlangs in Parijs}\\

\haiku{Bovendien is haar.}{leven ook op een ander}{niveau veranderd}\\

\haiku{{\textquoteleft}Ik ben heel jong ter.}{wereld                         gekomen in}{een heel oude tijd}\\

\haiku{Maar mij sprak het wel,.}{aan vooral omdat ik zo}{iets nooit zou durven}\\

\haiku{Veel  mensen                     .}{hebben op die leeftijd een}{leeg soort somberheid}\\

\haiku{Maar het hielp wel, want.}{hij dronk eindelijk                         uit}{en verliet de bar}\\

\haiku{Ziekten, die je zou,.}{kunnen krijgen spelen er}{een grote rol in}\\

\haiku{{\textquoteright} De naam luidde bij {\textquoteleft},{\textquoteright}.}{mij geen bel.De beroemde}{humorist riep ze}\\

\haiku{{\textquoteleft}Ach, ach, wat heb ik.}{vroeger om hem gelachen}{als hij optrad}\\

\haiku{Ik geloof dat het.}{beter uit zou komen op}{die andere muur}\\

\haiku{Maar hij raapte al:}{zijn krachten bijeen en}{wist uit te brengen}\\

\haiku{Hij kijkt uit op een.}{ambitieus grondwerk van}{de gemeente}\\

\haiku{Ze voeden                     zich.}{met alles wat in staat van}{ontbinding verkeert}\\

\haiku{{\textquoteleft}Elke dag warme{\textquoteright}.}{vis opeens een andere}{gevoelsinhoud}\\

\haiku{Helemaal op d'r,.}{eentje waarschijnlijk werd}{ze gemainteneerd}\\

\haiku{{\textquoteright} Hij boog zijn hoofd nu.}{dicht in mijn richting en de}{dranklucht werd sterker}\\

\haiku{Zijn regenjas week.}{in de hals en zijn                     hoed}{rustte op zijn oren}\\

\haiku{Ze gaf me een arm.}{en we liepen                     samen}{door de motregen}\\

\haiku{We stappen in zijn:}{antiquarisch voertuig}{en Mohammed zegt}\\

\haiku{{\textquoteleft}Lampie, je moet uit{\textquoteright},}{maar dat helpt niet                     want u}{weet hoe halsstarrig}\\

\haiku{{\textquoteright} Maar als ik haar op:}{wil tillen protesteert}{ze heftig en roept}\\

\haiku{Pas later heb ik.}{beseft dat hij                     te veel}{bier gedronken had}\\

\haiku{Men is heden ten.}{dage monkelend en}{onderhuids geleerd}\\

\haiku{De dame was in.}{haar                     ganse leven nog}{nooit wedersproken}\\

\haiku{{\textquoteleft}Wel, uit hun manier.}{van reageren op het}{voetgangerslicht}\\

\haiku{{\textquoteright} Tenminste bijna,.}{want hij was bezig de a}{te                         voltooien}\\

\haiku{Ik zei zo zacht dat:{\textquoteleft}.}{ik het zelf amper horen}{kon                    Moordenaars}\\

\haiku{Kerels die ouder,.}{waren dan hij trouwden mooie}{jonge                     meisjes}\\

\haiku{Toch                     is dat niet.}{de enige reden waarom}{ik hier buiten zit}\\

\haiku{Ik had het graag voor,.}{hem gedaan want                     hij is}{een aardige man}\\

\haiku{{\textquoteleft}En geloven die?}{acht                     lieve kindertjes}{in Jezus Christus}\\

\haiku{De chauffeur houdt met.}{zijn nog vrije hand de                         deur}{voor de jongen open}\\

\haiku{In mijn jeugd was Nol.}{veel ouder dan ik en nu}{maar zeven jaar}\\

\haiku{Ze had veel met hem.}{uitgestaan en ze was het}{niet                     vergeten}\\

\haiku{Dat is die vrouw bij.}{wie ze                     op een kamer}{woont komen zeggen}\\

\haiku{{\textquoteleft}Paraplu in de,{\textquoteright},.}{bak beveelt ze wijzend naar}{een vaas naast de deur}\\

\haiku{Het woord {\textquoteleft}vreemdeling{\textquoteright}.}{staat met koeieletters op}{mij                         geschilderd}\\

\haiku{{\textquoteright} Nadat ik het hoofd,:}{schudde keek ze zeer ernstig}{naar de kat en zei}\\

\haiku{En eigenlijk kan '.}{t me niet                     verdommen}{wat Joost Jan uitvoert}\\

\haiku{'t Is een corvee, ',.}{met vlammende flensjes aan}{t slot je weet wel}\\

\haiku{{\textquoteleft}Dat flauwe grapje.}{over meneer Bok had ik niet}{moeten                     maken}\\

\haiku{Gelachen heb ik,{\textquoteright} {\textquoteleft},.}{w\'el zegt de man naast me.Niet}{hier in Dordrecht hoor}\\

\haiku{Hij hoorde niet bij,.}{de reiniging maar bij zijn}{eigen creatie}\\

\haiku{Ik zag haar haarscherp,.}{voor me zoals ze er in}{mijn                     jeugd uitzag}\\

\haiku{Uit zijn binnenzak.}{haalde hij een plat flesje}{en nam een                     slok}\\

\haiku{Daar ik aannam dat,:}{het geen Abdij-siroop}{bevatte zei ik}\\

\haiku{Maar weet je wanneer?}{ik zin kreeg om die rails in}{mekaar te zetten}\\

\haiku{Er wordt op gewacht.}{door een jong stelletje dat}{al ondertrouwd is}\\

\haiku{En ik plakte de.}{antwoordenvelop van de}{Raad van Arbeid dicht}\\

\subsection{Uit: Dwaasheden}

\haiku{{\textquoteleft}Maar ik kan toch net!}{zo goed ergens anders}{telefoneren}\\

\haiku{Hij leek op een ets,.}{van Paul Klee zo scheef en}{gebarsten was-ie}\\

\haiku{Haar rechtschapenheid.}{had haar inmiddels niet voor}{ongeluk behoed}\\

\haiku{Het uit elkaar                     .}{nemen van voertuigen is}{een hartstocht van hem}\\

\haiku{Hij bewoonde een.}{portiekhuis met vochtplekken}{en een luchtje}\\

\haiku{De knaap zag mij even.}{aan en boog zich toen weer}{over zijn gefrutsel}\\

\haiku{Kleine tragedie;}{In de grote zaal speelt een}{amateur-blaaskorps}\\

\haiku{Ik legde de haak.}{op het toestel en bleef nog}{even staan kijken}\\

\haiku{Een jongetje, dat,.}{juist voorbijging stortte van}{zijn autopedje}\\

\haiku{Ik acteer de                     .}{houdingloze echtgenoot}{en raak op de gang}\\

\haiku{Als ik terugkom,.}{met de zakjes krijgen we}{hetzelfde nog eens}\\

\haiku{De man merkte dat.}{er helemaal geen triomf}{in haar stem                     was}\\

\haiku{Was het de regen,?}{die mij gistermiddag bij}{hem deed aankloppen}\\

\haiku{Toen ik binnentrad,.}{zat hij te lezen                         op}{zijn pantoffeltjes}\\

\haiku{Zoveel bedroeg ook.}{mijn                     kapitaal en ik}{was nog niet thuis}\\

\haiku{Toen ik klein was, kwam.}{de heer Nieuwkerk vaak bij mijn}{ouders over de vloer}\\

\haiku{Er kwam water aan,.}{te pas hij  bibberde}{tegen                     het glas}\\

\haiku{Mijn vrouw logeerde.}{namelijk bij haar moeder}{in de provincie}\\

\haiku{Geen nood, u komt hem.}{nog wel eens tegen op een}{donker weggetje}\\

\haiku{Eindelijk waren:}{we het er dan met elkaar}{over eens geworden}\\

\haiku{Ze droegen blauwe.}{kielen en vroegen of}{het hier moest wezen}\\

\haiku{{\textquoteleft}IJs,{\textquoteright} waarop hij                     .}{wantrouwig informeerde}{wat zoiets kostte}\\

\haiku{Die des anderen:}{daags waarschijnlijk tegen een}{kennis heeft gezegd}\\

\haiku{Zeg maar gerust dat}{we geen huur meer betalen}{als die kerel}\\

\subsection{Uit: Dwaasheden}

\haiku{{\textquoteleft}Maar ik kan toch net!}{zo goed ergens anders}{telefoneren}\\

\haiku{Hij leek op een ets,.}{van Paul Klee zo scheef en}{gebarsten was-ie}\\

\haiku{Haar rechtschapenheid.}{had haar inmiddels niet voor}{ongeluk behoed}\\

\haiku{Het uit elkaar                     .}{nemen van voertuigen is}{een hartstocht van hem}\\

\haiku{Hij bewoonde een.}{portiekhuis met vochtplekken}{en een luchtje}\\

\haiku{De knaap zag mij even.}{aan en boog zich toen weer}{over zijn gefrutsel}\\

\haiku{Kleine tragedie;}{In de grote zaal speelt een}{amateur-blaaskorps}\\

\haiku{Ik legde de haak.}{op het toestel en bleef nog}{even staan kijken}\\

\haiku{Een jongetje, dat,.}{juist voorbijging stortte van}{zijn autopedje}\\

\haiku{Ik acteer de                     .}{houdingloze echtgenoot}{en raak op de gang}\\

\haiku{Als ik terugkom,.}{met de zakjes krijgen we}{hetzelfde nog eens}\\

\haiku{De man merkte dat.}{er helemaal geen triomf}{in haar stem                     was}\\

\haiku{Was het de regen,?}{die mij gistermiddag bij}{hem deed aankloppen}\\

\haiku{Toen ik binnentrad,.}{zat hij te lezen                         op}{zijn pantoffeltjes}\\

\haiku{Zoveel bedroeg ook.}{mijn                     kapitaal en ik}{was nog niet thuis}\\

\haiku{Toen ik klein was, kwam.}{de heer Nieuwkerk vaak bij mijn}{ouders over de vloer}\\

\haiku{Er kwam water aan,.}{te pas hij  bibberde}{tegen                     het glas}\\

\haiku{Mijn vrouw logeerde.}{namelijk bij haar moeder}{in de provincie}\\

\haiku{Geen nood, u komt hem.}{nog wel eens tegen op een}{donker weggetje}\\

\haiku{Eindelijk waren:}{we het er dan met elkaar}{over eens geworden}\\

\haiku{Ze droegen blauwe.}{kielen en vroegen of}{het hier moest wezen}\\

\haiku{{\textquoteleft}IJs,{\textquoteright} waarop hij                     .}{wantrouwig informeerde}{wat zoiets kostte}\\

\haiku{Die des anderen:}{daags waarschijnlijk tegen een}{kennis heeft gezegd}\\

\haiku{Zeg maar gerust dat}{we geen huur meer betalen}{als die kerel}\\

\subsection{Uit: Een Hollander in Parijs}

\haiku{{\textquoteleft}Catch{\textquoteright} heet                     dit uit.}{Amerika overgewaaide}{divertissement}\\

\haiku{leest de                     vroegste,.}{editie                                              van de krant die}{net is uitgevent}\\

\haiku{Kijk eens wat een Kitsch{\textquoteright}.}{Op de bank tegenover me}{zaten drie priesters}\\

\haiku{Maar als Ernest een,:}{vader                     met zijn zoon ziet}{wandelen denkt hij}\\

\haiku{{\textquoteleft}Ze behoort aan een,{\textquoteright}.}{andere wereld luidt}{zijn motivering}\\

\haiku{Dan draait                     hij zich.}{gedesinteresseerd om}{en slentert weer weg}\\

\haiku{{\textquoteright} Omdat niemand geld,.}{biedt ontrolt het mannetje}{nu ook de tweede}\\

\subsection{Uit: Een Hollander in Parijs}

\haiku{{\textquoteleft}Catch{\textquoteright} heet                     dit uit.}{Amerika overgewaaide}{divertissement}\\

\haiku{Met de nuttiging.}{van het ooft gingen zowat}{tien minuten heen}\\

\haiku{Maar als Ernest een,:}{vader                     met zijn zoon ziet}{wandelen denkt hij}\\

\haiku{Dan draait                     hij zich.}{gedesinteresseerd om}{en slentert weer weg}\\

\haiku{Dat bevalt me. Men.}{amuseert zich en is toch in}{goed                     gezelschap}\\

\haiku{{\textquoteleft}Als er nou maar niet.}{zo'n lelijk oud vrouwtje in}{komt te                         wonen}\\

\haiku{Zolang de                         mens,.}{een beest is dient het beest de}{mens tot voedsel}\\

\haiku{Of u en ik dat -.}{ooit zullen leren ik}{twijfel eraan}\\

\haiku{Om het te breken.}{houd ik hem onhandig het}{doosje weer                     voor}\\

\haiku{Dit                         rijtuig is{\textquoteright}, {\textquoteleft}?}{niet gemakkelijk \'ofZiet}{gij die regenboog}\\

\haiku{Voor de deur van het{\textquoteleft}!}{reisbureau schalt ons dankbaar}{Hiep hiep hoera}\\

\subsection{Uit: Ik lieg de waarheid. De beste Kronkels}

\haiku{De beste Kronkels}{Editie Sylvia Witteman}{Colofon}\\

\haiku{De man merkte dat.}{er helemaal geen triomf}{in haar stem                     was}\\

\haiku{we hadden net zo.}{goed een huwelijksdatum}{kunnen vaststellen}\\

\haiku{En ze was meteen,.}{geschrokken omdat                     haar}{dochter al op was}\\

\haiku{{\textquoteleft}O, uitstekend, dank{\textquoteright}?}{u. Want waarom zou het zo'n}{man niet                     goed gaan}\\

\haiku{{\textquoteright} De leraar stapte,.}{met een klein zwart wichtje aan}{de arm op hen af}\\

\haiku{Hij zag de leraar:}{met zijn jas aan langskomen}{en hoorde hem gen}\\

\haiku{Maar net zoveel thee,,,,.}{als je wou                     drie vier vijf}{koppen dat gaf niet}\\

\haiku{De deur ging open en.}{de zwarte kelner stond}{weer op de drempel}\\

\haiku{Ze snelde weer naar,.}{de deur kennelijk op weg}{naar duizend plichten}\\

\haiku{{\textquoteright}        Grut Om zeven '}{uurs avonds luiden ze voor}{het vakantiehuis}\\

\haiku{Want dat is toch geen -?}{manier van doen iemand zijn}{tekst afnemen}\\

\haiku{{\textquoteright} {\textquoteleft}Maar als ik eruit,{\textquoteright}.}{ga moet ik op het koude}{zeil wierp ik tegen}\\

\haiku{Bijna                     plechtig.}{stond ik op en ging er op}{blote voeten heen}\\

\haiku{Plotseling sprong er:}{een bazige jongen}{van de fiets en riep}\\

\haiku{'t Is net als 't,.}{schippersvak je moet erin}{geboren worden}\\

\haiku{{\textquoteright} {\textquoteleft}Dat is waar, wij ben',{\textquoteright}.}{anders dan de mensen hier}{gaf de man toe}\\

\haiku{Voor ze opstapte,.}{probeerde ze de lamp maar}{die gaf geen                     licht}\\

\haiku{Zolang hij dus niets,.}{zegt bestaat de kans dat hij}{de                     waarheid kent}\\

\haiku{{\textquoteleft}O ja, meneer Van -,,.}{Driel                    gaat u zitten neemt}{u plaats alstublieft}\\

\haiku{De angst was over, maar.}{de wilde blijdschap keerde}{toch niet terug}\\

\haiku{Ik heb                     eens in...{\textquoteright} {\textquoteleft},,}{een boek gelezenChurchill}{die bijna verzoop}\\

\haiku{Dit embleem duidde.}{onze muzikale}{bedoelingen aan}\\

\haiku{En die andere,,.}{die dikke dat is een}{halve loodgieter}\\

\haiku{Nou ja, je moet                     ,.}{er niet over praten hoor maar}{die laat ik doodgaan}\\

\haiku{Het groen werd moe en.}{begon toen aan de punten}{te                     vergelen}\\

\haiku{{\textquoteright} Toen er gezoend en,.}{geknuffeld was streek men neer}{in de voorkamer}\\

\haiku{Thuis,{\textquoteright} antwoordde ik,?}{want leerde moeder niet dat}{jokken zonde is}\\

\haiku{Misschien is het wel,....}{op dat beetje talent dacht}{hij vol zelfbeklag}\\

\haiku{Ze groetten stijf toen.}{hij binnenkwam en bij het}{raam ging                     zitten}\\

\haiku{De dichter stond op,.}{en liep het caf\'e door dat}{lachje om zijn mond}\\

\haiku{{\textquoteleft}Ik mag toch zeker? '}{wel vragen of het al}{afgelopen is}\\

\haiku{Harry - ik droom nog.}{wel eens van hem als ik wat}{zwaar getafeld heb}\\

\haiku{{\textquoteleft}Die jongen die ze,,,.}{nou heeft och hij is goed}{maar sukkelachtig}\\

\haiku{Hij hield van moppen.}{vertellen en mensen voor}{de gek houden}\\

\haiku{Ze gingen zwijgend,,.}{zitten twee kleine oude}{verknochte mensen}\\

\haiku{En ofschoon hij nog,:}{steeds dacht aan de dood zei}{hij precies op tijd}\\

\haiku{H\`e, wat doe je nou,,{\textquoteright}.}{je handen aan je goeie broek}{afvegen zei Fie}\\

\haiku{Dan                     naar het blauw.}{worden en het naar buiten}{treden van de tong}\\

\haiku{Ik weet niet precies,.}{waarom het is maar                     ze}{vinden het niet goed}\\

\haiku{Ja, dat kan. 't Doet, '.}{geen pijn wantt gaat meteen}{uit in je mond}\\

\haiku{De man ging op het.}{bed liggen en liet de brief}{op de grond glijden}\\

\haiku{Er was                     geen geld.}{voor een echte acteur en}{toen mocht ik het doen}\\

\haiku{Het laatste wat hij,.}{zich herinnerde was uit}{zijn kindertijd}\\

\haiku{Hij probeert daar op.}{zijn schuchtere wijze iets}{tegen                     te doen}\\

\haiku{Hij neuriet liedjes.}{uit zijn                     jeugd en vindt dat}{hij goed bij stem is}\\

\haiku{Vader Ouwe Jan.}{kende ik al lang voordat}{hij in het huis zat}\\

\haiku{Vroeger zag ik hem,.}{vrijwel                     dagelijks in}{een kleine buurtkroeg}\\

\haiku{Ze zijn 't hem net.}{een halfuur geleden}{komen vertellen}\\

\haiku{{\textquoteleft}Ik heb het eraf,{\textquoteright}, {\textquoteleft}.}{geveegd zei hijomdat je}{niet in je zaak was}\\

\haiku{{\textquoteleft}Ik dacht dat je 't,{\textquoteright}.}{eraf geveegd had zei hij}{met geknepen stem}\\

\haiku{Hij is iets in de.}{haven en dat kun je wel}{aan hem zien ook}\\

\haiku{Ze heeft nooit iets in.}{haar                     blik van zo'n vrouw die}{haar man komt halen}\\

\haiku{En vandaag is Joop.}{verkwikt en tevreden aan}{het werk gegaan}\\

\haiku{Wie een penny voor,.}{zijn gedachten gaf zou zich}{bekocht voelen}\\

\haiku{Het jongetje had.}{zijn mond leeg en keek enigszins}{kritisch naar de man}\\

\haiku{{\textquoteright} De man, die juist een,:}{ochtendblad ontvouwde schrok}{een beetje en sprak}\\

\haiku{Hij liep weer naar de,:}{kast keerde met het boek bij}{mij terug en zei}\\

\haiku{Op die manier was.}{het dus                         mogelijk de}{mens uit te rekken}\\

\haiku{of ze, als ik het,.}{vertelde zou begrijpen}{wat ik hier                         most}\\

\haiku{Het wijfje bleef nog,.}{door de kier loeren wat de}{zaak compliceerde}\\

\haiku{{\textquoteleft}Piet, ken je me nog,.}{we hebben samen onder}{dienst                         gelegen}\\

\haiku{Ik draaide mij om.}{en daalde in zeer snel}{tempo de trap af}\\

\haiku{Haar doezelige ' - -.}{genotsogen zagent hem}{niet zonder haat doen}\\

\haiku{{\textquoteright}        Duiven In de.}{buurt van mijn Londens hotel}{bevindt zich een parkje}\\

\haiku{Zijn rechterpoot was,.}{normaal maar met de linker}{trok hij                     hevig}\\

\haiku{Maar dat brak ze dan,.}{lelijk op want nu was}{al dat brood voor hem}\\

\haiku{{\textquotedblleft}Als je het doet, dan.}{zal                     het altijd een kloof}{tussen ons blijven}\\

\haiku{{\textquoteleft}Wilt u tegen uw?}{man                     zeggen dat ik niet}{met hem kan trouwen}\\

\haiku{{\textquoteleft}Legt u het daar maar,,{\textquoteright},.}{neer juffrouw zei hij wijzend}{op een tafeltje}\\

\haiku{Ik had het gevoel.}{bij haar voor een klein examen}{te zijn                     geslaagd}\\

\haiku{Een eindje verder.}{ging ik aan een tafeltje}{bij het raam zitten}\\

\haiku{Er stond nu van die.}{miezerige nieuwbouw}{uit een krappe beurs}\\

\haiku{{\textquoteright} En dat was in die.}{dagen een synoniem voor}{zwarte reactie}\\

\haiku{Ze hebben 't te,...{\textquoteright} {\textquoteleft},{\textquoteright}.}{goed bij moeder zekerGunst}{zei de dominee}\\

\haiku{{\textquoteleft}Ik wou zo'n lampje,{\textquoteright}.}{van zestien gulden uit de}{etalage zei ik}\\

\haiku{Nou, die hoeven ze.}{ook zijn ogen alleen nog maar}{toe te                     drukken}\\

\haiku{Allemaal van                         ,.}{die haastige smurrie uit}{de idee\"enwinkel}\\

\haiku{Hij had geen talent,.}{voor geluk zoals Max Nord}{het eens uitdrukte}\\

\haiku{Ik heb laatst een                         .}{artikel gelezen over}{alcoholisten}\\

\haiku{Want bijna al die.}{winkeliertjes waren}{arm en neringziek}\\

\haiku{Ik had altijd een,.}{stukkie ijzer bij me}{op zaterdagavond}\\

\haiku{{\textquoteright} Fransje kwam terug.}{met de zak patat en ging}{op een kruk zitten}\\

\haiku{Hij was er nog niet.}{toen ik Jan de soldaat moest}{spelen in Indi\"e}\\

\haiku{En jij zat                     ook.}{nog in de verrekijker}{van je ouwe heer}\\

\haiku{Iets van Chopin -.}{of van zijn buurman ook}{geen vrolijke broek}\\

\haiku{Nu glimlachte ze.}{tegen me en kneep de ogen}{even helemaal dicht}\\

\haiku{De stem zweeg dan, maar,.}{de geur                     verplaatste zich en}{dat maakte veel goed}\\

\haiku{De oude was het,.}{vergeten maar knikte om}{eraf te wezen}\\

\haiku{{\textquoteright}        Bezoek Precies.}{om acht uur in de ochtend}{werd ik klaarwakker}\\

\subsection{Uit: Kroeglopen 2}

\haiku{Vader Ouwe Jan.}{kende ik al lang voordat}{hij in het huis zat}\\

\haiku{Vroeger zag ik hem,.}{vrijwel                     dagelijks in}{een kleine buurtkroeg}\\

\haiku{Ze zijn 't hem net.}{een half uur                     geleden}{komen vertellen}\\

\haiku{We gaan vanmiddag,.}{even naar Henk en Marie want}{Jopie is jarig}\\

\haiku{Ze staat schichtig op,:}{gaat bij de beschonken man}{zitten en vraagt}\\

\haiku{Toen ik de ronde,.}{net volgeschonken had ging}{de telefoon weer}\\

\haiku{Terwijl ik de fles,.}{weer hief ging de telefoon}{ten derden male}\\

\haiku{De man achter de.}{bierkraan haatte mij op het}{eerste                     geziect}\\

\haiku{Hij vertelde mij,,.}{in ons stamcaf\'e het eerst}{over Rastelli}\\

\haiku{Hij nam een ferme.}{teug en begon toen luid in}{zich zelf te praten}\\

\haiku{De een zei iets, langs.}{zijn neus weg en de ander}{bulderde daar om}\\

\haiku{{\textquoteleft}Ik dacht dat je 't,{\textquoteright}.}{eraf geveegd had zei hij}{met geknepen stem}\\

\haiku{{\textquoteright}        De vakantie.}{van Joop De vakantie van}{Joop zit er weer op}\\

\haiku{Hij is iets in de.}{haven en dat kun je wel}{aan hem zien ook}\\

\haiku{Ze heeft nooit iets in.}{haar                     blik van zo'n vrouw die}{haar man komt halen}\\

\haiku{En vandaag is Joop.}{verkwikt en tevreden aan}{het werk gegaan}\\

\haiku{Als ik er om vier.}{uur vijf binnen treed ben ik}{er de enige klant}\\

\haiku{Met een schuwe blik:}{naar                     de stoelen en de}{tafeltjes vroeg hij}\\

\haiku{{\textquoteleft}Zie je, ik ben niet.}{gewend om in zulke}{zaken te komen}\\

\haiku{Alleen - als kennis.}{is hij bepaald een                     wat}{onrustig bezit}\\

\haiku{Van                     buiten ziet.}{zo'n etablissementje er}{nog wel aardig uit}\\

\haiku{Wie een penny voor,.}{zijn gedachten gaf zou zich}{bekocht voelen}\\

\haiku{Hij nam een slok die.}{zijn bierglas ledigde en}{verlangde een nieuw}\\

\haiku{{\textquoteright} De man keerde door.}{het volle caf\'e naar het}{tafeltje terug}\\

\haiku{En we komen in.}{Amsterdam                     en dat geld}{brandde in me zak}\\

\subsection{Uit: Met de neus in de boeken}

\haiku{{\textquoteright} Pas later bleek dat.}{hij nooit eerder op een fiets}{gezeten                     had}\\

\haiku{{\textquoteleft}Hij mag dan haar op,,.}{z'n borst hebben maar zusters}{dat heeft Lassie \'o\'ok}\\

\haiku{{\textquoteleft}Als ik haar de                     ,?}{wol geef kan ze er dan voor}{mij \'o\'ok een maken}\\

\haiku{Ik was altijd blij.}{als ik de                     deur achter}{zo'n vent kon dichtdoen}\\

\haiku{Hij verhief zich voor,.}{het open raam sidderend over}{zijn hele lichaam}\\

\haiku{De natuur bracht hem.}{altijd zijn kindertijd}{in herinnering}\\

\haiku{En                     nou gaan we.}{uw mooie wasmachine uit}{de auto halen}\\

\haiku{De jongedame.}{in kwestie is inderdaad}{een Nederlandse}\\

\haiku{{\textquoteright} zei ik tegen mijn, {\textquoteleft}.}{vrouwmaar ik                     maak bezwaar}{tegen zijn aanhef}\\

\haiku{Want wat was er aan,?}{de hand met het boek dat mijn}{vrouw uit de kast trok}\\

\haiku{Er was \'e\'en zo'n klein,,.}{slecht ventje bij en                     die}{sloeg toch z\'o hard h\`e}\\

\haiku{Anders zouden ze,}{denken dat ik helemaal}{hoogmoedswaanzin heb}\\

\haiku{En misschien een                     .}{bijdrage tot het probleem}{van je depressies}\\

\haiku{Vandaar mijn vrees,                     .}{toen ik je zag staan voor het}{Citytheater}\\

\haiku{Particulier{\textquoteright} is -.}{het wel vervelend voor je}{toegegeven}\\

\haiku{D. had een afspraak,.}{met onze oude vriend}{de schrijver Jef Last}\\

\haiku{Hij klaagde over zijn.}{vreselijk leven in het}{bejaardenhuis}\\

\haiku{Toen begaf hij zich -.}{naar de toiletten al}{voor de derde keer}\\

\haiku{En schrijf vooral niet.}{terug als je het uit je}{tenen halen moet}\\

\haiku{{\textquoteleft}God, als ik alleen,.}{kom                     te staan laat me dan}{eenzaam kunnen zijn}\\

\haiku{Wat je schrijft over angst.}{en schuwheid voor mensen}{herken ik toch wel}\\

\haiku{Ik ben, net als jij,,.}{momenteel slecht op stoot maar}{ik knok ertegen}\\

\haiku{En zo ja - was die,?}{persoon de man over wie}{wij het laatst hadden}\\

\haiku{Ik herlas {\textquoteleft}De taal{\textquoteright}.}{der liefde en vond                     het}{opnieuw geweldig}\\

\haiku{Maar ik heb gewoon.}{zitten lezen in een}{eenvoudig bestek}\\

\haiku{de metselklinker {\textquoteleft}{\textquoteright}.}{is slechtsenigszins getrokken}{of                     beregend}\\

\haiku{O ja, men geeft hem.}{de hand. Maar telt vervolgens}{zijn vingers                     na}\\

\haiku{Breng het gesprek op,:}{Shakespeare en zeg}{dan langs uw neus weg}\\

\haiku{Toen de kelner op,:}{moeilijke platvoeten was}{weggestapt zei hij}\\

\haiku{Ze logeerden er.}{of kleefden er alleen aan}{de sublieme bar}\\

\haiku{Een zijner boeken.}{noemde hij Een diamant zo}{groot als de                     Ritz}\\

\haiku{Maar juist                     daarom.}{wilde Proust bij voorkeur door}{hem bediend worden}\\

\haiku{In de rue Blanche,,}{waar hij woonde ontmoette}{hij op een ochtend}\\

\haiku{van een grote                         .}{dichter heeft ze er vrede}{mee zo te heten}\\

\haiku{Ik zat bij hem en.}{na een tijdje vroeg hij wat}{mijn plannen waren}\\

\haiku{De vrouwen op straat.}{waren weggedoken in}{wijde                     jassen}\\

\haiku{De arrestatie?}{van Frech hebben we zeker}{aan jou te danken}\\

\haiku{Het is als met het -.}{geluid van Caruso jammer}{dat hij er mee praat}\\

\haiku{Maar hij heeft, hoop ik,.}{aan mijn toon gehoord dat ik}{het meende}\\

\haiku{je wordt                     er toch,.}{niet moe van je kunt er toch}{bij blijven zitten}\\

\haiku{Daarom ging Thomas.}{de trap niet af en Gerhardt}{de                         trap niet op}\\

\haiku{Die inspireerde.}{Mann tot het                         scheppen van}{zijn nieuwe figuur}\\

\haiku{Mijn vriend was er nog.}{niet en ik moest wachten in}{zijn werkkamer}\\

\haiku{In de spreektaal kunt{\textquoteleft}{\textquoteright}.}{u met het woordje                     mooi}{vele kanten op}\\

\haiku{{\textquoteright} De eerste jongen.}{floot niet zonder waardering}{door zijn voortanden}\\

\haiku{Nee meneer, tegen.}{snelheid en herrie is}{geen kruid gewassen}\\

\subsection{Uit: Onzin}

\haiku{{\textquoteright} vroeg mijn zoontje, die.}{de woorden allemaal niet}{zo precies weet}\\

\haiku{Ik                     lig in bed.}{en droom dat een steenbok me}{in het borstbeen bijt}\\

\haiku{dat ik hem nu,                      {\textquoteleft} -!}{onder de uitroepWat gij}{beledigt mijn st\'am}\\

\haiku{{\textquoteleft}Die                     jongen moet{\textquoteright}, {\textquoteleft}{\textquoteright},:}{opstaan ofHee vlegel en}{mijn tante zei paars}\\

\haiku{We namen elkaar,.}{op als twee worstelaars voor}{de wedstrijd begint}\\

\haiku{Opeens ging de deur.}{open en trad de oude heer}{Kortlever binnen}\\

\haiku{{\textquoteright} Hij bootste het na,.}{even omkijkend of niemand}{hem                     bespiedde}\\

\haiku{{\textquoteright} riep Kozels verschrikt,.}{als vreesde hij dat ik mij}{zou                     verdrinken}\\

\haiku{We zaten erg                     , ', '.}{ongemakkelijk maars}{lands wijss lands eer}\\

\haiku{Verdraaid,{\textquoteright} riep hij, {\textquoteleft}dat ',.}{zitm in de krachtfabriek}{als ik                     goed zie}\\

\haiku{De langste gaf, net,.}{toen ik passeerde                     de}{ander een schop}\\

\haiku{Het raadsel van de.}{naamloze voorbijganger}{kwam ter tafel}\\

\haiku{In ieder geval,{\textquoteright}.}{is uw oom een gourmet zei}{Annie inschenkend}\\

\haiku{Zij was half gedekt,,.}{er stond een bord op waarvan}{gegeten was}\\

\haiku{Bij Ina bevroren.}{de                     waterlanders op}{haar fijne kopje}\\

\haiku{{\textquoteright} riep de buurman, uit, {\textquoteleft},.}{het raam wijzenddaar met die}{blauwe jas                     aan}\\

\haiku{Koos kijkt nu bepaald,:}{op de staart getrapt maar het}{stemmetje klaagt}\\

\haiku{Ik kan er                     niets, -...}{aan doen maar hij st\'a\'at er weer}{voluit huilend nu}\\

\haiku{Ik strompelde naar:}{die helse bel en hoorde}{een meneer roepen}\\

\haiku{Dan wordt het duister.}{over zijn                     minuscule}{problematiek}\\

\haiku{{\textquoteright} roept de bestuurder,,{\textquoteleft}}{die de voornamen bepaald}{uit de mouw schudt}\\

\haiku{we hadden net zo.}{goed een huwelijksdatum}{kunnen vaststellen}\\

\haiku{Ik verloor opeens.}{mijn linkerschoen                         en moest}{er even naar zoeken}\\

\haiku{wat hij zei sloeg, naast,.}{deze haaibaai als een tang}{op een                     varken}\\

\haiku{En ge moet                     goed,.}{begrijpen dat ik er geen}{geld in steken wil}\\

\subsection{Uit: Onzin}

\haiku{{\textquoteright} vroeg mijn zoontje, die.}{de woorden allemaal niet}{zo precies weet}\\

\haiku{Ik                     lig in bed.}{en droom dat een steenbok me}{in het borstbeen bijt}\\

\haiku{dat ik hem nu,                      {\textquoteleft} -!}{onder de uitroepWat gij}{beledigt mijn st\'am}\\

\haiku{{\textquoteleft}Die                     jongen moet{\textquoteright}, {\textquoteleft}{\textquoteright},:}{opstaan ofHee vlegel en}{mijn tante zei paars}\\

\haiku{We namen elkaar,.}{op als twee worstelaars voor}{de wedstrijd begint}\\

\haiku{Opeens ging de deur.}{open en trad de oude heer}{Kortlever binnen}\\

\haiku{{\textquoteright} Hij bootste het na,.}{even omkijkend of niemand}{hem                     bespiedde}\\

\haiku{{\textquoteright} riep Kozels verschrikt,.}{als vreesde hij dat ik mij}{zou                     verdrinken}\\

\haiku{We zaten erg                     , ', '.}{ongemakkelijk maars}{lands wijss lands eer}\\

\haiku{Verdraaid,{\textquoteright} riep hij, {\textquoteleft}dat ',.}{zitm in de krachtfabnek}{als ik                     goed zie}\\

\haiku{De langste gaf, net,.}{toen ik passeerde                     de}{ander een schop}\\

\haiku{Het raadsel van de.}{naamloze voorbijganger}{kwam ter tafel}\\

\haiku{In ieder geval,{\textquoteright}.}{is uw oom een gourmet zei}{Annie inschenkend}\\

\haiku{Zij was half gedekt,,.}{er stond een bord op waarvan}{gegeten was}\\

\haiku{Bij Ina bevroren.}{de                     waterlanders op}{haar fijne kopje}\\

\haiku{{\textquoteright} riep de buurman, uit, {\textquoteleft},.}{het raam wijzenddaar met die}{blauwe jas                     aan}\\

\haiku{Koos kijkt nu bepaald,:}{op de staart getrapt maar het}{stemmetje klaagt}\\

\haiku{Ik kan er                     niets, -...}{aan doen maar hij st\'a\'at er weer}{voluit huilend nu}\\

\haiku{Ik strompelde naar:}{die helse bel en hoorde}{een meneer roepen}\\

\haiku{Dan wordt het duister.}{over zijn                     minuscule}{problematiek}\\

\haiku{{\textquoteright} roept de bestuurder,,{\textquoteleft}}{die de voornamen bepaald}{uit de mouw schudt}\\

\haiku{we hadden net zo.}{goed een huwelijksdatum}{kunnen vaststellen}\\

\haiku{Ik verloor opeens.}{mijn linkerschoen                         en moest}{er even naar zoeken}\\

\haiku{wat hij zei sloeg, naast,.}{deze haaibaai als een tang}{op een                     varken}\\

\haiku{En ge moet                     goed,.}{begrijpen dat ik er geen}{geld in steken wil}\\

\subsection{Uit: Een stoet van dwergen}

\haiku{Eindelijk waren:}{we het er dan met elkaar}{over eens geworden}\\

\haiku{Als hij  met een,.}{zaag terugkomt heb ik}{z\'o mijn broek weer aan}\\

\haiku{De man merkte dat.}{er helemaal geen triomf}{in haar stem                     was}\\

\haiku{{\textquoteright} vroeg mijn zoontje, die.}{de woorden allemaal}{niet zo precies weet}\\

\haiku{Hij bemerkte                         .}{mijn honger naar contact en}{lachte zindelijk}\\

\haiku{{\textquoteright} vroeg mijn zoontje, toen.}{we in de schemergrijze}{zijstraat                         liepen}\\

\haiku{Het laatst zag ik de.}{heer Cohen in de oorlog}{voor het station}\\

\haiku{Nou, aan zijn manier,.}{van lopen kon je zien dat}{hij mij gelijk gaf}\\

\haiku{En zij besloot hem,.}{die avond eens extra te}{raken in de gang}\\

\haiku{{\textquoteleft}Maar ik zal even mijn,.}{zaklantaarn halen dan kan}{ik beter                         zien}\\

\haiku{{\textquoteright} vroeg de chauffeur, die.}{bestoft doch ongebroken}{op de vloer                         zat}\\

\haiku{Met de nuttiging.}{van het ooft gingen zowat}{tien minuten heen}\\

\haiku{Maar als Ernest een,:}{vader                     met zijn zoon ziet}{wandelen denkt hij}\\

\haiku{{\textquoteright} Omdat niemand geld,.}{biedt ontrolt het mannetje}{nu ook de tweede}\\

\haiku{{\textquoteleft}O, uitstekend, dank{\textquoteright}?}{u. Want waarom zou het zo'n}{man                         niet goed gaan}\\

\haiku{Een half uur later:}{zat hij met haar op de}{trap en was al aan}\\

\haiku{{\textquoteright} De leraar stapte,.}{met een klein zwart wichtje aan}{de arm op hen af}\\

\haiku{hij wist het niet, maar.}{hij was in elk geval}{hogelijk bekoord}\\

\haiku{Kwam je 's ochtends,.}{beneden dan was de}{tafel al gedekt}\\

\haiku{Maar net zoveel thee,,,,.}{als je wou                     drie vier vijf}{koppen dat gaf niet}\\

\haiku{{\textquoteleft}Drink ze zelf maar op,,{\textquoteright}.}{artis zegt de man met een}{grimmig soort humor}\\

\haiku{{\textquoteleft}O ja, meneer Van -,,.}{Driel                    gaat u zitten neemt}{u plaats alstublieft}\\

\haiku{We hebben iets nieuws,.}{ingevoerd                     waardoor het}{werk wat sneller gaat}\\

\haiku{Want dat is toch geen -?}{manier van doen iemand zijn}{tekst afnemen}\\

\haiku{'t Is net als 't,.}{schippersvak je moet erin}{geboren worden}\\

\haiku{Ik heb wel eens in...{\textquoteright} {\textquoteleft},,}{een boek gelezenChurchill}{die bijna verzoop}\\

\haiku{Ze                     dronken wel,.}{hun glaasje maar mankeren}{deden ze nooit wat}\\

\haiku{Zacht en kalm was de.}{kroeg onder zijn stevige}{handen overleden}\\

\haiku{Steeds kleiner wordt het.}{schemerige rijk van de}{vaste jongens}\\

\haiku{Morgenochtend zijn,.}{lekker de slagers weer open}{dacht hij na{\"\i}ef}\\

\haiku{Ik hou van een meeuw,{\textquoteright}, {\textquoteleft}.}{vervolgde hij koppigen}{van Richard Tauber}\\

\haiku{Je komt er als mens.}{binnen en je gaat meteen}{op                         de knie\"en}\\

\haiku{{\textquoteright} Frits deed het en een.}{poosje later belde hij}{bij zijn moeder aan}\\

\haiku{Twintig Toen ik bij,.}{de haringkar kwam stond daar}{al een man te eten}\\

\haiku{Harry - ik droom nog.}{wel eens van hem als ik wat}{zwaar getafeld heb}\\

\haiku{{\textquoteleft}Die jongen die ze,,,.}{nou heeft                     och hij is goed}{maar sukkelachtig}\\

\haiku{{\textquoteright}        Maanlicht Om drie.}{uur in de nacht werd de man}{met een schok wakker}\\

\haiku{Op zijn kamertje.}{kleedde hij zich uit en ging}{in bed                     liggen}\\

\haiku{{\textquoteleft}Ach, wat heeft het voor?}{zin om het allemaal zo}{donker in te zien}\\

\haiku{{\textquoteleft}Je denkt toch niet dat?}{er louter parels van je}{lippen rollen}\\

\haiku{Anders verstrekt                     .}{Fie dat genoegen alleen}{op mijn verjaardag}\\

\haiku{Dat                         wil zeggen -,.}{ik was geen dwarse jongen}{maar ik wou vrij zijn}\\

\haiku{wachtte                         ik niet,.}{af en als ik wou deed ik}{een paar dagen niks}\\

\haiku{En alleen omdat,.}{hij het niet gedaan had moest}{jij de                         laan uit}\\

\haiku{Haar lach is echt en.}{ze kan niet tegen alles}{wat                     zielig is}\\

\subsection{Uit: Tussen mal en dwaas \& Klein beginnen}

\haiku{* ~ Mijn moeder is.}{altijd woedend als zij in}{een verhaal voorkomt}\\

\haiku{Uw vriend stelt vast                         .}{belang in uw verhaal over}{die krabbende vent}\\

\haiku{Nou moet er toch                     ,,.}{h\'e\'el wat gebeuren eer ik}{hier uit kom dacht ik}\\

\haiku{de vrede van een.}{naderende dut begon}{te                         vertonen}\\

\haiku{Het was een hele.}{stapel en er waren}{fraaie doorkijkjes bij}\\

\haiku{W\'at De Gaulle ook -.}{zeggen mag de wijn is weer}{goedkoop in Frankrijk}\\

\haiku{En dat maakt maar                     .}{droge-naaldetsen of}{het geen  geld kost}\\

\haiku{Wie een beroerde,,}{roman baart stelt alleen maar}{teleur maar                     schrijft}\\

\haiku{Zo'n Rubens kon                     ,.}{je de ruimte geven met}{zijn bolle dames}\\

\haiku{Maar dat presentje.}{heb ik gewoon weer in de}{plas gesmeten}\\

\haiku{Zij nam de arm                         .}{van haar galant en liep met}{hem het portiek uit}\\

\haiku{Ach, er zijn toch wel.}{fronten waarop Hitler heeft}{gezegevierd}\\

\haiku{Zij opende haar tas,:}{pakte er een zakje uit}{en zei                         opeens}\\

\haiku{Tot huis toe heeft dat.}{zuurtje als een kei op}{mijn maag gelegen}\\

\haiku{Waardig stapte ik.}{met mijn bordje                         terug}{naar de huiskamer}\\

\haiku{Hij vroeg dan ook geen,:}{twintig kaartjes                         voor het}{Stadion maar zei}\\

\haiku{{\textquoteleft}Nee, er h\'o\'eft immers,!}{nooit gekocht te worden}{dat liedje ken ik}\\

\haiku{Het vervelende.}{was echter dat die twee}{uit mijn beeld liepen}\\

\haiku{een troep b\^ete                         :}{schoolkinderen staat voor het}{tijgerhok en roept}\\

\haiku{Het zou niet haar                     .}{enige aanraking met de}{journalistiek zijn}\\

\haiku{Laat ons tezamen.}{in alle rust enige}{genres bekijken}\\

\haiku{Ze dachten dat ze.}{de brieven voor Holland in}{z\'e\'e moesten gooien}\\

\haiku{Die kerstavond was het.}{zwijgen nog drukkender in}{de keukenkamer}\\

\haiku{Toen hervatte de,:}{zwarte aarzelend doch niet}{onzakelijk}\\

\haiku{{\textquoteright} {\textquoteleft}H\`e, verdorie - dat,!}{w\'e\'et ik toch wel ik ben toch}{zeker niet kippig}\\

\haiku{{\textquoteright} begon hij                     op,:}{verlokkende toon maar ik}{wenkte af en vroeg}\\

\haiku{De                     uitspanning,:}{was reeds ver achter ons toen}{ik wanhopig vroeg}\\

\haiku{{\textquoteleft}Maar er moet wel een,.}{lappie om anders zou er}{vuil                     in komen}\\

\haiku{Moeizaam klom ik uit,.}{de veren                         want ik was}{heel dik geworden}\\

\haiku{{\textquoteright} Nu was het toch wel.}{zeker dat ik het een flink}{eind geschopt had}\\

\haiku{Maak het echter niet,... {\textquoteleft}}{\'al te suggestief want dat}{breekt je later op}\\

\haiku{Goed, als een vrouw in,.}{het parlement wil dan vind}{ik dat                     prachtig}\\

\haiku{Ik begreep er niets,,.}{van maar mijn moeder trok mij}{mee steeds huilend}\\

\haiku{{\textquoteright} En ze gaf me een,.}{klap op mijn hand zodat de}{punt in een plas viel}\\

\haiku{{\textquoteleft}Meester, we hebben,.}{er allemaal aan betaald}{behalve Frits}\\

\haiku{Het laatst zag ik de.}{heer Cohen in de oorlog}{voor het station}\\

\haiku{{\textquoteright} Alweer een grapje -?}{zou hij soms geheim abonnee}{van Kiekeboe zijn}\\

\haiku{Hij dacht even na, want.}{ondoordachte bescheiden}{geeft hij niet graag af}\\

\haiku{In de propvolle.}{trein wilde iedereen het}{meteen vertrappen}\\

\haiku{En n\'et toen ik in,.}{dat laatje rommelde kwamen}{die mensen binnen}\\

\haiku{Ik voor mij geloof.}{dat hij nu van dat zingen}{verlost is}\\

\haiku{{\textquoteleft}Wij gaan naar tante,{\textquoteright},.}{Aaltje zei het ene meisje}{toen wij weer reden}\\

\haiku{Hij zette het                     :}{toestel na  enig mikken}{aan zijn oor en sprak}\\

\haiku{De hele middag:}{op school hadie kennelijk}{zitten spinnen}\\

\haiku{Nou, aan zijn manier.}{van lopen kon je zien dat}{hij mij gelijk gaf}\\

\section{Jacob Cats}

\subsection{Uit: Huwelijk}

\haiku{Zo wordt nu 't oog,;}{hem los gedaan Dies ziet het}{zijn gevangen aan}\\

\haiku{Dit speeltjen heeft,.}{een grote sleep Men houdt daar}{eeuwig wat men greep}\\

\haiku{En wat de vrek in.}{d'aarde groef Dat is dan voor}{een malle oschroef}\\

\haiku{Veracht dan niet, o,.}{weerde vriend Wat u en mij}{ten goede dient}\\

\haiku{*~        T' samenspraak}{tussen Anna en Phyllis}{Gij die met vrucht}\\

\haiku{Phyllis Voor mij, ik,.}{wil hier open gaan Gelijk bij}{vrienden dient gedaan}\\

\haiku{Het gaat zo eender,.}{met de min Krakeeltjens}{brengen vriendschap in}\\

\haiku{Bij deze kwam een,.}{jong-gezel Een leerling}{in het minnespel}\\

\haiku{*~         De tweede valt,;}{hem in het haar En stelt een}{wonder vreemd gebaar}\\

\haiku{De last van 't huis,.}{de wil des mans En alle}{jaar een kind bijkans}\\

\haiku{Ziet, kind nu olijd ik,.}{dat ge gaat En dank u voor}{uw goeden raad}\\

\haiku{- Het meisje fluks en.}{onvermoeid         Kwam naar de}{kade toe geroeid}\\

\haiku{De vrijster wil naar, ';}{dezen kant Maar slaat het oog}{opt ander land}\\

\haiku{Schoon of een maagd een '.}{rugge biedtt En hindert}{aan het roeien niet}\\

\haiku{Ik prijze ja het,.}{echte-bond Maar niet als}{op den rechten stond}\\

\haiku{*~         Zegt wat ge wilt,,;}{slechts om de dracht Is menig}{mense hoog geacht}\\

\haiku{*~         Ik wil niet zo,}{versaagden geest Die ook zijn}{eigen schaduw vreest}\\

\haiku{*~         Hierachter woont,,}{een zeldzaam wijf En doet de}{jonkheid groot gerijf}\\

\haiku{*~         Door hem is al,.}{het stuk beleid Hem zij de}{lof in eeuwigheid}\\

\haiku{Een wijf, een krone,,.}{van den man Dat ospillen}{en dat sparen kan}\\

\haiku{als zich enig mens laat,;}{tot de lusten drijven De}{korte vreugd verdwijnt}\\

\haiku{wat ze brengen moet,.}{Omdat ze vreemde zucht in}{haren boezem voedt}\\

\haiku{een licht, een helder,.}{licht Dat roept u tot de zorg}{van uwen nieuwen plicht}\\

\haiku{Een wettig overheer.}{Moet ja de voorste zijn tot}{alle goede leer}\\

\haiku{Gij, die in echte,,,.}{paart Gaat heult met uwen man ook}{tegen uwen aard}\\

\haiku{hij, die het wonder,.}{ziet En prijst nog evenwel den}{groten Schepper niet}\\

\haiku{Wie dezen regel,,.}{houdt Die blijft dan even maagd ook}{als hij is getrouwd}\\

\haiku{gij, stelt uw dingen,.}{vast Eer u een hete koorts}{met smarten overlast}\\

\haiku{Vrouwen en mannen.}{moeten met een gelijke}{in leeftijd trouwen}\\

\haiku{klappers kletsmajoors;}{21 vuile zwangere 22}{sluiten opsluiten}\\

\haiku{met stijve kaken;}{onverschrokken 41 tot haar}{dagen bescheiden}\\

\section{August van Cauwelaert}

\subsection{Uit: Het licht achter den heuvel}

\haiku{Een enkele maal;}{kwam een vlaag van gezang den}{heuvel opgewaaid}\\

\haiku{Waarom was jonkheer?}{Leonce van de Burcht toen}{niet bijgesprongen}\\

\haiku{Dat duurde echter,.}{slechts eenige dagen hoogstens}{enkele weken}\\

\haiku{Hij zag den pastoor.}{staan praten tegen de vrouw}{met de kindertjes}\\

\haiku{{\textquoteleft}Ik zou 't op den,{\textquoteright}.}{duur nog te warm krijgen zei}{de geestelijke}\\

\haiku{{\textquoteright} De geestelijke '.}{vreesde dat zet jaar niet}{ten einde zou gaan}\\

\haiku{Willem klopte haar.}{vriendelijk op den schouder}{en dat deed haar goed}\\

\haiku{hij ontmoette was,;}{de veldwachter op gang met}{de lastenbrieven}\\

\haiku{Daarop schonk ze nog {\textquoteleft},{\textquoteright}.}{eens de glazen vol.Klara}{zei Willem gedempt}\\

\haiku{nog wankelend in,,.}{zijn wil aarzelend maar meer}{en meer verwonnen}\\

\haiku{Daar was geen een die,.}{nog verroerde maar ze}{keken hun oogen uit}\\

\haiku{De kinderen die,.}{onder het zeil te loeren}{lagen kletsten mee}\\

\haiku{Hij was blij dat hij.}{zijn militaire kleedij had}{mogen wegbergen}\\

\haiku{Willem zag vader,,.}{van het braakland komen naar}{hem toe recht en vlug}\\

\haiku{Het zong in Baltus.}{hart als een lied van kracht en}{voorspoed en geluk}\\

\haiku{Willem schrok toen hij.}{kort getrappel van voeten}{hoorde in den gang}\\

\haiku{{\textquoteleft}Het moet u wel vreemd,}{hebben geschenen zoo weer}{veilig thuis te zijn}\\

\haiku{{\textquoteright} {\textquoteleft}Nu is 't goed,{\textquoteright} zei,.}{Lucette toen Dorry maar}{doorging met zoenen}\\

\haiku{Ze was weer af en}{toe piano gaan spelen}{en ze las zooveel}\\

\haiku{Ze stapte nu 's,;}{zondags in losse lichte}{kleedjes en blouses}\\

\haiku{Tegen den kerkmuur.}{was Jan de grafmaker een}{put aan het delven}\\

\haiku{een lage vlucht van.}{duiven die uit het veld naar}{het dorp toeroeiden}\\

\haiku{{\textquoteright} vroeg Kardoentje, die.}{ongemerkt den veldwegel}{was afgekomen}\\

\haiku{Zijn bloed sloeg naar zijn.}{slapen en zijn hart ging plots}{aan het hameren}\\

\haiku{Niet zoo hard loopen,,.}{waarschuwde een vrouwestem}{die hij herkende}\\

\haiku{Toen kwam Lucette.}{zelf te voorschijn en wou dat}{ze even rusten zou}\\

\haiku{{\textquoteright} Theo lei 't nestje.}{in Mariette's hand als in}{een roze schelpje}\\

\haiku{waar ze te broeden.}{zaten en waar er kleintjes}{in het nest lagen}\\

\haiku{Want den volgenden;}{morgen kwamen er vreemde}{heeren op het dorp}\\

\haiku{{\textquoteright} 't Was vier uur v\'o\'or.}{de heeren klaar waren met}{hun onderzoek}\\

\haiku{Ze lachte een paar,.}{malen ironisch en keek in}{het glanzend water}\\

\haiku{tot ze heelemaal...}{voorbij was en spreken op}{zoo'n afstand evenmin}\\

\haiku{Dan heeft een boer geen.}{tijd om te rusten en geen}{tijd om te denken}\\

\haiku{Die zal nu duren.}{tot de nacht komt en de slaap}{over de  menschen}\\

\haiku{Moeder was nu thuis.}{gekomen en riep dat de}{koffie gereed stond}\\

\haiku{Maar daarop lag de.}{stilte weer onverbroken}{over land en hoeven}\\

\haiku{maar hij vermande,:}{zich lei zijne armen over}{haar schouders en zei}\\

\haiku{dat zeg ik niet, maar,.}{wat in mijne ooren valt}{dat knoop ik erin}\\

\haiku{{\textquoteright} {\textquoteleft}Laat het dan nu voor,{\textquoteright}.}{de eerste maal geschieden}{drong de jonkheer aan}\\

\haiku{{\textquoteleft}maar ik heb hem al,.}{maanden niet gezien tenzij}{van den predikstoel}\\

\haiku{Hij werd moe en ging '.}{even zitten in de schaduw}{vant prieeltje}\\

\haiku{{\textquoteleft}Mijn liefste meisje,{\textquoteright}, {\textquoteleft}.}{zei Kardoentjedat hoeft ge}{me niet te zeggen}\\

\haiku{{\textquoteright} riep hij terug, en.}{verdween achter het huis van}{den klompenmaker}\\

\haiku{Het werd een kermis.}{zooals zij er zes jaar lang geen}{meer gezien hadden}\\

\haiku{Harder door,{\textquoteright} riep Theo, '.}{den orgeldraaier toe in}{t voorbij walsen}\\

\haiku{maar Vital raakte.}{weer aarde en schoorde zijn}{beenen sterk als een muur}\\

\haiku{Maar Theo draaide zich;}{bliksemsnel om en tilde}{Vital op den rug}\\

\haiku{Theo had snel onder '.}{zijn linkerarmt hoofd van}{Vital gegrepen}\\

\haiku{Het leed en de zorg.}{kan een mensch tien jaar ouder}{maken op \'e\'en dag}\\

\haiku{De boeren waren.}{nu druk aan het ploegen en}{eggen en zaaien}\\

\haiku{De burgemeester;}{haalde er nauwelijks drie}{kandidaten door}\\

\haiku{*** ~ Maar Willem moest.}{den volgenden morgen al}{vroeg in de stad zijn}\\

\haiku{Toen Lucette thuis.}{kwam vond ze een briefje van}{Albert Durenne}\\

\haiku{Meneer pastoor moest.}{dus nog een tijdje wachten}{en hij wachtte nog}\\

\haiku{Een gelegenheid,.}{misschien een tijdverdrijf of}{een experiment}\\

\haiku{Ik groei er uit, dacht,.}{hij dan bij zichzelf ik voel}{dat ik er uit groei}\\

\haiku{En Klara zelf leek.}{lang zoo frisch en jong niet meer}{als in den zomer}\\

\haiku{Haar gelaat rees naar.}{het zijne op als een bloem}{en haar oogen glansden}\\

\haiku{Zijn hoofd helde naar.}{haar kopje toe en zijn arm}{boog om haar middel}\\

\haiku{De knecht die mest aan,.}{het onderploegen was hield}{zijn span stil en keek}\\

\haiku{want hij voelde zijn.}{hart onrustig en vaardig}{voor alle avontuur}\\

\haiku{Ze hangt boven de,;}{huizen in den hoogeren}{blauweren hemel}\\

\haiku{{\textquoteright} Willem keek verschrikt,.}{om want hij hoorde stappen}{zoo heel dicht bij hen}\\

\haiku{Zijn handen gleden.}{met een wonderen schroom en}{wijding over haar hoofd}\\

\haiku{Maar nu hoopte hij.}{dat hij haar niet meer onder}{de oogen komen zou}\\

\haiku{dat kon dan later.}{voor andere doeleinden}{worden aangewend}\\

\haiku{Fabrieken vergen.}{meer handen dan er in zoo'n}{streek beschikbaar zijn}\\

\haiku{{\textquoteleft}op voorwaarde dat...{\textquoteright}}{ge me niet een tweede maal}{met ontrouw betaalt}\\

\haiku{{\textquoteleft}Wie zich vergooit aan,,{\textquoteright}.}{een vrouw vergooit zijn toekomst}{bromde Baltus weer}\\

\haiku{Wat er verder met,.}{zijn leven gebeuren moest}{zou de toekomst leeren}\\

\haiku{Hij kwam verder op.}{den weg den pastoor tegen}{en hij hield hem staan}\\

\haiku{Ik zal u helpen.}{in den nood En vooral in}{het uur der dood}\\

\haiku{In de donkerte '.}{zag zet vuur van een pijp}{die werd aangepaft}\\

\haiku{Dat is tenslotte,{\textquoteright}.}{nog het beste wat we doen}{kunnen zei Mina}\\

\haiku{het is voor een vrouw,}{die begeert te voelen dat}{de jonge man dien}\\

\haiku{Ik zou zelf naar u,}{komen maar Dorry is nog}{wat onwel en nu}\\

\haiku{Een tijd bleef hij zoo,.}{nog zitten verstard in zijn}{leed en zijn wroeging}\\

\haiku{Willem leunde met.}{den rug tegen het raam en}{staarde naar den grond}\\

\haiku{Wat reeds voorbij was,.}{leek hem nu zoo jong ondiep}{en onervaren}\\

\haiku{Eindelijk hoorde.}{ze een stoel verschuiven en}{geklop van voeten}\\

\haiku{Zijne oogen lagen,.}{dieper in hunne holen}{maar glansden rustig}\\

\haiku{Moeder haalde nog.}{een paar kussens bij en hielp}{hem recht in zijn bed}\\

\haiku{Het gezang golfde.}{tegen de hoevemuren}{aan en over het dak}\\

\haiku{Moeder vroeg of ze '.}{t venster weer sluiten wou}{en hij vond het goed}\\

\haiku{Baltus fluisterde.}{hunne namen naarmate}{ze voorbijgingen}\\

\haiku{Ik was pas op de.}{universiteit en zij was}{nauwelijks achttien}\\

\haiku{{\textquoteright} {\textquoteleft}De pastoor zei toch,{\textquoteright}.}{dat ge er gelukkig zoudt}{mee zijn zei moeder}\\

\haiku{Hij stapte door den.}{mulligen landweg naar den}{grooten steenweg toe}\\

\subsection{Uit: Vertellen in toga}

\haiku{De slachter had de;}{wonde weer dicht gebrand met}{een gloeiend ijzer}\\

\haiku{Zoo stonden die twee,.}{elkaar daar uit te schelden}{ik weet niet hoe lang}\\

\haiku{En ze zou er heel.}{haar leven een lidteeken van}{dragen op haar been}\\

\haiku{Dan begon hij met,.}{zijn achterwerk te stooten}{lijk met een stormram}\\

\haiku{En den volgenden.}{morgen trok de Vilder naar}{zijnen advokaat}\\

\haiku{met verzoek dien dag.}{en dat uur te verschijnen}{in de raadskamer}\\

\haiku{Nu wou de rechter.}{nog weten wanneer de weg}{in orde zou zijn}\\

\haiku{En dan weet ge niet.}{meer of het van verdriet is}{of van zattigheid}\\

\haiku{Maar hij deed het niet,.}{en dien dag was hij overal}{bezig over zijn vrouw}\\

\haiku{Ze kwamen vragen.}{of hij dezen keer de boet}{niet kon kwijtschelden}\\

\haiku{Wanneer gaat ge toch,...}{ophouden vroeg de rechter}{ge drinkt u dood of}\\

\haiku{die komt er altijd.}{tusschen en dan vallen ze}{allemaal op mij}\\

\haiku{En het karretje,.}{reed weg de stad door en de}{vestingen voorbij}\\

\haiku{Meneer de rechter,,...}{zei ze ik zweer bij God en}{al zijn heiligen}\\

\haiku{- Da... da... wee... - Maar wat?}{hebt ge dan zelf verteld aan}{Verhulst Justine}\\

\haiku{Van den koster, zei,,?}{ik zoo ik weet van niks wat}{is daar mee gebeurd}\\

\haiku{Want ik ben nog wel.}{twee keeren blijven staan om Jef}{kwijt te geraken}\\

\haiku{ne satyr, Phanie,.}{dat is iemand die achter}{de kinderen loopt}\\

\haiku{Ja, die achter de,,.}{kinderen loopt zei Rinne}{dat is ne satyr}\\

\haiku{maar ge moet ons de,.}{volle waarheid zeggen juist}{zooals het is gebeurd}\\

\haiku{Maar de rechter hield:}{ze weer in en zei nog eens}{traag en met nadruk}\\

\haiku{Maar Buken liep de}{rest van den dag verloren}{in en rond het huis}\\

\haiku{Dat was een tweede.}{kermisdag en Buken liet}{nonkel niet meer los}\\

\haiku{Maar den volgenden '?}{middag was hij dadelijk}{aant reklameeren}\\

\haiku{maar iedermaal dat,:}{hij op een stoel klom moest hij}{weer eens zingen van}\\

\haiku{- 't Is toch kurieus,,...}{kloeg nonkel Baptist dat hij}{thuis zoo bot kan zijn}\\

\haiku{vroeg een vrouw, die met.}{een zwart korfken op haren}{schoot naast Buken zat}\\

\haiku{Buken was nog maar.}{pas den dorpel over of hij}{pakte nonkel aan}\\

\haiku{en met welk geld had,?}{hij dat perceel land betaald}{daar achter den smid}\\

\haiku{kocht een nieuwe klak;}{en een dubbelen col en}{een nieuwe kravat}\\

\haiku{en dat het koren,.}{schoon staat of dat de plaag aan}{de patatten zit}\\

\haiku{Maar toen Buken naar,;}{het onderzoek ging keurden}{ze hem ineens af}\\

\haiku{Buken werd triestig '.}{en krikkel ent werk werd}{allemaal te zwaar}\\

\haiku{Een droge hoest en,.}{daarop een kou en toen was}{het direkt gedaan}\\

\haiku{Het was duidelijk;}{genoeg dat het meisje op}{een verkeerd spoor was}\\

\haiku{En zei er nog iets.}{bij dat zuster Bernarda}{niet goed verstaan had}\\

\haiku{Mijn schatteke lief,,...}{kom dezen avond onder dit}{raam ik zal er zijn}\\

\haiku{Als er geen godsvrucht,.}{in zit is er niets aan te}{vangen met een kind}\\

\haiku{En de meisjes die,;}{binnenkwamen begonnen}{zooals het regel was}\\

\haiku{Een draai en de deur.}{ging open en Robbetje was}{weg de vrijheid in}\\

\haiku{Het was iets van dit,.}{alles het was misschien dit}{alles tegelijk}\\

\haiku{Toen de kleintjes van,.}{Robbetje weg waren kwam}{Ir\`ene naast haar staan}\\

\section{Charivarius}

\subsection{Uit: Nieuwe groene Charivaria. Deel 2}

\haiku{Zoo kan het vervoer.}{billijker en beduidend}{vlugger geschieden}\\

\haiku{{\textquoteright} (H.D.) {\textquoteleft}Dergelijke.}{landen zijn niet al te zeer}{in ernst te nemen}\\

\haiku{dit is vervelend,,.}{en Charivarius wou}{gaarne dat het N.v.d}\\

\haiku{20 - Wanneer gij u,.}{zwaarmoedig gevoelt wijt het}{niet aan anderen}\\

\haiku{Het schijnt ons niet meer,,.}{dan fair dat zij weten wat}{hun te wachten staat}\\

\haiku{, is te gast geweest.}{bij den chef van den Duitschen}{Generalen Staf}\\

\haiku{{\textquoteleft}In 't bijzonder.}{vestig ik de aandacht op}{den naam van de soep}\\

\haiku{Wellicht komt er uit.}{dezen oorlog toch iets goeds}{voor Nederland voort}\\

\haiku{{\textquoteright} Wij zijn benieuwd te,.}{hooren wie het ten slotte}{gekregen heeft}\\

\haiku{Wat wij nu zouden,,.}{willen weten is of er}{getrapt is of niet}\\

\haiku{Edelachtbare, De.}{Spaansche Vlieg is aanstootelijk}{voor de eerbaarheid}\\

\haiku{{\textquoteright}  (Lisser C.) ~ ;}{De Lisser C. is ons niet}{belangrijk genoeg}\\

\haiku{- Of 't hierop zal,.}{uitloopen kan voorshands nog}{niet beslist worden}\\

\haiku{Wij meenen onze.}{abonn\'e's met dat in kennis}{te moeten stellen}\\

\haiku{Meer menschen dan de,{\textquoteright}.}{kerk kon bevatten konden}{geen plaats vinden}\\

\section{J.B. Charles}

\subsection{Uit: Ontmoeting in den vreemde}

\haiku{Kraus Leermens, die ik!}{ergens in Nederland in}{een gesticht waande}\\

\haiku{Was het weer feest in?}{de stad en was  jij er}{deze keer \'o\'ok bij}\\

\haiku{, en het antwoord dat,,.}{vlot kwam bewees dat ik niets}{had moeten vragen}\\

\haiku{{\textquoteright} Hier onderbrak hij.}{zijn verbeelding om weer in}{het bier te kijken}\\

\haiku{Hij zou mij leren,,.}{kleinere jongens af te}{rossen schreeuwde hij}\\

\haiku{In de toekomst zal,.}{zij niet meer Moeder heten}{maar Minnares zijn}\\

\haiku{De bodem heeft wat.}{er op leefde altijd aan}{zich onderworpen}\\

\haiku{Ik ben niet met haar,.}{meegegaan maar heb haar met}{mij meegenomen}\\

\haiku{Maar nu die wezens,,?}{hier waarin zit het hem hun}{sociale sfeer}\\

\haiku{Ik ging natuurlijk,.}{niet maar bleef uren lang door de}{oude stad dwalen}\\

\subsection{Uit: Van het kleine koude front}

\haiku{waar in 1968 nog over,.}{X of Y gesproken is}{noem ik nu namen}\\

\haiku{Van daar uit had men.}{een beter gezicht op de}{ddr dan van Bonn uit}\\

\haiku{Alleen de boerin.}{blijft van Norfolk komen en}{blijft engels zijn}\\

\haiku{Het interesseert.}{hem helemaal niet waar ik}{dan w\`el vandaan kom}\\

\haiku{De duitsers, ook de,.}{kommunistiese konden}{niet saboteren}\\

\haiku{Ik zou van deze.}{man zelf horen wat hij zelf}{van de zaak denkt}\\

\haiku{Wij drinken een glas.}{bier in het kafee op de}{eerste verdieping}\\

\haiku{Men werkt daar dag en,.}{nacht in drie groepen van 90}{man telkens 8 uren}\\

\haiku{Zij vertellen mij.}{dat hij naar west gevlucht is}{en daar en daar zit}\\

\haiku{{\textquoteleft}die artikel-6'ers.}{denken dat ze beter zijn}{dan de anderen}\\

\haiku{In de eerste plaats.}{de Hortus Botanicus}{aan het Rapenburg}\\

\haiku{Misschien hoort het bij.}{een tijd van absolute}{tegenstellingen}\\

\haiku{Dimitroff spreekt, als,,.}{Van der Lubbe geen goed duits}{dat leek een voordeel}\\

\haiku{Ein Recht ist es der}{kommunistischen Partei in}{Deutschland illegal}\\

\haiku{Zijn vaders wens dat:}{hij officier zou worden}{heeft hij genegeerd}\\

\haiku{Dan is er een groot.}{stenen bord met veel namen}{daarin gebeiteld}\\

\haiku{Ik reis door dit land -?}{en geniet ervan terwijl}{ik broed op waarop}\\

\haiku{Een paar jaar later,,,:}{eind november 1961 bericht}{mijn krant uit Wenen}\\

\haiku{Hongarije v\'o\'or het?}{in deze rampzalige}{toestand geraakte}\\

\haiku{Een enkele heeft.}{ook wel es een beetje aan}{expansie gedaan}\\

\haiku{Nog steeds ben ik er.}{niet achter waarom zij dit}{staatshoofd zo haatte}\\

\haiku{Op 8 november;}{1931 wordt de eerste nieuwe}{kamer gekozen}\\

\haiku{En dan nog deze:}{kleine gedachte om mee}{naar huis te nemen}\\

\haiku{zij geven daarbij}{de nazi's van het land zelf}{de gelegenheid}\\

\haiku{kortom het is de.}{totale oorlog van het}{fascistiese beest}\\

\haiku{Hoe zwaarder men woog,.}{des te meer men zich blijkbaar}{kon veroorloven}\\

\haiku{Misschien zijn zij dood,,.}{waarschijnlijk zijn zij arm maar}{zij hadden gelijk}\\

\haiku{Ik had daar met mijn.}{buro aan meegedaan en}{hij keurde dat af}\\

\haiku{Anderen hebben.}{het er na de oorlog niet}{zo goed afgebracht}\\

\haiku{Een amsterdamse.}{konfektiefabrikant had}{gekollaboreerd}\\

\haiku{Ziehier de strekking.}{van mijn opmerkingen bij}{Martin korda dp}\\

\haiku{men zal dus misschien.}{willen begrijpen dat ik}{nu wel iets m\'oest doen}\\

\haiku{werkzaam waren op.}{hetzelfde uur dat Powers}{boven Rusland vloog}\\

\haiku{Van \'e\'en naoorlogs:}{verschijnsel was hij vooral}{onder de indruk}\\

\haiku{Het geweld valt het.}{geweld aan en de oorlog}{verdedigt zichzelf}\\

\haiku{Ik reken die hem,.}{toe ook al beweert hij dat}{het zijn schuld niet is}\\

\haiku{Daar stond echter een.}{onoverkomelijke}{barri\`ere voor}\\

\haiku{wat hij nog m\'e\'er was,.}{leren wij uit het gedicht}{van Engelman niet}\\

\haiku{Ja, maar ik h\'oef het.}{vod waar het in afgedrukt}{staat niet te lezen}\\

\haiku{Eros en Thanatos,,:}{liefde en haat voortbrenging}{en vernietiging}\\

\haiku{Het spijt mij, maar ik.}{breng je weer van Willy naar}{Siebe  terug}\\

\haiku{Dat is in onze.}{kring al dikwijls aan alle}{kanten bekeken}\\

\haiku{Maar  wat kan hij?}{doen om zijn inzichten tot}{gelding te brengen}\\

\haiku{Daarvan is helaas.}{in Oost-Duitsland op het}{ogenblik geen sprake}\\

\haiku{Kein Frieden ohne,.}{Freiheit aber keine Freiheit}{ohne Wahrheit}\\

\haiku{Het betrekkelijk.}{stellen van alle waarden}{laat er geen een over}\\

\haiku{het is uit angst voor.}{of kwaadwilligheid jegens}{bepaalde waarden}\\

\haiku{dan is het die in.}{het goede en het slechte}{konservatisme}\\

\haiku{Dat is ook moeilijk,.}{want er is namelijk niets}{bizonders gebeurd}\\

\haiku{Als wij w\`el van hem:}{vernemen en het is een}{ander geluid dan}\\

\haiku{De oorlog tegen.}{Egypte in 1956 was \'e\'en van}{die schietpartijen}\\

\haiku{Voor de joden is.}{deze nieuwe simpatie}{bijna even pijnlijk}\\

\haiku{De joden zijn nooit {\textquoteleft}{\textquoteright}.}{slechter geweest danwij en}{zijn nu niet beter}\\

\haiku{Het is een zwaar woord {\textquoteleft}{\textquoteright},.}{oorlogsmisdadiger maar}{het is niet te zwaar}\\

\haiku{Toen ik bij hem kwam,.}{was hij al steenkoud zo groot}{was het bloedverlies}\\

\haiku{Hij moest dansen met,.}{een kip in zijn handen zijn}{gebedskleden aan}\\

\haiku{Maar wij willen een;}{model van de bataafse}{kristenen tonen}\\

\haiku{Deze inleider,,:}{drs. G. Puchinger verklaart}{dat dit boek ons toont}\\

\haiku{Nu is het leven.}{dan ook eens een klein beetje}{naar voor de Colijns}\\

\haiku{Den Haag en vele,.}{plaatsen zijn vreselijk om}{aan te zien zegt men}\\

\haiku{Als het christelijk.}{is dan is het evengoed joods}{of mohammedaans}\\

\haiku{Men schijnt dit niet te.}{willen zeggen en evenmin}{hoe lang het nog duurt}\\

\haiku{Na de bijbelse.}{geschiedenis krijgen}{we aardrijkskunde}\\

\haiku{Het zou ook onjuist,.}{van hem geweest zijn zich pas}{weer op die XX}\\

\haiku{Hier zijn wij in het.}{gezelschap van Lunshof en}{wijlen Piet Bakker}\\

\haiku{Zo is het met dat,.}{grote humanisme het}{socialisme}\\

\haiku{Diezelfde stroom voedt,,.}{als men dat zo mag zeggen}{de loop van het boek}\\

\haiku{Het was dus een boek.}{dat niet klaar kon komen en}{toch een keer af moest}\\

\section{Ernest Claes}

\subsection{Uit: Clementine}

\haiku{, en hij wilde zich.}{zoo gauw mogelijk van het}{zaakje afmaken}\\

\haiku{- Clementine hield,.}{van niets en niemand zij dacht}{alleen aan zich zelf}\\

\haiku{{\textquoteleft}Clementine, ge,....}{moet niet ongerust zijn voor}{uw ouden dag wicht}\\

\haiku{De maanden gingen,.}{en daar kwamen er altijd}{weer andere}\\

\haiku{Het docht haar opeens.}{dat hij naar haar keek met een}{vreemd licht in de oogen}\\

\haiku{{\textquoteleft}Clementine, is?}{dat misschien voor mij dat ge}{die laat koud worden}\\

\haiku{maar dat hadt ge op.}{een andere manier ook}{kunnen regeleeren}\\

\haiku{Als er een brief kwam.}{deed ze dien zelf open en las}{hem mijnheerke voor}\\

\haiku{De pastoor is op.}{het kasteelke twee keeren}{komen aanbellen}\\

\haiku{Die koffie was zeer,.}{duti en de boterhams}{weinig in getal}\\

\haiku{{\textquoteright} De gierigheid van.}{Clementine vond daarin}{een groote voldoening}\\

\haiku{Hij heeft het deurtje.}{even open gezet om het vuur}{wat te temperen}\\

\haiku{tak komt er een dof,,:}{bijgeluidje bijna als}{een na-snikje zoo}\\

\haiku{Hier binnen leeft er,.}{iets onwezenlijks daar zijn}{geesten in dat huis}\\

\haiku{Clementine blikt,: - {\textquoteleft}'....}{op naar den wekker en zegt}{t Is negen uur}\\

\haiku{In het dorp blaften,.}{een paar honden en op den}{toren sloeg het uur}\\

\haiku{Als dat gedaan was, -....}{kwam ze voor de kachel staan}{aldoor kreunend \`eheu}\\

\subsection{Uit: De heiligen van Sichem}

\haiku{Ge leest er in uw.}{kerkboek gelijk ge thuis de}{gazet zoudt lezen}\\

\haiku{Z\'o\'o stom zijn we te.}{Sichem niet dat we daar ook}{geen oogen voor hebben}\\

\haiku{Daar moeten twee of.}{drie geslachten van menschen}{overheen zijn gegaan}\\

\haiku{Zooals hij daar staat is.}{Sint Rochus anders een heel}{kurieus manneke}\\

\haiku{Ze kijkt recht omhoog,.}{en ze lacht in heur eigen}{met wat ze daar ziet}\\

\haiku{Zijn gezicht is veel,.}{te weemoe{\"\i}g voor Sichem}{en te schoon ook}\\

\haiku{den uitgang gereed,,.}{wat verder de harmonie}{en ze komen af}\\

\haiku{En het is nu zoo '.}{stil gelijk oft heele}{dorp \'e\'en kerk was}\\

\subsection{Uit: Herodes}

\haiku{En deze, terwijl,.}{hij ze aankeek wist wat er}{in hun geest omging}\\

\haiku{Hij had geen tolk noodig,.}{want hij zelf sprak de meeste}{Oostersche talen}\\

\haiku{Misschien heeft het stuk...}{wel een wondere kracht in}{de Oosterlanden}\\

\haiku{waar hij anderen,.}{niet vertrouwde begreep hij}{er ditmaal niets van}\\

\subsection{Uit: Jeroom en Benzamien}

\haiku{ik geloof dat ik.}{niet verder op het geval}{zou zijn ingegaan}\\

\haiku{{\textquoteleft}Als ge daar niet in,.}{zijt verstaat ge dat niet zoo}{van den eersten keer}\\

\haiku{Ten slotte weze.}{nog herhaald dat zij beiden}{katholiek waren}\\

\haiku{Enkele keeren had.}{hij aangebeld en gevraagd}{de woning te zien}\\

\haiku{Liever vet smelten!...}{en biefstukken snijden tot}{hij er bij dood viel}\\

\haiku{{\textquoteright} {\textquoteleft}Altijd minder dan.}{een eigen appartement}{met een huishoudster}\\

\haiku{Die gingen ook zoo.}{moeilijk aan en uit over hun}{weeke slagersvingers}\\

\haiku{De R\'ev\'erende:}{M\`ere groette haar met een}{buiging van het hoofd}\\

\haiku{Ze zetten hun hoed.}{maar terug op toen ze aan}{het straathek kwamen}\\

\haiku{Het was hun of er,.}{een nieuw leven begon op}{een hoogeren trap}\\

\haiku{Ik dacht dat we daar...}{iederen dag naar de kerk}{zouden moeten gaan}\\

\haiku{In de Bermijnstraat.}{vroeg hen op een keer iemand}{in het Vlaamsch den weg}\\

\haiku{Benzamien was op,:}{het punt te antwoorden toen}{Jeroom ineens vroeg}\\

\haiku{eerste, tweede en,.}{derde klas naar het kostgeld}{dat men betaalde}\\

\haiku{Mijnheer B. Van Snick,..., '.}{Beenhouwer of Boucher als}{het int Fransch was}\\

\haiku{Van het drama dat?}{zich voor haar persoon voltrok}{in hun gezond hart}\\

\haiku{Hij hoorde de deur,...}{opengaan hij trok den handdoek}{van zijn gezicht weg}\\

\haiku{De R\'ev\'erende.}{M\`ere Sup\'erieure}{stond in de kamer}\\

\haiku{Haar week wit gezicht,.}{bleef daarbij even koud vroom en}{ontoegankelijk}\\

\haiku{Postcheckrekening) -.}{Nr 49911 bijgestaan door twee}{gewone priesters}\\

\haiku{Benzamien was niet.}{geheel zeker van zijn stuk}{met zijn vertaling}\\

\haiku{De barones de.}{Haricourt zei dat hij un}{humoriste was}\\

\haiku{Die roode sjawl op.}{de bank was van madame}{Van Holleb\`eque}\\

\haiku{Eens zat ze met haar.}{broertje op het grasplein voor}{het huis te spelen}\\

\haiku{Jeroom waagde nu,:}{toch een kleine opmerking}{bijna fluisterend}\\

\haiku{Van de familie.}{waren er hoop en al twee}{menschen gekomen}\\

\haiku{Een Spanjaard heeft dat, '.}{geschreven maart is even}{waar voor Anderlecht}\\

\haiku{De teerlingbak is. ...}{ons hart waarin het noodlot}{zijn dobbelsteenen gooit}\\

\haiku{De oogen van Jeroom,,.}{Meulepas waren ijskoud}{hard ongenadig}\\

\haiku{Het dikke blauwe.}{wangvleesch trok neerwaarts over}{zijn jukbeenderen}\\

\subsection{Uit: Kobeke}

\haiku{Bruu Kalot put een,.}{akerke water en laat zijn}{twee honden drinken}\\

\haiku{3 Die vader van,,.}{Kobeke och wat is dat}{een aardige vent}\\

\haiku{'t Wordt zoo erg dat.}{Broos het niet meer uithouden}{kan en hardop lacht}\\

\haiku{Pardoes draait op zijn,,.}{rug de pooten omhoog en}{krinkelt van plezier}\\

\haiku{Ze moeten alle,.}{twee lachen want dat was al}{lang afgesproken}\\

\haiku{Broos is al achter,.}{den eikenkant op weg naar}{huis als hij nadenkt}\\

\haiku{Ze loeren van tijd,.}{tot tijd eens om om den weg}{niet te vergeten}\\

\haiku{Boven de bosschen.}{komt het groote witte gezicht}{van de maan loeren}\\

\haiku{Pardoes dribbelt met,.}{zijn kop naar den grond zijn staart}{tusschen zijn pooten}\\

\haiku{De menschen loeren,,:}{van achter het gordijn en}{lachen en zeggen}\\

\haiku{{\textquoteleft}Gedoopt is gedoopt{\textquoteright},, {\textquoteleft}....}{zegt hijen ik doe geen twee}{missen voor een geld}\\

\haiku{Ze knoopt heur jakske '.}{los en geeft Kobeke en}{Nelleken mem}\\

\haiku{Bellemoeike wordt.}{wakker geschud en ze weet}{niet meer waar ze is}\\

\haiku{Maar Melle Komfoor.}{was ook wat locht in den kop}{van al die drupkens}\\

\haiku{Hij zal 't schaap zijn{\textquoteright},.}{nagelband nog doen springen}{zegt Bellemoeike}\\

\haiku{Hij trekt zijn ooren.}{achteruit en likt nu en}{dan eens over zijn snuit}\\

\haiku{Lulle-Mie kijkt.}{van uit heur stalleke door}{de spleet van de deur}\\

\haiku{Dat is alles wat '. '}{er over zoon onnoozel schaap}{te vertellen is}\\

\haiku{{\textquoteright} {\textquoteleft}Zeker, m'ne jong,.}{en dan kookt Tekla Penne}{daar hutsepot van}\\

\haiku{{\textquoteright} 't Is te zien dat.}{Kobeke en Nelleke}{het benauwd krijgen}\\

\haiku{Ze treiteren den}{koster omdat hij al twee}{keeren buiten gaan zien}\\

\haiku{Mieke Lies krest van ',.}{t giechelen precies of}{ze gekitteld wordt}\\

\haiku{Kajoet zit neven,.}{het vuur te slapen zijn oogen}{vast toegeknepen}\\

\haiku{Tegen den gevel,,.}{van de hut op een hoopke}{slommer ligt Kajoet}\\

\haiku{Dat konijn had hij,,.}{met een zuur gezicht zonder}{smaak opgevreten}\\

\haiku{Van dichterbij klinkt:}{nu de minneklagende}{lokroep van Purre}\\

\haiku{Lulle-Mie heeft.}{op iederen horen een}{dikke raap steken}\\

\haiku{{\textquoteleft}Nee Nelleke, maar.}{me docht precies dat er weer}{iemand op me riep}\\

\haiku{{\textquoteright} Broos is met hak en '.}{bijl aant struiken uitdoen}{tegen de zandbaan}\\

\haiku{Zijn broek zit er vol,.}{van en het kriebelt hem nog}{over zijn blooten rug}\\

\haiku{Ze lezen in hun.}{boek iets over paddestoelen}{en dolle kervel}\\

\haiku{Hij ziet de sneeuw op.}{de haag en op de zwik van}{de putkuip liggen}\\

\haiku{Hier en daar lukt het.}{toch en krijgen ze een cent}{of een koekske}\\

\haiku{Het raam van den hof,.}{staat open en de vogelkens}{zingen zijn oogen toe}\\

\haiku{Bellemoeike weet.}{van niks en leest maar voort aan}{heur paternoster}\\

\haiku{De jongens wringen.}{hun achterwerk efkens over}{den stoel weg en weer}\\

\haiku{ik zou dien Golo.}{de oogen uit zijn leelijken}{kop gekrabd hebben}\\

\haiku{Hij lacht niet zoo veel,.}{meer en ze gaan allang niet}{meer gelijk zwemmen}\\

\haiku{Ze gaan nog dikwijls '.}{gelijk hout of denappels}{rapen int bosch}\\

\haiku{En op een schoonen.}{dag in den vroegen voortijd}{komt Dorusoome binnen}\\

\haiku{Hij had niet gepeinsd.}{dat het zoo erg zou geweest}{zijn bij Nelleke}\\

\haiku{Hij meent dat hij een.}{gezicht moet zetten als een}{halve heilige}\\

\haiku{Broos kan Nelleke,,.}{ginder tegen dien boom niet}{uit zijn kop zetten}\\

\haiku{Ze zeggen dat die,.}{iederen dag vleesch eten}{en vrijdags stokvisch}\\

\haiku{Dat is een nieuwe,....}{Vader Abt die komt recht van}{den Paus van Rome}\\

\haiku{Had iemand van de?}{Broeders hem ooit gezien met}{een stuk in zijn kraag}\\

\haiku{Maar zelfs onder die.}{mis kondt ge hooren dat ze}{niet akkoord waren}\\

\haiku{Ge prakkezeert over,.}{alle dingen en ge weet}{niet van waar het komt}\\

\haiku{'t Was de tijd niet.}{meer dat de merels floten}{in den vollen dag}\\

\haiku{Maar ineens hoorde.}{hij ze zoo fel dat al de}{rest niet meer bestond}\\

\haiku{Die klanken botsten.}{achter tegen zijn kop lijk}{of er iemand sloeg}\\

\haiku{Al van er naar te.}{kijken deden Broederke}{Kobus zijn oogen zeer}\\

\haiku{En hij is er nu.}{zeker van dat hij geen kwaad}{doet met weg te gaan}\\

\haiku{Een vrouwmensch  stond, '.}{in haar hemd voor den open haard}{bezig aant vuur}\\

\haiku{Hij hielp nu mede,}{de kar laden en als ze}{de vracht had pakte}\\

\haiku{Over het klooster van.}{Zeveslote was het}{nu heimelijk stil}\\

\haiku{Wat had hij nu met?}{zijn heiligheid en zijn goed}{voorbeeld uitgehaald}\\

\haiku{{\textquoteright} Kobeke met zijn.}{pakske in zijn hand stond wat}{bedutst voor het wijf}\\

\haiku{{\textquoteright} Kobeke verstond,.}{dat niet en hij witte voort}{tot tegen den avond}\\

\haiku{Het jong ding zat al.}{bekanst heelegaar op den}{schoot van den koetsier}\\

\haiku{Een vrouwmensch, het jong,.}{ding docht hem antwoordde ook}{met een dubbel tong}\\

\haiku{Hij deed het met een.}{gezicht om iemand ineens}{nuchter te maken}\\

\haiku{Maar hij zag er niet.}{zooveel kwaad in omdat het}{een pastoorsmeid was}\\

\haiku{Ze hooren uwen hond ' '.}{wel bassen maart is of}{zet niet hooren}\\

\haiku{Hij luisterde met.}{al zijn attentie naar den}{man op het verhoog}\\

\haiku{Kobeke stond recht,,.}{kruchte van de pijn en ging}{door dit gat loeren}\\

\haiku{Een er van was de}{vieze werkman die den avond}{te voren neven}\\

\haiku{Hoort nu en ziet, hier '.}{begint nu echt het ende}{vant verhaal}\\

\haiku{Kobeke stond een.}{moment in de straat naar links}{en rechts te kijken}\\

\haiku{Straat uit en straat in,,.}{altijd rechtdoor naar den kant}{waar de zon opkomt}\\

\haiku{'t Is waar, het is,.}{zondag vandaag hij had dat}{bekanst vergeten}\\

\haiku{Rond den noen kwamen.}{ze in een parochie waar}{hij in den trein moest}\\

\haiku{{\textquoteright} Onder de wortels.}{van dien zelfden den liep een}{konijn door zijn pijp}\\

\subsection{Uit: Namen 1914}

\haiku{het is te koud en,.}{we zijn te plots opgejaagd}{uit onze nachtrust}\\

\haiku{{\textquoteleft}Zwijg, sergeant, on.}{ne fait pas d'omelettes}{sans casser des oeufs}\\

\haiku{Het is hoogstens tien,.}{minuten ver maar het is}{een akelige tocht}\\

\haiku{Ik weet niet wat het,.}{is maar daar is iets dat ik}{nog moet bijvoegen}\\

\haiku{{\textquoteright} Beneden in het.}{dal van Gelbress\'ee begint}{een huis te branden}\\

\haiku{Ik begrijp maar niet.}{waarom die batterij daar}{verlaten blijft staan}\\

\haiku{Ditmaal zijn het geen,.}{kleine groepjes maar een dicht}{gesloten bende}\\

\haiku{Hij kijkt even op, en:}{fluistert mij zachtjes toe als}{tot een oude vriend}\\

\haiku{'t Is een jonge,.}{kerel met fijne handen}{en een wit gezicht}\\

\haiku{Over zijn achterhoofd.}{en zijn hals loopt het bloed neer}{op zijn  ransel}\\

\haiku{Bruno komt eveneens.}{de gewonden verzorgen}{achter het huisje}\\

\haiku{Daar liggen nu meer,.}{gewonden in zijn loopgraaf}{en ook meer doden}\\

\haiku{Hij doet nog eens de,,:}{ogen wijd open staart mij vlak in}{het gezicht fluistert}\\

\haiku{Ik adem met wellust.}{de frisse lucht in en word}{bijna misselijk}\\

\haiku{Ik word duizelig,.}{langzaam aan onbewust van}{de plaats waar ik ben}\\

\haiku{Hij ziet er deerlijk,,.}{uit met zijn kortgeknipte}{haren zijn rond hoofd}\\

\haiku{Dan vraagt hij of er.}{niemand een kaart van de streek}{in zijn bezit heeft}\\

\haiku{{\textquoteleft}Trek je  toch niets.}{aan van wat die melkbaarden}{je daar vertellen}\\

\haiku{De ene wagen houdt,.}{stil na de andere vlak}{voor de ingangshal}\\

\haiku{Hij roept nog veel meer,.}{telkens de herhaling van}{dezelfde woorden}\\

\haiku{Wij zijn even goede...}{patriotten als gij en}{ik ben ook beschaamd}\\

\haiku{Het is een avond van,.}{bijster bange verschrikking}{en naar als een moord}\\

\haiku{{\textquoteright} {\textquoteleft}In het klooster, daar,,.}{achter die muur meen ik in}{het dorp zeker niet}\\

\haiku{{\textquoteleft}Moeten die arme?}{mensen nu de hele nacht}{in de kerk blijven}\\

\haiku{Ik leun tegen een.}{stoel en moet zin voor zin zijn}{toespraak vertolken}\\

\haiku{Het dienstpersoneel...}{moet eveneens de nacht in de}{kapel doorbrengen}\\

\haiku{Vele gewonden,,.}{vooral verbranden zijn in}{de nacht gestorven}\\

\haiku{hij schudt het hoofd heen:}{en weer en zegt nu en dan}{op hoog-ijle toon}\\

\haiku{{\textquoteright} komt hij mij nog eens,:}{knipogend nafluisteren en}{ik zeg in mezelf}\\

\haiku{Een enkele keer.}{hoor ik door het openstaande}{raam iemand zingen}\\

\haiku{Enkele dagen.}{later vertrekken de twee}{Belgische dokters}\\

\subsection{Uit: Oorlogsnovellen}

\haiku{Het is van ruw en,. :}{ongeschaafd hout maar het is}{mijn eenig eerekruis}\\

\haiku{-: Ik zat tegen den,.}{gevelmuur en Saelens lag}{languit op den grond}\\

\haiku{Hun stemmen waren. :}{zacht en kalm als de stilte}{van den zomernacht}\\

\haiku{{\textquoteleft}Il n'y a pas de,.}{consommations \`a dix}{centimes m'sieu}\\

\haiku{De kommandant kreeg '. :}{ern gegeneerden blos}{van op de wangen}\\

\haiku{A la guerre,.}{comme \`a la guerre}{zegt de Franschman}\\

\haiku{Hun handen waren;}{verweerd en hun gezichten}{gebruind door de zon}\\

\haiku{'t Moedertje deed.}{hare schoenen uit en liep}{op hare kousen}\\

\haiku{{\textquoteleft}Marcus, mijn jongen,!}{toch en hebben ze u zoo}{maar doodgeschoten}\\

\haiku{ons als van verre. :}{gezien door tegen den grond}{liggende wachters}\\

\haiku{Onze oogen deden '.}{pijn vant strakke staren}{in de duisternis}\\

\haiku{De Franschen hebben.}{een verbazend gemak om}{boeken te schrijven}\\

\haiku{Ik bleef en ben steeds {\textquoteleft}{\textquoteright}.}{in Mammy's oogenle cousin}{\`a M'sieu Raoul}\\

\haiku{Maar wanneer zij hem.}{aanspreken schrikt Vadertje}{Musset schichtig weg}\\

\haiku{Dan weet hij niet, o,.}{neen dan begrijpt hij niet wat}{het allemaal is}\\

\haiku{Ik vraag hem wat het,. :}{is hij trekt even de schouders}{op en gaat verder}\\

\section{Willem de Clercq}

\subsection{Uit: Woelige weken: november-december 1813}

\haiku{Men zegt dat de prins.}{van Eckm\"uhl heeft moeten}{capituleren}\\

\haiku{Men zegt dat er te.}{Utrecht een korps van 16 000 man}{gevormd moet worden}\\

\haiku{Onder vrienden legt.}{men Franse pakkers deze}{woorden in de mond}\\

\haiku{Niets, zie daar het feit.}{dat wij moeten toevoegen}{aan onze nieuwtjes}\\

\haiku{Vandaag was alles,.}{vervuld van neerslachtigheid}{van ontmoediging}\\

\haiku{Op de Leliesluis,,.}{houdt mij iemand in een kraag}{gewikkeld staande}\\

\haiku{Dit is geen lachend.}{beeld door het spelend vernuft}{eens dichters gevormd}\\

\haiku{dat het getal der.}{vrijwilligers in Den Haag}{zeer groot geweest is}\\

\haiku{Men geeft in Den Haag, -.}{aan elke vrijwilliger}{f 100 handgeld}\\

\haiku{Men zei dat morgen.}{onze ware regering}{zou worden gemaakt}\\

\haiku{Hij zat op een klein,.}{paard had een blauw buis aan en}{droeg een grote lans}\\

\haiku{Wij zullen nu weer,.}{een volk en de prins onze}{soeverein worden}\\

\haiku{Gisteren is hier.}{een bende Russisch voetvolk}{binnen getrokken}\\

\haiku{Klijn, Lierzangen van.}{Helmers Klijn moest in Felix een}{spreekbeurt vervullen}\\

\haiku{Kemper zei hierover.}{te zullen denken en reed}{naar Leiden terug}\\

\haiku{Bij de heer Collot.}{d'Escury ontvangen zij}{dezelfde boodschap}\\

\haiku{Kemper verklaarde.}{dat niets hem zo speet als het}{geval van Truguet}\\

\haiku{Harpe, J.F. de la (-) -,.}{17391803 Frans criticus en}{dichter hoogleraar}\\

\haiku{Scaevola = die.}{de linker in plaats van de}{rechter hand gebruikt}\\

\section{Alfons de Cock}

\subsection{Uit: Herdenkings-album 1850-1921}

\haiku{Alfons de Cock-}{Herdenkings-album 18501921}{Colofon}\\

\haiku{hij begon te 5 ' '.}{uurs avonds en eindigde}{te 2 uurs nachts}\\

\haiku{In zijn vertrouwde.}{omgeving was hij vooral}{te bewonderen}\\

\haiku{- Mevrouw Courtmans, geb. ( -).}{Johanna Desideria}{Berchmans1811 1890}\\

\haiku{De Molenaar, de.}{wever en de kleermaker}{uit de volksmeening}\\

\haiku{Gids15, jg. 1905, blz. - -.}{155 169 en 245 284.1906Het Liedje}{van Heer Halewijn}\\

\haiku{207 - 208.1911De Macht der.}{Kinderlijke Onschuld in}{de Sagenwereld}\\

\haiku{En 't en is maar '.}{een te kort Ent is Pe}{Joenens rostekop}\\

\haiku{{\textquoteleft}Keizer Karel mee.}{al zijn knechten Ging tegen}{de Franschen vechten}\\

\haiku{twee zijn Westvl, en,,,.}{komen voor in Rond de Heerd}{I 34 en III 32}\\

\haiku{{\textquoteleft}Heere Koning, die ',!}{ins Gravenhage zijt}{vervloekt zij uw naam}\\

\haiku{Van heden af{\textquoteright} zei, {\textquoteleft};}{toen de Heerontneem ik de}{boomen alle spraak}\\

\haiku{de goedhartige}{meid eener gravin had juist}{een hongerigen}\\

\haiku{{\textquoteright} - {\textquoteleft}Neen,{\textquoteright} antwoordde het, {\textquoteleft}.}{meisjevoor mijn meester wil}{ik alles wagen}\\

\haiku{Eer zij vertrekt, streelt.}{zij met haar vingers door den}{knaap zijn kroezelkop}\\

\haiku{Deze draalt niet, om.}{den getroffen persoon te}{komen onttooveren}\\

\haiku{tous mes cheveaux sont,{\textquoteright}.}{tomb\'es et voil\`a pourquoi je}{porte un bonnet}\\

\haiku{le guerrier.}{\`a la t\^ete d'or avait de}{nouveau disparu}\\

\haiku{Peu apr\`es une f\^ete:}{solennelle devait avoir}{lieu \`a la cour}\\

\haiku{je ferai passer}{tous les invit\'es devant}{moi et chacune}\\

\haiku{vaste \'etendue d'eau,).}{o\`u le pers\'ecuteur se}{noie avec son chameau}\\

\haiku{16) SEBILLOT, C.,, (:}{de la Haute-Bretagne}{III no 13brosse}\\

\haiku{un loup, un ours, un ().}{renardici il y a 3}{courses distinctes}\\

\haiku{27) KRAUSS, Sagen und,, (:}{Maerchen der S\"udslaven}{I no 89brosse}\\

\haiku{31Die Evangelien,,}{vanden Spinrocke 3e dag}{3e cap. 32Volgens}\\

\haiku{relat{\'\i}ves \`a...,.}{l'anatopiie \`a la}{m\'edecine 304}\\

\haiku{106Le h\'eros l\`eve,.}{l'\'ep\'ee il ne la jette}{pas derri\`ere}\\

\subsection{Uit: Vlaamsche sagen uit den volksmond}

\haiku{{\textquoteleft}Ja maar, meneer, dat,,{\textquoteright}.}{is geen vertelsel zulle}{dat is waar gebeurd}\\

\haiku{Een jongeling van.}{Kuntsem3 maakte het hof aan}{een beeldschoon meisje}\\

\haiku{- {\textquoteleft}Weet gij,{\textquoteright} vroeg hem op, {\textquoteleft}?}{zekeren dag een vrienddat}{gij met een heks vrijt}\\

\haiku{Sagenboek, I, 7-),.}{11 nl. de terugkeer van}{vrijer en vrijster}\\

\haiku{de bewoners van {\textquoteleft}{\textquoteright}.}{Iddergem worden dan ook}{Toovereers genoemd}\\

\haiku{WOLF, D. M\"archen,,,, {\textsection}.}{u. Sagen nr 159 WUTTKE D.}{Volksaberglaube 415}\\

\haiku{De zuster van den.}{jongeling ging een wijzen}{pater raadplegen}\\

\haiku{Bevend van schrik, wist;}{hij niet wat doen en liet de}{katten maar begaan}\\

\haiku{{\textquoteleft}Morgen zal ik een{\textquoteright}.}{kapmes meebrengen en eene}{haar beenen afsmijten}\\

\haiku{Het duurde niet lang.}{of hij zag de zwarte kat}{weer eens afkomen}\\

\haiku{Op eens springt er een,.}{schoone gevlekte kat uit}{een kreupelboschje}\\

\haiku{De weduwe legt,.}{de zaak uit zooals zij gestaan}{en gelegen is}\\

\haiku{{\textquoteright} - {\textquoteleft}Kom maar eens mee naar,,}{den stal en gij zult zien welk}{schoon wit paardeken}\\

\haiku{Een vrouw van Oelter.}{was naar Ninove drie}{brooden komen koopen}\\

\haiku{Er was toove ij ',,.}{int spel dat was zeker}{zei men eenparig}\\

\haiku{Wanneer de boer in.}{den morgen den betooverde}{v\'o\^or de staldeur vond}\\

\haiku{het voert zelfs al het,.}{geld mede dat er mede}{in aanraking komt}\\

\haiku{Hij bemerkte een.}{vrouwmensch op zijn stuk land en}{liet haar betijen}\\

\haiku{Steekt dit onder den,.}{dorpel van de deur dan kan}{zij niet meer binnen}\\

\haiku{{\textquoteright} De ouders van 't;}{kind waren in den hoogsten}{hemel van blijdschap}\\

\haiku{Intusschen werd de;}{zieke onderzocht door een}{tweeden geneesheer}\\

\haiku{Intusschen was de.}{zoon de Jezu{\"\i}eten van}{Aalst gaan raadplegen}\\

\haiku{- {\textquoteleft}Ge moet,{\textquoteright} zei ze, {\textquoteleft}een,.}{keermis laten doen om het}{paard te doen weerkeeren}\\

\haiku{het was de hond, en.}{den ganschen nacht moest hij er}{mede rondloopen}\\

\haiku{Dit diertje werd zoo,.}{groot dat hij er op den duur}{wel kon op rijden}\\

\haiku{Hij sprong met zijn twee.}{voorpooten op haar schouders}{en lachte luidop}\\

\haiku{Men bewaakte het,;}{paard en toch was het haar des}{morgens gevlochten}\\

\haiku{Maar de zoon gedroeg.}{zich uiterst slecht en deed niets}{anders dan jagen}\\

\haiku{Zij hebben maar te,.}{willen en onmiddellijk}{wordt hun wensch vervuld}\\

\haiku{En allen zetten,.}{zich aan tafel en aten en}{dronken smakelijk}\\

\haiku{{\textquoteright} 's Avonds was Jan in.}{de kerk en ging zich op den}{preekstoel verstoppen}\\

\haiku{De veldwachter stierf.}{evenwel zonder rechterlijk}{vervolgd te worden}\\

\haiku{M... kwam verleden,,.}{jaar op een Zaterdagavond}{van zijn werk terug}\\

\haiku{{\textquoteright} ~ (Wichelen.) ~ .}{Lezing b doet meer aan de}{mare denken}\\

\haiku{In Munkzwalm wijst een,.}{stalkaars de plaats aan waar een}{schat verborgen ligt}\\

\haiku{Steekt men den vinger,;}{naar een stalkaars uit dan komt}{zij er opzitten}\\

\haiku{toen bevond hij zich,.}{in Haasdonk anderhalf uur}{van zijn woning of}\\

\haiku{In Volkskunde, II, -,,-;}{89 vlg. Vgl. Wodana 23 vlg.}{verder blz. 154156}\\

\haiku{Men vertelt, dat er.}{in een hol te Lebbeke}{kabouters woonden}\\

\haiku{Hieruit blijkt tevens.}{hoe sagen afslijten en}{zich vervormen}\\

\haiku{{\textquoteright} - {\textquoteleft}J a, zeker,{\textquoteright} zei, {\textquoteleft}}{de weveren ik durf het}{u wel zeggen ook}\\

\haiku{Over het thema, {\textquoteleft}de{\textquoteright},.}{Duivel als kaartspeler zie}{men DE COCK-TEIRL}\\

\haiku{Ik heb het door mijn.}{grootmoeder wel honderd maal}{hooren vertellen}\\

\haiku{Vgl. ons volg. nr en.}{zie de aldaar gegeven}{referencies}\\

\haiku{er waren drie groote,,.}{merken als ingebrand op}{de deur te zien}\\

\haiku{Standaert, van Haaltert,.}{ging op een Zondag morgen}{met een vriend naar huis}\\

\haiku{daar bevond zich het,.}{huis van Schallekens en hier}{vluchtte hij binnen}\\

\haiku{uw eerste werk van,.}{dezen morgen zult gij heel}{den dag voortzetten}\\

\haiku{- {\textquoteleft}Uw eerste werk van.}{dezen morgen zult gij heel}{den dag voortzetten}\\

\haiku{) ~ Wordt ook verteld,.}{te Aalst te Wichelep en te}{Adegem Vgl. Wodana}\\

\haiku{Vgl onze nrs 235,,,.}{237 244 en 260 alsmede}{A. de COCK en IS}\\

\haiku{Vgl. onze nrs 224,,,,;}{en 226 SCHOUTENS Maria's}{Henegouw 54 130}\\

\haiku{Te Nevele, dicht {\textquoteleft}{\textquoteright},,.}{bij hetdorp ligt een weide}{de Oostbroek genaamd}\\

\haiku{Een tweede maal werd,.}{de deur geopend doch met}{hetzelfde gevolg}\\

\haiku{- {\textquoteleft}Houd u aan dit strooitje,;}{vast ik zal u tegen den}{donder beschermen}\\

\haiku{Sint-Leu ging binnen,.}{en vroeg aan den smid of hij}{bij hem werken mocht}\\

\haiku{(Zegelsem.) ~ Vgl.,,;}{P. DE MONT en A. de COCK}{Vlaamsche Vert. blz. 364}\\

\section{Frans Coenen}

\subsection{Uit: Burgermenschen}

\haiku{Zij trok hem aan 't,.}{handje mee den kruiwagen}{voor zich uitstuwend}\\

\haiku{Je leefde hier toch '...}{zoo ondert uitschot van}{de samenleving}\\

\haiku{En de trekken van.}{zijn schamelen kop stonden}{zorgelijk gegroefd}\\

\haiku{Hij zou d'r 's even,!}{een standje maken as ze}{terug-kwam}\\

\haiku{Maar ondertusschen '.}{bleef die Leen maar weg ent}{was al over vieren}\\

\haiku{as-ti dat hoort '...}{is t'r heelemaal geen huis}{metm te houen}\\

\haiku{En ze kleedde 'm '... '!}{toch zoo graag int witt}{Stond hem zoo lekker}\\

\haiku{En z\'o\'o aardig was.}{dat vrouwtje waarempel ook}{niet altijd voor d'r}\\

\haiku{Kom, gekheid,... ze zou '...}{t met een balletje straks}{wel weer goed maken}\\

\haiku{Maar ook die kleine.}{plek van rustiging wilde}{allengs verzinken}\\

\haiku{Het was al lang over,.}{en weer afgesproken maar}{telkens uitgesteld}\\

\haiku{Afijn... als de jongen...}{dan nog maar iets m\'e\'er werd dan}{zij geweest waren}\\

\haiku{Lena hoorde Truus'.}{zeurige verhalen over}{haar huishouen toe}\\

\haiku{Dat eeuwig grootdoen ',...}{vanm daarin was-i net}{zoo erg als z'n broer}\\

\haiku{Als ze 't niet om, '!...}{Truus liet en om de herrie}{dan zou zem toch}\\

\haiku{Het bloed schoot haar in,.}{de wangen terwijl ze hem}{van terzij aanzag}\\

\haiku{blaakte fel-wit, met,.}{het dood-strak keukenraam}{en de stille deur}\\

\haiku{Ze is knap, meer niet... ',...}{ent is een gezicht dat}{ook erg gauw verveelt}\\

\haiku{Binnen den rand van,.}{den stoel zag zij hem komen}{langzaam voortstappend}\\

\haiku{wat stak 'r nou voor ' '?}{kwaad in dat zoo'n jongens}{int water ging}\\

\haiku{Zoo haastten ze dan,.}{voort op niets lettend dan op}{hun eigen loopen}\\

\haiku{Tante Griet was al:}{v\'o\'or hen thuisgekomen en}{haar eerste vraag was}\\

\haiku{Je h\'et gelijk, Griet...,{\textquotedblright};}{je het schoon gelijk stemde}{ze benepen toe}\\

\haiku{Maar zij begreep 't, '.}{wel dat de jonge ant}{water zou wezen}\\

\haiku{Laat je los, laat je,... '...!}{dat kind los riep ze maar En}{t dee w\`at zeer no\`u}\\

\haiku{de hand d'r op zou,....{\textquoteright} {\textquoteleft}... '}{willen geven dat-i}{Och gofferdomme}\\

\haiku{Neem u er maar geen...,...}{notitie van heusch dat}{heb ik veel liever}\\

\haiku{{\textquoteright} haastte zich mama,,.}{verschrikt en verontwaardigd}{te protesteeren}\\

\haiku{... wat kan u dat nou,... '}{schelen wat er precies is}{voorgevallen als}\\

\subsection{Uit: In duisternis}

\haiku{En hij trachtte niet,.}{meer te denken met ogen dicht}{weer in te dutten}\\

\haiku{Hij was nu klaar en ':}{bezag zich nog eens scherp in}{t hangspiegeltje}\\

\haiku{Halfbewust onder ',...}{t denken had hij de kast}{horen openkraken}\\

\haiku{Een beetje verdoofd.}{trad hij voort over de straat vol}{glimmende plassen}\\

\haiku{Maar hij begon pas,.}{dit leven was ook voor iets}{beters opgevoed}\\

\haiku{Daar werd eerstens in.}{een sigarenwinkel een}{bediende gevraagd}\\

\haiku{Waar moest hij gaan over,?}{een paar dagen als hij zijn}{kamer ontruimd had}\\

\haiku{Vrijheid was toch wel,...!}{iets heerlijks eigenlijk iets}{absoluut nodigs}\\

\haiku{{\textquoteright} hijgde hij tegen, {\textquoteleft} '.}{haar inu deet expres}{om mijn te pesten}\\

\haiku{sal u de boeken ', '?}{weer int hangertje doen}{of mot ikt doen}\\

\haiku{Hij had zonder smaak,.}{gegeten schoon hij wel een}{kwartje verteerd had}\\

\haiku{Nee, dan toch niet op, '...}{die kamer en niet z\'o als}{hijt wilde doen}\\

\haiku{De jongeman zag.}{een poos door het glas heen en}{kreeg er eetlust van}\\

\haiku{in de groeiende,.}{verlegenheid in benauwd}{lichaams-voelen}\\

\haiku{maar ze hielden zich,...}{liever onnozel die er}{van profiteerden}\\

\haiku{... \'e\'en het er s'n oor,.}{beseerd en \'e\'en een snee over}{se wang geloof ik}\\

\haiku{Hij wilde er niet,.}{heen en wist tegelijk dat}{hij er komen zou}\\

\haiku{Allengs verzonk hij,,.}{al lopende weer in de}{vorige dofheid}\\

\haiku{Hij wachtte, de deur,.}{halfopen met een hand drukkend}{zijn jagende borst}\\

\subsection{Uit: Onpersoonlijke herinneringen}

\haiku{Men presenteerde.}{ons te koop veele vogels uit}{Egijpte en Afrika}\\

\haiku{Maar dan is er 's.}{Zondags een onverwachte}{kink in den kabel}\\

\haiku{Talloze menschen.}{en Corporaties buiten}{de Militaire}\\

\haiku{Bovendien heeft zij,.}{niets wezenlijks te doen zoo}{min als haar ouders}\\

\haiku{Zoo is Louise voor.}{den Vader eigenlijk een}{teleurstelling}\\

\haiku{onverschillige,,.}{verbazing en verachting}{als voor een zonde}\\

\haiku{dit liet gebeuren,,.}{zonder aanmoediging maar}{ook niet afstootend}\\

\haiku{En op dat boord staat,.}{het hoofd het gezicht diep en}{gewichtig peinzend}\\

\haiku{Er viel niets aan Le,.}{Roy meer te helpen zoo het}{al ooit had gekund}\\

\haiku{Rouaan, Chartres,,,.}{Amiens Saint Quentin om er de}{kerken te bezien}\\

\haiku{Iets, dat hem van zich?}{zelf af leidde en zijn}{leven een doel gaf}\\

\haiku{Maar zij geloofde,.}{eigenlijk niet dat hij zoo}{zelfbewust leefde}\\

\haiku{Er was niets meer te.}{verwachten dan een wellicht}{bezwaarlijk einde}\\

\haiku{Zoo wilde zij hem.}{dan nu nog recht doen en zijn}{naam vereeuwigen}\\

\haiku{Dat gebeurde toch,.}{al  vaak als het middel}{te vroeg uitwerkte}\\

\subsection{Uit: Reizen, een uitweiding en inwijding}

\haiku{En dan is er al,,:}{spoedig op eenigen afstand}{een stadje in zicht}\\

\haiku{Een nieuw mensch gaat op}{nieuwe wegen temidden}{van nieuwe dingen}\\

\subsection{Uit: Studies}

\haiku{Rillend stapte hij,.}{voort zoveel mogelijk de}{plassen vermijdend}\\

\haiku{hoe laat 't was, want.}{hij kon zich heel geen denkbeeld}{van de tijd maken}\\

\haiku{Enige ogenblikken, '.}{lag hij zo stil maar spoedig}{kreeg hijt te warm}\\

\haiku{Zij stond snel op en,.}{kwam tot hem die de armen}{naar haar uitstrekte}\\

\haiku{Kalm lag Paul daar nu,,.}{met de grote ogen wijd open}{zonder te ijlen}\\

\haiku{Meneer Dalman zelf,:}{was er maar kort in geweest}{jaren geleden}\\

\haiku{en nu vulde dat.}{zijn hele hoofd en trok zijn}{denken naar binnen}\\

\haiku{Als 't vandaag niet,.}{beter ging moest er morgen}{een dokter komen}\\

\haiku{Toen aten zij in de,.}{kamer die hij vroeger het}{zaaltje had genoemd}\\

\haiku{Mevrouw schreide nog,,.}{altijd wanhopig haar kracht}{uitgeput voelend}\\

\haiku{Dat zeker wel, want,...}{anders zou hij toch niet in}{bed zijn bij daglicht}\\

\haiku{Schuw rondblikkend, zag,...}{hij in de kamer waar de}{schemering waarde}\\

\haiku{om hulp roepen, 't,,...}{bed uitspringen de kamer}{uit naar beneden}\\

\haiku{een aanhoudende.}{wisseling van verwachting}{en teleurstelling}\\

\haiku{Ja, Vermeer had hen,...}{al gezien hij zou straks bij}{hen komen zitten}\\

\haiku{Maar hier, ook hier, drong ',,.}{t hem iets te doen dat hun}{genot zou geven}\\

\haiku{zij gaan, soms luider,,.}{op heftiger alsof het}{paar v\'o\'or hen twistte}\\

\subsection{Uit: Vluchtige verschijningen}

\haiku{ja... dat kwam er van,,}{bijna zes jaren getrouwd}{en vier kinderen}\\

\haiku{As je in de bos,.}{was dan hadt je tenminste}{de dokter voor nies}\\

\haiku{t Was waarlijk een,.}{otium cum dignitate}{dien hij hier genoot}\\

\haiku{een oogenblik zwaar....?}{wordt van doffe droefheid bij}{het herdenken}\\

\haiku{En dit duurde drie,.}{weken zonder dat hij er}{veel verder mee kwam}\\

\haiku{Hel-geelend licht:}{het uit de druischende}{machinekamer}\\

\haiku{Ja, 's winters was '.}{t varen lang geen plezier}{en vaak hard genoeg}\\

\haiku{Het was dadelijk,,:}{als ze een paar uur over hun}{tijd kwamen de vraag}\\

\haiku{zee en bergen en,...}{stad het lag alles roerloos}{en zeer verlaten}\\

\haiku{Hier en daar lag een.}{landhuis rustig temidden}{van nog jong plantsoen}\\

\haiku{Wij kijken uit het,.}{coup\'e-raampje terwijl de}{trein nog roerloos staat}\\

\haiku{daar wonen zullen.}{misschien en al die punten}{van nabij kennen}\\

\haiku{Over het plein is de.}{vochtige frischheid van den}{opengaanden ochtend}\\

\haiku{Maar er gaat lichte.}{onrust onder de groep bij}{den stalingang}\\

\haiku{Men zit er veel te.}{dicht op en ziet altijd de}{menschen in hun rug}\\

\haiku{twee mannen en vier,.}{jongens altemaal goede}{Duitsche gezichten}\\

\haiku{Zij waren voor het.}{meerendeel zeer mistroostig}{en zenuwachtig}\\

\haiku{Het waren deze,.}{die den aanstootelijken naam}{van Smouzen droegen}\\

\subsection{Uit: Zondagsrust}

\haiku{terwijl Verhoef met,...}{de leege theekop wegging en de}{alkoofdeur openbleef}\\

\haiku{Zij gromde terug,....}{dat Verhoef zich t'r niet mee}{hoefde bemoeien}\\

\haiku{10 guldes voor 't,.......}{huishoue in de week dat was}{toch ordentelijk}\\

\haiku{... helderde haar stem,.}{met dat geaffecteerde}{klankje haar eigen}\\

\haiku{We motte spare...,...}{zei-di altijd maar voor}{de kwaje dage}\\

\haiku{Ze most 'n echte,......}{dame zijn en natuurlijk}{altijd handschoene}\\

\haiku{De verpleger en '.}{de meid same haddenm}{nie meer kenne houe}\\

\haiku{Maar affijn, d'r was '...}{toch in alle geval nog}{s variatie}\\

\haiku{Nou, dat kon zij niet,....}{bekostige dan moste}{ze d'r maar vrijhoue}\\

\haiku{Een andere vent ',...}{keek nogs na z'n vrouw om}{hoe ze d'r uitzag}\\

\haiku{- ik... maar hij keerde.}{zich schokschouderend af en}{nam z'n krant weer op}\\

\haiku{Onder het drukke, '....}{smakken vroeg ze Dirk of hij}{r ook nog van wou}\\

\haiku{- Zus... aarzelde de,,......?...}{moeder die haar vergeten}{had ja wil ze wel}\\

\haiku{Haar zelf klonk het vreemd,.}{terwijl Verhoef en het kind}{er van opschrikten}\\

\haiku{Toen zich heelemaal ' '.}{afwendend vant raam om}{naart kind te zien}\\

\haiku{Het kind stond al klaar,.}{bij de deur vraagoogend van}{vader naar moeder}\\

\haiku{- Ze komt zeker 's...}{kijke of d'r wat te snaaie}{valt met de Zondag}\\

\haiku{wat is 't al weer,?}{lang da'k niet bij me lieve}{kindere was niet}\\

\haiku{riep Verhoef haar na,:}{dan uitschaterend in een}{zenuwige lach}\\

\haiku{En dat ze d'r niks - -!...}{van gemerkt hadde geen van}{beie Kato ook niet}\\

\haiku{In elk geval had,.}{hij haar liever zoo dan dat}{ze een kop toonde}\\

\haiku{Toen Marietje de,.}{deur had toegetrokken bleef}{het boven even stil}\\

\haiku{En hij had juist geen,!...}{poot verzet altijd maar op}{een stoel gehange}\\

\haiku{Ja... daar most-i,!}{bij haar maar mee ankomme}{met zukke dinge}\\

\haiku{As zij maar mooie kleere,...}{en opschik koope kon dan ging}{er de rest niet an}\\

\haiku{- Neen 't waren wel {\textquoteleft}{\textquoteright}...,..!}{allemaalvrindjes en wat}{voor godbeware}\\

\haiku{Zij voelden zich heel.}{thuis in hun koffiehuizen}{in hun eigen stad}\\

\haiku{Ja, je weet niet wat...}{je beginne moet om zoo'n}{dag om te krijgen}\\

\haiku{Maar hij meende 't... ...}{zoo goed en hij geloofde}{haar zoo onschuldig}\\

\haiku{t was zoo'n lange,.}{jongen zoo interessant}{bleek en met zwart haar}\\

\haiku{Ben was wel een goeie.......}{jongen zoo hartelijk en}{hij meende het zoo}\\

\haiku{Maar hij kwam altijd, '...}{en als hij er dan was was}{t toch ook weer goed}\\

\haiku{Maar nu werd zij zich.}{haar gedachten bewust en}{haar blijheid was heen}\\

\haiku{Maar hij was niet chic,......}{voor geen cent en hij kon haar}{ook niet veel geven}\\

\subsection{Uit: Een zwakke}

\haiku{Maar soms hadden ze.}{nog niet gekookt en dan moest}{hij t\`och melk nemen}\\

\haiku{Hij stond op, legde.}{een dubbeltje neer v\'o\'or haar}{groette en ging heen}\\

\haiku{Bij het eene raam, in,:}{de voltaire was n\`og eene}{vage gestalte}\\

\haiku{Hij was eerst in de.}{vierde klas burgerschool en}{voor weinig bruikbaar}\\

\haiku{Het was intusschen,;}{stil geworden binnen een}{stilte vol woede}\\

\haiku{Knagend woelde en '.}{wroette in hemt oude}{leegte-gevoel}\\

\haiku{Als hij tenminste,.....}{maar kans had later chef te}{worden dan of neen}\\

\haiku{Johan was even geschrikt '.}{van dat leven buiten hem}{int kamertje}\\

\haiku{Maar terwijl ze de,.}{knop omdraaide sloeg al de}{glazen keukendeur}\\

\haiku{Het duurde lang eer.}{de koppen zichtbaar werden}{boven den traprand}\\

\haiku{Ik vin 't heel lief,...}{van u dat u nog aan me}{heb willen denken}\\

\haiku{- Dag lieve mevrouw, ',.}{nout ga u goed heel}{veel pleizier op reis}\\

\haiku{Een zwakkeling en.....}{een ijdeltuit en voor geen}{cent zakenkennis}\\

\haiku{- Maar ik geloof toch,,:}{niet dat ik ooit zoo kijf zoo}{leelijk schreeuw als jij}\\

\haiku{Hij schudde met een.}{ongeduldigen ruk haar}{arm van zijn schouder}\\

\haiku{Dat kwam zoo te pas ....}{en zou hem zoo'n gevoel van}{meerderheid geven}\\

\haiku{Een breede zee van.}{nutteloosheid lag er nu}{tusschen hem en hen}\\

\haiku{God, we zijn al om '.}{half twaalf begonnen ent}{is nou over half twee}\\

\haiku{dat-i wel wat...}{minder lawaai kon maken}{als-ti uitgaat}\\

\haiku{Mevrouw stond op en.}{begon langs de ramen heen}{en weer te loopen}\\

\haiku{- Ach, dan heb u ze...}{zeker onder een ander}{stapeltje gelegd}\\

\haiku{en dan krijgen wij,....}{nog de schuld dat we niet op}{u gepast hebben}\\

\haiku{Het was goed en groot....}{hier op aarde te wachten}{en stil te zijn}\\

\haiku{Mevrouw voelde zich,.}{ellendig benauwd in de}{kamer in haar stoel}\\

\haiku{Heel langzaam zonk de.}{groote wijzer naar omlaag van}{cijfer tot cijfer}\\

\haiku{Maandag dan, dan kon,....}{Woensdag de boel in de kast}{zijn of Donderdag}\\

\haiku{Als ze dan maar de.}{boel niet te veel vernielden}{in dat mangelhuis}\\

\haiku{En dan was 't er,.}{zeker ook nog duur ook en}{dat kon niet lijen}\\

\haiku{Hij moest toch maar 's,....}{wat tegen haar zeggen een}{praatje beginnen}\\

\haiku{Hij wilde dat voor,,.}{hem geheel om er mee te}{doen wat hem lustte}\\

\haiku{suste mevrouw Dirks,.}{die achter liep en moeite}{had mee te komen}\\

\haiku{Waarom heb u pa,?}{dan nooit gevraagd of \`u de}{boeken mocht houen}\\

\haiku{Een klein, laag geluid,:}{van wielknarpen naderde}{links van den grindweg}\\

\haiku{Overal hetzelfde.}{koude bleeke licht schemerend}{tusschen de dennen}\\

\haiku{Johan had dat standje.}{en dat air van stout kind heel}{aardig gevonden}\\

\haiku{En met de handen.}{vooruit en gretige oogen}{liep hij op haar toe}\\

\haiku{Wat zou ze niet van?}{hem denken en zou ze zich}{niet beleedigd voelen}\\

\haiku{Langs de klammige.}{boomstammen streepte de vocht}{een donkere lijn}\\

\haiku{- Seg 's, wat het je,...,!}{me no\`u uitgevoerd jongie}{dat ga\`at zoo niet hoor}\\

\haiku{ik wou u nog wel ..}{even spreken over wat u me}{gistermiddag zei}\\

\haiku{dat ging zoo gauw en, .......}{was zoo onverwacht dat ik}{geen tijd had enz. enz}\\

\haiku{Na den even-kijk,:}{ter zijde had de ander}{weer voor zich gezien}\\

\haiku{Hij wist niet of de,.}{steken verscherpten maar zij}{vermoeiden hem reeds}\\

\haiku{Hij meende dat de,....}{steken al minder werden}{veel dragelijker}\\

\haiku{ik heb met m'n oogen ',....}{gezien datt er was en}{ik zeg dat je liegt}\\

\haiku{ze bedankte er...}{toch voor van die meid ook nog}{een gunst te vragen}\\

\haiku{Als 't eens een list '?}{was van zoo'n meid om haart}{huis u{\`\i}t te krijgen}\\

\haiku{- Och, Betje, zou je?}{me een pleizier willen doen}{en niet meer zingen}\\

\haiku{Dat kwam niet te pas,... '}{eigenlijk haar moeder maar}{te laten stikken}\\

\section{Ed Coenraads}

\subsection{Uit: Eiland van geluk}

\haiku{Ze zeggen dat de, -.}{maatschappij niet deugt ik heb}{dat nooit gevonden}\\

\haiku{{\textquoteleft}Het is beter niets.}{vooruit te horen en z\`elf}{maar te beleven}\\

\haiku{{\textquoteright} {\textquoteleft}Tschudi, tja... wel 'n.}{vent waar wat bij zat en waar}{je mee kon praten}\\

\haiku{Op Palmzondag lag ';}{ern paar honderd meter}{daarboven nog sneeuw}\\

\haiku{Met besten groet ook, -.}{aan juffrouw Errina en}{de anderen Uwdw}\\

\haiku{Dat had hij toch in.}{dienst als menagemeester}{ook altijd gedaan}\\

\haiku{En als Luzato...{\textquoteright} {\textquoteleft} ',?}{een brief aan des Voeux schreefBen}{jijt Errina}\\

\haiku{Och kom, was  ze,!}{in Lugano in een van}{de grote hotels}\\

\haiku{{\textquoteright} {\textquoteleft}Maar weet je wel dat?}{het anderhalf uur roeien}{is naar de overkant}\\

\haiku{Dan komt ze woedend.}{uit het witte huis en maakt}{een hele sc\`ene}\\

\haiku{Het was alsof de... {\textquoteleft}?}{zonnestralen zongenKom}{je naast mij zitten}\\

\haiku{Dan gaf je twintig, -.}{of dertig rappen wat je}{maar te missen had}\\

\haiku{Maar ik voel mij niet ', -?}{geroepen haar leed metr}{te gaan delen jij}\\

\haiku{{\textquoteright} {\textquoteleft}Jawel,{\textquoteright} pijnlikte, {\textquoteleft}?}{Julesmaar kun je dan niet}{tot morgen wachten}\\

\haiku{- De autobus van.}{elven schokte om de hoek}{bij de spitse rots}\\

\haiku{Doch de ander viel.}{hem bruusk in de rede dat}{dat toch niet aanging}\\

\haiku{Vooral Weckerlin.}{hield van het brede uitzicht}{van dat plateau af}\\

\haiku{Ja, ja, haastte Leo,,.}{zich dan weer daar was veel van}{waar heel veel van waar}\\

\haiku{Want zo als het hier ', '}{vandaag int klein was zo}{ging het int groot}\\

\haiku{Soms vielen ze van,.}{de gladde muur af op je}{hand of in je nek}\\

\haiku{hoe kletterde die '.}{regen weer op het platte}{dak vant atelier}\\

\haiku{Even, heel even haar in,:}{de lachende eerlike}{ogen te kunnen zien}\\

\haiku{Weckerlin zei een,.}{mop tegen Errina waar}{allen om lachten}\\

\haiku{Mi-ma mundus, -...,,.}{vult decipi mi-ma}{kom Free nog \'e\'en keer}\\

\haiku{{\textquoteleft}Ik weet niet of dat,.}{Albrecht's sterke kant wel is}{zulke houtsneetjes}\\

\haiku{Hij sloot de ogen om.}{de gedachten van zo even}{verder te dromen}\\

\haiku{Doch deze laatsten.}{hadden al genoeg van hem}{gezien en gehoord}\\

\haiku{Jaunes-citrons,...{\textquoteright}:}{jaunes et bleu\^atres Giulietta}{schaterde het uit}\\

\haiku{Dat was heel gew\'o\'on,, '.}{doodgewoont ging altijd}{zo in de wereld}\\

\haiku{Achter zich hoorde,;}{zij Jules die nu ook weer}{mee dorst te komen}\\

\haiku{Ik kijk om en toen...{\textquoteright} {\textquoteleft}!}{kwamen ze al op ons af}{En hij d'r van door}\\

\haiku{hoe bleef een raadsel,.}{want Luzato had beloofd}{te zullen zwijgen}\\

\haiku{Hij bukte zich juist,.}{over zijn tomaten de rug}{naar haar toegekeerd}\\

\haiku{Haar denken zocht zijn,:}{spoor terug kwam weer aan de}{pas verlaten plek}\\

\haiku{In de schachten van.}{de werkelikheid daalden}{haar gedachten af}\\

\haiku{Line sloot de ogen,.}{kuste hem en lei dan haar}{hoofd op zijn schouder}\\

\haiku{Het was een brief van,,.}{des Voeux zakelik-koud}{maar uitvoerig scherp}\\

\haiku{{\textquoteright} Albrecht haalde de,.}{schouders op alsof het hem}{onverschillig was}\\

\haiku{Om een eerlike,;}{zaak vooral om iets dat te}{verdedigen is}\\

\haiku{Doch de ontmoeting.}{met Albrecht viel haar lichter}{dan zij had gevreesd}\\

\haiku{Albrecht wendde zich,.}{af van het lichte venster}{keek Weckerlin aan}\\

\haiku{Hij was in een brand,.}{gesneld in de waan een mens}{te kunnen redden}\\

\haiku{er was geen mens, die,,.}{om hulp bad er waren geen}{vlammen geen gevaar}\\

\haiku{En die dreef hem uit,,.}{wierp hem op zij aleer de}{dag was opgegaan}\\

\subsection{Uit: Fakkeldragers}

\haiku{Het satiriese - -;}{gedicht op pag. 129 is meen}{ik van Erich M\"uhsam}\\

\haiku{Twee honderd in de, - '.}{maand waarachtign vijfde}{van zijn salaris}\\

\haiku{{\textquoteright} Ja gil maar toe, de.}{bediening was hier net als}{overal in Indi\"e}\\

\haiku{Dan bracht met traag en;}{stil beweeg de djongos de}{lange luie rietstoel}\\

\haiku{De direkteur van.}{de drukkerij had kortaf}{met neen geantwoord}\\

\haiku{Hij maakt er altijd,.}{een makan besar van en}{wil dan niet naar huis}\\

\haiku{Door geld,  door dat,:}{laatste middel dat stomme}{uiterste middel}\\

\haiku{Daar scharrelde de.}{Europeaan met vrouwen}{van zijn eigen ras}\\

\haiku{Nou, dan heb-ie;}{je aardrijkskunde ook slecht}{onthouwe meneer}\\

\haiku{in de verte het:}{licht van een vuurtoren aan}{de Afrikaanse kust}\\

\haiku{nee meneerr, - scheidt u, -.}{nu uit meneerr ik wil heus}{niet hebben m'neerr}\\

\haiku{Over Barbusse of...}{over de literatuur en}{de oorlog of over}\\

\haiku{En toen, na Durban,:}{en Kaapstad bij het kalmer}{worden van de zee}\\

\haiku{Zonder naar haar te,:}{kijken nam Marti haar toch}{van terzijde op}\\

\haiku{De reis is goddank,{\textquoteright}.}{achter de rug antwoordde}{hij ietwat nuchter}\\

\haiku{Er kwamen veel heel,:}{mensen en daarbij zulke}{interessante}\\

\haiku{mevrouw Schindler,.}{kwam uit kanton Glarus net}{als Heinrich's moeder}\\

\haiku{Hij ging niet meer naar,:}{Indi\"e terug hij wilde}{ook niet hier blijven}\\

\haiku{Waarom, waarom moest.}{het met Moeder altoos z\'o}{gaan en niet anders}\\

\haiku{daarh\'een trekken, waar.}{eindelik kleur kwam breken}{door den grauwen nacht}\\

\haiku{Het hele plan was.}{te zot om er twee woorden}{aan te verspillen}\\

\haiku{Het was al bij half.}{tien en een berg van werk lag}{op hem te wachten}\\

\haiku{Vraag aan \'een van de.}{andere Zwitsers of ze}{meer van hem weten}\\

\haiku{Officieren en.}{hoofdofficieren waren}{er v\'e\'el te weinig}\\

\haiku{Onaangenaam dat:}{Aloys Rantzau steeds weer op dat}{denkbeeld terugkwam}\\

\haiku{En zij zal de kop '.}{opsteken zo als zet}{nog niet heeft gedurfd}\\

\haiku{Dan legde zij haar.}{hand op Marti's arm en vroeg}{hem een sigaret}\\

\haiku{{\textquoteleft}Ik ben wel \'een van,{\textquoteright}.}{de weinigen vervolgde}{hij wat ernstiger}\\

\haiku{of voorzichtig met,.}{halve woorden aanduiden}{bang hem te kwetsen}\\

\haiku{Een adder dook op.}{in het warrig groeiende}{gras van zijn denken}\\

\haiku{Dan zal ik meteen.}{aan Forster zeggen dat u}{morgenochtend komt}\\

\haiku{Of kan hij vannacht,?}{nog op u rekenen als}{hij u nodig heeft}\\

\haiku{Ook die bende had.}{zich nu en dan nog in de}{Astoria gewaagd}\\

\haiku{Den eersten dag had,.}{Marti gemerkt dat die zich}{gepasseerd voelde}\\

\haiku{Thea Schumacher... zou ',:}{zijt begrijpen als hij}{haar straks vertelde}\\

\haiku{Elk woord geleek een,,.}{dorre oude munt glansloos}{en toch van waarde}\\

\haiku{iederen dag werd.}{het een grijs geschubde slang}{voor een dode deur}\\

\haiku{Mijn liefste is blank,.}{en rood hij draagt de banier}{boven tienduizend}\\

\haiku{Hij dacht nog eens na,:}{over haar bedoeling toen hij}{weer haar stem hoorde}\\

\haiku{In het begin was.}{de regering niet eens zo}{erg op mij gebrand}\\

\haiku{Anders Zou ik dat.}{ook allemaal niet tegen}{u gezegd hebben}\\

\haiku{en een gelaten.}{begrip van eigen kunnen}{en eigen onmacht}\\

\haiku{het kanon tegen.}{het nachtelik duister van}{Sinzheim's buitenwijk}\\

\haiku{hoe genoten zij,,...}{hoe groeiden zij in h\'a\'at als}{de tegenstander}\\

\haiku{Zij voelde in zijn:}{vlugge houding en in heel}{zijn jonge wezen}\\

\haiku{Zij kende Marti,.}{weinig hij was nog geen zes}{maanden in Sinzheim}\\

\haiku{O, die dorheid der,!}{praktijk dat harteloze}{praktiese leven}\\

\haiku{De beklemming, de.}{nijpende voetangel van}{het gegeven woord}\\

\haiku{Hij stond op van het,.}{bed dat rust beloofde en}{geen rust wou brengen}\\

\haiku{En toch... leek 't hem?}{z\'o omdat het alweer een}{jaar geleden was}\\

\haiku{Al wie oprecht en,.}{warm voor Sinzheim voelde moest}{daartoe meewerken}\\

\haiku{Hij zou koers houden,,.}{in de richting van Thea van}{Schwarz van Werner Stolz}\\

\haiku{- dat het verstand, het,:}{kille koele verstand straks}{ook zou oordelen}\\

\haiku{En verder, verder,...}{had hij willen verklaren}{moeten verklaren}\\

\haiku{De Meimaand was in.}{zachte zomerse dagen}{langzaam doorgebloeid}\\

\haiku{Maar schrik lag over de.}{felgeslagen daken en}{huizen van Sinzheim}\\

\haiku{{\textquoteleft}Ik kan er niets aan,{\textquoteright},.}{doen klonk het kil toen Marti}{had uitgesproken}\\

\haiku{De telefoon ging,.}{over luidruchtig getril door}{Hellmuth's werkkamer}\\

\haiku{Door het open raam klonk,.}{het kanonnengedonder}{onafgebroken}\\

\haiku{{\textquoteleft}Als ik je raden...,?}{mag maar je houdt niet erg van}{m'n raad geloof ik}\\

\haiku{{\textquoteright} ~ Tegen vier uur.}{begaf Marti zich w\'eer naar}{de universiteit}\\

\haiku{Zij keek hem aan door,,;}{haar wimpers als donkere}{violen fluweel}\\

\haiku{Uit een troepje op:}{de korridor der eerste}{verdieping klonk het}\\

\haiku{Dan klonk plotseling,:}{de hoge krankzinnige}{stem van Meierhofer}\\

\haiku{Zij drongen d\`an om,.}{Minsterberg dan om Schwarz heen}{en eisten hun geld}\\

\haiku{Een paar maal wist Schwarz,.}{het te winnen door vloeken}{snauwen en dreigen}\\

\haiku{ze roesden in de.}{vaalheid der halfduistere}{gangen over en weer}\\

\haiku{E\'en van de vuilste!}{verraders van de hele}{gele regering}\\

\haiku{Toen werd het drukkend.}{stil en geheel donker in}{zijn gevangenis}\\

\haiku{Langzaam at en dronk,;}{hij en genoot bijna van}{de stilte rondom}\\

\haiku{{\textquoteright} hoeveel maal had hij.}{dat de laatste dagen al}{niet moeten horen}\\

\haiku{Haatte hij wellicht,?}{niet genoeg kon hij dan niet}{fel genoeg haten}\\

\haiku{Naar en kleinzielig,.}{zo lang te mijmeren over}{eigen leed en strijd}\\

\haiku{Alles was immers;}{beter dan de val van de}{raden-republiek}\\

\haiku{Graaf Bermondt was aan.}{het hoofd van zijn troepen de}{stad binnen gerukt}\\

\haiku{Verwaarloosd gelijk;}{een huisdier dat men vergeet}{voedsel te geven}\\

\haiku{Eerst dienzelfden avond,,:}{terug in zijn cel flitste}{Marti te binnen}\\

\haiku{Nog steeds lijdend aan,.}{zwaarmoedigheid stemde hem}{dit mistroostig}\\

\haiku{Hij was tegelijk,.}{bruikbaar mens en betrouwbaar}{oprecht kameraad}\\

\haiku{zijn tussenkomst op?}{den middag van den moord in}{de universiteit}\\

\haiku{Toch berouwde hij,.}{niet dat hij zo tegen Schwarz}{was opgetreden}\\

\haiku{Vijf minuten v\'o\'or,:}{Zijn komst krabbelde Marti}{nog vlug onderaan}\\

\haiku{Van dat sterven gaat,.}{onmetelike kracht uit}{een grondeloos licht}\\

\haiku{maar ik zelf weet, dat,.}{ik ten slotte misbaar ben}{zo als iedereen}\\

\haiku{ieder die  hart,.}{heeft en verstand hoopt dat zo}{iets anders afloopt}\\

\haiku{de man die op het.}{laatste ogenblik de grote}{zaak in den steek liet}\\

\haiku{aan de volmaking,.}{de trapsgewijze opgang}{van het  mensdom}\\

\haiku{Hij zag opeens h\'aar,;}{verblijf de cel van Gertrud}{Faucherre voor zich}\\

\haiku{Maar hij geloofde.}{dat de werkelikheid met}{dit droombeeld spotte}\\

\haiku{De beloning die;}{Lilienfeld hiervoor ontving}{was niet zo gering}\\

\haiku{Hij richtte de spits:}{van zijn denken nu weer op}{het punt van uitgang}\\

\haiku{Je bent toch te zwak.}{om op je te nemen wat}{wij op ons nemen}\\

\haiku{Alles hebben de '.}{ploerten geprobeerd omt}{te onderdrukken}\\

\haiku{Als antwoord werd eerst,, -.}{een S. getikt toen een c.}{en dan h meer niet}\\

\haiku{dat ellendige!}{kloppen tegen de buizen}{wilde ophouden}\\

\haiku{en toen Marti niet:}{antwoordde liet hij er met}{een vloek op volgen}\\

\haiku{Ziet ge voor u het}{kleine portret dat stond op}{mijn schrijftafel dien}\\

\haiku{Denn dieses macht das,;}{Sterben fremd und schwer dass es}{nicht unser Tod ist}\\

\haiku{o Herr, gib jedem,,}{seinen Tod das Sterben das}{aus jenem Leben}\\

\haiku{en dat het - altans -.}{in zijn grondslagen niet mocht}{worden aangetast}\\

\haiku{op beleidvolle.}{wijze werd de produktie}{weer ingekrompen}\\

\haiku{welke dan terstond.}{onder anderen naam weer}{werden opgericht}\\

\haiku{Het was reeds gelukt,}{op Meierhofer de hand te}{leggen gevaarlik}\\

\section{Hendrik Conscience}

\subsection{Uit: Volledige werken 1. Batavia}

\haiku{voor een oud zeebonk,,.}{als ik is het niet goed aan}{land vrouw Pietersen}\\

\haiku{Maar, Hopman, geloof,!}{het mij ook slaat een Neerlandsch}{hart in den boezem}\\

\haiku{{\textquoteleft}Wie van u beiden,?}{het eerst den andere zal}{vergeten vraagt gij}\\

\haiku{{\textquoteleft}Walter, ik weet een.}{middel om niet lang van u}{gescheiden te zijn}\\

\haiku{{\textquoteright} {\textquoteleft}Kom, verstoor u niet,{\textquoteright}.}{om zoo weinig antwoordde}{zijne echtgenoote}\\

\haiku{{\textquoteleft}O, moeder, ik bid,,!}{u ik smeek u vertrek met}{mij naar het Oosten}\\

\haiku{Tegen den dikken.}{stam van eenen Billingbing6 lag}{de meid te slapen}\\

\haiku{Vier  jaren staat,,;}{Congo bij de zee bij de}{wijde stomme zee}\\

\haiku{{\textquoteleft}Wees niet droef, goede,{\textquoteright},.}{meester zeide Congo hem}{de hand nemende}\\

\haiku{De neger poogde;}{hem met versnelde stappen}{te doen vooruitgaan}\\

\haiku{Men moet overtuigd zijn,,.}{dat het Walter is om het}{te kunnen gelooven}\\

\haiku{Om uwer waardig te,.}{worden mag ik geenen mijner}{plichten verzuimen}\\

\haiku{Wij spraken bijna.....{\textquoteright} {\textquoteleft},,.....{\textquoteright}}{dagelijks van uNu ik}{ga tot straks Aleidis}\\

\haiku{{\textquoteleft}Rosalia, kom, kom,!}{dat ik mijnen heer vader}{ga verwittigen}\\

\haiku{Misschien is het nog,.}{iets anders dat hem wel te}{moede deed worden}\\

\haiku{nu zal de Hopman,?}{M. Walter ook vriendelijk}{onthalen niet waar}\\

\haiku{ik wil weten hoe.}{gij voert sedert uw vertrek}{uit  Amsterdam}\\

\haiku{Congo kan niets voor,,;}{u niets dan God bidden zooals}{gij hem hebt geleerd}\\

\haiku{maar dit toch zal hij,,.}{doen alle morgens elken}{avond den ganschen dag}\\

\haiku{{\textquoteright} {\textquoteleft}Ach, die brave vrouw,!}{Pietersen ik zal ze dus}{nooit meer wederzien}\\

\haiku{en is Aleidis nog,,}{jong meisje dan belet u}{niets mij te komen}\\

\haiku{maar zooals gij zegt, zij.}{zijn ten onzen gelukke}{zeer slecht gewapend}\\

\haiku{{\textquoteleft}Vertel ons eens in,.}{korte woorden hoe de zaak}{is afgeloopen}\\

\haiku{- Dadelijk maakten;}{wij den overloop klaar en het}{grof geschut vaardig}\\

\haiku{{\textquoteleft}Hopman, die heiden.}{met zijnen tulband moet van}{daar gelicht worden}\\

\haiku{Het is een kerel,,,.}{die het verre zal brengen}{wees daar zeker van}\\

\haiku{{\textquoteright} riep een der Factors,.}{die op dit oogenblik even}{door het venster keek}\\

\haiku{Dan antwoordde hij:}{den heere Stedevoogd in}{de Maleische taal}\\

\haiku{Dan hief hij weder:}{het hoofd op en zeide even}{kalm tot den gezant}\\

\haiku{maar Heemskerk, die als,}{een arend op hem losschiet heeft}{hem welhaast bereikt}\\

\haiku{{\textquoteleft}Mannen, broeders, laat;}{u de borst zwellen door het}{gevoel van den plicht}\\

\haiku{Ziet hier, welken last.}{wij van den heer Stedevoogd}{hebben ontvangen}\\

\haiku{Wel mogen wij over;}{het gebeurde als over eene}{zegepraal roemen}\\

\haiku{De bootsgezellen;}{zouden nog tot den avond de}{wallen bewaken}\\

\haiku{men kan er zich niet,.}{goed van bedienen zelfs niet}{tot verdediging}\\

\haiku{hij zal toonen, dat.....}{het hart hem goed is onder}{zijne zwarte huid}\\

\haiku{maar Van den Broeck stiet.}{hem terug en gebood hem}{te huis te blijven}\\

\haiku{{\textquoteright} riep Van den Broeck met, {\textquoteleft}!}{blijdschap uitdaar geven de}{Engelschen het op}\\

\haiku{{\textquoteleft}Makkers,{\textquoteright} riep Walter, {\textquoteleft},!}{de vijand van voren de}{vijand van achter}\\

\haiku{Wat deze mannen,;}{tot elkander zeiden was}{moielijk te verstaan}\\

\haiku{Soo drinckt dan hier}{Naer u plaisier Een pijp of}{vier Bij wijn of bier!31}\\

\haiku{{\textquoteright} Hier matigde de:}{Sjahbandar zijne stem en sprak}{op stilleren toon}\\

\haiku{nood en begeerten,.}{zijn de prikkels die den mensch}{sterk en groot maken}\\

\haiku{Weerhoud uwe tranen.}{en laat mij Congo eenige}{vragen toesturen}\\

\haiku{{\textquoteright} Dan bemerkte de,.}{Orang-kay's dat de Hopman}{hen had bedrogen}\\

\haiku{{\textquoteright} De Hopman hield den.}{blik ten gronde en schudde}{ontkennend het hoofd}\\

\haiku{Niet waar, Van den Broeck,,?}{gij zult u herinneren}{dat gij vader zijt}\\

\haiku{En zulk leven zou;}{ik pogen te behouden}{door trouweloosheid}\\

\haiku{De Portugees scheen.}{bedroefd en schudde het hoofd}{met medelijden}\\

\haiku{Eindelijk toch bleef.}{hunne slagorde voor des}{vorsten zitplaats staan}\\

\haiku{Elk voorval in hun,!}{lot elke verandering}{kon eene redding zijn}\\

\haiku{Ik ben hier onder.}{mijne landgenooten}{als een vreemdeling}\\

\haiku{hun gelaat werd geel,,;}{hunne lippen loodvervig}{hunne oogen verglaasd}\\

\haiku{ons allen bleef slechts,;}{over als trouwe soldaten}{te gehoorzamen}\\

\haiku{Wist gij nochtans, hoe!}{het gevoel der schaamte mij}{den boezem verscheurt}\\

\haiku{Zijn toestand sloeg de;}{hardvochtigste soldaten}{met medelijden}\\

\haiku{hij zeide weinig;}{en hield meesttijds den blik als}{beschaamd ten gronde}\\

\haiku{Oh,  konden wij,!}{hem levend verlossen wat}{schoone zegepraal}\\

\subsection{Uit: Volledige werken 2. De minnezanger. Een slachtoffer der moederliefde. Eene stem uit het graf}

\haiku{Een oude man met (.).}{zilverwitte haren en}{grijzen baardbladz 17}\\

\haiku{Waren de laatste,,:}{woorden die zij heden tot}{u sprak niet deze}\\

\haiku{hij hoorde zijnen.....}{naam als een noodkreet door veld}{en wouden galmen}\\

\haiku{{\textquoteright} {\textquoteleft}Met blijdschap, met groote,;}{blijdschap indien die zanger}{eene ware kunst toont}\\

\haiku{maar Ser Adelbert, met,:}{den gebiedenden vinger}{tot de dienaars riep}\\

\haiku{Het mildste van al,.}{wat er mild is op aarde}{Is de vrouwelach}\\

\haiku{Het sterkste van al,.}{wat er sterk is op aarde}{Is de vrouwelach}\\

\haiku{Het zoetste van al,.}{wat er zoet is op aarde}{Is de vrouwelach}\\

\haiku{ik wil nevens u,.}{zitten en u zeggen}{wat ik heb gedroomd}\\

\haiku{maar gij kondet niet.}{op Rotsburg blijven wonen}{en moest vertrekken}\\

\haiku{Bij het einde van:}{het ontbijt nam Ser Gunther}{het woord en zeide}\\

\haiku{{\textquoteleft}Mijn vader is wel...........}{een vrij man maar hij wint zijn}{brood met koophandel}\\

\haiku{Des avonds zal er een.}{vroolijk feestmaal op Rotsburg}{gehouden worden}\\

\haiku{het overige van.}{den tijd brengen wij door in}{vriendelijken kout}\\

\haiku{Vertrouwen wij op,,!}{Gods goedheid en gaan wij Hem}{dankend ter ruste}\\

\haiku{De ridders deden;}{de andere paarden meer}{naar de poort leiden}\\

\haiku{- Het wangedrocht viel.}{ter zijde en stierf in eene}{laatste stuiptrekking}\\

\haiku{Met overdreven wil,:}{met wanhoop worstelde ik}{tegen mij zelve}\\

\haiku{Gij zegt, dat gij ziek,?}{zult worden indien meester}{Wilfried ons verlaat}\\

\haiku{mijne dochter zal;}{hare zinnelooze zwakheid}{blijven beweenen}\\

\haiku{{\textquoteleft}O, mijn goede heer,!}{van Haviksberg wat ben ik}{blijde u te zien}\\

\haiku{want het rustte met!}{het aangezicht ter aarde}{en roerde zich niet}\\

\haiku{{\textquoteleft}Heer ridder,{\textquoteright} zeide, {\textquoteleft}?}{de kluizenaarwaarom zijt}{gij dus hopeloos}\\

\haiku{{\textquoteright} {\textquoteleft}Kom, Ser Wilfried, kom,}{telkens als de mismoed of}{de angst u aangrijpt}\\

\haiku{Wat hij in de kluis,;}{te vreezen had dit stond hem}{niet klaar voor den geest}\\

\haiku{de bleekheid op zijn.}{gelaat werd door den gloed der}{woede vervangen}\\

\haiku{Zij gedragen zich,.}{ten minste alsof zij het}{nooit hadden bemerkt}\\

\haiku{{\textquoteright} verklaren, en mijn.....}{arme zoon zou alle hoop}{moeten verzaken}\\

\haiku{Mijn vader trad met:}{M. Steurs in de kamer en}{zeide hem haastig}\\

\haiku{Dit voorval liet op.}{het gemoed mijner vrouw een}{diepen indruk na}\\

\haiku{Tegen den morgen,,.}{in de eerste dagklaarheid}{wekte mij Maria}\\

\haiku{Hij had veel praktijk;}{en was in aanzien als een}{schrander geneesheer}\\

\haiku{Het was het eenige,;}{middel dat nog kansen op}{genezing aanbood}\\

\haiku{{\textquoteright} {\textquoteleft}Kunt gij het kind niet?}{eenige oogenblikken van}{ons verwijderen}\\

\haiku{{\textquoteright} De grijsaard stond op.}{en ging in de weide bij}{het kleine meisje}\\

\haiku{Eindelijk werd ik,}{weder bedaard kuste nog}{herhaalde malen}\\

\haiku{Ik wensch en wil, dat.}{gij haar verpleegt en liefhebt}{als mijn eigen kind}\\

\haiku{Ja, daar zag ik zijn,}{rijtuig en met haast keerde}{ik naar het landgoed}\\

\haiku{{\textquoteleft}hier is mijnheer de,.}{dokter die u gaarne een}{handje zou geven}\\

\haiku{Hij bepeinsde zich:}{eene wijl en zeide dan met}{geestdrift in de oogen}\\

\haiku{In den hopeloozen.....}{toestand uwer echtgenoote is}{daar niets aan gewaagd}\\

\haiku{Geve God, dat gij!}{deze vermetelheid niet}{te betreuren hebt}\\

\haiku{Ik greep haar bij de,:}{hand trok haar in de kamer}{en riep met geestdrift}\\

\haiku{maar, o hemel, zij!}{moet de huwelijksakte}{onderteekenen}\\

\haiku{luister met kalmte.}{en wijsheid op hetgeen ik}{u ga openbaren}\\

\haiku{elke minuut is!}{eene eeuw van smart voor mijne}{arme verloofde}\\

\haiku{De meesten zouden;}{schrijven en hielden daartoe}{de pen in de hand}\\

\haiku{- Hebt gij bij de brug.....}{onderwege Meerhout eene}{samenkomst gehad}\\

\haiku{Ik ben gelukkig,,}{Klaas dat men uwe zaak bij mij}{erger gemaakt heeft}\\

\haiku{Ha, daar komt zij uit,!}{eene achterdeur met eene flesch}{wijn in de hand}\\

\haiku{Ofschoon gansch niet op,;}{zijn gemak stapte hij met}{koortsige haast voort}\\

\haiku{Hij stond op en liep.}{in \'e\'enen adem tot achter}{des schoolmeesters huis}\\

\haiku{dan weder stond des:}{brouwers lijk voor zijn bed en}{riep hem smeekend toe}\\

\haiku{Het geheele dorp,.}{stond er van overhoop zeide}{de onderwijzer}\\

\subsection{Uit: Volledige werken 3. Everard t'Serclaes}

\haiku{{\textquoteleft}Zie, zie, ginder bij,.}{Sint-Michielsheuvel komt}{Jan de goudslager}\\

\haiku{Dilbeek en al de!}{omliggende hofsteden}{staan in vollen brand}\\

\haiku{Ons gezantschap heeft;}{de heer graaf inderdaad met}{toorn en spot onthaald}\\

\haiku{De bestorming, de?}{plundering van Brussel een}{ingebeeld gevaar}\\

\haiku{- Eens in vrijheid, heb,;}{ik mij gespoed over Vorst de}{stad te bereiken}\\

\haiku{- een gulden leeuw op, -.}{zwart veld door den lieer van}{Assche gedragen}\\

\haiku{Het volk zweeg dus in;}{het openbaar en scheen lijdzaam}{zijn lot te dragen}\\

\haiku{het is tijd om den.}{dreigenden opstand zelfs in}{bloed te versmachten}\\

\haiku{Ik bezit nog meer.}{dan de helft van wat gij mij}{laatst hebt gegeven}\\

\haiku{Hij zweeg om zijnen.}{vader den tijd te laten}{den brief te lezen}\\

\haiku{{\textquoteleft}Welke innige,,!}{zuivere liefde zij u}{toedraagt Everard}\\

\haiku{Blijf voorzichtig met,,,.}{hem word voorzichtiger nog}{ik bid u mijn zoon}\\

\haiku{{\textquoteright} Everard greep de:}{handen des hopmans en riep}{opgetogen uit}\\

\haiku{{\textquoteright} {\textquoteleft}Gij bedriegt u niet, -,.....}{Goffredo en hebt gij mij}{niets meer te zeggen}\\

\haiku{{\textquoteleft}Ik zal mij houden,.}{alsof het toeval mij naar}{het bosch had geleid}\\

\haiku{{\textquoteright} {\textquoteleft}Ja, heer Willem,{\textquoteright} was, {\textquoteleft}.....{\textquoteright} {\textquoteleft}?}{het antwoordhet is zulk zacht}{wederWaar zijn ze}\\

\haiku{Hij gelijkt naar den}{jongen heer die vroeger zoo}{dikwijls aan onzen}\\

\haiku{het is, als zag ik.}{ze reeds in mijne volle}{handen glinsteren}\\

\haiku{Hoe de zucht tot geld!}{den menschelijken geest toch}{machtig kan maken}\\

\haiku{{\textquoteleft}Er zijn werklieden -;}{in mijns vaders slaapzaal ik}{had het vergeten}\\

\haiku{morgen toch, zoohaast mijn,.}{vader beneden is zal}{ik het schrijn openen}\\

\haiku{Wat kost het u, den?}{jongen t'Serclaes op uw}{feest te verzoeken}\\

\haiku{Hier kondigde een.}{hofmeester met luider stem}{hunne namen af}\\

\haiku{Deze gedachten.}{deden velen in hunnen}{afkeer wankelen}\\

\haiku{Hare schijnbare.}{koelheid had hem troost in den}{boezem gegoten}\\

\haiku{Zijn vader, die hein,:}{voorbijging sloeg hem op den}{schouder en vroeg hem}\\

\haiku{Gij moet veinzen en,.}{u houden alsof wij u}{niets hadden gezegd}\\

\haiku{{\textquoteright} {\textquoteleft}Ik ben Everard,{\textquoteright}.}{t'Serclaes antwoordde de}{jongeling spijtig}\\

\haiku{zeg den hopman, dat.}{ik voor den middag hem een}{bezoek zal brengen}\\

\haiku{Goffredo heeft wel.}{werkelijk naar Everard}{t'Serclaes gezocht}\\

\haiku{Die verpander was,?}{niemand anders dan hopman}{Goffredo niet waar}\\

\haiku{{\textquoteright} {\textquoteleft}Ik wensch te spreken,,{\textquoteright}.}{heer voorzitter zeide een}{zwaarlijvig brouwer}\\

\haiku{want hij opende de,:}{deur der zaal terwijl hij met}{luider stemme riep}\\

\haiku{{\textquoteright} {\textquoteleft}Wij zijn vijanden,,{\textquoteright}.}{sedert lang ik weet het wel}{zeide de Amman}\\

\haiku{{\textquoteleft}Het zou dus iemand,?}{uwer uitgenoodigden zijn die}{het juweel ontstal}\\

\haiku{Ha, ha, ziedaar dus,}{het geheim dat gij mij te}{veropenbaren}\\

\haiku{want nu zeide hij,:}{op smartelijken toon ja}{met nederigheid}\\

\haiku{Was de Amman geen,?}{boos en gewetenloos mensch}{tot alles bekwaam}\\

\haiku{Zwijgen, zwijgen moet,!}{gij of mijne gramschap rust}{op u voor eeuwig}\\

\haiku{Wat liefderijke!}{moeite heeft hij aangewend}{om mij te troosten}\\

\haiku{Is het de vrees van,?}{des Ammans boosheid die u}{zoo wreed doet lijden}\\

\haiku{Dezen avond was het;}{weder vergadering der}{eedgenooten}\\

\haiku{{\textquoteright} Tranen borsten hem.}{over de wangen en het hoofd}{viel hem op de borst}\\

\haiku{{\textquoteleft}Ja, ja, het is de,{\textquoteright}, {\textquoteleft}.}{heer t'Serclaes zeide de}{knechtik ken hem wel}\\

\haiku{Zoohaast ik maar weg kan,,.}{loop ik en ga hem melden}{wat er is geschied}\\

\haiku{{\textquoteright} Inderdaad, tot groote}{verbaasdheid der dienstboden}{volgde Everard}\\

\haiku{{\textquoteright} vroeg hij, terwijl hij.}{verrast luisterde en van}{zijnen stoel opstond}\\

\haiku{Uw vader, heer, is.}{mij sedert vele jaren}{een vurig vijand}\\

\haiku{Toen hij in de zaal,:}{trad kwam zijn vader hem te}{gemoet en zeide}\\

\haiku{{\textquoteright} {\textquoteleft}Ba, tot zooverre toch!}{zijn wij de slaven dezer}{grove lieden niet}\\

\haiku{{\textquoteright} {\textquoteleft}Dan gedragen wij.}{ons volgens de bevelen}{van meester Lankhals}\\

\haiku{{\textquoteright} Nog eene wijl zette:}{hij zijne overdenkingen}{voort en zeide dan}\\

\haiku{{\textquoteright} En hij sprong op met;}{gloeiende oogen en meende}{zijn zwaard te grijpen}\\

\haiku{{\textquoteright} Maar de overste der:}{wacht legde hem de hand op}{den mond en zeide}\\

\haiku{Maar t'Serclaes, voor,.}{zulken troost gansch gevoelloos}{antwoordde hem niet}\\

\subsection{Uit: Volledige werken 4. Het wassen beeld}

\haiku{maar de baljuw heeft,;}{Rosa zoo lief dat hij haar}{niets kan weigeren}\\

\haiku{zij slaapt zoo rustig,,!}{zij bloost zij heeft gelachen}{in eenen zoeten droom}\\

\haiku{Hoe verbleekten en,!}{beefden zij toen zij hunne}{woning ontwaarden}\\

\haiku{{\textquoteleft}Eilaas, Mevrouw, ik,{\textquoteright};}{ben half dood van schrik was het}{hopelooze antwoord}\\

\haiku{Zuchtend en kermend;}{trad de baljuw met zijnen}{schoonzoon in de zaal}\\

\haiku{daar lieten zij zich.}{op eene zitbank vallen en}{weenden in stilte}\\

\haiku{{\textquoteleft}Luitenant, gij hebt,,?}{gehoord waarover deze heer}{komt klagen niet waar}\\

\haiku{Ja, ja, een kind van,.....}{omtrent twee jaar een meisje}{in het wit gekleed}\\

\haiku{De roofzuchtige,.}{lieden gaan en komen maar}{keeren telkens terug}\\

\haiku{{\textquoteright} klaagde hij, terwijl,,.}{hij de vuisten krampachtig}{wringende bleef staan}\\

\haiku{Daarop vertelde.}{hij zijn wedervaren met}{volle oprechtheid}\\

\haiku{Wel had Frederic.}{zijne opzoekingen niet}{beslissend gestaakt}\\

\haiku{zie eens ginder, het,!}{kleine meisje dat zoo licht}{op de koorde danst}\\

\haiku{Indien zij eens in?}{handen van zulke lieden}{gevallen ware}\\

\haiku{{\textquoteleft}Maar gij moet u sterk,.}{houden en u niet laten}{ontroeren Dina}\\

\haiku{Gelukkig was de.}{dorpsdokter te huis en kwam}{hij onmiddellijk}\\

\haiku{De zinnelooze scheen;}{eensklaps al hare krachten}{terug te vinden}\\

\haiku{Sluit nu de deur der,.}{zaal achter mij en doe zooals}{ik u heb gezegd}\\

\haiku{Daarbij voegden zich.}{de gevolgen van slechten}{oogst en duren tijd}\\

\haiku{maar hij gelukte.}{daarin niet beter dan met}{Sebastiaan Groof}\\

\haiku{Onze gemeente!}{schijnt in een hol van dieven}{en roovers veranderd}\\

\haiku{Het belang, dat het?}{zichtbaar bedrukte meisje}{hem inboezemde}\\

\haiku{Antwoord slechts als men,{\textquoteright}.}{u iets vraagt viel de baljuw}{hem in de rede}\\

\haiku{{\textquoteright} {\textquoteleft}Zeker, Mijnheer, ik,.}{heb er van hooren spreken}{gelijk iedereen}\\

\haiku{Zoo niet, bevindt men,!}{u schuldig de galg zonder}{hoop op genade}\\

\haiku{{\textquoteleft}Preter, blijft gij met;}{drie uwer gezellen deze}{lieden bewaken}\\

\haiku{Heila, Heila, zoudt?}{gij mij kunnen verlaten}{in mijne droefheid}\\

\haiku{{\textquoteright} {\textquoteleft}Hoe, Samson, gij wilt?}{u zelven beschuldigen}{om ons te redden}\\

\haiku{Naar gewoonte zal.}{het ongetwijfeld op eene}{dwaasheid uitloopen}\\

\haiku{{\textquoteright} {\textquoteleft}Hawida Kavandar,?}{gij insgelijks blijft bij uwe}{eerste verklaring}\\

\haiku{Heila hield eene hand.}{haars broeders en besprengde}{deze met tranen}\\

\haiku{Krachtens een bevel.}{van den baljuw hebt gij mij}{aanstonds te volgen}\\

\haiku{, bleef ze eene lange:}{wijl met aandacht bekijken}{en mompelde dan}\\

\haiku{Wie heeft er niet ten,?}{minste \'e\'ens bemind in}{zijn leven niet waar}\\

\haiku{ik zwijg als het graf,{\textquoteright},.}{gromde de waarzegster zich}{naar de deur keerende}\\

\haiku{Dan geef ik bevel,;}{u naar de gevangenis}{terug te leiden}\\

\haiku{Rosa honderdmaal.}{gezien en ze dikwijls in}{de armen gedrukt}\\

\haiku{Ja, ik ben bereid;}{de kleine Rosa in uwe}{armen te brengen}\\

\haiku{{\textquoteright} {\textquoteleft}Weiger, heer, het staat,.}{u vrij maar uw kind blijft voor}{altijd verloren}\\

\haiku{De kaartenkijkster.}{zou dezen nacht in zijne}{woning overbrengen}\\

\haiku{O, mocht het waarheid,!}{zijn dat gij mijn arm kind mij}{gaat terugschenken}\\

\haiku{De lieden, die u,.}{hebben opgevoed zult gij}{niet meer terugzien}\\

\haiku{De meid had haar niet;}{alleen gewasschen en met}{liefde opgeschikt}\\

\haiku{{\textquoteright} {\textquoteleft}Katrien, indien de!}{goede God eens ons geluk}{wilde volbrengen}\\

\haiku{Het kon zes uren des,.}{morgens zijn en het was reeds}{lang klaar dag toen Mev}\\

\haiku{Misschien had zij het,,;}{bewusteloos zelf eenen zoen}{teruggegeven}\\

\subsection{Uit: Volledige werken 5. Bella Stock}

\haiku{{\textquoteleft}ik hoor nog zijne,.}{zware stem die zingt op de}{maat van het paardje}\\

\haiku{en dus droomende,,}{ben ik gaan overwegen dat}{gij ongelukkig}\\

\haiku{Er lag wel iets van;}{de logge vormen des beers}{in zijne leden}\\

\haiku{Djosep zal dan maar.....}{in zee met nuchteren mond}{en ledige maag}\\

\haiku{Maar ik ben vijf en.}{twintig jaar te vroeg op de}{wereld gekomen}\\

\haiku{Zes schellingen in,!}{dezen ongelukkigen}{tijd het is een schat}\\

\haiku{maar die lomperd lacht.}{mij uit en drijft zijnen ezel}{met meer geweld voort}\\

\haiku{{\textquoteleft}Wel, Ko, man lief, gij!}{zoudt een Christenmensch den dood}{op het lijf jagen}\\

\haiku{Ko volgde haar met:}{eene zure uitdrukking van}{spijt op het gelaat}\\

\haiku{{\textquoteright} riep de maagd verstoord, {\textquoteleft},?}{gij durft droomen dat ik uwe}{vrouw geworden ben}\\

\haiku{maar, God zij er om,,.}{geloofd droomen is bedrog}{zooals het spreekwoord zegt}\\

\haiku{- Eensklaps liet zij haar:}{voorschoot vallen en zeide}{tot den strandlooper}\\

\haiku{Niet waar, vader lief,,!}{wat God kan behagen is}{altijd wel gedaan}\\

\haiku{En indien ik zulks,,?}{geloof waarom zoudt gij dan}{wanhopen mijn kind}\\

\haiku{{\textquoteright} murmelde Bella,.}{met den helderen lach der}{blijdschap in de oogen}\\

\haiku{{\textquoteright} De vrouwen zagen,.}{op naar het kind dat met de}{hand naar de zee wees}\\

\haiku{noch de oude Stock,.}{noch zijne dochter wilden}{van trouwen hooren}\\

\haiku{Het is wel schoon van,.}{Bella dat zij  haren}{vader dus bemint}\\

\haiku{{\textquoteright} Allen klommen op,.}{tot het kleine kamertje}{achter M. Darings}\\

\haiku{{\textquoteright} zuchtte het meisje, {\textquoteleft}!}{met eenen stralenden glimlach}{wat ben ik blijde}\\

\haiku{voor zijnen armen,,!}{vader voor zijne moeder}{voor zijne zuster}\\

\haiku{{\textquoteleft}Ach, ja, ik bid u,!}{hij heeft de koorts nog gehad}{dezen namiddag}\\

\haiku{M. Darings deed eenen.}{stap naar de deur der hut en}{trok ze met zorg toe}\\

\haiku{Beklagen wij niet,.....{\textquoteright} {\textquoteleft}?}{de slachtoffers maar wel de}{moordenaarsMaar hij}\\

\haiku{Ik gevoel mij den,{\textquoteright}}{moed niet tot het vervullen}{dier wreede boodschap}\\

\haiku{Ik zal morgen, op,.}{den noen komen zien wat er}{kan worden gedaan}\\

\haiku{{\textquoteleft}Kom, mijn kind, laat ons,{\textquoteright};}{binnengaan en ontsteek de}{lampe zeide hij}\\

\haiku{{\textquoteleft}Laat mij nog wat met,,{\textquoteright}.}{u blijven waken Bella}{zeide de blinde}\\

\haiku{Tante Claar zal eerst,.}{te middernacht komen om}{u af te lossen}\\

\haiku{gij schat het veel te,,.}{hoog het weinige dat wij}{voor u kunnen doen}\\

\haiku{Indien ik tante,,}{Claar niet had ik zou nog niet}{eens weten voor wien}\\

\haiku{D\'a\'ar, tusschen die twee,,.}{duinen de donkergroene}{vlek dat is de zee}\\

\haiku{{\textquoteleft}Ja, ja, mijnheer,{\textquoteright} riep, {\textquoteleft}:}{Djosep half verlegenik}{durf het wel zeggen}\\

\haiku{Zijn broeder Louis ging,.}{te Duinkerken scheep op de}{labberdaanvangst}\\

\haiku{M. de Milval, gansch,.}{opgekleed daalde van de}{trap in de kamer}\\

\haiku{de wereld heeft dan.}{tusschen arme visschers zijn}{leven te slijten}\\

\haiku{Uwe afwezigheid,,;}{mijnheer zal mij voor langen}{tijd treurig maken}\\

\haiku{Zij scheen zeer bedrukt.}{en liet het hoofd mismoedig}{op de borst hangen}\\

\haiku{Gij  zoudt mij eene!}{geraaktheid doen krijgen met}{dat eeuwig talmen}\\

\haiku{hij eenen bos zeedoorn:}{aan en rukte zijne taaie}{stammen uit den grond}\\

\haiku{en d\'a\'ar eenen zijner,:}{bekenden op den schouder}{kloppende vroeg hij}\\

\haiku{{\textquoteleft}Jan, vriend, wat is er,?}{voorgevallen dat men hier}{zoo te hoopen staat}\\

\haiku{Zoudt gij bij geval?}{aan de domme uitvindsels}{der priesters gelooven}\\

\haiku{{\textquoteright} {\textquoteleft}Daarin bedriegt gij,,{\textquoteright}.}{u voorwaar viel zijn gezel}{hem in de rede}\\

\haiku{{\textquoteright} zeide de oude,.}{krijgsman op treurigen doch}{vriendelijken toon}\\

\haiku{maar ik weet nog zoo,;}{goed wat moeite gij deedt om}{mij te leeren lezen}\\

\haiku{U te zeggen, wat,.}{ik dien nacht heb geleden}{is onmogelijk}\\

\haiku{ik ben het niet, die.}{lust heb om met uwen barschen}{kozijn te vechten}\\

\haiku{{\textquoteright} En, zich tot zijnen,:}{broeder en zijne nichte}{wendende sprak hij}\\

\haiku{Het scheen hem, dat hij.}{het gerucht van stappen voor}{de deur ontwaarde}\\

\haiku{ik had mij belast.}{met onze beste dekens}{en met een kussen}\\

\haiku{Hij weigerde op,.}{een zachter bed te rusten}{uit edelmoed alleen}\\

\haiku{hij zal vertrekken,,;}{v\'o\'or den morgenstond verre}{verre zal hij gaan}\\

\haiku{Ik heb den Heer voor!}{uwe blijde terugkomst zoo}{vurig gezegend}\\

\haiku{Hij zou dus weten,.}{dat de \'emigr\'e langs de zee}{wilde ontvluchten}\\

\haiku{Onderwerp u dus,,.}{zonder tegenstand of zeg}{dat gij liever sterft}\\

\haiku{Bella schoof uit en;}{sloeg geweldig tegen den}{rand van het vaartuig}\\

\haiku{Ik bewonder u,,!}{ik eerbiedig u als waart}{gij eene heilige}\\

\haiku{{\textquoteleft}Bekommer u niet,.}{om de wonde die gij mij}{hebt  toegebracht}\\

\haiku{Ach, ik schreef het met,.}{zinnelooze haast onder de}{oogen mijner wachten}\\

\haiku{Gij hebt waarschijnlijk?}{nooit van liefde met mijne}{nichte gesproken}\\

\haiku{als hij eene reden,.}{tot groot verdriet heeft dan is}{alles hem pijnlijk}\\

\haiku{Het was zichtbaar, dat;}{een gedachtenvloed hem door}{de hersens stroomde}\\

\haiku{- want in der waarheid,.}{gij hebt geene reden om het}{ergste te gelooven}\\

\haiku{{\textquoteright} {\textquoteleft}Eilaas, heb ik u?}{dan reden gegeven om}{mij te verstooten}\\

\haiku{God zou mijne ziel.}{over zulke laffe baatzucht}{rekening vragen}\\

\haiku{{\textquoteright} Zij onderbraken.}{hunnen edelmoedigen twist}{en traden in huis}\\

\haiku{Kozijn Djosep heeft;}{mij vele goede dingen}{gezegd onderweg}\\

\haiku{{\textquoteright} zuchtte Djosep, zich.}{tranen van ontroering uit}{de oogen vegende}\\

\haiku{Nu, Bella, gij ziet,.}{wel dat gij geene reden hebt}{om dus te schrikken}\\

\haiku{{\textquoteleft}Ah, woorden, die M.!}{de Milval met eigene}{hand heeft geschreven}\\

\haiku{{\textquoteright} Louis Stock opende den,:}{brief hield eene wijl de oogen er}{op gericht en vroeg}\\

\haiku{en nochtans indien,,}{zijn broeder ja indien zijn}{eigen zoon tusschen}\\

\haiku{geef mij het middel,.}{om de plaats te herkennen}{waar hij zich bevindt}\\

\haiku{{\textquoteleft}Lieve nichte, gij?}{beschuldigt mij wellicht van}{gevoelloosheid}\\

\haiku{{\textquoteleft}Uit medelijden,.}{met mijne nichte verkort}{het treurig afscheid}\\

\haiku{Deze bloedstorting,;}{is nutteloos voor de zaak}{die gij verdedigt}\\

\haiku{Ginder, ginder, niet,!}{verre van het fort tusschen}{die hoopen aarde}\\

\haiku{want een hagel van.}{kogels en ballen kliefde}{de lucht rondom hem}\\

\haiku{en het huisje, dat,.}{zij heeft bewoond staat op drie}{boogschoten van hier}\\

\subsection{Uit: Volledige werken 6. Simon Turchi}

\haiku{evenwel, het was nog,:}{met zoete gelatenheid}{dat zij antwoordde}\\

\haiku{{\textquoteright} Een knecht opende de,,,:}{deur en terwijl hij iemand}{inleidde riep hij}\\

\haiku{{\textquoteright} {\textquoteleft}Maar met uw oorlof,,{\textquoteright}, {\textquoteleft}}{signore riep de oude}{ridder half vergramd}\\

\haiku{{\textquoteright} {\textquoteleft}Inderdaad, gij hebt,{\textquoteright},.}{wel zeer wel gedaan sprak de}{heer Van de Werve}\\

\haiku{hij zeide mij, dat!}{hij het noodige geld nog niet}{heeft kunnen vinden}\\

\haiku{zit daar neder op;}{die banke en laat mij aan}{uwe zijde zitten}\\

\haiku{Neen, daarom ben ik.}{op mijnen ouden dag niet}{over zee gekomen}\\

\haiku{Hem eensklaps op den,:}{schouder slaande riep Julio}{met eenen schaterlach}\\

\haiku{wat ik maak, moet slechts,.}{dienen om eene luim onzes}{meesters te voldoen}\\

\haiku{indien het spreken,.}{kon het zou u wondere}{dingen vertellen}\\

\haiku{gij zult dezen avond.....}{Geronimo afwachten}{en hem doorsteken}\\

\haiku{Volvoer uwen last met,.....}{schranderheid en ik  zal}{u goed beloonen}\\

\haiku{Gij zult een weinig}{v\'o\'or tien uren u ten huize}{van Geronimo}\\

\haiku{{\textquoteright} {\textquoteleft}Het is drie dagen,.}{geleden dat gij ze nog}{in de handen naamt}\\

\haiku{{\textquoteright} {\textquoteleft}Dit is volgens den.}{stand der personen en het}{gewicht der zaken}\\

\haiku{Er waren daar ook.}{eenige Carolusguldens}{te winnen nochtans}\\

\haiku{{\textquoteright} {\textquoteleft}Hij draagt gansch bruine;}{kleederen en eene witte}{veder op den hoed}\\

\haiku{Dat deze laatste,;}{nog had kunnen vluchten was}{hem onverstaanbaar}\\

\haiku{{\textquoteleft}De serenata;}{kan niet gegeven worden}{zonder luitspelers}\\

\haiku{De jonge Maria.}{was nu waarlijk schoon boven}{alle verbeelding}\\

\haiku{Tot Simon Turchi:}{sprak zij in het voorbijgaan}{met blijden glimlach}\\

\haiku{Wat ik doe, is slechts.}{om u uit een dreigenden}{toestand te redden}\\

\haiku{En zal de vreemde?}{koopman mij de somme ter}{hand stellen in geld}\\

\haiku{Er kon niets beters,.....}{uitgedacht worden om zijn}{leven te sparen}\\

\haiku{{\textquoteright} Eene onwillige.}{beweging van spijtigheid}{ontsnapte den knecht}\\

\haiku{{\textquoteright} {\textquoteleft}Zult gij nog hier zijn,,?}{signore als ik uit den}{kelder terugkeer}\\

\haiku{Hij nam de lamp en,.}{verliet den kelder zonder}{de deur te sluiten}\\

\haiku{Ik zal de deure.....}{sluiten en morgen mijn werk}{komen voltooien}\\

\haiku{Hij wist wel, dat zulks;}{niet kon geschieden dan na}{verloop van veel tijds}\\

\haiku{{\textquoteright} {\textquoteleft}Maar kon hij dan zoo?}{lichtelijk de hand mijner}{dochter verzaken}\\

\haiku{Geronimo nog,}{levend vinden hoe zouden}{wij al te zamen}\\

\haiku{ik zal zeer laat in.}{den avond Bernardo zenden}{om u te helpen}\\

\haiku{{\textquoteright} {\textquoteleft}Neen, ik zal hem op,.}{zijn leven gebieden u}{te gehoorzamen}\\

\haiku{Julio zette zich;}{op eenen stoel en legde het}{hoofd in de handen}\\

\haiku{Vergeet van nu af;}{aan uwe andere namen}{en wees voorzichtig}\\

\haiku{dan weder liet hij.....}{ze van de eene hand in de}{andere glijden}\\

\haiku{ik zal mij kleeden,,;}{als een edelman in satijn}{fluweel en zijde}\\

\haiku{een zenuwschok trof,.}{al zijne leden en hij}{verbleekte van schrik}\\

\haiku{Gij ten minste zult;}{daarboven de belooning}{der onschuld vinden}\\

\haiku{Morgen zal de Wet.....}{den speelhof en ook dezen}{kelder doorzoeken}\\

\haiku{moest zulk ijselijk?}{ongeluk uwe dagen op}{aarde verkorten}\\

\haiku{Ik gevoel mij de,.....}{kracht niet meer om de wreede}{taak te volvoeren}\\

\haiku{Ik zal uitgaan, om.}{te zien of ik ergens nog}{eenen winkel open vind}\\

\haiku{{\textquoteright} De schout keerde zich.}{om en stapte ter zijde}{van Simon Turchi}\\

\haiku{om mijnen besten.....?}{vriend te te mishandelen}{of te vermoorden}\\

\haiku{Wat mij betreft, ik,.}{bid u als vriend duid mij de}{zaak niet ten kwade}\\

\haiku{Merkbaar was het, dat;}{hij zijnen knecht niet zonder}{inzicht bespiedde}\\

\haiku{{\textquoteleft}Het is zonde, dat.}{zulke wijn nutteloos ten}{gronde moet vlieten}\\

\haiku{koop daar een goed paard;}{en vertrek in aller haast}{over Aerschot en Diest}\\

\haiku{akeligen schreeuw, als.}{hadde een geheime slag}{hem het hart doorboord}\\

\haiku{Ach, barmhartige,!}{God laat mij toch genade}{vinden voor uw oog}\\

\haiku{ik voelde een kort;}{oogenblik het bloed als eenen}{stroom mij ontvlieten}\\

\haiku{{\textquoteleft}Reeds acht dagen ligt,.}{het schip gereed dat hen naar}{Itali\"e moet voeren}\\

\haiku{{\textquoteright} Geronimo's hand;}{beefde in de hand zijner}{schoone echtgenoote}\\

\subsection{Uit: Volledige werken 7. De oom van Felix Roobeek}

\haiku{hij ware in staat.}{om u onaangename}{dingen te zeggen}\\

\haiku{hem zoo hard tegen,.}{het lijf dat hij er bijna}{van omtuimelde}\\

\haiku{Ik voelde allengs;}{weder mijne geneigdheid}{tot hem terugkeeren}\\

\haiku{Het is wonder, mijn.....}{vader verlaat van gansche}{dagen dien stoel niet}\\

\haiku{{\textquoteright} {\textquoteleft}Maar, goede heer, ik.}{heb nooit iets van zijne hand}{gehoord of gezien}\\

\haiku{{\textquoteright} {\textquoteleft}Wel, lieve hemel,!}{dan heeft de schurk weder aan}{alles gelogen}\\

\haiku{Maar hij bad mij, nog,:}{eenige oogenblikken te}{blijven en zeide}\\

\haiku{Ach, ik smeek u, poog!}{hem dit verzenmaken uit}{het hoofd te kouten}\\

\haiku{{\textquoteright} De brouwer stond op.}{en deed eenige stappen om}{tot mij te komen}\\

\haiku{Wij zijn rijk, schatrijk,.}{en ik lijd meer gebrek dan}{eene bedelaarster}\\

\haiku{{\textquoteleft}Gelief mij nu te,{\textquoteright},.}{volgen zeide hij zich den}{mond afvegende}\\

\haiku{{\textquoteright} En daarop ging hij,;}{weg zonder mij eenen glimlach}{te hebben gegund}\\

\haiku{Een kwart uurs later.}{draafden wij met dezelfde}{haast de stad Gent uit}\\

\haiku{Jan-oom had, dacht mij,!}{van den duivel gesproken}{en gevloekt misschien}\\

\haiku{grommend en het hoofd,.}{spijtig schuddende liep hij}{de spreekkamer uit}\\

\haiku{Achter hem stapte.}{ik door twee of drie gangen}{en beklom eene trap}\\

\haiku{gij zult onder hen.}{brave jongens en goede}{vrienden aantreffen}\\

\haiku{Die pocher van eenen.}{Waal kome nog eens hier om}{u te bespotten}\\

\haiku{Ik besloot daaruit,,,.}{dat hij hoe spotziek ook een}{goed hart moest hebben}\\

\haiku{Ik moest hem alles,.}{vertellen wat ik mij maar}{kon herinneren}\\

\haiku{De tijd, door mijnen,.....}{oom bepaald kon niet verre}{meer verwijderd zijn}\\

\haiku{uw oom is rijk, en.}{stoffelijke middelen}{ontbreken u niet}\\

\haiku{Moest ik niet voortaan?}{mij gansch toewijden aan het}{geluk van Jan-oom}\\

\haiku{{\textquoteright} mompelde hij, het.}{hoofd met eenen zonderlingen}{glimlach schuddende}\\

\haiku{Gij zult er binnen,.}{eenige dagen van weten}{te spreken Mijnheer}\\

\haiku{Daar is lekker bier,{\textquoteright}.}{uit de brouwerij van Frans}{Cools zeide hij}\\

\haiku{Mijn oom won  zijn,?}{fortuin door den handel den}{eerlijken handel}\\

\haiku{Mijne zuster is.}{gebrekkelijk en kan geene}{trappen beklimmen}\\

\haiku{Zijn   Elkander.}{de leelijkste dingen naar}{het hoofd geworpen}\\

\haiku{maar dit belet hem,.}{niet te eten en te drinken}{als een gezond mensch}\\

\haiku{Dat is de eenige,,.}{tijd dat wij vrij zijn om te}{doen wat wij willen}\\

\haiku{Nicht Margriet duwde;}{aan mijnen arm en wilde}{mij vooruit doen gaan}\\

\haiku{niets hooren dan die?}{eeuwige lamentatie}{van Jeremias}\\

\haiku{{\textquoteleft}Kozijn, wij gaan naar,,{\textquoteright}.}{boven om het avondmaal te}{nemen zeide zij}\\

\haiku{Snijd den staart van het,,!}{beestje en begin te eten}{rommeledommel}\\

\haiku{De kussens waren.}{niet opgeschud en ik moest}{ze beter schikken}\\

\haiku{Wat mij betreft, ik,.}{zal u daartoe behulpzaam}{zijn zooveel ik kan}\\

\haiku{Nog in mij zelven,.}{mompelende stond ik op}{en verliet den tuin}\\

\haiku{maar zij kwam telkens.}{terug met een barsch en}{afwijzend antwoord}\\

\haiku{Ja, donnerwetter,!}{gij zult later weten hoe}{Jan Roobeek zich wreekt}\\

\haiku{Het gezicht mijner;}{tranen voerde de woede}{van Jan-oom ten top}\\

\haiku{en die dag werd een.}{der bitterste in mijn reeds}{zoo bitter leven}\\

\haiku{de maandrozen, bij het,;}{huis had zij hem op zijnen}{naamdag geschonken}\\

\haiku{zij waren voor het,.}{genoegen dat ik hunnen}{kinderen aandeed}\\

\haiku{143.)overdreven zijn, want.}{eenen engel zelven kon men}{niet hooger prijzen}\\

\haiku{Daarover iemand te,;}{ondervragen dit zou ik}{niet gewaagd hebben}\\

\haiku{Zij had bruin haar en,.}{groote zwarte oogen vol gevoel}{en vol uitdrukking}\\

\haiku{{\textquoteright} vroeg mij Margriet, toen.}{ik haar in den gang onzer}{woning ontmoette}\\

\haiku{{\textquoteright} Zij bekeek mij eene.}{wijl met zonderlingen blik}{en schudde het hoofd}\\

\haiku{- dat ik wenschte,.}{u deel in onze blijdschap}{te kunnen geven}\\

\haiku{Dan zal mijn arme.}{vader zijne oogen niet meer}{moeten bederven}\\

\haiku{{\textquoteright} {\textquoteleft}Hemel, mijnheer, gij?}{zoudt wreed genoeg zijn om ons}{dien hoon aan te doen}\\

\haiku{Hebt gij Beatrix,?}{niet bekend dat gij smoorlijk}{op haar verliefd zijt}\\

\haiku{en om toch iets te,,.}{zeggen vroeg ik Helena}{of zij verdriet had}\\

\haiku{maar hoe kwam het toch,,,?}{dat zij dit zeggende eenen}{diepen zucht slaakte}\\

\haiku{Hij drukte mij de,:}{hand en zeide terwijl hij}{zich verwijderde}\\

\haiku{gij zijt zachtaardig,,,,,.....}{teergevoelig kiesch beleefd}{geleerd verstandig}\\

\haiku{Eene gansche week bleef,.}{ik te huis zonder eenen voet}{op straat te zetten}\\

\haiku{want anders zou een.}{van ons beiden uit dezen}{weg niet meer opstaan}\\

\haiku{Eindelijk keerde.}{zij terug en riep mij van}{beneden de trap}\\

\haiku{De arme jongen,.}{weet niet meer wat hij zegt en}{is waarlijk half dol}\\

\haiku{Volgens hem weigert.}{zij zijne hand om met u}{te kunnen trouwen}\\

\haiku{En wie heeft het recht?}{om u rekenschap van uwe}{daden te vragen}\\

\haiku{Hij komt altijd in;}{ons huis om haar leelijke}{woorden te zeggen}\\

\haiku{{\textquoteright} {\textquoteleft}Weet gij, Mijnheer,{\textquoteright} vroeg, {\textquoteleft}?}{hijwaarvan men Helena}{durft beschuldigen}\\

\haiku{{\textquoteleft}Alzoo, gij wist niet,?}{dat Helena onder de}{Linde zou komen}\\

\haiku{{\textquoteright} {\textquoteleft}Ja, er is moed toe,{\textquoteright}.}{noodig ging zij onbewogen}{en met nadruk voort}\\

\haiku{In Loochristy en te.}{Gent had ik wel een twaalftal}{kozijns en nichten}\\

\haiku{Het is, omdat gij}{een goede jongen zijt en}{ik u eene wrare}\\

\haiku{evenwel, de koele}{hardheid der laatste woorden}{van Jan-oom leenden}\\

\haiku{Ik kon mijnen angst.}{en mijne ontsteltenis}{niet meer bedwingen}\\

\haiku{Wij blijven daarna,,?}{toch immer vrij te doen wat}{wij willen niet waar}\\

\haiku{maar ik toch zal hem,.....}{zegenen en God bidden}{dat Hij hem loone}\\

\haiku{twee grepen mij bij.}{de armen en rukten mij}{eenige stappen voort}\\

\haiku{Zoo dreigend en van!}{zoo nabij had hij den dood}{hem zien aangrijnzen}\\

\haiku{{\textquoteleft}Beminde Neef, ~ {\textquoteleft}.}{Jan-oom heeft gisteren eenen}{aanval gekregen}\\

\haiku{Mijn oom was ontwaakt,.}{en ik had haast om mij voor}{hem aan te bieden}\\

\haiku{de hemel zou ons,.}{het geluk gunnen hem nog}{lang te behouden}\\

\haiku{Dit vierde deel kon.}{beloopen tot elfduizend}{vijfhonderd kronen}\\

\subsection{Uit: Volledige werken 8. Mengelingen. Het geluk van rijk te zijn. De podagrist}

\haiku{En niettemin, dat.}{verwelkt gelaat getuigt van}{vergane schoonheid}\\

\haiku{{\textquoteleft}Ik zal u wijzen,,{\textquoteright},.}{mijnheer antwoordde de man}{tot mij komende}\\

\haiku{Dan kon Sus blijven,!}{slapen totdat de pachter}{van zelf ontwaakte}\\

\haiku{onder zijne oogen;}{hing de loodvervige toon}{der onkuischheid}\\

\haiku{Daar zat de Lafheid,;}{die slapend het hoofd boog in}{den schoot des meesters}\\

\haiku{Eene uitdrukking van;}{razernij grijnsde op het}{gelaat des meesters}\\

\haiku{Ik zal met Geert er,.}{over spreken en zien hoe hij}{zelf de zaak verstaat}\\

\haiku{Ach, gevoeldet gij,,,!}{Spits wat ik lijd wat helsche}{pijnen ik doorsta}\\

\haiku{ons met minder vrees.}{dan wij de krekels op de}{heide vertrappen}\\

\haiku{gij zult leven als.....}{een landheer tusschen al de}{honden van Oostmal}\\

\haiku{Met bevende hand;}{deed de herder de riemen}{van den buidel los}\\

\haiku{Het is een geschrei,.}{en een leven van droefheid}{dat men hoort noch ziet}\\

\haiku{Betteken is een;}{lief kind met blonde haren}{en roode kaakjes}\\

\haiku{krullebol Susken;}{veegt zijn paard uit en legt de}{schalie op den grond}\\

\haiku{Maar, och arme, de!}{onnoozele schaapkens waren}{dood en bevrozen}\\

\haiku{Wat baat het mij, fraai?}{gekleed en liefelijk van}{aangezicht te zijn}\\

\haiku{die smullen aan een,;}{stuk spek en aan een schotel}{pap dat ze blinken}\\

\haiku{maar hunne wangen,!}{zijn doorgaans zoo bleek hunne}{leden zoo mager}\\

\haiku{Zoo bedriegen ze,,}{altijd de arme menschen}{die moeten erven}\\

\haiku{ten minste volgens,;}{het zeggen der lieden want}{ik weet er niets van}\\

\haiku{Zoo gij nu nog niet,,.}{aan het zot worden zijt Trees}{dan geef ik het op}\\

\haiku{verwonderd zag zij,.}{op dewijl zij geen licht op}{de trap bemerkte}\\

\haiku{De man naderde,:}{haar stak zijne hand uit en}{sprak op drogen toon}\\

\haiku{Maar, Trees, gij zult toch,,?}{redelijk zijn niet waar en}{ons geld wat sparen}\\

\haiku{Schouwvegers fijn van,,.....}{den A.B. Aardige kwasten}{Vroolijke gasten}\\

\haiku{Schouwvegers fijn van,,.....}{den A.B. Aardige gasten}{Vroolijke kwasten}\\

\haiku{Het scheen, dat deze;}{laatste overweging hem troost}{in den boezem goot}\\

\haiku{Nu was bazin Smet;}{sedert een kwart uurs uit de}{stad teruggekeerd}\\

\haiku{Die bleeke garnaat, met?}{al hare strikskens en al}{hare krullekens}\\

\haiku{Zulk goed kind, die voor,;}{haren Pauw zou sterven}{als het noodig ware}\\

\haiku{En ik zal zelfs nog,,}{bidden dat God u eene vrouw}{geve die u zoo}\\

\haiku{{\textquoteright} Van ongeduld op,:}{den grond stampende viel de}{jongen spijtig uit}\\

\haiku{het was niets anders.}{dan eene ontsteltenis}{van de zenuwen}\\

\haiku{Denk eens, Pauw, zij is....}{daar straks uitgegaan om eene}{dienstmeid te zoeken}\\

\haiku{ik geloof, dat ze!}{altemaal eenen slag van den}{molen weg hebben}\\

\haiku{{\textquoteleft}Moeder, mag ik u,?}{eens iets verzoeken eer gij}{uwen mantel aflegt}\\

\haiku{Moeten hooren, dat,!}{mijn vader gestolen heeft}{dat hij een dief is}\\

\haiku{Gij zult het erfdeel?}{van mijnen vader aan de}{Wet overleveren}\\

\haiku{{\textquoteleft}Maar Ons Heerken lief,,?}{toch weet gij nu hoe Jan Grap}{den slag heeft gedaan}\\

\haiku{{\textquoteright} {\textquoteleft}Wat kan Pauw daaraan,?}{doen dat zijnen vader een}{ongeluk overkomt}\\

\haiku{{\textquoteright} {\textquoteleft}Dit moet ik het best,{\textquoteright}.}{weten antwoordde de vrouw}{zonder aarzelen}\\

\haiku{Voor de laatste maal,,.}{ik bid u om uw eigen}{welzijn spreek waarheid}\\

\haiku{Dit liedeken van!}{mijnheeren en mevrouwen}{is uitgezongen}\\

\haiku{Dien ganschen dag, tot,.}{in den avond was er krakeel}{en droefheid in huis}\\

\haiku{{\textquoteright} galmde bazin Smet, {\textquoteleft},!}{met de angstbleekheid op het}{gelaatoh mijn geld}\\

\haiku{Het verlies van het,;}{geld moet u pijnlijk zijn}{ik gevoel het wel}\\

\haiku{het hoofd met fierheid.}{opheffen en ieder vrij}{onder de oogen zien}\\

\haiku{{\textquoteright}   Van allerlei,... (.}{vieze sprongen ter kamer}{ingehuppeldbladz}\\

\haiku{Schouwvegers fijn van,,!}{den A.B. Aardige kwasten}{Vroolijke gasten}\\

\haiku{{\textquoteright} Pauw greep Kaatje bij de;}{hand en wilde met haar de}{kamer ronddansen}\\

\haiku{{\textquoteleft}O, ja, ja, gij zult,{\textquoteright}.}{mijne goede moeder zijn}{zuchtte het meisje}\\

\haiku{En dan neuriede,:}{de dokter binnensmonds op}{dezelfde wijs}\\

\haiku{Ai, ai, het is als!}{woelde men met gloeiende}{ijzers in mijn lijf}\\

\haiku{Reeds is de kwaal tot;}{mijn ingewand en mijne}{maag opgeklommen}\\

\haiku{{\textquoteright} {\textquoteleft}Ja, mijnheer,{\textquoteright} was het, {\textquoteleft},;}{antwoordmaar hier is een brief}{die zeer haastig schijnt}\\

\haiku{{\textquoteright} Met den vinger aan,:}{het voorhoofd riep hij na een}{oogenblik stilte}\\

\haiku{O,  God, dan denk,.}{ik telkens dat mijn laatste}{uur is verschenen}\\

\haiku{mijne beslissing.}{ten uwen opzichte hangt af}{van uwen goeden wil}\\

\haiku{Er valt hier niet te,.}{aarzelen ik ga u een}{drankje bereiden}\\

\subsection{Uit: Volledige werken 9. De schat van Felix Roobeek}

\haiku{En, tot belooning,!}{had hij haar zelfs haar wettig}{erfdeel ontnomen}\\

\haiku{{\textquoteright} Zonder nog haren,,}{raad te bestrijden drukten}{wij de meening uit}\\

\haiku{{\textquoteright} En zonder onze,.}{toestemming af te wachten}{opende zij de deur}\\

\haiku{ik toonde den stal,.}{waar men de beide paarden}{zou laten rusten}\\

\haiku{Hoe menigeen had!.....}{reeds zijne verkleefdheid met}{het leven geboet}\\

\haiku{{\textquoteright} zuchtte de moeder,.}{met de handen vooruit om}{haar kind te grijpen}\\

\haiku{hoeveel last zij ons;}{dus in het midden van den}{nacht veroorzaakte}\\

\haiku{Zoudt gij weigeren,?}{het hier te houden totdat}{het genezen is}\\

\haiku{Ziet hier wat ik, in,:}{zulk geval van uwe goedheid}{voor ons kind verwacht}\\

\haiku{32.)reepels van het lijf en.}{ontnam hem insgelijks eene}{soort van grove vest}\\

\haiku{maar Margriet, die mijn,.}{inzicht merkte liep toe en}{ontrukte het mij}\\

\haiku{Ik betaalde mijn.}{glas bier en ging dubbende}{voort langs den steenweg}\\

\haiku{Zeker, dit verlies.}{moest den dapperen man zeer}{smartelijk vallen}\\

\haiku{De reismaal lag in;}{de kas zooals de \'emigr\'e ze}{er in had geplaatst}\\

\haiku{Ik ben insgelijks,.}{ervaren in allerlei}{naaiwerk gij weet het}\\

\haiku{gij zoekt een ander.}{middel van bestaan en ik}{treed in een klooster}\\

\haiku{{\textquoteleft}Niet de helft der winst,.}{zal mij toebehooren maar}{slechts het derde deel}\\

\haiku{gij beiden zult de.}{twee derden der huiskosten}{te dragen hebben}\\

\haiku{want de schout had zelf,.}{mij gezegd dat hij mij geen}{werk kon bezorgen}\\

\haiku{{\textquoteleft}Ik ben naar uw huis,{\textquoteright}, {\textquoteleft}.}{geweest zeide hijom u}{een voorstel te doen}\\

\haiku{Omdat gij het zijt,.}{zal ik u vijftien gulden}{in de maand geven}\\

\haiku{Het gezicht dezer,.}{aankondiging verraste}{noch bedroefde mij}\\

\haiku{{\textquoteright} {\textquoteleft}Het is eene erge,,{\textquoteright};}{tijding welke gij mij geeft}{zeide ik treurig}\\

\haiku{Is het niet om den?}{ons toevertrouwden schat te}{kunnen bewaren}\\

\haiku{Het werd ingezet;}{voor den spotprijs van duizend}{gulden wisselgeld}\\

\haiku{Om zulk eene zending,.}{te vervullen zijt gij niet}{stoutmoedig genoeg}\\

\haiku{het nagelaten.}{teeken van den eersten druk}{mijner vingeren}\\

\haiku{IX Margriet had slechts.}{een paar honderd gulden op}{hare reis verteerd}\\

\haiku{Hij kende de list,,.}{gromde hij en zou wel wijn}{weten te vinden}\\

\haiku{En dit was alleen.}{de voornaamste bron mijner}{bekommerdheid niet}\\

\haiku{Louis Gapet boet voor,;}{misdaden waaraan hij geen}{het minste deel had}\\

\haiku{{\textquoteright} Eenigen tijd daarna.}{verliet ik de hofstede}{en keerde naar huis}\\

\haiku{een zijner mannen.}{liep naar boven en haalde}{mijnen sleutelbos}\\

\haiku{Onze nicht had zich;}{met de pistool in de hand}{op de trap vertoond}\\

\haiku{Wij bleven nog lang;}{onder den invloed van den}{doorgestanen schrik}\\

\haiku{{\textquoteleft}Veronderstel, Felix,.}{dat wij De Blauwe Vos voor}{80,000 franken koopen}\\

\haiku{Zie eens, Felix, hoe veel!}{lieden er voor de poort der}{olieslagerij staan}\\

\haiku{Op de Markt zagen.}{wij vele lieden voor de}{olieslagerij staan}\\

\haiku{ik zag in stommen,.}{angst ten gronde want ik wist}{niet wat te zeggen}\\

\haiku{ter hand. Des avonds schreef,,.}{ik met verbrijzeld hart dit}{geld op ons debet}\\

\haiku{ik wras schatrijk en!}{behield eenen post van vijftig}{gulden in de maand}\\

\haiku{{\textquoteright} En met eenen lach, die,.}{mij op voorhand deed vreezen}{liep zij de deur uit}\\

\haiku{Het staat ons vrij, de,.}{dwalingen te vermijden}{die hij heeft begaan}\\

\haiku{{\textquoteright}        XVI Drie weken;}{later woonden wij in het}{fraaie huis aan de Markt}\\

\haiku{De prijzen waren.}{niet veranderd en neigden}{veeleer tot daling}\\

\haiku{maar ik bespeurde,;}{wel dat zij de ontrust van}{haar gemoed verborg}\\

\haiku{De gelukkigste.}{vooruitzichten maakten mij}{licht van geest en hart}\\

\haiku{Zij nam eenen stoel en,:}{zeide mij zeer bedaard doch}{met scherpen nadruk}\\

\haiku{maar sedert drie of.....}{vier jaar kan hij zijne beenen}{niet meer gebruiken}\\

\haiku{integendeel, ik.}{zal u dankbaar blijven voor}{deze eerste gunst}\\

\haiku{De jongeling had:}{een weinig van de inborst}{mijner nicht Margriet}\\

\haiku{Maar gij wilt veel meer;}{teruggeven dan wat men}{u ter hand stelde}\\

\haiku{{\textquoteright} {\textquoteleft}Uwe verkleefdheid voor,}{mij doet u de zaak in mijn}{voordeel verwringen}\\

\haiku{ik was gelukkig.}{geweest en had tweeduizend}{franken gewonnen}\\

\haiku{Maar zij, zonder zich,;}{te laten troosten zakte}{terug op den stoel}\\

\haiku{want verlaat hij ons,!}{dorp hij neemt mijn geluk en}{mijn leven mede}\\

\haiku{maar ik alleen ben,.}{de oorzaak dat hij hier niet}{langer durft blijven}\\

\haiku{Ach, hoe gelukkig!}{had ik mij geacht met u}{te kunnen blijven}\\

\haiku{Ik schudde eene wijl:}{in stille overweging het}{hoofd en zeide dan}\\

\haiku{ga, verwijder u,,.....}{van hier gij die afkeer en}{schande met u voert}\\

\haiku{haar man nam zijne,.}{muts af doch bleef in zijnen}{leunstoel gezeten}\\

\haiku{Na het wisselen,.}{van eenen groet vroeg ik hen of}{hun zoon te huis was}\\

\haiku{Ondanks al, wat ik,.}{hem kon zeggen bleef hij mijn}{aanbod verstooten}\\

\haiku{Nu bezat zij niet,;}{alleen een kind schoon en lief}{als een hemelwicht}\\

\haiku{{\textquoteright} {\textquoteleft}En gij zijt zeker,?}{dat Victor de zoon van een}{der Bohemers is}\\

\haiku{Was hij niet van edel?}{bloed en waarschijnlijk van eenen}{doorluchtigen stam}\\

\haiku{Zij had het hoofd op.}{de borst laten vallen en}{weende in stilte}\\

\haiku{{\textquoteright} {\textquoteleft}O, mijn God,{\textquoteright} riep hij, {\textquoteleft}!}{de blijdschap heeft mij belet}{daaraan te denken}\\

\haiku{broeders en zusters,;}{uws vaders en uwer moeder}{neven en nichten}\\

\haiku{{\textquoteleft}Hoe, gij biedt de hand?}{van uw engelachtig kind}{aan uwen armen klerk}\\

\haiku{{\textquoteright} {\textquoteleft}Weet gij,{\textquoteright} vroeg ik, {\textquoteleft}dat?}{dit kind aan zijnen hals een}{zeker teeken droeg}\\

\haiku{{\textquoteright} {\textquoteleft}Ik zelve, Mijnheer,,:}{heb het met een zwart snoer aan}{zijnen hals geknoopt}\\

\haiku{maar Fr\'ed\'eric, door,.}{zijn ongeduld weggerukt}{hoorde haar niet meer}\\

\subsection{Uit: Volledige werken 10. De plaag der dorpen. Eene welopgevoede dochter}

\haiku{Hij intusschen loopt,,;}{zingt tiert en vloekt tot schande}{van het heele dorp}\\

\haiku{{\textquoteright} {\textquoteleft}Het is, omdat hij.}{de pacht van verleden jaar}{nog niet heeft betaald}\\

\haiku{maar het einde is,......}{de bedelzak de misdaad}{of of nog erger}\\

\haiku{ik voelde in mij.}{eene uitnemende kracht en}{wonderlijken moed}\\

\haiku{{\textquoteright} {\textquoteleft}Maar nu, vader, nu.}{zal hij zijne toestemming}{met blijdschap geven}\\

\haiku{Ik moet er eerst nog,.}{over slapen en weten hoe}{haar vader het meent}\\

\haiku{Oh, ik wist niet, dat;}{een mensch zooveel liefde voor}{een beest kan hebben}\\

\haiku{Alle drie stonden!}{in tranen te smelten over}{den dood eener koe}\\

\haiku{{\textquoteright} {\textquoteleft}Om den achterstal,{\textquoteright}.}{zijner pacht te betalen}{zeide het meisje}\\

\haiku{en nu, nu zou ik,?}{leven tusschen vrienden}{die mij liefhebben}\\

\haiku{Jan Staers vatte de;}{boterham en nam er eenen}{beet van in den mond}\\

\haiku{hij is een brave,.}{jongen die arbeidt van den}{morgen tot den avond}\\

\haiku{{\textquoteright} {\textquoteleft}Zie, gebuur,{\textquoteright} sprak de, {\textquoteleft};}{oudehet is nutteloos}{met mij te veinzen}\\

\haiku{doch bij het einde.}{wondden hem de bloedige}{verwijten zeer diep}\\

\haiku{Weinig tijds daarna.}{opende zich de deur der}{steenen hoeve opnieuw}\\

\haiku{Waart gij in zijne,.}{plaats gij zoudt misschien nog meer}{vergramd zijn dan hij}\\

\haiku{{\textquoteright} {\textquoteleft}Zie, Lucas, zoo gij,.}{geen geduld wilt hebben ik}{kan er niet aan doen}\\

\haiku{het is vandaag slecht,.}{weder morgen zal de zon}{misschien wel schijnen}\\

\haiku{doch zijn vader was:}{hem vooruit en sprak met een}{bevelend gebaar}\\

\haiku{Zie, nu drijft hij de,.}{lieden uiteen omdat zij}{te nader komen}\\

\haiku{Kunt gij het zien en,?}{daar koud blijven staan gelijk}{een steen zonder ziel}\\

\haiku{ik heb hem dezen,}{morgen in den hollen weg}{ontmoet hij had eenen}\\

\haiku{Maar wij hebben geen.}{recht om de dochter van den}{vader te scheiden}\\

\haiku{{\textquoteleft}Kom, laat ons geenen tijd,,:}{verliezen Beth neem wat er}{noodig is tot schuren}\\

\haiku{maar ik zie wel, dat.}{de ellende zelve u}{niet heeft veranderd}\\

\haiku{de arbeid, dien men,;}{met goeden wil aanvaard heeft}{nog niemand gedood}\\

\haiku{ik zal u toonen,.}{dat gij geene redenen hebt}{om te wanhopen}\\

\haiku{Uw voorspoed was mij,.....}{een eeuwig verwijt dat ik}{niet kon verkroppen}\\

\haiku{{\textquoteright} {\textquoteleft}Maar onderwerpt ge,?}{u aan de proef met goeden}{wil en in vriendschap}\\

\haiku{Wij zullen samen.}{nog goede dagen op de}{wereld beleven}\\

\haiku{Ik zal loopen en,.}{hem gaan zeggen dat hij wat}{stiller moet rijden}\\

\haiku{Hij poogde hun te,;}{doen begrijpen dat alles}{nog ten beste ging}\\

\haiku{Zij greep zijne hand,,:}{en hem smeekende in de}{oogen ziende vroeg zij}\\

\haiku{Drinken, drinken en,!}{daar nedervallen zonder}{rede zonder ziel}\\

\haiku{De Schallebijter,.....}{zou u wegjagen zoo gij}{nog eens eenen druppel}\\

\haiku{Jan Staers, jongen, wat,!}{zijt gij leelijk met die groote}{glasachtige oogen}\\

\haiku{{\textquoteleft}Vaarwel,{\textquoteright} mompelde,.}{de zandboer zich weder tot}{de deure richtend}\\

\haiku{Men moet ze weten,.}{aan te pakken of het is}{er verkeerd mede}\\

\haiku{Een telloor, waar een,;}{stuk afgebroken is kan}{nog haren dienst doen}\\

\haiku{Wacht maar eens, totdat.}{gij aan het kapittel van}{de kinderen komt}\\

\haiku{maar ik trapte hem,:}{al gauw op zijnen teen en}{dan eerst zeide hij}\\

\haiku{Ik had hem tegen,?}{te half vier hier verzocht en}{het is al vier uur}\\

\haiku{Dat zal hem leeren, mij,.}{ook te plagen gelijk hij}{Zondag heeft gedaan}\\

\haiku{{\textquoteleft}Ja, maar ik was nog,.}{geen vijf stappen verder of}{daar lag er nog een}\\

\haiku{De ware reden.}{van mijnen brief is echter}{het voorgaande niet}\\

\haiku{Het meisje is niet,.}{schoon maar ze speelt uitmuntend}{op de piano}\\

\haiku{{\textquoteleft}Comment, Messieurs,.....?}{vous avez pu croir eque cette}{pauvre Virginie}\\

\haiku{{\textquoteright} De goede lieden.}{toonden zich tevreden over}{mijne belofte}\\

\haiku{hoe iedereen, van,:}{den morgen tot den avond haar}{met vleitaal overlaadt}\\

\haiku{Ik heb daareven een,.}{brief ontvangen die in het}{Fransch is geschreven}\\

\haiku{Van daar gingen wij.}{naar de Kroon en bleven er}{tot laat in den avond}\\

\haiku{Hare moeder was.}{boven om haar te troosten}{en te verzorgen}\\

\haiku{{\textquoteleft}Mijnheer en Mevrouw,.}{ik verlang een ernstig woord}{u toe te richten}\\

\haiku{{\textquoteleft}Kan geld opwegen?}{tegen het leven en het}{geluk van mijn kind}\\

\haiku{Dat zij M. Gustaf,;}{ten hoogste genegen is}{dit wil ik gelooven}\\

\haiku{{\textquoteright} stamelde hij, van.}{ontsteltenis op zijne}{beenen wankelende}\\

\haiku{Was mijn kameraad?}{langs eene andere baan mij}{vooruitgeloopen}\\

\haiku{Het verschrikte mij,;}{weder zoo geheel alleen}{te moeten blijven}\\

\haiku{{\textquoteleft}Ach, Mijnheer Bernard,,.}{ik weet wel waarover gij mij}{wilt onderhouden}\\

\haiku{Ik aarzelde om.}{uwen armen vriend dien wreeden}{slag toe te brengen}\\

\haiku{{\textquoteleft}O, Mijnheer Spronck,;}{wacht nog eenige dagen}{om te beslissen}\\

\haiku{{\textquoteright} {\textquoteleft}Blijf bedaard, Gustaf,{\textquoteright}, {\textquoteleft}.....}{zeide ikde zaken staan}{ginder niet gunstig}\\

\haiku{Spronck  schrijven.}{en hare toestemming tot}{zijn bezoek vragen}\\

\haiku{Mijne belofte;}{aan den kapitein moest ik}{evenwel vervullen}\\

\haiku{Hij is daar, achter,.}{gindsche hofstede uit ons}{gezicht verdwenen}\\

\haiku{Ik luisterde niet.}{meer en trad met jagend hart}{in Gustafs kamer}\\

\haiku{{\textquoteleft}Bernard,{\textquoteright} zeide hij, {\textquoteleft}.}{ik heb een ernstig verzoek}{u toe te richten}\\

\haiku{maar ik houd er aan,.}{dat gij nooit vergetet wat}{ik u ga vragen}\\

\haiku{M. Wijkevorst had,,.}{haar voor een paar maanden}{eenen brief geschreven}\\

\subsection{Uit: Volledige werken 11. De Boerenkrijg}

\haiku{{\textquoteleft}Zijt ge niet beschaamd,,?}{groot mensch dat ge daar ligt te}{huilen als een kind}\\

\haiku{Aan de andere.}{zijde van het doek stond eene}{vrouw met eene viool}\\

\haiku{De beul zal komen,.....}{af te kappen Of anders}{ben ik niet voldaan}\\

\haiku{{\textquoteright} {\textquoteleft}Ziet daar, burgers en,.}{boeren hoe Robespierre}{zelf staat te beven}\\

\haiku{Het volk, als razend, '.}{bond Charlot En sleurde haar}{naart duister kot}\\

\haiku{men sprak er juichend.}{en met luider stemme over}{eene goede tijding}\\

\haiku{Gansch Waasland is op....}{dit oogenblik overdekt met}{Fransche soldaten}\\

\haiku{Meer dan zeshonderd;}{priesters zijn reeds naar verre}{eilanden vervoerd}\\

\haiku{en, wil God hem de,.}{martelkroon gunnen hij zal}{ze niet ontvlieden}\\

\haiku{doch een licht deurken,,.}{voor den trap dat gesloten}{was we\^erhield hem}\\

\haiku{eenigen schoten op;}{den vliedenden priester en}{op zijnen gezel}\\

\haiku{Een blij gejuich, een:}{angstige vreugdekreet stond}{op in den tempel}\\

\haiku{In de verte zag;}{men nog eenige vrouwen en}{kinderen vluchten}\\

\haiku{hij had een papier.}{en pennen voor zich liggen}{en scheen te schrijven}\\

\haiku{Die domme baas heeft,;}{ons wijsgemaakt dat hij van}{onze komst niets wist}\\

\haiku{In twee, drie uren zou,;}{hij al de dieren slachten}{die in het dorp zijn}\\

\haiku{Ik zal Jan van den,.}{notaris verzoeken dat}{hij ze ga halen}\\

\haiku{{\textquoteright} Allen zagen hem,.}{met nieuwsgierigheid aan doch}{niemand antwoordde}\\

\haiku{{\textquoteright} Simon-Brutus.}{en zijne makkers schoten}{in eenen langen lach}\\

\haiku{Een paard kan naar de,.}{herberg niet gaan het is een}{menschelijk gebrek}\\

\haiku{{\textquoteright} Simon-Brutus.}{vatte de pen en schreef het}{gevraagde bevel}\\

\haiku{{\textquoteright} Maar de korporaal,:}{greep hem bij den kraag schudde}{hem hevig en riep}\\

\haiku{{\textquoteright} De vrouw antwoordde.}{niet en bleef met het hoofd in}{de handen zitten}\\

\haiku{gij zoudt mij grootelijks,.}{verplichten zoo gij mij dit}{wildet toelaten}\\

\haiku{boven haar hoofd, op,,.}{de trap brandde de lamp die}{zij er had geplaatst}\\

\haiku{Hij was uitnemend.}{bleek en scheen in al zijne}{leden te beven}\\

\haiku{{\textquoteright} Maar de sergeant:}{wees met verwondering naar}{buiten en zeide}\\

\haiku{het scheen hem zelfs, dat.}{hij eenig bijna onvatbaar}{gerucht had gehoord}\\

\haiku{Ik gebied u in,!}{naam der Fransche Republiek}{verklaar wat gij weet}\\

\haiku{Welnu, Citoyen,,,?}{Torfs nog eens zult gij zeggen}{wat gij weet of niet}\\

\haiku{een der soldaten,,.}{in het been getroffen viel}{neder in het zand}\\

\haiku{{\textquoteright} Simon-Brutus:}{stampte ongeduldig op}{den grond en zeide}\\

\haiku{Breng insgelijks den,.}{Citoyen binnen die mij}{verlangt te spreken}\\

\haiku{En toch, dit alles!}{brengen wij u in naam der}{Fransche Republiek}\\

\haiku{zijn leven is ten,.}{einde hij zal sterven in}{de gevangenis}\\

\haiku{gun hem genade!}{om nevens zijn kerkje in}{vrede te sterven}\\

\haiku{in de rechterhand;}{hield hij  met krampachtig}{geweld een geweer}\\

\haiku{{\textquoteright} antwoordde de knecht;}{met luider stemme en naar}{het dorp wijzende}\\

\haiku{{\textquoteright} Intusschen waren;}{de andere personen}{tot Jan genaderd}\\

\haiku{Ik beef, alsof uw.}{mond mij de schrikkelijkste}{tijding melden moest}\\

\haiku{{\textquoteright} {\textquoteleft}Ah,{\textquoteright} riep Bruno uit, {\textquoteleft},!}{ik denk aan mijnen vader}{aan mijne moeder}\\

\haiku{En nu, geen ontzien,,,, -,!.....}{meer geene rust geene vrees geene hoop}{zelfs wraak wraak alleen}\\

\haiku{daaruit bleek, dat de.}{knecht zich in zijne gissing}{niet had bedrogen}\\

\haiku{Niemand verroerde,:}{iedereen weerhield de kracht}{zijner ademhaling}\\

\haiku{{\textquoteright} Bevend liet hij de.}{maagd ten gronde zakken en}{viel bij haar neder}\\

\haiku{zijn haar was verward,.}{de oogen gloeiden hem in het}{hoofd van vermoeidheid}\\

\haiku{geen enkele straal:}{der hoop daalde in zijnen}{benauwden boezem}\\

\haiku{zij doet geweld om.}{uit het verbrijzeld lichaam}{zich los te rukken}\\

\haiku{{\textquoteleft}Vrienden, luistert met.}{koelen bloede op hetgeen}{ik u melden ga}\\

\haiku{Karel uit de Leeuw:}{wrong zijn geweer krampachtig}{in de vuist en riep}\\

\haiku{{\textquoteleft}Die brief bewijst, dat;}{ik een afgezondene}{van uwe vrienden ben}\\

\haiku{zij dan de benden,.}{die de Franschen ons op het}{lijf zenden willen}\\

\haiku{En  zoudt gij hem?}{dan verlaten om u aan}{mijn lot te hechten}\\

\haiku{ik ook beroep mij,.}{op uwe edelmoedigheid op}{uwe menschenliefde}\\

\haiku{{\textquoteright} {\textquoteleft}O, Simon,{\textquoteright} zuchtte, {\textquoteleft}.}{de maagd op grievenden toon}{ik durf niet spreken}\\

\haiku{Eene doodsche stilte;}{heerschte tusschen deze}{ongelukkigen}\\

\haiku{Eene doodsche stilte.}{heerschte tusschen deze}{ongelukkigen}\\

\haiku{Mij dunkt, die windels;}{om zijn hoofd drukken te veel}{op zijne wonde}\\

\haiku{{\textquoteleft}Ik vermeen, dat wij.}{binnen de twee uren zullen}{kunnen terug zijn}\\

\haiku{{\textquoteright} {\textquoteleft}Ach, Kaat lief,{\textquoteright} smeekte, {\textquoteleft},;}{Genovevaom Gods wil}{laat ons spoed maken}\\

\haiku{op die voorwaarde.}{alleen heeft zij toegestemd}{om mij te volgen}\\

\haiku{{\textquoteright} {\textquoteleft}Uw ontwerp is goed,,{\textquoteright}.}{en gelukkig Veva lief}{antwoordde de knecht}\\

\haiku{Wees zeker, Veva,.}{de ongelukkige man}{zal er van sterven}\\

\haiku{Waar men haar naartoe,.}{geleid had die schuilplaats kon}{niemand ontdekken}\\

\haiku{Nu staat hij gewis,.}{op den heuvel om uit te}{zien of Jan niet komt}\\

\haiku{{\textquoteright} De brouwer scheen te.}{ontwaken en zag de maagd}{ondervragend aan}\\

\haiku{Dat ik niet vroeger,.}{tot hier geraakt ben mag u}{niet verwonderen}\\

\haiku{Een der ruiters kwam.}{in vollen draf in de baan}{teruggereden}\\

\haiku{niet minder was de.}{vernieling onder eenige}{andere vendels}\\

\haiku{{\textquoteright} Een zelfde roep steeg;}{op uit het gansche leger}{der patriotten}\\

\haiku{Slechts een honderdtal.}{mannen had hij binnen de}{vesting gelaten}\\

\haiku{De Generaal deed;}{de trommels slaan en gaf het}{verlangde teeken}\\

\haiku{Twee uren tijds werden.}{hem vergund om zich tot den}{dood te bereiden}\\

\haiku{Alle volkeren,,.}{de wilden zelfs hopen op}{een beter leven}\\

\haiku{{\textquoteleft}Ach, vader, wat ben,!}{ik blijde dat ik u nog}{eens mag aanschouwen}\\

\haiku{Aanvaarden wij het.....{\textquoteright} {\textquoteleft},,,.}{lot zooals het isNeen Simon}{wanhoop niet mijn zoon}\\

\haiku{{\textquoteleft}Ik heb den dood van;}{Bruno als een langgewenscht}{geluk nagejaagd}\\

\haiku{Gij, kapitein van,;}{Waldeghem zult met uw}{vendel vooruitgaan}\\

\haiku{Intusschen vlogen;}{de kogels met vernieuwde}{kracht boven zijn hoofd}\\

\haiku{Allen verspreidden,.}{zich door het gehucht om eene}{rustplaats te vinden}\\

\haiku{sommigen hebben;}{hoofd of arm met bebloede}{doeken omwonden}\\

\haiku{Eindelijk, dezen,;}{merken wel dat geene tegen}{weer meer kan baten}\\

\haiku{bij het gezicht van.....}{dien nieuwen vijand bevangt}{hen een doodelijke schrik}\\

\haiku{Zij zien eene opening,;}{in den kring door de komst der}{ruiters veroorzaakt}\\

\haiku{De bezwijkende?}{deugd moet zich schamen voor het}{zegepralend kwaad}\\

\subsection{Uit: Volledige werken 14. De burgemeester van Luik}

\haiku{allen, klein en groot,{\textquoteright} {\textquoteleft}?}{staan wij hier rondom u.En}{de vijf anderen}\\

\haiku{{\textquoteright} {\textquoteleft}Het zijn verraders,{\textquoteright}.}{of bespieders zeide een}{der jonge mannen}\\

\haiku{maar dien linkschen, dien.}{schijnheiligen blik heb ik}{meer dan eens gezien}\\

\haiku{{\textquoteleft}Twijfel er niet aan,{\textquoteright}, {\textquoteleft}.}{was het antwoordik ben nog}{wat ik vroeger was}\\

\haiku{{\textquoteleft}Ik geloof waarlijk,.}{dat wij de goede richting}{hebben verloren}\\

\haiku{{\textquoteright} Het meisje hief het:}{hoofd op en antwoordde met}{eenen zoeten glimlach}\\

\haiku{Zij schoof terzijde,;}{op de bank als om hem eene}{plaats aan te wijzen}\\

\haiku{De droeve tijden,,.}{die wij beleven kunnen}{niet eeuwig duren}\\

\haiku{Zoo wreed zijn voor eenen,!}{kameraad voor eenen armen}{dienstbode als gij}\\

\haiku{{\textquoteleft}Het is de stem van,{\textquoteright}.}{Dani\"el Laruelle}{zeide Elisakeith}\\

\haiku{van den eersten dag?}{af heb ik uw inzicht ten}{volle begrepen}\\

\haiku{Ik dacht, dat er een;}{einde aan die comedie}{moest gesteld worden}\\

\haiku{{\textquoteright} {\textquoteleft}Elisabeth, gij weet,.}{het de macht der zwakken is}{de voorzichtigheid}\\

\haiku{Is het niet de stem,?}{van mijnheer de Saizan die}{ik in de gang hoor}\\

\haiku{En wat mag ik den?}{burgemeester in naam des}{konings aanbieden}\\

\haiku{Poog te weten, wat,.}{hem zou kunnen verleiden}{en beloof het hem}\\

\haiku{{\textquoteright} {\textquoteleft}Maar die wisselaar,?}{zal dus weten dat ik geld}{ontvang van Frankrijk}\\

\haiku{Haddet gij u aan?}{zulke onderscheiding wel}{durven verwachten}\\

\haiku{Ik heb hem gevraagd,,;}{of hij niet verlangde u}{te groeten mevrouw}\\

\haiku{maar hij wilde u.}{niet storen en zal tegen}{den avond terugkeeren}\\

\haiku{{\textquoteleft}Ik herken zijne!}{wijze van kloppen en loop}{om hem te openen}\\

\haiku{Mejuffer Clara.}{bovenal boezemde mij}{medelijden in}\\

\haiku{Laat mij evenwel u}{nog eenige woorden zeggen}{om de vernieuwing}\\

\haiku{En nu, mijn waarde,.}{vriend vraag ik u op mijne}{beurt om verschooning}\\

\haiku{Ha, Laruelle,,,!}{ha de Mouzon gij hebt nog}{niet gedaan met mij}\\

\haiku{Inderdaad, ik heb,.}{het goed geoordeeld mij niet}{te doen aanmelden}\\

\haiku{, want ik heb daar eenen.....}{bijzonderen bode tot}{mijne beschikking}\\

\haiku{Toen de graaf zijne,:}{uitlegging eindigde vroeg}{de burgemeester}\\

\haiku{{\textquoteleft}Neen, heer resident,.}{gij zult mij toelaten dien}{naam te verzwijgen}\\

\haiku{Wij roepen dezen,.}{tot ons alsook de leden}{der schuttersgilden}\\

\haiku{De knecht Jaspar:}{trad in de zaal en zeide}{tot zijnen meester}\\

\haiku{mijn geweten zou.}{het mij verwijten tot op}{den boord van mijn graf}\\

\haiku{{\textquoteright} {\textquoteleft}Ik ben maar een smid,{\textquoteright}.}{antwoordde de kapitein}{met eenen diepen zucht}\\

\haiku{{\textquoteright} {\textquoteleft}Welnu, het lot heeft,{\textquoteright}.}{recht over hen gedaan zeide}{de burgemeester}\\

\haiku{anderen, gekwetst,.}{hadden hoofd of armen met}{doeken omwonden}\\

\haiku{maar hij bedwong zich,:}{greep de beide handen zijns}{makkers en zeide}\\

\haiku{hij keerde zelfs zich,;}{geheel om als wilde hij}{het huis verlaten}\\

\haiku{{\textquoteright} Een hevig schaamrood.}{beklom haar voorhoofd en zij}{staarde ten gronde}\\

\haiku{Ik verberg het u,,,}{niet Dani\"el hij hoopt dat}{hij door zijnen raad}\\

\haiku{Hij is uw vriend, gij.}{hebt hem de gewichtigste}{diensten bewezen}\\

\haiku{En gij gelooft, dat?}{de graaf van Warfuz\'ee zou}{kunnen toestemmen}\\

\haiku{Trad ik dus binnen,.}{zonder aanmelding het is}{de schuld van Gobert}\\

\haiku{Ik zeg u dit slechts,}{omdat men somwijlen eenen}{bekenden Chiroux}\\

\haiku{Het is enkel een,.}{uitstel dat door zich zelf wel}{betamelijk is}\\

\haiku{Welnu,{\textquoteright} zeide De, {\textquoteleft}.}{Vrieseopenbaar mij het}{schrikkelijk geheim}\\

\haiku{Uw duistere brief.}{heeft ons alle te Brussel}{met angst geslagen}\\

\haiku{{\textquoteleft}Alzoo is voor het.}{oogenblik het doel onzer}{bijeenkomst bereikt}\\

\haiku{Alles drijft mij aan.}{om u een onmiddellijk}{vaarwel te zeggen}\\

\haiku{Hadde hij zijn doel,;}{getroffen alles ware}{verloren geweest}\\

\haiku{{\textquoteright} Deze overweging.}{ontrukte hem eenen kreet van}{hoogmoed en blijdschap}\\

\haiku{{\textquoteleft}Ja, ja, ik ben de!}{stadhouder des keizers in}{het prinsdom van Luik}\\

\haiku{Laruelle heeft,.}{zijn dood verdiend en ik ik}{spreek zijn vonnis uit}\\

\haiku{De kanunniken.}{hebben den verrader bij}{het volk aangeklaagd}\\

\haiku{Zie, zie, gij zijt niets,!}{dan bloed rookend bloed van het}{hoofd tot de voeten}\\

\haiku{Hij maakte met haast:}{de armen van Jaspar}{los en zeide hem}\\

\subsection{Uit: Volledige werken 15. De geldduivel}

\haiku{honderden stemmen}{klinken uit struik en heester}{en zingen den Heer}\\

\haiku{Eene wijl hield hij den:}{blik met klimmenden angst in}{de ruimte gericht}\\

\haiku{{\textquoteright} Over het gelaat van.}{M. Kemenaer was eene wolk}{van spijt neergezakt}\\

\haiku{{\textquoteright} {\textquoteleft}Het is schoon, omdat,{\textquoteright}.}{het u verblijdt murmelde}{de maagd met koelheid}\\

\haiku{aan de deur van het.....}{geld moet komen kloppen en}{eene aalmoes vragen}\\

\haiku{{\textquoteright} {\textquoteleft}Veel goedheid, te veel,,{\textquoteright}.}{goedheid mijnheer Kemenaer}{antwoordde Koenraad}\\

\haiku{Ik zal terugkeeren,;}{al ware het twee- of}{driemaal op eenen dag}\\

\haiku{Zij stak haren arm,:}{in den zijnen en zijde}{op streelenden toon}\\

\haiku{het is maar, opdat.}{ik bereid zou zijn tegen}{dat Berthold komt}\\

\haiku{Het geld heerscht over,;}{bijzondere menschen niet}{over de menschheid}\\

\haiku{Zijn het de namen,?}{van mannen die boogden op}{de macht van het geld}\\

\haiku{Monck liet zijne;}{armen eindelijk op den}{lessenaar vallen}\\

\haiku{Hij stapte zelfs naar,;}{de deure toe alsof hij}{iemand verwachtte}\\

\haiku{want, wees zeker, de.}{vrek zou iets kunnen krijgen}{en ons ontsnappen}\\

\haiku{Berthold heeft eenen;}{put v\'o\'or zijne eigene}{voeten gegraven}\\

\haiku{duivel bedriegen,?}{die zich in groot gevaar brengt}{om u te dienen}\\

\haiku{{\textquoteleft}Het zij dan zoo, mits!}{er geen ander middel is}{om mij te redden}\\

\haiku{{\textquoteright} {\textquoteleft}Nog de vrouw, aan wie,.}{ik gisteren zeide dat}{gij niet te huis waart}\\

\haiku{Zeg mij hoeveel gij,{\textquoteright}.}{noodig hebt morde Robyn met}{spijtig ongeduld}\\

\haiku{Maar Monck hoestte,;}{om de aandacht zijns meesters}{tot zich te trekken}\\

\haiku{- De grijsaard scheen voor:}{den raad van zijnen klerk te}{zwichten en zeide}\\

\haiku{mijnen zoon van de,.....}{gevangenis misschien van}{den dood verlossen}\\

\haiku{Waarschijnlijk waren.}{het duizend franken in het}{water gesmeten}\\

\haiku{Nog eenige dagen,.....}{indien het niet beter met}{mijne borst wil gaan}\\

\haiku{Gij hadt ongelijk,,.}{Monck hem zulk gevaarlijk}{spel aan te raden}\\

\haiku{hij zal nu niet veel.}{neiging hebben om verzen}{te hooren lezen}\\

\haiku{{\textquoteright} Een krachtige klank;}{der bel onderbrak hem in}{zijne uitroeping}\\

\haiku{Indien de kunst het?}{eenig middel ware om dit}{doel te bereiken}\\

\haiku{misschien was het de,:}{droefheid zijns dienaars die hem}{tot bewustheid riep}\\

\haiku{niet sterven, nog niet,{\textquoteright}.}{sterven stamelde Robyn}{met halve stemme}\\

\haiku{De voorzitter van.}{het gerechtshof mag alleen}{het zegel breken}\\

\haiku{God behoede u!}{voor de uitvoering van uw}{noodlottig besluit}\\

\haiku{Dus vertrouwend, treed.}{ik binnen en bied mijnen}{oom het boekdeel aan}\\

\haiku{maar van al deze.}{beschouwingen was mijne}{natuur afkeerig}\\

\haiku{Ik geraakte op,;}{eenen zolder ik leed honger}{en vernedering}\\

\haiku{{\textquoteright} riep de jongeling, {\textquoteleft};}{verbaasd en bevendmaar het}{is onmogelijk}\\

\haiku{laat een nieuw blad met.}{andere verzen drukken}{en er invoegen}\\

\haiku{hij treurt nu misschien,.}{reeds over het verdriet dat hij}{mij heeft aangedaan}\\

\haiku{Met het hoofd tegen:}{de borst des muzikants liet}{vallen.mij zelven}\\

\haiku{de zwoeging harer.}{borst vervult de kamer met}{eentonig geluid}\\

\haiku{Als al degenen,,!}{die de menschen bedriegen}{moesten terugkomen}\\

\haiku{want, ik weet niet, ik,!}{ben zoo flauw aan mijn hart en}{het is hier zoo koud}\\

\haiku{maar toch ben ik door.}{mijne gedachten te vroeg}{uit het bed gejaagd}\\

\haiku{maar het was toch niet,!}{voor zijne schoone oogen den}{knorrigen buffel}\\

\haiku{{\textquoteright} {\textquoteleft}Zeg eens, Margriet, is,?}{het waar dat hij boven het}{millioen rijk was}\\

\haiku{Het is mogelijk,;}{dat mij een onvoorzichtig}{woord is ontvallen}\\

\haiku{{\textquoteright} {\textquoteleft}Mijnheer heeft gelijk,,{\textquoteright}.}{volstrekt gelijk viel Monck}{hem in de rede}\\

\haiku{Allerlei droeve;}{overwegingen hadden mij}{belet te slapen}\\

\haiku{Zij meent, dat daarin;}{het middel bestaat om mij}{desnoods te dwingen}\\

\haiku{{\textquoteright} Het hoofd oprichtend,:}{antwoordde Berthold op}{wanhopigen toon}\\

\haiku{{\textquoteright} riep de jongeling,.}{al gaande de hand zijns vriends}{dankbaar drukkende}\\

\haiku{want het geld mijns  .....}{ooms zou mij het hart vervuld}{hebben met wroeging}\\

\haiku{Hij zegt, dat mijn oom.}{op zijne aanbeveling}{het hem heeft verzocht}\\

\haiku{In het huis, waar ik,:}{eene kamer bewoon staat een}{fraai kwartier ledig}\\

\haiku{het is een geluk,.....}{dat mij met dankbaarheid ten}{hemel doet blikken}\\

\haiku{eene ontroering, die,.}{hem zelven verraste had}{hem aangegrepen}\\

\haiku{Neen, neen, daartoe is.}{zijne vaderlijke}{liefde te innig}\\

\haiku{Heeft Berthold dan?}{nog de stoutheid gehad om}{ten uwent te komen}\\

\haiku{{\textquoteright} {\textquoteleft}Welaan, heer Monck,!}{dit gaat op uwe gezondheid}{en op uw geluk}\\

\haiku{Nu, in dit geval,.}{ziet gij mij voor de laatste}{maal heer Kemenaer}\\

\haiku{Mijn goede Monck,,,!}{onder ons gezegd gij zijt}{leelijk zeer leelijk}\\

\haiku{Zij stierve van schrik.}{bij de gedachte alleen}{van zulk huwelijk}\\

\haiku{zijn gelaat bewoog,.}{krampachtig zijne stem was}{dor en ratelend}\\

\haiku{{\textquoteleft}Alzoo, ik zou de,?}{echtgenoote worden van eenen}{man dien ik niet ken}\\

\haiku{Mijne hand en mijn?}{hart zouden de prijs worden}{eener somme gelds}\\

\haiku{ik verwachtte mij.}{aan deze genegenheid}{van uwentwege niet}\\

\haiku{voedsel zoeken voor,.....}{de koorts die mijne ziel en}{mijn lichaam verslindt}\\

\haiku{{\textquoteright} De muzikant greep:}{Bertholds beide handen}{en sprak met nadruk}\\

\haiku{Arme vriend, ik zal:}{u niet vragen welke kwaal}{u doet verkwijnen}\\

\haiku{Zoo deed ook de zoon,:}{van mijnen pachter toen hij}{op het trouwen stond}\\

\haiku{en ik uit goedheid,,;}{uit vriendschap ik laat u hier}{meesteresse zijn}\\

\haiku{Waarom komt Laura's,?}{naam altijd op uwe lippen}{als gij alleen zijt}\\

\haiku{Maar gij begrijpt wel,,;}{Margriet dat dit huwelijk}{moet worden belet}\\

\haiku{Blijf voortaan gerust.}{en luister niet meer naar den}{praat der geburen}\\

\haiku{Zij maakt zich ziek en.}{verkwijnt om haar huwelijk}{te doen verdagen}\\

\haiku{dieper waren de;}{rimpels des kommers in zijn}{voorhoofd gegraven}\\

\haiku{Hij liet de handen.}{nedervallen en legde}{ze aan zijn voorhoofd}\\

\haiku{Laura bemerkte,;}{haren vader eerst toen hij}{haar zeer nabij was}\\

\haiku{beloof mij, dat gij.....}{het beeld des doods van voor uwe}{oogen zult verjagen}\\

\haiku{Gij verlangt immers,?}{niet meer dat het huwelijk}{worde uitgesteld}\\

\haiku{En dat het de bloote,;}{waarheid bevat daaraan is}{niet te twijfelen}\\

\haiku{Gedurende eene.}{lange wijl wandelden zij}{allen zwijgend voort}\\

\haiku{gij zult de schoonste,.}{bruid zijn die men in lange}{jaren heeft gezien}\\

\haiku{Drie slechte stoelen;}{en eene tafel vormden er}{het gansche huisraad}\\

\haiku{{\textquoteright} De zieke hief de:}{oogen ten hemel en zuchtte}{op grievenden toon}\\

\haiku{{\textquoteleft}Welke waardigheid?}{blijft er mij in mij zelven}{te eerbiedigen}\\

\haiku{de bleekheid was op;}{zijn gelaat door den blos der}{hitte vervangen}\\

\haiku{daarom klopte het.}{hart der beide vrienden bij}{het openen des briefs}\\

\haiku{er zou toch voor mij.}{geen enkele rustige}{dag meer kunnen zijn}\\

\haiku{Morgen zal Laura.}{voor Gods altaar de hand van}{Monck aanvaarden}\\

\haiku{Nu, goede vrouw, zet,{\textquoteright}.}{u neer en wees gerust sprak}{hij met minzaamheid}\\

\haiku{{\textquoteright} viel de muzikant.}{met aangejaagd ongeduld}{haar in de rede}\\

\haiku{- en hij heeft mij wat,.....}{anders wijsgemaakt om dit}{te doen vergeten}\\

\haiku{en haar vader had.}{al lang in haar huwelijk}{met hem toegestemd}\\

\haiku{Al de anderen;}{luisterden met open mond en}{opgespalkte oogen}\\

\haiku{Laura stond in hun.}{midden met hangend hoofd en}{halfgesloten oogen}\\

\haiku{zoo vol spot, zoo vol,.}{droefheid dat de vrouw verbaasd}{achteruitdeinsde}\\

\haiku{verberg als ik voor,;}{iedereen wat pijnen uwen}{boezem doorwoelen}\\

\haiku{{\textquoteleft}Mijn God, mijn God,{\textquoteright} kreet,, {\textquoteleft}}{Kemenaer zich de handen}{voor de oogen slaande}\\

\haiku{Daar, neem uw handteeken,{\textquoteright}, {\textquoteleft}.}{zeide hijhet is Laura}{die ik hebben moet}\\

\haiku{{\textquoteright} Berthold legde.}{zich de handen voor de oogen}{en bleef sprakeloos}\\

\haiku{De rechter heeft het;}{noodlottig huwelijk niet}{kunnen beletten}\\

\haiku{uwe liefde alleen,.}{kan haar behoeden voor het}{graf dat op haar gaapt}\\

\subsection{Uit: Volledige werken 16. Eene gekkenwereld. De twee vrienden. Rikke-tikke-tak}

\haiku{niet alleen scheen hij;}{wel dertig voet loodrechte}{hoogte te hebben}\\

\haiku{Het was echter geene;}{eigenlijke verschriktheid}{die mij ontstelde}\\

\haiku{Wij hebben tijd om.}{wat van Antwerpen en de}{vrienden te kouten}\\

\haiku{Luister, zij zingt het,:}{eeuwige het eenige lied}{dat men hier kent}\\

\haiku{Daar opende men, recht,:}{over mij eene deur welke ik}{niet had opgemerkt}\\

\haiku{Houdt  uwe tanden,!}{gesloten of ik streel uwen}{rug met mijne knots}\\

\haiku{{\textquoteright} uit de alkoof, en.}{de kat springt huilend tusschen}{de gordijnen door}\\

\haiku{{\textquoteleft}Met dit haspelen.}{en dit schreeuwen geraken}{wij tot geen besluit}\\

\haiku{Lieve man, het is,{\textquoteright}.}{moeilijk met u te kouten}{begon vrouw Noppe}\\

\haiku{Hij is landmeter.}{en zal zich dit ambacht met}{meer vlijt aantrekken}\\

\haiku{{\textquoteright} Baas Noppe slaakte.}{eenen zucht en wreef zich met de}{hand over het voorhoofd}\\

\haiku{Neen, vrouw, dit doe ik,,,,!}{niet zeg ik u noch vandaag}{noch morgen noch ooit}\\

\haiku{Lisa, gij weet dat.}{gij met eenen korf eieren}{naar den winkel moet}\\

\haiku{{\textquoteright} {\textquoteleft}Gij zegt het om te,,{\textquoteright}.}{lachen majoor wedersprak}{zijn jonge gezel}\\

\haiku{zij zouden dus nu.}{maar heengaan en tegen den}{middag wederkeeren}\\

\haiku{Hebt gij ergens eene,.}{pijnlijke wonde zij kan}{slechts aan het hart zijn}\\

\haiku{Wij zijn insgelijks.}{jong en weten ook al iets}{van zulke dingen}\\

\haiku{{\textquoteleft}Lisa heeft hare.....{\textquoteright} {\textquoteleft},,!}{zinnen op TheodoorNeen}{neen verdenk haar niet}\\

\haiku{Geen verschil was er.}{tusschen de stof der schaal en}{die der letteren}\\

\haiku{{\textquoteright} riep moeder Noppe,.}{die eenen lichtstraal in haren}{geest voelde dringen}\\

\haiku{Toen de fourier hem,:}{genaderd was fluisterde}{hij hem in het oor}\\

\haiku{{\textquoteleft}Ik doe u gaarne,;}{uitgeleide gij twijfelt}{daar zeker niet aan}\\

\haiku{Let maar op, dat uw.}{ziekelijk medelijden}{u zelf niet zot maakt}\\

\haiku{Gij hebt tot nu toe,;}{maar vier onzer zotten}{gezien Mijnheeren}\\

\haiku{Carabos te zien,}{zou ik hun aanraden tot}{morgen te wachten}\\

\haiku{Leelijk - zooals wij het -;}{woord verstaan zijn ze boven}{alle beschrijving}\\

\haiku{op het gelaat van;}{den sergeant-majoor}{zweefde een glimlach}\\

\haiku{niets hebbende dan!}{den trouwen dienst van onzen}{goeden dwerg Topaas}\\

\haiku{{\textquoteright} zuchtte de fourier, {\textquoteleft},.}{opstaandemij dunkt ik zou}{er ziek van worden}\\

\haiku{want het verblijf te,.}{Gheel was voor hem niet goed dit}{gevoelde hij wel}\\

\haiku{Dien namiddag dreef.}{er een hevig onweder}{over de gemeente}\\

\haiku{De overdrevenheid.}{zijner ontroeringen had}{hem gansch genezen}\\

\haiku{Nu deed het gezicht;}{der zinneloozen bijna geenen}{indruk meer op hem}\\

\haiku{Welaan, gij zijt een;}{goede kameraad en een}{bescheiden  vriend}\\

\haiku{geen lichtstraal kan er,!}{nog binnen geene hoop meer voor}{mij dan in den dood}\\

\haiku{Lucia zit alleen,.}{beneden de prinses is}{op hare kamer}\\

\haiku{Vallen op het veld,,!}{van eer en zoo den worm dooden}{die mijn hart verscheurt}\\

\haiku{{\textquoteright} {\textquoteleft}Iedereen weet, dat.}{gij een gevoelig hart hebt}{en menschlievend zijt}\\

\haiku{{\textquoteleft}Ik heet Willem Hoofs,.}{en woon te Elsene}{bij het Keienveld}\\

\haiku{{\textquoteleft}Toen hij twee of drie,.}{uren later mij terugvond}{scheen hij zeer blijde}\\

\haiku{en toch, ik wilde!}{haar voorbij en deed eenen stap}{meer naar den afgrond}\\

\haiku{{\textquoteright} {\textquoteleft}O, Mijnheer,{\textquoteright} kreet de.}{ontstelde jongeling met}{tranen in de oogen}\\

\haiku{hij kon zijne vrouw.}{en zich zelven niet zoo van}{alles ontblooten}\\

\haiku{{\textquoteright} De jongeling sprong.}{op en scheen van verrassing}{en hoop te beven}\\

\haiku{{\textquoteright} kreet zij, tot in het.}{midden der kamer hem te}{gemoet komende}\\

\haiku{Maar zijn vriend, hem de,:}{hand grijpende zeide met}{geestdrift in de stem}\\

\haiku{{\textquoteright} De veekoopman liet;}{weder een oogenblik in}{stilte voorbijgaan}\\

\haiku{{\textquoteright} De dokter keerde.}{zich om en vervorderde}{langzaam zijnen weg}\\

\haiku{u gebruikt hebben?}{om de verlossing van mijn}{kind te bewerken}\\

\haiku{Mijne moeder zal.}{nu zoo gansch alleen dag en}{nacht aan mij denken}\\

\haiku{Hij heeft eene oude -.}{moeder eene deugdzame en}{edelhartige vrouw}\\

\haiku{{\textquoteleft}O, laat ze met ons!}{op het schoone landgoed te}{Boendale wonen}\\

\haiku{Hoe schilderachtig,:}{dit huis ook zij het biedt toch}{niets bijzonders aan}\\

\haiku{Geloof mij, moeder,!}{of ik bezweer het met eenen}{schrikkelijken eed}\\

\haiku{Deze vroolijke;}{muziek scheen den kolonel}{zeer te ontroeren}\\

\haiku{gras, heide, water,,.}{boomen alles groet mij in}{eene roerende taal}\\

\haiku{{\textquoteright} riep zij, {\textquoteleft}o, wees niet,?}{bedroefd ik zal immers nog}{wel wederkomen}\\

\haiku{{\textquoteright} De jonge boer sloeg.}{den blik ten gronde en bleef}{eenen tijd roerloos staan}\\

\haiku{{\textquoteright} {\textquoteleft}Vergeef mij, vader,{\textquoteright}.}{sprak de jongeling met waar}{berouw in de oogen}\\

\haiku{{\textquoteleft}Ik weet, dat gij niets;}{verlangt dan wat mij goed en}{voordeelig zou zijn}\\

\haiku{want sedert lang zijn.}{mijne droomen niets meer dan}{gal en bitterheid}\\

\haiku{Althans, zij had het.}{nooit aan zich zelve of aan}{anderen bekend}\\

\haiku{{\textquoteright} Trien trok een breiwerk:}{uit haren zak en zeide}{met even stille stem}\\

\haiku{de rijke menschen, -:}{geven hun geld en ik ik}{geef ook wat ik heb}\\

\haiku{de eene zegt dit, de,.}{andere zegt dat en op}{den duur weet men niets}\\

\haiku{{\textquoteright} {\textquoteleft}Maar, Meken, hoe hebt?}{gij hem kunnen verzorgen}{en onderhouden}\\

\haiku{of tenzij dat gij?}{ergens eene kous onder de}{pannen hebt steken9}\\

\haiku{Want een bruidegom.}{eischt meer tot zijn geluk}{dan koude vriendschap}\\

\haiku{het hart verdroogt, als.}{men het niet in een ander}{hart uitstorten kan}\\

\haiku{{\textquoteright} Zichtbaar beefde de,.}{maagd terwijl zij sprakeloos}{het hoofd voorover boog}\\

\haiku{Monica lag met;}{het hoofd op de tafel en}{moest bitter weenen}\\

\subsection{Uit: Volledige werken 17. De arme edelman. Eene 0 te veel}

\haiku{maar ik zou, door eene,?}{verkooping al mijne hoop}{gaan verbrijzelen}\\

\haiku{Nu toch verlicht een.....}{laatste straal der hoop onze}{duistere toekomst}\\

\haiku{Onze eenzaamheid:}{zal onze armoede niet}{langer verbergen}\\

\haiku{de edelman sprong recht,;}{zoo haast de bediende de}{zaal had verlaten}\\

\haiku{Daar staande, scheen hij;}{nog aan een ijselijken}{strijd overgeleverd}\\

\haiku{De edelman ging de;}{deur voorbij en wandelde}{de straat ten einde}\\

\haiku{En, inderdaad, het,.}{was een geheim zelfs voor den}{pachter der hoeve}\\

\haiku{Mijnheer De Necker.}{en zijn neef zullen hier het}{middagmaal nemen}\\

\haiku{want niettemin blonk.}{een liefdevolle glimlach}{op zijn aangezicht}\\

\haiku{{\textquoteright} De edelman zonk eene.}{wijl in de bespiegeling}{van zijns broeders lot}\\

\haiku{aldus twee flesschen.}{voor mijnheer De Necker en}{\'e\'ene voor zijn neef}\\

\haiku{Morgen zal het oog;}{der menschen  mistrouwend}{op u zich richten}\\

\haiku{- De koopman voelde}{zich door een waar gevoel van}{vriendschap tot mijnheer}\\

\haiku{hem kwam het vreemd voor,;}{dat men zich om zijn vertrek}{te verblijden scheen}\\

\haiku{doch zijne woorden.}{schenen het gewenschte doel}{niet te bereiken}\\

\haiku{Uwe oneindige;}{goedheid alleen geeft mij den}{noodigen moed daartoe}\\

\haiku{Wij zijn armer dan,.}{de pachter die de hoeve}{bij de poort bewoond}\\

\haiku{De eenzaamheid van.}{den Grinselhof bezielen}{door onze liefde}\\

\haiku{Eene vrouw moet haren.}{echtgenoot onderdanig}{volgen waar hij gaat}\\

\haiku{{\textquoteright} zuchtte de edelman,.}{zich de vuisten nevens het}{lichaam wringende}\\

\haiku{En, vermits gij mij,}{dwingt tot spreken vooraleer}{ik uw voornemen}\\

\haiku{dan, daartoe hebt gij,;}{eene  slechte keus gedaan}{heer Van Vlierbeke}\\

\haiku{Lenora bemerkte,.}{even  spoedig dat diepe}{smart hem ontstelde}\\

\haiku{Lenora had twee of;}{drie stappen gedaan om zich}{te verwijderen}\\

\haiku{Nu kome wat wil,.}{ik zal moedig zijn tegen}{verdriet en treurnis}\\

\haiku{den wreeden worm uit,!}{zijn hart rukken hem redden}{door mijne liefde}\\

\haiku{{\textquoteleft}Lenora, Lenora, mijn,,?}{kind zijt gij een bovenaardsch}{wezen een Engel}\\

\haiku{Zij herhaalde in;}{stilte zijne teederste}{bekentenissen}\\

\haiku{Te tien uren was de,,;}{zaal waar men beginnen zou}{met menschen vervuld}\\

\haiku{Een oogenblik slechts;}{duurde die hoonende houding}{der aanwezigen}\\

\haiku{Zij dragen beiden.}{een pakje in de hand en}{scheinen reisvaardig}\\

\haiku{Met trage stappen.}{gingen vader en dochter}{tot bij de hoeve}\\

\haiku{ik zou geheel mijn.}{leven het mij verwijten}{en er om treuren}\\

\haiku{En wiste ik het,{\textquoteright}.}{de voorzichtigheid zou het}{mij doen verzwijgen}\\

\haiku{{\textquoteright} Eene wijle stond hij,.}{beweegloos met de hand aan}{het voorhoofd gedrukt}\\

\haiku{Eene jonge dienstmeid.}{staat op den dorpel en lacht}{en praat met de knechts}\\

\haiku{de pachteresse.}{stond met het hoofd gebogen}{en was diep ontroerd}\\

\haiku{Misschien berust nog;}{in uw hart een rijke schat}{van moed en van hoop}\\

\haiku{Met het hoofd over haar,;}{werk gebogen schijnt zij ten}{gronde te blikken}\\

\haiku{Ah, hoe schatert gij,!}{van blijdschap hoe machtig slaat}{gij uwe vlerken uit}\\

\haiku{Gij zelve schijnt door;}{deze ongelukkige}{tijding getroffen}\\

\haiku{Welnu, welnu,{\textquoteright} vroeg, {\textquoteleft},?}{hijwat is het dan dat u}{zoo gelukkig maakt}\\

\haiku{hij zeide, terwijl:}{tranen van ontroering uit}{zijne oogen sprongen}\\

\haiku{maar hij stond tegen:}{zijne ontsteltenis op}{en zeide troostend}\\

\haiku{{\textquoteright} {\textquoteleft}Onder ons gezegd,,.}{ik heb nooit gedacht dat ze}{nog kon genezen}\\

\haiku{Somwijlen bekruipt;}{mij eene bekoring om Isidoor}{den hals te breken}\\

\haiku{{\textquoteright} {\textquoteleft}Neen, de knecht, dien wij,.}{zullen hebben zal de melk}{naar de stad voeren}\\

\haiku{Dan keerde hij zich,,!}{om en   Vaarwel goede}{edele vriendinne}\\

\haiku{en kom  mij dan,.}{zeggen of uwe moeder wel}{blijde is geweest}\\

\haiku{door zulke gekke.}{droomen brengt men de meisjes}{op eenen slechten weg}\\

\haiku{En gij spreekt juist, vrouw,.}{alsof Simon Storms niet meer}{op de wereld was}\\

\haiku{Ach,  kon zulk lot,:}{ons ook eens te beurt vallen}{zoudt gij niet zeggen}\\

\haiku{{\textquoteright} {\textquoteleft}Wel, goede man,{\textquoteright} kreet, {\textquoteleft}.}{de apothekergij zijt nog}{van den ouden tijd}\\

\haiku{Het antwoord liet zich,.}{wachten als was de koeboer}{in twijfel geraakt}\\

\haiku{Geef, lieve Heer, ons,!}{kost en kle\^er Het hemelrijk}{en dan niet meer}\\

\haiku{Hij was reeds tot bij,:}{de deur toen vrouw Storms hem met}{aandringen toeriep}\\

\haiku{{\textquoteright} {\textquoteleft}Gauw dan, moeder, ik,{\textquoteright},.}{heb geenen tijd morde hij tot}{haar terugkeerende}\\

\haiku{{\textquoteleft}Krijsch niet, moeder lief,.}{ik zal medegaan en u}{nimmer verlaten}\\

\haiku{{\textquoteright} {\textquoteleft}Simon, wij hebben:}{eene verkeerde gedachte}{over de Godshuizen}\\

\haiku{{\textquoteright} {\textquoteleft}Neen, ik ga niet weg,{\textquoteright}, {\textquoteleft}.}{morde hij beradenik}{moet u iets vragen}\\

\haiku{Hij zette zich op,:}{eenen stoel nam de hand zijner}{moeder en zeide}\\

\haiku{{\textquoteright} Vrouw Storms haalde de.}{sleutels uit haren zak en}{reikte ze hem toe}\\

\haiku{want zij vreesde, dat;}{deze aangejaagdheid}{hem ziek zou maken}\\

\haiku{Zoo ontsnapte hij,,.}{acht dagen later aan eenen}{gevaarlijken slag}\\

\haiku{Simon liet de pen (.}{uit zijne hand vallen en}{opende eene deurbladz}\\

\haiku{Ach, wat kon eene 0!}{te veel toch schrikkelijke}{gevolgen hebben}\\

\haiku{Simon onderging;}{eveneens den invloed van het}{moedbarend metaal}\\

\haiku{{\textquoteright} Baas Verhoeven en.}{zijne vrouw aanschouwden hem}{immer even verbaasd}\\

\haiku{Simon,{\textquoteright} vroeg zij, {\textquoteleft}is,,?}{de koets die daarbuiten voor}{de deur staat van u}\\

\haiku{{\textquoteleft}Ja, een dwaashoofd en,{\textquoteright}.}{een slecht mensch ben ik zeide}{de koeboer zuchtend}\\

\subsection{Uit: Volledige werken 18. De kwaal des tijds}

\haiku{Die oude izegrim;}{zit ons op den nek van den}{morgen tot den avond}\\

\haiku{Gij meent, dat ik voor?}{zoo weinig van schrik door den}{mestput zou loopen}\\

\haiku{Dan bracht hij zijnen,:}{stoel nader vatte de hand}{des grijsaards en sprak}\\

\haiku{over aan het lot, dat.}{hij zich zelven wetens en}{willens voorbereidt}\\

\haiku{betuig, bid ik u,;}{mevrouw Van Everdael mijne}{erkentelijkheid}\\

\haiku{Maar neen, het hart van!}{Dani\"el is een schat van}{goedheid en liefde}\\

\haiku{maar dat hij zijnen!}{ouden voedstervader niet}{meer zou beminnen}\\

\haiku{Gij hebt beiden eene!}{zonderlinge wijze van}{gelukkig te zijn}\\

\haiku{tranen der liefde.}{en der vriendschap stroomden in}{stilte rondom hem}\\

\haiku{met uw oorlof, de.}{schoone kleederen doen eene}{boerin ook geen kwaad}\\

\haiku{{\textquoteright} Dit zeggende, sprong.}{zij met de handen omhoog}{in de baan vooruit}\\

\haiku{Deze persoon droeg.}{eenen langen blauwen jas met}{vergulde knoopen}\\

\haiku{Moet gij mij daarom,?}{bekijken alsof gij mij}{wildet verslinden}\\

\haiku{Hij trekt een gezicht,.}{alsof de wereld tegen}{zijnen dank draaide}\\

\haiku{{\textquoteleft}Is het z\'o\'o, dat men?}{in Parijs zijne meesters}{leert eerbiedigen}\\

\haiku{Eindelijk daalde.}{het vertrouwen weder in}{des grijsaards boezem}\\

\haiku{{\textquoteleft}Ah sa, Dani\"el,!}{gij begint mij schrikkelijk}{te  vervelen}\\

\haiku{het eene staat onder;}{bevel van mijne rede}{en van mijnen wil}\\

\haiku{Deze twee zielen;}{strijden in mijn binnenste}{om de overwinning}\\

\haiku{{\textquoteleft}Zie de wereld zooals,.}{zij is en vraag haar niet wat}{zij niet geven kan}\\

\haiku{Ik heb haar bemind,.}{toen mijn hart even eenvoudig}{was als het hare}\\

\haiku{de stilte, de rust.}{alleen kan zijn gemoed tot}{bedaren brengen}\\

\haiku{{\textquoteleft}Ach, lieve tante,{\textquoteright}, {\textquoteleft}!}{zeide zij met verdoofde}{stemhij is zoo ziek}\\

\haiku{{\textquoteleft}Maar er is niets hoonends,.}{voor u in deze meening}{goede Willibald}\\

\haiku{IV De nacht moest de;}{arme Dani\"el niet veel}{rust gegund hebben}\\

\haiku{Wanneer gij eens ligt,:}{neergeknakt kan niets weder}{uwen stengel rechten}\\

\haiku{Eilaas, verloren,,!}{verloren voor altijd de}{kracht tot beminnen}\\

\haiku{Hoe komt het, dat gij?}{zijne plaats voor dit getouw}{ingenomen hebt}\\

\haiku{Ik heb ze beiden,;}{verzorgd nacht en dag alleen}{bij hun bed gestaan}\\

\haiku{{\textquoteleft}En gij, Rosalie,,?}{gij vervult uwe heilige}{belofte niet waar}\\

\haiku{{\textquoteright} {\textquoteleft}O, gij goede vrouw,{\textquoteright}, {\textquoteleft}!}{zuchtte de jonkheerwat moet}{gij gelukkig zijn}\\

\haiku{Oh, zij is slechts eene,.....}{boerinne een nederig}{wezen op aarde}\\

\haiku{Het is een handel,.}{waarin ik vroeger niet}{onervaren was}\\

\haiku{Nevens hem op eene;}{tafel lagen groote boeken}{opeengestapeld}\\

\haiku{Ik zal ze op uwe,.}{vraag gaan halen indien gij}{het mocht verlangen}\\

\haiku{dit alles is goed,;}{en wel en de boeken zijn}{met zorg geschreven}\\

\haiku{ik mijnen eigen.}{zoon gereed om zich in het}{verderf te storten}\\

\haiku{Het is in Frankrijk,,;}{te Parijs alleen dat ik}{kan en wil leven}\\

\haiku{{\textquoteleft}Verwacht hem heden,,{\textquoteright}.}{niet meer Celesta zeide}{de oude dame}\\

\haiku{En nochtans, aan die.}{noodlottige wreedheid kon}{hij niet ontsnappen}\\

\haiku{Tot nu toe heb ik.}{mij over die voorzienigheid}{niet te beklagen}\\

\haiku{Dom genoeg om ten!}{minste in de waarheid uwer}{vriendschap te gelooven}\\

\haiku{Welnu, zeg, dat gij,{\textquoteright} {\textquoteleft}}{niet van verveling op den}{Wulfhof wilt sterven}\\

\haiku{De grijsaard stapte;}{ter zaal in en ging langzaam}{rondom de tafel}\\

\haiku{In de duisternis;}{trapte hij op brokken en}{scherven van flesschen}\\

\haiku{maar, God zij er om,,!}{gezegend Dani\"el gij}{zijt zooverre nog niet}\\

\haiku{{\textquoteright} morde Gumbert, {\textquoteleft}dit.}{moet een slimme vogel of}{een dommerik zijn}\\

\haiku{{\textquoteright} {\textquoteleft}Bah, bah, waarom iets,?}{verbergen dat gansch gewoon}{en natuurlijk is}\\

\haiku{{\textquoteleft}Kom in de keuken,{\textquoteright},{\textquoteright},.}{zeide zei en leg mij eens}{uit wat dit beteekent}\\

\haiku{M. Willibald zal;}{dezen morgen zeker ons}{komen bezoeken}\\

\haiku{Iedereen zal over;}{dit onuitlegbaar vertrek}{zich verwonderen}\\

\haiku{gij begrijpt immers,?}{dat het afscheid hem te zeer}{zou ontsteld hebben}\\

\haiku{en, wanneer ik mijn,:}{hart te rade ga dan durf}{ik er bijvoegen}\\

\haiku{De notaris trok:}{den rentmeester een weinig}{ter zijde en sprak}\\

\haiku{{\textquoteleft}Ik zie ginder eene,.}{koets die in volle vaart naar}{hier komt gereden}\\

\haiku{Ga daarna in de.}{keuken en eet metterhaast}{insgelijks een stuk}\\

\haiku{in den stal hoorde;}{ik hem zingen van geluk}{en tevredenheid}\\

\haiku{In die gedachte.}{vertrok ik tegen den avond}{weder naar Brussel}\\

\haiku{{\textquoteright} {\textquoteleft}Wacht eens wat,{\textquoteright} zeide,.}{Katrien hare gezellin}{wederhoudende}\\

\haiku{{\textquoteright} Judocus verschoot,.}{en nog rooder werden hem}{wangen en voorhoofd}\\

\haiku{{\textquoteright} {\textquoteleft}En dat de Wulfhof?}{onzen jonkheer Dani\"el}{niet meer toebehoort}\\

\haiku{Och, Judocus, hij,!}{zal vernemen dat gij te}{veel gesproken hebt}\\

\haiku{Zij heeft wel gelijk,.}{te denken dat ik hare}{hulp zal weigeren}\\

\haiku{{\textquoteright} Hij trok zijn uurwerk,,,:}{uit en met het oog er op}{gevestigd sprak hij}\\

\haiku{Gumbert, mijn trouwe,,?}{gezel mijn boezemvriend mijn}{verkleefde broeder}\\

\haiku{271.)om, ging wankelend.}{tot eenen stoel en liet zich er}{op nedervallen}\\

\haiku{Alles, wat gij in,.}{uwe milde jeugd hebt gedroomd}{gaat waarheid worden}\\

\haiku{de verschrikte maagd,.}{was bleek en tranen rolden}{van hare wangen}\\

\subsection{Uit: Volledige werken 19. Geld en adel}

\haiku{Er liggen wel drie.}{kruiwagenvrachten er van}{op zijnen zolder}\\

\haiku{Sedert eenigen tijd,;}{hebben wij het nog al druk}{gehad inderdaad}\\

\haiku{Laat ons al deze.}{bedroevende gedachten}{ter zijde stellen}\\

\haiku{{\textquoteleft}Het zijn de jonge,{\textquoteright}.}{heeren uit den Gulden Arend}{zeide Jan Wouters}\\

\haiku{{\textquoteright} {\textquoteleft}De arme jongen,{\textquoteright}.}{kent geene beenen meer voegde de}{weduwe er bij}\\

\haiku{Sta mij maar zoo dom,!}{niet aan te kijken en reik}{mij mijne schoenen}\\

\haiku{{\textquoteleft}Is het zoo, ik dank,,{\textquoteright}.}{u uiterharte goede}{man mompelde hij}\\

\haiku{Gaat en vraagt het in.}{den Gulden Arend aan den baas}{en zijne dochters}\\

\haiku{{\textquoteright} {\textquoteleft}Ziet gij wel, moeder,{\textquoteright}, {\textquoteleft},?}{riep Linadat hij het nog}{niet heeft vergeten}\\

\haiku{- Begeef u op het.}{kantoor en deel den overste}{deze zaak mede}\\

\haiku{van zaken, goede -,.}{heer van Overburg mijn vriend zou}{ik durven zeggen}\\

\haiku{Kom morgen terug,.}{dan zal ik u mijn besluit}{te kennen geven}\\

\haiku{De edelman zag hem,.}{verbaasd aan als hadde hij}{hem niet begrepen}\\

\haiku{Indien gij in dit,{\textquoteright}, {\textquoteleft}}{huwelijk toestemt hernam}{Steenvlietleen ik u}\\

\haiku{Mijne dochter zal.}{waarschijnlijk uw voorstel met}{blijdschap vernemen}\\

\haiku{Mijnheer,{\textquoteright} kondigde, {\textquoteleft}.}{hij aanuw zoon is daareven}{te huis gekomen}\\

\haiku{{\textquoteleft}Het is de eerste,,;}{maal niet dat ik ze u zou}{aanwijzen vader}\\

\haiku{Ik moet bekennen,:}{dat ik werkelijk verre}{beneden hen sta}\\

\haiku{maar ik durf er niet,.}{voor instaan dat ik mijn woord}{zal kunnen houden}\\

\haiku{Is het Clemence,?}{van Overburg die men mij tot}{bruid wil voorstellen}\\

\haiku{Opspringende, vroeg:}{zij met eene uitdrukking van}{angst op het gelaat}\\

\haiku{In Brussel ging ik.}{onzen rijken vriend van den}{Kruisboom bezoeken}\\

\haiku{Die burgers mogen.}{nijverheid uitoefenen}{en handel drijven}\\

\haiku{Integendeel, gij,.}{zult mij helpen oprecht en}{zonder aarzeling}\\

\haiku{Deze samenspraak,.}{nam eene zeer ongunstige}{wending meende hij}\\

\haiku{Moeder, moeder, gij?}{hebt u ten mijnen koste}{willen vermaken}\\

\haiku{{\textquoteleft}Ik ben aan de Bank.}{tweehonderdvijftig duizend}{franken verschuldigd}\\

\haiku{{\textquoteright} {\textquoteleft}Het denkbeeld van zijn,.}{eerste bezoek verschrikt mij}{inderdaad vader}\\

\haiku{ik zal mij beleefd,.....}{en minzaam voor hem toonen}{zooveel mogelijk}\\

\haiku{{\textquoteright} {\textquoteleft}Nu, goede lieden,{\textquoteright},, {\textquoteleft}}{zeide de jongeling zich}{naar de deur keerende}\\

\haiku{{\textquoteright} En Herman Steenvliet.}{stapte door den voorhof en}{in den aardeweg}\\

\haiku{Jan Wouters wilde.}{Herman den boomgaard en den}{groenselhof toonen}\\

\haiku{Maar daar voelde hij,.}{eensklaps dat iemand hem op}{den schouder klopte}\\

\haiku{{\textquoteright} {\textquoteleft}O, mijnheer, heb toch!}{medelijden met mij en}{mijne kinderen}\\

\haiku{Niemand sprak nog een.}{woord en allen schenen min}{of meer verlegen}\\

\haiku{{\textquoteleft}De rijtuigen van,.}{M. den graaf van M. den}{markies en van Mev}\\

\haiku{{\textquoteleft}Het rijtuig van M.!}{den baron van Moorsbeke}{is ingespannen}\\

\haiku{{\textquoteright} De jongeling, die}{wel gevoelde dat het nu}{geen oogenblik was}\\

\haiku{gezondheid, die ons,{\textquoteright}.}{angstig maakte voegde de}{weduwe er bij}\\

\haiku{Nu begin ik eerst.}{goed te begrijpen waarvan}{men ons beschuldigt}\\

\haiku{Herman Steenliet zal.}{na dezen dag geenen voet meer}{in ons huis zetten}\\

\haiku{Wees zeker, kind, van,,;}{al wat men in het dorp zegt}{geloof ik niets niets}\\

\haiku{{\textquoteleft}Lina, belooft gij,,?}{de waarheid mij te zeggen}{geheel de waarheid}\\

\haiku{In \'e\'en woord, zij rooven.}{onze eer en bevlekken}{onzen goeden naam}\\

\haiku{Nauwelijks durf ik,.}{u openbaren wat men van}{haar zegt en gelooft}\\

\haiku{maar ik ben te vast.}{overtuigd dat uwe nieuwe gril}{geenen stand zal houden}\\

\haiku{{\textquoteright} En deze woorden,.}{murmelende ging Herman}{uit het kabinet}\\

\haiku{{\textquoteright} {\textquoteleft}En blijkt er uit de,?}{woorden zijns briefs dat hij even}{gunstig gestemd blijft}\\

\haiku{Zijne vroegere.}{kameraden ontmoeten}{hem  nergens meer}\\

\haiku{Ik geloof zulks niet,;}{en zou in alle geval}{het niet goedkeuren}\\

\haiku{Wij zullen dan in.....}{zijne tegenwoordigheid}{alles regelen}\\

\haiku{{\textquoteleft}Eilaas, ja, mijnheer,{\textquoteright}, {\textquoteleft}.}{was het antwoordik ben er}{nog diep van ontsteld}\\

\haiku{Zoohaast ik eene vaste,.}{verblijfplaats heb gevonden}{zal ik u schrijven}\\

\haiku{Het antwoord, dat uit,;}{hare gedachte oprees}{moest niet gunstig zijn}\\

\haiku{{\textquoteleft}Ik mag hem niet meer,.....}{wederzien en ik verlang}{niet hem nog te zien}\\

\haiku{Bedien  haar maar,,.}{allereerst bazin opdat}{ze spoedig wegga}\\

\haiku{men heeft mij dezen;}{morgen nog met slijken steenen}{uit het dorp gejaagd}\\

\haiku{{\textquoteleft}Kom eens hier bij mij,,{\textquoteright}, {\textquoteleft}.}{Clemence zeide hijik}{moet u iets vragen}\\

\haiku{hij maakte zich los,:}{uit hare armen terwijl}{hij somber gromde}\\

\haiku{Ach, ik ga afstand,;}{doen van mijne geboorte}{van mijnen adelstand}\\

\haiku{Houd op, Clemence,,.}{mijn toorn is wettig ik ben}{onverbiddelijk}\\

\haiku{Zij liep de kamer,.}{uit en zag den markies in}{den gang verschijnen}\\

\haiku{{\textquoteright} riep deze, zich nog, {\textquoteleft},!}{naar de zaal omkeerendeneen}{ik ken u niet meer}\\

\haiku{{\textquoteright} {\textquoteleft}Het schijnt echter, mijn,.}{jongen dat deze verre}{reis u niet toelacht}\\

\haiku{Niet waar, markies, het,?}{is om die reden alleen}{dat gij ons misprijst}\\

\haiku{Weldra werd hare.}{aandacht afgekeerd door eenig}{gerucht in den stal}\\

\haiku{Uwe eenvoudige,, -?}{goedheid de zuiverheid uws}{harten wat weet ik}\\

\subsection{Uit: Volledige werken 20. Het ijzeren graf}

\haiku{Het nederige.}{kerkje verheft zich boven}{den akker des doods}\\

\haiku{bloemen wiegelen;}{in den zoelen zuiderwind}{boven de graven}\\

\haiku{het is er alle;}{dagen Zondag en men speelt}{en zingt er altijd}\\

\haiku{{\textquoteright} {\textquoteleft}Och, onnoozel Mieken,,.}{het is de kluizenaar die}{ze daar altijd plant}\\

\haiku{En vertel maar niet!}{al te veel fabelen van}{het ijzeren graf}\\

\haiku{Eene stem, die mijnen,.}{naam noemde achter mij deed}{mij het hoofd omkeeren}\\

\haiku{het stomme kind was,.....}{niemand anders dan ik zelf}{die nu tot u spreek}\\

\haiku{hare oogjes blauw;}{en diep als de hemel op}{eenen helderen dag}\\

\haiku{al mijne leden,.}{wrongen zich krampachtig mijn}{aangezicht werd blauw}\\

\haiku{En als gij spreken,.}{kunt dan zal ik u nog veel}{schooner dingen geven}\\

\haiku{{\textquoteright} Het goede kind had;}{voorwaar geen ander inzicht}{dan mij te troosten}\\

\haiku{Gij begrijpt, dat wij;}{dit jaar op het kasteel niet}{zullen verblijven}\\

\haiku{Ik legde mij de.}{handen voor de oogen om haar}{niet te zien heengaan}\\

\haiku{Arme Leo,{\textquoteright} zeide, {\textquoteleft}.}{het goedhartig kindgij moogt}{daarom niet krijschen}\\

\haiku{Ik liep in \'e\'enen;}{adem door het dorp en in de}{dreve des kasteels}\\

\haiku{alles sterk en fraai,.}{gemaakt van fijn Engelsch staal}{zooals mijn vader zegt}\\

\haiku{Dan daalt in mij de,,;}{overtuiging dat ik verdrink}{dat ik ga sterven}\\

\haiku{Er was vooruitzicht;}{en edelmoedigheid in die}{kinderlijke scherts}\\

\haiku{Na eene lange wijl:}{trad M. Pavelyn terug}{in de school en sprak}\\

\haiku{, en zelfs in mijnen.}{slaap ontvielen mij dikwijls}{bittere tranen}\\

\haiku{en daarenboven,,,.}{hare stem hoewel fijn en}{zuiver was zeer zwak}\\

\haiku{ik zou ze nooit meer;}{zien zooals ze onverpoosd mij}{voor den geest zweefde}\\

\haiku{Ik was opgestaan.}{en had uit eerbied eenen stap}{ter zijde gedaan}\\

\haiku{{\textquoteleft}Kom aan, zeg ons toch,?}{wat is uwe gedachte over}{Leo's eerste werk}\\

\haiku{Een dezer heeren.}{was een man van fijn gevoel}{en diepe kennis}\\

\haiku{Ik was verheugd, dat;}{ik eene reden vond om mij}{neder te zetten}\\

\haiku{Ik zeide, dat ik;}{onpasselijk was en nu}{geenen lust tot eten had}\\

\haiku{indien ik iets noodig,.}{had zou ik kloppen om haar}{te verwittigen}\\

\haiku{de tranen rolden.}{in stilte als parelen}{over hare wangen}\\

\haiku{{\textquoteright} En hare stem nog,:}{meer bedwingende suisde}{zij schier onhoorbaar}\\

\haiku{{\textquoteright} Mij zwol de borst van,:}{moed mij popelde het hart}{van levensblijheid}\\

\haiku{Hij schudde mij nog.}{de hand met kracht en daalde}{dan de trappen af}\\

\haiku{de tranen zijn eene,.}{klacht een gebed om hulp of}{om medelijden}\\

\haiku{Hij hief mij met eene:}{korte beweging van den}{grond op en zeide}\\

\haiku{Tegen den avond had.}{ik het engelenhoofd schier}{geheel afgewerkt}\\

\haiku{Mijn werk was dus te.}{tenger van vormen en te}{mager van lijnen}\\

\haiku{maar hij liet mij den.}{tijd niet om uit te drukken}{wat ik gevoelde}\\

\haiku{{\textquoteright} {\textquoteleft}En indien ik u,?}{zeide dat ik de maker}{van dit beeld niet ben}\\

\haiku{maar ik wil mij de.}{verdiensten van anderen}{niet aanmatigen}\\

\haiku{ik deed geweld om.}{mijne smart te bedwingen}{en hief het hoofd op}\\

\haiku{{\textquoteright} {\textquoteleft}Neen, neen, wees gerust,,{\textquoteright}.}{Leo viel zij met eenen glimlach}{mij in de rede}\\

\haiku{Er moesten door Rosa's,;}{geest gepeinzen vlotten die}{zij niet uitdrukte}\\

\haiku{Er is iets, dat u,?}{bedroeft en gij weigert mij}{mijn deel van uwe smart}\\

\subsection{Uit: Volledige werken 21. De baanwachter. Gerechtigheid van Hertog Karel}

\haiku{Ongetwijfeld was;}{de baanwachter een getrouwd}{man met kinderen}\\

\haiku{Bij de blinde vrouw,:}{teruggekeerd antwoordde}{hij op hare vraag}\\

\haiku{het plantsoen onzer;}{koolen en onzer salade}{komt uit zijnen hof}\\

\haiku{Nu bemerkte de.}{jongen zijne grootmoeder}{onder het pri\"eel}\\

\haiku{Er liggen er nog,.}{wel acht of tien op mijne}{telloor ik voel het}\\

\haiku{Mie-Wanna hield zich.}{bezig met de kousen van}{Sander te stoppen}\\

\haiku{De  hagelsteenen.....}{sloegen zijne handen en}{wangen ten bloede}\\

\haiku{Bedwing uwe tranen,.}{en stapt recht vooruit nevens}{het spoor Mie-Wanna}\\

\haiku{, blijf toch rustig en,{\textquoteright}.}{doe de kinderen zoo niet}{schrikken zeide hij}\\

\haiku{Jan Verhelst zat op.}{zijne bank met de handen}{v\'o\'or het aangezicht}\\

\haiku{want nu en dan keek;}{hij op en luisterde met}{overspannen aandacht}\\

\haiku{Iedereen acht mij,.}{schuldig iedereen haat en}{vermaledijdt mij}\\

\haiku{Geef mij tijding van,!}{hen en ik zegen uwen naam}{tot op mijn doodbed}\\

\haiku{{\textquoteright} {\textquoteleft}Ach, Meken lief, ik,!}{zou het zoo gaarne gelooven}{het is toch zoo schoon}\\

\haiku{zelfs niet toelaten,!}{dat wij vernemen of hij}{gezond of ziek is}\\

\haiku{Zijn de menschen wreed,}{en onrechtvaardig jegens}{uwen armen vader}\\

\haiku{{\textquoteright} {\textquoteleft}Ja, mijnheer, ik ben.}{in de stad geboren en}{heb er lang gewoond}\\

\haiku{{\textquoteleft}Vader, hier is de,,.}{heer Masmans onze vriend die}{u komt bezoeken}\\

\haiku{Wat gaat gij doen met?}{uwe arme kinderen en}{uwe blinde moeder}\\

\haiku{Hij vatte weder:}{hare hand en zeide met}{aangrijpende ernst}\\

\haiku{maar Barbeltje moet.}{gij naar mijne zuster te}{Vilvoorden dragen}\\

\haiku{Ik heb slecht jegens,;}{u gehandeld omdat ik}{u schuldig waande}\\

\haiku{Integendeel, ik.}{had medelijden met uw}{ijselijk verdriet}\\

\haiku{Gij hebt er telkens.}{vermaak in gehad en er}{mede gelachen}\\

\haiku{Mijne moeder, mijn,;}{broeder en ik wij toonen}{ons onderdanig}\\

\haiku{De boter is schier,;}{in prijs verdubbeld en niet}{alleen de boter}\\

\haiku{Zou er eindelijk?}{recht te verhopen zijn voor}{den minderen man}\\

\haiku{hem scheen, dat hij den.....}{grond onder zijne voeten}{voelde daveren}\\

\haiku{maar eensklaps heft hij:}{de handen in de hoogte}{en roept kermend uit}\\

\haiku{Zeg mij eens, gij die,}{jong zijt en geenen last hebt te}{dragen waarom rijdt}\\

\haiku{Gij hebt een hof, eenen,,.}{burg in deze omstreken}{heeft men mij gezegd}\\

\haiku{{\textquoteleft}Ik woon te Brugge,,.}{in het hof van Uutkerke}{bij mijne moeder}\\

\haiku{Dan zal ik eens te;}{Hersberge aan de poort van}{uwen burg gaan kloppen}\\

\haiku{en mijn verlangen,,.....}{is hoort gij wel dat de prijs}{gewonnen worde}\\

\haiku{Geeft acht, heeren, ik!}{zal zien wie onder u de}{beste ruiters zijn}\\

\haiku{{\textquoteleft}Daar, Liedekerke,,{\textquoteright}.}{berg dit in uwe gordeltasch}{zeide de Hertog}\\

\haiku{Ik wil intusschen.}{hier wat wandelen en de}{frischheid genieten}\\

\haiku{Gij, Liedekerke,.}{ga uit het bosch en poog de}{jacht in te halen}\\

\haiku{Meent gij dan waarlijk,?}{dat ik hier gekomen ben}{om u kwaad te doen}\\

\haiku{De smart doet u het,,{\textquoteright}.}{kwaad overdrijven goede man}{zeide de Hertog}\\

\haiku{Ik heb mij voor den,,.}{schout aangeboden heer en}{hem mijn ramp geklaagd}\\

\haiku{zijne schildknapen,;}{niet spreken want hij zag er}{zeer verbolgen uit}\\

\haiku{{\textquoteleft}Zoo, gij weet dus wat?}{mij gisteren op de jacht}{is wedervaren}\\

\haiku{Er behoefden niet.}{min dan vier dienaars om het}{haar te ontnemen}\\

\haiku{Men zal toch wel zien,.}{dat zij anders u geheel}{onverschillig is}\\

\haiku{Wel zeker hebt gij.}{eenige dienaars op wie gij}{vast moogt vertrouwen}\\

\haiku{{\textquoteleft}Welnu, Martijn,{\textquoteright} vroeg, {\textquoteleft}?}{de vorstzal meester Antoon}{niet haast gaan komen}\\

\haiku{{\textquoteright} {\textquoteleft}Het is wel, geef mij;}{mijnen tabbaard en zet mij}{mijne kaproen op}\\

\haiku{{\textquoteleft}Is alle gevoel?}{van rechtvaardigheid dan in}{uwen geest gestorven}\\

\haiku{de jonkheer zal mijn.}{tweede vonnis aanvaarden}{en dus niet sterven}\\

\haiku{In alle geval.}{zal hij eindigen met zich}{te onderwerpen}\\

\haiku{{\textquoteright} antwoordde zij, op,.}{den hofmeester wijzende}{die bij de deur stond}\\

\haiku{De jonkheer Van der.}{Hameide heeft dus tweemaal}{het leven verbeurd}\\

\haiku{Gij zoudt u evenwel.}{in uwe verwachting kunnen}{bedrogen vinden}\\

\haiku{De schout van Brugge.}{verscheen welhaast in zijne}{tegenwoordigheid}\\

\haiku{Welk vonnis meent gij,?}{dat de Schepenbank tegen}{hem zal uitspreken}\\

\haiku{de opoffering,,.}{van geld hoe aanzienlijk ook}{schrikte hen niet af}\\

\haiku{{\textquoteleft}O, mijn God, indien!}{de Schepenbank hem tot den}{dood veroordeelde}\\

\haiku{Om Gods wil, heer schout,!}{folter eene arme moeder}{zoo wreedelijk niet}\\

\haiku{Er is een middel,,.....}{een onfeilbaar middel om}{Walter te redden}\\

\haiku{eene andere keus.}{dan de onderwerping is}{u niet gelaten}\\

\haiku{{\textquoteright} {\textquoteleft}Wij koesteren geene,,{\textquoteright}.}{wraakzucht eerwaarde vader}{antwoordde Thomas}\\

\haiku{Zij naderde met:}{hare dochter en zeide}{op smeekenden toon}\\

\haiku{{\textquoteleft}Vader, vader, ach,!}{heb medelijden met die}{arme edelvrouwen}\\

\haiku{{\textquoteleft}Ja, eerwaarde,{\textquoteright} was, {\textquoteleft}.}{het antwoordmaar  niemand}{mag hem naderen}\\

\haiku{{\textquoteright} {\textquoteleft}Het doet mij leed, voor,.}{uw gebed te moeten doof}{blijven eerwaarde}\\

\haiku{Nu scheen het hun toe,.}{dat zij geen hinderpaal meer}{konden ontmoeten}\\

\haiku{{\textquoteleft}Staat op, Mevrouw, en,,{\textquoteright}.}{gij goede lieden zeide}{de vorst zeer minzaam}\\

\haiku{De voorwaarde heeft,.}{onze geduchte vorst u}{zelf gesteld mijn zoon}\\

\subsection{Uit: Volledige werken 22. De loteling. Bavo en Lieveken}

\haiku{In het eene woonde;}{eene arme weduwe met}{hare  dochter}\\

\haiku{Maar op eens kwam men.}{van de leemen hutjes den tol}{des bloeds afeischen}\\

\haiku{{\textquoteright} Een onmerkbare;}{glimlach zweefde tusschen de}{tranen der maagd}\\

\haiku{Op die heilige,}{plaats waar elken dag iemand}{van hen den goeden}\\

\haiku{O, ik zal alle}{dagen een gebed lezen}{voor uwen heiligen}\\

\haiku{zij was bezig met -:}{het papier af te likken}{en riep half verstoord}\\

\haiku{Ik neem de pen in.}{de hand om te vernemen}{naar den staat van UL}\\

\haiku{Daarom, beminde,,.}{ouders als ge kunt zendt mij}{toch een beetje geld}\\

\haiku{maar ik heb hem, woord,.}{voor woord gezegd wat hij er}{in zetten moest}\\

\haiku{doch, daar men Fransch of,,;}{Waalsch sprak verstond Trien niet wat}{men over haar zeide}\\

\haiku{zijne glimmende;}{knevels waren met zwart was}{omhoog gestreken}\\

\haiku{{\textquoteleft}Och, goede vriend, gij,;}{moest bij mij blijven zitten}{als het u belieft}\\

\haiku{Zij voelen aan uwe,.}{kleederen om te weten}{van wat streek gij zijt}\\

\haiku{{\textquoteright}   Het meisje lag (.}{te weenen tegen de borst}{des jongelingsbladz}\\

\haiku{Laat ons liever in,;}{het donker hoeksken op de}{bank gaan zitten Jan}\\

\haiku{Ik heb er wel twee;}{uren voor dood van gelegen}{in eenen eikenkant}\\

\haiku{{\textquoteright} Ach, ik weet het wel,,{\textquoteright}, {\textquoteleft}?}{mevrouw zuchtte Trienwaar heb}{ik het toch verdiend}\\

\haiku{Welnu, beloof mij,;}{dat gij voor mij niets meer zult}{zijn dan eene zuster}\\

\haiku{op haar hoofd stond eene.}{zilveren kroon van zeven}{blinkende starren}\\

\haiku{- Laat ons nu maar wat,;}{beter voortstappen voordat}{de zon omhoog ga}\\

\haiku{Ik geloof, dat wij.}{verloren geloopen zijn}{met dit vertellen}\\

\haiku{Trien kwam bij hem staan.}{en schudde het water van}{hare kleederen}\\

\haiku{het zalig uitzicht,.}{was verdwenen de hoop van}{geluk was vergaan}\\

\haiku{{\textquoteright} De jongeling deed:}{geweld om te ademen en}{antwoordde zuchtend}\\

\haiku{en, de hand  der,:}{maagd vattende antwoordde}{hij met groote koelheid}\\

\haiku{Hij bond den kranke:}{onmiddellijk het lichtscherm}{voor de oogen en vroeg}\\

\haiku{Zeg aan de meid, dat.}{zij insgelijks eten brenge}{voor deze dochter}\\

\haiku{de kleine kruiden;}{sluiten hunne bloemkelken}{en bladeren toe}\\

\haiku{Hij scheen treurig en.}{bleef eene wijl met gebogen}{hoofde overwegen}\\

\haiku{Licht, waardigheid en!}{plichtbesef in het hart der}{moeders van het volk}\\

\haiku{Maar de vrouw, na eene,:}{lange poging mompelde}{met moedeloosheid}\\

\haiku{Zij wist het niet, en.}{evenwel dankte zij God uit}{den grond des harten}\\

\haiku{Op dit oogenblik,.}{werd de deur geopend en}{een man trad binnen}\\

\haiku{ik heb hem beloofd,,.}{dat ik zou komen indien}{het mogelijk was}\\

\haiku{{\textquoteright} herhaalden de twee,.}{zusterkens lachend in de}{handen kletsende}\\

\haiku{{\textquoteleft}Ik weet niet, ik kan.}{in dat woeste leven geen}{vermaak meer vinden}\\

\haiku{misschien zelfs zou hij.}{niet meer of zeer zelden met}{haar kunnen spelen}\\

\haiku{Inderdaad, vader.}{Wildenslag was een vijand}{van het onderwijs}\\

\haiku{Meer dan eens reeds had;}{hij met zijne vrouw over zijn}{inzicht gesproken}\\

\haiku{ik wil wel naar het,}{kantwerkhuis gaan ik zal er}{mijn best doen zooveel}\\

\haiku{Is het geluk van?}{haar kind de hoogste vreugde}{eener moeder niet}\\

\haiku{{\textquoteright} {\textquoteleft}Ja, maar zou zij dan?}{hare arme ouders wel}{blijven beminnen}\\

\haiku{mijn geluk, mijne.}{welvaart ben ik verschuldigd}{aan mijne moeder}\\

\haiku{In het eerst hadden;}{zij elkander getroost met}{de hoop op goed nieuws}\\

\haiku{Ik heb al moeite;}{gedaan om de kleinste naar}{de school te krijgen}\\

\haiku{zij was arm in de,.}{wereld en moest een ambacht}{leeren dit wist zij wel}\\

\haiku{Ja, ja, zeker, ik.}{neem uw edelmoedig voorstel}{aan uit gansch mijn hart}\\

\haiku{maar zoohaast er weder,.}{veel werk in Gent was zouden}{zij terugkomen}\\

\haiku{Dan werd het hem, als.}{hadde men met geweld iets}{uit zijn hart gerukt}\\

\haiku{somtijds stond hij op.}{en ging naar de  deur bij}{het minste gerucht}\\

\haiku{{\textquoteright} {\textquoteleft}Inderdaad, Adriaan,,.}{het is natuurlijk hij zal}{u gelukwenschen}\\

\haiku{{\textquoteright} {\textquoteleft}Ach, het is mijne,,{\textquoteright}.}{vrouw mijnheer antwoordde de}{ontroerde werkman}\\

\haiku{M. Raemdonck zou,.}{u goed willen doen indien}{het mogelijk is}\\

\haiku{{\textquoteright} ~ Bavo liet den:}{brief ter tafel vallen en}{begon te weenen}\\

\haiku{Maanden lang wachtte;}{hij op een tweede antwoord}{van Godelieve}\\

\haiku{Volgens zijn zeggen,;}{wonnen de Wildenslags veel}{geld ja veel te veel}\\

\haiku{Daar trad onverwachts,:}{een man in de kamer naar}{buiten roepende}\\

\haiku{Hij verborg hem zelfs,.}{niet dat hij het deed met een}{bijzonder inzicht}\\

\haiku{Dit is tamelijk.}{veel voor een jongeling van}{twee\"entwintig jaar}\\

\haiku{het zou mij dienen.}{om mijn vermaak een beetje}{af te wisselen}\\

\haiku{Het was ook wel wat;}{de meisjes eerst en daarna}{de ouders deden}\\

\haiku{maar vrouw Damhout zag,.}{wel dat iets anders zijnen}{geest benevelde}\\

\haiku{vreest gij nu, dat de?}{Damhouts ons gebed zouden}{kunnen verstooten}\\

\haiku{O, mijne schoone,!}{kindsheid hoe tergend ontstaat}{gij voor mijne oogen}\\

\haiku{en indien zij mijn,?}{gebed verwerpen omdat}{gij niet met mij zijt}\\

\haiku{dan zal ik alle,.}{schaamte alle gevoel in}{mijn hart versmachten}\\

\haiku{Gij hebt haar naar eene,;}{fabriek gezonden om geld}{uit haar te trekken}\\

\haiku{Mij ook beschaamde,;}{de poging welke ik bij}{u moest beproeven}\\

\haiku{hij liet zich op eenen:}{stoel zakken en zeide met}{schijnbare kalmte}\\

\haiku{Vergun mij als eene,.}{genade voor heden dit}{huis te verlaten}\\

\haiku{De jongeling stak,;}{de handen uit als om haar}{te wederhouden}\\

\subsection{Uit: Volledige werken 23. De jonge dokter}

\haiku{In deze kamer.}{woonde dus een student in}{de medicijnen}\\

\haiku{er was beweging.}{en gerucht van menschen in}{de straat gekomen}\\

\haiku{Gij erkent dus, dat?}{uwe gezondheid gevaar loopt}{van te bezwijken}\\

\haiku{hij lachte met de;}{anderen en sprak soms ook}{al een vroolijk woord}\\

\haiku{{\textquoteright} hem van de lippen,.}{viel wierp hij zich juichend aan}{den hals des grijsaards}\\

\haiku{{\textquoteleft}Daar, grootvader, drink,.}{dit glas het zal u deugd doen}{en u verkwikken}\\

\haiku{{\textquoteleft}kon ik gelooven wat,!}{uwe woorden mij voorspellen}{ik wierd zinneloos}\\

\haiku{Daar, grootvader, lees,{\textquoteright},.}{gij zelf sprak zij hem het schrift}{overhandigende}\\

\haiku{De voorzitter van;}{de Jury overlaadde mij}{met loftuitingen}\\

\haiku{De grootvader scheen.}{op dit oogenblik zich iets}{te herinneren}\\

\haiku{gedurende eene:}{lange wijl wist Adolf niet aan}{wien te antwoorden}\\

\haiku{Adelina alleen;}{hield zich meer ingetogen}{dan naar gewoonte}\\

\haiku{{\textquoteleft}Het is een man te,.}{paard die waarschijnlijk ook al}{om den dokter komt}\\

\haiku{{\textquoteright} Zij herkende dan.}{eerst de wachtende vrouw en}{stapte tot haar}\\

\haiku{{\textquoteright} Met droeven oogslag.}{wees de vrouw op het misvormd}{gelaat van haar kind}\\

\haiku{Zonder iemand den:}{tijd tot spreken te laten}{zeide de dokter}\\

\haiku{{\textquoteleft}Sa, vrienden, ik heb.}{slechts eenige oogenblikken}{om u te helpen}\\

\haiku{Ik kan nergens  ,!}{den voet zetten of men spreekt}{mij van Adolf Valkiers}\\

\haiku{Francisca heeft mij,.}{genoeg gezegd om mij het}{te doen vermoeden}\\

\haiku{Aan den knecht, die op,:}{het klinken der bel het hek}{kwam openen vroeg hij}\\

\haiku{Gij zult wel bemerkt,,?}{hebben zeker dat het niet}{zeer wel met mij gaat}\\

\haiku{Alzoo, Mijnheer, gij?}{zult den wintertijd in de}{stad gaan doorbrengen}\\

\haiku{Voorbij den molen,.}{gaande kwam hij weldra op}{eenen grooten steenweg}\\

\haiku{integendeel, er.}{speelde een stille glimlach}{op hare lippen}\\

\haiku{Ziet gij niet, moeder,?}{dat hij alle dagen meer}{en meer vermagert}\\

\haiku{Ik zal het u straks.}{beter en met meer vrijheid}{van geest bewijzen}\\

\haiku{{\textquoteleft}Adelina, kind, ik.}{weet niet welke macht uw woord}{op mij uitoefent}\\

\haiku{O, beroof mij niet,!}{van dien eenigen troost van die}{bron mijner sterkte}\\

\haiku{anderen zeiden,;}{dat het hem overkomen was}{door sterk te niezen}\\

\haiku{maar de zieke bleef.}{gedurig in eenen staat van}{halve bezwijmdheid}\\

\haiku{Ik eisch, dat er!}{oogenblikkelijk iemand}{worde geroepen}\\

\haiku{Wilt gij volstrekt uwe,}{nieuwe uitvinding op den}{pastoor beproeven}\\

\haiku{{\textquoteright} Bij het hooren van.}{Adolfs naam had Heuvels zich de}{leden gewrongen}\\

\haiku{{\textquoteright} {\textquoteleft}Inderdaad, maar het?}{middel om die vijfhonderd}{franken te vinden}\\

\haiku{indien het voor zijn,.}{geluk is moeten wij ons}{er over verheugen}\\

\haiku{Altijd, altijd zie,;}{ik hem onverpoosd waart zijn}{beeld voor mijne oogen}\\

\haiku{straks zal ik in den:}{hof wandelen en mij een}{beetje vermoeien}\\

\haiku{{\textquoteleft}Wel, wel,{\textquoteright} riep zij, {\textquoteleft}wie!}{hadde ooit zich aan dit nieuws}{kunnen verwachten}\\

\haiku{De handen tot een,:}{gebed samenvoegende}{smeekte Adelina}\\

\haiku{Gij zult mij helpen;}{in den strijd tegen mijne}{herinneringen}\\

\haiku{Waarom zou ik mij?}{laten wederhouden door}{eene zinnelooze hoop}\\

\haiku{Is de zucht om in,?}{eene groote stad te wonen nog}{nooit in u ontstaan}\\

\haiku{Daar ging ik nog het!}{bijzonder doel van mijn}{bezoek vergeten}\\

\haiku{De wind waaide uit,;}{het noordwesten en het was}{zeer koud en vochtig}\\

\haiku{en gij, meester, gij!}{verstoot meedoogenloos de}{bede van uw kind}\\

\haiku{{\textquoteright} De zieke zag de,.}{meid met verstoorden blik aan}{doch antwoordde niet}\\

\haiku{Langen tijd bleef de.}{volledigste stilte in}{de kamer heerschen}\\

\haiku{Gij poogt ons te doen,;}{gelooven dat zijne ziekte}{geen gevaar aanbiedt}\\

\haiku{Waar de hemel zou,,}{beslist hebben wat kan daar}{toch een geneesheer}\\

\haiku{maar de jongeling.}{behoefde voorwaar deze}{aanmoediging niet}\\

\haiku{{\textquoteright} kreet de meid, met de.}{handen opgeheven van}{de trappen springend}\\

\haiku{Het hart wilde van.}{zalig ongeduld mij in}{den boezem breken}\\

\haiku{{\textquoteleft}Hij toonde zooveel,.}{betrouwen niet als gij en}{scheen zelfs bekommerd}\\

\haiku{dat wij dezelfde?}{geneesmiddelen moeten}{blijven aanwenden}\\

\haiku{Het was voorwaar eene,}{openbaring des hemels die}{mij deed besluiten}\\

\haiku{De grootvader hield.}{de oogen biddend ten hemel}{opgeheven}\\

\subsection{Uit: Volledige werken 24. De burgers van Darlingen}

\haiku{Hunne goederen;}{bestonden in pachthoeven}{en landerijen}\\

\haiku{Een lakensch vestje,.}{grijpende begon zij er}{aan voort te naaien}\\

\haiku{{\textquoteright} {\textquoteleft}Eilaas, hoe is het,!}{toch mogelijk dat gij dus}{van uwe zuster spreekt}\\

\haiku{Kwas, over de zotte!}{kleederdracht mijner zuster}{Hermina spreken}\\

\haiku{De zaak is niet, te;}{weten of mijne zuster}{kwaad doet of kwaad meent}\\

\haiku{zooals het behoort, gij.}{zoudt redenen hebben om}{het te betreuren}\\

\haiku{De weenende vrouw,;}{sprong hem achterna om hem}{te wederhouden}\\

\haiku{Ik zal niet lang meer,,{\textquoteright}.}{hier blijven wonen mevrouw}{hernam de dienstmeid}\\

\haiku{{\textquoteleft}Daar, Sophie, daar.}{is een beetje geld op uwe}{huur van deze maand}\\

\haiku{Daarmede kan men.}{eene vrouw in het nieuw kleeden}{van hoofd tot voeten}\\

\haiku{Nauwelijks kon zij.}{eenige erkentelijke}{woorden stamelen}\\

\haiku{de gedachte van!}{het huwelijk vervult mij}{met eenen doodelijken schrik}\\

\haiku{{\textquoteleft}Hermina, mijne,,.}{arme Hermina schep moed}{ween niet zoo bitter}\\

\haiku{waarom M, Blondeel,}{zoo aandrong om Hermina}{te doen vertrekken}\\

\haiku{Is hunne eenige,}{studie niet hunnen armen}{pachters den laatsten}\\

\haiku{Theresia is eene.}{gezworene vijandin}{van het huwelijk}\\

\haiku{onderweg, mij dunkt,.}{ik hadde tranen gestort}{te midden der straat}\\

\haiku{maar wanneer men trouwt,.}{moet men toch een bestaan in}{de wereld hebben}\\

\haiku{hij was een ware,;}{Engelsch man doch hij had er}{niets bij verloren}\\

\haiku{dat gij, mijnheer, dat.}{uwe goede zuster mijnen}{wenschen gunstig waart}\\

\haiku{{\textquoteleft}Komt, komt, dit is toch;}{geene reden om het eten te}{laten koud worden}\\

\haiku{hij schijnt te vreezen,.....}{dat gij hem redenen tot}{gramschap zult geven}\\

\haiku{Ik voorzie, dat uwe.}{goestingen dezelfde niet}{zijn als de mijne}\\

\haiku{Uwe zuster schijnt mij.}{eene zoete inborst en een}{goed hart te hebben}\\

\haiku{{\textquoteright} Op dit oogenblik.}{hergalmde de straat van een}{plotseling gerucht}\\

\haiku{beiden gevoelen,.}{dat zij de reden zijn van}{elkanders verdriet}\\

\haiku{verkoop uw kind aan,.}{een grof mensch versleten naar}{ziel en naar lichaam}\\

\haiku{Het werd in den tuin,.}{stil en eenzaam als ware}{er niets geschied}\\

\haiku{{\textquoteright} {\textquoteleft}Ik weet niet tot wat,{\textquoteright}.}{te besluiten antwoordde}{Blondeel weifelend}\\

\haiku{De arme jongen;}{ligt nu bedolven in eene}{sombere wanhoop}\\

\haiku{De hemel gunne.}{mij krachten om tegen het}{verdriet op te staan}\\

\haiku{De vrederechter.}{en het tribunaal moeten}{er tusschenkomen}\\

\haiku{Dit zal mij toch niet.}{beletten hem aan te zien}{als ons beider kind}\\

\haiku{Hij zweeg eene wijl en.}{deed geweld om zich zei ven}{meester te worden}\\

\haiku{Dan zeide hij met:}{eenen scherpen grimlach en met}{beklemde gramschap}\\

\haiku{Ziehier, Romys, wat,,.}{ik om het kort te maken}{u te zeggen heb}\\

\haiku{Gij zijt volstrekt  ,.}{meester over uw kind evenals}{wij over ons fortuin}\\

\haiku{maar is uw fortuin?}{dan insgelijks niet het goed}{onzer familie}\\

\haiku{gij weet zoo goed als,.}{ik dat hij in het geheel}{geene familie heeft}\\

\haiku{Niettemin, al die;}{diamanten steken de oogen}{der aanschouwers uit}\\

\haiku{de meisjes tusschen.}{het volk murmelen woorden}{van bewondering}\\

\haiku{De bruidegom zegt,:}{met zoete blijde stem tot}{zijne echtgenoote}\\

\haiku{M. Pottewal en.}{zijne vrouw zitten aan het}{midden der tafel}\\

\haiku{zij houdt het hoofd recht.}{en toont een onbewogen}{en deftig gelaat}\\

\haiku{het is zichtbaar, dat.}{hij zijne ontsteltenis}{poogt te bedwingen}\\

\haiku{{\textquoteright} {\textquoteleft}Maar gij beeldt u dit,{\textquoteright}.}{alles in wedervoer de}{oude dame}\\

\haiku{M. Pottewal zal,?}{met den tocht van zes uren te}{huis komen niet waar}\\

\haiku{Eindelijk ging zij.}{ter kamer uit en stapte}{over eene opene plaats}\\

\haiku{Ik had eene goede;}{zaak ontmoet en tienduizend}{franken gewonnen}\\

\haiku{{\textquoteright} {\textquoteleft}Ik wacht, Theresia,,.}{totdat ik wete waarvan}{gij mij beschuldigt}\\

\haiku{Neen, bedrieger, er.}{staan wel andere dingen}{op uwe rekening}\\

\haiku{{\textquoteleft}Schier eene gansche maand!}{huishouden verzwolgen op}{eenen enkelen dag}\\

\haiku{Verwondert het u,?}{dat ik insgelijks zijne}{dame heb gegroet}\\

\haiku{Weet gij wat het eenig,?}{middel is om hier de rust}{te doen wederkeeren}\\

\haiku{Pottewal zette:}{zich nevens haar en wilde}{haar de hand nemen}\\

\haiku{Dan hief hij traaglijk:}{zijne oogen ten hemel en}{morde wanhopig}\\

\haiku{{\textquoteright} {\textquoteleft}De tweede maal liep,.}{ik hem schier tegen het lijf}{in de Melkstraat}\\

\haiku{Wat zij hem meldde,,;}{scheen hem noch buitengewoon}{noch wonderlijk}\\

\haiku{Hij deed een teeken,,;}{als toonde hij de meid die}{bij de wiege zat}\\

\haiku{Eerst was het een twist,.}{om te weten aan wie het}{wichtje wel geleek}\\

\haiku{M. De Cock zal mij.}{helpen om de wiege naar}{boven te dragen}\\

\haiku{Dit ontwerp schijnt slechts;}{uitgevonden om ons voor}{den zot te houden}\\

\haiku{{\textquoteright} {\textquoteleft}Mijn broeder Blondeel.}{verzekert nochtans dat het}{zal aanvaard worden}\\

\haiku{Schoon als zij is, zou.}{zij misschien een millioen}{gevonden hebben}\\

\haiku{God zij geloofd, ik{\textquoteright} {\textquoteleft},!}{zie u wel te pas.Wat schoon}{wat betooverend kind}\\

\haiku{zij had haren man,.}{gezien met het kind harer}{zuster op de knie}\\

\haiku{de blijdschap en de.}{opgeruimdheid waren weg}{van het gezelschap}\\

\haiku{{\textquoteleft}Inderdaad, ik heb,{\textquoteright}.}{het stoomtuig hooren fluiten}{bemerkte Romys}\\

\haiku{{\textquoteleft}Hadde ik mijn kind,.}{niet op den schoot ik vloog u}{daarvoor aan den hals}\\

\haiku{zij wierp hem de winst;}{en het geluk van Ernest}{in het aangezicht}\\

\haiku{Onderweg streelde,.}{zij den eenigen hond die was}{gespaard geworden}\\

\haiku{Hare oogen schenen,.}{te vlammen nochtans en ik}{was vervaard van haar}\\

\haiku{Een kreet van blijdschap,.}{ontvloog haar toen zij hare}{moeder bemerkte}\\

\haiku{de overmatige.}{blijdschap niet minder dan het}{overmatig verdriet}\\

\haiku{Theresia, ik heb.}{u arm gemaakt en den naam}{van uw kind onteerd}\\

\haiku{Heeft uw man slechte,.}{zaken gedaan het is voor}{zijne rekening}\\

\haiku{Ik liet u den naam;}{van mijnen echtgenoot met}{laster overladen}\\

\haiku{{\textquoteright} {\textquoteleft}Toe, onkel lief, van!}{Janneken en Mieken en}{van den suikerberg}\\

\haiku{{\textquoteright} {\textquoteleft}Het is inderdaad,,{\textquoteright}.}{zoo Hermina antwoordde}{Blondeel met fierheid}\\

\haiku{{\textquoteleft}Gij moet inzien, dat.}{ik geene vingeren heb van}{vijf en twintig jaar}\\

\haiku{Die muzikanten,.}{zij munten zeker niet uit}{door ootmoedigheid}\\

\haiku{gij zult niet zonder,.}{vrienden niet zonder hulp op}{de wereld blijven}\\

\haiku{{\textquoteleft}uwe goedheid kan den,,.}{noodlottigen slag die ons}{bedreigt niet afkeeren}\\

\haiku{Neen, neen, o hemel,!}{geen vlekje op het voorhoofd}{mijner Hermina}\\

\haiku{Het fortuin kan men;}{winnen en verliezen en}{nogmaals herwinnen}\\

\haiku{Dezen avond zult gij,.}{vertrekken naar Holland en}{van daar naar Londen}\\

\haiku{maar ik heb u geld,.}{geleend en gij hebt het mij}{teruggegeven}\\

\haiku{Ik ben schuldig, ten.}{minste aan eene zinnelooze}{onvoorzichtigheid}\\

\haiku{{\textquoteright} En zij ontvouwde.}{voor zijne oogen eenen ganschen}{bundel bankbriefjes}\\

\subsection{Uit: Volledige werken 26. Het Goudland}

\haiku{{\textquoteright} Een zucht ontsnapte,.}{den klerk en hij hief de oogen}{klagend ten hemel}\\

\haiku{Nog vier andere;}{schepen zendt de maatschappij}{naar Californi\"e}\\

\haiku{want een glimlach van.}{bewondering blonk als een}{licht op zijn gelaat}\\

\haiku{Eenige andere.}{booten zag men insgelijks}{op den stroom varen}\\

\haiku{Zou hij lamme beenen,?}{krijgen nu het beslissend}{uur gekomen is}\\

\haiku{gij moest mij maar eens,,.}{zeggen Mijnheeren wat ik}{hier in de hand heb}\\

\haiku{Het is Donatus,:}{toch niet die de eerste van}{honger zou sterven}\\

\haiku{Lage ik nu slechts!}{op onzen hooizolder te}{Natten-Haesdonck}\\

\haiku{Spaansche peper, ik,.}{ken het het dient om ezels op}{den loop te jagen}\\

\haiku{Wij zullen dien tijd.}{waarnemen om op alles}{orde te stellen}\\

\haiku{Tegen den middag.}{werden de reizigers op}{het dek geroepen}\\

\haiku{Niemand daaronder.}{verstond hem of betuigde}{hem eenige vriendschap}\\

\haiku{Daarenboven, voor.}{de minste reden stampt hij}{als een woedende}\\

\haiku{Gij wilt de Jonas?}{het lot bereiden van het}{Portugeesche schip}\\

\haiku{Evenwel, van dien dag.}{af aan had hij maar een slecht}{leven op het schip}\\

\haiku{Hij ziet altijd zuur,,;}{nochtans en het is hem te}{veel dat hij glimlacht}\\

\haiku{{\textquoteleft}Sedert het verlies.}{uwer banknoten ziet gij}{niets meer dan dieven}\\

\haiku{het was zichtbaar, dat.}{een heete wraakdorst hem in}{den boezem brandde}\\

\haiku{zijne tempels zijn,;}{de speelhuizen die gij hebt}{gezien of zult zien}\\

\haiku{Uit het antwoord bleek,;}{dat zulke verzending zeer}{gemakkelijk was}\\

\haiku{misschien zal hetgeen,.}{ik over hem schrijf hem nuttig}{zijn voor de toekomst}\\

\haiku{ik neem tien dollars;}{en zet ze te gelijk op}{\'e\'enen slag der kaart}\\

\haiku{De bediende liet.}{het zich geen tweemaal zeggen}{en vloog naar buiten}\\

\haiku{de revolvers en}{de andere messen heb}{ik gekocht voor u.}\\

\haiku{{\textquoteright} riep Jan, terwijl hij.}{zijne twee gezellen de}{deur uitduwde}\\

\haiku{Ik geloof, dat ik.}{er ten minste vier of vijf}{heb doodgeschoten}\\

\haiku{het is tevens de,.}{streek waar de wildemannen}{zich het meest toonen}\\

\haiku{wij zullen eiken.}{dag over onze aanstaande}{reis kunnen spreken}\\

\haiku{{\textquoteleft}Wie weet, of die sal......?}{sal die struikroovers niet reeds te}{San-Francisco zijn}\\

\haiku{vooruit dan..... en bij,!}{de minste vijandige}{beweging geeft vuur}\\

\haiku{maar richt uw oog naar.....}{alle kanten en houdt de}{geweren gereed}\\

\haiku{Van de schelmen, die,;}{ons hebben aangerand is}{niets meer te vreezen}\\

\haiku{Geen wonder dus, dat,;}{ik u gaarne zie omdat}{gij een muilezel zijt}\\

\haiku{{\textquoteleft}Ja, vriend Roozeman, het,{\textquoteright}.}{zijn verre uit de slimsten}{bemerkte Pardoes}\\

\haiku{{\textquoteleft}Het is geen half uur,.}{geleden dat ik u heb}{hooren aflossen}\\

\haiku{{\textquoteright} En toen men hem op,:}{een twintigtal stappen had}{gevolgd hernam hij}\\

\haiku{Wat de belooning,,;}{betreft die hij ons belooft}{vertrouwt daar niet op}\\

\haiku{de gereedschappen,.}{wie wil mij komt geen enkel}{stuk meer op den rug}\\

\haiku{Wij stelden ons te;}{weer en losten insgelijks}{onze vuurwapens}\\

\haiku{Dezen losten te;}{gelijk hunne geweren}{op de vijanden}\\

\haiku{maar Jan Creps had zich.}{vooruitgeworpen en den}{strop doorgesneden}\\

\haiku{{\textquoteleft}Langs het bed van dien.}{waterval zullen wij in}{de vlakte dalen}\\

\haiku{{\textquoteright} Zij bevonden zich.}{nu omtrent den winkel van}{eenen goudwisselaar}\\

\haiku{Achter dezen toog.}{stond een magere man met}{eenen bril op den neus}\\

\haiku{Gij meent, dat hij  ?}{die eenvoudige zwetsers}{niet heeft bedrogen}\\

\haiku{Zonder dit middel.}{zouden zij het hier niet lang}{kunnen volhouden}\\

\haiku{{\textquoteright} {\textquoteleft}Neen, voor eenen klomp goud,,!}{zoo groot als mijne vuist ga}{ik daar niet binnen}\\

\haiku{Toen hij vernam, dat,}{zij elk eenen grog gedronken}{hadden eischte}\\

\haiku{maar hier is geen draak.}{met zeven koppen noodig om}{venijn te spuwen}\\

\haiku{{\textquoteleft}Spoedig, mannen, geeft;}{mij nog een paar schoppen van}{dit roodachtig zand}\\

\haiku{{\textquoteright} In weinig tijds was.}{de koffie gekookt en de}{koeken gebakken}\\

\haiku{Zulke brokken goud!}{glinsteren er vele in}{den grond van den put}\\

\haiku{William stond voor.}{de deur en de moordenaar}{ongetwijfeld ook}\\

\haiku{Het was een galm, hol,.}{grof en ratelend als een}{lange donderslag}\\

\haiku{{\textquoteleft}Beer of geen beer, ik!}{zal het arme dier z\'o\'o niet}{laten vermoorden}\\

\haiku{{\textquoteleft}Vleesch is vleesch,?}{en indien het goed smaakt en}{niet schadelijk is}\\

\haiku{Ik betrouw mij niet,;}{op vrienden die beervleesch}{in hun lijf hebben}\\

\haiku{al wilden wij ze,.}{ontwijken wij zouden er}{niet in gelukken}\\

\haiku{een paar breede   ,.}{Tusschen zijne armen om}{hem te versmachten}\\

\haiku{Een van ons beiden!}{zal hier in de woestijn zijn}{gebeente laten}\\

\haiku{Kon ik mij het haar!}{tot het laatste pijltje maar}{uit den kop trekken}\\

\haiku{Het eenige, dat men,;}{nog hoorde was eene klacht over}{gebrek aan water}\\

\haiku{Wat hij hun zeide,;}{waren slechts uitvindsels om}{hun moed te geven}\\

\haiku{Dus deze beek is,.}{een teeken dat wij onzen}{placer naderen}\\

\haiku{Hij mompelde van,,:}{macht van eer van grootheid en}{scheen half dwaalzinnig}\\

\haiku{met door het water,.}{te gaan kunnen wij deze}{kloven bereiken}\\

\haiku{zullen wij dan elk?}{met een gewicht van tien pond}{aan den hals loopen}\\

\haiku{{\textquoteright} {\textquoteleft}Gij moet op voorhand.}{goed ademen en dan den mond}{gesloten houden}\\

\haiku{Victor was langer;}{dan de anderen onder}{water gebleven}\\

\haiku{Deze hoop gaf hun.}{moed en scheen hunne krachten}{te verdubbelen}\\

\haiku{Opdat het werk niet,,}{te zeer vertraagd wierd stelde}{hij voor van morgen}\\

\haiku{Hij naderde hem,:}{klopte hem op den schouder}{en zeide schertsend}\\

\haiku{{\textquoteleft}Voor al de schatten;}{der wereld zou Jan Creps hier}{niet meer wederkeeren}\\

\haiku{{\textquoteright} juichte de matroos,.}{zijnen mond naar het oor zijns}{makkers neigende}\\

\haiku{Deze teekenen.}{brachten hem tot aan den voet}{eener steile rots}\\

\haiku{hij knarsetandde,.}{en de oogen schenen hem uit}{het hoofd te komen}\\

\haiku{maar gij hebt te doen,.}{met magen die hunnen reuk}{verloren hebben}\\

\haiku{{\textquoteright} {\textquoteleft}Dit is te zeggen,{\textquoteright}, {\textquoteleft}.}{riep Donatusdat ik eens}{overvloedig zal eten}\\

\haiku{Het overige ei,.}{behoud ik voor mij om te}{weten hoe het smaakt}\\

\haiku{een mensch op zulke?}{onbarmhartige wijze}{uit enkel plezier}\\

\haiku{en misschien, misschien!}{zou hij hem de hand van zijn}{Anneken toestaan}\\

\haiku{Moest ik daarom naar?}{het vermaledijd land van}{Californi\"e gaan}\\

\haiku{de ziel, die leefde,!}{in haren blik zoo zuiver}{en zoo beminnend}\\

\haiku{{\textquoteleft}Madam, er is een,.}{man in den winkel die u}{volstrekt wil spreken}\\

\subsection{Uit: Volledige werken 28. Moeder Job. Een goed hart. Houten Clara}

\haiku{dan, na vervulling,:}{van den heiligen plicht zal}{het vreugde worden}\\

\haiku{8).daarom liet gij.}{de U uit nieuwe door uwe}{vingeren vallen}\\

\haiku{{\textquoteright} {\textquoteleft}Ik ben de beste,{\textquoteright}, {\textquoteleft}!}{schutter riep de brouweren}{iedereen weet het}\\

\haiku{Op de tien schoten!}{maar \'e\'ene roos en tweemaal}{buiten  het wit}\\

\haiku{Zij komt mij toe, en:}{ik zal ze door een ander}{moeten zien winnen}\\

\haiku{maar, zeg wat ge wilt,,.}{waar het geluk is  wil}{het zijn en blijven}\\

\haiku{{\textquoteright} {\textquoteleft}Eene dochter, die zal.....}{trouwen met Gabri\"el van}{onzen notaris}\\

\haiku{meesttijds hield hij het;}{hoofd gebogen en den blik}{ter aarde gericht}\\

\haiku{maar nu scheen een ver.}{gerucht zijne ooren te}{hebben getroffen}\\

\haiku{zijne lippekens,:}{verroerden en mij dacht dat}{het wilde zeggen}\\

\haiku{als men  het kind,.}{warm houdt dan zal de dokter}{het wel genezen}\\

\haiku{{\textquoteleft}Niemand weet het,{\textquoteright} was, {\textquoteleft},.}{het antwoorddan gij alleen}{misschien Rosina}\\

\haiku{reeds drie dagen..... zijn.....}{vader en zijne moeder}{vergaan in tranen}\\

\haiku{doch zij herstelde,:}{zich onmiddellijk schoof eenen}{stoel bij en zeide}\\

\haiku{- en trouw maar met eenen,!}{anderen man gij zult toch}{niet gelukkig zijn}\\

\haiku{Dit papier heb ik;}{den ganschen nacht met mijne}{tranen overgoten}\\

\haiku{de eenzaamheid en.}{de avondkoelte zullen mij}{misschien versterken}\\

\haiku{{\textquoteleft}Ik geloof niet, dat.}{de notaris daar nog zal}{willen van hooren}\\

\haiku{Kaat, de huismeid, was;}{met het hoofd op eene tafel}{in slaap gevallen}\\

\haiku{Verlicht mijnen geest,.}{of ik bezwijk onder het}{gewicht mijner smart}\\

\haiku{maar dit bewijst ten.}{voordeele van zijn eenvoudig}{en beminnend hart}\\

\haiku{Daar ziet gij het nu,.}{wel dat wij voor den rampspoed}{zijn geboren}\\

\haiku{Welnu, waar blijft gij '?}{met het liedeken vant}{zal wel beter gaan}\\

\haiku{Elk oogenblik, dat,.}{verloopt is eene hel van angst}{en lijden voor hem}\\

\haiku{Hugo ligt gebukt,;}{onder het akeligst lot dat}{eenen man kan treffen}\\

\haiku{zijne eer, zijnen,,.}{naam zijne vrijheid alles}{wil men hem ontrooven}\\

\haiku{Maar, notaris, het:}{moet u gemakkelijk zijn}{het geld te vinden}\\

\haiku{Zijne moeder heeft;}{sedert vier dagen nog niets}{gedaan dan weenen}\\

\haiku{{\textquoteright} {\textquoteleft}Welnu,{\textquoteright} hernam de, {\textquoteleft}.}{notarisGabri\"el zelf}{heeft het geschreven}\\

\haiku{Houdt het kind warm, dat.}{geen trek van deur of venster}{het raken kunne}\\

\haiku{Zijn vader zal de,,.}{zaak waarvan gij spreekt al lang}{vergeten hebben}\\

\haiku{laat mij door, ik moet,{\textquoteright}.}{eene haastige boodschap doen}{smeekte moeder Job}\\

\haiku{Nog eenige stappen;}{en zij zou van tusschen het}{koren geraken}\\

\haiku{Welhaast, als hadde,:}{de wanhoop haar overwonnen}{zuchtte zij verschrikt}\\

\haiku{dat Hij in zijne}{goedheid mij krachten late}{tot het volbrengen}\\

\haiku{{\textquoteright} {\textquoteleft}Ik heb geenen tijd, vriend,{\textquoteright},.}{Mols riep zij zonder haren}{stap te vertragen}\\

\haiku{de baron jaagde,,.}{hem om zoo te zeggen als}{eenen schelm van den hof}\\

\haiku{de minste traagheid.}{in de uitvoering zijner}{wenschen maakt hem gram}\\

\haiku{Er was immers niets?}{in zijnen beklaaglijken}{toestand veranderd}\\

\haiku{dit zijn woorden van,,.}{menschen die alles in het}{zwart zien gelijk gij}\\

\haiku{Hij is van hoofd tot;}{voeten in het zwart gekleed}{als een lijkbidder}\\

\haiku{zij verduisterde.}{uwen geest en deed u al wat}{kwaad is overdrijven}\\

\haiku{maar gij moest maar eens,!}{weten wat ik altemaal}{in mijn hart doorsta}\\

\haiku{Walter is op de;}{Pruisische grenzen door de}{gendarmes verrast}\\

\haiku{hun oog is zonder,.}{leven zij hebben honger}{en zoeken voedsel}\\

\haiku{- wat zoudt gij dan wel!}{met uwen enkelen wagen}{kunnen verrichten}\\

\haiku{Anders hadden zij.}{zekerde zware huispacht}{niet kunnen dragen}\\

\haiku{Het is gelijk, gij,!}{zult morgen naar Rijsel op}{den ijzeren weg}\\

\haiku{vindt gij daar nog de,.}{helft van twaalf franken in dan}{zal het alles zijn}\\

\haiku{Misschien is het maar,.....}{koud in eene kamer waar nooit}{vuur heeft  gebrand}\\

\haiku{Met Paschen of.}{weinige dagen daarna}{zouden zij trouwen}\\

\haiku{Hij trad binnen en.}{legde een groot papieren}{pak op de tafel}\\

\haiku{maar niet verschieten,,,}{Christina want het is iets}{dat u eenen fellen}\\

\haiku{Hij vond dezen met.}{eenen omwonden voet voor eene}{tafel gezeten}\\

\haiku{{\textquoteright} {\textquoteleft}Dit verlies brengt mij,{\textquoteright}.}{in eene groote verlegenheid}{zeide de koopman}\\

\haiku{{\textquoteright} {\textquoteleft}Mijnheer, o, Mijnheer,,!}{gij veroordeelt mij tot de}{schande tot den dood}\\

\haiku{mijne hersens zijn,.}{zoo vermoeid dat ik er gansch}{duizelig van ben}\\

\haiku{Ach, daar staat nog de,;}{stoof die mijne verstijfde}{leden heeft ontdooid}\\

\haiku{Zij was het, die mij,.}{ter schole geleidde toen}{ik nog een kind was}\\

\haiku{het is alsof een.}{nieuw leven met het bloed door}{mijne aderen vloeit}\\

\haiku{Hief den ijzeren.}{klopper der poort op en liet}{hem nedervallen}\\

\haiku{hare ziel hing nog.}{aan het mondelijn van het}{aangebeden kind}\\

\haiku{welk ook het geheim,:}{haars harten zij ik toch zal}{het niet verraden}\\

\haiku{Haar echtgenoot vond;}{in hare blijdschap eene bron}{van troost en genot}\\

\haiku{{\textquoteleft}Gravinne, laat ons,.}{in de kamer gaan waar de}{klavecimbel staat}\\

\haiku{zuster Cathelyne;}{uit het Falconsklooster heeft}{haar muziek geleerd}\\

\haiku{de Dwene zette;}{zich insgelijks nevens de}{moeder in eenen stoel}\\

\haiku{Nu,{\textquoteright} sprak de moeder, {\textquoteleft}.}{zing het lied van Met vreugde}{willen wij zingen}\\

\haiku{{\textquoteleft}Ja, 't is goed, nu......}{zing ik toch niet meer en van}{mijn leven niet meer}\\

\haiku{Uwe vereerende.}{vriendschap is mij eene schoone}{belooning genoeg}\\

\haiku{Als zij dan uit het,;}{huis trouwen verstrekt haar het}{gespaarde tot bruidschat}\\

\haiku{{\textquoteright} Met ontroering greep:}{de gravinne de hand der}{moeder en zeide}\\

\haiku{Eensklaps verbleekte,.}{de edelvrouw en zij begon}{van angst te beven}\\

\haiku{De Signora, door,.}{de Dwene gevolgd trad in}{hare slaapkamer}\\

\haiku{want haar uur is nooit,.}{zoo juist dat het som wijlen}{niet veel verschille}\\

\haiku{pijnlijke zuchten.}{waren het eenige antwoord}{op des meiskens vraag}\\

\haiku{evenwel was er nog.}{een schijn van ongeloof in}{haren vreugdelach}\\

\haiku{{\textquoteleft}Ik ben uwe eenige, -!}{uwe echte moeder en ik}{heb geen ander kind}\\

\haiku{{\textquoteright} De Dwene rustte.}{met het hoofd op den stoel en}{weende sprakeloos}\\

\haiku{ik ben in zijn oog;}{een vuig en verachtelijk}{schepsel geworden}\\

\haiku{Ik ben gekomen.}{om de hel des twijfels in}{uwen boezem te dooven}\\

\haiku{indien het waarheid,,.}{is wat gij zeggen gaat zoo}{zegene u God}\\

\haiku{{\textquoteright} {\textquoteleft}Eilaas,{\textquoteright} zuchtte de, {\textquoteleft}?}{graafwaarom mij die droeve}{tijden herinnerd}\\

\haiku{hoe dikwijls zij, voor,}{uwe voeten knielend onder}{eenen vloed van tranen}\\

\haiku{{\textquoteleft}Clara, het is de,.}{graaf d'Almata de man van}{uwe  beschermster}\\

\haiku{{\textquoteright} {\textquoteleft}Mijn vader is in,{\textquoteright}, {\textquoteleft}.....}{den hemel zuchtte Clara}{hij bidt God voor mij}\\

\haiku{{\textquoteright} riep d'Almata met, {\textquoteleft};}{ontroeringdit geheim wilt}{gij niet verraden}\\

\haiku{Ik zou de stem uws,.....}{vaders  miskennen zijn}{gebed verwerpen}\\

\haiku{- Maar gij zult  niets,?}{zeggen van hetgene hier}{geschied is niet waar}\\

\haiku{Reeds drie dagen was.}{het feest onder de meiskens}{in het Maagdenhuis}\\

\haiku{Zie nu maar, dat gij,!}{de waarheid kunt zeggen als}{het mogelijk is}\\

\subsection{Uit: Volledige werken 29. Valentijn. Eene verwarde zaak}

\haiku{Wanneer er ten uwent,.}{eene koe iets overkomt dan staat}{gij het beest wel bij}\\

\haiku{{\textquoteright} {\textquoteleft}Ik begrijp die hooge,{\textquoteright}.}{woorden maar half mompelde}{Monica dubbend}\\

\haiku{Maar ik ben eenig kind,.}{en mijne ouders willen}{daar niet van hooren}\\

\haiku{Zij is oud, en wat,.}{zij eens heeft besloten blijft}{onveranderlijk}\\

\haiku{en geen mensch op het,,,;}{dorp die mij nadert geen die}{mij de hand toereikt}\\

\haiku{want des avonds adem ik,;}{de geuren die van achter}{de haag opwalmen}\\

\haiku{En geen mensch die het,.}{reddend woord spreekt niemand die}{de hand hem toereikt}\\

\haiku{Moeilijk ware het,;}{geweest den ouderdom van}{dit mensch te raden}\\

\haiku{Wel waren zijne;}{kleederen tot op den draad}{versleten misschien}\\

\haiku{Nu is het te laat.}{om tot dit middel mijne}{toevlucht te nemen}\\

\haiku{Dan keerde hij zich.}{om en stapte mijmerend}{naar zijne woning}\\

\haiku{Helena wees met:}{den vinger naar eene kleine}{hoogte en zeide}\\

\haiku{in zulke zware.}{aarde als deze moet zij}{allengs versterven}\\

\haiku{Ik heb er in mijn,.}{leven vele duizenden}{gemaakt Mejuffer}\\

\haiku{{\textquoteright} De onderwijzer;}{aanschouwde het meisje met}{groote verwondering}\\

\haiku{En gij zelf, poogt gij?}{niet altijd op uw best voor}{den man te komen}\\

\haiku{- Intusschen wiesch zij,;}{mijne wonde met eene hand}{zoo zacht als fluweel}\\

\haiku{ja, hij zette zich.}{zelfs op de bank en schouwde}{eene wijl ten gronde}\\

\haiku{de gansche wereld,;}{die zich voor ons beglanst met}{een onbekend licht}\\

\haiku{Oh, wat zullen wij!}{altezamen gelukkig}{zijn op de wereld}\\

\haiku{Heb ik gevraagd om?}{die hel in mijnen boezem}{te voelen branden}\\

\haiku{{\textquoteleft}Ja, ja,{\textquoteright} zeide hij, {\textquoteleft}.}{in zich zelvende goede}{Hendrik heeft gelijk}\\

\haiku{Op een paar duizend.}{franken zal ik niet zien om}{u te beloonen}\\

\haiku{{\textquoteright} mompelde hij, {\textquoteleft}speelt,,?}{gij comedie Valentijn}{of is het gemeend}\\

\haiku{Dan de zegepraal.}{van het arglistige mensch}{zou niet lang duren}\\

\haiku{Anders, hoe zou hij?}{zelf het al lachende ons}{komen vertellen}\\

\haiku{Dit huwelijk moet.}{u ongelukkig maken}{voorgansch uw leven}\\

\haiku{Ik vermoedde niet,;}{dat er u zooveel nijd in}{het hart kon liggen}\\

\haiku{{\textquoteright} Het meisje misgreep;}{zich gewis over den echten}{zin dezer woorden}\\

\haiku{ongelukkig en,.}{bedroefd bovenmate en}{toch rustig en sterk}\\

\haiku{Het vuur der woede;}{heeft een oogen-blik mijn}{voorhoofd doen gloeien}\\

\haiku{maar de heiligheid.}{der plaats en mijn eerbied voor}{haar weerhielden mij}\\

\haiku{Uw hart is goed, gij,.}{hebt verstand en gij zoudt uw}{geld niet verkwisten}\\

\haiku{Het laatste woord van;}{Helena's vader verbrak}{echter zijnen droom}\\

\haiku{{\textquoteright} {\textquoteleft}Ik ben een wees en,{\textquoteright}.}{heb geene familie zeide}{de onderwijzer}\\

\haiku{Alleen zijnde, bleef;}{Valentijn nog eene wijl in}{zijne verbluftheid}\\

\haiku{Slechts bij het begin.}{der tweede bladzijde werd}{zijne stem luider}\\

\haiku{{\textquoteright} De onderwijzer;}{bezweek schier van geluk en}{van ontsteltenis}\\

\haiku{Helena moet de,.}{vrouw van u worden of de}{vrouw van Casimir}\\

\haiku{want het was voor hem.}{onder meer dan \'e\'en opzicht}{een plechtige dag}\\

\haiku{Al kwam de koning,,.}{zelf ik zou niet toelaten}{dat men u stoorde}\\

\haiku{maar daar klonk de stem,:}{van den kosterszoon die hem}{van buiten toeriep}\\

\haiku{Gij waart gereed tot,;}{alles om ons dit doel te}{helpen bereiken}\\

\haiku{{\textquoteright} M. Stoop gaf niet veel;}{acht op de gramme woorden}{van den olieslager}\\

\haiku{maar de olieslager;}{moest toch een zijner woorden}{hebben opgevat}\\

\haiku{Helena uwe hand.}{aanbieden en den bloodaard}{niet met haar spelen}\\

\haiku{want hare wangen.}{toonden nog de sporen van}{vergoten tranen}\\

\haiku{het eenige middel,,.}{daartoe het eenige is uw}{huwelijk met mij}\\

\haiku{zij stapte naar de:}{deur der kamer en morde}{nog in het heengaan}\\

\haiku{Dan zeide hij zeer,:}{stil tot de vrouwen die hem}{bevend aanzagen}\\

\haiku{De olieslager, reeds,:}{een beetje rood van den wijn}{trad binnen en vroeg}\\

\haiku{Mag ik hopen, dat?}{gij mij deze eenige gunst}{niet zult weigeren}\\

\haiku{ik heb  eenige.}{aanteekeningen in ons}{pachtboek geschreven}\\

\haiku{Met kalmte, zonder,.}{aangejaagdheid voor eenigen}{tijd nog ten minste}\\

\haiku{Ik wil u niet in.}{die oude diligence}{laten vertrekken}\\

\haiku{{\textquoteleft}Maar, Helena, het,.}{is niet volstrekt noodig dat ik}{heden vertrekke}\\

\haiku{{\textquoteleft}Wat drommel, Jan, is?}{dit nu rijden als een mensch}{met gezond verstand}\\

\haiku{de weiden waren,.}{zuur of te droog de akkers}{mager of vochtig}\\

\haiku{Hij leeft op zijne.}{renten en woont alleen met}{eene zijner zusters}\\

\haiku{De koets bleef welhaast.}{voor eene afspanning in het}{voorgeborchte staan}\\

\haiku{Bij de tafel zat.}{Helena met het hoofd op}{eenige papieren}\\

\haiku{Moeder Roosens zou.....}{hare dochter niet gaarne}{veel medegeven}\\

\haiku{Dit had den armen.}{knecht meer dan eens kletsen en}{stompen aangebracht}\\

\haiku{{\textquoteright} De jongeling was:}{opgestaan en morde met}{gebalde vuisten}\\

\haiku{{\textquoteleft}Ik misgrijp mij wel,{\textquoteright},.}{zeker zeide de pachter}{het hoofd schuddende}\\

\haiku{{\textquoteleft}Thomas, Thomas, laat,!}{uw hart vermurwen blijf niet}{onverbiddelijk}\\

\haiku{Ach, hoe kunt gij toch?}{zoo koel het doodelijk verdriet van}{ons arm kind aanzien}\\

\haiku{Is de dood niet daar,?}{om den menschelijken wil}{te verijdelen}\\

\haiku{Denk niet, Urbaan, dat.}{een gevoel van eigenbaat}{mij dus doet spreken}\\

\haiku{{\textquoteleft}Eh, eh, Blaas, Trees, komt,,:!}{geloopen gauw gauw ik mag}{trouwen met Cilia}\\

\haiku{En het is dus waar,,?}{Cilia gij gaat trouwen met}{Urbaan Couterman}\\

\haiku{Zij heeft haar kleed te!}{Brussel laten maken en}{zij weet wat het kost}\\

\haiku{{\textquoteright} {\textquoteleft}Zinnelooze woorden,,{\textquoteright};}{ijdele bedreigingen}{antwoordde Cilia}\\

\haiku{neen, bij den duivel,,!}{die mijne ziel beloert gij}{zijt nog niet getrouwd}\\

\haiku{{\textquoteright} Cilia hield de oogen.}{nedergeslagen en scheen}{van vrees te beven}\\

\haiku{het was, als liep daar.}{een mensch of een wild dier door}{het gebladerte}\\

\haiku{maar hoe hard en hoe,.}{dikwijls zij naar den knecht riep}{zij kreeg geen antwoord}\\

\haiku{Hij verborg zijnen.}{eigen angst om zijne vrouw}{te kunnen troosten}\\

\haiku{Zij zag hem zitten,;}{in den donkeren kerker}{op wat vochtig stroo}\\

\haiku{Maar de bewustheid;}{moest even ras weder in haar}{teruggekeerd zijn}\\

\haiku{de pachter had het.}{hoofd gebogen en scheen door}{wanhoop verpletterd}\\

\haiku{{\textquoteleft}Hum, hum, indien gij.}{maar de zaak niet te veel naar}{de eene zijde wringt}\\

\haiku{Zijne oogen waren,;}{rood en het was zichtbaar dat}{hij lang had geweend}\\

\haiku{Niemand vroeg ons, wie.}{van ons beiden den doodelijden}{slag heeft gegeven}\\

\haiku{{\textquoteright} {\textquoteleft}Weet gij dan niet, dat?}{hij verklaart den messteek te}{hebben gegeven}\\

\haiku{Was het niet uw zoon,?}{of waart gij het niet die den}{eersten slag toebracht}\\

\haiku{gij hebt het mij reeds,{\textquoteright},.}{gezegd morde de drossaard}{het hoofd schuddende}\\

\haiku{{\textquoteright} Het meisje vatte.}{eenen bezem en begon de}{kamer te vegen}\\

\haiku{{\textquoteright} {\textquoteleft}Gij, gij zoudt naar het?}{kasteel durven gaan om voor}{Urbaan te spreken}\\

\haiku{maar gelijk alle,.}{goedhartige menschen is}{hij zwak van gemoed}\\

\haiku{{\textquoteright} {\textquoteleft}Neen, in de hand van,.}{schutter Dierkx die ze naar den}{drossaard ging dragen}\\

\haiku{Gedurende eene,:}{wijl liet Karel haar begaan}{doch dan zeide hij}\\

\haiku{Wat hem terughield,.}{was alleen de dubbele}{zelfbeschuldiging}\\

\haiku{Maar, hoe akelig ook,.....}{indien God dit ongeluk}{heeft toegelaten}\\

\haiku{{\textquoteleft}Hoe is de arme!}{pachteresse vermagerd}{op zoo korten tijd}\\

\haiku{Hij wendde zich tot;}{de getuigen en stelde}{hun vele vragen}\\

\haiku{Urbaan, mijn kind, gij,?}{waart het dus niet die Marcus}{de wonde toebracht}\\

\haiku{{\textquoteright} {\textquoteleft}Thomas Couterman,?}{gij hebt de getuigenis}{van uwen knecht gehoord}\\

\haiku{totdat het allengs.....}{op de andere helling}{van het dal wegstierf}\\

\subsection{Uit: Volledige werken 33. Moederliefde. Lambrecht Hensmans}

\haiku{Marian zal mij,.}{den sleutel brengen zoohaast haar}{werk afgedaan is}\\

\haiku{En al wat gij mij,;}{te vragen hebt is zijne}{genade alleen}\\

\haiku{maar welhaast scheen eene;}{klimmende verwondering}{haar aan te grijpen}\\

\haiku{{\textquoteright} {\textquoteleft}Welaan, ik ga u.}{de reden mijner komst te}{Orsdael verklaren}\\

\haiku{{\textquoteright} gilde de boerin,.}{zich verstomd op haren stoel}{latende vallen}\\

\haiku{Maar ik zou Orsdael,!}{voor altijd ontvluchten}{om het nooit te zien}\\

\haiku{Er zijn misschien nog,?}{meer mevrouwen die den naam}{van Bruinsteen dragen}\\

\haiku{En na eene wijl te,:}{hebben gezwegen zeide}{zij op blijden toon}\\

\haiku{de nood dwingt mij tot.}{het zoeken eener plaats bij}{deftige lieden}\\

\haiku{Van Bruinsteen niets te?}{zien heeft in de aanvaarding}{harer dienstboden}\\

\haiku{Zij heeft u misschien,?}{aangegrijnsd gelijk zij het}{mij gewoonlijk doet}\\

\haiku{ik ben toch altijd.}{hier om u voor te staan en}{u te beschermen}\\

\haiku{Ik heb meenen te,;}{zien dat uw vertrouwen hem}{min of meer trotsch maakt}\\

\haiku{Van Bruinsteen met eene.}{grijns van verwondering en}{ontevredenheid}\\

\haiku{Eene lange wijl bleef;}{zij dus met uitge rekten}{hals luisterend staan}\\

\haiku{Terwijl de jonkvrouw,}{dus denkend voor den spiegel}{stond meende zij eenen}\\

\haiku{Zij legde zich de,:}{hand aan het voorhoofd slaakte}{eenen kreet en zuchtte}\\

\haiku{maar zij hoorde den.}{sleutel in het slot draaien}{en richtte zich op}\\

\haiku{maar alsdan waren,;}{er nog dienstboden die met}{mij mochten spreken}\\

\haiku{{\textquoteleft}Neen, zoo kan hij mij,.}{niet beminnen hij is schoon}{en ontzagwekkend}\\

\haiku{Houd u stil, en, kwam,.}{er iemand vergeet niet wat}{ik u heb gezegd}\\

\haiku{Daarin ligt echter.}{de uitlegging van den haat}{der gravin voor haar}\\

\haiku{{\textquoteright} {\textquoteleft}Ik moet wachten en,.}{zien welke middelen zich}{kunnen aanbieden}\\

\haiku{die ledigheid in,.....}{mijn hart als uw aangezicht}{mij niet tegenstraalt}\\

\haiku{Ik ben gekomen:}{om u eene gelukkige}{tijding te brengen}\\

\haiku{Het is reeds een kwart,.}{uurs dat ik geweld doe om}{haar te doen knielen}\\

\haiku{Ik ben geene vrouw om.}{mij door eenen knecht dagelijks}{te laten hoonen}\\

\haiku{Het is beter hem.}{te bedriegen dan hem tot}{vijand te hebben}\\

\haiku{Ik durfde vragen,,.....}{Martha dat God u mijne}{moeder liete zijn}\\

\haiku{{\textquoteright} Met eene ernstige:}{uitdrukking op het gelaat}{sprak de weduwe}\\

\haiku{Gij weet, dat ik op.}{vele uwer vragen niet mag}{of kan antwoorden}\\

\haiku{Ook glinsterde op.}{zijne borst de star der eer}{en der dapperheid}\\

\haiku{Ik wist, dat Hector;}{soldaat geworden was om}{mij te verdienen}\\

\haiku{Hij heeft mij nog meer;}{dan eens van den Leeuw zonder}{manen gesproken}\\

\haiku{mijn Hector bleef tot;}{den avond ongedeerd en vocht}{als een ware leeuw}\\

\haiku{geheimen, waarvan.}{gij nogtans de uitlegging}{eens zult bekomen}\\

\haiku{Ik weet wel, hoe hem.}{eenen afkeer van Helena}{in te boezemen}\\

\haiku{Het is een gezicht,;}{dat mij in mijnen slaap moet}{gemarteld hebben}\\

\haiku{het overige hangt.}{af van uw verstand en van}{uwe behendigheid}\\

\haiku{gij zoudt Helena.}{ongelukkig maken en}{u zelven met haar}\\

\haiku{Het is met moeite,.}{dat zij zich nog gewaardigt}{mij te antwoorden}\\

\haiku{het was zichtbaar, dat;}{zij tegen een smartelijk}{gepeins worstelde}\\

\haiku{{\textquoteleft}Hem laten gelooven,?}{dat ik toestem om zijne}{vrouw te  worden}\\

\haiku{Gansch roerloos bleef zij,;}{staan totdat zij overtuigd was}{van haren misgreep}\\

\haiku{Leen haar de kracht om!}{de stem harer benauwde}{ziel te verstikken}\\

\haiku{nochtans gij kent al.}{de uitgestrektheid mijner}{genegenheid niet}\\

\haiku{Mathijs aanschouwde.}{haar met eene uitdrukking van}{fierheid en geluk}\\

\haiku{{\textquoteright} {\textquoteleft}Waarom verbergt gij?}{mij dan deze reden zoo}{onverbiddelijk}\\

\haiku{maar wees zeker, dat,,.....}{het waar is want Martha zegt}{het en wat zij zegt}\\

\haiku{Deze laatste, toen,:}{hij genaderd was riep met}{grove barschheid}\\

\haiku{Wil zij wraak nemen,.}{over het gebeurde het zij}{dan op mij alleen}\\

\haiku{Wat kunt gij daaraan?}{doen dat die hooze Frederik}{onverwachts verschijnt}\\

\haiku{Zoudt gij wel gelooven,?}{dat zij in het geheel aan}{zich zelve niet denkt}\\

\haiku{mijne moeder heeft,.}{ons beiden vergiffenis}{geschonken zegt gij}\\

\haiku{Gij zult edelmoedig?}{genoeg zijn om mij uwe gunst}{terug te schenken}\\

\haiku{Ik zal zijne komst.}{afspieden en tot hem gaan}{op zijne kamer}\\

\haiku{{\textquoteleft}Ik doe al wat mij,.}{mogelijk is en gij schijnt}{nog niet tevreden}\\

\haiku{Zij zou zich zelve?}{dus in gevaar brengen om}{mij te verderven}\\

\haiku{maar door den dood van.}{dat kind ontsnapte haar het}{fortuin van den graaf}\\

\haiku{Omdat gij een schrift,?}{bezit dat met hare hand}{is onderteekend}\\

\haiku{{\textquoteright} {\textquoteleft}Het ligt daar in dien,{\textquoteright}.}{ijzeren koffer zeide}{de gouvernante}\\

\haiku{{\textquoteright} Toen de jager in,:}{de kamer getreden was}{zeide de boerin}\\

\haiku{Dries-Jan stiet de kolf:}{van zijn geweer zachtjes ten}{gronde en zeide}\\

\haiku{Ik wil, indien het,.}{mogelijk is hem heden}{nog zien en spreken}\\

\haiku{{\textquoteright} {\textquoteleft}Haar vader is dood,{\textquoteright}, {\textquoteleft}.}{antwoordde Marthamaar zij}{heeft nog eene moeder}\\

\haiku{die tijding, mocht zij,.}{waar zijn is gewichtiger}{dan gij kunt denken}\\

\haiku{Ook de weduwe.}{en Frederik weenden van}{blijde ontroering}\\

\haiku{Hij opende de deur.}{van Martha's kamer en sloeg}{den blik op het bed}\\

\haiku{Bleek en bevend stak.}{hij den anderen sleutel}{op de tweede deur}\\

\haiku{Hij zag de jonkvrouw.}{op eenen stoel in het diepe}{der kamer zitten}\\

\haiku{{\textquoteright} De notaris greep:}{haar de hand en zeide met}{sidderende stem}\\

\haiku{Verheug u, mijn kind,.}{gij wordt de gelukkige}{bruid van M. Bergmans}\\

\haiku{- daar beneden aan.}{mijne voeten bruisten de}{golven van den vloed}\\

\haiku{Ligt er misschien over?}{zijn leven een sluier van}{onbekende smart}\\

\haiku{maar eene akelige.}{verschijning nagelde haar}{vast op haren stoel}\\

\haiku{men doeme mijne,;}{onnoozele zusters als het}{kroost van eenen booswicht}\\

\haiku{Willem had lang en.}{stilzwijgend op de stem van}{Klara geluisterd}\\

\haiku{zij storte eenen traan, -, -!}{over het lot mijns vaders en}{verder niets niets meer}\\

\haiku{mijne liefde zal, -!}{eeuwig zijn ik heb het aan}{God zelven beloofd}\\

\haiku{Toen zij weder in,:}{boozen klap zich uitliet snauwde}{een arm wijf haar toe}\\

\haiku{Daar alleen is nog!}{hoop op rechtvaardigheid en}{op herstelling}\\

\haiku{IJselijk moest in;}{zulke geestgesteltenis}{hem het leven zijn}\\

\haiku{mijne toekomst op,;}{aarde is als eene kalme}{doodsche zeevlakte}\\

\haiku{Ik vergeef u dit, -.}{niet altijd dezelfde moet}{gij voor mij blijven}\\

\haiku{ik trap met voeten:}{wat daar nog van mijn vorig}{leven in overbleef}\\

\haiku{- daar, tegen den muur,.....}{stond de arme vrouw lachend}{haar kind te kussen}\\

\haiku{In het eerst wilde;}{de moeder zich het kind niet}{laten ontnemen}\\

\haiku{Zij wekte bevend,:}{haren man en als deze}{opsprong met den roep}\\

\haiku{de som was te groot,.}{en hij had op aarde geene}{andere vrienden}\\

\haiku{- {\textquoteleft}Ik ben een dief en,,;}{een schelm dit is waar ik heb}{mijne straf verdiend}\\

\haiku{want Willem zal gaan,.}{te huis komen en gij moet}{uwe les nog overzien}\\

\haiku{{\textquoteleft}Moeder lief, wil ik,?}{u eens iets zeggen dat u}{blijde maken zal}\\

\haiku{{\textquoteright} De vrouw scheen ditmaal:}{zeer goed te beseffen en}{sprak op blijden toon}\\

\haiku{{\textquoteright} Maar de vrouw vatte:}{hem bij de hand en sprak op}{zonderlingen toon}\\

\haiku{Het zijn niet alleen,.....}{tranen van blijdschap die over}{uwe wangen stroomen}\\

\haiku{Reeds had eene zoete;}{samenspraak de stille taal}{der oogen vervangen}\\

\haiku{De Besteek heeft in.}{Brabant immer plaats op den}{avond v\'o\'or den naamdag}\\

\subsection{Uit: Volledige werken 34. Levenslust}

\haiku{Ha, een gansch jaar van,.}{arbeid zorg en worsteling}{is weder voorbij}\\

\haiku{Was het de juffer,?}{of de norsche grijsaard die}{hem dus bespiedde}\\

\haiku{Nu zien wij slechts een.}{gedeelte der Alpen van}{het Berner Oberland}\\

\haiku{Deze laatste is.}{de witste berg van allen en}{een van de hoogsten}\\

\haiku{Ik gevoel mij hoogst.}{gelukkig aan zijnen wensch}{te mogen voldoen}\\

\haiku{Zij hoorden wel, dat;}{eene groote menigte menschen}{zich daar bevonden}\\

\haiku{{\textquoteright} Opvolgens werden;}{hun verschillige soorten}{van visch voorgediend}\\

\haiku{Uwe verbeelding droomt;}{en dweept onophoudelijk}{en laat u geene rust}\\

\haiku{{\textquoteleft}zij is waarschijnlijk.}{uitgegaan om in de stad}{eene boodschap te doen}\\

\haiku{{\textquoteright} {\textquoteleft}Neen, ditmaal misgrijpt{\textquoteright},,.}{gij u wedersprak Herman}{het hoofd schuddende}\\

\haiku{Mijne trefbare.}{verbeelding is de eenige}{reden daarvan niet}\\

\haiku{Hun voet baadt in de!}{blauwe zee en hunne}{kruin raakt den hemel}\\

\haiku{Herman meende naar;}{het roer te gaan om zijnen}{vriend te vervoegen}\\

\haiku{{\textquoteleft}Laat ons vergeten,.}{dat de bleeke juffer aan boord}{van de stoomboot is}\\

\haiku{onder anderen,...}{den Rothhorn Sigriswyl}{meer dan 7,000 voet hoog}\\

\haiku{Wij zullen den Rus:}{achternarijden en hem}{in het oog houden}\\

\haiku{Dit gezicht bracht eenen;}{plotselijken omkeer in}{zijn gemoed te weeg}\\

\haiku{eene kleur als die van.}{zekere parelen of}{van melkachtig glas}\\

\haiku{Ik verwar mij in;}{mijne neuswijzerij als}{in eene klis garen}\\

\haiku{Wat ik van den Rus,.}{zeg is enkel scherts om een}{beetje te lachen}\\

\haiku{Hij is een tooveraar,,.}{Herman en heeft zich voor ons}{onzichtbaar gemaakt}\\

\haiku{{\textquoteright} {\textquoteleft}Maar wat wilt gij toch?}{op die naakte rotsen gaan}{doen zonder leidsman}\\

\haiku{{\textquoteright} {\textquoteleft}Ik zwem in mijne{\textquoteright},.}{kleederen antwoordde de}{jonge advocaat}\\

\haiku{Het gerucht van den!}{piassenden regen doet}{zoo zalig slapen}\\

\haiku{{\textquoteright} {\textquoteleft}Het is eene zieke,;}{waarover men ons veel spreekt in}{onze studi\"en}\\

\haiku{{\textquoteright} {\textquoteleft}Neen, Max, ik bid u,,.}{stoor het stil genot niet dat}{mijne ziel overstroomt}\\

\haiku{Ik geloof waarlijk.}{dat gij het ditmaal doet om}{met mij te spotten}\\

\haiku{{\textquoteleft}Planken van boven,,;}{planken van onder planken}{van wederzijde}\\

\haiku{Wij zullen morgen.}{tijd genoeg hebben om}{den ijsberg te zien}\\

\haiku{Van beschrijving tot.}{beschrijving rekte zijn brief}{zich uitermate}\\

\haiku{{\textquoteleft}Dat is te zeggen{\textquoteright},, {\textquoteleft}.}{verbeterde Maxindien}{er geen gevaar is}\\

\haiku{{\textquoteright} {\textquoteleft}Neen, heeren, met eenen.}{goeden leidsman is er geen}{het minste gevaar}\\

\haiku{{\textquoteleft}Volgt mij, en gevoelt,.}{gij u vermoeid wij zullen}{onderweg rusten}\\

\haiku{{\textquoteright} Zij begonnen de;}{bestijging met goeden moed}{en spraken niet veel}\\

\haiku{Tot het dragen van;}{zulken zetel behoeven}{vier sterke mannen}\\

\haiku{Ik heb tot nu toe,;}{halvelings geloofd dat dit}{mogelijk kon zijn}\\

\haiku{{\textquoteright} {\textquoteleft}In den winter woon,{\textquoteright},.}{ik beneden in het dorp}{heer antwoordde zij}\\

\haiku{{\textquoteright} {\textquoteleft}Daaraan zijn wij van{\textquoteright},.}{onze kindsheid af gewend}{antwoordde de vrouw}\\

\haiku{geen kruid, geen vogel,.}{geen levend wezen meer in}{deze wildernis}\\

\haiku{{\textquoteleft}Staan wij hier boven?}{een grondeloos water op}{eene dunne ijskorst}\\

\haiku{{\textquoteleft}Breek ik onderweg,!}{zonder leidsman den hals zoo}{zal het uwe schuld zijn}\\

\haiku{Waarschijnlijk had de:}{ontsteltenis van Herman}{eene dubbele bron}\\

\haiku{{\textquoteright} {\textquoteleft}Gij toch, vriendje, gaat,?}{niet meer ter school vermits gij}{gids of leidsman zijt}\\

\haiku{Gij zult gaan hooren.}{dat hij hem misschien sedert}{zes maanden bezit}\\

\haiku{alle betrekking.}{tusschen hem en ons is voor}{altijd verbroken}\\

\haiku{{\textquoteleft}De vader der bruid;}{moet bij het huwelijksfeest}{tegenwoordig zijn}\\

\haiku{{\textquoteleft}Ziet ginder, heeren,{\textquoteright},.}{het hotel op den Faulhorn}{zeide de jongen}\\

\haiku{Ik zou wel willen{\textquoteright},, {\textquoteleft}}{weten zeide Hermanwie}{van ons beiden den}\\

\haiku{{\textquoteright} Max maakte eenen sprong,.}{als hadde hij lust om den}{waard te omhelzen}\\

\haiku{Het moet insgelijks.}{door de menschen op den berg}{gedragen worden}\\

\haiku{Deze ijswereld?}{zal dus dood blijven tot het}{einde der eeuwen}\\

\haiku{Ik zou niet gaarne.}{eene tweede maal door de zon}{gebraden worden}\\

\haiku{Indien wij te zes,.}{uren niet op weg zijn begaan}{wij eene groote domheid}\\

\haiku{{\textquoteleft}Vreest niet, heeren{\textquoteright}, sprak, {\textquoteleft},;}{de leidsmande weg is wel}{moeilijk inderdaad}\\

\haiku{want het is hier te.}{vochtig en te koud om er}{lang stil te blijven}\\

\haiku{Wij hebben geenen tijd.}{om ons hier met kluchten te}{blijven vermaken}\\

\haiku{{\textquoteright} {\textquoteleft}Vergeef mij deze{\textquoteright}, {\textquoteleft}:}{opmerking zeide Max.In}{mij ontstaat twijfel}\\

\haiku{misschien zijt gij zelf,,.}{zonder het te weten door}{den schijn bedrogen}\\

\haiku{Geloof mij, wat er,:}{ook geschiede het arme}{kind is veroordeeld}\\

\haiku{Spot niet met dien wensch,}{lach niet met dat inzicht of}{ik verwittig u}\\

\haiku{Max Rapelings had;}{zelfs eene samenspraak met de}{meisjes begonnen}\\

\haiku{{\textquoteright} De jonge dokter:}{verschrikte op zijne beurt}{en zeide zuchtend}\\

\haiku{Iedereen heeft zoo.}{zijne oogenblikken van}{geestafdwaling}\\

\haiku{Ik heb geenen tijd om.}{mij langer met deze zaak}{bezig te houden}\\

\haiku{Komt, stijgen wij in}{de koets en denken wij dat}{het nog beter is}\\

\haiku{Terwijl zij aan het,:}{eten waren zeide Max tot}{den tafeldiener}\\

\haiku{{\textquoteleft}Vergeef mij het erg.}{verdenken dat ik tegen}{u had opgevat}\\

\haiku{Wij gingen dikwijls,.}{naar Gent op de feesten der}{groote maatschappijen}\\

\haiku{{\textquoteright} {\textquoteleft}Hebt gij gedwaald, heer,{\textquoteright}, {\textquoteleft}}{dan was het slechts door te veel}{liefde zeide Max.}\\

\haiku{het denkbeeld alleen.}{van zulk voornemen zou haar}{doen terugschrikken}\\

\haiku{een twijfelachtig,.}{woord en gij doet mijn ontwerp}{in duigen storten}\\

\haiku{Rus, den gewaanden?}{verdrukker eener zieke}{maagd had toegewijd}\\

\haiku{Het is uwe maag, uwe.}{maag alleen die ontsteld of}{liever verzwakt is}\\

\haiku{Gave God dat ik;}{nooit ergere gevallen}{onder handen kreeg}\\

\haiku{{\textquoteright} {\textquoteleft}Ja, dan valt er niets,{\textquoteright}.}{meer te zeggen schertste de}{jonge advocaat}\\

\haiku{{\textquoteleft}Maar, oom lief, gij hadt}{besloten dat wij morgen}{zouden vertrekken}\\

\haiku{Er staat op een uur.}{en een half van de kruin des}{bergs een groot hotel}\\

\haiku{Ja, ja, tenzij gij,.}{zeer sterk zijt zult gij het kwaad}{hebben tegen mij}\\

\haiku{die zang, zoo vol ziel,:}{en vol gevoel heeft u de}{zenuwen ontsteld}\\

\haiku{{\textquoteright} {\textquoteleft}Ja, het moet wel zijn;}{dat uwe verbeelding weder}{op hol was geraakt}\\

\haiku{{\textquoteright} {\textquoteleft}Ah, dokter, gij hebt!}{mij de hoop op genezing}{in het hart gestort}\\

\haiku{Allen, behalve,;}{Herman slaakten eenen kreet van}{blijde verrassing}\\

\haiku{Laat ons nu onze,.}{reis voortzetten alsof er}{niets geschied ware}\\

\haiku{het was hier wel wat,.}{koud doch men kon wandelen}{en zich verwarmen}\\

\haiku{Gaan denken dat gij!}{mejuffer Halewijn hier}{op den Rigi ziet}\\

\haiku{zijne klagende.}{tonen hergalmden boven}{den Rigi-kulm}\\

\haiku{met zijne witte.}{bakkebaarden en zijne}{bruine paletot}\\

\haiku{Van dan af is zij.}{moedeloos gebleven en}{heeft zich ziek gevoeld}\\

\haiku{Dit troostend geloof,,.}{heeft in uwe afwezigheid}{haar gansch verlaten}\\

\haiku{ik begrijp het niet,}{en evenwel dank ik God voor}{den zoeten troost dien}\\

\haiku{{\textquoteright} Max Rapelings trok;}{eenen stoel bij en zette zich}{nevens Florentia}\\

\haiku{{\textquoteright} {\textquoteleft}Ik zou altijd met.}{u willen blijven en u}{nooit meer verlaten}\\

\haiku{Zij zouden dus te;}{zamen naar Fluelen en}{naar Amsteg reizen}\\

\haiku{{\textquoteright} {\textquoteleft}Op het tijdstip dat?}{eene doodelijke ziekte haar in}{dat gesticht overviel}\\

\haiku{Was het de wil van?}{God een derde wezen in}{dien band te sluiten}\\

\haiku{want zoo vernietigt.}{men zelfs de mogelijkheid}{der opoffering}\\

\haiku{Uwe opoffering,?}{wat ware zij anders dan}{een droeve zelfmoord}\\

\haiku{Dan zullen zij niets.}{zoo stil kunnen zeggen dat}{wij het niet hooren}\\

\haiku{Zie, wat beteekent toch?}{dat zonderling houten kruis}{tegen de rots daar}\\

\haiku{Het fiere hart van.}{Willem Tell weigerde dien}{smaad te onderstaan}\\

\haiku{De jongelieden.}{kregen elk eene kamer op}{een hooger verdiep}\\

\haiku{nog drie boogschoten.}{hooger en wij waren}{op de kruin des bergs}\\

\haiku{Ook luisterde hij;}{met gespannen aandacht op}{des dokters woorden}\\

\haiku{Daar voor mij staat de.}{bleeke schimme mijner zuster}{Frederika}\\

\haiku{maar ik ben bijna.}{zeker dat wij ons beiden}{hebben bedrogen}\\

\haiku{{\textquoteright} Herman reikte haar.}{den handschoen met zichtbare}{onverschilligheid}\\

\haiku{Hij deed nog eenige...}{stappen vooruit en beklom}{eene kleine hoogte}\\

\haiku{Zij zendt mij om te.}{vernemen hoe het met den}{heer Van Borgstal gaat}\\

\haiku{Zijn aangezicht was.}{bleek en getuigde van eene}{diepe verschriktheid}\\

\haiku{{\textquoteright} {\textquoteleft}Neen{\textquoteright}, antwoordde de, {\textquoteleft};}{burgeronze gemeente}{is daartoe te klein}\\

\haiku{{\textquoteright} {\textquoteleft}Ja, het arme kind,.}{ziet er diep bedrukt uit zij}{moet geweend hebben}\\

\haiku{{\textquoteleft}O, heer professor,!}{hoe dankbaar ben ik u voor}{uwe gedienstigheid}\\

\haiku{Gij zult hem den arm?}{houden en het bloed in die}{waschkom ontvangen}\\

\haiku{{\textquoteright} sprak M. Halewijn,.}{in zich zelven verschrikt van}{den stoel opstaande}\\

\haiku{{\textquoteleft}Ah, ja, ja, daar is,!}{hij daar is de engel die}{mij heeft genezen}\\

\haiku{De gelukkigste,!}{van allen zijt gij niet het}{is de arme Max}\\

\subsection{Uit: Volledige werken 35. De koopman van Antwerpen}

\haiku{ik drijf koophandel,.}{ik bemin den vrede en}{daarmede gedaan}\\

\haiku{{\textquoteleft}De koffie schijnt van.}{hare levendigheid te}{hebben verloren}\\

\haiku{wees rechtzinniglijk,.}{bedankt voor den troost dien gij}{mij hebt gegeven}\\

\haiku{hij schudde zelfs eens;}{het hoofd met eene uitdrukking}{van bekommernis}\\

\haiku{{\textquoteleft}Zeker, mijnheer, het,.}{is zijn portret dat gij mij}{hebt afgeschilderd}\\

\haiku{hij was nieuwsgierig,.....}{om te weten hoe ik te}{Brussel zou varen}\\

\haiku{{\textquoteleft}Daarenboven, ik,;}{heb u iets te melden dat}{u vermaak zal doen}\\

\haiku{{\textquoteright} De koopman sloeg den.}{blik ten gronde en overwoog}{eene wijl in stilte}\\

\haiku{Het verandert uwen,.}{toestand in de wereld mijn}{goede Rapha\"el}\\

\haiku{Ik wil spoedig rijk,,.....}{zijn op eenige jaren en}{gelukt mij dit niet}\\

\haiku{Maak de mat van de.}{mande los en begiet de}{bloemen voorzichtig}\\

\haiku{maar eindelijk, bij,.}{het verlaten van een hoog}{schaarbosch zag hij Mev}\\

\haiku{{\textquoteleft}Arme Verboord, hij.}{is altijd vol zorgen en}{bekommernissen}\\

\haiku{Ik zou welhaast gaan,.}{denken dat mijne woorden}{u verdriet aandoen}\\

\haiku{Er was er een, die.}{schier van den ganschen avond mij}{niet heeft verlaten}\\

\haiku{Het is reeds zoolang,.}{dat zij dit feest met vroolijk}{ongeduld afwacht}\\

\haiku{{\textquoteright} {\textquoteleft}En dan, mijnheer, het.}{is dezen avond feest in de}{groote Harmonie}\\

\haiku{Het middagmaal gaat.....{\textquoteright} {\textquoteleft};}{onmiddellijk opgediend}{wordenHet is zoo}\\

\haiku{maar nu zag hij, dat.}{de koopman hem een teeken}{van ongeduld deed}\\

\haiku{{\textquoteleft}Verboord, mijn vriend, gij.}{handelt noch rechtvaardig noch}{redelijk met ons}\\

\haiku{{\textquoteleft}Mevrouw weet wel, dat.}{mijnheer sedert drie maanden}{zeer zwaarmoedig is}\\

\haiku{Het is niet enkel,?}{om ons te troosten dat gij}{zulk vertrouwen toont}\\

\haiku{Zij alleen, en ik!}{zou het kunnen verzwijgen}{voor gansch de wereld}\\

\haiku{De prijsverhooging?}{van de koffie zouden zij}{toch moeten kennen}\\

\haiku{{\textquoteright} {\textquoteleft}Zijn de prijzen dan?}{eensklaps tot eenen gunstigen}{koers geklommen}\\

\haiku{Ik kom u spreken,,.}{van stoffelijke zaken}{van geld van fondsen}\\

\haiku{Mijne zaken zijn.....{\textquoteright} {\textquoteleft}?}{een beetje in de warUwe}{zaken in de war}\\

\haiku{- Welnu, ik bedank.}{u uiterharte voor dit}{bewijs uwer vriendschap}\\

\haiku{Gij kunt het,{\textquoteright} was het, {\textquoteleft}.}{antwoorden ik twijfel niet}{of gij zult het doen}\\

\haiku{Mijn huisgezin zou?}{door banden des bloeds aan het}{uwe worden gehecht}\\

\haiku{zij scheen zeer wel te,.}{begrijpen doch aarzelde}{nog in haar geloof}\\

\haiku{Felicita, mijn,.}{kind binnen eene maand zal men}{u groeten als Mev}\\

\haiku{{\textquoteright} Hij was reeds buiten,.}{de deur der kamer toen hij}{deze woorden sprak}\\

\haiku{Nu keerde hij zich.}{geheel om en richtte zich}{met haast naar de poort}\\

\haiku{{\textquoteright} Dit grievend gepeins,:}{deed hem rechtspringen terwijl}{hij nog herhaalde}\\

\haiku{Vooronderstel, dat.}{dit huwelijk inderdaad}{voltrokken worde}\\

\haiku{Ware ik in uwe,.}{plaats ik zou mij wreken door}{een diep misprijzen}\\

\haiku{Mijn vurigste wensch.}{was met hem voor altijd te}{worden vereenigd}\\

\haiku{Ik was gelukkig,.}{nochtans en niettemin ik}{schrikte en weende}\\

\haiku{Met mijne hand te,.}{aanvaarden verkreeg hij zulk}{kapitaal ruimschoots}\\

\haiku{Heeft het gebed u?}{dan niet versterkt tegen uwe}{ongegronde vrees}\\

\haiku{maar, wees zeker, v\'o\'or.}{dezen dag heb ik er nooit}{iets van geweten}\\

\haiku{maar dewijl men van,.}{buiten op de deur klopte}{bleef hij gezeten}\\

\haiku{Men zou zeggen, dat.}{gij mager geworden zijt}{sedert gisteren}\\

\haiku{Hij vatte de hand,:}{des jongelings trok hem naar}{een stoel en zeide}\\

\haiku{{\textquoteleft}Zij  heeft sterke;}{hoofdpijn en is voor een uur}{of twee gaan rusten}\\

\haiku{{\textquoteleft}Ik moet mij met spoed,.}{gereedmaken om naar M.}{Dorneval te gaan}\\

\haiku{Ach, ik smeek u, doe!}{geene moeite om mijn besluit}{te veranderen}\\

\haiku{Gij zult hem zeggen,.}{dat ik hem onmiddellijk}{verlang te spreken}\\

\haiku{Nu echter kan ik,.}{in veiligheid eene maand ja}{zelfs langer wachten}\\

\haiku{want zijn groet was schier,.}{onhoorbaar en zijn gelaat}{bleef ernstig en koel}\\

\haiku{Laat hooren, welk is?}{dit schrikkelijk geheim dat}{ik niet weten mag}\\

\haiku{{\textquoteright} M. Verboord sprong met;}{de beide handen in het}{haar achteruit}\\

\haiku{Dat ik arm worde,;}{maar dat ik toch eer moge}{doen aan mijn handteeken}\\

\haiku{{\textquoteright} De deur der zaal werd.}{onverwachts geopend na}{eenen haastigen klop}\\

\haiku{M. Dooms, de dokter,;}{van het dorp was redelijk}{in zijne eischen}\\

\haiku{maar zouden wij wel}{kunnen winnen wat er noodig}{is om den dokter}\\

\haiku{Walput genoeg liet,.}{gevoelen dat wij hare}{komst niet verlangden}\\

\haiku{Ik heb haar gezegd,.}{dat vader ziekelijk is}{en niemand wil zien}\\

\haiku{want Felicita:}{zeide met verdoofde stem}{tot hare moeder}\\

\haiku{{\textquoteleft}Felicita, hoe!}{zuiver is uw hart en hoe}{innig uwe liefde}\\

\haiku{Toen hij gevonden,:}{had wat hij zocht murmelde}{hij in zich zelven}\\

\haiku{{\textquoteright} {\textquoteleft}Maar, goede heer Banks,?}{weet gij dan niet meer wat wij}{u verschuldigd zijn}\\

\haiku{Sedert dan heeft de:}{fortuin niet opgehouden}{ons toe te lachen}\\

\haiku{Een mijner makkers,.....}{bleef er dood en ik kreeg eenen}{kogel door den arm}\\

\haiku{{\textquoteright} Walput schudde het.}{hoofd met verwondering en}{ontevredenheid}\\

\haiku{Hoe onverdelgbaar!}{is toch een ingeworteld}{gevoel in den mensch}\\

\haiku{Het was klaarblijkend,;}{voor hem dat de Verboords niet}{geheel arm waren}\\

\haiku{want anders zouden.}{zij dat rijke huisraad niet}{hebben behouden}\\

\haiku{Ik zal u zeggen.}{waarom hij van Amerika}{is wedergekeerd}\\

\haiku{Zal ik het gezicht?}{van den ondankbare wel}{kunnen verdragen}\\

\haiku{Mijn vader is ziek,.}{en zijne zenuwen zijn}{uiterst ontstelbaar}\\

\haiku{{\textquoteright} {\textquoteleft}Neen, neen, verlies den,{\textquoteright}, {\textquoteleft};}{moed niet zeide zijalles}{staat ten gunstigste}\\

\haiku{ik bid u, ik smeek,.}{u heb toch eenige deernis}{met mijn droevig lot}\\

\haiku{Wel had men nog niets;}{van hem gehoord sedert zijn}{vertrek naar Parijs}\\

\haiku{hij zal ontwaken.}{en allengs geheel uit de}{bezwijming opstaan}\\

\haiku{Verboord greep zijne:}{hand en zeide met diepe}{treurnis in de stem}\\

\haiku{Gij hebt gezien hoe.}{noodlottig die naam nog op}{zijne zinnen werkt}\\

\haiku{Zoeken wij vrede.}{en rust in de uitstorting}{onzer dankbaarheid}\\

\haiku{En zou daarom zijn?}{eindelooze edelmoed zonder}{belooning blijven}\\

\haiku{wierden de schulden,.}{vereffend dan moest hij het}{onfeilbaar weten}\\

\haiku{{\textquoteright} riep Felicita,.}{die hare ontsteltenis}{niet kon bedwingen}\\

\haiku{{\textquoteright} De grijsaard bleef eene;}{wijl met onvasten blik in}{de ruimte staren}\\

\subsection{Uit: Volledige werken 36. Schandevrees. Koning Oriand. Blinde Rosa}

\haiku{Ik geloof het niet,,{\textquoteright},;}{mijnheer zeide de dienstmeid}{het hoofd schuddende}\\

\haiku{maar nu twijfelde,;}{hij niet meer of hij moest zich}{misgrepen hebben}\\

\haiku{De arme man zou.}{naar alle waarschijnlijkheid}{niet lang meer leven}\\

\haiku{Of anders hebt gij.}{hem dien doodsteek wetens en}{willens toegebracht}\\

\haiku{Des anderen daags.}{is hij op reis gegaan met}{de arme jonkvrouw}\\

\haiku{Zij waren weg, en?}{gij hebt hen sedert dan niet}{meer wedergezien}\\

\haiku{Gelooft gij, Homans,?}{dat zij zich in nood zouden}{kunnen bevinden}\\

\haiku{Kan iets, kan geld tot,.}{uw geluk nog bijdragen}{zeg een enkel woord}\\

\haiku{{\textquoteright} De graaf de Hammes:}{greep de beide handen des}{grijsaards en zeide}\\

\haiku{Die teedere bloem,?}{men zou ze doen verstikken}{bij gebrek aan lucht}\\

\haiku{{\textquoteright} {\textquoteleft}Gij weet wel, moeder,,,?}{Jakob de jager die bij}{den Reigerspoel woont}\\

\haiku{Ach, in die kamer,,!}{binnen het somber gebouw}{leefde en leed Ida}\\

\haiku{Dan eerst wendde hij.}{den blik weder naar den kant}{van den Ouden Steen}\\

\haiku{{\textquoteleft}gij, zoo schoon, zoo zoet,, -?}{zoo zuiver sterven in de}{Lente des levens}\\

\haiku{Ons leven zou een,,.}{hemel zijn van vrede van}{blijdschap van liefde}\\

\haiku{Nu hij wist, dat Ida,.}{hem beminde gevoelde}{hij zich reuzensterk}\\

\haiku{{\textquoteright} gromde de oude.}{heer met eene nieuwe vlaag van}{gramschap in de oogen}\\

\haiku{ik moest eene hooge plaats....}{bekleeden in het bestuur}{van mijn vaderland}\\

\haiku{ik sluit haar op in.}{hare kamer en ik draag}{den sleutel op mij}\\

\haiku{Tegen den morgen,,;}{in mijnen slaap zag ik mij}{zeiven in het park}\\

\haiku{Dat zulke aanslag,.}{zou kunnen gelukken hoeft}{gij niet te vreezen}\\

\haiku{Mijn zoon Hugo is.}{vurig verliefd geworden}{op uwe kleindochter}\\

\haiku{{\textquoteleft}Ja, mijnheer, achter,.}{in het park niet verre van}{de groene rustbank}\\

\haiku{{\textquoteright} {\textquoteleft}Ginder tusschen het,.}{kreupelhout liggen de twee}{ladders en de plank}\\

\haiku{O, God, Gij, die weet,!}{hoe zuiver mijn inzicht is}{laat mij gelukken}\\

\haiku{Had zij het briefje,?}{gevonden dat hij over den}{muur geworpen had}\\

\haiku{hij ademde hijgend.}{en bewoog nu en dan de}{leden krampachtig}\\

\haiku{gij zult weten wat,.}{het is eene moeder doodelijk in}{haar kind te treffen}\\

\haiku{Hugo wrong zijne;}{leden krampachtig en scheen}{te willen opstaan}\\

\haiku{zij liet het hoofd op.}{de borst vallen en blikte}{zwijgend ten gronde}\\

\haiku{Kom, Maria, verhef.}{uwen moed in de maat der}{wreedheid van het lot}\\

\haiku{Zouden wij nu als?}{lafaards het hoofd bukken en}{den strijd opgeven}\\

\haiku{{\textquoteright} {\textquoteleft}Gij weet wel, mevrouw.....{\textquoteright} {\textquoteleft},?}{Von WeilerZoo mijnheer kent}{mijnen nieuwen naam}\\

\haiku{{\textquoteright} {\textquoteleft}Gij weet, mevrouw, dat.}{ik slechts de onschuldige}{oorzaak daarvan was}\\

\haiku{Mijne treurige,?}{geschiedenis ontrukt u}{eenen traan Hortensia}\\

\haiku{{\textquoteleft}Mijnheer..... Willem, het,.}{betaamt niet dat wij langer}{te zamen blijven}\\

\haiku{{\textquoteright} De graaf deinste een.}{paar stappen achteruit en}{verbleekte zichtbaar}\\

\haiku{In zijn  hart maakt.}{hij u verantwoordelijk}{voor ons bitter lot}\\

\haiku{{\textquoteright} {\textquoteleft}Vijf minuten voor,{\textquoteright},.}{tienen zeide de graaf op}{zijn uurwerk ziende}\\

\haiku{lief,{\textquoteright} murmelde zij, {\textquoteleft},.....{\textquoteright} {\textquoteleft}?}{wat de graaf mij zeide was}{integendeelHoe}\\

\haiku{De graaf De Hammes?}{was vermetel genoeg om}{ons op te zoeken}\\

\haiku{wel lijdt hij nog pijn,.}{van zijnen val maar hij zit}{reeds in eenen leunstoel}\\

\haiku{Gij zult toch v\'o\'or uw,.....}{huwelijk haar niet zeggen}{wie haar vader is}\\

\haiku{{\textquoteleft}Ja, lieve nicht, ik,{\textquoteright},;}{ben slechts uw oom sprak hij haar}{teeders treelende}\\

\haiku{{\textquoteright} riep de veldwachter.}{met eene soort van angstige}{verontwaardiging}\\

\haiku{{\textquoteright} Door deze laatste,:}{woorden vergramd gebood de}{koning zeer barsch}\\

\haiku{ik meende evenwel.....}{eenen glimlach op uwe lippen}{te hebben verrast}\\

\haiku{De minste glimlach;}{van haar voerde hem op tot}{het hoogste geluk}\\

\haiku{{\textquoteright} {\textquoteleft}Maar hoe zal Oriand,?}{vernemen wat de ridders}{en het volk zeggen}\\

\haiku{Men leide nu de;}{beschuldigden in mijne}{tegenwoordigheid}\\

\haiku{In de troonzaal van:}{het paleis waren vele}{lieden vergaderd}\\

\haiku{kreet, machtig genoeg.}{om op zekeren afstand}{te worden gehoord}\\

\haiku{{\textquoteright} murmelde hij, {\textquoteleft}maar;}{alle hoop op leven is}{voor mij verloren}\\

\haiku{dit glansrijke werk.}{van God is getuige van}{hetgeen hier geschiedt}\\

\haiku{Hij laat zich evenmin.}{uitroeien als het immer}{woekerend onkruid}\\

\haiku{Marcus en eenige.}{anderen stonden zelfs de}{tranen in de oogen}\\

\haiku{Gij, Coenraad, die een,.}{wijs man zijt neem eerst het woord}{en verberg mij niets}\\

\haiku{- uwe moeder, smeltend,.....}{in tranen en bezwijkend}{van smart toonde ons}\\

\haiku{Wassche haar schuldig!}{bloed zooveel mogelijk de}{schandvlek van uwe kroon}\\

\haiku{De koning bleef eene;}{wijl zwijgend en hield den blik}{nedergeslagen}\\

\haiku{{\textquoteright} {\textquoteleft}Indien Savary!}{het kind had gered en ons}{wilde verraden}\\

\haiku{{\textquoteright} {\textquoteleft}Neen, neen, uwe vrees is,{\textquoteright}.}{ongegrond wedersprak hem}{Mattabruna}\\

\haiku{Kon mijn dood zijne,}{ziel de verlorene rust}{terugschenken}\\

\haiku{dit zal mijnen zoon.}{verontwaardigen en hem}{ten uwen gunste stemmen}\\

\haiku{Allen zwegen en;}{hielden met angst het oog naar}{hunnen vorst gericht}\\

\haiku{Maar weder brak de.}{blik van Beatrix zijnen}{wil en zijne macht}\\

\haiku{Eindeloos is mijn,{\textquoteright}.}{medelijden antwoordde}{de  koningin}\\

\haiku{De overtuiging, dat,;}{hij zich niet bedroog stelde}{Helias gerust}\\

\haiku{Zoohaast wij dien voorbij,.}{zijn zien wij mijns vaders kluis}{in het groene dal}\\

\haiku{{\textquoteright} {\textquoteleft}Ik ben een ridder,{\textquoteright}.}{der Tafel-rond antwoordde}{de onbekende}\\

\haiku{{\textquoteright} {\textquoteleft}Ongetwijfeld kent,?}{gij den ongelukkige}{die op uwe hulp wacht}\\

\haiku{{\textquoteleft}Helias, ga naar.}{de weide en let op het}{paard van dezen heer}\\

\haiku{{\textquoteright} {\textquoteleft}Zeker, zeker, men.}{roemt hem als den dappersten}{leenman des keizers}\\

\haiku{De koning stapte.}{aan het hoofd van eenen langen}{stoet uit het paleis}\\

\haiku{{\textquoteright} {\textquoteleft}Arme vorstinne,,{\textquoteright}.}{denk aan uwe moeder zeide}{hare gezellin}\\

\haiku{Ik zal u daarvoor!}{gaarne zien en beminnen}{uit geheel mijn hart}\\

\haiku{Tranen glinsterden,:}{hem in de oogen terwijl hij}{overwonnen zeide}\\

\haiku{- maar het uitwerksel.}{van dit gezegde was niet}{zooals hij had verwacht}\\

\haiku{{\textquoteright} riep hij uit, {\textquoteleft}dat Hij.}{ten minste \'e\'enen mijner}{makkers gespaard heeft}\\

\haiku{{\textquoteleft}God zij gezegend,,!}{dat Hij u zoolang leven}{liet baas Joris}\\

\haiku{Ik belast mij met.}{u in Zijnen heiligen}{naam te beloonen}\\

\haiku{{\textquoteleft}Als  ik Rosa,;}{te bedelen leid dan spreekt}{ze altijd van u}\\

\haiku{Blinde Karel is.....}{besteed op eene hoeve langs}{de kanten van Lier}\\

\haiku{Er staat altijd een.}{bijzonder potteken voor}{haar in de assche}\\

\haiku{De blinde duwde.}{hem evenwel zachtjes met de}{handen van zich weg}\\

\haiku{Overblijfsel van een,!}{leger dat verslagen werd}{door het zwaard des tijds}\\

\subsection{Uit: Volledige werken 38. Jacob van Artevelde}

\haiku{die nijverheid en;}{handel opvoerde tot den}{hoogsten top van bloei}\\

\haiku{hij reisde, welke.}{last hem was opgelegd en}{wat hij uitvoerde}\\

\haiku{Op het verzoek des.}{Wijzen Mans vloog hier alles}{ten zijnen dienste}\\

\haiku{Welaan, gezellen{\textquoteright},, {\textquoteleft}:}{hernam Artevelde met}{koelheidzijt gerust}\\

\haiku{iets dichterlijks, iets;}{kwijnends lag er in zijnen}{langzamen oogslag}\\

\haiku{Die mannen, men zou.}{zeggen dat ze dorst hebben}{naar elkanders bloed}\\

\haiku{men heeft ons gezegd}{dat gij door uwe wijsheid en}{uwe groote goederen}\\

\haiku{daar heeft Graaf Gwijde;}{het schandelijk verdrag van}{Melun bekrachtigd}\\

\haiku{Dit alles kon hem:}{toch niet voor valschheid en}{onrecht behoeden}\\

\haiku{hun bestaan getuigt,.}{van Frankrijks snoode heerschzucht}{niet van onzen plicht}\\

\haiku{Ziet hier, hoe ik meen,:}{dat Vlaanderen vrijheid en}{nering zal hebben}\\

\haiku{maar dat men ook een.}{wapen hebben moet om ze}{te verdedigen}\\

\haiku{{\textquoteleft}Gezellen, wat men,;}{u daar gezegd heeft kan in}{den grond heel schoon zijn}\\

\haiku{Wat zullen wij, bij,?}{voorbeeld doen als onze Graaf}{terug wil komen}\\

\haiku{Zullen wij weder?}{kruipen onder de hand die}{ons gegeeseld heeft}\\

\haiku{ik geloofde een.}{bloedig zwaard te zien dat de}{vrijheid bedreigde}\\

\haiku{Indien hij komt als?}{Graaf van Vlaanderen en niet}{als Frankrijks veldheer}\\

\haiku{Hij zit krachtens recht;}{met de Schepenen van der}{Keure in den Raad}\\

\haiku{hooren hoeverre;}{de menigte op haren}{weg gevorderd was}\\

\haiku{{\textquoteright} {\textquoteleft}Ah, Muggelyn, er.}{moet met voorzichtigheid te}{werk gegaan worden}\\

\haiku{misschien denkt hij dat;}{gij de zaken reeds te diep}{begint in te zien}\\

\haiku{Zij weerhield zich en;}{naderde tot Lieven om}{hem te bedaren}\\

\haiku{Dat komt van in die...}{vervloekte kiezing voor uwen}{vader te loopen}\\

\haiku{Hij zou mij zulken,!}{stoel ten geschenke moeten}{geven uw vader}\\

\haiku{maar het zal er niet, '!}{bij blijven al moest Roeland}{int spel komen}\\

\haiku{Misschien wildet gij;}{Hoofdman in St-Michiels}{gekozen worden}\\

\haiku{voor de maat hunner,:}{kleine driften zij mogen}{laken en schelden}\\

\haiku{maar wat deze op,.}{minluiden toon antwoordde}{verstonden zij niet}\\

\haiku{hij zijnen mantel.}{en zijne huike aan en}{ging de kamer uit}\\

\haiku{als ik het Gentsche}{volk in alle straten hoor}{juichen en zingen}\\

\haiku{de bezetting van;}{Biervliet zal het grondgebied}{van Gent verwoesten}\\

\haiku{En vooronderstel:}{dat wij den eersten aanval}{des vijands afkeeren}\\

\haiku{maar hoe zullen wij?}{met de wacht van St-Jan door}{het volk geraken}\\

\haiku{Van Rupelmonde{\textquoteright},, {\textquoteleft}.}{was het antwoordik moet den}{Opperhoofdman zien}\\

\haiku{maar Artevelde;}{voorzag de mogelijkheid}{van zulken toestand}\\

\haiku{Gewis, de ziel van;}{Geeraart Denys moest door}{vreugde ontroerd zijn}\\

\haiku{de vischverkoopers met;}{hunne gestreepte kolders}{en lange lansen}\\

\haiku{De Overdeken liep;}{met zijne mannen tegen}{de ruiterij in}\\

\haiku{Artevelde had;}{reeds eene lichte wonde aan}{het hoofd bekomen}\\

\haiku{U ziende zou men!}{waarlijk zeggen dat gij de}{wereld hebt verroerd}\\

\haiku{Eindelijk besloot.}{men tot de misdaad zijne}{toevlucht te nemen}\\

\haiku{Laffe honden, die!}{tegen de zon blaft omdat}{haar licht u bedwelmt}\\

\haiku{{\textquoteright} {\textquoteleft}Opperhoofdman, zult,?}{gij over de Vrijdagmarkt gaan}{als gij naar huis keert}\\

\haiku{{\textquoteright} {\textquoteleft}Koude zucht naar de.}{waarheid en warme liefde}{voor mijn vaderland}\\

\haiku{mijn eed verplicht mij, -?}{en mijne graafschappen van}{Rethel en van Nevers}\\

\haiku{Men luistert  zoo.}{lichtelijk als men lijdt en}{ontevreden is}\\

\haiku{maar gij wijst mij daar;}{iets aan dat inderdaad niet}{te verzuimen is}\\

\haiku{- Gent, Brugge, Yperen,,,.}{Kortrijk Audenaerde}{Aelst en Geeraertsberge158}\\

\haiku{{\textquoteright} {\textquoteleft}Ik zou het doen en{\textquoteright},.}{geef daarop mijn ridderwoord}{antwoordde Edward}\\

\haiku{men ging ze nogmaals!}{gebruiken om Vlaanderens}{recht te verkrachten}\\

\haiku{Wees grootmoedig en!}{aanvaard het bestand zooals het}{u wordt voorgesteld}\\

\haiku{{\textquoteleft}Dezen morgen zal.}{het bestand ongetwijfeld}{worden bezegeld}\\

\haiku{- Belachelijke,,!}{spotternij die gij aanvaardt}{als een onnoozel kind}\\

\haiku{hoofde voor zijnen.}{vader en hoorde wellicht}{niet wat hij zeide}\\

\haiku{Sterven misschien als...}{eene bloem welker wortel van}{wormen is doorknaagd}\\

\haiku{Alles zal ijdel;}{worden in zijnen geest en}{in zijnen boezem}\\

\haiku{zijn gefolterde...}{geest stoot dit beeld terug en}{smeekt om genade}\\

\haiku{{\textquoteright} riep de Koning der, {\textquoteleft}!}{Ribauden lachendhet is}{onze vriend Lieven}\\

\haiku{wat gij mij nu ten.}{laste legt meende ik zelf}{u te verwijten}\\

\haiku{zij moge in uwen.}{geest geprent blijven tot het}{einde uwer dagen}\\

\haiku{Ver Artevelde{\textquoteright},.}{antwoordde Ghelnoot met eene}{minzame buiging}\\

\haiku{Heeft hij mijn geheim,.}{veropenbaard hij deed het}{zonder boos inzicht}\\

\haiku{maar ik wilde wel;}{weten welke lastertong}{u dit gezegd heeft}\\

\haiku{dat de koning der}{Ribauden niet afhouden}{zou vooraleer hij}\\

\haiku{Gij begrijpt wel dat:}{ik niet van den beginne}{vooruitspringen kan}\\

\haiku{ik verzoek u, mij;}{uwe gansche aandacht te leenen}{en niet te spotten}\\

\haiku{Daar zij van mij niet.}{weten zullen zij alles}{schreeuwen wat men wil}\\

\haiku{Ik weet het wel, en.}{hoop dat het bij mijne komst}{meest zal gedaan zijn}\\

\haiku{Naarmate de tijd,;}{verliep vermeerderde ook}{de toevloed des volks}\\

\haiku{doch men kon daarin}{geen ander doel vermoeden}{dan het inrichten}\\

\haiku{Ik bid God dat Hij!}{in dit plechtig oogenblik}{uwen geest verlichte}\\

\haiku{Het scheen dat zulks noch,;}{van Steenbeke noch zijne}{aanhangers beviel}\\

\haiku{{\textquoteleft}Neen, neen{\textquoteright}, antwoordde,.}{de Opperhoofdman hem de}{hand aangrijpende}\\

\haiku{maar zij zullen hem,!}{zelven eten hij moge hun}{smaken ofte niet}\\

\haiku{gebruikt is om de;}{Schepenen te bedriegen}{en te verschrikken}\\

\haiku{{\textquoteright} ~ {\textquoteleft}Welaan, vertrekt{\textquoteright},;}{gij maar van hier antwoordde}{Comyne met haast}\\

\haiku{Op dit oogenblik;}{is men bezig met over mijn}{lot te beslissen}\\

\haiku{{\textquoteright} Lieven sprong verschrikt.}{van zijnen stoel op en week}{bevend achteruit}\\

\haiku{Hij verliet alsdan.}{de Nederschelde en klom}{de Brabantstraat op}\\

\haiku{{\textquoteright} De jonge Denys:}{las het schrift over en sprak dan}{met verwondering}\\

\haiku{Gij wilt mij zonder,!}{hulp overleveren aan den}{beul ondankbare}\\

\haiku{{\textquoteright} {\textquoteleft}En zal de Jonkvrouw?}{dan niet herkennen welken}{weg zij gevolgd heeft}\\

\haiku{{\textquoteleft}Ah, Muggelyn, zijt}{gij reeds zoo oud geworden}{zonder te weten}\\

\haiku{Gij wilt weder langs,;}{kromme wegen tot uw doel}{geraken meester}\\

\haiku{Gij gevoelt dat de}{gelegenheid gunstig is}{om mij weder eenen}\\

\haiku{{\textquoteright} {\textquoteleft}Indien gij toch zoo,?}{moedig zijt waarom vermoordt}{gij haar dan zelf niet}\\

\haiku{Het is alsof het...}{haar ingegeven werd om}{mij te martelen}\\

\haiku{Geene beweging er,:}{in bespeurende zeide}{hij met blijdschap}\\

\haiku{belang hebben in?}{het gelooven aan de snoodste}{beschuldigingen}\\

\haiku{maar, Jacob, vriend, wij:}{mogen het zeggen in Gods}{tegenwoordigheid}\\

\haiku{laat mij afstand doen.}{van de ambten die het volk}{mij toevertrouwde}\\

\haiku{Oordeel in volle,;}{vrijheid beslis volgens uwe}{eigene inspraak}\\

\haiku{431.)water gieten?}{op de vlam die mijn vijand}{dreigt te verteren}\\

\haiku{Het is wel waar dat;}{al uwe pogingen op niets}{zullen uitloopen}\\

\haiku{gij maakt het verschiet,;}{wel zwart en veel zou daarop}{te zeggen vallen}\\

\haiku{229{\textquoteright} Geeraart Denys;}{viel eensklaps met gramschap uit}{en wilde spreken}\\

\haiku{Bevlekt u niet door.}{de onderwerping aan de}{verwaande Vollers}\\

\haiku{Ik spreek hier in naam.}{der gansche Weverij wier}{eer ik verdedig}\\

\haiku{{\textquoteright} {\textquoteleft}Neen, neen, vader{\textquoteright}, riep, {\textquoteleft}.}{Lieven met droefheidzeg mij}{zulke dingen niet}\\

\haiku{De Opperhoofdman.}{is zoolang gelasterd en}{vervolgd geworden}\\

\haiku{Gij ziet dus, dat het.}{morgen tijds genoeg is om}{uwe boodschap te doen}\\

\haiku{En hoe gij het ook,.}{beschouwet ik ga te bed}{en gij insgelijks}\\

\haiku{Geen laster loopt er,,,}{in Gent geen haat blaakt er geen}{bloed wordt er gestort}\\

\haiku{u toeroepen dat;}{ik het oogenblik mijner}{geboorte vervloek}\\

\haiku{Ongelooflijk en! -.}{toch waar Gij hebt nog vijftig}{moordenaars in huur}\\

\haiku{de eenige hoop op;}{verzoening met den Vorst moest}{worden verlaten}\\

\haiku{maar een andere.}{eisch kon zoo lichtelijk}{niet voldaan worden}\\

\haiku{Sedert de aankomst.}{der Engelsche vloot was de}{toestand veranderd}\\

\haiku{Hebt gij mij nog iets,?}{te zeggen of durft gij niet}{naar Gent wederkeeren}\\

\haiku{Geloof hem dan ook;}{op dit uur en wijs zijne}{smeekingen niet af}\\

\haiku{Deze verhuizing;}{ontsnapte in het eerst aan}{onze opmerking}\\

\haiku{Wat ik ben, werd ik;}{alleen door de keus mijner}{landgenooten}\\

\haiku{mijne vrienden zijn,.}{reeds naar boven gegaan zij}{brengen mij goed nieuws}\\

\haiku{{\textquoteright} Vrouw Artevelde.}{drukte nog met meer kracht de}{hand haars echtgenoots}\\

\haiku{Allen waren met,,.}{bijlen hamers zwaarden of}{daggen gewapend}\\

\haiku{98(3)Dat dit gebruik}{in Gent bestond is mij door}{den heer professor}\\

\haiku{99(1)Item ghaven zy,...}{meester Arnoude van Leene}{den stede surgien}\\

\haiku{se setten ende...}{spannen in dumysers voor}{scapiteins logyst}\\

\haiku{239{\textquoteleft}Verleenende...}{hem ter bewarenesse}{van zynen persoon}\\

\subsection{Uit: Volledige werken 39. Hlodwig en Clothildis}

\haiku{doch men kon zien, dat.}{hij daaronder een kleed van}{sneeuwwit linnen droeg}\\

\haiku{ik ben bereid mij.}{aan uwe uitspraak met ootmoed}{te onderwerpen}\\

\haiku{{\textquoteleft}Ik moet u nog iets;}{vragen over den toestand der}{zaken in Galli\"e}\\

\haiku{Lederen schoenen.}{waren hem met riemen aan}{de voeten gegespt}\\

\haiku{Hij zou eeuwig zijn,,?}{de band dien de zoete Freya}{niet gevlochten heeft}\\

\haiku{Voor de Wijtafel,;}{lagen bijlen messen en}{hamers van keisteen}\\

\haiku{{\textquoteright} Hierop besproeide.....}{hij de verloofden opnieuw}{met het offerbloed}\\

\haiku{zij twijfelden niet,,.}{of hij was het die tot den}{Maalberg naderde}\\

\haiku{Bij 's Heeren disch.}{waren nu de Scalden of}{dichters vergaderd}\\

\haiku{{\textquoteright} Langzaam voortgaande,;}{hield hij den blik vragend in}{de ruimte gericht}\\

\haiku{Allengs nochtans scheen;}{hij weder in eenen zachten}{droom weg te zinken}\\

\haiku{Eensklaps ontwaakte:}{hij uit zijne mijmering}{en zeide met spijt}\\

\haiku{Eensklaps ontwaart mijn;}{oor een hemelsch geluid van}{zang en snarenspel}\\

\haiku{Mij gewaardigt gij,.....{\textquoteright}}{niet te vragen of de nacht}{mij zacht is geweest}\\

\haiku{- de lucht is hier zoo,,!}{zuiver het lommer zoo frisch}{de natuur zoo mild}\\

\haiku{{\textquoteleft}Mijne dochter, ga;}{en wandel in den tuin met}{uwe gezellinnen}\\

\haiku{De toestand is zeer,.}{bedreigend voor mij men kan}{het niet miskennen}\\

\haiku{Zijne Weermannen;}{zouden tegen hem opstaan}{of hem verlaten}\\

\haiku{Om het gevaar te,?}{bezweren dat mijne eer}{en mijn hart bedreigt}\\

\haiku{Haar vader, hare,,.}{ooms hare broeders gansch haar}{geslacht is Ariaansch}\\

\haiku{maar laat mij sterven,!.....}{met de hoop dat ik eens uwe}{bruid zal  worden}\\

\haiku{Een lach van blijdschap.}{en zegepralenden nijd}{blonk op haar gelaat}\\

\haiku{Deze moest voor hen.}{als toegelaten gezant}{geheiligd blijven}\\

\haiku{Hier geschiedde de.}{beruchte wapendans der}{Noordervolkeren}\\

\haiku{Aurelianus trad,.}{in de tente om op den}{heirtog te wachten}\\

\haiku{Siagrius heeft.}{uwe uitdaging aanvaard en}{zal zijn woord houden}\\

\haiku{Luister en erken.....}{de onmogelijkheid uwer}{dwaze uitzichten}\\

\haiku{{\textquoteright} {\textquoteleft}Misschien bedriegt de,{\textquoteright}.}{koning zich niet bermerkte}{de Gallo-Remein}\\

\haiku{- Wat kan hij op mij,?}{die beschermd ben door al de}{Asen van het Glansheim}\\

\haiku{Hlodwig stond in de.}{laagte met de lieden van}{Doorniker-Gouw}\\

\haiku{- zoo spoedig had het!}{lot de bestemming van dit}{paleis veranderd}\\

\haiku{het heiligst is op,.}{aarde en in den hemel}{heeft er in gerust}\\

\haiku{Volg mij ter markte,;}{waar de buit op den middag}{zal verdeeld worden}\\

\haiku{{\textquoteright} {\textquoteleft}Neen, van mij zult gij,{\textquoteright}.}{het vat heden nog krijgen}{antwoordde Hlodwig}\\

\haiku{zij zijn de Goden, -.}{van ons geslacht niet van het}{zwartharige volk}\\

\haiku{en zeker, het zou.}{de inwoners van Galli\"e}{met recht bedroeven}\\

\haiku{Men bracht het bij den,,.}{Opperbloedman die het eenen}{doek voor de oogen bond}\\

\haiku{dan weder liep zij}{tot den Opperheirtog en}{herhaalde hare}\\

\haiku{{\textquoteright} {\textquoteleft}Maar waartoe zich met?}{zulke schoone uitzichten}{bezig gehouden}\\

\haiku{{\textquoteright} {\textquoteleft}Zij zal ook reeds den?}{Opperheirtog der Franken}{vergeten hebben}\\

\haiku{{\textquoteright} De Gallo-Romein.}{luisterde in verbaasdheid}{op Hlodwigs woorden}\\

\haiku{{\textquoteright} De heirtog wendde;}{zich om en meende tot de}{deur te naderen}\\

\haiku{Zij stak de handen:}{biddend tot hem op en riep}{in Latijnsche taal}\\

\haiku{De grond der straten;}{was overdekt met de lijken}{der weerlooze burgers}\\

\haiku{{\textquoteleft}Lutgardis, lieve,,.}{treur zoo niet om den hoon die}{ons is aangedaan}\\

\haiku{{\textquoteright} Een bittere lach',:}{bewoog Lutgardis lippen}{daar zij antwoordde}\\

\haiku{gij meent nog, dat de?}{dochter Hilperiks schuld heeft}{aan uw ongeluk}\\

\haiku{Elken kamper werd;}{een schild gebracht en aan den}{linkerarm geriemd}\\

\haiku{{\textquoteright} {\textquoteleft}Maar zij is Christin,{\textquoteright}.}{bemerkte een der Franken}{op spijtigen toon}\\

\haiku{{\textquoteright} {\textquoteleft}Machtige koning,{\textquoteright}, {\textquoteleft}}{der Burgonden antwoordde}{de Gallo-Romein}\\

\haiku{Het zijn de eenige,;}{bedreigingen waarvoor de}{koning zwichten kan}\\

\haiku{Zijt gij wel zeker,?}{dat de stomme jongeling}{Hilperiks zoon was}\\

\haiku{Neem uwe sleutels, wij.}{zullen de veroordeelde}{gaan verwittigen}\\

\haiku{Hij was de huisgraaf.....}{van koning Hilperik en}{stond nevens den troon}\\

\haiku{Zijn mijne ouders,,?}{en allen die mij dierbaar}{waren niet tot God}\\

\haiku{Uw oom stemt toe in;}{uwe echtverbintenis met}{mijnen heer Hlodwig}\\

\haiku{{\textquoteright} zeide Clothildis,.}{tot Aurelianus daar zij}{reeds ter deure ging}\\

\haiku{Zij klommen met de;}{Burgondische heeren in}{den breeden wagen}\\

\haiku{Het was een laag, maar,.}{zeer uitgestrekt gebouw van}{hout opgetimmerd}\\

\haiku{Ramold zal eischen,;}{dat de verloving in den}{Wijhof geschiedde}\\

\haiku{Ware der Franken!}{leger door de Romeinen}{vernield geworden}\\

\haiku{Glanzende sterren ';}{Ontvallen de lucht Aant}{einde der tijden}\\

\haiku{en heden nog wil.}{hij de zwartharige tot}{echtgenoote hebben}\\

\haiku{Verboden is het,;}{ons de zwartharige met}{bloed te besproeien}\\

\haiku{Kom, goede broeder,:}{stijg weder te paard en blijf}{dicht bij den wagen}\\

\haiku{hij deed gansch Frankrijk,.}{en gansch Belgi\"e doorzoeken}{om haar te vinden}\\

\haiku{{\textquoteright} Eenigen der vrouwen.}{verbleekten en maakten het}{teeken des kruises}\\

\haiku{Van onder hare.}{wangen vloot een tranenstroom}{over den lessenaar}\\

\haiku{Wij moeten onze,:}{wenschen bedwingen ons vriend}{der tijden maken}\\

\haiku{) Haar bij den schouder,:}{vattend bulderde Hlodwig}{op somberen toon}\\

\haiku{Aurelianus ging;}{ter rechterzijde nevens}{een der kinderen}\\

\haiku{{\textquoteright} Een burger legde,:}{den jonkman de hand op den}{mond hem zeggende}\\

\haiku{- Het naakt zwaard zich over,.}{den schouder leggende trad}{hij veldwaarts in}\\

\haiku{- Ik kon in het eerst;}{aan zulke schandelijke}{lafheid niet gelooven}\\

\haiku{{\textquoteright} morde de huisgraaf,.}{daar hij vol angst en ijzing}{zijn hoofd terugtrok}\\

\haiku{ik heb niet gaarne,.}{dat gij Hlodomar minder}{schijnt te beminnen}\\

\haiku{- Eene enkele bleef}{bij de koninginne met}{de zilveren kom}\\

\haiku{Ingomer gelijkt{\textquoteright} {\textquoteleft}?}{integendeel aan u.Mijn}{haar is toch niet bruin}\\

\haiku{- Hlodwig wist niet, wat,.}{hij voorhad en zag hem met}{verwondering aan}\\

\haiku{het is de eerste,;}{maal dat hij met ons ter wacht}{is opgeroepen}\\

\haiku{eenige druppelen,.....}{er van zijn genoeg om eenen}{os te dooden zegt zij}\\

\haiku{Zij bracht een voorwerp,:}{als een eikel in zijne}{hand daar zij zeide}\\

\haiku{tegen den laster.....}{is de verschooning zelve}{eene beschuldiging}\\

\haiku{{\textquoteleft}Luister, hoort gij de?}{koninginne zelve niet}{om bijstand huilen}\\

\haiku{Arme Ingomer,,{\textquoteright}, {\textquoteleft},!}{mijn lieveling kreet zijach}{mij scheurt het harte}\\

\haiku{Hlodwig sloeg den blik;}{op het loodvervig gelaat}{van zijn stervend kind}\\

\haiku{Kom, red mijn kind, ik,,.}{zal u schatten geven u}{danken u bidden}\\

\haiku{{\textquoteleft}O, mijn Ingomer,,,!}{mijn dierbaar kind neen de dood}{zal u niet hebben}\\

\haiku{Vergiffenis voor,.....}{eene moeder die verdwaalt van}{onzeglijke smart}\\

\haiku{het is omringd van,,,;}{licht het heeft vleugelen het}{juicht het zingt uwen lof}\\

\haiku{Zoo zegt men, dat het;}{leger ontevreden is}{en dreigt op te staan}\\

\haiku{eenen harden blik op:}{Aurelianus en sprak met}{koude bitsigheid}\\

\haiku{Siegebald hield zich;}{nevens den koning en sprak}{gemeenzaam met hem}\\

\haiku{Bij elke Gouw of;}{afzonderlijke bende}{bleef slechts \'e\'en edeling}\\

\haiku{en slechts de stoutsten.}{durfden geheel in zijne}{nabijheid blijven}\\

\haiku{Opstaande, zeide:}{de edeling met statige}{koelheid tot den vorst}\\

\haiku{{\textquoteright} En, Aurelianus,:}{met eenen korten wenk tot zich}{roepend beval hij}\\

\haiku{{\textquoteright} De Gallo-Romein.}{sprong met eenen luiden schreeuw uit}{zijnen zetel op}\\

\haiku{O, God, leg dit kruis;}{der bitterste martelie}{mij op de schouders}\\

\haiku{Vooronderstel, dat,.}{ik waarheid niets dan waarheid}{gesproken hebbe}\\

\haiku{mijnen zegen met.....}{het teeken des kruises op}{zijn voorhoofd schrijven}\\

\haiku{{\textquoteright} {\textquoteleft}Wil ik bevelen,,?}{heer dat men eenig warm water}{in het bad storte}\\

\haiku{zie, of de wachten.}{bij het paleis hunnen plicht}{doen en waakzaam zijn}\\

\haiku{Of wel, men zou hem.}{door de straten leiden als}{eenen onnoozelen dwaas}\\

\haiku{De voorbijgangers.....}{zouden met medelijden}{op hem nederzien}\\

\haiku{onwillig vat ik,;}{mijn zwaard en sla toe om den}{eerschender te dooden}\\

\haiku{Uw gemoed heeft mij,.....}{gelasterd uw geest heeft mij}{die eer ontstolen}\\

\haiku{uwe liefde, het woord.}{uwer genegenheid is haar}{noodig om te leven}\\

\haiku{verhaal mij, wat er,{\textquoteright},.}{is geschied sprak zij hem tot}{de banke leidend}\\

\haiku{zijn naam is Warnfried,.}{en hij hoorde te huis in}{Lominger-Gouw}\\

\haiku{{\textquoteright} Eene uitdrukking van'.}{spijtigen toorn ontstelde}{Lutgardis gelaat}\\

\haiku{Ik had ongelijk,,,;}{u te hoonen Siegebald}{lieve dierbare}\\

\haiku{{\textquoteright} Zij scheen eensklaps te:}{verschieten en zeide met}{angst op het gelaat}\\

\haiku{{\textquoteleft}Siegebald, het zijn,?}{geene blijde gedachten die}{u door het hoofd gaan}\\

\haiku{Ik twijfelde, of;}{ik nog wel een mannenhart}{in den boezem had}\\

\haiku{Ik zou haar alleen?}{ten prooi der wraakgierige}{Heidenen laten}\\

\haiku{Gij zult getuigen,,,.....}{heer dat ik mij zelven noch}{mijn paard heb gespaard}\\

\haiku{{\textquoteright} {\textquoteleft}En heeft onze heer,?}{koning u niet gelast mij}{iets meer te zeggen}\\

\haiku{ontferm U over het,.}{akelig lot van het kind dat}{U werd toegewijd}\\

\haiku{Dienvolgens, langs de:}{groote poort tegen het Forum}{kunnen wij niet gaan}\\

\haiku{{\textquoteright} De koningin wees.}{naar eene kleine tafel bij}{hare bedstede}\\

\haiku{gij weet niet, welke.}{benauwdheid uwe moeder in}{den boezem verstikt}\\

\haiku{{\textquoteright} {\textquoteleft}Beef zoo niet, Maria,{\textquoteright};}{antwoordde de koningin}{met gelatenheid}\\

\haiku{{\textquoteright} Maria wierp zich met;}{dwalende treurnis aan den}{hals der koningin}\\

\haiku{Hoort gij, dat mij een,.}{ongeluk overkomt vlucht dan}{op Gods genade}\\

\haiku{{\textquoteleft}Clothildis, goede,;}{een schrikkelijk ongeluk}{is mij overkomen}\\

\haiku{{\textquoteright} {\textquoteleft}D\'a\'ar, d\'a\'ar in het bed,{\textquoteright}, {\textquoteleft},!}{stamelde de koningin}{eene ademing een zucht}\\

\haiku{{\textquoteright} Als een pijl vloog hij.}{ter deure uit en verdween}{in de duisternis}\\

\haiku{geheug u alleen,.}{den schat van grootmoedigheid}{die zijn hart besluit}\\

\haiku{{\textquoteright} {\textquoteleft}Met u, met u wil,{\textquoteright}, {\textquoteleft}.....}{ik zijn zuchtte Clothildis}{u niet verlaten}\\

\haiku{de meesten echter.}{stonden in de verte bij}{de offerdieren}\\

\haiku{Tot nu was de vlucht;}{der raven en het lot der}{Runen ons gunstig}\\

\haiku{{\textquoteleft}Clothildis, waarom?}{deed gij een kruis bij mijne}{tente oprichten}\\

\haiku{Deze overweging ';}{was als een bliksem doors}{konings geest gegaan}\\

\haiku{Hier opent zich een graf,!}{voor ons allen ook voor ons}{ongelukkig kind}\\

\haiku{{\textquoteleft}Ik wil ten hunnen;}{gunste de knie voor mijnen heer}{en koning buigen}\\

\haiku{dit teeken voor den,:}{koning stellende sprak hij}{op plechtigen toon}\\

\haiku{De burgers hadden:}{zich daar in vaste drommen}{te zaam gedrongen}\\

\haiku{Dit is de echte,.}{naam van Clovis den eersten}{koning van Frankrijk}\\

\haiku{doch ten onrechte,,.}{zooals blijkt uit Gregorius}{Turonensis lib}\\

\haiku{God heet de derde ().}{dag der week Dysendag of}{Dynsdagdies Martis}\\

\haiku{Hij bekleedt in de.}{Noordsche Godenleer eenigszins}{de plaats des duivels}\\

\haiku{Dit is het wapen, ().}{door de Romeinen framea}{Fr. fram\'ee genaamd}\\

\haiku{71, waar er staat {\textquoteleft}The{\textquoteright}.}{AEthelings or chiefs of the}{Angles or Saxons}\\

\haiku{38Vienna is,.}{de stad Vienne op de}{Rh\^one in Frankrijk}\\

\haiku{60De kleine stad,.}{Auxonne op de rivier la}{Sa\^one in Frankrijk}\\

\haiku{73Zie over deze,.}{strafpleging Gregorius}{Turonensis Lib}\\

\subsection{Uit: Volledige werken 40. De kerels van Vlaanderen}

\haiku{Wat mij verhindert,!}{wat mij in den weg staat zal}{ik verbrijzelen}\\

\haiku{zij geschiedde op.}{geheel christelijke en}{stichtende wijze}\\

\haiku{{\textquoteright} vroeg de ridder met.}{eene stem die uit eenen kelder}{scheen op te klimmen}\\

\haiku{{\textquoteright} vroeg Burchard, nadat.}{hij den ridders eenen korten}{groet had toegestuurd}\\

\haiku{{\textquoteright} Eene uitdrukking van.}{ongenoegen trok Burchards}{lippen te zamen}\\

\haiku{{\textquoteright} {\textquoteleft}Dat geloove de!}{duivel indien hij zich wil}{laten bedriegen}\\

\haiku{want zij spreken er:}{van en beslissen den twist}{met de Walsche spreuk}\\

\haiku{Onze berichten.}{zijn niet zoo geruststellend}{als gij het voorgeeft}\\

\haiku{maar hij onderstond.}{deemoedig den uitval van}{zijnen ouden oom}\\

\haiku{Is alle gevoel?}{van eer en plicht eensklaps}{in u gestorven}\\

\haiku{{\textquoteright} {\textquoteleft}Toon u ten minste.}{een weinig lieftallig voor}{jonkver Placida}\\

\haiku{Wel vertraagde hij;}{zijnen stap en wel scheen hij}{soms te willen staan}\\

\haiku{maar de vrijheid, ziet,;}{gij is een kostbaardere}{schat dan het leven}\\

\haiku{Dit hoofd is nu de.}{machtige en gevreesde}{koning van Frankrijk}\\

\haiku{want zijne scherpe.}{lippen trokken bevend tot}{eenen grijns te zamen}\\

\haiku{Disdir Vos zag hem.}{achterna met een zuren}{lach op de lippen}\\

\haiku{{\textquoteright} In de kleeding dezer;}{Kerels heerschte vooral}{de blauwe verf30}\\

\haiku{Waarom dan voeren,?}{zij zwaarden als waren zij}{edelgeborenen}\\

\haiku{{\textquoteright} {\textquoteleft}Het is eene zaak van,{\textquoteright},.}{tijd heer hertog antwoordde}{de graaf zeer bedaard}\\

\haiku{{\textquoteright}, zeide de graaf, het, {\textquoteleft};}{hoofd schuddendedie toestand}{zal veranderen}\\

\haiku{{\textquoteleft}Het lust mij heden.}{niet den bedroevenden kant}{der dingen te zien}\\

\haiku{De dikke baard die;}{hem op de borst hing begon}{reeds te vergrijzen}\\

\haiku{Blijf immer dicht bij,,,,}{mij Strena Komt kinderen}{geeft mij elk eene hand.}\\

\haiku{Er moet een einde!}{aan onze schandelijke}{lijdzaamheid komen}\\

\haiku{hij zal hopen haar.}{te beminnen en hij zal}{er in gelukken}\\

\haiku{Tusschen u en mij;}{en uwen broeder zal het lot}{eenen afgrond delven}\\

\haiku{O, behoede de!}{barmhartige God u voor}{zulk akelig lijden}\\

\haiku{{\textquoteright} {\textquoteleft}Maar laat mij spreken{\textquoteright},,.}{morde Witta hare klacht}{onderbrekende}\\

\haiku{maar gij moet bedaard...}{blijven of ik verlaat u}{oogenblikkelijk}\\

\haiku{Maar de stilte die.}{haar omringde riep haar tot}{bewustzijn terug}\\

\haiku{{\textquoteleft}Wat men te Rijssel.}{heeft besloten zal men hier}{niet veranderen}\\

\haiku{{\textquoteright} {\textquoteleft}Maar is het zoo, mijn,.}{neef laat mij nog eene poging}{bij hem beproeven}\\

\haiku{Nu wil ik van dit}{huwelijk niet meer hooren}{en ik bevestig}\\

\haiku{Een onnoozel kind van,;}{veertien jaar met Kerlenbloed}{in de aderen toch}\\

\haiku{Gij ziet wel, heer proost,.}{dat er met lankmoedigheid}{niets is te winnen}\\

\haiku{{\textquoteright} {\textquoteleft}Mijn eigendom is{\textquoteright},.}{goed bewaard zeide Burchard}{met fieren glimlach}\\

\haiku{En hem niet weder,!}{levend kunnen maken zelfs}{niet in stroomen bloed}\\

\haiku{Hier sprong hij op zijn.}{brieschend paard en drukte het}{de spoor door de huid}\\

\haiku{Gaat nu naar huis, wekt.}{onderwege de Kerels}{en zendt ze herwaarts}\\

\haiku{de beste Runnen...}{zijne sterke leden en}{mannelijken moed}\\

\haiku{Hier steeg Burchard met.}{de Keurmans af en deed de}{Kerels stilhouden}\\

\haiku{Hij is in den burcht,!}{de vuige moordenaar van}{mijnen armen Eric}\\

\haiku{Hij snakt om tusschen.}{de Isegrims als een hunner}{te worden aanvaard}\\

\haiku{{\textquoteleft}Mij spijt het zeer eenen,.}{vriend dus toe te spreken maar}{gij dwingt er mij toe}\\

\haiku{Wordt er bloed tusschen,,.}{ons gestort het valle dan}{op u mher Disdir}\\

\haiku{Een strenge oogslag.}{van den proost dwong hem echter}{weder tot stilte}\\

\haiku{als gij nu zijt acht.}{gij mogelijk wat geheel}{onmogelijk is}\\

\haiku{Robrecht, ik herdenk;}{dat ik veroordeeld was tot}{eeuwige treurnis}\\

\haiku{Maar Ghyselbrecht, die,:}{het had gehoord antwoordde}{op tergenden toon}\\

\haiku{{\textquoteright} kreet Dakerlia op den, {\textquoteleft},!}{toon der diepste wanhoopo}{mijn arme vader}\\

\haiku{Wat Jakob de Leeuw,;}{betreft die had wel waarlijk}{den geest gegeven}\\

\haiku{{\textquoteleft}Alzoo, gij meent het?}{nog mogelijk dat Dakerlia}{Wulf uwe vrouw worde}\\

\haiku{Zijne wonde, die,.}{nog altijd was ontstoken}{is nu gesloten}\\

\haiku{mher Wulf kan niets van{\textquoteright},.}{de afkondiging verstaan}{bemerkte de proost}\\

\haiku{Maar Dakerlia, door eenen,.}{doodelijken schrik aangejaagd was}{niet te bedaren}\\

\haiku{{\textquoteright} riep zij uit, terwijl.}{men haar poogde van de deur}{terug te houden}\\

\haiku{Mocht zij hem dan niet?}{een laatst vaarwel wenschen en}{hem de oogen sluiten}\\

\haiku{hunne bestierders,,.}{jaarlijks door hen gekozen}{noemden zij Keurmans}\\

\haiku{Wij mogen geenen tijd.}{verliezen om ons zulken}{leidsman te geven}\\

\haiku{Op eene vertroosting:}{van Witta antwoordde zij}{met gelatenheid}\\

\haiku{Is er iets dat u,}{belet ons morgen vaarwel}{te komen wenschen}\\

\haiku{{\textquoteleft}De verwijdering,:}{van mher Sneloghe bedroeft}{mij ik beken het}\\

\haiku{Hier nam hij echter.}{afscheid van hen en keerde}{terug naar den burg}\\

\haiku{Ik wil niets gemeens}{hebben met lieden die eenen}{afschuwelijken}\\

\haiku{Overal waren de.}{lieden gevlucht en niemand}{bood ons wederstand}\\

\haiku{{\textquoteleft}De schalken wilden,;}{mij beletten tot u te}{naderen heeren}\\

\haiku{Daar zijn Houtkerels;}{die het lijk van graaf Karel}{allen smaad aandoen}\\

\haiku{wij verzoeken u '.}{naars graven kapelle}{te willen komen}\\

\haiku{Welnu, kastelein{\textquoteright},, {\textquoteleft}?}{vroeg hij hemgaat het werk goed}{voort aan de vesten}\\

\haiku{Men is dus verplicht.}{te denken dat het hun aan}{geenen strijdlust ontbreekt}\\

\haiku{Dakerlia, zoo op de?}{borst van Robrecht tranen van}{liefde stortende}\\

\haiku{ik dank u. Ga nu.}{tot den heer kastelein en}{draag hem uwe boodschap}\\

\haiku{Si ne connens niet,.}{ontgangen Si ne dogen}{niet sonder bedwanc}\\

\haiku{zooals ik doelmatig.}{zal oordeelen om hare}{hand te bekomen}\\

\haiku{Mher Van Praet zette.}{zich neder op eenen stoel en}{veegde zijn zweet af}\\

\haiku{God zegene u,,.}{mher Vos die ons ter hulpe}{komt in onzen nood}\\

\haiku{Weet gij wat ik hem?}{met saamgevoegde handen}{heb toegeroepen}\\

\haiku{Hij liet zich op de;}{knie\"en vallen en smeekte}{reeds om genade}\\

\haiku{Daar stonden voor de.}{poort een twintigtal Kerels}{op hen te wachten}\\

\haiku{{\textquoteright} {\textquoteleft}Maar, vriend Sneloghe,{\textquoteright},.}{wanhopen moogt gij toch niet}{murmelde Ludgard}\\

\haiku{Is de moordenaar,?}{Burchard niet van het bloed der}{Erembalds evenals ik}\\

\haiku{hij was belast, tot.}{het afweren van eenen storm}{in gereedheid was}\\

\haiku{Op de Kerels deed.}{deze afkondiging eenen}{min diepen indruk}\\

\haiku{{\textquoteright} {\textquoteleft}Weigert gij dan aan?}{het vriendelijk verzoek van}{den proost te voldoen}\\

\haiku{Uit den mond van hem.}{die de valschheid zelf is}{vloeit niets dan logen}\\

\haiku{Gij zegt dat ik den?}{vijand onze stad Brugge}{heb overgeleverd}\\

\haiku{Robrecht wilde den;}{wal aan den linkerkant der}{Hofpoort beklimmen}\\

\haiku{doch allengs zakte,.}{het hoofd hem op den schouder}{en hij sloot de oogen}\\

\haiku{Dan keerde hij, als,.}{uitzinnig van schrik terug}{tot achter de poort}\\

\haiku{{\textquoteright} En hij zakte, door,.}{de wanhoop verpletterd op}{eene bank neder}\\

\haiku{Na eene lange wijl,;}{nog naderde inderdaad}{een wapenbode}\\

\haiku{Eenige woorden slechts.}{wenschte hij met den proost}{te verwisselen}\\

\haiku{en het is zoo dat.}{ik belast werd als bode}{tot u te komen}\\

\haiku{Indien God over uw,.}{leven had beschikt dan wierd}{uw graf het mijne}\\

\haiku{Ach, wees goed, geef mij!}{het liefdebewijs dat ik}{u smeekend afbid}\\

\haiku{{\textquoteright} galmde Robrecht, de.}{verkrampte vuist tot Burchard}{vooruitstekende}\\

\haiku{Het verlies van den.}{kastelein liet de Kerels}{zonder opperhoofd}\\

\haiku{Een weinig voor elf.}{uur werd hun de reden dier}{beweging verklaard}\\

\haiku{opdat elk hunner.}{gestraft wierde in de maat}{zijner schuldigheid}\\

\haiku{Eene onduldbare:}{vermaledijding bonsde}{op uit Burchards borst}\\

\haiku{Tot nu toe had geen;}{hunner binnen Brugge zich}{durven vertoonen}\\

\haiku{479.)wenteltrap die zoo.}{nauw is dat men haar man voor}{man moet beklimmen}\\

\haiku{{\textquoteleft}Veldheer, heeft men nu?}{eene betere herberg voor}{mij doen bereiden}\\

\haiku{De refter van het.}{klooster was bijna hoog als}{de beuk eener kerk}\\

\haiku{Dan betuigde hij;}{zijne ontevredenheid}{aan meester Arnold}\\

\haiku{{\textquoteright} {\textquoteleft}Wat mij betreft, ik,;}{ben bereid om het werk voort}{te zetten veldheer}\\

\haiku{weigerden zij, de.}{beukram zou onmeedoogend}{zijn werk voltrekken}\\

\haiku{Zandkorrel dien het,!}{lot mede voert evenals de}{wind een vlokje stof}\\

\haiku{, uwe zuster en u:}{zelven daarboven in de}{armen te drukken}\\

\haiku{Wij smeeken op de!}{knie\"en uwe koninklijke}{grootmoedigheid af}\\

\haiku{, en met ongeduld.}{wachtte een ieder op het}{besluit des konings}\\

\haiku{Men zal zelfs zich niet.}{gewaardigen de Kerels}{te onderhooren}\\

\haiku{Indien de koning?}{van Frankrijk genade wil}{schenken aan Robrecht}\\

\haiku{maar Dakerlia liep tot,:}{hem en weerhield hem terwijl}{zij bevend kermde}\\

\haiku{{\textquoteright} {\textquoteleft}Dit is alles wat?}{gij tot uwe verdediging}{hebt in te brengen}\\

\haiku{{\textquoteleft}Ik zal u dankbaar...,,?}{blijven voor deze weldaad}{en wie weet wie weet}\\

\haiku{zij dienden later.}{tot zoogezegde tooverij}{of waarzeggerij}\\

\section{Antoon Coolen}

\subsection{Uit: Bevrijd vaderland}

\haiku{Stonden wij buiten?}{die tragische crisis der}{democratie\"en}\\

\haiku{De natuur heeft er.}{niet mee te maken en zij}{laat zich niet storen}\\

\haiku{Waarom in den trein,,?}{zoo het mogelijk is den}{ledigen coup\'e}\\

\haiku{Hij is de held in,.}{het gezelschap en zijn vrouw}{deelt in zijn glorie}\\

\haiku{Nu is het eenige,.}{uren diepstil en we voelen}{ons ge{\"\i}soleerd}\\

\haiku{Hij kwam bij ons huis,:}{toen ik niet thuis was en liet}{de boodschap achter}\\

\haiku{Maar de verwoesting,.}{hier is  zooveel jonger}{zooveel vreeslijker}\\

\haiku{Sie haben es hier,.}{h\"ubsch angelegt zei hij}{mij aan de voordeur}\\

\haiku{De menschen op de.}{trottoirs en in de caf\'e's}{kijken langs hem heen}\\

\haiku{Vlaanderen staat nu.}{voor de eerste maal op de}{winnende zijde}\\

\haiku{de kogels fluiten.}{door het hout der bosschen en}{over onze huizen}\\

\haiku{Dit uur der Duitsche.}{overwinning is het uur van}{Frankrijks nederlaag}\\

\haiku{De Gebroeders, De,,.}{Hoop De Eendracht Nooit gedacht}{en De Verwachting}\\

\haiku{De Godsgedachte.}{volk moet in deze dagen}{in vervulling gaan}\\

\haiku{De katholieke.}{priester is de dienaar der}{pauselijke kerk}\\

\haiku{Neen, die Domkerken {\textquoteleft}{\textquoteright}.}{en M\"unsters zijn gebouwdzur}{Andacht und K\"undung}\\

\haiku{Deze moeten hun.}{plaats vinden terzijde van}{altaar en kansel}\\

\haiku{Ziet hier, zegt hij, een.}{lijst van noordsche sterren aan}{den Duitschen hemel}\\

\haiku{dank zij Duitschland,!}{weet de wereld weer dat er}{een Vlaanderen is}\\

\haiku{Engeland mist zulk.}{een propaganda en zulk}{een propagandist}\\

\haiku{nuchterheid, vroomheid.}{en vrijheidszin er vreemd en}{vijandig aan is}\\

\haiku{beb kan die man ook}{wel want hij is bij der thuis}{geweest met zijn vrouw}\\

\haiku{- In Den Haag aan de,}{stations controleeren zij}{bij de uitgangen}\\

\haiku{Voor het eerst hoor ik:}{nu uit dezen kindermond}{de verzekering}\\

\haiku{overal waar ge in.}{die hooge wijdheid kijkt zijn ze}{in groepen bijeen}\\

\haiku{de kleine, goede.}{warmte van licht en feest van}{Zondag en Kerstmis}\\

\haiku{(Zoo angstig ben ik,.}{dat ik mijzelf zoo krachtig}{moet geruststellen}\\

\haiku{Als het eenigen tijd.}{stil is geweest worden de}{vensters geopend}\\

\haiku{De oudste heeft een.}{paar gedoode Duitschers zien}{liggen in den tuin}\\

\haiku{Na Kerstmis was het,...}{bij den arme weer armoe}{en bij den rijke}\\

\haiku{Please, it is a,:}{quarter to eight en zij brengt}{hun warm  water}\\

\haiku{Er is verteld, dat.}{zij de tranen in de oogen}{had bij den aanblik}\\

\haiku{En met elken dag,,.}{dien hij langer duurt wint het}{hart aan zekerheid}\\

\haiku{Reeds kon Eisenhower:}{een proclamatie tot het}{Duitsche volk richten}\\

\haiku{De Duitschers hebben:}{nooit dien onderstroom der groote}{krachten in ons volk}\\

\subsection{Uit: Herberg In 't Misverstand}

\haiku{De schildersbaas en}{drogist vond het juist prachtig}{als hij er weinig}\\

\haiku{Maar ook toen hij recht.}{zat zag hij de Rooy omhoog}{en omlaag wiegen}\\

\haiku{Er waren ook te,.}{veel gasten ze hadden een}{groote ruimte van doen}\\

\haiku{Een der meisjes kwam.}{den gemeenteontvanger}{bij haar wegtrekken}\\

\haiku{'s Morgens bij het.}{wakker worden dacht zij er}{niet dadelijk aan}\\

\haiku{In haar bitterheid.}{werd zij onverschilliger}{voor dat \`andere}\\

\haiku{En omdat ik het,.}{niet heb gewild hoeft hij het}{ook nooit te weten}\\

\haiku{Hij onderging de,.}{welgezindheid en rookte}{tevreden een pijp}\\

\haiku{Ze gebruikte de.}{uitdrukking van de vrouwen}{in haar omgeving}\\

\haiku{Over haar schoot tastte,.}{haar hand om haar schort bij den}{rand op te nemen}\\

\haiku{Op een buiigen.}{dag in den nawinter werd}{het kind geboren}\\

\haiku{Marjanne vroeg wat,?}{er aan de hand was had hij}{een zweer in den hals}\\

\haiku{Het zal uw eigen,.}{ondervinding zijn maar hier}{is het dan niet zoo}\\

\haiku{Toen tikte ze met.}{den scherpen kant van de kaart}{tegen de lippen}\\

\haiku{- Je zult zien als het,.}{kind er is heb je er de}{zachtste vrouw aan}\\

\haiku{Gedurende de}{dagen dat zij weg waren}{rekte Jan Jacob}\\

\haiku{Jan Jacob liet zich,.}{tracteeren ze wilden hem}{schadeloos stellen}\\

\haiku{In 't Misverstand,.}{totdat haar man zijn glas bier}{had leeggedronken}\\

\haiku{Verbeeldde hij zich?}{werkelijk dat hij hier iets}{te gelasten had}\\

\haiku{Daar zat - wacht, hij zou -.}{hem eens even vastpakken daar}{zat zijn revolver}\\

\haiku{Dat de kinderen.}{de school dichterbij hadden}{kon hem niet schelen}\\

\haiku{Ze gleden telkens,.}{naar onder weg ze wisten}{met hun beenen geen raad}\\

\haiku{Van dat sublimaat,,?}{die rattentarwe en dat}{koord dat wisten ze}\\

\haiku{Maar een m\'o\'oi dorp was,,.}{het daar ging niets van af je}{ging eraan hechten}\\

\haiku{Daar keek Anna naar,,.}{in haar opkamertje bij}{de Deysselbloemen}\\

\haiku{Thuis gebruikte haar.}{moeder voor het middageten}{een tafellaken}\\

\haiku{- Ik kan mij er niet,.}{mee bemoeien ieder is}{baas over zijn eigen}\\

\haiku{Daar zat zij, in de,.}{kilte maar zij had er een}{besloten toevlucht}\\

\haiku{Zij wist niet precies.}{wat haar in die vrouw aantrok}{en vertrouwen gaf}\\

\haiku{Zij zat op een stoel.}{tegen den muur en hield het}{hoofd ver achterover}\\

\haiku{De moeder nam hen,.}{stil bijeen Thijs Rooyakkers}{zei niets tegen hen}\\

\haiku{Als een man niet deugt,.}{dan zijn vrouwen er zacht en}{vriendelijk tegen}\\

\haiku{Ze sloeg er planken.}{voor en timmerde die vast}{in de kozijnen}\\

\haiku{Want schadepostjes,.}{nam hij zelf die schoof hij niet}{op de klanten af}\\

\haiku{En daar, ge krijgt het.}{beneden den prijs dien het}{me zelf heeft gekost}\\

\haiku{Toen ze drie weken.}{weg was geweest kwam Anna}{een avond weer terug}\\

\haiku{Bij dit werk werden.}{haar gedachten heel kalm en}{heel vriendelijk}\\

\haiku{Als er kinderen.}{bij waren deden  de}{grooten voorzichtig}\\

\haiku{- H\`e, h\`e toch ja, dat,!}{is nogal een mooi zeggen}{dat ge u niet schaamt}\\

\haiku{Maar na een tijdje,.}{hield de benauwdheid op hij}{kwam er weer doorheen}\\

\haiku{Toen Anna en haar '.}{mans avonds thuiskwamen was}{het gaan regenen}\\

\haiku{Even later kwam zij,.}{terug met haar groot kerkboek}{dat rood op snee was}\\

\haiku{Zij haalde er de.}{meegebrachte bidprentjes}{van haar vader uit}\\

\haiku{In de hitte van.}{den zomer wiedde Anna}{mee in de mangels}\\

\haiku{Bij den akkerzoom.}{nam hij met een aanloop den}{sprong over de bermsloot}\\

\haiku{Toen vroegen ze Jan,.}{Jacob hoe het bij hem thuis}{was afgeloopen}\\

\haiku{Als hij dan ziek werd,.}{alleen door bij haar te zijn}{dan moest hij maar gaan}\\

\haiku{Het is gedwongen.}{door het gebruik en omdat}{iedereen het doet}\\

\haiku{Dat was niet om den,.}{jongen alleen Marjanne}{begreep dat heel goed}\\

\haiku{Het heele huis was,.}{vol van het kind het kind was}{in alle dingen}\\

\haiku{En Lodewijk zet.}{voor de gelegenheid zijn}{hoed tot aan zijn rug}\\

\haiku{Na den doop liepen.}{ook Martiens schoonbroers met een}{sigaar in den mond}\\

\haiku{- Als ge dat ook maar,.}{moet aanzien zooals zij smijt met}{geld en met alles}\\

\haiku{Hij hoorde nog die,:}{vraag in de stem van zijn vrouw}{toen ze zijn naam zei}\\

\haiku{- die boer Martien! - Nee,,.}{voor een postzegel was het}{niet zei hij nog eens}\\

\haiku{De Rooy  zag, hoe.}{den boer het zweet in straaltjes}{langs het gezicht liep}\\

\haiku{Zij vroeg niet, of hij.}{het had gekund en of het}{moeilijk was geweest}\\

\haiku{Op het oogenblik.}{van overlijden was hij lid}{van de fanfare}\\

\haiku{Nadien zat zij weer.}{met groote oogen te kijken naar}{dat ijlende kind}\\

\haiku{Hij bleef ook lang op.}{een stoel bij het bed van den}{jongen zitten}\\

\haiku{Zij deed, of dat van,.}{het hijgen kwam omdat zij}{hard had geloopen}\\

\haiku{Marjanne vroeg, een,:}{beetje angstig maar zij deed}{daarbij zoo gewoon}\\

\haiku{Maar de drogist en.}{schildersbaas werd daardoor een}{beetje geprikkeld}\\

\haiku{Maar omdat ze laat,.}{begonnen bleven ze laat}{in den nacht zitten}\\

\haiku{Notaris Duchateau.}{stond aandachtig en ernstig}{naar hem te kijken}\\

\haiku{Er is geene cent van,,,,!}{jou bij geene c\`ent geene c\`ent hoort}{ge nog niet z\'o\'oveel}\\

\haiku{- En daarom zal ik:}{er niet over uitscheiden voor}{ik mijnen zin heb}\\

\haiku{Zij stond op, gaf hem,,:}{het kind dat zijn handjes naar}{hem uitstak en zei}\\

\haiku{- Als we goed boeren,.}{dan koopt ge dien grond er nog}{bij voor ontginning}\\

\haiku{Maar zooveel geeft ge, '.}{toch wel om geld dat get}{onze wilt hebben}\\

\haiku{Wie bijgeboden,.}{had werd opgeroepen hij}{moest zijn naam zeggen}\\

\haiku{Hij wrong zijn sigaar,.}{in den aschbak rond klopte}{telkens asch eraf}\\

\haiku{Dan, alsof  het,:}{een verpletterend vonnis}{was brulde Fleskens}\\

\haiku{Met den verkooper.}{hoefde niet lang beraad te}{worden gehouden}\\

\haiku{Hij kwam binnen bij.}{de kinderen Deysselbloem}{en vroeg naar Martien}\\

\haiku{Hij stak een sigaar.}{op en bood uit zijn koker}{Martien er een aan}\\

\haiku{De fanfare Sint '.}{Cecilia bracht hems avonds}{een serenade}\\

\haiku{Toen het sluitingsuur.}{sloeg trokken de feestende}{muzikanten af}\\

\haiku{De Rooy sloot en ging.}{daarop het gezelschap voor}{met zijn fakkels}\\

\haiku{- Nee, dat is aardig,,.}{zei hij dat je ondanks je}{voet gekomen bent}\\

\haiku{Met een paar dagen.}{ging hij naar de griffie voor}{de eedsaflegging}\\

\haiku{Tegen Jan Jacob,,.}{die liet merken dat hij het}{zag glimlachte hij}\\

\haiku{En die kunsten en.}{malligheden met dat touw}{en die revolvers}\\

\haiku{Hoe waren nu in '?}{s hemelsnaam die twee bij}{elkaar gekomen}\\

\haiku{Want Wilde Maria,.}{kon hen aankijken dat ze}{wonder wat dachten}\\

\haiku{Het oudste was van,.}{school dat kreeg lange beenen en}{schoot de hoogte in}\\

\haiku{De moeder zat bij,.}{het hoofdeind Jan Jacob zat}{bij het voeteneind}\\

\haiku{Overdag, in het licht,.}{hield hij ervan buiten in}{de velden te zijn}\\

\haiku{In alle weien, ',.}{staat het vee gezegend in}{t licht en graast}\\

\haiku{Maar ginds aan den berm,.}{stond zijn knecht te zwaaien dat}{Martien zou komen}\\

\haiku{Hij liep den bunder,.}{af sprong uit de ruigte van}{den berm over de sloot}\\

\subsection{Uit: Hun grond verwaait}

\haiku{In de wei en langs}{den wegkant vreet ze haar gras}{en onder den balg}\\

\haiku{In de verte is.}{de hemel vol van een kleur}{als van roode wijn}\\

\haiku{Daar op de tafel,.}{staat het komke waaruit hij}{heeft gedronken}\\

\haiku{Daarom zee ze ja.}{en ze was verstandig en}{wijs in dat ja}\\

\haiku{Johannes van Goch,.}{na den arbeid treedt zijn huis}{tegemoet zijn vrouw}\\

\haiku{Johannes had van '.}{t zomer zijnen klot van}{zijn veldje gehaald}\\

\haiku{da de hemel wit.}{was van sterren hun tweede}{kind wier geboren}\\

\haiku{Zijn vader en zijn.}{moeder vonden dat schoon en}{lachten van plezier}\\

\haiku{'t Was hem aan zijn,, '.}{hart gegaan waarachtig maar}{t ging niet anders}\\

\haiku{De avond buigt over ons.}{en de lente geurt en streelt}{ons tot op het bloed}\\

\haiku{Verdomme, zee Piet,.}{hij had weer wat vergeten}{en hij gong terug}\\

\haiku{Vroeg of laat had ik,.}{hier toch weg gemoeten hier}{kom ik nooit vooruit}\\

\haiku{- Ik zeg maar zoo, zee, '.}{Johannes dak pleizier}{van mijn jongens heb}\\

\haiku{Achterna maakt hij,.}{zijn doos open zoekt en haalt er}{een envelop uit}\\

\haiku{Zoo praten ze, over '.}{ent weer wa woorden mee}{stiltes ertusschen}\\

\haiku{hij en was hij mee '.}{een vergeeflijke zuchtn}{bietje apart te zijn}\\

\haiku{Da was Eimertje. ',.}{Schoonemanst Wier verteld}{hoe dat gegaan was}\\

\haiku{Toen ie Lammeke,,.}{zag toen liep ie er heen mee}{zijnen manken poot}\\

\haiku{- Ah, zegt Lodewijk,?}{begint de industrie tot}{hier door te dringen}\\

\haiku{Maar Lodewijk gaat -,?}{er op door Toen ze d'r over}{sprak wat zei ze toen}\\

\haiku{Waar zou nou wel juist?}{het plekje zijn waar ze toen}{hebben gezeten}\\

\haiku{De menschen worre.}{ouder en het aanschijn van}{ons liefland vernieuwt}\\

\haiku{Nee, Friedus bracht met.}{deze streek zijn ouders in}{groote moeilijkheden}\\

\haiku{Na 't heengaan van,,.}{Lodewijk indertijd was}{zijn verzet verzwakt}\\

\haiku{Vader komt binnen.}{en komt den buil tabak van}{den schouwrand halen}\\

\haiku{- Kijk 'es, zegt meester,,.}{Frunt maar dat blijft onder ons}{ik heb mijn plannen}\\

\haiku{En kijk es, ik heb,.}{je moeder hier gehad die}{kwam over je praten}\\

\haiku{Als hij thuiskomt kijkt,}{hij er zijn moeder op aan}{wa ze in haren}\\

\haiku{Mijn oogen kijken zacht.}{en sluiten zich onder de}{aandacht van jouw oogen}\\

\haiku{Bij iedere dood, '.}{zijn tranen van hen die over}{n hortje volgen}\\

\haiku{En ze stuurden  ,.}{hem een bidprentje voor hem}{en zijne vrouw}\\

\haiku{Te lente gaat hij '.}{trouwen meet durske van}{Verleijsdonke}\\

\haiku{zegt Friedus, 't ja, ',!}{t ja waarom heb ik mijn}{vrouw niet meegebracht}\\

\subsection{Uit: Jantje den schoenlapper en zijn Weensch kiendje}

\haiku{Alexandrine loopt de,,:}{achterdeur in de keuken}{door het gangske in}\\

\haiku{- Wier moeten essen,,.}{zegt Jan op zijn weensch als hij}{in de keuken komt}\\

\haiku{Ze droogt zorgvuldig.}{af en zet de borden op}{de tafel ineen}\\

\haiku{Ze hoeft heelemaal,.}{niet voorzichtig te zijn de}{borden zijn van blik}\\

\haiku{Boven haar knerpt en.}{knaagt een houtworm met korte}{rhythmische geluidjes}\\

\haiku{- Wat is tafel bij,.}{ellie vraagt Jan en hij slaat}{op het tafelblad}\\

\haiku{Ziedege, als ge,.}{goei water hebt dan zette}{ge goeje koffie}\\

\haiku{Xandrieke wier uit.}{den hof geroepen en de}{brief wier open gemaakt}\\

\haiku{ergernis, waar ze,.}{maar meent da de deugd geweld}{wordt aangedaan}\\

\haiku{Ja, zegt Gondeke,,.}{pak d'r kleerkes in ik breng}{ze bij de zusters}\\

\haiku{De slag kletst op het.}{jonge bruine vleesch van}{het kinderwangske}\\

\haiku{De hemel is weer,.}{diepblauw geworre over de}{vochtige aarde}\\

\haiku{daar zat een merel,}{in de wije wereld waarin}{het avond geworren}\\

\subsection{Uit: Kerstmis in de Kempen}

\haiku{Daar ging het niet over,,.}{zei hij maar recht was recht en}{reden was reden}\\

\haiku{Den oudste te zien,,,.}{zoo'ne groote schoone mensch dat}{geeft oe gedachten}\\

\haiku{Daar zat den ouden.}{Keunen zachtjes bij nee te}{schudden met het hoofd}\\

\haiku{Hij zette den kraag,.}{van zijnen jas op trok de}{pet over zijnen kop}\\

\haiku{In de volte kwam.}{Hanna van Bommel en zocht}{plaats op een stoelke}\\

\haiku{Zij trad in den herd,,.}{van het avondhuis daar wachtte}{Eimerd hare mensch}\\

\haiku{ik zeg het oe in,,.}{vertrouwen ge haalt er geen}{woord meer uit ze zwijgt}\\

\haiku{De man en de vrouw.}{vieten hunnen verket en}{aten uit den schotel}\\

\haiku{Onder den eten keek,.}{zij op van haar handen zij}{zag altijd het paard}\\

\haiku{Maar de plaveien,.}{trapte het kapot dat kon}{het niet voorkomen}\\

\haiku{Toen ging ze bleek en.}{met slepende voeten den}{akker af naar huis}\\

\haiku{Langzaam trok zij een.}{kruis over het rood cijfer van}{den eersten Kerstdag}\\

\haiku{Eimerd zat achter,.}{de plattebuis waarin hij}{een goed vuur stookte}\\

\haiku{{\textquoteleft}ga es op zij{\textquoteright}, zei, {\textquoteleft} ' '.}{zega d\'a\'ares zitten da}{k uit de weeg kan}\\

\haiku{- Ja, maar wie zegt ou,'?}{da gij oe eigen niet net}{zoo goed tekort doet}\\

\haiku{Hij lachte gegriefd.}{en troeste den geplooiden}{gerimpelden mond}\\

\haiku{- De naakten kleeden,.}{voor die dat verzuimt zal het}{er leelijk uitzien}\\

\haiku{- O, Kuunders, zei ze,,,.}{van de Panneschop ja daar}{heb ik af gehoord}\\

\haiku{Thuis dacht Govert Kuunders,.}{eraan waar de goejvrouw}{kon zijn gebleven}\\

\haiku{- 't Is het eenigste,.}{wat ik oe aan kan bieden}{het is oe gegund}\\

\haiku{Druk klopte hij de.}{asch van zijn sigaar in den}{koperen aschbak}\\

\haiku{Of er ook niet wat.}{geld vastgelegd moest worden}{voor zielemissen}\\

\haiku{gij daar en gij daar,.}{en gij daar en doorloopen}{en naar oe plaats gaan}\\

\haiku{Anne-Marie,.}{zei dat het toch met een paar}{dagen Kerstmis was}\\

\haiku{Hij maakte een breed,.}{overhangend dak van stroo dat}{hij bond met poppen}\\

\haiku{Daar waren 'nen hoop,.}{dingen waar zij hem nooit toe}{had kunnen krijgen}\\

\haiku{Veel hadden wij thuis,:}{niet maar met Kerstmis moest het}{er op overschieten}\\

\haiku{Nu daalde er, als,.}{met de sneeuw toen weer zoo iets}{goeds over de wereld}\\

\haiku{En de verre, hoog,.}{canada's die met eenen bocht}{van den weg meegaan}\\

\subsection{Uit: Lentebloesem}

\haiku{- Ik ben jouw moeder,,,,,,.}{ik ik ik ja kijk maar zoo}{lieve wijzeman}\\

\haiku{- In mijn oogen moogt ge,.}{n\'o\'oit anders zien dan liefde}{voor jou dan vreugde}\\

\haiku{Fritsje durfde.}{niet hard te stappen en geen}{geluid te maken}\\

\haiku{Want dat gescheurde.}{en gebarsten plafond was}{een klein wereldje}\\

\haiku{Vanavond gaan we met.}{z'n vieren naar een concert}{in de groote Staarzaal}\\

\haiku{Lieve grootvader,,.}{ik moet eindigen want daar}{komt net oom binnen}\\

\haiku{En wat heb ik een.}{leed over dit v\'o\'or mijn komst zoo}{leedlooze huis gebracht}\\

\haiku{- Oom Herman pakte, {\textquoteleft},{\textquoteright}.}{m'n handenje t'aime je}{t'aime zei hij snel}\\

\haiku{Miep E.J. Vanberghe.}{en   Max Wilde}{Ingenieur}\\

\haiku{Den Bosch, \} 25 Juni -, \} -}{19 Maastricht 25 Juni 19}{Kwezelke I}\\

\haiku{Hij lachte nu nog,:}{meer en antwoordde dat die}{rozen duur waren}\\

\haiku{{\textquoteleft}Hier,{\textquoteright} zei hij, {\textquoteleft}rozen,...{\textquoteright} {\textquoteleft},{\textquoteright}.}{LeentjeVoor de Lieve Vrouw}{kwam zij verlegen}\\

\haiku{Ze leefden stil en:}{rustig voort van den eenen dag}{in den anderen}\\

\haiku{Wa' zoude gij er,,,?}{van zeggen Peer van te gaan}{trouwen wij twee\"en}\\

\haiku{Dat was een angst en,,.}{een pijn een dag lang en op}{haar bed schreide ze}\\

\haiku{Toen begon hij weer,.}{te praten wat moeilijk zijn}{woorden zoekende}\\

\haiku{Ter bemoediging {\textquoteleft},.}{drukte hij toen wat vaster}{haar hand.Nou dag dan}\\

\haiku{Er viel een streepje,.}{zon naar binnen juist op het}{kruis boven de schouw}\\

\haiku{Hij met zijn goed hart -',,;}{t\`och nee immers hij is niet}{slecht hij is niet slecht}\\

\haiku{ik u geluk, op,{\textquoteright}}{dezen gedenkwaardigen}{dag in uw leven}\\

\haiku{Ze lachten luid, ze,.}{praatten zoo druk dooreen ze}{tierden rumoerig}\\

\haiku{Het gouden paar zat,!}{te glunderen en Manders}{had pleizier nee maar}\\

\haiku{Zij zongen, maar dat.}{was veel meer schreeuwen en maar}{tieren van pleizier}\\

\haiku{Zou hij nu staan te,?}{droomen bij dat witte ruw}{geschilderde kruis}\\

\haiku{De vrouw van zijn zoon,,!}{een flink vrouw-mensch maar een}{karnalli voor hem}\\

\haiku{als ik binnen kwam,.}{scheen hij te voren mijn komst}{te hebben vermoed}\\

\haiku{Toen besefte ik,}{plotseling hoe volstrekt en}{ontzettend alleen}\\

\haiku{Misschien, dacht hij, ligt;}{het verscholen onder het}{blaadje van een plant}\\

\haiku{en als zijn oogen het,:}{raadsel niet vinden konden}{zei hij bij zichzelf}\\

\haiku{Het was heerlijk bij '.}{dat zomerfeests morgens}{wakker te worden}\\

\haiku{Als ze slapen ging,,,.}{droeg moeder op haar rug haar}{naar haar slaapkamer}\\

\haiku{Hij lachte opnieuw,,,.}{een warme prettige}{lach knikte haar toe}\\

\subsection{Uit: De man met het Jan Klaassenspel}

\haiku{Verandert deze?}{kleine gebeurtenis het}{uitzicht van dit land}\\

\haiku{De kleine man laat,.}{dit beeld voorbijgaan hij noemt}{nadien een nummer}\\

\haiku{Ze kijken hem met,.}{hun groote doode oogen maar eens}{aan daar lacht Proens bij}\\

\haiku{- Nog nie, zee Nolda,.}{al waarde-gij den eenigsten}{mensch op de wereld}\\

\haiku{Toen nadien Proens weer,:}{bij Corneliske over den}{vloer kwam toen zee Proens}\\

\haiku{Anders, als ik maar,.}{eenen stoel krijg voor vannacht dan}{ben ik tevreden}\\

\haiku{nen toevallijder,.}{hij glijdt langs den deurstijl af}{langzaam op den grond}\\

\haiku{- Ge kant zeggen wa',,.}{ge wilt zegt Nolda hij kan}{tenacht hier blijven}\\

\haiku{Zij blijven beiden,}{met dien man bezig en met}{de vraag wat voor eene}\\

\haiku{Nolda is ook al,.}{zoo'ne gek eene zij gaat wat}{wroeten in den zak}\\

\haiku{Hij had niet veel te,.}{strijden zij stond aan zijnen}{kant in dezen strijd}\\

\haiku{Ze kost ook ruzie.}{maken en hem wegduwen}{met den elleboog}\\

\haiku{Nadien, het groote brood.}{voor de borst geplant snijdt ze}{driftig de sneden}\\

\haiku{Corneliske ging,.}{er tegen in maar hij kreeg}{zijn woorden terug}\\

\haiku{En Proens lachte van,.}{de blijdschap dat hij er niet}{mede getrouwd was}\\

\haiku{- Als ik dan morgen,,.}{oe fiets kan leenen vraagt hij om}{er heen te fietsen}\\

\haiku{Haar klompen klijven.}{en glijden in het slijk van}{haar pad langs den weg}\\

\haiku{Die kan niet lachend,.}{gaan liggen om over zich heen}{te laten schieten}\\

\subsection{Uit: De ontmoeting}

\haiku{we doen het om de.}{verloren waardigheid een}{beetje te redden}\\

\haiku{Maurits, thuis, had de.}{krant voor zich en staarde naar}{dat korte bericht}\\

\haiku{Toen ging hij het pad,.}{op naar den landweg waarlangs}{hij thuis zou komen}\\

\haiku{Hij hield zijn paard in,.}{legde de leidsels neer en}{ging den akker af}\\

\haiku{En eten, hij stond zelf.}{verbaasd dat hij niet meer te}{verzadigen was}\\

\haiku{Ze gingen zwijgend,.}{tot zij uit het gezicht der}{anderen waren}\\

\haiku{- Waaraan kan ik het,?}{te danken hebben dat ik}{vrij gekomen ben}\\

\haiku{De jonge boer deed.}{zijn best om zijn aandoening}{niet te verraden}\\

\haiku{- Zij had ons vroeger, -}{al eens een paar keer voor Don}{Jos\'e gewaarschuwd}\\

\haiku{- Barends zal je op,,}{de hoogte hebben gebracht}{zei Tom tot Maurits}\\

\haiku{Maar we kunnen dien.}{moord ook heelemaal buiten}{beschouwing laten}\\

\haiku{Wij nemen aan, dat.}{je een priester bij je wilt}{hebben voor je sterft}\\

\haiku{Nadat de deur was.}{dichtgedaan viel opnieuw een}{diepe stilte in}\\

\haiku{(naam) \_\_\_\_\_\_\_\_\_\_\_\_\_\_\_\_\_\_\_\_ \_\_\_\_\_\_\_\_\_\_\_\_\_\_\_\_\_\_\_\_ (adres) \_\_\_\_\_\_\_\_\_\_\_\_\_\_\_\_\_\_\_\_ (woonplaats) \_\_\_\_\_\_\_\_\_\_\_\_\_\_\_\_\_\_\_\_ meent, \_\_\_\_\_\_\_\_\_\_\_\_\_\_\_\_\_\_\_\_() {\textquoteleft}{\textquoteright}.}{dat de schrijverster is van}{De Ontmoeting}\\

\subsection{Uit: Peerke den Haas}

\haiku{Het vertrek hangt vol,,.}{rook die naar de lamp trekt waar}{hij dik om heen zweeft}\\

\haiku{Het bed, was is het,.}{een bed met wat todden van}{dekens en baalzak}\\

\haiku{Zij let niet op de,.}{bedoeling die hij daarmee}{kan hebben gehad}\\

\haiku{Peerke gaat weer bij,.}{het raam staan daar staat hij voor}{evenveel te kijken}\\

\haiku{Peerke is van den,.}{weg af gegaan de ruigte}{en het hooge bund in}\\

\haiku{Ze komen in de,,.}{Zwaan bij Jans de vierde vrouw}{van Jan den Trouwer}\\

\haiku{Ze staat daar nou te,.}{blinken en te lachen ze}{geeft nergens niks om}\\

\haiku{Een groote vlag, groen als,.}{de zee en met franje in}{allerlei kleuren}\\

\haiku{- Zijde gij zat, dat?}{gij mee oe goej dinge aan}{hebt liggen vallen}\\

\haiku{Hij gaat onder het.}{raam aan de tafel zitten}{en kijkt langs de hor}\\

\haiku{De tanden blijven,.}{er in steken zoodat Peerke}{met kracht moet trekken}\\

\haiku{Zij heeft zeker geen,.}{besef of ze schuld heeft aan}{de dingen of niet}\\

\haiku{Frans den Heete wordt.}{in het licht van de lamp eens}{extra bekeken}\\

\haiku{en 't plekte van ', ',.}{t bloedt was klaar bloed wat}{er op zijn gat zat}\\

\haiku{Dat was voor Frans den:}{Heete eenvoudig genoeg}{om te vertellen}\\

\haiku{- Nee, natuurlijk, hier.}{doe je geen mond open en kun}{je geen tien tellen}\\

\haiku{Hij salueert met.}{slappe vingers en wil ook}{hier parmantig doen}\\

\haiku{Peerke zijn vrouw werd,.}{gehoord ze hoefde den eed}{niet af te leggen}\\

\haiku{De vernieling, ja,,,,.}{die was zoo de kruiwagen}{het schuurke de ruit}\\

\haiku{Hij zette een fijn,}{gouden brilleke op en}{zette het weer af}\\

\haiku{Hij heeft gezeten,.}{hij heeft die schande over hen}{allebei gebracht}\\

\haiku{Jan kwam er wel, hij,.}{had meer van die huiskes tot}{in de voorpeel toe}\\

\haiku{Er vielen wat groote,.}{vlokken die waren verdwaald}{tusschen den regen}\\

\haiku{Toen hield de regen.}{op en sneeuwde het drukker}{in de stilte}\\

\haiku{Nou moest Ciska nog.}{even den sleutel naar Jan den}{Trouwer gaan brengen}\\

\haiku{Alexandrine loopt de,,:}{achterdeur in de keuken}{door het gangske in}\\

\haiku{Ze droogt zorgvuldig.}{af en zet de borden op}{de tafel ineen}\\

\haiku{'s Avonds, als de lamp,,.}{brandt raadpleegt Jan zijn grootboek}{dat kleine boekske}\\

\haiku{Ziedege, als ge,.}{goei water hebt dan zette}{ge goeje koffie}\\

\haiku{Xandrieke wier uit.}{den hof geroepen en de}{brief wier open gemaakt}\\

\haiku{ergernis, waar ze,.}{maar meent dat de deugd geweld}{wordt aangedaan}\\

\haiku{Ja, zegt Gondeke,,.}{pak d'r kleerkes in ik breng}{ze bij de zusters}\\

\haiku{daar zat een merel,}{in de wije wereld waarin}{het avond geworren}\\

\haiku{Het gezicht van den,,.}{jongen boer ontspande zich}{breed tot een lach}\\

\haiku{Het was zoo onnoozel,.}{dat zij den veger niet uit}{de handen legde}\\

\haiku{- Ja, daar moet ge eens,.}{ernstig over denken wat het}{zal moeten worden}\\

\haiku{Helder waaien de.}{tippen van den neteldoek}{en de baker lacht}\\

\haiku{Alla, met een breed,.}{gebaar nam hij een stoel ook}{Frederik lachte}\\

\haiku{Alsof zij koorts had,.}{zat zij nadien bij zijn bed}{te klappertanden}\\

\haiku{Niets is beter dan,,.}{een huishouden kinderen}{een levendig huis}\\

\haiku{- Weet ge, waarom ons,? -?}{moeder zei dat ik u naar}{huis moest brengen Nee}\\

\haiku{Verandert deze?}{kleine gebeurtenis het}{uitzicht van dit land}\\

\haiku{De kleine man laat,.}{dit beeld voorbijgaan hij noemt}{nadien een nummer}\\

\haiku{en het wiegelt van.}{verbeeldingen een bietje}{gek in hunnen kop}\\

\haiku{Ze kijken hem met,.}{hun groote doode oogen maar eens}{aan daar lacht Proens bij}\\

\haiku{Hij lachte rustig,,.}{alsof hij geloofde dat}{hij onkwetsbaar was}\\

\haiku{- Nog nie, zee Nolda,.}{al waarde-gij den eenigsten}{mensch op de wereld}\\

\haiku{Toen nadien Proens weer,:}{bij Corneliske over den}{vloer kwam toen zee Proens}\\

\haiku{Zij blijven beide,}{met dien man bezig en met}{de vraag wat voor eene}\\

\haiku{Nolda is ook al ',.}{zoone gek eene zij gaat wat}{wroeten in den zak}\\

\haiku{De oude man heft.}{in aandacht het hoofd op en}{ziet naar den zolder}\\

\haiku{Kobuske de Pint.}{en den  koster zijn daar}{praat over gaan maken}\\

\haiku{Hij had niet veel te,.}{strijden zij stond aan zijnen}{kant in dezen strijd}\\

\haiku{Ze kost ook ruzie.}{maken en hem wegduwen}{met den elleboog}\\

\haiku{Nadien, het groote brood,.}{voor de borst geplant snijdt ze}{driftig de sneden}\\

\haiku{Corneliske ging,.}{er tegen in maar hij kreeg}{zijn woorden terug}\\

\haiku{Ge hebt nog nooit eenen,,.}{mensch gehoord waar ge zoo om}{moet lachen zee ze}\\

\haiku{- Als ik dan morgen,,.}{oe fiets kan leenen vraagt hij om}{er heen te fietsen}\\

\haiku{Want ze hadden nou.}{diejen mensch  uit Scheijndel}{de deur uitgegooid}\\

\haiku{Haar klompen klijven.}{en glijden in het slijk van}{haar pad langs den weg}\\

\haiku{Daar zijn dingen, diep,.}{verholen zij hebben een}{verborgen teeken}\\

\haiku{- D'r zijn er zat die,.}{met tweejen zijn en die mij}{benijden zegt Proens}\\

\haiku{Die kan niet lachend,.}{gaan liggen om over zich heen}{te laten schieten}\\

\subsection{Uit: Stijntje}

\haiku{De groet wordt z\'o\'o lang.}{en z\'o\'o veel herhaald tot ik}{teruggegroet heb}\\

\haiku{Als vader daar vrij,?}{mag zijn waarom scheurt hij dan}{niet alles kapot}\\

\haiku{Ik schroef de dop van:}{de vulpen en de dreumes}{aan mijn knie\"en zegt}\\

\haiku{Die hem kleedt, die een.}{eindeloos geduld met hem}{heeft aan het ontbijt}\\

\haiku{- Eerst gaat de vleermuis,.}{vliegen dan gaat ze op de}{paddestoel zitten}\\

\subsection{Uit: Tsjechische suite}

\haiku{Maar als lustverblijf:}{voor de Tsjechische schrijvers}{is het een sprookje}\\

\haiku{Integendeel, ik.}{geloofde zelf even oprecht}{in het spel als hij}\\

\haiku{Naar dotters en lisch,.}{en de boot riekt een beetje}{naar teer en naar hout}\\

\haiku{Na twee maanden had:}{hij reeds zijn stempel van bloed}{op het land gedrukt}\\

\haiku{Hun ontzetting is.}{wellicht nog grooter dan die}{van de eerste tien}\\

\haiku{De Tsjechische pers.}{vermeldde haar met eenige}{hartelooze regels}\\

\haiku{De binnenkant van.}{de handen tegen den muur}{wordt klam en vochtig}\\

\haiku{de beschaduwde,;}{hellingen in hun plechtig}{fluweelig donker}\\

\haiku{Hij vertelt, hoe diep,.}{de put is en hij zal dat}{eens demonstreeren}\\

\haiku{Niets kan geluidloozer,.}{zijn dan de schokjes waarmee}{die lichtjes aangaan}\\

\haiku{Daar staan ijzeren}{tafeltjes en ijzeren}{stoeltjes en het zit}\\

\haiku{De groote filmstudio's,.}{van Barrandov daar op die}{hoogte buiten Praag}\\

\haiku{En ik moest aan die:}{kenschetsende benaming}{denken van Ehrenburg}\\

\haiku{Nee, zei hij lachend,,,}{maar hij liet wel merken dat}{het hem pleizier deed}\\

\subsection{Uit: Uit het kleine rijk}

\haiku{Als vader daar vrij,?}{mag zijn waarom scheurt hij dan}{niet alles kapot}\\

\haiku{Nu is het anders.}{dan het doorwaaid opene van}{een warme ruimte}\\

\haiku{De verrassingen:}{der werkelijkheid maken}{deze vraag klemmend}\\

\haiku{Ik schroef den dop van:}{de vulpen en de dreumes}{aan mijn knie\"en zegt}\\

\haiku{Die hem kleedt, die een.}{eindeloos geduld met hem}{heeft aan het ontbijt}\\

\haiku{- Eerst gaat de vleermuis,.}{vliegen dan gaat ze op den}{paddenstoel zitten}\\

\haiku{Ik beken, dat ik.}{op die vraag eenvoudig geen}{antwoord voor hem weet}\\

\haiku{hij zien, dat dat toch,:}{niet zoo erg geweest kan zijn}{hij toont zijn handjes}\\

\haiku{Ze hebben 't over,:}{als ze gr\'o\'ot zijn dan is het}{leven pas heerlijk}\\

\haiku{Dan brengen ze het,.}{naar den molen en dan wordt}{er meel van gemaakt}\\

\haiku{De eene rimpelvlaag.}{na de andere rent over}{het smalle water}\\

\haiku{Een anderen keer -,}{betrap ik hem en zooals hij}{opstaand uit zijn spel}\\

\haiku{En hij geeft weer voer,,,.}{de vingers die niets hebben}{strooien het neer}\\

\haiku{De oudste kon er,.}{niet bij zijn die zit met een}{zeeren voet in huis}\\

\haiku{Zwarte Piet moet het,.}{met minder doen die krijgt van}{elk een sigaret}\\

\haiku{Als ze naar bed zijn.}{blijft die kinderkamer vol}{van hun verwachting}\\

\haiku{deurtjes, die open en,,:}{dicht kunnen een mannetje}{er in en vooral}\\

\haiku{Dan komt hij midden,.}{in de kamer merkbaar}{met een bedoeling}\\

\haiku{En weer bekijkt hij.}{onderzoekend van alle}{kanten zijn auto}\\

\haiku{niet zooals ik het mij,,,.}{had voorgesteld wel groen zooals}{moest maar niet d\`at groen}\\

\haiku{dag Lieve Heer, dag,,.}{Lieve Vrouw dag engel die}{mij moet bewaren}\\

\haiku{En later nog, thuis,,.}{en bij de boterham en}{als ze naar bed gaan}\\

\haiku{De handjes op den,:}{rug zeggen ze kort en een}{beetje verlegen}\\

\haiku{Ze moeten bij de.}{hand worden genomen en}{naar de wieg geleid}\\

\haiku{In een nieuw huis, een,:}{nieuwe kinderkamer een}{nieuwe slaapkamer}\\

\haiku{- Geef mij maar een v\'e\'el,?}{grootere straf als ik maar}{w\`el mag luisteren}\\

\haiku{Ze luisteren met,,,.}{de ooren met den mond de}{oogen met heel hun ziel}\\

\haiku{Maar zoo gauw zit hij,.}{weer recht en roerloos want het}{verhaal gaat verder}\\

\haiku{Wat deed Andersen?}{als het oogsttijd was daar over}{de Deensche velden}\\

\haiku{Maar nu het portret,:}{van den grootvader dat in}{Hjalmars kamer hangt}\\

\haiku{ge v\'o\'or op zijn paard}{zitten en dan vertelt hij}{u een verhaaltje}\\

\haiku{- Dat zijn berichten,.}{die langs een draad gaan net als}{deze papiertjes}\\

\haiku{Den volgenden dag}{in denzelfden wind uit het}{zuidwesten komen}\\

\haiku{van een bronchitis.}{hersteld mocht de oudste voor}{het eerst weer buiten}\\

\haiku{Daar liggen de twee,,.}{anderen al even rustig}{maar ze slapen niet}\\

\haiku{Ik weet nog van niets,.}{en ik heb een zekerheid}{vaster dan alles}\\

\haiku{Langzamer rijdt de, '.}{motor maart donderend}{geraas mindert niet}\\

\subsection{Uit: Zegen der goedheid}

\haiku{Zij groet Joachim met,.}{den zachten eenderen groet}{van iederen avond}\\

\haiku{daar de zon flonkert}{met hare gerimpelde}{gensters en glanzen}\\

\haiku{Uit de bergen zijn.}{herdersknapen naar zijne}{kudden gekomen}\\

\haiku{En de stralende.}{blik harer oogen was bijna}{niet te verdragen}\\

\haiku{De oogen der zangers,.}{sloten zich daar half bij toe}{versluierden zich}\\

\haiku{Nereas echter.}{brulde de mannen toe met}{dreunende woorden}\\

\haiku{Daar verliest hij zijn,.}{hooghartig zwijgen voor om}{het te vertellen}\\

\haiku{En hij herhaalt en:}{hij doet alsof hij dit een}{beetje wil zingen}\\

\haiku{Maar Dismas, de zoon,.}{hij gevoelt zijn onrust en}{zwerft langs de wegen}\\

\haiku{Er is een stilte.}{waarin de stemmen voor de}{sprake aarzelen}\\

\haiku{O, het zijn zeker,.}{liederen van rooftochten}{misschien van moorden}\\

\haiku{De sterren en de.}{bevolen winden aan het}{raam houden de wacht}\\

\haiku{- Als ik daar in de,.}{eenzaamheid woon dan komt gij}{mij maar opzoeken}\\

\haiku{Het snelle zeil lag.}{trillend in het kabbelend}{water weerspiegeld}\\

\haiku{Zij hebben zeker,.}{een ver doel daar richten zij}{hun wandelstaf heen}\\

\haiku{Er zijn werklieden,.}{gekomen die hebben zijn}{huis grooter gebouwd}\\

\haiku{En de lach week niet,.}{van hun gezicht toen zij het}{evangelie hoorden}\\

\haiku{Daarom onderbrak.}{Goar zijn reis en keerde naar}{zijn woning terug}\\

\haiku{De goede blik van,.}{zijn oogen verandert er niet}{om zijn gezicht niet}\\

\haiku{Reprobus laat zich.}{op hetzelfde oogenblik}{niet meer bevelen}\\

\haiku{De getrokken en.}{gebaande wegen leiden}{naar den koning toe}\\

\haiku{De kleine koning.}{ziet met inspanning van de}{oogen op naar den reus}\\

\haiku{De duivel is in,.}{het zwart dat zijn lichaam glad}{als een huid omsluit}\\

\haiku{Als Reprobus het,.}{kruis is genaderd is de}{duivel verdwenen}\\

\haiku{Velen, die over de,.}{rivier willen verliezen}{er hun leven in}\\

\haiku{De booze vraagt altijd.}{de onschuld en het  bloed}{der onschuldigen}\\

\haiku{De stemmen van rouw,.}{de steenen in het plaveisel}{der straat verstrakten}\\

\haiku{Ik moet eerst op de.}{boerderij de koeien nog}{gemolken hebben}\\

\haiku{- Gij moet zien, hoe ik,,.}{krullen smeed zegt de smid dat}{zijn nog andere}\\

\haiku{Jan Hamers zegt, over,:}{de onderdeur tegen den}{boer die buiten wacht}\\

\haiku{In al het lawaai.}{gaan de steigeringen van}{het paard verzwakken}\\

\haiku{De handen en de.}{mond van een donkere man}{waren daar dicht bij}\\

\haiku{In de gewelven.}{trilt het eerste licht van den}{nieuwen dageraad}\\

\haiku{In haar slaap blijven.}{de oogen voor de sterren en}{de maan geopend}\\

\haiku{Zeven keer trokken.}{de zon en de sterren hun}{banen over haar heen}\\

\haiku{Hij ging bij haar in.}{het zand zitten en begon}{met haar te praten}\\

\haiku{Het belgerinkel,,.}{dat werd gedragen er was}{een adem die dit droeg}\\

\haiku{Zozimus, zei de,,.}{vrouw ik moet u danken dat}{gij gekomen zijt}\\

\haiku{Hij vond haar lichaam,.}{dat levenloos op den grond}{lag uitgestrekt}\\

\haiku{Nadien hief hij met.}{een ruk den kop overeind en}{schudde de manen}\\

\haiku{De zachte witte,.}{visch was zoet en edel van smaak}{om daarbij te eten}\\

\haiku{Theodotus bleef.}{in dit klein huis verscholen}{totdat de avond viel}\\

\haiku{Ook zij dachten hem,,.}{met deze waarschuwing een}{dienst te bewijzen}\\

\haiku{Hij gaf het gezicht.}{aan blinden terug en den}{gang aan kreupelen}\\

\haiku{De haken sloegen.}{zich opnieuw in zijn vleesch}{en scheurden het open}\\

\haiku{Theodotus, zei,.}{hij ik zie u wijken voor}{de pijnigingen}\\

\haiku{Nu zal ik u eens,.}{van mijnen wijn doen proeven}{geef mij eenen beker}\\

\haiku{Toen zag Fronto het.}{in lakens gewikkelde}{onthoofde lichaam}\\

\haiku{Een warm gekleede.}{vrouw hield hare schreden in}{naast hare vriendin}\\

\haiku{De  Heer is met.}{u. Maria versnelt hun vlucht}{en verblijdt hun hart}\\

\haiku{Hij heeft zijn pet zoo,.}{gek en zijn haren dat is}{iets grappigs aan hem}\\

\haiku{Hij warmt zich daaraan,.}{dicht naar het warm licht van het}{kerstkind gebogen}\\

\haiku{Hij vroeg voor den man.}{mildheid en verzachting en}{de groote verlossing}\\

\section{Dirk Coster}

\subsection{Uit: Marginalia}

\haiku{Het natuurlijke.}{leven is een proces van}{zelfvernietiging}\\

\haiku{Zij vallen het kwaad, '.}{in den man aan daar waart}{belachelijk is}\\

\haiku{Zooveel de mensch reeds,.}{is zooveel verstaat hij van}{de Evangeli\"en}\\

\subsection{Uit: Waarheen gaan wij?}

\haiku{Deze jeugd heeft zich.}{niets te herinneren en}{niets te vergeten}\\

\haiku{E\'en tegenspraak blijft.}{voortdurend boven deze}{regelen zweven}\\

\section{Louis Couperus}

\subsection{Uit: Antiek toerisme}

\haiku{- Schuif de gordijnen,,.}{dicht Tarrar beval hij het}{Libysche knaapje}\\

\haiku{Verandering van.}{spijs is het geheim van een}{goede gezondheid}\\

\haiku{En dan een gekraak....}{van zware koorden over groote}{katrollen heen}\\

\haiku{De ziel is h\'een uit,,....}{deze met kostbare zalf}{gebalsemde huls}\\

\haiku{Het Diversorium.}{bestond uit verschillende}{lage gebouwen}\\

\haiku{- Het is zoo mooi als,,:}{het maar kan meester Ghizla}{zei oom Catullus}\\

\haiku{Romein was aldaar:}{op twee rijen geschaard om}{hem af te wachten}\\

\haiku{Oom Catullus was,:}{blijde dat hij geen oesters}{kreeg en geen pauwbraad}\\

\haiku{Alleen, voelde hij.}{zijn leed en smartelijke}{verdrietelijkheid}\\

\haiku{Maar zij bereikten,.}{den uitgang zonder dat er}{bloed was vergoten}\\

\haiku{Als van een fijne,,,....}{vrouw ijl en dun een schim die}{heen en we\^er bewoog}\\

\haiku{Want Tarrar, niet meer,:}{verbonden zag er uit als}{een kleine wilde}\\

\haiku{Hij wilde op het,,....}{tempeldak gehuld in den}{droomensluier droomen}\\

\haiku{Wees gij het, die zegt!}{uw Eroten de Droomen mijn}{Meester te brengen}\\

\haiku{- Wees gij het, die zegt....}{uw Eroten de Droomen mijn}{Meester te brengen}\\

\haiku{De stad ruischte.}{van muziek en gloeide van}{illuminatie}\\

\haiku{Lucius neeg ter,,.}{aarde zonk op de knie\"en}{en kuste den vloer}\\

\haiku{Ge hebt gedroomd de,.}{vele roovers die geleken}{als dubbelgangers}\\

\haiku{De meester snikte,.}{het hoofd omwikkeld in zijn}{gouden droomsluier}\\

\haiku{Serapis had de.}{hemelsluizen geopend}{en het regende}\\

\haiku{Deze stad, kind, is,.}{een ontaarde stad al is}{zij schoon om te zien}\\

\haiku{in het Muzeum,,.}{in het Serapeum hier}{en te Canope}\\

\haiku{Zoo ik het woord niet,.}{te Memfis vind zal ik}{het verder zoeken}\\

\haiku{Zij naderden nu,.}{de hoofdplaats Sa{\"\i}s hoofdplaats van}{geheel Laag-Egypte}\\

\haiku{Ik ben te oud en,,.}{te dik Lucius voor een}{spokige orgie}\\

\haiku{IK BEN, DIE GEWEEST,,}{IS   IS   EN ZIJN ZAL}{EN NIEMAND HEEFT}\\

\haiku{Zij wrong zich als een,.}{witte waternymf die den}{vloed was ontstegen}\\

\haiku{- Het zijn de zeeroovers,,;}{geweest Lucius zeide}{afwendend Thrasyllus}\\

\haiku{- Meester Thrasyllus zal,!}{het niet tegen spreken hoe}{geleerd hij ook is}\\

\haiku{Maar des Profeten.}{donderende lach deed hem}{deinzen achteruit}\\

\haiku{- Ik dacht aan Kos, mijn,.}{vaderland en of ik het}{wel ooit we\^er zo\^u zien}\\

\haiku{henen boort, als hij,...}{honger heeft en die dorpen}{en steden in slokt}\\

\haiku{- Treed dan binnen in,.}{het Huis van de Zon noodde}{de opperpriester}\\

\haiku{De meester ging de,,.}{woestijn in en Tarrar steeds}{verwonderde zich}\\

\haiku{- De stem in mijn ziel,.}{zelve die de orakels in}{mij deden spreken}\\

\haiku{in zijn muil goten,,.}{lachende de priesters een}{kruik hydromel uit}\\

\haiku{Ik smacht naar een paar.}{malsche oesters en een jong}{gebraden pauwtje}\\

\haiku{Maar zij aten ook bloed,.}{en melk en kaas en er was}{geen ander voedsel}\\

\haiku{- Dan zijt gij terug,,....}{gekeerd en waar gij zijt voel}{ik mij het veiligst}\\

\haiku{Van de jacht terug,....}{komende zag ik daar even}{de lijn van de zee}\\

\haiku{Morgen zijn we te,.}{Dire bij de kolommen}{van Sesostris}\\

\haiku{Zij was voor hem ne\^er.}{gestort en zij snikte en}{kuste zijn voeten}\\

\haiku{Op de kaap, hand voor,....}{de oogen zag Kaleb uit en}{verwonderde zich}\\

\haiku{Luisteren er geen?}{slaven aan de deuren en}{is Kaleb verre}\\

\haiku{Bleek verscheen Kaleb,.}{voor Lucius die hem had}{laten ontbieden}\\

\subsection{Uit: De berg van licht}

\haiku{En hij trok op steenen,.}{bank den knaap tot zich den arm}{om zijn schouder heen}\\

\haiku{Bloed van menschen, van,,...}{dieren duizende menschen}{duizende dieren}\\

\haiku{een azuren lap, die...}{viel voor haar ne\^er met druiping}{als van zwaar water}\\

\haiku{morgenbries woei aan.}{de ziel der Syrische en}{Perzische rozen}\\

\haiku{De rinkelbommen,.}{rammelden en rommelden}{de trommen allen}\\

\haiku{De Menigte zag;}{Bassianus achter den}{Steen ommedansen}\\

\haiku{toen hij we\^er voortrad, -.}{naderde hem de tweede}{deerne Livilla}\\

\haiku{mystiek-helder als -;}{een priester des Lichts zoo zo\^u}{hij altijd blijven}\\

\haiku{Heil de eerwaarde,!}{Moeza en heil Mammea en}{Semiamira}\\

\haiku{hij oefende zijn,:}{Latijn en hij sprak sierlijk}{en zuiver die taal}\\

\haiku{De verhevene.}{Julia Moeza vereert mij}{met haar vertrouwen}\\

\haiku{En dan, hij is een... -?}{Christen Maar de Christenen}{zijn toch in aanzien}\\

\haiku{een eenvoudige,.}{hulde die je zeker ook}{wel zult willen doen}\\

\haiku{Zeg, ze zeggen, dat,:}{die geen mannetje is en}{geen meisje maar niets}\\

\haiku{Een tijger sloeg zijn;}{klauw door de tralies in den}{schouder van een vrouw}\\

\haiku{groote, donkere oogen,.}{die de verre Menigte}{zochten te boeien}\\

\haiku{Het volk koos partij,,.}{voor het Oosten voor de Zon}{voor Helegabalus}\\

\haiku{De gunstelingen,:}{sloegen mantel of toga}{af vlijden zich ne\^er}\\

\haiku{Ik weet meer van je,,...}{af dan je denkt mijn brave}{oude pappias}\\

\haiku{Beste pappias,;}{ik ben een leerling van den}{kn\`apsten Magi\"er}\\

\haiku{Hydaspes, die mij,,,...}{helaas niet gevolgd heeft heeft}{zelve mij geleerd}\\

\haiku{Want ik ben wel een,,?}{rank ventje voor zestien vindt}{je niet Maximinus}\\

\haiku{U is mooi als ik... -.}{nooit een knaap zag Maar ik ben}{ook Helegabalus}\\

\haiku{- Alleen om u trouw,!}{te dienen o zoon van mijn}{vroegere keizers}\\

\haiku{aan mijn vleesch en,!}{frisch in mijn bloed maar verwend}{word ik niet van daag}\\

\haiku{- Den eersten keer, dat,!}{hij je ziet spreekt hij tot je}{en duldt hij je kus}\\

\haiku{h\`em, Helegabalus,,}{hongerde naar h\`em voelde}{zich gloeien naar h\`em}\\

\haiku{Daar zij hem telkens,...}{anders zagen herkenden}{zij hem niet altijd}\\

\haiku{Des morgens had het,,;}{Hof hem gezien nijdig en}{norsch om Alexianus}\\

\haiku{nauwlijks konden de;}{acht dikken zich nestelen}{op de sigma klein}\\

\haiku{zij begrijpen niet,,;}{goed zij hebben noch de vrouw}{noch het kind herkend}\\

\haiku{Aanbiddelijk was,.}{hij zeer zeker bekoring}{oefende hij uit}\\

\haiku{Toen ik viel, stortte,.}{je op mij toe vroeg je mij}{of ik gewond was}\\

\haiku{- Je genade is,.}{onmetelijk je liefde}{zal mij alleen zijn}\\

\haiku{- Dat je alleen dreigt,,.}{Antoninus omdat je}{te veel mij lief hebt}\\

\haiku{Ik noem je voortaan,,.}{niet anders o mijn liefde}{dan Antoninus}\\

\haiku{De auriga droeg,;}{een lang zijden feestgewaad}{met gemmen bezaaid}\\

\haiku{en ik zo\^u wel heel...... -?}{veel van hem houden als ik}{niet in hem zag Wat}\\

\haiku{Hij is \`alles, hij!}{is de Zonneziel in heel}{haar veelvuldigheid}\\

\haiku{aan een zwarten baard.}{hangt een levend robijntje}{en tikkelt we\^er ne\^er}\\

\haiku{Is het omdat de,?}{avond kil is en zij zoo heel}{lang hebben gewacht}\\

\haiku{Hij stottert, hij durft,,;}{niet zeggen dat hij Hierocles}{aanbidt zijn Gemaal}\\

\haiku{Man van vorm, vo\`elde,...:}{hij zich al vrouw en had hij}{gehuwd zijn Gemaal}\\

\haiku{- Ik ben de alom!}{uitstralende Dubbelheid}{van Helegabalus}\\

\haiku{Keizerin zal je,.}{zijn en mijn vrouw middel tot}{mijn vervolmaking}\\

\haiku{maar zij had hooger.}{geklonken en zij dreigde}{te gelijker tijd}\\

\haiku{Zoo zuiver mikte,;}{de keizer dat hij er geen}{enkele miste}\\

\haiku{- Maar Matthias en...}{Ganadasa wisten zich}{staande te houden}\\

\haiku{het harde marmer,;}{van de levende statue}{die zijn lichaam was}\\

\haiku{klokkebengelend...,}{datura's en zij vielen}{zij vielen allen}\\

\haiku{Een begin van brand;}{laaide op in een hoek van}{het Triclinium}\\

\haiku{Mammea was uit het,.}{Vrouwenhof aangesneld zij}{ook in nachtgewaad}\\

\haiku{Severa rees op,,...}{opende de bronzen deur en}{zag omzichtig uit}\\

\haiku{- Waarom anders dan'!}{om de bezoedeling van}{Alexianus standbeelden}\\

\haiku{hij wierp zich \`op hem,';}{zijn vierkante knie drukte}{Antoninus borst}\\

\haiku{Misschien gingen wel,...}{handen schuw naar Alexander uit}{om hem te streelen}\\

\haiku{En zijn moeder... zijn......!}{moeder konkelt tegen mij}{tegen mijn gezag}\\

\haiku{- Hij is de Caezar,,!}{je bloedeigen neef en je}{aangenomen zoon}\\

\haiku{IV Dien middag kwam.}{de Senaat in de Oude}{Hoop ter audientie}\\

\haiku{zij roepen buiten:}{de wallen en grachten van}{het zomerpaleis}\\

\haiku{roept Antoninus,,:}{we\^er en waar zij knielen wijst}{hij tien slaven aan}\\

\haiku{Aristomachos,...}{is niet gekomen als toch}{was afgesproken}\\

\haiku{Ze dachten van hem,,...}{te eten van hem te drinken}{van hem te leven}\\

\haiku{Nu duwt zij en dringt...}{als een razende dwars door}{die volksmassa's door}\\

\haiku{lijken van vrouwen......}{in feestgewaad lijken van}{leeuwen en narren}\\

\haiku{haar zoon is een g\`od.........}{we\^ergeboren gebaard}{uit haar liefdelijf}\\

\haiku{je zoon is een knaap,,,,!}{een lieve flinke jonge}{Romein en niet meer}\\

\haiku{hij werpt zich - velen -;}{hooren toe plat over den grond}{aan Moeza's voeten}\\

\haiku{Onder geen keizer.}{is ooit gezien zulk een pracht}{van kavallerie}\\

\haiku{maar Alexander vermoord,,...}{als verrader die stond naar}{des keizers leven}\\

\haiku{Voor den drang van het...}{Volk zijn de opgestelde}{troepen bezweken}\\

\haiku{Had zij Alexander maar,.}{keizer gemaakt en hem te}{Emessa gelaten}\\

\haiku{de deur splintert breed,.}{open en door die opening wringt}{Hierocles zich binnen}\\

\haiku{Gij naamt ze aan, gij,!}{naamt ze aan gij naamt aan al}{die waardigheden}\\

\haiku{ik zal het niet te:}{erg maken en denken aan}{mijn Hollandsch publiek}\\

\haiku{het eerste deel zo\^u:}{bijna als apart boek kunnen}{gegeven worden}\\

\haiku{Het eerste deel is,.}{apart te lezen en wellicht}{voor grooter publiek}\\

\haiku{deel i en ii in,.}{november 1905 deel iii in}{februari 1906}\\

\haiku{{\textquoteleft}1 October zal:}{je het geheele boek in}{copie bezitten}\\

\haiku{Met het derde deel.}{werd begin 1906 waarschijnlijk}{hetzelfde gedaan}\\

\haiku{Gecorrigeerd is {\textquotedblleft}{\textquotedblright}'.}{sc\'ene in de vierde zin}{van Couperus tekst}\\

\subsection{Uit: De boeken der kleine zielen. Deel 1 en 2}

\haiku{Dorine trad in,.}{de zitkamer van haar bro\^er}{Karel van Lowe}\\

\haiku{dat Constance toch;}{even goed als wij allen een}{kind van mama was}\\

\haiku{Er heerscht een groote,.}{sympathie een warm gevoel}{tusschen allemaal}\\

\haiku{Je maakte altijd.}{de boterhammen voor de}{kinderen van Bertha}\\

\haiku{zoo geleidelijk,...}{tot de misdaad als had het}{niet anders gekund}\\

\haiku{Beiden hadden zij.}{wel hun leven in moeten}{zien als \'een groote fout}\\

\haiku{Maar moesten zij niet eerst,?}{weten hoe de familie}{hen ontvangen zo\^u}\\

\haiku{Mama heeft ons w\`el......}{het voorbeeld gegeven om}{h\`artelijk te zijn}\\

\haiku{Maar met Henri, met.}{Van der Welcke k\`on ik}{me niet uitspreken}\\

\haiku{Ik ben alleen in.}{Den Haag gekomen om veel}{van jullie te zien}\\

\haiku{Zij trok hem ook naar,.}{zich toe hield haar zuster en}{haar kind dicht bij zich}\\

\haiku{En hij hielp haar zich.}{wat uitkleeden en schikte}{hare kussens op}\\

\haiku{Bertha was al bijna -.}{geheel grijs grijzer zelfs dan}{mama Van Lowe}\\

\haiku{- Het is, alsof je,!}{het niet goed vindt dat je zoon}{op zijn vader lijkt}\\

\haiku{Ze doen veel goed, ze,...}{leven voor hun kinderen}{ze zien veel menschen}\\

\haiku{Die komen ook nooit.}{op hun diners en jours en}{bals etcetera}\\

\haiku{voor je vader om.}{je te omhelzen en te}{houden op zijn schoot}\\

\haiku{in de stem van het,:}{kind was een teederheid als}{wilde hij zeggen}\\

\haiku{de eiken deuren,.}{der vertrekken die zwijgend}{bleven gesloten}\\

\haiku{het voorhoofd welfde,.}{zich ivorig en hoog uit een}{dunnen krans grauw haar}\\

\haiku{hare schoonouders,.}{haar niet hadden willen zien}{als  een schande}\\

\haiku{- Ik merk, dat je nooit,.}{veel hebt nagedacht net zooals}{de meeste vrouwen}\\

\haiku{de oudste zoon nu:}{uit Indi\"e verwacht met vrouw}{en twee kinderen}\\

\haiku{zij genoot van het,,;}{solide degelijke}{officieele huis}\\

\haiku{Er was het Hof en.}{haar man had haar geleerd van}{grootheid te houden}\\

\haiku{Mama Van Lowe,.}{ging voorbij aan den arm van}{Otto van Naghel}\\

\haiku{Ik had het alleen,.}{heel beroerd gevonden als}{het weg was geraakt}\\

\haiku{de joviale...}{huzaar met de breede borst}{en de brandebourgs}\\

\haiku{Otto met Francis,! -,:}{waarom Zij voelde dat hij}{op de lippen had}\\

\haiku{- Omdat ze het zeer... -?}{eenzaam in Brussel hadden}{Maar de familie}\\

\haiku{- Zeg... die mevrouw Van...?}{der Welcke wat komt die}{hier eigenlijk doen}\\

\haiku{In een zitkamer,,.}{was Francis Otto's vrouw met}{de twee kinderen}\\

\haiku{Geen diners meer en,......}{japonnen en omslag om}{niets maar een geluk}\\

\haiku{Het is gelukkig,...}{dat Constance er niets op}{tegen zal hebben}\\

\haiku{Van der Welcke,,,.}{dan na den eten was blij dat}{het zijne beurt was}\\

\haiku{Waarom doen we het,,...}{dat alles al die drukte}{en al dien omslag}\\

\haiku{Dorine had maar,!}{eens moeten opperen dat}{regen nat maakte}\\

\haiku{alleen ho\^u ik niet...,,...}{van den snit van zijn vest te}{hoog vind ik zijn vest}\\

\haiku{hij had haar wel eens,...}{anders gezien dadelijk}{hooren uitvaren}\\

\haiku{Haar glimlach gaf een,.}{ronding aan haar wangen die}{haar verjeugdigde}\\

\haiku{Hoe ben je toch hier,...?}{komen wonen zeg tusschen}{twee kerkhoven in}\\

\haiku{en waarom is ze}{zoo verdraagzaam tegenover}{tante Adolfine}\\

\haiku{Kijk mevrouw Bruys haar...}{taartje eten met een bijna}{dierlijk genoegen}\\

\haiku{dat fluweel van den... -? -.}{kraag van Saetzema's rok Ja}{Dat is m\'ooi fluweel}\\

\haiku{zelfs Adolfine was,,...}{den laatsten tijd voor zij op}{reis ging heel aardig}\\

\haiku{Ja, zij vroegen het,:}{elka\^ar zonderdat zij het}{elkander zeiden}\\

\haiku{En de andere,...}{jongens ranselden Jaap af}{omdat hij het zei}\\

\haiku{Kom, Addy, bemoei...}{je voortaan maar niet veel met}{die boerenkinkels}\\

\haiku{Het was zijn eerste,,.}{verdriet en het was zoo zwaar}{zoo verstikkend zwaar}\\

\haiku{en omdat het dan,...}{alles voor niets was geweest}{niet voor mijn vader}\\

\haiku{wil je je brouilleeren.}{met de kinderen van een}{zuster van mama}\\

\haiku{En w\`at hij wist, woog,:}{hij in zijn naar zekerheid}{verlangende ziel}\\

\haiku{Maar toen schrikte Van,:}{der Welcke en als met}{een schok dacht hij na}\\

\haiku{- Waarom zouden wij...}{den kleinen jongen niet eens}{te logeeren vragen}\\

\haiku{- Ik dacht - en Van der,,.}{Welcke van het lachen}{schudde op en ne\^er}\\

\haiku{zoo jong, dat hij als,,.}{een bro\^er en zoo zwak dat hij}{als een kind was}\\

\haiku{- Ik kom je vragen.}{of je overmorgen op een}{diner bij me komt}\\

\haiku{- Je komst heeft dingen,...}{opgerakeld die al lang}{vergeten waren}\\

\haiku{Ook weet ik zeker,.}{dat je schoonouders je komst}{hebben afgekeurd}\\

\haiku{En om je huis te,,?...}{arrangeeren daar heb je ook}{niet veel idee van wel}\\

\haiku{Bertha doet die dingen,,?}{meer als vrouw van de wereld}{geloof ik niet waar}\\

\haiku{Ik dacht, dat je meer,.}{in de c\^oterie was van}{Bruis de Telefoon}\\

\haiku{mij misplaatst als een,,.}{indringer als iemand die}{niet is te avoueeren}\\

\haiku{Wie hebben wij, wie,!}{komt er bij ons bij wie zijn}{wij eenigszins in tel}\\

\haiku{In jaren had zij,.}{ze niet gezien in jaren}{niet van ze gehoord}\\

\haiku{er zo\^u niets komen,.}{en er zo\^u dus ook niets staan}{in den Dwarskijker}\\

\haiku{Ook dien dag bleef zij,,.}{thuis daar het stortregende}{en zij zag niemand}\\

\haiku{- Ik zo\^u heel gaarne,,...}{alles willen doen wat je}{mij vraagt Constance}\\

\haiku{Maar een andere,...}{gedachte gaf haar nieuwen}{strijdlust nieuwen moed}\\

\haiku{Mijn visite van... -!}{Dinsdag heeft hij mij kwalijk}{genomen Kwalijk}\\

\haiku{Voor mijn kind, en hem......}{later de carri\`ere}{de carri\`ere}\\

\haiku{Des te liever zal,...}{het mij zijn zoo ik niets met}{je te maken heb}\\

\haiku{Om h\'a\'ar... om zijn vrouw...!}{had hij Van Naghel op zijn}{gezicht willen slaan}\\

\haiku{en zelfs, dat zij haar...}{zoon achterliet had haar niet}{tot rede gebracht}\\

\haiku{al had hij ook aan,...}{haar vader geschreven zij}{zo\^u niet meer komen}\\

\haiku{- Ik heb ze verteld,,.}{dat je op reis was dat ze}{dus m\`oesten wachten}\\

\haiku{- Dat weet ik wel... - Maar......}{je moet niet zoo droefgeestig}{zijn op jou leeftijd}\\

\haiku{Maar dat ben ik al. -,... -,...}{Nu dan is het goed Kijk hoe}{donker is het Bosch}\\

\haiku{Van der Welcke,,.}{lachte blij gestreeld in zijn}{jonge ijdelheid}\\

\haiku{Er waren in mijn....}{leven veel vrouwen en toch}{waren ze er niet}\\

\haiku{- Ik ga lezingen,,.}{houden niet alleen hier maar}{overal in Holland}\\

\haiku{Men riep hem terug,,.}{maar hij kwam niet meer en het}{publiek stroomde weg}\\

\haiku{maar ik leefde van,,.}{mijn loon als een arbeider}{die ik toen ook was}\\

\haiku{vroeg Addy, die als,.}{er een vreemde was nooit me\^e}{ging naar den salon}\\

\haiku{- Kapitalisten,.}{zonder kapitaal lachte}{Van der Welcke}\\

\haiku{dat n\`a die jaren...,,...}{toen ik als Gerrit zeide}{een lief kindje was}\\

\haiku{- Wat meen je daarme\^e... -,.}{Die oude vriend van oom die}{over den Vrede spreekt}\\

\haiku{Soms... soms zo\^u ik hier... -?}{kunnen huilen Maak ik je}{zoo melancholiek}\\

\haiku{Er zijn menschen, die,,.}{nooit voelen en anderen}{die altijd zwijgen}\\

\haiku{Ik ben - ik weet niet -...}{waarom in een stemming om}{te schreien met u}\\

\haiku{Dat Marianne,.}{Van der Welcke naloopt}{zoodat het geen naam heeft}\\

\haiku{zei Adolfine, boos, {\textquoteleft}{\textquoteright}.}{omdat Floortje gesproken}{had vanoude vrouw}\\

\haiku{En Marianne,,.}{voor de variatie wordt}{verliefd op haar oom}\\

\haiku{In een hoek bij de.}{deur van de serre zaten}{de oude tantes}\\

\haiku{Zij voelde, dat Brauws,.}{naar haar keek en zij voelde}{dat Brauws nog boos was}\\

\haiku{- Dat komt, omdat ik,...}{het zoo prettig vind jullie}{bij me te hebben}\\

\haiku{Ik zeg andere,.}{dingen dan anderen en}{ik zeg ze anders}\\

\haiku{dan ga ik de straat,.......}{op ik weet niet waarheen naar}{Leiden naar Henri}\\

\haiku{Ik wil... - Ik wil, dat... -?}{je stil bent en geen sc\`ene}{maakt Waar is Emilie}\\

\haiku{En terwijl zij met,,:}{Henri de kamer verliet}{zeide zij hardop}\\

\haiku{Nu vond hij in de.}{kamer van Marianne}{zijn beide zusters}\\

\haiku{Maar als ik hem lief,......}{heb als ik hem lief heb als}{ik gelukkig ben}\\

\haiku{zij wilde als voor.}{hem en de anderen dooven}{dien te grooten glans}\\

\haiku{Marianne, zeg,... -?}{me dat het niet waar is Dat}{hij me het hof maakt}\\

\haiku{- was zij niet bang, of,.}{treurig want ze voelde in}{haar droom zich veilig}\\

\haiku{en niets blijft... niet de... -,...}{minste herinnering Neen}{dat gaat alles weg}\\

\haiku{Dat mijn leven heel....}{anders geweest moest zijn om}{goed te zijn geweest}\\

\haiku{hij heeft gevochten,...}{met Eduard is handgemeen}{met hem geworden}\\

\haiku{Misschien komt hij in,,...}{huis bij Adolf zijn voogd die heel}{streng voor hem zal zijn}\\

\haiku{Alleen met wat heel,...}{sympathiek was aan haar hield}{zij samenleven}\\

\haiku{Het was mij, of ik......}{in mijn kindersprookjes iets}{vermoedde van hem}\\

\haiku{in de intieme, -:}{schemeringen praatten zij}{dikwijls veel dacht zij}\\

\haiku{Zij vond zich als een.........}{schoolmeisje dat droomde en}{haar lessen leerde}\\

\haiku{hij voelde vol in...}{zich hun beider erfenis}{van jalouzie}\\

\haiku{want zoo dikwijls, mo\^e,...}{bewogen zag hij voor zich}{een ideaal van rust}\\

\haiku{God, God, wat zag ze,.........}{er aardig uit dat kleine}{ding en wij jongens}\\

\haiku{Constance schoof de,.}{tusschendeur open voerde het}{kind in den salon}\\

\haiku{zelfs tegenover Van,......}{Vreeswijck dacht zij zo\^u het}{misschien niet fair zijn}\\

\haiku{- Ja, ging zij voort, met,.}{die peinzend kalme stem een}{beetje gedwongen}\\

\haiku{Hare oogen vlamden,.}{op zij voelde zijn opzet}{haar te beleedigen}\\

\haiku{Arme Vreeswijck... -......,.}{Ja arme kerel zeide}{hij werktuigelijk}\\

\haiku{Er was een stilte,,,.}{terwijl zij zoo stond hij haar}{aanzag doordringend}\\

\haiku{Henri, laat ons het.......}{doen als het kan met iets van}{liefde voor elka\^ar}\\

\haiku{Toen zonk zij in een,.}{stoel terwijl om haar heen de}{kamer duizelde}\\

\haiku{Nou... je hebt lekker...,.}{liggen maffen zei Addy}{ruw makend zijn stem}\\

\haiku{Ik rust nu wel uit, -.}{nu ik het je zoo zeg en}{tegen je aan lig}\\

\haiku{toen voelde die... - Wat... -...;}{Dat die meer hield van jou dan}{van Marianne}\\

\haiku{Hij zo\^u hen immers...!}{toch spoedig verlaten zelf}{zijn leven zoeken}\\

\haiku{Zij, bevende, was,...}{gaan zitten omdat zij zich}{wankelen voelde}\\

\haiku{schel was de hoop, zoo,:}{verblindend dat hij haar eerst}{niet hoorde zeggen}\\

\haiku{de illuzie... - Ja......,.}{de illuzie sprak hij met}{een glimlach van pijn}\\

\haiku{En zij omhelsde,... -,.}{hem als vroeg zij vergeving}{Addy sprak zij zacht}\\

\haiku{De dagen waren,...}{langzaam voortgegaan de een}{na den anderen}\\

\haiku{De kleine zielen - {\textquoteleft}{\textquoteright}.}{geleidelijk in het net}{werd overgeschreven}\\

\haiku{Hij beloofde de}{kopij voor ongeveer half}{januari.49 Veen}\\

\haiku{{\textquoteleft}Ik hoop, dat ge van:}{ons ivoor-en-gouden boek}{pleizier zult hebben}\\

\haiku{\ensuremath{<} zijn h1236,32als hij ons, - -}{ziet fietsen \ensuremath{<} als hij ons}{fietsen ziet h1237,12wil}\\

\haiku{\ensuremath{<} lach, - die soms in,:}{een schater om vd W kon}{eindigen en zei}\\

\haiku{na h1370,10/11waait... alsof hij,...}{de ramen wil openen en}{binnen wil komen}\\

\haiku{h2461,24al \ensuremath{<} was het al}{h2463,20zonder \ensuremath{<} als zonder}{h2463,28bazuinen \ensuremath{<}}\\

\haiku{aan Veen, gedateerd,.}{31 december 1902 in het}{archief-Veen}\\

\haiku{Waarde Heer Veen, p..).}{221 75In h2 staan boven de}{komma drie puntjes}\\

\subsection{Uit: De boeken der kleine zielen. Deel 3 en 4}

\haiku{kinderen moet je,,}{hebben zonder kinderen}{heb je geen leven}\\

\haiku{De kussens van het;}{bed vertoonden maar even den}{indruk van Pauls hoofd}\\

\haiku{Gerrit sloot het raam,.}{de regen ruischte niet}{de kamer meer in}\\

\haiku{omdat ik me met...}{zorg kleed en je een kwartier}{langer laat wachten}\\

\haiku{De wereld is al,.}{smerig genoeg ook al is}{men nog zoo netjes}\\

\haiku{Hij keek, onder zijn,.}{parapluie razend naar}{den regenhemel}\\

\haiku{maar nu was hij den}{draad van zijn redeneering}{kwijt en daarbij moest}\\

\haiku{Ik heb ze om mij,.}{heen verzameld verzameld}{uit alle eeuwen}\\

\haiku{Soms zijn ze prachtig,...}{gekleed en zingen ze met}{heerlijke stemmen}\\

\haiku{Maar den laatsten tijd - -...}{hij schudde weemoedig het}{hoofd den laatsten tijd}\\

\haiku{Onder haar arm werd,,.}{hij hard als ijzer en boos}{hard zag hij haar aan}\\

\haiku{- Ja... - Dat de ploert ze... -...... -?}{niet wakker maakt en trapt Ja}{ja Beloof je dat}\\

\haiku{dat hij verkeerd deed......,.}{Als een vrouw waardig is slaat}{haar geen man mijn kind}\\

\haiku{Het is niet zoo, als....}{de jonge meisjes denken}{als ze verliefd zijn}\\

\haiku{en als hij dat doet.........}{lachen de menschen eerst en}{huilen ze daarna}\\

\haiku{het is ook een groot.......}{geheim voor de familie}{voor de kennissen}\\

\haiku{Ik heb juist gevoeld,...}{dat ik me los van alle}{banden moest maken}\\

\haiku{Zij vroeg zich af, hoe...}{zij het hier vier weken zo\^u}{moeten uithouden}\\

\haiku{Kunnen we niet eens,.}{het karretje huren dan}{zal ik je mennen}\\

\haiku{we kunnen het niet... -!}{iederen morgen nemen}{Iederen morgen}\\

\haiku{- Neen, hoor, zoo ben ik......}{niet om op mijn oom verliefd}{te worden voor niets}\\

\haiku{- De dokter heeft veel... -...,.}{hoop Ja zei Bertha nu alsof}{dat van zelve sprak}\\

\haiku{er is anders niets,......}{dan een beetje te zijn te}{doen voor anderen}\\

\haiku{Maar... soms... is het me... -,... -,...}{heel zwaar Kind mijn kind Ja soms}{is het me heel zwaar}\\

\haiku{Blind ook liep ze door,...}{dien droom van ijdelheid en}{ze dacht dat ze zag}\\

\haiku{Zij stonden op, en,,,,,.}{liepen voort de duinen op}{af op af zwijgend}\\

\haiku{Er gebeuren 's,,...}{nachts schandalen schandalen}{in alle kamers}\\

\haiku{Zij voelde zijn hand;}{op haar arm zwaar liggen als}{de hand van een man}\\

\haiku{zo\^u het leven en,.}{de loopbaan die zij voor zijn}{vader geknakt had}\\

\haiku{Ook niet aan papa,,.}{omdat ik voel dat hij het}{niet zo\^u begrijpen}\\

\haiku{Ernst, behoorende tot -;}{de donkere Van Lowe's}{het bloed van papa}\\

\haiku{iets van goedigheid:}{we\^erhield hem werkelijk}{driftig te worden}\\

\haiku{een aardig huisje,,,...}{een interieur een lief}{vrouwtje kinderen}\\

\haiku{de gedroogde visch,:}{die bij het bakken opzwol}{tot brosse schulpen}\\

\haiku{en daarbij is hij,...}{een melancholiek heer die}{zijn manie\"en heeft}\\

\haiku{Ik heb ook niet veel...,.}{honger jokte Gerrit die}{altijd honger had}\\

\haiku{En zeg me nu eens,,.}{Dorine ga je niet eens}{naar Nunspeet kijken}\\

\haiku{- Nu ja... dat weet ik,...?}{wel maar bij ons zo\^u je toch}{gezelliger zijn}\\

\haiku{Zij deed zich geweld.}{om de tranen in haar oogen}{terug te persen}\\

\haiku{Ach, mama weet wel,,...}{dat ze niet tegen je op}{kan niet waar Papa}\\

\haiku{Probeer u er me\^e,,.}{eigen te maken lieve}{oma dat ik niet mag}\\

\haiku{- Nou maar, ik vind er,,.}{niets aardigs aan en maak jij}{maar dat je weg komt}\\

\haiku{- Wacht jongens, papa,...}{moet eerst naar boven om zijn}{handen te wasschen}\\

\haiku{jij bent zoo lief en -...}{zoo zacht al ben je nog zoo}{een ruwe kerel}\\

\haiku{En - het was misschien -:}{allerstomst van hem}{hij had haar geloofd}\\

\haiku{Daar moeten we nu:}{altijd in bewondering}{voor ne\^erknielen}\\

\haiku{Zij zag vleierig,.}{tegen hem op haar handen}{streelden zijn lichaam}\\

\haiku{Zelfs den laatsten tijd...,... -? -.}{in Parijs Gerrit Wat Dacht}{ik wel eens aan jou}\\

\haiku{Al dat frissche - als, -;}{een vrucht waarin hij hapte}{van vroeger was weg}\\

\haiku{Zoo zo\^u het eenmaal -,...}{worden als hij heel oud heel}{oud was geworden}\\

\haiku{Nu was het zoo nog......}{niet nu daagde het nog uit}{het blonde troepje}\\

\haiku{Je denkt, dat je je...,...}{charme voor de eeuwigheid}{hebt Alles slijt kind}\\

\haiku{- Je hebt zeker nog......... -}{wel een portret een groepje}{van je kinderen}\\

\haiku{En zij zag ginds ver,,,,.}{weg te ver voor haar een vrouw}{oud als zij sterven}\\

\haiku{Maar je bent wel boos... -,,... -...,.}{geweest Stil stil mama Neen}{neen laat me spreken}\\

\haiku{Ik las, niet dat hij......}{ziek was maar dat hij uit zijn}{betrekking moest gaan}\\

\haiku{zo\^u in de toekomst -.}{dat hij uit dien verren dood}{terug zo\^u komen}\\

\haiku{Was het niet je nog,...}{wrokkende jeugd die niet de}{verzoening wilde}\\

\haiku{O, wat zo\^u er toch,!}{dreigen nu de oude vrouw}{ginds gestorven was}\\

\haiku{zo\^u die korrel niet...}{voldoende zijn om d\`at met}{wijsheid te dragen}\\

\haiku{hij, die nooit, na een,...}{gebroken carri\`ere}{had werken gekund}\\

\haiku{Het moedertje was:}{nog altijd overweldigd door}{haar stillen dollach}\\

\haiku{- Het is ook met dat... -,.}{beroerde we\^er Ja dat maakt}{de menschen somber}\\

\haiku{En ik zeg je ze,.}{en voel dat het nutteloos}{is ze te zeggen}\\

\haiku{Had hij nog koorts en,...}{had hij niet goed gedaan op}{te staan uit te gaan}\\

\haiku{Maar hij had niet meer,:}{kunnen in bed blijven hij}{had niet meer gekund}\\

\haiku{- Wil u dat aan de...}{juffrouw geven Stoppen in}{de hand van de vrouw}\\

\haiku{ze brengen het lijk,...}{van Henri me\^e oom en ze}{brengen Emilie me\^e}\\

\haiku{Was het leven dan...:}{niet meer gewoon Waren er}{dan niet als altijd}\\

\haiku{Constance opende,...}{de deur der eetkamer haar}{arm om Emilie heen}\\

\haiku{Zij snikte, als gek,.}{ne\^ergezonken tegen}{Constances knie\"en}\\

\haiku{Want hij zag niet meer:}{door de kamerwanden het}{heelal en het Beest}\\

\haiku{Nu was hij zoo ver,}{aangebeterd dat hij zat}{in een ruimen stoel}\\

\haiku{Onderzoek eens, wat,... -...}{er van is wil je Hoe heet}{zij en waar woont ze}\\

\haiku{- Ja, kerel... - Je wordt,... -,...}{nu we\^er beter h\`e Ja ik}{word nu we\^er beter}\\

\haiku{bleef het niet altijd,...}{kil en koud ook al laaide}{het nog zoo hoog op}\\

\haiku{Wat zocht ze in de,...}{diepe diepte wat smeet ze}{om hem heen het zand}\\

\haiku{Het is beter, dat... -,.}{je ook gaat Addy is nu}{al gegaan mama}\\

\haiku{Ze zijn nog niet naar... -... -.}{bed En Addy Addy moet}{er wel binnen zijn}\\

\haiku{Zeg haar... zeg haar nog......,....}{niets zeg haar zeg haar dat De}{wanhoopssnik schreeuwde}\\

\haiku{Toen staken zij het,.}{licht op en brachten zij de}{oude vrouw naar bed}\\

\haiku{hij, die toen, kleine,...}{jongen gegaan was naar den}{dikken aannemer}\\

\haiku{als het arme kind...}{nu maar niet ziek werd van die}{lange wandeling}\\

\haiku{O, wat was het een,...}{sombere gang de eiken}{deuren we\^erszijden}\\

\haiku{Maar trappeltripjes,:}{naderden snel en nu aan}{de deur klop-klop}\\

\haiku{En ze is de vrouw.}{van Addy en de moeder}{van zijn kinderen}\\

\haiku{Hij groette hier en,,}{daar in het rond maar kuste}{niemand gaf geen hand.}\\

\haiku{wat kan ik er aan... -,.}{doen Je moest eens eerlijk met}{me spreken zei hij}\\

\haiku{Maar als we daarheen,,.}{verhuizen Tilly moeten}{we heel zuinig zijn}\\

\haiku{- Nu, zeide zij met.}{haar matte gepiqueerde}{onverschilligheid}\\

\haiku{Addy fronste zijn,.}{brauwen en dat gaf hem een}{pijnlijk droeven trek}\\

\haiku{Tusschen de zorgen,:}{voor oom Gerrits kinderen}{had hij ze gevoeld}\\

\haiku{voor zichzelven wist,...}{hij niets en vooral wist hij}{het niet voor zijn ziel}\\

\haiku{De wind kwam binnen,.}{en blies met \'een ademtocht het}{licht uit van de lamp}\\

\haiku{- Ik ben altijd stil... -,,.}{Probeer eens Alex vertrouwen}{in me te hebben}\\

\haiku{Nu, laten wij dan...}{nu niet anders spreken dan}{over de Handelsschool}\\

\haiku{- Hij zal wel veel geld... -...,.}{al verdienen Zoo Niet zoo}{heel veel geloof ik}\\

\haiku{Ik zo\^u niet gaarne,...}{hebben dat hij Marietje}{hypnotizeerde}\\

\haiku{- Het is zoo kil aan... -,...}{de voeten Guy geef tante}{een voetenbankje}\\

\haiku{De ontzenuwing,:}{vereffende zich het lunch}{eindigde rustig}\\

\haiku{Adolfine was heel,,.}{ontroerd met roode huiloogen die}{zij telkens wischte}\\

\haiku{Ik heb geen geld om......}{om langen tijd ergens met}{haar buiten te gaan}\\

\haiku{Meneer is altijd,,.}{vreemd niet waar maar hij is niet}{lastig en vrij wel}\\

\haiku{In het grauwe licht.}{van het kleine kamertje}{rees het meisje op}\\

\haiku{Zeg, Addy, dat zijn,?}{alle de kinderen van}{oom Gerrit niet waar}\\

\haiku{O, zij had hem niet,,:}{getrouwd om zijn geld of zijn}{titel dat ook niet}\\

\haiku{haar rozigblanke,,!}{tint haar volle vormen haar}{jong-sterke leden}\\

\haiku{Zij had nog maar even,:}{den tijd zich te kleeden haar}{schaatsen te nemen}\\

\haiku{- Neen, dat is immers... -:}{het beste Het is altijd}{het zelfde geluid}\\

\haiku{Marietje zocht naar,.}{Constance om de sleutels}{van de linnenkast}\\

\haiku{Ik ben het type... -,...}{van een ouden vrijer Je}{moest nog trouwen Paul}\\

\haiku{Het is een mooie das,...}{maar ik heb er niet meer zoo}{een collectie van}\\

\haiku{Gelukkig, jullie,.}{kibbelen niet en jij zelfs}{niet meer met je man}\\

\haiku{- Neen, niet alleen... - U... -.}{heeft zich aan elka\^ar gewend}{Zonder veel woorden}\\

\haiku{Addy... zoo bij u...... -... -? -...}{altijd Mijn arme jongen}{Waarom Ik ben bang}\\

\haiku{- De innerlijke... -,}{dingen niet Wees gelukkig}{dat uw leven zoo}\\

\haiku{- Win haar dan... - Het is... -}{zoo heel moeilijk en waar er}{geen sympathie is}\\

\haiku{Zij ging  in de:}{aangrenzende kamer naar}{hare kinderen}\\

\haiku{En dan, dan hem ook,...}{jaloersch te maken van haar}{als zij was van hem}\\

\haiku{Waarom spreekt u van,... -...}{Den Haag en wat geven wij}{nu om een bal Juist}\\

\haiku{Hoe meen je kind... - Niet,,...}{zoo als tante Constance}{en Emilie en u}\\

\haiku{O hij hield van de,,... -...}{kinderen maar hield hij van}{ha\`ar zijn vrouw Addy}\\

\haiku{- Het is wel jammer,,...}{Tilly dat je het hier zoo}{weinig kunt schikken}\\

\haiku{voor niets... - Probeer te,......}{voelen Tilly dat ik me}{niet afsloof voor niets}\\

\haiku{- had hij schuld, - omdat!}{hij zijn vrouw alleen liefhad}{met de helft van zich}\\

\haiku{Ik zal doen als je,.}{zegt ik zal me in Den Haag}{een praktijk zoeken}\\

\haiku{Een nieuwheid, banaal,,;}{en frisch als de verf van haar}{huis was om haar heen}\\

\haiku{- Je hebt die koele,,,}{ver-affe stem kerel}{die ik zoo goed van}\\

\haiku{men zocht, de dames -,.}{vooral omdat hij er goed}{uitzag baron was}\\

\haiku{s avonds naar hare,,,...}{familie naar kennissen}{theedrinken alleen}\\

\haiku{het was mooi zacht, en.}{de lente weefde groentjes}{tusschen de boomen}\\

\haiku{meneer Erzeele -,}{zij kuste Mathilde gaf}{Erzeele de hand.}\\

\haiku{- Dat weet ik... - Hij kwam.......}{een afspraak maken om te}{tennissen morgen}\\

\haiku{En hoe de jongen...}{dat kind haar geest heeft weten}{te ontwikkelen}\\

\haiku{zoo heeft je leven,,...}{een doel zelfs mijn leven ook}{al doe ik zelf niets}\\

\haiku{Hare stem klonk als,.}{een stem van vroeger zeide}{dingen van vroeger}\\

\haiku{die jongen is te.}{gezond om altijd in die}{boeken te zitten}\\

\haiku{Maar juist dat alles............}{die gesprekken dat afscheid}{daarvoor vreesde hij}\\

\haiku{hij nam den brief voor... -....}{Adeline me\^e Hoe het haar}{te zeggen dacht hij}\\

\haiku{het is of het een......}{eigen zoon van me is die}{me heeft verlaten}\\

\haiku{Ik ho\^u van mijn man...... -,.}{van mijn kinderen en ik}{zo\^u Ja zeide hij}\\

\haiku{- Ja... maar... - Je houdt niet -... -...}{van me. Niet zoo Je zo\^u toch}{gelukkig worden}\\

\haiku{Hier - hij zocht in zijn -,.}{zak hier is een brief van Guy}{uit New-York}\\

\haiku{Het was na een avond.........}{dat hij gespeeld had in het}{circus en Eduard}\\

\haiku{Wat klonk hun gesprek,...}{zoo vertrouwelijk wat klonk}{het treurig bijna}\\

\haiku{Ik verwachtte je,,,,.}{morgen zei bleek Mathilde}{ondanks zichzelve}\\

\haiku{- Hier neem ik afscheid,,.}{zei Erzeele toen zij de}{zijstraat insloegen}\\

\haiku{ook is het beter... -... -...,... -}{Wat Dat je terug keert naar}{Driebergen Addy}\\

\haiku{- Dat je zoo veel van... -,...}{hem zal houden Ik weet het}{niet ik weet het niet}\\

\haiku{o God... nu zo\^u ik.........}{zoo gaarne wenschen om hen}{voor zijn kinderen}\\

\haiku{{\textquoteright} Het laatste deel zou {\textquoteleft} [...]:}{de cyclus  afsluiten}{met een nieuw belang}\\

\haiku{De laatste sc\`ene:}{van het Heilige Weten}{pakte mij erg aan}\\

\haiku{{\textquoteleft}Nu niet brommen dat:}{Het Late Leven wat kort}{is uitgevallen}\\

\haiku{Wil ik U eens een?}{paar bladzijden zenden om}{een proef te nemen}\\

\haiku{voltooid.55 Couperus.}{beloofde de proeven af}{te maken voor april}\\

\haiku{Op verzoek van Veen:}{gaf Thieme een overzicht van}{de stand van zaken}\\

\haiku{Ja56,35zijn van \ensuremath{<} gaan,}{voor57,23een \ensuremath{<} dan een58,24/25moest dat hij}{energie moest hebben}\\

\haiku{Zij had60,3gevonden.}{\ensuremath{<} geraden60,3zij had \ensuremath{<}}{zelve had zij61,5leven}\\

\haiku{de drie meisjes, de:}{jongen Herman100,19dan lachten}{tante en nichtjes}\\

\haiku{\ensuremath{<} niet om235,8/9twee,, -,}{machteloos \ensuremath{<} twee -235,9en}{van \ensuremath{<} van235,19kamer}\\

\haiku{\ensuremath{<} keer, alleen250,28die,}{\ensuremath{<} die zelfs251,11Constance}{\ensuremath{<} zij251,26zal \ensuremath{<} moet}\\

\haiku{Zelfs Alex272,3lang aan \ensuremath{<} aan,,}{272,4bijna in \ensuremath{<} bijna}{ziende om zich heen}\\

\haiku{- Ik ben altijd je,...-.}{vriend vadertje~ Ben je}{het nog altijd}\\

\haiku{De avonden363,23lazen \ensuremath{<},.}{stil lazen364,5een trap \ensuremath{<} de trap364,8hun}{\ensuremath{<} zijn364,34tusschendeur}\\

\haiku{een fout zoo men wil,, [...].}{een vergissing zeker maar}{zoo heel verklaarbaar}\\

\subsection{Uit: Brieven van Louis Couperus aan zijn uitgever}

\haiku{Het rekenen in.}{geld correspondeert met het}{rekenen in tijd}\\

\haiku{Enige malen deelt:}{hij Veen het geheim van zijn}{grote werkkracht mee}\\

\haiku{Ik hoop, dat wij het.}{eens zullen worden over de}{volgende edities}\\

\haiku{Tot mijn spijt kan ik;}{niet v\'oor 10 September in}{Amsterdam komen}\\

\haiku{Zoo mogelijk, zo\^u.}{ik het overige ook zeer}{gaarne ontvangen}\\

\haiku{Oscar Wilde Het.}{Portret van Dorian Gray}{vertaald door Mevr}\\

\haiku{U schreef eerst over 2de:}{druk van Extaze en nu}{van Lent v Vaerzen}\\

\haiku{Het boek zelf verschijnt.}{in de Duitse vertaling}{van Dr. P. Rach\'e}\\

\haiku{Vroegere Verzen,.}{in 1895 te verschijnen als}{Williswinde}\\

\haiku{Semiramis, twee.}{gedichten uit Gouverneurs}{Onze Huisvriend.46 5}\\

\haiku{Heeft de Josselin?}{de Jong met U gesproken}{over Majesteit}\\

\haiku{Het zo\^u mij leed doen.}{zoo U zich hier niet mede}{kon vereenigen}\\

\haiku{Achtend L.C. 62.}{Baarn   ten huize van Mr.}{J.J. van Santen}\\

\haiku{Maar laat mij voortaan,.}{een woordje me\^espreken waar}{het den band aangaat}\\

\haiku{Het boek is prettig.}{los ingebonden en ziet}{er nog al flink uit}\\

\haiku{Wanneer denkt U over []?}{hoek weggescheurd uitgave}{van Extaze}\\

\haiku{revizie is niet,.}{noodig als U er nog eens streng}{het oog over laat gaan}\\

\haiku{Ik weet niet, of ik,;}{ze afmaak hoewel ik er}{veel pleizier in heb}\\

\haiku{natuurlijk met de,!}{belofte dat U de Gids}{niet te vroeg inhaalt}\\

\haiku{verschijnt de rest in,.}{de Gids iets minder dan het}{eerste gedeelte}\\

\haiku{Heeft U dus nog niet,.}{met Minden afgesproken}{denk dan eens over hem}\\

\haiku{Van een tweeden druk.}{Majesteit alleen een paar}{exemplaren aan mij}\\

\haiku{ik meen, letters die,.}{niet goed aansluiten in het}{woord maar dwarrelen}\\

\haiku{L.C. ~ Mag ik U?}{de zorg van deze Poolsche}{dame opdragen}\\

\haiku{Ook de uitgave.}{in boekvorm laat daarna niet}{lang op zich wachten}\\

\haiku{Rotterdam Dokter:}{Vlaanderen Hilversum}{Zend U mij de ex}\\

\haiku{maar liever als ik.}{U mijn adres te Veneti\"e}{heb opgegeven}\\

\haiku{maar ik vind beider:}{uiterlijk niet zoo geslaagd}{als vorige keeren}\\

\haiku{trouwens Hooge Troeven,?}{alleen is toch te klein voor}{een boekje niet waar}\\

\haiku{Van Antonius,.}{zal ik wel drie proeven noodig}{hebben denk ik}\\

\haiku{Steeds gaarne Uw dw..}{L.C. 140 Rome   H\^otel}{du Sud   4.3.96}\\

\haiku{Met vriendelijke.}{groeten Steeds gaarne Uw dw}{L.C. 147 den Haag}\\

\haiku{Wat Else Otten,.}{betreft zal ik mij gaarne}{aan Uw raad houden}\\

\haiku{154 Parijs   18.}{Rue Chateaubriand   5.12.96}{Waarde Heer Veen}\\

\haiku{Het is heel aardig,.}{ook dat artikel in de}{Bergensche revue}\\

\haiku{Met vriendelijke.}{groeten van mijne vrouw}{Steeds gaarne Uw dw}\\

\haiku{Mag ik rekenen, {\textquoteleft}{\textquoteright}?}{dat U mij delauwers en}{disteltakken zend}\\

\haiku{ontving, kan ik U.}{nu zekere gegevens}{voor Psyche geven}\\

\haiku{Als de corrector,.}{het zorgvuldig doet laat ik}{het maar aan hem over}\\

\haiku{Steeds gaarne Uw dw:}{L.C. ~ Mevrouw informeert}{naar de vertaling}\\

\haiku{Wij zullen het heel:}{gezellig vinden U eens}{in Brussel te zien}\\

\haiku{In vriendelijken.}{dank het douceurtje van {\textflorin}}{25- ontvangen}\\

\haiku{Mag ik U dan mijn}{schuld terug betalen met}{1e druk Fidessa}\\

\haiku{mag ik dus hopen:}{de rest te ontvangen aan}{het adres van mijn broer}\\

\haiku{Daarme\^e is mij dus:}{voldaan de uitgave 1ste}{druk van den roman}\\

\haiku{Finantieel geeft,.}{het wel nooit iets maar daar leg}{ik mij maar bij ne\^er}\\

\haiku{B.:180 die las ik al..}{Andere kritieken heb}{ik helemaal niet}\\

\haiku{Het is mogelijk,,.}{dat dat alles zoo is maar}{het kwam mij nieuw voor}\\

\haiku{Ik zend U heden.}{in vriendelijken dank Uw}{testament terug}\\

\haiku{Wij rekenen er:}{vast op U te zien van den}{winter in Nice}\\

\haiku{kunt ge dus de drie?}{eerste vellen Tweede Deel}{nog eens zenden}\\

\haiku{Hoog Ed. Gestr Heer,,,.}{J.R. Couperus Rezident}{Bezoeki Java}\\

\haiku{Ik had gaarne 6,.}{exemplaren gebonden en}{niet gebonden}\\

\haiku{morgen gaan wij over,;}{in ons huis maar wij wachten}{nog onze meubels}\\

\haiku{en zitten dus met.}{een beetje een primitief}{comfort om ons heen}\\

\haiku{ik geloof, dat Van,??}{Hall ook geschrikt is maar is}{het heusch zoo erg}\\

\haiku{Louis Couperus 259.}{Nice   Villa Jules}{24.I.I. ~ Amice}\\

\haiku{Over een paar weken.}{zal het wel heelemaal af}{en in orde zijn}\\

\haiku{ik zeggen dat Van}{Deyssels kritieken ze ook}{verloren hebben:210}\\

\haiku{in boek IV is mijn,,}{kleine held nu 13 jaren}{een man.- Zoo ge}\\

\haiku{Het Late Leven ...,.}{vordert goed en wordt heel mooi}{al zeg ik het zelf}\\

\haiku{Ook de andere,.}{boeken zullen goed worden}{zullen we hopen}\\

\haiku{Juist na mijn brief kwam,;}{het pakket aan maar drukwerk}{gaat altijd vlugger}\\

\haiku{{\textquoteleft}Metamorfoze{\textquoteright}.}{en Fidessa's kopje vind ik}{altijd het mooiste}\\

\haiku{bv. Dan meld ik U.}{later precies hoeveel die}{twee bedragen zijn}\\

\haiku{het is eenvoudig.}{verregaande slordigheid}{met het afdrukken}\\

\haiku{Ik hoop, dat ge van:}{ons ivoor-en-gouden boek}{pleizier zult hebben}\\

\haiku{Wil ik U eens een?}{paar bladzijden zenden om}{een proef te nemen}\\

\haiku{Terwijl in \`al de!}{boeken de Hollandsche lucht}{grauw en angstig dreigt}\\

\haiku{Het heele huis ligt!}{overhoop en alles ruikt naar}{de naftaline}\\

\haiku{303 Wiesbaden   .}{Promenade-H\^otel}{10.VI.II ~ Amice}\\

\haiku{Nu niet brommen dat:}{Het Late Leven wat kort}{is uitgevallen}\\

\haiku{dan vrolijkt ge mij.}{wat op en dat heb ik wel}{noodig.- ~ Steeds t.\`a.v}\\

\haiku{in de toekomst zal.}{Het heilige Weten den}{cyclus voltooien}\\

\haiku{Is in Stefanie?}{misschien zelfs een portret van}{M\'elanie gegeven}\\

\haiku{Intussen is het.}{voor Couperus niet alles}{onbewolkt genot}\\

\haiku{Ik hoop in Rome.}{wat goed we\^er te krijgen en}{geen influenza}\\

\haiku{Het is dus alleen,.}{de finantieele kwestie}{die ik even aantik}\\

\haiku{Ik zal de proeven.}{van het Heilige Weten}{klaar maken voor April}\\

\haiku{Mijn schuld hoop ik je,}{spoedig af te doen maar het}{duurt ontzettend lang}\\

\haiku{Ach, beste kerel,,.}{ieder heeft het zijne een}{pakje te dragen}\\

\haiku{Dionyzos heeft.}{hij in de voorafgaande}{maand Juli voltooid}\\

\haiku{Met Veen overweegt hij.}{nu een complete editie}{van al zijn werken}\\

\haiku{Couperus laat de:}{maar al te juist gebleken}{opmerking volgen}\\

\haiku{Enfin, zoodra,.}{ik het geld heb wo\^u ik met}{je de schuld afdoen}\\

\haiku{Ik zal er zelve.}{van de laatste revizie}{nazien en herzien}\\

\haiku{mijn arme vrouw doet.}{wonderen om de boel aan}{de gang te houden}\\

\haiku{en Maart, iedere.}{maand {\textflorin} 500.- dan zo\^u ik}{je zeer verplicht zijn}\\

\haiku{In haast- ~ t.t.,!}{L.C. ~ Ik hoop dat het goed}{met U allen gaat}\\

\haiku{Mag ik je morgen?}{er een doos geconfijte}{vruchten voor zenden}\\

\haiku{Wij hebben het ook,,:}{nog geloof ik in een oud}{familie-album}\\

\haiku{Wil je een contract?}{opstellen over den roman}{van dezen zomer}\\

\haiku{Villa Jules, St..}{Maurice Nice ~ t.t. Louis}{Couperus        IV}\\

\haiku{Van den zomer heb,:}{ik niet gewerkt maar nu ga}{ik goed aan den gang}\\

\haiku{ik zal het niet te:}{erg maken en denken aan}{mijn Hollandsch publiek}\\

\haiku{Ik schrijf, of liever.}{geef voortaan geen letter meer}{uit in het Hollandsch}\\

\haiku{Ik vertelde U.}{reeds dat het is de roman}{van Helegabalus}\\

\haiku{het eerste deel zo\^u:}{bijna als apart boek kunnen}{gegeven worden}\\

\haiku{zelfs al zo\^u het niets,.}{zijn om we\^er het verlies in}{evenwicht te brengen}\\

\haiku{{\textflorin} 4500.-voor drie,.}{deelen waar je mij {\textflorin} 3000}{geeft voor twee deelen}\\

\haiku{Het eerste deel is,.}{apart te lezen en wellicht}{voor grooter publiek}\\

\haiku{Ik herhaal, dat ik;}{later gaarne bereid ben}{tot een concessie}\\

\haiku{En daarom kom ik,.}{nog eens hooren wat je denkt}{van mijn laatsten brief}\\

\haiku{Wil je het Eerste,:}{Deel voor het najaar dan is}{dat natuurlijk goed}\\

\haiku{1 October zal:}{je het geheele boek in}{copie bezitten}\\

\haiku{Ziet Couperus toch?}{op tegen het verschijnen}{van De Berg van Licht}\\

\haiku{zijn schoonmoeder is.}{bereid De Berg van Licht in}{ontvangst te nemen}\\

\haiku{Zend ook proef van dat,,.}{prospectus hoor want ik moet}{het beslist nazien}\\

\haiku{- Over het algemeen.}{ben ik niet erg tevreden}{over het afdrukken}\\

\haiku{Zend mij s.v.p. zoo veel.}{mogelijk kritieken en}{uitscheldpartijen}\\

\haiku{heel belangrijk vind,.}{ik het werk niet en het vult}{maar mijn koffer}\\

\haiku{Mijn vrouw zit rustig,.}{aan mijn zijde en is niet}{in Holland geweest}\\

\haiku{ik maak iederen;}{dag groote wandelingen en}{het is niets te warm}\\

\haiku{Wat wil je, het is{\textquoteright}.}{het eenige wat ik kan een}{gevaarlijk besluit}\\

\haiku{De waarheid is, dat}{er geld verdiend moet worden.345}{Aan Emma Garzes}\\

\haiku{het beste wensch ik.}{van mijn kant toe aan U en}{Uw huisgezin}\\

\haiku{Er is geen bezwaar ().}{dat de bundelselk in 2}{deelen verschijnen}\\

\haiku{Antwoord mij s.v.p. zoo,.}{spoedig mogelijk daar ik}{beslissen moet}\\

\haiku{Er is geen kwestie:}{van auteursrecht verkoopen}{aan het Vaderland}\\

\haiku{Van hier nodigt hij:}{Emma Garzes uit ook naar}{Rome te komen}\\

\haiku{we zullen dus maar {\textquoteleft}{\textquoteright}.}{niet meer denken overnog een}{bundel voor dit jaar}\\

\haiku{anders blijft het, als,.}{de Berg tien dagen liggen}{aan het Postkantoor}\\

\haiku{Wil je het boek in,.}{1 of 2 deelen geven}{en in welk formaat}\\

\haiku{wat pleizier hebt van}{mijn uitgaven en zal mijn}{best doen de boeken}\\

\haiku{L.C. 487 M\"unchen   .}{Pension Grebenau}{Wittelsbacherplatz}\\

\haiku{Maar ik kan heusch!}{zoo moeilijk reclame voor}{mijzelven maken}\\

\haiku{zo\^u je er reeds over}{kunnen denken wat je van}{mij kunt uitgeven}\\

\haiku{zend revizie, die.}{ik per ommegaande}{zal terugzenden}\\

\haiku{j'en fais mon deuil, als,:}{de Franschman zegt wanneer}{hij niet zeggen wil}\\

\haiku{ik zend je nog het,}{bundeltje waar je recht op}{hebt maar zo\^u je zoo}\\

\haiku{de Italiaanse.}{vriend is op 11 Juni naar}{Smyrna vertrokken}\\

\haiku{zij gaan mij ook niet.}{aan en ik heb heusch al}{genoeg aan mijn hoofd}\\

\haiku{Ik zo\^u je alleen;}{willen vragen er nog wat}{mede te wachten}\\

\haiku{Geef de bundels (met,):}{het eerste dat je reeds hebt}{uit in den vorm van}\\

\haiku{gumtig, voor gunstig,.}{Keyk voor Keyx terwijl overal}{anders goed Keyx staat}\\

\haiku{Zend mij ook proef van.}{Inhoud met de verwijzing}{naar Het Vaderland}\\

\haiku{Ik heb het nu in.}{het Duitsch gelezen en het}{is een prachtig boek}\\

\haiku{een hartelijken.}{handdruk van ons beiden en}{steeds gaarne ~ t.\`a.v}\\

\haiku{Ik hoop [dat] het niet,.}{verloren is want ik heb}{er geen brouillons van}\\

\haiku{L.C. 524 Florence.}{Pension Rochat}{Via dei Fossi 16}\\

\haiku{het is mijn laatste,.}{en ben er dus een beetje}{difficiel me\^e}\\

\haiku{Meld mij even ontvangst,,}{want als zij weg zijn is het}{een heele moeite}\\

\haiku{Vindt U het goed als?}{ik U de vertaling zend}{einde Augustus}\\

\haiku{Is de andere,...}{titel dus niet verkoopbaar}{ga je gang dan maar}\\

\haiku{Hij schrijft dus slechts zijn,:}{feuilletons en gaat door met}{die te bundelen}\\

\haiku{Ik wo\^u je alleen,,}{zeggen beste vriend dat ik}{voortaan toch wel we\^er}\\

\haiku{Couperus had die.}{in 1914 ontvangen voor zijn}{Antiek Toerisme}\\

\haiku{Louis Couperus ~ ??}{Is er papier voor de twee}{andere boekjes}\\

\haiku{Over een paar dagen,.}{zijn ze klaar en dan zend ik}{het heele deeltje}\\

\haiku{het publiek houdt er.}{niet van als de titel niet}{zijn belofte houdt}\\

\haiku{De Verliefde Ezel,.}{uit te geven die in Het}{Vaderland verscheen}\\

\haiku{Is de copie zoo?}{voldoende of wilt ge er}{nog een stukje bij}\\

\haiku{de prachteditie (van:).}{{\textflorin} 25.- het ex is zoo}{goed als uitverkocht}\\

\haiku{Het bundeltje zo\^u,.}{zeer verzorgd moeten worden}{breed en kort formaat}\\

\haiku{Ontvingt ge mijn brief,??}{waar over ik U het een en}{ander voorstelde}\\

\haiku{Ik had ook onlangs}{nog een prettige briefkaart}{van hem ontvangen.555}\\

\haiku{{\textquoteright} - Op donderdag 19:}{Juli schrijft Van Eeden dan}{de laatste regel}\\

\haiku{Couperus heeft la.}{Duse mogelijk maar een}{of twee keer ontmoet}\\

\haiku{Zijn naam stond voluit.}{op de achterzijde van}{zijn portretfoto's}\\

\haiku{Uit Engeland zijn.}{geen brieven van Couperus}{aan Wilde bekend}\\

\haiku{Een Lent van Vaerzen.}{was in 1884 bij J.L. Beyers}{te Utrecht verschenen}\\

\haiku{s Avonds draag ik nu,}{mijn oranje roos maar of ik}{het altijd zal doen}\\

\haiku{138Schertsenderwijs,.}{voor Trooper Peter naar de}{Engelse titel}\\

\haiku{Van een vertaling.}{van Noodlot van haar hand is}{het nooit gekomen}\\

\haiku{197De {\textquoteleft}beruchte{\textquoteright}.}{vrouwelijke hoofdpersoon}{uit De Stille Kracht}\\

\haiku{228Ten Brink is in.}{1901 op 67-jarige}{leeftijd overleden}\\

\haiku{289Vermoedelijk,.}{Mr. Dr. Willem Frans Donker}{Curtius geb. 1882}\\

\haiku{304Dit is, volgens,.}{wens dus van Couperus zelf}{zo ook uitgevoerd}\\

\haiku{Als patriot en.}{nationalist heeft hij}{veel gepubliceerd}\\

\haiku{Na de oorlog werd.}{hij ondersecretaris}{van Schone Kunsten}\\

\haiku{De twee bedongen.}{bundels groeiden tenslotte}{uit tot een vijftal}\\

\haiku{Roelvink nam regie.}{en mise-en-sc\`ene}{voor zijn rekening}\\

\haiku{Niet uit afgunst, maar '.}{omt publiek dat hem nu}{hemelhoog verheft}\\

\subsection{Uit: Nippon}

\haiku{De komedie                     ,.}{die hij anders speelde was}{nu werkelijkheid}\\

\haiku{Gedecideerd, hij,:}{w\`as een Mandarijn van het}{oude r\'egime}\\

\haiku{Dit is nog maar om.}{en bij Nagasaki}{en dit is nog niets}\\

\haiku{Er zijn steeds zieke.}{oogen                     en huidziekten aan}{wie u omringen}\\

\haiku{Gonse gaf                     rijk.}{ge{\"\i}llustreerde deelen}{over Japansche kunst}\\

\haiku{Maar ik geloof, dat.}{zij den geschiedschrijver te}{legendarisch zijn}\\

\haiku{Waardeeren doet hij zijn;}{landtongen en kaapjes en}{kronkelboomen}\\

\haiku{De voet                     zoekt den,.}{grooten steen zelfs na regen}{altijd droog en rein}\\

\haiku{Dat de Japanner,:}{hybridisch is zullen}{wij vaak bemerken}\\

\haiku{Zoowel het een als het:}{andere treffen tot in}{het uiterste}\\

\haiku{Dit laatste te doen;}{scheen het                     correctste en}{meest esthetische}\\

\haiku{Ik weet niet hoe ge,.}{over geld denkt                     maar ik meen}{dat als ik 3000 gld}\\

\haiku{iets ignobels is,.}{geschikt om duizend ziekten}{op                     te roepen}\\

\haiku{Het oude                     goud,.}{is het mooiste het is maar}{even dof of verweerd}\\

\haiku{plotseling draait de,,:}{gids die naast                     den chauffeur}{zit zich om en zegt}\\

\haiku{Zij verzorgt mij, wat.}{krachtdadig en toch teeder}{en toch weldadig}\\

\haiku{- Maar Araya, zeg ik, gel\`o\`of,??}{je nu dat je neef door een}{vos was bezeten}\\

\haiku{ik weet alleen, dat}{ik geen lust                     heb naar den}{Grooten Muur te gaan}\\

\haiku{O, ik ben maar een,:}{vlugge toerist maar ik}{ken mij een recht toe}\\

\haiku{Zeker om mij te,}{straffen slaat de vos klauwen}{uit en                     rijten}\\

\haiku{Het is zeer treffend,.}{hoe zeer velen hunner op}{apen gelijken}\\

\haiku{Nergens in dit land,.}{voel ik den gloed van                     een}{geestelijk Ideaal}\\

\haiku{En wij zouden om '?!}{elf                     uurs morgens in}{Yokohoma zijn}\\

\haiku{Hij is m\'e\'er voor hen,.}{dan de hoogste en schoonste}{berg van                     hun land}\\

\haiku{Want de dagen zijn,;}{lichtelang de zon rijst zeer}{vroeg ter kimme}\\

\haiku{Toen was beschaamd de.}{knaap en gaf aan de engel}{haar rok                     terug}\\

\haiku{De                     tuinman moet.}{van ieder boompje weten}{na hoeveel weken}\\

\haiku{Eens vond hij een dik,.}{zwaar boek                     over marine}{in een boekwinkel}\\

\haiku{- Kawamoto zeg ik,,?}{opstaande willen wij de}{karpers gaan voeden}\\

\haiku{Zijn er vijf, zes te,,.}{zamen dan spoort hij uren lang}{om ze te                     zien}\\

\haiku{Tot het schutsel, dat.}{rondom hen was opengeplooid}{hun aandacht trok}\\

\haiku{Helaas, als men geen,.}{schrijver geboren is kan}{men geen boek schrijven}\\

\haiku{Probeer ze nu eens.}{te gaan zien in den tempel}{van                     Horiuji}\\

\haiku{Haar vader, de zwaar,.}{gebouwde                     generaal}{hield veel van het kind}\\

\haiku{Zij had ook nog haar {\textquoteleft}{\textquoteright}, {\textquoteleft}{\textquoteright},.}{nurse hareama die}{alles voor haar was}\\

\haiku{Navrant is het leed.}{van het zieker en zieker}{wordende vrouwtje}\\

\haiku{de generaal brengt.}{zijn                     zieke dochter naar}{zijn buitenverblijf}\\

\haiku{Zoo ver van ons staand.}{als het Oosten maar                     kan}{staan van het Westen}\\

\haiku{Er is echter ook,.}{een verzameling goudlak}{die bizonder is}\\

\haiku{Het is                     klein, dit,,.}{kerkhof dit tempeltje een}{beetje rommelig}\\

\haiku{Zij zijn fun\`ebre,,.}{boomen grafboomen als}{onze cypressen}\\

\haiku{Zij rijen zich links,.}{en rechts meer dan                     twintig}{kilometer lang}\\

\haiku{zo\^u ik t\`och                     een?}{indruk krijgen van wat ik}{zo\^u hooren en zien}\\

\haiku{Wij zagen nog de.}{Wind-in-de-Pijnen}{en den Herfstregen}\\

\haiku{Ik herinnerde,:}{mij dien Slapenden Vos door}{Tetsuzan                     geschilderd}\\

\haiku{Ik herinnerde:}{mij de schilderij van}{Hiroshig\`e}\\

\haiku{Dit heeft in Japan.}{wel een heel ander type}{dan in China}\\

\haiku{Hij vindt de zonde;}{van de Yoshiwara}{verschrikkelijk}\\

\haiku{het groote steenen pad dwars,;}{door het mos zoodat het mos}{nooit betreden wordt}\\

\haiku{hij lijkt meer op een -:}{ouden                     markies dan op}{een zeebonk beslist}\\

\subsection{Uit: Noodlot}

\haiku{- Ik ben Robert van,... -,?}{Maeren misschien herinnert}{u zich Bertie jij}\\

\haiku{En dan altijd een, '!}{hooge hoed ens avonds altijd}{een rok met een bloem}\\

\haiku{Hij zo\^u zich schudden,...}{uit zijn zieleslaap hij zo\^u}{Bertie wegzenden}\\

\haiku{Bertie vroeg naar den;}{duur van de wandeling en}{wat men er zien zo\^u}\\

\haiku{Jullie kakelen.}{ook maar in plaats van eens naar}{het pad te kijken}\\

\haiku{De weg was breeder,}{geworden zij stegen dus}{gemakkelijker}\\

\haiku{Zij knikte en zij.}{hielpen beiden haar af te}{stijgen van de steenen}\\

\haiku{Hurk fatalistisch,;}{ne\^er als een Arabier en laat}{dag volgen op dag}\\

\haiku{Want hij was zooals hij,!}{was hij w\`as laf en kon zich}{niet veranderen}\\

\haiku{Toen een glas water,.}{en hij legde zich we\^er zich}{dwingend tot kalmte}\\

\haiku{Eigenlijk verborg,,...}{hij Frank dus Eve niets dan dat}{Bertie geen geld had}\\

\haiku{Intusschen, Bertie,.}{moest het dulden dat Frank zeer}{koel tegen hem werd}\\

\haiku{zijn fatalisme,.}{was als een godsdienst die hem}{sterkte en hoop gaf}\\

\haiku{Frank stond reeds op, om,.}{hem naar zijn kabinet te}{volgen Bertie ook}\\

\haiku{riep zij eindelijk,,,.}{uit toch nog vreezende hem}{Bertie te kwetsen}\\

\haiku{Hij is zoo open, zoo,...}{oprecht je weet zoo precies}{wat je aan hem hebt}\\

\haiku{Omdat hij een nieuw?}{leven wilde beginnen}{en nu niet meer kan}\\

\haiku{hij zo\^u daar, als het,.}{niet gerechtvaardigd is boos}{over kunnen worden}\\

\haiku{Maar dan komt het we\^er... -,,.}{terug Heusch Eve praat niet}{zulke gekkepraat}\\

\haiku{Het komt en het gaat,...}{voor een poosje weg en het}{komt en het gaat we\^er}\\

\haiku{Toe, o toe, spreek met,,}{hem enkele woorden maar}{ik bid er je om}\\

\haiku{Zij sidderde meer,:}{en meer en toen kwam het we\^er}{over haar en in haar}\\

\haiku{{\textquoteright} zeggen jullie, en.}{daarom noemen jullie niets}{wat ik wel iets noem}\\

\haiku{Maar... was het waarlijk,?}{de schuld van Frank dat hij Eve}{niet vergeten kon}\\

\haiku{Het is treurig, dat,...}{het zoo geworden is maar}{gooi het van je af}\\

\haiku{Je kan het dus niet,,.}{meenen als je zegt dat je}{er naar toe wilt gaan}\\

\haiku{In het halfduister.}{stiet Frank bij eene beweging}{even Bertie's hand aan}\\

\haiku{Ik heb net wat we.}{noodig hebben om naar Buenos}{Ayres te komen}\\

\haiku{Nou goed, hoor, ik zal,,:}{wel eens zien maar ga nu naar}{bed want ik heb slaap}\\

\haiku{Ho\^u me dicht tegen,,...}{je aan zoo in beide je}{armen in beide}\\

\haiku{Neen, neen... - Geloof je,... -... -?}{dan dat hij er belang bij}{had Ja Maar wat dan}\\

\haiku{Toen ik later over,.}{zijn woorden nadacht heb ik}{er dat in gevoeld}\\

\haiku{Maar Frank greep hem bij,:}{zijn schouders schudde hem en}{heesch brulde hij}\\

\haiku{En wat heb ik in...}{dat geslinger om me in}{evenwicht te houden}\\

\haiku{Het verleden werd:}{meer en meer het verleden}{en moest het blijven}\\

\haiku{ze wisten het toch...}{altijd beter en deden}{toch altijd hun zin}\\

\haiku{Integendeel, Eve:}{vreesde nu de geheele}{week voor dien Zondag}\\

\haiku{vroeg zij, verwonderd,.}{door heure tranen blikkend}{om zijn vreemden toon}\\

\haiku{Hare stem vloeide,}{zoet als balsem zij voelde}{om hem te sterken}\\

\haiku{Je hebt gezien, dat.}{Bertie een schurk was en je}{hebt hem doodgemaakt}\\

\haiku{Maar ik, ik voel, dat}{ik alles in mij mis om}{gelukkig te zijn}\\

\haiku{- Niets te veel, ik heb,.}{geleefd door jou zonder jou}{had ik nooit geleefd}\\

\haiku{Als de corrector,.}{het zorgvuldig doet laat ik}{het maar aan hem over}\\

\subsection{Uit: Oostwaarts}

\haiku{Ik doorleefde op...}{Java een schooljongenstijd}{van vijfjaren}\\

\haiku{vroegste uchtendmist.}{strekt dun mousseline uit}{over lucht en water}\\

\haiku{Stel je voor, dat de {\textquoteleft}{\textquoteright}?}{Prins der Nederlanden weg}{stoomde zonder ons}\\

\haiku{De vrouwen, die wij,,.}{zien zijn dikwijls blond al zijn}{ze Italiaansche}\\

\haiku{de loods besliste,,...}{echter dat het land was een}{meer en geboomte}\\

\haiku{Dan we\^er de engte,...}{van het kanaal in naar het}{Bittere Meer}\\

\haiku{Dat dringt zich aan je:}{op om je alles en nog}{wat te verkoopen}\\

\haiku{vertoonen, die U.}{nog treft aan zulke oude}{huisjes in Indi\"e}\\

\haiku{Het is een rijkdom,.}{een overstelping als alles}{is in het Oosten}\\

\haiku{eene verdieping is,;}{niet ouderwetschIndiesch en}{meer iets nieuwerwetsch}\\

\haiku{Het Europeesch,.}{effort dat hier zoo krachtig}{tot rezultaat kwam}\\

\haiku{De planter heeft zijn.}{eigen dagverdeeling en}{zijn eigen costuum}\\

\haiku{De gastvrouw is reeds,,.}{naar nieuwen trant gekleed in}{keurig wit toilet}\\

\haiku{Sarong en kabaai.}{worden nergens meer door de}{dames gedragen}\\

\haiku{Hij tjankoelt dit fijn,}{hij legt zijn zaadbedden aan}{en onderhoudt den}\\

\haiku{Ginds zijn de loodsen.}{en de administrateur}{komt ons te gemoet}\\

\haiku{De vrouwelijke.}{bloem kleurt zich van wit naar rood}{toe tot purperzwart}\\

\haiku{de afval dient tot.}{stookmateriaal van de}{locomobielen}\\

\haiku{De kar is een met,,:}{den weg een met de natuur}{een met het landschap}\\

\haiku{er me\^e, en beter,,.}{gesoldeerd hoor dat er geen}{drup ontsnappe}\\

\haiku{daarbij, de heeren.}{officieren zijn trouwe}{lezers van de h.p}\\

\haiku{- Dan zal ik voortaan,!}{zorgen dat de weg beter}{onderhouden wordt}\\

\haiku{Een rivier kronkelt.}{te voorschijn en verdwijnt we\^er}{tusschen blokken rots}\\

\haiku{Dit water noemt de, {\textquoteleft}{\textquoteright}:}{Maleier-om-de-kust}{desmakelooze zee}\\

\haiku{De Minnaar en de,.}{Bruid zij beheerschen voet bij}{voet den horizon}\\

\haiku{Nu zie ik al die.}{huizen weer en je lacht om}{die souvenirtjes}\\

\haiku{Maar hij had immers,!}{geen kindersouvenirs die}{hem bedrongen}\\

\haiku{Men houde zijn rijst.}{zelve zoo lang mogelijk}{wit en maagdelijk}\\

\haiku{Hij is te winnen,.}{door een enkel vriendelijk}{woord door een glimlach}\\

\haiku{Hij vindt het prettig.}{zijn meester een titel van}{grootheid te geven}\\

\haiku{Sangkoeriang roept,.}{zijn diengeesten te zamen}{zijne dewata's}\\

\haiku{Ik ken deze mooie.}{landen en deze wegen}{sinds twintig jaren}\\

\haiku{hier v\'o\'or u, tusschen,,}{deze bergen op deze}{aarde zij heffen}\\

\haiku{Vroolijk blijven zij,,,}{en hongerig geloof ik}{zijn zij of doen zij}\\

\haiku{er is nu niets aan,.}{te doen aan het feit dat acht}{koelies mij torsen}\\

\haiku{Af wiegelt het nu,.}{op hun schouders den bergweg}{af het oerwoud door}\\

\haiku{zijne ziel bleef nog.}{immer een feodale}{en middeneeuwsche}\\

\haiku{Het was geen spel van,.}{oorlog en helden het was}{een spel van liefde}\\

\haiku{De prins is reeds van,.}{kind af verloofd met zijn nicht}{prinses Schartadja}\\

\haiku{Een Boeddhistische,,.}{non zuster des konings komt}{den prins vermanen}\\

\haiku{Zij geven kniekus - -.}{en voetkus zeer lang deze}{kussen aan allen}\\

\haiku{Hoe moet de toerist}{het bewonderen als de}{Rezidenten zoo}\\

\haiku{Maar als ze hi\`er,...}{begonnen zijn is het d\`a\`ar}{al we\^er vol onkruid}\\

\haiku{het is zeker een,,.}{geschenk een kleine hulde}{van ik weet niet wie}\\

\haiku{Wat een vetuste!}{overblijfselen van langzaam}{wegteerende macht}\\

\haiku{de {\textquoteleft}translateur{\textquoteright} van,,.}{den Rezident vertolkte}{ze mij helaas niet}\\

\haiku{(Waarom heet deze {\textquoteleft}{\textquoteright} {\textquoteleft}{\textquoteright}?}{heer maar niet op zijn Hollandsch}{tolk ofvertaler}\\

\haiku{dan is het Rijk van.}{Kediri in deze tijden}{tot macht gekomen}\\

\haiku{De riffen in zee:}{geleken op den bouwtrant}{van poort en tempel}\\

\haiku{Het was met niets van.}{de groote Soenda-eilanden}{te vergelijken}\\

\haiku{nu en dan scheen het,;}{mij toe dat hij de stem der}{demonen nadeed}\\

\haiku{Zij gingen de hooge,,.}{gebeeldhouwde trap op in}{edele theorie}\\

\haiku{is even natuurlijk.}{als dat de rietstengel nijgt}{voor de waringin}\\

\haiku{Ik verlaat het schip,,.}{loop langen steiger af tuf}{naar het hoofdbureau}\\

\haiku{Een jonge man van ().}{wellicht nog geen veertigik}{schat moeilijk leeftijd}\\

\haiku{Dat is dus wel iets {\textquoteleft}{\textquoteright}.}{voor het Nederlandscheffort}{om trotsch op te zijn}\\

\haiku{\ensuremath{<} rij der bergreuzen}{staan153,6zingen \ensuremath{<} springen153,10in}{\ensuremath{<} op153,15kleine \ensuremath{<}}\\

\subsection{Uit: De stille kracht}

\haiku{De  marmeren;}{vloer van de voorgalerij}{spiegelde gladwit}\\

\haiku{Zij was bang, dat hij,;}{boos zo\^u blijven en dan had}{ze niets en niemand}\\

\haiku{Wat zij had willen,,...}{zijn als zij niet behoefde}{te zijn die zij was}\\

\haiku{- Die andere... zijn......,.}{\`ook valsch Mevrouw Van Does zag}{haar aan met pleizier}\\

\haiku{- Mevrouw Van Does laat,.}{ons een heele boel moois zien}{zeide zij streelend}\\

\haiku{Zij liet den brillant.}{even flonkeren en de steen}{schoot een blauwen straal}\\

\haiku{Het moet heusch van,,.}{boven zijn gevallen uit}{de goot door het raam}\\

\haiku{Hij was wat zwaar en,}{had aanleg nog zwaarder te}{worden maar toch had}\\

\haiku{Maar hij had ook een,.}{prettigen toon met ze al}{was het werken zwaar}\\

\haiku{Hij kon joviaal,.}{vriendschappelijk zijn al was}{hij de rezident}\\

\haiku{Hij zeide het niet,.}{ronduit maar iets ontsnapte}{hem in den Regent}\\

\haiku{Een paar jongelui,.}{in het wit wandelden en}{namen den hoed af}\\

\haiku{Eensklaps waren de}{huizen gedaan en langs een}{breeden weg strekten}\\

\haiku{Zij had zich gewend,;}{aan het spel der teenen aan de}{mest om de rozen}\\

\haiku{Zij was aan die haar,,:}{kenden of antipathiek}{of zeer sympathisch}\\

\haiku{En om iets goeds tot,,.}{stand te brengen heerschte}{zij met haar clubje}\\

\haiku{Zij stelde rok en,.}{witte das in en zij was}{onverbiddelijk}\\

\haiku{Het mannelijke.}{element mengde zich niet met}{het vrouwelijke}\\

\haiku{ze heeft die blauwe...}{irissen zelve geschilderd}{op Chineesche zij}\\

\haiku{Zij was vol kleine,,;}{geheimzinnige nukjes}{haatjes liefde-tjes}\\

\haiku{De rezident, hoe,.}{koel praktisch ook heeft daarin}{iets van een po\"eet}\\

\haiku{- Laten wij heusch,...}{eerlijk zijn anders is er}{geen aardigheid aan}\\

\haiku{- Ik moet over een half,...}{uur bij den rezident zijn}{maar ik ben te vroeg}\\

\haiku{Er is niets wat ik.}{buiten het geluk in mijn}{huis zoo hoog waardeer}\\

\haiku{De oudste dochter,;}{was gehuwd met een volbloed}{blonden Hollander}\\

\haiku{Zijn tuin was vol van,;}{bloemen die  zich alle}{hieven naar hem toe}\\

\haiku{maar wat het ook was,,;}{zij zag het dadelijk met}{een enkelen blik}\\

\haiku{Zocht in den cirkel,.}{een voet den hare zij trok}{den hare terug}\\

\haiku{Zij was verwonderd,.}{dertig te zijn en dit voor}{het eerst te voelen}\\

\haiku{Aan alle hoeken,.}{van het huis luisterden de}{bedienden talloos}\\

\haiku{de familie liep.}{buiten in den tuin en in}{de tjemara-laan}\\

\haiku{Zij hijgde naar adem,,,.}{half in zwijm steeds de kabaia}{open de haren los}\\

\haiku{Nooit had hij zijn vrouw,,.}{iets geweigerd hoe kostbaar}{het was wat zij vroeg}\\

\haiku{Een ontroering van.}{drukte voer veertien dagen}{door Laboewangi}\\

\haiku{de regenmoesson,,.}{onveranderlijk trad in}{op St. Nicolaas}\\

\haiku{Zij had zich nog nooit,.}{zoo gevoeld maar er was niet}{tegen te strijden}\\

\haiku{met een blikje op...?}{uw deur en een rijksdaalder}{in den hoeveel tijd}\\

\haiku{een blank gordijn van.}{regen daalde in rechte}{plooien van water}\\

\haiku{Zij misten in haar,.}{de vroolijkheid die hen eerst}{had aangetrokken}\\

\haiku{- In den waringin,,.}{van het achtererf hoog in}{de hoogste takken}\\

\haiku{Van Oudijck had,,.}{haar gehoord hij stond op kwam}{van achter het schut}\\

\haiku{Door den plassenden.}{tuin zagen zij een witte}{gedaante komen}\\

\haiku{- Als het iets is... stel,,.}{dat het iets is dat wij niet}{verklaren kunnen}\\

\haiku{Zij gaf gil op gil,.}{geheel krankzinnig van het}{vreemde gebeuren}\\

\haiku{En eens, 's avonds, even,.}{een paar passen me\^egaande}{met hem vroeg zij hem}\\

\haiku{zij maakte alle,}{jonge vrouwen en meisjes}{jaloersch en daar}\\

\haiku{genegenheid wat.}{zij er door haar behaagzucht}{in verloren had}\\

\haiku{Theo wilde zij niet,.}{meer en moederlijk deed zij}{voortaan met  hem}\\

\haiku{Want hij was het kind.}{van zijn moeder meer dan de}{zoon van zijn vader}\\

\haiku{Maar hij was te lui,.}{en te weinig helder om}{kwaad te kunnen doen}\\

\haiku{een hoogere plaats -.}{die altijd de lijn van zijn}{leven ge-weest was}\\

\haiku{onduidelijke,,.}{ridders het eene been vooruit}{in de hand een speer}\\

\haiku{Hij trok haar nu in,.}{zijn armen maar zij duwde}{hem zachtjes terug}\\

\haiku{Op Patjaram, je,,.}{oude moeder je zusters}{alles lik je maar}\\

\haiku{Men belde elka\^ar,.}{op om niets alleen om het}{pleizier te bellen}\\

\haiku{Waarlijk, Eva vond het.}{te Laboewangi toch nog}{veel gezelliger}\\

\haiku{Ziet u eens, hier in,.}{Indi\"e ben ik wat geweest}{daar zo\^u ik niets zijn}\\

\haiku{Het land heeft zich van.}{mij meester gemaakt en ik}{behoor het nu toe}\\

\haiku{Het was alles glad,.}{voor mij uit ten minste in}{mijn carri\`ere}\\

\haiku{Ik hield van mijn vrouw,,:}{ik hield van mijn kinderen}{ik hield van mijn huis}\\

\haiku{Mijn eerste vrouw was,}{een nonna die ik trouwde}{omdat ik verliefd}\\

\haiku{Hij zweeg even, toen ging,,:}{hij voort geheimzinniger}{fluisterender nog}\\

\haiku{Wat zal u blij zijn,.}{uw ouders te zien en mooie}{muziek te hooren}\\

\subsection{Uit: Van oude menschen, de dingen, die voorbijgaan...}

\haiku{hij had Lot, om te,.}{werken een van de beste}{kamers gegeven}\\

\haiku{Hij zeide het kalm,.}{heel onverschillig en nam}{zijn courant we\^er op}\\

\haiku{Laat Steyn nu rustig,,...}{wandelen denk niet meer aan}{Steyn denk aan niets meer}\\

\haiku{Dokter Thielens meent,...}{een voorbode van spoedig}{totaal doof worden}\\

\haiku{Lid van den Raad van,;}{Indi\"e nu reeds jaren lang}{gepensioneerd}\\

\haiku{- Je zo\^u misschien, met,,....}{wat tact het kunnen doen kind}{bij de Steyns wonen}\\

\haiku{Naar mevrouw Dercksz... -... -,,...}{Naar grootmama Ja ja zeg}{nu maar grootmama}\\

\haiku{In hoogruggigen,,,}{stoel als in troon zat zij recht}{gesteund door een stijf}\\

\haiku{Ik geloof, dat, \`als...,.}{ik gestorven ben hij hij}{me zal aanklagen}\\

\haiku{Dat zeg ik niet om,:}{je iets onaangenaams te}{zeggen hoor jongen}\\

\haiku{Notaris Deelhof... -?,:}{zei nog verleden Hoeveel}{zei Ina begeerig}\\

\haiku{Nu bevrijdt hij zich,:}{uit de tulle plooien en}{we\^er wil hij roepen}\\

\haiku{In den donker rilt,...}{hij en klappertandt en zijn}{oogen puilen puilen}\\

\haiku{Hij kruipt in bed, trekt,.}{de klamboe dicht trekt de sprei}{tot over zijn ooren}\\

\haiku{Hij heeft het sedert,...}{altijd gezien en hij werd}{een oude man}\\

\haiku{en... dr\`eigt er iets...?}{achter de boomstammen van}{dat eindelooze pad}\\

\haiku{maar het kwam niet van,:}{de Derckszen als tante}{Stefanie meende}\\

\haiku{Hij poogde zich te,,.}{beheerschen mannelijk te}{zijn flink en moedig}\\

\haiku{En hij streelde haar.}{op zijne beurt en gaf haar}{een innigen zoen}\\

\haiku{- Ja, zie je kind, een,:}{parapluie dat vind ik}{een ellendig ding}\\

\haiku{- Neen kind, een rijtuig.}{vind ik nog vreeslijker dan}{een parapluie}\\

\haiku{- Elly had alles,,;}{kunnen krijgen wat ze wo\^u}{zei de oude heer}\\

\haiku{De tandelooze mond,.}{trilt en mummelt de vingers}{sidderen hevig}\\

\haiku{Maar wat is die ziel,,!}{nu verstompt en wat is zij}{oud wat is zij oud}\\

\haiku{Zo\^u het dan toch zoo,:}{zijn als de menschen wel iets}{hadden gemompeld}\\

\haiku{in Indi\"e... vijftig... -,,.}{jaar geleden God wat een}{tijd huiverde Lot}\\

\haiku{- Dat is nu zestig...,.}{zestig jaren geleden}{zei Pauws droomerig}\\

\haiku{Elly wordt er maar -,!}{wit als een doek van kindje}{wat zie je er uit}\\

\haiku{Maar waarom ze niet,.}{trouwen willen dat is en}{blijft me een raadsel}\\

\haiku{Ik denk soms, dat het,...}{voorbij is dat dat alles}{voorbij is gegaan}\\

\haiku{- Niet zoo boos doen... Wat... -...}{nu van tante Ther\`ese We}{zullen maar niet gaan}\\

\haiku{En dat u niet eet... -,,.}{Ik eet ik eet zei tante}{Ther\`ese zacht en traag}\\

\haiku{Het was vaag, maar er,,.}{was eerzucht in en eerzucht}{uit liefde om h\`em}\\

\haiku{Hoe anders dan de,.}{huilwind van ons Noorden die}{zoo luguber giert}\\

\haiku{Onze natuur slaapt.}{den heelen zomer onder}{den brand van de zon}\\

\haiku{Net goed, had tante,.}{gedacht en toch hield ze wel}{van haar vogeltjes}\\

\haiku{En daar achter, wat...?}{verborg hij daar achter die}{Latijnsche boeken}\\

\haiku{We zouden van daag,,?}{samen er heen gaan niet waar}{tante Stefanie}\\

\haiku{Nog een heel korten... -,...}{tijd en d\`an Ja dan is het}{heelemaal voorbij}\\

\haiku{En tante Floor, oud,,...}{boos mandarijnengezicht}{zag naar haar man Ddaan}\\

\haiku{ik geloof dat niet -,.}{maar er is iets waarom oom}{Daan is gekomen}\\

\haiku{gezellig, vroolijk,;}{praatte Pol niet \`al te druk}{toch om grootpapa}\\

\haiku{n\`u wist hij het deel,...}{zijner moeder in de schuld}{de vreeslijke schuld}\\

\haiku{zijn buik wierp zich van,.}{links naar rechts hing nu over het}{gezonde been heen}\\

\haiku{Ik moest Harold zien,,,...}{jou zien ik moest mama zien}{ik moest Takma zien}\\

\haiku{ga dadelijk naar,.}{dokter Thielens en dan naar}{meneer Steyn de Weert}\\

\haiku{De dokter kan niets,...}{voor hem doen maar de dokter}{moet constateeren}\\

\haiku{- op den rechten stoel,...}{in die roode schemering}{van het raamgordijn}\\

\haiku{Hij ging en in het,,.}{sterfhuis de blinden dicht bleef}{tante Ad\`ele alleen}\\

\haiku{De menschen zouden,:}{er wel over praten misschien}{niet eens zoo h\'e\'el veel}\\

\haiku{En zij ging in de,.}{sterfkamer op de punten}{van haar pantoffels}\\

\haiku{Welke stemmen had,?}{hij gehoord welke stem had}{hij hooren roepen}\\

\haiku{Ik wil alleen op,,:}{de hoogte zijn \`of je erft}{en hoeveel je erft}\\

\haiku{Zij waren alleen,,:}{in het compartiment en}{zij zeide liefkoozend}\\

\haiku{Ook den volgenden,,.}{dag dien der begrafenis}{was tante Ad\`ele kalm}\\

\haiku{En dat hij me nu,,.}{en dan zoo aardig een paar}{honderd gulden gaf}\\

\haiku{- Mama zo\^u van avond,,.}{hier komen om jullie te}{zien zei tante Ad\`ele}\\

\haiku{Ik wo\^u... om al het,.}{geld van de wereld dat ik}{het niet gedaan had}\\

\haiku{lees... in Godsnaam... ter,...:}{wille van mij Steyn om het}{met me te deelen}\\

\haiku{Een afleiding was,,:}{ook dat Ina boven kwam met}{Netta op den arm}\\

\haiku{- Neen, zei Ad\`ele Takma,,,.}{en zij huiverde hier in}{huis sedert zij wist}\\

\haiku{Zij sprak niet veel, zat,...}{naast de moeder die hare}{hand had genomen}\\

\haiku{Maar zie je, kind, het...:}{z\'o\'o doen als jij het gaarne}{hebt dat kan ik niet}\\

\haiku{- Dat is lekker, met,...}{het gure we\^er meneer Lot}{en mevrouw Elly}\\

\haiku{integendeel, ze,,}{is kalm en heel blij dat ze}{mevrouw Ther\`ese we\^er}\\

\haiku{Daar stonden, zaten,,...}{de kinderen zij allen}{menschen van leeftijd}\\

\haiku{Hij plukte een paar,.}{van de druiven en zoog ze}{uit en belde toen}\\

\haiku{Zie je, John en.}{ik zien haar nooit en Mary}{komt gauw uit Indi\"e}\\

\haiku{twee zulke dotten,...}{van kinderen zulke mooie}{lieve kinderen}\\

\haiku{scheer me heel netjes,,...}{niet waar want met die baard ben}{ik afschuwelijk}\\

\haiku{Ik voel me herleefd.}{met mijn zachte vel en met}{mijn zijden hemdje}\\

\haiku{Lot nam zijn vaders -,... -!}{hand. Niet zoo spreken vader}{Verd\`omde vrouwen}\\

\haiku{Er waren mooie en,;}{interessante dingen}{vooral in Itali\"e}\\

\haiku{{\textquoteleft}Van den zomer heb,:}{ik niet gewerkt maar nu ga}{ik goed aan den gang}\\

\haiku{iets deed... \ensuremath{<} wat o\`ok...,,}{deed H14,24toch \ensuremath{<} toch toch GN}{H14,31jezelve \ensuremath{<} wij}\\

\haiku{39De correcte.}{lezing is overgenomen}{uit het kladhandschrift}\\

\section{Johanna Desideria Courtmans-Berchmans}

\subsection{Uit: De gemeente-onderwijzer}

\haiku{Op 25jarigen.}{ouderdom trad zij met hem}{in het huwelijk}\\

\haiku{*** In haren rampspoed.}{verloor Vrouwe Courtmans geen}{oogenblik het hoofd}\\

\haiku{Letterkunde als.}{zuivere kunst beschouwen}{daar dacht niemand aan}\\

\haiku{beiden vonden dat.}{het goed was en tevreden}{waren zonder meer}\\

\haiku{Het volk gaat me\^e of,}{gaat niet me\^e de schilders gaan}{hun gang en later}\\

\haiku{Zou men inderdaad?}{niet mogen spreken van een}{verkeerd uitwerksel}\\

\haiku{De meester nam zijn,;}{bolivard af en streek zijn}{dun grijs hair omhoog}\\

\haiku{die schoone kle\^eren.}{kan hij allemaal op den}{plak gekocht hebben}\\

\haiku{iets waarover hij zich.}{natuurlijk zeer blijmoedig}{in de handen wreef}\\

\haiku{Gij zijt immers bij?}{kruideniers en beenhouwer}{ten winkel geweest}\\

\haiku{{\textquoteright} {\textquoteleft}Gij trekt ook altijd,,}{partij voor dat vreemd volkje}{Mietje Raveschoot}\\

\haiku{{\textquoteright} De meester stond op.}{de teenen om de boeren over}{de schouders te zien}\\

\haiku{uit die zoo zoet, zoo,.}{hartroerend waren dat er}{Irma bij weende}\\

\haiku{neem de banknoot,.}{aan en geef de lessen in}{alle vriendschap voort}\\

\haiku{{\textquoteleft}Maar het nemen van.}{dat jongsken was ook al eene}{slechte spekulatie}\\

\haiku{{\textquoteright} vroeg de meester, die.}{belang in de samenspraak}{begon te stellen}\\

\haiku{Kozijn, zou er wel?}{een doove in W. zijn die}{het geval niet kent}\\

\haiku{nu stegen hare.}{kreten in luide klanken}{met het gebed op}\\

\haiku{{\textquoteleft}Maar waarom bleeft gij?}{aan het mastbosch met Mietje}{Raveschoot achter}\\

\haiku{{\textquoteright} {\textquoteleft}Denkt gij dat ik de?}{konkurrencie van zulk eenen}{man  moet vreezen}\\

\haiku{{\textquoteright} {\textquoteleft}Het zij zoo,{\textquoteright} was het,.}{antwoord en de twee vrienden}{gingen aan het spel}\\

\haiku{Ook vielen zij als.}{lavende hemeldauw op}{zijn brandend gemoed}\\

\haiku{zoo beminden zij,.}{wel zonder voldoende hoop}{maar ook zonder vrees}\\

\haiku{{\textquoteleft}Sa, pater kies er,{\textquoteright};}{een nonneken uit koos zij}{Mietje Raveschoot}\\

\haiku{en zij schonk de twee.}{beste vriendinnen nog eenen}{kus en eenen handdruk}\\

\haiku{maar helderer dan;}{de glans harer oogen straalde}{de blik harer ziel}\\

\haiku{Doch alhoewel de;}{inrigting der kostschool hem}{zeer verontrustte}\\

\haiku{ten voordeele van het,.}{volksonderwijs spreekt mogen}{wij niet herkiezen}\\

\haiku{{\textquoteright} {\textquoteleft}Ik geloof, om de,}{waarheid te zeggen dat gij}{beter dan iemand}\\

\haiku{{\textquoteright} {\textquoteleft}De ongesteldheid,,{\textquoteright}.}{is reeds voorbij Mijnheer Van}{Dale ze{\^\i} Irma}\\

\haiku{Viel mij echter dat,,;}{onverhoopt geluk te beurt}{dan gij kent Irma}\\

\haiku{{\textquoteright} {\textquoteleft}Ja, dit zal mijnheer,{\textquoteright}.}{Blommaert ons niet verhuren}{bemerkte Roosje}\\

\haiku{Ga, hij is in de.}{groote kamer bezig met de}{gazet te lezen}\\

\haiku{Anders zou men er,{\textquoteright},}{zoo druk niet in arbeiden}{antwoordde Tonia}\\

\haiku{Edward Van Dale,.}{en Irma Blommaert beiden}{alhier woonachtig}\\

\haiku{{\textquoteright} {\textquoteleft}In den tegenspoed,:}{stond Edward pal gelijk de}{volwassen woudeik}\\

\haiku{Het zwakke Roosje,,.}{was na eene kortstondige}{ziekte overleden}\\

\haiku{Slechts bij u, bij uwen.}{vader en uwe moeder vind}{ik mij hier te huis}\\

\haiku{de mantel die van,.}{haar hoofd tot op den grond daalt}{schittert als diamant}\\

\haiku{Prudens, die jonger,;}{is dan Hector blijft nog een}{jaar voortstuderen}\\

\haiku{een ongeluk komt,:}{nooit alleen en heden zou}{ik kunnen zeggen}\\

\haiku{{\textquoteleft}Vloet ik niet zorgen?}{om mijne kinderen iets}{achter te laten}\\

\haiku{Hoe jammer is het,{\textquoteright}.}{dat meester Save dood is}{zeiden de meisjes}\\

\section{Frits Criens}

\subsection{Uit: Vergaete aorlog}

\haiku{Ich k\'os waal janke.}{wie ze mich vanmiddig nao}{mien kamer brachte}\\

\haiku{Op eine kier waas.}{ich het kantoe\"er van de}{baas aan het poetse}\\

\haiku{Van die angere.}{wis allein de Boer waem en}{waat ich waas gewaest}\\

\haiku{Nemes geluifdje.}{det dees evakuasie weer zoe\"e}{v\"a\"orbie  zooj zeen}\\

\haiku{De Spass waas in\`ens.}{geine Spass mier en doerdje}{dan ouch mer efkes}\\

\haiku{Wie hae gen\'og haaj, sjtook}{hae de fles tr\"ok en sjpiensdje}{\'onr\"ostig de zaal}\\

\haiku{eine sjlaag van,.}{ziene houte poe\"et en}{ich waas d'r gewaest}\\

\haiku{Wie rijk wil worden,,{\textquoteright}.}{moet bij de mond beginnen}{waas h\"a\"ore vaste kal}\\

\haiku{waar, want die sjloge.}{alles in v\"a\"orraod op}{wie de eikuurkes}\\

\haiku{nog mier bewaeging}{in het gammel sjevaak det}{neet mier te haoje}\\

\haiku{Mer ich kwoom tr\"ok in,.}{ein kapot gesjaote}{sjtad moorzeels allein}\\

\haiku{het hoeshaoje waal}{\"orges t\"osse Li\"ewe en}{Frieslandj begrave}\\

\haiku{Allebei de borste,.}{ginge d'raan m\`et de kliere}{\'onger de erm}\\

\haiku{Achteraaf is.}{det mesjien sjt\'om want ich kwoom d'r}{door en m\'os nao hoes}\\

\haiku{De Magere waas.}{hel op waeg het werk aan mien}{lief aaf te make}\\

\haiku{M\`et waat ich in het,.}{laeve geli\"erdj h\"ob sjeet}{ich noe gein poes op}\\

\haiku{Mer ich h\"ob mich in.}{die k\`erk geine kier op}{mien gemaak geveuldj}\\

\haiku{Groe\"etvader en.}{vader wore allebei}{k\`erkemeister}\\

\haiku{{\textquoteright} vroog de geistelik,.}{wie hae de formuliere}{in m\'os v\"olle}\\

\haiku{Mer die vergote.}{det h\"a\"ore God miense ouch in}{de kop kan kieke}\\

\haiku{Vanaovendj moog ich,.}{in ein echt likbad v\"a\"or het}{ierst van mien laeve}\\

\haiku{Miene m\'ondj is de.}{bewaeging van het baeje}{nog neet verli\"erdj}\\

\haiku{as ze naeve de.}{trein op ginge \'om \'os nao}{binne te jage}\\

\haiku{Zoe\"e drejts se de.}{diekste liene dook nog in}{ein houmou kapot}\\

\haiku{Groe\"etvader waas.}{op waeg doe\"ed te waere}{mer ging neet doe\"ed}\\

\haiku{m\`et de ganse buurt,}{en wie hae de negendje}{daag nog laefdje zoog}\\

\haiku{ze v\"a\"or maagd oet Belsj.}{in B\"oggeme gek\'omme en}{dao later getrouwdj}\\

\haiku{Mer vanaaf dae tied dors.}{ich in b\`ed al gaaroets gein}{oug mier toe te doon}\\

\haiku{Daor\'om b\`en ich ouch,.}{det werk gaon doon \'omdet}{ich van dae miens heel}\\

\haiku{Ich goof te v\"a\"ol \'om,.}{hem \'om det ein laeve lang}{van hem te wille}\\

\haiku{Het waas door al die.}{kiere opgeloupe tot}{ein sjoe\"en bedraag}\\

\haiku{Same h\"obbe we,.}{de evakuasie \"a\"overlaefdj}{de w\`ekker en ich}\\

\haiku{Nao det bad h\"ob ich.}{mich \`ens tegooj bekeke}{in de sjpegel}\\

\haiku{In B\"oggeme sjt\'ong.}{in miene tied de waereldj}{zoe\"eget sjtil}\\

\haiku{Ich zooj neet weite.}{waat in miene j\'onge tied}{neet good is gewaest}\\

\haiku{Tien waas mer fien van.}{sjt\"ok en haaj zeker gein posjtuur}{\'om zich te houwe}\\

\haiku{En haod in de.}{duustere good en kwaod}{volk mer \`ens oetein}\\

\haiku{Daor\'om sjaote.}{die Ingelse drek as ze}{emes  trappeerdje}\\

\haiku{As ze eier te,,}{min haaj sjtook ze d'r ein paar}{bie oet ein anger}\\

\haiku{Op het l\`etst van;}{de aorlog t\`eldje allein}{mer dien sjaemel pens}\\

\haiku{Det gemurmel duit.}{mich dinke aan ein besjeting}{door de Ingelse}\\

\haiku{De piep loog m\`et de.}{aope kantj in de lingdje}{van de kogelbaan}\\

\haiku{Het insigste waat.}{nog ein bietje w\`erktj}{zeen mien gedachte}\\

\haiku{Sjt\'om eigelik, die:}{veugel doon niks angers as}{waat miense ouch doon}\\

\haiku{De Boer trouwdje in,}{het begin van de aorlog}{drek naodet zien}\\

\haiku{Duits gehuerdj,{\textquoteright} zag.}{de Boer mich sjtraks en hae}{jankdje opnoeuw}\\

\haiku{Ich wis neet wie het,.}{m\`et h\"a\"or ging allein det ze}{al bevriedj wore}\\

\haiku{Det vader net krank,,.}{waas wis ich neet det ze nao}{Baoksem m\'oste ouch neet}\\

\haiku{Aan zichzelf en h\"a\"or,}{twi\"e klein kranke wichter haaj}{ze de henj al vol.}\\

\haiku{Ich trok zoe\"e wit weg.}{det de kl\"a\"or allein al mich}{zooj verraoje}\\

\haiku{flot achterein m\`et de.}{luip van het gewaer in het}{kruuts en \'ongerlief}\\

\haiku{De naef probeerdje,.}{\'omhoe\"eg te k\'omme mer}{zakdje inein}\\

\haiku{Wie ich weer keek, zoog.}{ich h\"a\"or de ker same de}{br\"ok \"a\"overduje}\\

\haiku{Josef en Maria.}{geveuldj h\"obbe wie ze nao}{Betlehem ginge}\\

\haiku{{\textquoteleft}Nu mijn man toch niet,.}{meer hier is kan ik net zo}{goed naar Montfort gaan}\\

\haiku{zieptj, riets de vrucht die?}{aafkumtj van h\"a\"or weg en}{voors die aan de huunj}\\

\haiku{Ouch het sjlengske in,.}{mien naas m\'ot d'roet het kepke}{geit van miene m\'ondj}\\

\haiku{Achter mien ouge.}{haet emes ein groe\"ete lamp}{aangesjtaoke}\\

\haiku{Woe\"e is de d\"a\"or, woe\"e,,?}{is diene mantjel woe\"e b\`es}{se zelf woe\"e b\`en ich}\\

\section{Edward G. Croffts}

\subsection{Uit: Wie is S.S. 777?}

\haiku{{\textquoteleft}Aangenaam met u,.}{kennis gemaakt te hebben}{meneer Lewison}\\

\haiku{Degene, die hem,!}{dat koopje geleverd had}{zou ervoor boeten}\\

\haiku{- Er moet dus iemand,.}{langs gegaan zijn zonder dat}{u het gemerkt hebt}\\

\haiku{In dat geval kom;}{ik het bedrag zelf wel even}{uit uw kluis halen}\\

\haiku{{\textquoteleft}Ik ben Inspecteur,{\textquoteright}.}{MacNewvish antwoordde}{de Inspecteur droog}\\

\haiku{ik ben een beetje...,......{\textquoteright} {\textquoteleft},,}{een beetje nerveus ziet u}{enJa dat zie ik}\\

\haiku{We zullen dan eens,.}{zien of we daar geen stokje}{voor kunnen steken}\\

\haiku{Gibbs keek haar strak aan,:}{en het scheen dat zij zijn blik}{niet verdragen kon}\\

\haiku{Zij was de eenige,.........}{die hem er neergelegd kon}{hebben en dan dan}\\

\haiku{{\textquoteright} {\textquoteleft}Om u de waarheid,.}{te zeggen begrijp ik er}{minder van dan ooit}\\

\haiku{Maar als ik u een,:}{wijzen raad mag geven zou}{ik u aanraden}\\

\haiku{{\textquoteleft}Inderdaad,{\textquoteright} lachte,.}{Gibbs die aanstalten maakte}{om te vertrekken}\\

\haiku{Zijn lichaam raakte;}{echter verslapt door overwerk}{en ondervoeding}\\

\haiku{En natuurlijk wist.}{hij ook van de uitvinding}{van zijn employ\'e af}\\

\haiku{De employ\'e zag zijn;}{schoonste levensillusie}{in rook vervliegen}\\

\haiku{Lewison kende,.}{geen tegenwerking kende}{geen vernedering}\\

\haiku{nu hij oud was kwam -.}{hij tot dit bewustzijn ten}{koste van zichzelf}\\

\haiku{{\textquoteright} {\textquoteleft}Laten we er geen,{\textquoteright}.}{woorden meer over vuil maken}{antwoordde de reus}\\

\haiku{Tegelijkertijd.}{zocht hij naar zijn revolver}{in zijn achterzak}\\

\haiku{- Maar hij wist me op.}{het laatste oogenblik toch}{nog te ontsnappen}\\

\haiku{{\textquoteleft}Laat es zien,{\textquoteright} peinsde.}{hij en zware denkrimpels}{fronsten zijn voorhoofd}\\

\haiku{Van het onderzoek.}{hoorde hij gedurende}{deze dagen niets}\\

\haiku{{\textquoteright} {\textquoteleft}Natuurlijk, mijnheer,{\textquoteright}.}{Lewison antwoordde het}{meisje eenvoudig}\\

\haiku{In een ervan vond.}{hij de portefeuille van}{den automagnaat}\\

\haiku{{\textquoteleft}En lijkt het jou ook,?}{niet dat de stijl van dien brief}{erg mannelijk is}\\

\haiku{Ik geloof niet, dat.}{een vrouw op dergelijke}{wijze schrijven zou}\\

\haiku{Jim Gibbs bleef in het.}{priv\'e-kantoor achter}{en keek nog wat rond}\\

\haiku{{\textquoteright} {\textquoteleft}En zat miss Edwards,?}{op dien stoel daar naast dien van}{mijnheer Lewison}\\

\haiku{{\textquoteright} {\textquoteleft}Waar stond gewoonlijk?}{de presse-papier van}{mijnheer Lewison}\\

\haiku{Bovendien blijkt zij.}{plotseling van het kantoor}{verdwenen te zijn}\\

\haiku{{\textquoteleft}Kwaad en kwaad is twee,{\textquoteright}.}{antwoordde de Inspecteur}{op knorrigen toon}\\

\haiku{{\textquoteleft}Liefste Ned, Nog voor,.}{de nacht valt wil ik je een}{paar regels schrijven}\\

\haiku{Ik ben zoo uiterst,.}{beangst dat dit gevaarlijk}{voor je kan worden}\\

\haiku{Verder vond hij op.}{een enveloppe het adres}{van Ned Pambroke}\\

\haiku{{\textquoteleft}O, nee meneer, laat,,.}{de menschen die hier wonen}{er niets van merken}\\

\haiku{{\textquoteright} En de brave vrouw,.}{knipte de lamp aan die de}{trap verlichten moest}\\

\haiku{Dus miss Edwards heeft?}{u geholpen het gebouw}{binnen te dringen}\\

\haiku{{\textquoteleft}Gaat uw gang dan maar,, '.}{mijnheer Gibbs maar maakt u het}{kort alst u blieft}\\

\haiku{Dit hoorde ik van,.}{het personeel dat vlak bij}{hem in de buurt zat}\\

\chapter[15 auteurs, 1144 haiku's]{vijftien auteurs, elfhonderdvierenveertig haiku's}

\section{D.L. Daalder}

\subsection{Uit: Schimmenspel}

\haiku{over het gladde haar....}{sluit een wit plooikapje met}{heel fijne kantjes}\\

\haiku{Geen wonder, dat ze....}{een stoof nodig heeft om in}{evenwicht te blijven}\\

\haiku{het glas met water....}{vullen voor den dominee}{op Zondagmorgen}\\

\haiku{{\textquoteleft}Dominee is een -.}{grote man die moet goed eten}{om sterk te blijven}\\

\haiku{Ze kibbelen wel.}{graag samen en liefst over hun}{verschil in geloof}\\

\haiku{Drift flitst in mij op,,.}{maar ik heb niet de moed iets}{terug te zeggen}\\

\haiku{Bij Meester Douma;}{dringen de kinderen naar}{binnen met geweld}\\

\haiku{{\textquoteleft}Maar ik loop mank en.}{t\`och speel ik de baas over al}{die kwajongens hier}\\

\haiku{Ze kijkt met strenge.}{ogen over de klas en ze zegt}{heel scherpe dingen}\\

\haiku{Ik zie duidelijk,.}{dat Kees Keizer knikkers ruilt}{met Sime Vlaming}\\

\haiku{Dan valt m'n oog op.}{Rika Burger en het wordt}{me warm om het hart}\\

\haiku{En als ik op een,.}{van de knoesten klim kan ik}{net door het raam zien}\\

\haiku{Maar ik weet wel een.}{middel om hem aan zijn plicht}{te herinneren}\\

\haiku{dan krijgt ieder kind ' '.}{een cent voors morgens \`en}{\'e\'en voors middags}\\

\haiku{gekke Wullem van;}{Reyer Verdompeltje brengt}{ze mee uit Spiekdorp}\\

\haiku{{\textquoteright} En piekerend over.}{deze zonderlinge zaak}{vervolg ik mijn weg}\\

\haiku{ze zien duidelijk,.}{dat de blinkende koppen}{naderbij komen}\\

\haiku{Aai Burger zegt, dat:}{je helemaal onder de}{dekens moet kruipen}\\

\haiku{Ook mag ik hem niet,....}{omdat hij altijd zo zuur}{en benepen kijkt}\\

\haiku{Natuurlijk omdat.}{ik een jongen ben van den}{groven schoolmeester}\\

\haiku{je moet 't dak maar,....}{es op want grootmoeder klaagt}{over de  spreeuwen}\\

\haiku{je kunt op het dak.}{de schuiten bij Nieuweschild}{in zee zien liggen}\\

\haiku{En daarop schrijven,:}{ze een gedicht dat begint}{met de woorden}\\

\haiku{Schaduw zal de nacht. ', ';}{hun bi\^ent Grauwt almeet}{wordt stil en duister}\\

\haiku{Tot ik plotseling:}{de spottende blik ontdek}{van meester Douma}\\

\haiku{Ik val grootje om -:}{de hals ze weert me lachend}{af met de woorden}\\

\haiku{Waarom weet ik niet,.}{precies want niet alles is}{mij even duidelijk}\\

\haiku{Over den duivel, die,}{rondgaat als een briesende}{leeuw zoekende wien}\\

\haiku{Ik rilde van een.}{vreemde zaligheid bij dit}{schrikwekkend verhaal}\\

\haiku{Alle melk was op.}{en daarom moest Willem die}{avond cognac schenken}\\

\haiku{{\textquoteright} Maar Moeder offreert.}{een oud vloermatje en een}{bos ouwe kranten}\\

\haiku{Er gaat een gejuich,.}{op onder de jongens als}{de buit binnen is}\\

\haiku{Moeder zal wel kwaad,.}{wezen vanavond maar ook d\`a\`ar}{is niets aan te doen}\\

\haiku{De vete tegen {\textquoteleft}{\textquoteright} {\textquoteleft}{\textquoteright} '.}{de fijnen ende Roemsen}{zit haar int bloed}\\

\haiku{{\textquoteleft}As jullie 't weer,,....}{waagt hier te komme stuur ik}{de hond op je of}\\

\haiku{De fijnen moesten maar,....}{proberen nieuwe voorraad}{te verzamelen}\\

\haiku{en t\`och is er iets,}{geschonden in het  beeld}{van mijn Vader dien}\\

\haiku{Duizenden zijn er -.}{op het eiland je hoort ze}{altijd en overal}\\

\haiku{Alleen Kwantes, de,.}{rijksveldwachter is tegen}{hem opgewassen}\\

\haiku{Ik hol de dijk af,.}{de brug over en wip over het}{hek van het boestuk}\\

\haiku{Maar ik ervaar gauw,}{genoeg dat ze nooit \`echt raak}{schieten en voel me}\\

\haiku{er veiliger dan.}{in de straten van het vaak}{vijandige dorp}\\

\haiku{je moet een d\`ame....}{anders behandelen dan}{een boerendochter}\\

\haiku{Aan Oosterend is,,.}{dat niet veel bijzonders maar}{w\`at er is is goed}\\

\haiku{Als er alleen maar.}{boeren zijn krijg je hoogstens}{een tik met de zweep}\\

\haiku{er is iets vernield,....}{dat nooit meer volkomen is}{te herstellen}\\

\haiku{Tot plotseling een....}{hevige beweging golft}{door de mensenhoop}\\

\haiku{Trijn loopt gearmd met,,.}{haar moeder Nantje die}{hartstochtelijk huilt}\\

\haiku{Hij is heel bleek en....}{ik zie dat de hoge hoed}{in z'n handen beeft}\\

\haiku{voor den zoon van den....}{groven schoolmeester zal er}{geen genade zijn}\\

\haiku{bij Kinnebakkie,....}{ligt stro op de straat omdat}{er een zieke is}\\

\haiku{Ik ga naar grootje,.... '}{waar vader en moeder op}{de verjaring zijn}\\

\haiku{Na 11 November.}{volgen de feesten elkaar}{in snel tempo op}\\

\haiku{{\textquoteleft}Je moet maar an de, '}{Sunterklaas frage of die}{t een beetje goed}\\

\haiku{D\`an is er altijd.}{geld om mooie en lekkere}{dingen te kopen}\\

\haiku{Tegen vijf uur, als ',.}{t flink donker wordt ben ik}{niet meer te houden}\\

\haiku{de {\textquoteleft}streetfeger{\textquoteright} laat....}{de sarrende uitdaging}{niet over z'n kant gaan}\\

\haiku{Kinderen zijn er....}{niet meer geboren in dit}{late huwelijk}\\

\haiku{hij draagt een sikje,.}{dat hevig op en neer wipt}{als hij praat of eet}\\

\haiku{Verwoed vecht ik 's:}{middags met aardrijkskunde}{en geschiedenis}\\

\haiku{Later hoor ik, dat {\textquoteleft}{\textquoteright}.}{dit kosthuis bekend staat als}{de blikken emmer}\\

\haiku{De slaapkamers zijn,....}{klein en ongezellig in}{de nok van het huis}\\

\haiku{Schwantje houdt zich goed,:}{als zijn vader hem stevig}{op de schouder slaat}\\

\haiku{De voldoening van.}{gedaan werk begeleidt mij}{altijd en overal}\\

\haiku{verstolen kijk ik,....}{naar meisjes die lachend en}{pratend voorbijgaan}\\

\haiku{Hij staat v\`o\`or de klas,.}{geleund tegen een bordstijl}{en spreekt als een boek}\\

\haiku{Wij zien uit welke.}{delen die bestaat en hoe}{die zijn verbonden}\\

\haiku{Duidelijk zie ik,:}{dat Jan Lenstra piekert over een}{wiskundevraagstuk}\\

\haiku{z'n nagel krast nu.}{en dan een lijntje in het}{vernis van de bank}\\

\haiku{opstellen maken.}{is van ouds een geliefde}{sport in mijn leven}\\

\haiku{Het geritsel van.}{zijn papier accentueert}{de doodse stilte}\\

\haiku{{\textquoteleft}En bedenk wel, dat!}{het gewone werk daar niet}{onder lijden mag}\\

\haiku{En plotseling wordt,.}{de baas zich bewust welke}{fout hij heeft gemaakt}\\

\haiku{Hij gaat werken als,,.}{wij haastig en met grote}{grove contouren}\\

\haiku{ik verander van,.}{kosthuis ik word gedoopt en}{ik krijg een meisje}\\

\haiku{Op een meeting, vlak,,.}{bij ons huis op het land van}{Piet Dros spreekt Staalman}\\

\section{P.A. Daum}

\subsection{Uit: Nummer elf (onder ps. Maurits)}

\haiku{Na het eten ging hij.}{zijn partijtje maken in}{de soci\"eteit}\\

\haiku{{\textquoteright} {\textquoteleft}Laat me met rust{\textquoteright} was,.}{het onvriendelijk antwoord}{met een knoop er op}\\

\haiku{{\textquoteleft}Het is alles goed,.}{en wel maar met mij is het}{een ander geval}\\

\haiku{Wat schulden maken,.}{was en ze niet betalen}{dat wist ze precies}\\

\haiku{Men zegt dat hij knap.}{is voor zijn zaken en dat}{geloof ik ook wel}\\

\haiku{De heren rookten,.}{een havanna genietend}{als goede rokers}\\

\haiku{dan eens wat beter,.}{dan weer wat slechter en op}{den duur achteruit}\\

\haiku{Ik heb het gehoord, ....}{al haast een week geleden}{van onze baboe}\\

\haiku{Zij draaide de lamp,,.}{op keek in een handspiegel}{en zag hem die trek}\\

\haiku{{\textquoteright} Lena sliep die nacht,.}{zoals ze het in lange}{tijd niet had gedaan}\\

\haiku{Wie weet of het niet.}{een kleinigheid was over de}{medicijn of zo}\\

\haiku{De dokter kon er.}{niet veel meer aan doen dan de}{ziekte waarnemen}\\

\haiku{Zo'n kerel was nu.}{maar niet tot het geringste}{besef te krijgen}\\

\haiku{{\textquoteleft}Als ik er langs die,.}{weg zou moeten komen dan}{bedank ik er voor}\\

\haiku{Want hijzelf was weer.}{gewoon teruggebracht tot}{zijn pensioentje}\\

\haiku{Het scheen dat deze.}{neef een onbe- schaamde}{berenmaker was}\\

\haiku{enfin, dat het niet.}{rooskleurig gesteld is met}{de financi\"en}\\

\haiku{Hij zou er graag het,.}{zijne van gehad hebben}{maar dat was moeilijk}\\

\haiku{Hijzelf was ook niet ....}{eenkennig en als het maar}{buiten hem omging}\\

\haiku{Ik hoop niet dat je...{\textquoteright} {\textquoteleft},.}{me voor zo aartsdom aanziet}{Welnee zeker niet}\\

\haiku{Nu, als het u te,.}{Batavia niet mocht lukken}{schrijf me dan maar eens}\\

\haiku{Zij haalde het geld.}{uit haar eigen trommel en}{gaf het haar vader}\\

\haiku{{\textquoteleft}Breng het hem morgen,.}{zelf pa maar asjeblieft een}{bewijs op zegel}\\

\haiku{Het was een woord dat.}{de oude Bruce razend}{maakte van woede}\\

\haiku{Men begreep niet waar,. '}{de mensen vandaan kwamen}{maar ze waren er}\\

\haiku{Een appel en een,!}{ei dat zou ook hier wel het}{resultaat wezen}\\

\haiku{{\textquoteleft}Ik hoop,{\textquoteright} zei Vermey, {\textquoteleft}.}{dat je een beetje beter}{over me denken zult}\\

\haiku{{\textquoteright} Te Batavia moest.}{de oude Bruce in een}{draagstoel naar de wal}\\

\haiku{Ik zou wel een klein.}{souvenir van de oude}{heer willen hebben}\\

\haiku{Hij had meer dan ooit.}{vues op Lena of liever}{op haar vermogen}\\

\haiku{{\textquoteright} {\textquoteleft}Welnu, de waarheid.}{is dat hij me met Vermey}{getrouwd wilde zien}\\

\haiku{Wij hebben immers.}{zonder dat wel meer verschil}{van mening gehad}\\

\haiku{je herinnert je?}{nog wel de manoeuvre van}{de oude Bruce}\\

\haiku{Dat is te zeggen...,.}{ziet u ik ben geen jongen}{van achttien jaar meer}\\

\haiku{Men weet nooit of men.}{iemand wel ergens mee kan}{feliciteren}\\

\haiku{Ga jij nu naar huis,.}{dan schrijf ik nog dadelijk}{naar neef Voirey}\\

\haiku{{\textquoteright} Hij had een pret van.}{belang en lachte zoals}{hij maar zelden deed}\\

\haiku{Voirey was er ook.}{net een die dacht dat alles}{voor geld te koop was}\\

\haiku{{\textquoteright} {\textquoteleft}Maar er kon wel eens,{\textquoteright}.}{geen volgend maal komen zei}{ze sentimenteel}\\

\haiku{Zelfs in gedachten.}{wilde hij niet terug naar}{die Boheemse tijd}\\

\haiku{{\textquoteleft}Zo'n slet,{\textquoteright} dacht hij, stond,.}{op draaide het licht uit en}{ging naar zijn kamer}\\

\haiku{{\textquoteright} Tot een besluit was.}{Vermey ook de dag daarna}{nog niet gekomen}\\

\haiku{Ik heb geen plezier.}{zo alleen naar de muziek}{te gaan luisteren}\\

\haiku{Er waren geen lui,!}{op de weg tenminste haast}{niet dan inlanders}\\

\haiku{De postbode had.}{intussen zijn brieven en}{couranten gebracht}\\

\haiku{{\textquoteleft}Intussen,{\textquoteright} voegde, {\textquoteleft}}{Vermey er heel bedaard bij}{hoor ik wel van je}\\

\haiku{{\textquoteright} vroeg mevrouw Vermey,.}{opkijkend in de richting}{der bijgebouwen}\\

\haiku{Hij baadde, kleedde.}{zich en liet zijn bediende}{de koffers pakken}\\

\haiku{Zachtjes beurde hij.}{het hoofd op en schoof er zijn}{brede arm onder}\\

\haiku{Als men een slang vindt,.}{op zijn erf dan pakt men die}{niet met de handen}\\

\haiku{{\textquoteright} Haar moeder haalde:}{de schouders op en zei op}{minachtende toon}\\

\subsection{Uit: Verzamelde romans. Deel 3 [alleen Goena-goena]}

\haiku{Nijgh \& Van                     Ditmar,,-,-,-.}{Amsterdam 1998 p. 7208}{979985 10171027}\\

\haiku{Naast hem stonden op;}{een marmeren knaap{\textdegree} twee}{kopjes koffie}\\

\haiku{alleen miste hij.}{dat van zichzelve voor zijn}{eigen zaken}\\

\haiku{Als men 'n man heeft,,...{\textquoteright} {\textquoteleft}?}{die notaris is en men}{vertrouwt hem nietNu}\\

\haiku{{\textquoteright} {\textquoteleft}Ja,{\textquoteright} was 't antwoord, {\textquoteleft} '.}{met een zucht vol staatszorgdat}{isn lelijk ding}\\

\haiku{hoe haar zuster van;}{die ellendige kerel}{zou verlost raken}\\

\haiku{Reeds van de eerste;}{dag waren ze elkaar}{tegengevallen}\\

\haiku{{\textquoteright} De oude gaf door,.}{niets te kennen dat ze dit}{verstond of begreep}\\

\haiku{Ze durfde 'n                     .}{ogenblik haar gedachten niet}{te laten voortgaan}\\

\haiku{{\textquoteleft}Je bent erg pinter,,,.}{n\`eh dat je raden kunt wat}{anderen schrijven}\\

\haiku{De oude zal hem,.}{wel goed verzorgen en je}{kunt hier toch niets doen}\\

\haiku{{\textquoteright} {\textquoteleft}Mijn hemel, Borne,?}{je hebt toch op het kerkhof}{geen standjes gemaakt}\\

\haiku{We gaan anders niet,.}{om te spelen maar enkel}{en alleen voor Bets}\\

\haiku{{\textquoteleft}Ajakkes,{\textquoteright} zei Betsy,;}{zich woedend van dit Ezau{\"\i}sch}{schouwspel afwendend}\\

\haiku{Hij was bekend als {\textquoteleft}{\textquoteright},.}{eenhaai die altijd met}{de winst ging strijken}\\

\haiku{{\textquoteright} {\textquoteleft}Nu, 't was zo erg,...}{niet en als Pr\'edier me}{niet zo had verveeld}\\

\haiku{Ik zie niet, dat er.}{enige reden bestaat om}{hem te beklagen}\\

\haiku{Zij reikte hem over,.}{de tafel haar hand die hij}{maar flauwtjes drukte}\\

\haiku{Zij stond op, kwam naar;}{hem toe en stak haar gezicht}{vooruit om een kus}\\

\haiku{{\textquoteleft}Als hij rijk is en,?}{hij wil waarom zou hij dan}{niet als hij kon}\\

\haiku{en wat erop groeit,.}{beter dan de blanda's{\textdegree} die}{er overheen rijden}\\

\haiku{maar ze had                     er;}{nooit enig gevolg van gezien}{of ondervonden}\\

\haiku{Daarom achtte                     .}{Pr\'edier zich genoopt een}{besluit te nemen}\\

\haiku{{\textquoteleft}Waarom zeg je dat '?}{opn manier alsof het}{iets bijzonders was}\\

\haiku{Misschien is er iets.}{waars in en                     zou men nog}{verder kunnen gaan}\\

\haiku{{\textquoteleft}Ik vind wel, dat je.}{vanochtend gruwelijk zwaar}{op                     de hand bent}\\

\haiku{Als zij zich in het,.}{ongeluk willen storten}{is het hun zaak}\\

\haiku{{\textquoteright} Met saamgeknepen;}{lippen hief Bronkhorst het hoofd}{op en zag haar aan}\\

\haiku{{\textquoteright} Bronkhorst stond op en {\textquoteleft}....}{reikte haar de hand.Dus tot}{over een paar dagen}\\

\haiku{Je begrijpt toch wel,,.}{oudje dat ik mijn geld niet}{kan weggooien}\\

\haiku{Hoe lief zat ze daar,!}{met zijn ziek                     kind en hoe}{rustig sliepen ze}\\

\haiku{{\textdegree} Zacht en stil schoof;}{ze kleine Jean van haar}{arm in zijn bedje}\\

\haiku{Toen ze weg waren,.}{keek kapitein Borne hen}{na met aandoening}\\

\haiku{Betsy nam met haar.}{oude meid haar intrek}{bij de Bronkhorsten}\\

\haiku{Zij had met Sidin,.}{de conferentie gehad}{bij haar zoon Ketjil}\\

\haiku{De notaris was.}{kwistig                     geweest met zijn}{uitnodigingen}\\

\haiku{{\textquoteleft}Het is niet om de,.}{soesah maar omdat}{je niet brani bent}\\

\haiku{Alles moest zijn                     ,.}{tijd hebben en het zou nu}{niet lang meer duren}\\

\haiku{U kunt zo lekker,.}{kwee-kwee                     maken en daar}{houdt meneer zo van}\\

\haiku{het is zijn geheim,,.}{dat niemand aangaat omdat}{het van hem                     is}\\

\haiku{Ze hadden het 's;}{avonds daar altijd over in de}{bediendenkamers}\\

\haiku{Maar op haar schijnen.}{jouw duivelskunsten                     geen}{invloed te hebben}\\

\haiku{Maar Betsy trok de.}{wenkbrauwen hoog op en stak}{de lippen vooruit}\\

\haiku{{\textquoteleft}Ik had daar haast 'n,{\textquoteright}.}{brief opengemaakt aan jouw adres}{zei hij tot Marie}\\

\haiku{als vrouw stelde zij.}{temperament meer op}{prijs dan logica}\\

\haiku{Zoals                     ze haar,.}{plichten had vervuld zou ze}{staan op haar rechten}\\

\haiku{{\textquoteleft}Het scheelt, hoe men ook,.}{doet de ene dag toch altijd}{bij de andere}\\

\haiku{Dit was                     van een.}{andere hand dan dat van}{de vorige avond}\\

\haiku{{\textquoteright} {\textquoteleft}Komaan,{\textquoteright} zei ze met, {\textquoteleft}!}{bleke lippenen dat wordt}{mij gevraagd door jou}\\

\haiku{maak nu asjeblieft,,...{\textquoteright} {\textquoteleft}?}{geen bezwaren want als het}{nodig was danDan}\\

\haiku{d\'at nam zij                     zich,;}{ernstig voor om zijn goede}{naam te sauveren}\\

\haiku{Er werd niet verder,;}{over gesproken maar toch vond}{Marie het                     vreemd}\\

\haiku{Misschien wat vermoeid.}{van dat langdurig hossen}{tussen de wielen}\\

\haiku{zij wist dat hij                     .}{daarvan hield als hij uit was}{geweest en dorst had}\\

\haiku{De mensen mogen.}{voor mijn part precies zeggen}{wat zij willen}\\

\haiku{Het is niet                     waar,.}{en het zal op zo'n manier}{ook niet waar worden}\\

\haiku{en naarmate de,.}{klank                     verzachtte werden}{de woorden weker}\\

\haiku{van indiscretie,.}{geen spoor maar iedereen wist}{het niettemin}\\

\haiku{Hij had 't niet                     .}{louter gedaan om haar te}{contrari\"eren}\\

\haiku{'t Was er ditmaal.}{juist een die ze vertrouwde}{en gaarne                     mocht}\\

\haiku{integendeel, zij,.}{lachte allerliefst en}{stond dadelijk op}\\

\haiku{{\textquoteright} vroeg de oude, die.}{op een bal\'e-bal\'e}{haar hazenslaap sliep}\\

\haiku{En nu                     ving hij;}{aan met gemaakte kalmte}{te redeneren}\\

\haiku{Hij had dan toch ook '.}{wel eens                     meern sabel}{in de hand gehad}\\

\haiku{{\textquoteright} {\textquoteleft}Dus denkt u, dat het...{\textquoteright} {\textquoteleft}.}{van mijn kant niet nodig is}{Ik vermeen van neen}\\

\haiku{{\textquoteright} Daar de kapitein,.}{er momenteel althans geen}{raad op wist zweeg hij}\\

\haiku{{\textquoteright} {\textquoteleft}'t Is jammer, dat.}{ik me er persoonlijk niet}{mee kan bemoeien}\\

\haiku{En toen hij zich tot,.}{haar wendde ontroerde zij}{van zijn ontroering}\\

\haiku{Dit teken verwijst}{naar de woordenlijsten op}{blz. 1017                     e.v.}\\

\section{August Defresne}

\subsection{Uit: Het gehucht}

\haiku{In de schommeltent.}{ontstond enige tijd later}{een soortgelijk feit}\\

\haiku{Zij rukte zich met.}{een gil los en snelde om}{hulp naar haar vader}\\

\haiku{Het lach-kabaal,,.}{dat losbrak overstemde de}{kermisgeluiden}\\

\haiku{Zij holden door de,,.}{jankende jengelende}{schreeuwende kermis}\\

\haiku{Die oom was goed voor,.}{me maar na twee jaar verdronk}{hij op een zeiltocht}\\

\haiku{Door dat trekken ging.}{de koperen stang in haar}{steunpunten draaien}\\

\haiku{Maar naarmate hij,.}{onrustiger werd werd zij}{kalmer en kalmer}\\

\haiku{Zij waren een flits.}{van de oneindige tijd}{\'e\'en mens geweest}\\

\haiku{Vanuit zijn kamer.}{kon hij door een kleine gang}{in de kerk komen}\\

\haiku{Enige kinderen.}{hingen uit de donkere}{ramen te kijken}\\

\haiku{{\textquoteleft}Wat is het hier mooi{\textquoteright},.}{en stil zuchtte verlicht de}{opgejaagde man}\\

\haiku{{\textquoteright} De boeienkoning.}{keek met angstig ontzag naar}{het tabernakel}\\

\haiku{En voor mijn dochter.}{ben ik ook altijd zo goed}{mogelijk geweest}\\

\haiku{{\textquoteright} Tussen twee kramen.}{voor het logement stond een}{handkar schuin omhoog}\\

\haiku{Soms liep er een een.}{paar passen en ging dan weer}{naar zijn plaats terug}\\

\haiku{{\textquoteleft}Weet je{\textquoteright}, fluisterde, {\textquoteleft}.}{ze met gebogen hoofdik}{ben nog een meisje}\\

\section{Wies Defresne}

\subsection{Uit: Klanten}

\haiku{- Wie praat hier van door,!}{den neus boren dat gebeurt}{jou toch alleen maar}\\

\haiku{- Hij gapt ze op De,,?}{Porceleina de stinkerd}{en hoe doet hij dat}\\

\haiku{De capsules kan;}{ik aan de nonnen geven}{voor de nikkertjes}\\

\haiku{dan is je broertje, '.}{anders die brengts Zondags}{taartjes voor me mee}\\

\haiku{Mijn broer keek af en.}{toe door de spleten van de}{\'etalage-kast}\\

\haiku{- En jij komt voor de,,.}{honden ik zal ze halen}{zei kapelaan Roets}\\

\haiku{- Ik had eigenlijk,.}{in de kunst moeten  gaan}{vertelt hij Bertus}\\

\haiku{Ach arme, wat is!}{hij met zijn zwaan tegen de}{lamp geloopen}\\

\haiku{in 't museum.}{hangen de schilderijen}{ook op oogshoogte}\\

\haiku{Hij staat nu op en.}{gaat in den winkel tegen}{de vaten kloppen}\\

\haiku{Ik heb dikwijls zin,:}{om te doen wat Lamberts de}{vorige week zei}\\

\haiku{- Je zult er altijd,,.}{spijt van hebben zei ze dat}{is geen vrouw voor jou}\\

\haiku{- Och foei neen, Marie,,.}{dat moet je zoo niet zeggen}{dat arme Toontje}\\

\haiku{De vroolijke Sjef,.}{maakt zich daarover geen zorgen}{al is hij spion}\\

\haiku{En dat is zoo, want,.}{die krijgt vlagen van waanzin}{als ze op reis is}\\

\haiku{Je moet niet veel meer,,.}{hebben  ouwe je bent}{nou al lazerus}\\

\haiku{Toen de kwaliteit {\textquoteleft}{\textquoteright},!}{drank slechter werd voelden we}{wel dat het mis ging}\\

\haiku{Moeder griste hem:}{den brief uit de vingers en}{zei tegen Claesen}\\

\haiku{En ze bewoog haar '.}{handen rond haar hoofd alsof}{zem al op had}\\

\haiku{- Geef mijn peignoir en.}{pantoffels eens aan en laat}{me maar even met rust}\\

\haiku{wij vinden het fijn, '.}{wij vindent weer eens wat}{anders dan anders}\\

\haiku{Van onze klanten.}{ging de Engelsche spion}{wel eens mee fuiven}\\

\haiku{De fleschjes Stout,;}{werden achter onder langs}{de toonbank gezet}\\

\haiku{De dirigent loopt.}{langs de eerste rij met een}{fluitje in den mond}\\

\haiku{Zij is in het wit.}{gekleed en heeft een boquet}{rozen in den arm}\\

\haiku{- Bis, bis, en na nog.}{een klein toegiftje is dat}{moois ook weer voorbij}\\

\haiku{Jouez avec nous une,.}{partie de Whiste nous vous}{serons tr\`es oblig\'es}\\

\haiku{Maar ik bevries toch.}{nog  liever dan dat ik}{de klanten bedien}\\

\section{Maurits Dekker}

\subsection{Uit: Waarom ik niet krankzinnig ben (onder ps. Boris Robazki)}

\haiku{Immers is de mensch,,.}{in volstrekte eenzaamheid}{de juiste mensch niet}\\

\haiku{Uw argeloosheid.}{en zelfbewustheid zullen}{U parten spelen}\\

\haiku{Tijdens mijn ziekte,.}{had Dmitri Tomitch zich van}{het leven beroofd}\\

\haiku{Menigmaal heb ik.}{later alleen met mijn oogen}{een gesprek gevoerd}\\

\haiku{Dit is de laatste.}{herinnering die ik van}{mijn dorp bewaard heb}\\

\haiku{Ik kon nog geen werst,.}{hebben afgelegd toen de}{storm opnieuw opstak}\\

\haiku{Ik zou doen wat mijn.}{vader gezegd had en mij}{naar M. begeven}\\

\haiku{- Je bent moe, hernam,.}{hij de pit van zijn lamp nog}{wat hooger draaiend}\\

\haiku{Ik viel in slaap op,.}{den top van de draaiende}{golvende wereld}\\

\haiku{Het wezen van het (:}{meslet vooral op het woord}{dat ik hier gebruik}\\

\haiku{Ik ging in mijn bed.}{overeind zitten en begon}{hevig te huilen}\\

\haiku{Naar buiten riep ik,:}{neen maar van binnen uit kwam}{steeds weer het antwoord}\\

\haiku{Op dat oogenblik.}{betraden Marja en de}{vreemde haar kamer}\\

\haiku{Eens, toen ik 's avonds,.}{weer de wacht hield voor Marja's}{deur viel ik in slaap}\\

\haiku{Ik kon echter niet,.}{verhinderen dat mijn vrees}{hand over hand toenam}\\

\haiku{Dien avond wachtte ik.}{zijn komst met de revolver}{in mijn  hand af}\\

\haiku{Ja vriend, als je er,}{later over nadenkt begrijp}{je niet dat je zoo}\\

\haiku{dat Gij menschen zijt,.}{van vleesch en beenderen}{zooals ieder ander}\\

\haiku{Ik verstijfde, mijn.}{voeten werden zwaar en ik}{kon geen stap meer doen}\\

\haiku{Van hier uit was het.}{niet moeilijk de bakkerij}{terug te vinden}\\

\haiku{Wel herkende ik:}{de plaats waar ik mij op dat}{oogenblik bevond}\\

\haiku{Mijn ledikant stond.}{in een hoek en daarom had}{ik maar \'e\'en buurman}\\

\haiku{de mijne liggen.}{en ik begrijp waarom je}{dikwijls verdriet hebt}\\

\haiku{Ik hou van je, zoo.}{groot als de wereld en zoo}{diep als je oogen zijn}\\

\haiku{Na vijf dagen werd;}{ik genoodzaakt afscheid van}{mijn vriend te nemen}\\

\haiku{mijn medewerking.}{bij het ledigen zijner}{flesschen te verleenen}\\

\haiku{er omheen, welke.}{teekening er maanden lang}{op is blijven staan}\\

\haiku{Uw arm beweegt zich.}{onverbiddelijk in de}{richting van het wiel}\\

\haiku{Alles, van groot tot;}{klein en van hoog tot laag is}{daar middelmatig}\\

\haiku{Eens per maand, als de, '}{salarissen uitbetaald}{moesten worden kwam hij}\\

\haiku{s morgens reeds, maar.}{op die dagen bleef hij bij}{het theedrinken weg}\\

\haiku{- Een hulp-schrijver,.}{k\`an geen partij schaak van een}{klerk winnen zei hij}\\

\haiku{- Neem toch, vervolgde,.}{zij mij een schoteltje en}{een mes toeschuivend}\\

\haiku{Mijn oudste dochter.}{maakt shawls en dassen voor een}{modemagazijn}\\

\haiku{Jammer dat zij niet,.}{thuis is anders zou u haar}{werk eens kunnen zien}\\

\haiku{Het lawaai was nu.}{zelfs achter het gesloten}{venster onhoudbaar}\\

\haiku{- Ik ben heelemaal,,}{niet kwaad antwoordde ik wat}{je gezegd hebt is}\\

\haiku{Je moet dit alleen.}{inzien en begrijpen dat}{verzet je niets helpt}\\

\haiku{Zij waardeerde en,.}{beoordeelde mij geheel}{zooals ik verwacht had}\\

\haiku{Voorzichtig kroop ik:}{naar het bed en drukte zacht}{een kus op haar hand}\\

\haiku{- Neen, Vladimir, maar.}{ik geloof dat hij je spel}{min of meer doorziet}\\

\haiku{Volkomen waar, maar.}{dan toch in hoofdzaak voor hen}{die niets bezitten}\\

\haiku{Als iemand het ooit,.}{goed met mij gemeend had dan}{was hij het geweest}\\

\haiku{Welke pogingen,}{ik ook aanwendde ik bleek}{onmachtig te zijn}\\

\haiku{Ik werd pas stil, toen.}{ik haar gelaat plotseling}{veranderen zag}\\

\haiku{het was op haar drift.}{in te gaan en het geval}{ernstig te nemen}\\

\haiku{ik ben maar een mensch.}{en zelfs aan mijn vermogens}{zijn grenzen gesteld}\\

\subsection{Uit: De wereld heeft geen wachtkamer}

\haiku{Toneel en kerk zijn.}{in de loop der eeuwen van}{elkaar afgegroeid}\\

\haiku{Het Wereldvenster - -.}{het wereldgebeuren brengt}{ze weer bij elkaar}\\

\haiku{Alles behalve,:}{een vrolijke wetenschap}{maar in elk geval}\\

\haiku{De Wereld heeft geen.}{Wachtkamer stelt ons allen}{verantwoordelijk}\\

\haiku{- Als Mary mij straks,.}{een handje helpt zijn wij er}{vlug genoeg doorheen}\\

\haiku{Maar eerst ga ik even,.}{naar de cantine een}{kop koffie drinken}\\

\haiku{Dichtbundels op de,.}{plaats waar hij zijn rapporten}{had moeten vinden}\\

\haiku{Welke zekerheid,?}{hebt u dat die storing zich}{niet herhalen zal}\\

\haiku{Hij draaide zich om,:}{en zei met zijn duim naar de}{luidspreker wijzend}\\

\haiku{Misschien valt het toch,.}{nog mee en breng ik er het}{leven af dacht hij}\\

\haiku{Zijn ironische toon.}{ontging haar en zij keek hem}{bewonderend aan}\\

\haiku{- Dat zoiets nu juist,,.}{moest gebeuren terwijl ik}{er niet was zei hij}\\

\haiku{Het is niet anders.}{en wij zullen er ons bij}{neer moeten leggen}\\

\haiku{Merkwaardig ook, dat.}{hij nu reeds alle lampen}{had aangestoken}\\

\haiku{U bent toch ook maar,?}{een gewone vrouw al bent}{u dan een dokter}\\

\haiku{Is het soms gek van,?}{me als ik mij bezorgd maak}{over mijnheer James}\\

\haiku{Trek je in ieder.}{geval niets van de praatjes}{van de mensen aan}\\

\haiku{Jouw deel van het werk,.}{kun je rustig doen ik ben}{tot je beschikking}\\

\haiku{Neerslachtiger dan,.}{hij was weggegaan keerde}{hij naar huis terug}\\

\haiku{Zoiets deed je toch,.}{niet als iemands mening je}{onverschillig liet}\\

\haiku{Begrijp toch wat het:}{voor mij betekenen zou}{als ik weigerde}\\

\haiku{- Dag mijn jongen, zei.}{hij zacht en verliet langzaam}{het ziekenvertrek}\\

\haiku{En tenslotte  .}{help ik door mijn sterven ook}{de wetenschap nog}\\

\haiku{- Ik dank je, zei hij,,,.}{eindelijk voor je vriendschap}{voor alles Mary}\\

\haiku{Hij stond op, liep naar.}{het hoofdeinde van het bed}{en keek zijn zoon aan}\\

\haiku{Benner stond voor het.}{geopende venster en}{staarde naar buiten}\\

\haiku{Maar ook zijn geduld.}{had grenzen en men moest nu}{tot een eind komen}\\

\section{Grazia Deledda}

\subsection{Uit: De weg van het kwaad}

\haiku{Houd je voet schrap en.}{je zult zien dat hij je geeft}{wat je wilt hebben}\\

\haiku{{\textquoteright} - {\textquoteleft}Maar wat voor roddels,,?}{Zia Luisa wat valt er over}{mij te beweren}\\

\haiku{- {\textquoteleft}Morgen moet je naar,?}{ons weiland in de vallei}{weet je waar het is}\\

\haiku{{\textquoteright} Zio Nicola ging,.}{achter een klein ongedekt}{tafeltje zitten}\\

\haiku{Hij merkte dat ze,.}{hem niet alleen minachtte}{maar ook wantrouwde}\\

\haiku{- {\textquoteleft}Dat is \'e\'en{\textquoteright}, zei hij.}{en liet zich lenig uit de}{perenboom glijden}\\

\haiku{Een vluchtige blos.}{kleurde het bleke gezicht}{van het dienstmeisje}\\

\haiku{De hond was naar de,.}{poort toegerend en krabde}{eraan blij jankend}\\

\haiku{- {\textquoteleft}Dat werd tijd{\textquoteright}, zei ze,. -}{een punt van haar hoofdband naar}{haar schouder trekkend}\\

\haiku{Voor een poosje was}{het eentonige geluid}{van de koffiemolen}\\

\haiku{Ze lachte en schold.}{en dreigde het tegen Zia}{Luisa te zeggen}\\

\haiku{-{\textquoteleft}Baas, is dat?}{een manier om de mensen}{te laten werken}\\

\haiku{Droevige zaak als,.}{je niet rijk geboren bent}{uit een machtig ras}\\

\haiku{Misschien met een heer,,.}{een gestudeerd heer misschien}{met een rijke boer}\\

\haiku{Ze past niet bij me.}{en mijn verlangen kan me}{tot waanzin brengen}\\

\haiku{Geest van mijn moeder,.}{help me.  Bevrijd me van}{slechte denkbeelden}\\

\haiku{Kerstmis komt en dan.}{gaan we zingen en drinken}{met Zio Nicola}\\

\haiku{Hij werd midden in.}{de nacht wakker en trok zich}{in de schuur terug}\\

\haiku{eenden, schaakstukken,,.}{piramides kruisen en}{zelfs priestermutsen}\\

\haiku{Daar was Nuoro al,,.}{omgeven door de wind in}{de sombere avond}\\

\haiku{Het is koud, maar we.}{zijn niet zo fijn gebouwd als}{de hoge heren}\\

\haiku{horen de vrouwen,.}{in bed en nergens anders}{zo denk ik erover}\\

\haiku{Die liefde was als;}{de vlam die langs het droge}{mos had gestreken}\\

\haiku{Ik kan je kwaad doen,,.}{maar ik wil het niet het komt}{zelfs niet bij me op}\\

\haiku{- {\textquoteleft}Zo net brandde het,.}{nog ik begrijp niet waarom}{het is uitgegaan}\\

\haiku{Ze stribbelde een,.}{beetje tegen maar zonder}{geluid te maken}\\

\haiku{Hij stond op en ging:}{naar bed met steeds dezelfde}{vreugde in zijn hart}\\

\haiku{Ik verkoop meteen,,.}{het huis en het land alles}{en ik word koopman}\\

\haiku{En hij wilde haar,,.}{puur trouwen hoogstens gezoend}{en alleen door hem}\\

\haiku{in een goede biecht.}{wordt de geest gewassen als}{een doek in de bron}\\

\haiku{En hoe had hij het?}{gewaagd zijn  oog op haar}{te laten vallen}\\

\haiku{Wie van hen is het{\textquoteright}?}{waard Francesco Rosana}{s schoen te strikken}\\

\haiku{Maria bleef een paar.}{dagen op haar eigen land}{en bloeide er op}\\

\haiku{Uit elk dorp uit de;}{omgeving kwamen mensen}{naar Gonare}\\

\haiku{een blauwige damp.}{omhoog als de adem van de}{koortsige aarde}\\

\haiku{Maar ze durfde haar.}{duistere hartsgeheimen}{niet uit te spreken}\\

\haiku{{\textquoteright} - {\textquoteleft}Het spijt me, beste,.}{meid maar we hebben het langs}{de weg verloren}\\

\haiku{Het viel iedereen.}{op en hij deed geen moeite}{het te verbergen}\\

\haiku{Het is treurig om,.}{aan te zien zo mager als}{hij is geworden}\\

\haiku{{\textquoteright} De reiziger ging,.}{weg maar Pietro floot om hem}{terug te roepen}\\

\haiku{Ze kan me niet op{\textquoteright}.}{zon Judasachtige}{manier verraden}\\

\haiku{Hij greep hem bij zijn.}{halsband en trok hem achter}{de hoek van de muur}\\

\haiku{{\textquoteright} Maria voelde dat.}{zijn woorden voor haar bestemd}{waren en werd bang}\\

\haiku{Maar hij herhaalde.}{zijn minachtende gebaar}{en brak de zin af}\\

\haiku{Toen hij alleen was.}{gaf Pietro zich over aan zijn}{woede en wanhoop}\\

\haiku{Hoe graag had hij het!}{zaaigraan vergiftigd of in}{de wind gegooid}\\

\haiku{De weg hield niet op.}{en hij wist trouwens ook niet}{waar hij heen ging}\\

\haiku{-{\textquoteright}Als de winter.}{komt slaap ik weer onder dat}{rampzalige dak}\\

\haiku{- {\textquoteleft}Weet je zeker dat?}{ik me niet met opzet heb}{laten oppakken}\\

\haiku{{\textquoteright} - {\textquoteleft}Ja{\textquoteright}, zei Pietro, - {\textquoteleft}maar.}{bezit is niet genoeg om}{gelukkig te zijn}\\

\haiku{Na de nacht van zijn.}{vrijlating was Pietro niet}{meer langsgekomen}\\

\haiku{En ja, als je de,.}{spullen hebt dan moet je je}{niet karig tonen}\\

\haiku{Ze keerde terug.}{in haar kamer en legde}{haar bruidskleren klaar}\\

\haiku{{\textquoteright} Maar hij sloeg zijn arm{\textquoteright}.}{om Marias middel en}{wilde haar kussen}\\

\haiku{Ja, nu was alles,.}{voorbij nu was er niets meer}{om bang voor te zijn}\\

\haiku{Of moest hij naar de?}{keuken om op zijn plaats van}{knecht te gaan zitten}\\

\haiku{Het cadeau heb ik,.}{haar gegeven maar die zoen}{moet ik nog hebben}\\

\haiku{{\textquoteright} - {\textquoteleft}Als jij haar een zoen{\textquoteright},.}{geeft doe ik het ook zei de}{jonge eigenaar}\\

\haiku{De uitgestrekte.}{tanca was omgeven door}{begroeide muurtjes}\\

\haiku{- {\textquoteleft}Nee, ik vergis me{\textquoteright},.}{niet dacht Maria en keerde}{naar de hut terug}\\

\haiku{Wat gebeurde er,?}{daarginds in het nu geheel}{zwart geworden bos}\\

\haiku{dan kan hij zich op.}{zijn beurt ongerust maken}{als hij terugkomt}\\

\haiku{Wat had ik gelijk,!}{om ongerust te zijn ik}{was te gelukkig}\\

\haiku{Francesco zou,.}{niet dood zijn ik zou niet zo}{geleden hebben}\\

\haiku{Als ik hem kwaad had,.}{willen doen dan had ik het}{eerder kunnen doen}\\

\haiku{- {\textquoteleft}Daar ben ik{\textquoteright}, zei ze,, - {\textquoteleft}?}{vriendelijk maar niet teder}{Wat wil je van me}\\

\haiku{Ik ben klaar met het.}{inzaaien van het graan en}{ik ga hout kappen}\\

\haiku{{\textquoteright} - {\textquoteleft}Praat daar niet over{\textquoteright}, zei,.}{Pietro en beet zich in zijn}{vuist -{\textquoteright}Hou je mond}\\

\haiku{- {\textquoteleft}Daarna doe ik wel{\textquoteright},.}{het offici\"ele verzoek}{lachte Antine}\\

\haiku{Soms vroeg ze zich af.}{of ze opnieuw van Pietro}{was gaan houden}\\

\haiku{vandaag is het feest{\textquoteright},.}{en de boeren zijn in het}{dorp zei Antine}\\

\haiku{Hij was zo knap, zo,.}{goed gekleed met ogen die van}{geluk fonkelden}\\

\haiku{Een krachtige klop.}{op de poort wekte haar uit}{deze droomtoestand}\\

\haiku{{\textquoteright} En naarmate het.}{kwaad zich ophoopte weerklonk}{die vraag steeds sterker}\\

\haiku{Voor de mensen moest,..}{zij zich opofferen haar}{hele leven lang}\\

\haiku{Ze is geboren,,.}{om te vechten te strijden}{verraad te wreken}\\

\haiku{Misschien de eerste,:}{avond toen hij haar had omarmd}{en haar had gezegd}\\

\haiku{Pietro had een stap.}{achteruit gedaan en bleef}{naar haar kijken}\\

\section{Val\`ere Depauw}

\subsection{Uit: Het late geluk van Remi Zwartekens}

\haiku{waarom heeft liefje}{een dokus en eksteroog}{helemaal geen}\\

\haiku{Meneer Zwartekens,}{vooraleer gedichten uit}{te geven zoudt}\\

\haiku{Bestonden er dan?}{twee manieren om zonder}{fouten te schrijven}\\

\subsection{Uit: Niet versagen, Mathias}

\haiku{Val\`ere Depauw,}{Niet versagen Mathias}{Colofon}\\

\haiku{Hoe verheugd zou ze.}{nu zijn met dit boek van haar}{geliefden auteur}\\

\haiku{{\textquoteleft}Ik word opgejaagd...}{als wijd en dan moet ik nog}{wachten op mevrouw}\\

\haiku{{\textquoteleft}Ik had u nochtans,{\textquoteright}.}{gevraagd om zeven uur hier}{te zijn overdreef zij}\\

\haiku{En als het moest... ~ {\textquoteleft},{\textquoteright}.}{Maar hij moet voorzichtig zijn}{herhaalde Simon}\\

\haiku{Waarom?{\textquoteright} {\textquoteleft}Ja, hoe moet,{\textquoteright}.}{ik dat uitleggen begon}{Simon voorzichtig}\\

\haiku{Als de anderen,.}{de eerste gebruiken des}{te erger voor hen}\\

\haiku{{\textquoteleft}Nog niet, Lagrange...{\textquoteright} {\textquoteleft},?}{laat niet afMaar het komt wel}{in orde nietwaar}\\

\haiku{Maar Pieter-Jan bleek.}{aan dat doktersbezoek geen}{belang te hechten}\\

\haiku{Er was een nieuwe,.}{kracht in hem hij was sterk en}{niet te overwinnen}\\

\haiku{Hij had besloten:}{dat er een nieuwe fabriek}{gebouwd moest worden}\\

\haiku{{\textquoteright} 's Middags gingen,.}{zij samen naar Simon die}{den brief vertaalde}\\

\haiku{Hij was het evenbeeld,.}{van zijn moeder zoowel physiek}{als van karakter}\\

\haiku{En als er ook in...}{Anton niet veel liefde voor}{het bedrijf zijn moest}\\

\haiku{Als ge het geld weer,.}{missen kunt brengen wij het}{weer naar de spaarkas}\\

\haiku{Maar Maria kon zich.}{hierom niet verheugen en}{ze kon niet trotsch zijn}\\

\haiku{Het was een dure.}{les geweest en nooit meer zou}{hij zich zoover wagen}\\

\haiku{En zoo er toch iets,?}{moest gebeuren zoo er toch}{iets moest gebeuren}\\

\haiku{En het was ook lang,.}{geleden dat ze zoo had}{kunnen glimlachen}\\

\haiku{Lagrange was niet,.}{op zijn gemak dat kan ik}{u verzekeren}\\

\haiku{Over dien eersten Mei,:}{was er nog niet gesproken}{maar nu vroeg Simon}\\

\haiku{Hij was groot en sterk,.}{gebouwd maar hij had niet de}{fijnheid van Lambert}\\

\haiku{Henri Lagrange,.}{de kleinzoon van den ouden}{Charles Lagrange}\\

\haiku{Louis Lagrange moest.}{dicht bij de vijftig zijn en}{hij was weduwnaar}\\

\haiku{In 1905 brak tusschen...}{Lagrange en Wieringer}{het groote conflict los}\\

\haiku{En dat weten de.}{wevers en daar moeten wij}{gebruik van maken}\\

\haiku{En er is ook nog...}{de manier waarop die tien}{percent ge\"eischt wordt}\\

\haiku{In elk geval was.}{De Canni\`eres op de}{hand van Lagrange}\\

\haiku{{\textquoteleft}Wat er m\'e\'er kan mee?}{bereikt worden dan met een}{gewoon armuur}\\

\haiku{Op een morgen kwam.}{Simon Mathias op zijn}{bureau opzoeken}\\

\haiku{Het gebeurde in.}{het groote bureau en alleen}{de experts werkten}\\

\haiku{Ik, als griffier, heb.}{hier niet te bevelen en}{niet te verbieden}\\

\haiku{Het brevet, dat hij,,...}{den voorzitter voorlegt kan}{waardeloos zijn valsch}\\

\haiku{{\textquoteright} En het was goed, dat.}{Maria van alles op de}{hoogte werd gesteld}\\

\haiku{Meneer Lagrange,{\textquoteright}.}{heeft het recht de huiszoeking}{te doen zegde hij}\\

\haiku{En gingen zij ten,,!}{onder dan was het vechtend}{verbeten vechtend}\\

\haiku{En ginder tusschen...}{de getouwen stapte de}{oude Lagrange}\\

\haiku{Maar Wieringer zal,.}{het wel volhouden denkt men}{te Steenbrugge}\\

\haiku{En Anton, die nu.}{zestien was en steeds meer op}{zijn vader geleek}\\

\subsection{Uit: Toch lammeren, broers! (onder ps. Piet Canneel)}

\haiku{{\textquoteleft}Zij zijn moedig en,.}{sterk en zij hebben nog een}{lange tijd voor zich}\\

\haiku{En hopen, dat het,,.}{eens beteren zal ach dat}{kunnen zij niet meer}\\

\haiku{Zij was de laatste,...}{dagen alleen wat vermoeid}{maar dat ging wel over}\\

\haiku{Een eigen huis, dat,.}{prachtig ingericht was en}{heel wat contanten}\\

\haiku{Ge kunt het niet zien,:}{zoals hij daar zit maar hij}{heeft geen voeten meer}\\

\haiku{{\textquoteleft}En die andere,{\textquoteright}.}{jonge kerel aan het raam}{vervolgde Lode}\\

\haiku{Neen, neen{\textquoteright} zegde ik, {\textquoteleft},,}{op mijn beurtniet weigeren}{Mon niet weigeren}\\

\haiku{Zij dacht er ernstig.}{over na en dat was wellicht}{de beste uitkomst}\\

\haiku{{\textquoteleft}En nu wordt het tijd,.}{dat ge naar de steenweg gaat}{of ge komt te laat}\\

\haiku{{\textquoteleft}En nu stappen wij,.}{gauw op want het is te koud}{om te blijven staan}\\

\haiku{De kinderen zijn.}{bij mijn zuster en ik werd}{ook uitgenodigd}\\

\haiku{Hij hoefde echter,.}{niet te antwoorden want hij}{duwde het hek open}\\

\haiku{{\textquoteleft}Ik heb tot in de.}{minste bizonderheden}{alles meegemaakt}\\

\haiku{Hoogopgericht, trots,.}{alsof zij ook aan de paal}{vastgebonden was}\\

\haiku{Rikus, als tweede,,:}{oudste vond dat hij ook een}{woordje mocht plaatsen}\\

\haiku{Ivo stond voor hem en,.}{die grote sterke kerel}{weende als een kind}\\

\haiku{{\textquoteleft}En ze laat u nog.}{eens vragen of gij er nooit}{spijt zult om hebben}\\

\haiku{{\textquoteleft}En ik zal er nooit...}{spijt over hebben met Elza}{gehuwd te zijn}\\

\haiku{Een schoon kindje, het,...}{evenbeeld van de moeder trek}{voor trek het evenbeeld}\\

\haiku{En nu moet ge gaan,,.}{Lowieke want ik heb nog}{enkele klanten}\\

\haiku{laet ons singhen blij, Daer...}{meed oock onse leisen}{beghinnen vrij}\\

\section{Lodewijk van Deyssel}

\subsection{Uit: De scheldkritieken}

\haiku{Ik zet het je, ze.}{op een geschikte manier}{terug te leggen}\\

\haiku{De Europeesche,,.}{naties die heden stilstaan}{zijn morgen te niet}\\

\haiku{Thands behoort deze,;}{schrijver tot dien kliek gene}{tot een andere}\\

\haiku{Ontzachlijk in zich.}{zelf en ontzachlijk in het}{begrip der menschen}\\

\haiku{Wij willen Holland.}{hoog opstooten midden in}{de vaart der volken}\\

\haiku{Mij is het een lust,}{te schimpen op uw rimpels}{die geen groote Idee u}\\

\haiku{O, kon ik u zoo,!}{krenken dat uw hersens van}{den schrik verlamden}\\

\haiku{'t Is altijd feest, '.}{in onze Letterent}{is altijd kermis}\\

\haiku{Het wordt tijd om tot.}{een duidelijk begrip te}{komen van dit woord}\\

\haiku{{\textquoteleft}de meesten stelden{\textquoteright}.}{het uit tot morgen zegt een}{porder tot zijn vrouw}\\

\haiku{middelmatigheid,.}{en wankunst zich vertoonen}{ze te signaleeren}\\

\haiku{Gij vertoont den mensch,,...}{zeker maar als denkeres}{en kunstenares}\\

\haiku{Ja, malen, dat is ',;}{t wat gij doet en met een}{verheven penseel}\\

\haiku{Mag ik er dus op,?}{rekenen dat u niet meer}{zoo hard werken zult}\\

\haiku{welken indruk heb?}{ik van dat boekje in zijn}{geheel ontvangen}\\

\haiku{{\textquoteleft}er gaapte als een{\textquoteright};}{onoverkomelijke}{klove tusschen hen}\\

\haiku{je bent, ja, je bent,...,,...!}{een auteur een schrijver een}{romancier ondeugd}\\

\haiku{14 Juni 1890        [.}{Romans van Maurits 58}{Goena-Goena}\\

\haiku{wat duivel kan 'et,!}{m{\'\i}jn ook biete het raakt me}{eigelijk geen lor}\\

\haiku{Waarin Amor als naar,,.}{gewoonte bewijzen van}{zijn blindheid geeft IV}\\

\haiku{De opschriften dus.}{bederven eenigermate}{het geheele spel}\\

\haiku{De verhalende;}{gedeelten van dit werkje}{zijn niet de beste}\\

\haiku{Haar zuster 68~		 [,.}{Tooneelspel in vier bedrijven}{door Marcellus Emants}\\

\haiku{niet zoozeer om aan eene,;}{dame te zenden ook niet}{voor de kinderen}\\

\haiku{Doch, het is waar ook,.}{wij zouden een overzicht van}{het verhaal geven}\\

\haiku{Te gelijk is er.}{een boekje met gedichten}{van den heer G.H. Priem}\\

\haiku{Toen ben ik gretig.}{met mijn neus er op en er}{in gaan snuffelen}\\

\haiku{En... e... ze had - we -,?}{zijn onder  ons nog al}{fortuin geloof ik}\\

\haiku{In 't algemeen:}{zijn de waarheden omtrent}{den heer Byvanck}\\

\haiku{Byvanck dont le...}{livre sur Paris a \'et\'e}{publi\'e en fran\c{c}ais}\\

\haiku{Zij, de mannen van,.}{den dijk werden allengskens}{helden in mijn droom}\\

\haiku{Zij kenden geen God,.}{dan de God der waarheid zij}{dienden geene partij}\\

\haiku{Gisteren heb ik.}{een artikel van 3 blz.}{over Quack geschreven}\\

\haiku{maar ik, ten minste,,,.}{heb geld noodig niet voor mij maar}{voor mijn kinderen}\\

\haiku{Goes, die kan zulke.}{dingen terdege aardig}{aan de man brengen}\\

\haiku{Hoe gaat het in je.}{tweezaamheid en wanneer komt}{nu toch je roman}\\

\haiku{4, april 1888, blz. 163-,;}{169 en aldaar gesigneerd}{met de letters F.H.}\\

\haiku{(Een {\textquotedblleft}boekbe\"oordeeling{\textquotedblright}).}{van het werkje van dien naam}{door Soera Rana}\\

\haiku{die oude fraze,:}{die geen schoolkind van Delfzijl}{meer op zo\^u schrijven}\\

\haiku{Daarin onthulde.}{hij dat Marie de naam was}{van zijn verloofde}\\

\haiku{Netscher, niet in staat,.}{te troosten liet Professor}{ten leste alleen}\\

\haiku{Eindelijk, niet dan,...}{toevallig viel zijn blik op}{het leelijke woord}\\

\haiku{De van Eeden die,.}{erin komt is mijn vader}{de botanicus}\\

\haiku{Zoo een bruidegom;}{kan wanhopig in een hoek}{gaan zitten staren}\\

\haiku{Studi\"en op het,.}{gebied der Letterkunde}{door W.G.C. Byvanck}\\

\haiku{Il se donne tout,.}{entier voil\`a le secret de}{sa puissance}\\

\haiku{Gij zijt hier te dicht,,.}{te dicht zijt gij hier bij wat}{mij heilig is}\\

\haiku{Zien zij er niet uit?}{als een menigte kleine}{ovale grafzerkjes}\\

\haiku{-{\textquoteright} ~ 123 Hierachter (:}{in het handschriftvoorzien van}{de kanttekening}\\

\haiku{of ik dit ben of,,,...}{dat ben of niet ben of hoe}{het nu met mij z{\'\i}t}\\

\haiku{82, 249-56, 290, 350-,:}{3 ~ Calderon de}{la Barca Pedro}\\

\haiku{297 Hilman, J.: 161,,,:,,-,:}{183 328 Hofdijk W.J. 33 211}{2869 343 Holda}\\

\haiku{30, 33, 41, 52, 71-,,-,,,,,:}{3 116 1224 145 186 203}{343 Murger Henri}\\

\haiku{F. van: 80-84, 312,:,,,,,:}{Reijnders Karel 290 293 339}{347 348 358 Rembrandt}\\

\haiku{Een aangrijpend schoon.}{panorama lag dan aan}{mijn voeten ontrold}\\

\subsection{Uit: Verzamelde werken. Deel 2. De kleine republiek}

\haiku{Het portier stond al.}{open en er was bijna geen}{tijd meer voor Mietje}\\

\haiku{De vader van zijn,.}{vrouws bezorgdheid Willem ziek}{dadelijk schrijven}\\

\haiku{En er kwamen er,.}{achter Willem aan waarnaar}{hij niet dorst omzien}\\

\haiku{Maar het was half elf.}{en de klok bielebangde}{door de gebouwen}\\

\haiku{alles was stil, de,;}{slaapzaal le\^eg de jongens al}{lang naar beneden}\\

\haiku{Daarna keek Willem.}{tegen zijn spiegeltje of}{zijn h\'aar wel goed zat}\\

\haiku{Vertel nu eens, denkt,.}{ge dat ge U hier nogal}{zult kunnen schikken}\\

\haiku{- Zoo, en wat was dat,,?}{dan wel die ondeugendheid}{wat deed ge dan wel}\\

\haiku{- Och, dat weet ik niet,,,.}{meneer ja ze vonden het}{wel eens nog al erg}\\

\haiku{Pieterse, weet gij,}{er niet iets meer van gij zit}{daar zoo ernstig v\'oor}\\

\haiku{- Zoo, mot je scheiten,,.}{alle mattessen zitten}{vol eerst met me me\^e}\\

\haiku{Hij kauwde achter,.}{zijn lippen met chocola}{aan de mondhoeken}\\

\haiku{Kom, dat neukt niet, 't ' ',.}{isnn goeye hier die doet}{net of-i niks merkt}\\

\haiku{hee, zou Willem nog,.}{niet uit school komen het is}{toch al laat genoeg}\\

\haiku{Stelhuis en Brik, zijn,, '}{oudste plagers pakten hem}{beet brachten hem in}\\

\haiku{Hier tegen had hij:}{nog nooit in zijn leven iets}{gedaan of gebiecht}\\

\haiku{Maar er was niets, en.}{de lampen waren nog niet}{ne\^ergedonkerd}\\

\haiku{Hij zag den duivel,,.}{niet maar de duivel was een}{geest dus onzichtbaar}\\

\haiku{Hij lei zijn hand uit',:}{in de buiging van Mets arm}{zei tot hem en Breg}\\

\haiku{De groote nachtmis van '.}{den eersten Kerstdag was om}{vijf uurs ochtends}\\

\haiku{Hij was hier nu thuis,,.}{gemeenzaam onverschillig}{voor al het nare}\\

\haiku{Daar was 't weer, als,.}{een groot-lieve vogel}{die kwinkeleerde}\\

\haiku{Zij was niet dood, zij,,.}{was niet weg zij leefde nog}{zij zou bij hem zijn}\\

\haiku{Eindelijk was hij,.}{aangeheschen de kraag nog}{\'op tegen den hals}\\

\haiku{hier samenzijnd om,.}{dan los te gaan  ieder}{naar zijn kompanjie}\\

\haiku{En daar brandden de.}{zondaren eeuwig zonder}{ooit verteerd te zijn}\\

\haiku{Hij bad verder, maar:}{zijn gedachten bleven stil}{bij die woorden van}\\

\haiku{Zoudt ge dan rein zijn?}{geweest en in den hemel}{hebben kunnen gaan}\\

\haiku{- Ja, 't is toch h\'eel, ',.}{mooi je zalt n\'ooit zien dat}{mot ik bekenne}\\

\haiku{voor geen geld van de,...}{wereld we\^er wille slape}{in dat kamertje}\\

\haiku{- Nou, as ge d'r toch,.}{niet van krijgt dan hoefde dat}{ook niet te wete}\\

\haiku{Het was een zoete,, ',.}{vreugde ja nu voelde hij}{t wel langzaamaan}\\

\haiku{Na de wandeling,.}{veindse Willem vreeselijk}{hoofdpijn te hebben}\\

\haiku{Ze hadden gedamd,,:}{om wat te doen en telkens}{had Willem gezeid}\\

\haiku{Hij voelde de school,.}{nu in de verte in de}{vlak-bij-toekomst}\\

\haiku{- Hij is niet op school,,.}{geweest zei ze hij heeft een}{goeverneur gehad}\\

\haiku{De stem van Hoeffel.}{joeg hoog op voor de vensters}{achter Willems rug}\\

\haiku{Eindelijk waren,.}{ze in het bosch op den berg}{waar de kapel stond}\\

\haiku{Zij lachten om den,,,.}{kleine den leelijke den}{bleeke den gesarde}\\

\haiku{zeg-gij 't maar,,:,, ';}{Tiessen of kom Tiessen gij}{zultt wel weten}\\

\haiku{Pauli schuurde de.}{pop en en de proppen af}{van den katheder}\\

\haiku{Toen Pauli we\^er op,.}{z'n plaats zat kwam Piet-Suf}{we\^er voor de klas staan}\\

\haiku{En hij stond even stil.}{in de wijde nis van den}{groen-gouden dag}\\

\haiku{je, anders kan je '.}{wanneer je maar wiln pak}{slaag van me krijge}\\

\haiku{- Nou, als je maar wil,,,.}{van-daag morrege of}{met de vakancie}\\

\haiku{Er moest alleen op,.}{gelet worden dat Blaise}{d'r niks van merkte}\\

\haiku{Dani\"el had ook.}{allerlei streekjes waarme\^e}{hij Hoeffel plaagde}\\

\haiku{nu we\^er minder bij,.}{Scholten werd vernederd en}{geslagen zijnde}\\

\haiku{Hij nam zijn handen.}{uit zijn broekzakken en deed}{er een aan zijn kin}\\

\haiku{Dit was een gevoel,}{dat Willem niet gehad had}{sints het eerste jaar}\\

\haiku{- A jesses wat 'n,... -}{keerel zei Willem as je nou}{toch vooruit beloofd}\\

\haiku{- Kom, beloof 'et nou,:}{dan krijg je wat moois van me.}{Dani\"el lachte}\\

\haiku{- Nou, daar mot toch ies,,.}{voor zijn zei Willem late}{we wat verzinne}\\

\haiku{Hij trad voort tot aan,,.}{de stoep ging langzaam de stoep}{af en wachtte we\^er}\\

\haiku{die jongen... Zo\^u hij?...}{misschien ook al een beetje}{van Willem houden}\\

\haiku{hem, hij keek ne\^er en.}{het was of het windje de}{aandacht had verwaaid}\\

\haiku{die le\^e nu ook school;}{en ha\'ar vakancie was toen}{nog niet begonnen}\\

\haiku{Hij stak zijn tong uit.}{en trok rare gezichten}{tegen het kajee}\\

\section{Els Diederen}

\subsection{Uit: Platbook 5. Moder Maas}

\haiku{Ich hei aevel wel,,.}{mien druime en verlanges}{veural veul twiefel}\\

\haiku{M\`et mien kleinkinger '.}{veur oppe fiets geitt wie}{vanzelf richting Maas}\\

\haiku{Wat haet de Pletsjbaek '.}{toch gel\"ok en wat b\`en ichne}{gel\"okkige miensj}\\

\haiku{zoe st\`ellekes,.}{duide ritselend door de}{wind in cadans}\\

\haiku{Ik waas zoe\"e blie wie ',,.}{n kind wat neet zoe\"e maf waas}{ik waas nog ein kind}\\

\haiku{Miene pap dach dao,.}{aevel anders euver maar}{zag niks taege mam}\\

\haiku{Ich drejdje 't,:}{stuur verkie\"erdj precies d'n}{angere kantj op}\\

\haiku{zien bein kwoeam gekneldj.}{t\"osse de grindjbak en}{de baggermuuele}\\

\haiku{Det zien vroumes neet?}{k\'os verdrage det hae d'n}{hie\"ele daag thoes waas}\\

\haiku{door 't landjsjap struimp  ,:}{Waar de brede stroom der Maas}{statig zeewaarts vloeit}\\

\haiku{De Pruse h\"obbe}{de streek inne zestiger}{en zeventiger}\\

\haiku{Ich woondje doe in '.}{t centrum van Remunj op}{get mie\"e dan 20 m+NAP}\\

\haiku{De baek klatert de, ',}{wiejert inn M\"osj l\`esjt h\"a\"oren}{doorsj Woa birk en \`esj}\\

\haiku{'t Is weer voorbij,.}{die mooie zomer zong d'r Jerard}{Cox \'op d'r radio}\\

\haiku{En det geit 't b\`es '.}{mit angere same op}{t grote vaer}\\

\haiku{Doa brent nog jee lit. ' '.}{t Is jraad oft de sjtem}{van d'r Wullem huet}\\

\section{Jules Dister}

\subsection{Uit: Vermorzelde honden}

\haiku{De betrokkene.}{zou vaker met een koffer}{gesignaleerd zijn}\\

\haiku{Ergens tussen Luik.}{en Maastricht houden ook zij}{op te bestaan}\\

\haiku{Het enige wat hij,.}{niet kan is spreken anders}{was hij net een mens}\\

\haiku{{\textquoteright} Mijn oom Pie is geen.}{partij voor deze monsters}{van de didactiek}\\

\haiku{Wit haar wolkt om zijn,,.}{hoofd een stralenkrans een}{zilveren monstrans}\\

\haiku{Emoties zijn niet zijn,.}{sterkste kant laat staan dat hij}{ze zich herinnert}\\

\haiku{Aan het eind van de.}{oorlog deed ze opnieuw iets}{verbazingwekkends}\\

\haiku{En wat wil nu het,?}{verhaal ongelofelijk}{als het leven zelf}\\

\haiku{Het is een verhaal,?}{dat tot de verbeelding spreekt}{maar is het ook waar}\\

\section{R. Dobru}

\subsection{Uit: Wan monki fri. Bevrijding en strijd}

\haiku{De mensen die toen!}{de bovenlaag in de}{maatschappij vormden}\\

\haiku{Jozua pakt zijn tas.}{onder de arm en                     voegt}{zich bij de stakers}\\

\haiku{Hij is toen in de,.}{handel gegaan met het}{noodlottig gevolg}\\

\haiku{Bij springvloed konden.}{wij met de korjalen}{tot in huis komen}\\

\haiku{Een goede oogst van,;}{wie                     dan ook was voor de}{hele plantage}\\

\haiku{De gracht, een goot, werd.}{aan weerskanten begrensd door}{twee                     zandwegen}\\

\haiku{Enkele maanden.}{later kon het gezin weer}{bijeen gaan wonen}\\

\haiku{En u hebt God lief,.}{wanneer u uw                     naaste}{lief hebt als uzelve}\\

\haiku{Hij                     vertelde.}{ons dat zijn familie dat}{van hem had ge\"eist}\\

\haiku{Veredelen is.}{iets voeren van primitief}{naar                     ontsloten}\\

\haiku{Wij hadden allen.}{een kruidbad gekregen van}{mijn                     grootmoeder}\\

\haiku{Van mijn nicht waren.}{niet eens de                     haren aan}{haar benen verschroeid}\\

\haiku{De oude vrouw kwam(!).}{zowaar met een Bijbel in}{de hand aan tafel}\\

\haiku{Terwijl ik met hem,.}{zat te praten krijgt hij \'e\'en}{van zijn                     winti's}\\

\haiku{Ik vond het jammer.}{dat ze te                     arm was om}{het te betalen}\\

\haiku{Tot nu toe zal je.}{protesten horen van de}{Kreoolse meisjes}\\

\haiku{Tegenwoordig moet!}{je de Hindostaanse}{meisjes zien dansen}\\

\haiku{Wat je altijd bij;}{de Kreoolse meisjes vindt}{is bille baja}\\

\haiku{Een heleboel                     .}{oude dogma's waren weer}{kapotgeslagen}\\

\haiku{Wi Egi Sani                     .}{hield toen in 1957 haar eerste}{kultureel kongres}\\

\haiku{Naarmate de man,.}{sprak                     kreeg ik meer en meer}{sympathie voor hem}\\

\haiku{Wij maakten slogans,.}{en                     bralden die uit om}{gehoord te worden}\\

\haiku{De werkgever die.}{een paradijs had tijdens}{Pengels regiem}\\

\haiku{- De eenwording van;}{het Surinaamse volk wordt}{erdoor belemmerd}\\

\section{Neel Doff}

\subsection{Uit: De avond dat Mina mij meenam}

\haiku{De huurders die er ',;}{t best van konden komen}{woonden beneden}\\

\haiku{Intussen maakten,;}{de keukenmeid de min en}{ik de bedden op}\\

\haiku{{\textquoteright} En met haar handen.}{als kolenschoppen gaf ze}{mij een balletje}\\

\haiku{Ze liepen in een.}{wolk van Eau de Cologne}{en pepermuntgeur}\\

\haiku{Mijn zuster en de.}{andere meisjes droegen}{zulke stoffen niet}\\

\haiku{{\textquoteright} {\textquoteleft}Hoor eens, meisje, je,.}{komt hier nu al vijftien jaar}{altijd bij vlagen}\\

\haiku{Zo, en dat was dat..}{U moet niet denken dat u}{mij ooit terugziet}\\

\haiku{{\textquoteleft}Hier ouwe,{\textquoteright} zei ze, {\textquoteleft}.}{danhet is niet jouw schuld als}{je baas een hond is}\\

\haiku{thuis had hij het weer.}{te voorschijn gehaald en aan}{zijn vrouw gegeven}\\

\haiku{als je niets vertelt,.}{van dat wasgoed geef ik je}{de accordeon}\\

\haiku{Zijn benen waren,;}{hoog en gespierd zijn lange}{handen beweeglijk}\\

\haiku{Zindelijk houden.}{en naar school sturen kost echt}{niet teveel moeite}\\

\haiku{Bij tijd en wijle,.}{kwam hij naar huis maar alleen}{als ik er niet was}\\

\haiku{Op een morgen kwam.}{ze triomfantelijk bij}{ons binnentrippen}\\

\haiku{{\textquoteright} ~ Het dienstmeisje,,.}{waar Dirk hartstochtelijk van}{hield schonk hem een zoon}\\

\haiku{Bij de anderen,.}{zijn het de mannen waar ze}{niet tegen kunnen}\\

\haiku{Ik heb vijfhonderd,.}{frank ik wil alles voor m'n}{geld wat ervoor staat}\\

\haiku{Maar toen ik beter,.}{was is hij direkt bij me}{teruggekomen}\\

\haiku{{\textquoteright} {\textquoteleft}Dat kun je nu wel,...}{zeggen maar of Zouzou dat}{zo leuk zou vinden}\\

\haiku{Hier ben je wat, de,,}{mensen kennen je je staat}{goed aangeschreven}\\

\haiku{Jij lijkt haar te veel,!}{op een man en van mannen}{heeft ze al genoeg}\\

\haiku{een kind is een kind,.}{ik zal het niet verstoten}{om zijn velletje}\\

\haiku{{\textquoteleft}Als ze mijn kind was,.}{geweest had ik haar niet in}{het vak laten gaan}\\

\haiku{De rode is voor,.}{Fifi die is zo bruin als}{beukenootjes}\\

\haiku{{\textquoteright} {\textquoteleft}Als we allemaal,?}{een roos hebben zingt u dan}{een liedje met ons}\\

\haiku{Met de vuisten op:}{tafel leunde zij voorover}{en siste ademloos}\\

\haiku{ze lachte met een.}{helder geluid dat ver over}{het water schalde}\\

\haiku{{\textquoteright} En toen ze maar \'even,.}{een hand durfde losmaken}{gaf ze hem een klap}\\

\haiku{{\textquoteright} {\textquoteleft}Nou, Leen, als het aan:}{jou lag deed je nog zaken}{met de vissen hier}\\

\haiku{{\textquoteleft}Een vrouw met geld en;}{een neus voor zaken zou een}{schip kopen als dit}\\

\haiku{{\textquoteright} {\textquoteleft}Geef me tenminste,,.}{wat te drinken twee glazen}{dan heb ik weer moed}\\

\haiku{Ze keek even steels om,;}{zich heen alsof ze iemand}{zocht en ging toen mee}\\

\haiku{Je schaamt je omdat,.}{je een standje krijgt maar niet}{omdat je vuil bent}\\

\haiku{Ik bond het om haar;}{hoofd en met de strik midden}{tussen haar krullen}\\

\subsection{Uit: Dagen van honger en ellende}

\haiku{Geen van de kwalen.}{der armoede gaat aan de}{Oldema's voorbij}\\

\haiku{in hun huwelijk;}{kwamen twee ongerepte}{lichamen samen}\\

\haiku{hij toonde te veel,.}{dat domheid en gemeenheid}{hem tegenstonden}\\

\haiku{Een schok van de kar.}{deed de groote koffiemolen}{op mijn neus vallen}\\

\haiku{De hoofdgrachten van:}{Amsterdam boezemden mij}{grooten eerbied in}\\

\haiku{de roofjes op mijn.}{hoofd gingen open en het bloed}{liep mij in den hals}\\

\haiku{Ik had dus mijzelf,.}{beloofd dit jaar mijn eerste}{communie te doen}\\

\haiku{Ik koos een stoep aan,;}{een der grachten uit om mijn}{les te bestudeeren}\\

\haiku{Zij gaf antwoord met,.}{zoo'n zachte stem dat ik het}{nauwelijks verstond}\\

\haiku{wat verder was een.}{oase van boomen en een}{grasveld met bloemen}\\

\haiku{{\textquoteright} {\textquoteleft}Vader{\textquoteright}, zei ik, {\textquoteleft}laat;}{mij vannacht tusschen moeder}{en u in slapen}\\

\haiku{Heeft zij je nog niet,?}{verteld wanneer ze weer een}{kindje gaat koopen}\\

\haiku{onder 't draaien.}{net zooveel leven maakte}{als duizend bijen}\\

\haiku{Zijn gezicht straalde,,;}{zijn blauwe oogen werden zwart}{zijn lippen vochtig}\\

\haiku{De achterdeur van;}{een huis op den Nieuwendijk}{kwam in het slop uit}\\

\haiku{Wij legden hem in,.}{de wieg waar hij den heelen}{nacht bleef doorslapen}\\

\haiku{ik had opgemerkt,.}{dat rijke menschen spreken}{als in de boeken}\\

\haiku{Als er sprake was,.}{van reizen verloor vader}{alle bezinning}\\

\haiku{Aan den overkant van,.}{de gracht kwam een vrouw aan die}{iets in haar schort droeg}\\

\haiku{{\textquoteright} {\textquoteleft}Toe moeder, omdat.}{ie zoo geschrokken is toen}{ie van zoo hoog viel}\\

\haiku{Jij en ik wachten,,.}{nooit om wat voor ons staat te}{laten verdwijnen}\\

\haiku{Tegen het voorjaar,'.}{werd Baatje zoo dik en vet dat}{t een lust was}\\

\haiku{{\textquoteleft}En dan, je begrijpt,.}{de muizen loopen zoo maar}{tusschen zijn pooten}\\

\haiku{op het oogenblik;}{van koopen keert ze hem om}{en ontdekt een barst}\\

\haiku{Toen ging ik terug,.}{naar de volksstraten waar de}{verkoop beter ging}\\

\haiku{maar hij was nog zoo'n,;}{kind dat hij er nauwelijks}{eenig verdriet van had}\\

\haiku{hij volgde haar in.}{het donkere gangetje}{v\'o\'or onze kamer}\\

\haiku{als ik doortrokken;}{was geweest van parfums en}{in kanten gehuld}\\

\haiku{V\'o\'or hij wegging, gaf:}{hij me een paar gulden en}{herhaalde nog eens}\\

\haiku{Op de trap sprak hij ',.}{me int Fransch  aan maar}{ik verstond hem niet}\\

\haiku{{\textquoteleft}Er schiet niets anders,}{over dan den heelen nacht te}{blijven rondloopen}\\

\haiku{hij liep om de bank,;}{heen raapte het pakje op}{en ging langzaam weg}\\

\haiku{Ontmoedigd was ik.}{tot laat in den avond bij die}{vriendin gebleven}\\

\haiku{Hij kwam voorzichtig,.}{weer rechtop het geldstukje}{tusschen de tanden}\\

\haiku{{\textquoteright} {\textquoteleft}Moeder, nou haalt die!}{valsche meid me naar zich toe}{om me pijn te doen}\\

\haiku{{\textquoteleft}Ik denk, dat 't op,,}{vader neer zal komen als}{de zaak vervolgd wordt}\\

\haiku{Hij zag er bleek en;}{vervallen uit als een}{kleine vagebond}\\

\haiku{{\textquoteright} Wat deed 't mij goed,!}{tegen die geweldige}{borst aan te liggen}\\

\subsection{Uit: Dagen van honger en ellende}

\haiku{Zo kwam mijn broertje,.}{Kees altijd bij ons terug}{toen we klein waren}\\

\haiku{verschrikkelijk op.}{tot het niet veel scheelde of}{mijn ogen waren dicht}\\

\haiku{De hoofdgrachten van:}{Amsterdam vervulden mij}{met diepe eerbied}\\

\haiku{{\textquoteleft}Lieve hemel, waar?}{haalt dat kleine mirakel}{die woorden vandaan}\\

\haiku{Ik stond daar naast de,.}{zuster bevend van schaamte}{en vernedering}\\

\haiku{Al onze arme.}{kleintjes zijn op die manier}{op school behandeld}\\

\haiku{En met een wit bord.}{erop als deksel leek het}{ons heel behoorlijk}\\

\haiku{Ik had nooit 's nachts;}{bloemen gezien en kende}{dat verschijnsel niet}\\

\haiku{{\textquoteright} {\textquoteleft}Vader,{\textquoteright} zei ik, {\textquoteleft}mag?}{ik vannacht tussen moeder}{en u in slapen}\\

\haiku{Geholpen door een}{van de dienstboden van het}{huis pakte moeder}\\

\haiku{Wij legden hem in,.}{de wieg waar hij de hele}{nacht bleef doorslapen}\\

\haiku{{\textquoteright} Gelukkig stond er.}{een kamerschut tussen ons}{en de anderen}\\

\haiku{Als er sprake was,.}{van reizen verloor vader}{alle bezinning}\\

\haiku{Ze moesten mij in bed,.}{stoppen de opwinding had}{me koortsig gemaakt}\\

\haiku{Aan de overkant van,.}{de gracht kwam een vrouw aan die}{iets in haar schort droeg}\\

\haiku{de klompjes brandden;}{maar heel langzaam doordat ze}{kleddernat waren}\\

\haiku{Maar ze dachten dat.}{ik een arme duvel was}{die geen cent bezat}\\

\haiku{En wat jou aangaat,,.}{vrouwtje het is hoog tijd dat}{je goed verzorgd wordt}\\

\haiku{Aangezien ik vaak,:}{niet kon slapen hoorde ik}{ze soms overleggen}\\

\haiku{ik wist dan wat ze.}{voorhadden en deelde in}{hun ongerustheid}\\

\haiku{{\textquoteleft}Kijk eens of Keetje,.}{slaapt die meid ligt soms hele}{nachten te woelen}\\

\haiku{{\textquoteright} {\textquoteleft}Nou, dan moest je maar.}{eens een paar flessen wijn voor}{moeder meenemen}\\

\haiku{{\textquoteright} {\textquoteleft}Ja, zo zou het wel,{\textquoteright}.}{kunnen zei Mina na een}{ogenblik nadenken}\\

\haiku{Zij was bang dat de.}{klant zou denken dat ik al}{in het leven zat}\\

\haiku{Op de trap sprak hij,.}{me  in het Frans aan maar}{ik verstond hem niet}\\

\haiku{Hij draaide rond op,:}{zijn hakken sloeg zich op zijn}{dijen en brulde}\\

\haiku{Ik ben het zat, een.}{menselijk wezen hoort niet}{tussen jullie thuis}\\

\haiku{De eerste avond dat,:}{zij van hun werk thuiskwamen}{schrokken wij van hen}\\

\haiku{ik dommelde een,.}{beetje telkens oplettend}{of er onraad was}\\

\haiku{Hein en ik keken.}{elkaar aan en begrepen}{elkaars gedachten}\\

\haiku{{\textquoteleft}Ik weet wel dat het,,!}{onzin is maar vertel toch}{door het is zo leuk}\\

\haiku{daar heeft hij gelijk,.}{in ze heeft al even weinig}{beenderen als vlees}\\

\haiku{{\textquoteright} Wat deed het mij goed,!}{tegen die geweldige}{borst aan te liggen}\\

\haiku{De kinderen aten,;}{op tijd werden gewassen}{en gingen naar school}\\

\haiku{Maar opeens, als zij,.}{de vijftig al gepasseerd}{is komt het terug}\\

\section{A. den Doolaard}

\subsection{Uit: De druivenplukkers}

\haiku{wanneer ze lastig.}{werden zou hij ze wel in}{een hoekje drukken}\\

\haiku{niets te beginnen}{viel omdat hij een erfstuk}{was van de vader}\\

\haiku{- Dat meisje in de,,?}{keuken met dat donkere}{haar is dat jouw vrouw}\\

\haiku{Ze slenterden langs.}{de fontein omlaag naar het}{arbeiderslogies}\\

\haiku{- Je kunt vloeken, je,,.}{kunt zuipen zooveel als je}{wilt en ik doe mee}\\

\haiku{- zei hij, terwijl hij.}{zijn vuisten uit zijn warme}{broekzakken haalde}\\

\haiku{Zoo gauw ik geld heb,:}{voor een nieuw glazen oog laat}{ik kaartjes drukken}\\

\haiku{Ik wist dat hij dien, ',.}{avondt was een week daarna}{met haar slapen ging}\\

\haiku{Boven op de muur.}{stonden ze tegen de maan}{als twee groote katten}\\

\haiku{De zon zonk langzaam.}{naar de schuin toeloopende}{wanden van de schelf}\\

\haiku{Andr\'e lachte zoo.}{hard dat de hofhond er van}{begon te bassen}\\

\haiku{Heb jij er dan geen,,?}{verdriet van onding dat je}{druiven verrotten}\\

\haiku{- De markies is bang -.}{zich te compromitteeren}{werd er gefluisterd}\\

\haiku{Was zij iets anders,,?}{dan een koude vrouw die slechts}{nam maar nimmer gaf}\\

\haiku{Eindelijk kwam er:}{bericht uit Marseille van}{een arrestatie}\\

\haiku{bijen zoemden door;}{de wuivende schaduwen}{der dadelpalmen}\\

\haiku{Zoodra hij kon.}{nam hij afscheid en ging zijn}{lijfwacht bestellen}\\

\haiku{Wat hebben wij te?}{maken in het land waar de}{zon een krater is}\\

\haiku{de jonge Saporta,.}{ligt bij Verdun de jonge}{Blondel in het Rif}\\

\haiku{- troostte Pepe, - ik,,.}{ga m'n geweer halen en}{m'n hengel zoet maar}\\

\haiku{maar wanneer Vladja,.}{zijn handen hard in elkaar}{klapt fluit ze nijdig}\\

\haiku{Laat mij gelukkig,,!}{worden terwijl ik in U}{onderga natuur}\\

\haiku{Op je hersens moet,!}{je dragen dat houdt de booze}{gedachten binnen}\\

\haiku{- Sta niet te kijken!}{als een echte Hottentot}{met een eendenhoofd}\\

\haiku{Ze leken op groote;}{harten zooals je op plaatjes}{van heiligen zag}\\

\haiku{De leege tobbe was.}{licht en Vladja wandelde}{tevreden terug}\\

\haiku{Twee en tachtig, en,.}{zoo rechtop al leunde hij}{zwaar op zijn stokje}\\

\haiku{- 't Is Monsieur, ' -.}{Laforgue die opt}{kantoor werkt zei hij}\\

\haiku{Achter het strooblok.}{hoorden ze de Haaientand}{zich overeindwoelen}\\

\haiku{Hij zag Alice, die.}{het kralengordijn van de}{keuken opzij sloeg}\\

\haiku{Maar haar groote zwarte:}{oogen keken donker en ze}{fluisterde verschrikt}\\

\haiku{Gisteren heeft hij,}{mij de hand gedrukt en ik}{zag dat hij blij was}\\

\haiku{Wanneer je zoo'n mooi,;}{khakihemd bezit dan heb}{je geen tweede noodig}\\

\haiku{het leven is de;}{weerschijn der wolken in een}{vijver geworden}\\

\haiku{- Maar waarom heeft U,?}{de tuinspiegel omgegooid}{Monsieur Hu{\^\i}tre}\\

\haiku{Hoe nederiger,!}{de ondergeschikte des}{te strenger het recht}\\

\haiku{Voor elke kus die;}{ik U niet toestond een dag}{van brandend verdriet}\\

\haiku{een edelhert, zooals er;}{nog nooit een gezien werd in}{de heele vallei}\\

\haiku{hij heeft gezien, hoe.}{ik veertien jaar geleden}{het gewei ophing}\\

\haiku{Er was een scherpe,.}{pijn in zijn keel dat was van}{verdriet om Andr\'e}\\

\haiku{Mis mannetje, er,.}{is er nog altijd een die}{springlevend rondloopt}\\

\haiku{Lieg tegen hem en! '!}{kijk hem dan aant Is of}{de bliksem je treft}\\

\haiku{De nachten zijn koel,.}{en donker maar mijn hart blijft}{wakker en slaat wraak}\\

\haiku{hij was reeds in het.}{stroomgebied dat hij twaalf jaar}{lang ontweken had}\\

\haiku{Hij had de kijker.}{snel vastgedraaid en keek in}{haar donkere oogen}\\

\haiku{Drie dagen later.}{verraste de Saporta hun}{schuldige blikken}\\

\haiku{Voor zoover ik weet, is...}{de afstand tusschen deze}{plaatsen zeer gering}\\

\haiku{maar nu keek ze hem,.}{dringend aan en hij moest de}{oogen wel opheffen}\\

\haiku{Dit was de kleine,;}{gravin niet meer zooals hij haar}{vroeger gekend had}\\

\haiku{Dit was het water,!}{waaraan hij zijn gedachten}{had meegegeven}\\

\haiku{Ben ik het geweest,?}{die haar zooveel pijn deed dat}{zij zoo kijken moet}\\

\haiku{Hij boog zich langzaam,.}{voorover alsof het hem groote}{inspanning kostte}\\

\haiku{Jullie kunnen geen.}{glas ophebben of jullie}{bulken als stieren}\\

\haiku{De acrobaat breidde,.}{zijn armen uit en zette}{een hooge triller aan}\\

\haiku{Vladja lag kauwend.}{op zijn ellebogen en}{keek naar de verte}\\

\haiku{In de nevel werd,,,.}{gevochten en Stephane}{jouw man was de held}\\

\haiku{en hij kon haar niet,.}{afweren want zij greep als}{een verdrinkende}\\

\haiku{Ik heb tegen hem...}{gezegd dat we elkaar reeds}{lang weer ontmoetten}\\

\haiku{Ineens bukte hij '.}{zich en greep de wagenas}{int midden beet}\\

\haiku{Hij vroeg alleen aan.}{Monsieur Hu{\^\i}tre wie die}{Andr\'e geweest was}\\

\haiku{Hij nam vlug een slok,.}{water om de andere}{kant uit te kijken}\\

\haiku{Hij raakte haar zoo,.}{voorzichtig aan alsof ze}{een heilige was}\\

\haiku{Zooiets dragen ze,.}{bij ons thuis aan deze kant}{van de Karpathen}\\

\haiku{In het The\^atre de.}{Dix Heures zag ik op een avond}{Mr. en Miss Joering}\\

\haiku{Waarom moest iemand?}{die twee frank meer verdiende}{zoo'n kijftoon aanslaan}\\

\haiku{Opeens hoorden zij;}{een harde kreet en vlogen}{bevend uit elkaar}\\

\haiku{- Zij hielpen hem naar,.}{het raam waar hij loodzwaar in}{hun armen leunde}\\

\haiku{Andr\'e vloog overeind,.}{en zakte toen langzaam weer}{naar zijn plaats terug}\\

\haiku{- bromde Henri, - hij, '?}{is beter zag je hem dan}{niet aant raam staan}\\

\haiku{Een auto met  ,.}{lichten op stond voor de stoep}{het portier hing open}\\

\haiku{Buiten de parkmuur.}{ging Andr\'e mismoedig aan}{de wegkant zitten}\\

\haiku{- Jammer - fluisterde, -,.}{hij nu weet hij niet meer dat}{er recht gedaan is}\\

\haiku{Het is nu voorbij -,.}{mijn vader stierf in vrede}{dank zij ons bedrog}\\

\haiku{Twintig jaar kalme,?}{liefde is dat meer waard dan}{de roes van een maand}\\

\haiku{Door de leege velden,.}{togen de plukkers weg bij}{twee\"en en drie\"en}\\

\haiku{- En toch zal ik hem -.}{zien uitvaren knerste hij}{tusschen zijn tanden}\\

\haiku{Van dertig meter ',,}{hoog int  water dat}{overleefde je niet}\\

\haiku{Zoo heet het eiland,.}{waar Andr\'e naar toe gaat en}{de wijn heet ook zoo}\\

\haiku{Als hij \'e\'en vinger,.}{uitsteekt dan schiet ik hem een}{kogel in zijn beenen}\\

\subsection{Uit: Ori\"ent-Express}

\haiku{Drie uur lang bleven;}{zij samen opgesloten}{in de woonkamer}\\

\haiku{{\textquoteleft}Klimmen jullie naar.}{beneden met de kaars en}{laat het luik vallen}\\

\haiku{{\textquoteleft}Denk je dat ik de?}{minste hoop heb om nog drie}{maanden te leven}\\

\haiku{De pope schraapte,.}{een paar maal zijn keel dat het}{hol door de kerk klonk}\\

\haiku{Het licht viel op de.}{landkaart van rimpels in zijn}{donkerroode nek}\\

\haiku{Dat zal morgenavond!}{een vlam geven van hier tot}{Constantinopel}\\

\haiku{Dat had Damian,.}{toch maar slecht uitgerekend}{vlak voor de opstand}\\

\haiku{En de opstand bij!}{ons zal net zoo vallen als}{die vervloekte vlam}\\

\haiku{Kroum schreeuwde door, maar.}{Kosta kon in het laaien zijn}{woorden niet verstaan}\\

\haiku{{\textquoteright} {\textquoteleft}Ze heet Todor,{\textquoteright} zei, {\textquoteleft}.}{Milja glimlachenden geef}{nu die schaar maar hier}\\

\haiku{Er is geen golving,.}{in ze ligt precies vlak en}{oneindig rustig}\\

\haiku{Tusschen Prilep en het.}{klooster Svati Archangel}{ligt het dorp Markov Grad}\\

\haiku{De vleugelen van,.}{den aartsengel waren groot}{ruim en genadig}\\

\haiku{de heele tcheta.}{had dekking gezocht in de}{rotsen achter hem}\\

\haiku{morgen zouden het,.}{er negen zijn overmorgen}{een heel bataljon}\\

\haiku{De Turken waren.}{driester geworden en gingen}{tot den aanval over}\\

\haiku{maar voor het eerst van,.}{zijn leven was hij bang bang}{voor de kleeren der dooden}\\

\haiku{met het salvo der.}{anderen verdwenen drie}{andere schimmen}\\

\haiku{{\textquoteright} Zijn stem was kort en.}{hij schoot de woorden telkens}{met een hik eruit}\\

\haiku{Maar niemand kon Kroum,.}{terug houden voor Kroum was}{de vrijheid alles}\\

\haiku{Af en toe keek hij.}{naar de bergen om te zien}{of zij opschoten}\\

\haiku{Er stonden nu veel;}{verlaten boerderijen}{in Macedoni\"e}\\

\haiku{Toch was het land hier;}{langs de zijpaden leeg en}{er klonk geen gerucht}\\

\haiku{Zoo was hun leven.}{geweest het laatste jaar voor}{de revolutie}\\

\haiku{De zon glansde stil,.}{op zijn geel verdroogd hoofd en}{de roode tulband}\\

\haiku{Milja was van den.}{ezel gegleden en suste}{Stana die huilde}\\

\haiku{Want het waren net.}{uitgerekte huizen met}{hun ritsen raampjes}\\

\haiku{Ik heb er geen een,.}{en in die trein gezeten}{heb ik ook nog niet}\\

\haiku{Hij ging rustig door,.}{met priemen en had haar zelfs}{niet aangekeken}\\

\haiku{Mile bromde ook.}{en daarom maakte ze zich}{zoo klein mogelijk}\\

\haiku{Want nu zijn we in.}{Servi\"e en deze rivier}{heet de Morava}\\

\haiku{{\textquoteleft}Vervloekt kind,{\textquoteright} en sloeg.}{toen ineens spijtig met zijn}{vuist tegen zijn mond}\\

\haiku{Hij gaf haar een klap:}{in den rug met den steel van}{zijn zweep en snierde}\\

\haiku{Misschien gaat hij wel,.}{naar een gevangenis waar}{hij nooit meer uitkomt}\\

\haiku{Er waren wel tien.}{man noodig om hem opzij van}{den weg te trekken}\\

\haiku{Met dat al blijft een.}{paal een paal en Ristitch was}{een voorzichtig man}\\

\haiku{{\textquoteright} En tegelijk kijk;}{je om naar de deur of er}{niet iemand luistert}\\

\haiku{Dwars door het schelle,.}{licht dat hard tegen haar oogen}{aansprong liep een man}\\

\haiku{Het was alsof de.}{saaie heuvels ineens wakker}{geworden waren}\\

\haiku{vroeger was de muur,{\textquoteright}.}{hooger hij staat er al over}{de vijfhonderd jaar}\\

\haiku{Ik weet niet of je,.}{dit begrijpt maar ik kan het}{niet anders zeggen}\\

\haiku{Ik verlangde naar.}{m{\`\i}jn jonge vrouw en naar ons}{ongeboren kind}\\

\haiku{ik kan niet anders,.}{omdat ik eens onder mijn}{eigen dooden wegkroop}\\

\haiku{Een witte vlinder.}{fladderde langs haar heen maar}{ze merkte het niet}\\

\haiku{Ze trok haar gele.}{hoofddoek van haar haren en}{wuifde er woest mee}\\

\haiku{Jij denkt natuurlijk;}{aan woeste ritten te paard}{en God weet wat meer}\\

\haiku{Ja, voor het jonge,;}{geslacht is het beter in}{dat land zonder dwang}\\

\haiku{Dit was het nieuwe.}{dorp der kolonisten uit}{Noord-Servi\"e}\\

\haiku{{\textquoteright} zei hij met een zwaar.}{hoofdknikken terwijl hij zijn}{oogen bijna dicht deed}\\

\haiku{{\textquoteright} {\textquoteleft}Maar waarom leef je?}{dan in dit dorp Oom Kosta en}{niet in Radovo}\\

\haiku{Toch was het niet voor,...{\textquoteright}}{niets want ik vond dit dorp en}{ben er tevreden}\\

\haiku{Hij krabde aan zijn,.}{bult nam langzaam een stap en}{kwam breed voor haar staan}\\

\haiku{Het waren niet de,:}{oogen van een dichter maar van}{een man van de daad}\\

\haiku{en naar zijn oogen die,,.}{zooals hij plotseling zag zeer}{verschillend keken}\\

\haiku{Ik stond op die lijst.}{als No. 4 en ik heb die}{plaats niet gestolen}\\

\haiku{Want wanneer je door,:}{een slang gebeten bent dan}{geeft sabbelen niets}\\

\haiku{de vos van Stankovitch';}{en de schimmelponey van}{Stankovitch zoon Christo}\\

\haiku{Dooden spreken niet en.}{zoo was Damianovitch}{ten minste gered}\\

\haiku{Hij kende dat soort.}{gevoelens en hij wist hoe}{lang ze door vraten}\\

\haiku{Het was natuurlijk.}{een kleinigheid dit kind te}{laten verdwijnen}\\

\haiku{Een dubbele kreet.}{kwam uit de twee geschonden}{kindergezichten}\\

\haiku{Ze wist dat de zon;}{nog een uur noodig had tot den}{uitersten raamstijl}\\

\haiku{Daarna werd de zon.}{een laat vuur op den heuvel}{en de nacht begon}\\

\haiku{het rood van wilde,.}{zuring papaverrood en}{zonsondergangrood}\\

\haiku{Stankovitch nam haar mee!}{het bosch in en twee dagen}{later was ze dood}\\

\haiku{Dan gooide je zoo'n,.}{dop weg omdat hij leelijk}{en nutteloos was}\\

\haiku{hoe zijn vader die,.}{hij voor een heilige hield}{door en door slecht is}\\

\haiku{Doch de jongen sprak,}{meteen langzaam door alsof}{hij zijn woorden \'e\'en}\\

\haiku{Hij hing langzaam zijn.}{patroongordel en zijn tasch}{met papieren om}\\

\haiku{Er kwam alleen een.}{kleine rimpel tusschen haar}{strakke wenkbrauwen}\\

\haiku{Todor Alexandrov.}{had trouwens gezegd dat hij}{dit niet wenschte}\\

\haiku{De trein gaat om half,,.}{twaalf want je reist niet over Nisch}{maar over Bucarest}\\

\haiku{Naar dompig vette,.}{schapenvachten naar zuur zweet}{en scherpe uien}\\

\haiku{En geen wonder dat.}{hij eindelijk begreep hoe}{groot haar liefde was}\\

\haiku{Beheerschte hij?}{haar dan niet zoo volkomen}{als hij gedacht had}\\

\haiku{met oogen die te groot.}{en neusvleugels die te hoog}{en te rond waren}\\

\haiku{{\textquoteright} Het laatste woord was.}{een langgerekte kreet die}{wegzwierf in den wind}\\

\haiku{Er zijn honderden.}{handen die de leeuwenvlag}{willen borduren}\\

\haiku{Het was een vreemd en,;}{onaardsch licht niet van den dag}{en niet van den nacht}\\

\haiku{Zij volgde hem, en.}{scheurde haastig een tak vol}{roode bloesems af}\\

\haiku{De steenen hier waren;}{ook heel anders dan in de}{heuvels rond Kounovo}\\

\haiku{maar voor de rest leek:}{Chandanov op een pop in}{een panopticum}\\

\haiku{{\textquoteright} Meteen boog hij zich}{naar Christo toe en keek hem}{strak aan als wilde}\\

\haiku{De rest zullen we.}{in de groote Europeesche}{couranten lezen}\\

\haiku{Want wat je daar aan,!}{je kin hebt kan je heusch}{niet langer dragen}\\

\haiku{Uit de verte zag.}{hij enkel haar rug en haar}{breede bloote voeten}\\

\haiku{Hij keek haar vragend,:}{aan maar ze pakte zijn hand}{beet en zei haastig}\\

\haiku{De populieren.}{rekten zich nog kaal tegen}{het vale avondrood}\\

\haiku{Hij draaide haar om,.}{en duwde haar weg in de}{richting van het dorp}\\

\haiku{De eerste tijd had;}{het zware ding hem bij het}{zitten gehinderd}\\

\haiku{{\textquoteright} Meteen nam hij zijn.}{oogen van Christo weg en keek}{vluchtig de kring rond}\\

\haiku{Uit het kleine park.}{er naast kwam de zware}{geur van gesproeid gras}\\

\haiku{Het licht glansde in;}{de zwartzijden jurk die strak}{rond haar heupen lag}\\

\haiku{Je hebt best tijd om.}{een glas pruimen jenever}{met mij te drinken}\\

\haiku{Voivoda Petrov,?}{wie is de moordenaar van}{Todor Alexandrov}\\

\haiku{Ze stribbelde niet,.}{tegen maar legde haar hoofd}{tegen zijn schouder}\\

\haiku{{\textquoteright} Haar oogen stonden groot.}{en star en hij moest weer aan}{haar moeder denken}\\

\haiku{{\textquoteleft}Ik wil je helpen,,;}{eggen Oom Kosta en ploegen}{daar op je akker}\\

\haiku{maar ik ben bang dat!}{je naar vuur zoekt in de asch}{van verleden jaar}\\

\haiku{en ook zij had een,.}{dierbare doode en hij}{stond achter elk raam}\\

\haiku{Hij trad voor de deur.}{en keek treurig rond over het}{bijna boomlooze land}\\

\haiku{Hij besloot eerst het,;}{graf te graven nu terwijl}{iedereen weg was}\\

\haiku{Weldra moest hij de.}{losgewoelde steenen al hoog}{over de rand werpen}\\

\section{Aernout Drost}

\subsection{Uit: Hermingard van de Eikenterpen. Een oud vaderlands verhaal}

\haiku{Enige beknopte;}{toelichtingen kwamen mij}{noodzakelijk voor}\\

\haiku{Vrouwe wats geschied,,.}{Ik wint of ik bliver dood}{Nu blijft gezond}\\

\haiku{Zij leefden sinds in;}{gedurige worstelstrijd}{met hun overwinnaars}\\

\haiku{Ongeduldig sloeg;}{de Bard de grijsaard en de}{heldenzoon gade}\\

\haiku{thans durfde hij zich.}{in de toekomst weinig met}{Freya's gunst vleien}\\

\haiku{Het vroom gebed en.}{oprechte offer zijn hun}{welbehagelijk}\\

\haiku{Na de nacht doorwaakt,.}{te hebben trokken wij met}{de dageraad voort}\\

\haiku{De Goden zonden,;}{mij deze nacht een droom een}{vreselijke droom}\\

\haiku{Deze droom is een.}{vreselijke bode van}{de toorn der Goden}\\

\haiku{{\textquoteright} riep hij uit, en de.}{vreugde schitterde in het}{levendig bruin oog}\\

\haiku{de Usipeter is;}{niet terug en het lot der}{onzen onbeslist}\\

\haiku{-{\textquoteright} Met kinderlijke;}{trots zwaaide de grijsaard het}{oude krijgstuig}\\

\haiku{gij haatte mij, ik,.}{weet het de wil uwer Goden}{is u een wellust}\\

\haiku{Mijn ziel is los van '.}{de aarde en bovent}{aards verheven}\\

\haiku{ik gevoel grote,;}{zeer grote en verheven}{gewaarwordingen}\\

\haiku{Maar algemene,.}{schrik verspreidde zich toen de}{Bard terugkeerde}\\

\haiku{Men sleurde mij naar,,!}{de hal en wij vonden u}{aldaar gebiedster}\\

\haiku{Telkens heeft zij met;}{dringende belangstelling}{naar u vernomen}\\

\haiku{Huwelijksmin en!}{vaderliefde deden hem}{alles vergeten}\\

\haiku{De beker bleef op;}{de hertogelijke dis}{onaangeroerd staan}\\

\haiku{door de struwelen,.}{baande hij zich een weg want}{hij herkende mij}\\

\haiku{toen mijn krachten mij,,.}{begaven sleurde men mij}{met zacht geweld voort}\\

\haiku{Beklaag, beklaag u,!}{nimmer hem de naamloze}{te moeten noemen}\\

\haiku{Gaarne zou ik hem,.}{troosten maar nimmer wil hij}{mij zijn leed klagen}\\

\haiku{- Zo de reine Geest,{\textquoteleft}}{der Godheid Die zich uitstort}{van heur troon?-}\\

\haiku{Timotheus was op,:}{de knie\"en gezonken geen}{hunner die niet bad}\\

\haiku{de godsdienst, die wij,!}{heden beleden bij elk}{lotgeval heilig}\\

\haiku{Hoe beleidvol de,;}{Bard haar volgde de jonkvrouw}{ontdekte hem toch}\\

\haiku{eens droomde ik van,:}{het kind hetwelk mijn gade}{onder het hart droeg}\\

\haiku{een onvermengde.}{zaligheid zal de aardse}{onspoed vervangen}\\

\haiku{{\textquoteright} aldus ving zij aan, {\textquoteleft};}{mijn gebiedster bevindt zich}{niet in haar woning}\\

\haiku{Daarom was Hermingard;}{in de toren van Witte}{Geertrud gekerkerd}\\

\haiku{Schoon 't duister heers',, ',!}{geen licht ons oog mag treffen}{k Zal steeds min God}\\

\haiku{de Goden laten;}{niet toe dat hun gewijden}{bedrogen worden}\\

\haiku{{\textquoteright} riep zij uit, en wierp.}{zich voor de ontzaglijke}{vrouw op de knie\"en}\\

\haiku{{\textquoteleft}Bedrieg u niet,{\textquoteright} ging, {\textquoteleft};}{vrouw Geertrud voortik wil u}{niet  bedriegen}\\

\haiku{gij zult u in het;}{aanstaand geluk v\'o\'or deszelfs}{bestaan verheugen}\\

\haiku{Hem alleen vrees ik,.}{tot Hem alleen bepaalt zich}{mijn Godsdiensthulde}\\

\haiku{{\textquoteright} Hermingard keerde naar,:}{Geertruds verblijf terug en}{trad aan het venster}\\

\haiku{Nu toog hij naar de.}{schuilplaats der rovers en werd}{door hen gevangen}\\

\haiku{Toen Timotheus zweeg,.}{strekte zijn pleegvader de}{handen naar hem uit}\\

\haiku{Hij hoort mijn gebed......}{die bede moge in de}{oordeelsdag pleiten}\\

\haiku{De grijsaard was als,;}{versteend en luisterde van}{boezemangst vervuld}\\

\haiku{Weldra volgde er.}{een zeer treffend toneel van}{mannelijke rouw}\\

\haiku{als de maan aan de,;}{blauwe hemel bleekt zal uw}{as verzameld zijn}\\

\haiku{{\textquoteright} wendde Welf zich tot, {\textquoteleft}?}{haarweet gij de schuilplaats van}{de zoon der ondeugd}\\

\haiku{{\textquoteright} {\textquoteleft}Christus zal met haar,{\textquoteright}.}{zijn hernam Marcella met}{betraande blikken}\\

\haiku{nog stormt het daar wild,;}{en woedend wanneer ik aan}{het verleden denk}\\

\haiku{{\textquoteright} Zo sprekende, had;}{hij zich op de zodenbank}{nedergeworpen}\\

\haiku{zijn ouderdom bleef,?}{gespaard waarom zou die mijns}{vaders geknot zijn}\\

\haiku{er       Landbouw der*.}{Batavieren   ~         bleef}{hem geen twijfel over}\\

\haiku{aan de andere.}{zijde van het sterfbed stond}{de wichelares}\\

\haiku{Eindeloze troost.}{schenkt mij deze genade}{in het laatste uur}\\

\haiku{{\textquoteleft}Waarom zouden wij,,,!}{wenen als zij die geen hoop}{hebben Timotheus}\\

\haiku{Marcella had met;}{hem de weleer bewoonde}{woning betrokken}\\

\haiku{hoe gelukkig maakt,!}{mij uw leven uw liefde}{en uw onderwijs}\\

\haiku{De grote God schonk.}{mij zijn genade door het}{dierbaarste werktuig}\\

\haiku{mij een waardiger!}{tolk der hemelse leer en}{verlichtte mijn geest}\\

\haiku{Ik werd een Christen,,!}{geloofd zij het Lam dat voor}{onze zonden stierf}\\

\haiku{Wij vierden aldaar '.}{het twintigste jaar vans}{keizers regering}\\

\haiku{{\textquoteright} Met verrukking en.}{deelneming hadden allen}{het verhaal gehoord}\\

\haiku{sedert ongeveer.}{tien jaar was zijn naam hier te}{lande een begrip}\\

\haiku{Toen Drost Hermingard schreef,.}{kon hij dus putten uit een}{rijke traditie}\\

\haiku{de kardinale.}{christelijke deugden zijn}{reeds in hun bezit}\\

\haiku{namelijk toen zij (-).}{de apostel Paulus hoorde}{sprekenHand. 16:1415}\\

\haiku{uit lied cxvi van}{Ecclesiasticus oft}{de wyse sproken}\\

\haiku{ongetwijfeld wist (,,).}{E. Gibbon Decline and}{Fall ii Ch. 20}\\

\haiku{uit het gedicht {\textquoteleft}Aan{\textquoteright}.}{de ontluikende jeugd door}{Willem Bilderdijk}\\

\haiku{Camillus heette.}{de bruilofttochtgeleider}{bij de Romeinen}\\

\section{E. du Perron}

\subsection{Uit: Het land van herkomst}

\haiku{hoe ze ons kunnen?}{beletten om bijvoorbeeld}{de maan te voelen}\\

\haiku{Geloof H\'everl\'e;}{niet als hij zegt dat je veel}{van een Fransman hebt}\\

\haiku{Zeg maar eens eerlijk.}{wat voor jou re\"eel is in}{deze omgeving}\\

\haiku{nog anders dan je!}{gedacht had het persoonlijk}{te zullen maken}\\

\haiku{Als hij mijn vriend is,}{waarom ben ik dan niet met}{hem op mijn gemak}\\

\haiku{Hij stelt mij voor de:}{bar van Poccardi in}{te gaan en bestelt}\\

\haiku{Misschien heeft hij zelf,;}{te veel vadergevoelens}{oppert Goera\"eff}\\

\haiku{ik ben werkelijk,.}{zuiver een senorito}{een bourgeoiszoontje}\\

\haiku{De verandering,}{smolt langzaam weg ik lag naar}{mijzelf te kijken}\\

\haiku{de ziekte van mijn,.}{moeder zich spoedig genoeg}{moest laten gelden}\\

\haiku{{\textquoteleft}In die tijd, als je,{\textquoteright}.}{acht-en-twintig was was je}{meteen veertig}\\

\haiku{er was bijna geen,.}{weerstand meer in het kleine}{verzwakte lichaam}\\

\haiku{Hij is helemaal,!}{van streek want hij heeft nooit aan}{dit alles gedacht}\\

\haiku{heel broos en fijn, met.}{een prachtig teint en destijds}{gitzwart haar en ogen}\\

\haiku{altijd vertelde,,}{hij mij van mijn vader met}{een soort oude vrees}\\

\haiku{(Als mijnheer Ducroo,.)}{zijn knevels maar opdraaide}{beefden wij allen}\\

\haiku{omdat men degeen.}{is die de bekentenis}{heeft aangetrokken}\\

\haiku{De tweede was een {\textquoteleft}{\textquoteright};}{blanke nona en heette}{Jeanne Ende}\\

\haiku{Op een dag werd ik:}{daar door een dikke Duitser}{gefotografeerd}\\

\haiku{Ik ben vergeten;}{dat het licht werd uitgedraaid}{toen de film begon}\\

\haiku{Zij was een lange.}{vrouw met bijna spierwit haar}{maar een glad gezicht}\\

\haiku{{\textquoteright} Yoeng had vooral last;}{van geesten wanneer hij de}{ramen moest sluiten}\\

\haiku{De vensters ervan;}{waren toen smal en zwart en}{van tralies voorzien}\\

\haiku{De laatste ronde.}{werd tussen Tjang Panel en}{Lies uitgevochten}\\

\haiku{Bella is op haar:}{lach doorgegleden in een}{heel ander verhaal}\\

\haiku{Ongeveer in het.}{midden van de drijvende}{bal\'e was het dorp}\\

\haiku{het behoorde toe (),;}{aan de loerahdorpshoofd die}{Pa Djoewi heette}\\

\haiku{Toen hij stierf, was er.}{niemand bij hem en hijzelf}{was geheel vervuild}\\

\haiku{Mijn vader trok zich,;}{toen al bij voorkeur in bed}{terug of hij las}\\

\haiku{Ik kende deze.}{listen van weggaan trouwens}{al van heel jong af}\\

\haiku{\`en omdat zij niet,,}{alleen kon zijn zouden een}{boek kunnen vullen}\\

\haiku{Mijnheer was uit, zei,;}{het dienstmeisje maar zou wel}{dadelijk komen}\\

\haiku{Maar de liefde van.}{de personages gaat ons}{dan haast niet meer aan}\\

\haiku{Als de man van wie,.}{zij houdt evenzeer bedreigd wordt}{gaat ook dit niet op}\\

\haiku{Haar moeder trouwens,;}{was een Javaanse en kwam}{haar soms opzoeken}\\

\haiku{En wat krijg ik van?}{je als oom Edwin en ik}{later gaan trouwen}\\

\haiku{Door de tralies kon,;}{ik haar toch soms zien bidden}{languit op de vloer}\\

\haiku{ik proberen mij,.}{met hen te amuseren wat}{zelden gelukte}\\

\haiku{De vrouwen waren,:}{opgetogen over hem en}{mijn moeder zei mij}\\

\haiku{{\textquoteleft}Nah, en je zult niet,,,...}{gelo-oven maar oom ja-a}{droomde van een pauw}\\

\haiku{{\textquoteright} Mijn moeder was ook;}{in Brussel en Grouhy door}{hem teleurgesteld}\\

\haiku{{\textquoteleft}kijk, ik verdedig,,.}{mij niet eens ik heb er geen}{lust zelfs geen tijd voor}\\

\haiku{Hij liep mij hardop;}{pratend achterna om mij}{te laten lachen}\\

\haiku{{\textquoteright} Toen draaide ik hem.}{mijn rug toe en ging naar het}{karretje terug}\\

\haiku{verder had men hem;}{rijk getooid op een paardje}{laten rondrijden}\\

\haiku{op een dag stelde.}{ik hem voor deze kunst op}{mij te beproeven}\\

\haiku{omdat ik kans had.}{er de chef-corrector}{nog aan te treffen}\\

\haiku{Als mijn ouders dan,.}{in de fabriek waren kwam}{Pieng mij bekrijsen}\\

\haiku{{\textquoteleft}Wacht, Isnan, tot ik!}{groot ben en ik zal je net}{zo behandelen}\\

\haiku{Soms kwam hij toch met,:}{gaten in zijn vel thuis eens}{midden in zijn kop}\\

\haiku{braaf beest, hij heeft er{\textquoteright}.}{zelfs niet over gedacht om een}{van ons te bijten}\\

\haiku{Het zag er vreemd uit.}{en was toch nog een beetje}{tijger gebleven}\\

\haiku{Ik herinner mij.}{van deze laatste tijd niet}{veel meer dan een sfeer}\\

\haiku{hij schrijft terug dat.}{hij uitstel heeft aangevraagd}{en verkregen}\\

\haiku{ik wist de toon niet.}{te treffen die succes in}{grotere kring had}\\

\haiku{Het kwam niet in mij;}{op verlegen te zijn voor}{deze getuige}\\

\haiku{Met mijn boekentas;}{als gewoonlijk over de vrije}{arm sprong ik eruit}\\

\haiku{zij werd in zijn plaats,.}{door de agenten gegrepen}{maar zij hield van hem}\\

\haiku{hij zei dat hij er:}{niet over dacht om kostgeld voor}{hem te betalen}\\

\haiku{toch is zijrond in,.}{haar vlees geworden zooals gij}{haar niet gekend hebt}\\

\haiku{ik verbeeldde mij,:}{niet in de plaats van Darma}{te zijn maar ik dacht}\\

\haiku{Meneer Ducroo, heb,.}{ik gemerkt leert veel buiten}{de H.B.S. en hier niets}\\

\haiku{Later ontmoette}{ik hem bij de kapper en}{viel het mij weer op}\\

\haiku{Toen ik mijn moeder,.}{getroost had fietste ik naar}{de gevangenis}\\

\haiku{De cipier was een;}{Europeaan die eruit}{zag als een Alfoer}\\

\haiku{Het hinderde mij;}{toch dat niemand van ons bij}{Leni verder kwam}\\

\haiku{als zij die niet zelf.}{wilde uitkiezen moesten wij}{er iets op vinden}\\

\haiku{Junius, die nooit,;}{bij ons kwam zitten zag het}{uit de verte aan}\\

\haiku{zodra hij mij zag}{reed hij met het ringetje}{zwaaiend naar mij toe}\\

\haiku{Er was maar \'e\'en ding,,:}{merkte Junius op dat}{wel raar in hem was}\\

\haiku{Op een Zondagavond;}{in de soos kwam ik weer in}{de buurt van Hetty}\\

\haiku{Maar het ophalen.}{van de verloren beurten}{lukte mij niet meer}\\

\haiku{grijsachtig wit en:}{koel in zijn grote tuin vol}{romantisch lommer}\\

\haiku{Toen ik volwassen;}{was zocht ik in gesprekken}{soms toenadering}\\

\haiku{{\textquoteleft}want alleen door oom{\textquoteright}.}{Van Kuyck hebben we ons}{fortuin behouden}\\

\haiku{{\textquoteright} Welnu, Mad, Ida is,,:}{hier zij heeft mij leeren kennen}{ze zeide alleen}\\

\haiku{hier zag ik Trude.}{die zeker een halfuur om}{mij had gelachen}\\

\haiku{{\textquoteright} De jonge man, met:}{glinsterende ogen en een}{glimlach vol snaaksheid}\\

\haiku{Het was even donker;}{en omstreeks half 7 toen wij}{er terugkwamen}\\

\haiku{De derde maal was.}{ik 17 en het gebeurde}{in Tjitjalengka}\\

\haiku{Zij had mij gezegd}{dat zij zeker dagen zou}{laten voorbijgaan}\\

\haiku{ik was bij oom Van,}{Kuyck gelogeerd om 11}{uur nam ik op straat}\\

\haiku{ik kon naar Arthur, die.}{in de buurt woonde en in}{een tuinkamer sliep}\\

\haiku{die heeft je, ik weet...{\textquoteright}.}{niet die heeft je gemaakt tot}{een mensenhater}\\

\haiku{{\textquoteleft}Zie je, Ducroo, ik,!}{ben een goed mens maar d\`at kan}{ik nou niet hebben}\\

\haiku{ik bedacht dat ik;}{er kon komen met het geld}{dat ik nog over had}\\

\haiku{Wij hadden zolang;}{gewacht dat wij de kleine}{buit voor lief namen}\\

\haiku{{\textquoteleft}Als je niet oppast,,{\textquoteright},.}{An ga jij er eens ook zo}{uitzien zei Taco}\\

\haiku{{\textquoteright} heb ik mij dikwijls;}{afgevraagd als ik met hem}{samen was geweest}\\

\haiku{Voor een heel ras is.}{een groot man trouwens altijd}{de grote acteur}\\

\haiku{Zoals je op de{\textquoteright}.}{dansles ook van een meisje}{kunt leren dansen}\\

\haiku{hij kan er niet meer{\textquoteright}.}{aan doen dan ik en heeft er}{al last genoeg van}\\

\haiku{Hij was dol op zijn,;}{enige zoon die als kind een}{tenger knaapje was}\\

\haiku{Hij werd rood in zijn,:}{gezicht en ik ook maar hier}{lachte ik hem uit}\\

\haiku{Hij liep weg, zeggend,.}{dat het best was en die dag}{zag ik hem niet meer}\\

\haiku{We hebben daarbij,.}{17 man verloren het was}{een vrij grote troep}\\

\haiku{De rest is alleen}{thuisgekomen omdat een}{inlands sergeant}\\

\haiku{Van de aanvallers,,,.}{er waren er 11 geweest}{10 dood \'e\'en gesmeerd}\\

\haiku{je hebt niet de tijd.}{aan iets te denken als je}{met hem bezig bent}\\

\haiku{{\textquoteright} {\textquoteleft}Na Elly's dood... en,.}{ikzelf achteraf heeft het}{me toch aangepakt}\\

\haiku{hij heeft het gemerkt:}{toen hij dagelijks reisde}{per vliegmachine}\\

\haiku{hij is nu al 3.}{maanden hier en dit is dus}{zijn eerste rapport}\\

\haiku{In mijn schoongeboend:}{hotelletje een brief van}{Jane uit Bretagne}\\

\haiku{Ik vond zelf de grap.}{een beetje ver gedreven}{en ging naar Grouhy}\\

\haiku{en met 40 colli,.}{bagage die telkens moesten}{worden nageteld}\\

\haiku{en bleek spoedig mijn.}{moeder niet zo goed meer te}{kunnen masseren}\\

\haiku{de zogeheten,,.}{meesters de bedienden de}{log\'e's en ikzelf}\\

\haiku{zijn opmerkingen;}{over toestanden klonken zelfs}{niet meer als vroeger}\\

\haiku{In de eerste plaats;}{was er natuurlijk haar werk}{van iedere nacht}\\

\haiku{En bijwijze van...{\textquoteright}}{navelstreng een touwtje als}{bij kinderballons}\\

\haiku{Het is of ik een,.}{satire schrijf en toch is}{het niet geheel waar}\\

\haiku{Nu is het zelfs geen,;}{voorstad meer maar een wijk op}{de grens van Parijs}\\

\haiku{Ik zou haar een ring,?}{willen geven maar zou ze}{die niet inslikken}\\

\haiku{Je stelt een wet in,,.}{zei H\'everl\'e nog daarmee}{sta je altijd sterk}\\

\haiku{Toen ik zowat elf,;}{was vond ik op een dag mijn}{moeder huilende}\\

\haiku{Mijn angst dat ik d\`at!}{in die prachtige salon}{kon achterlaten}\\

\haiku{{\textquoteleft}Mm... en toen ik in,.}{Parijs was kreeg ik ook hier}{opeens genoeg van}\\

\haiku{je zult merken dat,{\textquotedblright}.}{ik gelijk heb maar ik laat}{je de plaats nu vrij}\\

\haiku{Ik ging met Sjoera:}{samenwonen en wilde}{Tanja niet meer zien}\\

\haiku{Toen zij wakker werd.}{had men haar tasje met het}{geld weggenomen}\\

\haiku{hij had haar in het;}{Schwarzwald ontmoet en gepoogd}{haar te verleiden}\\

\haiku{Ik bracht het gesprek:}{op Tanja en hij ging er}{dadelijk op in}\\

\haiku{in deze tijd zou.}{het tegendeel immers zo}{normaal zijn geweest}\\

\haiku{Ik kan er nu om}{lachen als ik bedenk met}{welke gevoelens}\\

\haiku{op het kwaadspreken.}{dat al begon zodra zij}{het hek uit waren}\\

\haiku{Wat er aan strijd was;}{geweest tussen de mama}{en haar ontging mij}\\

\haiku{{\textquoteleft}Ach, je weet er niets,,{\textquoteright}.}{van zei ze dat komt omdat}{het nog zo klein is}\\

\haiku{maar soms denk ik weer.}{dat ook deze formule}{te eenzijdig is}\\

\haiku{hoe kan men z\'oveel?}{verdriet hebben om iets dat}{z\'o vaag worden kan}\\

\haiku{hij wijdde er lang}{over uit en terwijl ik naar}{hem luisterde was}\\

\haiku{{\textquoteleft}Kom dichter bij me,.}{en zeg gerust als je het}{idee niet prettig vindt}\\

\haiku{tegen het licht van.}{de gang zag ik dat zij in}{een avondmantel was}\\

\haiku{maar, zei ze, ik moest.}{daarom niet denken dat zij}{niets had meegemaakt}\\

\haiku{Ik was niet langer,:}{dan vijf minuten weg en}{ik dacht onderwijl}\\

\haiku{In de Deux Magots,,;}{waar gevochten is spreekt men}{er nauwelijks over}\\

\haiku{Er zijn meer dan 5000,.}{manifestanten ditmaal}{en zij groeien aan}\\

\haiku{All\'e\'en een goede...{\textquoteright}.}{economische ordening}{Bij de H\'everl\'e's}\\

\haiku{Ik kwel mij met de -}{gedachte dat mensen als}{wij ik bedoel nu}\\

\haiku{Om dit leven uit,.}{te schakelen zou men niet}{meer moeten denken}\\

\haiku{je zou Deterding,:}{willen opblazen je kan}{het niet en je denkt}\\

\haiku{iets heel merkwaardigs,.}{werkelijk in het soort van}{je notarissen}\\

\haiku{{\textquoteright} {\textquoteleft}In de eerste plaats,.}{als niet-vee daarna als}{alles wat je wilt}\\

\haiku{we zoeken ze thuis,,;}{op volgens het adresboek en}{we vermoorden ze}\\

\haiku{binnen een halfuur.}{had hij drie verschillende}{reacties gewekt}\\

\haiku{en het was vroeger,...{\textquoteright}}{z\'o'n knappe jongen dat kan}{men ook nog wel zien}\\

\haiku{Ik sta tegenover.}{zulke woorden met een leeg}{hart en een leeg hoofd}\\

\subsection{Uit: Manuscript in een jaszak gevonden. Kroniek van de bekering van Bodor Gu{\'\i}la Buitenlander}

\haiku{Je suis all\'e \`a;}{Montmartre pour y chercher}{un ext\'erieur}\\

\haiku{Car malgr\'e tous mes.}{efforts je me suis compris}{en l'\'ecrivant}\\

\haiku{Want hoe ik ook mijn,.}{best deed ik begreep mezelf}{toen ik het schreef}\\

\haiku{Willekeurig in.}{stukken gehakt en onder}{elkaar geschreven}\\

\haiku{Alors tous deux furent.}{apais\'es et l'un ne faisait}{plus rien que baver}\\

\haiku{Mariette,~~~~~n'est,.}{pas ici elle d\'ejeune}{l'avocat r\^eveur}\\

\haiku{je vous remercie}{pour vos le\c{c}ons cher ma{\^\i}tre}{mais vous ne m'avez}\\

\haiku{reconnaissant sans}{cendrars sans moi tandis que}{j'\'ecris son portrait}\\

\haiku{de hemel is de}{rand van een horloge dat}{op kwart over elf staat}\\

\haiku{est-ce tout?}{simplement la langue de}{Poulo-Djawa}\\

\haiku{Ik heb een jeboek.}{gevenschre dat le zenprij}{noch le kenla is}\\

\haiku{de te logische!}{opstandeling tegen mijn}{jeugdige overmoed}\\

\subsection{Uit: Tegenonderzoek}

\haiku{Bloem's requisitoir,,.}{kent geen genade zegt Ter}{Braak en het is juist}\\

\haiku{Wilt u overigens?}{Clair-Obscur en Saturnus}{eens met mij doorzien}\\

\haiku{De Bezoeker heeft,,;}{Marsman zelf geloof ik als}{een aanwinst erkend}\\

\haiku{Ik verbeeld mij ook}{geenszins dat mijn streven om}{precies te zeggen}\\

\haiku{Z\'o afgepast en:}{verantwoord hoeft ieder woord}{hier toch niet te zijn}\\

\haiku{Geantwoord dat ik:}{deze psychologiese}{trouvaille zie als}\\

\haiku{Van Ostaijen wordt -:}{er bij gehaald en Ter Braak}{zei het immers reeds}\\

\haiku{Dit dekking zoeken;}{achter Van Ostaijen vind}{ik ook al zo lam}\\

\haiku{Men zou daar telkens,,.}{op terug komen en toch}{daar gaat het niet om}\\

\haiku{Sedert enige tijd,:}{kondigt hij aan onder de}{lijst van zijn werken}\\

\haiku{De soberheid van.}{toon is in het eerste deel}{bijna misleidend}\\

\chapter[16 auteurs, 1682 haiku's]{zestien auteurs, zestienhonderdtweeëntachtig haiku's}

\section{Pieter Ecrevisse}

\subsection{Uit: Vier verhalen uit het land van Zwentibold}

\haiku{tilde hem op uit,;}{den kring welke zich om den}{kerel gevormd had}\\

\haiku{Ik weet zeker, dat.}{gij vader en moeder zoudt}{gelukkig maken}\\

\haiku{maar dat hij ook eene.}{verantwoordelijkheid op}{zijne schouders droeg}\\

\haiku{want zwijgen, waar men,.}{spreken moet is een bewijs}{van lafhartigheid}\\

\haiku{dit jaar zie ik van,.}{mijn voorrecht af en zal maar}{de laatste komen}\\

\haiku{de voldoening moet,!}{openbaar zijn dat zweer ik u}{hier op mijne eer}\\

\haiku{Indien gij eventwel,,}{wist hoe rechtzinnig ik't}{met u meene gij}\\

\haiku{Hoe lang schenen mij,:}{de zes dagen die ons van}{dit bezoek scheidden}\\

\haiku{en nog hadden de.}{paarden te traagzaam gestapt}{voor ons ongeduld}\\

\haiku{Zoolang wij hier in,';}{bezetting liggen reken}{ik mij nog t huis}\\

\haiku{te meer, daar ik aan,;}{niemand mocht bekend maken}{wat in mij omging}\\

\haiku{Ik haalde u hier;}{niet alleen de colonne}{mobile in huis}\\

\haiku{Op deze wijze,.}{heb ik meer dan drie honderd}{mijlen afgelegd}\\

\haiku{Eenige minuten.}{verliepen onder deze}{heilige stilte}\\

\haiku{Langzamerhand heb;}{ik hem allerlei gerief}{medegenomen}\\

\haiku{Ook vermeed ik, uit,;}{dien hoofde alle verkeer}{met gezelschappen}\\

\haiku{mij zoo lastig, dat;}{ik beslote een einde}{daaraan te stellen}\\

\haiku{Let wel op, dat er,,.}{elke tien tot twaalf stappen}{zulk kruis geteekend staat}\\

\haiku{Dat ik in hunne,...}{handen viele daaraan is}{weinig gelegen}\\

\haiku{en Barbara sloot;}{den broeder met meer geestdrift}{in hare armen}\\

\haiku{Er ging nog een uur..,,!}{voorbij een uur hetgeen haar}{een eeuwigheid scheen}\\

\haiku{Op de vraag van het,:}{meisje begint Lemmens het}{volgende verhaal}\\

\haiku{Wat mij persoonlijk,.}{betrof vreesde ik deze}{agenten zeer weinig}\\

\haiku{De overwinningen.}{begonnen hoe langer zoo}{schaarscher te worden}\\

\haiku{{\textquoteright} De commandant sloeg:}{een register open en las}{wat er geboekt stond}\\

\haiku{want het eene schijnt mij.}{zoo pijnlijk om bekennen}{als het andere}\\

\haiku{doch ik zal mij niet!}{medeplichtig maken aan}{eene barbaarsche daad}\\

\haiku{Gij, franschmans, hebt;}{geen medelijden gehad}{met ons Vlamingen}\\

\haiku{{\textquoteright} Op deze wijze.}{maakte hij de aandacht der}{gendarmen gaande}\\

\haiku{Het is een eerlooze,!}{duitscher die tegen zijne}{stamgenoten dient}\\

\haiku{Zoo niet, dit zweer ik,!}{dat gij zonder ooren en}{neus van hier zult gaan}\\

\haiku{doch de smaak kwam hem,.}{zoo walgend voor dat hij het}{vocht moest uitspuwen}\\

\haiku{men kan niet weten, -, -;}{zegde hij of men den wolf}{niet in den muil loopt}\\

\haiku{Weinige woorden;}{werden gewisseld tusschen}{deze vier menschen}\\

\haiku{{\textquoteleft}- Mijn plichtbesef en;}{mijne dankbaarheid jegens}{de Voorzienigheid}\\

\haiku{stamelen, terwijl.}{hij den rug der rechter hand}{op zijn voorhoofd bracht}\\

\haiku{{\textquoteright} {\textquoteleft}- Neen, mijn kind, ik ben,:}{er verre af u deze}{hulp te verwijten}\\

\haiku{Deze dankbaarheid.}{ga ik oogenblikkelijk}{op de proef stellen}\\

\haiku{{\textquoteleft}- Uit hoofde van een,.}{kwalijk geplaatst eergevoel}{heeft hij gezwegen}\\

\haiku{bemint gij mijnen,?}{zoon en wilt gij hem uwe hand}{en uw hart schenken}\\

\haiku{Men bracht er schier den.}{geheelen nacht over in de}{hevigste spanning}\\

\haiku{verliet het huis met:}{even zooveel spoed als hij was}{binnengekomen}\\

\haiku{weldra stroomden de.}{overvloedigste tranen langs}{zijn bleek aangezicht}\\

\haiku{Zij vond het meisje ',:}{alleent huis en begon}{het gesprek aldus}\\

\haiku{Indien Isabella,!}{een arm meisje ware hij}{zou haar niet willen}\\

\haiku{Deze twee menschen.}{waren onafscheidbare}{vrienden geworden}\\

\haiku{De golven sloten...}{zich eenen oogenblik boven}{het hoofd des ruiters}\\

\haiku{Nooit zou het ruchtbaar,.}{geworden zijn zonder een}{zonderling toeval}\\

\haiku{Wij hadden deze:}{bewegingen zonder eenig}{overleg toegezien}\\

\haiku{Ik zelf begrijp niet,,,,;}{wat de heer graaf mijn vader}{wil of beraamd heeft}\\

\haiku{Wat belette hem?}{ten minste eenig teeken van}{bestaan te geven}\\

\haiku{De heer graaf en zijn;}{zoon moesten redenen hebben}{om bedroefd te zijn}\\

\haiku{omtrent al deze;}{omstandigheden was Jan's}{verhaal haar ontsnapt}\\

\haiku{De priester en ik...}{hadden den tijd niet om een}{antwoord te geven}\\

\haiku{Hier op mijn hart zult;}{gij den sleutel vinden van}{de deur zijns verblijfs}\\

\haiku{staarde verwilderd,.}{rond en toen hij ons beide}{zag scheen hij verschrikt}\\

\haiku{want de kristlijke.}{gelatenheid keerde op}{zijn gelaat terug}\\

\haiku{Het duurde eventwel,.}{niet lang of ik begon de}{oorzaak te gissen}\\

\haiku{O hemel, dacht ik,;}{thans gaat het woedende dier}{op den jager los}\\

\haiku{want zij zegde met,:}{bezorgdheid terwijl hare}{stem lichtlijk trilde}\\

\haiku{Ik beloofde en,;}{hoopte mijnen vader te}{zullen bewegen}\\

\haiku{Den dag voor Kersmis,,.}{des jaars 1683 had ik goeden}{buit ter jacht gemaakt}\\

\haiku{Met eene hijgende.}{borst sloop ik tot aan de deur}{van vaders kamer}\\

\haiku{ik beet op mijne.}{lippen en zocht naar eenen tekst}{om te beginnen}\\

\haiku{Ik stond op het punt,:}{de kamer te verlaten}{toen hij mij toeriep}\\

\haiku{, Walter, dat gij niet.}{ongenegen zijt voor den}{huwelijken staat}\\

\haiku{Zeg, bid ik u, dat!}{gij mij enklijk hebt willen}{op de proef stellen}\\

\haiku{Want hij behoort noch,.}{aan mij noch aan zich zelven}{maar aan zijn stamhuis}\\

\haiku{Hij straft dengenen,,;}{die zijne ouders niet eert}{reeds in dit leven}\\

\haiku{Daarna keerde ik.}{met haastige schreden naar}{het kasteel terug}\\

\haiku{Van honger wil ik,;}{niet omkomen dat ware}{eene soort van zelfsmoord}\\

\haiku{in Julia's oogen,;}{wilde ik geenszins voor een}{meineedige doorgaan}\\

\haiku{Tot hiertoe had de;}{zieke nog altoos met eenig}{gemak gesproken}\\

\haiku{Adriaan, Jan en ik.}{besproeiden het plekje grond}{met onze tranen}\\

\haiku{uwe hutten zullen!}{twee luiken ter straat en twee}{ter zijde hebben}\\

\haiku{Limbrichts grond, water,,,.}{straten bewooners in een woord}{alles behoort mij}\\

\haiku{want zij geleken.}{zich als twee droppels water}{uit het zelfde vat}\\

\haiku{want hij is bereid:}{om ook zijnen besten vriend}{te verloochenen}\\

\haiku{Maar maken wij de:}{lezers nader bekend met}{de beide zieken}\\

\haiku{De lieutenant;}{besloeg de eerste plaats op}{het ziekeleger}\\

\haiku{nochtans scheen 't hem;}{toe dat hij nog weinig of}{niets gewonnen had}\\

\haiku{{\textquoteleft}Wij hadden sedert,;}{30 jaren een tuinman met}{name Pelart}\\

\haiku{Waart gij niet vroeger,?}{Pierre P\'edart de zoon van}{mijnen hovenier}\\

\haiku{- Druk de woorden wel;}{in uw geheugen welke}{ik u ga zeggen}\\

\haiku{door den ouden adel!}{aanzien als een mengsel van}{alle boosheden}\\

\haiku{Ein herz von edelmuth,}{bewohnt Ist durch sich selbst}{am herrlichsten}\\

\haiku{Misschien viel de stok '!}{opt hoofd des bisschops van}{Trier uit gewoonte}\\

\haiku{De onschuldigste;}{daden van Johan werden ten}{ergsten uitgelegd}\\

\haiku{Aanstonds deed hij de;}{overige hovelingen}{voor zich verschijnen}\\

\haiku{geheel zijne macht:}{en wil schenen in de oogen}{alleen te heerschen}\\

\section{Frederik van Eeden}

\subsection{Uit: Gedachten}

\haiku{Parodieeren en.}{lachen zijn gevaarlijke}{wijzen van kritiek}\\

\haiku{Een organisme,.}{stort ineen met schokken maar}{groeit niet met schokken}\\

\haiku{Tot ons geluk zijn.}{onze instincten sterker}{dan onze rede}\\

\haiku{Wat voor een klein mensch,.}{ondeugd is kan zeker nooit}{deugd zijn voor een groot}\\

\subsection{Uit: De kleine Johannes. Deel 2}

\haiku{Zeker hebt ge niet, '?}{gedacht dat ik mijn woord zou}{houden ist wel}\\

\haiku{Gaat het niet, dan spijt,.}{me dat voor u maar ge moet}{er niet om liegen}\\

\haiku{- {\textquoteleft}wat wordt ge nat, bind,.}{mijn jasje om uw hoofd ik}{kan het wel missen}\\

\haiku{{\textquoteright} - - {\textquoteleft}En heel spoedig zult,.}{ge mij weer niet zien en toch}{ben ik er even goed}\\

\haiku{En zijn verheven,.}{metgezel was er ook een}{een gewone man}\\

\haiku{Hij dacht aan zijn dooden,.}{vader en dat hij nu naar}{een kermis-spel ging}\\

\haiku{Hij heeft veel verdriet,,.}{gehad Marjon zijn vader}{is pas gestorven}\\

\haiku{Hij heeft z'n verstand,,}{zoo goed als jelui met z'n}{vieren bij mekaar}\\

\haiku{In vlechten was nu '.}{t lichtblonde haar rondom}{haar hoofd gebonden}\\

\haiku{Het mooie en blijde.}{alleen is goed en dat wat}{wij moeten zoeken}\\

\haiku{Ga nu maar gauw, of.}{de blindheid van je kind komt}{op je geweten}\\

\haiku{Er werden woorden.}{in hem geboren die hij}{zorgvuldig vasthield}\\

\haiku{Want de wijze heeft.}{zijn ouders lief en wil hun}{kwaad wel goed maken}\\

\haiku{Vader en moeder.}{leven nog en gedijen}{door onze moeite}\\

\haiku{{\textquoteleft}Denk niet Johannes,.}{dat ik je telkens zeggeu}{zal wat je doen moet}\\

\haiku{Jamaar, tante, als,?}{ik vroom ben kom ik ook in}{den hemel niet waar}\\

\haiku{Rondom was het zeer,}{stil de blauwe en witte}{sterbloemen tusschen}\\

\haiku{hij moest leven, dat,,.}{hadden ze allen man en}{vrouw grijsaard en kind}\\

\haiku{Daar was schemerschaduw,.}{en een geheimzinnige}{plechtige stilte}\\

\haiku{- En dat na al het.}{goede wat ik dacht voor je}{gedaan te hebben}\\

\haiku{{\textquoteleft}Daatje wil je maar naar, '.}{de keuken gaan ik zalt}{verder wel afdoen}\\

\haiku{{\textquoteright} De gemeente keek.}{onthutst van den spreker naar}{dominee Kraalboom}\\

\haiku{{\textquoteleft}Ik verzoek u, de.}{orde in dit kerkgebouw}{niet te verstoren}\\

\haiku{Wie onzer zou de?}{genade niet begeeren en}{de zaligheid niet}\\

\haiku{Het leven van een,.}{oud mensch is zoo dor als er}{niets jongs bij opgroeit}\\

\haiku{Eindelijk bracht de.}{vlooien-temmer Johannes}{bij Marjon's wagen}\\

\haiku{En wijzer worden.}{was iets waartoe hij de kans}{niet graag verzuimde}\\

\haiku{{\textquoteright} zei Johannes en.}{klemde zijn handen samen}{in groote ontroering}\\

\haiku{Jij mot maar voor de,.}{woorden zorgen dan zorg ik}{wel voor de muziek}\\

\haiku{je zult wel motte,,.}{of je wilt of niet om niet}{te verhongeren}\\

\haiku{Het stonk er zoo, naar.}{uien en gebakken vet}{en naar veel erger}\\

\haiku{{\textquoteright} De meisjes klapten ' {\textquoteleft}!}{in de handjes en riepen}{omt hardstbravo}\\

\haiku{{\textendash} {\textquoteleft}Wij kunnen wel met,.}{zekerheid zeggen dat dit}{niet toevallig is}\\

\haiku{Misschien komt hij wel.}{tot den eersten met studie}{en goede cultuur}\\

\haiku{{\textquoteleft}Ik ben nog niet goed,.}{maar ik wil graag mijn best doen}{om het te worden}\\

\haiku{Nog eer hij te huis,.}{kwam was de taak hem reeds te}{machtig geworden}\\

\haiku{Wistik vooraan, want.}{die was gewend in duister}{en kende den weg}\\

\haiku{Zij riepen over de:}{gansche aarde en in den}{donkeren hemel}\\

\haiku{Ze hadden allen,.}{dier-vormen maar grooter}{en beter voltooid}\\

\haiku{Pan's doodbaar stond aan,.}{den oever der zee al het}{levende rondom}\\

\haiku{het blauw-en-witte,.}{kwam afdalen als een}{sneeuw-lawine}\\

\subsection{Uit: De kleine Johannes. Deel 3}

\haiku{Het land zag somber.}{en troosteloos als een land}{door lava verwoest}\\

\haiku{je kunt eens bij me,.}{op de vleugelen onzer}{dichter-vriendschap}\\

\haiku{Of geloof je 't? ',!}{nog niett Is heusch geen}{mopje van me hoor}\\

\haiku{Een wat ouder kind '.}{stond bij zijn knie en keek naar}{t dampende eten}\\

\haiku{boven je staat, v\`er,.}{en toch doet ze aldoor of}{ze de minste is}\\

\haiku{{\textquoteright} Deze, die voorover,.}{gebogen had gezeten}{richtte het hoofd op}\\

\haiku{- Als Jo nu fijner,....}{gezelschap noodig heeft dan moet}{hij zelf maar kiezen}\\

\haiku{Maar op 't laatste ',}{oogenblik schreef hijt af}{zonder te zeggen}\\

\haiku{Tot aan zijn neus kroop.}{hij onder de lakens en}{het zweet brak hem uit}\\

\haiku{Spoorrails en een trein.}{in de verte en opeens}{niet verder kunnen}\\

\haiku{Verwarde onzin,,.}{zonder ophouden met een}{zachte prevelstem}\\

\haiku{Toen ik bij haar kwam.}{zat ze te schreien en wou}{niets van me weten}\\

\haiku{Maar ze hield steeds vol,....}{dat zij leefde en terug}{zou komen en ook}\\

\haiku{Ja, ja, dat ware,.}{wel ontzettend al was het}{niet aan hen te zien}\\

\haiku{Dus hoorde wellicht.}{al die pracht aan de arme}{gekke Hel\'ene}\\

\haiku{Dat was echter van '.}{t poeder en hoorde zoo}{voor de deftigheid}\\

\haiku{Maar hij begreep niet.}{wat ze allen aan elkaar}{te zeggen hadden}\\

\haiku{Veel gereisd - papa -.}{kostschoolhouder van alles}{wat opgevangen}\\

\haiku{Hij deed zijn best en {\textquoteleft}.}{zong uit den treure vanO}{moeder de zeeman}\\

\haiku{Hij vroeg of hij naar,.}{huis mocht gaan daar hij moe was}{en hier niet hoorde}\\

\haiku{De deur ging open, de.}{verpleegster kwam er uit en}{liet de deur open staan}\\

\haiku{Alleen de sterren.}{schitterden strak en klaar aan}{den zwarten hemel}\\

\haiku{Zijn twee kinderen,}{waren nog even bekoorlijk}{maar zij maakten hem}\\

\haiku{{\textquoteright} - {\textquoteleft}Maar dan moet u toch {\textquotedblleft}{\textquotedblright}.}{noodzakelijk mijn boek over}{de Magie lezen}\\

\haiku{Toen zij dit zei, wierp.}{zij een verwijtenden blik}{naar den professor}\\

\haiku{Geen woord, noch blik, noch.}{teeken verraadde dat zij}{Johannes kende}\\

\haiku{{\textquoteright} ~ Maar Johannes'.}{verlangen naar Markus werd}{dagelijks sterker}\\

\haiku{- {\textquoteleft}Wat miste je dan,,?}{dat je bij hen niet vond maar}{wel ergens anders}\\

\haiku{{\textquoteright} - - {\textquoteleft}Nu dan, die Satan,.}{loert altijd op ons als een}{wolf op de schapen}\\

\haiku{Hij vond het echter:}{beter dat onopgemerkt}{te laten en zei}\\

\haiku{{\textquoteright} Van Lieverlee dacht,.}{na terwijl hij Johannes}{strak bleef aankijken}\\

\haiku{Een klerkje aan den.}{overkant werd opmerkzaam en}{hield op met zijn werk}\\

\haiku{Ook van Lieverlee,.}{keek belangstellend eenigszins}{onder den indruk}\\

\haiku{{\textquoteright} - Hierbij knikte het.}{mannetje fier en zette}{zijn mutsje vaster}\\

\haiku{{\textquoteright} - {\textquoteleft}Juist,{\textquoteright} zei Wistik, {\textquoteleft}weet?}{je nog wat Markus zei van}{de herinnering}\\

\haiku{Het water spoelde.}{en klotste aldoor om de}{uitgeholde steenen}\\

\haiku{Hij voelde zich nu,.}{een held na het aandurven}{van den Octopus}\\

\haiku{Het, dat ook bij den,.}{vijver zat toen het arme}{meisje zich verdronk}\\

\haiku{{\textquoteright} Johannes kon niet.}{laten te huiveren toen}{hij Bangeling zag}\\

\haiku{{\textquoteright} riep de stem van het,.}{mannetje nu als heel uit}{de verte omhoog}\\

\haiku{- {\textquoteleft}Dat dacht je niet, wel,?}{dat wij hier zulk een goede}{verlichting hadden}\\

\haiku{Door een laag en nauw '.}{gangetje gingen zij naar}{t volgend nummer}\\

\haiku{Bij duizenden rooft.}{hij de mooiste exemplaren}{van mijn collectie}\\

\haiku{Toen kwam een zeer lang,:}{en al smaller toeloopend}{zaaltje waarop stond}\\

\haiku{Het lange zaaltje.}{met de bedjes liep al door}{en werd steeds enger}\\

\haiku{{\textquoteright} Johannes zweeg en.}{de andere twee spraken}{een tijdlang samen}\\

\haiku{Toen hij weg was, was.}{eenige oogenblikken een}{gedwongen stilte}\\

\haiku{Hij kwam half overeind '.}{en staarde naar de zee en}{toen weer naart duin}\\

\haiku{Maar Johannes liet.}{haar niet in en zeide dat}{hij alleen wou zijn}\\

\haiku{Marjon had hij niet.}{gezien en hij wist niet of}{zij vertrokken was}\\

\haiku{Hij tikte den man,.}{op den schouder maar deze}{verroerde zich niet}\\

\haiku{- {\textquoteleft}Maar gij zijt nog geen,?}{mensch wilt gij een priester des}{Allerhoogsten zijn}\\

\haiku{{\textquoteright} {\textquoteleft}Bedelen hebt gij,.}{ze geleerd en de roede}{kussen die hen sloeg}\\

\haiku{{\textquoteright} {\textquoteleft}Goed voor u is het,?}{dat Hij niet anders doet want}{waar was uw redding}\\

\haiku{Ook eten en drinken,.}{is niet slecht maar alleen voor}{wie het noodig hebben}\\

\haiku{Mij hat-tie,.}{z'n bord in me snuit gegooid}{kijk h{\"\i}er wat een snee}\\

\haiku{{\textquoteleft}Perdon! - mevrouw en.}{de kleine worden niet mee}{derin geviteerd}\\

\haiku{Dat werd aanleiding.}{om hem voorloopig hier}{af te zonderen}\\

\haiku{- {\textquoteleft}Wilt u ons toestaan,,.}{mijnheer uw schedelmaten}{even op te nemen}\\

\haiku{En zoudt ge anders?}{beslissen als ik niet ben}{wat ge brutaal noemt}\\

\haiku{En Johannes ging.}{toen snel en dapper weg eer}{de tranen kwamen}\\

\haiku{Hij zag Marjon naast,.}{hem met wijd-gesperde}{oogen van ontzetting}\\

\haiku{{\textquoteright} - {\textquoteleft}Maar ik heb gezien,,.}{wat je uitstond Markus dien}{ellendigen avond}\\

\haiku{Doch tegen den avond,,.}{toen Marjon kwam was het met}{Keesje gedaan}\\

\haiku{En nogmaals hief hij,,:}{aan zachter maar vlijmend en}{pijnlijk doordringend}\\

\haiku{Maar Markus werd, straf,.}{gebonden door een zijdeur}{naar buiten gevoerd}\\

\haiku{Marjon's ijzeren,.}{bedje dat gansch schudde als}{ze zich even bewoog}\\

\haiku{Het klonk als een psalm,.}{maar zoo schoon en ernstig als}{hij nimmer hoorde}\\

\haiku{Met die vlam willen.}{de menschen hun brandende}{liefde aanduiden}\\

\haiku{{\textquoteright} - {\textquoteleft}En Aeschylus,{\textquoteright} zei de, {\textquoteleft}.}{vaderbij Marathon werd}{zijn hand afgehakt}\\

\haiku{Het instrument is,.}{gebarsten en zal in kort}{geen toon meer geven}\\

\haiku{Het is me of ik.}{nog weken lang dag en nacht}{zou moeten vragen}\\

\haiku{- we moesten toch maar eens, '?}{probeeren wat het mes hier doen}{kan ist niewaar}\\

\haiku{We zullen eens zien,?}{of je die hand niet nog weer}{gebruiken kunt he}\\

\haiku{{\textquoteright} Markus stak zijn hand,,: - {\textquoteleft}!}{uit die zij beiden kusten}{en sprakTot weerziens}\\

\subsection{Uit: De nachtbruid}

\haiku{Het leeft nog maar als.}{onbeteekenend deel van}{een grooter leven}\\

\haiku{Luister dan, lieve,,.}{lezer met wat geduld en}{geef u wat moeite}\\

\haiku{Ik wilde w\'a\'ar voor,.}{mijn geld want ik geloofde}{aan rechtvaardigheid}\\

\haiku{een dagblad lectuur, -:}{een afgeluisterd gesprek}{soms lichamelijk}\\

\haiku{{\textquoteright} - {\textquoteleft}Dat is duidelijk,! -.}{vader Maar toch houd ik nog}{iets te vragen over}\\

\haiku{- {\textquoteleft}E\'en ding is me nu,,.}{duidelijk mijn jongen dat}{je gauw trouwen moet}\\

\haiku{Ik wist dat zij nog.}{leefde en ik wist ook den}{naam van ons landgoed}\\

\haiku{Ik zie je nog zoo.}{duidelijk alsof ik je}{gisteren verliet}\\

\haiku{Want dat ik liever, -!}{met God omhoog wou dan met}{Satan omlaag nu}\\

\haiku{Bekommer je niet{\textquoteright} - - {\textquoteleft}!}{om wat er van me terecht}{komtArme jongen}\\

\haiku{Op 't punt van 't,.}{water van de zee en van}{het zeil-vermaak}\\

\haiku{Maar dat zijn eenmaal.}{zoo onze Italiaansche}{uitbundigheden}\\

\haiku{Het ware woord, de,,}{juiste redeneering het}{sluitend taalverband}\\

\haiku{Maar je hoeft die nu.}{tegenover de directie}{niet meer te spelen}\\

\haiku{de oplossing van.}{het geheim onzes levens}{ligt in den droom}\\

\haiku{{\textquoteleft}Dag Vico mio!,{\textquoteright} En, '.}{het was zijn stem meer nog dan}{t zijn gezicht was}\\

\haiku{Omdat ik toen eerst.}{zelf geseind had om naar haar}{te informeeren}\\

\haiku{Ik kwam langs boomen,.}{en groen en nam alles scherp}{en duidelijk waar}\\

\haiku{Ik verwaarloosde,}{mijn dagelijksch werk daarom}{niet integendeel}\\

\haiku{Links onder me was,.}{een geweldige afgrond}{ook een bergverschiet}\\

\haiku{We onderscheidden,.}{de menschen op den steenen pier}{die in zee uitstak}\\

\haiku{{\textquoteright} riep hij in 't Fransch,, {\textquoteleft}.}{toen het meisje voorbij was}{die je gister zag}\\

\haiku{Zoo ras als ik mij.}{een dag vrij kon maken ging}{ik weer uit zeilen}\\

\haiku{{\textquoteright} zei Elsje.  {\textquoteleft}Dat.}{geeft weer een heele week stof}{tot conversatie}\\

\haiku{{\textquoteright} - {\textquoteleft}Dat ik getrouwd ben?}{en een goede vrouw en}{vier kinderen heb}\\

\haiku{Maar ik weet niet of.}{het mij gelukken zal je}{dat te doen inzien}\\

\haiku{Ben je het oneens?}{met een der algemeene}{dingen die ik zei}\\

\haiku{met 't minste wat,}{je me geven wilt nu ik}{zoo oneindig meer}\\

\haiku{Niet Jan Baars, maar zijn,.}{zuster waar ik als kind aan}{huis genomen ben}\\

\haiku{Een laagheid van de,.}{soort waaruit ik mij juist met}{trots bevrijd voelde}\\

\haiku{Hij is geen vriend van,.}{tranen en geeft den zwaarmoed}{niet gaarne vat}\\

\haiku{{\textquoteright} en ik begreep dat.}{hij bedoelde dat ik mijn}{titel niet meer had}\\

\haiku{{\textquoteright} Toen sloeg Elsje haar:}{beide armen om mij heen}{en riep vreugdevol}\\

\haiku{Als wij kinderen.}{zijn houden we vader en}{moeder voor volmaakt}\\

\haiku{Maar ze deden niets,.}{spoedig weer in hun eigen}{belangen verdiept}\\

\haiku{Maar in mij woonde.}{een somber v\'o\'orgevoel met}{strakke zekerheid}\\

\haiku{als ons kindje wat, -?}{grooter is dat we weer in}{Holland gaan wonen}\\

\haiku{Ik ging naar buiten.}{en zag de blauwe lucht en}{een heerlijk landschap}\\

\subsection{Uit: Sirius en Siderius}

\haiku{Taede sloot het raam, '.}{stak een kaars op en ging met}{het licht naart bed}\\

\haiku{Moeder moest wel zeer.}{moe zijn om bij dit gerucht}{maar d\'o\'or te slapen}\\

\haiku{Neen, daar stond ze nog,, ',.}{rechtop midden int pad}{opziend naar zijn raam}\\

\haiku{Verschillend waren.}{ze van grootte en allen}{vreemd en bont gekleed}\\

\haiku{{\textquoteright} klonk achter hem het.}{hijgend-angstig fluisteren}{van zijn geleidster}\\

\haiku{Weet je wel tot wie '? -{\textquoteright} - {\textquoteleft} '?}{jet hebtZou ik anders}{doen als ikt wist}\\

\haiku{'t Is nu alles -....}{voor zijn verantwoording en}{van die anderen}\\

\haiku{Maar dat zag niemand.}{anders en neem ik niet op}{mijn verantwoording}\\

\haiku{{\textquoteleft}Ik heb een zwager,.}{die is leidekker en die}{heet Jezus Christus}\\

\haiku{Aan den wegkant was.}{een trog waarin een frissche}{waterstraal stroomde}\\

\haiku{'t Is goed dat hij,.}{gekomen is want er wordt}{met smart op gewacht}\\

\haiku{{\textquoteright} - {\textquoteleft}God zal hem net zoo,{\textquoteright}.}{lang laten leeven als hij}{noodig is zei Enna}\\

\haiku{De mooie liverei,,}{was erg vuil geworden zelfs}{het roode gladde}\\

\haiku{Boem! - klonk het van de, - -!}{nu nabije stad een dreunend}{kanonschot en Boem}\\

\haiku{{\textquoteright} - {\textquoteleft}Gun het ze dan, dat,{\textquoteright}.}{ze nu eens wijzer doen dan}{ze zijn zei Enna}\\

\haiku{Een officier te.}{paard kwam aanrijden en vroeg}{wat dat beduidde}\\

\haiku{En toen het gelukt, {\textquoteleft}!}{was juichten de passagiers}{en riepenhoera}\\

\haiku{Angstig luisterde.}{hij of Enna kreunde of}{Sirius schreidde}\\

\haiku{Toen bezon hij zich,, {\textquoteleft}!}{dat hij niet alleen wilde}{zijn en riepEnna}\\

\haiku{Het is alles veel.}{mooier en heerlijker dan}{het ooverdag zou zijn}\\

\haiku{De bedoeling van,,:}{zijn gebaar verstonden de}{ouders beiden als}\\

\haiku{{\textquoteleft}ik houd wel van een,.}{grapje maar ik laat me niet}{voor den mal houden}\\

\haiku{Daarom is hij de.}{eenige mensch op  aarde}{waar ze bang voor is}\\

\haiku{Zoo is dit schip een,.}{groot lijf bestuurd door een Ziel}{die niet het schip is}\\

\haiku{tot aan het toplicht,.}{dat als een kleine ster juist}{boven haar uitscheen}\\

\haiku{Maar ik geloof toch.}{niet dat de Oceaan deeze}{maal oover-wonnen is}\\

\haiku{Waarom verdient gij?}{het meer dan die allen die}{verongelukt zijn}\\

\haiku{- {\textquoteleft}U moest u schamen,,.}{mijnheer al deze menschen}{zoo te ontstemmen}\\

\haiku{{\textquoteright} {\textquoteleft}Ja, als de mist wil,{\textquoteright}.}{opklaren zei de jonge}{stuurman voorzichtig}\\

\haiku{{\textquoteright} Taede keek hem strak,.}{aan zonder te doen blijken}{dat hij hem verstond}\\

\haiku{Ik ben niet meer noodig,,.}{ik heb afgedaan maar jij}{moet nog beginnen}\\

\haiku{Je hebt zeker nog '.}{niet geslapen sinds wij van}{t schip afvoeren}\\

\haiku{Op zijn tijd zal het.}{donkere vlekje aan de}{kim wel verrijzen}\\

\haiku{Hij gaat den Herder.}{zoeken dien de aarde uit}{zicht verlooren heeft}\\

\haiku{- {\textquoteleft}Neen,{\textquoteright} zei de oude, {\textquoteleft},.}{ik ben niet blind maar de zon}{is niet meer zoo sterk}\\

\haiku{Maar mijn vrouw, die de,:}{wijste is van  alle}{menschen die zei mij}\\

\haiku{- {\textquoteleft}Het andere kind,.}{had ik \'o\'ok meer gezien maar}{ik weet niet meer w\'a\'ar}\\

\haiku{Maar voor mij, voor mijn.}{vrouw en mijn kind zijn het geen}{gewoone menschen}\\

\haiku{En ik begrijp nu,.}{dat het altijd om dat kind}{te doen is geweest}\\

\haiku{Ik geloof zelfs dat,.}{hij haar alleen getrouwd heeft}{omdat ik koomen moest}\\

\haiku{Zij waarschuuwde ons.}{ook als er booze plannen teegen}{ons beraamd werden}\\

\haiku{Ik begreep nu dat.}{mijn moeder samen met mij}{verbranden wilde}\\

\haiku{{\textquoteleft}Belet hem niet mij.}{van kant te maken als hem}{dat verligting geeft}\\

\haiku{Vier uuren duurde de,.}{spanning maar zij scheenen kort}{in Taede's gevoel}\\

\haiku{Maar daaruit besloot.}{zij ook dat hij nog onder}{de leevenden was}\\

\subsection{Uit: Van de koele meren des doods}

\haiku{De vorm van het brood,.}{dat elken morgen op de}{ontbijttafel lag}\\

\haiku{En toch was zij meest.}{van allen afkeerig van}{naar stad teruggaan}\\

\haiku{Er was iets boos en.}{verkeerds ingeslopen en}{zij begreep het niet}\\

\haiku{Zij liep door den gang,.}{over het witte hart bleef even}{staan en glimlachte}\\

\haiku{Zij dacht aan alles,.}{met genot en kon het niet}{genoeg herdenken}\\

\haiku{En Hedwig knikte dan ' {\textquoteleft}!}{vrindelijkt eerste en}{zeiDag grijsjasjes}\\

\haiku{De kinderen van,.}{het dorp waren onthaald zooals}{dat gewoonte was}\\

\haiku{Op school had zij les.}{van een jongen meester die}{veel aan kiespijn leed}\\

\haiku{De ernstige schijn.}{der dingen werd tegen den}{avond strak en somber}\\

\haiku{Wat leek het oude,.}{leven zalig waaraan die}{klank herinnerde}\\

\haiku{Hij zat plat neer met.}{gevouwen knie\"en en leek}{veel jonger dan Hedwig}\\

\haiku{ik wou dat ik met,.}{alle jongens met alle}{menschen zoo doen kon}\\

\haiku{Zwart en kaal ook de.}{heester-skeletten rondom}{langs den ouden muur}\\

\haiku{Het was in dezen.}{levenstijd dat zij muziek}{begon te verstaan}\\

\haiku{{\textquoteleft}Mijn zuster is zoo,{\textquoteright},.}{ziek zei hij niet wetend of}{dit onoprecht was}\\

\haiku{Ze kneep haar vingers {\textquoteleft}!}{in elkaar en prevelde}{Was ik toch maar dood}\\

\haiku{O neen, het was een.}{veel ouder en sterker en}{dierbaarder man}\\

\haiku{{\textquoteright} {\textquoteleft}Maar waarom doe je,?}{dan zoo anders den eenen dag}{of den andere}\\

\haiku{Maar had hem toen, in,}{dat uur iemand herinnerd}{of medegedeeld}\\

\haiku{nu bleef 't weg, of ',.}{t kwam door gelijkenis}{als schaduw of echo}\\

\haiku{En dat, nu zij toch!}{in liefde leefde en een}{heilstaat voor zich zag}\\

\haiku{Maar doodgaan scheen haar,.}{altijd nog veel beter nog}{veel begeerlijker}\\

\haiku{Zij wilde daarvan.}{zekerheid en beproefde}{zich op te richten}\\

\haiku{Eindelijk reed een,.}{wagentje aan met een boer}{en een veldwachter}\\

\haiku{Toen poogde zij te,.}{doen zooals de vromen doen en}{aan God te denken}\\

\haiku{{\textquoteright} En zij dacht hoe zij.}{zelve z\'o\'o liggen zou en}{gevonden worden}\\

\haiku{Alleen Johan had het.}{er mooi gevonden en hield}{van huis en menschen}\\

\haiku{, zij zag niets waaraan.}{zich vast te klemmen om niet}{weder te zinken}\\

\haiku{De oudste broeder,.}{was over zee Aernout in}{een andere stad}\\

\haiku{Toen besloot Gerard,,.}{door zorg en leed gedreven}{zich te vermannen}\\

\haiku{Al wat Gerard van,,.}{Ritsaart wist was hem tegen}{en leek ongunstig}\\

\haiku{Hij ging naar Hedwig en.}{bood geen weerstand meer aan de}{zoete betoovering}\\

\haiku{Deze ontdekking}{echter verruimde hem en}{stemde hem zachter}\\

\haiku{Maar word goed verliefd,.}{en bind dan want het rechte}{komt maar \'e\'ens goed}\\

\haiku{De maaltijden hield,.}{hij bij Joob daarvan was hij}{niet af te brengen}\\

\haiku{Dit is Monica,{\textquoteright}, {\textquoteleft},.}{zei Joobvulgo Jansje mijn}{beschermengel}\\

\haiku{Maar toen Ritsert haar.}{daarop aanzag begreep zij}{wat zij gezegd had}\\

\haiku{{\textquoteright} - - {\textquoteleft}Ja, maar, het \'e\'ene,.}{uur vind ik d{\'\i}t beter het}{andere uur d\'at}\\

\haiku{{\textquoteright} - - {\textquoteleft}Maar kan ik dan niet '?}{eeuwig heil verspelen door}{n verkeerde keus}\\

\haiku{Dit beangste hem, '?}{zeer zou Hedwig nu gaan baden}{en int duister}\\

\haiku{Snel tilde hij haar ',.}{uitt water het slappe}{hoofd op zijn schouder}\\

\haiku{{\textquoteright} Gerard zweeg, het hoofd,.}{afwendend zijn woede en}{afkeer verbijtend}\\

\haiku{En hoe kon in 't?}{hart der doodzonde de weg}{tot God zich openen}\\

\haiku{Maar dit kon zij niet,.}{ontwarren het was te zeer}{ineengevlochten}\\

\haiku{Omdat hij voor de.}{daglooners niet veel slechter zorgt}{dan voor zijn paarden}\\

\haiku{Het huisje lag hoog,.}{en alleen op open klip aan}{een inham der zee}\\

\haiku{Alles leek mooier,.}{en dierbaarder thuis zelfs de}{zee en de wolken}\\

\haiku{Zij was zoo overtuigd.}{en welsprekend dat Janet}{zich om liet stemmen}\\

\haiku{Daarna ging zij er '.}{mee int rijtuig en nam}{den trein naar Londen}\\

\haiku{{\textquoteright} - Toen wilde hij uit.}{minzaamheid Hedwigs koffertje}{op het rek zetten}\\

\haiku{{\textquoteleft}Geef me je beurs om,{\textquoteright}.}{den koetsier te betalen}{zei de vagebond}\\

\haiku{Niemand had kunnen.}{uitvorschen wie zij was of}{waar zij vandaan kwam}\\

\haiku{Zij at te middag,.}{bij den docter en zij was}{wat opgewekter}\\

\haiku{Er was een planken,,.}{vloer een ijzeren bed en}{twee stoelen meer niet}\\

\haiku{\'e\'en was, al voelde.}{zij tevens dat zij hem nooit}{meer mocht terugzien}\\

\haiku{Ik werd vreeselijk.}{weemoedig en moest aldoor}{aan Gerard denken}\\

\haiku{Het was zoo heerlijk.}{te hooren en hij nam mij}{mee naar een concert}\\

\haiku{Maar ik vond een klein.}{oud kerkje in een oude}{stille achterbuurt}\\

\haiku{Hedwig keek even \'op, met,,.}{snellen schuwen blik benieuwd}{wie z\'o\'o spreken kon}\\

\haiku{Grooter kinderen,,.}{zijn altijd nog zelfzuchtig}{hebberig twistziek}\\

\haiku{Het hoogste wezen,?}{is de hoogste vreugde wat}{kan het anders zijn}\\

\haiku{Maar het eigene.}{en persoonlijke aan ons}{is het meest kwetsbaar}\\

\haiku{{\textquoteright} - {\textquoteleft}Ook wel omdat ik,.}{mij schaamde omdat ik mij}{zelve redden wou}\\

\haiku{Elke dag probeer.}{ik nu den raad van zuster}{Paula te volgen}\\

\haiku{Nu zal God zich wel.}{weer van mij afkeeren en ik}{zal weer gek worden}\\

\haiku{Ik heb twee dagen,.}{op mijn kamer doorgebracht}{in angst en jammer}\\

\haiku{{\textquoteright} - {\textquoteleft}Is men in Holland?}{niet streng in de eischen voor}{liefde-zusters}\\

\haiku{{\textquoteright} - {\textquoteleft}Maar, mijn zuster, als,.}{het niet zoo ware dan zou}{het een wonder zijn}\\

\haiku{Het eigene sterft,.}{alleen door het leven niet}{door den lijfs-dood}\\

\haiku{{\textquoteright} - {\textquoteleft}Zouden wij dan na?}{den dood weer een nieuw leven}{moeten beginnen}\\

\haiku{Zij had dien al zoo, '.}{lang overdacht zij had hem maar}{voort opschrijven}\\

\haiku{Zij stond om half zes.}{op en dronk dan koffie met}{Harmsen en de vrouw}\\

\haiku{Ik schrijf nu al een,.}{jaar aan mijn boek en ben nog}{niet eens recht op gang}\\

\haiku{Van dat dagboek wordt.}{gezegd dat dit de oorsprong}{is van de roman}\\

\haiku{Pas jaren later,,.}{op 30 maart 1918 schreef zij hem}{in een brief erover}\\

\subsection{Uit: Van de koele meren des doods}

\haiku{De vorm van het brood.}{dat elken morgen op de}{ontbijttafel lag}\\

\haiku{En toch was zij meest.}{van allen afkeerig van}{naar stad teruggaan}\\

\haiku{Er was iets boos en.}{verkeerds ingeslopen en}{zij begreep het niet}\\

\haiku{Zij liep door den gang,.}{over het witte hart bleef even}{staan en glimlachte}\\

\haiku{Zij dacht aan alles,.}{met genot en kon het niet}{genoeg herdenken}\\

\haiku{En Hedwig knikte dan ' {\textquoteleft}!}{vrindelijkt eerste en}{zeidag grijsjasjes}\\

\haiku{De kinderen van,.}{het dorp waren onthaald zooals}{dat gewoonte was}\\

\haiku{De ernstige schijn.}{der dingen werd tegen den}{avond strak en somber}\\

\haiku{Wat leek het oude.}{leven zalig waaraan die}{klank herinnerde}\\

\haiku{Hij zat plat neer met.}{gevouwen knie\"en en leek}{veel jonger dan Hedwig}\\

\haiku{ik wou dat ik met,.}{alle jongens met alle}{menschen zoo doen kon}\\

\haiku{Zwart en kaal ook de.}{heester-skeletten rondom}{langs den ouden muur}\\

\haiku{Zij kwam niet, maar dat '.}{zij niet komen zou wist zij}{niet eert avond werd}\\

\haiku{Het was in dezen.}{levenstijd dat zij muziek}{begon te verstaan}\\

\haiku{{\textquoteleft}Mijn zuster is zoo,{\textquoteright},.}{ziek zei hij niet wetend of}{dit onoprecht was}\\

\haiku{Ze kneep haar vingers {\textquoteleft}!}{in elkaar en prevelde}{Was ik toch maar dood}\\

\haiku{O neen, het was een.}{veel ouder en sterker en}{dierbaarder man}\\

\haiku{{\textquoteright} {\textquoteleft}Maar waarom doe je,?}{dan zoo anders den eenen dag}{of den andere}\\

\haiku{Maar had hem toen, in,}{dat uur iemand herinnerd}{of medegedeeld}\\

\haiku{nu bleef 't weg, of ',.}{t kwam door gelijkenis}{als schaduw of echo}\\

\haiku{En dat, nu zij toch!}{in liefde leefde en een}{heilstaat voor zich zag}\\

\haiku{Maar doodgaan scheen haar,.}{altijd nog veel beter nog}{veel begeerlijker}\\

\haiku{Zij wilde daarvan.}{zekerheid en beproefde}{zich op te richten}\\

\haiku{Eindelijk reed een,.}{wagentje aan met een boer}{en een veldwachter}\\

\haiku{Toen poogde zij te,.}{doen zooals de vromen doen en}{aan God te denken}\\

\haiku{{\textquoteright} En zij dacht hoe zij.}{zelve z\'o\'o liggen zou en}{gevonden worden}\\

\haiku{Alleen Johan had het.}{er mooi gevonden en hield}{van huis en menschen}\\

\haiku{De oudste broeder,.}{was over zee Aernout in}{een andere stad}\\

\haiku{Toen besloot Gerard,,.}{door zorg en leed gedreven}{zich te vermannen}\\

\haiku{Al wat Gerard van,,.}{Ritsaart wist was hem tegen}{en leek ongunstig}\\

\haiku{Hij ging naar Hedwig en.}{bood geen weerstand meer aan de}{zoete betoovering}\\

\haiku{Deze ontdekking}{echter verruimde hem en}{stemde hem zachter}\\

\haiku{Maar word goed verliefd,.}{en bind dan want het rechte}{komt maar \'e\'ens goed}\\

\haiku{De maaltijden hield,.}{hij bij Joob daarvan was hij}{niet af te brengen}\\

\haiku{Maar toen Ritsert haar.}{daarop aanzag begreep zij}{wat zij gezegd had}\\

\haiku{{\textquoteright} - - {\textquoteleft}Maar kan ik dan niet '?}{eeuwig heil verspelen door}{n verkeerde keus}\\

\haiku{Dit beangste hem, '?}{zeer zou Hedwig nu gaan baden}{en int duister}\\

\haiku{Snel tilde hij haar ',.}{uitt water het slappe}{hoofd op zijn schouder}\\

\haiku{En hoe kon in 't?}{hart der doodzonde de weg}{tot God zich openen}\\

\haiku{Maar dit kon zij niet,.}{ontwarren het was te zeer}{ineengevlochten}\\

\haiku{Omdat hij voor de.}{daglooners niet veel slechter zorgt}{dan voor zijn paarden}\\

\haiku{Alles leek mooier,.}{en dierbaarder thuis zelfs de}{zee en de wolken}\\

\haiku{Zij was zoo overtuigd.}{en welsprekend dat Janet}{zich om liet stemmen}\\

\haiku{Daarna ging zij er '.}{mee int rijtuig en nam}{den trein naar Londen}\\

\haiku{{\textquoteright} - Toen wilde hij uit.}{minzaamheid Hedwig's koffertje}{op het rek zetten}\\

\haiku{Niemand had kunnen.}{uitvorschen wie zij was of}{waar zij vandaan kwam}\\

\haiku{Zij at te middag,.}{bij den docter en zij was}{wat opgewekter}\\

\haiku{Er was een planken,,.}{vloer een ijzeren bed en}{twee stoelen meer niet}\\

\haiku{\'e\'en was, al voelde.}{zij tevens dat zij hem nooit}{meer mocht terugzien}\\

\haiku{Dan ging zij op de,.}{drukke straat meenend weerstand}{te kunnen bieden}\\

\haiku{Ik werd vreeselijk.}{weemoedig en moest aldoor}{aan Gerard denken}\\

\haiku{Het was zoo heerlijk.}{te hooren en hij nam mij}{mee naar een concert}\\

\haiku{Maar ik vond een klein.}{oud kerkje in een oude}{stille achterbuurt}\\

\haiku{Hedwig keek even \`op, met,,.}{snellen schuwen blik benieuwd}{wie z\'o\'o spreken kon}\\

\haiku{Grooter kinderen,,.}{zijn altijd nog zelfzuchtig}{hebberig twistziek}\\

\haiku{Het hoogste wezen,?}{is de hoogste vreugde wat}{kan het anders zijn}\\

\haiku{Maar het eigene.}{en persoonlijke aan ons}{is het meest kwetsbaar}\\

\haiku{{\textquoteright} - {\textquoteleft}Ook wel omdat ik,.}{mij schaamde omdat ik mij}{zelve redden wou}\\

\haiku{Nu zal God zich wel.}{weer van mij afkeeren en ik}{zal weer gek worden}\\

\haiku{Ik heb twee dagen,.}{op mijn kamer doorgebracht}{in angst en jammer}\\

\haiku{{\textquoteright} - {\textquoteleft}Is men in Holland?}{niet streng in de eischen voor}{liefde-zusters}\\

\haiku{{\textquoteright} - {\textquoteleft}Maar, mijn zuster, als,.}{het niet zoo ware dan zou}{het een wonder zijn}\\

\haiku{{\textquoteright} - {\textquoteleft}Zouden wij dan na?}{den dood weer een nieuw leven}{moeten beginnen}\\

\haiku{Zij had dien al zoo, '}{lang overdacht zij had hem maar}{voort opschrijven}\\

\haiku{Zij stond om half zes.}{op en dronk dan koffie met}{Harmsen en de vrouw}\\

\section{Georges Eekhoud}

\subsection{Uit: Kees Doorik of een bloedig half-vasten}

\haiku{Kees verzamelde.}{het alaam in een hoek van de}{schuur en sloot de deur}\\

\haiku{Met een zichtbare,}{ingenomenheid vernam}{hij welk een klein eter}\\

\haiku{Nelis Cramp kwam hem.}{voor als de vrijgevigste}{van \`al de bazen}\\

\haiku{Op zijn vijftiende.}{jaar was Kees Doorik reeds een}{kranige jongen}\\

\haiku{Ze liet ieder maar,.}{razen en verloor daarom}{noch eetlust noch lach}\\

\haiku{Daar ze hun pijpen,.}{stopten uit een varkensblaas}{gaf Paulien wat vuur}\\

\haiku{Bijna op het bed,.}{uitgestrekt  draaide zij}{den rug toe naar Kees}\\

\haiku{'k Geloof toch, dat '!}{n knecht zoowel bloed in z'n lijf}{heeft als zijn meester}\\

\haiku{Kees was weg om het,,.}{veld van IJlwaal aan de}{Schelde te beren}\\

\haiku{'t Is markt morgen,...?}{en ik kwam uw boodschappen}{halen Geen belet}\\

\haiku{Wij hebben nog maar...}{de rogge en de toemaat}{binnen te halen}\\

\haiku{Mie, wa' zijt-de ',.}{gijn gelukkige het}{gelukskind zelf}\\

\haiku{hernam hij, terwijl.}{ze koffie inschonk en een}{boterham smeerde}\\

\haiku{Daar zijn nog genoeg...}{vondelingen en bastaards}{na hem te vinden}\\

\haiku{maar gij zijt toch niet,...}{daaronder te rekenen}{veronderstel ik}\\

\haiku{{\textquoteleft}Er is een middel:}{om dien zeldzamen knecht in}{uw dienst te krijgen}\\

\haiku{De aanhouding van ',;}{nen Duitschen smokkelaar door}{de tolbeambten}\\

\haiku{Jurgen Faas riep den, ':}{man met het varken toenen}{daglooner van Stabroeck}\\

\haiku{Bij dit spelleken '.}{barstte gansch de zaal haast van}{t danig lachen}\\

\haiku{maar hij beloofde,.}{zich vast er niet \'al te kort}{spel mee te maken}\\

\haiku{Ze vertrok met haar}{broer  Tist en den van haar}{niet weg te slagen}\\

\haiku{Kees antwoordde met ',,:}{nen vloek en vertrok na een}{onzedig gebaar}\\

\haiku{'t Is Jurgen, die... '}{zijn lierenaar tegen mij}{heeft uitgetrokken}\\

\haiku{Tranen welden op,.}{in den vervloekte zijn oogen}{en hij boog het hoofd}\\

\section{J.K. van Eerbeek}

\subsection{Uit: Lichting '18}

\haiku{Hij staat daar als een,.}{kind dat een prachtig geschenk}{heeft ontvangen}\\

\haiku{Een effen gezicht,.}{dat over de verbazing van}{den H.B.S.-er heenziet}\\

\haiku{Als een reus, zoo gaat.}{die aandacht staan boven de}{doffe schoolbanken}\\

\haiku{Het is of hij er.}{maar amper het zwarte bord}{mee durft te raken}\\

\haiku{hij glimlacht, grijnst, duikt,.....}{diep weg achter zijn eigen}{masker verslagen}\\

\haiku{De klas werkt weer, lang.}{niet ieder heeft begrepen}{wat er gebeurd is}\\

\haiku{Op tafel staat nog,.}{het glas waar de heer Speckmans}{uit gedronken heeft}\\

\haiku{een verschil, dat men.}{niet verklaren kan uit het}{groeiproces alleen}\\

\haiku{Twee tegenstanders,.}{die ieder door hun eigen}{angst verslagen zijn}\\

\haiku{dat is links-om, zoover,{\textquoteright}....}{ik weet had Toon Homan toen}{listig geantwoord}\\

\haiku{Van Veen,{\textquoteright} zei hij, {\textquoteleft}kun?}{je me niet eens aan een mud}{aardappels helpen}\\

\haiku{Een blauwe linnen;}{marktjas heeft hij zich om de}{schouders geschoven}\\

\haiku{Ik heb dat eenmaal....}{eens bij een man gezien op}{de markt te Nijkerk}\\

\haiku{en mijn tante ligt,.}{ziek versterkend eten is voor}{haar niet te krijgen}\\

\haiku{Het is al bijna,,....}{geen waarheid meer de waarheid}{die hij gezien heeft}\\

\haiku{Er begon zich een;}{vreemde beklemming op de}{menschen te leggen}\\

\haiku{snel overlegt hij, wat,.}{hem te doen staat wanneer een}{agent komt controleeren}\\

\haiku{En hij kende ook,,.}{een zekere trots van daar}{te staan waar hij stond}\\

\haiku{Van Toon's gedachten.}{uit stroomt het licht onder de}{grauwe pannen uit}\\

\haiku{De jonge Homan.}{probeert het eene gezegde}{na het andere}\\

\haiku{Hij houdt haar hand vast,;}{en strijkt met de beenen spoel van}{Coty over haar hand}\\

\haiku{bij 't begin...., zoover,.}{als ze bekend zijn staan de}{jaartallen erbij}\\

\haiku{Ik wed, dat je daar.}{de namen van deze muur}{op geschreven hebt}\\

\haiku{Daar kon hij goed mee,;}{uit want hij had een baard zoo}{hard als een spijker}\\

\haiku{dan zeilt hij van de;}{eene dolle situatie}{in de andere}\\

\haiku{Ik was al een paar;}{keer die week met m'n wagen}{de straat op geweest}\\

\haiku{Het komt me vreemd voor,.}{dat je met een kind in dit}{weer uit rijden gaat}\\

\haiku{om een kapotte.}{autoband zetten ze een}{volksfeest in mekaar}\\

\haiku{Kapitein Colff, de,;}{compagnies-commandant}{komt uit de barak}\\

\haiku{Er is niemand die....}{er bezwaar tegen heeft zijn}{lichaam te trainen}\\

\haiku{Enfin, Homan, als,....}{een oud model tank schommelt}{voorbij den ouwe}\\

\haiku{men beoefent het;}{schuinsrechts achterwaarts plaatsen}{van het rechterbeen}\\

\haiku{En de verhalen,.}{die over Wehmeyer in omloop}{zijn worden verteld}\\

\haiku{Hij voelt dat hij er,,;}{niet buiten kan buiten die}{zorg die risico}\\

\haiku{hij is niet zeker,,.}{of hij niet capituleeren}{zal op het eind}\\

\haiku{{\textquoteleft}Ik neem het je niet,....}{kwalijk dat je me gister}{uitgelachen hebt}\\

\haiku{{\textquoteleft}Ik begrijp niet,{\textquoteright} zegt,;}{hij als hij een tijdlang die}{kaart heeft bekeken}\\

\haiku{Ze kunnen me hier,.}{niet laten zitten met dat}{ijzer in me maag}\\

\haiku{hij let nu op zijn,,.}{oogen en daar ziet hij iets in}{dat hem niet aanstaat}\\

\haiku{Daarom gaat hij naar.}{den dokter en rapporteert}{zijn bevindingen}\\

\haiku{Er wordt besloten.}{dat men het nog drie dagen}{met hem zal probeeren}\\

\haiku{Hij licht een tip op,,.}{van de sluier maar er is}{niemand die nog ziet}\\

\haiku{Hij was er zich op,}{dat oogenbiik pijnlijk van}{bewust dat hij me\^e}\\

\haiku{In de kazerne.}{wordt met behoorlijk respect}{over voedsel gepraat}\\

\haiku{Homan begrijpt nu.}{waarom alle koks er zoo}{welgedaan uitzien}\\

\haiku{de leider is het,,.}{die het doel bepaalt en de}{verantwoording draagt}\\

\haiku{Er zijn er, die als;}{wild geworden inlanders}{met krissen zwaaien}\\

\haiku{De twee, van wie een,.}{naar de Distel zal moeten}{kijken elkaar aan}\\

\haiku{Het komt me voor, of,;}{ik deze kamer eerder}{heb gezien dacht hij}\\

\haiku{Ze bloosde, liet haar,,.}{viool glijden en wist niet}{wat te antwoorden}\\

\haiku{Het geval wilde,.}{namelijk dat hij geplaatst}{werd naar een kustwacht}\\

\haiku{{\textquoteleft}Eerste peleton,{\textquoteright}.}{allen present rapporteert}{de klasse-oudste}\\

\haiku{hij spreekt nog door, maar.}{zijn aandacht is al ginds op}{het kazerneplein}\\

\haiku{die een mengsel is.}{van de uitwaseming van}{menschen en leergoed}\\

\haiku{voor het feit dat ze.}{het niet eerder deden mag}{je God dankbaar zijn}\\

\haiku{Ik bedoel hier niet,;}{alleen de orthodoxen wat}{de godsdienst betreft}\\

\haiku{weet je, ik heb een,.}{gevoel of ik al jaren}{steenen gebakken heb}\\

\haiku{En werkelijk, men.}{kon ook niet anders dan er}{haar dankbaar voor zijn}\\

\haiku{men ziet de zoom zacht;}{in de schaduw achter de}{muur wegdoezelen}\\

\haiku{Voor het geluk van....}{tallooze gezinnen vechten}{de diplomaten}\\

\haiku{Wij hebben geen schijn,....}{van de macht die noodig is om}{dat te beletten}\\

\haiku{Het was de avond van,.}{de eerste lentedag die}{het jaar gebracht had}\\

\haiku{Hij ziet zich zelf als.}{het ware zitten op die}{plek op de wereld}\\

\haiku{Over de doktoren.}{praten ze als jongens over}{hun onderwijzer}\\

\haiku{Op de bestofte,,.}{zolder die hij eerst over moest}{kwam al niemand meer}\\

\haiku{Wat beteekent deze,?}{kamer die telkens in zijn}{droomen terugkeert}\\

\haiku{De bewoners van;}{het hospitaal vormen een}{kleine maatschappij}\\

\haiku{het personeel, de.}{bezoekers behooren bij}{een andere groep}\\

\haiku{Nous venons te voir,,,.}{Zuus le dimanche tu sais.{\textquoteright}11}{Maar hij komt terug}\\

\haiku{Wie aangewezen,,.}{is weet meteen wat hij van}{zichzelf denken moet}\\

\haiku{Nu eens kraakt in de,.}{nacht het bed van den een dan}{dat van den ander}\\

\haiku{Zoo nu en dan kijkt, {\textquoteleft}{\textquoteright}.}{hij tusschen de stammen door}{deverstuiving in}\\

\haiku{Hij kijkt alsof hij.}{het beeld van de heide met}{zich mee wil nemen}\\

\haiku{achter de blinde,....}{muur moest een boschage staan die}{dempte het geluid}\\

\haiku{naar van Beek, die met.}{eindeloos geduld telkens}{weer zijn spuwglas grijpt}\\

\haiku{maar voor die er van,.}{gedronken heeft is de helft}{over de rand gestort}\\

\haiku{{\textquoteright} Wanneer men met hem,,.}{spreekt bedaart het schokken dat}{door zijn lichaam trekt}\\

\haiku{Er is niemand, die.}{den vreemdeling een verwijt}{maakt van zijn gedrag}\\

\haiku{En de aralia die,.}{in de conversatiezaal}{staat is gaan werken}\\

\haiku{Je denkt wel eens, dat;}{je te maken hebt met de}{menschen om je heen}\\

\haiku{Dan is het immers,,,....}{of die schedel die je brein}{omspant zoo ruim is}\\

\haiku{Even zakelijk heeft,.}{hij dit vastgesteld alsof}{het hem niet aanging}\\

\haiku{Alleen als hij op,.}{zijn rechterzij ligt kan hij}{soms amper ademen}\\

\haiku{Donker was het in,....}{de kamer en alleen in}{die nis zag hij iets}\\

\haiku{Zou het mogelijk,,?}{zijn dat een wond geneest en}{geen lidteeken nalaat}\\

\haiku{Het kan een feest zijn,,,.}{gewoon ergens te staan te}{ademen daar te zijn}\\

\haiku{een schot is het, dat.}{men z\'o\'o tusschen de huizen}{vandaan nemen kan}\\

\haiku{Hij kwam hier - en er.}{was een gruwelijk hiaat}{in zijn gedachten}\\

\haiku{Eigenlijk, zoo denkt,,.}{hij wanneer hij alleen is}{ben ik twee menschen}\\

\haiku{Zoo een laat de wel,....}{van zijn persoonlijkheid niet}{nooit heelemaal los}\\

\haiku{, maar er is zulk een....}{rare looden onrust over zijn}{wezen gekomen}\\

\haiku{{\textquoteleft}Je houdt eenvoudig,.}{het geld onder je en je}{schrijft het Maandag in}\\

\haiku{En.... er bestaat een,.}{aapachtige scepsis die}{men niet kwalijk neemt}\\

\haiku{Een van z{\`\i}jn lichting,,.}{een die naast hem zat in het}{gymnastieklokaal}\\

\haiku{En Homan herkent,.}{de glimlach waarmee hun hun}{plaats gewezen is}\\

\haiku{telkens knikt ze met,:}{het hoofd bij iedere keer}{dat ze een naam noemt}\\

\haiku{En dan praat Anje,.}{weer zoo lang tot ze opnieuw}{moet afscheid nemen}\\

\haiku{hij verwacht het een,.}{of ander van dit bezoek}{en weet zelf niet wat}\\

\haiku{een antwoord op de,.}{vraag die dat stilstaan daar voor}{de kleine vrouw is}\\

\haiku{Tot grooter afscheid,.}{reikt ze hem nu de hand dan}{waartoe ze hem groet}\\

\haiku{En het lichtste licht,,....}{dat er op de wereld is}{gaat er over lichten}\\

\haiku{Hij was vervuld van;}{een diep medelijden met}{de oude Anje}\\

\haiku{Moet het leven, dat,?}{hem straks weer ontvangen zal}{zoo geleefd worden}\\

\haiku{Alle liedjes die,;}{hem aangesproken hebben}{gaat hij begrijpen}\\

\haiku{Een die zijn geduld;}{tegen de verlate tram}{staat uit te meten}\\

\haiku{Op die manier leeft.}{iemand met die afspraak als}{met een daimonion}\\

\haiku{En het wezen van:}{die afspraak is vastgelegd}{in deze woorden}\\

\haiku{Een oogenblik luwt,.}{de kramp die het dier in de}{ingewanden snijdt}\\

\haiku{er leeft wel zeker;}{nog een herinnering in}{deze pupillen}\\

\haiku{Als een onttakeld,,.}{schip dat dronken op de stroom}{drijft zoo ligt het daar}\\

\haiku{{\textquoteleft}Onze Toon heeft z'n,{\textquoteright}.}{diploma onze Toon voor}{en onze Toon na}\\

\haiku{En het grappige,....}{was dat zijn oom zelf evenzoo}{in het feit niets zag}\\

\haiku{de warme roode.}{gloed achter de lichtbakken}{lokt hem als een kind}\\

\haiku{Deze stond bij het,....}{buffet en keek als hij door}{het raam naar buiten}\\

\haiku{dat na{\"\i}ever is,....}{naarmate wij grooter \'echec}{in ons leven zien}\\

\haiku{{\textquoteright} {\textquoteleft}U is schromelijk,{\textquoteright};}{onbillijk tegenover de}{kerk roept nu iemand}\\

\haiku{Kil was de lucht, de.}{gele bladen zaten aan}{de eikenstammen}\\

\haiku{Hij ziet den kleinen,.}{man op jacht in dit veld en}{in zijn leven ook}\\

\haiku{Niets zagen deze.}{oogen van een ander dan wat}{hem vernederde}\\

\haiku{Als hij een spoor had,,,....}{geweten waardoor zijn ziel}{gaaf had kunnen gaan}\\

\haiku{Want op zijn manier,.}{is hij een van die welke}{moeilijk vergeten}\\

\haiku{fosforiseert de,.}{grauwe kalk op de witte}{muur die gaat lichten}\\

\haiku{Een tweede lamp gleed.}{over de ruiten der kleine}{voorstadswinkels mee}\\

\haiku{Meer ruimte had hij;}{eigenlijk niet noodig gehad}{om in te wonen}\\

\haiku{Een donderslag als.}{het knallen van een zweep en}{de angst in het hart}\\

\haiku{Telkens vingen de....}{grauwe muren der huizen}{een moment de glans}\\

\haiku{Hij verlaagt zich er;}{toe zijn tegenstander als}{gangmaker te zien}\\

\haiku{Een gedachte heeft;}{zich vastgehaakt aan wat daar}{op het doek gebeurt}\\

\haiku{hij ziet dwars door de,;}{atmosfeer die zich om zijn}{schedel heen verdicht}\\

\haiku{Ook zij zouden zich.}{legeren aan de voet van}{deze torens}\\

\subsection{Uit: Strooschippers}

\haiku{Het was 'n schip zooals.... '}{er die dagen alleen op}{de Gulle voeren}\\

\haiku{Van een gedeelte;}{van de Holterdiek had ie}{et zand afgebrocht}\\

\haiku{as et knappen wil,, '....}{laat et knappen dan leggen}{we d'rn knoop bij}\\

\haiku{De volgende dag ',.}{hadden zet in de wind}{en ze moesten jagen}\\

\haiku{en z'n vrouw ook) - Man, ' '.}{ikeb van  de Gulle}{nog geen druppelezien}\\

\haiku{De grauwe vloer, die,....}{de dunne planken van het}{schip draagt glijdt en schuift}\\

\haiku{De flarden van dat.}{bidden keven daar onder}{de dichte hemel}\\

\haiku{Want hij wist, hoe stijf;}{die menschen op de luiken}{van hun schip stonden}\\

\haiku{Rustig en ernstig.}{voeren ze de haven van}{het stadje binnen}\\

\haiku{Daar lagen ze nu,.}{met twintig vijf en twintig}{andere schippers}\\

\haiku{En s'avonds zaten,.}{ze onder de linden die}{langs de kaai stonden}\\

\haiku{Of mijnheer hen niet, '.}{wat wegwijs kon makenoe}{ze daar mee aan moesten}\\

\haiku{- Mijnheer wilde wel,.}{zorgen dat dat in orde}{kwam met dat liggeld}\\

\haiku{En m'n pake komt, ' ' '....}{daar en praat wat mitr en}{eeftr wat egeven}\\

\haiku{Ze zit steeds voor 'et, '.}{bed wanneer ze alleen is}{en geen menscher ziet}\\

\haiku{Ze had noch voor de,;}{zegen noch voor de vloek van}{haar defect een naam}\\

\haiku{Maar de spoken, waar,;}{hij in de somp mee gespeeld}{had plaagden hem nu}\\

\haiku{En het maakt zooveel,.}{wind dat de kleeren wapperen}{hun om het lichaam}\\

\haiku{'t Is goed, da'k 'ier,,....}{sta dacht hij anders vlogen}{ze menander an}\\

\haiku{er voer geen tweede,.}{zoo mooi van de werf dat kan}{men eerlijk zeggen}\\

\haiku{- Wanneer laden we,,.}{moest hij den jongen vragen}{omdat die niet sprak}\\

\haiku{- Je moest me even naar,, -.}{de wal brengen Jouk moest zijn}{vader herhalen}\\

\haiku{hij haatte het schip.}{zooals hij het zijn vader deed}{op dat oogenblik}\\

\haiku{hij zag er niet meer,.}{van dan de stofwolk die zij}{achter zich opwierp}\\

\haiku{ik ben nu toch aan,.}{de winnende hand zei hij}{maar tegen zich zelf}\\

\haiku{Toen had de goede;}{Bernard hem met zijn bruine}{oogen lang aangezien}\\

\haiku{Doch het duurde lang,.}{die avond eer de harten tot}{elkander kwamen}\\

\haiku{Toen werd 'et me toch.}{een beetje bar en ik schoof}{m'n stoel achteruit}\\

\haiku{Die beweerde van,;}{hem dat hij rustte in de}{schoot van Delila}\\

\haiku{De dokter kwam, en.}{meende de ouders gerust}{te kunnen stellen}\\

\haiku{Op die zelfde plaat {\textquoteleft}{\textquoteright}.}{heeftDe Vrouw Geertje nog twee}{jaar gevaren}\\

\haiku{de herinnering.}{aan helden en veldslagen}{hindert daar geen mensch}\\

\haiku{De schoonheid van die;}{landstreek is niet vermetel}{en ze is niet schuw}\\

\haiku{Hij had lust, hem in.}{de schouder te vatten en}{overboord te zetten}\\

\haiku{- Och, kreeg Katrien tot, - '.}{bescheid we kennenem}{langer as vandaag}\\

\haiku{- Verbazing, afkeer,.}{ongeloof stonden op de}{gezichten geteekend}\\

\haiku{Ik 'eb er samen '.}{mit de ouwe heer zelf de}{schuit door eenepeuterd}\\

\haiku{Eerder wel 'ad ie ',;}{geen beenen genoegad om de}{stad in te komen}\\

\haiku{Hij keek overboord, en.}{zag de kleine schilfers in}{het water schimmen}\\

\haiku{As die 'n visschien,.}{vangen kan prakkizeert ie}{over z'n werk niet meer}\\

\haiku{As je denken, dat, '.}{ie geld in brengt komt ie mit}{n paar visschies thuis}\\

\haiku{De onrust was wel '.}{heelendal de baas daar in}{dat huis aant Atje}\\

\haiku{Een krampachtige.}{stilte had het rumoer in}{de ban geslagen}\\

\haiku{Louwrens opende de,.... -,.}{mond wilde zich weren Mooie}{thuuskomst begon hij}\\

\haiku{En de vrouw, door dit,.}{zwijgen misleid roddelde}{voort met heete oogen}\\

\haiku{Hij had een gat in.}{het ijs geslagen om aan}{water te komen}\\

\haiku{Dat was geen kwaaie, maar.}{die had met alle schippers}{en vrouwlui schik}\\

\haiku{Jouk had die heele.}{middag een onzekerheid}{over zich  gehad}\\

\haiku{Sporen door het ijs,.}{zaagden ze waar ze met de}{schepen door konden}\\

\haiku{En hij ging met een;}{paar menschen kleine dennen}{uit het bosch halen}\\

\haiku{- We komen door dit,.}{ongeluk te laat bij de}{turf mopperde Jouk}\\

\haiku{En vlugger dan ooit, {\textquoteleft}{\textquoteright}.}{een schuit gelost is kwam toen}{De Vrouw Geertje leeg}\\

\haiku{Je kon  zoo niet, ';}{zeggen wat voor apartigs je}{aanem bespeurden}\\

\haiku{De kleine kajuit,.}{leek een hok van duiven die}{op uitvliegen staan}\\

\haiku{- Ik zou niet weten '.}{waarom de jongen minder}{is asn ander}\\

\haiku{Maar toen hij op de,.}{plecht stond kwam hem een zwart stuk}{leer onder de voet}\\

\haiku{- Maak nou maar dat je,.}{je schepen door de brug krijgt}{zegt m'n grootvader}\\

\haiku{we'ier 'n glas bier van,,.}{zes cent zei Sjoerd stil houdend}{voor een klein caf\'e}\\

\haiku{De olie-lucht van het.}{tentzeil en de kille damp}{van de avond buiten}\\

\haiku{En begon tegen,.}{hem te spreken alsof hij}{een broer van hem was}\\

\haiku{Met verwonderde.}{gezichten drong men te hoop}{om de twistenden}\\

\haiku{- Oordeelt niet, opdat.... -,,}{gij niet geoordeeld wordt Ja}{ja zei Weerselo}\\

\haiku{een zwakke gloor van.}{de gensters over de golven}{sloop er over de grond}\\

\haiku{Een kleine mensch, die.}{door het donker verraden}{en vereenzaamd was}\\

\haiku{Zijn voet zweefde al,.}{boven de plank toen hij zich}{scheen te bedenken}\\

\haiku{de vader en de.}{zoon stonden als vijanden}{tegenover elkaar}\\

\haiku{- Misschien weet ik 'n, ',.}{middel om je teelpen}{zei hij toen langzaam}\\

\haiku{nu, hij is maar een,?}{schipper wat heeft hij in dit}{kantoor willen doen}\\

\haiku{Nog een minuut, en,;}{hij kan weer in de straat staan}{flitst net door hem heen}\\

\haiku{het licht, de loopers,,....}{de linnen zak en hij stort}{naar de deur terug}\\

\haiku{het ligt achter de,.}{deur die hij zooeven niet open}{heeft kunnen krijgen}\\

\haiku{Hij is naar het schip {\textquoteleft}{\textquoteright},.}{deVrouw Geertje geloopen}{om Abel te spreken}\\

\haiku{En.... bewoog er zich?}{een licht achter het venster}{van het bovenraam}\\

\haiku{Sluipen zoo menschen,?}{die een eerlijke zaak te}{verrichten hebben}\\

\haiku{Het kleine kille,,;}{vlekje dat in de iris glimt}{gaat grooter worden}\\

\haiku{Wat een vreemd, angstig.}{geklepper van vleugels was}{dat in het vertrek}\\

\haiku{- Hoe komt het, dat Abel?}{op dit oogenblik deze}{woorden invallen}\\

\haiku{Zoo, als Jouk daar zijn,:}{vader op de rug ziet vliegt}{hem een onrust aan}\\

\haiku{- Dus op die manier ' ' '?}{ad je anet geld voorn}{schip willen komen}\\

\haiku{- Nee, maar ik zou hypteek,.}{op de praam kunnen nemen}{antwoordde toen Abel}\\

\haiku{Enfin, we raken,.}{de kolk uit en op eens is}{m'n knecht verdwenen}\\

\haiku{Er was nu niets geen,.}{rood meer in de streep die over}{de horizon liep}\\

\haiku{Er stond een emmer,, '....}{water een fornuispot en}{eenalve pot brei}\\

\haiku{- Wel, - zei Louwrens, - en,?}{je stond er bij of je de}{ruzie niet aanging}\\

\section{Noud van den Eerenbeemt}

\subsection{Uit: De berenkuil}

\haiku{Als ik jou was, zou ' '.}{ik maarnsn oogje op}{m'n dochter houden}\\

\haiku{Was dat een reden?}{om Toke te verbieden}{met hem uit te gaan}\\

\haiku{{\textquoteright} Hij keek naar Greets bed,.}{dat tegenover dat van Ria}{onder het raam stond}\\

\haiku{Ze griste het uit.}{zijn handen en wikkelde}{het in het papier}\\

\haiku{{\textquoteright} {\textquoteleft}Nou, daar moeten we,{\textquoteright}.}{eerst nog eens over praten zei}{Pierre geschrokken}\\

\haiku{Pierre, trek gauw een...{\textquoteright} {\textquoteleft},{\textquoteright}.}{jas aan en ga kijken of}{jeOnzin zei hij}\\

\haiku{{\textquoteleft}Toen ik vanmiddag,.}{bij Olivier was heb ik daar}{Helga gesproken}\\

\haiku{Het appelboompje.}{in een hoek van het kleine}{grasveld stond in knop}\\

\haiku{Hij glimlachte, nam.}{een paar slokken en wendde}{zich weer tot Pierre}\\

\haiku{Op school...{\textquoteright} {\textquoteleft}Natuurlijk!}{wordt de school er weer met de}{haren bij gesleept}\\

\haiku{{\textquoteleft}Hoor 'ns mens, er is ' '.}{n tijd van komen en er}{isn tijd van gaan}\\

\haiku{{\textquoteleft}... ziet er voor ons ook,...{\textquoteright} {\textquoteleft}......{\textquoteright} {\textquoteleft}...}{niet zo best uit vader veel}{minder verdienen}\\

\haiku{Toen ze op de top,.}{waren bleven zij staan en}{keken om zich heen}\\

\haiku{{\textquoteleft}Luistert u ook al,?}{naar de kletspraatjes die je}{hier in de buurt hoort}\\

\haiku{De dokter nam zijn.}{sigaar uit zijn mond en blies}{langzaam de rook uit}\\

\haiku{{\textquoteleft}Ik ken nog een paar.}{jongens als Frans en het is}{overal hetzelfde}\\

\haiku{{\textquoteright} {\textquoteleft}En ik dacht, dat je...{\textquoteright} {\textquoteleft}...{\textquoteright}}{met die jongen van Olivier}{in de stad zouAch}\\

\haiku{Toch zou het leuk zijn!}{geweest om morgen in de}{zaak te vertellen}\\

\haiku{n Duits type met.}{zo'n brede mond en van die}{opgemaakte ogen}\\

\haiku{Geen papieren, geen... ',.}{vergunningent Land van}{de toekomst Pierre}\\

\haiku{n Instorting...{\textquoteright} {\textquoteleft}We,.}{dachten dat het ergens in}{de Oude Man was}\\

\haiku{{\textquoteleft}Het was precies aan,.}{de andere kant op de}{vijfhonderdveertig}\\

\haiku{Hij reed de hoek om.}{en zag Pie voor de winkel}{van zijn ouders staan}\\

\haiku{Hij trok langzaam zijn.}{leren jekker uit en hing}{hem over een  stoel}\\

\haiku{Je zorgt er voor, dat...{\textquoteright}}{je binnen een maand aan het}{werk bent of anders}\\

\haiku{je iedere schicht.}{zes meter verder in het}{kolenveld werken}\\

\haiku{Dacht je, dat het mij,?!}{lag dag in dag uit onder}{de grond te zitten}\\

\haiku{Pierre voelde, dat.}{er van beide kanten niets}{meer te zeggen viel}\\

\haiku{Gevoelens, die uit,.}{een ver onbegrepen land}{diep in hem stamden}\\

\haiku{Het is verdomme...{\textquoteright}}{elf uur en jij ligt nog langs}{de straat te slieren}\\

\haiku{Je vader heeft me,.}{verteld dat jij alles van}{de provo's afweet}\\

\haiku{In het bijzonder,.}{de toekomst van uw bedrijf}{mag ik wel zeggen}\\

\haiku{Lange tijd was het.}{de enige winkel in de}{omgeving geweest}\\

\haiku{Haar handen lagen.}{in haar schoot en de vingers}{omknelden elkaar}\\

\haiku{{\textquoteright} zei hij, {\textquoteleft}maar je kunt.}{je het beste maar in het}{gebeurde schikken}\\

\haiku{{\textquoteleft}Laten we eerst de...}{uitslag van de kikkerproef}{maar eens afwachten}\\

\haiku{{\textquoteleft}U begrijpt wel, dat.}{die snelheid met de grootste}{zorg is uitgekiend}\\

\haiku{{\textquoteright} Toen hij weer in de,.}{kamer kwam stond er een kop}{koffie voor hem klaar}\\

\haiku{Allerlei dromen,,...}{die zij voor de toekomst had}{gaan nu in rook op}\\

\haiku{Haar vingers streken,.}{even langs zijn gezicht tastten}{toen naar zijn handen}\\

\haiku{Kun je hem nu niet,?}{eens zo ver krijgen dat hij}{naar de kapper gaat}\\

\haiku{En waarom moet hij?}{eeuwig en altijd in een}{spijkerbroek lopen}\\

\haiku{{\textquoteleft}We zijn naar Maaseik.}{gereden en hebben daar}{wat rond gelopen}\\

\haiku{{\textquoteleft}Het was net als al,{\textquoteright}.}{die andere Belgische}{caf\'es zei Frans}\\

\haiku{Hij stond rechtop, zijn.}{handen in zijn zakken en}{knikte bedachtzaam}\\

\haiku{{\textquoteleft}Word niet kwaad op 'r... ' -!}{t Is niet allemaal haar}{schuld vergeet dat niet}\\

\haiku{{\textquoteleft}Als hij daar niet was,!}{geweest zouden we nog niet}{weten waar ze zit}\\

\haiku{Tot het kind er was,!}{kon ze beter maar net doen}{alsof ze gek was}\\

\haiku{We zouden naar de...{\textquoteright}}{film kunnen gaan of misschien}{ergens iets drinken}\\

\haiku{{\textquoteleft}Ik moet zeggen, dat!}{je me wel de stuipen op}{het lijf hebt gejaagd}\\

\haiku{M'n vader en m'n.}{grootvader hebben hier in}{de winkel gestaan}\\

\haiku{Ze zullen je een.}{voor een in de steek laten}{en naar ons komen}\\

\haiku{Daar komt bij, dat je}{vrouw en jij ook niet meer van}{de jongsten zijn en}\\

\haiku{Aan onze kant staan.}{specialisten klaar om}{aan het werk te gaan}\\

\haiku{Hij besefte zelf,:}{wel dat zoiets niet haalbaar}{was en zei haastig}\\

\haiku{Maar het is nog maar '.}{n heel klein kiertje en het}{zal niet lang duren}\\

\haiku{Ze besefte, dat '.}{hetn strohalm was waar ze}{zich aan vastklampte}\\

\haiku{Greet is direct na...{\textquoteright} {\textquoteleft}?}{het eten weggegaan naar de}{InstuifEn Toke}\\

\haiku{Misschien voelde die '.}{zich eenzaam en kwam hijn}{kop koffie halen}\\

\haiku{Hij loopt echt niet in,!}{zeven sloten tegelijk}{mevrouw Vasterman}\\

\haiku{{\textquoteright} In het schenken van.}{troost was Charles Cleophas}{nooit erg goed geweest}\\

\haiku{Ze was naar dokter.}{Tijsen geweest en had wat}{zitten napraten}\\

\haiku{Weer was er de spijt,...}{dat ze het niet ongedaan}{had kunnen maken}\\

\haiku{Ze schrok er zelf van,.}{durfde in het begin niet}{verder te denken}\\

\haiku{Tot een huwelijk;}{tussen Toke en Pie zou}{het wel nooit komen}\\

\haiku{Ze liet zich van haar,.}{kruk glijden liep naar het raam}{en bleef ervoor staan}\\

\haiku{Het had in lang niet.}{geregend en het zand was}{droog en fijn als stof}\\

\haiku{{\textquoteleft}Bij 'n hond of 'n,.}{kat weet je altijd nog wel}{wat-ie bedoelt}\\

\haiku{De zon weerkaatste.}{schitterend in een open plek}{vlakbij de oever}\\

\haiku{Dokter Tijsen zat.}{op de rand van het bed zacht}{met Giel te praten}\\

\haiku{{\textquoteright} zei hij en het was,.}{duidelijk dat  hij het}{tegen zich zelf had}\\

\haiku{En toen had op een.}{avond meneer De Wever voor}{de deur  gestaan}\\

\haiku{{\textquoteright} {\textquoteleft}Hij vindt, dat hij 't,{\textquoteright}.}{ergens anders beter heeft}{dan thuis zei Pierre}\\

\haiku{{\textquoteright} Mia hoorde zelf hoe.}{schutterig die woorden over}{haar lippen kwamen}\\

\haiku{Waarom heb je me,?}{nooit gezegd dat je er met}{Pie over hebt gepraat}\\

\haiku{Eerst wilde ik het, '.}{hem niet zeggen maar later}{heb ikt verteld}\\

\haiku{{\textquoteright} Mia probeerde haar.}{eigen ongerustheid weg}{te redeneren}\\

\haiku{Die mocht eens denken,!}{dat Pierre al van alles}{op de hoogte was}\\

\haiku{Ze keek hem vragend,.}{aan herinnerde zich het}{gesprek met Toke}\\

\haiku{{\textquoteright} De dokter wachtte,.}{tot zij zelf was gaan zitten}{nam toen ook een stoel}\\

\haiku{Dan ben ik dus niet!}{het enige lid van de Mia}{Vasterman Fanclub}\\

\haiku{{\textquoteright} Charles Cleophas,.}{keek naar de doeken die aan}{de muren hingen}\\

\haiku{{\textquoteleft}'n Paar staan al op, '}{onze lijst zie ik maar dat}{zoekt de juffrouw op}\\

\haiku{Het was duidelijk,.}{dat het Berk absoluut niet}{interesseerde}\\

\haiku{Hij droeg een witte,.}{spijkerbroek en een oude}{haveloze trui}\\

\haiku{Twee dagen zou 't, ' '...}{goed gaan maar dan ist weer}{t ouwe liedje}\\

\haiku{Vertel ze, dat je '.}{me gesproken hebt en dat}{t best met me gaat}\\

\haiku{Ik ben gelukkig...{\textquoteright} {\textquoteleft} -!}{nog niet te oud omTrouwen}{daar komt niks van in}\\

\haiku{{\textquoteleft}Jullie gaan wanneer, ',}{je uit school komt meteenn}{nachtje logeren}\\

\haiku{{\textquotedblleft}'t Hele leger.}{van Napoleon is zo}{op de been gebracht}\\

\haiku{De televisie.}{bleef uit omdat Toke er}{misschien last van had}\\

\haiku{Als we de dokter,.}{niet gauw waarschuwen hoeft hij}{niet meer te komen}\\

\haiku{Opeens doken er.}{recht voor hem twee verblindend}{witte lichten op}\\

\haiku{Toen haar vader de,.}{kamer uitging gleed er een}{traan over haar wangen}\\

\haiku{Gisteravond toen hij...{\textquoteright} {\textquoteleft}!}{naar ons toe wilde komen}{Godallemachtig}\\

\section{Homme Eernstma}

\subsection{Uit: Liefdedood}

\haiku{Zij had juist dat, wat,.}{hij miste maar in wezen}{toch wel in zich had}\\

\haiku{Gedachteloos liep.}{hij als vanzelfsprekend het}{pad naar de tuin af}\\

\haiku{{\textquoteright} {\textquoteleft}Mijn moeder heeft een,{\textquoteright}.}{staart was het afwezige}{antwoord van Wibe}\\

\haiku{Daarnaast lagen twee.}{grote vijgenbladen bij}{wijze van bordjes}\\

\haiku{Tegelijk greep hij.}{een handvol frambozen uit}{de bloempot naast hem}\\

\haiku{Hij gaf de eerste,.}{portie aan Eabeltsje om}{haar te kalmeren}\\

\haiku{Zij greep zijn piemel.}{en probeerde die bij haar}{binnen te werken}\\

\haiku{Bij haar zesde maand.}{rende ze van onder de}{vijgenboom vandaan}\\

\haiku{Als Wibe wilde,.}{kon hij een rijtuig huren}{en er heen rijden}\\

\haiku{Plechtig legde hij.}{de boeken op tafel naast}{het bord van Wibe}\\

\haiku{Onze Wibe wordt,.}{een kleine Darwin en trouwt}{met zijn achternicht}\\

\haiku{Ten slotte zette:}{hij ze met een wijze raad}{op het goede spoor}\\

\haiku{Haastig liep ze naar.}{de keuken terug om het}{nieuws te vertellen}\\

\haiku{probeerde ze de,.}{muziek uit te blazen als}{een brandende kaars}\\

\haiku{Tegen zijn gezicht.}{voelde en rook hij Elises}{bezwete schaamhaar}\\

\haiku{Hij mengde het met.}{een teugje van zijn jonge}{rode bourgogne}\\

\haiku{Meestal hebben.}{vrouwen van sluiswachters het}{vrij wat naar hun zin}\\

\haiku{Het was op de avond.}{van de ineenstorting van}{de New Yorkse beurs}\\

\haiku{Het is weer precies.}{hetzelfde als met die tram}{en de melkfabriek}\\

\haiku{Het was hem op dat.}{ogenblik ontgaan dat hij nog}{commissaris was}\\

\haiku{Heilige Maria,,{\textellipsis}{\textquoteright}.}{Moeder van God bid voor ons}{Verder kwam ze niet}\\

\haiku{De zuster kwam naar.}{beneden om te zeggen}{dat die weer dicht moesten}\\

\haiku{Aan de vaart viel de.}{ophaalbrug bij de sluis dicht}{met een doffe dreun}\\

\haiku{Op die manier stijgt.}{de roman in po\"ezie}{boven zichzelf uit}\\

\section{Justus van Effen}

\subsection{Uit: Brief van een bejaard man en Reis naar Zweden}

\haiku{Maar hoezeer ze mijn,.}{trots ook pijnigden temmen}{deden ze haar niet}\\

\haiku{Ik vorderde zo}{geweldig in het leren}{dat ik weldra v\'o\'or}\\

\haiku{Zij beantwoordt ze;}{in een taal die ver boven}{haar leeftijd uitgaat}\\

\haiku{naar om mij neer te.}{leggen bij de machtspreuken}{van mijn leermeesters}\\

\haiku{Ik leverde er.}{meer op twee bladzijden dan}{zij in een heel boek}\\

\haiku{Ik dacht dat de hand.}{van mijn godinnetje me}{genoeg had gezegd}\\

\haiku{Ik meende zelfs te.}{zien dat ze mij nog verder}{in haar gunst brachten}\\

\haiku{Mijn eerste bezoek,.}{viel me niet tegen hoewel}{het erg lang duurde}\\

\haiku{Zij hoeft hem alleen.}{maar te laten merken dat}{zij smaak in hem vindt}\\

\haiku{Zij begon niet te.}{schreeuwen of als een viswijf}{van zich af te slaan}\\

\haiku{Zij verdween met de.}{zwakke oorzaken die haar}{hadden voortgebracht}\\

\haiku{Maar het kostte haar.}{veel meer moeite om zich van}{mij los te maken}\\

\haiku{Ze vallen meteen.}{op als men niet aan deze}{bouwtrant gewend is}\\

\haiku{Zolang alles in:}{het Italiaans gebeurt kan}{het er nog mee door}\\

\haiku{Zij schenen blij met.}{zijn ongeluk en trots op}{zijn vernedering}\\

\haiku{Hun lijf was mager,,.}{maar gespierd hun gang krachtig}{hun houding kaarsrecht}\\

\haiku{Ik ben er trouwens.}{ook geen smeden of barbiers}{tegengekomen}\\

\haiku{We kregen dikwijls.}{kinderen van elf \`a twaalf}{jaar als postiljon}\\

\haiku{Er zijn hier kroegen,.}{noch herbergen behalve}{dan in de steden}\\

\haiku{Heel vaak gaat het hier.}{om ondeugden onder het}{mom van hero{\"\i}ek}\\

\haiku{Het schip droeg alle.}{sporen van de pracht van de}{Engelse natie}\\

\haiku{Daarna trokken zij.}{elkaar de manchetten en}{dassen van het lijf}\\

\haiku{Toen de koningin,.}{zag dat hij niets at vroeg ze}{hem naar de reden}\\

\haiku{Men dacht daarom dat,.}{hij ziek was wat weer nieuwe}{vragen uitlokte}\\

\haiku{Zelfs zijn geringste.}{bedienden ontbrak het niet}{aan goud en zilver}\\

\haiku{{\textquoteleft}Ik hoef van niemand,,{\textquoteright}.}{mijn plicht te leren mijnheer}{antwoordde de Deen}\\

\haiku{R.G. voegt aan het eind.}{van zijn vertaling ook enig}{commentaar toe}\\

\haiku{Horatius, Brief (),.}{aan de PisonenDe arte}{poetica 412}\\

\section{Marcellus Emants}

\subsection{Uit: Inwijding}

\haiku{De enige raad van, ':}{oom diet hem moeilik viel}{op te volgen was}\\

\haiku{Als ze je vragen ',.}{inn bestuur te komen}{dan neem je dat aan}\\

\haiku{{\textquoteright} {\textquoteleft}Mag ik je dan zeer,,.}{danken Gertrude voor je}{keurige dienee}\\

\haiku{Als-t-ie nou in,.}{de toekomst ook maar op je}{rekene mag h\`e}\\

\haiku{Als de mensen eens,!}{wisten hoe pedant hij zich}{van avond wel voelde}\\

\haiku{letterlik schuw te....}{worde voor de omgang met}{fatsoenlike lui}\\

\haiku{'t Was, of de ogen.}{van die agent hem dwongen in}{de richting naar huis}\\

\haiku{Overal grote en;}{kleine schilderijen in}{vergulde lijsten}\\

\haiku{in de week was 'k.... ' '....}{in Utrechtn enkele keer}{ooks Zaterdags}\\

\haiku{Nieuwsgierigheid naar;}{haar verleden welde even}{in Theodoor op}\\

\haiku{{\textquoteright} Hij beloofde 't,.}{kuste haar op de handrug}{en wilde heengaan}\\

\haiku{Het antwoord van de.}{Griffier van de Rechtbank had}{hij maar half vertrouwd}\\

\haiku{Die hebbe gelijk{\textquoteright} {\textquoteleft} '{\textquoteright}.}{prevelden zijn lippendag}{wordtt toch niet meer}\\

\haiku{Jij bleef in Utrecht.... en........, '....}{en de meisjes alleen en}{ikn ouwe vrouw}\\

\haiku{van goeie familie....}{en die nog wat te wachte}{heeft van z'n tante}\\

\haiku{een kleur, die haar slecht.}{kleedde en die vloekte met}{de omgeving}\\

\haiku{Duurt 't je al te,,.}{lang snij dan maar uit terwijl}{die hier boven is}\\

\haiku{En wie weet, of de!}{lummel op dit ogenblik niet}{door haar werd gezoend}\\

\haiku{Wie weet, of je niet?}{veel meer van me verwacht dan}{ik betale kan}\\

\haiku{'k zal 't voortaan.}{ook wel doen en jij zal d'r}{geen last van hebbe}\\

\haiku{Hij meende, dat zijn.}{ernst werd miskend en wilde}{haar dit doen inzien}\\

\haiku{{\textquoteright} 't Was hem dan ook, '.}{te moede oft in hem}{schaterde van pret}\\

\haiku{Met vrouwelike:}{intu{\"\i}tievieteit riep}{zij eindelik uit}\\

\haiku{Toch wil ik wel 'ns;}{met je klinken en mag je}{me gelukwense}\\

\haiku{zolang ik zelf me,,.}{niet van onze borrel dat}{is de wijn onthou}\\

\haiku{Over mijn iedeaal '!}{vann man heb ik geen plan}{me uit te late}\\

\haiku{Toen het dienee was,}{afgelopen vroeg mevrouw}{van Onderwaarden}\\

\haiku{{\textquoteright} {\textquoteleft}Waarom heb ie me,?}{niet vooruit gezeid dat je}{misschien komme zou}\\

\haiku{houwe, die je elk....}{ogenblik in je ellende}{kan late stikke}\\

\haiku{'n Vrouw as ik mot ' '.}{n steen in d'r lijf hebben}{in plaats vann hart}\\

\haiku{Soms kon dit hem niets,.}{schelen vond hij die leukheid}{zelfs verkieselik}\\

\haiku{Verliefde lui - hij ' -.}{wistt immers luisteren}{nooit naar goede raad}\\

\haiku{{\textquoteright} Theodoor hield wel.}{eens graag een babbeltje met}{zijn jongste zuster}\\

\haiku{{\textquoteleft}Als 'k verstandig, '.}{kon zijn liet ik me int}{geheel niet neme}\\

\haiku{'t Is wat fijns!... Gaan?}{wij ooit met jonges om als}{met kamerade}\\

\haiku{tot an d'r dood en.}{dan alleen op kamers te}{moge gaan zitte}\\

\haiku{Ik zal me schame,.}{zoveel je verlangt als jij}{me dan ook maar helpt}\\

\haiku{kennelik had ze.}{weinig vertrouwen in dat}{voorspelde genot}\\

\haiku{{\textquoteright} {\textquoteleft}Zijn wij nog altijd?}{te slecht om ies van de goeie}{werke te hore}\\

\haiku{ik raai 't al. U,.}{mot uw geheime beter}{wegsluite moesje}\\

\haiku{want die gode en,,.}{die helde dat benne toch}{mooie ouwe dinge}\\

\haiku{Het trof Theodoor,,.}{dat ze keurig zuiver de}{melodie weergaf}\\

\haiku{{\textquoteright} Zulk soort klachten vond.}{Theodoor hoe langer hoe}{onaangenamer}\\

\haiku{{\textquoteright} Sinds een paar weken {\textquoteleft}{\textquoteright}.}{had zij de benamingschat}{voor hem ingesteld}\\

\haiku{vindt dit verstand dan.}{nu voor Willie elke man}{beter dan geen man}\\

\haiku{{\textquoteright} Drie dagen later:}{galmde Theodoor bij het}{aantafel-gaan}\\

\haiku{Tot de dokter of.}{de luitenant richtte hij}{hoogst zelden het woord}\\

\haiku{Het vond bij mevrouw.}{van Onderwaarden een zeer}{ongunstig onthaal}\\

\haiku{en zoende haar ogen,,,.}{haar mond het kuiltje in de}{ronde blanke hals}\\

\haiku{en daarom heb ie,.}{al die dage nie na me}{omgekeke h\`e}\\

\haiku{Zeg 't, h\`e, zeg 't. '....}{Dan gak weer na die man}{toe en weer en weer}\\

\haiku{{\textquoteright} Nog dreigend, maar nu:}{smekend tegelijkertijd}{gaat zij doffer voort}\\

\haiku{maar Theodoor hield.}{haar hand terug en schoof de}{kaarten op een hoop}\\

\haiku{als je ooit weer met,,.}{die nonsens aankomt dan word}{ik boos ernstig boos}\\

\haiku{Zo hoort 't.{\textquoteright} Daar kon.}{Theodoor onmogelik}{ernstig bij blijven}\\

\haiku{As 'k uit ete ga, ' '.}{zalk m'n ouwe blauwe}{nog wels andoen}\\

\haiku{Andere vrouwe.}{nisse altoos veel mooier}{gekleed dan Tonia}\\

\haiku{Op eens werd 't hem,;}{niet alleen klaar dat hij zich}{vergaloppeerd had}\\

\haiku{Ik ben d'r Theo heel,.}{dankbaar voor dat hij me d'r}{niet in laat lope}\\

\haiku{En als-t-ie zich,....}{eerst vrij heeft gemaakt v\'o\'or dat}{hij jou kwam vrage}\\

\haiku{Ik zal heel graag wat,;}{van u aannemen meneer}{van Onderwaarden}\\

\haiku{wat er zo van uw,........}{mama gezegd wordt is niet}{geheel zonder grond}\\

\haiku{Was ieder mens z\'o,?}{veranderlik z\'o weinig}{zeker van zich zelf}\\

\haiku{{\textquoteright} Stemmen galmden aan,;}{uit het kreupelhout waardoor}{hij gekomen was}\\

\haiku{As je zo opstuift, '.}{maak ie me maar bang en kan}{k toch niks zegge}\\

\haiku{Al wat ik verlang,.}{is behandeld te worden}{als fatsoenlik man}\\

\haiku{{\textquoteright} Even hield zij haar hand;}{met uitgespreide vingers}{zich weer voor de ogen}\\

\haiku{maar Theodoor wist.}{nog niet wat redeneren}{is met je gevoel}\\

\haiku{met de nadruk van:}{een verbitterde viel hij}{haar in de rede}\\

\haiku{Bedarend streek hij}{over het krullende haar en}{zonder dat hij wist}\\

\haiku{an 'n andere.}{late zien en lache ze}{me dan samen uit}\\

\haiku{{\textquoteright} Onder die wilde;}{woordenwarreling luwde}{Theodoors woede}\\

\haiku{{\textquoteright} Even glom de hoop in,.}{hem op dat hij haar nu zou}{kunnen overtuigen}\\

\haiku{kom weerom en je.}{zult geen halven dag op me}{hoeven te wachten}\\

\haiku{{\textquoteleft}'k Heb je maar nie;}{met m'n ongerustheid}{lastig gevalle}\\

\haiku{anders wenst ie je.}{naar de drommel en daar heeft}{ie gelijk in ook}\\

\haiku{op 'n bolstaande....',,:}{slip en d ouwe man die}{al bij God was roept}\\

\haiku{En 't kwam hem voor,.}{dat er voor dit huwelik}{veel te zeggen viel}\\

\haiku{O, die mode, wat!}{een mensonterend dwangbuis}{was dat in haar ogen}\\

\haiku{je luistert, welke,....}{amuzemente je bezoekt}{welke taal je spreekt}\\

\haiku{al is dat nieuwe....}{ook nog zo gemakkelik}{tot stand te brenge}\\

\haiku{hoe 't zou moete, '....;}{wezen en wat hij zou doen}{als-t-ie maarns}\\

\haiku{Is 'n argeloos?}{mens dan alleen bestemd om}{geplukt te worde}\\

\haiku{De aanstellerij '.}{zit bij de van Ouderhoorns}{immers int bloed}\\

\haiku{{\textquoteright} Nu lei zijn moeder:}{haar zwaardooraderde hand}{op zijn arm en zei}\\

\haiku{al herinnerde....}{hij zich de ellende met}{Tonia uitgestaan}\\

\haiku{n paar mooie dasse,:}{weg te gooie toen je zonder}{enige rede zei}\\

\haiku{Ze speelt met u als '....}{de kat met de muis enn}{betrekking zoeke}\\

\haiku{{\textquoteright} 't Was Theodoor,.}{of eensklaps zijn bloed in al}{zijn aderen stolde}\\

\haiku{'t Zit d'r aldoor....}{maar in d'r buik en harde}{koors hei ze gehad}\\

\haiku{die lange, lange, '....}{tijd toe je nooits na me}{ben komme kijke}\\

\haiku{{\textquoteright} Tegen de laatste.}{woorden was Theodoor weer}{niet opgewassen}\\

\haiku{We mogen ons niet,.}{zwak tonen niet maar door dik}{en dun toegeven}\\

\haiku{De opienies van,,.}{de mensen heb je nodig}{m'n vrind hoog nodig}\\

\haiku{- Weet je ook niet, dat....}{van Harmelen geen griffier}{is geworden en}\\

\haiku{Toch spande hij al.}{zijn krachten in om zich tot}{kalmte te dwingen}\\

\haiku{n dief, omdat ik ',....}{niet alsn monnik leef en}{wie weet of-t-ie}\\

\haiku{wat ik aan d' \'e\'en,';}{geef dat mot ik ook an d}{andere geve}\\

\haiku{'t Was al heel mooi, '.}{dat zijt uithield in dat}{ellendige gat}\\

\haiku{{\textquoteright} En met voorgewend:}{hooghartige kalmte liet}{hij er op volgen}\\

\haiku{Je mot 'm in toom,,.}{houwe zie je en nou en}{dan verbetere}\\

\haiku{maar dat je iemand, '....}{neme kan die je zelfn}{kwiebus hebt genoemd}\\

\haiku{Je hoeft de mense.}{toch overal d'r neus niet in}{te late steke}\\

\haiku{Twintig mienute....}{lang heeft ie van morge met}{me zitte prate}\\

\haiku{Onder het dooreten....}{nadenkend zag hij Dora}{voor zich oprijzen}\\

\haiku{maar ik maakte 't.}{de mense moeilik ies van}{me an te neme}\\

\haiku{Wat ik nou heb, zal.}{ik me hele leve lang}{wel motte houwe}\\

\haiku{'k Wou ze juist op;}{de muziekschool doen en dat}{kost nog al veel geld}\\

\haiku{maar u vraagt er naar,,.}{alsof u meent dat ik niets}{anders te doen heb}\\

\haiku{Al zijn grieven zou;}{hij kunnen uitspreken en}{toch prakties blijven}\\

\haiku{{\textquoteleft}Krijg ik antwoord of........}{ben je van plan stommetje}{te blijve spele}\\

\haiku{{\textquoteright} 't Was, of een zwart.}{glanzende straal Tonia's}{ogen ontbliksemde}\\

\haiku{Daarom huichelde,:}{hij nieuwe opwinding nog}{eenmaal opstuivend}\\

\haiku{Ze hebben mekaar.}{de verschrikkelikste}{dinge verwete}\\

\haiku{Plots voelde hij als:}{vroeger behoefte om zich}{stil af te vragen}\\

\haiku{Klein tegen Dorskamp,:}{en van zijn confr\`eres links}{die respondeerden}\\

\haiku{Nu ging hij na, op.}{welke wijze een werkman}{aan de fabriek komt}\\

\haiku{hij wist, dat zijn stem,.}{onzeker zou klinken en}{mat misschien wel schor}\\

\haiku{Misbruik-make.}{van hun poziesie doen die}{bazen allemaal}\\

\haiku{De slijtende tijd.}{zou zijn daad uitwissen in}{ieders gedachten}\\

\haiku{daardoor bewijs je,.}{ten minste voor je zelf dat}{je verstandig bent}\\

\haiku{Maar  oom had de.}{fles gegrepen en wilde}{hem wijn inschenken}\\

\haiku{ouwe koeien ben;}{ik niet voornemens  uit}{de sloot te halen}\\

\haiku{in huis was 't maar, '',....}{trap op trap af en ask}{d'r wa van zee dan}\\

\haiku{{\textquoteright} Een afschuw van zich.}{zelf barstte naar buiten in}{een walgend geluid}\\

\haiku{ze tokkelde zo '....}{graagn operawijsje op}{die ouwe toetse}\\

\haiku{dan loopt de derde,,.}{die geen partij achter zich}{heeft met de kluif weg}\\

\subsection{Uit: Jong Holland}

\haiku{Reeds vele jaren.}{zijn sedert de conceptie}{er van verlopen}\\

\haiku{- Dit is het tweede,.}{gelui het derde zal niet}{lang meer uitblijven}\\

\haiku{Aan jou alleen durf.}{ik de opvoeding van dat}{kind toevertrouwen}\\

\haiku{zij is gestorven,,.}{door mijn ruwheid mijn koelheid}{mijn nuchter verstand}\\

\haiku{- Wie weet of hij niet!}{juist van pas komt in deze}{wonderlijke tijd}\\

\haiku{Ons past het op het,.}{goede te letten en hij}{had veel zeer veel goeds}\\

\haiku{van het bord rees hij.}{weder naar het aangezicht}{van zijn oom omhoog}\\

\haiku{Ons past het op het,,.}{goede te letten en hij}{had veel zeer veel goeds}\\

\haiku{Evenals Pietekoo,}{het woord steeds aan Eveline}{liet gunde zij haar}\\

\haiku{De vrienden van den.}{huize dachten er niet aan}{haar te naderen}\\

\haiku{De bediende had,.}{stoelen aangeschoven het}{gezelschap nam plaats}\\

\haiku{{\textquoteright} Toch gevoelde zij.}{er meer bij dan de zusters}{konden beseffen}\\

\haiku{{\textquoteleft}Ja{\textquoteright} en wierpen dan,,.}{al breiend en bordurend}{blikken naar buiten}\\

\haiku{dat ik uw broer voor ',,{\textquoteright}.}{t laatst ontmoet heb dames}{ging Van Dijck voort}\\

\haiku{Evelines eerste;}{bemerking had dit getal}{van drie gegolden}\\

\haiku{Met een stem, welke,:}{van aandoening trilde riep}{hij plotseling uit}\\

\haiku{Een tik op de deur,.}{stoorde zijn overpeinzingen}{Gijsbrecht stond voor hem}\\

\haiku{{\textquoteright} Met de knop in de,:}{hand keerde hij zich evenwel}{weder om en vroeg}\\

\haiku{{\textquoteleft}Ik zal het met je,,;}{proberen misschien een half}{jaar misschien een jaar}\\

\haiku{{\textquoteleft}Anderen zouden,...}{de vruchten plukken wij voor}{niets hebben gewerkt}\\

\haiku{{\textquoteright} Sleek liep vluchtig de,:}{wijzigingen door hief toen}{het hoofd op en vroeg}\\

\haiku{{\textquoteright} Momstra echter had er,:}{blijkbaar niets geen lust meer in}{en galmde geeuwend}\\

\haiku{{\textquoteright} Van Dijck bromde.}{iets onverstaanbaars zonder}{de ogen op te slaan}\\

\haiku{{\textquoteleft}In 's hemels naam!}{geen letters eten na zulk een}{uitmuntend diner}\\

\haiku{Een zachtzinnige.}{aard verloochent zich wel eens}{na een goed diner}\\

\haiku{Gijsbrecht zorgde er.}{evenwel voor dat de honing}{niet al te zoet werd}\\

\haiku{Was het niet schoon door,?}{een dichter door een genie}{bemind te worden}\\

\haiku{-- Was hij dan in haar -?}{ogen even als in die van zijn}{broeder maar een kind}\\

\haiku{Ook had hij er in}{den beginne niets tegen}{gehad aan enigen}\\

\haiku{{\textquoteleft}Zodra er vier of,;}{vijf leden tegenwoordig}{zijn drinken wij thee}\\

\haiku{{\textquoteleft}Kan je waarachtig{\textquoteright}.}{nog spreken vroeg Gijsbrecht op}{minachtende toon}\\

\haiku{De eerste spellen,.}{wonnen zij zonder moeite}{en dit gaf Frits moed}\\

\haiku{Allen, Reelijn niet,.}{uitgezonderd keken met}{belangstelling op}\\

\haiku{Ah, jeune homme,!}{quelle triste vieillesse}{vous vous pr\'eparez}\\

\haiku{hert. Zolang ik geen.}{besluit genomen heb gaat}{het mij altijd zo}\\

\haiku{Hoe dikwijls moet ik?}{je zeggen dat ik mij niet}{met vrouwen ophoud}\\

\haiku{Daarna keerde hij.}{naar het plein terug en ging}{zijn woning binnen}\\

\haiku{{\textquoteleft}Zo vroeg al op, jij,?}{die anders om tien ure nog}{in de veren ligt}\\

\haiku{De kuur beperkte.}{het thans tot een glas melk met}{enige beschuiten}\\

\haiku{Uitgifte van 100,, -:}{000 Aandelen elk groot f}{10  directeur}\\

\haiku{{\textquoteright} {\textquoteleft}Frederik is er.}{zo onschuldig aan als een}{pasgeboren duif}\\

\haiku{{\textquoteright} Een donkere gloed,.}{liep over Mathildes wangen}{maar verdween spoedig}\\

\haiku{Haar verlegenheid,:}{was evenwel nog merkbaar toen}{zij ten antwoord gaf}\\

\haiku{maar, weet u, als hij,...}{de vorige avond veel wijn}{had gedronken dan}\\

\haiku{{\textquoteright} riep Emile op zijn,.}{gewone gemeenzame}{toon de bankier toe}\\

\haiku{Je weet, ik stelde,.}{vroeger grote prijs op zijn}{scherp geoefend oog}\\

\haiku{{\textquoteleft}Heb nu de goedheid{\textquoteright}}{hier plaats te nemen tussen}{mijn oudste en mij}\\

\haiku{Voor mij is en blijft,.}{het een heilige zaak de}{hoeksteen der moraal}\\

\haiku{De veertien jaren,.}{hadden haar wel iets gebracht}{wel veel ontnomen}\\

\haiku{{\textquoteright} {\textquoteleft}En hoe zijn de twee,?}{pupillen die Henri tot}{zich genomen heeft}\\

\haiku{Hij had eerst nu de;}{volle kracht bereikt van het}{mannelijk leven}\\

\haiku{zij hield echter zijn,.}{blik uit en hij moest ditmaal}{zelf de ogen neerslaan}\\

\haiku{Ik kan haar toch niet!}{met die oudgediende aan}{\'e\'en tafel plaatsen}\\

\haiku{een fijne fles zal,?}{beter smaken na de reis}{wat dunkt Johan er van}\\

\haiku{{\textquoteright} {\textquoteleft}Dan moet je maar eens.}{voor geruime tijd bij ons}{je intrek nemen}\\

\haiku{het voorrecht je tot,.}{leidsman te mogen dienen}{voor mij zelve op}\\

\haiku{{\textquoteright} {\textquoteleft}Wanneer ik er voor,?}{insta dat niemand de zaak}{zelfs vermoeden zal}\\

\haiku{Mevrouw Van Weerdt had.}{haar een onaangename}{indruk gegeven}\\

\haiku{Ondanks mijn brief en.}{telegram is de japon}{niet aangekomen}\\

\haiku{de woorden, welke,.}{zij opving kwamen van de}{overkant der tafel}\\

\haiku{{\textquoteleft}Frits, papa heeft op.}{de goede uitslag van je}{examen geklonken}\\

\haiku{- had u dan zulk een?}{machtige roeping toen u}{Nederland verliet}\\

\haiku{Gijsbrecht liet zich door,:}{die uitroep niet uit het veld}{slaan maar ging kalm voort}\\

\haiku{{\textquoteright} {\textquoteleft}Frits gaat nu niet meer{\textquoteright}, {\textquoteleft}'.}{uit merkte Elisabeth op}{t is al half tien}\\

\haiku{Smijt me in een gracht,.}{ik zal geen poging doen er}{weer uit te kruipen}\\

\haiku{Jij denkt er misschien.}{anders over omdat je nog}{altijd van haar houdt}\\

\haiku{Neen, laten wij nog ';}{wat rondlopen en ins}{hemelsnaam praten}\\

\haiku{Hij durfde echter,.}{niet weigeren klonk dus met}{iedereen en dronk}\\

\haiku{Reelijn vroeg welke.}{gasten de oude heer aan}{zijn dis had gehad}\\

\haiku{Doch eensklaps dacht hij.}{aan Frederika en zijn}{miskende liefde}\\

\haiku{{\textquoteright} Zwijgend ging Scheffer,.}{naar de deur en voerde haar}{de kamer binnen}\\

\haiku{Op een goede dag:}{namelijk had deze tot}{haar moeder gezegd}\\

\haiku{Het zwaard zinkt hem uit,.}{de hand machteloos zijgt hij}{op een stoel neder}\\

\haiku{Dan vraag ik Suze,.}{zondag avond en ben dus voor}{de regen binnen}\\

\haiku{{\textquoteright} Die aanwijzing was.}{voor Scheffer een lichtstraal in}{pikdonkere nacht}\\

\haiku{{\textquoteright} {\textquoteleft}Kijk eens Henri, wat,.}{jij als bankier doen kunt past}{ons daarom nog niet}\\

\haiku{{\textquoteright} De klemtoon, op het,.}{woord duurzaam gelegd ging voor}{Henri verloren}\\

\haiku{Buigend vertrekken,.}{zij doch ook nu vermindert}{het handgeklap niet}\\

\haiku{Opgewonden was;}{de gelukkige auteur}{naar boven gesneld}\\

\haiku{'t Is onwaar dat.}{wij uit eigen kracht iets goeds}{kunnen voortbrengen}\\

\haiku{wie durft beweren...{\textquoteright} {\textquoteleft},,.}{dat hij zijn evenmensKomaan}{oom geen praatjes meer}\\

\haiku{Integendeel vraag.}{ik mij af of Frits je wel}{waardig zal blijken}\\

\haiku{Slechts een nieuw doelwit,.}{een nieuwe eerzucht konden}{mij weer opheffen}\\

\haiku{{\textquoteright} Dit zou echter voor,.}{heden onmogelijk zijn}{beweerde Clara}\\

\haiku{Natuurlijk moet hij.}{afstuderen voordat het}{publiek kan worden}\\

\haiku{Was zij zich bewust?}{het toppunt van haar geluk}{bereikt te hebben}\\

\haiku{Eindelijk rees hij,,:}{op drukte haar onstuimig}{aan zijn borst en vroeg}\\

\haiku{je is zo spoedig.}{mogelijk te trouwen en}{dan eerst te werken}\\

\haiku{Geen terechtwijzing.}{van wie ook heeft mij ooit als}{deze getroffen}\\

\haiku{Niets belette hem.}{immers ogenblikkelijk naar}{boven te snellen}\\

\haiku{hij wist zelf niet meer,...}{hoe hij beiden voor het laatst}{gezien had maar dan}\\

\haiku{{\textquoteleft}Kom toch hier, en stel{\textquoteright}.}{u zo dwaas niet aan riep hij}{hem gemelijk toe}\\

\haiku{alle vertogen.}{van Gijsbrecht dienaangaande}{bleven vruchteloos}\\

\haiku{{\textquoteleft}Dag vrouw, dag moeder,!}{ziedaar mij gelukkig weer}{bij u aangeland}\\

\haiku{{\textquoteright} Terwijl Adolf aldus.}{sprak had Clara haar hand op}{zijn schouder gelegd}\\

\haiku{{\textquoteright} Dit zeggend scheurde.}{hij het telegram langzaam}{in kleine stukken}\\

\haiku{Nawoord Jong Holland ().}{1881 is de eerste roman}{van Marcellus Emants}\\

\haiku{Het was geenszins de.}{eerste maal dat Gijsbrecht ze}{ten beste  gaf}\\

\subsection{Uit: Lichte kost}

\haiku{Hij wond zich op, werd,.}{zenuwachtig en stootte}{meestal te hard}\\

\haiku{Monte Carlo is,!}{verloren en graaf Trogach}{strijkt met zijn schatten}\\

\haiku{Aan tafel was de;}{Italiaan nog steeds een en}{al bewondering}\\

\haiku{Ik maak mij sterk u.}{in twee minuten op de}{hoogte te brengen}\\

\haiku{Weer hield De Morrien,;}{vijftien Louis weer schoof Masset}{er dertig vooruit}\\

\haiku{- Mag ik de veertig?}{Louis aan de heer Masset voor}{u uitbetalen}\\

\haiku{Won hij nu en dan,;}{een zet dan verloor hij weer}{de twee volgende}\\

\haiku{Tienduizend franken,!}{verloren en bovendien}{veertig Louis geleend}\\

\haiku{{\textquoteright} {\textquoteleft}Juist daarom,{\textquoteright} voegde, {\textquoteleft}.}{Prudent er bijspeelde ik}{hoger en hoger}\\

\haiku{Zonder de minste:}{voorbereiding barstte zijn}{bekentenis los}\\

\haiku{Ik geloof niet, dat.}{De Morrien ooit in Monte}{Carlo heeft gespeeld}\\

\haiku{{\textquoteright} De commissaris.}{rees op en beantwoordde}{deze uitroep niet}\\

\haiku{in elk geval mocht.}{zij spoedig enig bericht van}{hem tegemoet zien}\\

\haiku{{\textquoteright} Gelukkig verstond.}{de commissaris geen woord}{van al dat gegrom}\\

\haiku{jongeheer, en met,.}{recht want ik stelde mij lang}{niet heerachtig aan}\\

\haiku{Geen van de kale.}{hoofden paste en geen van}{de pruiken voldeed}\\

\haiku{Cr\'ebillard hechtte,.}{aan typen en ik wilde}{er niet van horen}\\

\haiku{wij behielpen er,.}{ons echter mee zo goed en}{zo kwaad als het ging}\\

\haiku{De dames houden,;}{aan maar worden naar boven}{teruggezonden}\\

\haiku{{\textquoteleft}Monsjieu Cr\'ebillard... heeft.......}{al sedert een half uur zijn}{winkel verlaten}\\

\haiku{Je hebt zeker een.}{droppel van je mastik er}{op laten vallen}\\

\haiku{{\textquoteright} De lust tot praten;}{had onze kapper er niet}{bij ingeschoten}\\

\haiku{{\textquoteright} {\textquoteleft}Tot weerziens, mijnheer,.}{Cr\'ebillard tot weerziens in}{de nieuwe woning}\\

\haiku{{\textquoteright} {\textquoteleft}Welnu, heren, die,.}{dochter is een canaille}{een waar canaille}\\

\haiku{Nu denk je, dat ik '.}{mijt eerst weer wendde tot}{juffrouw Van Limburg}\\

\haiku{juffrouw Van Limburg.}{had zich nooit tot verpleegster}{mogen opwerken}\\

\haiku{Voor iedereen had,'}{hij een aardig woordje zelfs}{op juffrouw Timmers}\\

\haiku{juffrouw Van Limburg.}{begon mij bijzonder te}{interesseren}\\

\haiku{As ie thoch rijk is, '?}{wat mot so'n man dan inn}{sanatorium}\\

\haiku{mevrouw Van der Lof.}{sloeg het dampende vocht in}{een teug naar binnen}\\

\haiku{Maar als ze denkt, dat, '!}{ze zo van me afkomt dan}{heeft zet toch mis}\\

\haiku{Zij was er zeker;}{van deze melodie meer}{te hebben gehoord}\\

\haiku{Dit wilde ik in;}{dichterlijke beelden voor}{aller ogen stellen}\\

\haiku{De hele les is,!}{er aan heen gegaan en zij}{was opgetogen}\\

\haiku{Bovendien... jij doet.}{in de muziekkunst en ik}{doe in de bouwkunst}\\

\haiku{Doch plotseling boog:}{Frau Troistorff zich nog verder}{voorover en riep uit}\\

\haiku{zij had niet gedacht,.}{dat hij zo flink voor de dag}{zou durven komen}\\

\haiku{hij verbeeldde zich,:}{dat hij werkelijk voor de}{verleiding bezweek}\\

\haiku{Een oude dame;}{bette hem de slapen met}{eau de Cologne}\\

\subsection{Uit: Liefdeleven}

\haiku{Dat doet toch ieder,.}{die het bericht krijgt van z'n}{vriends engagement}\\

\haiku{ik dacht wel, dat je.}{de jaren van verliefdheid}{achter de rug hadt}\\

\haiku{dat is jouw idee en,;}{ik kan niet bewijzen dat}{je ongelijk hebt}\\

\haiku{Geloof jij, dat 'en '?}{groot man gelukkiger is}{danen obskuur baasje}\\

\haiku{daar heb je zo wat,.}{alles wat medici met}{hun kunst vermogen}\\

\haiku{ook de luitjes op.}{een dorp weten wat er in}{de wereld omgaat}\\

\haiku{Daar achter wist hij.}{de vaart en aan die vaart naast}{de koepel een bank}\\

\haiku{Misschien had hij toch.}{beter gedaan niet op dit}{bal te verschijnen}\\

\haiku{Fluks hief ze zich weer......}{recht omhoog kwam dichterbij}{een bloem in de hand.}\\

\haiku{Ja... als u vindt, dat......}{mijn mijn genegenheid niks}{te beduiden heeft}\\

\haiku{{\textquoteright} Weer aarzelde ze.}{en die aarzeling stelde}{hem opnieuw teleur}\\

\haiku{Als ik trouwde, zou.}{ik natuurlik dat beroep}{op moeten geven}\\

\haiku{maar ook, omdat hij.}{zich moeilik behelpen kon}{met een eng verblijf}\\

\haiku{Ik geloof, dat je.}{zo kalm bent en anderen}{zo kalm kunt maken}\\

\haiku{Christiaan vond, dat:}{ze flink optrad en voegde}{er alleen nog bij}\\

\haiku{maar overigens had:}{zij zich de naaste toekomst}{als volgt voorgesteld}\\

\haiku{maar die verkilden,...}{zodra een beeld oprees uit}{de nare dagen}\\

\haiku{{\textquoteleft}Ch\^atelaine, nou.}{ga ik je installeren}{in je nieuwe slot}\\

\haiku{Soms zeg je maar wat,.}{anders alleen om tegen}{te kunnen spreken}\\

\haiku{Dat je 't in zo'n!}{kamer ook maar \'e\'en nacht hebt}{kunnen uithouwen}\\

\haiku{Wat moeten ze er,?}{beneden van denken als}{we niet opdagen}\\

\haiku{wou je mij die ook...,?}{aanrekenen als iets goeds}{iets moois iets ideaals}\\

\haiku{Maar eerst ben je me......}{onverschillig geweest en}{nu nu haat ik je}\\

\haiku{Dan zou je weer met!}{je ouwe Trijn en met d'r}{dochter gaan heulen}\\

\haiku{Tot ze dicht genoeg,.}{bij hem stond dat haar hand de}{zijne kon grijpen}\\

\haiku{Misschien zal hier of ';}{daaren muur weggebroken}{dienen te worden}\\

\haiku{Enkel, dat ze geen {\textquoteleft}{\textquoteright},.}{sc\`enes maakte dat ze bleef}{zoals ze nu was}\\

\haiku{doorklinken, soms haar!}{zingen te horen galmen}{door voorhuis of gang}\\

\haiku{et Is ook wat die '.}{mensenen beetje naar de}{mond te babbelen}\\

\haiku{Dit antwoord schrijnde.}{nu weer door zijn zalige}{jubel-stemming}\\

\haiku{Zo zenuwachtig,,.}{dacht Christiaan heb ik ze}{vroeger nooit gezien}\\

\haiku{Maar dat belet niet, ' '.}{datet een of ander m'n}{aandacht wels trekt}\\

\haiku{Trijn heeft iets gezegd,,...}{waaruit ik heb opgemaakt}{dat wij gegeven}\\

\haiku{als we...{\textquoteright} {\textquoteleft}Zie je nu,?}{wel dat je de waarheid voor}{me verbergen wilt}\\

\haiku{Doch op alles zei:}{ze neen en altijd volgde}{dezelfde fraze}\\

\haiku{Ik heb nooit gedacht, '.}{dat ik vanen man z\'oveel}{zou kunnen houwen}\\

\haiku{Voor iets ernstigs, iets,,...}{moois en goeds iets dat alleen}{ik zou kunnen doen}\\

\haiku{want sinds ze weet, dat,.}{u komt is de toestand al}{merkbaar verbeterd}\\

\haiku{Mina heeft van de '.}{eerste dag afen hekel}{aan dat mens gehad}\\

\haiku{Weet u, dat ik haar?}{aangeboden heb Trijn te}{pensioneren}\\

\haiku{maar 't is misschien, '.}{toch beter dat ik je de}{ogen daars voor open}\\

\haiku{Je hebt mij de schuld.}{van alles gegeven en}{dat doe je altijd}\\

\haiku{Voorzeker had hij.}{niet de bedoeling haar slecht}{te behandelen}\\

\haiku{de randen van haar.}{ogen en een zilverig waas}{overtoog er het blauw}\\

\haiku{hoe waar 't is, dat.}{menig mens in zich zelf z'n}{ergste vijand heeft}\\

\haiku{Maar \'e\'en ding wil ik...}{je wel voorspelle al doe}{je alles voor d'r}\\

\haiku{{\textquoteright} Nu verweet ze hem '?}{z'n opstuiven en wie stoof}{altijdet eerst op}\\

\haiku{{\textquoteleft}Mijn God, 'en mens kan '!}{toch eindelik weles van}{idee veranderen}\\

\haiku{Maak ik 'es 'en plan,!}{dan heb jij altijd honderd}{duizend bezwaren}\\

\haiku{En hij voelde het:}{beetje nieuwsgierige vrees}{weer van hem wijken}\\

\haiku{Dat doet nou maar net,!}{of d'r geen mens anders op}{de wereld bestond}\\

\haiku{Totdat Mina plots,.}{stilhield lucht opsnoof en}{hoorbaar weer uitblies}\\

\haiku{je hebt 'et gezegd,...}{en je hebt er bij gevoegd}{dat je ook mij haat}\\

\haiku{het gouden haar, en,.}{in een nieuw kostuum dat haar}{biezonder goed stond}\\

\haiku{Trouwens,{\textquoteright} voegde hij, {\textquoteleft}.}{er bijmet de tijd wordt dat}{van zelf wel anders}\\

\haiku{grinniklachjes,;}{grapten waar niet ieder de}{reden van begreep}\\

\haiku{Over de tafel heen.}{vroeg hij Diepe naar een van}{zijn pasi\"enten}\\

\haiku{En eindelik en.}{ten laatste had zij de hoop}{ook opgegeven}\\

\haiku{maar niet zag hij haar,.}{die alleen hem zijn rust had}{kunnen hergeven}\\

\haiku{Dadelik was hij,.}{overeind haar gevolgd en nu}{liep hij naast haar door}\\

\haiku{Maar door haar schouders,:}{voer een licht schokje en zacht}{doch beslist sprak ze}\\

\haiku{Me dunkt, dat ie zich!}{niet ontzien heeft z'n afkeer}{van mij te tonen}\\

\haiku{{\textquoteleft}Als ik me maar door!}{iedereen vernederen}{en vertrappen laat}\\

\haiku{Twee dagen later.}{belde Christiaan om half}{vijf bij Diepe aan}\\

\haiku{de indruk, dat ze.}{tamelik prikkelbaar en}{opgewonden was}\\

\haiku{bleef hij bedaard, dan.}{schold ze op zijn tergende}{onverschilligheid}\\

\haiku{Ja... als er nou 's ', '...?}{en kleintje kwam zout je}{welkom zijn of niet}\\

\haiku{dat geef ik toe, Dus... '......, '.}{komtet nou wat mij betreft}{zalt welkom zijn}\\

\haiku{Soms is 't zelfs of '.......}{zeen tegenzin heeft in}{mij of in de zaak}\\

\haiku{Maar toch... toch speet 'et...}{hem nu dubbel zijn vriend te}{hebben geraadpleegd}\\

\haiku{En kwam hij... nu, dan '.}{wast nog niet te laat om}{te zien wat te doen}\\

\haiku{{\textquoteleft}Ach... wat ik daarbij,....}{voel kan jij niet kan niemand}{met me mee voelen}\\

\haiku{Van morgen wist je,.}{zeker nog niet dat ie je}{zou komen halen}\\

\haiku{Bij de andere...... ';}{vrouw een weduwe wast}{ook wel armoedig}\\

\haiku{Christiaan ving daar;}{nu en dan in zijn atelier}{wel een galm van op}\\

\haiku{{\textquoteright} {\textquoteleft}Dat ik niet mee wil,...;}{komen is de zaak niet heb}{ik ook niet gezegd}\\

\haiku{{\textquoteleft}Je zult wel moe zijn,,{\textquoteright}:}{ga maar vroeg naar bed dan was}{telkens haar antwoord}\\

\haiku{maar tans al... en nog......}{wel alleen naar Heijdestein}{terug te keren}\\

\haiku{Het lokaal, dat hij,.}{eindelik vond en nam was}{weinig naar zijn zin}\\

\haiku{Toen hij zich naar de,.}{deur wendde om weer heen te}{gaan stond zij voor hem}\\

\haiku{met z\'o'n dokter... dat... '.}{begrijp je toch is er niet}{et minste gevaar}\\

\haiku{{\textquoteright} {\textquoteleft}Beloof me dan, dat,...}{je geen vrouw zult nemen van}{wie je niet zeker}\\

\haiku{Hoe Mina dan zou..., '.}{worden hij kon hij dorstet}{zich niet voorstellen}\\

\haiku{Ik zou 't zo naar,.}{zo vreselik vinden als}{je me die afsloeg}\\

\haiku{Christiaan vond er,.}{dus niets vreemds in dat Mina}{Geertje te hulp riep}\\

\haiku{Een uur te voren.}{was Jantje ontwaakt en had}{hij weer even geschreeuwd}\\

\haiku{moest worden gedaan,.}{wat er nog verder in huis}{viel te verrichten}\\

\haiku{en zij... zou zij nog}{iets voor me voelen als ik}{niet de vader was}\\

\haiku{Hij zelf kon enkel,,.}{door angst te tonen haar nog}{angstiger maken}\\

\haiku{Zonder afscheid te.}{nemen vertrok Christiaan}{naar zijn atelier}\\

\haiku{Zonneschijn lag in;}{de kamer als een fletse}{vlek op het tapijt}\\

\haiku{Zodra Jantje aan,....}{werd geraakt begon hij te}{kreunen te krijten}\\

\haiku{Christiaan hoorde,.}{de fletse klanken verstond}{de zin er van niet}\\

\haiku{Dringend herhaalde,.}{ze die vraag als hij geen zeer}{stellig antwoord gaf}\\

\haiku{Was dat ook niet 'et, ',?}{bestet enige dat hem}{nog restte te doen}\\

\haiku{Voor geen geld zou hij.}{de herinnering aan die}{droom willen missen}\\

\haiku{Op die schijning ging,,,;}{hij in gedachten verdiept}{af doorschreed de deur}\\

\haiku{Maar... hier hing nog 'en... '.......}{fletse geuren geur van van}{Iris-poeier juist}\\

\haiku{En een tinteling.}{van hoop had nog eenmaal zijn}{somber denken doorflitst}\\

\haiku{Geen  blik was naar,;}{hem opgerezen geen woord}{tot hem uitgegaan}\\

\haiku{maar zijn alle, die,?}{er een grotere dosis}{van bezitten ziek}\\

\haiku{De afwijkende;}{mens speelt dus even goed een rol}{als ieder ander}\\

\haiku{het wegwerpen van.}{het kind met het badwater.11}{Marcellus Emants}\\

\haiku{Normaal doet zich voor -,!}{de psyche ziehier de lijn}{die de praktijk trekt}\\

\haiku{11Met opzet heb {\textquoteleft}{\textquoteright}.}{ik in dit betoog het woord}{schoonheid vermeden}\\

\subsection{Uit: Waan}

\haiku{'t Is zelfs wel... 't,?}{is in alle geval z\'o}{niet ordinair h\`e}\\

\haiku{Of er meer zulke?}{zonderlinge mensen als}{zij zouden bestaan}\\

\haiku{We kunne tante.}{toch de hele avond niet aan}{d'r lot overlate}\\

\haiku{Weer leunt zij over het.}{balkonhek heen en naast haar}{buigt Hendrik neder}\\

\haiku{Ik heb nooit meer te.}{hore gekrege dan zo'n}{paar losse mate}\\

\haiku{maar aan voorspele ' '.}{heb ikn hekel en van}{avond doe ikt niet}\\

\haiku{O, als ik dat niet...,;}{had geloof me dan was ik}{er al lang niet meer}\\

\haiku{Grootogig en met.}{open mond kijkt zij hem bijna}{verontwaardigd aan}\\

\haiku{En ik kan 't ook,.}{niet vele dat je me zo}{zit aan te stare}\\

\haiku{- Nou, kind, dat jij niks,.}{gebroke hebt is ook meer}{geluk dan wijsheid}\\

\haiku{Dra blijft ze ook staan,,....}{de hand op het hart gedrukt}{hijgend sprakeloos}\\

\haiku{- Nee, kind, dat moet je.}{juist niet en daar ben je zelf}{ook wel van overtuigd}\\

\haiku{Maggie zal zich niet;}{behoeven af te sloven}{in haar huishouden}\\

\haiku{Zonder een woord meer.}{te zeggen bereiken ze}{het pension}\\

\haiku{... wil u 't me nou ',?...}{s inpepere dat ik}{onlief ben geweest}\\

\haiku{- Denkt u dan, dat we......?}{niet voor mekaar passe of}{of iets dergelijks}\\

\haiku{- Ja... wat zal 'k je? ';}{daar nou van zegget Kan}{best aan mij ligge}\\

\haiku{maar... mijn hemel, als...!}{ik dat vergelijk met mijn}{engagementstijd}\\

\haiku{Bloemen en blade,.}{leven en pluk je ze af}{dan maak je ze dood}\\

\haiku{toch vond ik 't de,.}{eerste keer nog enger nog}{veel ontzettender}\\

\haiku{de lange, lege,,;}{weg zo rul bezond lijkt hem}{troosteloos eenzaam}\\

\haiku{maar er jets voor doen,,:}{er jets voor laten er voor}{in de plaats geven}\\

\haiku{'t Is genoeg, dat,;}{hij haar wil aanraken of}{ze duwt hem terug}\\

\haiku{Als ze terugkeert,,}{herhaalt de oude vrouw haar}{vraag gist ze vorst ze}\\

\haiku{Nu alles voorbij,.}{is kan ik dus helemaal}{vrij tot je spreken}\\

\haiku{Hoe dat komt, begrijp,.}{ik zelf niet want vroeger hield}{ik toch ook van je}\\

\haiku{zo kan je 't weer '...}{geen twee minute zonder}{t kind uithouwe}\\

\haiku{Hij ziet, dat ze van.}{alle kanten bekeken}{en befluisterd wordt}\\

\haiku{Rietzen wordt een fles.}{Heidsieck Monopol met drie}{glazen neergezet}\\

\haiku{hij heeft 't al lang, '.}{geweten al kont hem}{vroeger niets schelen}\\

\haiku{daar benne d'r heel...:}{wat die stemme toch zeker}{op u. Die zegge}\\

\haiku{Daarna begint ze.}{ineens weer opgewekt te}{spreken over het feest}\\

\haiku{toen heb je Rietzen '.}{geen rekenschap gevraagd of}{n klap gegeve}\\

\haiku{En nu ze geen vrees...}{meer koestert voor de scherpe}{klauwen van de beer}\\

\haiku{In het leven van.}{de man is de liefde maar}{een episode}\\

\section{Graad Engels}

\subsection{Uit: Det dank'tich d'n duvel}

\haiku{Ze konden echter.}{geen plaats passeren waar een}{kruis hing of stond}\\

\haiku{Ze lag daar echter.}{met een verband om haar hand}{te kermen van pijn}\\

\haiku{Ze was uiterlijk.}{een heel normale vrouw van}{rond de zeventig}\\

\haiku{Hij zei dat tegen.}{zijn vrouw en gebood haar de}{blaasbalg te trekken}\\

\haiku{Men adviseerde, {\textquoteleft}{\textquoteright}.}{dan ook te proberen van}{henbloed te trekken}\\

\haiku{Die ging schoorvoetend.}{mee en ook hij hoorde de}{wanmolen draaien}\\

\haiku{Het begon met de.}{plotselinge sterfte van}{een koe en een paard}\\

\haiku{Boer Jan raakte door.}{de miserie zijn geld en}{zijn gezondheid kwijt}\\

\haiku{Binnen een week viel.}{zijn van huis meegekregen}{paard dood in de kar}\\

\haiku{Deze noemde men {\textquoteleft}'{\textquoteright},.}{t Menke van Gengk uit de}{Belgische Kempen}\\

\haiku{Let dan goed op voor.}{het te laat is en schiet hem}{zijn gat vol hagel}\\

\haiku{Het werd zo laat dat.}{ze tegen middernacht nog}{onderweg waren}\\

\haiku{{\textquoteright} Een boer uit Helden {\textquoteleft}'{\textquoteright}.}{was op koehandel int}{Rooth onder Maasbree}\\

\haiku{Boven op de steel,.}{hing een klomp die telkens mee}{op en neer wipte}\\

\haiku{Men hoorde ze dan,,.}{kloppen maar niemand mocht noch}{durfde gaan kijken}\\

\haiku{Hij vertelde mij.}{dat daar in de wal vijftien}{kabouters woonden}\\

\haiku{Op de verjaardag.}{van een der leiders wilden}{ze eens wat aparts doen}\\

\haiku{De schaapherders uit.}{die tijd bleven graag uit de}{buurt van het Soemeer}\\

\haiku{Hij was daarover zo.}{kwaad dat hij een week  aan}{een stuk door vloekte}\\

\haiku{De duivel had hem.}{op den duur onder vloeken}{moeten loslaten}\\

\haiku{de hele dag zoet.}{zijn en er waarschijnlijk niet}{eens mee klaar komen}\\

\haiku{Zo leven hier nog.}{talrijke verhalen van}{dezelfde soort}\\

\haiku{Dat scheelde maar een.}{haar of er hadden hier twee}{doden gelegen}\\

\haiku{Het volgende lied.}{werd voor mij gezongen door}{een vrouw uit Helden}\\

\haiku{Ik wil met al mijn.}{oude leden Gaan leven}{naar mijn zin en lust}\\

\haiku{Op een avond waren.}{we met kameraden op}{buurtbezoek geweest}\\

\haiku{Nog erg slaapdronken.}{legde ik me languit op}{de kar te rusten}\\

\haiku{De moeder kwam op.}{bezoek om naar het graf van}{haar zoon te zoeken}\\

\haiku{{\textquoteleft}Als jullie menen,.}{dat er iets bijzonders is}{ga dan eens kijken}\\

\haiku{Voorzichtig kwamen,.}{de omwonenden uit hun}{huizen meest vrouwen}\\

\haiku{Het waren reuzen,.}{van kerels wel twee en een}{halve meter lang}\\

\haiku{Na wat gerust te.}{hebben besloten ze de}{Rijn te gaan graven}\\

\haiku{Die bevond zich op.}{het eerste verdiep aan de}{westkant in een gang}\\

\haiku{Ze vlogen wild uit,.}{elkaar maar een hele klis}{tuimelde omlaag}\\

\haiku{Een vertelling over.}{Onze Lieve Heer zelf van}{een man uit Maasbree}\\

\haiku{{\textquoteleft}Hoe kunnen jullie?}{nou geloven dat er wat}{bijzonders gebeurt}\\

\section{Desiderius Erasmus}

\subsection{Uit: De lof der zotheid}

\haiku{Vooreerst, wat kan er?}{zoeter of kostbaarder zijn}{dan het leven zelf}\\

\haiku{Hoofdstuk XIV   De.}{Zotheid verlengt de jeugd en}{weert den ouderdom}\\

\haiku{alsof niet juist het,.}{sterven hierin bestaat dat}{men iets anders wordt}\\

\haiku{Maar door die dwaling.}{weg te nemen bederft men}{het geheele stuk}\\

\haiku{Drinkt of ga heen, maar,.}{verlangt dat het tooneelstuk geen}{tooneelstuk meer zal zijn}\\

\haiku{Welk wijsgeer heeft ooit,?}{een staat uitgedacht die den}{haren nabijkomt}\\

\haiku{Nog zeldzamer is',.}{Mercurius geschenk de}{welsprekendheid}\\

\haiku{Zou ik soms Diana,?}{benijden omdat men haar}{menschenbloed offert215}\\

\haiku{Voorts bestaat een groot.}{gedeelte van hun geluk}{in hun bijnamen}\\

\haiku{{\textquoteleft}Hij vangt slapend visch{\textquoteright},:}{in zijn net werd toegepast}{en ook een ander}\\

\haiku{neem toch de misdaad,;}{uws knechts weg want ik heb zeer}{zottelijk gedaan441}\\

\haiku{44Jupiter wordt,.}{bedoeld die gezoogd zou zijn}{door de geit Amalthea}\\

\haiku{VII) wordt Bacchus op.}{de meest belachelijke}{wijze voorgesteld}\\

\haiku{114Dit fabeltje.}{wordt bij sommige oude}{dichters gevonden}\\

\haiku{122Iets dergelijks wordt,,.}{van Numa den tweeden koning}{van Rome verteld}\\

\haiku{142Grieksch wijsgeer, 396- (.}{314 v. Chr. 143Cato de}{jongerezie hoofdst}\\

\haiku{195{\textquoteleft}De eene muilezel,{\textquoteright} {\textquoteleft}}{krabt den anderen spreekwoord}{uit de oudheid =}\\

\section{Charles-Alexandre Chatrian en Emile Erckmann}

\subsection{Uit: De loteling van 1813}

\haiku{{\textquoteright} {\textquoteleft}Moeder, wij deden.}{hun wel kwaad en dat komen}{zij ons vergelden}\\

\haiku{{\textquoteright} Deze sterke vrouw.}{had haar toestand begrepen}{en den plicht haars zoons}\\

\haiku{Daarom wendde hij:}{het gesprek op een ander}{onderwerp en zei}\\

\haiku{de toestand van den.}{huisvader boezemde hem}{ongerustheid in}\\

\haiku{Abraham wilde zijn;}{zoon zelfs opofferen om}{te gehoorzamen}\\

\haiku{Het leven behoort,.}{u niet toe God alleen kan}{het u ontnemen}\\

\haiku{{\textquotedblleft}Noch aan u zelf, noch.}{aan anderen zult gij de}{schennende hand slaan}\\

\haiku{Maar wie heeft op de?}{kinderen toch meer recht dan}{vaders en moeders}\\

\haiku{Men rakelde den;}{haard op en allen schaarden}{zich om neef Laurens}\\

\haiku{Ten derden male,.}{doorkruist de tamboer de stad}{doch zonder gevolg}\\

\haiku{{\textquoteright} {\textquoteleft}Waarom,{\textquoteright} zei deze, {\textquoteleft}?}{staat het er overal niet zoo}{gunstig voor als hier}\\

\haiku{{\textquoteleft}Vendee\"ers, zijt gij?}{uw wettige koningen}{vergeten of niet}\\

\haiku{In de kazerne.}{vonden zij slechts twee dooden en}{vele gewonden}\\

\haiku{Daarna gelastte.}{zij een der omstanders een}{priester te halen}\\

\haiku{Gun hem den tijd om.}{zijn misdaad te beweenen}{en uit te boeten}\\

\haiku{Toch is het nog niet,.}{lang geleden dat ik den}{arbeid vaarwel zei}\\

\haiku{Hebt gij geen rust, en?}{geniet gij geen gezondheid}{op uw ouden dag}\\

\haiku{Aan mijn oudsten zoon,.}{laat ik mijn hut na hij zal}{mijn plaats innemen}\\

\haiku{Op dit gezicht sprong}{Berenger van zijn paard en}{door schrik bevangen}\\

\haiku{- Ik zal dit kasteel,.}{nooit verlaten antwoordde}{Berenger met kracht}\\

\haiku{- Mijn zoon, alvorens,.}{monnik te worden was ik}{een krijgsman zooals gij}\\

\haiku{Ik zie duidelijk,.}{den heiligen standaard die}{op dat schip wappert}\\

\haiku{- Berenger d'Elvaz,,?}{stamelde hij eindelijk}{is het mogelijk}\\

\section{Emile Erens}

\subsection{Uit: Korte verhalen}

\haiku{In een oofttuintje:}{achter het leemen boerenhuis}{zaten zij samen}\\

\haiku{Antje plaatste de lamp.}{op tafel en nu werd er}{koffie gedronken}\\

\haiku{Ook de heeren van.}{de zandgroeve speelden kaart}{dicht bij het buffet}\\

\haiku{Hij kroop een eindje:}{terug en opstaand ging hij}{rechtop op haar toe}\\

\haiku{Want ook Tinus was,.}{de vijand van Frits omdat}{deze geen boer was}\\

\haiku{Nevelwit hing de.}{lichtstilte der nacht over het}{heuvelige land}\\

\haiku{Bij den biechtstoel van:}{den kaplaan zat een jonge}{meid in groote onrust}\\

\haiku{{\textquoteright} {\textquoteleft}Och neen, nu niet, ik,{\textquoteright}.}{wil toch nog niet trouwen sprak}{zij met schuwen blik}\\

\haiku{{\textquoteleft}Ween niet zoo, Mieke,,{\textquoteright}.}{als we trouwen is immers}{alles goed zei Johan}\\

\haiku{Toen de winter kwam,.}{vond Liesje haar dood in den}{donkeren stal}\\

\haiku{zij schrok heel bleek van,.}{het plotselinge woord en}{voelde het op eens}\\

\haiku{Even buiten het dorp.}{liep de weg door een bosch met}{groote beukeboomen}\\

\haiku{En hij hoorde haar.....}{stille woorden zeggen in}{het donkere woud}\\

\haiku{Korte verhalen,!}{Noten 1{\textquoteleft}O Isra\"el}{hoe groot is Gods huis}\\

\section{Fons Erens}

\subsection{Uit: Bazen knuppels knechten. IJohannes Hendrijckos (1791-1866). Een dagloner in Wijnandsrade}

\haiku{In Maastricht viel mij.}{dat voor het eerst op in de}{dertiger jaren}\\

\haiku{De tantes zorgden.}{voor koffie en waren druk}{aan het babbelen}\\

\haiku{Nu ik oud ben, smaakt:}{die jeugd-boterham weer}{het allerbeste}\\

\haiku{'s Avonds kregen wij, {\textquoteleft}{\textquoteright}.}{dikwijls karnemelksepap}{die wijketsj noemden}\\

\haiku{Of de mensen zich?}{bij dat sobere bestaan}{gelukkig voelden}\\

\haiku{In ieder geval.}{was het dat zeker voor mij}{in het verleden}\\

\haiku{Nu wordt die heuvel;}{overhuifd met een koepel van}{hoge boomkruinen}\\

\haiku{De mensen moesten zelf.}{maar zien hoe zich tegen hen}{te verdedigen}\\

\haiku{Met vijf vliegende.}{vendels maken de Fransen}{jacht op deserteurs}\\

\haiku{Jean Henry Er(r).}{ens uit Wijnandsrade met}{trekkingsnummer 117}\\

\haiku{De mensen wisten,.}{het niet meer maar het kon hen}{ook weinig schelen}\\

\haiku{Al was ik wat bang, {\textquoteleft}{\textquoteright}.}{voor Han toch bracht ik graag de}{melk naargene Bek}\\

\haiku{De kachel zorgde.}{voor warm eten en voor warmte}{in de woonkeuken}\\

\haiku{Naast dat {\textquoteleft}sjaap{\textquoteright} hing een.}{achter glas geschilderde}{heiligenfiguur}\\

\haiku{In 1920 werd een pas.}{geslacht varken uit onze}{kelder gestolen}\\

\haiku{ik had meer te doen.}{met de jongen dan met het}{gestolen varken}\\

\haiku{Ik moet er als een.}{pasja in dat grote bed}{gelegen hebben}\\

\haiku{Soms kwamen meisjes.}{met hun spinnewiel samen}{op een boerderij}\\

\haiku{Niet minder dan 10\%.}{van alle gezinshoofden}{was zonder beroep}\\

\haiku{De {\textquoteleft}goeie ouwe tijd{\textquoteright},.}{heeft niet bestaan ook niet voor}{IJoannes Hendrijckos}\\

\haiku{Dit neemt niet weg, dat.}{ook daar de situatie}{erbarmelijk was}\\

\haiku{De kindersterfte.}{bleef ook in de Franse tijd}{nog ontstellend hoog}\\

\haiku{Hij is dan al 47!}{jaar lang schoolhoofd en zal het}{nog tot 1905 blijven}\\

\haiku{De mensen stonden.}{niet bepaald te springen om}{hun heer te dienen}\\

\haiku{Op mijn vraag naar de:}{geloofsbeleving in zijn}{jeugd vertelde hij}\\

\haiku{In het begin van.}{de middeleeuwen was de}{toga algemeen}\\

\haiku{{\textquoteleft}Ki\"esk\"opke{\textquoteright} zegt,,.}{de meester koster Coenen}{wel eens tegen hem}\\

\haiku{Per ongeluk schuurt - -.}{zijn stoppelbaard langs die o}{zo zachte wangen}\\

\haiku{Vader zit bij de,.}{schouw en dopt bonen moeder}{zorgt voor het avondeten}\\

\haiku{Even later gaat bij {\textquoteleft}{\textquoteright}.}{hen de buitendeur naar de}{neire krakend open}\\

\haiku{Trillend staat hij op,.}{zijn zware handen steunen}{op de tafelrand}\\

\haiku{Tot driemaal toe werd,.}{er geklopt maar telkens was}{niemand aan de deur}\\

\haiku{Aan die trekos kon,.}{iedereen zien dat vader}{een echte boer was}\\

\haiku{Ze bleven er twee,.}{dagen totdat Joean wat}{bijgekomen was}\\

\haiku{Heeft hij die bidweg?}{nu aan nonk Giel of aan zijn}{vader te danken}\\

\haiku{Het zand op het graf.}{was er toch en het water}{uit de put ook}\\

\haiku{Achter de grote.}{hoeve van Arensgenhout komt}{de winterzon op}\\

\haiku{Een kantkraag van sneeuw - -.}{spel van de wind krult over de}{berm langs het voetpad}\\

\haiku{Het is Bieldermans.}{van te Nietese met zijn}{zwager uit Hulsberg}\\

\haiku{De dienst is plechtig,,,.}{met veel misdienaars kaarsen}{wierook en gezang}\\

\haiku{De meeste boeren.}{zitten gebogen met de}{koppen bij elkaar}\\

\haiku{Twee mannen staan daar,,:}{nog nalachend tegen}{een muur te plassen}\\

\haiku{Hij gaat niet meer naar,.}{school hij heeft gebiecht en de}{kommunie gedaan}\\

\haiku{de drie koeien van {\textquoteleft}{\textquoteright}.}{Kleintjesomsjpanne in}{de stoppelklaver}\\

\haiku{Zij zal h\`em ook wel,.....}{horen alsof zij elkaar}{signalen geven}\\

\haiku{Hij gaat weer liggen,,.}{zijn gezicht naar de hemel}{de ogen gesloten}\\

\haiku{Joep is de jongste,.}{broer van hun overbuur Driek kan}{goed met hem overweg}\\

\haiku{Als een bronstig dier,....}{als een doldrieste duivel}{gooit hij zich op haar}\\

\haiku{Hij voert de paarden,.}{en streelt de kop van Bella}{zijn lievelingspaard}\\

\haiku{{\textquoteright} In \'e\'en adem vertelt.}{Driek hem alles wat die dag}{is voorgevallen}\\

\haiku{Joep is zijn beste,.}{kameraad hij begrijpt hem}{beter dan wie ook}\\

\haiku{En straks zullen hier....}{weer kinderen van Marie}{en Driek rondlopen}\\

\haiku{Na de mikken en.}{de vlaaien gaat een ham de}{nog warme oven in}\\

\haiku{Nonk Giel zet een lied:}{in op de melodie van}{de Marseillaise}\\

\haiku{Lange planken zijn:}{vastgespijkerd op hoge}{en lage paaltjes}\\

\haiku{Als gendarmes in,:}{de buurt komen begint een}{vrouw hard te roepen}\\

\haiku{Voor zichzelf had hij.}{al lang precies uitgemaakt}{wat hem te doen stond}\\

\haiku{Daar weet men alleen, '. '}{dat hijs morgens nog de}{paarden gevoerd heeft}\\

\haiku{Leike heeft Joean.}{bezworen er met geen woord}{over te beginnen}\\

\haiku{Ja maar, zeggen 'n,.}{paar grote boeren Leike}{je begrijpt dat niet}\\

\haiku{Diezelfde avond nog.}{rijdt Lommele Joean met}{zijn handkar naar Swier}\\

\haiku{Om tien uur in de.}{avond van 11 januari}{is hij gestorven}\\

\haiku{Niemand weet 't. Sinds.}{Napoleon is er al}{zoveel veranderd}\\

\haiku{{\textquoteright} Vrouw Dresen, die helpt,.}{bij geboorte en dood heeft}{hem al afgelegd}\\

\haiku{{\textquoteright} Rustig gaat hij voor,.}{hen uit ongeveer twintig}{mensen volgen hem}\\

\haiku{Als een woelrat kruipt.}{hij daarin weg onder het}{vochtige lover}\\

\haiku{Kort gezegd zijn het.}{vooral drie dingen die ik}{van hem geleerd heb}\\

\haiku{{\textquoteright} Vragend kijkt hij naar,:}{de pastoor die min of meer}{instemmend aanvult}\\

\haiku{En de pastoor.... 'n,!}{deserteur helpen dat is}{levensgevaarlijk}\\

\haiku{Ieder kind kreeg er.}{een en hield het angstvallig}{in beide handen}\\

\haiku{Een heel eind van het.}{karretje vandaan komen}{zij op het kerkpad}\\

\haiku{In de loop van de,.}{ochtend trekt de mist op er}{zit vorst in de lucht}\\

\haiku{De Fransen zouden.}{al allemaal over de Maas}{weggetrokken zijn}\\

\haiku{Hei, Kleintjes mit de,!}{korte beintjes zing ins get}{om werm te waere}\\

\haiku{Soms stonden ze zo.}{maar als reuzenfakkels in}{het veld te branden}\\

\haiku{Zij moet de garven.}{naar de wagen steken en}{Joep moet optasten}\\

\haiku{Hij staat alweer op.}{de wagen en zij heft een}{garf op de gaffel}\\

\haiku{Maar.... aan zijn arm loopt.}{Leineke en om hem heen}{jubelt de lente}\\

\haiku{ja, hij zal zorgen.}{dat hun kinderen leren}{lezen en schrijven}\\

\haiku{Maar nu is het een,.}{wildeman die ruig en vlug}{de mis afraffelt}\\

\haiku{Twee keer per dag moesten.}{de boeren er hun paarden}{en koeien drenken}\\

\haiku{Het gewas stond er,.}{schriel bij er kwam maar weinig}{korrel in de aren}\\

\haiku{De Vildersjtr\"op zit.}{met een wit weggetrokken}{gezicht bij de schouw}\\

\haiku{Op hoeveel van haar!}{huwelijksdagen heeft zij}{het niet gebeden}\\

\haiku{Hij foetert tegen,.}{Anna en Adam die het ook}{niet kunnen helpen}\\

\haiku{Bleef Driek aandringen.}{dan klauwde zij van zich af}{als een wilde kat}\\

\haiku{Kleintjes zelf zou hem.}{zeker aanraden om naar}{de Bongerd te gaan}\\

\haiku{De citer kraakt, een.}{snaar springt los en blijft langzaam}{heen en weer veren}\\

\haiku{Ku\"eb zegt, dat hij.}{even zal gaan vragen en loopt}{terug naar Anna}\\

\haiku{Op 30 april 1844 ging.}{hij haar dood aangeven bij}{de burgemeester}\\

\haiku{Daartussen trekt de.}{horizon een strakke lijn}{tegen de hemel}\\

\haiku{Een machtig kasteel.}{en een kerk liggen binnen}{brede slotgrachten}\\

\haiku{Zijn gouden borstkruis.}{en zijn ring glinsteren op}{in de schemering}\\

\haiku{Vertwijfeld grijpen.}{zij elkaar bij de hand en}{roepen om Sjtefan}\\

\haiku{Dat eigenbelang.}{uiteindelijk het belang}{van anderen is}\\

\haiku{Geschiedenis van{\textquoteright}.}{de beide Limburgen door}{Dr. Jappe Alberts}\\

\haiku{{\textquoteleft}Verloskunde en{\textquoteright}.}{kindersterfte in Limburg}{door Dr. J.H. Starmans}\\

\section{Frans Erens}

\subsection{Uit: Vertelling en mijmering}

\haiku{Hij sloeg aan ieder,.}{keer in den nacht wanneer er}{iemand voorbijging}\\

\haiku{Nu en dan kwam de.}{sikkel der maan achter de}{wolken te voorschijn}\\

\haiku{'Ja,' zei de vrouw, 'ik,,'.}{moet zeggen dat hij nooit zoo}{doet als dezen avond}\\

\haiku{Ik heb mij over niets.}{te beklagen gehad en}{bleef daar twee jaren}\\

\haiku{* * * ~ Zoo vertelde,.}{Nicolaas den nacht door tot}{vroeg in den morgen}\\

\haiku{breng mijnheer naar zijn,.}{kamer want hij zal zelf den}{weg moeilijk vinden}\\

\haiku{Toen ik mijn voeten,.}{onder de lakens strekte}{schrok ik ongemeen}\\

\haiku{Ik durfde  daar.}{zeker niet naar vragen den}{volgenden morgen}\\

\haiku{De zekerheid van.}{mijn eigen gedachten zou}{er onder lijden}\\

\haiku{Het was een grijze.}{haarlok ter lengte van meer}{dan een kwart meter}\\

\haiku{Kynon, zoon van Clydno,.}{verlangde van Kai wat de}{Koning had gezegd}\\

\haiku{De nacht duurde zeer.}{lang en de stilte werd door}{niets daar gebroken}\\

\haiku{Zij zullen zingen,.}{zooals gij het in uw eigen}{land nooit hebt gehoord}\\

\haiku{Geen wind rimpelde.}{het zilveren water aan}{den voet der muren}\\

\haiku{'Ik ben blij, dat gij,.}{geene andere oorzaak hebt}{mij weg te zenden}\\

\haiku{Misschien deed hij het.}{om niet ongehoorzaam te}{zijn aan den koning}\\

\haiku{Haar blauwen voorschoot}{had zij opgenomen en}{aanhoudend veegde}\\

\haiku{Van den vader kon,.}{men nooit veel zeggen dat was}{wel een brave man}\\

\haiku{Een paar zwaluwen,,.}{die op de luiken zaten}{te kweelen vlogen}\\

\haiku{De knecht kwam uit zijn:}{bed en naar binnen gaande}{vroeg hij aan Anna}\\

\haiku{Ongemerkt was zij.}{ingeslapen en zij had}{erg akelig gedroomd}\\

\haiku{Zet maar de koffie,.}{klaar ik sta op en ga er}{van morgen nog heen}\\

\haiku{Het is goed, dat de.'}{menschen zoo min mogelijk}{komen te weten}\\

\haiku{hij dompelde in,:}{de kom besprenkelde hij}{muren en bodem}\\

\haiku{'Wij moeten even gaan',, '.}{zitten zei de pastoorwant}{dat was hard werken}\\

\haiku{'Het verwondert me,.}{dat nicht Olfers nog niet bij}{ons heeft gelogeerd}\\

\haiku{Ja, ze zouden nu,}{hun hart eens ophalen aan}{een warm bordje soep}\\

\haiku{Zijn das is breed en,.}{zwart maar het zwart is door den}{ouderdom verkleurd}\\

\haiku{Hij is zorgvuldig.}{geschoren en draagt alleen}{een kleinen knevel}\\

\haiku{Men denke echter,}{niet dat hij het er op aan}{zou willen leggen}\\

\haiku{* * * ~ De heer Strowski.}{is een nog zeer jong docent}{van de Sorbonne}\\

\haiku{spreekt met hem niet uit '',.}{de hoogte maar als tot een}{vriend en gelijke}\\

\haiku{Zoo zag ik langs de.}{daken een spion door vier}{mannen achtervolgd}\\

\haiku{* * * ~ Uit steden en.}{dorpen trekken jammerend}{de vluchtelingen}\\

\haiku{* * * ~ Dat Goethe nooit,.}{een vurig patriot is}{geweest is een troost}\\

\haiku{Nu en dan hoort men.}{eens spreken van Joffre}{of van Hindenburg}\\

\haiku{De namen von Kluck,.}{Hausen enz. zijn verdwenen}{in de duisternis}\\

\haiku{* * * ~ De Pruisische.}{geest is een verschijnsel van}{den modernen tijd}\\

\haiku{Of het werkelijk,.}{overwonnen zal worden blijkt}{niet met zekerheid}\\

\haiku{Meent gij, dat hij zich?}{ooit van verdelgingszucht zal}{kunnen onthouden}\\

\haiku{Zij ziet dan geheel;}{wit als van karton met een}{blanke schittering}\\

\haiku{Zij vroegen het als.}{in een sleur en roeiden gauw}{weer naar land terug}\\

\haiku{Spoedig waren we.}{aan land en wij moesten eerst naar}{het customhouse}\\

\haiku{Daarom heen zaten.}{op banken gesluierde}{vrouwen te wachten}\\

\haiku{De lange baard in,,.}{twee gedeeld doch nog niet grijs}{doet hem herkennen}\\

\haiku{Doch de vrouwen bij,,,.}{de Spaansche Basken hebben}{dat naar ik meen wel}\\

\haiku{In hun land vonden.}{dan ook de Carlisten hun}{meesten aanhang}\\

\haiku{Maar zij glijden niet,.}{af zij kennen de waarde}{van iederen stap}\\

\haiku{Het zich bewegen,.}{is de strijd tegen den dood}{tegen den stilstand}\\

\subsection{Uit: Vervlogen jaren}

\haiku{In de betrachting.}{van dien plicht week Frans Erens voor}{niets op de wereld}\\

\haiku{Hij aanvaardde geen.}{zekerheid op gezag der}{domme herhaling}\\

\haiku{Dit was een gracht, door,.}{menschenhanden gegraven}{die diep en breed was}\\

\haiku{{\textquoteleft}Im Namen Gottes,,,.}{Vatters und des Sohns und}{des H. Geistes Amen}\\

\haiku{Het grondtype van.}{deze bouworde is de}{Romeinsche villa}\\

\haiku{{\textquoteleft}Geef mij dien eens hier{\textquoteright},, {\textquoteleft}.}{zei de pastooren rij nu}{die kar uit den mest}\\

\haiku{Soms liet ik het in.}{het midden langs het touw naar}{beneden vallen}\\

\haiku{slechts viel er nu en.}{dan een schietgebedje in}{het Fransch tusschendoor}\\

\haiku{Mijn grootmoeder zei.}{dat half voor zichzelf en dacht}{er verder niet over}\\

\haiku{Soms zag ik hem met.}{een groot houten plateel op}{de knie\"en zitten}\\

\haiku{Dit generaalschap.}{was voor Mathies de grootste}{vreugd van zijn leven}\\

\haiku{Hij was met alles.}{tevreden en maakte nooit}{eenige aanmerking}\\

\haiku{{\textquoteright} {\textquoteleft}O, ja{\textquoteright}, zei ik met, {\textquoteleft}.}{veel zelfbewustzijndaar heb}{ik veel aan gedaan}\\

\haiku{Wanneer ik eenige,.}{weken in het Noorden was}{rook ik het niet meer}\\

\haiku{De geur van een stad;}{of een land is toch wel iets}{zeer eigenaardigs}\\

\haiku{Een Leidsch professor;}{was voor mij een wezen van}{een hoogeren rang}\\

\haiku{{\textquoteleft}Conduisez-moi{\textquoteright}.}{au Quartier Latin \`a un}{hotel quelconque}\\

\haiku{Dubreuil was een zeer.}{begaafd artiest en een goed}{muziekcriticus}\\

\haiku{Hij schilderde in,.}{dien tijd een portret van mij}{dat ik nog bezit}\\

\haiku{Hij was uit Tours naar.}{Parijs gekomen om de}{Rechten te studeeren}\\

\haiku{Dat was het woord, dat.}{men gewoonlijk gebruikte}{voor supr\^emen lof}\\

\haiku{In ieder geval,.}{zal er een uitwerking zijn}{die hem gunstig is}\\

\haiku{la main mon cher ami,.}{Maurice Barr\`es   9}{Rue Victor Cousin}\\

\haiku{Die eerste tijd van,.}{Le Chat Noir was de beste}{ontegenzeglijk}\\

\haiku{Mor\'eas' ambitie.}{in Parijs was een groot Fransch}{dichter te worden}\\

\haiku{Den laatsten keer, dat,.}{ik hem heb gesproken was}{in La Vachette}\\

\haiku{Wij, Hollanders, wij.}{vonden dien toon op het laatst}{wel wat vermoeiend}\\

\haiku{Hij gedroeg zich als.}{een sto{\"\i}cijn en zag kalm}{den dood aankomen}\\

\haiku{hij zoo, evenals het, {\textquoteleft}{\textquoteright}.}{achterhoofd dat voor hemune}{face \'eteinte was}\\

\haiku{Den derden broer, die,;}{een gezien beeldhouwer was}{heb ik niet gekend}\\

\haiku{Rollinat was dus,,.}{wat men noemt arriv\'e met}{vijfendertig jaar}\\

\haiku{Aan de beweging.}{van zijn verzen ontbrak de}{spontane{\"\i}teit}\\

\haiku{Aan hoeveel smarten!}{en verwikkelingen was}{ik dan ontkomen}\\

\haiku{Toen Kloos er voor het,.}{eerst kwam zat ik toevallig}{naast hem in den kring}\\

\haiku{Van Deyssel nam het.}{bezoek zeer ernstig op en}{was zenuwachtig}\\

\haiku{Deze losheid was.}{juist voor ons een bekoring}{en hield ons bijeen}\\

\haiku{De nakomer heeft;}{er behoefte aan om te}{synthetiseeren}\\

\haiku{het boertige is,.}{niet zijn zaak maar wel het fijn}{humoristische}\\

\haiku{maar zooals veel feiten.}{uit die verre dagen is}{dat mij ontvallen}\\

\haiku{wellicht vergeet ik.}{bij deze opsomming nog}{den een of ander}\\

\haiku{Het is waar, dat wij.}{Feith en Bilderdijk op het}{oogenblik waardeeren}\\

\haiku{Eenigen tijd daarna,.}{hoorde ik dat dit Karel}{Alberdingk Thijm was}\\

\haiku{Het groote gebouw was.}{stampvol en Paap was toen een}{belangrijk persoon}\\

\haiku{Niemand was ooit op.}{het idee gekomen om die}{bron te controleeren}\\

\haiku{Veth was een man, wiens;}{conversatie bijzonder}{interesseerde}\\

\haiku{Goes of anderen,:}{werd er wel eens gescheld en}{als ik aan Kloos vroeg}\\

\haiku{Op een avond had zijn,.}{groote hond Kloos aangevallen}{maar niet gebeten}\\

\haiku{zij hield zich geheel}{onbeweeglijk en langs haar}{wang zag ik langzaam}\\

\haiku{Van het begin van}{zijn verblijf te Amsterdam}{ging hij met ons om}\\

\haiku{Soms zei hij dat hij.}{ging eten met dien of dien en}{vroeg of ik mee ging}\\

\haiku{hij behoefde er,.}{niet zuinig mee te zijn want}{hij had er genoeg}\\

\haiku{Als een werkman stond,.}{hij daar voor het doek deze}{kleine Hercules}\\

\haiku{Josef Israels had.}{in zijn gebaren iets meer}{magistraals dan Isa\"ac}\\

\haiku{en alleen aandacht.}{had voor hetgeen in Parijs}{over hem werd gezegd}\\

\haiku{ook aan mij bekend.}{waren en dat ik zijn vrouw}{dikwijls had ontmoet}\\

\haiku{Hij bleef passief en.}{liet zich die houding van Kloos}{goedig aanleunen}\\

\haiku{Verlaine was een,,.}{rakker een stijfhoofdige}{een onwillige}\\

\haiku{Tegenover het huis.}{en verderop lagen meer}{buitenverblijven}\\

\haiku{ik vergeten, ik.}{meen dat het op den linker}{Seine-oever was}\\

\haiku{Ik word gekend door.}{den Vader en dat is mij}{ten slotte genoeg}\\

\haiku{{\textquoteleft}Ik sta aan den rand...{\textquoteright},;}{is door den auteur kort v\'o\'or}{zijn dood gedicteerd}\\

\haiku{Ik herinner mij:}{nog heel goed hoe door een der}{onzen werd gezegd}\\

\haiku{Hier moet men letten:}{op de teekenen en men}{zou kunnen zeggen}\\

\haiku{4Gringoire was een.}{Fransch dichter van het einde}{der middeleeuwen}\\

\haiku{Hij heeft zich meen ik,.}{teruggetrokken omdat}{hij niet werd betaald}\\

\subsection{Uit: Vervlogen jaren}

\haiku{daarom had ik in.}{alle kamers papier en}{potlood klaar liggen}\\

\haiku{het allerlaatste.}{gedeelte heeft hij kort voor}{zijn dood gedicteerd}\\

\haiku{Mijn moeder had mij}{geleerd het afgepelde}{eitje plat te slaan}\\

\haiku{Dit was een gracht, door,.}{mensenhanden gegraven}{die diep en breed was}\\

\haiku{{\textquoteleft}Im Namen Gottes,,,.}{Vatters und des Sohns und}{des H. Geistes Amen}\\

\haiku{Het grondtype van.}{deze bouworde is de}{Romeinse villa}\\

\haiku{{\textquoteleft}Geef mij die eens hier{\textquoteright}, {\textquoteleft}.}{zei de pastooren rij nu}{die kar uit de mest}\\

\haiku{Soms liet ik het in.}{het midden langs het touw naar}{beneden vallen}\\

\haiku{Hijzelf was dronken,.}{geweest maar de dokter was}{volkomen nuchter}\\

\haiku{{\textquoteleft}Mijnheer Menten, wat,!}{moeten wij beginnen wij}{krijgen hongersnood}\\

\haiku{slechts viel er nu en.}{dan een schietgebedje in}{het Frans tussendoor}\\

\haiku{Mijn grootmoeder zei.}{dat half voor zichzelf en dacht}{er verder niet over}\\

\haiku{Soms zag ik hem met.}{een groot houten plateel op}{de knie\"en zitten}\\

\haiku{Dit generaalschap.}{was voor Mathies de grootste}{vreugd van zijn leven}\\

\haiku{Hij was met alles.}{tevreden en maakte nooit}{enige aanmerking}\\

\haiku{Hij kwam dikwijls bij.}{ons logeren en ging dan}{met mij wandelen}\\

\haiku{{\textquoteright} Als het donker was,.}{was de wagen van binnen}{dikwijls niet verlicht}\\

\haiku{Op de bodem werd '.}{s winters stro gelegd voor}{de koude voeten}\\

\haiku{Het was toen mode.}{zich de kop van de Franse}{keizer te maken}\\

\haiku{{\textquoteright} Een bravo-geroep.}{steeg op uit de bezoekers}{van het restaurant}\\

\haiku{Op het koor naast het.}{altaar zat een deftig heer}{met een jong meisje}\\

\haiku{Hij was een lange,.}{bleke jongen met blond haar}{en een plat gezicht}\\

\haiku{Hij vertelde met}{wie zij getrouwd waren en}{hoeveel kinderen}\\

\haiku{Hij sprak in het Frans,}{en toen hij aan het begin}{zei dat hij \'etranger}\\

\haiku{Wanneer ik enige,.}{weken in het Noorden was}{rook ik het niet meer}\\

\haiku{De geur van een stad;}{of een land is toch wel iets}{zeer eigenaardigs}\\

\haiku{Een Leids professor;}{was voor mij een wezen van}{een hogere rang}\\

\haiku{Men had de zaal door.}{een barri\`ere verdeeld}{in twee gedeelten}\\

\haiku{vier mensen van mijn.}{reisgezelschap in een open}{rijtuig gezeten}\\

\haiku{{\textquoteleft}Conduisez-moi.}{au quartier Latin \`a un}{hotel quelconque}\\

\haiku{Dubreuilh was een zeer.}{begaafd artiest en een goed}{muziekcriticus}\\

\haiku{Zij waren beiden.}{ongeveer \'e\'en- of}{twee\"entwintig jaar}\\

\haiku{Hij schilderde in,}{die tijd een portret van mij}{dat ik nog bezit.125}\\

\haiku{{\textquoteright} Dat was het woord, dat.}{men gewoonlijk gebruikte}{voor supr\^eme lof}\\

\haiku{In ieder geval,.}{zal er een uitwerking zijn}{die hem gunstig is}\\

\haiku{Ik bleef dan niet lang,.}{wilde hem  niet van zijn}{arbeid afhouden}\\

\haiku{C'est am\`erement peu,.}{pay\'e mais enfin c'est aussi}{peu de travail}\\

\haiku{Die eerste tijd van,.}{Le Chat Noir was de beste}{ontegenzeglijk}\\

\haiku{Mor\'eas' ambitie.}{in Parijs was een groot Frans}{dichter te worden}\\

\haiku{Plus de r\^eves d'}{azur au fond des bosquets verts O\`u}{le rossignolet}\\

\haiku{Hij heeft ook enige.}{werken uit de zestiende}{eeuw uitgegeven}\\

\haiku{Mevrouw Verlaine.}{zag er in die tijd breed en}{welgedaan uit}\\

\haiku{De laatste keer, dat,.}{ik hem heb gesproken was}{in de Vachette}\\

\haiku{Wij, Hollanders, wij.}{vonden die toon op het laatst}{wel wat vermoeiend}\\

\haiku{Hij gedroeg zich als.}{een sto{\"\i}cijn en zag kalm}{de dood aankomen}\\

\haiku{Alle genot is.}{enkelvoudig en zo ook}{het esthetische}\\

\haiku{De derde broer,229 die,;}{een gezien beeldhouwer was}{heb ik niet gekend}\\

\haiku{Rollinat was dus,,.}{wat men noemt arriv\'e met}{vijfendertig jaar}\\

\haiku{Aan de beweging.}{van zijn verzen ontbrak de}{spontane{\"\i}teit}\\

\haiku{Een figuur in het,.}{quartier Latin zoals ze}{zelden voorkwamen}\\

\haiku{Hij vertelde die,.}{zaak zeer uitvoerig onder}{ademloze stilte}\\

\haiku{Aan hoeveel smarten!}{en verwikkelingen was}{ik dan ontkomen}\\

\haiku{Toen Kloos er voor het,.}{eerst kwam zat ik toevallig}{naast hem in de kring}\\

\haiku{Van Deyssel nam het.}{bezoek zeer ernstig op en}{was zenuwachtig}\\

\haiku{Deze losheid was.}{juist voor ons een bekoring}{en hield ons bijeen}\\

\haiku{De nakomer heeft;}{er behoefte aan om te}{synthetiseren}\\

\haiku{het boertige is,.}{niet zijn zaak maar wel het fijn}{humoristische}\\

\haiku{Ik zie nog haar fraaie,,.}{bleke hand waarmee zij mij}{de regels aanwees}\\

\haiku{Frank van der Goes was.}{misschien wel de geestigste}{van de vriendenkring}\\

\haiku{wellicht vergeet ik.}{bij deze opsomming nog}{de een of ander}\\

\haiku{Alberdingk Thijm was,.}{gaf hij altijd een stuk van}{hemzelf ten beste}\\

\haiku{Het is waar, dat wij.}{Feith en Bilderdijk op het}{ogenblik waarderen}\\

\haiku{Enige tijd daarna,.}{hoorde ik dat dit Karel}{Alberdingk Thijm was}\\

\haiku{Daar zaten alleen;}{die drie jongemannen in}{een grote stilte}\\

\haiku{{\textquoteleft}Ik zet het je, ze.}{op een geschikte manier}{terug te leggen}\\

\haiku{Hij zei, dat hij er.}{niet werken kon en verliet}{Nieder-Ingelheim}\\

\haiku{Verder wist hij ook.}{niets over de kwaliteit van}{het grote gedicht}\\

\haiku{Een der paden die,;}{zijn geest heeft doorwandeld was}{ook een der mijne}\\

\haiku{Zo vertelde hij ',. '}{s avonds deze aankomst toen}{Groux er niet bij was}\\

\haiku{{\textquotedblright}{\textquoteright} Veth was een man, wiens;}{conversatie bijzonder}{interesseerde}\\

\haiku{Ook voelde zij zich.}{ongelukkig omdat zij}{geen kinderen had}\\

\haiku{zij hield zich geheel}{onbeweeglijk en langs haar}{wang zag ik langzaam}\\

\haiku{Van het begin van}{zijn verblijf te Amsterdam}{ging hij met ons om}\\

\haiku{Om twaalf uur gaf Ising.}{aan Betsy een revolver}{om af te schieten}\\

\haiku{Een houten beeld van,,.}{Vondel uit de achttiende}{eeuw stond in de gang}\\

\haiku{Hij woonde op de;}{Nieuwezijds Voorburgwal op}{twee achterkamers}\\

\haiku{Geloof aan God en.}{godsdienst heeft hij bij zijn dood}{teruggekregen}\\

\haiku{en alleen aandacht.}{had voor hetgeen in Parijs}{over hem werd gezegd}\\

\haiku{Hij bleef passief en.}{liet zich die houding van Kloos}{goedig aanleunen}\\

\haiku{Verlaine was een,,.}{rakker een stijfhoofdige}{een onwillige}\\

\haiku{Tegenover het huis.}{en verderop lagen meer}{buitenverblijven}\\

\haiku{ik vergeten, ik}{meen dat het op de linker}{Seine-oever was.556}\\

\haiku{Wanneer deze vorst,.}{een vrouw ware geweest kon}{hij niet anders zijn}\\

\haiku{Zoals ik reeds zei,;}{zag ik in Roosdorp een goed}{schrijverstalent}\\

\haiku{Het is genoeg, dat,.}{hij \'e\'en goed \'e\'en zeer goed boek}{heeft nagelaten}\\

\haiku{binnengingen en,:}{ik het lege bilart zag}{zei ik plotseling}\\

\haiku{Soms zei hij dat hij.}{ging eten met die of die en}{vroeg of ik meeging}\\

\haiku{Daarna gingen wij,.}{naar L'Isle Saint Louis naar de}{Quai de  Bourbon}\\

\haiku{hij behoefde er,.}{niet zuinig mee te zijn want}{hij had er genoeg}\\

\haiku{Hij maakte Isaac.}{nog een compliment over diens}{vloeiend Spaans-spreken}\\

\haiku{Als ik dat dronk, zei,.}{ze zou ik de volgende}{dag weer beter zijn}\\

\haiku{In geheel Spanje.}{is waarschijnlijk geen mooier}{kerkgebouw dan dit}\\

\haiku{En ook van buiten.}{dunkt mij deze kathedraal}{enigszins overladen}\\

\haiku{Als een werkman stond,.}{hij daar voor het doek deze}{kleine Hercules}\\

\haiku{Hij is volgens mij.}{niet overtroffen door die na}{hem zijn gekomen}\\

\haiku{Ik herinner mij:}{nog heel goed hoe door een der}{onzen werd gezegd}\\

\haiku{Hier moet men letten:}{op de tekenen en men}{zou kunnen zeggen}\\

\haiku{Ik word gekend door.}{de Vader en dat is mij}{ten slotte genoeg}\\

\haiku{Binnen een dag of.}{wat stel ik mij voor er u}{een cadeau te doen}\\

\haiku{110, 121, 125, 127-136,,,,,,,,,,-,.}{143 156 159 175 189 192 204}{212 291 303306 377}\\

\haiku{378, 383, 384, 386, 390-,,,,,,,:}{394 396 402 403 409 439 445}{Barr\`es Philippe}\\

\haiku{431 Belderok, A.J.:,,,, (-):}{244 245 423 424 Beljame}{Alexandre18431906}\\

\haiku{117, 250, 252-255, 260,,,,, (-):}{276 341 425 426 Diepenbrock}{Melchior17981853}\\

\haiku{117-119, 122, 132, 381,,,,, (-):}{382 387 389 393 Duinkerken}{Anton van19031968}\\

\haiku{30, 33, 34, 53-56,,, [] (-):}{73 74 76 Erens-Borghans}{grootmoeder17841867}\\

\haiku{57, 84, 180, 209, 210,,,,,,, (-):}{216 223 282 352 370 410 Vos}{Jan C. de18551931}\\

\haiku{Een van de laatste (-)}{portretfoto's van Isaac}{Israels18651934}\\

\haiku{Eins mach Zehn, / Und, /, /.}{Zwei lasz gehn Und Drei macht}{gleich So bist du reich}\\

\haiku{62In het handschrift, van,:}{8 september 1922 voegde}{Erens hierachter toe}\\

\haiku{Quoi qu'il en soit, c'est,.}{fait et il ne me reste}{qu'\`a dire c'est bien}\\

\haiku{90Gaston Dubreuilh.}{was in 1882 een jongeman}{van vijfentwintig}\\

\haiku{Vous voyez cela,.}{entre la rue Racine}{et la rue de l'Od\'eon}\\

\haiku{Je l'apprends l\`a, et.}{pour le vexer je m'occupe}{de la petite}\\

\haiku{Het blijkt vervaardigd,.}{te zijn door M. Alophe}{rue Royale 25}\\

\haiku{Hij heeft zich meen ik,.}{teruggetrokken omdat}{hij niet werd betaald}\\

\haiku{{\textquoteleft}en als u u nu,.}{helemaal ontkleedt dan laat}{mij dat nog ijskoud}\\

\haiku{301Jan C. de Vos (1855-),.}{1931 als acteur ook een groot}{karakterspeler}\\

\haiku{is haast de druppel,, '?}{daar Waar hij me\^e valt of gunt}{gem nog een poos}\\

\haiku{Het boek van Kuid en{\textquoteright},;}{God verscheen in De Nieuwe}{Gids oktober 1888}\\

\haiku{van Looy vertelde,:}{op 10 april 1890 in een brief}{aan Willem Witsen}\\

\haiku{Hij lag erg klein, net,.}{een beestje zoals Isaac}{Israels dat noemde}\\

\haiku{Vermoedelijk stond.}{een en ander hem niet meer}{helder voor de geest}\\

\haiku{{\textquoteleft}Fran\c{c}ois / doe me het / /.}{genoegen en laat onder}{geen pretext Juffr}\\

\haiku{520In het onder 517.}{genoemde boek valt zijn naam}{geen enkele maal}\\

\haiku{556Odilon Redon (-).}{18401916 woonde toen in de}{rue St. Romain 20}\\

\haiku{558Onder de schuilnaam:}{J. Staphorst schreef Jan Veth over}{Odilon Redon in}\\

\haiku{Er was een slapte.}{ingetreden en die hield}{nog enige tijd aan}\\

\section{Henrica van Erp}

\subsection{Uit: Kroniek van Vrouwenklooster in De Bilt}

\haiku{wij schonken die schout.}{met die buer die daer bij}{waren een vat bier}\\

\haiku{hij was in groote noot.}{op zee en sterf in Spangen}{op den 27 dach}\\

\haiku{'s Middags omstreeks.}{\'e\'en uur verbrandde het huis}{van onze priester}\\

\haiku{Een huis met een man.}{en vrouw en vijf kinderen}{erin dreef daar weg}\\

\haiku{De Schalkwijkse.}{wetering slibde voor een}{groot deel met zand dicht}\\

\haiku{Het ingezaaide.}{koren kwam onder water}{te staan en bedierf}\\

\haiku{de borgers van Utrecht]}{houden haar biscop buyten de}{stad ~ Anno 1527}\\

\haiku{Edoch wy brochten al}{ons beste goed en beesten}{tot Amersfoort ende}\\

\haiku{waren sy binnen,.}{Vyanen tot Bueren}{by hare ouders}\\

\haiku{Ende daer word die}{cloosteren voorgehouden}{dat sy verdingen}\\

\haiku{Ende hij lag 't.}{huys op St. Janskerkhof in}{des domdekens huys}\\

\haiku{tocht van Marten van]}{Rossem in Brabant ~ Int}{jaer Ons Heeren 1542}\\

\haiku{Die camenieren.}{hebben 5 hoorns guldens voor}{die bruyt klederen}\\

\haiku{Al die andere[].}{kneghs en maagden elx voor}{t een braspenning}\\

\haiku{Dinslo kwamen al;}{in 1277 en 1279 in handen}{van Vrouwenklooster}\\

\haiku{In 1506 vertrok hij.}{naar Spanje om zijn rechten}{te laten gelden}\\

\haiku{dit is dat gelt dat}{wij ghebuert hebben tot}{dat recht van saeken}\\

\haiku{Hij ging al vroeg over;}{tot de partij van hertog}{Karel van Gelre}\\

\haiku{86Bovenkleed voor.}{mannen met wijde slippen}{onder de gordel}\\

\haiku{Henrica van Erp.}{verantwoordt op deze plaats}{de 18 stuivers niet}\\

\haiku{143Mogelijk gaat;}{het hier om een zekere}{hopman Belrebus}\\

\haiku{Gijsbertss[oen] scout op[]}{Die Bilt aen meyster Peter}{Goesen woenen}\\

\haiku{Item die metzelaers[].}{scheenck ons convent}{een halliff aem wijns}\\

\haiku{van Bommel of van (),.}{Boemel{\textdagger}1549 waarschijnlijk te}{Bommel geboren}\\

\section{P.N. van Eyck}

\subsection{Uit: Opgang}

\haiku{Ik sloot mijn oogen en.}{zag niets dan de rozige}{helheid der leden}\\

\haiku{wanneer ik hem eens.}{zal ontvangen hebben als}{een roerlooze rijkdom}\\

\haiku{Het is goed in mij,.}{daar ik zijn kracht als een lust}{voel in mijn zwakheid}\\

\haiku{Men is een van hen:}{die haar bruising dwingt tot een}{zoete slavernij}\\

\haiku{De natuur is te}{groot dan dat zij rijker kon}{worden als de mensch}\\

\haiku{Zij geschiedt langzaam,,.}{onder verschrikkelijke}{krampen maar geschiedt}\\

\haiku{Daarom ook k\'unnen,.}{wij niet meer te gronde gaan}{als die voorgangers}\\

\haiku{een glans, hing het te.}{ademen in de glans waar het}{langzaam in verging}\\

\chapter[12 auteurs, 1228 haiku's]{twaalf auteurs, twaalfhonderdachtentwintig haiku's}

\section{Johan Fabricius}

\subsection{Uit: Leeuwen hongeren in Napels}

\haiku{de branding hief het;}{op en slingerde het weer}{in z'n koers terug}\\

\haiku{Mangia-tutto;}{wilde men heelemaal niet}{meer laten heengaan}\\

\haiku{na haar zes panters;}{had zij ook hem reeds onder}{haar ban gekregen}\\

\haiku{het bewijs daarvan.}{vormde de halfleege tent van}{den Zaterdagavond}\\

\haiku{Het terrein was nu,.}{tenminste halfweegs droog dank}{zij zijn ingrijpen}\\

\haiku{{\textquoteleft}En al die wagens,?}{dan die er elken dag maar}{komen aanrijden}\\

\haiku{Dadelijk  zou.}{de ellende van voren}{af aan beginnen}\\

\haiku{Een ander circus.}{zou er toevallig verlet}{om moeten hebben}\\

\haiku{En durft geen enkel...?}{circus het meer aan wat baat}{Saul dan zijn contract}\\

\haiku{{\textquoteright} Gottfried Sturm knikte,.}{langzaam alsof hij het zelf}{niet meer geloofde}\\

\haiku{Toen bleek eensklaps, dat.}{de Sardijnsche slager niet}{wilde teekenen}\\

\haiku{Zoo ging met dezen;}{gezamelijken inkoop}{geruime tijd heen}\\

\haiku{het scheen hem zelf toe.}{alsof hij al jaren lang}{met hen verkeerde}\\

\haiku{zij vertelden hem,;}{van moeilijke dagen die}{achter hen lagen}\\

\haiku{tegen het einde.}{van het leeuwennummer liep}{alles voorbeeldig}\\

\haiku{Ook aan het meisje.}{in de loge dacht hij daar}{tusschen door wel weer}\\

\haiku{Bedrukt, verward zocht.}{Gottfried Sturm nog weer naar een}{eervollen aftocht}\\

\haiku{- Ja... en hijzelf kwam.}{er dan natuurlijk niet meet}{voor in aanmerking}\\

\haiku{Rambaldo fronste.}{het voorhoofd en dacht snel en}{ingespannen na}\\

\haiku{Toen keek hij in zijn,.}{actentasch boordevol van}{circusgeheimen}\\

\haiku{Een weck was zeven.}{dagen en elke dag kon}{nog redding brengen}\\

\haiku{Hij was de gansche.}{zaak plotseling een weinig}{moede geworden}\\

\haiku{{\textquoteleft}Als u al optreedt,,!}{signor Saul maakt u het dan}{tenminste bekend}\\

\haiku{Deze gaf het weer,;}{aan Grazia door die juist met}{Gi-gi speelde}\\

\haiku{Bij het afscheid keek.}{zij hem nog eenmaal aan met}{haar Sphynxenblik}\\

\haiku{maar zijn leeuwen hier,.}{laten weghalen dat was}{nog weer wat anders}\\

\haiku{{\textquoteright} Rambaldo voelde.}{zich teleurgesteld en wist}{zelf niet goed waarom}\\

\haiku{En wij  hoeven,.}{alleen maar te zorgen dat}{ze hun thrill krijgen}\\

\haiku{{\textquoteright} Hij lachte zelf om.}{den goeden mop en sloeg Saul}{gul op den schouder}\\

\haiku{{\textquoteleft}Nou, en wat wou die,?}{vent nou van me die niet de}{burgemeester was}\\

\haiku{Levend vee en ook,.}{vleesch als ze aan boord een}{koelkamer hebben}\\

\haiku{Vroeger gaf mama,}{minder om zulke dingen}{maar sinds ze signor}\\

\haiku{Waarop Jeffries hem.}{van den tooverdrank met opium}{er in vertelde}\\

\haiku{{\textquoteleft}Voor den donder, neem,!}{me dan toch ook zoo zonder}{m'n zeeleeuwen}\\

\haiku{allen klopten hem.}{troostend op den schouder en}{vulden zijn glas weer}\\

\subsection{Uit: Venetiaans avontuur}

\haiku{Dat wil zeggen, in,,:}{de annonce waarop hij}{zich aanmeldde stond}\\

\haiku{{\textquoteright} De jongeman ziet,.}{hem afwachtend aan of er}{nog meer zal volgen}\\

\haiku{{\textquoteleft}Ons gesprek, mijnheer....}{is als een illustratie}{bij mijn artikel}\\

\haiku{{\textquoteleft}Hoe lang nog zal de;}{machtwaan van de werkgever}{kunnen voortwoeden}\\

\haiku{Het is stilzwijgend,.}{begrepen in het maandloon}{dat wij ontvangen}\\

\haiku{{\textquoteleft}Ik zelf duid het vers,...}{nog anders dan het hier is}{aangewend mijnheer}\\

\haiku{hij staat toch nog maar,.}{aan de overkant en tussen}{hen is het leven}\\

\haiku{{\textquoteleft}Daar moet en zal een,{\textquoteright},.}{eind aan komen zegt Stefan}{Kleingeld de jongste}\\

\haiku{misschien ook is de.}{lucht vandaag bijzonder zwaar}{van geuren en groei}\\

\haiku{{\textquoteleft}Mogen we voor dit?}{alles ook zo iets als een}{verklaring horen}\\

\haiku{Als ik dan niet slaag,.}{weet ik niet hoe het nog met}{me eindigen zal}\\

\haiku{Ik heb er nu geen,{\textquoteright}.}{aardigheid meer aan antwoordt}{Pepi onhandig}\\

\haiku{ze knipogen tegen,}{de meisjes die verwonderd}{om de hoek komen}\\

\haiku{{\textquoteleft}Hij moet zaterdag,{\textquoteright}.}{toch z'n vakantie hebben}{zeggen de pakkers}\\

\haiku{Een conducteur, die,,...}{als Walther juist wegdoezelt}{het biljet wil zien}\\

\haiku{Wie meer bagage,.}{heeft dan hij dragen kan valt}{hun weerloos ten prooi}\\

\haiku{Dan ontpopt zich de;}{hotelportier echter als}{redder in de nood}\\

\haiku{thans weer schijnt hem het:}{allerbelangrijkste wat}{hij zou kunnen doen}\\

\haiku{Niemand hier in de.}{Trattoria della Rosa}{interesseert het}\\

\haiku{het is namelijk.}{net alsof hij die vroeger}{al eens gezien heeft}\\

\haiku{Misschien heeft bij het!}{morgen al niet meer nodig}{om te rekenen}\\

\haiku{Zijn taille komt er,.}{goed in uit en zijn schouders}{lijken er fors in}\\

\haiku{maar zo'n jan- en!}{allemansfoto had ik}{van u niet verwacht}\\

\haiku{{\textquoteright} Lachend vat zij haar,:}{vriendin onder de arm en}{zegt haar meetrekkend}\\

\haiku{het komt vanuit de,;}{diepte alsof het uit het}{plaveisel opsteeg}\\

\haiku{Het lokt hem weinig.}{meer om naar de piazza}{terug te keren}\\

\haiku{- Het zou een flink gat,,...}{in zijn reisbeurs slaan maar wie}{niet waagt wie niet wint}\\

\haiku{Zijn avondkleding zal,...!}{hem de moed schenken op haar}{af te gaan vervloekt}\\

\haiku{- \'als je er een neemt,,?}{zal het toch alleen maar van}{de vrouw zijn niet waar}\\

\haiku{Zeg nu maar, dat u,!}{komt dan sturen we Bill om}{op u te wachten}\\

\haiku{Ze noemen hem in,:}{volgorde hun namen en}{Walther herhaalt ze}\\

\haiku{dat zou veel te saai!}{zijn als het altijd maar op}{een kus uitdraaide}\\

\haiku{Zoals wij zeggen,.}{zo is het spel en zo is}{het altijd geweest}\\

\haiku{{\textquoteright} {\textquoteleft}Dad is vanavond niet,{\textquoteright}.}{uitgenodigd stellen de}{meisjes hem gerust}\\

\haiku{Het komt hem echter,.}{verdacht voor dat zij hem zo}{lang laten wachten}\\

\haiku{{\textquoteleft}Ik ben ook verliefd,{\textquoteright}.}{bezweert hij en rilt erbij}{over zijn ganse lijf}\\

\haiku{Hij zal er trots op,:}{zijn dat hij een dichter in}{zijn familie krijgt}\\

\haiku{En juist mij gaf je...}{geen arm toen we daar zo door}{de stad rondliepen}\\

\haiku{Zij veegt ze vluchtig,:}{weg en troost zonder er zelf}{in te geloven}\\

\haiku{Heeft hij werkelijk,?}{ernstig gemeend op een van}{hen verliefd te zijn}\\

\haiku{Wat later buitelt}{hij in de branding tussen}{een aantal bruine}\\

\haiku{Intussen vindt hij {\textquoteleft}!}{de nieuwe grap uit om op}{elke vraag metwau}\\

\haiku{{\textquoteright} Dan zet hij zich voor.}{het kleine tafeltje en}{begint te schrijven}\\

\haiku{Enige tijd zit hij.}{zonder aan iets te denken}{en drinkt zijn koffie}\\

\haiku{Hardop herhaalt hij,.}{nog weer eens dat hij immers}{niet meer terug kan}\\

\haiku{hij weer vervalt als {\textquoteleft}{\textquoteright} {\textquoteleft}{\textquoteright}.}{hij ookDear wegwist en er}{Dearest van maakt}\\

\haiku{{\textquoteright} Daarna beveelt hij.}{zich nog voor een volgende}{dans aan en eclipseert}\\

\haiku{{\textquoteright} roept Walther en klapt,.}{passieloos in de handen}{zoals allen doen}\\

\haiku{Lilo beleeft van.}{deze gestolen dans ook}{al geen genoegen}\\

\haiku{{\textquoteright} zegt het meisje in.}{een volkomen Duits en wijst}{blozend op Walther}\\

\haiku{dan moet ze zich op.}{de lippen bijten om het}{niet uit te proesten}\\

\haiku{{\textquoteright} vraagt ze als Bibi.}{zich vlak voor haar voeten in}{het zand rondwentelt}\\

\haiku{Toen stond voor een man;}{met durf en lust tot avontuur}{de wereld nog open}\\

\haiku{Die gids vertelt het...?}{daar net aan zijn horde 70}{pas in de lengte}\\

\haiku{50 maal 25 maal 15...}{meter zou een kubieke}{ruimte geven van}\\

\haiku{Zo, daarom is de:}{kerel dus al de hele}{dag achter hem aan}\\

\haiku{daar leeft ze in 't,,;}{verborgen als een kleine}{arme studente}\\

\haiku{- Hoe laat arriveert,?}{de postkoets uit D. als ik}{u verzoeken mag}\\

\haiku{{\textquoteleft}Ik zal me op de.}{een of andere wijze}{wel zien te redden}\\

\haiku{hij wendt zich naar haar.}{om en doet een poging om}{haar toe te lachen}\\

\haiku{zij is moe en wil):}{ook nog een brief schrijven vraagt}{zij hem onverwachts}\\

\haiku{{\textquoteright} {\textquoteleft}Maar wat drommel, het,!}{gaat er toch niet om of er}{op hen gepast wordt}\\

\haiku{hij zelf denkt er niet.}{aan om zich voor Bibi in}{het touw te spannen}\\

\haiku{Hij is zo lief, kijk,;}{dan en hij wil zo graag mooie}{schilderijen zien}\\

\haiku{{\textquoteleft}U weet toch wel ma,...}{dat ik niet veel geef om al}{die oorlogsdingen}\\

\haiku{Als dit niet zo zou,...}{zijn ware het in hoge}{mate opwindend}\\

\haiku{{\textquoteleft}Maar natuurlijk, zelfs...{\textquoteright}}{dan zou u in 1918 nog maar}{een kind zijn geweest}\\

\haiku{Als zij samen bij,:}{ma terugkeren zijn de}{rollen verwisseld}\\

\haiku{Zij doet het aardig,,.}{en kinderlijk zo dat het}{eerder vleit dan kwetst}\\

\haiku{... zet morgen iets op...}{haar hoofd wat nog nooit op een}{hoofd gezeten heeft}\\

\haiku{{\textquoteright} Ma en Miep kijken;}{naar de diva en trachten}{dit te verwerken}\\

\haiku{Let op en schreeuw niet, -.}{ik ga je in je eigen}{belang opereren}\\

\haiku{De anderen zijn,{\textquoteright}.}{naar de Lido gegaan deelt}{Walther hem mede}\\

\haiku{{\textquoteleft}Als jij er me brengt, ',{\textquoteright}.}{zal ik wel zeggen of het}{t goeie is zegt hij}\\

\haiku{Het is onbillijk.}{om daar dan toch een oordeel}{over te verlangen}\\

\haiku{Zou je werkelijk,?}{durven volhouden dat dat}{alles je niets zegt}\\

\haiku{Zij houdt ermee op,;}{en zegt dat zij te vermoeid}{is om te dansen}\\

\haiku{Nu komt Doug als een;}{vluchtige donkere schim}{uit de deuropening}\\

\haiku{Hij trekt zijn zakdoek.}{te voorschijn en veegt er zich}{het voorhoofd mee schoon}\\

\haiku{hij had helemaal.}{niet anders kunnen springen}{dan hij gedaan heeft}\\

\haiku{hij moet zich haasten,.}{daar zij zich reeds vlak onder}{de wal bevinden}\\

\haiku{Bitter ontnuchterd,;}{wacht hij af wat er verder}{nog gebeuren zal}\\

\haiku{Als hij zijn ogen weer,;}{opent valt hem op hoe duister}{het geworden is}\\

\haiku{Marcolina ligt.}{met het bovenlijf over de}{tafel te snikken}\\

\haiku{Voor Marcolina,;}{was het te hopen dat dit}{maar gauw gebeurde}\\

\haiku{daglicht naar zee doen,.}{afdrijven en de vissen}{zorgden voor de rest}\\

\haiku{alsof hij gedoemd,}{is deze handen aan zijn}{lijf mee te dragen}\\

\haiku{Zwijgt hij tegenover,.}{de waard dan heeft hij hier nog}{twee dagen krediet}\\

\haiku{Vannacht moet ze als!}{een kwade meid onder mijn}{dak zijn weggevlucht}\\

\haiku{Ze bevallen hem.}{geen van beiden nu hij ze}{in het gelaat ziet}\\

\haiku{Hij verlangt naar ogen,;}{die in \'e\'en blik doorzien wat}{hij geleden heeft}\\

\haiku{Hij schijnt te denken,!}{dat hij hier voor zijn plezier}{op de uitkijk staat}\\

\haiku{Hij heeft nu genoeg.}{van dit plein en van deze}{mensen om hem heen}\\

\haiku{Kan hij geen stuk van?}{zich zelf verkopen om bij}{Peggy te komen}\\

\haiku{Dat zou mooi zijn als.}{de meneer hem zelf in het}{mandje kon leggen}\\

\haiku{Het hoofd natuurlijk.}{weer in diep gepeinzen naar}{de grond gebogen}\\

\haiku{En jij zult lachen,,.}{om dingen die mij hebben}{doen schreien Peggy}\\

\section{Louis Ferron}

\subsection{Uit: Turkenvespers}

\haiku{{\textquoteright} {\textquoteleft}De stoplichten niet,{\textquoteright}.}{meegerekend probeerde}{ik hem te troosten}\\

\haiku{Heb je eigenlijk?}{wel een idee waarom juffrouw}{Kamenow kaarsen brandt}\\

\haiku{Hun braaksel was zo.}{groen als de wouden die de}{akkers omzoomden}\\

\haiku{Hij legde zijn hand.}{op haar knie en knikte haar}{bemoedigend toe}\\

\haiku{Zes vrouwen had hij,.}{ge\"exploiteerd vijf had hij}{er doodgeranseld}\\

\haiku{Ik dacht, welke gek,.}{heeft mij eens verwekt daar in}{de Dorotheeengasse}\\

\haiku{Ze waren arm maar.}{jong en dat leek een reden}{om vrolijk te zijn}\\

\haiku{Maar omdat ik toen,.}{nog niet wist wat een sfinx was}{vergat ik het weer}\\

\haiku{Hammen vlogen uit,,.}{het keukenraam vergieten}{deksels en pannen}\\

\haiku{Het bloed spoot uit zijn,}{oren zijn neus en zijn mond en}{nog vraag ik me af}\\

\haiku{Ook mijn stiefvader.}{werd op de grond gesmeten}{en vastgebonden}\\

\haiku{Iemand vloekte toen.}{hij met zijn voet achter een}{wortel bleef haken}\\

\haiku{Ik had alleen nog.}{maar de nacht en de sterren}{om naar te kijken}\\

\haiku{Waarom moest ik juist,,?}{toen piekerend in die boom}{aan die film denken}\\

\haiku{{\textquoteright} De man blies een pluis.}{van zijn mouw en nam me nog}{eens aandachtig op}\\

\haiku{Nee, het was echt geen.}{inbeelding dat ik overal}{verval bespeurde}\\

\haiku{{\textquoteright} {\textquoteleft}Voor een kunstenaar,.}{hebt u wel een zeldzaam hard}{gemoed jonge vriend}\\

\haiku{Wat deugt er niet aan,?}{mij dat mensen van vlees en}{bloed mij ontwijken}\\

\haiku{{\textquoteright} {\textquoteleft}Ach,{\textquoteright} zei ik, {\textquoteleft}het zijn.}{de besten niet die op het}{uiterlijk afgaan}\\

\haiku{{\textquoteleft}Let op, er komt een.}{dag dat ze niet langer om}{me zullen lachen}\\

\haiku{Ik leerde de geur.}{van patchouli onderscheiden}{van die van muskus}\\

\haiku{Ik dacht, vliegend fort,.}{of Lancaster dat zou er}{een mooie naam voor zijn}\\

\haiku{Ik kon de vlieger.}{duidelijk zien zitten in}{zijn glazen koepel}\\

\haiku{{\textquoteleft}Alma,{\textquoteright} zei ik de,}{volgende dag toen Korngold}{naar de opera was}\\

\haiku{Eens zou een minnaar.}{haar vereeuwigen in een}{onthutsend portret}\\

\haiku{Ze sloeg haar armen,.}{om mijn middel haar handen}{streelden mijn liezen}\\

\haiku{{\textquoteright} Of, {\textquoteleft}jammer genoeg.}{is de techniek nog niet ver}{genoeg gevorderd}\\

\haiku{Als ik hoestte, klonk.}{mijn hoest pas als ik weer op}{adem was gekomen}\\

\haiku{Verliefd wordt men op.}{vrouwen wier voorkomen men}{niet beschrijven kan}\\

\haiku{{\textquoteleft}Dit had ik kunnen,{\textquoteright},.}{weten zei de stem die de}{stem van Korngold was}\\

\haiku{Een Zigeunerkind.}{was ik en het zou nog slecht}{met me aflopen}\\

\haiku{En ik vroeg me af.}{welke vorm die gedachten}{konden aannemen}\\

\haiku{Die jongeman joeg.}{zijn centen er al net zo}{vlot door als zijn neef}\\

\haiku{Wat heeft nog waarde,?}{in deze tijden wat mag}{nog een naam hebben}\\

\haiku{Tot mijn verbazing.}{rook zijn adem niet naar drank maar}{naar inkt en papier}\\

\haiku{Wat zouden de  ?}{meest geheime wensen van}{deze Freiherr zijn}\\

\haiku{{\textquoteleft}Goed,{\textquoteright} vervolgde Sayn, {\textquoteleft}.}{dan zult u het ook met me}{eens zijn dat \ensuremath{\sum}ipij=1}\\

\haiku{{\textquoteright} {\textquoteleft}O, zeker,{\textquoteright} viel ik, {\textquoteleft},.}{Sayn bijheel onschuldig geen}{onvertogen woord}\\

\haiku{wat liefde was, dan.}{zouden ze zich toch ook}{niet zo vergooien}\\

\haiku{{\textquoteright} {\textquoteleft}O, je houdt je van,,}{den domme Kaspar je weet}{het allemaal best.}\\

\haiku{Zo werkt dat systeem.}{dat iedereen op de plaats}{houdt waar hij thuis hoort}\\

\haiku{Ik had me al te.}{zeer laten meeslepen door}{mijn vooroordelen}\\

\haiku{Ik houd zo van je,,.}{Kaspar waarom wil je dat}{toch niet begrijpen}\\

\haiku{Daarna Wein, Weib und,,.}{Gesang Donauwellen het}{daverde maar door}\\

\haiku{{\textquoteright} {\textquoteleft}Zo wil de keizer,.}{het nu eenmaal hij wil dat}{we plezier hebben}\\

\haiku{Die opmerking scheen.}{zijn vermoedens alleen maar}{te bevestigen}\\

\haiku{Wij hebben altijd.}{ons best gedaan voor hen die}{over ons zijn gesteld}\\

\haiku{{\textquoteright} Buiten bulderden.}{de kanonnen en ik zocht}{naar zwavelstokken}\\

\haiku{{\textquoteright} En mijn stem dempend, {\textquoteleft}?}{Denkt u dat we het daar nog}{mee redden zullen}\\

\haiku{{\textquoteright} {\textquoteleft}Nee, mijnheer, maar ik.}{zou het met de nagel van}{mijn duim kunnen doen}\\

\haiku{{\textquoteright} Hij spuwde op de,.}{grond gorde zijn mand weer om}{en liep van ons weg}\\

\haiku{De hongerende.}{bevolking schraapte kalk en}{gips van de muren}\\

\haiku{Ik kon daar uren over.}{peinzen op mijn kamer en}{ik moest dat ook wel}\\

\haiku{En welke artsen?}{sprongen het slordigst met de}{hygi\"ene om}\\

\haiku{{\textquoteright} {\textquoteleft}En ik heb last van.}{huidverkleuringen en mijn}{tanden vallen uit}\\

\haiku{\v{C}elinek liet een.}{onderzoek instellen dat}{niets opleverde}\\

\haiku{Een vertwijfelde.}{dokter \v{C}elinek stormde}{de jachtzaal binnen}\\

\haiku{Met het zwellen van,.}{haar buik groeide haar gezag}{in de omgeving}\\

\haiku{De hygi\"ene.}{in het hospitaal nam er}{echter niet door toe}\\

\haiku{{\textquoteright} {\textquoteleft}Als wij het hier voor,.}{het zeggen hebben krijg je}{misschien nog wel meer}\\

\haiku{{\textquoteright} {\textquoteleft}Kunst, kunst aan mijn broek,.}{niets dan geiligheid waar jij}{mee loopt te leuren}\\

\haiku{Met juffrouw Kamenow.}{had ik nog nooit iets gehad}{en zij was zwanger}\\

\haiku{{\textquoteleft}Hij ademt,{\textquoteright} zei ze op, {\textquoteleft},.}{haar buik kloppendhij trappelt}{hij roept zijn moeder}\\

\haiku{{\textquoteleft}Je moet hem met U,.}{aanspreken hij is niet de}{eerste de beste}\\

\haiku{Dat was de Habsburgse.}{horoscopie waaraan ik}{onderworpen was}\\

\haiku{Vol afschuw stootte.}{ik hem van me af en hij}{viel zwaar op de grond}\\

\haiku{wie doorkneed was in.}{de anti-logica die}{hier ontwikkeld werd}\\

\haiku{Eynhuf had niet het.}{flauwste vermoeden van wat}{er in mij omging}\\

\haiku{Mackensen nam op.}{en begon een lang gesprek}{in een vreemde taal}\\

\haiku{De situatie.}{was omgeklapt en Kunz was}{zichzelf gebleven}\\

\haiku{Hij is op hun hand.}{en bazelt daarom maar wat}{over negermuziek}\\

\haiku{{\textquoteright} {\textquoteleft}Ach, excellentie,...}{als ik u dat allemaal}{moet gaan uitleggen}\\

\haiku{En wat hadden Kunz?}{er mee te maken en de}{Edler von Eynhuf}\\

\haiku{{\textquoteright} {\textquoteleft}Jou staat op zijn best,{\textquoteright}.}{de dood door verzakking te}{wachten sneerde ik}\\

\haiku{De zwart gelakte.}{marskramersmand op zijn rug}{hinderde hem zeer}\\

\haiku{Daarna moesten we in.}{groepjes van twee de lijken}{de kerk in dragen}\\

\haiku{De driften komen,{\textquoteright}, {\textquoteleft}.}{zoals ze gaan zei hijals}{een dief in de nacht}\\

\haiku{De oude knikte,.}{goedkeurend maar ik deed of}{ik het niet merkte}\\

\haiku{Zoon van een prins van{\textquoteright}.}{Baden schreven de boeken}{over mijn naamgever}\\

\haiku{Hij had zijn leven.}{in handen gelegd van de}{verdoemde machten}\\

\haiku{Het paradijs kon,.}{alleen werkers gebruiken}{geen pati\"enten}\\

\haiku{Ik dacht, in dit kind.}{moet alles gebeuren wat}{mij onthouden is}\\

\haiku{Dat liet hij, zei hij,.}{aan anderen over die daar}{in geschoold waren}\\

\haiku{{\textquoteright} Ik kuste het kind.}{op de fontanel en viel}{in een diepe slaap}\\

\haiku{Als men op zijn hoofd.}{ging staan kon men zien hoe de}{prins naar zijn hart greep}\\

\haiku{als thuis en ik zie.}{waarachtig niet in wat daar}{om te lachen valt}\\

\haiku{Tussen de spijlen.}{van de kroon stonden mensen}{met verrekijkers}\\

\haiku{Met zijn vlakke hand.}{klopte de man het ritme}{op de tafel mee}\\

\haiku{De man die Kaspar.}{heette knikte en zei dat}{ze voort moesten maken}\\

\section{Juul Filliaert}

\subsection{Uit: Jan Bart}

\haiku{Zijn hoog achterdek,,.}{dat hem in den rug beschermt}{wordt weggeschoten}\\

\haiku{Eens binnenloopen ',,,.}{int Boomstraatje ja ja}{dat zou Roosje wel}\\

\haiku{Keesje loopt aan.}{zijn hand en Jantje zit op}{den arm van moeder}\\

\haiku{van dekgevechten;}{met bijlen en pistolen en}{zwaarden en messen}\\

\haiku{De plunjezak wordt,.}{toegestropt in den hoek van}{de kamer geplaatst}\\

\haiku{Jan ziet moeder, van,.}{op den deurdorpel met de}{\'e\'ene hand wuiven}\\

\haiku{Iedere prooi was,.}{hem welkom als hij ze maar}{bemeesteren kon}\\

\haiku{Dit derde deel werd.}{in een aantal gelijke}{bedragen gesplitst}\\

\haiku{Een jaartje oorlog.}{tegen den Engelschman}{kan hem geen kwaad doen}\\

\haiku{- Waarom niet, antwoordt -.}{Jan Top. De overvaart eischt}{kunde en kennis}\\

\haiku{Wij ondervinden,,.}{dat allen best. Uw vader}{ondervond dat ook}\\

\haiku{Geen wilde vaart, want.}{het weder is helder en}{de wind is slapjes}\\

\haiku{Het smaldeel dat aan,.}{bakboord afzwaaide stevent}{naar de Midwaybocht}\\

\haiku{in zijn plaats had ik '.}{zoowel de Lords alst grauw te}{roosteren gelegd}\\

\haiku{Hij bezit ook een,.}{vierde aandeel in een mooie}{boot de Sint Michiel}\\

\haiku{Jan heeft het kleine.}{bootje dadelijk in de}{gaten gekregen}\\

\haiku{Het laken mag toch?}{niet heelemaal langs eenen kant}{getrokken worden}\\

\haiku{Als we persoonlijk,.}{ons zelf blijven dan gaat niet}{alles verloren}\\

\haiku{Langs den neus en op,.}{den rug van anderen haalt}{hij zijn oogst binnen}\\

\haiku{Jan Bart krijgt deze.}{gevaarlijke opdrachtjes}{voor zijn rekening}\\

\haiku{Hij was daarbij heel.}{anders dan de gewone}{zeeman aangelegd}\\

\haiku{Wat verdienste kon?}{er ook gemaakt worden aan}{boord van zoo'n prulding}\\

\haiku{Deze wendt het hoofd,.}{om doet de manschap teeken}{nog wat te wachten}\\

\haiku{begint er daar reeds '.}{een aant einde van de}{tafel te zingen}\\

\haiku{Waardin Martien Van,,.}{den Broucke vrouw Gouthiere}{glanst van voldaanheid}\\

\haiku{De wezens gloeien.}{hoogrood en glimmen boven}{de blauwe baais uit}\\

\haiku{In koortsachtige.}{gejaagdheid worden alle}{zeilen geheschen}\\

\haiku{Een lading Spaansche,.}{wijn die op een Engelsche}{smack werd veroverd}\\

\haiku{Wanneer ze, bijna,.}{boord aan boord zijn schiet Jan een}{laatste salvo af}\\

\haiku{Lassijn staat aan 't,,,.}{roer om met zijn buit Koning}{David te volgen}\\

\haiku{In het dekgevecht,,.}{dat een uur aanduurt weert}{hij zich als een leeuw}\\

\haiku{Daar waren nog de.}{bezoeken op de Groenplaats}{en in de Boomstraat}\\

\haiku{Haast geen tijd om met.}{Nicole een luchtje in}{de stad te scheppen}\\

\haiku{Met de roeibooten '.}{komen de schippers aan boord}{vant kaperschip}\\

\haiku{Zoodoende halen,.}{ze de geleden schade}{wellicht ruimschoots in}\\

\haiku{De killigheid van.}{de zieltjesweek druilt reeds over}{pleinen en haven}\\

\haiku{Wakte en natte '.}{dringen langs zijn broekbeurzen}{tot opt bloote vel}\\

\haiku{Mannen, in Godsnaam, '.}{t Laatste dat ge voor uw}{kapitein kunt doen}\\

\haiku{- Ik voel me alleen,,.}{gelukkig tevreden en}{voldaan Nicole}\\

\haiku{- Als men verloofd is,.}{dan stapt men fier en trotsch aan}{den arm van den man}\\

\haiku{- Maar, gij moet dan ook!}{begrijpen dat uw vrouw toch}{meer is dan uw schip}\\

\haiku{Zijn samenwerking.}{met de kapiteinmaats is}{winstgevend geweest}\\

\haiku{Er is pinkeling,.}{van licht dat wemelt in de}{plooien van de zee}\\

\haiku{De Neptunus is.}{ook al omgezwenkt en lost}{een tweede slavo}\\

\haiku{De twee mannen staan.}{tegenover elkander als}{razende honden}\\

\haiku{Ze gelijken twee,.}{kampers die een bloedige}{veete beslechten}\\

\haiku{- Stand houden, gilt en,, '.}{hijgt Cuyper die stuiptrekkend}{wentelt opt dek}\\

\haiku{Zij, immers, moeten,.}{vaart zee en oorlog niet meer}{leeren of leeren kennen}\\

\haiku{Ik heb ze van over.}{de tafel gehaald om haar}{een zoen te geven}\\

\haiku{En ten slotte nog,;}{twee en dertig zakken geld}{250 stuks gouden munt}\\

\haiku{De andere zijn?}{dan ook bij den kapitein}{terecht gekomen}\\

\haiku{Daar was wel goud op,.}{De Pelikaan maar toch geen}{zulke geschenken}\\

\haiku{Inmiddels wordt zijn.}{eerste klas fregat Mats in}{gereedheid gebracht}\\

\haiku{Wat zegt u, van zoo'n,,?}{kapitein als Bart meneer}{de havenbaljuw}\\

\haiku{de taal in eere.}{en aanzien te houden en}{te doen voortleven}\\

\haiku{Hij voelde zich een.}{soort hoofdman zonder bezit}{en zonder werkveld}\\

\haiku{Vauban,  trouwens,.}{bevond zich een beetje in}{het geval van Bart}\\

\haiku{Reeder spelen in,,.}{de politiek kan nooit geen}{kwaad vooral nu niet}\\

\haiku{Hij ziet toch wel, dat '?}{Duinkerke de stapelplaats}{vant Noorden wordt}\\

\haiku{Jammer dat Jan het,!}{vertikt baas en kapitein}{aan wal te spelen}\\

\haiku{Jan verdient het, want.}{hij is de uitblinker in}{de bloedverwantschap}\\

\haiku{Onder de kristen.}{slaven zijn verschillende}{stamgenooten}\\

\haiku{De nagalm zindert.}{weg in een vreugdetrilling}{of in smartgevoel}\\

\haiku{In de woonkamer.}{had hij de meesters angstig}{om bescheid verzocht}\\

\haiku{Dood had hij kunnen,.}{zaaien dood overwinnen lag}{buiten zijn bereik}\\

\haiku{zeemansrantsoen op,.}{zijn Vlaamsch dertig centimes}{per dag en per kop}\\

\haiku{In het groote huis op.}{de Groenplaats wil en kan hij}{niet meer verblijven}\\

\haiku{hij heeft oog en oor.}{open voor alle verlangens}{en alle wenschen}\\

\haiku{Twee dagen daarna,,.}{brengt Jan te Duinkerke zijn}{eersten buit binnen}\\

\haiku{Eigenlijke storm,.}{is  het nog niet wel de}{voorbode ervan}\\

\haiku{Bang is hij niet, maar.}{zoo'n wiegedans heeft hij nog}{niet medegemaakt}\\

\haiku{Met 't klaren van.}{den morgen verspringt de wind}{in een ander gat}\\

\haiku{Aan den grooten mast.}{laat Jan het signaal van den}{terugtocht hijschen}\\

\haiku{Patoulet had de.}{zaken op grootscheepsche}{wijze aangepakt}\\

\haiku{Maria liep toen, als,.}{driejarig kindje nog in}{haar eerste rokske}\\

\haiku{Een alledaagsch en.}{doodgewoon voorval in het}{boordleven aan wal}\\

\haiku{Van nu voortaan werd.}{ze losser en vrijer in}{haar doen en laten}\\

\haiku{men wandelt niet steeds,.}{met zijn heele zeemanshart}{de kaai op en af}\\

\haiku{Waarom was uw blik,,,?...}{z\'oo z\'oo droomerig-koel}{zoo hard dezen avond}\\

\haiku{De strooming van '.}{t water helpt den roeier}{in zijn zware taak}\\

\haiku{- Het vel van den beer.}{niet verkoopen vooraleer}{aan land te wezen}\\

\haiku{De boorden van zijn.}{hemd plooit hij tot boven de}{ellebogen op}\\

\haiku{Een oogenblik taakt,.}{ze het watervlak duikelt}{dan weg in de zee}\\

\haiku{Indien ik hier aan ',.}{t stuur moet blijven zitten}{dan val ik in slaap}\\

\haiku{Beweging zal me,.}{niet alleen wakker houden}{maar tevens deugd doen}\\

\haiku{Ze drinken rhum om,.}{zich op te monteren om}{wakker te blijven}\\

\haiku{Ze moeten wakker,.}{blijven willen ze in hun}{poging gelukken}\\

\haiku{Zestien dagen lang.}{bijna ter plaatse blijven}{draaien en laveeren}\\

\haiku{{\textquoteleft}Herinner U wie,.}{de oude Tromp De Ruyter}{en Duquesne waren}\\

\haiku{Waarom negeeren ze?}{ons werk en dwarsboomen ze}{onze inzichten}\\

\haiku{De drank heeft hem op,.}{dezen tocht ten slotte er}{onder gekregen}\\

\haiku{Maar dit ongeduld.}{van den minister werkt me}{op de zenuwen}\\

\haiku{, vliegt naar de kaai en.}{doet zijn schepen tusschen de}{staketsels meeren}\\

\haiku{In de bureelen.}{van von Heine is het een}{standje van belahg}\\

\haiku{Hij toont zich trouwens.}{waardig van het vertrouwen}{dat men in hem stelt}\\

\haiku{Maar Jan krijgt ook een {\textquoteleft}{\textquoteright},:}{plunderkommissaris aan}{boord die hem dwars zit}\\

\haiku{Ze gedragen zich,.}{echter niet als prinsen wel}{als aftroggelaars}\\

\haiku{Samuel Bernard... -.}{worstelt tegen stroom op Niet}{te verwonderen}\\

\haiku{Alleen de genster '.}{is nog noodig omt vuur aan}{de lont te brengen}\\

\haiku{- Een flinke grog, vrouw,.}{om dat beestje van onder}{mijn huid te jagen}\\

\haiku{Wanneer hij 't hoofd,.}{wil oprichten krijgt hij als}{een slag in den nek}\\

\haiku{Hij hoest nu en dan,.}{hard en droog en koud zweet klamt}{onder zijn oksels}\\

\haiku{Maar desondanks blijft,.}{zijn denken helder nu is}{zelfs merkbaar verscherpt}\\

\haiku{Achter het schip stroelt.}{en schuimt het witte kielzog}{van de laatste vaart}\\

\haiku{de Fransche werken,,;}{komt Jan Bart als zeeheld op}{het voorplan te staan}\\

\haiku{Wat men over Jan Bart,.}{wilde behouden werd op}{het voetstuk geplaatst}\\

\haiku{Als hij stierf was dit.}{stuk Vlaamsch land reeds vijftig jaar}{onder Fransch beheer}\\

\haiku{De bindselriemen.}{zitten in de roeimikken}{om houtsleet te sparen}\\

\haiku{In een stroommonding,.}{ook driepikkels waar men in}{nood aanleggen kan}\\

\haiku{Iemand die op zijn:}{stoel zit te wrikkellen van}{ongeduld enz. Zootje}\\

\subsection{Uit: Tijl's oog op den puinhoop}

\haiku{In den bak gedraaid,,.}{worden om de waarheid te}{zeggen d\`at kan niet}\\

\haiku{Mijn gebuur noemt dat,.}{z\'o\'o maar een eigenlijke}{spelonk is het niet}\\

\haiku{Dit stuk muur, eindigt, '.}{afgerond int profiel}{van een menschenhoofd}\\

\haiku{De zon schittert er,.}{door den heelen dag brandt er}{door bij avondzinken}\\

\haiku{Waar kunnen we nog!}{beter zijn   Dan in ons}{moeders keuken}\\

\haiku{Vooraleer wij de,;}{stad zouden binnentrekken}{bleef de voerman staan}\\

\haiku{We zullen spoedig.}{gaan zien indien ze nog niet}{ingenomen zijn}\\

\haiku{{\textquoteright} - {\textquoteleft}Maar we huizen nu, {\textquotedblleft}{\textquotedblright} '.}{wel in een spelonk in een}{abri ent gaat ook}\\

\haiku{Wat het gemis aan!}{barakken wel duizendmaal}{vergoelijken kon}\\

\haiku{Ik vraag een barak,.}{eerst en meest omwille van}{de kleine dutsen}\\

\haiku{Moet ik u zeggen?}{dat wij als van de hand Gods}{waren geslagen}\\

\haiku{, daar is geen nood voor,.}{Nele de menschen zouden}{het toch niet gelooven}\\

\haiku{Toch nooit een koffer,?}{of zoo iets waarin een schat}{kon verborgen zijn}\\

\haiku{Had mijn spade dien,!}{kop afgestekt wie weet wat}{er zou gebeurd zijn}\\

\haiku{Een onder hen, was,,}{zoo goed toch eens mede te}{komen om te zien}\\

\haiku{De kinders mochten.}{niet buiten piepen of we}{moesten hen achterna}\\

\haiku{De obussen hadden.}{die speelzieke kinderen}{als gefascineerd}\\

\haiku{In klas vernam de. '}{meester de oorzaak van het}{te laat-komen}\\

\haiku{Die blijven slechts een.}{oogenblik truntelen en}{toeteren verder}\\

\haiku{N\`u voelen wij ook!}{den polsslag van het leven}{door de wereld gaan}\\

\haiku{Als hij mij gewaar,.}{wordt schudt hij de veeren en}{wipt de ruimte in}\\

\haiku{De zucht naar roem en;}{naar macht leidt naar verdrukking}{en onderdrukking}\\

\haiku{Zijn glinsterende.}{oogen pinkelen achter de}{glazen van zijn bril}\\

\haiku{verdwijnt dan ergens,.}{achter een stuk muur als in}{een grondeloozen put}\\

\haiku{Dat ondervonden.}{wij weldra persoonlijk met}{dat ander konijn}\\

\haiku{we weten het al,.}{dertig jaar dat een gendarm}{geen genade kent}\\

\haiku{Ze stond er nog niet,.}{half of men ondervond dat}{het misloopen was}\\

\haiku{In den ondergang:}{van een heel gewest had ze}{haar roem gehandhaafd}\\

\haiku{Want opruiming en:}{heropbouw moeten hierheen}{veel vreemd volk lokken}\\

\haiku{Dit Fransch kerkhof is.}{een begankenisplaats van}{belang geworden}\\

\haiku{Het oorlogsgeweld.}{kwam uit het Noord-Oosten}{en uit het Oosten}\\

\haiku{Het eenige dat in,.}{deze stervende stad niet}{stierf was het kerkhof}\\

\haiku{Dat wil echter niet.}{zeggen dat wij den winter}{niet hebben gevoeld}\\

\haiku{Het geluk kwam ons.}{te gemoet en we zijn het}{niet voorbijgegaan}\\

\haiku{Ons onbekommerd,.}{vrij en los bestaan heeft een}{einde genomen}\\

\haiku{Er heerschte, al,:}{dien tijd in heel de stad slechts}{\'e\'en bekommering}\\

\haiku{Het wordt te allen.}{kant een gewedijver om}{ter eerst en ter meest}\\

\haiku{niest onder  de.}{geweldige prikkeling}{en hervat de taak}\\

\haiku{{\textquoteleft}Een naaste maal, tracht!}{dat karweitje te schikken}{buiten de werkuren}\\

\haiku{Pas had ik de deur:}{geopend of Nele was}{er al met de vraag}\\

\section{Emiel Fleerackers}

\subsection{Uit: Baveloo-Boetjes}

\haiku{{\textquoteright} zei Baveloo, {\textquoteleft}dat '...,?}{s telepathie Zijn mijn}{schoenen gelapt Boetjes}\\

\haiku{en een van hen, de,...}{bugel spitst zijn ooren en}{verneemt het geheim}\\

\haiku{Boetjes zat en zocht een,, -.}{derde woord ten minste een}{tweede maar hij zweeg}\\

\haiku{of ten minste naar, '!}{een tweede woordt woord van}{de redelijkheid}\\

\haiku{Maar zij had al meer.}{gekwakt dan een gewone}{kwakkel op zes jaar}\\

\haiku{sinds die kwakkel bij,.}{u hangt werkt dat kwaksel mij}{op de zenuwen}\\

\haiku{{\textquoteright} - {\textquoteleft}Enfin, d\`at heeft hij,.}{door mijn raam gesmeten vlak}{op een schilderij}\\

\haiku{Boetjes kwam dien avond op '.}{t studeerkamertje van}{den Heer Kanunnik}\\

\haiku{{\textquoteright}... - {\textquoteleft}Ja 't is schoon{\textquoteright} - zei.}{moeder Boetjes en slikte een}{pilleke binnen}\\

\haiku{{\textquoteright} - - {\textquoteleft}Neen, Mr Pastoor{\textquoteright} - zei.}{moeder Boetjes en keek of er}{een zot in huis stond}\\

\haiku{maar hij behoorde,: - {\textquoteleft},?}{plots tot de milddadigen}{enEen sigaar Boetjes}\\

\haiku{{\textquoteright} - {\textquoteleft}Dat kan waarschijnlijk '.}{nog heel goed vant bloed zijn}{van Karel den Groote}\\

\haiku{Maar als 't op den,...}{mesthoop stond en kraaide dan}{viel het omverre}\\

\haiku{{\textquoteright} zei Boetjes, {\textquoteleft}ge zult er,...:}{misschien mee lachen maar laat}{me rechtuit zeggen}\\

\haiku{{\textquoteright} - vroeg Sander, die twee. - {\textquoteleft},!}{uren lang gezwegen hadEen}{beetje geduld zoon}\\

\haiku{{\textquoteright} -        Boetjes vertelt van '... - {\textquoteleft} ',, -!}{ne muilezelDats heel goed}{gezegd Boetjes h\'e\'el goed}\\

\haiku{{\textquoteleft}Ga weg, snauwde dan,,,.}{Geert ga weg leelijkerd en}{kijk naar uw eigen}\\

\haiku{{\textquoteleft}En kijk nu zelf maar,,!}{Po-hotje of die geit ook}{maar \'e\'en gebrek heeft}\\

\haiku{En hij begaait mijn!... '',?}{wasch nogt Kan anders nie}{meer kapot zeker}\\

\haiku{{\textquoteright} - {\textquoteleft}Maar Mijnheer Pastoor{\textquoteright}, {\textquoteleft}?}{verweet moeder Boetjeshoe durft}{ge zoo iets zeggen}\\

\haiku{- {\textquoteleft}En toch, Boetjes{\textquoteright} hernam, {\textquoteleft}, ',.}{hij toen luiden tocht moet}{hem z\'o\'o zitten Boetjes}\\

\haiku{- {\textquoteleft}Gij denkt altijd, Boetjes{\textquoteright}, {\textquoteleft}...{\textquoteright}}{verweet hijdat ik u in}{de doeken wil doen}\\

\haiku{En voor de derde,,:}{maal en de derde maal heel}{driftig snauwde Boetjes}\\

\haiku{{\textquoteright} vroeg Baveloo 's.}{anderdaags en wreef zijn twee}{handen over malkaar}\\

\haiku{Dit is de vraag, Mr,,,:}{Pastoor n\`amelijk en te}{w\'eten en aldus}\\

\haiku{Te voet, om toch maar.}{bij uw meester te zijn en}{te kunnen broeien}\\

\haiku{Begaat ge zelf een,!}{groote dommigheid dat noemt ge}{een groote slimmigheid}\\

\haiku{Maar moeder Boetjes, als ',:}{ne man nam het op voor de}{eer van haar huis en}\\

\haiku{Maar in de kerk moet,...}{ge niet knikken tegen mij}{maar tegen O.L. Heer}\\

\haiku{Spreek met moeder daar ',,,...}{ns over alleen stillekes}{zonder koleire}\\

\haiku{Ze zouden toch van,, -!}{armoe thuis komen dacht ze}{Boetjes zeker en vast}\\

\haiku{Bij de ouders staat!}{het kinderspel maar al te}{dikwijls van geen tel}\\

\haiku{Ik zie Sander daar...}{yo-yo spelen en item zoo}{Mieke yo-yo maar}\\

\haiku{en die andere,,!}{yo-yo daar van Mieke dat}{was voor u moeder}\\

\haiku{Als ge niet wordt zooals,...}{kinderen zult ge nooit in}{den hemel komen}\\

\subsection{Uit: Brieven van Nonkel Pastoor}

\haiku{n.l. dat gij erom.}{beschaamd zijt geweest dat ze}{lachten met uw naam}\\

\haiku{Nu zal ik dat heel;}{in den korte en in den}{nauwe moeten doen}\\

\haiku{ge moet 'ne mensch geen,.}{toenamen geven vooral}{niet aan Pater Adams}\\

\haiku{En mag ik u eens?}{eventjes met een dilemma}{op uwen kop tikken}\\

\haiku{Denkt ge soms dat het,:}{plezant is voor mij als heel}{de parochie zegt}\\

\haiku{ne zekere Kant,!}{beweerde dat er geen tijd}{bestaat de sloeber}\\

\haiku{Wat zouden wij in ',?...}{s Hemelsnaam gaan doen als}{we geen tijd hadden}\\

\haiku{'t Slimste dat een}{gewoon mensch al doet met die}{twee instrumenten}\\

\haiku{een nagel trekken,.}{met de trektang dien hij met}{den hamer scheef sloeg}\\

\haiku{Een broek tusschen-in,, -.}{half-weg knoesel en}{knie dat is geen broek}\\

\haiku{maar Bertje, daar zijn,.}{dingen in de wereld die}{zoo simpel niet zijn}\\

\haiku{en hij verdient de...{\textquoteright} ' ';}{straf Ent schoonste vant}{geval is dit nog}\\

\haiku{en ge weet niet waar.}{het op aanloopt noch van waar}{het gekomen is}\\

\haiku{Is de huidige?}{jeugd van den dag van vandaag}{z\'o\'o teergevoelig}\\

\haiku{de jury stond er,.}{naast om te tellen en de}{parochie rondom}\\

\haiku{Na de 9e telloor;}{viel Tistje voorover met zijn}{gezicht in zijn 10e}\\

\haiku{{\textquoteright} - Of begon hij te ',?}{weenen alsne jongen die}{Nepos niet vertaald krijgt}\\

\haiku{'t Evangelie van.}{de arme weduwe met}{de twee penningskes}\\

\haiku{maar zoo juist is de...:}{perekwatie van de pastors}{binnengevloeid en}\\

\haiku{Sommige menschen,;}{zijn altijd bij de laatsten}{d\'oen wat ze willen}\\

\haiku{Maar in de tweede ',...}{helft vant leste bedrijf}{daar haperde wat}\\

\haiku{{\textquoteright} - En 'k heb altijd.}{ondervonden dat hij min}{of meer gelijk heeft}\\

\haiku{en 3o) als ik zooals,!}{gij nog zestien jaar oud was}{dan deed ik het wel}\\

\haiku{Ton is de man, die;}{thuis zit met dikke schijven}{en warme voeten}\\

\haiku{want de puntjes, die,...}{ge er mij in meedeelt zijn}{nog al van belang}\\

\subsection{Uit: Kijkkast}

\haiku{t Was de vader, -.}{zelf van den kleinen doode}{Petrus Nellemans}\\

\haiku{Binkske kwam voorbij -:}{het huis en al met eens zijn}{blij gemoed viel weg}\\

\haiku{en hoe juist bijtijds,,!}{op den boord van den afgrond}{gelukkig en rap}\\

\haiku{- En 't zou, den 23n, {\textquoteleft}{\textquoteright}.}{een elf-ure-lijk zijn}{in splendoribus}\\

\haiku{Een klein klokske kroop.}{van bedeesdheid weg achter}{Sint-Gummarus}\\

\haiku{En zoo, mijn bronzen,!}{predikers vergeet nooit uw}{plicht van predikers}\\

\haiku{eens voor henzelf, eens,;}{voor elk hun Pastoor eens voor}{elk hun parochie}\\

\haiku{Tegenover hem staat.}{Graaf Haditz in kersrood van}{begeerte naar macht}\\

\haiku{het lijk van hem die,...}{was Keizer Frans wordt naar den}{grafkelder gevoerd}\\

\haiku{Dien ken ik niet...{\textquoteright} De,,.}{Ceremoniemeester een}{derde maal klopt aan}\\

\haiku{De Schout floot, en twee,.}{honden vlogen op uit het}{struiksel naar Jan toe}\\

\haiku{{\textquoteleft}Ja, moederke, er,.}{aan moet hij toch want de Wet}{heeft lange armen}\\

\haiku{Ouwe-Jobbie stond, ':}{op en de rechterhand hoog}{bovent hoofd}\\

\haiku{- zoo stonden mij de.}{haijren te berghe aan}{mijnen lijve}\\

\haiku{Waer waerdij doen?}{ick de fundamenten der}{aerde leijde}\\

\haiku{{\textquoteright} - ~ Ouwe-Jobbie {\textquoteleft}{\textquoteright},...}{begon tescharren met de}{lakens te spelen}\\

\haiku{{\textquoteright} - Fiks viel opeens de, ',.}{Duitscher sloeg de rechterhand}{aant hoofd groette}\\

\haiku{{\textquoteright} zei Maarten, nu zelf.}{zoo verbaasd dat hij al Fransch}{begon te spreken}\\

\haiku{Hij was versleten '.}{nu van den ouderdom en}{t ruige leven}\\

\haiku{Mijn bullekarke..., '.}{en mijn gareel ik peins dat}{s hoop en alles}\\

\haiku{Maar Toghrai zelf, al,.}{reed hij niet op Basra ging}{Basra inhalen}\\

\haiku{- Basra vloog, - het vloog,,!}{als een pijl als een pijl als}{een snorrende pijl}\\

\haiku{En de Kadi liet,.}{hem nazoeken want hij had}{het zoo gezworen}\\

\haiku{Iedereen wilde,,.}{kost wat kost het dreigende}{gevaar afweren}\\

\haiku{E\'en oogenblik, \'e\'en,...! '}{h\'e\'el kort oogenblikje is}{er niets gebeurd niets}\\

\haiku{Wel, nu ik het ding,,, ', '.}{bepeins Pater ge zijt zelf}{int kleinne Paus}\\

\haiku{En morgen vroeg zal '.}{t weer uitvallen dat ik}{Schamel Binkske ben}\\

\haiku{(en hij snikte nu)!}{en de tranen droppelden}{langs zijn baard Binkske}\\

\haiku{De Pater knikte {\textquoteleft}{\textquoteright}:}{zoo iets vangemeenschap der}{heiligen en vroeg}\\

\haiku{t Kruisbeeld dat de.}{Heer den kleine bewaren}{en geleiden zou}\\

\haiku{Mijnheer Gerardus..........}{X   X straatje   X}{Belgium}\\

\haiku{Wat moest hij nog de!}{liefelijke schoonte leeren}{van de heiligheid}\\

\haiku{{\textquoteleft}En ze bekeken,, '...}{malkander Lieven halfdood}{ent spook heel kalm}\\

\haiku{Waar ergens groeit het?}{mastenhout zoo wild en knoest}{als in de Kempen}\\

\haiku{{\textquoteright} lachte de Keizer,:}{en hij keerde zich tot den}{stijven gouden heer}\\

\haiku{Al dadelijk moest.}{Patsken opgekleed voor zijn}{nieuwe bediening}\\

\haiku{en de \'e\'ene te,,;}{groot en de andere te}{klein en elk wist iets}\\

\haiku{{\textquoteright} - - {\textquoteleft}Daar is 'ne schout op '...}{elk dorp enne pastor op}{elke parochie}\\

\haiku{De Nederlanden.}{met hun nijverheid laat ik}{over aan Snotje Vandom}\\

\haiku{{\textquoteright} En de Keizer was:}{verstandig genoeg om de}{slotsom te trekken}\\

\haiku{Eerst spreken, aap, of!}{ik tast met mijn roeike naar}{uw ribbekast}\\

\haiku{Vanplan, met mijnen {\textquoteleft}{\textquoteright} -;}{van gij zijt de slimste van}{heel mijn kazerne}\\

\haiku{en al wat schoon is}{en lief en heilig en al}{wat een ideaal draagt}\\

\subsection{Uit: Kronijken}

\haiku{en al even stijf, langs,.}{zijn lange gezicht hingen}{lange snorpinnen}\\

\haiku{En hij deed den arts.}{en den generaal teeken}{dat ze heen mochten}\\

\haiku{{\textquoteleft}'t Es ol fiertig'}{jaor da me moeder den}{trottoir bij Menhier}\\

\haiku{eten we den baljuw, ' '!}{op ens morgens hebben}{wene zwaren kop}\\

\haiku{En Bernwald smeekte:}{opnieuw met vrees in de stem}{en op het gelaat}\\

\haiku{{\textquoteleft}Zalig zijn ze die...{\textquoteright} - {\textquoteleft}?}{Wat mogen die woorden te}{bedieden hebben}\\

\haiku{Dat we gebocheld,?}{zijn als we recht-op staan in}{den dienst des Heeren}\\

\haiku{en altijd zoo juist,, '.}{juist precies hairjuist nevens}{nen echten vloek af}\\

\haiku{De Pastor had hem:}{bij zijn vertrek zoo de les}{gespeld en gezegd}\\

\haiku{bewaar nu toch, in,;}{Gods naam uw ziele zuiver}{zoo lang ge maar kunt}\\

\haiku{de duivel hitste...}{de dazen al woedender}{tegen Maarten op}\\

\haiku{{\textquoteright} sprak de Heilige... {\textquoteleft} '!}{Opt oogenblik dat de}{donder u doodsloeg}\\

\haiku{maar een kind, dat men,!}{overlaat aan zijn wil doet zijn}{moeder schande aan}\\

\haiku{{\textquoteright} - {\textquoteleft}Zie, Mijnheer Pastoor,...}{stel geen vragen en ge zult}{geen leugens hooren}\\

\haiku{{\textquoteleft}En de ouders, die,!}{den Bode lezen mogen}{er een les in leeren}\\

\haiku{een paraplu is,...}{toch maar een vod een vod met}{een stokjen in}\\

\haiku{Toen knikte Pater, -...}{Rector tot afscheid ging heen}{en liet mij alleen}\\

\haiku{dat is nog iets, wat '...}{ik hun meen te vragen in}{t leste oordeel}\\

\haiku{en 'k weet niet uit '!}{wat voor een wolkt ding op}{me neerdonderde}\\

\haiku{En ik vertel hem ';}{int kort wat ik u zelf}{vandaag vertelde}\\

\haiku{gij, voed me met uw,!}{stillen vrede Beata}{Soror Paupertas}\\

\haiku{de legende nl..}{die gaat over den oorsprong van}{den Konijnenberg}\\

\haiku{Maar haring - haring, -.}{simpelweg zonder pekel}{is sympathieker}\\

\haiku{Soms vraagt wel eens een,:}{ouwe mensch die zijn eeuw niet}{meer kan bijblijven}\\

\haiku{Daarbij, 't heele... - {\textquoteleft}:}{Itinerarium krijgt ge}{nietAntifone}\\

\haiku{En ze praatten als,...}{broers en zusters gezellig}{en gemoedelijk}\\

\haiku{Maar Haroen bezag.}{het al met sombere oogen}{en hangende lip}\\

\haiku{'t Is een domme,.}{visch dien ze tweemaal vangen}{met denzelfden pier}\\

\haiku{{\textquoteright} Van hand tot hand gaan,.}{de teerlingen ratelen}{over het tafelblad}\\

\haiku{Spontijn keert het hoofd,: - {\textquoteleft},...}{stoot de deur aan en fluistert}{Hoog bezoek vrienden}\\

\haiku{de eenen zwegen uit,,.}{eerbied de anderen niet}{wetend wat te doen}\\

\haiku{{\textquoteright} Prins Jan komt op met,,,:}{een knaap een twaalf jaar oud stelt}{hem vlak v\'o\'or Bon roept}\\

\haiku{Serclaes op met een,,: - {\textquoteleft}!}{knaap elf jaar oud stelt hem vlak}{v\'o\'or BonNummer twee}\\

\haiku{- {\textquoteleft}Prins Jan,{\textquoteright} spreekt Vaartje nu, {\textquoteleft},'?...}{weer beleefduw vader ligt}{op sterven nie waar}\\

\haiku{Nu staat Bon hoog-op,,:}{de teenen zwaait met zijn armen}{als vleugelen kraait}\\

\haiku{En hij kreeg per kans,.}{Aldegonde te snappen}{trok hem bij de mouw}\\

\haiku{Hier, bij mijn voeten,...}{dartelt en spartelt muze}{Clio Serafientje}\\

\haiku{Geef mij een nagel,.}{en ik hang het op tegen}{den wand van mijn huis}\\

\haiku{En ik sta erbij,,...}{als Pontius-Pilatus}{en wasch mijn handen}\\

\haiku{Dat zijn \`uwe neefjes,...!...,?}{Jan Verantwoordelijkheid}{Krieuwelt er niets Jan}\\

\haiku{{\textquoteright} Bullarius buigt -}{even om te bedieden dat}{het zoo gebeuren}\\

\subsection{Uit: Opinies van Proke Plebs}

\haiku{en ze zouden nog,.}{niet op me willen spuwen}{als ik in brand stond}\\

\haiku{En wie de waarheid,...}{liefheeft eet het bitter brood}{van de minderheid}\\

\haiku{Als 'ne jager schiet,:}{met een tweeloop dan denkt hij}{altijd halveling}\\

\haiku{en ze komen te -!}{paard en gekroond op ons af}{en zegevieren}\\

\haiku{schaf God af, en geen.}{museum is groot genoeg}{voor uw afgoden}\\

\haiku{t k\`an... driemaal op,... '}{de tienduizend tweemaal op}{de twintigduizend}\\

\haiku{En Janeke Bots...}{zelf bekent en staaft al die}{getuigenissen}\\

\haiku{En zoo stapte 't.}{beestje waar de molenaar}{het hebben wilde}\\

\haiku{opheffen tot den,.}{hoogsten top van wetenschap}{welstand en plezier}\\

\haiku{en mee moet ge, en,.}{eeuwig zijt ge ten minste}{langs \'e\'enen kant}\\

\haiku{met al hun zalfjes,;}{en pottekes we zijn kaal}{en we blijven kaal}\\

\haiku{daar is een tijd van,;}{te spreken daar is een tijd}{van te  zwijgen}\\

\haiku{want, zeggen ze, wij.}{hebben de ster gezien en}{we zijn gekomen}\\

\haiku{niet met herders en,;}{schapewachters maar bankiers}{en pelsen mantels}\\

\haiku{want inderdaad, gij;}{hadt uw nikkeltjes slechter}{kunnen gebruiken}\\

\haiku{en dat geluk moet,;}{hem van buiten komen want}{binnen zit het niet}\\

\haiku{en als God het hun,.}{toelaat dan kunnen ze met}{ons komen spreken}\\

\haiku{en alle wegen,;}{loopen naar Gheel als ge zelf}{liever naar Gheel loopt}\\

\haiku{en alles hangt er,...}{van af langs waar toe het paard}{met zijn kop staat}\\

\haiku{Tegen den Paus en '...}{t Vatikaan is alles}{geprobeerd geweest}\\

\haiku{\'en tatati \'en,;}{tatata kwaad en koleirig}{en opgewonden}\\

\haiku{{\textquoteleft}Waarom gaat de Paus,?}{altijd zoo zitten met zijn}{twee vingers omhoog}\\

\haiku{{\textquoteright} - - {\textquoteleft}En het Kruis in de{\textquoteright}, -.}{plaats stellen duwt er Petrus}{bij simpel-weg}\\

\haiku{ge wilt en laat ze ';}{verzorgen door den besten}{veearts vant vak}\\

\haiku{dat kindeke is;}{de ware president van}{den Volkerenbond}\\

\haiku{of denkt ge soms dat?...}{ik mijn eigen kinderen}{niet meer kan tellen}\\

\haiku{de kindekes die,,...}{moesten zijn Van den Bemde en}{die niet geweest zijn}\\

\haiku{{\textquotedblleft}Ik ben te arm{\textquoteright}... en,.}{de schoonste rijkdom dat zijn}{de kinderzielen}\\

\haiku{En grooter zot leeft, - -!}{er niet dan een redelijk}{mensch die z\'ot wil zijn}\\

\haiku{Kunstmest... ~ (pooze) ~ , '.}{Als ik niet getrouwd wask}{ging op de markt staan}\\

\haiku{Maar 'ne clown in de ';}{cirk speelt muziek opne gam}{van zeven flesschen}\\

\haiku{beleefdheid staat ook.}{al slecht ge-hippotekeerd}{den dag van vandaag}\\

\haiku{Maar dat 's alweer,, '...}{voorbij Van den Bemde en}{t onweer is over}\\

\subsection{Uit: Proke vertelt...}

\haiku{Het Deezeke schudt -:}{zijn beddeken uit maar de}{groote menschen klagen}\\

\haiku{de boeren waren,...}{fier dat ze zoo de pluimkes}{op hun klak kregen}\\

\haiku{Rozeke had wel ' ';}{geen schoone stem ent leed}{aanne korten adem}\\

\haiku{Ik geloove in...:}{God den Vader enz. met de}{drie groetenissen}\\

\haiku{Al wat ik zeggen,, ', ';}{kan valt mis ent valt te}{min naark vreeze}\\

\haiku{Maar opeens schoot hij,:}{in zijn koleire en zijn}{fraksken uit en riep}\\

\haiku{{\textquoteright} - brulde Potje, en;}{ditmaal had de Kater geen}{tijd om te poeffen}\\

\haiku{Maar Potje zweeg nog,.}{en hij kon zijn ooren en}{zijn oogen niet gelooven}\\

\haiku{Op 'ne morgen plots,,,!}{daar lag vlak v\'o\'or hen op de}{heuvelen Rome}\\

\haiku{En den godslieven;}{dag door waschten ze maar}{en ze sjauwelden}\\

\haiku{kwam de Wijze Prins,;}{in de parochie met een}{troon en soldaten}\\

\haiku{Al die hun kemel, '!}{verliezen loopen vant}{gras naar de biezen}\\

\haiku{{\textquoteleft}'ne Goeie kemel is ', '.}{t want zijn vier pooten staan}{hem ondert lijf}\\

\haiku{Al die hun kemel, '!}{verliezen loopen vant}{gras naar de biezen}\\

\haiku{Al die hun kemel, '!}{verliezen loopen vant}{gras naar de biezen}\\

\haiku{Al die hun kemel, '!}{verliezen loopen vant}{gras naar de biezen}\\

\haiku{En wat die man niet,.}{wist dat was de moeite niet}{waard om te weten}\\

\haiku{en met 'ne manken.}{kemel geraakt ge verder}{dan zonder kemel}\\

\haiku{Al die hun hemel, '!}{verliezen loopen vant}{gras naar de biezen}\\

\haiku{{\textquoteright} gilde moederke, {\textquoteleft},,!}{Jan luister toch naar mijn raad}{en blijf bij moeder}\\

\haiku{En Jan valt bots het,...}{Kutteke kapot en plat}{in de koekepan}\\

\haiku{Die smid was altijd;}{een boezemvriend geweest van}{Godfried van Boeljon}\\

\haiku{en 't eerste van!}{al stapt Herenthals recht}{naar Jeruzalem}\\

\haiku{- Vooruit nu al wat... - -!}{Janssens heet en zoo voort en}{zoo voort en zoo voort}\\

\haiku{{\textquoteleft}Maar Majesteit, wat?}{houdt ge toch z\'o\'o van boeken}{en veelweterij}\\

\haiku{- {\textquoteleft}Kwestie of Koning!}{Baltazar de Koningsstar}{zal willen volgen}\\

\haiku{De starre schoot een,,}{straal op mij een engel zong}{me een melodij}\\

\haiku{En hij ging ook wal ',;}{klaver scheren int veld}{goeie vette klaver}\\

\haiku{- Hij stond op, stapte,,:}{bij en klaar staande in den}{maneschijn hij sprak}\\

\haiku{- {\textquoteleft}Dan is 't goed{\textquoteright} - zei.}{de veekoopman en liet ze}{begaan met den ezel}\\

\haiku{En toen de intocht,:}{over was toen streelde O.L. Heer}{het ezeltje en zei}\\

\haiku{en zei zoo, dat, wie, ';}{het wapen gebruikt doort}{wapen zal vergaan}\\

\haiku{zuchtte de pastoor,!}{wat hebben ze schoon over den}{vrede gesproken}\\

\section{Joos Florquin}

\subsection{Uit: Lente van het hart. Brieven van Tijl aan Neleke}

\haiku{waren de schakel, -.}{in de redeneering en}{zuchtte toen eens diep}\\

\haiku{Hij sprak geen woord, maar.}{ik zag aa nzijn glimlach dat}{hij gewonnen was}\\

\haiku{Eerst en vooral werd.}{het kacheltje opgesteld}{in ons hoofdkwartier}\\

\haiku{heet het jochie dat.}{zich in ons huis een vaste}{plaats heeft verworven}\\

\haiku{Maar ik wil me door.}{Mercator niet van mijn doel}{laten afleiden}\\

\haiku{Ik wil je al mijn.}{pech van de laatste week niet}{verder uitbeelden}\\

\haiku{het mooiste daarvan.}{is nu wel dat het vijgen}{na Paschen zijn}\\

\haiku{Een sentimenteel.}{tochtje kon dat nu precies}{niet genoemd worden}\\

\haiku{{\textquoteleft}Of als hij komt, of,.}{als hij scheidt heeft de oude}{Maart zijn gif bereid}\\

\haiku{{\textquoteleft}Na de vasten komt,{\textquoteright}.}{Paschen maar het was toch}{maar een schrale troost}\\

\haiku{{\textquoteleft}Zoo het de veertig,.}{martelaars vindt blijft veertig}{dagen weer en wind}\\

\haiku{{\textquoteleft}Als 't helder is,.}{op Jozefsdag een goed jaar}{men verwachten mag}\\

\haiku{{\textquoteleft}De Maartsche zon en,.}{de Aprilsche wind schendt er zoo}{menig koningskind}\\

\haiku{Ik kan je nu mijn.}{gedroomd verzoek van daar straks}{blijde herhalen}\\

\haiku{Zullen grenzen of?}{andere complicaties}{ons tegenhouden}\\

\haiku{Onze gevleide.}{glimlach had waarschijnlijk iets}{weg van een grimas}\\

\haiku{Een vriendelijke.}{waard verwelkomde ons met}{een breeden glimlach}\\

\haiku{Zoo waren we in.}{een wip uit het bed en kon}{de dag beginnen}\\

\haiku{Ik zei dat hij met.}{den graaf toch over politiek}{had kunnen spreken}\\

\haiku{is je vroolijkheid,.}{dat ze alle sombere}{gedachten ontkent}\\

\section{N.E. Fonteyne}

\subsection{Uit: Kinderjaren}

\haiku{Zaken sedert een.}{eeuw opgehoopt die ons voor}{weken rijk maakten}\\

\haiku{om het mysterie;}{dat met elke aardschop ons}{voor oogen gegooid werd}\\

\haiku{een felle klaarte,;}{alleenheerschend tusschen de}{andere schijnsels}\\

\haiku{Ze heeft ons volk te.}{veel ontnomen en niets in}{de plaats gegeven}\\

\haiku{Of gedruktheid om.}{een grijze lucht of om een}{plotse eenzaamheid}\\

\haiku{En bij ieder versch:}{geopend graf werd het een}{heele vertelling}\\

\haiku{In die dagen ben:}{ik honderde malen naar}{moeder geloopen}\\

\haiku{in mijn voorstelling.}{besloot haar dood meteen mijn}{eigen verdwijning}\\

\haiku{En van geen vreemde.}{heb ik ooit meer gehouden}{als toen van mijn hond}\\

\haiku{Slechts een enkele.}{dag teekent zich heller in}{die groote grijsheid uit}\\

\haiku{ik hoe kieskeurig.}{die kinderboekerij was}{bijeengevallen}\\

\haiku{Die dolven we nu,,:}{stofferig vergeeld uit een}{vergeten kast op}\\

\haiku{Ze geeft niet zoozeer de,.}{schoonheid als wel een droom een}{honger naar schoonheid}\\

\haiku{Het wordt voor zulk een.}{kunstenaar moeilijk om de}{natuur te vinden}\\

\haiku{Lente en Herfst die.}{voor het kind nog maar alleen}{te doorvoelen zijn}\\

\haiku{Spraken we thuis over,;}{klerikalisme dan trok}{vader de neus op}\\

\haiku{omdat ze nooit de.}{omvang van de gevreesde}{smart bereiken kan}\\

\haiku{Voor haar open venster {\textquoteleft}{\textquoteright}.}{speelde de pianiste}{God save the Queen}\\

\haiku{En nu ging het met.}{de hoop en het vertrouwen}{elke dag bergaf}\\

\haiku{wat hadden we een!}{plezier toen we eindelijk}{ook mochten vluchten}\\

\haiku{Een woelige stroom.}{die niet eens meer in de nacht}{onderbroken werd}\\

\haiku{Voor een huis werd een,.}{zwijn geslacht en elders hing}{een koe op de kaak}\\

\haiku{rechtop, vier aan vier,.}{gebonden tot boven de}{knie\"en in het zand}\\

\haiku{En dagen hadden.}{we werk met het afloopen}{der nieuwe lagers}\\

\haiku{Dagen naeen trok de:}{heele veestapel van het}{Noordvrije door het dorp}\\

\haiku{Aan ieder huis hangt.}{voor de eerste maal sedert}{vier jaar weer een vlag}\\

\haiku{wij hebben te zeer.}{voor hun erbarmelijke}{lafheden geboet}\\

\haiku{Zoeken we niet over?}{de vrouw naar een houvast in}{een schimmenwereld}\\

\haiku{{\textquoteright} Of zijn de menschen?}{voor die dingen werkelijk}{zoo onverschillig}\\

\section{Ellen Forest}

\subsection{Uit: Aleid}

\haiku{{\textquoteleft}Oh, heelemaal niet -.}{op een dag als vandaag k\`an}{haast niets te vroeg zijn}\\

\haiku{{\textquoteright} Mevrouw vermeed het,.}{woord kellner omdat ze zoo}{fel anti-Duitsch was}\\

\haiku{{\textquoteleft}Hoe heerlijk niet, om!}{je diaphragma zoo geheel te}{kunnen ontspannen}\\

\haiku{Ik k\`an niet in 't,,....}{gerij loopen opzitten}{pootjes geven}\\

\haiku{Marijke nam haar:}{later even apart en zei met}{tranen in haar stem}\\

\haiku{E\'en van de tantes.}{kwam even aanloopen en nam}{haar mee de stad in}\\

\haiku{Met het aanhooren.}{van dit ritornello kocht}{ze een uur vrijheid}\\

\haiku{Ik zal hem alleen.}{melden dat u iets zoekt en}{u introduceeren}\\

\haiku{Dan schudde ze zich.}{inwendig en begon om}{zich heen te kijken}\\

\haiku{Nu jaag ik u naar,{\textquoteright},:}{bed zei hij toen iemand in}{het voorbijgaan zei}\\

\haiku{Daaronder stroomden,. '}{traag en moeizaam gedachten}{verbonden met Rein}\\

\haiku{Hij had in alle -.}{stilte willen trouwen in}{zijn daagsche pakje}\\

\haiku{Ze had t\`och wel iets,.}{van hem begrepen dat gaf}{haar eenige vreugde}\\

\haiku{Er zit meer energie.}{in een Londonsche season}{dan in een oorlog}\\

\haiku{Ja - d\`at is het groote,.}{raadsel dat ik nog bezig}{ben op te lossen}\\

\haiku{Moeder denkt dat we.}{nooit mogen weigeren als}{we gevraagd worden}\\

\haiku{Aan dien plicht viel niet -.}{te tornen dien bepaalde}{zelfs Mr. Li Tiang niet}\\

\haiku{Aleid van Doornhagen{\textquoteright}, {\textquoteleft}{\textquoteright}.}{zei ze tot zichzelfmake}{the very best of it}\\

\haiku{{\textquoteright} {\textquoteleft}Ja Elsy, stoor me maar,....}{niet behalve als mevrouw}{mocht telefoneeren}\\

\haiku{Ik ben geschrokken.}{van de photo van den tuin}{die u me stuurde}\\

\haiku{totaal anders dan.}{het beeld wat ik van ons huis}{meegenomen heb}\\

\haiku{Haar grootste afkeer - {\textquoteleft}{\textquoteright}.}{geldt de Japanners die ze}{lie-Europeans noemt}\\

\haiku{Ik kom,  om eens.}{echt met een Europeesch}{meisje te praten}\\

\haiku{Ze zat nog steeds op,.}{den divan haar hand weer slap}{hangend langs den kant}\\

\haiku{niets beduidde Aleid.}{of hij den strijd gewonnen}{of verloren had}\\

\haiku{Ze moest nu alle.}{romantisch gedaas van Oost}{en West vergeten}\\

\haiku{Om niet achteruit,.}{te gaan moet China bij het}{Westen aankloppen}\\

\haiku{Er bestond voor mij.}{een wereld en achter die}{wereld een leegte}\\

\haiku{Daar was eigenlijk.}{geen woord in dien brief dat op}{direct gevaar wees}\\

\haiku{Ze stond tegenover.}{een natuurverschijnsel van}{ongewonen aard}\\

\haiku{Ze draaide om dat.}{verlangen maar durfde het}{geen naam te geven}\\

\haiku{Ze was dolblij met,.}{haar vondst hoewel die vreugde}{niet onvermengd was}\\

\haiku{De eentonige.}{sing-song van den dominee}{ontroerde haar nooit}\\

\haiku{Ge staat nu op steenen -.}{van den Eiffel ge loopt op}{grind uit den Donau}\\

\haiku{{\textquoteright} vroeg Ruth, {\textquoteleft}luchthartig,,?}{oppervlakkig lachend om}{alles en n\`og wat}\\

\haiku{Lieve kind, je bent.}{zoo jong en je gaat een heel}{leven tegemoet}\\

\haiku{Zoolang je jong bent,.}{is het leven d\'a\'ar om er}{op te antwoorden}\\

\haiku{{\textquoteright} {\textquoteleft}We zijn beste maatjes,,.}{maar heel oppervlakkig en}{dat is maar goed ook}\\

\haiku{Dat zeggen alleen -.}{menschen in wie de weerstand}{voor goed gedoofd is}\\

\haiku{{\textquoteright} Er klonk wrevel uit,.}{grootvaders stem iets dat ze}{van hem niet kende}\\

\haiku{{\textquoteleft}Als je dat denkt, mijn,}{kind hebben we er ons maar}{bij neer te leggen}\\

\haiku{Je kon een vrouw, van,?}{tante Arda's leeftijd dit}{toch niet weigeren}\\

\haiku{Hier begon hij, wat,,.}{hij voor zichzelf noemde zijn}{nieuw eenzaam leven}\\

\haiku{De wereld is zoo,,{\textquoteright}.}{slecht niet als zij er uit ziet}{zei Otto Vorendonk}\\

\haiku{Dit is alleen het,.}{bewijs voor mijn kameraad}{dat u betaald hebt}\\

\haiku{{\textquoteleft}Als de jeugd met haar?}{enthousiasme het nu}{eens in handen nam}\\

\haiku{Van zoo'n avond bleef niets.}{over dan een beetje weemoed}{over de mislukking}\\

\haiku{Als hij daar stond en}{de menschen gevangen hield}{in zijn stillen blik}\\

\haiku{Ze viel zich zelf op.}{als anders dan de meeste}{Europeanen}\\

\haiku{Z\'o\'o zag ze ook Rein,,.}{een man die naar den hemel}{had willen reiken}\\

\haiku{Dat hij weggegaan,,.}{was bewees dat zij beiden}{zich vergist hadden}\\

\haiku{hij had alleen haar.}{hand gegrepen en haar even}{heel diep aangezien}\\

\haiku{Een Chineesche vrouw,:}{mag geen emotie toonen maar}{haar jongen lachte}\\

\haiku{{\textquoteright} - Met den rug naar het.}{venster stond zij en las de}{woorden nog eens over}\\

\haiku{Bleek die waarde te,.}{klein dan zouden zij elkaar}{niet meer moeten zien}\\

\haiku{Toen zij eenmaal op.}{het balkon stond was alle}{twijfel verdwenen}\\

\haiku{Toch wist ze dat ze,,.}{geen kwaad gedaan had noch zij}{noch de anderen}\\

\haiku{Ze wilde niet in.}{de armen van Chung liggen}{en aan Rein denken}\\

\haiku{Vandaag voor het eerst.}{voel ik onze belangen}{als tegenstrijdig}\\

\haiku{Morgen kan alles,.}{uit zijn maar het kan even goed}{nog eeuwen duren}\\

\haiku{Dit is ook weer een {\textquotedblleft}{\textquotedblright},.}{van onze grootedrawbacks dat}{kasten-systeem}\\

\haiku{Die had er maar een,,.}{paar die op hem aasden maar}{die aasden zoo goed}\\

\haiku{En doet hij het niet,.}{dan is hij aan handen en}{voeten gebonden}\\

\haiku{{\textquoteright} En toen in een voor,:}{hem vreemde vervoering liet}{hij zich gaan en zei}\\

\haiku{En langzaam begon.}{de vastheid van zijn besluit}{te verminderen}\\

\haiku{Het was een brief van,.}{zijn moeder waarin deze}{schreef over de meisjes}\\

\haiku{Dat wil Wellington,.}{Koo ook en dat wil ik ook}{en velen met mij}\\

\haiku{{\textquoteright} Hij antwoordde niet,:}{dadelijk maar zei na een}{oogenblik wachten}\\

\haiku{Ze begreep hem weer,.}{niet maar ze voelde dat hij}{iets moois bedoelde}\\

\haiku{{\textquoteleft}Stellig kleurling of!}{halfbloed tot in het derde}{en vierde geslacht}\\

\haiku{Omzichtig schoof hij.}{tusschen de anderen door}{tot hij naast haar stond}\\

\haiku{Nu was een gevoel.}{van spanning het eenige dat}{ze registreerde}\\

\haiku{Hij liep in dien avond -.}{als in een wolk van goudstof}{recht op zijn doel af}\\

\haiku{Dan zouden ze aan.}{zichzelf en hun stemmingen}{overgelaten zijn}\\

\subsection{Uit: Passiebloemen (onder ps. Lucy d'Audretsch)}

\haiku{{\textquotedblright} dan dat de wereld}{voor je in aanbidding ligt}{en je eigen ziel}\\

\haiku{{\textquoteright} Moe van de emotie,.}{en van het waken sliep ze}{gauw in en droomde}\\

\haiku{{\textquoteright} {\textquoteleft}Och God, 't is niets,, '.}{heelemaal nietsk weet niet}{wat je van me wilt}\\

\haiku{Begrijp jij niet, dat,?}{als jij daar zoo zit dat ik}{dan niet weg kan gaan}\\

\haiku{telkens was je er... -,}{weer en dacht ik aan je en}{zag ik je voor me}\\

\haiku{die maanden waarin '... '...}{ze wachtten opt examen}{t allerlaatste}\\

\haiku{Waar dacht ze aan, toe! ',...}{t Zou wel gaan Mies wilde}{nu ook haar best doen}\\

\haiku{Nu, toen is hij een...{\textquoteright} {\textquoteleft}, '...}{eindje omgegaanOch ja}{t zal wel zoo zijn}\\

\haiku{ze waren ook z\'oo,.}{gek verliefd zij twee\"en en}{zoo altijd samen}\\

\haiku{Hij gaf geen antwoord, '...}{slierde zich inn grooten}{stoel en nam een krant}\\

\haiku{H\`e ja, je lieve -,...}{armen om mijn hals h\'eerlijk}{dat zachte kussen}\\

\haiku{God, zeg, zoo bij jou,...,...{\textquoteright}.}{te kunnen blijven nog \'een}{minuutje h\`e II}\\

\haiku{Vlugge, schuchtere,.}{blikken die afgleden voor}{zij ze goed merkte}\\

\haiku{Neen, stervende - dat ',.}{s beter dan had ze meer}{reden tot blijven}\\

\haiku{Eigenlijk vind ik '...{\textquoteright}}{t ook wel heerlijk dat je}{zoo'n dom vrouwtje bent}\\

\haiku{Maar als jij denkt dat...}{je goeddoet met z\'oo gekleed}{bij armen te gaan}\\

\haiku{Raar, dat zij twee\"en '...,...}{nooit overt kindje spraken}{n\'een zelfs niet voor Nic}\\

\haiku{Och, neen, 't was nu,...}{al weer weg die behoefte}{om alleen te zijn}\\

\haiku{... en t\`och, h\'eel diep - w{\`\i}st,...}{ze wel dat hij niet d\`at was}{wat ze gehoopt had}\\

\haiku{Eigenlijk had Nic,......}{haar pas geleerd wat passie}{was tellen nou toch}\\

\haiku{en in haar was ze, ' '... {\textquoteleft}}{toch wel bang bang voort \'een}{als voort ander}\\

\haiku{nu was hij w\`eg... en '...}{zij hadt gewild en nu}{had ze niets te doen}\\

\haiku{ze had hem toch wel,...,,...}{lief maar ellendig dat leege}{niet weten w\`at doen}\\

\haiku{en straks... uit 't bad - '...}{met ander linnen enn}{andere japon}\\

\haiku{die hoek nog - niet in '..., '...}{n vuile spiegel zien wacht}{even nogr haar los}\\

\haiku{De twee voor mij uit -'...}{h\'eel innig kussen elkaar}{voor Homerus beeld}\\

\haiku{En nu zijn ze weg - '...}{w\`eg int gewoel van de}{Parijsche straten}\\

\haiku{{\textquoteright} ~ {\textquoteleft}Lieve, 'k zou......{\textquoteright} {\textquoteleft}}{wat goed willen doen wat goed}{aan die menschjes}\\

\haiku{Ik ben vandaag naar ' - '...}{t kerkhof geweestt is}{Allerheiligen}\\

\haiku{Die twee zijn dood, Odin....}{en wij bezochten hen op}{Allerheiligen}\\

\section{Fritz Francken}

\subsection{Uit: Aan het vinkentouw}

\haiku{Voorzeker ook niet, (!}{uit leedvermaakde vlaag kon}{zijn vrouw overvallen}\\

\haiku{Met tranen in de.}{oogen vertelde mama wat}{haar overkomen was}\\

\haiku{Het was beslist een,,.}{aardig man die direkteur}{betrekkelijk jong}\\

\haiku{Als Calkoen met hem -,}{uitreed naar Holland en dat}{gebeurde dikwijls}\\

\haiku{Hij sprak z\'oo keurig,,!}{wist van alles iets \`af en}{van dat iets \`alles}\\

\haiku{Die Cornelissen... -...!}{is in Holland geweest en}{Och die sukkelaar}\\

\haiku{nog zoo kwaad niet... - Graaf...!}{een vaart en de Schelde mondt}{uit te Zeebrugge}\\

\haiku{Ge weet, bromde de,,!}{aspirant-Minister}{de gazetten he}\\

\haiku{Als hij zich spoedde,.}{haalde Hagedoorn nog den}{trein van tien v\'o\'or eenen}\\

\haiku{Hij verhaastte den.}{pas. Eindelijk ontwaarde}{hij het station}\\

\haiku{Hij heeft me gevraagd,.}{u te verzoeken zoolang}{op hem te wachten}\\

\haiku{Hij scheen beslist in.}{staat om iemand onverhoeds}{een lik te geven}\\

\haiku{Wou hij misschien even?}{de appartementen in}{oogenschouw nemen}\\

\subsection{Uit: De blijde kruisvaart}

\haiku{Wij werden op 't.}{oefenplein gedrild dat we}{draaiden als doppen}\\

\haiku{Den volgenden dag, ',.}{s middags kuierden we}{te Ieperen rond}\\

\haiku{Den 19en Oktober, ',.}{s middags gingen we in}{Le Havre aan wal}\\

\haiku{si nous n'avions,!}{pas eu la Belgique nous}{serions fichus}\\

\haiku{t Is een stadje,,.}{aan zee als een nest in de}{vallei gedoken}\\

\haiku{Oude wijvekens.}{verwelkomden de troepen}{met tandeloozen mond}\\

\haiku{En 't sneeuwde zoo.......}{nijg dat dat de peerden er}{niet niet door konden}\\

\haiku{U wil ik het wel,,...}{vertellen sergeant u}{kunt me begrijpen}\\

\haiku{In den kaos van.}{de frontdrukte verloor ik}{Seppe uit het oog}\\

\haiku{De boer kwam in de,:}{schuur een riek op den schouder}{en zei goedzakkig}\\

\haiku{zoolang d'r leven,,.}{is is er hoop beweerde}{Vodden-en-Beenen}\\

\haiku{Altijd koest blijven,... -!}{geduldig afwachten Met}{de vrinketten d'rop}\\

\haiku{Mismoedig brak de,.}{Spion de harde korst at}{met lange tanden}\\

\haiku{Goedschiks leenden de '.}{jongens mekaart beste}{stuk uit hun kleerkast}\\

\haiku{k Ben eens kurieus.}{of hij mijn kommissie zal}{bol gewerkt hebben}\\

\haiku{- 't Is al elf uur,, '.}{en de Spion is er nog}{niet zeit Zwierken}\\

\haiku{En op een schoonen '.}{dag verscheen hij terug in}{t Schipperskwartier}\\

\haiku{De panden van hun.}{verbalemonden kapoot}{flapten in den wind}\\

\haiku{Met wanhopigen,.}{blik zocht hij grabbelde die}{der gesneuvelden}\\

\haiku{'t Was een echte:}{verlossing toen de dokter}{den zesden dag zei}\\

\haiku{Rustig labeurde,.}{hij voort plaatste de zakjes en}{damde elke laag}\\

\haiku{Door de gaten van.}{de schietkanteelen loerden}{ze naar den vijand}\\

\haiku{De marsch van 't.}{regiment werd ingeluid}{door den Hondendief}\\

\haiku{Broer, ik schaam me en,.}{verwensch den dokter die me}{nog niet weerkeeren laat}\\

\haiku{In 't grauwe der,;}{schemering hingen ze te}{dansen de muggen}\\

\haiku{De Rik, 't Zwierken,,, '.}{de Voddetromp Drupneust}{Stropke mochten mee}\\

\haiku{Maar ze lazen een.}{akte van berouw na die}{onzinnige daad}\\

\haiku{- Als 't zoo voort gaat,.}{dan komen we nog bij den}{keunink te biechten}\\

\haiku{Wie geld verdiende,,}{noodigde broederlijk de maats}{uit bezocht met hen}\\

\haiku{- Bruggemans, geef me,!}{den hamer dat ik z'nen}{kop v\`astnagel}\\

\haiku{- Wacht sukkelaar, tot ',.}{we int Schipperskwartier}{komen zei Drupneus}\\

\haiku{Me dunkt dat ik dien.}{rakker nog zien ravotten}{heb op de Veemarkt}\\

\section{E. Franquinet}

\subsection{Uit: Maskeraad}

\haiku{{\textquoteleft}Kom{\textquoteright} zag er, {\textquoteleft}iech weet,,.....}{daste zwiege kins es et}{t'rop aon kump en dan}\\

\haiku{veur aon w\`erreke,.}{te dinke iers nog e flink}{st\"ok weijer te goon}\\

\haiku{men gedachtes, die.}{op tat momint wel erreg}{verward waore}\\

\haiku{Veer st\'onte dao wie,.}{geslage zeker wel e}{paar menute laank}\\

\haiku{veziet aon Pastoer,}{Bertels boe iech gistere}{den hielen aovend}\\

\haiku{sjoen st\"ok preuf en k\"a\"ort.}{in ze gehiel en in zen}{apaarte motieve}\\

\haiku{{\textquoteright} Et \'onweer leet nao,.}{dee slaag neet aof ieder woort}{et nog erreger}\\

\haiku{De bedeleer WAT,}{z\`ekste van deen erreme}{n\'onk Zjeraar dee dao}\\

\haiku{Euver de loch, die,.}{de maan van ziech aofgaof}{zal iech mer zwiege}\\

\section{Robert Franquinet}

\subsection{Uit: Drijfzand}

\haiku{Ze heeft een van de.}{meest opwindende ruggen}{die ik gezien heb}\\

\haiku{Een zware slag met.}{een stalen cylinder doet}{mijn kaakbeen barsten}\\

\haiku{De aftapper is.}{niet goed op de elektrische}{stroom aangesloten}\\

\haiku{De pennen dringen.}{langzaam binnen tussen de}{nagel en het bot}\\

\haiku{[II] Veel zoekt naar een.}{logische samenhang in}{mijn herinnering}\\

\haiku{Je hoeft er niet meer.}{op de juiste wijze te}{relativeren}\\

\haiku{Nu ik Clara van.}{kortbij heb gezien denk ik}{niet aan dat alles}\\

\haiku{Ik ben nog geen twee.}{dagen in de bungalow}{of ik spreek met haar}\\

\haiku{{\textquoteleft}U hebt een figuur.}{om een klassiek beeldhouwer}{mee te verrukken}\\

\haiku{De avond sluit tegen.}{de duinen als met luiken}{van beslagen zink}\\

\haiku{Al het andere.}{immers belemmert me}{direkt te schrijven}\\

\haiku{Mijn handen glijden.}{tussen je benen terwijl}{je de trap op gaat}\\

\haiku{Je familie is.}{in rep en roer want de man}{is wereldberoemd}\\

\haiku{Ze zegt dit laatste.}{met  een klemtoon die een}{beetje spottend is}\\

\haiku{Sedert jaren heeft.}{geen enkele dokter daar}{iets aan kunnen doen}\\

\haiku{Als je er levend,...}{aankomt vertel hen dan waar}{ik ben met het lijk}\\

\haiku{Ik keer me af van.}{Dalan die zijn hoofd in zijn}{armen houdt en schreit}\\

\haiku{We komen in een.}{residenti\"ele wijk}{met veel prikkeldraad}\\

\haiku{{\textquoteright} Ik begin haar uit.}{te kleden met een schroom of}{ik het nog nooit deed}\\

\haiku{Ik voel de spieren,.}{van haar benen haar buik en}{haar armen spannen}\\

\haiku{Ze doet haar ogen open.}{en begint in wanorde}{dingen te zeggen}\\

\haiku{Achter me is een.}{wagen uit een kleine steeg}{komen aanrijden}\\

\haiku{Mijn mensen brengen...}{u tot in de kleine straat}{achter de studio}\\

\haiku{Ik voel me niet thuis.}{in een leven waarin geen}{dingen gebeuren}\\

\haiku{Ik wil ook niet dat.}{je aan het leibandje van}{mijn gevoelens loopt}\\

\haiku{Ik heb de pest aan.}{jaloezie die ongegrond}{en kwaadaardig is}\\

\haiku{Hij strekt zijn benen}{languit en ik zie dat zijn}{broek verfomfraaid is}\\

\haiku{Een soldatenknoop.}{zit met een brede veter}{vast tegen haar keel}\\

\haiku{Zijn bleke pishuid?}{betasten als deeg van haar}{eigen substantie}\\

\haiku{{\textquoteright} {\textquoteleft}Laat je kapsel in.}{ieder geval natuurlijk}{en zonder linten}\\

\haiku{Ik vind die grote.}{kraag van zwart fluweel wel mooi}{op dat bleke blauw}\\

\haiku{Er zit een stukje.}{touw in  de klep van de}{smalle brievenbus}\\

\haiku{Zacht rochelend komt.}{langs de kapotte tong nog}{even adem naar buiten}\\

\haiku{o, nee, Marc, dat niet,,}{niet nu op deze trap wacht}{ik kom er van af}\\

\haiku{Al geiler wordend:}{fluisterde ze met hete}{lippen in mijn oor}\\

\haiku{Ze zegt dat zachtjes,,.}{niet hatelijk maar wel op}{een toon van verwijt}\\

\haiku{De Lardys hebben.}{me gevraagd om hen na mijn}{werk hier te treffen}\\

\haiku{En daarbij verzorgd,{\textquoteright}.}{als een prins zegt de dame}{nu gemoedelijk}\\

\haiku{De genotskrampen.}{wringen zich als vuurmessen}{door heel mijn lichaam}\\

\haiku{Zij kijkt de kamer.}{rond alsof ze niet meer heel}{goed weet waar ze is}\\

\haiku{Je wilt zeggen dat.}{je de transformatie als}{een totaalbeeld ziet}\\

\haiku{De slordige man.}{in de deuropening kijkt me}{achterdochtig aan}\\

\haiku{Ik kom weer in het.}{daglicht en zie haar op de}{stoep aan de overkant}\\

\haiku{Ze wentelt zich om.}{en gaat met het gezicht in}{de kussens liggen}\\

\haiku{Er stopt een auto.}{voor het huis in de stille}{straat met de tuintjes}\\

\haiku{Ik houd er niet van.}{dat anderen zich met mijn}{zaken bemoeien}\\

\haiku{Ze keek naar boven.}{en heeft mij ongetwijfeld}{aan het raam gezien}\\

\haiku{ik remmen verder,}{in de straat maar voor dat ik}{aan het tuinhek ben}\\

\haiku{Een vrouw rent uit een,:}{huis aan de overkant met de}{handen in de lucht}\\

\haiku{Een kind is voor mijn,.}{wagen gevallen maar ik}{heb hem niet geraakt}\\

\haiku{Het kan niet meer uit,.}{mijn nekmerg het kan niet meer}{van het netvlies af}\\

\haiku{Hij had honger want}{hij vertelde dat hij al}{acht dagen leefde}\\

\haiku{Ik schreef ook niet om.}{gefilmd te worden en nu}{is het toch gebeurd}\\

\haiku{{\textquoteright} Hij is ontstemd en.}{wordt het nog meer wanneer ik}{zijn aanbod weiger}\\

\haiku{Het provoceert een.}{druk op mijn ingewanden}{die me naar de w.c}\\

\haiku{En heb ik met mijn?}{achterhoofd de tegels van}{het toilet geraakt}\\

\haiku{Ik huil van wanhoop.}{en lig uren op mijn rug in}{de nacht te staren}\\

\haiku{{\textquoteright} {\textquoteleft}Fally,{\textquoteright} zegt ze, {\textquoteleft}erg.}{knap en intelligent maar}{zo links als de pest}\\

\haiku{Ik ben nog nooit een.}{intelligent journalist}{tegengekomen}\\

\haiku{Je hoeft dat niet te.}{zijn omdat de vrouwen je}{vanzelf graag mogen}\\

\haiku{Ze laat zich door Bunk.}{met zijn fluwelen neus op}{haar wang liefkozen}\\

\haiku{tussen haakjes, met...{\textquoteright} {\textquoteleft}}{het voorbeeld dat het kind thuis}{altijd gehad heeft}\\

\haiku{Je gaat een beetje,,{\textquoteright}.}{te ver Georges merkte}{Veronica op}\\

\haiku{Met mijn mond in haar.}{kut en mijn vingers in haar}{anus of omgekeerd}\\

\haiku{Wanneer we er uit:}{komen legt ze een hand op}{mijn schouder en zegt}\\

\haiku{loop je de kans dat '.}{ers ochtends een zwarte}{schorpioen in zit}\\

\haiku{Mij kop wordt onder.}{een kraan gehouden waaraan}{een waterzak zit}\\

\haiku{Ik voel het ritme.}{van mijn hartslag verhevigd}{slaan achter de ogen}\\

\haiku{dan word ik dronken...}{en als ik dronken ben kan}{ik niet schrijven}\\

\haiku{Zijn ogen rollen en.}{hij draait met zijn heupen als}{een buikdanseres}\\

\haiku{{\textquoteleft}Ze is zo warm dat.}{ze je aan het spreken zal}{brengen van waanzin}\\

\haiku{Boutan heeft de wacht,{\textquoteright}, {\textquoteleft}.}{zegt ze bedeesdik kan wel}{een uurtje blijven}\\

\haiku{Ze gaat tegen een,.}{muur staan met beide armen}{over de borst gekruist}\\

\subsection{Uit: Ghislaine la Bruy\`ere en ik}

\haiku{Zij was uitgestrekt,.}{met haar kleine buik tegen}{den grond gaan liggen}\\

\haiku{Ik verzocht haar een.}{middag in de stad om een}{rendez-vous}\\

\haiku{Den volgenden dag.}{sprak ik met Claude over de}{groote desillusie}\\

\haiku{Hij haalde een klein:}{verzenboek van Paul Geraldi}{uit zijn zak en zei}\\

\haiku{ik moest Ghislaine...,...}{ergens ontmoeten en dat}{geschiedde aldus}\\

\haiku{doch wie wilde plots!?}{dat uwe vrees en uw vreugde}{geen grenzen kende}\\

\haiku{Zij beheerschen u,;}{en het lokaas in beider}{handen is Adja}\\

\subsection{Uit: Marat, de marskramer}

\haiku{Niets of niemand mocht.}{hem aan zijn vorstelijke}{taak herinneren}\\

\haiku{Hij bleef bij de deur,.}{staan die achter hem door de}{knecht gesloten werd}\\

\haiku{Misschien werd wel juist.}{daarom zijn gevoel voor haar}{nog aangewakkerd}\\

\haiku{Duizenden mensen,.}{op straat die niets deden dan}{praten en zingen}\\

\haiku{Marat heeft zich in.}{het vest der zo misprezen}{meesters gestoken}\\

\haiku{Zijn voeten, waren.}{gehuld in het vilt van een}{oud soldatenvest}\\

\haiku{Even later werden.}{uit de hooikar enkele}{wapens geladen}\\

\haiku{Inmiddels stonden.}{de karpers in vuurvaste}{schotels in de oven}\\

\haiku{Mirabeau had zijn.}{stoutmoedige frase reeds}{tienmalen herhaald}\\

\haiku{Monarchisten en.}{Revolutionnairen}{zongen en juichten}\\

\haiku{{\textquoteright}... {\textquoteleft}Maar het gaat  er,.}{om dat men de jongens slaafs}{leert gehoorzamen}\\

\haiku{In de Assemblee.}{Nationale gingen}{stemmen op voor hem}\\

\haiku{Op de rand van zijn.}{dagblad schreef hij elke dag}{zijn aanmerkingen}\\

\haiku{{\textquoteleft}Ik had nochtans wat,...}{moeite om hem kapot te}{krijgen de bandiet}\\

\haiku{{\textquoteleft}Aan de lantaarn met,!}{de priesters die de eed niet}{willen afleggen}\\

\haiku{wel de traditie,.}{doch niet de Waarheid maakte}{er ons aan gehecht}\\

\haiku{De postmeester van.}{Chanitrix brengt vleesbouillon}{voor de kinderen}\\

\haiku{Een onbekende:}{nadert het rijtuig en roept}{met halfluide stem}\\

\haiku{vrienden wachten ons....}{nog anderhalf uur en wij}{worden afgelost}\\

\haiku{{\textquoteleft}Wij zullen wachten,{\textquoteright}.}{totdat het morgenlicht in}{de lucht is zucht hij}\\

\haiku{Hij wil emigreren!}{als de Cond\'e en d'Artois met}{het goud van Frankrijk}\\

\haiku{Maar het leek of de.}{stad Parijs met dit alles}{niets te maken had}\\

\haiku{Er kwamen nu af;}{en toe patrouilles aan bij}{het huis van Danton}\\

\haiku{De negentiende.}{Augustus schrijft Marat zijn}{groot requisitoir}\\

\haiku{In een wolk van stof.}{dringt deze duistere vloed}{tot aan de Abbaye}\\

\haiku{Een grote blonde:}{jongen werkte zich uit hun}{midden los en riep}\\

\haiku{Hij had bemerkt dat.}{zijn woorden geen rook waren}{geweest voor Marat}\\

\haiku{Barbaroux voelde!}{nu nog een vlaag van woede}{in hem opkomen}\\

\haiku{Een ogenblik leek het,.}{hem dat iemand zich in het}{trappenhuis bewoog}\\

\haiku{{\textquoteleft}Ik heb een boodschap,{\textquoteright}.}{voor citoyen Barbaroux}{herhaalde de stem}\\

\haiku{Morgenvroeg kun je.}{in de rue de la Perle}{je honger stillen}\\

\haiku{{\textquoteleft}Dit is de vrucht van.}{mijn grenzeloze liefde}{voor de Republiek}\\

\haiku{Met een tedere.}{streling gleed nu zijn hand over}{de flank van het dier}\\

\haiku{{\textquoteleft}Ik sterf zonder schuld,!}{aan de misdaden waarvan}{ik  wordt beticht}\\

\haiku{De vogeldiertjes,}{fluiten in de bessenstruik}{kom je nog niet aan}\\

\haiku{{\textquoteleft}Ik vind dat je me.}{toch eindelijk wel eens je}{voornaam noemen mag}\\

\haiku{{\textquoteright} {\textquoteleft}Wat ik er in draag,,!...}{is niet alleen voor vandaag}{het is voor altijd}\\

\haiku{Hij keek Barr\`ere:}{met een spottende blik}{aan en zij opeens}\\

\haiku{Dacht je soms, dat het,?}{volk vergeten zou wat je}{voor hen gedaan hebt}\\

\haiku{Dacht je soms dat de!}{loopjongens mijn krant uit hun}{mouw kunnen schudden}\\

\haiku{Maar het leek meer een!}{samenraapsel van Rouxisten}{dan Royalisten}\\

\haiku{{\textquoteleft}Kijk eens naar mij, mooi,,...}{duifje of ben je op weg}{naar je biechtvader}\\

\haiku{Toen viel Charlotte's.}{oog op een kleine dolk met}{een ivoren handstuk}\\

\haiku{Zij komt weer in haar.}{hotel en gaat zich in haar}{kamer opsluiten}\\

\haiku{Hij liet haar een kop.}{koffie naar boven brengen}{met een krakeling}\\

\haiku{Zij weet nu, dat ze.}{dagen en nachten lang zo}{zou kunnen wachten}\\

\subsection{Uit: Mijn hart zal niet vrezen}

\haiku{{\textquoteleft}Al legert zich ook,.}{tegen mij een krijgsmacht mijn}{hart zal niet vrezen}\\

\haiku{Maar in zijn toogzak....}{greep zijn hand instinctief naar}{de portefeuille}\\

\haiku{Blonken er een paar?}{sterren door het vreemdsoortig}{gewei van de boom}\\

\haiku{Hij stond lichtelijk.}{in een heup geknikt als de}{Venus van Milo}\\

\haiku{glou{\textquoteright}, deed hij met zijn,.}{keel alsof hij er een fles}{in liet uitlopen}\\

\haiku{Celeste was toen..}{twaalf jaar en wij maakten het}{voor haar in orde}\\

\haiku{De kostelijke.}{rechtvaardigheid richt zich door}{het schepsel tot God}\\

\haiku{je kinderen te....}{verwaarlozen om in de}{stad rond te hangen}\\

\haiku{Er werd een oude.}{bonbondoos uitgehaald met}{vergeelde foto's}\\

\haiku{Servaas was verbaasd.}{en verzekerde dat het}{geen sou kosten zou}\\

\haiku{Ik zal  pogen.}{hen een beetje gelukkig}{te leren leven}\\

\haiku{Misschien kan ik zelf....}{wel bij een van mijn vrienden}{daarvoor aankloppen}\\

\haiku{Maar de nooddruft heeft.}{van geslacht op geslacht de}{wegen verduisterd}\\

\haiku{Neen, hier werd hij de!}{ontzaglijke taak van het}{priesterschap gewaar}\\

\haiku{Hij versnelde zijn.}{passen en voelde het bloed}{door zijn aderen slaan}\\

\haiku{Motors van tanks en,.}{vlammenwerpers die nog nooit}{gebruikt zijn geweest}\\

\haiku{als men buiten de,!}{Kerk is en men twijfelt dan}{is men buit voor God}\\

\haiku{Het waren allen.}{dezelfde mensen in dat}{wonderlijke licht}\\

\haiku{Maar nu klonk het zo,.}{onrustbarend dat hij naar}{het tuinpoortje liep}\\

\haiku{Bij het venster stond,.}{Moron naar het bed van de}{priester te staren}\\

\haiku{{\textquoteleft}Wat wil je, kindje,.}{je kunt een dode geen nieuw}{leven inblazen}\\

\haiku{haal mijn broertje uit,..{\textquoteright} {\textquoteleft}....}{het graf en de goede man}{deed hetAls je blieft}\\

\haiku{Het was of zij de....}{kans kreeg om haar handen rond}{zijn keel te snoeren}\\

\haiku{{\textquoteleft}Is citroenwater?}{goedkoper zonder suiker}{dan op de prijslijst}\\

\haiku{Dat is een ramp voor{\textquoteright},.}{de caf\'e-houders vandaag}{zei hij stotterend}\\

\haiku{Het enige wat me,,.}{er in aanstaat is dat hij}{van angst barsten zal}\\

\haiku{Het betekende;}{voor de arbeiders ook het}{summum van weelde}\\

\haiku{Het voorval met de.}{Duitse motorfiets stond haar}{nog fris voor de geest}\\

\haiku{{\textquoteright} {\textquoteleft}De dokter heeft een..{\textquoteright} {\textquoteleft}}{rapport opgemaakt over een}{ernstige kwetsuur}\\

\haiku{Feitelijk voelde.}{hij reeds dat zijn komst hier iets}{belachelijks was}\\

\haiku{Het ontwapende.}{hem en beledigde hem}{tegelijkertijd}\\

\haiku{Hetgeen hen bindt, dat.}{zijn Uw verhoudingen tot}{hun eigen wereld}\\

\haiku{Het begrip dat zij,....}{zich van U vormen is dat}{Gij geschapen hebt}\\

\haiku{Ik moet u weer op{\textquoteright}, -.}{uw kleinheid van geest wijzen}{antwoordde Servaas}\\

\haiku{Soms verdroeg zij de.}{stilte van het landhuis niet}{meer en vluchtte weg}\\

\haiku{Temidden van de,.}{vijandig geworden hoon}{wandelt de priesteh}\\

\haiku{Tientallen malen.}{had ze iets bedacht om met}{hem alleen te zijn}\\

\haiku{Er was veel vis, maar.}{hij smaakte naar het vuil van}{de stadsriolen}\\

\haiku{In ieder geval{\textquoteright}.}{wel als hij een glas onder}{zijn neus heeft gehad}\\

\haiku{Op alle wijzen.}{die de mens veredelen}{en zijn hart openen}\\

\haiku{Dat het dieptepunt.}{van elk probleem reeds lang door}{Kristus is ontknoopt}\\

\haiku{{\textquoteleft}Zalig de armen,.}{van geest want hen behoort het}{rijk der hemelen}\\

\haiku{Boven het lage.}{bahut hing een stilleven}{met witte eenden}\\

\haiku{Toen begon hij met:}{een gebroken stemgeluid}{langzaam te spreken}\\

\haiku{{\textquoteleft}Als je zin hebt, mijn,.}{klein konijntje dan kun je}{met mij mee-eten}\\

\haiku{Het was alsof hij.}{in de uiterste hoek werd}{teruggeslagen}\\

\haiku{In die toestand vond,.}{C\'eleste hem die kijken kwam}{waar haar vader bleef}\\

\haiku{Tussen acht en elf '.}{uurs avonds had de wind de}{grond droog geblazen}\\

\haiku{Intussen was het.}{lijk van de priester in de}{gang blijven liggen}\\

\haiku{{\textquoteleft}Ik bedoel...., werp eens,....}{een oogje in die richting}{je kunt nooit weten}\\

\subsection{Uit: Spiegelgruis}

\haiku{Zij is niet bedroefd,.}{maar zij is gegrepen door}{iets onheilspellends}\\

\haiku{Entre ta chair et,.}{la mienne un r\^eve}{brulant m'a poss\'ed\'ee}\\

\haiku{Mon corps tout entier, -.}{s'est livr\'e \`a tes l\`evres}{infatigables H\'elas}\\

\haiku{Zijn lichaamskracht geldt,}{die van twee jonge vrouwen}{totdat zij beiden}\\

\haiku{In dit ogenblik wist.}{zich voor het eerst mijn telling}{der minuten uit}\\

\haiku{De ontelbaren,.}{voor wie de dood een zoete}{verlossing zou zijn}\\

\haiku{Niemand denkt meer aan,.}{het brood dat hij vanochtend}{niet gekregen heeft}\\

\haiku{Mijn hoofd  wordt als.}{het ware door een nevel}{van goud licht omstroomd}\\

\haiku{Dan staat hij wijdbeens.}{recht en tussen zijn knie\"en}{door zie ik de dood}\\

\haiku{Hij krijgt een blauwe.}{linnen werkbroek en een trui}{met mottengaten}\\

\haiku{Rustige avonden.}{vliegen met merkwaardige}{flitsen door mijn hoofd}\\

\haiku{In de duisternis.}{en door het slijk zoek ik de}{weg naar de Rijksbaan}\\

\haiku{se tirent les poils!}{des fesses pour se faire}{des cure-dents}\\

\haiku{De toren van het.}{mergelkerkje steekt boven}{de heuvelrug uit}\\

\haiku{- Het is bij twee uren,,.}{heren ik heb hier niets meer}{aan toe te voegen}\\

\haiku{Maar Tigre schijnt,:}{niets meer aux s\'erieux te}{nemen hij zingt}\\

\haiku{Ik heet Koenraad, zegt,.}{hij ik moet u de groeten}{doen van Tigre}\\

\haiku{Het landhuis van mijn.}{vader en het dochtertje}{van de notaris}\\

\haiku{Ik betrap mij er,.}{op dat ik haar als een dwaas}{zit te bestaren}\\

\haiku{Tegen middernacht.}{kwam ik ziek en dronken de}{slaapkamer binnen}\\

\haiku{Het zijn zij, die op,.}{het juiste ogenblik slaan waar}{zij te slaan hebben}\\

\haiku{Bij C\'eline is ',.}{het steedst onverwachte}{dat je overweldigt}\\

\haiku{Over een half uur ben,...}{ik aan de Pont Neuf waar we}{de metro nemen}\\

\haiku{Bij de uitgang ziet,.}{Tigre dat er een trein}{is aangekomen}\\

\haiku{Zij trok hem tegen.}{zich aan en nestelde haar}{gelaat in zijn nek}\\

\haiku{Zij zijn de plaatsen.}{waar de natuur de mens nog}{zijn crediet verschaft}\\

\haiku{Wat er verpulvert,;}{keert terug tot aarde en}{damp tot vruchtbaarheid}\\

\haiku{De anarchie doet.}{hiermede haar intrede}{in het cultuurbeeld}\\

\haiku{Doch dan weer, haar hand:}{over de zijne schuivend en}{met gedempte stem}\\

\haiku{Zij maakt een einde.}{aan al deze oprechte}{en geveinsde ernst}\\

\haiku{De belastingen,,!}{de flikflooierijen de}{stijgende waanzin}\\

\haiku{Maar sinds maanden is.}{er geen weekheid naar mijn keel}{geweld als deze}\\

\haiku{Het lijkt mij plots of.}{ik van de ene wereld in}{de andere leef}\\

\haiku{Een ontzettende,.}{gemeenzaamheid waarvan de}{tragiek mij aangrijpt}\\

\haiku{Haar handen liggen.}{gevouwen in het smalle}{vlak van haar bekken}\\

\haiku{neen C\'eline, ik;}{verwarm me in het zachte}{vuur van je schoonheid}\\

\haiku{Ik kruip op mijn buik,.}{door het sparrebos greppel}{na greppel zoekend}\\

\haiku{Voordat ik de tijd,}{heb om vier figuren te}{ontwaren vallen}\\

\haiku{Dan hoort hij knallen,,.}{achter zich onder op de}{weg langs het water}\\

\subsection{Uit: Uitdagend spel}

\haiku{Al wie zich tegen,.}{iets verzet heeft er nog niet}{mee afgerekend}\\

\haiku{Een mens wordt altijd.}{meer bepaald door zijn actie}{dan door zijn denken}\\

\haiku{De vrouw bekeek hem.}{even en ging toen door met het}{wassen van de vaat}\\

\haiku{Onder het slechte.}{licht zag Charat zijn vet zwart}{krullend haar glanzen}\\

\haiku{Maar hij bleef op zijn.}{hoede om elke plotse}{slag te ontwijken}\\

\haiku{Hij herinnerde:}{zich wat er gebeurd was en}{voegde eraan toe}\\

\haiku{Maar ik zou ook graag...}{weten of zijn handen er}{verzorgd uitzagen}\\

\haiku{Synthese is maar.}{een voorbijgaande ziekte}{van het intellect}\\

\haiku{{\textquoteright} zei Serge somber, {\textquoteleft}.}{ik voel er niets meer voor om}{nog te antwoorden}\\

\haiku{{\textquoteright} {\textquoteleft}Tegenwoordig zijn,.}{de omgangsvormen vrijer}{dan vroeger Daddy}\\

\haiku{Hij stond klaar om de.}{rechterarm van de Afrikaan}{uit het lid te slaan}\\

\haiku{Die eenzaamheid was.}{iets uit pijn en begeerte}{samengevlochten}\\

\haiku{IX Charat lag naast,.}{een stapel boeken op de}{grond lang uitgestrekt}\\

\haiku{... langgerekt en met.}{een pijnlijke klemtoon om}{nostalgisch te zijn}\\

\haiku{{\textquoteright} terwijl hij in zijn.}{barkast kijken ging wat hij}{hun aan kon bieden}\\

\haiku{Voordat hij de slag,.}{kon afweren dreunde het}{al  in zijn hoofd}\\

\haiku{wie er verder voor.}{jullie komst nog bij Perez}{aanwezig waren}\\

\haiku{{\textquoteright} De man sloop langs de.}{achterdeur naar buiten als}{een geslagen hond}\\

\haiku{De Huid nam de hoorn'.}{op van het toestel dat op}{Perez tafel stond}\\

\haiku{Als een in elkaar.}{gezakt ding ging hij in zijn}{bureaustoel zitten}\\

\haiku{Een diep en duister.}{heimwee doorspoelde hem als}{een vaag zwart water}\\

\haiku{Het gebeurde dat.}{er ogenblikken waren dat}{zij hem verraste}\\

\haiku{dat verticale.}{gevoel van de feniks die}{uit zijn as herrijst}\\

\haiku{Nergens vond hij wat,.}{hij zocht wat hij meende te}{hebben uitgedrukt}\\

\haiku{{\textquoteleft}Ik begrijp niet wat,.}{ik heb uit te staan met uw}{geheimen mijnheer}\\

\haiku{{\textquoteright} Zij bleef bij de deur,:}{van het atelier staan alsof}{zij zichzelf afvroeg}\\

\haiku{Hij ging naar haar toe.}{en liet zijn hand vriendelijk}{over haar hoofd glijden}\\

\haiku{{\textquoteright} {\textquoteleft}Maja,{\textquoteright} vroeg Charat, {\textquoteleft}.}{onverwachtnoem me eens iets}{waar je wel van houdt}\\

\haiku{{\textquoteleft}Soms hield ik met een,.}{grenzeloos gevoel van hem}{soms haatte ik hem}\\

\haiku{Hij aarzelde, maar:}{sprak toen zacht een zin uit die}{in hem opwelde}\\

\haiku{{\textquoteleft}Ik ben op weg naar,{\textquoteright}.}{het Paradijs ik heb mijn}{wandelschoenen aan}\\

\haiku{Serge zag dat hij.}{de oren spitste om niets van het}{gesprek te missen}\\

\haiku{Zij zag een bijna.}{trieste zinnelijkheid in}{zijn vragende blik}\\

\haiku{De werkelijkheid.}{balde zich te zamen in}{dit snikkend lichaam}\\

\haiku{Hij overwoog reeds op.}{welke wijze en met wie}{hij zijn slag zou slaan}\\

\haiku{De gedachte aan.}{een benzinepomp leek hem}{niet onuitvoerbaar}\\

\haiku{Twee agenten slopen.}{in gebukte houding naar}{de Byzantijn toe}\\

\section{E.L. Franken}

\subsection{Uit: Staatsgeheim}

\haiku{{\textquoteright} {\textquoteleft}Ik arme blijf dus,{\textquoteright}.}{met mijn koude dronk alleen}{zuchtte Mijnsbergen}\\

\haiku{{\textquoteleft}Voor mijn part, maar als.}{er wat gebeurt is het op}{jouw verantwoording}\\

\haiku{{\textquoteleft}Denk je dat ze me?}{alleen in het gezelschap}{van die geesten laat}\\

\haiku{Als u zich verder}{nog zoo aanstelt dwingt u mij}{u Knock-Out te slaan}\\

\haiku{Op de andere.}{kant van de straat stonden de}{parkeerende autos}\\

\haiku{Nou wees dan maar blij.}{dat je van avond de eer hebt}{me te leeren kennen}\\

\haiku{Mijn God hoe lang is.}{het geleden dat ik een}{druppel gezien heb}\\

\haiku{{\textquoteright} {\textquoteleft}Die deur ligt aan de.}{achterkant van het huis en}{komt in de tuin uit}\\

\haiku{Op de drempel stond {\textquoteleft}{\textquoteright},.}{met zijnheilig glimlachje}{Cornelis Baron}\\

\haiku{Vorst verstond de kunst.}{de pati\"enten op hun}{gemak te stellen}\\

\haiku{{\textquoteright} Nog lang nadat zij.}{de kamer verlaten had}{dacht hij over haar na}\\

\haiku{ofschoon ik geloof,.}{dat wij ook dien dader nog}{wel zullen pakken}\\

\haiku{een feit is het, dat.}{hij steeds in de buurt van de}{zaak te vinden was}\\

\haiku{{\textquoteright} Verveeld nam de heer.}{uit de portefeuille een}{diplomatentasch}\\

\haiku{Daar buiten in mijn.}{zeilboot heb ik de tijd om}{te leeren berusten}\\

\haiku{{\textquoteright} Een oogenblik dacht,.}{Mary na maar dan knikte}{zij hem alweer toe}\\

\haiku{Daar heb ik mij over,;}{verheugd omdat het onze}{werkkracht verhoogde}\\

\haiku{{\textquoteright} Het bellen van de.}{telefoon sneed den dokter}{ieder antwoord af}\\

\haiku{Toen voelde zij hoe.}{de handen van den man over}{haar lichaam gleden}\\

\haiku{Hij deed het licht uit,.}{sloot de deur af en begaf}{zich naar zijn woning}\\

\haiku{Met lichtveerende.}{stappen sprong hij de trap op}{en opende de deur}\\

\haiku{Haastig scheurde hij.}{de enveloppe open en}{ontvouwde de brief}\\

\haiku{{\textquoteright} Vorst vroeg zich af hoe.}{de brief op zijn schrijftafel}{kon zijn gekomen}\\

\haiku{De spelers spraken - -;}{deels uit afgunst en nijd over}{haar achteloosheid}\\

\haiku{Ik had nog een boek,.}{van U dat wilde ik U}{even terug brengen}\\

\haiku{{\textquoteright} En toen Baron haar,:}{zonder haar te begrijpen}{aankeek meende zij}\\

\haiku{Hij werd aan de kant,;}{geslingerd zijn rechterschoen}{was opengereten}\\

\haiku{Met gedempte stem.}{sprak Grudel in het mondstuk}{van de telefoon}\\

\haiku{De heeren stegen.}{uit en naderden zonder}{te spreken de deur}\\

\haiku{{\textquoteleft}Zoo'n keuken zou men,.}{hier zeker niet verwachten}{evenmin als de rest}\\

\haiku{Maar de moord op Dr.;}{Vorst zou ook uit een ander}{motief denkbaar zijn}\\

\haiku{Voorzichtig, latje,.}{voor latje liet Smit zich naar}{beneden glijden}\\

\haiku{Ik kon niets zien, maar,.}{toch had ik het gevoel dat}{hij er nog moest zijn}\\

\haiku{Als dit antieke!}{kamermeisje nu maar eens}{wat spraakzamer was}\\

\haiku{Het schoot hem opeens,.}{te binnen dat hij met haar}{had afgesproken}\\

\haiku{{\textquoteright} Een klopje op de.}{deur sneed het antwoord van den}{Hoofdinspecteur af}\\

\haiku{Hier kreeg hij voor de.}{tweede keer deze morgen}{een teleurstelling}\\

\haiku{{\textquoteright} {\textquoteleft}Kom, zet nu eens je,.}{detective-oogen op het}{is werkelijk niets}\\

\haiku{En al zou het tot -,.}{een fiasco komen ik}{ben jong inspecteur}\\

\haiku{{\textquoteright} Nijman lachte en.}{ook Smit kon zijn gevoelens}{niet onderdrukken}\\

\haiku{Stuk voor stuk nam Smit.}{ze eruit en liet ze door}{zijn vingers glijden}\\

\haiku{U zult dus Uw dienst.}{bij de Bank ongehinderd}{kunnen voortzetten}\\

\haiku{Veel meer zou het mij.}{interesseeren den heer}{Grudel te spreken}\\

\haiku{De Hoofdinspecteur.}{nam het wapen in de hand}{en onderzocht het}\\

\subsection{Uit: De vulpendetective}

\haiku{{\textquoteleft}Uw vrouw zal zonder,.}{twijfel vinden dat U zeer}{juist heeft gehandeld}\\

\haiku{Met oorverdoovend.}{lawaai zette de wagen}{zich in beweging}\\

\haiku{GEVONDEN: tusschen.}{Haarlem en Amsterdam LEEREN}{DAMESHANDTASCHJE}\\

\haiku{{\textquoteright} Th\'er\`ese Grenier.}{ging in een fauteuil naast de}{schrijftafel zitten}\\

\haiku{Of had U verwacht,?}{dat ik er onmiddellijk}{in zou toestemmen}\\

\haiku{{\textquoteright} Met belangstelling.}{sloeg zij de uitwerking van}{haar woorden gade}\\

\haiku{Alle kleur was uit.}{zijn gezicht geweken en}{zijn handen beefden}\\

\haiku{Onbeweeglijk zat.}{hij aan zijn schrijftafel en}{staarde voor zich uit}\\

\haiku{Hij stak zijn pijp weer,.}{aan maar legde haar na een}{paar trekken weer weg}\\

\haiku{Waarom bracht hij je,?}{niet naar de boot als hij je}{wilde laten gaan}\\

\haiku{{\textquoteright} {\textquoteleft}Het is de eenige,{\textquoteright}.}{belooning die ik vraag zei}{Baron glimlachend}\\

\haiku{{\textquoteleft}Gelooft U niet, dat,?}{het beter zou zijn wanneer}{U mij vertrouwde}\\

\haiku{Tusschen rijdende.}{taxi's en fietsers door liep hij}{naar de Kalverstraat}\\

\haiku{Dat  is nog vroeg,.}{genoeg om het vliegtuig naar}{Londen te halen}\\

\haiku{Ik dacht al, dat U,.}{mij heelemaal vergeten}{had mijnheer Baron}\\

\haiku{Ik zal U in de,.}{gelegenheid stellen er}{over na te denken}\\

\haiku{De dokter heeft een,.}{betere opinie over mij}{dan U inspecteur}\\

\haiku{{\textquoteright} Sadie was naast haar.}{gaan staan en had den arm om}{haar schouder gelegd}\\

\haiku{- De boodschap, die ik,.}{hem nu ga brengen is wel}{een rijksdaalder waard}\\

\haiku{Toen zij het schmink van,:}{haar gezicht verwijderde}{zei ze plotseling}\\

\haiku{{\textquoteright} vroeg Bronsdijk en hij,.}{zag hoe zij bewonderend}{naar het sieraad keek}\\

\haiku{Hij zou misschien wel,.}{in staat zijn geweld tegen}{U te gebruiken}\\

\haiku{{\textquoteright} {\textquoteleft}Dan moest je weten,.}{dat hij een dame ook als}{dame behandelt}\\

\haiku{Maar ik kan je w\`el,.}{zeggen dat ik niet graag in}{zijn schoenen zou staan}\\

\haiku{Bemoei je liever.}{met je eigen zaken en}{laat Bronsdijk met rust}\\

\haiku{Heb je wel eens van,?}{een tijger gehoord die goed}{voor een gazel was}\\

\haiku{{\textquoteright} {\textquoteleft}Al tien dagen lang.}{stuurt hij mij elken avond de}{prachtigste bloemen}\\

\haiku{Hij bewonderde.}{Th\'er\`ese en had eens veel}{van haar gehouden}\\

\haiku{Het was niet moeilijk,:}{voor haar zijn gedachten te}{raden en zij vroeg}\\

\haiku{Maar de boekhouder.}{was onverrichter zake}{teruggekomen}\\

\haiku{Wanneer U teekent,.}{blijft dit papier rustig in}{mijn safe liggen}\\

\haiku{Zoodra Bronsdijk,.}{alleen was begon hij heen}{en weer te loopen}\\

\haiku{Hij drukte op de.}{bel en dadelijk kwam de}{boekhouder binnen}\\

\haiku{heeft mijn chef er over,,?}{gesproken dat hij van plan}{is mij te ontslaan}\\

\haiku{Zij sloot de oogen, haar:}{gezicht was verwrongen van}{smart en zij kreunde}\\

\haiku{- Middeldorp moet haar.}{morgen vroeg om acht uur uit}{haar woning halen}\\

\haiku{{\textquoteright} Intusschen was dr..}{de Jong met het onderzoek}{gereed gekomen}\\

\haiku{- Toen ik de misdaad,.}{ontdekte probeerde ik}{te telefoneeren}\\

\haiku{{\textquoteright} Zij keek hem aan en,.}{uit haar oogen sprak een wanhoop}{die Smit ontroerde}\\

\haiku{Ik kreeg af en toe,,.}{den indruk dat zij het niet}{goed vond dat ik kwam}\\

\haiku{{\textquoteleft}Inspecteur, - is het -, -?}{absoluut noodzakelijk}{dat mijn man hier blijft}\\

\haiku{Ik hoorde, wat er,.}{gebeurd was en wilde haar}{niet alleen laten}\\

\haiku{{\textquoteright} Zij probeerde den.}{knop weer aan de paraplu}{te bevestigen}\\

\haiku{Ik moet U een paar,.}{vragen stellen die juffrouw}{Russell betreffen}\\

\haiku{Niemand weet, waar zij,.}{is want zij heeft geen boodschap}{achter gelaten}\\

\haiku{Als U van mijn schuld,,.}{overtuigd bent is het Uw zaak}{het te bewijzen}\\

\haiku{, vindt U het goed, dat?}{ik hier een paar woorden met}{mijnheer Baron spreek}\\

\haiku{{\textquoteright} {\textquoteleft}Inspecteur, weest U.}{niet zoo geheimzinnig in}{Uw uitdrukkingen}\\

\haiku{{\textquoteright} {\textquoteleft}Het is niet zoo gek,,,{\textquoteright}.}{als U denkt Sadie zei hij}{nauwelijks hoorbaar}\\

\haiku{Als U zegt, dat ik,,}{Amelie er mee help zal ik}{er niet naar toe gaan}\\

\haiku{Maar dikwijls, als zij,:}{met hem samen was en ik}{haar blik zag dacht ik}\\

\haiku{Ik telegrafeer,.}{nu naar de jongens dat de}{zaak in orde is}\\

\haiku{En als je nog wilt - -.}{wat je mij toen hebt gevraagd}{vandaag zeg ik ja}\\

\haiku{De hoofdinspecteur.}{stond op en ging haar een paar}{passen tegemoet}\\

\haiku{Maar dat komt zeker,,.}{door de thrillers die men leest}{en ook door de film}\\

\haiku{Maar zij kon toch niet,.}{gelooven dat Joe deze daad}{zou hebben begaan}\\

\haiku{Als ik zeg elf uur,,.}{dan bedoel ik ook elf uur}{geen minuut later}\\

\haiku{{\textquoteright} Met zijn linker hand.}{nam bij het blad papier aan}{en gaf het van Dam}\\

\haiku{Dank U. - U ziet er.}{eigenlijk veel te netjes}{uit voor een speurhond}\\

\haiku{Ik kon die rommel,,.}{die men mij voorzette niet}{naar binnen krijgen}\\

\haiku{{\textquoteright} Van Dam was bij hem.}{komen staan en nam een van}{de patronen op}\\

\haiku{{\textquoteright} {\textquoteleft}Het was een domheid,.}{haar in de dickey-seat}{te laten zitten}\\

\haiku{{\textquoteright} {\textquoteleft}Je wist, dat ik al.}{die papieren zoolang bij}{mij had gestoken}\\

\haiku{Tot nu toe hebben,{\textquoteright}.}{wij nog niets van hem gehoord}{antwoordde van Dam}\\

\haiku{{\textquoteleft}Ik begrijp er niets,.}{van dat het antwoord uit Ault}{nog niet binnen is}\\

\haiku{{\textquoteleft}Mocht ik over een uur,,.}{niet terug zijn dan weet U}{wat U te doen staat}\\

\haiku{Het leek hem beter,.}{zonder licht de gang van het}{huis te betreden}\\

\haiku{{\textquoteleft}Hallo, inspecteur -!}{U wordt door de hemel hier}{naar toe gezonden}\\

\haiku{Zoo onhebbelijk:}{en scherp mogelijk snauwde}{hij tegen Baron}\\

\haiku{{\textquoteleft}Ik geloof, dat ik.}{mij rustig aan Uw zorgen}{kan toevertrouwen}\\

\haiku{De wagen kwam van;}{de Weteringschans en scheen}{vaart te minderen}\\

\haiku{Hij draaide het licht -.}{op en keek in vijf op hem}{gerichte brownings}\\

\haiku{Ik krijg mijn ontslag - -.}{mijn carri\`ere alles}{is afgeloopen}\\

\subsection{Uit: Wie heeft de admiraal gewurgd?}

\haiku{De kapitein liet.}{den inspecteur en Clive}{Harrow binnengaan}\\

\haiku{Deze heer heeft zijn.}{dekstoel tot ongeveer vijf}{uur niet verlaten}\\

\haiku{{\textquoteright} Zwijgend, zonder hem,.}{aan te zien legde zij haar}{hand op de zijne}\\

\haiku{Ik hoop dat u het,.}{mij niet kwalijk neemt dat ik}{daar niet op antwoord}\\

\haiku{{\textquoteright} {\textquoteleft}Dan zullen wij u,.}{naar het station brengen}{als u het goed vindt}\\

\haiku{Misschien was van Soest.}{op reis en zijn bediende}{daarom met verlof}\\

\haiku{Hoofdinspecteur, de.}{bediende van mijnheer van}{Soest is gekomen}\\

\haiku{Daar zullen wij het.}{op het politiebureau}{nog eens over hebben}\\

\haiku{Tegen half negen,}{ging hij naar de hal maar hij}{zag den man niet dien}\\

\haiku{{\textquoteleft}Het doet mij net zoo,,}{veel pleizier als u en ik}{geloof dat dit komt}\\

\haiku{Zij was een paar jaar,,.}{jonger dan Jeanne slank}{blond en zeer sportief}\\

\haiku{{\textquoteright} {\textquoteleft}Ik geloof, dat u,.}{het meent wanneer u ons met}{de zon vergelijkt}\\

\haiku{En als zij het is,.}{noodigt u haar dan uit bij ons}{te komen zitten}\\

\haiku{Ik heb mij dikwijls,.}{afgevraagd of wij elkaar}{nog eens zouden zien}\\

\haiku{Maar vergeet u niet,.}{dat ik ook nu nog hiertoe}{zou kunnen overgaan}\\

\haiku{dat,{\textquoteright} bromde Wolters,:}{binnensmonds en zich tot van}{Dam wendend zei hij}\\

\haiku{{\textquoteleft}Wat weet u van de,?}{menschen die bij mijnheer van}{Soest over huis kwamen}\\

\haiku{Dan begaf hij zich.}{naar den hoofdcommissaris}{voor een bespreking}\\

\haiku{Smit reed vroeger als.}{anders naar zijn woning in}{de Euterpestraat}\\

\haiku{{\textquoteright} {\textquoteleft}Een vriendin van hem.}{is voor een korten tijd naar}{den Haag gekomen}\\

\haiku{Dit intermezzo.}{won voor hem door Mary's vraag}{nog aan beteekenis}\\

\haiku{{\textquoteleft}Corry, mijnheer hier,.}{vraagt hoelang mijnheer Wolters}{nu al bij ons woont}\\

\haiku{Een paar tafeltjes.}{van Harrow verwijderd zag}{zij Riemsdijk zitten}\\

\haiku{{\textquoteleft}Mister Murphy Trast -,{\textquoteright}.}{juffrouw Jeanne Morrees}{stelde Riemsdijk voor}\\

\haiku{{\textquoteright} {\textquoteleft}Wel - ik heb hem nog.}{nooit zoo gunstig over iemand}{hooren oordeelen}\\

\haiku{Uw vader schijnen.}{de tropen beter te zijn}{bekomen dan mij}\\

\haiku{{\textquoteright} Smit ging het kantoor.}{binnen en zag drie heeren}{tegenover zich staan}\\

\haiku{{\textquoteright} vroeg hij met zachte,.}{stem toen hij met Smit in de}{hal was aangeland}\\

\haiku{Het gaat hier om meer.}{dan de belangen van een}{schietvereeniging}\\

\haiku{Harrow en Sewell.}{zaten in diepe fauteuils}{tegenover elkaar}\\

\haiku{Hij heeft zich - zoover ons -.}{bekend is in ons land aan}{niets schuldig gemaakt}\\

\haiku{{\textquoteright} {\textquoteleft}O, dan is alles,{\textquoteright}.}{te begrijpen antwoordde}{Harrow glimlachend}\\

\haiku{Het heeft er den schijn,.}{van dat jouw loopbaan vooruit}{was afgebakend}\\

\haiku{{\textquoteright} Met een bitteren.}{glimlach had hij de laatste}{woorden gesproken}\\

\haiku{Eindelijk was de,.}{dag gekomen waarop hij}{met smart had gewacht}\\

\haiku{Ik herinner mij,,;}{dat het juist acht uur sloeg toen}{mijnheer Ruissaard kwam}\\

\haiku{Hij keerde naar de.}{bibliotheek terug en}{trad aan het venster}\\

\haiku{Voorzichtig koos hij,:}{zijn  woorden toen hij met}{gedempte stem zei}\\

\haiku{Als een gloeiende.}{bal verzonk de zon aan den}{horizon in zee}\\

\haiku{{\textquoteleft}Ik geloof,{\textquoteright} begon, {\textquoteleft}.}{Ruissaard ten slottedat wij}{gauw klaar zullen zijn}\\

\haiku{{\textquoteright} {\textquoteleft}Ja, waarschijnlijk een,{\textquoteright}.}{dubbelganger zei Ruissaard}{schouderophalend}\\

\haiku{{\textquoteleft}Ik geloof, dokter,,{\textquoteright}.}{dat dit spoedig noodig zal zijn}{antwoordde Smit}\\

\haiku{Hij had tijd genoeg,.}{en gaf zijn chauffeur opdracht}{langzaam te rijden}\\

\haiku{Ik verwijt mijzelf,.}{dikwijls dat ik u veel te}{dikwijls uw zin geef}\\

\haiku{Hij zag duidelijk,.}{dat zij steeds onrustiger}{begon te worden}\\

\haiku{Je mag er niet aan - -?}{te gronde gaan Jeanne}{hoor je wat ik zeg}\\

\haiku{Vanuit een kleine.}{kamer voerde een smalle}{trap naar den zolder}\\

\haiku{{\textquoteright} Langzaam ging de man,.}{zitten zonder Smit met zijn}{oogen los te laten}\\

\haiku{Vertelt u mij dus,.}{zoo kort mogelijk wat zich}{hier heeft afgespeeld}\\

\haiku{Alles, wat de man,.}{naast haar uitsprak waren haar}{eigen gedachten}\\

\haiku{Hij voelde een stoot,.}{en juist toen hij viel trof hem}{een felle lichtschijn}\\

\haiku{{\textquoteleft}Ik weet, inspecteur,.}{dat u mij aan Engeland}{zult uitleveren}\\

\haiku{Pas toen het te laat,.}{was heb ik de geheele}{waarheid vernomen}\\

\haiku{Hoofdinspecteur Smit:}{ziet in deze kwestie twee}{mogelijkheden}\\

\section{Kester Freriks, A.F.Th. van der Heijden, Oek de Jong, Frans Kellendonk, Nicolaas Matsier, Doeschka Meijsing en Geerten Meijsing}

\subsection{Uit: Over God}

\haiku{dat geluk bestaat,.}{uit schijn met een offer moet}{je het afdwingen}\\

\haiku{Ik leg mijn hoofd op.}{haar schouder en neurie een}{liedje in haar oor}\\

\haiku{Ik ben benieuwd of.}{al die mensen nog op het}{plein zijn verzameld}\\

\haiku{Met het naderen.}{van de engelen groeide}{zijn begeerte nog}\\

\haiku{Albert begon de.}{graankorrels van de eerste}{heuvel te tellen}\\

\haiku{{\textquoteleft}U vraagt mij wat het,?}{voortreffelijkste voor de}{mens is majesteit}\\

\haiku{Je houdt onbewust.}{rekening met zijn bestaan}{of je doet het niet}\\

\haiku{De Zoon herrees uit,,.}{het graf en verloste ons}{aldus van de dood}\\

\haiku{De Zoon predikte,:}{de liefde maar het was een}{bepaald soort liefde}\\

\haiku{De Here Here.}{deed mij alras verlangen}{naar verzonkenheid}\\

\haiku{Te ontdekken uit?}{welke ritmes evenwicht en}{gemoedsrust ontstaan}\\

\haiku{Om de Melkweg te.}{beschrijven vergelijk je}{hem met een sluier}\\

\haiku{Het liefst sloop hij door.}{de benauwde kruipgangen}{onder het gebouw}\\

\haiku{Niemand heeft me nog.}{duidelijk kunnen maken}{wat geloven is}\\

\haiku{{\textquoteleft}Het is mij bang om,,;}{u mijn broeder Jonathan}{gij waart mij zeer lief}\\

\haiku{Of deden we het,?}{zelf geschapen naar Zijn beeld}{en gelijkenis}\\

\haiku{Niet was het om de '.}{fraaie tonen ent geluid}{dat ik ontlokte}\\

\chapter[16 auteurs, 1627 haiku's]{zestien auteurs, zestienhonderdzevenentwintig haiku's}

\section{Bish Ganga}

\subsection{Uit: Lalbahadoer. Een fatale liefde}

\haiku{zij door toedoen van.}{haar familie niet zijn vrouw}{zou kunnen worden}\\

\haiku{Ze deed de deur van.}{de knip en duwde hem open}{met een licht gekraak}\\

\haiku{Want ze had hem wel,.}{binnengelaten maar nog}{geen woord gesproken}\\

\haiku{Alle relaties.}{waren verstoord en hoe moest}{het nu verder}\\

\haiku{De moeder werd in.}{haar voorhoofd geraakt en viel}{ter plekke dood neer}\\

\haiku{Politieagenten.}{bewaken de woning van}{Lalbahadoer}\\

\haiku{Het kostte hem veel.}{moeite de koeien in het}{gareel te houden}\\

\haiku{Ze waren bang om.}{doodgeschoten te worden}{door Lalbahadoer}\\

\haiku{In dat geval was.}{hij beter bewapend op}{patrouille gegaan}\\

\haiku{begaat, waarbij zijn.}{geliefde en haar moeder}{het leven laten}\\

\haiku{Het natrekken van.}{de tips liet naar hun gevoel}{te wensen over}\\

\haiku{Vanaf dat moment.}{werd Lalbahadoer zichtbaar}{nog onrustiger}\\

\haiku{Elke indringer.}{van het domein zou weldra}{opgemerkt worden}\\

\haiku{{\textquoteleft}Jammer dat het niet{\textquoteright}.}{tot een sprookjeshuwelijk}{is gekomen}\\

\haiku{Dit alles vond plaats.}{onder leiding en toezicht}{van de politie}\\

\haiku{Dat was voor Janssen.}{het signaal dat hij zich niet}{vergist kon hebben}\\

\haiku{Het was - zo op het -.}{eerste gezicht met respect}{in het graf gelegd}\\

\haiku{Met de valstrik is.}{een geweer met gespannen}{trekker verbonden}\\

\section{Rudolf Geel}

\subsection{Uit: Een afgezant uit niemandsland}

\haiku{Rudolf Geel Een}{afgezant uit niemandsland}{Colofon}\\

\haiku{Paul hing midscheeps over,}{de reling niet meer in staat}{te voelen hoe ziek}\\

\haiku{Zij danste voor zijn,.}{ogen hij kon haar niet recht in}{het vizier krijgen}\\

\haiku{Je doet me denken.}{aan een handelsreiziger}{die ik gekend heb}\\

\haiku{De jongen zakte.}{langzaam en statig terug}{op de achterbank}\\

\haiku{- Dat vind ik erg flink,.}{antwoordde Ellen weer met}{haar gewone stem}\\

\haiku{Terwijl de vingers,.}{hun lenig werk vervolgden}{wachtte Paul beleefd}\\

\haiku{- Ze staan allemaal,.}{tegen mij aan te zij ken}{fluisterde Asquit}\\

\haiku{Al te lang heeft de.}{ongare verveling u}{beziggehouden}\\

\haiku{Kunnen jullie door?}{de stad lopen zonder de}{beest uit te hangen}\\

\haiku{Een zwoele windvlaag.}{komt de kamer binnen en}{slaat op je darmen}\\

\haiku{Hij werd wakker, zo.}{abrupt dat hij nog even in}{zijn droom bleef steken}\\

\haiku{Toen plofte deze,.}{uit elkaar of eigenlijk}{was het dat niet eens}\\

\haiku{Buiten zijn kamer.}{liep iemand in pyjama}{hem voor de voeten}\\

\haiku{Links (als je van de).}{haven kwam waren hoge}{struiken en bomen}\\

\haiku{Ik moet kijken wie.}{van jullie het meest geschikt}{is voor commissies}\\

\haiku{Toen hij dit gezegd.}{had leek het alsof de lucht}{vol olifanten kwam}\\

\haiku{Ook onzin kon uit.}{zijn mond komen en niemand}{nam hem dat kwalijk}\\

\haiku{De ouvreuses in.}{het Luchtjeshuis stonden niet}{als prettig bekend}\\

\haiku{Dit dient om jullie.}{kennis te laten maken}{met de bevolking}\\

\haiku{Toen de man met de.}{mondharmonika klaar was}{werd hij toegejuicht}\\

\haiku{Maar het opletten.}{werd steeds moeilijker en er}{ging niets beginnen}\\

\haiku{- Wijs mij een vrouw, zei,. -.}{hij bijna plechtig met een}{hoge stem Ellen}\\

\haiku{Mijn oom droeg een groen.}{jachtkostuum om niet zo naast}{haar op te vallen}\\

\haiku{Daarna gaf hij ook.}{haar wat geld en tapte voor}{zichzelf een glas bier}\\

\haiku{Zo oud als ze wordt,,,;}{die lieve Spernip kippig}{een beetje aftands}\\

\haiku{Dit was verveling,.}{over mensen horen die in}{jouw hoofd leeg blijven}\\

\haiku{informeerde de.}{dichter met een overdreven}{vraagtoon in zijn stem}\\

\haiku{Het verwonderde.}{hem dat het onweer nog niet}{was losgebroken}\\

\haiku{- Als jij het eens wel,,.}{dacht zei Asquit steun zoekend}{tegen de deurlijst}\\

\haiku{En dan blijven ze,,.}{maar hier als opvrolijking}{desnoods met geweld}\\

\haiku{Een windvlaag sloeg in,.}{zijn gezicht tegelijk met}{wat regendruppels}\\

\haiku{De dakgoot liep over,.}{zodat een guts regen zijn}{kraag binnen plensde}\\

\haiku{Zijn hoofd kwam met een.}{klap op de grond en leek zelfs}{even te stuiteren}\\

\haiku{Zij nam zijn hand en.}{nam hem mee door het natte}{gras naar de rotsrand}\\

\haiku{Zijn voeten sopten.}{door de halmen die al lang}{niet gemaaid waren}\\

\haiku{Hij legde zijn hoofd.}{tegen haar schouder en liet}{zijn ogen dichtvallen}\\

\haiku{Hij zat op zijn bed.}{terwijl hij nauwelijks wist}{dat hij wakker was}\\

\haiku{Mensen mogen niet.}{teleurgesteld worden in}{hun aanbiedingen}\\

\haiku{Zoals Asquit had.}{geweten dat hij Senkar}{niets moest wij smaken}\\

\haiku{Ze zagen er niet.}{uit als soldaten die naar}{iemand op zoek zijn}\\

\haiku{- Dat zou ik niet zo,.}{prettig vinden hier in huis}{Jimmy zei Alissa}\\

\haiku{Ik geloof dat men.}{zich met zijn eigen woorden}{het beste uitdrukt}\\

\haiku{- Vanmorgen nog heb,.}{ik Gossep gesproken zei}{hij geheimzinnig}\\

\haiku{- We praten er niet,,.}{over zei hij voor hij het ding}{op zijn neus zette}\\

\haiku{Wij zullen trachten.}{u zo spoedig mogelijk}{te evacueren}\\

\haiku{Godverdomme, ik! -,.}{stik hier Luister zei Senkar}{een beetje vermoeid}\\

\haiku{- Kom nou verdomme,,.}{zei hij met zijn linkervoet}{op de grond stampend}\\

\haiku{Maar als jullie je.}{bek opendoen flikker ik je}{liever de straat op}\\

\haiku{Mary als je wist.}{wat de directeur op de}{wereld te koop weet}\\

\haiku{Hun voeten hadden.}{stof achtergelaten op}{de glanzende grond}\\

\haiku{De directeur ging,.}{achter een bureau zitten}{zijn rug naar de zee}\\

\haiku{Wij zouden het niet.}{tolereren wanneer hier}{een vijand vertrok}\\

\haiku{Zij renden naar zee.}{en verscholen zich bij het}{strand in struikgewas}\\

\haiku{Zij klommen tegen,.}{de heuvel op af en toe}{steentjes losschoppend}\\

\haiku{Elke meter die.}{hij nu aflegde zou hem}{minder nat maken}\\

\haiku{De bomen lagen,,.}{evenals bij de uitspanning}{te midden van gras}\\

\haiku{De drank maakte hem,,.}{warm overal scheen nu de zon}{behalve buiten}\\

\haiku{- Je hebt zes weken,.}{in mijn spoor gelopen zei}{Henri veel zachter}\\

\haiku{De deur tussen de.}{fabricagehal en de}{loods werd geopend}\\

\haiku{- Je hebt gelijk, zei,.}{hij kwaadaardig waarna hij}{op de grond spuugde}\\

\haiku{Het leek of alle.}{wolken zich in deze kring}{hadden verzameld}\\

\haiku{Het was duidelijk.}{dat hij de modder op zijn}{knie\"en niet voelde}\\

\haiku{Hij had de fles die.}{nog niemand had opgevraagd}{aan de mond en dronk}\\

\haiku{uit ieder van de.}{vletten kwamen twee witte}{mannen zonder helm}\\

\haiku{Op het ogenblik dat.}{Asquit wegvoer kon Paul zich}{juist aan boord hijsen}\\

\haiku{Zij liepen langzaam,.}{naar Asquit toe die op het}{dek was gaan zitten}\\

\haiku{Het water klotste.}{in kalme deining tegen}{de wand van het schip}\\

\haiku{- Dagen als deze,,.}{zei Gossep moeizaam terwijl}{ze Henri streelde}\\

\subsection{Uit: Al is de waarheid nog zo snel}

\haiku{Je hebt een beeld op.}{hem overgebracht waaraan hij}{nog wat houvast heeft}\\

\haiku{Ook vroeger deden.}{mensen op deze manier}{aan zelfbescherming}\\

\haiku{Hera, de vrouw van,.}{Zeus overigens beloofde}{hem macht en rijkdom}\\

\haiku{omdat mensen die {\textquoteleft}{\textquoteright}.}{doeg zeggen iets persoonlijks}{op het oog hebben}\\

\haiku{Mijn irritatie.}{wordt dus op verschillende}{manieren gewekt}\\

\haiku{Boontje vindt dit zo - -.}{grappig dat hij letterlijk}{barst van het lachen}\\

\haiku{het gaat hier niet om.}{een beloning maar om een}{welverdiende straf}\\

\haiku{En in Antwerpen {\textquoteleft}{\textquoteright}.}{heeft men het over iemandin}{het o'ken trekken}\\

\haiku{De zegswijze {\textquoteleft}een{\textquoteright}:}{nieuwsgierig Aagje zijn heeft}{vaak als toevoeging}\\

\haiku{Kostbare karaf,.}{onherstelbaar beschadigd}{amsterdam 27 jan}\\

\haiku{ze zijn dom, koppig,,,:}{gedoemd tot slavernij en}{wat het ergste is}\\

\haiku{Bovendien heb ik.}{geleerd dingen die ik niet}{begrijp te vragen}\\

\haiku{Ik sla een tijdschrift:}{open en meteen stuit ik op}{de brandende vraag}\\

\haiku{Aan haar voeten heerst,.}{een jong gevoel door die nieuw}{elastische kousen}\\

\haiku{En daar zat ik dan,.}{met die stijve trut en dat}{glas goedkope wijn}\\

\haiku{Maar geestelijke.}{pijn concentreert zich niet op}{een bepaalde plaats}\\

\subsection{Uit: De ambitie}

\haiku{Opa, vechtend tegen,.}{de slaap want hij was tien jaar}{ouder dan zijn vrouw}\\

\haiku{Die laatste zin had.}{hij voldoende op zijn zoon}{kunnen oefenen}\\

\haiku{Haar vader kwam zo.}{weinig mogelijk in het}{huis van zijn ouders}\\

\haiku{Ze waren bang dat.}{ik ophield met schoonmaken}{van hun zwijnestal}\\

\haiku{Ze hield altijd dat.}{rotsvaste vertrouwen in}{de goede afloop}\\

\haiku{Hij was er altijd,.}{vaag over geweest vond dat het}{een grote kans was}\\

\haiku{Maar toen ze het parkje,.}{binnenkwam wist zij al dat}{zij zich vergist had}\\

\haiku{Zij dacht aan Frits, die.}{met gesloten gordijnen}{bij de open haard lag}\\

\haiku{Daarna zou hij haar.}{in zijn armen nemen en}{met haar naar bed gaan}\\

\haiku{{\textquoteleft}Er kunnen nu toch,{\textquoteright}.}{geen boodschappen meer worden}{gedaan zei Nila}\\

\haiku{De discussie die.}{nu zou beginnen had zij}{al zo vaak gevoerd}\\

\haiku{Opnieuw was het een.}{onaangenaam idee alleen}{te moeten slapen}\\

\haiku{Ik wil niet langer.}{dan twee weken iets met hem}{te maken hebben}\\

\haiku{{\textquoteright} {\textquoteleft}Die heeft de MULO.}{niet eens zien staan op weg naar}{het gymnasium}\\

\haiku{Maar ik lag daar met.}{die  kinderen en zag}{ze de hele dag}\\

\haiku{Nooit had zij ook maar.}{enige preutsheid gekend ten}{opzichte van Frits}\\

\haiku{{\textquoteleft}Jullie kunnen het.}{beste met z'n twee\"en in}{het zwembad springen}\\

\haiku{{\textquoteright} Nila voelde een.}{spanning in haar buik die zij}{niet kon verdrijven}\\

\haiku{Hij sloeg zijn arm om.}{haar schouders en drukte haar}{tegen zich aan}\\

\haiku{Voor iemand die pas,.}{een dag in Spanje was zag}{hij bijzonder bruin}\\

\haiku{Wie garandeerde.}{haar dat Ronnie alleen op}{ontspanning uit was}\\

\haiku{Zijn werkelijke.}{bestaan speelde zich op een}{ander niveau af}\\

\haiku{Neuri\"end begon.}{zij aan het karwei waaraan}{zij een hekel had}\\

\haiku{Het kon niet anders.}{of dit was een signaal van}{hun generatie}\\

\haiku{Wie mis jij op dit,?}{ogenblik het meeste je man}{of je kinderen}\\

\haiku{Bij jou, wat zal ik,.}{zeggen klonk het helemaal}{niet automatisch}\\

\haiku{Ze trok de lipjes.}{uit de ijskoude blikjes}{en keerde zich om}\\

\haiku{{\textquoteleft}Het was bijna nog,{\textquoteright}.}{heel leuk geworden zei Henk}{op de achtergrond}\\

\haiku{Hij aarzelde, het.}{lege bierblikje tussen}{duim en wijsvinger}\\

\haiku{Je hebt er net zo.}{goed recht op te weten wat}{er gebeurt als ik}\\

\haiku{Zij bewoog snel heen.}{en weer tussen weerloosheid}{en irritatie}\\

\haiku{Alles kon nu met.}{haar gebeuren zonder dat}{zij eronder leed}\\

\haiku{Ben jij zo iemand.}{die het best vindt dat ze haar}{man een lui noemen}\\

\haiku{Probeerde iets te.}{zeggen in het besef dat}{het niet lukken zou}\\

\haiku{je alweer ergens.}{geweest op het moment dat}{je arriveerde}\\

\haiku{Dat was haar eigen.}{vraag die nooit door een ander}{aan haar was gesteld}\\

\haiku{Hij lag tegen haar.}{aan en warmde haar als zij}{het koud had in bed}\\

\haiku{Een noviteit van,.}{toen inmiddels afgeboekt}{van de rekening}\\

\haiku{Gaf haar nauwelijks.}{de kans die wens ook van haar}{kant uit te spreken}\\

\haiku{Hun gezamenlijk.}{leven bezat al te veel}{vanzelfsprekendheid}\\

\haiku{En toen stonden ze.}{tegelijk stil en keerden}{zich in haar richting}\\

\haiku{Daarachter was het,.}{land misschien zat hij tussen}{de bomen gehurkt}\\

\haiku{Een ogenblik later.}{wilde zij niet denken over}{verwarring of rust}\\

\haiku{Ze wilde dat hij,.}{naar haar toekwam wegging en}{weer terugkeerde}\\

\haiku{Toen Frank drie maanden,.}{oud was wilde zij op een}{morgen niet opstaan}\\

\haiku{Zij zat naast haar kind.}{en zag hem wakker worden}{en haar herkennen}\\

\haiku{Op een dag kwam haar.}{vader uit de inrichting}{waar oma verpleegd werd}\\

\haiku{Als ze een van haar,.}{buien had bracht ze niet veel}{meer dan gepiep voort}\\

\haiku{{\textquoteright} {\textquoteleft}Misschien omdat het,{\textquoteright}.}{verder altijd zo stil was}{antwoordde Nila}\\

\haiku{Aan de andere.}{kant irriteerde haar dit}{gebrek aan aandacht}\\

\haiku{Zij wilde haar hoofd.}{in Ronnie's schoot duwen en}{zo blijven liggen}\\

\haiku{Die herinnering.}{oproepen zonder de pijn}{ervan te voelen}\\

\haiku{De eerste keer dat,{\textquoteright}.}{je me erover vraagt na je}{vakantie zei Frits}\\

\haiku{Als je moet zeggen,}{wat je ervan vindt weet je}{niet welke woorden}\\

\haiku{Zij wist dat als Frits,.}{naar haar toekwam zij hem niet}{zou tegenhouden}\\

\haiku{Maar het was beter.}{voor haar dat ze hen rustig}{in Groningen liet}\\

\haiku{Je kon geen bezoek.}{verwachten of je maakte}{al ruzie met opa}\\

\haiku{Onsterfelijkheid.}{najagen door joden in}{elkaar te trappen}\\

\haiku{Je verdedigt hem,}{omdat je de rechtsorde}{wilt laten bestaan}\\

\haiku{Zij dacht erover na.}{hoe het zou zijn als Frits in}{elkaar zou klappen}\\

\haiku{Zij keerde terug.}{naar de slaapkamer en trok}{haar klerenkast open}\\

\haiku{Zij zag wel dat haar.}{omgeving pogingen deed}{haar te isoleren}\\

\haiku{Vroeg zich af of zij.}{niet beter de eerste trein}{terug kon nemen}\\

\haiku{Oog in oog met zijn.}{beul zou hij de pijn opnieuw}{voelen aanzwellen}\\

\haiku{Ze ging naar de stad.}{om haar eigen leven weer}{ter hand te nemen}\\

\haiku{wel een verzorgster,.}{maar bij nader inzien een}{gehalveerd gezin}\\

\haiku{Zij kleedde zich uit.}{met de bedoeling onder}{de douche te gaan}\\

\haiku{Terugkeren naar.}{het donzen gebied tussen}{slapen en waken}\\

\haiku{Zij was er zeker.}{van dat de fluitketel al}{op het gas stond}\\

\haiku{{\textquoteright} {\textquoteleft}Het is een kwestie,{\textquoteright}.}{van verantwoordelijkheid}{zei haar schoonmoeder}\\

\haiku{Zij moest niet te snel.}{toegeven aan eisen die}{de ander stelde}\\

\haiku{De ober keek haar aan,.}{schatte haar op de vrouw die}{vergeefs had verwacht}\\

\haiku{Ik sta hier maar in,.}{mijzelf te schreeuwen zonder}{dat iemand het merkt}\\

\haiku{En zo haalde zij.}{vakantiereizen in haar}{geheugen terug}\\

\haiku{Als je daarin klom,.}{en over de rand keek was je}{terug in de tijd}\\

\haiku{Ze kon zich alleen.}{niet herinneren hoe ze}{er gekomen was}\\

\haiku{Haar schoonmoeder kwam.}{de kamer binnen en liep}{naar de boekenkast}\\

\haiku{Niet naar school, niet erg,.}{ziek te weinig koorts om de}{dokter te roepen}\\

\haiku{Voor het eerst had hij.}{volkomen genoeg van de}{dingen die hij deed}\\

\haiku{Ze probeerde een.}{opgewekte toon in haar}{woorden te leggen}\\

\haiku{En daarvan weet ik.}{tenminste zeker dat het}{werkelijkheid is}\\

\haiku{Aan de andere.}{kant van de lijn begon een}{bel te rinkelen}\\

\haiku{Ze herinnerde,.}{zich geen woorden lieve noch}{afstandelijke}\\

\haiku{Haar dagboek was ze.}{begonnen met gedachten}{over haar grootmoeder}\\

\haiku{Die lui hebben door.}{de oorlog een soort mystiek}{over zich gekregen}\\

\haiku{Ik denk dat Van Raay.}{Melgers hoogst persoonlijk in}{zijn cel zou wurgen}\\

\haiku{Na afloop begon,.}{zij te huilen ontspannen}{en tevreden}\\

\haiku{Ze voelde zich niet,.}{overdreven zeker maar toch}{een stuk rustiger}\\

\haiku{Ze vroeg het hem en.}{hij antwoordde dat hij niet}{aan haar twijfelde}\\

\haiku{Ze stelde zich voor.}{hoe ze hier naakt met Ronnie}{door de keuken liep}\\

\haiku{En nu, dacht ze, heb}{ik een vriendin die op de}{wc zit te huilen}\\

\haiku{En hij zei dat Frits.}{misschien zijn best niet deed bij}{de verdediging}\\

\haiku{Ik vind dat je er,{\textquoteright}.}{te sjieke opvattingen}{op nahoudt zei Henk}\\

\haiku{We worden alleen.}{maar ongelukkig als we}{onszelf beheersen}\\

\haiku{Misschien heb jij wel.}{ontzettend veel zin om met}{Ronnie te neuken}\\

\haiku{Ze stak haar hand uit{\textquoteright}.}{en legde die een ogenblik}{op Frits onderarm}\\

\haiku{{\textquoteleft}Soms droom ik ervan.}{dat ik met een lief meisje}{in een hutje zit}\\

\haiku{Ik denk dat het kind.}{van het dek af zou waaien}{als ik haar meenam}\\

\haiku{Frits: ik heb het idee.}{dat mijn leven er anders}{door geworden is}\\

\haiku{Maar misschien zijn mijn.}{eigen eisen aan mensen}{wel te redelijk}\\

\haiku{Want ze voelde zich.}{ook prettig en vertrouwd in}{Ronnie's nabijheid}\\

\haiku{Ze liet toe dat hij.}{zijn hand op haar borst legde}{en haar betastte}\\

\haiku{{\textquoteleft}Ik kan begrijpen,}{dat je met een andere}{kerel iets begint}\\

\haiku{bij het vorderen.}{van de dag voelde zij de}{spanning toenemen}\\

\haiku{{\textquoteleft}Je maakt het leven,{\textquoteright}.}{zo beperkt met die nadruk}{op seks zei Ronnie}\\

\haiku{Alledrie hadden.}{ze hun eigen opvatting}{over het gebeuren}\\

\haiku{Maar ook dat had zich.}{misschien te gemakkelijk}{in haar vastgezet}\\

\haiku{Tussen Haarlem en.}{Den Haag hadden zij elkaar}{langdurig gekust}\\

\haiku{Sta ik eigenlijk?}{buiten de beslissing wat}{ik zal opschrijven}\\

\haiku{Geen lekke banden,.}{onweersbuien die ons koud}{en treurig maakten}\\

\haiku{Het ergste is dat,.}{je je niet voor kunt stellen}{wat het is dood zijn}\\

\haiku{En dan kan ik het,.}{nog even uitstellen omdat}{jullie bij me zijn}\\

\haiku{Mijn grootmoeder was,.}{een vrolijke en gastvrije}{vrouw voor de oorlog}\\

\haiku{{\textquoteright} Ronnie stopte het,.}{spiegeltje weg stond op en}{kwam naar Nila toe}\\

\haiku{Op de wastafel.}{lagen twee plastic builtjes}{gevuld met badschuim}\\

\haiku{Ze begon, op een,.}{uitgestelde manier haar}{jeugd te beleven}\\

\haiku{wat doen wij elkaar,.}{aan wat voor verwachtingen}{wek ik bij Ronnie}\\

\haiku{Het gaat er niet om,.}{dat hij dat doet maar dat hij}{mij vergeten is}\\

\haiku{Als ik een beetje.}{ziek ben en me voorstel dat}{het nooit meer overgaat}\\

\haiku{Sommigen van hen.}{hadden het kijken van de}{mensen veranderd}\\

\haiku{Maar toen de vriendin,:}{als een geslagen hond bleef}{staan zei de dame}\\

\haiku{ik praat er niet meer.}{over maar ik heb ook wel eens}{zin in wat anders}\\

\haiku{{\textquoteleft}En om eerlijk te.}{zijn weet ik niet waarom ik}{daar nu over begin}\\

\haiku{Misschien zat ze op.}{het Gare du Nord en kon}{geen trein meer krijgen}\\

\haiku{Ronnie's kennis van.}{de wereld was gelardeerd}{met kleurenfoto's}\\

\haiku{De opwindende.}{eenzaamheid wanneer zij uit}{het zolderraam keek}\\

\haiku{Maar de aankomst van.}{haar vader zou nog uren op}{zich laten wachten}\\

\haiku{Alles wat zij had.}{opgeschreven was nog maar}{een voorbereiding}\\

\haiku{En zij zou de tuin,,.}{betreden met de hoge}{donkere bomen}\\

\haiku{Hij stond op en liep,.}{naar een kastje waarvan hij}{het deurtje opende}\\

\haiku{{\textquoteright} {\textquoteleft}De slimste zul je,{\textquoteright},.}{nooit worden zei oma met iets}{spottends in haar stem}\\

\haiku{H\`e moeder je doet,{\textquoteright}.}{helemaal niet gezellig}{bouwde oma hem na}\\

\haiku{En toen ik je in,.}{mijn armen hield was ik zo}{verschrikkelijk blij}\\

\haiku{Maar ik vraag me af.}{of je wel weet waarom je}{niet meer terugwilt}\\

\haiku{En ik wist dat het,.}{niet alleen om jou was maar}{ook om je moeder}\\

\subsection{Uit: Bitter \& zoet}

\haiku{{\textquoteleft}Wel aardig, maar je.}{kon wel meteen zien dat ie}{alles beter wist}\\

\haiku{Henrietta draaide.}{zich om en legde haar kin}{op de rugleuning}\\

\haiku{Daar zat mijn moeder.}{op een keurige manier}{taart te lepelen}\\

\haiku{Ze praatte dan zacht.}{op hem in zonder dat ik}{verstond wat ze zei}\\

\haiku{Ik draaide mij een,.}{kwartslag naar Henrietta die}{strak voor zich uitkeek}\\

\haiku{{\textquoteright} {\textquoteleft}Ik zal laten zien{\textquoteright},.}{dat er hier tenminste \'e\'en}{fatsoen heeft zei pa}\\

\haiku{Weet je dat we hier?}{al kwamen toen we nog niet}{eens getrouwd waren}\\

\haiku{Toen ik terugkwam.}{zat mijn vader in een stoel}{een boek te lezen}\\

\haiku{Ik voelde me zo,.}{opgewonden jongen en}{ik was ook doodsbang}\\

\haiku{En dan zullen we.}{naar haar toegaan en zien dat}{het een ander is}\\

\haiku{Dat met je moeder{\textquoteright},.}{dat was ook vlak voor Pasen}{zei hij bedachtzaam}\\

\haiku{Toen iedereen me,,:}{toesprak op die receptie}{nou ja toen dacht ik}\\

\haiku{Hij wilde weten.}{hoe ze had gereageerd}{op zijn verhuizing}\\

\haiku{Zo reden wij met.}{ons drie\"en uren lang door de}{uitgestorven stad}\\

\haiku{Ze had altijd zo.}{de pest aan kots ruimen en}{dan ging ze maar weg}\\

\haiku{Dacht hij werkelijk?}{dat door zijn vertrek alles}{anders werd voor hem}\\

\haiku{Mijn vader stuurde,.}{de eerste fles terug met}{een knipoog naar mij}\\

\haiku{Het publiceren.}{van een artikel werd een}{obsessie voor hem}\\

\haiku{Maar eenmaal terug.}{in het mulle zand ploegden}{wij voort als vanouds}\\

\haiku{Hij zei dat als hij,.}{zijn hoofd schuin rechtop hield het}{draaien wel meeviel}\\

\haiku{En ik raakte in,.}{verwarring hoewel ze}{niet mijn hulp inriep}\\

\haiku{In werkelijkheid.}{weet zij zich misschien geen raad}{van de zenuwen}\\

\haiku{Zij tokkelde op,.}{een gitaar die een ander}{voor haar bespeelde}\\

\haiku{Zij had een grote.}{waardering voor mensen die}{iets bereikt hadden}\\

\haiku{De journalist bracht.}{haar om een uur of twee in}{zijn auto naar huis}\\

\haiku{{\textquoteleft}Ze haalde veertien.}{dagen diep adem om daarna}{de sprong te wagen}\\

\haiku{Heeft het publiek niet?}{ademloos de hele avond op}{hem zitten wachten}\\

\haiku{Die kieren zijn in.}{dit geval aangebracht in}{de werkelijkheid}\\

\haiku{Als hij me niet goed{\textquoteright},, {\textquoteleft}.}{vond zei zedan had hij me}{niet meer genomen}\\

\haiku{De filmpers had haar.}{drie keer zo luid geprezen}{als de regisseur}\\

\haiku{Ik verzekerde.}{haar dat zijzelf inderdaad}{heel goed meekon}\\

\haiku{Onder die laatsten,.}{bevond Mona zich en wel}{op de eerste rij}\\

\haiku{Toch maakte zij zich.}{al zorgen over het moment}{van uiteenspatten}\\

\haiku{Er bestonden nog.}{zoveel andere levens}{dan dat van haar}\\

\haiku{Tussen alles wat,.}{verging stonden bomen die}{ieder jaar bloeiden}\\

\haiku{Voor zijn derde film:}{kwam Preston met een geheel}{ander scenario}\\

\haiku{'s Avonds stond zij in,.}{de krant temidden van een}{aantal collega's}\\

\haiku{Een paar uur later.}{klopte ik op de deur van}{haar hotelkamer}\\

\haiku{uit de nevel komt.}{de schim van iemand uit een}{andere wereld}\\

\haiku{of ik van haar houd{\textquoteright},.}{zegt Mona Robbins in haar}{eerste artikel}\\

\subsection{Uit: Bloedmadonna}

\haiku{Zij wilde niet bij.}{de vrouwenpagina en}{nog minder bij sport}\\

\haiku{{\textquoteright} Hanna prikte een.}{stronkje bloemkool aan haar vork}{en hield het omhoog}\\

\haiku{Tegenover Smeets had.}{zij gehandeld volgens een}{vooropgezet plan}\\

\haiku{Vanaf 's morgens.}{negen uur had zij op het}{archief gezeten}\\

\haiku{Als goed marxist had.}{hij in gedachten zijn hand}{eraf gebeten}\\

\haiku{De zon zakte op;}{een gegeven moment weg}{achter de heuvels}\\

\haiku{een jaar geleden.}{had hij gedroomd dat zij naakt}{in zijn kamer stond}\\

\haiku{Haar vader wist niet,.}{hoe hij ze moest weigeren}{of tegenhouden}\\

\haiku{Hoe kwam de natuur?}{erbij zoveel ellende}{te verdubbelen}\\

\haiku{{\textquoteleft}Het zijn kwetsbare.}{en tere zielen en die}{heb ik nooit gehaat}\\

\haiku{{\textquoteright} {\textquoteleft}Het is zondag,{\textquoteright} zei,.}{Hanna en opnieuw begon}{zij te schateren}\\

\haiku{Van de huilende.}{madonna raakte hij niet}{onder de indruk}\\

\haiku{Eigenlijk vond hij,.}{het een smerig gezicht dat}{uitgelopen oog}\\

\haiku{liet, mede gezien.}{het feit dat zijn uitleg nog}{steeds over Agnes ging}\\

\haiku{{\textquoteright} {\textquoteleft}Dat die kerels naar.}{haar kijken met zo'n blik waar}{een vrouw eng van wordt}\\

\haiku{Bloedde zij voor hem,?}{om hem aan zijn gedachten}{te herinneren}\\

\haiku{Bij het openen van.}{de deur ving Hanna een glimp}{op van het beeldje}\\

\haiku{{\textquoteright} {\textquoteleft}Betekent dit dat?}{u de mogelijkheid van}{een wonder openhoudt}\\

\haiku{Ze bestonden uit.}{kalk en daaroverheen zat een}{laagje porselein}\\

\haiku{Nog even en zij had;}{alleen nog herdenkingen}{om over te schrijven}\\

\haiku{in haar hoofd begon,;}{iets te kraken de wereld}{was vol wonderen}\\

\haiku{Dit zinnetje sloeg.}{totaal niet op het stuk dat}{Hanna moest schrijven}\\

\haiku{Van een beeldje kun.}{je niet verwachten dat het}{maandverband gebruikt}\\

\haiku{En voor je het wist {\textquoteleft}{\textquoteright}.}{vervingen zeonderbuik}{nog door iets ergers}\\

\haiku{Deze gedachten,.}{die haar invielen en haar}{dagen verpestten}\\

\haiku{Op dit uur van de;}{avond was de snelweg niet meer}{bezaaid met auto's}\\

\haiku{Nuchelmans was niet.}{over de hoofdweg naar Uffel}{komen aanrijden}\\

\haiku{Waarom verwees hij?}{zo gemakkelijk naar God}{als het moeilijk was}\\

\haiku{Hij had het niet slecht,.}{gedaan was geliefd bij de}{parochianen}\\

\haiku{Zij stapte uit het.}{karretje en voelde de}{hand van haar moeder}\\

\haiku{Het bericht van de.}{moord had zich vanzelfsprekend}{razendsnel verspreid}\\

\haiku{De arts stelde het.}{tijdstip van overlijden op}{omstreeks middernacht}\\

\haiku{{\textquoteright} Streng keek hij Thieu aan,.}{maar deze voelde slechts een}{lachbui opkomen}\\

\haiku{Een ander idee was;}{de constructie van beeldjes}{met een reservoir}\\

\haiku{konden de mensen.}{de madonna hun eigen}{bloed laten huilen}\\

\haiku{Voorzichtig had hij.}{de vorige dag wat van}{het bloed afgekrabd}\\

\haiku{{\textquoteleft}Dus u gelooft ook?}{in de waarachtigheid van}{onze madonna}\\

\haiku{Hij trok een witte.}{zakdoek uit zijn broekzak en}{bette zijn voorhoofd}\\

\haiku{Ik wil dat u de.}{vastberadenheid laat zien}{van de moederkerk}\\

\haiku{Franske stond erbij,,.}{wees op haar riep luidkeels dat}{hij ging vliegeren}\\

\haiku{Luister eens,{\textquoteright} zei hij,.}{bijna fluisterend zich naar}{Hanna toebuigend}\\

\haiku{Op dezelfde toon.}{had zij kunnen zeggen dat}{ze er niets aan vond}\\

\haiku{Dan noem ik jou de.}{mooie jonge journaliste}{van het ochtendblad}\\

\haiku{Zij keerde juist naar}{mensen terug omdat zij}{wilde uitvinden}\\

\haiku{Een ander tochtje.}{had hen gebracht naar de kust}{in Noord-Frankrijk}\\

\haiku{Gemakkelijker.}{maakten zijn verdiensten het}{er voor hem niet op}\\

\haiku{Hij deed zijn best zijn.}{stem zo ironisch mogelijk}{te laten klinken}\\

\haiku{Zij bereikte het.}{doel na enige navraag en}{een kleine omweg}\\

\haiku{Van het eerste had;}{de burgemeestersvrouw zelf}{meer dan voldoende}\\

\haiku{Zij hep ernaartoe.}{en spiegelde zich in het}{heldere water}\\

\haiku{{\textquoteleft}E\'en,{\textquoteright} antwoordde zij,.}{zonder iets over de engel}{te durven zeggen}\\

\haiku{{\textquoteleft}Ik heb zojuist een,{\textquoteright}.}{theorie ontwikkeld zei}{Dantzig bedachtzaam}\\

\haiku{Erg ongesteld is,,.}{ze niet dacht Hanna met een}{lichte ergernis}\\

\haiku{{\textquoteleft}Wij zorgen er wel.}{voor dat ons madonneke}{een beetje rust krijgt}\\

\haiku{Een enkele keer.}{geeft een van die maaksels een}{zekere vreugde}\\

\haiku{Eerst bladerde zij.}{die even onverschillig door}{als de andere}\\

\haiku{In dat geval zou;}{ook het bloed van zijn dochter}{weer op gang komen}\\

\haiku{Franske maakte zijn.}{broek open en piste in de}{bloemen naast de kerk}\\

\haiku{Zij was in paniek,.}{geraakt zonder dat zij iets}{kon ondernemen}\\

\haiku{Voor de belangen.}{van de hemel wilden zij}{wel harder rennen}\\

\haiku{Of begon het te,?}{vervagen als een te kort}{gefixeerde foto}\\

\haiku{Als Zij Thieu in zijn.}{blote kont op een vrouw van}{lichte zeden zag}\\

\haiku{Met stijve knie\"en.}{stond hij op en begon op}{en neer te springen}\\

\haiku{Misschien is het zelfs.}{goed dat wij eerst als mannen}{onder elkaar zijn}\\

\haiku{{\textquoteleft}Ik zie voldoende.}{kansen om iets voor je zoon}{te betekenen}\\

\haiku{Hij omarmde de.}{jongen en zag voor zich hoe}{hij hem zou zoenen}\\

\haiku{Voor haar stond een glas,?}{oranje limonade was}{het prik of ranja}\\

\haiku{En in die wereld.}{bestond slechts een schromelijk}{gebrek aan kennis}\\

\haiku{Als het aan hen ligt.}{zullen al je plannetjes}{in duigen vallen}\\

\haiku{{\textquoteleft}Wat trekt volgens jou,{\textquoteright}.}{meer en vooral langer de}{aandacht vroeg Dantzig}\\

\haiku{Dit was nooit gebeurd,.}{die alles verterende}{woede bestond niet}\\

\haiku{die handen drukte,.}{hij tegen zijn oren tot ze}{barstten van de pijn}\\

\haiku{{\textquoteleft}Kalm, kalm toch,{\textquoteright} riep Rog,.}{hij legde een hand op het}{hoofd van het meisje}\\

\haiku{Zij glimlachten naar.}{elkaar en Wolffers zei iets}{dat zij niet verstond}\\

\haiku{Naar de opera van,.}{Veneti\"e nadat die in}{brand was gestoken}\\

\haiku{In een impuls deed.}{zij het gordijntje open en}{stapte naar binnen}\\

\haiku{{\textquoteleft}Ik werd al bang dat.}{ze je op een vuilnisbelt}{zouden afzetten}\\

\haiku{Zoek me en je zult.}{me tegenkomen als de}{tijd er rijp voor is}\\

\haiku{Zij zocht de kast na.}{en belichtte de muren}{met een zaklantaarn}\\

\subsection{Uit: Een dichteres uit Los Angeles}

\haiku{Daar stond de jonge.}{schilder in de aankomsthal}{van  een vliegveld}\\

\haiku{Henrietta draaide.}{zich om en legde haar kin}{op de rugleuning}\\

\haiku{Daar zat mijn moeder.}{op een keurige manier}{taart te lepelen}\\

\haiku{Ze praatte dan zacht.}{op hem in zonder dat ik}{verstond wat ze zei}\\

\haiku{Ik draaide mij een,.}{kwartslag naar Henrietta die}{strak voor zich uitkeek}\\

\haiku{{\textquoteright} {\textquoteleft}Ik zal laten zien,{\textquoteright}.}{dat er hier tenminste \'e\'en}{fatsoen heeft zei pa}\\

\haiku{Weet je dat we hier?}{al kwamen toen we nog niet}{eens getrouwd waren}\\

\haiku{Toen ik terugkwam.}{zat mijn vader in een stoel}{een boek te lezen}\\

\haiku{Ik voelde me zo,,.}{opgewonden jongen en}{ik was ook doodsbang}\\

\haiku{En dan zullen we.}{naar haar toe gaan en zien dat}{het een ander is}\\

\haiku{Toen iedereen me,,:}{toesprak op die receptie}{nou ja toen dacht ik}\\

\haiku{Hij wilde weten.}{hoe ze had gereageerd}{op zijn verhuizing}\\

\haiku{Zo reden wij met.}{ons drie\"en urenlang door de}{uitgestorven stad}\\

\haiku{Zij had altijd zo.}{de pest aan kots ruimen en}{dan ging ze maar weg}\\

\haiku{Dacht hij werkelijk?}{dat door zijn vertrek alles}{anders werd voor hem}\\

\haiku{Mijn vader stuurde,.}{de eerste fles terug met}{een knipoog naar mij}\\

\haiku{Het publiceren.}{van een artikel werd een}{obsessie voor hem}\\

\haiku{Maar eenmaal terug.}{in het mulle zand ploegden}{wij voort als vanouds}\\

\haiku{Hij zei dat als hij,.}{zijn hoofd schuin rechtop hield het}{draaien wel meeviel}\\

\haiku{sprekender dan de,.}{werkelijkheid hoewel dat}{bijna niet meer kon}\\

\haiku{Wuivende halmen.}{en boeren met de rode}{zakdoek om de hals}\\

\haiku{{\textquoteright} Nadat ik van de,.}{eerste schrik was bekomen}{voelde ik geen angst}\\

\haiku{Ik begreep dat ik.}{eenzaam was geweest toen ik}{als student aankwam}\\

\haiku{Maar de teksten die.}{de stilte van mij hoorde}{waren heel anders}\\

\haiku{Ik ging een dagje.}{vissen met een oom die in}{de Wormer woonde}\\

\haiku{{\textquoteleft}Wat zou een beschaafd?}{mens niet van zo'n schitterend}{pand kunnen maken}\\

\haiku{Toen hij na een uur,.}{of twee terugkwam verviel}{hij in stilzwijgen}\\

\haiku{Stel dat het echt een,,?}{incident was Hugo waar}{moest ik dan naar toe}\\

\haiku{En het kon je niet.}{schelen waarom Nicolas}{zo vaak zijn mond hield}\\

\haiku{Je krijgt nog eens tbc,}{van al dat rondhangen in}{bibliotheken}\\

\haiku{Opeens verscheen een.}{interview van hem met een}{bekende dichter}\\

\haiku{Waarom had hij nooit}{doorgekregen dat zij zich}{alleen maar inhield}\\

\haiku{Ons dochtertje had.}{nooit gedacht dat vakantie}{zo vrolijk kon zijn}\\

\haiku{Ik probeerde haar,.}{voor de geest te halen maar}{dat lukte maar half}\\

\haiku{{\textquoteleft}Als je erg nodig,{\textquoteright}, {\textquoteleft}.}{moet zei het dienstmeisjedan}{mag je hierbij mij}\\

\haiku{Tegelijkertijd,.}{wilde ik ook de tuin zien}{met zijn regendak}\\

\haiku{De tafels die er.}{stonden waren beladen}{met glazen en drank}\\

\haiku{Ik probeerde iets,.}{te zeggen maar nog steeds was}{mijn stem op de loop}\\

\haiku{De tijd heeft aan de.}{herinnering het een en}{ander veranderd}\\

\haiku{Ik legde de kaart.}{weg en ging bij een glas pils}{zitten nadenken}\\

\haiku{ik ooit kinderen,.}{zou hebben maar daar zag het}{nu even niet naar uit}\\

\haiku{Hij schrok af en toe.}{op en ging een half uur lang}{zitten luisteren}\\

\haiku{Tijdens het schrijven.}{kwamen andere dingen}{in mij naar boven}\\

\haiku{Maar het is de vraag.}{of ik me daarvan op dat}{moment bewust was}\\

\haiku{Zo belandden wij,.}{op een gegeven ogenblik}{bij het hek dat openstond}\\

\haiku{Is zoiets trouwens?}{mogelijk als het om je}{eigen leven gaat}\\

\haiku{Een paar weken per.}{maand verdiende zij wat geld}{met kleine baantjes}\\

\haiku{Als zij een vriend had.}{ging zij er van uit dat hij}{betaalde voor haar}\\

\haiku{Dan had ik tijdens.}{mijn huwelijk met Anna}{heel wat meer geschmierd}\\

\haiku{Julia, Duisterhofs,.}{vrouw bezorgt het bedrijf zijn}{bittere kantjes}\\

\haiku{Het viel mij op dat.}{zij zich goed verzorgde en}{met smaak gekleed ging}\\

\haiku{Maar  ook zelf was.}{ik allesbehalve een}{feestelijk type}\\

\haiku{Wat deed zij haar best.}{om iets te maken uit hun}{modieuze droom}\\

\haiku{Alles wat ik nooit,.}{geweten had kwam nu tot}{mij in vermoedens}\\

\haiku{Toen we elkaar pas,}{kenden wilde zij soms niet}{de deur uit verzon}\\

\haiku{Af en toe was het.}{fantastisch om haar eigen}{weg te kunnen gaan}\\

\haiku{{\textquoteright} zei Anna, nadat.}{Titia haar stoel verwisseld}{had voor het toilet}\\

\haiku{Ze waren wakker.}{geworden door het geluid}{van Anna's auto}\\

\haiku{Zij streelde Anna.}{over haar borsten en begon}{opnieuw te huilen}\\

\haiku{Ik zou er door sneeuw.}{willen lopen en het vuur}{laaiend opstoken}\\

\haiku{{\textquoteright} wilde zij weten,.}{toen hij haar voorstelde hem}{te vergezellen}\\

\haiku{{\textquoteleft}Het is toch vreemd dat.}{ik overal kan voorlezen}{en niet in Holland}\\

\haiku{So beautiful,{\textquoteright} sprak,.}{zij met de klemtoon op de}{laatste lettergreep}\\

\haiku{Aan Erwins gezicht.}{viel af te lezen dat hij}{zich niet amuseerde}\\

\haiku{Toch zou hij vroeger;}{geweigerd hebben het te}{fotograferen}\\

\subsection{Uit: Dierbaar venijn}

\haiku{Bij het volgende.}{bezoek aan zijn vader is}{er iets veranderd}\\

\haiku{Enkele van die.}{C\'ezanne-pastiches}{ontroeren mij zelfs}\\

\haiku{Bracht haar voor een paar.}{dagen naar een hotel aan}{de rand van een bos}\\

\haiku{Het is allemaal,.}{erg onbelangrijk wil hij}{daarmee uitdrukken}\\

\haiku{Uw moeder ligt er,{\textquoteright},.}{vredig bij zei hij in het}{mortuarium}\\

\haiku{Mijn moeder stond in,.}{de keukendeur haar handen}{glad van het zeepsop}\\

\haiku{Dat zij elkaar ook,.}{konden afbranden kwam niet}{in mijn moeder op}\\

\haiku{Zij kende zichzelf.}{een plaats toe in die wereld}{zonder eenzaamheid}\\

\haiku{{\textquoteleft}Stel je niet aan, Paul,{\textquoteright},.}{zei ze op haar gewone}{opgewonden toon}\\

\haiku{misschien verzetten;}{de aanstaande doden zich}{nog in hun dromen}\\

\haiku{De aanleiding tot.}{het nemen ervan verstrooid}{over de aarde}\\

\haiku{Tenzij hij iemand.}{tegenkwam met wie hij goed}{kon samenwerken}\\

\haiku{Daarna begon zij.}{te begrijpen wat hij haar}{wilde vertellen}\\

\haiku{Over eigenmachtig.}{daarbij ingrijpen hadden}{zij nooit gesproken}\\

\haiku{Bij het bed staan en.}{de arts vragen of hij de}{naald naar binnen drukt}\\

\haiku{Bud Powell speelde.}{in een kelder in de rue}{de la Huchette}\\

\haiku{Ze schopten je met.}{niet minder liefde van de}{wereld dan vandaag}\\

\haiku{{\textquoteright} schreeuwde zij, en mijn,.}{vader erachteraan op}{een sukkeldrafje}\\

\haiku{Niet lang geleden.}{had hij zijn zeventigste}{verjaardag gevierd}\\

\haiku{Een wankel evenwicht,,.}{liefje in het grensgebied}{van de waardigheid}\\

\haiku{{\textquoteright} Niet lang geleden.}{heeft hij zijn zeventigste}{verjaardag gevierd}\\

\haiku{{\textquoteright} Zijn ogen richtten zich.}{op de bessestruik opzij}{van het oprijpad}\\

\haiku{zo ver het oog reikt,,.}{mijn oog dat kan waarnemen}{zo lang er licht is}\\

\haiku{Maar wie zit op een?}{nagelaten bijdrage}{van hem te wachten}\\

\haiku{{\textquoteright} {\textquoteleft}Als je het weten,{\textquoteright}, {\textquoteleft}:}{wilt zei hijwant daar zit je}{toch naar te vissen}\\

\haiku{Wie zijn zij die 's:}{morgens uit bed stappen en}{opgewekt denken}\\

\haiku{Zij komen uit de,.}{stille witte wereld waar}{geen mensen wonen}\\

\haiku{Niet anders wil ik,.}{naar mijn leven kijken naar}{dat van mijn ouders}\\

\subsection{Uit: Dode bladeren}

\haiku{Dan kom ik aan de.}{ingang van het theater}{Th\'er\`ese tegen}\\

\haiku{zo  iemand ben.}{ik op dit ogenblik voor mijn}{eigen kinderen}\\

\haiku{{\textquoteleft}En in de hemel.}{krijgen we ongetwijfeld}{andere namen}\\

\haiku{Mijn generatie.}{benaderde de ouders}{met omzichtigheid}\\

\haiku{{\textquoteright} In die tijd was de.}{wolkenkrabber het hoogste}{gebouw van de stad}\\

\haiku{{\textquoteright} Als het zo ging, zou.}{ik van zakendoen ook wel}{niet veel begrijpen}\\

\haiku{Ik keek weer voor mij.}{en zag uit mijn ooghoeken}{dat zij glimlachte}\\

\haiku{En ik bereidde.}{mij voor op een avondje vol}{misverstanden}\\

\haiku{Je klampte je in.}{je slaap op een gegeven}{ogenblik aan mij vast}\\

\haiku{Als we dit straatje.}{uit zijn komen we op het}{Place des Vosges}\\

\haiku{Hij hield veel van je,{\textquoteright}.}{moeder meende zij ook te}{moeten opmerken}\\

\haiku{Als ik dingen zeg,.}{die je vervelend vindt moet}{je me waarschuwen}\\

\haiku{Van het ene op het.}{andere moment voel ik}{me als herboren}\\

\haiku{Beverig knoopte.}{ik de bovenste knoopjes}{van haar blouse los}\\

\subsection{Uit: Een gedoodverfde winnaar}

\haiku{Ik had me nooit  .}{bekommerd om de dingen}{die voorbij waren}\\

\haiku{Vergeten ook of.}{hij ooit thuis was toen ik nog}{niet zelf kon lezen}\\

\haiku{Of was ik het in?}{de drukte van mijn eigen}{leven vergeten}\\

\haiku{{\textquoteleft}Ik zou je kunnen,{\textquoteright}.}{helpen met dat archief zei}{Hugo even later}\\

\haiku{{\textquoteright} {\textquoteleft}Het is goedkoper,{\textquoteright}.}{om te lenen zei Hugo}{bij gelegenheid}\\

\haiku{Het plaatselijke.}{elftal betrad het veld en}{een gejuich ging op}\\

\haiku{De platen lagen.}{op een keurig stapeltje}{naast de grammofoon}\\

\haiku{In het gegeven.}{geval kwam zij niet over voor}{de begrafenis}\\

\haiku{Hem vragen hoe hij.}{reageerde op dit soort}{beledigingen}\\

\haiku{Wanneer ik nu zijn.}{stem hoorde zou ik dat ook}{moeilijk verdragen}\\

\haiku{Toen hij werkte was.}{hij houtbewerker op een}{machinefabriek}\\

\haiku{Hij was vijf jaar lang,.}{het spoor bijster bij gebrek}{aan initiatief}\\

\haiku{Kitsch bootste hij even.}{gemakkelijk na als een}{landschap van Ruysdael}\\

\haiku{Voor het eerst in zijn.}{aan mij bekende leven}{slikte hij iets weg}\\

\haiku{Op zijn bekende.}{wijze kwam hij al snel tot}{de kern van de zaak}\\

\haiku{{\textquoteleft}Ik weet niet hoe ik,{\textquoteright}, {\textquoteleft}}{het moet uitdrukken schreef mijn}{vaderen dat komt}\\

\haiku{Uiteindelijk was.}{collega Furstner hem maar}{twee jaar voor geweest}\\

\haiku{[VIII] Voordat ik.}{naar Bretagne reisde had ik}{twee ontmoetingen}\\

\haiku{{\textquoteleft}Als hij maar eens echt,{\textquoteright}.}{met me wou praten deelde}{Olga woedend mee}\\

\haiku{{\textquoteleft}Ik denk erover hem,{\textquoteright}.}{jaloers te maken met een}{ander zei Olga}\\

\haiku{Hij schudde zijn hoofd.}{in het vaste voornemen}{niet kwaad te worden}\\

\haiku{Vorrink evenwel had.}{het niet over patatkramen}{maar over danszalen}\\

\haiku{Maar het verval van.}{de achtergrond waarop zij}{steunden ging verder}\\

\haiku{{\textquoteright} {\textquoteleft}Moet je nou zien wat,{\textquoteright}.}{je gedaan hebt riep Alma}{tegen haar zoontje}\\

\haiku{{\textquoteleft}Hugo heeft zo naar,{\textquoteright}.}{je uitgezien zei ze de}{avond na mijn aankomst}\\

\haiku{Ze hadden elkaar.}{al zoveel jaren ergens}{mee bezig gezien}\\

\haiku{{\textquoteleft}Je irriteert me,{\textquoteright}.}{had Hugo de avond voor mijn}{aankomst opgemerkt}\\

\haiku{Ik nam Hugo mee.}{naar de keuken en zocht in}{de ijskast naar bier}\\

\haiku{Toen ze naast mij stond.}{legde ze haar hand op mijn}{haar en streelde het}\\

\haiku{Hij legde zijn hoofd.}{tegen haar bovenbeen en}{wilde een koekje}\\

\haiku{Ik had mijn vader.}{nooit  eerder in de weer}{gezien met een kind}\\

\haiku{Of was hij toen uit?}{angst niet toegekomen aan}{een afrekening}\\

\haiku{Hij hoefde niets meer.}{en besloot dus dat hij zich}{bezighield met mij}\\

\haiku{{\textquoteright} ~ Ik bestelde.}{nog een Calvados en keek}{op mijn horloge}\\

\haiku{Ik weet dat ik mij.}{in die droom tegen deze}{beelden verzette}\\

\haiku{Hij had waarschijnlijk.}{diezelfde wandeling met}{mijn vader gemaakt}\\

\haiku{Toen ik hun stemmen,.}{niet meer hoorde liep ik met}{Alma langs het strand}\\

\haiku{Vlak voor mijn vader.}{overleed had Hugo met mij}{over Olga gepraat}\\

\haiku{Die altijd zei dat.}{ze geen enkel probleem had}{met hun verhouding}\\

\haiku{Maar onmiddellijk.}{wordt ons duidelijk dat dit}{niet de manier is}\\

\haiku{Er was een tijd dat,{\textquoteright}.}{ik mijn moeder miste zei}{hij na enige tijd}\\

\haiku{{\textquoteleft}Ik zal je maar niet,}{zeggen dat je voor je op}{de weg moet kijken}\\

\haiku{{\textquoteleft}Als ik eraan denk!}{met wat voor belangrijke}{dingen jullie god}\\

\haiku{Het wild was van hem.}{gaan houden en niet meer weg}{te slaan van zijn spoor}\\

\haiku{Hij tilde zijn zoon.}{van de grond en drukte zijn}{hoofd tegen diens wang}\\

\haiku{Ik dacht eraan hoe.}{mijn vader op Fritsje}{had gereageerd}\\

\haiku{Hij wilde dat hij}{terug kon keren in de}{tijd om nog eenmaal}\\

\haiku{Ook wilde hij nog.}{een paar dingen beleven}{die hij gemist had}\\

\haiku{Omdat hij niet wist.}{wat hij dan tegen haar zou}{moeten zeggen}\\

\haiku{Ik stelde mij voor.}{dat Hugo al geruime}{tijd bij Olga was}\\

\haiku{Het irriteerde.}{hem dat Olga geen aandacht}{aan hem besteedde}\\

\haiku{Zij legde haar hand.}{tegen Asquits wang terwijl}{ze naar Hugo keek}\\

\haiku{Olga rukte zich.}{van hem los en gaf hem een}{klap tegen zijn wang}\\

\haiku{Daarna pakte ze.}{haar koffer op en liep de}{gang in naar de lift}\\

\haiku{Hugo  pakte.}{haar in haar nek en draaide}{haar hoofd naar zich toe}\\

\haiku{Wij liepen om het,.}{huis maakten een tochtje en}{deden boodschappen}\\

\haiku{Later op de avond.}{kwamen Alma en Hugo}{aangeschoten thuis}\\

\haiku{Toch  was het niet.}{alleen mode geweest maar}{tevens romantiek}\\

\haiku{Plotseling trad daar,.}{een ambtenaar op zijn pad}{als een soort Jezus}\\

\haiku{Maar hij kon er niet.}{onderuit te vertellen}{dat hij er wel was}\\

\subsection{Uit: Genoegens van weleer}

\haiku{sprekender dan de,.}{werkelijkheid hoewel dat}{bijna niet meer kon}\\

\haiku{Opeens begreep ik,}{dat zij mij een lafaard vond}{dat ze gehoopt had}\\

\haiku{Ze had altijd  .}{genoegen moeten nemen}{met de kruimeltjes}\\

\haiku{Zij was niet gesteld.}{op het uitwisselen van}{intimiteiten}\\

\haiku{Wuivende halmen.}{en boeren met de rode}{zakdoek om de hals}\\

\haiku{Toen bedacht ik mij {\textquoteleft}{\textquoteright}.}{dat het woordzonde niet het}{juiste was geweest}\\

\haiku{Ze dacht zeker dat.}{ik haar na afloop van de}{pret zou laten gaan}\\

\haiku{Ik begreep dat ik.}{eenzaam was geweest toen ik}{als student aankwam}\\

\haiku{Maar de teksten die.}{de stilte van mij hoorde}{waren heel anders}\\

\haiku{Ik ging een dagje.}{vissen met een oom die in}{de Wormer woonde}\\

\haiku{{\textquoteleft}Wat zou een beschaafd?}{mens niet van zo'n schitterend}{pand kunnen maken}\\

\haiku{Toen hij na een uur,.}{of twee terugkwam verviel}{hij in stilzwijgen}\\

\haiku{Stel dat het echt een,,?}{incident was Hugo waar}{moest ik dan naar toe}\\

\haiku{En het kon je niet.}{schelen waarom Nicolas}{zo vaak zijn mond hield}\\

\haiku{Je krijgt nog eens tbc,}{van al dat rondhangen in}{bibliotheken}\\

\haiku{Opeens verscheen een.}{interview van hem met een}{bekende dichter}\\

\haiku{Ik informeerde.}{of ze wel eens meeging met}{een vreemde kerel}\\

\haiku{Waarom had hij nooit}{doorgekregen dat zij zich}{alleen maar inhield}\\

\haiku{Ons dochtertje had.}{nooit gedacht dat vakantie}{zo vrolijk kon zijn}\\

\haiku{Ik probeerde haar,.}{voor de geest te halen maar}{dat lukte maar half}\\

\haiku{{\textquoteleft}Als je erg nodig,{\textquoteright}, {\textquoteleft}.}{moet zei het dienstmeisjedan}{mag je hier bij mij}\\

\haiku{Tegelijkertijd,.}{wilde ik ook de tuin zien}{met zijn regendak}\\

\haiku{De tafels die er.}{stonden waren beladen}{met glazen en drank}\\

\haiku{Ik probeerde iets,.}{te zeggen maar nog steeds was}{mijn stem op de loop}\\

\haiku{De tijd heeft aan de.}{herinnering het een en}{ander veranderd}\\

\haiku{Dat maakt natuurlijk.}{niets uit voor de waarde van}{de reconstructie}\\

\haiku{Ik legde de kaart.}{weg en ging bij een glas pils}{zitten nadenken}\\

\haiku{ik ooit kinderen,.}{zou hebben maar daar zag het}{nu even niet naar uit}\\

\haiku{Hij schrok af en toe.}{op en ging een half uur lang}{zitten luisteren}\\

\haiku{Tijdens het schrijven.}{kwamen andere dingen}{in mij naar boven}\\

\haiku{Maar het is de vraag.}{of ik me daarvan op dat}{moment bewust was}\\

\haiku{Is zoiets trouwens?}{mogelijk als het om je}{eigen leven gaat}\\

\haiku{Een paar weken per.}{maand verdiende zij wat geld}{met kleine baantjes}\\

\haiku{Zij trok haar jas aan,.}{en reed naar haar moeder die}{vlak bij haar woonde}\\

\haiku{Julia, Duisterhofs,.}{vrouw bezorgt het bedrijf zijn}{bittere kantjes}\\

\haiku{Het viel mij op dat.}{zij zich goed verzorgde en}{met smaak gekleed ging}\\

\haiku{Alles wat ik nooit,.}{geweten had kwam nu tot}{mij in vermoedens}\\

\haiku{Toen we elkaar pas,}{kenden wilde zij soms niet}{de deur uit verzon}\\

\haiku{Af en toe was het.}{fantastisch om haar eigen}{weg te kunnen gaan}\\

\haiku{{\textquoteright} zei Anna, nadat.}{Titia haar stoel verwisseld}{had voor het toilet}\\

\haiku{Ze waren wakker.}{geworden door het geluid}{van Anna's auto}\\

\haiku{Zij streelde Anna.}{over haar borsten en begon}{opnieuw te huilen}\\

\haiku{Ik zou er door sneeuw.}{willen lopen en het vuur}{laaiend opstoken}\\

\haiku{Anna zei dat ze.}{moe was en begaf zich naar}{de logeerkamer}\\

\subsection{Uit: De magere heilige}

\haiku{Zwijgend stonden ze,.}{elkaar aan te kijken zijn}{metgezel en hij}\\

\haiku{E\'en was lang en blond,,.}{de ander droeg een kort spits}{toelopend baardje}\\

\haiku{Nu is het zo dat.}{je zelfs kleine jongens niet}{met rust kunt laten}\\

\haiku{Zij kwamen niemand.}{tegen hoewel duizenden}{aanwezig moesten zijn}\\

\haiku{Maar  het meisje.}{had alles gezien en ze}{had het eten bij zich}\\

\haiku{Hij wil alleen dat.}{je naar het water kijkt en}{je kleren uittrekt}\\

\haiku{Als je mijn naam hoort}{en je ziet mij terug en}{je herkent mij niet}\\

\haiku{Niet bang zijn, ik heb -.}{je gered en ook jij De}{troep stond achter hem}\\

\haiku{Naakt en rillend in,.}{een kouder licht stond hij haar}{zo aan te gapen}\\

\haiku{{\textquoteright} lachte ze, maar hij,.}{zag niet dat ze lachte en}{zij lachte ook niet}\\

\haiku{De aanvoerder is,.}{toch altijd de lui die heeft}{zich kalm te houden}\\

\haiku{Onherkenbaar,{\textquoteright} zei.}{de jongen en keek alsof}{hem dat bedroefde}\\

\haiku{Hij keek naar haar door.}{de nog sterker wordende}{regenvlagen heen}\\

\haiku{Zij legde haar hoofd.}{even tegen zijn borst om niet}{te hoeven spreken}\\

\haiku{Maar dan volkomen, {\textquoteleft},{\textquoteright}.}{geheel en al.Ga mee naar}{de heuvel zei hij}\\

\haiku{Maar zij bleef in de,.}{abdij liep daar rond zonder}{om zich heen te zien}\\

\haiku{Dichtbij die uitgang.}{week zij plotseling naar rechts}{en volgde de muur}\\

\haiku{Maar daar, bij een luik,.}{in een half rondgebogen}{venstergat stond zij}\\

\haiku{Hij liep langzaam, met.}{zijn handen in zijn zakken}{van de abdij weg}\\

\haiku{Hij wilde het niet,,.}{geloven zo kon het niet}{zijn zo wanhopig}\\

\haiku{Paul gaf hem een hand,.}{wees hem een stoel en bood hem}{een sigaret aan}\\

\haiku{Niemand stak een hand.}{uit om hen tenminste naar}{boven te trekken}\\

\haiku{Vertel mij dat nog,{\textquoteright},.}{eens zei Rufus opeens zich}{naar haar toedraaiend}\\

\haiku{{\textquoteleft}Je bent een rothoer,.}{die geen man kan krijgen een}{vuile indringster}\\

\haiku{Alleen waar hij stond,.}{was een huis het wachthuis met}{daarnaast de kelder}\\

\haiku{Ook zij werd door de,.}{mannen gevonden vlak na}{hun verkrachtingen}\\

\haiku{Een algehele.}{gedaanteverwisseling}{had gezegevierd}\\

\haiku{Zij sliep vredig in.}{en de kerkklok luidde toen}{zij begraven werd}\\

\haiku{Zo zal ik je niet,.}{meer vinden zo zal je de}{afkomst nooit horen}\\

\haiku{Ellen die niet wil.}{weten dat jij nog kent wat}{zij heeft verloren}\\

\haiku{Het vulde zijn mond,.}{zijn tong bewoog er doorheen}{om het te vormen}\\

\haiku{Er kwamen mannen.}{die je meevoerden en niet}{durfden aanraken}\\

\haiku{Ik was net zo, en.}{daarom ben ik geworden}{zoals ik nu ben}\\

\haiku{Een meisje van op.}{zijn hoogst zestien jaar kwam uit}{de zee naar haar toe}\\

\haiku{Zij huilde en lag,.}{schokkend op haar buik met haar}{gezicht in het zand}\\

\haiku{Ik was bij hem en,.}{hij zorgde te goed  voor}{me en ik was jong}\\

\haiku{Zij is niets voor jou, -.}{en als je haar geloofd had}{Dat is de reden}\\

\haiku{{\textquoteright} Hij voelde hoe iets.}{in zijn keel schoot waardoor hij}{moeilijk kon praten}\\

\haiku{De boom helde en,.}{worstelde helde verder}{en brak langzaam af}\\

\haiku{- Want bedrogen in.}{haar poging nam de moeder}{haar dochter terug}\\

\haiku{Maar in het geheim.}{van hun verdwijning schreef zich}{de werkelijkheid}\\

\haiku{In die schijnbare}{seconde rust de hoop en}{groeit het raadsel dat}\\

\haiku{Soms voelde hij een.}{grote behoefte tussen}{mensen te zijn}\\

\haiku{De wijzer op de.}{snelheidsmeter boven het}{stuur kwam snel omhoog}\\

\haiku{Zijn metgezellen.}{letten erop dat hij niet}{in het gedrang kwam}\\

\haiku{Enkel de eenvoud:}{en enkel dat wat zij nog}{konden verkrijgen}\\

\haiku{Weggaan om in de.}{auto rustig te wachten}{op hun terugkomst}\\

\haiku{Je moet voor jezelf.}{zorgen en wij lieten het}{aan anderen over}\\

\haiku{Maar buiten het raam,.}{ontwaarde hij duisternis}{niets dan duisternis}\\

\haiku{Een grijsaard van het,.}{denken moe een idioot in}{zijn gevangenis}\\

\haiku{Het prikkeldraad was.}{tussen lange ruwhouten}{palen gespannen}\\

\haiku{Beiden keken zij,.}{recht voor zich uit wachtend op}{het eerste teken}\\

\haiku{Hij kon niet langer,.}{wankelde er naar toe en}{trok de dekens weg}\\

\haiku{Zij sloeg haar armen.}{om zijn hals en drukte zich}{vast tegen hem aan}\\

\haiku{Een ander die zich.}{in mijn plaats had gesteld of}{van wie zij dat dacht}\\

\haiku{{\textquoteleft}Je zal haar overal,.}{denken te vinden waar je}{zoekt zal je haar zien}\\

\subsection{Uit: Het moet allemaal nog even wennen}

\haiku{Toen Karel bij me.}{van de bank sprong knapten er}{meteen twee veren}\\

\haiku{Ik zag haar opeens.}{weer voor mij toen zij jong was}{en ik een kleuter}\\

\haiku{Zojuist samen lid.}{geworden voor f 3,50 van}{de McDonald's Club}\\

\haiku{{\textquoteright} riep iemand met een,.}{vrouwelijk accent wier stem}{ik niet herkende}\\

\haiku{De arts vulde met.}{vaste hand een spuit met sterk}{verdovend middel}\\

\haiku{Met mijn vader ging.}{ik naar de paardenraces}{op de boulevard}\\

\haiku{Daar is een kapper}{aan de gang geweest die zich}{ten doel heeft gesteld}\\

\haiku{De buren hebben.}{zojuist Rudolf en een hond}{langs de straat zien gaan}\\

\haiku{De onderste steen.}{zal bovenkomen bij het}{uitzoeken hiervan}\\

\haiku{Nemesis was bij.}{haar vergeleken een klein}{wurm in de luiers}\\

\haiku{Zij werd overreden,.}{terwijl zij in haar eentje}{door de natuur trok}\\

\haiku{Toen ik jong was deed.}{ik in de badkamer een}{jazztrompettist na}\\

\haiku{{\textquoteleft}En een prettige,{\textquoteright}.}{vakantie verder dan maar}{zei de saxofonist}\\

\haiku{het komende jaar.}{vooral goed te doen aan de}{gehandicapten}\\

\subsection{Uit: Ongenaakbaar}

\haiku{hij leverde het.}{absolute bewijs dat}{god nooit kon bestaan}\\

\haiku{Bovendien had hij.}{een zoon die dat allemaal}{wel voor hem uitzocht}\\

\haiku{{\textquoteright} {\textquoteleft}Lieveling, ik was.}{niet in de stemming voor de}{haute cuisine}\\

\haiku{Hij was er trouwens.}{niet aan gewend dat iets hem}{niet meteen lukte}\\

\haiku{Die zich uitkleedden.}{terwijl zij hun truttige}{kleren aanhielden}\\

\haiku{Maar bij die poging.}{vergat hij dat hij zichzelf}{niet kon ontlopen}\\

\haiku{Het speeksel droop langs.}{zijn brilleglas omlaag en}{drupte op zijn neus}\\

\haiku{Maar om die ander.}{te herkennen had hij geen}{helder zicht nodig}\\

\haiku{In die dagen vond,.}{Alice een vriend die haar met}{raad en daad bijstond}\\

\haiku{Philip zocht naar een.}{onderwerp waarover hij echt}{iets wilde zeggen}\\

\haiku{vond hij de zaak ook.}{weer niet ernstig genoeg voor}{al te veel tamtam}\\

\haiku{Hij besefte dat.}{die laatste uitspraak vooral}{iets over hemzelf zei}\\

\haiku{In zijn eigen huis.}{had hij de zaken perfect}{onder controle}\\

\haiku{Op die dagen ging}{Pietje Groenendijk gebukt}{onder een lading}\\

\haiku{Wat moest hij daarom?}{met een aanklacht tegen hem}{bij de politie}\\

\haiku{De dingen die hij.}{kon bedenken leken uit}{hem weg te zakken}\\

\haiku{Waarschijnlijk hadden.}{ze de avond met een wilde}{paring uitgeluid}\\

\haiku{En die vervolgens?}{opstond en in zijn wanhoop}{naar de bioscoop ging}\\

\haiku{{\textquoteright} {\textquoteleft}Omdat jij nog niet!}{eens een stukje deeg aan een}{haak durft te steken}\\

\haiku{Voorzichtig liet hij.}{het horloge uit zijn mouw}{op zijn schoot glijden}\\

\haiku{Die foto's wisten.}{de zorgelijke kanten}{van hun bestaan af}\\

\haiku{Dat de man van wie.}{zij had gehouden ver van}{haar af was geraakt}\\

\haiku{Hij schopte Vlasmans,.}{hoofd in elkaar terwijl hij}{hardop huilde}\\

\haiku{Het was opvallend.}{hoe prettig hij zich op dat}{moment al voelde}\\

\haiku{{\textquoteleft}Als we zeven keer.}{rond die school lopen zakt hij}{misschien in elkaar}\\

\haiku{{\textquoteright} Hij aarzelde, bleef.}{staan en keek omhoog langs de}{strenge voorgevel}\\

\haiku{{\textquoteleft}Als je een beetje.}{opschiet rijden we daarna}{vlug door naar Dinant}\\

\haiku{{\textquoteleft}Wilt u alstublieft.}{de rector bellen of hij}{mij ontvangen kan}\\

\haiku{{\textquoteright} {\textquoteleft}Daar stond niets over in,.}{zijn aantekeningen noch}{in het artikel}\\

\haiku{komaan, ik ga de.}{school van Vlasman eens met een}{bezoek vereren}\\

\haiku{{\textquoteleft}Ik vermoed dat u,}{in mij een medestander}{tegen Vlasman ziet}\\

\haiku{Zijn vader was geen.}{liefhebber geweest van een}{gevaarlijk leven}\\

\haiku{Hij wilde nog een,.}{keer met haar naar bed maar ze}{zei dat ze moe was}\\

\haiku{Ik ben bang dat je.}{van mij een verlengstuk van}{jezelf wilt maken}\\

\haiku{hij vreesde dat een.}{antwoord scherp en fluitend uit}{zijn strot zou komen}\\

\haiku{Zij kwam voor hem op.}{de grond zitten en legde}{een hand op zijn knie}\\

\haiku{Het proberen is,,.}{bedrog verwaandheid jezelf}{voor de gek houden}\\

\haiku{Vooral wanneer zijn.}{lichaam hem liet weten dat}{het haar nodig had}\\

\haiku{Ik heb er nog niet,{\textquoteright},.}{over nagedacht antwoordde}{hij niet naar waarheid}\\

\haiku{{\textquoteright} {\textquoteleft}Bij mij gaat het er,{\textquoteright}.}{niet helemaal om wat ik}{zelf wil zei Van Roon}\\

\haiku{Je vond het zeker?}{wel een buitenkansje wat}{je overkomen is}\\

\haiku{{\textquoteleft}Ik neem aan dat je.}{ervan genoten hebt het}{middelpunt te zijn}\\

\haiku{al was het alleen.}{maar omdat hij pas morgen}{hoefde optreden}\\

\haiku{Het liefst had hij de.}{verkoper een klap in zijn}{gezicht gegeven}\\

\haiku{Ik ben een van de.}{eersten in de wereld die}{iets van hem uitvoert}\\

\haiku{Ik had  ervoor.}{moeten zorgen dat ik me}{ook zo kon voelen}\\

\haiku{Zijn ouders kwamen.}{in die dagen om bij een}{auto-ongeluk}\\

\haiku{Eerst woonde hij met.}{Karen een paar maanden in}{een hotel in Aix}\\

\haiku{De bedrijfsleider.}{wilde weten van welke}{aard het bericht was}\\

\haiku{Ik was er zeker.}{van dat jij de hoorn niet van}{de haak zou nemen}\\

\haiku{De tederheid die.}{zij elkaar bewezen had}{iets vanzelfsprekends}\\

\haiku{Al in zijn laatste.}{jaar als leraar waren die}{dingen veranderd}\\

\haiku{{\textquoteright} {\textquoteleft}Stel dat ik bij jou.}{slaap en ik vertrek zonder}{je iets te zeggen}\\

\haiku{Als hij zo begon,.}{te redeneren kon hij}{nog een eind komen}\\

\haiku{Hij keek uitdagend,.}{naar zijn gastvrouw de kin een}{beetje geheven}\\

\haiku{En dat hij waar je?}{bijzit beslist over al of}{niet publiceren}\\

\haiku{{\textquoteright} {\textquoteleft}Zou je alles zelf?}{in handen willen houden}{wat je werk aangaat}\\

\haiku{Hij draaide zich naar,.}{Karen die tegen een der}{vleugels stond geleund}\\

\haiku{{\textquoteleft}Wat ons te doen staat.}{is het tegengaan van de}{vanzelfsprekendheid}\\

\haiku{Weer stond zij roerloos.}{op de plaats die zij eerder}{had ingenomen}\\

\haiku{Wanneer u zichzelf}{als een idioot beschouwt zou}{ik het niet wagen}\\

\haiku{Ik heb ze gezien,.}{in Parijs in de jaren}{twintig onder meer}\\

\haiku{Er bestaan dingen,.}{die van alle tijden zijn}{meneer Greveling}\\

\haiku{Verliest u dat niet,.}{uit het oog wanneer u voor}{uzelf op de vlucht slaat}\\

\haiku{Maar hij was te moe.}{om er op dit ogenblik diep}{in door te dringen}\\

\haiku{Bij de deur keerde.}{Alkowski zich nog een keer}{om en wenkte hem}\\

\haiku{{\textquoteright} {\textquoteleft}Op Henri na ben.}{jij de eerste aan wie hij}{dit wil laten zien}\\

\haiku{De zon stuiterde.}{een paar keer en rolde toen}{over de heuvels weg}\\

\haiku{Het voornaamste dat.}{mij aan hem bindt is dat ik}{hem moet beschermen}\\

\haiku{Zij verzekerde.}{hem dat hij zich geen zorgen}{hoefde te maken}\\

\haiku{Vervolgens rende.}{hij bijna in de richting}{van de aankomsthal}\\

\haiku{Hij probeerde zich.}{Alkowski voor te stellen}{op zijn  sterfbed}\\

\subsection{Uit: De paradijsganger}

\haiku{{\textquoteleft}Mag ik vanavond even?}{langskomen om te zien of}{alles naar wens is}\\

\haiku{{\textquoteright} Adela{\"\i}de loopt.}{met hem mee in de richting}{van het station}\\

\haiku{{\textquoteright} roept Adela{\"\i}de,.}{opeens met een verheugde}{lach op het gezicht}\\

\haiku{{\textquoteright} Johannes beseft.}{dat hij zich geen fouten meer}{kan veroorloven}\\

\haiku{Het gelach breidt zich.}{uit tot in de voorste en}{achterste coup\'es}\\

\haiku{De straal waarmee hij.}{doorgaans plast is niet meer zo}{krachtig als vroeger}\\

\haiku{Johannes knikt de.}{jongeman vriendelijk toe}{en steekt zijn hand uit}\\

\haiku{Tenslotte zou hij,.}{dichter worden zij het pas}{na het eindexamen}\\

\haiku{{\textquoteleft}Ik zou eigenlijk.}{best eens met vakantie naar}{Miami willen}\\

\haiku{{\textquoteleft}Je kunt niet surfen,{\textquoteright}.}{op de Amazone stelt zijn}{zoon spijtig vast}\\

\haiku{Een hem tegemoet.}{rijdende dame klemt haar}{handen om het stuur}\\

\haiku{Bij het nemen van.}{een lichte bocht begint zijn}{fiets te zwabberen}\\

\haiku{Alleen een bel. {\textquoteleft}En,{\textquoteright}.}{vraagt Debbie als hij later}{op de avond thuiskomt}\\

\haiku{Daarna stapt hij naakt,.}{de gang in waar een kast staat}{met zijn ondergoed}\\

\haiku{Eigenlijk zou hij.}{het short over zijn spijkerbroek}{moeten aantrekken}\\

\haiku{Dat zou prachtig zijn,.}{een flitsend schot dat als een}{streep de lijn passeert}\\

\haiku{{\textquoteleft}Wat een mafkees zeg,,{\textquoteright}.}{die ouwe hoort hij Bertus}{op de trap zeggen}\\

\haiku{{\textquoteleft}Waarbij de ster van.}{Bethlehem zal verbleken}{tot een vuurvliegje}\\

\haiku{Debbie zegt dat de,.}{cake van binnen zacht is}{met sinaasappelsmaak}\\

\haiku{{\textquoteleft}Ieder jaar wordt de.}{korting die het leven biedt}{een beetje minder}\\

\haiku{Hij gaat op tafel.}{klimmen en meedelen dat}{hij heeft gelogen}\\

\haiku{{\textquoteright} informeert hij, na.}{een korte observatie}{van de pati\"ent}\\

\haiku{{\textquoteleft}Als jullie dan niet,}{naar me willen luisteren}{vind ik het ook best.}\\

\haiku{Misschien valt hij in,.}{slaap of raakt anderszins in}{het ongerede}\\

\haiku{Johannes stapt in,.}{steekt het sleuteltje in het}{contact en toetert}\\

\haiku{Maar hij weet al te,.}{goed dat Debbie daar niet ligt}{noch iemand anders}\\

\haiku{Maar om de een of.}{andere reden heeft hij}{weinig zin in vis}\\

\haiku{En dan moet je nog.}{maar hopen dat die vrouwen}{ook de straat opgaan}\\

\haiku{\ensuremath{\star} ~ Johannes,,.}{zit met zijn vriend de dokter}{in het dorpscaf\'e}\\

\haiku{{\textquoteleft}Ik wil u een klein,{\textquoteright}.}{probleem voorleggen zegt hij}{tegen de dokter}\\

\haiku{Hoewel hij dan weer,.}{op een boekhouder lijkt zo}{keurig afgepast}\\

\haiku{In de tussentijd.}{kan de hond moeiteloos tot}{de aanval overgaan}\\

\haiku{dat ik hier niet loop.}{als een soort geleidemens}{voor bange hondjes}\\

\haiku{{\textquoteright} Het kwispelen van.}{de hond is niet langer een}{uiting van instinct}\\

\haiku{Als Johannes dicht,.}{bij haar is bukt hij zich om}{een schoen te strikken}\\

\haiku{Hij glimlacht en na.}{enige vertraging glimlacht}{zij naar hem terug}\\

\haiku{{\textquoteleft}Ik heb uw kaasjes,{\textquoteright},.}{gegeten zegt Johannes}{als een grote knul}\\

\haiku{{\textquoteright} Vroeger, toen hij zelf, {\textquoteleft}{\textquoteright}.}{studeerde waswerkstudent}{een bekend begrip}\\

\haiku{{\textquoteleft}Ik had de laatste.}{maanden erg veel mensen om}{mij heen Nadine}\\

\haiku{Verwacht zij dat hij?}{zal zeggen dat hij altijd}{naar haar verlangd heeft}\\

\haiku{Haar vervolgens ook.}{weer niet als een dolleman}{in de nek bijten}\\

\haiku{Het is alsof hij.}{zich aan iets vastklampt dat hij}{geen naam kan geven}\\

\haiku{Maar kijk, daar is het,.}{terug onverhoeds en met}{alle tederheid}\\

\haiku{Volzinnen wellen.}{in hem op als luchtbellen}{in kokend water}\\

\haiku{Waarna ik over de.}{relatie met mijn vriend zou}{moeten nadenken}\\

\haiku{Niet ver van hen, waar,.}{de stroom zich vernauwt tot een}{straaltje wacht een hond}\\

\haiku{{\textquoteleft}Maar ik moet toch eens,{\textquoteright}.}{zien waar je bent opgegroeid}{probeert Johannes}\\

\haiku{Hij weet zo gauw niets,.}{te verzinnen daarom laat}{hij de map schieten}\\

\haiku{Met zijn kloppende}{en naar aandacht en liefde}{hunkerende hand.}\\

\haiku{Zo Johannes, je?}{begrijpt zeker wel wat ik}{je kom aanzeggen}\\

\haiku{Aan de bomen hangt.}{een uitgelezen keur aan}{verboden vruchten}\\

\haiku{Johannes begint.}{te toeteren en met zijn}{lampen te seinen}\\

\haiku{Zij is veranderd.}{in een besnorde dikke}{man met een blauw hemd}\\

\haiku{{\textquoteright} Nathalie pakt een.}{badhandoek en wikkelt hem}{er helemaal in}\\

\haiku{Toch is het raar, het.}{onverwachte geschenk stemt}{hem niet  vrolijk}\\

\haiku{Ik wist niet dat je,.}{er zulke praktijken op}{nahield Johannes}\\

\haiku{{\textquoteleft}En dat ik na de.}{oorlog kon meewerken aan}{de wederopbouw}\\

\haiku{{\textquoteright} {\textquoteleft}Als kind zat ik het,.}{liefst in de box met een mooi}{boek over konijntjes}\\

\haiku{Ze plukt de zwarte.}{kater van de bank en stopt}{hem onder haar trui}\\

\haiku{Als hij het raam in,}{zijn kamer open heeft gezet}{komt Wanda binnen}\\

\haiku{Ik vond 75.000 meteen,{\textquoteright}.}{al aan de hoge kant moet}{hij toegeven}\\

\haiku{Maar op de een of.}{andere manier heeft hij}{dit niet meer nodig}\\

\haiku{In de verte, zo,.}{ver dat hij hem niet meer kan}{inhalen gaat Bas}\\

\subsection{Uit: De terugkeer van Buffalo Bill}

\haiku{Wisowsky dacht aan zijn,.}{kinderjaren aan vriendjes}{en aan schatgraven}\\

\haiku{Het leek of zijn brein.}{zich verzette tegen een}{uitspraak over de kaart}\\

\haiku{Bierflessen zorgden.}{voor de opvulling van open}{plekken op de grond}\\

\haiku{Hij liep regelrecht.}{op de antiquair af en}{sloeg hem om zijn oren}\\

\haiku{De kleinzoon van de:}{vervalser Costello}{vouwde een krant open}\\

\haiku{Hij vroeg zich af in}{hoeverre Marylou een}{idee had wat voor werk}\\

\haiku{Lorenzo opent zijn.}{ogen die niet meer dan een vlek}{kunnen waarnemen}\\

\haiku{ik word hoe meer ik.}{alleen nog kan lachen om}{Laurel en Hardy}\\

\haiku{Eerdaags zag hij die.}{meid misschien terug in een}{iets grotere rol}\\

\haiku{Geheimzinnige.}{tekens vielen er voor hem}{niet te ontdekken}\\

\haiku{we hebben twee uur.}{zitten kijken en alles}{is voor niets geweest}\\

\haiku{- Hij zou niet op mij,.}{lijken maar op u. Als u}{in hem geloofde}\\

\haiku{Op z'n hoogst lijkt het.}{of de steenmassa's nog wat}{zijn toegenomen}\\

\haiku{Nu vergaap ik mij.}{aan de uitgestrektheid van}{deze ru{\"\i}nes}\\

\haiku{Hij zat in bus 118.}{en sprak deze keer met een}{Amerikaans accent}\\

\haiku{Dat laatste is in.}{\'e\'en zin de inhoud van een}{klein miljoen preken}\\

\haiku{Het is duidelijk.}{dat iemand daarmee over het}{water kon skie\"en}\\

\haiku{Ik zeg niet dat ik,.}{er nu naar verlang maar ik}{ben er wel aan toe}\\

\haiku{Pompeji is nog,.}{beklemmender daar sloeg de}{dood plotseling toe}\\

\haiku{Dit alles had bij:}{nader inzien iets weg van}{een reclamespot}\\

\haiku{Amerika, dat is,?}{typisch een land voor zusjes}{vind je ook niet}\\

\haiku{En je onderwijst.}{studenten hoe ze in de}{grond moeten graven}\\

\haiku{Sauzen druipen uit.}{kant-en-klaarflessen over}{de rozestruiken}\\

\haiku{Jouw publiek beleeft.}{in ieder geval nog wat}{plezier aan de schoft}\\

\haiku{En dat geval met.}{die wielrenner heeft je dat}{duidelijk gemaakt}\\

\haiku{Hun prestatie is.}{al vergeten wanneer hij}{nog op de pers ligt}\\

\haiku{Die poppen staan er,.}{misschien nog ik ben daar in}{geen jaren geweest}\\

\haiku{Meestal hadden.}{ze een van die twee wel op}{hun kamer hangen}\\

\haiku{Als hij zijn kop in,.}{een t.v.-studio laat zien}{wordt hij gelynchd}\\

\haiku{Hoe ouder ik word,.}{Ronald hoe minder ik zou}{moeten aandringen}\\

\haiku{Een boek waarin je.}{geschreven moet hebbenom}{erbij te horen}\\

\haiku{Het zal een boek zijn.}{over de studenten en het}{omslag maak ik wit}\\

\haiku{Dit is Amsterdam{\textquoteright},.}{zei de reporter op de}{andere zender}\\

\haiku{Nu konden alleen.}{de mensen die zijknat wilden}{worden naar buiten}\\

\haiku{Enige tijd later.}{belde de burgemeester}{voor de tweede keer}\\

\haiku{- Vroeger was ik een,,.}{nobody een nitwit riep}{de medewerker}\\

\haiku{- Is heden op de,.}{Dam ongesteld geworden}{antwoordde Bertie}\\

\haiku{Ik wil vrede op.}{aarde en in de mensen}{een welbehagen}\\

\haiku{Daarin ligt de zin:}{van al die zogenaamde}{gelijkvormigheid}\\

\haiku{Zo waren ze weer.}{met het verleden in een}{onderonsje}\\

\subsection{Uit: Verleidingen}

\haiku{{\textquoteleft}Ik herinner me,{\textquoteright}.}{die ene kelner zei opa op}{zijn zeventigste}\\

\haiku{hij figuren die.}{hem doen achterblijven en}{voor zich uit staren}\\

\haiku{Jij hebt meer aan een,{\textquoteright}.}{stok zonder punt voegde mijn}{moeder eraan toe}\\

\haiku{Mijn vader hief zijn.}{staf en wees ermee in de}{richting van het raam}\\

\haiku{Oma vond het maar niks,{\textquoteright}.}{dat ze Frans praatten zei mijn}{vader op een keer}\\

\haiku{De caf\'es stonden.}{vol koffiemachines en}{fruitautomaten}\\

\haiku{Onmiddellijk dacht.}{ik aan mijn vader en zijn}{gepunte bergstok}\\

\haiku{Aan de andere.}{kant van de draaideur scheen een}{waterige zon}\\

\haiku{Het verbaasde hem.}{dat zij onmiddellijk in}{het boek verdiept was}\\

\haiku{Terwijl hij naar haar,.}{staarde keek zij even van het}{boek op en zag hem}\\

\haiku{Opnieuw voelde hij.}{de zachte aandrang van het}{mensje achter hem}\\

\haiku{Hij wist niet precies,.}{wie het was maar die dode}{daar hoorde bij hem}\\

\haiku{De directie was.}{van mening dat het hem aan}{niets mocht ontbreken}\\

\haiku{Misschien begaf hij.}{zich vaker onopgemerkt}{onder de mensen}\\

\haiku{Verzoeken om een,.}{optreden voor een vader}{die op sterven ligt}\\

\haiku{Alsof Dauzenberg.}{bij een sterfbed grappen zou}{kunnen verkopen}\\

\haiku{{\textquoteleft}Maar dan door iemand.}{die deskundig is op het}{gebied van dromen}\\

\haiku{Hij trok zich in zijn.}{studeerkamer terug en}{ik bleef bij Suze}\\

\haiku{Zij las mijn grappen,.}{lachte erom en vulde}{mij af en toe aan}\\

\haiku{Eerst leek het of hij.}{een herhaling bracht van zijn}{oude successen}\\

\haiku{Zij kreeg zoveel geld.}{in handen en toch was zij}{nog altijd een kind}\\

\haiku{Door de luidspreker.}{hoorden wij iedere avond}{dezelfde grappen}\\

\haiku{Ogenschijnlijk is dat.}{de slechtste plaats die ik had}{kunnen bedenken}\\

\haiku{Is dit niet het punt?}{waar overdrijving het wint van}{de redelijkheid}\\

\haiku{En ik voorvoelde.}{dat ik mij deze keer niet}{bekocht zou voelen}\\

\haiku{Deze keer is hij.}{hier voor de eerste keer niet}{alleen gekomen}\\

\haiku{Misschien zijn wij bang,}{om ons te binden vinden}{wij het allebei}\\

\haiku{Klachten bleven uit,.}{omdat niemand nog hier zijn}{vakantie doorbracht}\\

\haiku{Er was iets in haar.}{manier van lopen dat hem}{biologeerde}\\

\haiku{Het besef dat zij.}{niets met hem te maken had}{vergrootte zijn pijn}\\

\haiku{{\textquoteright} Zij schepte water.}{in haar hand en spatte het}{tegen mijn lichaam}\\

\haiku{Toen mijn vader het,.}{bootje kocht zag het er zo}{fris geschilderd uit}\\

\haiku{Dat je kunt denken.}{dat je tijd voldoende hebt}{om te verslapen}\\

\haiku{De vorige avond.}{had ik de inhoud voor het}{laatst gecontroleerd}\\

\haiku{Meneer Simon zat.}{op zijn knie\"en voor me en}{wreef over mijn wangen}\\

\haiku{En zoals zij zich,!}{moest wassen omgeven door}{andere vrouwen}\\

\haiku{toen ik zag dat dit,.}{niet lukte besloot ik de}{kamer uit te gaan}\\

\haiku{nu was ik het die.}{het ene miniflesje na het}{andere leegdronk}\\

\haiku{Vanavond mag je er,{\textquoteright}.}{net zo lang over doen als je}{wilt zei mijn vader}\\

\haiku{Ik keek door de ruit.}{naar de boulevard en zag}{daarin mijn moeder}\\

\subsection{Uit: Verliefdheid is een raar gevoel}

\haiku{Die Madock moet dus.}{weer een ander werk van de}{schrijver Willem zijn}\\

\haiku{Toch komt er bij dit:}{alles nog \'e\'en ding dat wij}{niet kunnen voorzien}\\

\haiku{Het jongetje dat,.}{deze tekst eens opschreef heet}{toevallig Jasper}\\

\haiku{Na die eerste zin.}{komt er iets dat de lezer}{aan het lachen maakt}\\

\haiku{Als je die koning,.}{dan toch Joop noemt streef je een}{bepaald effect na}\\

\haiku{Dus hij gebruikt ook,,.}{zijn vrouw en zijn moeder en}{misschien de slager}\\

\haiku{Iedere schrijver.}{heeft in zijn werk goede en}{slechte koningen}\\

\haiku{Intussen zag ik.}{heel wat mannen elke keer}{weer naar haar kijken}\\

\haiku{{\textquoteright} Bij besprekingen:}{in kranten en weekbladen}{zien wij hetzelfde}\\

\haiku{er zijn lezers die.}{doodgewoon niet leuk vinden}{wat zij schrijven}\\

\haiku{En misschien is het}{aardig om eerst eens over die}{vraag na te denken}\\

\haiku{Zat die processie,.}{er eenmaal op dan was het}{feesten geblazen}\\

\haiku{Die omtrekkende.}{bewegingen verhogen}{de natuurlijkheid}\\

\haiku{Die kwam al snel en.}{kon deze dief nog net op}{tijd arresteren}\\

\haiku{De man deed nog een.}{stap in de richting van de}{boom waar Peter stond}\\

\haiku{een boek mag nooit zo.}{zijn dat je erbij in slaap}{valt van verveling}\\

\haiku{Schrijvers van series.}{als Arendsoog beheersen hun}{vak helemaal niet}\\

\haiku{Dat leidt om zo te.}{zeggen een beetje af van}{de slordige stijl}\\

\haiku{) is een omheining,.}{of muur later ook gebruikt}{voor hek of slagboom}\\

\haiku{Een ouwe kluiver{\textquoteright},.}{is dus een oude snoeper}{een oude geilaard}\\

\haiku{Alleen de laatste,,.}{sc\`ene waarin Cyrano}{sterft is heel verstild}\\

\haiku{Mijn ontroering bij.}{de laatste sc\`ene zei wat}{dat betreft genoeg}\\

\haiku{In de eerste plaats.}{is daar de vraag of een tekst}{leesbaar is gedrukt}\\

\haiku{Trouwens, toen ik het,.}{boek zelf weer eens inkeek vond}{ik er niet veel aan}\\

\haiku{{\textquoteleft}Het was een leuk boek.}{dat over een jongetje ging}{die naar New York ging}\\

\haiku{Want zijn vader en.}{moeder kwamen in New York}{trug van vakantie}\\

\haiku{{\textquoteright} Of bij de dame,:}{met de slangen die immuun}{is geraakt voor gif}\\

\haiku{{\textquoteright} De man had dan aan}{die woorden genoeg gehad}{om te begrijpen}\\

\haiku{Dan had hij via die.}{dialoog de emotie van de}{spreker overgebracht}\\

\subsection{Uit: Een vrouw als een gedicht}

\haiku{Maar het gaat hier niet.}{om racefietsen maar om}{die nauwkeurigheid}\\

\haiku{in die kringen zag.}{men de beeldenstorm toch meer}{als een dieptepunt}\\

\haiku{{\textquoteleft}Als 't niet op de,{\textquoteright}:}{dominee drupt dan drupt het}{wel op de koster}\\

\haiku{Bij de koekoek als.}{duivel blijft het overigens}{ook een linke zaak}\\

\haiku{Persoonlijk wil ik.}{maar met heel weinig mensen}{iets tot stand brengen}\\

\haiku{Wij bieden deze,.}{aan via een formule een}{beleefdheidsfrase}\\

\haiku{{\textquoteright} Dit kun je zeggen.}{tegen iemand die altijd}{maar jammert en zeurt}\\

\haiku{{\textquoteright}        Opstand van de.}{Kamper uien De mensen}{in Kampen zijn kwaad}\\

\haiku{De volzin van Luns.}{die we hier ontleden is}{ook heel welsprekend}\\

\haiku{{\textquoteleft}tortre le cou \`a{\textquoteright} (:}{une bouteillede hals van}{een fles afwringen}\\

\haiku{Je had er geen flauw.}{benul van dat je bijna}{had moeten dansen}\\

\haiku{Het antwoord daarop.}{ligt voor de hand.             Een vrouw}{als een gedicht 1}\\

\subsection{Uit: De weerspannige naaktschrijver}

\haiku{Rudolf Geel De}{weerspannige naaktschrijver}{Colofon}\\

\haiku{in je hoofd draag je.}{dat fraaie boek op een zijden}{kussen met je mee}\\

\haiku{duivelse driften.}{werpen hun venijnige}{schaduwen vooruit}\\

\haiku{Een heimelijke,.)}{sigaret wij rookten geen}{van allen openlijk}\\

\haiku{, verdomde strakke -.}{rotonderbroek ik kom hier}{nooit op tijd vandaan}\\

\haiku{Misschien zullen wij (.}{aan het eind van de tochtIk}{moet nodig pissen}\\

\haiku{Ik leg mijn arm om.}{Ansje's schouder en probeer}{haar aan te kijken}\\

\haiku{Het oude wijfje,;}{klemt zich aan mij vast haar mond}{tegen mijn knie\"en}\\

\haiku{Een pijltje zeilt door.}{de lucht maar valt nog v\'o\'or de}{schutting op de grond}\\

\haiku{Een enkele dringt.}{met een dof geluid in het}{hout van de schutting}\\

\haiku{In de sluis zien wij.}{de bliksem als wij achter}{elkaar voortrennen}\\

\haiku{Ik heb mijzelf tot,}{gevangene gemaakt ik}{moet hier wegkomen}\\

\haiku{Het regent overal,.}{het is voortdurend herfst op}{de oude kopie}\\

\haiku{De verdovende.}{luchtdruk doet de vijanden}{tegen de grond slaan}\\

\haiku{En steeds keren wij,,.}{terug draaien om hen heen}{wij kennen geen angst}\\

\haiku{Ik moet gaan zitten,.}{wegzinken in een chaos}{of mij omdraaien}\\

\haiku{Meteen springt hij op.}{mij af en duwt mijn hoofd met}{kracht tegen de grond}\\

\haiku{En ik - wil zeggen,.}{dat wij samen over de hei}{moeten wandelen}\\

\haiku{Wat zal ik doen, ik.}{zet mijn pen op papier en}{begin te schrijven}\\

\haiku{Maar ik bedenk, dat,:}{daar geen kans op is laat}{ik niet nadenken}\\

\haiku{Ik klim het stenen.}{trapje op naar het platform}{voor de keukendeur}\\

\haiku{- Maar hij wil ons niet.}{vertellen waarom hij in}{mijn tuin wandelde}\\

\haiku{Deze eigenschap.}{begrijpen wij in hoge}{mate van elkaar}\\

\haiku{Op Zondagsschool het.}{nummer met de dikke en}{de dunne boekjes}\\

\haiku{En wanneer valt de?}{verwezenlijking van de}{wens schrijver te zijn}\\

\haiku{- Neem die rommel maar,,,,.}{mee naar huis zegt ze Johan Johan}{je bent een warhoofd}\\

\haiku{- Dan drinken we straks,.}{een slokje om het weg te}{spoelen zegt Elize}\\

\haiku{- Alle begin is,,.}{moeilijk zeg ik hees terwijl}{de hoop mij ontzinkt}\\

\haiku{Anneke is de,.}{eerste om de beurt gaan de}{meisjes naar de wc}\\

\haiku{Opnieuw beginnen,}{wij te lachen wij slaan de}{deken van ons af}\\

\haiku{Weet U nog, Pieter?}{die uitgleed met een cognacfles}{in zijn achterzak}\\

\haiku{Als je natuurlijk,.}{met een eigen meisje gaat}{ben ik \'e\'en teveel}\\

\haiku{Als het vervelend,.}{is kunnen we nog naar de}{tweede voorstelling}\\

\haiku{Dat is niet zo heel,.}{bijzonder want ik schreef er}{iedere dag \'e\'en}\\

\haiku{Je moet me echter.}{nooit vragen of ik het voor}{je wil opzoeken}\\

\haiku{- Nee hoor, zegt Elize,.}{maar mijn verloofde moest naar}{een begrafenis}\\

\haiku{Ik pak het glas op,.}{ledig het in \'e\'en teug en}{let scherp op de deur}\\

\haiku{Zij zorgen dat de.}{mensheid hem kennen leert via}{kranten en folders}\\

\haiku{- Luister je nu, vraagt,.}{de jongen verlegen het}{blad verfrommelend}\\

\haiku{De dichter buigt zich.}{over mij heen en verzoekt mij}{te blijven zitten}\\

\haiku{- Merkwaardig toeval,.}{mompelt de rektor die over}{haar schouder meeleest}\\

\haiku{Het is vertekend.}{en op de nieuwe vorm ben}{je niet ingesteld}\\

\haiku{Kevelkin kent mijn,.}{verrichtingen van horen}{zeggen uit haar krant}\\

\haiku{Hij loopt naar zijn plaats ().}{voor de toonbanknadat hij}{zich geordend heeft}\\

\haiku{Niet weg te slaan van.}{de stierenvechters in de}{glimmende pakjes}\\

\haiku{ze is geen maagd meer,,?}{waar bemoei je je mee hoe}{kan ik dat weten}\\

\haiku{Voorlopig vertel,,;}{ik van de hak op de tak}{zoals ik nooit doe}\\

\haiku{En Albert draait zich,,.}{ach verdomme het kan me}{geen flikker schelen}\\

\haiku{Ik heb slaap, de week.}{loopt ten einde en ik ben}{nog steeds oververmoeid}\\

\haiku{hier is Matti, wij.}{mogen hem geen moment aan}{zichzelf overlaten}\\

\haiku{Als ik terugkom,,!}{in Peru dan staan ze daar}{weer ze zijn overal}\\

\haiku{Ik ga alleen dood,,.}{net als jullie maar het is}{niet fijn met een mes}\\

\haiku{Het is misschien het,.}{laatste dat ik hoor maar ik}{zal dit meenemen}\\

\section{Reinier van Genderen Stort}

\subsection{Uit: Hinne Rode}

\haiku{Soms brandde de zon,.}{op haar wangen ondanks het}{groene bladerdak}\\

\haiku{Zij was nu vijftien.}{jaar en geleek haar vader}{meer dan haar moeder}\\

\haiku{Maar ook den harden,,}{vader had Hinne lief met}{een liefde dieper}\\

\haiku{de slag, waarmede,.}{zij dichtviel weergalmde in}{de stilte der gracht}\\

\haiku{- Ik weet nog niet of,...}{ik dan zal kunnen ik zal}{je nog berichten}\\

\haiku{- Ik voel er alles,...}{voor weer eenige dagen naar}{Gelderland te gaan}\\

\subsection{Uit: Kleine Inez}

\haiku{Zijn staatkundige;}{meerderheid werd door zijn felste}{vijanden erkend}\\

\haiku{Maar na eenigen tijd;}{hervatte de jonge man}{zijn zwervend leven}\\

\haiku{Een groote haviksneus,,.}{aan het einde lichtelijk}{afwijkend dreigde}\\

\haiku{Hij hoorde gejaagd,.}{loopen onderdrukt praten}{en slaan van deuren}\\

\haiku{vormde zich weer als,.}{op de vroolijke school die}{hij verlaten had}\\

\haiku{Zijn vader kwam over,.}{onderzocht hem langen tijd}{met den huisdokter}\\

\haiku{Dan werd een keulsche.}{aak door een kleine stoomboot}{stroomopwaarts gesleept}\\

\haiku{zoo bleef het licht ook.}{van den hellen zomerdag}{gedempt en rustig}\\

\haiku{zij zat bij hem en.}{liet spelenderwijs heur haar}{glanzen in de zon}\\

\haiku{- Ja oom, erkende,,...}{zij verward en kleurend ik}{zal het niet meer doen}\\

\haiku{dan vergat hij te.}{luisteren naar hetgeen de}{touwslagers zeiden}\\

\haiku{Een winterstorm had;}{een noorschen driemaster op}{het strand geworpen}\\

\haiku{Dan schoof langzaam en.}{sissend een locomotief}{achter hem voorbij}\\

\haiku{Hij stond op na een,'.}{poos slenterde terug tot}{kleine Inez huis}\\

\haiku{het scheen, als restte.}{haar geen kracht meer voor opstand}{en verbittering}\\

\haiku{Het sneeuwde en vaal.}{stroomde de rivier tusschen}{de witte oevers}\\

\haiku{Na dit gesprek werd.}{hun omgang weer gelijk hij}{vroeger was geweest}\\

\haiku{Peter las dezen,,.}{brief staande bij het raam een}{weinig gebogen}\\

\haiku{Een stoomschip, heel klein,.}{trok een matte rookstreep}{aan den einder}\\

\section{Caesar Gezelle}

\subsection{Uit: Uit het leven der dieren}

\haiku{liggen kijken door}{een grooten kijker naar den}{hemel en sterren}\\

\haiku{evengauw waren ze '}{we\^er open en ze wendden in}{hunne holten bij}\\

\haiku{Menig stonden daar ':}{de boomen int gelid}{op schuinsche reken}\\

\haiku{'Nen schoonen stillen:}{avond was er een van de twee}{buitengebleven}\\

\haiku{Sam keek ze na en,.}{wachtte wel indachtig wat}{er gebeuren moest}\\

\haiku{hij volgde haar op,:}{den voet al hijgen overal}{in en overal uit}\\

\haiku{met den hongerdood}{in zijn ingewand zat hij}{erop te bijten}\\

\haiku{ze trok haar we\^er in;}{en bleef zitten boven den}{trap voor een verbei}\\

\haiku{deed de goorpuid we\^er, '}{met eene zware basstem en}{ze dedent hem}\\

\haiku{Geschorst was ineens,;}{de male en geen puiden}{waren meer te zien}\\

\haiku{Zoo gauw ze u in,.}{de ooge krijgen  gapen}{ze lijk een afgrond}\\

\haiku{Maar we moeten van,}{den nood eene deugd maken en}{alles nemen zoo}\\

\haiku{Al krinkelen met,}{kop en eers verdween hij en}{in twee drie slokken}\\

\haiku{Ten langen, ten heel.}{langen laatste was hij het}{toch beu geworden}\\

\haiku{de grootste was hij,,;}{zoo daar geen meerdere en}{kwam de felste zanger}\\

\section{Josephine Giese}

\subsection{Uit: Van een droom}

\haiku{{\textquoteright} {\textquoteleft}Indien u voor mij,.}{gezongen hebt zal ik u}{zeggen wie u ben}\\

\haiku{Als in een droom, ging,:}{zij naar de piano en}{hoorde achter zich}\\

\haiku{daar gaat het leven,.}{om als gegrepen  in}{een reuzig vliegwiel}\\

\haiku{Eentonig als het:}{luiden eener doodsklok gaat}{door haar ziel de klacht}\\

\haiku{Wie is de man, die?}{de volkomen hoogheid vat}{van zulk een geven}\\

\haiku{Zij scheidde met groote,,.}{smart want zij wist dat na dit}{heengaan geen keeren was}\\

\haiku{Hoog aan den hemel,,,.}{stond aan wolkige lucht in}{rosse krans de maan}\\

\haiku{Het scheen te dalen,,.}{van de trap die wegvluchtte}{hoog in het donker}\\

\haiku{Maar telkens keerde.}{zij tot het lezen weer en}{tot het overdenken}\\

\haiku{Zij had in beide,,.}{stadi\"en als het ware}{hem voorbijgedroomd}\\

\haiku{Slechts liefde had haar - -.}{kunnen binden doch die was}{dood dus was zij vrij}\\

\haiku{Uit de brieven van.}{Ren\'e begon te spreken}{een haat aan het tooneel}\\

\haiku{Zou die wereld nu,:}{ook de oogen niet eens opengaan}{zou zij niet vragen}\\

\haiku{Toen hief zij als een,.}{bloem naar het licht haar gelaat}{naar het zijne op}\\

\haiku{Hij klemde haar in,,.}{zijn armen zoende haar op}{wangen mond en oogen}\\

\haiku{dit was de stem der,.}{maatschappij die sprak met de}{stem harer moeder}\\

\haiku{Zij nam zijn gezicht,;}{tusschen haar beide handen}{en keek hem aan lang}\\

\haiku{Later ging Richard.}{tot zijn vrienden en bleef het}{beneden doodstil}\\

\haiku{'t Was als de lang.}{verkropte behoefte van}{een gevangen geest}\\

\haiku{Reeds lang had Richard.}{stil gezonnen op iets dat}{uitkomst brengen zou}\\

\haiku{Want Richard had haar -, - {\textquoteleft}.}{lief ja dat begreep zij nu}{opzijne wijze}\\

\haiku{Ada nam haar hoed af,.}{legde het donkere hoofd}{stil aan zijn schouder}\\

\haiku{Doch op den bodem.}{van dien beker lag voor haar}{de alsemdruppel}\\

\haiku{wat was zij dan in,?}{wezen anders al werd zij}{naar de wet zijn vrouw}\\

\haiku{het stelde haar voor.}{een martelenden twijfel}{omtrent de zijne}\\

\haiku{{\textquoteleft}O, Richard, Richard,?}{waarom doe je jezelf en}{mij dit alles aan}\\

\haiku{een geruisch als.}{sloegen over de hoofden der}{menschen groote vleugels}\\

\haiku{Wanneer zij weten -,}{wil dan moet hij spreken heeft}{hij gesproken zoo}\\

\haiku{Vielleicht daas Unheil.}{dich erwartet Wird aller}{Welt es offen kund}\\

\haiku{Haar liefde moet hem,.}{veel vergelden haar liefde}{moet hem alles zijn}\\

\haiku{Hoe kon zij op den,?}{duur hem boeien als hij zich}{tot haar nederboog}\\

\haiku{Liefde en heimwee,.}{lag daar in haar stem maar niet}{beheerscht door kunst}\\

\haiku{Niet meer zichzelve,,.}{en niet de andere die}{zij wezen wilde}\\

\haiku{Eu nog altijd had,.}{zij hem niet gezien naar wien}{haar ziele smachtte}\\

\haiku{dat zou haar toch een}{feestelijk aanzien geven}{om te huldigen}\\

\haiku{Wat deerde het haar,.}{of hij haar al niet liefhad}{zooals zij had gewenscht}\\

\haiku{Hoe dikwijls had zij....}{zich in haar verbeelding aan}{zijn hals geworpen}\\

\haiku{Daar was de kloof, de.}{duistere spelonk waar zij}{haar licht ging blusschen}\\

\haiku{O, kon zij aan een.}{ieder zeggen hoe zalig}{het was te sterven}\\

\haiku{Langzaam - voet bij voet -,.}{veldwinnend op den zwaren}{den wijkenden nacht}\\

\haiku{En ook dat ging zich,,...}{vervullen wel niet precies}{zooals zij had gedacht}\\

\haiku{aan het eind, bij de,;}{breedoploopende trap}{stond een schuifraam open}\\

\haiku{die op die kleine....}{aardestip uw arm klein}{menschenkind nog ziet}\\

\haiku{Wanneer Elisabeth,}{dien blik ontmoette dan was}{het haar als zengde}\\

\haiku{{\textquoteright} vroeg dringend Ren\'e,.}{die voor niets oogen had dan voor}{dat beeldschoon wezen}\\

\haiku{Die kinderen hier,,!}{hoe lief waren zij voor haar}{hoe aanhankelijk}\\

\haiku{En toch ook nu was,.}{er in haar hart iets dat naar}{verwachten zweemde}\\

\haiku{wenn ich zu w\"unschen,}{wagte hoffen w\"urd ich auch}{zugleich wenn ich nicht}\\

\haiku{Maar daar hoorde zij,;}{haastige voeten die van}{beneden kwamen}\\

\haiku{Nog eenmaal wendde;}{zij haar lief gelaat omlaag}{en lachte hem toe}\\

\haiku{- Waren zij dan nu,?}{niet van elkander zij van}{hem en hij van haar}\\

\haiku{Zij zette haar hoed,.}{op hing een mantel om en}{begaf zich op straat}\\

\haiku{Sprakeloos leidde,.}{Lord Annavon Elisabeth}{den weg naar het Bosch}\\

\haiku{Wij zouden in een.}{schilderachtig oord een oud}{kasteel bewonen}\\

\haiku{Weer boog dat blonde.}{hoofd daar aan de overzij zich}{naar Lord Annavon}\\

\haiku{Ja, zij zou kunnen!}{leven op het geweten}{dragend zulk een dood}\\

\haiku{Zij zat gelijk een,.}{steenen beeld luisterend naar wat}{omging in het huis}\\

\haiku{... word wakker... en wist,.}{niet of het kind gevallen}{was of behouden}\\

\section{Maurice Gilliams}

\subsection{Uit: Elias of het gevecht met de nachtegalen}

\haiku{Maurice Gilliams,.}{Elias of het gevecht met}{de nachtegalen}\\

\haiku{Het is de eerste;}{keer niet dat ik haar van de}{blauwe hand vertel}\\

\haiku{En nu het eenmaal,.}{verontrust was wilde het}{niet meer stilliggen}\\

\haiku{Hij schijnt sterker en.}{mannelijker geworden}{tijdens deze tocht}\\

\haiku{Door het landgoed trekt,.}{onze kleine zwijgende}{stoet in de nacht}\\

\haiku{alleen de gouden.}{geest van het licht woont er nog}{rustig binnen in}\\

\haiku{de kin wrijft over de.}{bloedelooze kreukels van haar}{gevouwen handen}\\

\haiku{Zij draagt een kleed met.}{uiterst enge mouwen en}{witte manchetten}\\

\haiku{Blijkbaar zonder zich,.}{te haasten en toch in een}{omzien is hij weg}\\

\haiku{Hij staart mij aan, en.}{als ik me niet vergis is}{zijn gelaat beschreid}\\

\haiku{Elias, waarom hebt?}{ge uw vingertoppen met}{een naald gefolterd}\\

\haiku{Wanneer ik er een,:}{poos later over nadenk ben}{ik gerustgesteld}\\

\haiku{ik spring recht om de.}{vluchtelinge in de hall}{te achtervolgen}\\

\haiku{het is alsof hij.}{onverrichter zake van}{een eiland thuiskomt}\\

\haiku{Eerst wij jk zachtjes;}{naar binnen fluiten om hem}{welkom te wenschen}\\

\haiku{Ik durf niet zien waar,.}{ze ligt want het ergste zal}{met haar gebeurd zijn}\\

\haiku{mijn beenen worden moe,.}{als moesten ze onzichtbare}{draden voortslepen}\\

\haiku{De schali\"en aan;}{het torentje glimmen als}{met water bevloeid}\\

\haiku{Mijne moeder wordt,}{op eens weer levend en zij}{wandelt weg tusschen}\\

\haiku{tijdens het onweer.}{heeft zij aan het koperen}{kooitje staan prutsen}\\

\haiku{ik kom binnen als,.}{een klein stout kind aan de hand}{van mijne moeder}\\

\haiku{En waarom zie ik,?}{de gezichten niet wat zij}{van mij verwachten}\\

\haiku{Het is nu bijna}{middernacht en ik moet me}{nog wasschen v\'o\'or ik}\\

\haiku{morgen moet ik een.}{zuiver hemd en een frissche}{blouse aantrekken}\\

\haiku{laat ik me op de.}{knie\"en vallen en bespied}{zijn bedoelingen}\\

\haiku{ik voel een warmte,.}{naar het hoofd stijgen die mijn}{hersens geen pijn doet}\\

\haiku{En alsof ik hem:}{zijn voornemen uit het hoofd}{had willen praten}\\

\haiku{zij steekt haar hand in;}{een kous die zij tot aan haar}{elleboog optrekt}\\

\haiku{hij stapt de breede,.}{trap  op naast de portier}{die zijn valies draagt}\\

\haiku{Om mij dadelijk;}{te genezen heeft ze de}{pillen meegebracht}\\

\haiku{Zonder mijn mouw op.}{te stroopen steek ik mijn arm}{diep in het water}\\

\section{Anna van Gogh-Kaulbach}

\subsection{Uit: De hooge toren}

\haiku{{\textquoteright} {\textquoteleft}Nou,{\textquoteright} verdedigde, {\textquoteleft}.}{Henk zijn planze zien nog es}{wat van Sinterklaas}\\

\haiku{dadelijk week 't ':}{weer voor de gedachte aan}{t werk dat wachtte}\\

\haiku{In de bedste\^e werd,,.}{met hooge vogelgeluidjes}{het zusje wakker}\\

\haiku{{\textquoteright} spoorde Henk aan, als.}{was in die groote winkelstraat}{warmte te vinden}\\

\haiku{{\textquoteright} wees Wimpie en Henk,:}{meetrekkend vleide en drong}{zijn schraal stemmetje}\\

\haiku{Toch, in hun verval,:}{bewaarden zij restes van}{vroegere schoonheid}\\

\haiku{Of als vader weer,.}{werk had en moeder thuis kon}{blijven zooals vroeger}\\

\haiku{{\textquoteright} barstte Henk uit en,}{Wimpie loslatend goot hij}{met wild gebaren}\\

\haiku{{\textquoteright} Binnen smeet hij het,.}{boek op de plank veegde met}{zijn mouw over zijn oogen}\\

\haiku{Maar daarom moest zij, '.}{voorzichtig zijn dat zet}{niet merkten van Henk}\\

\haiku{Zoo kon Kees ook doen,.}{als ze vroeger eens naar de}{comedie gingen}\\

\haiku{{\textquoteright} En toen Henk haastig,,,:}{oprees nam ze zijn arm trok}{hem mee fluisterend}\\

\haiku{Met een ruk duwde,.}{hij de zware jassen op}{zij drong naar voren}\\

\haiku{hij voelde zich wee.}{en misselijk en in zijn}{oogen stonden tranen}\\

\haiku{{\textquoteright} gaf Henk terug, trok.}{meteen Wimpie mee in den}{kruidenierswinkel}\\

\haiku{{\textquoteright} zei hij vlug, met het.}{gevoel dat hij iets anders}{had moeten zeggen}\\

\haiku{{\textquoteright} smaalde hij en stak,.}{zijn tong uit naar Henk die snel}{zich had omgekeerd}\\

\haiku{Janus, plotseling,,.}{bevrijd riep een scheldwoord waar}{niemand op lette}\\

\haiku{{\textquoteright} De hevigheid van;}{Henks drift kalmde onder haar}{doen van oudere}\\

\haiku{Alleen Wimpie bleef.}{met Janus spelen en kwam}{soms bij hem in huis}\\

\haiku{Z'n vrouw zat er maar,}{mee die kon zich dood sloven}{en dat arme jong}\\

\haiku{Je vader is ook ',.}{n flinke vent dat hebben}{we altijd gezeid}\\

\haiku{Maar nu, onbewust,.}{voelde hij het vrije huilen}{als iets weldadigs}\\

\haiku{Gelukkig, dat de,.}{zon vandaag schijnt dacht Henk met}{nieuwe verheuging}\\

\haiku{En dat allemaal '.}{omdat ik die smeerlapn}{opzaniker gaf}\\

\haiku{En...{\textquoteright} hij aarzelde,:}{maar toen in eens gooide hij}{toch de vraag eruit}\\

\haiku{{\textquoteright} Ze zalen nog aan,;}{tafel toen er bedeesd aan}{de deur werd geklopt}\\

\haiku{{\textquoteright} En zichzelf in de,:}{rede vallend de oogen op}{de roode tulpen}\\

\haiku{die was tevrejen,.}{als hij wat moois had gemaakt}{dat eeuwen staan zou}\\

\haiku{hij wou niet weg juist,.}{nu in zijn nieuwe blijdschap}{om vaders thuiszijn}\\

\haiku{hij is niet gewoon,.}{om der vroeg in te leggen}{net zoo min as ik}\\

\haiku{{\textquoteright} Maar de vrouw, nu gansch,:}{zich gaan latend stootte uit}{door haar huilen heen}\\

\haiku{even - om Jen jongen -.}{gerust te stellen knipte}{hij met de leden}\\

\haiku{Over Vermeers gelaat.}{sloeg een heete driftblos en}{zijn oogen werden hard}\\

\haiku{Jans zonk terug in,:}{haar stoel en uitbarstend in}{huilen kermde zij}\\

\haiku{Moe was al naar bed,,}{maar ze had uit de alkoof}{geklaagd dat ze zoo}\\

\haiku{{\textquoteright} Vermeer keerde even,.}{zijn hoofd naar de slapende}{knikte tegen Henk}\\

\haiku{Nogal wonder, as!}{je de heele dag langs de}{straat mot schooieren}\\

\haiku{Nu lachten ze weer,.}{met hun drie\"en om haar dacht}{zij verdrietig}\\

\haiku{{\textquoteright} Hier dacht Vermeer weer.}{aan onder het loopen in}{de drukke marktstraat}\\

\haiku{Je mot heel vast in '.}{je schoenen staan omt vol}{te kennen houwen}\\

\haiku{{\textquoteright} Zij week iets voor hem,.}{op zij sloot de deur toen hij}{in de kamer was}\\

\haiku{Toen, terwijl zij het,:}{servet opvouwde zei ze}{ietwat aarzelend}\\

\haiku{t verdragen 'n,?}{schooier te blijven nou je}{goed werk ken krijgen}\\

\haiku{'t leek Vermeer dat.}{Jans nog bij geen bevalling}{z\'o\'o geleden had}\\

\haiku{Nu al maakte de {\textquoteleft}{\textquoteright} '.}{gedachte aan zijnneent}{moeilijker voor haar}\\

\haiku{Vermeer trachtte het.}{te onderscheiden in het}{weifelende licht}\\

\haiku{Jans deed de oogen open,.}{knipte ze even dicht als om}{Henk toe te knikken}\\

\haiku{{\textquoteright} Met zijn vuile hand,.}{streelde hij haar voorhoofd liep}{toen naar zijn vader}\\

\haiku{Tusschen hen was geen;}{woord meer gesproken over het}{voorgevallene}\\

\haiku{Zij schokte even met.}{de schouders en wendde haar}{gezicht naar bem toe}\\

\haiku{Op de hoeken van}{de straten groepten menschen}{samen en luchtten}\\

\haiku{Dus nu was hij er,,;}{de oorlog die ze al zoo}{lang voorspeld hadden}\\

\haiku{de kruijenier in de.}{Nieuwsteeg zou wat gort voor me}{bewaren en meel}\\

\haiku{{\textquoteright} Zijn oogen vingen een,:}{blik van Henk en sterker nog}{het hoofd geheven}\\

\haiku{Vermeer stond met Bosch;}{en Vermaas in een  hoek}{van het podium}\\

\haiku{Maar nou... verdom ik ',!}{t wat met die rotzooi te}{maken te hebben}\\

\haiku{In een flits herzag;}{Henk wat hij wist van moeders}{moeitevol leven}\\

\haiku{Bij 't hek van de,:}{textiel-fabriek waar zij}{werkte zei hij nog}\\

\haiku{de jongen deed haar,.}{nu telkens denken aan Kees}{zooals die vroeger was}\\

\haiku{Als een gerucht van.}{heel ver klonk achter hen het}{rumoer van de markstraat}\\

\haiku{{\textquoteleft}honderd doojen,{\textquoteright} {\textquoteleft}De,{\textquoteright}.}{oorlog kost er miljoenen}{gaf Vermeer terug}\\

\haiku{{\textquoteright} Vermeer legde even,.}{zijn hand op die van Henk die}{op de tafel lag}\\

\haiku{Ja, dammen ken je!}{en domineeren tot je der}{misselijk van bent}\\

\haiku{{\textquoteleft}En de andere.}{jonge kerels laten ze}{mekaar doodschieten}\\

\haiku{De luchtplaats lag als;}{een diepe kom tusschen groen}{begroeide wallen}\\

\haiku{Even blikte hij om.}{zich heen in de lachende}{jonge gezichten}\\

\haiku{Henk hoestte, droog en,;}{pijnlijk zooals hij was blijven}{doen na zijn ziekte}\\

\haiku{{\textquoteright} {\textquoteleft}En je berust er.}{in en weet tenslotte niet}{eens meer wat je mist}\\

\haiku{{\textquoteright} {\textquoteleft}Waarom berustte?}{hij der dan ook niet in om}{soldaat te worden}\\

\haiku{Hij gooide zich om,.}{op zijn stroozak vocht tegen een}{nieuwe hoestbui}\\

\haiku{Een schok van vreugde:}{voer door hem heen bij Vermeers}{plotselinge vraag}\\

\haiku{Hij streek zich over zijn,,.}{knevel nam de courant op}{begon te spreken}\\

\haiku{Maar 'n mensch wist soms '.}{niet meer waar hijt zoeken}{moest tegenwoordig}\\

\haiku{dat er een eind kwam.}{aan de beklemming van het}{gevangen-zijn}\\

\haiku{de mogelijkheid,;}{is niet uitgesloten dat}{hij verdronken is}\\

\haiku{Zwijgend kauwden zij,:}{het brood karig besmeerd met}{margarine}\\

\haiku{Vermeer rilde even.}{toen hij een slok nam uit de}{tinnen koffiekruik}\\

\haiku{hij v\'o\'or het Dagblad,;}{het bulletin waar Janssen}{over had gesproken}\\

\haiku{{\textquoteright} Wim knikte, zijn oogen.}{in kinderlijke peinzing}{voor zich uit starend}\\

\haiku{toen schoot Jans vooruit,;}{naar den ingang waar zij den}{schildwacht ontwaarde}\\

\haiku{{\textquoteright} de jongen had zijn.}{opstandigen geest blijkbaar}{niet van een vreemde}\\

\haiku{maar dan is er de.}{troost dat zij gevallen zijn}{op het veld van eer}\\

\haiku{Tegen Jans zei hij;}{niet hoe hijzelf alle hoop}{had opgegeven}\\

\haiku{Wim, begrijpend dat,;}{Henk dood was drong zich huilend}{tegen moeder aan}\\

\subsection{Uit: Jeugd}

\haiku{daarvoor ga je laat!}{na bed en zit je morgen}{op school te suffen}\\

\haiku{Toen de trein stil stond,,;}{stapte ze vlug uit liep met}{de anderen mee}\\

\haiku{{\textquoteright} {\textquoteleft}Ik ben anders zoo.}{gewend dat die lieve oogen}{me even ankijken}\\

\haiku{Ze liep in eens een,;}{zijstraatje in waar een paar}{meisjes aankwamen}\\

\haiku{ze liep de gang door,,;}{de huiskamer binnen waar}{ook al licht brandde}\\

\haiku{{\textquoteleft}Dat kan zijn, maar 't...}{past toch niet voor een meisje}{zooiets te vragen}\\

\haiku{{\textquoteright} Guust ging weer aan 't, ':}{rekenen maart ging nog}{moeielijker dan straks}\\

\haiku{je moet het zien met,.}{eigen oogen het voelen als}{een deel van je zelf}\\

\haiku{het papier was in, '...}{zijn zak maar hij durfdet}{haar haast niet geven}\\

\haiku{{\textquoteleft}Maar je moet denken, '.}{ik beschrijf niet enkel wat}{k mooi vind of goed}\\

\haiku{ze zou zoo graag haar,.}{moeder trotseeren maar ze}{durfde toch niet goed}\\

\haiku{{\textquoteright} {\textquoteleft}Dan doe je 't maar,.}{zonder zin je moet naaiwerk}{ook doen op z'n tijd}\\

\haiku{Nou ja, dat zijn van,.}{die bijzonderheden die}{haast nooit voorkomen}\\

\haiku{{\textquoteright} {\textquoteleft}Zou ze weer zooveel?}{te vertellen hebben over}{die neef uit Indi\"e}\\

\haiku{{\textquoteright} {\textquoteleft}Boven geloof 'k,,.}{ze was pas klaar met der werk}{daar komt ze net aan}\\

\haiku{Ze voelde altijd.}{weinig belangstelling voor}{zulke schandaaltjes}\\

\haiku{{\textquoteright} {\textquoteleft}Ja, ze zouen je,{\textquoteright}.}{maar exploiteeren viel de}{postdirecteur in}\\

\haiku{{\textquoteright} {\textquoteleft}Nee, Germinal is ',{\textquoteright}.}{t beste zei \'e\'en van de}{jongelui ernstig}\\

\haiku{hij wou iets doen wat,,}{ze zou bewonderen een}{kunstwerk scheppen z\'o\'o}\\

\haiku{Heil, heil, ik voel haar.....}{handen en den week en boog}{Van haren arm}\\

\haiku{Ze vond haar moeder, '.}{in de huiskamer waart}{al schemerig was}\\

\haiku{het volgende was,,;}{hooiland daar stond het gras hoog}{wachtend op de zeis}\\

\haiku{De tilbury was,:}{een zandweg ingedraaid waar}{het paard stappen moest}\\

\haiku{Guust trok haar arm door;}{de zijne en zoo liepen}{ze een eindje voort}\\

\haiku{hou je nou z\'o\'o maar,,?}{van me zonder dat ik wat}{ben wat gedaan heb}\\

\haiku{Haar moeder bleef haar,,:}{even aankijken ze nam haar}{hand zei bedarend}\\

\haiku{Z'n vader wil 't, '.}{immers ook die ziett maar}{w\`at verstandig in}\\

\haiku{En 't is voor z'n,.}{eigen geluk later zal}{ie je dankbaar zijn}\\

\haiku{vreemden bleven er;}{nu zoo ver van af met hun}{kijken en praten}\\

\haiku{Hij bleef in zijn boek, '.}{kijken om te laten zien}{datt hem ernst was}\\

\haiku{de groote massa dreef,.}{verder en waar hij langs streek}{bleef iets vaals achter}\\

\haiku{Of ja, ze moesten haar,.}{nemen anders verdiende}{ze immers geen geld}\\

\haiku{{\textquoteleft}Zeg vent,{\textquoteright} zei ze in,, {\textquoteleft},;}{eens naar hem opkijkendik}{vin dat je bleek ziet}\\

\haiku{Den laatsten tijd had:}{hij met Leida wel eens over}{godsdienst gesproken}\\

\haiku{{\textquoteleft}Vervloekt,{\textquoteright} mompelde,.}{hij woedend sloeg met zijn stok}{heftig op de steenen}\\

\haiku{{\textquoteleft}Ik hoop maar, dat Guust,{\textquoteright};}{eenvoudig zal blijven zei}{mevrouw Heerling weer}\\

\haiku{{\textquoteleft}hij krijgt den laatsten.}{tijd zoo iets verwaands over zich}{van jong studentje}\\

\haiku{{\textquoteleft}Ja, we zullen ons,.}{geluk veroveren een}{mooi leven hebben}\\

\haiku{Ze schrikte even, in.}{eens vooruitziende den strijd}{in vollen omvang}\\

\haiku{'t Was een zalig,.}{stil loopen een mooie rust na}{den mooien dag}\\

\haiku{hij wenschte weer,}{dit te leeren kennen nu niet}{alleen om te zien}\\

\haiku{{\textquoteright} 't Ontglipte haar,.}{op een toon van spijt die ze}{niet bedwingen kon}\\

\haiku{Maar evenmin kan ik:}{afstand doen van wat ik zie}{als m'n levenstaak}\\

\haiku{Toen Guust binnenkwam,,:}{keek hij op verbaasd en op}{zijn knorrigen toon}\\

\haiku{Hij zat luisterend,.}{naar den storm stilwachtend met}{zijn oogen op de deur}\\

\haiku{we hebben eenmaal,.}{bepaald dat alles tusschen}{jullie uit moet zijn}\\

\haiku{{\textquoteleft}och jongen, je haalt;}{jezelf en haar  zooveel}{verdriet op je hals}\\

\haiku{En de redactie,:}{van de courant rekende}{er op natuurlijk}\\

\haiku{'t Is wel aardig, ', '.}{n beetje droog bij alt}{nat dat je beschrijft}\\

\haiku{zijn hand vloog over 't,,.}{papier hij voelde geen kou}{meer geen moeheid ook}\\

\haiku{In den warmen trein ':}{gaft rhytmisch zacht schokken}{hem iets soezerigs}\\

\haiku{Nu zag ze hem, nu,;}{leefde haar gezicht op nu}{lachten haar oogen blij}\\

\haiku{wie weet hoe u uw,{\textquoteright}.}{hart nog op kunt halen aan}{vorst zei Willemien}\\

\haiku{Van Staaren reikte hem,.}{zijn twee handen drukte de}{zijne onstuimig}\\

\haiku{thuis trok hij zijn kleeren,,.}{uit liet ze liggen op den}{grond viel op zijn bed}\\

\haiku{We hebben misschien,.}{te veel geloopen je kwam}{om uit te rusten}\\

\haiku{Ik ben wat moe van ', '.}{t werken maart zal nu}{wel gauw beter zijn}\\

\haiku{{\textquoteright} {\textquoteleft}'N ei, nee juffrouw, ',.}{k heb der geen \'e\'en meer en}{dat gaot ook niet}\\

\haiku{ze namen hem mee,, '}{naar vergaderingen en}{gaven hem lectuur}\\

\haiku{hij was al jaren,.}{lang in de beweging gaf}{er al zijn tijd aan}\\

\haiku{ze had er dan toch '.}{bovendien nogt geluk}{bij van haar liefde}\\

\haiku{Ik heb heusch niet.}{genoeg om twee huishouwens}{te onderhouwen}\\

\haiku{Ze deed haar best, kalm,,,.}{gewoon te spreken stond op}{trok haar japon recht}\\

\haiku{Heerling trok haar arm,,:}{door de zijne begon toen}{langzaam voortloopend}\\

\haiku{Hij hoeft 't toch niet '?}{met alle personen uit}{t boek eens te zijn}\\

\haiku{{\textquoteleft}Kom,{\textquoteright} zei Willemien, {\textquoteleft} '!}{bedarendlees u maar eerst}{t heele boek uit}\\

\haiku{Guust voelde 't bloed,.}{opstijgen in zijn gezicht}{maar hij bleef toch kalm}\\

\haiku{'k Moet zeggen, dat '!}{je vadern prettige}{ouwe dag bezorgt}\\

\haiku{Ik vond er veel moois, '.}{in maark kan niet alles}{met je meevoelen}\\

\haiku{, wees heelemaal van,,.}{me leef mijn leven mee dan}{zal je leeren verstaan}\\

\haiku{schulden heb 'k niet,, ', '.}{toe laten wet probeeren}{v\'o\'ort te laat is}\\

\haiku{En aan dit boek kan',,,.}{k niets veranderen}{niets niets geen letter}\\

\haiku{En ook omdat strijd.}{eenmaal de natuurlijke}{en eenige weg is}\\

\haiku{{\textquoteleft}Schat,{\textquoteright} fluisterde hij, {\textquoteleft} '....?}{vlak bij haar oorje kuntt}{toch wel begrijpen}\\

\haiku{Ze stak haar hand uit,.}{als om afscheid te nemen}{hij drukte die vast}\\

\subsection{Uit: Het rijke leven}

\haiku{En vertel nu eens,.}{hoe alle kennissen in}{Boschvoort het maken}\\

\haiku{Je schijnt me niet te,,:}{willen begrijpen maar ik}{weet wat ik bedoel}\\

\haiku{ze liet er zich  ,.}{door omvangen zonk er in}{neer zonder denken}\\

\haiku{dat is nu al het,,.}{derde goede aanzoek dat}{je afslaat voor niets}\\

\haiku{Willy had geen hoed,.}{op liet den zoelen wind met}{haar haren spelen}\\

\haiku{Ze waren nu bij.}{de eerste huizen van het}{stadje gekomen}\\

\haiku{{\textquoteleft}Heeft u geen lust met?}{de dames eens naar het werk}{te komen kijken}\\

\haiku{Over het water hing,;}{een lichte nevel als een}{vochtige sluier}\\

\haiku{{\textquoteleft}Ik heb je al eens,.}{gezegd welke hooge eischen}{ik aan vriendschap stel}\\

\haiku{{\textquoteright} {\textquoteleft}Neen, zeg dat niet,{\textquoteright} viel, {\textquoteleft}.}{Willy driftig uithij heeft}{leelijk gehandeld}\\

\haiku{Mama begrijpt me,;}{niet en jammert alleen over}{het verloren geld}\\

\haiku{den dood van haar man,:}{maar deze had haar niet hard}{gemaakt of morrend}\\

\haiku{Ik moet u iemands,{\textquoteright}.}{groet brengen vervolgde hij}{na een oogenblik}\\

\haiku{'t Bloed vloog Willy ',.}{naart gezicht zelfs haar hals}{en voorhoofd kleurend}\\

\haiku{haar hart bonsde, in.}{haar gezicht waren vreemde}{zenuwtrekkingen}\\

\haiku{dit was dus alles.}{wat zij hem te zeggen had}{van haar verleden}\\

\haiku{Zou je werkelijk;}{niet kunnen gelooven in de}{liefde van dien man}\\

\haiku{het zich geven van;}{de vrouw aan den man scheen haar}{plotseling laag toe}\\

\haiku{ze zou George,;}{niet meer zien haar geluk zou}{voor altijd weg zijn}\\

\haiku{De kamer zag er,;}{zoo vreemd uit geheimzinnig}{in het schemerlicht}\\

\haiku{{\textquoteleft}Toe mama,{\textquoteright} begon, {\textquoteleft};}{ze weer na een oogenblik}{vraag me nu niet meer}\\

\haiku{{\textquoteleft}je hebt mij immers,.}{zelf gevraagd net te doen of}{er niets gebeurd was}\\

\haiku{Nu echter begreep, ',;}{hij dat d\`att was waardoor}{Willy zoo vreemd deed}\\

\haiku{toen hij weer sprak van;}{de biljartkamer zei ze}{als in gedachte}\\

\haiku{Het zou haar nu ook,;}{niets kunnen schelen al zag}{ze een kind doodslaan}\\

\haiku{{\textquoteright} {\textquoteleft}Dat weet ik juist niet, '.}{en u moet me helpen om}{t te bedenken}\\

\haiku{De ijzerfabriek;}{van den heer Dryfel lag een}{eind buiten Arnhem}\\

\haiku{toen was ze nog jong, ';}{na{\"\i}ef geloovend int}{mooie van het leven}\\

\haiku{{\textquoteright} {\textquoteleft}Weet je wat, kind, je,.}{moest oom en tante zeggen}{dat klinkt prettiger}\\

\haiku{Ze keek peinzend naar;}{zijn open jongensgezicht met}{de ernstige oogen}\\

\haiku{voor hen allen is,;}{meer arbeid te vinden dan}{ze af kunnen doen}\\

\haiku{Toch ben ik dikwijls,;}{bang voor Gerard nog minder}{dan later voor Louis}\\

\haiku{beloofde ze dien;}{winter een paar maanden thuis}{te zullen komen}\\

\haiku{Misschien is 't wel,,;}{goed als je daar dikwijls aan}{denkt zei ze langzaam}\\

\haiku{Die gedachte riep:}{hem een gezegde van zijn}{vader te binnen}\\

\haiku{{\textquoteleft}Ik zag gisteren,,}{op de tram aan je dat je}{zoo veranderd was}\\

\haiku{Wardorf was wel geen,;}{rijke partij maar toch heel}{goed aannemelijk}\\

\haiku{Gerard schreef haar een,,}{langen brief waarin hij wel}{tienmaal zei hoe blij}\\

\haiku{Maar ze moest toch nog,.}{eens aan hem denken hij kon}{haar niet meer missen}\\

\haiku{'t Begon haar te,.}{kwellen maar ze wist er niet}{van te beginnen}\\

\haiku{'t Scheen ook zoo dwaas,.}{dat ik beter wilde zijn}{dan de anderen}\\

\section{Paula Gomes}

\subsection{Uit: Sudah, laat maar}

\haiku{Wij worden vermoord.}{en Sonja verdwijnt in een}{huis voor geisha's}\\

\haiku{Tegen de ochtend.}{zagen we de stoet in de}{verte naderen}\\

\haiku{Je kon immers nooit,.}{weten er werden telkens}{mensen weggehaald}\\

\haiku{Ik had de kleine,.}{witte bloemetjes in het}{maanlicht zien prijken}\\

\haiku{Djongossen, nee, dat,.}{woord was taboe geworden}{pelayans die klaarstonden}\\

\haiku{Erna gaf mij een,.}{paarse steen mee die ik in}{mijn hand moest houden}\\

\haiku{Ik keek hoe iemand.}{eruitzag die een ander}{ging mishandelen}\\

\haiku{Mijn celgenoten.}{wisten het voor ik zelf op}{de binnenplaats kwam}\\

\haiku{De gevangene,.}{stond erbij de handen op}{de rug gebonden}\\

\haiku{Sylva en Erna.}{hielpen met het oprollen}{van de matrassen}\\

\haiku{We werden naar het.}{kamp gebracht dat eigenlijk}{al was opgedoekt}\\

\haiku{Een paar keer in de.}{week bestond het uit rijst met}{gebakken larons}\\

\haiku{We wachtten met z'n,,.}{allen in die ene kamer}{stiller dan de dood}\\

\haiku{Ik moest eerst het hek,.}{door langs de schildwachten met}{de bajonetten}\\

\haiku{Toen hij eindelijk,.}{afremde stonden we voor}{een gesloten huis}\\

\haiku{Alleen dat we in.}{Surabaja zaten toen}{de oorlog begon}\\

\haiku{Hierbij hoorden nu,.}{ook de doden de doden}{in het dodenrijk}\\

\haiku{Na de atoombom had.}{hij nog niets van zijn vrouw en}{kinderen gehoord}\\

\haiku{In een b\`etjak.}{zocht ik de plekjes op die}{mij bekend moesten zijn}\\

\haiku{Maar toen we bij het,.}{huis kwamen zag ik dat ze}{er niet meer woonden}\\

\haiku{je zag alleen de,.}{banden zonder dat je de}{titels kon lezen}\\

\haiku{De djongos zou me,.}{voorlopig naar zijn zuster}{brengen naar Sitih}\\

\haiku{Vannacht toen ik hem;}{naar buiten liet is hij niet}{teruggekomen}\\

\haiku{Niet alleen dat hij,.}{geen schoenen droeg in zijn hemd}{zaten slijtgaten}\\

\haiku{Indonesische.}{militairen liepen de}{huizen in en uit}\\

\haiku{Ik was verbaasd, maar.}{volgde hem naar de tuin tot}{bij het fietsenrek}\\

\haiku{Mijn grootvader was,{\textquoteright}.}{een Hollander vertelde}{een hotelportier}\\

\haiku{Legertrucks kwamen.}{de Hollanders overal in}{de stad ophalen}\\

\haiku{Ze zagen me niet,.}{staan verscholen tussen de}{struiken in de tuin}\\

\haiku{Die gaf het weer door.}{naar de volgende en zo}{de hele weg langs}\\

\haiku{Ze kreunde in haar.}{slaap na de injectie die}{ze gekregen had}\\

\haiku{Of dat ze zelfs maar.}{uit beleefdheid zeiden dat}{we welkom waren}\\

\haiku{Zo zat ik weer met.}{een heleboel mensen bij}{elkaar in een kamp}\\

\haiku{De mannen strekten.}{hun armen naar het licht en}{wiegden heen en weer}\\

\haiku{Misschien is voor dit.}{inzicht een soortgelijke}{ervaring nodig}\\

\section{Leon Gommers}

\subsection{Uit: Het uurwerk van Floor}

\haiku{Als ik dan zomaar:}{van het bed omhoogkom zie}{ik mezelf twee keer}\\

\haiku{De Dodenweg loopt.}{tussen die zilverig licht}{spattende kuilen}\\

\haiku{Ik vergeet, nog steeds,}{een beetje onrustig ik}{vergeet niet hoe laag}\\

\haiku{Ik knik dan zo lang.}{omdat hij daarop net zo}{lang terugtwinkelt}\\

\haiku{Ken je die van de?}{televisiekijkers op}{het Drielandenpunt}\\

\haiku{Soms zegt Floor dingen,.}{zonder dat ik hem iets vraag}{en dat is iets nieuws}\\

\haiku{De stiekjes rond een.}{kleurpotlood waar je soepel}{mee kunt trommelen}\\

\haiku{ze noemen het in.}{Duitsland natuurlijk niet voor}{niks een Rummelplatz}\\

\haiku{De rijtjeshuizen.}{verdwijnen in het eerste}{groen van de bomen}\\

\haiku{Dat schroef je bij dat,...}{paar gewone wijzers van}{zes uur Borstelkop}\\

\haiku{Ik probeer met een.}{vinger hoe koel een messing}{traproede kan zijn}\\

\haiku{Zijn duim gaat omhoog}{en deze bal komt niet zo}{hard en  vliegt aan}\\

\haiku{mijn hoofd terwijl ik.}{een vallende warrel van}{botten en vlees ben}\\

\haiku{het is en blijft met:}{jou hetzelfde als in de}{rest van de wereld}\\

\haiku{Broer heeft het stuur als.}{een vrijgezel vast en rukt}{aan het stuur en slipt}\\

\haiku{Dan stokt oma's gang.}{en de straat wordt drukker en}{zenuwachtiger}\\

\haiku{matrone vind ik,}{nog wel leuk maar mijn vader}{weet allang niet meer}\\

\haiku{boos naar Floor op de.}{grond en tussen de bomen}{van ons klokkenbos}\\

\haiku{Evenmin hou ik van.}{spelletjes waar veels te veel}{beweging in zit}\\

\haiku{De begerige.}{vingers van twee dringende}{neefjes gaan omhoog}\\

\haiku{Het zijn speelkaarten.}{en ze zijn er ook in het}{nachtblauw natuurlijk}\\

\haiku{Ik kijk dan door het}{zwarte raam naar de sterren}{buiten en dan hoor}\\

\haiku{, en Otto is een,.}{oude schoonzoon ongerust}{naar Anton kijken}\\

\haiku{En dan wachten tot.}{het uurwerk in een enkel}{ogenblik tot stand komt}\\

\haiku{Floors opengeknipte.}{scheermes hangt op hoogte van}{het blauwe schaamplekje}\\

\haiku{En als we snijden,.}{zijn wij trots en voelen niets}{van onze schaamte}\\

\haiku{Floor heeft het mes plat.}{op het vlees van de wang en}{tegen de oogkas}\\

\section{H.A. Gomperts}

\subsection{Uit: Intenties I, Kritieken en over kritiek}

\haiku{Eerst moet ik je in.}{de rede vallen om je}{gelijk te geven}\\

\haiku{5 Wrijving tussen.}{kunstenaars en publiek is}{geen nieuw verschijnsel}\\

\haiku{'Ik ben niet Cinna,.'}{de samenzweerder ik ben}{Cinna de dichter}\\

\haiku{De schrijver moet dus.}{inderdaad soms niet op zijn}{woord geloofd worden}\\

\haiku{Hetzelfde geldt voor,.}{geloof levensbeschouwing}{en filosofie}\\

\haiku{Later verschijnt de:}{vrouw bij Hemingway nog slechts}{in twee gedaanten}\\

\haiku{De vis is te groot.}{voor zijn bootje en daarom}{bindt hij hem langszij}\\

\haiku{Hij is discreter.}{en terughoudender dan}{de echte schrijver}\\

\haiku{Zij schrikt echter van,.}{de wildheid van zijn wanhoop}{kust hem en loopt weg}\\

\haiku{Zij bevatten geen.}{puzzel die uit het leven}{ge{\"\i}soleerd is}\\

\haiku{Poesjkin vertoont:}{allerlei kenmerken van}{de romanticus}\\

\haiku{Ik geloof dat men.}{hem onrecht doet door hem zo}{te klasseren}\\

\haiku{Na de dood van zijn ''.}{moeder schrijft Toergenjew het}{verhaalMoemoe}\\

\haiku{'Op een dag zullen.}{wij achter ons huis zitten}{om thee te drinken}\\

\haiku{Hij probeert het toch.}{en de inspanning bezorgt}{hem een attaque}\\

\haiku{Het probleem van de.}{psychologische waarheid}{bestaat voor hem niet}\\

\haiku{Hij is eerder een.}{losbandige figuur dan}{een wellusteling}\\

\haiku{Nu is hij dus weer.}{op die laatste verklaring}{teruggekomen}\\

\haiku{De heren van dit.}{landgoed doen niet veel anders}{dan eten en drinken}\\

\haiku{waaruit volgt dat men.}{nooit orthodox moet zijn in}{kwesties van smaak}\\

\section{Sam Goudsmit}

\subsection{Uit: Zoekenden}

\haiku{ze zagen, wie er,;}{op visite kwam en ze}{begrepen waarom}\\

\haiku{Och ja, Moeder, det,.}{vleesch kump wel ga\`ar zonder}{o\`e goa oe g\`ang maar}\\

\haiku{as ze de koppen, ',....}{bij mekare steken dan}{giett over mien d\`ah}\\

\haiku{zeg ze, och mensche, ' ' '.}{ik adder graag mien bord}{inr snoeteegooid}\\

\haiku{bah, 'k zol mien toch, '?}{mien oogen uut mien heufd schamen}{hef ze d\`etezegd \`e}\\

\haiku{{\textquoteright} {\textquoteleft}Ze mut zekers booles ',{\textquoteright}.}{alen veur Sjabbes probeerde}{Meijer te spotten}\\

\haiku{nou ebben we al....}{twintigduizend gulden schuld}{onder de boeren}\\

\haiku{{\textquoteright} riep Naatje, op den, {\textquoteleft} '!}{grond voor hem uitspugendwat}{n minne kerel}\\

\haiku{{\textquoteright} Sam zette 't luik '.}{voort winkelraam en liet}{de gordijnen neer}\\

\haiku{Maar enkel v\'o\'or zich....}{wou-ie zien in z'n boek en}{denken aan Sjabbes}\\

\haiku{baron de Nekoome, ',....}{of hoe heet die potsneus}{isn geschiewes}\\

\haiku{{\textquoteright} {\textquoteleft}Goa oe gang maar, ogod,,{\textquoteright}.}{h\`ee kreunde Vader van de}{ledenverwringing}\\

\haiku{Waarom kon-ie geen'?}{moppen krijgen en die hond}{van  n Moos wel}\\

\haiku{dat tuig wou zeker.}{voor tien lappies van honderd}{nog een graaf hebben}\\

\haiku{was 'n aardige... ' '....}{roestige sjiddesch ewestn}{sjiddisch metn luchien}\\

\haiku{maar moest die h\`ond 'm?....}{dan n\`ou ook vragen om die}{vervloekte centen}\\

\haiku{{\textquoteleft}hij vertrapte 't, ',{\textquoteright}.}{h\`aar kastanjes uitt vuur}{te halen zei-ie}\\

\haiku{nou,{\textquoteright} zei-ie, {\textquoteleft}ie 'em....}{ook al genog meeemaakt in}{d\`eze zeuventig}\\

\haiku{waar-ie niet an 't,?}{bod dan op det perceel an}{de Laragediek}\\

\haiku{wij ons kapitaal,}{beter in onze z\`ake}{gebruukenn doar ebbe}\\

\haiku{t kan ze t\`och niks,....}{verdommen waj koopen of}{niet koopen waj \`em}\\

\haiku{Rozette, dichter,:}{naar Naatje geschoven zag}{naar haar en Jette}\\

\haiku{met 'n schort voor kon ' ';}{zer met plezier w\`at graag}{n beetje helpen}\\

\haiku{Zij scheen ongewoon:}{beminnelijk vandaag aan}{Naatje en Jette}\\

\haiku{Ja, ze moest naar huis,, '.}{want die ze n\`ou weer had die}{snoeptet huis leeg}\\

\haiku{Goddank, as je geen,,,.}{meid noodig had zooals zij juffrouw}{Beem die Jette had}\\

\haiku{bah, wat had ze toch ';}{eigenlijk een hekel an}{t heele zoodje}\\

\haiku{David en Kobus.}{Koopmans en de amechtige}{Davids met z'n vrouw}\\

\haiku{{\textquoteleft}Ik zol zoo'n soten '.}{int zwart werachtig niet}{binnenloaten}\\

\haiku{{\textquoteright}, vroeg hem Moos in de', {\textquoteleft} '?}{Vries Brabantsch accentvinde}{t soms ni\`et mooi}\\

\haiku{in geen drie weken....}{had-ie een stukje vleesch}{an de haak gehad}\\

\haiku{de jonge Rebbe,,.}{het kerkbestuur met Mr. van}{Lier voorbijrenden}\\

\haiku{en as me dan an, '....}{iemand k\`ent dan weet me toch}{wat mer  ge\`eft}\\

\haiku{Gij hebt gezien dat,.}{er vreugde dat er blijdschap}{is in dit leven}\\

\haiku{als gij eenmaal zult;}{gekomen zijn aan de grens}{van deze woestijn}\\

\haiku{{\textquoteright} vroeg Moos naar De Beer,;}{en z'n vrouw die alleen nog}{gebleven waren}\\

\haiku{Joop, nou tegenover,.}{de sterk gemaakte zaak moest}{er totaal onder}\\

\haiku{ik kan mien doar van,:}{zelf niet mee bemujen}{maar d\`et weet ik wel}\\

\haiku{hij zal zien k\"ossien,;}{wel opscharrelen wees do\`ar}{maar verzekerd van}\\

\haiku{{\textquoteright} besloot Moos, {\textquoteleft}ja, ze,}{bin nog al zoo lekker de}{knechten um lief veur}\\

\haiku{wat heb 'k \'um oe, '!}{heen edreid net zoo lange tut}{ask vrij ware}\\

\haiku{'k Kan wel zien, da'j ',.}{t verliezen anders zo'j}{niet um thee denken}\\

\haiku{a'j speulen willen,,.}{dan speul-ie maar ik kan mien}{ier best verm\`aken}\\

\haiku{Is d{\`\i}t nu de stad,?}{die heet een zoo heerlijke}{schoonheidsvolbouwing}\\

\haiku{Die, schuw, was tusschen.}{de doorloopen naar Joede}{Rosenstein gegaan}\\

\haiku{minachtend had hij,;}{naar hen neergezien snauwend}{hen toegesproken}\\

\haiku{Lion met zijn pet,.}{af Joede altijd onder}{z'n groenigen hoed}\\

\haiku{{\textquoteright} {\textquoteleft}Nou, dan goa we maar,{\textquoteright}, {\textquoteleft}.}{zei Lionik blieve toch}{maar hier sloapenn}\\

\haiku{urn mien niet bij 't?}{bedde van mien Vader te}{willen loaten}\\

\haiku{k heb zorrege,....}{genog an mien kop ik kan}{oe niet hellepen}\\

\haiku{ik geleuve well....}{dat oe warrek hier nou is}{of-eloopenn}\\

\haiku{{\textquoteright} {\textquoteleft}Graag,{\textquoteright} zei Joop kort en, {\textquoteleft}.}{stond opMorgen za'k mien geld}{wel komen halen}\\

\haiku{Hij merkte het niet,;}{hoe hij overal \`om zich de}{teleurstelling vond}\\

\haiku{Zij gingen de straat,.}{op lieten h\`aar thuis om den}{boel te verzorgen}\\

\haiku{ewest urn de boel in....}{orde te maken en to\'e}{was jullie Sam egoan}\\

\haiku{was Joop, dezelfde,,....}{Joop soms niet met dreug brood noa}{schoole egoan ja of ne\'e}\\

\haiku{Ja, 't ging goed zoo,, '.}{hij kon gauw goan bedelen}{ast zoo voortging}\\

\haiku{{\textquoteright} {\textquoteleft}Goeienoavond,{\textquoteright} zei Joop:}{binnensmonds en dadelijk}{zachter tot Grietje}\\

\haiku{{\textquoteright} Hartog zat even te ',.}{denken overt voorstel dat}{ie Joop dacht te doen}\\

\haiku{Morren mun ze mien,....}{beloven dat ze niks meer}{zullen verlangen}\\

\haiku{{\textquoteright} {\textquoteleft}De's ofgespreuken,{\textquoteright}, {\textquoteleft} '....}{besliste Hartogen n\`ou}{goak noa huus toe}\\

\haiku{hellep mien tenminsten ' '....}{ann paar centen umn}{koegien te koopen}\\

\haiku{De boeren kregen.}{hun geld dan niet en gaven}{geen tweede stuk vee}\\

\haiku{{\textquoteleft}Ja, det kon 'k wel.......}{dadelijk denken meneer}{toe'j van lo\`od sprakken}\\

\haiku{{\textquoteright} {\textquoteleft}Moet jij zingen,{\textquoteright} riep,:}{Naatje en zachter tot den}{violist meteen}\\

\haiku{Kalm moest-ie voort, straks,....}{het huis met hem door en hem}{br\`engen naar het lood}\\

\haiku{dat er toch niks w\`as, '.}{datt onzin was om mee}{naar binnen te gaan}\\

\haiku{t Is in orde,{\textquoteright}, {\textquoteleft}.}{zei de rechercheurik leg}{beslag op dat lood}\\

\section{Johan Graafland}

\subsection{Uit: Van toen en thans}

\haiku{{\textquoteleft}Manne{\textquoteright}, begon hij, {\textquoteleft},.}{weerjullie bevalle me}{jullie kletse niet}\\

\haiku{Onmiddellijk liep '.}{n soldaat op hem toe om}{zijn pas te vragen}\\

\haiku{Schoppenheer rolde,,.}{neer morsdood kogel dwars door}{de halsslagaders}\\

\haiku{Toen hebben wij Haar:}{spreuk op Haar energiek gelaat}{belichaamd gezien}\\

\haiku{gij werdt genummerd,,,,.}{gekeurd inge\"ent gebaad}{geknipt en gericht}\\

\haiku{Die Dood, welke wij,,.}{Nederlandsche soldaten}{niet genoeg kennen}\\

\haiku{deze tijd van den.}{Dood heeft ons niet gebeterd}{en niet gelouterd}\\

\haiku{Deden zij dit, dan.}{werden zij op de krijtrots}{tot koning gekroond}\\

\haiku{{\textquoteleft}mon lieutenant,.}{voici le commencement}{de ma trag\'edie}\\

\haiku{hij stelde daarbij ' {\textquoteleft}{\textquoteright}.}{een dronk in opn spoedig}{pax hominibus}\\

\haiku{Tegen 'n boomstam.}{lag de oppasser van den}{vlieger te kermen}\\

\haiku{Na haar lied liep ze, ';}{naar den boom wiens takkenn}{kribbe overhuifden}\\

\haiku{Toen nam ze oma Reg,:}{bij haar rokken en draaide}{haar om zeggende}\\

\haiku{Diep en dwars drongen;}{er de karresporen in}{den vettigen grond}\\

\haiku{Met de uiterste.}{moeite handhaafde hij nog}{z'n autoriteit}\\

\haiku{Jij vecht elken dag ' ' '.}{voor me.k Moett tochns}{aan iemand zeggen}\\

\haiku{Hij kladt cahiers vol.}{met engelen-kopjes}{en koeien-pooten}\\

\haiku{- {\textquoteleft}'k Verbied je van,:}{Wijk onkrijgstuchtelijk over}{m'n chefs te praten}\\

\haiku{Toen mengde zich de ':}{luitenant-adjudant van}{Wijk int gesprek}\\

\haiku{Wat hij vanaf zijn,!}{twintigste jaar gevreesd had}{werd nu bewaarheid}\\

\haiku{Na eenigen tijd zei {\textquotedblleft}{\textquotedblright} {\textquotedblleft}{\textquotedblright}.}{de luitzijn hand wordt stijf en}{toenzijn hand is dood}\\

\haiku{{\textquoteright} commandeeren en de '....}{echo vant diepe graf gaf}{de schoten terug}\\

\haiku{E\'ens in die vier....}{maanden bezochten hem de}{majoor en Queen in}\\

\section{Jan Greshoff}

\subsection{Uit: Currente calamo}

\haiku{beste makker, daar,;}{geloof ik geen woord van dat}{is pure nonsens}\\

\haiku{dat alles is mooi,;}{daar valt voor de burgerij}{niet aan te tornen}\\

\haiku{En ik geloof dat.}{niemand hierop een precies}{antwoord kan geven}\\

\haiku{Zoo vergaat het mij,.}{wanneer ik zoo nu en dan}{mijn Kloos ter hand neem}\\

\haiku{Wij hebben, voor ons,.}{zelf een legendarischen}{held van hem gemaakt}\\

\subsection{Uit: In alle ernst}

\haiku{De ware minnaar.}{leeft met zijn liefde alleen}{in een leege wereld}\\

\haiku{Men behoort niet op}{reis te gaan zonder eenige}{teksten van zijn hand.}\\

\haiku{Het zijn geringe.}{gebeurtenissen en het}{vermelden niet waard}\\

\haiku{De bohemien voert.}{iederen dag opnieuw een}{strijd op twee fronten}\\

\haiku{Wie het meesterschap,:}{zoekt doet afstand van wat het}{ware genot is}\\

\haiku{Speenhoff heeft immer.}{een eigen stijl van schrijven}{en spreken gehad}\\

\haiku{- Voor mij - zegt hij - is,.}{een boek volgepropt met kunst}{altijd een slecht werk}\\

\haiku{Ik schrijf uitsluitend,.}{voor mezelf omdat ik niet}{weet wat eerzucht is}\\

\haiku{vroeg ik verbaasd en.}{nu toch wel in mijn diepste}{gevoelens geschokt}\\

\haiku{wat de moeite waard,,}{is gezegd te worden is}{de moeite waard goed}\\

\haiku{Van Schendel weet  :}{dat zelf en hij heeft dit ook}{willen openbaren}\\

\haiku{Met dit boek voltooit.}{Arthur van Schendel de trits van}{zijn noodlotromans}\\

\haiku{Wie er vijandig;}{tegenover staat ontdekt er}{niets van belang in}\\

\subsection{Uit: Rebuten}

\haiku{Maar dat doet er niet,.}{toe het komt er op aan om}{gedrukt te worden}\\

\haiku{En nog ben ik er.}{niet zeker van dat zij in}{hun ongelijk staan}\\

\haiku{Het eenige wat ons.}{in de krant interesseert}{is het Gemengd Nieuws}\\

\haiku{Omdat die maar \'e\'en,:}{levensdoel maar \'e\'en reden}{van bestaan hebben}\\

\haiku{En zoodra wij}{ophouden met alles wat}{wij niet begrijpen}\\

\haiku{Ik houd ook wel eens,,.}{en mijn goede vrienden niet}{minder van lachen}\\

\haiku{dat is nu alles,.}{wat wij verlangen alles}{wat wij noodig hebben}\\

\haiku{Hier onderbreek ik,.}{even mijn brief omdat ambtsplicht}{mij naar buiten drijft}\\

\haiku{Er volgen er nog,.}{vele even positief en}{even kinderachtig}\\

\haiku{{\textquoteright} {\textquoteleft}Het rythme van de.}{nabije toekomst trilt reeds in}{de verleden tijd}\\

\haiku{all\'e\'en, uitsluitend).}{en all\'e\'en het koffijhuis}{in ons kan wekken}\\

\haiku{Het dwaasklinkende {\textquoteleft}{\textquoteright},;}{spreekwoordliefde is blind is}{au fond niet zoo dwaas}\\

\haiku{Gij hebt er geen flauw.}{vermoeden van hoezeer mij}{dat alles koud laat}\\

\haiku{Het doet mij pijn het.}{tegenover u openlijk te}{moeten erkennen}\\

\haiku{dat het goed is z\'o\'o;}{te handelen en slecht om}{het anders te doen}\\

\haiku{Wilt gij nu, waarde,?}{strijdmakker met den heer X.}{gaan discussieeren}\\

\haiku{Ik zal mij dus wel.}{wachten om er een oordeel}{over uit te spreken}\\

\haiku{\'o\'ok hier weer precies.}{dezelfde verhaspeling}{der verhoudingen}\\

\haiku{Wanneer ik uitglij,:}{over de vuiligheid van den}{heer X zeg ik ook}\\

\haiku{Ik ook, Theodoor,.}{wil in m{\`\i}jn wereld leven}{en blijven leven}\\

\haiku{Maar nu kom ik op.}{mijn betoog terug en aan}{Marnix Gysen toe}\\

\haiku{Als hij niets anders,.}{op zijn geweten had zou}{het nog wel schikken}\\

\haiku{{\textquoteright} Mij trof hier in het {\textquoteleft}{\textquoteright}.}{bijzonder het boeiende}{gebruik vantelkens}\\

\haiku{Uit zijn dichtproeven {\textquoteleft}.}{spreekt niet zelden groote deernis}{met de misdeelden}\\

\haiku{Van Willem Elsschot,;}{zwijgen we omdat men daar}{gewoonlijk over zwijgt}\\

\haiku{Maar het ergste zijn.}{professors omschrijvingen}{en aanwijzingen}\\

\haiku{Is het u wel eens?}{overkomen mij werkelijk}{boos aan te treffen}\\

\haiku{En ik zelf heb \'e\'erst;}{Gallinaria tegenover}{Alassio veroverd}\\

\section{Alfons Grond}

\subsection{Uit: Schetsen van Heihoven}

\haiku{g'n loch is blieve.}{hange En doer d'r baemd is}{opgevange}\\

\haiku{Voelde hij niet hoe?....}{wrede klauwen geslagen}{werden in zijn arm}\\

\haiku{De vreemde heer hing....}{over de onderdeur van de}{schuur en betoogde}\\

\haiku{uch volmach, va mich ' '............}{zoltr geene las kriege al}{howtr em dat e}\\

\haiku{es vader toch wal,{\textquoteright}.}{im sjtand zie om zoene}{peumes te regeere}\\

\haiku{{\textquoteleft}zaat sjamt ier uuch nit,{\textquoteright}.}{los jevelles och jet uvver}{vuer angere}\\

\haiku{Om dat grote doel,.}{te bereiken was er bier}{en veel bier nodig}\\

\haiku{Pitter-Grades.}{wilde  eten en met rust}{gelaten worden}\\

\haiku{Ee tuurke wie een,;}{paeperbus Wees ooch noch}{sjeef noa boave}\\

\haiku{Nee ich ka mit d'r,.}{beste wil Dien oetzich neet}{erg loave}\\

\haiku{D'r Piet dach dat 'n ':}{t nogal wis en begoos}{aaf te ratele}\\

\haiku{Zonger mich veul doa,}{bie te dinke bloos ich dat}{dink op en zoot doe}\\

\haiku{mit eene roe kop en '}{der doem opt flutje te}{kieke en wos neet}\\

\haiku{Mae 't mot toch 't ' '.}{int ent angert uever}{m gezag waere}\\

\haiku{De bladeren van.}{de bomen en de heggen}{hingen slap en stil}\\

\haiku{Marie stond achter.}{het buffet met de kin op}{haar vuisten geleund}\\

\haiku{Da zien ich dek der,.}{boer mit sjup of hak Al op}{d'r akker sjtoa}\\

\haiku{puende-n 't, ',......}{Treesje althans ich wolm}{puene mae i}\\

\haiku{Op ins doa huer......}{iech jet hinger miech roespere}{en sjnoespere}\\

\haiku{Opins doa huer......}{iech jet hinger miech roespere}{en sjnoespere}\\

\section{Robert H. van Gulik}

\subsection{Uit: Vier vingers}

\haiku{Dat werkte, want hij.}{zag tersluiks dat de aap hem}{geboeid gade sloeg}\\

\haiku{Gan dronk zijn kop leeg.}{en ging eens verzitten op}{zijn houten krukje}\\

\haiku{Met die anderen!}{bedoel je dan zeker al}{die vieze vliegen}\\

\haiku{{\textquoteleft}In zo'n sjieke zaak,{\textquoteright}.}{komen geen landlopers zei}{Tao Gan tot zichzelf}\\

\haiku{Ik wou haar net te,}{woord staan toen de baas d'r an}{kwam waggelen as}\\

\haiku{Gan schoof de strengen.}{koperstukken met een vies}{gezicht in Leng's schoot}\\

\haiku{Amper was ik voor,;}{m'n poort uitgestapt of ik}{werd ineens erg naar}\\

\haiku{{\textquoteleft}Ik wist toch niet dat,?}{die meneer bij het gerecht}{hoort Edelachtbare}\\

\haiku{nou maar 's of je.}{d'r vijf andere precies}{zo naast ken legge}\\

\haiku{Hij bromde een paar.}{woorden van dank en liep de}{Rode Karper uit}\\

\haiku{Tao Gan  zag dat.}{van de linker pink alleen}{een stompje over was}\\

\haiku{Seng Kioe bekeek:}{de hoge stenen muur eens}{en zei met ontzag}\\

\haiku{Hij gaf de reus een.}{klap tegen zijn kuiten met}{het plat van zijn zwaard}\\

\haiku{het lijk niet te lang,.}{boven de grond te laten}{met dit warme weer}\\

\haiku{Seng Kioe vatte.}{het stilzwijgen van Rechter}{Tie als twijfel op}\\

\haiku{as me zus die oom!}{Twan an het lijntje houdt dan}{benne we binne}\\

\haiku{Zeg op, wanneer heeft?}{Leng de lommerdhouder jou}{in dienst genomen}\\

\haiku{Toen hij haar eens goed,:}{had opgenomen begon}{de rechter rustig}\\

\haiku{Hij was een goeierd,.}{ook al heb-ie me in}{de steek gelate}\\

\haiku{Dan trekke we het,!}{Kanaal op en af met vracht}{da's een fijn bestaan}\\

\haiku{Heb je dan nog niet?}{ingezien dat die moordzaak}{nu is opgelost}\\

\haiku{{\textquoteleft}Ja, ik beken dat,.}{ik Twan Mou-tsai vermoord}{heb Edelachtbare}\\

\haiku{{\textquoteright} Hij keek de rechter:}{smekend aan en vervolgde}{met trillende stem}\\

\haiku{Toen heeft hij met de.}{huismeester gegeten en}{is naar bed gegaan}\\

\haiku{Hij is altijd zo...}{goedig en behulpzaam en}{zo lief voor dieren}\\

\section{Maurits Gysseling en W. Pijnenburg}

\subsection{Uit: Corpus van Middelnederlandse teksten. Reeks II. Literaire handschriften. II-6. Sinte Lutgart, Sinte Kerstine, Nederrijns moraalboek}

\haiku{Anja de Man, Het:}{leven van sinte Lutgard}{door broeder Geraert}\\

\haiku{Het volgende vers.}{begint dan meestal met}{een kapitaaltje}\\

\haiku{Deze heeft soms de,.}{vorm van een schuin soms van een}{gebogen streepje}\\

\haiku{die hi249 g[eliken250] [}{sinen oghen woude jc}{seid mindan251 ic se}\\

\haiku{] [ente]n301 melk[e] [][] [........}{der minschlecheitdie dar uloy}{en vte cristus mont}\\

\haiku{walschelant wonen}{node maer si hadde eer}{te herkenrode}\\

\haiku{ghsciet691 692Doen tfolc}{dit oppenbarlec kinde}{so liet men rusten}\\

\haiku{ende sach dat hi}{in sorchleken leuene lach}{doen hi van herten}\\

\haiku{Doen si hem bat sus,}{jnnechlec sere 748 voer sijn}{siele so antwerd}\\

\haiku{di haer hadde haer.}{pine ghec\r{u}rt jnt veghvier}{so langh te sine}\\

\haiku{ende seide / haer.}{wiedwijs dat hi een nonne}{bedrogen hadde}\\

\haiku{alle sondaghe.}{al sint austijn maent dat doen}{sal elc goet kerstijn}\\

\haiku{962 Et es heilech}{salech end goet dat men den}{sieken dan gracia}\\

\haiku{metten nonnen sanc}{in den coer 998 daer sach een}{nonne di scegen}\\

\haiku{hi vuedet ende}{heuet lief ende smeket}{ende huedet Hien}\\

\haiku{dees waert O dochter}{van iherusalem stant oppe}{van dinen bedde}\\

\haiku{haer god een ander.}{manyre der lichamlec}{martyrien scire}\\

\haiku{die heilech waren}{beide ende goet die daer}{na dat si sturtte}\\

\haiku{hoe dat was wijlneer}{sijn name 1212 ende van}{wat verdientten hi}\\

\haiku{] seide dat si soud []}{werdden sciereverloest}{genedechlec vten}\\

\haiku{beddeken daer men []....}{af seghtdat scone bloyt}{ende blomen dreght}\\

\haiku{dit varen ende..}{ginghen spreken stappans van}{andren dinghen}\\

\haiku{Dies oec geloeft moet.}{werden de vrie di edele}{maght sinte kerstine}\\

\haiku{Op di vre dat men,}{voer mi sprac in de messe}{dierste agnus dei so}\\

\haiku{hem lief ocht waest hem..}{leet 1997 ende nayet aen}{haer clederkine}\\

\haiku{, hebben omgeghaen}{daer god sijn gracie hadde}{met gedaen ende}\\

\haiku{natuerlec sijn}{bloede 2170 Boude maeghde}{sijn selden bleuen}\\

\haiku{Jn dalder leste}{iaer dat si van ertrike}{sciet de maghet2245 vri}\\

\haiku{om en laet di mi}{nyet wedercomen ter}{erden daer ic af}\\

\haiku{135,34 136,3 136,4 136,4 136,6}{136,7 136,11 136,18 136,21 136,29 136,30 136,35}{136,37 137,1 137,3 137,5 137,13}\\

\haiku{) gewaerlec 69,8 () () (}{133,352 gewagen 9,141}{gewaghde 146,361}\\

\haiku{ysaac gemac ghemac [}{brac gebrac sprac trac uertrac}{vertrac sac stac.]ec}\\

\haiku{waarschijnlijk is dit.}{gebeurd door aankoop in de}{17de-18de eeuw}\\

\haiku{/   Urinscap is als.}{en minsche g\r{u}den wille}{he/uet tegens enen man}\\

\haiku{man / vint selden    /.}{dat sconheit inde suuerheit}{te gader bliuen}\\

\haiku{/ als lange als man /.}{wel helpen mag so heuet}{man   vrinde gn\r{u}g}\\

\haiku{/   D\r{u} al dat du /.}{d\r{u}s gelik of t\r{u}t v\r{u}r al}{den luden dedes}\\

\haiku{Jnde namelik /.}{dit gescrigt   is sulik}{dat varwe beh\r{u}ft}\\

\haiku{so / valt dicke dat /.}{sine slain in   dogen}{mit den vl\r{u}gele}\\

\haiku{Jnde als manne /.}{vindet s\r{u}nder dat lit.}{so lait manne gain}\\

\haiku{376,18 376,19 376,29 376,30 376,32}{376,36 376,38 376,40 376,43 377,2 377,2 377,11}{377,30 377,33 377,39 378,4 378,12}\\

\haiku{388,19 388,23 388,42 389,3 389,4}{389,11 389,28 389,33 389,38 390,10 390,33 390,35}{391,9 392,1 392,2 392,9 392,12}\\

\haiku{401,40 402,12 403,3 403,4 403,6}{403,11 403,11 403,13 403,15 403,18 403,34 404,6}{404,7 404,9 404,18 404,20 404,25}\\

\haiku{363,21 363,25 363,26 363,36 367,28}{368,1 368,17 369,16 370,30 370,37 371,36 372,12}{372,18 373,2 373,3 373,3 373,14}\\

\haiku{356,33 356,33 356,33 356,34 356,35}{356,35 356,36 356,37 356,40 357,1 357,4 357,4}{357,4 357,5 357,8 357,8 357,9}\\

\haiku{372,26 372,26 372,27 372,29 372,30}{372,34 372,36 372,39 372,41 372,42 372,43 373,4}{373,4 373,9 373,12 373,12 373,15}\\

\haiku{386,29 386,29 386,29 386,29 386,33}{386,33 386,33 386,41 387,1 387,6 387,6 387,7}{387,8 387,12 387,16 387,17 387,20}\\

\haiku{405,43 407,5 419,22 419,24 421,11 () () ()}{12 drag 389,71 drage 398,18}{1 dragen 356,38 370,36}\\

\haiku{366,8 366,15 366,15 372,19 (6) () (}{edelre 372,201 egen 359,35 359,35}{375,5 391,42 393,10 393,116}\\

\haiku{387,22 387,31 387,33 387,35 387,36}{387,38 387,38 387,39 388,1 388,1 388,11 388,14}{388,18 389,3 389,3 389,5 389,13}\\

\haiku{392,24 392,25 392,25 392,27 392,27}{392,33 392,33 392,35 392,35 392,37 392,39 392,42}{392,43 393,4 393,8 393,14 393,15}\\

\haiku{396,29 396,29 396,30 396,32 396,33}{396,36 396,37 396,40 397,4 397,6 397,6 397,7}{397,8 397,9 397,9 397,9 397,13}\\

\haiku{410,35 410,43 411,9 411,13 411,13 ()}{411,18 411,42 412,11 412,14 412,16 412,18 412,19}{412,2117 er 355,7 357,16}\\

\haiku{377,27 384,28 389,22 393,37 399,8 () (}{15 erliken 356,22 356,27 371,27}{372,14 387,25 394,8 397,347}\\

\haiku{392,24 392,25 392,27 392,40 394,30}{394,31 395,2 395,26 395,29 396,27 398,12 399,8}{399,33 400,8 402,3 404,23 408,14}\\

\haiku{384,43 390,35 391,7 410,5 (5) ()}{getempertheide 357,81}{getempertheit 357,12}\\

\haiku{366,34 367,1 368,25 370,20 371,39}{372,41 374,4 374,26 374,31 375,8 375,9 375,39}{376,17 376,18 377,8 377,11 377,17}\\

\haiku{372,27 373,42 382,7 382,8 382,11}{383,17 383,19 383,37 385,1 385,39 386,23 389,4}{389,39 392,11 393,21 397,38 398,17}\\

\haiku{414,15 414,34 414,36 414,39 414,41}{414,44 415,5 415,8 415,11 415,12 415,18 415,20}{415,21 415,29 415,40 416,1 416,9}\\

\haiku{373,37 373,41 373,41 373,43 374,2}{374,6 374,7 374,8 374,11 374,13 374,13 374,13}{374,24 374,40 374,40 375,12 375,20}\\

\haiku{375,29 375,35 375,44 376,8 376,30}{376,34 376,34 376,35 376,36 376,36 376,38 377,10}{377,16 378,15 378,15 378,26 378,31}\\

\haiku{420,33 421,23 421,33 421,36 422,7 () ()}{422,9 422,9 422,10 422,11327 idel 379,35}{389,3 399,93 idelen}\\

\haiku{370,26 370,27 370,28 370,29 370,31}{370,31 370,35 370,40 371,6 371,12 371,14 371,15}{371,19 371,30 371,30 371,32 371,33}\\

\haiku{414,24 (3) i\r{u}ng 367,39 367,39 () ()}{410,17 420,124 iunge 399,14 418,22}{2 iungen 385,37 388,7}\\

\haiku{378,20 378,21 378,24 378,26 378,31}{378,33 378,36 378,38 378,38 379,5 379,6 379,8}{379,8 379,9 379,11 379,17 379,30}\\

\haiku{408,27 408,31 408,35 408,36 408,38}{408,39 408,41 408,43 409,1 409,2 409,4 409,5}{409,6 409,9 409,12 409,13 409,16}\\

\haiku{) node 360,14 364,32 368,40 () ()}{392,20 399,21 408,396 nof 369,29 376,29}{376,29 407,294 nog 356,28}\\

\haiku{384,26 385,4 385,18  388,18}{388,21 391,12 391,17 400,29 401,32 405,8 408,10}{409,30 410,3 411,33 412,5 414,8}\\

\haiku{380,31 383,10 383,11 388,12 388,18}{388,32 389,8 389,12 390,7 391,37 391,37 394,23}{398,24 398,28 400,1 400,6 401,20}\\

\haiku{387,6 387,6 389,8 391,32 394,17}{396,19 397,3 401,3 401,15 403,2 403,18 404,15}{404,26 405,5 405,10 405,36 406,2}\\

\haiku{378,30 379,31 379,34 379,36 379,37}{380,10 382,2 382,15 382,31 383,15 384,8 385,1}{385,17 385,26 385,27 387,1 387,26}\\

\haiku{377,32 377,33 377,34 378,3 378,17}{378,21 378,24 378,25 378,27 379,4 379,30 380,9}{380,9 380,12 380,36 381,6 382,6}\\

\haiku{406,29 417,37 (7) simmen ()}{406,12 406,16 407,3 407,30 409,31 416,10 416,11}{416,29 417,169 sin 355,6}\\

\haiku{394,20 394,22 394,23 394,26 394,26}{394,33 394,33 394,35 395,12 395,13 395,14 395,20}{395,22 395,24 395,43 396,5 396,6}\\

\haiku{371,21 374,16 377,37 378,6 378,23}{379,36 380,25 384,17 385,42 396,7 397,7 400,38}{401,22 401,24 405,34 406,39 408,28}\\

\haiku{) tr\r{u}we 420,31 (1) tr\r{u}welik () () ()}{421,41 tr\r{u}welike 422,61}{tr\r{u}wen 415,5 421,52 tscip}\\

\haiku{362,32 362,33 (3) tsingen () () () ()}{405,11 tsloit 419,111 tspar 411,36}{1 tstert 411,37 411,382}\\

\haiku{408,26 408,39 410,9 410,10 413,4}{413,5 413,6 413,7 413,8 413,9 413,9 413,10}{413,11 413,12 413,13 413,13 413,13}\\

\haiku{366,33 370,30 372,18 377,31 385,35 ()}{385,36 392,6 407,5 413,36 417,2611 vir}{357,6 357,13 357,32 359,8 363,12}\\

\haiku{394,20 394,25 395,2 395,23 396,10}{396,11 396,19 396,33 397,23 399,8 399,20 400,8}{400,38 400,39 401,1 401,21 401,28}\\

\haiku{407,12 407,26 409,5 410,18 410,20}{410,33 411,3 411,6 411,13 411,14 411,19 411,38}{412,2 412,18 412,19 413,17 413,26}\\

\haiku{370,35 370,36 370,39 371,1 371,3}{371,7 371,9 371,11 371,15 371,21 371,23 371,35}{371,40 372,17 372,20 372,22 372,25}\\

\haiku{) wintelt 414,28 415,14 421,29 () () ()}{3 wirdet 418,141 wirken}{357,17 382,172 wirkene}\\

\haiku{373,42 391,1 399,6 399,22 404,31 ( ()}{405,32 408,208) wiuer 405,41 415,37}{2 wiuere 406,26}\\

\haiku{420,33 422,8 (30) wort 374,25 () ()}{379,29 379,30 382,6 382,9 403,41 418,47}{wragt 378,151 wrake}\\

\haiku{736165 737170 73837 ro 739175?}{740di op rasuur 741180 742i}{verbeterd uit e}\\

\chapter[28 auteurs, 5766 haiku's]{achtentwintig auteurs, vijfduizendzevenhonderdzesenzestig haiku's}

\section{Jacob Isra\"el de Haan}

\subsection{Uit: Pathologie\"en. De ondergangen van Johan van Vere de With}

\haiku{In de kamer naast,.}{de hare sliep haar man die}{het kind bij zich hield}\\

\haiku{Zij ging naar hare.}{eigen kamer terug om}{den dood te zoeken}\\

\haiku{{\textquoteright} Zijn vader verschrok,:}{zichtbaar voor Johan en de man}{dacht snel en beslist}\\

\haiku{{\textquoteright} Johan gevoelde, dat.}{de stem van zijnen vader}{brak in diens keel}\\

\haiku{Maar op dat juiste.}{oogenblik brak zijn lichaam}{in eene uitbarsting}\\

\haiku{{\textquoteleft}omdat ik juist voor,.}{vader zoo gevoel is ons}{leven toch verward}\\

\haiku{Johan zag het altijd.}{gaarne hoe de nacht in den}{dag veranderd werd}\\

\haiku{Johan gevoelde, dat.}{de linkerhand van zijnen}{vader heel zacht was}\\

\haiku{hoe de prachtige.}{rivier wezen zou met wind}{en met dat avondlicht}\\

\haiku{Johan zorgde geheel,.}{alleen dat zijn vader en}{hij nog thee kregen}\\

\haiku{{\textquoteleft}laat ik vader nu,.}{vertellen wat mijn groot en}{geheim verdriet is}\\

\haiku{Johan had twee dingen,.}{in dat gesprek gebracht die}{vreemd genoeg waren}\\

\haiku{Tusschen vieren en.}{half vijf moesten dan dikwijls de}{lampen nog licht op}\\

\haiku{{\textquoteright} Kor Koster sprak nu:}{met zacht-droeve stem van}{zeer ernstig verwijt}\\

\haiku{Johan wist, dat hij zoo,.}{gehavend was dat hij niet}{naar school kon komen}\\

\haiku{als je wilt mag je.}{gerust uit ons huis en uit}{onze stad weggaan}\\

\haiku{Ik dank je wel, dat,.}{je mij geschreven hebt dat}{het je vrij goed gaat}\\

\haiku{De vader van Johan:}{kwam uit Haarlem terug en}{daarna schreef hij Johan}\\

\haiku{Dat bovenhuis is,}{niet groot dus je kunt beslist}{niet meer dan \'e\'ene}\\

\haiku{Verder legde hij.}{de blonde vouwen van zijn}{haar goed in orde}\\

\haiku{De levenswijze.}{van het huis in Cuilemburg}{veranderde dus}\\

\haiku{Op eenen zonnigen,.}{middag zag Johan de groote zee}{die hij niet kende}\\

\haiku{Hij gaf in dien tijd,:}{verfijnde uitvoerige}{beschrijvingen van}\\

\haiku{Zij konden evengoed,.}{naar den bleeker gaan als}{ergens anders heen}\\

\haiku{Er kwam geen antwoord,.}{na zoovele biddende}{smekende brieven}\\

\haiku{{\textquoteleft}Het spijt mij, dat ze,...}{niet ruiken ik houd niet van}{bloemen zonder geur}\\

\haiku{{\textquoteright} Ren\'e werd nu diep,.}{teeder waardoor hij Johan nog}{veel meer verwarde}\\

\haiku{Hij wilde toen zelf.}{naar Londen gaan om Ren\'e}{naar huis te halen}\\

\haiku{Daarna zeide hij,.}{in de school dat hij thuis bleef}{omdat hij ziek was}\\

\haiku{{\textquoteright} {\textquoteleft}Dat weet ik niet... ik,...}{zeg je alleen eerlijk en}{precies wat ik voel}\\

\haiku{Maar ik kan nu niet,,}{dicht bij je blijven hier in}{huis dat voel ik wel}\\

\haiku{Zijn lichaam werd warm,.}{van licht trillend als fijne}{gevoelige vlam}\\

\haiku{dat is alles, wat.}{je met je onredelijk}{verlangen bereikt}\\

\haiku{Ook niet, omdat ik.}{dan wellicht van dit huis en}{Haarlem scheiden moet}\\

\haiku{Want Ren\'e zou hem.}{in het huis zijns vaders niet}{durven navolgen}\\

\haiku{En hoe dat einde,,.}{wezen zal dat hangt niet van}{jou af maar van mij}\\

\haiku{Ren\'e verhaalde.}{aan Johan van zijn leven in}{diepte van steden}\\

\haiku{{\textquoteleft}Kom Hannie,{\textquoteright} vleide, {\textquoteleft},}{Ren\'ehet is nu toch over}{twee\"en naar school gaan}\\

\haiku{maar 't is goed, ik,...}{zal je wel weer laten zien}{dat ik je liefheb}\\

\haiku{als u Ren\'e dwingt,....}{hier uit het huis te gaan dan}{ga ik ook beslist}\\

\haiku{het licht knipte in,,.}{zijne donkere oogen hij}{stond naast Johan die zat}\\

\haiku{Een avond van dien tijd,.}{van het leven van Johan was}{van prachtige rust}\\

\haiku{Rustig, met volmaakt,.}{bewerkte letters schreef hij}{hem eenen kleinen brief}\\

\haiku{Johan hoorde, dat hij.}{met water waschte en}{steenen dingen bewoog}\\

\haiku{{\textquoteright} {\textquoteleft}En dat je weer even,{\textquoteright}.}{ongelukkig worden zult}{zeide Ren\'e snel}\\

\haiku{... en daarom zul je...}{ten slotte aan je liefde}{voor mij ondergaan}\\

\haiku{en het is voor mij,.}{voordeelig dat het leven}{nu eenmaal zoo is}\\

\haiku{Hij wist niet, waarom,.}{hij Ren\'e had ontmoet door}{wien hij onderging}\\

\haiku{Maar daar heb ik op.}{het oogenblik niet zoo heel}{veel bezwaar tegen}\\

\haiku{Of de vrouw het eerst,,.}{daar ben ik nog benieuwd naar}{om dat te weten}\\

\haiku{{\textquoteleft}je denkt geloof ik,,.}{dat ik alles met je doe}{waar ik zin in heb}\\

\haiku{Hij rilde nu in.}{betrouwbare sterkte van}{eigen veiligheid}\\

\haiku{{\textquoteright} 30 Johan lag zonder.}{pijn en zonder ongeduld}{op zijne slaapplaats}\\

\haiku{De handen van Johan,.}{werden stukgeslagen en}{zoo ook zijn gezicht}\\

\haiku{Hij vroeg, of hij niet.}{voor eenige weken bij hen}{binnenwonen kon}\\

\haiku{{\textquoteright} {\textquoteleft}Dat geloof ik ook...,,?}{dus je vindt het niet noodig dat}{ik je dat voorlees}\\

\haiku{Zijn hoofd brandde van,.}{pijn en het stak zijne oogen}{aan die overspanden}\\

\haiku{Dat zijn de gaven,.}{des levens die ons met het}{leven verzoenen}\\

\haiku{De betekenis:}{van zijn dichterschap is nu}{algemeen erkend}\\

\haiku{Dit boek hoort bij het.}{levensverhaal van Jacob}{Isra\"el de Haan}\\

\subsection{Uit: In Russische gevangenissen}

\haiku{Maar zijn goede trouw.}{is boven den minsten}{twijfel verheven}\\

\haiku{De regeering beschikt,.}{over de verkeersmiddelen}{post en telegraaf}\\

\haiku{Daarvan maakt men zich.}{in het buitenland eene te}{zwarte voorstelling}\\

\haiku{Teleurstelling over,.}{het feit dat de Doema zoo}{weinig bereikte}\\

\haiku{Het zal duren tot,.}{de boeren begrijpen dat}{zij bedrogen zijn}\\

\haiku{en dat zij het weer,.}{zullen doen wanneer er geen}{ander bewind komt}\\

\haiku{Van Foinitzky en.}{van den Hoogleeraar en}{Senator Tag\'anzeff}\\

\haiku{In de kolonie.}{bij St. Petersburg volgt men}{het tweede systeem}\\

\haiku{De salarissen.}{in de koloni\"en en}{prioeten zijn laag}\\

\haiku{In de tuinen en.}{weiden waren de jongens}{nu aan  het werk}\\

\haiku{Russische knapen.}{hebben zuivere stemmen}{en een diep gevoel}\\

\haiku{Eene groote sterfte in.}{de gevangenissen is}{het gevolg hiervan}\\

\haiku{In Moscou vele.}{uit het zoo mogelijk nog}{meer beruchte Orel}\\

\haiku{Zij zijn gevangen:}{gezet door de civiele}{autoriteiten}\\

\haiku{in zijn oogen was de.}{vage dwaling die erger}{is dan eenige klacht}\\

\haiku{De redenen van.}{herroeping zijn talrijk en}{vaag geformuleerd}\\

\haiku{De gevangene.}{moest daar ook slapen wegens}{gebrek aan ruimte}\\

\haiku{Toen den vierden dag.}{een bewaarder kwam zag die}{het vuil in de cel}\\

\haiku{In zijn woede sloeg,.}{hij den man tegen den grond}{die in het vuil viel}\\

\haiku{Wij zaten samen.}{op den rand van haar bed als}{oude bekenden}\\

\haiku{De katorgisten,.}{werken twaalf uur per dag met}{een uur vrijen tijd}\\

\haiku{Z\'o\'oleverde 1907 reeds\%\%.}{24 meer op dan 1906 en 1908}{wederom 26 meer}\\

\haiku{Het gevolg is, dat.}{wij allen zenuwziek en}{ellendig worden}\\

\haiku{Ieder gesprek met.}{een gevangene alleen}{moest ik afdwingen}\\

\haiku{ach, krachtloos wordt hun,.}{verstand Dat het de dagen}{van hun strijd vergeet}\\

\haiku{Over eenigen hunner.}{moge in het bijzonder}{nog iets gezegd zijn}\\

\haiku{dat is al veel waard,,.}{dan vinden de lateren}{een steun een partij}\\

\haiku{ik was angstig voor,,.}{dien kalmen trotschen jongen}{nog niet twintig jaar}\\

\haiku{Nu ontbreekt ook de.}{schijn van recht die een vonnis}{toch altijd  heeft}\\

\haiku{Toen ik er was, werd.}{in Riga niet meer alle}{dagen geslagen}\\

\haiku{Ik antwoordde, dat.}{ik zijne carri\`ere}{niet bedreigen mocht}\\

\haiku{Ik ben 51 jaar, ik,.}{geloof niet dat ik nog vier}{jaar te leven heb}\\

\haiku{mijn voeten doen zeer,,.}{van die kettingen mijn maag}{is ziek mijn hoofd brandt}\\

\haiku{Ook Zjadanofski.}{werd geslagen tot hij zich}{niet meer verroerde}\\

\haiku{{\textquoteleft}Ik herinner mij.}{het boek van den Heer Harry}{de Wyndt nog zeer goed}\\

\haiku{Een half jaar later.}{was er te Riga weder}{een hongerstaking}\\

\haiku{Zij verzekerden,.}{mij dat de toestand in Orel}{schandelijk slecht is29}\\

\haiku{Zinge mijn lied nog,}{slechts van uw hartenbrekend}{lijden Vermijde}\\

\haiku{Daarom ben ik ook.}{zoo verbaasd over wat verder}{te Orel is gebeurd}\\

\haiku{Ze vlogen op mij.}{af en begonnen mij te}{slaan en te trappen}\\

\haiku{Hij genoot van het:}{schouwspel en hij hitste de}{bewakers nog aan}\\

\haiku{Ik blijf hier tot er.}{plaats is in een inrichting}{voor zenuwlijders}\\

\haiku{Ik verwachtte niet, {\textquotedblleft}{\textquotedblright}.}{anders of men zou ook mij}{eenbezoek brengen}\\

\subsection{Uit: Jerusalem}

\haiku{Het zal haar weer acht.}{pond kosten en iedereen}{zal haar uitlachen}\\

\haiku{Wij echter krijgen.}{een mooi biljet van goud op}{zijde-papier}\\

\haiku{Hamame zit als.}{een pop in wit met sluier}{van wit en zilver}\\

\haiku{En daartegen in,,.}{een kleine die gespannen}{luid kraait en schatert}\\

\haiku{{\textquoteleft}laat mij teruggaan.}{naar Jeruzalem en de}{bron alsnog openen}\\

\haiku{Gij glimlacht wellicht?}{over al dien eerbied en over}{al die etiquette}\\

\haiku{En Abdoel Salaam,,.}{die een wijs man is heeft hem}{laten omhakken}\\

\haiku{Ik hoop er nog wel.}{eens heen te gaan met Sa{\"\i}d}{Effendi samen}\\

\haiku{En Arabisch met den.}{vader en de andere}{familieleden}\\

\haiku{Een tafeltje bij.}{het raam met het mooie uitzicht}{op Jeruzalem}\\

\haiku{Natuurlijk komt het,,.}{niet te pas hem zonder meer}{daarnaar te vragen}\\

\haiku{Schatten aan boeken.}{zijn naar Engeland en naar}{Amerika gegaan}\\

\haiku{Een van mijn Joodsch (!}{vrienden te Amsterdamwat}{is Amsterdam ver}\\

\haiku{Maar achter al die.}{waarde-verschillen zal}{ik wel nooit komen}\\

\haiku{Maar er zijn hoeken,.}{waar de zon nooit komt en waar}{de lucht loodzwaar is}\\

\haiku{Op het terras van,,,,.}{het weeshuis vol vol vol van}{zon is nu bezoek}\\

\haiku{Van het Weeshuis gaan.}{we dus eerst weer het domein}{van de Russen over}\\

\haiku{Er is een rijweg,.}{door de Engelschen in den}{oorlog uitgelegd}\\

\haiku{Of opgezet als,.}{groote huizen en afgebouwd}{met een dwaas kort dak}\\

\haiku{Haar broer een van de.}{stoutsten en sterksten onder}{de Joodsche ruiters}\\

\haiku{Wie ook bestolen,.}{worden de gasten van den}{hoofdman zeker niet}\\

\haiku{Adil koopt ook bonbons.}{voor de vier vrouwen en de}{vele kinderen}\\

\haiku{En toch zijn de twee.}{slagen met een wreeden dood}{niet te duur betaald}\\

\haiku{Geen ander geluid.}{dan de stappende pooten}{en de echo daarvan}\\

\haiku{Maar nu Aboe Fares,.}{de sjech van allen is zijn}{er geene twisten meer}\\

\haiku{Waar wij rijden langs.}{zwarte tenten slaan de}{waaksche honden aan}\\

\haiku{Wij moeten elk van.}{de huisbedienden een half}{pond baksjisj geven}\\

\haiku{Ditmaal met een knecht,.}{die loopt en straks de paarden}{terugbrengen zal}\\

\haiku{{\textquoteright} Omdat de Polen.}{de lievelingen van mijn}{Volk hebben vermoord}\\

\haiku{Omdat de Polen.}{de lievelingen van mijn}{Volk hebben vermoord}\\

\haiku{Om twee uur zullen.}{wij bij den Klaagmuur komen}{voor de gebeden}\\

\haiku{Een geheelen dag.}{vasten is in dit heete}{weer wel al te zwaar}\\

\haiku{Verleden jaar  .}{is de Regeering het Weeshuis}{goed gezind geweest}\\

\subsection{Uit: Palestina}

\haiku{Deugdzaam sterven is,.}{mogelijk deugdzaam leven}{een ongerijmdheid}\\

\haiku{Voorloopig zeg,.}{ik dus alleen dat wij naar}{Hebron zullen gaan}\\

\haiku{Er is toch nu wel.}{iets veranderd bij een jaar}{of twee geleden}\\

\haiku{Een heel oud man met.}{het vertrouwen van een kind}{in zijnen vader}\\

\haiku{En zij zal water.}{morsen op de gordijnen}{en op de sofa}\\

\haiku{Wij zien van verre,.}{Bethlehem waar de rijke}{Christenen wonen}\\

\haiku{Abdoel Sala\"am heeft,.}{een witte warme deken}{medegenomen}\\

\haiku{De schavuit haalt een.}{diepe tabaksteug door zijn}{Turksche waterpijp}\\

\haiku{David deelde buit ().}{met de mannen van Hebron}{I Samuel 30:31}\\

\haiku{Godfried van Bouillon.}{gaf de stad in Leen aan den}{Heer van Avesnes}\\

\haiku{Vele kooplieden.}{dragen witte en groene}{doeken om hun fez}\\

\haiku{Vandaag zagen wij '.}{hen voort laatst gaan door de}{stad en door het dal}\\

\haiku{En wendt dan naar het,.}{Westen om bij Gaza in}{de zee te breken}\\

\haiku{Van boven af zien,,.}{wij het dal breed en edel waar}{de weg doorheenwindt}\\

\haiku{Wij kennen ook den,.}{onder-gouverneur uit}{Hebron gekomen}\\

\haiku{O, Berseba is.}{veel gemakkelijker dan}{Jeruzalem}\\

\haiku{De   B.F.C. regelt.}{de Dujet en de Qis\^as naar}{vaste tarieven}\\

\haiku{De Dujet hangt af.}{van rang en vermogen van}{beide partijen}\\

\haiku{Wij hopen ook een.}{bezoek te brengen bij de}{moeder van Jimmy}\\

\haiku{En zij is verloofd,.}{met den zoon van sjeikh Djadoeng}{die al zes jaar is}\\

\haiku{De twee leerjongens.}{blijven in hun holletje}{om af te werken}\\

\haiku{De h\^otelhouder.}{met vrouw en vijf kinderen}{in de andere}\\

\haiku{Weer verandert de.}{verhouding in den tijd van}{de Maccabee\"en}\\

\haiku{De Philistijnen.}{zijn onder de macht van de}{Grieken gekomen}\\

\haiku{De meesten zijn na.}{den oorlog uit Rusland in}{het Land gekomen}\\

\haiku{Abdoel Sala\"am is.}{h\'e\'el verheugd den ouden man}{mede te nemen}\\

\haiku{Hier, in dien tuin, waar,.}{het vernielde landhuis staat}{vestten de Turken}\\

\haiku{Zij duiken weg in.}{hunne schoudermantels van}{grijs en bruin gestreept}\\

\haiku{Wij willen vandaag,.}{nog rijden naar Esd\^ud het}{oude Ashdod}\\

\haiku{De schapen en de.}{geiten zullen er gaan in}{gemengde kudden}\\

\haiku{Door een prachtige.}{cactuslaan strompelen}{wij Yebnah binnen}\\

\haiku{Gelukkig werken.}{er sterke Joodsche jongens}{in de nabijheid}\\

\haiku{Kameelen trekken,.}{langs den weg die de kisten}{naar Jaffa voeren}\\

\haiku{Maar zij moeten ons,.}{vergeven dat wij er ons}{thans niet ophouden}\\

\haiku{En Allah zal ons,.}{leeren hoe mooi ons land ook in}{den regen kan zijn}\\

\haiku{Wij zouden een h\'e\'el.}{mooi uitzicht kunnen hebben}{van het platform af}\\

\subsection{Uit: Pijpelijntjes}

\haiku{{\textquoteright} In de kamer zit, '.}{de moeder mager ent}{dundonkere haar grijs}\\

\haiku{{\textquoteright} doofvraagt de oude.}{en haar woorden cadansen}{met de stopnaald mee}\\

\haiku{'t Kacheltje brandt,...}{een warme looming lijst}{in de kamer neer}\\

\haiku{{\textquoteright} {\textquoteleft}Ja... gesoesd, nou je ',, '...?}{t zegt hoor ik ook datt}{regent ben je klaar}\\

\haiku{{\textquoteright} {\textquoteleft}'t Regent zoo... 't, '.}{regent zoo laten wet}{morgenochtend doen}\\

\haiku{Een prettig gevoel.}{tintelde door mijn rug en}{langs mijn onderlijf}\\

\haiku{Toen vroolijker in '.}{t kleurige lampelicht}{dronken we koffie}\\

\haiku{Alles donker en,,.}{stil de straat leeg waardoor \'e\'en}{man zijn stappen klonk}\\

\haiku{{\textquoteright} Veertien dagen was '.}{Sam weg en ik wast huis}{niet uit geweest}\\

\haiku{... ik moet in eenen uit,.}{op deze brief wacht u maar}{niet met de koffie}\\

\haiku{nou op, dat Hec niet,.}{bijt als-ie los komt dan zorg}{ik voor m'neer Sam}\\

\haiku{{\textquoteright} {\textquoteleft}Kan me niks schelen, '....}{watr met dat beest gebeurt}{zijn m'n zaken niet}\\

\haiku{Uit de donkere,.}{huizenhoek naast ons doemde}{oud mannetje op}\\

\haiku{juffrouw Meks liet de,....}{gordijnen vallen met een}{harde rinkratel}\\

\haiku{Toosie ga jij naar bed,{\textquoteright},, '.}{strengde juffrouw Meks die vond}{datt verkeerd ging}\\

\haiku{welnou dan moet ze.}{maar deris bij dominee}{Deelman gaan hooren}\\

\haiku{{\textquoteright} {\textquoteleft}Zonder schandaal en,.}{die daalder betaalt u ook}{niet dat zal u zien}\\

\haiku{{\textquoteright} {\textquoteleft}Wat heb je d'r toch, '.}{mee noodigt lijkt waarachtig}{wel of je mee moet}\\

\haiku{Juffrouw Meks vond 't ', ' '.}{n goed id\'ee juffrouw Meks vond}{tn heel goed idee}\\

\haiku{{\textquoteright} {\textquoteleft}Nou{\textquoteright} zegt Hein {\textquoteleft}nou ga,.}{ik toch niet weg voor ik die}{ouwe moer ook heb}\\

\haiku{Om drie uur liep ik,,.}{de straat op die nog nat was}{maar zonder regen}\\

\haiku{En dan de drukke, '.}{stad-in-de-stad dwars}{door naart Rokin}\\

\haiku{Ze liepen met z'n,.}{twee\"en een ander was er}{bij dan gisteren}\\

\haiku{En dicht tegen Sam.}{aankreunend vertelde ik}{hem ellende}\\

\haiku{u kan ze gerust,.}{eten d'r zit geen margerien}{op en kaas van Noack}\\

\haiku{{\textquoteleft}Hoe vin-je me nou...,...}{laat Sam nou eerlijk zeggen}{of ik beter wor}\\

\haiku{morgenochtend zie '.}{ik je wel of vanavond als}{t niet te laat wordt}\\

\haiku{Maar dan zou ik nu,,.}{het andere nemen het}{gore niet goede}\\

\haiku{Maar ik wou toen niet,.}{met hem trammen hij was zoo}{gering en zoo vuil}\\

\haiku{{\textquoteright} In 't deurportaal,.}{stond Koos koudrillend met z'n}{handen in z'n zak}\\

\haiku{{\textquoteright} In de kamer stak,.}{ik de groote lamp ook op nu}{kon ik hem goed zien}\\

\haiku{Nou, toen woonde d'r ' ':}{bij ons opt dorpn oud}{vrouwtje en die zei}\\

\haiku{{\textquoteleft}Het is netjes, ik,,{\textquoteright}...}{moet u zeggen juffrouw Meks}{het is heel netjes}\\

\haiku{Sam moest naar Haarlem, '.}{een middag en ik had hem}{naart spoor gebracht}\\

\haiku{as m'neer Sam komt......}{zullen we wel zien steek u}{de lamp maar even op}\\

\haiku{eerst al die drukte.}{van je advocaat en zoo}{en van de rechtbank}\\

\haiku{d'r was niemand om,,}{me te halen nou en de}{rest dat snap je wel}\\

\haiku{{\textquoteleft}Ik heb nog wat geld,,?}{niet veel hoor zal ik nou maar}{bij jullie blijven}\\

\haiku{Met m'n slaapwarme '...}{voeten opt koudgladde}{vloerzeil rilde ik}\\

\haiku{{\textquoteleft}Zalle we niet naar,{\textquoteright}, {\textquoteleft}'.}{bed gaan fluisterde zijt}{is morgen vroegdag}\\

\haiku{Nee die is straks aan '.}{t station met Anna}{en de anderen}\\

\haiku{De kamer druk van,.}{menschen en ruw praten door}{mekaar kletsende}\\

\haiku{{\textquoteright} {\textquoteleft}Ja, dat wel, maar op.}{alle meniere hebt u}{z'n uitgaanskas nog}\\

\haiku{Anna help jij Stien, '.}{even voort naar beneden maar}{laatr niet vallen}\\

\haiku{Moet je begrijpen,,}{dat je daar nog na de kerk}{toe moet ook affijn}\\

\haiku{Hij dan weer langzaam.}{z'n woorden nou-kauwend}{en wikkewegend}\\

\haiku{ik zei gewoon, dat '...}{ikt verdomde en of}{ik ander werk kreeg}\\

\haiku{Maar ruw drukte ik ':}{m vlak tegen m'n beenen aan}{en zei heethijgend}\\

\haiku{maar zeg, ga d'ris even, ' '.}{kijkent is net of ze}{opt raam tikken}\\

\haiku{boven heb ik geen,.}{deel van leven meer ze doen}{daar niks as schelden}\\

\haiku{en we nemen wel...,,?}{goed afscheid vannacht voorgoed}{geloof u ook niet}\\

\haiku{------------- Maar rustig werd 't,.}{dan toch ze waren stil gaan}{liggen met mekaar}\\

\haiku{ik hou heel veel van,...}{d'r en ik weet gelukkig}{maar weinig van d'raf}\\

\haiku{ze h\`et in de Pijp,...}{gewoond zegt ze en es heeft}{ze niet dat weet ik}\\

\haiku{{\textquoteright} {\textquoteleft}Bij m'neer Driesse......,.}{ja ik kom dadelijk even}{m'n bloessie andoen}\\

\haiku{{\textquoteleft}Ziezoo, nou zijn we......{\textquoteright} {\textquoteleft}....}{d'r ik blijEn ik verdomd}{beroerd voor Overhoff}\\

\haiku{En lange dagen '.}{bleef ik bijm lijdend zijn}{onduurzieke onrust}\\

\haiku{Sprak niet tegen ons,, '.}{vroeg alleen maar wat geld als}{zet noodig had}\\

\haiku{Tegen het einde.}{van zijn boek noemt De Haan de}{Hoedemakersstraat}\\

\haiku{De naam van deze.}{straat werd later veranderd}{in Kuipersstraat}\\

\section{Hadewijch}

\subsection{Uit: Brieven}

\haiku{Misschien zou er met.}{den tijd wat meer licht in die}{duisternis opgaan}\\

\haiku{Een omgekeerde {\textsection}.}{D als paragraafteeken wordt}{hier door aangeduid}\\

\haiku{wat Gods goedheid hun (-).}{daar als wezenheid schenkt is}{hun waarheid1017}\\

\haiku{- De minnende ziel.}{wordt deelachtig gemaakt aan}{de klaarheid des Zoons}\\

\haiku{Daarom ook is zij,:}{niet alleen geroepen maar}{ook uitverkoren}\\

\haiku{Vooral zij die nog,}{jong is moet zich oefenen}{in alle deugden}\\

\haiku{zij verzekert haar:}{dat zij tot de Liefde}{uitverkoren is}\\

\haiku{omdat hij in zich,.}{heeft alles wat betaamt wat}{recht en rede is}\\

\haiku{Sijt blide altoes.}{in hope    om minne}{te vercrighene}\\

\haiku{Sijt op uwe hoede.}{ende in vreden van ||}{allen4041   dinghen}\\

\haiku{tegen een al te;}{stipte en te bepaalde}{levens-regel}\\

\haiku{Daer toe en constic.}{v niet bringhen dat    ghi}{mate daer ane hielt}\\

\haiku{Want de brief schijnt aan:}{te vangen te midden van}{een uiteenzetting}\\

\haiku{dat doet een deel der:}{trouwen gront dan herneemt zij}{deze uitdrukking}\\

\haiku{Er is hier nergens,.}{spraak van ontrouw nog minder}{van opstandigheid}\\

\haiku{De onvolmaakten;}{maken er hun liefdedienst}{afhankelijk van}\\

\haiku{Zij schikt en ordent,;}{zij drukt zich praegnanter en}{kernachtiger uit}\\

\haiku{als fragment G een.}{afschrift van den 6en en den}{10en brief volgens hs}\\

\haiku{Sine haken niet,.}{na smake Mer   335 si}{soeken orbere}\\

\haiku{tot de ontleding,:}{van een stemming waarin zij}{dikwijls verkeerd heeft}\\

\haiku{de overste zal doen;}{als Joseph die zijn broeders}{hoedde en leidde}\\

\haiku{zonder zich ooit op,}{zijn dienst te verheffen zal}{hij er naar streven}\\

\haiku{Liefde zoo trouw te.}{dienen is reeds het eeuwig}{leven beginnen}\\

\haiku{De MOTIEVEN die:}{daarbij ontwikkeld worden}{zijn voornamelijk}\\

\haiku{- Hier treffen wij de:}{grondgedachte aan van den}{strijd met de Liefde}\\

\haiku{de eeuwigheid die (-).}{de ware Liefde aan de}{ziel reeds schenkt1352}\\

\haiku{Mande heeft den brief;}{opgenomen tot r. 75}{tamelijk verkort}\\

\haiku{{\textquoteleft}blijdschap wordt ontsierd{\textquoteright}.}{door het vertrouwen op de}{duurzaamheid er van}\\

\haiku{zij kunnen er mooi,:}{over praten maar er zit wat}{anders in dan God}\\

\haiku{Ay en eest dan niet}{vrese- 424 425    lec}{roef dat wi vore}\\

\haiku{456 Dat ghenoeghet der,}{Minnen alre best datmen457}{458 te vollen}\\

\haiku{Aan het slot dan staat-,.}{r. 2531 om volgenden}{brief in te leiden}\\

\haiku{Want watmen dade,.}{buten   caritaten}{dat ware al niet}\\

\haiku{Hier omme steet wel,}{dat elc    mensche besie}{de gracie ende}\\

\haiku{Ende noch vraghet om,}{den wech dien die bi v}{502 sijn ende dien}\\

\haiku{om Minnen ere    .}{verdraghet den erren ende}{den onwetenden}\\

\haiku{geen menschelijke ().}{taal kan hemelsche dingen}{uitdrukken-122}\\

\haiku{Zoo werkt de ziel, door:}{de Liefde overheerscht en}{met Haar vereenigd}\\

\haiku{een prachtig beeld van:}{de hoogste waardigheid der}{ziel tegenover God}\\

\haiku{Ook in het Rike ().}{der Ghelieven wordt er op}{gezinspeeldc. XXV}\\

\haiku{de Volmaaktheid schenkt.}{aan de ziel het souverein}{gebied over haar zelf}\\

\haiku{Merk op dat er van.}{r. 69 af spraak is van de}{gansch onthechte ziel}\\

\haiku{ib. C. XLVI, blz.-,;}{9798 ook reeds C.XXXVI over}{Christus aan het Kruis}\\

\haiku{Hi velde sine,}{substancie dat was sinen}{heile-    ghen}\\

\haiku{wesene van.}{gode in een gheheel856}{gebruken comen}\\

\haiku{En toch schijnt deze:}{brief met den volgenden nauw}{verbonden te zijn}\\

\haiku{Ende alle die.}{v    belieghen die en}{weder segghet niet}\\

\haiku{Ay gheuoelt ende}{verstaet hoe    gherne ict}{saghe || ende}\\

\haiku{De gevoelens der ().}{ziel die met de Personen}{wandelt-64}\\

\haiku{toepassing van de?}{leer der circuminsessio}{personarum}\\

\haiku{Ik vlei me niet met:}{de hoop alles duidelijk}{te hebben gemaakt}\\

\haiku{God es mi met    .}{den vader gheheelleke}{met verweentheiden}\\

\haiku{dan komt 177 tot het,.}{einde dat ook het einde}{van zijn werkje is}\\

\haiku{Hier omme werden}{wi ghelettet in allen}{sinnen1106   Ende}\\

\haiku{Ende hier omme.}{en can nieman ande-}{ren ghehulpen}\\

\haiku{ieder vangt aan met,;}{een vers uit het Hooglied dat}{daarna verklaard wordt}\\

\haiku{Want  onse heer}{sedij selue eenen mensch Dat}{|| gerecht gebet1130}\\

\haiku{Die reden en can.}{god niet gesien dan in dat}{dat hi niet en is}\\

\haiku{Ende dat is daer,.}{om want si noch niet beset}{en sijn mit doechden}\\

\haiku{Want hi is god der.}{mynnen ende bekent wel}{die noet van mynnen}\\

\haiku{bijw. alleen, eeniglijk, -,.}{b.v. 14 39 e.e. niet 10 103}{e.e. alleinskine}\\

\haiku{soe sconen (schoon een),.}{beghin hebben zoo schoon}{beginnen 30 181}\\

\haiku{- znw. hare - n lien,,,,.}{dat waarover men beschaamd is}{schaamte 24 44 60}\\

\haiku{vgl. 3, 3 waar in.}{soortgelijk verband staat der}{heylegher doghet}\\

\haiku{- znw. vr. \'e\'enheid, het,,;}{in zichzelf vereenigd zijn}{b.v. van God 22 117}\\

\haiku{- der waerheit de,;}{genotvolle ervaring}{van wat is 15 14}\\

\haiku{bijw. naar wijze der;}{goddelijke natuur in}{den Vader 1 28}\\

\haiku{int - in het openbaar, ' (),.}{int algemeenin een}{gemeenschap 12 146}\\

\haiku{in ons selves - in, -,.}{eigen welbehagen 6}{286 na uwe 16 47}\\

\haiku{metten goede Gods,,,;}{met goddelijke dingen}{genade 10 67}\\

\haiku{- znw. vr. ook gratie,,,,;}{bovennatuurlijke gunst}{bijstand 10 13 61}\\

\haiku{- e ende ertsche,,;}{wat in den hemel en op}{aarde is 27 17}\\

\haiku{- maken een machtig, (),.}{geluid geven krachtig zijn}{van een stem 20 106}\\

\haiku{- znw. vr. {\textquoteleft}alreley{\textquoteright} (.);}{dair men wat af maken off}{besynnen kanTheut}\\

\haiku{- bnw. waarvoor geen naam,,.}{bestaat die geen naam hebben}{20 5 en daar pass}\\

\haiku{- gevolgd door inf. hem -,,;}{laten te iets verwaarloozen}{nalaten 4 42}\\

\haiku{dat ons ontbleven,;}{es waarin wij zijn te kort}{geschoten 6 337}\\

\haiku{- onverhaven bnw.,.;}{non elatus niet verheven}{22 22 en daar pass}\\

\haiku{hem - houden zich fraai,,,.}{onberispelijk houden}{gedragen 24 73}\\

\haiku{bi scoude ende,;}{bi rechte als rechtvaardig}{verschuldigd 17 90}\\

\haiku{met - n onderstaen,,,,.}{helpen steunen door vrede}{te schenken 5 4}\\

\haiku{werde 12, 229 en,;}{enkele dubieuze}{gevallen 6 68}\\

\haiku{ghi sijt te weec van.}{herten ende te kinsch}{in al uwen seden}\\

\haiku{Een brief, de XXXIe, uit.}{vroegere jaren werd er}{aan toegevoegd}\\

\haiku{Over Willem van St..}{Thierry zelf hoeven we niet}{lang uit te weiden1159}\\

\haiku{eam funditus et,,;}{interimens sicut mors}{interemit corpus}\\

\haiku{En zoo hebben de.}{L.S. ook voor de 41e preek wel}{Hadewijch gekend}\\

\haiku{en men wijst daartoe,.}{op het soort syllogisme}{waarmee hij aanvangt}\\

\haiku{met welk recht wordt zulk?}{een onbekende grootheid}{hier verondersteld}\\

\haiku{wat ook den stijl van.}{het slot kenmerkt tegenover}{dien van Hadewijch}\\

\haiku{Hoe komt het, dat ze:}{in al deze plaatsen op}{Richard terugvalt}\\

\haiku{Zoo ook fr. E, dat.}{immers nog uittreksels heeft}{uit den IVn Brief}\\

\haiku{Ziel is een weg waar;}{God door vaart in zijn vrijheid}{van uit zijn diepte}\\

\haiku{en dat juist maakt het,.}{mystiek in den ruimeren}{zin van het woord}\\

\haiku{dat is de Minne,,;}{die in de klaarheid het licht}{staat van de Godheid}\\

\haiku{Minne is de groote, (,).}{kracht die geheel de Schepping}{aandrijftXII 15 vlg.}\\

\haiku{Men heeft zelfs beweerd,:}{dat het Verlossingswerk voor}{Hadewijch niet leeft}\\

\haiku{Nu eens nu anders,,,.}{nu lief nu leet nu hier nu}{doer nu af nu ane}\\

\haiku{Ghi sult u proeven}{hoe ghi verdraghen moghet}{al dat u mescomt}\\

\haiku{Indien Hadewijch,}{zegt dat zij God haar zonde}{reeds zal bekennen}\\

\haiku{het is de hartstocht.}{van den H. Paulus om met}{Christus te zijn}\\

\haiku{Vereenigd, ja, hart;}{in hart en ziel in ziel en}{lichaam in lichaam}\\

\haiku{scone houdet na.}{de wet ende volcomen}{alsoe alst behoert}\\

\haiku{hoeveel meer we nog,!}{van Hem zouden ontvangen}{indien wij wilden}\\

\haiku{Daarom omgrijpt de.}{Vader den Zoon en den H.}{Geest in zijn enich recht}\\

\haiku{van zuiver leven,;}{in de liefde wat op zich}{zelf hoogst vruchtbaar is}\\

\haiku{Zoo geniet God ons.}{in zich zelven en alle}{heiligen met ons}\\

\haiku{Voert zij tot waarlijk?}{mystieke ervaringen}{en verschijnselen}\\

\haiku{de ingestorte,.}{mystieke Liefdestorm en}{orewoet van Beatrijs}\\

\haiku{Het is alles bij,,.}{Hadewijch zou men meenen}{natuur geworden}\\

\haiku{, dan zou men kunnen,.}{meenen dat de vrede er}{niet lang heeft geduurd}\\

\haiku{Heeft de hoogere?}{geestelijkheid zich ooit met}{het geschil gemoeid}\\

\haiku{zij heeft zich niet met,;}{hun kleederen getooid noch}{met hun verf en schijn}\\

\haiku{Ook den vreemden moet.}{wel wonderen van hare}{ende eysen}\\

\haiku{2Zie Inleiding op,.}{de Visioenen eerste}{en tweede hoofdstuk}\\

\haiku{hem selven ende,:}{allen creaturen =}{ten bate van voor}\\

\haiku{deen es vore 36}{dankes 37 B bezien 38 A}{glorilecheden}\\

\haiku{76 nemen met dank,.}{aanvaarden zonder er naar}{gezocht te hebben}\\

\haiku{130In allerlei te,.}{willen doen of vermijden}{wat vrijheid belet}\\

\haiku{119 seldi 120 B.}{zueten A haerre B}{harer 121 B warh}\\

\haiku{huedene 83 A.}{sine gratie B zijn B}{bezie84 B goet i.r}\\

\haiku{35829 b. 3592 ane B (.}{begert A gem. 3 haddic}{4 B noeytyt v.l.h}\\

\haiku{14 sonderlinghen?}{als comparatief gevoeld}{en daarom met dan}\\

\haiku{men..dech 30 soe dat;}{31 vanden B also 32}{dattene niemen}\\

\haiku{, zie 27 49 gronde (,.}{metdoch in A punt onder}{e in B doorgeh}\\

\haiku{) veruolghet Alsoe ();}{si nieti r. v. 50 no}{B gen. 52 B gen.}\\

\haiku{55 gheminneu 57.:}{A ghedincken 58 B}{so ongheeren}\\

\haiku{141 die (nose, nl.,):}{leed verdriet wij niet kunnen}{te boven komen}\\

\haiku{d.l.h.) egregie B;}{die eyghene d. 3 A}{heilegher B gen.}\\

\haiku{ververschen sonder,.}{vertraghen maar tot r. 43}{ontbreekt weer alles}\\

\haiku{eensgezindheid van,;}{gevoel die even als gij naar}{de Liefde streven}\\

\haiku{B hoge A wan...}{ge 104 selt 105 gi selt B}{van allen op ras}\\

\haiku{66 welk een zuiver;}{theocentrische reden}{voor onze blijdschap}\\

\haiku{27 dat ander waert.}{beteekent den Vader in de}{eenheid der natuur}\\

\haiku{onstech36 B dingen,.}{A ghebrukene in B}{is o uit u verb}\\

\haiku{of bet. dit = zij?}{die om hun zonden reeds ter}{helle behoorden}\\

\haiku{96-100 men ziet in:}{welken zin deze woorden}{moeten opgevat}\\

\haiku{naturlec 81 B;}{zien A twee 83 B gesien}{85 A gaet B ghaet}\\

\haiku{B vrems zeden 137:}{A na de werdecheit B}{nader werdecheit}\\

\haiku{niemanne 11 hen (.}{haren 12 B bezech wort}{in A iets onduid}\\

\haiku{De eerste is het.}{leven volgens Zijn eigen}{goddelijk wezen}\\

\haiku{B ia ende elc;}{wesen 353 A gheghoten}{es B ghegoten}\\

\haiku{met verw. naar hier).}{26 in confidentia opus}{871XLVJ-b}\\

\haiku{wat het is voor een.}{jonge herte Liefde te}{moeten ontberen}\\

\haiku{A te verwinne.}{B te verwinnen 9111 A}{iet 2 al ontbr}\\

\haiku{zere A diene:}{33 ziere 34 zinen B}{vart 35 na scone}\\

\haiku{A beziet zine.}{A ziere A ziere 13}{ziere w. 957.xxviij}\\

\haiku{De geestelijke.}{dronkenschap is een gewoon}{thema der mystiek}\\

\haiku{43 nedere B}{zinne 44 sijnse 46 B}{genuechten 49}\\

\haiku{142 ende dat... is:}{een nadere bepaling}{bij volwassene}\\

\haiku{A godelec B.}{godlec zijn 215 dingen 216}{wat ons C huus i.r}\\

\haiku{245 beteren =;}{met verzwegen vw. af te}{leiden uit wesen}\\

\haiku{5 tot enen mensche,.}{klaarblijkelijk tot haar zelf}{in een visioen}\\

\subsection{Uit: Visioenen}

\haiku{anders maken de,.}{letters C en B bekend}{dat ze uit die hss}\\

\haiku{twee sterretjes, dat.}{ze boven of onder aan}{de bladzijde staat}\\

\haiku{hieruit wordt de Lijst.}{der Volmaakten aangehaald}{door de letter 1}\\

\haiku{4de Boom (71-79) groot,,:}{met vele takken die door}{elkander groeiden}\\

\haiku{/ dan eneghen mensche /.}{die221 222    gheboren wart}{seder dat ic starf}\\

\haiku{Hij zegt niet, hoeveel.}{discipelen noch hoeveel}{talen er waren}\\

\haiku{I, 102, 1) dat dan.}{gedacht wordt op het hoogste}{der aarde te zijn}\\

\haiku{nochtan bleuen}{wi effenne/.    Ende}{si sal volwassen}\\

\haiku{heden/ ende met;}{di comen351   marghen}{in hare rike/}\\

\haiku{dat si in mi /;359 /;}{Ende dat si di cont}{van minen monde}\\

\haiku{De drie hemelen.}{worden haar dan verklaard als}{een beeld der Godheid}\\

\haiku{gaeft/, Ende ic noch}{niet doen en    bekinde}{uwe volcomene}\\

\haiku{Aan het slot worden, ':}{drie zaken vermeld doort}{visioen beteekend}\\

\haiku{vijf wegen, de eene,;}{hooger dan de andere}{leiden naar den top}\\

\haiku{Zoo neemt men Christus.}{enich op en beoefent men}{de eneghe Minne}\\

\haiku{Zij heeft Jezus in.}{zijn Menschheid en Godheid}{ten volle beleefd}\\

\haiku{Wat eunustus mag.}{beteekenen ben ik niet te}{weten gekomen}\\

\haiku{Eunuchus nu moet in:}{verband gebracht met het woord}{des Zaligmakers}\\

\haiku{een grauwen, jongen,,.}{en een blonden ouden met}{nieuwe vederen}\\

\haiku{Eindelijk heeft God;}{haar geleerd volcomene}{fierheit van Minnen}\\

\haiku{Als men nu overweegt,:}{dat God na te volgen in}{zijn Menschheid is}\\

\haiku{Dit kleed was gesierd,:}{met al die deugden die er}{hun naam op hadden}\\

\haiku{De uiteenzetting:}{ervan gaat duidelijk uit}{van het Schriftuurwoord}\\

\haiku{de zegels, die ze ',:}{langs buiten mett aanschijn}{verbonden waren}\\

\haiku{Sich wiltuus alsoe /}{voert meer ghebru-1110   ken}{alse ic Soe moestu}\\

\haiku{Verder wordt kort de;}{hoogheid en de klaarheid van}{dien troon aangeduid}\\

\haiku{want van een troon, of,;}{althans van een zetel wordt}{wel meer gesproken}\\

\haiku{Dat men alle dinc}{dore die claerheit van}{dien1153   trone sien}\\

\haiku{slechts 29 voor al de,!}{verloopen eeuwen maar 56}{voor haar eigen tijd}\\

\haiku{schreef, 6 liepen nog,.}{spelen en 5 zouden nog}{geboren worden}\\

\haiku{Weinig meer is er.}{te halen uit de meeste}{andere namen}\\

\haiku{naar verhouding, veel.}{minder namen van vrouwen}{noemt dan van mannen}\\

\haiku{Maar ook die kunnen,.}{kloosterzusters geweest zijn}{althans de maagden}\\

\haiku{verleyse ende.}{mijn vrouwe1328  nazaret}{kindese wel}\\

\haiku{afstaen enen -, zich op,,,;}{afstand van iem. houden hem}{ontrouw zijn 1 369}\\

\haiku{met - n werken met,,;}{de werken van innige}{vurigheid 7 19}\\

\haiku{F   familie,,,,,,.}{gevolg hofbedienden 9}{42 fenix 11 30}\\

\haiku{e.e. gheanxend, - der,,,,,.}{Minnen die in angst kwelling}{zich oefent 14 175}\\

\haiku{ghetrouwen, iem. iets,,,.}{toekennen daartoe in staat}{rekenen 8 91}\\

\haiku{hoe hi onse - in,,,;}{hemselven es ende ute}{hemselven 14 139}\\

\haiku{redeneerend verstand,,,,;}{en wat daardoor voortgebracht}{wordt 6 71 72 92}\\

\haiku{het Proza door Prof. (,.}{Dr. J. Vercoullie in 1895}{als n. 11 4e R.)1359}\\

\haiku{datten nieman soe.}{hertelike ghemint en}{heuet alse ic}\\

\haiku{Al heeft dus B bij '.}{t corrigeeren waarschijnlijk}{nog een ander hs}\\

\haiku{A telkens eke~,,,.}{C 10 ech A 9 ech 1}{ich in heilichBr}\\

\haiku{Men mag zeggen dat:}{in de rijmen de i klank}{veruit overheerscht}\\

\haiku{Door die schrijfwijze,,.}{ontstaat soms verwarring als}{goeds = goods godes}\\

\haiku{Enkele malen,, (.}{in A meer dan in C staat}{waert = woertB.v. Br}\\

\haiku{12, 98), enz. Deze.}{vormen worden gewoonlijk}{als ouder beschouwd}\\

\haiku{minne sout altoes,.}{thar wel duen waar E zoo}{schrikkelijken heeft}\\

\haiku{Daar is geen enkel.}{punt van overeenkomst in met}{E of F. ~ VI}\\

\haiku{Het kan zijn dat ik;}{hier en daar eenigszins anders}{had kunnen schikken}\\

\haiku{de redenen, die,.}{hij er voor aangeeft lijken}{me weinig dwingend}\\

\haiku{Dit voert mij tot een..}{ander bezwaar tegen de}{emendaties van Str}\\

\haiku{De algemeene:}{beschouwingen willen dit}{eenigszins toelichten}\\

\haiku{wat onwaarschijnlijk,.}{is daar ze niet aan \'e\'en stuk}{werden geschreven}\\

\haiku{Te voren nog had ();}{Hadewych gesproken van}{zijn dooddiere doet}\\

\haiku{wij moeten niet zoozeer,.}{Christus navolgen als wel}{Christus beleven}\\

\haiku{Dit zijn de twee groote,;}{mystieke toestanden die}{Hadewych vermeldt}\\

\haiku{noodzakelijke.}{eenheid van wil in minnen}{en haten met God}\\

\haiku{hare eenheid met,.}{hen die ten volle moeder}{Gods geworden zijn}\\

\haiku{de aandacht, omdat.}{daardoor het proza en de}{po\"ezie van Had}\\

\haiku{Een photographie.}{van dit zonnebeeld moet dit}{duidelijk maken}\\

\haiku{verklaard worden het.}{veelvuldig gebruik van het}{woord nuwe bij Had.1479}\\

\haiku{een nieuw sap vloeit door,.}{hun vezels en aderen een}{nieuw leven ontspruit1480}\\

\haiku{Et cum vidissem,.}{eum CECIDI AD PEDES}{EJUS tamquam mortuus}\\

\haiku{Sine HOEFT was groet;}{ende wijt ende KERSP van}{WITTER vaerwen}\\

\haiku{en dat zij niet naar,}{visioenen streefde niet}{in visioenen}\\

\haiku{In God is ook 's,:}{menschen ware leven van}{alle eeuwigheid}\\

\haiku{daar leven wij Gods,.}{eigen leven mede en}{zijn wij \'e\'en met God}\\

\haiku{- verderft helschen, hoedt,,;}{en voedt aardschen verheerlijkt}{hemelschen l. 68}\\

\haiku{{\textquoteleft}die terugkeer moet;}{gebeuren langs denzelfden}{weg als de uitgang}\\

\haiku{uitwerkselen, maakt,,;}{de ziel waardig om God te}{ontvangen 12 82}\\

\haiku{na zijn visioen,,;}{op Thabor heeft hij nooit meer}{gelachen 14 90}\\

\haiku{Op een Sinxendag {\textquoteleft}zong{\textquoteright} (,).}{men metten in de kerk en}{ik was daar7 1}\\

\haiku{Zij woonde niet in, {\textquoteleft}{\textquoteright}.}{een klooster dan warein}{de Kerk overbodig}\\

\haiku{De heeren van Diest.}{waren nauw verwant met de}{heeren van Breda}\\

\haiku{Uit een studie nu ':}{van dien katalogus van}{t Rooklooster blijkt}\\

\haiku{welnu ook deze.}{namen van de tweede laag}{komen daarin voor}\\

\haiku{Doch van veel grooter.}{belang is een andere}{plaats die in het hs}\\

\haiku{Uit die lofrede,.}{weten we mede niet slechts}{wat de kok over Had}\\

\haiku{De ipso Christo}{Sanctorum Sancto quidam}{animae illucens}\\

\haiku{Dit feest werd door Paus ({\textdagger});}{Joannes XXII 1334 voor heel}{de Kerk ingevoerd}\\

\haiku{in de rekening,,.}{rekening houdende met}{naar de mate van}\\

\haiku{vrees, bezorgdheid, niet;}{genoeg denzelfden dood te}{sterven als Christus}\\

\haiku{bij het uithouden, ().}{bij het volhardenvgl. te}{beghinnene}\\

\haiku{Men kan ook denken,}{aan het mystieke Lichaam}{van Christus tot wien}\\

\haiku{gij zult nog wel een ();}{beetje zulk een levenvan}{lijden doormaken}\\

\haiku{de lezing van C.,.}{geeft ook wel een zin maar dunkt}{me toch minder juist}\\

\haiku{enechleke vat ik:}{eerder op van den toestand}{van de zienares}\\

\haiku{309al vol onsteken.}{geheel deze voorstelling}{is apocalyptisch}\\

\haiku{dierre A omvaen (..):}{uit ontf 43 sinen 44 Doe}{45 ende alle}\\

\haiku{A gheheelre B.}{geheelre 53 B euweleke}{C nuwe ontbr}\\

\haiku{zij mag, zoo ze er,.}{eenig verschil in ontwaart den}{heerlijksten kiezen}\\

\haiku{de engel heeft haar,.}{zoo volkomen gemaakt dat}{zij niet meer twijfelt}\\

\haiku{A haer B har 83.}{sterke 345te sine of}{te siene ontbr}\\

\haiku{385na ons beide () (:}{n ghelijc = gelijk elk}{van ons beidenzooals}\\

\haiku{) 22 gene B zijn;}{23 A gratie B gescie}{25 B arbeyte}\\

\haiku{worden.dies van iets, ().}{om ietseenige verdienste}{dat ze niet hebben}\\

\haiku{is feitelijk een;}{verklarende herhaling}{van wat voorafgaat}\\

\haiku{samen (niet met Jc,):}{god ende mensche dit toch}{moet Had.'s leven zijn}\\

\haiku{hij had te lettel:}{minnen met affectien wat}{wel zal beteekenen}\\

\haiku{alle nakende;}{vervolgingen en smarten}{kunnen overwinnen}\\

\haiku{pen) ende heeren (.}{A heerscappentusschen die}{en ew geras}\\

\haiku{A auriolas (.)}{43 B geheele 44 B}{zijn B hierop ras}\\

\haiku{Wie tot die kennis,,.}{komt is zooals in 8,88 vlg. op}{weg ter heiligheid}\\

\haiku{hoe selke schinen.}{dolende ende nye ure}{daer ute en quamen}\\

\haiku{heilegen 123 A (.}{heilighen B heilegen}{B elkenop ras}\\

\haiku{811Dit is misschien,.}{ook een reden waarom ze}{niet \'e\'en met s. Aug}\\

\haiku{Daarin heeft zij nooit,.}{verpoozing volle vreugde en}{zaligheid gehad}\\

\haiku{ongrondelecheit;}{85 onderscedecheit A}{oercontse}\\

\haiku{bevestigd (in de).}{goddelijke Liefde door}{omgang met haar Zoon}\\

\haiku{door de middelste.}{vleugelen de geheele}{Minne aanschouwen}\\

\haiku{Hierdoor wordt misschien,.}{uitgedrukt dat de zeven}{gaven van den hl}\\

\haiku{223 B hogere ():}{A trouwe B geh. 224 B}{gewoutu uit n}\\

\haiku{1148sine zonder,.}{dat zij mij tormenten en}{mij niet tormenten}\\

\haiku{onder de Almacht,.}{der Liefde zitten rusten}{alle zaligen}\\

\haiku{De vriend had alles.}{over hare visioenen}{willen vernemen}\\

\haiku{Een populaire.}{heilige Brigida in}{de ME. was de hl}\\

\haiku{Men vindt, zegt hij, een.}{dergelijke signatuur}{in een aantal hss}\\

\haiku{Over de communie:}{onder beide gedaanten}{nog het volgende}\\

\section{Ben Haes}

\subsection{Uit: Moord op kasteel Valkensweerd}

\haiku{{\textquoteleft}Hendrik de Eerste{\textquoteright}!}{had ie zo maar voor de grap}{op bureau gezegd}\\

\haiku{{\textquoteright} Over 't gezicht van.}{den verdwijnenden agent gleed}{een air van gewicht}\\

\haiku{{\textquoteleft}We hebben maar niets{\textquoteright},.}{aangeraakt fluisterde de}{Indische dokter}\\

\haiku{Zo lang 't mooie weer '.}{aanhoudt blijven we nogn}{beetje hier hangen}\\

\haiku{Je weet wel h\`e, je,,!}{roept wat ik was helegaar}{in de war meneer}\\

\haiku{{\textquoteright} En de oude knecht,:}{keek den Inspecteur aan of}{hij zeggen wilde}\\

\haiku{{\textquoteleft}Je kunt tenslotte, '!}{nooit weten al ist maar}{voor de aardigheid}\\

\haiku{{\textquoteright} {\textquoteleft}Bracht de Jonker z'n '?}{nicht naar buiten of kwam ze}{all\'e\'ent huis uit}\\

\haiku{As ik ze tusse,,}{me klauwe krijg  zeg ik}{al tege Netje}\\

\haiku{Secretaris van.}{de directie der Hanborg}{Wapenfabrieken}\\

\haiku{{\textquoteleft}Tjonge, hoe is 't,.}{mogelijk z\`o iets moois heb}{ik nog nooit gezien}\\

\haiku{En toen is freule '?}{Julien poos later weer}{teruggekomen}\\

\haiku{We volgen hier 't,.}{Indische systeem ziet U}{bonnetjes schrijven}\\

\haiku{Losse verbinding?}{met officiele Britsche}{spionnagedienst}\\

\haiku{Mag ik weten, hoe '?}{U achtert feit van de}{moord gekomen bent}\\

\haiku{{\textquoteright} {\textquoteleft}Heus meheer, op ons, '.}{woord van gentelmenne me}{hebbet niet meer}\\

\haiku{{\textquoteright} {\textquoteleft}Thans eerst maar eens op{\textquoteright},.}{zoek naar Daan zonder Naam}{besliste Bolman}\\

\haiku{{\textquoteright}, bromde hij, terwijl ' ...... {\textquoteleft},?}{hijt toestel bediende}{Hallo Speur nog nieuws}\\

\haiku{Ik zal U van mijn, -.}{kant niet vragen hoe U aan}{die wetenschap komt}\\

\haiku{snoepreisje naar de '? ......}{lichtzinnige Stad aan de}{Amstel ent Y}\\

\haiku{Hij moest ook tanken '.}{enn slappe achterband}{laten oppompen}\\

\haiku{{\textquoteright} {\textquoteleft}Ik vrees, dat U de.}{ernst der situatie niet}{voldoende inziet}\\

\haiku{Maar ik twijfel er '.}{niet aan of haar onschuld zal}{aant licht komen}\\

\haiku{En nu even kalm he, ......}{anders kan ik m'n eigen}{woorden niet verstaan}\\

\haiku{{\textquoteleft}Ik zal U alles{\textquoteright},.}{vertellen hernam ze met}{licht bevende stem}\\

\haiku{er naar verlangde, '.}{z'n lege leven mett}{mijne te vullen}\\

\haiku{Freule Dudam:}{streek met de hand langs haar ogen}{en vervolgde dan}\\

\haiku{Maar jij Julie, kijk,,.}{me aan jij bent gered jij}{zult gered worden}\\

\haiku{Zoek de Kraaiepoot en ',!}{je bent de Panter ook op}{t spoor Bolletje}\\

\haiku{Weten jullie ook? ......}{onder welke naam ie daar}{staat ingeschreven}\\

\haiku{{\textquoteleft}Ik heb gehoord, dat '. '}{hiern ouwe kennis van}{me moet logeren}\\

\haiku{Allemaal \'e\'en pot,.}{nat op zeker ogenblik brand}{je je de vingers}\\

\haiku{{\textquoteright} {\textquoteleft}Niets, al bied je me '{\textquoteright},.}{n ton en hij deed een greep}{naar de tafel}\\

\haiku{Voorzichtig trachtte.}{hij het gesprek op Poolse}{Maria te brengen}\\

\haiku{Het licht flitste aan,.}{de kamer onthulde haar}{obscene geheim}\\

\haiku{{\textquoteright} {\textquoteleft}In orde Bol. 't?}{Arrestatiebevel heb}{je toch in je zak}\\

\haiku{{\textquoteright} {\textquoteleft}We zullen 'm G. ......{\textquoteright},, {\textquoteleft},?}{m\'eriten vloekte Speurik}{neem de auto h\`e}\\

\haiku{{\textquoteright} {\textquoteleft}Een goed rechercheur ',.}{moet ook een beetje vann}{wijsgeer hebben Speur}\\

\section{Paul Haimon}

\subsection{Uit: Gudela}

\haiku{{\textquoteright} {\textquoteleft}Was Ruprecht niet een,,?}{knappe jongen papa knap}{naar alle kanten}\\

\haiku{Veel bekwamer dan,.}{alle molenwerkers die}{ik ooit gehad heb}\\

\haiku{mensen laten hun?}{veld in de steek en willen}{met een geweer gaan}\\

\haiku{{\textquoteleft}Nu zal ik voor het!}{vreemde tuig tenminste niets}{behoeven te doen}\\

\haiku{Hij, die Ruprecht had, '.}{doorzien maar haar niet voort}{hoofd wilde stoten}\\

\haiku{Ze gaat weer naar haar.}{kamer om zich te kleden}{voor een lange dag}\\

\haiku{Zij wil zich haasten,.}{om het zeker te zien en}{het toch niet te zien}\\

\haiku{{\textquoteleft}Ai-ie, Fr\"aulein,{\textquoteright} en.}{ze laten de geweermond}{naar haar toedraaien}\\

\haiku{Hij kijkt lang en naar.}{alle kanten en dan draait}{hij zijn paard en wenkt}\\

\haiku{Hij rijdt rechtdoor, en.}{iedereen voorbij tot bij}{het gemeentehuis}\\

\haiku{Een meisje, dat op,.}{kostschool was geweest zag dat}{misschien maar alleen}\\

\haiku{Hij wenkte en dan:}{moet ze maar begrijpen en}{ze heeft begrepen}\\

\haiku{Ruprecht vereerde.}{de dood omdat hij een deel}{was van het leven}\\

\haiku{Het was of ze een,.}{wapen haalde om schielijk}{een aanslag te doen}\\

\haiku{Hoe kon zij nog aan,?}{iets anders denken dan aan}{haar eigen toestand}\\

\haiku{En eigenlijk was,.}{hij blij dat hij die Duitse}{knechten gezien had}\\

\haiku{het vaderland, dat,.}{nu was overgegeven was}{ook haar vaderland}\\

\haiku{Ja, dat doen ze, en.}{nu gooien zij hun bommen}{op mijn familie}\\

\haiku{Dan stond zij recht en,.}{keek de mensen aan waarvoor}{zij toch bang waren}\\

\haiku{En zij dacht, dat de.}{mens toch niet veel meer was dan}{een koe of een paard}\\

\haiku{Vader wil weten,.}{of ik in het geheim nog}{contact heb dacht ze}\\

\haiku{Het is of ik, sinds,.}{dat vliegtuig mijn naam draagt niet}{meer aard op de grond}\\

\haiku{hij gaat weer terug,.}{naar Compi\`egne als hij}{nog terug kan gaan}\\

\haiku{Het werd zo vreemd of:}{nu in haar lichaam alles}{reeds ging gebeuren}\\

\haiku{Het werd geen nacht, maar,.}{een wachten op een morgen}{die geen morgen werd}\\

\haiku{En zij hoorde weer:}{wat het laatste politiek}{gesprek was geweest}\\

\haiku{Hier hoor je niets van,.}{de oorlog dacht ze en ze}{liet ook het geweer}\\

\haiku{Heel het leven riep,.}{daar met de stem van iemand}{die haar nog lief had}\\

\haiku{{\textquoteright} en hij plantte een.}{paar stokken strak tegen de}{deur van de ingang}\\

\haiku{De wei werd als in:}{haar eerste groen gespreid rond}{hen en dan zei ze}\\

\haiku{De liefde kan men,.}{niet eens nadoen men kan ze}{nog niet vervangen}\\

\haiku{Hij toont haar aan ons,.}{soms gesluierd en soms in}{zilveren straling}\\

\haiku{Misschien ergeren,.}{zij zich daaraan het meest dat}{ze het zien kunnen}\\

\haiku{Christian heeft de,.}{brief in handen gehad maar}{hij kan niet lezen}\\

\haiku{{\textquoteright} ~ Nu zagen zij,.}{Grieta door de wei lopen en}{naar hen toe komen}\\

\haiku{Ik heb geen man meer,.}{nodig ik ben nooit aan \'e\'en}{man verslaafd geweest}\\

\haiku{Maar goed doorvoed en.}{gespierd al hadden ze dan}{geen boter gehad}\\

\haiku{{\textquoteright} De woudvrouw zwaaide.}{los en haast afwezig met}{haar hand ten afscheid}\\

\haiku{Hij dacht niet meer aan,.}{Herman die dit zijn leven}{lang zou onthouden}\\

\haiku{{\textquoteright} Haar moeder ging in,.}{de molen kijken zoekend}{of het meel er was}\\

\haiku{{\textquoteleft}Je liegt, o, je liegt,,{\textquoteright},.}{ik heb hem niet gehaald riep}{hij al rond lopend}\\

\haiku{Zich omdraaiend zag,.}{hij de ander naar binnen}{kijken naar zijn kind}\\

\haiku{{\textquoteleft}Ik heb nooit iemand,,.}{vermoord weet je maar vandaag}{zou ik het kunnen}\\

\haiku{In de kerk zou ik.}{misschien niet moe worden naar}{jou te luisteren}\\

\haiku{Was het misschien om?}{dit wat ik er later van}{terecht zou brengen}\\

\haiku{Hij had toch iets van.}{Ruprecht wanneer die over zijn}{nieuwe leer begon}\\

\haiku{Ze was ineens blij,.}{blij omdat deze man was}{zoals hij zijn moest}\\

\haiku{Ze was gelukkig,}{met dit voorbeeld van een mooi}{natuurlijk kind}\\

\haiku{diepe dingen van.}{het menselijk leven te}{kunnen bespreken}\\

\haiku{Er kwam een wilde.}{trek van teleurstelling in}{het gelaat der vrouw}\\

\haiku{De plek die ook de,.}{stroper had met een lijf zo}{koel als van een vis}\\

\haiku{{\textquoteright} {\textquoteleft}Het is Nicolaas,,.}{een boer die niet graag werkt en}{maar vast onderduikt}\\

\haiku{Misschien haalt hij wel,.}{een dokter die wil weten}{wie de vader is}\\

\haiku{Dan knipperde Grieta.}{en Gideon legde zijn}{steen op de slinger}\\

\haiku{{\textquoteright} Zijn gezicht ging open,.}{tot een grijns alsof er een}{masker overheen zat}\\

\haiku{Nu was zij alleen,.}{nog zijn zuster en niet meer}{in de eerste graad}\\

\haiku{Zij wees naar achter,,.}{naar een kleine dam door haar}{vader daar gelegd}\\

\haiku{Je komt vragen of.}{het goed is afgelopen}{met mijn ongeluk}\\

\haiku{Ze hield het niet lang,,.}{meer vol zag hij haar zware}{lichaam wankelde}\\

\haiku{Hij plaatste de lantaarn.}{op een verse hoop aarde}{en nam de schop vast}\\

\haiku{Misschien is het ook,.}{een gemene leugen want}{hij zegt de waarheid}\\

\haiku{{\textquoteleft}Neen,{\textquoteright} deed de jongen.}{alsof hij nu zijn vraag niet}{helemaal begreep}\\

\haiku{Het leek elkaar op,.}{te heffen als niet en niet}{dat vroeg en te vroeg}\\

\haiku{De auto's waren.}{reeds te zien en kwamen recht}{op de molen af}\\

\haiku{Hij was zo slecht niet,,.}{dacht ze weer hij is gewend}{te gehoorzamen}\\

\haiku{Als ik zou kunnen,.}{bidden wanneer het moet zou}{ik meer kunnen}\\

\haiku{Ze zag de rivier,.}{achter haar verglijden een}{weinig dof glanzend}\\

\haiku{Onder het oor was,.}{een bult en daaronder de}{kraag van een uniform}\\

\haiku{Die koude in haar,.}{steeg soms tot een gloeiend vuur}{dat haar verzengde}\\

\haiku{{\textquoteright} Dan knarsetandde,.}{de mond van de man die haar}{tot hier gebracht had}\\

\haiku{Jij bent de dokter,.}{niet jij bent Nicolaas en}{je bent weggegaan}\\

\haiku{Grieta is dat mens aan.}{de poort met haar geitenmelk}{en haar geitenstank}\\

\haiku{Waarom waren er,?}{zoveel achtervolgers die}{het wilden doden}\\

\haiku{Hij keek ook naar de,.}{zieke vrouw door de deur die}{was blijven openstaan}\\

\haiku{O, ik weet het al,.}{je staat reeds gereed hem aan}{hem over te geven}\\

\haiku{Hij begon nu te,.}{merken dat het bloed vlugger}{door zijn lichaam joeg}\\

\haiku{Twee ennen en een,.}{h de H van Hitler en}{de N van Nietsche}\\

\haiku{Ik heb hem horen.}{schreien en ze hebben hem}{niet meegenomen}\\

\haiku{In dat geval moet,{\textquoteright}.}{het kind uw naam dragen ging}{de dokter verder}\\

\haiku{Daarom zijn wij ook.}{zo blij na een verlossing}{met een goed einde}\\

\haiku{Dit is de stem van,.}{een die zeker vooraan in}{de kerk zit dacht hij}\\

\haiku{zo sterk als vroeger.}{vestingmuren er zouden}{hebben uitgezien}\\

\haiku{{\textquoteright} Gudela trok de;}{dekens nog meer om zich heen}{en staarde omhoog}\\

\haiku{Het was of hij met.}{de klauw van zijn hand een zwart}{duiveltje vastgreep}\\

\haiku{Maar ook andere:}{millioenen namen hem}{aan en doopten hem}\\

\haiku{En nog anderen,.}{namen hem aan en noemen}{hem rechtvaardigheid}\\

\haiku{Ze keek hem in het,.}{gelaat en zag dat hij op}{zijn lippen kauwde}\\

\haiku{Eduard had dat van.}{het kind gezegd voordat ze}{op zijn hoofd trapten}\\

\haiku{in ieder geval.}{is het geen schande wanneer}{men er geweest is}\\

\haiku{Waar het goed van eten,.}{en drinken was maar in het}{begin niet te veel}\\

\haiku{Ze begon met haar,.}{mond te trekken toen schudde}{ze neen met haar hoofd}\\

\haiku{Ik heb hem in het,.}{gezicht geslagen over zijn}{ogen en in zijn nek}\\

\haiku{Zij trok hem met een.}{schreeuw onder de resten van}{het voorhuis vandaan}\\

\haiku{Ik kon eerst in die.}{boom kruipen en dan naar de}{achterkant kijken}\\

\haiku{Het was zeker haar,.}{en zijn jongen een kind van}{vier jaar kon hij zijn}\\

\haiku{{\textquoteleft}Die ligt nu hier en.}{thuis wachten hem misschien een}{vrouw en kinderen}\\

\subsection{Uit: Jonkheid}

\haiku{het is vinnig koud,.}{en hij slaat met zijn handen}{onder de armen}\\

\haiku{Ginds komt al een troep,:}{herders de eerste van een}{heele karavaan}\\

\haiku{Hangt uwen lankrock voor, '.}{de wind Den voedstervader}{sorgt voort kind}\\

\haiku{Den lieven Jezus.}{krijt van dorst Zijn moeder geeft}{hem haere borst}\\

\haiku{Maer Joseph die was.}{heel verblijd Om dat het Kind}{niet meer en krijt}\\

\haiku{Vele anderen,:}{waren ook bang en wisten}{niet wat ze moesten doen}\\

\haiku{maar dat was nog zoo,.}{lang daar wilde zij liever}{niet meer op wachten}\\

\haiku{Het was bezaaid met,.}{een groen gewas waarover een}{man liep te eggen}\\

\haiku{Hij begroette oom.}{Gieljam hartelijk en ook}{Wale met warmte}\\

\haiku{Van daar uit gingen,.}{zij naar den stal waar het vee}{was ondergebracht}\\

\haiku{Uit de verte leek.}{zij heel mooi te zijn en haar}{tred was licht en vrij}\\

\haiku{Het jongste meisje.}{struikelde en viel hard met}{haar hoofd op den grond}\\

\haiku{Rein bekommerde,:}{zich daar echter niet om maar}{hij zei tot Wale}\\

\haiku{Het was misschien wel.}{daardoor dat hij altijd zoozeer}{leek op een diamant}\\

\haiku{Sint-Jan belooft mij,.}{veel ik zal vandaag nog meer}{gegeven krijgen}\\

\haiku{Hij was jong en blank.}{van gelaat met heel donker}{haar en wenkbrauwen}\\

\haiku{Die allen brachten.}{het er echter zonder groote}{ongelukken af}\\

\haiku{Ik dacht niet anders}{dan dat hij die daar speelde}{mij goed gezind was}\\

\haiku{Maar het was ook wel,.}{een zonderlinge zede}{die hier heerschte}\\

\haiku{Maar ik stond achter,.}{in de wei omdat ik pas}{laat was gekomen}\\

\haiku{Bij wijlen snorde.}{een tor of een kever met}{groote vaart den nacht in}\\

\haiku{- Niet goed voor u zou,.}{het zijn moest men u hier met}{mij te zamen zien}\\

\haiku{Hij wist nu geen woord,.}{meer te zeggen maar was ook}{de tranen nabij}\\

\haiku{hij zei echter geen,.}{woord maar keek daarna naar den}{kant waar Wale stond}\\

\haiku{Het werd heel stil toen:}{zij tot bij Rein en Wale}{ging en nu riep zij}\\

\haiku{{\textquoteleft}Ik zal altijd met,,.}{jou willen zijn Rein wat er}{ook mag gebeuren}\\

\haiku{Toen hij zich in zijn,,.}{bed uitstrekte merkte hij}{dat hij zeer moe was}\\

\haiku{De priester was reeds,.}{grijs aan zijn haren maar toch}{nog sterk en krachtig}\\

\haiku{Zij kropen beiden,;}{diep onder de mijt geheel}{tegen elkaar aan}\\

\haiku{Ze snelde toe, en.}{daar zag ze Rein weer verder}{draven op zijn paard}\\

\haiku{Neen, de kamillen,.}{deden het beest geen kwaad zij}{werkten zuiverend}\\

\haiku{Op het veld hing nog.}{een smoor en overal rook het}{naar gebrand koren}\\

\haiku{Ze sloeg haar armen.}{wild uit elkaar en toen nam}{haar vader haar vast}\\

\subsection{Uit: Het landgoed Solitudo}

\haiku{{\textquoteright} Johannes maande.}{hem dat hij voorzichtig moest}{zijn met zijn woorden}\\

\haiku{Dan zullen ze het.}{eerst denken aan hun vrienden}{onder de linde}\\

\haiku{Johannes die de,.}{bezoekers ontving was geen}{ogenblik onzeker}\\

\haiku{Johannes vergat.}{een ogenblik zijn honneursplichten}{toen zij bij hem kwam}\\

\haiku{Alsof vader door,.}{mijn blik gewaarschuwd was keek}{ook hij hun kant uit}\\

\haiku{Plotseling stond hij,.}{toen weer  op hij had nog}{haast niets genomen}\\

\haiku{Daar had hij plezier,.}{in die oude imker met}{zijn versleten toog}\\

\haiku{omdat hij niet eens;}{een poging had gedaan haar}{gerecht te proeven}\\

\haiku{Zebedeus was;}{weg en zou een vrouw die kon}{koken meebrengen}\\

\haiku{Kristie wees haar de.}{weg in het huis en hielp haar}{bij het uitpakken}\\

\haiku{Op onze vraag naar.}{haar naam antwoordde ze dat}{zij Philis heette}\\

\haiku{Op zijn kamer bleef.}{het licht naar Kristie's mening}{weer te lang branden}\\

\haiku{Nu denkt hij ook nog,{\textquoteright}.}{dat moeder hem is ontsnapt}{ging het door mijn hoofd}\\

\haiku{Hij had zeker nog,:}{het gordijn zien bewegen}{want opeens riep hij}\\

\haiku{Zebe zelf mocht mij.}{in opdracht van Johannes}{naar mijn bed dragen}\\

\haiku{{\textquoteright} Wist hij het nu, of?}{was hij nog verder weg van}{de werkelijkheid}\\

\haiku{Ik heb je gezegd.}{dat ook Johannes niet meer}{te vertrouwen is}\\

\haiku{Juist als na moeders,}{begrafenis liepen we}{met hem naar binnen}\\

\haiku{{\textquoteleft}Wat heb ik aan mijn,{\textquoteright},}{kinderen was zijn antwoord}{hij trok het zich aan}\\

\haiku{{\textquoteright} Onder die woorden:}{met potlood gekrabbeld had}{hij weer geschreven}\\

\haiku{Kristie zou er mij,.}{wel een vreugdig warm verslag}{van komen brengen}\\

\haiku{Dit was haar geschrift,.}{van gisteravond ze had iets}{met mij voorgehad}\\

\haiku{{\textquoteright} Zo bleef ze bezig}{en er kwam bovendien het}{een en ander uit}\\

\haiku{Dat glibberige?}{zuigdieren hun de lust uit}{merg en bloed zogen}\\

\haiku{Om Elza had op.}{Solitudo haast niemand}{zich zorgen gemaakt}\\

\haiku{Geloof jij ook dat?}{veel mannen door hun vrouwen}{groot zijn geworden}\\

\haiku{Wij doen niets dan hem.}{trachten te troosten met ons}{persoonlijk verdriet}\\

\haiku{Ik werd een feller.}{tegenstander van hem dan}{ik zelf vermoed had}\\

\haiku{Kristie fluisterde,:}{me nog toe terwijl ik naar}{een dik cahier zocht}\\

\haiku{De auto reed het.}{landgoed langzaam binnen en}{stopte bij de beuk}\\

\haiku{maar vroeg hij beleefd.}{het recept terug dat hij}{afgegeven had}\\

\haiku{Toen kwam hij weer bij,.}{hij herademde na een}{spookachtige droom}\\

\haiku{Ik ging op Kristie;}{toe die eerst als een versteend}{beeld was blijven staan}\\

\haiku{misschien ergert hem.}{dit en wil hij het dan zelf}{doen zoals het moet}\\

\haiku{Zebe knort om de;}{vele dorre takken die}{hij te ruimen krijgt}\\

\haiku{Nu heeft Johannes;}{hem gezegd dat hij spoedig}{geld moet verdienen}\\

\haiku{De eerste tijd moet,.}{hij beschrijven toen jullie}{gewonnen hadden}\\

\haiku{zijn gezicht wordt bleek,,.}{daarna rood van een grote}{spanning overtogen}\\

\haiku{We zien hem niet meer,.}{maar staren naar iets als uit}{een oud prentenboek}\\

\haiku{Misschien is hier een,{\textquoteright}.}{begrafenis nog mooi meent}{de vrouw van dertien}\\

\haiku{Als ik in de gang:}{kom zie ik het afscheid van}{vader en moeder}\\

\haiku{Ik zal mij nu al.}{met een bloemist verstaan die}{de hoogste prijs geeft}\\

\haiku{Het huis zal elke.}{vrouw te duur vinden als ze}{er nog geld bij krijgt}\\

\haiku{Ik wilde dat huis.}{echter als een present aan}{mevrouw aanbieden}\\

\haiku{Opeens horen we.}{vader met een stem die op}{een dreiging antwoordt}\\

\haiku{En daarom hebben.}{jullie de donderroede}{dus weggebroken}\\

\haiku{{\textquoteleft}Bemoei je nu ook.}{niet verder met de nieuwe}{kasteelbewoners}\\

\haiku{Ik vertrouw ze nog,{\textquoteright}.}{niet besluit vader als ze}{voorgoed weg zijn}\\

\haiku{Hij dacht bepaald iets,.}{heel slechts van mij alsof ik}{echt gevaarlijk was}\\

\haiku{{\textquotedblright} vroeg hij, ik huilde,.}{neen en hij vroeg wat ik dan}{van hem verwachtte}\\

\haiku{Zonder het moeras.}{kan het landgoed niet leven}{om zo te zeggen}\\

\haiku{Het was veel meer dan,.}{mooi het was vol majesteit}{om zo te zeggen}\\

\haiku{Bij de volgende:}{bocht houdt de chauffeur stil en}{vraagt aan zijn makker}\\

\haiku{Vanaf de wagen;}{grijpen de meisjes naar de}{zwarte vliertrossen}\\

\haiku{Nu is eindelijk.}{iets dat een succes voor hem}{zou worden mislukt}\\

\haiku{Hij steekt zijn handen.}{als vuisten even omhoog en}{slaat ze naar zijn hoofd}\\

\haiku{Het is de eerste.}{keer dat wij grote mannen}{zullen zien vechten}\\

\haiku{Hij schreeuwt en spartelt,.}{bijt mij in de handen dat}{het bloed eruit loopt}\\

\haiku{Het is of ik hem.}{naar haar zie lachen en zij}{die lach beantwoordt}\\

\haiku{Ik zie hoe de wind.}{tegen haar kleed slaat en haar}{figuur aftekent}\\

\haiku{Dat interesseert,.}{mij nog net maar dan is het}{ook genoeg geweest}\\

\haiku{Hebben ze je soms?}{ook geleerd hoe je een mooie}{vrouw moet verleiden}\\

\haiku{Zij hebben daarvoor.}{een Hubertusraam  aan}{de kerk gegeven}\\

\haiku{Dan rent moeder de.}{hut in en laat hem buiten}{voor de deur wachten}\\

\haiku{Zeker nog mooier,.}{dan toen ze kind was maar niet}{langer broos en vreemd}\\

\haiku{In haar ogen had ik.}{mij dus slechts geoefend voor}{mijn toekomstig werk}\\

\haiku{Alleen twijfel ik.}{eraan of je er ooit iets}{mee zult verdienen}\\

\haiku{Hij deed echter of;}{hij ons niet zag en verdween}{met grote stappen}\\

\haiku{was hetzelfde als.}{een mooie vrouw van haar schoonste}{sieraad beroven}\\

\haiku{Als iemand die zijn,.}{liefste verliezen gaat keek}{ik naar de eiken}\\

\haiku{Zij stonden er nog,.}{onbekend met wat er over}{hen werd bedisseld}\\

\haiku{Een gaai kon er in,,.}{dit seizoen niet zijn dan is}{het Zebe dacht ik}\\

\haiku{Alleen wil hij zich.}{zelf terugvinden tussen}{het geld en de macht}\\

\haiku{Ik zou notaris.}{moeten worden om voor jou}{een goudmijn te zijn}\\

\haiku{Ik zie een moeras}{dat bezig is zich over ons}{Goed uit te breiden}\\

\haiku{Ik beet eerst op mijn.}{tanden om de vraag nog}{tegen te houden}\\

\haiku{Johannes had weer.}{een papiertje met cijfers}{naast zich aan tafel}\\

\haiku{Ik geloof dat je.}{nog eens verliefd wordt op die}{boeren-Peter}\\

\haiku{Hij weet alles van,.}{ons veel meer dan je denkt als}{je hem bezig ziet}\\

\haiku{Stil, want Johannes.}{en Zebedeus mogen}{er niets van weten}\\

\haiku{Als hij het krijgt voor,.}{ons heb jij recht op een even}{groot deel als hij zelf}\\

\haiku{De vrouw trok zich aan.}{de kleren alsof zij ze}{wilde openscheuren}\\

\haiku{Het groene oog was,.}{het oog des verderfs dacht ik}{en wendde mij af}\\

\haiku{Als een lichtende,.}{kaars die beeft die zich verteert}{in haar binnengloed}\\

\haiku{Zo zou Johannes.}{ook gerekend hebben in}{mijn omstandigheid}\\

\haiku{In werkelijkheid.}{was er w\`el iets anders waar}{ik op moest letten}\\

\haiku{Ik wilde met haar,.}{meebidden maar ik had niets}{om aan te bieden}\\

\haiku{Het was of ik de.}{zwaarste boom uit het bos had}{horen neersmakken}\\

\haiku{{\textquoteright} {\textquoteleft}Zeg dat niet, Philis,,{\textquoteright}.}{er was nog niets gelukt kwam}{ik tussenbeide}\\

\haiku{5 Achter een beek.}{met een watermolen lag}{het bruine gebouw}\\

\haiku{Ze zwegen achter.}{mijn rug en ik voelde dat}{ze mij nakeken}\\

\haiku{Alleen aan zijn ogen,.}{zag ik iets vreemds alsof hij}{veel geleden had}\\

\haiku{De dokter zou het.}{nooit goed vinden dat ik u}{dit gesprek toestond}\\

\haiku{Mijn hoofd moet door een.}{lofwerk van spinnewebben}{als ik terugga}\\

\haiku{wilde ik vragen,.}{maar opeens waren we niet}{meer alleen binnen}\\

\haiku{In elk geval niet.}{om meneer De Roveren}{een plezier te doen}\\

\haiku{{\textquoteright} riep hij zo dat de.}{notaris het ook door het}{hout heen kon horen}\\

\haiku{De vogels die zo,;}{lang boven ons gecirkeld}{hadden vlogen weg}\\

\haiku{Wat kon Peter nog?}{meer van zijn historische}{dag hebben verwacht}\\

\haiku{Hij reikte hem mij,.}{over maar eigenlijk hoefde}{ik hem niet meer}\\

\subsection{Uit: Lentestorm}

\haiku{Waar is het rood dat,?}{verdween waar is het blauw dat}{duidt op een doode}\\

\haiku{De torens van den,.}{Wijngaardhof zijn ook rustig}{er waait geen vlag nog}\\

\haiku{Je hoort je ouden,,.}{geest die je zegt van toen van}{hoe het vroeger was}\\

\haiku{Maar Reinier van den,,.}{Branden knikt zijn zoon toe want}{hij weet dat het moet}\\

\haiku{De grijze priester,.}{die hen zegende beefde}{toen hij hen toesprak}\\

\haiku{Reinier begint van.}{het worstelen der jonge}{vrouw te vertellen}\\

\haiku{Geen moment slaat zij,.}{er acht op ofschoon hij ze}{niet kwaad gezegd heeft}\\

\haiku{ze kende immers.}{zoo Orbans gebaren en}{het tilde zoo licht}\\

\haiku{Zij waren vanzelf:}{stiller geworden in hun}{liefde-daden}\\

\haiku{{\textquoteleft}De ouders in een,{\textquoteright}:}{huis aan de kerk maar Machteld}{had hen verdedigd}\\

\haiku{Voor Machteld hoeft zij,.}{niet meer te vreezen zij is}{haar kamer niet uit}\\

\haiku{Er is even zoo iets.}{vreemds om zijn figuur en dat}{statige rijden}\\

\haiku{Nu hing alles van!}{dien koejongen af of hier}{zou gemaaid worden}\\

\haiku{Daar moest de boer weg;}{zijn en de oude boer te}{zwak om te maaien}\\

\haiku{Hij keek eens even op,,.}{naar de lucht naar de zon of}{zij nog niet weg ging}\\

\haiku{{\textquoteleft}Reinier,{\textquoteright} zuchtte zij,,.}{en staarde lang over hem heen}{als over haar leven}\\

\haiku{De meiden zeiden,.}{onder elkaar dat het nu}{wel mis zou loopen}\\

\haiku{Ze weet nu, dat ze,.}{zich dwaas gedragen heeft maar}{ze kon niet anders}\\

\haiku{Ze bleef er met heel:}{haar lichaam mee bezig en}{op eenmaal riep zij}\\

\haiku{Haar lichaam had het.}{willen inzuigen om haar}{ziel mee te voeden}\\

\haiku{{\textquoteleft}Heerlijk dier,{\textquoteright} dacht zij,.}{het wilde den nieuwen baas}{dus niet erkennen}\\

\haiku{'s Avonds komen de.}{witte avondnevels van den}{herfst om haar dwalen}\\

\haiku{eenzame uren, die....}{zij moest vullen met schoone}{vergeefsche beelden}\\

\haiku{Tot die op eenmaal,}{tezamen weefden om h\`em}{op eenmaal doortrok}\\

\haiku{En toen gebeurde,.}{alles wat in de Diepte}{maar gebeuren kon}\\

\haiku{zij was toen hij de}{koe geholpen had met het}{zware kalf en zij}\\

\haiku{Ze bijt haar tanden,.}{samen het is of ze op}{harde tranen bijt}\\

\haiku{{\textquoteright} {\textquoteleft}Moeder, kan Machteld,,{\textquoteright}.}{het helpen dat er oorlog}{is vroeg Walter dan}\\

\haiku{- Of zij vrouwen maar, ';}{namen zooals zij het eten tot}{zich nemens avonds}\\

\haiku{Zij werd heelemaal,.}{vervuld van iets zwaars iets dat}{grooter was dan zij}\\

\haiku{zijn liefde was met,.}{Frankrijk en zoo lang als dat}{was trof hem niemand}\\

\haiku{Ze slikte over haar,,:}{woorden heen een paar keer en}{dan ademde ze weer}\\

\haiku{Zou Godelief het,,?}{hebben afgestaan aan haar}{als het gekund had}\\

\haiku{{\textquoteleft}Het zal beter zijn,.}{als hij niet komt hij moet dit}{verdriet niet hebben}\\

\haiku{Zij heeft geprobeerd}{naar de grens te loopen om}{dan verder te gaan}\\

\haiku{Want haar hart schroeide.}{als werd het door twee sterke}{vuren aangetast}\\

\haiku{Meende Henricus,?}{met die nieuwe liefde dat}{de knecht haar liefhad}\\

\haiku{Ge weet het niet, oom,,?}{Henricus dat ik een knecht}{gedood heb om hem}\\

\haiku{telkens diep over het.}{lichaam of ze het met haar}{mond zou aanraken}\\

\haiku{{\textquoteleft}Het staat in de Schrift,{\textquoteright}.}{doch dan trok ze zich terug}{in een zijkamer}\\

\haiku{Hij had een diepe,.}{kap op zijn hoofd waardoor men}{zijn haar niet zien kon}\\

\haiku{Of zijn leger groot,.}{genoeg zal worden dat hij}{paarden kan krijgen}\\

\haiku{nu begint opnieuw,}{die geheime samenzweering}{en hij wenkte reeds}\\

\haiku{nooit, dacht zij, kwam hij ',.}{s nachts in dat zachte bed}{dat zij gemaakt had}\\

\haiku{Twee van haar moeten,.}{winnen en thans zijn beider}{oogen zwarte raadsels}\\

\haiku{iederen dag zag,,.}{zij den moord hier en thans thans}{brak hij uit dien mond}\\

\haiku{Men zei dat zij vuur,.}{konden eten maar dat ziet men}{wel wat zij kunnen}\\

\haiku{omdat Orban toch,!}{niet was gekomen terwijl}{hij toch vrij moest zijn}\\

\haiku{zoo ernstig en in,.}{de sfeer van zijn geval dat}{ze wel gelooven m\'oest}\\

\haiku{En staande naast zijn,,.}{paard met het hoofd geleund in}{den hals sliep hij in}\\

\haiku{de muren zijn stomp.}{als dateert hun verval van}{jaren geleden}\\

\haiku{zonder te spreken,.}{stond ze er maar z\'o\'o veel had}{zij hen nooit gezegd}\\

\haiku{het kind liet ze een.}{tijdlang huilen zonder dat}{ze er acht op sloeg}\\

\haiku{Maar dat reeds, dat hij,....}{bijna iemand gedood had}{dat reeds vond hij erg}\\

\haiku{Ik heb een mooie vrouw,.}{gezien wier man ik van het}{veld bij Leipzig droeg}\\

\haiku{Het was als in den,,.}{Bijbel Machteld en daarom}{ben ik weggegaan}\\

\subsection{Uit: De weg over de grens}

\haiku{Ook een weg over de,,.}{grens de duurste maar ook de}{gemakkelijkste}\\

\haiku{Ze werden er nog,.}{steeds rijker van en ook de}{trots zette goed aan}\\

\haiku{Je wende aan hun,,.}{taal aan hun discipline}{zelfs aan hun knevel}\\

\haiku{als er niemand was,}{en hun moeder niet op hen}{lette hadden ze}\\

\haiku{Als hij ze alleen,.}{laat gaan zal hij ze geen geld}{in hun zak geven}\\

\haiku{Plotseling ging ze.}{naar binnen en begon ze}{zich om te kleden}\\

\haiku{Bah!{\textquoteright} en de mannen.}{beginnen te schreeuwen}{als wild geworden}\\

\haiku{Zij kraaien vijftien.}{keer en dan zet de tater}{hen met moeite recht}\\

\haiku{Toen hij aan de balk.}{hing was zij trotser dan de}{haan ooit was geweest}\\

\haiku{of ben ik misschien?}{al verkocht en gebruikt hij}{mij als een slavin}\\

\haiku{Ik wou nu maar dat,.}{je die deur binnenkwam en}{dat je een paard had}\\

\haiku{Of hij dat gehoord,,.}{heeft vraagt Fr\"aulein Schaster en}{wat hij daar van denkt}\\

\haiku{{\textquotedblleft}Bitte, gn\"adige,{\textquotedblright},.}{Frau zei Drieka dan hoeft u}{dat niet meer te doen}\\

\haiku{Een verduiveld knap.}{wijfje heeft Peter Knarren}{als wijsvrouw gehaald}\\

\haiku{Peter ligt er naast,,.}{met vijftig groschen in zijn}{handen en snurkt}\\

\haiku{En hij zag dat een.}{Siegelbaron het met een}{kellner over hem had}\\

\haiku{Als het niet teveel,.}{is zullen we hem vragen}{dat hij ons laat gaan}\\

\haiku{En toen had hij zich.}{zo ver mogelijk van hen}{teruggetrokken}\\

\haiku{hij bevreesd dat zijn.}{eigen kinderen het hem}{ontroven zouden}\\

\haiku{Zij was de vrouw van,.}{Sep van Andr\'e geworden}{het kon raar lopen}\\

\haiku{Nu richtte ook Drik,.}{zich op om te tonen dat}{hij er ook mocht zijn}\\

\haiku{{\textquoteright} zei Peter nu en.}{de jongens keken hem met}{verraste ogen aan}\\

\haiku{Het was eigenlijk.}{te mooi geweest om na \'e\'en}{stuk gedaan te zijn}\\

\haiku{Zij zagen Sep die,.}{voorop liep een aanvoerder}{van een klein leger}\\

\haiku{Altijd was hij blind.}{en altijd was hij op tijd}{op de juiste plaats}\\

\haiku{Ze liep ermee naar,}{haar moeder het uit de korf}{tillend met een hand.}\\

\haiku{{\textquoteleft}Het zijn aardappels,.}{voor een varken die wij op}{kermisdag krijgen}\\

\haiku{Zij ging even naast Ria.}{zitten en legde haar een}{hand om de schouders}\\

\haiku{Hun broers kwamen thuis.}{en het eerst wat zij toonden}{waren hun handen}\\

\haiku{Roza had nog aan.}{een voorval deze middag}{niet willen denken}\\

\haiku{Ze had er niet om,.}{gehuild ze was er als door}{bevroren geweest}\\

\haiku{Hij was zo vrolijk,.}{en lustig als een jonge}{vrijer hoorde hij}\\

\haiku{Hij dacht na over wat.}{hij zou moeten doen en er}{viel hem niets meer in}\\

\haiku{Friedrich is niet.}{uit Duitsland gejaagd omdat}{hij heeft gestolen}\\

\haiku{Ze keerde zich met.}{de kandelaar in de hand}{om naar haar moeder}\\

\haiku{Eerst was ze er zelf;}{bij geweest en kon ze zien}{wat er gebeurde}\\

\haiku{Roza bleef over haar.}{vader gebogen en ze}{zag geen leven meer}\\

\haiku{{\textquoteleft}Je moet niet achten.}{dat ik zo iemand zijn}{widduwe naloop}\\

\haiku{{\textquoteleft}U moet willen dat,.}{hij leeft en voor u vecht en}{niet dat hij dood is}\\

\haiku{Wilde mijn hele.}{kop hebben om hem v\'o\'or op}{zijn kar te zetten}\\

\haiku{Hij greep naar zijn kiel,.}{stak de fles in zijn broekzak}{en verliet het veld}\\

\haiku{De broers geboden.}{Lotte dat ze zich niet als}{een wicht van elf jaar}\\

\haiku{Terwijl zij als een,.}{engel naar de kerk liep dacht}{ze aan wreedheden}\\

\haiku{Ze hield hem de soep.}{voor toen het scheen dat hij naar}{haar wilde grijpen}\\

\haiku{Immer zat,{\textquoteright} zei hij.}{en keek naar de andere}{kant dan waar zij was}\\

\haiku{anders was het Thies;}{altijd geweest die het bed}{voor hem opmaakte}\\

\haiku{Hij hield ze tegen,.}{zijn maagstreek alsof ze daar}{gekeurd moest worden}\\

\haiku{{\textquoteright} zei hij en hij schrok.}{een ogenblik omdat hij dat}{had durven zeggen}\\

\haiku{als haar vader ze.}{daar ging zoeken kon de man}{hem net ontvluchten}\\

\haiku{Op het gezicht van.}{Peter Knarren stond de angst}{in sterk reli\"ef}\\

\haiku{Hij voelde zich in.}{zijn recht aan de eigenaar}{van de hof gelijk}\\

\haiku{De boer stond met het,,.}{hoofd vooruit omdat hij te}{lang was in de deur}\\

\haiku{Emma knoopte haar.}{keurslijf je dicht en wilde}{de stal uitvluchten}\\

\haiku{Zichzelf aangeven.}{was soms even erg als iemand}{anders verraden}\\

\haiku{Hij gehoorzaamde,.}{op bevel niet omdat hij}{het er mee eens was}\\

\haiku{Het zal dan net zijn,.}{of ze voor ons luiden}{of we gaan trouwen}\\

\haiku{Dat was het lied van.}{de kermis als ze reeds meer}{dan een dag oud was}\\

\section{Ernest van der Hallen}

\subsection{Uit: Brieven aan Elckerlyc (onder ps. J. van den Wijngaerdt)}

\haiku{Draag het, indien het.}{voor uw werk of voor uw ziel}{noodzakelijk is}\\

\haiku{Wanneer de stof de.}{geest neerhaalt is de basis}{zelf der schoonheid weg}\\

\haiku{het venster zijner.}{ziel wijd open te werpen op}{het juichende licht}\\

\haiku{De opperste zin:}{en het uiteindelijk doel}{van het leven is}\\

\haiku{dat vele fouten;}{bedreven werden door hen}{die voorop gingen}\\

\section{Jacques Hamelink}

\subsection{Uit: Horror vacui}

\haiku{ze nog in leven).}{is weet ik niet waar ze woont}{en meneer Kobalt}\\

\haiku{De wagenwielen.}{knotsten en knersten over de}{ronde straatstenen}\\

\haiku{De wielen van de.}{kar rammelden daar knarsend}{en krakend overheen}\\

\haiku{Die zitten in hun.}{hol en horen toch wel waar}{het paard naar toe draaft}\\

\haiku{Daaronder was de.}{glimmende metalen knop}{van een riem te zien}\\

\haiku{Ik kon zien hoe ze,.}{slikte waarbij haar lippen}{iets vaneen gingen}\\

\haiku{Niet het geringste.}{ritselen van mijn kleding}{scheen haar te ontgaan}\\

\haiku{Het dier rekte zijn,.}{kop naar me toe en snoof}{onzeker leek het}\\

\haiku{Vertwijfeld schopte.}{en sloeg ik de geraakte}{poliep van me af}\\

\haiku{{\textquoteleft}Opgepast{\textquoteright} riep de.}{voerman met onnavolgbaar}{kwakende  stem}\\

\haiku{Af en toe gromde,.}{de voerman in zijn keel als}{een wolf hij zei niets}\\

\haiku{Wij zouden onze.}{vingers niet naar het toestel}{moeten uitsteken}\\

\haiku{Zijn stropdas hangt aan,,.}{een bedspijl zijn mouwen zijn}{los onopgerold}\\

\haiku{Ik zei dat ik het.}{doen zou als het de enige}{oplossing zou zijn}\\

\haiku{je bent ook gek net,{\textquoteright},.}{als allemaal net als die}{daar ze wijst naar mij}\\

\haiku{Hij is niet gerust,.}{er is een rimpel tussen}{zijn ogen gekomen}\\

\haiku{Dan kijkt hij naar me,,,.}{wil iets zeggen doet het niet}{knikt een paar maal kort}\\

\haiku{Nog dieper is de:}{zwarte sponsgrond die alles}{opzuigt en aanvaardt}\\

\haiku{Schijnbaar argeloos.}{preciserend met kleine}{haarfijne details}\\

\haiku{Een ochtendlijke,.}{nieuwsgierige kabouter}{gepensioneerd}\\

\haiku{{\textquoteleft}Ja{\textquoteright} zeg ik, {\textquoteleft}zeg me.}{welke ik moet gebruiken}{en hoeveel ervan}\\

\haiku{Ik word door de zon,;}{doodgestoken ruggelings}{val ik in water}\\

\haiku{Verder niet meer op,.}{de bedden overdag het is}{vaak genoeg gezegd}\\

\haiku{Zijn gezicht is nu.}{een glazig zwartstenen}{masker geworden}\\

\haiku{{\textquoteleft}dan schiet  ik je.}{met een gouden kogel en}{ik knip je snor af}\\

\haiku{De vijand at, aan,,.}{tafel met een arm op het}{blad om het bord heen}\\

\haiku{Daarna sloot ze de.}{balkondeuren en besloot}{een bad te nemen}\\

\haiku{Wat ze thans rook was.}{een tegelijk vager en}{penetranter lucht}\\

\haiku{Een ogenblik dreigde.}{mevrouw Siponelli in}{paniek te raken}\\

\haiku{Er was snachts een flink.}{pak sneeuw gevallen en het}{vroor dat het kraakte}\\

\haiku{Met het opknappen.}{van het huis bemoeide ze}{zich niet in het minst}\\

\haiku{{\textquoteright} Zijn stem had een vrij,.}{ironische klank wat ze niet}{scheen op te merken}\\

\haiku{Ze was toen op haar.}{vijftiende en als je goed}{keek nog steeds een kind}\\

\haiku{hoe die oom precies.}{aan die onderdelen kwam}{en waar hij woonde}\\

\haiku{En die had meer te.}{maken gehad met dingen}{als waarom het ging}\\

\haiku{Hij vroeg hun hoe het.}{met hun zaken ging en met}{het werk op het land}\\

\subsection{Uit: Het plantaardig bewind}

\haiku{Tenminste hij staat.}{met de armen omhoog naar}{de lucht te gillen}\\

\haiku{Ineens ren ik met.}{bemodderde voeten over}{kort fluwelig mos}\\

\haiku{Een rosse gloed rent,.}{laag over het water vreet zich}{in golven verder}\\

\haiku{Een soort walg om naar.}{beneden getrokken te}{worden bevangt mij}\\

\haiku{Bosneger, de Pok.}{en ik wendden voetenpijn}{en vermoeidheid voor}\\

\haiku{We hurkten bijeen.}{en zetten onze messen}{voor ons in het zand}\\

\haiku{Zijn handen wezen.}{iets aan ongeveer zo groot}{als een kokosnoot}\\

\haiku{Een geluid dat het,{\textquoteright}.}{maakte de stenen vlogen}{meters de lucht in}\\

\haiku{Het door kinderen.}{uitgegraven zand lag hoog}{om de zijkanten}\\

\haiku{Hij was krankzinnig,.}{maar op dat moment was hij}{er het dichtste bij}\\

\haiku{Er zwol iets in zijn.}{keel dat hij wegslikte en}{toen zat hij naast haar}\\

\haiku{(Hij had haar over zijn,.}{fantasie verteld ze had}{erom gelachen}\\

\haiku{Hij dacht vluchtig aan.}{de muur waarop de letters}{stonden geschreven}\\

\haiku{Morgenmiddag ga{\textquoteright}.}{je niet naar buiten hoorde}{hij nog achter zich}\\

\haiku{Hij bewoog alleen,.}{zijn lippen misschien zou ze}{hem ook zo verstaan}\\

\haiku{Haar haar was heel zwart,.}{hoog in dikke draaiingen}{om haar hoofd gelegd}\\

\haiku{Dan is het goed{\textquoteright} zei, {\textquoteleft}.}{zeje begrijpt de muziek}{beter dan wie ook}\\

\haiku{Hij durfde haar niet,}{aankijken alsof er nu}{iets tussen hen was}\\

\haiku{vrijdag Schichtig keek.}{hij door de deuropening van}{de school over het plein}\\

\haiku{Zijn pleegmoeder was,.}{alleen thuis en vroeg waar hij}{de schelp vandaan had}\\

\haiku{Een man die praatte.}{maar niet met de stem waarmee}{je gewoonlijk praat}\\

\haiku{Onder de tafel,.}{was een kleine vierkante}{koffer van wit leer}\\

\haiku{Mevrouw Daals was er,.}{niet over een paar dagen kwam}{ze weer zei tante}\\

\haiku{Het scheen hem nu veel}{minder iets zo gewoons als}{sigarettenlucht}\\

\haiku{Achter de deur was}{het zware monotone}{gepraat van de Stem}\\

\haiku{Haar mond die ik met.}{bramensap gekleurd heb is}{blauw en gesloten}\\

\haiku{Ik duw en trek tot,.}{het goed ligt met het gezicht}{naar boven gekeerd}\\

\haiku{En dan legt hij een:}{arm om mij heen en zegt met}{strakke ogen kijkend}\\

\haiku{Ervoor verbergen.}{gras en brandnetelvelden}{de lage toegang}\\

\haiku{Ik gaf niks om 'r.,.}{Gek maar dat met die ander}{kon ik niet hebben}\\

\haiku{Zijn dolzinnige.}{hoest klinkt tot me door als hij}{al op straat moet zijn}\\

\haiku{Haar mond is donker.}{en er zijn schaduwvegen}{om haar ogen en kin}\\

\haiku{Een dag of drie moet.}{hij toen al zo geweest zijn}{volgens de dokter}\\

\haiku{Hij ligt met de tong.}{uit de bek te reutelen}{als een stervende}\\

\haiku{je was bang, je kroop,{\textquoteright}.}{heel dicht tegen me aan het}{maakte mij ook bang}\\

\haiku{Deze gebaarde.}{Josias daarop voor hem}{uit te gaan lopen}\\

\haiku{{\textquoteleft}Josias Mure{\textquoteright}, {\textquoteleft}.}{zei hij bij zichzelfblijf wat}{er ook gebeurt kalm}\\

\haiku{De doos stond op de,.}{bovenste plank tussen twee}{stapels linnengoed}\\

\subsection{Uit: De rudimentaire mens}

\haiku{Misschien word ik wel,,.}{helemaal een sneeuwpop dacht}{hij goeie genade}\\

\haiku{Ze zuchtte diep en.}{stak hem haar lippen toe om}{gezoend te worden}\\

\haiku{Hun tot diep in de.}{aarde reikende wortels}{waren dezelfde}\\

\haiku{Op een goede dag.}{moest de proviandvoorraad}{aangevuld worden}\\

\haiku{Het vee was sterk en.}{vermenigvuldigde zich}{verwonderlijk snel}\\

\haiku{Geen hand reikte naar.}{een zorgeloos op het gras}{gegooid kledingstuk}\\

\haiku{De onderste tak.}{bevond zich meters boven}{de begane grond}\\

\haiku{het zou zich wel eens.}{kunnen ontpoppen in een}{fikse regenbui}\\

\haiku{Dat was al heel wat,.}{met dat feit als uitgangspunt}{werd veel mogelijk}\\

\haiku{In een duizel van.}{geluk en genot liet ze}{de handen begaan}\\

\haiku{Alles kon nog heel,,.}{goed terechtkomen ook nu}{juist nu welbeschouwd}\\

\haiku{De ruiters hadden.}{nietszeggende anonieme}{stoppelgezichten}\\

\haiku{Die boer pit terwijl,.}{hij op zijn poten staat als}{een paard verdomme}\\

\haiku{{\textquoteleft}Bedankt vrouw{\textquoteright} zei de.}{magere ruiter op niet}{onbeleefde toon}\\

\haiku{Naar zijn mening bleef:}{een mens altijd een mens en}{dat wilde zeggen}\\

\haiku{Maar het geweer dat.}{nog steeds tussen zijn knie\"en}{stond hinderde hem}\\

\haiku{Haar veestapel was.}{ze tengevolge van de}{oorlog kwijtgeraakt}\\

\haiku{Om hem van ons te.}{kunnen kopen liet ze hem}{die pot opgraven}\\

\haiku{De vrouw kon ze niet,.}{altijd zien hoe scherp haar ogen}{anders ook waren}\\

\haiku{Hij hervatte het.}{zoeken naar de verloren}{autosleutels niet}\\

\haiku{{\textquoteleft}Ik heb u dag aan,.}{dag om een teken gevraagd}{nu al twee jaar lang}\\

\haiku{ik heb het nu al:}{zo vaak gezegd dat ik het}{beu ben geworden}\\

\haiku{Nu eens hield hij hem}{zo dicht bij zijn gezicht dat}{het een wonder was}\\

\haiku{U zult wel gemerkt.}{hebben dat het zelfs al niet}{meer waait in dit bos}\\

\haiku{Hij stak zijn handen.}{uit om de regen beter}{te kunnen voelen}\\

\haiku{Met de angst van het}{voor bossen bevreesde kind}{dat hij geweest is}\\

\haiku{Nog steeds, maar flauwer,.}{al om zich heen slaand raakte}{hij snel uitgeput}\\

\haiku{De hoeven kwamen.}{recht op hem af en deden}{de grond daveren}\\

\haiku{Het hield zijn mondje.}{geopend en de oogjes}{waren gesloten}\\

\haiku{Wanneer de smaak haar.}{niet beviel spuwde ze het}{achteloos weer uit}\\

\haiku{Ook zij wierpen zich.}{op de grond en bedankten}{hem op hun manier}\\

\section{Maarten 't Hart}

\subsection{Uit: De droomkoningin}

\haiku{Het zijn er evenveel.}{als het volk Isra\"el dat}{door de woestijn trok}\\

\haiku{{\textquoteright} {\textquoteleft}Nee, maar wel heel licht,.}{en hij heeft moe even met een}{vleugel aangeraakt}\\

\haiku{Ik kon er alleen,}{maar naar luisteren ik kon}{het niet meezingen}\\

\haiku{{\textquoteright} In de kamer zat.}{een onbekende die me}{vriendelijk aankeek}\\

\haiku{Nu linksaf en dan,,.}{nog eens linksaf dacht ik maar}{toch aarzelde ik}\\

\haiku{{\textquoteleft}Ik woon op de dijk,{\textquoteright}, {\textquoteleft}.}{zei zekunnen we een heel}{eind samen lopen}\\

\haiku{Het was anders dan,.}{die andere muziek maar}{toch leek het erop}\\

\haiku{Ik weet het niet, ja,,.}{als het kon als we thuis een}{piano hadden}\\

\haiku{Voor het zover is,.}{probeert ze eerst om mij te}{leren klokkijken}\\

\haiku{Maar zolang als ik,:}{nog niet kan klokkijken kan}{zij ook niet zeggen}\\

\haiku{Ik verlaat mijn stoel,,.}{en loop met ingehouden}{adem naar de radio}\\

\haiku{Het huisje stond zo.}{dicht bij het pad dat ik de}{stemmen kon horen}\\

\haiku{zag ik er dan zo?}{onbenullig en weinig}{vreesaanjagend uit}\\

\haiku{{\textquoteright} {\textquoteleft}Nee,{\textquoteright} zei ze lachend,, {\textquoteleft},.}{en met iets van spot in haar}{stemnee dat kan niet}\\

\haiku{het gaat er alleen.}{maar om dat je van  de}{ander een Bach maakt}\\

\haiku{{\textquoteright} {\textquoteleft}Klassieke rotzooi,,!}{gadverdamme hadden we}{net zo leuk Elvis}\\

\haiku{her en der brandden.}{in het holst van de tuinen}{nog kleine lichtjes}\\

\haiku{Vandaag ben ik voor,.}{het laatst maandag komt jullie}{eigen bakker weer}\\

\haiku{Daar verliet ik het}{pad en omcirkelde ik}{zo snel mogelijk}\\

\haiku{Ik had les op de,.}{muziekschool bij een oude}{reusachtige vrouw}\\

\haiku{Het klonk als een oud,,.}{van ver voor mijn geboorte}{daterend bevel}\\

\haiku{hebben jullie er?}{weer \'e\'en betrapt die hier over}{het erf wou lopen}\\

\haiku{de wet van Metten.}{Anker is net zo'n wet als}{de wet van Hubble}\\

\haiku{Als mijn vader, vlak,:}{voor het avondeten de krant las}{en soms opmerkte}\\

\haiku{Voor het overige '.}{breide zes avonds als hij}{sliep zwarte sokken}\\

\haiku{Over mijn zuster kan.}{ik nog minder vertellen}{dan over mijn ouders}\\

\haiku{Toch bestaat ze niet,,:}{echt voor me of nee ik moet}{het anders zeggen}\\

\haiku{{\textquoteleft}Kom hier,{\textquoteright} en bleef staan.}{wachten terwijl de hond ging}{liggen op liet pad}\\

\haiku{dat ik mijn handen:}{vouwde en mijn ogen sloot en}{aldus God aanriep}\\

\haiku{bij een korter stuk.}{zou ze nooit kunnen laten}{horen wat ze kan}\\

\haiku{Daar moet je kunnen.}{horen dat de violist}{door smart overmand wordt}\\

\haiku{Zij speelde goed, maar.}{hij kraste erop los als}{een beginneling}\\

\haiku{Vervolgens vragen?}{of ze zin heeft om iets met}{me te gaan drinken}\\

\haiku{Ze rukte zich los, {\textquoteleft}{\textquoteright}.}{zeihou je handen thuis en}{vervolgde haar weg}\\

\haiku{die geur laat zich zelfs.}{niet door slecht trekkende open}{haarden verdrijven}\\

\haiku{{\textquotedblleft}Het is geen nette,,{\textquotedblright}.}{jongen Rens en dat vind ik}{zo verschrikkelijk}\\

\haiku{{\textquoteleft}Begrijp jij nou dat?}{de meeste mensen zo lang}{in hun bed blijven}\\

\haiku{Toch bleef zij hem slaan.}{met de riem totdat zij om}{een straathoek verdween}\\

\haiku{Het bloed gonsde in.}{mijn oren en mijn hart beukte}{tegen mijn borstbeen}\\

\haiku{En als je me nog,.}{twee geeltjes geeft maken we}{er echt een feest van}\\

\haiku{Vijftig gulden, dacht,.}{ik om te zien hoe iemand}{twee strikjes losknoopt}\\

\haiku{En als ze  over,,?}{me droomt sta ik er toch ook}{buiten waar of niet}\\

\haiku{het lijkt me niet de.}{weg om hoofdredacteur van}{een krant te worden}\\

\haiku{Dat artikel waar,}{ik het zo net over had was}{door jou geschreven}\\

\haiku{{\textquoteleft}We laten u eerst.}{het eerste deel uit het 22Ste}{trio van Haydn horen}\\

\haiku{Maar we zien elkaar,.}{misschien bij een overweg ik}{zal naar je wuiven}\\

\haiku{We liepen naar de.}{tafel waar de cassettes}{werden uitgereikt}\\

\haiku{{\textquoteright} {\textquoteleft}Nee, eigenlijk niet,.}{maar misschien begrijp ik niet}{goed wat je bedoelt}\\

\haiku{Wat vreemd,{\textquoteright} zei ik toen, {\textquoteleft}.}{we buiten warendat er}{niemand anders was}\\

\haiku{Het trok mijn aandacht.}{omdat het opeens geluid}{leek voort te brengen}\\

\haiku{{\textquoteright} En ook altijd zo.}{geweest en zelden zo sterk}{als op deze avond}\\

\haiku{Ze reikte me de,:}{rode wijn aan zocht naar de}{plaat en riep me toe}\\

\haiku{De kaart viel half open,,.}{vouwde zichzelf uit gleed toen}{van de tafel af}\\

\haiku{Een vrouw die mij even?}{zal laten merken dat ze}{sterker is dan ik}\\

\haiku{Je mag wel aan de,.}{dood denken maar je mag nooit}{aan zelfmoord denken}\\

\haiku{Maar het enige waar,.}{het opaan leek te komen had}{ik hem niet gezegd}\\

\haiku{Ik streelde haar nog,.}{steeds maar dat leek me niet de}{juiste methode}\\

\subsection{Uit: De vrouw bestaat niet}

\haiku{hoe ik ertoe kwam.}{er een bepaalde mening}{op na te houden}\\

\haiku{Zij was het enige.}{wezen op de wereld voor}{wie Krijnie bang was}\\

\haiku{Dat zorg en macht zo.}{nauw samenhangen is niet}{onbegrijpelijk}\\

\haiku{En eerst dan zal een.}{machthebber gebruik maken}{van zijn lichaamskracht}\\

\haiku{Is misschien dan toch:}{waar wat Harry Mulisch in}{een interview zei}\\

\haiku{Als hij een vaasje,.}{op de schoorsteen verschuift krijgt}{hij een grote mond}\\

\haiku{Volgens Joke Smit {\textquoteleft}:}{is depsychologische}{kern van het probleem}\\

\haiku{Voor de Goud-Elsje.}{serie liet ik de Bob Evers}{serie graag rusten}\\

\haiku{{\textquoteleft}De muziek behoort;}{tot eene geheel andere}{orde van zaken}\\

\haiku{Vrouwen konden hun.}{gecomponeerde werken}{ook publiceren}\\

\haiku{omdat ik wel, en,.}{zij niet later een beroep}{zou mogen kiezen}\\

\haiku{Overal om me heen.}{zie ik echter bewijzen}{van het tegendeel}\\

\haiku{Op 14 juni 1963:}{kocht ik een boek waar ik al}{lang naar had gezocht}\\

\haiku{dit soort werk diende.}{eerlijk over beide partners}{verdeeld te worden}\\

\haiku{Vooral Dostojewski is;}{in die dagen bij mij door}{de mand gevallen}\\

\haiku{Een dwaze maagd van,.}{Ida Simons maar daarna was}{het afgelopen}\\

\haiku{Toen ook daagde het.}{besef dat ik misschien zelf}{zou kunnen koken}\\

\haiku{Overigens zijn de.}{begrippen rotzooi en troep}{maar betrekkelijk}\\

\haiku{Maar, zou men kunnen,.}{zeggen de onderwerpen}{zijn toch heel anders}\\

\haiku{Ja, maar vergeet niet.}{dat Van der Meyden een heel}{snelle auto heeft}\\

\haiku{Ze weten het zelf:}{niet en daarom zal ik hen}{nu maar eens helpen}\\

\haiku{Overigens is het}{uitermate opvallend}{hoe merkwaardig groot}\\

\haiku{{\textquoteright} {\textquoteleft}Verbaast ons vaak{\textquoteright} klinkt {\textquoteleft}{\textquoteright}.}{toch bepaald anders dan het}{apodictischeis}\\

\haiku{{\textquoteright} Dat schreef Jung, deze.}{neo-platonische}{vernieuwer in 1934}\\

\haiku{Dat blijkt wel uit De.}{zwembadrnentaliteit van}{Andreas Burnier}\\

\haiku{Wie niet voor mij is,,.}{is tegen mij zei Jezus}{al zo mooi simpel}\\

\haiku{Op pagina 86:}{ongetwijfeld de mooiste}{uitspraak uit het boek}\\

\haiku{Alleen geloof ik.}{niet in haar oplossing of}{haar oplossingen}\\

\haiku{Zijn roman speelt in,.}{Boston aan het einde van}{de vorige eeuw}\\

\haiku{Daarom lijkt het mij.}{goed de raad van Virgina}{Woolf op te volgen}\\

\haiku{En een woord zonder,.}{betekenis is een dood}{woord een bastaardwoord}\\

\section{Henri Hartog}

\subsection{Uit: Sjofelen}

\haiku{- Ze kon 't pleizier,,.}{dat ze tot nogtoe had in}{d'r trouwen best op}\\

\haiku{As-t-ie ze nou,.}{niet kreeg kon zij ze voor zijn}{part wel opzouten}\\

\haiku{Maar morgen, dan zou,.}{ze gaan ze zou er geen gras}{over laten groeien}\\

\haiku{De meester, die sterk, '}{aan den draad trok wass avonds}{nog al eens gepoetst}\\

\haiku{Vrouw Scharrewou riep '.}{zachtjes zijn moeder en gaf}{diet lampje over}\\

\haiku{ze mosten van goeie,.}{huize zijn wouen ze hem}{d'r onder krijge}\\

\haiku{'t was toch al de,.}{vijfde dag ze most zich nou}{alleen maar redden}\\

\haiku{{\textquoteright} En Van Deesem, die,:}{ook buiten was gekomen}{zei tegen vrouw Muis}\\

\haiku{Maar nu kwam de Groef,.}{een woordje meepraten die}{naast vrouw Muis woonde}\\

\haiku{Hij was een beetje.}{in de olie en dan kon je}{niet voor hem instaan}\\

\haiku{Vrouw Van Deesem kwam,.}{nu terug met d'r neef die}{de huur ophaalde}\\

\haiku{Ze waren op 't.}{laatst allemaal van angst de}{deur uitgeloopen}\\

\haiku{'t jonge wijf had.}{ze an de buurvrouw verkocht}{voor een dubbeltje}\\

\haiku{Ze had jarenlang.}{op d'r zelve gewoond in}{een woning alleen}\\

\haiku{Ze kon toch niet over,.}{d'r hart verkrijgen om niet}{even te blijven staan}\\

\haiku{Toen vrouw Muis weer naar, '.}{binnen zou gaan kwam net de}{Groef langst portaal}\\

\haiku{Vrouw Muis was alweer,.}{weggesuft op haar stoel toen}{de jongen thuiskwam}\\

\haiku{Hei, scheeve, hoor'es even,,....}{ik geef-ie een kwartje as}{je op je broek trapt}\\

\haiku{Wel allemachtig,.}{d'r stond een groote knolraap in}{een pot v\`ol water}\\

\haiku{Ze kon niet velen,.}{dat-ie d'r an raakte}{door die rematiek}\\

\haiku{{\textquoteright} Jonge Miet werd kwaad, ', '.}{hoe had zet nou ze had}{t toch zelf gewild}\\

\haiku{hier en daar tegen, ';}{an geworpen en int}{bed zat ouwe Miet}\\

\haiku{Maar aan den overkant.}{waren de bovenhuizen}{in vieren verdeeld}\\

\haiku{{\textquoteleft}Daar leit een smeris ',,....{\textquoteright}}{int water verzuip nou}{maar verzuip nou maar}\\

\haiku{voeten slepen de,;}{straat voetpunten tjiepten als}{ingezet getjilp}\\

\haiku{Zij werd nuchterder,,;}{zij wist dat groote lui meisjes}{met cente namme}\\

\haiku{Die z'n vader of,.}{z'n moeder vermoord had was}{daar nog te goed voor}\\

\haiku{As 't een nette,?}{jongen was waarom bracht ze'm}{dan niet mee naar huis}\\

\haiku{En zonder dat zij,.}{er erg in hadden had-ie}{legge luisteren}\\

\haiku{Als ze 'm nou weg,.}{wou sturen dan most ze d'r}{maar veel over teemen}\\

\haiku{Toen die zestien jaar,.}{was liep-t-ie al na die}{verdomde hoeren}\\

\haiku{hij kon nog net met.}{zijne \'eene hand de Moer}{naar zijn keel grijpen}\\

\haiku{Louise boog zich naar,.}{de Brakel toe die zijn arm}{om haar hals legde}\\

\haiku{ze zal 'm wel voor.}{je in de watte legge}{op de beddeplank}\\

\haiku{dat is z'n vrouw, ze,...}{liep zoo driftig d'r na toe}{net of ze kwaad was}\\

\haiku{Nou mosten ze toch.}{met oneerlijke lui te}{doen gehad hebben}\\

\haiku{Geen een kwam d'r in,,.}{of hij most vooruit weten}{dat ze betaalden}\\

\haiku{dat is vroeger zoo,.}{geweest maar tegenwoordig}{is dat veranderd}\\

\haiku{{\textquoteright} En toen volgde hij}{zijne vrouw en achter haar}{pittigen opstap}\\

\haiku{{\textquoteright} Maar dat viel ook al,.}{tegen dat was alweer eene}{verkeerde tactiek}\\

\haiku{waar het zoo straks nog,.}{blauw was lagen er nu als}{schepen gegroepeerd}\\

\haiku{om haar oogen, die soms.}{heel dankbaar-zachtmoedig}{hem even a\`anschouwden}\\

\haiku{Hij vond, dat-ie,.}{meer vastigheid had als zij}{dat nu bepaalde}\\

\haiku{Dit verplichtte hem.}{op al deze wegen zijn}{aandacht te richten}\\

\haiku{zag veel jongens met,.}{meisjes die erg in hun schik}{leken met mekaar}\\

\haiku{Even, als een slok in.}{een nauwe keel was het dicht}{vallen van het slot}\\

\haiku{Dat ging dus in den.}{beginne allemaal van}{een leien dakkie}\\

\haiku{Ze kon moeilijk an ',.}{t fabriek blijven tot ze}{op alle dag liep}\\

\haiku{Ze zou natuurlijk,.}{zeggen dat de jongen te}{lang onderweg bleef}\\

\haiku{Maar die vent scheen maar,.}{niet beter te worden liet}{niks van zich hooren}\\

\haiku{En dan wisten ze,.}{wel hoe ze de pelisie}{op d'r hand kregen}\\

\haiku{anderen keken,,,.}{toe of ze kieskeurig ook}{fouten opmerkten}\\

\haiku{Post stond dicht bij haar,,,.}{in zijn boezeroen blootshoofds}{met Truus op zijn arm}\\

\haiku{{\textquoteright} vroegen ze, {\textquoteleft}wil je.}{vier dagen blijven of acht}{dagen of een maand}\\

\haiku{{\textquoteright} {\textquoteleft}Zeg voor mijn part wat,.}{je wil ze zijn allang naar}{de Godverdomme}\\

\haiku{As-t-ie dronken,,....}{was zocht-ie ruzie in de}{kroeg wou dan vechten}\\

\haiku{Zeg, zeit ze, die vent,,:}{van jou die lust ze ook koud}{en ik zeg nog zoo}\\

\haiku{{\textquoteleft}Ja maar, zoo as jij ',, '....}{t heb zoo'n enkel nachie}{jij hebt bed ruim}\\

\haiku{In Leie as meissie,'.}{zijnde was ze ook's bij een}{kaartlegster geweest}\\

\haiku{En dus besloot ze.}{morgen met de Sluische}{d'r man te prate}\\

\haiku{{\textquoteright} {\textquoteleft}'k Ben toch 's gaan,.}{hooren bij me zuster of}{die ergens van wist}\\

\haiku{hoe kort nog en dan,.}{liepen ze rond net zoo}{goed of ze dood was}\\

\haiku{Lekkerder dan al,.}{de menschen  die gedaan}{hadde gekrege}\\

\haiku{Je mocht 't wel, as ',.}{jet maar zoo dee dat ze}{je niks konden doen}\\

\haiku{Betaal de  lui,,,.}{maar die an je deur komme}{mane dweil stinkert}\\

\haiku{{\textquoteleft}ik hou wel van een,,, '}{schoone man wat jij ik wou}{dak er van nacht}\\

\section{Fran\c{c}ois Haverschmidt}

\subsection{Uit: Winteravondvertellingen}

\haiku{Deze gevaarten,,.}{bewogen maakten geluid}{en wezen op mij}\\

\haiku{Hoe de 1e kat hier,.}{te lande gekomen is}{is dood eenvoudig}\\

\haiku{Maar nu stond eensklaps,.}{Jelle's vader Henrij Gall op}{en vatte het woord}\\

\haiku{misschien wordt Klaauw nog,...!}{eens primus der provincie}{en dan welk een eer}\\

\haiku{Ach, ik dacht niet dat.}{kattenberekeningen}{meermalen falen}\\

\haiku{Het is wel slechts het,:}{woord van een  kat maar het}{is toch welgemeend}\\

\haiku{Vooraf evenwel nog, -,.}{\'e\'en woordje dat beteekent nog}{eenige volzinnen}\\

\haiku{Als gij er soms in,.}{mocht staan dan zal ik niets dan}{mooi's van u zeggen}\\

\haiku{Wie altoos met de,}{naakte waarheid voor den dag}{komt brengt het niet ver.}\\

\haiku{Plakken niet velen?}{bepaald valsche etiquetten}{op hunne flesschen}\\

\haiku{Ze tooit anderen.}{en wil ze wat meer laten}{schijnen dan ze zijn}\\

\haiku{Iemand, van wien ik,,,:}{durf wedden dat als hij sterft}{in de krant zal staan}\\

\haiku{Vooral voor iemand,.}{die getrouwd was met den man}{van jufvrouw Wawel}\\

\haiku{Iedereen weet, er,:}{zijn twee manieren om in}{opspraak te komen}\\

\haiku{Zij staat niet aan het;}{hoofd van philanthropische}{vereenigingen}\\

\haiku{- Wij kunnen deze.}{onbeduidende menschen}{dus gerust overslaan}\\

\haiku{Het zij verre van,,.}{mij hun talent hun genie}{te betwijfelen}\\

\haiku{Doch wat bewijst dit,, -, -?}{dan dat uw smaak niet vergeef}{het mij niet fijn is}\\

\haiku{Neen, schildknaap Kuno ',.}{wast veeleer Aan wien ze}{heur hartje gaf}\\

\haiku{Ziedaar een waardig!}{opvolger van den grooten}{zanger van zooeven}\\

\haiku{Zonder schoon te zijn,',.}{hadden zijn trekken iets edels}{iets beminnelijks}\\

\haiku{- Veel, zeide ik, was,,,.}{er dat hem aantrok veel ook}{wat hem van zich stiet}\\

\haiku{Hij koesterde een',.}{stille liefde voor alles}{wat schoon is en goed}\\

\haiku{- Wat zal ik er nog,?}{meer van zeggen om u zijn}{beeld te voltooien}\\

\haiku{Mogelijk was hij,,.}{zonder het te vermoeden}{aller geweten}\\

\haiku{Maar zijn plichtbesef,.}{noodzaakte hem van dezen}{wensch afstand te doen}\\

\haiku{Mijn geheele hart;}{is ingenomen door mijn}{academievrienden}\\

\haiku{Daar komen  toch.}{zeker alleen de echte}{letterkundigen}\\

\haiku{En toen heeft hij ook.}{een toast ingesteld op het}{jonge Vlaanderen}\\

\haiku{- Ik ben stom genoeg.}{om mij in verlegenheid}{te laten brengen}\\

\haiku{en als een ander}{wat zegt veroorlooft hij zich}{telkens ter zijde}\\

\haiku{Intusschen - ik heb.}{mij op den bewusten avond}{niet enkel verveeld}\\

\haiku{ik gaf wat, als ze.}{gedrukt waren en ik ze}{nog eens lezen kon}\\

\haiku{Maar, hoopte ik, gij,,.}{zoudt wel met iets dat minder}{was tevreden zijn}\\

\haiku{Maar terstond laat hij {\textquoteleft},.}{er op volgenKom ik ga}{er ook nog eens zien}\\

\haiku{- Zoolang haar Jan nog,...}{leefde had ze altoos nog}{goeden moed gehad}\\

\haiku{Maar Jan heeft nu wel.}{wat anders te doen dan zich}{te laten kussen}\\

\haiku{- Maar al genoeg - want}{ik kan mij nu niet langer}{met je ophouden}\\

\haiku{'t Was toch wat een,.}{deftige majoor en hij}{reed op een wit paard}\\

\haiku{Maar gij zeidet, dat.}{hij werk genoeg had om er}{niet af te vallen}\\

\haiku{Eigenlijk mag ik,.}{niet zeggen dat de jufvrouw}{haar meiden versleet}\\

\haiku{Zij deed er mee als.}{roekelooze jongeheeren}{met hun sigaren}\\

\haiku{Want versch waren ze,.}{juist niet de nieuwe meiden}{van mijn hospita}\\

\haiku{De jongste van uw.}{oudste kind is ouder dan}{uw eigen jongste}\\

\haiku{Of ik niet zie, dat?}{er een heele boel verkeerds}{in de wereld is}\\

\haiku{Kortom, wie zegt, dat,.}{gij geen gevoel hebt die moest}{zich liever schamen}\\

\haiku{Al was het de neus,,.}{van den burgemeester ik}{vrees dat hij plat ging}\\

\haiku{En nooit zondigde,.}{hij tegen het rijm evenmin}{als tegen de maat}\\

\haiku{{\textquoteleft}laat mij het om 's,}{hemels wil niet verraden}{wat voor een engel}\\

\haiku{- Gij hebt haar, zonder,;}{het te willen of het te}{weten gemankeerd}\\

\haiku{Hij liet zich dus niet;}{anders betitelen dan}{hem eerlijk toekwam}\\

\haiku{Want als zij in het,.}{water valt dan vertrekt de}{schavuit geen gezicht}\\

\haiku{Maar ach, hoe kan ik,?}{nog schertsen terwijl ik van}{den kleinen Bob spreek}\\

\haiku{- z\'o\'o iemand is noch,,.}{het een noch het ander is}{noch lui noch lekker}\\

\haiku{En dit laatste is,.}{het geval niet alleen met}{u maar ook met mij}\\

\haiku{Hoe ook aangelengd,.}{dat bedwelmend vocht voltooit}{onze ellende}\\

\haiku{Vertering had de.}{vreemde in het logement}{niet kunnen maken}\\

\haiku{Die roode kleur deed.}{vreemden wel eens vermoeden}{dat Mollemans dronk}\\

\haiku{die had dan een in '.}{t oog vallenden aanleg}{voor een beroerte}\\

\haiku{Het lijk opent de oogen,,}{de mond ontsluit zich en geeft}{een schreeuw de blaker}\\

\haiku{Hij moest met Trui en?}{den kleine bij Oom komen}{inwonen en Bram}\\

\haiku{De afwezigheid,.}{van den Baron is juist een}{goed teeken Mevrouw}\\

\haiku{Hoe d\`at: wat moet hij?}{op dit oogenblik met een}{geladen geweer}\\

\haiku{Hij stapte op de,:}{deur los belde en vroeg toen}{hem werd opengedaan}\\

\haiku{{\textquoteright} En met dat woord werd.}{hem de deur onzacht voor den}{neus toegesmeten}\\

\haiku{Bedenk gij zelf u,.}{maar eens of gij er niet wat}{op kunt verzinnen}\\

\haiku{wat trouwens ook niet,.}{behoefde want het was maar}{een meidenkamer}\\

\haiku{Z\'o\'o kreeg zij althans.}{een aanwijzing waar ze de}{baker vinden kon}\\

\haiku{Doch, alsof het er,}{Nol om te doen was geweest}{\`en den professor}\\

\haiku{{\textquoteright} De heer Frieseman had.}{een beetje moeite om het}{\'o\'ok niet te vinden}\\

\haiku{De tweede week na.}{zijn komst op het instituut}{kreeg Nol den kinkhoest}\\

\haiku{Zeker, er waren;}{sentimenteele ladie's}{en ook niet-ladie's}\\

\haiku{{\textquoteleft}zal je oppassen,,?}{buurman dat Johnny}{geen ongeluk krijgt}\\

\haiku{{\textquoteleft}Geef nu je moeder,,!}{maar een zoen ventje en dan}{gaan wij er van door}\\

\haiku{Zij dacht aan den prins,}{die beter geworden was}{en hoe gelukkig}\\

\haiku{En toen zag kleine.}{John wat geen mensch in de}{wereld gezien heeft}\\

\haiku{Er zijn menschen, die.}{liegen om u het geld uit}{den zak te kloppen}\\

\haiku{Toen proponeerde,.}{ik Heintje om haar in de}{waschkuip te smijten}\\

\haiku{Eindelijk werd ik,}{er melankoliek onder}{en op een goeien}\\

\haiku{En omdat we dat,;}{niet wisten deden wij dan}{ook geen van beide}\\

\haiku{Eenigen waren er,.}{blijkbaar voor maar anderen}{waren er tegen}\\

\haiku{Het beste was, je.}{maar in het geheel niet met}{hem te bemoeien}\\

\haiku{Het zat hem in zijn,;}{hart over die onverwachte}{pensioneering}\\

\haiku{Ik moest een ander,.}{in mijn plaats stellen maar dat}{ging niet op den duur}\\

\haiku{Als ik mijn boek voor,.}{mij heb dan kan ik naar geen}{praatjes luisteren}\\

\haiku{{\textquoteleft}Ferm, toe maar jongens,!}{laat hem geen duit houden van}{zijn gestolen geld}\\

\haiku{Of de advokaat,.}{het al buiten hem om deed}{dat gaf niet genoeg}\\

\haiku{Het is er mee, als.}{met de schilderstukken van}{mijn vriend van der Kwast}\\

\haiku{En gelukkig, als.}{men er niet midden in den}{nacht wakker van schrikt}\\

\haiku{{\textquoteleft}Het is een soort van,,:}{fantasie Oom en ik heb}{er boven gezet}\\

\haiku{ik U \'o\'ok toe (wel,).}{te verstaan dat bord pap op}{uw 80sten verjaardag}\\

\haiku{Als het maar mooi was,.}{en dat waren de verzen}{van Lucas stellig}\\

\haiku{{\textquoteleft}Pas eens op wat ik,,{\textquoteright};}{je zeg moeder als de wind}{straks niet gaat krimpen}\\

\haiku{ik was, hoe kort mijn,,}{geluk duurde en hoe diep}{diep ongelukkig}\\

\haiku{Zij was berekend,.}{voor elke aandoening voor}{iedere schakeering}\\

\haiku{Maar zijn dialoogstijl,,.}{die ook zonder de voordracht}{overeind blijft is uniek}\\

\haiku{Dit zijn er zoveel,.}{dat cursivering ervan}{storend zou werken}\\

\haiku{genie [der grootste \{\},,:}{sieraden van de balie}{genie p.164 r.7 r.14}\\

\haiku{In elk geval, zij,,.}{en ik zien elkaar zelden}{of juister nooit meer}\\

\section{Heere Heeresma}

\subsection{Uit: Vader vertelt}

\haiku{Het was inmiddels.}{wel duidelijk dat ik mijn}{stiel gevonden had}\\

\haiku{Hij stelde dan ook}{voor dat ik de rest uit de}{kassa zou gappen}\\

\haiku{- mijn sigaretten.}{tot de lippen schroeiende}{peuken verbrandde}\\

\haiku{Daarvoor ervaar ik.}{mijn verblijf aldaar als een}{te grote schande}\\

\haiku{128 p.  ~  ~Zesde,.}{druk  Onveranderde}{tweede druk 1971}\\

\haiku{88 p.~ ~Vijfde,.}{druk  Onveranderde}{derde druk 1969}\\

\haiku{108 p.~ ~Zesde,.}{druk  Onveranderde}{derde druk 1970}\\

\haiku{88 p.~ ~Achtste,.}{druk  Onveranderde}{derde druk 1971}\\

\haiku{96 p.~ ~Zesde,.}{druk  Onveranderde}{tweede druk 1971}\\

\haiku{Heere Heeresma,.}{vertelt over zichzelf  De}{Telegraaf 6.12.'69}\\

\haiku{3 Bijlemer Prinz' ();}{bekerzilmeta lezers}{uit de Bijlmermeer}\\

\haiku{Iets waar hij zich toen,.}{nameloos aan ge\"ergerd}{had miste hij nu}\\

\haiku{Richten kan je er,,!}{niet mee maar mensen wat gaat}{die gasgranaat hard}\\

\haiku{Dat krijg je ervan...}{wanneer je het kind met het}{badwater weggooit}\\

\haiku{Geen geleuter maar.}{to-the-point reispakketten voor}{jofele prijsjes}\\

\haiku{{\textquoteright} {\textquoteleft}Allemachtig, oom,{\textquoteright}, {\textquoteleft}?}{Nol riepen we in koorweet}{u wel wat u zegt}\\

\haiku{Gewoon een overall.}{aan en daar is de man van}{de luchtverversing}\\

\haiku{Gelukkig was ze.}{in de keuken bezig en}{deed hem meteen open}\\

\haiku{Een volk van dertien.}{miljoen zielen heeft geen recht}{op zoveel schrijvers}\\

\haiku{En zo kwam van het,.}{een het ander zoals u}{niet ontgaan zal zijn}\\

\haiku{En verder wordt de!}{heer Alberts voor mij nog steeds}{burgemeester}\\

\haiku{Daarvoor moet je niet.}{alleen heel positief zijn}{maar ook zeer modern}\\

\haiku{Proppers zijn het, die.}{het welzijn van hun gasten}{niets kan verdommen}\\

\haiku{Maar wat Marlboro.}{ervan bakte ging alle}{perken te buiten}\\

\haiku{mij diep toen ook ik...}{me omdraaide en deed of}{me neus bloedde}\\

\haiku{Geregeld treffen.}{we Kooimans titels eerder}{bij anderen aan}\\

\haiku{Zo kwam Rap op het.}{spoor van het tekenwerk van}{Beatrix Potter}\\

\haiku{Maar hoe maak ik het?}{de heren duidelijk dat}{we zigeuners zijn}\\

\haiku{We kombineren.}{daarvoor Leviticus 12}{en Lukas 2}\\

\haiku{Indachtig Jo Kal,.}{bij leven Godsdienstleraar}{te Amsterdam}\\

\haiku{Maar men weet hoe het.}{met uitgeverijen over}{het algemeen is}\\

\haiku{Zegt daar nu van {\textquoteleft}ach,.}{ik wist niet eens dat mijn naam}{op de cover stond}\\

\haiku{zuivel op zuivel,!}{is het werk van de duivel}{want zo is het toch}\\

\haiku{In de tram veerde.}{je op als de zitplaatsen}{volgeslibd waren}\\

\section{Herman Heijermans}

\subsection{Uit: Diamantstad}

\haiku{Was niet elk droomend?}{gezwijg g\'odlijker klank dan}{het puurste geluid}\\

\haiku{Achter de ruiten,.}{dichtst-bij was de starre}{aandacht gebroken}\\

\haiku{Jan antwoordde niet,.}{zacht schrapjes aaiend in den}{drekpoel beneden}\\

\haiku{Hij wil niet hebbe,, ',.}{v\`ader da'kn appel zoek}{die keerel van boven}\\

\haiku{Hou jij 'm teugen,,.}{met je lat Meijer anders}{flikkert-ie weer weg}\\

\haiku{De eene hand sloeg om, '.}{den deurpost vingerknekels om}{t molmende hout}\\

\haiku{Die eenzame m\'o\'est.}{in die dagen groot en stil}{hebben geleden}\\

\haiku{{\textquoteright} -, spotte Moppes, steen.}{zachtjens aanduwend over den}{zoetkring van zijn schijf}\\

\haiku{{\textquoteright} De schouders van den,:}{grijzen mageren slijper}{schokten ontkennend}\\

\haiku{'r droge lippen,.}{vrindlijk-rustig knikkend naar}{de zij van Bleazar}\\

\haiku{Nou ja, \`u heit goed, ':}{prate u weet niet watr}{komp-kijke}\\

\haiku{Achter het hoofdeind.}{was de deur van de kast waar}{de strontemmer stond}\\

\haiku{{\textquoteleft}Die is zoo g\'o\'ogem, '...}{zoo uitgeslape voorn}{kind van drie jare}\\

\haiku{Van morrege wor ':}{k wakker en daar zeit de}{gebenchte memme}\\

\haiku{Van 't Plein, dat zwart,.}{lag met krommende boomen}{kwam heftig gestuw}\\

\haiku{Moste ze Davy,!}{nie de darme uit z'n lijf}{trappe de pooiers}\\

\haiku{De vrouw van Semmie - '!}{die komp van de grach heit ze}{\`erg watr gebeurt}\\

\haiku{Soms sleepte z'n been,,.}{soms kon-ie niet loopen z'n}{water niet houden}\\

\haiku{Even lachte-die '.}{mal int geroes van de}{kijvende joden}\\

\haiku{Zelf was 'r mondje ',.}{n volwreven groezel z\'oo}{als ze gesmuld had}\\

\haiku{{\textquoteright} - Zij spande den duim.}{en den wijsvinger om de}{grootte te wijzen}\\

\haiku{{\textquoteright} {\textquoteleft}O jee zoo dikkels{\textquoteright},,:}{blufte ze weer blij dat ze}{die dingen mocht doen}\\

\haiku{Hij, etend 'n reepje, ', '.}{met stroop wenkter binnen}{vroeg naarr vader}\\

\haiku{- Wij haast geen vr\`ete - ' -!}{door de staking zij inn}{warreme Schoel bah}\\

\haiku{- As-je denk da-je ',,!}{mijnr tusschen neemt zeg mot}{je vroeger op staan}\\

\haiku{{\textquoteleft}toch zit 'r lucht in ',}{water en al zatr geen}{lucht in dan vin je}\\

\haiku{Het kind nam den visch,.}{in z'n zwart handje hield z'n}{pink dicht bij den bek}\\

\haiku{{\textquoteleft}godvergeefme de '!}{zonde wat ken jij metn}{glad smoel staan liege}\\

\haiku{Bij het station,.}{werkten ploegen met bezems}{schoppen en latten}\\

\haiku{{\textquoteleft}hoe komp iemand zoo, ' '!}{bemazzel ast water}{dich leit asn pot}\\

\haiku{Stop 'm derek in ',{\textquoteright}...}{t water anders kots je}{je hart uit je lijf}\\

\haiku{Geef mijn 'n kar met - '!}{negotie late z{\`\i}j zich}{n breuk sappele}\\

\haiku{{\textquoteleft}nee, danmod-u is -{\textquoteright}.}{wachte dan zal ik u wat}{anders late zien}\\

\haiku{Over 'r zittend aan ',}{t withouten tafeltje}{zag-ie eerst h\`oe wit}\\

\haiku{Geen dag ging voorbij '.}{of ze las opr bed als}{ze niet te moe was}\\

\haiku{Anders knikte ze,.}{anders herkende ze zijn}{manier van loopen}\\

\haiku{Met heet zand gong 't ' -.}{inn oogeblik had ze}{wel tienmaal beleef}\\

\haiku{Langzaam glee-ie in,,.}{rust z'n handen ontspanden}{z'n oogen knipten toe}\\

\haiku{Overdag hadden de '.}{kinderen al eerder bij}{m kennen kommen}\\

\haiku{{\textquoteleft}v\'oor Joozep na bed komp -{\textquoteright}.}{die zit nou nog lekker an}{tafel te slape}\\

\haiku{Wirrelend spetten.}{de vonkjes uit den dikken}{nek der machine}\\

\subsection{Uit: Droomkoninkje. Een verhaal voor groote kinderen}

\haiku{Jij heb 			 zeker,,?}{wat op je geweten dat}{je zoo bang ben h\`e}\\

\haiku{Ik zal op me teenen,....}{loopen en zoo zachies}{praten zoo zachies}\\

\haiku{{\textquoteright} {\textquoteleft}Dat denk 'k haast wel,, ' '.}{maar ze huilde zoo datk}{r niet 			 verstond}\\

\haiku{Heeft ze de mijne '.}{niet klaargemaakt met worstr}{op en met kaas}\\

\haiku{{\textquoteleft}Zeg moeder goeien,{\textquoteright},:}{dag zei vader die schik in}{z'n 			 bleuheid had}\\

\haiku{De rest zal 'k an,.}{de vogeltjes geven die}{d'r na 			 snakken}\\

\haiku{In de huiskamer - - '.}{de honnige wast}{schrikkelijk-eenzaam}\\

\haiku{{\textquoteright}   Toen vader van '.}{de fabriek kwam sliep \`alles}{int huisje}\\

\haiku{zeker de duim,   ;}{waarmee ze z'n 			 neus stuk}{had gesnoten}\\

\haiku{je mond houdt, ken je '....}{m ook hooren in de wind}{en in de regen}\\

\haiku{'t Is van mijn 'n....}{goedheid om hier te staan}{en geen verplichting}\\

\haiku{Toe nou, schat, toe nou{\textquoteright},,:}{zee 			 moeder nog liever}{dan anders lachend}\\

\haiku{{\textquoteleft}Ik wier ineene zoo,,{\textquoteright}:}{raar in me hoofd 			 moeder}{babbelde-ie na}\\

\haiku{{\textquoteright} {\textquoteleft}Omdat ik 'n brief{\textquoteright}....}{an tante Toos in de bus}{heb 			 gestoken}\\

\haiku{Hij, met 't hoofd op, '.}{de vrije hand trachttet boek}{te begrijpen}\\

\haiku{de 			 eerste trein,.}{en dan kom ik met de}{laatste werom}\\

\haiku{{\textquoteright}, vroeg de man, die zelf.}{kinderen had en bang}{was voor besmetting}\\

\haiku{{\textquoteleft}hij had niet 't lef '!}{motten hebbenn poot na}{me uit te steken}\\

\haiku{En wat zal vader,,}{as-ie uit Heerlen werom}{is en ze overtuigd}\\

\haiku{Zie je wel, dat je'....?}{niet huilen kan as ik}{t niet hebben wil}\\

\haiku{Moeder zat op den,,,.}{stoel bij de wieg 			 keek niet}{bewoog niet dacht niet}\\

\haiku{as je je man eens '!}{in de maand int Rooie Dorp}{mag opzoeken}\\

\haiku{{\textquoteright}, dreigde tante 't, ' ':}{aannemershuis metn}{kluif vann vuist toe}\\

\haiku{'k Geef je zoo van,,}{mijn armoei geverfde}{medam as je d'r}\\

\haiku{\'e\'en durft open te 			 ,.... '}{doen zoolang me zwager niet}{vrijgelaten wordt}\\

\haiku{Was 't reizen na '?}{Heerlen geen foefie geweest}{voorn allebie}\\

\haiku{{\textquoteleft}Ben ik daarvoor de?}{h\'e\'ele dag op de kinderen}{blijven passen}\\

\haiku{{\textquoteleft}maar ik 			 vind 't,}{schande en vooral zonde}{dat jij as moeder}\\

\haiku{{\textquoteright} En gebelgd op zijn.}{manier ging-ie naar buiten}{zitten 			 kijken}\\

\haiku{{\textquoteright} {\textquoteleft}Wel allemachtig,{\textquoteright}:}{zette de electrici\"en}{warm 			 loopend in}\\

\haiku{{\textquoteleft}Doet u 'r twintig,{\textquoteright}.}{gulden bij drong moeder bij}{de 			 leuning aan}\\

\haiku{{\textquoteright} {\textquoteleft}Daar mot je lekkers '....}{voor koopen en voor moeder}{n nieuwe 			 jurk}\\

\haiku{Nou zag-ie zichzelf,,,:}{ook een twee driemaal in de}{spiegels 			   en dacht}\\

\haiku{kles maar toe -			 en....}{toen het de kleine poes van}{mijn haar gegeten}\\

\haiku{Daar leit 'n hengst met....}{z'n vier pooten in de}{lucht te grabbelen}\\

\haiku{da-je 			 uit mot,....}{stappen en dat ze v\'o\'or in}{slaap zijn gevallen}\\

\haiku{{\textquoteright} {\textquoteleft}Laat is kijken, neef -,{\textquoteright}.}{ik heb ze nog nooit gezien}{soebatte-ie}\\

\haiku{En je scharrelt me!}{niet langer achter me rug}{in de keuken}\\

\haiku{Wie komt 'r onder, ',....?}{me hoed kijken hoe ikr}{uit 			 zie stommerd}\\

\haiku{{\textquoteleft}'n lolletje is ', ' ',,?}{t niet maark 			 wenr}{wel an wat jij Ko}\\

\haiku{{\textquoteright} {\textquoteleft}Prachtig{\textquoteright}, lachte de, ':}{electricien dier niet}{\`alles van snapte}\\

\haiku{Moeder, je mag wel,,?....}{meeluisteren maar je}{mag niks zeggen hoor}\\

\haiku{terwijl ik op 't,{\textquoteright} ':}{dak de 			 drajen hecht}{zei-ie tott kind}\\

\haiku{{\textquoteright}, gromde Kobus, niet,.}{goed wetend 			 wat-ie}{dee wat-ie zei}\\

\haiku{Toen sponsde moeder ',.}{snelt snuit van den jongen}{en kwam 			 omlaag}\\

\haiku{Toos, bijgekomen, '.}{zat mett hoofd op de}{tafel te grienen}\\

\haiku{Je zag niks van de, '.}{hooge 			 schoorsteenen niks vant}{wentelend schachtrad}\\

\haiku{Ze was zoo suf, dat ' '.}{ze opr 			 kousent}{tuintje door-stapte}\\

\haiku{Me niet eens na 			 ,....}{benejen dragen om me}{hier te laten eten}\\

\haiku{Hij was te laf 			 ,,.}{geweest om zich te bukken}{zich te overtuigen}\\

\haiku{naar 't ploffen 			  '.}{van de aardkluiten opt}{deksel geluisterd}\\

\subsection{Uit: Duczika}

\haiku{{\textquoteleft}'ns zien wie de baas ',...{\textquoteright} {\textquoteleft}!}{int leven is de man}{of de vrouwDe vrouw}\\

\haiku{{\textquoteleft}dat's gemeen en,!}{dat's laf om iemand z'n}{pols om te draaien}\\

\haiku{{\textquoteleft}want je heb me daar,,!}{gruwelijk beleedigd niewaar}{Poldi hahaha}\\

\haiku{{\textquoteleft}maar 'k hou niet van,!}{geneesmiddelen die je}{h\'e\'ele leven duren}\\

\haiku{zoo'n vroolijken avond,!}{hadden ze in geen maanden}{geen jaren gehad}\\

\haiku{{\textquoteright} klonk dadelijk de:}{tyranniek-schelle stem van}{onder den dameshoed}\\

\haiku{{\textquoteright}, zei Betty met 'n ' ':}{doodswit gezichtjen slok}{vanr glas nemend}\\

\haiku{hardnekkig als-ie, '.}{altijd was verdraaide-ie}{t toe te geven}\\

\haiku{{\textquoteright} {\textquoteleft}Omdat je met die, ',...{\textquoteright} {\textquoteleft}?}{ander metr nichtje nog}{meer flirtWat zou dat}\\

\haiku{Dat 's nonsens en - ' -,?...}{neurasthenisch gedoe weet}{k maar toch niewaar}\\

\haiku{Z\'o\'o beroerd na 't.}{drinken van wijn was-ie in}{geen tijden geweest}\\

\haiku{Om beurten zaten,,.}{de vrouwen de mannen om}{beurten sliepen ze}\\

\haiku{Nathan, kind{\textquoteright}, zei Semmy, ':}{Lubinsky z'n kin voort}{spiegeltje drogend}\\

\haiku{Drie-, viermaal had '.}{ze zacht-kermend gevraagd}{t raam te sluiten}\\

\haiku{{\textquoteright} Ze lachte 'r om,,.}{onoprecht met zorg om de}{bloedelooze lippen}\\

\haiku{Zacht bedrukte-ie,.}{den mouw van de blouse die}{ze zelf had gestikt}\\

\haiku{Ze hadden mekaar,.}{in de oogen gestaard hielden}{mekanders handen}\\

\haiku{Was 't 'n genot,...}{zoo adem te halen zoo voet}{naast voet te zetten}\\

\haiku{{\textquoteleft}en je geeft me je,, -,?}{woord dat als je morgen komt}{en dat d\'o\'e je niet}\\

\haiku{Goeien avond, meneer,{\textquoteright} ', '.}{Sch\"uler zein jong meisje dat}{n muziektasch droeg}\\

\haiku{Ook Erich Sch\"uler had 'r.}{bij z'n eerste bezoek een}{te zien gekregen}\\

\haiku{Dan zal ik morgen '!}{an Kr\"uger of an Neumann}{n briefie schrijven}\\

\haiku{Morgen z\'o\'o als ze,...}{wakker werd zou ze tienmaal}{zoo vlug beginnen}\\

\haiku{Maar ze haalde 't, '...}{in als Lotte en Betty}{n handje hielpen}\\

\haiku{Lotte zette 'n,.}{Tarantella in danste}{waarlijk voorbeeldig}\\

\haiku{Kwamen de menschen '?}{voorn piano  of}{voor de danseuse}\\

\haiku{Daar kan 'k Cl\'eo de,!...}{M\'erode Yvette Guilbert}{voor laten komen}\\

\haiku{de kat in Odessa,,}{uit den boom dan kan ik nog}{altijd nakomen}\\

\haiku{{\textquoteleft}Ook goed{\textquoteright}, babbelde.}{Duczika geduldig en}{innerlijkopgewekt}\\

\haiku{wat Schmink noodig hebben, '.}{als ze in Mannheim achter}{t voetlicht stonden}\\

\haiku{'k Zou d'r bij gaan ' '!}{zitten en ook isn duit}{int zakkie doen}\\

\haiku{{\textquoteright} Betty schonk z'n glas, ' ', ' '.}{in reiktetm over dronk}{r eerstn slok van}\\

\haiku{{\textquoteright} {\textquoteleft}Gedorie, steken!}{jullie niet allemaal je}{neus in mijn zaken}\\

\haiku{{\textquoteleft}en nou beweerde, '!}{ze vanmiddag nog wel dat}{zet om jou dee}\\

\haiku{Nacht Duczi{\textquoteright} - 'r hand' -:}{hield-ie vast of-ie  r}{niet los kon laten}\\

\haiku{Juffrouw Treibitz dee '.}{de wasch en Oom Ludwig hield}{t huishoudboek bij}\\

\haiku{{\textquoteleft}twee pond soepvleesch,...{\textquoteright} {\textquoteleft}!}{en twintig Pfennig beentjes}{vier augurkenNee}\\

\haiku{Nou dan weet ik 't - '{\textquoteright},.}{niet mot ut maar zeggen}{besloot de juffrouw}\\

\haiku{K\"onigsberger Klops...{\textquoteright} {\textquoteleft}{\textquoteright},.}{Met bloemkool knikte juffrouw}{Treibitz goedkeurend}\\

\haiku{Oom werd door juffrouw, ',!}{Treibitz dier ook patent}{uitzag vertroeteld}\\

\haiku{Zenuwbabbelen,,, '.}{aan een stuk door dee ze dat}{jer suf bij werd}\\

\haiku{{\textquoteleft}Ja, die kosten me, - -...}{Ludwig vier dochters en een}{zoon die eten wat op}\\

\haiku{Ze berukten de,.}{kettingen blaften dat je}{geen woord meer verstond}\\

\haiku{k Geloof dat 'k,!}{bedenkelijk boven m'n}{stand leef hahaha}\\

\haiku{{\textquoteleft}Zoo{\textquoteright}, sprak oom met 'n,:}{kleur van opwinding die z'n}{lippen verfletste}\\

\haiku{Ik geef je m'n woord,...{\textquoteright} {\textquoteleft}?}{dat-ie geen Hebreeuwsch leert}{Dat-ie niet wat}\\

\haiku{hoe laat 't op de - '.}{gangklok was telkens zat oom}{m op de hielen}\\

\haiku{Tenzij jij kans ziet...{\textquoteright} {\textquoteleft}!}{om met een van die lepels}{te racenChristus}\\

\haiku{Heb-ie geen kans, '?}{om hier of daar in de buurt}{n roeispaan te leenen}\\

\haiku{{\textquoteleft}op 't leven valt,,,...{\textquoteright}}{af te dingen naar meer dan}{\'e\'en kant maar toch toch}\\

\haiku{{\textquoteleft}Ja, ja{\textquoteright}, sprak Poldi, {\textquoteleft}{\textquoteright}:}{Nietzsche wonderlijk-vreemd}{in de oogen kijkend}\\

\haiku{Je ziet, 't klopt, klopt,...}{met de dingen waarover we}{zoo dikwijls spraken}\\

\haiku{Hij had 'r geen trek...}{in bij Duczi op heete}{kolen te zitten}\\

\haiku{{\textquoteright}, viel Erich geprikkeld - ' ' '?...}{uit zatt gedrochtmr}{tusschen te nemen}\\

\haiku{- naar 't z'n sigaar '.}{bepruimend monstertje bij}{t ladenbureau}\\

\haiku{{\textquoteright} Met oogen van diepste.}{walging keek Erich den in z'n}{stoel hijgende aan}\\

\haiku{Je stikte door de,,.}{bedomptheid den schemer de}{vadzige etenslucht}\\

\haiku{Al z'n leven was - '!}{de angst hoe lijkwit had-ie}{r niet uitgezien}\\

\haiku{Ik wou dat ik uw{\textquoteright},:}{jaren en uw gezondheid}{had zei ze goedig}\\

\haiku{Niettegenstaande ' {\textquoteleft}}{t bordjeUntertaillen}{besonders preiswert}\\

\haiku{Duizend tegen een...}{had-ie den knoop aan z'n vriend}{Poldi gegeven}\\

\haiku{Hij noemde 'r den ' -.}{naam vann restaurant in}{de buurt ze knikte}\\

\haiku{Zoolang 't rijtuig,.}{sjokkend door de straten glee}{spraken ze geen woord}\\

\haiku{Ben 'k niet altijd ' -?...}{n goed vriend voor jou en de}{anderen geweest}\\

\haiku{Dan sturen we 'n, '!...}{telegram dat je zuster}{n gekken kop heeft}\\

\haiku{van 'n leien dak,,!}{had je geen hulp van vreemden}{en wat voor vreemden}\\

\haiku{Wat lief, wat schattig,,' '.}{mijmerde ze en  r}{adem zong inr hoofd}\\

\haiku{Daar stond ze als 'n '.}{opgedirkte zottin met}{r zijden blouse}\\

\haiku{Twee-hoog ontmoette,.}{ze den Italiaan dien ze}{liefst mijlen ontweek}\\

\haiku{{\textquoteleft}zijn uw mama en...{\textquoteright} {\textquoteleft}?}{uw zusterIs Erich Sch\"uler in}{Friedrichshagen}\\

\haiku{... - schreef-ie, dat-ie...{\textquoteright}}{zich met uw nichtje Betty}{definitief had}\\

\haiku{De glazen staan in '...{\textquoteright} {\textquoteleft}{\textquoteright},,.}{t buffetGraag zei-ie blij}{dat ze op dreef kwam}\\

\haiku{En als-ie wel op,,!...}{de binnenplaats was was-ie}{even ver hahaha}\\

\haiku{{\textquoteright} En aan de overzij,, ':}{bij Caf\'e Bauer klonkt}{scherp en bevelend}\\

\haiku{Ze zei alleen dat, '.}{ze zoo moe was dat ze niet}{uitr oogen kon zien}\\

\haiku{Anders valt-ie...{\textquoteright} {\textquoteleft}!}{onder de termen vanWel}{God-allemachtig}\\

\haiku{Nee, onder deze. '}{omstandigheden mocht ze}{niet vervoerd worden}\\

\haiku{n Stoel bij 't bed, '.}{schuivend warmde-ier hand}{tusschen de zijne}\\

\haiku{Abfertigung der,!}{unter Kontrolle stehen}{den Personen hier}\\

\haiku{De eenige, die 'r,.}{niet met vragen lastig viel}{was Poldi R\"ose}\\

\haiku{Dat zou 'n laagheid, '... '}{zijnn afschuwelijke}{gemeenheidk Heb}\\

\haiku{Lieve Poldi, ik, '...}{smeek je laat mer niet meer}{aan terugdenken}\\

\haiku{Ze schudde 't hoofd - '}{ze hieldr omkringde oogen}{niet van de lamp af}\\

\subsection{Uit: Kamertjeszonde}

\haiku{Indien u mij in,:}{gemoede vraagt w\`at met het}{manuscript te doen}\\

\haiku{Nog lag ik niet in,.}{mijn stoel of de bel begon}{vinniger dan straks}\\

\haiku{'k Heb vergeten.}{de krenten te wasschen door}{de herrie van straks}\\

\haiku{we elkander het,,,.}{h\'e\'ele leven dag aan dag avond}{aan avond uur aan uur}\\

\haiku{maken - Juffrouw{\textquoteright}..... maar.}{dan weet-ie den naam van}{de oude vrouw niet}\\

\haiku{Bij den bakker aan,.}{de overzij haalt Dirk een versch}{p\`asgebakken brood}\\

\haiku{Van haar vingers druipt,.}{vet op de japon wat ze}{met haar mouw afveegt}\\

\haiku{{\textquoteright} - {\textquoteleft}Jij ben de heele!}{avond al zoo verrekt saai en}{vervelend geweest}\\

\haiku{{\textquoteright} - {\textquoteleft}'t Is gevaarlijk '.}{om iets an te halen met}{n getrouwde vrouw}\\

\haiku{{\textquoteleft}\`als het goed is voor,?}{jezelf waarom deugt het dan}{niet voor anderen}\\

\haiku{{\textquoteright} - Georgine laat,.}{mijn arm los gaat inloopen}{tusschen Guus en Duif}\\

\haiku{Toen ben ik voor hem '.}{en de kinderen weer an}{t zingen gegaan}\\

\haiku{Nou mot je weten,{\textquoteright}....}{da'k al om twaalf uur in me}{bed lag met maagpijn}\\

\haiku{Maar \`als ik niet te....}{eten had of mijn kinderen}{niet te eten hadden}\\

\haiku{{\textquoteleft}'k hoor je liever,....}{vertellen van honger en}{armoede dan van}\\

\haiku{{\textquoteright} roept de vrouw van den. - {\textquoteleft} ',?}{acteurDrink-ie mee van}{t rondje Sanet}\\

\haiku{Het kind staat in haar,.}{nachtpon op bloote voetjes zoo uit}{het bed gesprongen}\\

\haiku{Moosie, je vader meent ',,!}{t best met je Kom nou Moos}{wees nou verstandig}\\

\haiku{No\`u toen ik dat zei,.}{van dat kind had je oom en}{tante moeten zien}\\

\haiku{Hij me na een paar,?}{minuten achterna Waar}{is je huissleutel}\\

\haiku{{\textquoteright} {\textquoteleft}Voor die spijt 't me. '{\textquoteright}.... {\textquoteleft}'.}{Da'sn goeie vrouwk Zou ze}{twee woorden schrijven}\\

\haiku{Leipie holde de,, '.}{trap af pakte d'r in d'r}{nek smeetr op straat}\\

\haiku{Toen ben ik om half{\textquoteright}....}{zes even in de Bodega}{geweest om te zien}\\

\haiku{Wegens huwelijk,....}{van de tegenwoordige}{vraagt men te Arnhem}\\

\haiku{Hier ben ik, ik, zij ',.}{n chanteuse die lol in}{haar bordeelvak heeft}\\

\haiku{Ze is in rose.}{baltoilet met opstaanden}{kraag van geel satijn}\\

\haiku{'k Vind haar zoo wel.}{frisch met haar meisjesgezicht}{en haar zwarte oogen}\\

\haiku{Ik kijk naast Lisy, '.}{in den spiegel wrijft goed}{weg met m'n zakdoek}\\

\haiku{{\textquoteright}  {\textquoteleft}Zoo gingen wij{\textquoteright}}{naar de schouwburg heen Ad\`ele was}{niet erg vlug ter been}\\

\haiku{{\textquoteleft}Wacht u effen hier,{\textquoteright} - {\textquoteleft}.}{zegt de portieruw kamer}{is an de voorkant}\\

\haiku{Een, twee minuten,.}{alleen in de gang met de}{wuivende kaarsvlam}\\

\haiku{{\textquoteleft}Willen we is gaan '?}{toeren naar Scheveningen}{inn open bakkie}\\

\haiku{'k Maal om niks, om '....}{niks waarn hoop andere}{menschen om malen}\\

\haiku{Wat mot dat heerlijk.... '....}{zijn Watn akelig idee in}{zoo'n stad te leggen}\\

\haiku{Als je 'n man niet, '.}{volkomen vertrouwt mo\`et je}{niet metm trouwen}\\

\haiku{Hier h\`e-je mijn.... '{\textquoteright}....}{zakdoekk Heb d'r m'n neus}{nog niet in geveegd}\\

\haiku{{\textquoteright} {\textquoteleft}Als je an tafel.}{komt zitten en je hoed en}{je mantel uitdoet}\\

\haiku{{\textquoteleft}Dat zegt-ie Alf om,{\textquoteright}.}{niet met de deur in huis te}{vallen knipoogt Scherp}\\

\haiku{Ik kan niet gelooven}{dat u met v\'o\'orwetenschap}{m'n neef die ziek is}\\

\haiku{Ik verzeker je ',{\textquoteright},.}{da'kt je inpeper zei}{ik me inzeepend}\\

\haiku{{\textquoteright} {\textquoteleft}Wat doe je al niet.}{als de jonge meiden van}{rechts naar links dwijlen}\\

\haiku{Scherp sloeg de huisdeur, '.}{dicht kwam de keuken binnen}{metn hoop pakjes}\\

\haiku{{\textquoteright} {\textquoteleft}Wel nee, Sorsien, 'k '.... '....}{kom niet omt-geldt}{geld kan wel wachte}\\

\haiku{Kijk me goed anrecht -, -?}{in de oogen vrouw vr\'ouw vertrouw}{je me heelemaal}\\

\haiku{k dat-ie me - '.}{vermoorden zou stopt dek}{is achter me rug}\\

\haiku{- Toen h\'e'k drie weke '....}{geleefd metn officier}{van de marine}\\

\haiku{Dan kocht ik wel is ',....}{n zes pond paling ging die}{vrouwen tracteeren}\\

\haiku{Mok me dan zoolang}{verstoppe in je kelder}{of stel je me voor}\\

\haiku{Hij gaat met ons mee, '.}{alsk Georgine naar}{de Albert Cuyp breng}\\

\haiku{Als 'k nou met de, ' ' '....}{Ruhr begin benkr weer}{n end bovenop}\\

\haiku{Je mot is nagaan '....}{w\`atr an kolen verbruikt}{wordt in \'e\'en winter}\\

\haiku{In eens breekt-ie af,, ' {\textquoteleft}....}{loopt plomp naast me zegt nan}{poosje haast-woedend}\\

\haiku{{\textquoteleft}Daar kan jouw dikke -.}{kont niet tusschen kom maar n\`aast}{me in de leunstoel}\\

\haiku{Toen ik om twaalef,{\textquoteright}....}{uur opstond had ze nog niks}{gedaan die luie vlerk}\\

\haiku{Wie het jou geleerd?....}{ouwere mense in de}{rede te vallen}\\

\haiku{Met z'n kippige '.}{oogen zaagt-ie sneden vann}{halve vinger dik}\\

\haiku{- Meijer, de vader,,.}{van Stientje die was altijd}{verkouwen zei Duif}\\

\haiku{{\textquoteright} - {\textquoteleft}'k Heb schijt an jou,{\textquoteright} ':}{\`en an jou plappert Duif met}{n dikke tong uit}\\

\haiku{- Wat goddelek, om nou is ' -,?}{n heele nacht bij elkaar}{te zijn de \'e\'erste h\`e}\\

\haiku{{\textquoteleft}'t Is beter om '.}{Pijp-juffrouwe opn}{afstand te houen}\\

\haiku{{\textquoteright} Juffrouw Thomas brengt ',.}{n schotel bieten hel-rood}{in de witte schaal}\\

\haiku{Nou-ou, ik kan toch!}{niet op me bloote voeten naar}{beneje loopen}\\

\haiku{Maar als ze je weer,,:}{wat vrage van mamma hoor}{je moet je zegge}\\

\haiku{Geef an de kruier ',.}{t adres mee dan kom ik je}{van avond opzoeken}\\

\haiku{Voor 'n man is 't '.}{h\'e\'el gemakkelijk alsn}{vrouw z'n dienstmeid is}\\

\haiku{'t Is te voelen,.}{terwijl de leege woorden door}{de kamer drensen}\\

\haiku{{\textquoteleft}'k heb is eens 'n - '....}{kies late trekke dat zal}{k nooit vergete}\\

\haiku{en m'n eed toen - he'k ' '.}{je verteld datk toenn}{eed gezworen heb}\\

\haiku{Dat lijkt alles mal ' ',?}{opgewonden alskt}{hier zoo vertel h\`e}\\

\haiku{- Zij had gezegd dat ' '}{ze van me hield en datt}{r niks kon schelen}\\

\haiku{Over 'n paar weken.}{was ze over d'r tijd en was}{ze de machien kwijt}\\

\haiku{Maar 't grappigst was '.}{t eigenwijs opzetten}{van nieuwe regels}\\

\haiku{k Wil nou niet{\textquoteright} - en.}{ze ging in groote drukte de}{tafel afruimen}\\

\haiku{{\textquoteright} {\textquoteleft}Got, da's wel aardig,{\textquoteright},.}{zei juffrouw Doedelaar den}{doek verknuffelend}\\

\haiku{{\textquoteright} bobbelde juffrouw - {\textquoteleft},?}{Doedelaar met veel pretwat}{is v\`ochtig juffrouw}\\

\haiku{die kronkelt in de,,{\textquoteright}....}{verdieping da's arremoe}{siekte en hoofdpijn}\\

\haiku{Met z\`ulke kleine.}{oogjes stond je naar d'r bloote}{arme te kijke}\\

\haiku{En bij het licht, met,.}{woedend-harde rukken trok}{ze het portret stuk}\\

\haiku{Zij aan 't praten, ' '.}{aant overtuigen metn}{begin van berouw}\\

\haiku{Toen likte zij het, '.}{roervingertje af wou me}{t kopje brengen}\\

\haiku{{\textquoteleft}Asjeblief meneer '{\textquoteright} {\textquoteleft}.}{en w\`el bekommet u.}{Dank u wel mevrouw}\\

\haiku{heeft ze 't leven...}{zoo verzuurd dat-ie zich}{opgehangen heeft}\\

\haiku{maar tegenwoordig, '}{tegenwoordig benk toch}{wel een goed vrouwtje}\\

\haiku{Als ik me d'r niet, '.}{mee bemoeid had zout v\'eel}{beter geweest zijn}\\

\haiku{Wiel u asjeblief.}{er voortaan aan denke dat}{v\'oor die wienkel ies}\\

\haiku{ze zijn heel mooi,{\textquoteright} zei,.}{ik verheugd met meer honger}{dan lust tot gepraat}\\

\haiku{- {\textquoteleft}Oe, wat kn\`ars u oom,{\textquoteright},.}{zei Tilly de vingers in}{de ooren stoppend}\\

\haiku{{\textquoteright} - Zoo toonloos rustig ', '.}{als zet zei schriktet}{me w\`onderlijk op}\\

\haiku{Waarom was 'k maar?}{niet met Georgine \`en}{Scherp \`en Kaatje samen}\\

\haiku{Als je bericht kreeg - '?}{dat-ze d\'o\'od waren}{zout je wat doen}\\

\haiku{{\textquoteleft}'n jong mensch m\'ag iets ' -,?}{meer doen dann jong meisje}{wat zeg jij Jules}\\

\haiku{Zal 'k u ook is,{\textquoteright}.... - {\textquoteleft},,}{even legge meneerWel ja}{leg u mij ook is}\\

\haiku{{\textquoteright} - In 'n dronkenschap,, '.}{van herinnering springt ze}{op slaatn cancan}\\

\haiku{{\textquoteright} Om de lamp zaten,.}{we met z'n drie\"en pratend}{tot diep in den nacht}\\

\haiku{het was kaaremelk u}{begrijp jan was woedent en}{niet opgegeeten}\\

\haiku{{\textquoteleft}zij en Jan zijn nog,.}{niet bij me geweest z\'oolang}{ik met Alfred ben}\\

\haiku{Of 'k ook niet vond?}{dat Bij de wieg van Jan van}{Beers goddelijk was}\\

\haiku{Da's \'ook 'n mislukt,{\textquoteright} -....}{kaaretje meende schoonmama}{Kloos opnemend}\\

\haiku{Voor veertien dagen ' '{\textquoteright}.... {\textquoteleft}....}{zagkr nog gezond en}{welHartkwaal Meijer}\\

\haiku{Op het bruin geverfd,.}{haar spiegel-zwartte de}{glimmende hooge hoed}\\

\haiku{Guus, snikkerig, groot,.}{van vrouwelijk meelijden}{liep op Meijer toe}\\

\haiku{- Je \'eigen moeder,....}{die me niet gegund heeft je}{oogen toe te drukke}\\

\haiku{{\textquoteleft}Arme, arme Stien{\textquoteright} - - {\textquoteleft}.}{begon Meijer opnieuw\`arm}{ongelukkig kind}\\

\haiku{{\textquoteleft}'k Dacht niet dat 't,{\textquoteright} -.}{z\'oo gauw zou weze zei Dirk}{b\'ang voor die stilte}\\

\haiku{'k Wil die vent niet....{\textquoteright} {\textquoteleft} ';}{meer hier hebbeOf je di\`e}{neemt ofn ander}\\

\haiku{{\textquoteright} {\textquoteleft}Wat gezellig, h\`e,,?}{oome da'k u en mamma}{kan zien legge h\`e}\\

\haiku{{\textquoteright} {\textquoteleft}Dank u.{\textquoteright} 'k Stond al,. {\textquoteleft}......}{op de trap toen ze me nog}{even nariep Meneer}\\

\haiku{{\textquoteright}.... - {\textquoteleft}Over 't geld hoef u,{\textquoteright},.}{niet ongerust te zijn zei}{ik ongeduldig}\\

\haiku{k Sal nou maar gauw, '....}{weggaan h\`e anders ist}{w{\`\i}sselkantoor dicht}\\

\haiku{Bax was 's middags,.}{geweest had den toestand vrij}{gunstig gevonden}\\

\haiku{Ik keek haar alleen:}{maar aan en plots sloeg ze de}{armen om mijn hals}\\

\haiku{O God, Alf - je mot - '....}{niet d\`enke over wa'k zegk}{ben zoo ellendig}\\

\haiku{Want w\'e\'et je wel toen ',....}{jet \'e\'erst bij me kwam die}{avond bij juffrouw Bok}\\

\haiku{H\'e'k 'r nou n\`og niet?....}{twee weken geleden vijf}{dollars gegeven}\\

\haiku{U is te veel man '.}{van de wereld om niet te}{weten hoet hoort}\\

\haiku{Ja, al schud jij nog,, '.}{zoo je kop Dirkt kan me}{geen bliksem schelen}\\

\haiku{Wilt u nog trachten?}{wat te doen voor het behoud}{van de kinderen}\\

\haiku{Die vent en z'n vrouw '.}{moestk natuurlijk op mijn}{hand zien te krijgen}\\

\haiku{Iek weet niet wat voor.}{soort laarze dat zijn en Frits}{ies mienderjarig}\\

\haiku{{\textquoteright} Juffrouw Doedelaar ',.}{haalden sleutel uit haar}{zak dee mijn deur open}\\

\haiku{Da's 't onderscheid!}{en verder ontloope jullie}{mekaar geen flikker}\\

\haiku{Jullie scheppe hier.}{de peentjes op alsof je}{w\`onder wat inbrengt}\\

\haiku{Met 't h\`oofd in 't,,,.}{kussen moe toch-glimlachend}{luisterde ik weer}\\

\haiku{we ontvangen haar}{zoo goed en nu krijgen we}{geen taal of teeken}\\

\haiku{ik heb al teege pa}{gezegt als alfred soms een}{baantje voor hem heb}\\

\haiku{Nou leg ik leeg zoo,,.}{all\'e\'en zoo all\'e\'en op m'n}{bed in m'n bedstee}\\

\haiku{Zulleke schape,,?}{heelemaal geen lucht en pas}{ziek geweest niewaar}\\

\haiku{- Jeezis mierande, denk -!}{ik en ik de trap af as}{de verdommenis}\\

\haiku{toen ben 'k gauw met '!}{me kont op de tram gaan staan}{en nou weet jet}\\

\haiku{Maar 'k vin 't toch....}{beestachtig om zoo'n schaap nou}{weg te late gaan}\\

\haiku{Over 'n paar maande '....}{make zet zeker an}{Ka wijs da'k d\'o\'od ben}\\

\haiku{Ik verroerde me,.}{niet heeschwarm tintelend van}{zenuw-inspanning}\\

\haiku{De juffrouw vroeg of ' '.}{t voorn uurtje was of}{voor den heelen nacht}\\

\subsection{Uit: Kleine verschrikkingen (onder ps. S. Falkland)}

\haiku{Al dat geknoei most.}{ze in d'r nieuwe huis in}{Arnhem nie hebbe}\\

\haiku{Nog eens knikkend en '.}{glimlachend liep de brave}{kerel overt veld}\\

\haiku{Maar ze hadden een.}{anderen koers genomen}{of vreesden de kust}\\

\haiku{Het water gladde,.}{zonder geruisch zonder}{slag tegen het strand}\\

\haiku{De dag die zoo klaar,. '}{en luchtig begonnen was}{vaalde in scheemring}\\

\haiku{De menschen aan 't,.}{strand weken terug trokken}{de kinderen weg}\\

\haiku{dee\"en we al de '.}{povere dingen dien}{dokter gewoon is}\\

\haiku{Dan schuierde hij.}{weer en mijn handen trokken}{en hieven de polsjes}\\

\haiku{Met m'n vrouw samen,.}{maakte ze fleschjes klaar}{voor de zuigeling}\\

\haiku{Ja, dat zou je nie - '.}{geloovek hei in geen}{maande geslape}\\

\haiku{Dan bleekte de maan,.}{weer schuchtere glansjes door}{wolke-rand spinnend}\\

\haiku{Wij zijn 't!{\textquoteright} -, riepen,.}{we toen de buurvrouw angstig}{over de schutting keek}\\

\haiku{'k Ben gister den,{\textquoteright}.}{heelen dag van streek geweest}{zei mevrouw Dewaard}\\

\haiku{Op zoo'n manier zou.}{je je heele familie}{motten opsluiten}\\

\haiku{{\textquoteleft}'k Ben vast an 't,{\textquoteright}:}{kissie begonnen sprak-ie}{in afwezigheid}\\

\haiku{We hadden mekaar ' '.}{nog \'e\'ens ontmoet opt}{wegje naart strand}\\

\haiku{{\textquoteright} Hij knikte en 'n '.}{oogenblik later zatk}{mee aan de tafel}\\

\haiku{Even bekeek-ie 'm, '.}{van onder tot boven toen}{liet-iem vallen}\\

\haiku{Jan rookte, hevig,.}{de lucht doorhappend Chris en}{Trien vlochten bloemen}\\

\haiku{{\textquoteleft}Mijn krans is klaar,{\textquoteright} riep, ' '.}{Trien opspringendt groen van}{r boezelaar slaand}\\

\haiku{Toen wou Jan w\`eten ', '.}{hoe diept graf was maar Chris}{wout niet hebbe}\\

\haiku{{\textquoteleft}Ja - ja{\textquoteright}.  'r Oogen, '.}{gloeiden in de kassenr}{adem snoof gejaagd}\\

\haiku{{\textquoteright} {\textquoteleft}'n Kind blijft 'n kind - ',{\textquoteright}.}{n kind denkt niet na zei de}{wijze man nog eens}\\

\haiku{{\textquoteleft}{\`\i}k weet 't ook nie, ' '}{wantk mot binnent uur}{an de slachterij}\\

\haiku{{\textquoteright} sprak hij gelijk met,,.}{me op stappend linkervoet}{voor rechtervoet voor}\\

\haiku{Zij en Suus waren,.}{twee beesten bij tijjen}{varkes van meiden}\\

\haiku{{\textquoteleft}'t Is 'n slag... 'n '}{Moeder blijftn moeder al}{wordt ze n\`og zoo oud}\\

\haiku{De magere  .}{hand van knokels en vel scheen}{vleezig te worden}\\

\haiku{Vrouw Abel, verveeld, hield '.}{de boodschappenmand op de}{ronding vanr arm}\\

\haiku{De kist zal jou nie,{\textquoteright}.}{opvrete zei moeder met}{onrustige oogen}\\

\haiku{Den heelen avond was ', '.}{het daarn kwijlend gezoen}{n loddrig gegil}\\

\haiku{We krijgen nog nie '...}{eens de leege kisten vant}{goed na de zolder}\\

\haiku{Want 'n mensch ken d'r, '}{op en d'r af maarn kist}{al was die zoo klein}\\

\haiku{Ze was verzot op '.}{histories enn lijk had}{ze nog nooit gezien}\\

\haiku{Het deurgat zwalpte,.}{licht in de kamer tot bij}{de bruining der kist}\\

\haiku{{\textquoteleft}En hij is al weer, ' -,}{weg net watk zei hij is}{van hier gekommen}\\

\haiku{te bezwijke - zoo'n -.}{misselijke h\`ette}{zoo'n zon in je rug}\\

\haiku{Beter dat Bram 't}{nie beleeft dat zijn zoon met}{\`anzien terug komp}\\

\haiku{{\textquoteleft}dat is de afspraak ', ' '.}{dat weetk wel maar as jij}{tm zoo rauw zeit}\\

\haiku{de \`andere pijp{\textquoteright}... {\textquoteleft}',{\textquoteright}:}{van de schoorsteenn Sleepboot}{hardnekkigte oom}\\

\haiku{{\textquoteleft}Ik maak me nie blij ',{\textquoteright};}{metn dooie mosch zei oom}{wanhopig-kalm}\\

\haiku{{\textquoteleft}'t ken de boot van ' '.}{Sally zijn ent kenn}{\`andre boot weze}\\

\haiku{Ons mag-die t\`och,{\textquoteright}... {\textquoteleft}}{nie dadelijk zien hebbe}{we afgesproke}\\

\haiku{{\textquoteright} {\textquoteleft}As Mau 't 'm zoo - ' '{\textquoteright}... {\textquoteleft}!}{ra\'uw zeit as Mautr zoo}{maar uitpl\`appertN\`og}\\

\subsection{Uit: Schetsen. Deel 1 (onder ps. Samuel Falkland)}

\haiku{De violen, paars,.}{en geel fluweelen mollig}{naast de narcissen}\\

\haiku{Een breede ademplas.}{is als een aureool van}{bleekheid om zijn hoofd}\\

\haiku{Mijn God, wat zou er?}{van ons worden als er niet}{voor ons gekookt werd}\\

\haiku{Al de aschbakjes.}{van d'r familie leegde}{ze in een toetje}\\

\haiku{Ze liep rechtuit de,,.}{Weesperstraat in rondkijkend beduusd}{door zooveel menschen}\\

\haiku{Tot elf uur 's avonds.}{had-ie in het ruim van het}{kolenschip gewerkt}\\

\haiku{{\textquoteright} Angstig kroop-ie, ',.}{weg achtern reuzenzuil}{net in de schaduw}\\

\haiku{Je ziet alleen maar...,...}{z'n nakende voeten dan}{bont wit en rood bont}\\

\haiku{George begon.}{te huilen en de vrouwen}{keken krijtwit toe}\\

\haiku{Gister had zij ze,.}{gezien de werkeloozen in}{een langen optocht}\\

\haiku{De meneer, die met.}{de drie meisjes opkomt heeft}{niets buitengewoons}\\

\haiku{dikke kolommen,,.}{d\`an weer verwasemend tot}{mist dan vettig-wit}\\

\haiku{Als ze nou maar wat,.}{zure balletjes had wat}{zure balletjes}\\

\haiku{Maar dichtbij glijden,.}{ze effen weg als een mes}{dat over een plank slijpt}\\

\haiku{Ze had, als al de,.}{dames der gelegenheid}{een wit voorschootje voor}\\

\haiku{Pootig trapt-ie.}{met z'n hielen en slaat ze}{aan tegen z'n broek}\\

\haiku{Eerst 'n kind met 'n,.}{omslagdoek bleek onder de}{roodte van den doek}\\

\haiku{Onder den breeden,.}{stroohoed wuifde het haar dansend}{op den lentewind}\\

\haiku{* * * ~ Bij de huisdeur,, '.}{in een rieten stoel zat ze}{int zonnetje}\\

\haiku{Maar bij Wies was juist.}{het bijzondere aan het}{hoofd en de voeten}\\

\haiku{De twee lijvige,.}{paarse pompoenkoonen werden}{dus het eerst gezien}\\

\haiku{Als je fatsoenlijk,.}{man bent laat je je vrouw niet}{zoo berooid achter}\\

\haiku{h\`em, haar echtgenoot,.}{voor God den vader van haar}{pasgeboren kind}\\

\haiku{Dat heb je d'r van...* * *}{als je van die opvreters}{in huis neemt}\\

\haiku{ook zie je prachtig,.}{den Amstel met de bruggen}{rechts aan de overzij}\\

\haiku{Nou zal-ie blijven,...}{kijken naar zijn bord tot er}{w\'e\'er wat gezegd wordt}\\

\haiku{- Nou waren ze 'm '... '...}{ant beklagen ant}{bekla-a-age}\\

\haiku{nou is-ie dood voor.}{me. Achter de gordijnen}{zat ze sjiwwe}\\

\haiku{Ze hield de oogen niet,...}{af van de instrumenten}{stootend ademhalend}\\

\haiku{Je zou zwere dat '...}{d'r iemand int donker}{op de deur klopte}\\

\haiku{Zoo'n raam er uit, vind.}{ik het verschrikkelijkst van}{een verhuizerij}\\

\haiku{Hij zal toch wel wat,?}{tegen je gezeid hebben}{toen-ie nog hier was}\\

\haiku{Het wekkertje, dat,.}{om zes uur moest afloopen}{tikte kwaadaardig}\\

\haiku{Waarom heb je zoo -?}{lang met zulke groote oogen in}{het gas gekeken}\\

\haiku{Even staat-ie met het.}{touw om zijn middel op den}{rand van het bootje}\\

\haiku{Ik zat weer in mijn,.}{stoel zenuwachtig met het}{boek in de handen}\\

\haiku{Mijnheer Falkland,.}{ik ben maar zoo vrij geweest}{om mee te kommen}\\

\haiku{Zoo'n stilte moet er,.}{om Perrette geweest zijn}{toen de melkkan viel}\\

\haiku{Het is de zee bij,.}{het strand opkolkend golven}{in branding van schuim}\\

\haiku{Maar ze klapklapt naar.}{de gele bloemen boven}{de groene planten}\\

\haiku{{\textquoteright} De merrie stond schuin.}{op de tuintrap en snoof}{de biefstuklucht op}\\

\haiku{De meid maakte de,.}{buitendeur open veegde de}{voeten op de mat}\\

\subsection{Uit: Schetsen. Deel 2 (onder ps. Samuel Falkland)}

\haiku{Hij zat op de bank,.}{kijkend naar de kinderen}{die niet meer speelden}\\

\haiku{Ze zaten dicht bij,.}{nu hoofdjes gekeerd naar de}{zij van het boschje}\\

\haiku{Elken morgen reed;}{Daantje den rolstoel achter}{de aanrechtbank}\\

\haiku{{\textquoteleft}Wat zie jij me vuil,{\textquoteright}}{zij Suus en haar zakdoek met}{speeksel benattend}\\

\haiku{Ant nu gemaklijk.}{slobberde koffie en Suus}{roerde haar kopje}\\

\haiku{Aan het hoofd van de.}{tafels waren armstoelen}{voor de verpleegsters}\\

\haiku{Nu wou ik u nog '....}{eens laten zien hoet kind}{is bijgekomen}\\

\haiku{{\textquoteright}, informeerde Saar,.}{het gele gezicht dicht bij}{den schijn van het raam}\\

\haiku{z\'oo heeft iedereen -.}{ze zoo had ik ze al toen}{ik twintig jaar was}\\

\haiku{we waren nog steeds,,.}{verloofd verliefd genoten}{van de buitenlucht}\\

\haiku{Het tuintje, gesmoord,.}{in hooge houten schuttingen}{was triestig en zwart}\\

\haiku{In 't keukentje.}{hoorde hij haar scharrelen}{met den ketel}\\

\haiku{Inne...{\textquoteright} - verhaalde:}{l\`ang nadrukkelijk de vrouw}{van den herbergier}\\

\haiku{Toch begint ons paard.}{met groote opgewektheid van}{de ruif te vreten}\\

\haiku{{\textquoteleft}Wacht maar,{\textquoteright} zegt Lou en '.}{nog heeft hijt niet gezegd}{of Sorrie blijft staan}\\

\haiku{Alles is bij hem,,,.}{toegesmeerd zijn ooren zijn}{neusgaten zijn mond}\\

\haiku{Voor 'n daalder in,.}{de week maggie blij zijn als}{je onder dak ben}\\

\haiku{Wat mosten wij met?}{bij de tweehonderd gulden}{beestenpoeier doen}\\

\haiku{Want z\'o\'o'n bestelling '.}{geeft toch geen smid inn plaats}{van 60 inwoners}\\

\haiku{Wij hopen deze,.}{proefneming te herhalen}{hoogaanzienlijken}\\

\haiku{- Om dezen toorn te -}{motiveeren ge ziet dat}{ik wel zeer kalm ben}\\

\haiku{Die verweet je al, '}{de zonde van de wereld}{int bijzonder}\\

\haiku{Mevrouw het 'r wat '.}{saucijsjes enn stukkie}{klapstuk bijgedaan}\\

\haiku{Je mot rekenen, '...}{datk niks anders eet dan}{brood den heelen dag}\\

\haiku{Nu was zijn hoofd wat,,.}{meer gebogen in vreemde}{doffe gedachten}\\

\haiku{Het paard hinnekte,.}{zwiepte met den langen staart}{van witte haren}\\

\haiku{Ik zeg 't u nog, '.}{eens zegt u nog eens dat}{ik wil zijn all\'e\'en}\\

\haiku{In naam van God, in ',.}{naam vant kruis dat ik dien}{gebied ik n\'og eens}\\

\haiku{De hooge duin aan de.}{zeezij stompte op in den}{melkwitten hemel}\\

\haiku{'t Was de eerste, ' '.}{maal dat-iet land en}{t water zoo zag}\\

\haiku{Weer luisterde hij '.}{met starende oogen bijt}{klein stukje aarde}\\

\haiku{Edelachtbare, wat,?}{is er van den nacht van den}{Amsterdamschen nacht}\\

\haiku{Edelachtbare, wat,?}{is er van den nacht van den}{Amsterdamschen nacht}\\

\haiku{Mijnheer Duimelaar.}{leid met beslistheid z'n mes}{neer en luisterde}\\

\haiku{het zal noodig zijn u}{iets mede te deelen van}{mevrouw \`en meneer}\\

\haiku{'k Had de woning!}{wel ses-en-dertig keer}{kenne verhure}\\

\haiku{{\textquoteleft}... daar had 'k nou z\'o\'o.}{op gevlast om jullie me}{pertret te geve}\\

\haiku{- Als je dood ben, wat '?}{blijftr dan anders van je}{over dan je pertret}\\

\haiku{Boompjes en struiken,,.}{dor van leven harken in}{de blauwige lucht}\\

\subsection{Uit: Schetsen. Deel 3 (onder ps. Samuel Falkland)}

\haiku{dicht en z'n kouwe}{hand op haar w\'arme met den}{engagementsring}\\

\haiku{Je mot an vader '......}{zegge da'km niet achter}{me achter me lijk}\\

\haiku{en die kuiltjes van.......}{wit vel d'r tussche en z'n}{ongeschore kin}\\

\haiku{Vader stond altijd,.}{nog wat gebogen correct}{en stil naast moeder}\\

\haiku{Maar in ernstige,.}{gedachten ging ik naast mijn}{zwager die bleek zag}\\

\haiku{waaraan ik met groote,.}{moeite geholpen door de}{bemanning voldeed}\\

\haiku{Nooit zal ik den dag.}{vergeten toen de laatste}{kletskop verdeeld werd}\\

\haiku{Een ploeg arbeiders.}{van de sneeuwreiniging hield}{een vergadering}\\

\haiku{O, droevig is de.}{schimmende herinnering}{aan verre tijden}\\

\haiku{Van de gasten was '.}{de juffrouw van beneden}{t langst gebleven}\\

\haiku{Gek, h\`e.... als 'k me.}{waterchocola niet heb}{is me dag niet goed}\\

\haiku{De kamer had vier,.}{ramen zonder gordijnen}{m\`et jaloezie\"en}\\

\haiku{Komiek ook je moe\"e '}{hoofd inn hotelkussen}{en den heelen nacht}\\

\haiku{als je ligt met je,}{gesloten oogen op je warm}{ingedeukt kussen}\\

\haiku{Het is niet alles,.}{op te noemen wat er bij}{paarden te kijk is}\\

\haiku{Na de vijfde of.}{zesde lepel begint hij}{langzaam te drinken}\\

\haiku{{\textquoteright} Maar de moeder hief,:}{haar op kuste haar op oogen}{en wangen en zei}\\

\haiku{Elk oogenblik van.}{den dag kwam hij en plaatste zijn}{hand bij de tralies}\\

\haiku{De man wist er niets,.}{van en keek met verrassing}{toe toen hij thuis kwam}\\

\haiku{Leeuwtje zat over me., {\textquoteleft}.}{Leeuwtje is een klein ventje}{met grooteidealen}\\

\haiku{{\textquoteright} {\textquoteleft}Als je je mond had,.}{gehouden had je nu al}{alles geweten}\\

\haiku{Toen, Sam, je mag dat,.}{nou gek vinden toen ben ik}{ook stil geworden}\\

\haiku{Soo'n goszonde om '...}{n tapijt wat je pas krijg}{da\`alek neer te legge}\\

\haiku{Precies zoo iets van.}{ha\'ar om daar nou de heele}{nacht over te zeure}\\

\haiku{Ook zijn manchet was,,,.}{nieuw helder dragend een knoop}{verguld met rooden steen}\\

\haiku{Bij den preekstoel was.}{het zwart psalmenbordje naast}{de collectezak}\\

\haiku{Ze wist 't. Bruigom,,,.}{hoed in hand achter haar aan}{zette zich bij haar}\\

\haiku{{\textquoteright} Over de hoofden der,,}{hoorenden sprak hij half voor de}{vuist half aflezend}\\

\haiku{Zij zouden er een.}{flinke meid aan hebben en}{zorgzaam voor het kind}\\

\haiku{{\textquoteleft}wat 'n val... en dat...{\textquoteright} {\textquoteleft} '!}{je niet w\'e\'et waar je te land}{komtEn watn smak}\\

\haiku{{\textquoteleft}ik wou nou is van,...}{h\`em weten wat-ie met}{Botje wil zeggen}\\

\haiku{As je maar weet, als, '!...}{je maar weet da'kt an je}{vader zal zeggen}\\

\haiku{Wat dan! - Zouen we.... -, '!}{liever nog niet wat Ik zeg}{dat jet aanneemt}\\

\haiku{Bij de deur, buiten,:}{begon de student in een}{heel andere toon}\\

\haiku{bij elkaar krijg je.}{nog meer volgkoetsen dan de}{meneer die daar rijdt}\\

\haiku{Zoo zulks met gr\'o\'oten.}{tact geschiedde zou men niet}{licht er over spreken}\\

\haiku{Doode nummer wordt,.}{niet begraven geannexeerd}{door de wetenschap}\\

\haiku{{\textquoteright}, informeerde de,.}{salon-athleet Oliveira}{wakker geworden}\\

\haiku{De liedjes van den {\textquoteleft}{\textquoteright}.}{vroolijken beedlaar zou hij}{dien avond voordragen}\\

\haiku{{\textquoteright} zei hij mat, de oogen '.}{wrijvend die gloeiden vant}{staan boven het vuur}\\

\haiku{Lieve hemel 'k!}{ben zoo blas\'e en ik vind}{alles zoo banaal}\\

\haiku{Als ze me overdag,.}{konden zien zouden ze me}{n\`og meer bezingen}\\

\haiku{Dat l\'or van jou wil '!}{k nog niet-eens om me man}{z'n boterhamme}\\

\haiku{{\textquoteleft}'t Is me 'n reis{\textquoteright} -,.}{blaasde Saar terug terwijl}{ze hem knietjes gaf}\\

\haiku{Wanneer zou 't weer, {\textquoteleft}{\textquoteright}?}{gebeuren dat hij met z'n}{meissie buiten liep}\\

\subsection{Uit: Schetsen. Deel 4 (onder ps. Samuel Falkland)}

\haiku{Meer bij 't buffet,.}{kniezig en wreed stonden de}{tafels en stoelen}\\

\haiku{Je hoofd voelde zwaar '.}{en je neus was suf vant}{stof in de straten}\\

\haiku{Suf neemt het kind de,,}{nieuwe sigaar bijt ruw af}{de punt en mislijk}\\

\haiku{{\textquoteleft}Meneer,{\textquoteright} begon ze,;}{de worst weder in mijn kast}{plaatsend voor morgen}\\

\haiku{Gewichtig begon,.}{ze pillen te rollen de}{vingers zwart glimmend}\\

\haiku{Maar toen de kip al,.}{gepakt zat begon juffrouw}{Suzan te huilen}\\

\haiku{De melkboer die d\'e\'e ',.}{t \`achter in den tuin met}{een stevigen knauw}\\

\haiku{Op de stoep stond mijn,.}{palfrenier met de paarden}{een bruin en een zwart}\\

\haiku{Evenwel werd ik zeer,.}{onrustig niet wetend hoe}{ik er mee aan moest}\\

\haiku{Steviger dan straks,.}{bevestigde ik het touw}{keek over den gootrand}\\

\haiku{- Diep ademhalen - goed - - '.}{zoo best zoo wel dat loopt van}{n leien dakje}\\

\haiku{{\textquoteright} Zijn stem klonk hard, z\`elfs.}{in het kreunend lawaai van}{wielen en ruiten}\\

\haiku{Sien - in de keuken,?}{heb ik een pakje voor je}{klaar gezet hoor je}\\

\haiku{{\textquoteleft}Hebbe ze je niet '?}{gezegd d\`atr een soldaat}{hier op wacht mot staan}\\

\haiku{Het doek ging omhoog {\textquoteleft}{\textquoteright}.}{en dekomiek van het stuk}{maakte zijn entree}\\

\haiku{In 't koffiehuis,,.}{in den ouwen hoek zaten}{Pam en Hobbema}\\

\haiku{'n Mirakel zoo.}{snel als de vermicelli}{er in ronddraaide}\\

\haiku{Daar ging-die in.}{de kokende pan. En de}{boter werd al bruin}\\

\haiku{{\textquoteleft}'k Heb je nog eens, '.}{laten roepen omdatt}{z\'o\'o niet langer gaat}\\

\haiku{Evenzoo trachtten wij.}{relaties aan te knoopen}{met den kruidenier}\\

\haiku{Aan zieke boeren -.}{had hij een broertje dood als}{ze niet betaalden}\\

\haiku{Ze kauwden een poos -,.}{tegenover elkaar hij groote}{zij kleine happen}\\

\haiku{in 't koffiehuis - ' -}{bleef plakken zou jes avonds}{vroeger thuis komen}\\

\haiku{{\textquoteright} schaterde Jan en.}{de andere knechts stonden}{te schudden van pret}\\

\haiku{{\textquoteright} De hond trok stevig -.}{de man hield alleen maar z'n}{hand op den duwboom}\\

\haiku{Bij me eerste vrouw - '.}{tweemaal en bij deze is}{t de eerste keer}\\

\haiku{{\textquoteleft}Sjongen wat 'n pracht ',{\textquoteright}:}{vann krans hetgeen dezen}{vrijer deed zeggen}\\

\haiku{Bokje had rimpels.}{van grootemansgedachten om}{den gesloten mond}\\

\haiku{{\textquoteleft}Nou-nou,{\textquoteright} troostte ( '):}{Bet weerze moestt wel in}{zijn ooren schr\'eeuwen}\\

\haiku{Ze zou even naar huis,,.}{gaan haar meubeltjes nazien}{wat inkoopen doen}\\

\haiku{Vroeger hadden wij -.}{n\'a\'ast dat Caf\'e gewoond v\'e\'el}{jaren geleden}\\

\haiku{'t Is al mooi dat,,}{je elkaar zoo flauw-weg}{nog weet van vr\`oeger}\\

\haiku{Je dacht jong te zijn,, '.}{levensfuttig maar ann}{spiegel wende je}\\

\haiku{'t Is nou wel geen, ' '.}{sch\`ande maart had tochn}{boel beter gestaan}\\

\haiku{Voor vreemden leek 't.}{precies alsof niemand om}{de dooie  maalde}\\

\haiku{Wat bezielt je om?}{z\'oo je stervende vader}{toe te spreken}\\

\haiku{De erf laat ik na...}{aan de kinderen uit m'n}{tweede huwelijk}\\

\haiku{In den vijver, in,.}{de plassen op de blaeren}{ruischte het neer}\\

\haiku{Nou, ik ben Woensdag ' -?}{met opa uit geweest naart}{Rechthuis weel u wel}\\

\haiku{Ik veronderstel,.}{dat hij z\'e\'er vroeg getrouwd z'n}{vrouw verloren had}\\

\haiku{Onze jeugdige}{kruidenier begon op zijn}{stoel te bedenken}\\

\haiku{Ik hou 't geen uur,!}{meer met die kerel uit die}{ke\`e\`e\`e\`erel}\\

\haiku{Omdat zij deeg an ' '!}{t klaar maken was en hij}{r z'n neus in stak}\\

\haiku{De regen g\'o\'ot op,.}{je mantel op je hoed die}{zwaar  wer as lood}\\

\haiku{Schuw keek zij om zich.}{heen naar de schaduwen van}{boomen en heggen}\\

\haiku{En nou was 'r 'n,.}{plaasie binnen-in open waar}{ze kon uitrusten}\\

\haiku{Ruw schoof hij zijn stoel,,.}{bij het raam kwakte er op}{neer keek naar buiten}\\

\haiku{Nog eens trad hij op,.}{het bed toe betastte haar}{koudperlend voorhoofd}\\

\haiku{Schuw keek hij om of -.}{zij het niet zag of zij de}{oogen gesloten hield}\\

\haiku{liefst een dat honger.}{heeft opdat het v\'anz\'elf de}{armpjes uitstrekke}\\

\subsection{Uit: Schetsen. Deel 5 (onder ps. Samuel Falkland)}

\haiku{En datte is 'n -....}{j\`onge enne die jonge}{het zooveel gepraat}\\

\haiku{De schaar in z'n hand,}{wurmde voorzichtig tusschen}{de ijzerdraadjes}\\

\haiku{Je kan me daar de ',!}{waterleiding metn prop}{papier sluite och}\\

\haiku{Pan no. 6 stond op,.}{het vuur pan no. 7 op een}{petroleumstel}\\

\haiku{Ik ben kapot van,.}{dat groote kerkhof grooter dan}{P\`ere la Chaise}\\

\haiku{n zwakke borst had '....}{wast reden te meer om}{niet te gaan stake}\\

\haiku{Analyse van een.}{gemoedstoestand in verband}{met nieuwe schoenen}\\

\haiku{Raar. 't Kistje kwam.}{te staan op twee stoelen en}{er bij twee kaarsen}\\

\haiku{{\textquoteright}... {\textquoteleft}Ach Jeesis, J\`o,{\textquoteright} zei weer:}{de verteller van den moord}{op het dienstmeisje}\\

\haiku{Let wel, ik leg den,.}{nadruk op het bloed minder}{op de historie}\\

\haiku{{\textquoteleft}Kom aan mijn boesem,, '.}{lievelingk heb je is}{willen beproeven}\\

\haiku{Een postbode op,,.}{een kleine plaats vriend is een}{kerel van gewicht}\\

\haiku{Mijn vrouw slikte zijn,,.}{pillen zijn staaldrankjes werd}{met den dag zwakker}\\

\haiku{{\textquoteleft}je moet co\^ute qui.}{co\^ute voor versterkende}{middelen zorgen}\\

\haiku{{\textquoteright} {\textquoteleft}Vertel,{\textquoteright} zei ik, in,.}{perverse aandachtige}{Falklandluistring}\\

\haiku{Op eens, amice, was '.}{het mij alsof ik een klap}{int gezicht kreeg}\\

\haiku{Van avond zal-die.}{natuurlijk wel niet zingen}{van de vreemdigheid}\\

\haiku{Je doet ganschelijk.}{verkeerd tegen dezen tijd}{obscuur te worden}\\

\haiku{Als ze slapen ging,.}{d\'an op de logeerkamer}{en de deur op slot}\\

\haiku{- Karel die...{\textquoteright} {\textquoteleft}'k Vraag,{\textquoteright},.}{je excuses niet zei ze}{bits hem afwerend}\\

\haiku{'t Was een oude,.}{zwarte kater dien ze al}{meer dan tien jaar had}\\

\haiku{Dan hoorde je ze.}{telkens boven naar de kraan}{loopen en vloeken}\\

\haiku{{\textquoteright} Je moet weten dat '.}{Cateau voor geen geldn spin}{zou ged\'o\'od hebben}\\

\haiku{Den ganschen dag zie,.}{je er dames en heeren}{venters en koetsen}\\

\haiku{Ge weet niet w\'at eten,, -:}{w\'at praten w\'at kijken is}{en een mensch zien eten}\\

\haiku{Er zijn nog niet veel.}{bessen rijp en de warmte}{is zoo vermoeiend}\\

\haiku{En uitgelaten -,!}{als jongens draafden ze ja}{zoowaar tante draafde}\\

\haiku{We vonden niets dan,,.}{onrijp goed bessen die groen}{of verrot waren}\\

\haiku{Ik aarzelde in.}{de keus van het beroep dat}{ik zou aannemen}\\

\haiku{En van huis uit ben,.}{ik niet nerveus behalve}{in het voorjaar}\\

\haiku{Geeft u voorkeur aan?}{den eenen criticus boven}{den anderen}\\

\haiku{Johny - als jij, '?}{God als n\`eger z\`ag zou jij}{dan inm gelooven}\\

\subsection{Uit: Schetsen. Deel 6 (onder ps. Samuel Falkland)}

\haiku{de zich Marian,.}{verveeld-vies en kribbig}{den neus optrekkend}\\

\haiku{Moeder, goedig, t\`och -}{met angstig gebaar ruzie}{was zoo \`ell\`endig}\\

\haiku{Z'n broek, afgetrapt,,.}{slobbert om de groote plompe}{logge schoenen}\\

\haiku{{\textquoteright} {\textquoteleft}Zoo heet me broer ook,.}{die verleeje jaar an z'n hart}{gesturreve is}\\

\haiku{As-ie snurke wil.... ' '!...}{s\'al-ie snurket Lijkt wel}{n kertiermeester}\\

\haiku{v\'oor je 'n storm op, ', '.}{zee \`achter jen veilig}{hoteln warm bed}\\

\haiku{Hij zag er waarlijk,.}{ongunstig uit tot zelfs in}{z'n smerige kleeren}\\

\haiku{{\textquoteleft}z\'o\'o 'n w\`onder zou '!...}{t niet zijn voor menschen die}{dri\`e jaar getrouwd zijn}\\

\haiku{De phosphor-streep.}{vlamde zonderling in het}{donker der kamer}\\

\haiku{Hoe komt de duvel{\textquoteright} -.}{in z\'o\'o'n \`onschuldig kind zei}{die van beneden}\\

\haiku{{\textquoteleft}Wat zal Kenau 'r, '?}{wel van zeggen datt zoo}{laat geworden is}\\

\haiku{hij mevrouw den stok,, '.}{drukte op de veer bereid}{n moord te begaan}\\

\haiku{{\textquoteleft}.... Ik weet 'n ladder - ' -{\textquoteright}....}{as u dan bovenn raam}{openschuift ben u klaar}\\

\haiku{t Huisje had ze '.}{geruimd en moedern kop}{thee op bed gebracht}\\

\haiku{Die had 't ineens,.}{h\'ard te pakken en dat zoo}{kort voor de bruiloft}\\

\haiku{Dan met een vreemden, '.}{glimlach nam zen glas en}{\'e\'en der eieren}\\

\haiku{ze stilletjes om,.}{moeder die \`elk woordje van}{de krant las herlas}\\

\haiku{Niets leek veranderd.}{in de monotonie der}{kamertjesdingen}\\

\haiku{'t Was voor vier jaar '!}{n ruzie geweest tusschen}{Ant en de Snoepster}\\

\haiku{En 't dienstmeisje - '. '}{liep mee ast niet te ver}{uit de buurt voerde}\\

\haiku{As-die 't leeren, '.}{wou most-iet maar bij}{andren probeeren}\\

\haiku{{\textquoteleft}Nee,{\textquoteright} zei-die, z'n '.}{neus buigend tot onder den}{rand vant loket}\\

\haiku{Driftig lei ze 'r,.}{vinger op den mond toen-ie}{zoover de trap op was}\\

\haiku{De meneer wreef z'n, ' '.}{gezichtje waaidem lucht}{toe metn handdoek}\\

\haiku{As ze getrouwd was, -}{as moe at ze \`alles van}{chocola nou maar}\\

\haiku{'n Man kan boter '.}{zoo niet \`afkeuren of de}{vrouw stooftr stiekum mee}\\

\haiku{Net, toen de huisknecht, ':}{weerom keerde zag-iet}{l\'a\'atste gebeuren}\\

\haiku{{\textquoteright} {\textquoteleft}Juist,{\textquoteright} zei ik - en 'n.}{half uur later sprak ik den}{agent van de stoomtram}\\

\haiku{M'n vriend rukte het, '....}{roer om deedn paar extra}{slagen met z'n spaan}\\

\haiku{W\`at heb ik misdaan?}{om zoo in mijn eerst geslacht}{gestraft te worden}\\

\haiku{Bevend stak ze de, ',.}{lamp an liett gordijntje}{neer kleedde zich uit}\\

\haiku{De meester het 'r.}{an d'r tong getrokke en}{ze riep nie-eens au}\\

\haiku{Toen wou Jan w\`eten ', '.}{hoe diept gat was maar Chris}{wout niet hebbe}\\

\haiku{Santje grunnekte,.}{van pret doch met mate en}{onder de dekens}\\

\haiku{Dichtbij hoorde ze '.}{Anna's gegiegel die door}{n dekentuitje blies}\\

\haiku{As je moeder dood, '...}{was stopten ze je inn}{weeshuis bij vreemden}\\

\subsection{Uit: Schetsen. Deel 7 (onder ps. Samuel Falkland)}

\haiku{{\textquoteleft}... Hou je monden eens,{\textquoteright}.}{zei-ie en we luisterden}{allen glimlachend}\\

\haiku{Tegen negen, at '.}{man paar beschuitjes met}{marmelade}\\

\haiku{Nap dwaalde gebluft -, -.}{omlaag doorzocht de kamers}{de tuinkamer niets}\\

\haiku{Alles flapten ze '.}{in de kranten of int}{politierapport}\\

\haiku{Als 't 'n ander '?}{was zoue wijt toch niet uit}{de krant v\'oorleze}\\

\haiku{{\textquoteleft}je zal 'n fiets of '{\textquoteright}....}{n locomotief voor je}{neus gehad hebben}\\

\haiku{Voor 'n brievenbus - - - - {\textquotedblleft}!}{hield-ie stil toen toen z\`onder}{zweep z\`ondervort p\`erd}\\

\haiku{Of 'n evenpaard 'n - '?}{individu draagt of sleept}{w\`at ist verschil}\\

\haiku{Behoef ik nu nog,.}{te twijfelen dacht ik in}{vreugde en deemoed}\\

\haiku{De pen kroop over het '.}{blaadje enn krassende}{sluitstreep sloot den brief}\\

\haiku{Zes paar om d'r g\'o\'ed,{\textquoteright}.}{in te kommen van veertig}{cente zei moeder}\\

\haiku{Ze staarden naar 't, '.}{vierkant hok waarin het hooi}{lei ent lichaam}\\

\haiku{De vreemdeling zweeg,.}{een wijle bekeek me met}{loftuitende oogen}\\

\haiku{{\textquoteright} {\textquoteleft}Spausswasser ist Spausswasser,{\textquoteright},.}{snauwde ik uit m'n humeur}{wijd de deur openend}\\

\haiku{As-die niet in ',, '.}{t sterfhuis was zou-die}{aant graf kommen}\\

\haiku{'r Begon al heel '.}{wat stoppelgestuif opt}{laken te vallen}\\

\haiku{'k Draaide m'n stoel,.}{verzette den spiegel naar}{de  linkerzij}\\

\haiku{'t Zag 'r \`anders ' '.}{uit dankt wel bij den}{kapper gezien had}\\

\haiku{Pluimen en pluisjes ( ').}{dauwdenomt po\"etisch}{te zeggen omlaag}\\

\haiku{{\textquoteleft}Effen drinke,{\textquoteright} zei, '.}{ze de prop verslikkend die}{voorr keelgat wrong}\\

\haiku{Ja, daar schiet 'k mee, ',{\textquoteright} ':}{op w\`a\`art van komt snauwde}{de stem int gras}\\

\haiku{Mee-dreunend met 't,.}{gerommel in de verte}{kreunden de wielen}\\

\haiku{{\textquoteleft}de menschen van 't ' -{\textquoteright}.... {\textquoteleft}}{g\'asthuis zeggent en dan}{zou de politie}\\

\haiku{As de kindere ', ', '.}{t zeien dan w\`ast dan}{m\`ostt zoo wezen}\\

\haiku{D'r hoofd liep om van - '. '.}{de drukte en de schepen}{int zichtt Hielp}\\

\haiku{{\textquoteright}, schreeuwde Gerrit z'n:}{paarsbol gezichtje over de}{verschansing buigend}\\

\haiku{Gerrit en Adam en '.}{Gijs zwaaiden er mee datt}{spetterend knapte}\\

\haiku{En de kindren, bang '.}{voort donker liepen mee}{met de menschen}\\

\haiku{Angstig, alsof ze,:}{ze wekken wou riep ze met}{nadruk-accentjes}\\

\haiku{de pieters waren -.}{dood de pieters waren van}{h\`onger gestorven}\\

\haiku{De heele winkel,.}{was \'e\'en gezang \'e\'en zoet weeldrig}{getsilp en getril}\\

\haiku{En in 't voorjaar.}{zette meneer ze w\'e\'er bij}{mekaar in de broeikooi}\\

\haiku{{\textquoteright} {\textquoteleft}En je waterproof,{\textquoteright}.}{is opzij gescheurd lette}{m'n uitgever op}\\

\haiku{Neem 'n stoel en 'n ' '.}{stoof en gar bij zitten}{int zonnetje}\\

\haiku{{\textquoteright} {\textquoteleft}Zie je dan niet dat '?}{t de witte is met de}{veertjes an z'n poote}\\

\haiku{- N\`egen j\`onkies - 't ' - '{\textquoteright}....}{isn gezin ze hetr}{wat mee te stellen}\\

\haiku{{\textquoteleft}Suiker \`en boter -{\textquoteright},:}{en dan nog tw\'a\'alf eieren}{rekende besje}\\

\haiku{{\textquoteleft}me zakken zijn vol, - '{\textquoteright}....}{h\'e\'elemaal vol d'r was amper}{n plaasie voor me bril}\\

\haiku{Anders ben 'k zoo,{\textquoteright} ',:}{gezond klaagdet vrouwtje}{kurkig hijgend}\\

\subsection{Uit: Schetsen. Deel 8 (onder ps. Samuel Falkland)}

\haiku{{\textquoteright} {\textquoteleft}Dat ken 'n kl\`ein,{\textquoteright} {\textquoteleft}'!}{kind ruiken hield ma vol.n}{Klein kind in j\'o\'uw land}\\

\haiku{Toen ontspande ma,, ':}{minzamer blazend opn}{d\`erden stoel en zei}\\

\haiku{{\textquoteleft}Gebeurd,{\textquoteright} zei ma, 'r:}{voeten warmend op den rand}{van den kolenbak}\\

\haiku{zoo dikwijls van de,,}{hand in de tand zat je nou}{eens hier nou eens daar}\\

\haiku{Je zeg maar Owie en -{\textquoteright}....}{O-non en Merci de}{rest leer je vanzelf}\\

\haiku{{\textquoteleft}c'est tr\`es bon mais apr\`es...}{douze heures et moi avec pour}{le parle-ment}\\

\haiku{- {\textquoteleft}Zoo,{\textquoteright} kreunde ik m'n ':}{pijp bekauwend ent laatst}{slokje thee slurpend}\\

\haiku{Hier heb je geld en ' - '{\textquoteright}...}{loop opn draf dan krijg je}{n extra van me}\\

\haiku{{\textquoteright} 'n Nieuwe inval. '.}{n Schrijver z\`onder vrouw is}{geen sikkepit waard}\\

\haiku{{\textquoteright} {\textquoteleft}Dat is de br\'o\'er, die....{\textquoteright}:}{buiten woont en van z'n vrouw}{gescheiden is Of}\\

\haiku{{\textquoteleft}Nou, de huisheer van '.}{de overkant hett wel goed}{met z'n menschen voor}\\

\haiku{De een ging zijn weg,,.}{van lodderen bulderen}{stuiven en stampen}\\

\haiku{'n Stoomboot gulpte '.}{een roetstreep enn bom glee}{op zwarte vlerkjes}\\

\haiku{Het gulpte op den,.}{cementen bak toe om den}{kop van den zeehond}\\

\haiku{{\textquoteleft}Piet kijk is ga\`uw,{\textquoteright} zei, ' '.}{zet kommetje opt}{schoteltje smakkend}\\

\haiku{{\textquoteleft}En alles wat we,{\textquoteright}.}{v\`o\`orgelogen hebben zei}{ze haast hakkelend}\\

\haiku{Na zoo'n aanval van.}{bronchitis moest tante wat}{op verhaal komen}\\

\haiku{{\textquoteleft}'n Glaasje port of ' - '....}{n glas maderak heb}{van alles in huis}\\

\haiku{Vlamt het luidruchtig,,,.}{dan spreekt men van feest van een}{geboorte een bruid}\\

\haiku{{\textquoteleft}Luiken voor vensters ',{\textquoteright}.}{zijn de oogleden vann}{huis wijsgeerde ik}\\

\haiku{Verwonderd schelde - ' '.}{ik aann w\'e\'ek lang wask}{afwezig geweest}\\

\haiku{Maar hier coupeer ik.}{onmiddellijk \`elke op}{zichzelf misplaatste grap}\\

\haiku{En de keukendeur -.}{was dicht en stil \`en de deur}{van de goeie kamer}\\

\haiku{een draaimolen is,,.}{een beul een kwaadaardige}{liederlijke beul}\\

\haiku{En 's nachts, sliep-ie,.}{bij de waarzegster in de}{tent op de keien}\\

\haiku{Het gesprek lei 'n '.}{moment zoo plat alsn niet}{bewegende bot}\\

\haiku{je zoo maar niet je}{levens-hebben en houen}{en omgekeerd voelt}\\

\haiku{{\textquoteleft}Kinderen zijn geen,{\textquoteright}:}{getuigen zei ik en voor}{alle zekerheid}\\

\haiku{En zeg nou n\`og is - -....}{in me gezich wat jij in}{de krant heb gezet}\\

\haiku{De smid haalt je over '!,{\textquoteright}.}{t hekkie as-die je}{ziet gierde een-hoog}\\

\haiku{Vooral de groote meid - '!...}{de kwaje meid die opr}{broertje zou passen}\\

\haiku{{\textquoteleft}Laten we naar de,{\textquoteright}:}{Ringkade l\'oopen zeide}{mijn metgezellin}\\

\haiku{Eindelijk was-ie,.}{over z'n heesche woede heen}{kon-ie weer pr\`aten}\\

\haiku{Opgehitst, benauwd,,.}{knoopte hij de das rond z'n}{nek los ademde zwaar}\\

\haiku{Puf - 'k zit, blaasde, ' '.}{zer pappig handje als}{n waaier zwaaiend}\\

\haiku{{\textquoteleft}Marianne Prins,{\textquoteright},....}{zei ma de plombi\`ere}{verrast ne\`erzettend}\\

\haiku{Nou dan z\`al 'k 'r,{\textquoteright},.}{in d'r gezicht zien zei pa}{z'n vest toeknoopend}\\

\haiku{{\textquoteleft}heeft u iets tegen '?....}{n verkeering van mij met uw}{dochter Sophie}\\

\haiku{{\textquoteright} vroeg moe, die van 'r - '!}{fauteuil je zakter in}{weg van lekkerheid}\\

\haiku{De Haan, je doet je -:}{zaakjes toch nog best met je}{vijftig jaar dienst of}\\

\haiku{Dagen en weken ',,.}{leik in de kooi doodziek}{te lam om te eten}\\

\subsection{Uit: Schetsen. Deel 9 (onder ps. Samuel Falkland)}

\haiku{Er was daar een kof,.}{gezonken die ze zouden}{trachten te lichten}\\

\haiku{Zoo praatte vader ' -!}{en wij ant luistere}{dat ken je denke}\\

\haiku{Nou, toen kwam d'r 'n, '.}{motn gekrakeel van de}{andere wereld}\\

\haiku{die zit 'r in en ' -.}{die blijftr in daar was geen}{dichten mogelijk}\\

\haiku{de schipper is d'r - ' '.}{gloeiend bijt roer loopt as}{n hazewindje}\\

\haiku{{\textquoteleft}Ze zalle je niet,{\textquoteright},:}{opvrete zei dan Blanes}{zoo heette-die}\\

\haiku{Zij hadde kole - '.}{en koste en slijtagie}{hij deet met wind}\\

\haiku{En voor nimmendal -}{an de winkels levere}{nou dat zat zoo lang}\\

\haiku{{\textquoteright} En die simpele, -.}{woorden maakten me bang ik}{wist zelf niet waarom}\\

\haiku{As 'k 'm n\`og is, '.}{snap met jenever draat-ie}{r d\`adelijk uit}\\

\haiku{Op 'n Woensdag kwam ' '.}{k vant schaften in de}{machinekamer}\\

\haiku{En 'n dierlijke'.}{lodderlach schaterde uit}{Blanes zwarten bek}\\

\haiku{Daar dan,{\textquoteright} zei Zadok, ' '.}{t beestn afgekloven}{mergpijp toesmijtend}\\

\haiku{De eene Dubois had -.}{een aardig gezichtje de}{ander was leelijk}\\

\haiku{{\textquoteleft}zouen ze z\'o\'oveel,,!}{onrecht z\'o\'oveel moord z\'o\'oveel}{schandalen dulden}\\

\haiku{Bij onweer most je - -}{geen glimmende dingen \'open}{laten dat trok an}\\

\haiku{{\textquoteright}... 't Bevend tuitje '}{siepte koffiedik enn zwart}{strooperig straaltje}\\

\haiku{- want schudden mag je -.}{ze niet en op d'r kop staan}{mogen ze evenmin}\\

\haiku{Zelfs de kinderen,.}{zwegen angstig kijkend van}{vader naar moeder}\\

\haiku{En trokken naar een,.}{ander dorp toen de nacht de}{straten te schuil lei}\\

\haiku{Ze kromde  d'r,,.}{rug klauwde zich vast in de}{deur rukte en beet}\\

\haiku{Nou - ik heb 'n mand -.}{lekkere l\`evende schol}{ze sprong over de rand}\\

\haiku{We zatten mekaar...{\textquoteright} {\textquoteleft}}{moppen te vertellen tot}{de majoor langs kwam}\\

\haiku{r bij, Piet - as de ',....{\textquoteright} {\textquoteleft}.}{eenr praat van maakt praat de}{ander naJawel}\\

\haiku{{\textquoteleft}Geef u mijn is 'n '{\textquoteright}.}{ons zoetemelksche enn}{half ons leverworst}\\

\haiku{Op het podium.}{gromde geschuif van stoelen}{en taboeretten}\\

\haiku{van den \'e\'ersten tot;}{den l\`a\`atsten regel van het}{Lied von der Glocke}\\

\haiku{as pa weer na 't.... ' '.}{kantoor wast Wasr niet}{van gekomme}\\

\haiku{Z'n brilleglazen, '.}{schuchterden op haar toe b\`ang}{voorn uitbarsting}\\

\haiku{Ongetwijfeld was,.}{het een gentleman een die}{con amore werkte}\\

\haiku{- Wie zal d'r zoo gek '?}{zijn omn inbreker te}{herreberrege}\\

\haiku{Hij vond n\`ergens,,.}{logies de Duitscher vertrok}{met de laatste tram}\\

\haiku{Ik wil wel wete,{\textquoteright} ' - -:}{zeit boertje vinniger}{voor de tw\'e\'ede maal}\\

\haiku{Ik weet niets van 't '.}{lot vann bejaarde kip}{in de vrije natuur}\\

\haiku{De beschaving wijst,.}{bij bejaarde hennen en}{hanen naar soep kluif}\\

\haiku{Er kwam 'n flesch met.}{limonade en een met}{alcoholvrije-wijn}\\

\haiku{Zeg 'm gerust in{\textquoteright}....}{mijn naam dat vleesch de p\`est}{voor iedereen is}\\

\haiku{Thuis krijgen we 't, '{\textquoteright}....}{nooit as vader niet isn}{goeie verdienste het}\\

\haiku{Die met de zeere -.}{oogen stierf die met den boerekop}{viel van de trappen}\\

\haiku{D\`aar - in 't tuintje -.}{had je de ren met de acht}{kippen en den haan}\\

\haiku{Ze leek te drinken ',,.}{vant vuur de vonken de}{stuivende krinkels}\\

\haiku{n Baboe kon bij ' '!}{tijjen maller doen as}{n kind vann jaar}\\

\haiku{De natuur is een.}{materie-schalk en de}{menschen zijn blagen}\\

\haiku{Met 'n eenvoudig:}{gebaar wees-ie naar het}{oude pijpenrek}\\

\haiku{Als ze geen kop thee,.}{had gehouden zou ze in}{slaap zijn verzwommen}\\

\haiku{De kop hield 'r in, '.}{balans juist op de zotte}{limiet vant Zijn}\\

\haiku{{\textquoteright} zei ze bits, met 'n '.}{m\`a wakkerschrikkend j\`a enn}{brutaal-echo\"end p\`a}\\

\haiku{Jeanne haalde,.}{verveeld de schouders op dronk}{nijdige nipjes}\\

\haiku{\`op waren haakte.}{ze voor gelegenheden}{van liefdadigheid}\\

\haiku{'n Kleine twist smijt. '}{het huis-\'equilibre}{ondersteboven}\\

\haiku{Ze k\'e\'ek alleen st\`ar.}{naar pa's knie met de acht}{en veertig kolom}\\

\haiku{De kat loerde met,,.}{glazen blikkrende oogen klaar}{om z'n sprong te doen}\\

\subsection{Uit: Schetsen. Deel 10 (onder ps. Samuel Falkland)}

\haiku{Glimlachend liep de,,.}{juffrouw geen aanmerkingen}{makend niets zeggend}\\

\haiku{{\textquoteleft}Dag Corrie, snoesje,{\textquoteright},:}{zei Lies de schooltasch op den}{lessenaar leggend}\\

\haiku{daar was ze z\`oo bang -.}{voor  geweest en jawel}{niks as gekibbel}\\

\haiku{Het was een keurig,,.}{uitgezocht menu eenigszins}{ondeugend van toon}\\

\haiku{als ge Geldgebrek,,.}{hebt behoefte aan Brood schrijft}{dan Falklandjes}\\

\haiku{Stoffen voor 'n pak, '.}{kiezen als je niet uitt}{werk weet te komen}\\

\haiku{Schuif de boeken maar ',,.}{n beetje opzij Stom als}{ze je hinderen}\\

\haiku{188! 188! - Ja, als u, ' '.}{niet luistert ist niet noodig}{datk voorlees}\\

\haiku{Dan liep minstens 'n '!}{kwart van de gemeente met}{n gat in d'r buik}\\

\haiku{As ik hoofd van de, ' '.}{politie was steldek}{n vervolging in}\\

\haiku{Jij heb gisteravond! -}{mijn dienstmeid bekeurt op me}{\`eigen stoep bekeurd}\\

\haiku{Je blauwe potlood,, (,).}{commissarisluider daar}{Staal hem niet verstaat}\\

\haiku{In dat smaadstuk, in, '....}{dat sch\`andestuk heeftn zoon}{z'n \`eigen moeder}\\

\haiku{'k Zou haast zeggen, '.}{dat \`alle vrouwspersonen}{t mosten zien}\\

\haiku{de Duvel, voornoemd,.}{strooit z'n infaamheden in}{den reinsten akker}\\

\haiku{'t Ongelikte,.}{werd geaccentueerder}{dezelfde week nog}\\

\haiku{Kwam de slager aan '.}{het tuinhekje van Ib dan}{schoot-ier vandoor}\\

\haiku{{\textquoteright}, zei mevrouw, wakker.}{wordend in de linkerhelft}{van het lit jumeau}\\

\haiku{Dat geeft 'n dooie in,{\textquoteright}.}{de femilie zei mevrouw}{in de linkerhelft}\\

\haiku{Bij de visch - turbot -.}{sauce capres waren de}{aardappelen stijf}\\

\haiku{Daarom heb 'k m'n ' {\textquoteleft} '{\textquoteright}.... {\textquoteleft}}{souvenir ingezet met}{tToent kindje}\\

\haiku{We vonden 't best. {\textquoteleft} ' ',{\textquoteright}.}{Enn goospenning vann}{volle week zei ze}\\

\haiku{Zuchtend her-nam,.}{Barend de hand van z'n vrouw}{25 April 1903}\\

\haiku{En omdat moeder ' -}{meende datt gevaarlijk}{was te weigeren}\\

\haiku{Mie met de kaars die ',.}{straaltjes vet opt kleed spoot}{stonden voor m'n bed}\\

\haiku{Bij de {\textquoteleft}stedenten{\textquoteright} ' ' {\textquoteleft}}{wastn zoete inval}{en destedenten}\\

\haiku{Meneer - we hebben...{\textquoteright} {\textquoteleft}?}{centen bij u op de stoep}{gevondenW\`eer}\\

\haiku{Vrouwen verzetten - - '.}{zich dikwerf helaas tegen}{t best intellect}\\

\haiku{Je zweette van de,.}{morgen tot de avond maar je}{had wil van je werk}\\

\haiku{Een onderwijzer, '.}{of zoo dachtk. Een die de}{anderen waarneemt}\\

\haiku{Waarlijk, hij zei langs, '.}{z'n neus weg dingen diek}{absoluut niet wist}\\

\haiku{Daar staat 'n meester.}{bij met de tande in zijn}{mond as jij en ik}\\

\haiku{Jij ken 'r toch niet - ' - ' '.}{en {\`\i}k kenr nietk ken}{r zoo min als jij}\\

\haiku{Maar ze scharrelde, '.}{bij de kastn vergeten}{ding opbergend}\\

\haiku{Vanmorgen, nog geen,.}{drie uur gelejen was-ie}{\`even vochtig geweest}\\

\haiku{{\textquoteleft}Falkland - die man -{\textquoteright}.}{is je offer je heb dien}{man verjongejand}\\

\haiku{{\textquoteright} Kwaadaardig schoven ' '.}{r vingersn teen door de}{opstaande spanen}\\

\subsection{Uit: Schetsen. Deel 11 (onder ps. Samuel Falkland)}

\haiku{Nog grooter as 'n -?}{cocosnoot hei-jij wel}{is schijfies gekocht}\\

\haiku{{\textquoteright} {\textquoteleft}'k Zal me hand voor,, '.}{je ooge legge stommert dan}{zie jem niemeer}\\

\haiku{- Sientje met Bet en -...}{Zus an d'r hande Cor en}{Ansie d'r achter}\\

\haiku{Jessis as de zwaan ',!}{r op los vloog raakte ze}{onder de voete}\\

\haiku{Dat zag-ie nou ook -.}{z'n heele broek van binnen}{was geel van kaarsies}\\

\haiku{'t m\`ost - wou je 't.}{meissie niet in opspraak en}{ongeluk brengen}\\

\haiku{{\textquoteright} {\textquoteleft}Hoe wou jij negen!}{menschen en kindren van \'e\'en}{kip d'r buik vullen}\\

\haiku{Geef mijn 't maagie met '!}{wat sju enk doe me maal}{met aardappele}\\

\haiku{{\textquoteleft}en 'k wensch je veel,!}{genoegen maar ik zal niet}{van de partij zijn}\\

\haiku{De vrouwen konden '.}{z\`ulkn tekstverklaring niet}{laten passeeren}\\

\haiku{We\`ent gij, heerlijke:}{en machtige worstelaar}{voor kleine luyden}\\

\haiku{{\textquoteleft}'k Zel nog is luie,{\textquoteright},.}{zei de jongen superbe}{van demp-toon}\\

\haiku{Men zag er uw rok,, -.}{uw vest uw das ook uw kop}{door kaarsen bevlamd}\\

\haiku{'t Gebeurt meer, 'k.}{behoef er geen finesses}{van te vertellen}\\

\haiku{Ten slotte begrijp,.}{ik ook niet waarom ge die}{rooiekool zoo uitmaakt}\\

\haiku{Om me is lekker, '.}{dwars te zitten begon die}{metn advocaat}\\

\haiku{De uitmuntende,,:}{ziele-ontleedster freule}{Lohman vraagt terecht}\\

\haiku{{\textquoteleft}M'n honden geven,....}{mij hun liefde omdat ik}{hun geef de mijne}\\

\haiku{Als 'k me hier in -, ',....}{de schuld v\`erdrink ist voor}{jouw oogen jouw haar}\\

\haiku{(zij zwijgt nijdig, kijkt,)!}{vlug naar den waard giet haar glas}{over den grond Toe maar}\\

\haiku{Jij ben vanmorgen,.}{met je verkeerde been uit}{bed gestapt Marthaatje}\\

\haiku{(tast naar z'n das, zoekt, ',) ....}{snel de tafel af kijktr}{onder vraagt driftig}\\

\haiku{{\textquoteleft}Blijf u maar rustig ',,{\textquoteright}:}{int w\`arme graf mevrouw}{praatte-ie terwijl}\\

\haiku{{\textquoteright} {\textquoteleft}Betrekkelijk,{\textquoteright} zei:}{hij iets loslatend van de}{stemming van daar straks}\\

\haiku{{\textquoteright}, kommandeerde de,.}{hijgende man pogend zich}{te ontworstelen}\\

\haiku{'t Lukte zonder,,.}{aarzeling zonder gaping}{zonder incident}\\

\haiku{Koosje slobberde ', '.}{vanr kom bette met vrije}{handt achterhoofd}\\

\haiku{Vroeger dee tante,.}{Marretje dat vroeger zat}{ze mee aan tafel}\\

\haiku{De pijp in zijn hand,.}{rustte op z'n knie z'n hoofd}{boog wat naar voren}\\

\haiku{Hij was zoo helder,,. '}{van geest zoo bij de pinken}{meende oom Bernard}\\

\haiku{Mama leefde van '.}{wat Demoiselle enr}{zusters inbrachten}\\

\haiku{Nou zeg 'k niks meer,{\textquoteright}, ':}{redeneerde hij grimmig}{opn stoel ploffend}\\

\haiku{{\textquoteleft}As je \'e\'en kogel, ' '!}{vindt betaalk voor elke}{kogeln bankie}\\

\haiku{{\textquoteleft}ben je tevrejen? ' ' '!}{k Hadt motten weten}{v\'o\'ort huwelijk}\\

\haiku{d'r is niet \'e\'en meid,,!}{letterlijk niet \'e\'en of jij}{jaagt ze de deur uit}\\

\haiku{roemers, de borden.}{en schalen violette}{schulpen geworden}\\

\subsection{Uit: Vuurvlindertje}

\haiku{die had slaap, maar hield - ' ' -:}{zich koestt was weert naarste}{oogen-blikkie}\\

\haiku{{\textquoteleft}maar ik vin 't zoo '....}{fijnn andenken an me}{vader te hebben}\\

\haiku{De jaren hielpen ',,.}{t groote knagende verdriet}{langzaam vergeten}\\

\haiku{{\textquoteright} Met moeite hield ze,,.}{zich in om niet harder niet}{grover te schimpen}\\

\haiku{{\textquoteright}, klaagde moeder, die ':}{t nou zeker bedorven}{eten af had gezet}\\

\haiku{Ongezeggelijk.... '....}{wurmt Is elleke dag}{wat anders met je}\\

\haiku{niet mocht {\textquoteleft}uitvieren{\textquoteright},,.}{gaan uitrusten en in \'e\'en}{gierigheid huilen}\\

\haiku{{\textquoteright}, vroeg ze, 'r zoo naast, '.}{of-ie Engelsche woorden}{uitt Leerboek zee}\\

\haiku{{\textquoteleft}je zag enkel licht....{\textquoteright} {\textquoteleft},?}{van bovenHoe at je dan}{as je honger kreeg}\\

\haiku{{\textquoteright} gromde de vrouw, en '.}{r vingers tintelden van}{zenuw-opwinding}\\

\haiku{{\textquoteleft}Dat mot je heusch{\textquoteright}, ', {\textquoteleft}'}{niet meer doen praatte ze met}{tranen inr oogen}\\

\haiku{In de huiskamer '.}{leunde de grootmoeder in}{r stoel achterover}\\

\haiku{Koert probeerde 'r, '.}{in slaap te praten maart}{licht most opblijven}\\

\haiku{'r Was 'r niet een, '.}{bij geweest die van z'n hart}{n moordkuil maakte}\\

\haiku{{\textquoteright} De stoel niet tegen,.}{z'n drift bestand viel geknauwd}{ondersteboven}\\

\haiku{'t is links, toen weer ', - ' -....}{t is rechts en toen neemt}{me asje asje}\\

\haiku{En as je strakkies,....}{benejen komt zel je d'r}{van op staan kijken}\\

\haiku{zoo'n hoop verdomde, '....}{dingen dwars da'k me voel of}{k gek zal worden}\\

\haiku{'k Zal probeeren 't ' '....}{r uit te trappen al zal}{t niet glad zitten}\\

\haiku{Koert gaf geen antwoord, '.}{liett bed nog meer en nog}{knagender kreunen}\\

\subsection{Uit: Een wereldstad. Berlijnsche impressies en schetsen}

\haiku{{\textquoteright} {\textquoteleft}'k Zal Berlijn leeren{\textquoteright}, ',}{kennen namk me na een}{paar maanden t\`asten voor}\\

\haiku{Hier is 't,{\textquoteright} zei de,.}{wagenvoerder die beloofd}{had te waarschuwen}\\

\haiku{z'n oogenwit '.}{was bloedbeloopen als bij}{n opgejaagd dier}\\

\haiku{Drie-, viermaal, met,:}{maling an z'n slapende}{buren steunde-ie}\\

\haiku{De vagebond, die, '.}{zooeven z'n nood had geklaagd}{wast nijdigste}\\

\haiku{Dit nu is 'n h\'e\'el,{\textquoteright}:}{typisch hoekje van Berlijn}{zeide onze vriend}\\

\haiku{door schemerduister,.}{stapten we de deuren der}{bewaarplaatsen langs}\\

\haiku{Den Vater verliert,{\textquoteright}, ':}{man zei-ie als man die weet}{vant leven had}\\

\haiku{En 'k leef altijd,...}{in angst dat ze me van de}{straat zal wegnemen}\\

\haiku{We lagen opnieuw {\textquoteleft}{\textquoteright}.}{tegen de glooi{\"\i}ng van de}{B\"ohmischen W\"alder aan}\\

\haiku{Het eenige wat me.}{momenteel intresseert is}{Rusland in Berlijn}\\

\haiku{Morgen,{\textquoteright} zeide ik,:}{tot de maagd die er bleek bij}{geworden was}\\

\haiku{Je kunt nooit weten...{\textquoteright}}{welke curieuze vondst}{we gedaan hebben}\\

\haiku{Reuze-banket,,.}{Gister 10 December was}{het \`ongemeen druk}\\

\haiku{Voor 'n jour heb je.}{niet zoo schrikkelijk veel noodig}{en ook niet veel plaats}\\

\haiku{De punchgeur doortrok,.}{de kamer dee de mannen}{uitbundig praten}\\

\haiku{{\textquoteleft}heeft 'n sterveling...}{je wat gedaan da-je de}{pret gaat verstoren}\\

\haiku{Nou zeg 'k je \'e\'en{\textquoteright},, ':}{ding barstte mevrouw nan}{nieuwen ademhap los}\\

\haiku{{\textquoteleft}als je ons op die, '{\textquoteright}...}{manier blijft jagen benk}{nog in geen \'u\'ur klaar}\\

\haiku{Hoe de oogen van z'n...}{bruid bij die stomme woorden}{geglinsterd hadden}\\

\subsection{Uit: De wijze kater}

\haiku{Ik sta hier al een}{poos af te luisteren op}{wat voor manier jij}\\

\haiku{In mijn jeugd, toen ik,:}{nog graag op de schoot van mijn}{moeder sprong dacht ik}\\

\haiku{Die waren voor een!}{maand met bordpapier in plaats}{van met leer gezoold}\\

\haiku{Wat zou u in zo'n? '}{geval van schandelijke}{onverzorgdheid doen}\\

\haiku{(Wendt zich tot kater,.}{die zich met Jonathan heeft}{bezig gehouden}\\

\haiku{Helaas, Hoogheid, ik...}{moet mij aan de etiquette}{onderwerpen}\\

\haiku{Hoogheid, ik smacht naar!}{een t\^ete \`a t\^ete118 van}{meer tedere aard}\\

\haiku{(Af, uitgelaten,.}{door eerste lakei die hij}{genoeglijk toeknikt}\\

\haiku{als ik neerkniel in,;}{de kapel hoor ik hoe het}{altaar beknaagd wordt}\\

\haiku{De sallemanders,!}{en krengen hebben de koorts}{op het lijf Sire}\\

\haiku{Ik voel me ook als,}{een mens in een vreemd pakhuis}{maar de melk is best.}\\

\haiku{Ik had al het zeer,...:}{bijzondere genoegen}{Sire  Koning}\\

\haiku{Hoeveel ratten heeft?}{U hier sinds eergisteravond}{zelf niet gedood}\\

\haiku{Er zijn meer ratten,,.}{eerwaarde heer dan U en}{ik vermoeden254}\\

\haiku{Maar mijn poten heb.}{ik met groene zeep onder}{de pomp gewassen}\\

\haiku{Ik wou je alleen '...}{zeggen datt bijzonder}{verfrissend is}\\

\haiku{Ik zou hem op 't!...}{moment nog graag de ogen uit}{zijn kop halen}\\

\haiku{Juist, juist, Angorensis...,!...}{Geen tragische souvenirs}{arme kerel}\\

\haiku{Duizend tegen een,!}{dat hij zijn handtekening}{niet kan schrijven}\\

\haiku{Die was je wezen,...}{kopen bij een boer die ze}{al begraven had}\\

\haiku{Verdwijn uit onze!}{ogen en laat ons nooit meer iets}{van je horen}\\

\haiku{Als jij dat niet weet, '!}{heb je in rechten niett}{recht brood te bakken}\\

\haiku{96volgens het boek.}{Genesis schiep God op de}{zesde dag de mens}\\

\section{Albert Helman}

\subsection{Uit: Aansluiting gemist}

\haiku{Hij herinnerde.}{zich niet meer hoe lang hij daar}{was blijven zitten}\\

\haiku{Hij had zijn schouders.}{opgehaald en was hinkend}{de hut uit gegaan}\\

\haiku{Onwillekeurig.}{begint hij zo'n beetje te}{zoemen in zijn baard}\\

\haiku{voor het Consulaat.}{stond hij gelijk met een Turk}{of een Hottentot}\\

\haiku{De mannen echter,.}{praten gewoon terwijl ze}{naderbij komen}\\

\haiku{{\textquoteleft}Ik zou het prettig...}{vinden als we niet alle}{contact verloren}\\

\haiku{als hij maar een doel,,.}{kende of een halve gek}{was gelijk Baruch}\\

\haiku{Het is een dorp op,.}{zichzelf hier en ik ben bang}{voor indiscretie}\\

\haiku{de belediging.}{in zijn laatste gezegde}{steekt Baruch niet eens}\\

\haiku{Ik veronderstel...{\textquoteright} {\textquoteleft},{\textquoteright}.}{Laat ons er straks over spreken}{antwoordt Hornemann}\\

\haiku{{\textquoteright} {\textquoteleft}Och...{\textquoteright} Hornemann weet.}{eerlijk gezegd niet wat hij}{ervan denken moet}\\

\haiku{Sloeg zich er prachtig.}{doorheen met heilmassage}{en meer dergelijks}\\

\haiku{Je hebt het hele,.}{leven nog v\'o\'or je met}{allerlei kansen}\\

\haiku{Ongaarne... maar mijn.}{bangelijke zwerflust heeft}{me weer te pakken}\\

\haiku{Nu laat hij heel de.}{stroom van zijn onstuimigheid}{over haar heen storten}\\

\haiku{Ze zou  het niet.}{kunnen verdragen dat hij}{dit anders voelde}\\

\haiku{Als arts heeft hij een,.}{nederlaag geleden dat}{blijft hij zo voelen}\\

\haiku{Wie verstaat iets van?}{die dwaas-verwikkelde}{spaanse politiek}\\

\haiku{Hij wilde een rol,.}{spelen maar men vertelde}{hem niets van het spel}\\

\haiku{{\textquoteleft}Feitelijk is het.}{een driepersoons-huwelijk}{dat wij hier leiden}\\

\haiku{Schlauch echter wil.}{de dingen niet tragischer}{nemen dan ze zijn}\\

\haiku{nu zal het Schlauch.}{niet moeilijk vallen over hem}{te triomferen}\\

\haiku{Het is het enige.}{wat de ingehoudenheid}{der mensen verraadt}\\

\haiku{Daarom m\'oet hij met,.}{Hertha spreken voordat hij met}{de agenten meegaat}\\

\haiku{Daarom moedigt zij.}{Hornemann juist aan om zijn}{plan door te zetten}\\

\haiku{Hij heeft het zijne,.}{gezegd en zijn eten behoeft}{niet koud te worden}\\

\haiku{Des te groter zal,;}{haar wraak zijn op de dag dat}{afgerekend wordt}\\

\haiku{Hij kan haar alles,,,,.}{geven wat ze wil luxe zorg}{tederheid reizen}\\

\haiku{Er m\'oet een dokter,.}{komen zo kan hij niet de}{nacht in gaan met haar}\\

\haiku{En ergens verweg,.}{dreunen doffe slagen die}{misschien schoten zijn}\\

\haiku{Ze moet niet merken.}{dat hij al te veel prijs stelt}{op haar wederkeer}\\

\haiku{Schlauch wil gaarne.}{alles beloven wat de}{ander maar verlangt}\\

\haiku{Zijn woede moet eerst,.}{tot bedaren komen dan}{kan hij overleggen}\\

\haiku{Waar je ook komt, als.}{intellectuele jood}{word je slachtoffer}\\

\haiku{Van alle dingen {\textquoteleft}{\textquoteright};}{is niets betrekkelijker}{dan wat menrecht noemt}\\

\haiku{Ghetto-jood heeft hij.}{Mendel vaak verachtelijk}{bij zichzelf genoemd}\\

\haiku{{\textquoteleft}Dat soort natuurlijk,.}{dat in Duitsland nog altijd}{met de nazi's heult}\\

\haiku{maar de waarheid is,;}{dat hij zich best kan laten}{zien als zestiger}\\

\haiku{Eerst thans, in onze.}{nieuwe diaspora is het}{opnieuw ontsluierd}\\

\haiku{{\textquoteleft}En is dat ook in,?}{het kort uit te leggen hoe}{dit in elkaar zit}\\

\haiku{{\textquoteright} Lotte Bernstein kan.}{zich niet weerhouden een zucht}{van schrik te slaken}\\

\haiku{Wat ik er tot nu,.}{toe van ken bevalt me best.}{Ook van de mensen}\\

\haiku{Ik kan er helaas,.}{niet op wachten en ik heb}{er ook geen zin in}\\

\haiku{{\textquoteright} Sabine schudt van,:}{neen maar eerlijkheidshalve}{voegt ze eraan toe}\\

\haiku{Sabine gelooft,.}{echter dat ze daar op den}{duur wel overheen komt}\\

\haiku{Als Felipe haar,:}{bij het weggaan begeleidt}{vraagt hij hartelijk}\\

\haiku{De inboorlingen;}{proberen ieder geldstuk}{op zijn zilverklank}\\

\haiku{{\textquoteright} Ook Lotte Bernstein.}{spreekt op dit ogenblik zoals}{haar eigen moeder}\\

\haiku{Het is waar, ze heeft.}{dikwijls aan een verhouding}{met Tibor gedacht}\\

\haiku{maar die uitte hij.}{niet bij onverschillige}{gelegenheden}\\

\haiku{En toch zal ze het,,.}{doen om hem te bewijzen}{wat h{\'\i}j h\'a\'ar waard is}\\

\haiku{Sabine opent het.}{portier om hem naast haar te}{laten instappen}\\

\haiku{De contouren van.}{het fort op de top snijden}{hoekig in de lucht}\\

\haiku{Natuurlijk, alle.}{emigranten hebben hier de}{aansluiting gemist}\\

\haiku{een vrouw  kan hier,.}{niet komen in haar eentje}{zonder adjudant}\\

\haiku{{\textquoteleft}Ik geloof dat je,.}{niet erg gesteld bent op mijn}{gezelschap vanavond}\\

\haiku{En misschien ga je.}{me toch meer waarderen als}{je me beter kent}\\

\haiku{Ze haat zichzelf... en,.}{toch ze m\'oet en ze w{\`\i}l een}{spaans paspoort hebben}\\

\haiku{Dit is uiterste,...}{noodzaak want het gaat immers}{om leven en dood}\\

\haiku{Maar dat kan hij niet,.}{meer begrijpen de arme}{verdwaasde Tibor}\\

\haiku{Het overpeinzen komt,,.}{later Sabienchen als de}{strijd gestreden is}\\

\haiku{Onder hem ziet hij.}{nog slechts in het water een}{blauwwitte schuimkring}\\

\haiku{Een paar slaan een kruis, {\textquoteleft}{\textquoteright}.}{de overigen putten zich}{uit inme cago's}\\

\subsection{Uit: Afdaling in de vulkaan}

\haiku{Albert Helman}{Afdaling in de vulkaan}{Colofon}\\

\haiku{Het is al zo lang.}{geleden en je was nog}{geen vijf toen hij stierf}\\

\haiku{{\textquoteleft}Soms geloof ik, dat,{\textquoteright}.}{je mij moet haten sprak het}{meisje voor zich uit}\\

\haiku{Marjorie kwam naar,:}{me toe pakte mij bij de}{schouder en riep uit}\\

\haiku{Een hele poos moest,.}{er verlopen zijn toen ik}{mij hoorde roepen}\\

\haiku{Wanneer jij dan ook, -,.}{wat neemt de kralenketting}{die staat vast bij zwart}\\

\haiku{Haar blik viel op het.}{offermes dat nog boven}{in de koffer lag}\\

\haiku{De dame hield haar,;}{ogen meestentijds gesloten}{leek ietwat vermoeid}\\

\haiku{{\textquoteright} De betrekkelijk,.}{saaie stem die dit verteld had}{zweeg voor lange tijd}\\

\haiku{hoe kom ik nog ooit,;}{bijtijds terug aan het eind}{waar mijn coup\'e is}\\

\haiku{Daarom heb ik maar.}{gewacht tot wij elkander}{beter kunnen zien}\\

\haiku{En hij beriep zich.}{op al de dikke boeken}{die hij bij zich had}\\

\haiku{{\textquoteright} Een goed uur later,.}{pas wist ik wat er precies}{van mij verwacht werd}\\

\haiku{Er scheen hem iets te,:}{binnen te schieten want nu}{zei hij joviaal}\\

\haiku{{\textquoteright} zei Art\'egui op.}{een toon alsof hij mij de}{grootste dienst bewees}\\

\haiku{- was ik niet verdacht,.}{geweest en ik moest mij een}{ogenblik bezinnen}\\

\haiku{Ach, wat een trieste,.}{dag na de miserie van}{de nacht tevoren}\\

\haiku{Ik heb die oude.}{Amaral altijd voorspeld dat}{het mislukken zou}\\

\haiku{je krijgt van mij een,.}{eigen hond voor jou alleen}{wanneer je me trouwt}\\

\haiku{Ik zei - misschien ben -:}{ik onduidelijk geweest}{maar ik bedoelde}\\

\haiku{{\textquoteright} vroeg ik onderweg,.}{nog altijd gemelijk en}{terneergeslagen}\\

\haiku{Ik durf erom te,.}{wedden dat hij nog van zich}{zal laten spreken}\\

\haiku{{\textquoteright} vroeg ik, blij dat niet.}{de eeuwige Rancho meer}{het thema vormde}\\

\haiku{Het lot dat mij tot,.}{hier gevoerd had zou mij ook}{wel verder helpen}\\

\haiku{Niet alleen dat het,.}{mij niet verraste maar het}{was mij zelfs welkom}\\

\haiku{{\textquoteleft}De Alpen en de.}{Pyrenee\"en gaven mij}{datzelfde gevoel}\\

\haiku{{\textquoteright} Plotseling brak hij,.}{af en zweeg de handen op}{zijn knie gevouwen}\\

\haiku{In elk geval, er;}{was een meisje waar ik erg}{verliefd op raakte}\\

\haiku{Hoewel er toen ook.}{veel posada's en hotels}{te vinden waren}\\

\haiku{Zodat ze al met,.}{al toch met me trouwde en}{een brave vrouw was}\\

\haiku{Maar bultenaars zijn,.}{aanstonds driftig en ik had}{geen zin in ruzie}\\

\haiku{Ze vroegen of de...}{padr\'on het zich dan niet meer}{herinneren kon}\\

\haiku{Het was dezelfde}{als die van de veel minder}{wilde stroom waarlangs}\\

\haiku{Zelfs het brouwsel van.}{Remigio bood weinig baat}{meer op die hoogte}\\

\haiku{Nu ik echter weer,}{wat uitgerust was kropen}{een voor een al de}\\

\haiku{Maar het was geen raad, -.}{alleen dan zou ik het toch}{niet hebben gedaan}\\

\haiku{En het is goed om,.}{zo'n doel te hebben groter}{dan je eigenbaat}\\

\haiku{{\textquoteleft}Ja, mi jefe, hij,.}{die aanvoert moet zeer hard zijn}{en illusieloos}\\

\haiku{{\textquoteright} {\textquoteleft}Ik ben klaar wakker,{\textquoteright}, {\textquoteleft}.}{antwoordde ikna deze}{hele dag van rust}\\

\haiku{In de namiddag.}{van de tweede dag scheidden}{onze wegen zich}\\

\haiku{Het kon niet anders,.}{het moest een menselijke}{aanwezigheid zijn}\\

\haiku{Ik verhaastte mijn;}{stappen over de weg die naar}{het schijnsel voerde}\\

\haiku{Wij sliepen wakend,,.}{wachtten zonder haast nu het}{einddoel was bereikt}\\

\haiku{Zijn snor stond schriel en,.}{rafelig aan de punten}{dwars over zijn gezicht}\\

\haiku{{\textquoteleft}Was hij zo bang, dat?}{hij een aspirine nam}{als hij moest vechten}\\

\haiku{Moctezuma moest,:}{me wel erg na{\"\i}ef vinden}{want hij antwoordde}\\

\haiku{{\textquoteright} Wat later ging ik.}{met de Ingeniero het}{terrein verkennen}\\

\haiku{{\textquoteleft}Als we weer bij het,.}{woonhuis zijn zal ik je ook}{d\`at nog laten zien}\\

\haiku{{\textquoteright} {\textquoteleft}En je weet niet dat?}{er nog een slijpsteen staat in}{de grote corral}\\

\haiku{Als het waar is, zal,...}{hij het zeker merken nog}{v\'o\'or ieder ander}\\

\haiku{Daarbuiten is naar.}{mijn opvatting de wereld}{dood gelijk een berg}\\

\haiku{Door hielgeschop dreef;}{hij de onwillige ezel}{naar de bidkapel}\\

\haiku{Eerst had ik nog wel,.}{geduld toen ik slechts weinig}{wist en verweg was}\\

\haiku{Denk niet dat ik hier,.}{alleen maar rondkijk enkel}{maar beveel en vit}\\

\haiku{Zo kalm en rustig,.}{als de beste fokstier ook}{toen ik hem losliet}\\

\haiku{Dus nam ik hem het,.}{medaljon weer af wat hij}{braaf liet geschieden}\\

\haiku{{\textquoteright} {\textquoteleft}Als een bewijs dat,{\textquoteright}.}{u ook practisch werk doet op}{de rancho zei ik}\\

\haiku{{\textquoteleft}Geen echte zelfmoord,.}{natuurlijk maar zoals de}{ezels in Yucat\'an}\\

\haiku{Arme Antonio...,.}{Denk niet dat hij daarbij ook}{in iets tekort schoot}\\

\haiku{Hij kroop heel dikwijls,}{in mijn armen legde}{zich gelijk een kind}\\

\haiku{en wat een moeite,,.}{en verdriet om slechts tot hier}{vandaag te komen}\\

\haiku{Want wat ik in die,.}{flits gezien had overtrof mijn}{ergste fantasie}\\

\haiku{Misschien vergis ik,,.}{mij dacht hij of heb ik het}{in mijn slaap gedaan}\\

\haiku{hij had een nieuwe,.}{meegenomen en nu stond}{er maar een stompje}\\

\haiku{Daarna ging hij naar,,.}{binnen alle kamers door}{en zag alweer niets}\\

\haiku{De betovering}{van de hut scheen er juist door}{te verergeren}\\

\haiku{Maar geef ons vruchtbaar.}{zaad en goede lendenen}{aan onze vrouwen}\\

\haiku{De eerste keer maar,.}{enkele dagen later}{enkele weken}\\

\haiku{Toen was het blijkbaar.}{juist de hoogste uiting van}{liefde en fatsoen}\\

\haiku{En dacht u dat ik?}{voor mariachi langs de straat}{zou willen spelen}\\

\haiku{Zijn zuster die in,.}{Mexico gebleven was mocht}{zeggen wat ze wou}\\

\haiku{{\textquoteright} {\textquoteleft}Dat is heel ver. Maar,.}{ik ben blij dat je weer bij}{ons bent mi capit\'an}\\

\haiku{Terwijl ze nu zelf.}{op de vloer ernaast zich in}{hun deken rolden}\\

\haiku{Mijlen, denk ik, want,,.}{er kwam geen einde aan in}{geen half uur geen uur}\\

\haiku{Het klamme angstzweet,.}{brak mij uit om van het paard}{maar niet te spreken}\\

\haiku{Maar wanneer Ort{\'\i}z hem,.}{aangesteld heeft zal hij zijn}{verdiensten hebben}\\

\haiku{Wanneer ik daarbij;}{dan maar zorgde Veyt{\'\i}a uit}{de weg te blijven}\\

\haiku{Het is zeker dat.}{wij nu nog sterker zijn dan}{de cabrones ginds}\\

\haiku{aarzelend - wie gaat -:}{er graag voor d\'efaitist door}{maar bekommerd zei}\\

\haiku{Ginds, waar nu dat huis,.}{staat met dat ongewone}{dak je kent het wel}\\

\haiku{De indio is veel.}{te zeer verslaafd aan dit wat}{hem juist ondermijnt}\\

\haiku{{\textquoteright} {\textquoteleft}Zeker is geen woord,,{\textquoteright}.}{meer als u mij dat toestaat}{antwoordde Veyt{\'\i}a}\\

\haiku{Ik had het kunnen...,.}{weten dat ik nooit ach laat}{ik daarvan zwijgen}\\

\haiku{Maar mijn uur was zelfs,.}{toen nog niet geslagen ook}{dat stond geschreven}\\

\haiku{Ik zal hem een keer,...}{tortilla's bakken die hij}{daarna nooit meer eet}\\

\haiku{Aan Teobaldo, die,.}{toevallig in mijn buurt kwam}{vroeg ik wat het was}\\

\haiku{Een paar commando's,.}{met zijn stem die plotseling}{schril klonk in de nacht}\\

\haiku{Terwijl ik juist de - -.}{situatie hopeloos}{hier wilde redden}\\

\haiku{{\textquoteleft}En wij winnen de,.}{verbinding met Ort{\'\i}z die veel}{belangrijker is}\\

\haiku{{\textquoteright} En op het lichte:}{schouderophalen van don}{Salustiano}\\

\haiku{Daarom had hij al;}{die dagen nutteloos}{voorbij laten gaan}\\

\haiku{Slecht nieuws heeft de tijd,{\textquoteright}.}{sprak don Salustiano}{bijna fluisterend}\\

\haiku{Het was alsof er.}{buiten slechts gefluisterd werd}{en zacht geslopen}\\

\haiku{de maat die hem op.}{deze laatste verre reis}{moest begeleiden}\\

\haiku{{\textquoteleft}Ik wist niet,{\textquoteright} lachte, {\textquoteleft}.}{hijdat er in u ook nog}{een generaal stak}\\

\haiku{Een vriend in goede,?}{dagen moest het zeker zijn}{in slechte nietwaar}\\

\haiku{{\textquoteright} {\textquoteleft}Even duidelijk als,,.}{het feit dat dit uw dood zou}{zijn binnen het uur}\\

\haiku{Wie anders dan die?}{zoutelozen die het zout}{der aarde moesten zijn}\\

\haiku{{\textquoteright} Hoogst nieuwsgierig ging.}{ik naar de kamer van don}{Salustiano}\\

\haiku{Deze man was een,.}{van Art\'egui's verraders}{dat scheen vast te staan}\\

\haiku{Het lijkt misschien dat,....}{ik verloren heb maar ik}{zal verder zwijgen}\\

\haiku{Het gegeven woord.}{van onze jefe moeten}{wij eerbiedigen}\\

\haiku{{\textquoteleft}Juanito is.}{al onderweg om het u}{te komen zeggen}\\

\haiku{de grootste moed is,.}{misschien die waarbij je beeft}{maar op je post blijft}\\

\haiku{Hij jammerde steeds,}{maar hetzelfde tot opeens}{de negerkoning}\\

\haiku{het hoorde, hem naar:}{zich toe wenkte en hem in}{deugdelijk spaans vroeg}\\

\haiku{Don Nicol\'as was het,;}{hiermee volkomen eens en}{schoot het geld graag voor}\\

\haiku{Helaas, mijn trouwe.}{Candelario ontbrak daar}{in de voorhoede}\\

\haiku{Ik zal je op een.}{keer misschien bewijzen dat}{ik juist gegist heb}\\

\haiku{en ik stapte over,,.}{houtskool vastgekoekte as}{die zelfs nu nog stonk}\\

\haiku{Het alledaagse,.}{lot van zoveel meisjes in}{dit land bedacht ik}\\

\haiku{Want dan zullen ze,,{\textquoteright}.}{toch weer eens komen vroeg of}{laat verzuchtte zij}\\

\haiku{Wel, ik ben blij dat,.}{dit een droge streek is waar}{zoiets niet voorkomt}\\

\haiku{{\textquoteleft}Eindelijk is het,.}{toch gebeurd waarvoor ik al}{die tijd al bang was}\\

\haiku{{\textquoteright} {\textquoteleft}Dit is maar de helft,{\textquoteright}.}{van de waarheid meende don}{Salustiano}\\

\haiku{Daarom wordt elke.}{oorlog weer gevolgd door een}{of nog meer nieuwe}\\

\haiku{En als hij hem nu,.}{maar teruggeeft dan zijn wij}{beiden geholpen}\\

\haiku{{\textquoteleft}Maar denkt erom, je,,?}{steekt geen poot uit als ik het}{niet zeg begrepen}\\

\haiku{het kan voor hem ook,.}{nuttig zijn dit zonderling}{verhaal te horen}\\

\haiku{En toch...{\textquoteright} {\textquoteleft}U wordt de,{\textquoteright}.}{oudste hacendado van}{Tamaulipas zei Isidro}\\

\haiku{Ik heb er ook voor;}{gezorgd dat niets meer over is}{van al dat prutswerk}\\

\haiku{De breuknaadjes zijn,...}{weliswaar nog te zien maar}{als een haar zo fijn}\\

\haiku{Op de Rancho der.}{Drievuldigheid werd het zeer}{rustig en vertrouwd}\\

\haiku{{\textquoteleft}Dat heeft bovendien.}{niets met oude of nieuwe}{mensen te maken}\\

\haiku{Hij scheen een echte;}{fat en het beste was hem}{nog niet goed genoeg}\\

\haiku{Wij verzwegen dat,:}{het beest nog ongebroken}{was maar zeiden wel}\\

\haiku{{\textquoteright} {\textquoteleft}Marjorie ziet er,,}{heus niet naar uit dat ze een}{gebroken hart heeft}\\

\haiku{Maar ik stelde met:}{een soort deskundigheid reeds}{van tevoren vast}\\

\subsection{Uit: Amor ontdekt Aruba}

\haiku{{\textquoteleft}Maar u behoeft zich.}{gedurende de wachttijd}{niet te vervelen}\\

\haiku{U kunt dan daarmee,;}{naar de stad gaan of iets van}{het eiland gaan zien}\\

\haiku{Maar denkt u erom.}{dat wij al over een paar uur}{kunnen vertrekken}\\

\haiku{Er woei een adem van:}{oneindigheid door deze}{ritselende rust}\\

\haiku{Want ik ben verliefd,.}{geworden op uw eiland}{verliefd op Aruba}\\

\haiku{Landelijk en toch,.}{niet boers afgetrokken en}{toch niet eenzelvig}\\

\haiku{Toch werd hier in de,{\textquoteright}.}{buurt vrij veel goud gevonden}{vertelde Sjon Eli}\\

\haiku{{\textquoteright} {\textquoteleft}Het is een flauwe,,{\textquoteright},.}{rottige grap zei Cynthia}{ook even buiten adem}\\

\haiku{Integendeel, dan.}{tonen zij zich juist in hun}{ware gedaante}\\

\haiku{{\textquoteleft}Hij en ik geven.}{tenminste al onze vrije}{tijd aan dit eiland}\\

\haiku{{\textquoteleft}Eilandbewoners,{\textquoteright}.}{zijn gek op de zee merkte}{Cynthia lachend op}\\

\haiku{{\textquoteright} De steile, platte.}{kust bleef geaccidenteerd}{in zijn contouren}\\

\haiku{Nog standhoudend, een,,.}{eeuw twee eeuwen misschien maar}{zeker niet voorgoed}\\

\haiku{{\textquoteright} vroeg Cynthia, meer aan.}{zichzelf dan aan Sjon Eli tot}{wien zij zich richtte}\\

\haiku{{\textquoteleft}Er woei een adem van:}{oneindigheid door deze}{ritselende rust}\\

\subsection{Uit: De dolle dictator. Het ondoorgrondelijke leven van Juan Manuel de Rosas}

\haiku{Ook zij wier beide.}{ouders uit Europa naar}{Amerika kwamen}\\

\haiku{Mesties- halfbloed,.}{nakomeling van blanken}{en indianen}\\

\haiku{Misschien zijn ze nog,,.}{vijf leguas ver maar je}{zult zien ze komen}\\

\haiku{Maar don Clemente,.}{weet ze \'o\'ok te ruiken en}{langer dan jij}\\

\haiku{Don Clemente weet.}{dat zijn paard verloren is}{en springt op de grond}\\

\haiku{eerst zuidwaarts naar de, '.}{heuvels dan naart Oosten}{waar de zee moet zijn}\\

\haiku{{\textquoteright} De jongen houdt een,.}{harde kop een etmaal blijft}{hij opgesloten}\\

\haiku{Er stijgt geloei op,.}{dat verlangend uitklinkt in}{de oneindigheid}\\

\haiku{Hij zit toch zeker.}{even vast te paard gelijk de}{beste onder hen}\\

\haiku{De dieren paren,?}{toch zodra hun jonge kracht}{begint te rijpen}\\

\haiku{ze weet dat het niet,.}{waar is wat ze zei ze kent}{haar oudste immers}\\

\haiku{En ze weet ook dat.}{hij nu het toppunt van zijn}{woede heeft bereikt}\\

\haiku{Want ook Europa.}{beleeft weer een hausse in}{haar koningsschappen}\\

\haiku{Goed, waarvoor heeft hij?}{anders de lange avonden}{op Los Cerrillos}\\

\haiku{E\'enmaal in de;}{week brengt een koerier hem zijn}{brieven en zijn krant}\\

\haiku{Deze laatsten, en,.}{vooral zijn moeder nemen}{het hem hoogst kwalijk}\\

\haiku{Men schiet niet op, want:}{L\'opez is onvermurwbaar in}{een van zijn eisen}\\

\haiku{Verder laat hij niets,.}{uit en Rosas voelt zich steeds}{meer gekrenkt door hem}\\

\haiku{Die zwaait hij in de.}{wind met het bosje hooi dat}{plotseling \`opvlamt}\\

\haiku{Ten Noorden daarvan.}{mogen nu de christenen}{hun forten bouwen}\\

\haiku{Terwijl dit gaande,.}{is versterkt Rosas de grens}{zoveel mogelijk}\\

\haiku{Ook Rosas wordt zich.}{nu de herhalingsdwang in}{zijn leven bewust}\\

\haiku{zijn blonde haren;}{aan de conquistadores}{uit Noord-Spanje}\\

\haiku{Was het niet billijk?}{dat er meer indianen}{stierven dan koeien}\\

\haiku{Ze mogen zich bij;}{de regering vervoegen}{als ze lust hebben}\\

\haiku{Het zijn kisten vol:}{die hij meesleept aan boord om}{te bestuderen}\\

\haiku{Het is ontzaglijk}{hoeveel hij reeds weet en hoe}{dit vele weten}\\

\haiku{Hij is een gaucho;}{en door niemand laat hij zich}{in de kaart kijken}\\

\haiku{het spreekt vanzelf dat,.}{hij degene is bij wie}{de beslissing ligt}\\

\haiku{{\textquoteright} Juan Manuel.}{is gedwongen wederom}{hoog spel te spelen}\\

\haiku{Nu is Quiroga,:}{wakker en gebogen uit}{het raampje roept hij}\\

\haiku{Het antwoord is een:}{half-gefluisterde}{roep uit aller mond}\\

\haiku{Allen vliegen op,.}{haar wenken beven voor haar}{striemende woorden}\\

\haiku{In de familie.}{Rosas aarden de mannen}{steeds naar hun moeders}\\

\haiku{Hij moet er niet te,.}{veel aan denken het zou hem}{weemoedig maken}\\

\haiku{Hoewel het maar een.}{vodderig  berichtje}{van Cuiti\~no is}\\

\haiku{Maar hij wil niet de,,.}{enige zijn niet de enige}{verrader niet hier}\\

\haiku{{\textquoteright} Daarop gaat Ram\'on snel.}{naar huis om zich gereed te}{maken voor de reis}\\

\haiku{Het gaat van mond tot.}{mond dat Ram\'on Maza een der}{hoofd-raddraaiers is}\\

\haiku{Er is niets aan te,.}{doen al moet de halve stad}{eraan geloven}\\

\haiku{Rosas kijkt koel naar,,:}{haar handenwringen haalt de}{schouders op en zegt}\\

\haiku{{\textquoteright} Hiermee heeft hij de.}{ander nog eens extra aan}{het schrikken gemaakt}\\

\haiku{Ze imiteren hem.}{en weten hun lot aan het}{zijne gebonden}\\

\haiku{Hij ziet het vochtig;}{schitteren van haar ogen en}{haar lieve glimlach}\\

\haiku{Ze liegen er niet,,.}{om ze spreken klare taal}{tot in hun titels}\\

\haiku{De sluwheid waarmee,.}{hij daarbij te werk gaat zegt}{niets van zijn verstand}\\

\haiku{Hij wil lijken om,.}{zich heen hebben het komt er}{niet op aan van wie}\\

\haiku{Onmiddellijk had.}{de dikke mulat aan dit}{bevel gehoorzaamd}\\

\haiku{Las Balchitas, is?}{dat geen oord in het diepste}{van Patagoni\"e}\\

\haiku{{\textquoteright} - {\textquoteleft}Dan zullen wij hem,{\textquoteright}.}{waardig ontvangen antwoordt}{Juan Manuel}\\

\haiku{Juan Manuel:}{laat deze troep aantreden}{en verklaart plechtig}\\

\haiku{Daarna zullen wij.}{overgaan tot de plechtigheid}{van de wederdoop}\\

\haiku{Vaag, onkenbaar vaag.}{schemert hem het beeld van zijn}{moeder voor de ogen}\\

\haiku{Juan Manuel.}{heeft zich ook tegen deze}{liefde fel gekant}\\

\haiku{Nog wordt hij door de!}{commandant behandeld als}{een hoge eregast}\\

\haiku{En het andere...?}{dan haar vijf kinderen die}{ook de zijne zijn}\\

\haiku{Vrienden, vijanden,.}{ze zijn haast allen van het}{toneel verdwenen}\\

\haiku{- {\textquoteleft}Er is geen die de,,.}{wet stelt geen die eist alleen}{ik zelf tot mijzelf}\\

\haiku{bij sneeuw en wind blijft,.}{hij buiten werken net als}{op een lentedag}\\

\haiku{En Rosas met zijn;}{ijzeren energie begint}{weer te bevelen}\\

\haiku{En gezuiverd in,.}{de oude onbedorven}{indiaanse schoot}\\

\haiku{De trotse Pampas,?}{de Tehueltsches en}{de Pueltsches}\\

\subsection{Uit: Het euvel Gods}

\haiku{Maar jij, mijn vriend, ik.}{weet dat jij de jouwe kent}{als ik de mijne}\\

\haiku{Tot zoo laat was er;}{tusschen al Gods dagen geen}{verschil voor Jukkers}\\

\haiku{Spranger belde mij....}{op voor een spoedoperatie}{voor Arthur Christensen}\\

\haiku{De ander trachtte,.}{hem te kalmeeren maar hij}{wond zich steeds meer op}\\

\haiku{het gewelf was zoo,.}{hoog dat het in het midden}{niet meer te zien was}\\

\haiku{Als menschen uit een.}{andere wereld staarden}{wij elkander aan}\\

\haiku{Ik meen te mogen:}{veronderstellen dat hij}{heeft uitgeroepen}\\

\haiku{In ieder geval,.}{was het beestje omnivoor}{stelde Fowler vast}\\

\haiku{Het brak tusschen mijn,.}{tanden en ik bekeek het}{stuk dat ik vasthield}\\

\haiku{Ik zag hoe Fowler,.}{kokhalsde en was niet in}{staat te antwoorden}\\

\haiku{Vermoeid was zij nu,.}{na drie jaren geworden}{ten doode vermoeid}\\

\haiku{Vragen durfde hij, ':}{niet want wat hijt eerst zou}{willen vragen was}\\

\haiku{En omdat je uit.}{eigen ervaring reeds weet}{hoe het daar zijn kan}\\

\haiku{Maar de oude vrouw.}{trok me aan de mouw mee naar}{een andere reet}\\

\haiku{{\textquoteleft}Wij moeten nu de,.}{andere kant uit want het}{laatste huis is leeg}\\

\haiku{En weer is schemer.}{gevallen over de wankel}{voortstappende man}\\

\haiku{Ebenhezer rookte.}{uit een steenen pijpje en zon}{op al zijn godsdienst}\\

\haiku{Ze komen in het,.}{huis waar het niet noodig is het}{licht te ontsteken}\\

\haiku{Bijna is het dag,{\textquoteright},.}{zegt hij naar buiten stappend}{waar de paarden staan}\\

\haiku{juist zooals het op diens;}{minutieuze kaarten}{was aangegeven}\\

\haiku{De noordelijkste!}{kaart van pater Martinus}{was onnauwkeurig}\\

\haiku{aan sterke takels.}{rijdt de groote ketel gloeiend}{ijzer uit het vuur}\\

\haiku{Zoo moest ik het ook,.}{in mijn handen  kunnen}{vasthouden dacht Cobbs}\\

\haiku{{\textquoteright} Maar slap en willoos.}{liet ze zich doorgolven en}{Cobbs verloor de moed}\\

\haiku{Waarom heb ik ook,.}{geen moeder fluisterde het}{binnenste van Cobbs}\\

\haiku{Een mensch bewaart het,}{schoonst de herinneringen}{aan bergen zoolang}\\

\subsection{Uit: Facetten van de Surinaamse samenleving}

\haiku{Hun stemmen spreken,.}{door de puyai heen als}{bij een buikspreker}\\

\haiku{De therapie volgt.}{dan uit de diagnose en}{de etiologie}\\

\haiku{Dit hangt vaak meer van,.}{de pati\"ent af dan van}{de middeltjes zelf}\\

\haiku{Niet m\'e\'er overigens;}{dan de stadsbewoners van}{menig ander land}\\

\haiku{immers spoedig, als,:}{jonge vrouw had de dochter}{maar \'e\'en bestemming}\\

\haiku{{\textquoteleft}Avonturen{\textquoteright} winden.}{hem zelden op en hij zoekt}{ze ook niet bepaald}\\

\haiku{Ook bij andere;}{gelegenheden zingen}{ze spotliederen}\\

\haiku{Het huisraad staat steeds.}{glimmend te pronken in het}{halflicht van de hut}\\

\haiku{Het taboe van de...}{menstruerende vrouw zit}{ook nog vrij diep in}\\

\haiku{Een hangmat is voor.}{Indianen uiteraard}{een primair bezit}\\

\haiku{Dit dient hij op zijn.}{minst tot aan de geboorte}{van het kind te doen}\\

\haiku{{\textquoteleft}Natuurlijk, meneer,.}{de Voorzitter zodra er}{kinderen komen}\\

\haiku{een korjaal, een huis,:}{of de symbolisering}{van zijn gevoelens}\\

\subsection{Uit: De G.G. van Tellus}

\haiku{{\textquoteleft}Lijken net van die,;}{romanschrijvertjes kunnen}{niet samenvatten}\\

\haiku{{\textquoteleft}Die stommelingen.}{maken hun rapporten nooit}{volledig genoeg}\\

\haiku{Moet ik ze een voor?}{een bij me laten komen}{en met ze praten}\\

\haiku{Men kan alles uit,.}{de weg lopen behalve}{de medemensen}\\

\haiku{Dan verdwijn ik nu,{\textquoteright},,.}{maar zegt ze ondanks zichzelf}{maar toch half vragend}\\

\haiku{Hij vindt dat Edm\'ee.}{Duval en deze bloemen}{bij elkaar horen}\\

\haiku{Ze heeft een soort van,,.}{weerstand trots of een restant}{conventie misschien}\\

\haiku{Ach, moet nu reeds haar?}{betovering verbroken}{worden door een bruut}\\

\haiku{Op 19 dezer is.}{zijn gemelde novelle}{gereedgekomen}\\

\haiku{Inderdaad in een.}{toestand die weinig verschilt}{met die van beesten}\\

\haiku{{\textquoteleft}O, mon dieu, van.}{een bullebak zou ik niet}{erger verschrikken}\\

\haiku{Rosy denkt alleen.}{aan de vorige romans}{van Norman Angus}\\

\haiku{Ja, ja!) dan moet ik.}{u zeggen dat ik het een}{immoreel werk vind}\\

\haiku{{\textquoteright} vraagt Tom op een toon.}{waarin zijns ondanks toch een}{tikje spot doorklinkt}\\

\haiku{Ze huilen mee met...}{de wolven met wie ze het}{aas mogen delen}\\

\haiku{Het werken zelf is,.}{geen slavernij al is het}{nog zo zwaar en veel}\\

\haiku{Ouders zijn niet oud,.}{zolang de kinderen nog}{hun zorg behoeven}\\

\haiku{Nu beginnen de.}{jongens te praten en wordt}{over en weer verteld}\\

\haiku{Dan wordt het direct,.}{moord en doodslag echtscheiding}{en advocaten}\\

\haiku{een wijffie dat nog,{\textquoteright}.}{best te zoenen valt vertelt}{Eddy openhartig}\\

\haiku{Het brengt hem zelf ook,:}{in een speelse stemming en}{plagerig zegt hij}\\

\haiku{{\textquoteleft}Ik zit toch altijd.}{zo'n beetje te denken over}{wat ik zetten moet}\\

\haiku{Een mooie vrouw weet dat.}{ze een bankrekening heeft}{bij iedere man}\\

\haiku{Maar hij ontdekte,.}{dat dit gevoel zelf fout was}{een zonde in hem}\\

\haiku{Ze had hem kunnen,.}{schrijven misschien zou hij haar}{geholpen hebben}\\

\haiku{wanneer je je in.}{staat voelde zelfs je eigen}{moeder te haten}\\

\haiku{wist hij slechts hoe het,.}{te bereiken hoe het het}{diepst te treffen}\\

\haiku{hij is spontaner,,.}{ondernemender net als}{de vlam van een brand}\\

\haiku{Ten langen leste,.}{voor een spiegel waarin hij}{keek zonder te zien}\\

\haiku{Hij weet ook niets van:}{die andere smerige}{geschiedenissen}\\

\haiku{Alles aan hem is,,.}{rein onaangetast ofschoon}{hij een lamme is}\\

\haiku{En hoe vlijmend en.}{geestig kunnen zijn woorden}{een vijand wonden}\\

\haiku{{\textquoteleft}Neen, lieveling, je.}{weet dat ik op zee nooit zin}{heb om te lezen}\\

\haiku{Dacht je dat ik een?}{half uur kon werken zonder}{aan je te denken}\\

\haiku{De jonge Joachim,.}{Dijkman dan was nog maar een}{paar jaar in het vak}\\

\haiku{Niet voor niets had hij...}{aan het leven eigenlijk}{de brui gegeven}\\

\haiku{{\textquoteleft}Maar verzen kun je.}{toch nooit helemaal kennen}{door lezen alleen}\\

\haiku{Wat was de inhoud?}{van dit prettig gedrukte}{deemoedig dulden}\\

\haiku{Behalve dan {\textquoteleft}De{\textquoteright},.}{wapens neer een groots epos dat}{zijn jeugd had verblijd}\\

\haiku{Hij herinnerde.}{zich niet dat hij hem ooit was}{tegengekomen}\\

\haiku{Geen ogenblik was ze.}{Joachim Dijkman meer uit de}{gedachten geweest}\\

\haiku{En telkens opnieuw.}{trachtte hij zich haar beeld voor}{de geest te brengen}\\

\haiku{Dit bracht haar op een,.}{idee dat zij zonder dralen}{wilde uitvoeren}\\

\haiku{{\textquoteright} juichte Dijkman, en.}{hij snelde reeds weg om het}{boek te gaan halen}\\

\haiku{Er ontsnapte een,.}{woord aan Joachim zonder dat}{hij wist hoe het kwam}\\

\haiku{Hij zag duidelijk:}{de visioenen die het}{boek voor hem opriep}\\

\haiku{Soms schaamde hij zich.}{bijna over de nukkigheid}{van zijn denkfuncties}\\

\haiku{Ik heb een hekel,.}{aan romans maar in dit werk}{steekt veel waardevols}\\

\haiku{{\textquoteleft}Zeer ten onrechte,,{\textquoteright}.}{geloof ik voegde Walstra er}{gewichtig aan toe}\\

\haiku{{\textquoteright} {\textquoteleft}Vertalen is op,{\textquoteright}.}{zichzelf een grote kunst viel}{de leraar hem bij}\\

\haiku{Zijn zelfkritiek zei.}{hem dat hij beter schrijven}{kon dan vertellen}\\

\haiku{{\textquoteleft}Is dat niet een van?}{de zondagsgetijden van}{dominee Stuurman}\\

\haiku{{\textquoteleft}Het wordt tijd  dat.}{er weer eens goede romans}{geschreven worden}\\

\haiku{6307-18a (inzake).}{de voorstellingswereld van}{geobserveerde}\\

\haiku{Jesses, ze wilde.}{vooral niet meer denken aan}{die beroerde school}\\

\haiku{Dat juist de dingen,.}{waar je zin in had je zo}{verpest moesten worden}\\

\haiku{{\textquoteleft}God,{\textquoteright} zei mevrouw met, {\textquoteleft}.}{verbazingmisschien is ze}{niet goed geworden}\\

\haiku{{\textquoteright} vroeg de moeder, boos.}{nu ze niet meer ongerust}{behoefde te zijn}\\

\haiku{Mijnheer Steelinks voet.}{stiet tegen het open op de}{grond gevallen boek}\\

\haiku{{\textquoteright} {\textquoteleft}Zo draai je in een,,{\textquoteright}.}{cirkeltje rond mijn beste}{sprak de G.G. waardig}\\

\haiku{Voelde ze daarom?}{tegenwoordig zoveel voor}{reclasseringswerk}\\

\haiku{Maar hij weet dat hij.}{de zaak nog niet zo gauw}{eraan kan geven}\\

\haiku{{\textquoteright} De oude heer maakt,.}{zich nu weer boos en trappelt}{voor zich uit van drift}\\

\haiku{{\textquoteright} Dani\"el antwoordt,.}{maar niet buigt enkel het hoofd}{een klein beetje meer}\\

\haiku{{\textquoteleft}Ik moet, wou, zou graag.}{weten wie Ja\"el precies}{moet optrommelen}\\

\haiku{Dan zijn jullie een,.}{waardeloze bende aan}{de ontbinding toe}\\

\haiku{{\textquoteleft}Ja, wel zonderling,{\textquoteright},.}{zegt Rose Clark terwijl ze}{Angus de hand drukt}\\

\haiku{{\textquoteright} {\textquoteleft}Interessant kind,{\textquoteright},.}{bromt Van Stolwijk zonder het}{meisje aan te zien}\\

\haiku{Willy Roosendaal.}{staat er ook maar verlegen}{en eenkennig bij}\\

\haiku{Maar hij antwoordde:}{niet op de onderbreking}{en ging kalm verder}\\

\haiku{Iets waarover de G.G..}{zeer weinig gesticht is op}{velerlei gronden}\\

\haiku{Om zoiets zomaar,.}{openbaar te maken erger}{straf is ondenkbaar}\\

\haiku{Het wordt stiller en.}{stiller in de grote zaal}{met al die mensen}\\

\haiku{een oneindige,.}{levend-dode wereld}{die voor hem openligt}\\

\haiku{dat ondervindt zij,,.}{de voormalige Rosy}{Fiddle dubbel erg}\\

\haiku{Al haar verdrongen.}{afkeer van vroeger komt nu}{naar boven als haat}\\

\haiku{{\textquoteleft}Ik geloof niet dat.}{dit het goede moment is}{voor plichtplegingen}\\

\haiku{Hij had er een boek,.}{over geschreven zoveel had}{ze wel begrepen}\\

\haiku{{\textquoteright} {\textquoteleft}Het was niet gemeen,}{genoeg of u wist er een}{boek over te schrijven}\\

\haiku{{\textquoteright} {\textquoteleft}Troost is surrogaat,{\textquoteright} {\textquoteleft}.}{antwoordt de G.G.Steek uw hand}{in uw zak en voel}\\

\haiku{De wetsvoltrekking,.}{de bekrachtiging van je}{geschonden moraal}\\

\haiku{In het begin was,.}{hij goed voor mij hij hield op}{zijn manier van mij}\\

\haiku{uit wie ik voortkom;}{en bij wie ik eindelijk}{teruggekeerd ben}\\

\haiku{Hij heeft goed praten...}{tegen al de mensen die}{nu hij hem komen}\\

\haiku{Dat moet goed zijn, want,.}{ze wil niets meer er is niets}{meer waarop ze wacht}\\

\haiku{Ze wil zich op de.}{grond laten glijden om dit}{werkelijk te doen}\\

\haiku{{\textquoteright} brengt Simone in, {\textquoteleft}?}{het middenweet u dan ook}{waar Ren\'e uithangt}\\

\haiku{Wat ze ook wel doen,,.}{maar vaak te stuntelig te}{grillig of averechts}\\

\haiku{je moet er met een,.}{voorhamer op slaan eer ze}{wat weerklank geven}\\

\subsection{Uit: Hart zonder land}

\haiku{- {\textquoteleft}Nu zullen wij niet{\textquoteright},.}{op de anderen wachten}{sprak Anne halfluid}\\

\haiku{De gewoonheid van.}{haar kinderspel had niemand}{kunnen verbazen}\\

\haiku{In bed dacht zij dan,:}{nog dikwijls aan de pop en}{het was wonderlijk}\\

\haiku{Groucha antwoordde,:}{niet maar de student pakte}{de mand op en zei}\\

\haiku{Toen vouwde hij de.}{blaadjes voorzichtig dicht en}{stak ze in zijn zak}\\

\haiku{een die ik ben, en,.}{die toch meer is dan ik want}{ook een deel van jou}\\

\haiku{{\textquoteright} - {\textquoteleft}Onze plantages,{\textquoteright}.}{zijn meer Afrika dan je denkt}{sprak de oude man}\\

\haiku{Zonder dat ze 't,}{weten spreken die kerels}{van hun vaderland}\\

\haiku{De volgende dag.}{vroeg ik of Lindor met mij}{mee mocht gaan jagen}\\

\haiku{Eens had Wassilj aan.}{een man gevraagd of het nog}{ver was naar Moskou}\\

\haiku{Er vloeide geen bloed,,.}{maar de zwerm week achteruit}{verder en verder}\\

\haiku{de takken van een,!}{heester zie ik de open plek}{en welk een schouwspel}\\

\haiku{hoe grijze nevels.}{hangen in een huis waar licht}{en liefde woonden}\\

\haiku{In Alec's arm leunde,.}{Kesini zwaarder en vermoeid toen}{zij huiswaarts keerden}\\

\haiku{En omdat ik bang,.}{ben je te verliezen zooals}{Sunanda je verloor}\\

\haiku{{\textquoteright} Doodsbleek ging zij naar,,.}{binnen haar zuster trad haar}{reeds tegemoet blij}\\

\haiku{en zich omdraaiend,.}{spuwde hij in een nis waar}{een ikoon moest hangen}\\

\haiku{Het leven dat je.}{bij een stervende bijna}{als een onrecht voelt}\\

\haiku{{\textquoteleft}Laten we hem naar,.}{het hospitaal brengen hij}{is bewusteloos}\\

\haiku{Hijgend wilde hij,.}{roepen maar zijn stem was slechts}{een schor gerochel}\\

\haiku{Mijn lippen raken,,,;}{u Ichthus witte visch die}{geslacht zijt om ons}\\

\haiku{aanstonds, morgen, pijpt,.}{hij voor en allen dansen}{wij dien Orpheus na}\\

\haiku{Zij zag hoe of hij,.}{leed en nam de kevie mee}{naar de biljartzaal}\\

\haiku{het bloed ruischt aan,,.}{zijn slapen iets hamert in}{zijn hoofd doorschokt hem}\\

\haiku{En toen de avond kwam,.}{begaf de jonge vrouw zich}{reeds vroeg ter ruste}\\

\subsection{Uit: Kleine kosmologie}

\haiku{De dubbelster volgt.}{snel en weifelloos haar weg}{en stelt geen vragen}\\

\haiku{Is het leven een?}{samoem die opkomt uit de}{verre horizon}\\

\haiku{ruimte die gevuld,.}{moet worden wijl begrenzing}{om vervulling vraagt}\\

\haiku{integratie der;}{atomen en versplintering}{der fantasie\"en}\\

\haiku{dommekracht waarmee;}{een zwak verstand heelallen}{in hun assen tilt}\\

\haiku{{\textquoteright} Opziend naar het jong,}{gezicht vol levenslust en}{overmoed glimlachte}\\

\haiku{Maar nooit tevoren,:}{merkte ik wat ik thans met}{blij verbazen zag}\\

\haiku{hoe welgevormd zij,,...}{was hoe lieflijk haar gelaat}{hoe rank haar leden}\\

\haiku{Door de opening zag.}{ik de sterren schijnen over}{het bedauwde veld}\\

\haiku{Uranus Lang reeds was;}{hij bekend als een ster van}{bescheiden grootte}\\

\haiku{Of missen wij  ,,?}{het goddelijk span van Hoop}{Volharding Liefde}\\

\haiku{Gelaten kreunt de.}{oude aarde en heft zich}{uit haar slaap verjongd}\\

\haiku{Zo weegt de grote,:}{Waagschaal van de dood waarop}{wij alles zetten}\\

\haiku{nacht van diepe slaap,,.}{weerbarstig ondoordringbaar}{voor den vreemdeling}\\

\haiku{Het lukt misschien nog,{\textquoteright}.}{beter als je ook mijn kleed}{aantrekt zegt Yahyah}\\

\haiku{Wat zulk een oude?}{wijsgeer uit kan kuren als}{hij wordt ondervraagd}\\

\haiku{nagezeten door,.}{de schimmen van een beer een}{duivel en een draak}\\

\haiku{Wend u Oost-.}{en wend u Westwaarts langs uw}{eeuwenoude baan}\\

\haiku{Tot Herakles kwam,...}{met zijn stierkracht en allen}{verbaasde ook u}\\

\haiku{Tevergeefs welft de,.}{ruiter zijn borst zit  ik}{fier in het zadel}\\

\haiku{Steeds als ik hem weer,,:}{ontdek Dolfijn en redder}{van de kunst denk ik}\\

\haiku{Hoger, hoger dan.}{het altaar moet het beeld staan}{van den lieveling}\\

\haiku{En dat ik in ruil;}{daarvoor stil en alleen als}{een ster mag vergaan}\\

\haiku{Het gefluister der,;}{sferen werd duidelijker}{allengs verstaanbaar}\\

\haiku{{\textquoteright} Hen naderde ik, '.}{ondert afgrijselijk}{gesnurk van hun slaap}\\

\haiku{En dit alleen scheen.}{al voldoende om mij de}{arm te verlammen}\\

\haiku{Druppelend viel uit;}{de zak aan mijn zijde het}{bloed van Medusa}\\

\haiku{en iedere drup.}{in het zand der woestijn werd}{een giftig serpent}\\

\haiku{En langzaam nadert;}{Thuban weer het middelpunt van}{ons beperkt heelal}\\

\haiku{Het kind echter geeft;}{aan de nimf het geschenk met}{een glimlach terug}\\

\haiku{dat werd geboren,;}{om te branden en nimmer}{doven zal tot as}\\

\haiku{Het zonlicht slurpt de;}{nevelrafels en vlijt zich}{op de heuvelflank}\\

\haiku{Indien een man niet,,.}{komt Vindemiatrix moet}{een god u plukken}\\

\haiku{Vliegende Velox,...}{die vliedt en toch niet uit de}{kijker kunt lopen}\\

\haiku{Vliegende Velox,...}{die vliedt en toch niet uit de}{kijker kunt lopen}\\

\haiku{Vliegende Velox,...}{die vliedt en toch niet uit de}{kijker kunt lopen}\\

\haiku{Leer van de lenzen,;}{hoe gauw onze wensen de}{daad achterhalen}\\

\haiku{Vliegende Velox,...}{die vliedt en toch niet aan de}{wacht kunt ontlopen}\\

\haiku{sinds erger schande,.}{van herroepen wat hij w\'e\'et}{hiermee bespaard wordt}\\

\haiku{'t Groot gezang der:}{sferen overstemt het bidden}{der Inquisiteurs}\\

\haiku{Drie, vier eeuwen zijn.}{maar drie of vier seconden}{in de wereldtijd}\\

\haiku{Wanhoopt niet, het licht!}{zal eens de duisternis voor}{altoos overwinnen}\\

\haiku{Het ware weten,...}{springt te voorschijn in grote}{stilte binnenshuis}\\

\haiku{Het ware inzicht ';}{gaat slechts open voor wie int}{licht een blinde was}\\

\haiku{Ik echter kan thans,;}{niet meer falen mijn ster dwaalt}{nimmer van haar spoor}\\

\haiku{kringen vormend in.}{een spel dat even zinloos is}{als alle spelen}\\

\haiku{De boordeloze.}{ruimte heeft voor dit drijvend}{schip geen ankerplaats}\\

\haiku{En eerst - hoe lang is,}{dat geleden en hoe lang}{moest dat nog duren}\\

\haiku{met een plotseling,:}{weer hoger laaien om dan}{weer saam te trekken}\\

\haiku{gij zijt de oorzaak,;}{der geboorten verzoent ons}{met de reis des doods}\\

\haiku{het weten dat elk.}{aards beminnen een jacht ter}{zelfbevrijding is}\\

\haiku{ik ben nog zaad, nog.}{zucht van een gedachte die}{zich straks pas openbaart}\\

\haiku{hoezeer dit leven;}{in zijn broosheid dat van het}{heelal weerspiegelt}\\

\subsection{Uit: De laaiende stilte}

\haiku{Dan moet ik dubbel,.}{voorzichtig zijn want dat mag}{nooit grijpbaar worden}\\

\haiku{En Josephine...}{grijpt daarbij zijn hand vast en}{kijkt gelukzalig}\\

\haiku{Zij hebben gelijk.}{met te denken dat ook ik}{hier gelukkig ben}\\

\haiku{Ik koester mijn pijn.}{als een kind dat mijn borst bijt}{terwijl het zich voedt}\\

\haiku{Nachtenlang heeft mij.}{dit akelig gezicht tot in}{mijn slaap achtervolgd}\\

\haiku{Pas veel later ben,}{ik erover gaan nadenken}{hoe het mogelijk}\\

\haiku{want de dorre hand.}{van plicht en schijn drukt op mijn}{mond en smoort mijn stem}\\

\haiku{Het enige dat een,.}{kortstondig licht brengt in de}{brede wrede nacht}\\

\haiku{{\textquoteleft}Integendeel, zijn.}{strijdbare natuur dwingt hem}{tot gemeenzaamheid}\\

\haiku{er is geen plaats voor,?}{liefde meer en waar moet ik}{de mijne bergen}\\

\haiku{Morhang voor Raoul.}{kunnen beheren tijdens}{zijn afwezigheid}\\

\haiku{angst en onrust die.}{heel onze onzekere}{jeugd begeleid heeft}\\

\haiku{Maar zal Raoul ooit,?}{kunnen zullen wij vrouwen}{daartoe in staat zijn}\\

\haiku{Terwijl ik toch, in,.}{de grond van mijn hart elke}{verandering vrees}\\

\haiku{Toch is Amsterdam,.}{een drukke stad de drukste}{die ik ooit bezocht}\\

\haiku{Onrust drijft hem uit,.}{het woonvertrek het huis uit}{en de straten op}\\

\haiku{Toch is het beter,.}{geweest dit hier te horen}{dan nog op Morhang}\\

\haiku{Elk van ons is zo,,,.}{geslagen dat geen woord geen}{klacht geen troost meer klinkt}\\

\haiku{Met hem was het iets,.}{anders en daarom is hij}{van mij weggegaan}\\

\haiku{De storm die bijna,.}{een week lang geduurd heeft is}{eindelijk voorbij}\\

\haiku{Ik heb er altijd;}{naar verlangd en het gezocht}{met mijn verbeelding}\\

\haiku{Als ik dat nu ook;}{maar kon begrijpen v\'o\'or het}{einde van de reis}\\

\haiku{Er was niet enkel,.}{welkom maar ook afweer van}{het onbekende}\\

\haiku{Zoals ik toen al,.}{dadelijk verwachtte viel}{Willem Das hen bij}\\

\haiku{zijn gedragingen;}{tegenover de negers en}{de negerinnen}\\

\haiku{Tot mijn verbazing.}{vroeg hij toen zelfs niet waar mijn}{paard gebleven was}\\

\haiku{Wantrouwen, stomme,;}{woede en wrok ze loeren}{aan alle kanten}\\

\haiku{Een geldbedrag dat.}{hij bovendien bij lange}{na nog niet bezit}\\

\haiku{Het is zodoende.}{bijna tot een twist tussen}{de twee gekomen}\\

\haiku{Was dat echter waar,.}{hij moet mij dan al dikwijls}{hebben bijgestaan}\\

\haiku{Want er komt een dag.}{dat elke zoon ook op zijn}{beurt de vader wordt}\\

\haiku{Hij is veel vrijer,,.}{innerlijk dan wie ook hier}{op de plantage}\\

\haiku{Meer dan eens heb ik -}{mijn blikken voor de zijne}{neergeslagen ach}\\

\haiku{Het is wel ver met,}{mij gekomen dat ik dit}{zo onomwonden}\\

\haiku{Niemand behoeft meer,.}{te spreken niets behoef ik}{meer op te schrijven}\\

\haiku{Zolang hij er is,,.}{zullen zij niets verkeerds doen}{heeft hij mij beloofd}\\

\haiku{hun begeerte naar,.}{weelde onverzadigbaar}{door domme spilzucht}\\

\haiku{Hoe kon het anders?}{dan dat dit het laatste was}{wat hij deed en zei}\\

\haiku{Had ik toen iets kwaads,?}{gedaan wie zou er zich om}{bekommerd hebben}\\

\haiku{De enige die het,;}{zou hebben geweten zou}{Isidore zijn geweest}\\

\haiku{de wereld hier kent,.}{geen verleden zomin als}{zij een toekomst heeft}\\

\haiku{Nog altijd zie ik,;}{die blik van Isidore welke}{mij dit geleerd heeft}\\

\haiku{Nu is hij gevlucht,,.}{in de dood en keen telkens}{terug trouw te trouw}\\

\haiku{Zij willen immers,.}{breken nedersmakken en}{het diepste treffen}\\

\haiku{Het is alleen maar}{onbegrijpelijk dat er}{nog zovelen zijn}\\

\haiku{Ik vergat dat het.}{niet eender en niet even lang}{is voor een ieder}\\

\haiku{Met Raoul zijn vrees,....}{te delen zijn bezorgdheid}{voor het kind misschien}\\

\haiku{Ik mag blij zijn dat,.}{ik dienen kan daar waar die}{dienst nog wordt aanvaard}\\

\haiku{Niet langer verwringt.}{de succubus van hete}{begeerte hun vorm}\\

\haiku{Wel verliest het door.}{te steken en te snijden}{van zijn scherpte veel}\\

\subsection{Uit: Leef duizend levens (onder de naam Lou Lichtveld)}

\haiku{Hij is opgejaagd:}{als een rijwielkampioen}{en heeft slechts \'e\'en zorg}\\

\haiku{Natuurlijk alleen,.}{van het bizondere het}{belangwekkende}\\

\haiku{Een ander houdt van.}{haar en is aanstonds bereid}{voor haar te sterven}\\

\haiku{hij leeft niet geheel,.}{van eigen willekeur niet}{geheel autonoom}\\

\haiku{het liefdeleven,!}{de confrontatie met den}{erotischen partner}\\

\haiku{Gelijk de duivel.}{door zijn inblazingen en}{bedriegerijen}\\

\haiku{De meeste daden,;}{blijken heel anders te zijn}{dan wat ze lijken}\\

\haiku{{\textquoteleft}Allons, enfants de,{\textquoteright} {\textquoteleft}{\textquoteright}.}{la patrie of kan hetjij}{een veelvoud worden}\\

\haiku{De oude, ruwe;}{krijgsverhalen blijven voor}{het gewone volk}\\

\haiku{Knut de Snijder bleef,.}{lange tijd te Austegaard}{en werd een oud man}\\

\haiku{{\textquoteright} Moge een enkel.}{voorbeeld verduidelijken}{wat er bedoeld wordt}\\

\haiku{Het verhaal ontrolt, {\textquoteleft}.}{zich voor ons het is onder}{het lezenwordend}\\

\haiku{Want in het snelste.}{geval verhaalt men van het}{zojuist gebeurde}\\

\haiku{Niets is minder waar, {\textquoteleft},{\textquoteright}:}{zolang alle kunst slechts een}{herscheppen dat is}\\

\haiku{Wij worden attent,.}{gemaakt en de contr\^ole kan}{achteraf komen}\\

\haiku{zij verblinden niet,,,.}{zij winnen stap voor stap boek}{voor boek jaar na jaar}\\

\haiku{Beschouwing is een.}{confrontatie van het Ik}{met het Andere}\\

\haiku{Het kunstwerk wordt uit.}{een wisselwerking tussen}{beide geboren}\\

\haiku{Dit bewegende,,:}{voortstuwende zijn wij al}{tegengekomen}\\

\haiku{De fabel toont hem,.}{aan welk standpunt dat van den}{romanschrijver is}\\

\haiku{De oplossing der:}{opgeworpen problemen}{kan dan alleen zijn}\\

\haiku{hoeveel jaren haar,.}{nog vergund waren en het}{doet er ook niet toe}\\

\haiku{Het beantwoordt slechts - -.}{zoals iedereen weet aan}{een wensfantasie}\\

\haiku{Draagt het geen kiemen?}{in zich die het maar al te}{snel vernietigen}\\

\haiku{Hij was gekleed in;}{een flesgroene jas met een}{zwart-fluwelen kraag}\\

\haiku{en eindelijk in,.}{een afgelegen streek waar}{hij wederom faalt}\\

\haiku{{\textquoteleft}U beweert dus in?}{volle ernst een zelfstandig}{karakter te zijn}\\

\haiku{Het leesgezelschap{\textquoteright}.}{van Diepenbeek van P. van}{Limburg Brouwer voert}\\

\haiku{Sarcastisch zelfs geeft:}{hij midden in het boek van}{Mathilde toe}\\

\haiku{Zijn bedoeling blijkt,:}{duidelijk wanneer hij iets}{verder doorgaat}\\

\haiku{In een geval als {\textquoteleft}{\textquoteright}:}{Dostojewski'sSchuld en boete is}{voor hem de vraag niet}\\

\haiku{Van dezen laatste,:}{uit gezien moet het vraagstuk}{zo vertaald worden}\\

\haiku{Zulk een idee is op.}{zijn best een vaststelling en}{op zijn minst een vraag}\\

\haiku{Het ware, althans,.}{verifieerbare komt}{nu op de voorgrond}\\

\haiku{Bij ons maakte Van {\textquoteleft}.}{Lennep een aanvang door zijn}{Klaasje Zevenster}\\

\haiku{Stevenson in {\textquoteleft}Dr.{\textquoteright} {\textquoteleft}}{Jekyll and Mr. Hyde of}{Oscar Wilde in}\\

\haiku{The picture of{\textquoteright}.}{Dorian Gray op een veel}{verfijnder manier}\\

\haiku{Een klassiek voorbeeld {\textquoteleft}{\textquoteright}.}{daarvan hebben wij inUncle}{Tom's cabin van mrs}\\

\haiku{{\textquoteleft}Nederigheid is.}{niet de overheersende deugd}{der romanschrijvers}\\

\haiku{Zij zijn niet bevreesd.}{aanspraak te maken op de}{titel van scheppers}\\

\haiku{In dit opzicht is;}{het aesthetische echter}{onproblematisch}\\

\haiku{) Ongelukkigen,.}{of vervolgden te lijf gaan}{De zeden kwetsen}\\

\haiku{Ons 20ste-eeuws standpunt.}{is heel wat eenvoudiger}{en radicaler}\\

\haiku{Sommige tergen,,;}{achtervolgen ons laten}{de geest niet met rust}\\

\haiku{En hun oeuvre werkt, {\textquoteleft}.}{nog steeds na zo goed als dat}{van deklassieken}\\

\haiku{Degene die de {\textquoteleft}{\textquoteright} {\textquoteleft}{\textquoteright},.}{vraag en dus deprijs bepaalt}{dat is de lezer}\\

\haiku{Literatuur ter:}{verdere ori\"entatie}{Algemeen}\\

\haiku{Balzac 13, 53, 64,,,,,,,,,,,.}{65 128 149 v 267 294 296}{297 303 308 310 315}\\

\haiku{Boccaccio 79 v,,,,,,,,,,.}{81 v 158 v 161 255 269}{295 323 333 334 341}\\

\haiku{Mann (Thomas) 56, 59,,,,,,,,.}{123 191 208 222 227 262 297}{310 Manzoni 55}\\

\subsection{Uit: De medeminnaars}

\haiku{Een ogenblik stond hij.}{aarzelend te draaien op}{zijn lange benen}\\

\haiku{Toen, reeds achter de,:}{stoel van zijn moeder even naar}{haar overgebogen}\\

\haiku{hij paste niet meer.}{in de schoolbanken met zijn}{groot en hoekig lijf}\\

\haiku{Misschien zou ze hem,,.}{als hij haar weerzag niet eens}{meer willen kennen}\\

\haiku{Zie ik er uit als...{\textquoteright} {\textquoteleft},}{een van die vrouwen dieZo}{bedoel ik het niet}\\

\haiku{{\textquoteright} {\textquoteleft}Ik ben misschien niet,{\textquoteright}.}{zo ervaren bekende}{Joachim nederig}\\

\haiku{{\textquoteleft}Ken je dat grote?}{caf\'e op de hoek van de}{Markt en de Bredestraat}\\

\haiku{Ze had de afspraak.}{toch stellig niet gemaakt om}{van hem af te zijn}\\

\haiku{Een schok gaf hem de.}{ontdekking dat ze nu toch}{gekomen was}\\

\haiku{{\textquoteright} Carla zei niets meer,.}{hij rekende af en zij}{gingen de straat op}\\

\haiku{{\textquoteright} Met wijsneuzige:}{nadrukkelijkheid begon}{Joachim te oreren}\\

\haiku{Maar beloof me dan,....}{dat je mij een berichtje}{stuurt met een afspraak}\\

\haiku{In zijn handpalmen.}{voelde hij nog de warmte}{van haar polsen na}\\

\haiku{Zoals nu, hier voor,.}{zijn vader die hoorbaar aan}{zijn sigaar pufte}\\

\haiku{Maar hij had bergen,.}{werk te verzetten want zijn}{achterstand was groot}\\

\haiku{Zijn moeder kon hem,}{weer horen fluiten wanneer}{hij door het huis liep}\\

\haiku{Hoewel de leraar,,.}{er oudergewoonte geen}{acht op scheen te slaan}\\

\haiku{Daarna verdween ze,.}{achter de bomen waar hij}{haar niet meer kon zien}\\

\haiku{Maar hij was bezig,.}{zijn doel te bereiken weer}{een stap dichterbij}\\

\haiku{Ze spande zich in,.}{een ladder uit een zijden}{kous op te halen}\\

\haiku{maar wat wist het park?}{zelf van deze vluchtige}{wandelaarster af}\\

\haiku{een uitgetrokken.}{kledingstuk dat zij onder}{zijn ogen hanteerde}\\

\haiku{Het is dwaasheid niet.}{roekeloos te nemen wat}{het leven ons geeft}\\

\haiku{met een vinger op.}{de eigen mond beduidde}{ze hem te zwijgen}\\

\haiku{Een zacht fluitje klonk,.}{als een signaal waarmee men}{een bekende roept}\\

\haiku{Onmogelijk, - bij;}{zulke vrouwen ging alles}{immers anders toe}\\

\haiku{Duidelijk kon hij:}{de driftige woorden van}{zijn vader verstaan}\\

\haiku{Stel je eens voor, dat,:}{het Carla geweest was die}{hem had opgebeld}\\

\haiku{Als het tenminste.}{een vrouw was geweest en geen}{ondergeschikte}\\

\haiku{Eventueel,{\textquoteright} hernam, {\textquoteleft}.}{hijzou je bij mij in de}{zaak kunnen komen}\\

\haiku{Maar iets, iemand, een.}{gevaar verhinderde haar}{misschien te roepen}\\

\haiku{De duisternis viel,.}{reeds in toen Joachim bij het}{huis kwam aangefietst}\\

\haiku{{\textquoteright} {\textquoteleft}Het is geen zaak, het,{\textquoteright}.}{is iets doodonschuldigs zei}{Joachim met nadruk}\\

\haiku{Je wist toch heel goed.}{dat dit geen boodschap aan een}{kleine jongen was}\\

\haiku{{\textquoteright} {\textquoteleft}En jij bent zeker?}{in het geheel niet vals en}{niet bedriegelijk}\\

\haiku{de zorg om jou, om.}{wat je misschien belet heeft}{je woord te houden}\\

\haiku{{\textquoteleft}Laten wij elkaar,.}{niets vragen ieder zwijgen}{over zijn geheimen}\\

\haiku{{\textquoteright} Joachim trok haar hoofd.}{naar zich toe en legde zijn}{arm om haar schouder}\\

\haiku{{\textquoteleft}Ik laat me in met;}{een kinderachtigheid die}{me niet helpen kan}\\

\haiku{Zwijgend en met een.}{mismoedig gebaar gaf hij}{haar de brief terug}\\

\haiku{Het voertuig kwam gauw,.}{genoeg hij stapte in en}{noemde Carla's adres}\\

\haiku{{\textquoteleft}Het is goed dat het,{\textquoteright}.}{binnenkort afgelopen}{is zei de moeder}\\

\haiku{Wat moet ik dan doen?}{om mij nu voorgoed van haar}{te verzekeren}\\

\haiku{Deed ze tegenover?}{anderen misschien net zo}{als tegenover hem}\\

\haiku{een tijd van lange.}{zoete vrede zal misschien}{voor ons beginnen}\\

\haiku{Ze wisten, hoewel,.}{huisgenoten zo weinig}{van elkander af}\\

\haiku{Hij moest haar tonen,,.}{hoezeer hij dit samenzijn}{ook hier waardeerde}\\

\haiku{Vooral de oude,{\textquoteright}.}{heer mag hiervan niets weten}{vermaande Willem}\\

\haiku{Als een dorstige;}{temidden van de oceaan}{gevoelde hij zich}\\

\haiku{Velen woonden hier,.}{zoals de wereld ook door}{velen werd bewoond}\\

\haiku{{\textquoteright} en aanstonds daarop,:}{met  een gebaar naar zijn}{boeket vervolgde}\\

\haiku{Kroner's vakken in,.}{het bizonder als die hem}{zou willen treffen}\\

\haiku{Maar was het niet voor,?}{haar eigen bestwil om haar}{te kunnen helpen}\\

\haiku{In plaats van zich te.}{ontwarren werd haar raadsel}{ingewikkelder}\\

\haiku{Er viel voor Joachim.}{niets beters te doen dan hun}{voorbeeld te volgen}\\

\haiku{Die had dus Carla,...}{weggebracht en met een soort}{vanzelfsprekendheid}\\

\haiku{Maar zijn voorgevoel,.}{bevestigde hem dat ze}{ditmaal er zou zijn}\\

\haiku{{\textquoteleft}Maar je weet toch, het...}{ontbrak mij in de laatste}{tijd aan een bedrag}\\

\haiku{{\textquoteright} {\textquoteleft}Een vrouw als jij, met...?}{zoveel vrienden Zijn er dan}{geen rijke onder}\\

\haiku{{\textquoteright} Mainteneren wou,}{hij zeggen maar dat slikte}{hij natuurlijk in}\\

\haiku{{\textquoteleft}Niet bizonder,{\textquoteright} was.}{zijn antwoord op de vraag hoe}{hij het gemaakt had}\\

\haiku{Leo Dekking die het,.}{examen nog voor de boeg had}{kwam al op hem af}\\

\haiku{Geef me dan maar de.}{genadeslag en laat het}{afgelopen zijn}\\

\haiku{{\textquoteleft}Ik zie dat je al,,{\textquoteright}.}{begrijpt waar het op uitdraait}{ging de rector voort}\\

\haiku{Dit, wat erger was,.}{dan ronduit zakken weten}{waar je aan toe was}\\

\haiku{Van twee kwaden is,{\textquoteright}.}{dit nog het minste zei zijn}{vader laconiek}\\

\haiku{Een dergelijke.}{bezorgdheid was hij van zijn}{vader niet gewend}\\

\haiku{Twee maanden zijn een,,}{hele boel tijd als je die}{goed gebruikt Warden}\\

\haiku{{\textquoteleft}En je moet het doen,.}{omdat de eer van de school}{ermee gemoeid is}\\

\haiku{En deze had hem.}{bijna gesoebat om het}{herexamen te doen}\\

\haiku{Hij had de grootste.}{moeite om zijn opwinding}{te onderdrukken}\\

\haiku{Er zit niets anders,.}{op dan dat ik inderdaad}{het herexamen doe}\\

\haiku{Hij had geen recht van,.}{spreken meer wanneer hij er}{geen gevolg aan gaf}\\

\haiku{In ieder geval,}{zit je hier beter dan in}{zo'n kazerneflat}\\

\haiku{Deze ontmoeting.}{moet nu maar over ons beider}{leven beslissen}\\

\haiku{Hoe kon hij er \'e\'en?}{moment over gedacht hebben}{haar los te laten}\\

\haiku{De man die naar de.}{hond floot op het trottoir leek}{op meneer Rocquet}\\

\haiku{Practisch werken en,{\textquoteright}.}{alvast wat verdienen had}{Joachim geantwoord}\\

\haiku{In zijn dooie eentje.}{zou hij er zelfs van kunnen}{leven als het moest}\\

\haiku{Hoewel hij tevens,.}{blij was dat althans Carla}{niet ter sprake kwam}\\

\haiku{Hij liet zijn fiets maar,.}{staan en riep een taxi als een}{echte zakenman}\\

\haiku{Wie weet hoeveel er?}{waren en hoe weinigen}{slechts werden betrapt}\\

\haiku{Carla streek Joachim,.}{over zijn haren met een haast}{moederlijk gebaar}\\

\haiku{Hoe staat het met je?}{nieuwe functie op kantoor}{bij de oude heer}\\

\haiku{Er kwamen immers.}{genoeg aantrekkelijke}{langs of in hun buurt}\\

\haiku{Maar dat moet ik nog,{\textquoteright}.}{tegenkomen vertelde}{Willem goedgemutst}\\

\haiku{Dat heeft hij zeker;}{aan de een of andere}{zakenvriend geleend}\\

\haiku{Leo moest eens weten,,,... {\textquoteleft}}{hoe hij Joachim aan deze}{dingen was ontgroeid}\\

\haiku{Hoe had hij ook maar!}{\'e\'en ogenblik kunnen denken}{dat het Carla was}\\

\haiku{Nu zonk het terug.}{tot teleurstelling die op}{zwaarmoedigheid leek}\\

\haiku{het oude ideaal.}{dat zo heftig door al zijn}{zinnen verlangd werd}\\

\haiku{{\textquoteright} En hij loog erbij,,:}{hij wist zelf niet waarvoor het}{kwam automatisch}\\

\haiku{Maar nu is het mijn,.}{allerlaatste souvenir}{geworden weet je}\\

\haiku{de minnaars bleven.}{toch achter en zouden haar}{gauw doen vergeten}\\

\haiku{{\textquoteleft}Een daarvan is een,.}{van mijn leraren een van}{mijn examinators}\\

\haiku{Hij was er zelfs bij.}{toen ze mij zo ongeveer}{de deur uit zette}\\

\haiku{En als hij weet dat,.}{je erop verdacht bent zal}{hij het wel laten}\\

\haiku{Moeilijker dan voor?}{u om u aan het examen}{te onderwerpen}\\

\haiku{Niet dat de tijd die,.}{hij aldus won hem nog veel}{zou kunnen baten}\\

\haiku{Hij ging zitten op.}{een van de stoeltjes die nog}{steeds op hen wachtten}\\

\haiku{Zich ontworsteld had.}{aan Kroner's greep en aan de}{greep van iedereen}\\

\haiku{Of hij schoot haar neer,.}{en daarna ook zichzelf een}{kogel door de kop}\\

\haiku{En eerst misschien nog.}{wie zich tussen haar en hem}{zou willen dringen}\\

\haiku{Want dit was Carla,.}{maar toch niet de vrouw voor wie}{hij was gekomen}\\

\haiku{Zijn trage, vlakke,:}{stem zei toen de jongeman}{hem vragend aankeek}\\

\haiku{had Joachim zijn stem,?}{hier niet al eens gehoord door}{alle wanden heen}\\

\haiku{{\textquoteleft}Schiet maar op,{\textquoteright} zei de,.}{inspecteur nog zonder zelfs}{een zweem van boosheid}\\

\subsection{Uit: De medeminnaars}

\haiku{Toen, reeds achter de,:}{stoel van zijn moeder even naar}{haar overgebogen}\\

\haiku{hij paste niet meer.}{in de schoolbanken met zijn}{groot en hoekig lijf}\\

\haiku{Misschien zou ze hem,,.}{als hij haar weerzag niet eens}{meer willen kennen}\\

\haiku{Zie ik er uit als...{\textquoteright} {\textquoteleft},}{een van die vrouwen dieZo}{bedoel ik het niet}\\

\haiku{{\textquoteright} {\textquoteleft}Ik ben misschien niet,{\textquoteright}.}{zo ervaren bekende}{Joachim nederig}\\

\haiku{{\textquoteleft}Ken je dat grote?}{caf\'e op de hoek van de}{Markt en de Bredestraat}\\

\haiku{Ze had de afspraak.}{toch stellig niet gemaakt om}{van hem af te zijn}\\

\haiku{{\textquoteright} Carla zei niets meer,.}{hij rekende af en zij}{gingen de straat op}\\

\haiku{{\textquoteright} Met wijsneuzige:}{nadrukkelijkheid begon}{Joachim te oreren}\\

\haiku{Maar beloof me dan,....}{dat je mij een berichtje}{stuurt met een afspraak}\\

\haiku{In zijn handpalmen.}{voelde hij nog de warmte}{van haar polsen na}\\

\haiku{Zoals nu, hier voor,.}{zijn vader die hoorbaar aan}{zijn sigaar pufte}\\

\haiku{Maar hij had bergen,.}{werk te verzetten want zijn}{achterstand was groot}\\

\haiku{Zijn moeder kon hem,}{weer horen fluiten wanneer}{hij door het huis liep}\\

\haiku{Hoewel de leraar,,.}{er oudergewoonte geen}{acht op scheen te slaan}\\

\haiku{Daarna verdween ze,.}{achter de bomen waar hij}{haar niet meer kon zien}\\

\haiku{Maar hij was bezig,.}{zijn doel te bereiken weer}{een stap dichterbij}\\

\haiku{Ze spande zich in,.}{een ladder uit een zijden}{kous op te halen}\\

\haiku{maar wat wist het park?}{zelf van deze vluchtige}{wandelaarster af}\\

\haiku{een uitgetrokken.}{kledingstuk dat zij onder}{zijn ogen hanteerde}\\

\haiku{Het is dwaasheid niet.}{roekeloos te nemen wat}{het leven ons geeft}\\

\haiku{met een vinger op.}{de eigen mond beduidde}{ze hem te zwijgen}\\

\haiku{Een zacht fluitje klonk,.}{als een signaal waarmee men}{een bekende roept}\\

\haiku{Onmogelijk, - bij;}{zulke vrouwen ging alles}{immers anders toe}\\

\haiku{Duidelijk kon hij:}{de driftige woorden van}{zijn vader verstaan}\\

\haiku{Stel je eens voor, dat,:}{het Carla geweest was die}{hem had opgebeld}\\

\haiku{Als het tenminste.}{een vrouw was geweest en geen}{ondergeschikte}\\

\haiku{Eventueel,{\textquoteright} hernam, {\textquoteleft}.}{hijzou je bij mij in de}{zaak kunnen komen}\\

\haiku{Maar iets, iemand, een.}{gevaar verhinderde haar}{misschien te roepen}\\

\haiku{{\textquoteright} {\textquoteleft}Het is geen zaak, het,{\textquoteright}.}{is iets doodonschuldigs zei}{Joachim met nadruk}\\

\haiku{je wist toch heel goed?}{wat je deed toen je mij schreef}{dat ik kon komen}\\

\haiku{Je wist toch heel goed.}{dat dit geen boodschap aan een}{kleine jongen was}\\

\haiku{de zorg om jou, om.}{wat je misschien belet heeft}{je woord te houden}\\

\haiku{{\textquoteleft}Laten we elkaar,.}{niets vragen ieder zwijgen}{over zijn geheimen}\\

\haiku{{\textquoteright} Joachim trok haar hoofd.}{naar zich toe en legde zijn}{arm om haar schouder}\\

\haiku{{\textquoteleft}Ik laat me in met;}{een kinderachtigheid die}{me niet helpen kan}\\

\haiku{Zwijgend en met een.}{mismoedig gebaar gaf hij}{haar de brief terug}\\

\haiku{Het voertuig kwam gauw,.}{genoeg hij stapte in en}{noemde Carla's adres}\\

\haiku{Hij behoefde de,.}{brief  slechts open te maken}{en deed het dan ook}\\

\haiku{{\textquoteleft}Het is goed dat het,{\textquoteright}.}{binnenkort afgelopen}{is zei de moeder}\\

\haiku{Wat moet ik dan doen?}{om mij nu voorgoed van haar}{te verzekeren}\\

\haiku{Deed ze tegenover?}{anderen misschien net zo}{als tegenover hem}\\

\haiku{een tijd van lange.}{zoete vrede zal misschien}{voor ons beginnen}\\

\haiku{5 Met een spinsel;}{van leugens had hij thuis zijn}{wegblijven verklaard}\\

\haiku{Ze wisten, hoewel,.}{huisgenoten zo weinig}{van elkander af}\\

\haiku{Hij moest haar tonen,,.}{hoezeer hij dit samenzijn}{ook hier waardeerde}\\

\haiku{Vooral de oude,{\textquoteright}.}{heer mag hiervan niets weten}{vermaande Willem}\\

\haiku{Als een dorstige;}{temidden van de oceaan}{gevoelde hij zich}\\

\haiku{Velen woonden hier,.}{zoals de wereld ook door}{velen werd bewoond}\\

\haiku{Kroner's vakken in,.}{het bijzonder als die hem}{zou willen treffen}\\

\haiku{Maar was het niet voor,?}{haar eigen bestwil om haar}{te kunnen helpen}\\

\haiku{In plaats van zich te.}{ontwarren werd haar raadsel}{ingewikkelder}\\

\haiku{Er viel voor Joachim.}{niets beters te doen dan hun}{voorbeeld te volgen}\\

\haiku{Die had dus Carla,...}{weggebracht en met een soort}{vanzelfsprekendheid}\\

\haiku{Maar zijn voorgevoel,.}{bevestigde hem dat ze}{ditmaal er zou zijn}\\

\haiku{{\textquoteleft}Maar je weet toch, het...}{ontbrak mij in de laatste}{tijd aan een bedrag}\\

\haiku{{\textquoteright} {\textquoteleft}Een vrouw als jij, met...?}{zoveel vrienden Zijn er dan}{geen rijke onder}\\

\haiku{{\textquoteright} Mainteneren wou,}{hij zeggen maar dat slikte}{hij natuurlijk in}\\

\haiku{Niet bijzonder,{\textquoteright} was.}{zijn antwoord op de vraag hoe}{hij het gemaakt had}\\

\haiku{Leo Dekking die het,.}{examen nog voor de boeg had}{kwam al op hem af}\\

\haiku{Geef me dan maar de.}{genadeslag en laat het}{afgelopen zijn}\\

\haiku{{\textquoteleft}Ik zie dat je al,,{\textquoteright}.}{begrijpt waar het op uitdraait}{ging de rector voort}\\

\haiku{Dit, wat erger was,.}{dan ronduit zakken weten}{waar je aan toe was}\\

\haiku{Van twee kwaden is,{\textquoteright}.}{dit nog het minste zei zijn}{vader laconiek}\\

\haiku{Een dergelijke.}{bezorgdheid was hij van zijn}{vader niet gewend}\\

\haiku{{\textquoteleft}En je moet het doen,.}{omdat de eer van de school}{ermee gemoeid is}\\

\haiku{En deze had hem.}{bijna gesoebat om het}{herexamen te doen}\\

\haiku{Hij had de grootste.}{moeite om zijn opwinding}{te onderdrukken}\\

\haiku{Er zit niets anders,.}{op dan dat ik inderdaad}{het herexamen doe}\\

\haiku{Hij had geen recht van,.}{spreken meer wanneer hij er}{geen gevolg aan gaf}\\

\haiku{In ieder geval,}{zit je hier beter dan in}{zo'n kazerneflat}\\

\haiku{Deze ontmoeting.}{moet nu maar over ons beider}{leven beslissen}\\

\haiku{Hoe kon hij er \'e\'en?}{moment over gedacht hebben}{haar los te laten}\\

\haiku{De man die naar de.}{hond floot op het trottoir leek}{op meneer Rocquet}\\

\haiku{Praktisch werken en,{\textquoteright}.}{alvast wat verdienen had}{Joachim geantwoord}\\

\haiku{In zijn dooie eentje.}{zou hij er zelfs van kunnen}{leven als het moest}\\

\haiku{Hoewel hij tevens,.}{blij was dat althans Carla}{niet ter sprake kwam}\\

\haiku{{\textquoteright} Hij wachtte even, keek,:}{of Joachim hem wel begreep}{en  ging toen voort}\\

\haiku{Hij liet zijn fiets maar,.}{staan en riep een taxi als een}{echte zakenman}\\

\haiku{Ik beloof je, dat,.}{je het over uiterlijk een}{dag of drie vier hebt}\\

\haiku{Wie weet hoeveel er?}{waren en hoe weinigen}{slechts werden betrapt}\\

\haiku{Hoe staat het met je?}{nieuwe functie op kantoor}{bij de oude heer}\\

\haiku{Er kwamen immers.}{genoeg aantrekkelijke}{langs of in hun buurt}\\

\haiku{Maar dat moet ik nog,{\textquoteright}.}{tegenkomen vertelde}{Willem goedgemutst}\\

\haiku{Dat heeft hij zeker;}{aan de een of andere}{zakenvriend geleend}\\

\haiku{Leo moest eens weten,,,... {\textquoteleft}}{hoe hij Joachim aan deze}{dingen was ontgroeid}\\

\haiku{Hoe had hij ook maar!}{\'e\'en ogenblik kunnen denken}{dat het Carla was}\\

\haiku{Nu zonk het terug.}{tot teleurstelling die op}{zwaarmoedigheid leek}\\

\haiku{het oude ideaal.}{dat zo heftig door al zijn}{zinnen verlangd werd}\\

\haiku{{\textquoteright} En hij loog erbij,,:}{hij wist zelf niet waarvoor het}{kwam automatisch}\\

\haiku{Maar nu is het mijn,.}{allerlaatste souvenir}{geworden weet je}\\

\haiku{de minnaars bleven.}{toch achter en zouden haar}{gauw doen vergeten}\\

\haiku{{\textquoteleft}Een daarvan is een,.}{van mijn leraren een van}{mijn examinators}\\

\haiku{Hij was er zelfs bij.}{toen ze mij zo ongeveer}{de deur uit zette}\\

\haiku{En als hij weet dat,.}{je erop verdacht bent zal}{hij het wel laten}\\

\haiku{Moeilijker dan voor?}{u om u aan het examen}{te onderwerpen}\\

\haiku{Niet dat de tijd die,.}{hij aldus won hem nog veel}{zou kunnen baten}\\

\haiku{Je ziet eruit als,.}{een geest en het kind is een}{schatje om te zien}\\

\haiku{Zich ontworsteld had.}{aan Kroners greep en aan de}{greep van iedereen}\\

\haiku{Of hij schoot haar neer,.}{en daarna ook zichzelf een}{kogel door de kop}\\

\haiku{En eerst misschien nog.}{wie zich tussen haar en hem}{zou willen dringen}\\

\haiku{Want dit was Carla,.}{maar toch niet de vrouw voor wie}{hij was gekomen}\\

\haiku{Zijn trage, vlakke,:}{stem zei toen de jongeman}{hem vragend aankeek}\\

\haiku{{\textquoteleft}Schiet maar op,{\textquoteright} zei de,.}{inspecteur nog zonder zelfs}{een zweem van boosheid}\\

\subsection{Uit: Mijn aap schreit. Het euvel gods}

\haiku{Twee dingen waren,:}{het die mij aangetrokken}{hadden in de aap}\\

\haiku{De aap sprong rond in.}{mijn huis alsof hij er al}{jaren geweest was}\\

\haiku{Tevens was ik blij.}{dat het probleem zichzelve}{uitgewezen had}\\

\haiku{Mijn aap heeft een paar;}{eikeltjes gestolen die}{langs de weg lagen}\\

\haiku{was hij al aanstonds,.}{goede maatjes met haar toen het}{meisje binnen kwam}\\

\haiku{Ik zette het raam,.}{open want de kamer werd mij}{te eng en benauwd}\\

\haiku{Ik schold op hem, maar,.}{sloeg hem niet meer want hij was}{nog zwak en mager}\\

\haiku{Je moet niet alleen,.}{de verschillen leren maar}{ook de overeenkomst}\\

\haiku{In geen geval moest;}{ze zich echter met het beest}{kunnen bemoeien}\\

\haiku{{\textquoteright} {\textquoteleft}O, het is zo erg,{\textquoteright},.}{niet zei ik en bond mijn pols}{af met m'n zakdoek}\\

\haiku{Direct toen het licht,,.}{aanging zag ik hem zitten}{in zijn mand mijn aap}\\

\haiku{Stijf en koud ben ik.}{naar boven gegaan toen het}{haast ochtend was}\\

\haiku{de nieuwsgierigheid:}{die we allen hebben voor}{het einde van iets}\\

\haiku{Reeds enige dagen.}{was hij gehuwd en woonde}{hij in het paleis}\\

\haiku{Nauwelijks was hij,}{bij de vijver gekomen}{of hij wierp zo snel}\\

\haiku{Dan hief hij be{\^\i} zijn:}{armen op en begon een}{lied te zingen}\\

\haiku{{\textquoteright} De jager zette.}{zijn hoed recht en kruiste de}{armen over z'n borst}\\

\haiku{Tot zo laat was er;}{tussen al Gods dagen geen}{verschil voor Jukkers}\\

\haiku{Het eten was van avond,...}{slecht maar uwe liefkozingen}{zijn al mijn voedsel}\\

\haiku{anders was zoveel.}{leugen en bedrog immers}{niet nodig geweest}\\

\haiku{Spranger belde mij....}{op voor een spoedoperatie}{voor Arthur Christensen}\\

\haiku{De ander trachtte,.}{hem te kalmeren maar hij}{wond zich steeds meer op}\\

\haiku{Als mensen uit een.}{andere wereld staarden}{wij elkander aan}\\

\haiku{In ieder geval,{\textquoteright}.}{was het beestje omnivoor}{stelde Fowler vast}\\

\haiku{Het brak tussen mijn,.}{tanden en ik bekeek het}{stuk dat ik vasthield}\\

\haiku{Ik zag hoe Fowler,.}{kokhalsde en was niet in}{staat te antwoorden}\\

\haiku{Vermoeid was zij nu,.}{na drie jaren geworden}{ten dode vermoeid}\\

\haiku{Alles was van een.}{uiterste zindelijkheid}{en scheen te stralen}\\

\haiku{Vragen durfde hij, ',:}{niet want wat hijt eerst zou}{willen weten was}\\

\haiku{Ik knikte, nam een:}{slok van het onsmakelijk}{mengsel  en zei}\\

\haiku{Maar de oude vrouw.}{trok me aan de mouw mee naar}{een andere reet}\\

\haiku{{\textquoteleft}Wij moeten nu de,.}{andere kant uit want het}{laatste huis is leeg}\\

\haiku{Er zijn misdaden,.}{zo verschrikkelijk dat ze}{geen naam meer hebben}\\

\haiku{Bijna is het dag,{\textquoteright},.}{zegt hij naar buiten stappend}{waar de paarden staan}\\

\haiku{De noordelijkste!}{kaart van pater Martinus}{was onnauwkeurig}\\

\haiku{Het jongetje en.}{het meisje aan haar zijde}{zien nieuwsgierig rond}\\

\haiku{{\textquoteright} Maar slap en willoos.}{liet ze zich doorgolven en}{Cobbs verloor de moed}\\

\haiku{Waarom heb ik ook,.}{geen moeder fluisterde het}{binnenste van Cobbs}\\

\haiku{Want hij geloofde,.}{niet meer omdat er niets meer}{te geloven viel}\\

\subsection{Uit: Mijn aap lacht}

\haiku{Iedereen erkent.}{van sommige gevoelens}{de verhevenheid}\\

\haiku{Je leert ook al de.}{geluiden in de verte}{duidelijk verstaan}\\

\haiku{Bijen, wespen en.}{veel andere insekten}{zoemden er omheen}\\

\haiku{Onderwijl kneep ons,.}{de honger aangewakkerd}{door de lange tocht}\\

\haiku{V Diepdonkere.}{nacht stond geurig om mij heen}{toen ik weer bijkwam}\\

\haiku{lang zaten wij in}{het struikgewas waaronder}{onophoudelijk}\\

\haiku{Alleen dat \'ene was,:}{er nog dat \'ene dat ik moest}{en zou bereiken}\\

\haiku{De hele wereld.}{droop en uit de kruin lekten}{nog lange beken}\\

\haiku{Uit de moeder steeg,.}{een rauwe kreet op maar ze}{bood geen tegenstand}\\

\haiku{maar dan zijn ze snel,.}{en haastig gaan en komen}{botsend langs elkaar}\\

\haiku{, en waar ik tastte,.}{voelde ik koud gesteente}{gepolijste rots}\\

\haiku{Het was een hele,}{opgaaf waaraan ik mij toch}{niet kon onttrekken}\\

\haiku{Want als het nieuwtje,.}{eraf is komt de hele}{last toch op mij neer}\\

\haiku{{\textquoteright} Zijn aard was toch wat.}{minder ruw dan zijn woorden}{gewoonlijk klonken}\\

\haiku{{\textquoteright} Toen Mauriesje het,.}{zag werd hij boos en duwde}{de jongen opzij}\\

\haiku{{\textquoteleft}Je houdt je koest, hoor, '.}{anders draai ik je je nek}{om voor jet weet}\\

\haiku{Allemaal zijn wij,,.}{verschillend dat is waar maar}{willen  vriendschap}\\

\haiku{Want ze was niet bleek,;}{maar juist erg zwart van haar en}{van gezicht en ogen}\\

\haiku{Veeg je poten af,.}{en salueer tenminste}{voordat je vertrekt}\\

\haiku{{\textquoteleft}Je zult hem niet veel,.}{pleziertjes meer kunnen doen}{mijn beste mensen}\\

\haiku{{\textquoteright} Louise was, de vuist,.}{tegen haar mond gedrukt de}{kamer uit gevlucht}\\

\haiku{Zo kwamen zij al.}{smoezend tot onder aan de}{dakrand waar ik zat}\\

\haiku{En alsof er niets,.}{gebeurd was spraken zij van}{andere dingen}\\

\haiku{Toen zij haar uitliet,,.}{lette mevrouw Hudson op}{dat ik niet meeliep}\\

\haiku{iemand die mijn ziel,.}{begreep en ook al was hij}{ver dicht bij me stond}\\

\haiku{de tijden zijn er,.}{toch niet op vooruitgegaan}{de blanken ook niet}\\

\haiku{Al was het ook een,.}{krokodil geweest ze had}{hem aangevallen}\\

\haiku{Ik heb mijn adem in.}{uw borst geblazen en u}{met mijn geest verwekt}\\

\haiku{terwijl ze even zweeg,,.}{zoals ze vaker deed ook}{als ze alleen was}\\

\haiku{Zo'n schepsel denkt dat.}{er wel wat te halen valt}{of weet-ik-wat}\\

\haiku{Dacht je soms dat je,?}{de enige meid was die een}{kind  moest krijgen}\\

\haiku{Ze scheen hem iets toe,:}{te fluisteren want Winters}{antwoordde opeens}\\

\haiku{En nu gebeurde.}{iets waarop ik allerminst}{verdacht geweest was}\\

\haiku{{\textquoteright} Hetgeen mij niet veel,.}{hoop gaf haar bedrieglijke}{natuur kennende}\\

\haiku{{\textquoteleft}Ik heb in huis te,.}{veel oud porselein staan om}{het al te wagen}\\

\haiku{Het was vroeg in de,.}{morgen een troepje jongens}{ging op weg naar school}\\

\haiku{Terwijl ze met haar,:}{ene vrije hand mijn kooi opendeed}{riep ze bijna hees}\\

\haiku{Daar zat stellig heel,.}{wat achter dat ik moest zien}{uit te  vorsen}\\

\haiku{Maar toch, kijk eens naar!}{het schuldbesef van deze}{jeugdige zondaar}\\

\haiku{u is de priester,{\textquoteright}.}{antwoordde juffrouw Elfriede}{maar half tevreden}\\

\haiku{{\textquoteright} {\textquoteleft}Als een beest... als een,{\textquoteright}.}{beest kreunde Rosalina}{zonder op te staan}\\

\haiku{Planten leken ze,,.}{mij toe op weg om dieren}{te worden meer niet}\\

\haiku{Als je nog ziek mocht,.}{blijven kom ik je stellig}{weer eens opzoeken}\\

\haiku{{\textquoteright} Waarop ze weer een:}{keel opzette en opnieuw}{begon te roepen}\\

\haiku{Ze gaf een schamper,.}{lachje terwijl Preiselbeer}{op haar toe stapte}\\

\haiku{Ik naar binnen, door.}{de opening boven een van}{de lage deurtjes}\\

\haiku{Ik had moeite mij.}{vast te houden en voelde}{een flauwte nabij}\\

\haiku{weg van de mijnen;}{onder wie ik vreedzaam en}{gelukkig leefde}\\

\haiku{En zo vergeten;}{zij rechtuit te denken en}{rechtuit te voelen}\\

\haiku{In de verte klonk,,,.}{twee- driemaal ook gefluit}{maar anders korter}\\

\haiku{Nu wist ik pas wat,.}{leedvermaak wat plezier om}{andermans pijn was}\\

\haiku{En het was alsof.}{de bomen zelf zich rekten}{met een lui gebaar}\\

\haiku{Nooit had ik het bos,.}{zo mooi gezien als nu bij}{deze wederkeer}\\

\haiku{II Als een grote.}{blanke spiegel lag het meer}{tussen de bomen}\\

\haiku{Kwamen andere?}{dieren net als ik ooit tot}{eenzelfde besef}\\

\haiku{Een jong volk op zoek,.}{naar taken kon nergens meer}{voor terugschrikken}\\

\haiku{Iets moest ik doen, iets.}{geweldigs en groots om de}{troep te bezweren}\\

\haiku{Hoe smadelijk was.}{ik door mijn Mandrillen in}{de steek gelaten}\\

\haiku{Mijn hele leven,.}{was een droom geweest als dit}{hier ook een droom was}\\

\haiku{Heb je dan niet om,?}{je heen gekeken in woud}{en steppe warhoofd}\\

\haiku{Want even zeker als,}{ik die in deze laatste}{dag van je bestaan}\\

\subsection{Uit: Omnibus}

\haiku{Ziehier de fijne,,;}{schriele sprinkhaan het voedsel}{van Sint Jan Baptist}\\

\haiku{Twee dingen waren,:}{het die mij aangetrokken}{hadden in de aap}\\

\haiku{De aap sprong rond in.}{mijn huis alsof hij er al}{jaren geweest was}\\

\haiku{Tevens was ik blij.}{dat het probleem zichzelve}{uitgewezen had}\\

\haiku{Mijn aap heeft een paar;}{eikeltjes gestolen die}{langs de weg lagen}\\

\haiku{was hij al aanstonds,.}{goede maatjes met haar toen het}{meisje binnenkwam}\\

\haiku{Ik zette het raam,.}{open want de kamer werd mij}{te eng en benauwd}\\

\haiku{Ik schold op hem, maar,.}{sloeg hem niet meer want hij was}{nog zwak en mager}\\

\haiku{Je moet niet alleen,.}{de verschillen leren maar}{ook de overeenkomst}\\

\haiku{In geen geval moest;}{ze zich echter met het beest}{kunnen bemoeien}\\

\haiku{{\textquoteright} {\textquoteleft}O, het is zo erg{\textquoteright},.}{niet zei ik en bond mijn pols}{af met m'n zakdoek}\\

\haiku{Direct toen het licht,,.}{aanging zag ik hem zitten}{in zijn mand mijn aap}\\

\haiku{Stijf en koud ben ik.}{naar boven gegaan toen het}{haast ochtend was}\\

\haiku{XVIII Aan het eind;}{van alle overpeinzingen}{stond mijn besluit vast}\\

\haiku{de nieuwsgierigheid:}{die we allen hebben voor}{het einde van iets}\\

\haiku{Reeds enige dagen.}{was hij gehuwd en woonde}{hij in het paleis}\\

\haiku{Nauwelijks was hij,}{bij de vijver gekomen}{of hij wierp zo snel}\\

\haiku{Dan hief hij be{\^\i} zijn:}{armen op en begon een}{lied te zingen}\\

\haiku{{\textquoteright} De jager zette.}{zijn hoed recht en kruiste de}{armen over z'n borst}\\

\haiku{De gewoonheid van.}{haar kinderspel had niemand}{kunnen verbazen}\\

\haiku{In bed dacht zij dan,:}{nog dikwijls aan de pop en}{het was wonderlijk}\\

\haiku{Groucha antwoordde,:}{niet maar de student pakte}{de mand op en zei}\\

\haiku{Daar nam zij de pop,.}{die hard voelde en stijf van}{de oude kleren}\\

\haiku{des avonds stierf Groucha,,:}{glimlachend omdat zij zacht}{en ver nog hoorde}\\

\haiku{Toen vouwde hij de.}{blaadjes voorzichtig dicht en}{stak ze in zijn zak}\\

\haiku{{\textquoteright} {\textquoteleft}Onze plantages,{\textquoteright}.}{zijn meer Afrika dan je denkt}{sprak de oude man}\\

\haiku{De volgende dag.}{vroeg ik of Lindor met mij}{mee mocht gaan jagen}\\

\haiku{{\textquoteleft}dan kun je wel zes.}{dragers meenemen om ze}{naar huis te dragen}\\

\haiku{Eens had Wassilj aan.}{een man gevraagd of het nog}{ver was naar Moskou}\\

\haiku{Er vloeide geen bloed,,.}{maar de zwerm week achteruit}{verder en verder}\\

\haiku{hoe grijze nevels.}{hangen in een huis waar licht}{en liefde woonden}\\

\haiku{En terwijl Sunanda,:}{met haar hoofd in zijn schoot lag}{fluisterde zij zoet}\\

\haiku{In Alec's arm leunde,.}{Kesini zwaarder en vermoeid toen}{zij huiswaarts keerden}\\

\haiku{{\textquoteright} En zich omdraaiend,,.}{spuwde hij in een nis waar}{een ikoon moest hangen}\\

\haiku{Het leven dat je.}{bij een stervende bijna}{als een onrecht voelt}\\

\haiku{{\textquoteleft}Laten we hem naar,.}{het hospitaal brengen hij}{is bewusteloos}\\

\haiku{Hijgend wilde hij,.}{roepen maar zijn stem was slechts}{een schor gerochel}\\

\haiku{Mijn lippen raken,,,;}{u Ichthus witte vis die}{geslacht zijt om ons}\\

\haiku{Buiten was alles,,.}{stil geworden een kille}{ruisende stilte}\\

\haiku{aanstonds, morgen, pijpt,.}{hij voor en allen dansen}{wij dien Orpheus na}\\

\haiku{Zij zag hoezeer hij,.}{leed en nam de kevie mee}{naar de biljartzaal}\\

\haiku{{\textquoteright} {\textquoteleft}Maak je niet bezorgd,{\textquoteright}, {\textquoteleft}.}{lachte Alfredik ben zo}{safe als de paus}\\

\haiku{Of mijn dochter,{\textquoteright} dacht,.}{hij nog maar hij sprak deze}{aanvulling niet uit}\\

\haiku{En toen de avond kwam,.}{begaf de jonge vrouw zich}{reeds vroeg ter ruste}\\

\haiku{{\textquoteright} Zij leefden ieder,;}{apart teruggetrokken en}{vervreemd van allen}\\

\haiku{Zij dronken de wijn.}{die rode warmte geeft en}{stille vertroosting}\\

\haiku{Daar zegt de nachtwacht:}{opeens luidop de jongste}{van zijn gedachten}\\

\haiku{Als ik bewegen,,.}{kon denkt de nachtwacht zou ik}{zeker gaan kijken}\\

\haiku{De ratten namen.}{een doden gijzelaar mee}{in hun dodenrijk}\\

\haiku{En voor de eerste,:}{maal terwijl het dag is ziet}{ge hun gedaante}\\

\haiku{Nergens meer een schat.}{die niet hun scherpe tanden}{hebben beschadigd}\\

\haiku{Hij verschanst zich, duwt,.}{de tafel voor de deur de}{nachtwacht is bevreesd}\\

\haiku{{\textquoteleft}Hier zijn geen ratten,{\textquoteright},.}{gromt de muzikant terwijl}{de riemen plassen}\\

\haiku{Allen volgen den,.}{pastoor in een tumult dat}{sneller sneller gaat}\\

\haiku{{\textquoteright} bedenkt hij, en den '.}{bode reikt hijt glas om}{weer te vullen}\\

\haiku{Een koude is er,,.}{die door alles heendringt in}{de kern van je hart}\\

\haiku{Hij haakt de keten,,.}{los die om zijn hals hangt legt}{hem neer op tafel}\\

\haiku{{\textquoteright} Maar de muzikant.}{knoopt reeds zijn wambuis dicht en}{keert zich naar de deur}\\

\haiku{Het bootje is een,,....}{stip verdwenen nog te zien}{voorgoed verdwenen}\\

\haiku{Tot zo laat was er;}{tussen al Gods dagen geen}{verschil voor Jukkers}\\

\haiku{Het eten was vanavond,...}{slecht maar uwe liefkozingen}{zijn al mijn voedsel}\\

\haiku{anders was zoveel.}{leugen en bedrog immers}{niet nodig geweest}\\

\haiku{Spranger belde mij....}{op voor een spoedoperatie}{voor Arthur Christensen}\\

\haiku{De ander trachtte,.}{hem te kalmeren maar hij}{wond zich steeds meer op}\\

\haiku{Als mensen uit een.}{andere wereld staarden}{wij elkander aan}\\

\haiku{In ieder geval,{\textquoteright}.}{was het beestje omnivoor}{stelde Fowler vast}\\

\haiku{Het brak tussen mijn,.}{tanden en ik bekeek het}{stuk dat ik vasthield}\\

\haiku{Ik zag hoe Fowler,.}{kokhalsde en was niet in}{staat te antwoorden}\\

\haiku{Vermoeid was zij nu,.}{na drie jaren geworden}{ten dode vermoeid}\\

\haiku{Alles was van een.}{uiterste zindelijkheid}{en scheen te stralen}\\

\haiku{Maar de oude vrouw.}{trok me aan de mouw mee naar}{een andere reet}\\

\haiku{{\textquoteleft}Wij moeten nu de,.}{andere kant uit want het}{laatste huis is leeg}\\

\haiku{Er zijn misdaden,.}{zo verschrikkelijk dat ze}{geen naam meer hebben}\\

\haiku{Bijna is het dag,{\textquoteright},.}{zegt hij naar buiten stappend}{waar de paarden staan}\\

\haiku{De Noordelijkste!}{kaart van pater Martinus}{was onnauwkeurig}\\

\haiku{Het jongetje en.}{het meisje aan haar zijde}{zien nieuwsgierig rond}\\

\haiku{{\textquoteright} Maar slap en willoos.}{liet ze zich doorgolven en}{Cobbs verloor de moed}\\

\haiku{Waarom heb ik ook,.}{geen moeder fluisterde het}{binnenste van Cobbs}\\

\haiku{Want hij geloofde,.}{niet meer omdat er niets meer}{te geloven viel}\\

\haiku{Opnieuw verlaat een,.}{vogel de aarde-zee}{de tweede laag}\\

\haiku{Ik zag de jonge,.}{zachtgrijze dag achter de}{ijsbloemen wachten}\\

\haiku{Je mengt je onder,.}{de schreeuwers de schelders en}{de hoera-roepers}\\

\haiku{Desnoods reis je om,,,.}{over Frankrijk en Engeland}{Noorwegen Zweden}\\

\haiku{Niettemin ben ik,.}{glimlachend op weg gegaan}{zo lijkt het me nu}\\

\haiku{Ach, die dame die,.}{zich nimmer verbidden laat}{dat weet ik vandaag}\\

\haiku{Als ik hem over de...}{bomen of in de rivier}{had kunnen gooien}\\

\haiku{Zwijgzaam als altijd,.}{en zonder verwondering}{bemerkten ze mij}\\

\haiku{Men kon zien dat hij.}{slechts verachting voelde voor}{hun botte smoelen}\\

\haiku{Ik vertelde hun, -}{dat ik ongelukkig was}{maar niet van de brief}\\

\haiku{{\textquoteright} Een weinig verbaasd:}{over deze onverwachte}{vraag antwoordde ik}\\

\haiku{Ik greep hem uit haar,,.}{handen kuste hem stak hem}{jubelend omhoog}\\

\haiku{Ik bezag slechts mijn,.}{brief van alle kanten dat}{kostbare kleinood}\\

\haiku{Maar deze brief is,,?}{mijn manneneer mijn roeping}{mijn al. Begrijp je}\\

\haiku{Kwamen ze niet in,?}{de kast waar Marja mijn jas}{had weggehangen}\\

\haiku{Voorzichtig sloop ik,.}{de trap af dat haar tante}{mij niet zou horen}\\

\haiku{Ik zou gaan, als de.}{koperen pijl op de spits}{naar den vijand wees}\\

\haiku{De heuvelrug die,.}{ik beklimmen moest was steil}{en terrasvormig}\\

\haiku{Orion, die langzaam,.}{klom en zijn arm hief naar de}{horens van de Stier}\\

\haiku{- een sterke arm die,.}{mij tilde een warme borst}{waaraan ik insliep}\\

\haiku{Maar is het geen dwaas -,!}{die z\'oveel onderneemt en}{Marja en Marja}\\

\haiku{Het zou lijken op,.}{een verzoek om steun en ik}{had geen steun nodig}\\

\haiku{Stad met de vele,.}{bruggen over het blinkende}{troebele water}\\

\haiku{De onrust knaagde,,.}{de vertwijfeling die vroom}{maakt deed mij vloeken}\\

\haiku{Altijd dezelfde,,.}{straten dezelfde hoeken}{dezelfde huizen}\\

\haiku{En van het toeval,,.}{indien dat bestaat moeten}{we profiteren}\\

\haiku{Voor de dag ermee,!}{met de brief die je daar in}{je binnenzak hebt}\\

\haiku{ik leed meer dan ooit,.}{te voren onder dit steeds}{dringender gevraag}\\

\haiku{De onbekende,.}{best\'a\'at en daarom zal hij}{ook zeker komen}\\

\haiku{Hij kon zich haar niet,.}{goed meer voorstellen had haar}{maar weinig gezien}\\

\haiku{Het w\`as natuurlijk,,.}{zo en geen middel bestond}{daartegen geen troost}\\

\haiku{In mijn tijd had een.}{meisje als Genia niet lang}{hoeven te wachten}\\

\haiku{Je moeder was nog,.}{geen vijfentwintig toen ze}{trouwde ga maar na}\\

\haiku{En hij dacht [maar zei].}{het toch niet dat dit best de}{laatste keer kon zijn}\\

\haiku{{\textquoteleft}Met zo'n jong hart als.}{dat van Oom Lucas kan je}{honderd jaar worden}\\

\haiku{Ik ben vroeger ook,.}{in betrekking geweest toen}{moeder nog leefde}\\

\haiku{Een heide waarin.}{zich het pad verloren heeft}{tussen de bosjes}\\

\haiku{{\textquoteleft}Wat een zonde dat.}{Oom Lucas dit niet meer heeft}{kunnen meemaken}\\

\haiku{{\textquoteleft}Ik ben het, Cesar,{\textquoteright},.}{fluisterde hij opdat ze}{niet verschrikken zou}\\

\haiku{Had hij soms willen?}{scheppen toen hij met Genia}{naar het stadhuis ging}\\

\haiku{de boreling van.}{zijn moeder was een ander}{schepsel dan hijzelf}\\

\haiku{Naast een man zoals?}{Cesar die niets wilde en}{geen ambities had}\\

\haiku{Het  zijn was voor.}{hem nog slechts een ogenblik van}{oneindige duur}\\

\haiku{Hij droomde dan, dat.}{een klein en wrak scheepje met}{hem afdreef naar zee}\\

\haiku{hij vergenoegde.}{zich met zijn uitkijkpost op}{de vestingmuur}\\

\haiku{In werkelijkheid,:}{verkende ze hem en ze}{kirde aan zijn oor}\\

\haiku{Maar het was er niet,.}{minder levendig om het}{liet mij rust noch duur}\\

\haiku{Wel, vandaar kwam ik.}{weer bij de gereedschappen}{en de machines}\\

\haiku{Ik kreeg precies de.}{nodige brandstof om aan}{de gang te blijven}\\

\haiku{Hij vertelde, dat.}{hij ziek was geweest en nog}{niet geheel hersteld}\\

\haiku{We kunnen daarvan.}{profiteren door vanavond}{samen uit te gaan}\\

\haiku{Het is vast een stuk.}{vlees dat van de haak gescheurd}{is en gevallen}\\

\haiku{{\textquoteleft}Hoor eens lieveling,.}{je hebt vergeten om de}{ijskast dicht te doen}\\

\haiku{De trekken van den.}{dodelijk gewonden man}{gaan zich ontspannen}\\

\haiku{hij weet vooruit dat.}{deze bij na-dode}{vastbesloten is}\\

\haiku{Nog lang v\'o\'or het dag,,.}{is kraait er zelfs een haan hier}{midden in de stad}\\

\haiku{na jaren is haar....}{parfum het enige wat je}{bijgebleven is}\\

\haiku{Nog geen week later.}{werd hij telegrafisch naar}{Moskou ontboden}\\

\haiku{een hond blafte, de,.}{zee ruiste een frisse bries}{woei onverstoorbaar}\\

\haiku{De gardiaan van.}{de Capucijnen is al}{urenlang op het fort}\\

\haiku{{\textquoteright} brulde hij, zo hard.}{dat de anderen bij de}{toonbank omkeken}\\

\haiku{De terechtstelling.}{van de Madrileense}{opstandelingen}\\

\haiku{{\textquoteleft}Ik trek af,{\textquoteright} denkt hij, {\textquoteleft},.}{bevel of geen bevel het}{kan mij niet schelen}\\

\haiku{Als was niet de man,.}{met de sigaar maar hij de}{ge\"executeerde}\\

\haiku{Uit zichzelf alleen.}{zou hij toch geen moed vinden}{voor zulk een besluit}\\

\haiku{{\textquoteright} Alfredo kwam op.}{hem toe en legde zijn hand}{op Tristan's schouder}\\

\haiku{En Tristan nu,:}{met ferme slagen op de}{rug kloppend zei hij}\\

\haiku{in al de jaren,,.}{heb ik geloof ik ook nooit}{meer aan hem gedacht}\\

\haiku{Je hebt gelijk, we.}{zijn al tien jaar getrouwd en}{er is veel gebeurd}\\

\haiku{Het had geen zin de,.}{tijd te tellen z\'o dicht bij}{het tijdeloze}\\

\haiku{Maar allen waren,.}{even apathisch allen schenen}{te slaapwandelen}\\

\haiku{{\textquoteleft}Ik geloof dat wij.}{op weg zijn veel te grote}{vrienden te worden}\\

\haiku{{\textquoteleft}En welke liegt er,,?}{niet om wanneer ze zich te}{jong waant of te oud}\\

\haiku{Of wat jij misschien?}{niet reeds tegen andere}{vrouwen gezegd hebt}\\

\haiku{{\textquoteleft}Rubbermelk is het,}{grootste goed dat de hemel}{ons gegeven heeft}\\

\haiku{Je veegt de room van....}{je lippen af en begint}{opnieuw te zoeken}\\

\haiku{Die het met minder,.}{doen zijn beslist pechvogels}{of onwilligen}\\

\haiku{Een beste leerling, -,.}{die dan ook de kans heeft een}{groot man te worden}\\

\haiku{{\textquoteright} {\textquoteleft}Ze zijn nu ook veel,{\textquoteright}.}{langer antwoordde Carmen}{verontschuldigend}\\

\haiku{Zo wordt een viool, -.}{doortinteld van een toon als}{ik van Carmen was}\\

\haiku{Maar welk een klein mens,,!}{ben ik was ik dat zo-iets}{hem kon ombrengen}\\

\haiku{Zo stonden wij daar.}{tegenover elkander in}{de vestibule}\\

\haiku{Daarom ging ik maar.}{de kapel binnen om tot}{mijzelf te komen}\\

\haiku{thans ben ik in een,, -.}{ander landschap hier ben ik}{vreemd nochtans bekend}\\

\haiku{onherroepelijk.}{komt achter elke hoogste}{top weer een vallei}\\

\haiku{Zweetdruppels kwamen.}{als kleine torren over mijn}{voorhoofd gekropen}\\

\haiku{Ze kruipen op langs,,.}{de wangen de neusvleugels}{naar de oogkassen}\\

\haiku{Met een juichkreet zag,,.}{ik het bos de weide de}{hemel om mij heen}\\

\haiku{Met dit alles had,!}{ik mijn ogen gebet om weer}{ziende te worden}\\

\haiku{Namen en wegen,.}{ging ik weten om toch het}{spoor bijster te zijn}\\

\subsection{Uit: Orkaan bij nacht}

\haiku{die zeker van mij.}{denkt  zoals ik toenmaals}{van mijn vader dacht}\\

\haiku{Hij is mijn zoon, maar.}{ach hoe bitter weinig van}{mijzelf is in hem}\\

\haiku{Zou het mogelijk?}{zijn dat mensen werkelijk}{kunnen liefhebben}\\

\haiku{En je bewustzijn.}{van dit alles zoetjes maar}{in slaap te zingen}\\

\haiku{Immuun zijn en de.}{stoten op te vangen met}{voldoende weerstand}\\

\haiku{, weer eens andere,.}{gezichten om mij heen zag}{andere huizen}\\

\haiku{Ze heeft geloofd dat.}{het een plotseling en kloek}{besluit van mij was}\\

\haiku{Maar de leegte, de,.}{onnozelheid de weerzin}{is ook hier gelijk}\\

\haiku{{\textquoteleft}Weet je dan niet dat?}{er miljoenen meisjes zijn}{die Maria heten}\\

\haiku{Maar m{\'\i}j zou ze niet,.}{vangen al had ze dan geen}{slechte kijk getoond}\\

\haiku{Het medelijden,,.}{dat tot het Ik zijn punt van}{uitgang wederkeert}\\

\haiku{Hij is niets dan een,.}{schaduw dat heb ik je reeds}{vaak genoeg gezegd}\\

\haiku{Wat baat het om te?}{zeggen dat ik het vooruit}{had kunnen weten}\\

\haiku{nu ga ik morgen,.}{zeker weg nu kan ik het}{niet meer uitstellen}\\

\haiku{Kloppen niet alle?}{toeristenharten sneller}{bij deze overvaart}\\

\haiku{Het komt er net op ',,}{aan hoe jet wilt zien zegt}{hij tegen zichzelf}\\

\haiku{Maar Minne is toch.}{blij dat hij bij hen zit en}{een weinig meepraat}\\

\haiku{En om niet al te:}{vriendschappelijk te schijnen}{voegt hij er aan toe}\\

\haiku{De dame heeft nog,.}{een jongetje bij zich van}{een jaar of vier vijf}\\

\haiku{Ieder jaar heeft hij;}{het minder begrepen op}{de Afrikaanse zon}\\

\haiku{Daarom zal hij het.}{de tijd die hem rest hier ook}{nog wel uithouden}\\

\haiku{Doch hij zou het wel,.}{willen ontkennen want hij}{is tegelijk kwaad}\\

\haiku{zulke vrouwen zijn,,.}{zelf net grote lastige}{wrede kinderen}\\

\haiku{Want hij peutert nog.}{steeds met zijn zakmes in het}{automobieltje}\\

\haiku{je hebt natuurlijk.}{kou gevat met dat natte}{pak van gisteren}\\

\haiku{Dan legt ze haar hand.}{op zijn voorhoofd en voelt hoe}{heet en klam het is}\\

\haiku{Tot ziens...{\textquoteright} En ze vraagt,.}{zich of hoe ze ertoe komt}{juist dit te zeggen}\\

\haiku{hoe zou er anders.}{een saamhorigheid tussen}{hen kunnen bestaan}\\

\haiku{Hij herinnert zich.}{het aarzelende in haar}{spreken en kijken}\\

\haiku{Natuurlijk, ze heeft;}{het kind meegenomen om}{zich te wapenen}\\

\haiku{En nu zijn ze met,.}{hun beiden alleen voor de}{allereerste maal}\\

\haiku{zeker is het dat;}{onze sterren ieder hun}{eigen baan volgen}\\

\haiku{Dit is de enige:}{hoop die onverwoestbaar in}{mij is gebleven}\\

\haiku{Minne probeert zich.}{rekenschap te geven van}{deze verdwazing}\\

\haiku{Een tussenweg... dus,,.}{halfheid w\'e\'er een compromis}{w\'e\'er een mislukking}\\

\haiku{hoe kun je eerlijk,?}{zijn zonder eerst te weten}{wat eerlijkheid is}\\

\haiku{Het leven heeft niets,.}{goed met hem voorgehad en}{nu is het te laat}\\

\haiku{{\textquoteright} {\textquoteleft}Natuurlijk, daarin,{\textquoteright}.}{heb je volkomen gelijk}{zegt Minne beschaamd}\\

\haiku{Daar is de brede;}{vlakke kust en het wijde}{blauw van de oceaan}\\

\haiku{Maar als Claire het,,?}{zelf uithoudt als z{\'\i}j het kan}{waarom dan h{\'\i}j niet}\\

\haiku{Het past bij Claire's,,;}{uiterlijk maar niet bij haar}{wezen denkt Minne}\\

\haiku{laat hij zich nu maar.}{tevreden stellen met zijn}{commandantsdochter}\\

\haiku{de lange uren dat;}{hij alleen is en Claire}{verdiept in haar werk}\\

\haiku{De Noordafrikaanse.}{lente is ondertussen}{warmer geworden}\\

\haiku{Het zonlicht trillert.}{al over de heuvels en staat}{schel boven de zee}\\

\haiku{Dit is een nieuwe.}{lievelingsplek die Minne}{moet leren kennen}\\

\haiku{in het binnenste;}{gedeelte van de baai dringt}{slechts heel weinig zon}\\

\haiku{De drift die thans in,.}{hem is geeft hem lust op de}{rotsen te beuken}\\

\haiku{tot het uiteenscheurt.}{en wij in tijdeloze}{diepte verdwijnen}\\

\haiku{maar is de zuiverste?}{liefde geen concentratie}{van levensgevaar}\\

\haiku{De vrouw hoort het aan:}{het iets-gejaagde van}{zijn praten en zegt}\\

\haiku{Toch heeft dit lijflijk.}{contact met de storm hem goed}{gedaan voor een poos}\\

\haiku{Zij voelt haar warmte,.}{in de zijne overgaan zijn}{gloed in de hare}\\

\haiku{Het duister brandt aan,.}{hun randen maar het is dicht}{en ondoordringbaar}\\

\haiku{En vernieuwt zich het,.}{vuur in hem uitlaaiend in}{de zwartere nacht}\\

\haiku{Hij is jaloers, en.}{het is spijt die hem stokstijf}{doet staan bij de deur}\\

\haiku{En eens moet t\'och de.}{droom gekoppeld worden aan}{het alledaagse}\\

\haiku{Boos stampt hij met zijn:}{voet op de grond en zegt met}{tranen in zijn stem}\\

\haiku{De hoge trotse,;}{schoorsteen is gevallen de}{tqaf is gebroken}\\

\haiku{{\textquoteright} zegt de vrouw, veel meer.}{ge{\"\i}nteresseerd nu ze}{zich tot Minne wendt}\\

\haiku{Ik hoop dat je het,{\textquoteright}.}{eens bent met mijn voorstellen}{zegt Minne lachend}\\

\haiku{Hoe jong hij ook is,.}{zij ziet dat hij zijn eigen}{weg begint te gaan}\\

\haiku{Daarom juist wil hij.}{liever twijfelen en het}{lezen uitstellen}\\

\haiku{Marc heeft veel dingen,.}{in zijn kleine hoofd die hij}{zou willen zeggen}\\

\haiku{Hij kan Minne niet;}{zomaar verachten net als}{hij het Suze doet}\\

\haiku{Het begrip {\textquoteleft}kind{\textquoteright} is.}{groot en dreigend opgestaan}{in zijn bewustzijn}\\

\haiku{Zijn wrakhout verder,;}{spoelen achterlaten op}{verlaten kusten}\\

\haiku{Wat heeft hun liefde,?}{aan een halfslachtig dienen}{aan gebondenheid}\\

\haiku{En onderwijl vindt:}{Postma bij Hopkins het onthaal}{dat hij verwachtte}\\

\haiku{{\textquoteright} {\textquoteleft}Ja, onze harems...{\textquoteright}.}{zegt de oude Arabier even}{naar hem toegewend}\\

\haiku{De nacht geeft Sam iets,.}{milds het grootsprekerige}{is een beetje weg}\\

\haiku{Hij houdt de handen,;}{aan zijn hoofd alsof hij in}{vervoering luistert}\\

\haiku{Gevaarlijker dan.}{opium of hennepzaad is}{westerse muziek}\\

\haiku{De toon van Minne's.}{telefoongesprek geeft hem}{volle zekerheid}\\

\haiku{{\textquoteleft}En...{\textquoteright} stottert Minne, {\textquoteleft}...?}{en wat wil je zeggen met}{die combinatie}\\

\haiku{In de verte is.}{nog slechts het laatste restje}{zonnegloed te zien}\\

\haiku{Of zij in deze.}{levenskwestie niet zo denkt}{als Sam bijvoorbeeld}\\

\haiku{Geen enkele maal,.}{heeft hij haar naam genoemd al}{kent hij die zeer goed}\\

\haiku{Enkel wacht ze nog,.}{dat Minne het zal zeggen}{wat ze reeds vermoedt}\\

\haiku{de kinderen die,.}{de toekomst zijn ze willen}{ook niet dat ik blijf}\\

\haiku{En suizelend zwenkt.}{de auto door de tuinpoort}{naar de grote weg}\\

\haiku{{\textquoteleft}Is het mogelijk?}{dat een mens gelukkig is}{en het zelf niet weet}\\

\haiku{Wel gaan zijn voeten,;}{licht alsof ze nog niet goed}{de aarde raken}\\

\haiku{het kind was lastig.}{en Madame kon driftig}{worden als een man}\\

\haiku{Hij is weggegaan,.}{ik heb nooit meer gehoord waar}{hij gebleven is}\\

\haiku{Ik hoopte dat ik,:}{ooit zo'n mens ontmoeten zou}{om maar te weten}\\

\haiku{Ik wist alleen wat,,.}{sexe was ik wist dat liefde}{\'anders zijn moest m\'e\'er}\\

\haiku{Hij vertelt nu ook,.}{van zijn eigen leven met}{slechts weinig woorden}\\

\haiku{alle dreiging heeft,.}{het thans verloren en het}{raakt de diepte niet}\\

\haiku{Er blijft nog slechts het,.}{kleine menselijke het}{uiterlijk daarvan}\\

\haiku{Maar madame zal?}{nu misschien wel weer het heft}{in handen nemen}\\

\haiku{Ja, mijn waarde, zo,....}{z{\'\i}jn ze de vrouwtjes spreekt hij}{hem in stilte toe}\\

\haiku{Hij slaat zijn kleine,:}{armen om Minne's hals en}{zoent hem en zegt lief}\\

\haiku{{\textquoteright} De heerlijkheid van.}{hun geheim blijft veilig in}{zijn hoede achter}\\

\haiku{Haar hart klopt zwaar en ':}{t is of elke harde}{bloedstoot weer herhaalt}\\

\haiku{Niemand, indien hij.}{niet voluit de prijs van zijn}{hart durft betalen}\\

\haiku{Ik kan niet meedoen.}{aan het verder kweken van}{de oude leugens}\\

\haiku{Dichterbij nog is.}{het zachte praten van een}{welbekende stem}\\

\haiku{Eens heb ik naar de,...}{dood verlangd het storten in}{een donkere kloof}\\

\subsection{Uit: Peis noch vree}

\haiku{wat hij gewild had,.}{dat voor eeuwig zou bestaan}{nu was het genoeg}\\

\haiku{Man en vrouw in bed,.}{ze nemen weinig m\'e\'er plaats}{dan een enkeling}\\

\haiku{Ze gaan erin, en,,!}{blijven tot ze betere}{mensen zijn erin}\\

\haiku{Ook dit kommerlijk.}{bestaan echter zou hun niet}{lang vergund blijven}\\

\haiku{De vrouw haalde de:}{schouders op en antwoordde}{zacht maar resoluut}\\

\haiku{Als wij iets hadden,.}{ik zou het gezegd hebben}{om h\'a\'ar te sparen}\\

\haiku{Zwijgend boog Josef.}{zich over zijn vrouw en tilde}{haar in zijn armen}\\

\haiku{Koffie,{\textquoteright} sprak de man.}{en trachtte monter wat licht}{in de lucht te zien}\\

\haiku{{\textquoteleft}Honden, bedelaars.}{en lieden van het vreemde}{ras hier ongewenst}\\

\haiku{Gelukkig, dacht zij,.}{dat Davidson tenminste}{dit niet heeft gemerkt}\\

\haiku{{\textquoteright} Toen had zijn moeder:}{het alweer weggetrokken}{met de berisping}\\

\haiku{waar het alles vond:}{waarmee het van meet af aan}{vertrouwd geraakt was}\\

\haiku{Wat een dwaasheid om.}{je op een tocht als deze}{te gaan poeieren}\\

\haiku{Tenslotte kun je,.}{alles overdrijven dat moest}{je toegeven}\\

\haiku{En aan je mooiheid,.}{komt gauw een eind vooral als}{je kinderen krijgt}\\

\haiku{Hij keek alleen een.}{beetje schuinsweg naar haar ogen}{en knikte zachtjes}\\

\haiku{Ik was toch met hem.}{getrouwd omdat ik op mijn}{manier van hem hield}\\

\haiku{De kalk was niet heel,.}{wit meer scheen opgelost in}{het schemergrauwen}\\

\haiku{De visser zette.}{zijn hengel tegen de wand}{en stiet de deur open}\\

\haiku{Ik heb er mijn loon,.}{voor gehad zo zou je het}{kunnen opvatten}\\

\haiku{Hier, waar het vredig.}{is en ik u lastig val}{met mijn geraaskal}\\

\haiku{{\textquoteright} {\textquoteleft}Zo zijn vrouwen,{\textquoteright} vond,.}{de hengelaar terwijl hij}{het bed klaarmaakte}\\

\haiku{{\textquoteleft}Al wat er is, al,.}{wat er leeft houdt Hij in zijn}{versteende handen}\\

\haiku{Achter haar hoorde:}{ze de scherpe neuriestem}{van Henri zingen}\\

\haiku{Hij droeg een Duitse,.}{uniformjas maar zijn broek was}{die van een burger}\\

\haiku{Een ijlheid die zelfs.}{met het koudste water niet}{viel af te wassen}\\

\haiku{Daar lag het landschap,.}{zoals zij het gisteren}{nog niet gezien had}\\

\haiku{Je bent nog aan 't -.}{begin alleen een kind ziet}{tegen jaren op}\\

\haiku{Hoe glunder had hij.}{haar niet aangekeken toen}{hij haar kwam wekken}\\

\haiku{Zo zullen het ook,.}{de landen moeten doen en}{de werelddelen}\\

\haiku{{\textquoteleft}Vrijwillig,{\textquoteright} troostte,.}{ik mijzelf want ik had toch}{kunnen weigeren}\\

\haiku{{\textquoteleft}Jullie zouden pas,{\textquoteright}.}{gelukkig zijn wanneer je}{een kind had zei hij}\\

\haiku{Daarom weet ik nu.}{ook heel precies hoe het toen}{moet zijn gelopen}\\

\haiku{ze) daar straffeloos.}{en haast bij wijze van sport}{worden uitgemoord}\\

\haiku{Ik weet niet of er.}{gevolg gegeven is aan}{haar gratieverzoek}\\

\haiku{je vader,{\textquoteright} zei de.}{kluizenaar toen de grijsaard}{eindelijk klaar was}\\

\haiku{{\textquoteleft}Niet onze woorden,.}{maar onze daden vormen}{onze verdienste}\\

\subsection{Uit: De rancho der X mysteries}

\haiku{Terwijl hij naar het,:}{geld zocht trok hij een couvert}{voor de dag en zei}\\

\haiku{Maar ik zal hem wel.}{zorgvuldig bewaren tot}{aan mijn eigen dood}\\

\haiku{Dat is een van de.}{hoofdopgaven van wat men}{hier beschaving noemt}\\

\haiku{Die ligt nog altijd.}{achter slot in een lade}{van dit schrijfburo}\\

\haiku{{\textquoteleft}Want inderdaad heb,,:}{ik een ogenblik niet langer}{dan een flits gedacht}\\

\haiku{Hoe mijn geest zich zou,.}{voelen leek mij voorlopig}{nog raadselachtig}\\

\haiku{Toen viel mijn oog op,.}{een langwerpige krat die}{geheel anders was}\\

\haiku{{\textquoteleft}Het is toch jammer.}{dat wij de politie niet}{geroepen hebben}\\

\haiku{{\textquoteright} De knecht had weer moed,.}{gevat liet den Amerikaan}{niet alleen werken}\\

\haiku{{\textquoteright} {\textquoteleft}Toen is de {\textquoteleft}Timor{\textquoteright}.}{gezonken van de zwaarte}{van al die vlinders}\\

\haiku{Ik ging dus alleen.}{en moet bekennen dat ik}{teleurgesteld werd}\\

\haiku{Dan schenk ik hem u..}{Een vriendelijk geschenk mag}{men niet weigeren}\\

\haiku{Gelijk zo vaak in}{mijn leven verweet ik een}{onverschrokkener}\\

\haiku{Ik was nog niet veel,:}{wijzer maar de conducteur}{die langs kwam en riep}\\

\haiku{Ik moet even verschrikt,:}{gekeken hebben want hij}{verklaarde aanstonds}\\

\haiku{Geloof niet, dat ze.}{er iets bijgeleerd hebben}{in al die eeuwen}\\

\haiku{Het enige wat zij,.}{leerden is bidden in de}{kerk en berusten}\\

\haiku{en voelde ik mijn,,.}{koffer die ik nog in de}{hand hield weggerukt}\\

\haiku{Er waren meer dan.}{dertig doden en meer dan}{honderd gewonden}\\

\haiku{Mijn verstand zei me,,,.}{dat ik om dit te kunnen}{doen moest hertrouwen}\\

\haiku{Sommigen slaakten,.}{een zucht van verlichting of}{misschien wel van spijt}\\

\haiku{hij wankelde en.}{de omstaanders begonnen}{weer op te dringen}\\

\haiku{Dat is verboden,{\textquoteright}.}{verklaarde Candelario}{met plechtige ernst}\\

\haiku{{\textquoteleft}Zo'n auto is voor.}{hen een mysterie en voor}{ons iets alledaags}\\

\haiku{{\textquoteleft}Uw vader heeft een,.}{kortgeknipte snor als van}{zwarte kokosbast}\\

\haiku{Het liep tegen de,.}{middag de tijd dat alles}{onbeweeglijk wordt}\\

\haiku{Maar dan onder de,.}{mensen zodat ze worden}{als redeloos vee}\\

\haiku{{\textquoteleft}Ik wou dat alles.}{zo licht op mijn geweten}{woog als deze Yaqui}\\

\haiku{Het zal wel een uur,.}{geduurd hebben voordat de}{kleine deur open ging}\\

\haiku{Daarna heb je mij,.}{verschrikt en dan krijg ik soms}{van die toevallen}\\

\haiku{maar de rest van de.}{droom was overduidelijk en}{liet mij niet meer los}\\

\haiku{{\textquoteleft}Ik zal nooit al die.}{personen en stromingen}{leren ontwarren}\\

\haiku{Hij houdt er alleen,.}{niet van zijn paarlen voor de}{zwijnen te werpen}\\

\haiku{het zal het eerste,.}{zijn wat ik doe zodra ik}{mijn geld gebeurd heb}\\

\haiku{Ik probeerde de.}{panama en daarna zo'n}{weke vilten}\\

\haiku{{\textquoteright} zei ik onderweg,.}{nog lachend om het dwaze}{visioen van Efra{\'\i}n}\\

\haiku{De lange golfslag.}{der regenvlagen zong mij}{eindelijk in slaap}\\

\haiku{{\textquoteright} Hij leek opgelucht,:}{door dit antwoord wendde zich}{af en zei schamper}\\

\haiku{Het zou van alles, -...}{kunnen zijn van een wapen}{tot een stuk speelgoed}\\

\haiku{maar nauwelijks ziet,......}{hij het hij hoort alleen en}{sterft twintig doden}\\

\haiku{maar ik kon alleen,.}{inwendig vloeken omdat}{ik zo weerloos was}\\

\haiku{Daar hielden de vier.}{bandieten stil en stegen}{van hun paarden af}\\

\haiku{Maar toen deze hem,:}{zag aankomen fluisterde}{hij geruststellend}\\

\haiku{slechts een kleine haag.}{van stevige rietstengels}{scheidde mij ervan}\\

\haiku{Harris, toen zij zich.}{opzettelijk door mij had}{laten inhalen}\\

\haiku{Met zachte klopjes.}{begon ik de hand van het}{meisje te strelen}\\

\haiku{Maar het hart dat hij,.}{bezat was nog wijder en}{opener dan dit land}\\

\haiku{Het huttendorp was.}{echter uitgestorven toen}{wij er aankwamen}\\

\haiku{Nauwelijks was de,.}{Indio goed en wel weg of}{de vrouw kwam terug}\\

\haiku{Vandaag zal ik u.}{niet lastig vallen met zo'n}{afgodendienaar}\\

\haiku{Agapito heeft ook;}{kruiden gegeven aan de}{vrouw die mij verzorgt}\\

\haiku{{\textquoteright} Maar de Schoolmeester,,:}{schudde het hoofd en zei half}{lachend half spijtig}\\

\haiku{Hij deed het rustig,,;}{vleiend bijna maar op zeer}{suggestieve toon}\\

\haiku{Ze hadden best een.}{spelletje met Felipe}{willen beginnen}\\

\haiku{En nog verder moesten.}{bruine  adobe-hutten}{van dc armen staan}\\

\haiku{Ik antwoordde van,.}{ja maar voegde er aan toe}{dat hij bezoek had}\\

\haiku{{\textquoteleft}Neen,{\textquoteright} zei hij, {\textquoteleft}ik zal.}{je liever laten zien dat}{ook het licht niets baat}\\

\haiku{{\textquoteleft}Juana Sierra,{\textquoteright}.}{weer met hetzelfde verzoek}{om een gebedje}\\

\haiku{uit alles bleek, dat.}{hij een even goed theoloog}{als doodgraver was}\\

\haiku{{\textquoteright} {\textquoteleft}De Heilige Maagd,}{heeft zeker niets met deze}{dingen te maken}\\

\haiku{De figuur droeg  .}{een kroon en hield de armen}{op de borst gekruist}\\

\haiku{het is een soort van,.}{leegte een verlangen naar}{niet-verlangen}\\

\haiku{Hijzelf weigerde.}{rond te hangen bij zulk een}{uitgestorven boel}\\

\haiku{Baboso nogmaals.}{de magische strepen van}{zijn toorn over de grond}\\

\haiku{Maar laat hem eens aan...}{zijn lot over en kom dan over}{zes maanden terug}\\

\haiku{Maar niemand weet of,.}{ik mijzelf straks kan redden}{laat staan een ander}\\

\haiku{{\textquoteright} Met alles wat in,.}{mij was verzette ik mij}{tegen dit denkbeeld}\\

\haiku{Zie je nu wel, dat?}{mijn waarschuwing maar al te}{goede gronden had}\\

\haiku{Maar Mexico heeft bij.}{al het boosaardige iets}{van de eeuwigheid}\\

\haiku{{\textquoteleft}Er is een man van,.}{je rancho gekomen die}{je dit gebracht heeft}\\

\haiku{Morgenochtend komt.}{hij bij mij op kantoor om}{je te ontmoeten}\\

\haiku{Zo niet, dan is het,.}{misschien nog wel het beste}{dat gij ze niet weet}\\

\haiku{Groet mijn neef Isidro en,.}{zeg hem dat hij gelijk heeft}{om weg te blijven}\\

\haiku{Het viel moeilijk uit,.}{te maken maar tenslotte}{kon het heel best zijn}\\

\haiku{Maar wat is er nog,?}{anders in te vinden dan}{stof stof en knekels}\\

\haiku{Om generaal te,.}{zijn moet men in Mexico ook}{zo\"oloog wezen}\\

\subsection{Uit: Serenitas}

\haiku{wat al die vrienden,.}{zochten leek hem hazardspel}{onzindelijkheid}\\

\haiku{{\textquoteright} Slechts eenmaal had zijn,.}{baas gedacht dat het ook met}{Dorus mis zou lopen}\\

\haiku{Na dagen werk sprak,.}{zij hem weer aan een avond toen}{hij van zijn werk kwam}\\

\haiku{Nou ja, je hebt ook{\textquoteright},, {\textquoteleft}'.}{gelijk zei het meisjet}{Is nog niet eens aan}\\

\haiku{Als je kwam moest je.}{maar meteen zeggen wanneer}{we gingen trouwen}\\

\haiku{{\textquoteright} {\textquoteleft}Een man heeft altijd{\textquoteright},.}{bedoelingen met een vrouw}{zei Marietje bits}\\

\haiku{Met een wonderlijk,:}{hoge jongensstem die even}{maar schor klonk zong hij}\\

\haiku{Ze zuchtte verlicht,.}{toen ze hem daar nog zag staan}{geduldig wachtend}\\

\haiku{Aan onszelf mankeert,.}{ook wat en je moet trachten}{samen goed te zijn}\\

\haiku{De liefde is een.}{geluk en een offer had}{de pastoor gezegd}\\

\haiku{Toen hij de pastoor, '.}{weer sprak vertelde Dorus van}{t mislukte plan}\\

\haiku{Wat zal het moeilijk.}{zijn om die te openen voor}{een vriendelijk woord}\\

\haiku{ze was nog zwak, en,:}{toen Dorus weer alleen zat met}{zijn tante sprak hij}\\

\haiku{{\textquoteright} {\textquoteleft}Een mens kan zoveel,{\textquoteright}.}{honger hebben dat het hem}{niet meer kan schelen}\\

\haiku{ze zal geloven,.}{dat het medelijden van}{je is of zoiets}\\

\haiku{{\textquoteright} En toen zij bezig:}{was te drinken van het glas}{dat hij nog vasthield}\\

\haiku{Je ziet het wel, ze{\textquoteright},.}{is niet erg toeschietelijk}{zei de oude vrouw}\\

\haiku{{\textquoteleft}Je moest een ander,.}{huis zien te krijgen als we}{ooit zover komen}\\

\haiku{Want zo gesloten,,:}{zo afwezig was ze soms}{dat hij moest denken}\\

\haiku{'t Is alleen erg,.}{dat je zo verloren loopt}{en zonder veel doel}\\

\haiku{En ik ben juist hier{\textquoteright}.}{komen wonen om van veel}{geklets af te zijn}\\

\haiku{zoals het gaan moet.}{en die maakt dat alles ten}{goede terecht komt}\\

\haiku{Een man die nooit door;}{hartstocht weifelde in zijn}{koninklijk gebaar}\\

\haiku{Dorus echter zat in.}{diep nadenken verzonken}{toen Winters thuis kwam}\\

\haiku{Het wil me niet uit{\textquoteright},, {\textquoteleft}.}{de gedachten sprak hijdat}{je ongelijk hebt}\\

\haiku{{\textquoteleft}Het is iets prachtigs.}{als je zo zorgeloos en}{zo gelukkig bent}\\

\haiku{{\textquoteleft}Als ze nog van mij,{\textquoteright}.}{houden dan zou u ook van}{me moeten houden}\\

\haiku{Zo'n huichelaar, zo'n,!}{stiekemerd zo'n vuilpoes van}{een farizee\"er}\\

\haiku{De kinderschaar aan,,.}{uwe voeten O Jezus komt}{U blij begroeten}\\

\haiku{Zonde van iemand!}{die zo van het leven zou}{kunnen genieten}\\

\haiku{Als de knechts reeds zijn,.}{schaften staat Dorus nog steeds in}{de deur en luistert}\\

\subsection{Uit: De stille plantage}

\haiku{En wie ze ooit van,.}{te voren zag hij hervindt}{ze                         nimmermeer}\\

\haiku{{\textquoteright}  Josephine.}{sloeg haar arm om hem heen en}{streelde zijn haren}\\

\haiku{Raoul wil niet naar, '.}{Gen\`eve bij alt getwist}{om bijbelwoorden}\\

\haiku{{\textquoteright}                        Maar C\'ecile,:}{zag de blaren vallen en}{in haar zei een stem}\\

\haiku{{\textquoteright}  {\textquoteleft}Daarom heb ik,.}{ook niet omgezien maar in}{mij en nu vooruit}\\

\haiku{{\textquoteleft}Dit land is te klein,.}{voor ballingen te rijk}{voor armen als wij}\\

\haiku{los genoeg wezen;}{om daarheen te gaan waar het}{schoonste ons                     roept}\\

\haiku{En stellig spreken,.}{wij elkander nader}{over vijf of tien jaar}\\

\haiku{{\textquoteright}  {\textquoteleft}Zoudt gij denken?}{dat er een dusdanige}{voorbestemming was}\\

\haiku{{\textquoteright}  {\textquoteleft}O neen, ik ga,{\textquoteright}.}{om niet weer te keren zei}{Raoul beraden}\\

\haiku{Wantrouw de hitte,,.}{wantrouw het land wantrouw zelfs}{wat u schoon                     lijkt}\\

\haiku{Reeds woog de harde.}{scherpe hitte op hun hoofd}{en op hun handen}\\

\haiku{{\textquoteright}  {\textquoteleft}De grond is hier,{\textquoteright}, {\textquoteleft},.}{ook goed sprak Daszo goed dat}{het een nadeel heeft}\\

\haiku{De heer                     Morhang.}{zal zien dat goedheid ondeugd}{is voor dit zwart vee}\\

\haiku{Een stoel die niet deugt,}{hak je tot brandhout                     een}{slaaf die je ergert}\\

\haiku{Hij kon aan wal gaan.}{om het terrein nauwkeurig}{te                     verkennen}\\

\haiku{Hoe verder van                     ,.}{de mensen hoe groter de}{kansen voor geluk}\\

\haiku{{\textquoteleft}Verbannen zult ge,.}{u                     zeker voelen zo}{bij tijd en wijle}\\

\haiku{met de                     aanleg.}{van tabaksaanplantingen}{was reeds begonnen}\\

\haiku{{\textquoteleft}Deze plantage.}{zal eerst over twintig jaar ten}{volle bloeien}\\

\haiku{op wie de stilte.}{der plantage zwaarder weegt}{dan nodig is}\\

\haiku{het zwart van                     zijn.}{halfnaakte lichaam leek slechts}{schaduw van een wolk}\\

\haiku{Er is een blanke,.}{geestelijke dicht hier in}{de buurt verscholen}\\

\haiku{De angsten van haar.}{verbeelding                     werden er}{nog tastbaarder door}\\

\haiku{Op de plantages.}{was de regentijd die van}{het grootste gevaar}\\

\haiku{Hier was het een park;}{dat wachtte op het rustig}{treden van een mens}\\

\haiku{Ons hart is een boot;}{die geankerd ligt in een}{zeer                     stille baai}\\

\haiku{Een deel slechts van 't,.}{lied verstond Raoul en het}{ergerde                     hem}\\

\haiku{{\textquoteright}  De anderen.}{noemden hem indringer en}{hoogmoedige dwaas}\\

\haiku{Slechts                     dat kon hem,:}{redden van de ondergang}{het doemwaardige}\\

\haiku{De vruchtbare geest,.}{van het woud houdt zich verre}{van ons dacht zij}\\

\haiku{Als zelfs                     hij het,...}{niet begreep hoe zouden dan}{ooit de anderen}\\

\haiku{Agnes was nu de;}{uiterste grenzen van de}{velden genaderd}\\

\haiku{Bijna verlaten.}{lagen de kostgronden en}{de tabaksvelden}\\

\haiku{Wat nog gered werd,.}{was verweg het kleinste en}{schamelste deel}\\

\haiku{Als Willem Das weer,.}{iets d\'a\'arover zeggen zou kon hij}{hem                     antwoorden}\\

\haiku{Zonder te schreeuwen '.}{kromden zij zich ondert}{snerpen van zijn zweep}\\

\haiku{Heeft een meisje eens,,?}{geen woorden                     los daaruit}{verwaaid gestameld}\\

\haiku{Recht boven hun hoofd,.}{scheen de koperen zon en}{geen vogel zong meer}\\

\haiku{Zelf droeg hij haar, zwaar,.}{op zijn armen maar zonder}{te hijgen en kalm}\\

\haiku{Snel doet de dood zijn.}{werk in                     het land van het}{brandende leven}\\

\haiku{{\textquoteright} stond op en wenkte.}{de negers dat zij hem weg}{konden                     dragen}\\

\haiku{Waanzinnig stormde,:}{Agnes                     naar binnen de}{haren verwilderd}\\

\haiku{Duizend geluiden;}{omvingen het blokhuis en}{heel de plantage}\\

\haiku{Een                     vlaag joeg haar,.}{de kamer uit naar het bed}{van de opzichter}\\

\haiku{een man die zij had,.}{kunnen                     liefhebben die}{rechten had op haar}\\

\haiku{Des avonds speelde zij.}{vaak met het bruine kind dat}{nu een vrije was}\\

\haiku{Een harde onwil.}{deed Raoul de tanden op}{elkander klemmen}\\

\haiku{meer grond dan hij                     ,,,,...}{betreden kon veel huizen}{slaven vrouwen macht}\\

\haiku{Hij zegt dat ik niets,.}{van het                     werk begrijp en}{dat is misschien waar}\\

\haiku{Er was geen denken,.}{aan                     ze in de bossen}{te achtervolgen}\\

\haiku{{\textquoteleft}Als het moet, kunnen.}{wij hier later altijd weer}{terug                     komen}\\

\haiku{Dit licht was nieuw en.}{luisterrijker dan zij ooit}{tevoren zagen}\\

\haiku{Het was niet in te.}{denken dat je dit alles}{nooit meer zou zien}\\

\haiku{Niet met de dagen,.}{leef je daar maar met de}{snelheid van dromen}\\

\haiku{Des ochtends was zij,.}{het eerste buiten op de}{voorplecht van de boot}\\

\haiku{Altijd door spoelde;}{de zee langs                     het schip en}{wast zijn wanden schoon}\\

\haiku{{\textquoteleft}Soms was het toch nog,{\textquoteright}.}{mooier ginds sprak zij als tot}{zichzelve}\\

\haiku{{\textquoteright}  {\textquoteleft}Ja, je had er,{\textquoteright}.}{prachtige kansen bromde}{Raoul in zijn baard}\\

\haiku{eeuwig is de                     ,.}{stroom de enige die deze}{wildernis trotseert}\\

\haiku{Ik heb een oude...}{neger hier eens er over}{horen vertellen}\\

\haiku{Doch met de grootste,:}{stelligheid antwoordde}{de oude neger}\\

\haiku{Maar gaat gerust uw,,.}{gang                     heren producers}{mij maakt het niets uit}\\

\subsection{Uit: De stille plantage}

\haiku{Een zonnescheut, en:,,.}{zij denken verder verder}{verder moet het zijn}\\

\haiku{zou zijn leven niet...}{verloren zijn wanneer dit}{alles er niet was}\\

\haiku{Raoul wil niet naar, '.}{Gen\`eve bij alt getwist}{om bijbelwoorden}\\

\haiku{{\textquoteleft}Je begrijpt wat het.}{is voor Raoul om alles}{achter te laten}\\

\haiku{Zij bleef staan en trok.}{haar bij zich op het lage}{muurtje langs de weg}\\

\haiku{{\textquoteleft}Ik moet je nog wat,{\textquoteright}, {\textquoteleft}.}{vertellen sprak hijwat ik}{straks verzwijgen moest}\\

\haiku{D\'a\'ar was Raoul, die;}{buiten nog haastig stond te}{spreken met zijn oom}\\

\haiku{D\'a\'ar Josephine,;}{die haar arm om C\'ecile}{heen geslagen had}\\

\haiku{die luistert en ver;}{staat de gefluisterde taal}{van wind en wolken}\\

\haiku{{\textquoteleft}Bekroop u nog nooit,?}{de lust om daarginder te}{blijven kapitein}\\

\haiku{Reeds woog de harde.}{scherpe hitte op hun hoofd}{en op hun handen}\\

\haiku{Maar als ge wilt, zal.}{ik gaarne van uw raad en}{dienst gebruik maken}\\

\haiku{Met statig en toch;}{vriendelijk gebaar ontving}{de landvoogd  hem}\\

\haiku{En ik denk dat gij,}{die het leven hier niet kent}{en niet gewend zijt}\\

\haiku{Of zegt de bijbel?}{niet dat elke meester zijn}{knecht kastijden mag}\\

\haiku{{\textquoteleft}Vergeet morgen niet,{\textquoteright}:}{de dijken op te hogen}{of tegen Raoul}\\

\haiku{Maar ik denk toch niet,.}{dat hij de rechte man is}{hier op deze plaats}\\

\haiku{{\textquoteleft}'t Is niet goed in.}{deze oorden dat een man}{te lang alleen zij}\\

\haiku{Wanneer je met een,.}{vrouw wilt wonen zal ik het}{de meester vragen}\\

\haiku{{\textquoteright} {\textquoteleft}Laat mij het nog niet,,{\textquoteright}.}{zeggen misses smeekte hij}{het hoofd gebogen}\\

\haiku{Er is een blanke,.}{geestelijke dicht hier in}{de buurt verscholen}\\

\haiku{Op de plantages.}{was de regentijd die van}{het grootste gevaar}\\

\haiku{Hier was het een park;}{dat wachtte op het rustig}{treden van een mens}\\

\haiku{Waar zij was liet zij,.}{iets dierbaars achter stierf een}{deel af van haar zelf}\\

\haiku{Dit was zijn rijk, dat.}{hij regeren zou naar zijn}{eer en geweten}\\

\haiku{Wat heb je eraan.}{met alle anderen in}{onvrede te zijn}\\

\haiku{Steeds herkende zij.}{weer een hoek of een boom of}{een dak van Bel Exil}\\

\haiku{{\textquoteleft}Wie honger heeft moet.}{de aarde bebouwen in}{het zweet zijns aanschijns}\\

\haiku{Agnes was nu de;}{uiterste grenzen van de}{velden genaderd}\\

\haiku{{\textquoteright} {\textquoteleft}Kan je geduldig,?}{zijn als je eensklaps het spoor}{bijster geraakt bent}\\

\haiku{Bijna verlaten.}{lagen de kostgronden en}{de tabaksvelden}\\

\haiku{Lang nog blijven ze,.}{wanneer uw woning reeds leeg}{en verlaten is}\\

\haiku{Zonder te schreeuwen '.}{kromden zij zich ondert}{snerpen van zijn zweep}\\

\haiku{Recht boven hun hoofd,.}{scheen de koperen zon en}{geen vogel zong meer}\\

\haiku{Zelf droeg hij haar, zwaar,.}{op zijn armen maar zonder}{te hijgen en kalm}\\

\haiku{Rustig, zonder te.}{haasten trad hij naar buiten}{toen Raoul hem riep}\\

\haiku{Duizend geluiden;}{omvingen het blokhuis en}{heel de plantage}\\

\haiku{En daarin, opeens,,.}{tjuikte een vogel kraaide}{de eerste haan}\\

\haiku{Het leek of ze niet.}{het minst begrepen wat er}{gedaan moest worden}\\

\haiku{niemand stoorde meer.}{het nachtelijk leven van}{de negerinnen}\\

\haiku{daarin was zij zelf.}{somtijds een vreemde die zij}{plotseling hervond}\\

\haiku{Een harde onwil.}{deed Raoul de tanden op}{elkander klemmen}\\

\haiku{Te weinig om de.}{strijd tegen de wildernis}{voorgoed te winnen}\\

\haiku{En hij nam haar in,.}{zijn armen kuste haar de}{tranen uit de ogen}\\

\haiku{{\textquoteright} {\textquoteleft}Zijn jeugd...{\textquoteright} glimlachte,.}{Josephine vrolijk door}{Raouls zekerheid}\\

\haiku{Josephine dacht.}{niet langer aan heengaan en}{niet meer aan blijven}\\

\haiku{Haar eindelijk weer.}{te paard te zien gaf hem een}{onverwacht plezier}\\

\haiku{Dit licht was nieuw en.}{luisterrijker dan zij ooit}{tevoren zagen}\\

\haiku{Langs onbekende,:}{bossen varen de boten}{en toch denkt een mens}\\

\haiku{Des ochtends was zij,.}{het eerste buiten op de}{voorplecht van de boot}\\

\haiku{{\textquoteleft}'t Is bij wijze,.}{van vergiffenis die ik}{u vraag mejuffer}\\

\haiku{{\textquoteright} {\textquoteleft}Ach, zeg toch liever,}{nog eens hoe die tijgers uit}{de bossen kwamen}\\

\haiku{{\textquoteright} Maar Josephine,:}{verdedigde haar zoon en}{spotte op haar beurt}\\

\haiku{{\textquoteright} {\textquoteleft}Als u dat bedoelt,{\textquoteright}, {\textquoteleft}.}{zei toen Gastonik ga er}{later vast naar toe}\\

\haiku{Maar gaat gerust  ,,.}{uw gang heren producers}{mij maakt het niets uit}\\

\subsection{Uit: De stille plantage}

\haiku{Een zonnescheut, en:,,.}{zij denken verder verder}{verder moet het zijn}\\

\haiku{zou zijn leven niet...}{verloren zijn wanneer dit}{alles er niet was}\\

\haiku{Raoul wil niet naar, '.}{Gen\`eve bij alt getwist}{om bijbelwoorden}\\

\haiku{{\textquoteleft}Je begrijpt wat het.}{is voor Raoul om alles}{achter te laten}\\

\haiku{Zij bleef staan en trok.}{haar bij zich op het lage}{muurtje langs de weg}\\

\haiku{{\textquoteleft}Ik moet je nog wat,,.}{vertellen sprak hij wat ik}{straks verzwijgen moest}\\

\haiku{D\'a\'ar was Raoul, die;}{buiten nog haastig stond te}{spreken met zijn oom}\\

\haiku{D\'a\'ar Josephine,;}{die haar arm om C\'ecile}{heen geslagen had}\\

\haiku{Bekroop u nog nooit,?}{de lust om daarginder te}{blijven Kapitein}\\

\haiku{Reeds woog de harde.}{scherpe hitte op hun hoofd}{en op hun handen}\\

\haiku{Maar als ge wilt zal.}{ik gaarne van uw raad en}{dienst gebruik maken}\\

\haiku{En ik denk dat gij,}{die het leven hier niet kent}{en niet gewend zijt}\\

\haiku{Of zegt de bijbel?}{niet dat elke meester zijn}{knecht kastijden mag}\\

\haiku{Maar ik denk toch dat,.}{hij niet de rechte man is}{hier op deze plaats}\\

\haiku{{\textquoteleft}'t Is niet goed in.}{deze oorden dat een man}{te lang alleen zij}\\

\haiku{Krijgen ze uitbouw,?}{aan hun loods waarover ik je}{laatst gesproken heb}\\

\haiku{Wanneer je met een,.}{vrouw wilt wonen zal ik het}{de meester vragen}\\

\haiku{{\textquoteright} {\textquoteleft}Laat mij het nog niet,,{\textquoteright}.}{zeggen misses smeekte hij}{het hoofd gebogen}\\

\haiku{Er is een blanke,.}{geestelijke dicht hier in}{de buurt verscholen}\\

\haiku{Op de plantages.}{was de regentijd die van}{het grootste gevaar}\\

\haiku{Hier was het een park;}{dat wachtte op het rustig}{treden van een mens}\\

\haiku{Waar zij was liet zij,.}{iets dierbaars achter stierf een}{deel af van haar zelf}\\

\haiku{{\textquoteleft}Wat zijn zij nog ver,.}{van het Christendom zei hij}{tegen C\'ecile}\\

\haiku{Dit was zijn rijk, dat.}{hij regeren zou naar zijn}{eer en geweten}\\

\haiku{Wat heb je eraan.}{met alle anderen in}{onvrede te zijn}\\

\haiku{Steeds herkende zij.}{weer een hoek of een boom of}{een dak van Bel Exil}\\

\haiku{Wie honger heeft moet.}{de aarde bebouwen in}{het zweet zijns aanschijns}\\

\haiku{Bijna verlaten.}{lagen de kostgronden en}{de tabaksvelden}\\

\haiku{Lang nog blijven ze,.}{wanneer uw woning reeds leeg}{en verlaten is}\\

\haiku{Ik geloof niet dat.}{ze gaarne naar andere}{meesters verhuizen}\\

\haiku{Recht boven hun hoofd,.}{scheen de koperen zon en}{geen vogel zong meer}\\

\haiku{Zelf droeg hij haar, zwaar,.}{op zijn armen maar zonder}{te hijgen en kalm}\\

\haiku{Rustig, zonder te.}{haasten trad hij naar buiten}{toen Raoul hem riep}\\

\haiku{Duizend geluiden;}{omvingen het blokhuis en}{heel de plantage}\\

\haiku{En daarin, opeens,,.}{tjuikte een vogel kraaide}{de eerste haan}\\

\haiku{t Leek of ze niet.}{het minst begrepen wat er}{gedaan moest worden}\\

\haiku{niemand stoorde meer.}{het nachtelijk leven van}{de negerinnen}\\

\haiku{daarin was zij zelf.}{somtijds een vreemde die zij}{plotseling hervond}\\

\haiku{Een harde onwil.}{deed Raoul de tanden op}{elkander klemmen}\\

\haiku{Te weinig om de.}{strijd tegen de wildernis}{voorgoed te winnen}\\

\haiku{Het zweet droop langs zijn,.}{hoofd en armen zo snel liep}{hij naar het blokhuis}\\

\haiku{En hij nam haar in,.}{zijn armen kuste haar de}{tranen uit de ogen}\\

\haiku{{\textquoteright} {\textquoteleft}Zijn jeugd...{\textquoteright} glimlachte,.}{Josephine vrolijk door}{Raouls zekerheid}\\

\haiku{Haar eindelijk weer.}{te paard te zien gaf hem een}{onverwacht plezier}\\

\haiku{Langs onbekende,:}{bossen varen de boten}{en toch denkt een mens}\\

\haiku{Des ochtends was zij,.}{het eerste buiten op de}{voorplecht van de boot}\\

\haiku{Slechts Agnes stond nog '.}{laat opt achterdek te}{staren in dit zwart}\\

\haiku{Hij heeft ook u doen,,}{ontwaken v\'o\'or het te laat}{was en mogelijk}\\

\haiku{{\textquoteleft}'t Is bij wijze,.}{van vergiffenis die ik}{u vraag mejuffer}\\

\haiku{{\textquoteright} {\textquoteleft}Ach, zeg toch hever,}{nog eens hoe die tijgers uit}{de bossen kwamen}\\

\haiku{{\textquoteright} Maar Josephine,:}{verdedigde haar zoon en}{spotte op haar beurt}\\

\haiku{{\textquoteright} {\textquoteleft}Als u d\`at bedoelt,,.}{zei toen Gaston ik ga er}{later vast naar toe}\\

\subsection{Uit: Het vergeten gezicht}

\haiku{Ze hadden onrust,;}{in zijn leven gebracht en}{maar schamel plezier}\\

\haiku{Het was trouwens al;}{de zoveelste keer dat hij}{in Veracruz was}\\

\haiku{Maar de ander zei,:}{lachend om de ergernis}{van zijn kameraad}\\

\haiku{Kon hij niet altijd?}{weer een schip vinden wanneer}{het hem berouwde}\\

\haiku{Hij schaamde zich, maar.}{hij wist dat de Cubaan hem}{niet verraden zou}\\

\haiku{Slechts een enkele:}{maal vond hij gezelschap dat}{hem aangenaam was}\\

\haiku{Hij zocht de aarde,,;}{maar niet de mensen dat wist}{hij nu al zeker}\\

\haiku{Rufino haatte,;}{het gezicht daarvan dat hem}{protserig toescheen}\\

\haiku{Hij was dom geweest;}{zonder geleide ook dit}{stuk af te leggen}\\

\haiku{De enige kleine,:.}{zekerheid die hem nog bleef}{was weer teruggaan}\\

\haiku{Zijn weg moest westwaarts,.}{voeren dat was het enige}{wat hij nog wist}\\

\haiku{Tenochtitl\'an, dat {\textquoteleft}{\textquoteright}.}{later Spaans en pronkerig}{Mexico genoemd werd}\\

\haiku{Zo kwam hij, door het,.}{licht geleid waar het centrum}{van de stad moest zijn}\\

\haiku{misschien vond  hij.}{voor deze nacht een bank in}{een van de parken}\\

\haiku{Lang kon hij niet meer, -.}{zonder werk lopen hoogstens}{enkele weken}\\

\haiku{Maar toen hij op straat,,.}{kwam sloeg het licht hem tegen}{evenals het geraas}\\

\haiku{Met een warme blik,;}{een glimlach die het begin}{werd van herkennen}\\

\haiku{een middelpunt van,;}{alle leven van alle}{wereldgebeuren}\\

\haiku{lichtgrijze nevels,.}{die het helle verkleurde}{zonlicht verstoorden}\\

\haiku{Wat was er nog over?}{van de reine hoogvlakte}{die hij verwacht had}\\

\haiku{Ondanks alles was,...}{hij er misschien toch beter}{aan toe geweest daar}\\

\haiku{iemand was die zich,,.}{al was het maar voor kort om}{hem bekommerde}\\

\haiku{Rufino meende,.}{dat zij een vreemdelinge}{moest zijn net als hij}\\

\haiku{- en ook, dat hij  ...}{een portret bezat van het}{meisje van zijn oom}\\

\haiku{ze kon dat met haar.}{ervaring meteen aan het}{soort bezoeker zien}\\

\haiku{Rufino voelde,.}{haar harde lichaam dat warm}{was in zijn handen}\\

\haiku{Ze kon op deze;}{wijze alleen maar van kwaad}{tot erger komen}\\

\haiku{{\textquoteright} Matilde was voor,.}{hem komen staan en schudde}{hem bij de schouders}\\

\haiku{Vervolgens ging hij,.}{met langzame dreunende}{schreden naar buiten}\\

\haiku{Hij had gedacht met,.}{zijn hulp aan haar zichzelf te}{kunnen bevrijden}\\

\haiku{{\textquoteleft}Maar haar Agust{\'\i}n is,.}{een goed zakenman dat moet}{ik hem nageven}\\

\haiku{{\textquoteleft}Het is goed dat je,,}{nu gekomen bent en niet}{later in de avond}\\

\haiku{Rufino begon.}{te geloven dat hij haar}{onrecht had gedaan}\\

\haiku{En hij had zich door;}{zijn illusies omtrent haar}{laten misleiden}\\

\haiku{{\textquoteright} Dan zich plotseling,:}{weer oprichtend zonder de}{doos los te laten}\\

\haiku{Met zijn vaardige.}{linkerhand wierp hij achter}{zich de huisdeur dicht}\\

\haiku{{\textquoteright} Eerst toen gaf hij zich.}{rekenschap dat Rufino}{ongewapend was}\\

\haiku{Daarna wendde hij,:}{zich weer tot Rufino en}{vroeg verachtelijk}\\

\haiku{{\textquoteleft}Haast je wat,{\textquoteright} beval,.}{Agust{\'\i}n zijn revolver weer}{opgericht houdend}\\

\haiku{dat geleefd had in...}{een ver stadje van een streek}{die hij niet kende}\\

\haiku{Niet de lafheid van,,{\textquoteright}.}{een klein flikkerend schiettuig}{bedacht Rufino}\\

\haiku{Je kunt Matilde,,}{de groeten van mij doen als}{je haar nog ooit ziet}\\

\haiku{Het is jammer dat.}{je wegloopt en hier niet in}{zaken bent gegaan}\\

\haiku{Ik reken op je,.}{vriendschap het enige waaraan}{ik nog durf denken}\\

\haiku{Maar daarvan wist zij,.}{niets af en hij kon het haar}{ook niet bijbrengen}\\

\haiku{Niemand wist wanneer,.}{of waar hij schering  was}{en wanneer inslag}\\

\haiku{het werd geboren.}{uit de zee en gaat later}{weer daarin onder}\\

\haiku{{\textquoteright} {\textquoteleft}Ik ben een klant als,.}{iedere andere en}{betaal voor een nacht}\\

\haiku{Hij ging voorzichtig,}{naar binnen om te kijken}{wat er gebeurde}\\

\haiku{Na de ontzetting,;}{van het drama alleen te}{zijn met de dode}\\

\haiku{Weldra viel de nacht,.}{bij het naderen van een}{volgend gebergte}\\

\haiku{de chocolade,.}{ging daar van hand tot hand en}{zij beraadslaagden}\\

\haiku{{\textquoteleft}Denk je dat {\`\i}k me?}{nog veel herinner van waar}{ik geboren ben}\\

\haiku{Niemand sprak een woord.}{toen Rufino zich bij het}{troepje kwam voegen}\\

\haiku{De uitreis was niets,.}{dan zoete-koek geweest}{gesmeerde boter}\\

\haiku{Ik vraag me alleen,,}{af wat de gil was die van}{het achterdek kwam}\\

\haiku{Toch klonk het hem zelf,.}{potsierlijk toe nu hij het}{eenmaal gezegd had}\\

\haiku{Hij hield zich aan de,;}{reling vast toen hij langzaam}{naar achteren liep}\\

\haiku{Hij zei weer tegen;}{zichzelf dat het tenslotte}{maar inbeelding was}\\

\haiku{het was de eerste.}{keer dat iemand hier iets van}{zijn dochter ervoer}\\

\haiku{Bij alle mensen,.}{is het zo maar alleen bij}{mij kun je het zien}\\

\haiku{In dat geval was,.}{hij werkelijk gelukkig}{heel wat dagen lang}\\

\haiku{{\textquoteleft}Integendeel, ik.}{ben het met je eens dat je}{niet voor vrouwen voelt}\\

\haiku{De lichtmatroos riep,:}{hem hoog en beangst zijn nog}{kinderlijke stem}\\

\haiku{Maar Rufino vond;}{het eigenlijk heel rustig}{en plezierig zo}\\

\haiku{Te omhelzen wat,.}{verzwelgt te baren wat voor}{altijd ons omhult}\\

\haiku{Dan zou dat overal,.}{hetzelfde wezen op geen}{enkel schip beter}\\

\haiku{Rufino zat met,.}{het hoofd in de handen op}{de rand van zijn brits}\\

\haiku{{\textquoteright} Rufino's gezicht.}{vertrok zich nogmaals tot een}{pijnlijke glimlach}\\

\haiku{{\textquoteleft}Er is niet altijd,.}{rechtvaardigheid in deze}{wereld mijn jongen}\\

\haiku{Hij keerde terug;}{met de vrouw die hij haar mand}{afgenomen had}\\

\haiku{Nu zwierf hij wie weet,.}{waar en don Cosme maakte}{zich zorgen over hem}\\

\haiku{en de andere,{\textquoteright}.}{zoon is nu weg sprak do\~na}{Anita met een zucht}\\

\haiku{Uw dienaar... en hoogst...}{erkentelijk als ge u}{bij ons wilt zetten}\\

\haiku{Hij denkt dat ik hier,......}{slaap zoals vaker en jij}{in je kamertje}\\

\haiku{Zij vatte hem bij:}{de beide armen en drong}{hem zachtjes terug}\\

\haiku{Maar hier had ze zelfs,...}{geen flauwe kans gezien bij}{gebrek aan vrouwen}\\

\haiku{Opeens kwam het hem;}{voor dat hij dit alles reeds}{tienmaal beleefd had}\\

\haiku{Verwijfd leken die,;}{vreemdelingen zo groot en}{blank als zij waren}\\

\haiku{Al het andere,.}{was bijzaak slechts theater}{en fantasterij}\\

\haiku{Daar moet je vrouw voor,.}{zijn om zo'n subtiel bedrog}{meteen te doorzien}\\

\haiku{Laat ons graven waar,.}{wij staan want waar wij staan is}{altijd Klondyke}\\

\haiku{in zo korte tijd.}{raakte niemand ontwend aan}{het steedse leven}\\

\haiku{Tot elke prijs zou.}{zij trachten voorlopig bij}{Anita te blijven}\\

\haiku{{\textquoteleft}We hadden nooit met.}{deze aangelegenheid}{moeten beginnen}\\

\haiku{Hij liet de waard bij.}{zich komen en vroeg naar het}{adres van een bordeel}\\

\haiku{Het spijt me van dat,,{\textquoteright}.}{litteken Palomino}{zei ze vriendelijk}\\

\haiku{Mijn vriend Rufino....}{L\'opez is ook gebleven toen}{hij er zin in had}\\

\haiku{hij had niet eens de,.}{poncho over zich getrokken}{zich niet uitgekleed}\\

\haiku{De ochtenden hier.}{vervulden haar steeds met een}{wonderlijk ontzag}\\

\haiku{Ook de rechtste lijn,.}{is zo krom dat hij weer tot}{zichzelf terugkeert}\\

\haiku{Alleen Rufino.}{was een kind geweest zonder}{kinderachtigheid}\\

\haiku{Nu zag hij er, zo,.}{stram als hij daar stond meer als}{een militair uit}\\

\haiku{we zullen een heel,.}{stuk kunnen zeilen met uw}{beider welnemen}\\

\haiku{Macario was het,.}{eerst boven rekte zich en}{spiedde om zich heen}\\

\subsection{Uit: Waarom niet}

\haiku{Als hij heelemaal,}{buiten is waar het daglicht}{altijd heller schijnt}\\

\haiku{Hij is de sterkste,.}{van hun drie\"en vast nog veel}{sterker dan Rientje}\\

\haiku{Drie zweetdroppeltjes.}{komen in de bocht van haar}{wipneus te zitten}\\

\haiku{Eensgezind waren,.}{ze alleen in het gevaar}{als ze bang waren}\\

\haiku{Zoolang het leven,.}{zijn gewone gangetje}{ging kibbelden ze}\\

\haiku{Het woord was taboe,;}{maar je mocht het gerust in}{je eentje zeggen}\\

\haiku{Haar wipneusje vroeg,,:}{en ze besliste nog een}{beetje onzeker}\\

\haiku{Daar eindigden de.}{dagen en begonnen de}{nieuwe ochtenden}\\

\haiku{{\textquoteright} Rientje haalde de.}{bladeren uit haar blonde}{verwarde haren}\\

\haiku{Hij liet zijn armen;}{over de oppervlakte van}{het water zweven}\\

\haiku{de groepjes boomen;}{die in het wilde weg hier}{en daar uitstulpten}\\

\haiku{- {\textquoteleft}Fijn is dat gruis daar{\textquoteright},,.}{beneden zei Jan en wees}{naar de fjordenkust}\\

\haiku{De zon scheen bijna.}{loodrecht omlaag en om hen}{heen danste de lucht}\\

\haiku{Zelfs het grassprietje.}{waarmee hij ze plaagde bracht}{ze niet daarvan af}\\

\haiku{Ze stond het dichtst bij.}{Karel en rukte hem zijn}{steen uit de handen}\\

\haiku{Zij timmerde er,.}{op los en Jan verdween luid}{krijtend uit de grot}\\

\haiku{{\textquoteright} - {\textquoteleft}Jouw neus is net zoo{\textquoteright},.}{leuk als een kromme bloem zei}{Karel onverwachts}\\

\haiku{Bovendien konden;}{ze vanaf de grot precies}{zien wat je er deed}\\

\haiku{Een andere kracht.}{echter vocht tegen zijn lust}{om op zee te gaan}\\

\haiku{de gekartelde.}{rand van rotsen waartusschen}{wit waterschuim lag}\\

\haiku{Twee dagen bleef het,.}{zoo voordat het stormen en}{de regen ophield}\\

\haiku{Karel had wel aan,.}{Jan willen vragen waar ze}{was maar hij dorst niet}\\

\haiku{Voor de tweede maal.}{voelden de jongens zich een}{beetje opgelucht}\\

\haiku{hij dacht alleen dat.}{nu de kooi uiteensloeg en}{hij zou verdrinken}\\

\haiku{Maar zijn kleeren plakten.}{stijf en plankerig vast en}{maakten hem benauwd}\\

\haiku{De menschen die hier,.}{wonen weten stellig ook}{hoe weg te komen}\\

\haiku{Maar God die hem tot,.}{hier geholpen had zou hem}{ook verder helpen}\\

\haiku{Een tak met een soort.}{dennenappels roosterde}{en was half verbrand}\\

\haiku{En weer begon hij;}{te hopen dat er toch wel}{menschen zouden zijn}\\

\haiku{Hij zou dat kale,.}{weleens willen aanraken}{als hij maar durfde}\\

\haiku{Maar Rientje had het,;}{zoo besloten en Karel}{was het er mee eens}\\

\haiku{Haar eigen vader;}{zou geen gemakkelijke}{aan  haar hebben}\\

\haiku{de Paps van vreemde,.}{kinderen nog minder dat}{beloofde ze hem}\\

\haiku{Maar de man sloeg er,.}{geen acht op en dat maakte}{Rientje nog veel boozer}\\

\haiku{Zij sprong op en liep.}{hem achterna tot aan de}{ingang van de grot}\\

\haiku{Tenslotte begon:}{ze de jongens allerlei}{verwijten te doen}\\

\haiku{Zij namen ieder,.}{een van de dieren die de}{man ze aanreikte}\\

\haiku{Hij rukte eens aan,.}{het stuk dat af was en liet}{de jongens trekken}\\

\haiku{Karel raapte een;}{paar vezels op en begon}{het ook te probeeren}\\

\haiku{Het scheen dat er toch.}{altijd iemand eenzaam moest}{zijn op het eiland}\\

\haiku{Hij wist dat hij op,;}{deze wijze zwak was en}{niet mocht toegeven}\\

\haiku{En hij trachtte zich,.}{te vermannen zette de}{tanden op elkaar}\\

\haiku{Manuel schudde,;}{het hoofd en bedacht dat hij}{weer dom had gedaan}\\

\haiku{- {\textquoteleft}Neen, het is een hoed{\textquoteright},.}{zei ze toen de man haar het}{schort wou ombinden}\\

\haiku{Wel was hij al flink.}{versleten en zaten er}{overal gaten in}\\

\haiku{{\textquoteright} vroeg Manuel toen. - {\textquoteleft},.}{alles op wasUit de beek}{natuurlijk dommert}\\

\haiku{Die vertelde ook.}{zulke verhalen van een}{orang-weet-ik-veel}\\

\haiku{Het leek wel de stem,.}{van Paps en die stem scheen uit}{de rots te komen}\\

\haiku{Iets donkers kwam naar,.}{voren en de man bad met}{verdubbelde kracht}\\

\haiku{Maar de eerste naam.}{maakte  het spelletje}{aanlokkelijker}\\

\haiku{- {\textquoteleft}Als jij in de grot,{\textquoteright},.}{komt slapen gaan wij in de}{hut zei Rientje slim}\\

\haiku{Hij, de groote-mensch,,,,!}{de dwaas de betweter de}{indringer de tyran}\\

\haiku{dat het van deze.}{wachtpost zou afhangen of}{hij wegkwam of niet}\\

\haiku{Toch merkte ze heel;}{goed dat de man vlak achter}{haar was gekomen}\\

\haiku{Hij had ze dit woord;}{vaak hooren gebruiken voor}{iets verschrikkelijks}\\

\haiku{{\textquoteright} - {\textquoteleft}Je ben eigenwijs,{\textquoteright},.}{je w{\`\i}lt het niet begrijpen}{riep Manuel uit}\\

\haiku{Ze bezorgden hem.}{ook allerlei gemak met}{kleine karweitjes}\\

\haiku{- {\textquoteleft}Als hij de vesten{\textquoteright},.}{meeneemt kunnen we zelf nooit}{meer weg bedacht Jan}\\

\haiku{nu en dan bracht hij.}{de hand boven zijn oogen en}{tuurde naar de zee}\\

\haiku{{\textquoteright} En hij schopte en.}{begon om zich heen te slaan}{als een dolleman}\\

\haiku{De fladderende.}{haren van Rientje voor hem}{uit maakten hem dol}\\

\haiku{In de arm knaagde,.}{een stekende pijn en ze}{hing nog altijd slap}\\

\haiku{Het begon nacht te,;}{worden en ze bleven maar}{op dezelfde plaats}\\

\haiku{Hij was heet, overal,.}{waar je hem aanpakte en}{lag maar te kermen}\\

\haiku{{\textquoteright} - {\textquoteleft}Als hij beter is,{\textquoteright},.}{beginnen we direct aan}{die kuil zei Rientje}\\

\haiku{Jan dacht dat hij met.}{een groote stok gekomen was}{om hem dood te slaan}\\

\haiku{Ik dacht eerst dat het,.}{een dorre tak was en wou}{hem eruit trekken}\\

\haiku{{\textquoteright} - {\textquoteleft}Ik zal er ook over{\textquoteright},.}{denken zei Rientje met een}{gezicht vol modder}\\

\haiku{{\textquoteright} - {\textquoteleft}Nou ja{\textquoteright}, zei Karel.}{die op dat oogenblik geen}{ander antwoord wist}\\

\haiku{Ze kwam tot bij de,.}{grot keek voorzichtig om het}{hoekje naar binnen}\\

\haiku{Hij bleef ook niet lang.}{aandringen dat zij in de}{grot zou overnachten}\\

\haiku{hij trok er zich niets;}{van aan dat Jan nu met een}{lamme arm rondliep}\\

\haiku{Er bestond ook geen.}{twijfel omtrent de plaats die}{ze had aangeduid}\\

\haiku{Het gelig beetje,;}{licht dat door de nevels scheen}{kroop alweer lager}\\

\haiku{Ze konden nog zien,.}{waar Manuel geloopen}{had zijn groote stappen}\\

\haiku{Het was veel beter.}{om nu maar een onschuldig}{gezicht te trekken}\\

\haiku{het had hem geleerd.}{om op zijn hoede te zijn}{en snel te vluchten}\\

\haiku{Op zoo'n klein eiland.}{zou het ventje hem toch niet}{kunnen ontloopen}\\

\haiku{Maar het kon ook iets,.}{anders zijn wat hij niet wist}{iets ongeneeslijks}\\

\haiku{Hij droomde dat het.}{verpletterd werd onder een}{geweldig rotsblok}\\

\haiku{Even maar, en hij zou,.}{het wanhopigste doen het}{uiterste riskeeren}\\

\haiku{je ging tenlaatste de.}{dingen zien waarop je zoo}{hardnekkig hoopte}\\

\haiku{Onderwijl stonden.}{Karel en Rientje maar met}{groote oogen te kijken}\\

\haiku{{\textquoteright} Hij was veel te bang.}{dat Manuel zich misschien}{nog bedenken zou}\\

\haiku{{\textquoteright} - {\textquoteleft}Morgen ga ik mijn{\textquoteright},.}{sabel halen liet hij er}{meteen op volgen}\\

\haiku{Ze vond dat jongens.}{nooit erg veel verder denken}{dan hun neus lang is}\\

\haiku{Val me niet lastig,.}{met al die histories met}{de  kinderen}\\

\haiku{Ik moet beginnen,,.}{te schreeuwen te wuiven dat}{ze mij opmerken}\\

\haiku{Vroeger dachten ze,.}{een paar duizend maar dat is}{nonsens natuurlijk}\\

\haiku{- {\textquoteleft}Vraag of dat eiland{\textquoteright},.}{heelemaal onbewoond was}{zei de kapitein}\\

\haiku{In Durban vind je.}{zeker een paar landslieden}{en jullie consul}\\

\haiku{Om een jongen die;}{nog steeds niet met zijn aanzoek}{op de proppen komt}\\

\haiku{dit is de eerste,.}{vrouw sedert mijn terugkomst}{die ik z\'o\'o bekijk}\\

\haiku{Je kunt net zoo goed.}{met ongewasschen voeten}{een meisje aaien}\\

\haiku{Eerst insisteerde.}{Lola om er meer van te}{weten te komen}\\

\haiku{Ik wou dat je wat{\textquoteright},.}{ego{\"\i}stischer dacht viel Gil}{hem in de rede}\\

\haiku{Integendeel, het;}{huwelijksaanzoek was haar}{een pak van het hart}\\

\haiku{Als je nu naar al,.}{die booten toe moet kan het}{je te pas komen}\\

\haiku{ik heb het immers,;}{niet gewild het kind moet het}{mij maar vergeven}\\

\haiku{Waarom moet ik nu?}{al gestraft worden dat ik}{hem niet gewild heb}\\

\haiku{Als het niet iets heel,.}{gewoons was had ik immers}{mijn mond gehouden}\\

\haiku{{\textquoteright} Manuel stond aan,,.}{de andere kant van de}{tafel bleek nerveus}\\

\haiku{hoeveel tijd zal er?}{noodig zijn voordat ik ervan}{zal kunnen houden}\\

\haiku{Hij ging werkelijk.}{naar de slaapkamer waar de}{wieg stond met het kind}\\

\haiku{Maar waarom zeggen,?}{die kinderen hetzelfde}{precies hetzelfde}\\

\haiku{Ik heb hem nadien,,.}{maar eens gezien en hij had}{een lam smal armpje}\\

\haiku{Je hebt een duivel,.}{uit mijn lichaam gerukt een}{duivel van verdriet}\\

\haiku{Trek ze toch uit, je,.}{bent almachtig nu en ik}{sterf aan deze pijn}\\

\haiku{Maar nog had hij geen,.}{besluit kunnen nemen wist}{hij niet wat te doen}\\

\haiku{Manuel kon er.}{niet toe besluiten zich nog}{meer bloot te geven}\\

\haiku{Met hem kunnen wij,.}{opschieten hij vindt al die}{gevoelens onzin}\\

\haiku{{\textquoteright} - {\textquoteleft}Het hangt er van af.}{waar je zelf gaat staan om die}{zaak te bekijken}\\

\haiku{{\textquoteright} - {\textquoteleft}Heb je nooit gemerkt?}{dat iets in ons zich verzet}{om na te denken}\\

\haiku{Natuurlijk vroeg ze. - {\textquoteleft}{\textquoteright},.}{of hij ook van reizen hield}{Neen zei Manuel}\\

\haiku{Jammer genoeg, zou,.}{vader zeggen ofschoon hij}{er nooit geweest is}\\

\haiku{Je zou in staat zijn.}{om uit wraak stilletjes met}{ons weg te varen}\\

\haiku{Wat is er precies?}{met die verloren korrels}{aarde aan de hand}\\

\haiku{- {\textquoteleft}Ik wed dat het een{\textquoteright},.}{citaat is uit Epictetus}{juichte Edwina}\\

\haiku{Lang stond Manuel,:}{ze nog na te wuiven tot}{Gil eindelijk zei}\\

\haiku{{\textquoteright} - {\textquoteleft}Zoomin als jij een{\textquoteright},.}{volmaakt anarchist plaagde}{Manuel terug}\\

\haiku{En het was alsof.}{ze hier op de boot eens zoo}{gek eruit zagen}\\

\haiku{Zij bedacht niet dat.}{het meisje ongeveer zou}{oud moest zijn als zij}\\

\haiku{Hij voelde nogmaals.}{of hij zijn revolver bij}{zich had gestoken}\\

\haiku{Dat ziet eruit als,.}{een schietstrik of het moet een}{verdraaid toeval zijn}\\

\haiku{{\textquoteleft}Zelfs tegen onze.}{dwaasheden is hun vernuft}{niet opgewassen}\\

\haiku{Als medemenschen.}{hebben wij geen enkel recht}{meer tegenover hen}\\

\haiku{Hij was mokkend in,}{de hoek gaan staan waar hij met}{heel scherp toekijken}\\

\haiku{als je ook met een....}{heele troep kerels op ze}{afkomt geen wonder}\\

\haiku{- {\textquoteleft}Ik geloof dat w{\`\i}j{\textquoteright},.}{wel vrienden zullen worden}{zei de oude man}\\

\haiku{{\textquoteright} De golven komen;}{en nemen het schip op en}{dragen het verder}\\

\haiku{{\textquoteleft}Ik ken daar een man,.}{die op ons wacht en die wel}{voor ons zorgen zal}\\

\haiku{hij was, hoe prachtig,,...}{hij zijn armen bewegen}{kon springen duiken}\\

\haiku{We hebben het hem{\textquoteright},.}{dikwijls genoeg gezegd zei}{Karel dan telkens}\\

\haiku{Er is geen erger.}{overwonnene dan de man}{die niets meer verwacht}\\

\haiku{En dan moest Rientje:}{tenslotte toegeven en}{uit zichzelf zeggen}\\

\haiku{Drie jaar later werd,.}{een jongen geboren die}{Jan Willem heette}\\

\haiku{Nu weet ik waarvoor.}{ik al deze jaren heb}{moeten doormaken}\\

\haiku{Ik begrijp...{\textquoteright} - {\textquoteleft}Op de{\textquoteright},.}{knie\"en zou ik u moeten}{danken zei Voorberg}\\

\haiku{Hij begreep dat de.}{vader hem dat waarschijnlijk}{kwalijk zou nemen}\\

\haiku{Ik kan in weelde;}{mij natuurlijk niet meten}{met Lord Curdington}\\

\haiku{{\textquoteleft}Kom-kom, wie zelf.}{zijn fouten inziet heeft ze}{al half verbeterd}\\

\haiku{{\textquoteleft}Je weet toch dat de?}{isra\"elieten niet aan}{den Christus gelooven}\\

\haiku{Je houdt van iemand,.}{of je houdt niet van hem daar}{kun je niets aan doen}\\

\haiku{Ik hield meteen van.}{Winny en van Edgar en}{van de kapitein}\\

\haiku{Maar als ik ze niet,.}{begrijpen kan heb ik er}{niets mee te maken}\\

\haiku{De god tot wie ze,;}{baden rook naar kamfer en}{zat \`onder de stof}\\

\haiku{{\textquoteright} - {\textquoteleft}Het kan me schelen.}{omdat de banden van het}{bloed ons verbinden}\\

\haiku{Je hebt geen enkel,,.}{recht want ik heb niets van je}{en niets van je noodig}\\

\haiku{En dat is hoog noodig,,.}{want veel soeps is het niet wat}{je om je heen ziet}\\

\haiku{wij beiden hierom,.}{en veroordeelden het als}{zeer onwelvoeglijk}\\

\haiku{Het is onzin om.}{in belangrijke dingen}{niet rechtuit te zijn}\\

\haiku{{\textquoteright} - {\textquoteleft}Er zijn menschen die;}{spoorwegen bouwen uit een}{soort van amusement}\\

\haiku{Het hoogste wat wij.}{bereiken kunnen is het}{weten hoe ze zijn}\\

\haiku{Maar het zal eerder,.}{goed zijn dan slecht dat we van}{elkander houden}\\

\haiku{Hij wilde na een,;}{groet gewisseld te hebben}{haastig doorloopen}\\

\haiku{Wat kon een jonge,?}{onwijze knaap als Paco}{daarvan begrijpen}\\

\haiku{Maar een bepaalde,,?}{tactiek een bepaalde lijn}{geven ze die aan}\\

\haiku{{\textquoteleft}... beken jezelf tot...:}{het proletariaat durf}{te zijn wat je bent}\\

\haiku{Ze maken je zelf,.}{tot hun tegenstander die}{verdommelingen}\\

\haiku{aan menschenliefde.}{ontspruit onze haat tegen}{de onderdrukkers}\\

\haiku{Elk plantje moet zijn.}{plekje hebben en ieder}{beest zijn holletje}\\

\haiku{En opeens, als kwam,:}{hij eindelijk tot zichzelf}{zei hij resoluut}\\

\haiku{Koeprow woonde niet,.}{anders dan de anderen}{niet anders dan Jan}\\

\haiku{Koeprow had het ook:}{wel eens als een verwijt te}{hooren gekregen}\\

\haiku{De ander, gewend,;}{aan Koeprow's manier van doen}{doorstond ze rustig}\\

\haiku{Er waren een paar,.}{aardige kameraden}{daar anarchisten}\\

\haiku{Sinds ik partijlid,.}{ben praat ik immers niet meer}{over vrijbuiterij}\\

\haiku{Bolsjewisme moet,.}{je in het bloed zitten in}{het protoplasma}\\

\haiku{- {\textquoteleft}Dat beteekent koppen{\textquoteright},,.}{zei Iwan en hij maakte een}{gebaar naar zijn hals}\\

\haiku{In een millioen.}{jaren heeft de menschheid}{nog niets bijgeleerd}\\

\haiku{Hoe kon ze over hen.}{praten terwijl het verdriet}{nog zoo versch schrijnde}\\

\haiku{{\textquoteleft}Ook wij gelooven, om,{\textquoteright},.}{te kunnen vechten vader}{Rosenblum zei hij}\\

\haiku{Zijn verbeelding was.}{reeds bezig met de nieuwe}{taak die hem wachtte}\\

\haiku{{\textquoteright} - {\textquoteleft}We zijn met veel te.}{weinigen om nog hard van}{stapel te loopen}\\

\haiku{en dit zou dan de.}{plek wezen waarop Jan zijn}{aanval kon richten}\\

\haiku{van nabij had hij.}{nooit veel sterk effect gezien}{van al dat geschrijf}\\

\haiku{Een waarheid die geen,;}{geld oplevert is niets waard}{voor de krantenman}\\

\haiku{een luizenleger.}{van millioenen kleine}{gemeene leugens}\\

\haiku{{\textquoteleft}Je onderschrijft het,}{negatieve programma}{je doet mee zoolang}\\

\haiku{Onbewust merkte;}{ook Jan dit spiegelbeeld van}{zichzelf aan haar op}\\

\haiku{{\textquoteright} Jan had niet zooveel.}{vertrouwen in die klasse}{van medemenschen}\\

\haiku{Ik ben een man uit,?}{het publiek en wat is een}{krant zonder publiek}\\

\haiku{Waarschijnlijk hielden.}{ze er in het geheel geen}{theorie op na}\\

\haiku{Wat wij nastreven.}{moet in de eerste plaats een}{mogelijkheid zijn}\\

\haiku{- {\textquoteleft}Wat men ze geven,,:}{moest inplaats van geld is een}{briefje waarop staat}\\

\haiku{Deze Hollandsche?}{episode zou dus nu reeds}{afgeloopen zijn}\\

\haiku{{\textquoteleft}Luister  eens, je.}{hebt die zoon van mij een heel}{eind ver gekregen}\\

\haiku{{\textquoteright} Toch klampt ze zich aan,.}{de mogelijkheid van dit}{toeval vast dacht Jan}\\

\haiku{Ongenaakbaar voor,.}{al het kleine gemeene}{en vuile rondom}\\

\haiku{Reeds lang deden ze;}{daar hun best de koning een}{beentje te lichten}\\

\haiku{Als je hard ging, liep;}{je vast tusschen de dijken}{en de duinen hier}\\

\haiku{Men komt zelden in,.}{het kamertje waar men niet}{van oudsher thuishoort}\\

\haiku{{\textquoteright} vroeg Jan zuurzoet en.}{in de parabel-stijl}{die hier mode was}\\

\haiku{{\textquoteright} - {\textquoteleft}En dacht je soms dat?}{ze in werkelijkheid niet}{op elkaar leken}\\

\haiku{Anders sukkelen.}{we in enkele maanden}{tien jaar achteruit}\\

\haiku{ze schreef dat zij met;}{haar handelsdelegatie}{naar Parijs zou gaan}\\

\haiku{Waarom zijn jullie?}{noorderlingen toch zoo van}{vrouwen geschrokken}\\

\haiku{{\textquoteright} - {\textquoteleft}Zelfs nu behoeft het{\textquoteright},.}{er niet te zijn zei Jan een}{beetje achteloos}\\

\haiku{{\textquoteright} - {\textquoteleft}Dat is het juist{\textquoteright}, zei.}{Gil terwijl hij een gebaar}{van wanhoop maakte}\\

\haiku{Zon, palmen, felle,,...}{kleuren eindelooze ruimten}{van land zee en lucht}\\

\haiku{hoe anders dan ik.}{denk zou de werkelijkheid}{misschien geweest zijn}\\

\haiku{{\textquoteright} - {\textquoteleft}Zoolang ik je ken,.}{is het of je zoekt naar iets}{wat onvindbaar blijft}\\

\haiku{Daarom begon Jan,:}{maar te vertellen hoe het}{alles geweest was}\\

\haiku{zelfs de gitaar was.}{in zijn onzichtbare tuin}{in slaap gevallen}\\

\haiku{met behulp van de.}{arbeid van een paar honderd}{arme inlanders}\\

\haiku{In Engeland, of,.}{als je met alle geweld}{zou willen ook hier}\\

\haiku{Jan begon driftig.}{op en neer te loopen toen}{de ander weg was}\\

\haiku{en altijd bleek dat.}{je in je wezenlijkste}{verlangens faalde}\\

\haiku{En nu vooral, nu.}{ik aan zooveel andere}{dingen moet denken}\\

\haiku{Hoe zeer moest ze hem.}{dan niet aan zich gebonden}{en geknecht hebben}\\

\haiku{Het zijn mystici, -.}{of desperado's of}{beide tegelijk}\\

\haiku{{\textquoteright} - {\textquoteleft}Je liefde voor de.}{partij was dus grooter dan}{je liefde voor Jan}\\

\haiku{Ze vond het gek van.}{zichzelf tot deze slotsom}{gekomen te zijn}\\

\haiku{Tot zoo lang is het.}{beter al die dingen uit}{de weg te blijven}\\

\haiku{Je hebt nog liever.}{die hut stukgemaakt en ons}{plezier bedorven}\\

\haiku{Anders sterven we.}{aan het gevaar van onze}{eigen wapenen}\\

\haiku{{\textquoteright} Van wie het nieuwe,.}{plan afkomstig was kon hij}{moeilijk  zeggen}\\

\haiku{Hij heeft er zich dood,.}{op geknutseld maar drie van}{die dingen zijn klaar}\\

\haiku{lieden die zich een.}{geluk droomen dat niet van}{deze wereld is}\\

\haiku{{\textquoteright} vroeg Jan, zonder nog.}{zijn blikken af te wenden}{van de zoldering}\\

\haiku{Jan verdedigen,.}{wilt ook terwijl hij niet meer}{verdedigbaar is}\\

\haiku{hem, terwille van,.}{jou zal ik hem verdragen}{wat er ook gebeurt}\\

\haiku{De rest komt er niet,.}{op aan een menschenleven}{komt er niet op aan}\\

\haiku{Alles heeft dwaze,.}{namen de eene is zoo slecht}{als de andere}\\

\haiku{Maar nu, op deze.}{eigen oogenblikken moest}{het toch gebeuren}\\

\haiku{Zelfs met de wind mee.}{was dat onmogelijk in}{deze drukke stad}\\

\haiku{Dan drukte hij, nog,}{steeds de linker hand in zijn}{zak de knop zoo diep}\\

\haiku{Hoe was dan z\'o\'o snel?}{de revolver uit zijn zak}{omhoog gekomen}\\

\haiku{Waren nu reeds uit?}{die massa enkelen op}{hem losgelaten}\\

\haiku{Dan keek hij overal,;}{om zich heen er was niemand}{in de nabijheid}\\

\haiku{Hard trok hij de deur,.}{achter zich dicht dat het lang}{na-dreunde in huis}\\

\haiku{Als ik vooreerst maar,.}{wegkom uit Madrid dan zie}{ik wel weer verder}\\

\haiku{dat is het eenige:}{wat je te doen hebt als vrij}{en redelijk mensch}\\

\haiku{Over de vlakke weg,.}{die nu volgde hernam hij}{het dolle tempo}\\

\haiku{Alle andere.}{zijn verloren droomen of}{nog ongeboren}\\

\haiku{wat lang voor de eerst.}{bekende geschiedenis}{op aarde plaatsgreep}\\

\haiku{Straks zou het alweer,.}{te laat zijn dan kwam al het}{andere terug}\\

\haiku{En ze durfde nu.}{niet te telefoneeren met}{Manuel's woning}\\

\haiku{{\textquoteright} - {\textquoteleft}In naam van vand\'a\'ag zijn.}{er offers die we met recht}{mogen weigeren}\\

\haiku{Rien had niet meer de;}{moed om aan te dringen dat}{Mirjam heen zou gaan}\\

\haiku{{\textquoteleft}Te wonen in wat,.}{er buiten ligt zal meer pijn}{doen dan hier te zijn}\\

\haiku{Maar zijn werk was er,,?}{toch nog zijn idee het beste}{en edelste aan hem}\\

\haiku{Alleen, voor haar was;}{hij nooit enkel de drager}{van een idee geweest}\\

\haiku{D\`an zal ik danken,:}{wanneer er dank is verdiend}{en ik zeggen kan}\\

\haiku{{\textquoteright} - {\textquoteleft}Dat wat wij toeval,.}{noemen is misschien juist de}{zin van ons leven}\\

\haiku{Elk had zijn deel in,.}{het werk in het slagen en}{in de mislukking}\\

\haiku{{\textquoteright} Rien ving zijn gezicht.}{binnen de glansboog van haar}{groote blauwe blikken}\\

\haiku{Ik moet hem toch zijn;}{plaats en zijn kans geven in}{dit ondermaansche}\\

\haiku{zoo was zijn aard, dat.}{had hij noodig om zichzelf te}{verwezenlijken}\\

\haiku{zulke vrouwen moest...}{ik in mijn leven nog eens}{kunnen ontmoeten}\\

\haiku{Vroeger ben ik vaak}{terneergeslagen geweest}{omdat ik bang was}\\

\haiku{Het gaf haar iets van.}{een boerin en iets van een}{wild ongetemd dier}\\

\haiku{Het is een soort van.}{idealisme waarvoor ik}{geen begrip meer heb}\\

\haiku{moeten wij ze ook,,?}{nog op papier in verf in}{marmer namaken}\\

\haiku{En dan moet ik die,.}{droom maar weer opschrijven om}{te kunnen leven}\\

\haiku{Als ik leerlooier.}{was of landlooper zou ik}{het \'o\'ok weigeren}\\

\haiku{Het eenige wat wij,.}{meenemen op de vlucht is}{onze oprechtheid}\\

\haiku{Elk besef van tijd,.}{heeft opgehouden er is}{slechts aanwezigheid}\\

\haiku{met het spelletje.}{van de conjecturen was}{het evenzoo gesteld}\\

\haiku{Het was altijd heel,,.}{nabij geweest en toch ik}{heb het niet herkend}\\

\haiku{En bij haar laatste,,:}{zwijgen nadat ook dit was}{gezegd  dacht ik}\\

\haiku{je denkt aan afscheid,;}{ouderdom en doodgaan met}{een voldaan gevoel}\\

\haiku{wat was er anders,?}{noodig om gelukkig dankbaar}{en voldaan te zijn}\\

\haiku{mijn bizonder deel.}{van uitgestooten-zijn}{dat ik moet dragen}\\

\haiku{Toen ik haar begon,,.}{te volgen werd ze grooter}{groeide tot een berg}\\

\haiku{Zoolang... zoolang... tot,.}{later het geschikte uur}{zou komen later}\\

\haiku{En dan, laat ik je.}{n\`ogmaals herinneren aan}{Robinson Cruso\"e}\\

\subsection{Uit: Waar is Vrijdag gebleven?}

\haiku{Een ernstig conflict.}{tussen de naburen was}{onvermijdelijk}\\

\haiku{De juiste toedracht.}{van zaken valt moeilijk meer}{te achterhalen}\\

\haiku{en als men hem niet, (...).}{had laten gaan zou hij in}{zee gesprongen zijn}\\

\haiku{het achterwege,{\textquoteright}.}{te laten ofschoon hij het}{ten slotte wel deed}\\

\haiku{{\textquoteleft}in vrede{\textquoteright} zijn de.}{twee allerlaatste woorden}{in Crusoe II}\\

\subsection{Uit: Zomaar wat kinderen}

\haiku{de mooie nieuwe fiets.}{van meester Tjon Sie Kwie te}{mogen schoonpoetsen}\\

\haiku{{\textquoteright} Het was laat in de.}{namiddag eer zij terug}{in de stad waren}\\

\haiku{Ik zie je nooit iets,.}{lezen je doet alsof er}{geen boeken bestaan}\\

\haiku{Zelfs al had Pa {\textquoteleft}met{\textquoteright},.}{geen sterveling gezegd op}{zo'n dreigende toon}\\

\haiku{Ik zal je maar niet,.}{zeggen hoe want dat kan je}{toch niet begrijpen}\\

\haiku{{\textquoteleft}Zet ze maar achter!}{het huis in die baskiet met}{gebroken kleren}\\

\haiku{{\textquoteleft}Al zit je vol, je,?}{kunt toch wel een plaatsje voor}{hem inruimen h\`e}\\

\haiku{de meesten echter,.}{liepen onverschillig al}{pratend lang hem heen}\\

\haiku{Of zoals je dat {\textquoteleft}{\textquoteright}.}{zelfs bijhalve-Chinezen}{onmiddellijk zag}\\

\haiku{{\textquoteleft}Er zijn in onze.}{groep ook drie of vier Turkse}{jongens en meisjes}\\

\haiku{Nou, met zo iemand,.}{kon je toch nooit omgaan waar}{hij ook vandaan kwam}\\

\haiku{zo lekker smaakte.}{hem deze schotel nog in}{zijn herinnering}\\

\haiku{Anders verlies ik,?}{misschien deze baan en wat}{beginnen we dan}\\

\haiku{Dit moest sneeuw zijn, wist,.}{hij maar nog niet hoe het er}{van nabij uitzag}\\

\haiku{Van Annemiek die {\textquoteleft}!}{nog steeds met hem aanpapte}{en ook altijdDoei}\\

\haiku{Misschien kom ik nog,}{wel eens kijken bij jullie}{in Suriname}\\

\subsection{Uit: Zuid-Zuid-West}

\haiku{eenzaam te zijn, want.}{alleen de eenzame geeft}{acht op de stilte}\\

\haiku{Arme Inca, die.}{vijftien eeuwen wachten moest}{voor het stalleke}\\

\haiku{Geeft rekenschap van,.}{uw rentmeesterschap gij die}{het goud begeerd hebt}\\

\haiku{Zijn ogen spiegelen;}{zich in het kalme zwarte}{water der kreken}\\

\haiku{Zij leggen die over,.}{elkaar en weten niet dat}{het steeds een kruis wordt}\\

\haiku{Het hart van dit land;}{is een stille kern waar geen}{geluid meer kan zijn}\\

\haiku{{\textquoteright} En in zijn blijde.}{verwondering citeerde}{hij ganse verzen}\\

\haiku{Na een paar uren kwam.}{de heerlijke sensatie}{van het uitstappen}\\

\haiku{Geen van de planten,.}{die groeien in dit zand zijn}{te zien in de stad}\\

\haiku{De verpoeierde.}{regen was zilver in de}{ongeboren dag}\\

\haiku{De dag is een luie,{\textquoteright}.}{en trage mulat zei hij}{tegen de bomen}\\

\haiku{hoe alles groeide,,.}{hoe groot het land was dat nu}{weer ontgonnen werd}\\

\haiku{De zon scheen de glans.}{van versgebakken brood over}{zijn kop en handen}\\

\haiku{Nu is het als een,,.}{wijde vijver waar ik wacht}{in een schaduwplek}\\

\haiku{Dan zag je de nacht,.}{door het venster als een heel}{lelijk zwart gezicht}\\

\haiku{terwijl je wachten.}{moest onder een balkon werd}{ze zelfs feestelijk}\\

\haiku{Maar binnen in het.}{gesloten huis was alles}{helemaal anders}\\

\haiku{Ik bid u, denk met.}{een liefdevol hart aan de}{jeugd van Maldoror}\\

\haiku{ik herinner mij.}{eigenlijk geen enkel feest}{in ons oude huis}\\

\haiku{Dit alles hangt heel,,.}{nauw samen met het huis met}{de stad met het land}\\

\haiku{Wie bevreesd is, kan.}{nimmer het binnenste van}{mijn land betreden}\\

\haiku{Uitgeteerd keren,.}{ze stadwaarts en hun schat is}{na \'e\'en dag een droom}\\

\haiku{Onontkoombaar zijn.}{wij ingesloten in zulk}{een benauwenis}\\

\haiku{{\textquoteleft}Op de toppen des}{levens ben ik gestegen}{om u te zingen}\\

\haiku{Een fijngekauwde;}{wortel spuwen zij allen}{in de lege boot}\\

\haiku{De donder verliest.}{zich in deze ruimte tot}{een klein gemorrel}\\

\haiku{Uit zijn schuilplaats komt,,.}{een tijger te voorschijn en}{gluurt naar links naar rechts}\\

\haiku{Ik schoof de grijsaard.}{weg van de piano en}{speelde de tango}\\

\haiku{En een kind dat te,.}{jong op reis gaat komt soms te}{zeer als man terug}\\

\haiku{De stad beweegt zich,.}{om een lichaam de zee woelt}{rond een dood lichaam}\\

\haiku{Niet in de landen,.}{zijn wij vreemdelingen maar}{in elkanders droom}\\

\haiku{Kropina-kreek,.}{een kleine zijtak van de}{Para-rivier}\\

\haiku{Hij schreef ook enige.}{merkwaardige boeken tot}{hun verdediging}\\

\subsection{Uit: Zusters van liefde}

\haiku{het was niet besteed,....}{aan zo'n wezen geen parels}{voor zulke nou ja}\\

\haiku{Roerloos, ongestoord.}{door het binnenkomen van}{de twee bezoekers}\\

\haiku{Isolatie in een.}{afzonderlijke barak}{was onafwendbaar}\\

\haiku{Toen al waren zijn,.}{gedachten bij dat meisje}{maar hij vroeg nog niets}\\

\haiku{Hij merkte dat haar,,.}{haren gitzwarte nog niet}{waren afgeknipt}\\

\haiku{Want inderdaad moesten,.}{dat er meerdere geweest}{zijn stelde men vast}\\

\haiku{Maar t\'och, ook zo in.}{zijn eentje doet een jong mens}{veel ervaring op}\\

\haiku{Je ziet het ze al:}{schrijven en je hoort het de}{mensen al zeggen}\\

\haiku{Vertrouwensman van,.}{de Cristero's of de hemel}{mag weten wat meer}\\

\haiku{haar werkzaamheden,}{kon hij niet beoordelen}{die kende zij zelf}\\

\haiku{Daarom heb ik je?}{immers aangeraden je}{daar te vestigen}\\

\haiku{Een staat binnen de.}{staat is onduldbaar volgens}{elke logica}\\

\haiku{Hij is moeilijk voor....}{iedereen en eigenlijk}{mag je niet klagen}\\

\haiku{Hoe dikwijls ben ik,{\textquoteright}.}{zelf niet opnieuw begonnen}{troostte Amaral hem}\\

\haiku{{\textquoteleft}Nou, de afloop van.}{deze boevenstreek kunt u}{zich wel voorstellen}\\

\haiku{Zonder veel geestdrift,.}{maar ook zonder merkbare}{onvriendelijkheid}\\

\haiku{Anders zat je hier.}{nu niet te luisteren naar}{mijn gelamenteer}\\

\haiku{Maar ondertussen,.}{zit uw vrouw in spanning te}{wachten Manolo}\\

\haiku{Je ziet, het zijn en.}{blijven toch heidenen in}{hun diepste wezen}\\

\haiku{op hun manier het.}{dodenfeest vieren voor de}{gestorven Christus}\\

\haiku{nieuw inheems broedsel,....}{zoals ik zelf en Lino}{en wie je maar wilt}\\

\haiku{Wat zij te doen had,;}{was zichzelf en ook Lino}{de tijd te gunnen}\\

\haiku{De wijsheid woont bij,.}{het simpele volk niet bij}{de professoren}\\

\haiku{Daar was het mij \'o\'ok,.}{om begonnen hoewel niet}{op de eerste plaats}\\

\haiku{degene die hij,?}{naar de kliniek gebracht had}{waaruit zij vluchtte}\\

\haiku{Want de landmeter.}{toonde zich uitermate}{nerveus en beangst}\\

\haiku{En om zijn vriend in,:}{dezelfde trant van repliek}{te dienen zei hij}\\

\haiku{Wat natuurlijk nooit.}{gebeurt als deze dingen}{hier blijven doorgaan}\\

\haiku{{\textquoteleft}Laten we in 's;}{hemelsnaam ophouden over}{dat nonnengedoe}\\

\haiku{Iets verder hingen:}{ook allerlei andere}{folterwerktuigen}\\

\haiku{{\textquoteleft}Maar wat versta je?}{dan onder het vinden van}{een veilige weg}\\

\haiku{Hun gesprek stokte.}{dan ook verder en liep even}{later ten einde}\\

\haiku{Hopelijk zou hij;}{Isidro daar zonder veel moeite}{weten te vinden}\\

\haiku{Wie het land bebouwt,.}{weet er alles van en houdt}{er rekening mee}\\

\haiku{Maar mensen uit de.}{stad zijn alles vergeten}{wat van belang is}\\

\haiku{{\textquoteleft}We kwamen er pas,{\textquoteright}.}{achter door een stom toeval}{ging Isidro opeens voort}\\

\haiku{Wat tegenwoordig,.}{verboden was behalve}{voor de navelstreng}\\

\haiku{Toch wil ik er nog,{\textquoteright}, {\textquoteleft}.}{weieens naar toe zei Hector}{maar dat heeft de tijd}\\

\haiku{Hij deed zijn zegje,,{\textquoteright}.}{hard en duidelijk dat vind}{ik leuk sprak Chole}\\

\haiku{H\'a\'ar scheppingsdrang is.}{een andere dan die van}{de meeste mannen}\\

\haiku{Maar die andere.}{naam heeft de zuigelingen}{geen geluk gebracht}\\

\haiku{{\textquoteleft}Wij houden onze,.}{rancho's opzettelijk klein}{zoals ze nu zijn}\\

\haiku{Nog maar twee of drie.}{zittingen schatte hij en}{het zou voltooid zijn}\\

\haiku{Maar hoe Chole dit.}{alles zo aanstonds opneemt}{is mijn grootste zorg}\\

\haiku{De kamer met het;}{beste licht bestemde hij}{meteen tot atelier}\\

\haiku{Daarenboven was.}{zij hier in dit gehucht van}{onschatbaar veel nut}\\

\haiku{oudere vrouw was.}{al opgestaan toen hij met}{zijn relaas begon}\\

\haiku{{\textquoteleft}Se\~nor,{\textquoteright} zei ze, nu, {\textquoteleft}.}{bijna fluisterendik moet}{u nog wat zeggen}\\

\haiku{{\textquoteright} Na hem even zwijgend,:}{te hebben aangezien ging}{Leocadia voort}\\

\haiku{En misschien zou hij,,.}{haar als het meezat toch nog}{kunnen bepraten}\\

\haiku{De enige kerk van,.}{de geheel verlaten plaats}{zoals hij merkte}\\

\haiku{ook het kerkhof waar.}{al tientallen lichtjes bij}{de graven brandden}\\

\haiku{Zo lang zij hem nog,.}{dienen kon vormden zij een}{ware twee\"eenheid}\\

\haiku{Hij voelde zich niet.}{al te zeer bezwaard door haar}{aanhankelijkheid}\\

\haiku{{\textquoteright} veinsde Hector met.}{het onschuldigste gezicht}{dat hij kon trekken}\\

\haiku{Wat niet wegnam dat.}{er soms heel vervelende}{dingen gebeurden}\\

\haiku{Hij was er slecht aan,}{toe en is naar het grote}{hospitaal vervoerd}\\

\haiku{En schenk ons een goed,...{\textquoteright} {\textquoteleft}?}{leven een nuttig bestaan}{Is dat niet prachtig}\\

\haiku{Nu moest ze gaan, er,...}{wachtten zoveel anderen}{zoals Ena wel wist}\\

\haiku{Ofschoon hij in zijn.}{eentje al moeite genoeg}{had rond te komen}\\

\haiku{{\textquoteleft}Laat toch eens zien wat.}{je hier zoal gemaakt hebt}{in deze chaos}\\

\haiku{Waarop Laurette,:}{alsof zij hem begreep en}{zich excuseerde}\\

\haiku{hoe hij dit doen kon:}{en keek verrast op toen zijn}{vrouw plotseling zei}\\

\haiku{{\textquoteleft}Het is hier niet groot.}{en er is niets anders in}{dit huis beschikbaar}\\

\haiku{Samen gingen zij;}{hun eerste inkopen doen}{in de stadsdrukte}\\

\haiku{Maar met haar zuiver...}{instinct stelde zij immers}{zuivere daden}\\

\haiku{Ik zou hier misschien...{\textquoteright}.}{Ena's ogen keken de kant uit}{van het lazaret}\\

\haiku{Zij behielden hun.}{waardigheid totdat zij uit}{het gezicht waren}\\

\haiku{Een vreemde dame,...}{eigenlijk maar ijverig}{genoeg en nuttig}\\

\haiku{Ze begrijpen niet.}{dat de natuur veel helpers}{voor ons heeft klaarstaan}\\

\haiku{Het leek wel alsof,.}{hij altijd iets in het schild}{voerde jazeker}\\

\haiku{Intussen steeg de}{drukte op het marktplein waar}{tientallen kraampjes}\\

\haiku{Alleen wat de Kerk.}{aangaat geef ik toe dat je}{gelijk kunt hebben}\\

\haiku{Ik heb vanavond nog,,}{veel te doen maar moet jullie}{dringend iets melden}\\

\haiku{Dit was immers iets.}{van nog veel groter belang}{dan een Sint-Jansfeest}\\

\haiku{Of hij soms meende?}{dat men op een feestelijk}{pleziertochtje ging}\\

\haiku{Niet allen echter.}{genoten toen al van zo'n}{vreedzame sluimer}\\

\haiku{Wat moest hij van haar,?}{nu hij immers wist dat zij}{toch maar een vrouw was}\\

\haiku{Ik heb het niet, heb.}{het nooit zo maar gegeven}{aan een gezonde}\\

\haiku{Zie dan maar verder.}{te komen op deze reis}{die nog weken duurt}\\

\haiku{als zij zich al te.}{hoog en vermetel in zijn}{nabijheid wagen}\\

\haiku{En don Benito,.}{was toch de slimste al was}{hij maar een Indio}\\

\haiku{zijn tweekleurige.}{glazige blik groot en star}{op haar gevestigd}\\

\haiku{Luider dan eerst ging}{hij voort met haar te zeggen}{dat hij gezien had}\\

\haiku{Hij is slim, mij te.}{slim af moest Urbina L\'opez}{zichzelf bekennen}\\

\haiku{{\textquoteleft}Hij dacht zeker er,.}{naar toe te kunnen vliegen}{met al zijn grootspraak}\\

\haiku{Ik geloof dat het.}{waar is wat er in dat boek}{van mijn maestro stond}\\

\section{Albert Helman en Albert Kuyle}

\subsection{Uit: Van pij en burnous}

\haiku{de dingen met vijf,.}{zinnen te benaderen}{en niet met het bloed}\\

\haiku{Overal te zoeken.}{naar de \'e\'ene weg die nog}{onversperd moet zijn}\\

\haiku{Aan niets wordt zooveel;}{zekerheid geofferd als}{aan nieuwsgierigheid}\\

\haiku{Hoor nu het nacht is,.}{hoe dit zwarte water klotst}{tegen den steiger}\\

\haiku{Met een vaartje neem ik,.}{de hoeken en kom buiten}{adem in de dorpsstraat}\\

\haiku{Het hindert weinig.}{op welken tijd van het jaar}{gij in Umbri\"e komt}\\

\haiku{Maar Umbri\"e ligt van.}{de verdere wereld ook}{zoo afgezonderd}\\

\haiku{uit alle stemmen;}{klokte een donkere roep}{naar nieuwe broeders}\\

\haiku{Zij slaan hun handen,.}{tegen de zwarte muur die}{korrelig aanvoelt}\\

\haiku{Een nieuwe omdraai.}{geeft uitzicht op de heele}{vlakte van Umbri\"e}\\

\haiku{Totdat om de hoek}{van de straat het heldere}{zingen van water}\\

\haiku{En weer spiegelt het,,,.}{water de tijd de liefde}{de Gerechtigheid}\\

\haiku{De stad van Petrus,}{die niet warm voor u open ligt}{als ge komt van ver}\\

\haiku{Wij wachten op Sint,.}{Petrus die zegenend ons}{allen voorbijtrekt}\\

\haiku{Hij zegent allen.}{met hostie en kelk die hij}{in zijn handen draagt}\\

\haiku{Zijn woorden slaan neer.}{als zware keien voor de}{voeten van Arius}\\

\haiku{De abt is een groote,.}{rijzige man als hij daar}{voor zijn zetel staat}\\

\haiku{De menschen loopen,.}{er heen en weer de jongens}{ravotten er wat}\\

\haiku{Wie kent Amsterdam?}{als hij noch de Jordaan noch}{de Kolk gezien heeft}\\

\haiku{Verderop voor een.}{der huisjes zit een oude}{man op een bankje}\\

\haiku{Ze reizen langs de.}{vooruitgeschoven posten}{van het moederland}\\

\haiku{De dienstgang is vol}{leven en vlak achter een}{troepje matrozen}\\

\haiku{Salammb\^o echter {\textquoteleft}{\textquoteright};}{bleef voor mij het maximum van}{wat wijstijl noemen}\\

\haiku{De tambourijn schuift,.}{schelle ronde vlakjes af}{als email dat schilfert}\\

\haiku{Een slank, fransch meisje.}{schenkt ons den gloeiend-rooden wijn}{van M\'egrine in}\\

\haiku{Warme bronnen met ',.}{chloorzure soda int}{water of zooiets}\\

\haiku{de wegen waarlangs.}{hij ons voert naar het feest der}{Verrijzenis}\\

\haiku{Door de open deur zie.}{je een stuk mimosa-laantje}{als een schilderij}\\

\haiku{Arme Vincent; ik,.}{bid voor hem en voor alle}{schilders die ik ken}\\

\haiku{Het is oud als een.}{bedoe{\"\i}nentent en jong}{als deze lente}\\

\haiku{Met gorgel- en.}{sis-geluiden leidde}{hij de kameelen}\\

\haiku{Dan verdwijnt alles.}{in het trillend zoemen van}{bewegende lucht}\\

\haiku{Hij keek op naar het,.}{vroom gelaat den vollen baard}{en de wijze oogen}\\

\haiku{Het jonge hoofd naast.}{hem boog zich nog dieper naar}{den gelen stofweg}\\

\haiku{De bedoe{\"\i}nen.}{praten fluisterend met scherp}{schrapen van de keel}\\

\haiku{Want de tapijten.}{zijn hier de rijkdom en de}{welvaart van de stad}\\

\section{Emiel van Hemeldonck}

\subsection{Uit: Agnes}

\haiku{Lange uren, waar geen,;}{eind aan scheen te komen en}{ik kon niet klagen}\\

\haiku{{\textquoteleft}Ziet ge, het zal niet!}{lang duren of hij is een}{eerste klas koetsier}\\

\haiku{En ook moeder moet,.}{het gehoord hebben want zij}{antwoordde niet meer}\\

\haiku{En het meisje met.}{het stroohoedje en de blauwe}{korenbloemen}\\

\haiku{Dat ik een haas kon,.}{verrassen snoek stroppen en}{een eekhoorn vangen}\\

\haiku{Geluk er mee, en...{\textquoteright}}{leer het later zorgvuldig}{aan uw kinderen}\\

\haiku{Een magere, hooge;}{stem riep uit de verte en}{het meisje verschrok}\\

\haiku{Ik voelde de oogen,.}{van het meisje op mij maar}{ik vond geen woorden}\\

\haiku{Hij knikte, en weer,.}{lag er spot in zijn stem maar}{het deed mij geen pijn}\\

\haiku{Het was of ik uit,.}{een tooverwereld kwam uit een}{vreemden droom ontwaakt}\\

\haiku{In den stal stampten,.}{de paarden het dreunde hol}{in het groote gebouw}\\

\haiku{Tist betrapte mij.}{bij dit werk en het was hem}{niet onaangenaam}\\

\haiku{in het licht van de.}{kaars danste zijn grillige}{schaduw op den muur}\\

\haiku{De klauw van zijn hand.}{greep naar de kaarsvlam en het}{was donkere nacht}\\

\haiku{ik leefde feller,.}{heviger dan ik het tot}{nog toe gedaan had}\\

\haiku{Er was geen tijd meer,,.}{dorpen hadden geen naam er}{was geen herkennen}\\

\haiku{maar er was iets in,,.}{zijn blik houding en gebaar}{dat ik niet kende}\\

\haiku{Het was als een even,.}{afgebroken gesprek dat}{hij thans voortzette}\\

\haiku{maar dan verstarde:}{zijn glimlach tot een grimas}{en hij fluisterde}\\

\haiku{met moeite kon ik,.}{in de maat van zijn tragen}{zwaren stap blijven}\\

\haiku{{\textquoteleft}Er komt onweer,{\textquoteright} zei,.}{hij en het was alsof hij}{van man tot man sprak}\\

\haiku{Dan voelde ik het.}{licht verzwakken en opende}{voorzichtig de oogen}\\

\haiku{Het geldstuk brandde.}{in mijn hand. Als een golf sloeg}{de woede over mij}\\

\haiku{Zijn magere hand.}{rilde op den dikken knop}{van zijn wandelstok}\\

\haiku{Ik had het vroeger,;}{wel gezien maar er nooit veel}{aandacht aan besteed}\\

\haiku{Op de werktafel:}{stond een bronzen beeldje op}{marmeren voetstuk}\\

\haiku{ik wist alleen dat,.}{ik niet buigen zou wat er}{ook mocht geschieden}\\

\haiku{Toen  hoorde ik,.}{den naderenden stap van}{Ben mijn oudsten broer}\\

\haiku{{\textquoteright} Wanneer ik naar hem,.}{opkeek zag ik dat hij de}{schouders ophaalde}\\

\haiku{Wij hadden tegen,;}{het bosch aan een plek liggen}{die moerassig was}\\

\haiku{Ik begreep en schreef,.}{ook de enkele woorden}{die hij dicteerde}\\

\haiku{Achter haar beklom,,.}{ik de smalle trap \'e\'en twee}{verdiepingen hoog}\\

\haiku{Schouderophalend,}{keek zij mij aan en op dit}{oogenblik hoorden}\\

\haiku{de geboorte van.}{het  eenige kind had haar}{het leven gekost}\\

\haiku{Louis is aan 't werk,,...{\textquoteright}}{en zoodra hij verlof}{kan krijgen komt hij}\\

\haiku{Wanneer ik aan de,.}{deur stond zag ik dat hij nog}{iets zeggen wilde}\\

\haiku{In kladden drijven,;}{de vogels over de wijde}{rustende akkers}\\

\haiku{De stilten tusschen.}{ons waren vaak langdurig}{en soms ondraaglijk}\\

\haiku{En ik voelde haar,.}{tegenwoordigheid al zag}{noch hoorde ik haar}\\

\haiku{De buiging die hij,.}{maakte kwam mij houterig}{en potsierlijk voor}\\

\haiku{Met zeer gemengde.}{gevoelens bracht ik den avond}{op mijn kamer door}\\

\haiku{Hij groette verstrooid,.}{keek even naar de pakkage}{en nam zijn plaats in}\\

\haiku{{\textquoteright} Plots kwam hij terug,:}{bij de auto greep een paar}{valiezen en vroeg}\\

\haiku{Ik moest verhalen,.}{en de vinnige vragen}{doorflitsten mijn relaas}\\

\haiku{maar meer dan deze,.}{woorden trof mij de wijze}{waarop hij ze sprak}\\

\haiku{Over de binnenkoer,.}{kwam Jos\'e gegaan den arm}{reikend aan Spitsmuis}\\

\haiku{Dit alles kwam mij,.}{zoo raadselachtig voor dat}{het mij verwarde}\\

\haiku{Doelloos liep ik door,.}{de straten alleen om den}{tijd om te krijgen}\\

\haiku{zonder mij nog \'e\'en,.}{woord te gunnen zette zij}{het op een loopen}\\

\haiku{Onzeker was zijn,:}{blik en vaag het gebaar van}{zijn hand als hij zei}\\

\haiku{De geur van dieren,.}{gestapeld hooi en versch groen}{vergezelde mij}\\

\haiku{Dit woord greep mij aan,.}{geen schooner lof had mij dieper}{kunnen ontroeren}\\

\haiku{Agnes zal maar eerst,...}{dezen avond komen tenzij}{gij ze laat roepen}\\

\haiku{Alleen de jonker,.}{bewoog niet maar ik kon niet}{gelooven dat hij sliep}\\

\haiku{Een kleine mis, met.}{wat oude moederkens en}{een paar kinderen}\\

\haiku{De grafmaker keek.}{op toen ik bleef staan tot hij}{den kuil gevuld had}\\

\haiku{Zij keek naar mij op.}{en ik meende argwaan in}{haar oogen te lezen}\\

\haiku{in den koelen avond.}{werden de slachtoffers naar}{het gasthuis gebracht}\\

\haiku{Ik gehoorzaamde.}{werktuiglijk en de auto}{schoot den zijweg in}\\

\haiku{Boven een donker;}{massief rees de scherpe spits}{van een kerktoren}\\

\haiku{Na een wijl zette,.}{hij zich in het gras leunend}{tegen den schuurmuur}\\

\haiku{{\textquoteleft}Ik heb hem bij ons,.}{gehaald wanneer hij gered}{moest worden diefstal}\\

\haiku{De auto schoot in,;}{gang ik sloeg den eersten den}{besten weg  in}\\

\haiku{Ik schoof dichter naar,.}{haar toe en nam haar hand die}{zij niet terugtrok}\\

\haiku{Beter en dieper,.}{dan ik dit ooit gevoeld had}{wist ik mij hier thuis}\\

\haiku{Toen ik haar vroeg hoe,}{haar jonge meesteres het}{stelde schudde zij}\\

\haiku{Misschien had hij dit,:}{wel bemerkt want naar Spitsmuis}{omkijkend zei hij}\\

\haiku{Dan schudde zij het,.}{hoofd en ging zonder mij een}{antwoord te gunnen}\\

\haiku{Maar hij gebaarde.}{van mijn aanwezigheid niet}{meer bewust te zijn}\\

\haiku{{\textquoteright} Werktuiglijk nam ik,.}{het stuk in handen en plots}{doorvoer mij een schok}\\

\haiku{In de garage:}{wees ik den chauffeur naar het}{hoogste schap en zei}\\

\haiku{het was of zij mij,.}{iets zeggen wilde maar ik}{wachtte vruchteloos}\\

\haiku{Misschien was het wel,.}{goed dat ge intusschen naar}{iets anders uitzaagt}\\

\haiku{ik zag en hoorde,.}{alles maar de beteekenis}{drong niet tot mij door}\\

\haiku{Achter de tafel,,.}{zat rustig zijn pijp rookend}{een politieman}\\

\haiku{Honger naar kennis,,.}{honger naar macht en dit is}{zijn verderf geweest}\\

\haiku{Het verlangen dat,;}{mij verteerde heb ik niet}{kunnen verbergen}\\

\haiku{Door de wolken brak,.}{de bleeke zon en het schrille}{licht deed mijn oogen pijn}\\

\subsection{Uit: De harde weg}

\haiku{Zij antwoordde niet,.}{maar hij meende een glimlach}{gezien te hebben}\\

\haiku{{\textquoteleft}Niet waar, heer pastoor,?}{we krijgen wonderlijke}{dingen te hooren}\\

\haiku{{\textquoteright} Haar stem was z\'o\'o, dat.}{het bevel geen tweede maal}{moest herhaald worden}\\

\haiku{{\textquoteright} Haar spot was scherp en,,.}{neerhalend maar geen lach geen}{glimlach trad haar bij}\\

\haiku{In het rijtuig, dat,.}{zij zelf voerde zat de knaap}{Micha\"el naast haar}\\

\haiku{En dan bereikte.}{het magere geluid van}{een vedel haar oor}\\

\haiku{{\textquoteright} kraaide hij, maar dan.}{was zijn lied uit en hij kroop}{langs de banken weg}\\

\haiku{Er was geruisch.}{van kleeren en het geluid van}{een gedempten stap}\\

\haiku{Hij ging haar v\'o\'or, de,.}{trap af boog naar de deur van}{de gastenkamer}\\

\haiku{een baken voor het,.}{kleine werkzame volk van}{stad en platteland}\\

\haiku{En dit kind... {\textquoteleft}Er kan,.}{geen dringende haast bij zijn}{hoogedele Vrouwe}\\

\haiku{Buiten zijn uniform.}{was niets martiaals meer aan}{hem te bespeuren}\\

\haiku{Hij stond in de deur:}{van het prieeltje en riep}{de kinderen toe}\\

\haiku{fluweelen buis en.}{zilveren beengespen op}{de witte kousen}\\

\haiku{{\textquoteright} Door de groote poort, langs,,.}{de hal waar reeds wakend licht}{brandde naar de trap}\\

\haiku{{\textquoteright} {\textquoteleft}Cerberus die zijn,{\textquoteright}.}{schatten goed bewaakt spotte}{de provisor mild}\\

\haiku{de hooge muren, bij,.}{dit geluidloos sneeuwen dat}{de wereld afsloot}\\

\haiku{de zotte wind, die,.}{onder de sneeuwvlokken joeg}{wervelde ze mee}\\

\haiku{tegen de bermen.}{bij de poelen sterden de}{eerste anemonen}\\

\haiku{{\textquoteright} Vaag vermoedde hij,.}{wat van hem verlangd werd en}{hij knikte nogmaals}\\

\haiku{{\textquoteright} knikte zij spottend.}{en het was alsof zij hem}{als een vod wegsmeet}\\

\haiku{haar lippen werden,.}{scherp haar hand trilde wanneer}{zij het mes neerlei}\\

\haiku{Het kon de oude,:}{gewoonte zijn die plots in}{hem weer leven kreeg}\\

\haiku{{\textquoteleft}Gij hebt uw werk goed,,.}{verricht ma m\`ere en uw}{opzet is geslaagd}\\

\haiku{Nu moest zij niet meer;}{wachten op den koerier van}{Hare Majesteit}\\

\haiku{Het was niet noodig dat;}{zij het angstige meisje}{op den voet volgde}\\

\haiku{Wanneer zij stil hield,;}{op het voorplein brandde de}{kramp in hand en voet}\\

\haiku{De eenige weg, heeft...}{Hij gezegd die de waarheid}{is en het leven}\\

\haiku{Als gravin Martha,.}{omkeek zat pastoor Grang\'e}{daar als vernietigd}\\

\haiku{{\textquoteleft}Neen,{\textquoteright} schudde hij, als.}{de koetsier het portier van}{het rijtuig open wierp}\\

\haiku{Het rhythme van zijn,.}{tragen stap vertraagde nog}{hij moest zich voortsleepen}\\

\haiku{vaag schoof het landschap,,,.}{voorbij struikgewas boomen}{een boerenwoning}\\

\haiku{dat een gezonde.}{ziel alleen in een gezond}{lichaam kan wonen}\\

\haiku{{\textquoteright} Maar hij voelde den,:}{lichten hoon niet knikte naar}{Micha\"el en zei}\\

\haiku{Hij had tijd noch kans.}{om zich van zijn vermoeden}{te vergewissen}\\

\haiku{Hij had een zwaren,.}{nacht gehad in zijn gewond}{been knaagde de pijn}\\

\haiku{{\textquoteright} raadde Alexander, maar.}{zij gebaarde alsof zij}{hem niet gehoord had}\\

\haiku{Als de kamerdeur,.}{geopend werd spoelde het}{licht de gang binnen}\\

\haiku{Het afscheidnemen,.}{zou hem niet t\'e zwaar vallen}{zijn hart was niet hier}\\

\haiku{Binnen een kwartier,.}{wordt de klok geluid en dan}{schiet hij toch wakker}\\

\haiku{Hij vond een glimlach,,.}{derwijze berusting en}{schudde traag het hoofd}\\

\haiku{In de kamer naast.}{hem ging het geluid van oom}{Alexanders licht gesnork}\\

\haiku{{\textquoteright} zei hij, maar het klonk.}{of hij daar heelemaal niet}{zoo zeker van was}\\

\haiku{Maar Claire stond er,.}{op te vertrekken hoe ook}{aangedrongen werd}\\

\haiku{de vlakten plooiden.}{open onder het magere}{licht van den sneeuwnacht}\\

\haiku{{\textquoteright} vroeg hij, opkijkend.}{en haar het dichtgevouwen}{papier toereikend}\\

\haiku{Als er geluid van,:}{een stap klonk en Amelie bij}{het vuur stond zei hij}\\

\haiku{Wij kunnen naaien,, ',.}{en alst moet is er nog}{andere arbeid}\\

\haiku{{\textquoteright} De Bellincourt keek.}{hem scherp aan en zijn lachje}{was medelijdend}\\

\haiku{{\textquoteleft}Ik denk dat het niet.}{zoo moeilijk zou zijn om den}{ketting op te gaan}\\

\haiku{Hij riep haar, wees op,.}{het vertrappelde gras maar}{zij lachte spottend}\\

\haiku{{\textquoteright} Misschien had hij haar.}{niet verstaan of raakte haar}{dwaas bevel hem niet}\\

\haiku{Zonder eenig dankwoord,.}{nam zij den kroes aan en dronk}{zonder op te zien}\\

\haiku{De vraag verraste.}{hem en hij begreep niet wat}{zij van hem wilde}\\

\haiku{Dezen avond kunt gij,{\textquoteright}.}{te Loven zijn zei hij en}{Micha\"el begreep}\\

\haiku{Van der Noot maakte;}{zich driftig bij de lezing}{van de missive}\\

\haiku{{\textquoteleft}De Keizer toont eens.}{te meer dat hij het goed meent}{en het verkeerd doet}\\

\haiku{Tot mijn spijt zal ik.}{echter de plechtigheid niet}{kunnen bijwonen}\\

\haiku{In lange rijen;}{zaten de fraters aan de}{withouten tafels}\\

\haiku{Hij ging hen v\'o\'or, en.}{na hem volgde de lange}{rei hem naar de kerk}\\

\haiku{Langen tijd bleef hij,.}{haar achternastaren}{maar zij keek niet om}\\

\haiku{ergens weerklonk het.}{lustige geklepper van}{een watermolen}\\

\haiku{Zij keek naar haar zoon,:}{en er klonk fierheid in haar}{stem wanneer zei zij}\\

\haiku{{\textquoteleft}Het doel van mijn reis.}{zou den leeraar wellicht}{interesseeren}\\

\haiku{Zoo zette dan 's.}{anderen daags Micha\"el}{alleen de reis voort}\\

\haiku{Hij keek haar vragend,:}{aan en daar zij zwijgen bleef}{zei hij gebiedend}\\

\haiku{Maar hij kon het niet,.}{over zijn hart krijgen terug}{naar haar toe te gaan}\\

\haiku{{\textquoteleft}Ik ben er verheugd,}{om in u den advokaat}{van ons volk te zien}\\

\haiku{{\textquoteleft}Nu, als de oude......}{heer zich nog interesseert}{aan zijn zeg hem dan}\\

\haiku{Micha\"el wuifde,.}{met de hand maar zijn groet werd}{niet beantwoord}\\

\haiku{{\textquoteleft}Mijn verblijf alhier.}{duurt langer dan wel eenigszins}{kon voorzien worden}\\

\haiku{Maar oom Alexander had.}{reeds rechtsomkeer gemaakt en}{traag gleed de deur dicht}\\

\haiku{De broeder-portier,.}{kende hem niet liet hem in}{de groote spreekkamer}\\

\haiku{Er is bezoek dat.}{u mogelijk belang}{kan inboezemen}\\

\haiku{{\textquoteright} {\textquoteleft}Voor ons was het ook,{\textquoteright}.}{een heele verrassing ging}{prelaat Hermans voort}\\

\haiku{Huygens gaf hij een.}{omstandig relaas van hun}{belevenissen}\\

\haiku{Ja, ja, ze mochten,.}{hem zonder fout verwachten}{lachte de prior}\\

\haiku{Samen bogen zij,.}{over de wieg over het bed waar}{Aim\'ee sluimerde}\\

\haiku{Zelden kon hij zich.}{losrukken om weer naar zijn}{werktafel te gaan}\\

\haiku{wat hij te zeggen,.}{had kon ook met andere}{woorden geschieden}\\

\haiku{Zoo trok hij door de.}{dreef naar het bosch en verder}{plooide de hei open}\\

\haiku{De honden liepen,.}{snuffelend in het grauwe}{verregende kruid}\\

\haiku{In hun nood namen,}{de boeren hun toevlucht tot}{Micha\"el waarvaf}\\

\haiku{{\textquoteleft}Als honden over de...,.}{wegen gejaagd Wie ons vangt}{heeft een belooning}\\

\haiku{{\textquoteleft}Neen, maar misschien hebt?}{ge wel een drukker onder}{uw goede vrienden}\\

\haiku{Een leurder neemt de.}{oude kleeren over en draagt den}{dood en weet het niet}\\

\haiku{{\textquoteleft}Haal bij wie naar uw,{\textquoteright}.}{meening helpen kan gebood}{hij den chirurgijn}\\

\haiku{Hij liep de gang op,.}{en neer onrustig als een}{wild dier in de kooi}\\

\haiku{* * * ~ 's Anderen.}{daags schelde hij vruchteloos}{om het dienstmeisje}\\

\haiku{{\textquoteright} Perrin wenkte de.}{gendarmen en stiet moedig}{de achterdeur open}\\

\haiku{De bedreiging, dit,:}{wapen der zwakkelingen}{hanteerend zei hij}\\

\haiku{{\textquoteright} {\textquoteleft}Ha,{\textquoteright} fluisterde zij.}{en het was of zij onder}{een slag ineenkromp}\\

\haiku{{\textquoteright} {\textquoteleft}Waar een wil is, is,{\textquoteright}.}{ook een weg antwoordde de}{Choiseul geprikkeld}\\

\haiku{Meer wist Micha\"el,.}{niet maar het was voldoende}{om zijn rust te rooven}\\

\haiku{de naam klonk in hem,.}{na dan verhelderde een}{glimlach zijn gelaat}\\

\haiku{{\textquoteright} Geen antwoord hoorend, keek,.}{hij naar Micha\"el op zag}{hoe zwak hij nog was}\\

\haiku{Als zij stilstonden,;}{verraste hen het geluid}{van een geweerschot}\\

\haiku{{\textquoteright} Zijn oogen rustten op,.}{haar gelaat maar onder zijn}{blik roerde zij niet}\\

\subsection{Uit: Johan van der Heyden, magister}

\haiku{We zullen u die}{oneerbiedige woorden}{niet aanrekenen}\\

\haiku{De Zwarte Molen?}{is nog altijd eigendom}{van de van Wessem's}\\

\haiku{{\textquoteleft}Geert Dielis, met wat.}{spellen en een paar liedjes}{is geen school te doen}\\

\haiku{Altijd tegen avond,,.}{zegt hij zoo dan speel ik mijn}{geestelijk schaakspel}\\

\haiku{Hij moest de vraag niet,.}{stellen hij las genoeg op}{de aangezichten}\\

\haiku{de deken gaf hem.}{twee kriaaltjes mee om hem}{den weg te wijzen}\\

\haiku{Dat zal zoo...{\textquoteright} {\textquoteleft}Neen, dat{\textquoteright},.}{zal nu niet deed de nieuwe}{meester opmerken}\\

\haiku{die vreugd van elken,.}{dag haar broertjes en zusjes}{bezorgd te weten}\\

\haiku{Zoo lag ze daar bleek,.}{en stil tot geen enkele}{inspanning bekwaam}\\

\haiku{De grootste hoorde;}{ze roepen en tieren in}{de lage bosschen}\\

\haiku{Ik doe mijn best.{\textquoteright} Als,:}{hij van der Heyden tot aan}{de deur bracht vroeg hij}\\

\haiku{In {\textquoteleft}'t Heibloemke{\textquoteright}...{\textquoteright}:}{vinden wij elkaar Johan keek}{hem glimlachend aan}\\

\haiku{het praatje over weer.}{en wind en aan tafel de}{nieuwtjes van de stad}\\

\haiku{{\textquoteleft}Mijn vader, die in ',.}{t leger diende bracht hem}{mee van zijn tochten}\\

\haiku{Als hij terug op,.}{de binnenkoer kwam was het}{venster gesloten}\\

\haiku{{\textquoteright} Johan keek hem strak aan,, {\textquoteleft}!}{maar zijn mond bleef gesloten}{Hoed u voor dien man}\\

\haiku{de wind speelde in.}{haar hoofddoek en hij wuifde}{haar van verre toe}\\

\haiku{Tegen avond maakte.}{hij den hond los en trok langs}{het Akkerpad op}\\

\haiku{{\textquoteright} {\textquoteleft}Ga met mij nu mee,.}{ik moet zoo rap als ik kan}{terug naar Turnhout}\\

\haiku{{\textquoteright} Johan meende verder,.}{te vragen maar daar was de}{oude met Rina}\\

\haiku{{\textquoteleft}Zeg maar dat Marten!}{van Rossum het beestje zal}{komen betalen}\\

\haiku{Zoo geraakten ze.}{aan de eerste huizen van}{de Pottersstraat}\\

\haiku{Hij kende de de, -.}{Roode's al van langer hij}{was zelf van Turnhout}\\

\haiku{Hij kende er wat,.}{van had veel van die boomen}{geplant en ge\"ent}\\

\haiku{En plots doorstroomde.}{breed orgelspel den wijden}{beuk van de kapel}\\

\haiku{hier in de abdij,.}{is hulp Rina zal haar het}{brood toereiken}\\

\haiku{Dit kind en...{\textquoteright} {\textquoteleft}Toe nu,{\textquoteright},.}{Johan drong ze aan en dat klonk}{haast als een bevel}\\

\haiku{In de kamer nam.}{hij den kandelaar op en}{hield hem over de wieg}\\

\haiku{De stemmen klonken,,;}{en daalden wentelden en}{keerden stegen dan}\\

\haiku{{\textquoteleft}Ach ja{\textquoteright}, glimlachte, {\textquoteleft}?}{hijdie  Geert Dielis van}{den Zwarten Molen}\\

\haiku{Geert had niet gezien,.}{wat het boek was hij had het}{op de schouw gelegd}\\

\haiku{Zij liet een schreeuw en.}{viel op de knie\"en v\'o\'or het}{kruis boven den haard}\\

\haiku{ik heb het al zoo,,...}{dikwijls gezegd hij mag dat}{niet doen hij mag niet}\\

\haiku{Hij keek van boven.}{zijn brilleglazen uit en}{lei zijn brevier weg}\\

\haiku{{\textquoteright} {\textquoteleft}De Bijbel van Geert...{\textquoteright}.}{Dielis Van Wessem keek hem}{ongeloovig aan}\\

\haiku{Hij is geen kind meer,.}{dat heeft hij trouwens al meer}{dan eens bewezen}\\

\haiku{hier en eet eerst, ik{\textquoteright},.}{zorg wel voor de twee klassen}{zei hij vriendelijk}\\

\haiku{Midden de kamer;}{hing een bronzen luchter met}{brandende kaarsen}\\

\haiku{donker en dreigend.}{rolde de holle roffel}{door de leege stede}\\

\haiku{Hij scheen op iets te,.}{dubben de vage blik was}{naar binnen gekeerd}\\

\haiku{Aan een der hutten.}{stak hij zijn witten stok in}{het riet van het dak}\\

\haiku{Ginder zag hij den.}{glimmenden waterspiegel}{van een droomend veen}\\

\haiku{Zijn bloed stond stil, nog.}{nooit had zij al die jaren}{z\'o\'o v\'o\'or hem gestaan}\\

\haiku{Johan hoorde zijn roep.}{over de hei en er kwam een}{warm gevoel over hem}\\

\haiku{Dan brak geestdriftig.}{handgeklap en stormachtig}{hoerageroep los}\\

\haiku{het is nutteloos,.}{op te blijven het zou wel}{laat kunnen worden}\\

\haiku{Ge ziet, als d'ijzers,.}{afgetrokken zijn hebt ge}{niets meer te zeggen}\\

\haiku{Dat is nu jaren, '.}{voorbij ent is of het}{maar pas gebeurd is}\\

\haiku{Laatst in {\textquoteleft}De Roode{\textquoteright}.}{Schild heeft een heel gezelschap}{het kunnen hooren}\\

\haiku{De Roode Schild{\textquoteright}, waar.}{de bode voor Herentals}{en Lier uitspande}\\

\haiku{{\textquoteleft}Nooit, nooit meer zeggen,...{\textquoteright},,.}{jongen Traag met starre oogen}{knikte het kind}\\

\haiku{Maar wie zou hem in?}{den avond laten gaan hebben}{langs de moerassen}\\

\haiku{Nog een kind, had zijn.}{scherp verstand het doorzicht van}{den volwassene}\\

\haiku{Het verheugt mij dat{\textquoteright},.}{mijn zoon de waarheid liefheeft}{zei hij kort en hard}\\

\haiku{De Magistraat stond.}{op de pui van het stadhuis}{en schouwde den stoet}\\

\haiku{de landvoogdes is:}{onwetend van de nooden die}{het volk martelen}\\

\haiku{{\textquoteleft}Nooit werd met scherper.}{ironie de ondergang van}{de Kerk geschilderd}\\

\haiku{Een hoek der kamer:}{werd vrij gemaakt en daar nam}{het gezelschap plaats}\\

\haiku{Het was vreemdstil in.}{de gang en de stilte in}{haar was nog grooter}\\

\haiku{Hij zag wel dat hij.}{oud geworden was en hoe}{dof die oogen stonden}\\

\haiku{haar handen lei ze.}{op zijn schouders en haar oogen}{lieten hem niet los}\\

\haiku{We zijn reeds in de...{\textquoteright} {\textquoteleft}{\textquoteright},.}{VastenJa knikte D'Hose}{en keek Johan strak aan}\\

\haiku{Hij sloeg de oogen neer.}{onder den vragenden blik}{van van der Heyden}\\

\haiku{Want de kettersche,.}{leering die u betooverd heeft kan}{mij niet bekoren}\\

\haiku{{\textquoteright} De kamerdeur werd:}{opengeworpen en een blije}{stem verwelkomde}\\

\haiku{V\'o\'or het hoogaltaar.}{zonk koordeken Coomans}{neer en boog het hoofd}\\

\haiku{Maar ik kan me al}{voorstellen hoe hij ginder}{in zijn fluweelen}\\

\haiku{{\textquoteright} Vruchteloos trachtte.}{de jonge man de kerels}{tegen te houden}\\

\haiku{Onwillekeurig.}{vouwde hij zijn handen en}{begon te bidden}\\

\haiku{{\textquoteright} En de volgende '.}{week was de gezant van duc}{d'Alf opt Kasteel}\\

\haiku{Patroeljes ruiters,.}{reden door de straten zoo}{bij dag als bij nacht}\\

\haiku{De wacht bracht hem op,.}{het binnenplein zoo naar de}{kleine wachtkamer}\\

\haiku{het was haast of er,.}{vreugde lag in haar stem om}{die onzekerheid}\\

\haiku{De ziekte van haar...,.}{moeder Vrees niet het heeft haar}{aan niets ontbroken}\\

\haiku{Heel 't gezin is.}{groot en niemand die niet voor}{zichzelf zorgen kan}\\

\haiku{Wel viel de gang hem,.}{zwaar er was zooveel dat hij}{niet vergeten kon}\\

\haiku{Duc d'Alf's strengheid had;}{de rumoerigste geuzen}{voorzichtig gemaakt}\\

\haiku{En ze hebben hem,...{\textquoteright} {\textquoteleft}{\textquoteright},.}{gepijnigd gemarteldJa}{zei de priester stil}\\

\haiku{{\textquoteright} ~ V\'o\'or hij zich ter,:}{ruste legde had Rina}{hem nogmaals gevraagd}\\

\haiku{{\textquoteright} {\textquoteleft}Neen{\textquoteright}, antwoordde hij, {\textquoteleft}...}{zachtmisschien is de tros naar}{Limburg afgezakt}\\

\haiku{Zoo den langen nacht,.}{door met de gedachten die}{hem niet loslieten}\\

\haiku{Het was als de stem,.}{van een angstig kind dat plots}{zijn moeder ontwaart}\\

\subsection{Uit: Kroniek}

\haiku{De heer klopt hem op,,;}{den schouder hij moet mee gaan}{in de groote kamer}\\

\haiku{Fien zijn vrouw, ze is,.}{te vroeg gegaan ze had dat}{moeten beleven}\\

\haiku{Stanske doet dat niet,.}{ze heeft haar werk en ze zal}{dat niet verlaten}\\

\haiku{Maar dat hij dan juist,.}{voorgoed zou heengaan daar heeft}{ze nooit aan gedacht}\\

\haiku{De eenzaamheid die.}{haar aangrijnst en de armoe}{die aan haar hart bijt}\\

\haiku{Een goed werk deed ze,;}{er mee dat is heel het dorp}{door verteld geweest}\\

\haiku{{\textquoteleft}Wat buurten, bazin{\textquoteright}.}{Langs het poortje komt hij den}{hof ingedrenteld}\\

\haiku{dreef naar de meerschen,.}{waar de burgemeester zijn}{nieuwe hoeve bouwt}\\

\haiku{Een meester, dat zit.}{niet alleen binnen de vier}{muren van de school}\\

\haiku{En toekomende,,}{week begint de school ja de}{volgende week al.}\\

\haiku{Hij staat in de deur,,.}{de burgemeester en ziet}{den jongen man na}\\

\haiku{Hij blijft daar achter.}{in de klas staan en zegt geen}{gebenedijd woord}\\

\haiku{Hij ziet dat meester.}{Van Deun geboeid toeluistert}{en hij gaat maar door}\\

\haiku{Ze glimlachen maar,,.}{hij verschiet daar niet van hij}{kent die zwijgers wel}\\

\haiku{{\textquoteright} {\textquoteleft}Ja,{\textquoteright} glimlacht Karel, {\textquoteleft},...{\textquoteright}}{ik heb op een anderen}{akker geploegd maar}\\

\haiku{Hij trok voor een dag,.}{of acht naar zijn zoon die te}{Antwerpen woonde}\\

\haiku{Hij trekt  driftig,:}{aan zijn pijp hij staat recht en}{zegt kort en beslist}\\

\haiku{duidelijk kan hij,.}{den timmerman hooren dat}{donkere geluid}\\

\haiku{Hij tikt er met den.}{nagel tegen en past ze}{tusschen zijn tanden}\\

\haiku{als ze thuis komen,.}{staat Stanske al in de deur}{ze wacht met haar eten}\\

\haiku{hij moet zien hoe schoon,.}{dat koren op wil op dien}{mageren heigrond}\\

\haiku{Nieuw is ze niet, maar.}{ge hebt zelf wel gehoord dat}{er nog klank in zit}\\

\haiku{en de piano.}{is honderd maal meer waard dan}{den vorigen keer}\\

\haiku{{\textquoteright} Zoo, dat is niet waar,?}{en waar heeft Goor dat anders}{gehoord of gezien}\\

\haiku{Hij kent dien stap langs.}{het lage venster en het}{geruttel van de deur}\\

\haiku{De kleintjes zwijgen,.}{en kruipen weg ze krijgen}{hun deel van de bui}\\

\haiku{Hij zit aan tafel,.}{en haalt papier en potlood}{boven en zijn boek}\\

\haiku{Goor schuift papieren,.}{en boek bijeen hij sluipt haast}{de zoldertrap op}\\

\haiku{Zij zet een lied aan,,;}{er zijn geen woorden bij noodig}{haar stem klimt en daalt}\\

\haiku{Haar handen beefden.}{zoo sterk dat ze de tas haast}{niet vasthouden kon}\\

\haiku{Het licht van de lamp,,.}{speelt in haar oogen die zacht}{zijn hij weet dat wel}\\

\haiku{Hij hoort alleen den.}{hollenden wind als een ver}{gezoem in de schouw}\\

\haiku{De koorts verteert dit,.}{jonge lichaam hij voelt die}{kleine hand branden}\\

\haiku{Bij waken,{\textquoteright} zegt hij, {\textquoteleft}.}{en binnen een paar uur nog}{een lepel geven}\\

\haiku{Hij buigt over het kind,.}{er is een zware eerbied}{over hem gekomen}\\

\haiku{en dat gelaat is.}{hard met die scherpe lijnen}{van pijn en armoe}\\

\haiku{Hij moet weer buiten,,,.}{zijn in de zon de huizen}{uit de velden in}\\

\haiku{Als hij opkijkt, blikt,.}{hij in Anna's oogen die hem}{angstig aanstaren}\\

\haiku{meester geeft hem de.}{verhalenboekjes met de}{na{\"\i}eve prentjes}\\

\haiku{Meester kijkt ze na,;}{en dan verglijdt de glimlach}{van zijn aangezicht}\\

\haiku{De Ruyck voelt zich,.}{onzeker hij weet niet wat}{hij nog zeggen zal}\\

\haiku{De ondervraging.}{is komen bevestigen}{wat ik vermoedde}\\

\haiku{{\textquoteright} Meester Van Deun kijkt,.}{op zijn oogen worden klein en}{hij verzet zijn klak}\\

\haiku{Dan merkt hij plots het.}{piepen van een poortje en}{een stap op het pad}\\

\haiku{Ze kijkt hem strak aan,.}{dan worden haar oogen klein en}{de harde blik breekt}\\

\haiku{Na 't middageten,.}{wachten de schriften hij mag}{dat niet verwaarloozen}\\

\haiku{E\'en schakel uit den.}{ketting en heel het ding valt}{rinkelend dooreen}\\

\haiku{Hij blijft er wat naar,.}{kijken het is of hij staat}{op iets te wachten}\\

\haiku{De burgemeester.}{herhaalt het nog eens en stopt}{dan een versche pijp}\\

\haiku{Een ruime voorplaats,,;}{daar moet de piano staan}{een studeerkamer}\\

\haiku{Hij gaat eerst naar zijn.}{vader en hij weet niet hoe}{hij het zeggen zal}\\

\haiku{ginder is 't zand,,.}{zand en nog zand vliegakkers}{waar de wind mee speelt}\\

\haiku{En weer een nieuwe.}{dag en de arbeid is daar}{als een bedreiging}\\

\haiku{Als de furie over,.}{was heeft Barbara er nog}{eens mee gelachen}\\

\haiku{en de kinderen,.}{de eeuwige zorg en de}{slapelooze nachten}\\

\haiku{Een vrouwmensch om aan.}{het roer van een schip te staan}{en te bevelen}\\

\haiku{Een goed mensch, dat mag,,{\textquoteright}.}{gezegd worden een sterk mensch}{zegt meneer pastoor}\\

\haiku{Ge moogt me gelooven...{\textquoteright} {\textquoteleft}?}{dat ik ze gebruiken kan}{Maar wat wilt ge dan}\\

\haiku{{\textquoteleft}Er is geen geschrift,...}{en als ge geen geloof hecht}{aan wat wij zeggen}\\

\haiku{Hij wandelt met haar,,.}{langs de veldpaden door de}{bosschen door de hei}\\

\haiku{Hij loopt eens naar de,;}{school de bloemen mogen niet}{vergeten worden}\\

\haiku{Meester heeft zijn zoon,.}{in den arm genomen z\'o\'o}{voor Anna gestaan}\\

\haiku{Het kind groeide, de,,.}{donkere zoekende oogen}{het zwarte krulhaar}\\

\haiku{Soms gebeurt dat wel, '.}{als hij aan die bemoste}{steenen aant werk is}\\

\haiku{Het krijgt eten, zwarte,,.}{blinkende steenen en dan moet}{het terug binnen}\\

\haiku{Hij hoort den sleutel.}{in het slot knarsen en het}{gepiep van de poort}\\

\haiku{Een roode bloem, een,,.}{gele bloem een witte bloem}{nog veel bloemen}\\

\haiku{Hij staat daar met de.}{handen in de zakken aan}{zijn pijp te trekken}\\

\haiku{{\textquoteleft}Ja,{\textquoteright} zegt zijn vader, {\textquoteleft}.}{zoodaarmee gaan we weerom}{naar den winter toe}\\

\haiku{Dat was de laatste.}{maal dat hij ginder op de}{hoeve geweest was}\\

\haiku{De tijd verglijdt, er,.}{is geen pijn meer zij heeft dit}{geluk nooit gekend}\\

\haiku{Wat, dat weet hij niet,.}{juist maar het is als een roes}{over hem gekomen}\\

\haiku{Het kan het voorjaar,,.}{zijn of de voorbije stormnacht}{of dit ongeluk}\\

\haiku{Hij dronk niet, en we.}{geraakten elken winter}{door zonder vragen}\\

\haiku{Hij ziet het smalle,,.}{gelaat van het kind en de}{harde grijze oogen}\\

\haiku{Hij was een van die,,.}{kinderen die veel nemen}{maar weinig geven}\\

\haiku{Het is hem als een;}{verlichting als hij hoort dat}{ze liefst boven blijft}\\

\haiku{Zijn handen in zijn,.}{zakken hij fluit in den avond}{en is gelukkig}\\

\haiku{Niets moet hij vragen,.}{ze zitten gereed om hem}{alles te geven}\\

\haiku{De zon werkt in den,.}{jongen mast de geur van warm}{hars hangt in de lucht}\\

\haiku{Dat is de meester,,.}{van Rielen en dat is zijn}{vrouw en daar het kind}\\

\haiku{Tweemaal moet ze de,.}{vraag stellen Jane kijkt haar}{met koele oogen aan}\\

\haiku{Hij kijkt haar zwijgend,.}{aan maar die blik is een vraag}{en zij voelt dit zoo}\\

\haiku{{\textquoteleft}Ik zal 't eens aan,{\textquoteright}.}{meneer pastoor vragen zegt}{hij ten einde raad}\\

\haiku{{\textquoteleft}En we hebben thuis,...{\textquoteright}.}{een schoon katje zoo'n schoon jong}{katje en        XXII}\\

\haiku{Hij kan toch al die?}{mannen niet d'een na d'ander}{gaan te voet vallen}\\

\haiku{{\textquoteleft}Ik zal Jane een,{\textquoteright}.}{tas melk warmen zegt hij en}{Anna knikt hem toe}\\

\haiku{hij zet een ernstig.}{aangezicht of hij iets heel}{gewichtigs bedenkt}\\

\haiku{Wat ze naar St. Job,.}{kunnen brengen kunnen ze}{ook naar Vossel\`er doen}\\

\haiku{Hij werpt de kachel,.}{open de vlammengloed danst op}{de hooge zoldering}\\

\haiku{Meester De Ruyck.}{heeft hem trouw verslag over zijn}{poging gegeven}\\

\haiku{En op den koop toe, -.}{heiboeren daar doe'de nog}{niet mee wat ge wilt}\\

\haiku{Het vriest vinnig en.}{op het oksaal tocht het van}{alle windstreken}\\

\haiku{Zijn oogen gaan naar de,?}{mannen rond de tafel wat}{zullen ze zeggen}\\

\haiku{Maar dan wordt zijn blik,.}{weer streng en in dit gelaat}{komt iets hard hoekig}\\

\haiku{Er staat een lampje, -.}{op de tafel en daar dat}{rustbed in den hoek}\\

\haiku{Ze hooren gerucht.}{in de kamer en Anna}{Wouters luistert toe}\\

\haiku{het wilde lied van;}{een dronken vogel op den}{eersten lentedag}\\

\haiku{Als hij weer thuis is,,.}{ziet hij haar hoogrood gelaat}{en haar oogen glanzen}\\

\haiku{Zijn oogen zoeken, hij.}{vindt haar oogen die glanzen en}{onbeweeglijk staan}\\

\haiku{Er is geen tijd meer,,.}{zijn oogen doen pijn het gebeurt}{alles buiten hem}\\

\haiku{Hij schreit niet en toch.}{voelt hij de tranen over zijn}{aangezicht vloeien}\\

\haiku{In den namiddag:}{na de begrafenis heeft}{zijn vader gezegd}\\

\haiku{Als hij buiten gaat,,.}{volgt Karel hem als een hond}{met geslagen oogen}\\

\haiku{Elke slag gonst met,.}{een donker geluid in den}{stam gonst in zijn hoofd}\\

\haiku{Maar dan voelt hij dat,;}{de weerstand vermindert het}{is haast onmerkbaar}\\

\haiku{Hij hoort de stappen,.}{op de trap er is nog dit}{gerucht in den stal}\\

\haiku{{\textquoteright} Hij wacht niet op een,.}{antwoord hij is de heer en}{meester en beveelt}\\

\haiku{Het leven, hij heeft,.}{het ondergaan hij heeft er}{over gezegevierd}\\

\haiku{Maar het kan dan als.}{een storm over hem komen en}{hij ligt verslagen}\\

\haiku{De lauwe dooiwind,.}{speelt in zijn haren het kind}{is ongeduldig}\\

\haiku{Het kleine, scherpe;}{aangezichtje en de groote}{onrustige oogen}\\

\haiku{Meester kan dit kind,.}{niet benaderen hij voelt}{dat en het pijnt hem}\\

\haiku{hij zit met starre.}{droomoogen aan tafel en zijn}{glimlach is geluk}\\

\haiku{{\textquoteright} Mester knikt, hij heeft.}{het luisterend kind in de}{deuropening zien staan}\\

\haiku{En... dat ge hier nu,,?}{zijt Janneke daar is toch}{niets met de jongens}\\

\haiku{Zal ik hem eens over,?}{dien grond spreken daar is nu}{toch nog geen haast bij}\\

\haiku{{\textquoteright} {\textquoteleft}Als ge dat eens wilt,{\textquoteright}.}{doen zegt Janneke Berten}{en hij knikt voldaan}\\

\haiku{{\textquoteleft}Dat was wel eens om, '!}{te doen alsne mensch niet op}{geld of hulp moest zien}\\

\haiku{In dien man leeft het,.}{kind dat hij gekend heeft het}{is lang geleden}\\

\haiku{Den volgenden keer,.}{nemen we een blijspel of}{maar weer een drama}\\

\haiku{Rijpe bessen en.}{de vochtige geuren van}{mos en rottend hout}\\

\haiku{Misschien ontdekken,:}{ze hem wel en glimlachen}{ze hem toe zeggend}\\

\haiku{Er staat werkelijk.}{iemand over hem gebogen}{en zijn hart valt stil}\\

\haiku{{\textquoteright} De blauwpurpere.}{bessen reiken tot den rand}{van de kleine kan}\\

\haiku{{\textquoteleft}We moeten naar huis,,.}{gaan het is middag Stanske}{heeft al geroepen}\\

\haiku{Hij gaat met lichten,, -.}{stap zoo in den droom hij heeft}{kabouters gezien}\\

\haiku{{\textquoteright} En plots is daar een.}{lang gerekte angstkreet en}{onderdrukt gesnik}\\

\haiku{Hij luistert scherp toe,.}{of hij dat ingehouden}{geschrei niet meer hoort}\\

\haiku{{\textquoteright} In den zoon heeft hij,.}{den vader ontdekt het is}{niet te loochenen}\\

\haiku{ze maakt zijn sjaal los,.}{ze heeft zijn sloffen onder}{de kachel gezet}\\

\haiku{ze staat aan 't vuur, '.}{ze bukt en zit wat int}{vuur te leuteren}\\

\haiku{Meester heeft het maar.}{\'e\'ens moeten vragen aan}{Janneke Berten}\\

\haiku{Zij steekt haar handen,,.}{uit zij ziet dit glimlachend}{mager aangezicht}\\

\haiku{Zijn moeder die naar.}{hem toekomt en haar hand die}{op zijn voorhoofd ligt}\\

\haiku{{\textquoteleft}Ja, een erg geval,{\textquoteright}.}{knikte de baas en zijn stem}{glijdt er licht over heen}\\

\haiku{Fik van Sooi Delles,.}{zit thuis het gaat ellendig}{traag met die wonde}\\

\haiku{de menschen zetten.}{dat op een kast of hangen}{het tegen den muur}\\

\haiku{Fik heeft er zoo maar,.}{wat verf aangestreken zoo}{wat voor zijn plezier}\\

\haiku{Als ze buiten is,,.}{gaat meester naar de kast daar}{staat zijn papierke}\\

\haiku{{\textquoteleft}Ge moet niet van de.}{slimsten zijn om te weten}{vanwaar die wind komt}\\

\haiku{Na d'hoogmis is hij,.}{er mee begonnen het is}{geen kleine karwei}\\

\haiku{En hoe legt Fienke?}{van Sooi Delles dat aan boord}{met al dat klein volk}\\

\haiku{De Ruyck heeft geen,.}{last met de melodie ze}{hangt in zijn vingers}\\

\haiku{Maar hij wordt naar haar.}{gezogen en hij weet het}{en verweert zich niet}\\

\haiku{Als hij weer over het,,}{klavier gebogen zit zal}{hij piet opzien}\\

\haiku{Over den toog ligt de.}{herbergier gebogen in}{gesprek met \'e\'en klant}\\

\haiku{Hij ziet nog licht door.}{de raamspleten als hij in}{het gangetje komt}\\

\haiku{Hij staat ginder aan.}{den toog en luistert naar haar}{los en luchtig woord}\\

\haiku{Als hij buiten staat.}{in de eenzame straat moet}{hij zich losrukken}\\

\haiku{Met harde oogen staart.}{hij in den nacht en uren lang}{ligt hij wakker}\\

\haiku{Dat rooft wel veel tijd,,.}{en zijn kinderen die zijn}{nogal veel alleen}\\

\haiku{Zijn oogen gaan over de,,.}{werf de lage gebouwen}{den rustigen dries}\\

\haiku{De zegenende hand.}{gaat over hem en deemoedig}{maakt hij het kruisteeken}\\

\haiku{Er is een rust die,.}{hem aangrijpt in de lucht hangt}{een blije verwachting}\\

\haiku{Als 't schooljaar uit,.}{is komt Fred van den meester}{met prijzen naar huis}\\

\haiku{Hij staat v\'o\'or het volk.}{en toont den bloedbevlekten}{mantel van Cesar}\\

\haiku{Ze zit nog niet op,.}{den hoogsten kant maar dat hij}{zoo iets durft zeggen}\\

\haiku{{\textquoteright} Hij rilt onder de.}{harde woorden die hem als}{zweepslagen treffen}\\

\haiku{En zijn plaats op de,,.}{eerste bank blijft open vader}{tot hij terug is}\\

\haiku{Als ze daar met de.}{appels zijn moet de bende}{op een lange rij}\\

\haiku{{\textquoteright} De laatste woorden,.}{fluistert hij of een pijn hem}{onverhoeds overvalt}\\

\haiku{Zijn blokken gaan naar,,.}{Turnhout naar Herentals tot}{in de Walen toe}\\

\haiku{Ze komt voor haar kind.}{vragen wat zij vroeger voor}{zichzelf gevraagd heeft}\\

\haiku{s Anderdaags vroeg.}{in den morgen vertrekt hij}{met de eerste tram}\\

\haiku{Hij zit in de stad,.}{en dat is ver. En een mensch}{met een geleerdheid}\\

\haiku{Het is begonnen,.}{met de nieuwe stallen nu}{jaren geleden}\\

\haiku{Hij zit er over na '.}{te peinzen als zes avonds}{in den heerd zitten}\\

\haiku{Merieke kent zijn {\textquoteleft}!}{stap al.Die\"e van Janneke}{Berten is daar weer}\\

\haiku{{\textquoteleft}Merieke van den.}{meester en dien jongste van}{Janneke Berten}\\

\haiku{{\textquoteleft}En als ge 't nog,.}{beter wilt weten moet'te}{maar eens komen zien}\\

\haiku{De vacantie is, '.}{daar en int meestershuis}{wordt trouwfeest gevierd}\\

\haiku{Hij kuiert wat in,.}{het tuintje hij loopt daar als}{een leeuw in de kooi}\\

\haiku{Hij vraagt naar Sooi en.}{hij verneemt dat hij met den}{ploeg in d'akkers zit}\\

\haiku{Hij slaapt als een steen,,.}{maar als de eerste haan kraait}{staat hij op de werf}\\

\haiku{Een goeien dag en,,.}{anders niets wel vrank en open}{maar daarmee gedaan}\\

\haiku{{\textquoteleft}Ik ben nu toch nat, '.}{k zal die medicijnen}{maar eerst wegdragen}\\

\haiku{Steeds begrijpend, - maar.}{nu is zijn jongen daar en}{d\`at begrijpt hij niet}\\

\haiku{{\textquoteright} Nelleke kijkt om.}{naar het natte spoor van haar}{voeten op den vloer}\\

\haiku{Hij glimlacht als hij,.}{het kind ziet zoo'n leerlinge}{heeft hij nooit gehad}\\

\haiku{Misschien van Rielen,, '.}{ook wel of neen daar zijns}{avonds zoo laat geen trams}\\

\haiku{Zijn oogen glijden over,.}{zijn stoere krijgers over het}{jammerende kind}\\

\haiku{Hij legt alles op.}{het tafeltje en zijn oogen}{hangen aan het kind}\\

\haiku{In de deur van de.}{kamer staat een meisje en}{hij herkent ze niet}\\

\haiku{Meester is langs de.}{akkers gegaan en hij heeft}{de weien gezien}\\

\haiku{Dat is het niet, dat,,,!}{is het niet maar onze Peer}{onze Peer zoo vroeg}\\

\haiku{Fred jongen, als hij,,.}{nu nu op dit oogenblik}{in de stad kon zijn}\\

\haiku{Rielen, Rielen, het,.}{doode Rielen dat aan zijn}{oogen voorbij wandelt}\\

\haiku{Voor een artiste;}{gaat het van het eene seizoen}{in het andere}\\

\haiku{En Mieke moet de.}{kleine mannen eten geven}{en dan naar bed doen}\\

\haiku{Hij kan niet vluchten,.}{in een melodie in de}{wereld van een boek}\\

\haiku{{\textquoteright} De stem is zeer zacht,.}{een donker gefluister als}{van een die moe is}\\

\haiku{{\textquoteright} vraagt meester, hij kan.}{dit angstige vermoeden}{haast niet verbergen}\\

\haiku{Hij zal met een lied,.}{beginnen dat brengt stemming}{en bindt de aandacht}\\

\haiku{Het moet hem zoet zijn...}{terug te schouwen op den}{afgelegden weg}\\

\haiku{En Lina van den,?}{burgemeester wat heeft hij}{hooren vertellen}\\

\haiku{Zijn eigen fout, neen,,.}{niet Freds fout maar hijzelf mag}{dat op zich nemen}\\

\haiku{Waarom heeft hij hem,?}{laten gaan waarom heeft hij}{steeds toegegeven}\\

\haiku{Meester trekt er de.}{eerste vacantiedagen}{mee naar de hoeve}\\

\haiku{{\textquoteright} Meester hoort hoe die,.}{stem mild geworden is en}{dat grijpt naar zijn hart}\\

\haiku{En zij zelf koopt maar,,.}{kinderen niets dan meisjes}{alle jaren \'e\'en}\\

\haiku{Hij ligt alleen en.}{ziet de drijvende wolken}{en den hoogen hemel}\\

\haiku{Het leven dat hem,.}{geraakt heeft hij weet niet wat}{er van komen moet}\\

\haiku{De avond daalt, er ligt.}{een onzeglijke teerheid}{over al de dingen}\\

\haiku{Dat is lijk geld, als '.}{get kunt gebruiken hebt}{g'er gewoonlijk geen}\\

\haiku{Meester kijkt van zijn,.}{schrijfboeken op de zware}{stilte is voelbaar}\\

\haiku{{\textquoteright} Meneer de deken,.}{knikt hij kent Rielen en hij}{kan niet weigeren}\\

\haiku{Hij ziet wel hoe zijn.}{vicaire wenkt en hij schudt}{glimlachend het hoofd}\\

\haiku{Hij leeft er veertien.}{dagen van en dan is het}{verlangen daar weer}\\

\haiku{{\textquoteright} zegt Tistje en hij,.}{prakkezeert nog wat over wind}{regen en slecht weer}\\

\haiku{Hij weet wel dat hij,.}{beter zwijgen zou maar hij}{kan er niets tegen}\\

\haiku{Dan buigt Tistje, hij;}{is heel en al deemoed en}{onderdanigheid}\\

\haiku{Het verrast hem niet,.}{hij glimlacht vreemd als hij den}{brief in handen heeft}\\

\haiku{de nieuwe meester,.}{wacht al ongeduldig zijn}{leven begnit}\\

\haiku{{\textquoteleft}Jongens,{\textquoteright} zegt meester,.}{de Ruyck maar hij voelt dat}{hij niet verder kan}\\

\haiku{Hij kent hun namen,;}{ze hebben bij hem op de}{banken gezeten}\\

\haiku{Volk dat hier geweest,,.}{is een echte processie}{dat mag ik zeggen}\\

\haiku{En ja, ik vergeet,...}{er nog ze zullen mij wel}{te binnen vallen}\\

\haiku{In de kerk is dit.}{donker geroezemoes}{van de hoogdagen}\\

\section{Fr. Hendrichs}

\subsection{Uit: De maaier van den dood (onder ps. Edw. Halliwells)}

\haiku{Ik zal Uw koffers,.}{halen ik ben blij dat het}{zoo onschuldig is}\\

\haiku{{\textquoteright} {\textquoteleft}Ik weet nog niets van{\textquoteright},,}{een gestolen kapdoos af}{verbaasde hij mij}\\

\haiku{Maar de hooge Schot kwam:}{eerst met uitgestoken hand}{op mij toe en sprak}\\

\haiku{{\textquoteright} {\textquoteleft}Alles hangt er van,{\textquoteright},, {\textquoteleft}.}{af Edward verstomde hij}{mijluister verder}\\

\haiku{Miss Allwough is,;}{inderdaad zijn nicht en ik}{weet dat zij hem vreest}\\

\haiku{Waarom de nicht haar,;}{honingzoeten neef ducht kon}{ik niet ontdekken}\\

\haiku{Den geheelen dag,}{had ik geen verstandig woord}{met hem gewisseld}\\

\haiku{Het was werkelijk,.}{geen verlokkend vooruitzicht}{dat hij mij opende}\\

\haiku{ik verwacht, zal hij,,;}{je te lijf gaan indien hij}{begrijpt wat je zoekt}\\

\haiku{ik voelde dat mijn.}{lach niet oprecht was en ik}{rilde als van kou}\\

\haiku{De tuinarchitect.}{Allcott een nog grootere}{geheimzinnigheid}\\

\haiku{het stond bij mij vast,;}{dat ik niet in zou breken}{bij miss Rudgewood}\\

\haiku{{\textquoteleft}Iemand klom op een,,.}{stoel om die schilderij te}{verschuiven zegt hij}\\

\haiku{Was deze oude,?}{David zulk een huichelaar}{dat hij Ethel verried}\\

\haiku{eerst toen hij wist, dat....}{Ethel's papieren open en bloot}{op tafel lagen}\\

\haiku{De persoon, die zij,.}{aldus in vertrouwen nam}{had haar verraden}\\

\haiku{Nu werd mijn ijver.}{eerlijker beloond dan in}{de vorige uren}\\

\haiku{{\textquoteleft}Mag ik weten hoe,,?}{je werkdag gelijk je het}{uitdrukt werd besteed}\\

\haiku{een krachtig meisje,.}{voorvoelt instinctmatig het}{gevaar dat haar dreigt}\\

\haiku{{\textquoteleft}Ik weet het niet meer,.}{ik weet ook niet met welk recht}{U mij ondervraagt}\\

\haiku{Doch mijn pleegvader,.}{is niet in staat tot eenige}{laagheid Mijnheer Gould}\\

\haiku{even lichtte zij het,.}{hoofd om mij een dankbaren}{blik na te zenden}\\

\haiku{Een pleegvader, die!}{zijn kind zielslief heeft en het}{rampzalig maakte}\\

\haiku{Ik geloof, dat mijn.}{vriend een spoor ruikt en dan is}{hij verschrikkelijk}\\

\haiku{dan sprak Elliot:}{met zijn aantrekkelijke}{gemoedelijkheid}\\

\haiku{{\textquoteright} {\textquoteleft}Nog ongeveer een,.}{jaar toen vonden zij elders}{een betere plaats}\\

\haiku{{\textquoteleft}Ik zal haar schrijven,,,.}{dat je haar bedroog Joe er}{rest mij niets anders}\\

\haiku{{\textquoteright} {\textquoteleft}Ik zal open met U,?}{spreken U gelooft immers}{niet in Harold's schuld}\\

\haiku{Gingen er geheel?}{nieuwe personen in het}{drama optreden}\\

\haiku{{\textquoteright} ~ {\textquoteleft}Ik denk, dat het;}{gele boekje mij nog even}{te pas zal komen}\\

\haiku{{\textquoteright} {\textquoteleft}Is de rijksweg de?}{eenige weg van Bathhurst}{naar de grasvelden}\\

\haiku{{\textquoteright} {\textquoteleft}De Draak{\textquoteright} lag eveneens {\textquoteleft}{\textquoteright},?}{rechts vanDe Schotsche Leeuw als}{ik mij niet bedrieg}\\

\haiku{Dat was alles wat.}{hij losliet en ik bleef even}{wijs als tevoren}\\

\haiku{Vrouwen verbeelden,{\textquoteright}.}{zich wel meer wat wierp ik er}{oppervlakkig uit}\\

\haiku{het is meer dan eens.}{een vraag van wichelroede}{en telepathie}\\

\haiku{Na een paar honderd:}{schreden kwam Joe ons uit een}{zijpad in de flank}\\

\haiku{{\textquoteright} Dan drukte hij mij {\textquoteleft},{\textquoteright}, {\textquoteleft}.}{hartelijk de hand.Om Gods}{wil sprak ikspaar je}\\

\haiku{hij nam Mac Pherson's}{papieren uit diens zakken}{en wij verlieten}\\

\haiku{Omdat de spoorlijn,,;}{waar Doberney aan ligt toen}{nog niet gebouwd was}\\

\haiku{Joe zal er wel niet,.}{op tegen hebben dat ik}{een pijp bij hem rook}\\

\haiku{Ik vroeg, waarom Mr..}{Mac Pherson zooveel belang}{in ons kon stellen}\\

\haiku{ik vond de vraag zoo,.}{zot dat ik weer uitbarstte}{in een schaterlach}\\

\haiku{{\textquoteleft}Het kon immers niet,,{\textquoteright}.}{anders zijn Joe vervolgde}{hij gemoedelijk}\\

\haiku{{\textquoteleft}Neen, Joe, kijk mij maar,.}{frisch en vroolijk aan want ik}{heb goed nieuws voor je}\\

\haiku{Dan stond Elliot,.}{op greep zijn glas vast en hief}{het bevend omhoog}\\

\haiku{Blijf onze vriend, wij,.}{blijven de Uwe onze dank}{is onsterfelijk}\\

\section{Leo Herberghs}

\subsection{Uit: Gehuchtenboek}

\haiku{Het kloostertje is.}{te groot voor de weinige}{zusters die er zijn}\\

\haiku{Hotels in het groen.}{verborgen die je bereikt}{langs stenen trappen}\\

\haiku{Oud-Eijsden.}{is grijzig en verward in}{dromen van eertijds}\\

\haiku{{\textquoteleft}'n Varken haalt zelfs{\textquoteright},.}{uit deze modderbrei nog}{voedsel zegt de boer}\\

\haiku{De zon gaat onder,.}{achter wolken hoewel het}{nog lang geen avond is}\\

\haiku{Naast de sleepboot rust.}{een Kempenaer die in}{reparatie is}\\

\haiku{Die berg met aarde.}{die u hier ziet hebben we}{naarboven gehaald}\\

\haiku{Een derde boer zit.}{op de hooizolder waar hij}{hooi bij elkaar veegt}\\

\haiku{Ik heb er een die:}{helemaal van hout is maar}{die staat op zolder}\\

\haiku{In de stoel bij de,,.}{haard bij een breiwerkje zit}{een slapende kat}\\

\haiku{bosjes die vermist.}{aan de grens van vindbaar en}{onvindbaar liggen}\\

\haiku{Soms denk je dat een.}{rij bomen de plaats aangeeft}{waar je zoeken moet}\\

\haiku{Rott*~         Overheersend.}{grijze daken die diep naar}{beneden buigen}\\

\haiku{Er zijn antieke, {\textquoteleft}{\textquoteright} {\textquoteleft}{\textquoteright}.}{gelige treinstellen met}{Raucher enBuffet}\\

\haiku{Een kindergrafje,,.}{aandoenlijk blauw geschilderd}{ligt half gekanteld}\\

\haiku{{\textquoteright} De vrouw brengt ons naar.}{het bakhuisje en doet de}{sluitpen van de deur}\\

\haiku{De boerenvrouw wijst.}{naar het huis tegenover het}{hare dat leeg ligt}\\

\haiku{Achter de heuvels,,.}{links ligt Cottessen en rechts}{ook onzichtbaar Epen}\\

\haiku{Op het uiteinde.}{van de bebouwing wisselt}{het land de stad af}\\

\haiku{zodat ik haastig.}{retireer als een soldaat}{die vlucht voor spervuur}\\

\subsection{Uit: De laatste nachtegaal}

\haiku{Geen stank, geen lawaai!}{en zelfs geen toeristische}{route voor je deur}\\

\haiku{De burgemeester,.}{had toevallig net spreekuur}{dus dat kwam goed uit}\\

\haiku{De burgemeester ().}{zijn naam was Grotermaar was}{blij dat hij hen zag}\\

\haiku{In ons nabuurdorp.}{Rode trekken ze zich niks}{van de natuur aan}\\

\haiku{Die goeie raad van zijn.}{collega had hij zelf ook}{kunnen bedenken}\\

\haiku{Maar buurman Teuge.}{interesseerde zich niet}{voor het weggetje}\\

\haiku{Boer Strijdbaar, de man,.}{van Mien had suffend bij de}{kachel gezeten}\\

\haiku{Heeft hij ook gedacht?}{aan een dienstencentrum met}{sjoelbakkenkamer}\\

\haiku{Als ik jou was zou{\textquoteright}.}{ik maar eens met de heren}{kontakt opnemen}\\

\haiku{{\textquoteleft}Zo,{\textquoteright} zei Krombrood, {\textquoteleft}dat,.}{wist ik niet maar ik geloof}{wel dat het goed is}\\

\subsection{Uit: Mijn gezamenlijk geknor}

\haiku{Ik ga dan naar die}{persoon toe en duw hem er}{een paar in de hand.}\\

\haiku{Hij weet niet eens of.}{de waarheid een staart heeft of}{alleen maar een rug}\\

\haiku{Met het gevolg dat.}{je nooit een vis boven op}{de trap ziet staan}\\

\haiku{Nu eens slaat de deur,.}{te hard dicht dan weer klemt het}{raam of lekt het dak}\\

\haiku{en je je kieuwen.}{en je zwemvliezen niet hoeft}{te gebruiken}\\

\haiku{Dat moeten ze zelf.}{weten en je moet je er}{niet mee bemoeien}\\

\haiku{Ik zeg vaak knie als.}{ik voet bedoel en neus als}{ik oor wil zeggen}\\

\subsection{Uit: Wie zwemt is keg. Een handleiding voor actief niet-zwemmen}

\haiku{Als het niet naar school,.}{had moeten gaan had het ook}{geen schoolslag geleerd}\\

\haiku{Mij hebben ze het.}{tenminste nooit gevraagd en}{ik ken de schoolslag}\\

\section{Jacob Hiegentlich}

\subsection{Uit: 1907-1940. Een joods artist tussen twee oorlogen}

\haiku{{\textquoteleft}Kort geding tegen {\textquotedblleft}{\textquotedblright}{\textquoteright} (,), {\textquoteleft}}{deJongerenDe Nieuwe}{Gids 1931 I p. 615}\\

\haiku{Wat er omgaat in:}{deze jonge dichter is}{zuiver getekend}\\

\haiku{De compositie.}{van zijn romans is lang niet}{onberispelijk}\\

\haiku{C.L. Sciarone.}{heeft echter in Den Gulden}{Winckel van Sept}\\

\haiku{Siegfried was verheugd,;}{al sprak Pastoor wellicht ook}{tot anderen zo}\\

\haiku{Hij was ontwaakt en,:}{uit z'n roes herrezen z'n}{ogen stonden ernstig}\\

\haiku{Het was misschien een, - '.}{gave zoals ook dichter}{zijnn gave was}\\

\haiku{Op school hadden de, '}{leerlingen er van gehoord}{ze vroegen oft}\\

\haiku{ze ziet en 's avonds.}{schreef hij verzen en toch was}{hij niet gelukkig}\\

\haiku{Hij betreurde nog,.}{steeds de loopbaan die men voor}{hem gekozen had}\\

\haiku{Maar in 't begin.}{van de begrafenis was}{hij zeer geschrokken}\\

\haiku{{\textquoteleft}Er is maar \'e\'en God.}{en dat is die van ons en}{dat is de beste}\\

\haiku{Dan zitten ze op.}{een verwarmd terras bij de}{grote potkachel}\\

\haiku{Zelfs in Frankrijk schrijft,:}{men over hem vertaalt men een}{gedicht aan zijn vrouw}\\

\haiku{men lacht zuurzoet, uit '.}{angst voorn slecht artist te}{worden gehouden}\\

\haiku{{\textquoteleft}Nee, 't is nog zo'n, '.}{mooie avond ik ga nog liever}{wat opt terras}\\

\haiku{Onder de douche,:}{viel hem Baudelaire in}{hij declameerde}\\

\haiku{onafhankelijk.}{van elkaar hier en b.v. in}{China opduiken}\\

\haiku{De Jodenhaat is ',.}{n gevoel dat bij velen}{onverdelgbaar tiert}\\

\haiku{(Wunderhorn) Maar ook.}{deze Stolz zal Hitler c.s.}{ten kwade duiden}\\

\haiku{De cultuur van den.}{overwinnaar overstroomt die van}{den overwonnene}\\

\haiku{Het gezicht uit mijn}{raam op de uitgestrekte}{groenwitte velden}\\

\haiku{Welnu, bij dezen;}{vitalen Oostjood is van}{waanzin geen sprake}\\

\haiku{(Posthuum verschenen,).}{met een Voorwoord door Roelfien}{van Blokhuysen}\\

\section{Henri van der Hoek}

\subsection{Uit: De man van Timbuctu}

\haiku{{\textquoteright} {\textquoteleft}Dat kan best zijn, maar.}{ik begrijp toch niet hoe gij}{dat zoo raden kunt}\\

\haiku{Aldus kwam er eene.}{geheele verandering}{in het huishouden}\\

\haiku{{\textquoteleft}Ik hoor daar iets, wat,}{niet voor eene goede wijze}{van opvoeding pleit}\\

\haiku{ligt nog z\'o\'o versch in,.}{mijn geheugen alsof het}{pas gister voorviel}\\

\haiku{Het leventje, dat,.}{ik nu leidde beviel mij}{buitengewoon goed}\\

\haiku{{\textquoteright} vroeg ik mijn oom, die.}{al mijne bewegingen}{had gadeslagen}\\

\haiku{Was hij vroeger al,.}{niet erg mededeelzaam thans}{sprak hij maar zelden}\\

\haiku{Ik besloot dus maar.}{om niets meer te vragen en}{zoo bleef het er bij}\\

\haiku{Er stond een groote C {\textquoteleft}.}{boven aan het papier en}{daaronder stondPars}\\

\haiku{{\textquoteright} Hierop liep hij op.}{het bureau toe en plaatste de}{lamp op het schrijfblad}\\

\haiku{Vervolgens schoof hij.}{een stoel bij en bekeek het}{slot van het bureau}\\

\haiku{Iemand die rechts is!}{zal toch aan den linker kant}{geen afdruk maken}\\

\haiku{{\textquoteright} Hierna wees hij mij,.}{naar de plaats waar de pinnen}{behoorden te zijn}\\

\haiku{Daar is, geloof ik,,{\textquoteright}.}{de vijver hernam Born in}{de verte ziende}\\

\haiku{Indien gij even hier,.}{wilt komen kunt gij  er}{u van overtuigen}\\

\haiku{Het is de indruk,.}{van een touw waarmede de}{boot was vastgemaakt}\\

\haiku{Ondertusschen zijn.}{wij de lengte der boot te}{weten gekomen}\\

\haiku{Neen, gij behoeft niet,,{\textquoteright}.}{te zoeken dat heb ik al}{gedaan vervolgde}\\

\haiku{{\textquoteleft}Indien gij de lamp,,}{wilt vasthouden zal mij dit}{hoogst aangenaam zijn}\\

\haiku{{\textquoteleft}Er valt hier niets meer,,.}{te verrichten meester we}{zullen naar huis gaan}\\

\haiku{Maar nu hoorde ik,:}{de stem van mijn hospita}{die tot de meid riep}\\

\haiku{Ik was natuurlijk;}{heel vriendelijk voor haar en}{vroeg hoe zij heette}\\

\haiku{Ik herhaal het, om.}{\'een of twee uur was het nog}{tijd genoeg geweest}\\

\haiku{doe dus of gij thuis,.}{waart steek eens op en neem er}{dan uw gemak van}\\

\haiku{{\textquoteleft}Ja, en wij hebben -.}{allen kans van slagen dank}{zij het vroege uur}\\

\haiku{Niet \'e\'en enkele;}{keer behoefde Born hem een}{order te geven}\\

\haiku{En nu stond hij te;}{ginne-gappen met mijn}{sigaar in zijn mond}\\

\haiku{{\textquoteright} {\textquoteleft}Dat zag ik aan het.}{eindje sigaret dat hij}{weggeworpen had}\\

\haiku{En vertrek nu, mijn,'!}{brave en dat des Heeren}{zegen op u daal}\\

\haiku{Ik bemerkte een.}{ondeugende flikkering}{in de oogen van Born}\\

\haiku{{\textquoteleft}Neem plaats, inspecteur,{\textquoteright};}{sprak hij op zijn gewonen}{vriendelijken toon}\\

\haiku{Maar ik liet niets van.}{mijn vermoeden blijken en}{liep kalm naast haar voort}\\

\haiku{Het papier, dat eerst,;}{op de borst van den doode}{lag was verdwenen}\\

\haiku{En toen nam ik haar.}{mede naar het bureau en}{daar zit ze nu nog}\\

\haiku{want om de eerste.}{nacht reeds te laten waken}{vond ik overbodig}\\

\haiku{Full Speed rende hij.}{nu de Singel op en \'een}{der zijstraten in}\\

\haiku{Dus eindelijk zou!}{dan de oude heer Dubois}{gewroken worden}\\

\haiku{{\textquoteright} {\textquoteleft}Of hij heeft een meer,{\textquoteright}.}{winstgevend baantje kunnen}{krijgen ging ik voort}\\

\haiku{{\textquoteleft}Inspecteur,{\textquoteright} ging hij, {\textquoteleft};}{voorter zal natuurlijk wel}{een loods aan boord zijn}\\

\haiku{Na hem volgden ik,;}{inspecteur van Noort en ten}{laatste de agenten}\\

\haiku{Het was, alsof er.}{een troep wilde dieren aan}{het vechten waren}\\

\haiku{Wij legden hem bij}{het vuur neder en deden}{zooveel mogelijk}\\

\haiku{Op de plaats waar hij,.}{met het hoofd gelegen had}{ontdekte ik bloed}\\

\haiku{Weldra kwamen wij,.}{aan het station waar de}{trein juist gereed stond}\\

\haiku{ik wachten v\'o\'or ik.}{hem het doodende staal door}{het hart kon stooten}\\

\haiku{Met Dubois had ik.}{afgesproken  daar om}{drie uur te komen}\\

\haiku{{\textquoteright} Deze raad was nog.}{zoo slecht niet en vol vreugde}{spoedde ik mij heen}\\

\haiku{ik was en uitte.}{vol dankbaarheid een niet te}{weerhouden juichkreet}\\

\haiku{Wij drongen het bosch.}{binnen en spoedden ons zoo}{snel mogelijk voort}\\

\haiku{Ondertusschen zocht.}{ik naar een gelegenheid}{om te ontsnappen}\\

\haiku{Welnu dan kon het.}{mij ook niet schelen of hij}{leefde of dood was}\\

\haiku{Ik was koortsig en,.}{opgewonden wat niet te}{verwonderen was}\\

\haiku{Tot mijn groote blijdschap,.}{bemerkte ik dat wij naar}{het noorden trokken}\\

\haiku{Ik zal als tolk voor,.}{u dienen want mijn vader}{verstaat uw taal niet}\\

\haiku{En toch, bij nader,.}{inzien moest ik hem wel van}{den moord verdenken}\\

\haiku{Ikzelf was nu heer.}{en meester en stond dus in}{rang gelijk met hem}\\

\haiku{{\textquoteleft}Als vader niet goed,,{\textquoteright},}{is zal ik je wel komen}{waarschuwen zei ik}\\

\haiku{{\textquoteright} {\textquoteleft}Neen,{\textquoteright} drong ik aan, {\textquoteleft}de,;}{weg lijkt mij nu vrij ik kan}{niet langer wachten}\\

\haiku{Hadt je nu ooit wel,?}{kunnen denken dat ik je}{nog eens vinden zou}\\

\haiku{Vandaar zal hij wel.}{naar het ooglijdersgesticht}{worden overgebracht}\\

\section{B\`er Hollewijn}

\subsection{Uit: Als ratten in de kerk. Een dorpsgeschiedenis uit een nog jong verleden}

\haiku{Maar in plaats van zijn,:}{toorn de vrije loop te laten}{zei pastoor Klabbers}\\

\haiku{In de naam van de,.}{Vader en de Zoon en de}{Heilige Geest Amen}\\

\haiku{Ze liet hem in de.}{wachtkamer en spoedde zich}{naar haar heer-broer}\\

\haiku{Je kunt niet op straat.......}{lopen of je wordt door u}{weet wel overvallen}\\

\haiku{Maar ja, de freule.}{was raar en rare mensen}{doen rare dingen}\\

\haiku{Als je een kat bij,.}{een stuk vlees zet moet het al}{een sterke kat zijn}\\

\haiku{Doe de groeten van.}{mij thuis en zeg dat ik eens}{gauw kom aanlopen}\\

\haiku{{\textquoteleft}Stop,{\textquoteright} zei Verstegen,}{nog voordat mijnheer pastoor}{was uitgesproken}\\

\haiku{{\textquoteright} {\textquoteleft}Wis en waarachtig,{\textquoteright}.}{protesteerde de boer weer}{met de oude gloed}\\

\haiku{{\textquoteright} Zijn vrouw verkeerde.}{in een vergevorderde}{staat van opwinding}\\

\haiku{Maar je ervaring}{met de jongemannen hier}{komt mij wel van pas.}\\

\haiku{Men had de huizen.}{geverfd en gewit en de}{straten schoongeveegd}\\

\haiku{Maar als er iets is,.}{gebeurd heb  je het aan}{je zelf te wijten}\\

\haiku{{\textquoteleft}Ik heb weinig lust.}{om me door jou in het gips}{te laten drukken}\\

\haiku{{\textquoteleft}Denkt u misschien,{\textquoteright} vroeg, {\textquoteleft}.}{hijdat zo'n nacht me op de}{knie\"en zou krijgen}\\

\haiku{Het leven begint.}{bij veertig en ik ben nog}{geen vijfenvijftig}\\

\haiku{Totdat u me zei,.}{dat ik hen zelf hun leven}{moest laten maken}\\

\haiku{Ik blijf u eeuwig.}{dankbaar voor de raad die u}{mij hebt gegeven}\\

\haiku{Evengoed als ik de,.}{kerk respekteer dient u mijn zaak}{met rust te laten}\\

\haiku{{\textquoteleft}Laat me met rust,{\textquoteright} klonk.}{een schor baritongeluid}{tussen de knie\"en}\\

\haiku{{\textquoteright} zei de pastoor, om.}{het gesprek een andere}{wending te geven}\\

\haiku{Zachtjes sloop ze de.}{trap op en luisterde aan}{de slaapkamerdeur}\\

\haiku{{\textquoteright} hoorde ze gedempt, {\textquoteleft}?}{mevrouw jammerenwaarom}{komt er geen kind in}\\

\haiku{Het kristendom is,,.}{de bezieling het zuurdeeg}{dat alles doordringt}\\

\haiku{Met een onbestemd.}{gevoel in zijn binnenste}{zwoegde hij verder}\\

\haiku{De oorzaak lag nu,.}{niet bij de voetbalklub maar bij}{mijnheer pastoor zelf}\\

\haiku{Je hebt in een uur.}{tijd zes sigaretten de}{lucht in geblazen}\\

\haiku{{\textquoteleft}Wanneer het nog ooit,.}{zover komt zal ik  u}{nader berichten}\\

\haiku{Vier en vier... zestien...,...}{en zestien tien en tien is}{twintig zes en zes}\\

\haiku{Lena hoorde hem.}{naar boven strompelen en}{kwam de keuken uit}\\

\haiku{Ik nam haar onder.}{de armen en probeerde}{haar op te trekken}\\

\haiku{{\textquoteleft}Ik zou niet graag in,{\textquoteright}.}{de schoenen van Suske staan}{meende Verstegen}\\

\haiku{Mijnheer pastoor had.}{zich met een dansleraar in}{verbinding gesteld}\\

\haiku{{\textquoteright} {\textquoteleft}Staat u boven de?}{wet en de plaatselijke}{verordeningen}\\

\haiku{Het optreden van.}{de pastoor begon Suske}{knap te vervelen}\\

\haiku{Hij trok zijn handen.}{van het buffet weg en kwam}{er achter vandaan}\\

\haiku{{\textquoteright} Radeloos liep ze,,.}{de kamer uit de gang op}{naar de buitendeur}\\

\haiku{De dokter...{\textquoteright} Bevend,...}{rende ze weer naar binnen}{naar de telefoon}\\

\haiku{Ze stond op en ging.}{naar de keuken om een glas}{water te halen}\\

\haiku{{\textquoteleft}Ik geloof dat het,{\textquoteright}.}{wel los zal lopen stelde}{hij Lena gerust}\\

\haiku{Geloof me pastoor,.}{het wordt de hoogste tijd dat}{u tot u zelf komt}\\

\haiku{{\textquoteleft}Morgen gaan we een.}{eind verder en dan heb je}{mij niet meer nodig}\\

\haiku{Als je altijd in,.}{de weer bent geweest kun je}{niet zo maar niets doen}\\

\haiku{Het was duidelijk.}{dat de lange priester het}{flink te pakken had}\\

\haiku{Wenen is voor ons.}{een danszaal met heerlijke}{Straussmelodie\"en}\\

\haiku{{\textquoteleft}Als hij mij dat had,.}{geleverd was er geen ruit}{meer heel gebleven}\\

\haiku{Ze hebben u een,{\textquoteright}.}{flink kopje kleiner gemaakt}{stelde de boer vast}\\

\haiku{Voor mij staat vast, dat.}{de burgemeester eerlijk}{en betrouwbaar is}\\

\haiku{In mijn jeugd ging ik.}{altijd bij de oude Driek}{in ons dorp kijken}\\

\haiku{{\textquoteright} {\textquoteleft}Wat hij wil, weet ik,{\textquoteright}.}{niet richtte de pastoor zich}{weer tot de meester}\\

\haiku{Ik heb zo'n idee, dat.}{je het hele gezin hier}{op stang hebt gejaagd}\\

\haiku{Misschien is dit het,{\textquoteright}.}{beste gaf deze met een}{somber gezicht toe}\\

\haiku{Pastoor Klabbers keek.}{ontstemd vanaf de preekstoel}{naar het tafereel}\\

\haiku{Men zag haar knie\"en.}{en ongedekte dijen}{beven en trillen}\\

\haiku{Ze knagen aan de!}{eer en de goede naam van}{hun medemensen}\\

\haiku{Maar in plaats van zijn,:}{toorn de vrije loop te laten}{zei pastoor Klabbers}\\

\haiku{Abraham, Isaak en.}{Jakob leefden duizenden}{jaren geleden}\\

\haiku{Als we de kerk trouw,.}{blijven beloont de hemel}{ons duizendvoudig}\\

\haiku{Dank u hartelijk,{\textquoteright}.}{was alles wat de pastoor}{wist  te zeggen}\\

\subsection{Uit: Betsche (onder ps. Orenz)}

\haiku{Met 'r korte lijf '.}{komt ze nauwelijks boven}{t tafelblad uit}\\

\haiku{Dadelijk begint. ' '.}{dat jong weer overnieuwt Is}{n bodemloos vat}\\

\haiku{Lewieke was 'n,,.}{gave gezonde jongen}{van bijna acht pond}\\

\haiku{Ze moest maar zien met '.}{haar vijf kinderen int}{leven te blijven}\\

\haiku{Leen spaarde alles '.}{wat ze krijgen kon en gaf}{t d'n Hollander}\\

\haiku{Ze was met twee van ',.}{bij hun opt portaal naar}{de kanaal gegaan}\\

\haiku{Dat is niet zoveel.}{werk en ze heeft maar de helft}{leverworst nodig}\\

\haiku{Van Dorus van uit de, '.}{eerste bouw heeft hij ook eens}{n portret gemaakt}\\

\haiku{Bij hen onder woont ', '.}{ookn schilder maar daar kun}{jet niet aan zien}\\

\haiku{Ze kw\'amen, met 'n '.}{deftigheid alsof ze naar}{n soiree gingen}\\

\haiku{'t Viel niet mee om '.}{int hartstikke duister}{de weg te vinden}\\

\haiku{Ze moest niet denken, ' ',.}{dattne salon de luxe}{was zoals bij hun}\\

\haiku{Opeens schrok mevrouw ',.}{doorn geluid dat uit de}{hoek aan de deur kwam}\\

\haiku{Ze had kippenvel.}{en hing haren pelsmantel}{los over de schouders}\\

\haiku{Sodemerakel, wat.}{had die schullever op z'n}{geweten gehad}\\

\haiku{De blanken moesten over '.}{t algemeen niet veel van}{de negers hebben}\\

\haiku{Als ze 'n meisje ' {\textquoteleft}{\textquoteright},.}{hadden enn flessnaps kon}{je over hen lopen}\\

\haiku{Laatst hebben ze hem '.}{metn meisje achter de}{walmuur uitgehaald}\\

\haiku{Ze had gezegd, dat ',.}{t godgeklaagd was wat hij}{allemaal aanving}\\

\haiku{Bij die briefjes maakt,.}{de juffrouw tekeningen}{die ze met krijt kleurt}\\

\haiku{De hele middag{\textquoteright} {\textquoteleft}?}{al.Waarom ben je dan niet}{naar huis gekomen}\\

\haiku{Harie en Lambeer.}{slapen samen in \'e\'en bed}{onder de pannen}\\

\haiku{Ze zagen er uit '.}{alsof ze uitn brandend}{huis waren gevlucht}\\

\haiku{Ze zijn vergeten,,.}{dat ik hun op de bok heb}{geholpen zegt ze}\\

\haiku{Daar konden ze zich '... {\textquoteleft}}{n plaats voor reserveren}{Wat denken ze wel}\\

\haiku{Of 't verstandig,,.}{is dat ie zich Christien pakt}{valt nog te bezien}\\

\haiku{Je hoort 't aan 't,.}{spektakel dat ze nu vlak}{bij de poort maken}\\

\haiku{En ook omdat hij.}{dan weer werken kan en hij}{z'n volle loon krijgt}\\

\haiku{t Is voor Betsche '.}{n hele rekenarij}{om er te komen}\\

\haiku{t Is nu toch goed,{\textquoteright} '.}{wast enige wat ze had}{kunnen antwoorden}\\

\haiku{Z'n vrouw brengt hem z'n.}{eten en dikwijls brengt ze dan}{ook het hare mee}\\

\haiku{Daarom moet hij naar.}{beneden kijken als ie}{tegen Betsche spreekt}\\

\haiku{'n Langgerekte.}{geeuw van Manus doet beide}{vrouwen opkijken}\\

\haiku{Hoe ouder Manus,.}{werd hoe ongelukkiger}{dat hij zich voelde}\\

\haiku{Snotverdutju, dan.}{krijgen ze me met geen tien}{paarden meer in bed}\\

\haiku{{\textquoteright} {\textquoteleft}Als ze dan ook maar,.}{weet dat ze me stikken kan}{met haar gezauwel}\\

\haiku{{\textquoteright} {\textquoteleft}Daar heb je met de...{\textquoteright}.}{Pater niet over te klagen}{Betsche heeft gelijk}\\

\haiku{De Pater heeft nog.}{nooit zo iets gezegd als wat}{Christien er uit flapt}\\

\haiku{Als ie er kans voor, '.}{kreeg zocht hij er altijdn}{stil plekje voor uit}\\

\haiku{Gisterenavond had ', {\textquoteleft}{\textquoteright}.}{iet dan toch gemeend toen}{hijeindelijk zei}\\

\haiku{Je hebt me 'n les '.}{gegeven en toen kreeg ik}{n echte sigaar}\\

\haiku{Dat was trouwens wat,.}{van de conferentie zei}{hij en niet van hem}\\

\haiku{De Pater hielp hem.}{met flessen aangeven en}{de melkkruik vullen}\\

\haiku{Z'n borstkas, die nu,;}{met ribben is getekend}{zal dan gevuld zijn}\\

\haiku{Zo ongeveer is,.}{de voorstelling die Manus}{zich van z'n zoon maakt}\\

\haiku{Lewieke heeft z'n.}{broek aangetrokken en helpt}{haar ijverig mee}\\

\haiku{Als ze 's avonds naast, '.}{elkaar in bed liggen komt}{n hand aan Betsche}\\

\haiku{Als je de tering,.}{hebt moet je je toch al in}{de gaten houden}\\

\haiku{Begin maar eens met.}{vijf kinderen als je de}{hele dag weg bent}\\

\haiku{Ze komen ook niet, ' {\textquoteleft}{\textquoteright}.}{bij haar want dan zouden ze}{gauwnnaam krijgen}\\

\haiku{Als er geholpen,.}{moet worden hoef je niet met}{geld te rammelen}\\

\haiku{De twee halen de...}{schouders op en geven de}{zatlap geen antwoord}\\

\haiku{Enfin, ik ga eerst ',.}{evenn pot bier drinken dan}{kom ik naar je toe}\\

\haiku{Daar werd 't opeens '. '}{stil en toen hoorde Betsche}{n hels spektakel}\\

\haiku{Hij lag tussen z'n,.}{wimpers door te kijken toen}{Betsche binnenkwam}\\

\haiku{Ze wist wel niet wat,.}{saldo en rente was maar}{wat kon dat schelen}\\

\haiku{Hier...{\textquoteright} Op dat moment.}{ging de kaffeedeur open en}{verscheen de Pater}\\

\haiku{Geloemige nog,.}{aan toe wat had Betsche in}{haar rats gezeten}\\

\haiku{Als Manus dood ging.}{kregen ze honderd gulden}{van het dooiefonds}\\

\haiku{Dat zijn wat wilde,,.}{haren die gaan er wel uit}{redeneerde ie}\\

\haiku{Ze keek niet om, maar '.}{begon te schreeuwen alsof}{ze aann mes hing}\\

\haiku{{\textquoteleft}Vuile sokus,{\textquoteright} {\textquoteleft}.}{griende zewaar heb je zo}{lang uitgehangen}\\

\haiku{Met \'e\'en scheur rijt de.}{een de ander z'n hemd tot}{aan de broeksrand open}\\

\haiku{{\textquoteleft}Was je eerst,{\textquoteright} gebiedt, {\textquoteleft}.}{Betsche noganders maak je}{het hele bed geel}\\

\haiku{In het straatje giert.}{de storm van gemaskerden}{naar het hoogtepunt}\\

\haiku{{\textquoteleft}'t Is nog maar tien,{\textquoteright}, {\textquoteleft},.}{minuten zegt Betschekom}{O.L. Heer wacht op ons}\\

\haiku{Gij zijt allemaal,.}{erg bedankt dat ge zo veel}{voor ons gedaan hebt}\\

\subsection{Uit: Brandende aarde. Een brok geschiedenis van de mijnstreek}

\haiku{Het hout kraakte nu.}{achter hen en dezelfde}{holle lach weerklonk}\\

\haiku{Ze kwamen op de.}{ontgonnen heuveltop en}{zochten naar hun hut}\\

\haiku{Ze trachtte hem door.}{gebaren te bewegen}{met hen mee te gaan}\\

\haiku{Enkele dagen.}{later werd er op de deur}{van de hut geklopt}\\

\haiku{Voor hen stond het vast,.}{dat het uitwerpselen van}{de hel waren}\\

\haiku{Door zijn verstorven.}{en harde leefwijze werd}{hij vroegtijdig oud}\\

\haiku{De zwelling van zijn.}{spierballen tekende zich}{onheilspellend}\\

\haiku{{\textquoteright} Christel sloeg angstig '.}{n kruis en schoof schuw langs haar}{broer door naar moeder}\\

\haiku{Erik trok de geldzak.}{te voorschijn en legde hem}{voor zich op tafel}\\

\haiku{Maar toen de Schele,.}{de daad bij het woord voegde}{riep hij hem terug}\\

\haiku{'t Liefst zou hij nooit '.}{\'e\'en brok van z'n glorie aan}{n ander overdoen}\\

\haiku{Hij zou de oude,.}{laten zien dat hij niet bang}{was voor de duivel}\\

\haiku{{\textquoteright} In 't vertrek trok.}{hij zijn buis uit en ging door}{het raam staan turen}\\

\haiku{Hij zocht gezelschap.}{en dronk pinten bier in een}{onmatig aantal}\\

\haiku{Beneden zocht de,,.}{Leeuw stotend en struikelend}{naar de  uitgang}\\

\haiku{In plaats van de vrouw.}{kwam daarna een tweede man}{aan de lier te staan}\\

\haiku{Aan zijn knie\"en en.}{ellebogen vertoonden}{zich schilferplekken}\\

\haiku{Hijgend wreef hij zijn.}{ogen vrij en keek voor zich uit}{zonder iets te zien}\\

\haiku{Nooit meer zou hij een '.}{hand uitsteken omn stuk}{kool te verwerven}\\

\haiku{Ga maar terug van.}{waar je gekomen bent met}{je duivelsgebroed}\\

\haiku{Twee jaren later.}{werd de oude naast zijn vrouw}{in het graf gelegd}\\

\haiku{'t Waren ruwe,.}{kerels voor wie slechts drank en}{vrouwen bestonden}\\

\haiku{Lang heb ik gewacht.}{om de bezitster mijn wraak}{te doen gevoelen}\\

\haiku{Met 'n minachtend.}{lachje draaide hij zich om}{en spoog op de grond}\\

\haiku{'n Vrouw zuchtte en.}{een man liep mopperend de}{kamer op en neer}\\

\haiku{{\textquoteright} Toen hij de ladder,:}{naar de galg beklom bad hij}{met trillende stem}\\

\haiku{Ik zou ook liever,.}{boer blijven als jullie maar}{niets te kort kwamen}\\

\haiku{De zon bleef echter,.}{onberoerd doorschijnen fel}{en genadeloos}\\

\haiku{Voldoening blonk in:}{zijn zwartomrande ogen toen hij}{haar verzekerde}\\

\haiku{Wilhelm liet de arm.}{van z'n vader los en ging}{op de bank liggen}\\

\haiku{Hij voelde zich als '.}{n vreemdeling aan de deur}{van zijn eigen huis}\\

\haiku{Wilhelm lag op de.}{bank tegen de muur met zijn}{slaap te worstelen}\\

\haiku{Er brak een rad en.}{een der mijnen moest geheel}{worden stilgelegd}\\

\haiku{{\textquoteleft}Moeder ook proeven,{\textquoteright} '.}{beet ze eveneens een stuk van}{t veevoeder af}\\

\haiku{Dan verdien ik met {\textquotedblleft}{\textquotedblright}.}{de jongens genoeg om spek}{enweck te kopen}\\

\haiku{In de nabijheid.}{van de Holz werden nieuwe}{schachten aangelegd}\\

\haiku{Met een diepe zucht,.}{liet ze de jongen los die}{blij naar buiten liep}\\

\haiku{Zelfs de winkeliers.}{zagen liever haar rug door}{de deur verdwijnen}\\

\haiku{Ze leken meer op.}{de Tsaar van Rusland dan op}{goede christenen}\\

\haiku{Voorzichtig legden.}{de Ratten hun hoofdman voor}{het huis op de grond}\\

\haiku{Mensen zagen haar '.}{gebogen overt lichaam}{van Rudolf liggen}\\

\haiku{{\textquoteright} Tegen de sterke.}{armen van de pumper was}{Rudolf machteloos}\\

\haiku{Als de baas alles, '.}{weten wilde zou iet}{te horen krijgen}\\

\haiku{Met een nors gezicht, {\textquoteleft}{\textquoteright}.}{zei de man dat hij in het}{vervolgslepen moest}\\

\haiku{Het geweld barstte;}{los en de citadel werd}{in puin geschoten}\\

\haiku{Altijd door zag hij.}{haar lachend gezicht met de}{blinkende tanden}\\

\haiku{Z'n verlangen dreef,.}{hem vooruit in de richting}{van de boerderij}\\

\haiku{Is er bij jullie '?}{op de boerderij nooitn}{ongeluk gebeurd}\\

\haiku{Als hij tot U komt,;}{gehuld in schapenvacht en}{aan Uw oren femelt}\\

\haiku{In ieder geval. '}{had hij de directeur voor}{de poort laten staan}\\

\haiku{Z'n broer Anton had.}{een roodharig kind met een}{bleek sproetengezicht}\\

\haiku{Driekus Joep werd bij.}{deze gelegenheid als}{portier aangesteld}\\

\haiku{De scheldende stem.}{van het rode vrouwmens klonk}{op het kamertje}\\

\haiku{Eindelijk zou het {\textquoteleft}{\textquoteright}.}{in de crypte vanzijn huis}{worden bijgezet}\\

\haiku{Liesbeth kende de '.}{oorzaak van deze spot en}{sloegn zijpad in}\\

\haiku{Als de vroedvrouw kwam,.}{zou het kind niet lang meer op}{zich laten wachten}\\

\haiku{{\textquoteright} zei Driekus Joep fier,.}{tegen Jozef terwijl hij}{z'n jas dichtknoopte}\\

\haiku{Ze had hem een zoon. ',.}{geschonkent Mooiste wat}{een vrouw geven kan}\\

\haiku{De ellende in.}{de arbeidersgezinnen}{groeide met de dag}\\

\haiku{Ze was een flinke, '.}{jonge vrouw geweest metn}{welgevormd lichaam}\\

\haiku{De ranke klipgeit '.}{van weleer groeide uit tot}{n kamerolifant}\\

\haiku{In een winkel kocht ',.}{hijn paar broodjes die hij}{onderweg opat}\\

\haiku{Hans, de jongste, 'n,.}{stumperig ventje dat nooit}{had kunnen lopen}\\

\haiku{'t Was een proper,,.}{huisje goed in de verf met}{witte gordijnen}\\

\haiku{{\textquoteleft}'t Is rotzooi,{\textquoteright} zei,.}{Wilhelm terwijl hij weer op}{de stoel ging zitten}\\

\haiku{De mannen zaten.}{met grimmige gezichten}{aan het tafeltje}\\

\haiku{Dat moet de Bond maar,{\textquoteright}.}{opknappen bromde hij en}{ging naar de voordeur}\\

\haiku{Uit de kooi stapte.}{Wilhelm op de losvloer in}{de grote schachtplaats}\\

\haiku{De Dodekop kwam.}{als laatste door het gat en}{zag de etende mannen}\\

\haiku{Degene, die hij,.}{betrapte kon minstens op}{ontslag rekenen}\\

\haiku{Nauwelijks was hij,,:}{terug of weer klonk het hoog}{en laag hard en hees}\\

\haiku{{\textquoteright} 's Avonds zaten ze.}{samen en spraken over de}{mijn en de lonen}\\

\haiku{De oude Jozef,.}{meende dat hij het gesprek}{wilde afbreken}\\

\haiku{Zelfs in de pijler,,.}{als hij de kool loswroette}{dacht hij aan Greetje}\\

\haiku{{\textquoteright} Met gestrekte arm.}{en bevende vinger wees}{Gerard naar de deur}\\

\haiku{In z'n gedachten.}{had hij met haar geleefd in}{een glanzend huisje}\\

\haiku{Elke Zondag ging '.}{hij naar het huisje op de}{hoek vant pleintje}\\

\haiku{'n Witte zuster.}{begeleidde hem naar de}{r\"ontgenafdeling}\\

\haiku{{\textquoteright} De zuster verstond,.}{wel wat de soldaat zeide}{maar gaf geen antwoord}\\

\haiku{'n Onwillige.}{motor aan de schudgoot had}{hem opgehouden}\\

\haiku{Het ontstuimige.}{bloed van zijn voorvaderen}{joeg door z'n aderen}\\

\haiku{{\textquoteright} Greetje legde haar.}{verstelwerk op tafel en}{kwam bij Wilhelm staan}\\

\haiku{Hij zag haar aan de,.}{deur staan met opgeheven}{hoofd en harde ogen}\\

\haiku{Tranen welden in.}{z'n ogen en drupten door de}{groeven langs zijn neus}\\

\section{P.C. Hooft}

\subsection{Uit: Liederen en gedichten}

\haiku{niet langer dan het,.}{weigeren duurt niet langer}{duurt het minnen}\\

\haiku{Maar 't schijnt wel wie,.}{geen rust en waagt kan kwalijk}{lust gewinnen}\\

\haiku{Indien dit bosje, '!}{klappen kon wat melddet}{al boelage}\\

\haiku{wie geboden dienst, '.}{versmaadt wenst er wel om als}{t is te laat}\\

\haiku{{\textquoteright} ~ {\textquoteleft}Reine liefd' van',{\textquoteright}, {\textquoteleft},!}{d allerreinste zei hij}{Sijbrech bolle meid}\\

\haiku{{\textquoteright} ~ Zij heeft een zweep,;}{ontbo\^on uit Polen die ze}{bij haar kammen hangt}\\

\haiku{Zij mogen niet, uit,,.}{heter borst gemind zijn maar}{slechts aangebeden}\\

\haiku{Zo gaat het waar men.}{naar waardij de godlijkhe\^en}{verzuimt te vieren}\\

\haiku{Uw dubbel nat, door,.}{deze nood verijsd wordt stijf}{als stenen vonder}\\

\haiku{Treurt rozelaar, treurt,,.}{bollen treurt laat vrij de mol}{uw plaats verwild'ren}\\

\haiku{Om daar een parel,,.}{af te halen en streeft zo}{niet door duizend do\^on}\\

\haiku{Zijn hoogste lof, in,,;}{mensenkelen noch donder}{is noch bliksemjacht}\\

\haiku{Hoe kunt gij hen, die,,,.}{u niet zoeken bestoken}{in hun voordeel gaan}\\

\haiku{hij gaat dan ook weer.}{meer liefdespo\"ezie voor}{Christina schrijven}\\

\haiku{Zo noemt hij zichzelf, {\textquoteleft}{\textquoteright}.}{regelmatig Cephalo}{dat is Grieks voorhoofd}\\

\haiku{Amaryl, de deken.}{zacht 17 Opnieuw een lied voor}{Ida Quekels uit 1604}\\

\haiku{7 hem... ga hem (nl.);}{Narcissus bij gebrek aan}{een levensgezel}\\

\haiku{1 Periosta,;}{een rivier mogelijk de}{Amstel of de Vecht}\\

\haiku{5 dode... ruimen ();}{de doden verenvan het}{bed te verlaten}\\

\haiku{In 1630 stuurde hij:}{haar de eerste kersen met}{dit korte versje}\\

\haiku{1-2 plicht... staat de;}{taken bevat van een mens}{die in aanzien staat}\\

\haiku{2 eerst... schoot eerst op,.}{Delfts schoot tenslotte in de}{schoot van de aarde}\\

\section{Samuel Coster en P.C. Hooft}

\subsection{Uit: Warenar}

\haiku{beter eeren,30 En,.}{ick in plaets des giericheyts}{dit huys beheeren}\\

\haiku{gheslaghen,75 Dat hy?}{mijn thienmael op een dach ten}{huys uyt gaet jaghen}\\

\haiku{Een hielen dach sit,}{hy in huys ghelijck as}{op de winckels}\\

\haiku{Gelijc hier Warnar,.}{buers dochter die mocht mijn}{kommer stelpen}\\

\haiku{eenich dralen, Keunje,:}{me niet helpen soo moet ick}{een ander halen}\\

\haiku{Daer mat hy't feyt ten, //}{breetsten uyt en gingh staen met}{veel menty temen,289}\\

\haiku{Ien doodts-hooft seydse,,}{ien doots-hooft wat binje ien}{nuwelijck bloedt?498}\\

\haiku{warnar En as je,?}{ghelt ontfanghen zoudt wat}{doeje dan hier}\\

\haiku{geertruyd Jae wel 'tis, //}{te wonder wat rancken}{datter om gaen.580}\\

\haiku{Wat komt my over, ick //,:}{bin ien bedurven man.635 Hout}{den dief hout den dief}\\

\haiku{al gaf jou de Beul,?}{een kerf.685  ritsert Ic een}{Pot waer van daen}\\

\haiku{De Heer heb heur ziel,,}{wy warent soo wel iens We}{villen nimmermeer}\\

\haiku{die aan het eind van.}{het spel van zijn behoudzucht}{zal zijn genezen}\\

\haiku{De voorrede houdt:}{de afloop van het spel met}{moeite verborgen}\\

\haiku{Giericheydt kondigt.}{aan dat hij de vrek nog \'e\'en}{lesje zal leren}\\

\haiku{Essentieel in.}{morele zin is ook de}{afloop van het spel}\\

\haiku{{\textquoteleft}kaleteyt{\textquoteright} is een {\textquoteleft}{\textquoteright},.}{woordspeling metkaal wat aan}{berooid doet denken}\\

\haiku{hoewel hij er nu:}{volstrekt niet toe gedwongen}{zou zijn 21stede}\\

\haiku{nog geen duit (op veel)...?:}{munten stond een kruisvormig}{teken 86Of taecken}\\

\haiku{aan de voorname,:}{kant de rechterkant 296hebbe}{kick gheen ghenuyghen}\\

\haiku{per pond een reaal (),!}{gouden munt ter waarde van}{ca 2{\textonehalf} stuiver kind}\\

\haiku{alsof (het lot van):}{Holland ervan afhing 587op}{waren ghenomen}\\

\haiku{dubloenen (Spaanse,):}{dukaat ter waarde van ca.}{zeven gulden 634of}\\

\haiku{anders had men in}{het huis van Warnar niet zo}{lang op mij gewacht}\\

\section{C.P. Hoogenhout}

\subsection{Uit: Geskiedenis van Josef en Catharina, die dogter van die advokaat}

\haiku{Ik voel al 'n paar.}{dagen of ik niet lang meer}{op aarde zal zijn}\\

\haiku{{\textquoteleft}Ek praat eendag met,}{vrind Pannevis daaroor en}{laat vir mij ontval}\\

\haiku{En toe hulle ver,,.}{Hem sien bid hulle Hem aan}{maar party twyfel}\\

\haiku{Veelseggend is die,:}{Voorrede wat ons hier in}{sy geheel laat volg}\\

\haiku{{\textquoteright} Vader en skoonseun,.}{gesels in die Ne\"ende}{Gesprek Z.A. 5 Des}\\

\haiku{{\textquoteright} Die geskiedenis.}{van die G.R.A. moet as bekend}{veronderstel word}\\

\haiku{{\textquoteright} Soms trek hy liewers ' {\textquoteleft}{\textquoteright}.}{nsluier oor die sielelewe}{van sy karakter}\\

\haiku{Hy beveel jou ook,, {\textquoteleft}}{nie maar s\^e net wat hy dink}{die beste is bv.}\\

\haiku{{\textquoteright} {\textquoteleft}Hoe sal die arme;}{Wagner sig nou verbly in}{syn salige rus}\\

\haiku{{\textquoteright} Die perdeboer word,.}{ook nie vergeet nie want die}{papies is lastig}\\

\haiku{(2) C.P. Hoogenhout.}{in Die Nuwe Brandwag deel}{I nrs. 1 en 2}\\

\haiku{Laaste Stem uit die,.}{Genootskap van Regte}{Afrikaners 1918}\\

\haiku{{\textquoteleft}Das tog al te erg.}{om ons broer van honger en}{dors te laat doot gaan}\\

\haiku{Die ander kinders.}{was nie van Ragel nie maar}{van een ander vrou\"e}\\

\haiku{God die alles weet,,}{die alles siet kan ons ook}{net soo goed sien as}\\

\haiku{{\textquoteleft}Ons het niks meer om,,.}{te eet nie gee voor ons brood}{anders gaat ons dood}\\

\haiku{Hulle weet goed, dat.}{Jakob voor Benjamin nooit}{sal laat saam gaan nie}\\

\haiku{En toe hulle nou.}{regtig na die tronk gaan toe}{skrik hulle nog meer}\\

\haiku{Voor Benjamin wort.}{vyf maal soo veul opgeskep as}{voor die andere}\\

\haiku{Josef was bly om,.}{al syn broers by hem te heh}{vooral Benjamin}\\

\haiku{{\textquoteright} Toe die broers dat hoor,:}{was hulle soo maar lam van}{skrik en hulle seh}\\

\haiku{{\textquoteleft}Og meneer mot tog.}{nie kwaad wort nie dat ek nog}{een woordje wil seh}\\

\haiku{Myn kinders mot tog!}{ook altyd ou mense met}{agting behandel}\\

\haiku{en as myn storie,.}{nie na jou sin is nie stuur}{hom dan maar terug}\\

\haiku{'n Afrikaander, wat,;}{geen Engels verstaat sou daar}{min an gehad h\^e}\\

\haiku{Hulle stemme gaat, '.}{te hoog dit lyk of daarn}{twis of rusie is}\\

\haiku{Appelkoos, perskes,,,,.}{appels pere pruime was}{by hom in o'ervloed}\\

\haiku{want Ma roep dat die.}{karn af is en ik moet die}{botter gaat af haal}\\

\haiku{Die rente  was.}{al 3 maande verval en}{nog nie betaal nie}\\

\haiku{{\textquoteright} {\textquoteleft}Weet jy nog, wat jy}{eendag op skool geseg het}{toen ik jou geplaag}\\

\haiku{By 'n draai in die '}{groot pad terwyl hulle weer}{opn stap getrek}\\

\haiku{Maar David dit was,!}{daarom nie jou skuld as dit}{so gekom het nie}\\

\haiku{n Heerlike troos,}{lesers wat kerkerade is}{of familie het}\\

\haiku{Die laaste handdruk.}{toen hy al op die wage}{sit was gewissel}\\

\haiku{Die voorperde trap, ',.}{reg. Een klapn harde klap}{en die wagens rol}\\

\haiku{- twe elemente, wat.}{met makaar mar volstrek nie}{kan verenig nie}\\

\haiku{dat hy gerus mag,.}{slaap en dat syn reis verder}{voorspoedig mog wees}\\

\haiku{dit was mar so haar -.}{gewoonte tant Mimie is}{dit al gewoon}\\

\haiku{{\textquoteright} Toen sy dit geseg,, '}{het sy kombuis toe en net}{nou kann mens hoor}\\

\haiku{In kort, hy is weg.}{en die onrusstokers het}{hulle sin gekry}\\

\haiku{as dit so kom nie,,.}{maar meer mag ik nie se nie}{en wil ik oek nie}\\

\haiku{David was in die.}{Kaap en siet Catharina}{en wie weet wat meer}\\

\haiku{maar die goeje siel,;}{had nie anders had sy dit}{seker gegewe}\\

\haiku{E\'en weg is daar voor,,;}{myn Vrystaat of Transvaal toe daar}{is ik onbekend}\\

\haiku{Sy weet hoe David.}{moet voel as hy dit lees in}{die onderveld}\\

\haiku{Ver 'n man, wat deur ',!}{n ongeluk bankrot raak}{het ik reg jammer}\\

\haiku{Hulle wa'ens, karre.}{en negosiegoed was goed}{verkoop en verruil}\\

\haiku{dat hulle brandma'er.}{bowe gekom was en dit}{sou nie betaal nie}\\

\haiku{Juis dit doet jou hart,.}{eer an dat jy nou myn woord}{terug wil gewe}\\

\haiku{insuur en o'ersuur,;}{gaat al goed brood knee is nog}{bietjie swaar seg sy}\\

\section{Cor de Hoon}

\subsection{Uit: Bitter lemon}

\haiku{Hij dacht aan de twee.}{bedden en de spiegel met}{de kromme pootjes}\\

\haiku{Maar daar praten we{\textquoteright}.}{nog wel eens over als je in}{de vijfde klas zit}\\

\haiku{, maar voor hij naar de,.}{docentenkamer ging moest}{hij naar het toilet}\\

\haiku{Dat citeren is,{\textquoteright}.}{een beroepsziekte aan het}{worden geloof ik}\\

\haiku{Er was een pijnlijk.}{plezierige steekvlam door}{hem heengeschoten}\\

\haiku{Voorzichtig liet hij.}{zich terugzakken in zijn}{vorige houding}\\

\haiku{Bovendien had hij.}{at genoeg zorgen om zijn}{blindwordende ogen}\\

\haiku{De rector keek hem.}{vragend aan in afwachting}{van een verklaring}\\

\haiku{Vincent kan dan aan.}{het strand liggen terwijl ik}{de musea bezoek}\\

\haiku{Ik zag vanmiddag{\textquoteright}.}{een vlieg over de rand van een}{flatgebouw lopen}\\

\haiku{Terwijl hij daar stond}{begon de klas langzaam voor}{zijn ogen te draaien}\\

\haiku{Maar ik kan niks doen{\textquoteright}.}{met een valpeip die te kort}{afgeschneden is}\\

\haiku{Misschien was het toch.}{niet zo vreemd dat Fred er zich}{niet rijp voor voelde}\\

\haiku{Er zit te veel stof.}{in de lucht en dat is niet}{goed voor stoflongen}\\

\haiku{Er klonk een zweem van.}{trots in haar stem en de man}{glimlachte hijgend}\\

\haiku{Uit de struiken kwam,,.}{de egel te voorschijn bleef even}{staan zijn snuit omhoog}\\

\haiku{Het had geen zin in.}{bed te liggen en naar het}{plafond te staren}\\

\haiku{{\textquoteleft}Mijn vader zegt dat{\textquoteright}.}{een dokter nog aanzien heeft}{in de maatschappij}\\

\haiku{Vincent vouwde de,:}{kaart op schonk een glas whisky}{in en mompelde}\\

\haiku{De rector zegt dat{\textquoteright}.}{ik te veel de neiging heb}{om te citeren}\\

\haiku{Vincent voelde zich.}{verlegen en begon nog}{harder te zweten}\\

\haiku{Ze bogen samen,.}{het hoofd lachten geruisloos}{en dronken whisky}\\

\haiku{Zeg maar dat ik bij.}{de dokter geweest ben en}{lage bloeddruk heb}\\

\haiku{Hij grijnsde tegen,.}{zijn spiegelbeeld het grijnsde}{minzaam terug}\\

\haiku{Hij liep tussen de.}{rekken door op zoek naar de}{bleke jongeman}\\

\haiku{In de tropen aten.}{de meeste mensen flink en}{dronken overvloedig}\\

\haiku{De ergste hitte.}{zou nu langzaam wegebben}{uit het schoolgebouw}\\

\haiku{{\textquoteleft}O, ik dacht dat het{\textquoteright}.}{weer een voorbeeld was van jouw}{gevoel voor humor}\\

\haiku{Hij kleedde zich uit.}{in het donker en ging naakt}{op bed liggen}\\

\haiku{De laaste druppel.}{die er aan hing spatte uit}{elkaar in zijn oog}\\

\haiku{De bak liep vol en.}{zijn theorie werd door de}{praktijk bevestigd}\\

\haiku{Hij wist ook zeker.}{dat hij dit werk geen dag zou}{kunnen volhouden}\\

\haiku{Nou ja, we hebben.}{elkaar toch ook weer niet zo}{veel te vertellen}\\

\haiku{Het eerste ogenblik.}{deinsde Vincent terug toen}{hij de hal betrad}\\

\haiku{Was beschteld door zo'n.}{kakmadam die ieder jaar}{een nieuw toilet wil}\\

\haiku{De kans zat er dik.}{in dat sommige mensen}{overgeplaatst werden}\\

\haiku{Toen Eric aandrong en;}{haar wilde kussen duwde}{ze hem weg en zei}\\

\haiku{Met deze woorden.}{liet ze Eric staan en voegde}{zich weer bij de groep}\\

\haiku{Vincent wachtte tot.}{hij uitgesproken was en}{ging naar de keuken}\\

\haiku{Hij zou hem vragen.}{een borrel met hem te gaan}{drinken in de stad}\\

\haiku{Maar Vincent voelde.}{diep in zijn binnenste dat}{hij boete moest doen}\\

\haiku{Ik heb een hekel{\textquoteright}.}{aan die krantenberichten}{over Muis in Melkfles}\\

\haiku{Hij moest eigenlijk,,.}{niet meer rijden maar ja wat}{moeten we met hem}\\

\haiku{Hij zette zijn bril.}{af en trok zijn onderste}{oogleden omlaag}\\

\haiku{Ben je van plan om?}{ook over te schakelen op}{natuurlijk voedsel}\\

\haiku{of ze ook wel weer.}{te voorschijn kwamen als ze}{gepasseerd waren}\\

\haiku{Terwijl hij hier nog.}{mee bezig was ging bij de}{buren de deur open}\\

\haiku{Ik had een virus{\textquoteright}.}{infectie en mijn dikke}{darm was onrustig}\\

\haiku{Glas kraakte onder,.}{zijn voeten het wegdek was}{nat en glibberig}\\

\haiku{De routine van.}{dertig jaar kon niet zonder}{uitwerking blijven}\\

\haiku{De jongen in de.}{derde bank van de tweede}{rij trok zijn aandacht}\\

\haiku{Jou herken ik ook{\textquoteright},.}{zei hij tegen een jongen}{achter in de klas}\\

\haiku{not be ever again The,,{\textquoteright}.}{marked of many loved}{of one Gentlemen}\\

\haiku{{\textquoteleft}Waarom trekt U uw,,?}{jasje niet uit als U het}{zo warm hebt meneer}\\

\haiku{Het zou beter zijn.}{als er iemand bij was die}{hij kent en vertrouwt}\\

\haiku{En laatst belde de}{postbode bij ons omdat}{hij geen gehoor kreeg}\\

\haiku{{\textquoteleft}Ik zal jullie nog.}{\'e\'en sc\`ene laten horen}{en dan stoppen we}\\

\haiku{Uit het verwarde.}{geschreeuw ving hij nog slechts hier}{en daar een kreet op}\\

\haiku{Ik heb me \'e\'en keer,.}{naakt vertoond maar voortaan hou}{ik mijn kleren aan}\\

\haiku{Vincent wilde niet.}{meer naar hen luisteren en}{verliet zijn kamer}\\

\haiku{{\textquoteleft}But you must know, your,,{\textquoteright}.}{father lost a father that}{father lost lost his}\\

\haiku{Pas toen hij vlak voor,:}{Vincent stond keek hij op stak}{zijn hand uit en zei}\\

\haiku{Jouw vreemde gedrag.}{was natuurlijk een signaal}{van onbehagen}\\

\haiku{{\textquoteleft}Ik heb mijn hele,{\textquoteright}.}{opvatting over het leven}{gewijzigd Vincent}\\

\haiku{Dat zou Kyle of.}{Lochalsh kunnen zijn en}{dat Killiecrankie}\\

\subsection{Uit: Zij op het nachtkastje}

\haiku{Het ging tenslotte,.}{niet om de pati\"ent maar}{om rust en orde}\\

\haiku{Hij was opgestaan,.}{had zich aangekleed en was}{naar buiten gegaan}\\

\haiku{Hij keek nog eenmaal.}{naar de gang maar zag niets dat}{hem verontrustte}\\

\haiku{Het was in zijn ogen,.}{een Grote Zonde zo met}{de tijd te knoeien}\\

\haiku{De man vroeg zich af,.}{hoe lang het nog zou duren}{voor de dag aanbrak}\\

\haiku{Ze vroeg zich af wat.}{de man deed in die lange}{ziekenhuisnachten}\\

\haiku{Ze begon door de.}{nog bijna verlaten straat}{naar huis te lopen}\\

\haiku{Hij bestuurde zijn.}{parochie met vaste hand}{en genoot ervan}\\

\haiku{De man had in de;}{kudde een onbestemde}{plaats ingenomen}\\

\haiku{Hij had achteraf, {\textquoteleft}}{besloten dat de titel}{van haar boek moest zijn}\\

\haiku{Hij gluurde even, toen.}{hij het achteruit schuiven}{van haar stoel hoorde}\\

\haiku{Zo stond ze voor hem.}{in haar jurk en zo had ze}{haar nachtgewaad aan}\\

\haiku{een deur, die zo maar,,.}{ineens in het slot sprong een}{plank die zich voegde}\\

\haiku{Het was triest in een,.}{straat gewoond te hebben waar}{nooit iets was gebeurd}\\

\haiku{Zonder een kik te.}{geven zakte het als een}{pudding in elkaar}\\

\haiku{Deze liep de weg,,.}{op maar keek dan om om te}{zien waar zijn hond bleef}\\

\haiku{Het ritme van de.}{tijd werd hier bepaald door de}{thema's dag en nacht}\\

\haiku{Het was een hele.}{vooruitgang weer een hoofd op}{het kussen te zien}\\

\haiku{Alle gesprekken.}{verstomden en iedereen}{keek vol spanning toe}\\

\haiku{Niemand wist hoe ze,.}{heette wat ze dacht of wat}{ze met het bloed deed}\\

\haiku{Onder het scheren.}{had hij besloten te voet}{naar kantoor te gaan}\\

\haiku{het dennebos ging.}{over in loofhout en hij liet}{het bos achter zich}\\

\haiku{Die zou wel lauw zijn,}{en het was bovendien geen}{drank die geschikt was}\\

\haiku{Bovendien kon hij.}{de vrouw naast hem niet zo maar}{in de steek laten}\\

\haiku{Even bleef ze staan om.}{op adem te komen en zich}{te ori\"enteren}\\

\haiku{Hij had gezien, dat.}{ze een blouse droeg met een}{hele rij knoopjes}\\

\haiku{Hij wilde, dat ze.}{ophield met dat vervloekte}{heen en weer wiegen}\\

\haiku{Toch ging hij 's avonds.}{vaker dan gewoonlijk}{alleen wandelen}\\

\haiku{Waarom dit vreemde.}{mannetje liep te dwalen}{was hem ontschoten}\\

\haiku{Zijn vrouw zat nog steeds.}{in dezelfde houding over}{hem heen te staren}\\

\haiku{Het was kort na die,.}{nacht geweest waarin hij niet}{thuis gekomen was}\\

\haiku{Er kwam geen trein aan.}{en er was geen trein waar zij}{mee vertrekken kon}\\

\haiku{Het was duidelijk,.}{dat ik de hoofdpersoon van}{de bijeenkomst was}\\

\haiku{Wie weet hoe vaak dit.}{hoofd al boven de modder}{uitgekomen was}\\

\haiku{Fluweelachtig blauw,.}{en rood tere roze en}{zeegroene kleuren}\\

\haiku{In een oogwenk stond.}{er een hele verdieping}{op zijn bungalow}\\

\haiku{Ik wist niet of ik.}{kon helpen en waar ik het}{eerst naartoe moest gaan}\\

\haiku{Ze keken niet op.}{of om en lazen uit hun}{heilige boeken}\\

\haiku{De tafel, waaraan,.}{hij zat te werken zag er}{keurig netjes uit}\\

\haiku{Op het scherm waren.}{nu alle inwendige}{organen zichtbaar}\\

\haiku{Zo verdween hij  .}{in het inwendige van}{het medisch centrum}\\

\haiku{Het water was lauw.}{en ik bereikte zonder}{moeite het midden}\\

\haiku{Dit werd bevestigd.}{toen er een stroom van kritiek}{losbrak in de zaal}\\

\haiku{Als ik mezelf in,.}{veiligheid wilde brengen}{moest ik nu vluchten}\\

\haiku{Ik ben me er niet.}{van bewust dat er iets aan}{mijn paspoort mankeert}\\

\haiku{Hij is natuurlijk.}{een gevolg van alles wat}{ik meegemaakt heb}\\

\section{Roel Houwink}

\subsection{Uit: Novellen (1920-'22)}

\haiku{Tevergeefs zoeken.}{hun vingers steun tusschen het}{welkend tafelgroen}\\

\haiku{Eerst toen zij weken,:}{lag trok om haar zwerend lijf}{den zachten sluier}\\

\haiku{Maar toen de duinen,.}{waren weggevloeid in mist}{bukte  hij zich}\\

\haiku{Dagen ontweek hij,.}{de menschen hongerend om}{de norsche hoeven}\\

\haiku{Hoe zij hem vingen....}{strompelend door den laatsten}{akker voor de grens}\\

\haiku{Hij wordt gewaarschuwd,.}{en slentert ter zijde elk}{bewegen gestremd}\\

\section{Th\'er\`ese Hoven}

\subsection{Uit: Naar Holland en terug}

\haiku{Het Noodlot was hem,,.}{op wonderlijke wijze}{te hulp gekomen}\\

\haiku{de keuken vlak naast -.}{de eetkamer hoort toch in}{de bijgebouwen}\\

\haiku{Niet meer naar Indi\"e,.}{kunnen gaan dat was altijd}{in Holland blijven}\\

\haiku{Zoo, nu en dan deed, '.}{Jan maar als of hij dacht dat}{t een grapje was}\\

\haiku{jij hoeft je er niet,.}{op te verheugen want jij}{gaat zeker niet mee}\\

\haiku{{\textquoteright} Jan stond met een geeuw,,:}{op rekte zich een paar maal}{uit bromde iets van}\\

\haiku{Wat een baas toch, die, '.}{Njo van haar zoo deftige}{meneer int zwart}\\

\haiku{{\textquoteleft}Foei Gerarda, wat,.}{beoefen je slecht wat je}{zooeven gehoord hebt}\\

\haiku{{\textquoteright} vroeg de Kapitein,.}{die zich met zijn nichtjes had}{beziggehouden}\\

\haiku{iedereen weet toch,.}{zij een inlandsche vrouw maar}{hier heel wat anders}\\

\haiku{Ik had het al zoo,.....}{lang willen doen maar ik kon}{er niet toe komen}\\

\haiku{Een neef van haar, die,.}{naar Indi\"e ging had haar ten}{huwelijk gevraagd}\\

\haiku{De tijd, die nu kwam,.}{was een gezegende voor}{de arme L\'eonie}\\

\haiku{{\textquoteleft}Weet u.... begon hij...., {\textquoteleft}'....}{aarzelendt Is toch een}{lamme gewoonte}\\

\haiku{Mijn arme moeder,,....}{kassian vindt me een echt}{Hollandschen jongen}\\

\haiku{{\textquoteleft}Vindt u dat Pa het....?}{leven van Ma bedorven}{heeft of omgekeerd}\\

\haiku{In Indi\"e had ze,.}{nooit gewandeld in Holland}{zoo min mogelijk}\\

\haiku{in den Haag was 't,.}{Jan geweest die haar nog wel}{eens mee had getroond}\\

\haiku{je heele leven.}{te moeten boeten voor een}{dwaasheid van je jeugd}\\

\haiku{{\textquoteleft}Ik ga met Pa een,{\textquoteright}.}{voetreis in den Hartz maken}{kondigt de een aan}\\

\haiku{En Jan Weitinga,.}{gaat treurig naar huis omdat}{hem niets is beloofd}\\

\haiku{De berichten uit.}{Berchtesgaden werden hoe}{langer hoe slechter}\\

\haiku{{\textquoteright} {\textquoteleft}Maar zijn arme vrouw,,.}{die geen woord Duitsch kent en zijn}{jongen die hier is}\\

\haiku{Nee, als ik u een,.}{raad mag geven zou ik hem}{maar stil hier laten}\\

\haiku{Nou, 't beste zou,.}{dan maar zijn dat hij met zijn}{Ma naar Indi\"e ging}\\

\haiku{Een paarsroode blos....}{bedekte het gelaat der}{zedige Tony}\\

\haiku{zoo'n jongen van 18....,....}{jaar zonder vader met een}{inlandsche moeder}\\

\haiku{Naar frankfort is al?}{bijna een heele dag en}{dan nog naar M\"unchen}\\

\haiku{{\textquoteright} Nu is het de beurt.}{van den Notaris om er}{niets van te vatten}\\

\haiku{Ik wou eigenlijk,.}{eens met u overleggen wat}{het beste zou zijn}\\

\haiku{Ik ken ze wel niet,.}{allemaal maar beter dan}{jij kan er geen zijn}\\

\haiku{Ook nu wil ze, zoo, '.}{spoedig mogelijkt weer}{goed maken bij Jan}\\

\haiku{Maar L\'eonie kan er,,.}{bij verder nadenken niets}{aardigs in vinden}\\

\haiku{Thuis, als kind, Gondang,.}{Legi werd er niet veel acht}{op haar geslagen}\\

\haiku{Maar ze voelde zich.}{nog ongelukkiger en}{nog meer verlaten}\\

\haiku{{\textquoteright} Als zieke eenmaal, '....}{vindt niet lekker ist al}{bedorven voor hem}\\

\haiku{Hun verhouding van.}{broeder en zuster had toch}{niet kunnen duren}\\

\haiku{Zij kleedt zich nu weer,,?}{heelemaal op zijn Indisch}{toch zoo lekker ja}\\

\haiku{{\textquoteleft}Hoor eens, kameraad, '.}{lan we mekaar nu eens}{klaren wijn schenken}\\

\haiku{Als dokter moogt u,....}{niet oververtellen wat uw}{pati\"enten scheelt}\\

\haiku{ik ga naar huis, naar.}{L\'eonie dan en ik zet oogen}{en ooren goed open}\\

\haiku{Nu ja, als 't er,....}{op aan kwam had hij wel zijn}{zinnen bij elkaar}\\

\haiku{We blijven bij je,,,?}{tot neef Adriaan komt Majin}{als je het goed vindt}\\

\haiku{De laatste jaren;}{in Holland waren ook al}{heel ongelukkig}\\

\haiku{{\textquoteright} L\'eonie poogde de.}{schim van een lachje op haar}{gelaat te tooveren}\\

\haiku{Geen enkele man,,.}{zet zich zonder wrok of spijt}{over een blauwtje heen}\\

\haiku{{\textquoteright} {\textquoteleft}Voor mij was ze de,,.}{trouwste vriendin die ik ooit}{gehad heb Clara}\\

\haiku{Ze verzocht Majin:}{op een ochtend bij haar te}{komen en zei toen}\\

\haiku{Dan hadden wij ons.}{allebei verveeld en u}{hier in uw eentje}\\

\haiku{Ze lacht wel zoo en -.}{zegt lieve woordjes maar ze}{kan toch niet gelooven}\\

\haiku{Jammer, dat zij niet.}{wat vertrouwelijker met}{zijn moeder omging}\\

\haiku{Hij lachte - omdat '....:}{zet zoo raar zei zuchtte}{toen en fluisterde}\\

\haiku{Mijn vader heb ik;}{nooit gekend en Mama sprak}{bijna nooit over hem}\\

\haiku{ik was nog een kind,, '.}{ik wist niet wat liefde was}{ik weett nog niet}\\

\haiku{Voor beiden was 't,.}{een foltering ter wille}{van Jan doorgestaan}\\

\haiku{Ik dacht maar zoo, dat....}{u toch wel weer verlangen}{zoudt naar uw huisje}\\

\haiku{Menschen werden wel.}{eens gek van veel en vooral}{van moeielijk denken}\\

\haiku{{\textquoteleft}Maar nu, ik begrijp,?}{nog niet waarom jij haar in}{mijn huis wil brengen}\\

\haiku{De jonge moeder,;}{probeerde van alles en}{alles mislukte}\\

\haiku{{\textquoteright} Jan had er in al;}{dien tijd niet met zijn moeder}{over durven spreken}\\

\haiku{Jan, die den dokter,...}{had uitgelaten was weer}{binnengekomen}\\

\subsection{Uit: Van Koningsplein naar Gang Ket\`apan}

\haiku{Ze kwamen in den, '.}{schouwburg waart tamelijk}{vol en heel warm was}\\

\haiku{{\textquoteleft}Ja, werkelijk, en, '.}{vooral nu dat ikt niet}{meer behoef te doen}\\

\haiku{{\textquoteright} {\textquoteleft}Onverschillig, nee, '....}{maart Indische klimaat}{werkt zoo verslappend}\\

\haiku{{\textquoteleft}Waarom zijn jelui,,?}{in al dien tijd nooit eens naar}{Europa gegaan}\\

\haiku{Het meenemen van.}{een der kinderen was voor}{de convenances}\\

\haiku{Het was haar, als was,}{zij de bezitster van een}{kostbaren schat dien}\\

\haiku{ik zou willen' dat,....}{ik jou ontmoet had in plaats}{van Catherine}\\

\haiku{Waarop hem gevraagd,.}{werd wie die stommiteit op}{zijn rekening kreeg}\\

\haiku{Trouwens, die zuster,.}{is nog zoo lang geleden}{niet uitgekomen}\\

\haiku{Ik vind niet, dat we....}{nu juist in een stemming zijn}{om feest te houden}\\

\haiku{Pa was mooi boos, dat.}{je op je verjaardag niet}{eens thuis wou komen}\\

\haiku{{\textquoteright} {\textquoteleft}Ga je gang, als je,{\textquoteright},.}{er pleizier in hebt zei Jack}{op smalenden toon}\\

\haiku{thuis was of als de,}{kinderen er waren dan}{lachte ze wel weer}\\

\haiku{ze trapte er even,:}{met haar blooten voet op toen}{zei ze heel gewoon}\\

\haiku{op de scholen was;}{er voor de kleintjes gestrooid}{door een heuschen Sint}\\

\haiku{{\textquoteright} {\textquoteleft}Dat niet, maar Liekie....}{en ik hebben nog wel wat}{in onzen spaarpot}\\

\haiku{{\textquoteright} Gonne luisterde,;}{niet verder en ging er ook}{niet verder op door}\\

\haiku{Wat hoefde hij zich '....}{altijd muizenissen in}{t hoofd te halen}\\

\haiku{Gonne zei, dat ik, '.}{vroeg thuis moest zijn omdatt}{zoo vol is op straat}\\

\haiku{'t Was zoo leeg in, '....}{huis nat vertrek van haar}{man en van Liekie}\\

\haiku{Hij zou Liekie thuis,.}{brengen dat beteekende}{bij Catherine}\\

\haiku{en wilde hij 't,.}{pak niet weer meenemen moest}{hij haar wel volgen}\\

\haiku{{\textquoteright} In \'e\'en seconde.}{trachtte de zakenman den}{toestand te overzien}\\

\haiku{naar die andere........}{dat ze hem misschien in lang}{niet terug zou zien}\\

\haiku{Hij mocht haar niet meer,.}{steunen en och arme ze}{had dien steun zoo noodig}\\

\haiku{'t Was zijn zoon, hij,....}{moest hem vormen tot een braaf}{tot een bruikbaar mensch}\\

\haiku{hij was anders wel,.}{wat ongenaakbaar die heer}{en meester van haar}\\

\haiku{ze wilden een roes, '.}{en Catherine moestt}{gelag betalen}\\

\haiku{Een schok ging haar door ',!}{t lichaam een schaterlach}{ontsnapte haar keel}\\

\haiku{Ze streek zich met de ',....}{hand overt voorhoofd ze kon}{er niet aan gelooven}\\

\haiku{jammer, stomme zet,,.}{begreep nog niet hoe hij er}{toe gekomen was}\\

\haiku{Hij zag van uit de,.}{verte dat er overal licht}{aan was in zijn huis}\\

\haiku{{\textquoteleft}Pien moet der na toe,, '.}{als iemand nog der wat an}{kan doen ist Pien}\\

\haiku{Als toewan53 dokter,.}{obat geeft en niet gelpt dan laat}{Pien nenn\`eh komen}\\

\haiku{Ik sukkelde veel,,.}{ik zag er slecht uit ik was}{leelijk geworden}\\

\haiku{Die overtuiging, dat....}{ze hem haten moest en het}{bewustzijn dat ze}\\

\haiku{Zij, de willooze, de........}{apathische die hem van zich}{had laten weggaan}\\

\haiku{als je nu tot mij,.}{zoudt terugkeeren zou je even}{schuldig zijn als toen}\\

\haiku{En van de hoogste.}{illusie verzonk ze in}{de diepste wanhoop}\\

\haiku{Na dien nacht, wreed als,.}{een terechtstelling stortte}{Catherine in}\\

\haiku{{\textquoteright} {\textquoteleft}Ik begrijp, dat jij,!}{er naar snakt om van hier weg}{te komen maar o}\\

\haiku{Integendeel, ik,.}{zou je zoo dankbaar zijn als}{je mij met rust liet}\\

\haiku{{\textquoteleft}Ik kom obat brengen,,,.}{zoo goede obat wekt mevrouw}{gier heelemaal op}\\

\subsection{Uit: Zoo men zaait}

\haiku{na eenigen tijd zond,,}{hij haar naar Buitenzorg bij}{een dokter in huis}\\

\haiku{Nee, daar behoorde,,.}{meer moed toe dan hij op zijn}{leeftijd nog bezat}\\

\haiku{en toen volgde een '.}{opsomming vant geen ze}{zich herinnerde}\\

\haiku{Mijn arm kind, kon ik,,.}{je slechts een tehuis als mijn}{vrouwtje aanbieden}\\

\haiku{hij behandelt me,.}{zoo als een jongetje een}{burgerjongetje}\\

\haiku{Ze wist niet wat te,.}{zeggen ze kon zich toch niet}{aan hem opdringen}\\

\haiku{Meneer komt terug,,!}{met twee dochters de eene zoo}{mooie meisje prachtig}\\

\haiku{Maar.... om nu nog eens.}{op de bedden-quaestie}{terug te komen}\\

\haiku{We hebben, v\'o\'or ons,.}{vertrek een maand met hem in}{Parijs doorgebracht}\\

\haiku{{\textquoteleft}Nee,{\textquoteright} zeg mijn man, {\textquoteleft}veel,.}{te omslachtig en te duur}{zoo lang een wagen}\\

\haiku{ze had ook zoo vroeg.}{al de ellende van haar}{moeder meegemaakt}\\

\haiku{Ze zou er nu niet,.}{aan toegeven doch rechtstreeks}{op haar doel afgaan}\\

\haiku{Waarom zendt u die....,,?}{weduwe Janssens met haar}{troepje dan niet weg}\\

\haiku{Het was de tweede,.}{dien ze sedert haar aankomst}{in Indi\"e ontving}\\

\haiku{Ik ben, in vele,.}{opzichten niet zoo sterk en}{niet zoo flink als jij}\\

\haiku{vindt hij 't beter,:}{dat hij 2e klasse reist dan}{seint hij eenvoudig}\\

\haiku{als altijd, enkel.}{en alleen vervuld van haar}{eigen persoontje}\\

\haiku{Als 't er een van, '.}{Jack was geweest zou ikt}{ook gedaan hebben}\\

\haiku{Ik hoef er toch niet, ',}{bij te zetten dat ikm}{weggenomen heb}\\

\haiku{Het scheen, dat Nonna,,:}{Pien haar gedachte raadde}{ten minste ze zei}\\

\haiku{der zat haast niks in}{me kisten en ik kon me}{niks anschaffen ook}\\

\haiku{{\textquoteright} Liekie bracht hem tot.}{aan de voorgalerij en}{nam lachend afscheid}\\

\haiku{{\textquoteleft}En dan... denk er aan,.}{dat je niet te luidruchtig}{of te vroolijk bent}\\

\haiku{ze vond dat het te.}{fleurig en te frisch voor de}{gelegenheid was}\\

\haiku{die had haar, zoo ze,.}{op was natuurlijk naar de}{badkamer zien gaan}\\

\haiku{Ze zou er zich dus '.}{maar zelve van overtuigen}{en gingt huis in}\\

\haiku{En nu was ze z\'o\'o,.}{stil geweest dat Gonne er}{niets van gemerkt had}\\

\haiku{{\textquoteleft}Je weet toch, hoeveel '.}{eenvoudiger ikt in}{Holland gewend was}\\

\haiku{Je moet niet denken, '.}{dat ik totaal onbekend}{ben mett Maleisch}\\

\haiku{Ze moest voor hem de,.}{eerste zijn gelijk hij voor}{haar de eerste was}\\

\haiku{Henk.... heusch, tracht mij,.}{en mogelijk ook je zelf}{niet te misleiden}\\

\haiku{{\textquoteright} Hendrik de Berg was '.}{reeds naar buiten gesneld en}{naart rijtuig toe}\\

\haiku{De ongewone,....}{omgeving de slapende}{Nonna voor haar bed}\\

\haiku{Ik heb niets meer van ',{\textquoteright}.}{t leven te wachten zei}{Gonne gelaten}\\

\haiku{werd ze opgeschrikt.}{door haastige stappen in}{de voorgalerij}\\

\haiku{{\textquoteleft}Ik luisterde niet,....}{verder doch voelde me wel}{honderd pond lichter}\\

\haiku{Maar Jack dacht er niet,,,.}{aan ook maar \'e\'en haarbreed voor}{wie ook te wijken}\\

\haiku{Zij, die zelve hoog,.}{stond en achtenswaardig was}{had hem geminacht}\\

\section{Marijke H\"oweler}

\subsection{Uit: Bij ons schijnt de zon}

\haiku{Daar heb jij bij mijn.}{weten nooit zo over in de}{zorgen gezeten}\\

\haiku{Maar de bank was leeg.}{en Arnold liep ongerust}{naar hun slaapkamer}\\

\haiku{zo zit het,{\textquoteright} zei Roos,.}{toch enigszins tevreden over}{de samenvatting}\\

\haiku{{\textquoteleft}Nee ik bedoel dat,{\textquoteright}.}{van die huisgenoot van je}{zei mevrouw De Zeeuw}\\

\haiku{een huisgenoot van.}{een kennis van mij heeft last}{van aanstellerij}\\

\haiku{{\textquoteright} {\textquoteleft}H\`e, zou jij even het,.}{raam open willen zetten je}{ziet er zo warm uit}\\

\haiku{{\textquoteright} {\textquoteleft}Nee, dat bedoel ik,{\textquoteright}.}{niet wuifde mevrouw De Zeeuw}{haar ongemak weg}\\

\haiku{{\textquoteleft}Mijn ogen kleven en.}{mijn bril en mijn zakdoek en}{mijn broekzak kleven}\\

\haiku{{\textquoteleft}'n Jaar of zes,{\textquoteright} zei,.}{Roos wetend dat dit een wat}{lage schatting was}\\

\haiku{{\textquoteleft}Ja,{\textquoteright} zei moeder, {\textquoteleft}en,.}{voor je ophangt h\`e breng even}{wat geld voor me mee}\\

\haiku{{\textquoteleft}Het is jammer dat.}{de mensen tegenwoordig}{geen melk meer willen}\\

\haiku{{\textquoteright} vroeg Leo zo neutraal,.}{mogelijk ofschoon Roos het}{toch te gretig vond}\\

\haiku{Hij bekeek Rosa.}{en scheen daar inspiratie}{aan te ontlenen}\\

\haiku{{\textquoteright} {\textquoteleft}Maar waarom ga je.}{dan zo vaak naar haar toe als}{je niet van haar houdt}\\

\haiku{{\textquoteright} vroeg Arnold en liet.}{zich verbouwereerd op de}{eetkamerstoel neer}\\

\haiku{{\textquoteright} {\textquoteleft}Weet je wat je doen,,{\textquoteright}.}{moet in zo'n geval Arnold}{fluisterde Mattheus}\\

\haiku{{\textquoteright}        13 Het waarom}{van een innerlijk als men}{een uiterlijk heeft}\\

\haiku{deze was dat het.}{zo prettig verlopen was}{met de inboedel}\\

\haiku{{\textquoteleft}Je hebt er niets aan.}{om het allemaal zo hoog}{te willen spelen}\\

\haiku{{\textquoteright} Autorit\"atsgl\"aubig,.}{als Hugo was maakte het}{wel enige indruk}\\

\haiku{Dat had Hugo nog.}{niet zo gezien en hij keek}{Rosa vragend aan}\\

\haiku{wilde ik het ook,.}{nog met je over hebben ze}{heeft t\'och iets aardigs}\\

\haiku{En Leo begreep maar.}{al te goed wat zijn moeder}{daarmee bedoelde}\\

\haiku{Heb jij dat nou ook?}{soms dat je eenvoudige}{dingen niet verstaat}\\

\haiku{Maar moeder maakte.}{dat Leo niet aan kiezen toe}{hoefde te komen}\\

\haiku{{\textquoteleft}Dat zou dat moeder,{\textquoteright}.}{zich allerlei dingen in}{het hoofd haalt zei Leo}\\

\haiku{{\textquoteleft}Dat ze er niet van,.}{slapen kan dat ze er dag}{en nacht mee rondloopt}\\

\haiku{In de keuken sneed}{Rosa de uien en als}{men haar vragen zou}\\

\haiku{Arnold aarzelde.}{zoals hij de halve nacht}{had liggen dubben}\\

\haiku{Arnold probeerde.}{een snikachtig geluid in}{zijn stem te leggen}\\

\haiku{{\textquoteright} Arnold legde het.}{boterhamtrommeltje op}{bed en ging zitten}\\

\haiku{{\textquoteright} {\textquoteleft}Ga met \'ons mee,{\textquoteright} zei, {\textquoteleft}.}{Arnold spontaanwij willen}{naar het buitenland}\\

\haiku{{\textquoteleft}Wat enig om je nu,...}{eens in een gewoon kostuum}{te zien nou gew\'o\'on}\\

\haiku{{\textquoteright} {\textquoteleft}Koken, oh nee kind,,,,.}{voor geen goud ik kook nooit meer}{n\'o\'oit stel je voor zeg}\\

\haiku{zou ze w\'el, zou ze,?}{niet zou Van der Loo zijn mond}{dicht kunnen houden}\\

\haiku{Nu moest hij ze nog.}{met alle twee zijn voeten}{zien op te tillen}\\

\haiku{{\textquoteleft}Mattheus heeft nog maar,{\textquoteright}.}{pas z'n rijbewijs gehaald}{aarzelde Arnold}\\

\haiku{Sorry dat ik het,.}{zeg maar meer is het niet die}{schoonmoeder van jou}\\

\haiku{Toch keek Van der Loo '.}{zijn prachtige map int}{geheim nog even na}\\

\haiku{Al z\'o lang, dat ik.}{was gaan geloven dat ik}{het verkeerd begreep}\\

\haiku{Omdat ik wist dat.}{je vergeten zou om het}{aan ze te vragen}\\

\haiku{Ik wil niet dat je ',{\textquoteright}, {\textquoteleft}.}{r treitert zei Arnolddat}{wil ik niet hebben}\\

\haiku{In elk geval, Leo.}{was altijd al een beetje}{onstuimig geweest}\\

\haiku{{\textquoteright} {\textquoteleft}Oh lieve mamaatje,{\textquoteright}, {\textquoteleft}}{zongen ze verderzeg het}{niet tegen papaatje}\\

\haiku{Hij nam haar handen.}{in de zijne en streek er}{met zijn voorhoofd langs}\\

\haiku{{\textquoteleft}Wil je wel zorgen,{\textquoteright}.}{dat je met je vlerken van}{me afblijft zei Roos}\\

\haiku{En wat mevrouw De}{Zeeuw vooral zo waardeerde}{in de kinderen}\\

\haiku{Nee, mevrouw De Zeeuw.}{was niet ontevreden over}{haar investering}\\

\haiku{{\textquoteleft}Kun je misschien een?}{briefje geven waarop staat}{wat je geleend hebt}\\

\haiku{Bovendien moest men.}{een kind vroeg leren om iets}{af te kunnen staan}\\

\haiku{Dat maakte de sfeer.}{in die mooie Citroen er niet}{gezelliger op}\\

\haiku{Ze had zich een paar.}{jurken aangeschaft om dat}{te onderstrepen}\\

\haiku{{\textquoteleft}We moesten eigenlijk,{\textquoteright}.}{naar bed vond Mattheus en keek}{eens op naar Arnold}\\

\haiku{Dat die cognac blijft.}{stilstaan in je glas terwijl}{je ermee ronddraait}\\

\haiku{{\textquoteright} {\textquoteleft}Nee,{\textquoteright} gaf Hugo na,.}{enig nadenken toe het was}{een nieuw gezichtspunt}\\

\haiku{Zo, zo,{\textquoteright} zei mevrouw, {\textquoteleft}?}{De Zeeuwlopen jullie niet}{te hard van stapel}\\

\haiku{{\textquoteleft}Kijkt u eens hier,{\textquoteright} en.}{hij haalde een pakketje}{uit zijn binnenzak}\\

\haiku{{\textquoteright} En mevrouw De Zeeuw.}{wilde elk een pakketje}{in de hand stoppen}\\

\haiku{{\textquoteleft}Zo,{\textquoteright} zei Arnold toen.}{mevrouw De Zeeuw haar laatste}{slokje thee op had}\\

\haiku{Loof de Heer, dacht Leo,.}{Gij zijt de gezegende}{onder de vrouwen}\\

\haiku{En Thomas vroeg zich.}{af of hij jaloers was op}{Leo en dat was zo}\\

\subsection{Uit: Tobias}

\haiku{De kleinste ja, als,,.}{die zo oud was geweest maar}{de Tobias nee}\\

\haiku{Waar het alleen om,,.}{gaat dat is dat hij er niet}{gezien zal worden}\\

\haiku{Rachel hoort haar niet,,.}{die lijkt niet hier te zijn zo}{staart die voor zich heen}\\

\haiku{Hij haalt de schouders,:}{op kijkt haar dan aan en wil}{weer verder lopen}\\

\haiku{Dan sluipt hij naar de,,.}{deur links van de tap en doolt}{wat door de gangen}\\

\haiku{{\textquoteright} En Tobias voelt.}{het paard onder zich rillen}{en geeft de sporen}\\

\haiku{dat hoef je niet te,{\textquoteright},.}{zeggen en hij neemt de lap}{legt die op de grond}\\

\haiku{{\textquoteright} {\textquoteleft}Dat kan wel zijn, maar,,}{nooit zo achterlijk als de}{Tobias dat blijkt}\\

\haiku{Ik ben al ruim vier.}{jaar gered en elk jaar is}{een jaar geredder}\\

\haiku{Ze keken wel mooi.}{uit om ze een strobreed in}{de weg te leggen}\\

\haiku{Hoeveel  moet ik?}{jou bieden als hij vannacht}{bij jou zou slapen}\\

\haiku{Ik heb er wel vijf ',{\textquoteright}, {\textquoteleft}.}{gezien diet konden houdt}{Joshua volof zes}\\

\haiku{Als Belle er maar,.}{een beetje tevreden mee}{was met die uitleg}\\

\haiku{Een theepot en een,,.}{paar boeken met een touwtje}{erom komen mee}\\

\haiku{{\textquoteright} {\textquoteleft}Trouwen zeker,{\textquoteright} snikt,.}{Judith maar ze laat nu haar}{arm tenminste los}\\

\haiku{En nu hij haar kust,.}{blijken zijn tranen zouter}{dan de hare}\\

\haiku{Dat kwam, omdat hij.}{vergeten was dat hij niet}{met paard en kar was}\\

\haiku{{\textquoteleft}Waar was je dan zo,{\textquoteright}, {\textquoteleft}?}{bezorgd over vraagt zewaar til}{je dan zo zwaar aan}\\

\haiku{{\textquoteright} {\textquoteleft}Dat weet ik niet,{\textquoteright} zegt.}{Tobias en ze lacht en}{strijkt hem door zijn haar}\\

\haiku{{\textquoteleft}Maak open,{\textquoteright} zegt ze, {\textquoteleft}en.}{kijk of je niets hebt wat als}{een doek kan dienen}\\

\haiku{{\textquoteright} {\textquoteleft}Ach God,{\textquoteright} zegt ze, {\textquoteleft}staat.}{de Tobias jou daar voor}{niets op te wachten}\\

\haiku{{\textquoteright} Ze knikken, vegen.}{het laatste brood nog van hun}{mond en staan dan op}\\

\haiku{{\textquoteright} {\textquoteleft}De Joshua is al,.}{in het dorp de Joshua die}{kan toch niet zwijgen}\\

\haiku{Het klinkt als de wind.}{waarin men alles horen}{kan wat of men vreest}\\

\haiku{Nooit was ze  nog.}{zo ongerust geweest over}{geluid dat uitbleef}\\

\haiku{Ooit,{\textquoteright} zegt ze, {\textquoteleft}heb ik,.}{veel gepraat voornamelijk}{tegen de dingen}\\

\haiku{{\textquoteright} Maar Tobias schijnt.}{niet te willen begrijpen}{waar hun redding ligt}\\

\haiku{Hij is bewaker,.}{zogenaamd dat niemand het}{mijnenveld betreedt}\\

\haiku{{\textquoteleft}Nee,{\textquoteright} zegt hij dan, {\textquoteleft}jou,{\textquoteright}.}{laat ik niet weer gaan en kijkt}{weer naar zijn handen}\\

\haiku{Als Judith haar het}{water brengt en het nadien}{weer op wil halen}\\

\haiku{{\textquoteright} {\textquoteleft}Een heks, een heks,{\textquoteright} zegt}{Joshua en aan de dekens}{valt te zien hoe kort}\\

\subsection{Uit: Tranen van niemand}

\haiku{Kijk dan naar die kast,,?}{tegenover je die wil je}{toch hebben nietwaar}\\

\haiku{Winnie sloeg de ogen.}{neer en begon zedig haar}{taartje op te eten}\\

\haiku{En langzamerhand,}{werden de bomen kaal want}{hoe groter de plicht}\\

\haiku{Hij liep de deur uit,,}{deed zijn jas aan en voordat}{ze precies begreep}\\

\haiku{Het lag ook wel aan,;}{mij ik kon nooit eens meedoen}{als zij gek deden}\\

\haiku{Zij hoorde hem zijn.}{step uit de schuur halen en}{knarsend wegrijden}\\

\haiku{III Zijn kind had nooit,.}{geleefd voor hem als nu nu}{het gestorven was}\\

\haiku{Waarom wilden wij,?}{ons voortzetten zijn wij z\'o}{op ons zelf gesteld}\\

\haiku{Marie-Jeanne,,,.}{gaat het nog het is nu tijd}{ik kom zo terug}\\

\haiku{Kom, zei het meisje,,}{laten we ons plezier niet}{laten bederven}\\

\haiku{Nee, zei hij, laten.}{we eerst gaan kijken hoe de}{mensen het hier doen}\\

\haiku{Ze keken in de,.}{emmer hoogmoedig volgde}{de man hun blikken}\\

\haiku{De obers zagen het.}{en namen niet de moeite}{ze weg te jagen}\\

\haiku{Het was vier uur, hij.}{rekte zich en bleef voor zich}{uit liggen kijken}\\

\haiku{He, doe nou niet zo,,,!}{triest we zijn hier toch voor ons}{plezier zei ze kom}\\

\haiku{Laat kwamen ze aan,,,}{iedereen sliep ze maakten}{de leider wakker}\\

\haiku{de voorzitter ons}{op en vroeg of wij  het}{gevonden hadden}\\

\haiku{, ik werd voorgesteld,.}{aan zijn vrouw gewogen en}{te licht bevonden}\\

\haiku{Misschien niet vandaag,,.}{maar morgen zeker hoorde}{ik mezelf zeggen}\\

\haiku{Ik weet niet, begon,,;}{ze ik ben bang dat hij niet}{goed ziet en dan snel}\\

\haiku{We weten het niet,.}{het is zo hopeloos ver}{van huis allemaal}\\

\haiku{De recensie van,,.}{een toneelstuk enkele}{films een cabaret}\\

\haiku{K\'erels, dan kon je.}{echt horen dat ze de pest}{had aan het leven}\\

\haiku{Evengoed vroeg je je,}{af hoe zo'n griet zo stom kon}{wezen dat wordt vast}\\

\haiku{Ach schat, zei ze, trek,?}{je er niks van aan je heb}{het nou toch lekker}\\

\subsection{Uit: Van geluk gesproken}

\haiku{{\textquoteright} Zij bedoelde dan.}{niet alleen de lucht maar zijn}{hele verschijning}\\

\haiku{Soms  ontbrak de}{hele week het toetje aan}{moeders eten nu eens}\\

\haiku{Het lag Wouter op.}{de tong om te vertellen}{wat hij van plan was}\\

\haiku{Martijn heeft een 6.}{voor staatsrecht en ik heb een}{4 voor statistiek}\\

\haiku{Eerst klonk er geklos.}{op de trap en daarna de}{stem van meneer Kalk}\\

\haiku{{\textquoteleft}Dat heb jij toch al,{\textquoteright}.}{gedaan zei Martijn en dat}{was natuurlijk zo}\\

\haiku{{\textquoteright} {\textquoteleft}Als jij m'n petje ',}{van de achterbank haalt dan}{doe ikt dak open}\\

\haiku{{\textquoteleft}Ik wilde niet aan,{\textquoteright}.}{\'e\'en stuk doorrijden zei Leo}{verontschuldigend}\\

\haiku{Hij kocht zelfs nog een {\textquoteleft}{\textquoteright}.}{flesje wijn omthuis bij de}{tv op te drinken}\\

\haiku{Hij ging daarom maar.}{aan de grote tafel in}{het midden zitten}\\

\haiku{Het bericht kwam toch:}{niet helemaal goed door want}{meneer de Bruin riep}\\

\haiku{Jij moet je er niet.}{mee bemoeien als je er}{geen verstand van hebt}\\

\haiku{{\textquoteright} {\textquoteleft}Ik mag geen verstand,}{van stereo hebben maar van}{jou weet ik alles}\\

\haiku{{\textquoteright} {\textquoteleft}Geen nieuws,{\textquoteright} zei De Bruin.}{en probeerde zich uit de}{voeten te maken}\\

\haiku{Thomas beloofde,.}{Thomas bezwoer hen dat hij}{zich beheersen zou}\\

\haiku{{\textquoteright} Thomas keek op en.}{zag hoe Martje de tafel aan}{het afruimen was}\\

\haiku{{\textquoteleft}Laat toch staan kindje,,.}{morgen gaan we verder ga}{toch lekker naar bed}\\

\haiku{Ik geloof dat we,{\textquoteright}.}{beter naar bed kunnen gaan}{zei Thomas praktisch}\\

\haiku{Dat kunt u zich niet,{\textquoteright}.}{voorstellen zei mevrouw de}{Bruin nu iets kalmer}\\

\haiku{{\textquoteleft}Heeft hij u,{\textquoteright} Martijn, {\textquoteleft}?}{kon het woord maar niet vinden}{geslagen of zo}\\

\haiku{{\textquoteright} zei ze toen weer en.}{voor Martijn op kon staan was}{het haar al gelukt}\\

\haiku{Barbara had op:}{zich genomen met Martje te}{praten of beter}\\

\haiku{{\textquoteleft}Ik ga even vragen,{\textquoteright}.}{of ze iets wil zei Thomas}{en liep naar boven}\\

\haiku{Niettemin deed het.}{hem goed in de vertrouwde}{omgeving te zijn}\\

\haiku{Leo scheurde door de.}{drie bochten welke zijn huis}{van zijn werk scheidden}\\

\haiku{Van vrouwen wist je.}{eigenlijk nooit hoe preuts ze}{nou precies waren}\\

\haiku{Hij had daar nooit veel.}{goeds van verwacht want van het}{een kwam het ander}\\

\haiku{Nog nooit had Martje een!}{snackbar gezien met zoveel}{Telegraaflezers}\\

\haiku{Daardoor kwam het ook.}{dat ze de auto's en zelfs}{de bus niet hoorde}\\

\haiku{{\textquoteright} zei Leo geschrokken.}{en voelde driekwart van zijn}{energie wegzakken}\\

\haiku{{\textquoteleft}Welnee kerel,{\textquoteright} zei.}{Van Zutphen en zwaaide de}{rook uit z'n buurt weg}\\

\haiku{Hij legde z'n hand' {\textquoteleft},,}{op Thomas mouw en schonk diens}{glas vol.Hier drink op}\\

\haiku{{\textquoteright} {\textquoteleft}Nee,{\textquoteright} zei Thomas, {\textquoteleft}uit,,.}{haar briefje v\'o\'ordat ze naar}{ons toekwam bleek niets}\\

\haiku{{\textquoteleft}Ik heb ook nog aan,{\textquoteright}.}{de studentenpsycholoog}{gedacht zei Thomas}\\

\haiku{Nee, niet wat me te,,.}{wachten staat wat er gebeurt}{tegelijk gebeurt}\\

\haiku{{\textquoteright} {\textquoteleft}Nee,{\textquoteright} zei Roos, {\textquoteleft}dat hoeft,.}{niet ik hoef het deze keer}{niet meer te horen}\\

\haiku{Martje maakte zich los.}{van haar eigen avontuur en}{knikte naar Geesje}\\

\haiku{Ik kan toch beter,.}{gewoon bij m'n verslaafde}{zootje blijven dacht Gees}\\

\haiku{En Thomas was er.}{al helemaal niet voor in}{het goede humeur}\\

\haiku{{\textquoteleft}Meisje,{\textquoteright} zei Thomas, {\textquoteleft},{\textquoteright}.}{wat fijn dat je meeging en}{verder zwegen ze}\\

\haiku{'t Is trouwens niet,,!}{warm ook schiet eens een beetje}{op ouwe jongen}\\

\haiku{Net op tijd om het.}{eten om zes uur op tafel}{te kunnen zetten}\\

\haiku{{\textquoteright} Leo wist zich geen raad,.}{en ging maar weer eens achter}{z'n bureau zitten}\\

\haiku{{\textquoteright} {\textquoteleft}Nee,{\textquoteright} zei Thomas, {\textquoteleft}weet?}{je waarom het een rare}{vraag is geworden}\\

\haiku{{\textquoteright} Barbara stond nog.}{steeds met de sandwiches en}{de limonade}\\

\haiku{Maar ze had het nog.}{niet opgeborgen of het}{meisje wilde m\'e\'er}\\

\haiku{Misschien een wieg of....}{luiers of Leo's hoofd liep}{om van de zorgen}\\

\haiku{{\textquoteleft}Dag Anke,{\textquoteright} zei Leo, {\textquoteleft}?}{vriendelijkwat heb je voor}{me te doen vandaag}\\

\haiku{En mevrouw Elisa {\textquoteleft}{\textquoteright}.}{wiegde het kindbiem-bahm}{op het klokgelui}\\

\haiku{Nee meneer Evers, niet,...}{doen komt u hier even zitten}{en neemt u meneer}\\

\haiku{Het was een eindje,,.}{om en ja zo was het het}{was een eindje om}\\

\haiku{{\textquoteright} En ze begon aan,.}{Wouter te trekken die heel}{langzaam overeind kwam}\\

\haiku{Dus De Bruin knikte.}{maar een beetje dat het een}{goed idee van haar was}\\

\haiku{{\textquoteleft}En je potloodje,{\textquoteright},.}{zei De Bruin maar dat verzon}{hij in zijn overmoed}\\

\haiku{{\textquoteright} zei de man die het.}{telefoonnummer bijna}{goed had gelezen}\\

\haiku{We zouden misschien,,}{het huis in twee\"en kunnen}{delen jij boven}\\

\subsection{Uit: Van geluk gesproken}

\haiku{{\textquoteright} Zij bedoelde dan.}{niet alleen de lucht maar zijn}{hele verschijning}\\

\haiku{Soms  ontbrak de}{hele week het toetje aan}{moeders eten nu eens}\\

\haiku{Het lag Wouter op.}{de tong om te vertellen}{wat hij van plan was}\\

\haiku{Martijn heeft een 6.}{voor staatsrecht en ik heb een}{4 voor statistiek}\\

\haiku{Eerst klonk er geklos.}{op de trap en daarna de}{stem van meneer Kalk}\\

\haiku{{\textquoteleft}Dat heb jij toch al,{\textquoteright}.}{gedaan zei Martijn en dat}{was natuurlijk zo}\\

\haiku{{\textquoteright} {\textquoteleft}Als jij m'n petje ',}{van de achterbank haalt dan}{doe ikt dak open}\\

\haiku{{\textquoteleft}Ik wilde niet aan,{\textquoteright}.}{\'e\'en stuk doorrijden zei Leo}{verontschuldigend}\\

\haiku{Hij kocht zelfs nog een {\textquoteleft}{\textquoteright}.}{flesje wijn omthuis bij de}{tv op te drinken}\\

\haiku{Hij ging daarom maar.}{aan de grote tafel in}{het midden zitten}\\

\haiku{Het bericht kwam toch:}{niet helemaal goed door want}{meneer de Bruin riep}\\

\haiku{Jij moet je er niet.}{mee bemoeien als je er}{geen verstand van hebt}\\

\haiku{{\textquoteright} {\textquoteleft}Ik mag geen verstand,}{van stereo hebben maar van}{jou weet ik alles}\\

\haiku{{\textquoteright} {\textquoteleft}Geen nieuws,{\textquoteright} zei De Bruin.}{en probeerde zich uit de}{voeten te maken}\\

\haiku{Thomas beloofde,.}{Thomas bezwoer hen dat hij}{zich beheersen zou}\\

\haiku{{\textquoteright} Thomas keek op en.}{zag hoe Martje de tafel aan}{het afruimen was}\\

\haiku{{\textquoteleft}Laat toch staan kindje,,.}{morgen gaan we verder ga}{toch lekker naar bed}\\

\haiku{Ik geloof dat we,{\textquoteright}.}{beter naar bed kunnen gaan}{zei Thomas praktisch}\\

\haiku{Dat kunt u zich niet,{\textquoteright}.}{voorstellen zei mevrouw de}{Bruin nu iets kalmer}\\

\haiku{{\textquoteleft}Heeft hij u,{\textquoteright} Martijn, {\textquoteleft}?}{kon het woord maar niet vinden}{geslagen of zo}\\

\haiku{{\textquoteright} zei ze toen weer en.}{voor Martijn op kon staan was}{het haar al gelukt}\\

\haiku{Barbara had op:}{zich genomen met Martje te}{praten of beter}\\

\haiku{{\textquoteleft}Ik ga even vragen,{\textquoteright}.}{of ze iets wil zei Thomas}{en liep naar boven}\\

\haiku{Niettemin deed het.}{hem goed in de vertrouwde}{omgeving te zijn}\\

\haiku{Leo scheurde door de.}{drie bochten welke zijn huis}{van zijn werk scheidden}\\

\haiku{Van vrouwen wist je.}{eigenlijk nooit hoe preuts ze}{nou precies waren}\\

\haiku{Hij had daar nooit veel.}{goeds van verwacht want van het}{een kwam het ander}\\

\haiku{Nog nooit had Martje een!}{snackbar gezien met zoveel}{Telegraaflezers}\\

\haiku{Daardoor kwam het ook.}{dat ze de auto's en zelfs}{de bus niet hoorde}\\

\haiku{{\textquoteright} zei Leo geschrokken.}{en voelde driekwart van zijn}{energie wegzakken}\\

\haiku{{\textquoteleft}Welnee kerel,{\textquoteright} zei.}{Van Zutphen en zwaaide de}{rook uit z'n buurt weg}\\

\haiku{Hij legde z'n hand' {\textquoteleft},,}{op Thomas mouw en schonk diens}{glas vol.Hier drink op}\\

\haiku{{\textquoteright} {\textquoteleft}Nee,{\textquoteright} zei Thomas, {\textquoteleft}uit,,.}{haar briefje v\'o\'ordat ze naar}{ons toekwam bleek niets}\\

\haiku{{\textquoteleft}Ik heb ook nog aan,{\textquoteright}.}{de studentenpsycholoog}{gedacht zei Thomas}\\

\haiku{Nee, niet wat me te,,.}{wachten staat wat er gebeurt}{tegelijk gebeurt}\\

\haiku{{\textquoteright} {\textquoteleft}Nee,{\textquoteright} zei Roos, {\textquoteleft}dat hoeft,.}{niet ik hoef het deze keer}{niet meer te horen}\\

\haiku{Martje maakte zich los.}{van haar eigen avontuur en}{knikte naar Geesje}\\

\haiku{Ik kan toch beter,.}{gewoon bij m'n verslaafde}{zootje blijven dacht Gees}\\

\haiku{En Thomas was er.}{al helemaal niet voor in}{het goede humeur}\\

\haiku{{\textquoteleft}Meisje,{\textquoteright} zei Thomas, {\textquoteleft},{\textquoteright}.}{wat fijn dat je meeging en}{verder zwegen ze}\\

\haiku{'t Is trouwens niet,,!}{warm ook schiet eens een beetje}{op ouwe jongen}\\

\haiku{Net op tijd om het.}{eten om zes uur op tafel}{te kunnen zetten}\\

\haiku{{\textquoteright} Leo wist zich geen raad,.}{en ging maar weer eens achter}{z'n bureau zitten}\\

\haiku{{\textquoteright} {\textquoteleft}Nee,{\textquoteright} zei Thomas, {\textquoteleft}weet?}{je waarom het een rare}{vraag is geworden}\\

\haiku{{\textquoteright} Barbara stond nog.}{steeds met de sandwiches en}{de limonade}\\

\haiku{Maar ze had het nog.}{niet opgeborgen of het}{meisje wilde m\'e\'er}\\

\haiku{Misschien een wieg of....}{luiers of Leo's hoofd liep}{om van de zorgen}\\

\haiku{{\textquoteleft}Dag Anke,{\textquoteright} zei Leo, {\textquoteleft}?}{vriendelijkwat heb je voor}{me te doen vandaag}\\

\haiku{En mevrouw Elisa {\textquoteleft}{\textquoteright}.}{wiegde het kindbiem-bahm}{op het klokgelui}\\

\haiku{Nee meneer Evers, niet,...}{doen komt u hier even zitten}{en neemt u meneer}\\

\haiku{Het was een eindje,,.}{om en ja zo was het het}{was een eindje om}\\

\haiku{{\textquoteright} En ze begon aan,.}{Wouter te trekken die heel}{langzaam overeind kwam}\\

\haiku{Dus De Bruin knikte.}{maar een beetje dat het een}{goed idee van haar was}\\

\haiku{{\textquoteleft}En je potloodje,{\textquoteright},.}{zei De Bruin maar dat verzon}{hij in zijn overmoed}\\

\haiku{{\textquoteright} zei de man die het.}{telefoonnummer bijna}{goed had gelezen}\\

\haiku{We zouden misschien,,}{het huis in twee\"en kunnen}{delen jij boven}\\

\section{J. Huf van Buren en Maurits Sabbe}

\subsection{Uit: De twee invasies. Hoe grootvaders broer uit den oost weerkeerde}

\haiku{En 't is ook een'.}{ijselijke bazar op}{straat met al da volk}\\

\haiku{Een luide lach, en {\textquoteleft},!}{eenmerci monsieur ne}{vous d\'erangez pas}\\

\haiku{Water in korten,;}{tijd gesproken werd zal ik}{hier niet herhalen}\\

\haiku{Albertine gaf;}{ten antwoord dat hij dan maar}{bij haar blijven moest}\\

\haiku{Dan spijt het mij dat,.}{het zoo geloopen is zei}{de jonge baron}\\

\haiku{Ik wil  niet, dat.}{er ooit over gesproken of}{om getreurd worde}\\

\section{Gerard van Hulzen}

\subsection{Uit: De ontredderden. Eerste bundel}

\haiku{Wie zal de lijn hier?}{trekken en wie heeft het bij}{het goede eind}\\

\haiku{'t Ging belabberd,, ',.}{vanochtend merkte-iet}{dadelijk al goed}\\

\haiku{- Welja, ik mot maar,.}{altoos hellepe weerde}{hij onwillig af}\\

\haiku{Het Jantje voelde,,.}{toch iets ervan bleef staan en}{keek om verwonderd}\\

\haiku{Ze schudde haar door,.}{elkaar om haar gauwer te}{laten bekennen}\\

\haiku{- Nou jij of je broer...}{dacht je soms dat ik kledder}{in m'n ooren heb}\\

\haiku{- Jawel, je hebt 'et,.}{maar voor kommandeere schampte}{Hein brutaal terug}\\

\haiku{Betje bedelde.}{nu alleen en waagde zich}{weer in de avondstraat}\\

\haiku{- Ja zeker, ik meen,.}{het we kunnen je toch het}{heele jaar niet ho\^ue}\\

\haiku{Op zijn verslonsde;}{gele haren kleefde een}{beetje schuin de pet}\\

\haiku{Voor de bewoners '.}{waren die schelknoppen in}{t geheel niet noodig}\\

\haiku{Hij bromde nog eens,.}{hum en zij werkte en wreef}{weer voort aan haar stoel}\\

\haiku{- Je kon anders best,.}{wachten tot ik weg was dan}{heb je de ruimte}\\

\haiku{- Hier heb je de prijs.}{die je koopt en hier is de}{primie die je wint}\\

\haiku{Nou hoef ik je niks.}{meer te zeggen wat bloemen}{wel tot stand brengen}\\

\haiku{Hij rekte dit woord,,:}{primie rekte opnieuw de}{woorden en bralde}\\

\haiku{'t Gaat hier net as,! '}{bij de staatsloterij zoo}{eerelik als goud}\\

\haiku{Maar ze liet zich niet,;}{verdringen duwde met haar}{achterste terug}\\

\haiku{t Is heelemaal......?}{zes-en-twintig cente}{wat wou je d'ervan}\\

\haiku{Ze had het hem nog,.}{niet verweten nee dat moest}{ze ook eens lappen}\\

\haiku{Mies, nu ook wakker,.}{geworden kwam eruit en}{kuste haar moeder}\\

\haiku{O, hij begreep het, ',.}{t kwam door dat vriesweer dat}{alles zoo opklonk}\\

\haiku{zijn oogen staarden blind.}{tegen haar strak-harde rug}{als tegen een muur}\\

\haiku{Dan kwakte hij de,:}{gevulde schepper in de}{bak terug gromde}\\

\haiku{zeg ik je, as we,.}{op straat komme te staan trek}{ik ertusschen uit}\\

\haiku{een loopende hond.}{valt allicht wat in de mond}{was toch het spreekwoord}\\

\haiku{En toch 't moest, als, '.}{hij langer wachtte gingt}{heelemaal niet meer}\\

\haiku{Hij liet zijn jas los,,:}{die op de stoel gleed schreeuwde}{dan ineens gedurfd}\\

\haiku{t vriest toch veel te...,!!}{hard om te kunne plakke}{ja maak mijn dat wijs}\\

\haiku{Hij holde door, was,.}{de onderste trap al af}{buiten haar bereik}\\

\haiku{Wat kocht ze in al,, '.}{die tijd bij een bedroefd drupje}{t meeste voor h\`em}\\

\haiku{Thuis hadden ze 't,.}{niet breed maar de kast zag er}{toch behoorlijk uit}\\

\haiku{De kwaadheid, zoo lang,,.}{bedwongen ziedde brak uit}{naar alle kanten}\\

\haiku{je zwijgt en zwijgt, doet ',.}{of jet niet merkt om de}{vree te bewaren}\\

\haiku{Ach, ach, wat maakte,?}{een vrouw niet mee wat kreeg ze}{al niet te doorstaan}\\

\haiku{wel tien keer streek ze ',;}{t zelfde vouwtje uit en}{haar oogen staarden blind}\\

\haiku{Een oogenblik dacht,,.}{ze dat Baller er aan kwam}{maar ze zag verkeerd}\\

\haiku{Ze voelde haar hoofd.}{sufzwaar worden van al dat}{moeizaam overleggen}\\

\haiku{Van zoo'n steggel, zoo'n.}{penningfokker had ze niet}{veel te verwachten}\\

\haiku{Lekker zou-ie haar ',!}{noues troeven nou had ze}{net niks te zeggen}\\

\haiku{Een vermoeden van,.}{vuilheid golfde in haar op}{maakte haar dol}\\

\haiku{Gejacht liep ze de,.}{straat ten einde zonder te}{beseffen waarheen}\\

\haiku{Een huiver van kou,.}{en killigheid omving haar}{drong door alles heen}\\

\haiku{In elk geval eerst!}{die zak ergens neerzetten}{en dan verder zien}\\

\haiku{Wie kan dat zeggen, '....}{t gaat van zelf bijna als}{geboren-worden}\\

\haiku{- Wat is er gebeurd,,.}{herhaalde Greet om haar tong}{wat los te krijgen}\\

\haiku{'t Viel haar on-. '}{noemlijk zwaar tot haar ouwe}{doen terug te keeren}\\

\haiku{De glad gevroren.}{straten maanden haar weer aan}{tot voorzichtig gaan}\\

\haiku{Ze hoorde of zag,.}{niets meer van wat om haar heen}{liep keek niet meer uit}\\

\haiku{O, hij zou zich wel,,!}{redden hij had haar niet noodig}{o nee volstrekt niet}\\

\haiku{Wat gaf het of hij,.}{hier al vuur aanle{\^\i} als ze}{toch niet opdaagde}\\

\haiku{Het voorgevoel nam '.}{in een paar telt begrip}{van zekerheid aan}\\

\haiku{De mannen raakten,.}{in verwarring wisten niet}{wat ze moesten zeggen}\\

\haiku{De werkelijkheid.}{leek hem minder erg dan al}{die grijnsgezichten}\\

\haiku{Waarom wou-ie zich?}{toch in eigen oogen beter}{maken dan hij was}\\

\haiku{De kroegen zouden,.}{gauw gaan sluiten nu kon hij}{nog eentje nemen}\\

\haiku{Achter hem in 't.}{dorre hout meende hij te}{hooren ritselen}\\

\haiku{Een afgezakte,......}{een afgetrapte was-ie}{kostte geld voor niets}\\

\subsection{Uit: De ontredderden. Tweede bundel}

\haiku{Nu werd zijn aandacht ' '.}{getrokken doorn groepje}{menschen opt strand}\\

\haiku{Met 'n enkele ',:}{handstoot duwde hijn paar}{jongens op zij riep}\\

\haiku{Dat was een leelijk,.}{geval alles behalve}{een buitenkansje}\\

\haiku{Maar terwijl hij zich.}{dit inpraatte voelde hij}{zich  niet zeker}\\

\haiku{'t Zou 't beste.}{zijn even te gaan kijken om}{zich te overtuigen}\\

\haiku{- Daarom hoef je niet,...!}{dadelijk te ranselen}{je lijkt wel een beul}\\

\haiku{- Wat zeur je dan... je '...}{mott natuurlik zoolang}{mogelijk volho\^ue}\\

\haiku{'t Was 't relaas,,.}{dat ze elkaar aldoor voor}{ze{\^\i}en al jaren}\\

\haiku{- Ik denk 't wel... we...}{kunne natuurlik bij de}{direktie probeere}\\

\haiku{Naar alle kanten,.}{vluchtten de menschen in trams}{in koffiehuizen}\\

\haiku{Hij voelt de raakheid,.}{van die woorden doch wil er}{niet aan toegeven}\\

\haiku{De vrouw, stil in het,.}{achtervertrek voelt een glimp}{van verheugenis}\\

\haiku{Nou  hebben ze,,.}{de tijd blijven ze plakken}{verteren toch niet}\\

\haiku{Zij wil wel weg, blijft '.}{toch staan en weet niet hoe ze}{t aanleggen zal}\\

\haiku{Affijn, eerst probeeren -!}{het zaakje te verkoopen}{en dan verder zien}\\

\haiku{Weer ging een schijfje;}{appel over de lip van de}{meneer naar binnen}\\

\haiku{Nee, niet dadelijk,.}{gaan als je de centen hebt}{zei hij bij zichzelf}\\

\haiku{kijk wat 'n lef, liet,.}{daarop de oogen weer zakken}{als bedacht hij zich}\\

\haiku{Onderwerping, haat,.}{en weer onderwerping ze}{kwamen in \'e\'en tel}\\

\haiku{Hij merkte ook, dat,.}{ze op straat bleven staan dat}{ze naar hem keken}\\

\haiku{waar-je je hoofd kon,;}{neerleggen alle avonden}{op dezelfde plek}\\

\haiku{Be menschen weten.}{bij lange niet wat voor goeds}{in een borrel zit}\\

\haiku{Och, als hij nou weer, '.}{geld had zout zeker niet}{anders gaan als toen}\\

\haiku{Laatst nog, nee nu al,!}{weer drie jaar geleden met}{die zes week ziekte}\\

\haiku{Triestig staarde hij '.}{voor zich uit overt water}{van de Vijverberg}\\

\haiku{zoo'n vent was 't niet.}{waard om er twintig jaar voor}{achter slot te gaan}\\

\haiku{Meer met smeekende. '}{blikken dan met woorden zocht}{hij te vermurwen}\\

\haiku{- Drink ook 'es effe,,.}{riep er \'e\'en die gulzig aan}{de emmer slokte}\\

\haiku{- Werk je anders maar,, '...}{niet kapot schreeuwde Jaap die}{nu aant vuur stond}\\

\haiku{ze mochten Roelf graag,.}{lijden wilden hem niet over}{de kop jakkeren}\\

\haiku{Er valt altijd wat.}{te vinden of te vitten}{als ze d\`at willen}\\

\haiku{- Dat geloof ik... heb... '!}{motte anpietsek was}{heel wat ten achter}\\

\haiku{Tegelijk maakte:}{hij een beweging waarmee}{hij wilde zeggen}\\

\haiku{- Je hebt gelijk, ik,}{heb er ook al drie binne}{en morge vroeg is}\\

\haiku{- Het is niet goed te,.}{drinken waagde ze enkel}{schuchter te zeggen}\\

\haiku{het bleeke vermoeden,.}{dat ook haar jongen eens zoo}{iets zou overkomen}\\

\haiku{- 't Is zonde hem,,!}{te roepen zei ze in haar}{zelf maar het moet toch}\\

\haiku{De baas stond naar hem,;}{te gluren en hij voelde}{dat-ie knoeide}\\

\haiku{'t Was of met de.}{komst van Trees een heele blije}{wereld binnen schoof}\\

\haiku{Ik ben er niks op!}{gesteld wat jullie onder}{elkaar verknussen}\\

\haiku{Van half drie, tot half,,,}{vier van half vier tot half vijf}{zat ze aangekleed}\\

\haiku{Tegen vijf uur trok,.}{ze d'erop uit nam de tram}{naar Scheveningen}\\

\haiku{Maar hij vatte haar ',,:}{omt midden kuste haar}{wild op de mond zei}\\

\haiku{- En misschien nog wel,.}{wat anders ook giechelde}{Marie niet erg kiesch}\\

\haiku{Zijn oogen waren zoo,.}{schelvischachtig-grauw maar}{zijn neus vond ze leuk}\\

\haiku{ze moest het zich nog,.}{eens inprenten dat ze niet}{weer met hem uitging}\\

\haiku{Hij vroeg zich aldoor.}{af waar ze gisteravond naar}{toe kon zijn gegaan}\\

\haiku{Alwat ze zei of '.}{deed zou immerst geval}{enkel vergrooten}\\

\haiku{Maar een jong, wild ding,.}{als Trees die wil niet altijd}{stemmig thuis zitten}\\

\haiku{Ze wist van haar zelf.}{wat het beteekent als een vrouw}{zoo'n zwakke man krijgt}\\

\haiku{Met een strakke blik,.}{stond ze over hem gebogen}{de lippen bevend}\\

\haiku{- H\`e, gelukkig dat,!}{het eruit is dat h\^et me}{lang genog benauwd}\\

\haiku{Zij kon geen antwoord,;}{vinden suste dat hij zich}{rustig zou houden}\\

\haiku{Dan zakte ze op,.}{de stoel neer verborg haar moe\"e}{hoofd in de handen}\\

\haiku{Haar jongen had toen.}{overal loopen zoeken en}{haar niet gevonden}\\

\haiku{goed, jubelde Trees,.}{harte-gereed blij met}{deze oplossing}\\

\haiku{'s Nachts in haar bed, '.}{dacht ze toch aan Roelf en vond}{zet doodjammer}\\

\haiku{zijn moeder en 'n,.}{tante en nog een nichtje}{stonden er eveneens}\\

\haiku{Soms mocht-ie al op ',.}{t balkon komen en in}{de tuin wandelen}\\

\haiku{Een vochtglinstering,.}{in haar oude oogen weersprak}{dit goed vertrouwen}\\

\haiku{- Jij, vroeg z'n moeder,,!}{ontzet welnee jongen dat}{zal niet gebeure}\\

\haiku{- Zoo, maar ik heb er... ',!}{genog vant hangt me de}{keel uit d\`at zoeken}\\

\haiku{Vijlen beraspten,.}{de doorgeslagen nagels}{werkten de hoef bij}\\

\haiku{'t maakte haar zelf.}{week  en onmachtig om}{verder te kunnen}\\

\haiku{Ze hield de wagen.}{in en liet het kind van haar}{moede arm zakken}\\

\haiku{al hoorde ze niet,.}{de woorden ze voelde toch}{wat daar werd gezegd}\\

\haiku{Wat stapten ze vlug,.}{ook al droeg die juffer nog}{zulke hooge hakken}\\

\haiku{Hard ratelden de.}{wrakke radertjes over de}{vlakke klinkerweg}\\

\haiku{Smadelijk keek de,.}{inspekteur naar haar dan naar}{de vier kinderen}\\

\haiku{Die mooie lucht zag ze.}{dreigend met zijn rosse en}{violette pracht}\\

\subsection{Uit: Wrakke levens}

\haiku{Zoo was het lange, '.}{tijd maart gaat gelukkig}{nu veranderen}\\

\haiku{Die moest ze nog eerst,....}{afmaken maar dan dan kon}{ze wat uitrusten}\\

\haiku{'t kwam zeker van!}{al dat gewoel langs zijn oogen}{en van de warmte}\\

\haiku{- Ja, klakte Sijpert,,?}{het lijkt er wel wat op maar}{wat zul je eran doen}\\

\haiku{- Dat meen ik ook! - 't,,,?}{Zou gaan maar o mijn vrouw waar}{moet die dan blijven}\\

\haiku{- Zeker, je hebt hier, ',}{ook kans van genezen maar}{wat ist geval}\\

\haiku{Hij moest het nog eens,.}{aanzien d\`at groeide nu van}{zelf weer in hem aan}\\

\haiku{En dan, hij wil je,.}{ook graag zelf genezen dat}{begrijp je toch wel}\\

\haiku{- Hou-je goed kerel,,.}{tot het volgende jaar dan}{zei luchtig De Greef}\\

\haiku{ze mocht niet bij hem,.}{blijven al had hij het hard}{te verantwoorden}\\

\haiku{- Gaat u maar naar hem... '.}{toe hij h\`et zoo-evenn}{benauwdheid gehad}\\

\haiku{Maar ze kwam weer in,,.}{beklag en dat troostte even}{vergoedde wel wat}\\

\haiku{Als je in Zurich,.}{overnacht spreek je toch van geen}{poste-restante}\\

\haiku{Het eene voorwendsel '.}{leek nog onwaarschijnlijker}{dant andere}\\

\haiku{Ze wilde de brief,.}{aan haar lippen drukken maar}{ze hield zich nog in}\\

\haiku{Uit haar ooghoeken,,;}{zoo pijnlijk van vermoeidheid}{perelde een traan}\\

\haiku{Ze vond zich zelf weer.}{kinderachtig met haar}{ingebeelde angst}\\

\haiku{Zijn patroon had hem,,.}{door goede menschen gesteund}{hier heen gezonden}\\

\haiku{voor Rekeltje bleef.}{nagenoeg geen kansje om}{een plaats te vinden}\\

\haiku{Vooruit Rekeltje,,!}{kommandeerde de baas nou}{tegen mij jouw werk}\\

\haiku{Maar dat zag de boer ',!}{in de spiegel en die aan}{t opspele nou}\\

\haiku{Maar het afloopen '.}{ging toch gemakkelijker}{dant opkomen}\\

\haiku{ze wist dus niet of.}{ze wel zoo goed vooruitging}{als ze zelf meende}\\

\haiku{Soms keerde een zich ',.}{om of sprakn paar woorden}{anders bleef het stil}\\

\haiku{Maar Fiene Tas, die,,:}{beter wist vinnigde al}{op haar in schampte}\\

\haiku{Zij deed het alleen,.}{uit angst om de praatsters een}{beetje in te toomen}\\

\haiku{ik zou wel zegge,... '.}{dat het niet zoo erg meer is}{alleens morgens}\\

\haiku{Ik zou natuurlijk,...}{graag terug gaan maar als u}{denkt dat het niet kan}\\

\haiku{Ze verzond de brief, '.}{nog dezelfde avond maart}{gaf haar weinig rust}\\

\haiku{- Omdat jij nou niet,!... -,...?}{weg mag daarom scheld je op}{mijn Niet weg niet weg}\\

\haiku{Je moest het toch al! -,! -? - '...}{hebben Ja n\`et gekregen}{En Ik weetet niet}\\

\haiku{ze bleven ieder, '.}{in haar eigen hoekje voor}{t eigen raampje}\\

\haiku{De jongen boven,,.}{dat wist ze nu zeker ging}{langzaam achteruit}\\

\haiku{Een tweede snit werd.}{gedaan en ongemerkt trok}{de zomer voorbij}\\

\haiku{Grietje Groen, weer wat,.}{opgekwakkeld luchtkuurde}{nu en dan maar eens}\\

\haiku{Vlijm en scherp hoorde,:}{ze weer de woorden van de}{dokter die toch zei}\\

\haiku{Maar 't gerucht trok.}{aldoor zijn blikken en dat}{maakte hem nerveus}\\

\haiku{- Wat denkt u ervan,.}{vroeg hij eindelijk toen de}{dokter bleef zwijgen}\\

\haiku{- Dat in geen geval... -... -,!}{En ik dacht Beste jongen}{laat mij nu denken}\\

\chapter[2 auteurs, 88 haiku's]{twee auteurs, achtentachtig haiku's}

\section{Victor Ido}

\subsection{Uit: De paupers}

\haiku{Als ik 't maar heb,...{\textquoteright} {\textquoteleft}, '.}{BoongNonsens dat zwijn ist}{waarempel wel waard}\\

\haiku{{\textquoteright} {\textquoteleft}Nou ja... als ze Pa ', '.}{maarn groot present geven}{dan ist ook goed}\\

\haiku{Dat wist Boong heel goed, ':}{en daarom had ie stierlijk}{t land aan Vincent}\\

\haiku{{\textquoteright}, kwam hij, in nachtbroek,.}{en kabaai gestoken weer}{in den kring zitten}\\

\haiku{Aller aandacht was.}{nu gespannen op hetgeen}{hij vertellen zou}\\

\haiku{{\textquoteleft}Je weet, kind, grootpa, '....}{kan d'r niet meer tegenn}{half glaasje maar z\'o\'o}\\

\haiku{Zij ziet niet rood van, ',.}{trots maar vant vuur in de}{keuken natuurlijk}\\

\haiku{Tjang Sina, zorgzaam,}{door Tietie bediend maalde}{onder zacht gesmek}\\

\haiku{En ongeduldig,.}{werden ze wijl het al zoo}{laat was in den nacht}\\

\haiku{Deze mopperde, ';}{zei datt al zoo laat en}{zijn paard zoo moe was}\\

\haiku{hij had liever dat,.}{hij nu maar betaald werd dan}{kon hij naar huis gaan}\\

\haiku{O, die meisjes en,, ',.}{vrouwen vooral de mooie zijn}{n raadsel vond Krol}\\

\haiku{Zij snakte er naar,,....}{daarmee kennis te maken}{daarin te verkeeren}\\

\haiku{Dat was al jaren.}{zoo geweest en dat zou wel}{nooit veranderen}\\

\haiku{{\textquoteright} Sam voelde een groot*.}{verdriet rijzen in zijn}{gemoed en zweeg}\\

\haiku{{\textquoteright} Malie beurde het,,:}{hoofd dit heftig schuddend op}{en riep beslist uit}\\

\haiku{Ik houd niks van Boong, ' ',.}{t isn leeglooper die}{voor de galg opgroeit}\\

\haiku{De twijfel kwelde.}{hem al zoo lang in zijn werk}{en in zijn droomen}\\

\haiku{{\textquoteright} Vincent poogde kalm,.}{te blijven ofschoon hij ook}{niet meer gerust was}\\

\haiku{Vincent kon niet meer,.}{spreken het bloed stroomde uit}{en langs zijn lippen}\\

\haiku{In dien stillen nacht:}{streed hij een hevigen strijd}{in zijn binnenste}\\

\haiku{De vermoorde stond.}{bekend als een fatsoenlijk}{en oppassend man}\\

\haiku{Maar dan zou ze ook '....}{nog bemind worden doorn}{ander dan dien neef}\\

\haiku{'t Zou dom van haar, '.}{zijn indien zet al niet}{lang geprobeerd had}\\

\haiku{hij merkte hoezeer.}{het jonge meisje zich het}{gebeurde aantrok}\\

\haiku{{\textquoteright} {\textquoteleft}O, zeker, hij moet,,.}{je mooi vinden want je bent}{werkelijk mooi Da{\"\i}}\\

\haiku{*~          Spijtig vond hij ',;}{t niet dat grootpa weer naar}{zijn werk was gegaan}\\

\haiku{Wat bliksem, hij zou.}{toch zelf wel weten wat ie}{doen en laten moest}\\

\haiku{Kon ie dat beest maar;}{ald\'o\'or aan den praat krijgen tot}{Tjang uitgehuild had}\\

\haiku{Wat zou dat toch zijn,?}{dat ie daaraan niet langer}{weerstand kon bieden}\\

\haiku{{\textquoteleft}Je weet niet wat je,,.}{vraagt Da{\"\i} en daarom zal ik}{er niet boos over zijn}\\

\haiku{En we houden veel.}{meer van de vrijheid dan de}{totoks wel denken}\\

\haiku{De wijze, waarop,;}{zij hem toesprak stelde hem}{geheel gerust}\\

\haiku{hij was 'n Indo,*!}{en daarom wilden ze}{hem niet hebben}\\

\haiku{{\textquoteright} Boong's oogen schitterden.}{van ingehouden toorn en}{verontwaardiging}\\

\haiku{Nou ja, toch sama,,{\textquoteright}*.}{djoega zeg plaagde}{Perisa vroolijk}\\

\haiku{{\textquoteleft}We moeten trachten,....}{een verandering in den}{toestand te brengen}\\

\haiku{{\textquoteleft}Maar de bami is, ',{\textquoteright}.}{verrukkelijk dats waar}{vervolgde Reumer}\\

\haiku{Vroeger had-tie ', '....}{toch nooitn oogje op Da{\"\i}}{ik begrijpt niet}\\

\haiku{Nooit te voren had.}{hij haar ge{\"\i}nviteerd om}{samen uit te gaan}\\

\haiku{Dadelijk verdween.}{de droeve uitdrukking van}{haar mooi gezichtje}\\

\haiku{Zoo was hij op z'n:}{best gekleed en voelde zich}{deftig aangedaan}\\

\haiku{Men zou dan zien, dat,.}{zij niet alleen was maar een}{sterk geleide had}\\

\haiku{Snel was die liefde,....}{in haar opgekomen op}{het eerste gezicht}\\

\haiku{Nu, als u er geen, '{\textquoteright},.}{bezwaar tegen hebt doe ik}{t graag sprak Reumer}\\

\haiku{{\textquoteleft}U zegt nu wel, dat ', '.}{t niet waar is maar hoe kunt}{ut bewijzen}\\

\haiku{{\textquoteright} zuchtte kwasi-ernstig,.}{Reumer Nini's glas witten}{wijn opnieuw vullend}\\

\haiku{Almachtige God,,....}{Boong k\`ende die gestalte}{hij k\`ende die stem}\\

\haiku{{\textquoteleft}Ik zal tot 't laatst,,....}{toe om Lien vechten maar als}{je me vermoordt Boong}\\

\haiku{Vincent wilde niet, -.}{hebben dat hij Lien trouwde}{dat was duidelijk}\\

\haiku{Heden en toekomst.}{behoorden van toen af niet}{meer hem alleen toe}\\

\haiku{Oedit = stoffen.}{ceintuur waarmee de sarong}{wordt opgehouden}\\

\haiku{182.Serimpi =.}{hofdanseres der Sultans}{van Midden-Java}\\

\section{Ad van Iterson}

\subsection{Uit: Zuiderlingen}

\haiku{{\textquoteleft}and his mama cried...{\textquoteright}:}{En toen zei de diskjockey}{wat er was gebeurd}\\

\haiku{Je zal ons altijd.}{en overal treffen en dan}{drinken we er \'e\'en}\\

\haiku{de gangen van de -:}{economische faculteit}{om precies te zijn}\\

\haiku{Het lirium Van.}{de winter hebben we oom}{Pierre begraven}\\

\haiku{{\textquoteleft}En nu is hij dood,,{\textquoteright}.}{die arme Pierre zei de}{moeder van Nelly}\\

\haiku{maar... maar had hij nu,!}{maar naar haar geluisterd dan}{was het niet gebeurd}\\

\haiku{{\textquoteleft}Het spijt me, jong, we -,?}{hebben geen oud papier ach}{val om ben jij het}\\

\haiku{Mijn zoons hebben hem,.}{Pele genoemd omdat het}{ook zo'n zwarte is}\\

\haiku{{\textquoteright} zei hij, maar voegde,,:}{er toen ik het hek al had}{opengedaan aan toe}\\

\haiku{{\textquoteright} zei hij, terwijl hij.}{maar met moeite zijn evenwicht}{wist te bewaren}\\

\haiku{Oom Pierre, die dan,.}{al uren in Brand's Bierhuis zat}{zag ik haast nooit meer}\\

\haiku{Hij zeeg helemaal,.}{achterover van het lachen}{totdat zijn hoed viel}\\

\haiku{De volgende dag.}{in alle vroegte zijn ze}{hem komen halen}\\

\haiku{Daar hebben we de:}{klassieke tegenstelling}{uit de zielsleer weer}\\

\haiku{vijf met de claxon en.}{drie met de pneumatische}{deurinstallatie}\\

\haiku{De witte pater.}{beschreef hen in enkele}{brede verfstreken}\\

\haiku{De weg tonen en -?}{de waarheid brengen wordt dat}{nog wel eens gedaan}\\

\haiku{Daar in de buurt op.}{een hoek wist hij een caf\'e}{waar veel werd gekaart}\\

\haiku{{\textquoteleft}Vanmiddag kreeg hij,.}{opeens geen lucht meer hebben}{ze me net verteld}\\

\haiku{Omdat Pierre te,.}{muzikaal bleek werd hij op}{de tuba gezet}\\

\haiku{{\textquoteright} zei Caroline, die.}{nog steeds duizelig van de}{pirouettes was}\\

\haiku{{\textquotedblleft}De allergrootste.}{reden was toch dat hij het}{gewoon in zich had}\\

\haiku{Deze herschikking.}{gaf de groepsdans n\'og meer vuur}{en locomotie}\\

\haiku{{\textquotedblright} En ik spring, tussen,,.}{twee boten door whaaf het meer}{van Bolsena in}\\

\haiku{zou hij soms naar haar?}{benen kijken in plaats van}{naar haar onderrok}\\

\haiku{Als ze een heuvel,,}{opreden keek ze aan \'e\'en}{stuk achterom bang}\\

\haiku{{\textquoteright} {\textquoteleft}Meer op Belgi\"e en,{\textquoteright}.}{Duitsland geori\"enteerd}{knikte Tiny Ummels}\\

\haiku{Hier groeien planten.}{die nergens anders op de}{wereld voorkomen}\\

\haiku{Ze wierp een steelse,.}{blik opzij maar ze zag niets}{bijzonders aan hem}\\

\haiku{{\textquotedblleft}Het onweer zal wel,,{\textquotedblright}:}{overdrijven want het komt van}{Belgi\"e af zei hij}\\

\haiku{Meneer Rudy staat.}{met een glunderend gezicht}{onder de schoolbel}\\

\haiku{En of ze nou echt,,.}{precies zo zijn gebeurd jong}{dat maakt geen tuit uit}\\

\haiku{Ik herinner me.}{dat we een keer in de trein}{naar Brussel zaten}\\

\haiku{Ich k\"usse ihre Hand.{\textquoteright} -!}{ik geef hun een handkus gans}{in het nette hoor}\\

\haiku{En wat een geduw!}{en een getrek van al die}{carnavalsgekken}\\

\chapter[5 auteurs, 426 haiku's]{vijf auteurs, vierhonderdzesentwintig haiku's}

\section{A.M. de Jong}

\subsection{Uit: Frank van Wezels roemruchte jaren \& Notities van een landstormman}

\haiku{dat hij met zoveel.}{inspanning om zich heen had}{weten te krijgen}\\

\haiku{Zoiets moest je aan,.}{den lijve ervaren eer}{je er oog voor kreeg}\\

\haiku{'t Is al zo lang,.}{geleje da'k geen uniform}{om me donder had}\\

\haiku{{\textquoteright} {\textquoteleft}Make ze nooit meer,?}{behoorlijke soldaten}{van geloof jij wel}\\

\haiku{{\textquoteleft}O{\textquoteright}, vroeg Rengers, de, {\textquoteleft}?}{studentnoemen ze dat in}{het leger een bed}\\

\haiku{Zij staken de borst,:}{vooruit trokken de schouders}{vierkant en hoonden}\\

\haiku{{\textquoteleft}Waarom zijn we toch?}{zo gek om dat allemaal}{maar goedsmoeds te doen}\\

\haiku{Ik kan me toch niet?}{as een kleine jongen uit}{laten kankeren}\\

\haiku{en als het eenmaal, '.}{gezegd was wist jet voor}{je hele leven}\\

\haiku{ze schoven immers!}{alleen maar zover ze zelf}{geschoven werden}\\

\haiku{Och{\textquoteright}, antwoordde Frank, {\textquoteleft}'.}{t is eigenlijk toch een}{komieke wereld}\\

\haiku{Je bent een sekreet{\textquoteright},:}{grauwde Van Wezep en zijn}{vriend voegde er bij}\\

\haiku{{\textquoteright} Frank keek nadenkend.}{naar de slapende kleine}{jongen in de wieg}\\

\haiku{As je dienst weigert,.}{verzet je je tegen de}{bestaande orde}\\

\haiku{Langzaam wendde hij,.}{zijn ogen naar hun gezicht keek}{ze een voor een aan}\\

\haiku{{\textquoteleft}Wie weet waar we zelf. '.}{nog toe koment Is een}{treurige wereld}\\

\haiku{Nou goed de kolf in.}{je schouder drukken en niet}{bang zijn voor het schot}\\

\haiku{Dus salueerde,.}{hij maakte rechtsomkeert en}{ging het bureau af}\\

\haiku{{\textquoteright} En even later kwam,}{hij met een somber gezicht}{de kamer weer in}\\

\haiku{Hun tong lag verdroogd,.}{in hun mond en zij waren}{hongerig en moe}\\

\haiku{om uit te houden,.}{in de andere is geen}{redding mogelijk}\\

\haiku{een officier wordt,.}{geboren op de dag dat}{men hem be\"edigt}\\

\haiku{Ze vervloekten de.}{hele wereld in de zwartst}{denkbare termen}\\

\haiku{Zaterdagsmorgens.}{zouden ze met de eerste}{trein naar huis reizen}\\

\haiku{Het gezicht van de.}{kompagnieskommandant werd}{bijna menselijk}\\

\haiku{Hij dacht ook aan de.}{aanstaande promotie en}{wist zich verloren}\\

\haiku{Hij was nog steeds op,,.}{z'n hoede vertrouwde de}{zaak maar half en zweeg}\\

\haiku{Met een beetje goeie '.}{wil zullen wet onder}{mekaar wel vinden}\\

\haiku{hun leven, hun werk,,.}{hun vak hun denkbeelden op}{allerlei gebied}\\

\haiku{{\textquoteright} De schaduw van een.}{glimlach gleed heerlijk over de}{bruine gezichten}\\

\haiku{En ze spraken af,.}{dat ze doodeenvoudig niet}{mee zouden zingen}\\

\haiku{Dit strekte niet om.}{de woede der landstormers}{te doen bedaren}\\

\haiku{Jij dacht, dat ik je?}{nou elke week een spoorpas}{zou moeten geven}\\

\haiku{{\textquoteleft}Mag ik misschien ook,',?}{weten luit waarom u me}{hebt laten roepen}\\

\haiku{Het was de eerste.}{schrede op het pad naar de}{rang van officier}\\

\haiku{{\textquoteleft}Ga d'r eerst es as{\textquoteright},.}{een fatsoenlijk mens bijstaan}{gromde de stuurman}\\

\haiku{Wel, De Groot, as je,.}{niet boksen kunt wordt de zaak}{veel eenvoudiger}\\

\haiku{{\textquoteright} De overrompelden.}{keken stom verbaasd naar de}{vreemde snoeshanen}\\

\haiku{Maar hij voelde nog.}{meer voor een open brief aan die}{deftige burgers}\\

\haiku{Alleen het rooie blad:}{kwam er mee en dadelijk}{was Leiden in last}\\

\haiku{Jullie hadden geen.}{klachten mogen hebben over}{je inkwartiering}\\

\haiku{Maar verdomd lelijk '...?}{blijftet Is dat een bakkes}{voor een militair}\\

\haiku{Ik vind de stukjes{\textquoteright},.}{prachtig geschreven deelde}{Barten rustig mee}\\

\haiku{Hij greep onverwijld:}{zijn kwartiermuts van de bank}{en zei opgelucht}\\

\haiku{{\textquoteright} {\textquoteleft}Zeker, kolonel,,.}{maar ik geef er de voorkeur}{aan dat niet te doen}\\

\haiku{Hij verdomde het.}{nu zeker om een direkt}{antwoord te geven}\\

\haiku{Toen hij terug op,}{de kamer kwam wisten de}{kameraads het al.}\\

\haiku{je gauw ziek wordt, laat '.}{ik je van avond nog naart}{hospitaal brengen}\\

\haiku{Zouden de cellen?}{in de gevangenissen}{ook onverwarmd zijn}\\

\haiku{Hij kon precies langs,.}{de brits schuurde dan al met}{z'n arm langs de muur}\\

\haiku{Hij voelde zijn pols,,.}{lei de hand op zijn hart keek}{op z'n horloge}\\

\haiku{De redaktie zou.}{iemand sturen om alle}{bijzonderheden}\\

\haiku{Het toneeltje met.}{het Kind stond te vlak vooraan}{in zijn bewustzijn}\\

\haiku{Bijna allemaal.}{stamgasten van provoost en}{politiekamer}\\

\haiku{Ben je een haartje?}{belazerd om je de boel}{zo aan te trekken}\\

\haiku{Ik ben Tolman, neem,.}{me niet kwalijk da'k me niet}{eerst voorgesteld heb}\\

\haiku{{\textquoteright} Hij lachte dreunend,.}{en Frank lachte mee maar het}{ging niet van harte}\\

\haiku{k heb helemaal,.}{geen ethisch bloed of zo net zo}{min as principes}\\

\haiku{Waarachtig, kerel, ' '.}{t kan je opt laatst geen}{bal meer verrotten}\\

\haiku{Jij denkt te veel{\textquoteright}, zei, {\textquoteleft}.}{hij wijsgerigen v\'e\'el te}{veel aan anderen}\\

\haiku{En Frank van Wezels.}{hart was donker van weemoed}{en sombere haat}\\

\haiku{En nou leek het hem,.}{een onherbergzaam oord een}{echt verbanningsoord}\\

\haiku{Een eerste en een.}{tweede luitenant stonden}{bij hem te praten}\\

\haiku{Hij sloeg z'n hakken,:}{tegen mekaar richtte zich}{tot de kapitein}\\

\haiku{Jij zoekt een kwartier,.}{voor deze man waar ie een}{kamer alleen heeft}\\

\haiku{{\textquoteright} {\textquoteleft}Jawel, kaptein, 'k{\textquoteright},.}{zal m'n best doen beloofde}{de soldaatschrijver}\\

\haiku{Een man, vrouw en drie.}{kinderen zaten om de}{tafel en aten pap}\\

\haiku{{\textquoteleft}Ik geloof, dat ik ',?}{et niet slecht getroffen heb}{met de kompie wat}\\

\haiku{{\textquoteright} {\textquoteleft}Ja... nog al{\textquoteright}... {\textquoteleft}'k Weet.......}{er alles van uit de krant}{en uit de stukken}\\

\haiku{voor een maffie in,!}{de week onderhou ik je}{hele zwikkie hoor}\\

\haiku{{\textquoteleft}En donder nou m'n,, '!}{bureau af vaandrig en hou}{et je voor gezegd}\\

\haiku{{\textquoteleft}Ik laat me door geen,.}{ene brigges wegboksen al}{heit ie drie baarden}\\

\haiku{In plaats van een cent:}{schoof hij een dubbeltje naar}{het midden en zei}\\

\haiku{Z{\'\i}j zouen zich zo'n.}{buitenkansje niet hebben}{laten ontglippen}\\

\haiku{Maar... eh, majoor... eh...:}{kan ik van middag mijn pas}{niet krijgen alvast}\\

\haiku{Och, majoor, ik kan.}{me eigenlijk overal zo'n}{beetje in schikken}\\

\haiku{Je hebt natuurlijk,.}{gelijk dat je je kiezen}{niet van mekaar doet}\\

\haiku{Daarna zaten ze,.}{tot middernacht bij mekaar}{rookten en praatten}\\

\haiku{{\textquoteright} De stakker met zijn.}{pijnlijke benen keek hem}{vol afgrijzen aan}\\

\haiku{Dat was toch voor het,.}{rijk veel voordeliger en}{voor de man beter}\\

\haiku{{\textquoteleft}Kan je je misschien?}{zo'n geval uit de laatste}{tijd herinneren}\\

\haiku{{\textquoteleft}Na de zware dag.}{van gisteren kunnen ze}{dit niet meemaken}\\

\haiku{Toen ze buiten het,:}{dorp waren en de bossen}{in trokken vroeg Frank}\\

\haiku{{\textquoteleft}Op 't ogenblik is.}{Staak bezig de luitenant}{Dijkstra weg te pesten}\\

\haiku{Zijn hele gezicht}{straalde van voldoening en}{hij had een papier}\\

\haiku{Ze liepen samen:}{de dorpsstraat in en Frank viel}{met de deur in huis}\\

\haiku{Maar de laatste straf{\textquoteright},.}{is drie maanden geleden}{herinnerde Frank}\\

\haiku{{\textquoteright} Even zweeg Frank om z'n.}{antwoord beter tot z'n recht}{te laten komen}\\

\haiku{de mening van de,.}{mensen die zo'n geval in}{de kranten lezen}\\

\haiku{Ik zal er nog es.}{over denken en Toon es bij}{me laten komen}\\

\haiku{Frank was benieuwd, of.}{de boer hun niet een hapje}{zou laten mee eten}\\

\haiku{Mensen, die gozer '.}{is zo vies bij int open}{snijen van koppen}\\

\haiku{Frank was juist op het,.}{kompagniesbureau toen Jan}{binnengebracht werd}\\

\haiku{{\textquoteleft}Die hebben ze van,.}{me gestolen de vuile}{keleiradieven}\\

\haiku{En hij vergat om.}{te keren en is sedert}{dien nooit meer gezien}\\

\haiku{Maar een paar weken.}{geleden was ie tegen}{de lamp gelopen}\\

\haiku{Op de binnenplaats.}{waren de mannen bezig}{piepers te jassen}\\

\haiku{\'als dat tenminste...}{het eind was en hij nog niet}{tot erger verviel}\\

\haiku{{\textquoteleft}Ik neem \'o\'ok niet meer,.}{verantwoordelijkheid dan}{nodig is majoor}\\

\haiku{Er werd zes weken.}{militaire hechtenis}{tegen hem ge\"eist}\\

\haiku{De man was getrouwd,.}{had drie kinderen en een}{ziekelijke vrouw}\\

\haiku{dat ze behoorden,...}{bij het leven van eeuwen}{die voorbij waren}\\

\haiku{Dan kon er toch ook?}{geen sprake zijn van een zwaar}{gevecht in Limburg}\\

\haiku{{\textquoteright} {\textquoteleft}In zeker opzicht...{\textquoteright},.}{misschien gaf de luitenant}{met tegenzin toe}\\

\haiku{{\textquoteright} {\textquoteleft}Mag ik misschien ook,,?}{zien meneerde president}{welke twee dat zijn}\\

\haiku{Maar ze waren hier.}{op neutraal terrein en hij}{moest de pil slikken}\\

\haiku{De opleiding was.}{op het terrein gekomen}{en stond te rusten}\\

\haiku{{\textquoteleft}Maar wat heb je dan?}{te zeggen op wat ik naar}{voren gebracht heb}\\

\haiku{Ze moesten er dus maar.}{blijven en deelnemen aan}{de verdediging}\\

\haiku{toen wendde hij zich:}{tot de hospitaalsoldaat}{en zei lakoniek}\\

\haiku{En dat onder de!}{strenge kritische blikken}{van de kolonel}\\

\haiku{{\textquoteright} Franks gezicht werd \'e\'en.}{grote demonstratie van}{schrik en afgrijzen}\\

\haiku{{\textquoteleft}En nou gaan we es{\textquoteright},.}{een ronde over de kamer}{doen grinnikte Frank}\\

\haiku{Dat noemt ie rustig,!}{as-t-ie je pas op de}{bon geslingerd heit}\\

\haiku{{\textquoteright} vroeg de sergeant.}{van de week met een volmaakt}{onschuldig gezicht}\\

\haiku{Maar de wereld is.}{nou eenmaal ondankbaar en}{niet vlug van begrip}\\

\haiku{Nou kwam het er op.}{aan een smoes te vinden om}{verlof te krijgen}\\

\haiku{Maar al luidden de,.}{berichten ongunstig het}{doodsbericht bleef uit}\\

\haiku{{\textquoteright} {\textquoteleft}Maar je zult wel een{\textquoteright},.}{fatsoenlijk pensioentje}{krijgen troostte Frank}\\

\haiku{Maar de vent hield z'n.}{poot scheef en schinkelde als}{een kreupele rond}\\

\haiku{{\textquoteright}... {\textquoteleft}Laat ie de pokke!}{krijge met ze sjokkela}{en z'n sigaren}\\

\haiku{En begrepen niet,.}{eens hoe akelig doorzichtig}{hun spelletje was}\\

\haiku{H\'et moment was het.}{ontvangen van het werkpak}{en de kwartiermuts}\\

\haiku{dubbele gouden.}{streep en gouden kroon op de}{linkerbovenarm}\\

\haiku{De hitte was om,,,.}{ons overal overal en het}{was onverdraaglijk}\\

\haiku{geef me wat terug,.}{voor m'n goeie zorgen en ze}{maakt je officier}\\

\haiku{En toch - toch waren.}{we nauwelijks een maand lang}{bij elkaar geweest}\\

\haiku{Het was ergens op,,.}{een groot zandig terrein waar}{men aan athletiek doet}\\

\haiku{Ik heb nog nooit een.}{pijp zo wetenschappelijk}{langzaam zien stoppen}\\

\haiku{Ik zal dromen, dat.}{ik de kunst versta ook zo}{de lijn te trekken}\\

\haiku{'k Had de moord in,?}{en ik verrekte van de}{pijn maar wat doe je}\\

\haiku{Daar zou ik staaltjes,.}{van kunnen openbaren die}{opzien verwekten}\\

\haiku{Alles gaat vrij wel,.}{je ser vet zelfs valt nog al}{zo onhandig niet}\\

\haiku{Maar in eens heb je,.}{een stuk aardappel in je}{mond dat je niet smaakt}\\

\haiku{De schijtlijster het '.}{et lef niet om d'r mee voor}{de draad te komme}\\

\haiku{Daarmee waren \'en',?}{post \'en luit naar de weerga}{zou je zo zeggen}\\

\haiku{De nood was op 't,!}{hoogst gestegen maar nu was}{ook redding nabij}\\

\haiku{En toen verklaarde'.}{de luit dat zijn afdeling}{het gewonnen had}\\

\haiku{{\textquoteright} En als bij afspraak:}{stijgt uit de gelederen}{het gezang op}\\

\haiku{Het spijt me, dat dit,.}{tegen de krijgstucht is maar}{de levenstucht eist het}\\

\haiku{'t Ziet er nogal.}{vies uit en helemaal niet}{begerenswaardig}\\

\haiku{de eerste aanloop.}{tot de onthulling van m'n}{griezelig geheim}\\

\haiku{{\textquoteleft}Nee{\textquoteright}, zei hij koppig, {\textquoteleft},,,:}{nee k'praal daar m\`o je nou niet}{van praten ik zeg}\\

\haiku{Dientengevolge {\textquoteleft}{\textquoteright},.}{klaagt hij nooitover het eten of}{wat daarvoor doorgaat}\\

\haiku{is 's avonds terug '.}{gekomen van verlof in}{plaats vans middags}\\

\haiku{Hoe hij plotseling,:}{was gereduceerd tot nul}{tot minder dan nul}\\

\haiku{Dus spreekt het vanzelf,.}{dat je dagen te voren}{je komst aankondigt}\\

\haiku{Vaandrig zouden ze,.}{worden dat is zoiets als}{leerling-luitenant}\\

\haiku{Daar was er een bij,,.}{die woonde als burger dan}{vroeger in Den Haag}\\

\haiku{(Bij een dikte van!)}{1.3 cm namelijk kan de}{oorlog niet doorgaan}\\

\haiku{dat weet ik wel, maar,!}{wat duivel dan moeten ze}{maar niet gebeuren}\\

\haiku{Nou zullen ze hard...}{gaan werken en met plezier}{officier worden}\\

\haiku{De pijn lag op  .}{zijn verwrongen gezicht en}{hij steunde hoorbaar}\\

\haiku{Ik ontsloot de deur,.}{en kwam in een gang waar het}{nog heel donker was}\\

\haiku{De korporaal van.}{de week haalde de dekens}{en strozakken weg}\\

\haiku{En dat straf is om,.}{te verbeteren niet om}{te verbitteren}\\

\haiku{De kommandant van:}{de tegenpartij wenkt naar}{de onze en roept}\\

\haiku{Maar na een minuut.}{of wat komt de lenigheid}{een beetje terug}\\

\haiku{om mooie cijfertjes,.}{te halen want dat wil ik}{van jou ook altijd}\\

\haiku{Hier en daar, eenzaam,.}{ligt een stukje vlees in een}{wasblik te dromen}\\

\section{Max de Jong en Hans van Straten}

\subsection{Uit: Ik ben een echt genie. De briefwisseling van Max de Jong en Hans van Straten 1942-1951}

\haiku{Naar die juffrouw in ().}{Den Haag2 ga ik nietwillen}{ze thuis ook al niet}\\

\haiku{Misschien kan dit als.}{het af is een klein boekje}{op zichzelf worden}\\

\haiku{Ik zou je wel een,}{ex. willen sturen maar jij}{leest toch geen proza?35}\\

\haiku{een opeenhoping (...,...,... ).}{van korte bijzinnetjes}{wanneer als wanneer}\\

\haiku{Theosophie,?}{in zijn geheel moet dat niet}{zijn in haar geheel}\\

\haiku{Uit de la van een.}{kast kwam een schets tevoorschijn}{van een toneelstuk}\\

\haiku{Het is trouwens niet.}{eens nodig dit partijtje}{nog uit te spelen}\\

\haiku{Ik dacht natuurlijk, '.}{ook dat het niks was omdat}{hij int Fries schrijft}\\

\haiku{Ik stel Lehmann hoog,,.}{dat weet je maar Wadman Staat}{nu minstens even hoog}\\

\haiku{redactie heb ik!}{niet zoveel te vertellen}{als de vorige}\\

\haiku{{\textquoteleft}Wanneer het denken,.}{rondtolt en het hart breekt roept}{de mens om een god}\\

\haiku{Dat Neeltje je Heet.}{van de Naald slecht vond bewijst}{haar nuchter oordeel}\\

\haiku{D\`er Mouw moet erbij,.}{maar hij publiceerde niet}{in De Beweging}\\

\haiku{Zijn smaak is van een.}{hoogst boekenkasterige}{po\"ezie\"erigheid}\\

\haiku{We kunnen hier wel,.}{enkele concessies doen}{maar toch niet teveel}\\

\haiku{Engelman, Jet Holst,,.}{Campert en Van der Graft meer}{kunnen er niet bij}\\

\haiku{Zes uur per dag, en.}{als je ingewerkt bent kun}{je ook thuis werken}\\

\haiku{We zijn nu toch weer,.}{met De Neve bezig maar}{die wil pas in Sept}\\

\haiku{Hans, Ik d\'enk er niet, -.}{aan om over de kop te slaan}{maar ik z\'al meedoen}\\

\haiku{In de bl\'oemlezing,.}{moet Du Perron vanzelf w\'el}{dat kan jij dan doen}\\

\haiku{Adriaan van der Veen,,.}{heeft opgemerkt dat ik op}{Jezus lijk jasses}\\

\haiku{Ook met het oog op!}{de uitgave ervan is}{dat noodzakelijk}\\

\haiku{Hoe doen we nou met,?}{die bloemlezing maken we}{die nou af of niet}\\

\haiku{Overigens heb je.}{nog aardig wat boeken staan}{om te verpatsen}\\

\haiku{Zelfs alles-lezers.}{als Vermeulen en Schwithal}{lezen dat niet meer}\\

\haiku{Max        Amsterdam, [].}{ca. 13 januari 1948}{briefkaart Beste Hans}\\

\haiku{De anderen gaan,.}{hun gang daarom wil het ACC}{ook eens zijn gang gaan}\\

\haiku{van Roger Martin,.}{du Gard de geschiedenis}{van een bekering}\\

\haiku{Vroeger leek zoiets,.}{nog wel wat toen gebruikten}{ze er nieuw hout voor}\\

\haiku{Zeg nou zelf, moet je}{om van dat soort dingen zo}{lekker te worden}\\

\haiku{Documenteer je,?}{evenwel even of hoe maken}{we dat anders uit}\\

\haiku{Het bewijst weer eens, ().}{dat goed schrijven een kwestie}{van geldis tijd is}\\

\haiku{Daarbij schijnt de NVSH.}{jaarlijks een behoorlijke}{duit over te houden}\\

\haiku{Hoe staat het met die?}{fusie van Podium en}{Libertinage}\\

\haiku{Je moet eens op die.}{nieuwe Soc. Bew. letten van}{Frits Kief en Jef Last}\\

\haiku{1949 [briefkaart] Beste,.}{Max ~ Dinsdagmiddag kom}{ik naar Amsterdam}\\

\haiku{al dat verwerpen}{van waar je je nog niet mee}{bezig gehouden}\\

\haiku{de Jong en Rudie.}{van Lier worden er alleen}{belangrijker door}\\

\haiku{Het is maar een klein,.}{boekje en je kunt overslaan}{wat je niet bevalt}\\

\haiku{Ik denk maar dat ik,.}{fascist word dat komt straks weer}{erg in de mode}\\

\haiku{Schrijf me even hoe of,.}{wat dan weet ik weer waar ik}{beginnen moet}\\

\haiku{Dat je Liefde en.}{Goudvissen wilt gaan lezen}{is verdienstelijk}\\

\haiku{{\textquoteleft}Ik bracht de bloemen,.}{aan een andere vrouw die}{ik ook wel lief vond}\\

\haiku{10, 115, 117, 118, 141,,,,,, (-):}{145 153 168 169 171 Reydon}{Hermannus18961943}\\

\haiku{29[onder de brief heeft]:}{MdJ geschreven Rodenko}{stottert en is links}\\

\haiku{89Zie de noten.}{bij de brief MdJ van ca. 18}{januari 1947}\\

\section{Pieter Joossen}

\subsection{Uit: De kroniek van Pieter Joossen Altijt Recht Hout}

\haiku{Voort ist water van,}{de Nieuhaven geloopen}{duer de strate}\\

\haiku{En tot dien eynde}{hadde hy gesonden twee}{vaendelen Waelen}\\

\section{Sjouke Joustra}

\subsection{Uit: Vertrouw nooit een zeeloods}

\haiku{{\textquoteright} Op korte afstand.}{van Scheltema stopte hij}{en draaide zich om}\\

\haiku{{\textquoteright} De genoegzame.}{trek verdween van het gezicht}{van de Germaan}\\

\haiku{De kapitein liep.}{echter op de wandkast toe}{en deed de deur open}\\

\haiku{Hier de sleepboot voor,,?}{de Mitsu Maru wat is}{de bedoeling loods}\\

\haiku{We zwaaien zeker?}{met de kop in de richting}{van Cittershaven}\\

\haiku{De vierde sleepboot,{\textquoteright}.}{maakt op de haven vast klonk}{door de marifoon}\\

\haiku{Kunnen jullie bij?}{het loodswezen nu niet eens}{een planning maken}\\

\haiku{naar wat later de.}{invasie van Normandi\"e}{zou blijken te zijn}\\

\haiku{Frans Naerebout werd.}{op 30 augustus 1748 in}{Veere geboren}\\

\haiku{Naerebout geniet.}{slechts korte tijd van deze}{late waardering}\\

\haiku{Rondom het schip lag.}{niets anders dan een grauwe}{wollige massa}\\

\haiku{{\textquoteleft}Dat kan niet, of er,{\textquoteright}.}{moet nog volk beneden zijn}{mompelde Jacob}\\

\haiku{wat moeten wij doen,?}{nu de anderen het schip}{verlaten hebben}\\

\haiku{Doelbewust hield het.}{patrouillevaartuig op de}{Scheldemond aan}\\

\haiku{Vlagerige wind,,.}{uit het noordwesten kracht 8}{Bft zee onstuimig}\\

\haiku{{\textquotedblleft}Rivier,{\textquotedblright} antwoordde.}{ik schuchter en beledigd}{om het woord leerling}\\

\haiku{Het komt niet meer voor:}{dat men van jongsaf aan met het}{besef heeft geleefd}\\

\haiku{Het was stikdonker.}{en zware zee\"en beukten}{over beide schepen}\\

\haiku{Tegen de ochtend.}{ruimde de wind en ook werd}{de zee handzamer}\\

\haiku{{\textquoteleft}Veel schepen zullen,,.}{hier niet varen stuurman maar}{je kunt nooit weten}\\

\haiku{Volgens mij is zijn.}{positie nog ten minste}{een half uur te gaan}\\

\haiku{{\textquoteright} Het was de langste.}{zin die Jacob hem ooit had}{horen uitspreken}\\

\haiku{Jan besloot zich in.}{te laten schrijven voor de}{cursus eerste rang}\\

\haiku{Natuurlijk lagen.}{in de oorlog de lintjes}{voor het oprapen}\\

\haiku{It is time to}{go now Haul away your anchor}{Haul away your anchor}\\

\haiku{Ik kreeg orders om.}{honderd ton aardappels in}{Hansweert te laden}\\

\haiku{Hij was er zeker.}{van dat het de Congoboot}{voor hem zou worden}\\

\haiku{Hij knoopte zijn jas.}{dicht en verdween zonder groet}{naar de Roeierswacht}\\

\haiku{Nog voor de week ten.}{einde was bleek de staking}{verlopen te zijn}\\

\haiku{De rozenkrans was.}{nu van de ochtendjas naar}{zijn broekzak verhuisd}\\

\haiku{Langzaam maar zeker.}{naderde de sleep nu de}{roergang tot de sluis}\\

\haiku{Yes I told you you,.}{are a very good pilot an}{excellent pilot}\\

\section{Tjibbe Joustra}

\subsection{Uit: Superpop}

\haiku{Zo langzamerhand.}{zijn er vandaag alleen maar}{andere kanten}\\

\haiku{Mijn moeder zegt dat.}{ze water genoeg gezien}{heeft van de zomer}\\

\haiku{Wij staken de straat.}{over om ons bij de lui naast}{de stok te voegen}\\

\haiku{Ze springt erin als.}{ik en komt met een vlugge}{zwemslag op me af}\\

\haiku{Ze proberen over,.}{hun schaduwen te springen}{wat niet lukken wil}\\

\haiku{We komen langs een.}{tot monument verklaarde}{openbare belcel}\\

\haiku{We pakken vlug zijn.}{zak en goed op en zwaaien}{er uitbundig mee}\\

\haiku{We troffen het wel.}{en trokken ons terug in}{een kleine ruimte}\\

\haiku{Wanneer ze langs de.}{wagen van de sr komt is}{viervoet verdwenen}\\

\haiku{Het verkeer golft in.}{een onafgebroken stroom}{langs het drietal heen}\\

\haiku{Ik geef hem een waai,.}{voor zijn kop wil hem een waai}{voor zijn kop geven}\\

\haiku{Mar en ik uit het.}{vorige moment over naar}{dit nieuwe ogenblik}\\

\haiku{Rol buigt zich over de.}{rand van de kade waar een}{vlot tegenaan drijft}\\

\haiku{Kris rammelt met het.}{opgepakte in haar hand.}{Gooi dan als je durft}\\

\haiku{De wagen is paars,.}{de overal van de jongen}{die er uitstapt ook}\\

\haiku{Mar gaat eens kijken,.}{terwijl de jongen de deur}{voor haar openhoudt}\\

\haiku{) Dank je wel, zeg ik,.}{wij zijn hier niet gekomen}{om weer weg te gaan}\\

\haiku{En zo pakt hij de.}{doos op en weet de deur van}{de zaak te vinden}\\

\haiku{En ja, voor niets gaat.}{de zon hier zeker als een}{jojo op en neer}\\

\haiku{Hoe uitgestrekt kan.}{een plek zijn om de plaats heen}{waar je naar toe wilt}\\

\haiku{Uitkijken naar wat.}{ons verder brengt zal hier niet}{makkelijk worden}\\

\haiku{Mar probeert ondanks.}{de zichtbeperking om ons}{heen te kijken}\\

\haiku{Een wegglijdende.}{spiegelende schim aan de}{rand van gedachten}\\

\haiku{Nou dank je hoor, wat,.}{ben jij knap zeg zeggen we}{tegen het meisje}\\

\haiku{Ontzettend aardig,,,.}{zeg maar we moeten weg zie}{je we zijn al laat}\\

\haiku{De fig klikt om de,}{hoek van de deurpost het licht}{te voorschijn onthult}\\

\haiku{Mar, wil ik zeggen,,.}{draai mij om maar ook Mar is}{nergens meer te zien}\\

\haiku{Ik ontdek dat de.}{mist opgelost is en de}{avond nacht geworden}\\

\haiku{Klauwt haar vingers, grist.}{met een razendsnelle graai}{de maan uit de lucht}\\

\haiku{Dichte nevel hangt.}{boven de vlakte en er}{is niets meer te zien}\\

\haiku{Doordat onze vriend,.}{ophoudt met trappen want de}{weg gaat hier omhoog}\\

\haiku{Mar slaat een arm om,.}{mijn schouders ik sla een arm}{om de jongen heen}\\

\haiku{Onderwijl kunnen ().}{enige zaken onsgeestes}{oog gaan passeren}\\

\haiku{Mar fluistert in mijn,.}{oor ik kijk naar mijn glas wat}{inderdaad leeg is}\\

\haiku{Voordat het meisje,,.}{op de laadvloer klautert kijkt}{ze even opzij lacht}\\

\haiku{De laatste spitst zijn.}{lippen als hij denkt dat het}{meisje het niet ziet}\\

\haiku{Das gooit dat wat hij.}{gevonden heeft weg ergens}{tussen de struiken}\\

\haiku{Mar weet haar kamer,.}{te vinden het slot op de}{tast open te krijgen}\\

\haiku{Ze drukt zich langzaam.}{omhoog tegen de gladde}{groenbemoste schors}\\

\haiku{In een tel zijn de.}{takken op dit deel van het}{grasveld opgeruimd}\\

\haiku{Leeg, op de schipster,.}{na en zonder rechts zie je}{niet wat je links ziet}\\

\haiku{De deining verzwakt,.}{tot een nauwelijks merkbaar}{dobberen tot niets}\\

\haiku{De jongen draait zich,.}{om gooit de kaart terug in}{de bak van de fiets}\\

\haiku{Misschien een kans om.}{van een ogenblik een ander}{moment te maken}\\

\haiku{Verwonderd kijkt de,.}{fietser om zich heen wie weet}{dat hij op straat zit}\\

\haiku{Kijk nog eens goed, het.}{is duidelijk dat dit daar}{heel iets anders is}\\

\haiku{Opgewonden staan.}{we met ons vijven op het}{dak om het blik heen}\\

\haiku{In de ramen aan.}{mijn kant wordt de wereld aan}{mijn kant weerspiegeld}\\

\haiku{Neuri\"end loop ik,.}{over mijn dak met mijn voeten}{spullen verschuivend}\\

\haiku{wat me geschikt lijkt,.}{om iets in neer te leggen}{in op te bergen}\\

\haiku{Met een klap die over.}{het water weg stuitert slaat}{daarop het raam dicht}\\

\haiku{Ik gris rondom me,.}{heen allerlei rommel van}{het dek ga gooien}\\

\haiku{Ik zou in lachen.}{willen uitbarsten maar laat}{dit achterwege}\\

\haiku{Ik zwaai met de top,.}{in mijn hand heen en weer wat}{zal ik besluiten}\\

\haiku{Hee, hoor ik Jansen,.}{naast mij zeggen ga je iets}{voor ons opvoeren}\\

\haiku{Nee, Jonna roert met.}{de staaf van ijzer in de}{inhoud van het blik}\\

\haiku{Doggo kom op, steek.}{je struisvogelpolitiek}{eens door dit open gat}\\

\haiku{Maar nee hoor, het komt,.}{wel dieper te liggen maar}{blijft keurig drijven}\\

\haiku{het minder moeilijk,,.}{springen van het dak op het}{dek elk aan een kant}\\

\haiku{Ik, Doggo, strijk met.}{mijn hand door mijn haar of over}{een stugge borstel}\\

\haiku{En de handmeester,.}{die ik er uithaal papier}{is geduldig}\\

\haiku{Ik ontdoe mij van,.}{mijn meesteres stop deze}{terug in de tas}\\

\haiku{Ik kijk nog net op.}{tijd op om te zien dat het}{op mij gericht was}\\

\haiku{Ratta staart naar het,.}{raakvlak van zijn voeten met}{de straat ik staar mee}\\

\haiku{Bijvoorbeeld in een.}{verhaal dat bijna niet na}{te vertellen is}\\

\haiku{Ik houd mijn glas op.}{ooghoogte en kijk er door}{naar een stad in glas}\\

\haiku{Achter mij zegt een.}{reiziger in vertaalde}{taal goed meisje}\\

\haiku{Mijn meisje pakt de,.}{bril het meisje zet mijn bril}{op onder bijval}\\

\haiku{Op de rand van de.}{kade staat een deinende}{figurantenbrij}\\

\haiku{Ik doe of ik zweet,,}{het zweet van mijn voorhoofd veeg}{verschuif met mijn voet}\\

\haiku{Doggo, luister, dat,}{is om de vaargeul uit te}{diepen hoor je dat.}\\

\haiku{De boot drijft traag rond,.}{onder een wolk van woorden}{tot  ze dwars ligt}\\

\haiku{Een verklaring brengt.}{je trouwens zelden dichter}{bij de oplossing}\\

\haiku{Met een boog gooi ik,,.}{de toeter over mijn schouder}{weg de gracht in}\\

\haiku{Jonna probeert de.}{opengesprongen parasol}{weer dicht te vouwen}\\

\haiku{Ik steek over nadat,.}{alles voorbijgegaan is}{en beklim de brug}\\

\haiku{Hoe komt het toch dat.}{sommige dingen soms wel}{breken en soms niet}\\

\haiku{Weg er mee, dit  ,.}{maakt mij dorstig of het doet}{mij mijn dorst voelen}\\

\haiku{Ik veer overeind, doe,.}{mijn ogen open knipper tegen}{het felle zonlicht}\\

\haiku{Hij zet het achter.}{de parasol en de palm}{rechtop op het strand}\\

\haiku{In mijn ongeduld.}{duw ik hem bijna voor een}{langs rollende tram}\\

\haiku{Links naast mij ligt een,.}{gitaar op de straatstenen}{rechts een racefiets}\\

\haiku{Iemand loopt met haar.}{reiszak tegen de gitaar}{in mijn linkerhand}\\

\haiku{Vind ik het niet vreemd,.}{dat ik hem had en dat hij}{paste de sleutel}\\

\haiku{Zo'n gitaar op je.}{rug heeft eigenlijk toch wel}{iets compleets vind ik}\\

\haiku{Ze klinken na in,,.}{de gitaar net als de bel}{van de tram exact zo}\\

\haiku{Bereik het punt van.}{de straat waar ik deze zal}{moeten oversteken}\\

\haiku{Ik draai mij om om,.}{ergens naar te gaan zoeken}{wat weet ik nog niet}\\

\haiku{Met schorre keel schreeuw.}{ik de tonen van deze}{improvisatie}\\

\haiku{hoe zit het met de,.}{uitgestrektheid strekt het mee}{of strekt het tegen}\\

\haiku{Een eind voor mij uit.}{is iets te zien wat opeens}{weer verdwenen is}\\

\haiku{Mijn lippen sluiten,.}{zich gespannen kijk ik naar}{een luchtspiegeling}\\

\haiku{Ik knijp mijn ogen dicht.}{om beter naar Majjo}{te kunnen kijken}\\

\haiku{ja, daar, ja die kant,,}{zie je dat meisje zonder}{die gitaarkoffer}\\

\haiku{De welving van haar,.}{buik verlokkend priemen van}{vuurrode tepels}\\

\haiku{Een eind verder loop.}{ik likkend aan in de zon}{en uit de drukte}\\

\haiku{Voordat het beeld met.}{hem mee rent probeert ze aan}{mijn staart te trekken}\\

\haiku{Maak een vaag gebaar.}{terug dat ergens bij het}{begin blijft steken}\\

\haiku{Nou, nee, laat maar, maar.}{heb je toevallig geen kist}{om op te zitten}\\

\haiku{ik heb er trouwens,,.}{alleen maar even op gestaan}{niks gepoetst plaatstaal}\\

\haiku{Ik krabbel overeind,.}{laat een wind en vind een plek}{om op te schijten}\\

\haiku{Ik rek mij uit en.}{voel met mijn handen aan de}{palmboom op mijn hoofd}\\

\haiku{Okee goed jij wint,  ,.}{brengt hij moeizaam uit maar mag}{ik nog even voelen}\\

\haiku{Oh schone met je,.}{palm je zoete lippen en}{je volle tieten}\\

\haiku{Vanuit het water,,}{gespartel en geproest schijt}{stik verdomme blub}\\

\haiku{Onderzoekend kijkt,,.}{ze nog steeds in gebukte}{houding om zich heen}\\

\haiku{Wat jij daar hebt is,}{geen voorwerp maar alleen maar}{iets wat er op lijkt}\\

\haiku{De warme, hijgend.}{uitgeblazen voelbare}{adem van Fatala}\\

\haiku{Fanata met je.}{afdalende wielen en}{je zekere zit}\\

\haiku{De weg vrijgemaakt,.}{voor als de drie meisjes nog}{terugmoeten straks}\\

\haiku{Van de hellingkant.}{van de brug komen nu twee}{meisjes aanfietsen}\\

\haiku{Trams, auto's, fietsers.}{en voetgangers trekken op}{naar de binnenstad}\\

\haiku{Totala grinnikt,.}{Fatala beziet de stad}{en de lucht zo blauw}\\

\haiku{Totala ziet het.}{weerspiegeld in de glazen}{van haar zonnebril}\\

\haiku{Uit het portiek waar.}{ze nu langslopen komt een}{meisje naar buiten}\\

\haiku{Nou, ik ga naar Mar,.}{onze eigen avonturen}{beleven weet je}\\

\haiku{Ja nee, een goed plan,,,.}{hoor maar eh ik heb ook een}{idee zegt Labarbra}\\

\haiku{Weet je dat ik het,.}{al een keer uitgeprobeerd}{heb zegt Labarbra}\\

\haiku{Een van de wielen.}{blijft steken in de railgoot}{van een kadekraan}\\

\haiku{Labarbra duwt haar.}{handen in de zakken van}{haar fantasiejurk}\\

\haiku{Meisje rijdt rondjes,,.}{straatorkest aan de overkant}{van de straat auto's}\\

\haiku{Ze haalt het mes licht,.}{drukkend even heen en weer trekt}{het dan iets terug}\\

\haiku{Volg mij, want wij gaan,.}{opstijgen roept hij tegen}{het stadslawaai in}\\

\haiku{Wat kan hij in zijn.}{schild voeren onder de rest}{van de sluier}\\

\haiku{Weet je, wanneer ze,.}{maar vroeg genoeg met vliegen}{beginnen wie weet}\\

\haiku{Onnono tikt de.}{vermaker van zich af die}{achteruit tuimelt}\\

\haiku{Jammer alleen voor,.}{haar dat het water leeg}{is zegt Fatala}\\

\haiku{De vraag is nu, denkt,:}{Fanata welke boodschap}{hier het best overkomt}\\

\haiku{Gek dat Totala.}{al zo ver is dat ik hem}{nergens meer zien kan}\\

\haiku{Bij het uitspreken.}{van deze woorden heeft zij}{haar ogen gesloten}\\

\haiku{Hij wil haar op de.}{zich dwaas terugtrekkende}{figuren wijzen}\\

\haiku{Fatala wil naar,.}{het heft van haar mes grijpen}{maar ziet er vanaf}\\

\haiku{je boft nog dat je.}{hier niet net zo vaak staat als}{dat ik hier langskom}\\

\haiku{Wacht, onderbreek me,.}{niet er staat nog meer op gaat}{Totala verder}\\

\haiku{Onnono barst in,.}{schateren uit verslikt zich}{in een hap gras}\\

\haiku{Animee geeft hem een,.}{duw zodat hij in de berm}{zal terechtkomen}\\

\haiku{Animee wijst met haar.}{vinger grillige wegen}{over het kaartoppervlak}\\

\haiku{Het meisje buigt zich,.}{nog wat verder naar buiten}{lacht naar beneden}\\

\haiku{Het ding zeilt een eind,.}{door de lucht komt ergens net}{buiten beeld terecht}\\

\haiku{Animee bukt zich om.}{haar op de grond liggende}{hamer te pakken}\\

\haiku{Ze houdt haar adem in,.}{en luistert hoort Fatala}{een stem fluisteren}\\

\haiku{Vanuit de richting.}{van de stad komt niets naar hun}{eiland toe varen}\\

\haiku{Plotseling blijft ze,.}{roerloos liggen laat haar hoofd}{op de grond rusten}\\

\haiku{Fatala springt op,,.}{laat haar ogen over haar lichaam}{over het eiland gaan}\\

\haiku{Hee Onnono, een,.}{nieuw gezichtspunt roept hij naar}{de uitzichteter}\\

\haiku{zijn schouders op, maakt,.}{een gebaar of hij er iets}{overheen gooit wegwerpt}\\

\haiku{Logisch, want als het.}{andersom zou zijn zou ik}{andersom zitten}\\

\haiku{Hoe ziet vanaf de.}{overkant iemand er uit die}{naar de overkant kijkt}\\

\haiku{Fanata doet een,.}{stap naar voren schopt er met}{haar voet tegenaan}\\

\haiku{scandeert Onnono,.}{zijn handen als een roeper}{om zijn mond houdend}\\

\haiku{En, zegt Fanata,.}{wil je je liefdesverdriet}{soms gaan begraven}\\

\haiku{Juist, knikt Onnono.}{en probeert gevaarlijk te}{doen op de dakpunt}\\

\haiku{Onnono schudt zijn,,.}{hoofd flarden tekst wolken op}{dwarrelen omlaag}\\

\haiku{zich heenkijkend of.}{er iets is om zich koelte}{mee toe te wuiven}\\

\haiku{Totala kijkt  .}{naar Fatala's gezicht van}{onderen gezien}\\

\haiku{Zij knippert met haar.}{ogen tegen het felle licht}{van de nabije zon}\\

\haiku{De ontmoeting, denkt,.}{Animee evaluatie voor}{meer dan een persoon}\\

\haiku{Aanval vanuit de,,.}{lege ruimten de lege}{tijd denkt Fatala}\\

\haiku{Bloed, denkt Animee, ik.}{zou natuurlijk ook wat voor}{haar kunnen pissen}\\

\haiku{Het licht beweegt in.}{haar ogen mee op het ritme}{van haar benen}\\

\haiku{De meisjes raken.}{elkaar tegelijk met hun}{vingertoppen aan}\\

\haiku{Fatala loopt om.}{het pak stenen heen naar de}{kant van het water}\\

\haiku{Haar haren vegen.}{bij het bewegen van haar}{hoofd over de vloer stof}\\

\haiku{De jongen houdt met.}{een aangedikt gebaar zijn}{hand achter zijn oor}\\

\haiku{Zonder dat hij op.}{het idee is gekomen naar}{boven te kijken}\\

\haiku{De samenloop van,.}{uitgesproken gedachten}{zegt Labarbra's stem}\\

\haiku{Daarachter de straat,.}{vol auto's fietsen en trams}{die het plein verdeelt}\\

\haiku{Onnono legt zijn,.}{handen in zijn nek spant de}{spieren van zijn borst}\\

\haiku{Samen tikken zij.}{een ingewikkeld ritme}{op de straatstenen}\\

\haiku{Zonder grip op dat,.}{wat gebeurd is alles wat}{nog moet gebeuren}\\

\haiku{Omdat ik wel zie,.}{dat je haast hebt hoor zegt hij}{tegen Fanata}\\

\haiku{Sorry maar weet jij.}{misschien de weg naar boven}{of naar beneden}\\

\haiku{dat men heeft ziet er.}{alleen daarom zo al een}{stuk draaglijker uit}\\

\haiku{Allicht ja, logisch,.}{dat je hier naar beneden}{kijkt waar anders naar}\\

\haiku{Wiens hamer's steel om.}{alles in het straatzand op}{te tekenen}\\

\haiku{voorbijgaat zonder,.}{daartoe uitgenodigd te}{zijn gaat hij verder}\\

\haiku{Wie zet zo'n ding ook.}{op de gang zonder er zelf}{naast te gaan liggen}\\

\haiku{Zij weet nog net te.}{voorkomen dat de klanken}{haar mond verlaten}\\

\haiku{Labarbra gaat naast.}{Animee voor het grachtkant raam}{van het caf\'e staan}\\

\haiku{Toen zij fluisterde.}{maar jij mijn eindeloze}{tong van lik en taal}\\

\haiku{Wat die zijn dus, die,.}{inwoners in het genot}{van alle rechten}\\

\haiku{Wie of wat bepaalt.}{het groeien van de ruimte}{tussen de muren}\\

\haiku{Langs de randen loopt.}{een galerij die aan de}{buitenkant open is}\\

\haiku{His feet sinking away,.}{in the burning hot loose sand}{Yessimo toils on}\\

\haiku{He stoops to pick up,.}{the book but it lies a few}{meters further on}\\

\haiku{The question is,,}{Yessimo says as if lost}{in thought where're}\\

\chapter[19 auteurs, 1415 haiku's]{negentien auteurs, veertienhonderdvijftien haiku's}

\section{Mathias Kemp}

\subsection{Uit: De felle novene}

\haiku{Pastoor Grompers was,.}{een beste kerel maar wat}{van de oude school}\\

\haiku{- Hij bedoelt het zoo,.}{kwaad niet haastte zich pater}{Herman te sussen}\\

\haiku{Ja, dat kon Grompers,.}{niet ontkennen doch hij bleef}{niettemin critisch}\\

\haiku{wij studeeren na het.}{seminarie in biechtstoel}{en bij huisbezoek}\\

\haiku{Het voorbeeld van Sint,,.}{Paulus die eenmaal Saulus}{was voor oogen houden}\\

\haiku{Eigenlijk gold hij.}{zoowat als het bedorven kind}{der communauteit}\\

\haiku{liet Ernestine,.}{Woltmakers die toevallig}{met ritmeester Jhr}\\

\haiku{Voorop het bestuur,;}{der Broederschap met zijn nieuw}{processievaandel}\\

\haiku{Het werd een dreigend.}{opdringen rondom Debrit}{en zijn trawanten}\\

\haiku{Waarom studeeren, zich?}{blootstellen aan proefjaren}{en tegenslagen}\\

\haiku{Je leeft nu in een.}{sfeer van heiligheid die ik}{je benijden moet}\\

\haiku{haar laatste wenschen,,.}{zijn mij nu ik alles weet}{heilig geworden}\\

\haiku{- Wat mij betreft, kunt,;}{U het tot morgen laten}{meende P. Thomas}\\

\haiku{{\textquoteright} MEER en meer raakte.}{Torenen in beroering}{door de Novene}\\

\haiku{Waarom P. Herman?}{baron Richelle met een}{bezoek lastig viel}\\

\haiku{hard zou loopen, ze.}{is namelijk een zuster}{van graaf Lahnenstein}\\

\haiku{Ik tusschen elf en - -.}{twaalf het uur van ebbe aan}{den biechtstoel er heen}\\

\haiku{Ze zond ons uit om,.}{te bedelen anders kan}{ik het niet noemen}\\

\haiku{Ik telde voor oud,!}{vuil en ik was toch een frisch}{en aardig meisje}\\

\haiku{Toen onze oudste,,.}{kwam Lowieke werd het een}{paar maanden beter}\\

\haiku{- Vertel nu eens wat...... ',.}{van jezelf en neem eensn}{snuifje dat doet goed}\\

\haiku{- Menschen van uw slag,.}{zien we hier maar zelden viel}{Nelia spinnig uit}\\

\haiku{Men ging ook wel eens,.}{na wie zich nog niet in de}{kerk vertoond hadden}\\

\haiku{Er ketste iets, met,.}{hol en droog geluid op den}{steenen vloer van de kerk}\\

\haiku{Daar rondom drongen,,.}{de menschen stom van schrik in}{afgrijzen terug}\\

\haiku{- Breng Guus Pollissen, '.}{ook nog op den goeden weg}{dan zijnt er twee}\\

\haiku{- Men weet nooit waar het,.}{zaad terecht komt  troostte}{zich zijn confrater}\\

\haiku{Hoe die bij het volk,.}{ingeslagen is zult U}{wel gemerkt hebben}\\

\haiku{- Ge kunt gerust eens,.}{inschenken moedigde de}{Conciliant aan}\\

\haiku{ze verzekerde,.}{dat hij een spion van de}{belastingen was}\\

\haiku{De bewoners van.}{het reuzenkrot hadden hun}{verblijf zelfs versierd}\\

\haiku{Een goed loon in den,, '.}{zomer geen zin voor sparen}{s winters gebrek}\\

\haiku{Dat heftige wijf.}{heeft niet zoo'n ongelijk met}{haar verbittering}\\

\haiku{Het mag U wel niet,.}{verwonderen dat de droom}{mij interesseert}\\

\haiku{Met een hoopvol en.}{dankbaar gemoed ben ik naar}{Brussel getogen}\\

\haiku{Er gebeurde iets,.}{in mijn kop of een bol vuur}{erin ontplofte}\\

\haiku{Noch Marie noch haar.}{moeder hadden hem ooit als}{arts geconsulteerd}\\

\haiku{De eigenlijke:}{aard van dat schepsel leek hem}{niet twijfelachtig}\\

\haiku{daarmee wilde de.}{vreemde persoonlijkheid niets}{te maken hebben}\\

\haiku{P. Medardus kon:}{niet nalaten even in dien}{hoop te snuffelen}\\

\haiku{Een fijne flesch {\textquoteleft}Mumm,{\textquoteright}.}{cordon rouge verhoogde}{de intimiteit}\\

\haiku{Ze dansten af en.}{toe in een klein zaaltje met}{eigen strijkorkest}\\

\haiku{Even beet wel een  :}{vluchtig wantrouwen in de}{verhitte koppen}\\

\subsection{Uit: Sterren, musschen en ratels}

\haiku{Zegt het spreekwoord niet,?}{dat alleen het ongeluk}{geschiedenis schrijft}\\

\haiku{Niemand kende, als,}{hij de vorming en den aard}{van den zoo rijken}\\

\haiku{Remerswael ligt in.}{de goede jaargetijden}{paradijselijk}\\

\haiku{met de drie kleine '.}{siroopstokerijen ging}{t ook al niet meer}\\

\haiku{En hij, de vinder,?}{wat zouden de gevolgen}{voor hem zelf wezen}\\

\haiku{- En ik geloof aan,.}{geen eeuwigen dood stemde}{een opzichter in}\\

\haiku{Montluce had van.}{verre deze tragedie}{der geesten gevolgd}\\

\haiku{Hij zat uren aan uren,.}{in zijn werkkamer star in}{de verten turend}\\

\haiku{Maar de pastoor sprong.}{bij het hooren dier oude}{galmen ontzet op}\\

\haiku{Toen hij plots oversloeg,.}{in godslasteringen viel}{een stilte rond hem}\\

\haiku{De vrouwen deinsden,.}{angstig de mannen slopen}{ruimer naar voren}\\

\subsection{Uit: Vallende vogels}

\haiku{La belle blonde{\textquoteright},.}{nam de hooge heeren in den}{tuin van een caf\'e}\\

\haiku{toen ze het twisten.}{der jonge meisjes hoorden}{treuzelden ze wat}\\

\haiku{Het gaat op leven,.}{en dood kreet de heesche stem}{van een blokbreker}\\

\haiku{Van dezen mensch hield,,.}{ze nu opeens met heel haar}{hart met heel haar wil}\\

\haiku{Al haar vroegere,,,;}{sentimenten leken vaag}{flauw verzinsels zwak}\\

\haiku{zeker, ze was mooi,,,.}{en zelfs met smaak hoewel zeer}{eenvoudig gekleed}\\

\haiku{Als kind zag ze in.}{Achiel de verwezenlijking}{van dat andere}\\

\haiku{Hij schreef het in een.}{plotsen roes en voelde het}{als een bevrijding}\\

\haiku{Hij kende Wanroth;}{genoeg om zich met reden}{te verontrusten}\\

\haiku{dat ik ze voor niks!}{beter meer aanzie dan die}{mamsellen uit Luik}\\

\haiku{Ze merkten dat ze.}{raak schoten en zetten hun}{wraakneming voort}\\

\haiku{- Laten we liever,.}{wat gaan wandelen ik houd}{het binnen niet uit}\\

\haiku{- En ik, onnoozele,......}{die je vertrouwde die je}{idealiseerde}\\

\haiku{een uitbarsting van.}{redelooze drift had ze hem}{niet in staat geacht}\\

\haiku{Wat dat beteekende,.}{wist ze zelf niet precies je}{zag een hollen weg}\\

\haiku{Voor ze zoover was, greep.}{hij haar om het midden en}{trok haar naar zich toe}\\

\haiku{Het viel Sanne nu,.}{op dat ze dit erg zwaar en}{overdreven deden}\\

\haiku{- Die fotos nam ik,,.}{zelf hakkelde hij en ik}{liet ze vergrooten}\\

\haiku{Willy hield zich goed,.}{al steeg een verscheurende}{woede in hem op}\\

\haiku{Daarom wilde ze,.}{niet meer van hem houden niet}{meer aan hem denken}\\

\haiku{Alles leek haar dan,,.}{onverschillig doelloos grauw}{en onbelangrijk}\\

\haiku{Inderdaad had een.}{diepe desillusie zijn}{karakter misvormd}\\

\haiku{- Met je dronken kop.}{vertelde je me wat je}{me nuchter verzweeg}\\

\haiku{'s Middags Stegen,,.}{beiden de parachute}{op den rug omhoog}\\

\haiku{Op de kade wat,}{wenkende en wuivende}{menschen op het schip}\\

\haiku{Maar een boeketje.}{bloemen voor Sanne wilde}{hij niet vergeten}\\

\haiku{Onder hysterisch.}{gelach van Conny viel de}{debutante flauw}\\

\haiku{mompelde hij, na.}{de voorgeschiedenis te}{hebben vernomen}\\

\haiku{Het werd een totaal,.}{ineenzinken van alle}{hoop energie en trots}\\

\haiku{Ze steeg met Willy,.}{op boven een landschap met}{spitse bergtoppen}\\

\haiku{- Sanneke, vergeef......... -}{me vergeef me brulde hij}{als een zinnelooze}\\

\haiku{Willy was toch geen,.}{slecht mensch geweest had altijd}{hoog en edel gewild}\\

\haiku{De gewonde ging,.}{achteruit maar kon toch nog}{bezoek ontvangen}\\

\haiku{Sanneke vergleed,,...}{in een eindelooze witte}{zalige stilte}\\

\haiku{Voor Demerrel zou,.}{ze een paar prachtige groote}{afdrukken maken}\\

\section{Pierre Kemp}

\subsection{Uit: Limburgs sagenboek}

\haiku{Dit scheen een teken,.}{van boven te zijn om op}{die plaats te graven}\\

\haiku{Deze vluchtten nu.}{naar Maastricht en verhaalden}{daar het gebeurde}\\

\haiku{De plaats waar deze,;}{gebouwd werd was toen nog een}{waterloze streek}\\

\haiku{het welcke hy.}{volmaeckt heeft door synen Apostel}{den H. Jacobus}\\

\haiku{Verhaal nu hetgeen,!}{u is geschied tot meerder}{glorie van Zijn naam}\\

\haiku{Niemand had die struik.}{geplant en hij bloeit er nog}{tot heden toe voort}\\

\haiku{evenmin konden ook.}{nu de einden aan elkaar}{bevestigd worden}\\

\haiku{Die plek had precies.}{de vorm van de grondslag van}{een kapelletje}\\

\haiku{Weer beproefde hij.}{van wal te steken en weer}{lukte het hem niet}\\

\haiku{Gedurende de;}{H. Mis overviel een diepe}{slaap de hertogin}\\

\haiku{{\textquoteleft}Oh, nu hebben wij.}{toch eindelijk Margreetje}{teruggevonden}\\

\haiku{Zij voeren verder.}{en verder tot zij aan de}{stad Athene kwamen}\\

\haiku{Zij werd veroordeeld.}{om de volgende morgen}{verbrand te worden}\\

\haiku{Het was een wreed en,.}{hardvochtig man die niets gaf}{om God noch gebod}\\

\haiku{Het bleek een monnik,.}{te wezen die hier blijkbaar}{wilde overnachten}\\

\haiku{{\textquoteleft}Ik was juist van plan,?}{m'n ziel te verdobbelen}{wat dunkt u ervan}\\

\haiku{Hij was dapperder:}{dan zijn naam en een buurvrouw}{van hem riep hem toe}\\

\haiku{{\textquoteright} riep de visboer, blij.}{verrast en wilde zich door}{de rijen dringen}\\

\haiku{Als ik je eens een,.}{raad mag geven laat het dan}{zo en zo maken}\\

\haiku{{\textquoteright} In hun angst holden.}{zij tussen de menigte}{door naar de prior}\\

\haiku{Daar vertelden zij.}{het gebeurde en toonden}{de bebloede draad}\\

\haiku{Dat is natuurlijk.}{honderden en honderden}{jaren geleden}\\

\haiku{Maar in de huizen.}{op de Heerestraat is nooit meer}{iemand gaan wonen}\\

\haiku{Maar de koning, die {\textquoteleft}{\textquoteright},.}{links te paard stijgt zal te Mook}{vluchten over de brug}\\

\haiku{Hij wilde het niet,.}{geloven hij had het te}{duidelijk gehoord}\\

\haiku{Bij het herenhuis,.}{van Tielens gekomen schreed}{zij daar de stoep op}\\

\haiku{Zij zochten nog toen.}{het al twaalf uur sloeg op de}{kerkklok van Gulpen}\\

\haiku{Maar toen kwam daar een,:}{zwarte vlugge heer aan en}{zei tegen de knecht}\\

\haiku{Onderwijl beval.}{de heer de aap de gast eens}{goed te bedienen}\\

\haiku{Hadt gij ook dat nog,!}{nagelaten dan waart gij}{in zijn macht geweest}\\

\haiku{De man biechtte de.}{tweede maal en later nog}{eens tot acht maal toe}\\

\haiku{De man werd na dat.}{uur beter en beter en}{herstelde spoedig}\\

\haiku{Dit duurde tot er,.}{een steen werd gekapt die juist}{in het gat paste}\\

\haiku{{\textquoteleft}Ga naar de molen!}{en maal de zak koren die}{is aangekomen}\\

\haiku{Hij gaf hem de kom.}{met zaad en gebood hem de}{korrels te tellen}\\

\haiku{Want zo aanstonds moet,!}{ik mij wreken al is het}{op mijn beste vriend}\\

\haiku{Terzelfdertijd kwam:}{de bezetene weer tot}{bezinning en riep}\\

\haiku{moorden, diefstallen.}{en branden waren aan de}{orde van de dag}\\

\haiku{Dit had zo enige.}{jaren geduurd en de graaf}{was ten einde raad}\\

\haiku{Die mensen waren.}{zeer beangstigd en spoedden}{zich biddend naar huis}\\

\haiku{Ouden van dagen, '.}{waarschuwen us nachts nooit}{door de Pas te gaan}\\

\haiku{{\textquoteright} {\textquoteleft}'t Is geen ekster, ',{\textquoteright}, {\textquoteleft}!}{t is een raaf snauwde de}{koningwerk maar voort}\\

\haiku{Geen bergkant was hem,,.}{te steil geen water te breed}{geen moeras te diep}\\

\haiku{Het was winter, de.}{sneeuw lag nog al hoog en het}{had flink gevroren}\\

\haiku{Zo kwamen zij op,.}{een plaats die de schijnwerker}{nog nooit had gezien}\\

\haiku{Het spook nam de steen.}{weg en nu vertoonden zich}{twee potten met geld}\\

\haiku{Bij een volgende;}{bevalling bleef de man thuis}{bij zijn vrouw achter}\\

\haiku{Wel hoorden beiden,:}{de stem van de berggeest die}{hen spottend toeriep}\\

\haiku{Het naderde het.}{bed en hield de soldaat het}{licht onder de ogen}\\

\haiku{Dat was hem welkom,.}{want hij had grote honger}{na dat ronddwalen}\\

\haiku{{\textquoteright} De gedaante liet.}{hem echter geen tijd om nog}{verder te praten}\\

\haiku{Vroeger zou op die.}{plaats alle avonden een licht}{hebben gebrand}\\

\haiku{De jonge heks zou,.}{al gauw merken hoe lelijk}{zij zich had vergist}\\

\haiku{het wijf lag daar met.}{stevige hoefijzers aan}{handen en voeten}\\

\haiku{Op een donkere - -}{avond het was winter begaf}{hij zich dan met}\\

\haiku{Hij wierp het kamrad.}{op de mestvaalt en maakte}{zich uit de voeten}\\

\haiku{{\textquoteleft}Zie je nu, dat het ' ....,:}{t kamrad van is zoals}{ik je gezegd heb}\\

\haiku{Hij wilde dus iets.}{verzinnen om de kat in}{de oven te werpen}\\

\haiku{{\textquoteright} riep de vader boos.}{en zette de appel op}{de schoorsteenmantel}\\

\haiku{Hij zag ook dat het;}{dier moeite genoeg deed om}{vooruit te komen}\\

\haiku{En toen hij dan ook,:}{zijn nood klaagde aan die man}{antwoordde deze}\\

\haiku{Het meisje bracht de.}{boodschap over aan mijnheer en}{deze liet haar gaan}\\

\haiku{na die tijd zijn geen.}{ongelukken meer in de}{brouwerij gebeurd}\\

\haiku{Je hebt mijn dochter!}{met een kwade hand geraakt}{en ze doen kwijnen}\\

\haiku{{\textquoteleft}Wat zijt gij toch een,!}{gelukkige vrouw die zo}{een goede man hebt}\\

\haiku{{\textquoteright} {\textquoteleft}Zeg dat wel,{\textquoteright} meende,.}{de vrouw gevleid dat haar man}{zo geprezen werd}\\

\haiku{Die heb ik dat ook!}{geraden en nu zijn ze}{maar wat gelukkig}\\

\haiku{De pastoor keek eens.}{om en bad toen luider in}{het  Latijn voort}\\

\haiku{Toen zij weg waren,.}{nam de jonkman ook de pot}{en zalfde zich ook}\\

\haiku{Onderwijl sloop de.}{knecht stilletjes naderbij}{en nam de zeef weg}\\

\haiku{Heel de berg zat vol.}{katten en er kwamen er}{nog maar altijd bij}\\

\haiku{Hij wilde weten.}{waar hij aan toe was en ging}{weer naar beneden}\\

\haiku{Dat wijf kan ook nog,{\textquoteright}.}{wel wat anders dan brood eten}{antwoordde de man}\\

\haiku{Zij wisten geen raad.}{ertegen en de plaag werd}{met de dag erger}\\

\haiku{{\textquoteright} Op deze wijze,.}{zanikte zij zolang de}{mannen er waren}\\

\haiku{Die man was dood en.}{het was zijn doodshemd dat zij}{in de  hand hield}\\

\haiku{{\textquoteleft}Haaf water, haaf m\`elk,,}{iech h\"ob get krie gemete}{En iech h\"ob m'n ziel}\\

\haiku{Wel vond men enige.}{dagen later zijn lijk dat}{boven kwam drijven}\\

\haiku{Die dag was het een,,.}{weer dat men er geen hond door}{joeg laat staan een mens}\\

\haiku{{\textquoteleft}Gij moet de vrouw weer.}{leggen op de plaats waar zij}{is dood gebleven}\\

\haiku{Toen zij 's morgens, '.}{wakker werd lag haar man weer}{naast haar int bed}\\

\haiku{{\textquoteleft}Het kan gaan, gelijk,!}{het wil maar morgennacht schiet}{ik het veulen dood}\\

\haiku{Het werd hem nu wel,}{wat bang te moede maar hij}{besloot toe te zien}\\

\haiku{Zo bleef hij wachten,.}{in zijn duistere schuilhoek}{tot klokslag een uur}\\

\haiku{Ze hoorden de kat;}{uit de verte aankomen}{of over de daken}\\

\haiku{De mensen van de.}{hoeve wisten niet wat zij}{ervan denken moesten}\\

\haiku{Zo werd de knecht dan.}{ook stilaan ervan verdacht}{een weerwolf te zijn}\\

\haiku{want de kogel vloog.}{over de bomen in plaats van}{de hond te treffen}\\

\haiku{De pastoor gaf het,;}{gevraagde doch liet zich over}{de zaak zelf niet uit}\\

\haiku{De stem van boven.}{Te Schaesberg zaten enige}{mannen te kaarten}\\

\haiku{Onderweg werd hij.}{telkens lastig gevallen}{door een zwarte hond}\\

\haiku{Een jager was in.}{het Fazantenbos onder}{Oh\'e en Laak op jacht}\\

\haiku{De dronkemannen,.}{wilden hem wel eens zien dat}{was juist iets voor hen}\\

\haiku{Zo kwam hij over de.}{weg van Mari\"enwaard naar}{het Limmerlerbroek}\\

\haiku{{\textquoteright} lachte de lange.}{juffrouw nog en toen was ze}{ineens verdwenen}\\

\haiku{{\textquoteright} Eindelijk liet de.}{veerman zich toch bepraten}{en zette hem over}\\

\haiku{Ook vertoonde de,.}{kanunnik zich soms zijn hoofd}{in de hand dragend}\\

\haiku{wanneer zij het nog,.}{eens zag moest ze het vragen}{wat het begeerde}\\

\haiku{De eerste pater,.}{die je tegenkomt daar moet}{je goed op letten}\\

\haiku{Die overste zal je,.}{aannemen dan kun je voor}{priester studeren}\\

\haiku{Buiten de kom van.}{het dorp waren twee maaiers}{in het hooi werkzaam}\\

\haiku{maar het was of zij.}{versteend waren en iets hen}{op de plaats vasthield}\\

\haiku{hij kon evenwel niet, {\textquoteleft}{\textquoteright}.}{merken dat hij het dier ook}{maard\`at verwondde}\\

\haiku{En nu bekende.}{hij dat hij zijn ziel aan de}{duivel had verkocht}\\

\haiku{{\textquoteleft}Ik zie wel, dat je!}{erg bedrukt bent en dat het}{je niet naar wens gaat}\\

\haiku{{\textquoteleft}God zegene u,{\textquoteright}.}{mocht de duivel zich van haar}{ziel meester maken}\\

\haiku{{\textquoteright} Met deze boodschap,,.}{door de bankier overgebracht}{kon de duivel gaan}\\

\haiku{{\textquoteright} verliet de boze,.}{hem onder het breken van}{vele boomtakken}\\

\haiku{Verontwaardigd wees;}{mevrouw het misdadige}{voorstel van de hand}\\

\haiku{Zij wilde in dat.}{spookhuis geen dag meer blijven}{en zei haar dienst op}\\

\haiku{Nu heb ik geen rust,!}{meer zelfs niet in het graf en}{in de eeuwigheid}\\

\haiku{Het was volle maan,.}{toen hij ging en buiten zo}{helder als de dag}\\

\haiku{Weer hernamen zij.}{hun geweldige arbeid}{van voren af aan}\\

\section{Paul Kenis}

\subsection{Uit: F\^etes galantes}

\haiku{Zij was geen {\textquoteleft}fille{\textquoteright},;}{galante niet een vrouw uit}{de lichte wereld}\\

\haiku{Ondanks den fellen.}{wind bleef het zooals altijd erg}{druk op de groote brug}\\

\haiku{{\textquoteright} Met moeite baanden;}{de twee wandelaars zich door}{dat alles een weg}\\

\haiku{Zij was eene {\textquoteleft}fille{\textquoteright},,.}{eene meid van lichte zeden}{van de armste soort}\\

\haiku{de lange bruine {\textquoteleft}{\textquoteright}}{lokken ongepoeierd lijk}{eenincroyable}\\

\haiku{op de grenzen der.}{heide stierf alle geluid}{van de wereld weg}\\

\haiku{Ook als Fabre;}{haar aanspreekt kijkt ze daarom}{niet verwonderd op}\\

\haiku{De waterkruik staat;}{on-aangeraakt op den rand}{van het bronbekken}\\

\haiku{Over het met gekleurd:}{zand bestrooid paadje stapten}{beiden nevens een}\\

\haiku{een zuur gezicht te.}{trekken tegen de heeren}{die lief wilden zijn}\\

\haiku{Over al de dingen:}{begon de avond zijn grijzig}{webbe te spinnen}\\

\haiku{Ten volle was dan.}{ook de danseres op haar}{aanbidder verliefd}\\

\haiku{het oude regiem.}{tegen de omwenteling}{zou verdedigen}\\

\haiku{Maar mag ik nu ook}{weten wat er verder met}{mij gebeuren zal}\\

\haiku{de heer Cazotte.}{had ons allen nog den angst}{op het lijf gejaagd}\\

\haiku{in het gevang waar.}{hij het verband van zijne}{wonden zou rukken}\\

\haiku{het verhaal mijner;}{wederwaardigheden zou}{u slechts vervelen}\\

\haiku{toren stevenden.}{die den ingang van de}{haven verdedigt}\\

\subsection{Uit: Historische verhalen}

\haiku{En de Heer heeft mij,.}{gezegend want mijn arbeid}{gedijt voor dit land}\\

\haiku{Zoo vermochten die,;}{heeren van de wet het de}{rust te herstellen}\\

\haiku{ook had hij nog het,.}{schootsvel voor waarmede hij}{te arbeiden placht}\\

\haiku{De magistraat bleek.}{niet bij machte er paal en}{perk aan te stellen}\\

\haiku{Mijnheer van Egmont,;}{was in een draagstoel met twee}{muilezels bespannen}\\

\haiku{Hier en daar vonkte.}{de bleeke zon een schittering}{in al dat metaal}\\

\haiku{De Cellebroeders.}{kwamen aan om de lijken}{van de galg te doen}\\

\haiku{moeizaam leunend op,.}{den arm van een metgezel}{strompelde hij voort}\\

\haiku{Maanden lang had de;}{late Winter alles blank}{gelegd en verstard}\\

\haiku{Toen hadden luwe;}{regens de sneeuw doen smelten}{en den grond doorweekt}\\

\haiku{bijtenden hoogmoed,.}{en knagenden nijd die hem}{het hart wegvraten}\\

\haiku{koortsgloed vlamde in,.}{de blikken die deemoedig}{ten gronde keken}\\

\haiku{Buiten verblindde.}{hen weer het licht van de nu}{reeds schuin staande zon}\\

\haiku{Met de kruisbroeders;}{was ook veel vreemd volk de stad}{binnengekomen}\\

\haiku{een koffer werd van.}{de tweede verdieping in}{de straat leeggeschud}\\

\haiku{V Jonker Wenzel;}{trof schikkingen om den burcht}{te overmeesteren}\\

\haiku{Clio, de muze van,.}{de geschiedenis moge}{het mij vergeven}\\

\haiku{De moeilijkheden.}{van mijn onderneming heb}{ik niet onderschat}\\

\haiku{Te meer daar ik met.}{al mijn plannen zoo weinig}{blijk op te schieten}\\

\haiku{Elk tuintje sluimert.}{als in een bocht van de snel}{vlietende rivier}\\

\haiku{Wat verder voert een,;}{vlonder over het water naar}{de dennenbosschen}\\

\haiku{Waarom had ik aan:}{mijn hospita eenvoudig}{niet de vraag gesteld}\\

\subsection{Uit: De kleine Mademoiselle C\'erisette}

\haiku{Het eerste wat ik '}{nu deed als iks morgends}{vroeg ontwaakte was}\\

\haiku{, dagen lang hadden;}{wij er nieuwe schoonheden}{kunnen genieten}\\

\haiku{keeltjes trilden bij,.}{het me\^eneurie\"en van het}{lied borstjes golfden}\\

\subsection{Uit: De roman van een jeugd. Een ondergang in Parijs}

\haiku{Eigenlijk had hij:}{ook wel gemeend de vrienden}{wat te overbluffen}\\

\haiku{hij voelde hoe 't}{hem te benauwd zou worden}{op zijne kamer}\\

\haiku{eventjes kwam het in {\textquoteleft}{\textquoteright}.}{hem op dat deweltschmerz ook}{ingebeeld kon zijn}\\

\haiku{Zoo stellig had hij;}{vader beloofd er ditmaal}{door te geraken}\\

\haiku{hij trok de laden:}{open waaruit hij \'e\'en voor \'e\'en}{zijne zaken kreeg}\\

\haiku{den briefschrijven die.}{hen thuis van zijn voornemen}{moest verwittigen}\\

\haiku{Tusschen het gewoel:}{heen drongen verkoopers van}{velerlei dingen}\\

\haiku{Zou hij niet op eene?}{bank wat uitrusten of in}{een koffiehuis gaan}\\

\haiku{eerst moest hij het wat.}{gewend worden en met de}{stad kennis maken}\\

\haiku{De kade lag koel;}{en rustig in het lommer}{van oude boomen}\\

\haiku{het oude {\textquoteleft}quartier{\textquoteright}:}{du Temple met historisch}{klinkende namen}\\

\haiku{slechts mogelijk was...}{zich in een blad of tijdschrift}{bekend te maken}\\

\haiku{een paar keeren ging hij:}{nog en dan was er een kort}{briefje gekomen}\\

\haiku{Waarschijnlijk zou hij:}{vandaag den bestuurder niet}{meer kunnen spreken}\\

\haiku{Begoochelingen,,!}{ach die waren zoo lang en}{zoo verre voorbij}\\

\haiku{Zoo kon hij taalman;}{worden in een hotel of}{een groot magazijn}\\

\haiku{Zelfs in armoede;}{was het vrije leven heerlijk}{in deze groote stad}\\

\haiku{- Neen, illuzies wel,;}{niet meer die hadden reeds te}{dikwijls bedrogen}\\

\haiku{Vroeger had Vincent,.}{dat alles niet opgemerkt}{nu leerde de nood}\\

\haiku{Hoe kwam je er toch,;}{op zoo maar heereknecht}{te willen worden}\\

\haiku{hij zou twintig frank;}{per week ontvangen en in}{huis eten en slapen}\\

\haiku{nu bemerkte hij.}{dat de zon over de huizen}{begon te nijgen}\\

\haiku{zijn kennis had hem.}{in de steek gelaten en}{zou niet meer komen}\\

\haiku{Na 't eten gingen.}{zij heen en namen afscheid}{op den boulevard}\\

\haiku{soms schoven over den;}{vloer de slepende passen}{van een dansend paar}\\

\haiku{daarbij ze hadden;}{gezegd dat hij tweemaal per}{week kon terug keeren}\\

\haiku{Zonder overtuiging;}{stemde Vincent met heel die}{redeneering in}\\

\haiku{Alvorens toe te:}{geven stelde hij echter}{zijne voorwaarden}\\

\haiku{de vijf frank van het,;}{cachet de opbrengst van het}{verkochte uurwerk}\\

\haiku{Dan had Hettner zijn:}{beschermeling een paar frank}{in de hand gestopt}\\

\haiku{toen de andere.}{bleef aandringen wees hij hem}{af met stug gebaar}\\

\haiku{anderen deden,.}{het wel die het minder noodig}{hadden dan hij zelf}\\

\haiku{een eerste groep ging,;}{vroeg in den avond uit en bleef}{slechts tot tien elf uur}\\

\haiku{Soms schalde in de;}{verte het gezang van een}{huiswaarts keerende groep}\\

\haiku{wel had hij niet veel,.}{verdiend maar genoeg om zijn}{honger te stillen}\\

\haiku{Vincent keek rond of;}{hij niet eene geschikte plaats}{om te slapen vond}\\

\haiku{de {\textquoteleft}crocheteurs{\textquoteright} en;}{andere arme stakkerds}{hadden weer arbeid}\\

\haiku{De rij volgend ging,}{Vincent een smal gangetje}{door naar eene groote zaal}\\

\haiku{ondanks de lucht en.}{ondraaglijke hitte kon}{je veilig rusten}\\

\haiku{De zaal was zoo vol.}{dat er niet \'e\'en enkele}{meer bij zou kunnen}\\

\haiku{De {\textquoteleft}hotels{\textquoteright} hadden:}{allen hetzelfde vuile}{schurftige uitzicht}\\

\haiku{en zoo je tien cent;}{opleg betaalde kreeg je}{zuivere lakens}\\

\haiku{ook het supplement.}{voor reine lakens wilde}{hij wel uitgeven}\\

\haiku{maar zijn walg om \'e\'en.}{bed met twee te deelen kon}{hij niet overwinnen}\\

\haiku{De eerste nacht, dien,:}{Vincent hier doorbracht leek hem}{bijzonder akelig}\\

\haiku{daarbij, je kon 't:}{in die vuile atmosfeer}{toch niet uithouden}\\

\haiku{Ook Vincent kreeg zijn,;}{part zoodat hij twee dagen lang}{kaas bij zijn brood had}\\

\haiku{Een andermaal kwam;}{l'Asticot aangedragen}{met een zak noten}\\

\haiku{zij gingen altijd,;}{bij paren met denzelfden}{afgemeten stap}\\

\haiku{iederen dag gaan...}{schrijvers weg en komen er}{nieuwe in de plaats}\\

\haiku{Met muffen reuk sloeg.}{het stof neer en weer kwam er}{wat verademing}\\

\haiku{Als zijn wrokkige:}{menschenhaat smolt weg voor de}{warmte van dien blik}\\

\haiku{Slechts eventjes wou hij}{haar terug zien en laten}{weten hoe dankbaar}\\

\haiku{Voor den geringen:}{prijs van vijftien cent had hij}{een overvloedig maal}\\

\haiku{hij wou er niet aan,.}{denken en schreef maar voort maar}{altijd voort adressen}\\

\haiku{Zij verlangden slechts.}{iedereen even ellendig}{als zich zelf te zien}\\

\haiku{hij wilde niet meer,.}{aan dat alles  denken}{het was te pijnlijk}\\

\haiku{Anderen dosten:}{hun bedienden uit in een}{bont vastenavondpak}\\

\haiku{zes potloodkrabbels:}{en vier penseelvegen en}{daar stond het landschap}\\

\haiku{De meesten waren {\textquoteleft}{\textquoteright},:}{l\^acheurs die je onmiddellijk}{in den steek lieten}\\

\haiku{ze zouden je van;}{niets hebben ingelicht en}{niets voor je gedaan}\\

\haiku{hij was erg bij de.}{hand en zijne papieren}{waren in orde}\\

\haiku{Voor de eerste maal,;}{weer sedert den langen tijd}{dacht hij aan de vrouw}\\

\haiku{n\'ecessit\'e faict,.}{gens mesprendre et faim le}{loup saillir du bois}\\

\haiku{hoe hij zes maand had.}{gekregen voor diefstal in}{een automobiel}\\

\haiku{de welstand van een.}{ander was als een hoon op}{eigen ellende}\\

\haiku{gisteren avond nog;}{had zij een klant gehad die}{niet betalen wou}\\

\haiku{Zoo kwam de deur voor;}{een oogenblik vrij en was}{hij buiten gesneld}\\

\haiku{Hij was hongerig,.}{en vermoeid de kou beet hem}{in het aangezicht}\\

\haiku{Een ander had er;}{van gesproken om naar het}{Zuiden te trekken}\\

\section{M.J.H. Kessels}

\subsection{Uit: Der Koehp va Hehle in de sjlag va Waterloo}

\haiku{zit ee aoth versjrummeld,{\textquoteright}.}{menke noh sjatting wiet in}{de nuhgentig joar}\\

\haiku{Ich zaan, da gao ich ' '.}{t iesjte evvelt Niehs}{noch adieh zage}\\

\haiku{adieh Koehp, osse,.}{leeve Hergot bewaart dich}{ich zal dervuur beehne}\\

\haiku{Der ruksjtrank doog}{mich geweldig pieng en de}{rubbe krakde mich}\\

\haiku{Noew wosj heh mich der '.}{ruk duchtig aaf en vreefm}{in mit veerevet}\\

\haiku{{\textquoteleft}Sjmak het, Koehp{\textquoteright}, vroog ich, {\textquoteleft}{\textquoteright},, {\textquoteleft}{\textquoteright}.}{of dat antwoordde hehnoch}{behter wie bookeskook}\\

\haiku{Die sjpas hei der,.}{motte zieh vier sjlooge ze neer}{wie de wil kanieng}\\

\haiku{{\textquoteright} -  {\textquoteleft}Dan ginne tied{\textquoteright},{\textquoteleft}:}{verlore zaan ich tegen}{der Bam   Ich zaan}\\

\haiku{Ze zooge het dan ooch.}{in en leepe wie echte}{winkhong dervan durch}\\

\haiku{maar de plank brik aaf.}{en ich val weer oppen ruk}{tusje de verke}\\

\haiku{Zouwe 't flits de{\textquoteright} ().}{Pruuse zieh of der Kroehsjel}{generaal Grouchy}\\

\haiku{dat niks mieh doh woar ().}{wie de ouw Garde onger}{der FrijanFriant}\\

\haiku{- {\textquoteleft}Drei joar a ee sjtuk ' '.}{is ze neet oehtt bed en}{t lieje gewehs}\\

\section{Mensje van Keulen}

\subsection{Uit: Van lieverlede}

\haiku{In ieder geval '.}{gaatt niet over als ik me}{steeds moet verkleden}\\

\haiku{Toen Coby nog thuis,.}{woonde werd er beneden}{ook wel gezongen}\\

\haiku{Ze keek toe hoe de.}{vrouw haar moeder op de rug}{begon te kloppen}\\

\haiku{Mevrouw Beijer kreeg.}{haar boterham niet op en}{klaagde over haar maag}\\

\haiku{ik heb je daar nog,.}{nooit gezien of je daar geen}{zin in zou hebben}\\

\haiku{De kleine stond te.}{dreinen en in paniek aan}{haar rok te trekken}\\

\haiku{{\textquoteright} In de woning van.}{haar oudste dochter begon}{ze er opnieuw over}\\

\haiku{{\textquoteright} Coby lachte en.}{negeerde Dannie die een}{lelijk gezicht trok}\\

\haiku{Dan is Dannie dat,{\textquoteright}, {\textquoteleft}.}{ook zei mevrouw Beijerwant}{die heeft ook vrienden}\\

\haiku{{\textquoteright} Dromerig staarde.}{Coby voor zich uit en toen}{schudde ze haar hoofd}\\

\haiku{Ze legde een hand.}{over haar ogen en hoorde haar}{moeder wegsloffen}\\

\haiku{De panty was hard.}{aan de teenstukken en moest}{nodig gewassen}\\

\haiku{Over hem wisten ze,.}{niet zoveel behalve dat}{ie erg laat thuiskwam}\\

\haiku{{\textquoteright} {\textquoteleft}Als je me dat zou.}{gunnen mag je me wel het}{dubbele geven}\\

\haiku{En jullie waren,}{jullie waren vreselijk}{onaardig voor me.}\\

\haiku{Denk eraan dat opa.}{vanmiddag komt om paardjes}{voor je te bakken}\\

\haiku{Haar handen lagen,,.}{gebald als twee geplukte}{duifjes op de sjaal}\\

\haiku{{\textquoteright} zei mevrouw Beijer:}{met een gesmoorde stem die}{piepend eindigde}\\

\haiku{De mis heeft veel geld, ',,.}{gekostt was in de kerk}{sjiek met een loper}\\

\haiku{Van die dingen die.}{ze in flessen afsteken}{kan ik hier niets zien}\\

\haiku{{\textquoteright} Hanna hield de fles.}{boven het glas tot er geen}{druppel meer uitkwam}\\

\haiku{Buiten klonk er nu,.}{en dan nog een knal de fik}{lag na te smeulen}\\

\haiku{{\textquoteright} Hanna ging zitten.}{en duwde haar hand in het}{biddende gezicht}\\

\haiku{Hanna legde het.}{slechts van een naam voorziene}{poststuk voor haar neer}\\

\haiku{of ie nou mee-,.}{of tegenvalt je bent een}{ervaring rijker}\\

\haiku{Zijn vrouw negerend, doch,.}{de voordeur openlatend liep}{hij het portiek uit}\\

\haiku{Ik doe het licht uit,,.}{dacht ze zodat ie denkt dat}{er niemand thuis is}\\

\haiku{Ze eindigt altijd.}{met zielig en spijtig doen}{als zeuren niet helpt}\\

\haiku{Hij heeft me wel eens...}{in de steek gelaten en}{dan komt het slechte}\\

\haiku{Er bewoog iets rechts,.}{onder in de hoek ze ging}{op haar tenen staan}\\

\haiku{Blijven hangen aan.}{die stomme leuning van dat}{stomme rotportiek}\\

\haiku{{\textquoteleft}Hij gelooft nooit dat '.}{jij met dat ouwe lijkn}{dagje op stap gaat}\\

\haiku{{\textquoteright} {\textquoteleft}Welja, begin jij, '.}{ook nog maar eens iedereen}{neemtt voor hem op}\\

\haiku{{\textquoteright} Hanna stopte de.}{bestelling in haar zak en}{tilde de tas op}\\

\haiku{Hij is haar ontrouw,{\textquoteright}.}{geweest zei mevrouw Beijer}{en stak haar hand uit}\\

\haiku{De poten waren,.}{geknakt de rugleuning lag}{voor de commode}\\

\haiku{{\textquoteleft}Ik heb de hele,{\textquoteright}.}{dag nog geen honger gehad}{zei mevrouw Beijer}\\

\haiku{Twee platte, smalle.}{latjes zaten als pleisters}{tegen de deurpost}\\

\haiku{Hanna voelde haar.}{maag zwellen en nog leek haar}{dorst niet te lessen}\\

\haiku{En niet alleen 'n,.}{flesje limonade doe}{er wat stevigs in}\\

\haiku{Ik heb 'n hekel.}{aan mensen die er trots op}{zijn dat ze sparen}\\

\haiku{{\textquoteright} Hij duwde de deur.}{verder open en zette zijn}{voet op de drempel}\\

\haiku{Ze schraapte er de.}{schimmel uit en zette de}{deur naar de tuin open}\\

\haiku{Ze keek naar de korst.}{in haar hand en stond op om}{hem weg te gooien}\\

\haiku{Toen mevrouw Beijer,.}{bij bewustzijn kwam zag ze}{dat het donker was}\\

\section{Jan J. Klant}

\subsection{Uit: De geboorte van Jan Klaassen}

\haiku{Want soms viel er een.}{rukwind op de Dam en trok}{aan de gordijnen}\\

\haiku{De heer Duivel liet.}{mij rustig uithuilen en}{stak een sigaar op}\\

\haiku{Welk een zegening,.}{dit blijde uitstromen van}{het watercloset}\\

\haiku{{\textquoteleft}Je zoent niet als een,{\textquoteright}, {\textquoteleft}.}{kantoorman zei het meisje}{maar als een dichter}\\

\haiku{Een ogenblik nog,{\textquoteright} zei, {\textquoteleft},?}{hijjuffrouw hoe laat begint}{mijn vergadering}\\

\haiku{{\textquoteleft}Zeg Sat, ik zou je,,.}{v\'o\'or we aankomen graag nog}{even willen spreken}\\

\haiku{Katrijn had deze.}{foto in de rand van de}{spiegel gestoken}\\

\haiku{s Nachts greep ik naar,.}{de deurknop Waar wanhoop streed}{om door te breken}\\

\haiku{De secretaris.}{zat aan zijn bureau en mijn}{chef stond achter hem}\\

\haiku{{\textquoteright} {\textquoteleft}Integendeel,{\textquoteright} zei, {\textquoteleft}.}{de secretaris snelwij}{zijn zeer tevreden}\\

\haiku{{\textquoteright} riep ze, stampvoetend, {\textquoteleft}}{van woedejullie dichters}{gaan altijd te ver.}\\

\haiku{Ik ging achter het,.}{voor mij bestemde bureau}{zitten bij het raam}\\

\haiku{Voortdurend ruist er,,}{hier water dacht ik en nu}{ontdek ik het pas.}\\

\haiku{Mijnheer Duivel kwam.}{terug en nodigde mij}{uit hem te volgen}\\

\haiku{Velenzijn er die,.}{verdorstten Mil sloot zij niet}{haar deuren dicht}\\

\haiku{Is het een wonder?}{dat men zich ergert aan hun}{gemaskeerd misbaar}\\

\haiku{{\textquoteright} Toen ik weer in het,.}{portaal stond hoorde ik ze}{bulderend lachen}\\

\haiku{Tenslotte stopte,.}{het toestel toen mijn hoofd juist}{het plafond raakte}\\

\haiku{Hij ging voor mij op,.}{zijn bureau zitten met zijn}{armen over elkaar}\\

\haiku{Wat vond hun holle?}{zoekende blik in deze}{doodse wildernis}\\

\haiku{Ik nam de pion,.}{niet maar richtte mijn loper}{op zijn zwakste punt}\\

\haiku{Ze zijn nog banger,.}{dan ik want zij durven zich}{zelfs niet bewegen}\\

\section{Jos Kleinjans}

\subsection{Uit: Het acces van Meijel}

\haiku{{\textquoteleft}Ik kan u alleen.}{maar complimenteren met}{uw discipline}\\

\haiku{Toen hij de kurk weer,;}{op de fles sloeg zag hij de}{commandant staren}\\

\haiku{Plotseling sprong een.}{van de jongere jongens}{voor hen op het pad}\\

\haiku{De man verviel steeds.}{met overdreven ernst in zijn}{commandantenrol}\\

\haiku{Hij spoog iets in de,.}{lap keek er even naar en borg}{de zakdoek vlug weg}\\

\haiku{{\textquoteright} De vrouw sloeg betrapt.}{haar schort voor haar gezicht en}{vluchtte de stal in}\\

\haiku{Latour stelde zich - -:}{en stond zich niet anders toe}{slechts drie kenmerken}\\

\haiku{Als je een paar maal {\textquotedblleft}{\textquotedblright},.}{het woordvolk herhaalt lijkt het}{niet meer te bestaan}\\

\haiku{{\textquoteleft}Het tijdpotlood is.}{een bijzonder nuttige}{en slimme vinding}\\

\haiku{Het geratel van.}{de mitrailleurs kwam over de}{spoordijk dichterbij}\\

\haiku{Hij richtte zich half.}{op en rukte de Colt uit}{zijn okselholster}\\

\haiku{De jongen met de:}{hamer greep voor de tweede}{maal doeltreffend in}\\

\haiku{De officieren:}{van het Militair Gezag}{volgden op de voet}\\

\haiku{Het valt dus wel mee.}{met mijn kennis ontrent uw}{activiteiten}\\

\haiku{Besloten werd om.}{RVV-groepen selectief}{te bewapenen}\\

\haiku{Over de top van de.}{organisatie heb ik}{geen informatie}\\

\haiku{Ik accepteer dit.}{rapport en daarmee is uw}{opdracht ten einde}\\

\haiku{Morgen meldt u zich.}{bij mij en ik deel u uw}{nieuwe taken mee}\\

\haiku{{\textquoteright} Schuurman knikte en.}{krabbelde een paar regels}{op een stuk papier}\\

\haiku{{\textquoteright} Het sarcasme in,.}{haar blik ontging hem niet maar}{hij negeerde het}\\

\haiku{{\textquoteright} Latour haalde diep.}{adem na de provocatie}{en schudde zijn hoofd}\\

\haiku{Maar daarom hoeven,?}{we hun systeem nog niet te}{omhelzen nietwaar}\\

\haiku{en nog vroeger, een.}{sergeant-majoor die}{stamrozen kweekte}\\

\haiku{{\textquoteleft}Ik ben bang dat ik,.}{niet helemaal begrijp wat}{u bedoelt majoor}\\

\haiku{{\textquoteright} Latour ontweek nu.}{de autoriteit in de}{stem van de majoor}\\

\haiku{{\textquoteright} Schuurman donderde.}{een gebalde vuist op het}{blad van zijn bureau}\\

\haiku{De enige dekking,.}{die ze hadden werd gevormd}{door hun vrachtwagen}\\

\haiku{Latour duwde haar.}{met geweld verder open en}{glipte de gang in}\\

\haiku{{\textquoteright} vroeg Latour en liep.}{zonder haar antwoord af te}{wachten de trap op}\\

\haiku{Latour pakte het.}{lichaam bij de schouder vast}{en trok het terug}\\

\haiku{Haal jij die even op,,.}{maar maak voort want wij moeten}{Serv\'e wegbrengen}\\

\haiku{Een geelgroene fluim.}{bleef op het eikehout van}{een trede liggen}\\

\haiku{het vergeten zijn,.}{we moeten Serv\'e naar het}{hospitaal brengen}\\

\haiku{{\textquoteleft}Nu geloven de.}{mensen misschien nog dat de}{LO echt heeft bestaan}\\

\haiku{{\textquoteright} Latour nam zelf een,.}{slok hield ondertussen de}{man bij diens arm vast}\\

\haiku{{\textquoteleft}En ik ben Little,{\textquoteright}.}{John zei de reus en sloeg}{zichzelf op de borst}\\

\haiku{Jij bent net zo'n stom,{\textquoteright}.}{Frans wijf antwoordde Gleason}{triomfantelijk}\\

\haiku{Jij brengt je schrijfsels.}{bij allerlei idioten}{onder de aandacht}\\

\haiku{{\textquoteright} {\textquoteleft}Ik twijfel er niet,.}{aan dat u uw promotie}{verdiend heeft majoor}\\

\haiku{Ook Latour had het,.}{koud ondanks de gevoerde}{parka die hij droeg}\\

\haiku{Leroy frunnikte.}{een moment aan de leren}{kinband van zijn helm}\\

\haiku{{\textquoteleft}Als de luitenant,.}{mij wil verexcuseren}{ik moet even pissen}\\

\haiku{De deur zwaaide wijd,.}{open nog voordat Latour had}{kunnen aankloppen}\\

\haiku{Hij kon eind vijftig,.}{zijn maar met gemak voor tien}{jaar ouder doorgaan}\\

\haiku{Het was al avond en,.}{donker in de hut van de}{Bisschop widdege}\\

\haiku{De man stond zonder,.}{iets te zeggen op en ging}{voor de gang in}\\

\haiku{enig begrip voor de.}{abnormaliteit hiervan}{bereikte haar niet}\\

\haiku{De regen van de'.}{afgelopen weken heeft}{nie veel goed gedaan}\\

\haiku{Blijf nie' achter, mijn.}{sporen verdwijnen heel wat}{vlotter als gij peinst}\\

\haiku{Ze schenen hem niet.}{te horen en gingen hun}{eigen hut binnen}\\

\haiku{{\textquoteleft}Eh... wat is... eh...{\textquoteright} {\textquoteleft}De,,{\textquoteright}.}{deur luitenant herhaalde}{de man geduldig}\\

\haiku{Op het zand lagen,,.}{in de vorm van een ruit vier}{dode konijnen}\\

\haiku{{\textquoteright} {\textquoteleft}Graag,{\textquoteright} zei Latour en.}{voelde zich belachelijk}{in zijn gretigheid}\\

\haiku{Wil men overleven,.}{dan moet straf koste wat kost}{vermeden worden}\\

\haiku{{\textquoteleft}We zijn er, dat is,{\textquoteright}.}{een gegeven antwoordde}{hij na een ogenblik}\\

\haiku{Ik wilde immers.}{revanche voor mijn verlies}{van de schoolmeester}\\

\haiku{De morgen van de.}{achtentwintigste was het}{weer niet verbeterd}\\

\haiku{D'n Pie  had een.}{stengun die hij ook leek te}{kunnen gebruiken}\\

\haiku{D'n Pie sloeg Wijngaards:}{op diens schouder om hem te}{bedanken en zei}\\

\haiku{totdat u mij die.}{papieren geeft waar ik}{voor gekomen ben}\\

\haiku{Toen liep hij op het,,.}{stilliggende lichaam zijn}{eigen lichaam toe}\\

\section{Johannes Kneppelhout}

\subsection{Uit: Studentenschetsen. Deel 1. Teksten (onder ps. Klikspaan)}

\haiku{s bekannt, Und ' '.}{wo ihrs packt da ists}{interessant}\\

\haiku{- O gulden vrijheid,!}{der Studentenwereld ik}{zal u nooit kennen}\\

\haiku{hij moet getuige:}{zijn van een gesprek waarin}{wreedelijk voorkomt}\\

\haiku{Is hij een Stoicus?}{die zich in de lijdzaamheid}{zoekt te oefenen}\\

\haiku{Drie weken daarna.}{bragt mij het toeval voor de}{derde maal bij hem}\\

\haiku{Gelukkig hij wiens!}{Aeskulaap de vrienden van}{zijne sponde weert}\\

\haiku{men komt zijn vriend geen,.}{gezelschap houden men komt}{koffijhuis houden}\\

\haiku{- zoo, ben jij daar nog? -.}{krijgt Hendrik een stoel bij het}{vuur en vat eene pijp}\\

\haiku{- Kr... - maar ik zou geen -.}{politiek aanroeren en}{andere dassen}\\

\haiku{Maar die menschen hier,.}{souperen nog comme au}{temps de nos p\`eres}\\

\haiku{Hij plaatst zich boven,,,.}{buiten ja tegenover de}{Studentenwereld}\\

\haiku{hij slaat zich de borst.}{kampot om iets te vinden}{dat er naar gelijkt}\\

\haiku{Op de zee van het;}{leven laat hij de hulk van}{den doctor zweven}\\

\haiku{Alleen omdat het.}{eene Dissertatie is zal}{men het niet inzien}\\

\haiku{Daar komt onverwachts:}{eene vervaarlijke stem van}{de achterkamer}\\

\haiku{- Toen werd papa, tot,;}{het uiterste gedreven}{woedend en razend}\\

\haiku{monumenten, voor?}{niemand toegankelijk dan}{die latijn verstaat}\\

\haiku{Het boek zag er uit,.}{als de dief zelf overal met}{smetten en scheuren}\\

\haiku{Ik vraag verschooning,.}{wij hebben slechts den naam met}{elkander gemeen}\\

\haiku{De hinderpalen,.}{waar hij niet over kan mogen}{links blijven liggen}\\

\haiku{Wie uwer schaamt zich zulk,?}{eene armhartige vleitaal}{niet mijne vrienden}\\

\haiku{Nu haast de fleemkous '.}{zich zijne makkers opt}{tapijt te brengen}\\

\haiku{Neen, 't is daar nog,.}{te vroeg voor bovendien is}{dit thans het doel niet}\\

\haiku{Het is een krakeel,,,.}{een oproer eene vischmarkt een}{bordeel van klanken}\\

\haiku{- Och, die gemeene,.}{Theologant ik weet zelf}{niet meer hoe hij heet}\\

\haiku{T is lichte maan.}{en op de sneeuw onderscheidt}{men gemakkelijk}\\

\haiku{En zij gehaat als,!}{trappenschuren Steeds zij zijn}{buidel zonder geld}\\

\haiku{Men vreest bij ons geen,.}{witte mouwen Wij smijten}{uit al wat ons knelt}\\

\haiku{de kerel is vast.}{een half uur te laat in de}{wereld gekomen}\\

\haiku{dat iemand joolig.}{en luchtig maakt zoodra}{hij er binnen komt}\\

\haiku{'T zij 'k avondrood,!}{of morgen zie Ik drink mijn}{glas Crambamboeli}\\

\haiku{Men bedenke slechts,!}{dat tijd hier niet staat voor dag}{of week maar voor maand}\\

\haiku{, zou het geen voortgang,.}{hebben gehad daar kunt gij}{verzekerd van zijn}\\

\haiku{, was het om vele.}{redenen zaak dat over dag}{vermeden werden}\\

\haiku{Flanor antwoordde met.}{een vreesselijken vuistslag}{op een Leidschen neus}\\

\haiku{Ge hebt immers zelf.}{gezien hoe de Leidenaars}{mij met steenen smeten}\\

\haiku{- You see how these,,,.}{fellows drink and smoke and}{roar replied Mr. Pickwick}\\

\haiku{Fest gemauert in.}{der Erden Steht die Form aus}{Lehm gebrannt}\\

\haiku{bij voorkeur neemt men,.}{C die reeds zoo dikwijls zijn}{hoofd heeft gestooten}\\

\haiku{s bekannt, Und ' '.}{wo ihrs packt da ists}{interessant}\\

\haiku{Naauwelijks bij de,.}{Stads-Gehoorzaal daar}{moest het er op los}\\

\haiku{En de vreemdeling:}{wijst er op met den vinger}{en zegt met deernis}\\

\haiku{Minerva, die op,:}{het Academiegebouw prijkt}{zijne schutsgodin}\\

\haiku{Gusje van Eijkens.}{beeldtenis teekent zich in}{de opening der deur}\\

\haiku{fluistert de deugniet.}{op plegtigen toon en den}{vinger voor den mond}\\

\haiku{dan liever zijne:}{schamelheid openlijk bekend}{en ronduit gezegd}\\

\haiku{- Ik zou gaarne mijn... -?}{testimonium hebben}{Van welk collegie}\\

\haiku{- Wat hebben we er!}{voor je ingezeten op}{sommige plaatsen}\\

\haiku{- Mijnheer heeft belet -:}{of Mijnheer schreeuwt zelf uit al}{zijne waardigheid}\\

\haiku{Maar wien van deze,?}{beide nu zal het gelden}{Lisse of ten Deyl}\\

\haiku{luidkeels uit, die het.}{naadje van de kous nog maar}{niet juist vinden kan}\\

\haiku{- Dat 's eene vervloekt.}{gemeene hatelijkheid}{op de jongelui}\\

\haiku{Maar dat tot aan uw,!}{dood mijn naam in uw gemoed}{Toch blijve wonen}\\

\haiku{Met een zoet lijntje.}{trachtte men hem weder naar}{binnen te krijgen}\\

\haiku{- Nog al! - En hoeveel?}{partijen worden er wel}{op een jaar gespeeld}\\

\haiku{zoekt hem vooral in.}{de laatste vaderlandsche}{gebeurtenissen}\\

\haiku{Vergunt uwen schrijver!}{een voorbeeld en wilt het hem}{ten goede houden}\\

\haiku{Dit alles weet ik,}{van alles wat gij mij daar}{tegenwerpt loochen}\\

\haiku{- En ben jij dan zoo?}{pedant van te denken dat}{je hier bent geweest}\\

\haiku{een glas jenever!}{aan mijnheer Bivalva voor}{mijne rekening}\\

\haiku{Van daar de twist, doch -!}{hij zal morgen wel met een}{bittertje of neen}\\

\haiku{Of het Studentje ', -!...}{t hoorde maar om er een}{ui op te zeggen}\\

\haiku{Ewoud brandt zich bij het;}{aansteken van zijn cigaar}{eene blaar op den neus}\\

\haiku{Velen overweldigt.}{de slaap en zitten stom als}{Egyptische beelden}\\

\haiku{De dageraad breekt.}{aan en nog duurt de nacht voor}{de feestvierders voort}\\

\haiku{Gusje van Yken met;}{de zijnen gaat ontbijten}{aan het Haagsche Schouw}\\

\haiku{Praat zoo tegen je,.}{sletten maar niet tegen een}{ordentelijk mensch}\\

\haiku{De een begraaft zich;}{onder boeken en sluit zich}{op tusschen muren}\\

\haiku{ik heb mijn tijd al}{staan te verbiljarden in}{den Paauw en ten zes}\\

\haiku{- meer dan gewone.}{veerkracht en liefde voor het}{Genootschap vereischt}\\

\haiku{Hoe ongelukkig -?}{dat door eene zekere hoe}{zal ik het noemen}\\

\haiku{hoe menigmaal men}{dezelfde phrase telkens}{op hare hielen}\\

\haiku{als er toch niet aan,!}{te doen is bedank dan ten}{minste in verzen}\\

\haiku{Dit herstelde hem.}{van zijne huivering en}{versterkt hief hij aan}\\

\haiku{Niemand onzer of,;}{hij heeft nog veel te leeren veel}{te verbeteren}\\

\haiku{Op elk bord lag \'e\'en;}{gebakken aardappel en}{\'e\'en gebraden ui}\\

\haiku{eene vete tusschen.}{de Buitengewone en}{Werkende Leden}\\

\haiku{want, indien iemand,}{hij was zwaar en moeijelijk}{ter  sprake.90}\\

\haiku{Verder, midden uit,;}{dien zwarten hoop verheft zich}{een boven allen}\\

\haiku{Wij bevelen ons,,.}{voor het vervolg in hare}{welwillendheid aan}\\

\haiku{Hier  niet meer de,;}{ruwste vuist die den schepter}{der heerschappij zwaait}\\

\haiku{Het is hier vooral,;}{het zedelijke overwigt}{dat gehuldigd wordt}\\

\haiku{Mevrouw Iburg kon men.}{slechts een weinig overdrijving}{te laste leggen}\\

\haiku{het zijn mannen en,;}{vrouwen die om den broode}{rollen opzeggen}\\

\haiku{Er moest een band zijn:}{tusschen den Hoogleeraar}{en den Muzenzoon}\\

\haiku{Dat nu deze soort!}{van Afleggers allen maar}{Bivalva's waren}\\

\haiku{waar den jongeling;}{de toegang steeds openstaat tot}{gezellig verkeer}\\

\haiku{Spoedig zongen het,,;}{tien spoedig honderd spoedig}{duizend straatjongens}\\

\haiku{{\textquoteright} Het gezelschap rukt,,.}{niet zonder gevolg de poort}{van Arnhem binnen}\\

\haiku{Men bevindt zich, naar,,.}{huis keerende voor een meer dat}{het Wielermeer heet}\\

\haiku{denk je, dat ik het,,!}{niet vervloek om voor jullie}{pleizier jan domie}\\

\haiku{zij zelve werken,;}{mede tot de betoovering}{die u overmeestert}\\

\haiku{Alzoo is ook de,,.}{tijd als zoodanig aan de}{muzijk vijandig}\\

\haiku{il appelle \`a;}{lui tous les enfans pauvres}{qui ont de la voix}\\

\haiku{zij doet stappen als,!}{een dragonder en wat zit}{zij dik in het vet}\\

\haiku{Gelieft mij slechts te.}{volgen bij het eindigen}{van het paardenspel}\\

\haiku{De wetenschap, het,,.}{gekozen vak dit zij de}{pit het middenpunt}\\

\haiku{je viendrai retremper!}{mon \^ame \`a l'abreuvoir de}{la gourmandise}\\

\haiku{Je t'assure.}{qu'il n'est pas facile de}{se faire juste}\\

\haiku{het ontbreekt mij aan.}{moed om de redoutables}{te negligeren}\\

\haiku{het is alles vorm,,,,:}{alles geest altijd lagchen}{stoeijen malligheid}\\

\haiku{il m\'eriterait.}{d'\^etre clou\'e \`a ter Gou ou}{\`a Egmondbuiten}\\

\haiku{Ah \c{c}a, mon cher, la.}{cloche sonne midi et}{mon caf\'e m'attend}\\

\haiku{Met regt mogt hij de,.}{verzekering geven dat}{hij niet dronken was}\\

\haiku{Me voil\`a \`a vingt-trois,.}{et tu n'es encore que}{vig\'esimaire}\\

\haiku{Une chaise de nuit,;}{et un gros bonnet fourr\'e}{voil\`a ce qu'il me faut}\\

\haiku{Il y avait plus de.}{trente estomacs au}{d{\^\i}ner doctoral}\\

\haiku{Mes saluts \`a tous.}{les amis et crois-moi}{quem nosti}\\

\haiku{De Dissertatie,.}{wordt gedrukt weldra zal de}{Promotie volgen}\\

\haiku{Mon cher ami,    Il;}{y a fort longtemps que je}{ne t'aper\c{c}ois plus}\\

\haiku{Loisir doublement.}{m\'erit\'e apr\`es tant de mois}{de piochage}\\

\haiku{roept Pluyx vrij luid, en.}{twintig stemmen herhalen}{dat heerlijke woord}\\

\haiku{- Och, zwijg toch, vriend, 't.}{is immers maar om onze}{dubbeltjes te doen}\\

\haiku{Ook is de soort van;}{geestigheid in het Verhaal}{hoogst oorspronkelijk}\\

\haiku{Het tweede couplet.}{van het eerste versje kan}{er naauwelijks door}\\

\haiku{De overbrenger had.}{er den oorspronkelijken}{geest uitgesneden}\\

\haiku{De twee volgende.}{coupletten zijn evenzeer fraai}{en vloeijend vertaald}\\

\haiku{- waart ge ellendig,,,;}{beroerd geesteloos ontbloot}{van alle talent}\\

\haiku{Geen liever is er, '.}{ooit ontmoet Geen trouwer ooit}{int minnen}\\

\haiku{Men moet zeggen, dat.}{we wonder wel aan onze}{bestemming voldoen}\\

\haiku{Gij, Redacteurs, zijt,,,.}{niet slecht niet onbeschoft niet}{lomp niet hatelijk}\\

\haiku{ik zou in ue.'s plaats;}{nooit den moed gehad hebben}{zoo iets te schrijven}\\

\haiku{De wetenschap, het,,.}{gekozen vak dit zij de}{pit het middenpunt}\\

\haiku{het is alles vorm,,,,:}{alles geest altijd lagchen}{stoeijen malligheid}\\

\haiku{Men heeft mij gezegd...:}{dat P mij gisteren is}{komen opzoeken}\\

\haiku{Deze vervloekte,!}{furore heeft mijn pen stomp}{gemaakt \'e\'en moment}\\

\haiku{ik snijd er een punt,,!}{aan een scherpe punt en wat}{voor een scherpe punt}\\

\haiku{Met regt mogt hij de,.}{verzekering geven dat}{hij niet dronken was}\\

\haiku{Voor een zin die niet,.}{van Bossuet afkomstig is}{is hij lang genoeg}\\

\haiku{Weer een jaar dat we.}{aan onze verzameling}{kunnen toevoegen}\\

\haiku{Is dandy-achtig?}{gedrag als het jouwe voor}{mij nog weggelegd}\\

\haiku{Ik zou willen dat.}{jij er een vertaling in}{versvorm van maakte}\\

\haiku{Naar uw stranden Hef ';}{k iedre avondstond mijn}{zegenende handen}\\

\haiku{De Dissertatie,.}{wordt gedrukt weldra zal de}{Promotie volgen}\\

\haiku{zo zul je zeggen, {\textquoteleft}.}{wat je alleen maar vanuit}{Parijs kunt schrijven}\\

\subsection{Uit: Studentenschetsen. Deel 2. Commentaar (onder ps. Klikspaan)}

\haiku{En de vreemdeling:}{wijst er op met den vinger}{en zegt met deernis}\\

\haiku{Op het omslag van, {\textquoteleft}{\textquoteright},:}{de tweede aflevering}{Wuftheid staat vermeld}\\

\haiku{Toen kwam ook het plan.}{op de afleveringen}{te illustreren}\\

\haiku{De student Schrijver,.}{tekenaar en lithograaf}{werkten nauw samen}\\

\haiku{Over de prent bij {\textquoteleft}Flanor{\textquoteright}:}{bijvoorbeeld schreef Ver Huell}{aan Kneppelhout}\\

\haiku{f 2,50 (Nieuwsblad voor,):}{den boekhandel 14 juni}{1860  Oplage}\\

\haiku{f 1,90 (gebonden) (,);}{Nieuwsblad voor den boekhandel}{4 oktober 1872}\\

\haiku{19 december 1873 (,);}{Nieuwsblad voor den boekhandel}{19 december 1873}\\

\haiku{Exemplaren daarvan.}{zijn tot op heden echter}{niet aangetroffen}\\

\haiku{het overgrote deel.}{van de lesuren werd besteed}{aan Latijn en Grieks}\\

\haiku{caf\'e-biljart,, ().}{aan de Nieuwe Rijn wijk 7}{nr. 27nu nr. 20}\\

\haiku{(Blok en Martin, De,-):;}{Senaatskamer p. 23}{Simplex eenvoudig}\\

\haiku{{\textquoteleft}violen{\textquoteright} - voor de).}{betaling van het gelag}{laten zorgen}\\

\haiku{\'e\'en voet telt twaalf duim.}{en is ongeveer dertig}{centimeter}\\

\haiku{de intocht van Jan ():}{van Beieren in Leiden}{1420   154ongewacht}\\

\haiku{Van Zonneveld, {\textquoteleft}Het{\textquoteright},):}{Leiden van Piet Paaltjens}{p. 15~210Waalboer}\\

\haiku{(Van Zonneveld, De,-):}{Romantische Club p. 81}{128~342den Burg}\\

\haiku{(Brom, Omkeer in 't,):}{studenteleven p. 63}{371novicius}\\

\haiku{- Kr... - maar ik zou geen -:}{politiek aanroeren en}{andere dassen}\\

\haiku{{\textquotedblleft}Kon ik maar een broek,!}{krijgen zoals jij ze in}{de Typen teekent}\\

\haiku{Organiek Besluit,):}{van 2 augustus 1815 art.}{84~258oosterling}\\

\haiku{Organiek Besluit,).}{van 2 augustus 1815 art.}{148 en 156       26}\\

\haiku{Organiek Besluit,):}{van 2 augustus 1815 art.}{56~389jura}\\

\haiku{(Schneider en Hemels,,) [...]:}{De Nederlandse krant p.}{150~16slaat een kout}\\

\haiku{Een nieuw treurspel over:}{de grootvorst ontlokte aan}{Beets de verzuchting}\\

\haiku{Organiek Besluit,) ':}{van 2 augustus 1815 art.}{205~467leit af}\\

\haiku{de Leidse kermis.}{duurde van hemelvaartsdag}{tot Pinksteren}\\

\haiku{want hetgeen ik wil,,,.}{dat doe ik niet maar hetgeen}{ik haat dat doe ik}\\

\haiku{er was in die tijd.}{geen hospes of hospita}{met deze naam}\\

\haiku{waarschijnlijk is het ().}{de voornaamVincent van de}{caf\'ebediende}\\

\haiku{wellicht is dat de.}{reden dat hij ruggelings}{is afgebeeld}\\

\haiku{Het was een dispuut {\textquoteleft}{\textquoteright},.}{voor deNieuwe letteren}{opgericht rond 1839}\\

\haiku{volksliedje op de {\textquoteleft}{\textquoteright},:}{wijs vanLes \'etudiants met}{als derde regel}\\

\haiku{21-22En de pet - Regt -, -:}{coquet Op \'e\'en oor Zwiert de}{breede straten door}\\

\haiku{{\textquoteleft}Daar ziet hij - maar met, (?), [...]{\textquoteright}.}{ernstige oogen/'t Hoofd met}{een Oreool omkranst}\\

\haiku{(Gids voor Leiden en,-):}{omstreken p. viii en}{7778~80Houri}\\

\haiku{Lady Macbeth draagt,, {\textquoteleft}{\textquoteright}.}{een kaars haar ogen zijn openbut}{their sense are shut}\\

\haiku{(Vademecum voor,-):}{den student p. 122123}{529kardinaal Puff}\\

\haiku{928-929dat zijne:}{pomp wel verstopt zou wezen}{van de haarpruiken}\\

\haiku{Het duel ({\textquoteleft}Mensur{\textquoteright});}{was een onder- ~deel}{van de erecode}\\

\haiku{Volgens Bilderdijk:}{wist Beyling welk gruwelijk}{einde hem wachtte}\\

\haiku{(Wetboek van het Strafregt,)-...}{p. 189 en 287~13781392Het}{doel zal toch wel zijn}\\

\haiku{de Leidse kermis.}{duurde van hemelvaartsdag}{tot Pinksteren}\\

\haiku{Klikspaan lijkt hier het:}{omgekeerde te zeggen}{van wat hij bedoelt}\\

\haiku{(De La Bruy\`ere,,):}{Oeuvres compl\`etes p. 478}{38wezenlijkheid}\\

\haiku{beslaat een deel van.}{de provincies Zuid- en}{Noord-Holland}\\

\haiku{9-10waar de vetste...:}{melk vloeiten de geurigste}{kaas bereid worden}\\

\haiku{der over elkander,:}{geslagene armen der}{duimpjesdraaijerij}\\

\haiku{met hun familie.}{wandelen en daar lelijk}{mee inzitten}\\

\haiku{289-298Wij sluiten in...}{geen kamermurenZoolang}{er wijn in flesschen}\\

\haiku{het zonder afspraak.}{door elkaar ondervragen}{van de studenten}\\

\haiku{eigenlijk een huis,.}{loods of schuur waar bokkingen}{gerookt worden}\\

\haiku{Organiek Besluit,):}{van 2 augustus 1815 art.}{87~23reizen}\\

\haiku{{\textquoteleft}Alle examina{\textquoteright}.}{zonder onderscheid moeten}{een vol uur duren}\\

\haiku{verwijzing naar de.}{mythe over de bruiloft van}{Peleus en Thetis}\\

\haiku{In deze schets speelt;}{het Academiegebouw een}{belangrijke rol}\\

\haiku{Openingswoord van de,.}{promovendus volgens een}{vaste formule}\\

\haiku{702-703als een god van:}{Homerus weggescholen}{in eene wolk van dauw}\\

\haiku{hij was gevestigd,, ().}{op de Koepoortsgracht wijk 2}{nr. 97nu nr. 34}\\

\haiku{verwijzing naar de:}{slotregel van het elfde}{couplet van psalm 118}\\

\haiku{1126-1127En een onzer:}{meest bekende geleerden}{het woord hybridisch}\\

\haiku{de verteller laadt;}{zijn pistool door een kogel}{in de loop te doen}\\

\haiku{Op 4 november.}{1839 was het opgevoerd in}{de Leidse schouwburg}\\

\haiku{1865-1866Want huwlijksheil:}{en vadervreugd/Boeit vaster}{dan een droom der jeugd}\\

\haiku{Vanaf 1839 werd de.}{redactie gekozen uit}{leden van het lsc}\\

\haiku{de daar genoemde)-:}{wet niet teruggevonden}{401402onder het zeil}\\

\haiku{(Lunsingh Scheurleer e.a.,,,-):}{Het Rapenburg dl. 1 p.}{271272~968vaak}\\

\haiku{Zij besloten de.}{dag met een groot feest op de}{soci\"eteit}\\

\haiku{Hij maakte reizen.}{door heel Europa en naar}{het Midden-Oosten}\\

\haiku{officier van de.}{met lansen bewapende}{cavalerie}\\

\haiku{Organiek Besluit,):}{van 2 augustus 1815 art.}{109 en 111~1490atoom}\\

\haiku{op 1 juni 1841.}{bracht koning Willem ii een}{bezoek aan Leiden}\\

\haiku{Chr.), het beroemde.}{boek over de retorica}{van Cicero}\\

\haiku{(Gids voor Leiden en,-).}{omstreken p. viii en}{p. 7778~      89}\\

\haiku{den tol, vreze, dien,,.}{gij de vreze eer die gij}{de eer schuldig zijt}\\

\haiku{Kevers sterven kort.}{na de bevruchting en het}{eieren leggen}\\

\haiku{Vluchtte na de val,.}{van Athene naar Perzi\"e waar}{hij werd vermoord}\\

\haiku{Chr.) bouwde voort op.}{de fundamenten die zijn}{vader had gelegd}\\

\haiku{respectievelijk.}{vergiffenis schenken en}{vergiftigen}\\

\haiku{[Namen der leden] [].}{volgens de orde waarin}{zij zittenZegel}\\

\haiku{de grootste beker,.}{de kleinere beker en}{de kleine beker}\\

\haiku{mottoQu'il est grand,...}{qu'il est beau de se dire}{\`a soi-m\^eme}\\

\haiku{Het accent lag nu.}{meer op het voordragen van}{werk van anderen}\\

\haiku{vanaf 1829 student),;}{theologie thesaurier}{of penningmeester}\\

\haiku{De ondergang der (-).}{eerste wareld18091810 van}{Willem Bilderdijk}\\

\haiku{De geschiedenis;}{van het gezelschap is goed}{gedocumenteerd}\\

\haiku{{\textquoteleft}violen{\textquoteright} - voor de).}{betaling van het gelag}{laten zorgen}\\

\haiku{144-145de almagt,:}{der kunst die steden bouwde}{en tijgers temde}\\

\haiku{uitbreiding van de {\textquoteleft}{\textquoteright} ().}{verwensingloop naar de maan}{en pluk starren}\\

\haiku{bepaalde pijpen,;}{bestonden uit een losse}{kop steel en mondstuk}\\

\haiku{Programma eerste:}{invitatieconcert sc}{286verdieping}\\

\haiku{354-356Beethoven [...] [...] [...],,,:}{Weber Rameau Mozart Gl\"uck}{Spohr Cherubini}\\

\haiku{De lezing {\textquoteleft}hunne{\textquoteright}.}{is ontleend aan de derde}{en vierde druk}\\

\haiku{verwijst mogelijk.}{naar discussies voorafgaand}{aan de dies van 1841}\\

\haiku{verkorte vorm van {\textquoteleft}{\textquoteright}, {\textquoteleft}{\textquoteright}.}{zetteden de verleden}{tijd vanzetten}\\

\haiku{verkorte vorm van {\textquoteleft}{\textquoteright}, {\textquoteleft}{\textquoteright}.}{zetteden de verleden}{tijd vanzetten}\\

\haiku{(Worp, Geschiedenis,,;}{van het drama en van het}{tooneel dl. 2 p. 377}\\

\haiku{Het stuk werd op 31.}{mei 1838 voor de eerste maal}{in Leiden vertoond}\\

\haiku{in mei 1843 trad hij.}{op ter gelegenheid van}{de Leidse kermis}\\

\haiku{Hij was de zoon van,.}{Ward Bingley een van de grootste}{acteurs van zijn tijd}\\

\haiku{bijlage 875 en):}{876~722ontstond er vrij}{wat minder nachtrumoer}\\

\haiku{De uitdrukking gaat (;}{terug op een passage}{in Tartuffe1669}\\

\haiku{eigenlijk een huis,.}{loods of schuur waar bokkingen}{gerookt worden}\\

\haiku{(Stokvis, De wording,-):}{van modern Den Haag p. 245}{246~397aanspraken}\\

\haiku{Organiek Besluit,):}{van 2 augustus 1815 art.}{104~436koesten}\\

\haiku{(Usener, {\textquoteleft}Maatschappij {\textquotedblleft}{\textquotedblright}{\textquoteright},;}{ijzergieterijDe prins}{van Oranje p. 389}\\

\haiku{Frans auteur (1806-1866),.}{van romans toneelstukken}{en feuilletons}\\

\haiku{het risico dat.}{hij zou worden nagevolgd}{was dus gering}\\

\haiku{Citaat uit Les chants ().}{du cr\'epuscule xiv1835}{van Victor Hugo}\\

\haiku{Zij mogen aan de [...]{\textquoteright}.}{huizen der professoren}{gehouden worden}\\

\haiku{verhaal van hetgeen,.}{elke dag voorvalt vooral}{van reizigers}\\

\haiku{Kopie\"en van het.}{beeld waren toen al over heel}{Europa verspreid}\\

\haiku{Voor het tweede deel.}{van de omschrijving is geen}{bron gevonden}\\

\haiku{Gerrit de Clercq (-),.}{18211857 vanaf 1839 student}{rechten te Leiden}\\

\haiku{n'interdis pas \`a:}{ma cendre les caveaux}{de Saint-Denis}\\

\haiku{Naar uw stranden Hef ';}{k iedre avondstond mijn}{zegenende handen}\\

\haiku{ook hier combineert;}{Klikspaan een wel en een niet}{bestaande senaat}\\

\haiku{Digesta, p. 860),:}{222Ornatissime quaenam}{fuerunt ultima}\\

\haiku{(Anoniem, Aballino,,):}{de groote bandiet p. 61}{7eigenaardigheid}\\

\haiku{[Anoniem], {\textquoteleft}Goethe en{\textquoteright}.}{eenige zijner beroemdste}{tijdgenooten}\\

\haiku{Dyserinck, J.,.}{Het studentenleven in}{de literatuur}\\

\haiku{2 dln. Paris, z.j.,, {\textquoteleft}.}{Heemskerk G.De bloem van de}{Leydsche academie}\\

\haiku{Gedenkboek van het.}{Collegium classicum}{cui nomen M.F. cond}\\

\haiku{Revue de Paris,, (),-.}{seconde \'edition dl.}{vi1834 p. 101116}\\

\haiku{Twee en veertigste.}{jaarboek van het genootschap}{Amstelodamum}\\

\haiku{Lochem, 1970 Koppen,,.}{C.A.J. van De geuzen van de}{negentiende eeuw}\\

\haiku{Nijland, J.A., Leven (-).}{en werken van Jacobus}{Bellamy17571786}\\

\haiku{Schulze, F./P. Ssymank,}{Das Deutsche Studententum}{van den aeltesten}\\

\haiku{Een zedekundig.}{tafereel uit het begin}{der vijftiende eeuw}\\

\haiku{2e vermeerderde, ',,.}{dr. z.p. z.j. Vissert Hooft}{H.P. De student Beets}\\

\haiku{88, 212, 360 Alexander,:}{Nikolajevitsj Grootvorst}{van Rusland   i}\\

\haiku{617 ii: 30, 345, 347,,,():}{424 544 Clercq Mathurin}{Joseph le   i}\\

\haiku{128, 141 Brieven over:}{den aard en de strekking van}{hooger onderwijs}\\

\haiku{664 ii: 175, 269, 415,,,,, (:}{504 525 557 573 Scriblerus}{Martinuszie ook}\\

\haiku{54Mededeling ().}{op het omslag van Leven}{v bis1 april 1842}\\

\haiku{uba 2350 h 9-10,.}{102Leidsch Dagblad 11 en 15 april}{1868 en 16 mei 1868}\\

\section{J.M.W. Knipscheer}

\subsection{Uit: De blauwe draak}

\haiku{Blijkbaar zie jij kans,.}{om nog iets van deze zaak}{terecht te brengen}\\

\haiku{Denkt u er w\`el om {\textquoteleft}{\textquoteright}!}{dat het woordGenade voor}{ons niet bestaat}\\

\haiku{En ze toonde mij,.}{het stofblik volgeladen}{met doode muggen}\\

\haiku{Al mijn spieren en.}{zenuwen waren nu op}{het hoogst gespannen}\\

\haiku{- Over een drietal uren,.}{zal het mij geoorloofd zijn}{u toe te laten}\\

\haiku{gezicht ging ik weer {\textquoteleft}{\textquoteright}.}{zitten enverdiepte mij}{weer in mijn lectuur}\\

\haiku{Ik volgde ze, bij, '.}{mezelf concludeerend datt}{wel eens mis kon zijn}\\

\haiku{In dat geval liet,!}{die andere car het niet}{op zich zitten hoor}\\

\haiku{Ik zei niets tegen.}{de anderen en wachtte}{nog maar eens even af}\\

\haiku{En ik vertelde.}{mijn wederwaardigheden}{op den terugweg}\\

\haiku{op je, omdat je!}{zoo'n stevig pantser tegen}{zijn aanvallen hebt}\\

\haiku{meestal komen;}{de hoofdpersonen zich al}{uit zichzelf melden}\\

\haiku{Het pistool gleed in.}{mijn zak en ik wendde mij}{nu naar het bureau}\\

\haiku{Zonder meer haastte.}{ik mij naar de woning van}{den betrokkene}\\

\haiku{barstte hij \'even los,.}{zoodat alle omstanders ons}{ineens aankeken}\\

\section{anoniem}

\subsection{Uit: 'Het dagverhaal van een onbekende. Een Gouds dagboek uit het jaar 1788 en later'}

\haiku{{\textbullet} De {\textquoteleft}Waarschouwing{\textquoteright} van--.}{07021788 staat ook in het}{Publicatieboek}\\

\haiku{de leede van de ',;}{oraniesocitijd voort Harthuijs}{meede gewapent}\\

\haiku{97Koestraat, thans is ().}{dit de Markthet deel tussen}{Groenendaal en Hoogstraat}\\

\haiku{de tegenstanders),.}{van elkaar te scheiden uit}{elkaar te houden}\\

\haiku{105Voluit is de.}{naam van dit regiment Bosc}{de la Calmette}\\

\section{Marie Koenen}

\subsection{Uit: Het hofke}

\haiku{{\textquoteright} Willem had vluchtig.}{omgezien naar Sanderkens}{kunstenmakerij}\\

\haiku{Maar niet lang liet hij,.}{zijn zachte zegenende hand}{op Milia's hoofd}\\

\haiku{{\textquoteright} 't Was Willem, die, '.}{daar voor haar stond voort eerst}{na die \`eenen avond}\\

\haiku{Vaag schemerden er.}{de muren van het Hofke}{tusschen de boomen}\\

\haiku{{\textquoteright} Grave was met zijn.}{sjees stedewaarts geweest voor}{den paardenhandel}\\

\haiku{{\textquoteleft}Terstegen, hoe oud,.}{is die dochter van jou ik}{bedoel je voorkind}\\

\haiku{Hij geeft den kleinen.}{buurman een gemoedelijk}{tikje op het hoofd}\\

\haiku{En toch is het de,.}{vergiffenis verzoenend}{en vereffenend}\\

\haiku{{\textquoteright} {\textquoteleft}Wel dan, ik zal h\`aar!}{de deur van het Hofke niet}{gesloten houden}\\

\haiku{{\textquoteleft}Hij slaapt{\textquoteright}, beduidt hij,.}{zijn kameraad wenkend en}{wijzend met den blik}\\

\haiku{wat bleek, wat ernstig,.}{en stil en schijnt te droomen}{van verre dingen}\\

\haiku{Er vlamt iets op in,.}{zijn gedachten zijn hoofd wordt}{warmer en warmer}\\

\haiku{Waarom zouden wij?}{samen geen nieuwen tijd op}{het Hofke brengen}\\

\haiku{Zij bestond alleen,.}{al het overige verzonk}{in nevels voor hem}\\

\haiku{En nu die Thielde,!}{met haar aanstellerij en}{haar brutale oogen}\\

\haiku{{\textquoteleft}Wie had gedacht, dat?}{de dingen nog zoo'n goeden}{keer zouden nemen}\\

\haiku{Op het Hofke zat;}{Milia dien middag voor het}{wijd-open venster}\\

\haiku{En had ze ooit zooals,:}{nu kunnen begrijpen wat}{het beduiden wil}\\

\haiku{Hij had verwijten,.}{verwacht uitbarstingen van}{verdriet bij Milia}\\

\haiku{dat de beesten ziek,,;}{en dood ook al in den stal}{werden gevonden}\\

\haiku{Daar ging hem dan ten!}{laatste het Hofke toch nog}{in handen vallen}\\

\haiku{Hij durfde er niet,}{over spreken dat zijn vader}{hem gezonden had}\\

\haiku{{\textquoteleft}Het is niet om mij,{\textquoteright},.}{het is om de kinderen}{zei hij nog zachter}\\

\haiku{Ze heeft nooit zoo op.}{haar moeder geleken als}{in dit oogenblik}\\

\haiku{Het huis van Sander.}{en Thielde lag in dezen}{tijd reeds verlaten}\\

\subsection{Uit: De korrel in de voor}

\haiku{Onderwijl was op.}{Garversberg de heele buurt}{al in opschudding}\\

\haiku{{\textquoteright} Plonia joeg Paulus.}{en den oudste van Ruiters}{haar zoon achterna}\\

\haiku{Zwarte Marjan, om, ';}{met geen tang aan te pakken}{enr drie jongens}\\

\haiku{{\textquoteleft}Niet toegapen, maar,{\textquoteright}.}{bidden allemaal samen}{beheerde Drikus}\\

\haiku{Zelfs de belhamels.}{van Zwarte Marjan trokken}{de pet van het hoofd}\\

\haiku{een dochter van de.}{baronnen van Laag Case}{met den koetsierszoon}\\

\haiku{Dat wist hij toch wel, '?}{waart bij den rentmeester}{van den baron was}\\

\haiku{Alzoo - bij den heer.}{notaris zeggen juist zooals}{bij de anderen}\\

\haiku{Waar was dan ook z'n?}{goede moed gebleven van}{midden op den dag}\\

\haiku{Daar stond hij opeens, '.}{recht met een schok omt van}{zich af te schudden}\\

\haiku{{\textquoteright} Zoo pleitte Nelis,,.}{voor zichzelf barmhartig z'n}{doen met den hond goed}\\

\haiku{- Zou Nelis Broens soms, '?}{niet weten hoet behoort}{op een heerenhof}\\

\haiku{al zoo vroeg met de,...}{melk in het voorhuis juist als}{zij naar de kerk ging}\\

\haiku{Daar zullen we dan,,!}{om bidden moeder om zoo'n}{bouwknecht uit duizend}\\

\haiku{De beslissing hoeft.}{immers niet vandaag nog te}{worden genomen}\\

\haiku{Na Amen en kruisteeken ':}{was het eerste woord vant}{Rosalien voor h\'em}\\

\haiku{Ze had zijn naam zelfs.}{nog niet voor haar eigen en}{niet voor God genoemd}\\

\haiku{'k Zou je zoo graag '.}{bij de Zusters van Overdael}{int klooster zien}\\

\haiku{Niks hoef je te doen,,:}{dan er te vragen wat wij}{er wilden vragen}\\

\haiku{{\textquoteright} Dat was al, wat hij '.}{aant Rosalien terug}{had te boodschappen}\\

\haiku{{\textquoteright} {\textquoteleft}Maar, - je hebt voor ons?}{je plaats bij de Zusters toch}{niet opgegeven}\\

\haiku{{\textquoteright} Wel, zie, - dacht Nelis -.}{of die Leonardus ook}{oogen in z'n kop heeft}\\

\haiku{En echt blij was ze,:}{haar met een gerust hart te}{kunnen antwoorden}\\

\haiku{Een woord te weinig,,.}{kind is dikwijls nog erger}{dan een woord te veel}\\

\haiku{dat het heele dorp...}{blij mocht zijn met zoo'n aanwinst}{als Leonardus}\\

\haiku{Vandaag klonk hem haar.}{hartewoord als een lach en}{een snik tegelijk}\\

\haiku{En toch - ze kon er.}{zich hoegenaamd niks goeds meer}{van voorspiegelen}\\

\haiku{{\textquoteleft}Wat eenmaal den stoot,...}{heeft gekregen bergaf blijft}{naar omlaag rollen}\\

\haiku{Zooals Leonardus.}{dat zoo goed onthouden had}{uit het Evangelie}\\

\haiku{Nelis stond, eer hij ',.}{t wist aan de schuurdeur naar}{Peereneer uit te kijken}\\

\haiku{{\textquoteleft}Dat ziet er hier nu!}{toch anders uit dan omtrent}{Allerheiligen}\\

\haiku{Almaar berg-op,,.}{berg-af als een blind paard}{in den tredmolen}\\

\haiku{{\textquoteleft}Mariajoos nog toe, '!}{alst hem maar niet in den}{kop is geslagen}\\

\haiku{{\textquoteright} Opgewonden sloeg,.}{ze zich door den oploop heen}{recht op den bergrand aan}\\

\haiku{Den varkensketel.}{vullen met koolbladen en}{stronken en afval}\\

\haiku{Had die stiekemerd '?}{vann Neliske daar den}{wind van gekregen}\\

\haiku{Hij kwam recht en dronk,,,.}{gehoorzaam zwolg zoo vlug als}{eenigszins mogelijk}\\

\haiku{Met den rug van haar.}{hand streek ze zich schichtig de}{tranen uit de oogen}\\

\haiku{Leonardus stond.}{hem midden op den zolder}{al op te wachten}\\

\haiku{{\textquoteright} - Al zou hij er ook.}{alleen in de uiterste}{noodzaak aan roeren}\\

\haiku{Als ge ze nou nog...}{met u twee\"en naar boven}{zoudt willen dragen}\\

\haiku{{\textquoteleft}Franciscus, ga, en,,.}{bouw Mijn kerk weer op die in}{puin valt zooals ge ziet}\\

\haiku{Hij sloeg het dek op,,.}{en daar stond moeder Plonia}{ook met een lantaarn}\\

\haiku{nou is de weg naar.}{Garversberg voortaan niet meer}{te ver voor die twee}\\

\haiku{het zei, toch wees hij.}{met een los gebaar den kant}{van Laag Case uit}\\

\haiku{{\textquoteright} Het eigenlijke,:}{van de Derde Orde dat}{wist Nelis nu wel}\\

\haiku{Mijnheer pastoor bij,...}{ons zal er mettertijd wel}{een weg op weten}\\

\haiku{{\textquoteleft}Geef, goede God, dat...{\textquoteright}}{het tweede weer een zoon mag}{zijn Den volgenden}\\

\haiku{Heil en zegen bracht,.}{ze mee beweerden Plonia}{en de buurvrouwen}\\

\haiku{- {\textquoteleft}Vijf stappen vaneen,,...}{en niet eens te kunnen gaan}{zien of ze goedligt}\\

\haiku{En zelfs Ferdinand,,.}{stond het span als een kenner}{te bewonderen}\\

\haiku{Ze probeerde 't,.}{wel vooral ook om hem wat}{op te vroolijken}\\

\haiku{{\textquoteright} {\textquoteleft}Als we Nelis maar '?}{vastns om de vigilant}{naar Overdael stuurden}\\

\haiku{Maar daarmee ook - tot -!}{den laatsten halven cent voor}{Bella geblazen}\\

\haiku{Nelis was al den,,...}{inrij uit handen in de}{zakken blik ver weg}\\

\haiku{Want wat zou 'k daar, '?}{kunnen zeggen zonder dat}{gijt me voorzegt}\\

\haiku{Bij de bocht klom-ie,, -.}{den hoogen kant op om den weg}{te overzien naar links}\\

\haiku{Mijn jagershoed met.}{het fazantenpluimke zou}{hem beter kleeden}\\

\haiku{Ons den oudste nou?}{al en voorgoed heelemaal}{afhandig maken}\\

\haiku{- {\textquoteleft}Hij kijkt je naar de, '.}{oogen en ziet meteen oft}{meenens is of niet}\\

\haiku{Maaien... {\textquoteleft}Achter mij,{\textquoteright}.}{blijven gebood Plonia den}{ganschen morgen door}\\

\haiku{Ze was zelfs nog nooit.}{zoo zwijgzaam geweest als van}{dit oogenblik af}\\

\haiku{onmogelijk van '.}{Truuke te kunnen houden als}{n man van z'n vrouw}\\

\haiku{s Maandags nog voor ':}{den middag vernam hijt}{al van z'n moeder}\\

\haiku{{\textquoteright} En v\'o\'or den eten, in,:}{de volle achterkeuken}{Truuke zelf al direct}\\

\haiku{{\textquoteright} - Iets anders wilde.}{er niet in z'n gedachten}{opleven voor haar}\\

\haiku{Dat k\'on Nelis toch, '.}{den schoft z'n verachting in}{t gezicht blazen}\\

\haiku{Over het kozijn heen - -!}{zag hij een bleeke schim in den}{sterrenschemer Truuke}\\

\haiku{- Zonde voor God zou ',.}{t zijn hem in z'n roeping}{tegen te werken}\\

\haiku{- enkel en alleen,.}{op kattekwaad gespitst geen}{moment opletten}\\

\haiku{Te zien, hoe er niets...}{was dan volop reden tot}{overgroote dankbaarheid}\\

\haiku{Uit den hoofddoek keek '.}{r gezicht ouwelijk bleek}{en spits geworden}\\

\haiku{{\textquoteright} {\textquoteleft}Nog geen vier maanden,{\textquoteright} '.}{meer kniktet Rosalien}{haar veelzeggend toe}\\

\haiku{of die genegen..}{is zijn part van Garverskamp}{aan ons over te doen}\\

\haiku{Dat had Gradus hem '.}{bijt eerste woord evengoed}{willen toetellen}\\

\haiku{{\textquoteright} - {\textquoteleft}Vraag dat maar 'ns op,!}{den Bulthoek waar Nelis Broens}{goed genoeg voor is}\\

\haiku{God in den Hemel:}{en dien naasten voorspreker}{van hem daarboven}\\

\haiku{Het derde drietal,:}{in den stoet ze dienen meer}{tot opluistering}\\

\haiku{Onder de kanten,:}{doopdekens uit worden ze}{te voorschijn gebracht}\\

\haiku{- Tot eindelijk ook,...}{die donkere oogen opengaan}{groot en verwonderd}\\

\haiku{Toch weet Nelis wel,.}{hoe hij siddert en beeft voor}{dien woesten Colla}\\

\haiku{{\textquoteright} zegt 't Reeke, en strekt.}{z'n rechterhandje al uit}{om hem te aaien}\\

\haiku{de tortels van Sint,{\textquoteright}.}{Siskus en hij wijst omhoog}{het hellingbosch in}\\

\haiku{Verleden zomer.}{eerst is het andere er}{tusschengekomen}\\

\haiku{Al wekenlang zie ' '.}{k uit om jouns alleen}{te kunnen spreken}\\

\haiku{Als ze vast maar aan '..}{t wankelen zijn in hun}{vertrouwen op mij}\\

\haiku{- {\textquoteleft}Loop hard, Trinette.}{vragen er stevig een nat}{verband om te doen}\\

\haiku{{\textquoteleft}Ik kom v\'o\'or den avond ',?}{ns even hooren wat er toch}{aan de hand is hier}\\

\haiku{{\textquoteleft}Je doet of hij wel...}{de grootste misdaad van de}{wereld heeft begaan}\\

\haiku{Maar aldoor bleef 'k,.}{hopen dat het ware toch}{nogwel komen zou}\\

\haiku{ten opzichte van;}{Colla meer en meer gestijfd}{in z'n tegenzin}\\

\haiku{Met Ferdinand daar.}{en Trinette van u als}{tusschenpersonen}\\

\haiku{{\textquoteleft}Laat mij nou eerst 'ns,{\textquoteright},.}{uitspreken Wevers kalmeert}{hem de bezoekster}\\

\haiku{ik heb ditonderwerp.}{tot nog toe maar liever niet}{willen aanroeren}\\

\haiku{geen stroobreed zal ze.}{het geluk van haar heerke}{in den weg leggen}\\

\haiku{Haar eigen portret!}{in een breloque voor z'n}{horlogeketting}\\

\haiku{{\textquoteright} {\textquoteleft}Die Zwaan doet opgeld!}{tegenwoordig voor de twee}{boezemvriendinnen}\\

\haiku{En 't Rosalien,:}{is echt opgelucht dat ze}{er uit kan brengen}\\

\haiku{Al zal hij hun wel, -!}{de ooren van het hoofd eten}{zoo'n reus in z'n groei}\\

\haiku{Je weet nogwel. - {\textquoteleft}Naar,{\textquoteright} {\textquoteleft} '! - -?}{moeder dacht ikoft komt}{er van En dan zij}\\

\haiku{Als de meester ze,}{maar weer allemaal rondom}{zich ziet de zijnen.\ensuremath{\therefore}}\\

\subsection{Uit: De moeder}

\haiku{Ja, ze kan met een.}{vrij en gerust hart op haar}{leven terugzien}\\

\haiku{Tila begon mee.}{te naaien en het werk vloog}{hun uit de handen}\\

\haiku{{\textquoteright} {\textquoteleft}En Manders zou bij{\textquoteright}.....}{den bovenmeester Kroes een}{goed woord kunnen doen}\\

\haiku{Door 't eerste raam,,.}{dat van den winkel schijnt wat}{flauw licht den weg over}\\

\haiku{Moeder Severiens,.}{talmt verwonderd opgeschrikt}{uit haar gemijmer}\\

\haiku{De winkeldeur staat,,.}{open en achter den winkel}{de keukendeur ook}\\

\haiku{{\textquoteright} Er flitst een booze vlam '.}{uit Tila's oogen en het bloed}{slaat haar naart hoofd}\\

\haiku{{\textquoteleft}Ge moet beginnen,.}{met morgen naar Manders te}{gaan tusschen schooltijd}\\

\haiku{Nu met Paschen.}{was er over dat vergane}{loof een dunne rijm}\\

\haiku{Tila heeft al haar.}{gebeden gepreveld en}{toch niet gebeden}\\

\haiku{Zie, nergens is de.}{Geul zoo vreedzaam en gedwee}{als hier in Vlake}\\

\haiku{Of ze aldoor d'r.}{uitval van dien Aprilavond}{wilde goedmaken}\\

\haiku{Ik heb nu gezien - '.}{hoe dat kann nevel van}{goud hing door het dal}\\

\haiku{D'r tanden blinken.}{tusschen de in trotschen lach}{trillende lippen}\\

\haiku{Nooit heeft ze zoo scherp.}{alles gehoord en gezien}{als dezen middag}\\

\haiku{Moeder Severiens '.}{bukt naarn klaproos tusschen}{de schriele halmen}\\

\haiku{Maar losjes wippend:}{op z'n voetzolen lacht hij}{ineens luchthartig}\\

\haiku{{\textquoteright} {\textquoteleft}Leonie{\textquoteright} maant d'r, {\textquoteleft}, '.}{moederkom liever binnen}{t wordt onze tijd}\\

\haiku{Tot Sint-Jan{\textquoteright} mompelt,.}{Leonie hem toe terwijl}{ze de deur uitgaan}\\

\haiku{{\textquoteright} {\textquoteleft}Die heb ik ook{\textquoteright} poogt, '.}{het meisje te lachen maar}{t is als een snik}\\

\haiku{{\textquoteleft}'k Moet gaan{\textquoteright} ineens.}{verlegen trekt Treeske zich}{terug naar de deur}\\

\haiku{{\textquoteleft}Jongen, nu raad 'ns,?}{wie er is geweest wie daar}{op den stond wegging}\\

\haiku{{\textquoteright} hervat ze bijna,.}{verwijtend nu ze de lamp}{heeft aangestoken}\\

\haiku{{\textquoteleft}Ge moet met Louis naar.}{Berghof om uw huwelijk}{bekend te maken}\\

\haiku{Hier, je krijgt op den '.}{koop toen suikerboon om}{op te sabbelen}\\

\haiku{{\textquoteleft}Is 't waar - hebt gij,?}{dat allemaal klaargemaakt}{madame Curvers}\\

\haiku{'t Zingt in hun hart,,.}{maar ze spreken niet droomen}{maar en luisteren}\\

\haiku{In de gang en in.}{de voorzaal gaat het roezen}{en rumoeren voort}\\

\haiku{Jules' moeder legt.}{haar warme hand op die van}{Treeske in haar schoot}\\

\haiku{En Jules zoekt en ',.}{zoekt naart rechte woord nu}{weet niet wat en hoe}\\

\haiku{{\textquoteleft}Goddank{\textquoteright} verademt, '.}{moeder Severiens als ze}{opt plein komen}\\

\haiku{Zij stappen snel en.}{schuw het donker in van de}{Daelhoverstraat}\\

\haiku{Dat hij zich nu in,.}{Godsnaam toch sterk houdt zorgt haar}{alles te zeggen}\\

\haiku{t Gesprek gaat voort,.}{deint telkens tusschen gescherts}{en bespiegeling}\\

\haiku{Hier in de keuken ',.}{konden wet hooren of}{we er bij waren}\\

\haiku{Ze zijn op den spronk.}{en treden haastig op de}{twee wachtenden toe}\\

\haiku{t Lijkt Jules of.}{ze zich schreiend aan z'n hart}{wil komen bergen}\\

\haiku{{\textquoteright}.... Heeft hij die woorden?}{van het Hooglied  op z'n}{viool gezongen}\\

\haiku{Moeder Severiens'.}{gedachten zwerven weg in}{een zonnigen droom}\\

\haiku{t Ergste is dat.}{wij daardoor op schrikkelijk}{hooge lasten zitten}\\

\haiku{{\textquoteright} {\textquoteleft}Ge hadt het aan uw.}{moeder eerder en anders}{dienen te zeggen}\\

\haiku{En toch zal ze voor!}{de zooveelste maal er maar}{weer overheen moeten}\\

\haiku{Altijd dat studeeren,,.}{en sterk is hij nooit geweest}{nooit als anderen}\\

\haiku{hij zit verscholen,.}{de voeten langs de lage}{oeverglooiing neer}\\

\haiku{Nu eerst voelt ze hoe,.}{koud en bevend z'n handen}{zijn hoe klam z'n kleeren}\\

\haiku{Natuurlijk dat hij.}{in z'n koortsdroom juist over dit}{alles bezig was}\\

\haiku{Vaal-blauw komt het}{vroegste schemeren van den}{nieuwen dag over hem}\\

\haiku{{\textquoteright} Moeder Severiens,.}{weet niet waarom ook haar spraak}{ineens zoo beklemt}\\

\haiku{Nu de andere,,:}{haar aanziet wachtend op het}{eindwoord zegt ze stroef}\\

\haiku{Heel z'n lichaam is.}{in siddering en door z'n}{hoofd warrelt een storm}\\

\haiku{Z'n laatste maanden.}{in Vlake sprak Jules geen}{woord meer over Treeske}\\

\haiku{Is't de eenzaamheid?}{die haar in korte maanden}{zoo heeft veranderd}\\

\haiku{Er liggen 'n paar,.}{dorre olmblaren die ze}{met den voet wegschuift}\\

\haiku{{\textquoteright} Louis z'n oogen en z'n.}{mond blijven gesperd in een}{verdwaasden glimlach}\\

\haiku{Neen, Jules zal haar!...}{niet heelemaal als een oud}{mensch terugvinden}\\

\haiku{Haastig schuift ze 't.}{takje tusschen twee pakken}{wol in de muurkast}\\

\haiku{Ze ziet nu eerst dat.}{d'r moeder met starende}{oogen zit te schreien}\\

\haiku{En waarom niet trotsch,....}{op haar jongen dien de ernst}{uit de oogen donkert}\\

\haiku{Maar bij z'n zware....}{stem schudt moeder Severiens}{afwerend het hoofd}\\

\haiku{Dat zal haar goeddoen{\textquoteright},.}{begint dan ineens tegen}{Dolfke te spelen}\\

\haiku{Maar Treeske heeft met:}{sluippassen een stoel gehaald}{en prevelt haar toe}\\

\haiku{{\textquoteleft}Waarom zou ik 't, '?}{niet goedvinden als zijt}{zoo graag zou hebben}\\

\haiku{Moeder Severiens.}{ligt stil met opgewend hoofd}{en gesloten oogen}\\

\subsection{Uit: Wassend graan}

\haiku{{\textquoteright} Tot haar gelukkig:}{groote Maria en Lucia in}{duo te hulp komen}\\

\haiku{{\textquoteright} z'n schoonmoeder kan ':}{niet laten ook een duit in}{t zakje te doen}\\

\haiku{We hebben 't hier,.}{waarachtig best naar onzen}{zin moeder Wevers}\\

\haiku{Haar {\textquoteleft}Heerke{\textquoteright} is er,.}{voor in de wieg gelegd dat}{weet z'n moeder wel}\\

\haiku{Frans verschrikt van de.}{schrille jaloezie in de}{stem van z'n meisje}\\

\haiku{ze spelt hem dit als ',.}{t ware voor en hij moet}{het wel herhalen}\\

\haiku{{\textquoteright} - {\textquoteleft}Vraag dat liever maar ',{\textquoteright}.}{ns aan Frans wijst Anneke}{haar vinnig terecht}\\

\haiku{{\textquoteleft}Moest ik dien braven?}{heerenknol hier soms van dorst}{laten versmachten}\\

\haiku{Aanvankelijk had:}{z'n aanstaande schoonmoeder}{daar veel op tegen}\\

\haiku{Lucia monter als, '.}{nog van-haar-leven niet met}{n kleur van plezier}\\

\haiku{Je hebt 't nou zelf,{\textquoteright},.}{bijgewoond begint hij z'n}{stem nog niet meester}\\

\haiku{{\textquoteleft}Wie zou dien deugniet,?}{in toom moeten houden als}{ik er niet meer was}\\

\haiku{{\textquoteright} {\textquoteleft}Dus in elk geval '.}{is hij daar nogn jaarlang}{goed opgeborgen}\\

\haiku{{\textquoteleft}Ik weet wel, 't is, -.}{niks voor jou de dochter van}{den burgemeester}\\

\haiku{{\textquoteright} Plotseling stelt hij,!}{zich ook schrap met even veel recht}{toch zeker als zij}\\

\haiku{{\textquoteright} 't Ontbreekt Frans aan.}{kalmte en kracht om op haar}{woorden in te gaan}\\

\haiku{{\textquoteright} Om hem te sparen, ' ':}{wilt Rosalient hem}{niet laten hooren}\\

\haiku{- Maar dat kan ik bij.}{jou immers heelemaal niet}{veronderstellen}\\

\haiku{{\textquoteleft}En graag of niet, u.}{zult mij moeten dulden als}{meester op den Hof}\\

\haiku{Allesbehalve, '!}{een parmante schalk maarn}{wijsneus in persoon}\\

\haiku{'t Lijkt haar eerder.}{of er een eeuwigheid ligt}{tusschen toen en thans}\\

\haiku{t Rosalien weet,.}{genoeg dat er in het dorp}{om gelachen wordt}\\

\haiku{{\textquoteleft}Is me dat nou 'n,!}{partuur voor ons Anneke}{die jonge Wevers}\\

\haiku{Als ik je zeg, dat! '}{de ware Jozef al op}{haar te wachten staat}\\

\haiku{Jong als Frans is, en.}{vol illusies als hij was}{over z'n Anneke}\\

\haiku{- Maar nou wordt het dan,.}{toch meer dan tijd er mee voor}{den dag te komen}\\

\haiku{- Voor 't eerst van haar.}{leven op Zondag niet naar}{de Heilige Mis}\\

\haiku{Maar ja, dezen keer '.}{ist natuurlijk om dat}{lied van de bruiloft}\\

\haiku{Ondanks z'n negen.}{kinderen heeft hij dat nooit}{mogen beleven}\\

\haiku{{\textquoteright} {\textquoteleft}Alsof ik ookmaar '!}{n halfuur met dien vlegel}{overweg zou kunnen}\\

\haiku{{\textquoteright} {\textquoteleft}Willen is kunnen,, -?}{Frans dat hoeft je moeder je}{toch niet meer te leeren}\\

\haiku{Deze Cecile.}{Steeg scheen nou opeens Rita}{te hebben ontdekt}\\

\haiku{{\textquoteleft}Al van negen uur,.}{op weg winkelwaar voor me}{halen in Overdael}\\

\haiku{Al zou dat er nog, '.}{van komen alst zoo moest}{aanhouden met mij}\\

\haiku{Van dezen middag,,{\textquoteright}.}{te beginnen als je wilt}{zet Gregoire door}\\

\haiku{{\textquoteright} - Waarin heer Frans dan.}{niets dan wrange jaloezie}{meende te hooren}\\

\haiku{{\textquoteleft}Kom zeg, bemoeien.}{jullie je liever alleen}{maar met de poppen}\\

\haiku{Gejaagd streek ze het,.}{voorhuis in terwijl hij haar}{hardop uitlachte}\\

\haiku{Geen wonder dus, dat '.}{zet hoe langer hoe meer}{bij elkaar zochten}\\

\haiku{Wezenlijk, ik ben.}{niet van kraakporcelein zooals}{Anneke Reinders}\\

\haiku{{\textquoteright} Wat ze feitelijk,,}{bedoelde begreep Frans niet}{goed maar meer en meer}\\

\haiku{Allebei hadden.}{ze den adem ingehouden}{om te luisteren}\\

\haiku{hoe bitter weinig.}{uithoudingsvermogen hij}{eigenlijk nog had}\\

\haiku{Ga  'ns naar h\`em, -.}{toe laat dat uw eerste gang}{zijn op eigen beenen}\\

\haiku{Want immers juist iets,:}{voor Anneke hem zoo fijn}{te laten verstaan}\\

\haiku{En als jij er nou,... ':}{ook bijkomt zoomaar vanzelf}{t zou zoo goed zijn}\\

\haiku{opeens alles weer,...}{in orde en op z'n ouds}{ook met Trinette}\\

\haiku{Je kunt niet gelooven, '.}{watn aantrekkingskracht zoo'n}{kind in de wieg heeft}\\

\haiku{M'n geweten heeft!}{me nou lang genoeg gekweld}{over die eereschuld}\\

\haiku{{\textquoteright} - {\textquoteleft}Allemaal goed en, '!}{wel maar veels te veel ineens}{voorn eersten gang}\\

\haiku{Verwijten zal ze, '.}{hem niets enkel hem maar op}{n afstand houden}\\

\haiku{{\textquoteleft}respect{\textquoteright}, en {\textquoteleft}hoe jij '{\textquoteright}...}{voor ons allemaal t\`ochn}{vader bent geweest}\\

\haiku{{\textquoteleft}Als nonk Nelis niet,,!}{naar hem komt komt hij naar nonk}{Nelis die van ons}\\

\haiku{Immers keer na keer:}{werd hem dat opgelegd in}{den loop der jaren}\\

\haiku{- aan een moeder de?}{eerste tijding brengen van}{den dood van haar kind}\\

\haiku{Er na greep ze z'n... {\textquotedblleft},{\textquotedblright}, {\textquotedblleft}}{handen en riep hem bij}{z'n naamFrans riep ze}\\

\haiku{{\textquoteleft}Neen, Nelis... laat mij -.}{maar ik heb meer aan hem goed}{te maken dan jij}\\

\haiku{Ik heb hem eerst ook '.}{allesbehalven goed}{hart toegedragen}\\

\haiku{om het kribje, om, - ':}{het zingen zonderdat ze}{t zelf besefte}\\

\haiku{Aarzelend bij den,...}{drempel kon ze niet anders}{dan een kruis maken}\\

\haiku{{\textquoteleft}En mag 'k dan v\'o\'or?}{m'n vertrek nog \'e\'en keer bij}{u terugkomen}\\

\haiku{{\textquoteright} {\textquoteleft}Alle respect voor,{\textquoteright} '.}{uw courage kniktet}{Rosalien hem toe}\\

\haiku{Dank zij ook de  , '!}{goede verzorging diek}{allewijl geniet}\\

\haiku{'t Rosalien was,.}{bekans al opweg maar toen}{bedacht ze zich toch}\\

\haiku{Al kenden we dien -.}{al sindslang te veel kwam er}{zich tusschen stellen}\\

\haiku{Enkel en alleen '.}{dan tochmaar omdatt hem}{zelf haast overmand had}\\

\haiku{vroeger had hij wel:}{meermalen voor korter of}{langer tijd geloofd}\\

\haiku{Dat crucefix zal,, -:}{er hangen eer iemand er}{erg in heeft ja juist}\\

\haiku{Dienzelfden avond nog,:}{opeens de heele keuken}{vol van z'n geluk}\\

\haiku{{\textquoteright} {\textquoteleft}Maar dan zullen we '!}{toch eerst Jeskens nader}{moeten leeren kennen}\\

\haiku{Nelis uit, zonder '?}{er tegen hemn woord van}{gerept te hebben}\\

\haiku{{\textquoteleft}Was jij maar zoo ver,.}{dat je de teugels hier in}{handen kon nemen}\\

\haiku{{\textquoteleft}Zeg aan Nelis,{\textquoteright} al.}{eveneens telkens iets anders}{dat geen uitstel leed}\\

\haiku{t Boterde niet!}{bijster tusschen vader en}{zoon Alexander Doree}\\

\haiku{Die kwestie zou hij.}{weten te regelen tot}{aller voldoening}\\

\haiku{Andr\'e Ruiters zal '.}{t zoolang als pachter voor}{z'n neef beheeren}\\

\haiku{Maar ik, \'o\'ok niet van,:}{gisteren zooals je weet zei}{zoo langs m'n neus weg}\\

\haiku{Want bij dat groote nieuws:}{van Garverskamp kon ze zich}{niet meer goedhouden}\\

\haiku{En hoe Moeder hier?}{opeens zonder iemand zou}{komen te zitten}\\

\haiku{Dadelijk na de,,{\textquoteright}.}{Vasten in alle stilte}{beslist Trinette}\\

\subsection{Uit: Wat was en werd. Verhalen uit Limburgs legende en historie}

\haiku{De gunst en ook de.}{manschappen der grooten zijn}{hem onontbeerlijk}\\

\haiku{roep 't maar van de,.}{daken af dat heel de stad}{alles hoort en weet}\\

\haiku{{\textquoteleft}Ja, ik versta 't - ' -....}{nu ik verstat hij is}{bijna weggeruimd}\\

\haiku{Karel, haar zoon, z'n....}{vaders opvolger worden}{in Austrasi\"e}\\

\haiku{{\textquoteleft}Laat ons dat alles,.}{eens flink maar rustig onder}{de oogen zien vrouwe}\\

\haiku{Lambertus begrijpt,...}{dat en heft de hand om haar}{te zegenen}\\

\haiku{Eerst op den drempel,:}{der voorhal spreekt Hubertus}{die hem uitgeleidt}\\

\haiku{{\textquoteleft}Aangewezen zijn,.}{we den een aan den ander}{dat is duidelijk}\\

\haiku{wat ik redden en}{richten kan voor Christus en}{Zijn rijk met dit mij}\\

\haiku{{\textquoteleft}Wiens naam gezegend{\textquoteright},.}{zij met dien van Lambertus}{bidt Karel Martel}\\

\haiku{Gaat het niet rechtstreeks,.}{dan maar langs een omweg zijn}{vertrouwen winnen}\\

\haiku{ik werd veertien en,.}{zestien en had aldoor nog}{aan mijn droom genoeg}\\

\haiku{Dat is de Vita{\textquoteright},,.}{Sancti Servatii weifelt}{hij niet begrijpend}\\

\haiku{{\textquoteright} {\textquoteleft}Dan hoef je nooit je,.}{zelf te verwijten het niet}{beproefd te hebben}\\

\haiku{De sleutelbos aan.}{zijn leeren gordel klinkelt bij}{iederen voetstap}\\

\haiku{, komen we je niet?}{al te onverwacht uit je}{studie opjagen}\\

\haiku{Hij nestelde zich.}{onder haar mantel en liet}{haar arm niet meer los}\\

\haiku{{\textquoteright} Richardis ziet den:}{ruigen knoestigen man naast}{haar nadenkend aan}\\

\haiku{{\textquoteright} {\textquoteleft}Waarom een weerstand,,?}{die ons het afscheid pijnlijk}{laat worden Gerhard}\\

\haiku{{\textquoteright} {\textquoteleft}Lofwaardig is de{\textquoteright},.}{goede meening antwoordt de}{monnik bedachtzaam}\\

\haiku{{\textquoteright} {\textquoteleft}Gerhard, hoe zou een?}{moeder de liefde van haar}{kind kunnen weerstaan}\\

\haiku{{\textquoteright} Chris Vaesen is nooit.}{verlegen geweest en lacht}{mee met de twee}\\

\haiku{En als Chris trotsch knikt,.}{wisselen de twee een blik}{van verstandhouding}\\

\haiku{{\textquoteright} {\textquoteleft}Wel, ik heb m'n woord,....}{gegeven aan den Prins wiens}{vlag ik gevolgd ben}\\

\haiku{De Prins van Oranje,.}{is daar binnen en wil je}{zelf zien en spreken}\\

\haiku{Totdat de weergalm,.}{opeens wegzinkt omdat de}{ruimte zich verwijdt}\\

\haiku{Hij kan alleen nog,.}{maar bang wachten dat de boot}{ergens zal landen}\\

\haiku{Door de schuld van hem,,, -!}{dwaashoofd luchthart opsnijer}{door zijn schuld alleen}\\

\haiku{God vergeve hem,.}{wat hij zich zelf nooit meer zal}{kunnen vergeven}\\

\section{R.A. Kollewijn}

\subsection{Uit: Verweghe en zijn vrouw (onder ps. C.P. Brandt van Doorne)}

\haiku{Kompromitteerde,!}{hij er z'n schoonzuster mee}{dat r\'a\'akte hem niet}\\

\haiku{'t Was de vijfde '.}{maal dat hijt verhaal nu}{deed op die middag}\\

\haiku{Want hij was nu zich,,.}{zelf niet meer hij was Brikhof}{de postdirekteur}\\

\haiku{Merkwaardig dat ze.}{zo fris en bevallig bleef}{onder dat werken}\\

\haiku{Kwam het d\'a\'arvandaan,,,?}{dat hij kalm was niet opsprong}{in drift of wanhoop}\\

\haiku{hij begon iets te.}{voelen van schaamte over z'n}{argwaan en bitsheid}\\

\haiku{Je denkt nu zo,  ,.}{maar als ik het deed zou je}{er spijt van hebben}\\

\haiku{Hij reikte haar koel,.}{en stijf de hand keerde zich}{om en ging heen}\\

\haiku{Er was daarbij geen.}{sprake van opzettelik}{anders zich voordoen}\\

\haiku{Toen ze boven de,.}{twintig kwam dacht ze niet dat}{ze ooit zou trouwen}\\

\haiku{Ze kwam nu en dan,.}{bij Verweghe aan huis als}{vriendin van Marie}\\

\haiku{Natuurlik was het,.}{niet alles z\'o als zij het}{zou hebben gewenst}\\

\haiku{maar erger was 't,.}{dat nu leugen lag in haar}{woorden en blikken}\\

\haiku{Was hij gekomen?}{om te vertellen dat hij}{haar raad had gevolgd}\\

\haiku{Hij onderzocht haar - -.}{z\'e\'er tegen haar zin en vond}{niets verontrustends}\\

\haiku{Maar d\`an was de zaak -....}{ook uit gesteld dat-ie}{ooit had bestaan}\\

\haiku{En hij voelde aan, -!}{z'n bezorgdheid z'n angst die}{toch \`ongegrond was}\\

\haiku{opgetrokken haar, '....}{voeten alst water haar}{stoel bereikte}\\

\haiku{Niet zoals laatst, in,.}{een half wanhopige bui}{maar na kalm overleg}\\

\haiku{Het scheen dat men voor.}{z'n schuchtere avances niet}{onverschillig was}\\

\haiku{En eindelik het.}{wanhopig besluit om hem}{alles te zeggen}\\

\section{Gerrit Komrij}

\subsection{Uit: Heremijntijd. Exercities en ketelmuziek}

\haiku{En altijd laten,.}{ze me in de steek wanneer}{ik een pen vasthoud}\\

\haiku{Maar zodra ik 'n,.}{pen in mijn hand houd ben ik}{niet meer te stuiten}\\

\haiku{Ze heeft meer oog voor.}{functionaliteit dan}{voor menselijkheid}\\

\haiku{Er is behoefte.}{aan ontwerpers die de}{klerken trotseren}\\

\haiku{Zinloos gingen ze.}{verder waar de anderen}{gebleven waren}\\

\haiku{In werkelijkheid,.}{is er maar \'e\'en vrouw en dat}{is al erg genoeg}\\

\haiku{z\'o verguld mee dat {\textquoteleft}{\textquoteright}, ' {\textquoteleft}{\textquoteright}.}{hij opsex niets tegen heeft}{dat hijtspeels noemt}\\

\haiku{Ik heb het gevoel,.}{dat ik iets goddelijks ja}{God in hun ogen zie}\\

\haiku{De tweede is een:}{literatuur waarmee ik}{nogal verguld ben}\\

\haiku{Over de vetplanten?}{die de onderste helft in}{beslag nemen heen}\\

\haiku{Snufferds van oudjes '.}{kijken om kwart voor achts}{avonds televisie}\\

\haiku{Zij bloeit op waar zij.}{wordt vertroeteld door zielen}{waar de rek in zit}\\

\haiku{Je kan immers \'o\'ok {\textquoteleft}{\textquoteright}.}{niets vertellen door onzin}{aaneen te breien}\\

\haiku{Elke dichter houdt.}{zijn eigen po\"ezie voor}{de enige ware}\\

\haiku{honderd om hem te,.}{begaan en honderd om er}{achter te komen}\\

\haiku{{\textquoteleft}Neem een emmer en.}{zet die tegen de muur of}{een andere wand}\\

\haiku{Hij hield ditmaal geen,.}{papiermand of emmer vast}{maar een tabouret}\\

\haiku{Ik weet niet hoe het,.}{zo kwam maar er wilde geen}{evenwicht meer komen}\\

\haiku{Maar mijn liefde voor.}{solitaire spelletjes}{was nog niet geblust}\\

\haiku{Vaag staat me nog bij.}{dat het  beest er niet echt}{florissant uitzag}\\

\haiku{Nu gebeurt er iets,.}{heel tragisch als we er goed}{over nadenken}\\

\haiku{Ik heb ze meteen ' '.}{bijt begin de stuipen}{opt lijf gejaagd}\\

\section{Dirk Ayelt Kooiman}

\subsection{Uit: De grote stilte}

\haiku{Ik ben scheefgegroeid - -.}{dat blijkt maar weet niet eens waar}{het begonnen is}\\

\haiku{een arrogante -.}{klootzak vond wat ook niet te}{verwaarlozen was}\\

\haiku{Zelfs h\`em viel het op}{dat men vergeleken met}{wat hij gewend was}\\

\haiku{Hij vroeg zich af of.}{het initiatief van haar}{kant gekomen was}\\

\haiku{En wie zwijgt stemt niet,.}{alleen toe hij wordt ook wel}{de wijste genoemd}\\

\haiku{Waarop ze gevraagd.}{had of hij nou ook met al}{die vriendinnen sliep}\\

\haiku{{\textquoteleft}Weet je wat die naam...?}{betekent in het land waar}{ik net vandaan kom}\\

\haiku{Er waren dingen.}{die je bij nader inzien}{namelijk niet zei}\\

\haiku{(Pauze.) Haar koffer.}{kregen we een paar weken}{later opgestuurd}\\

\haiku{Er zaten nog wat,,.}{kleren in beschimmeld de}{rest was gestolen}\\

\haiku{- Het was alsof er.}{een deur werd opengegooid en}{weer dichtgeslagen}\\

\haiku{Hij boog zich naar haar,.}{over voelde zich ijzig kalm}{en vastberaden}\\

\haiku{{\textquoteright} Ze aarzelde, en.}{even zag het er naar uit dat}{ze zou gaan gillen}\\

\haiku{[7] Op de dag af.}{een week later gebeurde}{er het volgende}\\

\haiku{Ze vertrouwde me.}{toe dat ze op het punt stond}{jarig te worden}\\

\haiku{Ze omhelsde me,.}{op mijn beurt waarbij ze op}{haar tenen moest staan}\\

\subsection{Uit: Een romance}

\haiku{{\textquoteleft}Wat was er gebeurd,?}{en hoe gebeurde het wat}{zou de uitkomst zijn}\\

\haiku{Geen zuchtje wind dreef.}{door de wijdgeopende}{vensters naar binnen}\\

\haiku{Want die vogel, die,,...}{neerschoot op de aarde af}{naar de put immers}\\

\haiku{ze draaide me om,.}{mijn as zodat ik in haar}{lengterichting hing}\\

\haiku{Daar kon u best eens,,.}{gelijk in hebben meneer}{antwoordde ik vroom}\\

\haiku{Ik vegeteerde -.}{en die luxe kon ik me nog}{permitteren ook}\\

\haiku{Wat was er gebeurd,?}{en hoe gebeurde het wat}{zou de uitkomst zijn}\\

\haiku{- Zo, daar is ie dan! -,.}{Aardig van je om me op}{te halen Bomdal}\\

\haiku{Ik hoorde hem al:}{fluisterend achter zijn hand}{verduidelijken}\\

\haiku{Ik beklopte het:}{koetswerk als betrof het de}{flank van een renpaard}\\

\haiku{Een giechelige,!}{dithyrambe van glas naar}{urinoir wat een tijd}\\

\haiku{logeerpartijtjes,.}{in de vakantie een jaar}{of tien gelede}\\

\haiku{In zo'n koekblik, u,{\textquoteright}.}{weet wel zo'n afgeroste}{militaire kist}\\

\haiku{Het glas verhief zich,,.}{ademnood de tranen liepen}{me over de wangen}\\

\haiku{Mijn bewustzijn {\textquoteleft}toont{\textquoteright} -!}{me namelijk mijzelf want}{dat is wat het doet}\\

\haiku{ik heb ze in mijn,.}{macht ze zijn volledig aan}{mij onderworpen}\\

\haiku{Alles goed en wel,,?}{helemaal mee eens maar wat}{moet ik daar n\'u mee}\\

\haiku{Dat schudden om te...?}{zetten in een knikken het}{te onderbreken}\\

\haiku{Hij veerde op uit -.}{zijn stoel het leek wel of de}{zitting in brand stond}\\

\haiku{Door een jubelend,:}{koortje begeleid had ik}{hem de hand gereikt}\\

\haiku{Ik keek de mensen,.}{langs die zich tot een massa}{hadden verenigd}\\

\haiku{En dat het zo snel,.}{gebeuren zou kon ik nog}{minder vermoeden}\\

\haiku{Een enorme spiegel,.}{was het zoals men die ziet}{in modehuizen}\\

\haiku{We dansten echter,,.}{al heel opeens zonder dat}{ik er bij nadacht}\\

\haiku{op en neer, heen en,.}{weer synchroon als de stokken}{van een strijkorkest}\\

\haiku{Dat ze een blunder.}{hadden begaan door mij zo}{te onderschatten}\\

\haiku{Ik liep naar een van,.}{de auto's toe betastte}{de benzinedop}\\

\haiku{Ze schijnt hem af te,.}{wijzen te oordelen naar}{de intonatie}\\

\haiku{Ik sluit de deur van.}{mijn kamer en wacht tot hij}{is teruggekeerd}\\

\haiku{Bomdal amuseert zich,.}{wel onnodig me over hem}{zorgen te maken}\\

\haiku{Bovendien, hij is.}{het die me zijn vaarwater}{heeft binnengeloodst}\\

\haiku{Aan weerzijden van,.}{het pad een web van dunne}{naakte worteltjes}\\

\haiku{De schedel van een,.}{konijn gesteriliseerd}{door zon en regen}\\

\haiku{Nu kan ik, dat wil,.}{ik best bekennen niet zo}{erg best  vangen}\\

\haiku{Ik richt mijn blik op,.}{een witte beweeglijke}{stip in de verte}\\

\haiku{Vaag drong het tot me.}{door dat hij af en toe in}{zichzelf mompelde}\\

\haiku{Een angstaanjagend,.}{verschijnsel temeer omdat}{het ook de stem gold}\\

\haiku{Ongeloofwaardig:,,!}{wij op een zandverstuiving}{op een heideveld}\\

\haiku{De bossen in het.}{verschiet verdwenen achter}{een loodkleurig waas}\\

\haiku{Koppig bleef hij staan.}{en vroeg met dikke tong wat}{de bedoeling was}\\

\haiku{De bliksemslagen.}{worden luider en volgen}{elkaar sneller op}\\

\haiku{Merkwaardig... - Maar is?}{het niet gevaarlijk om in}{zo'n schuur te zitten}\\

\haiku{daar ligt - met Martha -.}{en dat hij best mag zien dat}{zij hier ligt met mij}\\

\haiku{Ik sta erbuiten,!}{ik wil er verder niets mee}{te maken hebben}\\

\haiku{- Ja, wie zou er nu...!}{een wesp in jouw glas werpen}{Zijzelf natuurlijk}\\

\haiku{Ik hoop dat ik je.}{niet deprimeer op deze}{feestelijke avond}\\

\haiku{herhaalde Bomdal,.}{nerveus trommelend op de}{leuning van zijn stoel}\\

\haiku{Had hij me misschien?}{iets gevraagd en moest ik hem}{nu antwoord geven}\\

\haiku{Haar andere arm.}{had ze om de schouder van}{Martha geslagen}\\

\haiku{- Zeg Martha, zou je?}{misschien een kop koffie voor}{me kunnen maken}\\

\haiku{Beter  te koud,.}{dan te warm daar verweken}{de hersens maar van}\\

\haiku{Joyce streek met een.}{bevochtigde wijsvinger}{langs haar wenkbrauwen}\\

\haiku{Of was het een vonk?}{van verontwaardiging die}{in hem opgloeide}\\

\haiku{Plotseling drong de.}{betekenis ervan ten}{volle tot me door}\\

\haiku{We schuifelen in,.}{de richting van de auto}{alleen en samen}\\

\haiku{En daarbij was het.).}{gebleven Maar nu zat zij}{naar mij toegewend}\\

\haiku{En wij zouden op.}{afstand van elkaar staan en}{elkaar opnemen}\\

\haiku{Wellicht zouden haar... -?}{borsten me in verwarring}{brengen Maar zijzelf}\\

\haiku{Om niets te doen, niets,.}{te laten ontstaan niets te}{laten gebeuren}\\

\haiku{Maar, drong tot me door,.}{dat zou niet van het minste}{belang geweest zijn}\\

\haiku{Een zwaar wolkendek,,,.}{struiken rondom een hek}{daar stond de auto}\\

\haiku{Van haar gezicht kon.}{ik me later nooit meer een}{voorstelling maken}\\

\haiku{En in het bordeel:}{met de duizend trappen en}{de duizend deuren}\\

\subsection{Uit: De vertellingen van een verloren dag}

\haiku{Uit weinig m\'e\'er valt.}{op te maken hoezeer hij}{zich voelt aangedaan}\\

\haiku{Het doffe gevoel:}{van meewarigheid wanneer}{zijn moeder uitroept}\\

\haiku{Omdat ze het zelf.}{niet durven te doen gaan ze}{naar hun buurtcaf\'e}\\

\haiku{dat blijkbaar ook zij -.}{bestaat en bovendien in}{mijn gezelschap is}\\

\haiku{Dat is gratie, maar.}{niet een gratie omwille}{van het behagen}\\

\haiku{Doen alsof het ook.}{voor mij de gewoonste zaak}{van de wereld is}\\

\haiku{Een blik achter zich,.}{alsof daar geheugensteun}{te verwachten is}\\

\haiku{- Het gaat hem, wanneer,.}{we hem geloven moeten}{zelden voor de wind}\\

\haiku{hij duidt aan met zijn),,:}{hand eminentie stond ik op}{een schip en ik riep}\\

\haiku{{\textquoteleft}En de terugweg,.}{is korter dan de heenweg}{valt me altijd op}\\

\haiku{Er klinkt verbazing,.}{in hun stem verbazing over}{zoveel avonturen}\\

\haiku{Maar dat was wel een.}{beetje hinderlijk wanneer}{er al iemand was}\\

\haiku{Wie zegt me dat iets?}{alsvolgt zal gaan omdat het}{altijd al zo ging}\\

\haiku{Hij knikt gevleid, maar {\textquoteleft}{\textquoteright}.}{houdt het voorlopig toch maar}{liever opsignor}\\

\haiku{Hij was bijvoorbeeld.}{dodelijk nieuwsgierig naar}{mijn toekomstplannen}\\

\haiku{Het patent komt op,}{jouw naam te staan dat zal ik}{meteen regelen}\\

\haiku{Kompaan staat kreunend,.}{op en rekt zich uit zijn haar}{glanzend van de dauw}\\

\haiku{En het moet gezegd,.}{dat karweitje is in een}{ommezien geklaard}\\

\haiku{Hij pakt de cognacfles,:}{giet het laatste bodempje}{naar binnen en zegt}\\

\haiku{Waar de dimensie?}{die aan het verslag ontbreekt}{vandaan getoverd}\\

\haiku{Wat te beginnen:}{met het gevoel dat hij op}{dit moment ervaart}\\

\haiku{Zeker, zo was het,.}{onmiskenbaar er is een}{gesloten verband}\\

\haiku{Ik trek haar zonder,:}{uitstel aan de arm en vraag}{iets in de geest van}\\

\haiku{Nu! - En dat dan in.}{tegenstelling tot wat er}{volgde op die avond}\\

\haiku{{\textquoteright} heeft hij gevraagd, een,.}{beetje afwezig zonder}{haar aan te kijken}\\

\haiku{Nooit, nooit zou ik haar,.}{weer terugzien het leven}{had geen waarde meer}\\

\haiku{Het zonlicht op mijn.}{haren is van een zon die}{nooit meer schijnen zal}\\

\haiku{Het rosarium,.}{alleen nog herkenbaar aan}{de tegelpaadjes}\\

\haiku{En mijn hand op zoek.}{naar welvingen die ze nog}{niet te bieden had}\\

\haiku{{\textquoteleft}Wanneer we onze}{bagage nou eens later}{op gingen halen}\\

\haiku{Een gammel houten.}{bruggetje voert ons over een}{snelstromende beek}\\

\haiku{ieder woord, ieder.}{gebaar geeft uitdrukking aan}{die situatie}\\

\haiku{{\textquoteright} - Die vraag moet me wel.}{op de lippen bestorven}{hebben gelegen}\\

\haiku{Dat ging door tot er:}{een nooit  te voorspellen}{eindstreep werd bereikt}\\

\haiku{wanneer het nu zou.}{regenen zou de damp van}{het plaveisel slaan}\\

\haiku{die scherpgepunte,,.}{glasscherven zo dichtbij zo}{binnen handbereik}\\

\haiku{Hij is begonnen,.}{bij de A en inderdaad}{in slaap gevallen}\\

\haiku{Aan het tafeltje.}{tegenover het zijne zit}{een verliefd paartje}\\

\haiku{Zweet op zijn voorhoofd -.}{zelfs om het af te vegen}{ontbreekt hem de kracht}\\

\haiku{{\textquoteleft}Neem me niet kwalijk,.}{dat ik u stoor maar ik zit}{met een probleempje}\\

\haiku{Informeert hoe laat.}{de ochtendpostbestelling}{gewoonlijk plaatsvindt}\\

\haiku{Naar welke van de,?}{twee zou uw voorkeur uitgaan}{als ik vragen mag}\\

\haiku{Ze passeerden het,.}{Casino de schijnwerpers}{diffuus door de sneeuw}\\

\haiku{Voetje voor voetje volgde}{Merkuur door een dikke laag}{krakende  sneeuw}\\

\haiku{Jazeker, een man,,.}{en twee meisjes de ene rood}{de andere blond}\\

\haiku{k Gooi je er hier,.}{uit want verderop is het}{eenrichtingsverkeer}\\

\haiku{Wanneer de werkster.}{geweest is kan ik een dag}{lang niets meer vinden}\\

\haiku{Zeno's kooi wordt eens.}{in de week verschoond en van}{vers zand voorzien}\\

\haiku{{\textquoteright} {\textquoteleft}Door ze geregeld.}{te laten reinigen en}{ze te borstelen}\\

\section{Anton Koolhaas}

\subsection{Uit: De laatste goendroen}

\haiku{De eerste keer dat,;}{ze dit probeerde werd het}{Bladroes wel wat te bar}\\

\haiku{Ze maakte een zacht.}{jankend geluid en deed dat}{toen nog een paar keer}\\

\haiku{En voor zover dat:}{voor andersdenkenden niet}{zo'n probleem zou zijn}\\

\haiku{Hij zou alleen zijn;}{eigen weg door het leven}{gevonden hebben}\\

\haiku{Jonas antwoordde.}{dat hij dat maar al te graag}{voor hem wilde doen}\\

\haiku{dat veel bruin bevat.}{buitengewoon moeilijk zijn}{te ontdekken}\\

\haiku{Dat moet U zeker{\textquoteright},, {\textquoteleft}}{doen riep de heer R\"utl\"u van}{het departement}\\

\haiku{{\textquoteleft}Inderdaad dit kon{\textquoteright}.}{wel eens een vondst van enorme}{betekenis zijn}\\

\haiku{Misschien deze ene,}{nacht nadat hij vandaag zou}{hebben besloten}\\

\haiku{Het was prof Ruddell}{die de impasse doorbrak}{door voor te stellen}\\

\haiku{Zojuist had je het.}{over die ene gehoorzenuw}{van  de sprinkhaan}\\

\haiku{Nou, ik peins er niet{\textquoteright},.}{over riep Strutt en hij trok zijn}{handschoenen weer uit}\\

\haiku{ik u met mijn hand -.}{op het hart dan dat het hier}{zo'n saai rotgat is}\\

\haiku{Even voortvarend als,.}{Ruddell binnengekomen}{was verdween hij weer}\\

\haiku{alleen kalmte is{\textquoteright} {\textquoteleft}{\textquoteright}.}{hier vereist en zei nog een}{keerMijne heren}\\

\haiku{{\textquoteleft}Ja ik weet niet of{\textquoteright},, {\textquoteleft}}{het iets van waarde is zei}{Jonas verlegen}\\

\haiku{{\textquoteleft}Wel, wel{\textquoteright}, riep Ruddell, {\textquoteleft},{\textquoteright}.}{en dr Laeet zeiDront je bent}{onverbeterlijk}\\

\haiku{{\textquoteleft}Maar ze hebben het,{\textquoteright},.}{niet opgeschreven daar zegt}{Fenja riep Jonas}\\

\haiku{{\textquoteleft}Wat onze kleine{\textquoteright}.}{geleerde hier gesteld heeft}{vraagt om dat beraad}\\

\haiku{Professor Ruddell.}{wist eigenlijk ook niet meer}{wat hem te doen stond}\\

\haiku{Eigenlijk konden.}{er niet zo goed drie mensen}{in het kamertje}\\

\haiku{Maar ze deden of}{het iets heel gewoons was en}{dat verbaasde me.}\\

\haiku{{\textquoteleft}En ik geloofde,{\textquoteright},.}{het meteen wat ze zeiden}{mompelde R\"utl\"u}\\

\haiku{Maar aangezien hij.}{van haar weggelopen was}{deerde hem dat niet}\\

\haiku{Het dient vastgesteld {\textquoteleft}{\textquoteright}.}{dat het woordbedreigen hem}{eigenlijk niets zei}\\

\haiku{Een enkele keer,.}{als er niet gevoetbald werd}{vergezeld van Jomp}\\

\haiku{Misschien kan ik daar{\textquoteright}.}{dan iemand anders nog eens}{een plezier mee doen}\\

\haiku{Er was een gesprek.}{gaande tussen zijn vader}{en zijn grootvader}\\

\haiku{Daarna verklaarde,.}{hij dat hij helemaal niets}{rook laat staan voorjaar}\\

\haiku{Maar waar had hij de?}{vleugels nog die zorgden voor}{de goede afloop}\\

\haiku{Dat je zijn blik nu.}{uitgesproken treurig kon}{noemen is niet zo}\\

\haiku{Maar toch wel zo hoog.}{dat de beek te breed was om}{overheen te springen}\\

\haiku{Toen dacht hij echter:}{aan wat hem vlak daarvoor door}{het hoofd was gegaan}\\

\haiku{Bij elkaar is dit.}{wel wat je aftakeling}{zou kunnen noemen}\\

\haiku{De avond daarna was.}{hij te moe om een derde}{bed weg te graven}\\

\haiku{De zwarte grond was,.}{onder opzij en boven}{Bladroes en achter hem}\\

\haiku{Bladroes moet veranderd,,.}{zijn na voorbereiding en}{inkeer in een plant}\\

\haiku{{\textquoteright}, fluisterde Hiske, {\textquoteleft}.}{en ineens stak ik toen mijn}{handen in die plant}\\

\haiku{En daarom schrok ik.}{zo ontzettend en holde}{ik naar julie toe}\\

\subsection{Uit: Vanwege een tere huid}

\haiku{Anton Koolhaas}{Vanwege een tere huid}{Colofon}\\

\haiku{bang worden terwijl.}{er niets te zien is waar je}{bang voor hoeft te zijn}\\

\haiku{Wel goed dat ik je{\textquoteright},.}{nu eindelijk eens buiten}{tref zei Jokke zacht}\\

\haiku{Een zinnetje om.}{op te krijgen om in het}{Duits te vertalen}\\

\haiku{Jokke knikte - {\textquoteleft}Maar.}{stap dan meteen op je fiets}{als ik er aankom}\\

\haiku{Het is de groei{\textquoteright}, zei, {\textquoteleft}{\textquoteright}.}{zijn moederdaar tobben al}{die kinderen mee}\\

\haiku{Hij was verliefd op,.}{Takkie mateloos verliefd}{er in verdronken}\\

\haiku{dacht hij vaak en hij.}{huiverde dan en wou dat}{hij er van af was}\\

\haiku{Daar zal Hij het niet,.}{druk mee hebben want er zijn}{er nog maar een paar}\\

\haiku{{\textquoteright} Jokke was daar niet.}{vatbaar voor en voelde er}{zich door beledigd}\\

\haiku{Jezus meid, met je,,.}{kouwe drukte dacht Jokke}{net of ik dat weet}\\

\haiku{Gisteren hadden.}{de sproeten op haar armen}{hem toch gehinderd}\\

\haiku{Hij maakte zich dan.}{wel een voorstelling van hoe}{dat dan zou toegaan}\\

\haiku{de Koning had het,.}{in de gaten dat Takkie}{eindelijk beet had}\\

\haiku{De kano's waren;}{te huur bij een zwembad aan}{de rand van de stad}\\

\haiku{{\textquoteleft}Meervallen, je weet;}{wel die vissen die zo zwart}{zijn met die snorren}\\

\haiku{Ze weten trouwens.}{evenmin dat ze zelf practisch}{zijn uitgestorven}\\

\haiku{De ooms en tantes.}{hadden lijven gehad in}{zwarte omhulsels}\\

\haiku{Op een gegeven.}{ogenblik valt het je op dat}{je geen pijn meer hebt}\\

\haiku{zo kind, zo zijn we,.}{toch zekerlijk allemaal}{gelukkig vandaag}\\

\haiku{{\textquoteleft}Nee, nou moet je het{\textquoteright},.}{niet nog mooier maken dan}{het al was riep Cor}\\

\haiku{Ik zal er snel in ().}{voorzienhij plakt de zegels}{er op stond er dan}\\

\haiku{Hier zijn ze opnieuw,{\textquoteright}.}{Adriaan en ik bedank u}{voor uw waakzaamheid}\\

\haiku{Cor en Annepiet,.}{trouwens ook maar die kregen}{wat ze verdienden}\\

\haiku{Uit verdriet{\textquoteright}, jammert,:}{ze heel zacht maar ze zou het}{uit willen gillen}\\

\haiku{Natuurlijk denkt ze.}{aan de volgende dag en}{wat ze zullen doen}\\

\haiku{En na enige niet,.}{te overleven momenten}{waren ze er af}\\

\haiku{Daarna zullen ze.}{het gesprek voortzetten over}{dokter de Koning}\\

\haiku{Zijn ogen vormen een.}{neutrale materie in}{zijn kleurloze kop}\\

\haiku{Takkie was al heel;}{wat keren in en weer uit}{de wherry gestapt}\\

\haiku{Zo lang gewacht, tot.}{ze er eenvoudig te moe}{door geworden was}\\

\haiku{Mogelijk hebben;}{ze een ander niet gebruikt}{terrein gevonden}\\

\haiku{Van jongetjes en.}{meisjes vragen we niet waar}{ze ge bleven zijn}\\

\section{Pieter Korthuys}

\subsection{Uit: Menschen in malaise}

\haiku{s Avonds na het eten,.}{toen wij samen waren kwam}{het hooge woord eruit}\\

\haiku{Eerst zat Va nog wat,.}{in de krant te lezen maar}{toen kwam het eruit}\\

\haiku{Nog een week of twee,.}{kon hij blijven maar dan was}{het afgeloopen}\\

\haiku{{\textquoteright} {\textquoteleft}Nou, ik geloof, dat.}{ik dan nog liever in een}{achterbuurt hier zit}\\

\haiku{{\textquoteleft}Wij kunnen morgen,,,.}{eens kijken h\'e Flosje of}{wij iets goeds vinden}\\

\haiku{Overmorgen is het,.}{al de 28ste dan hebben wij}{nog maar twee dagen}\\

\haiku{{\textquoteright} {\textquoteleft}Lottie nemen wij,}{mee die kan bij mij in het}{mandje achterop}\\

\haiku{het was deze week,.}{een late boot een van de}{oudere typen}\\

\haiku{Maar breek er je nek,!}{niet als je morgen naar je}{bezit komt kijken}\\

\haiku{Een doeltje, dat wij,.}{samen nastreven om jouw}{ouders te helpen}\\

\haiku{Van tweehoog hoort hij.}{de sliffers van pantoffels}{omlaag klepperen}\\

\haiku{Z\'o\'o gemakkelijk.}{als zij het voorstelde zou}{het zeker niet zijn}\\

\haiku{Maar ze zijn dan toch,}{motte verhuize om ons}{te kenne helpe}\\

\haiku{Hij wil later ook,.}{naar Indi\"e zien te komen}{als zijn oudste broer}\\

\haiku{{\textquoteright} Een gelukkige.}{blik van verstandhouding is}{haar eenige antwoord}\\

\haiku{Maar van Nel weet ze,.}{dat je tegenwoordig z\'o\'o}{billijk kunt koopen}\\

\haiku{Flos eet dan ook vijf.}{dagen van de week alleen}{met de kinderen}\\

\haiku{Die vent denkt zeker,.}{in haar een goede klant te}{hebben gevonden}\\

\haiku{Hij licht zijn slappe,:}{vilthoed en vraagt haar of ze}{niets kan gebruiken}\\

\haiku{Zijn haren zijn nat,.}{hij was zich klaarblijkelijk}{aan het mandi\"en}\\

\haiku{{\textquoteleft}En zelf heb ik in!}{geen maand een sigaar gerookt}{uit bezuiniging}\\

\haiku{{\textquoteleft}Het is acht uur, ik!}{word nog wel geholpen als}{ik dadelijk ga}\\

\haiku{{\textquoteleft}'t Is Meijers,{\textquoteright} zegt,:}{Frans en ter verklaring voor}{Rudolf en Hanny}\\

\haiku{Frans zoekt een geschikt.}{woord om zijn afgebroken}{zin te hervatten}\\

\haiku{En ook wil hij zijn.}{vrouw helpen met de kamer}{aan kant te brengen}\\

\haiku{De blauwe rook trekt.}{in breede banen het half}{openstaande raam door}\\

\haiku{Ze ziet het weer voor,.}{zich die blanke pracht van de}{bloeiende bongerds}\\

\haiku{Eigenlijk heeft zij,'.}{er nu reeds wroeging over als}{ze Frans gezicht ziet}\\

\haiku{En Meijers is ook,!}{geen vent om zich nu eens voor}{je in te spannen}\\

\haiku{Dan durven ze zelfs.}{geen honderd gulden in de}{maand te weigeren}\\

\haiku{{\textquoteright} vraagt hij verwijtend, {\textquoteleft},!}{anders kwam je er altijd}{mee bij me vroeger}\\

\haiku{Hij is verdwenen.}{in de ontelbaarheid van}{het inlandsche volk}\\

\haiku{Nou, in vredesnaam,{\textquoteright}.}{dan maar zoo'n suikerdrankje}{doet hij wanhopig}\\

\haiku{tusschen de dikke.}{wolkengroepen die de zon}{reeds verduisteren}\\

\haiku{Ze gevoelt zich hier,.}{als met Frans en ze hebben}{nog geen kinderen}\\

\haiku{Zwijgend zitten ze,.}{naast elkaar die drukkende}{rust overweldigt je}\\

\haiku{Het begint harder.}{te waaien en Flos trekt haar}{sjaal onder de keel}\\

\haiku{Zoo opgewekt en!}{moedig als nu heeft ze zich}{haast nog nooit gevoeld}\\

\haiku{En Lottie, achter,,.}{Moesje bedelt om er ook}{te mogen zitten}\\

\haiku{Frans kijkt elken avond,.}{de kranten nauwkeurig na}{maar er is niet veel}\\

\haiku{Enfin, het zou wel,.}{droog worden en anders de}{regenjas maar aan}\\

\haiku{Je kunt toch niet voor?}{ieder wissewasje naar}{den dokter loopen}\\

\haiku{{\textquoteright} Aan den anderen,}{kant komt wat spottend de vraag}{dat het dan blijkhaar}\\

\haiku{Op het platje is, ',.}{het om dezen tijd \'e\'en uur}{s middags te warm}\\

\haiku{Lottie heeft hem al,.}{ontdekt Dickie heeft aandacht}{voor iets op den grond}\\

\haiku{{\textquoteleft}Maar ik wilde nog.}{minstens een maand hier blijven}{uitzien naar een baan}\\

\haiku{Met een sprong zijn haar.}{gedachten dan weer bij haar}{eigen belangen}\\

\haiku{Een volgende brief,,.}{met de landmail logenstraft}{haar luchthartigheid}\\

\haiku{{\textquoteright} {\textquoteleft}Nooit, dat weet je wel,,!}{hoe blij ik ben die rijkdom}{is niet te meten}\\

\haiku{Jij bent in dien tijd,...}{voor mij de vrouw geworden}{die ik moest hebben}\\

\section{Alfred Kossmann}

\subsection{Uit: De moord op Arend Zwigt}

\haiku{Met wortel en tak.}{zou hij die vertedering}{uit moeten roeien}\\

\haiku{Ik kots van dat huis,,.}{van die schijnheiligheid van}{die smeerlapperij}\\

\haiku{{\textquoteright} vroeg ze verveeld maar, {\textquoteleft},?}{dringendeen glaasje likeur}{een glaasje klare}\\

\haiku{Ik heb honger{\textquoteright} zei, {\textquoteleft}.}{hijwe moeten eens iets te}{eten zien te krijgen}\\

\haiku{Hij begreep dat het.}{verdacht zou zijn om op dit}{voorstel in te gaan}\\

\haiku{{\textquoteright} Simon vroeg het op,.}{een onpersoonlijke maar}{dreigende manier}\\

\haiku{als het warm is moet,.}{je liever niet eten dat is}{goed voor de lijn ook}\\

\haiku{{\textquoteleft}Als de heren niet}{betalen kunnen er twee}{dingen gebeuren}\\

\haiku{En in die valse.}{atmosfeer begon Arend als}{een kind te huilen}\\

\haiku{{\textquoteright} zei Simon, {\textquoteleft}de deur,,.}{opendoen het geld stelen het}{is onherstelbaar}\\

\haiku{Hij wist niet waar hij.}{aan toe was en zocht naar een}{gevoel dat overheerste}\\

\haiku{Waarom had hij zo,?}{slecht gewerkt op school terwijl}{hij het beter kon}\\

\haiku{Bij toeval of bij;}{gratie van instinct kon het}{wonder gebeuren}\\

\haiku{Het wonder was er,.}{wel maar het leek nog niet het}{beslissende}\\

\haiku{Zij keerde zich om,.}{keek rond door de nu lege}{winkel en ging weg}\\

\haiku{{\textquoteleft}Het was mijn vader{\textquoteright}, {\textquoteleft},.}{zei hijik ben blij dat hij}{ons niet heeft gezien}\\

\haiku{Hij voelde hoe zijn,,.}{vlees waar zij het aanraakte}{begon te leven}\\

\haiku{Het lome onheil,,.}{van de atmosfeer smachtend}{naar wind deed hem niets}\\

\haiku{ik heb de wereld,.}{van binnenuit veroverd}{ik kan nu alles}\\

\haiku{Hij stond op en liep,.}{verder overwegend of hij}{maar zou gaan slapen}\\

\haiku{Hij snikte hardop.}{en wenste dat hij zich om}{zou durven draaien}\\

\haiku{Het liefst had hij zijn.}{hoofd in zijn handen gelegd}{en was doodgegaan}\\

\haiku{De tellen die hij,,.}{niet meer uitsprak tikten weg}{verprutst verloren}\\

\haiku{De wereld ligt in,.}{scherven en ik moet er een}{eenheid van maken}\\

\haiku{{\textquoteright} voegde hij tot zijn.}{schaamte nog bijna smekend}{aan zijn betoog toe}\\

\haiku{je denkt zeker dat,.}{ik gek ben je denkt zeker}{dat ik het niet doe}\\

\haiku{Zij keek hem kalm aan,.}{maar in haar vage antwoord}{klonk enige dreiging}\\

\section{Wim Kuipers}

\subsection{Uit: Platbook 4. Fitsprovins}

\haiku{Wie veer aankaome.}{biej de sjtart woort dat geveul}{allein mer sjterker}\\

\haiku{Fietse, wie vrouwluuj, '.}{mit tesse vol w\`ek poor en}{erpel aant stuur}\\

\haiku{Vaals, dan is 't nie '.}{mie\"er asn tepelke}{op de werreldbol}\\

\haiku{op \'os sportfietse.}{waor we de spatborde}{vanaf han gesloop}\\

\haiku{de wolke... alles......}{zwart de waeg smaal gehoebeld}{miene fiets rammelt}\\

\haiku{'t Waas al laat in '.}{t sezoen en de blajer}{vele van de buim}\\

\haiku{Inne va h\"on hauw '.}{de mie\"etste kans umt}{renne tse winne}\\

\haiku{Wool d'r Piet ing kans,.}{maache moe\"et he\"e inne fiets}{mit versjnellinge han}\\

\haiku{Sjouwer a sjouwer '.}{varete zet letste}{sjtuk noa d'r knietsjroam}\\

\haiku{Wie veer ouch achter,.}{h\"a\"om aan karde veer krege}{geine maeter good}\\

\haiku{Vreuger kwaam ich dich...,?}{nog al ins teenge mer duis}{te t'r niks mie\"e aan}\\

\haiku{Van doe aaf waor,.}{ich verkoch aan dees vorme}{van sjport m\`et aafzeen}\\

\haiku{'t Irriteert mich.}{dat bie de euverheid sjport}{is blieve ligke}\\

\section{Albert Kuyle}

\subsection{Uit: De bries}

\haiku{Men had haar hooren,.}{gillen aan het keldergat}{maar meer wist men niet}\\

\haiku{Fedor Fedorowitz.}{was de laatste naam die op}{de rol werd gezet}\\

\haiku{Fedor is bij de.}{acht die met een sloep den weg}{hakken naar het land}\\

\haiku{Hij voelt hoe elke.}{beweging van de bek in}{de spieren doortrilt}\\

\haiku{Als hij de oude.}{man goedendag gezegd heeft}{zal hij haar kussen}\\

\haiku{haar haren hangen.}{in een zware wrong op een}{nieuw geruit kleedje}\\

\haiku{{\textquoteleft}we gaan trouwen, wees,.}{niet koppig geef een brief mee}{aan de veekooper}\\

\haiku{En hoe meer hij er,.}{over nadacht hoe meer hij dacht}{dat zij gelijk had}\\

\haiku{Toen ze uitging was.}{het omdat de kamer te}{klein was voor haar droom}\\

\subsection{Uit: Jonas}

\haiku{Als een kleine pijn.}{begint achter in zijn hoofd}{een angst te groeien}\\

\haiku{Na ieder uur een,.}{top gewonnen na ieder}{uur een dal gedaald}\\

\haiku{Nooit meer water dan.}{de bron zich verzamelde}{in haar kleine kom}\\

\haiku{Dit is geen zwerver,.}{zooals er zoovelen aan de}{havenkaden zijn}\\

\haiku{Het is een rond en,.}{blauw luchtledig waarin het}{schip gewichtloos zweeft}\\

\haiku{De kapitein keert,.}{het gezicht naar boven als}{hij de beker draait}\\

\haiku{De zee\"en waarop{\textquoteright}.}{ik vluchtte en de aarde}{die ik ontvluchtte}\\

\haiku{Zijn barmhartigheid.}{is als de morgenwolk en}{als de morgendauw}\\

\haiku{Hij luistert naar de:}{muziek die hem beneden}{en ter weerzij is}\\

\haiku{Nog veertig dagen.}{en Niniveh zal omver}{geworpen worden}\\

\haiku{Die het water zijn,.}{schatten ontnamen het vuur}{dreef ze nu landwaarts}\\

\haiku{Of zou het waar zijn,?}{dat hij den smid bij diens vrouw}{had toegenomen}\\

\haiku{De roeden worden.}{geheven in de huizen}{van de buitenbuurt}\\

\haiku{Jonas kent hun reuk,:}{uit de duistere holten}{van het visschenlijf}\\

\haiku{Laat dan af van uw.}{boosheid en weest den Koning}{een gewillig volk}\\

\haiku{Zij staan scherp tegen.}{den witten weg die naar de}{tempeltrappen voert}\\

\haiku{In bloed smoort gij uw,.}{vruchtbaarheid in bloed zal de}{Heer u verstikken}\\

\haiku{wat baat brengt mij het,?}{l\'even waar ik de w\'o\'orden}{heb om te dienen}\\

\haiku{staat eerbiedig stil,.}{dat het woord van uw wijsheid}{hem verkwikken zal}\\

\haiku{allen is hij, en.}{Jahwe legde de wereld}{in zijne handen}\\

\haiku{Wie bewaarde het?}{graan van den overvloed voor de}{dagen vol honger}\\

\haiku{Wie sprak wijsheid door?}{zijnen mond en gaf acht op}{den vrede des rijks}\\

\haiku{Een oogwenk staat hij.}{onbeweeglijk temidden}{van zijn knielend volk}\\

\haiku{De Koning heeft de:}{ketens genomen van die}{geketend waren}\\

\haiku{Bitter in den mond,.}{smaakt Jahwe en als alsem}{zijn Zijne daden}\\

\haiku{Het einde wenkt, en.}{hij buigt zich van verlangen}{over naar dat einde}\\

\haiku{No\'e ziet om zich heen,.}{als was hij kinderen en}{aarde vergeten}\\

\haiku{Dan schrijdt hij uit, naar,.}{buiten waar de zon laag over}{de akkers wentelt}\\

\chapter[16 auteurs, 3167 haiku's]{zestien auteurs, drieduizendhonderdzevenenzestig haiku's}

\section{L.H.J. Lamberts Hurrelbrinck}

\subsection{Uit: Het beulsjong}

\haiku{beroerd - lamzalig -;}{toch moesten ze er heen voor hun}{eigen ponteneur}\\

\haiku{{\textquoteright} {\textquoteleft}Allo dan, ik zal, -}{mijn best doen al was het maar}{daarvoor alleen}\\

\haiku{{\textquoteright} {\textquoteleft}Dat wil niet zeggen,.}{dat we het dezen keer ook}{niet zullen hebben}\\

\haiku{Visioenen van, -,.}{smaad en verachting van roem}{en eer toen de man}\\

\haiku{{\textquoteright} {\textquoteleft}Justement, 't doet,}{mij pleizier dat je dat zoo}{goed onthouden hebt}\\

\haiku{Dat - dat nooit meer... maar..., -...}{wat dan wat dan nondediu}{weer d'r van door gaan}\\

\haiku{{\textquoteright} {\textquoteleft}Neen, er ontbreekt niks{\textquoteright},, {\textquoteleft},}{als Wetzels terugkeertgeen}{rooie duit contrarie}\\

\haiku{{\textquoteright} {\textquoteleft}Ja effectief, dat,...}{weet ik maar daarom hoeft het}{toch niet uit te zijn}\\

\haiku{koude windvlagen,.}{welke golven over den zwart}{geregenden grond}\\

\haiku{{\textquoteright} Dan weer zich wendend:}{tot de nog aanwezigen}{met tergenden lach}\\

\haiku{hij had hem  zoo -...}{valsch aangekeken toen hij}{dat zei of Kouwen}\\

\haiku{alles om hem heen;}{overtrokken met gordijn van}{nevelachtig waas}\\

\haiku{{\textquoteright} {\textquoteleft}O dat went wel als - -}{je eenigen tijd hier geweest}{bent kom nou maar mee}\\

\haiku{{\textquoteright} Een woedeblik van;}{Peter naar dien heen en weer}{zwaaienden dronkaard}\\

\haiku{{\textquoteright} {\textquoteleft}Werken Harieke,,.}{net als alle jongens als}{ze van school af zijn}\\

\haiku{Schuchter, verlegen.}{neemt Tineke plaats op den}{nog ledigen stoel}\\

\haiku{-Harieke had}{gelogen en Peter had}{ook gelogen zou}\\

\haiku{hij hun niet, en weer.}{een behoedzaam sluipen door}{de struiken naar huis}\\

\haiku{{\textquoteright} {\textquoteleft}Goed, doen jelui dat,.}{en laat me dan weten wat}{afgesproken is}\\

\haiku{maar dat andere...,.}{neen Tineke dat mag ik}{niet van je vergen}\\

\haiku{hij verdient het voor - '.}{wat hij gepresteerd heeftt}{was mirabilant}\\

\haiku{{\textquoteright} {\textquoteleft}Kan jij hem helpen{\textquoteright},.}{aan een andere manier}{met felle bitsheid}\\

\haiku{{\textquoteleft}Vindt u niet, dat hij,,{\textquoteright}.}{mooi geschoten heeft vader}{na geruime poos}\\

\haiku{{\textquoteleft}Neen, dat is niet waar,{\textquoteright},.}{dat is niet waar met heftig}{vlugge ratelstem}\\

\haiku{{\textquoteright} {\textquoteleft}Ik zal gaan, vader,,;}{sebiet maar dat zeg ik u}{en onthoud het goed}\\

\haiku{Jij zult Hari niet,?}{het erf afdonderen als}{hij hier kwam voor mij}\\

\haiku{een nevelwaas voor; ', '...}{zijn oogent draaitt wentelt}{alles om hem heen}\\

\haiku{Bartels, Hubertus,...}{Bartels gonst en raast het in}{zijn brandenden kop}\\

\haiku{{\textquoteright} {\textquoteleft}Neen, neen{\textquoteright}, met driftig - ',.}{afwerend gebaart is}{niks zeg ik je toch}\\

\haiku{zij zou hem vragen.}{waarom en hij zou weer niet}{kunnen antwoorden}\\

\haiku{Hazen, konijnen,.}{die bengelen aan om het}{lijf vastgesnoerd touw}\\

\haiku{Eindelijk niet meer,.}{zichtbaar die twee voor hen die}{gebleven zijn}\\

\haiku{Zoo den geheelen - -,,;}{dag zou hij ziek zijn Jezus}{Maria Jozef nog}\\

\haiku{als ze serieus.}{ziek was zou Peter hem wel}{gewaarschuwd hebben}\\

\haiku{{\textquoteright} {\textquoteleft}Ja, 't is maar voor,.}{een paar minuten dan kom}{ik het weer halen}\\

\haiku{{\textquoteleft}Wat is er met hem{\textquoteright},.}{gebeurd tot Hari geknield}{aan zijne zijde}\\

\haiku{ik heb je alleen.}{maar gevraagd of je die plaats}{zoudt kunnen wijzen}\\

\haiku{{\textquoteright} {\textquoteleft}Hari, Hari{\textquoteright}, als.}{de gerechtsdienaar de hand}{op zijn schouder legt}\\

\haiku{Ik beloof het u.{\textquoteright}.}{Zoo is de dienaar Gods bij}{Bartels gekomen}\\

\haiku{ik zal haar terstond -,?}{schrijven is er nog iets wat}{ik voor je doen kan}\\

\haiku{{\textquoteright} {\textquoteleft}Zoo absoluut heeft,,.}{hij dat niet gezegd maar wel}{dat de kans bestaat}\\

\haiku{{\textquoteright} Niet spreken - en zij.}{wil hem juist verzorgen om}{hem te doen spreken}\\

\haiku{{\textquoteleft}We wenschen Bartels{\textquoteright},.}{te spreken de oudste der}{heeren tot Peter}\\

\haiku{zij hooren niet het;}{rammelend gepraat van hun}{medereizigers}\\

\haiku{{\textquoteleft}Hoeber{\textquoteright}, terwijl Bertha.}{vat de slap neerhangende}{hand in de hare}\\

\haiku{een beefschrikken, als}{een der knechten hem nadert}{om hem te vragen}\\

\haiku{{\textquoteright} {\textquoteleft}Dat het Hari niet,.}{was die hebben ze dan ook}{al losgelaten}\\

\haiku{Nieuwe angst in hem -!}{zouden ze toch nog daar zijn}{om hem te halen}\\

\haiku{{\textquoteright} {\textquoteleft}Stil nou kinders, stil,.}{nou niet zoo'n spektakel bij}{zoo'n zwaren zieke}\\

\haiku{{\textquoteright} {\textquoteleft}Wat wil je daarmee,{\textquoteright}.}{zeggen Bertus terwijl zij}{hem verbaasd aanstaart}\\

\haiku{{\textquoteright} {\textquoteleft}Als u dat zoo braaf -?}{en christelijk vindt waarom}{doet u het dan niet}\\

\haiku{handen, die drukken,,.}{handen der mannen vrouwen}{die kussen vrouwen}\\

\subsection{Uit: De heks van Heinsbroek}

\haiku{Ongeveer een jaar;}{na hun huwelijk werd hun}{een zoon geboren}\\

\haiku{Een luide juichkreet,,;}{een hartelijk galmende}{lach dien hij uitstoot}\\

\haiku{hij had ze willen,;}{bespieden van nabij hun}{geheimen kennen}\\

\haiku{enkele dagen;}{later was deze zelf in}{het dorp gekomen}\\

\haiku{'t gat in den grond.}{was verborgen achter dik}{begroeid kreupelhout}\\

\haiku{Wie zou hem kunnen,?}{opvolgen wie zou zijn plaats}{kunnen innemen}\\

\haiku{zonder meitskes, da,{\textquoteright}.}{h\"ob ich gaar oet gein spasz in}{antwoordt zijn buurman}\\

\haiku{niets anders dan het;}{krakend geritsel over de}{roode plavuizen}\\

\haiku{Enkelen, die nog;}{trachten te dansen op de}{waggelende beenen}\\

\haiku{er zijn enkele;}{nieuwe huizen gebouwd in}{de kom van het dorp}\\

\haiku{'t Zou werkelijk,.}{niet kwaad zijn als Wessels haar}{eens bij zich riep}\\

\haiku{Zoo bent u daar{\textquoteright}, vangt,, {\textquoteleft}}{hij aan in het Duitsch om haar}{ter hulpe te zijn}\\

\haiku{dat was ook geen werk,.}{voor u want u schijnt mij toe}{eene geleerde vrouw}\\

\haiku{vooral tegen de;}{koorts had ik een uitstekend}{recept gevonden}\\

\haiku{stijf, bewegingloos,.}{staat ze voor hem starend hem}{aan met holle oogen}\\

\haiku{een vaag twijfelen,.}{dat is geslopen in dat}{kinderlijk gemoed}\\

\haiku{{\textquoteright} {\textquoteleft}O Jeui, o Jeui, o,?}{Jeui waat zal er dan van mien}{erme vrouw w\`ere}\\

\haiku{{\textquoteleft}Stil Marieke, stil,;}{ich wil neet da's te dao nog}{oits euver  spriks}\\

\haiku{ich zal et seffens{\textquoteright}:}{in mien gebeibook ligke}{en dan tot Wessels}\\

\haiku{Hij zal de eerste,;}{weken niet kunnen zien heeft}{de dokter gezegd}\\

\haiku{Tevergeefsch heeft;}{Ubachs zijn kameraden dat}{alles gewezen}\\

\haiku{hij drukt haar hand weer,,.}{innig hartstochtelijk steeds}{dank stamelend}\\

\haiku{Marieke{\textquoteright} heeft hij:}{haar toegevoegd met ietwat}{haperende stem}\\

\haiku{Hij houdt plotseling, '.}{zijn woorden in terwijlt}{bloed naar zijn hoofd stijgt}\\

\haiku{een wraak, die leed zou,,.}{doen die zou pijnigen wreed}{verschrikkelijk wreed}\\

\haiku{ze aanschouwt Wessels,;}{vroolijk lachend in blijde}{opgewondenheid}\\

\haiku{{\textquoteleft}Waat h\"ob ich gehoird,{\textquoteright},:}{Marieke roept hij uit de}{verte reeds haar toe}\\

\haiku{Toen ineens weer een,.}{ontzettende angst die mijn}{lach deed verstommen}\\

\haiku{Achter een boom heb,,.}{ik gewacht loerend dat er}{iemand zou komen}\\

\haiku{Mijn broer heeft mij het;}{mij toekomend deel voor de}{voeten geworpen}\\

\haiku{waarom dat je nu.}{eerst erkent de moeder van}{Marieke te zijn}\\

\haiku{zou ik op mijn beurt,?}{nu ook iets mogen weten}{wat ik niet begrijp}\\

\haiku{{\textquoteright} {\textquoteleft}Dat kan ik je niet,,.}{zeggen mijn kind dat moet je}{vader zelf vragen}\\

\haiku{Dan heet zie mich ouch,.}{gelag veur eur deur zie heet}{mich verlaote}\\

\haiku{{\textquoteright} Spoedig is ze weer,.}{terug in de hut die ze}{straks heeft verlaten}\\

\haiku{ontzet aanschouwt hij,;}{dat gevaar dat hem omringt}{van alle zijden}\\

\haiku{Een waanzinnige,}{drift in hem als hij ziet die}{blijdschap als hij hoort}\\

\haiku{nooit zouden zij zich;}{meer neerzetten op die bank}{vlak voor zijn schuilplaats}\\

\haiku{dat woest dansen in;}{de lucht van het aan het koord}{bengelend lichaam}\\

\haiku{wij kunnen hem hier,{\textquoteright}.}{niet laten liggen een der}{gerechtsdienaars}\\

\haiku{{\textquoteleft}U moet er ook uit,,{\textquoteright}.}{burgemeester voegt een der}{gendarmen hem toe}\\

\haiku{hij huivert, als hij,.}{de beide vrouwen ontwaart}{gebukt over een bed}\\

\haiku{zij wil hooren uit,;}{zijn mond of haar kind in het}{leven zal blijven}\\

\haiku{{\textquoteleft}'t waor toch zien,,{\textquoteright}.}{keend al waor er ouch slech}{gewes stottert hij}\\

\haiku{{\textquoteright} {\textquoteleft}Als ik dat gedaan -?}{had wat zou je dan wel van}{mij gedacht hebben}\\

\subsection{Uit: Limburgiana}

\haiku{Ich h\"ob dich genog;}{in de gate gehauwe}{um dat te weite}\\

\haiku{m\`e noe wil ich gei,,.}{ruizie ich goon mit ich h\"ob}{ouch doors es e peerd}\\

\haiku{even draait het den kop,.}{om terwijl het mij aanstaart}{met vreesschuwe oogen}\\

\haiku{{\textquoteright} {\textquoteleft}Negen glazer beer,,?}{Madame verdaolt9}{geer uch neet dao m\`et}\\

\haiku{zoo gesmaden man,:}{dan lijdt het bijna immer}{eensluidend antwoord}\\

\haiku{allebei kunnen.}{ze hun op dit oogenblik}{gestolen worden}\\

\haiku{van allebei is..{\textquoteright}!}{ooze keel neuchter gebleve}{Vervloekte kerels}\\

\haiku{De Hemel zij dank,,,.}{zij is het niet een oude}{man die binnentreedt}\\

\haiku{geer zeet toch geine?}{badraof47 al maakt geer e}{bitteke ambras48}\\

\haiku{{\textquoteleft}'n raar taol, dat,.}{Hollandsch m\`et niks anders es}{fransche weurd er in}\\

\haiku{hij zal derhalve.}{slechts hebben te tellen en}{te teekenen}\\

\haiku{{\textquoteright} {\textquoteleft}Maar verklaar mij dan ',.}{toch ins Hemelsnaam waar}{je man is vrouwtje}\\

\haiku{Alles op zijn plaats,,....}{ja alles terwijl z'n oogen}{vorschend rondstaren}\\

\haiku{Zie zoo, eindelijk,:}{als overal onzichtbaar is}{gemaakt het bewijs}\\

\haiku{In Roosdaal troont de {\textquoteleft}.}{RederijkerskamerIn}{Liefde Bloeiende}\\

\haiku{Toen eindelijk de,;}{plechtige dag maar tevens}{de dag der vreugde}\\

\haiku{wat is dat veur 'n,?}{negerij woe de hoeser}{gein nommers h\"obbe}\\

\haiku{{\textquoteleft}Tr\"uk{\textquoteright} dondert deze,.}{terwijl hij de getrokken}{sabel hoog zwaait}\\

\haiku{Hei, juffrouw, bezeet}{uch dat mer ins good en kint}{geer casueel ouch}\\

\haiku{Zoo steeds in hunne;}{gedachten een beurtelings}{hopen en vreezen}\\

\haiku{de muzikanten;}{brengen de instrumenten}{aan hunne lippen}\\

\haiku{vreugderazend hollen zij}{weg naar alle kanten om}{uit te schetteren}\\

\haiku{veer hauwe ze toen,;}{nog neet die van noe die mer}{altied door scheete}\\

\haiku{in de bevende,.}{perkamenthanden steeds een}{dikken ruwen stok}\\

\haiku{hij heeft er gebracht,;}{z'n enkele povere}{meubeltjes z'n bed}\\

\haiku{we zullen eens zien:}{of je voor de Rechtbank ook}{niets weet -denk d'raan}\\

\haiku{{\textquoteright} {\textquoteleft}De pastoor heeft hier,.}{niets te maken laat die er}{asjeblieft buiten}\\

\haiku{{\textquoteleft}'t Zal veur dich nog{\textquoteright},.}{al schaoi zien zegt een hunner}{tot den kastelein}\\

\haiku{die vrem prei\"en et -,{\textquoteright}.}{us komme opvrete noets}{noets van z'n leve}\\

\haiku{roode haren, die.}{stijf opstaan op het lage}{bollende voorhoofd}\\

\haiku{Hij ziet kinderen;}{van zijn leeftijd met elkaar}{spelen en stoeien}\\

\haiku{de menschen met voor.}{het meerendeel ons geheel}{vreemde gezichten}\\

\haiku{e paar jaor es,}{dich alles verruniweerd}{is dan kinste d'n}\\

\haiku{Van af dat moment,;}{een geheel zich geven zich}{wijden aan dat kind}\\

\haiku{{\textquoteleft}Lieske, ene tourn\'ee{\textquoteright},.}{veur mich roept hij nu zelf tot}{de herbergierster}\\

\haiku{{\textquoteleft}ongelukkige,.}{k\`el ene zoeplap gewore}{allein oet chagrijn}\\

\haiku{{\textquoteright} Een even vragende,.}{blik tot haar man beantwoord}{met korten hoofdknik}\\

\section{Lambrecht Lambrechts}

\subsection{Uit: Het wingewest}

\haiku{want een schoolmeester,....}{krijgt maar elfhonderd frank}{om te beginnen}\\

\haiku{Jasperken is het,.}{eigenlijk die mij op de}{beenen geholpen heeft}\\

\haiku{Ofwel, zie, hij zou!}{er eene moeten opleiden}{naar zijn eigen hand}\\

\haiku{- Een der talen, die.}{het minst gesproken worden}{op het wereldrond}\\

\haiku{Ge zoudt voorwaar niet,!}{zeggen dat hij in een leemen}{huis geboren is}\\

\haiku{- Ik wed dat hij zelfs!}{deken benoemd zal worden}{met den eersten keer}\\

\haiku{zijn broer leest de mis,.}{bij den graaf van Sonnebeek}{die senator is}\\

\haiku{Zij kan v\'o\'orkomen,,,?...}{waar het noodig is de menschen}{te woord staan wa blieft}\\

\haiku{Daarmee zouden de.}{jongens van de straat af en}{de herberg uit zijn}\\

\haiku{Die van den bakker,,;}{zijn niet veel zegt gij en gij}{hebt wellicht gelijk}\\

\haiku{Vrouwen dulden geen,.}{verdeeldheid van gezag heb}{ik altijd gehoord}\\

\haiku{{\textquoteright} zonder te weten,.}{of een rups werkelijk kwaad}{k\`on zijn ja of neen}\\

\haiku{- Zoo lang ze niet met,.}{een anderen getrouwd is}{bestaat er nog kans}\\

\haiku{- Worden ze bij u?}{misschien verkocht tegen drie}{ellen voor een frank}\\

\haiku{Hoe meer kwaad zij over,.}{Jasper hoorde zeggen hoe}{liever zij het had}\\

\haiku{In dergelijke.}{bezigheidjes vond hij een}{roerend genoegen}\\

\haiku{Zij kende dagen,.}{waarop zij buitengewoon}{lekker gemutst was}\\

\haiku{En hennen, ja, ook...}{een tiental hennen zouden}{we kunnen houden}\\

\haiku{En een mooien haan, ' {\textquoteleft}!}{die er tusschen loopt ens}{morgensKukeluuk}\\

\haiku{Tot op zijn derde,.}{vel haddet gij hem moeten}{invetten Alexander}\\

\haiku{{\textquoteright} riep ik onlangs tot,.}{\'e\'en van de drie ik weet niet}{meer welke het was}\\

\haiku{die zal thuis wel niet.}{veel meer dan aardappelen}{en droog brood vinden}\\

\haiku{Fier-gelukkig.}{liet Jasper het mooie beeld van}{hand tot hand rondgaan}\\

\haiku{Een onzer eerste:}{nationale helden}{was een Limburger}\\

\haiku{De mooie mode kwam.}{het getal der uitvoerders}{nog verdubbelen}\\

\haiku{De paarden die de,:}{haver verdienen moeten}{ze ditmaal krijgen}\\

\haiku{- Zulk een vrouw bestaat,!}{in Helseghem niet misschien}{in heel Limburg niet}\\

\haiku{wil dien brief eens even, -.}{doorloopen het werk van een}{concurrent wellicht}\\

\haiku{- Heer Minister, ik,.}{kan de zaak dadelijk klaar}{spinnen zoo gij wilt}\\

\haiku{- Als dat z\`oo is, dan.}{kent de meester wellicht meer}{armoe dan weelde}\\

\haiku{Vergeleken met.}{de bakkersdochters is het}{een geleerde vrouw}\\

\haiku{Die is den heelen!}{dag bezig met jachthonden}{en watersneppen}\\

\haiku{een halfdonkere,;}{cirk waarin een juffrouw over}{een ijzerdraad liep}\\

\haiku{Toen het donker was,}{geworden sneed hij eenige}{bloeiende takken}\\

\haiku{- Ge gelooft toch niet,,?}{kerels dat ge mij iets op}{de mouw zult speten}\\

\haiku{{\textquoteleft}Aan den hoogsten boom!}{van Helseghem moesten ze dien}{schavuit opknoopen}\\

\haiku{Het rende heen en.}{weer in de kamer gelijk}{een das in het krijt}\\

\haiku{Neen, verwacht niet, dat,.}{een ernstig mensch uw taktiek}{goed zal keuren man}\\

\haiku{Te Paschen had}{zij hem verstooten en den}{heelen zomer door}\\

\haiku{En gij, gij zijt maar,,.}{een mensch een zwakke zondaar}{een arme stumperd}\\

\haiku{Gij eet uw hart op,?}{van verdriet en gij wilt het}{mij niet zeggen he}\\

\haiku{En als hij binnen,:}{kwam noodde hij niet altijd}{meer als te voren}\\

\haiku{- Waarom heb ik die?}{vreemde dingen eigenlijk}{allemaal geleerd}\\

\haiku{Maar de verwer was,.}{klaar-nuchter en daarbij}{rap als een weerlicht}\\

\haiku{- De hoovaardigen,.}{zullen vernederd worden}{leert het Evangelie}\\

\haiku{Zondag, na het lof,.}{moet gij beiden eens in de}{pastorij komen}\\

\haiku{Maar dat waagden de.}{schuchtere Limburgers dan}{toch weer niet te doen}\\

\haiku{Ook heeft de Vlaming.}{geen het minste gevoel van}{eigenwaarde meer}\\

\haiku{- En zoo zal het na.}{tien jaren ook gaan in de}{Limburgsche Kempen}\\

\section{Olaf J. de Landell}

\subsection{Uit: De appels bloeien}

\haiku{zijn schuwe, blauwe,.}{ogen de kleine mond en het}{smalle gezichtje}\\

\haiku{Waarop Agneta,.}{zei heel best z\`elf het kind te}{kunnen opvoeden}\\

\haiku{- ~ Achter het huis.}{Wynendael ligt een prachtig}{rosariumpje}\\

\haiku{Ze sloot de deur even, '.}{nadrukkelijk als zem}{had geopend}\\

\haiku{Toen zij de trap af,,.}{liep was ze recht en haar ogen}{keken opgewekt}\\

\haiku{Dit was de eerste,.}{maal dat Coen voor publiek was}{opgetreden}\\

\haiku{Ze vonden er Coen,.}{met schitterende ogen en}{gloeiende wangen}\\

\haiku{Ze had een patroon,.}{gekocht in de stad en wol}{in twee kleuren blauw}\\

\haiku{Hij liet me 'n heel,,!}{lang lint zien van bloemen en}{die waren van glas}\\

\haiku{{\textquoteright} informeerde Coen,.}{daar hij behoefte hieraan}{scheen te gevoelen}\\

\haiku{Tante Agneta.}{wendde haar gelaat naar de}{tuin en belde Braam}\\

\haiku{{\textquoteleft}Ik heb geen vader,{\textquoteright},.}{en moeder meer vervolgde}{Coen en zuchtte diep}\\

\haiku{Omda'k nie deftig,,{\textquoteright}.}{bin zeit me moeder lichtte}{de Lange scherp toe}\\

\haiku{Lientje zuchtte. {\textquoteleft}Tot,{\textquoteright}.}{elke mogelijke prijs}{zei Agneta hard}\\

\haiku{{\textquoteright} Hij overdacht met een,.}{wee gevoel of hij hier lang}{zou kunnen blijven}\\

\haiku{Coen zat rechtop in,.}{bed met warrige krullen}{en een bleek gezicht}\\

\haiku{Niemand had ooit de.}{freules zulke dingen in}{hun gezicht gezegd}\\

\haiku{Beneden, onder,:}{de schemerlamp zei Alexander}{tegen zijn ouders}\\

\haiku{En in 't vervolg.}{ga je niet meer om met die}{proleet van Gaalders}\\

\haiku{Wat deden zij daar,?}{ook op een uur dat niemand}{hen daar verwachtte}\\

\haiku{Doch Coen zag hem met,.}{zulke vlammende ogen aan}{dat Braam terugweek}\\

\haiku{Ze hadden bijna,.}{even lang op het Huis gewoond}{Agneta en Braam}\\

\haiku{Ze waren fragiel,.}{en uit de tijd zoals ze}{daar binnen gleden}\\

\haiku{Wij hebben hem dat,'...}{tweede afscheid niet bespaard}{na zijn ouders dood}\\

\haiku{de klank bleef hangen {\textquotedblleft},,{\textquotedblright}.}{in een blauw bordje metOost}{West thuis best erop}\\

\haiku{Hij vroeg beteuterd,.}{of die vader niet op een}{stoel kon plaatsnemen}\\

\haiku{{\textquoteright} murmelde ze, {\textquoteleft}is,... -{\textquoteright};}{dat die Verbrinke die Coen}{Agneta kuchte}\\

\haiku{Lientje stak de brief,.}{kalm in haar tasje en deed}{er het zwijgen toe}\\

\haiku{Als ik later groot,...,,}{ben mag er niemand weggaan}{die ik aardig vind}\\

\haiku{s Middags kochten.}{zij cadeautjes voor meneer}{Alexander z'n ouders}\\

\haiku{- Alexander, voorzichtig,.}{de deur sluitend wendde zich}{zuchtend naar de trap}\\

\haiku{De Lange had ook,.}{een vuurpijl afgeschoten}{naar eigen zeggen}\\

\haiku{Coen had de geestkracht,.}{niet om een verpletterend}{zwijgen te ontgaan}\\

\haiku{{\textquoteright} zei Klop, en legde.}{liefkozend zijn hand op de}{glanzende zijkant}\\

\haiku{Drie huizen verder '.}{was die morgen een man van}{t dak gedonderd}\\

\haiku{De Lange zat recht,.}{tegenover hem en kauwde}{op een grassprietje}\\

\haiku{Al die tijd hadden -.}{zij verschillende dingen}{gedacht en gezien}\\

\haiku{Meneer Alexander kon,.}{zo hartelijk lachen met}{het hoofd achterover}\\

\haiku{{\textquoteleft}Zouwe die tuntels?!}{van jou nou nog altijd nie}{groot genog weze}\\

\haiku{Dit alles in streng,;}{geheim zoals alles in}{het dorp placht te gaan}\\

\haiku{Dat veroorzaakte.}{een nadenkend zwijgen bij}{de tegenpartij}\\

\haiku{Ten slotte zat Coen,;}{op een dun-potig stoeltje}{wit-met-goud}\\

\haiku{{\textquoteleft}We zijn er nog te...,{\textquoteright}.}{jong voor en wilde daarmee}{zijn afschuw sussen}\\

\haiku{Nee, Coen meende wel.}{zijn ganse leven alleen}{te zullen blijven}\\

\haiku{{\textquoteright} zei boven haar de,.}{stem die eensklaps zo z\'e\'er op}{de hare geleek}\\

\haiku{dat ik 'm van u -?...}{k\'o\'op en d'r nog een m\`ens van}{probeer te maken}\\

\haiku{Maartje tolde terug.}{in haar slaafse toewijding}{aan de familie}\\

\haiku{{\textquoteleft}Alles,{\textquoteright} fluisterde,.}{ze en begon haastig de}{scherven te ruimen}\\

\haiku{- ~ In de loop van.}{de namiddag werd er aan}{de voordeur gescheld}\\

\haiku{Woarom h\`e' je ze '??!}{nie stijf gevloekt en de oge}{uitr kop gesch\`opt}\\

\haiku{Op meneer Alexander,;}{die zo-maar trouwt terwijl}{ik er niets van weet}\\

\haiku{En ik dacht erover,...{\textquoteright} {\textquoteleft}?!}{dat ik zelf misschien nooit zou}{trouwenWoarom niet}\\

\haiku{een hoofd groter dan,.}{Kaarsberg jong en zich bewust}{van zijn lichaamskracht}\\

\haiku{Zij liepen hard door,.}{het weiland en sprongen hoog}{en v\`er over slootjes}\\

\haiku{{\textquoteleft}Ook met de handen.}{{\`\i}n mijn zakken ben ik een}{Borgh van Wynendael}\\

\haiku{Hadden ze dan niet,?}{steeds het goede gedaan voor}{deze jongen}\\

\haiku{{\textquoteleft}En d\`enk erom, neef,{\textquoteright}, {\textquoteleft}.}{Barend drong Agneta aan}{n{\'\i}\'et per trein terug}\\

\haiku{Van zijn gastvrouw kreeg.}{hij een haastige handdruk}{met een  glimlach}\\

\haiku{{\textquoteleft}We zullen toch maar,,{\textquoteright}.}{pogen iets beters voor je}{te vinden zei hij}\\

\haiku{In stilte hoopte,:}{hij dat Peun niet aan dat pak}{mee-betaalde}\\

\haiku{Nicht Ida wuifde nog,.}{even doch de tram moest wachten}{op een verkeerslicht}\\

\haiku{De stenen van dit!}{huis laten elkander los}{van liefdeloosheid}\\

\haiku{En al die tijd heb,!}{ik gewerkt en gestudeerd}{om niet te sterven}\\

\haiku{Gelukkig begon,.}{hij toen weer over geesten met}{bloedende monden}\\

\haiku{Een stuk jeugd viel van -.}{hem af en de Lange moest}{daar even van stilstaan}\\

\haiku{Bij ons ligt brood nooit,{\textquoteright}, {\textquoteleft} -}{op de grond zei hijen wij}{gooien er niet mee}\\

\haiku{Onbegrijpelijk,.}{dat hij daarnet over de dood}{had lopen soezen}\\

\haiku{{\textquoteleft}Mar Jezus, jonker, ',{\textquoteright}.}{t is bij tiene zei de}{politie-agent}\\

\haiku{Elk in zijn branche,.}{zei vrolijk dat het wel in}{orde zou komen}\\

\haiku{{\textquoteleft}Ik bin d'r kepot,{\textquoteright}.}{van vervolgde de Lange}{met een lage stem}\\

\haiku{{\textquoteleft}Maar ik kan niet zo, '.}{erg lang want ik m\'o\'et nogn}{massa werk leren}\\

\haiku{Daar had hij altijd,,.}{geweten wat hij wilde}{zeggen en zou doen}\\

\haiku{langs slingerende,.}{paadjes achter in de tuin}{waar het hout dicht was}\\

\haiku{als kind had hij er.}{niet mogen spelen omdat}{er zoveel glas was}\\

\haiku{de sfeer in huis was.}{er een van plechtig bezoek}{en ongewoonte}\\

\haiku{Het was misschien wel,.}{gek zo op stel en sprong het}{huis te verlaten}\\

\haiku{hel, purper-roze,,.}{met schichten van het witste licht}{dat er kan bestaan}\\

\haiku{Het ontbijt verliep,.}{zo snel dat Coen geen tijd vond}{om na te denken}\\

\haiku{{\textquoteright} Zijn vriend bekeek hem:}{afkeurend en gaf als enig}{antwoord te kennen}\\

\haiku{{\textquoteright} Toen klommen ze uit.}{de wagen en gingen aan}{de wegkant zitten}\\

\haiku{{\textquoteleft}Dat heb ik toch m'n,{\textquoteright}.}{hele leven gedaan gaf}{Coen vriendelijk toe}\\

\haiku{Er tjilpte ergens,.}{een schelle vogel-stem}{en Coen glimlachte}\\

\haiku{de mond gaf niet af!).}{en zonk schokkend in het open}{portier op de mat}\\

\haiku{{\textquoteleft}Als u maar niet zo,,,}{boos was zou ik vragen of}{u met me wilt eten}\\

\haiku{Hij had eigenlijk.}{alleen maar de warme toon}{van haar stem gehoord}\\

\haiku{{\textquoteright} Maar daarna bedacht,.}{hij hoeveel kostbaar eten ze}{hem had afgestaan}\\

\haiku{Ik zou 't prettig,{\textquoteright}.}{vinden zei Verbrinke met}{zijn warmste glimlach}\\

\haiku{een vrouwenlach en,.}{een babbelend kwetterend}{kinderstemmetje}\\

\haiku{{\textquoteleft}Nee,{\textquoteright} zei de Lange, {\textquoteleft}{\textquoteright}!}{botik k\`en dat. Toen moest Coen}{d\`at nog vertellen}\\

\haiku{Hoe kan een mens zo!}{veel ervaringen in zo'n}{korte tijd opdoen}\\

\haiku{Hoewel hij de pijn.}{in haar blik niet zou hebben}{kunnen verdragen}\\

\haiku{Elisa en Alexander;}{keken haar met grote ogen}{aan en glimlachten}\\

\haiku{Hij schokte op, en,.}{ze had durven wedden dat}{hij alweer rood werd}\\

\haiku{Ze hielden zulke, '.}{gesprekken samen als Coen}{s morgens opstond}\\

\haiku{Na een h\'e\'el zachte,,:}{mislukte klap van hem zei}{de bokser nog eens}\\

\haiku{maar hij zweeg, en kon,.}{niet voorkomen dat hij n\`og}{eens hevig bloosde}\\

\haiku{{\textquoteright} vroeg Coen, zijns ondanks.}{geboeid door de ernst van de}{Lange z'n gezicht}\\

\haiku{Toen hij ongeveer,:}{acht maanden les had zei de}{danser op een keer}\\

\haiku{Neef Barend tikte {\textquoteleft}...{\textquoteright}.}{aan zijn hoedrand en zeiD\`aggg}{en stapte voorbij}\\

\haiku{Hij had zich alles.}{duidelijk door Verbrinke}{laten uitleggen}\\

\haiku{Nu blijft het boven,{\textquoteright},.}{nog even na-klinken dacht}{Coen met bonzend hart}\\

\haiku{{\textquoteleft}Maar dat is toch geen!}{excuus voor dergelijke}{nalatigheden}\\

\haiku{Hij keek de kring rond,.}{zoals hem nooit tevoren}{vergund was geweest}\\

\haiku{Zij liet het huis zien.}{met de onpersoonlijkheid}{van een rondleidster}\\

\haiku{{\textquoteleft}Die oude mensen,}{moeten een geweldige}{staat hebben gevoerd}\\

\haiku{Maar dat besefte,.}{Coen niet en het zou hem ook}{niet hebben geraakt}\\

\haiku{Iedereen kon n\'u,.}{begrijpen dat hij Mona}{alleen bedoelde}\\

\haiku{{\textquoteright} Hij glimlachte ook,.}{maar deed geen moeite om het}{vriendelijk te doen}\\

\haiku{Maar dat was niet het,,.}{antwoord wat iemand jou had}{mogen geven toen}\\

\haiku{- Dit gebeurde de,.}{zesde dag dat Coen in de}{revue danste}\\

\haiku{{\textquoteleft}Omdat ik het heb,.}{met die arme boeren die}{teveel betaalden}\\

\haiku{{\textquoteleft}Mooi is belangrijk,,{\textquoteright}.}{in het lelijke leven}{leerde Mona hem}\\

\haiku{En het brugje over,.}{de gracht dat tante Lientje}{had laten leggen}\\

\haiku{De vier hoeken van.}{de lijst droegen het wapen}{Borgh van Wynendael}\\

\subsection{Uit: Ave Eva}

\haiku{En Eva begreep, hoe.}{grof hij zou zijn tegen de}{eerstvolgende klant}\\

\haiku{{\textquoteleft}Ja,{\textquoteright} stemde Sally,.}{in vol medeleven met}{Eva's medeleven}\\

\haiku{{\textquoteleft}Zeg, Eef, denk erom,,.}{dat je tegenover Derk de}{brutale speelt hoor}\\

\haiku{Ze had geen tijd om.}{te lachen of te huilen}{of adem te halen}\\

\haiku{Het is eigenlijk,.}{voor niemand een kunst op die}{manier mooi te zijn}\\

\haiku{Er klonk flirt in zijn,.}{stem wat zij tot elke prijs}{wilde omzeilen}\\

\haiku{Het was lang niet zo,.}{moeilijk als ze gedacht had}{met hem te praten}\\

\haiku{En als ik denk, dat,,.}{het genoeg is dan scheur ik}{me los dat het knalt}\\

\haiku{Toen begreep Eva pas,,.}{dat hij niet om haar woorden}{lachte maar om h\'a\'ar}\\

\haiku{'t Was wel weer een,,!}{leugen maar ach aan een boom}{zo volgeladen}\\

\haiku{{\textquoteright} en daar moest Derk weer,.}{zo ontzettend om lachen}{dat hij haar losliet}\\

\haiku{{\textquoteright} {\textquoteleft}En laat een kamer.}{in orde maken voor dit}{schattige meisje}\\

\haiku{Ze stond gewoonweg,?}{in brand van het blozen want}{wat was dat voor taal}\\

\haiku{Ze zag zichzelf al.}{onbeheerst liggen ronken}{tegen Derks arm}\\

\haiku{Zowel de vader,:}{als de moeder lachte en}{haar verloofde zei}\\

\haiku{{\textquoteright} bekende mevrouw,.}{Van Hellenduyn met iets van}{eerbied in haar stem}\\

\haiku{En morgen was het,...}{zaterdag daar kon Eva dus}{niets tegen hebben}\\

\haiku{Hij zuchtte en sloeg.}{een arm om haar middel en}{trok haar naar zich toe}\\

\haiku{Maar Eva kon niet met,.}{zekerheid vaststellen dat}{hij had gelachen}\\

\haiku{Ze vertelde hem.}{van haar vader en tante}{Gien en het kantoor}\\

\haiku{De trekpot deed zijn,.}{plicht met choquant geluid in}{de zware stilte}\\

\haiku{Derk zette haar de.}{volgende ochtend in de}{stad af bij Sally}\\

\haiku{{\textquoteleft}En nou is 'ie al,{\textquoteright}.}{veertig jaar met haar getrouwd}{besloot Derk bedrukt}\\

\haiku{En te weinig geld.}{gaf hij slechts als zij teveel}{uitgegeven had}\\

\haiku{De hit, die liep te,.}{slenteren schoot ervan in}{een haastige stap}\\

\haiku{Hoeveel?{\textquoteright} {\textquoteleft}Nou,{\textquoteright} - Eva dacht;}{aan Amanda P\`eche en aan}{Monsieur Fran\c{c}ais}\\

\haiku{{\textquoteleft}Lach maar gerust, - dacht,?!}{je dat iedereen jou hier}{au s\'erieux neemt}\\

\haiku{En ze voelde zich.}{doodongelukkig in haar}{kant en juwelen}\\

\haiku{{\textquoteright} Z'n haar had alle,.}{brillantine verzaakt en}{golfde om zijn hoofd}\\

\haiku{Het werd tijd, dat ze.}{het pad der dame trachtte}{terug te vinden}\\

\haiku{Was het niet, om \'uit,?}{te stappen en desnoods naar}{de stad te lopen}\\

\haiku{Op dergelijke,.}{lichtzinnige taal wist ze}{niets te antwoorden}\\

\haiku{Ze stonden in de,.}{hal en haar begeleider}{knipte het licht aan}\\

\haiku{Maar dat andere - - -.}{meisje en Derks gedrag}{dat stak haar het meest}\\

\haiku{{\textquoteleft}Ik heb een heerlijk,{\textquoteright}.}{ontbijt voor je trachtte Derk}{haar te verlokken}\\

\haiku{{\textquoteright} {\textquoteleft}Och joa, ieder wat,{\textquoteright}.}{antwoordde Gert P. met \'e\'en}{dichtgeknepen oog}\\

\haiku{Was het wel beleefd,,?}{om zo te schateren toen}{ze over de fuif sprak}\\

\haiku{Daar was gelukkig.}{afleiding genoeg om wat}{op adem te komen}\\

\haiku{Pa bok wist van niets,.}{meer dan van horenstoten}{en mummelde woest}\\

\haiku{Eva dacht, dat hij het.}{meest ge\"ergerd was over zijn}{eigen warme kop}\\

\haiku{Er zou wel iets niet,.}{in orde zijn dat er geen}{receptie van kwam}\\

\haiku{{\textquoteright} En daar moest pa Van.}{Hellenduyn zijn wenkbrauwen}{weer van rimpelen}\\

\haiku{Verschil in leeftijd,{\textquoteright}.}{misschien dacht mevrouw Van der}{Dam pessimistisch}\\

\haiku{{\textquoteright} kon niet verhoeden,.}{dat mevrouw Van Hellenduyn}{licht gaf van vreugde}\\

\haiku{Op 't laatst kon je '.}{wel voort hele leven}{gekke ogen zetten}\\

\haiku{Het was duidelijk,.}{dat zijn trots hierbij niet op}{stal kon blijven staan}\\

\haiku{En de verliefde,.}{roos die het tuinpoortje zo}{lyrisch omhelsde}\\

\haiku{Had hij niet gezegd, {\textquoteleft}{\textquoteright}?}{van dinsdagnacht en had zij}{nietja geantwoord}\\

\haiku{{\textquoteright} zei Derk voor zich heen,.}{en striemde het rulle zand}{met een korenaar}\\

\haiku{En Aatje Broek vond '.}{t meisje van meneer Derk}{een fijne dame}\\

\haiku{{\textquoteright} {\textquoteleft}'t Lijkt me heerlijk,,{\textquoteright},.}{als er niemand komt overwoog}{Eva veel te verhit}\\

\haiku{{\textquotedblright}{\textquoteright} En een vrachtrijder,,:}{overtuigende schakel met}{de stad had beweerd}\\

\haiku{Dat was hoog nodig,.}{want de oude Eva groeide}{door de nieuwe heen}\\

\haiku{{\textquoteright} wuifde mevrouwtje,.}{Van Hellenduyn vanaf de}{marsepeinen stoep}\\

\haiku{De auto schoot met,.}{zo'n vaart weg dat tante's hoofd}{bijna achterbleef}\\

\haiku{Het was helemaal,!}{niet aardig want er kon toch}{best iets gebeuren}\\

\haiku{{\textquoteright} gilde tante aan,,.}{Derks oor met priemende}{vijandige ogen}\\

\haiku{Ze werd zolang bij,.}{Sally gedeponeerd die}{druk zat te typen}\\

\haiku{Sally vroeg niets en,.}{zei weinig want haar nieuwe}{artikel moest af}\\

\haiku{Zij beheerste zich voor, {\textquoteleft}}{\'e\'enmaal niet en gaf een}{klap onder zijn hand.}\\

\haiku{Ja, Derk had haar maat,.}{onthouden van die eerste}{ring met amethysten}\\

\haiku{Lang zal ze leven{\textquoteright},.}{hoewel het orgel een aria}{van Tosca draaide}\\

\haiku{Eerst vader, toen de ', - - {\textquoteleft}!}{baas vant kantoor enDat}{vervloekte geklets}\\

\haiku{Toen zag ze Derks,,:}{wit verbeten gezicht en}{restaureerde weer}\\

\haiku{{\textquotedblleft}Ga nou eindelijk,.}{eens mee naar huis nu heb je}{het niet te druk meer}\\

\haiku{12 {\textquoteleft}Eefje,{\textquoteright} zei Derk,, {\textquoteleft}}{plechtig toen ze in de tuin}{waren aangeland}\\

\haiku{Zij probeerde met ',.}{m te praten zonder dat}{de baas het merkte}\\

\haiku{Ze wilde het graag,.}{vergeten maar kon aan niets}{anders meer denken}\\

\haiku{ze moest er haar neus,.}{van snuiten want alles had}{zo mooi kunnen zijn}\\

\haiku{{\textquoteright} Eva had het gevoel,,.}{of ze op haar hoofd stond en}{kon geen woord spreken}\\

\haiku{En wie weet, hoeveel!}{er nog met achttienhonderd}{gulden te doen viel}\\

\haiku{Als ik je met iets,, '.}{kan helpen dan weet je dat}{k voor je klaar sta}\\

\haiku{{\textquoteleft}Vader is over de,.}{kop geweest en dat had ik}{moeten begrijpen}\\

\haiku{Waarom gaat een man, '?}{houden van juist die vrouw die}{m niet wil hebben}\\

\haiku{Ze stond op en even.}{staarde ze peinzend over Eva's}{hoofd heen het raam uit}\\

\haiku{En op dat ogenblik.}{werd ze pas verblind door de}{gloed van het geluk}\\

\haiku{Daarom wist ik toen,,.}{opeens dat je geschikt was}{om Derk te helpen}\\

\haiku{En begon maar vast.}{de haakjes van Eva's grijze}{ruit los te maken}\\

\haiku{{\textquoteleft}Het is jammer, dat,{\textquoteright}.}{je zo weinig met hem op}{hebt zei Sally zacht}\\

\haiku{En Sally had de,...}{brief waarin Derk zijn liefde}{voor Eva bekende}\\

\haiku{Pas loater is, '...}{de dokter gekommes}{morgens om drie uur}\\

\haiku{Die richtte zich op.}{en duwde het baretje}{in de juiste stand}\\

\haiku{Ze begroette Eva,.}{achteloos alsof deze}{niet was weggeweest}\\

\haiku{en Aatje Broek had '.}{rimmetiek en haar dochter}{n zere vinger}\\

\haiku{{\textquoteleft}Zijn jullie daar nog,,?}{monsters winterwortels en}{zure augurken}\\

\subsection{Uit: Blonde Martijn}

\haiku{Ik kreeg een kleur en.}{probeerde een andere}{kant op te kijken}\\

\haiku{Zijn blik rustte kalm {\textquoteleft}?}{en vriendelijk op me.Wat}{is er aan de hand}\\

\haiku{{\textquoteleft}Dit is gedaan op.}{de dag dat Nellie in haar}{wasbeurt werd gestoord}\\

\haiku{Want o, wat weet ik -.}{nog goed dat verhaal het {\`\i}s}{Nellie van der Grijp}\\

\haiku{God, wat heb ik veel,.}{van hem gehouden in die}{schaarse ogenblikken}\\

\haiku{Hij moet een hele.}{tijd doodstil op de zolder}{hebben gezeten}\\

\haiku{een appelige.}{rijpheid was met het blote}{oog kenbaar aan haar}\\

\haiku{De werklui hebben;}{het dakbeschot nu bijna}{geheel verwijderd}\\

\haiku{En opeens heeft hij -.}{er eentje gepakt fel en}{raak om haar middel}\\

\haiku{{\textquoteleft}Eigenlijk heb jij,...{\textquoteright}}{me ook gekust toen je die}{lippen aanraakte}\\

\haiku{Maar ja, ik was toen,.}{pas zestien en Martijn was}{al een-en-twintig}\\

\haiku{-        5 Zijn tante,,.}{mevrouw Van Haysmaal heb ik}{pas later ontmoet}\\

\haiku{Toen kwam mijn vader,.}{net binnen niets wetend van}{enig ernstig gesprek}\\

\haiku{Iets heel liefs en toch,.}{flink en sterk waarmee je kon}{stoeien en praten}\\

\haiku{{\textquoteright} Dat Martijns pink het,.}{nadeed maakte de zaak slechts}{begrijpelijker}\\

\haiku{- Hij liep het huis uit,.}{klom op z'n fiets en hijgde}{naar de politie}\\

\haiku{als vrouwen onder.}{elkaar kon je de details}{beter bespreken}\\

\haiku{Hij had zijn woede.}{gelaafd met de aanblik van}{moeder en dochter}\\

\haiku{De zwijgzaamheid zonk.}{met steeds zwaarder weefsel op}{Alice Bronneberg}\\

\haiku{Die gewoon naar het.}{buffet ging en er in een}{lade rommelde}\\

\haiku{Hij schoofde lade.}{dicht en wendde zijn ogen naar}{Alice Bronneberg}\\

\haiku{Later trof ik mijn,.}{moeder terwijl ze stond te}{telefoneren}\\

\haiku{je bent een stukje,!}{stille lente waar ik van}{zou kunnen janken}\\

\haiku{Angelique was:}{toen op zijn schoot gaan zitten}{en had geantwoord}\\

\haiku{Dat mocht niet van mijn -.}{moeder ik drentelde dus}{niet te dicht bij huis}\\

\haiku{{\textquoteleft}Hij houdt zo veel van,{\textquoteright}.}{Emilie van Wijdevelden}{fluisterde Agnietje}\\

\haiku{Van de werkster, die,;}{dikke Marie moest helpen}{tweemaal in de week}\\

\haiku{{\textquoteleft}Martijn,{\textquoteright} zei ik toen, {\textquoteleft},?...}{eindelijk als een ventje}{houdt van Emilie h\`e}\\

\haiku{Hoewel ik toch zelf.}{zo ten diepste was geboeid}{door Agnietje Weisse}\\

\haiku{Je hebt gedacht dat;}{je een uurwerk was om de}{tijd aan te wijzen}\\

\haiku{Wij weten ook, dat,...}{dit een vliegtuig is en dat}{er een mens in zit}\\

\haiku{Wellicht was dat de,.}{grief waarom het stuk naar de}{zolder was verhuisd}\\

\haiku{{\textquoteleft}Ja, \`of je weet wat,;}{hij gedaan heeft en dan moet}{je me waarschuwen}\\

\haiku{En als je denkt dat,.}{ik het niet begrijp moet je}{het me uitleggen}\\

\haiku{Hij zuchtte ervan,.}{en haastte zich om het huis}{heen naar het balkon}\\

\haiku{De notaris zou.}{inlichtingen inwinnen}{over een avondcursus}\\

\haiku{Dezelfde brede, -.}{soepele mond iets verhard}{door ervaringen}\\

\haiku{Hij blikte van haar,.}{weg over de Meent waar paarden}{liepen te grazen}\\

\haiku{{\textquoteright} want ik vond het maar,.}{een bedompt hokje daar aan}{het eind van de gang}\\

\haiku{Ze konden erop.}{zweren blonde Martijn te}{hebben aangepakt}\\

\haiku{En nu zou hij de.}{stilte gaan vieren in de}{Blauwe Kamer}\\

\haiku{{\textquoteright} Achter mij hoorde,.}{ik Bart lachen ver boven}{het radiootje uit}\\

\haiku{waarom sloeg ze niet,?}{een arm om me heen en vroeg}{gewoon naar Martijn}\\

\haiku{Op een avond waren.}{er twee vriendinnen bij mijn}{moeder op bezoek}\\

\haiku{Het drong zuur door hout.}{en kalk en vond mij in m'n}{eigen kamertje}\\

\haiku{{\textquoteleft}Dat is h{\'\i}j,{\textquoteright} zei ik,.}{in een wrange poging om}{iets te doorbreken}\\

\haiku{Ze legde haar wang.}{tegen de mijne en hield}{me tegen zich aan}\\

\haiku{En dan komen er.}{weer andere auto's die}{het puin ophalen}\\

\haiku{{\textquoteright} Of een laatste wuif.}{van een lichaamsdeel dat het}{niet wil afleren}\\

\haiku{U hebt waarschijnlijk,{\textquoteright}.}{nooit een vriend verloren door}{de dood zei ik dan}\\

\subsection{Uit: De dief stelen}

\haiku{{\textquoteright} {\textquoteleft}Lafbek,{\textquoteright} gromde Pol, {\textquoteleft}!}{om zo'n kind met die goudsbloem}{te laten trouwen}\\

\haiku{{\textquoteleft}We mogen in geen,,}{geval vergeten dat de}{goede God ons ziet}\\

\haiku{{\textquoteleft}Ze moet veel gekker,{\textquoteright}, {\textquoteleft}!}{doen dacht Polanders wordt ze}{zenuwpati\"ent}\\

\haiku{Hij keek er kies bij,.}{op z'n bestelling alsof}{hij deze oplas}\\

\haiku{Morgane boog zich:}{naar Pol en vroeg met weer die}{duidelijke stem}\\

\haiku{En een salade..., -....}{Italienne en eh une}{fromage vari\'ee}\\

\haiku{Maar ja, ze was zo,....}{dom telkens die garenklos}{te laten vallen}\\

\haiku{zodat hij als een!}{vlieg langs de muur naar boven}{zou kunnen lopen}\\

\haiku{{\textquoteleft}Alors,{\textquoteright} Toine haalde, {\textquoteleft} -.}{diep ademkus de Madonna}{je mag niet vallen}\\

\haiku{Bah! - Morgane trok:}{het licht uit met het gouden}{koord boven haar bed}\\

\haiku{{\textquoteleft}Ja, velen zien mij,,}{voor een dichter aan maar ik}{ben meer een opener}\\

\haiku{Ik heb u trouwens,.}{vanavond al verteld dat ik}{niet dicht als ik werk}\\

\haiku{De gasten schreden.}{op droombenen  aan en}{groetten Pol beleefd}\\

\haiku{Morgane trachtte.}{hem kennelijk in een val}{te laten lopen}\\

\haiku{Nee, niet die witte -,.}{de gouden waar je straks mee}{in je handen stond}\\

\haiku{{\textquoteleft}Als de inbreker,.}{succes had gehad zou dit}{zijn winst zijn geweest}\\

\haiku{met een schatrijke,,.}{vrouw die nu al bedacht wat}{hij moest aantrekken}\\

\haiku{Ido liet zijn ogen uit {\textquoteleft}?}{de azuren lucht neerdalen}{op Pol.Wat zeg je}\\

\haiku{Vlot, vormelijk, met.}{rollende zinnen en een}{ouwelijk handje}\\

\haiku{{\textquoteleft}Ik wou graag, dat je,.}{met me meeging papa en}{mama afhalen}\\

\haiku{{\textquoteright} Deze opmerking.}{lichtte de ontmoeting in}{de trein wel schel uit}\\

\haiku{Ik zal haar zeggen,,.}{dat ze de buitendeur los}{moet laten vannacht}\\

\haiku{Moest ze het eigen.}{bloed niet loslaten aan de}{Middellandse zee}\\

\haiku{Een proleet, zeg, die!...}{op straat ligt met allerlei}{goedkope slungels}\\

\haiku{Elke gast kon met.}{de directie een ander}{code vaststellen}\\

\haiku{Ik wil, dat jullie,{\textquoteright}.}{verdomd goed op me passen}{zei Pol vermanend}\\

\haiku{{\textquoteright} en hij legde de.}{hevig murmelende hoorn}{tevreden terug}\\

\haiku{En heeft de sterke?}{dame je voldoende met}{rust kunnen laten}\\

\haiku{Toen er negentien,;}{schelpjes in haar zak zaten}{was de zon onder}\\

\haiku{Morgane was uit -.}{wandelen met Van Aadel wat}{haar geen plezier deed}\\

\haiku{{\textquoteright} converseerde Pol. {\textquoteleft}, -!}{Kijk straks bruiste de regen}{en w\`eg is alles}\\

\haiku{Je moet je veilig...}{stellen voor je ouders en}{hun beslissingen}\\

\haiku{Ido was geslagen,.}{weggegaan en Pol had een}{gedicht geschreven}\\

\haiku{{\textquoteleft}Heb je een boompje,?..}{kunnen vinden waarvan de}{vruchten rijp waren}\\

\haiku{Een gevoel of je,?...}{in bloei staat als een van die}{struiken buiten h\`e}\\

\haiku{Ja, oppassen dat....}{niemand anders hem bereikt}{met een beter bod}\\

\haiku{De titel - nee, dat?...}{ik uw dochter dat boekje}{ten geschenke geef}\\

\haiku{{\textquoteleft}Een vrouw gaat voorbij,,{\textquoteright}.}{en haar mantel is van puur}{goud vertelde hij}\\

\haiku{Hij grijpt de vrouw aan.}{en ontneemt haar het witte}{kleed met de parels}\\

\haiku{En ze m\'o\'est hem in -.}{zijn beweging volgen het}{gelukte haar ook}\\

\haiku{Ze vroeg zich met angst,.}{af wat d\`at nu weer had te}{betekenen}\\

\haiku{Er was dus iemand,.}{die hem in de gaten hield}{en nut van hem had}\\

\haiku{{\textquoteright} Hij mocht niet geheel -.}{en al loochenen dat kon}{gevaarlijk blijken}\\

\haiku{Ido had hem een paar,:}{keer strak aangekeken en}{\'e\'enmaal gezegd}\\

\haiku{De nabije lantaarn.}{stond in een omarming van}{oleander-groen}\\

\haiku{Hij stapte ferm door,.}{perken en gazons tot hij}{onder de vlieg stond}\\

\haiku{Lieve kind,{\textquoteright} zei Pol, {\textquoteleft}}{met een grinnik die alle}{ernst moest ontberen}\\

\haiku{{\textquoteright} Hij bekeek het stuk.}{met welbehagen en gaf}{het haar toen terug}\\

\haiku{{\textquoteright} zei ze nadenkend, {\textquoteleft},.}{vooral als we weten dat}{d\`at waarheid behelst}\\

\haiku{Meneer Merlin is!...}{gezond en sterk en zo lief}{voor alle mensen}\\

\haiku{Ik kom tegenover.}{de scherven van een derde}{huwelijk te staan}\\

\haiku{{\textquoteleft}Ik zal helemaal,{\textquoteright}.}{opnieuw moeten beginnen}{ging hij peinzend voort}\\

\haiku{Ze wankelde - - of - - - -:}{ze even zou spelen met hem}{Dat kon ze ook niet}\\

\haiku{Dat heb ik in jou -.}{gevonden en daar moet ik}{mijn deel van hebben}\\

\haiku{{\textquoteleft}Erger,{\textquoteright} antwoordde, {\textquoteleft}.}{Polin het gevang zit je}{met kameraden}\\

\haiku{Ze sloeg haar armen,.}{om zijn hals en kuste hem}{aandachtig terug}\\

\haiku{Pol, geef monsieur,,.}{mijn pas dan kan hij zien dat}{ik de waarheid spreek}\\

\haiku{Ja, - ze moesten koffers - -.}{pakken en afrekenen}{ze gingen naar huis}\\

\haiku{Je mag die stukken,.}{hebben als je in vrede}{met ons kunt leven}\\

\haiku{- zo is het mij ook,{\textquoteright}.}{niet opgevallen hoonde}{Venens sinister}\\

\haiku{Morgane voelde;}{een vreselijke spanning}{in haar geest komen}\\

\haiku{Ze keek nog eens om.}{en gierhuilde weer met de}{handen voor de ogen}\\

\haiku{{\textquoteleft}Met brandtrapjes en.}{afgesloten vensters en}{elektrische seinen}\\

\haiku{Vooral, nu er een.}{wettige echtgenote}{bij is gekomen}\\

\haiku{{\textquoteright} wikte Pol. {\textquoteleft}Maar je.}{vader zal alles vatten}{wat hij kan krijgen}\\

\subsection{Uit: In het hol van de tamme leeuw}

\haiku{Daarom was ze niets.}{minder romantisch in zijn}{leven gekomen}\\

\haiku{{\textquoteleft}Als u eens een keer,.}{n{\'\i}\'et langs kwam zou ik beslist}{ook op \'u wachten}\\

\haiku{- Ze moest zich haasten,,.}{om Paps te zeggen dat ze}{met Bob zou trouwen}\\

\haiku{{\textquoteright} {\textquoteleft}O, gos, nee, da's waar,{\textquoteright}.}{verdoezelde meneer zijn}{onvoorzichtigheid}\\

\haiku{Trouwens, je hebt die,...{\textquoteright}}{auto nog geen jaar als ik}{me goed herinner}\\

\haiku{Maar niets maakte hem,.}{ooit zo giftig als smelten}{door anders toedoen}\\

\haiku{meneer Van der Spa,;}{starend in een toekomst vol}{blijde gezichten}\\

\haiku{{\textquoteright} {\textquoteleft}Ik wou, dat ik een,{\textquoteright}.}{auto van hem kreeg zuchtte}{Liza uitgeput}\\

\haiku{Dus had Hetty niet:}{kunnen vertrekken zonder}{een opdracht aan Kees}\\

\haiku{{\textquoteright} zei meneer Van der,.}{Spa en brandde zich flitsend}{aan het belknopje}\\

\haiku{Misschien was hij in.}{zijn prilste dagen een goed}{jongetje geweest}\\

\haiku{{\textquoteleft}Ik vind die nieuwe.}{tuinjongen op \'e\'en na de}{grootste idioot hier}\\

\haiku{En mams heeft ze nooit,.}{gekregen dus die kan er}{niet over oordelen}\\

\haiku{er zijn er die een -,.}{man prachtig vindt er zijn er}{die hij niet kan zien}\\

\haiku{Maar elke man heeft,.}{er \'e\'en waar zijn hand zonder}{bedenken naar grijpt}\\

\haiku{Dat ziet alleen de,{\textquoteright},.}{buitenstaander antwoordde}{Bob haar aankijkend}\\

\haiku{{\textquoteleft}Kind,{\textquoteright} zei hij, {\textquoteleft}jou kan,.}{ik mijn zegen geven want}{dat maakt niet veel uit}\\

\haiku{Zijn gelaatstint klom.}{van tomaat via aardbei en}{radijs tot biet op}\\

\haiku{Liza richtte zich.}{op in haar huilstoel en hing}{haar ogen aan de deur}\\

\haiku{{\textquoteleft}Als je belooft dat,.}{ik met Bob mag trouwen zal}{ik je ophijsen}\\

\haiku{wat hij wellicht had,.}{ge\"erfd van de dame naar}{wie hij was genoemd}\\

\haiku{Het gaat me zo aan,{\textquoteright}.}{mijn hart van Kees zei ze met}{extra veel hoofdpijn}\\

\haiku{en hij wreef zich in.}{de handen tot er een lucht}{van geschroeid vlees hing}\\

\haiku{Van Dalen stond toen,.}{ook op en vroeg verlof om}{zijn hond te fluiten}\\

\haiku{Dat gedoe tussen,{\textquoteright}.}{jou en mijn zoon verklaarde}{de trotse vader}\\

\haiku{Een schaap zou sluik haar,.}{krijgen van schaamte als het}{z\'o moest mekkeren}\\

\haiku{en dat d{\`\i}t dus was,.}{hoe mensen een kapitaal}{konden verdienen}\\

\haiku{En of er wel eens,??}{was ingebroken in dat}{schatrijke gedoe}\\

\haiku{{\textquoteleft}Ik wil er meer van,,!}{weten Joris en op}{zeer korte termijn}\\

\haiku{{\textquoteleft}Het is \`opgebruikt,{\textquoteright}.}{constateerde meneer Van}{der Spa lamzakkig}\\

\haiku{Toekomst maakte je,.}{met jiu jitsu of met een}{aardappelmesje}\\

\haiku{Dat betekent dat,.}{ze \`of iets in haar schild voert}{\`of is afgeleid}\\

\haiku{{\textquoteleft}Ik denk, dat hij zijn,.}{schade inhaalt van een paar}{dagen tegelijk}\\

\haiku{Dat is de wulpse,...!}{bloeddorstige grimas van}{een bezetene}\\

\haiku{\`of ze heeft er n{\'\i}\'et,.}{mee te maken en dat is}{n\`og penibeler}\\

\haiku{Lotje bleek weinig.}{tijd te hebben voor verder}{omslachtig gesprek}\\

\haiku{{\textquoteright} wilde hij haastig,.}{weten kennelijk bezwaard}{door het kusverbod}\\

\haiku{Je ziet eruit als -,.}{een pistache en dat ligt}{niet aan jou dit keer}\\

\haiku{Zou juffrouw Parels,...?}{niet denken dat ze daarin}{onherkenbaar was}\\

\haiku{{\textquoteright} vertelde ze haar, {\textquoteleft}!}{moeder zeer stralenden daar}{kan ik niet meer uit}\\

\haiku{Als ze praatte, klom.}{er een onbekend beest in}{haar strot op en neer}\\

\haiku{Dit was het laatste,,.}{wat ze had verwacht van haar}{prilste jaren af}\\

\haiku{Eigenlijk kwam hij.}{veel te snel bij het hek van}{de grote villa}\\

\haiku{Ze zouden toch niet...?}{allemaal zijn gevlucht voor}{een ontzettendheid}\\

\haiku{{\textquoteleft}Ik weet zelf niet, waar, -{\textquoteright} {\textquoteleft}}{ik soms mijn gedachten heb}{neemt u me dus niet}\\

\haiku{Lieve God, wat zat!}{hij onwrikbaar in de greep}{van de vrolijkheid}\\

\haiku{De gastheer wist nog.}{altijd niet of het te veel}{of te weinig was}\\

\haiku{Zo moest Eva ook zijn,.}{begonnen toen ze wakker}{werd uit het scheppen}\\

\haiku{De temperatuur;}{in de woning was hoger}{dan het bouwsel zelf}\\

\haiku{en dat laatste zou.}{een ongetwijfeld droeve}{ervaring worden}\\

\haiku{Hij loog gitzwart dat - -...}{hij zat te werken en dat}{hij niets te kort kwam}\\

\haiku{{\textquoteleft}Zou uw moeder geen...?}{belangstelling hebben voor}{een nieuwe wagen}\\

\haiku{{\textquoteright} sputterde meneer,.}{Van der Spa krachteloos en}{derhalve bozer}\\

\haiku{V\'o\'orda't'ie zich loeiend,!}{van zonde in de hel gooit}{met al die wijve}\\

\haiku{{\textquoteright} lalde de laffe,.}{schurk en hees zich min of meer}{aan haar stoel overeind}\\

\haiku{En zonder dat hij,!}{het wist deed hij oefening}{no. 4 volmaakt goed}\\

\subsection{Uit: De kant\'elen k\`antelen}

\haiku{alsof er tijd en,.}{ruimte in het gesprek was}{om te antwoorden}\\

\haiku{En wij zullen dan,.}{de tijd nemen tezamen}{iets te gebruiken}\\

\haiku{{\textquoteleft}Anisette voor de...,,.}{dames en brandewijn voor}{de heren Simon}\\

\haiku{{\textquoteright} Waarna de lieve,:}{stralend antwoordde dat dit}{geenszins hinderde}\\

\haiku{Het juffertje, met,:}{een halve rev\'erence}{nam haar glas en zei}\\

\haiku{Hij is nog zo jong -!...}{dan doen wij allemaal wel}{eens dwaze dingen}\\

\haiku{De heer Bertels zond,.}{een brief welke opwindend}{dreigde te zijn}\\

\haiku{, zoals mij door de.}{heer Ter Wamel van Heuvell}{is aangeraden}\\

\haiku{Hij had getoond rijp.}{te zijn voor de veiligheid}{van het huwelijk}\\

\haiku{Beschaafd, zeer juist van,.}{woordkeus en vol van de tact}{die ware adel is}\\

\haiku{Ja, ze voelde zich,.}{rood worden en glimlachte}{tot krimpens toe}\\

\haiku{{\textquoteleft}Zij is gekwetst en,{\textquoteright}.}{verlangt toch naar haar kind dacht}{Emilia van Heuvell}\\

\haiku{{\textquoteleft}Lieve Allaer,{\textquoteright} schreef, {\textquoteleft}?}{zijhoe gaat het toch met Jou}{en Je echtgenoote}\\

\haiku{Bijna geruisloos,.}{slipte zij de kamer uit}{een smalle gang in}\\

\haiku{{\textquoteleft}Als iemand tracht over,!}{te zwemmen of te varen}{dan schiet je subiet}\\

\haiku{{\textquoteright} Eindelijk had ze,.}{doel getroffen ze zag het}{aan de oude ogen}\\

\haiku{Hij had een gevoel,.}{alsof zijn eten eensklaps niet}{meer wilde zakken}\\

\haiku{Uw keuken is puik,,{\textquoteright}.}{lieve jongen voegde hij}{er nadenkend bij}\\

\haiku{Ik bezweer u, dat.}{mijn man en ik niets van uw}{nood hebben bevroed}\\

\haiku{De lopers werden -.}{toen al gelegd alles ging}{met kostbare haast}\\

\haiku{De Herengracht wist,.}{toen dat daar de familie}{Sytz kwam te wonen}\\

\haiku{Ergens moest toch een,?...}{brug van menselijkheid zijn}{in dit mysterie}\\

\haiku{Ach! - kon Betje Sytz,!?}{werkelijk zo menselijk}{zijn en zo geestig}\\

\haiku{Een blonde kerel,,.}{met geweldige schouders}{en ogen als een arend}\\

\haiku{de kleintjes grienden,.}{hem te hard hij zat maar met}{gebalde vuisten}\\

\haiku{Va is al zo lang,' '!}{bij meneer Carel da we}{t motten vieren}\\

\haiku{{\textquoteright} En Rinus knikboog,.}{en struikelde verzaligd}{het kantoortje uit}\\

\haiku{een blijmoedig, stil.}{ventje met een dunne stem}{en een dunne hals}\\

\haiku{Die nacht lag meneer,.}{Mathijs wakker zoals wel}{vaker gebeurde}\\

\haiku{{\textquoteleft}Wat een geluk, dat -!}{j{\'\i}j me dat kunt zeggen \`en}{dat ik het inzie}\\

\haiku{Ze stond op, en zocht.}{tussen de banken door haar}{pad naar het midden}\\

\haiku{De bakkersvrouw vroeg.}{of ze verkering had met}{die zwarte jongen}\\

\haiku{Doch een paar dagen.}{later bracht de postillon}{een brief voor haar mee}\\

\haiku{Nu was de verte,.}{een losse wolk drijvend op}{onbekende wind}\\

\haiku{Met naar kroezig haar,.}{terwijl golvende vlechten}{werden bezongen}\\

\haiku{De zusters huwden,...}{eveneens enkelen met een}{z\'e\'er goede partij}\\

\haiku{een man geweest van,?}{klein karakter zodat hij}{zich liet ompraten}\\

\haiku{{\textquoteright} Waarover Gerrit, nog,.}{altijd te goeder trouw weer}{schaterlachte}\\

\haiku{Ja, het leek of ze,.}{zich strekte en hoger zat}{dan de anderen}\\

\haiku{Maar ja, Graddus zei:}{dus op een dag tegen een}{kwitantieloper}\\

\haiku{Nu begreep hij, hoe.}{gemakkelijk Graddus met}{anders goed omsprong}\\

\haiku{{\textquoteright} zeiden de mensen, {\textquoteleft}!...}{in de kleine stadGraddus}{loopt met een handkar}\\

\haiku{Een enkele man,?!}{vroeg zich luidop af of hij}{soms met vodden liep}\\

\haiku{Nee, beste lezer,,.}{denk nu niet dat Graddus de}{honderdduizend won}\\

\haiku{{\textquoteright} Op dat ogenblik kwam,.}{bedoelde heer naar buiten}{vriendelijk lachend}\\

\haiku{Op de Postweg hield.}{naast hem een reusachtige}{luxe-auto stil}\\

\haiku{{\textquoteleft}Trouwens - wie zegt u, - -...,?...!}{dat ik inderd\'a\'ad niet mal}{doe op zo'n toneel}\\

\haiku{Hij schreef er na haar,.}{weggaan een rapport over voor}{het procesverbaal}\\

\haiku{Was haar prestatie?...}{eventueel w\`erkelijk}{zoveel drukte waard}\\

\haiku{De directeur van:}{de Radiant Dancing Group}{zei met een grinnik}\\

\haiku{Daarna begonnen -.}{haar armen te bewegen}{het lichaam trilde}\\

\haiku{De armen waren.}{onwerkelijk blank in hun}{snelle gebaren}\\

\haiku{Nog wist het publiek,.}{niet of deze val  ook}{opzet was geweest}\\

\haiku{{\textquoteleft}Hoe weet u nou, dat -?}{die dame naar de bank gaat}{om geld te halen}\\

\haiku{Het stemde mevrouw.}{Thijmens op een prettige}{manier weemoedig}\\

\haiku{{\textquoteleft}Ik moet daar dienst doen,.}{en heb geholpen bij het}{onderzoek en zo}\\

\haiku{{\textquoteleft}Goed, en degene,,,.}{die haar heeft vermoord wist dat}{ze geld in huis had}\\

\haiku{Maar bij mij heeft hij...{\textquoteright}}{nooit het voordeurslot hoeven}{te repareren}\\

\subsection{Uit: Het klooster van de lichtgroene paters}

\haiku{De bomen weken.}{en toonden een veld van even}{hoge afmeting}\\

\haiku{Het jonge paar werd;}{een middelbaar paar met een}{zoon en een dochter}\\

\haiku{Het geluk van de.}{voorouders moet hem door het}{hoofd hebben gespeeld}\\

\haiku{Een drankje om nooit.}{te vergeten en nimmer}{weer te nemen}\\

\haiku{Tommy was door een.}{kwaadaardig stuk rolpens de}{kamer opgezegd}\\

\haiku{God, zeg niet zulke,,}{onzin ik zou immers nog}{even blijven leven}\\

\haiku{We zijn toch niet dol,!}{dat we teruggaan naar die}{suffe kamer-troep}\\

\haiku{{\textquoteleft}Welke opdracht heeft?...{\textquoteright}.}{die man van wie gehad Hij}{was alweer terug}\\

\haiku{{\textquoteright} vroeg de meneer, nu.}{toch definitief over z'n}{hele ziel bevreemd}\\

\haiku{{\textquoteleft}Maar u voelde zich....}{unaniem meer aangetrokken}{tot dit matte groen}\\

\haiku{Tommy richtte een.}{paar schroeiende ogen naar de}{vorige spreker}\\

\haiku{Daar werd het gelach,.}{zo veelvuldig dat niemand}{meer verstaanbaar was}\\

\haiku{{\textquoteleft}Of iemand bemint,,}{hem en heeft Hans een geschenk}{willen aanbieden}\\

\haiku{Maar ze toeterde:}{bliksemscherp uit een hoek van}{haar verknepen mond}\\

\haiku{God helpe je, als!}{je in de dorpsgemeenschap}{wordt opgenomen}\\

\haiku{{\textquoteright} vroeg Tommy, en hief '.}{een rood gezicht op metn}{paar zielige ogen}\\

\haiku{Gijs stond stil bij een.}{gemakkelijke crapaud}{van bleek rood fluweel}\\

\haiku{Gijbertje was een.}{mooi meidje met felrood haar}{en donkere ogen}\\

\haiku{Dat haar haren een.}{gouden licht afstraalden in}{de luikende avond}\\

\haiku{De woordkeus van de.}{verteller was naief en}{ongecompliceerd}\\

\haiku{Nee, meneer Joan.}{greep in z'n  zak en gaf}{de boer een gulden}\\

\haiku{Meneer Joan werd,.}{een wildebras die geen wijf}{met rust kon laten}\\

\haiku{Ach, het werd een slaatje;}{van zure komkommer met}{uitjes en sardientjes}\\

\haiku{{\textquoteleft}Daar moet dan rode,{\textquoteright}.}{wijn bij worden geserveerd}{bestelde Tommy}\\

\haiku{Sommige mensen.}{ondergaan het leven zo}{bizonder simpel}\\

\haiku{Ze trokken hem aan,,}{slappe armen overeind maar}{hij boog overal uit}\\

\haiku{Die dure stoelen,{\textquoteright}.}{zijn van Kareltje kefte}{Tommy waarschuwend}\\

\haiku{Ze renden de deur,.}{uit en stoven weg in de}{wagens rechts en links}\\

\haiku{Het liep afgetobd,;}{en zonder enige moed het}{liet de kop hangen}\\

\haiku{Terwijl hij hen toch.}{ietwat had gesticht met zijn}{pianogetingel}\\

\haiku{{\textquoteleft}Het is jammer, dat,{\textquoteright}.}{ze zo van ons schrok klaagde}{Rogier zoetsappig}\\

\haiku{{\textquoteleft}En as julle geen,?}{poaters ben woarom draoge}{julle dan rokke}\\

\haiku{{\textquoteleft}Niks nie,{\textquoteright} weerde ze, '.}{met toch nogn losse moer}{in het lachcentrum}\\

\haiku{En kijk, al praatten -!}{die kerels nou volslagen}{zot w\`erk kreeg ze}\\

\haiku{{\textquoteleft}Maar dit is dan ook,,?}{de laatste kip die hier wordt}{geintroduceerd niet}\\

\haiku{Miquel was overeind.}{en stak zijn hoofd om de hoek}{van de kamerdeur}\\

\haiku{Rinus zette het.}{laatste theekopje neer en}{applaudisseerde}\\

\haiku{...{\textquoteright} Ukje hijgde en.}{blafte kleine geluidjes}{langs de buitendeur}\\

\haiku{want wie garandeert?}{dat niet ook de liefdoener}{je leven begeert}\\

\haiku{dat ik de stam die - -,?}{avond fuifde op eh zeg wat}{{\`\i}s dit voor gerecht}\\

\haiku{Nou ja, dat wisten,.}{ze wel maar als hij gepest}{werd was dat logisch}\\

\haiku{{\textquoteright} Miquel grinnikte.}{vaak en knikte terwijl hij}{z'n blik afwendde}\\

\haiku{{\textquoteleft}Nee, aan mijn servet,{\textquoteright}.}{kleeft geen lippenrood voegde}{hij er peinzend bij}\\

\haiku{Toen de wagen was,.}{weggereden liep Miquel}{langzaam naar boven}\\

\haiku{met visioenen.}{of God tegen hem met de}{vingers knipte}\\

\haiku{Die smoking is ook.}{echt niet het goedkoopste stuk}{van de uitverkoop}\\

\haiku{Zijn handen bleven,.}{langs z'n lichaam toen hij boog}{en zich afwendde}\\

\haiku{Rinus praatte half.}{grinnikend gedempt en zweeg}{opeens als verrast}\\

\haiku{In de hal vroeg hij.}{het telefoonboek en keek}{er een beetje in}\\

\haiku{{\textquoteright} Miquel haalde de.}{schouders op en glimlachte}{half naar Angelo}\\

\haiku{Tommy keek van de.}{een naar de ander en trok}{de wenkbrauwen op}\\

\haiku{Hij bukte zich toch.}{en begon de zaken in}{het tasje te doen}\\

\haiku{Ze maakten een kooi.}{van een flinke lap gaas uit}{de Welkomwinkel}\\

\haiku{{\textquoteright} Maar Angelo stond,:}{bewegingloos naar de maan}{te kijken en zei}\\

\haiku{Op kantoor was een,,{\textquoteright}.}{vent waar ik al lang de pest}{aan had ging Hans door}\\

\haiku{Hij zat altijd te,.}{lachen en hij had {\'\i}\'ets wat}{me irriteerde}\\

\haiku{vooral Rogier en.}{Tommy hadden zoiets nog}{onlangs meegemaakt}\\

\haiku{{\textquoteright} {\textquoteleft}Nee,{\textquoteright} gaf Hans toe, {\textquoteleft}maar...}{iedereen kon begrijpen}{wat hij bedoelde}\\

\haiku{{\textquoteleft}Hans,{\textquoteright} bitste hij, {\textquoteleft}als je,!}{nog \'e\'en woord zegt sla ik je}{persoonlijk tot moes}\\

\haiku{{\textquoteleft}Mijn meisje heeft twee,,}{jaar ziek gelegen en van}{de winter stierf ze}\\

\haiku{{\textquoteright} {\textquoteleft}Dat is al maanden,{\textquoteright},.}{geleden zei Tommy half}{tegen Angelo}\\

\haiku{{\textquoteright} Angelo begreep.}{dat hij ook niet bij hem had}{kunnen aankloppen}\\

\haiku{Dwing me niet, strenger,.}{maatregelen te nemen}{want ik sta voor niets}\\

\haiku{De veldwachter liet.}{zich op de knie\"en zinken}{in het autolicht}\\

\haiku{In de stad was een.}{huis van hem vrij gekomen}{door een sterfgeval}\\

\subsection{Uit: Koninklijke omnibus}

\haiku{wat had de jongen,!}{zijn verlegen trotse hart}{toen voelen krimpen}\\

\haiku{Te ver naar voren,.}{zodat hij telkens tegen}{de treden schopte}\\

\haiku{In de bovengang,,:}{langs holle stapgeluiden}{zei de jonge vorst}\\

\haiku{Terwijl hij sprak, zag.}{hij in de ogen van Dov\`ec}{schrik en verrassing}\\

\haiku{Doch hij speelde met,.}{overgave mee en voelde}{zich heel gelukkig}\\

\haiku{{\textquoteleft}Als ik de eerste,?}{keer al niet de tijd neem wat}{moet ik dan later}\\

\haiku{{\textquoteleft}Hij ontdoet zich van.}{zijn tooi en strekt zich uit op}{zijn leger en rust}\\

\haiku{{\textquoteright} Hij vlijde handig.}{een brede hermelijnen}{kraag om de schouders}\\

\haiku{{\textquoteright} Langs het vorstelijk,:}{oor fluisterde Kudja aan}{de kraag frunnikend}\\

\haiku{Hij klopte Adalbert.}{op de schouder en wees naar}{de prachtige kroon}\\

\haiku{{\textquoteleft}Je bent helemaal,.}{geen familie van me en}{je opa ken ik niet}\\

\haiku{Ze maken rovers...{\textquoteright}.}{van ons Adalbert keek hem met}{brandende ogen aan}\\

\haiku{{\textquoteleft}Alleredelste,{\textquoteright}, {\textquoteleft} - -!}{hakkelde hiju kunt zich}{hier niet uit praten}\\

\haiku{{\textquoteleft}U schijnt de ernst van,.}{het gebeurde niet te zien}{Alleredelste}\\

\haiku{Hij liet zich languit.}{vallen en keek tussen het}{hoge bloeisel door}\\

\haiku{Zijdelings stond op.}{het veld een jongen van iets}{oudere leeftijd}\\

\haiku{De jongen wendde.}{zich geheel naar hem om en}{bekeek hem opnieuw}\\

\haiku{{\textquoteright} {\textquoteleft}Precies,{\textquoteright} zei Karalj,.}{en trok z'n benen onder}{zich om op te staan}\\

\haiku{Toen hij de jongens,.}{met de bloemen zag kreeg hij}{een schok door z'n lijf}\\

\haiku{Dov\`ec zette zijn,:}{glas voorzichtig neer en zei}{zacht maar zeer stellig}\\

\haiku{Het moet nu vier jaar,...}{geleden zijn hij was dus}{ongeveer zestien}\\

\haiku{{\textquoteleft}Ze zijn op raad van - -{\textquoteright} {\textquoteleft}!}{de koetsier uitgestegen}{bij het moerasN\'e\'e}\\

\haiku{Maar u sluit zich van,,.}{mij af met uw cynisme}{en dat is jammer}\\

\haiku{De beste raadsman.}{leek Adalbert altijd nog het}{vriendje uit zijn jeugd}\\

\haiku{Hij lachte en sprong,.}{over de stenen heen van de}{weg af het veld in}\\

\haiku{Altijd was er om.}{Dj\'ura Gw\'ano een sfeer van haat}{en onrust geweest}\\

\haiku{Pas toen Gw\'ano de,.}{zaal had verlaten barstte}{er een tumult los}\\

\haiku{{\textquoteleft}Die is nou blij, maar '.}{over een poosje heeftie weer}{iets anders nodig}\\

\haiku{Maar die avond kwamen,:}{Dov\`ec en Bartenstein bij}{Adalbert en zeiden}\\

\haiku{- ze hadden het als.}{goede soldaten stellig}{zelf kunnen vinden}\\

\haiku{Onverwacht sprak de -.}{priester zijn stem viel als een}{galm in het zwijgen}\\

\haiku{Die jongen stond weer,.}{net zo star als hij in het}{begin had gedaan}\\

\haiku{Maar er was iets in,.}{het snuiven van het dier dat}{hem waakzaam maakte}\\

\haiku{{\textquoteleft}Als ik gewoon in,!}{Jarb\'ogic was gebleven}{zou hij nog leven}\\

\haiku{de tafel kraakte.}{met enorm geraas onder hun}{handen in elkaar}\\

\haiku{{\textquoteleft}Maar als je 's nachts,.}{kouwe voeten hebt laat ze}{je mooi bibberen}\\

\haiku{Hij richtte het hoofd}{een beetje waanwijs op en}{keek uit het venster}\\

\haiku{{\textquoteleft}En ik acht u een -.}{betrouwbaar man en die zijn}{schaars in de wereld}\\

\haiku{Hij deed een greep in.}{zijn zak en bood haar plat op}{zijn hand de solid}\\

\haiku{{\textquoteleft}Er zijn altijd zo,{\textquoteright}.}{veel vraagstukken tegelijk}{zei hij voorzichtig}\\

\haiku{En het drong pas die,.}{avond laat tot hem door wat ze}{kon hebben bedoeld}\\

\haiku{Dat heeft mij na het.}{gesprek met mevrouw Kurgic}{een beetje bevreemd}\\

\haiku{en dit standpunt had.}{nog nimmer enige Vorst zo}{simplistisch verwoord}\\

\haiku{Dit waren woorden.}{die Adalbert een prik in zijn}{hersenpan gaven}\\

\haiku{Ik geef het je niet,.}{in handen want men zou naar}{ons kunnen kijken}\\

\haiku{Dat is veel erger.}{dan boze koningen en}{zure prinsessen}\\

\haiku{{\textquoteleft}Wie beseft er, wat,?}{het wil zeggen een kroon op}{je hoofd te krijgen}\\

\haiku{Niet alleen had ik,...{\textquoteright}}{zo veel begrip niet verwacht}{zo ver van mijn land}\\

\haiku{U bent reeds te lang,...}{opstandig geweest om te}{kunnen vertrouwen}\\

\haiku{Een gans apparaat;}{van ceremoniekenners}{kwam in beweging}\\

\haiku{Hij heeft er toch goed,,}{aan gedaan al die wijven}{eruit te trappen}\\

\haiku{Om drie minuten.}{over half elf remde de trein}{piepend en hijgend}\\

\haiku{Dat moet u niet doen,{\textquoteright}, {\textquoteleft}.}{verzocht Adalbert haastigwij}{zijn incognito}\\

\haiku{van het onderhoud.}{met Gw\'ano had hij alleen}{vermoedens gehad}\\

\haiku{Volgens Adalbert moest,.}{er arnika op en dat}{deed Karalj dus}\\

\haiku{Baron Bartenstein.}{hoorde dit alles met een}{brede glimlach aan}\\

\haiku{{\textquotedblleft}Wij verlenen het -{\textquotedblright} {\textquotedblleft}}{agr\'ement U zoudt eventueel}{kunnen doen schrijven}\\

\haiku{{\textquoteright} De baron boog en.}{liet zich uitgeleide doen}{door de knecht van dienst}\\

\haiku{Je bent je eigen -{\textquoteright} {\textquoteleft},{\textquoteright}.}{L\'a\'at me nou zei Adalbert met}{een wit schuimgezicht}\\

\haiku{{\textquoteright} en ontsloot schielijk,:}{de stroom van woorden welke}{hij had klaarliggen}\\

\haiku{Toen ontmoette zijn -.}{blik die van de jongeman}{en Bartenstein zweeg}\\

\haiku{het water spatte.}{als twee dunne vlerken langs}{de smalle boeg op}\\

\haiku{Het was wonderlijk -.}{alsof hij een jong diertje}{in bescherming nam}\\

\haiku{Drie van de paarden.}{knikten met hun hoofd terwijl}{het vierde brieste}\\

\haiku{De laatste deur was.}{van binnen afsluitbaar en}{aldus deden zij}\\

\haiku{Zij gingen speurend,;}{nu langs alle laden en}{kasten en kisten}\\

\haiku{Maar nu blonk achter,.}{elke gedachte die ene}{heel lieve glimlach}\\

\haiku{{\textquoteleft}En d\`an moeten we,,{\textquoteright}.}{uitvissen waar die baron}{woont vulde hij aan}\\

\haiku{Terwijl de Fijne,:}{het netnummer draaide van}{Grifsdorp sprak de Boom}\\

\haiku{Hij vingerde aan,;}{een microfoon die gillen}{uitzond als een big}\\

\haiku{{\textquoteleft}Verknepen knoedels,,!}{drekdoerakken rotsmerige}{stinkstommelingen}\\

\haiku{maar in dienst had hij.}{het niet verder geschopt dan}{rekruut en cachot}\\

\haiku{n theemuts op je,{\textquoteright},.}{hoofd antwoordde de Fijne}{wat niet aardig was}\\

\haiku{{\textquoteleft}Als de zaak morgen,,{\textquoteright};}{mislukt is het jouw schuld zei}{hij gemakkelijk}\\

\haiku{{\textquoteright} Hier moest de Fijne -.}{toch om glimlachen wellicht}{vleide hem het beeld}\\

\haiku{Pingel deed zijn jack -.}{uit en een sjiek colbert aan}{de broek droeg hij reeds}\\

\haiku{{\textquoteleft}Ik wou toch \`erg graag,{\textquoteright}.}{effe de vlag probere}{zei Jules bedeesd}\\

\haiku{{\textquoteleft}Ik mot toch wete,,....}{hoe of'tie beweegt als ik}{mot saluere}\\

\haiku{{\textquoteright} {\textquoteleft}Dora wordt ervoor,{\textquoteright}.}{betaald wees de hertog haar}{hooghartig terecht}\\

\haiku{En waarom meteen!}{van die onbeheerste grote}{trossen van alles}\\

\haiku{En terwijl ik wil,....}{doorlope zie ik dat de}{pin d'r half uitsteekt}\\

\haiku{De post is geweest,.}{en die smeris komt over zes}{minuten terug}\\

\haiku{{\textquoteleft}Ach,{\textquoteright} sprak de eerste, {\textquoteleft},....}{blas\'edat is een grapje}{van Jules Thera}\\

\haiku{{\textquoteright} {\textquoteleft}Nee, merci,{\textquoteright} ontkwam, {\textquoteleft}....}{de ouderedat draag ik}{altijd zelf bij me}\\

\haiku{Wat deed ze ook met....}{zo'n onzinnig accent vlak}{onder haar steekneus}\\

\haiku{De koffie stond haar,.}{aan de boorden van haar ziel}{dat was duidelijk}\\

\haiku{De Fijne leek wat -,.}{te horen hij boog zich uit}{zijn bank en riep iets}\\

\haiku{Hij gebaarde naar -.}{de schommelbank zij konden}{samen gaan zitten}\\

\haiku{{\textquoteleft}Dat beleef ik maar,!...{\textquoteright}}{\'e\'enmaal en ik heb het}{totaal niet verwacht}\\

\haiku{Zeg dat me moeder,,!...}{ziek is dat ik met jullie}{mee moet bed\`enk iets}\\

\haiku{Hij ging dus open doen,,.}{en was totaal niet verbaasd}{politie te zien}\\

\haiku{Hij hoorde Dora,,.}{beneden achter een deur}{jankerig praten}\\

\haiku{De complicatie.}{had hem een rake klap op}{het hoofd gegeven}\\

\haiku{Het kwam niet te pas,.}{opeens in je eentje de}{tuin in te lopen}\\

\haiku{hij, de Fijne, was,!....}{zonder het zelf te willen}{uit zijn huid geglipt}\\

\haiku{Hij voelde edele.}{tranen prikken achter zijn}{schurke-oogleden}\\

\haiku{Waarom hebben ze,?}{dat nog niet uitgevonden}{en die atoombom wel}\\

\haiku{Zijn sterke kluiven '.}{wendden de hertog alsof}{hij aant spit hing}\\

\haiku{U kunt het er met -.}{water en zeep afhalen}{aan u de keuze}\\

\haiku{De politie knijpt!...}{me in alle uiteinden}{van m'n zenuwen}\\

\haiku{En bijna nog v\'o\'or,:}{zij was uitgesproken bood}{de hofdame aan}\\

\haiku{Vergeet niet, dat we,.}{niet weten w\`at er in die}{juwelenkist zit}\\

\haiku{D'r is d'r n\`og een,,{\textquoteright}.}{die koffie mot hebbe zei}{de Boom dromerig}\\

\haiku{Amad\'e trok zijn das recht.}{en de baron zat weer te}{vingeroefenen}\\

\haiku{Jij hebt nog geen deur,{\textquoteright}.}{opengedaan beschuldigde}{Pingel korzelig}\\

\haiku{Het was een oude,.}{heer die half buigend een hoed}{van zijn hoofd plukte}\\

\haiku{Nou, dag mevrouw, dag,!}{schat magikwelzeggen}{dag lekkere kluit}\\

\haiku{- {\textquoteleft}ik heb panne, ziet,.}{u en ik m\'o\'et een luchtpomp}{hebben en een schaar}\\

\haiku{De barones sloot,.}{de deur goed en leunde even}{tegen het paneel}\\

\haiku{{\textquoteright} kreet ze, en keek toch,.}{weer even om naar de tijger}{die nog steeds niet kwam}\\

\haiku{{\textquoteleft}Ja, maar we kunnen,!...{\textquoteright}}{natuurlijk ook wel iets van}{jou verwachten Door}\\

\haiku{{\textquotedblright} toen hadden een paar....}{dames en heren ze bont}{en blauw geslagen}\\

\haiku{Ze bracht enkele.}{haren ter plaatse waar zij}{hoorden in de taart}\\

\haiku{En hoeveel hadden,?}{jullie gedacht Dora dan}{later te geven}\\

\haiku{{\textquoteleft}Ik begrijp het niet,{\textquoteright}.}{meer sliste hij nederig}{en teleurgesteld}\\

\haiku{En die hofdame.}{vertrouwde ze nog minder}{dan Roje Gerrit}\\

\haiku{Gelukkig had de -.}{inspecteur hem gezien hij}{kon niet ver komen}\\

\haiku{{\textquoteleft}Ik geloof dat mijn,{\textquoteright}.}{ring in de terrine is}{gevallen zei ze}\\

\haiku{De prinses verloor,:}{haar voorname kalmte niet}{en zei vriendelijk}\\

\haiku{De Fijne richtte.}{zijn ogen in een dringende}{smeekbede naar haar}\\

\haiku{Pingel had zich op.}{een knie laten vallen en}{greep Fijne's benen}\\

\haiku{Ze had geen oog van -.}{hem af kunnen houden hij}{zag d'ruit als vijftien}\\

\haiku{De baron veegde,.}{over zijn voorhoofd dat vochtig}{was van de warmte}\\

\haiku{Een grote tafel,.}{stond in het midden omringd}{van hoge stoelen}\\

\haiku{Het was vreselijk -.}{prinses Eline de la Tour}{Olmberg schreide}\\

\haiku{Hij poetste over zijn.}{gezicht met gesloten ogen}{en haalde diep adem}\\

\haiku{{\textquoteleft}Ik heb nauwkeurig.}{de portretten van prinses}{Eline bekeken}\\

\haiku{Hij trok het bankje.}{onder zijn zitspieren en}{opeens speelde hij}\\

\haiku{{\textquoteleft}En dan gaan we met,.}{mekaar naar een verdomd goed}{eethuisje vanavond}\\

\haiku{Toen begon hij te,;}{lopen in zijn tennispak}{veerkrachtig en snel}\\

\haiku{Nou geef ik een schop,!}{tegen de tafel zodat}{de theepot kantelt}\\

\haiku{Waren er dan toch?...}{schepselen die over haar en}{haar vader praatten}\\

\haiku{Tussen al deze:}{individuutjes was Jaapje}{nooit opgevallen}\\

\haiku{Hij legde zijn arm.}{om veel vrouwelijke en}{manlijke schouders}\\

\haiku{Tante Sanna had.}{de resten van haar glimlach}{teruggevonden}\\

\haiku{{\textquoteleft}Je kunt altijd bij,{\textquoteright}.}{ons aankloppen om goede}{raad vulde ze aan}\\

\haiku{Alsof iemand haar.}{werkelijk au s\'erieux}{had genomen}\\

\haiku{{\textquoteleft}God hebbe zijn ziel,{\textquoteright}.}{en stemme hem gelukkig}{antwoordde Agniet}\\

\haiku{zo'n bezonken stem,!}{als ze aan het werk was met}{een behandeling}\\

\haiku{zij waren over het.}{nichtje dat eensklaps alleen}{was komen te staan}\\

\haiku{Dat maatschappelijk:}{werk had haar op andere}{gedachten gebracht}\\

\haiku{Zo  wakker en.}{actief was ze de hele}{dag nog niet geweest}\\

\haiku{hij had nog kuiltjes,!..}{in zijn wangen ook en wat}{een prachtige mond}\\

\haiku{Iedereen spreekt over.}{geld en niemand herkent de}{waarde van vriendschap}\\

\haiku{zij eten gevaarlijk.}{voedsel dat hun karakter}{en geest ondermijnt}\\

\haiku{Het was alsof ze.}{in een schitterend belicht}{panorama keek}\\

\haiku{Maar haar hoofd begon,:}{te gloeien en om een hoek}{dacht ze in paniek}\\

\haiku{{\textquoteleft}een schaakspeler die.}{geen enkele partij ten}{einde wil spelen}\\

\haiku{En Agniet, op geen,.}{geluid voorbereid voelde}{dat ze een kleur kreeg}\\

\haiku{Het ding hing daar al -,.}{minstens tien jaar ze wist niet}{waar het vandaan kwam}\\

\haiku{Dat wist ze op dit,.}{moment aan de scherpte van}{haar teleurstelling}\\

\haiku{{\textquoteright} Maar hij greep, zacht en,.}{ferm haar enkel en nam het}{schoentje van haar voet}\\

\haiku{- -{\textquoteright} Met verbijstering,.}{bemerkte Agniet dat ze}{uit haar japon gleed}\\

\haiku{En dat, besefte,:}{ze was de heiligheid van}{het wedervaren}\\

\haiku{Loom sloot ze later,.}{de voordeur af liep de trap}{op naar haar kamer}\\

\haiku{- ~ Ze bedacht, dat,,.}{ze eigenlijk niet wist welk}{werk Idris deed en waar}\\

\haiku{Ze lag weer in het.}{duister en de kille hand}{knelde haar hart dicht}\\

\haiku{{\textquoteleft}Ik heb daarbij hard -...,}{moeten werken ik heb een}{examen afgelegd}\\

\haiku{Ach, dat was ook niet,.}{erg geweest want ze had veel}{van hem gehouden}\\

\haiku{Waarom?{\textquoteright} {\textquoteleft}Omdat jouw,{\textquoteright}.}{land te ver van het mijne}{ligt zei Agniet zacht}\\

\haiku{Van Idris hoorde zij,.}{in deze dagen niets en}{dat begreep ze wel}\\

\haiku{Mijn man heeft zijn naam.}{genoteerd en naar deze}{heer ge{\"\i}nformeerd}\\

\haiku{Wij hebben thans de.}{naam opgekregen van een}{keurige dame}\\

\haiku{Er was weinig strijd.}{geweest bij de uitroeping}{van de nieuwe vorst}\\

\haiku{Ze moest geen enkel,.}{bijvoeglijk naamwoord overslaan}{als het Idris betrof}\\

\haiku{{\textquoteright} Agniet voelde zich.}{als een geigerteller die}{positief ontmoet}\\

\haiku{de expressie van,.}{Agniets gelaat in dat het}{hartverwarmend was}\\

\haiku{{\textquoteleft}Zou ons land daar een, - -?}{ambassade hebben of}{eh een legatie}\\

\haiku{Stel, dat Idris w\`el met, -....}{die Emier heeft gesproken maar}{inderdaad niet niet}\\

\haiku{{\textquoteleft}Al neem ik niet aan,,.}{dat hij nog een harem heeft}{hedentendage}\\

\haiku{Het was lang niet zo,.}{heerlijk als ze had gedacht}{om te vertellen}\\

\haiku{{\textquoteleft}Bleekblauw, zilver, heel,.}{blank goud in de tint van je}{haar met iets van zwart}\\

\haiku{Geen grote kunst, geen,.}{olie geen antiquiteiten}{of oude steden}\\

\haiku{{\textquoteleft}Ik verbeeld het me,,{\textquoteright}.}{net als de meeste mensen}{antwoordde Agniet}\\

\haiku{{\textquoteleft}Ik ken iemand uit,.}{zijn familie die zal me}{introduceren}\\

\haiku{Agniet volgde een,.}{donker meisje naar een hal}{waar ze moest wachten}\\

\haiku{Anderen bogen.}{in het stof en brachten hun}{hoofden ter aarde}\\

\haiku{De mensen keken.}{allemaal alsof ze een}{revuenummer was}\\

\haiku{Ze zou natuurlijk.}{worden ondergebracht in}{het vrouwen verblijf}\\

\haiku{een zeer donker, slank.}{meisje met ravenzwart haar}{in een zware wrong}\\

\haiku{Ze vroeg zich vaag af,.}{wat haar programma voor die}{avond zou mogen zijn}\\

\haiku{Zij besefte goed,,.}{het voedsel te eten wat Idris}{dus gewend moest zijn}\\

\haiku{{\textquoteleft}Maar Allah weet, wat -.}{er gaat gebeuren de Emier}{moet het afwachten}\\

\haiku{De bevestiging.}{van haar vermoeden liet niet}{lang op zich wachten}\\

\haiku{- Toen besefte ze,.}{dat ze met het boek voor de}{Emier in haar hand zat}\\

\haiku{Zijn ogen richtten zich -.}{weer naar haar en daar had ze}{geen weerstand tegen}\\

\haiku{De orchidee - of,,.}{welke bloem ook zit niet aan}{de buitenkant Agggniet}\\

\haiku{Hij trok haar tegen,.}{zich aan onweerstaanbaar van}{omzichtige kracht}\\

\haiku{{\textquoteright} en wachtend op de,:}{traktatie voegde hij zacht}{aan zijn verhaal toe}\\

\haiku{Een wachter buiten.}{een van de voorhangen gaf}{hij een kort bevel}\\

\haiku{Uit de verte had -}{nog onbekommerd gelach}{geklonken v\'o\'or hen}\\

\haiku{Die poincettia in -?...}{Holland was zo roodbladig}{geweest wat w{\`\i}st ze}\\

\haiku{Er trok een fijne.}{rimpeling van glimlach langs}{Djamilas kaken}\\

\haiku{En toen begon een:}{zeer naarstig onderricht in}{allerlei woorden}\\

\haiku{De hoofddoek welke:}{buiten werd gedragen met}{een rolband van koord}\\

\haiku{{\textquoteleft}Dit is koorts - ik ben,.}{besmet met iets waarvoor geen}{inenting bestond}\\

\haiku{Terwijl ze van de,}{spiegel weg naar buiten keek}{zag ze door de tuin}\\

\haiku{{\textquoteright} Djamila wendde.}{haar ogen naar haar en schudde}{vriendelijk het hoofd}\\

\haiku{{\textquoteright} Daarmee ontnam hij.}{aan het samenzijn een al}{te formele toon}\\

\haiku{bladen vol bekers,.}{binnen die voor de gasten}{werden neergezet}\\

\haiku{Terwijl ze naar hem,:}{luisterde zag Agniet de}{hand van de dienaar}\\

\haiku{Zijn lichaam boog als.}{een brug hol \`opkrommend in}{gruwelijk lijden}\\

\haiku{Ze schaamde zich en.}{meende liefde en geluk}{te hebben verspeeld}\\

\haiku{Agniet vroeg zich af, -.}{waar de dode dienaar was}{gebracht en zijn vrouw}\\

\subsection{Uit: Kroelen met de kroon}

\haiku{Nou geef ik een schop,!}{tegen de tafel zodat}{de theepot kantelt}\\

\haiku{Waren er dan toch?...}{schepselen die over haar en}{haar vader praatten}\\

\haiku{Tussen al deze:}{individuutjes was Jaapje}{nooit opgevallen}\\

\haiku{Hij legde zijn arm.}{om veel vrouwelijke en}{manlijke schouders}\\

\haiku{Tante Sanna had.}{de resten van haar glimlach}{teruggevonden}\\

\haiku{{\textquoteleft}Je kunt altijd bij,{\textquoteright}.}{ons aankloppen om goede}{raad vulde ze aan}\\

\haiku{Alsof iemand haar.}{werkelijk au s\'erieux}{had genomen}\\

\haiku{'God hebbe zijn ziel,{\textquoteright}.}{en stemme hem gelukkig}{antwoordde Agniet}\\

\haiku{zo'n bezonken stem,!}{als ze aan het werk was met}{een behandeling}\\

\haiku{zij waren over het.}{nichtje dat eensklaps alleen}{was komen te staan}\\

\haiku{Dat maatschappelijk:}{werk had haar op andere}{gedachten gebracht}\\

\haiku{Zo  wakker en.}{actief was ze de hele}{dag nog niet geweest}\\

\haiku{hij had nog kuiltjes,!..}{in zijn wangen ook en wat}{een prachtige mond}\\

\haiku{Iedereen spreekt over.}{geld en niemand herkent de}{waarde van vriendschap}\\

\haiku{zij eten gevaarlijk.}{voedsel dat hun karakter}{en geest ondermijnt}\\

\haiku{Het was alsof ze.}{in een schitterend belicht}{panorama keek}\\

\haiku{Maar haar hoofd begon,:}{te gloeien en om een hoek}{dacht ze in paniek}\\

\haiku{{\textquoteleft}een schaakspeler die.}{geen enkele partij ten}{einde wil spelen}\\

\haiku{En Agniet, op geen,.}{geluid voorbereid voelde}{dat ze een kleur kreeg}\\

\haiku{Het ding hing daar al -,.}{minstens tien jaar ze wist niet}{waar het vandaan kwam}\\

\haiku{Dat wist ze op dit,.}{moment aan de scherpte van}{haar teleurstelling}\\

\haiku{{\textquoteright} Maar hij greep, zacht en,.}{ferm haar enkel en nam het}{schoentje van haar voet}\\

\haiku{- -{\textquoteright} Met verbijstering,.}{bemerkte Agniet dat ze}{uit haar japon gleed}\\

\haiku{En dat, besefte,:}{ze was de heiligheid van}{het wedervaren}\\

\haiku{Loom sloot ze later,.}{de voordeur af liep de trap}{op naar haar kamer}\\

\haiku{- ~ Ze bedacht, dat,,.}{ze eigenlijk niet wist welk}{werk Idris deed en waar}\\

\haiku{Ze lag weer in het.}{duister en de kille hand}{knelde haar hart dicht}\\

\haiku{{\textquoteleft}Ik heb daarbij hard -...,}{moeten werken ik heb een}{examen afgelegd}\\

\haiku{Ach, dat was ook niet,.}{erg geweest want ze had veel}{van hem gehouden}\\

\haiku{Waarom?{\textquoteright} {\textquoteleft}Omdat jouw,{\textquoteright}.}{land te ver van het mijne}{ligt zei Agniet zacht}\\

\haiku{Van Idris hoorde zij,.}{in deze dagen niets en}{dat begreep ze wel}\\

\haiku{Mijn man heeft zijn naam.}{genoteerd en naar deze}{heer ge{\"\i}nformeerd}\\

\haiku{Wij hebben thans de.}{naam opgekregen van een}{keurige dame}\\

\haiku{Er was weinig strijd.}{geweest bij de uitroeping}{van de nieuwe vorst}\\

\haiku{Ze moest geen enkel,.}{bijvoeglijk naamwoord overslaan}{als het Idris betrof}\\

\haiku{{\textquoteright} Agniet voelde zich.}{als een geigerteller die}{positief ontmoet}\\

\haiku{de expressie van,.}{Agniets gelaat in dat het}{hartverwarmend was}\\

\haiku{{\textquoteleft}Zou ons land daar een, - -?}{ambassade hebben of}{eh een legatie}\\

\haiku{Stel, dat Idris w\`el met, -....}{die Emier heeft gesproken maar}{inderdaad niet niet}\\

\haiku{{\textquoteleft}Al neem ik niet aan,,.}{dat hij nog een harem heeft}{hedentendage}\\

\haiku{Het was lang niet zo,.}{heerlijk als ze had gedacht}{om te vertellen}\\

\haiku{{\textquoteleft}Bleekblauw, zilver, heel,.}{blank goud in de tint van je}{haar met iets van zwart}\\

\haiku{Geen grote kunst, geen,.}{olie geen antiquiteiten}{of oude steden}\\

\haiku{{\textquoteleft}Ik verbeeld het me,,{\textquoteright}.}{net als de meeste mensen}{antwoordde Agniet}\\

\haiku{{\textquoteleft}Ik ken iemand uit,.}{zijn familie die zal me}{introduceren}\\

\haiku{Agniet volgde een,.}{donker meisje naar een hal}{waar ze moest wachten}\\

\haiku{Anderen bogen.}{in het stof en brachten hun}{hoofden ter aarde}\\

\haiku{De mensen keken.}{allemaal alsof ze een}{revuenummer was}\\

\haiku{Ze zou natuurlijk.}{worden ondergebracht in}{het vrouwenverblijf}\\

\haiku{een zeer donker, slank.}{meisje met ravenzwart haar}{in een zware wrong}\\

\haiku{Ze vroeg zich vaag af,.}{wat haar programma voor die}{avond zou mogen zijn}\\

\haiku{Zij besefte goed,,.}{het voedsel te eten wat Idris}{dus gewend moest zijn}\\

\haiku{{\textquoteleft}Maar Allah weet, wat -.}{er gaat gebeuren de Emier}{moet het afwachten}\\

\haiku{De bevestiging.}{van haar vermoeden liet niet}{lang op zich wachten}\\

\haiku{- Toen besefte ze,.}{dat ze met het boek voor de}{Emier in haar hand zat}\\

\haiku{Zijn ogen richtten zich -.}{weer naar haar en daar had ze}{geen weerstand tegen}\\

\haiku{De orchidee - of,,.}{welke bloem ook zit niet aan}{de buitenkant Agggniet}\\

\haiku{Hij trok haar tegen,.}{zich aan onweerstaanbaar van}{omzichtige kracht}\\

\haiku{{\textquoteright} en wachtend op de,:}{traktatie voegde hij zacht}{aan zijn verhaal toe}\\

\haiku{Een wachter buiten.}{een van de voorhangen gaf}{hij een kort bevel}\\

\haiku{Uit de verte had -}{nog onbekommerd gelach}{geklonken v\'o\'or hen}\\

\haiku{Die poincettia in -?...}{Holland was zo roodbladig}{geweest wat w{\`\i}st ze}\\

\haiku{Er trok een fijne.}{rimpeling van glimlach langs}{Djamilas kaken}\\

\haiku{En toen begon een:}{zeer naarstig onderricht in}{allerlei woorden}\\

\haiku{De hoofddoek welke:}{buiten werd gedragen met}{een rolband van koord}\\

\haiku{{\textquoteleft}Dit is koorts - ik ben,.}{besmet met iets waarvoor geen}{inenting bestond}\\

\haiku{Je blik is dauw over -.}{mijn eenzaamheid en ik kan}{slechts je ogen strelen}\\

\haiku{Terwijl ze van de,}{spiegel weg naar buiten keek}{zag ze door de tuin}\\

\haiku{{\textquoteright} Djamila wendde.}{haar ogen naar haar en schudde}{vriendelijk het hoofd}\\

\haiku{{\textquoteright} Daarmee ontnam hij.}{aan het samenzijn een al}{te formele toon}\\

\haiku{bladen vol bekers,.}{binnen die voor de gasten}{werden neergezet}\\

\haiku{Terwijl ze naar hem,:}{luisterde zag Agniet de}{hand van de dienaar}\\

\haiku{{\textquoteleft}Als er niets is, ben!}{ik voor altijd het malle}{mens van de slokjes}\\

\haiku{Zijn lichaam boog als.}{een brug hol \`opkrommend in}{gruwelijk lijden}\\

\haiku{Ze schaamde zich en.}{meende liefde en geluk}{te hebben verspeeld}\\

\haiku{Agniet vroeg zich af, -.}{waar de dode dienaar was}{gebracht en zijn vrouw}\\

\subsection{Uit: Maneschijn over uw hart}

\haiku{Eigenlijk leek het,:}{wel of hij vleugels achter}{zijn schouders voelde}\\

\haiku{De wandelaar bleef,.}{uit beleefdheid stil staan en}{wist niets te zeggen}\\

\haiku{En als hij baadde,.}{lag er een boekje met een}{potlood naast zijn kuip}\\

\haiku{Het verlangen in.}{zijn blik was te zwaar voor de}{Eerste Minister}\\

\haiku{De ochtend daarna.}{ging de Eerste Minister}{niet naar zijn bureau}\\

\haiku{Men zweeg, zoals dat.}{behoort onder ministers}{van goede komaf}\\

\haiku{Het uitmuntende;}{en voortreffelijke lag}{in onze snelheid}\\

\haiku{{\textquoteright} zeiden ze, {\textquoteleft}je komt!}{op Pasen misschien wel in}{een valstrik terecht}\\

\haiku{{\textquoteright} Zo kan een mens z'n.}{leven afhangen van \'e\'en}{onbedacht half uur}\\

\haiku{Op een nacht werd de.}{vrouw wakker omdat ze haar}{man hoorde zingen}\\

\haiku{{\textquoteright} Want iedereen kan,.}{begrijpen dat zoiets geen}{pas geeft in de nacht}\\

\haiku{En daar kon Onze.}{Lieve Heer op zijn beurt niets}{tegen bedenken}\\

\haiku{De dokter zette,.}{zijn bril op en verzocht de}{heer diep te zuchten}\\

\haiku{s Middags was de,.}{oven afgekoeld en er was}{aldoor niets gebeurd}\\

\haiku{{\textquoteleft}Beseft u alleen,,.}{reeds het feit dat niemand zich}{verbaast als \'u praat}\\

\haiku{Doch naast zijn bed stond,,.}{een radio die heel zachte}{mooie muziek speelde}\\

\haiku{Doch Jan Luchtbel viel {\textquoteleft},!}{voor hem op zijn knie\"en en}{zeiHier ben ik God}\\

\haiku{Zend mij asjeblieft,;}{terug naar de aarde m\`et}{armen en benen}\\

\haiku{En de koning lag,.}{op zijn purperen bed zo}{zwak als een leeg hemd}\\

\haiku{En de kroonprins stond.}{bij het lijk van zijn vader}{en keek naar de ring}\\

\haiku{{\textquoteleft}Laten we onze,{\textquoteright}.}{zaken bepleiten stelde}{de oudste neef voor}\\

\haiku{Maar zij is een vrouw,!}{ze kan niet vergeleken}{worden bij een man}\\

\haiku{Allen zongen een,:}{vreemd lied dat heerlijk en toch}{melancholiek klonk}\\

\haiku{{\textquoteright} zei het meisje, {\textquoteleft}want!}{je hebt me zo ontzaglijk}{gelukkig gemaakt}\\

\haiku{Hij onderzocht de,.}{tong van de pati\"ent en}{voelde hem de pols}\\

\haiku{Ja, dat was voor een;}{man des geloofs heel moeilijk}{te beantwoorden}\\

\haiku{Laat het even over aan,.}{de Schepper die u en uw}{klacht heeft geschapen}\\

\haiku{{\textquoteright} Nou, dat hadden er,.}{m\'e\'er geweten dus daarin}{stond zij niet alleen}\\

\haiku{Hij liep steeds voort, en.}{om hem heen zonk de stilte}{der verlatenheid}\\

\haiku{Hij kon het weten,.}{want hij had een oogje op}{een van de dochters}\\

\haiku{Hij wenste Pirre,:}{tot Ho-ho te bekeren}{en zei derhalve}\\

\haiku{Hij zei midden in {\textquoteleft}{\textquoteright}, {\textquoteleft}{\textquoteright};}{de geschiedenisTot ziens}{engoedemorgen}\\

\haiku{Doch de goede man,.}{was zo geschokt dat hij hun}{niet kon antwoorden}\\

\haiku{Want waar gebrek wordt,.}{geleden daar teert de mens}{in gezondheid weg}\\

\haiku{{\textquoteleft}Ik had het zo graag,,{\textquoteright}.}{willen voleinden tijdens}{mijn leven zei hij}\\

\haiku{{\textquoteleft}Ik heb een appel, -!}{met een wurm erin en ik}{h\'o\'ud zo van wurmen}\\

\haiku{Doch wanneer \`allen,.}{hun zorgen verliezen slinkt}{de dankbaarheid snel}\\

\haiku{{\textquoteleft}Je hebt hun alles,.}{ontnomen wat de hemel}{hun had beschoren}\\

\haiku{Zo naderde de,.}{dag dat de Keizer voor het}{eerst zou afdalen}\\

\haiku{Iedereen vermocht,.}{te begrijpen dat hem de}{dood slechts kon wachten}\\

\haiku{Want het was stil - veel,.}{stiller dan het ooit in een}{leeg bedehuis is}\\

\haiku{Nu was de pastoor (}{niet zo-maar de eerste}{de beste priester}\\

\haiku{Maar het Kruisbeeld hielp,.}{niets en de vragen werden}{heel goed beantwoord}\\

\haiku{Het was hem aan te,.}{zien dat hij in gedachten}{haastig iets telde}\\

\haiku{Niettemin was de.}{liefdadigheid gebaat met}{haar tien miljoen tun}\\

\haiku{{\textquoteleft}Je bent uit de hel -.}{opgeborreld en daar kun}{jij \'o\'ok niks aan doen}\\

\haiku{Het was ontzettend,.}{enerverend we wisten geen}{naam te bedenken}\\

\haiku{{\textquoteleft}Ik word dadelijk,.}{begraven en dat gebeurt}{niet iedere dag}\\

\haiku{- Maar een paar dagen.}{later vond ze de eerste}{cursus in de bus}\\

\haiku{Zo begon dat, en,.}{zo ging het door tot zij er}{niet meer buiten kon}\\

\haiku{Nu zal iedereen,!}{weten dat ik werkelijk}{een heilige ben}\\

\haiku{{\textquoteright} De oude pater.}{voelde zijn ziel vollopen}{met medelijden}\\

\haiku{Hij was de eerste,.}{die haar sedert lange tijd}{bij haar naam noemde}\\

\haiku{Daar kwam het larfje,.}{weer en het hielp ook deze}{zieke naar boven}\\

\haiku{Want ook boven het,.}{water is een hogere}{macht die grenzen legt}\\

\haiku{God schiep de mens, en.}{Hij schept hem nog aldoor en}{Hij zal hem scheppen}\\

\haiku{Zijn voeten heten {\textquoteleft}{\textquoteright} {\textquoteleft}{\textquoteright} {\textquoteleft}{\textquoteright}.}{Vader enMoeder en zijn}{hoofd heetNageslacht}\\

\haiku{De ruiten waren,.}{stuk de gordijnen hingen}{vergaan op de grond}\\

\haiku{{\textquoteright} hoorde men de Pang.}{in zijn troonkamertje voor}{zich heen bulderen}\\

\haiku{Een feit is, dat de.}{Pang op statige benen}{de Kamers doorschreed}\\

\haiku{Hij troonde me mee,.}{naar een lunchroom voor een kop}{koffie met slagroom}\\

\haiku{Anderen keken,.}{en glimlachten ook het leek}{wel besmettelijk}\\

\subsection{Uit: Met liefde en respect. Deel 1: Het devies}

\haiku{als een donkere,.}{holte waarin iemand een}{sleutel omdraaide}\\

\haiku{Weer voelde Didier:}{enige hoop in zijn diepste}{verlangens doemen}\\

\haiku{een pijl, gonzend en.}{haarscherp afgeschoten op}{het verliefde hart}\\

\haiku{En observeerde, -.}{hoe zij rechter-op kwam te}{zitten ze kuchte}\\

\haiku{{\textquoteleft}Wij moeten erop -,{\textquoteright}.}{achten dat het op peil blijft}{voltooide hij week}\\

\haiku{Soms dacht hij in z'n,}{eentje te weten wat voor}{een lief dom gansje}\\

\haiku{Mary bepeinsde;}{intussen dat Egelsbergh haar}{heel anders vasthield}\\

\haiku{{\textquoteleft}Ja,{\textquoteright} beaamde hij, {\textquoteleft}.}{peinzenddat zou ik aardig}{hebben gevonden}\\

\haiku{Mary leunde heel.}{voorzichtig naast hem tegen}{het gelakte hout}\\

\haiku{gebaad als ze werd.}{door het stortbad van pastoors}{diepzinnigheden}\\

\haiku{Mary was nog zo -.}{jong Antoine was ook nog}{maar vierentwintig}\\

\haiku{Het land veerde op.}{en ontwaakte voorzichtig}{tot bloei-poging}\\

\haiku{naar den hemel ging!}{wou ze zo g\`ere mevrouw}{gesproken hebben}\\

\haiku{Haar handen waren,;}{klaar met troosten breien en}{eten-bereiden}\\

\haiku{{\textquoteright} {\textquoteleft}Zeker niet, als het,{\textquoteright}.}{een biechtgeheim is stemde}{Antoine vroom in}\\

\haiku{Zijn vrouw blikte snel.}{en wantrouwend van onder}{haar dwaas chasseurtje}\\

\haiku{Maar ze kreeg een schoon,{\textquoteright}.}{jungsken vertelde meneer}{pastoor gedempt}\\

\haiku{Vreet God en Maria,,!}{op en verbrand m\`en d\`e ik}{niet meer bestoai}\\

\haiku{Dan mocht de Vrouwe.}{wel een prachtige poffer}{tentoonstellen}\\

\haiku{{\textquoteleft}Moar ge krijgt 'et ok,{\textquoteright}, {\textquoteleft} '!}{nie had Sjef geantwoordik}{hout lekker zelf}\\

\haiku{opeens zag men in,.}{dat hier de juiste woorden}{waren gebezigd}\\

\haiku{Mevrouw van het Huis -.}{stapte sjust van heuren fiets}{zij was hier ook klant}\\

\haiku{{\textquoteleft}En dan zetten we,{\textquoteright}, {\textquoteleft}:}{d'r nen bordje bij vulde}{ze aanwaarop stoat}\\

\haiku{In stilte besloot,:}{ze de eventu\'ele zoon naar haar}{vader te noemen}\\

\haiku{{\textquoteright} Mary verkilde.}{en mompelde haperend}{dat mijnheer uit was}\\

\haiku{Maar Mary vond het,.}{dienstig nu van onderwerp}{te veranderen}\\

\haiku{{\textquoteleft}Voorzover ik dit,.}{huis waardeer heb ik deze}{ruimte het liefste}\\

\haiku{En meteen gaf hij,.}{zo'n verschrikkelijke gil}{dat hun oren tuitten}\\

\haiku{Een doodstille roep.}{die alles zou overklinken}{als Het Ogenblik kwam}\\

\haiku{Het sluike stappen,.}{van de dragers de kist die}{zich maar liet voeren}\\

\haiku{En ze schaamde zich -.}{ze sloot de ogen om zichzelf}{niet te beseffen}\\

\haiku{{\textquoteleft}Wat ben ik blij, dat,{\textquoteright}.}{ik dit eens mag zien zei ze}{zeer  hartgrondig}\\

\haiku{Aan Leentje maakte.}{zij haar wens kenbaar Mevrouw}{te mogen spreken}\\

\haiku{Ach ja, ach ja, de.}{bezoekster  moest daarvan}{heen en weer wiegen}\\

\haiku{en met dat doel had.}{zij tamelijk lang in het}{klooster verbleven}\\

\haiku{Juffrouw Van Toossen,{\textquoteright}}{komt om mij te waarschuwen}{tegen Miet Lintjen}\\

\haiku{En die afgang was:}{eveneens statiger gewenst}{dan geworden}\\

\haiku{Ze legde haar hand,:}{even op de andere arm}{van Miet en zei zacht}\\

\haiku{dit alles in de.}{dagen van winterfeest op}{haar pad te vinden}\\

\haiku{{\textquoteleft}Ja, dat geloof ik, -,,,...{\textquoteright}}{wel ja Joujou of Kiki}{of Trala of Foeifoei}\\

\haiku{{\textquoteright} Nee, ze voelde zich.}{onder de schijn van volmaakt}{geluk heel alleen}\\

\haiku{En Janus had zich -?}{onhandig gevoeld wat moest}{je daarop zeggen}\\

\haiku{In de winkel werd.}{druk gesproken over Nilles}{van den Bollebek}\\

\haiku{{\textquoteright} lispelde een vrouw,.}{die haar mond niet langer kon}{weerhouden van spraak}\\

\haiku{En Miet glom alsof.}{haar ziele gepoetst brons was}{uit engelhanden}\\

\haiku{Een beven zonk door,,}{haar leden alsof ze met}{iets gruwelijks sprak}\\

\haiku{Het hinderde niet -.}{of er iemand in huis was}{of niet ze praatte}\\

\haiku{Nee, dat dee ze niet -.}{ze trok de benen onder}{het lijf en stond recht}\\

\haiku{- Antoine zal je,.}{snel berichten en alles}{in orde maken}\\

\haiku{Ze liep heel langzaam,.}{naar het terras en zocht een}{papier en potlood}\\

\haiku{{\textquoteright} Maar vriendin Marie.}{Antoinette heette ook}{niet naar de duivel}\\

\haiku{Het Huis beleefde.}{dagen van glorie en de}{gloed van jong leven}\\

\haiku{Wist een vrouw ooit, waar...?}{haar echtgenoot de gelden}{vandaan toverde}\\

\haiku{En dan stondt ge in,}{het keurige vertrek waar}{onder glimstolpen}\\

\haiku{{\textquoteleft}Want d\`e huis was toch,!}{van Egelsbergh en d'r kan wel}{fortuin in zitten}\\

\haiku{Soms dacht Antoine,,.}{herhaling te zien alsof}{het een spel betrof}\\

\haiku{{\textquoteleft}Ik dacht dat je in,...{\textquoteright},.}{Itali\"e zat of ergens Ja}{dat deed Claire ook}\\

\haiku{Toine voelde zich!}{onprettig alert tegenover}{zijn eigen vrouw}\\

\haiku{Wat kon er dus nog?}{voor roerends tot stand komen}{in hun lieve kerk}\\

\haiku{En die lach had hij.}{nu om de lippen van het}{mevrouwtje gezien}\\

\haiku{En ik doe het nou,...{\textquoteright} {\textquoteleft},!}{ook pastoorM'n goeie vriend wat}{bent ge zwaarwichtig}\\

\haiku{Zelfs de goede wil.}{van een zieleherder is}{niet altijd genoeg}\\

\haiku{Een Kerstnacht-mis...?}{mee echt levende mensen}{in het stalleken}\\

\haiku{Sinterklaas kwam uit.}{Spanje en voer na enig gul}{gedrag weer terug}\\

\haiku{het smadelijke.}{grinniken en w\`egkijken}{van dorpsgenoten}\\

\haiku{{\textquoteleft}Ik denk, dat ik maar!}{eens naar de commissaris}{van politie ga}\\

\haiku{Maar in het bos had.}{een grote jongen haar het}{geld afgenomen}\\

\haiku{- {\textquoteleft}Waarom moeten wij?}{medelijden hebben met}{die armoedzaaiers}\\

\haiku{en men vroeg elkaar,?}{wat de buitenwacht daarmee}{ook van node had}\\

\haiku{{\textquoteleft}Zoudt ge d'r niks voor, '?}{voelen opt Gavenoord}{te komen wonen}\\

\haiku{Ze oogde hem z\'o,.}{recht in zijn hart dat hij er}{bijna van verschoot}\\

\haiku{{\textquoteright} en in zijn werkkiel;}{en pillow-broek stapte}{Ruur in naast mijnheer}\\

\haiku{{\textquoteleft}Moet ik me door een,?}{vrouw laten voorschrijven hoe}{ik moet handelen}\\

\haiku{Het was verfrissend -, -.}{iemand die haar nodig had}{die haar een taak bood}\\

\haiku{{\textquoteleft}De hekken van mijn,,{\textquoteright}.}{hart zijn d\`e nie kaploan}{zei ze vriendelijk}\\

\haiku{{\textquoteright} {\textquoteleft}Ach,{\textquoteright} antwoordde haar, {\textquoteleft}!}{man over-volwassenwat}{h\`elpt je best doen nou}\\

\haiku{En intussen wist.}{ze dat kindje groeiende}{binnen haar lichaam}\\

\haiku{Och, mevrouw, - 'n mens,{\textquoteright}.}{hee wel es zurge zei vrouw}{Van Mosse klankloos}\\

\haiku{Hij bleek toch in de;}{kerk al een ietwat schrille}{figuur te worden}\\

\haiku{Het lieve ouwe.}{mens had een jurkje gebreid}{voor de luiermand}\\

\haiku{Bij het vuur, in een;}{soort lila tent omdat ze}{zo omvangrijk werd}\\

\haiku{Amadeetje was nu -!}{ruim twee-en-een-half het zou}{zo allerliefst zijn}\\

\haiku{De oude Koning,.}{boog het eerst en vertelde}{van hun verre tocht}\\

\haiku{Vrouw Besonder was,.}{de ganse avond kalm geweest}{zoals meestal}\\

\haiku{Dat zij zo totaal,.}{ontkleed waren shockeerde}{haar een kort moment}\\

\haiku{{\textquoteright} Maar Toine lachte.}{met zijn al te heldere}{ogen op haar gericht}\\

\haiku{Antoine zweeg, raar,. {\textquoteleft}?}{op zijn hoede als voor een}{verkeerd seinClairtje}\\

\haiku{{\textquoteright} {\textquoteleft}O,{\textquoteright} antwoordde Sjef, {\textquoteleft},.}{gauw dan mar want me klaante}{stoan de wachte}\\

\haiku{Hij was zo totaal,.}{niet de beklaagde dat ze}{allemaal zwegen}\\

\haiku{maar ze d\`acht aan die - - {\textquoteleft}?}{twee ranke lichamenBlijf}{je altijd bij me}\\

\haiku{Een vriendelijke,.}{mannenstem informeerde}{of mijnheer thuis was}\\

\haiku{En de boerin die,,:}{had gesproken wendde zich}{tot mevrouw en vroeg}\\

\haiku{De Egelsberghs volgden.}{in een geleende wagen}{van goede vrienden}\\

\haiku{Toen de wagen voor,.}{het bordes stopte schrok de}{echtgenoot wakker}\\

\haiku{En nog nasnuivend,;}{stapte ze uit bed en ging}{naar de linnenkast}\\

\haiku{hij was geweest op,.}{het fr\^ele blonde meisje}{Mary van Genthen}\\

\haiku{Tussen twee wee\"en,.}{bedacht ze dat hij verloofd}{was en sprak daarover}\\

\haiku{de dorpsjongen wist,!}{een holle boom waarin je}{je hand kon steken}\\

\haiku{Omdat het bloed ons -.}{bindt aan leven en aan dood}{we blijven soamen}\\

\haiku{en oorhangers als - -... {\textquoteleft}}{van een koningin en een}{armband en een speld}\\

\haiku{Samen begonnen.}{ze hem te wassen en het}{bloed af te weken}\\

\haiku{Johan reed de wagen,.}{tot vlak voor de kerk waarvan}{de deuren openstonden}\\

\haiku{maar tegelijk was:}{er een vaag gevoel in haar}{van teleurstelling}\\

\haiku{Mary besloot, dat,;}{Johan alleen naar huis moest de}{auto wegbrengen}\\

\haiku{Wat wist zij ook van?}{woede en van Ter Tuynen}{Egelsberghse speelsheid}\\

\haiku{Maar Classen had hem,.}{meegetroond en Mary had}{hem binnengehaald}\\

\haiku{en daarbij was zij.}{met een voet in een molshoop}{te land gekomen}\\

\haiku{Ach, als ze het slechts (...)}{had gez\`egdze had het niet}{hoeven te menen}\\

\haiku{Z{\'\i}j was begonnen,.}{vrouw Van Mosse haar stukske}{land af te nemen}\\

\haiku{een levensgroot zicht:}{op huwelijksgeluk in}{de rijke wereld}\\

\haiku{Johan, als je tijd hebt,?}{wil je dan eens kijken naar}{de kinderwagen}\\

\haiku{{\textquoteleft}En waarom hedde?}{gij d\`e dan loater nie}{tegen m\`en gezegd}\\

\haiku{Antoine moest weg -, -;}{ze wist niet waarheen zij kreeg}{de auto met Johan}\\

\haiku{{\textquoteleft}Ik dreig ginnen mens,{\textquoteright}, {\textquoteleft} {\textquotedblleft}{\textquotedblright},.}{antwoordde de Muntik zeg}{heu tege m'n perd}\\

\haiku{Het wapen glipte,.}{Bollebek uit de hand de}{man zelf wankelde}\\

\haiku{Barntje werd namens.}{de pastoor thuisgebracht door}{een grote jongen}\\

\haiku{En hij hield pas op,.}{toen Noud hem een dubbeltje}{beloofde \`en gaf}\\

\haiku{Ze wankelde op.}{tastende voeten naar de}{deur en opende die}\\

\haiku{zodat men toch als, {\textquoteleft}{\textquoteright}.}{een soort gezin tezamen}{bleefonder vrienden}\\

\haiku{En ver achter haar.}{leunde Sjef Castel tegen}{een pilaar en keek}\\

\haiku{Haar glimlach droeg een -.}{vernis van schreien haar ogen}{speelden niet meer mee}\\

\haiku{{\textquoteright} maar terwijl ze sprak, -!}{zag ze een schrik over zijn ogen}{zwemen snel weer weg}\\

\haiku{{\textquoteleft}Je bent een prachtig,!}{wijf ik  vind je na elk}{kind mooier worden}\\

\haiku{{\textquoteleft}Ik met een dikke,,!}{buik en zwangerschapsvlekken}{en luiers tellen}\\

\haiku{Maar hij greep haar hand,,.}{en kuste die terwijl hij}{haar stevig vasthield}\\

\haiku{{\textquoteleft}Ja, maar als ik zo,!}{dicht mogelijk bij je blijf}{word je weer zwanger}\\

\haiku{Ik worstel ook niet,{\textquoteright}.}{spiernaakt met hem bracht Mary}{zeer lief naar voren}\\

\haiku{Aan de telefoon -.}{kwam een lieve stem dat was}{de vrouw des huizes}\\

\haiku{As d'r engelkes,!}{z\`en dan hebbe ze genoeg}{aander toak te doen}\\

\haiku{Hij nam het epistel.}{in handen en liet er zijn}{vingers over tasten}\\

\haiku{en zoons en dochters.}{hielden zich gereed om de}{moeder bij te staan}\\

\haiku{En iedere boer,!}{kon van nabij zien van welk}{edel ras zij waren}\\

\haiku{Harry poogde, de -.}{rem te grijpen maar dat zou}{ook gevaarlijk zijn}\\

\haiku{{\textquoteright} en hij zag even iets.}{als begin van een glimlach}{om haar mond plooien}\\

\haiku{Hij had allerlei -.}{zinnen op de lippen en}{sprak er niet een uit}\\

\haiku{{\textquoteleft}Hij moest jou maar es,{\textquoteright}.}{in je nakie zien snibde}{de jonge moeder}\\

\haiku{De afzender had,.}{alleen adres vermeld en dat}{was haar onbekend}\\

\haiku{Niet alleen in de;}{kerk was enkele malen}{voor hen gebeden}\\

\haiku{{\textquoteleft}Binnen enkele!}{jaren rijden de kleine}{burgers in auto's}\\

\haiku{{\textquoteright} en dan knelde hij,.}{haar in zijn armen dat ze}{geen adem kon vangen}\\

\haiku{Maar was de wijze,...?}{statige priester ook niet}{van groter waarde}\\

\haiku{{\textquoteleft}O ja, zegt u maar,!}{dat wij er om ongeveer}{tien uur zullen zijn}\\

\haiku{Mary huiverde.}{en was blij dat Antoine}{haar begeleidde}\\

\haiku{maar met het oog op.}{mijn wettige zoon kan ik}{haar kind niet echten}\\

\haiku{Als goed arts had hij,.}{meer oog voor haar verdriet dan}{voor het eigene}\\

\haiku{Voorzichtig nam hij,.}{de vracht over en legde die}{naast zijn eigen plaats}\\

\subsection{Uit: Met liefde en respect. Deel 2: Maskeravond}

\haiku{De buslijn loopt via.}{Den Deun door Woenselsven naar}{Rogunen en voort}\\

\haiku{Toen dat allemaal,...}{gebeurde waren wij ook}{in Woenselsven}\\

\haiku{Zijn rokken zijn te}{lang en zijn kop te vuil en}{zijn stem snijd mijn deur}\\

\haiku{Hij had een zwembad,.}{geopend in 1928 en een}{fiets-school in 1932}\\

\haiku{{\textquoteleft}Ik hoop, dat die naam,{\textquoteright}.}{jullie allen geluk zal}{brengen zei Mevrouw}\\

\haiku{En toen dit meneer,:}{pastoor ter ore kwam knikte}{hij beheerst en zei}\\

\haiku{Hij zei gewoon dat.}{ze niet mee kon doen daar ze}{het type niet was}\\

\haiku{Antoine was een,,!}{aardige vlotte kerel}{ze hield veel van hem}\\

\haiku{IK GELOOF NIET   .}{en meer onderaan stond zijn}{bloedeigen naam}\\

\haiku{Misschiens moest hij nou,.}{wel veur de radio komen}{om te proaten}\\

\haiku{Mijnheer pastoor De.}{Wett was een man van uiterst}{strenge begrippen}\\

\haiku{En Ceeske voelde.}{zich gebrand en gerafeld}{tussen deze twee}\\

\haiku{{\textquoteleft}Voel je zelf niet, dat?...}{het ongehoorzaamheid preekt}{en de geest zwak maakt}\\

\haiku{En van Antoines.}{kant kende ze eigenlijk}{alleen zijn vader}\\

\haiku{Hij vestigde een.}{paar geduldige ogen op}{de brutale vent}\\

\haiku{Hij wentelde zich,.}{in eigen rouw en trok daar}{het gezin in mee}\\

\haiku{hij zoende -) en ze -...}{leunde met haar hoofd tegen}{zijn borst \'e{\'\i}ndelijk}\\

\haiku{De zoon keek naar de.}{tassen en vandaar naar de}{liggende vader}\\

\haiku{Hij zweeg, terwijl zij.}{de tafel dekte en de}{borden neerzette}\\

\haiku{In de gang rook het.}{naar slechte tabak en het}{was er smoezelig}\\

\haiku{{\textquoteright} En van zijn hoge:}{toren uit repliceerde}{haar fiets-leraar}\\

\haiku{Twee dagen later;}{dwarrelde er een bonte}{postkaart in de bus}\\

\haiku{En wat was er veel,.}{voor een Egelsbergh-zoon}{om nu te denken}\\

\haiku{Snotterend verrees.}{moeder Van Drimmelen en}{voegde zich bij haar}\\

\haiku{Ze schudden mekaar.}{stevig de hand. En elk ging}{weer zijn eigen weg}\\

\haiku{Het was een kille,,.}{dag nevelig en kaal een}{afwijzend ogenblik}\\

\haiku{{\textquoteleft}Goeiendag,{\textquoteright} zei de,,.}{deken glimlachte knikte}{en schoof zijn stoel bij}\\

\haiku{{\textquoteright} herhaalde Deetje,.}{hardop en bezichtigde}{Sjef nadrukkelijk}\\

\haiku{Hij knielde bij haar,.}{neer sloeg zijn arm om haar hals}{en vroeg wat er was}\\

\haiku{Laat die nacht, sliep ze -.}{eindelijk in diep-weg}{voor enkele uren}\\

\haiku{Mary dacht, dat ze!}{later nooit weer marsmuziek}{zou kunnen horen}\\

\haiku{Mary had op de -.}{radio moeten letten het}{maakte haar doodziek}\\

\haiku{Het was haar, of er.}{een groot licht in de kamer}{begon te schijnen}\\

\haiku{Als de koningin,!}{zou worden doodgeschoten}{hadden we niets meer}\\

\haiku{Ach - we voelen ons,{\textquoteright}.}{zeker niet \`on-veilig}{antwoordde Toine}\\

\haiku{Hij sloeg de armen, ().}{om haar heen waar Noud bij stond}{die wendde zich af}\\

\haiku{Geen zwezerik, geen,,;}{ossetong geen nierstuk geen}{biefstuk van de haas}\\

\haiku{Dat was een bijnaam.}{geweest voor een jongen die}{dikwijls dronken was}\\

\haiku{De distributie,.}{was merkbaar doch er was van}{alles nog genoeg}\\

\haiku{aan het slot van de.}{dienst leken zij er geheel}{bij te behoren}\\

\haiku{Ze sprak met velen.}{van hen en wenste allen}{een zalig Kerstfeest}\\

\haiku{Toen het meisje hem, -}{aandiende kreeg Evelien de}{prikkels over de rug}\\

\haiku{Toen dit bericht het,.}{Huis bereikte keek Mary}{haar echtgenoot aan}\\

\haiku{hij mocht een beetje,.}{spelen in de huiskamer}{en moest vroeg naar bed}\\

\haiku{{\textquoteleft}Het wordt nu zaak, heel,.}{goed uit te kijken wie wij}{in ons huis halen}\\

\haiku{Hij liep moeilijk, zijn.}{gezicht was vervallen en}{hij had een blauw oog}\\

\haiku{Hij was dan toch een -.}{juweel al had ze dat nooit}{achter hem gezocht}\\

\haiku{Toen werd de priester -.}{witgloeiend wat natuurlijk}{toch zwak van hem was}\\

\haiku{Hij wou er verder -!...}{niets van weten een mens zou}{d'r ziek van worden}\\

\haiku{En waarom daar, als?...}{je zulke brede muren}{had in elk vertrek}\\

\haiku{Meteen liet hij zich,.}{over haar heen vallen en sloeg}{zijn armen om haar}\\

\haiku{Alleen Mevrouw wist, -.}{waar die vandaan kwamen het}{was een zoet geheim}\\

\haiku{Kortbesloten stond,.}{Mary op klaar om boos te}{worden op wie ook}\\

\haiku{hij zette alles,.}{neer in de kamer en sloot}{de gordijnen}\\

\haiku{Het wonder, dat zo'n,!... {\textquoteleft}}{grote meneer hem begr\'e\'ep}{en meteen maar hielp}\\

\haiku{Zijn  biechtvenster,,.}{uitzicht naar vergiffenis}{was toegevallen}\\

\haiku{De eerste gast op,.}{de Woens was papa Egelsbergh}{die geld kwam lenen}\\

\haiku{Hij was zo geplet,.}{dat hij zijn thee dronk en zich}{geweldig brandde}\\

\haiku{De vilten deur door,,.}{een poortje van latwerk een}{donkere trap af}\\

\haiku{Ze gleed bijna uit - -.}{ze moest beter kijken de}{trap was slecht verlicht}\\

\haiku{Hun stemmen galmden.}{bol en onbeheerst tegen}{het lage gewelf}\\

\haiku{En het lachende,:}{jongenskopje dat om de}{hoek van de deur keek}\\

\haiku{Daar wachtte hem een.}{goeiendag en vragen naar}{zijn welbevinden}\\

\haiku{Het leed geen twijfel.}{of de pastoor verdiende}{dit brood zeer moeizaam}\\

\haiku{Gr\"ucklich,{\textquoteright} herhaalde -.}{hij enkele malen en}{trok haar naar zich toe}\\

\haiku{Zij was volslagen.}{overstuur thuisgekomen met}{gescheurde kleren}\\

\haiku{En kijk nu, wie zij:}{om de hoek van de Lange}{Kruisstraat tegenkwam}\\

\haiku{Hoe vaak hadden de...?}{struiken om dat bankje heen}{gebloeid en gegeurd}\\

\haiku{Opeens dook er een.}{veelvuldige schaduw op}{uit het dorre blad}\\

\haiku{- Enkele mensen.}{vergaten de spertijd en}{vlogen hun huis uit}\\

\haiku{Ze hurkte neer en,, {\textquoteleft}!}{zamelde natte hete}{scherven roependAu}\\

\haiku{Mevrouw Mary ging -.}{ook na het eten niet weg zij}{bleef hen geleiden}\\

\haiku{en over die woorden.}{heeft juffrouw Marie Calchoen}{werkelijk geschreid}\\

\haiku{Het zoontje Aartje mocht.}{even binnenkomen om de}{gast te begroeten}\\

\haiku{{\textquoteleft}Wij hebben genoeg,.}{en van andere grond smaakt}{het soms lekkerder}\\

\haiku{En toen opeens werd, {\textquoteleft}!!}{haar peinzen verflard door een}{wilde stem dieHALT}\\

\haiku{Traag, druppelsgewijs.}{vormde zich een vermeld feit}{uit zijn gestamel}\\

\haiku{en voelde zich zo,.}{vreselijk arm dat ze er}{bijna ziek van werd}\\

\haiku{Hoe kwam het, dat zo,?...}{velen wisten waarheen hun}{schreden zich richtten}\\

\haiku{En toen een kreet - z\'o,.}{afgrijselijk dat allen}{naar boven stormden}\\

\haiku{Vader, moeder en.}{broertjes en zusje vlogen}{de kamer binnen}\\

\haiku{{\textquoteright} De gast knikte en {\textquoteleft}{\textquoteright}.}{glimlachte en herhaalde}{memorerendRuur}\\

\haiku{Hij kon zo prachtig!}{controleren of de stof}{echt begrepen was}\\

\haiku{De jongen was nu,.}{bijna veertien jaar oud en}{fors voor zijn leeftijd}\\

\haiku{Ze werd angstig, en,.}{als ze z\'e\'er nerveus was keek}{ze in de spiegel}\\

\haiku{En elk uur kwam het -.}{geraas nog nader het schoof}{bijna in hun huid}\\

\haiku{{\textquoteleft}Ik heb geen kleren - - -...{\textquoteright}}{en geen geld ik weet niet wat}{er zal gebeuren}\\

\haiku{Ze liep de kamer,.}{uit naar de achterdeur om}{die te ontsluiten}\\

\haiku{Berooid was ze, en -.}{verlaten straatarm in haar}{diepste gevoelens}\\

\haiku{De volgende dag!}{kwam de mare dat Breda}{eveneens was bevrijd}\\

\haiku{Zo had Mary dus.}{twee Duitse jongens gemeld}{bij de bevrijders}\\

\haiku{Nou ja, zij waren,{\textquoteright}.}{niet altijd onze vrienden}{vulde Toine aan}\\

\haiku{Zij kusten Amad\'e Auf,.}{Wiedersehen als echte}{Duitse kinderen}\\

\haiku{Mevrouw bleef binnen,.}{de deur die nog door de knecht}{werd open gehouden}\\

\haiku{Dieje Barnt was ne,,.}{vuil ventje moar joa zo'n keind}{wist toch van niks nie}\\

\haiku{Een schaduw van iets.}{onaangenaams somberde}{over Mary's denken}\\

\haiku{12 Het leven in.}{Nederland leek spoedig weer}{normaal te worden}\\

\haiku{De winkel stond vol,.}{mensen toen Bollebek daar}{binnenkwam dwalen}\\

\haiku{Hij praatte gewoon,.}{verder mee Piet en nam een}{geducht pak vlees mee}\\

\haiku{Zo vriendelijk en,.}{blank alsof er nimmer een}{conflict was geweest}\\

\haiku{Daar drentelde zij -,;}{dus mevrouw Mary in haar}{eigen stille tuin}\\

\haiku{Ik hoop dat je er,{\textquoteright}.}{nog eens over wilt nadenken}{drong zijn moeder aan}\\

\haiku{Het zakje viel op,.}{de grond hij had zijn vingers}{vol stroperigheid}\\

\haiku{Gewond en geknakt,.}{trok de troep weer naar Spanje}{en vandaar noordwaarts}\\

\haiku{Ze had Toine nog.}{nooit z\'o duidelijk en zo}{plat horen praten}\\

\haiku{De eigen teelt bleek,.}{niet goed dat jaar de zomer}{was te warm geweest}\\

\haiku{Sommige dames,.}{of heren negen zeer diep}{al voortwandelend}\\

\haiku{Het viel Mary op;}{dat meisjes meer wuifden dan}{de jonge kerels}\\

\haiku{Oma had flossig bleek,.}{haar gekregen met grote}{lokken wit daardoor}\\

\haiku{De blijdschap was zo,.}{onverbloemd dat Mary weer}{geheel opveerde}\\

\haiku{En ze beloofde,.}{hemelsblank de groeten te}{zullen overbrengen}\\

\haiku{{\textquoteright} kefte Wine - en.}{zij zetten hun aarzeling}{om in halve draf}\\

\haiku{Johan en Amad\'e waren.}{daarneven en brachten haar}{omzichtig overeind}\\

\haiku{{\textquoteleft}Gistelbergen,{\textquoteright} had,.}{hij gezegd en hij had zijn}{hand uitgestoken}\\

\haiku{Zo'n veurnoame mens had,.}{de Munt nie in z'ne buurt}{en d\`e kon \^ok nie}\\

\haiku{Maar in gesprekken.}{bemerkte je nooit iets van}{bezorgdheid of smart}\\

\haiku{Maar naast hem zat Barnt,,.}{ruiger van uiterlijk een}{beetje onrustig}\\

\haiku{Mary wist dat hij.}{zelf ook nooit een ander zou}{hebben geholpen}\\

\haiku{Tabak en te lang.}{gedragen ondergoed en}{roestige spijkers}\\

\haiku{Dat eten is al gek -?}{maar waar wou je die ouwe}{man van betalen}\\

\haiku{De ouwe heer had.}{dus gepeinsd over begraven}{van zijn geldwaarden}\\

\haiku{{\textquoteright} En toen zijn vader,}{zich na een uurtje nogmaals}{absenteerde liep}\\

\haiku{Nu was het vrede,.}{en nog was de wereld vol}{snikken en tranen}\\

\haiku{{\textquoteright} Babette lachte -!}{vrolijk maar het klonk zo ver}{en los van het bed}\\

\haiku{{\textquoteright} Babette belde.}{met een ijzerdraad-dun}{handje de zuster}\\

\haiku{Die middag brandde;}{ze haar kindervleugels aan}{de kunstmatigheid}\\

\haiku{Kleine, ontkleurde,...}{schim die zo dapper op dat}{bed had gelegen}\\

\haiku{Uit het huis kwam vaag -,.}{gerucht van spreken een deur}{die opende en sloot}\\

\haiku{Jan,{\textquoteright} zei Mary, {\textquoteleft}ik;}{ben gekomen om afscheid}{van haar te nemen}\\

\haiku{Dat was bijna op -.}{de hoek aan het begin van}{de Gevloekte Weg}\\

\haiku{maar een uur later:}{stond er op het bord met nog}{veel oranjer letters}\\

\haiku{Dit is vreselijk,,{\textquoteright},.}{voor een vader stemde ze}{toe met zachte stem}\\

\haiku{Maar ze wist - aan het -.}{wriemelen van zijn handen}{dat hij was geraakt}\\

\haiku{Hij zonk en verloor -.}{zijn laatste expressie hij}{werd een leeg lichaam}\\

\haiku{Nu en dan opende.}{ze de ogen en dacht aldoor}{te hebben gewaakt}\\

\haiku{nie zien d\`e de twee...,!}{families vechten om ne}{heilige Mis nee}\\

\haiku{De dode moest op.}{een schone eigen wagen}{worden gereden}\\

\haiku{dat \'e\'en de eerste.}{moest zijn in het middenpad}{van Sinte Maria}\\

\haiku{En met een lichte.}{hoofdnijging liep hij naar het}{Wit Engelpad}\\

\haiku{Op de bioscoop van:}{wijlen Sjef Castel stond nu}{met gouden letters}\\

\haiku{Hij was maar \'e\'en jaar -.}{ouder dan Toontje van de}{chauffeur die was tien}\\

\haiku{Alle dieren in,.}{de stal waren geluidloos}{hoewel er geen sliep}\\

\haiku{Wat m\'o\'est zo'n vrouw nou,...}{allenig in huis en dan}{achter in de tuin}\\

\haiku{Alleen Amad\'e keek blij,.}{met glimmende ogen en een}{aardige grinnik}\\

\haiku{Bij navraag bleek dat.}{Amad\'e erover had gejubeld}{tegen zijn broer Barnt}\\

\haiku{wat verschrikkelijk - - -{\textquoteright}.}{hoe k\`an dat Doch hij sloeg dicht}{op haar kille blik}\\

\haiku{Het konijn wipte.}{lekker naar binnen alsof}{hij had geoefend}\\

\haiku{en d\`e knijn kennik...{\textquoteright},!}{nie En ach bijna had hij}{Buikje vergeten}\\

\haiku{Ze meende bij al...}{deze dieren iets liefs in}{de ogen te speuren}\\

\haiku{Die mevrouw ontzet,,:}{dat begrijpt u. Maar hij blijft}{kalm en toont haar aan}\\

\haiku{Mary had zich snel -.}{willen afwenden maar iets}{had haar weerhouden}\\

\haiku{en knipte af, en -...}{rolde snel het volgende}{beeldvlak voor haastig}\\

\haiku{Ze hebben te veel,!}{gezien en gehoord en zelf}{aan de kant gestaan}\\

\haiku{Ze sprak over haar bruur, '...}{die ze altijn bietje}{gek hai gevonden}\\

\haiku{Veel bomen in het.}{bos bezweken en zwiepten}{omver als latten}\\

\haiku{Toine vertelde.}{zo breedvoerig als vleiend}{bleef voor de dochter}\\

\haiku{Binnen twee weken!}{sturen ze je terug als}{een geplukte kip}\\

\haiku{Daarna het ontbijt,.}{dat in volkomen stilte}{moest worden gebruikt}\\

\haiku{ik heb kinderen.}{van jouw zaad gedragen en}{ter wereld gebracht}\\

\haiku{een enkele groep,...?}{argelozen om over de}{grens te geraken}\\

\haiku{Maar wellicht zou zij.}{binnenkort een heel echte}{lady worden}\\

\haiku{Zij k\`on niet denken.}{aan een demonstratie van}{huwelijksgeluk}\\

\haiku{In de Bijbel staan.}{alleraardigste dingen}{over dat onderwerp}\\

\haiku{En om haar hals had.}{ze een gouden rozenkrans}{met paarse stenen}\\

\haiku{Ach, zij was ook zo,!...}{broos en lichtvoetig als zij}{door de gangen ging}\\

\haiku{Mary probeerde.}{altijd enige vreugde te}{brengen aan dit graf}\\

\haiku{Ik denk ook vaak aan,{\textquoteright}.}{dat arme kleine jonkje}{antwoordde Mary}\\

\haiku{En Mary vroeg niet,.}{verder zij vertrouwde haar}{prachtige dochter}\\

\haiku{Het begrijpen van.}{de situatie zonk als}{waanzin op Mary}\\

\haiku{Moar hoe d\`e ze dan?...}{tot dizze bepoaling}{had kunnen komme}\\

\haiku{Toen knielde meneer.}{pastoor neer en hij vouwde}{de haand en hij bad}\\

\haiku{Freer begreep tot diep,;}{in zijn vezels wat dat te}{betekenen had}\\

\haiku{Want ja, verbeeldje '!...}{dat Corrie doar beneen}{sluksken te kort kwam}\\

\haiku{Dat w\`as ze ook - maar...}{de werkelijkheid smaakte}{anders dan de droom}\\

\haiku{Antoine breidde.}{zijn armen uit en Derk deed}{een stap naar voren}\\

\haiku{Mary spitste de blik,.}{erheen maar sloeg meteen haar}{ogen neer als betrapt}\\

\haiku{Derk bracht hen naar een.}{grote kamer met blanke}{houten meubelen}\\

\haiku{{\textquoteright} Toine zat met een.}{vreemd scherpe expressie naar}{zijn kind te kijken}\\

\haiku{In 't voorbijgaan.}{zag Mary een paar keren}{de negerjongen}\\

\haiku{slank, met een tragisch,.}{masker alsof hij net uit}{een huilbui verrees}\\

\haiku{maar door geldgebrek.}{was hij gaan poseren voor}{naaktfotografie}\\

\haiku{Zij begreep nu dat;}{ze nooit de moeder van Aartje}{had kunnen worden}\\

\haiku{Toine begon steeds.}{dringender te klagen over}{alle verliezen}\\

\haiku{Daarnaast pijnigde.}{haar de vergeefse reis naar}{Denemarken}\\

\haiku{{\textquoteright}... en {\textquoteleft}Lisabeth, o,!,!...}{moar m'n h\'emel ge het oew}{voeten nie geveegd}\\

\haiku{{\textquoteleft}Moar, zuster Porta,!...}{w\`e bende slonzig mee oew}{ermen en bene}\\

\haiku{Kom, stoat op en doe,,!}{dees over want dit is lillek}{rauw gedoan heur}\\

\haiku{{\textquoteright} Andere zusters.}{kwamen toegelopen en}{stonden geschokt stil}\\

\haiku{daar stond, met haar kap,.}{dwars over het hoofd en zonder}{ijver in haar ogen}\\

\haiku{{\textquoteleft}Maar hij is nog geen, -!}{twintig hij woont daar maar hij}{verdient te veel geld}\\

\haiku{Ze wandelde met,;}{een boerin vertrouwelijk}{naar elkaar geneigd}\\

\haiku{Johan wipte van zijn,,.}{plaats liep om de wagen heen}{hield het portier open}\\

\haiku{Mevrouw Egelsbergh kreeg.}{haar zoon en haar juwelen}{mee terug naar huis}\\

\haiku{wat was ze geroerd,...}{geweest toen hij voor het eerst}{bij hen mocht branden}\\

\haiku{Ze kon toch ook wel?}{een paar goede kinderen}{hebben voortgebracht}\\

\haiku{Zo zaten zij te.}{zamen voor een kop koffie}{in de eetkamer}\\

\haiku{maar hielden alert oog.}{op gebalde vuisten en}{vlammende blikken}\\

\haiku{Hij sprak steeds sneller,.}{om niet in de rede te}{worden gevallen}\\

\haiku{Met zeventien jaar.}{modellen gestolen en}{adressen verraden}\\

\haiku{Hij was spierwit en.}{keek zijn bloedverwanten aan}{met brandende ogen}\\

\subsection{Uit: De mooiste verhalen}

\haiku{Dat is niet duur, voor,.}{zoveel historie als daar}{ligt opgetast}\\

\haiku{En dan mot 't snoer,,.}{in die kast ligge Wullem}{bij de ring van Hoen}\\

\haiku{{\textquotedblright} zei ze, {\textquotedblleft}want moe is...{\textquotedblright} -,?}{vannacht weggegaan Hoe vindt}{u zo-iets meneer}\\

\haiku{De onderwijzer.}{zesde klas keek het lijstje}{van de rollen na}\\

\haiku{Ze was ook dom, en.}{buiten de klas maakte ze}{geen uitzondering}\\

\haiku{Ik kleurde en dacht,.}{n\`og veel meer terwijl ik mijn}{mouw weer neerstroopte}\\

\haiku{{\textquoteleft}Noblesse oblige,{\textquoteright},.}{want we leerden al Frans en}{verstonden dit reeds}\\

\haiku{En we hebben nooit.}{weer iets geks van Elvira}{gehoord of gezien}\\

\haiku{Nee, het hondje had -,.}{haar hart maar toch niet zo als}{dat van Enze}\\

\haiku{Maar wie moesten ze nog? '.}{vragens Nachts luisterden}{ze tot in hun slaap}\\

\haiku{{\textquoteleft}Al zal ik d'r aan,{\textquoteright}, {\textquoteleft}!}{dood gaan zei hijik rij de}{hele weg terug}\\

\haiku{Ze kwam haastig naar,:}{binnen want ze begreep zijn}{hulpeloosheid niet}\\

\haiku{Gert heeft er wakker,.}{van gelegen want hij was}{opeens geen mens meer}\\

\haiku{Om dit waanbeeld te,:}{temmen heeft hij op een dag}{tegen Jans gezegd}\\

\haiku{Nou ja,{\textquoteright} antwoordde, {\textquoteleft},!}{Ganterdat moet je geheim}{houden vrouw Blokker}\\

\haiku{Hij kon nog geen drie.}{minuten in het water}{hebben gelegen}\\

\haiku{E\'en drager drukte,.}{op zijn borst om het water}{eruit te pressen}\\

\haiku{{\textquoteleft}Heb niet het hart, die,{\textquoteright}.}{ketting ook maar \'e\'en dag af}{te leggen zei Gert}\\

\haiku{Daar was ze ook even,.}{beteuterd van en ze wist}{zo gauw geen antwoord}\\

\haiku{En - misschien was het,...}{ook alleen medelijden}{wat haar zo diep trof}\\

\haiku{en ik ben zo bang,{\textquoteleft} -}{dat je n zult denken dat}{ik medelijden}\\

\haiku{De dokters zeggen,,}{dat ik de eerste tijd niet}{zal kunnen lopen}\\

\haiku{Kees waardeerde zijn,.}{vrouwtje bizonder en zij}{schatte hem ook hoog}\\

\haiku{Zo kwam hun laatste,.}{avond aan de C\^ote d'Azur}{en Hetty was stil}\\

\haiku{Ze waren altijd,.}{netjes gekleed en beleefd}{tot in de puntjes}\\

\haiku{maar Gijbert kwam met,.}{een waterpistool en schoot}{de vent in zijn ogen}\\

\haiku{Op een dag - hij was -.}{toen al zevenentwintig}{werd hij heel anders}\\

\haiku{Die borstelige -.}{wenkbrauwen en de rechte}{mond hij kende ze}\\

\haiku{Duidelijk zicht op,.}{het orkest een aardig punt}{in de zaalruimte}\\

\haiku{Bella was enige.}{tijd een veelgeziene gast}{bij hen thuis geweest}\\

\haiku{Ja, zo had hij het,.}{als kind genoemd wanneer hij}{naar die handen keek}\\

\haiku{En ze legde een.}{eerbiedige hand op haar}{eigen naveltje}\\

\haiku{Ze begon zo zacht,:}{te praten en opeens leek}{alles ijl aan haar}\\

\haiku{Ze voelde alleen,.}{heel sterk dat grote mensen}{schrokken van doodgaan}\\

\haiku{Ze kon er 's avonds,.}{bijna niet van slapen en}{kreeg een nachtlichtje}\\

\haiku{Bij de school stoeiden.}{de kinderen joelend en}{schreeuwend door elkaar}\\

\haiku{De knechten wisten:}{het hem niet te melden en}{de vrouwen schreiden}\\

\haiku{Zo kwam hij op een,.}{dag terug te keren naar}{de keizer Karel}\\

\haiku{{\textquoteleft}Zal ik vergaan van?}{smartelijke dorst naar de}{meren van haar ogen}\\

\haiku{Ik heb palmwijn voor,{\textquoteright}.}{je gesneden vleide de}{zon zijn beminde}\\

\haiku{Een enkele maal.}{echter werd haar verlangen}{sterker dan haar angst}\\

\haiku{{\textquoteright} En dan zou de zon,.}{weer weten dat de maan hem}{niet had vergeten}\\

\haiku{Er was geen wanklank,.}{of zorg tussen ons dan dat}{zijn bezit klein was}\\

\haiku{Haar levenloze.}{ogen waren wijd-open van}{onduldbaar lijden}\\

\haiku{Als de dood haar   -...}{had betoverd zodat zij}{hem niet meer liefhad}\\

\haiku{En in Kwangsi zou,.}{hij zijn waren verkopen}{en goud ontvangen}\\

\haiku{Hij kon van niemand,.}{anders  meer dromen dan}{van deze schone}\\

\haiku{{\textquoteright} Wellicht had Uma hem,;}{willen zeggen dat zij h\`em}{toch had gekozen}\\

\haiku{{\textquoteleft}Weet jij nog, wat je,?}{wilde overdenken toen ik}{was weggezonden}\\

\haiku{Het was een soort dolk,,.}{in vreemde golven gesmeed}{met een scherpe punt}\\

\haiku{Op een dag zag hij;}{een bouwmeester de stand van}{pilaren meten}\\

\haiku{Ze verlangden eten,.}{en waren zeer vriendelijk}{voor de jonge vrouw}\\

\haiku{De jongelingen.}{aten geconfijte vruchten}{en zij dronken wijn}\\

\haiku{Dan word je prachtig,}{donkerrood en je bevat}{vitamine C.}\\

\haiku{De kolonel doet -:}{zelf open en dat is een klap}{op Van Bommels hart}\\

\haiku{{\textquoteleft}Het is Serafien,{\textquoteright}.}{verbeterde een van de}{andere boeren}\\

\haiku{En nu is het laatst,.}{gebeurd dat Annabartje naar}{de bushalte liep}\\

\haiku{Hij vloog scherend laag.}{over  de grond en ging op}{het hekje zitten}\\

\haiku{Maar van het raam uit,.}{kon ze zien dat de merel}{van de beschuit at}\\

\haiku{En toen sloegen de,.}{twee klokken allebei vijf}{want zo laat was het}\\

\haiku{Het parelhoen moest.}{deze inzichten winnen}{in zijn eenzaamheid}\\

\haiku{{\textquoteleft}Vriend,{\textquoteright} zei ze stil, {\textquoteleft}ik, -}{weet niet wat er is gebeurd}{ver-weg klonk}\\

\haiku{Hij schaamde zich, dat.}{zo'n doodgewone plant een}{hekel aan hem had}\\

\haiku{Hij hoorde er iets,.}{vervelends in alsof dat}{een wonder zou zijn}\\

\haiku{Ze zouden me v\'o\'or.}{zonsondergang de ware}{wijsheid doen kiezen}\\

\haiku{Hij liep op stille.}{voeten tempelwaarts met zijn}{kostbare besluit}\\

\haiku{Op 't laatst zult u.}{niet meer geloven in pijn}{en slapeloosheid}\\

\haiku{{\textquoteright} Toontje zuchtte weer,.}{hij leek het gelach niet te}{hebben gewaardeerd}\\

\haiku{Ik meende aarde,,.}{scherven kralen en schatten}{te horen lachen}\\

\haiku{{\textquoteright} Maar reeds bukte hij,.}{nogmaals en wroette kort en}{fel in de kluiten}\\

\haiku{Er was duidelijk,:}{verschil in kleur en glazuur}{hij toonde het ons}\\

\haiku{Hij keek naar de agent,.}{die rookte en oplettend}{naar de lamp staarde}\\

\haiku{Ik heb er nooit pijn,.}{van gehad behalve de}{laatste twee dagen}\\

\haiku{Hun huwelijk was.}{nu niet bepaald een vorm van}{rose harmonie}\\

\haiku{Kees ging veel naar de.}{soci\"eteit en Betty}{had negen clubjes}\\

\haiku{{\textquoteleft}Ik zal een doosje,.}{uit mijn tas halen dan kun}{je hem meenemen}\\

\haiku{Daar was hij dus zelf,.}{met een stekende vierde}{rib als verklaring}\\

\haiku{{\textquoteright} Hij probeerde te,.}{glimlachen maar zijn vierde}{rib stak te hevig}\\

\haiku{er was meer wit in,.}{een schrille glimmer in zijn}{goedig zwart gezicht}\\

\haiku{{\textquoteleft}Het is kennelijk,{\textquoteright}.}{een intoxicatie stelde}{de professor vast}\\

\haiku{Langs de weg zitten,.}{was ook geen genoegen daar}{groeiden brandnetels}\\

\haiku{Mar die van jou, die,,{\textquoteright}}{het alles opgespaard toen}{jij most komme bitste}\\

\haiku{De inspecteur wou,.}{nog niet maar kon een grinnik}{ook niet beheersen}\\

\haiku{maar de kundigheid.}{van zijn huisdokter blijkt daar}{niet minder om}\\

\haiku{De pan stond treurig,.}{verlaten terzijde het}{bord was leeg en koud}\\

\haiku{Het was vreselijk -,?...}{zou hij dan toch een beetje}{geschokt zijn door iets}\\

\haiku{{\textquoteright} informeerde ze.}{nonchalant en schudde haar}{veren nog es op}\\

\haiku{Toen had hij haar te.}{pakken en beet haar met \'e\'en}{hap de kop af}\\

\haiku{Hij had haar toch tot,;}{het laatste toe gegeven}{wat ze verwachtte}\\

\haiku{Afscheid nemen van,.}{opa en oma vermaningen}{en groeten innen}\\

\haiku{Mamma bromde dat,.}{het een vieze lap was van}{een onbekende}\\

\haiku{{\textquoteright} zegt ze, {\textquoteleft}je weet 'r ',...!}{toch marn krui{\"\i}g broodje}{van te bakken h\`e}\\

\haiku{{\textquoteleft}Tante Merlina,{\textquoteright}, {\textquoteleft}.}{wees ik haar kies terechteen}{heks is geen dame}\\

\haiku{{\textquoteleft}Wat ben je toch een,!...}{afschuwelijk mens je bent}{precies je moeder}\\

\subsection{Uit: De porselein tafel}

\haiku{Ik ben blij, dat er.}{zoveel vrouwen in onze}{familie waren}\\

\haiku{De nieuwe woning,.}{lag buiten de Waterpoort}{aan de Lemsterweg}\\

\haiku{Alberdien moest het,.}{hoofd koel houden daar Speyer}{lang niet goedkoop bleek}\\

\haiku{Van het eens grote.}{gezin waren slechts twee broers}{en een zuster over}\\

\haiku{want Orne had de,.}{doek om zijn buik gewonden}{tussen broek en jas}\\

\haiku{zij was bevreesd voor,.}{verwijdering juist nu ze}{het derde kind droeg}\\

\haiku{Zij riep het kind bij,,.}{zich en vroeg hem ernstig wat}{hij daar gedaan had}\\

\haiku{Daar kon Orne maar,.}{nauwelijks om lachen want}{de mop was niet fijn}\\

\haiku{Was het leven dan?}{niets anders dan geboren}{worden en sterven}\\

\haiku{Hij vond haar om half.}{elf met kloppende hoofdpijn}{in de woonkamer}\\

\haiku{Nicht Pietje geleek.}{eensklaps een olielichtje op}{de laatste druppels}\\

\haiku{Op de weg naar huis.}{werd het rijtuigje omgonsd}{door de oostenwind}\\

\haiku{Deze, welke zij,;}{niet meer nodig had vond ze}{in een hoedendoos}\\

\haiku{En drie kommetjes,,.}{van groen net erwtensoep met}{draken buitenop}\\

\haiku{Er waren zelfs tot.}{in Leeuwarden verhalen}{over doorgedrongen}\\

\haiku{De avond verviel tot,;}{nacht de duisternis sloot haar}{hand om de ramen}\\

\haiku{Ze had het vreemde, - -...}{gevoel dit verdiend bijna}{verwacht te hebben}\\

\haiku{idem 1938 - De laars op- - -.}{de nek 19391944 Jozef duikt}{1946 De knopenman}\\

\haiku{TONNY VAN DER HORST,.}{Ik schreef een Kerstverhaal voor}{jonge meisjes 1946}\\

\subsection{Uit: Probleem in Aerdenberg}

\haiku{{\textquoteleft}Ja,{\textquoteright} zei Terry, die,.}{sedert hij detective}{is recht door zee gaat}\\

\haiku{Zelfs niet \'e\'en, want dan.}{zou Netty Brand dadelijk}{gealarmeerd zijn}\\

\haiku{Pas op, dat je kop,{\textquoteright}.}{d,'r niet af valt waarschuwde}{Terry sarcastisch}\\

\haiku{Ik heb mijn leven,{\textquoteright}.}{ingedeeld in kwartieren}{legde Terry uit}\\

\haiku{Wij werden volgens.}{afspraak bekend gemaakt als}{vrienden van Henri}\\

\haiku{Doch deze haalde,.}{de schouders op en wees op}{de commissaris}\\

\haiku{Maar is er dan geen?}{sprake geweest van n\`og een}{ruzie of zoiets}\\

\haiku{Toen hij weg was, keek,:}{Burgheem ons met een vage}{glimlach aan en zei}\\

\haiku{We beloofden het.}{woordelijk namens haar over}{te zullen brengen}\\

\haiku{Hij keek eens naar ons,:}{via het spiegeltje bij de}{voorruit en lachte}\\

\haiku{{\textquoteleft}Maar uw spreektrant zal.}{deze gevoelens zeker}{niet versterkt hebben}\\

\haiku{Hij voelde aan de,....}{toetsen en streek over de kast}{klopte op het hout}\\

\haiku{Hij tilde de sprei,.}{op en bezag het gladde}{lak met een loupe}\\

\haiku{{\textquoteright} {\textquoteleft}En -,{\textquoteright} hernam Terry, {\textquoteleft},....{\textquoteright} {\textquoteleft},.}{kiesexcuseer de vraag Gert}{Tot uw dienst meneer}\\

\haiku{En die keer dasse?!}{zo'n herrie kreeg met meneer}{Martijn over die pook}\\

\haiku{Veel erger, dan ik.}{van haar verwacht had en het}{verbaasde me zeer}\\

\haiku{{\textquoteright} vroeg ze glimlachend,.}{terwijl ze haar voet op de}{eerste tree zette}\\

\haiku{Het klonk spookachtig,,.}{en ik schaam me niet om te}{zeggen dat ik schrok}\\

\haiku{{\textquoteright} {\textquoteleft}Als ik het wist, Han,,{\textquoteright}.}{zou ik het je graag zeggen}{gaf hij ten antwoord}\\

\haiku{Dan had het in mijn.... {\textquoteleft}}{eigen verschijning w\`el zo}{aardig kunnen zijn}\\

\haiku{Het galmde kolkend,.}{langs de holle gangen in}{daverende echo's}\\

\haiku{Haalde achter de:}{rug van de sidderende}{vrouw zijn schouders op}\\

\haiku{Ik kwam naast hem staan:}{en luisterde vol spanning}{naar zijn conclusie}\\

\haiku{{\textquoteleft}Stel, dat Booner een klein, -{\textquoteright}.}{teder plekje heeft in haar}{hart Terry lachte}\\

\haiku{Want dat verwachtte,?}{je zeker toen je over iets}{griezeligs praatte}\\

\haiku{Ik heb rechercheurs,!}{gekend die bijna net zo}{slim waren als ik}\\

\haiku{Ik zag, dat Terry,:}{de vuisten balde maar hij}{zei beminnelijk}\\

\haiku{Om dan zodoende,?....}{aan Gert te verraden dat}{ze van Henri hield}\\

\haiku{Ik zou in staat zijn,.}{het je te vertellen van}{louter opwinding}\\

\haiku{en toen is 'ie naar....,....}{meneer Martijn gegaan om}{voorschot te vrage}\\

\haiku{{\textquoteright} {\textquoteleft}Dat is niet in een,{\textquoteright}.}{vloek en een zucht uitgelegd}{antwoordde Terry}\\

\haiku{Ik kan niet zeggen,.}{of het haat was of woede}{of vernedering}\\

\haiku{Van mij zul je je,.}{geld moeten afwachten tot}{je eraan toe bent}\\

\haiku{Menters stond op en.}{ging naar buiten om een paar}{agenten te roepen}\\

\haiku{{\textquoteleft}Ben je met meneer,?}{Martijn in de kamer van}{zijn broer geweest Gert}\\

\haiku{{\textquoteleft}Nou moet u zich niets,.}{laten vertellen door de}{dorpsmensen meneer}\\

\haiku{{\textquoteleft}En dan zul je juist....}{buiten de gemeente een}{dwarsweg links vinden}\\

\haiku{Je herkent 'm niet,.}{dus kijk hem ook niet te erg}{aan in het donker}\\

\haiku{Waarschijnlijk zal hij,.}{je willen binden en dat}{mag je niet toestaan}\\

\haiku{om me verslag uit,.}{te brengen en dan ga je}{naar dat afspraakje}\\

\haiku{De lange man naast.}{me maakte een beweging}{van verbazing}\\

\haiku{{\textquoteleft}Het had maar weinig,!....}{gescheeld of ik had je niet}{weer levend gezien}\\

\haiku{In plaats daarvan keek,.}{hij naar de grauwe hemel}{buiten en kuchte}\\

\haiku{{\textquoteleft}Ik wou u alleen,.}{maar even plagen om u wat}{op te monteren}\\

\haiku{Die ruzie met m'n....}{broer zat me eerlijk gezegd}{nog zo in het hoofd}\\

\haiku{Ik mocht hem graag om.}{de manier waarop hij zijn}{onkunde beleed}\\

\haiku{{\textquoteleft}En draait die nou ook, '?!}{de bak in omdat ik wat}{vanm verteld heb}\\

\haiku{Wie weet, hoe hij ons!}{en een gesprek met ons van}{zijn kant schilderde}\\

\haiku{Als ze wisten, hoe, -,}{verdacht ze zichzelf maken}{dat is juffrouw Booner}\\

\haiku{- Ik wou, dat je je,.}{eigen gezicht zo nu en}{dan eens kon zien Han}\\

\haiku{- Hier zat ik nou, met,....}{veel werk zodat ik me niet}{kon wijden aan kunst}\\

\haiku{Ik stak op mijn dooie:}{gemak een sigaret aan}{en informeerde}\\

\haiku{Het was duidelijk,.}{dat de twee Van der Lindens}{lucht voor haar waren}\\

\haiku{- Toen boog Terry het,,.}{hoofd alsof hij nadacht hoe}{hij moest beginnen}\\

\haiku{Terry ging in de,.}{gang waar we hem met Crommer}{hoorden fluisteren}\\

\haiku{Het was precies de,....}{wond die Martijn van Doff aan}{de slaap had gehad}\\

\haiku{Met een auto is '.}{zes nachts weer in het dorp}{hier teruggekeerd}\\

\haiku{Het was natuurlijk,.}{de overzetting waarbij ik}{hem geholpen had}\\

\haiku{Op het ogenblik dat,:}{hij de eerste hand drukte}{zei Arlette koel}\\

\haiku{{\textquoteright} De jonge vrouw kon.}{een helle triomf in haar}{ogen niet beheersen}\\

\haiku{{\textquoteleft}Zie je wel, dat je?!}{geen schoon ondergoed hoefde}{te laten komen}\\

\haiku{{\textquoteleft}Wat ben je toch een,{\textquoteright}.}{afschuwelijk ondier zei}{Terry afkeurend}\\

\haiku{Hij keek sprakeloos.}{en gekrenkt voor zich uit en}{trok zijn lippen in}\\

\subsection{Uit: Spiegel aan de wand}

\haiku{Om half acht is hij.}{de vrouwen voorbijgegaan}{met een korte groet}\\

\haiku{tot  de ouwe.}{Berrends op een ochtend stierf}{aan hartverlamming}\\

\haiku{Frank ontdekt er steeds,.}{weer een diepte in waarin}{hij betoverd staart}\\

\haiku{Voor een man is er,.}{toch altijd nog het werk dat}{hem interesseert}\\

\haiku{De tweede dag heeft {\textquoteleft}{\textquoteright},.}{zeverrek gezegd midden}{in de rose zaal}\\

\haiku{Er is geen mooier,,.}{huwelijk denkbaar zelfs nu}{nog dan het hunne}\\

\haiku{Ze dacht, dat hij maar...,?}{\'e\'en bewonderaarster had}{Pardon madame}\\

\haiku{De zaak zal haar niet, '....}{meer kunnen  gebruiken}{alst zichtbaar wordt}\\

\haiku{{\textquoteleft}Ik ben toch - {\textquoteright} ze zweeg, {\textquoteleft},{\textquoteright}.}{een fatsoenlijke vrouw had}{ze willen zeggen}\\

\haiku{Het lijkt wel, of er.}{een snelle trilling door haar}{hele lichaam gaat}\\

\haiku{Fr\"oken Hannsen.}{lacht en neemt een nieuw blaadje}{complexion paper}\\

\haiku{Ik voor mij vind het..{\textquoteright} {\textquoteleft},?}{geen aanbeveling voor de}{zaakPardon mevrouw}\\

\haiku{Iedereen weet, dat {\textquoteleft}{\textquoteright}.}{hetBella Monica geen}{filialen heeft}\\

\haiku{Haar japon zit van;}{achteren net zo volmaakt}{als van  voren}\\

\haiku{En zo zet Helen.}{Carfew haar snoepzuchtige}{man op dieet}\\

\haiku{Juffrouw Liza bijt.}{op haar lip en vervloekt in}{stilte Mr. Carfew}\\

\haiku{Haar gezicht straalt zo,.}{op dat de afkeuring in}{Frank's ogen wegsmelt}\\

\haiku{Om die v\`ent, die haar.}{de glorie van het nieuwtje}{zo doodleuk ontneemt}\\

\haiku{Knikt, voor het oog der,.}{wereld en schrijdt regelrecht}{naar cabine 11}\\

\haiku{- Er moet natuurlijk.}{een antwoord komen van de}{ondergeschikte}\\

\haiku{Ik heb haar een paar.}{dagen geleden gezien}{met een jongeman}\\

\haiku{Mademoiselle:}{dient eigenlijk alleen maar}{als chaperonne}\\

\haiku{{\textquoteleft}Alle kleuren zijn,{\textquoteright}.}{bij uw gelaat en type}{aangepast zegt Frank}\\

\haiku{{\textquoteleft}Levendig en diep,.}{zonder onaangenaam in}{het oog te lopen}\\

\haiku{de kleur is alleen.}{een sterkere nuance}{van die van de lait}\\

\haiku{- O, ja! 't Maakt je.}{oren aan het gloeien en je}{hart aan het kloppen}\\

\haiku{Hij hoeft haar nooit iets,.}{te vertellen en zij zegt}{ook bijna nooit wat}\\

\haiku{Maar het is een lang.}{slank meisje met blond haar en}{een ernstig gezicht}\\

\haiku{De rouge ligt als.}{een los poeierig waas op}{de jukbeenderen}\\

\haiku{Maar dan zou hij toch.}{niet zo snel zijn hart aan een}{ander verliezen}\\

\haiku{Onnozel houdt ze.}{het dampende kopje bij}{haar bevende mond}\\

\haiku{Dag juffrouw,{\textquoteright} zegt ze,.}{vriendelijk en toch niet al}{te gemoedelijk}\\

\haiku{Dag mevrouw,{\textquoteright} antwoordt:}{het meisje en grijpt meteen}{de huis telefoon}\\

\haiku{Dan zinkt ze met een.}{sidderende zucht neer in}{de behandelsfoel}\\

\haiku{{\textquoteright} En als je 't zelf,?}{hoort moet je haast lachen want}{dat k\`an immers niet}\\

\haiku{op een gegeven:}{ogenblik legde hij zijn hand}{over de hare heen}\\

\haiku{Het overhemd was van,.}{zo dunne zijde dat het}{de huid zelf geleek}\\

\haiku{Of was ze alleen?}{maar niet zo geraffineerd}{als de anderen}\\

\haiku{Als hij zijn gezicht,,,.}{dichtbij het hare opzij}{wendt herkent Bob hem}\\

\haiku{Mimi's stem klinkt hoog.}{en haar lachjes zijn een snoer}{kristallen kralen}\\

\haiku{Is er niet een tram,?}{of een bus of iets waarmee}{hij naar Ina kan gaan}\\

\haiku{ze vindt het zelf niet,.}{aangenaam want ze is er}{niet rustig onder}\\

\haiku{- Frank bedenkt, dat hij.}{niets van haar leven buiten}{het instituut weet}\\

\haiku{Is het geen teken,?}{dat hij ontevreden is}{over haar uiterlijk}\\

\haiku{Ze kijkt aldoor in.}{de diepte van haar rijkdom}{en dan duizelt ze}\\

\haiku{Juffrouw Diller neemt.}{een abonnement voor twintig}{behandelingen}\\

\haiku{dan was die heer toch.}{nooit een ondergeschikte}{van een der dames}\\

\haiku{Het pijnlijkste vindt,.}{de Barones echter dat}{Jo toen kribbig werd}\\

\haiku{De vrolijkheid op.}{Johanna's gezicht stolt tot}{een kille hardheid}\\

\haiku{Achter Johanna.}{schrijft de assistente snel}{het rapport even bij}\\

\haiku{Het doet er niet toe,,.}{hoe ze hem zal begroeten}{of helemaal niet}\\

\haiku{Voor de huidvoeding, {\textquoteleft}{\textquoteright}:}{koos ze net als elke vrouw}{allespecials}\\

\haiku{Als zodanig is;}{Maria's dood-zijn ook}{onbegrijpelijk}\\

\haiku{Ze vindt het opeens,.}{grof van zichzelf om zo vroeg}{gekomen te zijn}\\

\haiku{{\textquoteright} De blonde en de:}{bruine schateren en de}{eerste informeert}\\

\haiku{Ze komt uit een land,;}{waar de vrouw minstens evenveel}{waard is als een man}\\

\haiku{{\textquoteleft}Wie iets niet begrijpt,.}{of ergens over spreken wil}{moet bij mij komen}\\

\haiku{Meneer Geerts is de.}{oprichter van de zaak en}{ook de bezitter}\\

\haiku{De assistente.}{laat zich er dan ook niet door}{van de wijs brengen}\\

\haiku{{\textquoteleft}Ik heb nog nooit een,.}{assistente ontmoet die}{zittend kon werken}\\

\haiku{{\textquoteright} {\textquoteleft}Oh,{\textquoteright} miss Jean wiegt op, {\textquoteleft}?}{haar kruk van bevreemdingmaakt}{that a verskil}\\

\haiku{Hierop slaat Frank zijn:}{ogen neer en zijn stem breekt het}{vragend zwijgen open}\\

\haiku{{\textquoteleft}En dat ik me niet,....}{vergis bemerk ik uit de}{woorden van die vrouw}\\

\haiku{En juist omdat ze,,.}{nu doorv\'o\'elt wat hij meent kan}{ze hem niet troosten}\\

\haiku{{\textquoteleft}En als je je plaats,.}{verlaat dan kijk {\`\i}k je niet}{meer aan Frank Berrends}\\

\haiku{Zij neeg heel even het.}{hoofd en richtte het toen wel}{tweemaal zo hoog op}\\

\haiku{De mensen zijn wreed,....}{en wellicht ziet het er erg}{onsmakelijk uit}\\

\haiku{we mogen zelf bij, {\textquotedblleft}{\textquotedblright} {\textquotedblleft}{\textquotedblright}.}{alles beslissen of het}{ja ofnee zal zijn}\\

\haiku{Dan komt er eensklaps.}{een grote geldkrapte in}{zijn zakenleven}\\

\haiku{{\textquoteright} {\textquoteleft}De handel behoort,{\textquoteright}.}{een schaakspel te zijn zegt Geerts}{met trillende stem}\\

\haiku{{\textquoteright} {\textquoteleft}Meestal,{\textquoteright} antwoordt,.}{Frank onbewogen en schenkt}{een borrel in}\\

\haiku{Dan is er nog een,:}{wonderlijk huis dat vele}{aanbouwsels vertoont}\\

\haiku{of alleen maar het:}{plotselinge ervaren}{van de romantiek}\\

\subsection{Uit: Ter ere van}

\haiku{Vort, voortrennen - we,!}{moeten nog negentienmaal}{het dorp rond vandaag}\\

\haiku{{\textquoteleft}Zoek dan tenminste,!...}{een mooi medailletje uit}{of draag iets \`anders}\\

\haiku{Hij slurpte beschaafd.}{zijn koffie en keek over de}{rand van het kopje}\\

\haiku{- Ze schudden Ron om,.}{beurten de hand terwijl ze}{naar de deur gingen}\\

\haiku{Als d'r es 'n wijf,.}{opkomt knikt ze mee d'ren}{kop en goait weer}\\

\haiku{Pak dus je koffer,!}{ik kom oe vanmiddag om}{drie ure hoalen}\\

\haiku{{\textquoteright} zo voldaan, alsof.}{hij ze allemaal met de}{hand had gevangen}\\

\haiku{en twee dagen na:}{de begrafenis zei Huub}{tegen zijn moeder}\\

\haiku{Zo kwam hij op het.}{idee dansles te gaan nemen}{in het naaste dorp}\\

\haiku{Spanjaarts richtte zijn:}{onrustige oogjes op}{hem en vroeg lijnrecht}\\

\haiku{Langs alle kanten.}{werden matten geklopt en}{ramen gewassen}\\

\haiku{Zijn ogen waren in.}{de haven van de nis tot}{stilstand gekomen}\\

\haiku{De betovering;}{omtrok hem met een wand van}{stille melodiek}\\

\haiku{{\textquoteleft}Ach, denkte, d\`e'k?}{nie meer kan spitten omdat}{ik zeventig ben}\\

\haiku{moar het wordt toch tijd,, ' '!}{vind ik datr n\'o\'u esn}{Gemeenschapshuis komt}\\

\haiku{Hij groette met de,:}{anderen doch \'e\'en zuster}{wendde zich tot hem}\\

\haiku{{\textquoteleft}Excuseert u, dat,{\textquoteright},.}{ik eens even de benen strek}{zei hij en stond op}\\

\haiku{{\textquoteleft}Het spijt me, dat Rieks,{\textquoteright}.}{van Brugge ziek is bracht Ron}{bezorgd naar voren}\\

\haiku{Maar toen had hij stil,.}{gestaan om te bedenken}{waar hij moest strooien}\\

\haiku{Meneer Pastoor,{\textquoteright} zei, {\textquoteleft} -,?...}{hij bevendwoar is oew oew}{barmhartigheid man}\\

\haiku{As die hier nou es -,?}{verstrooid wou zijn w\`e zoude}{dan zeggen pastoor}\\

\haiku{Pastoor had ze hoog,.}{geroemd en God geprezen}{voor zulk schoon schepwerk}\\

\haiku{Maar Huub was hem recht.}{blijven aankijken met zijn}{allerzwartste ogen}\\

\haiku{Slechts \'e\'en jongen die,.}{hij niet eerder had gezien}{was ongedurig}\\

\haiku{Buiten deinde het;}{bolle wezen van Spanjaarts}{voorbij het venster}\\

\haiku{Diejen knol kan nog -}{genen wind loaten of}{hij zegt Weesgegroet}\\

\haiku{Hier is oew koffie,,{\textquoteright}, {\textquoteleft} '?}{Spanjaarts zei Annet zorgzaam}{wilden kuuksken}\\

\haiku{{\textquoteleft}Ik geloof, dat God, '!}{h\'e\'el gelukkig was de dag}{datie jou maakte}\\

\haiku{Nu meende Ron hem.}{met tact zijn verwondering}{te mogen tonen}\\

\haiku{En die kleine van -,!}{Monders d\`e was nen mesken}{van zes nen nichtjen}\\

\haiku{Pietje Monders mocht.}{niet vaster in de zadel}{zitten dan hij}\\

\haiku{En Jef ontzag zich,.}{niet het Mariaspel weer naar}{voren te schuiven}\\

\haiku{Dat was nu eenmaal,.}{een soort kreet van hem waarmee}{hij altijd goed zat}\\

\haiku{Daar wil ik liggen,,}{zodat mijn oog net zo hoog}{is als de bloemen}\\

\haiku{{\textquoteright} en ze vertelde.}{de belevenis aan Huub}{en de kinderen}\\

\haiku{Ron had een gevoel,,.}{of hij stikte terwijl hij}{zijn hand uitstrekte}\\

\haiku{Groter was Rieks in -.}{zijn beheerstheid kleuriger}{Martien in zijn wil}\\

\haiku{De machine sleet,.}{bij de dag ongewend en}{verroest als ze was}\\

\haiku{{\textquoteright} Ron vestigde hun,.}{aandacht op de prachtige}{grote veldbloemen}\\

\haiku{En zo kwam Ron weer,:}{te dwalen die middag langs}{Akkerweg en Stroopad}\\

\haiku{t is gauw zo ver.{\textquoteright} {\textquoteleft}{\textquoteright}.}{Maar bijna niemand kent zijn}{rol barstte Ron uit}\\

\haiku{{\textquoteright} zei ze bleekjes, want.}{daaraan had ze heel nare}{herinneringen}\\

\haiku{Gadverju, kerel,,!}{ge bent toch nen vent en geen}{gedoopte rochel}\\

\haiku{En ze houden oe,!}{allemoal teugen ge}{kumt nie oan den kop}\\

\haiku{De mooie vrouw, die je -.}{ongetwijfeld bent als je}{magerder zou zijn}\\

\haiku{Ben de Weyn was daar,.}{lid van hij wilde graag te}{paard opkomen}\\

\haiku{Met koesterende,:}{opwinding met triomf sprak}{Spanjaarts de woorden}\\

\haiku{{\textquoteright} {\textquoteleft}Daar geeft Rennevoirt,{\textquoteright}.}{nog geen enkel bewijs van}{repliceerde Ron}\\

\haiku{Hij klopte haar op.}{enig klinkend lichaamsdeel en}{was Ron vergeten}\\

\haiku{Want op de laatste.}{brief naar Amerika was geen}{antwoord meer gevolgd}\\

\haiku{{\textquoteleft}Het is z\'o dringend,{\textquoteright}, {\textquoteleft}!}{zei meneer Pastoorhet wordt}{bekant wijwater}\\

\haiku{en in de verte,;}{zong hun de klank van een zeis}{toe die gescherpt werd}\\

\haiku{{\textquoteright} schreeuwde Huub, langs het.}{schaterend gelach van de}{beide anderen}\\

\haiku{wat wist je weinig!...}{op wie je kon bouwen van}{ogenblik tot ogenblik}\\

\haiku{Tafels en stoelen.}{neerzetten en een spiegel}{in de grimeerhoek}\\

\haiku{Alle ketenen,,.}{de ringen en hangers de}{kronen en gespen}\\

\haiku{Ze zag zijn handen.}{voorzichtig borststukken en}{kappen gladstrijken}\\

\haiku{En gij, Ron, hedde?!}{\`alle gedachten achter}{oe geloaten}\\

\haiku{De akkers lagen.}{grijs en verlaten achter}{het mooie bouwwerkje}\\

\haiku{Het comit\'e zocht.}{mekander en wist van uur}{op uur geen uitkomst}\\

\haiku{Om zeven uur was.}{het betraand droog boven een}{versopte aarde}\\

\haiku{Maar het bleef droog en.}{het pleintje voor de kapel}{hernam zijn aanzicht}\\

\haiku{Hoe warm en direct,!}{waren ze gebleken en}{hoe raadselachtig}\\

\haiku{Buiten legde hij,.}{zijn hand tegen de muur waar}{de Dood had geleund}\\

\haiku{Pas toen hij het in,.}{de hand hield zag hij dat het}{een telegram was}\\

\haiku{Natuurlijk was hij -.}{woedend er zou altijd wel}{een aanleiding zijn}\\

\subsection{Uit: Versprongen ster}

\haiku{Er was geen regen,.}{meer alleen een aandachtig}{auditorium}\\

\haiku{Ze sprong beslist niet;}{zo technisch en zo mooi als}{haar lotgenote}\\

\haiku{De baas kon dat ook,,:}{zo vragen en als je dan}{een klacht had zei hij}\\

\haiku{{\textquoteright} {\textquoteleft}Nee,{\textquoteright} zei Betsie, met, {\textquoteleft}!}{haar tanden tegen het glas}{dat begrijpt u niet}\\

\haiku{{\textquoteleft}Ik word binnenkort,.}{vijftig dus een baan zult u}{mij niet aanbieden}\\

\haiku{Het is een tikje,...}{vergezocht en daardoor niet}{onverwacht genoeg}\\

\haiku{{\textquoteleft}Kom nou,{\textquoteright} zei hij, en, {\textquoteleft},!...}{streelde haar hulpeloze}{handkom nou Betsie}\\

\haiku{Zijn wangen waren,.}{hoogrood hij keek Betsie aan}{met ontzette ogen}\\

\haiku{Hij bleek opeens veel,.}{minder onaantastbaar dan}{hij ooit was geweest}\\

\haiku{Er hoeft maar een reep,.}{chocola in te zitten}{of een zakdoekje}\\

\haiku{En vervolgens bleek:}{de bakker een grote doos}{bezorgd te hebben}\\

\haiku{Ze had nu juist die,!}{dag zo veel gedachten die}{niet aardig waren}\\

\haiku{Daarom kon je juist,.}{zo gemakkelijk gek doen}{en eens wat zeggen}\\

\haiku{Ze herinnerde,.}{zich hoe ze voor het eerst naar}{een baan was gestapt}\\

\haiku{Of wilde de vrouw?}{van de directeur kennis}{maken met Betsie}\\

\haiku{Nog drie tellen, en.}{Alie zou de bevreemding een}{kop thee aanbieden}\\

\haiku{Maar die kanarie - -...}{lieve Cor en Aal die moet}{toch nog even wachten}\\

\haiku{Een jongeman in.}{een groen satijnen broek stond}{midden in de zaal}\\

\haiku{Betsie struikelde.}{door de stoot in haar rug op}{het gezelschap toe}\\

\haiku{Ze beschouwen je,.}{als een excentrieke ster}{die mij intiem kent}\\

\haiku{Stephans moest spelen,;}{was een donkere vrouw van}{een jaar of vijftig}\\

\haiku{Ze stond daar, en wist,.}{niet eens wat ze eigenlijk}{precies gezegd had}\\

\haiku{{\textquoteleft}dan zal ik er een,.}{canap\'e van maken met}{de rug naar je toe}\\

\haiku{Betsie van de Pen.}{werd langzaam maar heel zeker}{Betty Vandepen}\\

\haiku{Ik heb 's morgens,...}{altijd een basstem dat is}{nog uit mijn diensttijd}\\

\haiku{Hemel, ja, een mens!...}{kan toch niet altijd rijk en}{deftig geweest zijn}\\

\haiku{naar haar kleedkamer,,.}{gooide de bontcape af}{trok de japon los}\\

\haiku{Veel sneller dan ze,.}{gedacht had  stond ze weer}{op het podium}\\

\haiku{Slechts langzaam drong het,.}{tot Betsie door dat zij de}{ouwe werkster was}\\

\haiku{Loretta was na.}{de premi\`ere niet meer}{terug gekomen}\\

\haiku{Je ziet er uit naar,{\textquoteright}.}{geldzorgen constateerde}{Betsie moederlijk}\\

\haiku{{\textquoteleft}Toen ik je voor het -{\textquoteright} {\textquoteleft}-,{\textquoteright}.}{eerst zag in de Kiekenstraat}{vulde Betsie aan}\\

\haiku{Hij deed een snelle,.}{stap hogerop en stond \'e\'en}{trede onder haar}\\

\haiku{Ze durfde zich te,.}{laten gelden en was veel}{zelfverzekerder}\\

\haiku{Alie voorzag verdriet;}{en ontnuchtering voor haar}{jongere zusje}\\

\haiku{{\textquoteright} Betsie stond stil en,.}{vroeg zich ijlings af wat hij}{gehoord kon hebben}\\

\haiku{Ze zat opeens in,.}{een luchtledig en begreep}{de mopjes maar half}\\

\haiku{{\textquoteleft}Dat hij nog met haar,!}{wilde praten na alles}{wat hij van haar wist}\\

\haiku{{\textquoteright} Een jonge kerel}{met een fluwelen jasje}{en hoog opgekamd}\\

\haiku{Overigens was het,.}{geen plezier de kamer te}{delen met Lyra}\\

\haiku{Toen keek ze in de.}{al te twinkelende ogen}{van haar directeur}\\

\haiku{Ze had eens zo'n soort,.}{mop gelezen maar wist die}{toch niet precies meer}\\

\haiku{{\textquoteleft}Voor de vrouw, die zo...{\textquoteright},.}{prachtig op de vleugel springt}{stond er op het kaartje}\\

\haiku{Ze boog stralend en,.}{glimlachend naar de zaal die}{klapte en stampte}\\

\haiku{{\textquoteright} zei het jongetje,.}{lichtelijk verbaasd dat zo'n}{mevrouw d\`at niet wist}\\

\haiku{Wat wist je weinig!}{van je medemensen en}{van hun gevoelens}\\

\haiku{{\textquoteright} Dat werd gehoord door,.}{een kwieke oudere heer}{die voor hen  liep}\\

\haiku{{\textquoteleft}Ik zou geen moment,.}{geloofd hebben dat je zo'n}{feestpet kon dragen}\\

\haiku{Ze keken op een,.}{vreemde verwachtingsvolle}{manier naar Betsie}\\

\haiku{{\textquoteleft}Mijn zuster had eens,,{\textquoteright}.}{een strijkijzer dat onder}{stroom stond zei ze toen}\\

\haiku{{\textquoteright} {\textquoteleft}O,{\textquoteright} stelde de man, {\textquoteleft}!}{haar geweldig ongerust}{het is z\'o gebeurd}\\

\haiku{Ze stond erop met,.}{een wijdopen mond en een been}{als een voorhamer}\\

\haiku{En nog was ze zo,.}{vermusicald dat ze er}{een ballet in zag}\\

\haiku{Ze liep naar boven,.}{en zette haar koffer met}{een bons op de grond}\\

\haiku{Het ergste was, dat,.}{ze niet wist wat ze doen moest}{toen het eten op was}\\

\haiku{Maar wie gaat er nu,?}{ook voor het raam zitten als}{ze zo bekend is}\\

\haiku{Op die manier moet,!...}{iedereen wel denken dat}{je in de film speelt}\\

\haiku{{\textquoteleft}Maar dat mag voor hen,.}{geen  aanleiding zijn je}{te belemmeren}\\

\haiku{{\textquoteleft}Ik breng u hier het,.}{meest spierwitte schaap thuis dat}{u zich denken kunt}\\

\haiku{Het was n\'a\'ar, dat ze!...}{zo met bedrog een geschenk}{lekker moest maken}\\

\haiku{{\textquoteleft}Ik heb laatst iemand,...}{ontmoet die zei dat hij haar}{kende van vroeger}\\

\haiku{Nu komt het,{\textquoteright} dacht ze,.}{wraakbelust en volgde hem}{zonder enig gerucht}\\

\haiku{In de gang maakte,:}{hij hoorbare voetstappen}{en praatte met haar}\\

\haiku{Een misselijke,.}{vraag die Betsie het bloed naar}{de hersenen dreef}\\

\haiku{de mensen in de.}{zaal kenden en ontzagen}{mekander te zeer}\\

\haiku{{\textquoteleft}Ik heb dat arme,{\textquoteright}.}{kind een glaasje tonicum}{gegeven zei hij}\\

\haiku{Vreemd, vond Betsie, dat,...}{hij zo'n geveinsde indruk}{maakte nu en dan}\\

\haiku{Ze hoorde toch wel,}{bij deze troep ze hoorde}{er inderdaad bij}\\

\haiku{Nog daargelaten -:.}{of ze het k\'on ze werd er}{niet voor geroepen}\\

\haiku{Wat hadden zij toch,!}{een malle simpele kijk}{op deze dingen}\\

\haiku{{\textquoteright} informeerde Alie,.}{later gebiologeerd}{door de gedachte}\\

\haiku{{\textquoteright} En Sandor, over de,.}{toetsen gebogen begon}{weer te spelen}\\

\haiku{Je springt in elk stuk,,{\textquoteright}, {\textquoteleft}!}{dat je speelt vertelde Thea}{al word je honderd}\\

\haiku{Het publiek is er,,!}{dol op en \'e\'en sprong is toch}{ook niets zeg nou zelf}\\

\haiku{En toen zei Betsie,:}{eindelijk zacht wat ze al}{lang op het hart had}\\

\haiku{De mensen in de,.}{zaal hadden even gelachen}{terwijl ze opkwam}\\

\haiku{Doch Robbens zag met,.}{grote scherpte dat Lyra}{de dialoog speelde}\\

\haiku{Hij greep haar bij de,.}{arm en duwde haar in haar}{eigen kleedkamer}\\

\haiku{{\textquoteleft}Dat is het bij mij...{\textquoteright}.}{vanavond niet De andere}{drie zeiden geen woord}\\

\haiku{{\textquoteleft}Ik zou nooit willen,!}{spelen als het moest op kracht}{van de alcohol}\\

\haiku{Zou ze nu nog een?...}{verlopen actrice op}{sterk water worden}\\

\haiku{Ze zei het niet luid,,.}{ze gilde het niet zoals}{was afgesproken}\\

\haiku{{\textquoteright} zei Robbens zorgzaam, {\textquoteleft},.}{ik wacht buiten en breng je}{met de wagen thuis}\\

\haiku{Ze moest sportief  ,.}{kunnen zijn en op tijd aan}{de kant kunnen gaan}\\

\haiku{De taxi reed haar in.}{zwierige bochten naar het}{kantoor van Robbens}\\

\subsection{Uit: Winterverhalen}

\haiku{{\textquoteleft}Ik eh - ben niet de,{\textquoteright}.}{jongen die duizendjes vindt}{zei hij eenvoudig}\\

\haiku{je sluit het raam dat,.}{je hebt opengemaakt en je}{gaat weg door de deur}\\

\haiku{Een straatlantaarn en:}{de beroemde maan uit het}{liedje hielpen hem}\\

\haiku{Toen schoof hij alle.}{brandbare zaken uit de}{buurt van de brandkast}\\

\haiku{{\textquoteright} {\textquoteleft}Ach ja,{\textquoteright} antwoordde, {\textquoteleft}!}{Jan beschaafdmaar ik wilde}{niet herkenbaar zijn}\\

\haiku{Hij rimpelde er.}{zijn wenkbrauwtjes van en zag}{er zorgelijk uit}\\

\haiku{Morgen mag ik met,{\textquoteright}.}{vader en moeder naar de}{kerk vertelde Cor}\\

\haiku{Terwijl moeder zei,...!}{dat hij al lang naar Spanje}{was teruggegaan}\\

\haiku{{\textquoteleft}Soms,{\textquoteright} zei Cor, die het.}{mysterie toch ook nog niet}{geheel doorgrondde}\\

\haiku{Het eten dat nog op,.}{tafel kwam was niet vet meer}{en krap gemeten}\\

\haiku{In de maatschappij,,.}{die een groot uurwerk was mocht}{je niets zelf willen}\\

\haiku{Was ze maar machtig,...:}{of tenminste verstandig}{dat ze zeggen kon}\\

\haiku{Iedereen in de,.}{buurt wist dat Monseigneur het}{nooit zelf mocht weten}\\

\haiku{De mensen om hem,.}{heen waren ook geroerd zij}{bogen en knielden}\\

\haiku{Vader zat met de,.}{benen op een andere}{stoel en wou niet uit}\\

\haiku{Hij zweeg, en trachtte.}{het verhaal voor het opstel}{te achterhalen}\\

\haiku{Hij vroeg nog zachter,:}{met veel te veel geduld voor}{zo'n klein mannetje}\\

\haiku{En hij wist al, wat,:.}{hij zou vragen en vooral}{wat hij zou z\`eggen}\\

\haiku{De geestelijken:}{hoorden haar in gedachten}{allemaal zeggen}\\

\haiku{Pietje Schiplijn bleef.}{al die tijd gevangen in}{het visioen staan}\\

\haiku{In de eerste plaats,,.}{hoop ik dat gij gezond moogt}{zijn en gelukkig}\\

\haiku{Wij waren in de.}{greep van het grootst ontzag dat}{ik ooit heb gevoeld}\\

\haiku{Hoewel ze net zo,.}{scherp keek als nu verzachtten}{haar trekken zich even}\\

\haiku{Die laatste dagen,.}{waren zo schoon als elk uur}{hier in de hemel}\\

\haiku{De kleine jonkvrouw,.}{sjorde haar kleed op en bond}{het met dat goudkoord}\\

\haiku{{\textquoteright} De woorden riepen.}{een vaag medelijden in}{de jongen wakker}\\

\haiku{Hij was vergezeld,.}{van zes knechten waarvan twee}{koffers meevoerden}\\

\haiku{Daarna ging het in,,.}{fiere gestrekte draf de}{laatste paar mijlen}\\

\haiku{Hij verraadde haar,.}{in halve woorden wat de}{gezant gemeld had}\\

\haiku{{\textquoteright} Dit allerliefste...?}{kinderfiguurtje met de}{Maria-ogen}\\

\haiku{Indien hij gedurfd,.}{had zou hij haar kinnetje}{hebben opgelicht}\\

\haiku{Gij zijt van verre,{\textquoteright}.}{tegen die pilaar gebotst}{zei de jongeman}\\

\haiku{al zou een kamer.}{voor hem en mij tezamen}{mij eenzaam lijken}\\

\haiku{Sweder van Urssen,.}{dronk en begon luider en}{luider te spreken}\\

\haiku{{\textquoteleft}Het viel me woord op,...!}{woord mee dat hij mijn dochter}{vermocht te volgen}\\

\haiku{En reeds was daar aan:}{zijn deur die vervelende}{oude biechtvader}\\

\haiku{Ze leken beiden,.}{in gedachten verdiept zo}{zwijgzaam waren zij}\\

\haiku{De jonkvrouwe kwam.}{als een levend geworden}{spooksel de zaal in}\\

\haiku{Dan vraag ik mij af,,}{waarom zo veel soldaten}{u vergezellen}\\

\haiku{Iets in zijn woorden.}{bracht het reisdoel terug in}{Rychilda's gedachten}\\

\haiku{Ik heb u beproefd,,.}{op kracht op beheersing op}{dapperheid en zelfdwang}\\

\haiku{{\textquoteright} polste Rychilda,.}{alsof ze zulks anders n{\'\i}\'et}{zou hebben gedaan}\\

\haiku{De dag kwam, waarop:}{de dochters van de hertog}{van Gordon zeiden}\\

\haiku{Wat weten wij van?}{liefdes kracht en zwakte in}{het hart van een vrouw}\\

\haiku{en daartussen lag.}{de droom van vlakke zee en}{wazige kusten}\\

\haiku{Ja, hij reed hoog te,.}{paard als in de dagen toen}{hij kapitein was}\\

\haiku{want de edelman aan.}{het hoofd van de troep had een}{bezwegen haast}\\

\haiku{{\textquoteleft}Mijn voedsel is uw,.}{voedsel mijn bezittingen}{zijn geheel de uwe}\\

\haiku{En de stille zaal.}{aan haar voeten was als een}{schaal voor haar tranen}\\

\haiku{de vrouwen bij het,.}{braadspit de oude heerser}{in zijn hoge stoel}\\

\haiku{Het was de vreemdste,.}{Kerstmis welke iemand zich}{had kunnen denken}\\

\haiku{maar waarom heb je?}{die fonkelende witte}{eruit gelaten}\\

\haiku{Hij keek telkens naar,.}{zijn vrouw die met stroeve ernst}{haar kousen heelde}\\

\haiku{hij leek tussen hen,.}{te staan en evenzeer zijn adem}{in te houden}\\

\haiku{Aan de andere.}{kant van de stille straat stond}{haar vriendin Lida}\\

\haiku{{\textquoteright} Met een nuance:}{van ontevredenheid in}{haar stem zei Lida}\\

\haiku{en daarnaast te arm.}{om in beschaafde kringen}{zijn plaats te vinden}\\

\haiku{{\textquoteleft}Wat ben je toch een,!}{voorbeeld van ijver als er}{iemand naar je kijkt}\\

\haiku{Sander was toen al.}{aan zijn vijfde tekening}{voor de kleine zaal}\\

\haiku{Bij nacht werk ik het,,...{\textquoteright}}{beste dan is het stil dan}{zijn de dromen los}\\

\haiku{Hij keek uit het raam.}{en strekte werktuigelijk}{zijn pijnlijke rug}\\

\haiku{Om dat gedicht te,...?}{mogen zeggen  tussen}{al die pretmakers}\\

\haiku{Dat had hij immers...!}{van het eerste ogenblik af}{kunnen verwachten}\\

\haiku{Nee, stoor hem nu niet!}{met je flauwe gepraat over}{koffie en pastei}\\

\haiku{Hij kwam nader, en.}{stond een beetje verlegen}{stil bij de tafels}\\

\haiku{Zij hadden op een.}{verjaardag zitten praten}{over Kerst-sagen}\\

\haiku{In een donkere,,.}{stinkende koeiestal bij}{het vak van de stier}\\

\haiku{{\textquoteleft}Nee,{\textquoteright} zei de stier, met.}{weer die rare grimas van}{lippen en wangen}\\

\haiku{Zelfs de man in de.}{opname-wagen sprak over}{goedkope mopjes}\\

\haiku{Alles wat restte.}{was een geluidsband met een}{brommerige stem}\\

\haiku{Hun zoontje Piet, een,.}{jongen van twaalf jaar was al}{een tijdje niet goed}\\

\haiku{alsof een Kerstboom,!}{het uithield in regen en}{storm en desnoods sneeuw}\\

\haiku{Op het Dweelsepad {\textquoteleft}}{schudde meneer pastoor de}{notaris de hand.}\\

\haiku{{\textquoteleft}Weet je wat, Kors,{\textquoteright} zei, {\textquoteleft},.}{hij danik kom eens terug}{als je alleen bent}\\

\haiku{Aan zijn praat was te,.}{merken dat hij weinig werd}{tegengesproken}\\

\haiku{Verlejen jaar heeft...{\textquoteright}}{hij Kerstliedjes gezongen}{voor zieke mensen}\\

\haiku{{\textquoteright} Zijn woorden leken.}{vleugels te krijgen tegen}{de koele hoogte}\\

\haiku{Dirk heeft gelijk,{\textquoteright} zei, {\textquoteleft},}{Theuns aarzelendals ik nou}{die boom geef komen}\\

\haiku{Daarna wendde hij,.}{zich tot Dirk en legde een}{hand op zijn schouder}\\

\haiku{Nee, hij kwam niet voor,;}{Kors die al met een brede}{grijns de stoel bijschoof}\\

\haiku{- Zo kwam Jaantje bij.}{Pietje van Rien en Mien in}{de kamer te staan}\\

\haiku{We hebben boven,...{\textquoteright} {\textquoteleft},}{nog zo'n dunne koperen}{bak van ome KreelJa}\\

\haiku{Iets in het duister,.}{gaf hem inspiratie haar}{hand vast te pakken}\\

\haiku{Vrouw Martje Theuns zei, dat.}{ze naar de stad wilde gaan}{om engelenhaar}\\

\haiku{Op het gebroken.}{hekje van Miebartjes tuin zat}{een van Barnds duiven}\\

\haiku{Daarachter lag het,.}{wijde veld met een stukje}{van het Dweelsepad}\\

\haiku{{\textquoteleft}Dat heeft notaris,,{\textquoteright};}{al gedaan toen we jouw brood}{gingen eten zei ze}\\

\haiku{Twee dagen later.}{werd er in de avond geklopt}{bij de notaris}\\

\haiku{De avond viel, en spon.}{de boom met al zijn sier in}{een rust van schaduw}\\

\haiku{het was net zoals:}{mevrouw Marie tegen de}{dokter had gezegd}\\

\haiku{{\textquoteleft}O,{\textquoteright} zei hij ademloos, {\textquoteleft}!}{tegen zijn oudersalles}{is vanavond \'even mooi}\\

\haiku{Hij celebreert in,.}{grafse stilte de Mis en}{zingt zijn litanie}\\

\haiku{Men kan, wanneer men,;}{verloren is beter voor}{de dag uit varen}\\

\haiku{Op de middenbank.}{zaten twee oude mannen}{in rode mantels}\\

\haiku{Terwijl hij toch zo'n.}{welbesteed leven achter}{zich had liggen}\\

\haiku{{\textquoteleft}Ik ben de Goede,{\textquoteright},.}{Bedoeling zei hij als om}{zich voor te stellen}\\

\haiku{Mij is niemand te,.}{machtig en ik zal u de}{reden verklaren}\\

\haiku{{\textquoteleft}Doorzoek mijn huis dan,{\textquoteright}, {\textquoteleft}.}{raadde Sylvester hemwant}{ik hecht daaraan niet}\\

\haiku{{\textquoteleft}Waarom de mens faalt,{\textquoteright}, {\textquoteleft}.}{antwoordde zedat is al}{zo dikwijls gezegd}\\

\haiku{V\'o\'or hem lag dat strand,.}{en hij had zo juist zijn boot}{aan wal getrokken}\\

\haiku{En hij dacht in een:}{plotseling verlangen naar}{stilte en puurheid}\\

\section{Ferdinand Langen}

\subsection{Uit: In pyama}

\haiku{Toen liet hij zijn hand.}{wippen voor zijn open mond en}{begon te kauwen}\\

\haiku{Een flauwe smaak kwam,}{in mijn mond en ik voelde}{dat ik hoofdpijn kreeg}\\

\haiku{In de tijd tussen.}{twee schokken luisterde hij}{naar den dominee}\\

\haiku{Eigenlijk schaamde,.}{ik mij voor hem maar ik wist}{niet precies waarom}\\

\haiku{Die is flauw{\textquoteright}, zei de, {\textquoteleft}}{Weesneem maar mee naar huis en}{leg zout op z'n pens.}\\

\haiku{Ik lachte hem uit.}{en zei dat hij nauwlijks tot}{de helft zou komen}\\

\haiku{Aan de gevolgen,.}{durfde ik niet denken die}{moesten vreselijk zijn}\\

\haiku{Er gebeurde niets,.}{maar ik durfde niet weer over}{de rand te kijken}\\

\haiku{Het teer was taai en.}{ik rolde het in mijn hand}{tot een balletje}\\

\haiku{Ik herinnerde.}{mij alleen nog duidelijk}{een voldaan plezier}\\

\haiku{en nou jou, om die,!}{kat en zijn huishoudster is}{nog wel zijn zuster}\\

\haiku{Maar ik voelde het,.}{wel kreeg blauwe plekken en}{huilde stil van pijn}\\

\haiku{Ik probeerde de,.}{klauw te grijpen maar ik greep}{Fijt en hij was nat}\\

\haiku{De uitgestrekte.}{houten arm bleef mij tergend}{in de rug steken}\\

\haiku{Bij de kerk wilde,.}{ik het hek in draaien maar}{Fijt hield me tegen}\\

\haiku{Des avonds zat ik met.}{de vreemde jongens in een}{weiland aan de gracht}\\

\haiku{Ik draafde door het.}{weiland tot het water me}{in de schoenen stond}\\

\haiku{Waarschijnlijk hoopte.}{hij op die manier iets van}{Hein los te krijgen}\\

\haiku{{\textquoteleft}Ja, ja{\textquoteright}, viel Evert mij,.}{dadelijk bij terwijl hij}{schuchter naar Hein keek}\\

\haiku{In de winter die,.}{volgde zocht ik opnieuw het}{gezelschap van Evert}\\

\haiku{Zij was veel zwaarder:}{dan ik en het resultaat}{bleef altijd gelijk}\\

\haiku{Van hun gesprekken,}{begreep ik de helft niet toch}{wonden ze me op}\\

\haiku{Nu begrijp ik, dat.}{dat niet helemaal buiten}{zijn eigen wil was}\\

\haiku{We liepen zwijgend.}{in de karresporen van}{het zandweggetje}\\

\haiku{Waarom, dacht ik, zijn.}{er zoveel mannen door hen}{verleid geworden}\\

\haiku{Haar benen en haar,,.}{armen haar neus en haar oren}{maar vooral haar mond}\\

\haiku{{\textquoteleft}Wat is er met jou,?}{Memmeling heb je zo slecht}{geslapen vannacht}\\

\haiku{Ik was naakt, alleen.}{over mijn ene schouder vielen}{haar blonde haren}\\

\haiku{Toen het begon te,.}{regenen leerde ik in}{kroegen te schuilen}\\

\haiku{{\textquoteleft}Ze zal toch nog  ,{\textquoteright}.}{wel een staartje hebben dat}{zeemeerminnetje}\\

\haiku{Ze nam het  in.}{haar handen en streelde er}{over met haar vingers}\\

\haiku{Je vader het wel,{\textquoteright}.}{geld maar die verdomt het \'o\'ok}{om af te schuiven}\\

\haiku{nu vermoord ik hem,.}{hoorde zij het bekende}{fluitje weer op straat}\\

\haiku{Op de trappen van.}{het stadhuis kreeg ze ruzie}{met haar schoonmoeder}\\

\haiku{Hij kon dat moppig.}{overdrijven van Cornelis}{niet erg waarderen}\\

\haiku{Wat ons bond was geen,.}{vriendschap meer het was van mijn}{kant een obsessie}\\

\haiku{Hij weet weer waar hij,,....}{is en hij kan nog roken}{Goddank roken}\\

\haiku{De bloemen waren,.}{nu zo dicht bij dat ik ze}{meende te ruiken}\\

\haiku{Ik glimlach daar nu,.}{om maar ik ben eigenlijk}{nog niets veranderd}\\

\haiku{Ik kan dat rustig,.}{zeggen nu ik alleen ben}{in deze kamer}\\

\haiku{Bovendien ziet hij.}{alles in de verte als}{door beslagen ogen}\\

\haiku{Hij tekent met zijn.}{pen kleine figuurtjes op}{zijn bolle nagels}\\

\haiku{Zullen de bellen?}{tussen zijn vingers hem nog}{kunnen verleiden}\\

\haiku{Maar Vita haalt haar.}{schouders op en opnieuw raakt}{haar lichaam zijn arm}\\

\haiku{Hij denkt de dagen.}{door en des nachts is hij te}{moe om te slapen}\\

\haiku{Het personeel roept.}{hij bijeen en hij speecht als}{een feestredenaar}\\

\haiku{Ze moeten het goed,,:}{begrijpen het is niet erg}{belangrijk alleen}\\

\haiku{De ramen van de.}{huizen knipogen hen toe en}{de bomen wuiven}\\

\haiku{Mijn tijd komt, want nog.}{altijd geloof ik dat het}{goede beloond wordt}\\

\section{Jef Last}

\subsection{Uit: Partij remise}

\haiku{Hun lachen smoort als.}{het piraatje tusschen de nat}{geworden vingers}\\

\haiku{De \'e\'ene  hoop.}{van iederen dag kleurt hun}{wangen en voorhoofd}\\

\haiku{Langs de dijken ligt.}{de levende guirlande}{van jonge menschen}\\

\haiku{Nog later in den.}{nacht komen de pinose}{jongens de straat op}\\

\haiku{Met fellen slag jaagt,.}{de spoel heen en weer heen en}{weer door den inslag}\\

\haiku{Zoo staan ze, wanneer.}{ze een standje krijgen of}{ontslagen worden}\\

\haiku{Dan treedt Sternheim op:}{om een acte uit Goethe's}{Faust voor te dragen}\\

\haiku{Is het wonder dat?}{Gerrit tevreden is over}{den gang van zaken}\\

\haiku{Wie klaar is, mag de{\textquoteright}.}{sommen van gisteren in}{het net gaan schrijven}\\

\haiku{, wij zullen er ons{\textquoteright} (,).}{bij neer gelegen hebben}{fout zegt de meester}\\

\haiku{Argumenten en.}{scheldwoorden stuiten af op}{een koppig zwijgen}\\

\haiku{{\textquoteleft}Ach{\textquoteright}, zegt hij, {\textquoteleft}een haal{\textquoteright}.}{uit de mok kreeg ik thuis ook}{wel eens van mijn oue}\\

\haiku{Een enkele ligt.}{met zijn laarzen aan reeds op}{de stroozak te slapen}\\

\haiku{Een korte kreet wordt.}{zoo snel gesmoord alsof ze}{niet had geklonken}\\

\haiku{Ja, domin\'e, we{\textquoteright}.}{spraken net over de krijgstucht}{bij de marine}\\

\haiku{{\textquoteleft}U bent nog te kort,,.}{op de vloot domin\'e U}{hebt nog idealen}\\

\haiku{Brieven zijn aan boord.}{van een opleidingsschip geen}{priv\'e bezitting}\\

\haiku{{\textquoteright} roept ze, {\textquoteleft}ja, thuis is, '{\textquoteright}.}{ie twee trappen op maar en}{overt portaaltje}\\

\haiku{Bij de Neptuun bar,:}{brandt een gloeiende lichtbaak}{in blauw rood en geel}\\

\haiku{jaagt het bloed door zijn,}{lichaam naar zijn kop toe in}{twee groote teugen zuipt}\\

\haiku{{\textquoteleft}Verdomme{\textquoteright}, fluistert, {\textquoteleft},{\textquoteright}.}{hijals ik maar slof als ze}{maar niet zuur is}\\

\haiku{Een anderen keer.}{heeft hij haar meegenomen}{naar het Gebouw toe}\\

\haiku{- Dokter Noppes, - De, -, -.}{oue va\`ar Kapitein Slikjas}{Luitenant Roosje}\\

\haiku{Voorzichtig, zachtjes,.}{om niet te storen brengt zijn}{vrouw de thee binnen}\\

\haiku{Een volle heen, een -,,.}{leege terug f\"ordern f\"ordern}{jakkert de steiger}\\

\haiku{{\textquoteleft}Wat ik zeggen wou{\textquoteright},, {\textquoteleft}}{vervolgt van Garderenvoor}{mij is de wereld}\\

\haiku{Maar de mensch is juist.}{vermoeid geworden van het}{eeuwen rechtop staan}\\

\haiku{Grijze lucht, grauwe.}{golven en de regen die}{in zijn gezicht striemt}\\

\haiku{Ineens herwint de.}{schipper zijn bezinning en}{breekt de verstarring}\\

\haiku{Op dat oogenblik.}{is de mijn geen drie meter}{meer van het schip af}\\

\haiku{Willem hier, liep nog.}{geen drie maanden geleden}{op z'n tandvleesch}\\

\haiku{Nou heeft ie een paar,.}{nieuwe moli\`eres een}{das en een dasspeld}\\

\haiku{Maar  dan met de,.}{mannetjes in de hand met}{de vuist op tafel}\\

\haiku{Troelstra herinnert zich:}{een tafelgesprek tegen}{David en Heine}\\

\haiku{De Twerskaja is,.}{zwart van demonstranten de}{Soljanka vloeit over}\\

\haiku{{\textquoteright} Op het hout van zijn.}{brits drukt fuselier Hannes}{nijdig een luis dood}\\

\haiku{{\textquoteleft}Een arrebeier{\textquoteright},, {\textquoteleft},.}{zegt Hannesis nou een keer}{geen mensch dat weet je}\\

\haiku{Hoe meer herrie d'r{\textquoteright},, {\textquoteleft}}{komt merkt de mijnheer bedaard}{ophoe moeilijker}\\

\haiku{Niet afwachten en.}{protesteeren en er het}{beste van maken}\\

\haiku{Slingerend in den.}{wind strooit een bleeke booglamp z'n}{licht op de spijlen}\\

\haiku{{\textquoteright} Jaantje klampt zich aan,,.}{Gerard die z'n arm in een}{verband draagt  vast}\\

\haiku{Alsof 't van een!}{legertje en een vloot als}{de onze afhing}\\

\haiku{Maar in '16 is 't{\textquoteright}.}{ernst geworden en verdomd}{bloedige ernst ook}\\

\haiku{In Tjimahi een '.}{reuzen vergaderings}{nachts op de renbaan}\\

\haiku{Die dienstweigering.}{op de Regentes is te}{vroeg gekomen}\\

\haiku{honderd meiden die.}{ik beter gekend heb dan}{m'n eigen zuster}\\

\haiku{Als ze met Gerard,.}{uitging moest hij een boordje}{om en een dop op}\\

\haiku{Je kunt probeeren weg,.}{te loopen dan krijg je de}{kogel in je rug}\\

\haiku{Zie je hoe ze olie?}{krijgen voor hun motor en}{geld voor hun manschap}\\

\haiku{Natuurlijk zijn er.}{andere  factoren}{ook bij gekomen}\\

\haiku{Met de winsten uit.}{Indi\"e ging de illusie}{samen verloren}\\

\subsection{Uit: Van een jongen die een man werd}

\haiku{Het merkwaardige,.}{was dat ik mij in beide}{kringen thuis voelde}\\

\haiku{nergens zoo goed als;}{stil in mijn kamer met}{mijn boeken om me}\\

\haiku{een pak slaag, als het,.}{eenige wat hij er aan doen}{kon voor remedie}\\

\haiku{Hij, sinds twee jaar dood,?}{voor zichzelf leefde nog in}{hun herinnering}\\

\haiku{Een breede bundel,}{glimmende rails straalde door}{een dal het land in}\\

\haiku{Weer trokken ze met.}{breede trossels jongens des}{Zondags den weg langs}\\

\haiku{{\textquoteleft}kom laten we gaan,{\textquoteright}.}{zwemmen en dan naar huis de}{boer wacht met melken}\\

\haiku{Aan de gootsteen hielp.}{een meisje haar moeder de}{vaten te spoelen}\\

\haiku{Frenske, die aan zoo'n,.}{woordenvloed niet gewoon was}{werd er beduusd van}\\

\haiku{Wat wou nou zoo'n vent!}{hem vermanen die zelf niets}{eens naar de kerk ging}\\

\haiku{Hij zag er uit of.}{hij ons met zijn sabel in}{stukjes wou hakken}\\

\haiku{Er was tusschen hun.}{twee\"en nog nooit een woord over}{liefde gesproken}\\

\haiku{De bitterheid drong.}{zijn hart binnen en keerde}{zich tegen hemzelf}\\

\haiku{Eerst verloren ze.}{hun werk en dan zette men}{hen uit hun huisje}\\

\haiku{Eigenlijk was de.}{geestelijkheid door dezen}{aanval overvallen}\\

\haiku{Frenske liet zich door,.}{een stroom meestuwen een}{andere straat in}\\

\haiku{Het was hardbruin met.}{aan den kanten leelijke}{houten barakken}\\

\haiku{Boven hen in de.}{hooge boomen zat ergens een}{vogel te zingen}\\

\haiku{Dan kom ik uit de}{donkere mijn En hoor bij}{vagen schemerschijn}\\

\haiku{Het was of van de.}{beantwoording dezer vraag}{zijn leven afhing}\\

\haiku{de meesten het in.}{d'r hart als je met kunst je}{brood moet verdienen}\\

\haiku{Voor de kroegen 's}{avonds en voor de meiden had}{hij geen centen meer}\\

\haiku{, smokkelen dorst hij,.}{ook niet meer en het werk bleef}{altijd even rottig}\\

\haiku{Aan het fornuis was.}{moeder nog bezig met de}{pannen en schotels}\\

\haiku{{\textquoteleft}Heilige Maria,,.}{vol van genade gij weet}{dat ik haar liefheb}\\

\haiku{Je kon dat met den.}{besten wil van de wereld}{niet geluw noemen}\\

\haiku{Was hij dan niet in,?}{de mijn geweest toen dat groote}{ongeluk plaats vond}\\

\haiku{Frenske was te veel.}{arbeider om dit ooit te}{kunnen begrijpen}\\

\haiku{Aan den anderen.}{kant van het plein hoorden ze}{hem nog even schelden}\\

\haiku{mocht je rondkruipen,.}{over het land dat je met hun}{strond bemest had}\\

\haiku{Het kon geen toeval,.}{zijn dat ze hier al dien tijd}{op hem gewacht had}\\

\haiku{Om een inbeelding,,?}{een gril het terugschrikken}{voor een nieuwigheid}\\

\section{Gheraert Leeu}

\subsection{Uit: Dialogus Creaturarum dat is Twispraec der creaturen}

\haiku{Dit geldt ook voor de.}{mens als hij zich bij tijd en}{wijle niet ontspant}\\

\haiku{De dialoog wordt in:}{de inhoudsopgave als}{volgt aangekondigd}\\

\haiku{op de initiaal.}{geeft de regelhoogte van}{de initiaal aan}\\

\haiku{als die heylighe}{leerraer sinte thomas van}{aquinen | bescrijft}\\

\haiku{|  [30] dat xxxi}{van mandragora ende}{venus dat ons leert}\\

\haiku{dat wij an | een.}{anders qualick varen ons}{spieghelen sellen}\\

\haiku{ons maeteliken}{ende saetliken te re}{| gieren |}\\

\haiku{vier Van eenen ionghen [].}{bock die alte groten goec}{|10 kelaer was}\\

\haiku{Die maen antwoorden.}{Ghanck van mi want ick di}{niet | lief en heb}\\

\haiku{dye haer seluen:}{seer began te verheffen}{| ende seyde}\\

\haiku{ende hoy eeten [}{als een os ende leuen}{tijden sel- |}\\

\haiku{tot dat die seuen}{iaren | volbracht waren}{In welken seuen}\\

\haiku{Ist dattu yegens mi}{vechten wils so en bistuniet}{|  wel bedacht}\\

\haiku{Ghelijck als een [].}{|25 drachticheet cleyne}{dinghen tot groot brenghet}\\

\haiku{Du selste weder}{op een ander tijt | mit}{hem werden verblijt}\\

\haiku{mer si dede an}{hoor wa | penen ende}{began te strijden}\\

\haiku{Quade harders en.}{hebben gheen sorghe van}{horen sca | pen}\\

\haiku{Die zee antwoorden.}{Siluer ende goudt}{en is niet met my}\\

\haiku{mit boesheyt ende.}{houerdicheyt te | ghen}{horen ouersten}\\

\haiku{int vader leuen}{dat een broeder | vraechde}{enen ouden vader}\\

\haiku{Dat neghentiende[]}{dyalogus |          15}{TOt 134 dat gout quam}\\

\haiku{dat dye vrouwe van}{eenre duyf die van bouen}{| quam ende nam}\\

\haiku{gheliken noch mit {\textparagraph}}{hem veel hanteren Want die}{wi | se man seyt}\\

\haiku{veel luden spraken [].}{wort gheuraghet waer |25 om}{hy dat dede}\\

\haiku{beloghen ende .}{bedraghen doe seyde hy}{totten keyser |}\\

\haiku{om dijnre lieften []}{ont- |10 fanghen heb}{die mogen spreeken}\\

\haiku{Wi bidden di dattu}{dy van die werlt wechdoeste}{ende dattu dy |}\\

\haiku{LAet 177 ons soecken.}{den besochten meester van}{| medicinen}\\

\haiku{lijfs ontgaen mochten.}{dat si dan hoor son- |}{den souden biechten}\\

\haiku{Ende doe si te}{lande ghecomen waren}{| spraken si hoor}\\

\haiku{also | verdreef.}{mandragora dat onreyn}{wijf ende seyde}\\

\haiku{goeden wil dye hi}{had om te clooster te gaen}{| ende segghen}\\

\haiku{bistu mi seer waert}{| Daer om wil ick mi mit}{di verenighen}\\

\haiku{want alsoe sal hi.}{iv oeck doen ist dat ghy |}{v niet en wachtet}\\

\haiku{Augustinus dat}{god den danckbarighe ghe}{| gheuen hadde}\\

\haiku{machtighen mensche}{op dattu | niet en valles}{in sinen handen}\\

\haiku{Mer dat hondekijn}{spoelde myt sinen heer dat}{hy oeck | voeden}\\

\haiku{bekenden dattie}{voghel astur steruen most}{en woude hi |}\\

\haiku{Hi antwoorden Ia}{ick hebber noch wel drieDie}{vader | seyde}\\

\haiku{dat vleys crighen dat.}{hi heeft Hier om seyde hi}{totten ra | uen}\\

\haiku{Daer om salstu my}{gheloeuen ende |}{betrouwen Coom daer}\\

\haiku{ende leyden dat.}{pack of | ende die ezel}{ontfinck sijn loon}\\

\haiku{alle die werlt o}{| uer Daer om seyt sinte}{Ian in sijn epistel}\\

\haiku{seyt is een voghel}{vanden | gheslachte des}{ghiers wit van verwen}\\

\haiku{Ende om dat si.}{gheen goet beghin en |}{hadde inder ioecht}\\

\haiku{an die merct om te.}{verkopen | seyde tot}{sinen ghesellen}\\

\haiku{wanneer die wijse}{gheslagen is wat sal hi}{doen | Hi antwoort}\\

\haiku{alle die stocken.}{om ende veriaghede}{alsoe die bijen}\\

\haiku{dat sijns va- | [].}{5 ders leeringhe goet ende}{profitelick was}\\

\haiku{Hi leit in sijn nest.}{een | smaragd teghen die}{vernijnde dieren}\\

\haiku{recht of si op een}{muer ghe | staen hadden}{Dese beesten sijn}\\

\haiku{Dyalogus xc}{|          SAtirus 352 is}{een mensch die hoornen}\\

\haiku{hem beyden om te {\textparagraph}}{vliegen | ende van daen}{te come Ende}\\

\haiku{Het is beter den}{| bosen prior of te}{setten ende enen}\\

\haiku{Ic ben selue wijs}{ende verstandel ghenoech}{Ick en wil van v}\\

\haiku{366 |          EEnhoorn}{367 is een dyer dat een}{hoorn in sijn voerhoeft}\\

\haiku{perikel was in}{die zee ende hy sat op}{enen groten stoel |}\\

\haiku{oec die qualen der}{zielen gepurgeert mitten}{siecten ende |}\\

\haiku{hem den hals ende.}{seyde Qualiken 384 |}{heeft hi wreaek ghedaen}\\

\haiku{quam ende nam dat [}{sackelkijn mytten goude}{ende liep daer |}\\

\haiku{oec som | wijlen.}{gheschieden als dat beest een}{goet stuck van hem was}\\

\haiku{ende dan dicwijl}{gheen van allen krijghen en}{mochte | alsoe}\\

\haiku{Paulus Inden staet [].}{daer ghi ge |25 ropen}{sijt daer bliuet in}\\

\haiku{wi sijn ghesont ende.}{en besighen nym- |}{mermeer medicijn}\\

\haiku{hoe langhe sal men}{| silencium houden Die}{oude antwoorden}\\

\haiku{Maeckt hier of meester}{een leringhe | te rijm}{dat vwe konst daer wt}\\

\haiku{dat een velthen was}{die veel iongher kuyken had}{die si seer nau |}\\

\haiku{rusten in hoer hol []}{ende als si drie dagen}{geslapen |10}\\

\haiku{ver- | droech hi}{oeck al goetelick Van den}{keyser augustus}\\

\haiku{een ander geweest []}{haddeGaet haes- |20}{teliken van hier}\\

\haiku{Het is beter |}{dat wi hem doden dan dat}{wi ons seluen}\\

\haiku{wil ic di leren}{in die clergi dattu ghe-}{| leert mogeste}\\

\haiku{acker ghebouwet}{had ende hadde daer |}{in ghesaeyt vlasschen saet}\\

\haiku{huden en heeft hi.}{niet mo- | ghen ontgaen}{die scutten des doots}\\

\haiku{Die oghen plaghen}{mit ghenoechten scoon dinghen}{| te sien ende}\\

\haiku{wes ontfanck is}{sonde wes gheboerte is}{| onsalicheyt}\\

\haiku{dat is Twispraec der,,;}{creaturen Gheraert Leeu}{Gouda 4 april 1481}\\

\haiku{39a7 (boek begint op),,,,,,,;}{a2r b8-i8 k6 l8-m8 n6}{nn8-z8 {\cyryhcrs}6 96}\\

\haiku{400H initiaal, 3,.}{regels hoog geen zichtbare}{representant}\\

\subsection{Uit: Dye hystorien ende fabulen van Esopus}

\haiku{In zijn Latijnse ().}{editie1486 komen deze}{anekdotes wel voor}\\

\haiku{Na de Latijnse.}{vita volgt het leven van}{Esopus in het Duits}\\

\haiku{Het papier bevat,.}{veel gaatjes met name langs}{de binnenrand}\\

\haiku{een cleyne tijt daer}{hij hem tusschen onscul}{| dighen mochte}\\

\haiku{Soe coempt die ghene [}{die de sorghe hadde}{vanden hee- |}\\

\haiku{alzoe ghij siet dat.}{ick ben die minste en |}{de die cranckste}\\

\haiku{| Mer nae haren [].}{goetduncken gauen sij hem |}{5 den meesten last}\\

\haiku{Ick biddu dat ghij}{mijne woir | den int quaet}{niet nemen en wilt}\\

\haiku{Ende als esopus}{dese ant | woerde was}{hoorende begonste}\\

\haiku{dach als xanctus.}{hem baden | wilde met}{sijnen studenten}\\

\haiku{ghenomen hadde}{Ende als die | voeten}{ghenoech ghesoden}\\

\haiku{In dien daghe |}{des ordeels als die menschen}{verrisen sullen}\\

\haiku{| den welcken.}{ghij die costelijcke}{spyse gheseyndt hebt}\\

\haiku{Nu siet ghij wel dat}{v wijf die wech is ghegaen}{v niet lief en |}\\

\haiku{Mijn heerschap doet dy.}{bidden dat ghij coemt eten v}{myddachmael met | hem}\\

\haiku{hoe dattet sij Ick.}{wil vollen mynen buyck}{ende altijt eten}\\

\haiku{ende als dese}{tafelen aldus gheset}{| ende bereydt}\\

\haiku{fabule ende}{hystorie die welcke}{vertelt van twee rauen}\\

\haiku{Gaet en- | de}{deylt den scadt van goude die}{ghij gheuonden hebt}\\

\haiku{dat die honden den}{woluen ghegheuen ende}{gheleuert souden}\\

\haiku{dat ghij my wijset die}{ste- | de daer ghij den}{voirghenoemden toorn}\\

\haiku{ghelykerwijs als}{die merye peerden die ghi hebt}{la | ten comen}\\

\haiku{so namen sij eenen}{gulden cop vten tempel van}{appollijn haren}\\

\haiku{Ende badt sijnen}{meester dat hij eens mocht |}{die steden besien}\\

\haiku{hem dit gonnende}{so hebben sij hem op eenen}{wae | ghen gheset}\\

\haiku{dat hi tot ghenen}{daghen eenich broot | van hem}{en hadde ghehadt}\\

\haiku{Daer nae als die teue.}{haer cleyne honden ghe-}{| worpen hadde}\\

\haiku{Ende daerom.}{| en is haer gheselscap}{noch goet noch vruchtbaer}\\

\haiku{Waer af dat esopus.}{vertelt eene aldusdane}{| ghe fabule}\\

\haiku{Dat die ghene die [].}{altijt quaet doet |15 die}{wijle dat hij vroom}\\

\haiku{seg- | [15]}{| ghende totten afgod}{iupiter aldus}\\

\haiku{des doots want het is}{veel beter in sekerheyt}{groffelijc | ken}\\

\haiku{soe vergheue ict.}{v Mer alsoe langhe als}{ick | sal leuen}\\

\haiku{| Die dinghen die.}{ouermits Crachte ende}{vreese ghelooft sijn}\\

\haiku{|    HEt was een}{wolf die op eenre tijt vant}{een hooft van enen man}\\

\haiku{aldusdanighen}{sentencie ende sprack |}{totten wolf aldus}\\

\haiku{Die vijfthienste {\textsection}}{fabule is vanden wolf}{ende vanden hont}\\

\haiku{Hier om en is die}{ghene niet goet die twee |}{contrary heeren}\\

\haiku{ende dan sal ic}{| doen dat ghene dat v}{belieuen sal}\\

\haiku{want ist sake dat}{die | ossen dryuers oft}{onse meester dy}\\

\haiku{want ghij een Coninc.}{sijt so ist alte samen}{tot uwen ghebode}\\

\haiku{v eten oft hij sal}{v brenghen | op die merct}{om te vercopen}\\

\haiku{| sijn oft eyghen}{te wesen ouermits dat}{hij wrake van eenen}\\

\haiku{Ende daer om en}{salmen inden flatterers}{gheene | gheloue}\\

\haiku{Ha arme sotten.}{ghi en weet niet waer om dat}{| ick voerlope}\\

\haiku{mer hier is twerck}{der waerheyt gheble |}{ken ende gheschiet}\\

\haiku{ym] [e] een zweert in[]}{sinen wech ghevondene}{hebbe Ende |}\\

\haiku{die nochtans zeer sot[] {\textsection} ()}{is ende gheen wijsheyt en}{doet |         3586}\\

\haiku{Soe bidde ic v.}{dat | ghij my hier inden}{wech niet eten en wilt}\\

\haiku{want hi was in sijn}{hollekijn bi tlogijs vanden}{leeuwe daer hi |}\\

\haiku{wel ghespijset werden}{Ende sprac | tot desen}{tween scapen aldus}\\

\haiku{die scapen | [25]}{staen op beyde die hoecken}{vanden kamp Ende}\\

\haiku{een lam Ende doe}{dijne naersticheyt omme}{my dat te nemen}\\

\haiku{die wolf en soude.}{onse lammeren niet wech}{ghedraghen hebben}\\

\haiku{waer af dat ic dy []}{zeer dancke Ende die |}{20 wolf seyde hem}\\

\haiku{alle tgheent dat groen}{ende drooch is | Ende}{die derde seyde}\\

\haiku{ende alle |}{die telgheren Ende die}{rechter sprack tot hem}\\

\haiku{sal moghen alst noot []}{|10 is Ende die wolf}{antwoerdede hem}\\

\haiku{Ic wilt gheerne doen}{ende bins wel te vreden}{| Ende mettien}\\

\haiku{ende dat scaep}{saghen vercleet wesende}{mittet vel vanden}\\

\haiku{Het is die mensche []}{Ende hij seyde weder}{om |25 tot hem}\\

\haiku{want ic niet voerder}{gaen en mach Ende die man}{seyde | tot hem}\\

\haiku{Ende mettien so}{begonste hem die man op}{sijn hooft te smijten}\\

\haiku{Ghelijck alst blijct}{by deser nauolghender}{fabulen |}\\

\haiku{5] helpen wilt soe.}{sullen wi wel ter stont wt}{desen put comen}\\

\haiku{wat ghi doet dat doet{\textsection} ()}{wijslick ende aensiet dat}{eynde |          101}\\

\haiku{god aenbedende.}{was op dat hi hem vele}{goets verleenen wilde}\\

\haiku{hem geuende so.}{groten slach mitten hoofde}{teghens die muer}\\

\haiku{Ende als hi sach []}{datse niet dansen en |}{10 wilden soe werdt}\\

\haiku{hoofde ende wt []}{sinen baerde om dat}{hij |5 haer so}\\

\haiku{daeromme so wie.}{den anderen be- |}{rispen oft leeren wil}\\

\haiku{meynende dathet.}{eenen leeuwe geweest ware}{ende liepen wech}\\

\haiku{die welcke mal.}{| canderen ghemoetten}{op een watere}\\

\haiku{hoe sijdi al |}{dus dwaes van v seluen}{aldus te prijsen}\\

\haiku{fabule is van.}{eenen visscher ende | van}{een cleyn visschelkijn}\\

\haiku{ende te dijnen}{passe moghen teten so}{ic my betrouwe}\\

\haiku{ende groot profijt.}{dat hi van te voren te}{hebbene | plach}\\

\haiku{wachtu lieue sone.}{dat die mire niet wi |}{ser en sij dan ghi}\\

\haiku{zo laet dan voir mij.}{comen alle de vrou |}{wen van uwen huyse}\\

\haiku{Die lantman seide}{Daeromme | hebbe ic}{v gheuangen om}\\

\haiku{ne dat seker is[] {\textsection} ()}{voir tghene dat onseker}{is |         20156}\\

\haiku{Als sij ghegeten []}{hadden quam nedius ten}{|10 aenganghe}\\

\haiku{alle die ghene}{diemen hem aenbrochte die}{ontsinnich waren}\\

\haiku{Henderson, Arnold, '',:}{ClaytonHaving Fun with the}{Moralities in}\\

\haiku{Dit is ook de druk.}{die Hecker in zijn editie}{heeft getranscribeerd}\\

\subsection{Uit: Het ongelukkige leven van Esopus}

\haiku{En wat het ergste,.}{was hij was stom zodat hij}{geen woord kon spreken}\\

\haiku{Wie goed doet, mag dus.}{goede hoop op God hebben}{beloond te worden}\\

\haiku{{\textquoteleft}Maar ik heb een knecht,.}{die niet mooi maar wel van de}{juiste leeftijd is}\\

\haiku{Volgens ons hebt u.}{hem meegebracht om ons voor}{de gek te houden}\\

\haiku{Mijn echtgenote}{is zo verwaand dat ze het}{niet zou verdragen}\\

\haiku{{\textquoteleft}Heb je niet in de?}{gaten dat ik meer van haar}{houd dan van mijzelf}\\

\haiku{{\textquoteright} Vervolgens wendde,:}{Xanctus zich wederom}{tot Esopus en zei}\\

\haiku{{\textquoteleft}Ga naar de markt en.}{koop het allerbeste eten}{dat je kunt vinden}\\

\haiku{Toen de studenten,:}{het gerecht zagen spraken}{zij tot Xanctus}\\

\haiku{{\textquoteright} {\textquoteleft}Breng ons het derde,{\textquoteright}.}{gerecht beval Xanctus}{Esopus even later}\\

\haiku{Waarna Esopus als.}{derde gerecht wederom}{tongen opdiende}\\

\haiku{Kom daarom vandaag,.}{opnieuw bij mij dan zullen}{we wat anders eten}\\

\haiku{{\textquoteleft}Vrouw, doe water in.}{het hekken en was deze}{pelgrim de voeten}\\

\haiku{De steen die u daar,.}{ziet liggen lag eerst in het}{begin van het bad}\\

\haiku{{\textquoteleft}Niet u geeft mij de,.}{schat maar degene die hem}{hier neergelegd heeft}\\

\haiku{{\textquoteleft}Omdat de letters,{\textquoteright}, {\textquoteleft}:}{antwoordde Esopusons dat}{duidelijk maken}\\

\haiku{Weet je dan niet dat!}{de god die wij vereren}{er net zo uitziet}\\

\haiku{Nadat ze lekker,:}{gegeten hadden sprak de}{kikker tot de rat}\\

\haiku{{\textquoteleft}De armen moeten.}{door de rijken niet bespot}{of versmaad worden}\\

\haiku{Hij voegt daarom de:}{gecursiveerde woorden}{toe aan Macho's tekst}\\

\section{Aart van der Leeuw}

\subsection{Uit: Ik en mijn speelman}

\haiku{Wij schertsten dit weg,.}{of we een lastig insect}{van ons afsloegen}\\

\haiku{Spoedig verloor ik.}{hem uit het gezicht in het}{warnet der straten}\\

\haiku{Ook hij haalde de,.}{schouders op en keerde zich}{kalm tot zijn klanten}\\

\haiku{{\textquoteright} Mijn speelman ligt te,,;}{ijlen dacht ik zeker is}{de koorts gestegen}\\

\haiku{Toen ik de oogen weer,.}{opende lag mijn hoofd in den}{schoot van een meisje}\\

\haiku{Intusschen is het,.}{tijd geworden om aan den}{uitgang te denken}\\

\haiku{Het was al laat in,.}{den nacht geworden voor ik}{ter ruste kon gaan}\\

\haiku{Behagelijk schik.}{ik mij terecht tusschen de}{geurige halmen}\\

\haiku{Waar het schild uithing van,.}{een goudgele krakeling}{stapte ik binnen}\\

\haiku{{\textquoteright} De boer dankte en,}{draafde op zijn dunne beenen}{naar den kant heen dien}\\

\haiku{Valentijn keek een,.}{beetje bedroefd naar den kluif}{dien ik hem aanbood}\\

\haiku{Mijn makker diept een.}{dik geknopten stok op uit}{wat ouden rommel}\\

\haiku{{\textquoteleft}Valentijn,{\textquoteright} zeg ik, {\textquoteleft}.}{speel ons een liedje en noodig}{den nacht tot den dans}\\

\haiku{We buigen voorover,.}{en met de handen hollen}{we een grafkuil uit}\\

\haiku{Dadelijk daarop}{echter hooren wij achter}{ons in de herberg}\\

\haiku{Zoo ten minste werd,.}{er gemompeld en sedert}{bleef die plaats geschuwd}\\

\haiku{Mijn hart klopt luid als,}{van een kleinen jongen en}{waarlijk voel ik mij}\\

\haiku{{\textquoteright} Dan fluistert ze met,:}{bevende lippen terwijl}{ze mij bang aanziet}\\

\haiku{Ik moet uit zien te,;}{vinden waar hij zijn hoofdpijn}{uit ligt te slapen}\\

\haiku{Nieuwsgierig sloop ik,.}{naderbij en maakte een}{slip los van het pak}\\

\haiku{{\textquoteright} Dezelfde kamer,,.}{op het meer uitkomend werd}{mij aangewezen}\\

\haiku{Ik sprong van den stroozak,,.}{dankte hem en zwoer hem zijn}{hulp te vergelden}\\

\haiku{{\textquoteright} Van verwondering.}{rolde hij met ton en al}{ondersteboven}\\

\haiku{hij zoo ruimschoots had,.}{genoten hem nog door het}{brein gemoesseerd heeft}\\

\haiku{Ik wilde hem naar:}{aanleiding er van een vraag}{stellen en zeide}\\

\haiku{{\textquoteright} riep hij ge\"ergerd,.}{en haalde medelijdend}{de schouders op}\\

\haiku{{\textquoteleft}Aha, die Mathilde,,.}{d'Almonde die feeks en dat}{dekselsche manwijf}\\

\haiku{Een stilte, en dan,.}{de trapleer die met zwaren}{stap wordt bestegen}\\

\haiku{{\textquoteright} {\textquoteleft}En {\'\i}k zat juist aan;}{een anderen held uit de}{oudheid te denken}\\

\haiku{Toch verwijt me mijn,.}{geweten geen oogenblik}{dat ik bedrog pleeg}\\

\haiku{Uit straf er voor werd}{ik dadelijk daarop aan}{de mouw getrokken}\\

\haiku{{\textquoteright} Verscheen mijnheer de.}{Pomponne en maakte een}{deftige buiging}\\

\haiku{De dorpsklerk leest nu,;}{het stuk voor en als hij er}{onder gezet heeft}\\

\subsection{Uit: De kleine Rudolf}

\haiku{En nu klettert een,,}{glazen deur open je stommelt}{een paar trapjes op}\\

\haiku{Dat de kussenkast.}{ze in zijn oud-eiken}{binnenste berge}\\

\haiku{je afstamt, en dat,.}{je een man bent die in zijn}{jeugd gestudeerd heeft}\\

\haiku{Op een morgen stormt.}{Koba met een hoogrood hoofd}{mijn kamer binnen}\\

\haiku{Neen, waarlijk, slechts goed,.}{over doden en oom Jakob}{stierf niet lang daarna}\\

\haiku{En dan plotseling,.}{bruist een stormvlaag nader doet}{de ruiten trillen}\\

\haiku{Verschrikt sta ik stil,.}{buiten terwijl ik langzaam}{tot bezinning kom}\\

\haiku{Daar zijn er op mijn,,.}{hoedrand gevallen op mijn}{knie\"en mijn handen}\\

\haiku{{\textquoteleft}Ach veel te weten,{\textquoteright}, {\textquoteleft}.}{zucht ze mismoedigik ben}{vroeg van huis gegaan}\\

\haiku{Ernstig inspecteert!}{hij de spijzen en snort als}{een oud spinnewiel}\\

\haiku{Natuurlijk, dat hij,.}{zwijgt erover dat hij me de}{weg heeft gewezen}\\

\haiku{Roomwit is Clara,.}{en met een waasje mosgroen}{om de natte muil}\\

\haiku{dat als een hamer.}{op zijn aambeeld in mijn borst}{begint te kloppen}\\

\haiku{Een wildvreemd meisje,,,}{vraagt je op bezoek neen zeg}{je onmogelijk}\\

\haiku{Het blijkt een nauwe,.}{zitplaats als we er ons in}{hebben gewrongen}\\

\haiku{Maar de ander heeft,.}{al naar mijn hand gegrepen}{krachtig schudt hij die}\\

\haiku{Tussen ons en het.}{doelwit spreidt zich zij{\"\i}g het}{groene gazon uit}\\

\haiku{Ik zal beginnen,,.}{natuurlijk met het schiettuig}{dat het kleinste is}\\

\haiku{Nu schrik je bijna,;}{van de ruk waarmee ze de}{hand naar het oor brengt}\\

\haiku{En wat blijft me dan,.}{verder nog over nu alles}{uitgesproken is}\\

\haiku{Ik begin met de,:}{aanhef nadat ik een blad}{glad gestreken heb}\\

\haiku{En ik, die ze me.}{altijd als onnoemelijk}{rijk heb voorgesteld}\\

\haiku{Als ze me uitlaat,,;}{glimlacht ze tegen me god}{weet waarom dankbaar}\\

\haiku{{\textquoteright} En nu haak je de.}{koperen sloten van de}{statenbijbel open}\\

\haiku{Bij het weggaan merk.}{ik dat ze een hoogrode}{kleur heeft gekregen}\\

\haiku{Ademloos laat ik bij;}{een straathoek mijn hartklop tot}{bedaren komen}\\

\haiku{Terwijl de wereld:}{van je vrijheid vlak bij je}{wenkt door de ruiten}\\

\haiku{Op een ongelijnd:}{blad schrijf ik hem uit in mijn}{fraaiste schoonschrift}\\

\haiku{Ik beloof haar een,.}{hulp bij de arbeid die {\'\i}k}{zal bekostigen}\\

\haiku{toch kunnen er in.}{de kortst mogelijke tijd}{met je gebeuren}\\

\haiku{Dat ze om de schat,.}{moet denken waarvan ze de}{draagster mag wezen}\\

\haiku{{\textquoteright} en ik zie Martha.}{met een beweging van schrik}{achteruit wijken}\\

\haiku{Nu wil ze, dat ik.}{zelf het onderwerp van het}{gesprek zal worden}\\

\haiku{{\textquoteright} Door het aanzetten.}{van de motor gaat me haar}{antwoord verloren}\\

\haiku{En dan breekt hij in,.}{een stille steeg de stenen}{los en gaat graven}\\

\haiku{Terwijl ze in de,,,}{houding blijven staan kijk ik}{neer op ze ja n\'u}\\

\haiku{Goed, ja, uitstekend,.}{en binnen twee jaar had ik}{mijn examen gedaan}\\

\haiku{{\textquoteright} {\textquoteleft}Goed,{\textquoteright} zeg ik, \'e\'en woord,.}{slechts maar met het brandmerk van}{mijn eerste leugen}\\

\haiku{Mogelijk dat ik.}{daarvoor op een emmer zou}{hebben te klimmen}\\

\haiku{Het is alles een,,.}{beeld slechts nevel geschetst in}{een paar wolkstrepen}\\

\section{Jan de Liefde}

\subsection{Uit: Uit drie landen}

\haiku{hoe spoedig kan men....{\textquoteright}}{mij berooven van mijn jeugdig}{leven en dan zou}\\

\haiku{De Lollards waren '.}{overt algemeen een stil}{en lijdzaam volkje}\\

\haiku{maar ging hij met list.}{en overleg te werk om zich}{den weg te banen}\\

\haiku{Misschien is er op.}{een of andere manier}{wel iets aan te doen}\\

\haiku{Wel verbazend, welk,!}{een zware ijzeren deur}{is dat kapitein}\\

\haiku{Nu draagt hij de kroon,.}{der heerlijkheid en niemand}{kan hem die ontrooven}\\

\haiku{Daarbij kwam, dat wij.}{volstrekt niet wisten wat er}{van ons worden zou}\\

\haiku{Na haar kwamen er,.}{nog andere vrouwen die}{hetzelfde deden}\\

\haiku{{\textquoteright} En nu verhaalde.}{hij den moord in al zijne}{bijzonderheden}\\

\haiku{De opzichter nam,.}{hiermede genoegen want}{hij kon niet anders}\\

\haiku{In plaats van banken;}{waren er eenige planken}{op blokken gelegd}\\

\haiku{Daarom drong hij er.}{nu des te meer op aan dat}{de boom vallen moest}\\

\haiku{{\textquotedblright} - Zijn smeeken hielp hem,.}{niets hij werd gedood en in}{stukken gehouwen}\\

\section{P. van Limburg Brouwer}

\subsection{Uit: Romantische werken. Deel 2. Diophanes}

\haiku{Hij  worstelde,,,,,.}{liep jaagde schoot zwom evenals}{al zijne makkers}\\

\haiku{Men doet dat bij ons, -;}{zoo niet en vooral men doet}{het niet te Sparta}\\

\haiku{Dit tooneel is  voor.}{het vervolg belangrijker}{dan gij misschien denkt}\\

\haiku{- Wel zeg mij dan eens,,?}{jonge vriend wat keurt gij het}{beste voor den mensch}\\

\haiku{Ongetwijfeld, het.}{zijn de stappen van iemand}{die haastig voortgaat}\\

\haiku{Hoe weet ik dan of,,?}{dit een meisje was of een}{godin of een nimf}\\

\haiku{Hoe 't ook zij, de.}{uitkomst bewees dat ik mij}{niet bedrogen had}\\

\haiku{Zoodra zij mij,.}{zag wees zij mij den weg dien}{ik te volgen had}\\

\haiku{Ik bracht er deze}{woorden zoo schielijk uit dat}{zij niet in staat was}\\

\haiku{Eene godin was zij,.}{niet maar toch ook geene vrouw zooals}{andere vrouwen}\\

\haiku{doch ik bemerkte.}{aldra dat hij zich uit de}{voeten gemaakt had}\\

\haiku{Zij, gij weet wel, heeft,.}{er genoeg maar die zitten}{altijd achter slot}\\

\haiku{in een woord, hij had, -.}{een zware kou gevat zoodat}{hij niet kon uitgaan}\\

\haiku{Dat ik de lier van,,.}{Timotheus en de Scias47}{moest zien begrijpt zich}\\

\haiku{Het heeft ook zijne,....}{lasten eene jonge vrouw zoowel}{als zijne lusten}\\

\haiku{Pedonomus,  ,!}{hier een uwer kweekelingen die}{uit het gelid loopt}\\

\haiku{Maar nu die man, dacht,!}{ik alweder verliefd op}{de schoone Gorgo}\\

\haiku{Leon had het mij,;}{niet verzocht maar hij had mij}{toch zijn nood geklaagd}\\

\haiku{en zou een jongen?}{er dan zijne grillen niet}{aan opofferen}\\

\haiku{- Ja, zie nu eens, ik.}{wed dat gij nog niet eens weet}{wat hier gebeurd is}\\

\haiku{- Maar wat, vroeg ik nu,?}{heeft u bewogen u aan}{mij over te geven}\\

\haiku{- Wat, zeide hij op,.}{een vrij hoogen toon wat anders}{dan hunne ontrouw}\\

\haiku{Gij zoudf ons van een, ....}{onberekenbaren dienst}{kunnen zijn en zelf}\\

\haiku{Hoe dit zij, er was.}{niet anders op dan mij in}{mijn lot te schikken}\\

\haiku{Jongenlief, dat gaat.}{zoo gemakkelijk niet in}{die nauwe straten}\\

\haiku{Er behoefde hier.}{naar geen balsem of stlengis}{gezocht te worden}\\

\haiku{Ik moet echter tot,}{mijne schande bekennen}{dat ik zoodra}\\

\haiku{- Een slaaf, ja, maar dat.}{zegt hier wat anders dan in}{andere steden}\\

\haiku{Hier was het in den.}{beginne nog moeilijker}{gehoor te krijgen}\\

\haiku{maar gij moet nog een ().}{bewijs van den apographeus}{controleur hebben}\\

\haiku{Evenwel er waren,;}{er daar behoefde ik niet}{aan te twijfelen}\\

\haiku{Daar is die oude,.}{zaniker weer hoorde}{ik er een zeggen}\\

\haiku{Eensklaps, als door schrik,,:}{verbijsterd vloog zij op en}{vroeg mij hijgende}\\

\haiku{- Maar bij Hercules,,.}{zeide Polycles nog eens}{wacht hen gerust af}\\

\haiku{Bij Hercules, ik.}{geloof dat Plutus er zelf}{in gekropen is}\\

\haiku{- Zoo, ik dacht dat hier.}{in Athene allen gelijk}{waren voor de wet}\\

\haiku{- Ja, en waarom niet,,.}{ten minste om kwaad te doen}{hernam ik schielijk}\\

\haiku{Hij zal natuurlijk;}{de welwillendheid zijner}{moei niet versmaden}\\

\haiku{Gij verwondert u,;}{zeker daarover ten hoogste}{wijze Demeas}\\

\haiku{Als gij hem spraakt, zoudt.}{gij nooit zeggen dat hij zulk}{een groot wijsgeer is}\\

\haiku{Voor den ingang der.}{Academie stond een altaar}{aan Eros geheiligd}\\

\haiku{Maar is niet Plato's?}{wijsbegeerte geheel op}{de Liefde gegrond}\\

\haiku{Ja, wie weet of gij,,}{nog niet zooals zij zult zeggen}{dat ik wel dwaas ben}\\

\haiku{die heeren vliegen,.}{over de rivieren of zij}{ooievaars waren}\\

\haiku{Maar (en ik ben het)}{aan mij zelven verplicht er}{dit bij te voegen}\\

\haiku{Ik had, ja, mij niet.}{zoo onvoorbereid daarheen}{moeten begeven}\\

\haiku{mij ging alleen het,.}{resultaat aan en dat werd}{ik spoedig gewaar}\\

\haiku{Gelukkig verstond.}{ik dien blik en had ik tijd}{mij te herstellen}\\

\haiku{en een vreemdeling.}{wil ik altijd gaarne eene}{inlichting geven}\\

\haiku{de Eleusinia te willen,.}{bijwonen127 en mij zelfs te}{laten inwijden}\\

\haiku{(ik wilde hier iets,).}{anders zeggen maar ik had}{er den moed niet toe}\\

\haiku{Honderdmaal dacht ik:}{om het gezegde van den}{wijzen Aristippus}\\

\haiku{Honderdmaal stond ik}{op het punt tot hem te gaan}{en hem te smeeken}\\

\haiku{Verwonderd over dit,,,:}{gezicht neem ik den brief scheur}{het koord los en zie}\\

\haiku{Ik vlieg de kamer,,.}{uit roep de huishoudster vraag}{haar naar Lagisca}\\

\haiku{Uit vrije keuze heb.}{ik u gekozen boven}{honderd anderen}\\

\haiku{- Elpinice, die, -.}{gij reeds vroeger op Creta}{bemind hebt schreef zij}\\

\haiku{- Vergeef, vergeef mij,,}{zeide hij bedenk dat ik}{om zoo te zeggen}\\

\haiku{Ik herken er u.}{aan en zou niet anders van}{u verwacht hebben}\\

\haiku{- Dat zal moeielijk zijn,,,.}{Lamprias zeide ik want}{ik heb geen huis meer}\\

\haiku{De wet der natuur.}{wil dat de sterkere meer}{heeft dan de zwakke}\\

\haiku{Zijn doel is vrijheid,.}{en geluk zooals dat van den}{onrechtvaardige}\\

\haiku{- Lampis is er nog,,.}{niet zeide mijn geleider}{maar hij komt zeker}\\

\haiku{- De stem die u dat,,.}{zeide sprak nu Athenagoras}{was die der godheid}\\

\haiku{Mij dacht, eene vrouw als.}{Elpinice kon toch zoo}{onbekend niet zijn}\\

\haiku{gij hadt mij gezegd,.}{waar gij woondet en ik ben}{nooit bij u geweest}\\

\haiku{Neen, Diophanes, stel,.}{u gerust gij hebt u niets}{te verwijten}\\

\haiku{, zoo weinig met u,;}{bekend wel gewacht hebben}{u te waarschuwen}\\

\haiku{hervatte hij met.}{eene zachte stem en bijna}{bedeesde houding}\\

\haiku{Vraag mij nu niet hoe;}{de rechtvaardigheid eene kunst}{kan genoemd worden}\\

\haiku{20Er is op Creta,,.}{zegt Plutarchus een beeld van}{Zeus zonder ooren}\\

\subsection{Uit: Romantische werken. Deel 1. Een leesgezelschap te Diepenbeek. Een ezel. Eenig speelgoed}

\haiku{het verbond, reeds daar,.}{gemaakt te hernieuwen en}{te bevestigen}\\

\haiku{Ik, voor mij, zou het;}{dwaasheid vinden nu reeds aan}{eene plaats te denken}\\

\haiku{maar, al gaat het in,.}{Utrecht of in den Haag beter}{dat helpt ons hier niet}\\

\haiku{De boeken, die ter,;}{tafel moeten komen zijn}{mij al gezonden}\\

\haiku{En, ik weet niet of;}{gijlieden allen zulke}{sterke bollen hebt}\\

\haiku{Als ik in uw plaats,,.}{was baas Hartman dan gaf ik}{hem het roer maar over}\\

\haiku{Ik laat nu daar dat:}{de drie-eenheid er met geen}{woord in vermeld wordt}\\

\haiku{In het tweede deel,}{werd nu de zaak omgekeerd}{en aangetoond welk}\\

\haiku{- Wie uwer, zeide hij,?}{onder anderen zou nog}{kunnen aarzelen}\\

\haiku{om malkanderen,}{te verdragen in liefde}{en te behouden}\\

\haiku{Het groote onderscheid,.}{was of zij wakker waren}{dan of zij sliepen}\\

\haiku{- Hoe dat, vader, vroeg,?}{nu de burgemeesterske}{was het dan niet goed}\\

\haiku{des kapiteins hand,:}{hartelijk en zeide zijn}{lach versmorende}\\

\haiku{- Met uw permissie,,.}{hernam de kapitein dat}{heb ik niet gezegd}\\

\haiku{Ik zeg maar dat die,.}{leer God niet verheerlijkt maar}{hem oneer aandoet}\\

\haiku{Wij willen daarom.}{het lieve meisje echter}{niet veroordeelen}\\

\haiku{Ik heb u zelfs nog,.}{verscheiden vragen te doen}{voor wij zoover zijn}\\

\haiku{Vreest gij niet dat gij.}{uzelven van het koninkrijk}{Gods zult uitsluiten}\\

\haiku{- Dus het uitkiezen.}{en het niet uitkiezen hangt}{alleen van God af}\\

\haiku{Eerst hebt gij alleen.}{gezegd dat gij het niet met}{Hellenbroek eens waart}\\

\haiku{Ik wil nu niet eens {\textquoteleft}{\textquoteright}.}{vragen of dat woordgodsdienst}{hier wel gepast is}\\

\haiku{En nu, vrienden, nu,,.}{zooals ik straks zeide eens over}{een anderen boeg}\\

\haiku{Schielijk nam nu de:}{chirurgijn-diaken het}{boekje en zeide}\\

\haiku{{\textquoteright} Dat gaat best. Geef mij ', ';}{t boekske maar mee ik zal}{t voor u wel doen}\\

\haiku{In hoeverre die,.}{conclusie juist was zullen}{wij nu daarlaten}\\

\haiku{, maar daarom is het.}{niet minder zooals ik u daar}{voorgelezen heb}\\

\haiku{- Wat niet mogelijk,!}{is bij de menschen dat is}{mogelijk bij God}\\

\haiku{Nu, hernam de heer,}{Van Groenendaal het doet mij}{genoegen te zien}\\

\haiku{voor zooveel gij dit,}{aan de armen zult gedaan}{hebben voor zooveel}\\

\haiku{- Ja, wel een booswicht,,.}{hervatte Willem en een}{listige booswicht}\\

\haiku{Gij begrijpt dat ik;}{in het eerst als verpletterd}{was van verbazing}\\

\haiku{En wat lezen wij?}{in den tweeden brief aan de}{Thessalonicensen}\\

\haiku{en op dat drietal.}{prijkte bovenaan de naam}{van Jacobus Klos}\\

\haiku{Was dit alles niet?}{meer dan het gevolg van Gods}{eeuwig raadsbesluit}\\

\haiku{Evenwel, zoo verblind,.}{was hij niet of hij begon}{het zelf in te zien}\\

\haiku{- Hoe komt het toch dat?}{die Klos bovenaan op het}{drietal bij u staat}\\

\haiku{Dat zal een ander;}{leventje zijn dan met dien}{dominee Wilbrink}\\

\haiku{hij was vriendelijk,.}{genoeg maar ik kon het zoo}{niet met hem vinden}\\

\haiku{Hij begreep dus ook.}{eens een anderen toon te}{moeten aannemen}\\

\haiku{Alle schrift is van,,;}{God ingegeven dat staat}{er gij hebt gelijk}\\

\haiku{wie zou in hare?)}{omstandigheden anders}{gehandeld hebben}\\

\haiku{Hartman en Kootje,.}{gaan naar Amerika met de}{afgescheidenen}\\

\haiku{Ik heb u eenige.}{gewichtige tijdingen}{mede te deelen}\\

\haiku{Ik ontving hem, om,;}{u de waarheid te zeggen}{niet heel vriendelijk}\\

\haiku{Doch wees voor Willem.}{niet bevreesd die zal den Heer}{niet ontrouw worden}\\

\haiku{Het was haar wel niet.}{ongewoon den kapitein}{te hooren lachen}\\

\haiku{Want als ik Cesar,.}{was dan zou ik mijzelven}{wel kunnen helpen}\\

\haiku{Van welk een geest denkt,,?}{gij dat Caligula de}{Cesar bezield is}\\

\haiku{Gij vraagt immers niet ', '.}{naart geen hij mag maar naar}{t geen hij kan doen}\\

\haiku{Palestra kwam om.}{het noodige voor den maaltijd}{gereed te zetten}\\

\haiku{Geduld dus, lieve,;}{jongen bedenk dat gij het}{om mijnentwil doet}\\

\haiku{Maar weldra begon.}{ik ernstig over mijn eigen}{lot na te denken}\\

\haiku{Hij onderzocht ook.}{nauwkeurig naar de plaats van}{mijne geboorte}\\

\haiku{{\textquoteright} De oude man vroeg,.}{nu of ik wel mak was en}{geen kribbebijter}\\

\haiku{Dit gedeelte van.}{mijne geschiedenis is}{wezenlijk tragisch}\\

\haiku{De geheele buurt,.}{wordt bijeengeroepen om}{het wonder te zien}\\

\haiku{- een vernieuwd en nog.}{luider gelach was al het}{antwoord dat ik kreeg}\\

\haiku{- Doch goed, ik zal mij,,!}{zelven wel helpen wacht maar}{wreede Palestra}\\

\haiku{Palestra deed mij, -.}{beloften en kuste mij}{en ik bleef een ezel}\\

\haiku{Vrienden, om zijne,.}{genietingen mede te}{deelen heeft hij niet}\\

\haiku{De eerste liefde,,.}{is nog zoo gij het noemt de}{liefde der onschuld}\\

\haiku{Wat zeg ik, het is.}{niet eens noodig het geprezen}{werk zelf te lezen}\\

\subsection{Uit: Romantische werken. Deel 3. Charicles en Euphorion. Grillus}

\haiku{want hij zou het voor.}{geen geld uit zijne handen}{gegeven hebben}\\

\haiku{Euphorion dit,:}{bemerkende zeide op}{een lachenden toon}\\

\haiku{de getrouwe, de,.}{deelnemende vriend hij was}{de gelukkige}\\

\haiku{In \'e\'en woord, lieve,.}{Charicles de Platonist}{werd een Cynicus}\\

\haiku{Het klamme zweet brak.}{Hermotimus eensklaps bij}{deze woorden uit}\\

\haiku{Polydorus nog het.}{best geschikt om de zaken}{te vereffenen}\\

\haiku{Lyde trippelde,.}{naar het bestje toe en bracht}{haar in mijn vertrek}\\

\haiku{Charicles vond bij,:}{Polydorus hetgeen hij lang}{vergeefs gezocht had}\\

\haiku{Zij nam het stoute,;}{besluit het haren vader}{bekend te maken}\\

\haiku{- Zoo, zoo, dan neemt gij!}{maar vast de eerste klasse}{voor u zelve in}\\

\haiku{Op deze wijze;}{nam het gesprek eene eenigszins}{andere wending}\\

\haiku{Uwe ziel, het beste,.}{van uw aanzijn behoort niet}{tot deze wereld}\\

\haiku{Of gij bij deze,.}{ruiling wint of verliest kan}{ik niet bepalen}\\

\haiku{maar zoo ik u zeg,;}{gij behoeft niet te vragen}{of gij dat doen moogt}\\

\haiku{Sosandra begon met.}{eene geveinsde koelheid en}{onverschilligheid}\\

\haiku{Met dat al droeg men,.}{nauwkeurig zorg zijn geduld}{niet uit te putten}\\

\haiku{Het genot, dat hij,.}{als het hoogste goed stelde}{had hem bedrogen}\\

\haiku{- Ziedaar dan het loon,!}{der liefde ziedaar de prijs}{mijner weldaden}\\

\haiku{Wat zij op weg nog,;}{verder gesproken hebben}{is niet zoo bekend}\\

\haiku{, en door Charicles.}{lang gevreesde oogenblik}{was eindelijk daar}\\

\haiku{- Zult gij eene slavin?}{tegen hare meesteres}{laten getuigen}\\

\haiku{Gij hebt, zeide hij,, -.}{met moeite voortgaande mij}{het leven gered}\\

\haiku{Maar was het dan een,?}{zoo noodzakelijk een zoo}{verdienstelijk werk}\\

\haiku{Charicles maakte,:}{van deze gelegenheid}{gebruik en zeide}\\

\haiku{- Die echtgenoot is,.}{Charicles van Athene de}{zoon van Glaucias}\\

\haiku{Wij hebben beiden.}{een tegenovergestelden}{weg ingeslagen}\\

\haiku{Het gezegde dient.}{alleen om u  tot de}{orde te roepen}\\

\haiku{Hebben wij er geen,.}{eer van wij hebben ook geen}{schande te vreezen}\\

\haiku{Het komt zelfs niet eens.}{te pas over het een of het}{ander te denken}\\

\haiku{52De twee eerste,,.}{beroemde dichteressen}{zijn genoeg bekend}\\

\section{Manuel van Loggem}

\subsection{Uit: Insecten in plastic}

\haiku{De kleuren waren,.}{hard en scherp net tegen de}{werkelijkheid aan}\\

\haiku{{\textquoteleft}Je had niet zo gauw.}{je reserves van weerstand}{moeten uitputten}\\

\haiku{Het was alles zo.}{onzinnig en eigenlijk}{onbegrijpelijk}\\

\haiku{Dit schilderij is{\textquoteright}.}{mooi en wilde daarmee mijn}{weerstand verdoven}\\

\haiku{Als ik maar wist dat.}{het donker was buiten zou}{me dat rust geven}\\

\haiku{Het is zelfstandig.}{geworden en ook Richard}{is onderhorig}\\

\haiku{Ik vind het mooi,{\textquoteright} zei.}{de jongen en ik wist dat}{hij het  meende}\\

\haiku{Het komt toch alleen.}{aan op de ontroeringen}{die een kunstwerk wekt}\\

\haiku{{\textquoteright} Hij had een droge.}{stem en sprak of hij me een}{lesje wou geven}\\

\haiku{Als je het ziet dan.}{voel je dat een moeder haar}{kinderen liefheeft}\\

\haiku{Wat bedoelde hij?}{er mee dat het de laatste}{keer zou kunnen zijn}\\

\haiku{Wie  die ander.}{was kon ik me achteraf}{nooit herinneren}\\

\haiku{Dat ik de tijd niet.}{kon indelen vond ik niet}{meer zo hinderlijk}\\

\haiku{Soms betastte ik}{mijn gezicht om mij ervan}{bewust te worden}\\

\haiku{Dageraad 1946 - Het -.}{volksgezicht 1948 De tocht van}{de dronken man 1950}\\

\haiku{Het werk van Gerrit -.}{Achterberg 1950 Inleiding}{tot het toneel 1951}\\

\section{Harie Loontjens}

\subsection{Uit: Wieker Lui}

\haiku{Zou dit misschien een}{reminiscentie zijn aan}{een oud volledig}\\

\haiku{Meint d'r mesjiens of 't}{plezerig is veur miech um}{vaan oet de winkel}\\

\haiku{Iech snapde neet gaw ', '.....}{gen\'og watr meinde meh toen}{zagr eur water}\\

\haiku{Ze m\^os strak toch eve,.}{denao touw goon zoe gaw es}{Tonia toes waor}\\

\haiku{Jao, meh zoe'ne nuije, '.}{dee w\`et alweer get miejer}{esnen aandere}\\

\haiku{'n maog vaan e '.}{paar hoezer weijer m\`etne}{verbranden errem}\\

\haiku{Meh noe d'r toch heij,.}{zeet noe hoof iech ouch neet nao}{Uuch touw te komme}\\

\haiku{Meh ze wis ouch wie ' '.}{t oetzaog in aander}{tije vaant jaor}\\

\haiku{Ze zaog altied '.}{oet wiet iewig leve}{en heel h\"a\"or good hum\"or}\\

\haiku{Nein, es 'r 't k\^os, '.}{veerdig kriege d\'a\'ann hoes}{heij in de straot}\\

\haiku{Eder hoeshawwe.}{kraog zie kruus en noe ouch}{mesjiens weer dat vaan h\"a\"om}\\

\haiku{{\textquoteleft}Geer.... gere s\^okker?.... '.}{h\`et Waor goddaank gei}{resep wat koed k\^os}\\

\haiku{rege en zoe get '.}{kaw boeste besj eemp veur bis veur}{t kaw te neume}\\

\haiku{had 't zellef es '}{kraol gez\'onge en nog}{waort ein vaan}\\

\haiku{Quand le J\'esus venait,.....}{au monde Le bon sauveur}{au barbe blonde}\\

\haiku{'n Sjoen m\`es hadde,.}{ze gelierd z\`esstummig m\`et}{de jonges debij}\\

\haiku{Eigenaordig.}{tot zoe wienig Mastreechse}{keersleedjes waore}\\

\haiku{Dee had al koffie.}{gehad m\`et de zengers en}{z'n stumpkes al op}\\

\haiku{boe ze gans allein.}{st\'onge m\`et um hun niks es}{blaanke witheid}\\

\haiku{Dat waor veur d'n}{iersten erreme mins dee}{sanderendaogs}\\

\haiku{En noe kraog eder - -.}{nog altied vaan oonder de}{serv\`et e st\"okske}\\

\haiku{De winter kump en,....}{deft us beve De winter}{kump en dao is noed}\\

\haiku{{\textquoteright} meh de res had 't:}{leedsje euvergenome}{en z\'ong weijer}\\

\haiku{Z'n sjotel broed zat ' ' '.}{r neer en op zien tiene}{leepr naot leech}\\

\haiku{Zien femilie woort}{gezegend en gaof v\"a\"ol}{vaan h\"a\"or kinder aon}\\

\haiku{Meh nein, daan zouwe....}{ze toch ouch tegen  eur}{vrijaasj zien gewees}\\

\haiku{Me geit 'ne jonge '....}{geistelik aofhole en}{d\'at ist veurnaomste}\\

\haiku{De gouwen tieger{\textquoteright} ' '.}{waort daonaon}{dr\"okde vaan belang}\\

\haiku{Eder muzikant en}{zenger k\^os dao z'n b\"ongskes}{inwissele veur}\\

\haiku{In 383 verplaatste hij,.}{zijn zetel naar Maastricht waar}{hij in 384 stierf}\\

\haiku{Aanvankelijk heeft.}{het gestaan in de Oude}{Minderbroederskerk}\\

\haiku{bij Roermond bevindt.}{zich een bron welke genoemd}{wordt naar Sint Servaas}\\

\section{Jac. van Looy}

\subsection{Uit: Feesten}

\haiku{Met hollige oogen:}{lag juffrouw Broense in de}{kamer te kijken}\\

\haiku{och, juffrouw verlost,.}{U me van een koppie ze}{benne met suiker}\\

\haiku{Hij knikte naar den,:}{bedste\^e-hoek vroeg met zijn}{korte beveelstem}\\

\haiku{{\textquoteleft}Leukerd, jij begrijpt,,?}{me geef jij me eens de hand}{hoe oud b\`e-je}\\

\haiku{jou trommel is ook,,{\textquoteright}.}{nog niet geborsten nog lang}{niet zei oome Piet}\\

\haiku{{\textquoteright} De Bruid, verheerlijkt,,:}{stijf op haar stoel groette toen}{het leven uit was}\\

\haiku{De kachel was ook, ', '?}{me\^e verhuisdt was warmpies}{genogt wast niet}\\

\haiku{je kan nooit weten,{\textquoteright}....}{met die presenten gaf zij}{zich zelve antwoord}\\

\haiku{wij weten er hier,{\textquoteright}.}{wel raad me\^e snaakte oome}{Willem goedhartig}\\

\haiku{{\textquoteright} {\textquoteleft}Een goed geloof en,.}{een kurke ziel dan drijft een}{mensch altijd boven}\\

\haiku{O, die draaiende,;}{liedjes de kamer begon}{er van te draaien}\\

\haiku{{\textquoteleft}da's mooi gezeid,{\textquoteright} gaf:}{hij dadelijk toe en me\^e}{zong de kamer}\\

\haiku{Soms had ze geloofd.}{dat de lage zolder kwam}{ne\^erdreunen op haar hoofd}\\

\haiku{Ze hield de hand aan, {\textquoteleft}.}{haar voorhoofddie warmte gaat}{nou we\^er beginnen}\\

\haiku{maar h\`un zou dat ten.}{eeuwigen dage kwalijk}{worden genomen}\\

\haiku{{\textquoteleft}Als ie opstaat, valt,{\textquoteright},;}{ie kletste oome Willem}{zijn bro\^er bedoelend}\\

\haiku{och, och, ze wist niet,,,....}{waar te kijken het werd haar}{paars voor de oogen foei}\\

\haiku{Onder de starre.}{lamp verhieven de bazen}{zich met boller oogen}\\

\haiku{hij at met smaak en,.}{een appel in zijn zak voor}{Grietje dat had hij}\\

\haiku{de gladde flesschen,;}{kwamen want iedereen had}{er aardigheid in}\\

\haiku{Weelsen werd v\`o\`or door.}{Willem geholpen aan het}{zoeken naar zijn jas}\\

\haiku{{\textquoteright} terugzakte langs}{den zeepmast en zoo smachtend}{was blijven kijken}\\

\haiku{de gouvernante,,.}{gehoorzaam kwam waarschuwend}{een eindje aan}\\

\haiku{mannen met sikken,;}{en uit den nek geschoren}{haar oude zeelui}\\

\haiku{een sieraad van een:}{meid heeft ze zoo mooi lang zwart}{als een gitana}\\

\haiku{Toch zijn er ook wel;}{straten waar het tot diep-in}{ellendig zwart is}\\

\haiku{koekenbakkers en;}{de komenijs die glazen}{vol kokinjes heeft}\\

\haiku{Maar er zijn er nog,:}{wakker genoeg hoor hoe ze}{leven maken}\\

\haiku{van uit de bedste\^e,}{zijn geeuwen zacht klaagde en}{we\^er wat later kwam}\\

\haiku{iedereen was toch,;}{zoo goed voor hem de klantjes}{vergaten hem niet}\\

\haiku{dat doet goed van een, '.}{vreemde te ervarent}{geeft wel een dag steun}\\

\haiku{Ja, de zon zou de.}{jongen branden tusschen de}{koren-velden}\\

\haiku{{\textquoteright} Niet op zijn gemak ',:}{voor dat gezicht int bed}{meende hij verder}\\

\haiku{En Juffrouw Weelsen.}{was toch wel blij dat Broense}{er al geweest was}\\

\haiku{Zij waren als met.}{blauwe verf besmeerd en in}{doorsopte kle\^eren}\\

\haiku{En hij had geen duur,.}{in zich gehad alles moest}{nog worden geboekt}\\

\haiku{Of ze die passen?}{nog maakte en de handen}{hield boven het hoofd}\\

\haiku{{\textquoteleft}hoor es, moe,{\textquoteright} ook die:}{haar hand en allebei de}{handen wippend zoo}\\

\haiku{Hoe ze eindelijk.}{morgen voor de melange}{zouden klaar komen}\\

\haiku{je weet geen raad, ik '.}{houd het er voort loopt nog}{eens falikant uit}\\

\haiku{{\textquoteleft}Hoor me nou zoo'n schaap,{\textquoteright},:}{an mopperde Antoon en}{Geertrui vervolgde}\\

\haiku{dan zegt-ie {\textquotedblleft}bijen{\textquotedblright}:,:}{en zeg je dan ik zie er}{geen een dan zegt hij}\\

\haiku{{\textquoteright} Gestookt onder het:}{gevlei harer stem drongen}{de woorden aan}\\

\haiku{{\textquoteright} {\textquoteleft}Neen, zoo ver kwam het,,.}{niet eens ze wou niet met hem}{uitgaan dat was het}\\

\haiku{{\textquoteleft}Weet je wat, Antoon,{\textquoteright}, {\textquoteleft};}{zei Weelsen toenweet je wat}{je nu eens doen moest}\\

\haiku{Hoe lang was het niet;}{geleden dat hij er het}{laatst op had gespeeld}\\

\haiku{Andere jaren;}{had de boer nog wel eens een}{maat me\^egegeven}\\

\haiku{het houtje belegd,.}{met een hardgruizige laag}{te voorschijn haalde}\\

\haiku{Een felle schichting,:}{lichtte door de oogen van den}{jongen toen hij zei}\\

\haiku{Thijs,{\textquoteright} riep hij hem toe, {\textquoteleft}?}{toen hij onder bereik was}{heb je nog drinken}\\

\haiku{of er hier katten,,.}{hadden liggen stoeien hei}{zoo'n plok van geweld}\\

\haiku{Toen had 't Wimme.}{geleken dat zijn lot zou}{gaan  beteren}\\

\haiku{{\textquoteright}... Tusschen de acht en,?}{negen honderd gulden wil}{je daar goed voor zijn}\\

\haiku{doffer oogelden.}{de bloempjes weg onder de}{vlamming van de zeis}\\

\haiku{Gisteren had ik ',}{een oogenblik dat ikm}{zag maar daarna zag}\\

\haiku{een boterwarmer, '.}{glom aan de andre kant}{evenver vant brood}\\

\haiku{{\textquoteright} {\textquoteleft}Nu, opsnijen kan,,{\textquoteright}, {\textquoteleft}}{je dat moet worden gezegd}{grunnikte Roota we\^er}\\

\haiku{ik ben nog niet au,}{bout de mon latin dat moet}{u toch nog hooren}\\

\haiku{hij bemoeide zich;}{met het modelletje dat}{stijf naar hem opzag}\\

\haiku{je maakt dat ik me}{daar ineens zie zitten op}{het tabouretje}\\

\haiku{Het was nu lang niet;}{meer zoo vinnig en ze had}{al zeven stuivers}\\

\haiku{Het jongetje kwam {\textquoteleft}!}{met zijn vader de winkel}{uit enadie heeren}\\

\haiku{Dan keek zij uit haar.}{krullen op in eens of wou}{ze wat van haar}\\

\haiku{Soms kon ze door het.}{raampje een heer met glimhoed}{rechtop zien zitten}\\

\haiku{de diender klom er,.}{op den drempel en had zijn}{touwen op zijn rug}\\

\haiku{Moeder bleef lang uit;}{en zij had meer verdiend dan}{moeder had verdiend}\\

\haiku{Ze knoopte nu de,;}{doek los van haar lenden ruig}{en wit bespikkeld}\\

\haiku{Ze klopte met de,,.}{doek verscheien malen deed}{hem om toen anders}\\

\haiku{De smeltende hars;}{der vuurmakers smookte uit}{de kachel kieren}\\

\haiku{{\textquoteleft}Wat,{\textquoteright} grauwde we\^er de, {\textquoteleft},.}{vrouw in eensheb jij op je}{kop te krauwen zeg}\\

\haiku{Hij bromde iets van {\textquoteleft}{\textquoteright};}{te laat en werd al drukker}{van bewegingen}\\

\haiku{{\textquoteleft}Je kunt toch nooit eens,{\textquoteright},, {\textquoteleft}'}{praten vond ze terwijl ze}{haar hoed recht zette}\\

\haiku{Dat was je eerst een,.}{toer geweest mannen maken}{zulke groote stappen}\\

\haiku{Onder de rouw van;}{de binnenboog zagen zij}{de opene poort in}\\

\haiku{Had zij wel ooit zoo'n.}{zuivere boog gezien en}{in zoo een hemel}\\

\subsection{Uit: Jaapje}

\haiku{hij droeg een brief in;}{zijn hand en een driekanten}{steek had 			 hij op}\\

\haiku{Jaapje 			 hield zijn.}{hand in zijn zak en klemde}{het harde kluitje}\\

\haiku{als een vlieg was hij,.}{opgegaan koud en licht door}{de 			 warme lucht}\\

\haiku{Jaapje rolde zich.}{om als een hoopje en}{van het licht we\^er af}\\

\haiku{zijn hart trommelde.}{in 			 zijn lijf en toen werd}{het vreeselijk stil}\\

\haiku{Heer, waarom verstoot?}{Gij mijn ziel en verbergt Uw}{aanschijn voor 			 mij}\\

\haiku{{\textquoteright} zei Nico barsch.}{en greep hem vast bij de split}{van zijn borstrok}\\

\haiku{Maar nu dacht hij 		  ,;}{daaraan niet er leefde in}{Jaapje wat anders}\\

\haiku{Hij voelde dat zij,}{heel dicht bij de tobbe kwam}{staan onderwijl}\\

\haiku{{\textquoteright} Jaapje zag uit het;}{bladerige boek felle}{kleurtjes fladderen}\\

\haiku{Hij keek oplettend,.}{naar de mannekoppen}{we\^erszijds de kapel}\\

\haiku{{\textquoteleft}Zoo, kom je maar es,{\textquoteright},.}{kijken bromde Rijs of kwam}{het uit de 		  grond}\\

\haiku{Jaapje had armen;}{zien liggen van buizen en}{pijpen van broeken}\\

\haiku{die niet door 'n goeie,.}{bril kijkt kijkt licht door de bril}{van een 		  ander}\\

\haiku{De meisjes zaten,.}{op de voorste 		  bank de}{jongens achteraan}\\

\haiku{in 't midden was,.}{een stralende zon daar}{glom een nummer in}\\

\haiku{{\textquoteleft}Wanneer ik groot ben,{\textquoteright}, {\textquoteleft}.}{troostte hij zich zelfword ik}{ook 		  generaal}\\

\haiku{{\textquoteleft}Net zoo blauw als de,{\textquoteright}, {\textquoteleft}.}{rest zei Doorik geloof het}{ten minste 		  wel}\\

\haiku{{\textquoteleft}Groomoe,{\textquoteright} zei Jaapje, {\textquoteleft}?}{danmag ik Door tot voor de}{poort 		  wegbrengen}\\

\haiku{wij moeten altijd,'.}{wederstaan den 		  Booze t}{aller uur en tijd}\\

\haiku{de ziel heeft het 		  ,.}{altijd warm ze is van de}{duvel bezeten}\\

\haiku{Laat hij mij maar voor,;}{de heeren roepen ik}{zal mijn woord wel doen}\\

\haiku{Dan zette groomoe.}{haar bril op en dan zat ie}{er altijd 		  in}\\

\haiku{stak zij haar hand door.}{de leuning van haar stoel en}{hield de zijne vast}\\

\haiku{Jaapje hoorde hem.}{schuifelen over de 		  sneeuw}{en eventjes hoesten}\\

\haiku{{\textquoteright} {\textquoteleft}Da\`arvoor heb ik het,{\textquoteright}.}{niet gekregen van Rudolf}{antwoordde Jaapje}\\

\haiku{hij groeide veel te,,;}{hard 		  zei moeder er was}{geen bijhouen aan}\\

\haiku{hij werd uit de bank.}{geroepen en moest in}{de eetzaal komen}\\

\haiku{Ze zaten altijd.}{aan je lijf te wriemelen}{als het niet noodig was}\\

\haiku{Jaapje snufte en.}{deed zijn boezelaar van}{achteren zelf vast}\\

\haiku{{\textquoteleft}vraagt u maar gerust,{\textquoteright}.}{of ik wat krijg maar 		  bleef}{heel stijf staan kijken}\\

\haiku{Ze gingen er bij.}{zitten 		  en spalkten hun}{oogen vastberaden}\\

\haiku{ze was we\^er op het, {\textquoteleft},}{hoopje zand gaan kloppen}{de rijstebrijberg}\\

\haiku{Onmiddellijk was;}{Jaapje opgekomen en}{liep hard naar de pomp}\\

\haiku{Nico was de poort, {\textquoteleft}{\textquoteright}.}{gauw ingegaan omdat hij}{koetsier wou worden}\\

\haiku{uit 		  de hand te,.}{eten een sneedje grof en een}{sneedje roggebrood}\\

\haiku{{\textquoteleft}Zal je eens goed je,{\textquoteright}, {\textquoteleft}.}{oogen uitwasschen zei Doorze}{zijn we\^er erg 		  zwart}\\

\haiku{je 		  wordt al te,.}{groot voor een talhout maar je}{verdiende het wel}\\

\haiku{Jaapje pinkte en.}{lachte vergevingsgezind}{naar de kant van Koos}\\

\haiku{{\textquoteleft}Ziet onze lieve?}{Heer dan door de jassen van}{de soldaten heen}\\

\haiku{Je grootvader en {\textquotedblleft}{\textquotedblright}.}{ik waren getrouwd onder}{delamme koning}\\

\haiku{toen klikte we\^er de,.}{deurtjes buiten 		  dicht de}{eene na den ander}\\

\haiku{er lag een knoop in,}{met een wilde zwijnekop}{er op zijn eenig}\\

\haiku{Deijlius was,.}{zelf   gekomen had een}{rond 		  hoedje op}\\

\haiku{en ten leste had.}{Kareltje ook een nieuwe}{pet 		  gekregen}\\

\haiku{{\textquoteright} Jaapje kreeg een 		  ,.}{bangig wezen daar had hij}{nog nooit aan gedacht}\\

\haiku{Je hoorde almaar:}{loopen boven je hoofd en}{toen Koos had gezegd}\\

\haiku{de hoek van inval;}{is gelijk aan den hoek}{van terugkaatsing}\\

\haiku{Ze hadden w\^eer,:}{een poosje zitten visschen}{toen de meester zei}\\

\haiku{Wat kiest gij schuw uw,:}{pad  of van de boer die}{tot zijn paard zei}\\

\haiku{Doch nu het kermis,.}{was geworden las hij niet}{meer in zijn prijsje}\\

\haiku{Er waren altijd {\textquoteleft}{\textquoteright}.}{krieken en 		  peertjes ook}{met rooie wangetjes}\\

\haiku{Er stond niet erg veel,;}{in omdat het zulke groote}{drukletters had}\\

\haiku{Hij hield zijn adem in,,}{en fronsde er schuin tegen}{aan hij zag het goed}\\

\haiku{Telkens was het of,.}{hij op wou zitten 		  dan}{viel hij slap we\^er ne\^er}\\

\haiku{Er liep een vrouw met;}{wafels op een 		  blaadje}{tusschen de banken}\\

\haiku{hij nam het van zijn.}{pruikebol en zwaaide}{naar al de menschen}\\

\haiku{{\textquoteright} vroeg de loodgieter,.}{kijkend 		  naar twee schrammen}{op Jaapjes gezicht}\\

\haiku{Wanneer wij hem straks,}{zullen toevertrouwen aan}{den schoot der aarde}\\

\haiku{{\textquoteright} had tante gevraagd, {\textquoteleft},}{hoe heb je het gewaagd we}{houen het nooit}\\

\haiku{voor de ruitjes 		  ,.}{waren hekjes soms dat ze}{niet zouen breken}\\

\haiku{Toen zag hij in een}{groote glazen kast een heer die}{zat te schrijven}\\

\haiku{{\textquoteright} zei de jongen {\textquoteleft}as,;}{je dat niet heb gezien heb}{je ook niks gezien}\\

\subsection{Uit: Jaap}

\haiku{man die de {\textquoteleft}platen{\textquoteright}.}{had in zijn handen als de}{moeder een schoon hemd}\\

\haiku{{\textquoteright} {\textquoteleft}Het nieuwe is hem,{\textquoteright}}{te machtig zei groomoe en}{le{\^\i} haar hand op}\\

\haiku{En zoo ging het om,.}{en om van de eene dag in}{de 			 andere}\\

\haiku{{\textquoteleft}Nou, domkop dan,{\textquoteright} riep, {\textquoteleft},{\textquoteright} {\textquoteleft}}{Dolfhou maar je groote snavel}{we kommen 			 al.}\\

\haiku{Door gaf een zoen aan.}{groomoe en aan Koos en toen}{er een aan 			 hem}\\

\haiku{{\textquoteright} {\textquoteleft}Met alle soorten,{\textquoteright}.}{van genoegen zei Koos en}{hupte op de been}\\

\haiku{Hendrik zat 			 Jaap.}{recht aan te kijken met zijn}{bruin-blauwe oogen}\\

\haiku{{\textquoteright} {\textquoteleft}Groomoe,{\textquoteright} liet Jaap we\^er, {\textquoteleft}?}{hoorenwanneer krijgt Door nou}{een 			 koppie thee}\\

\haiku{Groomoe le{\^\i} haar hand.}{naar Jaap's blauwe 			 arm en}{liet die daar rusten}\\

\haiku{{\textquoteright} antwoordde Jaap en.}{leunde zich 			 meteen los}{tegen zijn stoel aan}\\

\haiku{{\textquoteleft}Ze zijn op 't laatst,}{niet meer te tellen al die}{tikken op je}\\

\haiku{Piet Pollee raapte.}{een driekant brokje op}{en ging er van eten}\\

\haiku{{\textquoteright} vroeg hij, {\textquoteleft}dat is mijn,;}{bibliotheek dat bennen}{mijn klassieken}\\

\haiku{de 			 linkerhoek.}{van zijn mond gaan lachen en}{toen de rechterhoek}\\

\haiku{ik zou haar gauw haar,{\textquoteright}.}{luier geven raadde}{groomoe zacht naar Door}\\

\haiku{{\textquoteleft}Maak voort,{\textquoteright} bromde Jaap.}{naar het paard en loende schuin}{en naar 			 boven}\\

\haiku{{\textquoteleft}Kijk es aan,{\textquoteright} zei van, {\textquoteleft},;}{Essennou zie dan of ik}{geen gelijk heb}\\

\haiku{De tree\"en en de.}{slagijzers waren gedofzwart}{en de lantarens}\\

\haiku{{\textquoteleft}'t Is Woensdag, je!}{moet vanmiddag naar de}{catechesatie}\\

\haiku{die van de jonge.}{baas was wat dikker gedeukt}{van 			 onderen}\\

\haiku{{\textquoteleft}Ik zal het zeggen,{\textquoteright}, {\textquoteleft}.}{zei Koosmet een le\^ertje om}{aan te 			 dragen}\\

\haiku{stijf op 			 een leeuw.}{en met je voeten in de}{beugels als je kon}\\

\haiku{De palen staken,.}{uit de keien op er lei}{maar \'een 			 kei bij}\\

\haiku{Koenraad zal 			 ze;}{dan na schafttijd voor me in}{de stopverf zetten}\\

\haiku{De looper was een,;}{hompige kei aan een kant}{glad geslepen}\\

\haiku{rol dan eerst 			 de,.}{stopverf door de rauwe olie}{dan houdt ze beter}\\

\haiku{'t Was toch wel een,;}{mooi vak 			 schilderen en}{glazen-maken}\\

\haiku{{\textquoteleft}Gauw,{\textquoteright} zei de witte, {\textquoteleft}, '.}{ventbezorg hem heett is}{meer dan je 			 weet}\\

\haiku{Het was 			 Psalm 1, '.}{waarvan hijt eerste vers}{van 			 buiten kon}\\

\haiku{Jaap was soms benauwd,.}{door het 			 orgel om wat}{er uit kon komen}\\

\haiku{zijn kwasten spoelen.}{in terpentijn tot er niets}{geen olie meer in was}\\

\haiku{Toen bromden ze vlak,.}{naast elka\^ar alsof ze}{oneenigheid hadden}\\

\haiku{hij deed het nochtans,.}{naarstig mede tot 			 er}{de rook van af vloog}\\

\haiku{Ze hadden Jaap eens.}{opgetild en toen had hij}{het ook 			 gezien}\\

\haiku{Bertus was heel kort.}{geknipt en dikker door de}{lucht 			 geworden}\\

\haiku{zoo schoten zij het.}{jaar uit en alles wat er}{in was 			 gebeurd}\\

\haiku{maar Jaap was bij zijn.}{kopje blijven zitten en}{had geschud van neen}\\

\haiku{Alles wat hij meer:}{nog leerde liet Jaap als}{achter in de school}\\

\haiku{achter hem sprak een.}{schorre stem en siste wat}{over 			 zijn schouder}\\

\haiku{of allemaal in, {\textquoteleft}{\textquoteright}.}{sn\'el gevolg na een wat de}{kaskade heette}\\

\haiku{wanneer de lommerds.}{maar eens spreken konden en}{als het dan woei}\\

\haiku{Gezichten bleven,?}{goor en handen want waarin}{moest je je wasschen}\\

\haiku{Vrijdags hooren kwam.}{of er niet een paar voor}{Zondag waren noodig}\\

\haiku{al reed je op 			 {\textquoteleft}{\textquoteright}.}{kunstschaatsen allicht mooier}{op de korte baan}\\

\haiku{ze sloeg haar schaatsen.}{dadelijk uit 			 en reed}{terug naar Gerard}\\

\subsection{Uit: Jacob}

\haiku{Zijn vader werkte,;}{aan de spoor daar konden zij}{hem niet gebruiken}\\

\haiku{Hij was toen erg 			 , ....}{verschrokken maar je wou toch}{ook wel eens spelen}\\

\haiku{{\textquoteleft}hij h\'et hem 			 we\^er{\textquoteright}.}{en wist dan dat pikeur de}{Boer de winner was}\\

\haiku{Oom keek meestal;}{blijmoedig en soms met open}{lippen naar zijn zoon}\\

\haiku{Meinardus liep almaar,;}{vroolijk te praten wist veel}{te vertellen}\\

\haiku{tante sprak altijd.}{of ze de tijd 			 had en}{lachte onverwacht}\\

\haiku{Af en toe 			 drong;}{het laag geratel van een}{rijtuig naar boven}\\

\haiku{Dan hoorde je hun,.}{lachen door het huis soms}{tot in de keuken}\\

\haiku{Koos was er ook, had,.}{kringen onder haar oogen was}{erg nieuwsgierig}\\

\haiku{{\textquoteright} Jakob voelde dat,.}{er toch niets aan te doen was}{hij hield zijn mond}\\

\haiku{Nu 			 hij niet al, '}{de lessen aan de avondschool}{meer volgde zat hij}\\

\haiku{Jaantje was zoo kwiek,.}{als een jongen zou er om}{willen vechten}\\

\haiku{Dat weten we wel,{\textquoteright}, {\textquoteleft},.}{praatte Doorkom drink nou je}{kopje eerst 			 le\^eg}\\

\haiku{Koos droogde haar neus,.}{nam we\^er haar kopje 			 bij}{het schoteltje op}\\

\haiku{Ze waren voor het.}{hek en in de schijn 			 van}{de tuinlantaren}\\

\haiku{hij lichtte zijn pet,.}{krabde even zijn voorhoofd}{en spoelde nog eens}\\

\haiku{{\textquoteright} stemde Jakob grif.}{we\^er toe en met den platsten}{toon van de 			 straat}\\

\haiku{Het feest van meester;}{Boudewijnse was reeds een}{poosje geleden}\\

\haiku{Ik heb mijn bril niet,{\textquoteright}, {\textquoteleft};}{bij me zei de baaszing het}{me maar es 			 voor}\\

\haiku{{\textquoteleft}ik zelf speelde niet,.}{l\'ang met 			 poppen ging al}{gauw in betrekking}\\

\haiku{Zoo zijn we,{\textquoteright} had Door,, {\textquoteleft}}{toen geantwoord schenkend de}{kopjes we\^er vol}\\

\haiku{Geesteraag, hij en;}{ik zijn 			 ongeveer van}{dezelfde leeftijd}\\

\haiku{Het stelde een heer,,.}{en een dame voor 			 pas}{getrouwd dat zag je}\\

\haiku{{\textquoteright} De baas voerde elk,.}{we\^er mede nu zou het pas}{goed beginnen}\\

\haiku{Al liep iemand weg,.}{moest je nog niet denken dat}{hij je gelijk gaf}\\

\haiku{, zit goed vast, 			 maar.}{ik zou toch niet graag een paard}{van u willen zijn}\\

\haiku{Hij nam zijn zakdoek.}{uit zijn zak en wreef 			 zijn}{hand terdege schoon}\\

\haiku{Gras was er niet en,.}{geen mos maar wel sprieten scherp}{voor het gevoel}\\

\haiku{wolken maken 			 .}{het duin tot golvend zand of}{tot \'e\'ene groote kuil}\\

\haiku{Soms vloog er over je,.}{hoofd een vogel dat wel een}{zeearend zijn 			 kon}\\

\haiku{het genie is een, {\textquoteleft}{\textquoteright}.}{lang geduld stond inHelp u}{zelf te lezen}\\

\haiku{De wind en de zee;}{moest 			 je uit elkander}{weten te houden}\\

\haiku{het 			 zal zooveel.}{niet verschillen en kan toch}{evengoed zoo bestaan}\\

\haiku{{\textquoteright}  De komiek zong.}{het eene Parijsche liedje}{na het andere}\\

\haiku{Hij was toen door de.}{stilste straatjes gegaan naar}{het spel van 			 Basch}\\

\haiku{dikwijls als je keek,.}{bleek het alwe\^er anders dan}{de vorige maal}\\

\haiku{Jakob dacht aan het:}{Wilhelmus en aan de}{Geuzenliederen}\\

\haiku{Tinus kwam er nooit,.}{omdat hij te ver van de}{winkel woonde}\\

\haiku{K. Kommer eischt,.}{geduld en moed L. Luiheid}{leidt tot tegenspoed}\\

\haiku{Q. Quasi-deugd,.}{is slijm en slijk R. Rijkdom}{maakt niet altoos rijk}\\

\haiku{We zijn ten slotte,{\textquoteright}.}{allemaal weezen zei Baas}{met zijn oogen 			 ne\^er}\\

\haiku{Een jongen bracht 't.}{karbiesje we\^erom Dat zij}{verloren had}\\

\subsection{Uit: Nieuw proza}

\haiku{{\textquoteleft}Het wordt tijd dat ik,{\textquoteright}.}{mijn conterfeitsel maken}{laat dacht hij weder}\\

\haiku{{\textquoteright} {\textquoteleft}Niet denken doet men,{\textquoteright}.}{altijd genoeg meende we\^er}{zoetjes Ambroise}\\

\haiku{kom laten wij het,!}{nog eens samen hebben nog}{eens een duetje}\\

\haiku{het tintelde hem,.}{tegen het kleurde er als}{in geslepen glas}\\

\haiku{Het was maar goed dat.}{mijnheer van Oudentijd niets}{van dit alles zag}\\

\haiku{{\textquoteright} {\textquoteleft}Je oogen bennen goed,{\textquoteright}, {\textquoteleft}.}{oordeelde Marretjeen}{ik weet je drinkt niet}\\

\haiku{het stond er toch reeds,.}{in het rijtje huizen toen}{ik nog heel jong was}\\

\haiku{Het varen ging nooit,,.}{recht altijd in zwenkende}{baan het water trok}\\

\haiku{{\textquoteright} scherp keek ze naar over,;}{het bermpje naar het grotje}{waar ze had gestrooid}\\

\haiku{ze liet haar boekje.}{zinken en plaatste haar voeten}{ordelijk nevenseen}\\

\haiku{E. daarentegen.}{zie ik altijd dadelijk}{als in de vlakte}\\

\haiku{Hij had in Chigi {\textquoteleft}{\textquoteright}.}{het meeste te zeggen en}{stelde nogal eens}\\

\haiku{Ik weet niet juist meer,.}{wat ik neuriede het was}{iets van Beethoven}\\

\haiku{Mei-regen maakt,,}{dat ik grooter word Grooter}{word Ender dat wensch}\\

\haiku{achter hem stond de.}{geest van Zorolla in zijn}{geruite pakje}\\

\haiku{Niet alleen zijn vrouw,;}{aanbad hem ieder die hem}{kende hield van hem}\\

\haiku{Het was een gulle,,;}{ouwerwetsche zeeman een}{aangespoeld matroos}\\

\haiku{{\textquoteright} vroeg de kerel, {\textquoteleft}naar,?}{het wijfje kijken dat zoo}{aorig lullen kan}\\

\haiku{Hij was toen we\^er gaan,.}{dwalen hij had zijn messen}{in zijn pelerien}\\

\haiku{er schoven telkens;}{levende tronies achter}{elkander voorbij}\\

\haiku{De knaap rolt om op,....}{het voetpad de agent liet zijn}{blik op hem wegen}\\

\haiku{Wanneer zal eens een?....}{alliance hollandaise}{worden opgericht}\\

\haiku{{\textquoteright} hielp mede zijn vrouw, {\textquoteleft}.}{herinnerenlag alles}{wijd onder de sneeuw}\\

\haiku{hij zal wel we\^er zijn,,.}{beste beentje v\'o\'or zetten}{weet ik mijn broeder}\\

\haiku{{\textquoteright} {\textquoteleft}Jullie moesten hem over,{\textquoteright}.}{zien te houden vond tante}{Van Hoevelaken}\\

\haiku{ieder hield zich bij.}{zijn bordje of onderhield}{zich met zijn naaste}\\

\haiku{De doeken daar dit, '.}{kind in leit Ist purper}{van zijn majesteit}\\

\haiku{Van Hoevelaken;}{zat met een der hulstblaadjes}{voor hem te spelen}\\

\haiku{{\textquoteright} liet ingehouden.}{straf oom Vervieren over heel}{de tafel hooren}\\

\haiku{{\textquoteleft}En een enkel geel,{\textquoteright}.}{gordijntje vulde mevrouw}{Stavoren aan}\\

\haiku{het is mij niet recht.}{duidelijk waar Maria met}{het kindje op rust}\\

\haiku{{\textquoteright} Vervieren richtte.}{zich achterover en pufte}{zijn rook naar boven}\\

\haiku{Mevrouw Vervieren.}{lachte het heele klavier}{van haar tanden bloot}\\

\haiku{wees er van zeker,....}{Rembrandt zou er ook niet om}{gelachen hebben}\\

\haiku{hij leek we\^er in zijn,;}{zwak getast de voorliefde}{voor de novelle}\\

\haiku{Tante Agnes kneep,.}{haar eene oog deed alsof ze}{zoog op iets heel fijns}\\

\haiku{{\textquoteleft}En u krijgt een warm,{\textquoteright}.}{kruikje vleide Dora aan}{haar andere zij}\\

\haiku{{\textquoteleft}Wat een gezellig,{\textquoteright},.}{voorwerp zei die kijkend scherp}{naar den kandelaar}\\

\haiku{Zijn uitzicht was dan,.}{streng en week hij leefde er}{geheel in mede}\\

\haiku{hebt Gij U eere,....}{ontzegd Werdt Gij in stroo en}{in doeken gelegd}\\

\haiku{Landoue zat stil te.}{glunderen en grunnikte}{naar het schermutsel}\\

\subsection{Uit: Proza}

\haiku{Maar op het plein San.}{Marco vierde de sneeuw feest}{in de groote stilte}\\

\haiku{Van toen aan was het.}{plein niet langer verlaten}{en ongeschonden}\\

\haiku{Hij sloeg zich met de.}{handen tegen de schouders}{om warm te worden}\\

\haiku{- {\textquoteleft}Zit stil, ni\~na,{\textquoteright} zei,.}{de schilder zenuwachtig}{haastig arbeidend}\\

\haiku{- {\textquoteleft}Dood, heertje,{\textquoteright} zei ze, {\textquoteleft}.}{verleden jaar gestorven}{aan de cholera}\\

\haiku{- {\textquoteleft}Es tonto,{\textquoteright} zei de, {\textquoteleft},,.}{deernehet doet hem geen pijn}{heertje es tonto}\\

\haiku{{\textquoteright} zei de jonge man.}{tot zijn onbewegelijk}{rookenden buurman}\\

\haiku{Beelden gehaald van,.}{ver uit verre jaren en}{uit verre streken}\\

\haiku{{\textquotedblright} en staande, met \'een,,.}{teug dronk hij het glas le\^eg dat}{de vrouw hem inschonk}\\

\haiku{Na een klein poosje.}{kwam de stem van baas van D.}{achter zijn rug om}\\

\haiku{het silhouetje.}{van het buitenhek stond er}{donker tegen uit}\\

\haiku{Dat waren altijd.}{oogenblikken geweest van}{bizonder genot}\\

\haiku{want laag achter hem,.}{rees de klare maan zuiver}{in den nacht stijgend}\\

\haiku{In het smettelooze;}{maanlicht was alles z\'o\'o ijl}{en onstoffelijk}\\

\haiku{{\textquoteright} - {\textquoteleft}En de begeerte,{\textquoteright}.}{is eeuwig prevelde de}{afgekeerde bloem}\\

\haiku{en daar flikkerde;}{het lichtende karkas van}{een gebouw omhoog}\\

\haiku{Als een straatvlam een,}{rij in het gezicht sloeg kon}{hij zien hoe verlept}\\

\haiku{En nu vol van drank,.}{en pret klotsten en dansten}{ze hier den nacht uit}\\

\haiku{Ja, dat was groot, dat,.}{was groot dat stond vast in den}{dag en in den nacht}\\

\haiku{van dien dwaas die meer.}{wijsheid spreekt dan tien wijzen}{van den kouwen grond}\\

\haiku{De dorre adem die,.}{uit de vlakte zwoegde hing}{in de houten kast}\\

\haiku{De felle zon sloeg.}{van het land op en brandde}{ne\^er op den wagen}\\

\haiku{'t was als de rest, '.}{t zocht zijn pleizier en haar}{eigen goed leven}\\

\haiku{Dood, dood, gevoelloos,....}{voor koud en naar weder en}{voor mijn roepen doof}\\

\haiku{{\textquoteright} Zijn hals en krop was ' ';}{ondert vertellen aan}{t zwellen gegaan}\\

\haiku{en toen ben ik naar........}{zijn bed gegaan bij ons in}{de wagen ziet u}\\

\haiku{met schrik om het hart,.}{met kloppende keel stond ik}{het aan te staren}\\

\haiku{Had ik niet een klein?}{donker geslinger gezien}{naar het open buiten}\\

\haiku{En even daarna kwam:}{de stem der padrona hun}{weggaan nagalmen}\\

\haiku{Hij vond dezen met,.}{zijn rug tegen de tafel}{zeker in zijn hoek}\\

\haiku{En de comedor;}{was geheel stil geworden}{van menschengepraat}\\

\haiku{Hij schuurde met zijn.}{hoofd langs den muur om een goed}{plaatsje te vinden}\\

\haiku{laat, zich mooi maakt voor,.}{haar liefste voor hem haar schoon}{vermenigvuldigt}\\

\haiku{De snoer was maar half,}{afgewonden de dobber}{schommelde dichter}\\

\haiku{Onder geruisch.}{zit zij van serafijnen}{die haar overkroonen}\\

\haiku{vlak, vermindert in.}{het minst niet de kloekheid van}{het volgroeide blad}\\

\haiku{Die zon en zomer, ':}{te beminnen leeren ent}{treuren gehad}\\

\haiku{Er bewoog zich in.}{alle opzichten veel jongs}{om dat jonge boek}\\

\haiku{Is dit zich als even '?}{verraden een schade voor}{t boek-geheel}\\

\subsection{Uit: Reizen}

\haiku{Kom, heer gemaal, al.}{deze dingen moeten niet}{dus worden overpeinsd}\\

\haiku{De zaak is, waarde,{\textquoteright}, {\textquoteleft}}{compeer meesmuilde hijdat}{ge er zelf razend}\\

\haiku{als neef-lief laat.}{voor de Bar blijft plakken met}{Tanger's fine fleur}\\

\haiku{Chopin: speelde....}{hij en fantaseerde dat}{het liep als water}\\

\haiku{honderdmaal liep ik.}{dit blanke straatje en dat}{ik dit pas oplet}\\

\haiku{{\textquoteright} Toen voor een groote plaat,:}{bleef hij aandachtig staan het}{onderschrift beturend}\\

\haiku{we gaan hetzelfde,, {\textquoteleft}{\textquoteright}.}{weggetje waarin de stoet}{verdwijnt inGekken}\\

\haiku{Hij is toch mooi wit,.}{ze heeft zoo'n aardig kuifje}{en zulke goeie oogen}\\

\haiku{Naar Tetuaan werd '...}{het wel twaalf uur enk was}{nog heelemaal frisch}\\

\haiku{ik geloof dat ik...}{nog liever op een pak zit}{dan in een boxsaddle}\\

\haiku{Rijden was heerlijk,....}{je werdt een sterker man daar}{bovenop zoo'n beest}\\

\haiku{{\textquoteleft}Hoe was ook we\^er de,{\textquoteright}...}{naam van die rivier zocht hij}{in zijn gedachten}\\

\haiku{die intonatie,{\textquoteright},.}{sprak Emilia beduusd toen zij}{waren in hun tent}\\

\haiku{Hasj stil en waardig,.}{reikte nog wat eieren}{warm uit den ketel}\\

\haiku{het is hetzelfde.}{gevoel dat ik als jongen}{op een schommel kreeg}\\

\haiku{plotseling brandde,....}{het zonvuur raasde over den}{bodem en verzwond}\\

\haiku{zij trokken onder....}{de groote triomfpoort door en}{waren toen thuis}\\

\haiku{{\textquoteright} {\textquoteleft}Dat is dus het eind,.}{van het complot als onze}{vriendin het noemde}\\

\haiku{zoo was het zeker.}{geworden dat zij Laraisj}{zouden bezoeken}\\

\haiku{op elke wang en.}{op het voorhoofd iets als een}{sterretje droegen}\\

\haiku{{\textquoteleft}Lekoes, Lekoes,{\textquoteright} klapte er,.}{de naam van uit het af en}{aan loopende volk}\\

\haiku{De Lekoes rolde zijn.}{gele golven te midden}{van zanderijen}\\

\haiku{ze we\^er plannen en,}{keek ze maar gewoon naar haar}{minnaar misschien ook}\\

\haiku{{\textquoteright} Om den heuvelzoom.}{fladderden vale vlinders}{en zetten zich ne\^er}\\

\haiku{je wordt er dronken,{\textquoteright};}{van riep Emilia uit en ze}{bukte en bukte}\\

\haiku{{\textquoteright} Koud tinkelden reeds,:}{sterren hij hoorde verdwaasd}{het schreeuwen van Hasj}\\

\haiku{koppen van kerels,,.}{gehurkt op den vloer gloeiden}{boven den drempel}\\

\haiku{{\textquoteright} Emilia hield een nieuw.}{schrijfboek op haar schoot en was}{druk met Mohammed}\\

\haiku{Maar door den lagen.}{stand der zon gelukte het}{nu niet zoo spoedig}\\

\haiku{{\textquoteright} lachte gelukkig,.}{zijn ellendig hoofd toen hij}{den zonprik voelde}\\

\haiku{Hij is zeker eens,{\textquoteright}.}{gebeten op het Zocco}{meende Theobald}\\

\haiku{Voor de drassige;}{sleuring daar hield nu ook de}{andere stoet halt}\\

\haiku{Vlakbij hoorden zij;}{de bange stemmetjes van}{de moorsche vrouwen}\\

\haiku{schrik maakt blind als toorn,,...}{de oogen gaan inwaarts zien iets}{anders dan er is}\\

\haiku{Evangeline wou}{rusten en in eens gaf toen}{Roosevelt bevel}\\

\haiku{De drijver stampte.}{zijn voet tegen den grond om}{te zien of het hield}\\

\haiku{een  reikte een '.}{sachet tot aan zijn neus en}{snooft parfum}\\

\haiku{zijn voorhoofd echter;}{was veel nietiger en zijn}{uitkijk blauw en struw}\\

\haiku{af en toe klaagt de.}{roep van een blindeman door}{de rulle ruimte}\\

\haiku{Een van de jonge,{\textquoteright}.}{generatie lachte Miel}{toen hij afscheid nam}\\

\haiku{Zeg aan uw meester,{\textquoteright}:}{antwoordde mijn gezel op}{we\^er zulk een boodschap}\\

\haiku{Er is van Fez nog,, {\textquoteleft}{\textquoteright}.}{niets gemaakt zei de gezant}{mij hoorend dat ikschreef}\\

\haiku{bang en siddrend beestje, ', ..............}{O watn verbijstering}{in je borstje}\\

\subsection{Uit: De wonderlijke avonturen van Zebedeus}

\haiku{De reisverhalen,;}{die wij u hebben te doen}{zijn de peine waard}\\

\haiku{de zwaarste spijzen,,.}{de vruchten onzer wijsheid}{ze verteren licht}\\

\haiku{de vogel zal zich,,}{nimmermeer verblijen Mijn}{kooi wordt le\^eg van zang}\\

\haiku{(betracht zich aldus)}{andermaal maar nu naar de}{andere zijde}\\

\haiku{voor je hoofd de proef.}{op de som en een gloedje}{onder de borstkuil}\\

\haiku{Wat moet ik dan doen?}{om je we\^er te brengen in}{je knolletuin}\\

\haiku{(houdt de handen nog.)}{wat vooruit gestoken en}{haalt ze dan snel in}\\

\haiku{geen eerlijk middel.}{te klein  om te komen}{tot een groot einde}\\

\haiku{jammervollen  ,;}{zij wier lippen pas lachen}{in den vasten dood}\\

\haiku{Vier, vijf, achter de....}{zevende daar groeien de}{dikste bramen}\\

\haiku{dit is het ware,}{reuzen-voedsel wat ik}{zeer wel ken laat ik}\\

\haiku{droomen, nat, voedsel,.}{en zonnelicht de aarde}{heeft het overvloedig}\\

\haiku{- Het is de rechte,,,,?}{tijd zei daarop kortaf de}{eerste dichter w\`aar}\\

\haiku{Koekoek, koekoeke,...}{Liefde-gelok maar geen}{pake en moeke}\\

\haiku{sommigen noemen...}{het regen en anderen}{noemen het tranen}\\

\haiku{ik bewonder, ik,,.}{bewonder vergeef wend uw}{aangezicht niet af}\\

\haiku{le kerk Vindt je hier...,......}{werk Hemeltjes-blauw}{in het stroo O O}\\

\haiku{Veel vragen zijn even}{jong als vele vraagsters en}{vele antwoorden}\\

\haiku{ben ik toch niet, dat,...}{er de grond van wankt dat er}{de grond van davert}\\

\haiku{Hoor, hoor, fluisterde,.}{Zebedeus ze hebben}{het over de grasjes}\\

\haiku{hij keek, of keek hij.}{naar de achterdeelen van}{een denkbeeldig span}\\

\haiku{En zijn ze, na een,, '?}{tijdje zoo gedroogd dan gaat}{ge ist niet zoo}\\

\haiku{dat is omdat het,,...}{kil is omdat het kil is}{huiverde de reus}\\

\haiku{Velen uwer was het,,}{toen niet duidelijk wat dit}{zeggen wou en wij}\\

\haiku{Wat is uw meening?}{daaromtrent en welke houdt}{gij voor de beste}\\

\haiku{Zoek maar niet langer,,.}{viel Ruigrok in de rede}{zij leit aan den weg}\\

\haiku{Hij tuurde naar het;}{slijpende geglibber en}{wist niet waar te zien}\\

\haiku{wat jong is, verheft,,;}{zich gereede wat oud wordt}{allicht zich verdiept}\\

\haiku{De grootste schrijder}{onder ons doet ten leste}{uit geen anderen}\\

\haiku{En rond den pegel.}{Onder den regel Schimmert}{het rood en blauw}\\

\haiku{Eensklaps de pegel.}{Onder den regel Stort naar}{bene\^e met een krak}\\

\haiku{{\textquoteright} - {\textquoteleft}Uit ouwe kennis,{\textquoteright},.}{zei ze naar Zebedeus}{toen de dienaar ging}\\

\haiku{{\textquoteright} had zich de stem van ', {\textquoteleft}}{den gastheer gemengd int}{gesprekhoe hooger}\\

\haiku{{\textquoteright} niesde plotseling, {\textquoteleft}.}{Zebedeusik ben het}{geheel met u eens}\\

\haiku{Maar deze w\`as van,.}{de zee de vertrouweling}{der groote elementen}\\

\haiku{Door het raam verscheen,.}{nog het park in den eindloozen}{droevigen schemer}\\

\haiku{{\textquoteright} Tourniput tilde.}{zijn knie\"en tot zijn kin en}{schaterde het uit}\\

\haiku{Ik geloof dat 'k,,...?}{even maar Zou kunnen slapen}{nu Vindt je het naar}\\

\haiku{Er wies geen bloem, er;}{groeiden stakkelstruiken Ter}{we\^erszij van den weg}\\

\haiku{zoo vertrouwelijk,}{meestal onderhield hij}{ons dan uit den schat}\\

\subsection{Uit: De wonderlijke avonturen van Zebedeus}

\haiku{Wanneer hij spot als,}{nu en ongevoelig zich}{toont verbeurt hij juist}\\

\haiku{Wanneer Narango's,?}{gordel is geslonken naar}{wien verlangt hij dan}\\

\haiku{hij was, dat abele;}{lieden Afstandelijk hem}{groeten met ontzag}\\

\haiku{Toen werd zijn gansche,;}{leven De gansche wereld}{werd die maagd voor hem}\\

\haiku{) Hoor, ze dwarrelen.}{als kapelletjes tusschen}{de stammen telkens}\\

\haiku{Aardige onzin.}{verbergt vaak meer klaarte dan}{onaardige zin}\\

\haiku{Als een plaatse der.}{zoete bijeenkomst staat mijn}{hart altijd open}\\

\haiku{een hinde is toch,.}{geen wolf geen wezen dat haat}{inboezemt of angst}\\

\haiku{Die zijn geweer soms {\textquoteleft}{\textquoteright},.}{hem een zorg Bij moeder de}{vrouw in bed verborg}\\

\haiku{{\textquoteleft}Maak me zoo'n ding,{\textquoteright} zei,, ';}{Jan heel straf Wanneer het deugt}{koop ik jet af}\\

\haiku{Konden ter kerk zij,...}{Zondag rijden Net zoo goed}{als de grootste hans}\\

\haiku{Hij trok zijn geldbeurs, '.}{naar het licht Een mollevel}{metn veter dicht}\\

\haiku{Zij had haar jak nog ',.}{onderr rok En trijpten}{slof om wollen sok}\\

\haiku{{\textquoteleft}Als of 't gesmeerd,{\textquoteright},.}{was meende Bram En daarop}{staan moest een oorlam}\\

\haiku{'t Lemoen lag met;}{zijn spoorstok vast En aan de}{haken wel gepast}\\

\haiku{Af van de deur en '.}{kwam al gauw Schuddend de bel}{aann eindje touw}\\

\haiku{{\textquoteleft}Nee, die is goed,{\textquoteright} riep, {\textquoteleft},!}{moeke LijsHeere me God}{de knul is niet wijs}\\

\haiku{Een meisje achter, {\textquoteleft}{\textquoteright}.}{een spiegelglas Zei dat het}{de arke Noach's was}\\

\haiku{En toen de zon naar ',:}{t Westen zonk In roode}{schijnsels alles blonk}\\

\haiku{'t Kwam wel terecht,,.}{Floor schonk op pof En morgen}{rekenden zij of}\\

\haiku{Geen kunst heeft het nog;}{kunnen stellen buiten wat}{men noemt de natuur}\\

\haiku{Men heette hem een,:}{aristocraat dan lachte hij}{weemoedig en zei}\\

\haiku{Daar wordt gescheld en.}{een jonkman komt op om zijn}{dienst aan te bieden}\\

\haiku{Ik geef het U, zooals:}{ik het heb gevonden in}{mijn ooms papieren}\\

\haiku{Vertel me wat je,,;}{wilt praat mij over je meester}{praat mij over je zelf}\\

\haiku{Wat ik wenschte,,.}{te weten R\^evard is de}{waarde der wrake}\\

\haiku{{\textquoteright} {\textquoteleft}Maar, R\^evard,{\textquoteright} uitte, {\textquoteleft}?}{Zebedeushoe komt gij}{aan die gedachte}\\

\haiku{Gij hebt haar met uw,}{Lorrijnsche redenen zelf}{den weg gewezen}\\

\haiku{{\textquoteleft}Wilt u mij langzaam?}{en duidelijk de woorden}{nog eens herhalen}\\

\haiku{hij haalde zijn adem.}{fluitend in en a\^emde dien}{fluitend w\^eer uit}\\

\haiku{m\'a-ar... toen de baard......}{er eens was fluit-te ik}{van vo-ren-af-aan}\\

\haiku{Geen mensch weet dat, ik...,...}{fl\'uit geen mensch komt meer naar mij}{luisteren ik fluit}\\

\haiku{om-dat u......}{mijn neus niet hebt u d\`enkt het}{zal sn\`orken worden}\\

\haiku{{\textquoteright} {\textquoteleft}Doet het mij maar eens,{\textquoteright},.}{na gromde de man gebelgd}{of hij beleedigd was}\\

\haiku{Op vele vragen.}{geeft mijn oom mij nog altijd}{het beste antwoord}\\

\haiku{Er kwamen er nog,...}{meer Naast mij zeeg een juffrouw}{op het stoeltje ne\^er}\\

\haiku{Laten wij eens in,.}{de groene kamer kijken}{als het u belieft}\\

\haiku{{\textquoteright} Dit zeggende had.}{Philippus weder naar zijn}{meester omgezien}\\

\haiku{Voldaan, wanneer, al,...}{wist ik niet om wat Er iets}{in mij ontbloeide}\\

\haiku{Het kwam zoo voor een,.}{glimploos vlak te staan Als oud}{ijs in de maan}\\

\haiku{Wij hebben mannen,,;}{noodig zegt doctor Swellius}{mannen van de daad}\\

\subsection{Uit: De wonderlijke avonturen van Zebedeus}

\haiku{Wij zaten er als.}{kleinen aan wie een vol bord}{kersen werd beloofd}\\

\haiku{Ik zag hem, bromde,.}{R\^evard laatstelijk met een}{bloem in het knoopsgat}\\

\haiku{Zebedeus, de,.}{blaadjes ter hand zit weer in}{vragende houding}\\

\haiku{op 't oud Atheensche,:}{voorrecht Wijl ze is van mij}{beschik ik over haar}\\

\haiku{Demetrius vindt,;}{het niet Die wil niet zien wat}{ieder ander ziet}\\

\haiku{Een stevig stuk werk,,.}{dat verzeker ik je en}{een verrukkelijk}\\

\haiku{De duif vervolgt den,.}{griffioen de hinde Spoedt}{zich ter tijgervangst}\\

\haiku{Al heeft hij Hermia,,,,;}{lief Heer liefheeft o Toch heeft}{ze alleen u lief}\\

\haiku{{\textquoteright} {\textquoteleft}Gij ziet met deze ',{\textquoteright}.}{drank int lijf bromt R\^evard}{met zijn basstem we\^er}\\

\haiku{Zij keert zich op de,:}{rustbank maakt zich klein als een}{diertje en fluistert}\\

\haiku{Nee, doe er twee bij,.}{laat het geschreven worden}{in acht en acht}\\

\haiku{{\textquoteleft}Antropologen,,{\textquoteright}, {\textquoteleft}.}{mevrouw zegt R\^evardzijn geen}{antropophagen}\\

\haiku{Dan gaan twee er plots,;}{\'een vrijen Dat moet tot een}{spelletje leien}\\

\haiku{hang me niet aan, Of.}{ik zal je als een slang mijn}{lijf afschudden}\\

\haiku{Maar hij verjoeg mij,}{schimpend en hij dreigde Te}{slaan te schoppen mij}\\

\haiku{Hij kwam er recht mee:}{aangeloopen achter uit}{de sneeuwjacht en zei}\\

\haiku{mijn koningin, uw,.}{hand Betreed de slaapvloer hier}{naar wiegetrant}\\

\haiku{Mij dunkt, ik zie dit,.}{al met loenschen blik Als elk}{ding dubbel schijnt}\\

\haiku{Ik kon die oude,.}{fabels Die kinderlijke}{sprookjes nooit gelooven}\\

\haiku{Hoe te verschalken,?}{Den lakschen tijd anders als}{door wat vreugde}\\

\haiku{'k vertelde reeds,.}{mijn lief Van Herkules mijn}{glorieus verwant}\\

\haiku{Ik heb order hier.}{gekregen Achter de deur}{het stof te vegen}\\

\haiku{{\textquoteleft}Ja,{\textquoteright} zuchtte ze uit, {\textquoteleft};}{ze dansten bij de kom waar}{de koe komt drinken}\\

\haiku{{\textquoteleft}O,{\textquoteright} kwam de stem van, {\textquoteleft}.}{Dorinde binnen sprekener}{was geen sleutelgat}\\

\haiku{{\textquoteleft}Vraag mij dat nu niet,{\textquoteright}:}{bromde Zebedeus en}{vervolgde luider}\\

\haiku{{\textquoteright} {\textquoteleft}Denk er eens over na,{\textquoteright}.}{antwoordde Zebedeus}{op de oude toon}\\

\haiku{De gastvrouw had hun.}{nadering bespeurd en trad}{rustig naar voren}\\

\haiku{hij zag de gastvrouw.}{afkijken ook en ging met}{Dorinde terzijde}\\

\haiku{{\textquoteright} antwoordde de heer;}{met een open afgrijzen naar}{zijn zegsman starend}\\

\haiku{ze heeft haar buis aan'.}{en haar hoedje op en rookt}{uit Manus pijpje}\\

\haiku{tuinders, R\^evard, zijn.}{soms de grootste stoorders van}{het werk der bijen}\\

\haiku{{\textquoteright} meende moeder Mie,.}{aanloopend omdat zij zich}{even verwijderd had}\\

\haiku{het geloofde aan.}{den boom en dat was wat de}{kinderen boeide}\\

\haiku{{\textquoteleft}Ik heb het hem niet,{\textquoteright}.}{gevraagd antwoordde verstrooid}{Zebedeus}\\

\haiku{{\textquoteleft}Het ging niet langer,,{\textquoteright}.}{niet l\'anger herhaalde het}{beeld in de spiegel}\\

\haiku{{\textquoteright} Mevrouw Popotte.}{leek opstond de reiniging}{der kooi vergeten}\\

\haiku{Ik verheelde u.}{niet in welk een toestand zich}{de boedel bevond}\\

\haiku{Hij is in de lucht,,;}{van ons in onze wolken}{verduisteringen}\\

\section{Louise B.B.}

\subsection{Uit: Janneke de pionierster}

\haiku{Nog dien ochtend had,:}{vader vol trots me in de}{wangen geknepen}\\

\haiku{Met gebogen hoofd,,;}{fronsende wenkbrauwen stond}{ik te luisteren}\\

\haiku{En nu klopte mijn,:}{hart zoo bevangen toen ik}{zachtjes mompelde}\\

\haiku{de gebeden der,....}{stervenden opdreunend zooals}{ik later vernam}\\

\haiku{Sidin dengar, hoor, {\textquoteleft},!}{apa-apa!{\textquoteright}24Wat heb-je}{gehoord spreek vrij uit}\\

\haiku{Als je in Holland,}{een dinertafel ziet dan}{vallen de bloemen}\\

\haiku{Ik zag ons zitten,,;}{moeder en ik over elkaar}{Henk tusschen ons in}\\

\haiku{dat ik die droge!}{harde rijst aanzie voor een}{smedigen pudding}\\

\haiku{En weet je wat me?}{nu op het oogenblik het}{meest interesseert}\\

\haiku{{\textquoteleft}Kom, Henk, nu, voor wij {\textquotedblleft}{\textquotedblright},,?}{naarkooi gaan ons gewone}{halfdekje slaan h\`e}\\

\haiku{ik ben z\'o\'o blij, z\'o\'o,!}{innig dankbaar dat ik je}{gevolgd ben hierheen}\\

\haiku{Vrouwen gevoelen!}{en  denken nu eenmaal}{anders dan mannen}\\

\haiku{{\textquoteleft}Ik dank u voor uw,,;}{bitter maar stellig gezond}{drankje dokter Spaan}\\

\haiku{Als u soms eenige....?}{kisten blikjes en wijn van}{mij wilt overnemen}\\

\haiku{{\textquoteright} Onwillekeurig,.}{in mijn groote vreugde liet ik}{mijn handen rusten}\\

\haiku{Ik riep hem luide,.}{toe en na wat geschuifel}{ontsloot hij de deur}\\

\haiku{{\textquoteleft}Wat zijn jullie druk,,!}{ik ben niets nieuwsgierig laat}{mij nu maar met rust}\\

\haiku{Ik knikte, het was,.}{waar Johnstone alleen hield}{kippen op zijn erf}\\

\haiku{Men wil niet weten,!}{voor de rechter- wat de}{linkerhand wegschenkt}\\

\haiku{de respectieve {\textquoteleft}{\textquoteright}!}{paddenstoelen-cultuur}{in de woningen}\\

\haiku{Van Offenberg, dat.}{is zware concurrentie}{die u mij aandoet}\\

\haiku{Ik ook, ik ben heel,,....}{trotsch tegen haar geweest dat}{weet ik wel enne}\\

\haiku{Weldra stond ik in.}{de achtergalerij van}{Johnstone's woning}\\

\haiku{Eindelijk klopte:}{ik haar tot afscheid nog eens}{op den schouder}\\

\haiku{En zoodra zij,.}{zaten bood Henk sigaren}{aan schonk  ik thee}\\

\haiku{Nu toost ik op uw:}{lang verblijf alhier met dit}{geurig kopje thee}\\

\haiku{Wil-je dadelijk,,?}{gaan over veertien dagen met}{de volgende boot}\\

\haiku{{\textquoteright} Des avonds kwamen, als,.}{naar gewoonte de heeren}{op het theeuurtje}\\

\haiku{Mijn heer en meester.}{met zijn dankbaar kiekgezicht}{drentelde mij na}\\

\haiku{allen, ieder voor, {\textquoteleft}{\textquoteright}!}{zich had om diegrroote eer}{willen smeeken}\\

\haiku{{\textquoteright} Spaan sloeg zijn armen,:}{om mij heen tilde me op}{en droeg me naar bed}\\

\haiku{Sidin begon ook!}{met zoo raar te beven en}{zich flauw te voelen}\\

\haiku{Toean heeft mij zoo!}{gezegd niet te vergeten}{u in te geven}\\

\haiku{Er klonk angstige....}{droefheid uit die wanhopig}{berustende stem}\\

\haiku{niet meer was en heel.}{goed een behoorlijk toilet}{had kunnen maken}\\

\haiku{maar dat weet je wel,,!}{kind ik moet hier blijven tot}{mijn contract om is}\\

\haiku{hier te blijven, maar.}{juist de groote apathie waarvoor}{dokter Spaan bang is}\\

\haiku{en nu wilden zij.}{het je zoo gemakkelijk}{mogelijk maken}\\

\haiku{Geen wonder dat je,!}{het niet te boven komen}{zou mijn eenige schat}\\

\section{Virginie Loveling}

\subsection{Uit: Bina}

\haiku{Zijn boerderij was ':}{de grootste van het dorp en}{t omliggende}\\

\haiku{Lietje en Merie.}{geleken noch op Bina}{noch op elkander}\\

\haiku{{\textquoteleft}Er bestaat maar een.}{vrouwmensch op de wereld voor}{my en gij zijt het}\\

\haiku{maar gij, rijke boer,.}{zult toch uw dochter in haar}{hemd niet laten gaan}\\

\haiku{dat is te vetten,,.}{elk  in een kleine ren}{in een donker kot}\\

\haiku{{\textquoteright} zei Deodaat, haar.}{minzaam van onderen op}{in de oogen ziende}\\

\haiku{Deodaat, hare,.}{verschijning beloerend had}{op de ruit getikt}\\

\haiku{wat die zich eens in,.}{het hoofd had gezet was er}{niet uit te krijgen}\\

\haiku{Ge kunt ze immers,,{\textquoteright}.}{kwijt zijn zoodra ge wilt}{verweet haar Merie}\\

\haiku{Het duurde eenige;}{dagen eerdat de dokter}{was gerustgesteld}\\

\haiku{Van gansche dagen,.}{sprak Bina schier geen woord meer}{stroef in zich gekeerd}\\

\haiku{Bina zoo werkzaam,,!}{zoo waakzaam scheen voor alles}{onverschillig thans}\\

\haiku{Zoodra het vet,}{in de pan siste schepte}{zij een lepel deeg}\\

\haiku{{\textquoteright} Wat Merie zei, toen,.}{ze kort daarop binnenkwam}{was te bevroeden}\\

\haiku{Wij kunnen het toch,,?}{niet helpen nietwaar mijnheer}{de onderpastoor}\\

\haiku{Bina was weder,.}{ter been doch tot den rang van}{slonsmeid afgedaald}\\

\haiku{Ik kom u vragen?....}{of Bina mij ten langen}{laatste hebben wil}\\

\haiku{{\textquoteright} was het eenige wat,.}{haar verbazing uitbracht toen}{ze hem ontwaarde}\\

\haiku{{\textquoteright} zei hij gebiedend,,.}{als boosaardig in zijn smart}{en het geschiedde}\\

\haiku{Bij het vloerstroo bleef,.}{hij staan als ontdekte hij}{voor het eerst iets nieuws}\\

\haiku{Was het niet genoeg, {\textquoteleft}?}{dat baasken van Dorpeden}{penning werd gejond}\\

\haiku{daar een, nog recht, zijn;}{rosse flarden-armen in}{gebed openhoudend}\\

\haiku{{\textquoteright} {\textquoteleft}Eene belofte, gij,?}{zult mij niet bespieden niet}{volgen in mijn tocht}\\

\haiku{Dien had hij reeds vast,.}{terwijl hij omzichtig de}{trappen afdaalde}\\

\haiku{De daders waren....}{tot dusverre aan elke}{hinderlaag ontsnapt}\\

\haiku{Niet straffeloos had.}{Jasper het fijne polsje in}{zijne hand gedrukt}\\

\haiku{Zijn ergernis en....}{zijne verootmoediging}{waren grenzeloos}\\

\haiku{zijne schreden hem,.}{van zelf naar de kade waar}{hij vroeger woonde}\\

\haiku{Hij wreef de hand over,.}{zijn voorhoofd waar het koude}{zweet op parelde}\\

\haiku{{\textquoteright} En zoo werd het nieuws,.}{van deur tot deur van mond tot}{mond overgeleverd}\\

\haiku{Waren zij allen,?}{te beklagen zoo diep als}{hun ellende scheen}\\

\haiku{Hij sprak een persoon;}{of twee aan met zijn holle}{schooiersbetoning}\\

\haiku{En hij vertelde.}{de geschiedenis van den}{brief en het erfdeel}\\

\haiku{{\textquoteleft}Onze vader die,{\textquoteright}:}{in de hemelen zijt aan}{de deuren en zei}\\

\haiku{Hij betrok het dan.}{ook mits zesmaandelijksche}{betaling voorop}\\

\haiku{{\textquoteright} {\textquoteleft}Die wat krijgt moet wat,{\textquoteright},.}{geven antwoordde Peetje}{als een hondgeblaf}\\

\haiku{{\textquoteleft}Mijd u,{\textquoteright} zei de meid,.}{tegen hem met een emmer}{kolen aankomend}\\

\haiku{{\textquoteright} Peetje ging naar den,,.}{vlaskoopman deed zijn beklaag}{hevig uitvallend}\\

\haiku{{\textquoteright} {\textquoteleft}Och Heere, mijn geld,,,!}{mijn schoon geld alles kwijt en}{het zoo noodig hebben}\\

\haiku{ik zou gaan, nu eens?}{stappen in deze dan in}{gene richting deed}\\

\haiku{Lag de stad v\'o\'or mij,?...}{lag ze aan mijn rechterhand}{of aan mijn linker}\\

\haiku{{\textquoteright} vroeg ik mij af, en.}{toen trok ik doelloos verder}{met haastigen stap}\\

\haiku{{\textquoteleft}Bellotje,{\textquoteright} zei ik,,.}{schuchter als een schooljongen}{die iets heeft misdaan}\\

\haiku{Een kopje met melk, '.}{stond onaangeroerd naast haar}{opt vensterbord}\\

\haiku{Kalmte is in mijn,.}{gemoed gekomen kalmte}{en melancolie}\\

\haiku{{\textquoteleft}Mejuffrouw, ik kan.}{u niet genoeg danken voor}{uw gedienstigheid}\\

\haiku{Ik keek door 't raam,;}{zag boeren en boerinnen}{trekken naar de kerk}\\

\haiku{Du gr\"unst nicht nur in,.}{Sommerzeit Im Winter auch}{wen's friert und schneit}\\

\haiku{Dat struikelen in!}{de bramen op den barm van}{genen diepen weg}\\

\haiku{Ik wendde 't hoofd.}{ter zijde en worstelde}{om los te komen}\\

\haiku{Wel zag ze dat ik,.}{bleek was en gedrukt dat ik}{geen eetlust meer had}\\

\haiku{De boomen staan  ,.}{nog naakt maar lentewalmen}{hangen in de lucht}\\

\haiku{20 Maart 19... ~ Welk?}{een goddelijke hand heeft}{mij daarheen gestuurd}\\

\haiku{{\textquoteleft}Moeder heeft nooit weer.}{het portret van Cecile}{willen bekijken}\\

\haiku{Omdat ik het zoo,{\textquoteright}}{schoon vind stamelde ik met}{het beeld in de hand.}\\

\haiku{{\textquoteright} {\textquoteleft}Of al kwam het zelfs,{\textquoteright}.}{niet levend ter wereld zei}{ze godsdienstschendend}\\

\haiku{Ofschoon op 't feit,:}{betrapt trachtte hij zich te}{verontschuldigen}\\

\haiku{Zij wachtte op de,.}{diligence die destijds}{nog de spoor verving}\\

\haiku{In het vervolg werd.}{nooit een woord meer tusschen hen}{over die zaak gerept}\\

\haiku{Hij trad juist binnen,,.}{hij had staan luisteren hij}{had alles gehoord}\\

\haiku{Hij sleepte zich tot.}{aan de voordeur en trok den}{ondergrendel uit}\\

\haiku{'t Was ook niet noodig,.}{iets er bij te voegen het}{feit sprak voor zich zelf}\\

\haiku{Dit jaar was 't heel.}{den morgen een geratel}{en gerij geweest}\\

\haiku{De stoet kwam langs den.}{Steenput voorbij in zijne}{omreis rond het dorp}\\

\subsection{Uit: Erfelijk belast}

\haiku{ge zoudt mij zeker?}{in een muit willen houden}{gelijk een vogel}\\

\haiku{Madame D'Haeyer.}{zat uit te kijken aan haar}{bovenvenstertje}\\

\haiku{Mietje donderde.}{los op haar en ik spaarde}{haar ook geen verwijt}\\

\haiku{Zij had Colette,,,:}{vroeger nog gezien ging bij}{haar gedrukt en sprak}\\

\haiku{{\textquoteright} {\textquoteleft}Goed,{\textquoteright} sprak het kind met.}{sombere schaduwen van}{wanhoop in het oog}\\

\haiku{Het was te zien, dat.}{het er bedrijvig toeging}{en geen werk ontbrak}\\

\haiku{Berenice had.}{nooit andere hoven dan}{stadshoven gezien}\\

\haiku{{\textquoteleft}Tante, wie was die ',?}{kleine int groen met het}{anker in de hand}\\

\haiku{Welnu, dat duurt reeds,,,....}{laat zien wel vijftien jaren}{zij wachten naar haar}\\

\haiku{Hoe innig lief, steeds,.}{hopend dat de beterschap}{ernstig zou wezen}\\

\haiku{dien wierp zij eerst los,.}{over zich toen waagde ze het}{hem aan te trekken}\\

\haiku{Te Vroden had hij.}{geen leeftijdsgenooten}{van zijn geestespeil}\\

\haiku{de dagen waren,,,,.}{kort koud mistig donker ten}{beste dat het ging}\\

\haiku{Otto wist het wel.}{en herhaalde het dikwijls}{genoeg aan zich zelf}\\

\haiku{oom Eed schudde het,.}{hoofd en zei rechtuit dat hij}{het niet begeerde}\\

\haiku{Colette zuchtte.}{reeds met het vooruitzicht van}{een dronken thuiskomst}\\

\haiku{Tante, weet ge wat,.}{breng Berenice eens mee}{als ge terugkomt}\\

\haiku{Ondanks dat kwamen.}{de klanten na de vroegmis}{en v\'o\'or de hoogmis}\\

\haiku{Er was een hanglamp,,.}{en toen die brandde werd het}{eerst recht genotrijk}\\

\haiku{{\textquoteright} {\textquoteleft}Zeker niet, zeker, '.}{niet doch spreken wij niet meer}{overt gebeurde}\\

\haiku{Ik u wel, ik ging, '.}{u te gemoet toenk u}{hierin zag vluchten}\\

\haiku{Berenice mocht,.}{het niet voort vertellen was}{haar aanbevolen}\\

\haiku{Wat was het toch, dat?}{haar zoo drukte op dien tot}{vreugd bestemden avond}\\

\haiku{{\textquoteright} {\textquoteleft}Mijn eigen hart rijdt,,}{op een karreken16 als ik}{dat allemaal hoor}\\

\haiku{{\textquoteleft}Clette, er mag toch?}{zeker wel een druppel17 af}{op zulk een einde}\\

\haiku{wij hebben reeds te,{\textquoteright}:}{veel tijd verloren en hij}{nam een boek ter hand}\\

\haiku{Toen zei hij in eens,:}{den loop van inwendige}{gedachten uitend}\\

\haiku{{\textquoteleft}Het is een schande,}{zulke kinderen op de}{wereld te brengen}\\

\haiku{En hij dacht aan de,.}{tormenten der hel waarvan}{de priesters spraken}\\

\haiku{Zij hoorde hem aan,,}{reeds half verloomd en telkens}{was het haar weder}\\

\haiku{{\textquoteright} en zij nam al haar,.}{krachten te zamen stond op}{en wankelde heen}\\

\haiku{Aan 't avondmaal werd.}{hij tot de werkelijkheid}{teruggeroepen}\\

\haiku{Dat geestbesliste:}{in hem breidde zich tot het}{stoffelijke uit}\\

\haiku{Tante Colette,.}{trad binnen ze waarschuwde}{dat het etenstijd was}\\

\haiku{hij gaat ten onder,!}{van gebrek van uitputting}{en vermoeienis}\\

\haiku{{\textquoteright} en hij strekte de, {\textquoteleft}?}{armen uit boven zijn hoofd}{wilt ge nu heengaan}\\

\subsection{Uit: Een revolverschot}

\haiku{Hij bracht een handvol,.}{kaartjes een goeden wensch en}{een plakalmanak}\\

\haiku{Men moet overal een {\textquotedblleft}{\textquotedblright},{\textquoteright}.}{harden pakken op zulke}{dagen zei de een}\\

\haiku{Zij nam 't al eens, '.}{op den arm zij mindet}{eindelijk met drift}\\

\haiku{Toen Georgine,.}{elf jaar telde was Marie}{er twee en twintig}\\

\haiku{De kleine{\textquoteright}, aldus,.}{noemde haar Marie en hun}{vader zei het na}\\

\haiku{hij scheen overal een.}{zonneglans van opwekking}{mede te brengen}\\

\haiku{Altijd zeker is,.}{het dat hij het goed van zich}{wist af te schudden}\\

\haiku{Ik heb haar gezien,,,.}{een schoone doode men zou}{zeggen dat ze slaapt}\\

\haiku{En waar eertijds, en,:}{gedurende zoovele}{jaren de plaat met}\\

\haiku{Luc Hancq en de:}{twee raadsleden-jonkmans}{deden hetzelfde}\\

\haiku{{\textquoteright} met haar vingeren.}{woelde zij werktuiglijk de}{mulle aarde om}\\

\haiku{hij wist het immers,,...}{wel hij moest het weten dat}{zij veel van hem hield}\\

\haiku{Waarom liet hij zich?}{van den eersten den besten}{de loef afsteken}\\

\haiku{Zij gingen rond op ':}{t enge ruim en keken}{naar alle winden}\\

\haiku{{\textquoteleft}Marie, dat is niet,{\textquoteright},.}{wel gedaan sprak hij ontsteld}{zichtbaar beangstigd}\\

\haiku{{\textquoteleft}Uw beenen zijn jonger,,{\textquoteright}.}{dan de mijne ga zelve}{zei ze eenigszins scherp}\\

\haiku{Zij was donkergroen,,.}{aan fluweel gelijk niet of}{weinig vertreden}\\

\haiku{{\textquoteright} herhaalde Marie,, {\textquoteleft} '?}{verslagenwaarom hebt ge}{t mij niet gezegd}\\

\haiku{{\textquoteleft}Mijnheer Luc Hancq,, '?}{heeft niet gewild dat ik u}{riep verstaat get}\\

\haiku{Georgine scheen,.}{zoo onverschillig zoo niet}{vijandig gestemd}\\

\haiku{Het regende thans,.}{bijna bestendig zoodat ze}{zich binnen hielden}\\

\haiku{De beide zusters.}{troostten hem in de maat van}{het mogelijke}\\

\haiku{Zou Luc misschien een?}{voorwendsel zoeken om te}{Vroden te blijven}\\

\haiku{Hij kwam niet dien avond '.}{daar er repetitie van}{t muziekkorps was}\\

\haiku{ginder, dan alleen!}{met het gruwelijk geheim}{bekend te wezen}\\

\haiku{Sprakeloos had ze,.}{het hoofd geschud dat ze er}{niet bij gaan wilde}\\

\haiku{{\textquoteright} zei ze wrevelig, {\textquoteleft},,.}{doe uw werk ik zal bellen}{als ik u noodig heb}\\

\haiku{Stasius werd door.}{den vrederechter van het}{kanton ondervraagd}\\

\subsection{Uit: De twistappel}

\haiku{Die broeders zouden ',.}{t saam bewonen er hun}{leven eindigen}\\

\haiku{Weldra was al het,;}{grof werk gedaan de metsers}{en de dienders weg}\\

\haiku{{\textquoteleft}Indien alzoo een!}{schepselken eens eeuwig in}{de hel moest branden}\\

\haiku{en mijnheer die niet, '!}{thuis ist zal al aan mij}{geweten worden}\\

\haiku{{\textquoteright} Zij ijlde vooraan,.}{toomloos heen gedreven als}{een waanzinnige}\\

\haiku{{\textquoteright} Was 't eerste, wat;}{ze aan Kathelijntje met}{doffe stemme vroeg}\\

\haiku{haar glimlach was een,.}{zoen elke handeling een}{streeling voor het kind}\\

\haiku{zij hinkte pijnlijk,.}{eene hand steunzoekend in}{de zijde houdend}\\

\haiku{maar nu gebeurt dat,.}{allemaal niet meer aldus}{het geloof is weg}\\

\haiku{{\textquoteright} gewichtig knikte, {\textquoteleft},,.}{Petruskiekens konijnen}{en duiven juffrouw}\\

\haiku{Aan het hek staan was, '.}{streng verboden door mijnheer}{zij waagdet niet}\\

\haiku{{\textquoteright} antwoordde mevrouw, {\textquoteleft},.}{Duquennehij heet Gaspard}{hij behoudt zijn naam}\\

\haiku{Gaspard is een van, '.}{de Drie Koningen ik heb}{t u reeds verklaard}\\

\haiku{Fernande bedacht.,..}{zich verstandiglijk het voor}{en tegen wikkend}\\

\haiku{Dit dacht hij nu en.}{dankbaarheid borrelde weer}{boven in zijn hart}\\

\haiku{Thans brak voor hem en:}{voor Fernande een tijdperk}{aan van groot geluk}\\

\haiku{Doch ze wisten 't,:}{niet en velen moesten wel hun}{meening deelen want}\\

\haiku{{\textquoteright} antwoordde deze,.}{geringschattend zich tot de}{anderen richtend}\\

\haiku{Niets, niets, het was een,{\textquoteright}.}{grap aldus wilde Petrus}{het verbeteren}\\

\haiku{zij was zijn eigen.}{moeder niet en had zich als}{deze aangesteld}\\

\haiku{Op zijne kamer;}{gesloten heel den dag te}{water en te brood}\\

\haiku{En Fernande stond:}{pal onder dien stortvloed van}{beschuldigingen}\\

\haiku{{\textquoteleft}Heere, verleen mij!}{de kracht om die beproeving}{te doorworstelen}\\

\haiku{Diep kwetste 't haar doch.}{zij was veel te fier om het}{te laten merken}\\

\haiku{de zekerheid en....}{de bevestiging van het}{reeds lang gevreesde}\\

\haiku{In zijne hand houdt....}{hij nog enkele bladen}{van een reisgids vast}\\

\haiku{Zij tilt hem overeind,:}{in hare armen drukt zijn}{hoofd aan hare borst}\\

\haiku{Hem met hare hand,.}{opgetild hem aan haar hart}{geaaid als een kind}\\

\haiku{een stofje, een atoom,,,....}{onzichtbaar iets wegblaasbaar}{weggeblazen reeds}\\

\haiku{haar bloote borst was met,}{zeven zwaarden doorstoken}{en plots begreep hij}\\

\haiku{hij was thans aan zijn {\textemdash}?}{laatste examen altijd het}{recht verdedigen}\\

\section{Lambert Rijckxz Lustigh}

\subsection{Uit: 'Kroniek I' van Lambert Rijckxz Lustigh (1656-1727)}

\haiku{Gevangenneming-;}{en executie Dick Jansen}{Spilt151153~1711}\\

\haiku{, en dat hij oock,}{wel staande Houde en noch}{met eenen dierbaren}\\

\haiku{daar mede hare}{keleren tongen wascht}{ende op den 16}\\

\haiku{van hem sijn Laaste:}{drie schoone koijen af}{op den 26 decemb}\\

\haiku{hebbe u besten}{gedaan om uwe koijen}{sieck te maken dogh}\\

\haiku{1714 doen sterft van klaas:}{meijnsen sijn Laaste koe}{Op den 12 febr}\\

\haiku{Ik meende dat Ik ',}{een deuselingh int hooft}{kreegh daarom soo gingh}\\

\haiku{de grietenijen}{terwetardeel en beijde}{de dongalen staan}\\

\haiku{alles wat uijt de,:}{voorz vaart sal proveneren}{ook geven haar Ed}\\

\haiku{van Ebbe Willemsz}{koij voor seven stuvers aen}{rottekruijt gekogt}\\

\haiku{1721 van amsterdam}{in koetsen tot weesp wierden}{gevoert om aldaer}\\

\haiku{1722 wiert gedaan,}{een waar en waragtighe}{beschrijvinge uijt}\\

\section{Christiaan Creemers, Jos. Habets, Maria Luyten en A. Nieuwenhuizen}

\subsection{Uit: Kronijk uit het klooster Maria-Wijngaard te Weert, 1442-1587. Eene bijdrage tot de voorgaande kronijk, op het jaar 1566. Een vijftal stukken betrekkelijk de Hervorming te Weert 1583-1584}

\haiku{wij meynden dat de;}{kerck op ons hooft soude}{gevallen hebben}\\

\haiku{drij stonden gebaert;}{op de spincamer tusschen}{ider pilaer \'e\'ene}\\

\haiku{daerom is sij;}{verhoort en de gevangen}{sijn vrijgelaeten}\\

\haiku{waeren wij in seer,.}{grooten noot en lijden niet}{wetende wat doen}\\

\haiku{\'e\'en pont boter twee,.}{stuijver min \'e\'en oort \'e\'en ton}{haringh aght gulden}\\

\haiku{en strackx naer den oogst.}{begonsten alle dingen}{seer dier te worden}\\

\haiku{sij beraede haer;}{met twee of drij  van de}{oudste susteren}\\

\haiku{wij hadden nogh soo.}{grooten schaede niet in het}{gewas des ackers}\\

\haiku{Eenige hebben hun;}{verhangen en eenige}{op wege geweest}\\

\haiku{Precies doen de clock.}{vier uuren sloegh wiert het light en}{men sagh het niet meer}\\

\haiku{Hij zou voortaan zijn.}{gebed aan de kerk-deur}{mogen verrichten}\\

\haiku{39Het handschrift van den.}{Heer Habets heeft Buren in}{plaats van Beijeren}\\

\chapter[24 auteurs, 3257 haiku's]{vierentwintig auteurs, drieduizendtweehonderdzevenenvijftig haiku's}

\section{H.H.J. Maas}

\subsection{Uit: Het goud van de Peel}

\haiku{- Dat kund' ummers nie... -,, '...}{Och erm dingske mit \'e\'e\"en}{haend draagk ow weg}\\

\haiku{En het eind was, dat.}{de meid over zes weken zou}{moeten vertrekken}\\

\haiku{Niemand nam immers? '}{werkvolk in dienst om het te}{laten leegloopen}\\

\haiku{Toen moest de mest over.}{het land gebracht worden voor}{het winterkoren}\\

\haiku{Ondanks het gezwoeg,.}{voelde zij dat een kou haar}{lijf overhuiverde}\\

\haiku{Dat kwam bij jonge, {\textquoteleft}{\textquoteright}.}{vrouwen wel meer voor vooral}{bij den eersten keer}\\

\haiku{en als de winter, '...}{om was zou ment weer duur}{moeten inkoopen}\\

\haiku{Hij wist er niets van,,.}{keek er verwonderd van op}{niet-geloovend}\\

\haiku{Hij was echter groot,.}{en sterk en daarom was hij}{heel vroeg gaan dienen}\\

\haiku{En de zorg begon,.}{pijnlijk te knagen aan hun}{leven voor het eerst}\\

\haiku{Was de winter toch,...}{maar om de zomertijd zou}{werk genoeg brengen}\\

\haiku{Maar kort daarna wist, {\textquoteleft}'{\textquoteright},.}{ze datt al zoo was toen}{ze den kerkgang deed}\\

\haiku{Ze was toen al weer {\textquoteleft}'{\textquoteright},.}{n heel eind maar ze spande}{zich in wat ze kon}\\

\haiku{Dan zou het ergste,.}{gelejen zijn beurden ze}{zich zelf hopend op}\\

\haiku{'t Was er een van....}{een ander dorp na een week}{ging-ie weer terug}\\

\haiku{dat het er met z'n.}{twaalf en een halven gulden}{leelijk zou uitzien}\\

\haiku{Maar de twee helpers,,!}{die Toon gevraagd had moesten toch}{ook betaald worden}\\

\haiku{Hij zat in 'n klein,.}{huisje met z'n vrouw die men}{zelden te zien kreeg}\\

\haiku{D'r werd gezopen,...}{zooveel als ze maar door de}{keel konden krijgen}\\

\haiku{... maar later werd het...}{verbranden op de meeste}{plaatsen verboden}\\

\haiku{Die uitdrukking had.}{de spreker onthouden van}{den burgemeester}\\

\haiku{Na eenige jaren.}{had de maatschappij al het}{grauwe veen verwerkt}\\

\haiku{de eenvoudige...}{dorpelingen ophitsen}{tegen het gezag}\\

\haiku{Bovendien bracht het.}{hem enkele honderden}{guldens per jaar op}\\

\haiku{Dat orgel-spelen...}{zou-ie voorloopig niet}{laten varen}\\

\haiku{'t Ging allemaal, '.}{zoo goed in zijn tijdt was}{een plezier geweest}\\

\haiku{Daarna sneed zijn stem,:}{knerpend als het scheuren van}{een ijzeren plaat}\\

\haiku{hij liet het nog eer, '...}{in brand steken dant voor}{zoo'n geld te geven}\\

\haiku{- Ja, het w\`as wel wat,..}{veel maar dat wilde hij er}{toch voor betalen}\\

\haiku{Haar broers verlolden.}{op \'e\'en avond meer dan hij in}{een week verdiende}\\

\haiku{Plotseling rukte,.}{het dier aan schichtig door het}{slaan en het geraas}\\

\haiku{Gewoonlijk werden.}{in plaats hiervan andere}{regels gezongen}\\

\haiku{Wat had hij toen veel!}{gezien zonder er iets van}{geleerd te hebben}\\

\haiku{Sinds hij echter den,.}{strijd aanvaard had kon hij dien}{niet meer ontvluchten}\\

\haiku{Als daar een kanaal...}{naar toe werd gegraven of}{een tram aangelegd}\\

\haiku{De meesten hadden,,.}{klompen bruin van aangeplakt}{moer aan de voeten}\\

\haiku{wel wil ik gaarne,...}{toegeven dat we niet te}{veel keuze hebben}\\

\haiku{daarom vind ik het,...}{juist zoo vreemd dat u niet van}{Den Schoolmeester houdt}\\

\haiku{Eindelijk begon,:}{hij geirriteerd doordat hij}{geen ontkomen zag}\\

\haiku{Neen, door den dag kwam. ' '.}{er weinigs Zaterdags}{ens Zondags meer}\\

\haiku{Het mes sneed er aan,,.}{twee kanten helderde hij}{nog op grinnekend}\\

\haiku{Hij had den naam van...}{in allerlei gemeene}{straatjes te sjouwen}\\

\haiku{Hoe konden ze nu?}{toch eigenlijk in zoo iets}{nog plezier hebben}\\

\haiku{Daar kan je jezelf,...}{gaan liggen kietelen as}{je wat hebben wilt}\\

\haiku{- Ja, die was verdomd,...,...}{goed z\`eg en een mooie meid ook}{die de mop voordroeg}\\

\haiku{Ze zetten hem op,.}{de tafel en dan sloeg ie}{een onzin uit bar}\\

\haiku{Aan het station.}{Lizaveen stapte niemand}{in dan De Visscher}\\

\haiku{ik had niet gedacht,...}{dat ik mij in dien mensch zou}{kunnen vergissen}\\

\haiku{En wat ze na haar,.}{dood kon nalaten dat was}{voor de familie}\\

\haiku{Opeens wendde zij,.}{zich naar hem toe of-ie niet}{naar de hoogmis ging}\\

\haiku{Deed bestraffend, dat '...}{de kapelaant toch ook}{al goed zou meenen}\\

\haiku{Waar haalden zij de?}{kunst vandaan om hun leven}{zoo in te richten}\\

\haiku{Hij ging nog wel naar, '.}{de kerk maar dat deed-ie}{alleen voort oog}\\

\haiku{'t Schijnt, dat ik in...}{de wieg gelegd ben om den}{strijd uit te vechten}\\

\haiku{Voor het onderhoud.}{van het gezin schiet er dan}{zooveel niet meer over}\\

\haiku{Nou, dan maar voor drie...?...}{maanden en zou er dan goed}{voor gezorgd worden}\\

\haiku{De gemeente zou,.}{er wel bij varen als zij}{haar veen verkocht had}\\

\haiku{Het bevatte een:}{lange correspondentie}{als hoofdartikel}\\

\haiku{iemand zijn toevlucht,.}{neemt die verblind is door haat}{tegen het goede}\\

\haiku{Beter dan thuis,  .}{want zij hadden geen tijd om}{maar hen te kijken}\\

\haiku{ik dacht, breng het prul,...}{aan meneer pastoor dat die}{het verbranden kan}\\

\haiku{Om het verband goed,.}{te kunnen leggen spelde}{hij het hemd omhoog}\\

\haiku{Na een paar dagen.}{verzonden zij de copie}{naar de drukkerij}\\

\haiku{- Ja, jongen, 'k zal... - '...}{toch weg moetent Leven}{is niet makkelijk}\\

\haiku{Nu vele Groeten.}{van mijn vrouw of Fientle wie}{ik ze noemen wil}\\

\haiku{Hen uit de verdienst,,.}{schoppen alsof zij honden}{waren dat kon-ie}\\

\haiku{Daar moest het publiek,...}{buiten blijven het had er}{niets mee te maken}\\

\haiku{Ja, zooals verteld werd, ',...}{wast h\`em immers ook zoo}{gegaan met zijn baas}\\

\haiku{Maar met de verdienst '.}{wast bij de maatschappij}{toch een boel beter}\\

\haiku{Zij zelf hadden voor.}{hun trouwen in Pruisen een}{ledikant gekocht}\\

\haiku{in 't ongeluk,...}{stooten als-ie een woord te}{v\'e\'el naar hun zin zei}\\

\haiku{En hij was dan toch...}{een bittere vijand van}{den secretaris}\\

\haiku{Jennesen en zijn.}{aanhangers waren woedend}{over zijn kandidaatuur}\\

\haiku{Maar de ander was,...}{toch wel bang dat Van Eijzen}{het niet halen zou}\\

\haiku{Allee, de lui moesten,...}{maar drinken al kostte het}{hem duizend gulden}\\

\haiku{Zijt gij vergeten,?}{dat de steenen van den oven van}{een raadslid kwamen}\\

\haiku{Er kwamen handen.}{tekort om jenever en}{bier aan te voeren}\\

\haiku{Als het ingevuld,.}{was met den naam Van Eijzen}{dan deugde het niet}\\

\haiku{'s Morgens hing een.}{mand zonder bodem bij van}{Eijzen aan de deur}\\

\haiku{De gemeente moest,.}{ook voor de godsdienstige}{belangen zorgen}\\

\haiku{Hij woelde door zijn, {\textquoteleft}{\textquoteright}...}{bed al maar pratend over turf}{en aken enstriekke}\\

\haiku{Fien en Dien snikten.}{hardop en vluchtten uit het}{kleine slaaphokje}\\

\haiku{Ze moest zeker dien...}{vreemden bocht ook nog laten}{zat-vreten}\\

\haiku{Hij had gezegd, dat...}{de vrome menschen niet te}{vertrouwen waren}\\

\haiku{De regeering zoog hen,.}{uit deed totaal niets om den}{landbouw te steunen}\\

\haiku{Bij het onderzoek...}{van hoogerhand was er zelfs}{nog te v\'e\'el in kas}\\

\subsection{Uit: Landelijke eenvoud}

\haiku{Het was dan  toch,.}{ook schand voor de ouwers als}{de dochter zoo deed}\\

\haiku{Ja, het eerste jaar,!}{maar over een tijdje zou het}{wel anders worden}\\

\haiku{En die was tegen,...}{haar uitgevallen dat ze}{haar mond moest houden}\\

\haiku{Dit zou ze haar voor,,,,....}{de voeten gooien neen nog}{beter harder z\'o\'o}\\

\haiku{ge het ummers ow,...}{meid ien de stad daor zien}{ja veul m\`oier megjes}\\

\haiku{Dat was zoo maar een,,.}{kaart geweest beweerde hij}{om te versturen}\\

\haiku{Ze wist, dat moeder,.}{dan bad dat deed ze altijd}{in zoo'n gevallen}\\

\haiku{Iederen dag zou, '....}{ze voortaan naar hem toegaan}{dat zet zagen}\\

\haiku{Hard zette zich haar.}{willen tegen het geweld}{van haar ouders in}\\

\haiku{Nel hoefde d'r nooit,.}{op te rekenen dat hij}{zijn toestemming gaf}\\

\haiku{De trekken van zijn,.}{gezicht verstrakten omdat}{Nel nog tegenhield}\\

\haiku{In vol ontwaken.}{van hun verlangen hielden}{zij elkander vast}\\

\haiku{maar niet meer schreien...,...}{maar naar huis gaan als vader}{en moeder riepen}\\

\haiku{De dokter trachtte,.}{haar te bedaren dat het}{toch z\'o\'o erg niet was}\\

\haiku{Ze zou nu wel gaan,.}{trouwen en dan was alle}{leed weer vergeten}\\

\haiku{dat iets vreeselijks,.}{haar wachtte van haar ouders}{en van de menschen}\\

\haiku{Maar zij begreep het, {\textquoteleft}{\textquoteright}.}{wel dat kwam allemaal van}{hetfusegerei}\\

\haiku{Ze lieten de lui, {\textquoteleft}{\textquoteright}...}{maar raak kletsen die konden}{hunde mars lekken}\\

\haiku{eigenlijk zou men...}{met h\`em geen medelijden}{hoeven te hebben}\\

\haiku{'t Was nu eenmaal,,.}{zoo en waar d'r twee kijven}{hebben d'r twee schuld}\\

\haiku{Klokslag twaalf ging hij,.}{aan tafel zitten of het}{eten klaar was of niet}\\

\haiku{Terwijl hij wegging,.}{snauwden ze elkander de}{scheldwoorden nog toe}\\

\haiku{In alles deed hij,.}{zijn wil zonder haar zelfs ooit}{iets te vertellen}\\

\haiku{*** Driek had thuis nog niet, {\textquoteleft}.}{dadelijk gezegd dat hij}{met Nelmoest trouwen}\\

\haiku{Hij meende zeker,,.}{dat hij den vogel af had}{als hij getrouwd was}\\

\haiku{De teleurstelling.}{zweepte zijn woede nog aan}{tot h\`arder zwoegen}\\

\haiku{Voor een beetje geld.}{kan hij dan toch zonder zorg}{alles afwachten}\\

\haiku{Zijn kameraden.}{durfden eerst niet ronduit zijn}{partij te kiezen}\\

\haiku{Hannes Knik zeurde.}{zijn haat uit tegen zijn}{eigen gezelschap}\\

\haiku{Goemans raasde nog,:}{na zijn vuist telkens op de}{tafel bonkerend}\\

\haiku{Hij had gezien van,.}{stadsche heeren Hoe hij moest}{gaan in fijne kleeren}\\

\haiku{Maar mooie Nel kon hem,,.}{behagen Hij dacht ik zal}{het maar eens wagen}\\

\haiku{Die lag als een brand.}{op zijn ziel en gloeide den}{haat voortdurend aan}\\

\haiku{Eindelijk, opeens,,.}{schoten de schreeuwstemmen uit}{allemaal gelijk}\\

\haiku{Zij zou er geld voor,..}{kwijt willen zijn als Driek die}{Nel nooit had gezien}\\

\haiku{- Jao mer, ik wee\"et,..}{ummers van niks hoe wil ik}{wat kunne zegge}\\

\haiku{Als ze zeiden, dat,.}{ze wat gezien hadden dan}{zou\"en ze liegen}\\

\haiku{Voor h\`a\`ar was daarom ',,.}{alt zware werk spitten}{zaaien en maaien}\\

\haiku{Dikwijls moest ze even,,.}{rusten steunend op de schop}{tot het wat over trok}\\

\haiku{Want het zou er nu.}{toch wel alle dagen om}{te doen kunnen zijn}\\

\haiku{Langzaam reide zich.}{de eene zwarte aardevoor}{aan de andere}\\

\haiku{Daar, ze begreep n\`og, '.}{niet dat zer het leven}{bij had gehouden}\\

\haiku{Anders mankeeren die.}{mevrouwen altijd w\`at in}{de laatste dagen}\\

\haiku{Daar was niks aan te,..}{veranderen zoo was de}{wereld geschapen}\\

\haiku{En wie trouwde er,?}{in dezen slechten tijd nog}{v\'o\'or het moetens werd}\\

\haiku{Die dan, begrijpend,.}{elkander aankijken en}{naar Willem wijzen}\\

\haiku{Alle middelen.}{wendden zij aan om hem dat}{gebrek af te leeren}\\

\haiku{Zijn dunne lippen.}{scherpen zich vast op elkaar}{tot een harden rand}\\

\haiku{'k wil niet, 'k doe '.. -, '.}{t niet Verschoor ik hebt}{goed met je gemeend}\\

\haiku{'t Is jou eigen,,?}{schuld dat je straf hebt zie je}{dat nou zelf ook niet}\\

\haiku{Hoe kan ik dan je, '?..}{belofte gelooven dat je}{t niet meer doen zult}\\

\haiku{Vlijmend wee doorschrijnt, '..}{zijn borst zijn ziel kreunt vant}{jammerende leed}\\

\haiku{Nogmaals klopte ze:}{op de deur en riep met eene}{mistroostige stem}\\

\haiku{{\textquoteleft}'k Wacht op iemand,{\textquoteright},,}{antwoordt Marie maar hij zal}{toch niet meer komen}\\

\haiku{Zijn hart sprong op van.}{blijdschap en hij begon nog}{sneller te loopen}\\

\haiku{ze zou trachten zich.}{niet meer door haar gevoel te}{laten medesleepen}\\

\haiku{Och, kom, ik vind het,.}{nu zelfs aardig om me zoo}{eens te verrassen}\\

\subsection{Uit: Peel omnibus}

\haiku{Ja, als haar moeder,;}{weer beter was zou zij een}{nieuwe dienst zoeken}\\

\haiku{Nee, in het dorp had,.}{ze niks in de stad was het}{veel plezieriger}\\

\haiku{Ze moest maar naar huis,.}{gaan als ze niet wilde doen}{wat zij aanraadden}\\

\haiku{Door de praatjes van. '}{het volk zouden die daartoe}{genoodzaakt worden}\\

\haiku{En het is God zelf,,.}{die het gezegd heeft dat er}{armen moeten zijn}\\

\haiku{Hij had een huisje,.}{op het oog met wat grond}{dat paste hem juist}\\

\haiku{In bijzijn van haar.}{man durfde Floortje hem niet}{meer voor te spreken}\\

\haiku{Toen hadden ze een,.}{zware mars gehad honger}{en dorst geleden}\\

\haiku{Eer de dag om was,.}{was het gepraat het hele}{dorp doorgetrokken}\\

\haiku{De zieke ging hard '.}{achteruit en opeens was}{t afgelopen}\\

\haiku{Het zou een eerste,.}{klas begrafenis zijn zeer}{deftig en plechtig}\\

\haiku{Als hij maar eens kwam,.}{en wat geld meebracht dan zou}{het wel weer goed zijn}\\

\haiku{Jan moest maar zuipen,, '.}{zoveel als hij lusttet}{kwam er niet op aan}\\

\haiku{En dikwijls al had,.}{het hele dorp verteld dat}{hij weer ging trouwen}\\

\haiku{Dat zou wat anders.}{zijn dan zich bij de boeren}{kapot te werken}\\

\haiku{De vrienden zouden.}{hen uitgeleide doen naar}{het station}\\

\haiku{Wat van zijn verdienst,?}{afgeven waarvoor hij zo}{hard moest arbeiden}\\

\haiku{Alleen een klaagbrief.}{van zijn moeder kwam Jan nu}{en dan hinderen}\\

\haiku{Ik wil hem nu niet,,!}{in huis hebben ziet ge en}{daarmee is het uit}\\

\haiku{Allemaal naar de,,!}{piotten potverdikke}{vivat de piot}\\

\haiku{Maar de Rooie van de.}{wethouder kwam met hoge}{stem tussenbeide}\\

\haiku{Wankelde nu naar,.}{deze dan naar die kant door}{het tartend stoten}\\

\haiku{Het ergste is dat.}{de ouwe niet wil hebben}{dat ik ga dienen}\\

\haiku{Anders laten ze,!}{je zitten met je armoe}{die mooie kanaljes}\\

\haiku{{\textquoteright} Vrouw Van der Poorten.}{kwam neven de aanbouw en}{liep juist bij Jan uit}\\

\haiku{Terwijl zij met vlug.}{armbeweeg de stenen aan}{elkander rijden}\\

\haiku{Ze maakte het met,.}{Jan nooit af al moest ze met}{hem weglopen}\\

\haiku{'t Was niet mooi van,.}{Jan dat hij daarover zoveel}{te mopperen had}\\

\haiku{Gauw genoeg zou hij.}{een-te-veel zijn bij}{de Van der Poortens}\\

\haiku{Veranderen ging.}{niet meer of er zou grote}{herrie van komen}\\

\haiku{Ze lieten het geld, '.}{wel rollent kwam er hun}{op een paar niet aan}\\

\haiku{Op een plaats werd zelfs.}{op een avond de lamp van de}{zolder geslagen}\\

\haiku{Zijn kameraden ' ',.}{vant werk meendent niet}{kwaad dat wist hij wel}\\

\haiku{Liepen toen nog eens,.}{over het terrein naar alle}{kanten uitkijkend}\\

\haiku{Zij had de beste '.}{arbeiders dan ook maar voor}{t nemen gehad}\\

\haiku{De burgemeester.}{had op haar vragen niet veel}{willen antwoorden}\\

\haiku{Nou verdomde hij.}{het toch ook om nog langer}{te gaan bedelen}\\

\haiku{Dat Mien nou zelf wel.}{zag dat ze maar met de Rooie}{had moeten trouwen}\\

\haiku{'t Was een streek van.}{die kerel geweest om de}{boel te bedriegen}\\

\haiku{Een andere kant, ',.}{uitkijken doen of iem}{niet zag dat schoelje}\\

\haiku{Ik heb al een paar,{\textquoteright},.}{glazen bier gedronken loog}{ie groot-doende}\\

\haiku{Wie dat ni\'et kon, ',.}{wasn sul daar deden ze}{mee wat ze wouen}\\

\haiku{Maar de Rooie spotte:}{met gewichtig-doening}{in stem en gezicht}\\

\haiku{Jan had ook nooit met, '.}{Mien moeten trouwen dat was}{stom vanm geweest}\\

\haiku{Nou moesten ze maar eens.}{voor de dag komen met de}{lekkere brokjes}\\

\haiku{Vooruit, voor de dag,, '.}{er mee of hij zou eens zien}{watm te doen stond}\\

\haiku{En Van der Poorten,.}{ging stiekem aan de fles dat}{was toch nog erger}\\

\haiku{Waar dat naar toe moest,,.}{wisten ze niet maar dat kon}{z\'o toch niet blijven}\\

\haiku{Die meenden maar, dat.}{de mannen het met een bak}{koffie konden doen}\\

\haiku{Gelach en gepraat,.}{rumoerde druk rond \'al meer}{bezoekers lokkend}\\

\haiku{Dan waren ze als,,.}{duivels die anders toch heel}{goed waren nuchter}\\

\haiku{Eerst tussen Jan en ',.}{Mienr moeder en later}{met Mien zelf ook niet}\\

\haiku{Als men 't zo nam, ',.}{wast geen wonder dat Jan}{ook wel eens kwaad werd}\\

\haiku{Altijd en altijd,.}{datzelfde gezanik dat}{hij niets verdiende}\\

\haiku{En aansluitend bij:}{zijn eigen denken bitste zij}{hem schamperend toe}\\

\haiku{Zij zouden liever,.}{gehad hebben dat hij maar}{was weggebleven}\\

\haiku{Daar zo zat als een,.}{kwajongen die kijven kreeg}{over een domme streek}\\

\haiku{Alles moest weer goed,.}{worden geen oude koeien}{uit de sloot halen}\\

\haiku{'t Was dezelfde,.}{Jan niet meer als toen hij aan}{het klooster werkte}\\

\haiku{Een mens moest nog al.}{wat ondervinden om aan}{zijn eind te raken}\\

\haiku{In een hoek lagen.}{witte en zwarte klompen}{opeengestapeld}\\

\haiku{Langs de muur planken.}{met kruidenierswaren en}{katoenen stoffen}\\

\haiku{En al sloeg hij hem,.}{d'r neer dan kon niemand hem}{daar wat over maken}\\

\haiku{Van der Poorten zou,,.}{wel zorgen dat ie wat in}{huis had dreigde hij}\\

\haiku{Even ontmoetten de.}{blikken van de Rooie en die}{van Marie elkaar}\\

\haiku{Ge hebt vandaag uw,{\textquoteright} {\textquoteleft},,?}{portie wel gehad dunkt me.}{Kom kom wat is dat}\\

\haiku{De mensen zullen,,{\textquoteright}.}{wel denken dat hier ruzie}{is drift Drieka op}\\

\haiku{hij zijn arm los en.}{zonder verder op haar te}{letten kijft hij voort}\\

\haiku{U ophouden met,!}{dat wijf en mijn vrouw gemeen}{schoelje dat ge bent}\\

\haiku{Dan, verzoenend, stelt,.}{hij voor dat ze samen maar}{moeten afdrinken}\\

\haiku{Van der Poorten was ',.}{s middags uit geweest was}{zo zat als een snip}\\

\haiku{Zou nou wel liggen.}{te ronken als een os om}{zich uit te roesen}\\

\haiku{Een brede wonde,.}{gaapt in zijn achterhoofd als}{een zwarte holte}\\

\haiku{Een huivering van,.}{angst doorkilt haar lijf telkens}{als zij wat horen}\\

\haiku{Zo'n ouwe mens nog.}{geen tijd te geven om een}{jas aan te trekken}\\

\haiku{Het was dan toch ook,.}{schande voor de ouders als}{de dochter zo deed}\\

\haiku{{\textquoteright} 't Rood van lichte.}{opwinding schemerde door}{haar gelaatshuid heen}\\

\haiku{Ja, het eerste jaar,!}{maar over een tijdje zou het}{wel anders worden}\\

\haiku{En die was tegen,.}{haar uitgevallen dat ze}{haar mond moest houden}\\

\haiku{Als wij zoveel geld,...{\textquoteright} {\textquoteleft},?}{hadden danJa hoe zijn ze}{d'r aangekomen}\\

\haiku{Dit zou ze haar voor,,,,.}{de voeten gooien neen nog}{beter harder z\'o}\\

\haiku{Dat was zo maar een,,.}{kaart geweest beweerde hij}{om te versturen}\\

\haiku{Ze zouden wel gauw.}{tegenspoed krijgen op stal}{of op de akker}\\

\haiku{Met een angstkreet vloog,.}{ze op en krijste hem toe}{haar los te laten}\\

\haiku{En met angst-kracht,.}{trok ze hem van Nel weg weer}{uit de slaapkamer}\\

\haiku{Ze wist, dat moeder,.}{dan bad dat deed ze altijd}{in zo'n gevallen}\\

\haiku{Iedere dag zou, '.}{ze voortaan naar hem toegaan}{dat zet zagen}\\

\haiku{Nel hoefde d'r nooit,.}{op te rekenen dat hij}{zijn toestemming gaf}\\

\haiku{De trekken van zijn,.}{gezicht verstrakten omdat}{Nel nog tegenhield}\\

\haiku{In vol ontwaken.}{van hun verlangen hielden}{zij elkander vast}\\

\haiku{Hun blikken felden,.}{elkaar tegen in het wild}{bewogen gelaat}\\

\haiku{De dokter trachtte,.}{haar te bedaren dat het}{toch z\'o erg niet was}\\

\haiku{Ze zou nu wel gaan,.}{trouwen en dan was alle}{leed weer vergeten}\\

\haiku{Ja, maar vader en.}{moeder waren zo kwaad op}{de Jorissen}\\

\haiku{Maar zij begreep het,.}{wel dat kwam allemaal van}{de boterfabriek}\\

\haiku{'t Zou gemeen van,.}{hem zijn als hij het niet deed}{en onnozel ook}\\

\haiku{Driek zou ook wel eens,.}{zelf met haar vader spreken}{of Jorissen}\\

\haiku{Eigenlijk zou men.}{met h\'em geen medelijden}{hoeven te hebben}\\

\haiku{'t Was nu eenmaal,,.}{zo en waar d'r twee kijven}{hebben d'r twee schuld}\\

\haiku{Ze wou nog liever,.}{altijd hard werken als het}{maar geen zondag werd}\\

\haiku{Klokslag twaalf ging hij,.}{aan tafel zitten of het}{eten klaar was of niet}\\

\haiku{'n Schande voor zo'n,.}{ouwe kerel zo in de}{herbergen te doen}\\

\haiku{Nadat de ergste,.}{woede gelucht was bleef hij}{nog nazaniken}\\

\haiku{{\textquoteright} Terwijl hij wegging,.}{snauwden ze elkander de}{scheldwoorden nog toe}\\

\haiku{In alles deed hij,.}{zijn wil zonder haar zelfs ooit}{iets te vertellen}\\

\haiku{Driek had thuis nog niet, {\textquoteleft}{\textquoteright}.}{dadelijk gezegd dat hij}{met Nelmoest trouwen}\\

\haiku{Hij meende zeker,,.}{dat hij de vogel af had}{als hij getrouwd was}\\

\haiku{De teleurstelling.}{zweepte zijn woede nog aan}{tot h\'arder zwoegen}\\

\haiku{Maar zijn vader, die,;}{kinkelig zwetste dat hij}{nooit bang was geweest}\\

\haiku{Zijn kameraden.}{durfden eerst niet ronduit zijn}{partij te kiezen}\\

\haiku{{\textquoteright} {\textquoteleft}Ja, nou moet ie het,.}{zelf maar weten nou Nel er}{mee zit te kijken}\\

\haiku{Goemans raasde nog,:}{na zijn vuist telkens op de}{tafel bonkerend}\\

\haiku{Maar mooie Nel kon hem,,.}{behagen Hij dacht ik zal}{het maar eens wagen}\\

\haiku{Die lag als een brand.}{op zijn ziel en gloeide de}{haat voortdurend aan}\\

\haiku{Eindelijk, opeens,,.}{schoten de schreeuwstemmen uit}{allemaal gelijk}\\

\haiku{Zij zou er geld voor,.}{kwijt willen zijn als Driek die}{Nel nooit had gezien}\\

\haiku{Een ogenblik later.}{kwam Hannes met een grote}{troep binnenkabalen}\\

\haiku{Als ze zeiden, dat,.}{ze wat gezien hadden dan}{zouden ze liegen}\\

\haiku{Dan was er nog iets,,:}{een misbruik in sommige}{dorpen ingeroest}\\

\haiku{En het eind was, dat.}{de meid over zes weken zou}{moeten vertrekken}\\

\haiku{Er moest gegeten.}{worden en brood gaven de}{meesters in school niet}\\

\haiku{Niemand nam immers?}{werkvolk in dienst om het te}{laten leeglopen}\\

\haiku{Toen moest de mest over.}{het land gebracht worden voor}{het winterkoren}\\

\haiku{Ondanks het gezwoeg,.}{voelde zij dat een kou haar}{lijf overhuiverde}\\

\haiku{Dat de boerin maar,.}{eens wat mee aanpakte zo}{druk was het niet meer}\\

\haiku{Hij was echter groot,.}{en sterk en daarom was hij}{heel vroeg gaan dienen}\\

\haiku{En de zorg begon,.}{pijnlijk te knagen aan hun}{leven voor het eerst}\\

\haiku{Was de winter toch,.}{maar om de zomertijd zou}{werk genoeg brengen}\\

\haiku{{\textquoteright} Maar kort daarna wist, {\textquoteleft}'{\textquoteright},.}{ze datt al zo was toen}{ze de kerkgang deed}\\

\haiku{Ze was toen al weer {\textquoteleft}'{\textquoteright},.}{n heel eind maar ze spande}{zich in wat ze kon}\\

\haiku{Gedurende de.}{zomer hadden ze het al}{hard genoeg gehad}\\

\haiku{Dan zou het ergste,.}{gelejen zijn beurden ze}{zich zelf hopend op}\\

\haiku{{\textquoteleft}Ja, Peeters, mijn goeie,,.}{man ziet ge dat zult ge nog}{wel niet begrijpen}\\

\haiku{Zij trok zich dat aan, '.}{en driftte hem tegen dat}{zet ook niet op\'at}\\

\haiku{Maar de twee helpers,,!}{die Toon gevraagd had moesten toch}{ook betaald worden}\\

\haiku{Hij zat in een klein,.}{huisje met zijn vrouw die men}{zelden te zien kreeg}\\

\haiku{Er werd gezopen,.}{zoveel als ze maar door de}{keel konden krijgen}\\

\haiku{Ze hielden hem vast!}{en twee trokken hem de broek}{van de benen af}\\

\haiku{Maar later werd het.}{verbranden op de meeste}{plaatsen verboden}\\

\haiku{Die uitdrukking had.}{de spreker onthouden van}{de burgemeester}\\

\haiku{Bovendien bracht het.}{hem enkele honderden}{guldens per jaar op}\\

\haiku{Daarna sneed zijn stem,:}{knerpend als het scheuren van}{een ijzeren plaat}\\

\haiku{Zag hij dan ook, of?}{de arbeiders werkten zo}{hard als ze konden}\\

\haiku{hij liet het nog eer,.}{in brand steken dan het voor}{zo'n geld te geven}\\

\haiku{{\textquoteright} {\textquoteleft}Ja, het w\'as wel wat,.}{veel maar dat wilde hij er}{toch voor betalen}\\

\haiku{{\textquoteleft}wij hebben maar een,.}{prul van behuizing maar er}{zitten guldens in}\\

\haiku{12 De armoede.}{joeg hen op van de ene plaats}{naar de andere}\\

\haiku{Haar broers verlolden.}{op \'e\'en avond meer dan hij in}{een week verdiende}\\

\haiku{Plotseling rukte,.}{het dier aan schichtig door het}{slaan en het geraas}\\

\haiku{Gewoonlijk werden.}{in plaats hiervan andere}{regels gezongen}\\

\haiku{Wat had hij toen veel!}{gezien zonder er iets van}{geleerd te hebben}\\

\haiku{Sinds hij echter de,.}{strijd aanvaard had kon hij die}{niet meer ontvluchten}\\

\haiku{{\textquoteright} {\textquoteleft}Wat ik meen, dat is,.}{geld dat de winkeliers hier}{konden verdienen}\\

\haiku{Als daar een kanaal.}{naar toe werd gegraven of}{een tram aangelegd}\\

\haiku{De meesten hadden,,.}{klompen bruin van aangeplakt}{moer aan de voeten}\\

\haiku{Een moedeloze.}{loomheid somberde om die}{groep arbeiders heen}\\

\haiku{Een huivering van.}{aanstuwende somberte}{trok over zijn lichaam}\\

\haiku{Als ge er een jaar,.}{of wat in gestaan hebt dan}{voelt ge het niet meer}\\

\haiku{Bij harde winter.}{kan het wel eens gebeuren}{dat een ketting knapt}\\

\haiku{{\textquoteright} {\textquoteleft}Ja, ik heb er wel,.}{eens een gehad maar hij is}{kapot gevallen}\\

\haiku{{\textquoteright} {\textquoteleft}Ja, heel eenvoudig,{\textquoteright},.}{stemde De Visscher toe om}{toch iets te zeggen}\\

\haiku{Wel wil ik gaarne,.}{toegeven dat we niet te}{veel keuze hebben}\\

\haiku{Daarom vind ik het,.}{juist zo vreemd dat u niet van}{De Schoolmeester houdt}\\

\haiku{Juffrouw Verhoeven.}{nodigde hem uit toch even}{binnen te komen}\\

\haiku{{\textquoteright} {\textquoteleft}Zeker, maar ik heb '{\textquoteright} {\textquoteleft},,.}{t ook niet tegen u.O}{nee nee dat is waar}\\

\haiku{Het geld regeert de,.}{wereld het geld is de ziel}{van de negotie}\\

\haiku{Neen, door de dag kwam. ' '.}{er weinigs Zaterdags}{ens zondags meer}\\

\haiku{Het mes sneed er aan,,.}{twee kanten helderde hij}{nog op grinnekend}\\

\haiku{Hij had de naam van.}{in allerlei gemene}{straatjes te sjouwen}\\

\haiku{Hoe konden ze nu?}{toch eigenlijk in zo iets}{nog plezier hebben}\\

\haiku{{\textquoteright} Ja, bitterde het, '!}{in zijn somber denken op}{zoals ment ziet}\\

\haiku{Hard werken was 't,.}{er ook maar d\'at was toch niks}{tegen turfgraven}\\

\haiku{Daar kan je jezelf,.}{gaan liggen kietelen als}{je wat hebben wilt}\\

\haiku{Ze zetten hem op,.}{de tafel en dan sloeg ie}{een onzin uit bar}\\

\haiku{Maar als ze in de,.}{hel liggen dan zal het er}{wel anders spannen}\\

\haiku{Nu en dan sloop een,.}{paar weg de lijven tegen}{elkaar gedrongen}\\

\haiku{Een ander schopte,.}{de benen terug die hem}{in de weg kwamen}\\

\haiku{Aan het station.}{Lizaveen stapte niemand}{in dan De Visscher}\\

\haiku{ik had niet gedacht,.}{dat ik mij in die mens zou}{kunnen vergissen}\\

\haiku{En wat ze na haar,.}{dood kon nalaten dat was}{voor de familie}\\

\haiku{Opeens wendde zij,.}{zich naar hem toe of ie niet}{naar de hoogmis ging}\\

\haiku{Deed bestraffend, dat '.}{de kapelaant toch ook}{al goed zou menen}\\

\haiku{Waar haalden zij de?}{kunst vandaan om hun leven}{zo in te richten}\\

\haiku{Die waarderende.}{brief bracht De Visscher in een}{roes van vreugde}\\

\haiku{'t Schijnt, dat ik in.}{de wieg gelegd ben om de}{strijd uit te vechten}\\

\haiku{Voor het onderhoud.}{van het gezin schiet er dan}{zoveel niet meer over}\\

\haiku{Eigenlijk wel geen,,.}{familie zo gesproken}{maar toch meer als vreemd}\\

\haiku{De gemeente zou,.}{er wel bij varen als zij}{haar veen verkocht had}\\

\haiku{{\textquoteleft}Wie de mensen wil,.}{leren kan hen nooit te dom}{veronderstellen}\\

\haiku{Het bevatte een:}{lange correspondentie}{als hoofdartikel}\\

\haiku{Ik dacht, breng het prul,.}{aan meneer pastoor dat die}{het verbranden kan}\\

\haiku{Op een schrijven aan.}{Het Nieuws van Peelland kreeg De}{Visscher geen antwoord}\\

\haiku{Een hees gebrul als.}{van een gepijnigd dier ging}{uit de hoop omhoog}\\

\haiku{Als je de beest wilt,.}{uithangen zie dan maar dat}{je de deur uitkomt}\\

\haiku{Om het verband goed,.}{te kunnen leggen spelde}{hij het hemd omhoog}\\

\haiku{Na een paar dagen.}{verzonden zij de kopij}{naar de drukkerij}\\

\haiku{dat is, dat wij veel.}{te weinig verdienen voor}{zo'n harde arbeid}\\

\haiku{Met gezondhijd moet}{ik U nu schrijfen dat het}{zoo lang geduurt heeft}\\

\haiku{Nu vele Groeten.}{van mijn vrouw of Fientje wie}{ik ze noemen wil}\\

\haiku{Hen uit de verdienst,,.}{schoppen alsof zij honden}{waren dat kon hij}\\

\haiku{En met de bezem,.}{de straat laten vegen waar}{hij had gelopen}\\

\haiku{Maar met de verdienst.}{was het bij de maatschappij}{toch een boel beter}\\

\haiku{Men heeft er toch al.}{genoeg aan onze lieve}{Heer moeten geven}\\

\haiku{Zij zelf hadden voor.}{hun trouwen in Pruisen een}{ledikant gekocht}\\

\haiku{Nee, er zit zo'n schelm.}{in het hele land niet in}{de gevangenis}\\

\haiku{Maar de ander was,.}{toch wel bang dat Van Eijzen}{het niet halen zou}\\

\haiku{Daarna werd nog een:}{circulaire verspreid door}{de gemeente}\\

\haiku{Allee, die lui moesten,.}{maar drinken al kostte het}{hem duizend gulden}\\

\haiku{De mensen waren,,.}{blij als hij niet snauwde dat}{het niet nodig was}\\

\haiku{Een kapelaan uit.}{het dorp ging met het hoofd van}{de openbare school}\\

\haiku{Er kwamen handen.}{tekort om jenever en}{bier aan te voeren}\\

\haiku{Als het ingevuld,.}{was met de naam Van Eijzen}{dan deugde het niet}\\

\haiku{'s Morgens hing een.}{mand zonder bodem bij van}{Eijzen aan de deur}\\

\haiku{De gemeente moest.}{ook voor de godsdienstige}{belangen zorgen}\\

\haiku{En het hoofd van de.}{dorpsschool had verhoging van}{salaris gevraagd}\\

\haiku{Hij woelde door zijn,.}{bed al maar pratend over turf}{en aken en staken}\\

\haiku{Fien en Dien snikten.}{hardop en vluchtten uit het}{kleine slaaphokje}\\

\haiku{ik woon nu ver van}{de post af zuster in de}{week heb ik geen tijd}\\

\haiku{Bij het onderzoek.}{van hogerhand was er zelfs}{nog te v\'e\'el in kas}\\

\subsection{Uit: Verstooteling}

\haiku{Ja, als haar moeder,;}{weer beter was zou zij een}{nieuwen dienst zoeken}\\

\haiku{Nee, in het dorp had,.}{ze niks in de stad was het}{veel pleizieriger}\\

\haiku{Ze moest maar naar huis,,.}{gaan als ze niet wilde doen}{wat zij aanraadden}\\

\haiku{Door de praatjes van. '}{het volk zouden die daartoe}{genoodzaakt worden}\\

\haiku{En het is God zelf,,.}{die het gezegd heeft dat er}{armen moeten zijn}\\

\haiku{En of hij haar geen.}{proces kon maken wegens}{grove beleediging}\\

\haiku{Chiek in huis, en 's...}{middags zich aankleeden en}{gaan wandelen}\\

\haiku{ie klimt zoo\"e mer in....}{de b\"oem en scheurt de t\"ak en}{lacht ow nog uut ok}\\

\haiku{In bijzijn van haar.}{man durfde Floortje hem niet}{meer voor te spreken}\\

\haiku{In Holland was 't,?}{soldaat-zijn niks wat}{had je daar nou aan}\\

\haiku{Eer de dag om was,.}{was het gepraat het heele}{dorp doorgetrokken}\\

\haiku{De zieke ging hard '.}{achteruit en opeens was}{t afgeloopen}\\

\haiku{'s Nachts kon ze  .}{geen oog toe doen van al de}{zorg en het verdriet}\\

\haiku{Als hij maar eens kwam,.}{en wat geld meebracht dan zou}{het wel weer goed zijn}\\

\haiku{wer, we l\`even mer...{\textquoteright},,,...}{eens  Sauf broeder sauf den}{daalder d\`e mos auf}\\

\haiku{{\textquoteright} Tot Jan zijn woede.}{uitheeschte en zich op}{hem wilde werpen}\\

\haiku{Jan moest maar zuipen,, '...}{zooveel als hij lusttet}{kwam er nikt op aan}\\

\haiku{En dikwijls al had,.}{het heele dorp verteld dat}{hij weer ging trouwen}\\

\haiku{Dat was de smart, die.}{in zulke oogenblikken}{in hem opleefde}\\

\haiku{Dat zou wat anders.}{zijn dan zich bij de boeren}{kapot te werken}\\

\haiku{De vrienden zouden.}{hen uitgeleide doen naar}{het station}\\

\haiku{Wat van zijn verdienst,?}{afgeven waarvoor hij zoo}{hard moest errebeie}\\

\haiku{Alleen een klaagbrief.}{van zijn moeder kwam Jan nu}{en dan hinderen}\\

\haiku{en gii most eier,,, ()}{st\`ele h\`e vur eur en dan}{opnieuw knipoogend}\\

\haiku{Wankelde nu naar,.}{dezen dan naar dien kant door}{het tartend stooten}\\

\haiku{Sakkerloot nee, d'r, '...}{maar niet aan denket was}{niet om uit te staan}\\

\haiku{Ze maakte het met,...}{Jan nooit af al moest ze met}{hem wegloopen}\\

\haiku{Zoolang als ze met,.}{hun beien waren hadden}{ze niet veel plaats noodig}\\

\haiku{'t Was niet mooi van,.}{Jan dat hij daarover zooveel}{te mopperen had}\\

\haiku{Gauw genoeg zou hij.}{een-te-veel zijn bij}{de Van der Poortens}\\

\haiku{Veranderen ging.}{niet meer of er zou groote}{herrie van komen}\\

\haiku{Op een plaats werd zelfs.}{op een avond de lamp van den}{zolder geslagen}\\

\haiku{Liepen toen nog eens,.}{over het terrein naar alle}{kante uitkijkend}\\

\haiku{Die er boven op,}{stond haalde het aan en de}{overigen trokken}\\

\haiku{Zij had de beste '.}{arbeiders dan ook maar voor}{t nemen gehad}\\

\haiku{De burgemeester.}{had op haar vragen niet veel}{willen antwoorden}\\

\haiku{Nou verdomde'n ie '.}{t toch ook om nog langer}{te gaan bedelen}\\

\haiku{Dat Mien nou zelf wel, {\textquoteleft}{\textquoteright};}{zag dat ze maar met denRooie}{had moeten trouwen}\\

\haiku{'t Was een streek van.}{dien kerel geweest om den}{boel te bedriegen}\\

\haiku{kon hij maar ergens,...}{wat leenen maar daar hoefde hij}{niet om te komen}\\

\haiku{{\textquoteright} Een schampering van.}{d'r-alles-van-weten}{lag in z'n woorden}\\

\haiku{barst maar... z\'o\'o leep was,,?...}{dat mirakel toch niet dat}{ie d\`at snapte h\`e}\\

\haiku{toen was die kerel...}{begonnen met een groote mond}{van verdammter lump}\\

\haiku{Maar de {\textquoteleft}Rooie{\textquoteright} spotte:}{met gewichtig-doening}{in stem en gezicht}\\

\haiku{Mien en de {\textquoteleft}Rooie{\textquoteright}... als {\textquoteleft}{\textquoteright}, '...}{ze met denRooie getrouwd was}{zout beter zijn}\\

\haiku{nou moesten ze maar eens...}{voor den dag komen met de}{lekkere br\"okskes}\\

\haiku{Vooruit, voor den dag,, '...}{er mee of hij zou eens zien}{watm te doen stond}\\

\haiku{hoe Marie was, dat... '...}{wist iedereen en Mien zou}{welt zelfde zijn}\\

\haiku{Gelach en gepraat,.}{rumoerde druk rond \`al meer}{bezoekers lokkend}\\

\haiku{Men wist niet meer, wat..., ',...}{men gelooven moest jat was}{dan toch te gek h\`e}\\

\haiku{een maand of drie... ja,, '}{d\`a\`ar zorgde hij wel voor maar}{met de verdienst ging}\\

\haiku{Wat een gezwets zou ',.}{t geven in het dorp over}{hem en zijne vrouw}\\

\haiku{Altijd en altijd,.}{datzelfde gezanik dat}{hij niets verdiende}\\

\haiku{Onbewust verviel.}{hij een oogenblik in een}{nadenkend zwijgen}\\

\haiku{En aansluitend bij:}{zijn eigen denken bitste zij}{hem schamperend toe}\\

\haiku{Daar zoo zat als een,...}{kwajongen die kijven kreeg}{over een dommen streek}\\

\haiku{{\textquoteleft}Wat denk'te wel,...}{ge het ok noo\"et wat te}{moppere geh\`ad}\\

\haiku{Jan moest maar dikwijls,...}{naar huis komen moeder was}{er toch nog altijd}\\

\haiku{'t Was dezelfde,.}{Jan niet meer als toen hij aan}{het klooster werkte}\\

\haiku{Een mensch moest nog al...}{wat ondervinden om aan}{zijn eind te raken}\\

\haiku{In een hoek {\^\i}agen.}{witte en zwarte klompen}{opeengestapeld}\\

\haiku{Langs den muur planken.}{met kruidenierswaren en}{katoenen stoffen}\\

\haiku{En al sloeg hij hem,.}{d'r neer dan kon niemand hem}{daar wat over maken}\\

\haiku{Van der Poorten zou,,.}{wel zorgen dat ie wat in}{huis had dreigde hij}\\

\haiku{Even ontmoetten de {\textquoteleft}{\textquoteright}.}{blikken van denRooie en die}{van Marie elkaar}\\

\haiku{{\textquoteleft}Der, mer da's ok het,,...}{leste dan tap ik nie mer}{drink nou mer gaauw uut}\\

\haiku{hij zijn arm los en.}{zonder verder op haar te}{letten kijft hij voort}\\

\haiku{Dan, verzoenend, stelt,.}{hij voor dat ze samen maar}{moeten afdrinken}\\

\haiku{Een geweldige.}{trap tegen de deur doet hem}{half-opspringen}\\

\haiku{Een breede wonde,...}{gaapt in zijn achterhoofd als}{een zwarte holte}\\

\haiku{Een huivering van,.}{angst doorkilt haar lijf telkens}{als zij wat hooren}\\

\haiku{De Zondagsche kleeren.}{rijen op de stoelen in}{alle vertrekken}\\

\section{Marcel Maassen}

\subsection{Uit: Blauwe damp}

\haiku{Want als Jos er niet,.}{was geweest dan was het nooit}{meer iets geworden}\\

\haiku{Ik bleef lekker thuis,.}{een rustig en tevreden}{hamstertje wezen}\\

\haiku{Ik zat achter de.}{computer en vroeg me af}{wat ik moest schrijven}\\

\haiku{Bekijk 't maar, ga,...}{maar lekker schrijven ga maar}{gauw beroemd worden}\\

\haiku{Dat lachje van haar,,:}{die valsverliefde ogen dat}{handje door mijn haar}\\

\haiku{Ja, schatje gaf wel,,.}{kusje ik wel maar schatje}{wist heus wel beter}\\

\haiku{was vroeg je, of ik,.}{lekker met hem geneukt heb}{dat bedoel je toch}\\

\haiku{Nee, nooit heb ik haar.}{kunnen dwingen te buigen}{en te bekennen}\\

\haiku{Tenminste, niet meer.}{vanaf het moment dat ik}{erop ging letten}\\

\haiku{Vriendinnen bleven, '.}{ze al zagen ze elkaar}{opt laatst niet meer}\\

\haiku{{\textquoteright} Moeder kreeg eerst twee.}{andere kinderen en}{daarna kreeg ze mij}\\

\haiku{Met die vier jaren,.}{die daartussen liggen weet}{ik me niet goed raad}\\

\haiku{En vooral: je moest.}{beter zijn dan iedereen}{die je omringde}\\

\haiku{Natuurlijk haatte.}{ik leren maar ik was er}{redelijk goed in}\\

\haiku{Allemaal dingen.}{die je allang wist en die}{nooit veranderden}\\

\haiku{Eerst alleen naar de,.}{thuiswedstrijden later ook}{naar uitwedstrijden}\\

\haiku{{\textquoteleft}Nog even{\textquoteright}, dacht hij, {\textquoteleft}nog,.}{heel even meelachen ze}{houden vanzelf op}\\

\haiku{Twee pilsjes in de.}{handen en nog twee op de}{bar en dan zuipen}\\

\haiku{Trouwens, jijzelf ook,.}{niet je zou niet weten wat}{je moest beginnen}\\

\haiku{Ze bleef gewoon waar,,.}{ze was fijn bij Walter en}{ik kon doodvallen}\\

\haiku{Bijna zeventig,.}{pagina's die niet je niet}{zomaar aan elkaar}\\

\haiku{Dat vond vader een, ({\textquoteleft}}{prima plan precies wat hij}{zelf al had bedacht}\\

\haiku{Zelf had ik ook niets,.}{tegen damesbezoek maar}{het kwam er niet van}\\

\haiku{En als ik er mijn,.}{toekomst voor moet opgeven}{dan moet dat ook maar}\\

\haiku{Aan schrijven had ik.}{niet gedacht toen ik zei dat}{ik zou gaan schrijven}\\

\haiku{Maar ze hoort me weer.}{eens niet want de tv begint}{alsnog te spelen}\\

\haiku{Zoef, we kunnen niet.}{slapen als jij de hele}{tijd staat te lullen}\\

\haiku{Hij knijpt z'n ogen toe,.}{en z'n mond elke spier in}{zijn lijf trekt samen}\\

\haiku{Nog elke dag zijn,.}{zij samen een keer of drie}{en soms wel vaker}\\

\haiku{{\textquoteright} Dan begint Jos te.}{bulderen zoals alleen}{Jos bulderen kan}\\

\haiku{Twee woorden maar, want {\textquoteleft}{\textquoteright}.}{datmevrouw had voor moeder}{niet eens gehoeven}\\

\haiku{Natascha was dik.}{en lelijk en altijd veel}{te zwaar opgemaakt}\\

\haiku{We waren veertien.}{en er was weer eens een kamp}{van Jong Nederland}\\

\haiku{Vervolgens gaf hij.}{mij een knipoog en ik gaf}{er eentje terug}\\

\haiku{Voor haar huis zou ik,}{zeggen dat ik haar leuk vond}{zou ik haar kussen}\\

\haiku{In z'n blote kont.}{zat hij langs de kant van het}{water en keek toe}\\

\haiku{{\textquoteleft}De ogen van Jeffrey.}{zijn niet meer dan twee zwarte}{gaten in zijn hoofd}\\

\haiku{Toen stonden we naakt,.}{tegenover elkaar nog een}{moment onwennig}\\

\haiku{{\textquoteleft}Stel dat je het zou,?}{mogen overdoen hoe zou je}{het dan aanpakken}\\

\haiku{Op dit moment ben.}{ik niet verliefd en ik wil}{me ook niet binden}\\

\haiku{Ze draagt een groene.}{onderbroek en een witte}{bh en verder niets}\\

\haiku{Zij werd kwaad en ik.}{werd kwader en we sliepen}{in zonder nachtzoen}\\

\haiku{We hebben al vier.}{maanden verkering en je}{bent nog niet zwanger}\\

\haiku{{\textquoteright} Ieder had gewoon.}{z'n eigen schoenendoos vol}{pijnen en pijntjes}\\

\haiku{{\textquoteright} {\textquoteleft}Wat is dat nou voor{\textquoteright},, {\textquoteleft}?}{onzin zegt moederben jij}{een volwassen vent}\\

\haiku{Hij glimlachte, hij,.}{grijnsde keek naar de plaat die}{voor zijn voeten lag}\\

\haiku{En het was niet eens}{zozeer om het bedrog dat}{ik hem inniger}\\

\haiku{Op Tweede Kerstdag, '.}{bijvoorbeelds avonds met z'n}{allen in de kroeg}\\

\haiku{Geleen en Sittard,.}{is water en vuur vooral}{met carnaval}\\

\haiku{We gingen er wel,,.}{naartoe gingen wel kijken}{maar we keken niet}\\

\haiku{Ik word razend en:}{zij lacht alsmaar liever en}{spreekt alsmaar zoeter}\\

\haiku{Walter en zij en,.}{nog iemand met z'n drie\"en}{in \'e\'en botsauto}\\

\haiku{Je tas stond open en.}{dus dacht ik dat ik er wel}{in mocht snuffelen}\\

\haiku{Zij gaat weg en ik,.}{blijf hier zij is met Walter}{en ik ben alleen}\\

\haiku{Ze slaat een hand voor,.}{haar mond buigt zich over mij heen}{en tuit haar lippen}\\

\haiku{Zij over kunst, Bertram.}{over filosofie en ik}{over literatuur}\\

\haiku{Het beddegoed is.}{afgetrokken en schoon goed}{ligt klaar op de stoel}\\

\haiku{Toen waren we weer,,.}{alleen gezellig met z'n}{tweetjes muis en ik}\\

\haiku{Zo weet ik dus niet,.}{of ze wakker is of slaapt}{of ze denkt of droomt}\\

\haiku{Ik kruip weer tegen,:}{haar aan sla mijn arm om haar}{middel en fluister}\\

\section{Herman de Man}

\subsection{Uit: Aardebanden}

\haiku{Aan Walter Thiry mijn.}{tweejarigen vriend ~ I.}{Moeder's uitvaart}\\

\haiku{ik wensch, dat het voor{\textquoteright}.}{lange jaren uw laatste}{leed geweest zal zijn}\\

\haiku{Zijn kinderachtig.}{benauwde laatste woorden}{bleef ze onthouden}\\

\haiku{Willem stond achter,.}{de tapkast zij kon dus goed}{even gemist worden}\\

\haiku{Zijn sterkte, zijn niet.}{te breidelen mannekracht}{deed haar duizelen}\\

\haiku{Met twee, drie stappen.}{als van een grooten beer was}{hij de kamer door}\\

\haiku{ze wist dat het t\`och.}{komen moest en dat zij het}{te aanvaarden had}\\

\haiku{die kleurde altijd.}{subiet als een meneer uit}{de stad hem aansprak}\\

\haiku{{\textquoteleft}wilt U mij volgen,.}{de meid zal uw koffers straks}{wel boven brengen}\\

\haiku{{\textquoteleft}En Pa, is het nu,?}{heusch waar dat hij op het}{water loopen kon}\\

\haiku{Aan weerskanten van,;}{de kaai waren slooten de}{een daarvan was breed}\\

\haiku{Dat deed de stijve.}{Juffrouw Th\'erees goed tot in}{haar diepste wezen}\\

\haiku{Dat deed haar angstig - '.}{omzien en vreezen vreezen}{t allerergste}\\

\haiku{Die man was Liesje's {\textquoteleft}{\textquoteright}.}{Vader en zij was Th\'erees}{Versteeg uitDe Zalm}\\

\haiku{daarmee depte ze.}{haar heete voorhoofd en dat}{gaf koelte en rust}\\

\haiku{als ik goed zie met,.}{dien vrijer die er met zijn}{vrouw op logies is}\\

\haiku{Toen nam ze moedig.}{twee wijnglazen en keerde}{ze om bij de voetjes}\\

\haiku{{\textquoteleft}hij zal pijn in zijn.}{buik hebben of anders in}{zijn potteman\'e}\\

\haiku{{\textquoteright} De vreemdeling sloeg.}{zijn oogen op en keek lang en}{vorschend naar Jacques}\\

\haiku{Neen jongen, ga jij {\textquoteleft}{\textquoteright}.}{maar weervan alles doen en}{laat mij met vrede}\\

\haiku{Juffrouw Th\'erees zat;}{stijf als een ijspegel op}{het keukenstoeltje}\\

\haiku{Over zijn gezicht kwam,.}{een vredige trek tijdens}{hij den brief doorlas}\\

\haiku{- {\textquoteleft}En nu, burgers, met,!}{z'n allen driemaal hoera}{voor de Koningin}\\

\haiku{Toen stonden ze op,.}{en liepen terug in de}{richting van de Lek}\\

\haiku{Ze liet hem maar weer,.}{doen en meteen was hij van}{dat gesprek weer af}\\

\haiku{Toen ze heur groene,,:}{stijve bloeze dichtgeknoopt}{had riep hij luchtig}\\

\subsection{Uit: Omnibus}

\haiku{Jochem plant dan zijn:}{vleeshompen vaster op de}{aard en zal zeggen}\\

\haiku{Hij kocht, en 't wijf,.}{mag dat niet weten \'o\'ok een}{loterijbriefje}\\

\haiku{{\textquoteright} {\textquoteleft}Kerel, je maakt me, {\textquotedblleft},.}{nog eens gek jij met jeja}{meneer nee meneer}\\

\haiku{Met vreemde ogen keek,, '.}{die even maar int wezen}{van deze schooier}\\

\haiku{Toen ze het franshuis, '.}{naderden zagen zem}{al in de tuin staan}\\

\haiku{De organist staart.}{verbaasd over z'n boek naar z'n}{twee onte gasten}\\

\haiku{Luuk, dees twee mannen,,.}{zeggen dat het kul is die}{bedelaarstekens}\\

\haiku{Maar wel hebben ze '.}{t geluid al half om de}{aarde heen gestuwd}\\

\haiku{Hij met z'n lompe,.}{korsterige handen en}{z'n onwetendheid}\\

\haiku{Hij is misschien iets,.}{als een stuk oom van een neef}{van een neef z'n broer}\\

\haiku{Wist hij maar een woord,.}{om d'r gedachte op wat}{anders te richten}\\

\haiku{{\textquoteleft}Je kan een boer nooit,.}{te veul centjes afhandig}{maken zeg ik maar}\\

\haiku{Overal was geld, veel,,....}{geld veel te veel geld overal}{behalve bij hem}\\

\haiku{Z\'o maar te grijp, als.}{eerst maar de drager zich niet}{meer verweren kon}\\

\haiku{Ze wisten goed, dat.}{het nu een kwaad uur werd voor}{de bedelaren}\\

\haiku{Maar mannen, ik zeg,.}{maar zo kom d'r in en la\^an}{we er een vatten}\\

\haiku{Toen grepen we beur.}{en metselden de ouwe}{zog helegaar in}\\

\haiku{Hij stond op en sloeg.}{met z'n pezige vuist op}{de keukentafel}\\

\haiku{arme lat, blijf jij,.}{maar hier wat wonen op de}{werf tot het je past}\\

\haiku{Als je maar geld hebt, '.}{dan mag jen keer de brand}{steken in je keel}\\

\haiku{{\textquoteright} {\textquoteleft}Maar stellig, en wel.}{volgens art. 432 ten eerste}{Wetboek van Strafrecht}\\

\haiku{En Jochem loopt met,.}{smoesjes te leuren die t\'och}{niet zallen helpen}\\

\haiku{Wat dee die ander?}{ook krek op dat moment langs}{te kommen lopen}\\

\haiku{niet zeggen... 't is ', '.}{de wet dust is de wet}{wantt is de wet}\\

\haiku{ja, dan, dan... wat mot,?}{uedele dan doen as u}{dat met ons eens zift}\\

\haiku{Ze hebben over de,.}{witkop niet meer gepraat die}{dag al viel dat zwaar}\\

\haiku{weerom zal zijn van.}{z'n vechtpartij tegen de}{lucht en de wolken}\\

\haiku{Je bekomt hier, deur,.}{ons toedoen wat vrijplezier}{en een centje toe}\\

\haiku{{\textquoteright} En de jongens, wild,.}{op een kluit gooien elk een}{cent in de lijmpot}\\

\haiku{Negen gezonde,,.}{blije kinderen woesten en}{stillen alderhand}\\

\haiku{Z'n twee getrouwe,.}{bedelgasten die nooit of}{nooit zullen overslaan}\\

\haiku{En waar, in Lopik,?}{bestaan nog overig zulke}{mooie lichtende ogen}\\

\haiku{Dat zou echtig ook.}{wel de dochter kunnen zijn}{van mijnheer Koekkoek}\\

\haiku{Bastiaar Six zat,,.}{als een woestgemaakt beest naast}{Aartje en Aartje beefde}\\

\haiku{{\textquoteright} riep de voerman woest, {\textquoteleft} ',!}{hum smijten wet lijk veur}{z'n voeten verstaan}\\

\haiku{{\textquoteright} En 't was toen maar,.}{goed dat mijnheer Koekkoek den}{huis uit kwam lopen}\\

\haiku{ze gaven om 't,.}{rouwbeklag allemaal wat}{centen uit de knip}\\

\haiku{Allenig Jochem,.}{hield van z'n portie tien rooie}{centen over niet meer}\\

\haiku{Want het geld van m'n '.}{maat is me even lief alst}{geld van een jonker}\\

\haiku{{\textquoteright} En toen 't duppie, '.}{weer in mijn vingers zat wier}{t me daar zo heet}\\

\haiku{Want we zaten daar.}{alleen maar wat lustig langs}{de dijk te zingen}\\

\haiku{Zo zie je, zaken,,.}{doen mag niet allenig rijk}{zijnde d\'an mag het}\\

\haiku{As ik nog 'ns zou,, '.}{motten beginnen met jou}{dan deek het niet}\\

\haiku{Hij heeft geleerd, geld,.}{te porren uit arm en rijk}{uit leed en pleizier}\\

\haiku{Hij had 'em de kop,{\textquoteright}.}{in motten slaan en jou d'r}{bij barst Jochem uit}\\

\haiku{zij lachten mee om.}{de kwaaie aard in die ouwe}{meid te bezweren}\\

\haiku{Heel de aarde met,.}{alle mensen er op draait}{om de rijksdaalders}\\

\haiku{Daarom bleef hij van;}{toen af maar rond lopen in}{een kleine cirkel}\\

\haiku{wat liefde en wat,,.}{haat geboren worden wat}{groeien en sterven}\\

\haiku{as je gaat rechten,.}{om een plank van drie gulden}{dat kost drie duzend}\\

\haiku{En... wij gaan alvast,.}{de ouwe plank over als ie}{nog lang genog is}\\

\haiku{Best. Maar la\^an ze dan,.}{naar d'r eigen streek gaan die}{smerige Belzen}\\

\haiku{En nou heb ik niks......}{niks Geen koei en geen werkloon}{om wat te prutsen}\\

\haiku{En dat weer omdat,.}{jullie geen geld hebben om}{er slecht mee te doen}\\

\haiku{je moet je in die.}{zaken nooit door je gevoel}{laten meeslepen}\\

\haiku{{\textquoteright} {\textquoteleft}Juist Chef, kinderen.}{vragen hartelijkheid en}{huisgezinsgeluk}\\

\haiku{{\textquoteright} {\textquoteleft}Schaam jij je eigen,,.}{maar wat om dat te zeggen}{waar die twee bij zijn}\\

\haiku{Ik geef als 't mij,,?!}{belieft en dat gaat jou geen}{bliksem aan verstaan}\\

\haiku{Ik docht al, wat een....}{hoop zoveel Pausen bennen}{d'r niet eens gewist}\\

\haiku{{\textquoteright} {\textquoteleft}Jaat, en 't mankeert,.}{er nog maar aan das ze je}{antwoord gaan geven}\\

\haiku{{\textquoteright} zei Jochem toen hij ',.}{t stuivertje eerst goed vast}{had ter verklaring}\\

\haiku{Hij is met z'n knecht,.}{de zevende boom van twee}{en halfpond al kwijt}\\

\haiku{{\textquoteleft}En Bart, as je an, '.}{de kaart komt slaan ik jen}{paar ribben kapot}\\

\haiku{Vooral, omdat ze.}{veel nieuws wisten van tussen}{Schoonhoven en hier}\\

\haiku{{\textquoteright} - Maar Brandewijn met,.}{Suiker gaat er nou veur zes}{jaar achter minstens}\\

\haiku{Een stuk pleizier is, '.}{van z'n avond aft geluk}{van leed uitdragen}\\

\haiku{Als Piet Miltenburg,.}{honderd jaren wordt zal ie}{er n\'og van schenden}\\

\haiku{{\textquoteright} {\textquoteleft}Verdomd, nee, bliksem,;}{Goof daar vat jij eigentlijk}{de pan bij de steel}\\

\haiku{wat is er ievers?}{in die zekere orde}{voor h\'em weggelegd}\\

\haiku{Elk ander mens  .}{zou neergestort zijn bij zo'n}{liefelijk klapje}\\

\haiku{Wij zijn nou eenmaal,,.}{aangewezen om tot het}{end niks te hebben}\\

\haiku{wij vragen niet wijex,.}{dan een homp leverworst te}{maggen bekommen}\\

\haiku{z\'o beroerd kan u.e, ',.}{t nooit hebben of wij zijn}{er naakter aan toe}\\

\haiku{Wij hebben niks, maar....}{heel de wereld is van ons}{om op te lopen}\\

\haiku{{\textquoteright} vond de Notaris,.}{gram en hij moedigde Chef}{aan met z'n stokje}\\

\haiku{al stem ik toe dat,.}{het een gedachte is waard}{om te overwegen}\\

\haiku{Als 't niet naar recht, '.}{is dan motten wet niet}{goed willen praten}\\

\haiku{E\'en ding is zeker,{\textquoteright}, '.}{zei Jochem toen ze buiten}{t gehoor waren}\\

\haiku{Maar als hij nader...}{stuift en veel plaats is er niet}{in hun scheepskeuken}\\

\haiku{dat jonk is nou z\'o.}{uit het nest gekropen en}{z\'o voor vreugd bestemd}\\

\haiku{Hij valt daar buiten,.}{ook al houdt hij stijf z'n pet}{op zijn glibberkop}\\

\haiku{Twintig uren in de,,...}{sloep en dan roeien zonder}{richting te weten}\\

\haiku{{\textquoteleft}Volgende keer krijg,.}{je kerels van me mee om}{aarpels te jassen}\\

\haiku{En ineens hoorde.}{hij de stem terug van Bartje}{Rijkelijkhuizen}\\

\haiku{Naar Rotterdam is.}{hij helemaal eens in zijn}{leven heen geweest}\\

\haiku{Als je te bedde,.}{ligt dan is de wereld stil}{en ook de mensen}\\

\haiku{Toen kwam er wat wilds.}{in zijn kop en hij greep zijn}{knikkerbuiltje vast}\\

\haiku{{\textquoteright} Maar Aai wou bij Brok.}{heel niet zeggen waarvoor hij}{naar die kooplui trok}\\

\haiku{De juffrouw vatte.}{daarop heur man bij de arm}{en fluisterde wat}\\

\haiku{Ze hebben 't hem.}{niet nagewezen en er}{niet op gezinspeeld}\\

\haiku{{\textquoteleft}het vlees is beter;}{dan de benen en niemand}{is te vertrouwen}\\

\haiku{Waarom had nou die?}{Gert Koeimans zo'n hekel aan}{de drie serpenten}\\

\haiku{Maar van toen af moest.}{Gert Koeimoes dag aan dag die}{loop er bij maken}\\

\haiku{Bijgelovig ben,,.}{ik niet maar dat ze heksten}{dat geloof ik w\'el}\\

\haiku{Maar 't ergste was,. '}{wel dat ze als honden op}{de landpacht waren}\\

\haiku{{\textquoteright} Daarop riepen ze,.}{in de nacht dat ze van de}{politie waren}\\

\haiku{Willen ze soms niet,.}{lopen dan kittelen ze}{met de zwiep hun oren}\\

\haiku{la\^an ze dan maar blij,;}{zijn dat ze met de kast mee}{omgevallen zijn}\\

\haiku{Dewijl gij zegt, dat.}{ik door Be\"elzebub de}{duivelen uitwierp}\\

\haiku{{\textquoteleft}En d\'an, 't is toch,.}{maar toeval dat we hier bij}{mekare zitten}\\

\haiku{{\textquoteright} {\textquoteleft}Als je z\'o begint,{\textquoteright}, {\textquoteleft} '.}{vond Jasdan hou jet maar}{netjes onder je}\\

\haiku{Zou er toen ergens,?}{wat bij hem gesprongen zijn}{een bloedaar of zo}\\

\haiku{{\textquoteright} Maar d\'at had Jas niet,.}{moeten zeggen want to\'en was}{het hek van de dam}\\

\haiku{uedele verkoopt,....}{geen lullificatie maar}{lullificatie}\\

\haiku{Niemand heeft uit zijn.}{eigen levensloop en van}{zijn vak wat verteld}\\

\haiku{Want als ze eenmaal,,.}{wat zei juffrouw Naatje dan}{duisterde ze niet}\\

\haiku{{\textquoteleft}Ach,{\textquoteright} zei de meester, {\textquoteleft}?}{heb ik daar werkelijk nog}{Jenny Lindjes}\\

\haiku{En nou zeg ik maar,,...}{met de hand op de Grondwet}{dat als het waar is}\\

\haiku{{\textquoteright} Maar Kees liet weten,.}{dat hij nog zeker een half}{uur te stoken had}\\

\haiku{{\textquoteright} {\textquoteleft}Dat je je geld kreeg,.}{een hand en een brief van de}{geheelonthouding}\\

\haiku{{\textquoteleft}en hier, meneer de,...{\textquoteright}.}{commissaris hebben we}{en toen zag hij mijn}\\

\haiku{{\textquoteright} Daarop kreeg ik van.}{die meneer een gulden en}{een fijne sigaar}\\

\haiku{- En eer Chef er op,.}{bedacht was had zijn maat dat}{alw\'e\'er verkondigd}\\

\haiku{ik zat zelf aan de,;}{cassa onze voorstelling}{moest nog beginnen}\\

\haiku{Sommige wouwen,.}{het kooitje nog niet uit maar}{daar wist ik raad op}\\

\haiku{Zo'n anker van een;}{Maasstroom-bootje is in}{een oortje gelicht}\\

\haiku{En madame was,.}{w\'e\'er in gebreke ze heeft}{het w\'e\'er niet voorzien}\\

\haiku{{\textquoteleft}hij kent dat liedje,.}{niet goed want ineens raakt hij}{uit de melodie}\\

\haiku{had geld tekort in.}{z'n kas  en die wou zich}{voor zijn raap schieten}\\

\haiku{Maar 't was aardig,.}{om te zien hoe hef ze met}{elkaar omgingen}\\

\haiku{{\textquoteright} {\textquoteleft}En Jacq de Zeehond,?}{wat is er naderhand nog}{van hem geworden}\\

\haiku{En nog, als ik in,.}{Amsterdam kom ga ik wel}{eens naar Francientje}\\

\haiku{Tussen de benen,.}{der mensen door hobbelde}{hij voorzichtig voort}\\

\haiku{die vent is aan 't,;}{malen maar geld is geld en}{handel is handel}\\

\haiku{En dan... je kan nooit;}{weten wat zo'n snijer op}{z'n hoofd heeft lopen}\\

\haiku{{\textquoteleft}Ja,{\textquoteright} ging hij voort, {\textquoteleft}want.}{dezer dagen heb je er}{niet \'e\'en gesleten}\\

\subsection{Uit: Een stoombootje in den mist}

\haiku{'t is jammer voor,.}{Dorus maar hij heeft  een zeer}{hoofd onder z'n pet}\\

\haiku{Jas heeft het niet op,.}{die walle-bazen ze}{heulen met den wal}\\

\haiku{IJs is er dat jaar,.}{niet veel geweest maar water}{en sneeuw niet zuinig}\\

\haiku{Een mensch kan maar aan,.}{den gang blijven helpen doet}{het geen donderament}\\

\haiku{Half drie is het ruim,.}{als hij de schotten van de}{loopplank laat sjorren}\\

\haiku{Hun kleer was er van,,,.}{overtogen hun haren het}{anker de trossen}\\

\haiku{{\textquoteright} {\textquoteleft}Kan me geen donder,!}{schelen ik jaag de schuit niet}{op het zand verstaan}\\

\haiku{Jullie weten nou, '.}{hoet een schipper vergaan}{kan in den bliksem}\\

\haiku{t Was mijn jong niet, ',.}{t was Toontje niet dat zee}{ik toch al d\^alijk}\\

\haiku{'t Laken gong er.}{af en we zagen niks dan}{rauw vleesch en bloed}\\

\haiku{dan slijt de nacht \'o\'ok.}{en valt er lichtelijk nog}{wat te verdienen}\\

\haiku{En Kees, de stoker,.}{was ook al van de schuit en}{ook de sloep was weg}\\

\haiku{En als je niet meer,.}{bekomen kan d\`an eerst voel}{je de ontbering}\\

\haiku{Als vandaag een boer,.}{sterft morgen vat een ander}{z'n spaai bij den steel}\\

\haiku{En z'n ouwe vrouw;}{gong hemelen en toen heit}{ie dat veurtgezet}\\

\haiku{Maar daar sturen we,,}{den bond op af m\'a\'ar dat is}{voor later zorg m\'a\'ar}\\

\haiku{Nou moet je weten, ';}{die schijf opt lijf van den}{tamboer is knap groot}\\

\haiku{Hij regende nat ';}{en toen maakten we vant}{zeil wat meer luifel}\\

\haiku{En 'k docht al, nou,.}{gaat het uit zijn maar hij nam}{n\`og voor een kwartje}\\

\haiku{We hadden op 't....}{lest handen te kort \'e\'en klant}{en een tent vol werk}\\

\haiku{Hij raakte er een,...}{keer acht van de tien hij kocht}{w\'e\'er voor een kwartje}\\

\haiku{En direct docht ik,.}{op dien daggelder ik weet}{zuiver niet waarom}\\

\haiku{Vreet heel 't schip maar!}{leeg en zuip al de melk op}{en al het water}\\

\haiku{Zulk een bericht gaat,.}{als op den wind van den een}{op den ander over}\\

\haiku{Maar als hij nader...}{stuift en veel plaats is er niet}{in hun scheepskeuken}\\

\haiku{dat jonk is nou z\'o\'o.}{uit het nest gekropen en}{z\`o\`o voor vreugd bestemd}\\

\haiku{Hij valt daar buiten,.}{ook al houdt hij stijf z'n pet}{op zijn glibberkop}\\

\haiku{Twintig uren in de,,...}{sloep en dan roeien zonder}{richting te weten}\\

\haiku{Toe nou Dorusje lief,...{\textquoteright} {\textquoteleft},{\textquoteright}, {\textquoteleft}.}{kereltje toe nouNee riep}{de rostejij niet}\\

\haiku{Volgenden keer krijg,.}{je kerels van me mee om}{aarpels te jassen}\\

\haiku{En ineens hoorde.}{hij de stem terug van Bartje}{Rijkelijkhuizen}\\

\haiku{jullie waagde en?}{dat de reederij nog geld}{toegeven moest \'o\'ok}\\

\haiku{Naar Rotterdam is.}{hij heelemaal eens in zijn}{leven heen geweest}\\

\haiku{Als je te bedde,.}{ligt dan is de wereld stil}{en ook de menschen}\\

\haiku{Toen kwam er wat wilds.}{in zijn kop en hij greep zijn}{knikkerbuilje vast}\\

\haiku{- Maar Aai wou bij Brok.}{heel niet zeggen waarvoor hij}{naar die kooplui trok}\\

\haiku{De juffrouw vatte.}{daarop heur man bij den arm}{en fluisterde wat}\\

\haiku{{\textquoteright} {\textquoteleft}'t Is,{\textquoteright} zei toen het, {\textquoteleft}',}{hoofd van Jutt is dat jij}{eigens nog te jong}\\

\haiku{Ze hebben 't hem.}{niet nagewezen en er}{niet op gezinspeeld}\\

\haiku{Waarom had nou die?}{Gert Koeimoes zoo'n hekel aan}{de drie serpenten}\\

\haiku{Maar van toen af moest.}{Gert Koeimoes dag aan dag dien}{loop er bij maken}\\

\haiku{ze hadden een hoop.}{van die uitheemsche katten}{met lange haren}\\

\haiku{Bijgeloovig ben,,.}{ik niet maar dat ze heksten}{dat geloof ik w\`el}\\

\haiku{Maar 't ergste was,. '}{wel dat ze als honden op}{de landpacht waren}\\

\haiku{) op een kippenhok.}{gevallen en half kreupel}{van den val gevlucht}\\

\haiku{De dienders dan, want,:}{daar was ik gebleven de}{dienders die zee\"en}\\

\haiku{Daarop riepen ze,.}{in den nacht dat ze van de}{politie waren}\\

\haiku{la\^an ze dan maar blij,;}{zijn dat ze met de kast mee}{omgevallen zijn}\\

\haiku{Dewijl gij zegt, dat.}{ik door Be\"elzebub de}{duivelen uitwerp}\\

\haiku{Zou er toen ergens,?}{wat bij hem gesprongen zijn}{een bloedaar of zoo}\\

\haiku{{\textquoteright} Maar d\`at had Jas niet,.}{moeten zeggen want t\`oen was}{het hek van den dam}\\

\haiku{Uedele verkoopt,....}{g\'e\'en lullificatie maar}{lullificatie}\\

\haiku{Niemand heeft uit zijn.}{eigen levensloop en van}{zijn vak wat verteld}\\

\haiku{Want als ze eenmaal,,.}{wat zei juffrouw Naatje dan}{fluisterde ze niet}\\

\haiku{En nou zeg ik maar,,...}{met de hand op de Grondwet}{dat als het waar is}\\

\haiku{{\textquoteright} Maar Kees liet weten,.}{dat hij nog zeker een half}{uur te stoken had}\\

\haiku{{\textquoteleft}Dat ig zoo'n ouwe ',.}{truc Gewoonwegt kistje}{omkeeren anders niks}\\

\haiku{{\textquoteright} {\textquoteleft}Dat je je geld kreeg,.}{een hand en een brief van de}{geheelonthouding}\\

\haiku{Daarop kreeg ik van.}{dien meneer een gulden en}{een fijne sigaar}\\

\haiku{- En eer Chef er op,.}{bedacht was had zijn maat dat}{alw\'e\'er verkondigd}\\

\haiku{ik zat zelf aan de,;}{cassa onze voorstelling}{moest nog beginnen}\\

\haiku{Sommigen wouwen,.}{het kooitje nog niet uit maar}{daar wist ik raad op}\\

\haiku{Zoo'n anker van een;}{Maasstroom-bootje is in}{een oortje gelicht}\\

\haiku{En madame was,.}{w\'e\'er in gebreke ze heeft}{het w\'e\'er niet voorzien}\\

\haiku{- hij kent dat liedje,.}{niet goed want ineens raakt hij}{uit de melodie}\\

\haiku{Maar 't was aardig,.}{om te zien hoe lief ze met}{elkaar omgingen}\\

\haiku{{\textquoteright} {\textquoteleft}En Jacq de Zeehond,?}{wat is er naderhand nog}{van hem geworden}\\

\haiku{En nog, als ik in,.}{Amsterdam kom ga ik wel}{eens naar Francientje}\\

\haiku{die vent is aan 't,;}{malen maar geld is geld en}{handel is handel}\\

\haiku{En dan... je kan nooit;}{weten wat zoo'n snijer op}{z'n hoofd heeft loopen}\\

\haiku{{\textquoteleft}Ja,{\textquoteright} ging hij voort, {\textquoteleft}want.}{dezer dagen heb je er}{niet \'e\'en gesleten}\\

\haiku{Een vakman uit de {\textquoteleft}{\textquoteright},.}{kleine wereld door Herman}{de Man beschreven}\\

\haiku{We laten 't hem:}{echter liever persoonlijk}{voor U vertellen}\\

\subsection{Uit: Het wassende water}

\haiku{Ze was gierig met,.}{haar liefde want ervaring}{had haar slim gemaakt}\\

\haiku{Daarom wier er ten.}{avond zoo dukkels over moeders}{gepraat in de schuit}\\

\haiku{Zoo wijd hij zien kon,.}{leek het of er toen geen rund}{meer lag in het land}\\

\haiku{{\textquoteright} {\textquoteleft}Zoo, nou dan kan 't,{\textquoteright}.}{n\`et over zijn zegde hij droog}{en liep den huis in}\\

\haiku{Hij aardde d\'a\'arin zijn.}{vader die ook nooit veel van}{woorden was geweest}\\

\haiku{Hij buigt het hoofd en,,}{heft het weer hij dwaalt en zwaait}{onvast en aleer}\\

\haiku{wou en keek verward '.}{doort koekoekraamt over de}{landen tot de Lek}\\

\haiku{Hij heeft zijn barre.}{woorden opgevreten en}{het hoofd gebogen}\\

\haiku{Moeders kende die}{drift en \'o\'ok de luwing en}{ze viel heur rauw jonk}\\

\haiku{Dirk Hoogerzeil hun,.}{gebuur toen kon ze b\`est diens}{plagen verdragen}\\

\haiku{Mogelijk had je.}{me d\`an toendertijd niet}{zoo kwalijk geraa{\"\i}en}\\

\haiku{En Willem... wou jij?}{rechtevoort op vader zijn}{stoel zien te kommen}\\

\haiku{Daarom is 't zoo,.}{jammer dat jij en moeders}{verschillen hebben}\\

\haiku{een koebeest dat wel.}{twee honderd gulden boven}{prijs wordt afgemijnd}\\

\haiku{Weggedrongen zat.}{zijn schamende gestalte}{tegen een schot aan}\\

\haiku{Doch ie, da 'k me?}{w\'e\'er op dat punt zou laten}{bepraten deur haar}\\

\haiku{het boerenbedrijf.}{dat is in mijn leven zoo}{noodig als het asemen}\\

\haiku{Hij keert manhaftig.}{terug tot het mooie werksche}{leven van een boer}\\

\haiku{'t Is goed rond en, '.}{glanzend best doort kalf heen}{en in vollen geef}\\

\haiku{Hij wordt Jaap van Jan, '.}{de Pauw genaamd naart huis}{de Pauw in Teckop}\\

\haiku{{\textquoteleft}En as Willem dan ' '...{\textquoteright} {\textquoteleft}?}{trouwen gaat en moeders doet}{emt land overH\`e}\\

\haiku{Wat een kw\^agast, om '.}{t veur zijn eigen bro\^er zoo}{lang op te vreten}\\

\haiku{Hij is fortuinlijk,,.}{in koop en verkoop op den}{huis en de markten}\\

\haiku{{\textquoteright} 't Was toen maar weer,.}{moeders die er later op}{den dag over begon}\\

\haiku{In weinig uren tijds,.}{was aan beider leven een}{groote draai gegeven}\\

\haiku{{\textquoteleft}ge het mijn vader.}{gedacht en ik zal hier mijn}{vader gedinken}\\

\haiku{Vliet en Dijkveld mocht;}{nooit of nooit onderloopen}{ofwel vervuilen}\\

\haiku{{\textquoteright} En ze begreep, dat,.}{hij \`al de schouwbrieven thans}{langs ging het groote keind}\\

\haiku{Stuur morgen 'an den '.}{dag maar een man omt ding}{bij mijn in te slaan}\\

\haiku{v\'o\'or heden had hij.}{nooit of nooit aan Geertrui de}{Goei aldus gedacht}\\

\haiku{Zal je later met,?}{me me\^e willen naar onze}{woning gunterwijd}\\

\haiku{{\textquoteright} En d\^alijk zocht hij,.}{van heur oogen af te lezen}{hoe ze dat wel vond}\\

\haiku{{\textquoteright} {\textquoteleft}En nou 'n elkeen}{hieromtrent er van afweet}{dat wij trouwen gaan}\\

\haiku{och toch gelukkig, '.}{alt lieve was nog niet}{verdroogd in zijn lijf}\\

\haiku{Maar haar verdoken?}{verlangen vroeg mogelijk}{\`andere vreugde}\\

\haiku{En ik bin d'r niet,.}{zeker van of ik jou veur}{den kop mag stooten}\\

\haiku{Ik zal gaan, ik zal,.}{opschuiven veur den ware}{veur de jeugdigheid}\\

\haiku{in beslagenheid -.}{gaat onze Gieljan Beijen}{zijn vaders weg op}\\

\haiku{En vraag mijn ook maar,.}{gien raad meer je doen t\`och je}{eigen believen}\\

\haiku{Maar toch kwam deze.}{nieuwe verdrietigheid niet}{op haar hard wezen}\\

\haiku{'t Allereerst zijn.}{benoeming tot Hoogheemraad}{van het dijkgebied}\\

\haiku{Zoo'n rustig en t\`och;}{bespraakt man zond een ieder}{graag naar het Dijkhuis}\\

\haiku{E\'en zak chili is,.}{duur een wagon chili maakt}{den prijs al lager}\\

\haiku{Hij wier er niet wild,.}{van maar neep aandachtig zijn}{vingers tot vuisten}\\

\haiku{{\textquoteleft}Ik bin den Dijkgraaf,!}{van den Lekkendijk meneer}{de Burgemeester}\\

\haiku{Het was hem daarna,:}{een blije veraseming toen}{gerapporteerd wier}\\

\haiku{Mogelijk ware.}{alzoo het water te keeren}{uit zijn laag gebied}\\

\haiku{We weten, dat thans,.}{een land volloopt dat daardeur}{veul schaai zal lijen}\\

\haiku{Maar met minder schaai,!}{is nievers het water te}{laten w\`el met meer}\\

\haiku{Na deze woorden.}{brak de ontstelling eerst in}{waren omvang uit}\\

\haiku{Een broer teugen een,.}{broer da's in het openbaar gien}{verheven gevecht}\\

\haiku{as hem zijn eigen,,?!}{nest wil bevuilen dan mot}{hem dat maar doen waar}\\

\haiku{Gij als ik, we zijn,!}{toch mannen van den arbeid}{mannen van het land}\\

\haiku{Maar w\`el onder het.}{boerenvolk was hittigheid}{om die benoeming}\\

\haiku{Hebt eerbied voor den,.}{boerenstand Den gulden stand}{van Nederland}\\

\haiku{12Met behulp van.}{bijvoer meer koeien houden}{dan het land toelaat}\\

\section{Lidy van Marissing}

\subsection{Uit: De omgekeerde wereld}

\haiku{Steeds minder slaat hij.}{als een razende met z'n}{handen op de grond}\\

\haiku{Hij tilt zijn prooi net.}{even van de grond en laat haar}{plotseling weer los}\\

\haiku{Z{\'\i}j een bruine rug -!}{en w{\'\i}j de zenuwen dat}{pikken we niet meer}\\

\haiku{Z{\'\i}j een bruine rug -!}{en w{\'\i}j de zenuwen dat}{pikken we niet meer}\\

\haiku{Wanneer ik hem zeg,.}{dat hij moet kijken kijkt hij}{niet naar behoren}\\

\haiku{Hij neemt de prijskaart.}{van een rode lakjas en}{hangt die op zijn borst}\\

\haiku{zij werd tegen de).}{Arabiese gebruiken in}{aan mannen getoond}\\

\haiku{- dat hij net zo goed,.}{iemand anders had kunnen}{zijn op dat moment}\\

\haiku{Vlak v\'o\'or hij tegen:}{de deur duwde hoorde hij}{een korte  klik}\\

\haiku{- Soms denk ik wel eens.}{aan dingen die te slecht zijn}{om over te praten}\\

\haiku{Deze techniek mag.}{alleen dan gebruikt worden}{als het terecht is}\\

\haiku{Ook deze techniek;}{mag alleen worden gebruikt}{als het terecht is}\\

\haiku{hij zal verstandig{\textquoteright}).}{zijn en volmaakt en alles}{doen wat ik zeg}\\

\haiku{- Ik heb een wereld,.}{van dagdromen waarover ik}{niemand iets vertel}\\

\haiku{{\textquoteright} ~ Dat wat niet te,.}{verdragen is blijft in ons}{opgesloten}\\

\haiku{Men heeft dan meer kans {\textquoteleft}{\textquoteright} {\textquoteleft}{\textquoteright},}{omregt te zien enregt te}{loopen zoodra}\\

\haiku{men hoorde zeggen {\textquoteleft}, {\textquoteleft},!}{Als het u gelieft Mevrouw!{\textquoteright}en}{Na u Eerwaarde}\\

\haiku{Voor het overige.}{beeldt men zich in alleen te}{kunnen leven}\\

\haiku{{\textquoteright} Een minuut later.}{hoorden we de auto de}{straat uit scheuren}\\

\haiku{zijn arbeid blijft hem,.}{vreemd zijn werksituatie}{blijft onbegrepen}\\

\haiku{{\textquoteleft}Ja, met een duur woord.}{mag je het nu inderdaad}{wel koncern noemen}\\

\haiku{Jullie hebben al.}{twee keer zoveel loon als een}{paar jaar geleden}\\

\haiku{Voor het schrijven van.}{deze tekst beschikken wij}{maar over weinig tijd}\\

\haiku{Zoals elke munt.}{of medalje heeft ook elk}{begrip twee kanten}\\

\haiku{Bijvoorbeeld door het.}{kloppen met een steen of stuk}{hout op de tafel}\\

\haiku{Iedereen is het,.}{met een plan eens maar eentje}{gaat er tegen in}\\

\haiku{Hij stapte uit met,.}{een vaag schuldgevoel dat hij}{zelf onderdrukte}\\

\haiku{WIJ PROTESTEREN?}{TEGEN        Hoofdstuk XX Als}{ratten in de val}\\

\haiku{Inmiddels waren.}{de eveneens schaars geklede}{vrouwen gekomen}\\

\haiku{de woorden af die.}{vaak slechts misverstanden en}{illusies scheppen}\\

\haiku{Wij bekleden een.}{vooraanstaande positie}{in de maatschappij}\\

\haiku{1969 - Ruim twintig man.}{zijn gebleven en hebben}{de fabriek bezet}\\

\haiku{Maar voor je er bent,.}{moet je een heel stuk lopen}{wel acht straten door}\\

\haiku{Hier en daar was een.}{huis als bij toeval op het}{land neergevallen}\\

\haiku{In de zekerheid}{dat wet en orde aan hun}{kant staan generen}\\

\haiku{D.w.z. het is al eens,?)}{gebeurd maar wanneer gaat het}{weer gebeuren}\\

\haiku{Vertelt hij wat hij?}{heeft gezien of wat hij denkt}{te hebben gezien}\\

\haiku{Nou, luister 's goed,{\textquoteright}.}{zei hij en ging recht overeind}{op de bank zitten}\\

\section{Pauline Marres}

\subsection{Uit: En toen brak de hel los. De verwoesting van Maastricht door Parma in 1579}

\haiku{De verwoesting van}{Maastricht door Parma in 1579}{Colofon}\\

\haiku{Weet je dat ik je?}{alleen heb meegenomen}{om haar te treffen}\\

\haiku{Kom Mirza, nu gaan '.}{we proberenn beetje}{geluk te vinden}\\

\haiku{{\textquoteright} Stapvoets reed hij tot,.}{bij de dichtstbijzijnde poort}{de Tongerse Poort}\\

\haiku{{\textquoteleft}Als Parma komt, zal,}{hij hier ergens wel een brug}{over de rivier slaan}\\

\haiku{de Proenens zijn aan,?}{hen verwant daar hebt u toch}{wel eens van gehoord}\\

\haiku{Weet u dat keizer?}{Karel zelfs bij hen aan huis}{kwam in Antwerpen}\\

\haiku{Die zwetskop van een,.}{waard vertelt meer dan nodig}{is straks ook van mij}\\

\haiku{{\textquoteright} {\textquoteleft}Jan, misschien is hij '.}{opt ogenblik de enige}{wijze in Maastricht}\\

\haiku{Maar als ik geen geld...{\textquoteright} {\textquoteleft}.}{krijgDe stad kan nauwelijks}{haar schuld betalen}\\

\haiku{Toen ze lager keek,,;}{naar hun sluisje sloeg haar een}{plotselinge schrik}\\

\haiku{Ja, liefje, je lag,.}{in de Jeker we hebben}{je eruit gehaald}\\

\haiku{Veel kinderen in ',.}{t kamp hadden er wel een}{altijd dezelfde}\\

\haiku{'t Lag zomaar op,,:}{een bankje voor een tent er}{was niemand die zei}\\

\haiku{{\textquoteleft}Misschien beschermt 't,{\textquoteright}.}{je nog eens als er kogels}{rondvliegen zei ze}\\

\haiku{En bovendien een,,?}{kale muts zonder kant zelfs}{geen geplooide strook}\\

\haiku{{\textquoteright} zei moeder Mina, {\textquoteleft}.}{kanten en stroken horen}{bij rijke mensen}\\

\haiku{Toen ze in de tuin,.}{waren kwam opeens een hond}{blaffend op hen af}\\

\haiku{Toen reikte de knecht,.}{hem het wambuis aan eveneens}{bruin en zwart gestreept}\\

\haiku{als ik er weer een,.}{moet verkopen zal ik naar}{Luik of Leuven gaan}\\

\haiku{{\textquoteright} zei hij verbluft en.}{tevens beledigd door de}{gemeenzame toon}\\

\haiku{Alleen maar kijken '.}{naar de mooie mensen ent}{lekkere  eten}\\

\haiku{Toen sprong ze brutaal '.}{op Nol's bed om te voelen}{hoe zachtt wel was}\\

\haiku{Van Daelen had wel;}{verwacht dat hij Manzano}{hier zou aantreffen}\\

\haiku{die elk half uur op,.}{zijn trompet blaast ten teken}{dat hij wakker is}\\

\haiku{Er werd hard gewerkt,,;}{aan de wallen en poorten}{torens en grachten}\\

\haiku{'t Oude geschut,.}{werd weer bruikbaar gemaakt ook}{nieuw werd gegoten}\\

\haiku{In ploegen werkten.}{alle gildeleden mee}{met de soldaten}\\

\haiku{{\textquoteright} herhaalde hij en.}{keek de ander een hele}{poos weifelend aan}\\

\haiku{maar een mokkend of.}{ontevreden gezicht kon}{hij niet ontdekken}\\

\haiku{{\textquoteright} Langzaam, en onder.}{veel gesteun haalde hij een}{mand vol houtblokken}\\

\haiku{{\textquoteleft}Jij, je vader, je.}{moeder zullen zelf moeten}{zorgen voor alles}\\

\haiku{Proenen kwam na wat.}{heen- en weergeloop de}{kamer binnen}\\

\haiku{Ze moet dan morgen,,.}{een goede kip brengen Nol}{en wat eieren}\\

\haiku{Onderwijl moet jij, '.}{hier opruimen ookt bed}{in orde maken}\\

\haiku{{\textquoteright} Manzano had een.}{tijdlang in noordelijke}{richting staan kijken}\\

\haiku{Dat was nu heel wat.}{eenvoudiger geworden}{dan de eerste keer}\\

\haiku{Omdat door deze.}{deur zo'n heerlijke etenslucht}{komt als ze opengaat}\\

\haiku{De poorten waren,.}{dichtgegooid de bezetting}{betrok de wallen}\\

\haiku{Dag-in dag-uit zat '.}{ze in haar stoel en prikte}{haar naald int goed}\\

\haiku{{\textquoteleft}Hoe kon 'n mooie vrouw:}{als mevrouw Suetendael zich}{zo toetakelen}\\

\haiku{Oude monniken,.}{liepen ernaast met dikke}{brandende kaarsen}\\

\haiku{Vrouwen, kinderen,;}{grijsaards en zieken volgden}{in dichte rijen}\\

\haiku{Links en rechts verkocht.}{hij kaarsen aan mensen die}{zich mee aansloten}\\

\haiku{Maar als je wat vindt,;}{zorg dan dat niemand op straat}{kan zien wat je draagt}\\

\haiku{Aan de overkant van,, '.}{de Maas in Wijk int huis}{van sinjeur Tapijn}\\

\haiku{En terwijl hij naar,.}{boven klom greep de wanhoop}{hem weer bij de keel}\\

\haiku{De wal was hoog, de.}{gracht daaronder heel diep en}{met water gevuld}\\

\haiku{Als reden werd hem.}{verteld dat de grond er te}{vochtig was voor kruit}\\

\haiku{hij meende zelfs nu.}{en dan een druppel water}{te horen vallen}\\

\haiku{Als hij maar onder, '.}{die vervloekte gracht door was}{zout beter gaan}\\

\haiku{Inderdaad keek de.}{man met een schuin oog naar Nols}{manipulaties}\\

\haiku{Hij miste de moed.}{en de lust om zich naar zijn}{post te begeven}\\

\haiku{Hier en daar stond een.}{verlaten houten hut van}{een warmoezenier}\\

\haiku{'t Restje luie zweet,.}{dat je nog hebt zullen we}{er wel uitpersen}\\

\haiku{Waar zal ik schuilen?}{als de Spaanse benden over}{de stad losbreken}\\

\haiku{Blijkbaar herkende,:}{ze hem want hij hoorde een}{stem zachtjes roepen}\\

\haiku{Toen ze de eerste,;}{tak had bereikt zwiepte ze}{de ladder omver}\\

\haiku{Ze hield een hand in.}{haar zij om haar lachen in}{bedwang te houden}\\

\haiku{{\textquoteright} Bella zag niets, ze;}{keek weer recht voor zich uit als}{een slaapwandelaar}\\

\haiku{Ze zuchtte, omdat.}{ze zo opeens duidelijk}{de plaats herkende}\\

\haiku{Over 't dak kon hij,.}{niet gevlucht zijn de raampjes}{waren gesloten}\\

\haiku{Hij doorboorde 't;}{gezwel zodat de vuile}{inhoud eruit spoot}\\

\haiku{Ze stelden zich in,.}{twee rijen op voor zijn tent}{trokken hun degens}\\

\haiku{Als u mee wilt gaan '...}{ent voor deze keer wilt}{verontschuldigen}\\

\haiku{voor deze keer want,;}{die vreemde heren blijven}{niet zo lang ziet u}\\

\haiku{Hij keerde zich om.}{en zag een vrouw zitten die}{groente schoonmaakte}\\

\section{H. Marsman}

\subsection{Uit: De dood van Ang\`ele Degroux}

\haiku{Dit is mijn klooster,.}{jammer voor jou dat jij er}{moeder-overste bent}\\

\haiku{landschap na landschap;}{vouwde zich in hem open en}{verschemerde weer}\\

\haiku{Hij liep door drukke,.}{en stillere straten maar}{hij merkte het niet}\\

\haiku{Het was op een avond.}{in September geweest in}{het vorige jaar}\\

\haiku{Met dit gedrocht had,!}{zij geleefd met die vrouw had}{dit monster geleefd}\\

\haiku{- Wij kennen elkaar,... -}{nog zoo weinig maar als het}{u interesseert}\\

\haiku{Toen, terugkeerend uit,,:}{zijn gedachten zeide hij}{Charles strak aanziend}\\

\haiku{Maar toen hebben uw.}{oogen mij haar beeld met een schok}{teruggegeven}\\

\haiku{Neen, bewijzen kan,?}{ik het niet maar wat kunnen}{we w\`el bewijzen}\\

\haiku{Rutgers zat, klein en,.}{wanstaltig aan zijn voeten}{en keek in het vuur}\\

\haiku{Toen het roepen aan.}{den concierge om de deur}{voor hem open te doen}\\

\haiku{De avenue was een.}{zwarte spiegel waarin de}{lantarens dansten}\\

\haiku{Haar antwoord klonk zacht,,,.}{al lag er dacht Charles iets}{van afkeuring in}\\

\haiku{Het omgaf zijn  .}{lichaam met een frissche en}{intieme warmte}\\

\haiku{Het was laat in den.}{avond toen Charles begon te}{spreken over Ang\`ele}\\

\haiku{Maar het gevoel van,,.}{verwantschap waarover hij sprak}{ontbrak haar geheel}\\

\haiku{Maar hij kreeg ondanks.}{herhaalde en dringende}{aanvraag geen gehoor}\\

\haiku{Ik geloof dat u.}{ook de overeenkomst tusschen}{haar en uzelf overschat}\\

\haiku{Hij had alleen, na,.}{hun breuk maandenlang als een}{kluizenaar geleefd}\\

\haiku{Maar de dood deed niets,.}{overhaast hij ging langzaam en}{overwogen te werk}\\

\haiku{Maar misschien had hij,....}{haar wel vergeten misschien}{ook was hij getroost}\\

\haiku{Van der Mark was een;}{lange magere man van}{omstreeks veertig jaar}\\

\haiku{De scherpte van zijn:}{onderzoekenden blik had}{echter niets agressiefs}\\

\haiku{- Het is ontzettend,,.}{zei Charles maar hij had het}{gevoel dat hij loog}\\

\haiku{Van der Mark zag dat.}{hij streed met een moeilijkheid}{en bekortte die}\\

\haiku{de eenzaamheid zijn.}{kan en hoe weinig wij voor}{elkaar kunnen doen}\\

\haiku{wij zijn het alleen...}{w\`el geweest vijf jaar lang in}{het hart van Ang\`ele}\\

\subsection{Uit: Vijf versies van 'Vera'}

\haiku{- tenzij je het op.}{een heel                         bizondere}{wijze verantwoordt}\\

\haiku{Ik weet het niet, hoop.}{daarover binnenkort een}{groot stuk te schrijven}\\

\haiku{V\'o\'or het einde van.}{het jaar waren er 1200}{exemplaren verkocht}\\

\haiku{Vera overhandigt den-?}{man met de stofjas haar pas.}{Waar woont u}\\

\haiku{Zal ik mij moeten,;}{verkwanselen voor weinig}{geld voor veel geld}\\

\haiku{Waar loopt                       d\`eze,*?}{op uit        56  ~          waar}{loopt alles op uit}\\

\haiku{en                     hij schatte}{de soepele spanning der}{borsten waartusschen}\\

\haiku{De jongen trok met,;}{ruime krachtige slagen}{de boot door het meer}\\

\haiku{Daarbinnen leefden:}{zij    alsof dit eiland}{de wereld                     was}\\

\haiku{Het is alsof ik.}{een bron ben geworden}{of een fontein}\\

\haiku{Niemand van hen    .}{dorst te breken met een}{ontspoord resultaat}\\

\haiku{bitterheid stond     '.}{in zijn                     mond als hij dacht}{aan November18}\\

\haiku{Zij liep haastig, licht,.}{voorovergebogen dicht langs}{de huizen}\\

\haiku{hun armen stonden.}{op elkaars    schouders en}{maakten een                     brug}\\

\haiku{Berlijn was    een,.}{stad                     vol inventie vol}{wilskracht en energie}\\

\haiku{Hij was slank en                     .}{forsch en onweerstaan-}{baar-krachtig}\\

\haiku{de oude boomen.}{voor het    hotel in}{de Victoriastrasse}\\

\haiku{Dan nam zij zijn hoofd;}{in haar handen en streelde}{hem door zijn haar}\\

\haiku{Is het niet beter}{\'e\'en klein maar stellig geluk}{te bezitten}\\

\haiku{De trein remde en.}{reed knarsend de dreunende}{overkapping binnen}\\

\haiku{plotseling was zij.}{onzichtbaar    geworden}{in het gedrang}\\

\haiku{Plotseling riep zij,.}{den kellner betaalde en}{ging haastig weg}\\

\haiku{Zij riep een taxi, die.}{de straat afzakte en gaf}{haar                     hotel op}\\

\haiku{Och, dat had ik je.}{ook van    te voren wel}{kunnen voorspellen}\\

\haiku{wat heeft zij nu nog,(}{over nu zij haar    geloof}{heeft verloren}\\

\haiku{Het is misschien niet}{waar dat ik niet in het}{klooster gegaan ben}\\

\haiku{ik begrijp Judas,,,.}{die hem verried en Petrus}{die                     hem verried}\\

\haiku{Ik heb als dienstmeid,,.}{gewerkt in Dahlem in}{Hermsdorf in Gr\"unau}\\

\haiku{ik rekende op,;}{zijn wel-levendheid op}{zijn ridderlijkheid}\\

\haiku{Nu mag er niets dood,,;}{zijn en wij wij mogen ook}{nooit meer doodgaan}\\

\haiku{een 11                     droomloos [, [;}{en droomloos en 12 leven}{is en leven is}\\

\haiku{Zij kreunde, haar mond,.}{bleef gesloten ook haar}{oogen waren nu dicht}\\

\haiku{Maar hij had daarna, -:}{toen hij                     voelde dat dit}{gemis of liever}\\

\haiku{- Maar... ik doe toch niet.}{enkel uit heerschzucht het}{werk op het atelier}\\

\haiku{Misschien [ van Dreyer,,.}{d\'aarom althans niet boven}{de Moeder stellen}\\

\haiku{- h\`eldere pijn 15--.}{17                      Hoe zwak was zij en}{hoe klein was haar hart}\\

\haiku{7 - Ja ik ben moe,?}{en nu wil ik alleen}{zijn begrijp je dat}\\

\haiku{Ik [ - Ja, ik ben moe,,?}{en nu wil ik alleen zijn}{begrijp                     je dat}\\

\haiku{17Tekst C. Zie                             ,.}{Verantwoording blz. 19 van}{deze uitgave}\\

\haiku{ander woord                                                                                                                        12-}{13                                Het laken over haar}{schouder ademt heel zacht}\\

\haiku{en hij, na den nacht,.}{met Hedda voelde zich}{verjongd en ontsmet}\\

\haiku{] wijd en hij, na den,.}{nacht met Hedda                                     voelde}{zich als verjongd}\\

\haiku{88 c                                                                7-13 [...].}{zoo sterk contrasteerde}{machtige boeren}\\

\haiku{altijd {\textquoteleft}voor                                     het...{\textquoteright}- [...]...}{eerst                                                                                                                        2324                                Voor het}{eerst aarzelde hij}\\

\haiku{33                                z\'o\'o [ z\'oo                                                                         - [...}{110 c                                                                45                                dan}{het lieflijk verband}\\

\haiku{] simpele                                                                                                                        28-....{\textquoteright}.}{32                                Walter Hedda schuift}{de krant van zich af}\\

\haiku{overbodig                                                                                                                        24 ],}{was vergeleken was}{vergeleken}\\

\haiku{valsch omlijnd                                                                                                                        12,, -:}{binnen zou komen groot}{jong en krachtig OPM}\\

\haiku{- h\`eldere pijn het.}{zou temperen maar                                     toch}{tevens doorvlijmen}\\

\haiku{er ] het kan niet meer,,....}{hij is al veranderd}{er                                                                                                                        29                                mij}\\

\haiku{Toen zag                                                                                                                        33                                 ]:}{zichzelve zichzelf                                                                         138}{c Aan het slot OPM}\\

\section{H. Marsman en Simon Vestdijk}

\subsection{Uit: Heden ik, morgen gij}

\haiku{s Avonds hadden wij.}{elkander niet gezien of}{althans niet herkend}\\

\haiku{Waaraan ontleenen die?}{vlerken dat air behalve}{aan hun onnoozelheid}\\

\haiku{omdat het hunkert.}{naar krasse bewijzen van}{vriendschap en liefde}\\

\haiku{veel lucht (en vergeet:}{niet de monden waar die lucht}{uit is gekomen}\\

\haiku{- Ja, zei hij, vooral.}{in Huelva heb ik heel}{veel geluk gehad}\\

\haiku{Ik verheug me in,.}{je woede dat je dit weer}{van me slikken moet}\\

\haiku{dan kan je meteen....}{die watten voor nuttiger}{dingen gebruiken}\\

\haiku{Acht uur ongeveer.}{werd ik door stemmen uit een}{diepe slaap gewekt}\\

\haiku{Wij staan 's morgens.}{vroeg op en na het ontbijt}{gaat ieder zijn gang}\\

\haiku{Alleen zijn hoofd is,.}{gevonden door een jongen}{die zwom in de beek}\\

\haiku{Daarop gingen wij,.}{met ons vieren het veld in}{op zoek naar het lijk}\\

\haiku{Paul keek dan vragend,.}{naar Wevers in afwachting}{van diens beslissing}\\

\haiku{- Mijne heeren, ik!}{drink op de duizendste meid}{die ik gehad heb}\\

\haiku{Toch ben ik na het.}{verslag van je fuifje niet}{volkomen gerust}\\

\haiku{de geschiedenis,.}{met Annie is van later}{datum zooals je weet}\\

\haiku{blonde meisjes met,.}{een sentimenteel lachje}{dom en opgeprikt}\\

\haiku{Aan zijn manier van,.}{spreken hoorde ik dat zijn}{lippen droog moesten zijn}\\

\haiku{Ik begrijp nu ook;}{wel dat mijn vorige brief}{je geprikkeld heeft}\\

\haiku{De rivier slingert.}{rondom de stad en op \'e\'en}{plaats even er doorheen}\\

\haiku{{\textquoteleft}U hebt volkomen,,;}{gelijk Don Pedro hij werkt}{inderdaad te hard}\\

\haiku{Snellen overschat de.}{dood inderdaad omdat hij}{het leven overschat}\\

\haiku{Maar goed, wat geeft het,.}{Holland heeft zijn tijd en zijn}{beschaving gehad}\\

\haiku{zoo gelukkig als.}{een niet-bekeerde zich}{zelfs niet kan droomen}\\

\haiku{{\textquoteright} Maar Don Pedro, haar,,,:}{vraag negeerend zei nauwelijks}{ironisch tot Wevers}\\

\haiku{{\textquoteleft}Neen, wij slapen nog,{\textquoteright} {\textquoteleft}}{zei Nettie lachend en greep}{hartelijk zijn hand.}\\

\haiku{Ik kan je zeggen,,.}{v. M. dat het mij nu toch}{een opluchting is}\\

\haiku{Wevers' verdere.}{ontwikkeling wacht op mijn}{welversneden pen}\\

\haiku{Zonder 't misschien.}{te willen dring je je in}{mijn intimiteit}\\

\haiku{dag meneer Wevers,,.}{mag ik uw schoenen likken}{of iets dergelijks}\\

\haiku{weet je, dat je me,,?}{die avond verleden jaar had}{kunnen vermoorden}\\

\haiku{{\textquoteright} Maar waarom Nettie,.}{daar nu ineens heen wil is}{mij niet duidelijk}\\

\haiku{Maar bewijst het niet,?}{iets dat ik nu juist deze}{stemmingen doormaak}\\

\haiku{ik zal maar blijven.}{en trachten de zaak nog in}{orde te krijgen}\\

\haiku{- Ik dacht wel dat je,,}{er niets van begrijpen zou}{zei ze iets zachter}\\

\haiku{Het spijt me, maar nu.}{zal ik je illusie toch}{moeten verstoren}\\

\haiku{Begrijp je nu dat?}{ik toch nog geloof dat je}{verliefd op hem bent}\\

\haiku{Intusschen is het,}{heel goed mogelijk dat hij}{mediumiek niet zoo}\\

\haiku{eerst liet ze me vijf;}{minuten in datzelfde}{kamertje wachten}\\

\haiku{Op een avond zaten.}{Wevers en ik bij elkaar}{in een Haagsch caf\'e}\\

\haiku{Ik kon rekenen;}{op een betrekking aan een}{werf in Rotterdam}\\

\haiku{Ze liep traag en met.}{iets van geremdheid zooals men}{dat in droomen heeft}\\

\haiku{dat ik op dit stuk,.}{iets te verdedigen heb}{of te verraden}\\

\haiku{Als ik mij goed in,.}{die mogelijkheid verdiep}{word ik wanhopig}\\

\haiku{{\textquoteleft}Feitelijkheden.}{over hun leven heeft Wevers}{mij weinig verteld}\\

\haiku{hield zij van hem of,,?}{van M. hield deze van haar}{hield Wevers van haar}\\

\haiku{Ik denk nog altijd,,}{aan haar zij leeft in m{\`\i}j ik}{draag haar in mij om}\\

\haiku{Kom spoedig, lieve, -;}{Wevers maar natuurlijk met}{andere woorden}\\

\haiku{Walging voelde ik,,:}{niet alleen ontgoocheling}{en medelijden}\\

\haiku{wat mij op Wevers'.}{kamer te wachten stond voor}{ik in zou grijpen}\\

\haiku{Ik hoop, dat je je.}{vanavond geamuseerd hebt}{onder mijn leiding}\\

\haiku{Maar ik had er toch.}{blijkbaar Praag bij noodig om tot}{schrijven te komen}\\

\section{Marita Mathijsen}

\subsection{Uit: Seks in Limburg. Gevolgd door dezelfde tekst in het Belfelds}

\haiku{Nee,  dat wist ik,.}{niet en ik vond het toen ook}{geen plausibel idee}\\

\haiku{als je de ene plek.}{dichtplakt barst die wel op een}{andere plaats open}\\

\haiku{t gedeurd had, en.}{of ich drek \'ongerop waas}{k\'omme te ligge}\\

\haiku{Netuurlik goof \'os ',}{mamt kink de bors aevel}{altied zoe\"e det weej}\\

\haiku{Op 'n gegaeve}{moment beg\'os ich mich z\"org}{te make euver}\\

\haiku{n plasgaetje en ' '.}{n poepgaetje ouch nagn}{anger gaetje had}\\

\haiku{As ich d'r zeker,}{van waas det nemes binne}{k\'os k\'omme deej ich}\\

\haiku{Weej krege ze flink.}{euver \'os klungels en weej}{woorte oetgehuuerd}\\

\haiku{zoe\"e det meugelik,}{de h\`elf van mien toete te}{zeen ware gewaes}\\

\haiku{Wie ich  d'r mei,.}{te make kreeg waas ich al}{verhoes oet Limburg}\\

\haiku{Aevel ich b\"on d'r:}{ouch hie\"el zeker van det}{d'r ouch bedoeld woort}\\

\section{Justus van Maurik}

\subsection{Uit: Burgerluidjes}

\haiku{Eindelijk op een ';}{morgen zag ik iemand voor}{t venster zitten}\\

\haiku{Een groote stoel stond bij,;}{de tafel en in dien stoel}{zat een oude vrouw}\\

\haiku{ik hoop niets meer, ik, -;}{verwacht niets meer ik zal hier}{wel eenmaal sterven}\\

\haiku{zij had weinig, maar.}{ze wist het weinige wat}{ze had te waardeeren}\\

\haiku{Hendrik moest een vrouw; '.}{zoekenkwou nog zoo graag mijn}{kleinkinderen zien}\\

\haiku{Ja, zij had hem lief;}{uit al de volheid van haar}{onbedorven hart}\\

\haiku{Toen zij alleen was,.}{met haar moeder was zij naast}{haar stoel neergeknield}\\

\haiku{{\textquoteleft}Heb je er nog nooit,?}{over gedacht wat we met hen}{beginnen moeten}\\

\haiku{Ik zal die oudjes, -....{\textquoteright} {\textquoteleft},...?}{in Godsnaam maar mee laten}{varen maarNu maar}\\

\haiku{{\textquoteright} klonk op eens de stem.}{van den kindschen man uit de}{andere kamer}\\

\haiku{vader en moeder,{\textquoteright} -,.}{riep Truitje en zonk in de}{kussens terug}\\

\haiku{er is dan toch nog,,.}{iemand die mij liefheeft die}{ik liefhebben mag}\\

\haiku{langzaam opdraaien,,:}{en dan hou je aan totdat}{meneer Hopkamp zegt}\\

\haiku{Ziehier in korte.}{trekken den inhoud van Dr.}{Penners eersteling}\\

\haiku{Eindelijk komt het, '.}{oogenblik datt uur van}{middernacht zal slaan}\\

\haiku{Dat vindt Penner ook,:}{en daarom zwijgt hij totdat}{de directeur zegt}\\

\haiku{maar als hij 't me,.}{al te bont maakt loopt hij een}{pak slaag van mij op}\\

\haiku{Zoodoende staat hij, ':}{alleen in zijn oordeel en}{dan zegtt publiek}\\

\haiku{Zeer fideel slaat Hart:{\textquoteleft}!}{zijn arm om Andr\'ees taille}{en neuriet  Ach}\\

\haiku{Elke gedachte,,;}{die Willem voedde deelde}{hij met zijn moeder}\\

\haiku{'k Zou me niet goed,.}{kunnen voorstellen dat je}{van me weg zoudt gaan}\\

\haiku{als vastgenageld!}{bleef zij op haar stoel zitten}{en hoorde alles}\\

\haiku{Toen omvatte zij;}{haar kind met inspanning van}{haar laatste krachten}\\

\haiku{{\textquoteright} - Bram zwijgt even, doet eeu:}{paar flinke trekken aan zijn}{pijp en vervolgt dan}\\

\haiku{{\textquoteright} {\textquoteleft}En wij gaan reizen(!){\textquoteright}.}{door de wollek-khik}{gilt de andere}\\

\haiku{omdat je weet, dat,.}{je ouwe niet naar bed gaat}{v\'o\'or jij binnen bent}\\

\haiku{- Buiten klettert de '.}{regen op het zinken dak}{vant wachthuisje}\\

\haiku{Alleen datgene,,.}{wat opvalt of bijzonder}{is wordt opgemerkt}\\

\haiku{{\textquoteleft}'t Is mooi weer voor, -.}{een inbreker zoo'n nacht is}{geld waard voor een boef}\\

\haiku{{\textquoteright} zegt Bram, die de vrouw.}{met beide handen onder}{de armen vasthoudt}\\

\haiku{{\textquoteleft}Die lap zal 't hem,.}{net doen voor een gordijn als}{hij lang genoeg is}\\

\haiku{dat is dan toch een, -.}{heele lastpost voor je zoo'n}{zwak schepsel in huis}\\

\haiku{Dan komt de schimmel, -!}{niet in je krullebol dat}{zou zonde wezen}\\

\haiku{{\textquoteright} en terwijl zij dit.}{vraagt buigt zij zich voorover met}{haar oor naar zijn mond}\\

\haiku{Komt hij thuis, dan slaapt.}{hij uit en kan op avontuur}{morgen weer werken}\\

\haiku{'t Is maar voor de. '}{schandaligheid onder weg}{en voor de buren}\\

\haiku{'t Was alleen maar, ':}{een Baron diet zwaar op}{z'n zenuwen had}\\

\haiku{Ik ben het met u,.}{eens al de nieuwigheid is}{geen verbetering}\\

\haiku{Zij nam het glas, dat,.}{op mijn beddetafeltje}{stond en rook er aan}\\

\haiku{{\textquoteright} zei ik, blij dat er,.}{een oogenblik was dat haar}{woordenvloed ophield}\\

\haiku{{\textquoteright} {\textquoteleft}Ja, spot er maar mee,,,:}{meneer maar het is een goeie}{raad dien ik je geef}\\

\haiku{drie klontjes witte....}{suiker met een droppel of}{zes Haarlemmerolie}\\

\haiku{Uw\'e heeft mij een dienst;}{gedaan en daarom heb ik}{voor u ook wat over}\\

\haiku{zij naderde mijn,:}{bed lei zachtkens haar hand op}{mijn voorhoofd en zei}\\

\haiku{{\textquoteright} riep het goede mensch,.}{met een ontdaan gelaat toen}{zij in het bed keek}\\

\haiku{{\textquoteright} Glimlachend dacht ik}{na over het sic transit en}{richtte mij zoo goed}\\

\haiku{Nu eens trilde ik,;}{op mijn beenen als een door den}{wind bewogen riet}\\

\haiku{Ik zal zoo vrij zijn,.}{om dan ook dit eindje voor}{hem over te laten}\\

\haiku{dat is naar mijne.}{meening een kort begrip van}{de geneeskunde}\\

\haiku{{\textquoteright} {\textquoteleft}Neem je pillen trouw,.}{in lees het boek Job en laat}{de rest aan mij over}\\

\haiku{{\textquoteright} {\textquoteleft}Goeie hemel! 't schijnt;}{wel alsof de heele buurt}{weet dat ik ziek ben}\\

\haiku{{\textquoteleft}Als uw\'e dan partoet,;}{uit wil moet je maar in je}{sjamperloepie gaan}\\

\haiku{Ik voelde, dat ik,:}{wit werd van kwaadheid maar ik}{hield mij kalm en zei}\\

\haiku{{\textquoteright} Een tweede slag op:}{den kant van de tafel deed}{mij verschrikt vragen}\\

\haiku{Anders kon jij net...{\textquoteright} {\textquoteleft},!}{als ieder ander voor mijn}{part naar denMaar oom}\\

\haiku{Hij ging vlak voor mij,:}{staan streek zijn knevels op en}{bulderde mij toe}\\

\haiku{Zij zag in hem een:}{bondgenoot en voegde hem}{daarom halfluid toe}\\

\haiku{{\textquoteright} waagde ik in het, {\textquoteleft}!}{midden te brengendat is}{toch puur bijgeloof}\\

\haiku{hij dweept er mee en {\textquotedblleft}{\textquotedblright}.}{daarom he\`eft hij zijn huis ook}{villa Bay genoemd}\\

\haiku{'k geloof dat de ';}{commandants avonds wel wat}{veel aan de wieg stoot}\\

\subsection{Uit: Op reis en thuis}

\haiku{'k habbe 'm van '.}{t versoepe gered en}{noe is d'r so trouw}\\

\haiku{'t Volk krijgt aan boord!}{twee maal per dag een oorlam}{en daarmee basta}\\

\haiku{Zie je, meneer, dat.}{is zoowat schering en inslag}{van z'n redenasies}\\

\haiku{*** Op 't achterdek {\textquoteleft}{\textquoteright};}{klinkt vroolijk het orkest van}{de gloeiende pook}\\

\haiku{Als hij er maar bij,.}{was kon je zeker zijn dat}{een fuif goed afliep}\\

\haiku{Zuiver en helder.}{klonk het eenvoudige lied}{in den stillen nacht}\\

\haiku{- Dat is een van de,,;}{mooiste melodi\"en die}{ik ken zei hij zacht}\\

\haiku{Toen de aanvoerder,}{gevallen was kwamen al}{de apen naar hem toe}\\

\haiku{Ik maakte gebruik '.}{van de gelegenheid en}{koost hazenpad}\\

\haiku{Een nieuwe, zeker,.}{door hem uitgevonden term}{voor verliefd te zijn}\\

\haiku{de dame begint.}{zachtjes te kuchen en wrijft}{nu en dan haar oogen}\\

\haiku{- Dat geveugel an,.}{m'n lijf kan ik niet velen}{laat dat maar waaien}\\

\haiku{ik heb zoo nu en:}{dan wat van u gelezen}{en daarbij dacht ik}\\

\haiku{Sina's vader, dit,.}{wou ik u straks vertellen}{was niets inhalig}\\

\haiku{Volgens de wet was,:}{hij haar erfgenaam maar hij}{kwam bij me en zei}\\

\haiku{Aan boord waren reeds;}{de meeste passagiers in}{in zalige rust}\\

\haiku{dan moet hij voor mij,....}{voor een schuifje werken dan}{helpen we mekaar}\\

\haiku{J. Bruin, Beeldhouwer,,,.}{3 {\texttimes} schellen toonde hem}{dat hij terecht was}\\

\haiku{- maar, ziet u, d'r is '....}{tegenwoordig al niet veel}{te doen int vak}\\

\haiku{bedaard aan, we zijn, ';}{nog zoo ver niet laat maar eens}{eerstt model zien}\\

\haiku{- Daar zijn we nu, - zei,:}{Capelli zijn hoed en stok}{op een stoel leggend}\\

\haiku{En d\`en, om je de,,.}{waerheid te zeggen z'n}{neus die w\`es \'enders}\\

\haiku{- D'r is van voren.}{wat afgehaald en aan de}{eene zij bijgebracht}\\

\haiku{{\textquoteright} - Zie je, juffrouw Bruin, '.}{dat is nut r\'esum\'e}{van al de opinies}\\

\haiku{Hij keek somber v\'o\'or:}{zich en zei met een weinig}{gemaakte tragiek}\\

\haiku{Mijn familie is.}{nog bijna geheel en al}{ten mijnen laste}\\

\haiku{Onlangs ben ik zes;}{weken op Tourn\'ee geweest}{in de provincie}\\

\haiku{wat een ordinair:.}{gezicht wat kijkt die kerel}{knorrig als hij gaapt}\\

\haiku{'t is me alsof -.....}{ik draai ik dommel in en}{ik droom van mooi we\^er}\\

\haiku{De voerman heeft een,.}{blauwen kiel aan die zwart ziet}{door de nattigheid}\\

\haiku{Daar nadert iemand,, '.}{ik herken hemt is een}{kurgast evenals ik}\\

\haiku{Hun vesten hangen.}{open en de veelkleurige}{bretels zijn zichtbaar}\\

\haiku{{\textquoteright} was de eenig troost, die:}{hij van haar kreeg en vinnig}{voegde zij erbij}\\

\haiku{toch moet hij {\textquoteleft}voor das.}{fraumensch ergens in abendbrot}{zoesammenblasen}\\

\haiku{Enfin! 't doet me ' '.}{toch pleizier datkreis weer}{Hollandsch kan spreken}\\

\haiku{dat's waar ook, hoe?}{heet de burgemeester van}{Amsterdam ook weer}\\

\haiku{k ontmoette hem.}{een jaar of wat geleden}{ook bij de courses}\\

\haiku{Schuins d'r over kocht ik}{altijd mijn sigaren in}{een klein winkeltje}\\

\haiku{, meneer, tegen die.}{dingen kunnen de Fransche}{patissiers niet aan}\\

\subsection{Uit: Oude kennissen}

\haiku{- denk je om de leege....}{flesschen van de bessensap}{te laten terug}\\

\haiku{gossiemijne, neem, ';}{me niet kwalijk ik kont}{heusch niet helpen}\\

\haiku{ik ben jaren bij, '}{de van Palens over den vloer}{gekomenk Was}\\

\haiku{Ja, ja, en dan zoo, ' '.}{een en ander meert is}{toch nooitt eigen}\\

\haiku{en tien minuten.}{later waren ze al bij}{onzen lieven Heer}\\

\haiku{- Wil je die Cantenac,.}{van Burgers niet eens proeven}{voor de aardigheid}\\

\haiku{ga ik me niet gauw,;}{te waag vooral niet als ik}{de firma niet ken}\\

\haiku{Gul {\textendash} Uche! - den - neen, maar! -?}{dat is onmogelijk Vindt}{je hem dan te duur}\\

\haiku{Zeg Meijer, nu kun.}{je meteen kennis maken}{met je concurrent}\\

\haiku{zegt de voorzitter,, -!}{met welgevallen zich zelf}{hoorend Mijne heeren}\\

\haiku{- Onze taal is rijk, -! -! -!}{aan klanken die klinken hum}{klanken vol klank hum}\\

\haiku{- en ik zou hier voor -!}{uw pleizier over de vreugd staan}{schetteren merci}\\

\haiku{zijn er intusschen,?}{ook nog stukken of brieven}{gekomen Bode}\\

\haiku{{\textquoteright} {\textquoteleft}Haal dadelijk den,.}{agent die hier altijd aan den}{hoek van de straat staat}\\

\haiku{ge neemt afscheid van,, -!}{een goede lieve vriendin}{die ver ver weg gaat}\\

\haiku{Wie kent ze niet, wie -?}{heeft ze niet lief en wie is}{er niet benauwd voor}\\

\haiku{wat 'n mooie meid was -, -!}{dat mooi meneeren een engel}{om te stelen}\\

\haiku{je herinnert 't,.}{je wel je hebt er laatst nog}{zoo over geroepen}\\

\haiku{Voor hem, van hem, om,,,.}{hem is alles zijn wil is}{wet zijn ik alles}\\

\haiku{heeft ze die w\`el, dan.}{is ook ook haar stervensuur}{nog een marteling}\\

\haiku{Er is veel geestkracht - ';}{toe noodig om zacht te zijnt}{is het deel der vrouw}\\

\haiku{er zich al gauw mee -.}{wennen ik doe alles met}{Jantje op mijn arm}\\

\haiku{- Wat zij vandaag weer,,;}{heeft begrijp ik niet maar ze}{is uit haar humeur}\\

\haiku{- Ik zou wel uit de,? ......................}{zaken willen want waar werk}{ik eigenlijk voor}\\

\haiku{Dat begint hem toch;}{te h{\`\i}nderen en hij zoekt}{zich te verstrooien}\\

\haiku{'t Is maar voor de,.}{meiden die moeten sluiten}{en op tijd naar bed}\\

\haiku{Honderd maal op een.}{dag hoort men tegenwoordig}{deze verzuchting}\\

\haiku{wel de moeite waard,}{om n\'og eens een schoonpapa}{te vereeuwigen}\\

\haiku{Om te beginnen,,!}{dan vergun mij dat ik mijn}{vriend Pluim even voorstel}\\

\haiku{je moest dat liever - ', '.}{niet doent hindert me ik}{kant niet velen}\\

\haiku{Sophie is een......}{hoogst fatsoenlijke vrouw en}{ik verbied je om}\\

\haiku{Jaloersche mannen,.}{worden er ook gevonden}{maar meer sporadisch}\\

\haiku{'k zou toch wel eens,.}{even willen kijken hoe die}{baker er uitziet}\\

\haiku{Voordat zij 't zelf.}{weet is ze teruggegaan}{en heeft aangeklopt}\\

\haiku{- Honnig! - Nou Mevrouw! ';}{t ziet er dan maar reintjes}{en illegant uit}\\

\haiku{Wij menschen hebben, ';}{er al gauw een kijkkie op}{watt worden zal}\\

\haiku{Hij heeft grooten lust,?}{om maar dadelijk weer heen}{te gaan maar waarheen}\\

\haiku{De kantoorjas, met ',;}{t stof van den dag er op}{wordt weggehangen}\\

\haiku{dan zal 't eten je,,,!}{des te beter smaken Komt}{jongens aan tafel}\\

\haiku{je zult papa  ,;}{knorrig maken Vader denkt}{er echter niet aan}\\

\haiku{Als hij vertrokken,;}{is gebruikt zij haar ontbijt}{en haar twaalf-uurtje}\\

\haiku{ze gaven ieder, -}{het zijne en hielden zelf}{een kleinigheid over}\\

\haiku{Pardon - ik zag zoo,...;}{dadelijk niet dat U is}{alles wat hij uit}\\

\haiku{- Nu ben je heusch,,!}{een lieve man dat je me}{helpt opruimen hoor}\\

\haiku{zoodat de ruiter zijn - -.}{arm de sabel ontbreekt we\^er}{dreigend vooruitstreekt}\\

\haiku{Ik was dol op 'r,;}{en ik heb altijd geloofd}{dat zij mij ook mocht}\\

\haiku{Krampachtig knelde:}{hij zijn handen om zijn hoed}{en heesch zei hij}\\

\haiku{Als ze 't op d'r ' '.}{heupen had kon niemandt}{metr uithouwen}\\

\subsection{Uit: Stille menschen}

\haiku{Hei jij niet grappies,!}{verteld an de ouwe man}{terwijl ie werkte}\\

\haiku{- dat's bloed voor bloed,,!}{dat heb je aan Berbertje}{verdiend valsche hond}\\

\haiku{- Al ben je ook nog,...!}{zoo goedkoop ze deuge niet}{ze deu-eugen niet}\\

\haiku{Berbertje zei dat ' '.}{t z\'o\'o was en daarom zou}{t wel z\'o\'o wezen}\\

\haiku{zorg jij maar dat je,.}{goed en mooi werk maakt dan zorg}{ik voor klandisie}\\

\haiku{{\textquoteright} Te Amsterdam zou.}{zij een ruimer veld vinden}{voor haar werkzaamheid}\\

\haiku{- dat ik ook dood was - ' - '!}{God gaftk heb toch niets}{meer op de wereld}\\

\haiku{mijn vrouw was dan al ',,!}{n buitengewoon mensch zoo'n}{geestige goeie ziel}\\

\haiku{- Ben je gek, dikke,,!}{mogol wat verbeeld jij je}{wel oud gedierte}\\

\haiku{Denk je misschien dat.}{ik voor jou borsteltjes zal}{gaan zitten maken}\\

\haiku{Oome Daan richtte,:}{zich een weinig op en vroeg}{haastig verwijtend}\\

\haiku{ik had er niet meer,.}{naar omgekeken evenmin}{als naar mijn model}\\

\haiku{{\textquotedblleft}juffrouw{\textquotedblright} mag je d'r,{\textquoteright}.}{wel aflaten riep schamper}{de wijnhuishoudster}\\

\haiku{- Dat's niet waar - en -?}{je hoed vol deuken wat is}{er met je gebeurd}\\

\haiku{- - Ze verkochten ze - -.}{d\'a\'ar ten voordeele van van de}{kamerontbinding}\\

\haiku{{\textquoteleft}U is hier zeker,,?}{verkeerd meneer of wenscht}{u mij te spreken}\\

\haiku{nu begon ik te,;}{begrijpen dat ik te doen}{had met een slimmerd}\\

\haiku{- Welnu, dan moeten '.}{de goeient maar met de}{kwaaien ontgelden}\\

\haiku{Een goeie geeseling.}{of een brandmerk is misschien}{ook al voldoende}\\

\haiku{je ziet er toch niet,.}{naar uit dacht ik en nam den}{oude beter op}\\

\haiku{En Brammetje, die ',:}{zich mett geval amuseert}{zingt eensklaps solo}\\

\haiku{op 't portaal niest, '.}{hij een paar maal wantt is}{er koud en tochtig}\\

\haiku{- Daar is de ketel;}{en nu weet ik meteen waar}{die tocht vandaan komt}\\

\haiku{En 't kind schreeuwt zich -,;}{een ongeluk ik begrijp}{er niets van niets helpt}\\

\haiku{hoe komt zoo'n kind nu -,.}{ineens aan influenza}{hoor hij hoest ook weer}\\

\haiku{Zij zag zijn profiel;}{scherp afsteken tegen het}{verlichte venster}\\

\haiku{{\textquoteright} {\textquoteleft}U kent me niet meer,,;}{meneer Bernard maar dat wil}{ik wel aannemen}\\

\haiku{hij verzon voor mij,,:}{allerlei ik lachte d'r}{wel reis om en zei}\\

\haiku{{\textquoteleft}'k Moest kijken of, ';}{de juffrouw al weg wast}{eten staat op tafel}\\

\haiku{{\textquoteleft}Lieber Bernard, du,!}{bist ein so guter Mensch ein}{so pr\"achtiger Kerl}\\

\haiku{{\textquoteleft}Als ik haar niet had,,!}{stond ik hier niet voor u dan}{was ik allang weg}\\

\haiku{hij rookt ze van vijf, - -.}{om een dubbeltje dat is}{toch geen overdaad Aug}\\

\haiku{hij weet zelfs niet, dat!}{hij zonder overjas de deur}{is uitgeloopen}\\

\haiku{'t Bleef toch nog licht, '.}{in de kamer want de maan}{scheen doort venster}\\

\haiku{ik, daar heb 'k al,.}{reis een paar maal voor in de}{nor gezeten dankie}\\

\haiku{k heb Toon niet eens - '....}{meer kunnen inschenken hij}{zit opn droogie}\\

\haiku{Ik nam dadelijk,?}{men duiten en Klaas keerde}{zijn zak om niet waar}\\

\haiku{- En, vulde Klaas weer,;}{aan toen stonden we op straat}{zonder dubbeltjes}\\

\haiku{zoo na aan 't hart, ',;}{k word er ziek van als ik}{ze niet spreken mag}\\

\haiku{toen jij al lang sliep,...}{heb ik nog een pak gemaakt}{van die boeken en}\\

\haiku{Al 't water in,?}{die kist met boeken wat doe}{jelui aan die kom}\\

\haiku{Plof! - De linnenkast.}{glipt uit de leng en barst op}{straat uit elkander}\\

\haiku{Tegen tien uur houdt.}{de kruiersbaas met zijn knechts}{een half uur schafttijd}\\

\haiku{In 't huishouden! -;}{een kleine handreiking zoo}{iets doet hij gaarne}\\

\haiku{Langzamerhand weet;}{hij juist hoeveel spinasie}{of wittekool kost}\\

\haiku{Na 't diner moet,.}{hij zijn dutje doen dat is}{levensbehoefte}\\

\subsection{Uit: Toen ik nog jong was}

\haiku{je zult zoo aanstonds.}{je adem wel noodig hebben op}{de steile trappen}\\

\haiku{Kijk nu maar eens goed,, '}{rond maar hou je pet goed vast}{wantt waait nog al.}\\

\haiku{- D\'a\'ar was eenmaal de,....}{Buitensingel daar is nu}{de Nassaukade}\\

\haiku{ze dronken d'r ook -}{wel ris een roemertje wijn}{of wat anders en}\\

\haiku{U begrijpt, daardoor,}{snapten ze eindelijk dat}{ze gefopt waren}\\

\haiku{En de schepen, waar,;}{ze dat goed mee aanbrachten}{le{\"\i}en in de gracht}\\

\haiku{Je had daar aan de,.}{Ouwe-brug tegen de}{Korenbeurs gebouwd}\\

\haiku{Het Damrak, Oude.}{brug en Korenbeurs met de}{kleine winkeltjes}\\

\haiku{Ze rollen ze over:}{een uitgelegde plank aan}{wal met een vroolijk}\\

\haiku{Te deksel, Teun, wat. ' '.}{maek je ze blauwt Laikt}{t firmement wel}\\

\haiku{zoo'n kuip doodstil staan,, '.}{dan komt er een dik vel op}{netn stuk le\^er}\\

\haiku{Jaco is reeds meer,.}{dan anderhalve eeuw dood}{want hij werd in Ao}\\

\haiku{ze heit 'r nou een ' (), '.}{metn kriekiebochel dat}{s ook nog een tref}\\

\haiku{hij leest alles wat - '}{ie te pakke kan krijge}{metn ouwe krant}\\

\haiku{Zoolang een van hen, '.}{nog geld of drank had hadden}{zet allemaal}\\

\haiku{Blijf met je poote van,, '....}{de deur ze zelle wel ope}{doen ast tijd is}\\

\haiku{- Ik heb nog 'n neef,,?}{in Nieuwjork die ken jij ook}{wel niewaar Manus}\\

\haiku{- 't Is waarachtig,,!}{of ze binne ruike dat}{ze der ankomme}\\

\haiku{Daarom tikt hij den.}{dikken man zachtjes aan en}{herhaalt zijn verzoek}\\

\haiku{- Broeder en zusters,,.}{geliefde vrienden laat ons}{samen zingen}\\

\haiku{Luid verheft hij zijn:}{stem en met een akelige}{neusklank zingt hij}\\

\haiku{Ischt daar iemand,,,}{die mijn nicht bejreipt das}{hij frage ich soll}\\

\haiku{- Profeet, ik kan u,.}{niet erg goed zien de kachel}{staat zoo in den weg}\\

\haiku{- Daar heb je nummer,,.... -!}{drie Trui die zal ook wel over}{de kachel Profeet}\\

\haiku{{\textquoteleft}Sister, come to,!}{the front der heilige Jeist}{sal doerch you schpreeken}\\

\haiku{we zullen eerst die,.... '....}{vuile boel afwasschen laat}{je oogris kijken}\\

\haiku{En ik wil niks met ' - ';}{m te make hebbe ik}{ben schuchter vanm}\\

\haiku{Jan d'r naar toe en,:}{op z'n luister en toen ie}{terugkwam zei ik}\\

\haiku{en als ik ze weer, ',?}{snap neem ik ze mee naart}{bureau versta je}\\

\haiku{'t Heele Hol ruikt -.}{naar gehak maar wij blijve}{d'r nuchtere van}\\

\haiku{Iedereen had in {\textquoteleft}{\textquoteright}.}{datGemeenebest zoo'n soort}{van liefhebberij}\\

\haiku{All\'e\'en is toch ook,....}{maar all\'e\'en in gezelschap}{geniet je dubbel}\\

\haiku{- Als je soms dorst krijgt,,!}{steek je zoo'n partje in je}{mond dat frischt zoo op}\\

\haiku{Zou je denke, dat.}{ik voor juillie plezier hier}{sta te blauwbekke}\\

\haiku{'t Was me, alsof.}{ik een engelenstem uit}{den Hemel hoorde}\\

\haiku{Piet was destijds nog, - '?}{al erg hoog op de beenen geef}{jen zoet slokkie}\\

\haiku{of denk jij soms, dat ',?}{k voor jou pet mijn riem zal}{weggooie mooie jonge}\\

\haiku{maar toen zaten we.}{heel deftig in het eerste}{Amphitheater}\\

\haiku{{\textquoteleft}Wandel met den Heer{\textquoteright} -,?}{ik wandel liever met een}{knappe meid snap je}\\

\haiku{De bezoeker wipt ':}{t restantje uit zijn glas}{naar binnen en vraagt}\\

\haiku{- Ga eerst de cente,...}{hale ze hebbe vandaag}{zeker veel verkocht}\\

\haiku{Jetje Meloen slaat hard:}{en vinnig met haar stok op}{de zinken toonbank}\\

\haiku{... scheldt de meid, en zich:}{knorrig omdraaiend zegt ze}{luid tot haar vriendin}\\

\haiku{Mokkie is een van.}{de typigste figuren}{uit de Duvelshoek}\\

\haiku{Schouderophalend,:}{ziet hij den ander lachend}{aan en zegt kalmpjes}\\

\haiku{Keizersgragt No. 380,, -:}{geeft niet aan de deur w\`el aan}{blindemanne of}\\

\haiku{Doch 't gebeurde,.}{meermalen dat hij er geen}{herberg kon vinden}\\

\haiku{- Dat is een cadeau,.}{van een Italiaander die}{hier loosjies heit gehad}\\

\haiku{'t Is een wenk, een.}{waarschuwing ten nutte van}{mijn collegas}\\

\haiku{daar zat de kruier.}{al op me te wachten met}{een nijdig gezicht}\\

\haiku{- Dat waren zoo de,.}{gewone scheldnamen die}{je naar je hoofd kreeg}\\

\haiku{'t Zat er aan bij;}{die lui en alles was van}{de bovenste plank}\\

\haiku{- Gerrit zouen we?}{er niet een halfie ouwe}{klare op gieten}\\

\haiku{Gerrit, neem dat kind, '!}{die nare steek toch aft}{is God verzoeken}\\

\haiku{Ik werd er compleet,.}{ziek van want ik maakte me}{voortdurend driftig}\\

\haiku{- kleed je aan en zet,;}{je pet op we gaan van avond}{naar de komedie}\\

\haiku{- Karel de Groote met, '!}{een pijp en lezendt was}{waarachtig komisch}\\

\haiku{Nu is Veltman 83.}{jaar en treedt sedert eenige}{jaren niet meer op}\\

\haiku{zijn bewegingen.}{waren volstrekt niet die van}{een hoogbejaarde}\\

\haiku{- Veltman, z\'o\'o moet    ';}{Veltman als Gijsbreght.jijt}{ook trachten te doen}\\

\haiku{We bleven trouw bij.}{mekaar en dit hinderde}{Duport geweldig}\\

\haiku{Het kamertje van.}{een waschmeisje{\textquoteright}.man met}{geen gezicht meer zien}\\

\haiku{'t Was {\textquoteleft}de uitgang{\textquoteright}.}{voor de Amsterdammers en}{voor de buitenlui}\\

\haiku{Dat het publiek bij,. '}{z'n entr\'ee vertering kreeg}{was dus bepaald noodig}\\

\haiku{Hij is feitelijk,.}{dood zoodra hij niet meer}{op de planken staat}\\

\haiku{we binne vijftien,.}{jaar getrouwd en ze heit nooit}{getwijfeld meheer}\\

\haiku{Ik wou uwe vrage,, ',}{zou je niet denke dat ik}{t best doe er mee}\\

\haiku{- Och, schei uit met je.... -!}{moffinnen gezeur Ick doe}{toch alles voor jou}\\

\haiku{zij liep terug en,.}{zette een vervaarlijke}{keel op hulp roepend}\\

\haiku{ze verdwijnen niet,.}{doch verplaatsen zich als men}{hun nesten verstoord}\\

\subsection{Uit: Van allerlei slag}

\haiku{Grootvader vraagt, of;}{u eventjes een knoop aan zijn}{overjas wil zetten}\\

\haiku{Haar gezicht, misschien,,.}{eenmaal schoon is thans mager}{vervallen en beenig}\\

\haiku{Maar, alle gekheid,,;}{op een stokje vertel me}{eens wat er aan scheelt}\\

\haiku{'k heb 't destijds, '.}{willen doen maart is bij}{willen gebleven}\\

\haiku{en neem je 't niet,.}{dan beschouw ik de vriendschap}{tusschen ons als uit}\\

\haiku{Is 't niet een traan,,:}{die hij steelswijze afwischt}{terwijl hij antwoordt}\\

\haiku{{\textquoteright} {\textquoteleft}Dat zeit grootvader,{\textquoteright} -.}{ook en opnieuw begint de}{kleine te hoesten}\\

\haiku{{\textquoteleft}'k Hoop morgen na!}{de koffie een visite}{te komen maken}\\

\haiku{Jongens, Wilders, hij,;}{is zoo wel-off zooals de}{Engelschen zeggen}\\

\haiku{Is 't geen schande,?}{zoo'n ouwen stakkerd aan zijn}{lot over te laten}\\

\haiku{Ze is zooveel als - -! - '?}{president van die nou hm}{hoe heett ook weer}\\

\haiku{ge geeft alleen de.}{bagatel van 20 centen}{of twie dubbelkens}\\

\haiku{Pas nu maar op de,.}{kleintjes dan blijft er wat over}{voor den ouden dag}\\

\haiku{Ze maken geloof,,{\textquoteright}.}{ik pret over ons oudjes voegt}{hij er zachtjes bij}\\

\haiku{{\textquoteright} Mina en Bertha zien.}{eerst elkander en daarna}{Sarah vragend aan}\\

\haiku{onwillekeurig,:}{sloeg zij de oogen neder en}{toen hij zachtkens vroeg}\\

\haiku{'k laat hem in den,.}{brand zitten totdat we hier}{in de kamer zijn}\\

\haiku{Kom hier, laat ik je,.}{nog eens extra bedanken}{en jou ook Mientje}\\

\haiku{{\textquoteright} {\textquoteleft}Wel, uw maar sluit mijns,;}{inziens in dat u niet tot}{kiezen kan komen}\\

\haiku{Ze voelde iets, dat,:}{als een steek door haar hart ging}{toen hij vervolgde}\\

\haiku{{\textquoteleft}Houd een oogje in ',{\textquoteright};}{t zeil sprak ze bij't afscheid}{nemen tot Mina}\\

\haiku{{\textquoteright} {\textquoteleft}Nu, goed dan, Sarah,,.}{ga mee naar boven maar loop}{zachtjes op de trap}\\

\haiku{{\textquoteleft}Dat 's bijna net,,.}{zooals den vorigen keer maar}{toch veel veel mooier}\\

\haiku{maar ik kan 't niet,,:}{helpen dat ik lachen moet}{als ik om haar denk}\\

\haiku{Tante Saar is in;}{je oom Scheffler's huishouden}{totaal opgegaan}\\

\haiku{Gunst nog toe, ik ben.}{heelemaal in de war door}{al die drukte hier}\\

\haiku{De kinderen? 't;}{Zijn immers bijna allen}{volwassen menschen}\\

\haiku{zij, de oudste, had, '.}{de jongsten grootgebracht en}{t was goed gegaan}\\

\haiku{Wou je nu nog de,!}{oude jongejuffrouw gaan}{spelen malle Saar}\\

\haiku{Toen keerde Scheffler:}{zich plotseling om en vroeg}{met ernstige stem}\\

\haiku{ik dank je, Adriaan,.}{dat je me z\'o\'o lang bij je}{hebt willen houden}\\

\haiku{Waarom ga je niet,.}{aan de Station daar is}{meer te verdienen}\\

\haiku{Is den Dam me toch,.}{lief geworden omdat hij}{voor mij alles is}\\

\haiku{Nou hoef je niks niet, '.}{meer te vragen w\'a\'ar zes}{nachts geweest bennen}\\

\haiku{Veel bennen er bij,,!}{die je door een ringetje}{kunt halen keurig}\\

\haiku{Isa\"ak is dol op een,:}{ouwe jas maar die pakjes}{geeft hij je cadeau}\\

\haiku{Dan slaapt Isa\"ak als een,,.}{roos aan \'e\'en stuk door totdat}{hij er weer uit moet}\\

\haiku{'t is alsof het,.}{geen geld kost want ze vegen}{er de straat mede}\\

\haiku{Moosie smeert ze in, en.}{mijn jongste dochter past op}{den kleinen Sampie}\\

\haiku{Er klonk iets in die,;}{stem dat mij sympathetisch}{voor dien man stemde}\\

\haiku{hij keek strak voor zich, ',.}{ent scheen mij toe dat zijn}{onderlip trilde}\\

\haiku{maar - we waren aan.}{den dijk bij de zwemschool en}{ik moest uitstappen}\\

\haiku{op de hoogte van.}{Madera is hij over de}{fokkeschoot gegaan}\\

\haiku{Liesbeth wist niet wat ';}{ze mij doen zou van plezier}{datk weer thuis was}\\

\haiku{In die drie jaren,, '.}{dat je weg waart heb ikt}{dikwijls opgemerkt}\\

\haiku{{\textquoteleft}Ik w\`el, - ze houdt niet,.}{van hem ze houdt van geen een}{van de jongens hier}\\

\haiku{ben je gek, Tijs, - zei, -?}{ik tot mezelf zoo'n jonge}{blom zou jou nemen}\\

\haiku{Hier zweeg de man en.}{wischte zich een traan van de}{verweerde wangen}\\

\haiku{{\textquoteleft}Dat ziet er goed uit,...{\textquoteright}.}{goed uit als hij de dames}{in baltoilet ziet}\\

\haiku{de tanden buigt de,:}{candidaat-notaris}{terwijl hij vervolgt}\\

\haiku{{\textquoteright} Bevallig dankend.}{laat Marie zich door Emmer}{naar haar plaats brengen}\\

\haiku{maar daaraan wen je.}{al net zoo gauw als aan de}{lucht van de klare}\\

\haiku{Zoo'n slijterij is, - ',!}{geen broodwinningt is een}{goudwinning meneer}\\

\haiku{{\textquoteright} {\textquoteleft}Of 't! 'k Heb 't, ' '}{dikwijls genoeg tegen m'n}{vrouw gezegd alsk}\\

\haiku{Salomons wijsheid,.}{ten tweede Jobs geduld en}{ten derde Simsons kracht}\\

\haiku{die kwam driemaal in,:}{de week met een flesch daar een}{\'etiquetje op stond}\\

\haiku{- Eens op 'n keer, dat,,.}{ik ziek was deed mijn maat er}{in wat er op stond}\\

\haiku{daarom wou ik haar.}{op dien middag'n paar centen}{in de hand drukken}\\

\haiku{hij zag er uit als.}{de geletterde dood en}{hij kreunde van pijn}\\

\haiku{Ik merkte al gauw,.}{dat de stumperd stijf van de}{rheumatiek was}\\

\haiku{Op eens maakte 't,:}{kind zich van mij los ging vlak}{voor me staan en zei}\\

\haiku{t past mij niet om!}{iets van de klanten van mijn}{patroon te zeggen}\\

\haiku{Nou zal je reis wat,{\textquoteright}, {\textquoteleft}!}{hooren vervolgt hij maar een}{barschHou je mond}\\

\haiku{Met een scherpen gil '.}{eindigt de klarinet en}{t scherm gaat weer op}\\

\haiku{Wij verlaten de.}{keuken en zien in de zaal}{het laatste bedrijf}\\

\section{Pieter van der Meer de Walcheren}

\subsection{Uit: Mijn dagboek. Dagboek 1. 1907-1911}

\haiku{O onvindbare,,!}{onzichtbare Vader heb}{deernis met mij}\\

\haiku{Zijn wil en Zijn plan.}{worden openbaar en liggen}{bloot als een zee}\\

\haiku{Ik sta aan den rand.}{van het lichte geheim en}{buig mij stil voorover}\\

\haiku{Die dag brak aan als.}{alle andere dagen}{van alle weken}\\

\haiku{die voor vrijheid en;}{menselijkheid met Christus}{hun leven wagen}\\

\haiku{Ik voel mij in hem,,,,.}{mijn medemens vernederd}{bezoedeld verlaagd}\\

\haiku{Deze beeldhouwer.}{is kloek en rustig en vol}{helderen eenvoud}\\

\haiku{Ik begrijp zelf al.}{niet goed wat ik ben komen}{doen in het leven}\\

\haiku{- Laat ik uitscheiden,.}{ik voel dat ik mijn schone}{vreugde storen ga}\\

\haiku{Welk een afschuw had,}{zij van die tamme mensjes}{wier hoogste ideaal}\\

\haiku{Wij moesten elkander,, -?}{vinden wij werden geleid}{door een blind toeval}\\

\haiku{In de enkele}{dorpen aan den groten weg}{gelegen welken}\\

\haiku{Des nachts om \'e\'en uur,,.}{na het diner bij H. een}{jeugdig geleerde}\\

\haiku{wij vormden de kern.}{van een leger waarvan hij}{de aanvoerder was}\\

\haiku{De boulevard was,.}{verlaten slechts af en toe}{reed een rijtuig langs}\\

\haiku{Zij was zijn bode,,;}{zijn sekretaresse zijn}{vertrouwelinge}\\

\haiku{een vijfde zoekt het,;}{oneindige in diepste}{gemeenste zonden}\\

\haiku{Blijde levenskracht,,.}{doorgloeit mij ik voel mij sterk}{ik ben welgemoed}\\

\haiku{bij het zien van den,,.}{sterrenhemel van een bloem}{van gans het leven}\\

\haiku{- voelt uw verbeelding?}{den grondelozen afgrond}{naar alle kanten}\\

\haiku{Het weder blijft zacht;}{en zonnig als bij ons in}{een schone lente}\\

\haiku{Wanneer de grote,.}{zon over hen heen straalt dan is}{de aarde een feest}\\

\haiku{Ik wilde wel dat,,.}{door een wonder zij genas}{opeens volkomen}\\

\haiku{Het leven is een,.}{kalme bezigheid is een}{doodgewone zaak}\\

\haiku{Schuw spoedden zij zich,,.}{langs de huizen of zochten}{een rijtuig een taxi}\\

\haiku{De Melkweg boogt zijn.}{ontzagwekkende baan door}{den besterden nacht}\\

\haiku{Wij zijn op orde.}{en het geregeld leven}{gaat weer beginnen}\\

\haiku{En het schouwspel van.}{het leven is werkelijk}{niet zielverheffend}\\

\haiku{- {\textquoteleft}Hij bederft zelf zijn,{\textquoteright}, - {\textquoteleft}.}{succes zeide mij iemand}{hij mist soepelheid}\\

\haiku{Welk een vrome ziel!}{vol heimwee bezat deze}{grote muzikant}\\

\haiku{Die Durand is het.}{type van den mislukten}{intellektueel}\\

\haiku{En ik verheug mij.}{grotelijks op de lezing}{dezer boeken}\\

\haiku{In plaats van een rok,;}{hing een vuile blauwe lap}{om haar naakt lichaam}\\

\haiku{Hetzelfde drukke,.}{leven van mensen-zien}{werken en uitgaan}\\

\haiku{ontzagwekkend zijt.}{Ge en ik honger naar Uw}{tegenwoordigheid}\\

\haiku{laat door een wel schoon,,.}{maar geen werkelijkheid in}{zich dragend waanbeeld}\\

\haiku{de oneindige.}{wereld van den geest zal u}{geopend worden}\\

\haiku{ik leg haar in de;}{holte van Uw Hand. Zij is}{vuil en vol modder}\\

\haiku{Ik heb de macht van;}{de gewijde vingers van}{den priester gevoeld}\\

\haiku{Wij volgen Jezus,,.}{schrede na schrede op zijn}{smartelijken tocht}\\

\haiku{Hij herinnerde.}{zich mijn bezoeken van den}{vorigen winter}\\

\haiku{En toen ineens heb,}{ik het gevraagd ik had het}{uitgesproken v\'o\'or}\\

\haiku{Later, later zal.}{ik wel eens die donkere}{dagen verhalen}\\

\haiku{De rest liet ik, eerst,;}{toen ik heiden was over aan}{het onverwachte}\\

\haiku{En ik vroeg mij af:}{in die tijdloze uren van}{de stilte van God}\\

\haiku{God kan alles, - nu...}{Hij het klaar speelt van mij een}{priester te maken}\\

\section{J.H. Meerwaldt}

\subsection{Uit: 'Pid\'ari'. Of de strijd van het licht tegen de duisternis in de Bataklanden}

\haiku{Hij staat met den rug,.}{naar het huis toe dat aan Si}{Panggoe toebehoort}\\

\haiku{Den avond van dien dag.}{bezoeken wij het huis van}{Si Panggoe nogmaals}\\

\haiku{Bij de haardsteden.}{heeft de engel des slaaps reeds}{zijn werk begonnen}\\

\haiku{Met \'e\'en sprong is Si}{Panggoe op de been en ijlt}{zijn woning binnen}\\

\haiku{Terwijl hij naar zijn,}{luid schreeuwende strijdmakkers}{omkeek om te zien}\\

\haiku{De geslepen schelm.}{heeft na een paar dagen een}{uitweg gevonden}\\

\haiku{U behooren zij.}{toe en gij alleen hebt over}{hen te beschikken}\\

\haiku{Dat is waarlijk een.}{groot bewijs uwer liefde en}{goedheid jegens mij}\\

\haiku{{\textquoteleft}Als wij het kind eens!}{naar den zendeling te Pansoer}{na pitoe brachten}\\

\haiku{zoo meer tot hem zijn,.}{toevlucht in de velerlei}{nooden die het drukte}\\

\haiku{Nog even ootmoedig.}{als vroeger achtte hij dat}{ambt voor zich te hoog}\\

\haiku{De Heer zette hem.}{daartoe ook nog verder tot}{een rijken zegen}\\

\haiku{De woede van het.}{wilde volk werd hoe langer}{zoo meer opgewekt}\\

\haiku{Van uit de verte.}{rommelde reeds de donder}{grommend door het woud}\\

\haiku{{\textquoteleft}Waarom zoo haastig,,.}{vriend het onwe\^er kunt ge toch}{niet meer ontkomen}\\

\haiku{Eindelijk raakte}{de maat van Panggalam\"ei's}{euveldaden vol.}\\

\haiku{Er heerschte een.}{droeve bedrijvigheid in}{Loboe Hamindjon}\\

\haiku{Zijn geest moest er voor,.}{beloond worden dat hij in}{hem gebleven was}\\

\haiku{Op die wijze werd.}{dus eenvoudig twist gezocht}{tegen Ama Laboe}\\

\haiku{als hij maar naar de.}{gouvernementsschool wilde}{gaan en vlijtig leeren}\\

\haiku{{\textquoteright} - {\textquoteleft}Zeker, mijnheer, ik,{\textquoteright}.}{ben werkelijk geen Maleier}{luidde het antwoord}\\

\haiku{Ten slotte heeft ook.}{een brand nog een gedeelte}{van het dorp verwoest}\\

\section{Johan de Meester}

\subsection{Uit: Geertje (2 delen)}

\haiku{ach kijk, spatten op ', '.}{er mouw bij de pols en een}{groote vlok oper rok}\\

\haiku{- Ga nu weer zitten,,.}{de tijd is kort hoorde ze}{Grootvader zeggen}\\

\haiku{Nou hei je toch wat.... -,,.}{Heel vriendelijk van je hoor}{Lina zei Meester}\\

\haiku{Net of zij niet heel,!}{goed wist dat die eigenlijk}{op haar neer zagen}\\

\haiku{Die had er ook gauw,.}{een vrouw gekregen een mooie}{rijke Duitsche vrouw}\\

\haiku{En almaar, almaar -....}{vloog de trein en ze was er}{zeker nog lang niet}\\

\haiku{Hij ging enkel maar,.}{naar Utrecht maar tot zoover konden}{ze samen reizen}\\

\haiku{Wie zingt niet mee op, '....}{dee-ee-zen dag Vant}{Lindenhout ter eer}\\

\haiku{Het was aan de Schie,,,.}{een korte zijstraat als een}{l\^a zoo hol hoog-recht}\\

\haiku{- Zie je, daar kwam te,. '}{veel konkerensie hier heb}{ik het rijk alleen}\\

\haiku{Altijd treurde hij, ':}{over die dingen Groo'moe had}{noges eens gezeid}\\

\haiku{Tante zei niets meer, '.}{als aarzelend sjokte ze}{t keukentje in}\\

\haiku{vijand was ze dus,,....}{met Tante en ze had nog}{geen dienst ze had niks}\\

\haiku{Maar een duimvlak op,.}{de brief zei dat Tante hem}{had gelezen}\\

\haiku{- als die het wist van,....}{Oom z'en afval \'en hoe de}{zaak verloopen was}\\

\haiku{Groo'va was altoos,.}{dage lang van streek als u}{om geld had gevraagd}\\

\haiku{mit j' eens, da's geen.}{zaak om je grootvader mee}{an boord te komme}\\

\haiku{medeplichtig, door,;}{te zwijgen aan zijn verraad}{tegenover Groo'va}\\

\haiku{Heef u er nog over,?}{gesproke tegen mevrouw}{Gobius Tante}\\

\haiku{die naam k\`omp hier niet!.}{te p\`as galmde Maandag en}{lachte het eerste}\\

\haiku{- Hai je da' gehoord,,?}{Riek wat Gobius mit de}{koster gehad he't}\\

\haiku{- {\textquoteleft}Zoo, zit jij maar weer,'?}{te leze wat hei je nou}{uit de kas gehaald}\\

\haiku{Het was de eenige,.}{keer dat Oom van de nieuwe}{krant had gesproken}\\

\haiku{In de winkell\^a.}{lagen drie dagen lang vier}{en dertig centen}\\

\haiku{Wat een rijkdom, wat,.}{vreemde ramen maar je kon}{niks naar binnen zien}\\

\haiku{- Geer, morrege ga,....}{je met mijn uit dan mag je}{mee na Heins z'en huis}\\

\haiku{Hu, ze werd er haast,:}{duizelig van toen ze pal}{naar beneden keek}\\

\haiku{Toch bleef 't hoofdje,, '.}{branderig hoogrood van kleur}{ent mondje droog}\\

\haiku{Wacht, ze zou de deur,.}{maar toedoen Truus mocht is te}{veel overend komme}\\

\haiku{Nooit kon Meneer wat,!}{doen in huis of de Juffrouw}{had wat te zegge}\\

\haiku{Laa'st, toen Truusje zoo -?}{d\'o\'odziek was kon je wat an}{de Juffrouw merke}\\

\haiku{Wat had het anders,,.}{prettig kunnen zijn van avond}{weer daar op het plat}\\

\haiku{Zooas ie bevoorbeeld '!}{overet huwelijk en de}{liefde kon prate}\\

\haiku{En die moeder is -....}{zijn zuster en hij houdt van}{de Maagd Maria}\\

\haiku{Schuw keek ze naar Oom, ',;}{z'en glast hoeveelste was}{dit al ze wist niet}\\

\haiku{Tante scheen 't nie',....}{naar te vinde anders zou}{ze niet z\'o\'o lang staan}\\

\haiku{Ze lag op de rug,,.}{het hoofd half op zij ze kon}{Tante zoo niet zien}\\

\haiku{En zag, vol angst voor,.}{nieuwe schande Tante in}{de deur verschijnen}\\

\haiku{onder het wasschen.}{zou ze de tranen weten}{in te houden}\\

\haiku{Wat zou d'er zijn, dat,,?}{de feeks d'er was nou al en}{nog wel op Zondag}\\

\haiku{Zij deed het, zonder,:}{zich verder rekenschap te}{geven vrijmoedig}\\

\haiku{Wat had dat-er, '!}{mee te maken dat zet}{prettig had hier thuis}\\

\haiku{- Wat hei'j de kinders',.}{blij gemaak hoorde ze zijn}{vriendelijke stem}\\

\haiku{En kijk daar us, hoe,.}{aardig die twee moeders met}{al dat kleine grut}\\

\haiku{zag hij er weer {\textquoteleft}wel '{\textquoteright},.}{asen heer uit bekende}{Geertje zichzelve}\\

\haiku{Als ze haar liever, '.}{niet meer hadden moesten zet}{maar ronduit zeggen}\\

\haiku{Wij gaan alleen bij.}{Domenee Gobius en}{Domenee De Valk}\\

\haiku{O jee, komp Arie je! -.}{hale Hij had-t-er al}{wel kunne weze}\\

\haiku{die armen, die als,,.}{klauwen waren dat beestesnoet}{dat op haar toedrong}\\

\haiku{Zij hield de oogen nog,.}{gesloten tranen welden}{de oogleden door}\\

\haiku{Doch zoodra hij,.}{kwam te spreken was er voor}{Geertje slechts zijn stem}\\

\haiku{Wat had ze ook voor,;}{slaap gehad zoo kort en dan}{in die benauwdheid}\\

\haiku{V. Toen zij na de,.}{kerk thuis aanbelde trok de}{juffrouw aan het koord}\\

\haiku{het was of zij de,;}{handdruk na-voelde in}{haar borst door haar lijf}\\

\haiku{and're jongens,;}{bleven staan meenende dat}{er wat gebeurde}\\

\haiku{Ze wou wel graag naar,.}{de middagkerk maar dan moest}{ze eerst nog thuis zijn}\\

\haiku{O, die z\'alige,,;}{lucht van et hout toch na de}{nacht et elzenhout}\\

\haiku{Dat was zoolang as -!}{Geertje heugde en n\'o\'oit was}{er wat an gedaan}\\

\haiku{Bij het thuiskomen;}{van de wandeling vond ze}{Wouter Heukelman}\\

\haiku{Als instinktmatig.}{loenschte Geertje even naar}{de kant van Groo'va}\\

\haiku{De larikse, de, ',;}{konifeeret glansde}{zilver op et groen}\\

\haiku{{\textquoteleft}Mijn liefste is blank,.}{en rood hij draagt de banier}{boven tien duizend}\\

\haiku{Vele wateren:}{zouden deze liefde niet}{kunnen uitblusschen}\\

\haiku{Nicht was er al van ',.}{s morgens af Rika zou}{niet zijn gekomen}\\

\haiku{De jongens zouden,.}{straks wel komen zij waren}{nu nog aan het werk}\\

\haiku{- Dee je niet beter,,?}{Geertjen as je nou bij}{je grootmoeder bleef}\\

\haiku{leege kamer er nog.}{ongezelliger uit dan}{eerst in de schemer}\\

\haiku{Jan had het over Gijs,,,.}{hun knecht die nu getrouwd was}{maar niet gelukkig}\\

\haiku{Koos naar school en Moe,,!}{boven in bed o het was}{zoo heerlijk zitten}\\

\haiku{'t Was of er oogen,....}{al grooter werden of die}{hem omvatten moesten}\\

\haiku{ze wist et niet meer,,,!....}{ze had een ijskoud hoofd dat}{leeg was ze wier g\`ek}\\

\haiku{- H\`e ja, en nou met.... '.}{zoo'n regent Kind keek haar}{aan en zij het kind}\\

\haiku{Radeloos trok ze;}{de schouders op om de bloote}{nek te beschermen}\\

\haiku{Gotogot en d'er,,!}{was toch niks ze weerde hem}{af zooveel ze kon}\\

\haiku{Hij mocht de Juffrouw, '.}{niet verlatet zou de}{grootste zonde zijn}\\

\haiku{{\textquoteright} Maar ze kon niet, ze,....}{m\`oest Tante spreken over die}{boodschap van Groo'moe}\\

\haiku{Sophie, met 'en -!}{schoone japon an die wist}{ook wel wat ze dee}\\

\haiku{maar opeens, daar was! ',;}{de werkvrouwt mensch w\`as daar}{al de deur al toe}\\

\haiku{'En drukkie op d'er,....}{lijf of ze de vrouw is van}{de burgemeester}\\

\haiku{Ze was geen kind meer,,.}{ze wist toch wel wat het was}{getrouwd-zijn}\\

\haiku{Tusschen hem en zijn,,;}{vrouw was het uit was er n{\`\i}ets}{meer sedert lang al}\\

\haiku{Zij, alleen-wakker,.}{als om af te luisteren}{dat allen sliepen}\\

\haiku{Zij voelde dat zij,,;}{hevig kleurde want ja ze}{had aan Hem gedacht}\\

\haiku{Hij was nog uit - stond}{ze er in haar kamertje}{lang mee in de hand.}\\

\haiku{Maar de kinderen,,.}{bleven  slapen heel het}{huis sliep of was stil}\\

\haiku{t Zou zijn, als kon,.}{het haar niet schelen of hij}{nog meer geld uitgaf}\\

\haiku{- Heerejee, nou ben ' ''.}{k weer te stil en strakjes}{mochk nie prate}\\

\haiku{En net zoo wat op....}{etzelfde mement was goeie}{Groo'moe gestorven}\\

\haiku{Ze gaf nu nog om,.}{Hem alleen om het geluk}{dat Hij bij haar vond}\\

\haiku{Niets op aarde was.}{haar iets waard meer bij wat zij}{zich nu bewust werd}\\

\haiku{blij keek ze neer op,:}{de ontbijtboel die nu gauw}{het eerst aan kant moest}\\

\haiku{Wel aan haar gezicht,,,,;}{haar kleeding ook aan haar figuur}{heur haar heur handen}\\

\haiku{Waarom 'en weekblad? - ''.}{k Dach dat Heins je d'er een}{had meegegeve}\\

\haiku{Ze wilde 't graag,,.}{uit medelijden doch dorst}{het niet goed vragen}\\

\haiku{Geertje nam nu dus, '.}{maar afscheid buurvrouw zout}{verdere wel doen}\\

\haiku{Zondag had ze kou,,.}{gevat met dat lange}{verschoonen boven}\\

\haiku{Zoo aak'lig als 't,,.}{was zoo laag en zoo somber}{toch had zij het lief}\\

\haiku{Ze zou'en het nu wel, {\textquoteleft}.}{niet meer gel\'o\'oven dat zelaa'st}{wat kou gevat had}\\

\haiku{Zij had haar lange.}{uitgaansavond en om vijf uur}{trok ze de deur dicht}\\

\haiku{maar zou Oom dan niks,?....}{verdienen wat deed hij dan}{toch aldoor op straat}\\

\haiku{- en thuis moest d'er nog;}{z\'o\'oveel gedaan voor de groote}{partij van vanavond}\\

\haiku{{\textquoteleft}'t Hijgend hert, der{\textquoteright} -....}{jacht ontkomen dat ze zelf}{zoo'o hijgend hert was}\\

\haiku{{\textquoteleft}Hou je dan heelem\'a\'al',?}{nie meer van me wil je nou}{volstrekt m'en verderf}\\

\haiku{als het kon in zijn,.}{eigen dienst dat ze misschien}{met Oom kon deelen}\\

\haiku{Maar norsch liep het.}{mensch langs haar heen zonder eenig}{antwoord te geven}\\

\haiku{Krabben wou ze, dat, ',.}{\`andre menscht slet dat ze}{h\'a\'ast m\'e\'er nog haatte}\\

\haiku{En die vent die z'en, ':}{centen op-zijn met drie}{juffers omem heen}\\

\haiku{H\`e, die gevel van ',!}{t Loterijketoor net}{een brief met rouwrand}\\

\haiku{En dat netstille, '?}{van de winkels w\'a\'arom maakte}{t haar zoo angstig}\\

\haiku{O, h\`e, 't gaf haar,.}{een verruiming die joodsche}{slagerij nog open}\\

\haiku{Waar\`om zou ze vand\'a\'ag juist,!}{bevallen z\'o\'oveel maanden}{en maanden te vroeg}\\

\haiku{Toen die heer, die haar,,,....}{z\'o\'o maar aansprak lachend met}{knipoogjes gemeen}\\

\haiku{Toch - soms z\`onk ze weg,,:}{als in ijs kreeg ze een drang}{om weg te h\`ollen}\\

\haiku{- Ge mee, hieraufer,....}{dan drinke me-n-en kep}{keffie en prate}\\

\haiku{Ze dacht, huiv'rend, aan -,'....}{de middag zij daar tot schand}{van een heele buurt}\\

\haiku{hij, in een willen,;}{woedend blijven woedend van}{verontwaardiging}\\

\haiku{mejelij mot je,,.}{met ter hebben mejelij}{niks as mejelij}\\

\haiku{Driftig schokkend met,,,.}{de schouders liep ze om te}{kunnen zwijgen weg}\\

\haiku{En ze knielde v\'o\'or,,....}{de koffer prikte met het}{sleuteltje draaide}\\

\haiku{Oom vertelde van,.}{Cohen dat die zijn neef pas}{weer had belazerd}\\

\haiku{Als ze stil op zoo'n,.}{kamer woonde zou hij wel}{weer durven komen}\\

\haiku{voordat hij met zijn,.}{stoel had geschoven was z{\`\i}j}{uit haar leunstoel op}\\

\haiku{V\'o\'or 'um te staan, z'en,.}{stem te hoore dat z'en hoofd}{weer boog naar haar over}\\

\haiku{As die de brief in,,'.}{hande kreeg tien tegen een}{da jij um nooit zag}\\

\haiku{maar hij moest het doen,,.}{om thuis voor de kinders want}{anders geen leven}\\

\haiku{{\textquoteleft}Zou mijn aangezicht?}{moeten medegaan om u}{gerust te stellen}\\

\haiku{zoo mag je niet doen, '.}{je ratelt ze af als een}{roomschet latijn}\\

\haiku{Vele wateren....}{zouden deze liefde niet}{kunnen uitblusschen}\\

\haiku{Zij zou er toch ook,;}{geen kunnen krijgen want de}{planken lagen leeg}\\

\haiku{Maar Donderdags was,:}{ze weer zooveel beter dat}{hij had bevolen}\\

\haiku{Ze zei zich, dat ze,.}{niet verlangde dat haar angst}{te hevig was}\\

\haiku{De ontnuchtering,.}{die in haar viel maakte haar}{weer plotseling moe}\\

\haiku{Maar Geertje keek sta\^ag,,.}{lusteloos zij zat er bij}{suf-onverschillig}\\

\haiku{Heele weke gaan,.}{d'er voorbij da'k alleen met}{Cohen te doen heb}\\

\haiku{Weer nam zij deel aan,,.}{den dienst van God in het Huis}{van God weer mocht zij}\\

\haiku{thuis gevoeld had ze,.}{zich in beide als nooit een}{oogenblik bij Oom}\\

\haiku{Zenuwachtig, wist,;}{ze zelf niet waar ze naar toe}{zou gaan en waar niet}\\

\haiku{Strompelend kwam hij.}{het trapjen af en bleef}{schuins v\'o\'or Geertje staan}\\

\haiku{Met de oogen toe had,:}{ze aan tafel gezeten}{tot Tante zelf zei}\\

\haiku{Wat het slet dan toch '!}{wel dachtt Geld lag zeker}{te grabbel op straat}\\

\haiku{Zoo pijnigde het,:}{denken haar telkens wanneer}{ze besloten was}\\

\haiku{Als een priemsteek kwam:}{dan de gedachte aan de}{brief in haar koffer}\\

\haiku{Ze moest blijven in?}{Jan zijn stad en wie had ze}{hier anders dan Oom}\\

\haiku{Geertje wist, dat zij,.}{nu het moest zeggen doch zij}{vond de woorden niet}\\

\haiku{Jasses, om zoo ie's, '.}{gemeens te denken en dat}{enkel vanen grap}\\

\haiku{Nu, Maandag zou dan,.}{op de hoek van de straat aan}{de Schie blijven staan}\\

\haiku{Om toch geen gerucht,.}{te maken bleef ze in haar}{gebogen houding}\\

\haiku{de laatste woorden.}{van Groo'moe prevelden een}{gebed voor jou}\\

\haiku{daar nu scharrelen.}{moeten om bij de lage}{kapstok te komen}\\

\haiku{om Geertje nau bai,,....}{u thuys te hebbe \`as et}{kan as Geertje will}\\

\haiku{- Och ik kan hier toch',!}{z\'o\'o nie weg de bedde ben}{nog niet eens aan kant}\\

\haiku{Hij, hij, Groo'va, net, ', '.}{als Tanteen slet was ze}{in hun oogenen slet}\\

\haiku{dat, nu zij niet schreef,.}{aan Jan God hem misschien zou}{neigen tot schrijven}\\

\haiku{Hij ijlde naar zijn,,;}{bedstee trok er een deken}{uit nam het kussen}\\

\haiku{{\textquoteright} 't Was m\'ogelijk,.}{dat ze enkel van die schrik}{was flauw gevallen}\\

\haiku{nee, duisternis, nee,, '}{zij was niet meer ziek zij lag}{hier iner bedste\^e}\\

\haiku{Bij Maandag kon niet,,,?}{ze wou niet bij Oom dus toch}{weg met Groo'va mee}\\

\haiku{Waarom dringt Maandag,?}{toch aldoor m\'e\'e aan dat ze}{Willen nemen zal}\\

\haiku{Bij menschen spreekt en,.}{lacht zij mee houdt de schijn op}{van jonge Geertje}\\

\haiku{- {\textquoteleft}Vindt je dat dan niet,?}{mooi van die jongen dat hij}{z\'o\'oveel van je houdt}\\

\haiku{Eerst d'er ziekte, toen,,, '.}{de miskraam en veel werken}{nu dater zwaar viel}\\

\haiku{Ze stond achter een,,}{hekje in een groep menschen}{die opeens om d'er}\\

\haiku{Zij gaan nu nogmaals.}{samen naar Oom en nog loopt}{hij te zaniken}\\

\haiku{In de menigte:}{nam hij haar arm en toen zij}{die los wou trekken}\\

\haiku{hier aan het eind, hij,.}{weet het wel dan ontmoet hij}{de trem van het Park}\\

\subsection{Uit: De zonde in het deftige dorp}

\haiku{als het, met heel veel}{geduldigen takt van flink}{zijn en kalm je wensch}\\

\haiku{maar het was Stork, of:}{hij opeens een anderen}{kijk op de meid kreeg}\\

\haiku{Emmy vond het toen ',.}{z\'o\'on eer dat zij voor de}{frankeering wou zorgen}\\

\haiku{- Vrinden hebben me,.}{verweten dat ik een vrouw}{met geld heb getrouwd}\\

\haiku{het d\`orp zou hij hem,!}{hebben ontzegd als het nog}{in zijne macht stond}\\

\haiku{vast de voerman uit -.}{de stad dan kreeg men zijn goed}{alweer pas morgen}\\

\haiku{een nooit voorziene,.....}{bloei van geluksvreugd somtijds}{bijna overstelpend}\\

\haiku{En hij aarzelde,,....}{hij was bang terwijl zijn kind}{zijn hulp behoefde}\\

\haiku{Even het geloovig;}{geduld der oude freule}{Sitsen bewonderd}\\

\haiku{Sprekend groeten bleef, {\textquoteleft},{\textquoteright} {\textquoteleft}.}{uitzondering bewaard voor}{DokterDaumenee}\\

\haiku{Het openen van het.}{raam had stagnatie in het}{gesprek veroorzaakt}\\

\haiku{De vader en zijn.}{vriend waren in de warme}{kamer gebleven}\\

\haiku{Joop vond het van Leo,:}{een prachtig idee maar stond er}{wat beteuterd bij}\\

\haiku{Het meisje wipte,,.}{van haar kruk en keek pruilend}{of ze zou huilen}\\

\haiku{Neuri\"end stapte.}{hij door de eenzaamheid van}{zijn kamer terug}\\

\haiku{Hij verzekerde,.}{zich zooveel mogelijk te}{hebben afgeschud}\\

\haiku{ingebroken op,.}{drie leege buitens maar nergens}{een beurs gevonden}\\

\haiku{- Ik dacht juist dat u.}{met dit weer graag van de hei}{zou zijn weggevlucht}\\

\haiku{- Dan moeten we hem '.}{ookes naar Holland halenl}{grappigde Berkie}\\

\haiku{Bronchitis en ik,.}{weet al niet wat enkel door}{onvoorzichtigheid}\\

\haiku{Het gaf hem eenige,.}{voldoening Dina rust te}{kunnen bezorgen}\\

\haiku{In de andere.}{zaal meende Stork onderdrukt}{gelach te hooren}\\

\haiku{- Schaken ken ik ook,,.}{zei de ander zelfbewust}{maar onverschillig}\\

\haiku{Ach, maar daar hoorde,:}{hij Loesje roepen vlakbij}{op de bovengang}\\

\haiku{Hij bleef naar de deur,.}{gewend staan totdat Wedelaar}{haar had gesloten}\\

\haiku{Ik merk aan je, dat.}{je Dina's verzekering}{niet in twijfel trekt}\\

\haiku{- Arme goeie man, had.}{zij gefluisterd en hem op}{het voorhoofd gekust}\\

\haiku{Wedelaar vouwde de.}{briefblaadjes v\'o\'or zich dicht en}{legde ze bijeen}\\

\haiku{- Dina, altoos grootsch,,.}{bleef stug doen ook terwijl ze}{van Neeltje afhing}\\

\haiku{Daar was moeder voor,,.}{doende geweest Dinsdagsavonds}{ook l\`ang nog in bed}\\

\haiku{nooit hoorde je d\'a\'ar ',.}{vanen motje praten als}{bij d'erlui menschen}\\

\haiku{Doortastend nam zij,, -.}{dezen op blies blies nog eens}{het lichtje was uit}\\

\haiku{Doch zij wilde de -,;}{schuld niet aanvaarden niet uit}{liefde voor Wedelaar}\\

\haiku{Op \'e\'en had zij het,'.}{lang gezet van dat z een}{kind van elf was af}\\

\haiku{Het was het eerste,:}{geweest wat ze met mekaar}{hadden meegevoeld}\\

\haiku{De schrikkelijke.}{tijding had de lieve z\'o\'o}{droef doen ontstellen}\\

\haiku{Toen haar brief kwam met,:}{haars vaders adres achterop}{had hij begrepen}\\

\haiku{{\textquoteleft}want zij was eene maagd,,.}{zoodat het in Amnons oogen zwaar}{was haar iets te doen}\\

\haiku{zij wilden Hugo,.}{beslist nog spreken v\'o\'or dit}{plotseling vertrek}\\

\haiku{Maar nu komt Hendrik,.}{uit het dorp met een vreemde}{treurige boodschap}\\

\haiku{Met een wenk, dat hij,.}{zwijgen zou ging Aleid hem in}{de huiskamer voor}\\

\haiku{Nu was ook Jan van.}{Loodijck opgeschrikt uit zijn}{rustige houding}\\

\haiku{- Zegt u me eerlijk,,;}{Dominee uw zelfverwijt}{brengt me tot die vraag}\\

\haiku{De vraag met een knik,,,:}{beantwoordend bond Hovink}{aangemoedigd aan}\\

\haiku{nou, de meid dan had,....}{verteld dat Dina en het}{medisch studentje}\\

\haiku{VROUW Van Rooien zat,,,.}{verslagen scheef op haar stoel}{gekromd moe hijgend}\\

\haiku{Mijn gaf-t-ie niet,.}{op nou liep-t-ie de meid}{d'er broers achterop}\\

\haiku{Toen zij melk voor Wim, ':}{ging halen had zijt uit}{de verte gezien}\\

\haiku{Het ligt geenszins in,.}{mijn bedoeling pressie op}{hem te oefenen}\\

\haiku{Hij dorst, hij kon niet,.}{alles schrijven juist zooals het}{Papa gegaan was}\\

\haiku{Dus stuurde Papa,....}{het rijtuig-terug toen}{moest het nog wachten}\\

\haiku{Berkie gaf een draai,,.}{aan zijn stoel nam het papier}{en deed of hij las}\\

\haiku{Allen stonden, doch.}{op aandringen van broeder}{Van Loodijck was Ds}\\

\haiku{Krookje was een heel,.}{goed meisje maar men kon niets}{aan haar overlaten}\\

\haiku{al die bezoeken,;}{dikwijls zoo ver. Wedelaar was}{immers niet jong meer}\\

\haiku{toen ze blij van de -.}{huisbel schrikte eindelijk}{zou daar Wedelaar zijn}\\

\haiku{*** Vrouw Van Rooien vroeg,.}{wel freskuus dat zij Mevrouw}{dorst lastig vallen}\\

\haiku{, Leo was twee uren thuis,.}{en nog had hij niet met zijn}{Vader gesproken}\\

\haiku{haastte hij zich tot,.}{den bediende nog voordat}{hij gezeten was}\\

\haiku{Toen bleef hij zitten,.}{het glas in de hand. Stork zag}{tranen in zijn oogen}\\

\haiku{En dien had hij nog ',,!}{es ten eten gevraagd denkend}{dat misschien met Em}\\

\haiku{hij moest altoos de,.}{menschen nog vinden waar zijn}{geld niet welkom was}\\

\haiku{Als opblazend van,:}{voldaan gegrinnik legde}{Hovink aan Stork uit}\\

\haiku{Stork en Claartje van.}{Lakervelde zaten in}{zijn studeerkamer}\\

\haiku{Maar wel had hij sterk:}{de bekoring gevoeld van}{dat enkele ding}\\

\haiku{De jongen was er,.}{zuinig op wat Leo plezier}{deed als blijk van smaak}\\

\haiku{Als dat te lastig,;}{is wil jij het misschien van}{jou boekje nemen}\\

\section{J.T. de Meesters}

\subsection{Uit: Memento mori}

\haiku{gelukkig dat hij.}{zich een oogenblik aan zijn}{studie kon wijden}\\

\haiku{{\textquoteleft}Je weet dat de kans,.}{dan groot is dat ik dwaze}{dingen doe of zeg}\\

\haiku{Maar, zooals ik je zei,.}{probeer nu eerst een beetje}{tot rust te komen}\\

\haiku{Toen zij geen antwoord,.}{kreeg waagde zij het nog niet}{naar binnen te gaan}\\

\haiku{De collega van.}{den specialist liet niet}{lang op zich wachten}\\

\haiku{vond hij toch dat hij.}{de meeste dingen beter}{deed dan anderen}\\

\haiku{Naar Willy's meening!}{in ieder geval veel meer}{dan wenschelijk was}\\

\haiku{Ik ben nog onder,.}{behandeling geweest van}{den dokter haar man}\\

\haiku{{\textquotedblleft}een ziekelijke{\textquotedblright}.}{vrouw is het ergste wat een}{man kan overkomen}\\

\haiku{van den toestand van,.}{zijn passagiere besloot}{hij tot het laatste}\\

\haiku{{\textquoteright} zei Jonkmans streng, het.}{ijzer willende smeden}{terwijl het heet was}\\

\haiku{Ineens verlies ik,.}{hem uit het oog spring op m'n}{fiets en zie niets meer}\\

\haiku{Wij zullen, vriendje,.}{dit zaakje eens fijn met z'n}{twee\"en opknappen}\\

\haiku{Denk je dat ik mijn?}{reputatie door jou wil}{laten bekladden}\\

\haiku{Wat zou het - hebben?}{we niet allemaal een of}{andere manie}\\

\haiku{driftig zou kunnen,!}{worden dat hij je een pak}{slaag gaf dan ril ik}\\

\haiku{Een volwassen vrouw....}{die beweert magere-Hein}{gezien te hebben}\\

\haiku{{\textquoteleft}Jij bent den laatsten,!}{tijd bepaald niet heelemaal}{in orde Hermans}\\

\haiku{Als u uw kleeding even,.}{los wilt maken zal ik uw}{hart onderzoeken}\\

\haiku{Op het bureau zat.}{Jonkmans al vol ongeduld}{op hem te wachten}\\

\haiku{{\textquoteright} {\textquoteleft}Jawel, chef, maar hij,.}{knipt niet alleen zijn eigen}{baard maar ook zijn haar}\\

\haiku{Mevrouw Hobbel had.}{meer dan \'e\'en reden om naar}{buiten te kijken}\\

\haiku{Toen ik nu hoorde,.}{bellen vertrouwde ik het}{heelemaal niet meer}\\

\haiku{begrijpelijk, daar.}{zij slechts zelden aan licht en}{lucht werd blootgesteld}\\

\section{Paul Meeuws}

\subsection{Uit: Badhuis in de sneeuw}

\haiku{Maar aan die woorden.}{zat het parfumluchtje van}{een vreemd accent}\\

\haiku{Blom had het over zijn,.}{ontmoeting met Feininger}{eerder op de dag}\\

\haiku{Ik ook, daarom heb.}{ik geprobeerd je tegen}{hem te beschermen}\\

\haiku{Misschien hebben ze,,.}{gelijk denkt Feininger en}{klinkt het toch te zacht}\\

\haiku{Blom turft het aantal.}{gespeelde stukken op een}{notitieblokje}\\

\haiku{Er was geen durf voor.}{nodig om er je vinger}{doorheen te steken}\\

\haiku{Hij zou rechtstandig,,.}{omlaag storten de grond in}{samen met het huis}\\

\haiku{De soldaten voor.}{het Stadhuis keken verbaasd}{in onze richting}\\

\haiku{{\textquoteright} {\textquoteleft}Pas jij maar beter,{\textquoteright}.}{op je vrouw zei mijn moeder}{met trillende stem}\\

\haiku{Ik kon mij bij het {\textquoteleft}{\textquoteright}.}{woordheulen eigenlijk niets}{verkeerds voorstellen}\\

\haiku{Mijn broertje en ik,.}{gingen in de houding staan}{net als iedereen}\\

\haiku{Binnenkort zou het.}{klavichord verhuizen naar}{de voorkamer}\\

\haiku{Al wekenlang was,.}{het volop zomer het hout}{was droger dan ooit}\\

\haiku{Hier en daar dook haar.}{glinsterend oog tussen het}{gebladerte op}\\

\haiku{Die vrolijkten het,.}{heilig huisgezin op met}{zaag schaaf en hamer}\\

\haiku{Ze boog een beetje.}{voorover en leunde met haar}{kin in haar handen}\\

\haiku{Ik stelde me het.}{hebben van borsten tot in}{de finesses voor}\\

\haiku{Van afrastering;}{tot afrastering draafde}{hij als een jong paard}\\

\haiku{Eindelijk rolde.}{de bal tussen zijn blote}{voeten op de grond}\\

\haiku{De galg verhief zich.}{tegen een schitterende}{diepblauwe hemel}\\

\haiku{De mannen op het.}{verlichte terras keken}{op van hun kaartspel}\\

\haiku{moest je het spoor over,,.}{de stad uit richting M. waar}{ook de ijsbaan lag}\\

\haiku{Ze trok haar hand van.}{hem weg en rende pardoes}{de drukke straat op}\\

\haiku{Ongrijpbaar als een.}{vogeltje fladderde ze}{boven zijn bereik}\\

\section{Geerten Meijsing}

\subsection{Uit: Tussen mes en keel}

\haiku{Als aangeklede;}{aap was ik degene die}{belachelijk was}\\

\haiku{Wat mij betrof, voor.}{het allemaal tot een goed}{einde was gebracht}\\

\haiku{Wel wil ik zeggen,.}{waar ik woon al heb ik thuis}{niets aan te bieden}\\

\haiku{We wachtten tot de.}{vrouwelijke collega}{ook was ingestapt}\\

\haiku{Als hij het te kwaad,,.}{heb ben ik er altijd voor}{hem en andersom}\\

\haiku{Veel beter af dan.}{met het leven dat ik haar}{had kunnen bieden}\\

\haiku{Woorden van liefde -...}{zijn in water geschreven}{als {\'\i}emand dat wist}\\

\haiku{In de luwte van.}{het bestaan was alles tot}{stilstand gekomen}\\

\haiku{Zijn ijdelheid wordt.}{licht gekrenkt en dan laat hij}{de ander vallen}\\

\haiku{Vlot in de omgang,}{maar hij moet oppassen niet}{uit zijn nek te gaan}\\

\haiku{Zij was altijd een.}{beetje boos wanneer ze mij}{van advies diende}\\

\haiku{Maar voor die dingen,.}{had ik nu geen tijd want ik}{moest weer aan het werk}\\

\haiku{En steeds opnieuw dat.}{kleine reservoirtje van}{die vulpen vullen}\\

\haiku{wanneer ik me in.}{het midden van een witte}{nacht bij haar voegde}\\

\haiku{Ik was zo bang haar.}{te irriteren dat ik}{haar irriteerde}\\

\haiku{We hadden ze slechts,.}{voor het grijpen zo dichtbij}{waren de sterren}\\

\haiku{Daarin  waren.}{we niet beter af dan het}{redeloze vee}\\

\haiku{Maar waarom was het,?}{dan zo zoet veel lekkerder}{dan voedsel of drank}\\

\haiku{aan Martha, beide,;}{Mireilles Montalcino en}{de Magere Brug}\\

\haiku{{\textquoteright}, dan glimlachte ze:}{geheimzinnig en haalde}{ze haar schouders op}\\

\haiku{Ik wist ook - en van;}{geen enkele wetenschap}{word je vrolijker}\\

\haiku{En dat degene.}{die het meest liefhad altijd}{in het nadeel was}\\

\haiku{Want als zij eenmaal.}{alle sluizen openzette}{was ik nergens meer}\\

\haiku{Je werd beroofd, niet.}{alleen van je vrijheid maar}{ook van je wilskracht}\\

\haiku{Te minder omdat.}{er weinig te verwachten}{was als beloning}\\

\haiku{{\textquoteright} ~ Een van onze.}{grootste problemen was het}{verschil in tempo}\\

\haiku{Dit was het enige,.}{wat ik nog had mijn geloof}{in dit laatste boek}\\

\haiku{Het was er wel, als,.}{een baksteen in de hand maar}{je kon er niets mee}\\

\haiku{De mogelijkheid,.}{kreeg pr\'esence de spoken}{kregen gestalte}\\

\haiku{Een tweede keer zou.}{ik deze komedie niet}{kunnen opvoeren}\\

\haiku{Maar paniek kan de -.}{toestand wel omschrijven al}{is dat niets voor mij}\\

\haiku{Ik herkende het.}{soort vlees niet onmiddellijk}{en vroeg wat het was}\\

\haiku{Ik wil kinderen -,.}{is het niet van jou dan maar}{van iemand anders}\\

\haiku{De meegebrachte.}{studieboeken liet ik even}{voor wat ze waren}\\

\haiku{Vooral als zij haar,.}{formule uitsprak kon ik}{het niet meer houden}\\

\haiku{Ik wilde daar geen.}{kwestie van maken en dat}{deed zij ook niet}\\

\haiku{Ten leste was ik,.}{nauwelijks meer bruikbaar zelfs}{niet als seksobject}\\

\haiku{Maar wat mij het meest,.}{verontrustte was dat ik}{genoot van mijn smart}\\

\haiku{Hij was uit Holland.}{weggegaan om in New York}{beroemd te worden}\\

\haiku{Waarschijnlijk had ze.}{tegenover hem heel wat meer}{over mij te klagen}\\

\haiku{Ze zei wel {\textquoteleft}liefje{\textquoteright},,.}{tegen me afstandelijk}{ironisch en gemeend}\\

\haiku{En in de oranje.}{streep van de ondergaande}{zon zag ik de hoop}\\

\haiku{Ik hoef hem niet uit.}{de kast te trekken om de}{hoes terug te zien}\\

\haiku{Het deed pijn (zoals:}{mijn moeder gezegd had na}{elke bevalling}\\

\haiku{Tenslotte kon ik.}{ook op eigen kracht tegen}{de berg opklimmen}\\

\haiku{En kracht hadden ze,.}{mij bepaald niet gegeven}{deze roespillen}\\

\haiku{Toen ik haar leerde.}{kennen woonde ze tussen}{uitdragersspullen}\\

\haiku{Net als bij klussen.}{begint goed koken met het}{juiste gereedschap}\\

\haiku{Zij was er ook niet,.}{graag wist niet zo goed wat te}{doen als ze thuis was}\\

\haiku{Als mijn boek later,.}{uitkwam dan gepland zou ik}{gewoon wegblijven}\\

\haiku{Die wisten ervan,.}{maar konden niet zeggen hoe}{serieus het was}\\

\haiku{Deed de afwas, en.}{werkte een middag met de}{zeis in de boomgaard}\\

\haiku{Zoals Tolstoj zei ():}{die ik ooit van onder haar}{aanrecht bevrijd had}\\

\haiku{Voor de duur van een.}{kop thee is het oor van de}{ander gewillig}\\

\haiku{Zwarte schoonmakers.}{in witte jassen kwamen}{traag dweilend voorbij}\\

\haiku{Maar koffie hoort bij,.}{de psychiatrie zoals}{we zullen merken}\\

\haiku{Daarna zou hij met.}{zijn staf bespreken of ik}{in aanmerking kwam}\\

\haiku{Ik zei sorry toen.}{onze handen aan het stuur}{elkaar even raakten}\\

\haiku{Ze was gekleed om,.}{te winnen niet op punten}{maar met een knock-out}\\

\haiku{En stelt u zich dan.}{uw vriendin voor in de rol}{van Schopenhauer}\\

\haiku{Het water klotste.}{met de wind mee heen en weer}{onder de bruggen}\\

\haiku{Als ik mijzelf een,:}{graf uitzocht kon de tombe}{niet groot genoeg zijn}\\

\haiku{een groene kapel,.}{de hele natuur moest mij}{tot kathedraal zijn}\\

\haiku{Het begraven der.}{doden in het laagland is}{levenloos en kil}\\

\haiku{Aldus verging het,.}{mij de eerste keer en zo}{is het gebleven}\\

\haiku{Hoe meer Pierlala,}{mij te na komt hoe harder}{en uitbundiger}\\

\haiku{Lange tijd heb ik. '}{het idee gehad dat zij er}{alleen voor mij was}\\

\haiku{Als ik niet verliefd,.}{op of idolaat van iemand}{was schortte er iets}\\

\haiku{Mijn ouders waren,;}{voor langere tijd verreisd}{mijn broer zat in dienst}\\

\haiku{Want de gedachte.}{aan zelfmoord vergezelde}{me nu voortdurend}\\

\haiku{Daarna vertrok mijn.}{vrouw en bleef ik alleen met}{mijn dochter achter}\\

\haiku{De concurrentie.}{met mijn dochter was bijna}{iedereen te zwaar}\\

\haiku{Kost het u moeite,?}{om uit bed te komen zelfs}{later op de dag}\\

\haiku{Ik leed dus aan iets.}{waarvoor andere mensen}{behandeld werden}\\

\haiku{Maar ik wilde mij,.}{niet laten behandelen}{nooit van mijn leven}\\

\haiku{Liever had ze die.}{pas in gebruik genomen}{na de verhuizing}\\

\haiku{Maar ze begreep dat.}{ik mijn behoefte moeilijk}{in het park kon doen}\\

\haiku{Ik moest mijn werk doen.}{en tegemoetkomen aan}{haar verwachtingen}\\

\haiku{In ieder geval -.}{moest ik op het werk zijn dat}{was vaak al genoeg}\\

\haiku{{\textquoteleft}Meneer Provenier,,,.}{wij voeren een gesprek u}{en ik een dialoog}\\

\haiku{En u beschouw ik,.}{niet als analysant die maar}{wat aan kan kletsen}\\

\haiku{In Itali\"e had ik.}{nog een kettingzaag die ik}{wilde ophalen}\\

\haiku{zijn afkeer voor de.}{soort was onverholen en}{welgeformuleerd}\\

\haiku{Voor zo'n dor leven.}{als het zijne had ik mij}{willen behoeden}\\

\haiku{Jij blijft hier tot het,!}{laatst al  was het om mij}{een plezier te doen}\\

\haiku{Ik kneep hard in mijn.}{arm om te voelen of ik}{nog iets kon voelen}\\

\haiku{Met mijn stemming was.}{mijn libido tot onder}{het nulpunt gezakt}\\

\haiku{het leek me alleen.}{beter als we allebei}{even tot rust kwamen}\\

\haiku{Ze hadden mij niet.}{eerder zo trefzeker op}{mijn plaats gewezen}\\

\haiku{koffers uitpakken,,.}{wat opruimen eenvoudig}{voedsel bereiden}\\

\haiku{Een diep geworteld.}{pessimisme moest eraan}{ten grondslag liggen}\\

\haiku{Toen hij de nachtmis,.}{uitkwam dacht hij dat hij het}{ergste gehad had}\\

\haiku{Laatst had hij nog een.}{paar knopen aangezet en}{zijn schoenen gepoetst}\\

\haiku{Daar liep een lange.}{weg tussen de pijnbomen}{door tot op het strand}\\

\haiku{dat hij geen vrije keus -.}{meer had bij weid gedwongen}{een stap te zetten}\\

\haiku{{\textquoteleft}Meneer Provenier,{\textquoteright}, {\textquoteleft}.}{zei hij onrustig snuivend}{doet u niets overhaast}\\

\haiku{Mijn bezigheden.}{en gedachten waren in}{\'e\'en keer stilgelegd}\\

\haiku{Vergis je niet - jij.}{hebt een staat van dienst om u}{tegen te zeggen}\\

\haiku{Alles wat je daar,.}{hebt bereikt mag je niet in}{\'e\'en keer weggooien}\\

\haiku{Ik zag voortdurend,,.}{hoe zij in haar standje knieen}{wijd bovenop zat}\\

\haiku{In deze laatste -.}{rit had zij haar ziel gelegd}{van 2000 cc inhoud}\\

\haiku{Mijn auto's waren.}{een waarborg dat ik nergens}{zou blijven steken}\\

\haiku{Ooit had mijn vader.}{mij verteld dat z{\'\i}jn vader}{nooit gelachen had}\\

\haiku{Elk woord dat op mijn,.}{zwakte duidde maakte mijn}{moeder radeloos}\\

\haiku{Die scheerspullen kun,{\textquoteright}.}{je weer meenemen was het}{eerste wat ze zei}\\

\haiku{De psychiater,.}{liet mij plaatsnemen in zijn}{stoel en bleef zelf staan}\\

\haiku{Natuurlijk hield zij.}{van mij en zou zij me nooit}{in de steek laten}\\

\haiku{{\textquoteright} Bewegingen van.}{benen en armen kon ik}{niet meer beheersen}\\

\haiku{De foto's van mijn.}{dochter en van h\'a\'ar prikte}{ik op het prikbord}\\

\haiku{De rokers kwamen.}{op hun elfendertigst rond}{de snoepkar drommen}\\

\haiku{Ik wilde bellen,.}{een verbinding leggen met}{de buitenwereld}\\

\haiku{{\textquoteright} {\textquoteleft}Als je een verzoek,.}{indient kan ik vragen of}{hij morgen tijd heeft}\\

\haiku{Zij komt onder de.}{douche vandaan en stapt in}{haar ciabatten}\\

\haiku{het linkerbeen recht,.}{uitgestrekt het rechter over}{de leuning gehaakt}\\

\haiku{Zij trok haar dikke.}{lippen weg om alles aan}{het licht te brengen}\\

\haiku{Zou ik het nog eens,.}{na kunnen vertellen dan}{bleef niets onvermeld}\\

\haiku{Zelfs voor de ergste.}{types liep ik plotseling}{over van sympathie}\\

\haiku{of als het hoppen,.}{van een overvoed konijn vlak}{voor de kerstdagen}\\

\haiku{Alles geregeld,,.}{natje en droogje aanspraak}{en mededogen}\\

\haiku{deze apen met een.}{overmaat aan sociale}{intelligentie}\\

\haiku{Wij mensen hadden.}{ons opgericht om naar de}{sterren te kijken}\\

\haiku{{\textquoteright} {\textquoteleft}Mevrouw, ik heb drie.}{kinderen en voorlopig}{twaalf kleinkinderen}\\

\haiku{Bijna had ik hem.}{mijn Italiaanse shampoo}{ook nog toegestopt}\\

\haiku{Die dingen doe je,.}{in de ergo of anders}{in een spreekkamer}\\

\haiku{omdat ik al mijn.}{oude correspondenties}{had afgebroken}\\

\haiku{Niet veel, de mensen.}{weten niet meer wat het is}{een brief te schrijven}\\

\haiku{van het andere.}{uiteinde was je evenwel}{pas als laatste weg}\\

\haiku{{\textquoteright} Gemakkelijk viel '.}{het me niets ochtends om}{twaalf uur warm te eten}\\

\haiku{We hebben toen nog.}{lang moeten zoeken naar}{een kurkentrekker}\\

\haiku{Natuurlijk was ik.}{solidair met mensen die}{ook wilden vallen}\\

\haiku{De enige die wel,,.}{eens voor mij kwam was Cindy}{zonder haar vriendin}\\

\haiku{{\textquoteright} Heel voorzichtig sneed {\textquoteleft}{\textquoteright}.}{zeg-alsjeblieft-Hella}{haar tafelplan aan}\\

\haiku{We mogen toch wel?}{\'e\'en of twee keer per week met}{onze arts spreken}\\

\haiku{Mijn ongeschoren;}{wangen hadden niets zieligs}{of onderworpens}\\

\haiku{Ik nam een appel.}{uit de fruitkrat en wreef die}{glimmend over mijn mouw}\\

\haiku{{\textquoteright} Wat kunnen mensen,.}{flink zijn onderweg naar hun}{eigen ondergang}\\

\haiku{Nog voor ik langs het,.}{kantoortje kwam schokten mijn}{schouders geluidloos}\\

\haiku{Op geen enkele.}{manier kon ik mijzelf tot}{bedaren brengen}\\

\haiku{Daar stond hij met zijn.}{twee valiezen en de doos}{bij de tramhalte}\\

\haiku{Ik dacht zo dat je.}{dan precies weer terugkomt}{op je uitgangspunt}\\

\haiku{{\textquoteright} Twee zones en een.}{overstapje waren genoeg}{om thuis te komen}\\

\haiku{Niet van die dunne}{tissues waarvan altijd een}{doos klaarstond in geval}\\

\haiku{Het leven dat hij.}{leiden moest als hij zich aan}{haar aan zou passen}\\

\haiku{omdat je daardoor.}{de dingen gaat zien zoals}{ze werkelijk zijn}\\

\haiku{Elke dag dat je.}{nog geld had uit te geven}{was meegenomen}\\

\haiku{Hij was weer terug,.}{uit de grote landen ver}{van de mooie dingen}\\

\haiku{Adriaan moest niet gaan.}{denken dat hij voor hem zou}{blijven koerieren}\\

\haiku{Ze zullen hun best.}{doen om te bewijzen dat}{het niet aan hen ligt}\\

\haiku{Om zichzelf wakker.}{te schudden begint hij aan}{haar kraag te trekken}\\

\haiku{hoe kon je weten?}{of iemand maar enigszins in}{de smaak zou vallen}\\

\haiku{Zijn knie\"en waren,,.}{stram zijn kin voelde ruw zijn}{kleren waren nat}\\

\haiku{Ondertussen was.}{het imperatief dat hij}{zijn blaas ledigde}\\

\haiku{Hoe vaker hij dat,.}{deed hoe gedecideerder}{de afwijzingen}\\

\haiku{Hij was al niemand,.}{meer dus zijn spullen konden}{het best de deur uit}\\

\haiku{Of had hij nog dat,?}{van gisteren aan waarin}{hij had geslapen}\\

\haiku{je bent, dat je de,.}{beste was dat het nog nooit}{zo goed is geweest}\\

\haiku{Ik ben net zo goed.}{en net zo slecht als alle}{andere mensen}\\

\haiku{Wanneer twee mensen,.}{bijeen zijn praten ze graag}{over  een derde}\\

\haiku{vier lieve meisjes.}{die zich onder die lampen}{over mij heen bogen}\\

\haiku{Ik schaam mij niet voor,.}{deze littekens en ben}{er ook niet trots op}\\

\haiku{Met mijn vrienden, en,.}{zeker met mijn exen had zij}{nooit contact gezocht}\\

\haiku{was ik te laf om?}{de zelfdoding tot een goed}{einde te brengen}\\

\haiku{Vroeger had het hier.}{aangenaam geroken naar}{boeken en tabak}\\

\haiku{Goedkoop was het ook,.}{niet bepaald maar gelukkig}{was ik verzekerd}\\

\haiku{Ik wist niet wat me,.}{overkwam zo hevig als ik}{door haar werd bemind}\\

\haiku{en deze tussen;}{novelty seeking en non}{novelty seeking}\\

\section{Johannes van Melle}

\subsection{Uit: Bart Nel, de opstandeling}

\haiku{Een meisje van een,.}{jaar of negen kwam binnen}{haar schort vol kuikens}\\

\haiku{{\textquoteleft}Maar dit s\^e ek vir,}{oom as Botha ons in die}{oorlog wil insleep}\\

\haiku{Fransina staarde.}{naar buiten en oom Giel zat}{maar stil te roken}\\

\haiku{wist al dat zij zich.}{de gehele morgen had}{lopen opwinden}\\

\haiku{Hy is niks meer as '.}{n werktuig in die hande}{van Engeland nie}\\

\haiku{Ek wens jy wil by.}{die huis bly en jou nie met}{die dinge moei nie}\\

\haiku{Telkens weer stond een.}{van de ouden op om tot}{kalmte te manen}\\

\haiku{Maar dat gaf je nog.}{niet het recht om tegen het}{gezag op te staan}\\

\haiku{{\textquoteright} {\textquoteleft}Jy wil in die veld.}{gaan en oorlog maak en jou}{boerdery verwaarloos}\\

\haiku{of zijn kaffers wel.}{werkten en of er geen vee}{in de landen kwam}\\

\haiku{Je hoort veel in een.}{enkel woord en je begrijpt}{veel uit een zwijgen}\\

\haiku{Ze zei niet veel meer,.}{dekte met Annekie en}{de meid de tafel}\\

\haiku{Sy kan miskien praat.}{en dinge uitbring wat my}{planne verydel}\\

\haiku{Twee dagen later,,.}{nog voor de zon op was liep}{Bart al in zijn land}\\

\haiku{het ek 'n kaffer ' '.}{opn bycicle metn}{brief na hom gestuur}\\

\haiku{Gisteraand laat kom.}{die kaffer toe terug en}{hier is die antwoord}\\

\haiku{Maar trachten, hem tot.}{andere gedachten te}{brengen deed ze niet}\\

\haiku{Hij kon het nu toch.}{niet meer geheim houden en}{het hoefde ook niet}\\

\haiku{{\textquoteright} {\textquoteleft}Maar as ek iets vir;}{jou kan doen terwijl jij van}{die huis af weg is}\\

\haiku{Ek wil julle nie{\textquoteright},.}{verbloem nie dat die toestand}{ernstig is zei hij}\\

\haiku{Nou sal die Jood mos.}{partij goed byskryf en}{pryse verander}\\

\haiku{{\textquoteleft}Ek wonder of ons{\textquoteright},.}{dan maar nie eers na hom toe}{moet ry nie zei Bart}\\

\haiku{Die motorkarre.}{spioen ons en hulle weet}{presies waar ons is}\\

\haiku{Het kommando hield.}{stil en de paarden draaiden}{hun rug naar den wind}\\

\haiku{Na de dienst zocht hij.}{zijn ligplaats een eindje van}{de anderen af}\\

\haiku{Aan de luchtkimmen}{schoten telkens bliksems en}{een enkele maal}\\

\haiku{Op een  afstand.}{leek het of het den grond met}{een kleed bedekte}\\

\haiku{Hier is twee k\^erels,{\textquoteright},.}{van die kommando Venter}{kommandant zei hij}\\

\haiku{{\textquoteleft}Gauw{\textquoteright}, zei Fransoois en.}{zij joegen naar een ruig stuk}{blauwe haakdorens}\\

\haiku{Ons weet nou dat die.}{regeringsmense weer ons}{spore gekry het}\\

\haiku{{\textquoteleft}Nee wat{\textquoteright}, vertelde, {\textquoteleft}.}{de een mismoedigdaar is}{geen rebellie nie}\\

\haiku{Ons kan die Vader.}{dank dat die rebellie so}{gou doodgeloop het}\\

\haiku{Ek is bly dat Bart{\textquoteright},.}{eindelik rede verstaan}{zei de kommandant}\\

\haiku{{\textquoteright} {\textquoteleft}Elkeen moet self weet, '.}{wat hy doen maar ek vind dit}{n verkeerde ding}\\

\haiku{Sal oom bietjie sit.}{dan sal ek gou bietjie vir}{oom koffie ingooi}\\

\haiku{{\textquoteleft}Kornelia vra of;}{jy nie liewer in die nag}{by ons wil wees nie}\\

\haiku{Hij zag hoe kwaad zij,.}{was en stond op gaf haar koel}{de hand en reed heen}\\

\haiku{{\textquoteleft}die huis beheer en.}{dan nog buitekant sake}{behartig daarby}\\

\haiku{{\textquoteright} Maria's gezicht,.}{werd geheel rood tot haar hals}{en oren waren rood}\\

\haiku{{\textquoteright} {\textquoteleft}As Bart net weer by{\textquoteright},.}{die huis is sal alles wel}{regkom troostte hij}\\

\haiku{Dit moet ellendig.}{wees as jy in so'n tyd van}{mekaar verskil}\\

\haiku{Of sou hulle so?}{kwaad wees vir mekaar dat hy}{nie  wil skryf nie}\\

\haiku{Dat was natuurlik,.}{een vergissing de man had}{een verkeerde voor}\\

\haiku{Nu eerst bemerkte,.}{zij hoe fel zij gehoopt had}{Bart terug te zien}\\

\haiku{Dat zijn brief te laat,.}{gekomen was maakte voor}{hem weinig verschil}\\

\haiku{wie het was zag hij.}{haar achter onder de tent}{zitten en hield stil}\\

\haiku{{\textquoteright} {\textquoteleft}Sal ek nie vir jou?}{wag nie en jou gou na huis}{ry met die motor}\\

\haiku{Zacht, om hem niet te,.}{wekken knapte zij zich in}{haar kamer wat op}\\

\haiku{Zij deed verkeerd, gaf.}{de mensen aanleiding om}{over haar te praten}\\

\haiku{Toen hij bij haar hek.}{afsteeg kwam Fransina naar}{buiten gelopen}\\

\haiku{{\textquoteleft}Dag Fransina{\textquoteright}, zei.}{hij op een toon alsof hij}{haar wilde troosten}\\

\haiku{Die brief oor Basson ',.}{wasn onvergeeflike fout}{maar Bart sal verstaan}\\

\haiku{Sy moet hom nou maar.}{opreg alles skryf en dan sal}{dinge weer regkom}\\

\haiku{{\textquoteright} {\textquoteleft}Nee eintlik nie, maar,.}{tog Japie van Martha was}{gister daar by ons}\\

\haiku{Wat een giftige,.}{bende mensen was dat toch}{die buren van haar}\\

\haiku{Jy moet self maar kos,.}{vir jou opskep ek wil}{perbeer bietjie slaap}\\

\haiku{Het was er heet en}{benauwd en lui lagen ze}{allen uitgestrekt}\\

\haiku{{\textquoteleft}Jammer{\textquoteright}, dacht hij, {\textquoteleft}dan.}{sou ek nou iets gehad}{het om in te lees}\\

\haiku{Het was een man, die.}{in de buurt van Mooiplaats een}{kleine winkel had}\\

\haiku{Ag meneer Nel, jou{\textquoteright},.}{regering maak darem nie}{goed nie zei de Jood}\\

\haiku{het ges\^e dat as.}{die gras in sy sate is}{alles sal oor wees}\\

\haiku{{\textquoteright} vroeg hij op een dag,.}{nadat ze daartoe verlof}{verkregen hadden}\\

\haiku{{\textquoteright} Toen hij de brief van,.}{Malherbe kreeg liet hij oom}{Gawie die lezen}\\

\haiku{Ze kregen hout en.}{gereedschap en zaagden en}{schaafden de tijd om}\\

\haiku{Oom Gawie had een,,.}{jaar gekregen en ging toen}{zijn tijd om was heen}\\

\haiku{{\textquoteleft}Ons rebelle het{\textquoteright},, {\textquoteleft}.}{swaar gekry zei hijmaar ons}{moet probeer vergeef}\\

\haiku{Japie kan tog nie.}{daarvan maak wat hy daarvan}{behoort te maak nie}\\

\haiku{{\textquoteright} {\textquoteleft}Dan wou ek sommer,.}{vandag nog ry so gou as}{Japie terug is}\\

\haiku{Alles wat zij deed.}{ging alsof het geen moeite}{of overleg kostte}\\

\haiku{Hij legde even zijn.}{hand op de hare en zij}{glimlachte hem aan}\\

\haiku{Dat was nu eenmaal.}{voorbij en zij was nu de}{vrouw van Ferdinand}\\

\haiku{Het was evenwel toch.}{bijna half tien toen hij op}{Bassons Rust aankwam}\\

\haiku{Hij reed zijn auto}{de garage in en toen}{hij terugkwam stond}\\

\haiku{{\textquoteleft}Jammer dat ek tog{\textquoteright},.}{maar nie na swaer Jan toe gery}{het nie dacht hij}\\

\haiku{Zij zag het ineens,.}{alsof hij het zelf haar in}{woorden had gezegd}\\

\haiku{Het was nog vroeg en.}{na een nacht van regen woei}{er een koele wind}\\

\haiku{Die beste is om '{\textquoteright},.}{haar nou maarn tydjie aan haarself}{oor te laat dacht hij}\\

\haiku{As sy nog altyd,.}{meer van hom hou as van my}{laat haar dan maar gaan}\\

\haiku{'n Vrou wat by my '.}{is en aann ander dink}{wil ek nie h\^e nie}\\

\haiku{Maar hij vermoedde,.}{toch ook dat zij hem liever}{zag gaan dan komen}\\

\haiku{Dan hoop ek dat hy{\textquoteright},.}{haar terugvat en dat hy}{haar hel gee dacht hij}\\

\haiku{Anders sal jy tog.}{maar langs die pad moet stilhou}{om iets te gebruik}\\

\haiku{Sy s\^e hy doen wat.}{hy kan om haar die lewe}{aangenaam te maak}\\

\haiku{Hij stak zijn pijp aan.}{de kaars aan en geraakte}{diep in gedachten}\\

\haiku{het, wat se oneer?}{steek dan daarin om jou eie}{ouers te gaan vra}\\

\haiku{dit moet ek eerlik,.}{s\^e al kan ek die vent ook}{nie meer verdra nie}\\

\haiku{{\textquoteleft}Van Fransina het.}{ek gehou soos min mans van}{hulle vrouens hou}\\

\haiku{Soms als zij tegen,.}{hem aangeleund zat had zij}{tranen in haar ogen}\\

\haiku{Nou my kind, dat ek.}{nog baie baie jare op jou}{gesondheid mag drink}\\

\haiku{{\textquoteleft}'n Kind is darem ' '{\textquoteright},.}{n groot aantreklikheid in}{n huis zei Maritz}\\

\haiku{Jy kan haar maklik.}{elke week of eenmaal in}{die veertien dae sien}\\

\haiku{Ze ging weer tegen;}{haar tante aan staan en keek}{naar de gezichten}\\

\section{Josepha Mendels}

\subsection{Uit: Als wind en rook}

\haiku{{\textquoteright} {\textquoteleft}Visite of niet,{\textquoteright}, {\textquoteleft}.}{zei hij weervoor ons is het}{vandaag een werkdag}\\

\haiku{Herman zorgt nu voor,.}{zichzelf dat maakt het mij wel}{gemakkelijker}\\

\haiku{Maar ja, hij had me,...}{niet gezoend mij niet in zijn}{armen genomen}\\

\haiku{En wat wil je, dan.}{ben je meteen weer eens in}{Amsterdam geweest}\\

\haiku{{\textquoteright} {\textquoteleft}Je haar,{\textquoteright} antwoordde, {\textquoteleft}.}{hij toenalleen je haar heb}{je van je moeder}\\

\haiku{Ik probeerde op,,.}{te staan maar het lukte niet}{de stoel was te diep}\\

\haiku{Simon vroeg mij op.}{te staan en bracht me naar zijn}{warme bed terug}\\

\haiku{Hij zoende haar zo.}{nu en dan en de nacht bracht}{hem knapendromen}\\

\haiku{{\textquoteleft}Als een mens alles.}{vooruit wist zou hij heel veel}{dingen anders doen}\\

\haiku{Meer zal voorlopig,.}{niet mogelijk zijn want dan}{wordt het huis te klein}\\

\haiku{{\textquoteleft}Nu zal ik dan toch,{\textquoteright}.}{eindelijk mijn vrouw van je}{maken verdween het}\\

\haiku{{\textquoteright} {\textquoteleft}O,{\textquoteright} antwoordde ik, {\textquoteleft}...{\textquoteright} {\textquoteleft},}{het arme schaapLaten we}{er iets op drinken}\\

\haiku{Ik liet hem begaan,.}{zoals ik van de eerste}{keer af gedaan had}\\

\haiku{{\textquoteleft}Want weet je, Hoefstar,.}{ik moet die zelf vanavond in}{de trein meenemen}\\

\haiku{{\textquoteleft}Het is nu drie uur,.}{voor de melkboer belt zal het}{wichtje er wel zijn}\\

\haiku{Voor zover Simon.}{zich met haar bemoeide was}{hij heel lief voor haar}\\

\haiku{Judith leek heel veel,;}{op Simon was donker en}{had zijn bruine ogen}\\

\haiku{Het was de kiem van.}{haat voor Simon die in mijn}{handen opwelde}\\

\haiku{je vond sommige,.}{even mooi als weer andere}{even middelmatig}\\

\haiku{{\textquoteright} antwoordde Elisa, {\textquoteleft}.}{ik ben zo gewend aan mijn}{eigen instrument}\\

\haiku{wanneer Judith nog,.}{zoiets gezegd zou hebben}{maar dit stille kind}\\

\haiku{{\textquoteleft}Hij is al weg,{\textquoteright} zei,.}{ik lachend en toen zijn wij}{gaan musiceren}\\

\haiku{Elisa had veel werk.}{om zijn garderobe in}{orde te brengen}\\

\haiku{in strijd met Simons.}{opvattingen geeft Elisa}{een ander gezicht}\\

\haiku{Na dit verblijf aan.}{zee gingen wij nog een week}{naar mijn ouders toe}\\

\haiku{Het zou de eerste.}{keer zijn dat ze haar na de}{vakantie weer zag}\\

\haiku{De denneboom keek.}{door het halfgeopende}{gordijn naar binnen}\\

\haiku{Als mijn moeder nog,?}{leefde wat zou zij daarvan}{wel gezegd hebben}\\

\haiku{Als mijn moeder nog,?}{leefde wat zou ze wel van}{dit meisje denken}\\

\haiku{{\textquoteleft}U bent zo blootshoofds.}{toch veel leuker en die hoed}{maakt u nog ouder}\\

\haiku{Zij ruimde af, trok.}{hem zijn jasje uit en bond}{hem een vaatdoek voor}\\

\haiku{{\textquoteright} riep een stem, en een.}{arm strekte zich uit naar een}{lampje naast het bed}\\

\haiku{Minouche was de.}{tweede vrouw in zijn leven}{die dit had gedaan}\\

\haiku{Men heeft mij verteld.}{dat het zo buitengewoon}{interessant is}\\

\haiku{En boven op dat.}{blanke wit lagen zelfs wat}{oude broodkruimels}\\

\haiku{Ik verbeeldde me,...}{toen dat ik ongelukkig}{was ongelukkig}\\

\haiku{De hangklok slaat juist,.}{tien uur de kamer is reeds}{keurig opgeruimd}\\

\haiku{De dag ligt open voor,.}{me ik kan eten wanneer ik}{wil en wat ik wil}\\

\haiku{Maar het is al zo,...}{lang geleden dat hij dit}{gedaan heeft zo lang}\\

\haiku{Of denk je dat je,?}{te oud bent om kinderen}{te krijgen Louise}\\

\haiku{{\textquoteright} zegt ze dan, {\textquoteleft}of je.}{vanavond weer de seider bij}{ons komt doorbrengen}\\

\haiku{{\textquoteright} {\textquoteleft}Maar als ze een kind,?}{moet krijgen hoe kan ze dan}{nog dikker worden}\\

\haiku{{\textquoteright} {\textquoteleft}En heb je het niet,,?}{te druk zo alleen met een}{dagmeisje Elisa}\\

\haiku{Nu zegt ze opeens.}{dat dit het enige is waar}{ze plezier in heeft}\\

\haiku{Heeft Simon zijn werk?}{voor het concours ingestuurd}{en maakt hij een kans}\\

\haiku{ik ben er, Richard,.}{Palmers ik zou hem zo graag}{terug willen zien}\\

\haiku{Voor Richard is er,.}{altijd plaats al zou het maar}{in haar schooltas zijn}\\

\haiku{Nu legt ze alle... {\textquoteleft}}{uitgescheurde bladzijden}{op haar schoot en wacht}\\

\haiku{Hij geeft het haar en.}{zij lacht en likt met haar tong}{die boter eraf}\\

\haiku{{\textquoteleft}Ik heb ook tot tien,.}{jaar met poppen en beren}{gespeeld Rebecca}\\

\haiku{Maar je ziet er niets,,.}{van integendeel het kind}{lijkt wel opgelucht}\\

\haiku{Vanzelfsprekend kan...}{niemand mij beletten niet}{aan hem te denken}\\

\haiku{het daar altijd zo,.}{fris ik herinner mij niet}{precies meer naar wat}\\

\haiku{Maar je weet zelf hoe,.}{dat gaat de helft zal er wel}{van gelogen zijn}\\

\haiku{Maar dat geklets gaat,,.}{vanzelf weer over Simon dat}{begrijp je ook wel}\\

\haiku{{\textquoteright} En hij duwde me.}{achter het scherm opdat ik}{mij zou uitkleden}\\

\haiku{Het hield niet op, het.}{leek op het laatst bijna of}{hij ze zelf verzon}\\

\haiku{Hoe kan ik u de?}{stemming uitbeelden die er}{op vrijdagavond heerste}\\

\haiku{Ik huilde bijna.}{dat ik niet groter was en}{mijn stem niet sterker}\\

\haiku{Zou Klazien willen,?}{blijven en zal Simon de}{kinderen houden}\\

\haiku{Want voor mij is die.}{met het vertrek van Elisa}{reeds voorgoed voorbij}\\

\haiku{De groeten aan je,{\textquoteright}, {\textquoteleft}.}{vader zei Louiseik kom}{wel gauw weer eens langs}\\

\haiku{Jij blijft dus maar hier.}{in ons midden en vertrekt}{niet naar het zuiden}\\

\haiku{Eerst wist hij er geen,:}{antwoord op maar toen werd hij}{er zich van bewust}\\

\haiku{Maar zijn goede oog.}{is nu ook aangetast en}{hij ziet steeds slechter}\\

\haiku{Misschien zal hij pas;}{bij mij komen wanneer hij}{geheel blind zal zijn}\\

\haiku{hij komt dan als een,.}{hulpeloze maar dat zal}{hij zelf niet weten}\\

\haiku{{\textquoteright} Nu echter kijkt ze,.}{vol vertrouwen naar Richards}{kleine blauwe ogen}\\

\haiku{{\textquoteleft}Het sneeuwde zo bij,.}{ons het was er koud en hier}{is het heerlijk warm}\\

\haiku{Met sangh of spel haer, '?}{man verquickt Alst nodigh}{huyswerk is beschikt}\\

\section{Victor de Meyere}

\subsection{Uit: Langs den stroom}

\haiku{t Was de eenige;}{maal dat hij op zwier ging en}{hij deed het dan goed}\\

\haiku{Sneller liep hij voort,,.}{altijd maar sneller alsof}{hij achtervolgd werd}\\

\haiku{Ook vong hij muizen.}{die hij hem dan met ware}{zelfvoldoening bracht}\\

\haiku{Hij herinnerde.}{zich nog de eerste muis die}{hij gevangen had}\\

\haiku{De uil zat in 't.}{midden van de kooi op eenen}{poot en pinkoogde}\\

\haiku{hij stak ze in de,,.}{kooi een voor een den arm in}{de lange slobkous}\\

\haiku{'t reutelde maar,.}{even over zijne tong om ze}{niet te doen schrikken}\\

\haiku{'t Duurde echter '.}{niet lang of hij ging weer aan}{t prakkezeeren}\\

\haiku{Toen  werd het weer;}{duister in hem en duister}{v\'o\'or zijne oogen ook}\\

\haiku{De boeien nepen ' ';}{hem diep int vleesch en}{t bloed stond er v\'o\'or}\\

\haiku{De uil werd wakker,.}{zette zich wat verder en}{dommelde weer in}\\

\haiku{Hij lei zich terug.}{op de brits met de handen}{onder  den kop}\\

\haiku{En hij huiverde.}{van de gedachte die den}{lierenaar opriep}\\

\haiku{Ge moogt gerust zijn.}{en spreken als tegen uw}{eigen moeder t'huis}\\

\haiku{Duidelijk hoorden ';}{wijt lawaai van het volk}{op de kasseide}\\

\haiku{'t Was gedrest tot.}{in zijn haar dat met lange}{klissen saamplakte}\\

\haiku{Door 't veel werken.}{van den morgen tot den avond}{verkort hun leven}\\

\haiku{Had hij het zelf niet?}{gedaan toen hij dacht dat de}{tijd gekomen was}\\

\haiku{Z\'o\'o gauw hij maar kon.}{kroop hij onder de dekens}{in de groote alkoof}\\

\haiku{En 's morgens werd,.}{hij wakker met het eerste}{gekraai van den haan}\\

\haiku{de borst blaasbalgde,.}{op en neer om een beetje}{asem op te vangen}\\

\haiku{{\textquoteleft}ge zult het hier goed,{\textquoteright}.}{hebben in uwen ouden dag}{verzekerde zij}\\

\haiku{- En toch is het z\'oo,,....}{peinsde hij voort ik kan die}{pree niet verdienen}\\

\haiku{ik wacht niet langer,,.}{zegde hij terwijl hij zich}{langzaam aankleedde}\\

\haiku{En dan, ik weet het,:}{immers toch mijn geld kan ik}{niet meer verdienen}\\

\haiku{Hij wist dat het voor.}{altijd was en dat hij niet}{meer zou wederkeeren}\\

\haiku{Toke schreef het nieuws.}{in een langen brief aan den}{meester van de werf}\\

\haiku{Er vielen er in ',,.}{t water niet ver van de}{boot met luid geplons}\\

\haiku{men vertelde goed,,.}{en kwaad alles ondereen}{maar kwaad wel het meest}\\

\haiku{Als men hem terug ',.}{naart prison bracht voelde}{hij zich gelukkig}\\

\haiku{Hij wierp zich te bed,, '.}{en sliep in een zwaren roes}{tots morgens toe}\\

\haiku{'t waren altijd.}{gevangenen die men met}{dat werk gelastte}\\

\haiku{'t Zou hem overal,!}{blijven vervolgen overal}{waar hij komen zou}\\

\haiku{dat is niet goed voor.}{u. Trek ergens naar een vreemd}{dorp als schoenmaker}\\

\haiku{Heel den Rupelkant,,.}{liep hij af dorp na dorp maar}{nergens vond hij iets}\\

\haiku{'t Was nu weeral,?}{aan den waterkant maar wat}{kon hij er aan doen}\\

\haiku{Een Charleroische {\textquoteleft}{\textquoteright}.}{bakDen Jongen Jan volgde}{den beurtschipper op}\\

\haiku{Hij sloot de deur en '.}{ging op den driepikkel aan}{t venster zitten}\\

\haiku{- En dan, - ging hij voort, -,.}{die kerel daar heeft niet}{veel op te vragen}\\

\haiku{Hij moest maar weggaan,,....}{zoo gauw mogelijk naar een}{ander dorp ver weg}\\

\haiku{Een pijl uit eenen boog '!}{kon hem met zoo'n geweld niet}{int hart snorren}\\

\haiku{Ge zijt gij Andries,?}{Eyckmans geboren te}{Wintham niet waar}\\

\haiku{t briefken weer te.}{voorschijn en ontplooide het}{voor de tweede maal}\\

\haiku{Ge waart gij bij de,,?}{bende der woldieven over}{een jaar of vier he}\\

\haiku{Hongerig viel hij ',}{aant eten een korst brood van}{den proviand dien}\\

\haiku{Dezen nacht, als 't,.}{dorp in slaap lag zou hij er}{mede vertrekken}\\

\haiku{Dat ge beter moogt,,.}{varen Dries jongen dat ge}{beter moogt varen}\\

\haiku{hoe dieper en schooner,,.}{van kleur het werd als donker}{blauwachtig fluweel}\\

\haiku{Zij hadden ook hun.}{vreugde aan de fantasie}{van de kinderen}\\

\haiku{Als de kalkman is,;}{geboren Gaan we naar Dort}{om kaf en koren}\\

\haiku{En de zatlap - ik -;}{moet het zeggen heeft het niet}{lang meer getrokken}\\

\haiku{Als hij buitenkwam,,...}{zette de ziekte zich op}{hem vast ineens d\'aar}\\

\haiku{Op den stroom, volgde.}{hij de wondere speling}{van licht en schaduw}\\

\haiku{overal in heel het, ' '.}{land was er wat gaandet}{een oft ander}\\

\haiku{Men ontwaarde nog,.}{alleen een wit mat streepken}{tusschen de wimpers}\\

\haiku{Een glimlach zweefde,.}{om zijnen mond alsof hij}{zich verlicht voelde}\\

\haiku{Er lag zooveel leed.}{in dat enkel woord dat de}{moeder niet aandrong}\\

\haiku{Jef was eens lange,.}{maanden kwaad geweest op haar}{door haar eigen schuld}\\

\haiku{Maar eens de eerste,.}{bangheid voorbij bleef het eene}{leute heel den dag}\\

\haiku{Eens, toen een schip van,.}{stapel liep gaf men een fooi}{voor de werklieden}\\

\haiku{De moeder hief 't.}{hoofd op en hoorde zijne}{stappen versterven}\\

\haiku{de koster heeft me.}{gezeid dat ik u maar een}{handje moest helpen}\\

\haiku{daverde van 't.}{geweld waarmede de wind}{over het dak scheerde}\\

\haiku{- En dat het tijdens,.}{zoo'n onweer gebeuren moest}{vervolgde Nette}\\

\haiku{Zij gunde zich geen.}{oogenblikje rust en vond}{altijd een doening}\\

\haiku{Monica, nu zij,.}{weer alleen was luisterde}{angstig naar den storm}\\

\haiku{'t was of hij, in,...}{zwaren stormloop gestadig}{het huis bebeukte}\\

\haiku{Zij viel hem aan den.}{hals en kuste hem in een}{wilde omarming}\\

\haiku{Weer greep hij 't lijk,.}{op met moede handen die}{van pijn tintelden}\\

\haiku{Groote, lange droppen.}{zweepte de wind als hagel}{in zijn aangezicht}\\

\haiku{Hij voelde alleen,.}{de vochtigheid onder de}{voeten in het gras}\\

\haiku{Het leek een hortend.}{gevaarte dat naderde}{in een snellen rit}\\

\haiku{Met luide gillen,.}{snikte hij het uit in een}{stortvloed van tranen}\\

\haiku{Ik weet niet hoe ik,!}{tot hier ben geraakt met dat}{lijk in mijn armen}\\

\haiku{Het water kwam en.}{ik ben gaan vluchten met het}{lijk van mijn vader}\\

\haiku{Z\'oo geraakte ik '.}{t water v\'o\'or en kon ik}{den dijk bereiken}\\

\haiku{Maar eerst twee borrels,...}{klare jenever een voor}{u en een voor mij}\\

\haiku{En toch, als hij recht,.}{voor zich uitkeek scheen Neel niets}{vreemds te bemerken}\\

\haiku{De menschen van het.}{dorp stonden er met heelder}{hoopen saamgeschaard}\\

\subsection{Uit: De Vlaamsche vertelselschat. Deel 1}

\haiku{dat lekker soepken,,.}{kermde het manneken ik}{heb toch zoo'n honger}\\

\haiku{Hoe het ook poogde,.}{het kon zijn soep maar niet tot}{in zijn mond brengen}\\

\haiku{Nu, aanstaanden keer}{is het mijne beurt om te}{blijven en luiden}\\

\haiku{Begeef u zoo gauw;}{mogelijk naar de opening}{van deze spelonk}\\

\haiku{En als ge 't nu,.}{nogmaals probeeren moest sla ik}{u dood als ne pier}\\

\haiku{- Maar ik, met mijn vier,;}{pooten zal dan veel vroeger}{aankomen dan gij}\\

\haiku{- Als ge dan toch naar,.}{de stad meewilt kruipt dan maar}{in mijn achterste}\\

\haiku{'t Halfhaantje was.}{uiterst tevreden en liet}{het zich goed smaken}\\

\haiku{In een ommezien.}{waren al de schapen dood}{en opgevreten}\\

\haiku{- Water, kom uit mijn.}{achterste en zet het hier}{allemaal onder}\\

\haiku{Ik heb hier iets in.}{mijnen zak dat ik daartoe}{wel bezigen kan}\\

\haiku{De herder trok blij.}{met zijn tafeltje weg en}{de prins sprong te paard}\\

\haiku{{\textquoteleft}Knuppel sla den hoed{\textquoteright},.}{in den zak zei hij en het}{gebeurde alzoo}\\

\haiku{En toen de prins hem,.}{uitnoodigde mee te eten liet}{hij zich niet wachten}\\

\haiku{Hij vermoedde wel.}{wie de moordenaar was en}{werd woedend van toorn}\\

\haiku{Maar de jonge prins:}{kreeg het in de mot en riep}{op den kluppelzak}\\

\haiku{En hij ging recht door,,,.}{van dorp tot dorp van stad tot}{stad van land tot land}\\

\haiku{- Dat is ook al niet,.}{slecht geantwoord vervolgde}{de kapitein}\\

\haiku{Ik heb zin in uwen,.}{snuit maar zoo meteen nemen}{wij u nog niet mee}\\

\haiku{En de kapitein,.}{die ook kon zijn oogen van de}{prinses niet houden}\\

\haiku{Z\'o\'o werd hij van den.}{hoogsten berg van de stad naar}{beneden gerold}\\

\haiku{t Geraamte ging.}{de groote zaal uit tot in de}{gang en Jan volgde}\\

\haiku{Zij stonden v\'o\'or een,.}{arduinen draaitrap die naar}{den kelder leidde}\\

\haiku{Te dien einde moest.}{ik alle nachten op het}{kasteel gaan spoken}\\

\haiku{De duivels van de.}{hel hadden mij de grootste}{wreedheid opgelegd}\\

\haiku{Ge ziet het he, hoe?}{ge door een oog van een}{naald gekropen zijt}\\

\haiku{Jan Snoef en Kroes*~         .}{Er was eens een jongen en}{die heette Jan Snoef}\\

\haiku{om u het geheim,,.}{dat hij mij toevertrouwde}{mede te deelen}\\

\haiku{Knechten hebt gij tot,.}{uwe beschikking meer dan er}{u dienen kunnen}\\

\haiku{Hij nam het schip van.}{zijne schouders en zette}{het in het water}\\

\haiku{En daar hoorde hij.}{opeens in de verte een}{schrikkelijk gevecht}\\

\haiku{En zonder wachten.}{deed Hans een nieuwen wensch en}{hij werd ook een duif}\\

\haiku{- Beter iets dan niets,.}{al is dat iets dan ook zoo'n}{krimpieperig ding}\\

\haiku{De visscher moest bij:}{zijn thuiskomst het vischje in}{drie stukken snijden}\\

\haiku{Stijg van uw paard en}{neem uw zwaard in de hand. En}{zoo deed Gouden Lok}\\

\haiku{Zijn ouders zouden.}{kunnen nagaan wat er met}{hem geschieden zou}\\

\haiku{En de tooverheks was.}{toen wel verplicht van de nood}{eene deugd te maken}\\

\haiku{vroeg de ridder aan.}{den type en begon hem}{de mauw te veegen}\\

\haiku{De duivel bleef daar.}{maar staan blinken en zag ze}{niet buiten komen}\\

\haiku{De duivel trok al.}{jankende af en liet ne}{geweldigen vloek}\\

\haiku{'t zoog den man de '.}{hersens uit den kop ent}{bloed uit het hart}\\

\haiku{Een oude heer kwam.}{opendoen en die vroeg hem wat}{er hem beliefde}\\

\haiku{- Jongen, zegde hij,,?}{nogmaals ge weet wat ik u}{gezegd heb niet waar}\\

\haiku{- Weet gij wel, zei het,?}{paard dat het hier voor u een}{verboden plaats is}\\

\haiku{Nu, de meester heeft,.}{een paard dat sneller loopt dan}{ik maar dat is niets}\\

\haiku{een doornenhaag zoo.}{dicht begroeid dat mensch noch dier}{er doordringen kon}\\

\haiku{Hij liep te allen.}{kant om hem te vinden en}{de hand te drukken}\\

\haiku{- Zie maar eens hoe ik,.}{werpen kan en hij wierp den}{vogel in de lucht}\\

\haiku{En Jan Onversaagd.}{sprong zoo hoog hij maar kon en}{de tak deed de rest}\\

\haiku{De tijden zijn zoo}{lastig en onzeker en}{elke sterke man}\\

\haiku{Maar het zwijn liep veel.}{harder dan hij en haalde}{hem maar altijd in}\\

\haiku{gelukte ik er.}{in een van de reuzen bij}{de beenen te grijpen}\\

\haiku{Jan, echter, had al.}{spoedig in de gaten wat}{er op handen was}\\

\haiku{Als zij allemaal:}{binnen waren begon hij}{luidop te droomen}\\

\haiku{En Duimken, terwijl, '.}{al zijn broeders reeds ronkten}{aant luisteren}\\

\haiku{En met een kroontje.}{op zijn eigen hoofd lei hij}{zich daarna te rust}\\

\haiku{Met heel veel moeite.}{gelukte het hem nog uit}{de teil te kruipen}\\

\haiku{Voortaan was zij rijk,.}{en werken deed zij niet meer}{evenmin als Duimken}\\

\haiku{Ge kunt er van avond.}{met de schoolmeesteres eens}{lekker van smullen}\\

\haiku{Krak, krak, krak ging het.}{maar altijd bij elk woord dat}{over haar lippen kwam}\\

\haiku{Deze week is het.}{lot op de dochter van den}{koning gevallen}\\

\haiku{en Pakt en verslindt.}{en Vecht en overwint hielden}{dapper voet bij stek}\\

\haiku{En het arm boerken,.}{kwam naar voren schuchter en}{meer dood dan levend}\\

\haiku{- Maar zeg mij nu eens,,?}{vroeg hij verder wie heeft u}{dat ingeblazen}\\

\haiku{dat is zeker, en;}{staat het niet geschreven dan}{heb ik het gedroomd}\\

\haiku{- Vaarwel dan, zei de,!}{oude man u ook zal ik}{niet meer wederzien}\\

\haiku{En toen, onder 't,:}{opsmullen van zijn prooi zei}{de draak voortdurend}\\

\haiku{Hij sneed den buik van.}{den draak open en haalde den}{lever te voorschijn}\\

\haiku{De verloren zoon*.}{Er was eens een man en}{die had twee zonen}\\

\haiku{ik heb opgepast.}{terwijl mijn broeder alles}{vertierelierde}\\

\haiku{- Wacht, zei de moeder,.}{ik zal dien stok eens roepen}{om den hond te slaan}\\

\haiku{En hij meende al.}{gelijk dat zij een verzoek}{deed aan de H. Maagd}\\

\haiku{Zij ging en ging en,.}{bleef gaan tot zij aan den voet}{van een hoogen berg kwam}\\

\haiku{- Liever dan hier te,!}{blijven staan wil ik in een}{reiger vergaan}\\

\haiku{Tk zou ook gaarne.}{van dat leelijk ding op mijn}{rug afgeraken}\\

\haiku{Zij bedankten het.}{Manneken voor al wat hij}{voor hen had gedaan}\\

\haiku{Ge kunt denken hoe.}{het Manneken Miserie}{er oin lachen moest}\\

\haiku{Ik ben te ziek en.}{te kramakkelijk om in}{den boom te kruipen}\\

\haiku{Zie, sinds de halve!}{uur dat ik hier zit sterven}{er geen menschen meer}\\

\haiku{De draak spartelde,.}{nog een stondeken maar viel}{daarop mors-dood}\\

\haiku{Daar ze eene tooveres,.}{was zou ze de prinses wel}{uit den weg ruimen}\\

\haiku{Nog had de jongen;}{niet het minste nieuws over zijn}{verloren prinses}\\

\haiku{Allen kwamen weer.}{zonder inlichtingen en}{zonder de prinses}\\

\haiku{Aan den haard zat een,.}{oude kluizenaar die zijn}{paternoster las}\\

\haiku{V\'o\'or den morgen zult.}{gij uw stiefmoeder levend}{moeten verbranden}\\

\haiku{Hij ging maar altijd.}{door want hij wilde naar het}{eind van de wereld}\\

\haiku{Op mijn smeeken heeft.}{hij me een laatste uitstel}{van drie maand verleend}\\

\haiku{De pastoor liep zeer.}{naar de pastorij en Jan}{den Dief achter hem}\\

\haiku{dacht hij dat het waar.}{gebeurde en hij op reis}{naar den hemel ging}\\

\haiku{- Laat maar gaan, ging Jan,.}{den Dief voort klagen  op}{voorhand helpt geen zier}\\

\haiku{Een sukkelaar van,,.}{een pelgrim gelijk ik zal}{uw paard niet stelen}\\

\haiku{- Dat is geen refuus,.}{aan u die z\'o\'o braaf en goed}{zijt geweest voor mij}\\

\haiku{Haal maar water en.}{wat gauw of er worden geen}{koeken gebakken}\\

\haiku{Wanneer ge dan nog.}{te biechten komt hebt ge maar}{keitjes te tellen}\\

\haiku{Naarmate 't vlas.}{opgeraakte bracht hij een}{nieuwen voorraad aan}\\

\haiku{De overeenkomst was.}{gesloten en de duivel}{wipte de deur uit}\\

\haiku{Hij vloekte en liep....}{haver-abaver de deur}{uit        XLVIII}\\

\haiku{Kom met mij naar de.}{stad en ik zal er u een}{deel van meegeven}\\

\haiku{- Ta, ta, ta, 't is,,!}{al meer dan goed En dan ik}{heb geen kogels meer}\\

\haiku{En de roover gaf hem '.}{seffenst geld terug en}{won op den loop gaan}\\

\haiku{- Ik weet het wel, baas,, '.}{antwoordde Jan maar ik ben}{niet voort zoet}\\

\haiku{In 1917 kwam mij een.}{bijna gelijkluidende}{lezing in handen}\\

\haiku{- Halfhaantje vindt een.}{geldbeurs en een man ontsteelt}{ze of ontleent ze}\\

\haiku{I. - Aan den voet van.}{de galg ontkomt hij echter}{door zijn fluitjesspel}\\

\haiku{Van Jaaksken met zijn ( -).}{FluitjeA.B.C. slechts het fluitje}{als toovermiddel}\\

\haiku{De motieven van:}{dit sprookjestype kunnen}{als volgt aangeduid}\\

\haiku{A. Lootens, Oude,:.}{Kindervertelsels blz. 39}{Pietji en Jantji}\\

\haiku{J.F. Vincx, Dit zijn,,:}{grappige Vertelsels en}{Sprookjes I blz. 41}\\

\haiku{De drie Gebroeders,.}{Nr 654 Antti Aarne en}{Maurits de Meyer}\\

\haiku{Maurits de Meyer:}{rangschikt het thema van dit}{vertelsel als volgt}\\

\subsection{Uit: De Vlaamsche vertelselschat. Deel 2}

\haiku{- Hier zal ik wel een,,.}{onderkomen vinden dacht}{ze en zij belde}\\

\haiku{De koning deed de.}{klokken luiden op al de}{torens van het land}\\

\haiku{De varkenshoedster.}{kwam in haar werkkleedij met de}{ezelsmuts op het hoofd}\\

\haiku{Zij dooden den gouden,.}{fezant braadden hem aan het}{spit en aten hem op}\\

\haiku{Zoo was hij dan op.}{het eind van het jaar rijker}{dan de zee diep is}\\

\haiku{Een man tikte hem.}{evenwel op de schouders en}{vroeg waar hij heenging}\\

\haiku{In een oogwenk had.}{hij ze allemaal dood voor}{zijn voeten liggen}\\

\haiku{- Maar hier, zegde zij,,}{schenk ik u tot bewijs dat}{gij mij gered hebt}\\

\haiku{ik voelde dat gij,.}{de appels had afgeplukt}{was ik genezen}\\

\haiku{Een vrouw was bezig.}{met den overschot van den oogst}{in te zamelen}\\

\haiku{Maar ik zal u, in,.}{het naaste dorp er eens een}{staaltje van geven}\\

\haiku{Zooals ge weet, werden.}{de paters er uit verjaagd}{in den Franschen tijd}\\

\haiku{Ineens keerde hij.}{zich om en keek verbaasd naar}{alle kanten uit}\\

\haiku{Iedereen keek hem,.}{na want hij leek aan een die}{geen gewest meer weet}\\

\haiku{De knecht van den boer.}{schoot huilend op den loop en}{ging de wet halen}\\

\haiku{De twee andere.}{bulten zagen dat echter}{met leede oogen aan}\\

\haiku{Een man naderde '.}{ent leek waarachtig of}{de bult daar weer was}\\

\haiku{- Wat zal er van u?}{geworden in de wereld}{als gij niet leeren walt}\\

\haiku{Ik ben genezen!}{en gezond gelijk ik nog}{nimmer ben geweest}\\

\haiku{Als gij er u drie. '}{dagen op te broeden zet}{hebt gij jonge ezels}\\

\haiku{'t Moest een rijke.}{van elders zijn dien zij eens}{tot man zou nemen}\\

\haiku{Beiden lieten een.}{verschrikkelijken schreeuw en}{sloegen op de vlucht}\\

\haiku{Ook ben ik overtuigd.}{dat Albaan u deze drie}{veeren zal brengen}\\

\haiku{Albaan groef op de.}{aangewezen plaats en vond}{den gouden sleutel}\\

\haiku{Albaan nam zulks aan,:}{en die vragen luidden bij}{den tweeden koning}\\

\haiku{Hij zag daar, op een,.}{boogscheut afstand het slot van}{Vogel Veen liggen}\\

\haiku{- Neen, vervolgde het,:}{meisje nu blijft er nog de}{vraag van den veerman}\\

\haiku{Zij wierp zich om den.}{hals van Albaan en kuste}{hem wel duizendmaal}\\

\haiku{- Neen, zoo kunt ge niet,.}{blijven loopen ik zal nog}{eens voor u zorgen}\\

\haiku{Met die woorden trok '.}{t manneken terug naar}{zijn sjees en reed voort}\\

\haiku{En de filosoof.}{moest beslissen wat men er}{mee aanvangen zou}\\

\haiku{Een anderen keer.}{moest de koning een bezoek}{te Kessel brengen}\\

\haiku{Dan hoort ge altijd}{waar hij zit en den dag voor}{dat de koning komt}\\

\haiku{Om twaalf uur kwamen,.}{ze allemaal af ieder}{met twee tellooren}\\

\haiku{Eerst ging de pastoor,,.}{dan de burgemeester dan}{de twintig boeren}\\

\haiku{Op eens echter kwam:}{de baas vol alteratie}{binnengeloopen}\\

\haiku{En als dat niet pakt,}{beginnen ze te vloeken}{en te sakkeren}\\

\haiku{Zij dronken een pint:}{en rookten een pijp en dan}{zegde het smidje}\\

\haiku{- Maar ik, ik, ik vloog.}{tot het uiterste puntje}{van de eeuwigheid}\\

\haiku{We kwamen aan een.}{rivier waar we over moesten langs}{een smal bruggetje}\\

\haiku{En als ik opsta,.}{moet ge er ook uit of de}{duivel houdt de kaars}\\

\haiku{Sinds twee dagen had.}{het geen grummelken eten over}{de lippen gehad}\\

\haiku{Gestolen is 't! -,,, '.}{Goed goed pastoor z\'o\'o moet ge}{t nu maar zeggen}\\

\haiku{En 't was eten van.}{goud en drinken van goud dat}{men haar voorzette}\\

\haiku{Maar ginds zie, in 't,,.}{aardsch paradijs staat een boom}{de boom van de spraak}\\

\haiku{Daarom moet ge me.}{laten leven en terug}{in de zee werpen}\\

\haiku{- Mijn vrouw zou graag in.}{een steenen huis wonen met een}{grooten hof er aan}\\

\haiku{Zijn vrouw woonde nu.}{in een prachtig steenen huis met}{een grooten hof}\\

\haiku{De vrouw echter was. '.}{het niett Leek wel of er}{haar nog wat ontbrak}\\

\haiku{'t Manneken kwam,.}{gelukkig thuis maar zijn vrouw}{was niet gelukkig}\\

\haiku{En toen zij paus was,,...}{wilde zij nog meer worden}{ja meer nog dan Paus}\\

\haiku{Na elk schot raapte.}{hij iets op van den grond en}{stak het in zijn zak}\\

\haiku{De ingeslapen.}{looper sprong wakker van het}{daverend gerucht}\\

\haiku{Hij opende al zijn.}{zakken en joeg de muggen}{naar de soldaten}\\

\haiku{Geen een die tijdens.}{zijn levensjaren nog aan}{werken moest denken}\\

\haiku{- Dan zal het haantje,.}{geen hartje gehad hebben}{zei de schoenmaker}\\

\haiku{Beiden peuzelden.}{het haantje op en togen}{in stilte verder}\\

\haiku{De schoenmaker die.}{zulks vernomen had kwam het}{zeggen aan Ons Heer}\\

\haiku{Maar de nacht die kwam.}{geschiedde alles gelijk}{den vorigen nacht}\\

\haiku{Dan bleef hij in zijn, '.}{bed wakker liggen als een}{muisje int meel}\\

\haiku{Eer men 't wist lag.}{zij aan den steenen trap van het}{koninklijk paleis}\\

\haiku{Enkele stonden.}{later kwam ook de prinses}{de kamer binnen}\\

\haiku{'t Leek wel of hij.}{uren ver werd weggedragen}{door een wervelwind}\\

\haiku{En hij ging naar de.}{reuzen en werd door hen als}{een broer ontvangen}\\

\haiku{Na lang zoeken vond.}{hij er een hut waarin een}{kluizenaar woonde}\\

\haiku{Als gij dien hond doodt.}{is al wat in de kamer}{ligt uw eigendom}\\

\haiku{En daar stond, op den,.}{vliegenden minuut een zak}{goud v\'o\'or zijn voeten}\\

\haiku{ze den volgenden.}{morgen op het eerste uur}{te doen optrekken}\\

\haiku{riep de kapitein,.}{toen de witte gedaante}{zich voor hem ophief}\\

\haiku{Welnu, raadt, raadt wat?}{er het schoonste is in mijn}{tuin en mijn kasteel}\\

\haiku{Zijne vrouw had het.}{schot gehoord en kwam eens zien}{wat er gebeurd was}\\

\haiku{De schoenmaker ging.}{aan zijne vrouw vertellen}{wat er gebeurd was}\\

\haiku{De schoenmaker trok.}{het zich niet verder aan en}{Het het lijk liggen}\\

\haiku{Toen hij weer lange,.}{dagen gegaan had kwam hij}{voorbij een moeras}\\

\haiku{Hij schudde met de.}{beurs zoo gelijk hij het de}{mannen had zien doen}\\

\haiku{Hij zal hij daar wel,.}{voor te vinden zijn als wij}{hem vijf frank geven}\\

\haiku{Hij nam de maat en.}{de schaar want hij moest de stof}{ter plaatse snijden}\\

\haiku{Ten slotte sneed hij, '.}{er nog een stuk af dat hij}{doort venster wierp}\\

\haiku{- Sint Antonius.}{geef mij eens twee frank om naar}{de kermis te gaan}\\

\haiku{- Wat is hier gebeurd,,?}{zegde deze leeft die man}{nog of is hij dood}\\

\haiku{Haar {\textquoteleft}venken{\textquoteright}3 zocht met,,,.}{veel gedruisch Trap op trap}{af door heel het huis}\\

\haiku{Slaat uw oogen neer, 't.}{is de heilige geest die}{nu nederdaalt}\\

\haiku{En deze won het.}{en kreeg de honderd kronen}{van den koning}\\

\haiku{De koster had, ten,.}{slotte een gedachte die}{aangenomen werd}\\

\haiku{De duivel liet zich.}{in den zetel vallen en}{Smidje Smee lachte}\\

\haiku{Daarop nam hij den,.}{ring van den vinger zoodat hij}{opnieuw zichtbaar werd}\\

\haiku{Hij is in een ver.}{land en aan de menschen vraagt}{hij den weg naar huis}\\

\haiku{Daar stond die koning,.}{nu terug in zijn eigen}{land maar heel alleen}\\

\haiku{*~         Er was eens een,.}{jongen op het dorp die niet}{van de slimste was}\\

\haiku{- Een mensch is wel geenen,,.}{vorsch zei Jan maar hij springt}{ook al eens gaarne}\\

\haiku{De schoonste vrouw van,.}{heel het land Is Sneeuwwitje}{rein Zoo teer en fijn}\\

\haiku{De schoonste vrouw van,.}{heel het land Is Sneeuwwitje}{rein Zoo teer en fijn}\\

\haiku{De booze vrouw stak hem:}{zelf in de lokken van het}{arm meisje en zei}\\

\haiku{Als de kabouters.}{nu thuis kwamen schoten zij}{dadelijk ter hulp}\\

\haiku{Het aan dit thema:}{verwante sprookje van de}{gebroeders Grimm heet}\\

\haiku{Verteld te Boom, in,,,.}{1890 door vrouw J.D.K. waschvrouw}{geboren te Ranst}\\

\haiku{Vader geboren,.}{te Wuestwezel moeder}{van Duitsche afkomst}\\

\haiku{Rond den Heerd, II, blz.: ',:}{79t Manneken uit de}{Mane V. blz. 262}\\

\haiku{'t Vertelselke ',,:}{vant Manneke uit de}{Mane VII blz. 27}\\

\haiku{Wodana, blz. 47 en,,:}{J.W. Wolf Deutsche M\"archen}{und Sagen blz. 105}\\

\haiku{der Tod zu F\"ussen ().}{des Krankendas Bett oder den}{Kranken umgelegt}\\

\haiku{der Schmied wird weder.}{in den Himmel noch in die}{H\"olle gelassen}\\

\haiku{Een dergelijke.}{lezing werd in Vlaanderen}{nog niet geboekt}\\

\subsection{Uit: De Vlaamsche vertelselschat. Deel 3}

\haiku{'t Scheen wel of er.}{daar een stoet van engelen}{kwam aangezwommen}\\

\haiku{Den vierden dag, toen,.}{hij opstond was hij als dood}{van honger en dorst}\\

\haiku{Wij moeten dien ring.}{bemachtigen en zullen}{dan even machtig zijn}\\

\haiku{Nu ge toch niet naar,.}{de school moet kunt ge er u}{eens goed vermaken}\\

\haiku{Al gaande sloeg hij.}{zijn lokken van links naar rechts}{en van rechts naar links}\\

\haiku{- Kruip eens rap in den,,.}{oven zei ze en zie eens of}{hij heet genoeg is}\\

\haiku{Hij trok recht naar de.}{Hel om zijn voornemen ten}{uitvoer te brengen}\\

\haiku{Eerst moest hij weten.}{wat er daar onder  den}{grond ruizemuisde}\\

\haiku{Buiten gekomen.}{floot hij op de drie paarden}{van de drie ridders}\\

\haiku{De juffrouw dacht niet.}{anders of Jan de Koeter}{zou niet weerkomen}\\

\haiku{- Welnu, zei de heer,.}{tegen zulken baas heb ik}{niets in te brengen}\\

\haiku{De tweede zoon ging.}{steeds naar het oosten en kwam}{aan het Zilverland}\\

\haiku{Eindelijk zag hij,,.}{op een klein zandheuveltje}{een witte roos staan}\\

\haiku{Hij stapte af en.}{ging er een wandeling doen}{in de groote bosschen}\\

\haiku{Vier maanden later.}{kwam hij dan ook in de stad}{van zijn vader aan}\\

\haiku{Het was twaalf uur in.}{den nacht en daarom ging hij}{in den hof slapen}\\

\haiku{Alles werd door den.}{vijand gebombardeerd en}{kapotgeschoten}\\

\haiku{De prins nam al de.}{lapjes van den neusdoek en}{smeet ze overal heen}\\

\haiku{*~         Koning Alexander.}{had van boven op het hoofd}{een grooten hoorn staan}\\

\haiku{Zij besloten die.}{geboorte te vieren met}{een wafelenbak}\\

\haiku{Zoo ging het meisje,,.}{voort zoo ver zoo ver tot het}{eind van de wereld}\\

\haiku{Zij trok daarop naar,.}{de overzij waar Madam de}{Maan haar kasteel stond}\\

\haiku{- Ik moet de zeven.}{kauwkens spreken die hier in}{den IJsberg wonen}\\

\haiku{Het meisje had slechts.}{den tijd zich even achter de}{deur te verbergen}\\

\haiku{Eens gebeurde het.}{dat Toon in de Schelde een}{bad wilde nemen}\\

\haiku{Toon bleef zitten aan, '.}{den kant altijd maar diep in}{t water kijkend}\\

\haiku{De kinderen van.}{de buurt speelden daar juist en}{maakten veel lawaai}\\

\haiku{Jan was de zoon van.}{een weduwe die op een}{klein pachthoef woonde}\\

\haiku{- Te naasten keer zal ',.}{t beter gaan antwoordde}{de jongen daarop}\\

\haiku{Eens moest de moeder.}{van slimme Jan enkele}{kiekens verkoopen}\\

\haiku{En die boer deed zoo, ',.}{maar alst oogsttijd was zat}{hij met leege armen}\\

\haiku{- Kom, Heer, we trekken,,.}{weg zei Sinte Pieter die}{beefde als een riet}\\

\haiku{Een weinig verder.}{kwam daar een man aan met een}{mand rijpe kersen}\\

\haiku{- En wat hebben de,?}{boeren uit den Polder van}{mij gezeid Pieter}\\

\haiku{Sinte Pieter en.}{Sint Jan mochten beiden hun}{rechten doen gelden}\\

\haiku{Hij had al de kracht.}{van zijn lijf noodig om het net}{boven te halen}\\

\haiku{Gij, die niets bezit,.}{zult niets bijbrengen als ik}{morgen te kort kom}\\

\haiku{Daarop liep hij recht.}{naar het paard om het naar zijn}{stalling te brengen}\\

\haiku{En zie, hij vond hem.}{veel verder dan zijn oudste}{broer geschoten had}\\

\haiku{De duif, waarvan de,.}{prinses gesproken had vloog}{voor hem de lucht in}\\

\haiku{eten en drinken naar,!}{ons hartje lust zonder dat}{het ons een cent kost}\\

\haiku{De baas, die zulks in ',,:}{t oog kreeg vond dat aardig}{kwam buiten en vroeg}\\

\haiku{- Ge moet weten, zei,.}{de student dat het kermis}{is op het kasteel}\\

\haiku{Eindelijk bood zich,.}{Siepe aan verkleed in een}{Spaanschen geneesheer}\\

\haiku{De kabouters moesten ',,.}{s avonds als loon een groote teil}{zoete melk hebben}\\

\haiku{- En nochtans, ging hij,,}{weer voort daar waar ik mijn sprong}{nam had ik het nog}\\

\haiku{- Misschien had ik het.}{daar toch nog en is het op}{den grond gevallen}\\

\haiku{Een boer, die daar wat ',.}{verder aant zaaien was}{kwam toegeloopen}\\

\haiku{Geef mij uw riek, ik,.}{zal zelf zoeken ik weet best}{wat ik hebben moet}\\

\haiku{En hij klom op een,...-}{boom met zijn twee molensteenen}{en zijn koevel}\\

\haiku{'t Bleef  echter.}{aan een der laagste takken}{van den boom hangen}\\

\haiku{En de man ging de.}{brandende kool halen en}{lei ze in den haard}\\

\haiku{die had een stok in,.}{zijn mond met een gat erin}{en daar kwam rook uit}\\

\haiku{De jeugd was er wars '.}{vant werk en alleen op}{spel en dans verzot}\\

\haiku{Hij trok de groote baan.}{op en al de koppels hem}{dansend achterna}\\

\haiku{De baren sloegen,.}{toe speleman en dansers}{ineens verzwelgend}\\

\haiku{t sloot zich zoo dicht,.}{achter hem toe dat er geen}{mensch meer door en kon}\\

\haiku{'t Was altijd een.}{en hetzelfde antwoord dat}{er gegeven werd}\\

\haiku{Haast u nu naar huis.}{en wees verstandig in uw}{doen en handelen}\\

\haiku{Ha, ik begrijp het,.}{die herbergier is u nog}{eens te plat geweest}\\

\haiku{\'en uw tafelken,.}{kunt weerkrijgen als ge maar}{uit uw oogen wilt zien}\\

\haiku{Ja, de bol kreeg een.}{eereplaats in de beste}{kamer van hun huis}\\

\haiku{En de boer liet het.}{wit konijn in gewijden}{grond begraven}\\

\haiku{Joosken bond den zak:}{stevig dicht en de herder}{begon te roepen}\\

\haiku{Altijd-aan, dag en,.}{nacht had hij op menschen en}{dingen nagedacht}\\

\haiku{De koningszoon zag, '.}{nu duidelijk dat de heks}{aant sterven was}\\

\haiku{Zoo sprak de heks met. '}{stille stem en nam daarop}{den derden spiegel}\\

\haiku{Hij spoedde zich naar,:}{huis sleurde de geit binnen}{en riep op zijn vrouw}\\

\haiku{Na nog lang op hem,.}{gewacht te hebben viel hij}{eindelijk in slaap}\\

\haiku{Ik, voor mijn paart, ik;}{neem het poeder mee dat doof}{maakt en stom en blind}\\

\haiku{'t Was meteen een '.}{herrie van belang int}{vijandelijk kamp}\\

\haiku{De koning, zooals gij,.}{wel denken kunt was daar zeer}{ongelukkig over}\\

\haiku{Ga u verhuren,.}{bij den reus die ginder op}{den Wolkenberg woont}\\

\haiku{Als ge dat doet, zal.}{het u de schuilplaats van de}{prinses doen kennen}\\

\haiku{Zij zit gevangen.}{in den ondersten kerker}{tegen den vijver}\\

\haiku{Gloeiende kolen,,.}{gloeiende kolen moet ik}{hebben anders niet}\\

\haiku{Met een slag van zijn.}{sabel sloeg Fluppen den bol}{in duizend stukken}\\

\haiku{Hij droeg evenwel goed.}{zorg zijn kluppeltje onder}{den arm te houden}\\

\haiku{Nog vele landen.}{en nog meerdere steden}{moest hij doortrekken}\\

\haiku{Daarom heb ik reeds.}{een boontje voor u en zal}{ik u aanhooren}\\

\haiku{Hij ronkte meteen, '.}{dat men hem op een uur in}{t rond kon hooren}\\

\haiku{*~         Er was eens een.}{arme weduwe en die}{had maar eenen jongen}\\

\haiku{Hij was nog maar pas.}{het dorp uit of hij kwam een}{ouden ezel tegen}\\

\haiku{Jan, al had hij maar,.}{weinig gaf de vijf centen}{en de hond liep mee}\\

\haiku{Ik zat daar onder '!}{t stoelken van ons meken}{te spinnen en krak}\\

\haiku{Nu, als de wereld,.}{gaat vergaan ben ik liever}{op de wijde baan}\\

\haiku{De haan vloog van den.}{boom en trok met Jan en zijn}{kameraden mee}\\

\haiku{Iedereen sprong op,.}{want zij meenden allemaal}{dat er onraad was}\\

\haiku{- 't Is een licht dat.}{men ginder ver in een huis}{aangestoken heeft}\\

\haiku{Ge weet wel wat voor,:}{een spel dat is al wordt het}{nu niet meer gespeeld}\\

\haiku{Meteen hoorde hij.}{dan weer den man achter hem}{dapper doorstappen}\\

\haiku{De kwezel pakte.}{haar dikken kerkboek en wierp}{hem achter de kast}\\

\haiku{Heel hun leven lang.}{hadden die twee menschen naar}{kinderen getracht}\\

\haiku{Zij waren beiden '.}{zoo gelukkig dat zijt}{niet zeggen konden}\\

\haiku{'t Was of er een.}{jongetje van rond de vier}{jaar in hun bed lag}\\

\haiku{De goede arme.}{vrouw besloot het kindje mee}{naar huis te nemen}\\

\haiku{Een pintje voor den.}{reus was als een groote ton voor}{een gewonen mensch}\\

\haiku{Heeft er ooit iemand?}{regelmatiger dan ik}{zijn tol aanbetaald}\\

\haiku{Zij hadden reeds van.}{het Goudland hooren spreken}{en trokken er heen}\\

\haiku{Maar de manschappen.}{met de pieken stonden hun}{heeren dapper bij}\\

\haiku{Als het regende;}{was het reukwater en als}{het sneeuwde suiker}\\

\haiku{Maar toen meteen had,.}{hij dorst gekregen dorst lijk}{honderdduizend man}\\

\haiku{Hij ontwaakte op,.}{den dijk van de rivier daar}{dicht tegen zijn dorp}\\

\haiku{Duimelingsken en;}{de Wolf en de gewone}{sprookjes van Duimken}\\

\haiku{Dit sprookje werd mij.}{door den heer Florimond Van}{Duyse meegedeeld}\\

\haiku{der Knabe gem\"astet die (:}{Hexe in den Ofen geworfen}{Vergelijk nr 1121}\\

\haiku{Het overeenstemmend:}{sprookje bij de Gebroeders}{Grimm draagt als titel}\\

\haiku{- De Duivel en de ().}{HouthakkerSprookjes van den}{dankbaren duivel}\\

\haiku{Van Kluppelken uit,;}{den Zak en Van het Vrouwken}{en heur Kanneken}\\

\haiku{Vlaamsche Moppen van,;}{Victor de Meyere en}{Leo Verkein nr 11}\\

\haiku{H. Befreiung aus ();}{dem SackeKasten durch Tausch}{mit einem Hirten}\\

\haiku{Vertelling van de,,,,;}{Kat den Hond de Zwaan de Koe}{het Peerd en den Haan}\\

\haiku{waar zij voldoende.}{verklarend zijn worden de}{thema's niet vermeld}\\

\haiku{- De boerenzoon komt:}{van den troep en gebaart geen}{Vlaamsch meer te kennen}\\

\haiku{Daarna zijn vijand ':}{in den zak steken en in}{t water werpen}\\

\haiku{Hoe Onze Lieve,,.}{Heer den Duivel machteloos}{maakte III nr 187}\\

\subsection{Uit: De Vlaamsche vertelselschat. Deel 4}

\haiku{En hij zag al de!}{sterren van den hemel v\'o\'or}{zijn oogen flikkeren}\\

\haiku{wauw! deed de hond, en.}{hij beet alom  waar hij}{den wolf kon grijpen}\\

\haiku{De kraai en de puit*:}{De kraai zat aan den kant}{van eenen put en riep}\\

\haiku{Eindelijk kwam de. ' '.}{wintert Sneeuwde ent}{vroor dat het kraakte}\\

\haiku{Na lang zoeken vond.}{hij hem eindelijk in een}{verlaten spelonk}\\

\haiku{- 't Is waar, zei de,,.}{bie ik heb het moeilijk maar}{ik doe niemand kwaad}\\

\haiku{Het woog z\'o\'o zwaar, dat.}{hij al zijn macht noodig had om}{het op te halen}\\

\haiku{De vorschen maken.}{nu voortdurend kennis met}{zijn koningswet}\\

\haiku{Zoo konden ratten.}{en muizen zich bijtijds uit}{de voeten maken}\\

\haiku{zei Sinte Pieter,.}{die altijd goed betrouwen}{had in de menschen}\\

\haiku{Zij dringen in al.}{de woningen en spelen}{er heer en meester}\\

\haiku{De haas was aan 't.}{napeinzen hoe dat men de}{afzetting zou doen}\\

\haiku{Kom met mij mee, ik.}{weet een schoonen kaas zitten}{op gindsche hoeve}\\

\haiku{Nu eerst ging de musch,}{voor een goei te werk schreeuwend}{en huilend gelijk}\\

\haiku{De twee laatste die.}{aan de beurt kwamen waren}{de musch en de eend}\\

\haiku{het had voorzegd, maar.}{de uil heeft de woorden van}{den gier onthouden}\\

\haiku{- Tot uw straf zult gij.}{voortaan geen tien meter ver}{meer kunnen vliegen}\\

\haiku{En de ekster en.}{de tortelduif begonnen}{hun nest te bouwen}\\

\haiku{Mijn nest mag ik niet.}{verlaten of mijn jongen}{sterven in de schaal}\\

\haiku{Onze moeder heeft.}{sneeuwwitte pootjes en de}{uwe zijn zwart als roet}\\

\haiku{Op den gestelden.}{dag stond hij met zijn wonder}{dier op den kruisweg}\\

\haiku{daarom kwam hij maar,.}{met zijn eigen beest vooruit}{om tijd te winnen}\\

\haiku{De duivel voelde,.}{zich verloren vloekte en}{wilde wegloopen}\\

\haiku{- Ziedewel, dacht het,.}{schaap mijn eerste gedachte}{was de beste}\\

\haiku{Daartoe hadden zij.}{allerhande steenen op een}{groote baan saamgebracht}\\

\haiku{En bovendien heeft.}{zij drie duivelsharen op}{haren kop staan}\\

\haiku{'t Duurde echter,}{niet lang of er kwamen er}{al naar beneden}\\

\haiku{Tot koning ben ik, ' '.}{verheven Enk blijft}{mijn leven lang}\\

\haiku{Lang vloog zij en haar.}{aankomst meldde zij met een}{nijdig hoorngeschal}\\

\haiku{En de vos was de,.}{pijp uit maar bleef van verre}{op den loer liggen}\\

\haiku{Het koningsken bleef.}{diep in het hol zitten en}{kikte noch mikte}\\

\haiku{De arend ontving het.}{beestje met fatsoen en dacht}{een oogenblik na}\\

\haiku{Van toen af, werd de.}{vijgeboom een heilige}{boom geheeten}\\

\haiku{Sindsdien worden er. '}{madeliefjes met roode}{vlekken gevonden}\\

\haiku{En de sneeuw ging tot.}{de roode kollebloem en}{vroeg haar roode kleur}\\

\haiku{Erger ging het met!}{diegenen die een tijdlang}{geloopen hadden}\\

\haiku{Verteld te Hamme,,,.}{in l909 door S.B. een meisje}{van Elverzele}\\

\haiku{Ons Volksleven, II,,,.}{blz. 126 P.J. Cornelissen}{en J.B. Vervliet Vl}\\

\haiku{Pourquoi les chats se;}{lavent-ils la figure}{quand ils ont mang\'e}\\

\haiku{Der Fuchs verleitet;}{den Hahn mit geschlossenen}{Augen zu kr\"ahen}\\

\haiku{Pol de Mont en A.,,:}{de Cock Zoo vertellen de}{Vlamingen blz. 19}\\

\haiku{J. Cornelissen,,:}{en J.-B. Vervliet Vlaamsche}{Vertelsels blz. 225}\\

\haiku{Van Triene Giet (slechts);}{twee van de drie motieven}{die wij meedeelen}\\

\haiku{- Waarom de Menschen.}{in alle Richtingen over}{de Wereld loopen}\\

\haiku{waar zij voldoende,.}{verklarend zijn worden de}{thema's niet vermeld}\\

\haiku{De haan vleit de gaai:}{en gelukt er in den buit}{voor zich te houden}\\

\haiku{De wolf in don put ():}{de weerkaatsing van de maan}{gelijkt aan een kaas}\\

\haiku{-Wat de Mug in ',.}{t Kamp van de loopende}{Dieren verneemt 425}\\

\section{Herman Middendorp}

\subsection{Uit: De schaduw van Capoulet}

\haiku{Mijnheer de graaf had,.}{geloof ik gedacht dat u}{vroeger zou komen}\\

\haiku{de overvloedige,.}{regen verminderde maar}{hield niet geheel op}\\

\haiku{Natuurlijk, ik wist,....}{zeker dat ik de deurknop}{had zien bewegen}\\

\haiku{de onheilen, die,.}{zich later voltrokken had}{moeten voorvoelen}\\

\haiku{Ik keek naar buiten,;}{over de golvingen van de}{beboschte bergen}\\

\haiku{Het beloofde een.}{mooie na-zomersche dag}{te zullen worden}\\

\haiku{Nu wil het verhaal,;}{dat zij na haar dood geen rust}{heeft kunnen vinden}\\

\haiku{Het is een eenigszins.}{pijnlijke geschiedenis}{met dien jongeman}\\

\haiku{Monique had een,.}{brief in de hand dien ze haar}{vader overreikte}\\

\haiku{hij wou hem aan u,,.}{brengen en toen zei ik dat}{ik het wel doen zou}\\

\haiku{eerst tegen half acht;}{behoefde ik weer thuis te}{zijn voor het avondeten}\\

\haiku{ik zal den graaf er,.}{van op de hoogte stellen}{wat er gebeurd is}\\

\haiku{We hebben hier wel ',}{middelen om u aant}{spreken te krijgen}\\

\haiku{In dat geval zou.}{de bende dus uit nog m\'e\'er}{personen bestaan}\\

\haiku{{\textquoteright} zei hij zacht, {\textquoteleft}als we,.}{gesnapt worden vermoorden}{ze ons alle twee}\\

\haiku{Het meisje, dat voor,;}{mij uitging keek eerst spiedend}{naar beide kanten}\\

\haiku{Zeg Donia, heb jij?}{vandaag niet weer iets vreemds aan}{De Fontenay bemerkt}\\

\haiku{als er vanavond nog,.}{geen hulp komt zal ik Parijs}{moeten opbellen}\\

\haiku{Dat wou ik nu weer,,;}{probeeren maar het is luk-raak}{als ze hem vinden}\\

\haiku{Dan kunnen we straks,.}{met ons drie\"en zien wat we}{verder zullen doen}\\

\haiku{Op den grond, naast de,.}{tafel lag het lichaam van}{den graaf De Fontenay}\\

\haiku{Ik zweeg er over,  .}{dat ik Vergniaud naast}{het huis gezien had}\\

\haiku{Met de handen op.}{den rug liep de maire de}{kamer op en neer}\\

\haiku{er is geen bezwaar,.}{tegen dat de klok van het}{kasteel wordt geluid}\\

\haiku{{\textquoteright} {\textquoteleft}Ja,{\textquoteright} zei ik, eenigszins, {\textquoteleft}.}{verbaasd over deze vraagwij}{zijn goede vrienden}\\

\haiku{Hij vertelt veel goeds,.}{van u. En ik geloof dat}{hij zich niet vergist}\\

\haiku{Hij maakte hem niet,,.}{open waar we bij waren maar}{ging de kamer uit}\\

\haiku{Het adres was slordig,,.}{geschreven met groote krullen}{maar zonder fouten}\\

\haiku{Mijnheer de graaf keek.}{vluchtig naar het adres en stak}{den brief in zijn zak}\\

\haiku{{\textquoteleft}Maar ik vermoed, dat.}{we dat bewijsstuk ook niet}{noodig zullen hebben}\\

\haiku{{\textquoteright} {\textquoteleft}Nu, heb jij zelf wel,?}{eens wat gezien hier in huis}{dat je bang maakte}\\

\haiku{trouwens, de heele,,.}{manier waarop hij sprak was}{mij antipatiek}\\

\haiku{{\textquoteleft}Hoe laat was het, toen,?}{u het schot hoorde dat op}{den graaf gelost werd}\\

\haiku{{\textquoteleft}U was zeker wel,,}{verbaasd toen mijnheer Donia}{u dat vertelde}\\

\haiku{Ongerust keken.}{professor Chalosse en}{ik elkander aan}\\

\haiku{Op de vraag, wat haar,.}{broer voor den kost deed zei ze}{dat hij handel dreef}\\

\haiku{Bij de deur hield de.}{heer Armandy mij nog een}{oogenblik terug}\\

\haiku{Hij werkt samen met,{\textquoteright}.}{den Service de S\^uret\'e}{vervolgde Firmin}\\

\haiku{Er klonk een kreet in,.}{de stilte onmiddellijk}{gevolgd door een schot}\\

\haiku{Direct werd het mij,.}{duidelijk dat hier geen hulp}{meer te bieden was}\\

\haiku{Wij lieten hem in.}{zijn alteratie achter}{en gingen naar huis}\\

\haiku{{\textquoteright} {\textquoteleft}Wat zou u gedaan,,?}{hebben professor als hij}{het niet geweest was}\\

\haiku{we weten was hij,.}{de eenige die een hekel}{aan mijn vader had}\\

\haiku{De misdadiger,,}{kan dicht bij het huis gestaan}{hebben toen hij schoot}\\

\haiku{De ramen zijn niet,.}{hoog niet meer dan een voet of}{vier boven den grond}\\

\haiku{De professor liet.}{Pernod brengen en dronk er}{twee groote glazen van}\\

\haiku{{\textquoteright} Terloops merk ik hier,.}{even op dat ik dit stopwoord}{van Crampton kende}\\

\haiku{Natuurlijk was hij,;}{bang dat hij dan zelf ook in}{de val zou loopen}\\

\haiku{De detective.}{ging voort met boeken van de}{planken te nemen}\\

\haiku{{\textquoteright} Men kon duidelijk,.}{zien dat de bladzijden er}{uit gescheurd waren}\\

\haiku{De detective;}{en de inspecteur kropen}{in het struikgewas}\\

\haiku{hij alleen kende.}{het huis en het bestaan van}{het famieliboek}\\

\section{P.H. van Moerkerken jr.}

\subsection{Uit: De bevrijders}

\haiku{In de gesprekken;}{der ouderen sprankelde}{vernuftige scherts}\\

\haiku{In de trekken van.}{het blonde meisje kon zij}{echter niet lezen}\\

\haiku{op wiens vol gelaat;}{een sterke overtuiging van}{eigenwaarde lag}\\

\haiku{er waren landen\%,.}{die geen 20 er waren er}{die niets betaalden}\\

\haiku{Zijn plan om hen in;}{Holland op te zoeken werd}{telkens uitgesteld}\\

\haiku{Doch Ter Wisch zag dat.}{zijn kleine groene ogen naar}{Th\'er\`ese loerden}\\

\haiku{Maar de heftigste;}{revolutionnairen}{hadden zich verzet}\\

\haiku{Kort na den middag.}{reed de wagen het voorplein}{van Den Ulenhoek op}\\

\haiku{zij wist de hoge.}{waarde van het verborgen}{leven des harten}\\

\haiku{het opdringend volk,.}{beschimpte hen bauwde hun}{vreemden tongval na}\\

\haiku{Anne-Marie.}{ontving haar broeder met een}{vermoeiden glimlach}\\

\haiku{In het geboomte;}{der hofsteden langs den weg}{zongen  vogels}\\

\haiku{tijd om naar de stad,,.}{naar Haarlem te gaan hadden}{zij geen van beiden}\\

\haiku{een leeuwerik hoog;}{boven de weiden en de}{verre wateren}\\

\haiku{Zij moet een man van, ... ...}{talenten hebben en zij}{is voor mij te oud}\\

\haiku{Zijn innigsten wens.}{vernam hij nu uit den mond}{van een vriend als raad}\\

\haiku{Bovendien, hij was;}{niet presentabel op een}{huwelijksaanzoek}\\

\haiku{Schrijf dit over, met je{\textquoteright}}{elegantste hand. In zoeten}{droom liet Tobias}\\

\haiku{Toen Floris de deur.}{had horen dichtslaan barstte}{hij in lachen uit}\\

\haiku{Op de kermis had.}{zij dien heer in den groenen}{rok om geld gevraagd}\\

\haiku{De twe vuren  ;}{rustten in evenwicht aan de}{einden der aarde}\\

\haiku{Maar zulk een kind, ... het;}{was nieuw voor hem en hij gaf}{er toch zijn geld voor}\\

\haiku{Het verdriet mij zeer.}{dat ik u een deceptie}{moet veroorzaken}\\

\haiku{De Katholieken,{\textquoteright}.}{zouden het niet met u eens}{zijn hernam Aagje}\\

\haiku{Hij stond op en liep.}{onrustig in het grote}{vertrek heen en weer}\\

\haiku{al zijn vrienden bij.}{den vijfden Beurspilaar spraken}{er hun vrees over uit}\\

\haiku{{\textquoteright} Maar Floris boog zich:}{over het tafeltje en gaf}{er een vuistslag op}\\

\haiku{Zo-even had hij;}{de zakjes met dukaten}{uit de kluis gehaald}\\

\haiku{s Avonds riep Santje.}{met zwakke stem haar broeder}{Bart bij de bedste\^e}\\

\haiku{En hij schreef haar over;}{de rampen die vrouw Breevoort}{hadden getroffen}\\

\haiku{Zij antwoordde niet,;}{maar gaf hem brandewijn en}{zelfgebakken brood}\\

\haiku{Doch zij herkenden.}{den weg aan de donkere}{plekken der lijken}\\

\haiku{Zelfs het geschut van.}{den vijand wekte hem niet}{uit zijn mijmering}\\

\haiku{naar wezen was zo.}{anders dan van haar die hem}{nu vergezelde}\\

\haiku{En zo verzoelde;}{een weldadige warmte}{den Decembernacht}\\

\haiku{David zelf verscheen ' ';}{alleens morgens ens}{avonds aan het ziekbed}\\

\haiku{Den Zaturdagavond.}{en den Zondag bracht hij op}{Wijckervelt door}\\

\haiku{Des Maandagmorgens,,.}{den 22sten was hij weer in de}{Kalverstraat terug}\\

\haiku{leefde zij nog, was,? ...}{zij wellicht getrouwd wist zij}{nog van zijn bestaan}\\

\haiku{de oude zaak van.}{zijn vader en grootvader}{mocht niet verlopen}\\

\haiku{Naar dien man had zij,;}{verlangd om zijn leven was}{zij angstig geweest}\\

\haiku{Aagje Fabian.}{en Jacob ter Wisch zagen}{elkander niet meer}\\

\haiku{De schone vrede ...}{van een leven vol liefde}{was toch niet voor hem}\\

\haiku{de rivieren staan,,;}{niet stil de zee golft aldcor}{de wolken jagen}\\

\haiku{Aan den rand van het}{woud gekomen hoorde hij}{op de hoogvlakte}\\

\haiku{hij glimlachte om}{de anderen die in de}{dwaze verblinding}\\

\haiku{het jonggestorven;}{hoofd was met den klassieken}{lauwerkrans getooid}\\

\haiku{De beide vrouwen.}{luisterden met in den schoot}{gevouwen handen}\\

\haiku{zorgvol zag zij naar,;}{haar kleinen grond waarover zij}{alleen nu waakte}\\

\subsection{Uit: De ondergang van het dorp}

\haiku{wier kruinen het verst.}{zichtbaar waren uit heide}{en akker en vloed}\\

\haiku{Eindelijk drong de.}{leer der Hervorming in de}{naastbije steden door}\\

\haiku{Doch in het volk bleef.}{nog lang het ruwe gemoed}{der oorlogstijden}\\

\haiku{Tien jaren hadden,.}{zij daar geleefd eer hun een}{zoon geboren werd}\\

\haiku{met de rechter hield,.}{hij een klein in perkament}{gebonden boekje}\\

\haiku{Zij arbeidden elk.}{aan een historie hunner}{beminde landstreek}\\

\haiku{De arme stomme.}{was opgestaan aan de hand}{van moeder Tuinder}\\

\haiku{ik dacht aan mijn jeugd.}{en aan alles wat ik toen}{hoopvol en mooi vond}\\

\haiku{het nauwgeboren,;}{licht over de oude akkers}{de oude stulpen}\\

\haiku{Hij keerde zich om.}{en kwam de volgende twe}{maanden niet buiten}\\

\haiku{van S. Thomas den:}{Apostel en het mirakel}{zijner relikwie}\\

\haiku{Verveloos was het,.}{hout der kozijnen verweerd}{de kleine ruiten}\\

\haiku{{\textquoteleft}Het was niet de wil,.}{van de boeren maar de wil}{van den Heilige}\\

\haiku{{\textquoteright} Pastoor Hedel zag.}{even naar de grijze urnen}{op het kabinet}\\

\haiku{Zijn grootvader had,.}{hem die verhalen gedaan}{voor een halve eeuw}\\

\haiku{Toen stond hij langzaam.}{op en pakte ordeloos}{zijn gerei bijeen}\\

\haiku{De notabelen:}{van Aarloo en Nierode}{traden toe als lid}\\

\haiku{In dien nacht klopte.}{hij aan de lage deurtjes}{bij de slaapsteden}\\

\haiku{de bewijzen voor;}{de burgemeesterlijke}{bevoegdheid gevraagd}\\

\haiku{hij had er oude.}{komforen neergezet en}{tinnen asbakjes}\\

\haiku{of niet zijn zachte.}{en vaste leiding te zeer}{een dwang was geweest}\\

\haiku{De zoon dacht aan  ,.}{de toekomst aan het leven}{dat nu eerst begon}\\

\haiku{{\textquoteright} {\textquoteleft}En als de ruimte,;}{eens wel eindig was enkel}{maar  onbegrensd}\\

\haiku{Hij trof haar met de.}{doofstomme moeder in den}{boomgaard voor het huis}\\

\haiku{Kan 't bloed in mijn{\textquoteright}.}{als gist'ge wijn Weer tot}{het brein doen springenn}\\

\haiku{Na drie weken was;}{de staking in de grote}{steden verlopen}\\

\haiku{nu hoopte hij dat,....?}{Marretje er bij was zou}{zij hem herkennen}\\

\haiku{was hier niet overal?....}{de schone vrede van dien}{beminden grijsaard}\\

\haiku{En die eerste nacht,,.}{dien ik nu weerzie was de}{schoonste mijns levens}\\

\haiku{Was het geen wanhoop,?}{aan Uw macht geen wantrouwen}{jegens Uw wijsheid}\\

\haiku{En de mare der.}{nieuwe modelboerderij}{ging snel over het land}\\

\haiku{Te midden van mijn.}{lieve boeken heeft de tijd}{mij neergeslagen}\\

\haiku{Ik lig hier nu stil,,....}{en goed het is alles schoon}{en lief om mij heen}\\

\haiku{maar om zes uur reed.}{hij met sierlijke rijzing}{weg van de aarde}\\

\haiku{Hij wilde verder.}{en de herinneringen}{beangstigden hem}\\

\haiku{meer-en-meer werd.}{de streek door renteniers en}{forensen gezocht}\\

\haiku{Frajer en groter.}{werden er de openbare}{gebouwen hersticht}\\

\section{Pol de Mont}

\subsection{Uit: De amman van Antwerpen}

\haiku{Neen, ligh nu stille...,{\textquoteright},;}{gans stille zei hij haastig}{toen ze zich bewoog}\\

\haiku{Wat zou ze nu doen,?}{de volgende dag als het}{licht gekomen was}\\

\haiku{{\textquoteleft}Hebbic uw verlof,,?}{Jonkver Veerle om een woord}{met u te spreken}\\

\haiku{Ze was bedwelmend,}{mooi in haar mollige}{naaktheid met haar huid}\\

\haiku{Hij voelde het, - hij, -.}{meende het te voelen dat}{hij overwonnen had}\\

\haiku{Ze lag gerust te,}{slapen in een heel vreemde}{houding met de voetjes}\\

\section{Henricus van Moorsel}

\subsection{Uit: Kronijk, of Aantekening der merkwaardige voorvallen binnen de gemeente Heeze en eenige omliggende dorpen en enkelde welken algemene belangstelling verdienen}

\haiku{Op Onze Lieve,.}{Vrouw-Geboorte ~ 8}{September 1952}\\

\haiku{1680193 heeft men gezien.}{eene staartster die zich in het}{Noorden vertoonde194}\\

\haiku{deze waren aan}{de staken bevroren en}{kosten 6 gulden}\\

\haiku{1799 can sterken en,.}{langdurigen winter heeft}{geduurd tot 1 April306}\\

\haiku{men dagt dat er geen / /;}{gebouw Pag. 26 meer zoude}{hebben blijven staan}\\

\haiku{29 December des.}{nachts gedonderd en gestormd}{met hagelbuijen}\\

\haiku{het metselwerk was;}{vroeger aanbesteed tot aan}{de vengster durpels480}\\

\haiku{6 Augustus de.}{kappen op de nieuwe kerk}{te Heeze gesteld}\\

\haiku{14 Februarij;}{zijn de eerste pijpen in}{het orgel gesteld}\\

\haiku{der 8 Afd. Inf. van;}{Someren na Heeze en}{Leende gekomen660}\\

\haiku{dat drie maal per dag,;}{hervat wordt des  morgens}{smiddags en des avonds}\\

\haiku{op het ogenblik stond;}{de Peer of Knop des torens}{als in volle vlam}\\

\haiku{Men kon naauwelijks;}{te voet droog aan het Kasteel}{te Heeze komen}\\

\haiku{- Weder veel regen.}{gevallen gelijk ook de}{volgende dagen}\\

\haiku{Kronijkschrijver.}{heeft zich in zake de brand}{van Helmond vergist}\\

\haiku{Hermanni, H. / Jongh, /, /, /,.}{A. de Laats J. Raucamp J.}{Chr. Sprang E. van}\\

\haiku{G. BANNENBERG, Sint,.}{Willibrord in Waalre en}{Valkenswaard p. 24}\\

\haiku{is ook bekend als.}{beneficiant van het}{altaar van de HH}\\

\haiku{121L. VAN AITZEMA,,,-.}{Saken van Staet en Oorlogh}{dl. II p. 451452}\\

\haiku{Uit den toorn zijn wij.}{gestooten Alexius Jullien}{heeft ons gegooten}\\

\haiku{M. THEODARDUS,,-.}{ROELOFS Geschiedenis van}{Grave p. 3839}\\

\haiku{M. THEODARDUS,,-.}{ROELOFS Geschiedenis van}{Grave p. 3839}\\

\haiku{A. FRENKEN, Helmond,,-.}{in het Verleden dl. II}{p. 184225 passim}\\

\haiku{376Het woonhuis van de;}{drossaard van Heeze is nooit}{pastorie geweest}\\

\haiku{Middelprijzen der;}{levensmiddelen oyer}{de maand Julij 1824}\\

\haiku{PROVINCIAAL BLAD,,,-;}{VAN NOORD-BRABAND 1830}{n. 92 p. 7778}\\

\haiku{PROVINCIAAL BLAD,,*,-;}{VAN NOORD-BRABAND 1832}{n. 110 p. 910}\\

\haiku{N. 6 van deze.}{Memorie luidt als volgt Den}{10 Augustus 1831}\\

\haiku{Deze notitie.}{is door kronijkschrijver}{later toegevoegd}\\

\haiku{PROVINCIAAL BLAD,,,-.}{VAN NOORD-BRABAND 1833}{n. 101o p. 34}\\

\haiku{Had het commando.}{overgenomen van Majoor}{W. Senn van Bazel}\\

\haiku{603De 2e Comp. was.}{reeds te Heeze sedert 15}{Januari 1833}\\

\haiku{694Koop II bekend.}{als Asten Sectie E nrs.}{796 tot en met 803}\\

\haiku{697Sectie B n., {\textquoteleft}{\textquoteright} {\textquoteleft}{\textquoteright}.}{114 bekend alsDe Donk of}{Den Ommelschen Bosch}\\

\haiku{PROVINCIAAL BLAD,,,,.}{VAN NOORD-BRABAND 1837}{n. 99 p. 17 53}\\

\haiku{742PROVINCIAAL,,,.}{BLAD VAN NOORD-BRABAND}{1838 n. 90o p. 5}\\

\haiku{Zie de afbeelding.;}{van deze pastorie op}{p. 140 van het Hs}\\

\haiku{772PROVINCIAAL,,,.}{BLAD VAN NOORD-BRABAND}{1839 n. 107 p. 39}\\

\haiku{(PROVINCIAAL BLAD,,,-).}{VAN NOORD-BRABAND 1838}{n. 90 p. 4445}\\

\haiku{807PROVINCIAAL,,,.}{BLAD VAN NOORD-BRABAND}{1842 n. 115 p. 49}\\

\haiku{PROVINCIAAL BLAD,,,-.}{VAN NOORD-BRABAND 1842}{n. 115 p. 2223}\\

\haiku{873De toren der.}{kerk oorspronkelijk bekend}{als Sectie A 862}\\

\haiku{PROVINCIAAL BLAD,,,-.}{VAN NOORD-BRABAND 1844}{n. 80 p. 4445}\\

\haiku{1028Zie voor deze,,.}{benoeming DE GODSDIENSTVRIEND}{dl. XXXVI p. 156}\\

\haiku{Reeds eerder was een.}{poging tot verkoop van de}{Heerlijkheid mislukt}\\

\haiku{beroepen Juli,.}{1689 vertrokken naar Waalre}{11 November 1693}\\

\haiku{PROVINCIAAL BLAD,,, \&;}{VOOR NOORD-BRABAND 1836}{nr. 8 p. 7 12}\\

\section{Lodewijk Mulder}

\subsection{Uit: Humor en satire}

\haiku{den tijd, dien dan nog,;}{overschiet gebruiken ze voor}{nuttigezaken}\\

\haiku{- hanc tuemur hac -,!}{nitimur o vaderland}{o dierbre grond}\\

\haiku{Dat is er nog een, '.}{zooals ik ze int jaar 1500}{heb zien gebruiken}\\

\haiku{want ik werd het hoe.}{langer hoe minder met de}{Hollandsche maagd eens}\\

\haiku{Verbeeldt u mijne,:}{verbazing toen ik daarop}{het adres zag staan}\\

\haiku{{\textquoteleft}Dat is een {\textquoteleft}kwaal,{\textquoteright} zei, {\textquoteleft} ';}{ze toendie heb ik int}{jaar 1814 opgedaan}\\

\haiku{U behoef ik het,,:}{niet te zeggen wie ik ben}{en wat ik bedoel}\\

\haiku{maar iedereen voelt, -.}{zoo niet en dat is jammer}{voor het gewormte}\\

\haiku{Het model van zulk.}{sterk haar zal wel hier of daar}{te vinden wezen}\\

\haiku{Hofdijk, Aeddon I,}{zang ~  Hoewel dat ghij zijt}{schichtigh als een rhee}\\

\haiku{zooals ik, en op een,}{verheven onpartijdig}{standpunt te gaan staan}\\

\haiku{{\textquotedblleft}Ik heb hetzelve,!}{bewerkstelligd gelijk gij}{mij hadt bevolen}\\

\haiku{Ik wist het op dat.}{oogenblik niet en bleef in}{gedachten verdiept}\\

\haiku{{\textquoteright} {\textquoteleft}Niets liever dan dat,{\textquoteright}, {\textquoteleft} -;}{zei hijmaar wat maakt je in}{eens zoo pleizierig}\\

\haiku{Ik wil me op een,,!}{verheven onpartijdig}{standpunt plaatsen Joost}\\

\haiku{Die drenken in den,}{morgendouw De duifjes met}{haar trekkebekken}\\

\haiku{{\textquoteright} {\textquoteleft}En zoudt gij op uwe,?}{beurt denken dat die taal zoo}{moeilijk te leeren is}\\

\haiku{Lees er \'e\'ene maand,.}{in en gij zult geheel over}{dat vreemde heen zijn}\\

\haiku{een wit piqu\'e vest,.}{glac\'e handschoenen en een}{fijne bruine rok}\\

\haiku{{\textquoteleft}Jongen,{\textquoteright} zei hij, na, {\textquoteleft} ';}{eenig aarzelenmaar vindt je}{t toch niet wat gek}\\

\haiku{maar 't is een kunst,,!}{om dat geheim te houden}{want die intriges}\\

\haiku{Nu of nooit,{\textquoteright} dacht ik, {\textquoteleft}.}{een gevoelvol discours over}{het buitenleven}\\

\haiku{als wij eens een keer ',;}{of vier in de maands avonds}{uitgaan is het veel}\\

\haiku{gaat men liever niet,.}{naar den eenen kant dan gaat men}{naar den anderen}\\

\haiku{maar dat is gezond,;}{in de morgenlucht dat wekt}{de appetijt op}\\

\haiku{{\textquoteleft}de majoor zal hem, '.}{ook wel kennen wantt was}{ook een militair}\\

\haiku{{\textquoteright} vroeg de majoor, {\textquoteleft}'t,;}{staat me niet voor dat ik er}{ooit van gehoord heb}\\

\haiku{{\textquoteleft}Ik ben officier,{\textquoteright}, {\textquoteleft};}{zei de officieren ik}{ben uit A. vandaan}\\

\haiku{{\textquoteleft}En daar heb je de,{\textquoteright}, {\textquoteleft}.}{kippen vervolgde Holman}{en dat is de haan}\\

\haiku{{\textquoteright} {\textquoteleft}Wel foei, mevrouw,{\textquoteright} viel, {\textquoteleft}!}{mevrouw Holman haar in de}{redewat een woord}\\

\haiku{{\textquoteleft}Mijnheer Rentink, mag?}{ik u eens verzoeken de}{glazen te vullen}\\

\haiku{{\textquoteleft}'t Is toch een lief,{\textquoteright}.}{talent zei mevrouw Seller}{mij in vertrouwen}\\

\haiku{Ik heb het met een, '.}{jongen naar huis gestuurd want}{ik draagt zelf nooit}\\

\haiku{U hebt u zeker,{\textquoteright}.}{hier of daar aan geblesseerd}{ging de jager voort}\\

\haiku{ik heb nooit aan de, -}{jacht gedaan maar ik heb er}{heel veel van gehoord}\\

\haiku{maar de tweede helft:}{van zijne toespraak maakte}{mij dat duidelijk}\\

\haiku{{\textquoteleft}en al was dat  ,.}{zoo dan ben ik toch niet van}{plan om heen te gaan}\\

\haiku{Adieu, tot Dinsdag,,{\textquoteright}.}{dan we eten om half drie en}{hij ging de deur uit}\\

\haiku{Laten we 't dan, '.}{maar afspreken zooals jet}{me voorgesteld hadt}\\

\haiku{Nu wij opnieuw van.}{onzen toren De klok van}{twaalven hooren slaan}\\

\haiku{{\textquoteright} riep hij eindelijk, {\textquoteleft}?}{uitrepeteer je een rol}{voor een Duitsch drama}\\

\haiku{Ga maar zitten, neem.}{zes vel papier voor u en}{een vollen inktpot}\\

\haiku{Laten we 't nu,.}{nog eens overlezen tot waar}{we gekomen zijn}\\

\haiku{{\textquoteleft}Maar komaan, vooruit,, ',!}{op denzelfden weg en als}{t kan nog doller}\\

\haiku{, en ik gevoelde.}{mij wel honderd pond lichter}{dan voor zijne komst}\\

\haiku{{\textquoteleft}Mijnheer,{\textquoteright} zei hij, {\textquoteleft}ik.}{bewonder uw richtigen}{po\"etischen blik}\\

\haiku{Op eenmaal mijne,?}{rust weer in gevaar brengen}{of hem glad afslaan}\\

\haiku{{\textquoteright} {\textquoteleft}Wilt gij mijn gesp van?}{vijfentwintigjarigen}{dienst in pand hebben}\\

\haiku{voor andren de)!}{stoeitijd Reeds langs den straatweg}{een karretje trok}\\

\haiku{vreeslijk doet hikken?}{En slikken en proesten en}{snurken en snikken}\\

\haiku{Hoeveel wonderen, '!}{zagen ze zwervend int}{prachtige London}\\

\haiku{Ook in het British '.}{Museumt gesprek met}{Athena beluistrend}\\

\haiku{Een ernstig woord over ',?}{sonnetten ~ Mooi ist}{maar een sonnet}\\

\haiku{ure zal de gort en.}{om twee ure het middageten}{worden gebruikt}\\

\haiku{ure zal de gort en.}{om twee ure het middageten}{worden gebruikt}\\

\haiku{Geen voorrechten, ten,!}{minste niet zulke groote in}{het oog vallende}\\

\subsection{Uit: Mengelwerk}

\haiku{als 't donker was,}{werd dat naar voren gedraaid}{met een dun kaarsje}\\

\haiku{Ja mijn kind, ik ben,?}{de baddokter wou je me}{gesproken hebben}\\

\haiku{Dat is zeker niet -.}{goed laat dat liever aan de}{geleerden over}\\

\haiku{En zei hij dus, als,?}{hij verliefd werd zou hij je}{laten trouwen}\\

\haiku{Ik geef de maan van -,.}{al hun kuren om vijf uur}{eten versta je}\\

\haiku{de een vertelt iets,,;}{dat hij weet aan een ander}{die het ook al weet}\\

\haiku{vergeef mij, ik heb.}{u in uw samenspraak met}{den dokter gestoord}\\

\haiku{Dat is een gevolg,;}{van de overgroote teerheid van}{uw gestel Fr\"aulein}\\

\haiku{Fr\"aulein v. S. Neem,,?}{mij niet kwalijk vriend mag ik}{u even iets vragen}\\

\haiku{Kijk, mijnheer, daar heb, '.}{ik er eent mooiste wat}{er te vinden was}\\

\haiku{En zeg er bij, dat,.}{het van iemand komt die je}{niet moogt noemen}\\

\haiku{Dat zou ik ook niet,.}{kunnen want ik weet niet eens}{hoe mijnheer heet}\\

\haiku{ik moet hem zien, en.}{dan kan ik oordeelen hoe}{hij over mij denkt}\\

\haiku{Ik dien nu heen te.}{gaan en hem het eerst hier te}{laten komen}\\

\haiku{Best, mijnheer  (ter),!}{zijde onder het weggaan}{Ja laarzenpoetsen}\\

\haiku{(Hij neemt den zakdoek,).}{in de hand staat op en gaat}{naar de Fr\"aulein}\\

\haiku{U kent allebei,?}{dat clubje Hollanders wel}{zooals ze dat noemen}\\

\haiku{het nieuwe verband,.}{wordt gelegd en van daar gaat}{het naar een ander}\\

\haiku{Daarom roep ik het:}{met den meesten aandrang al}{mijnen lezers toe}\\

\haiku{Mogelijk deel ik,}{daarvan en van hetgeen ik}{nog later hier zien}\\

\haiku{Dan is er leven;}{en drukte op dien anders}{zoo stillen oever}\\

\haiku{van de groote wijnbowl (),;}{ik meen van twee ankers die}{daar geledigd werd}\\

\haiku{Ces toujour \`a vous.}{M. et Madame que l'ons}{doit ces remersiment}\\

\haiku{din\'e, eens goed, de,.}{tweede maal middelmatig}{de derde maal slecht}\\

\haiku{Onder den indruk,}{van al die beschouwingen}{begonnen wij met}\\

\haiku{de noordsche tafel.}{was best en alles was net}{en behagelijk}\\

\haiku{maar ik had buiten:}{de wisselvalligheid van}{het weer gerekend}\\

\haiku{Geen zweem van woning;}{of bebouwing aan de kust}{van het vaste land}\\

\haiku{de vrouwelijke.}{leden van zijn gezin had}{hij nooit gesproken}\\

\haiku{Eindelijk scheen het:}{oogenblik tot beginnen}{gekomen te zijn}\\

\haiku{- Het was duidelijk,.}{dat niemand iets wonderlijks}{in een wonder zag}\\

\haiku{ik eene kleine deur,;}{waardoor ik op straat meende}{te zullen komen}\\

\haiku{de bergen in de;}{rondte liggen zwijgend en}{grijs te sluimeren}\\

\haiku{Wat een Maggio is,.}{wisten we toen evenmin als}{waarschijnlijk 99 pCt}\\

\haiku{hij genezen was,.}{moest hij naar dat feest om zijn}{herstel te vieren}\\

\haiku{Intusschen was het.}{altijd imposant en het}{bezoek dubbel waard}\\

\haiku{twee zware koffers,.}{waarvan de eene niet minder}{dan 68 kilo woog}\\

\section{Multatuli}

\subsection{Uit: Volledige werken. Deel 1. Geloofsbelydenis. Max Havelaar. Brief aan ds. W. Francken Azn. Brief aan den gouverneur-generaal in ruste. Aan de stemgerechtigden in het kiesdistrikt Tiel. Max Havelaar aan Multatuli. Het gebed van den onwetende. Wys my [...]}

\haiku{Zie, daar richt zy zich,... -....}{op en ziet ons dankbaar aan}{Juist riep de vader}\\

\haiku{Ik ben makelaar,,.}{in koffie en woon op de}{Lauriergracht No. 37}\\

\haiku{Men vlucht met het een.}{of ander voorwerp naar het}{einde der aarde}\\

\haiku{weet niet, waarom? - En!}{vraag eens naar den prys van een}{stel biljartballen}\\

\haiku{Het is zo niet in, '.}{de wereld ent is goed}{dat het niet zo is}\\

\haiku{Hy had dan ook wel,.}{iets van een Duitser en van}{een reiziger ook}\\

\haiku{Ja, ja, hy was het,!}{die my uit de handen van}{den Griek had verlost}\\

\haiku{Ik schreide, en bad,.}{om genade want ik zat}{vreselyk in angst}\\

\haiku{Hy zag zeer bleek, en, '.}{toen ik hem vroeg hoe laat het}{was wist hyt niet}\\

\haiku{{\textquoteright} Myn vrouw zei hierop.}{dat Frits dan voortaan niet meer}{mee zou naar den krans}\\

\haiku{Maar Louise schreide,.}{weer en de dames zeiden}{dat het heel mooi was}\\

\haiku{Over hydraulische.}{onderwerpen in verband}{met de rystkultuur}\\

\haiku{Maar hy zegt, dat de.}{invloed zich eerst openbaart in}{het tweede geslacht}\\

\haiku{Natuurlyk had hy,.}{my om geld gevraagd en van}{zyn pak gesproken}\\

\haiku{hoe zou het wezen,,?}{dacht ik als ik hem de plaats}{van Bastiaans gaf}\\

\haiku{Dat myn naam niet op,.}{den titel zou staan omdat}{ik makelaar ben}\\

\haiku{Bovendien, ik houd.}{niet van mensen die altyd}{ontevreden zyn}\\

\haiku{- Je weet immers dat.}{die m'nheer gister alles}{heeft meegenomen}\\

\haiku{Wat hy zeide, was,}{gewoonlyk lang overdacht een}{eigenaardigheid}\\

\haiku{Daar is me juist iets,...?}{voorgekomen dat zou de}{Regent ons verstaan}\\

\haiku{hy is beschaafd, niet? -... -?}{waar O ja En hy heeft een}{grote familie}\\

\haiku{Ik beveel me zeer,!}{aan voor uw medewerking}{m'nheer Verbrugge}\\

\haiku{{\textquoteright} niet uitsprak, voor hy ' {\textquoteleft}{\textquoteright}.}{haartniet zondigen had}{mogelyk gemaakt}\\

\haiku{En, kinderen als,.}{ze waren vermaakten zy}{zich met hun nieuw huis}\\

\haiku{Rypt niet uw padi?}{dikwerf ter voeding van wie}{niet geplant hebben}\\

\haiku{Wat zal er gezegd?}{worden in de dorpen waar}{wy gezag hadden}\\

\haiku{zyn schryver heeft het,...!}{my gezegd en bovendien}{dat bruske vragen}\\

\haiku{- Ook de staten die,,.}{ik vandaag ontving zyn vals}{ging Havelaar voort}\\

\haiku{Het was myn hart dat!}{ge daar hebt opgeslikt als}{een versnapering}\\

\haiku{En wacht geen meisjes ',.}{op alst uit is want dit}{neemt de stichting weg}\\

\haiku{niemand kan hem dus.}{een grondige kennis der}{zaken betwisten}\\

\haiku{Er is een juffrouw...}{flauw gevallen toen hy van}{dat zwarte kind sprak}\\

\haiku{Heimlich erz\"ahlen.}{die Rosen Sich duftende}{M\"archen ins Ohr}\\

\haiku{Marie is daar in -,?}{dien rooien tuin waarom rood}{en niet geel of paars}\\

\haiku{{\textquoteleft}deze kapel is...}{opgericht door den bisschop}{van Munster in 1423}\\

\haiku{Ze bewegen zich,.}{als een hobbelpaard minus}{nog het va et vient}\\

\haiku{Eigenliefde en.}{verveling dringen u iets}{dergelyks te doen}\\

\haiku{- Welnu, dan weet je.}{dat er peperkultuur in}{het Natalse is}\\

\haiku{Bovendien, ik had... -'!}{allerlei gekheden in}{het hoofd Ta oesah}\\

\haiku{Ik zei dat elk mens.}{in zyn medemens een soort}{van konkurrent ziet}\\

\haiku{Den gevangene.}{immers is men onderhoud}{en voedsel schuldig}\\

\haiku{Hebt ge niet geweend,...}{by die moeder vruchteloos}{zoekend naar haar kind}\\

\haiku{, zou die betaling.}{spoedig zyn middelen zyn}{te boven gegaan}\\

\haiku{Gesichtchen nahe,,.}{dr\"uckt Dann lachst du freudig das}{ist auch Gef\"uhl}\\

\haiku{Men verkrygt daardoor '.}{den roep van bekwaamheid en}{yver voors lands dienst}\\

\haiku{Gisteren heb ik:}{Sjaal man gezien met zyn vrouw}{en hun jongetje}\\

\haiku{Hy is bleek als de,,.}{dood zyn ogen puilen uit en}{zyn wangen staan hol}\\

\haiku{Men ziet uit alles.}{dat Stern jong is en weinig}{ondervinding heeft}\\

\haiku{Ook de ouders van.}{zyn vrouw woonden altyd in}{datzelfde distrikt}\\

\haiku{Ik zal uitblyven....}{driemaal twaalf manen deze}{maan rekent niet mee}\\

\haiku{Zie, Adinda, kerf.}{een streep in je rystblok by}{elke nieuwe maan}\\

\haiku{het vastgeknoopt aan '.}{t eindelyk terugzien}{onder den ketapang}\\

\haiku{{\textquoteleft}wie van ons zal het,?}{lichaam verslinden dat daar}{daalt in het water}\\

\haiku{En ook vroeg hy zich,?}{wie er toch wel wonen zou}{in zyns vaders huis}\\

\haiku{- en ze hadden zich.}{neergelaten aan sterke}{rotan-koorden}\\

\haiku{Maar nog altyd is...!}{myn ziel En myn hart bitter}{bedroefd Adinda}\\

\haiku{De man was zeker!}{bezig met een jaarverslag}{over rustige rust}\\

\haiku{Er is altyd een.}{soort van belediging in}{dat niet herkennen}\\

\haiku{En ze hebben niets,,!}{te eten en ze slapen op}{den weg en eten zand}\\

\haiku{Daar zit hy rustig,...}{by vrouw en kind en tekent}{borduurpatroontjes}\\

\haiku{{\textquoteright} Ja, ik hoor 't wel, ',!}{ik hoort wel dat roepen}{om wraak over myn hoofd}\\

\haiku{Het gesprek liep over.}{de verwachte beslissing}{van de Regering}\\

\haiku{Maar ik zal tot hem,.}{gaan en hem aantonen hoe}{hier de zaken staan}\\

\haiku{wederlegging der!}{hoofdstrekking van myn werk}{is onmogelyk}\\

\haiku{voor gezondheid en, '.}{leven maar voorn gering}{deel van hun welstand}\\

\haiku{Fluks wierp hy 't in, '.}{de goot en reinigde het}{metn stalbezem}\\

\haiku{Nederland heeft niet.}{verkozen recht te doen in}{de Havelaarszaak}\\

\haiku{de herinnering '.}{int leven roept aan den}{Javasen oorlog}\\

\haiku{{\textquoteleft}man, bloemlezer, weet?}{je wel wat je beweert ons}{te willen leren}\\

\haiku{Dit is al iets in!}{onzen tyd van jammerlyk}{ordinarisme}\\

\haiku{Me dunkt toch dat ze,,.}{vooral met het oog op z'n}{dood zeer treffend zyn}\\

\haiku{Op strandplaatsen als,.}{Marseille verbasteren}{de rassen zeer snel}\\

\haiku{) Het is zeer gewaagd '.}{dit op grond vann enkel}{woord aan te nemen}\\

\haiku{mannelyk voor, en '.}{even onsmakelyk alst}{bedoeld delikt zelf}\\

\haiku{Men behoorde den.}{moed te hebben zyner}{gewetenloosheid}\\

\haiku{Het blad zit, als 't ',.}{yzer vann houweel loodrecht}{op den houten steel}\\

\haiku{{\textquoteright} Maar nooit bleek er dat.}{er iets gedaan werd om dit}{doel te bereiken}\\

\haiku{De terechtwyzing:}{van den heer Bensen is my}{te meer welkom}\\

\haiku{Doch juist, omdat het;}{u niet maar te doen was om}{een boek te schrijven}\\

\haiku{Gy duldt geen spat op, '...}{uw kleed geen schampschot int}{gelaat ge dwingt my}\\

\haiku{5 - Men begrypt, hoe.}{ik er aan hecht dat deze}{brief bewaard blyve}\\

\haiku{maar in de buurt van.}{Job vond ik werkelyk den}{brief waarvan hy spreekt}\\

\haiku{- men wachtte op een ',;}{ommekeer vant lot op}{Himmels einfallen}\\

\haiku{en al ware het, -.}{dat ik my weder bedroog}{ik kan niet anders}\\

\haiku{Zy het dezen avond,,,...{\textquoteright}}{schreef ik zy het heden nacht}{zy het morgen vroeg}\\

\haiku{Ik ging niettemin,.}{met zachte vriendelyke}{vermaningen voort}\\

\haiku{dat ik dwaalde, toch.}{getoond had in die dwaling}{eerlyk te zyn}\\

\haiku{Wederom is een.}{nieuwe parlementaire}{zitting geopend}\\

\haiku{Neen, niet omdat hy,.}{gedaan heeft wat hy kon maar}{omdat hy iets kon}\\

\haiku{Het geheel bestaan.}{van ons Vaderland staat of}{valt met dat gebouw}\\

\haiku{dan zoude het een.}{onbillykheid wezen aan}{valsheid te denken}\\

\haiku{- D\^as en broekie f\^a,,.}{me bro\^ertje weet uwe f\^a me}{bro\^ertje d\^a doot is}\\

\haiku{Ik zelf vergelyk,.}{het by een kalf met zeven}{staarten zonder kop}\\

\haiku{zeer goed, - dat stuk nu,,.}{was w\'elgeboren en heeft}{geen geluk gehad}\\

\haiku{Er is geen enkel, '}{boek goed geschreven en het}{uwet minst van al.}\\

\haiku{Wie zich toelegt op.}{goed schryven kan nooit v\'e\'el voor}{den dag  brengen}\\

\haiku{de kleine moet in,.}{de lucht en de bonne volgt}{U met het kindje}\\

\haiku{Fanny moet er by,... {\textquoteleft}{\textquoteright} '.}{en de onsterfelykheid}{fancy int eind}\\

\haiku{Wat had hy, met zyn!}{nuchter verstand nog veel goeds}{kunnen uitrichten}\\

\haiku{Uw ouders zullen,.}{schateren over de grappen}{die lieve mensen}\\

\haiku{Wie 't goede doet,}{Opdat een God hem lonen}{zou maakt juist d\'a\'ardoor}\\

\haiku{Het geloof aan God '.}{heeft geen vaster grond dant}{geloof aan spoken}\\

\haiku{De vader zal het -,;}{kind omhoog houden neen dat}{zal de moeder doen}\\

\haiku{Hy spoorde na, en... '!}{snuffelde daar blaasde iets}{int geboomte}\\

\haiku{Welnu, de arbeid,...}{is volbracht wy stonden u}{onze akkers af}\\

\haiku{22 - Ryst stampen in. ' '.}{een houten blokt Woord is}{n onomatopee}\\

\haiku{De afgedrukte.}{tekst is overeenkomstig de}{Bloemlezing van 1865}\\

\haiku{zyde555kan worden -;}{gelyk gesteld met die op}{Java volgens Hs}\\

\haiku{het leek ongewenst,, -;}{die te bestendigen.13313Broek}{Berlyn volgens 1881}\\

\haiku{op de15010iets tot stand te -;}{brengen ten voordele van}{Natal volgens Hs}\\

\haiku{ik bezit daarvan -;}{nog de minuut16636hadden plaats}{gehad volgens 1881}\\

\haiku{van het vermoeden ' -;}{der25914de opvatting vant}{begrip volgens 1881}\\

\haiku{residentie op,:}{West-Java verdeeld in}{drie afdelingen}\\

\haiku{in militaire;}{dienst terug belegerde}{hij Utrecht en Naarden}\\

\haiku{Zweeds schrijfster (1801-1865),.}{die voor de emancipatie}{der vrouw ijverde}\\

\haiku{Op 15 November:}{1851 zond hij dit spel onder}{de nieuwe titel}\\

\haiku{later ambtenaar.}{aan het Ministerie van}{Handel te Parijs}\\

\haiku{Rangkasbetoeng en) (:}{Sadjira en Lebak}{Kidoeldistricten}\\

\haiku{Translated from the.}{original Manuscript by}{Alphonse Nahuys}\\

\haiku{boomsoort, uitstekend,.}{geschikt voor timmer- en}{sierhout z.g. teakhout}\\

\haiku{18.5 gedrukt had, en.}{Multatuli in 1875 uit}{het hoofd invulde}\\

\subsection{Uit: Volledige werken. Deel 2. Minnebrieven. Over vrijen arbeid in Nederlands-Indi\"e. Brief aan Quintillianus. Idee\"en, eerste bundel}

\haiku{En halen kan ik,.}{haar ook niet omdat ik geen}{geld heb voor de reis}\\

\haiku{Helpt ze hem niet, troost, -?}{ze hem niet zou hy sterker}{wezen zonder h\'a\'ar}\\

\haiku{Tot nader order,,,!}{als de jongens op zyn de}{jongens die v\'o\'orgaan}\\

\haiku{En men speculeert '?}{opt gebrek-lyden}{onzer kinderen}\\

\haiku{Papa is moe, moe,,;}{van gort stroop Oostenrykers}{en assurantie}\\

\haiku{een vrouw, die meer was,;}{dan de paus schoon ze maar eten}{had voor drie dagen}\\

\haiku{Le D\'esespoir van,....}{Lamartine is veel}{te lang en te mooi}\\

\haiku{Ik heb gemerkt dat,.}{er veel zaken zyn die men}{niet zegt aan vrouwen}\\

\haiku{En God, of de god...? '!}{begrypt ge dat woordt Is}{verwant met weten}\\

\haiku{Dit is alles zeer, ',.}{duidelyk en wiet niet}{begrypt is verdoemd}\\

\haiku{Gy, die groter zyt,,.}{dan ik steek uw arm uit en}{pluk opdat ik ete}\\

\haiku{- Het middel is zeer,.}{eenvoudig antwoordde de}{vrome heremiet}\\

\haiku{Wat moet ik doen, dat,?}{myn kind niet van my wegga}{als het lopen kan}\\

\haiku{- Wie nu geduldig,,.}{melkt tot het laatste toe brengt}{vette melk tehuis}\\

\haiku{- Eilieve, wie zal?}{beletten dat ze weet wat}{ik haar niet leerde}\\

\haiku{Ook wy zouden niets,.}{geweten hebben als ge}{ons niets hadt gezegd}\\

\haiku{Hassan wenste den,.}{vogel lengte van veren}{en noemde hem Rock}\\

\haiku{Hoe kunt gy zo scherp,,?}{zyn Max gy die toch zo goed}{weet lief te hebben}\\

\haiku{Maar om haar goed te,.}{beschryven zou ik haar by}{my moeten hebben}\\

\haiku{En zonder yverzucht,.}{hoor ik aan wat zy gezegd}{heeft zonder schaamte}\\

\haiku{Dat beestje... welnu,... '...... '}{uit verveling heb ikt}{wel eens geknepen}\\

\haiku{Want er komt veel wet.}{in de historie die my}{nu overkomen is}\\

\haiku{De dochters van myn,.}{man stoppen scheef en stikken}{dat het schande is}\\

\haiku{men dwingt u niet tot;}{goddienen op een manier}{die u niet aanstaat}\\

\haiku{Dit alles leidt nu:}{tot de volgende vreemde}{conclusi\"en}\\

\haiku{In Indi\"e zal de.}{stryd gevoerd worden om de}{wereldheerschappy}\\

\haiku{Ik denk het wel, van,...}{tyd tot tyd omdat ik hier}{zoveel verdriet heb}\\

\haiku{- dan nog kan ik niet.}{uitgaan om te schryven in}{die boekenkamer}\\

\haiku{Dat wegstoppen van.}{aandoeningen is my te}{lastig op den duur}\\

\haiku{- Ik dank u voor 't,,...}{schryven heer Ridder doch myn}{gezin lydt gebrek}\\

\haiku{Waarom laten zy?}{ons vier lange maanden in}{de onzekerheid}\\

\haiku{Ga vrymoedig tot,:}{hem en zeg aan den Keizer}{van Insulinde}\\

\haiku{Liever had hy nog, ',.}{slechter gespeeld of int}{geheel niet dan z\'o}\\

\haiku{Hy is wat moe van ',,!}{t gaan maar als hy hangt Is}{dat terstond voorby}\\

\haiku{Vriend Nathan, help my 't,...}{kind Eens overzetten op dien}{and'ren schouder}\\

\haiku{daar valt het weer, En...-}{ryst nu langzaam weer omhoog}{Dat zie ikzelf}\\

\haiku{kind'ren bauwt hem na,, '...}{En sart hem dat hy n\'og wat}{zegge aant kruis}\\

\haiku{Is dat ironie, is,,...?}{dat sarkasme is dat hoon}{of is dat domheid}\\

\haiku{het afnemen van!}{buffels aan de bevolking}{is het ergste niet}\\

\haiku{Die zo-even nog?}{daarginds natie en Koning}{representeerde}\\

\haiku{{\textquoteleft}Als hy langer hier,.}{was gebleven ware hy}{stellig vergeven?{\textquoteright}32}\\

\haiku{Ik weet niet of gy ',,!}{t genoeg vindt o Kiezers}{maar ik vind het v\'e\'el}\\

\haiku{t Is schande, dat!}{ge my aan zoiets blootstelt}{door uw slordigheid}\\

\haiku{{\textquoteright} Misschien heb ik 't.}{antwoord van dien ouden man}{niet goed begrepen}\\

\haiku{Dat begreep ik toen,,.}{niet omdat ik maar een kind}{was beste Tine}\\

\haiku{Dit althans schreef my,, '}{de kleine maar ik verdenk}{haar nu van foppery}\\

\haiku{5 - Zie 't boek over,, (.}{de Veilingen eerste deel}{pag. 145eerste druk)32}\\

\haiku{30 - Ziehier wat Prof.:}{Veth daarvan zegt in den Gids}{van Augustus 1860}\\

\haiku{Multatuli, niet,.}{alleen in aangenomen}{naam maar inderdaad}\\

\haiku{{\textquoteright} Verbeeld u het lid,:}{ener jury die  tot een}{kollega zeide}\\

\haiku{Maar toen ik 't las,.}{was ik beschaamd en heb de}{P\`ene bedankt}\\

\haiku{Waar de fantaisie.}{kristal meende te zien vond}{ze schimmel en vuil}\\

\haiku{Later blykt dat ze}{schoon is en dat ik pedant}{was door te meenen}\\

\haiku{Uit dien strijd tussen (,.}{vurige begeerte naar}{waarheidexact maths}\\

\haiku{Want, Sire, ik lees,,.}{geen Engels en geen dichters}{ook dat begrypt ge}\\

\haiku{ik vragen wat hy,,}{is door niets te zeggen door}{niets te verrichten}\\

\haiku{*~         Ja, er waren, '!}{veel bylagen by dien brief}{t was een bundel}\\

\haiku{Vier jaren lang was,}{Sjaalman bespot en gesard}{door Droogstoppel v\'o\'or}\\

\haiku{Volgt niet oorlog op,,?}{oorlog roering op roering}{onrust op onrust}\\

\haiku{Is 't zyn schuld dat,?}{uw vaderland zo bar is}{zo onvriendelyk}\\

\haiku{Of als hy, al te, ',?}{lang geperst aant gisten}{was gegaan hyzelf}\\

\haiku{Altyd zulke fraaie,!}{naar studie en wetenschap}{riekende woorden}\\

\haiku{Ziet hier de staten,,.}{van invoer van verkoop van}{netto provenu}\\

\haiku{En als de schryver?}{van zulke ongepaste}{klaagbrieven aanhoudt}\\

\haiku{Take it away...{\textquoteright} To...!}{honester men maakt plaats}{voor eerlyker lui}\\

\haiku{Ik wil trachten u, ',.}{dit te zeggen schoont me}{moeite kost lezer}\\

\haiku{Wie twyfelt er aan,?}{of het goede beter is}{dan het kwade}\\

\haiku{De kleine plichten,.}{die zyn leeftyd hem oplegt}{vervult hy gaarne}\\

\haiku{In \'e\'en woord, tot nog.}{toe arbeidde hy geheel}{en al vrywillig}\\

\haiku{En ik ben blyde:}{toegegeven te hebben}{in een wens die}\\

\haiku{Er zal geen hagel '.}{vallen zolangt gewas}{op het veld staat}\\

\haiku{Ieder inwoner.}{van den Staat zal behoorlyk}{gevoed worden}\\

\haiku{o d\'at zou ik ook, '...}{wel kunnen en denk aant}{ei van Columbus}\\

\haiku{Ik citeer myzelf, '}{gaarne alst dienen kan}{om aan te tonen}\\

\haiku{{\textquoteleft}dat de Javaan het, '.}{zyne ontving al ist}{dan ook wat weinig}\\

\haiku{Ook een tweede stuk.}{van gelyke strekking is}{reeds drie jaren oud}\\

\haiku{Ge smoort met zulke.}{kinderachtigheid uzelf en}{uw eigen belang}\\

\haiku{Multatuli z\'al,.}{niet bezwijken omdat hij}{niet bezwijken k\'an}\\

\haiku{Dat ik moeilyk werk ',.}{met zo'n logge vracht opt}{gemoed is ook waar}\\

\haiku{Een verzameling,,,.}{van hout steen kalk enz. is niet}{altyd een gebouw}\\

\haiku{Een vergadering.}{van mensen is niet altyd}{een gezelschap}\\

\haiku{Eens-voor-al, het:}{woordjen is gebruik ik}{tot verkorting van}\\

\haiku{Er zyn meer muggen,.}{dan wespen meer kappellui}{dan droogstoppels}\\

\haiku{Ik bied 'n vel druks.}{voor een goed voornaamwoord van}{de tweede persoon}\\

\haiku{Als 'n hardloper, '.}{z'n been breekt ist bal par\'e}{by de kruipers}\\

\haiku{Ik weet niet hoe 't,}{komt maar  Ein M\"archen}{aus alten Zeiten}\\

\haiku{Ik viel neer op 'n,,,...}{stoel afgemat uitgeput}{byna moedeloos}\\

\haiku{Ik was zwaar gewond, '.}{en kont hoofd niet keren}{naar de pendule}\\

\haiku{Zy hadden de vraag, '.}{der moeder niet verstaan en}{zagent kind niet}\\

\haiku{Maar als men heenstapt,.}{over dien eersten regel dan}{volgt de rest vanzelf}\\

\haiku{A en B wilden,.}{een rivier overtrekken die}{van Oost naar West liep}\\

\haiku{Dan heb ik in myn.}{Idee\"en wat meer ruimte voor}{andere zaken}\\

\haiku{een voorzichtigheid,;}{die soms aan bezorgdheid voor}{teleurstelling grenst}\\

\haiku{zijn van de totaal,:}{bedorven politieke}{atmosfeer dat is}\\

\haiku{Ik vind dat dit de.}{passage naar den kelder}{belemmeren zou}\\

\haiku{ik 't af, dat men.}{z'n aangezicht gebruikt om}{er op te liggen}\\

\haiku{Na al 't bidden,.}{om den Geest verbaas ik my}{over die verbazing}\\

\haiku{Dat komt er van, als '!}{men een opzichter niet aan}{t woord laat komen}\\

\haiku{Maar overigens - o!}{heerlyke overeenstemming}{in verscheidenheid}\\

\haiku{{\textquoteright} Gy meent misschien dat,...!}{ze naar bed gingen pour tout}{de bon ditmaal mis}\\

\haiku{{\textquoteright} Hy vraagt om zegen ' {\textquoteleft} '{\textquoteright}.}{opt weeshuiswaarin wij}{t zo goed hebben}\\

\haiku{Er zou voor ons geen, '.}{plus bestaan wanneer wet}{minus niet kenden}\\

\haiku{Ik wandelde met.}{kleinen Max. Voor ons uit ging}{een man met zyn kind}\\

\haiku{Welke werking heeft?}{op dien steen de hefboom die}{1/x deel korter is}\\

\haiku{Onsterfelykheid '.}{zonder eeuwigheid isn}{koord met \'e\'en eind}\\

\haiku{By 't opstaan 's,.}{morgens zag ik dien berg en}{wat hy uitblaasde}\\

\haiku{Ik klaag niet \'omdat,.}{het zo is maar juist d\'a\'arover dat}{het zo wezen moet}\\

\haiku{Die som kan niet meer,:}{wezen dan twee ze kan niet}{minder zyn dan twee}\\

\haiku{Die God voegt samen,,,,,,,,...}{ontbindt maakt vermaakt richt wendt}{buigt heft perst en plet}\\

\haiku{Een schoolknaap kan ze,.}{u aantonen en ditmaal}{zonder vals vernuft}\\

\haiku{Dien Constantyn noemt,.}{ge groot en de rest is even}{waar als die grootheid}\\

\haiku{Agatha vraagde niet ':}{naart verband tussen h\'a\'ar}{zonden en Adams val}\\

\haiku{Wie wysheid spreekt in den,,.}{tempel en dwaasheid geeft aan}{zyn vrouw is een dief}\\

\haiku{wat Origenes deed {\textquoteleft}{\textquoteright},.}{om des hemelryks wil zie}{d\'at begryp ik niet}\\

\haiku{Zo spreekt geen meisje...{\textquoteright}.}{Ziedaar schering en inslag}{van de opvoeding}\\

\haiku{*~         Ik wou wel eens ' {\textquoteleft}{\textquoteright},}{nHeer zien die de macht had}{u te beletten}\\

\haiku{Ook moest gy weten.}{wat er te doen was tegen}{de apen-attractie}\\

\haiku{Franschen zouden des.}{anderen daags myn voorbeeld}{nagevolgd hebben}\\

\haiku{eerste hoofdstuk te.}{schryven boven den aanvang}{dezer historie}\\

\haiku{Jazelfs, hy begon.}{dien klank lief te krygen om}{de betekenis}\\

\haiku{- Oui, mon fils, du plus,...}{grand malheur du seul malheur}{qui soit au monde}\\

\haiku{Men leeft met en in,,.}{de heilige Rosalia}{Lucia Monica}\\

\haiku{Zo-even had hy}{den monnik verdriet gedaan}{door de betuiging}\\

\haiku{Nous sommes pay\'es,!}{pour le savoir nous autres}{passagers de pont}\\

\haiku{{\textquoteleft}dat dit epitheton!}{naar omstandigheden moet}{gewyzigd worden}\\

\haiku{d\'at boek kon wel eens.}{de vonk aanbrengen om die}{te doen ontvlammen}\\

\haiku{Er waren altyd ',.}{Grieken diet geloofden}{vooral in Boeoti\"e}\\

\haiku{Ik verwacht nu 'n:}{annonce van dezen of}{genen bierbrouwer}\\

\haiku{Ge wist dat ik 't.}{Volk misselyk had gemaakt}{van uw programmen}\\

\haiku{aan 't bouwen van ',.}{n huis met dikke muren}{en hoge torens}\\

\haiku{Het Nederlandse.}{Volk kan gerust wezen over}{die 115 millioen}\\

\haiku{En, Sire, dat is, ' {\textquoteleft}{\textquoteright}.}{wel waar maart was de vraag}{niet of ikmooi schreef}\\

\haiku{Het was niet blauw of,,:}{geel niet rood of groen of grijs}{maar alle verf door\'e\'en}\\

\haiku{Er zyn velen die,.}{geloven in myn God de}{Noodzakelykheid}\\

\haiku{Want, ook daarvoor zorgt,, '.}{de Natuur als men d\'at doet}{zalt niet vallen}\\

\haiku{'t Moet dus geweest.}{zyn v\'o\'or de ontdekking der}{staathuishoudkunde}\\

\haiku{Ja, van de toekomst,.}{als ons die muffe schoollucht}{zal afgewaaid zyn}\\

\haiku{die Wouter vergeefs '.}{tegent licht hield om er}{meer van te weten}\\

\haiku{- Ik heb gevraagd aan,,.}{m'n moeder zei hy maar ze}{wil me niets geven}\\

\haiku{- 't Is maar, weet je,.}{om je te zeggen dat we}{heel fatsoenlyk zyn}\\

\haiku{Lang na Habakuk,:}{dacht Wouter nog meermalen}{aan haar deemoedig}\\

\haiku{Ze was foei-lelyk,,.}{nogal vuil en bovendien}{wat onrecht van leest}\\

\haiku{Vyf vingers heb ik '.}{aan myn hand Ter eer vant}{lieve vaderland}\\

\haiku{De pruik maakte 'n,.}{vreugdesprong en de krullen}{omhelsden elkaar}\\

\haiku{En ook ken-i al '.}{de werrikwoorden f'nt}{frouwelik cheslacht}\\

\haiku{- Skenkerrissin, Trui,... '.}{en blaas es in de tuit d'r}{sitn blaatje foor}\\

\haiku{want ik heb altyd ',... '}{heel int fatsoenlyke}{gebakerd weet je}\\

\haiku{want-i was in de,,!}{granen en d\'a\'ar kan je na}{me vragen hoor je}\\

\haiku{Maar als men driftig, ' ' '.}{is neemt men welns meert}{een voort ander}\\

\haiku{O Baker, hoor ons,.}{juichen aan Als wij met U}{uit baak'ren gaan}\\

\haiku{Weg van Hem, en vaart,.}{naar de diepe gewelven}{waar geen Baker is}\\

\haiku{{\textquoteleft}De verhouding met.}{het ryk der Pieterse's is}{allercordiaalst}\\

\haiku{En juffrouw Laps hield,.}{veel van oefenen zoals}{wy gezien hebben}\\

\haiku{- Zie je, jongeheer, '.}{dat kan ik ze maar niet aan}{t verstand brengen}\\

\haiku{Die blik slaagde, maar,}{ik kan niet ontveinzen dat}{men verwonderd was}\\

\haiku{Neen, de verbazing ' '.}{vant gezelschap hadn}{heel anderen grond}\\

\haiku{- Maar wat moeten we?}{dan in godsnaam aanvangen}{met dien kwajongen}\\

\haiku{Jazelfs wordt spotten,.}{plicht waar redenering te}{vergeefs wezen zou}\\

\haiku{Maar toen m'n zuster, ' {\textquoteleft}{\textquoteright}.}{trouwde heeft zen grote}{kastmeegekregen}\\

\haiku{Dat deed Fancy niet,.}{dat deed de priester om die}{zes schellingen}\\

\haiku{Dat deed ik reeds te,:}{Lebak toen ik tot de}{bevolking zeide}\\

\haiku{Ik wou zo graag 't.}{kopyrecht daarvan aan m'n}{kinderen laten}\\

\haiku{Ook is 't onjuist.}{dat ik niet meer over hem zou}{te zeggen hebben}\\

\haiku{Doch byna overal ':}{elders sluiten wy zulke}{woorden metn t}\\

\haiku{Nu, als gyzelf op,.}{z'n nederlands denkt hebt ge}{kans van juist raden}\\

\haiku{Elk mot wordt weer op.}{zyn beurt door andere mots}{krachteloos gemaakt}\\

\haiku{Dat talent is my,}{niet gegeven waarin dan}{ook de reden ligt}\\

\haiku{Dit wordt wel gedaan}{in de voor omstreeks een jaar}{by den uitgever}\\

\haiku{De voorbeelden die,.}{ik hiervan kan aanhalen}{zyn anecdotisch}\\

\haiku{niet iedere sloep '.}{geoorloofdn schip van dien}{kant te naderen}\\

\haiku{op 't herstellen.}{der latynse u in haar}{wezenlyken klank}\\

\haiku{De etymologie '.}{vant woord presenning kan}{ik niet opgeven}\\

\haiku{Daarin schitteren.}{trouwens de meeste scheepstermen}{door afwezigheid}\\

\haiku{We zyn denkdieren,,:}{kunnen denken en voelen}{aandrang tot denken}\\

\haiku{het uitroeien der.}{vervloekte gewoonte van}{niet-begrypen}\\

\haiku{Des te gegronder '.}{is alzo de klacht int}{nu volgend nummer}\\

\haiku{Men zie hierin geen - '!}{bewondering van m'n werk}{t lykt er niets naar}\\

\haiku{Die zogenaamde.}{tussenscholen zullen nu}{wel niet meer bestaan}\\

\haiku{Maar dat men hem nu,.}{ook als auteur verheft is}{me enigszins  nieuw}\\

\haiku{Ziehier weder 'n,.}{fout van dezelfde soort als}{in 183 385 en 409}\\

\haiku{Met dit woord spreekt men.}{op beleefde wys een niet}{zeer jonge vrouw aan}\\

\haiku{de ene heeft op het:}{titelblad van het eerste}{deel de vermelding}\\

\haiku{En dit~3183'n K, ' ' -;}{die slechtsn C is metn}{stokje volgens 1879}\\

\haiku{Eerst in het najaar.}{van 1857 gelukte het de}{opstand te dempen}\\

\haiku{In 1753 hield Buffon:}{in de Acad\'emie Fran\c{c}aise}{zijn intreerede}\\

\haiku{derde regel van:}{het tot volkslied geworden}{gedicht van Heine}\\

\haiku{Het oratorium ().}{Die Sch\"opfung1798 is een van}{zijn meesterwerken}\\

\haiku{kozakkenhetman, (-).}{vertrouweling van Peter}{de Grote16441709}\\

\haiku{Aanvankelijk bij;}{de rechterlijke macht in}{Nederlands-Indi\"e}\\

\haiku{Sappho: grootste Griekse, (.}{dichteres geboren op}{Lesbos\ensuremath{\pm} 600 v}\\

\haiku{resp. staatsman (1623-1672) (-).}{en raadpensionaris}{van Holland16251672}\\

\subsection{Uit: Max Havelaar of De koffiveilingen der Nederlandsche Handelmaatschappy}

\haiku{Maar geen wonder dat,!}{later de toon bitterder}{het woord scherper werd}\\

\haiku{ik laat inspannen;}{en rij naar de Fille de}{madame Angot}\\

\haiku{Moet ik u zoeken,?}{in de wolken of in de}{straten eener stad}\\

\haiku{want niet ieder had.}{een hoornen huid waarop de}{indruk afschampte}\\

\haiku{{\textquoteright} En zij gaf eerst den,,.}{wil dan de kracht eindelijk}{de overwinning}\\

\haiku{De minnebrieven;}{zijn een vonkelend vuurwerk}{van vernuft en geest}\\

\haiku{dat de kunstenaar,.}{Goethe hier verkeerd deed maar}{de mensch Goethe juist}\\

\haiku{wat ik verlang... maak...}{dat je in drie maanden de}{eerste bent op school}\\

\haiku{- Heerejesis, zeit zijn,!}{moeder waar haalt de jongen}{de dingen van daan}\\

\haiku{Ik Heb van uw roem,.}{als rechtsman veel gehoord En}{wilde Van Huisde}\\

\haiku{Onder den titel;}{Verspreide stukken zijn er}{eenigen verzameld}\\

\haiku{Nu, zeker, Laura.}{Ernst of de school des levens}{is een juweeltje}\\

\haiku{lord Ci-devant:}{en de tabakshandelaar}{van de Brakke Grond}\\

\haiku{Lezer, ik heb u.}{genoeg gezegd om u op}{den weg te helpen}\\

\haiku{Wie redeneert, dient,.}{de rede en de rede}{zal u vrijmaken}\\

\haiku{Men vlucht met het een.}{of ander voorwerp naar het}{einde der aarde}\\

\haiku{De familie die,.}{van dezen arbeid leven}{kan heeft weinig noodig}\\

\haiku{Het is zoo niet in, '.}{de wereld ent is goed}{dat het niet zoo is}\\

\haiku{Hy had dan ook wel,.}{iets van een Duitscher en van}{een reiziger ook}\\

\haiku{Ja, ja, hy was het,!}{die my uit de handen van}{den Griek had verlost}\\

\haiku{Ik schreide, en bad,.}{om genade want ik zat}{vreeselyk in angst}\\

\haiku{Hy zag zeer bleek, en, '.}{toen ik hem vroeg hoe laat het}{was wist hyt niet}\\

\haiku{Maar Louise schreide,.}{weer en de dames zeiden}{dat het heel mooi was}\\

\haiku{Over hydraulische.}{onderwerpen in verband}{met de rystkultuur}\\

\haiku{Maar hy zegt, dat de.}{invloed zich eerst openbaart in}{het tweede geslacht}\\

\haiku{Ook vond ik brieven,.}{waarvan velen in talen}{die ik niet verstond}\\

\haiku{Natuurlyk had hy,.}{my om geld gevraagd en van}{zyn pak gesproken}\\

\haiku{Bovendien hy zag,...}{er armoedig uit en wist}{niet hoe laat het was}\\

\haiku{hoe zou 't wezen,,?}{dacht ik als ik hem de plaats}{van Bastiaans gaf}\\

\haiku{Ik ben zeker dat.}{hy met tweehonderd gulden}{tevreden zou zyn}\\

\haiku{Wat hy zeide, was -}{gewoonlyk lang overdacht een}{eigenaardigheid}\\

\haiku{Daar is me juist iets,...?}{voorgekomen dat zou de}{Regent ons verstaan}\\

\haiku{Maar toon my iets uit,.}{je weitasch dan denkt hy dat}{we d\'a\'arover spreken}\\

\haiku{D\'a\'ar moet geleden,,!}{zyn veel geleden daar is}{ondervonden}\\

\haiku{hy is beschaafd, niet? -... -?}{waar O ja En hy heeft een}{groote familie}\\

\haiku{Ik beveel me zeer,!}{aan voor uw medewerking}{m'nheer Verbrugge}\\

\haiku{, en daar het spreken, '.}{zoo moeielyk viel brak men}{t gesprek af}\\

\haiku{De toestand is er!}{sedert dien tyd niet beter}{op geworden}\\

\haiku{{\textquoteright} niet uitsprak, voor hy {\textquoteleft}{\textquoteright}.}{haar datniet zondigen had}{mogelyk gemaakt}\\

\haiku{een argument, zooals,.}{men weet waartegen niets valt}{intebrengen}\\

\haiku{Hy wie zyn woord spreekt,.}{opdat zy zich oprichten}{in hun ellende}\\

\haiku{Rypt niet uw padie?}{dikwerf ter voeding van wie}{niet geplant hebben}\\

\haiku{Wat zal er gezegd?}{worden in de dorpen waar}{wy gezag hadden}\\

\haiku{Hyzelf heeft dat geld, '.}{noodig en de kollekteur wil}{t hem voorschieten}\\

\haiku{- Ook de staten die,,.}{ik vandaag ontving zyn valsch}{ging Havelaar voort}\\

\haiku{t Was myn hart dat!}{ge daar hebt opgeslikt als}{een versnapering}\\

\haiku{Er is een juffrouw...}{flauw gevallen toen hy van}{dat zwarte kind sprak}\\

\haiku{Is niet reeds dit boek - -}{dat Stern me zoo zuur maakt een}{bewys hoe goed}\\

\haiku{Heimlich erz\"ahlen.}{die Rosen Sich duftende}{M\"archen ins Ohr}\\

\haiku{{\textquoteleft}deze kapel is...}{opgericht door den bisschop}{van Munster in 1423}\\

\haiku{Ze bewegen zich,.}{als een hobbelpaard minus}{nog het va et vient}\\

\haiku{Eigenliefde en.}{verveling dringen u iets}{dergelyks te doen}\\

\haiku{Ik zeg niet, daar heb,,:}{ik een vrouw gezien die z\'o\'o}{of z\'o\'o schoon was neen}\\

\haiku{waarom ik in uw!}{schatting verheven moest zyn}{boven verkoudheid}\\

\haiku{{\textquoteright} Havelaar scheen te,:}{verstaan wat Tine meende}{want hy antwoordde}\\

\haiku{- Welnu, dan weet je '.}{dat er peperkultuur in}{t Natalsche is}\\

\haiku{Myn verdienste was.}{te grooter omdat zy heel}{weinig antwoordde}\\

\haiku{wat is dit, dat die,?}{man macht heeft boven my en}{steenen houwt uit myn schoot}\\

\haiku{Als de ommelet,,... -}{overigens goed was zou dat}{geen bezwaar zyn maar}\\

\haiku{Maar de generaal.}{wilde my niet naar Natal}{laten vertrekken}\\

\haiku{De gevangene.}{immers is men onderhoud}{en voedsel schuldig}\\

\haiku{Hebt ge niet geweend,...}{by die moeder vruchteloos}{zoekend naar haar kind}\\

\haiku{- Ich weiss es nicht, ~ .}{Die Sterne Zahl hat Niemand}{noch gez\"ahlt}\\

\haiku{Men verkrygt daardoor '.}{den roep van bekwaamheid en}{yver voors lands dienst}\\

\haiku{Gisteren heb ik:}{Sjaalman gezien met zyn vrouw}{en hun jongetje}\\

\haiku{Hy is bleek als de,,.}{dood zyn oogen puilen uit en}{zyn wangen staan hol}\\

\haiku{Men ziet uit alles,.}{dat Stern jong is en weinig}{ondervinding heeft}\\

\haiku{En ook Sa{\"\i}djah's,.}{vader was zeer bedroefd doch}{zyn moeder het meest}\\

\haiku{Ook de ouders van.}{zyn vrouw woonden altyd in}{hetzelfde distrikt}\\

\haiku{Zie, Adinda, kerf.}{een streep in je rystblok by}{elke nieuwe maan}\\

\haiku{het vastgeknoopt aan '.}{t eindelyk terugzien}{onder den ketapan}\\

\haiku{{\textquoteleft}wie van ons zal het?}{lichaam verslinden dat daar}{daalt in het water}\\

\haiku{langs de knie neerviel...}{in heerlyke golving op}{den kleinen voet}\\

\haiku{En ook vroeg hy zich,?}{wie er toch wel wonen zou}{in zyns vaders huis}\\

\haiku{Ik weet het wel, ik,!}{weet het wel dat myn verhaal}{eentoonig is}\\

\haiku{Rangkas-Betoeng,,.}{25 Februari 1856 des}{avends te 11 ure}\\

\haiku{En ze hebben niets,,!}{te eten en ze slapen op}{den weg en eten zand}\\

\haiku{{\textquoteright} Ja, ik hoor het wel,,!}{ik hoor het wel dat roepen}{om wraak over myn hoofd}\\

\haiku{Het gesprek liep over.}{de verwachte beslissing}{van de Regeering}\\

\haiku{ze zyn in eere,!}{van hier gegaan en men schryft}{aan U zulk een brief}\\

\haiku{Maar ik zal tot hem.}{gaan en hem aantoonen hoe}{hier de zaken staan}\\

\haiku{Ik zou te Ngawi:}{hetzelfde moeten doen wat}{ik hier gedaan heb}\\

\haiku{Eindelyk liet hy.}{op-nieuw verzoeken om}{gehoord te worden}\\

\haiku{Fluks wierp hy 't in, '.}{de goot en reinigde het}{metn stalbezem}\\

\haiku{Nederland heeft niet.}{verkozen recht te doen in}{de Havelaarszaak}\\

\haiku{kort overzicht van de,.}{Woutergeschiedenis door}{Holda in den Ned}\\

\haiku{de herinnering.}{in het leven roept aan den}{javaschen oorlog}\\

\haiku{Dit is al iets in!}{onzen tyd van jammerlyk}{ordinarisme}\\

\haiku{Me dunkt toch dat ze,,.}{vooral met het oog op z'n}{dood zeer treffend zyn}\\

\haiku{Op strandplaatsen als.}{Marseille verbasteren}{de rassen zeer snel}\\

\haiku{) Het is zeer gewaagd '.}{dit op grond vann enkel}{woord aantenemen}\\

\haiku{Men behoorde den.}{moed te hebben zyner}{gewetenloosheid}\\

\haiku{Het blad zit, als 't ',.}{yzer vann houweel loodrecht}{op den houten steel}\\

\haiku{{\textquoteright} Maar nooit bleek er dat.}{er iets gedaan werd om dit}{doel te bereiken}\\

\haiku{wat hem te wachten '}{stond vann chef die toch even}{als hy gezworen}\\

\haiku{Multatuli, niet,.}{alleen in aangenomen}{naam maar inderdaad}\\

\haiku{Doch ik dacht er niet,.}{aan en heb geen verdienste}{van m'n discretie}\\

\haiku{staatkundige kleur -.}{nog altyd voor byzonder}{achtenswaardig door}\\

\subsection{Uit: Max Havelaar. Deel 1. Tekst}

\haiku{Men vlucht met het een.}{of ander voorwerp naar het}{einde der aarde}\\

\haiku{De familie die,.}{van dezen arbeid leven}{kan heeft weinig noodig}\\

\haiku{Het is zoo niet in, '.}{de wereld ent is goed}{dat het niet zoo is}\\

\haiku{Hy had dan ook wel,.}{iets van een Duitscher en van}{een reiziger ook}\\

\haiku{Ja, ja, hy was het,!}{die my uit de handen van}{den Griek had verlost}\\

\haiku{Ik schreide, en bad,.}{om genade want ik zat}{vreeselyk in angst}\\

\haiku{Hy zag zeer bleek, en, '.}{toen ik hem vroeg hoe laat het}{was wist hyt niet}\\

\haiku{Maar Louise schreide,.}{weer en de dames zeiden}{dat het heel mooi was}\\

\haiku{Over hydraulische.}{onderwerpen in verband}{met de rystkultuur}\\

\haiku{Maar hy zegt, dat de.}{invloed zich eerst openbaart in}{het tweede geslacht}\\

\haiku{Ook vond ik brieven,.}{waarvan velen in talen}{die ik niet verstond}\\

\haiku{Natuurlyk had hy,.}{my om geld gevraagd en van}{zyn pak gesproken}\\

\haiku{Bovendien, hy zag,...}{er armoedig uit en wist}{niet hoe laat het was}\\

\haiku{hoe zou 't wezen,,?}{dacht ik als ik hem de plaats}{van Bastiaans gaf}\\

\haiku{Ik ben zeker dat.}{hy met tweehonderd gulden}{tevreden zou zyn}\\

\haiku{Wat hy zeide, was -}{gewoonlyk lang overdacht een}{eigenaardigheid}\\

\haiku{Daar is me juist iets,...?}{voorgekomen dat zou de}{Regent ons verstaan}\\

\haiku{Maar toon my iets uit,.}{je weitasch dan denkt hy dat}{we d\'a\'arover spreken}\\

\haiku{D\'a\'ar moet geleden,,!}{zyn veel geleden daar is}{ondervonden}\\

\haiku{hy is beschaafd, niet? -... -?}{waar O ja En hy heeft een}{groote familie}\\

\haiku{Ik beveel me zeer,!}{aan voor uw medewerking}{m'nheer Verbrugge}\\

\haiku{, en daar het spreken, '.}{zoo moeielyk viel brak men}{t gesprek af}\\

\haiku{De toestand is er!}{sedert dien tyd niet beter}{op geworden}\\

\haiku{{\textquoteright} niet uitsprak, voor hy {\textquoteleft}{\textquoteright}.}{haar datniet zondigen had}{mogelyk gemaakt}\\

\haiku{een argument, zooals,.}{men weet waartegen niets valt}{intebrengen}\\

\haiku{zoo geen menschen rond,.}{die van problematische}{beroepen leven}\\

\haiku{Hy wie zyn woord spreekt,.}{opdat zy zich oprichten}{in hun ellende}\\

\haiku{Rypt niet uw padie?}{dikwerf ter voeding van wie}{niet geplant hebben}\\

\haiku{Wat zal er gezegd?}{worden in de dorpen waar}{wy gezag hadden}\\

\haiku{Hyzelf heeft dat geld, '.}{noodig en de kollekteur wil}{t hem voorschieten}\\

\haiku{- Ook de staten die,,.}{ik vandaag ontving zyn valsch}{ging Havelaar voort}\\

\haiku{t Was myn hart dat!}{ge daar hebt opgeslikt als}{een versnapering}\\

\haiku{Er is een juffrouw...}{flauw gevallen toen hy van}{dat zwarte kind sprak}\\

\haiku{Marie is daar in -,?}{dien rooien tuin waarom rood}{en niet geel of paars}\\

\haiku{{\textquoteleft}deze kapel is...}{opgericht door den bisschop}{van Munster in 1423}\\

\haiku{Ze bewegen zich,.}{als een hobbelpaard minus}{nog het va et vient}\\

\haiku{Eigenliefde en.}{verveling dringen u iets}{dergelyks te doen}\\

\haiku{ik zeg niet, daar heb,,:}{ik een vrouw gezien die z\'o\'o}{of z\'o\'o schoon was neen}\\

\haiku{waarom ik in uw!}{schatting verheven moest zyn}{boven verkoudheid}\\

\haiku{- Welnu, dan weet je '.}{dat er peperkultuur in}{t Natalsche is}\\

\haiku{Myn verdienste was.}{te grooter omdat zy heel}{weinig antwoordde}\\

\haiku{En de koning des,.}{lands toog voorby met ruiters}{voor zyn wagen}\\

\haiku{wat is dit, dat die,?}{man macht heeft boven my en}{steenen houwt uit myn schoot}\\

\haiku{Twaalfde hoofdstuk  -,,.}{Beste Max zei Tine ons}{dessert is zoo schraal}\\

\haiku{Ik zei dat elk mensch.}{in zyn medemensch een soort}{van konkurrent ziet}\\

\haiku{Maar de generaal.}{wilde my niet naar Natal}{laten vertrekken}\\

\haiku{te veel reeds gewerkt,.}{dan dat ik me verschuilen}{zou achter myn jeugd}\\

\haiku{Den gevangene.}{immers is men onderhoud}{en voedsel schuldig}\\

\haiku{Hebt ge niet geweend,...}{by die moeder vruchteloos}{zoekend naar haar kind}\\

\haiku{Men verkrygt daardoor '.}{den roep van bekwaamheid en}{yver voors lands dienst}\\

\haiku{Gisteren heb ik:}{Sjaalman gezien met zyn vrouw}{en hun jongetje}\\

\haiku{Hy is bleek als de,,.}{dood zyn oogen puilen uit en}{zyn wangen staan hol}\\

\haiku{Men ziet uit alles,.}{dat Stern jong is en weinig}{ondervinding heeft}\\

\haiku{Ook de ouders van.}{zyn vrouw woonden altyd in}{hetzelfde distrikt}\\

\haiku{Ik zal uitblyven....}{driemaal twaalf manen deze}{maan rekent niet mee}\\

\haiku{Zie, Adinda, kerf.}{een streep in je rystblok by}{elke nieuwe maan}\\

\haiku{het vastgeknoopt aan '.}{t eindelyk terugzien}{onder den ketapan}\\

\haiku{{\textquotedblleft}wie van ons zal het?}{lichaam verslinden dat daar}{daalt in het water}\\

\haiku{langs de knie neerviel...}{in heerlyke golving op}{den kleinen voet}\\

\haiku{En ook vroeg hy zich,?}{wie er toch wel wonen zou}{in zyns vaders huis}\\

\haiku{Ik weet het wel, ik,!}{weet het wel dat myn verhaal}{eentonig is}\\

\haiku{- Ach, zeide zy, er! -,.}{is zooveel slecht volk Zeker}{dat is er overal}\\

\haiku{Ik heb de meeste,}{hoogachting voor u maar ik}{ken den geest dien men}\\

\haiku{vroeg de moeder.175~		 -, -? -!}{Ik zei kleine Max. En wat}{beduidt dat Bedtyd}\\

\haiku{En ze hebben niets,,!}{te eten en ze slapen op}{den weg en eten zand}\\

\haiku{{\textquoteright} Ja, ik hoor het wel,,!}{ik hoor het wel dat roepen}{om wraak over myn hoofd}\\

\haiku{Het gesprek liep over.}{de verwachte beslissing}{van de Regeering}\\

\haiku{Duclari, een zeer,:}{beschaafd mensch berstte in een}{wilden vloek uit}\\

\haiku{ze zyn in eere,!}{van hier gegaan en men schryft}{aan U zulk een brief}\\

\haiku{Maar ik zal tot hem.}{gaan en hem aantoonen hoe}{hier de zaken staan}\\

\haiku{Ik zou te Ngawi:}{hetzelfde moeten doen wat}{ik hier gedaan heb}\\

\haiku{Eindelyk liet hy.}{op-nieuw verzoeken om}{gehoord te worden}\\

\haiku{Fluks wierp hy 't in, '.}{de goot en reinigde het}{metn stalbezem}\\

\haiku{de herinnering '.}{int leven roept aan den}{javaschen oorlog}\\

\haiku{Dit is al iets in!}{onzen tyd van jammerlyk}{ordinarisme}\\

\haiku{Me dunkt toch dat ze,,.}{vooral met het oog op z'n}{dood zeer treffend zyn}\\

\haiku{Op strandplaatsen als.}{Marseille verbasteren}{de rassen zeer snel}\\

\haiku{) Het is zeer gewaagd '.}{dit op grond vann enkel}{woord aantenemen}\\

\haiku{Men behoorde den.}{moed te hebben zyner}{gewetenloosheid}\\

\haiku{Het blad zit, als 't ',.}{yzer vann houweel loodrecht}{op den houten steel}\\

\haiku{{\textquoteright} Maar nooit bleek er dat.}{er iets gedaan werd om dit}{doel te bereiken}\\

\haiku{wat hem te wachten '}{stond vann chef die toch even}{als hy gezworen}\\

\haiku{staatkundige kleur -.}{nog altyd voor byzonder}{achtenswaardig door}\\

\chapter[8 auteurs, 659 haiku's]{acht auteurs, zeshonderdnegenenvijftig haiku's}

\section{Top Naeff}

\subsection{Uit: In mineur}

\haiku{{\textquoteleft}Ik ga morgen naar,,...{\textquoteright}}{Deventer altijd Zondag's}{naar de oude lui}\\

\haiku{re... {\textquoteleft}de Brinkhorst{\textquoteright}, 't...?}{buiten van Meneer de Ras}{Alting weet je dat}\\

\haiku{Dit is 't. Mooie streek...{\textquoteright}.}{h\`e De woorden verstikten}{in zijn schorre keel}\\

\haiku{Wreed was hij, wreed in ',.}{t onvermijdelijke}{wreed ook in het noodelooze}\\

\haiku{Met haar oogen groette,,.}{ze stijf recht-\'op zwaar geleund}{tegen de tafel}\\

\haiku{{\textquoteright} En zij speurde niet.}{den toon van ironie onder}{die luchtige scherts}\\

\haiku{{\textquoteleft}Nacht juffrouw Brantsberg,{\textquoteright}, {\textquoteleft}?}{zei hijkan ik u nergens}{meer mee van dienst zijn}\\

\haiku{{\textquoteright} Bruusk ging hij heen, riep,,.}{springend in het rijtuig zijn}{adres aan den koetsier}\\

\haiku{Om hem een geur van,...}{oranjebloesem rozen en}{heliotrope}\\

\haiku{De kellner bracht bier,.}{aan morste kleine plasjes}{op den vuilen grond}\\

\haiku{Misschien zou 't kind, ',.}{hem gauw volgen hij hoopte}{t zoo'n stakkerdje}\\

\haiku{Vanmiddag wacht hij.}{haar weer voor nog een kleine}{verandering}\\

\haiku{{\textquoteright} Gebogen glipt zij,, '.}{de deur uit die achter haar}{dicht slaat int slot}\\

\haiku{'s Morgens, om half,:}{zeven wekte hij haar weer}{met schorre slaapstem}\\

\haiku{Aldoor grinnikend,,,,.}{slipte Christien de deur uit}{sloeg die knallend dicht}\\

\haiku{Verder was op den,.}{ledigen langen weg geen}{mensch te bespeuren}\\

\haiku{Christien zakte op,,, '.}{de knie\"en leunde haar kin}{spits opt kozijn}\\

\haiku{Christien's hart trilde,.}{van trots toen zij daar het hek}{intrad met een heer}\\

\haiku{Zoo eindigde de.}{liefdesgeschiedenis van}{juffrouw Christien}\\

\haiku{De goejen moeten,{\textquoteright},,.}{met de kwajen lijden}{zei vinnig Christien}\\

\haiku{{\textquoteleft}Je moest toch ook nog,.}{eens gaan naar Suze en Frits}{zij drongen zoo \'a\'an}\\

\haiku{Hoe waardeerde zij,;}{dan niet te wonen in een}{groote stad vol gevaar}\\

\haiku{Zij telde... nog een......,,......}{keer stilte nog eenmaal kort}{als een snik nog een}\\

\haiku{En daarom deinst hij,.}{nu terug voor een daad die}{juist moeder verbood}\\

\haiku{Barend, op den kant,,:}{trapt forsch af met zijn voet}{tegen den bootrand}\\

\haiku{De andere mouw,,,.}{opgeblazen klaar lijkt een}{arm te omsluiten}\\

\haiku{Mien God, hoe vake ' ',...}{haktm veureholden}{hoe gevaorlijk}\\

\section{Edouard de N\`eve}

\subsection{Uit: Muziek voorop}

\haiku{{\textquoteleft}U bent zeker van,?}{de eerste lichtingen die}{op moeten komen}\\

\haiku{Hij weet niet meer wat,,.}{te zeggen en eet diep in}{gedachten verder}\\

\haiku{Ondertusschen denkt.}{hij aan het onrecht dat hij}{Francine aandoet}\\

\haiku{En als ze ziet dat:}{Jean teleurgesteld is voegt}{ze er haastig bij}\\

\haiku{Zij heeft alleen met.}{Jean te maken omdat ze}{met h\`em getrouwd is}\\

\haiku{Beroerd, eindeloos,.}{getik dat haar verhindert}{te slapen vannacht}\\

\haiku{{\textquoteleft}Je weet toch dat ik,}{niet anders kon dat ik m'n}{moeder moest zien v\'o\'or}\\

\haiku{Aan mij heeft ze   -.}{altijd een hekel gehad}{al v\'o\'or we trouwden}\\

\haiku{Daarmee schoof hij reeds.}{een eind van het uiterste}{linksche plaatsje weg}\\

\haiku{Dit lachen verbreekt,.}{de angstige spanning die}{hen omvangen houdt}\\

\haiku{Elk woord kan een klacht,.}{zijn elke klacht kan barsten}{tot een huilpartij}\\

\haiku{{\textquoteright} Francine wendt zich.}{om om niet te laten zien}{dat zij nu toch huilt}\\

\haiku{{\textquoteleft}Waarom stuurde hij,?}{hem dan niet direct aan mij}{zooals hij altijd doet}\\

\haiku{En Jean heeft er me,{\textquoteright}.}{nooit een verwijt van gemaakt}{zei Francine koel}\\

\haiku{Maar onmiddellijk.}{daarna was Francine weer}{tot vechten gereed}\\

\haiku{Maar het was misschien,....}{geen grap dat lintje van het}{Legioen van Eer}\\

\haiku{De duimschroeven aan,,,!}{zeg ik je en de poet in}{zonder pardon rang}\\

\haiku{{\textquoteright} Uit de rijen der.}{Hollandsche compagnie stapt}{een man naar voren}\\

\haiku{Hij wentelt er om,....}{en om als een scherpe boor}{in een rotte plank}\\

\haiku{Even later, als hij,.}{in zijn bed ligt kan hij niet}{in slaap  komen}\\

\haiku{Hier en daar kromde.}{een enkele vrouw zich over}{haar  veldarbeid}\\

\haiku{En zeggen dat er....}{nog kaffers zijn die blij zijn}{naar het front te gaan}\\

\haiku{En nu kan het me,,.}{niets meer schelen niets meer wat}{er met me gebeurt}\\

\haiku{Voor mijn part krijg ik '....}{n kogel zoo gauw als}{ik aan het front kom}\\

\haiku{En zij weten niet.}{wanneer zij hun bestemming}{zullen bereiken}\\

\haiku{{\textquoteright} {\textquoteleft}Ja, natuurlijk, als,....}{je de geldpest hebt zooals jij}{kun je zooiets doen}\\

\haiku{Hij had zoo graag \'o\'ok,.}{naar Parijs willen gaan maar}{hij heeft niet gedurfd}\\

\haiku{je zal zeggen hoe,}{graag ik bij je zou zijn en}{als ik gedurfd had}\\

\haiku{Terwijl Monsieur:}{Lasalle zijn auto ging}{halen vroeg L\'eonie}\\

\haiku{{\textquoteright} Francine haalde.}{haar schouders op en wist niet}{goed wat te zeggen}\\

\haiku{Goed,{\textquoteright} zei ze, {\textquoteleft}vanavond,.}{ga ik met je mee daar kun}{je op rekenen}\\

\haiku{{\textquoteright} {\textquoteleft}Daar moet je ze geen,{\textquoteright}.}{gelegenheid voor geven}{ried L\'eonie heftig}\\

\haiku{Zij weet niet hoe zij.}{haar bekentenis onder}{woorden moet brengen}\\

\haiku{Haar eigen lafheid,.}{vergetend kon zij die van}{Jean niet vergeven}\\

\haiku{Onderwijl zou zij.}{doen alsof er niets was dat}{haar verontrustte}\\

\haiku{Maar nooit sprak zij hem.}{over wat zij besloten had}{hem  te zeggen}\\

\haiku{Maar onmiddellijk.}{daarop begon hij over iets}{anders te praten}\\

\haiku{Onder tentdoek en.}{dekens kruipen ze weg en}{trachten te slapen}\\

\haiku{En als je nog \'e\'en,,....}{woord zegt dan nom de Dieu}{sla ik je bek in}\\

\haiku{Vergeefs trachtte hij.}{zich wijs te maken dat hij}{daar geen recht op had}\\

\haiku{Het was een uitkomst.}{voor hem dat die soldaat hem}{had uitgelachen}\\

\haiku{Hij was alleen maar}{ongeduldig geworden}{omdat hij bang was}\\

\haiku{De wereld was vol.}{van scharrelaars en Parijs}{in de eerste plaats}\\

\haiku{Jean zou de eerste...}{zijn van zijn regiment om}{met verlof te gaan}\\

\haiku{Verlof zouden ze,,.}{krijgen dat wisten ze maar}{niemand wist wanneer}\\

\haiku{{\textquoteright} Het leven in de.}{eerste linie was voor hen}{vol onzekerheid}\\

\haiku{Hij spreekt er met zijn,:}{reisgenooten over en}{een adjudant zegt}\\

\haiku{Denk je dat ik me....}{afvroeg waar\`om ik w\`el en de}{anderen nog niet}\\

\haiku{Bij den uitgang van.}{het station nemen ze}{afscheid van elkaar}\\

\haiku{E\'en woordje tegen.}{den chauffeur en zijn taxi brengt}{hem naar Francine}\\

\haiku{Als de Duitschers hem,.}{dan maar niet misten als hij}{er dan maar in bleef}\\

\haiku{Hij was overtuigd dat.}{Francine's huwelijk zoo}{goed als kapot was}\\

\haiku{Francine zou   '.}{nooit verliefd opn man als}{Lemerre worden}\\

\haiku{En ook omdat ik.}{niet wilde dat je minder}{over mij zou denken}\\

\haiku{Maar als het is wat,.}{je moeder me verweten}{heeft dan is het waar}\\

\haiku{Hij heeft nooit van z'n.}{moeder een edelmoedige}{opwelling gezien}\\

\haiku{Onmiddellijk na.}{de opleving is haar hoop}{weer ineengestort}\\

\haiku{{\textquoteright} Francine had maar.}{wijselijk verzwegen dat}{zij \'o\'ok zoo'n jurk had}\\

\haiku{Morgen, over 'n paar,.}{dagen ten hoogste zullen}{ze niets meer krijgen}\\

\haiku{Vijftig meter zijn.}{genomen over een lengte}{van honderd misschien}\\

\haiku{Zijn blonde haar hangt,,.}{in pieken met bloed besmeurd}{over z'n bleeke gezicht}\\

\haiku{Ze wachten op het.}{fluitsignaal dat den aanval}{moet ontketenen}\\

\haiku{En Jean zou er haar,.}{misschien komen opzoeken}{een sc\`ene maken}\\

\haiku{Dat had zij al die,.}{maanden opgespaard daar had}{zij niet aan geraakt}\\

\haiku{Hij was niet streng voor,.}{zijn soldaten hij leefde}{ook niet met hen mee}\\

\haiku{Nu wist hij niet eens.}{meer of Francine nog in}{hun woning wachtte}\\

\haiku{Indien zij niet schreef,,.}{niet antwoordde beteekende}{dat dat zij weg was}\\

\haiku{V\'o\'or hij terug gaat.}{naar het front wil hij dit ook}{\'e\'en keer meemaken}\\

\haiku{{\textquoteright} V\'o\'or hij zijn derde.}{galon krijgt ontvangt hij het}{Legioen van Eer}\\

\haiku{Ze denken dat ze.}{\'o\'ok iets voor het vaderland}{moeten over hebben}\\

\haiku{En hij weet dat die.}{rust bedenkelijk is en}{niet van langen duur}\\

\haiku{De mannen vallen,.}{om van slaap maar zij durven}{niet te gaan liggen}\\

\haiku{Alles is onze,.}{schuld van  jou en van mij}{en van den oorlog}\\

\haiku{Hij w\'e\'et dat hij geen,.}{held is dat hij niets gedaan}{heeft uit echten moed}\\

\haiku{De Russen zijn een.}{onordelijke bende}{en staken den strijd}\\

\haiku{Dat heeft zij ge\"eischt.}{omdat  zij haar vrijheid}{wilde behouden}\\

\haiku{De Italianen.}{worden teruggeslagen}{op de Piava}\\

\haiku{Zij houdt zich echter,:}{goed en zegt haar hand over haar}{voorhoofd strijkende}\\

\haiku{Zij begrijpt vaag dat.}{hij iets anders heeft gewild}{dan beroemd worden}\\

\haiku{Iedere natie.}{wil de eer der overwinning}{voor zich opeischen}\\

\section{Edmond Nicolas}

\subsection{Uit: Brocaat en boerenbont. Schering en inslag van een fabrikantenleven}

\haiku{Na een paar jaren:}{begon hij de wevers zijn}{wil op te leggen}\\

\haiku{laat hem geloven.}{dat hij afhankelijk is}{van de producent}\\

\haiku{Prosper bloosde tot in:}{zijn hals en antwoordde met}{een klein stemmetje}\\

\haiku{Enfin, hij werd door:}{Fr\`ere Canis met volle}{muziek ontvangen}\\

\haiku{{\textquoteright} Op een avond sprak Prosper.}{over deze uitingen van}{Oom Jan met Papa}\\

\haiku{{\textquoteright} Maar geleidelijk,.}{begon Prosper te betalen}{binnen zes maanden}\\

\haiku{Toen het testament,.}{geopend werd meenden ze}{het te begrijpen}\\

\haiku{En zorg nu maar eens.}{dat die order voor Funck en}{Co de deur uitkomt}\\

\haiku{onder de prijs van,.}{Leiden maar zo dat de markt}{niet bedorven werd}\\

\haiku{{\textquoteright} Dat antwoord bracht de;}{dokter zijn moeilijkheden}{weer in gedachte}\\

\haiku{Als hij nu nog aan,.}{lager wal raakte dan was}{alles in orde}\\

\haiku{Maar meneer Snakkers,,.}{had geen tijd dank U wel een}{andere keer graag}\\

\haiku{goed{\textquoteright} zei hij, {\textquoteleft}en ze.}{zullen de eerste jaren}{nog wel beter gaan}\\

\haiku{Maar voor Anna was.}{die zeventiende Mei een}{groot wonder gebeurd}\\

\haiku{Goed dus Dinsdag om,.}{zes uur dan kon ze de Mis}{zelf ook bijwonen}\\

\haiku{Prosper had zich achter,.}{het oor gekrabt en had haar}{lang aangekeken}\\

\haiku{De tuinman nam het,;}{glaasje aan en dronk het bij}{kleine teugjes uit}\\

\haiku{{\textquoteright} Langzaam liep hij met,}{Anna op het klooster toe}{en trok aan de bel.}\\

\haiku{de tweede vleugel.}{zou dan worden bestemd voor}{koetshuis en stallen}\\

\haiku{Hij hoorde van de.}{notaris dat het land erg}{vast in de hand zat}\\

\haiku{{\textquoteright} Bruysten zette zijn.}{ijzeren bril stevig op}{zijn mopsneus en keek}\\

\haiku{Ik zei gisteren?}{toch duidelijk dat ik een}{kasteel ging zetten}\\

\haiku{En toen begon de.}{wonderlijke en goede}{tijd van het convent}\\

\haiku{Het was echter geen,.}{erg resolute Fanny}{die Anna aantrof}\\

\haiku{Maar er was wel een,.}{hoge officier bij van}{de Spahi's nog wel}\\

\haiku{{\textquoteleft}Als ze nu nog \'e\'en....}{keer iets aan het plan willen}{veranderen dan}\\

\haiku{En nu komt U wel,,.}{zeggen dat het schande is}{om zo te bouwen}\\

\haiku{Het enige wat hem,:}{interesseerde  was}{of we opschoten}\\

\haiku{Dat was een kunstwerk,.}{en het is zonde dat het}{afgebroken is}\\

\haiku{{\textquoteleft}Als er nu weer een,.}{mond bij is moest je maar wat}{meer gaan verdienen}\\

\haiku{Maar nu luister eens,,,?}{mijn jongen je hebt het niet}{royaal is het wel}\\

\haiku{{\textquoteright} {\textquoteleft}Dat is te zeggen,,,.}{met een hit wel meneer Prosper}{of een mak beestje}\\

\haiku{Maar dat kon hij, zei,.}{Prosper wel informeren bij}{de franse paters}\\

\haiku{En geleidelijk:}{ontstond een gewoonte in}{het jonge gezin}\\

\haiku{En toen verstoorde,,.}{Prosper volkomen onbewust}{het prachtige feest}\\

\haiku{Nee, meneer Prosper, met,.}{alle respect voor U dat}{gaat zo niet langer}\\

\haiku{De tweede avond was,.}{er weer een volle fles en}{de derde avond weer}\\

\haiku{Maar Marius bleef.}{geen uur langer in Parijs}{dan strikt nodig was}\\

\haiku{De stoffenkoopman,.}{placht van huis tot huis te gaan}{en sloeg geen deur over}\\

\haiku{{\textquoteright} Even leek het, alsof,.}{Prosper iets tegen wou zeggen}{kwaad worden misschien}\\

\haiku{{\textquoteright} {\textquoteleft}Dat is goed en wel{\textquoteright}, {\textquoteleft}.}{zei Mariusmaar daar is}{ook mode  in}\\

\haiku{En voor je het weet,.}{heb je een assortiment}{waar geen eind aan is}\\

\haiku{de concurrentie.}{van Lyon en het Rijnland}{was trouwens te sterk}\\

\haiku{Het werd tien uur, elf,.}{uur middag en  Margot}{was zoek en bleef zoek}\\

\haiku{dat wil zeggen, zeer.}{binnen de grenzen van het}{betamelijke}\\

\haiku{we verkopen, des,.}{te beter is het voor ons}{de aandeelhouders}\\

\haiku{De combinatie:}{had \'e\'en merkwaardige}{specialiteit}\\

\haiku{zijn de mensen waar,.}{ik vanaf kom en waar jij}{ook vanaf komt Prosper}\\

\haiku{Hij betaalde zijn,.}{mensen goed hij zorgde zelfs}{voor hun oude dag}\\

\haiku{Des avonds was er op.}{het kasteel groot diner voor}{veertig personen}\\

\haiku{Een zangwedstrijd van.}{dubbelmannenkwartetten}{zou op zijn plaats zijn}\\

\haiku{Toen de duisternis,;}{langer duurde gingen de}{mensen stil naar huis}\\

\haiku{Donderdag kwam er.}{een uitvoerig telegram}{van Prosper uit Berlijn}\\

\haiku{Het pakte anders.}{uit dan Wagemans en zijn}{vrienden verwachtten}\\

\haiku{Zo betaalde dus;}{eigenlijk Prosper het loon van}{de werkelozen}\\

\haiku{Prosper voelde erg veel,.}{voor de Jezu{\"\i}eten in}{Katwijk bijvoorbeeld}\\

\haiku{eerlijk spelen{\textquoteright} zei, {\textquoteleft}.}{hij half-ernstigniet de}{jongen opstoken}\\

\haiku{Prosper richtte zich op,,.}{zover dat zijn bovenlijf}{achterover helde}\\

\haiku{Maar die rijkdommen,:}{gaven hem geen voldoening}{en hij wist waarom}\\

\haiku{Ze kunnen niet in,,.}{hun blootje gaan werken Prosper}{gebruik je hersens}\\

\haiku{{\textquoteright} {\textquoteleft}Als je gezien hebt,,.}{waar ze mee lopen dan zou}{je dat niet zeggen}\\

\haiku{Des avonds kwam Herman,,.}{uitgenodigd door Anna}{eten op het kasteel}\\

\haiku{En toen kwam er een.}{merkwaardige brief in de}{brievenbus van Prosper}\\

\haiku{Prosper was eigenlijk,,:}{een acteur een groot acteur}{die maar \'e\'en fout had}\\

\haiku{dat hij daar in de,.}{volksbond woonde maar hij was}{ongezeggelijk}\\

\haiku{Als je later denkt,.}{dat je wat bereikt hebt doe}{er dan afstand van}\\

\haiku{ANNA WONEN in.}{een pension voor oude}{dames in Brussel}\\

\subsection{Uit: De erfenis}

\haiku{Die nonnenkostschool,.}{niet meegerekend want die}{heb ik niet geteld}\\

\haiku{{\textquoteleft}Nee, dat was het niet,,.}{maar we spraken over onze}{plannen Driek en ik}\\

\haiku{{\textquoteright} {\textquoteleft}Dat had niets met de,{\textquoteright}, {\textquoteleft}}{erfenis te maken zei}{de vader waardig}\\

\haiku{Ang\`ele zette haar,:}{bril af om op afstand te}{kunnen zien en zei}\\

\haiku{Jules, geef meneer.}{Claudius iets om de keel}{te bevochtigen}\\

\haiku{Het testament, zou,.}{men kunnen zeggen bestaat}{uit twee gedeelten}\\

\haiku{Even ontstond er een.}{geprikkeld gezwatel door}{de hele kamer}\\

\haiku{{\textquoteright} Plechtig en correct.}{begon Jules de brede}{trap te beklimmen}\\

\haiku{{\textquoteleft}Als je me nu de,,.}{post brengt Jules dan zal ik}{die eerst afwerken}\\

\haiku{{\textquoteright} De stem zweeg, en het.}{leek Paca of van haar een}{antwoord verwacht werd}\\

\haiku{Die krijgt het van twee,.}{kanten van de Frenckens}{\`en van de Wevers}\\

\haiku{Ze glimlachte haar.}{openste glimlach en vroeg of}{het zo beter was}\\

\haiku{Deze Grace was,.}{een wrak een ru{\"\i}ne van}{wat ze geweest was}\\

\haiku{En wou jij nu de?}{zaak voortzetten met deze}{kieteltuinmeubels}\\

\haiku{Merkwaardig,{\textquoteright} dacht hij, {\textquoteleft}.}{wat zo'n erfenis toch een}{geluk kan brengen}\\

\haiku{{\textquoteleft}Paca heeft een baan,.}{en is voor het eerst van haar}{leven zelfstandig}\\

\haiku{{\textquoteleft}Je me demande,{\textquoteright},:}{begon hij maar de monnik}{interrumpeerde}\\

\haiku{{\textquoteleft}Als die vrouwmensen,.}{aan de gang zijn geweest kun}{je niets meer vinden}\\

\haiku{Dat moet geweest zijn,,.}{eens kijken bij de bruiloft}{van Henri W\"osten}\\

\haiku{Toen hij het voertuig}{in het oog kreeg wenkte hij}{met zijn grote hoed}\\

\haiku{Dom Willem leidde:}{heel wat herinneringen}{in met de woorden}\\

\haiku{Onze auto kan.}{U wel naar een andere}{bestemming brengen}\\

\haiku{Maar ik zal wel een.}{paar fleskes achterhouden}{voor eigen gebruik}\\

\haiku{En ik wil je wel.}{mager laten eten ook als}{je daarnaar verlangt}\\

\haiku{{\textquoteleft}Als er een oorlog,.}{komt hoeven we niet over de}{toekomst te spreken}\\

\haiku{Nog een paar jaar, en,.}{dan is Clotje een schone}{engel hopen we}\\

\haiku{Ja, ik had wel werk,.}{genoeg maar je wilt er toch}{wel eens over praten}\\

\haiku{Matthieu stond op, en.}{begon opgewonden heen}{en weer te lopen}\\

\haiku{{\textquoteleft}Omdat meneer de.}{Vries in de gaten heeft dat}{ik geen kind meer ben}\\

\haiku{Ik ben natuurlijk,.}{niet gegaan maar mijn broertje}{Jim is gaan kijken}\\

\haiku{Maar als U denkt dat,.}{dit lesje geholpen heeft}{dan vergist U zich}\\

\haiku{Het antwoord was een,,.}{beetje vaag dat hing ervan}{af zei de ander}\\

\haiku{De Vries vermande,.}{zich en kreet dat Claudius}{beledigend was}\\

\haiku{Wat ze anders zou,.}{kunnen betekenen is}{me niet duidelijk}\\

\haiku{{\textquoteright} zei hij na een paar.}{seconden ingespannen}{getuurd te hebben}\\

\haiku{{\textquoteleft}Denk eraan, Alexander,,{\textquoteright}.}{dat het een zakenbezoek}{is zei Clotje streng}\\

\haiku{Tenslotte is een.}{politiek vluchteling een}{man van karakter}\\

\haiku{Maar opeens barstte.}{Ang\`ele uit in een bijna}{hysterisch gelach}\\

\haiku{brocaten tasjes,.}{fluwelen reticules}{en dergelijke}\\

\haiku{dit was de enige.}{manier om U onder vier}{ogen iets te zeggen}\\

\haiku{Even aarzelde ze,.}{en sneed toen resoluut de}{enveloppe open}\\

\haiku{{\textquoteleft}Ik ben verbaasd over,.}{de mate van gelijk die}{je hebt Claudius}\\

\haiku{Hij was trouwens eerst.}{voorgisteren in den Haag}{geweest voor zaken}\\

\haiku{{\textquoteleft}Zo, Driek, wat zei je?}{ook weer over ontmoetingen}{op vreemde plaatsen}\\

\haiku{het luik had geen slot,.}{en zonder moeite openden}{Matthieu en Driek het}\\

\haiku{{\textquoteleft}Hij is wel een zoon,.}{van Gertrude maar hij schijnt}{toch niet te deugen}\\

\haiku{Hoge rijgschoenen,,.}{bijvoorbeeld met een nogal}{hoge brede hak}\\

\haiku{{\textquoteright} {\textquoteleft}Volgens haar was het,{\textquoteright}.}{gevaar niet voorbij lichtte}{Claudius haar in}\\

\haiku{Hoe, daar heb ik nog,.}{geen idee van maar dat mogen}{we wel aannemen}\\

\haiku{{\textquoteright} kreet Clotje opeens,.}{en ze deed een vlugge greep}{naar de paraplu}\\

\haiku{En nu begrijp je,,.}{wel meneer dat ik nu geen}{risico's meer neem}\\

\haiku{kreeg verlof van de.}{verpleegster om even met zijn}{zwager te praten}\\

\haiku{Het lijkt wel of je!}{de hele nacht niet uit de}{kleren bent geweest}\\

\haiku{Maar waarvoor in 's,?}{hemelsnaam waarvoor wou hij}{dat graf openmaken}\\

\haiku{En wat mijn train de,!}{vie betreft ik heb goddank}{mijn eigen fortuin}\\

\haiku{En aan hem dacht zij.}{het eerste toen Claudius}{sprak over sanering}\\

\haiku{{\textquoteleft}En ik stel voor, dat.}{we dit onderhoud ergens}{anders voortzetten}\\

\haiku{Hij wist dat hij hier.}{een pertinente leugen}{had neergeschreven}\\

\haiku{{\textquoteleft}Kom,{\textquoteright} zei Claudius, {\textquoteleft},,.}{we stappen op anders vat}{je nog kou meisje}\\

\haiku{{\textquoteleft}En ik,{\textquoteright} besloot Don, {\textquoteleft}!}{Carlosben de tweede zoon}{van die Gertrude}\\

\haiku{Men kon, om deze,.}{te construeren uitgaan}{van twee gedachten}\\

\haiku{Sommigen dachten.}{dat men goed oud moest worden}{om goed gek te doen}\\

\haiku{Maar hij is zover,,.}{dat hij zijn lichaam niet meer}{opmerkt naar hij zegt}\\

\haiku{{\textquoteleft}Tenslotte,{\textquoteright} zei ze, {\textquoteleft}.}{later tegen Servaasis}{geld een idioot iets}\\

\haiku{{\textquoteleft}Ik vind het wel een,,}{beetje gewaagd wat U doet}{nicht Berendina}\\

\haiku{Tenslotte verscheen,,.}{de moeder een beetje moe}{een beetje ontdaan}\\

\haiku{Ik ga iets drinken.}{dat ik sinds de watersnood}{niet meer heb geproefd}\\

\haiku{Dan gaat ze maar nu.}{en dan eens een dagje naar}{Engeland terug}\\

\haiku{Ik geloof zelfs dat {\textquoteleft}{\textquoteright}.}{zeU tegen zichzelf zegt}{als ze in bad zit}\\

\haiku{{\textquoteright} Bewonderend keek.}{Claudius zijn liegende}{echtgenote aan}\\

\haiku{Maar dat nam niet weg.}{dat Paca zich een beetje}{verwaarloosd voelde}\\

\haiku{En onze dokter.}{zegt dat het best kan komen}{door die verwantschap}\\

\haiku{{\textquoteright} Zelfs Berendina:}{Volleboezem was mild in}{haar uitlatingen}\\

\haiku{wie van jullie meent?}{me te kunnen wijzen op}{ernstige fouten}\\

\subsection{Uit: De heer van Jericho}

\haiku{Buiten dien muur ligt,.}{de hofstede waar de halfer}{van Jericho boert}\\

\haiku{Dit is goed, maar er,.}{ontbreekt iets aan dat ik niet}{kan definieeren}\\

\haiku{{\textquoteright} Nu en dan echter.}{reed de baron Aboe niet naar}{zijn oefenweide}\\

\haiku{hijzelve zeide,}{dat het zoo eenvoudig was}{dat het niet loonde}\\

\haiku{Hij kreeg de helft van.}{het land en meer dan de helft}{van de contanten}\\

\haiku{{\textquoteright} vroeg hij, toen hij met.}{baron Delsain aan een licht}{glaasje moezel zat}\\

\haiku{En de stalknecht dook:}{op uit het donker van den}{stal en riep terug}\\

\haiku{Voor U moet ik de,}{orde der Schitterende}{Schoonheid cre\"eeren}\\

\haiku{Het is waard om leed,.}{te hebben wanneer men op}{dit licht kan hopen}\\

\haiku{Ik vind het verkeerd,,{\textquoteright}.}{dat in mijn huis gewed wordt}{zeide de baron}\\

\haiku{{\textquoteleft}Ik vraag je excuus,,{\textquoteright}, {\textquoteleft}.}{Schenck zeide hijmaar ik}{liet me meeslepen}\\

\haiku{je broer was gek, je,.}{zuster was gek jij bent het}{gekste van allemaal}\\

\haiku{Het leven van den.}{jonker in Brussel kostte}{inderdaad v\'e\'el geld}\\

\haiku{De dood kan nooit zoo,.}{verschrikkelijk zijn als het}{deinen der golven}\\

\haiku{Hoe mijn geachte,.}{voorvader dat gelapt heeft}{is me een raadsel}\\

\haiku{op een anderen{\textquoteright}.}{dronk dan dit aftreksel van}{chineesche kruiden}\\

\haiku{{\textquoteright} {\textquoteleft}Je wint,{\textquoteright} zeide de,.}{baron met een geste van}{ontwapend te zijn}\\

\haiku{En haar adem bleef net,.}{zoo kalm alsof ze in den}{stal stond te vreten}\\

\haiku{Na de verloving.}{hoorden we ineens niets meer}{over publiek worden}\\

\haiku{Als er hoop is, wil,.}{ik wachten zoolang wachten}{als U  verlangt}\\

\haiku{Dan wil ik afstand,:}{doen van allerlei wat ik}{nu aangenaam vind}\\

\haiku{hij was begonnen,.}{te typen en het boek zou}{getypt worden}\\

\haiku{Ik ken de menschen,,.}{ik ken den grond en ik ben}{niet zonder invloed}\\

\haiku{De pastoor is er,}{al de misdienaars kan ik}{elken dag vangen}\\

\haiku{Het is een kwestie.}{van tijd en medewerken}{met de genade}\\

\haiku{deze man heeft een.}{kramersziel en onteert het}{notarieele ambt}\\

\haiku{{\textquoteright} Dien raad volgde de.}{jonge Heer van Jericho}{gedeeltelijk op}\\

\haiku{ze wenden aan hun,.}{steel je ziet ze groen worden}{en weer zilvergrijs}\\

\haiku{En er is bij God.}{toch niets in zijn gedrag dat}{er op zou wijzen}\\

\haiku{{\textquoteright} vroeg Cecily in, {\textquoteleft}.}{verrukte verbazingdan}{had ik toch gelijk}\\

\haiku{{\textquoteright} zei de notaris, {\textquoteleft}.}{heftigen een huiszoeking}{zal het bewijzen}\\

\haiku{{\textquoteright} Hij ging naar de stad.}{en bestelde rolluiken}{voor al zijn vensters}\\

\haiku{Hij neemt de centen,,.}{waar hij ze vindt Van man of}{vrouw van weeuw of kind}\\

\haiku{Kort daarop reed de.}{Heer van Jericho in het}{rijtuigje stadswaarts}\\

\haiku{Ik hoor zoo het een,.}{en ander als ik bij mijn}{cli\"ent\`ele kom}\\

\haiku{Deze kwam na ast:}{den stoel staan en zei met een}{opgewekte stem}\\

\haiku{Sta op, als je een,{\textquoteright}.}{man bent commandeerde de}{Heer van Jericho}\\

\haiku{{\textquoteright} Maar zonder op zijn:}{antwoord te letten ging de}{oude dame voort}\\

\haiku{De groote hoofddeur was,.}{van gebeeldhouwd eikenhout}{met brons beslagen}\\

\haiku{De markiezin is.}{een kwezel en de markies}{was een esprit fort}\\

\haiku{Men had, zoo zeide,.}{men den herder gehoond en}{met vuil geworpen}\\

\haiku{Zooiets is ons nog,{\textquoteright}.}{nooit overkomen zei ze met}{tranen in haar stem}\\

\haiku{Ge laat den Heer in.}{den hof staan en spuit hem schoon}{met de glazenspuit}\\

\haiku{Smoesjes om niet te.}{luisteren naar den pastoor}{worden niet geduld}\\

\haiku{Evenals gij, ben ik;}{sinds geruimen tijd zonder}{berichten van haar}\\

\haiku{Maar nooit vond ik het,,,.}{noodig haar om zoo te zeggen}{te reserveeren}\\

\haiku{Al die effecten.}{en aardsche schatten kunnen}{mij niets meer schelen}\\

\haiku{{\textquoteleft}Ik geloof niet dat,{\textquoteright}.}{dit in dit geval noodig is}{zeide hij halfluid}\\

\haiku{Maar alles kan niet.}{op zijn best gaan in deze}{treurige wereld}\\

\haiku{Hij heeft ons, om het,.}{zoo maar te zeggen uit de}{vuiligheid gehaald}\\

\haiku{Ik bemerkte, dat.}{Duchatel zijn hoeve zonder}{verlies had verkocht}\\

\haiku{Toen, aan het einde,.}{wilde hij niet terugkeeren}{langs hetzelfde pad}\\

\haiku{{\textquoteleft}Er zijn allerlei,.}{redenen waarom men met}{een vrouw wil trouwen}\\

\haiku{Het is tenslotte,.}{een fran\c{c}aise monsieur}{le Baron begrijpt}\\

\haiku{{\textquoteleft}We gaan Monsieur.}{Ernest Lacroix verslaan met}{zijn eigen wapens}\\

\haiku{{\textquoteright} Toen hij van het plan,.}{gehoord had wilde hij in}{elk geval meedoen}\\

\haiku{Als derde was een.}{machtig groot en dik man te}{voorschijn gekomen}\\

\haiku{Het was etenstijd en;}{men vierde in een goede}{herberg het afscheid}\\

\haiku{morgen ontwaakte,.}{de man niet toen ik op zijn}{kamerdeur klopte}\\

\haiku{Ik zette door, want.}{zulk een ontbindend lijk in}{huis stond me niet aan}\\

\haiku{Ik nam dus een pan,,.}{maakte vuur sneed spek en brak}{een paar eieren}\\

\haiku{Reeds voor den middag.}{lag het werk op de hoeven}{om Jericho stil}\\

\haiku{Toen de deur achter.}{hen beiden gesloten was}{zetten zij zich neer}\\

\haiku{{\textquoteright} Met een spring begon.}{Aboe den weg naar Jericho}{af te leggen}\\

\haiku{Dan trekt het beter,{\textquoteright},.}{door zeide Baptiste en}{hij zwaaide de zweep}\\

\haiku{{\textquoteright} zei Cecily, en.}{haar hand zocht de hand van den}{Heer van Jericho}\\

\subsection{Uit: De president}

\haiku{Hij loopt dicht langs de ':}{huizen vant Prinsenvest}{en de Loskade}\\

\haiku{ternauwernood ziet;}{hij de goederen die op}{de kade liggen}\\

\haiku{En tenslotte was.}{hij toch blijven staan om een}{aalmoes te geven}\\

\haiku{De agent kwam terug,:}{aanmerkelijk beleefder}{en eerbiediger}\\

\haiku{hij droeg 'n flambard,{\textquoteright}.}{en een wijde cape zei}{de president vlot}\\

\haiku{Maar dat was niet noodig -:}{de commissaris zei uit}{eigen beweging}\\

\haiku{In dien val streek door:}{hem heen het beeld van een wit}{en zwarten monnik}\\

\haiku{Hij sloeg met z'n hand:}{op de armleuning van den}{stoel en zei hardop}\\

\haiku{En terwijl ik zoo,:}{bedremmeld sta te kijken}{zei de dokter nog}\\

\haiku{Als een heel oude.}{herinnering bezon hij}{zich op het stigma}\\

\haiku{{\textquoteleft}Natuurlijk heb je -}{dat geld van me gekregen}{en dat je hard was}\\

\haiku{{\textquoteleft}Ik weet natuurlijk,.}{niet hoe sterk Uw gevoelens}{waren voor die vrouw}\\

\haiku{En toch wist ze wel,.}{dat het kind niet altijd}{kon blijven zwerven}\\

\haiku{En bezinnend zag.}{hij de vlakke handen van}{den zwerver voor zich}\\

\haiku{Met belangstelling.}{keek de wachtende grijsaard}{naar die voorstelling}\\

\haiku{{\textquoteleft}Een net van vragen,{\textquoteright}:}{en het beeld ontwikkelde}{zich als vanzelve}\\

\haiku{als ik \'e\'en knoop los.}{wil maken moet ik al de}{knoopen ontwarren}\\

\haiku{Ongeveer z\'o\'o als, '.}{Hugo toen was zout kind}{der vrouw nu wel zijn}\\

\haiku{Hij was echter een,;}{te hoffelijk man om dit}{te laten merken}\\

\haiku{{\textquoteright} {\textquoteleft}Ja, Dientje, dat die '.}{een veel beter hand mett}{kind heeft dan zijzelf}\\

\haiku{Ze begrijpt maar niet,{\textquoteright}.}{waar\`om U dit voor haar doet en}{hij aarzelde even}\\

\haiku{Toch moest hij z\'o\'o de.}{taak vervullen die hij op}{zich had genomen}\\

\haiku{Nella had haar hals '.}{n beetje gebogen en}{keek recht voor zich uit}\\

\haiku{{\textquoteright} Toen ze dit gedaan,.}{had bleef de vrouw verlegen}{bij de tafel staan}\\

\haiku{en tegelijk gaf.}{hem dat feit een vaag gevoel}{van onbehagen}\\

\haiku{{\textquoteright} zei de president, {\textquoteleft}.}{glimlachendeen verzetje}{doet iedereen goed}\\

\haiku{{\textquoteleft}Maar hou een volgend,.}{maal Uw trouwring maar aan dat}{voorkomt misverstand}\\

\haiku{{\textquoteleft}Nu kan ik me toch.}{niet meer herinneren waar}{ik gebleven was}\\

\haiku{{\textquoteright} Met moeite hield de:}{president zijn gekrenktheid}{uit stem en gelaat}\\

\haiku{{\textquoteright} De spreker fronste,:}{zijn wenkbrauwen en langzaam}{en moeilijk zei hij}\\

\haiku{{\textquoteleft}Maar U weet toch wel,{\textquoteright}, {\textquoteleft}.}{zei de presidentdat die}{praatjes onjuist zijn}\\

\haiku{eerst tweemaal voor de, '.}{keukenmeid toen eenmaal voor}{t kamermeisje}\\

\haiku{En nu, zooveel jaar,.}{later moest hij trachten dien}{vriend in te halen}\\

\haiku{Telkens als \'e\'en zijn.}{voet op de treeplank zette}{zwiepte de bus over}\\

\haiku{Wilde hij dan 'n?}{martelaar worden voor wat}{hij als zijn plicht zag}\\

\haiku{Als in een koortsdroom:}{gingen de laatste maanden}{aan zijn oog voorbij}\\

\haiku{Ze trokken samen,:}{op een middelpunt zoo ver}{van hem verwijderd}\\

\haiku{het zou een teeken,;}{zijn van hartelijkheid dat}{hij niet verwacht had}\\

\haiku{Het was alsof hij,.}{vreesde naar huis te gaan door}{de stad te loopen}\\

\haiku{Wij, en gij vooral,.}{mogen dit uit den grond van}{ons hart betreuren}\\

\haiku{Maar eveneens waren,,.}{er die gezondheid geld of}{liefde bezwoeren}\\

\haiku{De boer, een norsche,,.}{sombere duitendief had}{zich opgehangen}\\

\haiku{En heel misschien was.}{met een operatie iets te}{bereiken geweest}\\

\haiku{Alles wat ik voor.}{de genezing noodig had kon}{ik mij verschaffen}\\

\haiku{Het was alsof ze.}{voor mijn oogen voetje voor voetje naar}{haar graf schuifelde}\\

\haiku{ik was door het lot,,.}{van Marie gedeclasseerd}{onttroond ontheiligd}\\

\haiku{Ik wilde heel sterk,,.}{heel vast onderzoeken of}{Marie nog leefde}\\

\haiku{Dagen lang lag ik,.}{in bed en ik zweefde in}{een stille wereld}\\

\haiku{Misschien is ze dan -.}{gestorven in wanhoop God}{vervloekende}\\

\haiku{Hij vreest van tijd tot,.}{tijd dat hij de antichrist}{zal blijken te zijn}\\

\haiku{Die onderzocht de {\textquoteleft}{\textquoteright},.}{zaakConchita niet eens maar}{verscheiden malen}\\

\haiku{{\textquoteleft}Antichrist{\textquoteright} en in.}{zijn hart niets dan berouw over}{zijn nieuwsgierigheid}\\

\haiku{Rustig bad pater.}{Colango de completen}{en ging te ruste}\\

\haiku{Tenslotte vroeg hij,;}{den ouden man bij hem te}{blijven overnachten}\\

\haiku{Omdat monseigneur - -.}{drie achtereenvolgens het}{beter z\'o\'o vonden}\\

\haiku{{\textquoteleft}Als de baron tot.}{God terugkeert zijn er geen}{moeilijkheden meer}\\

\haiku{Opeens viel hij op {\textquoteleft}{\textquoteright}.}{de knie\"en en zeiJube}{benedicere}\\

\haiku{En elk dorpeling,,.}{was lid moest lid zijn moest de}{avonden bezoeken}\\

\haiku{Hij was 't die steeds {\textquoteleft}{\textquoteright};}{zijn zaal verhuren moest aan}{Kunst en Wetenschap}\\

\haiku{ze hadden allen,;}{schuld aan het kasteel van pacht}{of retributie}\\

\haiku{tenslotte zei ze.}{den dienst op en een nieuwe}{was niet te krijgen}\\

\haiku{De gebeden van '.}{t begin der Mis liepen}{al als een uurwerk}\\

\haiku{op Pinksterdag was '.}{er g\'e\'en die aans Heeren}{Lijf behoefte had}\\

\haiku{En ze vertelden:}{aan den pastoor dat ze het}{beeld moesten vernielen}\\

\haiku{Degene, die de,:}{bijl gedragen had bleef staan}{en hij zei opeens}\\

\haiku{Maar 't werd bekend,.}{en monseigneur schreef een brief}{om opheldering}\\

\haiku{Twee dagen later:}{kreeg de pastoor een briefje}{van den baron}\\

\haiku{Die eerder de bijl,.}{gedragen had was door den}{bliksem verslagen}\\

\haiku{De rentmeester droeg,, '.}{de bijl \'e\'en klap ent beeld}{zou er geweest zijn}\\

\haiku{Maar op eenmaal zag -.}{hij een licht dat uitging van}{een hoek der kamer}\\

\haiku{Hij stond op - en de ':}{armen uitgestrekt alsn}{bedelaar vroeg hij}\\

\haiku{De drie arbeiders,:}{keken elkaar aan en ze}{zeiden als in koor}\\

\haiku{Op d\'eze plek - op,.}{d\'eze plek wilde hij zijn}{liefde bekennen}\\

\haiku{het,{\textquoteright} zei de oude,.}{man en er waren klare}{tranen in zijn oogen}\\

\section{Rob Nieuwenhuys}

\subsection{Uit: Vergeelde portretten uit een Indisch familiealbum}

\haiku{De avond tevoren:}{had tante Sophie aan}{tafel al gezegd}\\

\haiku{Buiten verkleurde.}{de nacht en langzaam brak een}{nieuwe dag aan}\\

\haiku{Ik 				herkende.}{haar eigenlijk nauwelijks}{zoals ze daar lag}\\

\haiku{Ook het inschuiven.}{in de auto geschiedde}{vlug en geruisloos}\\

\haiku{Hij deed het naar 				.}{inlandse trant en snoof}{meer dan hij kuste}\\

\haiku{het vervolg is het {\textquoteleft}{\textquoteright}.}{verhaal van oom 			   Tjen over}{deooievaarsjacht}\\

\haiku{In naam van de tijd.}{kon er nu gerust over}{gesproken worden}\\

\haiku{gelukkig niet uit,.}{Batavia maar hij zou op}{zichzelf gaan wonen}\\

\haiku{Als een geest of een,?}{spook 			 waarvan ik zo vaak}{had horen spreken}\\

\haiku{Ze was eigenlijk.}{erg vriendelijk voor me en}{lachte telkens}\\

\haiku{toen ineens zoende.}{ze hem en 			 ging daarna}{vlak voor hem zitten}\\

\haiku{Soms waren hele;}{levensgeschiedenissen}{erop afgebeeld}\\

\haiku{Het moet in ieder.}{geval een merkwaardig}{gezicht zijn geweest}\\

\haiku{Een knappe vrouw, met.}{saamgeknepen lippen en}{een brede 			 mond}\\

\haiku{Bepaalde trekken,}{abstraheerde ze gewoon}{ze zette ze apart}\\

\haiku{Ze vroeg en snakte,?}{naar erkenning maar wie kon}{haar 			 die geven}\\

\haiku{Toen hoorde ze haar:}{moeder uit bed glijden}{en langzaam zeggen}\\

\haiku{Het sprak vanzelf dat.}{Midin de volgende}{dag werd uitgehoord}\\

\haiku{er was eigenlijk:}{maar \'e\'en lied waarvan 			 hij}{de wijs goed kende}\\

\haiku{Daar moeten tante.}{Sophie en oom Tjen ook}{hebben gelopen}\\

\haiku{{\textquoteleft}Zeg zeun,{\textquoteright} hoorde ik, {\textquoteleft},?}{hem tegen oom Tjen zeggen}{zeg zeun willen we}\\

\haiku{Tante Sophie,.}{naast hem in haar Japanse}{zijden kimono}\\

\haiku{Enige dagen voor.}{de bevalling viel tante}{Christien in onmacht}\\

\haiku{En zelfs als men het,.}{minimum gewicht neemt}{mocht het kind er zijn}\\

\haiku{Hij 			 glimlachte:}{opgelucht en stemde in}{met een ander plan}\\

\haiku{Maar ze wilde nooit;}{herinnerd worden aan}{sc\`enes als deze}\\

\haiku{In het voorjaar bleek.}{het gewenst dat oom Tjen nog}{\'e\'en winter bleef}\\

\haiku{Ik zie mijzelf het.}{telegram aannemen en}{aftekenen}\\

\haiku{Zelfs de kamer scheen.}{een andere 			 dan een}{ogenblik tevoren}\\

\haiku{het sprak vanzelf 			 .}{dat tante Sophie bij}{ons zou komen}\\

\haiku{Zou het zijn werking,?}{verder doen ook in ons huis}{en 			 ons gezin}\\

\haiku{Het kind was al lang{\textquoteleft}{\textquoteright},.}{overgegaan jaren en}{jaren geleden}\\

\haiku{De heer Treves gaf.}{soms ook 			 precies aan waar}{een kwaal zetelde}\\

\haiku{Rienkie leek deze.}{keer wat teruggetrokken}{en 			 verlegen}\\

\haiku{Ze zong niet alleen,,.}{maar praatte nu ook druk vlug}{en 			 geestdriftig}\\

\haiku{Wat mijzelf betreft,}{er was mij die 			 middag}{veel aan gelegen}\\

\haiku{Welnu, we slaagden.}{er soms in iets daarvan te}{realiseren}\\

\haiku{Rienkie 			 genoot.}{ervan en dat gaf mij een}{grote voldoening}\\

\haiku{Rienkie huiverde.}{toen ze in haar avondmantel}{naar de auto liep}\\

\haiku{{\textquoteleft}Kasian,{\textquoteright} hoorde.}{ik haar op een afstand al}{tegen me zeggen}\\

\haiku{Ze zetten \'e\'en 			 .}{voor \'e\'en hun instrumenten}{achter hun standaard}\\

\haiku{Bij het naar buiten.}{gaan sloeg ik mijn arm losjes}{om Rienkie heen}\\

\haiku{Het spel was alleen.}{te 			 doorzichtig om haar}{de prijs te gunnen}\\

\haiku{Ze was eenvoudig:}{in een lachstuip gevallen}{en had gezegd}\\

\haiku{Tijdens het gesprek}{zei hij dat hij een van zijn}{relaties voor mij}\\

\haiku{Heel 			 anders toch,,?{\textquoteright}:}{dan Kitty deze meisjes}{ja Ze bedoelde}\\

\haiku{Geen wonder dat de.}{oudste het nu voortdurend}{in de buik 			 had}\\

\haiku{De conferentie.}{had plaats in de kamer van}{tante Sophie}\\

\haiku{{\textquoteleft}Ja, wel d\'onker, Lien,.}{maar toch niet z\'o donker}{als deze meisjes}\\

\haiku{Nooit, n\'o\'oit waren die}{kinderen eens lief voor haar}{of deden ze iets}\\

\haiku{In deze fase.}{trad Kitty op en kwam haar}{rechten opeisen}\\

\haiku{je moet er voor een,,.}{tijd uit weg uit dit huis dat}{zal je goed doen}\\

\haiku{vierwielig rijtuig,,:}{genoemd naar de ontwerper}{Deeleman 			   desa}\\

\haiku{zacht gekookte rijst,:}{au bain-marie bereid}{njonja 			 b\u{e}sar}\\

\section{A.H. Nijhoff}

\subsection{Uit: De dagen spreken}

\haiku{Ze beschouwden je.}{als een parasiet die hun}{belasting kostte}\\

\haiku{Maar jij wist weer niet,.}{dat het lot zijn laatste poets}{nog niet gespeeld had}\\

\haiku{Gij hebt den oorlog.}{en zijn dooden en verminkten}{reeds vergeten}\\

\haiku{dat die geuzen de.}{aarde vormen waarop uw}{gouden tempel steunt}\\

\haiku{En daarom vraagt gij.}{met verbazing wat men u}{eigenlijk verwijt}\\

\haiku{Neen, z\`elfs niet als een,...}{hond want een hond verdedigt}{nog zijn meester}\\

\haiku{Velen die nimmer,.}{zonder arbeid waren zijn}{werkeloos geweest}\\

\haiku{Zeker, bezorgde,.}{ouderen de jeugd is min}{of meer verwilderd}\\

\haiku{Gij zijt weggevlucht.}{en velen hebben zelfs uw}{heengaan niet bemerkt}\\

\section{Peter J.A. Nissen}

\subsection{Uit: De akkoorden van het gemoed}

\haiku{Zij stal het hart van,- {\textquoteleft}}{Van Deyssel die aan haar in}{188485 het sonnet}\\

\haiku{Het betoog werd even}{vurig ontvangen als het}{werd uitgesproken}\\

\haiku{Ook over de eerste.}{opvoering doet een verkeerd}{jaartal de ronde}\\

\haiku{In 1883 werd Seipgens.}{benoemd tot leraar aan de}{Rijks H.B.S. te Leiden}\\

\haiku{Desalniettemin}{is het toch wel een aardig}{liefdesgedichtje.143}\\

\haiku{De Goddelyke,:}{kindervriend Door de onschuld}{ingenomen Sprak}\\

\haiku{- Gij eindelijk zet.}{de kerk niet meer beurtelings}{in gloed en tranen}\\

\haiku{O Gods naam, Gods daad,,!}{Gods Wijsheid Gods Liefde zij}{eeuwig gezegend}\\

\haiku{Hij is Heer van den,.}{tijd Hij is Heer en Meester}{van de Eeuwigheid}\\

\haiku{- dat is toch een man,;}{van fatsoen Dient netjes en}{deftig te leven}\\

\haiku{Glimlagchend, ziet zij,,:}{hem dan aan Verheugt zich om}{zijn spijt En zegt dan}\\

\haiku{Een ieder wacht van,.}{d'aanval slechts Het nog te}{geven teeken}\\

\haiku{Zelfs veegt men zich een ',.}{traan uitt oog En drinkt en}{klinkt nu weder}\\

\haiku{Emile Seipgens   , '}{Onrust  Wolken met uw}{snelle vaartk Heb}\\

\haiku{Ik kwam - doch gij werdt,....}{daadlijk bang En gij begont}{bijna te weenen}\\

\haiku{Het heel publiek, det '....}{knipt zich uigskes Mer opt}{inj applaudisseert}\\

\haiku{Men ging dan ins nao,.}{Hinsberg h\`er Dao gaof men}{ei fameus concert}\\

\haiku{Ein kalfsborst mit ein {\textquoteleft}!}{voos of zesEn meuglik oug}{daonao ein fles}\\

\haiku{Hai is van leefde,;}{half verschmacht Hai dinkt aan}{trouwe daag en nacht}\\

\haiku{Wie praoper zit,.}{dai stevelet Waat is det}{veutje klein en net}\\

\haiku{wie ein beugelp\'ort{\textquoteright} is.}{een Roermondse uitdrukking}{voor o-benen}\\

\haiku{Is eine rieke,.}{get verkaajd Dai leet mer nao}{den dokter schikke}\\

\haiku{Mer velt van oos luuj,.}{eine drin Dai mot mer zelf}{kartoesche biete}\\

\haiku{Den erme miens, de,.}{slumste droet Dai kan dich kluut}{en klompe make}\\

\haiku{De m\"oggen dansden,.}{om os haer De krekel had}{zich heis gezongen}\\

\haiku{De zon zonk ne\^er in.}{volle glans Wie in ein zee}{van golje water}\\

\haiku{{\textquoteleft}Had ich toch eer aan!}{de oer228 van m{\^\i}nen dood}{as aan uch229 gedacht}\\

\haiku{In mijn verbeelding,!}{was ze reeds een arm meisje}{dat ik redden moest}\\

\haiku{Ik gevoelde mij,.}{als gered als ontsnapt aan}{een dreigend gevaar}\\

\haiku{Renilde nam den.}{arm van Jeanne en trad}{met haar naar binnen}\\

\haiku{Des anderen daags .........................................................}{vertrok ik en heb Maasloo}{nooit weergezien}\\

\haiku{{\textquoteright} 't Was mij of een,.}{berg verplaatst werd die op mijn}{hart had gelegen}\\

\haiku{Paschen is de;}{jubel over de opstanding}{na de lijdensweek}\\

\haiku{Hoe jammer dat gij,,,!}{mijn trouwe vriend mijn leidsman}{afwezig moest zijn}\\

\haiku{Nog een korten draai -;}{en plotseling stond ik aan}{den zoom van het bosch}\\

\haiku{gestorven is en.}{hem slechts een eenig dochtertje}{heeft nagelaten}\\

\haiku{En daar ligt nu de!}{lieve vrede van mijn lief}{dorpje verbroken}\\

\haiku{Op een avond, dat ik,.}{mij door de dorpsstraat spoedde}{trad hij op mij toe}\\

\haiku{Omdat wij twee geen,.}{vrouw geen familie en geen}{kinderen hebben}\\

\haiku{Soms  was het mij,......}{onmogelijk te gelooven}{wat ik gezien had}\\

\haiku{Non vastator,Niet als,;}{dwingland~Sed salvator}{rukt hij nader}\\

\haiku{Elk vreemde scheepskiel,,!}{strijkt de vlag O Vaderland}{voor Uw gezag}\\

\haiku{L\`a, plus de trouble,,;}{plus de peine La paix y}{r\`egne pour toujours}\\

\haiku{Ig zel opstoan,:}{en noa mie vader goan en}{ig zel hem z\`egge}\\

\haiku{Nu staat Jantje nog,:}{daar met zeep en met zout Denkt}{steeds bij zichzelve}\\

\haiku{Als Sasse bij ons,,}{het graag aangenaam heeft Wel}{drommels dat hij zoo}\\

\haiku{Daartusschen staken {\textquoteleft}{\textquoteright} {\textquoteleft}{\textquoteright}.}{de velden metweit en met}{haver donker af}\\

\haiku{iedereen kende {\textquoteleft}{\textquoteright} {\textquoteleft}{\textquoteright}.}{hem in Wiel en devr\`emde}{gingen hemniks aan}\\

\haiku{Hier weigerden zijn,,.}{beenen den dienst en besloot hij}{wat rust te gunnen}\\

\haiku{De kathedraal van.}{Roermond v\'o\'or en na de brand}{van 20 mei 1892}\\

\haiku{De kathedraal van.}{Roermond v\'o\'or en na de brand}{van 20 mei 1892}\\

\haiku{Zij waren goed en,!....}{lief de bewoners van Boschdorp}{in mijn kindertijd}\\

\haiku{of wel een ander,.}{van de talrijke liedjes}{toen te Boschdorp in zwang}\\

\haiku{want toen werd ik, door,.}{een derde zusje voorgoed}{er uit verdrongen}\\

\haiku{Eindelijk meen ik,.}{dit gevonden te hebben}{en slaap rustig in}\\

\haiku{Veur zollen dig in,.}{det dupke sjt\`eken En dig}{lever zelver \`ete}\\

\haiku{Intusschen zitten {\textquoteleft}{\textquoteright}:}{moeder en ik alleen in}{onzesjtoofkamer}\\

\haiku{- En, omde's te zoo,,.....}{braaf bus kr{\^\i}gs te oug die sjoon}{knuip allemoal}\\

\haiku{Terstond word ik door;}{de spelende kornuiten}{op straat opgemerkt}\\

\haiku{Om het zeerst word,,,.....}{ik bekeken bewonderd}{en wellicht benijd}\\

\haiku{Bij den wrevel, die, ',;}{bijt zwoegen Somtijds uit}{het binnenst welt}\\

\haiku{En dan, n\'ogeens, het,,,,.}{k\'on niet neen heusch het k\'on}{niet dat van Marie}\\

\haiku{Hij kon t\'och nergens.}{aan dat vuil ontkomen dat}{over zijn ziel hing}\\

\haiku{Maar daar floot het van:}{de locomotief en de}{conducteurs riepen}\\

\haiku{Hier vlak bij, als groote,,.}{jongen daar in Roermond had}{hij vroeger geleefd}\\

\haiku{De h\^otelhouder,.}{kende hem niet meer en hield}{hem voor een vreemde}\\

\haiku{Er zouden nu toch.}{wel geen wandelaars zijn op}{de Kapellerlaan}\\

\haiku{Christus daaraan te,.}{sterven zwaar-bloedend uit}{gruwbare wonden}\\

\haiku{Kom, hij zou zich nog,...}{maar eens omdraaien dan zou}{de slaap wel komen}\\

\haiku{Het was hem of er,.}{iets in hem zou barsten en}{zijn hoofd duizelde}\\

\haiku{Negen hoofdstukken,,-.}{over Roermonds geschiedenis}{Roermond 1985 177200}\\

\haiku{Gerhard van Wessem (), () ().}{1203 De laatste Noorman891}{en Jan van Weert1635}\\

\haiku{44Exemplaar van de,.}{eerste druk in SBM van de}{tweede druk in UBN}\\

\haiku{47Een echte goede.}{biografie van Cuypers}{is er nog steeds niet}\\

\haiku{Eerste ontmoeting -{\textquoteright}, (),,-.}{Uit brieven De Beiaard 5}{1920 deel II 516}\\

\haiku{De katholieke{\textquoteright}.}{architectuurtheorie}{van Alberdingk Thijm}\\

\haiku{Bijlage tot de-,,-.}{Handelingen van 19091910}{Leiden 1910 1933}\\

\haiku{Bijlage tot de-,,-.}{Handelingen van 18961897}{Leiden 1897 120}\\

\haiku{Borel heet er Paul.}{Waerens en Sophie heet}{Corrie van Meeden}\\

\haiku{Jacob Lodewijk () (-),.}{Louis Herten18461920 koopman}{en schoolopziener}\\

\haiku{321E\'en in een zet, met.}{kracht door den ring geworpen}{bal telt zes punten}\\

\section{Nel Noordzij}

\subsection{Uit: Het kan me niet schelen}

\haiku{het was een rompje.}{met een hoofdje en op dat}{hoofdje een hoedje}\\

\haiku{ik moet er niet aan,.}{denken maar ze zou hem zelfs}{in zijn jas helpen}\\

\haiku{Natuurlijk hoor je,.}{bij ons ik zie geen verschil}{tussen jou en mij}\\

\haiku{Jenny stond het niet,.}{te denken ze fluisterde}{het de kamer in}\\

\haiku{Met spitse lippen.}{slurpte ze de gloeiende}{koffie naar binnen}\\

\haiku{{\textquoteleft}Er gebeurde iets.}{op de mannenzaal waar ik}{niets mee nodig had}\\

\haiku{Naakt was ze alleen,.}{een karkasje geschikt voor}{de vuilnisemmer}\\

\haiku{Als Vincent in de,.}{eetkamer bij de radio}{zit hoort hij me niet}\\

\haiku{Meneer had het al.}{gezien en de bende de}{bende gelaten}\\

\haiku{Die schrijverij is.}{een zoeken en tasten naar}{klaarheid met zichzelf}\\

\haiku{Als ik de kalk heb,.}{weggeveegd ga ik naar de}{Emmalaan dacht ze}\\

\haiku{van je,{\textquoteright} Volgens haar,.}{hoorde het er bij dat je}{je meisje zoende}\\

\haiku{Ze duwde hem weg,}{maar keek nieuwsgierig naar de}{foto's in zijn hand.}\\

\haiku{Had Vincent de doos?}{bonbons ook teruggelegd}{in het nachtkastje}\\

\haiku{{\textquoteright} De vrouw maakte een.}{stotend geluid met haar keel}{en blies voor zich uit}\\

\haiku{Ze hield haar ogen op,.}{dezelfde hoogte ook toen}{de tram weer doorreed}\\

\haiku{We schreven elkaar,.}{briefjes want we zaten niet}{in dezelfde klas}\\

\haiku{De blinddoek, die ze,.}{voor hadden was een voile}{maar een keiharde}\\

\haiku{Enfin, maak je niet,.}{ongerust ik heb geleerd}{me te beheersen}\\

\haiku{Misschien leek het er,,.}{op maar ik was vijftien dat}{maakt een groot verschil}\\

\haiku{Ren\'ee zal niet met.}{die vriendin naar bed willen}{vandaag aan de dag}\\

\haiku{Dus eigenlijk houdt,.}{ze alleen van zichzelf daar}{komt alles op neer}\\

\haiku{ik hou van vrouwen,?}{en meer niet of pluis je de}{oorzaak daarvan uit}\\

\haiku{Ze legde haar hoofd.}{voorover op de tafel en}{begon te brullen}\\

\haiku{In het bed in de.}{uiterste hoek van de zaal}{lag een dikke man}\\

\haiku{Zingende zagen,,.}{cirkels alles zit in dat}{hoofd en dan val je}\\

\haiku{Ik zou hem moeten,,:}{zoenen zijn dikke bange}{kop tegen me aan}\\

\haiku{{\textquoteleft}Waarom weigerde,?}{U mij gisteren Uw adres}{niet toen ik het vroeg}\\

\haiku{{\textquoteright} Jenny vertelde,.}{dat hij altijd dronken is}{ik merk er niets van}\\

\haiku{Misschien gaat Meerboom,,,.}{even kijken nee hoe kan dat}{nou die is bezig}\\

\haiku{Niet huilen, dit is,,.}{waanzin dacht ze ik heb er}{geen  reden voor}\\

\haiku{hij trok het mes er.}{uit en legde het naast het}{zijne op zijn bord}\\

\haiku{Ik deed net alsof,.}{vond het een mooie manier om}{het af te leren}\\

\haiku{(En Uw vader en),.}{Uw moeder straal flauw ben ik}{ik wou het zelf ook}\\

\haiku{Maar mam, hij verveelt,,,,.}{me zo eerst mild dan bars dan}{weer mild dan weer bars}\\

\haiku{Je moet eens een avond,.}{bij me komen ik zie je}{zelden bij mij thuis}\\

\haiku{Anderen staken.}{hun hand omhoog en gingen}{op hun tenen staan}\\

\haiku{Met haar elleboog.}{schoof ze de vrouw opzij en}{maakte de deur open}\\

\haiku{Ze boog haar gezicht.}{naar voren en deed haar mond}{wijd open voor de spijl}\\

\haiku{Ren\'ee bestelde.}{een half flesje witte wijn}{en begon te eten}\\

\haiku{{\textquoteleft}Zal ik je iets van.}{vertellen en laat het je}{een lesje wezen}\\

\haiku{{\textquoteright} Och kom, dacht Jenny,.}{in mijn eigen kamer mag}{ik niet meer praten}\\

\haiku{Het komt er niet op,.}{aan wat en voor wie je voelt}{als je maar \`echt voelt}\\

\haiku{schone lakens, een.}{grote asbak en nergens}{stof of viezigheid}\\

\haiku{{\textquoteleft}Heb jij {\textquotedblleft}Les jeux sont{\textquotedblright},?}{faits van Sartre gelezen}{of de film gezien}\\

\haiku{Ze keerde zich half.}{van Ren\'ee en Lucas af}{en keek de zaal in}\\

\haiku{Ze hoorde stemmen.}{uit de muren van lege}{lokalen bruisen}\\

\haiku{Alles wat je zegt,,.}{fraai en meestal minder}{fraai berust daarop}\\

\haiku{Zij bewogen hun.}{handen zenuwachtig langs}{elkanders lichaam}\\

\section{Cees Nooteboom}

\subsection{Uit: Een middag in Bruay}

\haiku{Aan de overkant een,.}{bloemenwinkel een Maison}{de Confiance}\\

\haiku{De andere stem,,.}{die achter in het toestel}{kabbelt rustig door}\\

\haiku{Tegen de paal van {\textquoteleft}}{een stoplicht zie ik een geel}{pamflet waarop staat}\\

\haiku{Lege, open ruiten.}{waaruit belachelijke}{gordijnen zwaaien}\\

\haiku{O liefelijke,,!}{Natie o poel van rust o}{ongestoord eiland}\\

\haiku{Hoe lang zal het nog?.}{duren eer er werkelijk}{zo gedacht wordt p.s}\\

\haiku{Definitiever.}{ziet de laatste foto van}{de serie er uit}\\

\haiku{Hij heeft een vlag bij,,!}{zich de moordenaar en wat}{draagt hij een gek pak}\\

\haiku{Eigenlijk is dat,,.}{de reden geweest voor mij}{om dit te schrijven}\\

\haiku{Verzoend met Duitsland,,.}{Algerije geregeld de}{ziekte genezen}\\

\haiku{Een kind loopt voorbij.}{met een oranje muts op en}{blaast op een toeter}\\

\haiku{Ik was elf jaar toen,.}{ik werd bevrijd en ben dus}{niet zo erg bevrijd}\\

\haiku{Dat is vlug werk, dacht -.}{hij en wachtte even met de}{hoorn op te nemen}\\

\haiku{{\textquoteright} Buylers noemde zijn {\textquoteleft}.}{bedrag en voegde erbij}{zware roebels graag}\\

\haiku{Er sterven nu in.}{onafhankelijk Kongo}{mensen van honger}\\

\haiku{Hij zegt dat hij hem,.}{nu niet af kan doen maar dat}{ik hem morgen krijg}\\

\haiku{Het zonlicht heeft de.}{halve cirkel rondgemaakt}{en is verdwenen}\\

\haiku{Het twaalfde uur van.}{de dag glijdt voorbij zonder}{dat er iets gebeurt}\\

\haiku{Hij is heel klein, en,.}{kijkt verschrikt naar het grote}{dier de menigte}\\

\haiku{Wie het nu niet weet,.}{zal het nooit weten niet door}{woorden tenminste}\\

\haiku{Een van de mannen.}{heft zijn arm hoog op en laat}{die auto stoppen}\\

\haiku{Het wordt allemaal,.}{opgeschreven en ik moet}{ondertekenen}\\

\haiku{Er staan metalen,.}{stoeltjes maar het is te koud}{om te gaan zitten}\\

\haiku{Een Duitse middag [].}{17 januari 1963 Drie}{uur in de middag}\\

\haiku{Het is ook moeilijk.}{om aan de indruk van die}{stem te ontkomen}\\

\haiku{menigte omdat}{ik aanwezig ben en help}{een hal te vullen}\\

\subsection{Uit: Een nacht in Tunesi\"e}

\haiku{Na dat halve uur}{komt er een mevrouw die zegt}{dat meneer Sadek}\\

\haiku{We blijven daar een,.}{tijdje staan hierna zullen}{we niet verder gaan}\\

\haiku{een meneer in het.}{Hollands Maandblad beweert dat}{ik gelogen heb}\\

\haiku{{\textquoteleft}Ik ben formateur,{\textquoteright}}{had hij toen geantwoord en}{zijn blik verlegen}\\

\haiku{Zo werd hij in drie {\textquoteleft}{\textquoteright}.}{maanden een negatieve}{held van deze tijd}\\

\haiku{Hij weigert naar de.}{microfoon te gaan en spreekt}{vanuit zijn koorstoel}\\

\haiku{Verderop drijven.}{ze voorbij en ik zie dat}{het water snel stroomt}\\

\haiku{{\textquoteleft}We kunnen wel even,.}{naar oma gaan daar zijn we in}{geen tien jaar geweest}\\

\haiku{Hij gaat naar Lagos, en,?}{waar ga ik naar toe en of}{ik iets wil drinken}\\

\haiku{Het is koud, voor het.}{eerst dit jaar voel ik dat het}{winter zal worden}\\

\haiku{dat er op neerkomt.}{dat die werf nog wel goed is}{voor een verkiezing}\\

\haiku{Drieduizend mensen,...}{plus zesduizend die van hen}{afhankelijk zijn}\\

\haiku{Tussen hen in staat.}{een glas cognac waar ze om}{de beurt van drinken}\\

\haiku{Maar dat kan ook een.}{kwestie zijn van simpelweg}{de prijs opdrijven}\\

\haiku{Maar niemand heeft het,.}{ooit kunnen zien en ik ga}{naar het monument}\\

\subsection{Uit: Een ochtend in Bahia}

\haiku{Terug was alles,,.}{hetzelfde dat wil zeggen}{even anders als heen}\\

\haiku{Ik weet dat het mooi,,.}{is dat het bestaat en dat}{ik er niet bij hoor}\\

\haiku{Het licht is al wit.}{genoeg om geen detail te}{laten ontsnappen}\\

\haiku{Het eerste wat ik.}{zie is de zwarte modder}{onder de huizen}\\

\haiku{De stemming is er,.}{onverdeeld vrolijk en deelt}{zich zelfs aan mij mee}\\

\haiku{Dan zullen we ons.}{er auf eigene Faust mee}{moeten uitrusten}\\

\haiku{Aan de overkant stond,.}{een heel anders geaarde}{menigte en keek}\\

\haiku{Ik laat mijn perskaart,.}{zien hij kijkt er minachtend}{naar en duwt me weg}\\

\haiku{Op de straten is.}{nog duidelijk te zien waar}{het gebrand heeft}\\

\haiku{Het asfalt is niet,.}{alleen nat het is ook erg}{onafhankelijk}\\

\haiku{Door er niet meer te,.}{zijn bestaat hij nog als de}{vraag naar wie hij was}\\

\haiku{Het wordt al donker, '.}{en buiten is hett uur}{van de paseo}\\

\haiku{De jeugd van de stad,.}{flaneert over de straten steeds}{dezelfde route}\\

\haiku{De vreemdeling loopt,.}{er tussendoor en wordt als}{vreemdeling herkend}\\

\haiku{Maar het ergste, in,,.}{die hitte zijn de zware}{leren beenstukken}\\

\haiku{Vijf keer ben ik in,.}{Portugal geweest en ik}{heb het nooit gehaald}\\

\haiku{Langs de weg karren -.}{met wielen zonder spaken}{dichte stukken hout}\\

\haiku{In een klein stadje.}{in het zuidoosten krijg ik}{pech aan mijn auto}\\

\haiku{Hoe ver is deze,,!}{kerk van laten we zeggen}{de Nederlandse}\\

\haiku{Londen heeft vandaag,.}{een zeer grijze hoed op en}{ik loop er onder}\\

\haiku{En aan het eind van,.}{Brighton de boulevard en}{daaronder de zee}\\

\haiku{In Beaune, in de,.}{Bourgogne was er geen plaats}{meer in de herberg}\\

\haiku{Een voorwereldse.}{portier achter een hoge}{bruine lessenaar}\\

\haiku{Zo ontstond, niet zo,.}{ver van Madrid de vallei}{der gevallenen}\\

\haiku{Als ik binnenkom.}{voel ik een door machines}{gemaakte koelte}\\

\haiku{Koud is het, dat is,.}{het enige terwijl buiten}{de zon verder brandt}\\

\haiku{hij vocht was beter,.}{geweest maar de farao wou}{een pyramide}\\

\subsection{Uit: De Parijse beroerte}

\haiku{Cees Nooteboom}{De Parijse beroerte}{Colofon}\\

\haiku{Herakleitos    ,!}{Leve Heraclites weg}{met Parmenides}\\

\haiku{Een enkeling spreekt.}{over een rechtse staatsgreep maar}{wordt weggelachen}\\

\haiku{Gisteren schreef ik {\textquoteleft}{\textquoteright},.}{dat hij er al niet meer was}{dat was gisteren}\\

\haiku{In een gaanderij.}{van de Sorbonne hangt een}{recept voor bommen}\\

\haiku{De ontkenning van...}{alle verlangens van de}{arbeidersklasse}\\

\haiku{Daar is het heet, ik,.}{rijd er in en er uit op}{weg naar het westen}\\

\haiku{In de dorpen zijn,}{de raadhuizen open door de}{open deuren zie ik}\\

\haiku{En wat zou je er,?}{ook mee moeten hier tussen}{al die weilanden}\\

\haiku{ga heeft Mitterrand.}{zijn treurige liedje al}{tien keer gezongen}\\

\chapter[7 auteurs, 612 haiku's]{zeven auteurs, zeshonderdtwaalf haiku's}

\section{Karel van den Oever}

\subsection{Uit: Het inwendig leven van Paul}

\haiku{Het begin dezer.}{werkdadige liefde had}{voor Paul geen einde}\\

\haiku{de lucht Delftsch-blauw,.}{Beider soort  zichtbaarheid}{was bijna identiek}\\

\haiku{Vo\'or dat Paul de kerk,.}{intrad had hij reeds den smaak}{harer heiligheid}\\

\haiku{de dag nadien las {\textquoteleft}{\textquoteright};}{hijDe ware Wijnstok van}{Bonaventura}\\

\haiku{{\textquoteleft}Und eine neue Welt{\textquoteright}:}{entspring auf Gottes Wort zag}{Paul de kleine Haydn}\\

\haiku{Paul overdacht hoe het}{aardsch leven \'een ren was naar}{God en hoe ijdel}\\

\haiku{bloedkegels hingen;}{stijf uit de doornen-kroon}{langs zijn aangezicht}\\

\haiku{De diepte hief haar{\textquoteright}.}{hand op herinnerde zich}{Paul een psalm-vers}\\

\haiku{als voorheen zag hij;}{een boter-gele wolk}{rooken voor de zon}\\

\haiku{De heide lag er,.}{gedroomd en onwezenlijk}{bijna stoffeloos}\\

\haiku{de {\textquoteleft}Ark des Verbonds{\textquoteright}, {\textquoteleft}{\textquoteright}, {\textquoteleft}{\textquoteright}:}{deGeestelijke Roos de}{Toren van David}\\

\haiku{Zijn eucharistisch}{verlangen was zoo fel dat}{hij bereidwillig}\\

\haiku{hij vreesde ook de {\textquoteleft}{\textquoteright}.}{schaduwen der buien die}{verkoeling brachten}\\

\haiku{een kleine inzet,.}{op het eeuwig Leven een}{hervormde voor-spijs}\\

\haiku{Paul, die nu schuilde,:}{onder een portiek zegde}{luid-op den psalmtekst}\\

\haiku{De menschen moesten het,,.}{Vagevuur begeeren meende}{Paul en niet vreezen}\\

\subsection{Uit: Kempische vertelsels}

\haiku{me dunkt zelfs dat ik}{heur goudklaar taaiken tot hier}{in mijn ooren hoor}\\

\haiku{mutseken sloeg er.}{eentonig op en neerewaarts}{in een zot bedrijf}\\

\haiku{{\textquoteright} en Peere sufte, den,...}{denkenden kop schuddend naar}{zijnen leegen stoel weer}\\

\haiku{Daar was een vrouwken ';}{dat spon Giele giele gon}{zoo hard datt kon}\\

\haiku{{\textquoteleft}Eh, vrouwken, schei toch!}{eens uit en laat astemblieft}{toch dat licht branden}\\

\haiku{{\textquoteright} zei hij eindelijk {\textquoteleft}',... '}{t zijn altijd wel zware}{patatten vrouwken}\\

\haiku{toen kwam de morgen...}{op in den Oosten met een}{stille klaarte}\\

\haiku{dempigen mist die.}{alle geluid en doening}{heimelik verdook}\\

\haiku{blauwen kiel die dan.}{omhoog en neersloeg lijk een}{klepperende vlag}\\

\section{Frans van Oldenburg Ermke}

\subsection{Uit: Limburg aan de galg. De legende van de Bokkerijders en de geschiedenis van hun lot}

\haiku{Om hem werden de.}{kraaien en raven niet in}{hun maaltijd gestoord}\\

\haiku{Doch wie arm is, heeft.}{de plicht het te blijven tot}{de dood erop volgt}\\

\haiku{En de koning is,.}{steeds een vr\'e\'emde koning de}{vrouw een vr\'e\'emde vrouw}\\

\haiku{Waarom stond je ook,?!}{niet op toen de hond bij de}{buren zo blafte}\\

\haiku{Een kermis is er,...}{niets bij en de marskramers}{doen goede zaken}\\

\haiku{Doch spoedig daarna.}{al was er weer en b\'eter}{werk aan de winkel}\\

\haiku{En zij die de wacht,.}{hielden wachtten gespannen}{op het eerste alarm}\\

\haiku{Met een stevig stuk -!}{hout had zekere Andries}{je moest hem kennen}\\

\haiku{naar Arret L\"utgens -}{in Bank bij Kohlscheid opbrengst}{drie rijksdaalders}\\

\haiku{Wie het in slechte,.}{tijden niet slecht gaat moet op}{alles verdacht zijn}\\

\haiku{En er is geen dief,.}{zo groot of hij acht de drost}{een n\`og grotere}\\

\haiku{Want je kunt nooit z\'o,!}{arm zijn of door zo'n oorlog}{word je nog armer}\\

\haiku{Maar er is altijd:}{het zekere weten dat}{eens het einde komt}\\

\haiku{Het was zijn eerste,.}{glaasje maar zijn laatste zou}{het nog lang niet zijn}\\

\haiku{\'o\'ok afgetrokken,.}{bekeerd en beschaamd en als}{betere mensen}\\

\haiku{Hij zwoegt op akkers.}{waar het graan wast dat voor hem}{niet wordt gemalen}\\

\haiku{in het zwert neffens;}{hem eenen tweeden met eenen boek}{in zijne handen}\\

\haiku{een list sedertdien.}{door menighe filou met}{goet gevolgh herhaeld}\\

\haiku{Die werd gepakt, op,.}{23 januari 1751 te}{Landsraad-Gulpen}\\

\subsection{Uit: Pastoor Pius Paerel. Een geschiedenis zonder moraal}

\haiku{Oud-pastoor.}{Hendriks is altijd een goed}{predikant geweest}\\

\haiku{Ik droeg den koster.}{op om een briefje bij je}{in de bus te doen}\\

\haiku{En aan het hart van.}{het volk zelf legde hij zijn}{oor te luisteren}\\

\haiku{Adriaan van Harte:}{glimlachte en antwoordde}{alleen maar met een}\\

\haiku{De gastheer lachte.}{dreunend en stak een vinger}{dreigend naar hem op}\\

\haiku{Zij scheen het drinken.}{van koffie zoo'n groote zonde}{nog niet te vinden}\\

\haiku{- Het doet den mensch goed,,! - {\textquoteleft},!}{te hooren dat de mensch slecht}{isO die menschen}\\

\haiku{{\textquoteright} vroeg Joachim nu, door.}{zooveel vroolijkheid aan het}{twijfelen gebracht}\\

\haiku{Het was inderdaad:}{een oogenblik geweest om}{nooit te vergeten}\\

\haiku{{\textquoteleft}Nog \'e\'en maand les bij!}{z\'o\'o'n man en het ventje is}{voorgoed bedorven}\\

\haiku{{\textquoteleft}Toen ik nog zong in....}{het Cecilia-koor}{te Rammelrade}\\

\haiku{Jongens, wat er ook,...!}{gebeure we blijven bij}{elkaar en vr{\`\i}enden}\\

\haiku{Maar zijn verloofde.}{was  inderd\'a\'ad van z\'e\'er}{goede familie}\\

\haiku{Voor een retourtje!}{van zeven gulden vijftig}{heb ik mijn kerk vol}\\

\haiku{{\textquoteleft}Dat zou niet gek zijn,{\textquoteright}, {\textquoteleft}.}{antwoordde Joachimal was}{het ook de \'e\'erste keer}\\

\haiku{Hij droomde van steenen.}{engelen en een hemel}{van verschoten zij}\\

\haiku{{\textquoteleft}'t Is wel geen  ,{\textquoteright}, {\textquoteleft}.}{Afrika lachte Joachimmaar}{toch bijna Parijs}\\

\haiku{Maar boven zijn hoofd.}{dreunde het geronk van den}{slapenden koster}\\

\haiku{Ze hadden nachtdienst,.}{gehad maar wilden deze}{kans niet verzuimen}\\

\haiku{En  zelfs van de.}{bruikbare kon men niets met}{zekerheid zeggen}\\

\haiku{{\textquoteright} Het was duidelijk,.}{dat hij dit alles niet voor}{het eerst vertelde}\\

\haiku{{\textquoteleft}Het lijkt meer op een,{\textquoteright}.}{luchtschip met een strijkje aan}{boord grapte Joachim}\\

\haiku{{\textquoteright} vloekte Adriaan, die.}{over Goedeltje's dekschild was}{komen struikelen}\\

\haiku{Het werd zomer en.}{op het oksaal hoopten stof}{en modder zich op}\\

\haiku{Hij laat den heeren.}{verzoeken toch vooral niet}{op hem te wachten}\\

\haiku{Maar gelukkig riep {\textquoteleft}!}{Pastoor Pius Paerel}{zelf zoo hardHoera}\\

\section{Alfons Olterdissen}

\subsection{Uit: Prozawerken in Maastrichtsch dialect}

\haiku{{\textquoteleft}Van stad en lui veur{\textquoteleft}.}{50 jaor I. Algemeen}{voorkomen der stad}\\

\haiku{vestingm\"or, wie mien {\textquoteleft}{\textquoteright}.}{grameer in h\"a\"orecrinoline}{m\`et nege reipe}\\

\haiku{Es te slachter de,}{boer op trok um bieste te}{koupe dan k\'oste z'em}\\

\haiku{En de waor st\'ont.}{minder in de vitrien en}{mie in de winkel}\\

\haiku{Me k\'os ze kriege.}{van de Vuelta Abajo}{aon tien sent te tien}\\

\haiku{Alloh, in eeder.}{geval heele ze leve}{in de brouwerij}\\

\haiku{Iers kraoge ze vrij}{en k\'oste ze speule op te}{groete speulplaots}\\

\haiku{Kwaom heer em, neet hoole,}{dan marsjeerde de v\`elder m\`et}{ze slachoffer aon}\\

\haiku{Dan bleek wel, of te.}{kuipster kontent k\'os zien of}{tatse rouwkoup had}\\

\haiku{Euveral zit er.}{m\`et z'n doeme-n-aon}{en heer k\"op toch niks}\\

\haiku{Pa Bonk waos ene,.}{struise kerel mer zoe lui}{es er groet waos}\\

\haiku{te bloete hakke,.}{wie twie oliekeuk bove z'n}{klompe oetkomme}\\

\haiku{de zwingel en m\`et.}{twie kamprajer woort te boel in}{beweging gebrach}\\

\haiku{Ze pombde diech neet,.}{allein de wiesheid in ze}{slooge ze debinne}\\

\haiku{ze m\`et ze kuijke.}{aon enen andere spieker}{gehange h\"obbe}\\

\haiku{Lam gen\'og, daste toch.}{nog tao rekene m\'os en}{zinne ontlede}\\

\haiku{als een ei 3.5 cent,.}{kost hoeveel verdient dan wel}{die kip in een jaar}\\

\haiku{zoe vas opein, tot.}{zene mond achter de punt}{van z'n neus zaot}\\

\haiku{m\`et Sinterklaos ',.}{n humme ene steve en}{ene peperkooke maan}\\

\haiku{Wel klobde ziech te,}{leef kinder nao zien doed mie}{onderein met m\`et}\\

\haiku{{\textquoteright} M\`et t'n {\textquoteleft}timbre{\textquoteright} van:}{enen twieden tenor liet ze}{noe vollege}\\

\haiku{{\textquoteright}, doog Graadsje, dee noe,.}{vas euvertuig waor dat}{niemes em zaog}\\

\haiku{e sjeep meister woort.}{en tot et tege d'n ierste}{pileer aon Wiek stoot}\\

\haiku{Van d'n Eker en de.}{Zooj is allewijl in de}{stad niks mie te zien}\\

\haiku{totse heij zeker.}{van e raozetig sjaop}{hadde gevrete}\\

\haiku{Is et einmaol,;}{geslach dan is et aal}{good wat t'raon is}\\

\haiku{te goon wandele.}{of nao de stad gesjik um}{kommissies te doen}\\

\haiku{ze achteroet en,.}{es ze nate veuj kriege}{kriege ze de pups}\\

\haiku{Veur de varjaassie,.}{zeet heer dan ouch wel ins tot}{et enen olifant is}\\

\haiku{{\textquoteleft}dao moot iech ouch get{\textquoteright},:}{van h\"obbe mer wie de vrouw}{hiel kordaot zag}\\

\haiku{iefer aon de gaank, '.}{um toch op te ierste balsn}{gooj figuur te sloon}\\

\haiku{Op v\"a\"ol plaotse {\textquoteleft}{\textquoteright},}{waor dansmeziek ofspeul wie}{ze zagte en m\`et}\\

\haiku{Ouch patattefrits,.}{lever en peerdsvleis woorte}{gerikkemandeerd}\\

\haiku{V\"a\"ol lui g\'onge ziech}{sm\"orreges e kruiske hoole}{en et leve g\'ong}\\

\haiku{{\textquoteright} Noe woort et ouch tied.}{veur Greteke en ze kwaom}{sjouw aongesloeperd}\\

\haiku{- {\textquoteleft}Nein, iech mein, h\"obder{\textquoteright},.}{ze al verkoch klonk et weer}{van et z\"olderke}\\

\haiku{Binne kwaome de.}{drei ongel\"oksveugel m\`et en}{m\`et tot bedare}\\

\haiku{H\"a\"ore maan zaoliger}{waor enen ierste fitseleer}{gewees en dee had}\\

\haiku{d'n ingaank van et {\textquoteleft}}{str\"a\"otsje ene paol en dao}{leunde noe Zjann\`et}\\

\haiku{Koelek had er evels,}{zene mond opegedoon of}{Janneske st\'ont veur}\\

\haiku{H\"a\"ostig bont heer de.}{lanteerie aon de kaord}{en leet ze zakke}\\

\haiku{ins in zenen droum,,.}{want es heer sleep k\'os er de}{r\"os nog neet vinde}\\

\haiku{Et {\textquoteleft}meitske{\textquoteright} keek h\"a\"or}{onger\"os euver h\"a\"ore br\`el}{aon en zeker neet}\\

\haiku{Ik had u willen,{\textquoteright}.}{verrassen want ik ben naar}{Maastricht overgeplaatst}\\

\haiku{Ene reuzeketel.}{woort op et vuur gezat en}{de kookerij beg\'os}\\

\haiku{veel em ouch nog wel,}{ins in tot heer onder z'n}{vreuger kinnesse}\\

\haiku{Et waor in et.}{jaor 1868 tot Mastreech zou}{ontmanteld weurde}\\

\haiku{Dan had heer altied.}{ene blouwen humperok aon}{en ene sjollek veur}\\

\haiku{medamme em al. '}{t\"osse hun twieje-n-in}{nao z'n meer gebroch}\\

\haiku{Ze waore toen}{good bevrund gewees en dooge}{ziech geere onderein}\\

\haiku{{\textquoteright}, waarsjouwde nog te,.}{jong vrouw van Tienes mer et}{waor al te laat}\\

\haiku{Es et evels op lol,.}{make aonkwaom leete zij}{ziech ouch neet lompe}\\

\haiku{er neet te deep nao,.}{um de invoudige reije}{tot heer et neet k\'os}\\

\haiku{{\textquoteright}, klonk tao opins ene,.}{pistolsjeut door de spin of}{e kanon aofg\'ong}\\

\haiku{Ze prouzde wie 'n,.}{kat die ze mosterd aon h\"a\"or}{neus hadde gesmeerd}\\

\haiku{Noe arriveerde.}{de deputaassies oet te}{ander d\"orrepe}\\

\haiku{ene barmeter en.}{die van Eeksterbos m\`et e}{paar houte vaze}\\

\haiku{Es er dan ouch al,.}{ins get oet et loed g\'ong k\'os}{er toch gei koed mie}\\

\haiku{Noe begreep er de:}{belangst\`ellende vraog}{van d'n andere}\\

\haiku{andere sloog ze.}{in de gawwigheid ene}{knien in zene nak}\\

\haiku{d'n onderein, um.}{ziech evekes te verpoeze}{van et dr\"ok get\'offel}\\

\haiku{{\textquoteright} - {\textquoteleft}Dan mooste miech ouch,.}{mer medein de sj\`elderije}{ophange Stiene}\\

\haiku{Iech waor ouch aon '.}{t witte wie noe en dao}{loerde dee sjijns op}\\

\haiku{Dao zant iech in de.}{sjerrever en m\`et waor}{de keuningin langs}\\

\haiku{{\textquoteleft}Vivat St. Jean{\textquoteright} (Ene;}{sjieke s\'okkerbekker deit}{alles op ze Frans}\\

\haiku{Wikske sjreef al ':}{mer z\"ochtenteeren 3 en}{ondervroog wijer}\\

\haiku{De {\textquoteleft}sc\`ene{\textquoteright}, die ziech,.}{tao zou aofspeule k\'oste ze}{ziech wel indinke}\\

\haiku{um h\"a\"or gemond ins,:}{te luchte of ze beg\'os}{tege de dames}\\

\haiku{Dao woort geklop en '.}{medam spoojde ziech naon}{ander t\"a\"ofelke}\\

\haiku{- {\textquoteleft}Trijnsje, iech m\'os 'n,?}{nuij kindermeitske h\"obbe}{h\"obste eint veur miech}\\

\haiku{zij neet ete en van.}{enen andere z'n drei sent}{nog profeteere}\\

\haiku{m\`et ene prachtige.}{sprunk te vinster oet nao et}{Leliestr\"a\"otsje in}\\

\haiku{D'n hiemel g\'ong d'n}{hook um van de Wolfstraot}{en op et kruuspunt}\\

\haiku{Noe moot me bij de,.}{Waole komme da's toch}{kokende m\`ellek}\\

\haiku{D'n eine voesslaag.}{nao d'n andere kwaom neer}{op kop en sjouwers}\\

\haiku{{\textquoteright} Teun verdween achter. '}{de deure van de buroo}{onder et Stadhoes}\\

\haiku{Dee wouw mer altied,.}{opgen\'omme weurde anders}{heel heer die moul neet}\\

\haiku{Iech verdrej et{\textquoteright}, zag, {\textquoteleft},.}{zeiech staon neet mie op}{doeg tiech et noe mer}\\

\haiku{Iech laot miech toch{\textquoteright}.}{noe al neet sjendeleere door}{m'n eige kinder}\\

\haiku{Et waos neet um,.}{te geluive wie goojekoup}{tao alles waor}\\

\haiku{- {\textquoteleft}Sagen sie maal{\textquoteright}, vroog, {\textquoteleft}?}{erhaben zie de kleine}{joenge niech gezeen}\\

\haiku{W\"omke beg\'os te,}{begriepe tot alles oet}{waor tot te meeg}\\

\haiku{{\textquoteright}, vroog er aon eine, '.}{dee em m\`etn lanteerie in}{ze geziech luugde}\\

\haiku{Mie es z\`es, zeve,.}{jaor waor er neet mer}{bij-der-hand en}\\

\haiku{- {\textquoteleft}Iech moot em evels aon{\textquoteright},.}{d'n aptieker laote}{zien deilde Greet m\`et}\\

\haiku{begaeden), ziech - =,:}{overdadig eten en drinken}{begrepder~~uit}\\

\haiku{Kempesch Kauf),kenaar (,.:}{de)-het kanaal van Maastricht}{naar Luikkenkee-(Fr}\\

\haiku{Krack),krammerin (den.:}{uitgang in uit te spreken}{als in het Fr.)-(Fr}\\

\haiku{versierde tak op,.}{de bekapping van een in}{aanbouw zijnd huis 4}\\

\haiku{Reefel),reigere-rillen,, (.}{remmelke-rammelaarreng}{mv. v. raank)-1}\\

\section{Adriaan van Oordt}

\subsection{Uit: Irmenlo. Deel 1}

\haiku{Alkwert voelde in zich,.}{een nieuw leven een leven}{van geheime smart}\\

\haiku{De oude Goden,.}{lieten er geen vrouwen toe}{geen lijfeigenen}\\

\haiku{{\textquoteleft}De kerkheer was boos,.}{omdat de nieuwe bisschop}{ontevreden is}\\

\haiku{Ze antwoordde niet, ',}{maar bracht de handen aant}{hoofd met een blik dien}\\

\haiku{Alkwert bevestigde,.}{met enkele woorden met}{goedige knikken}\\

\haiku{Hij hield er niet van,.}{de vrouw na te gaan in haar}{werken en weven}\\

\haiku{Weifelend haalde.}{hij een buidel te voorschijn}{en bood hem Alkwert aan}\\

\haiku{Ze reikte Warnef '.}{de hand en boogt hoofd in}{een weerlooze houding}\\

\haiku{En toen ze in de,;}{schaduw zijner liefde bleef}{hunkeren sprak hij}\\

\haiku{En ziende, dat hij,.}{haar blikken ontweek liet ze}{zich terugvallen}\\

\haiku{De zijnen wisten.}{andere wegen door de}{heilige wouden}\\

\haiku{Den strijdhamer zou.}{hij over hun vette nekken}{zwaaien als Donar}\\

\haiku{Naast hem gezeten,.}{scheelde ze telkens naar den}{blikkerenden hoop}\\

\haiku{en dan den bijvang,:}{verlatend aan de hoeven}{der buren klopten}\\

\haiku{Winkhorst, gevolgd door, '.}{een sleep mannen en vrouwen}{rende doort dorp}\\

\haiku{En toen ze vanavond,....}{niet terugkeerde heb ik}{gezocht en gezocht}\\

\haiku{Zijn haren vielen,.}{verward vuil vergrijsd over zijn}{voorhoofd en schouders}\\

\haiku{Zich oprichtend om,.}{te zien zag hij kersen als}{bloeddroppels vallen}\\

\haiku{{\textquoteright} Zacht spreidde hij de,.}{armen uit Alkwert aanziende}{met vochtige oogen}\\

\haiku{Hij had de handen,.}{uitgestrekt om de gunst van}{God af te dwingen}\\

\subsection{Uit: Irmenlo. Deel 2}

\haiku{En beneden hem.}{waren de oldermans in}{hun zetels van zand}\\

\haiku{Hij stortte zich neer, '.}{t gelaat naar de aarde}{en bad en dankte}\\

\haiku{Thuis gekomen, vond.}{hij Marfa met een ontsteld}{gelaat aan den haard}\\

\haiku{Ze kon niet den eenen.}{dag God en den anderen}{duivels erkennen}\\

\haiku{Hij hoorde haar niet,,:}{en tot zich zelven komend}{kreet hij smartelijk}\\

\haiku{Ze bouwden hem een,.}{hoeve statig genoeg voor}{een Wodanpriester}\\

\haiku{en bleef, de lange,.}{ledematen intrekkend}{hardnekkig zwijgen}\\

\haiku{en daar ze telkens ',.}{t hoofd schudde trachtte hij}{haar te overreden}\\

\haiku{En de wijkwasten. '.}{kwijldent Bloed regende}{over de gewaden}\\

\haiku{Zwijgend voegde hij,}{zich bij de Irmenlo\"ers}{die den vermoeiden}\\

\haiku{{\textquoteright} De lippen in een:}{wreed spotachtig beven stiet}{ze hem af en sprak}\\

\haiku{Zijn woorden brokten,.}{van elkander af terwijl}{zijn borst zich bewoog}\\

\haiku{Eeuwig draadde de,,.}{regen eeuwig als de tijd}{sinds Woonfred weg was}\\

\haiku{Alles klopte aan,,,.}{haar lijf aan haar hoofd aan haar}{hals aan haar polsen}\\

\haiku{Misschien zat hij in '.}{de hoeve tegent licht}{van den haard geleund}\\

\haiku{{\textquoteleft}Gonda, kom, 't stormt, ',,.}{alsoft in mijn hoofd ook}{stormt kom moeder wacht}\\

\haiku{t hooren van die.}{ongewoon verzoete stem}{een wellustbeving}\\

\haiku{Hij wilde wel, 't,.}{lijf door het tentezeil heen}{uitdagend schreeuwen}\\

\haiku{Zijn sterke smart werd?}{overstoft door het ootmoedig}{wachten en op wat}\\

\subsection{Uit: Warhold}

\haiku{{\textquoteright} Daar de anderen,:}{duister voor zich zagen stond}{hij op en schreeuwde}\\

\haiku{ik afschaffing van,,.}{de schande aan de kerk dus}{ook aan mij gedaan}\\

\haiku{Haar gelaat in de,,:}{handen liet ze zich voorover}{gaan al stamelend}\\

\haiku{en zijn handen in,,.}{elkander peinsde hij hoe}{hij zich redden zou}\\

\haiku{Hij rilde, en met.}{een koude stem klaagde hij}{zichzelf aan bij God}\\

\haiku{Hij gaf zich over aan,.}{de hooge gedachten die zijn}{leven vervulden}\\

\haiku{Hij rilde - toen hij.}{een schaduwgestalte door}{de hamei zag gaan}\\

\haiku{Hij dacht niet over wat.}{hij had misdaan en wat er}{verder te doen stond}\\

\haiku{Uw gelaat is als.}{zon in het moer en uw gang}{als der honden vlucht}\\

\haiku{'t Was hem, alsof.}{door haar woorden een slag in}{hem te vallen kwam}\\

\haiku{Zij had al zoo veel}{maren gehoord van nood daar}{binnen en van nood}\\

\haiku{Maar dan werd een groot.}{gekal van stemmen en een}{kraakgestoot gehoord}\\

\haiku{en taal noch teeken,.}{noch minder een maere werd}{van hem vernomen}\\

\haiku{Zijn armen schokten,:}{en het speeksel schuimde uit}{zijn mond toen hij riep}\\

\haiku{{\textquoteleft}Wij zijn gewoon, ons.}{zelf te genezen en ons}{zelf te wapenen}\\

\haiku{Maar niet willende,,.}{denken zweepte hij zich voort}{nog sneller loopend}\\

\haiku{Ik leefde zwaar aan,,.}{de aarde te loom om mijn}{handen te heffen}\\

\haiku{Maar zoetjes aan werd,.}{hij vervuld van een warmte}{die hem beven deed}\\

\haiku{Toen deze Warhold,:}{herkende schoot hij in een}{schaterlach en riep}\\

\haiku{Met stijve lippen,.}{bevroeg hij Janne hoe dit}{zoo geworden was}\\

\haiku{Maar ik zal u een.}{maal doen bereiden en dan}{alles verhalen}\\

\haiku{Maar steeds zijn handen,.}{om de hare lengde hij}{zich tegen haar aan}\\

\haiku{Zij liepen terug,.}{en begonnen weer de vaart}{nog eens en nog eens}\\

\haiku{weifelen deed en.}{gaandeweg de dingen van}{zijn blikken leidde}\\

\haiku{Hij kon zijn gevoel,.}{zijn blikken en zijn schreden}{niet meer beheerschen}\\

\section{G.A. van Oorschot}

\subsection{Uit: Twee vorstinnen en een vorst (onder ps. R.J. Peskens)}

\haiku{Ik hoefde niet te,.}{kijken want ik wist hoeveel}{streepjes er stonden}\\

\haiku{Ik weet nog hoe ik.}{liep te klappertanden van}{kou en ellende}\\

\haiku{Onder de schoorsteen.}{lagen forse houtblokken}{hoog opgestapeld}\\

\haiku{Op tafel stond een.}{volle schaal met bolussen}{en krentebollen}\\

\haiku{En sindsdien begint.}{voor mij elke kerstdag met}{scherven op de muur}\\

\haiku{Hij wou geen kolen,,.}{geven zei ik terwijl ik}{moeilijk overeind kwam}\\

\haiku{ontnam een van de,.}{knechten zijn schop en vulde}{de beide emmers}\\

\haiku{Ik richtte mij op.}{en zag haar hele gezicht}{overstroomd met tranen}\\

\haiku{Ga eens even op zij,.}{zei moeder en duwde me}{de kamer weer in}\\

\haiku{Omdat de agent zo.}{beleefd en verlegen deed}{werd ik nog banger}\\

\haiku{Ik heb geen vriendjes,.}{opgezocht maar ben naar mijn}{grootmoeder gegaan}\\

\haiku{Vader trok aan het.}{touw de voordeur open en kwam}{ook naar beneden}\\

\haiku{Daardoor wist ik dat.}{de afbetaling een vol}{jaar had geduurd}\\

\haiku{Ik zag dat alle.}{kinderen hun boeken en}{schriften al hadden}\\

\haiku{En wat die boeken.}{betreft moet je maar even naar}{de concierge gaan}\\

\haiku{Wat heb je een rooie,,.}{kop zei moeder toen ik de}{keuken binnenkwam}\\

\haiku{Ik ben niet naar mijn,.}{klas teruggegaan maar de}{school uitgelopen}\\

\haiku{Ik wilde eerst naar.}{mijn grootmoeder gaan omdat}{ik veel van haar hield}\\

\haiku{Ze pakte mij bij.}{de linkerpols en sleepte}{mij achter zich aan}\\

\haiku{Daarna werd de deur.}{op een kier geopend en}{gevraagd wie er was}\\

\haiku{Boezeroen heeft een,.}{trui joelde het die ochtend}{door de kleedkamer}\\

\haiku{Ik weet niet wat je.}{begonnen bent hem naar die}{rotschool te sturen}\\

\haiku{Van mijn gespaarde.}{zondagscenten kocht ik mij}{zelf stiekum een zwembroek}\\

\haiku{Het strand was in twee,.}{helften verdeeld een kleine}{en een grote helft}\\

\haiku{Geen van de jongens.}{merkte op dat ik niet mee}{het water inging}\\

\haiku{alsof ik een steen,.}{wegwierp in de richting van}{de leraar gooide}\\

\haiku{Zelfs per ongeluk.}{heb ik nooit een bal door de}{korf kunnen gooien}\\

\haiku{Toen ik op straat kwam,.}{zei ze dat ik niet altijd}{zo treuzelen moest}\\

\haiku{Piet Zeeman aaide.}{over mijn haar en daar werd ik}{erg verdrietig van}\\

\haiku{meelopend konden.}{ze elkaar gelukkig nog}{net een hand geven}\\

\haiku{Links en rechts sloeg ze.}{me om de oren en joeg me}{voor straf naar bed}\\

\haiku{Dan laten we het,.}{licht nog maar even uit voegde}{hij er aan toe}\\

\haiku{Toen ze de kamer.}{binnen kwam wierp ze het met}{een smak op tafel}\\

\haiku{Maar die zaten nog.}{helemaal aangekleed op}{de rand van hun bed}\\

\haiku{Die eerste weken '.}{werd ers nachts om toerbeurt}{bij vader gewaakt}\\

\haiku{Al was je honderd,, '}{keer de burgemeester zei}{ze je komt niet bij}\\

\haiku{Mijn vader las de,.}{kranten debatteerde weer}{met zijn bezoekers}\\

\haiku{Na een minuut of.}{tien riep vader of moeder}{even komen wilde}\\

\haiku{Bij het naar bed gaan.}{legde ik mijn broek onder}{mijn hoofdkussen}\\

\haiku{Op de hoogste bank.}{kon je met je handen het}{tentzeil aanraken}\\

\haiku{Heb je nooit ergens,.}{voorgoed willen blijven vroeg}{moeder plotseling}\\

\haiku{waar zou moeder nou.}{wezen en wanneer zou ze}{nou terugkomen}\\

\haiku{Eerst gaf vader geen.}{antwoord en er ontstond een}{moedeloos zwijgen}\\

\haiku{In het begin vroeg.}{hij haar nog wel of ze de}{brief niet lezen moest}\\

\haiku{Het enige waar ze.}{met hartstocht aan deelnam was}{het toneelspelen}\\

\haiku{En ze vroeg of er.}{een of andere pestkop}{was die hem dwars zat}\\

\haiku{Ik heb gezegd, zei,.}{hij plechtig na een poos}{gewacht te hebben}\\

\haiku{En toen heb je met,.}{je zakdoek het zweet van je}{kop geveegd zei ik}\\

\haiku{hoe mevrouw Coster}{de aardrijkskundestok op}{mijn rug kapot sloeg}\\

\haiku{Als je het liever,.}{niet vertelt zei ik ineens}{vol medelijden}\\

\haiku{Ach, lachte hij, dat.}{zeg ik natuurlijk maar bij}{wijze van spreken}\\

\haiku{Je begrijpt hoe ik.}{schrok en totaal van de kaart}{was van dit antwoord}\\

\haiku{En hoe lang heeft dat,.}{met elkaar geduurd in de}{voorkamer vroeg ik}\\

\haiku{Ze haalde de lijst.}{er af en hing de foto}{boven het buffet}\\

\haiku{VII Mijn vader werd.}{op hoge leeftijd naar het}{ziekenhuis gebracht}\\

\haiku{Twee dagen voor zijn.}{dood lagen de brieven nog}{in de linnenkast}\\

\haiku{Ik ben humeurig.}{omdat ik de hele dag}{niets heb uitgevoerd}\\

\haiku{Ik zou letterlijk.}{kunnen opschrijven wat er}{gezegd gaat worden}\\

\haiku{Ze zet haar knie\"en,.}{naast elkaar trekt haar rok en}{onderrok omhoog}\\

\haiku{Dat kan wel wezen,.}{zegt moeder maar ik heb er}{geen eigenschap aan}\\

\haiku{Tw\'e\'e en tachtig zegt.}{vader op een toon alsof}{hij hem de baas is}\\

\haiku{De burgemeester.}{vraagt of er veel zieken zijn}{in de gemeente}\\

\haiku{Ze richt plotseling.}{haar hoofd op en zegt dat ze}{geen zin in eten heeft}\\

\haiku{Een dunne witte.}{kaars is vastgebonden aan}{een stukje boomschors}\\

\haiku{Zal ik het grote.}{licht nou ook maar een beetje}{aandoen vraag ik haar}\\

\haiku{Vader helpt moeder.}{naar bed en zet daarna de}{televisie aan}\\

\haiku{Maar ik begrijp best.}{dat je morgen vroeg weer op}{je werk moet wezen}\\

\haiku{Mijn moeder heeft een.}{paar maanden geleden een}{attaque gehad}\\

\haiku{Hij wist echt niet hoe,.}{het kwam maar de hele stad}{was op de hoogte}\\

\haiku{En de huisdokter.}{die ik erover aansprak vindt}{dat eigenlijk ook}\\

\haiku{De lijkdienaren.}{stonden opgesteld met de}{kist op hun schouders}\\

\haiku{Ach, zei hij, ik heb.}{de bloemen nog geen water}{gegeven vandaag}\\

\haiku{We moeten haar een,.}{proces aandoen had een van}{de broers gezegd}\\

\haiku{Hij trok de schuif weer.}{helemaal open en vulde}{wat antraciet bij}\\

\haiku{Ik doe mijn broek en.}{overhemd uit en ga aan het}{tafeltje zitten}\\

\haiku{En hij verzuchtte.}{dat een mens het helaas niet}{voor het zeggen heeft}\\

\haiku{Mij komt de zondag,.}{in verband met mijn werk het}{beste uit zei hij}\\

\haiku{Mijn zuster Ka heeft,.}{eens gezegd dat onze broer}{gemakkelijk leeft}\\

\haiku{En langer dan een.}{halve dag moeten we er}{niet aan besteden}\\

\haiku{Ja, zei tante Cor,.}{ik hoop niet dat jullie het}{mij kwalijk nemen}\\

\haiku{Hij trok zijn jasje.}{uit als een werkman die voor}{een groot karwei staat}\\

\haiku{Zit-ie klem die,.}{deur vroeg hij en duwde me}{een eindje op zij}\\

\haiku{O marianne,,.}{O proletaren Aan de}{strijders Morgenrood}\\

\haiku{Ze zat recht op in,.}{haar bed met haar handen om}{de spijlen geklemd}\\

\haiku{Toen ze helemaal.}{niet  reageerde ben}{ik maar weggegaan}\\

\haiku{doe net of je me,.}{omhelst en druk dan even je}{vingers om mijn hals}\\

\haiku{En daar op tafel.}{vindt u een bakje met wat}{kleine spulletjes}\\

\haiku{Ik zal enkel dat,.}{gouden broche en haar ring}{meenemen zei ik}\\

\haiku{Ik wou het haar toch,.}{liever zelf zeggen want ik}{hield toen ook van h\`a\`ar}\\

\section{Jo Otten}

\subsection{Uit: Bed en wereld}

\haiku{{\textquoteleft}Ik ben even langs de.}{werkster geweest en heb haar}{een tientje gebracht}\\

\haiku{Hij vertelt verder;}{hoe hij Ann kort daarna voor}{goed had verloren}\\

\haiku{Hij weet het niet, hij}{weet alleen dat hij haar zou}{hebben weggehaald}\\

\haiku{Zo verdiept was hij.}{in zijn lectuur dat hij mij}{niet heeft opgemerkt}\\

\haiku{Zeker, zeker, had,.}{hij dat kunnen doen maar hij}{heeft het niet gedaan}\\

\haiku{Ik werd bedwelmd door,.}{de rook de zweetlucht en de}{goedkope parfums}\\

\haiku{Maar terugkomend.}{kreeg ik lust om bij een van}{haar binnen te gaan}\\

\haiku{Voor een gebarsten.}{kapspiegel installeerde}{zich Little Esther}\\

\haiku{Ik ging het trapje,.}{af dat van haar kamertje}{naar de gang leidde}\\

\haiku{Langs ons heen is een;}{va-et-vient van negers}{en negerinnen}\\

\haiku{Ik wil geen vrouwen,,.}{ik wil slapen geef mij toch}{wat om te slapen}\\

\haiku{We komen op een...}{kamer met een groot bed en}{alweer met spiegels}\\

\haiku{Nee, nee, dat zal zij,.}{zeker niet doen dat zal zij}{toch zeker niet doen}\\

\haiku{zij vindt het zelf veel.}{te aangenaam dat mijn blik}{over haar lichaam strijkt}\\

\haiku{Zij hebben bressen.}{geschoten en andere}{harten vrij gemaakt}\\

\haiku{de rest vreet zich in.}{het lichaam en blijft voor het}{daglicht verborgen}\\

\haiku{Nee, er was niemand,.}{alleen maar die oude vrouw}{en die oude man}\\

\haiku{Ik wil niet eenzaam,.}{sterven laat ik nu maar dood}{blijven in mijn bed}\\

\haiku{Ik kom een zaal met,,}{mensen binnen ik loop naar}{voren doe of ik}\\

\haiku{je portret aan de...}{muur van de slaapkamer was}{het enige wat bleef}\\

\haiku{Luisteren naar de...,,...}{klok staren in de vlam niet}{naar rechts niet naar links}\\

\haiku{Ik moet op tijd zijn,,;}{ik moet het voorbeeld geven}{ik heb nog een uur}\\

\haiku{geen tijd, geen klok, geen,...}{zwarte wijzers die sluipen}{over een wijzerplaat}\\

\haiku{slaperig kijkt haar,.}{hoofd boven het gestikte}{zijden overdek uit}\\

\haiku{Voor het naar bed gaan,.}{bidt zij haar knie\"en worden}{koud op de stenen}\\

\haiku{zij is de vrouw van,.}{een vossenkweker niet van}{een warmoezenier}\\

\subsection{Uit: Muizen en demonen}

\haiku{zij houdt z\'o\'oveel van,.}{hem m\'e\'er dan ik ooit van hem}{zal kunnen houden}\\

\haiku{Na den dood van mijn.}{vader bleef ik geruimen}{tijd bij mijn moeder}\\

\haiku{W\'e\'ervinden, mag}{ik van weervinden spreken}{nu hij z\'o\'o dichtbij}\\

\haiku{Altijd wist hij mij,.}{moed in te spreken nooit liet}{hij zich temeer slaan}\\

\haiku{d\'a\'arom  moest hij, om,.}{zijn leven te redden in}{de diepte springen}\\

\haiku{zijn gestamelde.}{explicaties zijn nooit tot}{mij doorgedrongen}\\

\haiku{Tenslotte bood zij,;}{geen weerstand meer alles deed}{zij wat hij wilde}\\

\haiku{Dat {\textquoteleft}eigen werk{\textquoteright} ging:}{een voorname plaats in mijn}{leven innemen}\\

\haiku{Iederen morgen}{stond ik op met nieuwen moed}{en meer dan vroeger}\\

\haiku{Ik dacht alleen maar,.}{aan Marceline die hulp}{en  zorg noodig had}\\

\haiku{Ik was overtuigd dat;}{ik toch altijd dezelfde}{voor haar zou blijven}\\

\haiku{Ik trachtte hem een,;}{beeld te geven van de angst}{waarin wij leefden}\\

\haiku{Steeds was ik beangst,.}{voor ons kind dat het eenige}{zou moeten blijven}\\

\haiku{Het was een leven,.}{in een paradijs dat wreed}{zou worden verstoord}\\

\haiku{Het speelde vaak door,.}{mijn hoofd in de nachten dat}{ik slapeloos lag}\\

\haiku{Nooit was zij liever.}{dan in dien tijd toen zij mij}{z\'o\'o vermagerd zag}\\

\haiku{koud en afwerend,.}{stond zij tegenover mij ik}{wist niet wat te doen}\\

\haiku{ik zag haar oogen en,.}{die oogen waren als altijd}{trouw en toegewijd}\\

\haiku{Een beetje schuldig:}{zal hij behoedzaam binnen}{komen en vragen}\\

\haiku{Wanneer de bloemen,,.}{bloeien denkt zij kan ook de}{liefde niet sterven}\\

\haiku{Wij vonden elkaar,...}{in een omarming die niet}{wilde eindigen}\\

\haiku{Zij ging terstond naar.}{de kleine keuken om wat}{eten klaar te maken}\\

\section{J. van Oudshoorn}

\subsection{Uit: Achter groene horren}

\haiku{Thans echter werden.}{van den nacht de schaduwen}{niet uiteen gevaagd}\\

\haiku{In allerhande.}{vormen stonden er ook reeds}{gevlochten manden}\\

\haiku{Nu lag zij er door.}{straten en huizenblokken}{in toom gehouden}\\

\haiku{Een drabbige sloot,,.}{met begin van iets als een}{sluisje dwars er door}\\

\haiku{Waarom juist nu dat?}{andere zich tusschen hen}{beiden had gesteld}\\

\haiku{Een lekkertje, zij,,,,.}{met Verwaaien en die en}{die daar in die keet}\\

\haiku{Tien jaar lang geduld,.}{te moeten hebben enkel}{om iets te worden}\\

\haiku{Ten slotte uit de,,;}{bank gesleurd struikelend nog}{aan den muur gekwakt}\\

\haiku{Minstens drie, en zoo:}{mocht hij er wel drie dagen}{minstens over denken}\\

\haiku{liet den ander het.}{laatste stuk door de stad een}{eind voor zich uit gaan}\\

\haiku{Maar toen reed hij al.}{achterwaarts om voldoende}{aanloop te krijgen}\\

\haiku{{\textquoteright} stond er enkel nog.}{met reuzenletters in het}{midden der heining}\\

\haiku{En tot geen prijs ter.}{wereld er met wie dan ook}{ooit over begonnen}\\

\haiku{Dat begreep hij niet,,.}{hij die zich reeds ongestraft}{voorschot had verschaft}\\

\haiku{Wat getemperd licht,.}{ook zacht over afgeronde}{lage meubeltjes}\\

\haiku{Voorschot, genomen....}{weer op zoo'n voorschot nog in}{het verschiet Elastiek}\\

\haiku{Voor een tragedie.}{ontbrak het aan mede-}{en tegen-spelers}\\

\haiku{Vreemde gezichten,.}{maar niet \'e\'en van die enkel}{dom-gezonde}\\

\haiku{Hier werd, met kalfsvel,.}{en trompetten een hooger}{leven ingeluid}\\

\haiku{En toch, iets reins en.}{veiligs was zooeven uit zijn}{leven heen gegaan}\\

\haiku{Dat haar zulk verschil.}{in postuur op den duur niet}{onuitstaanbaar werd}\\

\haiku{Langer dan een maand.}{was hij hier door centrale}{verwarming verwend}\\

\haiku{Had daarbij de chef?}{toch niet de hand vluchtig aan}{zijn schouder gebracht}\\

\haiku{Want hoe natuurlijk,.}{hoe bevrijdend werkte thans}{daartegen de straat}\\

\haiku{Door de harmonie,.}{van het landschap die eens tot}{teekenen noopte}\\

\haiku{Al was hij er niet,...}{meer op in gegaan liet het}{hem toch geen rust}\\

\haiku{Maar die mocht hij, na,.}{aankomst daar declareeren}{en bleef zoo safe}\\

\haiku{En toch, niet enkel -}{in gedachten hij merkte}{het met voldoening}\\

\subsection{Uit: Bezwaarlijk verblijf}

\haiku{De handtasch trouwens.}{bevatte het hoognoodige}{voor zoo'n eerste nacht}\\

\haiku{Intusschen had hier.}{en daar een der andere}{eters plaats genomen}\\

\haiku{geen liflafjes bij.}{een half fleschje azijn als}{in dat warenhuis}\\

\haiku{Dit vooral om den,,}{ander te toonen hem eens}{te laten voelen}\\

\haiku{Vroegen daar beide?}{riolen niet even gretig}{om volle aandacht}\\

\haiku{Van haar salaris.}{bij de kapel alleen kon}{zij niet rond komen}\\

\haiku{Was zij, liever dan,.}{nadeel te riskeeren weer haar}{eigen weg gegaan}\\

\haiku{Op een leugen meer.}{of minder scheen het daarbij}{niet aan te komen}\\

\haiku{Aan crediet voor die.}{laatste dagen kon dit slechts}{ten goede komen}\\

\haiku{Het d\'efil\'e werd.}{door het algemeen gedrang}{telkens weer gestremd}\\

\haiku{Nog geen voertuig of.}{voetganger was intusschen}{voorbij gekomen}\\

\haiku{Dan bevond hij zich.}{in zoo'n overvolle coup\'e}{met houten banken}\\

\haiku{Zijn ademhaling even,.}{deed stokken als in een te}{eng bedompt vertrek}\\

\haiku{Terwijl geld toch het.}{aller-onontbeerlijkste}{hier beneden was}\\

\haiku{Van haar salaris.}{bij de kapel alleen kon}{zij niet rondkomen}\\

\haiku{Was zij, liever dan,.}{nadeel te riskeeren weer haar}{eigen weg gegaan}\\

\haiku{Want dat die hem een,.}{beentje gelicht had daarvan}{ging evenmin iets af}\\

\haiku{Wel jammer, nu hij...}{zich hier wat op dreef en thuis}{begon te voelen}\\

\haiku{Hij had het gevoel.}{te moeten wijken voor het}{neefje van de chef}\\

\haiku{Ik heb me - als de -.}{doctorandus19 van iets}{moeten vrijmaken}\\

\haiku{wanneer je in de,.}{put zit doe dan juist hetgeen}{je het moeilijkst schijnt}\\

\haiku{Van Oudshoorn schreef het.}{tussen begin oktober}{en eind december}\\

\haiku{Ich war also ganz,}{ruhig aber in Antwerpen}{konnte ich ihn}\\

\haiku{Schreibe auch sehr oft.}{und sei herzlich gek\"usst von}{deinem trauen Mann Jan}\\

\haiku{Tobias en de,,:}{dood Deventer Dagblad 25}{november 1925 W.}\\

\haiku{Pinksteren, Weekblad,:}{voor Rotterdam 4 mei 1929}{Israel Querido}\\

\haiku{Pinksteren, Dagblad,:}{van Rotterdam 27 juni}{1929 Henri Borel}\\

\haiku{Pinksteren, De Vrije,,,-:}{Bladen vi december 1929}{blz. 406407  N.N.}\\

\haiku{Holland's Welvaren,,:}{De Groene Amsterdammer}{19 juli 1930 N.N.}\\

\haiku{De Java Bode, []:}{21 februari 1931 N.N.}{Dr. P.H. Ritter Jr}\\

\haiku{Hendrik Hagenaars,,:}{Hoek De Nieuwe Courant 20}{oktober 1949 N.N.}\\

\haiku{Van Oudshoorn bekroond,,:}{Algemeen Handelsblad 9}{augustus 1950 N.N.}\\

\haiku{Schrijven uit noodzaak,, []:}{Trouw zaterdag 4 mei 1968}{Verz. W.I. W.A.M. de Moor}\\

\haiku{De eerste druk van.}{het handboek vermeldt immers}{het juiste tijdstip}\\

\haiku{64Waarover Feijlbrief,.}{met zijn neef gesproken heeft}{is onduidelijk}\\

\subsection{Uit: De fantast}

\haiku{Daar de vrouw hem niet,:}{meer weg joeg versterkte hem}{dit in het besef}\\

\haiku{- De bedoeling was, '.}{s dichters werkkamer te}{laten verbouwen}\\

\haiku{Alleen opzij van,.}{een groepje mensen die daar}{op de tram wachtten}\\

\haiku{{\textquoteright} {\textquoteleft}Het vreemdste lijkt toch,.}{wel dat je ook kunt dromen}{ni\`et te dromen}\\

\haiku{{\textquoteright} {\textquoteleft}Vroeger dachten ze,.}{dat de aarde zo plat als}{een pannekoek was}\\

\haiku{Van geluk naar de,.}{godsdienst en geharrewar}{was nog slechts een stap}\\

\haiku{{\textquoteleft}Maar, maar heeft U dan?}{niet genoeg aan wat ik U}{zelf reeds gezegd heb}\\

\haiku{Hij had niets tegen,.}{Dani\"el al was het ook}{een raar stuk schoonzoon}\\

\haiku{Jawel, al zet U,.}{ook grote ogen op zo was}{het en niet anders}\\

\haiku{Zou het niet goed zijn,?}{wanneer je er eens voor een}{paar weken uitkwam}\\

\haiku{Van een buitenlands.}{paspoort was dan ook nog steeds}{geen werk gemaakt}\\

\haiku{Trouwens, Doortje hield.}{alweer een nieuw karweitje}{voor hem in petto}\\

\haiku{De geruchten over....}{een nieuwe oorlog werden}{steeds hardnekkiger}\\

\subsection{Uit: In memoriam}

\haiku{Hoe warmer het werd,.}{hoe meer ter Laan zich op dat}{verlof verheugd had}\\

\haiku{Het mag zijn, dat de.}{dikke h\^otelhouder het}{een en ander dacht}\\

\haiku{Toen dat ook zonder,:}{zichtbare uitwerking bleef}{haar eenig antwoord een}\\

\haiku{Niettegenstaande,.}{herhaald opbellen meldde}{er zich niemand meer}\\

\haiku{Daartoe was de les,,.}{naar beide zijden al te}{gevoelig geweest}\\

\haiku{Zijn innerlijke.}{verwarring schrijnde tot het}{ondragelijke}\\

\haiku{Ze zaten in een.}{ruime rieten bank in een}{hoek der veranda}\\

\haiku{Van de boulevard.}{beneden viel geen voetstap}{meer te vernemen}\\

\haiku{Een elkander even.}{aanzien leek thans intiemer}{dan een omhelzing}\\

\haiku{zijn terughouding,,.}{zijn wantrouwen waren nog}{eenmaal doorbroken}\\

\haiku{Met opgeslagen.}{jaskraag bevond de student}{zich weder op straat}\\

\haiku{Even leek het hem, als.}{was hij sinds jaren niet meer}{in die club geweest}\\

\haiku{Daarbij wist hij te {\textquoteleft}{\textquoteright}.}{goed deZeester niet meer te}{willen verlaten}\\

\subsection{Uit: Louteringen}

\haiku{Fagin, de roode,.}{jood voor zijne rechters en}{ter dood veroordeeld}\\

\haiku{De dood en Fagin's.}{schrikkelijke aantijging}{tegen het leven}\\

\haiku{Zelfs met het oor aan.}{het sleutelgat kon hij nog}{niets bepaalds verstaan}\\

\haiku{Tot beneden de:}{trapdeur open gerukt werd en}{Cato's vreugde galm}\\

\haiku{Maar wanneer zij naakt,.}{geweest was had zij de deur}{niet toegeworpen}\\

\haiku{Pas in de winkel.}{hoorde hij het algemeen}{gelach losbreken}\\

\haiku{bij hetgeen hem den.}{daarop volgenden laatsten}{kermisdag overkwam}\\

\haiku{Tegen Simon bleek.}{hij dien eersten middag al}{niet opgewassen}\\

\haiku{O, o, nog slechts twee.}{nuttelooze dagen scheidden}{hem van zijn vertrek}\\

\haiku{Het was niet meer te,.}{loochenen dat hij op den}{verkeerden weg was}\\

\haiku{En wat kon het hem.}{schelen dat hij wederom}{een les verzuimde}\\

\haiku{Hij haalde diep adem.}{en keerde zich met een schok}{weder naar de klas}\\

\haiku{Hij wist precies hoe.}{het binnen die andere}{jongens was gesteld}\\

\haiku{Ja, wanneer hij haar.}{voor zich alleen gehad had}{in zijn kamertje}\\

\haiku{Plots voelde hij zich.}{koortsig eenzaam en durfde}{niet meer op te zien}\\

\haiku{Alleen met schrijven.}{was het hem na dien eersten}{keer niet meer gelukt}\\

\haiku{En toch wanneer zijn.}{moeder een tweede sleutel}{van het kastje had}\\

\haiku{Vind je hier soms niet.}{veel meer dan jij noodig hebt en}{bij vertrouwde lui}\\

\haiku{Toch had deze hem.}{nog het adres van Moleschot}{medegegeven}\\

\haiku{Eindelijk alleen,.}{maar de kelner met zijn oog}{voor het sleutelgat}\\

\haiku{Na afloop van den.}{dag stond zijn gelaat nog scheef}{van het huichelen}\\

\haiku{Welk een geluk, dacht,.}{hij dat er nimmer halfheid}{in mijn leven was}\\

\haiku{Slechts hier en daar werd.}{aan het lage water een}{lantaarn ontstoken}\\

\haiku{Hij achtte het even,...}{natuurlijk om keizer te}{zijn als een ander}\\

\haiku{Hij kon weenen van,.}{geluk maar weemoed drong niet}{binnen tot dien rust}\\

\haiku{De natuurlijke.}{mensch leek spoorloos in hem te}{zijn ondergegaan}\\

\haiku{Het breken van de.}{zee werd er door het geroesch}{der menschen overstemd}\\

\haiku{Zijn aschblond haar hing.}{in een vreemde schuine punt}{over het leege voorhoofd}\\

\haiku{Om zich een houding.}{te geven frommelde hij}{het programma open}\\

\haiku{De hemel stond nog.}{in klaren weerschijn van de}{ondergaande maan}\\

\haiku{Wel stond hij nog in.}{de kinderschoenen van zijn}{geestelijk streven}\\

\haiku{Nog nimmer had hij.}{de gevaren van dezen}{omgang onderkend}\\

\haiku{Er was daar ergens.}{een verwaarloosd archief in}{orde te brengen}\\

\haiku{Jammer, dat hij met.}{haar nooit eens iets dergelijks}{ondernomen had}\\

\haiku{Hij ontwaarde het.}{jonge model als door een}{vreemde hand geteekend}\\

\haiku{Slechts voor de vensters}{was een schuine ruimte vrij}{gebleven tusschen}\\

\haiku{Maar hij moest toch weer.}{eens terloops de juffrouw haar}{meening uitlokken}\\

\haiku{Ook wist hij waarom.}{hij alle verkeer met hen}{had afgebroken}\\

\haiku{Zich in te beelden.}{dat hij zelf een dier kleine}{zwarte menschen was}\\

\haiku{Het is de nevel,.}{die in het bewustzijn zelf}{verhelderen moet}\\

\haiku{Verbergen was het,.}{onbewuste beginnen}{leugen het einde}\\

\haiku{Maar zou dit ijdel?}{verbeeldingswerk dan nooit een}{einde meer nemen}\\

\haiku{Zoo moest ook Paula.}{er hebben uitgezien toen}{zij negen jaar was}\\

\subsection{Uit: Pinksteren}

\haiku{Onwillig keerde.}{zich de andere jongen}{naar het drietal om}\\

\haiku{Zij kende dat reeds.}{tot vervelens toe en het}{bleef steeds hetzelfde}\\

\haiku{Arie, met opzet  ,.}{natuurlijk was een heel eind}{achter gebleven}\\

\haiku{Een doordringende.}{reuk van broeiend hooi steeg van}{alle kanten op}\\

\haiku{Want geen twijfel, of....}{er kwam een onweer van je}{welste opzetten}\\

\haiku{Het hardnekkigste.}{echter had haar die jonge}{violist vervolgd}\\

\haiku{zooals die sinds lang dien?}{laten avond tusschen hen was}{wedergekeerd}\\

\haiku{Maar toen werd het toch...}{een zich haasten om niet te}{vroeg te wezen}\\

\haiku{Arie den volgenden.}{Zondagmiddag nog heel wat}{moeten slikken}\\

\haiku{Zoo met zijn vieren.}{kon het natuurlijk tot geen}{bespreking komen}\\

\haiku{Plotseling had zij.}{het daar op het grasveldje}{benauwd gekregen}\\

\haiku{Het bleef dus, voor dien,.}{avond tenminste bij enkel}{groot-doenerij}\\

\haiku{Een heete bloedgolf,.}{steeg Arie alles verdoovend}{naar de hersenen}\\

\haiku{De hotelhouder.}{wist zoo'n nauwlettendheid ten}{zeerste te waardeeren}\\

\haiku{Want nu lag Marie,.}{in bed zij het dan van de}{kamer afgewend}\\

\haiku{Neen, dan was het maar.}{beter zwijgend alle schuld}{op zich te nemen}\\

\haiku{Maar geen twijfel meer,.}{woord voor woord had hij Marie's}{gestamel verstaan}\\

\haiku{Maar wel zeker, en,.}{wat d\`at betreft Arie had er}{al over nagedacht}\\

\haiku{Ook keek hij niet op,.}{toen Marie dan eindelijk}{de stilte verbrak}\\

\subsection{Uit: Tobias en de dood}

\haiku{Dat hem aanlokte.}{en tegelijkertijd met}{afkeer vervulde}\\

\haiku{Eenmaal zoo dicht in,.}{de buurt had hij toch even goed}{naar huis kunnen gaan}\\

\haiku{Vol zelfbewustheid.}{streek Tobias zijn lange}{zwarte snorren op}\\

\haiku{Men maakte er zich.}{in dit gezelschap hoogstens}{belachelijk door}\\

\haiku{Hoe dan ook, met die.}{bende daarachter had hij}{voor goed gebroken}\\

\haiku{Thans liet hij een paar {\textquoteleft}{\textquoteright}.}{nieuwe platen draaien om}{zein te studeeren}\\

\haiku{Met haar hoed op en,.}{haar mantel aan want zij was}{er slechts op bezoek}\\

\haiku{Met de handen in.}{de schoot bleef deze roerloos}{aan tafel zitten}\\

\haiku{Tobias lachte.}{tot hem de tranen over de}{wangen biggelden}\\

\haiku{O! o!, daartegen.}{zou de Roode het beslist}{moeten afleggen}\\

\haiku{Daarin week hij nu.}{eenmaal geen duimbreed van zijn}{beginselen af}\\

\haiku{Tot hem een stem aan.}{de telefoon in die rust}{was komen storen}\\

\haiku{Ja het ontbrak er,.}{nog maar aan dat zij bij de}{gramophoon ging zingen}\\

\haiku{Dat was al jaren,.}{geleden maar nu begreep}{hij het toch op eens}\\

\haiku{Zonder te drinken.}{had Tobias het glas weer}{van zich afgezet}\\

\haiku{Alles goed en wel,?}{maar waarom rekenschap te}{willen afleggen}\\

\haiku{gruwelijk zat te,.}{vervelen hardnekkig haar}{eigen gang gegaan}\\

\haiku{Maar aan het verstand.}{van Tobias twijfelde}{zij daarom niet meer}\\

\haiku{In de biljartzaal;}{waren eenige vreemden zeer}{luidruchtig bezig}\\

\haiku{Als zoo dikwijls had.}{Tobias haar discreete}{kloppen niet gehoord}\\

\haiku{X. Na langen tijd.}{komt Tobias weder in}{beter gezelschap}\\

\haiku{In dit verband moest.}{hij dikwijls aan zijn eerste}{huwelijk denken}\\

\haiku{Peet vernam, dat de.}{dokter zelf een bekende}{stille drinker was}\\

\haiku{De betreffende.}{vertrok nog haastiger dan}{zij gekomen was}\\

\haiku{Kitty, Tobias',.}{verloofde geeft blijken van}{groote zelfstandigheid}\\

\haiku{Verduiveld, hoe had....}{hij dat walgelijke tuig}{kunnen vergeten}\\

\haiku{zooals die zich tijdens.}{hun alleen zijn nog nimmer}{kenbaar had gemaakt}\\

\haiku{Er scheen nog iets van.}{den ouden zee-officier}{in hem te leven}\\

\haiku{Zulke voorschriften.}{waren even aanmatigend}{als belachelijk}\\

\haiku{Het speet Tobias,.}{nu dat er besloten was}{om op te breken}\\

\haiku{{\textquoteleft}Tobias is moei{\textquoteright}, {\textquoteleft}{\textquoteright}.}{zeide zijn schoonzustermen}{kan het aan hem zien}\\

\haiku{Dat was, om er niet,.}{meer van te zeggen toch wel}{hoogst eigenaardig}\\

\haiku{Om kort te gaan, de,!}{aangifte was gedaan de}{aanklacht ingediend}\\

\haiku{Wie of er dan van?}{afwist en daarvan misbruik}{trachtte te maken}\\

\haiku{Maar nu kwam toch de.}{beurt aan Tobias om groote}{oogen op te zetten}\\

\haiku{Hier moest Irma nog,}{meer naar het licht gaan staan en}{hield Tobias haar}\\

\haiku{Blijkbaar had thans de;}{tegenpartij twee volle}{uren op hem gewacht}\\

\haiku{Zoo behoefde hij,.}{Irma niet te roepen of}{om haar te bellen}\\

\haiku{{\textquoteright} herhaalde hij dus,, {\textquoteleft}?}{als had hij Irma niet goed}{verstaanBij de hand}\\

\haiku{Maar, wat hij van den,.}{aanvang af vermoed had was}{waarheid geworden}\\

\haiku{De toespraak van den.}{geestelijke wilde maar}{geen einde nemen}\\

\haiku{Goede hemel, op}{de vergevensgezindheid}{van een heiligen}\\

\haiku{Maar genoeg hiervan,.}{Tobias wilde vooral}{geen spelbreker zijn}\\

\subsection{Uit: Verhalen}

\haiku{Nachtgeest IN deze.}{buurten voelde de geest zich}{onvrij en bedrukt}\\

\haiku{Ja, zoodra die.}{ongedurige geest de}{overhand behield}\\

\haiku{Wanneer hij dus op!}{den duur niet aan zich zelf als}{koning gelooven kon}\\

\haiku{{\textquoteleft}Ha, ha, ziet daar ons,{\textquoteright}.}{volk vervolgde de eenzaam}{overgeblevene}\\

\haiku{Hij lachte er om.}{tot hem de tranen uit de}{zwarte oogen rolden}\\

\haiku{Dat stond als zware.}{bedreiging tegen den zoo}{zachten avond-val}\\

\haiku{Maar is dan ieder!}{gevoel voor recht bereids in}{jelui verstorven}\\

\haiku{Soms werd het even stil.}{maar zonder overgang tot het}{bevrijdend afscheid}\\

\haiku{Hij had gedaan wat,......}{anderen deden getracht}{zich aan te passen}\\

\haiku{Was hij soms toen in?}{gelijkmatigheid en rust}{zich zelve geweest}\\

\haiku{Hoe stellig wist hij!}{thans wat deze oorlog voor}{zijn bewustzijn was}\\

\haiku{Een gezondere;}{buiten-kleur begon bij}{haar op te komen}\\

\haiku{Haar te verlaten,.}{heette opnieuw alles op}{het spel te zetten}\\

\haiku{Overal om hen was.}{gedempt gepraat als in een}{besloten ruimte}\\

\haiku{Ik moest mij in een.}{dicht gewoel van haastige}{menschen bevinden}\\

\haiku{Een lauwbedwelmend.}{mengsel van slechte parfums}{vervulde de zaal}\\

\haiku{Nog zelden had ik.}{mij van deze omgeving}{zoozeer vervreemd gevoeld}\\

\haiku{Mijn vader kon het.}{late feestmaal blijkbaar al}{evenmin verdragen}\\

\haiku{Zou hij haar opnieuw?}{bereid vinden van meet af}{aan te beginnen}\\

\haiku{{\textquoteleft}Maar nee, dat kon toch,.}{niet zoo vol als het kleine}{ding met vlooien zat}\\

\haiku{Het was de herberg,.}{enkele schreden van de}{hoeve verwijderd}\\

\haiku{Want dit hier was iets,.}{nieuws waaraan zij bereids geen}{deel meer nemen kon}\\

\haiku{Van den hevigen.}{regenval was bijna niets}{meer te bemerken}\\

\haiku{De naakte aarde.}{om de hut was hier en daar}{nog zwart en drassig}\\

\haiku{Toch kon hij er niet.}{toe besluiten met Nelly}{naar binnen te gaan}\\

\haiku{Maar vervloekt, daar moest.}{iemand met zijn pooten aan}{gezeten hebben}\\

\haiku{Door Verkade op,.}{sokken begroet steeg het}{gemeente-hoofd uit}\\

\haiku{Achteruit tredend.}{kwam hij tusschen een heer en}{een dame te staan}\\

\haiku{Waren de laatste?}{jaren daarginds niet immer}{zonder haar geweest}\\

\haiku{{\textquoteright} Johanna volgde,.}{hem in het portaaltje nog}{immer fluisterend}\\

\haiku{Van Lier, met wien hij.}{een jaar lang in de groote stad}{had samengewoond}\\

\haiku{Ook de voorstelling.}{in de zaal gebeurde zoo}{als iets bijkomstigs}\\

\haiku{Hij wist dat hare.}{gedachten immer om hem}{toefden en ook thans}\\

\haiku{Hoe zou hij alleen?}{nog de kracht vinden met dat}{verval te breken}\\

\haiku{En schier onbewust.}{had zijn verlangen hem aan}{zee teruggevoerd}\\

\haiku{Bijna had hij nog.}{ruzie met den man van de}{juffrouw gekregen}\\

\haiku{Wanneer hij den trein!}{miste en ditzelfde nog}{eens moest beginnen}\\

\haiku{Of ja, hij zou van.}{het station terugkeeren}{en haar verrassen}\\

\subsection{Uit: Willem Mertens' levensspiegel}

\haiku{Het besef eener.}{lichamelijke ikheid}{sprak bijna niet aan}\\

\haiku{Nog noodde, met een,.}{drieste hoofdbeweging zij}{hem na te komen}\\

\haiku{In den aanblik van.}{het bewegelijk stadsbeeld}{kwam iets ongekends}\\

\haiku{Het was de doffe,.}{nadreun van een doffen slag}{die alles velde}\\

\haiku{Hij was een dwaas te,.}{meenen dat zij nu eveneens}{naar hem verlangde}\\

\haiku{hij plots de moeder,,, {\textquoteleft},{\textquoteright}.}{die hem zacht werendgekke}{vreemde jongen zei}\\

\haiku{Het was voorbij en.}{weldra zou een ander hier}{zijn plaats vervullen}\\

\haiku{Toen borg hij het met.}{een schouderophalen in}{zijn portefeuille}\\

\haiku{Na het middageten.}{had hij de laatste weken}{meestal hoofdpijn}\\

\haiku{naar binnen en  .}{liet het orgel spelen om}{den schijn te redden}\\

\haiku{Ze zaten in de.}{warme huiskamer in den}{zachten lampeschijn}\\

\haiku{{\textquoteleft}Ja, en als jelui,.}{me niet helpen kom ik in}{de gevangenis}\\

\haiku{{\textquoteleft}Wel donders{\textquoteright} dacht hij {\textquoteleft}?}{hebben de dingen zich dan}{op hun kop gesteld}\\

\haiku{Toen rekte hij zich,,.}{uit en zocht zich binnensmonds}{pratend een schoon boord}\\

\haiku{Geen sterveling in,.}{geen enkelen tijd die het}{niet beweenen zou}\\

\haiku{Hij lag in bed met.}{de brandende lamp op het}{nachtkastje naast hem}\\

\haiku{Dat had alleen de,.}{tante die alle brieven}{zelf naar de post bracht}\\

\haiku{Het was er als een,.}{kippenhok waarin men een}{steen geworpen had}\\

\haiku{Hij snikte in het}{duister van radeloozen angst}{en de oogen puilden}\\

\chapter[13 auteurs, 973 haiku's]{dertien auteurs, negenhonderddrieënzeventig haiku's}

\section{Willem Paap}

\subsection{Uit: De doodsklok van het Damrak}

\haiku{Hy  zal sneller,.}{loopen opdat hy gauw aan}{de Rozengracht is}\\

\haiku{De storm, dien hy recht,;}{in het gezicht had gehad}{woei nu van terzy}\\

\haiku{Verbeeld ik my dat,?}{of ziet u er niet zoo}{goed uit als anders}\\

\haiku{Wie wat, ervaring,!}{in het leven heeft zal dat}{niet tegenspreken}\\

\haiku{dan zouden ze eerst.}{maar eens de overjassen aan}{den stander hangen}\\

\haiku{Voorheen hadden zy,.}{steeds dokter Arends gehad maar}{die was overleden}\\

\haiku{- Jy doet hier altyd;}{je best om je klantjes by}{mekaar te houden}\\

\haiku{die verdobbelt z'n,.}{geld aan de beurs maar aan z'n}{huizen doet hy niets}\\

\haiku{Kyk jy eens in die,,.}{kast en jy op de kast ik}{zelf kyk er onder}\\

\haiku{Corbelyn viel snel,,.}{op zyn boterham aan want}{hy had haast groote haast}\\

\haiku{Baby had altyd.}{over Emmy en Corry het}{moedertje gespeeld}\\

\haiku{de voorkamer, die,;}{salon werd genoemd met drie}{ramen op de gracht}\\

\haiku{Doch reeds kwam Thilde:}{achter in de gang uit de}{huiskamer en riep}\\

\haiku{- Wel beste meid, maar.}{gauw je mantel af en maar}{gauw in de kamer}\\

\haiku{Laat je bedienen, '.}{door de dames waar jes}{avonds je tyd doorbrengt}\\

\haiku{Want de naam doet er,;}{veel toe by een papier dat}{je aan de markt brengt}\\

\haiku{- Nou, hy dankte de.}{heeren voorloopig al}{wel voor de moeite}\\

\haiku{Als Schibaieffs van,.}{middag niet aan de beurs zyn}{zorg dan voor een koers}\\

\haiku{Hy sluit, gelyk men,.}{dat noemt de orders van zyn}{klanten in mekaar}\\

\haiku{De Waert enz., op de,.}{kantoren waar ze gekocht}{hebben gezegd wordt}\\

\haiku{Nu, dat was morgen,.}{verkoopen en wat er te}{kort kwam bypassen}\\

\haiku{- Want als je wilt, had,.}{hy gezegd kan je dat niet}{veel moeite kosten}\\

\haiku{je zelf zou het niet,;}{hebben gewaagd die dingen}{te laten maken}\\

\haiku{wat hy daarover zoo,,,.}{eens vertelde was gelyk}{spreekt slechts oppervlak}\\

\haiku{ik kan me dat best -.}{begrypen toen heeft hy de}{tweede genomen}\\

\haiku{het zonnetje schynt,?}{zoo lekker buiten maar als}{het u te koud is}\\

\haiku{Want 'n meisje zoo,.}{haar leven lang alleen dat}{was niet het rechte}\\

\haiku{Een jongen van een,,:}{veertien jaar kwam haar tegen}{liep dicht langs haar zei}\\

\haiku{maar je kunt toch wel,.}{zoo nu en dan doen alsof}{zy aanwezig is}\\

\haiku{- Ja, m'nheer! - En heb,?}{ik het mis of leven je}{beide ouders nog}\\

\haiku{want Alida lag een.}{verdieping hooger op de}{slaapkamer te bed}\\

\haiku{God, het is zoo gek,,.}{zei-d-i maar daar leef je}{heelemaal van op}\\

\haiku{Voor het eerst in zyn.}{leven zat hy op de weeke}{kussens der weelde}\\

\haiku{Nu, zoo vlug zullen.}{we met de bankjes toch maar}{niet om ons gooien}\\

\haiku{dat zou-d-i wel, ';}{eens willen zien dat-i}{t niet terug kreeg}\\

\haiku{Want als je doorgaat,, ';}{dat is bekend genoeg dan}{wordt hetn hartstocht}\\

\haiku{daar komt de stroom van,;}{beursheeren de Kalverstraat}{uit het Rokin af}\\

\haiku{Destyds woonde in.}{het huis van Dries Corbelyn}{de vreugde niet meer}\\

\haiku{die Unions, die nu,;}{124 staan stonden toch nog geen}{jaar geleden 190}\\

\haiku{De gongslag in de.}{rotonde verkondigde}{het aangaan der beurs}\\

\haiku{Het gemurmel wordt.}{stil daar by den hoek van de}{Santa Parulla's}\\

\haiku{zy duizelde, zy,,.}{kon niet over dien weg die daar}{zoo breed zoo hol was}\\

\haiku{En zy grepen haar.}{by den arm en gingen met}{haar terug naar huis}\\

\haiku{En dat, terwyl hy.}{het zoo benauwd had met dat}{kloppen van het hart}\\

\haiku{'t Was hem, of hy,,.}{weg moest van elke plaats waar}{hy zat waar hy stond}\\

\haiku{Pas op, niets merken,,!}{laten hier in huis houd je}{goed houd je kranig}\\

\haiku{Nou de Kimberleys,!}{niks meer waard zyn ben je naar}{de verdommenis}\\

\haiku{Waarom had zyn vrouw?}{hem aangepord om aan de}{beurs te speculeeren}\\

\haiku{En Geurtje bracht het.}{theewater en zette het}{theeblad op tafel}\\

\haiku{En Dries Corbelyn.}{liep wild heen en weer als een}{roofdier in zyn kooi}\\

\haiku{want, duivels, het had, '.}{hem toch aangegrepen hy}{wasn beetje moe}\\

\haiku{Iets vroolyker was het.}{in de woonkamer van Arie}{Zuydam en Alida}\\

\haiku{Zyn advocaat had,,,}{hem gezegd dat nu hy toch}{niets van haar hoorde}\\

\subsection{Uit: Vincent Haman}

\haiku{In der Beschr\"ankung,:}{zeigt sich der Meister en}{zei niet 			 Hegel}\\

\haiku{Kyken hoe iets in, '.}{elkander 			 zit moet hem}{n pleizier worden}\\

\haiku{- Wel, als het korter,;}{was dan zond ik het aan de}{Amstelbode}\\

\haiku{{\textquoteleft}'t Is heel knap van{\textquoteright},:}{zoo'n kind en oom 			 Gabriel}{Haman beweerde}\\

\haiku{de kranten, 			 Buckle.}{en de brochure van een}{dubbeltje}\\

\haiku{Hy heeft je stuk 			 .}{gelezen en zegt dat er}{goeie zinnen in zyn}\\

\haiku{Dit nu is zeer mooi,}{maar een auteur die niets te}{zeggen heeft dan}\\

\haiku{'k Weet immers wel,.}{dat u die niet voor me}{zou willen koopen}\\

\haiku{Met een sarrige.}{treiterlust zat hy het werk}{te concipieeren}\\

\haiku{Haar 			 vader vond ', '.}{hemn deugniet dus was het}{n flinke jongen}\\

\haiku{hy zou haar in z'n,}{proza precies volgen de}{houtsnee alleen wat}\\

\haiku{Esther lachte 			 :}{haar blymoedigen lach op}{het opene gelaat}\\

\haiku{Hy streek zich met de,.}{linkerhand over het voorhoofd}{ging zwygend 			 mee}\\

\haiku{Wacht, die brief in de,...}{zak 			 steken mama kon}{eens binnen komen}\\

\haiku{En 			 bovendien,,.}{och zeuren hoe kon ze over}{die drift zaniken}\\

\haiku{Marie zei dat er,;}{iets heel byzonders moest zyn}{w\'at wist ze 			 niet}\\

\haiku{Maar er kwam geen brief,,;}{dien wachtenden dag niet}{den volgenden niet}\\

\haiku{Zy had 'n rytuig '.}{genomen enn krans}{gebracht naar het graf}\\

\haiku{'t Was een achttal,.}{jaren geleden nu dat}{Vincent gegaan was}\\

\haiku{Een jaar later was,.}{haar moeder gestorven kort}{daarna haar vader}\\

\haiku{zoo teer als ik die,.}{tonen 			 wilde hebben}{kryg ik ze toch niet}\\

\haiku{Hoog en rank op haar.}{elegante Columbia zat}{de slanke Esther}\\

\haiku{die 's altyd zoo -;}{zwaar op de hand. Schimp nou niet}{meer zoo op Reinhold}\\

\haiku{- Kom, zei Esther, een,.}{weinig geschokkeerd laten}{we naar huis 			 gaan}\\

\haiku{Zy waren Halfweg,,}{nu lang voorby gingen den}{tol door en v\'oor}\\

\haiku{Esther moest iets doen,.}{aan de ellende die zy}{daar voor zich zag}\\

\haiku{Zy waren zwygend,.}{gegaan en ook zittende}{zwegen zy een wyl}\\

\haiku{Een notaris uit;}{Den Haag stuurde 			 wel eens}{geld per postwissel}\\

\haiku{Maar 'n paar maanden ';}{later was er 			 weern}{vonnis gekomen}\\

\haiku{{\textquoteleft}God, Puck wat wor je{\textquoteright},.}{zwaar zette 			 hem op het}{laken van het bed}\\

\haiku{Vroeger kwam Van den,,.}{Berg er veel 			 later Van}{Wheele nu Jules}\\

\haiku{In eens bleven ze,}{dan gewoonlyk weg zonder}{dat je 			 hoorde}\\

\haiku{Esther slank in haar,,.}{eenvoudig grys Marie}{kleiner wat dikjes}\\

\haiku{Wat ik je vragen,,?}{wou 			 Vincent heb je nog}{over het tydschrift gedacht}\\

\haiku{Witsen wist altyd,;}{precies het moment te}{pakken mooi van toon}\\

\haiku{'t is half tien, en.}{ze hadden hier 			 om acht}{uur zullen wezen}\\

\haiku{Wat spyt me dat, wat,...}{spyt me dat dat hy daar nu}{weer mee 			 begint}\\

\haiku{Nu gaat ook het licht;}{van dat hooge raam daar ver aan}{den overkant 			 uit}\\

\haiku{Als ik wist dat hy,...}{alleen was om aan my te}{kunnen denken}\\

\haiku{Beroerd idee, ik 			  '....}{wilt niet meer hebben Laat}{ik wat rondkyken}\\

\haiku{Vincent, onder die,;}{woorden was een weinig in}{pose gekomen}\\

\haiku{Ik hou van 'r. Zy,,.}{heeft mooie oogen zoo raar grys}{grauw zou je zeggen}\\

\haiku{Toen hy een wyle,,.}{gelezen had stond hy op}{liep in de kamer}\\

\haiku{October ging heen,.}{November kilde in de}{gangen der huizen}\\

\haiku{Esther, dien zomer,,.}{en herfst was wat nerveus zooals}{zy het noemde}\\

\haiku{zy had hem 			 niet.}{moeten ontvangen daar in}{dat schemerend uur}\\

\haiku{zei weer Marie, en,}{lei de 			 hand op den}{schouder van Esther}\\

\haiku{wat 			 zat ze nu ':}{hier byt ontbyt haar tyd}{te verbeuzelen}\\

\haiku{Dat was voor Esther,,.}{by haar schilderen 			 weer}{een heele drukte}\\

\haiku{'k 			 Heb vandaag,.}{behoorlyk gewerkt en mag}{nu wel wat praten}\\

\haiku{Daar had-i nou z\'oo,.}{op gehoopt dat-i weer}{zou gaan 			 schryven}\\

\haiku{Vincent vond het w\'at.}{prettig dat er zoo over hem}{geschreven 			 werd}\\

\haiku{Wel, dat 			 zou iets, '.}{nieuws zyn zoo eens iets uitn}{heel andere sfeer}\\

\haiku{Maar 			 nu, 't was '.}{enorm zoo veel als-i soms schreef}{opn voormiddag}\\

\haiku{Maar de man had nu '.}{werkelykn 			 heel goed}{inzicht in de zaak}\\

\haiku{Die heftigheid van, ',;}{ons van vroeger dat wast}{jonge 			 bloed oom}\\

\haiku{Wel, dat vond Vincent ', '.}{n belangryk ideen zeer}{belangryk idee}\\

\haiku{En 'n predikant, ', ';}{ook inn toga staat heel}{goedn 		  toga}\\

\haiku{- Och, je weet toch wel,.}{dat ik daar tegenwoordig}{zoo 			 niet meer kom}\\

\haiku{zou   moeten leeren,...}{zou het by hem toch ook zoo}{vlug niet meer 		  gaan}\\

\haiku{Ze moest niet denken,,;}{had-i gezeid dat het niet}{precies waar was}\\

\haiku{Zoo zaten zy daar.}{vredig by den komenden}{September-avond}\\

\haiku{'t Was wel geen slot,.}{maar het heette dan toch}{het   Vondelhuis}\\

\haiku{Pag. 170, regel 14-,.}{21 is citaat uit diens}{Huysman's L\`a-Bas}\\

\section{Gerrit Paape}

\subsection{Uit: Reize door het aapenland (onder pseudoniem J.A. Schasz)}

\haiku{Bijgekomen, duwt.}{hij haar met kracht van zich af}{en vlucht ijlings weg}\\

\haiku{Voorlopig staan de '.}{papieren van Pietert}{Hoen dus nog het sterkst}\\

\haiku{Had hij intussen?}{zijn bekomst gekregen van}{de revolutie}\\

\haiku{De reiziger leert.}{de apentaal en loopt ook op}{handen en voeten}\\

\haiku{In drie bedrijven,,,,,.}{door J.A. Schasz M.D. Utrecht bij G.T.}{van Paddenburg 1779}\\

\haiku{hoe is het mooglijk,?}{dat ik   mijn arme Paard}{kan vergeeten}\\

\haiku{en heeft het zelf wel,?}{ooit gedagten gemaakt om}{op hol te   gaan}\\

\haiku{Dan zou het er slegt,}{met hem uitzien wanneer hij}{gekreegen werd?129}\\

\haiku{- en gelust ons den,;}{Aapenstaat   dan is de staart}{volstrekt on\"ontbeerlijk}\\

\haiku{In 't eerst, werd het.}{geval wonderbaarlijk door}{een gehaspeld}\\

\haiku{\'e\'en ten hoogsten, doch.}{keurde dat van nommer}{vijf ten sterksten af}\\

\haiku{Gij moet er door uw!.}{eigen oogen van overtuigd}{worden zei hij}\\

\haiku{dagt ik, deeze zijn!}{gekomen,232  om mijn Staart}{aftehakken}\\

\haiku{Maar wij zijn hier met!}{ons zo veel wijze Mannen}{bij elkander}\\

\haiku{Het gevolg van dit, '.}{alles was   een zundvloed}{overt Aapenland}\\

\haiku{want, zei   hij, gij,.}{zegt toch dat het volmaakt met}{mijn concept strookt}\\

\haiku{Ten minsten waren:}{negen tiende deelen voor}{de afkapping}\\

\haiku{gij weeten, dat elk;}{Mensch een goede en   een}{kwaade zijde heeft}\\

\haiku{Veelen hunner:}{hadden dien gan-  schen}{nagt niet kunnen slaapen}\\

\haiku{wordende elke.}{kring geduurig honderd}{Aapen talrijker}\\

\haiku{Nommer 17  En.}{al uw Mede\"aapen zijn er}{bijna om koud}\\

\haiku{van de Reize door (.}{het Aapenland zijn aanwezig}{in KB Den Haagsign}\\

\haiku{Zie Mia I. Gerhardt,, (),-:}{Two Wayfarers UtrechtU.P.A.L. nr.}{5 1964 p. 2026}\\

\haiku{A study in the, (.).}{history of an idea3}{CambridgeMass 1948}\\

\haiku{allusie op de.}{partijnamen Orangisten}{en Patriotten}\\

\section{Jan van Panders}

\subsection{Uit: Kroniek van Alkmaar in de Bataafs-Franse tijd, 1787-1797}

\haiku{Dit mij ten vollen.}{verzeekerd door juffrouw}{Van der Slot aldaar}\\

\haiku{Verscheide van hun.}{hadden tevooren voor zwaar}{patriot gespeeld}\\

\haiku{De thermomeeter.}{teekende deze morgen}{4 graaden onder 0}\\

\haiku{1 maart In deze}{middag kwamen weder hier}{in van Purmerend}\\

\haiku{Dit dankuur werd alhier}{bij alle gezindheedens}{gehouden en schoon}\\

\haiku{(Ik heb niet kunnen.)}{goedvinden om dit request}{te teekenen}\\

\haiku{Den 28 dito heeft '.}{het klokgelui voort eerst}{opgehouden}\\

\section{Willem Pijper}

\subsection{Uit: Het papieren gevaar. Verzamelde geschriften (1917-1947) (3 delen)}

\haiku{Het is de echo van:}{wat hij in 1930 al schreef over}{Pijpers teksten}\\

\haiku{Het moet hem vele.}{maanden voorbereidingstijd}{hebben gekost}\\

\haiku{Pijpers uitspraak {\textquoteleft}de.}{muziek is een volkomen}{nutteloze zaak}\\

\haiku{In dat geval wordt:}{echte kunst geschapen met}{eeuwigheidswaarde}\\

\haiku{Toen ik kwam, had ik.}{Johan Rasch en Meyer Drukker}{als concertmeesters}\\

\haiku{men miste hier te.}{zeer contrastwerking.{\textquoteright}95 In Utrecht}{was het niet anders}\\

\haiku{Daarbij ben ik zelfs,,.}{nu nog in twijfel of ge}{vals bewust vals zijt}\\

\haiku{hij deed alsof dit.}{concert verlopen was als}{elk ander concert}\\

\haiku{de natuur van het.}{instrument wordt dan het minst}{geweld aangedaan}\\

\haiku{Daarentegen was;}{het Adagio niet gezonken}{en niet eindeloos}\\

\haiku{Men heeft toch vroeger!}{Beethoven en Berlioz ook}{moeten doorzetten}\\

\haiku{alles werd slapper -,.}{doch niet de heer Van Gilse}{treft hier alle schuld}\\

\haiku{Over de waarde der}{cadens zou nog het een en}{ander te zeggen}\\

\haiku{Maar, wat Wagenaar,.}{met zijn Berlioz bereikt is}{in \'e\'en woord volmaakt}\\

\haiku{Ik weet het ook niet -!}{ik kan er me geen noot meer}{van herinneren}\\

\haiku{Het vertolkt alle,.}{emoties het heeft accenten}{voor elk sentiment}\\

\haiku{Ik heb haar dit en,}{veel meer nog vergeven toen}{ik bemerkte welk}\\

\haiku{Wij Utrechtenaren!}{mogen onze concertzaal}{wel in ere houden}\\

\haiku{Kubbinga, Helvoirt,.}{Pel en Van der Ploeg vielen}{mij niet mee ditmaal}\\

\haiku{En ik moet helaas,}{zeggen de heer Caro is}{er niet in geslaagd}\\

\haiku{En als zodanig.}{vind ik de suite niet meer}{of minder dan zwak}\\

\haiku{En dat zij, die door}{haar natuur het volste recht}{heeft lief te hebben}\\

\haiku{Het werk bespreek ik,:}{binnenkort nader over de}{uitvoering slechts dit}\\

\haiku{Tod und Verkl\"arung'.}{Strauss Tod und Verkl\"arung}{besloot het concert}\\

\haiku{Tegen een opera:}{van Verdi valt inderdaad}{niets in te brengen}\\

\haiku{Eerste symfonie -!}{Welk een programma en}{welk een uitvoering}\\

\haiku{misschien lukte Faur\'es () -.}{kwartet even goedof bijna}{Mozart zeker niet}\\

\haiku{In hoeverre dit,.}{juist gezien is wil ik in}{het midden laten}\\

\haiku{De herenrollen,,.}{waren dunkt mij ook zo goed}{mogelijk bezet}\\

\haiku{mijn lijstje geeft niets.}{dat in dit blad niet reeds voor}{kort werd besproken}\\

\haiku{De positie van;}{dirigent van het U.S.O. is}{een zeer moeilijke}\\

\haiku{{\textquoteleft}Doe dat nu niet, die!}{stukken hebben ze al zo}{vaak in Utrecht gehoord}\\

\haiku{Als contrapuntist;}{staat Franck geheel onder}{Bachs sterke invloed}\\

\haiku{En het tweede deel,,:}{Scherzo herinnert weer aan}{een ander liedje}\\

\haiku{een kunstwerk dat een,:}{microkosmos is waarin}{alles zijn plaats heeft}\\

\haiku{het Schone en het,.}{Onschone het Sublieme}{en het Trachtende}\\

\haiku{Comme son coeur, sa,:}{religion est toute}{amour toute piti\'e}\\

\haiku{Elke techniek heeft,.}{haar tegenheden zo ook}{natuurlijk deze}\\

\haiku{het ergste wat we -:}{zouden kunnen verwachten}{en ik voeg erbij}\\

\haiku{En bij Liszts:}{Faust-wals werd het andere}{uiterste bereikt}\\

\haiku{Haar kostbare toon;}{tovert u onvermoede}{zaligheden voor}\\

\haiku{Laat men toch de o!}{niet zo dik en grof in de}{ruimte plamuren}\\

\haiku{Hun kwartetspel is.}{de grootste tovermacht die}{ik nog onderging}\\

\haiku{Van de doorvoering,.}{noch van de terugkeer valt}{veel te verhalen}\\

\haiku{Alles wat erin,.}{gebeuren gaat is reeds van}{tevoren bepaald}\\

\haiku{Dit is juist weer een,.}{theatereffect dat men}{liever zou missen}\\

\haiku{Het orkestspel heeft.}{mij ook in veel opzichten}{genot geschonken}\\

\haiku{Scherts, lezer, is dit,!}{misschien voor u en mij maar}{niet voor iedereen}\\

\haiku{Waarom ik met de,,?}{heer Z. aanvang lezer en}{niet met miss Parlow}\\

\haiku{Meesterlijk is een,:}{zeer veelzeggend woord doch het}{kan ook beduiden}\\

\haiku{De Finale lijdt.}{ook onder die barokke}{tegenstellingen}\\

\haiku{Hoe dit nu zij, opus.}{59.1 laat nooit na een diepe}{indruk te maken}\\

\haiku{Haar travestie was,.}{meer flatteus dan waarschijnlijk}{op zijn zachtst gezegd}\\

\haiku{Het zingen van dit.}{koor was thans beter dan ik}{het wel eens hoorde}\\

\haiku{Maar hij vat de zaak.}{aan op een wijze die mij}{niet de juiste lijkt}\\

\haiku{een stemming dus in ().}{kwarten met een terts in het}{middenluitstemming}\\

\haiku{het elektriseert u,;}{niet gelijk het vioolspel}{van een Fritz Kreisler}\\

\haiku{Und bedenkt man nun {\textquotedblleft}...}{dass diese Regeln in der}{Form desDu darfst nicht}\\

\haiku{Waarom volgt ook zij,?}{het  funeste gebruik}{zwaar zwaar te spelen}\\

\haiku{De zes liederen,.}{opus 68 intoneren ook}{niets anders niets nieuws}\\

\haiku{Het Interm\`ede,.}{is zeer ironisch hier en daar}{bepaald beklemmend}\\

\haiku{Cyclusconcert    ().}{5 februari 1920UD}{Tivoli   U.S.O. o.l.v}\\

\haiku{Regers werkwijze.}{voor al zijn composities}{kunnen aantonen}\\

\haiku{de heer Petri heeft.}{Bachs grandioze fuga}{uitmuntend gespeeld}\\

\haiku{Cyclusconcert654   ().}{17 februari 1920UD}{Tivoli   U.S.O. o.l.v}\\

\haiku{Zoiets is typisch'.}{voor de mentaliteit van}{Ewers bewonderaars}\\

\haiku{Zij verkonden de, {\textquoteleft}{\textquoteright}.}{winter zij groeien slechts daar}{waarVerwesung heerst}\\

\haiku{de Vreugde (Scherzo), (), ().}{de LiefdeAdagietto de}{LevensmoedRondo}\\

\haiku{Doch het aanvaarden.}{als een uiting van ziel tot}{ziel kan ik ook niet}\\

\haiku{effectbejag - is.}{helaas manifest genoeg}{in dit Requiem}\\

\haiku{Bringen Herz und Hirn -,.}{in Not  Ruhe ruhe}{meine Seele}\\

\haiku{Waar was het gaande,?}{statig-rouwende tempo}{van het eerste deel}\\

\haiku{\ensuremath{\cup} - heeft Schubert z\'o.}{genoteerd dat de tweede}{tel het accent krijgt}\\

\haiku{Het orkest had zijn,.}{slechtste en Zimmermann had}{zijn beste avond niet}\\

\haiku{Beroemd heeft het werk,.}{hem overigens niet gemaakt}{zomin als Lecocq}\\

\haiku{Das Lied vom Kummer...}{soll auflachend in die}{Seele euch klingen}\\

\haiku{Doch, houd de mensen, {\textquoteleft}}{de kunstenaars v\'o\'or dat de}{hoogste wijsheid dit}\\

\haiku{Orgelconcert    ().}{29 september 1920UD}{Buurkerk   Dirk Corn}\\

\haiku{Niet overal speelde;}{Rijnbergens reproductie}{op het hoogste plan}\\

\haiku{een royaal werk van,.}{Hans Brandenburg getiteld}{Der moderne Tanz}\\

\haiku{Dit behoorde tot().}{het hoogste dat een zanger}{es ooit kan geven}\\

\haiku{En het is helaas.}{te zelden dat men deze}{lof neerschrijven kan}\\

\haiku{Ik wenste dat Van;}{Gilse zich aan dit nabije}{voorbeeld spiegelde}\\

\haiku{Hun timbre iriseert,.}{te zelden flonkert niet in}{vele facetten}\\

\haiku{Bezitten we niet?}{evenzeer een muziek die de}{Heldenmoed prikkelt}\\

\haiku{Thans rangschikken wij.}{het onder de Muzieken die}{de Slaap oproepen}\\

\haiku{Het is het oude:}{liedje van de lyrische}{muziekbespreking}\\

\haiku{Het gaat niet zonder,,.}{een programma van acht of}{tien of twaalf nummers}\\

\haiku{Er is geen spoor van,.}{verzwakking in dit late}{werk integendeel}\\

\haiku{Charlotte Bara's.}{grootste vergissing is dat}{zij op muziek danst}\\

\haiku{En beschouwingen ', -?}{overs meesters artiestschap}{zienerschap waartoe}\\

\haiku{Het is een daad om.}{het publiek het Heden te}{doen accepteren}\\

\haiku{De reproductie.}{van dit laatste werk was zeer}{verre van volmaakt}\\

\haiku{En technisch (de twaalf).}{delen zijn alle wel zeer}{beknopt en psychisch}\\

\haiku{Kamermuziekavond - ():}{Das Rheinische Trio   18}{maart 1921UD ~ Kaun}\\

\haiku{Doch pijnlijk is het,,;}{soms en zwaar om zichzelf te}{kunnen uitspreken}\\

\haiku{Laat uw motieven.}{juist zo terugkomen als}{ze afgerold zijn}\\

\haiku{{\textquoteleft}faire du th\'e\^atre{\textquoteright} {\textquoteleft}}{gaat nooit samen metfaire}{de la musique{\textquoteright}.986}\\

\haiku{Zij zijn verplicht om:}{hun eigen wezen in hun}{muziek te zoeken}\\

\haiku{Licht in vele, naar.}{goedvinden te rangschikken}{betekenissen}\\

\haiku{En zo met bijna,.}{alle halfgoden van het}{tweede derde plan}\\

\haiku{Zie Lalo, Piern\'e, (!),,,,.}{Ravelja Bizet Chabrier}{Chausson Charpentier}\\

\haiku{Anders staat het ook.}{al niet met mijn opinie over}{zijn Sneeuwimpressie}\\

\haiku{Het kan dit, doch het.}{d\'e\'ed dit gisteravond zonder}{uitzondering niet}\\

\haiku{De Weser Zeitung {\textquoteleft}.}{roddelt wat overEen rijpe}{jonge musicus}\\

\haiku{{\textquoteleft}Gr\"uner Pfau{\textquoteright} (die wit),:}{was of op dat der land-}{en volkenkunde}\\

\haiku{modulaties der.}{stemmingen stonden open bij}{iedere maatstreep}\\

\haiku{{\textquoteleft}Als-ie er wat van?}{wist zou-d-ie ommers niet}{in de kr\'ant schrijven}\\

\haiku{Maar haal er nu de,,!}{muziek niet bij laat de kunst}{met rust ik bid u}\\

\haiku{de melancholie...}{dezer muziek bleven zelfs}{niet te vermoeden}\\

\haiku{Maar zijn Schelomo,,.}{een sterk een persoonlijk werk}{noemen mag ik niet}\\

\haiku{Het moet lokale,.}{politiek worden voor het}{interessant is}\\

\haiku{Welke gedachte:}{verstopt zich achter een zin}{als de volgende}\\

\haiku{Een omstandigheid,.}{buiten mijn wil buiten haar}{schuld waarschijnlijk ook}\\

\haiku{Ik zeg overigens -.}{niet dat ik dit een gemis}{vind integendeel}\\

\haiku{er bleven niet veel.}{landen ter wereld over die}{hij niet bezocht heeft}\\

\haiku{Tot zijn werken uit:}{de laatste jaren van zijn}{leven behoren}\\

\haiku{Zij heeft echter een,:}{niet zeer omvangrijke noch}{klankvolle mezzo}\\

\haiku{[...] Meine Symphonie,!}{wird etwas sein was die Welt}{noch nicht geh\"ort hat}\\

\haiku{Merk op dat hij \'o\'ok -:}{last had van de drie B's doch}{bij h\'em heetten ze}\\

\haiku{De heer Van Gilse.}{vermoedt uitsluitend uit vrees}{voor het Utrechtsch Dagblad}\\

\haiku{Het Utrechtse orkest () [}{woensdag 22 december}{1921Het Vaderland}\\

\haiku{Het is de mening.}{van de heer Pijper dat hij}{het niet heeft geleerd}\\

\haiku{Holland is w\'el het.}{voorbeschikte land voor de}{Begrafenissen}\\

\haiku{En ik meen dat de.}{muziek ouder rechten heeft}{dan het Commentaar}\\

\haiku{De muziek, ook hier,.}{in Holland iriseert reeds in}{andere sferen}\\

\haiku{Veel extase zal,.}{wellicht verbleekt blijken veel}{rust zou trekvoeten}\\

\haiku{Doch wat men w\'el, na,,:}{vijf minuten luisteren}{kon vaststellen is}\\

\haiku{Het deed denken aan';}{Berlioz land der muzikaal}{gelukzaligen}\\

\haiku{{\textquoteleft}Auferstehn, ja,,{\textquoteright}.}{auferstehn wirst du mein}{Staub nach kurzer Ruh}\\

\haiku{met meer Geloof (dat).}{allicht een klein heuveltje}{verzet dan Inzicht}\\

\haiku{Klanglich sch\"on{\textquoteright} (nu, nu!).}{en de houtblazers hebben}{er veel in te doen}\\

\haiku{Zulke foutjes zijn -.}{gauw gemaakt gelukkig even}{spoedig verbeterd}\\

\haiku{En dit gevaar had.}{Martine Dhont kunnen en}{moeten vermijden}\\

\haiku{Koene speelde het.}{concert beter dan ik het}{nog van hem hoorde}\\

\haiku{De muzikale:}{waarden lijken mij daarvoor}{ook hier te gering}\\

\haiku{Dit is het beste,.}{het persoonlijkste fragment}{van het Requiem}\\

\haiku{Zijn Beethoven voelt,,.}{hij niet stijf niet dogmatisch}{niet puriteins aan}\\

\haiku{Men heeft gisteravond:}{een zijner meest geslaagde}{liederen gehoord}\\

\haiku{iets anders is dan.}{het elkander te woord staan}{van twee musici}\\

\haiku{Maar de vraag die wij,:}{ons vandaag in de eerste}{plaats  stelden is}\\

\haiku{We hebben vandaag ();}{8 augustus ons eerste}{schandaaltje gehad}\\

\haiku{Waarom heeft hij dan...?}{ook geen An der sch\"onen blauen}{Donau geschreven}\\

\haiku{Het komma-punt:}{van uitgang onzer scepsis}{is het volgende}\\

\haiku{Een nocturne van.}{Chopin voor viool en}{piano eveneens}\\

\haiku{Een symfonie van.}{Bruckner voor piano \`a}{quatre mains evenzeer}\\

\haiku{Ik vraag een ogenblik...}{aandacht voor de portee der}{vier adjectieven}\\

\haiku{Frankrijk-Itali\"e,:}{Oostenrijk-Itali\"e zich}{ook voortsponnen tot}\\

\haiku{Maar zover zal het;}{met de Nationale}{Opera niet komen}\\

\haiku{De voorstelling werd,,.}{nog iets nieuws gedirigeerd}{door Rudolf Tissor}\\

\haiku{Hij blijkt een onzer.}{meest talentvolle jonge}{violisten}\\

\haiku{En ook het laatste.}{programmapunt bevatte}{levende namen}\\

\haiku{Er valt redelijk:}{vrij weinig op dit muziek}{maken te zeggen}\\

\haiku{haar inzicht zou zich.}{aangepast hebben aan de}{muziek van vandaag}\\

\haiku{te zeggen heeft) dan!}{Debussy's of Bachs Myra}{Hess stellig zou doen}\\

\haiku{De mannenrollen:}{waren beter bezet dan}{de vrouwenrollen}\\

\haiku{De tijdgenoten.}{van Moesorgski misten het}{gevoel voor zijn kunst}\\

\haiku{we gooien net zo;}{lang met het mooiste speelgoed}{tot het kapot is}\\

\haiku{Proberen - hij heeft (...).}{daarvoor te veel talentniet}{kwaadaardig bedoeld}\\

\haiku{zijn werk zal daar, hoop,.}{ik op den duur belangrijk}{genoeg voor worden}\\

\haiku{Tivoli-concert - Jan ().}{Dekker   2 november}{1922UD ~ U.S.O. o.l.v}\\

\haiku{Doch dit is wellicht.}{een atavisme waarvoor ik}{mij te schamen heb}\\

\haiku{Men kan er gewis,:}{een Wagneriaans klinkend}{stuk van maken doch}\\

\haiku{in 1922 lijken Brahms.}{en Wagner ternauwernood}{nog tijdgenoten}\\

\haiku{de composities.}{van zijn leermeesters Bart\'ok B\'ela}{en Kod\'aly Zolt\'an}\\

\haiku{Ook in de Etude.}{van Chopin waren zeer}{schone ogenblikken}\\

\haiku{Expressie volgens {\textquoteleft}{\textquoteright},:.}{mijn mening wat overdreven}{mild dus meewarig}\\

\haiku{{\textquoteright} De Telegraaf schijnt,,:}{het sinds een paar jaar z\'o te}{interpreteren}\\

\haiku{de kunstrubriek van:}{De Telegraaf voorlopig}{niet meer te lezen}\\

\haiku{Deze ambrozijn -!}{was een godenvoedsel maar}{in Beethovens tijd}\\

\haiku{Scherts, van meester X,,.}{Y en Z gecompileerd}{door Hans Pfitzner}\\

\haiku{Wat men in Holland.}{nimmer was en hopelijk}{ook nooit zal worden}\\

\haiku{voorstellen en ik,,.}{heb dat zelfs in een opera}{vaak genoeg gehoord}\\

\haiku{{\textquoteleft}Die ik de laatste (){\textquoteright},.}{keer gehoordof gespeeld heb}{was nog zo dwaas niet}\\

\haiku{Ik durf te zeggen.}{dat wij de Roussel van 1920}{hier nog niet kennen}\\

\haiku{Op die wijze is,.}{Roussels vioolsonate}{te lang geloof ik}\\

\haiku{Hinderen deed de.}{tamboerijn in abstracto}{overigens geen mens}\\

\haiku{Zo niet - poenitet).}{me peccasse monotoon}{en fantasieloos}\\

\haiku{Er waren ook wat ().}{onpreciesheden in de}{strijkerseerste deel}\\

\haiku{Bazel, of Z\"urich,,,.}{of Gen\`eve zijn dat op dit}{ogenblik althans niet}\\

\haiku{De uitspraak van het.}{koor is niet zeer zuiver en}{men zet wat traag in}\\

\haiku{Het is niet praktisch.}{vijf mensen een taak van twee}{toe te vertrouwen}\\

\haiku{Die verhouding zou.}{wel eens bij benadering}{vast te stellen zijn}\\

\haiku{ook de beeldstormer,.}{destructeur heeft soms recht op}{een herinnering}\\

\haiku{Zo hakt men bomen,.}{met een schaaf snijdt camee\"en}{met een koubeitel}\\

\haiku{Ik vind in zijn werk.}{geen kiemen van een zelfs maar}{korte eeuwigheid}\\

\haiku{Een lief dialoogje,,.}{een dito muziekje een}{dito aankleding}\\

\haiku{Ik voor mij, ik houd,.}{wel van zulk soort dubbel het}{animeert tenminste}\\

\haiku{Het is op twee na.}{de bekendste van Schuberts}{acht symfonie\"en}\\

\haiku{Das M\"archen von,.}{der sch\"onen Melusine}{van Heinrich Hofmann}\\

\haiku{Zijn muziek is op.}{dit ogenblik in Holland nog}{vrijwel onbekend}\\

\haiku{Ook in het vierde,,.}{deel staan welbewust enige}{italianismen}\\

\haiku{Chopin, Liszt),:}{honderdmaal veelzijdiger}{gefacetteerd dus}\\

\haiku{Natuurlijk klinkt Les - []:}{soir\'ees de P\'etrograde}{anders zeggenwij}\\

\haiku{En in de tweede:}{relatie overtreft Schumann}{Brahms vele malen}\\

\haiku{Die keerzijde bleef,,.}{bij Stravinsky tot \ensuremath{\pm} 1920}{erg onafgewerkt}\\

\haiku{Dit toch is het doel.}{der kunstbewerking genaamd}{psychoanalyse}\\

\haiku{Kod\'aly's werken zijn;}{hier tot dusverre bijna}{niet doorgedrongen}\\

\haiku{Het vrouwenkoor van.}{Toonkunst bevat overigens}{goed materiaal}\\

\haiku{En men kan dan zelfs}{tot conclusies komen die}{hier en daar lijnrecht}\\

\haiku{Bezinningen - en.}{te weinig elans werden tot}{autoprojecties}\\

\haiku{De reien bleven,.}{de hoogtepunten ook van}{deze voorstelling}\\

\haiku{Die eerste strofen,.}{die een verhaspelde Van}{Eeden voorstellen}\\

\haiku{En dus eigenlijk.}{al geantiquiseerd v\'o\'or}{het geschreven was}\\

\haiku{dat do\'et het leven.}{in Europa van onze}{dagen ook niet meer}\\

\haiku{En Der Abschied heet.}{het beste fragment van Das}{Lied von der Erde}\\

\haiku{Het is pijnlijker.}{hem te moeten verwerpen}{dan Strauss of Reger}\\

\haiku{Dat de Eerste een,.}{Volgelingenwerk is kan}{het stuk niet schaden}\\

\haiku{Spaanderman schijnt (zie);}{ook zijn dirigeren geen}{geboren leider}\\

\haiku{{\textquoteright}1782 ~ Ziehier een.}{gloednieuwe definitie}{van het contrapunt}\\

\haiku{de Debussynse,:}{Pastorale voor zang en}{piano van 1906}\\

\haiku{Praktisch zal men daar,.}{dus nimmer voorbeelden van}{vinden vermoed ik}\\

\haiku{Het verschijnsel op.}{zichzelf behoeft ons dus niet}{te verontrusten}\\

\haiku{die er genoeg van,}{hebben Zimmermann altijd}{links vooraan te zien}\\

\haiku{{\textquoteleft}Met hoeveel fluiten?}{wilt u het concert van Bach}{uitgevoerd hebben}\\

\haiku{{\textquoteright} Het is gelukkig.}{alleen voor de schrijver van}{dat stukje maar waar}\\

\haiku{En men kon dus de.}{muzikale factoren}{laten beslissen}\\

\haiku{Laten wij maar niet;}{op nieuwe openbaringen}{gaan zitten wachten}\\

\haiku{Dit is constructief,.}{beter dan Ferroud het}{is zelfs scholastisch}\\

\haiku{{\textquoteright} In het origineel {\textquoteleft},,.}{staatJe t'aimerai Seigneur}{d'un amour tendre}\\

\haiku{Misschien heeft hij het?}{thema bij ongeluk in}{kreeftengang gebracht}\\

\haiku{Maar er ligt weinig:}{belofte voor de toekomst}{in opgesloten}\\

\haiku{Zij schijnen daar voor:}{rituele doeleinden}{gebruikt te worden}\\

\haiku{Pas na de oorlog.}{waagde Sch\"onberg zich weer aan}{de compositie}\\

\haiku{Doch juist hier ligt het.}{verschil tussen hem en de}{beide genoemden}\\

\haiku{Zijn motivering,:}{was een muzikale de}{andere waren}\\

\haiku{De opera stelt nu.}{eenmaal andere eisen}{dan de concertzaal}\\

\haiku{zijn constructies zijn'.}{uitbreidingen van Stamitz}{en Mozarts structuur}\\

\haiku{Maar de kunstenaar,.}{van deze van onze tijd}{gedraagt zich anders}\\

\haiku{\'e\'en meesterwerk, de,;}{Sonate voor fluit alt en}{harp van Debussy}\\

\haiku{Een stuk als Saul en.}{David heeft niemand hem hier}{nog nagemaakt}\\

\haiku{Melodisch blijft het.}{wat vlak en er zijn een paar}{doffe plekken in}\\

\haiku{in 1909 bestond de.}{Pell\'eas van Debussy}{reeds bijna tien jaar}\\

\haiku{te veel wildheid nog,.}{te veel oppervlakkige}{scherts en ontroering}\\

\haiku{Want Sch\"onberg was de,;}{dogmaticus de louter}{destructieve geest}\\

\haiku{Weberns F\"unf S\"atze -.}{zijn hier voor de eerste maal}{gespeeld tweemaal zelfs}\\

\haiku{het aambeeld, de man, -.}{het beest laat zich natuurlijk}{wel analyseren}\\

\haiku{Hoe vaak brengen de?}{internationale}{virtuozen nieuws}\\

\haiku{Maar het lijkt mij dat.}{gij aan het jazzsymptoom veel te}{veel waarde toekent}\\

\haiku{Dit lijkt mij, in beeld,.}{gebracht de zaak die wij aan}{de orde stelden}\\

\haiku{Maar in 1784 had hij ().}{reeds een Pianoconcert}{in Es geschreven}\\

\haiku{Dit orkest is een.}{buitengewoon subtiel en}{willig apparaat}\\

\haiku{Maar Berlioz ervoer,.}{altijd alles anekdotisch}{buiten samenhang}\\

\haiku{Eerste en Derde,.}{symfonie Ouverture}{Coriolanus}\\

\haiku{De tijden voor de.}{Nederlandse muziek zijn}{gunstiger dan ooit}\\

\haiku{Louise Wijngaarden, ( {\textquoteleft}...}{van het Gebouwklinkt het niet}{alsde l'Institut{\textquoteright}?}\\

\haiku{twee koningen voor -?).}{\'e\'en gemenebest wie kon}{zich dat voorstellen}\\

\haiku{Paul Sanders laat zijn;}{executanten meer vrijheid}{dan Pieter Tiggers}\\

\haiku{Zowel Tierie als.}{de heer en mevrouw De Boer}{komt daarvoor lof toe}\\

\haiku{Een periode.}{waarvan Mahler een typisch}{representant was}\\

\haiku{Er wordt angstwekkend (),:}{geslagentoch heeft zij geen}{groot forte ofwel}\\

\haiku{In aanleg is er.}{zelfs literair vermogen}{bij hem aanwezig}\\

\haiku{de reis is erg duur.}{en Itali\"e garandeert ons}{geen Frankfurts comfort}\\

\haiku{Maar als opera is.}{het stuk van Gounod beter}{dan dat van Busoni}\\

\haiku{Bewijst dit nu de?}{juistheid of onjuistheid van}{Sch\"onbergs principes}\\

\haiku{De werkelijke.}{waarde van een kunstwerk wordt}{er niet door bepaald}\\

\haiku{Het hoofdthema van.}{dit rondo is bovendien}{bepaald triviaal}\\

\haiku{De wezenlijke.}{waarde van een kunstwerk wordt}{er niet door bepaald}\\

\haiku{Het hoofdthema van.}{dit Rondo is bovendien}{bepaald triviaal}\\

\haiku{Het hoofdthema van.}{dit Rondo is bovendien}{bepaald triviaal}\\

\haiku{precies even typisch.}{Hollands zijn als Das Lied von}{der Erde Chinees}\\

\haiku{Maar Saskia lijkt een}{meesterwerk naast Nico van}{der Lindens toonstuk}\\

\haiku{Henri Zagwijn (1878).}{is the most modern of this}{generation}\\

\haiku{let us hope that,,.}{music in general will}{reap the fruits}\\

\haiku{Dat kan iedereen.}{met goede oren en goede}{hersenen leren}\\

\haiku{Superieur zijn (;}{de karikaturen van}{BrahmsIntermezzo}\\

\haiku{Pastiche nr. 16,:}{bestaat niet uit noten doch}{uit een regel druks}\\

\haiku{Mettez beaucoup de,,.}{notes n'importe lesquelles}{sauf celles qu'il faut}\\

\haiku{Het is trouwens niet;}{erg om zijn wiegeliedjes}{vergeten te zijn}\\

\haiku{Voor de ter zake.}{kundige lezer is dit}{nu wel voldoende}\\

\haiku{De woorden van het,.}{liedje dat zij daarbij zingt}{zijn van Debussy}\\

\haiku{En het publiek kan,...}{over iets wat het nooit gehoord}{heeft kwalijk denken}\\

\haiku{Hij {\textquoteleft}laadt{\textquoteright} de noten ();}{vanbijvoorbeeld de Vijfde}{met zijn eigen geest}\\

\haiku{Zijn de noten van,?}{een compositie dode}{symbolen of meer}\\

\haiku{ondergeschikt aan.}{de muziek en daardoor juist}{wat het wezen moest}\\

\haiku{Waarom geschiedt in?}{dit drama alles zoals}{het geschieden moest}\\

\haiku{Hoe sterker deze,.}{drang is des te sterker ook}{de vernielingsdrang}\\

\haiku{Hij dirigeert niet,.}{zonder entrain hij stelt zich}{het klankbeeld juist voor}\\

\haiku{De orkesten zijn.}{talrijk en de concerten}{worden goed bezocht}\\

\haiku{Van die bedoeling}{is de componist echter}{teruggekomen.{\textquoteright}2376}\\

\haiku{Die ontroering kan,,:}{direct ongecontroleerd}{aan de dag treden}\\

\haiku{Instructief was in.}{dit opzicht vooral het Oud}{driekoningenlied}\\

\haiku{Bach was noch een dor,:}{theoreticus noch een}{gevoelig dichter}\\

\haiku{het mangelt hem aan,,.}{felheid aan kritische zin}{aan vitaliteit}\\

\haiku{toen Mengelberg de.}{Suite uit L'oiseau de}{feu dirigeerde}\\

\haiku{Hiermee waren wij.}{dus voor de tweede keer de}{gasten van Itali\"e}\\

\haiku{Webern heeft zich het:}{merk van Sch\"onberg-adept}{te diep ingebrand}\\

\haiku{De  langzame,;}{delen zijn zeer grof voor een}{pianomuziek}\\

\haiku{Deze opera is.}{van 1865 en er staat gewis}{al veel Bizet in}\\

\haiku{Men offert veel op,.}{aan de logica ja zelfs}{aan de scholastiek}\\

\haiku{Parijs, Berlijn en,:}{Wenen verloren veel van}{hun betekenis}\\

\haiku{hier in Amsterdam.}{behaalde het nauwelijks}{een succ\`es d'estime}\\

\haiku{Het eerste deel is,.}{goed maar de rest is bijster}{weinig overtuigend}\\

\haiku{Karl Marx, Leo\v{s} Jan\'a\v{c}ek}{Onbevredigende}{kamermuziek}\\

\haiku{Adriano Lualdi's.}{Le furie d'Arlecchino}{is een luchtledig}\\

\haiku{Belangrijk was ook ().}{het StrijktrioSerenade}{van Alexander Jemnitz}\\

\haiku{Niemand zal tegen.}{deze gedachtegang iets}{willen inbrengen}\\

\haiku{Maar dat is ook niet;}{in eerste instantie de}{taak van het publiek}\\

\haiku{het intermezzo ().}{heeft aan de helft genoegzes}{maten plus opmaat}\\

\haiku{Het is bijna een,.}{revenant van het eerste}{deel in het verkort}\\

\haiku{Men herkent soms in;}{de aanvang de melodiek}{niet onmiddellijk}\\

\haiku{Dit melodische ().}{gegeven loopt totniet tot}{en met de slot-a}\\

\haiku{wat wordt ons, zowel, {\textquoteleft}{\textquoteright}?}{hier als elders voorgespeeld}{alsnieuwe muziek}\\

\haiku{Hij sleept niet mee, hij,.}{verkondigt niet hij vleit niet}{en hij dreigt evenmin}\\

\haiku{Wij komen op de.}{betekenis van Anton}{Webern nog terug}\\

\haiku{Hierop berust ten {\textquoteleft}{\textquoteright}.}{dele de dooddoener van}{DoppersHollandsheid}\\

\haiku{Zijn lyriek ontroert,.}{ons niet zijn climaxen grijpen}{ons niet in het hart}\\

\haiku{Misschien verwondert:}{deze uitlating u uit}{mijn mond een weinig}\\

\haiku{Ik had in hoofdzaak.}{hetzelfde willen vragen}{als de heer Gomperts}\\

\haiku{Ten slotte zijn ook ({\textquoteleft}}{de varianten op het}{oude Wilhelmus}\\

\haiku{Ik ben verheugd dat,.}{Wozzeck hier komt omdat het}{een prestatie is}\\

\haiku{Want voor de kunst als.}{zodanig loopt het grote}{publiek niet warm}\\

\haiku{En vangen aan te,.}{schelden of te loven in}{hun halve wijsheid}\\

\haiku{De grote massa.}{der tijdgenoten kan dit}{niet observeren}\\

\haiku{Het is, inderdaad,,.}{het allervoornaamste maar}{het is niet genoeg}\\

\haiku{{\textquoteright} zijn wij het in het.}{gegeven geval zonder}{enig voorbehoud eens}\\

\haiku{Maar neutraliteit.}{is de meest funeste vorm}{van tegenwerking}\\

\haiku{Dit hier lijkt op een.}{hedendaagse versie van}{Das Narrenschneiden}\\

\haiku{de hoofdrolspelers...}{verzuimden tot nu toe hun}{baard te laten staan}\\

\haiku{Op twee punten moet.}{ik in het bijzonder de}{aandacht vestigen}\\

\haiku{Met de muziek heeft.}{dit alles ternauwernood}{nog iets te maken}\\

\haiku{Julius R\"ontgen,,.}{bijvoorbeeld was het type}{van de muzikant}\\

\haiku{Maar dit is niet het.}{tijdstip voor stijlkritische}{bespiegelingen}\\

\haiku{Er zijn daar zalen;}{waarin het een genot is}{te musiceren}\\

\haiku{Maar hoezeer is de,!}{stemming erdoor be{\"\i}nvloed}{de sfeer vergiftigd}\\

\haiku{Aandelen Grieg zijn;}{tegenwoordig nauwelijks}{verhandelbaar meer}\\

\haiku{De situatie.}{is dus veel ernstiger dan}{wij gevreesd hadden}\\

\haiku{Stel daartegenover:}{een soortgelijk muzikaal}{probleem uit Lulu}\\

\haiku{Op het terrein der.}{muziekpedagogie valt}{nog zeer veel te doen}\\

\haiku{De Maneto zal,,.}{concerteren op 5 6}{8 en 12 juni}\\

\haiku{een Engels (Purcell), () ().}{een HollandsVoormolen en}{een FransDelibes}\\

\haiku{Wij signaleren,.}{dit niet voor de eerste niet}{voor de tiende maal}\\

\haiku{En dan verwondert.}{men zich nog dat de gasten}{dit maal niet lusten}\\

\haiku{Dich ber\"uhrte in,...}{Welschland fremder s\"usser}{Kunst neue Verk\"undung}\\

\haiku{Dich ber\"uhrte in,...}{Welschland fremder s\"usser}{Kunst neue Verk\"undung}\\

\haiku{Hij kwam, met deze,.}{conceptie schijnbaar in de}{buurt van Sch\"onbergs school}\\

\haiku{Wat dacht je, zou daar?}{nog een artikel voor de}{Groene in zitten}\\

\haiku{Het Concertgebouw;}{vernieuwt principieel zo}{weinig mogelijk}\\

\haiku{Weinig, te weinig,.}{van die beloften is in}{vervulling gegaan}\\

\haiku{de actuele.}{situatie geeft niet veel}{hoop op de toekomst}\\

\haiku{Maar de wereld der.}{klanken blijft onberoerd door}{het oorlogsgeweld}\\

\haiku{Terugziend op het:}{festival 1946 lijkt mij dat}{meer dan mogelijk}\\

\haiku{Jardins sous la pluie (1903)}{waarin de chromatisch}{dalende baslijn}\\

\haiku{Hier in Nederland.}{wordt Schmitts muziek zelden ten}{gehore gebracht}\\

\haiku{De composities (;}{uit de jaren 1904 en 1905}{Acht Lieder opus 6}\\

\haiku{H\'aba eveneens, doch.}{bovendien verwierp hij het}{hele toonsysteem}\\

\haiku{Tot dusverre bleef.}{de beweging tot het land}{van oorsprong beperkt}\\

\haiku{Natuurlijk klinkt Les -:}{soir\'ees de P\'etrograde}{anders zeggen wij}\\

\haiku{Mettertijd zal dit (}{het geval niet meer zijnmen}{went nergens vlugger}\\

\haiku{Harmonisch is het,.}{stukje zo ge wilt a-}{of polytonaal}\\

\haiku{sluit het werk met een.}{uit \ensuremath{\beta} afgeleid ritmisch}{figuur abrupt}\\

\haiku{Ook in het vierde,,.}{deel staan welbewust enige}{italianismen}\\

\haiku{Ons Europese,.}{\'e\'en-tw\'e\'e heeft zich wel}{overleefd naar het blijkt}\\

\haiku{Het eerste stukje;}{heeft het karakter van een}{marche fun\`ebre}\\

\haiku{A en b stijgen,.}{tot een climax c verloopt}{pianissimo}\\

\haiku{Het is alsof de;}{inwijdeling zich op gaat}{heffen tot het Al}\\

\haiku{Wil men zulks t\'och doen,.}{dan wachte men tot onder}{het Broedermaal}\\

\haiku{(Voor een overzicht van,,-.}{de inhoud zie Bijlage}{3 HPG 2 922927}\\

\haiku{in het Engels met.}{kapitalen en in het}{Frans in onderkast}\\

\haiku{{\textbullet} Acht\'elik, Josef,.}{Der Naturklang als Wurzel}{aller Harmonien}\\

\haiku{Annalen van de-.}{operagezelschappen in}{Nederland 18861995}\\

\haiku{Debussy, Claude,- ().}{Correspondance 18841918}{ed. Fran\c{c}ois Lesure}\\

\haiku{Verkade, Eduard \&,.}{dr. E.F. Cartier van Dissel}{Eduard Verkade}\\

\haiku{{\textbullet} Vuillermoz, \'Emile, {\textquoteleft}{\textquoteright}.}{Le style orchestral}{de Maurice Ravel}\\

\haiku{{\textbullet} Wolff, Betje \& Aagje,.}{Deken Historie van den}{Heer Willem Leevend}\\

\haiku{100 I 430, 431, 656,,,.}{657 658 Vioolsonate}{nr. 3 in d op}\\

\haiku{La vita nuova,}{I 801 Paradiso I 507}{807 Purgatorio}\\

\haiku{82 I 662 II 64,,,,.}{235 240 Gli\`ere Reinhold}{Strijkkwartet op}\\

\haiku{I 390 Mein Herz I,:}{390 Ord-Hume Arthur W.G.}{Pianola}\\

\haiku{832 Hoek, H.G. van de,,,, (-}{organist I 354 355 685}{686 Hoer\'ee Arthur18971986}\\

\haiku{) schrijver/librettist,,, (-)}{II 67 717 747 Hol J.C.1874}{1953 musicoloog}\\

\haiku{Simon van (1849-1929),, (-)}{musicoloog II 596 602}{Mittler Franz18931970}\\

\haiku{mecenas II 789, (-)}{Winding August Hendrik1835}{1899 componist II}\\

\haiku{115Ingezonden brief, (),.}{van Ovink opgenomen in}{Van Gilse2003 413}\\

\haiku{Pijper gebruikt het.}{woord grondtoon hier wellicht in}{figuurlijke zin}\\

\haiku{281Zie voetnoot 100.}{bij de recensie van 30}{januari 1918}\\

\haiku{289Zie voetnoot 83.}{bij de recensie van 16}{januari 1918}\\

\haiku{308Zie voetnoot 125.}{bij de recensie van 21}{februari 1918}\\

\haiku{341Zie voetnoot 83.}{bij de recensie van 16}{januari 1918}\\

\haiku{{\textquoteleft}Het is de eerste.}{maal dat de heer Pijper op}{zijn stuk terugkomt}\\

\haiku{slechts zal hier en daar.}{de nodige beperking}{worden ingevoerd}\\

\haiku{18.4 en het kwartet,.}{van Ravel n\'a de pauze}{stond Dvo\v{r}\'aks kwartet op}\\

\haiku{Het citaat {\'\i}n het,,-.}{citaat is van Berlioz \`A}{travers chants 89}\\

\haiku{{\textquoteleft}L'affirmation!}{de l'impuissance y est}{pouss\'ee jusqu'au dogme}\\

\haiku{452Voor Stephan P\'artos.}{zie de recensie van 19}{januari 1919}\\

\haiku{Zie ook voetnoot 130.}{bij de recensie van 27}{februari 1926}\\

\haiku{471Zie voetnoot 22.}{bij de recensie van 23}{januari 1918}\\

\haiku{481Parafrase:}{van een regel uit Schillers}{Ode an die Freude}\\

\haiku{511Zie voetnoot 68.}{bij Pijpers recensie van}{28 november 1919}\\

\haiku{514Pijper in de.}{programmatoelichting van}{26 november 1919}\\

\haiku{{\textquotedblleft}Dieu le voit{\textquotedblright}, maar.}{ik zie het ook en anders}{dan hij zich verbeeldt}\\

\haiku{Zie de recensie.}{van 28 november 1917 en}{de voetnoot aldaar}\\

\haiku{Zie ook voetnoot 126.}{bij de recensie van 21}{februari 1918}\\

\haiku{{\textquoteright} is een regel uit ().}{Der AbschiedDas Lied von der}{Erde van Mahler}\\

\haiku{Zie verder voetnoot.}{91 bij de recensie van}{20 december 1919}\\

\haiku{de namen van Van,,,.}{Anrooy Averkamp Viotta}{Wagenaar en Zweers}\\

\haiku{Deze schrijft, als hij:}{het heeft over de kenmerken}{van Franse muziek}\\

\haiku{Zie ook voetnoot 78.}{bij de recensie van 14}{januari 1918}\\

\haiku{631Minstrels waren,.}{geen negers maar als negers}{geschminkte blanken}\\

\haiku{644Zie voetnoot 240.}{bij de recensie van 13}{januari 1919}\\

\haiku{657Zie voetnoot 199.}{bij de recensie van 12}{februari 1920}\\

\haiku{Op 19 maart en 8.}{oktober 1921 noemt Pijper}{de opera nogmaals}\\

\haiku{713Op het programma,.}{na de pauze stond Sindings}{Vioolconcert op}\\

\haiku{722Ruyneman schreef zijn ();}{lied L'absolu1919 op tekst}{van Arthur P\'etronio}\\

\haiku{Matthijs Vermeulen.}{deed er ironisch verslag van}{in De Telegraaf}\\

\haiku{809Zie voetnoot 161.}{bij de recensie van 21}{januari 1920}\\

\haiku{818Zie voetnoot 122.}{bij de recensie van 12}{februari 1918}\\

\haiku{820Zie voetnoot 161.}{bij de recensie van 21}{januari 1920}\\

\haiku{828Zie voetnoot 207.}{bij de recensie van 17}{februari 1920}\\

\haiku{837Zie voetnoot 125.}{bij de recensie van 21}{februari 1918}\\

\haiku{852Zie voetnoot 171.}{bij de recensie van 23}{januari 1920}\\

\haiku{{\textquoteright} 876Zie voetnoot 218.}{bij de recensie van 21}{februari 1920}\\

\haiku{903Zie voetnoot 109.}{bij de Muziekkroniek van}{24 december 1920}\\

\haiku{938Zie voetnoot 125.}{bij de recensie van 21}{februari 1918}\\

\haiku{944Zie voetnoot 138.}{bij de recensie van 12}{januari 1920}\\

\haiku{963Zie voetnoot 78.}{bij de recensie van 14}{januari 1918}\\

\haiku{985Pijper schept hier.}{waarschijnlijk een beetje op}{over zijn geheugen}\\

\haiku{987Zie voetnoot 205.}{bij de recensie van 14}{februari 1920}\\

\haiku{988Zie voetnoot 223.}{bij de recensie van 21}{februari 1920}\\

\haiku{Het roode lampje,,.}{signifische gepeinzen}{verschenen in 1921}\\

\haiku{Wellicht is deze ().}{Sneeuwimpressie identiek aan}{de Dansimpressie1920}\\

\haiku{1024Zie voetnoot 223.}{bij de recensie van 21}{februari 1920}\\

\haiku{{\textquotedblright} Daarop, tot angst en,,.}{schrik van mijn vrouw liep ik de}{deur uit de straat op}\\

\haiku{1054Zie voetnoot 138.}{bij de recensie van 12}{januari 1920}\\

\haiku{1079Zie voetnoot 154.}{bij de recensie van 11}{februari 1921}\\

\haiku{1082Zie voetnoot 133.}{bij de recensie van 11}{januari 1921}\\

\haiku{1087Zie voetnoot 54.}{bij de recensie van 14}{februari 1921}\\

\haiku{1149In de Camera (,:}{ObscuraDe familie}{Kegge hoofdstuk 7}\\

\haiku{Diepenbrock aanvaardt.}{de dedicatie en woont}{de uitvoering bij}\\

\haiku{Zie verder voetnoot.}{165 bij de recensie van}{13 augustus 1922}\\

\haiku{De verzameling ().}{is uitgegeven bij W.}{de Haan te Utrechtz.j.}\\

\haiku{Musikbl\"atter des,,-,.}{Anbruch jrg. 4 nr. 34}{februari 1922}\\

\haiku{1328Zie voetnoot 46.}{bij de recensie van 27}{januari 1922}\\

\haiku{1343Zie voetnoot 19.}{bij de recensie van 6}{januari 1922}\\

\haiku{Vooral de laatste.}{paragraaf van het tweede}{deel werd herschreven}\\

\haiku{1404Zie voetnoot 109.}{bij de recensie van 7}{januari 1920}\\

\haiku{Musikbl\"atter des,,-,.}{Anbruch jrg. 4 nr. 34}{februari 1922}\\

\haiku{1469Zie voetnoot 138.}{bij de recensie van 12}{januari 1920}\\

\haiku{1518Zie voetnoot 166.}{bij de recensie van 21}{januari 1920}\\

\haiku{/ Zu singen um das, /',':}{Meisterst\"uck Gewinn es}{Kunst gewinn es Gl\"uck}\\

\haiku{zie de recensie.}{van 17 oktober 1921 en}{voetnoot 252 aldaar}\\

\haiku{Nu 'k over Claire '.}{Jache spreek moetk nog even}{verder vertellen}\\

\haiku{1604Zie voetnoot 171.}{bij de recensie van 23}{januari 1920}\\

\haiku{{\textquoteright} 1610Zie voetnoot 19.}{bij de recensie van 6}{januari 1922}\\

\haiku{{\textquoteright} Hendrik Marsman, {\textquoteleft}De{\textquoteright}.}{positie van de jonge}{Hollandse schrijver}\\

\haiku{1707Tekstregel uit ().}{Der AbschiedDas Lied von der}{Erde van Mahler}\\

\haiku{{\textquoteright} 1737Pijper beschouwt.}{de Scythische suite als}{programmamuziek}\\

\haiku{Zie ook voetnoot 126.}{bij de recensie van 21}{februari 1918}\\

\haiku{1978Herbert Antcliffe, {\textquoteleft},{\textquoteright}.}{De Planeten suite voor}{orkest van Gustav Holst}\\

\haiku{1995Dit essay is.}{later opgenomen in}{De Quintencirkel}\\

\haiku{2013Bedoeld zijn de.}{componisten van d'Indy's}{Schola cantorum}\\

\haiku{2048Zie voetnoot 176.}{bij de recensie van 25}{januari 1920}\\

\haiku{nr. 103 in Es (Mit) ().}{dem Paukenwirbel en nr.}{104 in DLondon}\\

\haiku{2080Zie voetnoot 167.}{bij de recensie van 21}{februari 1927}\\

\haiku{2116Een Equale is een.}{stuk voor gelijke stemmen}{of instrumenten}\\

\haiku{[...] Sie sehen bereits, []?}{wie sehr mich diese Musik}{Carmen verbessert}\\

\haiku{Beethoven, voor vijf,.}{pianoconcerten meer}{dan een half dozijn}\\

\haiku{De voorstellingen,,.}{waren op 10 12 14 en}{16 november 1927}\\

\haiku{2327Zie voetnoot 187.}{bij het essay in het Haagsch}{Maandblad van maart 1927}\\

\haiku{2339La puerta del.}{vino is de vijftiende}{van de 24 Pr\'eludes}\\

\haiku{Ook verschenen zijn.}{prenten in De Vrijheid en}{De Ware Jacob}\\

\haiku{2346Adolf Oberlander (-).}{18451923 was een beroemd Duits}{karikaturist}\\

\haiku{{\textquoteleft}Willem Pijper... c'est.}{une gloire musicale}{de la Hollande}\\

\haiku{{\textquoteright} 2350Zie voetnoot 130.}{bij de recensie van 27}{februari 1926}\\

\haiku{2358Zie ook de open ({\textquoteleft}{\textquoteright}.}{brief van Pijper aan Rutters}{Moderne muziek}\\

\haiku{2369Zie voetnoot 96.}{bij de recensie van 24}{januari 1928}\\

\haiku{Pijper schrijft op 4:}{juni 1946 in een brief aan}{Louise Bolleman}\\

\haiku{2380Zie voetnoot 161.}{bij de recensie van 21}{januari 1920}\\

\haiku{De voorstellingen,,.}{waren op 15 17 19 en}{21 november 1928}\\

\haiku{Op deze plaats is.}{de formulering uit De}{Muziek gehandhaafd}\\

\haiku{2474Zie voetnoot 161.}{bij de recensie van 21}{januari 1920}\\

\haiku{het staat er onder {\textquoteleft},{\textquoteright}-.}{het kopjeZ\"urich juni}{1926 op p. 119131}\\

\haiku{Maar een tijd die reeds,.}{gekomen is behoeft men}{niet meer te beiden}\\

\haiku{2520Midas staat hier.}{voor de kunstrechter die zich}{belachelijk maakt}\\

\haiku{2526Dit Strijkkwartet.}{uit 1927 was opgedragen}{aan Willem Pijper}\\

\haiku{de boston is een.}{Amerikaanse versie van}{de langzame wals}\\

\haiku{2744Toen Alban Berg.}{stierf was zijn tweede opera}{Lulu onvoltooid}\\

\haiku{mit allen Mitteln.}{der vorhandenen Technik}{eine Welt aufbauen}\\

\haiku{De reeks concerten.}{waarover Pijper hier schrijft was}{daarvan de eerste}\\

\haiku{2847De Anschluss van.}{Oostenrijk bij Duitsland vond}{plaats op 12 maart 1938}\\

\haiku{] Hindemith {\textquoteleft}hoort{\textquoteright} als.}{componist slechts tonale}{mogelijkheden}\\

\haiku{zie de recensie.}{in het Rotterdamsch Nieuwsblad}{van 9 juli 1927}\\

\haiku{] Georg Schaeffner, ().}{Claude Debussy und das}{PoetischeBern 1943}\\

\haiku{2896Pijper verwijst.}{hier met een voetnoot naar de}{huidige p. 804}\\

\haiku{2917De naam van Arthur.}{Hoer\'ee is vanaf de tweede}{druk weggelaten}\\

\haiku{2983Zie de voetnoot.}{bij de recensie van 12}{januari 1920}\\

\haiku{En ook anders dan.}{in een volmaakt geslaagde}{atonale muziek}\\

\haiku{{\textquoteright} 3006Zie de voetnoot.}{bij de recensie van 15}{januari 1920}\\

\haiku{3019Zie de voetnoot.}{bij de recensie van 21}{januari 1920}\\

\haiku{3022Zie de voetnoot.}{bij de recensie van 21}{januari 1920}\\

\haiku{tevens werden er.}{muziekvoorbeelden in de}{tekst opgenomen}\\

\haiku{3036Dit is niet de.}{eerste maal dat Pijper het}{woord kiemcel gebruikt}\\

\haiku{Zijn leerling Jan van.}{Dijk voltooide het werk in}{1992 als zijn opus 839a}\\

\section{Filip de Pillecyn}

\subsection{Uit: Elizabeth}

\haiku{Het mooie, donkere.}{kind was een vroeg ontwikkeld}{meisje geworden}\\

\haiku{Met donkere blik.}{had Konrad von Marburg op}{hen neergekeken}\\

\haiku{Zij nam het kind op.}{en \'e\'en voor \'e\'en bekeken}{de monniken het}\\

\haiku{Van ver het land in,.}{kwamen de mensen naar de}{voet van de Wartburg}\\

\haiku{Reeds uren wacht zij daar.}{met een knecht die de paarden}{aan de toom rondleidt}\\

\haiku{Dezelfde morgen.}{had zij hem toegefluisterd}{dat zij zwanger was}\\

\haiku{Hij keert terug naar.}{zijn tent en strekt zich uit op}{zijn legerstede}\\

\haiku{In \'e\'en ogenblik ziet;}{zij voor zich al de vreugden}{die zij heeft gekend}\\

\haiku{De woeste vreugde;}{van de feestvierenden dringt}{tot Elizabeth door}\\

\section{Herman Pleij}

\subsection{Uit: Van schelmen en schavuiten}

\haiku{{\textquoteright} Daar gaat hij op in,.}{en hij bedriegt de boeren}{vanwege hun geld}\\

\haiku{Wanneer men deze,.}{mensen niet kent dan zijn het}{doorgaans oplichters}\\

\haiku{Daarmee stichten ze.}{onder gehuwden grote}{angst en verwarring}\\

\haiku{Hij biedt drie of vier,.}{stuiver en zo is er dan}{weer \'e\'en bedrogen}\\

\haiku{Kooplieden, die hun.}{waren duur inkopen en}{goedkoop aanbieden}\\

\haiku{Hij, die het werk van,.}{een ander overneemt maar daar}{niets van terecht brengt}\\

\haiku{Zo blijft Aernout,.}{een ware schavuit in al}{zijn doen en laten}\\

\haiku{Ik versta de taal.}{der vogels even goed als gij}{die van uw pastoor}\\

\haiku{Een dergelijke.}{orde dienen de broeders}{in ere te houden}\\

\haiku{Meldt u derhalve,.}{in deze leerschool aan en}{weest nu allen stil}\\

\haiku{Maar hoe snel hij de,.}{koop tot stand bracht zo traag was}{hij met betalen}\\

\haiku{{\textquoteright} {\textquoteleft}Mijn lieve zoon, spot,.}{niet maar denk aan God en bid}{om Zijn ontferming}\\

\haiku{Derhalve gebood:}{hij zijn makkers om achter}{te blijven en sprak}\\

\haiku{Daarentegen was.}{hij zeer bedreven in de}{kunst van het schooien}\\

\haiku{En wanneer hij dat,.}{gevonden had moest hij hun}{dat laten weten}\\

\haiku{Ze kwamen tot de,.}{slotsom dat ze \'e\'en van hen}{zouden blinddoeken}\\

\haiku{{\textquoteright} Toen kwam de boer uit,.}{zijn bed en de pastoor liet}{de lakens wassen}\\

\haiku{Dus ging hij maar op,.}{pad zonder te weten wat}{hij precies moest doen}\\

\haiku{In Antwerpen ging.}{een schoenmaker eens uit eten}{in De Berkentak}\\

\haiku{{\textquoteleft}Breng me nog een pot,.}{bier dan komt het precies uit}{op zeven stuiver}\\

\haiku{wie niet werkt is een,.}{luie oplichter die buiten}{de maatschappij hoort}\\

\section{Sybren Polet}

\subsection{Uit: De geboorte van een geest}

\haiku{Keek haar na, deurknop.}{in de hand en glimlachte}{toen ze ook omkeek}\\

\haiku{Standaert / welcke:}{sy volghen ende lollen}{op haeren wijze}\\

\haiku{hadden zich lekker,;}{gewassen voor ze hier naar}{toe gingen met zeep}\\

\haiku{Iedereen vliegt wel!}{eens met een groen takje in}{z'n bek Op niemand}\\

\haiku{- als straf of boete?}{voor zonden die mijn ouders}{eens begaan hebben}\\

\haiku{De enorme glimlach,...}{boven mij glimlach in de}{vorm van een gezicht}\\

\haiku{want men daeromme;}{in toecomende tyden}{gheen recht doen en sail}\\

\haiku{-Waarom heb je.}{nooit plastiese chirurgie}{laten toepassen}\\

\haiku{Hij moest zich nu al.}{inhouden terwijl het spel}{pas begonnen was}\\

\haiku{daarna, op 't laatst,.}{nauwelijks nog bewegend}{bijna stilliggend}\\

\haiku{misschien kan ik straks}{in een ander te weten}{komen wat voor ras}\\

\haiku{omme gebragt te ',}{werden opt pleyn alhier}{alwaer men gewoon}\\

\haiku{deze zijn weg en;}{steekt messcherp door het lichaam}{van de bezoeker}\\

\haiku{verlengt zich nog een,}{meter of zo en veert dan}{weer terug terug}\\

\haiku{Daarnaast nog wel vijf.}{andere lichtpunten en}{-ornamenten}\\

\haiku{-In zwart wit kun.}{je ze helemaal niet meer}{uit elkaar houden}\\

\haiku{-O, niks aan de ' -.}{hand.t Was gelukkig maar}{psychies krijg pillen}\\

\haiku{alleen had ik me.}{nooit gerealiseerd dat}{ze utopies waren}\\

\haiku{-Schenk mij dan maar,,...}{eens in zei Birgitta mijn}{geest is zo aards dat}\\

\haiku{woensdagmiddag weer...}{\'een van die machteloze}{schijnvertoningen}\\

\haiku{Daar komen dan flats,.}{zoals in de Bijlmer met}{een huur van f 400}\\

\haiku{Daar geraakte de, '.}{schaafbank op de straat en de}{karel spronger uit}\\

\haiku{Wat duyvel of je '!}{wel meent de beurgers en al}{t volk te villen}\\

\haiku{hoe fyn singt dien hont,!}{nou hoe zit hij nu op met}{hangende pootjes}\\

\haiku{Jij kaerel, seg, wil,?}{je aanstonds onse maats hier}{doen koomen ofte niet}\\

\haiku{mijn kleine klauwen,.}{zijn scherpe tandjes die in}{mijn nekvel bijten}\\

\haiku{De kamer waarin.}{ik binnengelaten word}{is geweldig groot}\\

\haiku{Er komt een schuwe.}{blik in zijn zwarte ogen als}{we hem aanspreken}\\

\haiku{Ook zonder pil voelt;}{hij zich geladen met een}{grote vrolijkheid}\\

\haiku{- Liegen alsof het,.}{waar is schrijven alsof het}{werkelijkheid is}\\

\haiku{Dat Doelenhotel.}{was daarom in mijn jeugd het}{toppunt van rijkdom}\\

\haiku{Hier wonen dacht hij,.}{verder lopend en pillen}{bijna overbodig}\\

\haiku{het is echter een,.}{puimsteen welke men deze}{vorm gegeven heeft}\\

\haiku{Het gevolg was dat.}{velen hun behoefte maar}{ter plaatse deden}\\

\haiku{En over drie jaar ga,,.}{je dus met pensioen-}{Ja zei hij blij toe}\\

\haiku{Dat was de enige.}{heerlijke avond die ik in}{Amsterdam doorbracht}\\

\haiku{met \'e\'en woord, 't was,.}{het yslykste schouwspel dat}{men bedenken kan}\\

\haiku{zelfs de kleuren van,...}{het hout hebben iets grijs het}{groen en het beige}\\

\haiku{-... -Soms denk ik.}{dat de mensheid meer pijn dan}{plezier heeft beleefd}\\

\haiku{Hij had al urenlang;}{met zijn offici\"ele}{gast opgetrokken}\\

\haiku{ik moet hem zien te.}{pakken te krijgen voor hij}{het terrein verlaat}\\

\haiku{En herinneren - (:),: (:), (:),:}{etengoed en openengoed en}{opnemengoed en}\\

\haiku{Had het gevoel dat,,...}{als hij maar bleef lopen er}{niets kon gebeuren}\\

\haiku{Hij kon echter met {\textquotedblleft}{\textquotedblright}.}{zijnmeesters niet overweg en}{werd steeds lastiger}\\

\haiku{om reden dat ze.}{geen slaven hadden die voor}{hen konden werken}\\

\haiku{een kort verslag van ().}{hoe het boekongeveer tot}{stand is gekomen}\\

\subsection{Uit: In de arena}

\haiku{Zeer vreemde dingen,.}{heb ik daar gezien te veel}{om te beschrijven}\\

\haiku{zoet water zonder.}{te weten of er water}{op het eiland was}\\

\haiku{Voorts kwam men overeen'.}{dat Lucretia Cornelisz}{eigendom zou zijn}\\

\haiku{Cornelisz zelf had.}{zoals gezegd Lucretia}{voor zich opge\"eist}\\

\haiku{bij zich had hij een,.}{ton water een ton met brood}{en een vaatje wijn}\\

\haiku{De afgang van de,:}{held was nu kompleet want Pelsaert}{schreef in zijn verslag}\\

\haiku{Daarna liepen we.}{op het gebouwenkompleks}{toe en luisterden}\\

\haiku{Hij liet  ons toe.}{tot de salon en belde}{om de huishoudster}\\

\haiku{De jongen in bed.}{knikte heftig van ja en}{begon te hoesten}\\

\haiku{De jongen dacht na.}{en we lieten hem verder}{maar alleen denken}\\

\haiku{Ook Perdie ontging,.}{dit uiteraard niet maar zijn}{houding verried niets}\\

\haiku{6 De treinreis naar.}{Baywater nam niet meer dan}{twintig minuten}\\

\haiku{de moeder wordt nooit,.}{echt zelfs niet als de vader}{het kind ge\"echt heeft}\\

\haiku{de deur was op slot,.}{maar de sleutel stak er aan}{de buitenkant in}\\

\haiku{Nu keek de man op,.}{fronste eerst zijn wenkbrauwen}{en glimlachte toen}\\

\haiku{mogelijk heeft hij,.}{geen motieven ik bedoel}{eigen motieven}\\

\haiku{Hij gaat schuil achter}{de werkelijkheid die hij}{konstateert of schept}\\

\haiku{hooguit spreekt het lijk.}{de waarheid via ons of via}{onze buikstem}\\

\haiku{trouwens alles was,,.}{halfzichtbaar en leek zelfs in}{het halfdonker grauw}\\

\haiku{Wel spraken vader.}{en zoon de hele winter}{niet tegen elkaar}\\

\haiku{Op een avond keerde,,.}{Ekil en met hem twaalf man niet}{naar het schip terug}\\

\haiku{De Koeren gingen.}{daarop eten en drinken en}{waren zeer vrolijk}\\

\haiku{we gaan terug naar.}{de hoeve en zeggen hem}{wat er gebeurd is}\\

\haiku{Wanneer hij kwaad werd.}{kreeg zijn gezicht harde en}{grimmige trekken}\\

\haiku{- Niet tegen u of,,.}{tegen een overmacht zei Ekil}{maar man tegen man}\\

\haiku{Nadat Ekil aan boord.}{was gegaan hesen ze het}{zeil en gingen scheep}\\

\haiku{Op zee, ja Ekil is,.}{nu op zee varend in de}{richting van IJsland}\\

\haiku{Hij sloeg werktuiglijk.}{een kruis en riep tegelijk}{in zijn hoofd Odin aan}\\

\haiku{de orewoet zich niet.}{op maar in de aarde man}{vrouw buurvrouw dochter}\\

\haiku{Op de drempel neemt,.}{de monnik de wachtenden}{in ogenschouw broedend}\\

\haiku{Ik voel dat ik het.}{leven zal verlaten v\'o\'or}{1 januari}\\

\haiku{Of ze tonen hem.}{beter dan hij is en dan}{zijn het ook schoften}\\

\haiku{Hij staat echter niet.}{het Absolute Toeval}{toe dat hem ontkent}\\

\haiku{Pericelcius keek.}{met \'e\'en oog en ik zag zijn}{gezicht verstrakken}\\

\haiku{Hij was die hij is,,.}{in alle eeuwigheid hij}{is niet die hij was}\\

\haiku{Cellinius sprak;}{zich uit voor het verfraaien}{van het lelijke}\\

\haiku{Men keek mij bevreemd,,.}{aan behalve Brusano die}{nadenkend knikte}\\

\haiku{Toch realiseert.}{de verbeelding zich misschien}{vooral via het woord}\\

\haiku{En dacht - en haar beeld:}{werd er alleen maar aardser}{en vuriger door}\\

\haiku{Wie twijfelt moet de.}{kok z'n rug krabben en z'n}{wangen opblazen}\\

\haiku{En die het meest op.}{gewone mensen lijken}{zijn het gevaarlijkst}\\

\haiku{Deze vertrok geen,.}{spier terwijl het gezelschap}{in spanning toekeek}\\

\haiku{- Ik realiseer,!}{mijzelf via een ander via}{een keizer nog wel}\\

\haiku{- Breng ons nog een vis,,!}{een meerval of morene}{het doet er niet toe}\\

\haiku{Zijn geest behoorde.}{niet meer tot het domein der}{veranderingen}\\

\section{Servatius Josef Ponten}

\subsection{Uit: Die Bockreiter}

\haiku{Und die ganze Zeit,,.}{\"uber die du bei uns wohnst}{habe ich dich lieb}\\

\haiku{In der Nacht verliert?}{Ihr einen Finger in der}{H\"ackselmaschine}\\

\haiku{{\textquoteright} fluchte der Doktor, {\textquoteleft}?}{wo zum Teufel habt Ihr denn}{den Finger gelassen}\\

\haiku{Sonderbar, sagte,!}{er f\"ur sich wie das Kind im}{Manne nicht ausstirbt}\\

\haiku{{\textquoteright} frug Frau Elisabet.}{freundlich erstaunt und streichelte}{ihr das schwarze Haar}\\

\haiku{Hahn vom Turme der,.}{Abtei wenn es auch nur ein}{blecherner Hahn war}\\

\haiku{{\textquoteleft}Ich glaube, er kann.}{sie besser auswendig als}{die Psalmen Davids}\\

\haiku{Die Blumen in den}{G\"arten waren geschlossen}{und schliefen genau}\\

\haiku{Du erschienst oben \"uber,.}{der Dachtraufe die leichten}{Leitern nachziehend}\\

\haiku{Und wenn die Sonne,?}{verl\"oscht ist z\"undet Gott}{dann den Mond nicht an}\\

\haiku{Die Nacht ist die Zeit,.}{der Diebe und Dichter der}{Gespenster und Engel}\\

\haiku{Die G\"ange schienen.}{aus den Kehlen der Winkel und}{Ecken aufzuschreien}\\

\haiku{Auch die Nonnen und!}{Heiligen haben keine}{englischen Leiber}\\

\haiku{Alle Fenster der,:}{Gasse \"offneten sich}{und man rief sich zu}\\

\haiku{Jetzt klatschte die.}{Jauche auf und verschlang}{ein ringendes Paar}\\

\haiku{Sollen sie uns noch?}{weiter unsere gr\"une}{Jugend vorwerfen}\\

\haiku{Die R\"auber stellen.}{bei Beginn der Nacht rund um}{das Dorf Wachen aus}\\

\haiku{Durch viele blinde.}{Flintensch\"usse t\"auschen sie}{eine gro{\ss}e Zahl vor}\\

\haiku{Alles klatschte,}{und tausend grobe H\"ande}{streckten sich den wei{\ss}en}\\

\haiku{in der linken Hand.}{und die R\"uckenlehne des}{Stuhles in der rechten}\\

\haiku{So geschehe mit,,{\textquoteright}.}{Euch was Rechtens ist sagte}{seufzend der Richter}\\

\haiku{{\textquoteright} {\textquoteleft}Wenn das Gericht mehr,?}{wei{\ss} als ich selbst was will es}{dann von mir wissen}\\

\haiku{Es gab nur eine,,.}{Strafe den Galgen f\"ur ihn und}{alle Bockreiter}\\

\section{Elisabeth Maria Post}

\subsection{Uit: Het land, in brieven}

\haiku{Alleenlijk heb ik;}{mij met het nazien van de}{drukproeven belast}\\

\haiku{Welk een vrolijkheid:}{spreidt deze gedachte op}{mijn toekomstig lot}\\

\haiku{zowel wanneer zij -.}{predikt als wanneer zij op}{het toneel verschijnt}\\

\haiku{welk een gunstige!}{bestiering derhalve van}{een menslievend God}\\

\haiku{Ik zag de zwakheid,!}{van mijn lichtverleid hart dat}{zo dikwijls helaas}\\

\haiku{Nu verlang ik naar,.}{de rust het is meer dan \'e\'en}{uur na middernacht}\\

\haiku{de schone vlokken!}{dalen met een statige}{gelijkheid neder}\\

\haiku{De tedere bloem, ',:}{diet veld versiert Blijft uit}{zijn handen leven}\\

\haiku{- op hoop van spoedig,;}{antwoord geef ik hem vandaag}{aan de post mede}\\

\haiku{vergenoegdheid en.}{deftigheid waren op zijn}{gelaat geschilderd}\\

\haiku{De oude man ging.}{voort met verhalen van de}{dagen zijner jeugd}\\

\haiku{uit hun vette wei,;}{mijn paarden grinneken of}{mijn schapen blaten}\\

\haiku{Haar man poogt zoveel;}{mogelijk zijn droefheid voor}{haar te verbergen}\\

\haiku{Ik ging naar een klein,;}{somber zijvertrek waar het}{lijkje geplaatst was}\\

\haiku{uit het kiempje van.}{dit verdervend stof zal een}{Engel voortkomen}\\

\haiku{De tuinman snoeide;}{de vruchtbomen wier knopjes}{dagelijks zwellen}\\

\haiku{Ik zuchtte wel eens ';}{bijt vooruitzicht van zijn}{donkere dagen}\\

\haiku{O Eufrozyne,?}{wie zou hier niet voelen dat}{de Schepper goed is}\\

\haiku{en het zijn vette,.}{droppelen die een frisse}{geur medebrengen}\\

\haiku{Waar een ijdele,!}{wereldslaaf wanhopig is}{daar juicht een Christen}\\

\haiku{als de natuur ons;}{hoger opleidt dan vervult}{zij de ziel volmaakt}\\

\haiku{en de gehele.}{natuur kreeg een vrolijker}{gedaante voor mij}\\

\haiku{U een aangenaam;}{ogenblik te verschaffen is}{mij uren moeite waard}\\

\haiku{Oostwaards zocht ik, met,;}{mijn teleskoop de stad van}{mijn Eufrozyne}\\

\haiku{- Een gevoelige:}{ziel moet zelfs dikwijls meer dan}{gewoon vrolijk zijn}\\

\haiku{dan gevoel ik de ':}{kracht vant schone gezang}{van lavater}\\

\haiku{en deze geeft ons.}{hier de natuur in dit stil}{verblijf der onschuld}\\

\haiku{hoe sterk werken zij!}{beide om mijn gehele}{ziel te verrukken}\\

\haiku{Laat ons daarom ons:}{geheel aan de leiding der}{vriendschap overgeven}\\

\haiku{en zoudt gij tot zulk?}{een prijs uw verwondering}{wel kopen willen}\\

\haiku{Zijn schoon welgevormd;}{lichaam was de woning van}{een nog schoner ziel}\\

\haiku{- Al mijn eindige!}{begeerten waren voldaan}{in mijn Melidor}\\

\haiku{en nooit dacht ik dit.}{verlies zo lang te zullen}{kunnen overleven}\\

\haiku{Gisteren deden,,;}{wij na ons avondmaal nog een}{lieve wandeling}\\

\haiku{Hoe vriendlijk zijn uw,,,!}{bleke stralen O lieve}{zachte schone Maan}\\

\haiku{Gij kort zijn bange,.}{lijdensnachten Uw blij gelaat}{verkwikt zijn ziel}\\

\haiku{Zij gebruiken schild -, ' '.}{noch wapent Hoofd is hun}{vant schreien warm}\\

\haiku{Want hoe gevreesd ons,,?}{deze stond zij hij komt toch}{zeker en wanneer}\\

\haiku{Waar vond men toch  ,.}{een genoegen dat niet met}{verdriet gemengd is}\\

\haiku{achter ons was de;}{aarde nog door de vale}{schemering bedekt}\\

\haiku{mijn nauwkeurigste:}{teke  ning zou al het}{schoon doen verdwijnen}\\

\haiku{emilia Grootser en;}{treffender gezicht levert}{de natuur niet op}\\

\haiku{- een boer zelfs staat stil,,;}{op zijn land en ziet de Zon}{aan die het koestert}\\

\haiku{mijn vriendin had mij.}{lang begerig gemaakt om}{dit te bezoeken}\\

\haiku{Opgetogen van,.}{nieuwsgierigheid hielden wij}{hier een ogenblik stil}\\

\haiku{ziet en hoort gij in?}{dit alles niet een loflied}{voor de Algoedheid}\\

\haiku{{\textquoteleft}wordt{\textquoteright} een aanzijn gaf,,;}{werd een machteloos kind lag}{in een arm verblijf}\\

\haiku{en vooral nog korts '.}{bijt bezoeken van een}{hunner ondervond}\\

\haiku{aan de andere.}{kant een onafzienbaar vak}{van korenvelden}\\

\haiku{Hoe veel vlijt, hoe veel,!}{kunstvermogen bezitten}{deze schepseltjes}\\

\haiku{Hier bedankten wij.}{onze heuse geleidster}{voor haar gul onthaal}\\

\haiku{de half gerotte.}{vensters hingen scheef in de}{verzakte sponning}\\

\haiku{{\textquoteleft}De vossen hebben,;}{holen en de vogelen}{des hemels nesten}\\

\haiku{Het smaaklijk brood dat,;}{zij ons schenken Doet ons aan}{uwe goedheid denken}\\

\haiku{Nu beginnen de;}{eerste flikkeringen der}{morgenschemering}\\

\haiku{hoe verkwikkelijk!}{zal de gezuiverde lucht}{mij tegenkomen}\\

\haiku{het een, noch ander,.}{kan ooit anders dan in mijn}{verbeelding bestaan}\\

\haiku{slaafse banden, die!}{mij meer drukken nadat ik}{de vrijheid kende}\\

\haiku{Over enige tijd is***;}{mijn moeders voornemen met}{mij naar te reizen}\\

\haiku{maar een harmonie,.}{die ik wel gevoelen doch}{niet beschrijven kan}\\

\haiku{Gij hebt de knopjes,.}{voort doen komen Gerijpt in}{warme zomerlucht}\\

\haiku{Zo ver ik haar ken,;}{heeft ook haar karakter een}{zweemsel van het uwe}\\

\haiku{Gaarne had ik hier.}{de verrijzing der maan en}{sterren afgewacht}\\

\haiku{mijn luister zal ook,;}{groeien Als mij de zon der}{eeuwigheid bestraalt}\\

\haiku{mijn jeugd onsterflijk,.}{bloeien Terwijl ook gij met}{eeuwge schoonheid praalt}\\

\haiku{- Mijn verbeelding doet,;}{mij nog de nachtegaal nog}{het duifje horen}\\

\haiku{De roos die op uw,,;}{koontjes gloort Cefize zal}{niet eeuwig bloeien}\\

\haiku{De schoonheid is een;}{bloem die sterft Als tijd en smart}{haar blaadjes krullen}\\

\haiku{wat haar luister moog,;}{bestaan Nooit heft ze uw ziel}{tot hoger orden}\\

\haiku{Zij heeft waarschijnlijk;}{op ons zeereisje koude}{op de long gevat}\\

\haiku{- de mijne kan haar,.}{op het levenspad geen troost}{meer geven ik sterf}\\

\haiku{{\textquoteright} Zij wees mij in een,:}{lade een klein doosje aan}{en voegde er bij}\\

\haiku{En laat Sofia ook,;}{uw vriendin worden als ik}{voor u niet meer ben}\\

\haiku{Schoonder dan alle.}{bloemen is mij het mos dat}{er nu rondom groeit}\\

\haiku{De seksuele;}{component maakt immers de}{liefde onzeker}\\

\section{J. Presser}

\subsection{Uit: De nacht der Girondijnen}

\haiku{Oppassen, dat ik,.}{Schiller niet vergeet ook al}{zo'n citatenkast}\\

\haiku{Denk nu niet  *~          ,.}{alleen aan dat baardje maar}{ook aan die bloedplas}\\

\haiku{ik richt me tevens,.}{tot een meelezer die over}{mijn schouder heen kijkt}\\

\haiku{{\textquoteright} {\textquoteleft}Over u. U weet toch,.}{ook waarachtig wel hoe de}{Joden ervoor staan}\\

\haiku{Je zou je toch rot,,:}{lachen zo'n meneer Acohen}{die tegen me zegt}\\

\haiku{U vond het heerlijk, - -,.}{u u bent toch niet boos nou}{u zwelgde erin}\\

\haiku{In Westerbork, waar,.}{ik met de vacantie ben}{geweest bij Vati}\\

\haiku{Maar daar is maar \'e\'en.}{weg heen en die  heb ik}{je aangewezen}\\

\haiku{ik schreef daar al iets -,,.}{over niet weer doen zegt Jacob}{maar vergeet het niet}\\

\haiku{ik heb zelf daarna,,).}{vrouwen vlak v\'o\'or de baring}{naast de ton gezet}\\

\haiku{toen ik even inhield,:}{om te braken kreeg ik van}{Adelphi zelf een trap}\\

\haiku{Komt me op een dag -,?}{de Raaf in de les je weet}{toch wie de Raaf is}\\

\haiku{Nu had ik net een.}{juweel van een onderwerp}{voor de kinderen}\\

\haiku{Want waar hij was, daar,.}{was het gezellig in een}{kamp een zeldzaamheid}\\

\haiku{dit was de eerste,;}{joodse geestelijke die}{ik ooit had ontmoet}\\

\haiku{{\textquoteleft}Heb je zo'n hekel?}{aan de Protestanten en}{de Katholieken}\\

\haiku{Ik bedoelde het,,}{uiteraard ironisch maar \`of}{dat ontging hem \`of}\\

\haiku{Toch weet ik dankzij,.}{de kampklets wel ongeveer}{wat er gebeurd is}\\

\haiku{wat hij verkondigt,;}{is de  waarheid alleen}{de toekomstige}\\

\haiku{Ik kan alleen maar,;}{zeggen dat er sindsdien niets}{meer van me over is}\\

\haiku{Ze berust, ze is,:}{heel dapper maar tevens erg}{bezorgd om haar man}\\

\haiku{Meer kon hij me ook}{al niet vertellen en ik}{helde ertoe over}\\

\section{Sientje Prijes}

\subsection{Uit: Een bewogen vrijdag op de Breestraat (onder pseudoniem Sani van Bussum)}

\haiku{Want die was er al, '.}{den vorigen dag en is}{s nachts gebleven}\\

\haiku{Eigenlijk verlangt,,,....}{hij nu juist n\`u naar die vrouw}{die daar zoo ziek ligt}\\

\haiku{Want de zieke moet,....}{immers w\`el pijn hebben wil}{er voortgang komen}\\

\haiku{Die was niet meer in,....}{te halen die was naar zijn}{pati\"enten toe}\\

\haiku{Al gappen ze me,....}{maar een goud horloge ben}{ik al gesjochten}\\

\haiku{Hij drukte zich den,....}{hoed diep in het achterhoofd}{terwijl hij nadacht}\\

\haiku{Daar had hij willen,....}{troosten ezel die hij was en}{nog wat kwaads gezegd}\\

\haiku{{\textquoteleft}Nee, ik ga niet weg,{\textquoteright}, {\textquoteleft}.}{zei hij geruststellendik}{blijf erbij zitten}\\

\haiku{Al leg je d'r goud,,....}{bij d\'a\'ar dan kan ik n\`og niks}{door me keel krijgen}\\

\haiku{{\textquoteright} vroeg ze, toen de meid,,.}{weer beneden kwam kwaad dat}{die haar zag eten}\\

\haiku{De stem van Costa Gomez....}{donderde tegen die van}{de barende op}\\

\haiku{Gekheid, hij zal eens,:}{gaan hooren zooals men bij een}{zieke gaat hooren}\\

\haiku{De smaak was Jolie,....}{vergaan het walgde hem voor}{zijn tweede glaasje}\\

\haiku{Bezorgt u d'r 'n, ' ' ',....!}{paar mooiet komtr niet op}{an watt kost daar}\\

\haiku{n ouwen man ben, ', '}{ik niet jaloersch laatm gaan}{n ouwe bok lust}\\

\haiku{Inderdaad, hij was '.}{zoo lekker dik en vetjes}{alst maar hoefde}\\

\haiku{Zijn ruggetje was,,.}{van spek zijn zachte zijen}{nekje zuiver room}\\

\section{Arij Prins}

\subsection{Uit: De heilige tocht}

\haiku{Het z.g. beschaafde,,.}{publiek tot voor kort kende}{geen schrijver Arij Prins}\\

\haiku{D., {\textquoteleft}indertijd aan.}{Dr. ten Brink zond en waar ik}{niets meer over hoorde}\\

\haiku{Er kan geen sprake -:}{zijn van moralistische}{bedoeling laat staan}\\

\haiku{ik zijn wezen te {\textquoteleft}}{karakteriseeren in}{den aanhef van mijn}\\

\haiku{Zoo ging de ridder,.}{lange voort en wel hem was}{alsof hij daalde}\\

\haiku{, en op zijn borst was,.}{zware druk door soepelen}{last die zich bewoog}\\

\haiku{De ridder somber,.}{ging daarheene en zond zijn}{knaap met fakkel weg}\\

\haiku{Wind koelde  af.}{zijn aangezicht waarvan de}{druppelen vielen}\\

\haiku{De muren, die dit,:}{droegen hoog kleurden uit in}{wisseling van steen}\\

\haiku{De Saracenen,;}{donkere groepen die om}{de vuren lagen}\\

\haiku{Prins nam toen Alfred,:}{St\"urken in zijn firma op}{die voortaan heette}\\

\subsection{Uit: Een koning}

\haiku{De koning, alleen,.}{in een stijf hoogen zetel voor}{het open vensterluik}\\

\haiku{een streep weggaande.}{zonne-kracht boven de}{steengrijze wallijn}\\

\haiku{hij maakte  groot,.}{misbaar en wenschte zijn}{gunsteling terug}\\

\haiku{Naast hem zijn slagzwaard,,.}{zoo lang als een knaap dat hij}{niet kon hanteeren}\\

\haiku{Maar Harold al naar,,.}{de stad voor-over op zijn paard}{dat laag-draafde}\\

\haiku{Om hem steigeren,.}{en slaan met armen en beenen}{naar de dreighoeven}\\

\haiku{In zijn gulzige;}{schouwen het woninghout van}{menschen zonder vuur}\\

\haiku{Toen tweemaal openslaan,.}{in grilligheid het Boek dat}{op het graf gelegd}\\

\haiku{wel twist als \'een steeds,.}{winnen en trekken dan in}{toorn breede messen}\\

\haiku{En in den witten,.}{den nacht-dag nog menig}{man zoo omgebracht}\\

\haiku{En speren oppe.}{splinters schenen met glans van}{spitsen bovenaan}\\

\haiku{Glad de weg, die steeg,,.}{en moeijelijk het gaan dat}{was met ongeduld}\\

\haiku{Vlijm-glans van daggen,,.}{dreigend hoog in vuisten en}{bloed langs wanden droop}\\

\haiku{Ook stijve voeten.}{bloot-gemaakt door die slecht}{schoeisel hadden}\\

\haiku{schoten schaarsch, en,,.}{veel te hoog met droog geluid}{als hout dat kraken}\\

\haiku{hij kwam door hoorn van,}{een handlantaarn waarachter}{was een oud gelaat}\\

\haiku{De bode echter,.}{sprak en werd tersluiks in het}{huis gelaten}\\

\haiku{Zij dwaalden op het.}{dek met zacht ge-krab als}{van een stervende}\\

\haiku{een dichtheid grijs van.}{fijn gesprenkel in het stil}{ten-avond-gaan}\\

\subsection{Uit: Uit het leven}

\haiku{Spinoza vroeg aan?}{Janus hoeveel hij er daar}{wel mee had gemold}\\

\haiku{hare oogen waren.}{vochtig en een dikke traan}{rolde langs haar wang}\\

\haiku{Zonder door iemand,.}{te zijn opgemerkt was zij}{binnengekomen}\\

\haiku{Freek hield het paard vast,.}{dat opeens zoo mak als een}{duif was geworden}\\

\haiku{Werken ging echter,.}{niet en Oliehoek zette zich}{tegen een hooiberg}\\

\haiku{{\textquoteright} De dokter haalde,.}{de schouders op en ging weer}{naar de bedstede}\\

\haiku{Zij stemde toe, en.}{tegen zes uur des avonds ging}{het tweetal van huis}\\

\haiku{Zijn hoofd dreigde te,.}{bersten en hij viel als een}{meelzak op den grond}\\

\haiku{maar opeens kreeg hij,.}{een geduchten stoot waarvan}{hij nooit meer opkwam}\\

\haiku{Twee dagen later,.}{had hij het op de borst en}{moest te huis blijven}\\

\haiku{De patroon hield hem.}{dan ook eigenlijk alleen}{uit medelijden}\\

\haiku{Des Maandags ging Jan.}{Zomer er op uit om een}{baantje te zoeken}\\

\haiku{Hap, en 't glas is, ' '.}{leeg net alsoft inn}{laars is leeggegooid}\\

\haiku{{\textquoteright} {\textquoteleft}Dat had ik al lang,.}{gedaan als jij maar niet zoo}{sullig was geweest}\\

\haiku{Hij heeft 't weer op,.}{de borst en leit te hijgen}{als een koespaard}\\

\haiku{Een enkele maal;}{kwam een der kinderen of}{Krijns naar hem kijken}\\

\haiku{De reis werd te voet,.}{gemaakt want hij had geen geld}{om per spoor te gaan}\\

\haiku{Eindelijk, na veel,.}{hijschen en trekken kregen}{zij hem op een stoel}\\

\haiku{Hare geheele.}{persoonlijkheid had iets luis}{en onverschilligs}\\

\haiku{Zij was toch ook niet,!}{geboren om onder de}{boeren te leven}\\

\haiku{Als dat nu nog eens, '.}{gebeurt dan zal ikt van}{je loon afhouden}\\

\section{Jan Pelgrum Pullen}

\subsection{Uit: Die navolginghe Christi}

\haiku{\'e\'en enkele maal ().}{Beda en Gregorius}{de Grotecap. 10}\\

\haiku{Wat was Sijn Passie,.}{Sijn steruen anders dan}{ootmoedicheijt}\\

\haiku{dat Hij dick wils den,}{menschen dit ende  dat}{beneempt hier in proefs}\\

\haiku{ende    als hij}{sus in die gelaetenheijt}{gestelt is dan sal}\\

\haiku{Den mensch en moet daer,.}{niet sijn hij moet daer sijn}{oft hij niet en waer}\\

\haiku{Dat is seker, wort,}{den mensch inden wille Godts}{geset   soe can}\\

\haiku{{\textquoteleft}daer den mensch staet in,,}{die   puerheijt daer is}{hij in Godt ende}\\

\haiku{Den   mensch is een,}{edel creatuer sijn siel}{is geschaepen nae}\\

\haiku{dat hij Godt, Sijnen,}{Heere in alles dat Hij}{hem gegheuen}\\

\haiku{Lucam, dat Hij des}{daechs leerden inden tempel}{ende des nachs}\\

\haiku{dat hij dick sijn knien.}{boechden in sijn gebedt voor}{alle menschen}\\

\haiku{Helias was een}{mensch sterffelijck als}{wij ende hij badt}\\

\haiku{Dat is seker, waer}{den mensch niet en begeert}{hem te beteren}\\

\haiku{Siet alsoe moet dat;}{ock in ons sijn ende daer}{moeten wij   sijn}\\

\haiku{bij alle menschen.}{ende in alle steeden}{afghescheijden}\\

\haiku{comen, jae, niets niet:}{en achten ende dat}{en is niet wonder}\\

\haiku{Dan ist, dat wij hier}{duer in Godt ghetoghen}{werden ende}\\

\haiku{Hoe ghij met Christo.}{altijt opghericht sult staen in}{Godt den Vader}\\

\haiku{sij, ende soe veel,.}{haer leuen doot was   soe}{veel wast een wesen}\\

\haiku{Sij hadden, dat  .}{was in gheelheijt ghewesent}{in den wesen Godts}\\

\haiku{in God bevangen,,,.}{besloten opgenomen}{van God vervuld}\\

\haiku{het doordringen van.}{het goddelijk licht in het}{menselijk gemoed}\\

\chapter[3 auteurs, 1151 haiku's]{drie auteurs, elfhonderdeenenvijftig haiku's}

\section{H.P.G. Quack}

\subsection{Uit: Herinneringen uit de levensjaren van Mr. H.P.G. Quack 1834-1913}

\haiku{Maar er was een groot,.}{onderscheid die schilder doet}{het niet om zichzelf}\\

\haiku{Het werken in de.}{samenleving kon zoo schoon}{zijn en is het niet}\\

\haiku{Den aard der school heb.}{ik pogen te schetsen in}{mijn opstel over Joh}\\

\haiku{Wij plaagden wreed als -}{naar gewoonte de arme}{stakkers van Zwitsers}\\

\haiku{Doch zelfs op onze.}{vergeten school kwam er nu}{als een soort van koorts}\\

\haiku{Het profetisch lied {\textquoteleft},?}{van Isa\"ac da CostaWachter wat}{is er van den Nacht}\\

\haiku{Langs twee kanten kreeg '.}{ik een nieuwen kijk opt}{leven daar buiten}\\

\haiku{Het was wederom,.}{August Zimmerman die mij}{daarop deed letten}\\

\haiku{Doch dat was slechts de.}{uiterlijke aanleiding}{van het verschijnsel}\\

\haiku{Het knellen van 't,.}{gareel was al te voelbaar}{al te hinderlijk}\\

\haiku{Dit stond voor mij vast,.}{dat ik wakker moest zijn in}{allerlei opzicht}\\

\haiku{Die aandacht ging over,.}{tot verbazing wanneer men}{hem hoorde spreken}\\

\haiku{Voor ons, leden van,,}{het Unica dier dagen werd}{hij de leider dien}\\

\haiku{Die serenade {\textquoteleft}{\textquoteright}.}{was eenpanache van ons}{bestaan dier dagen}\\

\haiku{In mijn oogen reikte.}{hij over de eeuwen he\^en de}{hand aan Martinus}\\

\haiku{Hij, die zijn leven,.}{daaraan ten offer brengt vindt}{het ware leven}\\

\haiku{Daer eick bij eick soo,.}{vrolijck groeit Hel velt vol}{soete boeckweit bloeit}\\

\haiku{De kracht en werking.}{en wisseling dier vormen}{moest worden bepaald}\\

\haiku{Ik had zelfs geen geld.}{meer om plaats te nemen voor}{een tocht naar elders}\\

\haiku{Men mocht ook bij hem.}{niet spreken over de dingen}{die men niet goed wist}\\

\haiku{Hij was in dit en {\textquoteleft}{\textquoteright}.}{elk ander opzicht de man}{van derechte lijn}\\

\haiku{Het uitzicht op het;}{Spaarne vlak tegenover de}{groote brug boeide hem}\\

\haiku{{\textquoteleft}on ne discute,{\textquoteright}.}{pas avec ses adversaires}{on les supprime}\\

\haiku{Let wel, dat hij in.}{zijn vormen de beleefdste}{man ter wereld was}\\

\haiku{Eens slechts kreeg ik het, {\textquoteleft}{\textquoteright},.}{met mijn officieren als}{corps bijna te kwaad}\\

\haiku{Met het begin van.}{September 1861 kwam ik in}{Amsterdam terug}\\

\haiku{Bij wijlen ging hij.}{hierin tot aan de grenzen}{van uitbundigheid}\\

\haiku{Met zijn vriend van Braam.}{had Vriese een groot landgoed}{aan de Ruhr gekocht}\\

\haiku{Trouwens waar waren?}{in ons land de mannen die}{daarvoor oog hadden}\\

\haiku{Toch ging alles, zij ',.}{t met eenige schokken en}{rukken nog vooruit}\\

\haiku{Zoowel met den staat als.}{met den heer baron de Hirsch}{werd onderhandeld}\\

\haiku{Hij bewonderde.}{altijd het gezegde van}{Quesnay aan den dauphin}\\

\haiku{Ik wist thans beter,}{dan vroeger welk een zware}{taak der menschheid}\\

\haiku{Zij kwamen dikwijls}{des Zondags bij ons en wij}{gingen veel malen}\\

\haiku{Ik moest daarvoor stil,,.}{in mij zelf gekeerd zonder}{ophouden werken}\\

\haiku{Mij niet gevangen.}{te geven in het gareel}{der partijschappen}\\

\haiku{economie was slechts.}{een fragment van de leer der}{sociologie}\\

\haiku{Het zal blijken dat.}{in Holland het hart nog op}{de goede plaats klopt}\\

\haiku{Vondel teekende:}{den Hollandschen staatsman met}{het \'e\'ene woord}\\

\haiku{En wij moesten het in:}{de Gids van Augustus 1866}{ternederschrijven}\\

\haiku{In de dagen van.}{mijn secretariaat had}{ik hem leeren kennen}\\

\haiku{Maar zijn werkkring aan '.}{de staatsspoorwegen boeide}{hemt allermeest}\\

\haiku{elk zijner woorden.}{en handelingen droeg een}{Hollandschen stempel}\\

\haiku{Het was een rede {\textquoteleft}.}{overTraditie en Ideaal}{in het Volksleven}\\

\haiku{De beste krachten.}{van vroeger werden als met}{lamheid geslagen}\\

\haiku{Dank zij dat middel.}{had mijn vrouw nimmer meer iets}{van die kwaal gemerkt}\\

\haiku{Men deed het reeds voor;}{de stroo-mingen in de}{lucht en in de zee}\\

\haiku{Toen \'e\'ens de stoot,.}{was gegeven zette de}{beweging zich voort}\\

\haiku{Neen, hij was de man;}{der souvereiniteit van}{het individu}\\

\haiku{Charles Edmond zelf.}{had zich in die dagen ook}{niet veel rust gegund}\\

\haiku{Ik meende op die;}{wijze mijn vaderland trouw}{te kunnen dienen}\\

\haiku{Godin had - als echo -:}{der woorden van Fourier tot}{zich zelven gezegd}\\

\haiku{een nog niet langen:}{tijd uitgekomen boek van}{Friedrich Nietzsche}\\

\haiku{{\textquoteright} - {\textquoteleft}Das der Verbrecher,.}{sich selbst richtete war sein}{h\"ochster Augenblick}\\

\haiku{In de bureaux der.}{Bank werd hij volkomen de}{meester en leider}\\

\haiku{toen hij president).}{was geworden voelde hij}{zich volkomen thuis}\\

\haiku{Want hij was in den,.}{vollen zin van het woord een}{braaf trouw en edel man}\\

\haiku{Bank zilverbons uit,.}{te geven van f1.- f}{2.50 en f 5.-}\\

\haiku{Doch de twee eerste.}{onderdeelen behielden}{toch haar groote waarde}\\

\haiku{Elk uur van den dag,,.}{was hij aan den arbeid kloek}{slim en onstuimig}\\

\haiku{In het gezicht van.}{de haven liet hij het stuur}{aan anderen over}\\

\haiku{Nooit heb ik zulk een.}{droeven gang gedaan als op}{dien tocht naar Den Haag}\\

\haiku{Het wapen van zijn.}{woorden en daden was een}{degen voor den Raad}\\

\haiku{Op het begrip der {\textquoteleft}{\textquoteright}.}{gemeenschap kon geen beroep}{meer worden gedaan}\\

\haiku{van Kretschmar voort,.}{om aan al die verlangens}{het hoofd te bieden}\\

\haiku{Ik waagde toen reeds.}{een terugblik te werpen}{op het verleden}\\

\haiku{Het bezigen van;}{staats-geweld kan slechts een}{uitzondering zijn}\\

\haiku{Dus {\textquoteleft}verder{\textquoteright} is de,.}{leus want de maatschappij heeft}{haar schepen verbrand}\\

\haiku{Zij daarentegen.}{hadden het oog op alle}{gedeclasseerden}\\

\haiku{Toch is het - volgens -.}{Ruskin niet z\'o\'o bezwaarlijk}{dit te ontdekken}\\

\haiku{Een vriendelijke.}{herinnering blijft mij bij}{van deze bonden}\\

\haiku{Dit alles werd door.}{Izoulet merkwaardig puntig}{uit\'e\'engezet}\\

\haiku{Op die wijze had {\textquoteleft}{\textquoteright}.}{hetNut gewerkt in de eeuw}{die het had beleefd}\\

\haiku{w\`el heb ik hem bij.}{uitstek gewaardeerd en nu}{en dan bewonderd}\\

\haiku{Op de zeden en.}{gewoonten van het volk moet}{worden ingewerkt}\\

\haiku{Zijn kloek v\'o\'orgaan in.}{die Kamer is algemeen}{bekend en geroemd}\\

\haiku{zij zijn tevreden;}{als zij des Zondags hun preek}{hebben gehouden}\\

\haiku{Voor 't oogenblik.}{bepaalden wij ons tot dat}{wat voor de hand lag}\\

\haiku{Over en we\^er zouden.}{wij ons hier elkander de}{hand kunnen bieden}\\

\haiku{Les d\'elicats sont,.}{malheureux Rien ne saurait}{les satisfaire}\\

\haiku{Ik durf dus, zelfs niet,.}{fluisterend die woorden op}{mijn lippen nemen}\\

\haiku{Ik wensch voortaan liefst,,,}{te zwijgen eerbiedig als}{het kan op te zien}\\

\haiku{naar de hoogte, en.}{te blijven vereeren wat}{te\^er is en heilig}\\

\haiku{Baudriliart, H.J.L.,,,.}{prof. staathuishoudkunde te}{Parijs geb. 1821 55}\\

\haiku{Broere, prof., dichter, {\textquoteleft}{\textquoteright},.}{en wijsgeer redacteur van}{De Katholiek 41}\\

\haiku{Cervantes Saavedra,,,-,,,.}{Migu\"el de Spaansch dichter}{15471616 14 134 146}\\

\haiku{Damlust{\textquoteright}, fabriek van,,,,.}{spoorwegmate-rieel te}{Utrecht 107 111 117 196}\\

\haiku{Goethe, J.W. von, de,-,,,,,.}{Duitsche dichter 17491832 30}{87 92 214 392 noot}\\

\haiku{Hirsch, M. baron de,,,,,,,,.}{bankier te Brussel 103 111}{116 117 120 121 187}\\

\haiku{Kasteele, J.C. van,,.}{de ambtenaar aan een der}{ministeries 123}\\

\haiku{, 1824-82, 34, 92 v.,,,,,,,,,,,,.}{94 95 120 128 171 184 192}{194 234 302 335 336}\\

\haiku{Vissering, S., jur.,,,,,,,,.}{prof en minister 43 87}{126 140 141 205 335}\\

\haiku{minister, 106, 107V.,,,,,,,,,,,,.}{110 112 113 114 115 119 120}{122 137 139 189 196}\\

\haiku{Zimmerman, August,,,,,,,,,,,,.}{8 9 10 13 14 16 17}{22 23 29 40 416}\\

\haiku{{\textquoteleft}De natuur wil niet.}{het uitsluitend bezit van}{een enkelen sijn}\\

\haiku{{\textquoteleft}evenwijdig loopen.}{het recht van den mensch en het}{recht van eigendom}\\

\haiku{De lezers, evenals,}{de schrijver zullen hem den}{versctmldigden}\\

\section{Em. Querido}

\subsection{Uit: Het geslacht der Santeljano's. De verweerde jaren (onder ps. Joost Mendes)}

\haiku{De stad In 1876 was.}{Dortendam de groote stad van}{het kleine Holland}\\

\haiku{Ze ademden dieper,,,;}{en forscher ze hijgden ze}{huilden ze lachten}\\

\haiku{kon moeilijk op als.}{hij zat en ging dadelijk}{zitten als hij stond}\\

\haiku{Het was als wekte.}{haar Lot telkens heel zacht tot}{het leven terug}\\

\haiku{Mordechai's gezicht.}{was onder het vertellen}{even vaal-gebleekt}\\

\haiku{Maar dan kwam na zijn;}{stil versnikten weemoed een}{diep onmachtsgevoel}\\

\haiku{het heeleparket;}{had er lol in gehad en}{naar hem gekeken}\\

\haiku{- Heb jullie 't ook,, ...}{gehoord begon Jonas}{opgewekter nu}\\

\subsection{Uit: Het geslacht der Santeljano's. Het licht dat gloorde (onder ps. Joost Mendes)}

\haiku{Aan het andere,,;}{eind ook over en weer zaten}{de meer gegoeden}\\

\haiku{Langzamerhand was.}{de straat dicht-gebouwd en}{ook de sloot gedempt}\\

\haiku{wat niet weg ging in,.}{het gezin lapte hij er}{grof voor zichzelf door}\\

\haiku{Naar zich toe had hij.}{dat gezin gehaald met een}{overstelpenden drang}\\

\haiku{In het wezenloos;}{wegzinken van Zadok prees}{hij de bedaardheid}\\

\haiku{een wilde draai{\"\i}ng.}{en verwarrende mijmer}{ging zwaar door zijn hoofd}\\

\haiku{De wegsmuigeming.}{van hun hartelooze pret had}{Daan opslag gezien}\\

\haiku{- Nee, jij niet, kleine......}{geweldenaar dank je voor}{je Nova Zembla}\\

\subsection{Uit: Het geslacht der Santeljano's. De dorrende akker kiemt (onder ps. Joost Mendes)}

\haiku{Zwaar stapelde het.}{materiaal tegen de}{jongens zich daar op}\\

\haiku{Zoo'n dag had alles.}{dan zijn blij-na{\"\i}eve}{belangstelling}\\

\haiku{En wanneer zij hen,;}{vasthadden kwamen ze niet zoo}{gemakkelijk los}\\

\haiku{hij vulde hun  ,,.}{barre norsche werkdagen}{met stil fier geluk}\\

\haiku{Die klaarde met zijn,;}{spits vernuft hun overvolle}{wazige gevoel}\\

\haiku{Dan kwamen ze tot,,.}{hem gejaagd en hunkerend}{om hulp en uitleg}\\

\haiku{Zoo groeide, warm en,;}{oprecht hun leven om dat}{van Van Collem heen}\\

\haiku{Van Collem kreeg er.}{zijn deel in alsof hij een}{Santeljano was}\\

\haiku{Ko hield vast dan en,.}{vuurde aan Daan boog zich er}{onder en droeg}\\

\haiku{- Ko lacht, hield Jonas,,.}{gemeen-opzettelijk even}{onderbrekend vast}\\

\haiku{de mannen quasi,;}{steviger ieder in zijn}{malle eigenheid}\\

\haiku{Als op een eersten,,.}{rang zoo lekker en op zijn}{gemak zat Jonas}\\

\haiku{t waren jongens......}{die durfden gepakt waren}{door nieuwe idee\"en}\\

\haiku{- Goede morgen, zei,,.}{hij zuiver beschaafdwaardig}{naar de jongens toe}\\

\subsection{Uit: Het geslacht der Santeljano's. De revolte-dagen (onder ps. Joost Mendes)}

\haiku{materieel en,.}{geestelijk gonsde het er}{al meer en sterker}\\

\haiku{Maar al meer was hun.}{maatschappelijke macht in}{de stad gestegen}\\

\haiku{die lachte breed en,.}{juichte zacht zijn oogen diep van}{glans en aandoening}\\

\haiku{secuur gedaan met.}{zijn naarbuiten geperste}{gladgestrekenheid}\\

\haiku{Niet op het gebied;}{van de uitbuiting was dit}{compromittante}\\

\subsection{Uit: Het geslacht der Santeljano's. Het wonderschone rijpen (onder ps. Joost Mendes)}

\haiku{Hij had gezien dat,}{Ko niets wist enkel leefde}{in den brandenden}\\

\haiku{Dan verstierf al het;}{kabaal van Ko's wetenschap}{en werd hij heel stil}\\

\haiku{de groote taart al 's ':}{morgens v\'o\'or achten gebracht}{metn rijmpie}\\

\haiku{Ze sprak weinig, vroeg,;}{niets luisterde enkel en}{was vol zorgzaamheid}\\

\haiku{En nooit was in haar.}{bewegen de inspanning}{der veerkracht merkbaar}\\

\haiku{Even, \'e\'en moment, greep}{Daan dan alles wat er in}{zijn ziel aanliefde}\\

\haiku{op het stadhuis was,.}{geen luchie meer het stonk nog maar}{alleen ter plaatse}\\

\haiku{Vogelendaal was.}{voor de jongens het blije oord}{van hun blanke jeugd}\\

\haiku{Hij gr\'e\'ep niet naar de, ...}{dingen die hij wilde hij}{stortte zich er op}\\

\subsection{Uit: Het geslacht der Santeljano's. 's Wasdoms volle tooi (onder ps. Joost Mendes)}

\haiku{zijn voorhoofd vluchtte}{als in verbijstering met}{een vaart naar achter}\\

\haiku{te zwaar drukkenden.}{last van al die vereering en}{aanhankelijkheid}\\

\haiku{De Dort kostte Ko,.}{halve dagen zoover liep hij}{langs de oevers door}\\

\haiku{Want het geweldig.}{torsen van de ijs-Dort}{was het groote wonder}\\

\haiku{- Goed Kootje, goed, gaf, ...... -}{Balront innig toe maar waarom}{ons niet meegevraagd}\\

\haiku{Ze lachten als hij,...}{lachte keken op slag weer}{stroef als hij het deed}\\

\haiku{Pekel en Koe strije...,... -...!}{z{\`\i}j zegge ijs is niks als}{water Waterstof}\\

\subsection{Uit: Het geslacht der Santeljano's. 's Werelds daverende wedloop. Eerste boek (onder ps. Joost Mendes)}

\haiku{Zij was er niet voor,.}{de dingen maar de dingen}{waren er voor haar}\\

\haiku{Haar destructieve,;}{desorganiseerende}{geest vrat alles aan}\\

\haiku{Toen had Myriam,,.}{haar oom Jacques Pardo op}{kantoor genomen}\\

\haiku{Als u werken wilt, ',...}{neemt u maarn schoon doekje}{d'r staan  er wel}\\

\haiku{... en ik vermoed, ik ',, - '}{weett niet ik gis maar hoor}{s menschen wegen}\\

\haiku{zij lachte zich de,,.}{tanden blinkend bloot hij bleef}{zonder krimp immuun}\\

\haiku{En met een vaart sloop,.}{hij ineens toen de kamer}{door op den haard aan}\\

\haiku{Geen levend wezen,.}{met gevoel gedachte en}{spraak verkozen ze}\\

\haiku{maar zij, lachend er,.}{op bedacht was ineens weer}{door het deurtje weg}\\

\haiku{Nu ging ze zich voor,.}{de koffie kleeden riep ze}{Ko in haar kamer}\\

\haiku{Heel het personeel,.}{de familie en de kring}{lagen voor hem neer}\\

\haiku{In haar kloek en fraai,;}{imponeerend handschrift schreef ze}{dozijnen brieven}\\

\subsection{Uit: Het geslacht der Santeljano's. 's Werelds daverende wedloop. Tweede boek (onder ps. Joost Mendes)}

\haiku{En na een poos, toen,,.}{het stond een vesting geleek}{het een citadel}\\

\haiku{Hij voelde zich een.}{door het socialisme}{direct bedreigde}\\

\haiku{- Dit alles was de.}{bezielende glans hunner}{vergaderingen}\\

\haiku{Nu vorderde het.}{algemeen aanzien van het}{groote gild met den dag}\\

\haiku{Het huis, allereerst,,.}{was nu opengezet wijder}{dan ooit te voren}\\

\subsection{Uit: Het geslacht der Santeljano's. 's Werelds daverende wedloop. Derde boek (onder ps. Joost Mendes)}

\haiku{En de rooien, ze ....}{lachten enverdubbelden}{hun heilzame hel}\\

\subsection{Uit: Het geslacht der Santeljano's. De hooge lichte kim der stilte (onder ps. Joost Mendes)}

\haiku{opheffing van De,.}{Fakkel of royement de}{zaal ingestooten}\\

\haiku{met derzelver hoog,!}{geroemden raken dialoog}{merde verdomme}\\

\haiku{Van vrijwel alles,;}{wat gebracht was had hij het}{record verbeterd}\\

\section{Isra\"el Querido}

\subsection{Uit: De Jordaan: Amsterdamsch epos. Deel 1}

\haiku{Bi j\'ei d'r d\`en noar,.... '!}{Folled\`em s\`el me doar auk}{n leindertje sa\`an}\\

\haiku{- - Sau'n kern\`ek,.. beet de.}{aalman Manus nijdig den}{kant van Karel uit}\\

\haiku{Bang door het geweld.}{keken de burgers benauwd}{uit naar politie}\\

\haiku{- Sch\`af je snurkert,.... op ' '!}{t petrauleim-f\`et}{leit nogn snei braud}\\

\haiku{huylklep.... en nau de,.....}{steine op en ploag d'r nau}{je schorremorrie}\\

\haiku{Toorn en twist waren.}{de dreigendste gedaanten}{van hun hartstochten}\\

\haiku{tr\`ep je foader nie....}{op se pink en staur s'aage}{niet in se tukkie}\\

\haiku{lachte hij weer en.}{stapte met een streel onder}{de kin van Bet weg}\\

\haiku{Ook nu weer weende,}{ze haar woorden uit terwijl}{ze toch zoo zoetjes}\\

\haiku{Hij slorpte zoetjes.}{zijn koffie in en keek weer}{peinsdiep naar den grond}\\

\haiku{Voor haar met Frans had.}{de meid net zooveel ontzag}{als voor een deurknop}\\

\haiku{Ze stonden in haar.}{insteekje en ze zag ze}{de h\'e\'ele week v\'o\'or zich}\\

\haiku{Ze jammerde haar.}{leef-ellende in alle}{bizonderheid uit}\\

\haiku{En niet \'e\'en keer in,.}{de week maar iederen dag}{slurpte hij zich zat}\\

\haiku{Ze herinnerde}{zich precies al de knusse}{babbeltjes tusschen}\\

\haiku{- Stijn zwierf nu al 's,;}{nachts op zee met de vlet de}{kwakken tegemoet}\\

\haiku{ze zou ze spuwen.}{op hun kliergezichten en}{moes van ze trappen}\\

\haiku{Stijn had Thijs beet die.}{zich telkens wou opwerken}{en op Neel smakken}\\

\haiku{Hij voelde nu zelf.}{dat hij er afschuwelijk}{gemeen moest uitzien}\\

\haiku{dat was zijn leven,.}{nu de verontrusting van}{hem was \'afgetild}\\

\haiku{Stijn stopte hem het.}{tabakszakje toe en keek}{kalm over het water}\\

\haiku{Thijs gromde woedend,,.}{wat terug maar aarzelde}{liet het zuipje los}\\

\haiku{Je zoudt ze tegen.}{hun dekschaal de eeuwigheid}{wilen intrappen}\\

\haiku{als die kwak maar zijn....}{zeiltjes wou draaien en te}{loefter oversteken}\\

\haiku{In een vuil-goor.}{onderhemd was hij blootshoofds}{de straat opgehold}\\

\haiku{Ze hadden allen.}{in de buurt gloeiend het land}{aan Dien's gierigheid}\\

\haiku{lachte Neel vadsig,.}{haar kleffe handen over haar}{hooge buik afstrijkend}\\

\haiku{Ze leek van binnen,.}{wel een doodzerk zoo drukkend}{stil was het in haar}\\

\haiku{Wil je haar beter,.}{hebben doe dan alsof je}{gelooft wat ze zegt}\\

\haiku{- Mie de Roeier kwam.}{weer van het plaatsje en de}{wasch-lijntjes}\\

\haiku{Plots schreeuwde ze naar.... -,,....}{Mie de Roeier Mie ka\`ak doar}{slentert je ma\`ad \'an}\\

\haiku{Groote bekommering.}{voor ziekte en droogte sloeg}{angst in de harten}\\

\haiku{Dronkenschap, armoe.}{en vuil besmetten al hun}{daden en woorden}\\

\haiku{Ze werden ontzien {\textquoteleft}{\textquoteright},;}{alsnette menschen door het}{gemeenste crapuul}\\

\haiku{mit achttien stuyfer.... -,}{f'rdient tie d'r achttien Nou}{en ikke seg d\`et}\\

\haiku{- Vlak voor de Wijde,.}{Gang hield een zwarte koets stil}{in de Willemstraat}\\

\haiku{Maar hij deed niets en.}{zijn schelden krijtte zich dood}{op eigen onmacht}\\

\haiku{Al dat kletsende?}{janhagel wou zijn jool nu}{in wroeging verkeeren}\\

\haiku{Een onbeschaamder.}{vrouwspersoon had hij in zijn}{leven niet ontmoet}\\

\haiku{- - Blinde Jaonus,....;}{so\`agt se luchtkesteile lichtte}{de gelagbaas in}\\

\haiku{Weer draaide Karel.}{bij en strafte den sleeper}{met hoonende woorden}\\

\haiku{Benauwde luchten.}{van peulen en bleekpoeier}{mengden zich dooreen}\\

\haiku{Nou speet het Karel.}{dat hij zijn harmonica}{niet had meegetild}\\

\haiku{Dien sproetneus zou hij.}{hebben laten hinken en}{springen als een clown}\\

\haiku{Traag slenterden, bij,.}{stoetjes achter elkaar de}{vrinden de markt op}\\

\haiku{Eindelijk was het.}{trotsche dienstmeisje door een}{bres heengedrongen}\\

\haiku{Ze lachten elkaar,.}{toe in verdekte schaamte}{overmoedig van woord}\\

\haiku{Nou moest hij wel zijn.}{brandendst begeeren naar die meid}{weg-leugenen}\\

\haiku{Een heetige koorts.}{van verlangen groeide \'a\'an}{in den menschendrom}\\

\haiku{in zijn armen, beenen,,.}{romp hoofd en handen werd de}{muziek lijn en stand}\\

\haiku{loat ik nou jo\`a '!....}{n biggetje kra\`age die}{d'r me b\`uyk uytd\`enst}\\

\haiku{De Lindengracht lei.}{bekrioeld van dansende}{meiden en kerels}\\

\haiku{het was zoo een gul.}{wereldje bij tante Nel}{uit de Goudsbloemstraat}\\

\haiku{Vlug was haar hulp, rad.}{haar tong en guitig keken}{haar dartele oogen}\\

\haiku{Maar als de drank in.}{hem gistte wist hij nooit wat}{hij eigenlijk deed}\\

\haiku{Ze verlangde toch?}{geen lust-slot en geen tuin}{met karseboomen}\\

\haiku{En plots weer met een.... -,....}{luimig lachje versleepte}{hij zijn woordjes Uch}\\

\haiku{Haar heele lichaam.}{voelde ze drijven in een}{heete vochtigheid}\\

\haiku{Moest ze zich nou weer,?}{zuur denken om het geld n\'a}{de negen dagen}\\

\haiku{- Die m\`elligha\`ad m\^o '!....}{jen Bokkebek nie an}{se geifel h\`enge}\\

\haiku{Werkeloos,.... \'eten van,!}{zijn vrouw's standje van wat zij}{bijeen had gezwoegd}\\

\haiku{Door d\'at en door die,....}{feuilletonsche boeken had}{ze Gronjee na\'ast haar}\\

\haiku{Neel, nog een beetje,.}{erger bijgeloovig dan}{anders schrok ervan}\\

\haiku{Hij baggerde door,,.}{door zonder te zien waar hij}{eigenlijk heenging}\\

\haiku{Alles deinde nu.}{op het rustige rhythme}{van haar ademhaal mee}\\

\haiku{Vreemd,.... nu beklemde.}{haar niet meer de gedachte}{aan zijn werkloosheid}\\

\haiku{Mientje luisterde,.}{nu naar een comedie niet}{zelf door Lien gezien}\\

\haiku{Mientje had den zin.}{voor plagerigen humor}{van Neeltje in zich}\\

\haiku{waspit = Jordaansch.}{meisje van de vroegere}{waskaarsenfabriek}\\

\haiku{pl\`ekje van rood.}{haar opleggen = je kop}{tot bloed ranselen}\\

\subsection{Uit: De Jordaan: Amsterdamsch epos. Deel 2: Van Nes en Zeedijk}

\haiku{Corry's aandacht was.}{alweer afgedwaald naar een}{bokking-venter}\\

\haiku{Joden Jet bleef koers,.}{houden begon weer met haar}{lokkende koopjes}\\

\haiku{Maar wat had ze een...!}{goddelijk blank velletje}{en wat een knar haar}\\

\haiku{Ze had nog nooit zoo.}{een aanhaalderigmooie meid}{uit het volk gezien}\\

\haiku{De blanke boezems.}{der vrouwen fonkelden van}{valsche juweelen}\\

\haiku{In tijden had ze.}{van zulk een mak handeltje}{niet z\'o\'o genoten}\\

\haiku{8Ze schold en keef met,.}{haar compagnon ze trapte}{op haar inkoopen}\\

\haiku{Ja toch, ook om zijn.}{snapsie vond hij het leven nog}{wel de moeite waard}\\

\haiku{Een mensch moest als de.}{wind ongezien verschijnen}{en ongezien gaan}\\

\haiku{Dat is jonkheer Van... -,,....}{Overhoof Soo edelaardige}{Dirk schertste Peet door}\\

\haiku{De wrok groefde de.}{vette rimpels nog dieper}{in zijn laag voorhoofd}\\

\haiku{Jij smakkert,... jij bint.........!}{t'r ijser ballast schot en}{tonne bij mekaar}\\

\haiku{- W\`a sou 't!... zei Jaap,......}{geraakt en haastig slikte}{hij zijn borrel ik}\\

\haiku{'n krats! - Hoeveel?... vroeg,.}{Jo haar gepoetste nagels}{zacht beademend}\\

\haiku{Ze schaterden valsch.}{en een gniepige drift vrat}{Jo's mooi gezicht op}\\

\haiku{Schuw kwam Toontje op.}{en drillend dwongen ze hem}{-te luisteren}\\

\haiku{... gierde Jo naar de.}{koppelaarster en trapte}{een hoedendoos in}\\

\haiku{Toen vertelde de.}{Rooie dat zij altijd te laat}{kwam op de b\^uhne}\\

\haiku{- De laatste van 't...!}{jaar bij de burgemeester}{op de bouleva\'ard}\\

\haiku{Ze wou, wo\'u en geen.}{enkel ondersmoord begeeren zou}{haar doen wankelen}\\

\haiku{Een mensch als zij, kon.}{nou niet maar ineen goud uit}{een hoorntje drinken}\\

\haiku{Ze had nooit in haar,.}{gansche leven zoo gevloekt}{gehuild en gesnikt}\\

\haiku{Dien wou terug en.}{stapte op de tram bij een}{lijn-driehalte}\\

\haiku{Met zijn duiven kon.}{hij veel beter omgaan dan}{met zijn kinderen}\\

\haiku{Een kluchtspel, al dat,,.}{grut dat ze g\'af terwijl ze}{zelf betalen moest}\\

\haiku{Haar ruim vierjarig.}{Leendertje was toch zoo een}{zoete lieveling}\\

\haiku{Een zware sigaar.}{bungelde losjes tusschen}{zijn kinderlippen}\\

\haiku{Annetje zat vlak.}{bij het raam inspannend een}{brok krant te lezen}\\

\haiku{Als God toch haar kind,.}{in het ongeluk stortte}{dan liever nog zo\'o}\\

\haiku{Als die meid niet van,.}{kersen en honing leefde}{dan deugde het niet}\\

\haiku{Nee, ze groef door heel.}{andere gangen in haar}{donker binnenste}\\

\haiku{Hoe had haar eigen.}{meisjeshoogmoed haar zelf niet}{geplaagd en gekweld}\\

\haiku{ze was, hoe ze den,}{slaap van zijn oogen roofde hoe}{krankzinnig verliefd}\\

\haiku{hij zou haar smeeken,...!}{all\'e\'en voor h\'em te leven}{alleen h\'e\'el alleen}\\

\haiku{duwde hij liever!}{zijn twee klavieren in het}{kokende water}\\

\haiku{Het was er stampvol.}{en de menschen zeurden rond}{hem de honderd uit}\\

\haiku{Nou zou hij nog even.}{op zijn bed blijven rekken}{en ren- gelen}\\

\haiku{Wat had hij de rooie.}{kanes van Mie nou toch vlak}{voor zijn oogen gezien}\\

\haiku{De meid had er een,}{kleur van gekregen toen ze}{het zag z\'o\'o prachtig}\\

\haiku{Manus kende geen,.}{guller hartelijker en}{meewariger vent}\\

\haiku{Waarom bleef hij bij?}{Joden Jet als hij haar toch}{zoo tegenwerkte}\\

\haiku{dat hij voor zichzelf,.}{niets te bedisselen had}{noch iets te ontzien}\\

\haiku{En Manua werd het.}{een troost te weten dat het}{toch nog geen herfst was}\\

\haiku{maar toch kon ieder.}{mensch een stemmetje geven}{als hij in nood zat}\\

\haiku{Een sterker zagend,,.}{gesnork als hoorde ze zijn}{spot was het antwoord}\\

\haiku{En weer geeuwde ze.}{vervaarlijk naar de kamer}{toe waar Manus zat}\\

\haiku{... suste Jet weer en {\textquoteleft}{\textquoteright}.}{meteen duwde ze zacht haar}{vangst de kamer in}\\

\haiku{Ze leek hem wel een.}{beetje erg beverig en}{zeer kort van begrip}\\

\haiku{Zijn heele klucht met.}{dit juffie kon hij morgen}{vergeten wezen}\\

\haiku{Corry was woedend.}{dat de Bochel haar niet eens}{had aangekeken}\\

\haiku{... herhaalde Corry,.}{giftig stampend dat ze niet}{verder kon komen}\\

\haiku{ze den vorigen.}{avond ieder voor tien gulden}{hadden afgezet}\\

\haiku{Corry leek zoo lang,,.}{als zij zag ze maar tienmaal}{mooier en blanker}\\

\haiku{Over alles ademde,.}{de losse wellustige}{bandeloos- heid}\\

\haiku{De koppelaarster {\textquoteleft}{\textquoteright},.}{wou voor de meid in hetstof}{knielen zei ze zelf}\\

\haiku{Ze beloofden haar {\textquoteleft}{\textquoteright}.}{eengoeden nacht en zooveel}{drank als ze lustte}\\

\haiku{De andere liep.}{in het grijs flanel met een}{groote bloem op zijn borst}\\

\haiku{- Zeg liefert,... hoorde,...?}{ze Gonda zeggen mag ik}{effe anschuife}\\

\haiku{- Steek jij maar je borst!...,... '!}{op joolde hij voort wantn}{kop heb je t\'och niet}\\

\haiku{De maintenee Fucht {\textquoteleft}{\textquoteright}.}{sprong plots bij en zong van de}{Happies days in Dixie}\\

\haiku{Ze begluurden hem,.}{ademloos maar waagden nimmer}{den kleinsten uitval}\\

\haiku{De neger had haar.}{opgehaakt en neergesmakt}{als een vuil pak kleeren}\\

\haiku{Het seil, mamselle,...,......}{is de Liefde en de Deugd}{en het anker}\\

\haiku{Ze knikte, maar in.}{stilte nam ze zich voor geen}{slok meer te nemen}\\

\haiku{De neger had zijn.}{heer weer doodomzichtig op}{een stoel neergezet}\\

\haiku{Joden Jet grijnsde,.}{en zong mee stomdronken van}{al de champagne}\\

\haiku{En fijner plaagde,:}{Corry terug overzeker}{van haar overwinning}\\

\haiku{En ieder moest zijn.}{rol worden toebedeeld zooals}{zij dat verlangde}\\

\haiku{Neen, ni\'et veel, nog,....}{niets meer dan afgepast kreeg}{ze geen achterdocht}\\

\haiku{as t'r \'e\'en manke '...!}{vlieg int land is gaat tie}{op jo\'u neus sitte}\\

\haiku{dat ze den vent moesten!}{rossen tot hij er half dood}{bij te krimpen lag}\\

\haiku{Nooit vroeg hij iemand.}{op zijn krot en ook nimmer}{bezocht hij een buur}\\

\haiku{Snikkend lei ze wat,.}{neer drukte hem de polsen}{en ging toen snel weg}\\

\haiku{- Soo achtbare heer,....}{spotte Karel op den knik}{van den souteneur}\\

\haiku{Een bloedstroom gulpte,.}{door het haar over de schouders}{en op de keien}\\

\haiku{Dan zouden ze een.}{potje rollen waarvan de}{lui konden roojemen}\\

\haiku{Maar die twee waren;}{van avond op nieuw avontuur de}{stad ingekuierd}\\

\haiku{- En se spinnekop!....}{sat nog op se das wist Guus}{Sand te vertellen}\\

\haiku{- W... \`at?... aarzelde,.}{Willem zijn spel-stomheden}{nog zelf niet overziend}\\

\haiku{- Je krijg nog 'n cent,...!}{fan d'r mag se morrege}{veur komme singe}\\

\haiku{Annemie was de,.}{buur van Dien en Blonde Kee}{woonde onder Nel}\\

\haiku{Hij kon geen brood meer,!}{verdienen dan moesten het maar}{de kinderen doen}\\

\haiku{Op het Oudekerksplein.}{bleef hij dralen zonder het}{zelf te beseffen}\\

\haiku{Treuzelend begon.}{ze te verhalen en haar}{stem zonk heesch in}\\

\haiku{op de lat. Cobus.}{wist precies van de kerels}{hoever hij kon gaan}\\

\haiku{Anne Looy's geelbleeke,.}{gezicht staarde weer den Dijk}{op zonder te zien}\\

\haiku{Joden Jet pakte,.}{weer in met het eeuwige}{stroohoedje op haar hoofd}\\

\haiku{Op Lou wachten,... nee,,.}{dat wierd te wisselvallig}{kon te lang duren}\\

\haiku{- Doch je d\`a 'k nie ' '?}{asn stammene\'e achter}{t vlot kan komme}\\

\haiku{- Doch je... doch je... d\`a ' ' '?}{kn salmpie l\'a uitsuige}{fann palinkie}\\

\haiku{Sterven vond Piet een,.}{ijselijkheid al hield hij}{niks van het leven}\\

\haiku{Anders zou hij wat,...!}{beleven om op die kist}{te durven zitten}\\

\haiku{Waarom de spullen?}{dan niet geborgen waren}{in kasten of zoo}\\

\haiku{Ze zou er koud van.}{hebben kunnen worden als}{ze dat al niet was}\\

\haiku{Toen ging ze, met een.}{hooge stem Karel overdreven}{hartelijk groetend}\\

\haiku{mi Moed, Beleid en... ' '!}{Trouw kompern mensch aln}{heel end achterop}\\

\haiku{Dien kreeg een hartbons.}{toen Karel met een ooiijk}{gebaar instemde}\\

\haiku{Frans Leerlap kwam met.}{zijn sleependen jichtvoet van het}{duivenplat hinken}\\

\haiku{Wat h\'ede suipt, falt,,!}{gistere in de Kuil Foor}{oud fuil Foor oud fuil}\\

\haiku{blauwkop,... lachte weer,....}{Karel en l\'a je vrou nie}{in de drek sitte}\\

\haiku{Ze wisten immers.}{niet eens wa\'ar en met wie ze}{gevochten hadden}\\

\haiku{Door een troep beesten,...!}{van kerels gehavend door}{meiden verraden}\\

\haiku{Ze vertelde dat,.}{er vijf door de politie}{waren opgepikt}\\

\haiku{Een paar uur later.}{ke\'erden de gedachten}{en plannen opnieuw}\\

\haiku{Zoo verbijsterd en.}{wild had  hij hem nog nooit}{om een deern gezien}\\

\haiku{Wa\'ar je haar roerde,.}{brandde een creatuur als}{hij zich de toppen}\\

\haiku{Zo\'o kroop hij den Dijk,}{weer op waar hij zooveel wind}{ving voor en aleer}\\

\haiku{- Soo'n warmte,... bromde,....}{hij ferbeel je je eige}{an de keerkringe}\\

\haiku{... stootte Simbad er,.}{nijdig uit die voorvoelde}{waar Manus heen wou}\\

\haiku{Ik wacht  op 'n,......}{bekende jonge h\`e die}{me in de flank falt}\\

\haiku{n lachie,... 'n rookie ',... '...}{enn babbeltje mitn}{schorpioen d'r op}\\

\haiku{Hoe zou nou zoo een?}{gouden vlinder h\'em ooit op}{de hand gaan zitten}\\

\haiku{- 't Varen was niks... ',....}{gedaann Beroerd beetje}{leven bromde hij}\\

\haiku{Zag hij nu heusch,?}{niet dat die deern geen portuur}{was voor een matroos}\\

\haiku{Hij wierd niet gauw week,.}{in het hart maar d\'at zou hem}{toch laten grienen}\\

\haiku{En 't skip nam toen '!}{n draai An de kop fan de}{handelskaai Skip \'ehooy}\\

\haiku{die vigeleerde!...}{als ze er lust in had of}{voor eigen nood moest}\\

\haiku{Wat schreeuwde, schermde,.}{of vloekte dat \'overschreeuwde}{of \'overvloekte ze}\\

\haiku{Ze deinsde voor niets,.}{terug al werd het een strijd}{op leven en dood}\\

\haiku{ze verlangde naar,.}{een handdruk een gebaar van}{mekaar-verstaan}\\

\haiku{Het bleef zacht in haar.}{roepen en kreunen naar}{181bevrediging}\\

\haiku{Even lachte ze, het,.}{bleeke mooie gelaat zacht-sarrend}{naar hem toegekeerd}\\

\haiku{Al wat danste, zat,.}{en stond rende naar den hoek}{waar gevochten werd}\\

\haiku{De blonde zou hij.}{op een andere manier}{wel inpeperen}\\

\haiku{Voor de zooveelste.}{maal dwong Karel den kerels}{bewondering af}\\

\haiku{We\'er gaf hij haar een.}{ruk dat ze nog lager voor}{zijn voeten uitzonk}\\

\haiku{mannen die leven.}{van het geld dat een vrouw met}{haar lichaam verdient}\\

\haiku{losse werklieden,.}{met een boekje van het Veem}{genieten voorkeur}\\

\subsection{Uit: De Jordaan: Amsterdamsch epos. Deel 4: Mooie Karel}

\haiku{De Jordanertjes.}{schreeuwden alsof de heele}{wijk in brand laaide}\\

\haiku{De heimelijke;}{v\'o\'orvreugd van Pinksteren moest}{in Luilak gisten}\\

\haiku{Want van die vlotschuit,,.}{7Mooie Karel hadden de}{rekels niet terug}\\

\haiku{En Alie, haar meid van,?}{zeventien lei die ook nog}{vadsig op haar stroo}\\

\haiku{Zijn laatsten spaander,.}{schoot hij uit als hij zag dat}{ze maar \'even grienden}\\

\haiku{Voor een paar dagen,...}{althans nijp-zorgen weg}{en kommer om Greet}\\

\haiku{Waarom poekelde?}{Karel eigenlijk niet met}{die kluif-gasten}\\

\haiku{je swarte kouse... '!}{en je rooje petoffels en}{m\'e\'et ope erf op}\\

\haiku{Want nou leken al,;}{zijn kracht en behendigheid}{ingetoomd geschoold}\\

\haiku{Bij Corry viel er.}{niets te redderen van haar}{intiemste wezen}\\

\haiku{Het bedrog van zijn.}{hart hield hij niet langer vol}{tegenover Corry}\\

\haiku{Mooie Karel kende.}{geen beuzelingen en geen}{onverdraagzaamheid}\\

\haiku{- Slurp je m\'e\'e, 'n slok,...?}{uit me flesch pofsak of bi je}{fol op je borsie}\\

\haiku{n natte klap op!}{je kruintje geife d\^a je}{ooge uitsplintere}\\

\haiku{- Die komp bij mijn niet,... '!}{goed dan haal ikm tug s\'o\'o}{se hart uit se lijf}\\

\haiku{- Fan je mollege... -!...}{arrempies En ikke fan}{je lekkere grens}\\

\haiku{Karel hield zijn grauw.}{petje op en bedelde}{den kring verder rond}\\

\haiku{Karel stapte weg.}{en bekommerde zich nauw}{meer om den zatlap}\\

\haiku{Thijs snurkte zwaar als,;}{een dronken bruigom in de}{duistere bedste\^e}\\

\haiku{s Morgens al om.}{klokke-vier schoot Karel}{met een schrik wakker}\\

\haiku{Maar even daarna wist;}{hij niet eens meer waarom hij}{was losgebarsten}\\

\haiku{Greet, al dieper en,.}{schrijnender van spijt gekweld}{sou maar naar bed gaan}\\

\haiku{Ze schenen compleet.}{engelen van goedheid en}{inschikkelijkheid}\\

\haiku{Ze was van boven,...}{tot onder pluim en bel. Ja}{een pantemime}\\

\haiku{Greet voelde zichzelf,,!}{zoo ernstig zoo schrikkelijk}{schrikkelijk ernstig}\\

\haiku{En waar de schat lei,.}{daar kronkelden helblauwe}{vlammen uit den grond}\\

\haiku{En de klantjes van.}{de Frans Halsstraat hielp zij schuw}{en gedachteloos}\\

\haiku{Doordringender en.}{nijpender keek de malle}{kastelein hem aan}\\

\haiku{De commissaris.}{preekte saai en gluurde loensch}{naar vader en zoon}\\

\haiku{Tweemaal per week moest.}{hij een heelen avond bij hem}{komen doorbrengen}\\

\haiku{Had hij nou maar een,...}{eindje cigaret om te}{trekken te zuigen}\\

\haiku{Hij tikte Zacht, iets,,:}{harder nog iets harder t\'ot}{hij zekerheid kreeg}\\

\haiku{Hij vreesde van de,.}{dieven en inbrekers om}{hem heen geen verraad}\\

\haiku{Zij stond te snikken:}{voor den commissaris en}{jammerde al maar}\\

\haiku{En nog was er geen!}{koopster te vangen onder}{een strooien dakje}\\

\haiku{Je laat 'r je schaats,,!...}{je hengels je danse en}{je duife foor staan}\\

\haiku{En tartend met zijn,,:}{heerlijk-buigzame hooge}{mannenstem zong hij}\\

\haiku{ferslikke se d'r '!}{eige fast mitn falsche}{guide in d'r keel}\\

\haiku{En madam Bedil,.}{was z\'elf een gifmengster als}{het er op aankwam}\\

\haiku{Greet had een brief in,;}{de  hand pas door den post}{haar schoot opgegooid}\\

\haiku{Guns, wat had ze met,!}{zoo een blok aan haar been aan}{haar jonge leven}\\

\haiku{Heel de Jordaan was,.}{op de been nog kooplustig}{in de smoorhitte}\\

\haiku{- Bij jo\'u jonke de ',?}{pietermanne achtert}{stille luikie h\`e}\\

\haiku{Meiden en kerels,.}{bulderden van het lachen}{tierden en spotten}\\

\haiku{De Goudsbloemdwarsstraat,;}{als kruispunt ving van allen}{kant het buurtrumoer op}\\

\haiku{- Kopsorg... as 't '!}{n slachterij wordt bin ik}{d'r tug s\'o\'o te piel}\\

\haiku{Stil draaide hij zich.}{om en stapte voet voor voet}{het kamertje in}\\

\haiku{Karel weerde het.}{wellustige gesmeek der}{meiden boertig af}\\

\haiku{En hij danste op,.}{zijn stoepje Greet woest in een}{warrel meetrekkend}\\

\haiku{En even, tusschen de,:}{bekoorlijke tandjes zong}{zij nazinnend}\\

\haiku{... jammerde Teun uit,,...}{bed naar Ant die stil op een}{kruk was neergesakt}\\

\haiku{Hoe lam-ellendig.}{voelde Karel zich onder}{al de scherts en boert}\\

\haiku{Gebbetjes van 'n,......}{kitsige adviseerde}{Poort langs den neus weg}\\

\haiku{Harmen, ja di\'e had;}{hem eigenlijk naar alle}{slechtigheid gejaagd}\\

\haiku{- 's Morgens kwam de.}{cipier en we hoorde de}{sleutels rinkele}\\

\haiku{Hij rook niks anders.}{dan poffertjes en wafels}{van een kermiskraam}\\

\haiku{Zoo een halfblanks vent,.}{was waard dat hem de duim in}{de oogen gedrukt wierd}\\

\haiku{Frans zag de wanhoop.}{en het kwaad geweld op zijn}{verwrongen tronie}\\

\haiku{Het bleek niets anders.}{dan een verachtelijke}{baantjesjagerij}\\

\haiku{Op een dag liet de'.}{Rechter-Commissaris}{Frans vader roepen}\\

\haiku{Hij smeekte Frans en.}{hij wroette in de pijn van}{zijn onrust en angst}\\

\haiku{Dan kon hij ieder.}{woord scherper en krenkender}{van scherts uitstooten}\\

\haiku{Die schepte nooit op,.}{deed nooit dik en speelde niet}{met een zakdoekje}\\

\haiku{Hun wandelstokjes.}{plakte hij in een hoekje}{166achter de deur}\\

\haiku{- Als we nou aan de,!....}{wieg stooten zijn we geknipt}{fluisterde Frans weer}\\

\haiku{Maar de spanning hield.}{nijpend het bezetene}{vreugdegevoel vast}\\

\haiku{Daan Blikkie zuchtte,.}{dat di\'e gedwongen kalmte}{zenuwkracht verslond}\\

\haiku{Frans leefde weer in.}{een koortsroes en onder een}{heete betoovering}\\

\haiku{- Ik wou maar d\^a 'k '...!}{jo\'u inn lijsie had kon}{ik je ophange}\\

\haiku{Plots overduizelde.}{hem weer een stoutmoedige}{geluksgedachte}\\

\haiku{as me fader 'n,}{bok slacht mit dolle kerfel}{mag je me raje}\\

\haiku{en je frommes breit '...}{r hier d'r trui ferder en}{je kooters gaan}\\

\haiku{- Hij krijgt op se jak,...,...!}{soofeul astie lust Trui foor de}{goksie maar hij lust niks}\\

\haiku{Maar telkens lachte,.}{Karel weer zijn heerlijken}{uitdagenden lach}\\

\haiku{Wat een heerlijke,.}{heupen had die meid om haar}{zoo \'op te tillen}\\

\haiku{hei-je niks mee te...!}{make en slobber alleenig}{maar je taskef\'ee}\\

\haiku{Dat had toch als een?}{menschelijk voornemen in}{hem rondgezongen}\\

\haiku{Een kille windvlaag.}{suizelde huiverig door}{de nachtstilte heen}\\

\haiku{Nou moest hij maar niet.}{zijn kousen verzolen en}{zichzelf bedriegen}\\

\haiku{Hij zou wel kunnen,!}{kuiltjes-schieten met}{knikkers van pleizier}\\

\haiku{Waarom wierd zijn loop,?}{nu een stoffig slenteren}{slof-slof-slof}\\

\haiku{Vroeger wipte hij.}{dat vogelnestje op zijn}{hand als een draaitol}\\

\haiku{Dat wisten Thomas......?...}{Slokkebier toch wel en Jan}{de Gans en Hompie}\\

\haiku{Nu schepte zij voor,.}{een iegelijk op borden}{gort-met-stroop}\\

\haiku{- Soo'n Schoppeboer... hei '!}{k nog nooit-nie in me}{klefiere gehad}\\

\haiku{Er dreigde iets om.}{hem heen in de lucht als bij}{onweer-op-komst}\\

\haiku{Karel doorzag de:}{omhulling van haar jaloersch}{spel en mompelde}\\

\haiku{De Afslager, een,.}{vriend van Karel keek hem met}{ontzette oogen na}\\

\haiku{Op de Lindengracht,,.}{tusschen zijn kameraden}{wierd Karel begekt}\\

\haiku{Moest misschien ruimte,...}{gemaakt met een opsteker}{e\^er gaf het geen lucht}\\

\haiku{Begon hij weer te?}{razen en te schimpen in}{bezeten tarten}\\

\haiku{Wat verlangde hij?}{nog tusschen de joeldrukte}{van de Noordermarkt}\\

\haiku{Terwijl hij alleen,.}{snakte naar het vreemde naar}{het onbekende}\\

\haiku{Hij schold zichzelf nog.}{een proper maagdelijntje}{in de tiejeiskraak}\\

\haiku{Het mistte zwaar en.}{grauw op den laten middag}{van de insluiping}\\

\haiku{Frans had een gevoel;}{alsof zij zich telkens op}{hem wilden werpen}\\

\haiku{heel het heldere.}{licht scheen nijpend gedoofd in}{haar beschroomde oogen}\\

\haiku{dat hij overeind moest,!}{rijzen opdat hij hem uit}{elkaar kon scheuren}\\

\haiku{de heele Jordaan;}{beefde en rilde voor zijn}{zwijgende zinning}\\

\haiku{En weer verklaarde,;}{Frans dat hij hem alles zou}{vertellen alles}\\

\haiku{Natuurlijk over jouw...}{link liefdesspel met meide}{van allerlei slag}\\

\haiku{Maar toch lei er een.}{nieuwe vraag te martelen}{op Karel's lippen}\\

\haiku{dat Corry van h\'em,.}{misschien wel hield maar Karel}{Burk alleen liefhad}\\

\haiku{Dat, d\^at vooral moest.}{Karel het allereerst zijn}{knar inhameren}\\

\haiku{Zijn ouders hadden,.}{hem schrikkelijk bejammerd}{maar zichzelf nog meer}\\

\haiku{een heel mooi vrouwtje.}{met een droomerig-zacht}{meisjesgezichtje}\\

\haiku{als een vermomden,?}{duivel in priesterrok nu}{hij niet meer inbrak}\\

\haiku{Hij lanterfantte,,,;}{hij danste speelde zong voor}{de tippelaarsters}\\

\haiku{bij alle twee, een.}{even hevig en vlijmend in}{het leven kerven}\\

\haiku{Maar Karel, tot Frans',.}{vinnige verbazing vroeg}{nauwelijks meer wat}\\

\haiku{Wat verlangde die?}{troebele zwammer toch van}{hem en met welk recht}\\

\haiku{Nou kom ik mit de...}{heele schutterij-kep\`el}{op theefesite}\\

\haiku{Burk rilde voor de.}{allergemeenste groeven}{om Poort's mondhoeken}\\

\haiku{Nou kwam Poort met de.}{bijl in de hand en joeg hem}{de inspringer uit}\\

\haiku{behallefe de... -!....}{ferstaancente Bink is-ie}{gierden de meiden}\\

\haiku{Hij was vanavond weer,.}{in een verschrikkelijke}{ellendige bui}\\

\haiku{Maar in\'e\'en sloeg zijn,.}{chagrijn \'om in een dolle}{grijnzende dreigpret}\\

\haiku{Maar in een vrijen;}{karakterdans kon Karel}{veel eruit gooien}\\

\haiku{Burk liep heen en weer,,.}{onder een schuwe schaamte}{den kop ingebukt}\\

\haiku{Karel huiverde,...:}{en toch daemonisch gromde}{hij er tegenin}\\

\haiku{Het woede-licht.}{in zijn z\'elf nog bloedende}{oogen scheen uitgebluscht}\\

\haiku{339Ze stemde in,.}{maar alleen wanneer de twee}{kerels meestapten}\\

\haiku{Het wierd over en weer,.}{spreken tot het bittertje}{erbij te pas kwam}\\

\haiku{In den donkeren.}{stal waschte Thijs het af}{en voedde het goed}\\

\haiku{Ook met den nieuwen.}{compagnon kreeg hij heibel}{en wierd het knokken}\\

\haiku{Eindelijk geheel {\textquoteleft}{\textquoteright},.}{los strompelde hij weer bij}{zijn eega binnen}\\

\haiku{Hij konkelmonkte '!...}{t gebefte gajes en}{de affekate}\\

\haiku{Het geschal scheen hem.}{onvoorziens een oogenblik}{te ontnuchteren}\\

\haiku{In Caf\'e {\textquoteleft}Marktzicht{\textquoteright}.}{joelde opgewonden de}{avond-Kerstdrukte}\\

\haiku{Uitkijk, een makker,.}{hield de politie links en}{rechts in de gaten}\\

\haiku{Greet, Alie en Sientje,.}{hadden er voor gezorgd en}{zelfs Thijs wou meedoen}\\

\haiku{Het liefelijke,.}{en argelooze er in bracht}{haar geheel van streek}\\

\haiku{Karel voelde haar.}{adem heet beven tegen zijn}{trillende handen}\\

\haiku{Het was alsof de.}{heele Jordaan meejuichte}{met hun vereendheid}\\

\haiku{In Corry was het,.}{wonderlijke gebeurd dat}{Poort niet had bevroed}\\

\haiku{nu was Corry niet...}{op Pinksteren maar op Kerst}{tot hem weergekeerd}\\

\haiku{iemand die al op.}{jeugdigen leeftijd in de}{gevangenis zat}\\

\subsection{Uit: Levensgang: roman uit de diamantwerkerswereld}

\haiku{Dan heb ik ni\'et,;}{meer te spreken over maar van}{menschen en dingen}\\

\haiku{In z'n onderbroek,,.}{was Hein zacht met ingeho\^uen}{vaart opgesprongen}\\

\haiku{Een vuurspat van.}{woest lucifers-gestrijk}{brak door haar kijfvaart}\\

\haiku{Zou je niet z\'oo  ,?}{wegloopen en den heelen}{boel laten stikken}\\

\haiku{Zijn baas zou dan \'o\'ok.}{zien dat hij die vodden niet}{langer dragen kon}\\

\haiku{- Late we'm na de, ' '!}{kantien drage mitn slok}{bier is ier op}\\

\haiku{- Had jelui nou die?}{dolle schreeuwleelik daar niet}{kenne tegehou\^e}\\

\haiku{De benauwing van, '}{voorspellingen te hooren}{weer druktem al.}\\

\haiku{En as die 'n groote,,.}{smoel zet die jood dan l\^a je}{je niet afbluffe}\\

\haiku{Dat voelde ie ook.}{veel beter passen bij z'n}{heldenkarakter}\\

\haiku{Toen begon er als '.}{vanzelf plotsn zacht-innig}{spel in z'n zieltje}\\

\haiku{Het leven thuis met ' '.}{z'n o\^uers en broers wasmn}{gruwel geworden}\\

\haiku{In kleinigheidjes,.}{had ie wel veel gejokt maar}{altijd voor zichzelf}\\

\haiku{Hein zat, zat zonder ',.}{n woord te spreken met z'n}{hand onder de kin}\\

\haiku{Smoorheet blakerde ' '.}{t kacheltje int eng}{krotje om zich heen}\\

\haiku{Woedend was Hein dat '.}{z'n moedert arme kind}{bij d'r scheldnaam riep}\\

\haiku{as 'k de fasch breng... ' '...}{soene se me rejoal}{t isn merakel}\\

\haiku{Giftig voelde ze '.}{zich worden bijt zien van}{Heins koppig volhou\^en}\\

\haiku{Bij elken nieuwen.}{opschep duwde ie z'n bord}{verder van zich af}\\

\haiku{Z'n lippen voelde,.}{ie aan kurkdroog korrelig}{van vleeschbultjes}\\

\haiku{dreigend gekijk van,.}{aanhangers nog dreigender}{van tegenstanders}\\

\haiku{in zwaar rumoer en:}{dreunend geweld werden z'n}{woorden afgehakt}\\

\haiku{Scherp-bevend',:}{klonk Veeges stem losgescheurd}{uit z'n heeschheid}\\

\haiku{sentimenteel bi, '!}{je dat ken je van mijn op}{n briefie krijge}\\

\haiku{Bakounine mot, -, -!}{je leze kon je maar fransch}{en Kropotkine}\\

\haiku{Toen, inspannend z'n,.}{wil had ie ook dat nare}{gevoel weggedacht}\\

\haiku{- Goeie heer, met wat 'n.}{vorsten-branie staan ze}{je n\`et even te woord}\\

\haiku{Maar dat onder z'n ',!}{makkerst leven zoo duf}{zoo uitgesuft was}\\

\haiku{Dan had ie pas voor.}{zich gezien brokken wei aan}{de Utrechtsche zijde}\\

\haiku{Ja, maar zooveel groote '.}{lui vant geloof hadden}{dat toch \'ook gedaan}\\

\haiku{Hij wist zelf niet of ' ' ';}{iet vann man of van}{n vrouw verlangde}\\

\haiku{- Lepper, van middag,,,?}{bin je toch af en u ook}{meneer Spauer niet}\\

\haiku{- Zes h\`arde h\`o\'ek-k\`e... '...,,......}{h\`etn heel st\`e\`entje door}{z'n d{\`\i}ks door z'n pl\'ats}\\

\haiku{- Adden\`om blijf staan, maak - '.}{me geen m\`erchel17 schreeuwde}{ier dreigend toe}\\

\haiku{- N\`oh d\`ammaar, hou u '... '!}{t dammaar \`af t\`och laatk}{me niet afzette}\\

\haiku{Lepper, Sprauer, Mopens,,, '}{Remeni Mierikstein enn}{hoop meisjes drongen}\\

\haiku{- vroeg Pronkman op den, '.}{man af in dreig-houdingn}{antwoord afwachtend}\\

\haiku{- laat je moeder je '!}{maar iedere dagn paar}{eiere klosse}\\

\haiku{Intusschen hadden.}{veel kijkers geprobeerd den}{tiltoer na te doen}\\

\haiku{Maar 't gestommel,.}{geschuif en gevloek hield aan}{zonder resultaat}\\

\haiku{Bleekman was plots naar ';}{m toegestapt om in z'n}{nota te kijken}\\

\haiku{Al de chipswerkers;}{klaagden mee en vertelden}{van hun blindkijken}\\

\haiku{Heviger draaide ',.}{en rommeldet in z'n}{keel boven z'n maag}\\

\haiku{stil bleef ie even staan,,;}{zonder kaakbeweeg met de}{kauwvracht in zijn mond}\\

\haiku{Rooier bestoof de;}{jagende kongestie z'n}{kop in bloedaandrang}\\

\haiku{moest ie zich nou door?}{dien beroerden kerel voor}{den mal laten ho\^uen}\\

\haiku{- Meneer Broos, daar is '...,... '}{n beheime35 om je te}{spreke oggen\`ebbiech}\\

\haiku{verlegen wou ie,,.}{wegkruipen de gang in om}{z'n woesten uitval}\\

\haiku{hij voelde dat ie.}{geen houvast meer zou hebben}{aan z'n eigen drift}\\

\haiku{iets dat afstootte.}{in voorname meerderheid}{en toch te\^er meeging}\\

\haiku{Eindelijk had Hein.}{weer z'n geluid losgeschord}{uit schokbedwelming}\\

\haiku{hoe zou\^en ze in hun,.}{vuistje lachen als Hols nou}{weer werd afgemaakt}\\

\haiku{Eva triumfeerde.}{in zichzelf om den scherpen}{koelen trots van Hein}\\

\haiku{Ze voelde, na zoo'n.}{uitgescheurde smart niets meer}{te moeten vragen}\\

\haiku{lief klonk haar stem weer,,}{in doordringend gevraag wat}{ie dan bedoelde}\\

\haiku{Heel stil was ik mit '}{m d'r-naa-toe gegaan}{en opgewonde}\\

\haiku{Maor je seit, d\^e.... -,, '!}{je mit juff'r E\'ef\'a\'a Ja stil}{nou daar k\`omk op}\\

\haiku{Dol verlangend was,;}{Hein in\'e\'ens te weten}{wat ze dan wel dacht}\\

\haiku{Bij Bresser was 't.}{heele personeel al drie}{weken werkeloos}\\

\haiku{Hij had iets dors, iets, '.}{stroefs gevoeld plotseling dat}{eerst niet inm was}\\

\haiku{Toch wou z'm nog even, '.}{d\`oorplagen al meende ze}{t meeste ervan}\\

\haiku{Z'n oogen traanden van,.}{geestdrift en z'n mond lachte}{breed en bezielend}\\

\haiku{- Nee juffrouw, u weet ', '...}{niet hoeveel wij naat mooie}{t hooge verlange}\\

\haiku{omdat 'k uit me.....}{zelf heelemaal verlang naar}{wat mooi en goed is}\\

\haiku{Al z'n drukte en;}{woordgespat bestierf op z'n}{lippen aan haar zij}\\

\haiku{, met strak voorbijzien, '.}{van Sak wel wetend dat ie}{m daarmee pestte}\\

\haiku{Met z'n hand stond ie, '.}{klaar en Bresser hadm nieteens}{goeiendag gezegd}\\

\haiku{'n Kristen schoonschoon ',!}{vond ien ideaal en z'n}{eigen kind dokter}\\

\haiku{Want de klub sprak van, '.}{z'n huisgezin als vann}{muzikaal wonder}\\

\haiku{hoe as 't toe-d'r!}{tijd wat \`anders was as van}{d\'aag-des-daags}\\

\haiku{ze verzuipe de, '!}{boel toch of verteret}{an de m\`eide}\\

\haiku{jij h\`et jou porsie,,!}{dubbelt en dwars gehad maar}{ik ik bin nog jong}\\

\haiku{Met  de hitte '.}{vant gesprek golfden \`op}{weer de dierkreten}\\

\haiku{'t publiek voorover.}{boog en dan in beefbeweeg}{verdween in klokzij}\\

\haiku{- en plots zich bewust,:}{wordend schorde ie nijdig}{met zwaardere stem}\\

\haiku{'n verschwartzter nar,!...}{die elkeene do\'odslage}{wil mit ze ch\`ente}\\

\haiku{Jij bent zoo, blijft zoo, '.}{al zou je voor haar doort}{vuur willen vliegen}\\

\haiku{Buiten Eva, die nog,.}{niet beneden was zat de}{heele familie}\\

\haiku{in de tragedie '.}{vann landleven dat ze}{voor zich zag bloeden}\\

\haiku{wat roerend verteld '.}{vann leven waar ze nooit}{flauw aan gedacht had}\\

\haiku{Daar zat ze weer in,;}{haar kamer met allemaal}{weelde-dingen}\\

\haiku{Op stoelen en schoot.}{had ze dan eerst den vollen}{bloemschat uitgestort}\\

\haiku{Als ie 't maar niet, ';}{dwaas vond dat verbergen en}{half schuilgaan vanr}\\

\haiku{ik wil niet hebbe ',?}{meer dat je werklui opt}{kantoor laat verst\`aan}\\

\haiku{en je weet hoe 'k ', '!}{r de pest an heb ann}{vreemd op m{\`\i}jn kantoor}\\

\haiku{je eige vader, '!}{afstaan voor zoo'n arreme}{gojn versteller}\\

\haiku{dat as je 't hart...}{h\^et w\'e\'er die kerel op m'n}{kantoor te neme}\\

\haiku{... zal 'k verrekke,!}{Eef die h\`et-je je kop}{mechogge gemaakt}\\

\haiku{Stil liet Bresser zich.}{op den stoel zakken waarop}{Eva gezeten had}\\

\haiku{Hij barstte, hij had ',,.}{r wel kunnen worgen maar}{hij kon niet k\`on niet}\\

\haiku{En hoe dol blij zal, '.}{Hols zijn als ie hoort dat ze}{eens metm mee wil}\\

\haiku{- Och Zeelt, vertel me, '...}{maar niet verdert maakt me}{zoo gloeiend driftig}\\

\haiku{anders gaat ie me...}{nog op de late avond j\'ou}{en mij anblaffe}\\

\haiku{- en weer schoffelden.}{z'n stappen weg zonder dat}{ie Hein gezien had}\\

\haiku{Vreemd-lekker had '.}{ie nou es uitgeslapen}{alsn gewoon mensch}\\

\haiku{Alles wat ie 'r,.}{wou zeggen stokte in z'n}{keel wou niet verder}\\

\haiku{Ja, ze voelde dat ';}{in dien man heelemaalt}{kind gebleven was}\\

\haiku{en ik ben juist heel...... -,...}{laat opgestaan heusch O}{nee juffrouw dat treft}\\

\haiku{dieper in te gaan '.}{opt socialisme}{aan alle kanten}\\

\haiku{haar onzedelijk.}{gewoeker en vernielen}{van menschenlevens}\\

\haiku{Waarom durfde ze ' '?}{niet gewoonn wandeling}{metm te maken}\\

\haiku{ie weer vrijer, en.}{terug leefde ie weer den}{tijd van extaze}\\

\haiku{Rozalie was even '.}{stil blijven staan en keekn}{donkere steeg in}\\

\haiku{Als j't maar uitzegt......}{met eigen temperament}{is alles gloednieuw}\\

\haiku{ik geloof dat de......}{juffrouw ziek is z'n dochter}{die in de zaak is}\\

\haiku{Maar och... z{\`\i}j wou 't....}{toch ook niet dan had z'm toch}{wel laten roepen}\\

\haiku{Banger joeg 't door'm, '.}{heviger dat ie n\`og niet}{wist hoet met'r was}\\

\haiku{En sterker schreide ' ' '.}{t inm dat ier niet}{dadelijk mocht zien}\\

\haiku{Nog pas had ie 'r,.}{gezien vol leven en nooit}{was ze ziek geweest}\\

\haiku{je weet, 'n gojsche147!}{honderd gulde is meer dan}{e jiddische148 tauzend}\\

\haiku{Toen ie de kamer, '.}{uitging zag ie onder aan}{t trapje Hein staan}\\

\haiku{dan maar eerst van zich,,...}{afschuiven dat die dooie daar}{die stille Eva was}\\

\haiku{ie Zeelt nu in z'n,,.}{warm geduldig opporren}{in z'n taai aanhou\^en}\\

\haiku{toch wou ik je in}{het kort iets verklaren dat}{verklaring eischt}\\

\haiku{Nu kan ik me al.}{levendig voorstellen wat}{jij antwoorden zult}\\

\haiku{je weet niet wat een,...}{toer het voor mij is om daar}{nog in te komen}\\

\haiku{- Leg jij je mo\^er 'n,,!}{nijfie rotsodemieter}{ik ken soo nie voort}\\

\haiku{Soo as hij 't ons, ' '!}{uitgelege h\`et kann}{kindt begrijpe}\\

\haiku{... nou most 'k enkel,?}{nog maar vier hoog afsakke}{om te fluite seg}\\

\haiku{hoe meer g\^ojjem hoe! -.}{meer de mazele driftte}{Kiehl er tegenin}\\

\haiku{verleuje weuk as '... -!...}{k hier geweus binne Leg}{niet te o\^uehoere}\\

\haiku{... ik neem 'm, as ie,!}{velooht is op slag therug}{teuge zes ghulde}\\

\haiku{Schatergelach en '.}{stemmengespot omgierde}{t worstelgroepje}\\

\haiku{in hem was die stem ', ';}{gaan klinken doort leven}{t volle leven}\\

\haiku{Want zoo hoog als ging,}{de galm-golf van hun}{opstandingslied zoo}\\

\subsection{Uit: De Jordaan: Amsterdamsch epos. Deel 3: Manus Peet}

\haiku{- Hier bifakeer ik'... '!}{m eige hier enter ik}{teget want op}\\

\haiku{Hier peleton, hier,,...!}{sta je Manus mit al je}{fisemente bens}\\

\haiku{Hij verafschuwde,.}{de hitte de schroeiende}{huizen en straten}\\

\haiku{Hij jammerde en.}{perste de ellende zijn}{menschelijk merg uit}\\

\haiku{Dat hiette met de,....}{grove bijl er inhakken}{mompelde Manus}\\

\haiku{de mensch die God wou,....}{naderen je naderde}{alleen de wormen}\\

\haiku{hij dacht aan den dood,.}{hij van binnenuit zich als}{leeggeschept voelde}\\

\haiku{wat beseffen wij,,?}{tijdelijke schepseltjes}{van het tijdelooze}\\

\haiku{deze menschen zijn,.}{al niet meer terwijl zij toch}{nog spreken en gaan}\\

\haiku{- Heil en sege in ',......}{t ouwe mompelde Peet}{zich sarcastisch toe}\\

\haiku{een mensch moest als de,.}{wind ongezien verschijnen}{en ongezien gaan}\\

\haiku{Dat was nu  zijn,.}{eerste kaars die branden zou}{zonder sprankeling}\\

\haiku{En dadelijk het.}{wreed-tartende in haar}{krenkenden spreektoon}\\

\haiku{Waterverf Manus,,.........}{alles dun spoelsel jawel}{overmorgen ziet u}\\

\haiku{Corry, vlijmend in,,;}{haar hoon haar stukscheurende}{bloedige ironie}\\

\haiku{En toch wilde hij.}{weten wat Corry van hun}{tweetjes mompelde}\\

\haiku{Wier naam bij vele,...}{was bekend Se liet sich in}{mit vuile dinge}\\

\haiku{- De lichte schaduw {\textquoteleft}{\textquoteright}.}{van hetGulden Boekske valt}{over zijn zoldertje}\\

\haiku{Zoo vurig bonsde.}{en ziedde het bloed niet meer}{in den naspruit Peet}\\

\haiku{Doch w\'o\'orden duidden,.}{aan bij rechtzinnigen en}{bij vrijzinnigen}\\

\haiku{heulegaar \`onder...... -?}{de serrekies binne se}{geschaakt Serrekies}\\

\haiku{Mit 'n daai en 'n '...}{soeslap geef je sen trap}{feur hullie kiese}\\

\haiku{- Gut, aume,... ontviel,...!}{hem plots je sit hier s\'o\'o mit}{de dooje an tafel}\\

\haiku{hij zoo een beetje.}{naar zijn buurt en drentelde}{dan kalmpjes tot huis}\\

\haiku{Nou het zoo in de,.}{praat te pas kwam mocht hij het}{bestig vertellen}\\

\haiku{Geen jongen waagde.}{zich zoo hachelijk op de}{nauwste dakgoten}\\

\haiku{De kleine zusjes.}{en broertjes beefden voor het}{gezag van Bromtol}\\

\haiku{Ook had hij al lang,.}{in de kluisgaten dat die}{drie h\'em noodig hadden}\\

\haiku{Dan stotterde Nel,:}{gebluft door het uitblijven}{van Dirkje's verweer}\\

\haiku{Turrefeglomsel ', '.}{inn testje Veur me moes}{t kwaje besje}\\

\haiku{Met een angstdriftkreet.}{was hij dien dwarreltol op}{de keel gesprongen}\\

\haiku{Zij hielden liever,.}{een flikflooier die gaapte}{dan een vent die beet}\\

\haiku{Ook de blik- en.}{gaslucht maakte hem ziek en}{sloom-onderworpen}\\

\haiku{Hij verdroeg nauw den.}{spotschimp van zijn maats om zijn}{vlammige bakkes}\\

\haiku{Jan Gouwenaar zag.}{allerlei vreemde streken}{en vreemde landen}\\

\haiku{toen naar Malaya en,,.}{Barcelona en toen naar}{Oneglia in Itali\"e}\\

\haiku{En 't Skip nam to\'e ',!}{n draai An de kop fan de}{Handelskaai Skip ehooi}\\

\haiku{Je kon die radde,.}{kerels die mastlieren toch}{nooit overk\'akelen}\\

\haiku{die balanceerden;}{zoo een mooi fonteintje op}{hun gesjochten neus}\\

\haiku{a\^ars krijg je 't op... -,!}{je graatje Olie jij maar je}{kruiskoppe jobstraan}\\

\haiku{- Van mijn spekkertje, ' '!}{n goud maffie ofn pot}{lood d'r bovenop}\\

\haiku{zoo zoetjes en zoo,...!}{fijn-beverig door den}{stillen hemel o}\\

\haiku{Toen gelastte hij.}{Jan driemaal om zijn bullen}{weer aan te trekken}\\

\haiku{Jan zonk kermend in.}{elkaar en werd weggebracht}{naar de buitencel}\\

\haiku{enne... behoorde!}{zijn stamva\^ar nog wel tot het}{Utrechtsche Kapittel}\\

\haiku{... snauwde die loods, die.}{geen aasje sjoege had van}{wat hij bedoelde}\\

\haiku{Werom komp gij nie,?}{as feur deise Bij mijn in}{mijn schape-kouw}\\

\haiku{Ook in dat muffe.}{hol zat hij weer weken aan}{weken te fronsen}\\

\haiku{Me legge de man '.}{n kwartje uit Bij Piet in}{de Kopere Tuit}\\

\haiku{- De mismaakte stelt.}{den volmaakte eischen en}{peilt de jaloezie}\\

\haiku{En andersom moest.}{zijn vrouw van en in hem dit}{precies zoo begeeren}\\

\haiku{Simon Twei-Duim,,,}{Kau de Reus Hein Uilje mit}{Na de Neus Auk Piet}\\

\haiku{En toch, het kookte,.}{en gistte in Peet gelijk}{nimmer te voren}\\

\haiku{problemen  die.}{alle toch tot daden moesten}{worden \'omgeleefd}\\

\haiku{zoodra ze de?}{onpeilbare diepten van}{muziek niet verstond}\\

\haiku{Camille hoonde.}{en schimpte openlijk in zijn}{vurige pamphletten}\\

\haiku{en nog erger van,,.}{hun lach hun vertier van hun}{verdierlijkend kroost}\\

\haiku{En eeuwig door, een.}{frissche klatering van het}{eindelooze water}\\

\haiku{Doch ook, in reine,:}{overgave en geloof vroeg}{hij zich telkens af}\\

\haiku{Alle somberte.}{en verdriet waren in\'e\'en}{uit hem weggelicht}\\

\haiku{Manus zwol van trots.}{en tegelijk kromp hij in}{van nederigheid}\\

\haiku{Zij hoorde hem weer.}{laat in den nacht spelen op}{zijn harmonika}\\

\haiku{- Als je zoo begint, ',....}{smeer ikm op slag dreigde}{Corry minachtend}\\

\haiku{Maar Manus liet zich.}{niet uit het spoor lichten door}{verdekte tweespalt}\\

\haiku{De goudvlammige.}{hoofddos krinkelde los langs}{ooren en wangen}\\

\haiku{Zijn diplomatiek.}{Mongolensnuit deed niets dan}{gluren en loenschen}\\

\haiku{Hij wou zelf zien, ho\'e,,;}{beperkt ook zelf oordelen}{ho\'e onjuist misschien}\\

\haiku{Telkens als Manus,}{van Bakoenine wat las}{dan was het alsof}\\

\haiku{Hier en daar had hij,.}{inrichtingen afgelensd}{zonder veel behaai}\\

\haiku{l\'a-je dan vast voor!}{engeltje oproepe bij}{Onseliefeheir}\\

\haiku{Het blauwe maanzaad,.}{smaakte compleet als mosterd}{zoo scherp en giftig}\\

\haiku{Nu, in het voorjaar,,.}{eind Mei begon de herrie}{hel \'op te laaien}\\

\haiku{Die werd vast majoor,!}{van de ratelwacht als hij}{zoo voortslofte}\\

\haiku{De dame legde,.}{Frans den stamboom van haar hond}{voor dien zij kwijt wou}\\

\haiku{Daarom voelde zij,.}{zich verplicht iets vriendelijks}{te moeten zeggen}\\

\haiku{Nel had er nu weer,.}{een kind bijgekregen een}{wolk van een jongen}\\

\haiku{Zij flapte er nou,.}{eenmaal alles uit wat haar}{jeukte op de tong}\\

\haiku{Nel vond het toch maar,.}{fijn dat zij zich voor het eerst}{zoo liet bedienen}\\

\haiku{- Om te griesele,....}{viel eigendunkelijke}{Mie Nat rillend bij}\\

\haiku{Frans leek een beetje.}{verbaasd dat Corry nog niet}{was thuisgekomen}\\

\haiku{... vroeg Frans plots ontzet,.}{toen hij Eenpoot zag zuigen}{op zijn linkerhand}\\

\haiku{Gisteren had hij,;}{ook nog een paar schoothondjes}{verkocht heel prijzig}\\

\haiku{Maar Bromtol zag het {\textquoteleft}{\textquoteright} {\textquoteleft}{\textquoteright}.}{dadelijk aan hetsaurtement}{vansmoeltjestrekke}\\

\haiku{'t Was natuurlijk,.}{omdat Corry thuis sprak van}{Bad-Aap in haar wrok}\\

\haiku{maar jij loopt naar de.}{Kromme Palmstraat toe en ik}{naar de Lindegracht}\\

\haiku{En dat kon je niet,,.}{wegpraten slecht en recht met}{geen duizend woorden}\\

\haiku{Zij begreep niets van.}{het georganiseerde}{arbeidersverzet}\\

\haiku{Maar nou... nou... was hij...!}{voor Frans maar een kind van het}{gezin maar een mensch}\\

\haiku{Nou, na een maand, had.}{zij geen  143voedsel meer}{voor haar eigen wicht}\\

\haiku{En daar zat Frans maar.}{\'al over te treuren en te}{kniezen in zichzelf}\\

\haiku{De deklief-hebbers.}{in den Jordaan hielden niet}{eens meer sierduiven}\\

\haiku{Plots schoot Frans in een...,.}{lach o die grapjasserij}{van Bromtolletje}\\

\haiku{Alleen die blauwe,.}{Dragonder die liet zich niet}{wegdringen door hem}\\

\haiku{Manus Peet had nog '.}{tweemaal Frans Leerlaps avonds}{aan huis opgezocht}\\

\haiku{Dan voelde hij dat;}{de geest en de vervoering}{echt in hem waren}\\

\haiku{Dit was misschien wel.}{het sterkste en tegelijk}{het zwakste standpunt}\\

\haiku{een droef beetje, dat!}{nauwelijks gemalen of}{geraspt kon worden}\\

\haiku{Kom, hij ronkte als,!}{een slagveer een seconde}{v\'o\'or het afloopen}\\

\haiku{In het gezin van.}{Frans Leerlap mocht Manus niet}{meer worden gemist}\\

\haiku{En, even bevend van,.. -!}{stem spotte Manus schijnkoel}{Schuin over je Dahlia}\\

\haiku{tusschen de ketels.}{werken en de vlampijpen}{uitvegen op schip}\\

\subsection{Uit: Menschenwee. Roman van het land. Deel 2}

\haiku{komp bai de huur t'recht.. ' '!....}{doent sellefers sel je}{n vrachie voele}\\

\haiku{Verduufeld aa's tie nie.}{twee Piets en twee Dirke veur}{sain rekening nam}\\

\haiku{voortzwoegend tot den,,.}{avond zwijgzwaar geradbraakt in}{pijning van elk lid}\\

\haiku{Witte hoeden en.}{strooien kiepen lichtten en}{blondden in de wei}\\

\haiku{- Ou\"e Gerrit voelde,.}{zich nijdig worden hoorde}{dol-klank maar half}\\

\haiku{- Hou je bek, snauwde,....}{woest-driftig Kees ik}{vroag je sinte nie}\\

\haiku{Ze voelden wel, de, ',.}{werkers datt nu ging om}{hun rust h\`un bestaan}\\

\haiku{F'rduufeld, nou gonge ',.}{de kerelsr koejeneere}{bromde ou\"e Gerrit}\\

\haiku{- Kees, Kees, bromde Piet,,,,!....}{weer die hep gain rug gain stuit}{gain kop die hep niks}\\

\haiku{Dirk en neef Hassel '.}{konden met hun karrent}{zijhek nog niet in}\\

\haiku{je laikt puur daa's,,,....}{se benne bestig loog Dirk}{om zich te redden}\\

\haiku{En van allen kant,,.}{de zwoeggezichten keken}{strakker vermoeider}\\

\haiku{De eerste gouden;}{zomerjubel van wei en}{boom was weggelicht}\\

\haiku{Z\'o\'o, elken dag bleef,}{Wimpie moederziel alleen}{huilde hij soms als}\\

\haiku{- Heen en weer liep ie,.}{dol-angstig als plots zoo vreemd}{z'n hart stil bleef staan}\\

\haiku{Om de kerels had,:}{ie rondgedrenteld ze in}{z'n tel-rhytme}\\

\haiku{Van alle kanten,.}{tegelijk krakeelde bod}{tegen elkaar \`op}\\

\haiku{- Saa'k f'rbrande aa's '' '....}{k snap w\'at waif d'r mee}{uithoale mot}\\

\haiku{Is d'r netuurlik,!}{daa's sullie d'r honde snachts}{skiete loate}\\

\haiku{m\^o je laine.. en '......}{borge beklomppe ens}{winters hai je nood}\\

\haiku{.. je heb van ochend tut............}{nacht te werke en nie \`om}{te kaike fort fort}\\

\haiku{Veel venters, bek-\`af,,.}{stemden toe in prijs lam en}{gebroken van sjouw}\\

\haiku{De kerels roerden,,.}{smakkerden en de meiden}{lachten en likten}\\

\haiku{Den wagen had Kees '.}{n eindje hooger smaller}{dijk opgereden}\\

\haiku{Z'n lijf stond in walm,.}{van zweet poriede open in}{rauwen vleeschgeur}\\

\haiku{licht dat schroeide z'n,, ';}{vleesch groef en dreunde sloeg}{en vrat int hooi}\\

\haiku{- Nog 'n vaif vorke,,.}{galmde ie terug met z'n}{rug naar Dirk gekeerd}\\

\haiku{Als was z'n oog met ''}{n zuur gif volgedrupt zoo}{vrat en brandde}\\

\haiku{Gunter stoan d'r..!}{Bolk en Hannes Skrepel en}{Piet Steinstroa en Gais}\\

\haiku{Achter 'm stonden,.}{in bloei van lichtend paars de}{vroege aardappels}\\

\haiku{hoonde Dirk den Ou\"e,.}{na die achter de deur hem}{nog wou meetronen}\\

\haiku{de Ou\"e is d'r puur,!}{tuureluurs van nou die soo feul}{kwait k\`en aa's tie wil}\\

\haiku{- D\`at veel liever, dan,, '.}{op stap op den gloeizandweg}{naar zeen uur gaans}\\

\haiku{En dan maar brommen,, '.}{en klagen luid luid om iets}{r t\`egen te doen}\\

\haiku{Wisselend in gang,.}{trokken de werkers drie maal}{\`op naar de groote stad}\\

\haiku{Soms, als ie wat beet, ';}{had weer konm de heele}{boel niet meer schelen}\\

\haiku{Nu en dan zag ie.}{z'n wijf verdwaald rondzoeken}{in de tuinderij}\\

\haiku{'n Paar dagen had,.}{ie achtereen in huis wat}{bollen gesorteerd}\\

\haiku{Vloeken k\`on ie, als.}{ie daar niet dikwijls genoeg}{vrij mocht afzakken}\\

\haiku{Kees was bedankt, mocht '.}{weer eens aanslenteren in}{t drukst van den pluk}\\

\haiku{Hun morsige bloote '.}{voeten ploeterden int}{warmzonnige gras}\\

\haiku{Maar Trijn wou niet dat,.}{Geert d'r hoofd zou breken om}{die narigheidjes}\\

\haiku{Maar Jan was z\`eker '.}{bij  Dirk minstensn paar}{keer Guurt te treffen}\\

\haiku{Wai.. wille d'r vast.. '!}{mi de haire meegoan aa't}{t in frinskap lait}\\

\haiku{De meiden vonden.}{dat de knapen zich kranig}{gehouden hadden}\\

\haiku{in stikdonker, bij;}{schaduwrood schijnsel van wat}{eenzame lantaarns}\\

\haiku{Most dus puur tug 'n...}{toeval weuse a\`as tie die}{vint dur nog antr\`of}\\

\haiku{En Wimpie had dat.}{met smeekjes bij d'r moeder}{voor haar klaar gespeeld}\\

\haiku{Donker bonkten z'n.}{schonken boven de lage}{rij pofpetten uit}\\

\haiku{gain groasje, dood,.. '!}{an die swaine trek jullie}{d'rn poar kiese}\\

\haiku{Moordhol timbreerde, ' '....}{z'n stem en vlak naastm klonk}{n andere zang}\\

\haiku{Over twee minute....}{sal de nieuwe voorstelling}{een aanvang neme}\\

\haiku{voorsiet u van een, -.}{plaats suggereerde de stem}{van de estrade}\\

\haiku{Onder 't vreten ' ';}{rimpelder kop alsn}{oudwijvenmasker}\\

\haiku{Daa't is d'r mit 'n, '.}{half uur d\`a\`an hoonden meid}{belust op emotie}\\

\haiku{'n Zangeres was.}{achter gala-chanteur naar}{voren gekropen}\\

\haiku{- Ze drentelden en,.}{draaiden tot Dirk en Willem}{er uit wankelden}\\

\haiku{menschen die elkaar.}{in ego{\"\i}stischen angststuip}{knellend verdrongen}\\

\haiku{- In\'e\'ens voelde,,.}{ie zich weer laf kruiperig}{laf klam in doodsnood}\\

\haiku{stilte die 'm deed.}{rillen en huiveren van}{al stijgender angst}\\

\haiku{Dat was d'r eerst met,;}{snijboon negentig cent de}{duizend voor fabriek}\\

\haiku{Ze schreeuwden 'm toe, ', ',.}{vant hok uitt erf dat}{ie zich bergen zou}\\

\haiku{Ou\"e Gerrit huilde,,,.}{snikte rochelde van angst}{ontzetting en drift}\\

\haiku{Als 'n rouw, ging er '.}{stomme ontzetting en stil}{geween overt land}\\

\haiku{- Hij voelde iets heel ',.}{bangs opm drukken iets ergs}{dat gebeuren mo\'est}\\

\haiku{De vraag ontshutste,.}{ou\"e Gerrit z\'o\'o dat ie sip}{voor zich bleef kijken}\\

\haiku{Loenschig kippigden,.}{z'n oogen op Hassel en zwaar}{donderde z'n stem}\\

\haiku{Plots zei Beemstra 'm iets ', - '.}{int oor en Troost lispte}{t over aan Stramme}\\

\haiku{Als ik nog meer van,..}{die klanten had zou ik zelf}{op de valreep staan}\\

\haiku{je grond is er niet,..}{slechter op geworden dat}{zal ik niet zeggen}\\

\haiku{ik seg mo\`ar.. da\`as 'n..!}{kerel die help je nie van}{de wal in de sloot}\\

\haiku{Ja, hij most de boel,.}{vergoeilijken met meelij}{met verkleineering}\\

\haiku{Enn... veur w\^a gong die?}{nou nie in hande van de}{aere netoaris}\\

\haiku{Vrouw Zeune was links,,;}{gaan staan klaar om hem op te}{vangen als ie viel}\\

\haiku{En dan nog.. aa's 't!}{donkere hokkie opestong}{t'met de loodpot}\\

\haiku{ik hep d'r puur ses....!}{pop wonne heb d'r perdoes}{op Peloone wed}\\

\haiku{op 't krotje daar, ';}{op den Duulweg \'e\'en kamer}{metn achterend}\\

\haiku{Expres heb ik mijn..}{lens late zitte zag je}{niet dat ik merkte}\\

\haiku{Dat had ie eerder,.}{moeten doen toen ie alleen}{thuis was gekommen}\\

\haiku{In laatsten stamel,:}{van angst half-verstikt}{stotterde ie uit}\\

\haiku{stapte Piet naar Dirk,, '.}{liepen ze stom naast elkaar}{n havenkroeg in}\\

\haiku{En daarna Ant, in ';}{snikkende huilkramp voort}{lijkje van Wimpie}\\

\haiku{- Hij slenterde weer, ',.}{werkeloos rond en niks kon}{m meer schelen niks}\\

\haiku{- In 't lage krot;}{knaagde weer naderende}{winter-ellende}\\

\haiku{- 'n Geweldige,.}{uitstorting van haat woede}{en wrok barstte los}\\

\subsection{Uit: Menschenwee. Roman van het land. Deel 1}

\haiku{Maar toch, \'e\'en ding was,....}{er voor hem gebleven in}{al z'n ellende}\\

\haiku{se hewwe main, ',....}{neudig al wi'k veurn sint}{p'r uur puur nog nie}\\

\haiku{Met schrik-gezicht,,.}{draaide vrouw Hassel om wild}{grijpend naar d'r schort}\\

\haiku{die z'n rekening......}{hewwe wou en veurskot}{van grondbelasting}\\

\haiku{pats, dan d'r van langs,,....}{mit d'r stevige knuiste}{dan seie se niks meer}\\

\haiku{- Bar, bin soo koud aa's, '.... - '....}{m'n paip mokten ander}{Kruip inn stuk hout}\\

\haiku{dan lager dalend:}{met spannender stilterust}{in cijferzakking}\\

\haiku{Guurt was 'n meid die,.}{alleen aan d'r zelf dacht dat}{voelde ze nog wel}\\

\haiku{wacht se mos sich nou....}{moar puur inprate da't van}{selvers betert gong}\\

\haiku{Ze groef 't in 'r, ' ',.}{hoofd metseldet inr}{geheugen met drift}\\

\haiku{Ze zat zoo lekker,.}{zoo lekker d'r kansen te}{berekenen}\\

\haiku{Niks meer noodig, voor se ',.}{aigen paar kan en de}{rest veur de venter}\\

\haiku{Dat was 't eenige ', '.}{datr staande hield enr}{verdriet verdoofde}\\

\haiku{doedelsak, je skenkt,.... -,....}{op main poote helhoak Main}{kristus wa\^a jokkes}\\

\haiku{en jullie.. jullie........ '?....}{wete d'r ook gain snars van}{weet jait moeder}\\

\haiku{St. Nikolaas was '.}{in wit gewaad neergedaald}{int stedeke}\\

\haiku{de hoogste hep sain,..... -....}{twee pond paling of ses pond}{speek'laos twoalf gooit}\\

\haiku{- Nou aa's j'r senie '....}{in hep ka je f'nacht int}{koarte-huis maffe}\\

\haiku{Dronken tronies van;}{mannen en meisjes lachten}{al \`a\`an in zuipgrijns}\\

\haiku{Een had partij voor '.}{Piet getrokkenn ander}{voor den Duinkijker}\\

\haiku{Zacht vlokte sneeuw neer,.}{wemel-schimmig op zwak}{lichtend haventje}\\

\haiku{Huiverig en koud,,.}{keek ie uit overal waar ie}{het sneeuwstraatje langs ging}\\

\haiku{Voor 'n koek- en,.}{broodwinkel bleef ie staan met}{groote gulzige oogen}\\

\haiku{Want hij zelf had 'r ', ',.}{n gruwel van vond hetn}{lam ellendig werk}\\

\haiku{Dan joeg 't in 'm,, '.....}{brandde bonsdet die lui}{die alles hadde}\\

\haiku{ikke  en de........}{hufters sou je je aige}{nie f'rmorsele}\\

\haiku{In huilsnikken barstte, '.}{ze los metr vuile schort}{tegen d'oogen gedrukt}\\

\haiku{De kinders hurkten,,,.}{bij Wim's bed saamgeklonterd}{in doodsnood spraakloos}\\

\haiku{De kinders konden,,.}{beginnen gulzigden in}{met oogen en handen}\\

\haiku{Jammerlijk vaalgroen '.}{bleekte z'n kopje int}{schuwe val-licht}\\

\haiku{Z'n vuile hansop,,.}{liet z'n beentjes uitspaken}{latjesplat recht uit}\\

\haiku{- En je moeder dan,,....}{ken die nie blaife meskien}{is d'r feur main wa}\\

\haiku{Ze had gloeienden ',.}{hekel aanm omdat ie}{van z'n vader hield}\\

\haiku{V\'o\'or 't huisje van,.}{Grint bleef ie staan trapte ie}{even tegen de deur}\\

\haiku{- Nou, ik wou moar da,,....}{se d'r sno\`ater hield driftte}{Jan Grint is me da}\\

\haiku{Klaas was woedend, hij.}{begreep niet waar de kerel}{zich mee bemoeide}\\

\haiku{zat ie gezwollen,.}{van kleine afgunst kleine}{indringerijtjes}\\

\haiku{twai volle ure en....'....}{nou gong de hailige en}{d appeteker}\\

\haiku{Z'n bleek groenige '.}{kop zag in schrik \`op naar z'n}{vader int licht}\\

\haiku{moar jullie hep sain,,}{achter de mouw zei Kees straf}{onverschillig voor}\\

\haiku{Bolk, spraakzaam, blij, dat,.}{ie wat takkebosjes te}{maken had ging door}\\

\haiku{Loomer schuurden de,,.}{takken op langs de kar bij}{kleine optrekjes}\\

\haiku{lijf naar beneden,.}{kijkend of Krelis al klaar}{was om te sjorren}\\

\haiku{komp, spotte Heelis,'!}{mo je eerst fesoenlik}{tellegrefeere}\\

\haiku{hai kon ook tug nie '....}{betoale en nou weer}{n dokter bai Piet}\\

\haiku{Op den dag kwam er,, '}{toch nooit niks van was ie bang}{dat zem snappen}\\

\haiku{Met 'n bons had ie..}{de bedsteedeurtjes achter}{zich dichtgeslagen}\\

\haiku{- Moeder, nou effe.. '?}{et vuurskutje van de}{stal hoal jait}\\

\haiku{Sou ie niet gille ',.}{aa's tiet sag aa's tie d'r}{was in sain kelder}\\

\haiku{Z'n hart bonkte en....}{wilder gloeiden z'n oogen in}{z'n grijzen glanskop}\\

\haiku{Het lampje eerst uit,,.}{dat wou ie nou eenmaal zoodat}{ie niemand zien kon}\\

\haiku{Maar heel bedaard bleef, '.}{zem loom vragend of ie}{d'r uit was geweest}\\

\haiku{Nooit kon ie slapen, '.}{of moeder moest ook liggen}{al werdt elf uur}\\

\haiku{Rams aansjokken uit,}{achterend vaagtastend in}{schuifel-pasjes}\\

\haiku{- Ik seg moar, scherpte,'........}{ze stil uit da se d'r van}{bekold is behekst}\\

\haiku{aa's je nie je lol....'....}{d'rin houwt sou je t'met j}{aige f'rsuipe}\\

\haiku{De half-open.}{slaapmondjes zuurden adem uit}{in eng ruimtetje}\\

\haiku{Enn ikke... ikke... '....}{bin protestant moart sit}{d'r dunnetjes op}\\

\haiku{Ant frommelde 'r,.}{plunje los zwaar gapend met}{zenuwgeluid}\\

\haiku{Maar nou mo' je sain..!..}{pakke hoor!nou van niks meer}{prakkeseere hoor}\\

\haiku{Dan nog liever met,,.}{sprenkels maar da gong nie meer}{nou ie niks bezat}\\

\haiku{hai mos t'r soo wa'........}{baif'rdiene ken se kossie}{bestig hoale}\\

\haiku{Met de anderen,,.}{was ie zeker in z'n sprong}{de sloot ingestapt}\\

\haiku{Breugel stapte naast,.}{Kees die z'n achterlader}{weer be-hageld had}\\

\haiku{'t zwart-stille, '.}{pad \`opstappend tegent}{donkere krot aan}\\

\haiku{Hij zou d'r 'n pats '....}{tegen d'r kop geven aa's}{se nogn woord zei}\\

\haiku{Laag groezelde 't '.}{lampje wat licht neer int}{killige vertrek}\\

\haiku{Dit jaar had ie zich '.}{nog voorn prijsje van de}{koeien afgemaakt}\\

\haiku{en al \'e\'en keer s'n, '.}{borge ansproke die niks}{meer mitm wilde}\\

\haiku{in tait van nood skil '? -....'!....}{je oarepels mitn bail}{hee Moar da k\^e soo}\\

\haiku{'t Zwaarste werk ging '.}{m licht van de hand. Stank van}{beervuil rook ie niet}\\

\haiku{nou dacht ie aan z'n,,.}{meid voor vanavond in heete}{opjoelende lol}\\

\haiku{Gisteren hadden.}{ze de eerste centen van}{spinazie gebeurd}\\

\haiku{Hoe kon ie zoo stom '.}{zijn te gelooven datt bleef}{staan wachten op hem}\\

\haiku{Onrustig joegen.}{de tuinders elkaar \`op in}{hun verborgen angst}\\

\haiku{Zoo bleef de Meiemaand '.}{rondgaan int stedeke}{en dorpjeszeeweg}\\

\haiku{Stil zat dominee, ',.}{in mijmer inn rieten}{tuinstoel aan den weg}\\

\haiku{niks had d'een voor d'a\^er, '....}{over of ze motte wete}{datt vast goed gong}\\

\haiku{En Kees was al heel.}{blij als ie zeven pop kreeg}{voor de heele week}\\

\haiku{loom gezang van de, ',.}{aarde naart lichtende}{groeiende leven}\\

\haiku{An Peters van de....}{Baanwaik h\'e'k femurge twee}{hoek oarbeie f'kocht}\\

\haiku{Hommels streepten van;}{allen kant fluweelige}{kleurtjes door de lucht}\\

\haiku{vroeg ie in-\'e\'en,,.}{door dikke vrouw negeerend die}{nog wat zeggen wou}\\

\haiku{Strak drukte ie z'n,.}{bril in neusgleufje montuur}{tegen z'n ooren}\\

\haiku{zei verbleekend, mank... '}{vrouwtje schuiner afzakkend}{r linkerschouder}\\

\haiku{- Wat, schreeuwde woedend,!.}{assistentje donder op}{wordt hier niet verkocht}\\

\haiku{maar ik zeg dat 't,,;}{nou genoeg is barstte uit}{z'n nijdige stem}\\

\haiku{hoho.... aa's d\'a' 't waif '....}{miskien bestiges inn}{gesticht sel konne}\\

\haiku{- Dokter Troost bleef 'r,,.}{stil bekijken schudde soms}{even zwak z'n log hoofd}\\

\haiku{Dokter stond nog wat,.}{te bedisselen met Guurt}{die fijntjes lachte}\\

\haiku{en ook mi' sonder.. '....}{aige oapeteek en nooit komp}{ien keer teveul}\\

\haiku{noa niks! - Moar, komp 'r,, '!}{t'met rege schreeuwde Piet}{bestig veurt raip}\\

\haiku{of onze lieve ', '?....}{Heerm zou ranselen dat}{ier bij neerviel}\\

\subsection{Uit: De oude waereld I: Het land van Zarathustra}

\haiku{Samir wachtte een,.}{blik of de Gebieder er}{iets van begeerde}\\

\haiku{All\'e\'en bij het volk.}{perst gij het zoetste sap van}{de levenskern uit}\\

\haiku{het lokkend kweelen.}{van een loofvogel achter}{doornen-struweel}\\

\haiku{Inhevige, toch:}{bedwongen bewogenheid}{ging Darius voort}\\

\haiku{En toch drukte er.}{op zijn ziel de onheilstilte}{der zee v\'o\'or den storm}\\

\haiku{Scythia, waar zij geen.}{voedsel meer vinden en geen}{uitweg meer kennen}\\

\haiku{al wat het diepe,.}{menschen-wezen aan zichzelf}{ontleent alleen blijft}\\

\haiku{Gij hebt mijn zoon door,.}{zelfmoord in bloedstortingen}{laten omkomen}\\

\haiku{- Dek je zondestem...,....}{toe en zwijg zei hij barsch}{tot den voorlezer}\\

\haiku{Stil wenkte hij den,.}{gunsteling met de schalksche}{oogen aan te vangen}\\

\haiku{V\'o\'or wij doof worden,.}{moeten wij hooren en v\'o\'or}{wij blind worden zien}\\

\haiku{Ik zeg u Koning,,.}{uw grootvizierszoon sprak stout}{maar ook laf en half}\\

\haiku{Maar Zerubbabel,.}{verlangde niets voor zichzelf}{doch veel voor zijn volk}\\

\haiku{Een juichenden drom;}{van werkmeesters vereenden}{de Hebreeuwers tesa\^am}\\

\haiku{in het Zuiden naar;}{het Idumeesche gebergte}{en Arabia Petraea}\\

\haiku{W\'e\'er reed Nadab, de,.}{grijze Susi\"er ontschroomd naast}{Koning Darius}\\

\haiku{waar hij bezweek van.}{gekweel en gevley zijner}{heidensche tortels}\\

\haiku{Zerubbabel, den,.}{peha op te trekken naar}{den heiligen grond}\\

\haiku{Tooyt u niet op als.}{een satyrhoender in}{zijne pronkveeren}\\

\haiku{Zion brandt... Zion,......}{brandt de veste verkoolt tot}{een zwarten bouwval}\\

\haiku{Want nu, n\'u mannen,.}{Isra\"els richt God ons weer}{goedgunstiglijk \'op}\\

\haiku{De leschem van Naphtali,.}{golfde zeegroen-blauwe}{purperingen toe}\\

\haiku{Raad mij broeder, bij... -...}{de heilge Fravashi van}{Zarathustra Hoor}\\

\haiku{Toch, het vuur was hem,,.}{ontstolen het vuur ontspat}{aan steen tegen steen}\\

\haiku{Firdusi wordt een {\textquoteleft}{\textquoteright}.}{gloeiend vereerder genoemd}{der oude tijden}\\

\haiku{een herschepping van.}{het leven der Oudheid in}{al zijn uitingen}\\

\haiku{Dit  feit lijkt een.}{tegenstrijdigheid en is}{het toch slechts schijnbaar}\\

\haiku{Omdat hij met een,.}{essentieel orgaan mensch}{en tijden herschept}\\

\haiku{Ik bekommer mij}{om historische feiten}{en bronnen zooveel}\\

\haiku{Jammer dat de man.}{niet op de hoogte van de}{jongste feiten is}\\

\haiku{Vgl. ook Ed. Meyer,,, {\textsection}-:}{Geschichte des Altertums}{Erster Band 117118}\\

\haiku{kennis van tempels,,...:}{priesterleven en feesten}{in dit \'e\'ene groote}\\

\haiku{Omdat deze zich.}{evenzeer in groepeeringen van}{feiten openbaren}\\

\haiku{Het verleden en.}{het tegenwoordige is}{een en hetzelfde}\\

\haiku{In den Bijbel b.v..}{geeft Cyrus de vaten uit}{Babylon terug}\\

\haiku{In zijn Lehrbuch der,;}{alten Geographie schrijft}{H. Kiepert p. 193}\\

\haiku{Nog iets anders dan.}{het minachtende woord van}{onzen prof. Hartmann}\\

\haiku{Der J\"ager ist im.}{orientalischen M\"archen}{der Unterweltsmann}\\

\haiku{Deuteronomium.}{wierd gevonden en op naam}{van Moyzes gesteld}\\

\haiku{Avant l'aube, il fait,,}{froid tout est triste l'homme}{reste inquiet}\\

\haiku{Ze liggen echter,.}{niet altijd op hooge punten}{doch ook in dalen}\\

\haiku{Bolland bleef zijn slurf.}{zwaayen en vermorzelde}{wie hem te na kwam}\\

\haiku{Ed. Meyer blijkt ook.}{nog een groot bewonderaar}{van de Achaemeniden}\\

\haiku{Hij bestrijdt Winckler,.}{die beweert dat Astyages}{geen Meed is geweest}\\

\haiku{Firdusi Zie E.,,.}{Burnouf Commentaire sur}{le Yasna p. 400}\\

\subsection{Uit: De oude waereld II: Zonsopgang}

\haiku{Richt de hoorns en zingt....}{met een schallende stem mijn}{mannen der bergen}\\

\haiku{Een misvormende.}{ontsteltenis wrong er over}{heel zijn schoon gelaat}\\

\haiku{En de vingeren.}{mijner kinderen in de}{bescherming van Thot}\\

\haiku{Op zijn rug, dezen....}{tartenden beschimper van}{Azi\"e's Majesteit}\\

\haiku{Hij, Heer van het Al,....}{voortbrengende stier onder}{de negen Goden}\\

\haiku{Eens was Aethiopi\"e,:}{vastgeklonken aan Aegypte}{nu begeerde het}\\

\haiku{Toch kroop het eerst naar,.}{hem toe Peshtudibart en de}{anderen volgden}\\

\haiku{- Groot is uw Heer,... viel.}{schijn-nederig Haman}{op zijn beurt nu in}\\

\haiku{Zij worden door de.}{pijlen onzer knaapjes in}{de vlucht gedreven}\\

\haiku{Doch zij, een stonde,.}{verdwijnen als schaduwen}{in het nevelland}\\

\haiku{- Doen wij altijd... doet.}{ook de heilige kat met}{de zwarte ooren}\\

\haiku{- O Yima, wist ik...}{hoe het zal worden aan het}{einde der tijden}\\

\haiku{Maar de booze geesten.}{hebben ze gevloekt en hun}{trony's roodgebrand}\\

\haiku{- Ik neem hem levend,...}{mee naar Susa tusschen de}{smaragdduifjens}\\

\haiku{Houdt jij ook zoo van?}{ivoor en struisveeren als die}{zwarte grijnzers hier}\\

\haiku{De papegaay zweeg,.}{en er gonsde een nare}{klemmende stilte}\\

\haiku{dat de knaap dien zij,.}{sidderend begeerde haar}{niet lief mocht hebben}\\

\haiku{Laat uw doekdrager.}{u een scheut welriekende}{specerij gunnen}\\

\haiku{hoe zie ik in het,!}{gouden licht van uwe oogen de}{waarheidsdrift fonklen}\\

\haiku{- Ja mist, nevel, mist!}{die tusschen de reten uwer}{vingers weer ontsnapt}\\

\haiku{De gevretene.}{wordt vreter en de vreter}{weer gevretene}\\

\haiku{Ik ken een Spartaan,!}{die veel liever een edelman}{is dan een edel man}\\

\haiku{Uw vruchtboomen laat.}{gij met Calybonischen}{wijn besprenkelen}\\

\haiku{- Al drie dagen leeft.}{hij gillend en hijgend op}{de paalpen gespit}\\

\haiku{Of vallen vrees en?}{verschrikking en duizeling}{ook over jou Spamitres}\\

\haiku{Hij zegt dat ik geen,;}{rijk van het Oosten maar van}{de aarde begeer}\\

\haiku{Mijn Sidonische....}{galey zal mijn troonhemel}{onbezwalkt torsen}\\

\haiku{Ieder woord dat zij....}{zeggen neemt een gedaante}{aan van een wezen}\\

\haiku{Je riekt naar zoeten,...}{narcis en sandelhout deern}{en naar saphraanbrood}\\

\haiku{- Bij den heiligen,......}{Darius ik zie dat je}{gelitteekend bent}\\

\haiku{Rond den Hellespont moet....}{de bodem dreunen van het}{bouwen der schepen}\\

\haiku{Ik deern, ben altijd.}{gelukkig met de dingen}{die ik niet begrijp}\\

\haiku{- Spamitres,... viel hevig,.......}{Xerxes uit ik wil de}{vrouw van mijn broeder}\\

\haiku{schepsel wordt nu al,!...}{vervreten door de groote}{smachtende zonde}\\

\haiku{Want weet Spamitres,... weet....}{schoon meisjen het geslacht der}{Achaemeniden vreest niets}\\

\haiku{Ze beeft, het fijne,....}{zangstertje het lokkende}{dochtertje van Maya}\\

\haiku{- Vertrek kamerling,...,....}{zei in zachte fluistering}{Xerxes en bid}\\

\haiku{{\textquoteright} Aan het slot van zijn, {\textquoteleft}{\textquoteright}:}{boekL'exotisme am\'ericain}{schrijft Gilbert Chinard}\\

\haiku{A very low door in.}{the western side serves}{as an entrance}\\

\haiku{{\textquoteleft}Verhandeling over{\textquoteright}:}{geestenzien en wat daarmee}{samenhangt beweert}\\

\haiku{Vgl. over Hrihor Chapter,:}{II The History of the}{Philistines I}\\

\haiku{Abessyni\"e noemt hij.}{een land der contrasten in}{meer dan een opzicht}\\

\haiku{F\"ur die biblische.}{Literatur ist das von}{groszer Wichtigkeit}\\

\haiku{In het Hebreeuwsch is.}{de rhythmische vaart en dreun nog}{veel geweldiger}\\

\haiku{{\textquoteright} Over {\textquoteleft}the meadow{\textquoteright}.}{in the suburd of the city}{volgt uiteenzetting}\\

\haiku{Perrot en Chipiez}{halen er Herodotus bij}{en doen uitkomen}\\

\haiku{Daneben finden.}{wir auch den Titel K\"onig}{des Nuhselandes}\\

\haiku{il explore en;}{navigateur le fleuve}{des temps r\'evolus}\\

\haiku{Deuxi\`eme Partie, {\textquoteleft}{\textquoteright}.}{Am\'elineau eenExpos\'e}{du syst\`eme geeft}\\

\haiku{{\textquoteright} 45) verbonden met.}{waardevolle opgaven}{over zijn regiment}\\

\haiku{Zie ook ibid. Zweiter,-:}{Band vooral p. 4647 en}{het gansche hoofdstuk}\\

\subsection{Uit: De oude waereld III: Morgenland}

\haiku{Vernietigd wordt het,.}{leugenwoord Het Ware zal}{vernietigen hem}\\

\haiku{Hij werpt zich het \'e\'erst,.}{op de ziel in zwakheid en}{in vreeze ademend}\\

\haiku{De wijze mint de.}{armoe en het stille grauw}{der nederigen}\\

\haiku{Haman moest op zijn.}{hoede wezen en d'ooren}{rekken tot den grond}\\

\haiku{De poort der aarde.}{omstralen zij met den glans}{hunner wetenschap}\\

\haiku{Dadelijk ontstond:}{het eerste visioen van}{het schrikbarende}\\

\haiku{Langzamerhand zag;}{Haman al helderder en}{wonderbaarlijker}\\

\haiku{\'op deed rijzen tot;}{den vorstentulband boven}{de stierenhorens}\\

\haiku{de adder kronkelt.}{zich in den kroondiadeem}{onzer koningen}\\

\haiku{Beseft ge dan n\'u?}{reeds dat gij den twintigsten}{van Thoth moet vreezen}\\

\haiku{Blijf eerst nog zingen.}{bij het vuur der Perzen en}{weerstreef uw lot niet}\\

\haiku{grooter raadsel nog.}{dan het marmer-vlammen}{van hun aardewerk}\\

\haiku{- Gelooft gij dat wij,?}{bestaan uit stof schaduw en}{ongeschapen ziel}\\

\haiku{En ik, Bes, ik laat.}{de zon lachen en jaag de}{treurigheid weg}\\

\haiku{- Omdat zij niet uw,,....}{volk maar \'u bespionneeren}{viel wrang Ruba uit}\\

\haiku{Bij het sterven van.}{den dag verkwijnde ook zijn}{drang tot handelen}\\

\haiku{Ook de natuurstreek}{van Anatoli\"e bedwelmde}{hem weer gansch en al.}\\

\haiku{Hydarnes, aan het,.}{hoofd der Onsterfelijken}{leek van louter goud}\\

\haiku{Ruba zag dat zij}{zich lieten bedienen door}{naakte slavinnen}\\

\haiku{, gelaat aan gelaat.}{nog \'e\'en tel in doodswanhoop}{het licht inhieven}\\

\haiku{Ook hij heeft niets in.}{te brengen bij den Raad der}{Amphyctionen}\\

\haiku{Droomerig, tegen,':}{Ruba in klonk Xerxes}{stem weer aarzelend}\\

\haiku{- Braaf zoo... heel braaf zoo,,....}{mijn Gebiedertje gij leert}{in de ruimte zien}\\

\haiku{Wankel niet altijd.}{naar de tragische grenzen}{der dingen terug}\\

\haiku{het is dood-uit met.}{Pan. En de vallei is heel}{diep en half duister}\\

\haiku{- Het purperkrijt op,,....}{uwe geschminkte wangen mijn}{Koning korrelt los}\\

\haiku{Neem u in acht voor!}{het bloedend gebit van het}{geranselde paard}\\

\haiku{- O Volschapene,.}{hoe zoet is uw geluid bij}{dit broze peinzen}\\

\haiku{Geweld... geweld, dat....}{is uw schuimende kracht uw}{brijzelende drift}\\

\haiku{Kermend-zacht van,':}{stem klonk het haperend van}{Xerxes lippen}\\

\haiku{Weet wel, de goden.}{van Hellas verduisteren}{de menschenbreintjes}\\

\haiku{O mijn heerlijke...}{Achaemeniden-Koning als}{gij wist hoe lekker}\\

\haiku{- De dood sluipt er op,.....}{de teenen rond zong Pan eens in}{liefde-verdriet}\\

\haiku{- Uw bode naar het,.}{orakel der Branchiden kwam}{terug zonder stem}\\

\haiku{Hoor Arche{\"\i}sche,,.}{Hera juichen haat juichen}{rauw van stem en kreet}\\

\subsection{Uit: Zegepraal}

\haiku{Dit boek is voor jou,.}{subliem kind van verbeelden}{en werkelijkheid}\\

\haiku{Maar dan, mijn geluk,,!}{mijn bevend geluk dat ik}{jou heb Florence}\\

\haiku{mij iederen avond.}{en iederen doodstillen}{nacht op mijn kamer}\\

\haiku{Florence, zie die,....?}{prachtige amfora is}{dat geen lijntooverij}\\

\haiku{Het gigantische,.}{er in komt op me \`af en}{ik stort er op neer}\\

\haiku{Ik hoorde door m'n '.}{oorent geruisch van}{oceanen vloeien}\\

\haiku{Hoe vervloekte ik!}{mijn minnarijen met die}{heerlijke vrouwen}\\

\haiku{riep ze heel zacht, zoo.}{verbergend haar schrik om mij}{met te ontstellen}\\

\haiku{ik smeekte 'r jou,,.}{te beschermen te helpen}{als ik toch dood ging}\\

\haiku{Je weet Florence.}{hoe mijn hoofd altijd vol zat}{met melodie\"en}\\

\haiku{In ieder dezer '!}{vlamtt vuur van een Carmen}{uit de gouden oogen}\\

\haiku{Zoo stom, zonder kreet.}{vergaat de vrouwendans in}{woesten zinnenwaan}\\

\haiku{Je handen wiegen,.}{nog traag boven je hoofd in}{verslappenden wuif}\\

\haiku{- Als ik iemand zie '.}{vallen word ik kwaad en schiet}{direkt inn lach}\\

\haiku{Elk vingertje zingt,.}{een gracieus zangetje}{van lijnbekoring}\\

\haiku{de maandelijksche.}{post zond die ik ook v\'o\'or m'n}{ziekte van hem kreeg}\\

\haiku{En nu moet je den.}{bitsen ineengedrongen}{kerel naast haar zien}\\

\haiku{Daarom wachtte ik,,.}{al maar wachtte ik tot ik}{wat meer zeggen kon}\\

\haiku{met toekomst-wezens,,....}{vol vrijen ademhaal groot en}{overvloedig van kracht}\\

\haiku{Dat geluk liet me,.}{nooit meer los ook in mijn n\`og}{bangere uren niet}\\

\haiku{'t Was soms of ik.}{telkens iemand de trappen}{hoorde afrollen}\\

\haiku{goud en violet.}{van weerschijn-verrukking}{en droomrige pracht}\\

\haiku{Als je hoofdje een,}{beetje zakt op je borst en}{je oogen kijken}\\

\haiku{ze wondert in mij.}{als de zachte zang van heel}{stil gemijmer}\\

\haiku{En daarom, daarom,.}{Florence is mijn snakken}{naar jou nog grooter}\\

\haiku{Nu wachtte ik m'n,.}{benauwingen af vreesde}{ik ook veel minder}\\

\haiku{Die geur beroerde.}{toen ontzettend en akelig}{m'n doodsgedachten}\\

\haiku{O Florence, hoe '.}{t lieve vrouwtje me toen}{treurig aankeek}\\

\haiku{Ik had ze in 't '.}{hartje van de stad verlangd}{opt Rembrandtsplein}\\

\haiku{Ik was overstelpt van,.}{ontroering om zooveel moois}{zooveel goddelijks}\\

\haiku{de stralende sneeuw,.}{den verblindenden schitter}{van blank-wit licht}\\

\haiku{Zijn zelfvergoding '.}{isn andere pool van}{\`ons kommunisme}\\

\haiku{m'n jonge vriend Sam,,.}{van Daalen mijn dokter vier}{jaar ouder dan ik}\\

\haiku{Ik wist niet waar ik,.}{heen moest ik alleen met m'n}{kosmischen godsdienst}\\

\haiku{wat 'n troostelooze;}{wolken drijven er van daag}{in de hemelen}\\

\haiku{Wat 'n grauwpaarse!}{daken triesten er in de}{schemerende stad}\\

\haiku{Er zijn parijsche.}{schepsels die ik nooit intiem}{zou willen  zien}\\

\haiku{Mijn verhuizing naar.}{achter gebeurde op een}{somberen herfstdag}\\

\haiku{Jou levensgeur woei '.}{op me \'a\'an als d'aroma}{vann druivenveld}\\

\haiku{Ik verdoemde de!}{engelachtige streeling}{van vrouwenmonden}\\

\haiku{De liefde als 'n, '!}{hemelsch preludium Eros}{opn goud troontje}\\

\haiku{Er is in mij zelfs.}{een bange schuchterheid om}{je te naderen}\\

\haiku{Maar 't is alles,, ' '}{\`echt waar echt want iedereen}{ziett en zegtt.}\\

\haiku{En dan te denken,,!}{dat ik uit de verte zoo}{dikwijls jaloersch was}\\

\haiku{Ik sidder bij 't, '.}{hooren van je stap bijt}{kreuken van je rok}\\

\haiku{De zonneboog kleurt.}{de hemelpoorten daar met}{wondre vuurbloemen}\\

\haiku{Ik hoor nu je lach,.}{je zilver-trillenden}{zangerigen lach}\\

\haiku{Het grootst en machtigst.}{opneemvermogen is toch}{m\`enschelijk begrensd}\\

\haiku{Ach Florence, wat.}{mij dat weer heeft doen snikken}{van ontsteltenis}\\

\haiku{Ik zag v\'o\'or me, den, '.}{blinden Milton maart scherpst}{den dooven Beethoven}\\

\haiku{Want hij durft \`alles,.}{dadelijk nadoen zonder}{aarzelen of schrik}\\

\haiku{Dan botsen ze, loopt.}{Arie ontdaan naar z'n moeder}{of mij en huilt stil}\\

\haiku{je kunt, wat je dan,.}{te pakken hebt heeft je ten}{minste goed gedaan}\\

\haiku{Iedere stem die.}{ik hoorde achter den muur}{gaf ik gestalte}\\

\haiku{Ik voelde all\'e\'en.}{menschelijke eenheid en}{zielegelijkheid}\\

\haiku{'t Vonkelde er,}{heet in de tropische zon}{en nu en dan hield}\\

\haiku{Toen hoorde ik hoe;}{hij zelf aan vreeselijke}{slapeloosheid leed}\\

\haiku{Want 't geluk van.}{het socialisme zit}{niet in de boeken}\\

\haiku{Ze hadden strakke,,.}{weeke hartstochtelijke en}{droomfijne monden}\\

\haiku{die boekenverkoop.}{heeft mij bang-verweende}{nachten gegeven}\\

\haiku{Florence, was jij, '!}{hier geweest ik zout je}{afgesmeekt hebben}\\

\haiku{Florence, heb je,,?}{me toen in die schemeruren}{niet hooren roepen}\\

\haiku{Telkens, onder nieuw,,.}{lichtreflex vergoocheling}{van buik kop en staart}\\

\haiku{Toch voelde ik me, '.}{m\`ett loopen wat beter}{en gelukkiger}\\

\haiku{Soms ijlde ik, riep, '.}{ik Zus w\`a\`art moedertje}{toch verborgen zat}\\

\haiku{Kuische vrouwtjes;}{bolderde hij de rokken}{\`op als windzeilen}\\

\haiku{Ze laten 'n brok ' '.}{vant groote leven int}{kleine leven zien}\\

\haiku{'t Leek me, of je ',.}{vant boereland kwam zoo}{frisch en geurig}\\

\haiku{'n Week lang bleef ik.}{dood voor alle andere}{dingen om mij heen}\\

\haiku{Maar ik kon bijna,.}{niet ademen zoo zwaar werkte}{de lucht op me in}\\

\haiku{Op 'n middag nam,.}{hij zich plots voor alleen met}{me op straat te gaan}\\

\haiku{Het klonk me soms heel,,,!}{geheimzinnig zacht en ver}{en zoo teer zoo teer}\\

\haiku{Tot die schepsels leek.}{hij te behooren in die}{verbitteringsuren}\\

\haiku{aan den man, die mij ' '.}{zelf weern endt leven}{ingetrokken had}\\

\haiku{Hij zelf erkende,,.}{dat ik er vreeselijker}{aan toe was dan hij}\\

\haiku{Ze stonden er door,.}{elkaar uit kweekkassen en}{wild uit de natuur}\\

\haiku{hoe ontroerde mij ',.}{t landgerucht meer nog dan}{de stem van de stad}\\

\haiku{En hoe verder ik,.}{ging hoe meer de landarbeid}{zich voor mij opdrong}\\

\haiku{Lieve, ken je de?}{voorstelling dier seringen}{plukkende schepsels}\\

\haiku{T\`ot 't dier weer in '.}{m opgetergd wordt door nood}{en woesten aandrift}\\

\haiku{'t Duurt niet lang. 't,!}{Dier woest en rauw gromt heel gauw}{\`op in die menschen}\\

\haiku{Of trappen ze hun!}{vrijster tegen den buik als}{de nachtbuit slecht was}\\

\haiku{Prachtig geveeg en,.}{lijngekras elk streepje z'n}{beduidenis toch}\\

\haiku{, want alles bijna!}{is de moeite waard apart er}{wat van te zeggen}\\

\haiku{'t Is er gesmook,,,.}{gezuip gelal gestoei en}{potsierlijk gedans}\\

\haiku{Wat 'n buurt, wat 'n ' '.}{huiver van sombertes}{winters ens avonds}\\

\haiku{Florence, in mij '.}{schreit niet de smart van \'e\'en ziel}{maar vann wereld}\\

\haiku{In mij jubelt niet ' '.}{t geluk van \'e\'en mensch maar}{vant menschdom}\\

\haiku{dat 't leven z\'o\'o,!}{groot zoo allergeweldigst}{in mij wil werken}\\

\haiku{Had ik 't voor drie,.}{maanden gehoord ik zou er}{onder bezwijmd zijn}\\

\haiku{dat er niets, niets van,.}{me bleef geen heugenis en}{geen herinneren}\\

\haiku{t Golfgeklots is '.}{t zangerige verhaal}{van dat verlangen}\\

\haiku{ik geef ze w\`at van,.}{m'n beetje en zij geven}{mij van hun beetje}\\

\haiku{Ik zorg altijd voor.}{prachtige aanwijzingen}{en introdukties}\\

\haiku{Maar je houdt je in,.}{uit vrees dat die menschen je}{toch niet begrijpen}\\

\haiku{er is 'n vreugde. '}{dan in me die me alle}{leed doet vergeten}\\

\haiku{'t leven dat de.}{heele menschheid aandraagt}{haar smart en geluk}\\

\haiku{kind was ik nu niet, '}{zwak zoo zwak dat mijn lichaam}{siddert ondert}\\

\haiku{Dacht je dat 't 'n, '?}{ongelukkige liefde}{wast nonnetje}\\

\haiku{soort waarzeggerij,?}{en al uiterlijk vertoon}{van mystiek gedoe}\\

\haiku{Je rammelt leeg, en '.}{je wordt uitgeschud alsn}{zak aardappelen}\\

\haiku{Ik weet wel dat met,.}{alles gezegd k\`an worden}{niet alles verwerkt}\\

\haiku{Want weet lieve, dat '.}{ik me niet door de smart van}{t leven laat slaan}\\

\haiku{Florence, wil je,?}{weten hoe precies hoe ik}{mij dat voorstel}\\

\haiku{En hoe schreiend slecht.}{is niet de Clotilde in}{Dokter Pascal}\\

\haiku{Dan kon hij schreien;}{van angst dat er niet alles}{tegelijk uitkon}\\

\haiku{Van de lui die hun {\textquoteleft}{\textquoteright}}{w\`erk z.g. van zich kunnen}{afzetten begrijp}\\

\haiku{In 't Hollandsch kan, ';}{je teederste dingen broos}{alsn adem vatten}\\

\haiku{dat stille happen, '....}{die droefnis om den mond bij}{t droge slikken}\\

\haiku{En ik ben blij, dat.}{ik me verzet heb tegen}{die smartaandoening}\\

\haiku{Werelden bouwen,,,.}{dat is groot werk dat werk wil}{ik doen mo\`et ik doen}\\

\haiku{Deze eeuw eischt.}{de belichaming van het}{wereldepos}\\
 
\chapter[12 auteurs, 946 haiku's]{twaalf auteurs, negenhonderdzesenveertig haiku's}

\section{Henri Rademaekers}

\subsection{Uit: Sancta Innocentia}

\haiku{Kerdju, det waar get!}{gewaes es d'r noe eine}{drin had gezaete}\\

\haiku{Het waar flink gaon}{vrere en de windj bloos hem}{door zien dun jeske}\\

\haiku{Ware die knien dan,?}{t\'och van de Keuningin wie}{Wulmke had gezach}\\

\haiku{{\textquoteleft}det kos hem eine{\textquoteright}.}{maondj of ettelik in}{de Pollartsjtraot}\\

\haiku{Angers zoot d'r gaer,.}{in de b\"os mer vandaag waar}{d'r gei plezeer aan}\\

\haiku{In-ins ware, ',.}{ze dao mert waar nog v\"a\"ol}{te vreug zach moder}\\

\haiku{Bang det ze hem dao,}{hole waar d'r neet meer want}{vader had gezach}\\

\haiku{{\textquoteright} {\textquoteleft}Ja, mijnheer, hij heeft,{\textquoteright}.}{het w\`el gedaan zach opins}{Jeuke zie moder}\\

\haiku{Krintebreudjes en,.}{flaetjes dao z\`ol d'r hem ins}{ein vaeg aan gaeve}\\

\haiku{dzjing dzjing ging 't en.}{opins sjt\`ong veur h\"a\"or naas waat}{ze betale m\`oste}\\

\haiku{Moder sjt\`ong get te}{drentjele en ging toen nao}{eine lange gaam}\\

\haiku{Die hadde z\`onne,.}{veldjwachter auch waal gel\"os}{die Bataviere}\\

\haiku{{\textquoteright} en Jeuke, dae zich,:}{pas gebiech had durfde neet}{te lege en zach}\\

\haiku{Toen had moder hem}{eine wermen dook \`om de}{kop gedaon mer}\\

\haiku{Det had ze zellef,.}{al ins \`ongerv\`onje wie}{det ging mit zo fonds}\\

\haiku{Hae verzoop haos,......}{in dae sjtool en op ins ging}{d'r zjwoek zjwoek}\\

\haiku{Jeuke sjt\`ong t\"osse,.}{de luuj in mit vader dae}{de late sjich had}\\

\haiku{{\textquoteright} Moder bloos ins get}{sjtaof van de kraon aaf}{en bloos hem vaort}\\

\haiku{In Remunj, langs 't,.}{sjpaor dao sjt\`elde zich}{de persessie op}\\

\haiku{En later z\`ol ze, '.}{verrijze op d'n daag van}{t laatste oordeel}\\

\haiku{Jeuke had nao d'n.}{opt\`och gekeke wie dae}{bie h\"a\"or veurbie kwaam}\\

\haiku{* * * ~ Berb, de maag van,;}{meneer Pesjtoor waar ei good miens}{mit ein hert van g\`oldj}\\

\haiku{D'n Enesbejer ging '.}{mit zienen h\`ondjt veldj in}{\`om ein oer of twee}\\

\haiku{nette p\`oppekop.}{had wie die dames die in}{de zjoernaal sjt\`onge}\\

\haiku{neet meer verder k\`os '.}{zinge en midde int}{leedje bleef sjtaeke}\\

\haiku{Toen w\`ol Oswald auch}{zie vader op dezelfde}{meneer oet d'n tied}\\

\haiku{gezichter taege,.}{hem sjneej en toen k\`os d'r zich}{neet haaje van de lach}\\

\haiku{D'n deef m\`os door de.}{achterdeur van de keuke}{zeen binne gek\`omme}\\

\haiku{De res zoot veilig,.}{in de brandjkas wo d'n deef}{neet aangewaes waar}\\

\haiku{Aan Juffrouw Mieke.}{de Krekel te Zandbergen}{Limburg Nederland}\\

\section{Hilda Ram}

\subsection{Uit: Schetsen, novellen en vertellingen}

\haiku{Nu krijg ik moed en.}{wie weet word ik u nog geen}{lastige klante}\\

\haiku{5e afl. bl. 233).}{en volg. moeten wij hier ook}{terloops aanstippen}\\

\haiku{sa po\'esie montre.}{alors l'homme du m\'etier et}{non le po\"ete}\\

\haiku{Het duurde tot in,.}{1894 vooraleer Hilda Ram}{nog iets liet drukken9}\\

\haiku{Ik kampte om eenen,,,!}{blik Die thans vol hulde me}{als een zon bestraalt}\\

\haiku{Als verloofde van ';}{Johan trekt Eleonora naart}{hof van Portugal}\\

\haiku{{\textquoteright} Emmanu\"el is:}{getroffen door Eleonora's}{kinderlijke trouw}\\

\haiku{, En wanneer ze 't}{voorhoofd beurde Scheen zoo ver}{haar blik te zwerven}\\

\haiku{Het metrum is niet:}{dichterlijk genoeg en niet}{bijzonder verzorgd}\\

\haiku{Zijn zij niet op de,}{hoogte van wat hen gevraagd}{wordt zoo verzoeken}\\

\haiku{en ik moest nu een,!}{sermoentje ondergaan dat}{niet van de poes was}\\

\haiku{Zoo kon ik zeker,!}{mijnen stiel niet leeren d\'at kon}{niet blijven duren}\\

\haiku{{\textquoteright} - Nog eenige dagen.}{leefde hij en dan luidde}{de doodsklok voor hem}\\

\haiku{Eenieder begon!}{te zoeken en haast was het}{raadsel opgelost}\\

\haiku{Zooals ik gezegd heb,,.}{droeg ze Oomken opgevuld}{uit schoonheidsgevoel}\\

\haiku{Janneken hield zich ',!}{stil en als hijs morgens}{wakker werd wat vreugd}\\

\haiku{en zijn paard er bij.}{en hoopen met lekkers voor}{hem en zijn broerken}\\

\haiku{Wapper voor haar, die,:}{met zijn vinger naar haar hoofd}{wijzend haar toeriep}\\

\haiku{Maar nauwelijks was,.}{het er af of hij ging weer}{zijnen ouden gang}\\

\haiku{Lange Begge!}{zou de les gespeld hebben}{aan wie het durfde}\\

\haiku{Op Lange Wapper.}{had de arme weduwe}{niet  eens gedacht}\\

\haiku{Ze bogen hun hoofd.}{onder den bevrijdenden}{zegen des priesters}\\

\haiku{Ons Janneken moet '.}{eenen bloempot  dragen naar}{t Kindergasthuis}\\

\haiku{{\textquoteright} {\textquoteleft}Vrouwken, spreekt hij, er,.}{kwam tot mij iemand die me}{geheel overreedde}\\

\haiku{Loon het, Heere, den,!}{goeden Engel die zijne}{redding bewerkte}\\

\haiku{{\textquoteleft}Daar, vrouw, ga, koop een,!}{bloempot met mijn borrelgeld}{ik drink niet vandaag}\\

\haiku{niet om langer te,,.}{kunnen slapen neen om naar}{de vroegmis te gaan}\\

\haiku{ze is binnen nog ' ' '.}{t eene ent andere}{aant wegschikken}\\

\haiku{{\textquoteright} fleemt hij en moedigt.}{haar aan met een stootje van}{zijnen elleboog}\\

\haiku{{\textquoteright} {\textquoteleft}Ik ook niet, moeder,,.}{onderbreekt haar Toon laat ons}{maar seffens trouwen}\\

\haiku{de bekers en al '.}{t ander tafelgerief}{blinken als zilver}\\

\haiku{Afwasschen en het,.}{huis uitkeren gaan gauw als}{de jongens weg zijn}\\

\haiku{Later komt vader '.}{thuis en om 7 uren gaan de}{kleinen naart bed}\\

\haiku{{\textquoteright} En op het bewijs,,.}{dat Token gelijk heeft moet}{Toon niet lang wachten}\\

\haiku{In zijne jongens.}{weet hij iederen aanleg}{te onderscheiden}\\

\haiku{En zij vinden geene.}{woorden om er de schoonheid}{van te beschrijven}\\

\haiku{Nooit zoudt ge denken, '.}{dat zet minste uitstaans}{met vodden hebben}\\

\haiku{antwoordt vieze Trien,!}{in Token's plaats men moet nooit}{den moed opgeven}\\

\haiku{{\textquoteright} roept Toon en, met de,.}{handen in zijne haren}{loopt hij de deur uit}\\

\haiku{Weet, dat Token en,.}{Toon hun deel hadden van smart}{en angst en droefheid}\\

\haiku{Zalig degenen, '!}{diet gegeven is er}{zulke te weenen}\\

\haiku{En wil er een den,:}{slechten weg op gij zult tot}{hem gaan en zeggen}\\

\haiku{Sedert dien was hij.}{van een kleinen een grooten}{baas geworden}\\

\haiku{want die naam, Poeier,.}{en Kruit zou den drager er}{van zelf overleven}\\

\haiku{Dat {\textquoteleft}zoo{\textquoteright} klonk weinig.}{aanmoedigend en een wolk}{betrok zijn gelaat}\\

\haiku{ik zou niet gaarne,!}{hebben dat hij in mijn huis}{armen en beenen brak}\\

\haiku{Zoo zat hij eens te,,.}{wiegen te wiegen dat ik}{er wakker van werd}\\

\haiku{{\textquoteright} {\textquoteleft}Ja, ja, antwoordde,!}{Hoppert maar ik dank liever}{alles u alleen}\\

\haiku{Zij wist ook wel, als,.}{het ernstig was hoe ze haar}{vischje vangen moest}\\

\haiku{{\textquoteleft}Hang het nu niet uit,,{\textquoteright}.}{Peerken zei Mevrouw Hoppert}{met ontroerde stem}\\

\haiku{Geen kwartier verliep,,.}{of terug kwam hij met een}{politiedienaar}\\

\haiku{Peerken was niet meer,.}{te zien en er werd niet meer}{van hem gesproken}\\

\haiku{Er waren zooveel,.}{antwoorden als hoofden maar}{Otto's stem klonk luidst}\\

\haiku{{\textquoteleft}Hoort eens, meisjes, voor,.}{we beginnen wil ik u}{eene kunst laten zien}\\

\haiku{Miss Holding was van.}{haren troon gedaald om den}{oproer te dempen}\\

\haiku{Zij is de oudste.}{en levert ons meer spel dan}{de andere twee}\\

\haiku{in hare zwarte,,.}{oogen spraken bewondering}{ja verrukking uit}\\

\haiku{Eene nieuwe wereld,,.}{de wereld van den geest was}{voor haar opengegaan}\\

\haiku{want daar waren er.}{in oorlogstijd honderden}{vermoord geworden}\\

\haiku{huisde met  haar,.}{en dan nog het eenig kind van}{die dochter Fonsken}\\

\haiku{In gansch de wereld.}{ging eene rilling door alle}{eerlijke harten}\\

\haiku{zijn ooren waren.}{zoo doorschijnend en stonden}{zoo ver van zijn hoofd}\\

\haiku{Waar blijven ze uit,,?}{de vijanden als het iets}{of iemand goed gaat}\\

\haiku{Hij droeg de vlag en?}{ging vooruit en d\'at zou hij}{moeten opgeven}\\

\haiku{{\textquoteright} {\textquoteleft}Begin maar gauw te,.}{eten gij hebt een gezicht als}{een hongerlijder}\\

\haiku{Nu is hij kreupel,.}{en zit in een rolstoel maar}{hij heeft het verdiend}\\

\haiku{{\textquoteright} En weer liep hij rond.}{om zijne volgelingen}{bijeen te zoeken}\\

\haiku{Daar lag haar Fonsken.}{met het hoofd in bebloede}{doeken gewonden}\\

\haiku{Och, manneken, slaap,.}{nu wat en zwijg anders zult}{ge niet genezen}\\

\haiku{{\textquoteright} En dan legde hij}{zijne hand op de vlag en}{fluisterde zooveel}\\

\haiku{Dood van den hertog.}{van Brabant.- Geboorte}{van Prins Boudewijn}\\

\subsection{Uit: Slachtoffers voor Transvaal}

\haiku{want daar waren er.}{in oorlogstijd honderden}{vermoord geworden}\\

\haiku{In gansch de wereld.}{ging eene rilling door alle}{eerlijke harten}\\

\haiku{zijn ooren waren.}{zoo doorschijnend en stonden}{zoo ver van zijn hoofd}\\

\haiku{Waar blijven ze uit,,?}{de vijanden als het iets}{of iemand goed gaat}\\

\haiku{Hij droeg de vlag en?}{ging vooruit en d\'at zou hij}{moeten opgeven}\\

\haiku{{\textquoteright} {\textquoteleft}Begin maar gauw te,.}{eten gij hebt een gezicht als}{een hongerlijder}\\

\haiku{{\textquoteright} En ze stelde hem,.}{allerlei vragen doch hij}{kon niet antwoorden}\\

\haiku{Nu is hij kreupel,.}{en zit in een rolstoel maar}{hij heeft het verdiend}\\

\haiku{{\textquoteright} En weer liep hij rond.}{om zijne volgelingen}{bijeen te zoeken}\\

\haiku{Daar lag baar Fonsken.}{met het hoofd in bebloede}{doeken gewonden}\\

\haiku{Och, manneken, slaap,.}{nu wat en zwijg anders zult}{ge niet genezen}\\

\haiku{{\textquoteright} En dan legde hij}{zijne hand op de vlag en}{fluisterde zooveel}\\

\section{Relham}

\subsection{Uit: De man achter de schermen}

\haiku{De auto bracht ons.}{intussen naar de villa}{van mijn gastheer}\\

\haiku{- Bepaald vinden zult!}{U de juwelen en het}{geld zeker ni\`et}\\

\haiku{En soms beweer je,!}{dat ik de nieuwsgierigste}{mens op aarde ben}\\

\haiku{Een koel windje streek,.}{door het gebladerte dat}{zachtjes ritselde}\\

\haiku{Maar op het ogenblik.}{is het mij onmogelijk}{meer te vertellen}\\

\haiku{Z\'o hartelijk te,.}{lachen als ik hem slechts zeer}{zelden gezien had}\\

\haiku{Ik heb hem gevraagd.}{hier te komen en hij heeft}{het me toegezegd}\\

\haiku{Precies te twaalf uur.}{verscheen de commissaris}{met z'n inspecteurs}\\

\haiku{En ik begreep, dat,.}{ik me aangesteld had als}{een klein dom meisje}\\

\haiku{\'e\'en onvoorzichtig.}{woord kan mr. Tarani v\'e\'el}{kwaad doen en den m.a.d.s}\\

\haiku{Mocht U mij nodig,.}{hebben dan ben ik altijd}{voor U te spreken}\\

\haiku{{\textquoteright} Donald knikte met.}{een onverschillig gezicht}{en sloot even z'n ogen}\\

\haiku{{\textquoteright} {\textquoteleft}Juist, mijn jongen, dat.}{is het volgende punt van}{ons werkprogramma}\\

\haiku{Hij zweeg en Donald:}{trok de conclusie uit z'n}{mededelingen}\\

\haiku{Mijn personeel heeft - -.}{sinds \'e\'en week dus al lang v\'o\'or}{de diefstal vrij af}\\

\haiku{{\textquoteleft}Mr. Tarani, ik,}{weet op het ogenblik m\'e\'er van}{de zaak af dan U.}\\

\haiku{De zon scheen nog even, -.}{opgewekt de hemel was}{nog even vlekkeloos}\\

\haiku{{\textquoteright} {\textquoteleft}Je bedoelt, dat ik?}{er niet al te snugger uit}{zie op dit ogenblik}\\

\haiku{{\textquoteright} {\textquoteleft}Goed, dan zal ik je.}{nu een kort college in}{het schieten geven}\\

\haiku{De gemaskerde.}{onbekende herstelde}{zich verrassend gauw}\\

\haiku{Doch een gepaste.}{vermakelijkheid is wel}{aan te bevelen}\\

\haiku{) Er werden massa's.}{vruchten-ijs besteld en naar}{binnen gelepeld}\\

\haiku{De ander viel hem {\textquoteleft}!}{echter dadelijk in de}{rede met eenAha}\\

\haiku{We hadden toch naar {\textquoteleft}{\textquoteright}.}{denbaas gevraagd en waren}{hier binnengeleid}\\

\haiku{Niet in de richting,.}{van de speelzaal doch juist naar}{de andere kant}\\

\haiku{{\textquoteright} begon ik de \'e\'erste.}{van een hele rij vragen}{te formuleren}\\

\haiku{{\textquoteleft}Johny, - n\'u komt!}{pas het belangrijkste deel}{van onze speurtocht}\\

\haiku{en heb  ik al.}{mijn conclusies tot nu toe}{op los zand gebouwd}\\

\haiku{U kunt me schrijven,.}{naar Milaan of naar Rome}{poste restante}\\

\haiku{{\textquoteleft}Ja, maar{\textquoteright}, viel me toen, {\textquoteleft}!}{inover een kwartier vertrekt}{het avond-vliegtuig}\\

\haiku{Zo konden we in '.}{t voorbijgaan slechts enige}{woorden opvangen}\\

\haiku{Maar nu is er een,:}{kleinigheid die ik niet van}{me kan afzetten}\\

\haiku{{\textquoteleft}Vanavond precies om,,?}{half acht hier in de villa}{h\`e afgesproken}\\

\haiku{Ik weet alleen, dat.}{signor Corelli vreselijk}{opgewonden is}\\

\haiku{{\textquoteright} {\textquoteleft}Hoe weet je dan...{\textquoteright} {\textquoteleft}Geen,!}{vragen nu alsjeblieft ik}{moet nu rust hebben}\\

\haiku{Wij zwegen beiden - -:}{wat moest je d\'a\'arop antwoorden}{en hij vervolgde}\\

\haiku{{\textquoteright} {\textquoteleft}Dan zal ik me met.}{deze verzekering maar}{tevreden stellen}\\

\haiku{In de volgende, -:}{seconde hoorden we een}{gil dan uitroepen}\\

\haiku{{\textquoteleft}Ik zal hem direct{\textquoteright},.}{naar een ziekenhuis brengen}{zei Corelli tot mij}\\

\haiku{Resultaat van dit.}{gesprek kon ik nog niet te}{weten komen}\\

\haiku{Het gezelschap stond,.}{ook op omdat de pauze}{juist was begonnen}\\

\haiku{Wij stonden enige.}{ogenblikken roerloos op \'e\'en}{plaats en luisterden}\\

\haiku{Hij bukte zich en.}{bleef enkele seconden}{gespannen kijken}\\

\haiku{Johny{\textquoteright} - hij greep - {\textquoteleft}.}{mijn handje moet proberen}{je te beheersen}\\

\haiku{We stapten haastig.}{in en de auto vloog over}{de verlaten weg}\\

\haiku{{\textquoteright} {\textquoteleft}Geduld, Johny,!}{nog twee dagen en je zult}{ook d{\`\i}t begrijpen}\\

\haiku{{\textquoteright} vroeg hij, alsof het ' '.}{niet \'e\'en uurs nachts maar pas}{n uur of tien was}\\

\haiku{Noch dat iemand het,.}{huis binnenkwam noch dat een}{schot afgevuurd was}\\

\haiku{{\textquoteright} {\textquoteleft}Wij allen, zoals, -.}{U ons hier ziet en dan nog}{signor Sibilio}\\

\haiku{Als hij komen wil,.}{zal hij in uiterlijk 10}{minuten hier zijn}\\

\haiku{Nu zal ik mij tot.}{het aller-voornaamste}{moeten beperken}\\

\haiku{Ik lette meer op,;}{dat eigenaardige beeld}{dan op zijn woorden}\\

\haiku{Even later was de,...}{grote zwarte auto om}{de hoek verdwenen}\\

\haiku{Misschien zou het in, -?}{Amerika mogelijk zijn}{maar in Europa}\\

\haiku{Ik geloof, eerlijk,,!{\textquoteright}}{gezegd dat je me voor het}{lapje houdt Donald}\\

\haiku{{\textquoteright} {\textquoteleft}Omdat ik wou, dat -.}{U samen met ons er heen}{ging naar zijn woning}\\

\haiku{Dat was de enige,.}{keer hedenmorgen dat de}{deur geopend werd}\\

\haiku{nee, die niet, - U moet.}{rekenen dat die lange}{gang er tussen ligt}\\

\haiku{Alleen als de deur,.}{van binnen-uit geopend}{wordt blijft het alarm uit}\\

\haiku{{\textquoteleft}Sluit alle deuren,.}{af zodat niemand ons in}{de rug zal komen}\\

\haiku{Als een van de drie,!}{ook maar een vinger beweegt}{dadelijk schieten}\\

\haiku{een medewerker.}{van hem ging met den patient}{naar het ziekenhuis}\\

\haiku{Ondertussen was.}{de \`echte Corelli naar de}{villa gereden}\\

\section{M. Revis}

\subsection{Uit: 8.100.000 m{\textthreesuperior} zand}

\haiku{Helaas is het slechts,.}{om te zien dat  het \'o\'ok}{geld verdienen kan}\\

\haiku{Dan komt men er wel,.}{achter dat alles op zijn}{pooten terecht komt}\\

\haiku{{\textquoteright} Ook Kees voelde, dat.}{hij voor iets meer bestemd was}{dan polderwerker}\\

\haiku{Een rossig verlicht.}{stationnetje kwam soms}{in het duister op}\\

\haiku{{\textquoteright} De journalist vindt.}{de vergelijking met het}{kantwerk zeer geslaagd}\\

\haiku{Paarden, arbeiders,,,,:}{paardenzweet menschenzweet zand}{keien en karren}\\

\haiku{Hij zou alleen graag,,.}{kinderen gehad hebben}{of \'e\'en kind een zoon}\\

\haiku{En 1930 (dit verhaal,)!}{gaat niet verder want toen stierf}{van Dool 2020 en 8.100.000}\\

\haiku{De dragline is,.}{een prachtmachine ze schept}{op en stort 2 M3}\\

\haiku{Als je een dijk kunt,.}{leggen ga je vanzelf aan}{spoorbanen denken}\\

\haiku{En zoo groeit de weg,.}{uit den grond de nieuwe weg}{voor het snelverkeer}\\

\haiku{Maar zijn daar dan niet?}{de eerste ketenen van}{het Rotsgebergte}\\

\haiku{Morgen is er weer,,,.}{een dag grauw troosteloos weer}{een Novemberdag}\\

\haiku{Er is flauw licht in,.}{de kamer maar het bed is}{in de schemering}\\

\subsection{Uit: Kringloop. De geschiedenis van een schip}

\haiku{aan den anderen,;}{kant komt het weer te voorschijn}{versmald en verlengd}\\

\haiku{Het een ontstaat uit,.}{het ander zooals de eene mensch}{uit de andere}\\

\haiku{Hij wordt nu door de.}{teekenaars theoretisch}{in elkaar gezet}\\

\haiku{In het voorjaar van {\textquoteleft}{\textquoteright}.}{1911 wordt de kiel van dePrins}{van Oranje gelegd}\\

\haiku{Het heeft den nieuwen,,.}{nog slapenden reus gezien}{daar aan den oever}\\

\haiku{Een dag later ligt {\textquoteleft}{\textquoteright}.}{dePrins van Oranje op de}{reede van Tanger}\\

\haiku{Nevels rijzen 's.}{nachts uit zee op en drijven}{de schepen uiteen}\\

\haiku{kinabast, huiden,,,,,,,.}{kapok suiker tabak rijst}{thee koffie copra}\\

\haiku{Alleen zijn snelheid {\textquoteleft}{\textquoteright}.}{is iets grooter dan die van}{dePrins van Oranje}\\

\haiku{Op een dag met ruw.}{weer vaart de U X door de}{sluizen naar buiten}\\

\haiku{In dat geluid kan,.}{men zich niet vergissen dat}{was een kanonschot}\\

\haiku{Dat gebeurt op 13,.}{Juli een mooie zomerdag}{vol zon en wind}\\

\haiku{Ten slotte doet een.}{officier de ronde door}{het verlaten schip}\\

\haiku{Ter weerszijden van,.}{het schip is nu de breede}{blinkende rivier}\\

\haiku{De wind is Noord, de,.}{zee wordt vlakker naarmate}{de middag vordert}\\

\haiku{Een tankboot stuurt zijn {\textquoteleft}{\textquoteright}.}{lange voorschip achter de}{Prins van Oranje langs}\\

\haiku{uit den schoorsteen komt,:}{een dikke walm waarin het}{schijnt te weerlichten}\\

\haiku{De andere boot {\textquoteleft}{\textquoteright} {\textquoteleft}{\textquoteright}.}{van deMaas is dePrins van}{Oranje genaderd}\\

\subsection{Uit: Paviljoen van glas}

\haiku{Potgieter steekt met;}{hoofd en schouders boven zijn}{tijdgenoten uit}\\

\haiku{Diep in hem is een,,.}{ruimte waar geen gedachten}{zijn een gapend gat}\\

\haiku{In de statuten:}{wordt het doel der maatschappij}{aldus omschreven}\\

\haiku{Een groot orkest speelt.}{de Jubel-ouverture}{van C.M. von Weber}\\

\haiku{Gelukkig blijft het.}{Paviljoen door zijn bouwtrant}{de aandacht trekken}\\

\haiku{{\textquoteright} Mevrouw Cramp slaat het,.}{boek dicht waarin zij las en}{schudt langzaam het hoofd}\\

\haiku{Cramp kijkt opmerkzaam,.}{naar de boerderijen die}{hier en daar liggen}\\

\haiku{Soms gaan de paarden.}{van een rijtuig in volle}{draf over de straatweg}\\

\haiku{Overblijfselen van,.}{een oude muur van een poort}{en een wachttoren}\\

\haiku{Het eten is hier slecht,{\textquoteright}, {\textquoteleft}.}{merkt Cramp de tweede dag op}{en de mensen stijf}\\

\haiku{Nu, die Roderick,{\textquoteright}, {\textquoteleft},?}{gaat hij verderwie is dat}{ook weer Roderick}\\

\haiku{{\textquoteleft}Dames en heren,{\textquoteright}, {\textquoteleft}.}{roept hijdie oude man moet}{geholpen worden}\\

\haiku{Maar de Turkse vrij;}{scharen trappen de vonken}{in Bulgarije uit}\\

\haiku{Anderen vallen.}{op de knie en laten het}{hoofd langzaam zakken}\\

\haiku{Maar het gebouw staat.}{er al. Aan voortvarendheid}{mankeert het hem niet}\\

\haiku{Dat zijn karakter,.}{niet iedereen aanstaat is}{begrijpelijk}\\

\haiku{Ik bedoel dat ik.}{het niet wenselijk acht zulks}{bekend te maken}\\

\haiku{Er moet veel meer kort.}{papier in omloop zijn dan}{de balans vermeldt}\\

\haiku{Het gehoor luistert,.}{aandachtig alsof het ging}{om een reis naar Mars}\\

\haiku{Staat daar nog niet de,?}{oude Cramp doorzichtig als}{een geestverschijning}\\

\haiku{De schepping van Dr!}{d'Espina in handen van}{iemand als Mr Cramp}\\

\haiku{Gelukkig voor Het.}{Bericht doet zich plotseling}{een nieuw geval voor}\\

\haiku{De directeur wrijft.}{zich in de handen over zijn}{tactisch optreden}\\

\haiku{hoogste notering,.}{tussen 1900 en 1907 is 16}{geweest laagste 9}\\

\haiku{Intussen, dingen.}{als deze komen bij meer}{maatschappijen voor}\\

\haiku{Reeds lang is Mr Cramp.}{gewend met het Paviljoen}{te doen wat hij wil}\\

\haiku{Hij installeert een.}{maitresse op kamers in}{de Hemonylaan}\\

\haiku{Schrijf, ploerten, schreeuw je,:}{hees op vergaderingen}{schuimbek van woede}\\

\haiku{En een opvolger,.}{moet {\`\i}k aanwijzen maar voor}{U doe ik het niet}\\

\haiku{Je bent de langste,.}{tijd directeur geweest wees}{daar maar zeker van}\\

\haiku{De verbijstering,.}{maakt plaats voor misnoegen het}{misnoegen wordt haat}\\

\haiku{Eigenlijk is zijn.}{levenslange vleierij}{niets dan haat geweest}\\

\haiku{Tegen iemand die}{knoeit mag je nooit zeggen dat}{het een knoeier is}\\

\haiku{Hij heeft Romeinse,;}{munten verkocht die in Aken}{werden geslagen}\\

\haiku{{\textquoteright} vraagt Stomaski, {\textquoteleft}.}{en neemt het deksel van een}{kistdit is marmer}\\

\haiku{Na de crisis van.}{1921 is het scheepvaartverkeer}{weer toegenomen}\\

\haiku{gegadigden voor,.}{zijn project te vinden met}{betuiging van spijt}\\

\haiku{Hij houdt hen een poos,.}{aan de praat om hen geld uit}{de zak te kloppen}\\

\haiku{De mensen moeten;}{de inkomstenobligatie uit}{overtuiging nemen}\\

\haiku{Stugheid van de een,.}{die de ander nog meer op}{zijn hoede doet zijn}\\

\haiku{Aan dat soort mensen.}{heb ik alle ellende}{te danken gehad}\\

\haiku{Vooruitstrevende;}{architecten spreken van}{het Nieuwe Bouwen}\\

\haiku{Het Paviljoen is,.}{oud het gaat gebukt onder}{de last der jaren}\\

\haiku{Meermalen bezoekt.}{Mr Cramp de voorstellingen}{in de schouwburgzaal}\\

\haiku{{\textquoteright} Paraat betoont zich, {\textquoteleft}?}{ongeduldigwaar blijft dat}{oude kavalje}\\

\haiku{Zulke dingen zijn?}{toch schering en inslag in}{het zakenleven}\\

\haiku{Het publiek heeft in:}{deze jaren Mr Cramp een}{bijnaam gegeven}\\

\haiku{Hij is mensenschuw,,,.}{zegt men hij heeft kind noch kraai}{hij ontmoet niemand}\\

\haiku{Hoe kunnen deze?}{en deze obligaties nog}{te voorschijn komen}\\

\haiku{Nu heeft het publiek.}{bijna geen belang meer bij}{de gang van zaken}\\

\section{F.H. Rikken}

\subsection{Uit: Codjo, de brandstichter}

\haiku{{\textquoteright} hernam Codjo.}{als begreep hij de herkomst}{van de kippen niet}\\

\haiku{Maar zou je er zelf,,?}{niet heengaan daar jij hem toch}{beter kent dan ik}\\

\haiku{overal vervolgd en, '}{als een stuk wild opgejaagd}{mag je eersts avonds}\\

\haiku{Slaap nu wel, nu geen.}{meester de uren uwer rust meer}{telt en beknibbelt}\\

\haiku{Werktuigelijk ging,:}{Present naar haar toe waarop}{zij tot hem zeide}\\

\haiku{Waarom had hij haar?...}{niet eerst met zijn voornemen}{in kennis gesteld}\\

\haiku{{\textquoteleft}Dan zullen wij hen {\textquotedblleft}{\textquotedblright}!}{ook denSpaanschen bok laten}{voelen en goed ook}\\

\haiku{mijn jongen{\textquoteright} riep Tom, {\textquoteleft},.}{blijde uitjij weet wat een}{ouden man toekomt}\\

\haiku{{\textquoteleft}Ik voel, dat het goed,{\textquoteright}.}{doet aan mijn oude knoken}{zeide Tom vroolijk}\\

\haiku{daar kreeg je driemaal,.}{daags een soopie behalve wat}{men er nog bij nam}\\

\haiku{Het was een zware.}{arbeid op de plantages}{met waterwerken}\\

\haiku{Ik heb hem geheel,.}{in mijn macht en kan met hem}{doen wat ik verkies}\\

\haiku{{\textquoteleft}Wat zou jij kleine,,?}{nietige spin vermogen}{tegen den tijger}\\

\haiku{Je zult mij in de,!}{oogen der menschen in eere}{herstellen brrr}\\

\haiku{{\textquoteright} hernam de tijger.}{door haar tegenstreving al}{meer en meer gesard}\\

\haiku{{\textquoteright}, vroeg Codjo na.}{eenige oogenblikken van}{pijnlijk stilzwijgen}\\

\haiku{Ik wilde weten,,,.}{Afie of gij mij z\'o\'o bemint}{als ik u liefheb}\\

\haiku{Ik houd veel van mijn.}{geloof en ik zal daaraan}{nooit ontrouw worden}\\

\haiku{{\textquoteright} {\textquoteleft}Wie heeft je gezegd,?}{dat je veracht wordt en dit}{omdat je slaaf bent}\\

\haiku{De blanken hebben;}{ten koste van ons zwoegen}{rijkdom verworven}\\

\haiku{de oude wil ons,.}{jongeren gebruiken om}{voor hem te werken}\\

\haiku{Ik wilde daarom,,.}{maar dat wij wisten waar wij}{iets kunnen vinden}\\

\haiku{Zie eens, dat prachtig.}{stuk zoutevleesch hebben}{zij meegenomen}\\

\haiku{{\textquoteleft}Wil jij mij in den ()?}{nacht naar de kapoewerie}{struikgewas brengen}\\

\haiku{{\textquoteleft}Ik kwam hem vragen.}{of ik wariembo van hem}{zou kunnen krijgen}\\

\haiku{Hij herstelde zich:}{echter zoo goed mogelijk}{en vroeg brutaalweg}\\

\haiku{Ik zal jou even zoo.}{goed ls de overigen in}{mijn wraak verdelgen}\\

\haiku{{\textquoteleft}Maar nu hij er zelf,}{op uit gaat vertrouw ik hem}{weer geheel en al.}\\

\haiku{Dan zullen ook wij!}{eens gen{\'\i}eten van de vruchten}{van onzen arbeid}\\

\haiku{Zij veracht mij even,.}{diep in haar hart als ik haar}{haat in het mijne}\\

\haiku{Wil je haar dooden of?}{haar alleen een langdurig}{lijden overzenden}\\

\haiku{Dat is iets, wat ik,.}{van iemand gekregen heb}{om te verkoopen}\\

\haiku{misie kan alles,.}{koopen later zal ik het}{geld wel ontvangen}\\

\haiku{baas Willem behoeft,. '}{niet te weten hoe ik er}{aan gekomen ben}\\

\haiku{{\textquoteright} {\textquoteleft}Ach, hoe zou iemand,?}{het maken die steeds in vrees}{en angst moet leven}\\

\haiku{{\textquoteright} {\textquoteleft}Daarom heb ik het,{\textquoteright}.}{zelf maar gedaan antwoordde}{Codjo spottend}\\

\haiku{{\textquoteleft}Ik ben nu al een.}{paar maanden vrij en het gaat}{mij nog zoo slecht niet}\\

\haiku{Daar zullen wij dus,.}{nogal iets kunnen vinden}{wat ons te pas komt}\\

\haiku{Maar zou je er dan?}{tegen hebben mij all\'e\'en}{den weg te wijzen}\\

\haiku{Het wordt eindelijk,.}{eens tijd om de handen uit}{de mouw te steken}\\

\haiku{En wat geeft het ook,{\textquoteright}, {\textquoteleft},.}{ging hij voortis het hier niet}{dan ergens anders}\\

\haiku{In de keuken heb.}{ik bananen en switi}{moffo gevonden}\\

\haiku{Een oogenblik bleef}{Codjo met een helschen}{lach om de lippen}\\

\haiku{de eene trok naar den,.}{Waterkant de andere}{naar de Maagdenstraat}\\

\haiku{Zou je ze niet voor?}{mij kunnen bewaren of}{wellicht verkoopen}\\

\haiku{Dit is nu nog maar,{\textquoteright}.}{het begin voegde hij er}{overmoedig aan toe}\\

\haiku{Hij bracht als buit een.}{kalkoen en eenige stukken}{katoen mede}\\

\haiku{Welnu, ouroe:}{fajatiki no de plei}{foe teki faja81}\\

\haiku{{\textquoteleft}En zoo juist wilde!}{je mij overhalen in het}{bosch te gaan leven}\\

\haiku{{\textquoteleft}Ik ben het{\textquoteright}, zeide,.}{Present toen hij de stem van}{Betsy herkende}\\

\haiku{{\textquoteright} {\textquoteleft}Ik ga er niet heen,{\textquoteright}, {\textquoteleft}.}{zeide deze beslistals}{Present niet meedoet}\\

\haiku{Codjo blies er.}{met kracht in en weldra sloeg}{de vlam er uit op}\\

\haiku{Ik nam den koffer,,.}{op het hoofd hij was niet zwaar}{daar er niets in was}\\

\haiku{Ik schoof den koffer.}{bij de andere kisten}{en ging de trap af}\\

\haiku{Kom jij intusschen.}{maar voorloopig met mij}{mee naar het Piket}\\

\haiku{{\textquoteleft}Ik heb vanmorgen,{\textquoteright}.}{den korf half open gevonden}{hernam Frederik}\\

\haiku{{\textquoteright}, beval hij, {\textquoteleft}kijk eens,.}{hier en daar rond of de kip}{niet te vinden is}\\

\haiku{Vannacht moet ieder,.}{onzer maar zien dat hij een}{onderkomen vindt}\\

\haiku{Ik dacht bepaald, dat,.}{er onweer zou komen z\'o\'o}{benauwd vond ik het}\\

\haiku{Men wil mij voor het,.}{gerecht dagen nadat men}{mij bestolen heeft}\\

\haiku{Jammerend liep Tia.}{door het huis bij het verhaal}{van de ontdekking}\\

\haiku{{\textquoteleft}Het is hier te koud,.}{in het gras laat ons een goed}{heenkomen zoeken}\\

\haiku{Dit gezegde was,:}{dubbelzinnig daar het even}{goed kon beteekenen}\\

\haiku{Deze zeide niets;}{en gaf zelfs niet door kreten}{zijn smart te kennen}\\

\haiku{{\textquoteleft}Wat heb je dan met,?}{dat stuk bont gedaan dat ik}{je gegeven heb}\\

\haiku{{\textquoteleft}Wat heb ik je dan,?}{gedaan dat jij je ziekte}{aan mij wilt wijten}\\

\haiku{Omdat tante bang,....}{was dat je haar daarmede}{kwaad zoudt aanbrengen}\\

\haiku{De herinnering,;}{aan al het leed dat zij had}{moeten verduren}\\

\haiku{{\textquoteleft}Ik bid u, tante,,.}{vergeef mij indien ik u}{leed veroorzaakt heb}\\

\haiku{{\textquoteleft}Het is te spoedig,.}{geleden om ons veel vrees}{in te boezemen}\\

\haiku{Hij heeft dat alles,.}{geleden opdat gij niet}{verloren zoudt gaan}\\

\haiku{{\textquoteleft}Codjo{\textquoteright}, hernam,.}{zij terwijl haar eene rilling}{door de leden voer}\\

\haiku{hij was de slaaf van.}{Mary Rose Herbert en}{niet van Makenzie}\\

\section{Herman Robbers}

\subsection{Uit: De bruidstijd van Annie de Boogh}

\haiku{de sterren te zien...,}{schijnen door de zoldering}{van zijn slaapkamer}\\

\haiku{Hij passeerde al '.}{gauw het huis en keek het in}{t voorbijgaan aan}\\

\haiku{Je wilt je zeker ',....}{nog wel wat opfrisschen voor}{teten h\`e jongen}\\

\haiku{wat kwam het er op,!...}{aan tien of twaalf dagen van}{onnadenkendheid}\\

\haiku{ze gaven niets om.......}{wat voor hem het belangrijkst}{was en omgekeerd}\\

\haiku{t Waren ook zoo'n,.}{paar besten en zij hielden}{van hem om hem zelf}\\

\haiku{ze zouden er z\'o\'o,,}{passeeren maar je kondt er}{bij avond niets van zien}\\

\haiku{'t zou eigenlijk.}{heel goed voor hem zijn als hij's}{wat tegenspoed kreeg}\\

\haiku{Misschien hebben we,....}{wel tijd er even heen te gaan}{na de receptie}\\

\haiku{De moeder gaf hem;}{bijna geen gelegenheid}{naar haar te kijken}\\

\haiku{{\textquoteleft}En wat vindt-je van,{\textquoteright}, {\textquoteleft}?}{de anderen vroeg Louisvan}{Papa en Mama}\\

\haiku{Maar zoo met fixeeren....}{en naloopen was hij niet}{verder gekomen}\\

\haiku{Hij had ook al gauw}{weten op te merken wat}{voor soort van meisje}\\

\haiku{Hij had nooit gedacht,....}{dat het z\'o\'o groot kon zijn het}{verschil tusschen hen}\\

\haiku{Voor zich-zelf wou ';}{hijt niet weten dat het}{was om haar alleen}\\

\haiku{Maar ze moest het toch....}{ook wel hooren en Annie}{schaamde zich voor haar}\\

\haiku{Ze had nu uit de;}{verte kunnen hooren dat}{hij z'n rok aanhad}\\

\haiku{hij was er veel te,....}{hijgerig-onrustig voor}{liep al maar verder}\\

\haiku{Hij ging, den Maaskant,.}{volgend langs het park tot bij}{den ouden Heuvel}\\

\haiku{Dat was goed en mooi,....}{een schoone opbloei van zijn}{beste neigingen}\\

\haiku{moe van 't zitten....}{maakte elk bewegingen}{van willen opstaan}\\

\haiku{de nieuwe zou niet.}{in huis komen voor  dat}{Annie er uit was}\\

\haiku{Het bruidje was er,.}{aan gewoon aan die stemming}{en aan die standjes}\\

\haiku{Op een morgen met}{haar moeder alleen had ze}{er van gesproken}\\

\haiku{of mama dan een....}{beetje vriendelijker voor}{haar zijn wou voortaan}\\

\haiku{Het zou een al te '.}{wreede teleurstelling zijn}{alst niet zoo was}\\

\haiku{wat was dat nu stom,!...}{hij had zoo gemakkelijk}{uit kunnen blijven}\\

\haiku{Toen de bruigom weg '.}{was bleef ze een heelen tijd}{alleen aant woord}\\

\haiku{Toen kwamen ze in...,,....}{de drukke winkelbuurt het}{Boymansplein de Blaak}\\

\haiku{Ze maakte 't dus,....}{gauw af zei dat ze nog wel}{terug zou komen}\\

\haiku{Ze waren nu klaar,....}{met hun boodschappen konden}{dus wel naar huis gaan}\\

\haiku{Langzaam, een beetje....}{stroef en knarsend ging de deur}{naar achteren open}\\

\haiku{Hij had het alles...,,....}{te voren wel vermoed ja}{bijna geweten}\\

\haiku{Rillend en pijnlijk,,}{kroop hij z'n bed in maar z'n}{moeheid was zoo groot}\\

\haiku{s Middags sloot hij,.}{zich op in zijn kamer ging}{op z'n bed liggen}\\

\haiku{Het zingen was iets,.}{te laat ingevallen maar}{dat herstelde zich}\\

\haiku{als hij het los mocht...,,,!...}{maken dat het kon zwieren}{leven bewegen}\\

\haiku{Hij verkneukelde,....}{zich als een smulpaap genood}{op een fijn diner}\\

\haiku{Annie zat met haar,....}{rug naar zijn tafel gekeerd}{aan de volgende}\\

\haiku{Hij was sinds lang aan,.}{zoo weinig gewoon hij kon}{er niet meer tegen}\\

\haiku{Drinken kon hem nu....}{den roes niet geven waaraan}{hij behoefte had}\\

\haiku{Louis ging trouwen, Paul...,!}{wou nog geen tien dagen bij}{haar blijven och God}\\

\haiku{Toen hij 's middags, '.}{thuis kwam was Paul al opt}{punt van vertrekken}\\

\haiku{Die geschiedenis,!...}{van gisteren-avond dat}{was toch zoo erg niet}\\

\haiku{waarom?... Ze deed ook,....}{den mond wel open maar haar keel}{liet geen klanken door}\\

\haiku{Ze voelde zich of,....}{ze op-eens was veranderd}{herkende zich niet}\\

\haiku{hij wist het toch maar,!...}{hij had den slag beet om zoo'n}{meisje te boeien}\\

\haiku{En ze sloot het raam,....}{weer met een licht gevoel van}{teleurgesteld zijn}\\

\haiku{Maar \'e\'en tegelijk, ',....}{nement is vergif had}{de dokter gezegd}\\

\haiku{Haar hart bonkte zoo,....}{hevig op dat ze bang was}{dat het geluid gaf}\\

\haiku{Om kwart over vijven ',.}{zette zen hoed op deed}{haar manteltje om}\\

\haiku{Langzaam, voorzichtig,...,....}{draaide ze den sleutel om}{toch knarste die even}\\

\haiku{Maar een schilder is,,....}{een man en hij een jonge}{sterke jonge man}\\

\haiku{Een ander had haar,,....}{nu bezat haar nu had macht}{en rechten over haar}\\

\haiku{Zij snikte, snikte..., '....}{maart was alsof dit leed}{door tranen groeide}\\

\haiku{Mocht hij haar dan nu,:}{laten merken dat hij haar}{begeerde vragen}\\

\haiku{toch liet hij haar niet,,:}{los maar bracht z'n mond bij haar}{oor nu fluisterde}\\

\haiku{toch  wist hij nooit....}{het goddelijke zoo dicht}{te zijn genaderd}\\

\subsection{Uit: Een mannenleven. Deel 2. Op hooge golven}

\haiku{{\textquoteright} Terwijl liepen Huib.}{en Bl\'ecour met Gerbrandts mee}{naar zijn kleedkamer}\\

\haiku{Maar nog v\'o\'or ze daar,!}{waren traden hun haastig}{twee vrouwen opzij}\\

\haiku{En dan Janne en,,....}{Ruth nog nietwaar en Driesse}{en Melchior Spin}\\

\haiku{Schoonheid, kunst, beide;}{zoo volmaakt subjectieve}{begrippen trouwens}\\

\haiku{Men zat er veilig,,,;}{warm en onder elkaar in}{dit nachtelijk uur}\\

\haiku{Ik kom morgen bij,,.}{je Gerbrandts en dan zullen}{wij erover praten}\\

\haiku{Toch, zou ik zeggen,.}{moest je die menschen nu hun}{gang maar laten gaan}\\

\haiku{'is,{\textquoteright} vervolgde zij,;}{en haar gezichtje werd}{plotseling ernstig}\\

\haiku{Hoe denkt hij er nu,?}{over onze beste vriend en}{vereerde auteur}\\

\haiku{Gerbrandts lachte met,:}{verruktgroote oogen stak Huib zijn}{beide handen toe}\\

\haiku{Geen repetities,....}{vandaag dank zij n\^otre cher}{ma{\^\i}tre Huib Hoogland}\\

\haiku{je moet het wel vreemd....}{vinden dat een meisje je}{zulke dingen zegt}\\

\haiku{{\textquoteleft}Ik zal trouwens niet,....}{zeggen dat je heelemaal}{ongelijk hebt maar}\\

\haiku{{\textquoteleft}Ja,{\textquoteright} zei Janne, {\textquoteleft}ik...., '.}{heb ook een moeder gehad}{neek heb haar n\'og}\\

\haiku{Wat een leed heeft het,.}{me gedaan dat ik niet kon}{komen gisteravond}\\

\haiku{Gisterenavond is,...., '....}{hij er weer geweest en nou}{k heb dan beloofd}\\

\haiku{al zou ik niet met,....}{hem kunnen trouwen al was}{hij getrouwd desnoods}\\

\haiku{{\textquoteleft}Bedenk toch, ik ben,....}{\'e\'en-en-dertig en}{ik ben een meisje}\\

\haiku{Al scherper zag hij,....}{de sc\`enes h\'o\'orde hij den}{toon der dialogen}\\

\haiku{Het k\'on toch niet, dat,....}{het hem enkel om d\'at zou}{te doen zijn om d\'at}\\

\haiku{Hoe vreemd en ernstig, ',!}{die brandende oogen int}{strakke bleeke gezicht}\\

\haiku{En dit zonder dat.}{zij er zich een van beiden}{over verwonderden}\\

\haiku{Hij blikte zijn vrouw,.}{in het oogenzwart dat hem}{helder toeglansde}\\

\haiku{{\textquoteleft}Nou zeg, wat vinden,,?}{jelie zal ik doorlezen}{of de rest morgen}\\

\haiku{de bitterheid van,.}{haar nederlaag het scheen de}{zijne geworden}\\

\haiku{Ik dacht, om je de,.}{waarheid te zeggen dat het}{juist het zwakste was}\\

\haiku{Ze stond stil, hield nog,.}{altijd zijn hand vast zoodat ook}{hij moest blijven staan}\\

\haiku{Plotseling schreef hij,,;}{een nieuw begin en zie dit}{werd dadelijk goed}\\

\haiku{{\textquoteleft}O jij - jij - jij,{\textquoteright} kwam.}{er eindelijk schor van zijn}{trillende lippen}\\

\haiku{Maar een bittere....}{trekking van zijn onderlip}{bleef nog wijlen}\\

\haiku{Er is dan ook geen,.}{grappenmakerij bij zooals}{de vorige maal}\\

\haiku{Die amsterdamsche,.}{voorstellingen Huib sloeg er}{geen enkele over}\\

\haiku{Hij forceerde het,.}{wel maar dan wreekte dat zich}{door verergering}\\

\haiku{Huib was er zacht en, '.}{wijd ontroerd van toen zes}{avonds naar huis spoorden}\\

\haiku{En gejacht liep hij -;}{door naar de trem wat was het}{weer laat geworden}\\

\haiku{Til trouwens, bang voor,....}{scherpe woorden deed haar best}{om af te leiden}\\

\haiku{jij bent nou wel erg,;}{beroemd maar wat heb je daar}{nou eigenlijk aan}\\

\haiku{Een gesprek met de.}{Doescates had dien twijfel}{voedsel gegeven}\\

\haiku{Heb je geen partij,....}{gekozen voor dat vrouwtje}{en je kwaad gemaakt}\\

\haiku{Werk jij maar, zoek jij.... '....}{maark Heb geweldig veel}{feducie in jou}\\

\haiku{Huib zou hij heeten, ',.}{alst een jongen was had}{Co al geschreven}\\

\haiku{En hij schreef het ook,.}{dadelijk aan Co dat hij}{er zoo blij mee was}\\

\haiku{Maar nog zoo zelden.}{had hij iets van de moeder}{in Janne ontdekt}\\

\haiku{Maar ze dorst daar haast,.}{nooit naar te vragen daarop}{te zinspelen zelfs}\\

\haiku{Wie zou gelukkig?}{kunnen zijn met een vrouw die}{hem geheel begreep}\\

\haiku{En ook zijn werk voor.}{de zaken kreeg de aandacht}{niet die het noodig had}\\

\haiku{Meneer van der Kamp,....}{u als onze technische}{specialiteit}\\

\haiku{Daar had ik al zoo,{\textquoteright}.}{eenig idee van knikte Huib met}{zijn zelfden glimlach}\\

\haiku{De heeren drukten,,,....}{hem \'e\'en voor \'e\'en de hand en}{waren vertrokken}\\

\haiku{{\textquoteright} {\textquoteleft}'k Wou 'k het \'o\'ok,{\textquoteright},.}{kon zeggen bromde Huib met}{even een blik naar Til}\\

\haiku{Wat kan jou nou in.}{godsnaam de sympathie van}{het publiek schelen}\\

\haiku{Z\'o\'o alleen kan je........}{hun wat schoonheid geven wat}{troost en verheffing}\\

\haiku{Ik ben narrig en,,....}{humeurig tegenwoordig}{prikkelbaar lastig}\\

\haiku{En wendde schielijk.}{het hoofd om de betraande}{oogen te verbergen}\\

\haiku{Bedrog is volmaakt,.....}{onnoodig juist omdat we zoo}{vrij zijn allebei}\\

\haiku{{\textquoteleft}Onmogelijk,{\textquoteright} zei,, {\textquoteleft},.}{Spin rustignee-nee dat}{is onmogelijk}\\

\haiku{Ze kwam haast nooit meer.}{uit zichz\'elf en Huibs vrouw was}{daar verwonderd over}\\

\haiku{Huib mocht haar brengen,,,,?}{o ja heel graag maar dan tot}{aan de trem niet waar}\\

\haiku{Want bepaald met de, '.}{trem  wou ze gaan als hij}{t niet kwalijk nam}\\

\haiku{{\textquoteright} En ze glimlachte,.}{met een doffen blik kuste}{Janne vaarwel}\\

\haiku{trek je daar dan niet ',,....}{te veel vanan hoor en blijf}{geduldig met hem}\\

\haiku{{\textquoteright} De oude vingers {\textquoteleft},,,,.}{drukten Tils hand.Niks kindje}{niks hoor schrik maar niet}\\

\haiku{{\textquoteright} Vier weken later - -.}{Til en Huib aan haar bed stierf}{mevrouw Molano}\\

\haiku{Dol!{\textquoteright} En of hij straks,;}{even mee naar beneden ging}{een oogenblik maar}\\

\haiku{Even keek hij op zijn,,.}{horloge dacht aan thuis aan}{die hem wachtten daar}\\

\haiku{Hoe had hij het \'o\'oit,,....}{kunnen denken vroeger dat}{zulke gedachten}\\

\haiku{{\textquoteleft}Als 'k mevrouw dan....{\textquoteright} {\textquoteleft},,!}{maar niet ophoud tenminste}{O welnee welnee}\\

\haiku{Hoogland deed een stap,;}{achterwaarts maar zij bleef zich}{aan hem vastklemmen}\\

\haiku{in een seconde;}{van aarzeling had hij haar}{schouders gegrepen}\\

\haiku{{\textquoteright} Intusschen waren,.}{ze doorgeloopen zwegen}{beiden  een poos}\\

\haiku{Alsof er gevaar:}{voor iets was en hij ijlings}{terugverlangde}\\

\haiku{Een oogenblik scheen,.}{het inderdaad of Huib zich}{op hem werpen ging}\\

\haiku{Het haar schrijven zou,,.}{gemakkelijker zijn maar}{onmannelijk laf}\\

\haiku{Hij is veel ouder,,,....}{dan ik dan jij zelfs en zoo}{ziek zoo ellendig}\\

\haiku{Volkomen waar is,;}{het zeker niet zei een stem}{in zijn binnenste}\\

\haiku{Maar zich ingedacht ',.}{had hijt nog in geenen deele}{het spreken tot Til}\\

\haiku{Daar was ze dan nu,,.}{zijn herwonnen vrijheid zijn}{vrede zijn zielsrust}\\

\haiku{Het zal zoo heerlijk,,....}{zijn daar te wonen samen}{jij altijd bij me}\\

\haiku{Een vroegrijp meisje,....}{wat al  te ontijdig}{bewust en vroegrijp}\\

\haiku{Zou hij dan alles,....?}{\'e\'ens moeten verliezen}{z\'o\'o eenzaam blijven}\\

\haiku{En niets meer dat Til,,.}{en Huib scheidde vervreemdde}{dat tusschen hen stond}\\

\haiku{IJlte alleen, van,,....}{verzwegen gedachten nu}{ja dat bleef altijd}\\

\haiku{{\textquoteleft}Vergeef me, kerel,,....}{je hebt gelijk het was een}{misselijke grap}\\

\haiku{{\textquoteleft}Ik begrijp je, Mels,.}{ik begrijp je zorg en je}{verontwaardiging}\\

\haiku{als je dit maar weer,....}{eens gehad hebt dan kan je}{er wel weer tegen}\\

\subsection{Uit: Roman van een gezin. Deel 1. De gelukkige familie}

\haiku{Wel, u begrijpt toch, '!....}{dat zet volgend jaar om}{het restje kwamen}\\

\haiku{Ieder jaar - een tijd -!....}{lang zelfs iedere maand was}{de winst gestegen}\\

\haiku{Agenten beproefden.}{vergeefs de beweging er}{weer in te brengen}\\

\haiku{{\textquoteright} Voor de kinderen,;}{was het een bizondere}{dag belangwekkend}\\

\haiku{Intusschen zijn er....}{bij ons toch een stuk of wat}{binnengekomen}\\

\haiku{{\textquoteleft}Excuus vragen, denk,!}{ik d'r hangende pootjes}{laten bekijken}\\

\haiku{Je behoeft zoo bang,,,!}{niet te kijken Emmie zoo'n}{ramp is het niet hoor}\\

\haiku{{\textquoteleft}Och! 't Waren een,....}{paar van die lui die vandaag}{nog gewerkt hebben}\\

\haiku{De menschen konden......}{ook dikwijls zoo aanhouden}{dringen en liefdoen}\\

\haiku{Als mama haar dan, '?}{toch niet gelooven wou waarom}{vroeg zet haar dan}\\

\haiku{Jan zei met recht, wees, ',?....}{maar blij als jer niets mee}{te maken hebt h\`e}\\

\haiku{Jans vage {\textquoteleft}geloof{\textquoteright},.}{was het hare ook Jan sprak}{haar meeningen uit}\\

\haiku{Slecht zagen ze er,,.}{uit vreeselijk bleek en voozig}{de zetters vooral}\\

\haiku{Ze geloofde ook.}{eigenlijk niet dat Theo er}{zelf naar verlangde}\\

\haiku{Ook het andere,....}{raam werd hoog opgeschoven}{gordijnen op zij}\\

\haiku{- hij proefde vooruit.}{al het geestesgenot van}{erover te praten}\\

\haiku{heb je 't gehoord,{\textquoteright},.}{riep hem Adam toe hoogheesch}{door zijn opwinding}\\

\haiku{De Bries alleen hield,.}{zich meest wat op zij schoon hij}{lachende toekeek}\\

\haiku{En een nieuwe brief {\textquoteleft}{\textquoteright}.}{van hetlooncomit\'e dat}{niet eens werd erkend}\\

\haiku{Ze vroeg of Noortje.}{haar eigen kamer nu ook}{eens wou laten zien}\\

\haiku{Er zijn er wel, die,.}{iets toegeven willen ten}{minste in werkuren}\\

\haiku{Ru zat rechts naast haar -,;}{hij praatte zoo kregel zoo}{heftig in-eens}\\

\haiku{De vorige rees....}{als een naar visioen voor}{zijn tobbenden geest}\\

\haiku{Enfin, zoo'n jongen,.... '!...}{dat gaat wel overn Beetje}{sentimenteelig}\\

\haiku{Nieuw, jong, frisch, telkens.}{rumoeriger ontwaken}{de Maandagmorgens}\\

\haiku{Gejuich, zwaar dreunend,,.}{hoed-en-pet-gezwaai}{van alle kanten}\\

\haiku{Die vroegen of de,,....}{opslag een cent per uur ook}{daar werd gegeven}\\

\haiku{Wel vijftig, zestig, {\textquoteleft}{\textquoteright} - '.}{bleven op de keien zoo}{stond int volksblad}\\

\haiku{Want zij hadden het, '....}{nu immers ondervonden}{t hielp altijd iets}\\

\haiku{ging een oogenblik,.}{later de deur uit nog even}{een boodschap doen}\\

\haiku{soeziger zat hij.}{een beetje verzakt aan den}{fonkelenden disch}\\

\haiku{Louise Heugens en - '!}{Gonne van de Palsn paar}{contrasten die twee}\\

\haiku{'k zou ook nog best ',?,... '!}{is de lucht in willen h\`e}{t Is zoo heerlijk}\\

\haiku{Maar Jeanne ging,:}{met de anderen mee tot}{bij Baatz voor de deur}\\

\haiku{Nou bonjour,{\textquoteright} zei ze, {\textquoteleft},,,....,....}{haastigtoe Th\'e zeg vooruit}{nou Dag Gonne dag}\\

\haiku{En ze vond ook maar, -.}{goed dat het uit was nu dat}{hij maar niet meer kwam}\\

\haiku{als Jeaan nu toch...!}{eenmaal niet genoeg van hem}{hield wel natuurlijk}\\

\haiku{'t scheen wel als had.}{ze geen flauw vermoeden van}{dieper bestaan}\\

\haiku{Ook was er volstrekt,,,;}{geen reden voor Emma om}{bang te zijn vond hij}\\

\haiku{Jaloersch was Theo, en,,,.}{Ada veeleischend coquet liet}{zich duchtig gelden}\\

\haiku{Waren papa en,,....}{Theo weg na het koffiemaal}{viel er rust in huis}\\

\haiku{Die was trouwens rijk,, ' '!....}{van z'n vrouw wat kontem}{ten slotte schelen}\\

\haiku{Of - en ze bloosde - '!....}{bij die gedachte \`oft}{werd nog veel mooier}\\

\haiku{Maar Frans ging er heen,,;}{een visite maken met}{Theo en Jeanne}\\

\haiku{Wat we buitendien,.... '....}{moeten uitgeven zie je}{t Dagelijksche}\\

\haiku{waar moeten we d\'an,, ' '!}{op bezuinigen Jank}{weetet heusch niet}\\

\haiku{Hij was dan niet in!}{de wereld geschopt om voor}{vreemden te zorgen}\\

\haiku{Om in korten tijd,....}{veel bij elkaar te krijgen}{was d\'at de manier}\\

\haiku{Ongetrouwd van je, '!}{leven genietent is}{wel zoo verkieslijk}\\

\haiku{Een deftig meisje,,....}{van oude familie veel}{goede connecties}\\

\haiku{de werkelijke,,....}{trouwdag dag van receptie}{cadeaux en bloemen}\\

\haiku{Zijn houding  moest,...}{vroolijk en vriendelijk zijn}{blij en hoffelijk}\\

\haiku{De sneeuw lag, als smet,....}{van de straat op de bloemen}{die later kwamen}\\

\haiku{telkens keek ze haar,.}{bruigom aan  met plezier}{in zijn vroolijkheid}\\

\haiku{niets geen plezier meer,....}{kon hebben daarom in de}{zilveren bruiloft}\\

\haiku{den meesten scheen het;}{toch wat kras zoo dadelijk}{na het vele eten}\\

\haiku{En werkelijk, de,.}{kop werd prachtig leek sprekend}{op den ouden Croes}\\

\haiku{In h\'o\'ofdzaak was Croes,,....}{toch prachtig geslaagd en hij}{hoopte Van Oosthoff}\\

\haiku{Ze kon niet verder,,.}{komen gaf hem een hand en}{hij lachte nerveus}\\

\haiku{Ze bleven beleefd,:}{en welwillend maar ietwat}{teruggetrokken}\\

\haiku{ofschoon dat toch het,,}{natuurlijke doel van zoo'n}{studie is nie-waar}\\

\haiku{Kijk 's, als 't je....}{nou voornamelijk te doen}{is om wat meer werk}\\

\haiku{Dat wordt er dan w\'e\'er....}{een die voortdurend alleen}{op d'r kamer zit}\\

\haiku{Zijn vrienden hadden.}{al klaar gestaan met het open}{rijtuig en den krans}\\

\haiku{Ze had geen nieuwen '.}{wintermantel gekochtt}{vorige najaar}\\

\haiku{Jeanne was er,,;}{het laatste jaar ook al niet}{op vooruitgegaan}\\

\haiku{de hartelijkheid.}{van den zilverenbruilofsdag was}{gauw voorbij geweest}\\

\haiku{En letteren, hoe,?}{k\'om je aan letteren wat}{wou je dan worden}\\

\haiku{Doch dit wisten zijn,.}{vrienden alleen zijn vader}{merkte er niets van}\\

\haiku{Het wordt nou toch wel, ', '....{\textquoteright} {\textquoteleft}}{erg warm hier ik g\'a maar weer}{is geloofkNee}\\

\haiku{Als-t-ie 't erg,....}{druk heeft gehad op kantoor}{of slechte zaken}\\

\haiku{dan mag je 'n mooie ';}{nieuwe japon koopen of}{n hoed of zoo iets}\\

\haiku{Ik vind alleen, een, '...{\textquoteright} {\textquoteleft}....}{grachtehuist heeft toch nog}{veel meer cachetJa}\\

\haiku{{\textquoteright} Emma begon te,.}{huilen ze was in-eens}{heelemaal overstuur}\\

\haiku{Jan, het gaat niet met....}{enkel  twee meiden in}{zoo'n groot huis als dit}\\

\haiku{En zij zijn er nou,....}{ook eenmaal aan gewend dat}{juf van alles doet}\\

\haiku{werd hij een walging,,,}{in zich gewaar een afkeer}{niet te overwinnen}\\

\haiku{hen beiden kwam tot,:}{een praatje uitgebreider}{dan het gewone}\\

\haiku{De familie ging;}{niet naar buiten en het was}{een natte zomer}\\

\haiku{En klagen, klagen,,...}{steen en been tegen ieder}{die ze ontmoette}\\

\haiku{wat of het wel was,,...}{voor een mensch die vrouw en hoe}{of het er toeging}\\

\haiku{Zoo zou hij dan toch,!....}{iets gedaan eindelijk iets}{voortgebracht hebben}\\

\haiku{O, wat een slap, flauw,!...}{kinderachtig ventje was}{hij vroeger geweest}\\

\haiku{Ze is d'r akelig,,,....}{vandaan gekomen doodmoe}{hijgende bezweet}\\

\haiku{Theo veegde zich, met, ' '.}{z'n zijden zakdoekjet}{zweet vant voorhoofd}\\

\haiku{Ze lette haast niet,!}{op haar huisgenooten}{ze gaf niet om hen}\\

\haiku{Overal en aldoor....}{dreigde iets waaraan ze geen}{naam dorsten geven}\\

\haiku{Het sjofele mensch,;}{keek geschrokken op gaf niet}{dadelijk antwoord}\\

\haiku{Zoo waarachtig as,....}{God leef ik durf er haas niet}{over te beginnen}\\

\haiku{haar gezicht bleef naar;}{Emma gekeerd onder het}{verdere praten}\\

\haiku{Jesis, zoo'n kreng van 'n....,....}{jongen dan toch ook neemt u}{nie kwalijk mevrouw}\\

\haiku{naar haar zakdoekje,.}{wischte zich het zweetige}{gezicht ermee af}\\

\haiku{Ja, zooals ik ook al,....}{zei het is niet plezierig}{voor u en papa}\\

\haiku{Hij gaf haar een stoel.}{en stuurde een jongen om}{meneer te halen}\\

\haiku{Daar willen ze 'm,,....}{zeker niet hebben zoo'n rooie}{zoo'n socialist}\\

\haiku{{\textquoteright} {\textquoteleft}En als u 't niet,{\textquoteright}, {\textquoteleft},....}{voelt ging Theo voortnou dan zult}{u later misschien}\\

\haiku{z'n eene arm, die er,;}{ver overheen lag hief hij een}{paar malen even op}\\

\haiku{In 't ouderlijk,,.}{huis kwam Theo v\'o\'or z'n trouwen}{een paar malen nog}\\

\haiku{hij had gezegd dat;}{hij wel begreep door wie oom}{Herman was gestuurd}\\

\haiku{eigenlijk had hij;}{van jongsaf nooit bizonder veel}{met hem opgehad}\\

\haiku{- dan winnen en z'n,....}{inzet verdubbeld zien \`of}{dien zien weggraaien}\\

\haiku{Het waren papa.}{en mama die de grootste}{verliezen leden}\\

\haiku{Vijf minuten voor.}{twaalven kon Croes zich haast niet}{langer beheerschen}\\

\subsection{Uit: Roman van een gezin. Deel 2. E\'en voor \'e\'en}

\haiku{Ze zeggen immers....}{altijd dat juist mannen die}{sterk geleefd hebben}\\

\haiku{Was het trouwens niet?}{een klaargemaakte zaak waar}{hij voorgezet was}\\

\haiku{nu maar h\'e\'el erg graag ',.}{of anders veel liever in}{t geheel niet hoor}\\

\haiku{Als het kwam zou het}{de partijen ditmaal niet}{zoo onvoorbereid}\\

\haiku{Dat wil zeggen,{\textquoteright} liet, {\textquoteleft}....}{hij er scherp-fluisterend}{op volgenhij zei}\\

\haiku{Neem een rijtuig, en....,....{\textquoteright} {\textquoteleft}....}{haal laats kijken wie zijn er}{ook alzoo bijNou}\\

\haiku{Een ieder mot toch,!}{vrij wezen in z'n doen en}{laten wat weerlach}\\

\haiku{Maar daar was geld voor,,;}{noodig natuurlijk je moest wat}{kunnen inkoopen}\\

\haiku{Jammer, ellendig, '....}{jammer dat zer zooveel}{minder uitzag al}\\

\haiku{'t sprak van zelf dat;}{ze allemaal fel tegen}{het volk zouden zijn}\\

\haiku{Hij zou er niet meer,....}{in terug willen in die}{wereld van vroeger}\\

\haiku{Die toppen vooral,,,! ...}{die zijn zoo ruw en hard en}{heelemaal grauw kijk}\\

\haiku{dat hij dit over een, '....}{korten tijd n\'og eens zien zou}{en dan voort laatst}\\

\haiku{{\textquoteleft}Kom,{\textquoteright} zei hij, {\textquoteleft}ik moet,,? ....}{weer eens opstappen ga je}{misschien mee zoover Ru}\\

\haiku{Ze hadden elkaar.}{den heelen avond nog zoo goed}{niet begrepen}\\

\haiku{{\textquoteleft}Welnee, welnee! - och,!}{kom het loopt ommers met een}{sisser af misschien}\\

\haiku{Jelie blijft rustig,.}{thuis we zouden mekaar maar}{in de weg loopen}\\

\haiku{- ze scheelden dan ook...}{maar twee jaar en daar Jan de}{scherpzinnigste was}\\

\haiku{het was beroerd, ze,,?}{gaven het toe volmondig}{maar wat wou hij dan}\\

\haiku{Met bitterheid en.}{opwinding bleef Croes praten}{en zich verzetten}\\

\haiku{Hij had geen hoop meer.}{op een verbetering in}{z'n financi\"en}\\

\haiku{hij erkende dat '.}{hijt er naar gemaakt had}{in den laatsten tijd}\\

\haiku{Hij zou 't morgen,.... '}{maar in-eens zeggen dat}{hij er geweest was}\\

\haiku{Toch nog aarzelig,,,....}{langzaam wrikte Croes aan den}{deurknop duwde open}\\

\haiku{Hij sprak er over met,;}{Jeanne want die had de}{huishoudkas nu toch}\\

\haiku{Noortje kostte nu ',.}{weln boel geld maar ze ging}{dan ook het huis uit}\\

\haiku{Hij zal maar ergens,....}{in de leer moeten bij een}{timmerman of zoo}\\

\haiku{Ik heb heelemaal,...}{zoo'n verbazend duur jaar ik}{weet waarachtig niet}\\

\haiku{{\textquoteright} Maar even later, zich,:}{vermannend en een stap naar}{den knaap toetredend}\\

\haiku{{\textquoteleft}Henk, kereltje, win,,,....{\textquoteright} {\textquoteleft}?}{je nou niet zoo op toe hou}{je nou kalm ikKalm}\\

\haiku{En dan, in-eens,,.}{keerde hij zich om en liep}{weg de kamer uit}\\

\haiku{hij gaf dan bijna, '.}{nooit antwoord maar zuchtte als}{vann diep verdriet}\\

\haiku{Het A-examen had;}{ze een paar jaar geleden}{met sukses gedaan}\\

\haiku{Ook Jeanne, bij,.}{een bezoek aan haar drong er}{hartelijk op aan}\\

\haiku{'k Geloof dat jij ',{\textquoteright}.}{t nog veel meer noodig heb dan}{ik riep Gonne uit}\\

\haiku{En dat jij, die zoo'n '!....}{model vann echtgenoote}{en moeder zou zijn}\\

\haiku{Soms, \'even, gluurde zijn....}{blik van de bladzij weg en}{naar Gonne's gezicht}\\

\haiku{Wanneer de meisjes '.}{elkaar aanzagen proestten}{zet bijna uit}\\

\haiku{Anders kan die zoo.... '}{gezellig zitten praten}{en gekheid maken}\\

\haiku{{\textquoteleft}Laat ik u toch niet,,.}{storen meneer Driebeek blijft}{u rustig liggen}\\

\haiku{{\textquoteright} riep hij uit, met 'n, {\textquoteleft},,!}{armzwaainee u begrijpt me}{niet ik merk het wel}\\

\haiku{ze spraken elkaar.}{nu immers ook alle drie}{bij den voornaam aan}\\

\haiku{Je kunt met hem niet,?}{anders dan vertrouwelijk}{praten vin-je wel}\\

\haiku{Voor Gonne iets te,.}{zijn haar beter te helpen}{maken was het doel}\\

\haiku{{\textquoteright} {\textquoteleft}Heb-je dan geen....}{innerlijke behoefte}{daaraan te gelooven}\\

\haiku{afschuwelijk, vond,....}{mama zoo'n ontstemming nu}{vlak v\'o\'or de bruiloft}\\

\haiku{Wat kan 't me ook,, '!}{allemaal schelen dacht ze}{k word toch de bruid}\\

\haiku{{\textquoteright} trachtte Noortje te, {\textquoteleft}',.}{lachenna slechte inkt of}{zoo iets geloof ik}\\

\haiku{je kreeg er meelij,.}{mee zooals ze zich inspannen}{moest om mee te doen}\\

\haiku{De bruid alleen kreeg....}{een kleur en voelde tranen}{in haar oogen prikken}\\

\haiku{{\textquoteright} En ook de bode,,.}{haastig toegeschoten boog}{zich over de tafel}\\

\haiku{Nu kon ze verder,,....}{niets meer dan wachten wachten}{tot het uit zou zijn}\\

\haiku{Iedere man van....}{dertig jaar heeft wel iets in}{z'n verleden dat}\\

\haiku{'t werd tijd nu, tijd;}{om naar boven te gaan en}{zich te verkleeden}\\

\haiku{{\textquoteright} En beneden, in,,.}{de zaal tegen Croes hem er}{niet bij aankijkend}\\

\haiku{Het zal een heele....}{deun voor haar zijn om er weer}{van op te komen}\\

\haiku{{\textquoteright} zei Jan, en kuste ', {\textquoteleft}....{\textquoteright} {\textquoteleft}....}{haar zacht opt voorhoofdkom}{wees nu maar kalmJa}\\

\haiku{de tafel stond er -.}{gedekt toen ze een voor een}{beneden kwamen}\\

\haiku{Ze beseften nu, '....}{allen dat hett ergste}{was voor mama zelf}\\

\haiku{ze w\'ou er ook nooit,....}{over praten ze hield niet van}{zulke gesprekken}\\

\haiku{maar dat kon niet, want,.}{het was heel stil buiten had}{ze daar straks gemerkt}\\

\haiku{De jongen huilde,.}{blijkbaar in zijn bed kreunend}{met diepe snikken}\\

\haiku{Och ja,{\textquoteright} riep hij uit, {\textquoteleft}....}{ik weet ook zelf niet waarom}{ik zoo huilen moet}\\

\haiku{'t Was natuurlijk,,!}{waar wat iedereen hem zei}{hij hield heel wat over}\\

\haiku{Hij was niet onnoozel,....}{of krankzinnig hij wist wat}{er van komen moest}\\

\haiku{Trouwens, hij wilde,:}{het zelf eerst haast niet gelooven}{maar het was toch zoo}\\

\haiku{de woorden klankten,.}{dan in haar hoofd maar kwamen}{niet over haar lippen}\\

\haiku{Jeanne vond het, ';}{ellendigt zichzelf te}{moeten bekennen}\\

\haiku{Moet je nog al mee,....}{aankomen tegenwoordig}{zei hij in zichzelf}\\

\haiku{Anna, onthuis en,;}{verlegen bewoog zich nog}{plomper dan anders}\\

\haiku{{\textquoteright} herhaalde hij, en.}{krampte zijn fijne handen}{tot vuisten samen}\\

\haiku{Stil was ze altijd,,;}{ook aan tafel nu men was}{dat van haar gewoon}\\

\haiku{Wat m\'e\'er welvaart, wat, ...}{voorspoediger zaken ja}{dat was het eenige}\\

\haiku{{\textquoteright} De jongen praatte,.}{nog door maar Theo luisterde}{er al niet meer naar}\\

\haiku{Vlug schepte ze haar '.}{oudste de koolraap opt}{wit-steenen bordje}\\

\haiku{Het ventje keek nu {\textquoteleft}{\textquoteright}.}{en dan wat schuw en schichtig}{naar z'nmeneer op}\\

\haiku{Tegen iemand moest!}{hij zoo nu en dan toch eens}{wat kunnen luchten}\\

\haiku{Nou nacht Anna,{\textquoteright} riep, {\textquoteleft}, ',!}{ze gedempttot ziens ik kom}{wel weeris gauw hoor}\\

\haiku{O nee, het is maar....}{flikkering in de ruiten}{van die kleerenwinkel}\\

\haiku{{\textquoteleft}Denk je soms dat ik,! ...}{niet weet wat echte fijne}{galanterie is}\\

\haiku{Jij ook, Noortje,{\textquoteright} liet, {\textquoteleft},!}{hij er zachter op volgen}{z\'o\'o te kijven foei}\\

\haiku{Maar zij week schichtig,,.}{opzij haar stoel meetrekkend}{greep haar vaders arm}\\

\haiku{{\textquoteright} antwoordde hij dan,,.}{triestig-gedempt en hij}{zuchtte geluidloos}\\

\haiku{Jou ongelukkig!}{te weten en er niets aan}{te  kunnen doen}\\

\haiku{{\textquoteright} De beweging in, {\textquoteleft}!}{de zaal nam toe telkens was}{er rumoer ensst}\\

\haiku{Zij kon het nu niet,.}{meer enkel uitspattende}{woorden begreep ze}\\

\haiku{Hij had verwacht dat.}{De Bries pittiger kost zou}{gegeven hebben}\\

\haiku{Goed, goed, hij zou dat.}{laten rusten en bij de}{principes blijven}\\

\haiku{Hij begon met een.}{kalmte die na Moks gekrijsch}{bijna matheid scheen}\\

\haiku{Hoe moest het toch gaan,.}{daar in het winkeltje van}{feestartikelen}\\

\haiku{Maar kom, k\'om dan toch,! ....}{niet aan het allerergste}{denken in-eens}\\

\haiku{Elken dag kwam hij, '.}{thanss morgens tusschen elf}{en twaalf gewoonlijk}\\

\haiku{Hij draaide zich af,....}{deed alsof hij iets zocht op}{het nachttafeltje}\\

\haiku{Niemand dorst vragen, '.}{maar ze bleven Herman in}{t gelaat turen}\\

\haiku{alleen de wielen,....}{knarsten het rijtuig piepte}{of kraakte soms}\\

\section{Maurice Roelants}

\subsection{Uit: Komen en gaan}

\haiku{En zij mijn vrouw, waar?}{is er haar hart op dezen}{oogenblik aan toe}\\

\haiku{Een aftroeving vond.}{hij doelmatiger dan een}{proces-verbaal}\\

\haiku{Hij beminde mij,:}{en sloeg mij op den schouder}{terwijl hij zeide}\\

\haiku{{\textquoteright} Zij sprak eenvoudig,,.}{zonder coquetterie die}{op tegenspraak aast}\\

\haiku{Ik gooide haar een.}{bal toe en wilde haar in}{het spel betrekken}\\

\haiku{Er knalden op de.}{lucht eenige rappe schoten}{van nabije jagers}\\

\haiku{{\textquoteright} - {\textquoteleft}Zijt gij zeker dat?}{al zijn inspanning te uwer}{intentie niet was}\\

\haiku{{\textquoteright} - {\textquoteleft}Wij wel,{\textquoteright} zeide ik,.}{snel met een gedachte aan}{mijn vriend Berrewats}\\

\haiku{De dag was te warm.}{geweest voor het laat seizoen}{en het raam stond open}\\

\haiku{Haar was het die zij,.}{volgde terwijl mijn woorden}{geen weergalm wekten}\\

\haiku{- {\textquoteleft}Herinnert gij u?}{dan niet meer wat gij mij}{vroeger hebt gezegd}\\

\haiku{{\textquoteright} zei Jaak krachtig, trots.}{het eerste lallen van een}{verlammende lip}\\

\haiku{Ik keerde huiswaarts.}{met een leedwezen dat mij}{bijna zuiverde}\\

\haiku{Wij gebruikten het:}{avondeten in een zwijgzamer}{stemming dan destijds}\\

\haiku{Doch Emma nam geen:}{notitie van mijn grapje}{en zeide effen}\\

\haiku{Ik heb verleden.}{jaar zes fruitboomen in den}{tuin laten planten}\\

\haiku{Ik verzeker u,.}{dat gij van mij niets meer zult}{te vreezen hebben}\\

\haiku{Tenslotte zag ik.}{slechts uitkomst in de hulp van}{den onderpastoor}\\

\haiku{En dan, gij weet het,.}{zij heeft totnogtoe geen voet}{in de kerk gezet}\\

\haiku{{\textquoteright} Trots mezelf kwam in,.}{mijn geest een argwaan dien ik}{met afkeer verdrong}\\

\haiku{{\textquoteright} Hij was rechtgestaan}{en boven zijn soutane}{was zijn gelaat}\\

\haiku{Niettemin lei ik,.}{mijn hand in de zijne die}{droog en gloeiend was}\\

\haiku{waarom ik wreed moest,.}{zijn tegenover iemand wie}{ik gansch was onthecht}\\

\haiku{- {\textquoteleft}Ik schrijf nog vanavond,.}{aan mijn vader dat ik bij}{mijn man terugkeer}\\

\haiku{- {\textquoteleft}Toch vroeger dan wij,{\textquoteright}.}{hadden verwacht wedervoer}{ik omzichtiger}\\

\section{A. Roland Holst}

\subsection{Uit: Verzamelde werken. Deel 3. Verzameld proza. Deel 1. Deirdre en de zonen van Usnach. Het Elysisch verlangen. Tusschen vuur en maan. De afspraak. Voorteekens}

\haiku{Daden van toorn en,,.}{verraad zullen om u zijn}{o vlam van schoonheid}\\

\haiku{Zij gaven haar in,.}{de hoede van een vrouw die}{de voorspelling wist}\\

\haiku{Alleen de wind zong,,.}{er het oude vreemde lied}{dat geen woorden heeft}\\

\haiku{Lavarcham en,.}{de oude man liepen langs}{den muur geruischloos}\\

\haiku{Maar zijn hoofd neeg, zijn,.}{handen grepen in zijn borst}{en hij wankelde}\\

\haiku{{\textquoteleft}Zij, die daar komen,.}{zullen de trouwe wachters}{zijn van ons leven}\\

\haiku{Zij negen het hoofd,.}{voor de eenzame oude}{vrouw die voor hen stond}\\

\haiku{Wij weten alleen,.}{dat er veel verloren ging}{sinds zij verdwenen}\\

\haiku{Toen strekte Noisa,.}{zijn handen naar haar uit en}{wilde tot haar gaan}\\

\haiku{Maar zij weerde hem,,.}{en ging snel heen wankelend}{brekend in snikken}\\

\haiku{Zij had begrepen,;}{dat Noisa nog een ander}{leven beminde}\\

\haiku{Zij nam haar hand weg,.}{uit die van Noisa en ging}{een schrede terug}\\

\haiku{{\textquoteright} Het weinige, dat,.}{er dien morgen nog te doen}{was werd snel gedaan}\\

\haiku{En Deirdre voelde,.}{zich loopen en zij hoorde}{de stem van Fergus}\\

\haiku{Nergens meer de oogen -.}{van mijn geliefde zijn stem}{niet meer te hooren}\\

\haiku{Toen zij onder de.}{muren voorbijreden zag}{men hun gelaten}\\

\haiku{En haar brekende,:}{stem sidderde van smart en}{liefde toen zij loog}\\

\haiku{O, hoe schoon was zij,!}{in de goede jaren toen}{ik over haar waakte}\\

\haiku{Dwars door die zaal schreed,,.}{hij recht naar de harp die niet}{ophield te spelen}\\

\haiku{En rustig klonk zijn:}{stem van nabij en zonder}{dat het hoofd bewoog}\\

\haiku{In haar oogen heb ik,.}{een licht gezien dat niet meer}{is van zon of maan}\\

\haiku{Aan den voet van zijn.}{harp hadden luisterend hun}{harten gelegen}\\

\haiku{Toen hief zij het hoofd.}{en zag om naar de woeste}{honger van het vuur}\\

\haiku{Zij liep het smalle,.}{strand ten einde tot aan de}{grens van dit geweld}\\

\haiku{Zoo vaak hij om zag,,.}{was het steeds weer achter hem}{dat de muziek bleef}\\

\haiku{De bronzen voeten;}{glinsteren in de wijde}{stilte van den tijd}\\

\haiku{In sterke, zuivre;}{kleuren staat de grazige}{wei van paarden vol}\\

\haiku{Over dat leven breekt;}{de lach als de lichte zee}{op een gouden strand}\\

\haiku{de moeder zal de,.}{maagd de zoon de Koning der}{ontelbren zijn2}\\

\haiku{maar ik zie van mijn.}{wagen uit een bloemig veld}{waar hij op rijdt}\\

\haiku{Bran ziet schittert het;}{spel van de paarden der zee}{in den zomergloed}\\

\haiku{Wij zijn hier van den ';}{aanvang vrij van jaren en}{de dwang vant graf}\\

\haiku{ons vond de zonde,.}{niet en geen wien ooit de kracht}{der jeugd begaf}\\

\haiku{door dit bedrog werd.}{ouderdom en ondergang}{der ziel uw lot}\\

\haiku{van zijn stam zal een.}{korten tijd een schoon wit mensch}{op aarde gaan}\\

\haiku{geheimen zal hij,.}{aan den mensch onthullen blij}{en zonder schroom}\\

\haiku{hij zal een draak zijn,.}{in den strijd hij zal een wolf}{zijn in het woud}\\

\haiku{Doch hij wilde niet,,.}{spreken maar keek slechts naar hen}{lachend met open mond}\\

\haiku{Zoo stil werd toen zijn,.}{liefde dat hij open ging tot}{een ander leven}\\

\haiku{Daarbuiten, op den,.}{omgang hadden zij zich toen}{aan het spel gezet}\\

\haiku{Niet langer staarde;}{de vreemde tegenstander}{weg over den omtrek}\\

\haiku{Met het aanbreken:}{van den derden dag begon}{het weer om te gaan}\\

\haiku{Maar een hand werd op.}{mijn schouder gelegd en gij}{wendde uw hoofd af}\\

\haiku{Snel ontkleedde ik,,.}{mij blies de kaarsen uit en}{legde mij in bed}\\

\haiku{Zacht sloot gij de deur,,,.}{weer en gij liept langs mijn bed}{dwars door de kamer}\\

\haiku{, en hurkend bij den.}{kleinen haard hieldt gij een vlam}{onder het rooster}\\

\haiku{zulke oogen als den;}{mensch bevreemden en het kind}{naar zich toe trekken}\\

\haiku{Op dat oogenblik,}{nam ik mij alleen stellig}{voor later te doen}\\

\haiku{Eindelijk, toen het,.}{zomer was geworden kwam}{ik waar de zee is}\\

\haiku{Achter den dunnen,,.}{wand waartegen ik lag liep}{iemand op de gang}\\

\section{Richard Roland Holst}

\subsection{Uit: Overpeinzingen van een bramenzoeker}

\haiku{{\textquoteleft}En de derde{\textquoteright}, vroeg, {\textquoteleft}?}{mijn jeugdige vriendhij die}{nu gestorven is}\\

\haiku{De muziek maakte.}{mij wrevelig droefgeestig}{en toch gelukkig}\\

\haiku{hij schreef, terug zou,.}{gaan naar het huis vanwaar hij}{eens de reis begon}\\

\haiku{Hij weifelt, hij talmt,,.....}{hij blijft staan iederen dag}{opnieuw die strijd}\\

\section{Anton Roothaert}

\subsection{Uit: De vlam in de pan}

\haiku{Alleen aan zijn broek,.}{was te zien dat hij niet als}{meisje bedoeld was}\\

\haiku{Dr. Hiemstra had verder.}{den storm onbewogen over}{zijn hoofd laten gaan}\\

\haiku{Ruim tien jaar zit het.}{wagentje muurvast en er}{m\'o\'et iets gebeuren}\\

\haiku{Die sergeant was,.}{er een die nooit iets deed op}{zijn eigen houtje}\\

\haiku{Hij glimlacht, nu hij,.}{hoort dat het resultaat niet}{op zich laat wachten}\\

\haiku{Dit bataljon heeft.}{een goeden naam en hier is}{de kantine}\\

\haiku{{\textquoteleft}Denk maar niet, dat de{\textquoteright},.}{eerste klap een daalder waard}{is roept Koen hem na}\\

\haiku{Dat geeft natuurlijk.}{een onpleizierig gevoel}{van onzekerheid}\\

\haiku{Vanavond suist er een}{kwade noordooster en niet}{ver van den laatsten}\\

\haiku{Dit is het huis van,:}{mijnheer Jansen-van der}{Aa dat wil zeggen}\\

\haiku{Telkens moet hij van.}{het pad af om fietsers te}{laten passeren}\\

\haiku{Maar wat gaan we doen?}{met deze ferstoteling}{van edelen bloede}\\

\haiku{In de rustkamer,.}{heerst een ongewoon rumoer}{dat schielijk verstomt}\\

\haiku{Michel zoemde zeer {\textquotedblleft} -{\textquotedblright}:}{belangstellendM m en}{bot er boven op}\\

\haiku{{\textquoteright} {\textquoteleft}Dat klinkt inderdaad,.}{vlot maar zo had hij nog steeds}{geen platte wagens}\\

\haiku{{\textquoteright} {\textquoteleft}Ja, ik verwacht met.}{spanning het optreden van}{Michel Lanslot}\\

\haiku{Hij had al gauw een!}{groten troep om zich heen en}{lieve  hemel}\\

\haiku{Met die pis-praetjes,!}{heb ik niks te maken ad}{fundum govvedomme}\\

\haiku{{\textquoteright} {\textquoteleft}Een onderwijzer.}{komt altijd in zijn eigen}{winkeltje terecht}\\

\haiku{Het lijkt veel op ons,.}{eigen kwaaltje het is dom}{en onvoordelig}\\

\haiku{En als je zoiets,.}{ziet lijkt ons vraagstuk opeens}{veel eenvoudiger}\\

\haiku{Al twee nachten heb.}{ik \`a\`akelig gedroomd van}{lepels en vorken}\\

\haiku{Om vier uur is het -!}{timmerwerk al klaar en niet}{verder vertellen}\\

\haiku{Zij zetten den rug,.}{hol en lopen met losse}{lijven als dansers}\\

\haiku{Buitel maar met je,.}{speelgoed-vlindertje}{in de zon jongen}\\

\haiku{Als hij nu een klas,.}{van die jongens bezig ziet}{vreet hij zijn hart op}\\

\haiku{er wordt geen papier.}{vuilgemaakt en het fluimen}{is afgelopen}\\

\haiku{En waarvoor hadden?}{ze den sergeant Slotboom}{uit zijn werk gehaald}\\

\haiku{Pom-pom-pom,...}{wie heeft de suiker in de}{erwtensoep gedaan}\\

\haiku{Nu zijn baard in de,.}{waskom drijft schijnt het leven}{minder hopeloos}\\

\haiku{Waarschijnlijk zal het,:}{grootste deel B-kleding}{zijn dat wil zeggen}\\

\haiku{We waren op veel,.}{voorbereid maar zo'n uitschot}{had niemand verwacht}\\

\haiku{Hij had de jongens.}{aangespoord om  trots te}{zijn op hun uniform}\\

\haiku{Maar als hij zover,:}{is laat Ras Wenkiboe den}{instructeur vragen}\\

\haiku{In ieder geval,.}{juich ik het toe dat u niet}{langs den kant blijft staan}\\

\haiku{Schurfie zit hem op.}{enkele schreden afstand}{trouw aan te kijken}\\

\haiku{Maar op dit ogenblik.}{heeft zij een grief tegen al}{wat militair is}\\

\haiku{Net als die kale.}{luis van een ambtenaartje}{uit de Beeldenstraat}\\

\haiku{{\textquoteright} {\textquoteleft}Dat is de leider,{\textquoteright}.}{van een kring zegt Koen en buigt}{zich weer over zijn werk}\\

\haiku{Maar hun dienst zouden -, -!}{ze god hier en daar bons op}{de tafel goed doen}\\

\haiku{een huishouden met...}{niks als ruzie was voor haar}{ook zo lollig niet}\\

\haiku{En hiermee was haar.}{positie in den huize}{aanzienlijk versterkt}\\

\haiku{De compagnie gaat.}{banken sjouwen en Quinten}{heeft veel schrijverij}\\

\haiku{je vel tussen haar -.}{nagels dat je haast bleef in}{je eerste schreeuwstuip}\\

\haiku{De geschiedenis.}{van het onderwijs in het}{kwartier van Baarschot}\\

\haiku{ze konden gezien,.}{worden hoefden maar \'e\'en keer}{binnen te komen}\\

\haiku{Toch schijnt mijnheer Haak,.}{te voelen dat er niet de}{juiste stemming heerst}\\

\haiku{Toch is het verder,,.}{dan zij dachten want de}{knal laat zich wachten}\\

\haiku{O Heer, geef mij lust,...}{tot werken maar wil dien lust}{enigszins beperken}\\

\haiku{De zon blinkt op zijn,.}{groot knipmes waarmee hij den}{zesponder aansnijdt}\\

\haiku{Plons, plons, daar gaan er,.}{al twee maar zijn bliksemsnel}{weer op het droge}\\

\haiku{Oei-oei, nu gaat het,.}{verkeerd daar halverwege}{den linkervleugel}\\

\haiku{Ginds heeft Schipper een.}{walletje van een halven}{meter gevonden}\\

\haiku{Zo ver hij kan zien,,...}{zijn er geen achterblijvers}{dus geen verliezen}\\

\haiku{Brinkman, houd den troep,.}{op zijn plaats anders wordt het}{een beestenbende}\\

\haiku{Ja, dat is pech, vindt.}{Beumke en hij trekt een}{meewarig gezicht}\\

\haiku{Toch staat op de borst.}{van dezen ruwen kiel een}{adelaar geborduurd}\\

\haiku{Ja, ze zullen op '.}{t ogenblik wel alle drie}{bij den dokter zijn}\\

\haiku{O, de geest onder,...}{de jongens is uitstekend}{buitengewoon zelfs}\\

\haiku{We waren verspreid.}{opgemarcheerd en kregen}{een beestachtig vuur}\\

\haiku{Toen zag hij, dat links.}{voor ons uit de troep begon}{samen te klitten}\\

\haiku{{\textquoteright} Zij rijden door het.}{dorp en bij de oude kerk}{laten zij stoppen}\\

\haiku{Daar stond de tweede.}{met een machine-pistool}{onder den oksel}\\

\haiku{Toen kreeg de majoor.}{een idee en liet Hollandse}{signalen blazen}\\

\haiku{Gauw een patrouille.}{erheen en het resultaat}{melden aan Stuurman}\\

\haiku{Een zenuwlijder.}{zou ons hier vannacht rustig}{laten verkleumen}\\

\haiku{Drost zet voorzichtig.}{zijn bajonet op en gaat}{vooruit door de gang}\\

\haiku{weg beveelt hij luid.}{praten en dit hoeft hij geen}{tweemaal te zeggen}\\

\haiku{Die kopraal van ons vloekt,{\textquoteright}.}{een mens het geweer uit de}{vingers zegt Klinker}\\

\haiku{Op bliksemoorlog.}{is een recrutenschool dan}{ook niet ingericht}\\

\haiku{V\'o\'or je het weet, heb.}{je tijdens een aanval den}{vijand in den rug}\\

\haiku{Hij ziet duidelijk.}{de verbeten razernij}{op hun gezichten}\\

\haiku{Want Klinker ligt half.}{bedolven onder het zand}{en hij is zeer dood}\\

\haiku{Zo is er ook geen,.}{woord gerept over Karel Koen}{zelfs niet door Michel}\\

\haiku{Zij zien een mijnheer,.}{in officiersuitrusting}{gemaskerd en wel}\\

\haiku{{\textquoteright} Deze mensen zijn '.}{van lichting23 en toen was}{Vlot nog sergeant}\\

\haiku{En omdat meneer,.}{niet te overtuigen is gaat}{hij mee naar dat huis}\\

\haiku{Gelukkig hebben...}{onze wakkere mannen}{het bijtijds ontdekt}\\

\haiku{Uit een der laagste.}{hoeken drupt langzaam het bloed}{op den trottoirband}\\

\haiku{Van de bevelen,.}{begrijpt hij wel niet veel maar}{het zijn bevelen}\\

\haiku{De groeten aan den?}{reserve-kapitein}{Beumke of zo}\\

\haiku{Hij heeft het zoete,.}{gevoel dat hij hierdoor wraak}{neemt op het serpent}\\

\haiku{Wegkruipen was dus...}{het volledig bewijs van}{een slecht geweten}\\

\haiku{Het is niet nieuw, geen,.}{eigen vinding maar voor hem}{toch een ontdekking}\\

\haiku{Hij blijft boven op.}{het walletje staan en kijkt}{lachend op hen neer}\\

\haiku{Dat wil dus zeggen,?}{dat de brug eerst in handen}{van den vijand was}\\

\haiku{Neen, Beumke wist,....}{er niets van komt recht uit den}{polder Zo-zo}\\

\haiku{{\textquoteright}, gilt er een achter.}{uit den hoop en de stem slaat}{over van opwinding}\\

\haiku{En toen ik naar dat,!...}{vliegtuig tippelde zong ik}{zelfs hardop van boum}\\

\haiku{Nocker zelf spreekt van.}{een doodgeboren kindje}{met een lam handje}\\

\haiku{Dit is nog het fraaie,.}{handschrift van Karel's vader}{het lijkt wel steendruk}\\

\haiku{Hij keert haar koppig,.}{den rug toe en nu eerst ziet}{zij hoe bleek hij is}\\

\section{Felix Rutten}

\subsection{Uit: Daags veur Krismes}

\haiku{Camillo nuimde, -!}{mich eine pries en dat waar}{geine kattendrek}\\

\haiku{Wie  angesj is:}{dat noe neit bie Camillo}{et geval gewaes}\\

\subsection{Uit: Novellen}

\haiku{En auch b\`en ich, zo,.}{lankzaam aan al einen daag}{awwer gewoorde}\\

\haiku{{\textquoteright} De Prior dach nao'.}{en keek den ermen hals e}{tiedje sjwiegend aan}\\

\haiku{En de sjtein vlogen,.}{en vlogen of et b\`ontje}{veugelkes ware}\\

\haiku{{\textquoteright} Einen ougeblik.}{sjt\`ong Broeder Balderik gans}{verbawwereird}\\

\haiku{zo auch nog ummer.}{de eesjte fiool}{in h\"o\"or awd gedouns}\\

\haiku{En zo haw ich dus.}{de freule Margarethe}{lere k\`enne}\\

\haiku{H\`ei\"e ze same'?}{neit \`ens e feeske op touw}{k\`onne z\`ette}\\

\haiku{Fruit lachde einen.}{aan van kompotiees van}{zilver en kristal}\\

\haiku{M\`onter g\`ong ze mit '.}{de m\`onter bende de waeg}{op ent landj in}\\

\haiku{Auch wirkde ze 't,.}{j\`onk v\"olkske gaer in de}{handj wie ze mer koos}\\

\haiku{{\textquoteright} Taenge den aovend ' '.}{veil de deur vant hoes veur}{gouwd achterm toe}\\

\haiku{En al gaw ware,,.}{veer twee kameraote}{dr\"ok aan et moele}\\

\haiku{Camillo nuimde,.}{mich eine pries en dat waar}{geine kattendrek}\\

\haiku{En \`om noe euver,:}{et dood punt haer te k\`omme}{vraogde ich h\"o\"om}\\

\haiku{Krism\`es is dan toch,,?}{et fees van \`os herders en}{van alle veevolk}\\

\haiku{Wie angesj is dat:}{aevel neit bie Camillo}{et geval gewaes}\\

\haiku{{\textquoteleft}Dokter, zou et neit '?}{gouwd zeen es ich mesjien toch nog}{mern waek hie bleef}\\

\haiku{Hae hei dat dan toch,.}{k\`onne doon hae hei dat mesjins}{zelfs waal m\`otte doon}\\

\haiku{Bie j\`ong vrouwluuj kan.}{me dat jao nooit zo presies}{te weite k\`omme}\\

\haiku{Mer et weur eine,,;}{sjoe\"e eine lelike kael}{mit al zien marke}\\

\haiku{Moritz h\`ei mit gein, ' '....}{weurd k\`onne z\`egge wiem}{dat aant hart g\`ong}\\

\haiku{{\textquoteright} {\textquoteleft}Geer woont zo sjoon, haet,{\textquoteright}.}{Albaer mich gezag trachde}{Moritz aaf te lei\"e}\\

\haiku{Nog get korter bie,.}{en ich vuilde zien wang langs}{mien wang sjtrieke}\\

\haiku{Me gouf dat neit sjus,.}{oet mitlie\"e mer \`om de kael}{kwiet te waerde}\\

\haiku{en zo h\`olp 'r hun.}{auch nog mit et hout dat ze}{neudig hawwe}\\

\haiku{Et liek droug ein w\`onj,.}{aan de veurkop wo et k\`endj}{aan doodgeblouwd waar}\\

\haiku{dat zal mich egaal zeen{\textquoteright},, {\textquoteleft} '.}{zag de Sjoltheises veerm}{mer kwiet waerde}\\

\haiku{Doe, Reinder, wits de,{\textquoteright}.}{waeg jao nog waal daohaer de}{kamer baoven-in}\\

\haiku{En zo g\`onge veer.}{op bezuik in den tore}{van et guitje}\\

\haiku{Dan b\`en ich waal zo.}{blie dat ich daen aovend neit}{in sjlaop kan k\`omme}\\

\haiku{{\textquoteleft}K\`enjer leif, gouwd.}{aeten en dr\`enke hiltj}{lief en zeel bie-ein}\\

\haiku{t Is et erfgouwd,}{van mien k\`enjer waat ich}{in zien h\`enj l\`eg}\\

\haiku{{\textquoteright} {\textquoteleft}Vader, en ich maak '.}{mich sjterk dat ichm \`onger}{den doem zal hawte}\\

\haiku{Noe zou {\textquoteleft}nonk Sevrien{\textquoteright}, -.}{al gaw onneudig en mesjins}{waal te v\"o\"ol zeen auch}\\

\haiku{Den tied waar gek\`omme,'.}{dat er zich moos \`omzeen nao}{e nuuj besjtaon}\\

\haiku{Hubertien g\`ong \`ens,:}{onger de h\`egge t\"osje de}{netele kieke}\\

\haiku{Die v\`enj ich get,.}{bleik en die zeen mich auch get}{te kl\`ommelechtig}\\

\haiku{Et doerde noe nog,.}{weier get en dao koum ein}{kar aangeroddeld}\\

\haiku{Tilke haw de maad}{van de pastoor gevraog}{ze oet te k\`omme}\\

\haiku{Noe haw ich aevel.}{in zien pastorie neit mit}{h\"o\"om allein te doon}\\

\haiku{Ich h\"ob mich waal \`ens,.}{aafgevraog of ze ooit}{loon getrokken haet}\\

\haiku{ze deig alles om,,}{mager te waerde zag}{ze mer bleef zo get}\\

\haiku{{\textquoteright} {\textquoteleft}N\`onk, wie h\"obs doe dan,.... ?}{auch zo d\`om konne zeen zo}{gans zonger bez\`ei}\\

\haiku{veer z\`egge das doe;}{mit eine m\"orgestrein}{bievandan weggeis}\\

\haiku{Waat veer dao gekaok,.}{en gebakke h\"obbe kan}{ich neit mee z\`egge}\\

\haiku{{\textquoteright} - dan waar ze weier '.}{gans Calypso en veilm}{in sjt\`orm in de erm}\\

\haiku{Mer hae maagde dat., {\textquoteleft}{\textquoteright},.}{In wirkelikheidau fond}{lachde hae mit h\"o\"or}\\

\haiku{{\textquoteright} {\textquoteleft}Wae h\`ei dan auch zo{\textquoteright},.}{get k\`onne d\`enke zag er}{sjtil t\"osje zien tenj}\\

\haiku{Den eine zag dat;}{hae d'roet wol en zich ziene}{waeg waal zou zuike}\\

\haiku{{\textquoteright} {\textquoteleft}Ich kan papa in{\textquoteright}.}{zienen toesjtandj neit bie}{allein laote}\\

\haiku{Koos et Schaldus dan, '?}{allein laote esof}{et neit vanm h\`olj}\\

\subsection{Uit: Onder den rook der mijn}

\haiku{{\textquoteright} Zij stapte moedig '.}{int hooge kruid met hare}{tengere beentjes}\\

\haiku{Ieder ging naar de ',.}{Vespers middags wat nu}{ook niet meer gebeurt}\\

\haiku{De blauwe oogjes,.}{pinkelden schalks maar het wicht}{verroerde zich niet}\\

\haiku{Maar Willem keerde:}{zich nog eens naar de meisjes}{om en riep woedend}\\

\haiku{Een frank per dag en,.}{zweten als een os van den}{morgen tot den avond}\\

\haiku{{\textquoteright} Willem rekte zich ',}{uit overt hek  toen ze}{heenging om te zien}\\

\haiku{{\textquoteright} {\textquoteleft}Ik ben een arme,.}{man die vijf  kinderen}{heb groot te brengen}\\

\haiku{Het was een jongen.}{die op de wijze van zijn}{lied kwam aanstappen}\\

\haiku{Zijn stem bulderde,:}{de heksen tegen die stil}{hielden om hem heen}\\

\haiku{{\textquoteright} {\textquoteleft}Dan zullen we den,,.}{rozenkrans bidden Trina}{en ter ruste gaan}\\

\haiku{Mijnwerkersvolk is,.}{gevaarlijk volk had zijn groote}{makker hem geleerd}\\

\haiku{Zij bedwelmden hem.}{als de jenever die hij}{hem leerde zwelgen}\\

\haiku{Het was alsof er,.}{een zee in ruischte die}{steeg en viel en steeg}\\

\haiku{Zij wilde geluid,.}{geven in haar angst maar kon}{niets uitbrengen}\\

\haiku{Alleen de groene.}{boomen stonden als verstomd}{in al die drukte}\\

\haiku{De vogel was er,.}{af en de schutters hadden}{een nieuwen koning}\\

\haiku{Alles vloog overeind,.}{en stormde den boomgaard uit}{naar het schietterrein}\\

\haiku{Ook was er naar de,,.}{overzij van den weg geen vrij}{schoon wiegend veld meer}\\

\haiku{Nieuwe steenovens dan,!}{en nieuwe huizenreeksen}{om ze te bergen}\\

\haiku{dat Hary Gerards.}{dood geslagen was in de}{Brunssumer heide}\\

\haiku{Zijn heel gezicht was ' '.}{opgezwollen ent haar}{stond stijf vant bloed}\\

\haiku{maar Hary voelde.}{niet dat hare stem eventjes}{onzeker trilde}\\

\haiku{die kermisavond, de:}{overval waarbij de ander}{zich gewroken had}\\

\haiku{Maar ik wilde je,.}{maar eens zeggen dat d\'at zoo}{niet verder gaan kan}\\

\haiku{Het was een gedicht,.}{en Lize bedekte den}{naam van den schrijver}\\

\haiku{Ik had me moeten.}{offeren voor zijn geluk}{en zijn ommekeer}\\

\section{Jan van Ruusbroec}

\subsection{Uit: Het sieraad der geestelijke bruiloft}

\haiku{Zij geldt tot heden.}{voor de beste en er werd}{veel zorg aan besteed}\\

\haiku{Die blind blijven, en,.}{dit gebod verzuimen die}{zijn allen verdoemd}\\

\haiku{en hieruit ontspringt,:}{het andere punt en dat}{komt voort uit de ziel}\\

\haiku{En de groote nood der,.}{menschelijke natuur en}{de eer zijns Vaders}\\

\haiku{Gods barmhartigheid,.}{en onze nood Gods mildheid}{en ons verlangen}\\

\haiku{Zachtmoedigheid maakt.}{in den mensch vrede en pais}{van alle dingen}\\

\haiku{En hij zal gesierd:}{zijn en gekleed met een kleed}{in twee\"en verdeeld}\\

\haiku{Gods is eenvoudig,;}{en niet uit te spreken door}{de creaturen}\\

\haiku{deze dragen naar.}{God alle goede werken}{en alle deugden}\\

\haiku{die hoovaardig is,,.}{is niet ootmoedig en die}{behoort niet aan God}\\

\haiku{en omdat Hij in;}{ons en met ons eeuwig wil}{wonen en blijven}\\

\haiku{Wij moeten ook God.}{loven met al waarmede}{wij dat vermogen}\\

\haiku{Hoofdstuk XXVIII'}{Van de vierde wijze}{van Christus komst NU}\\

\haiku{Zij benemen den.}{grond en de oefening van}{alle innigheid}\\

\haiku{Deze ziekte is,.}{gevaarlijker dan eenige}{van de andere}\\

\haiku{en hij weet niet van,.}{God noch van zich zelven in}{de oprechte deugd}\\

\haiku{maar zij is even wijs.}{op den eersten dag als zij}{het ooit zal worden}\\

\haiku{En zij scheidt niet van,;}{God en nimmermeer doet zij}{dit op deze wijs}\\

\haiku{En daar voor eischt,,;}{noch wil hij iets dat hij niet}{aan God wil geven}\\

\haiku{Al levende sterft.}{hij en al stervende wordt}{hij weder levend}\\

\haiku{en hij gaat in God.}{met genietende neiging}{in eeuwige rust}\\

\haiku{En zij zeggen, dat,.}{zij rusten in Dengene}{dien zij niet minnen}\\

\haiku{Dan ware zij ook,;}{niet meer heilig of zalig}{dan een steen of hout}\\

\haiku{4Het woord wallen door.}{R. gebruikt leeft nog in de}{Limburgsche volkstaal}\\

\chapter[33 auteurs, 7666 haiku's]{drieëndertig auteurs, zevenduizendzeshonderdzesenzestig haiku's}

\section{Maurits Sabbe}

\subsection{Uit: 't Pastorken van Schaerdycke}

\haiku{- Zij hebben mij ook,!}{allemaal helpers troosten}{mijn broers en zusters}\\

\haiku{Jan Potagie, die, ';}{den domme speelt maar het in}{t geheel niet is}\\

\haiku{- Ge hoort het, kind, dat,.}{moeten ze hebben en dat}{kan ik niet geven}\\

\haiku{Verlegen trok ze.}{haar tengere hand terug}{en zocht heen te gaan}\\

\haiku{Dat kind kan voor u... -... '}{geen kwaad doen En voor u kan}{dat kind geen goed doen}\\

\haiku{Hij voelde zich als,.}{een die wacht houden moest bij}{een dierbaren schat}\\

\haiku{een preekheerentaart,;}{met merg van ossen krenten}{en rozewater}\\

\haiku{Vlugger dan hij 't, '.}{verwacht had was de knop aan}{t bloeien gegaan}\\

\haiku{'t Was Jozijnken, '. -?}{die int geval een dol}{plezier had Waa-aar}\\

\haiku{Dankbaar zijn om 't,,... -?}{goede ja en niet te veel}{droomen Niet droomen}\\

\haiku{De ontgoocheling,.}{komt van de werkelijkheid}{niet van de droomen}\\

\haiku{Op knechts en meiden.}{kon ze niet rekenen om}{haar te vervangen}\\

\haiku{- Zoo is mijn broer, de,.}{jonkheer pensionaris}{juffrouw Jozijnken}\\

\haiku{Adhemar was al!}{smoorlijk verslingerd op die}{listige dante}\\

\haiku{Agla\"e greep naar de,.}{bel doch eensklaps kreeg ze een}{anderen inval}\\

\haiku{Fiete besteedde.}{haar laatsten zorgen aan den}{opschik van den tuin}\\

\haiku{En zoo kwamen ze:}{weer tot de bekommering}{van al hun dagen}\\

\haiku{- Ik ga, Fiete, ik......}{ga maar ik ben als van de}{hand Gods geslagen}\\

\haiku{het was, Jozijnken,.}{hier te zoeken zonder een}{aanwijzing of wenk}\\

\haiku{- Juist, daar gaan vele, '!}{schavuiten bij omdat ze}{ert veiligst zijn}\\

\haiku{s Pastorkens hart.}{schreide van weedom en zijn}{oogen waren vochtig}\\

\haiku{- Ja, en nu heb ik,,,?}{helaas mijn vrijheid verbeurd}{God weet voor hoe lang}\\

\haiku{De officier bleef.}{een lange poos bedremmeld}{en besluiteloos}\\

\haiku{Hij kon toch zoo maar.}{niet ineens het pastorken}{laten aanhouden}\\

\haiku{Dat ging ook niet, zij.}{was al de patronesse}{van de schrijnwerkers}\\

\haiku{Weet gij, eerwaarde,,?}{dat ons Jozijnken weg is}{spoorloos verloren}\\

\haiku{Met zijn gepeinzen.}{en gevoelens verlangde}{hij alleen te zijn}\\

\haiku{Eerst toen hij zelf wou,.}{gaan besloot de maarte hem}{te gehoorzamen}\\

\haiku{'t Was Coddebiers,.}{die hen in den donkeren}{avond kwam verrassen}\\

\haiku{- Fiete, zei toen de,!}{oude man we moeten aan}{onzen gast denken}\\

\haiku{Bij 't krieken van.}{den morgen schokkelde de}{wagen naar de stad}\\

\haiku{drong 't pastorken.}{aan om een einde aan die}{kwelling te maken}\\

\haiku{- Aan wie berouw heeft,.}{en weer goed wil leven schenkt}{Hij zijn genade}\\

\haiku{wees lijdzamig en;}{laat uw barmhartigheid zijn}{op alle vleesch}\\

\haiku{De waardinne houdt... -,;}{me gevangen Laat dit aan}{mij over Jozijnken}\\

\haiku{'t Duurde niet lang.}{of de uitvaartstemming was}{er heelemaal uit}\\

\haiku{Wat jammer, dat bij,!}{hier niet is hij zou wel goed}{meegedaan hebben}\\

\haiku{Wat zou ze hier nog,?}{te doen hebben indien hij}{haar ontnomen werd}\\

\haiku{- Weet ge, Coddebiers,?}{wat Jozijnken doen zal als}{ik er niet meer ben}\\

\haiku{Alles cirkelde,;}{om hem heen de sneeuwvelden}{en de hemelboog}\\

\section{Annie Salomons}

\subsection{Uit: Daadlooze droomen (onder ps. Ada Gerlo)}

\haiku{kan ik me jou nog.}{niet goed voorstellen tusschen}{al dat ethisch gedoe}\\

\haiku{Ze lachte, en die.}{lach streek de harde lijnen}{weg van haar gezicht}\\

\haiku{iederen dag bij ';}{t ontbijt wordt daar in den}{bijbel gelezen}\\

\haiku{Hij is goed voor me,,.}{geweest elken dag al die}{veertig jaren lang}\\

\haiku{{\textquoteright} verzwaarde ze weer, {\textquoteleft} ' {\textquotedblleft}{\textquotedblright}, ' {\textquotedblleft}{\textquotedblright}.}{dat ze welsja zegt als}{tneen moest wezen}\\

\haiku{Hij kwam dichter naast,;}{haar rijden en stak  zijn}{arm door den hare}\\

\haiku{Dat is eigenlijk,.}{de eenige plaats ter wereld}{waar ik me thuis voel}\\

\haiku{- toen ik ineens h\'a\'ar.}{zag staan in de rij met de}{andere menschen}\\

\haiku{haar mond trilde aan,;}{den eenen hoek en ze liep toch}{met de boekentasch}\\

\haiku{{\textquoteleft}Dat zullen wij nu,.}{ons heele leven blijven}{koningskinderen}\\

\haiku{Ze had tegenover;}{hem immers geen gedachten}{en meeningen meer}\\

\haiku{Nu zag je toch maar,!}{eens hoe ver je met zulke}{neuswijzigheid kwam}\\

\haiku{{\textquoteleft}Nou, maar toen ik met ',,.}{m zou gaan trouwen zei ik}{toch \'o\'ok niet waarom}\\

\haiku{Nu 't eenmaal zoover,.}{was had hij dan maar moed voor}{twee moeten hebben}\\

\haiku{en ik hoor hier, bij.}{de wijsheid en de sjawls en}{de rheumatiek}\\

\haiku{ze wist meteen, dat,.}{dat nu niet meer noodig was maar}{het roerde haar niet}\\

\haiku{knoeide zelf in een.}{laboratorium met}{allerlei proeven}\\

\haiku{En Amy was voor mij {\textquotedblleft}{\textquotedblright},...}{het meisje geweest van m'n}{jongensjaren af}\\

\haiku{ik voelde, dat ik,;}{nu wel sterven wou en ook}{wel eeuwig leven}\\

\haiku{Het was een angst voor,;}{ieder uur dat ik haar niet}{onder m'n oogen had}\\

\haiku{En toen heeft hij het,.}{opengemaakt en toen zat er}{een wollen das in}\\

\haiku{{\textquoteleft}Ik heb me toch niet,...}{op hem geabonneerd toen}{ik met hem trouwde}\\

\haiku{{\textquoteright} {\textquoteleft}Ja,{\textquoteright} zei ze vaag, en.}{de tranen raakten tot de}{randen van haar oogen}\\

\haiku{in den doodstillen.}{nacht bleven hun stappen nog}{een heel eind hoorbaar}\\

\haiku{{\textquoteleft}Laten we 't ons;}{allemaal zoo makkelijk}{mogelijk maken}\\

\haiku{Ze zal mij niet noodig;}{hebben gehad om in den}{hemel te komen}\\

\haiku{En zoo wachtend en.}{werkend was ze achter in}{de dertig geraakt}\\

\haiku{we aten lekker en,.}{volop en we zaten goed}{in onze spullen}\\

\haiku{alles wat er over,,.}{was werd aan haar besteed en}{m\'e\'er dan wat over was}\\

\haiku{en dat ze op een...,...;}{zomeravond samen en dat}{hij toen ineens zei}\\

\haiku{{\textquoteright} {\textquoteleft}O, maar dit w\`as toch,{\textquoteright}.}{heerlijk en romantisch vroeg}{de backfisch dringend}\\

\haiku{dat zie ik wel aan,}{uw oogen en die tram en het}{warme bed van straks}\\

\haiku{t voor jou zal zijn... ';}{t Zal  voor jou \'o\'ok wel}{de liefde wezen}\\

\haiku{Wat de eene mensch den,,.}{ander daarin aandoen kan}{onwaardig fnuikend}\\

\haiku{Ze wilde het zoo, ';}{duidelijk uiteen zetten}{waaromt niet m\'ocht}\\

\haiku{Een beetje verslapt...}{natuurlijk ieder mensch in}{daaglijkschen omgang}\\

\haiku{niet ziende bewoog;}{ze haar handen onzeker}{over het theegerei}\\

\haiku{De laatste jaren.}{van zijn leven bracht hij in}{het buitenland door}\\

\haiku{zonder behoefte,;}{om te bespiegelen om}{in verband te zien}\\

\haiku{naaktstudies kan je,,.}{dat noemen nou je kan er}{alles op studeeren}\\

\haiku{{\textquoteright} {\textquoteleft}Misschien ben je nog,{\textquoteright};}{te jong en te onstuimig}{zei hij voorzichtig}\\

\haiku{U begrijpt toch wel,.}{dat ik zoo iets niet in m'n}{kamer wil hebben}\\

\haiku{hij converseerde;}{met een chimpans\'e en een}{ziekelijken orang}\\

\haiku{de zon scheen er vol,.}{naar binnen en het raam was}{wijd open geschoven}\\

\haiku{{\textquoteleft}Kijk 's, ik vind het,;}{gezellig dat ik je ben}{tegengekomen}\\

\haiku{{\textquoteleft}Als jij met 'm uit, '.}{gaat zou ikm ook wel eens}{willen leeren kennen}\\

\haiku{en nu zeurde hij,.}{tien minuten over iets dat}{er niet op aan kwam}\\

\section{Anna de Savornin Lohman}

\subsection{Uit: Om de eere Gods}

\haiku{Dat oogenblik, het!}{had immers beslist over heel}{zijn jonge leven}\\

\haiku{{\textquoteleft}Ik hoop heel spoedig.}{mijn opwachting bij mevrouw}{te komen maken}\\

\haiku{- Mevrouw heeft het dus '.}{op h\'a\'ar geweten als zij}{t goede doel schaadt}\\

\haiku{- En ik vind het z\'o\'o,,.}{veel prettiger dat je nu}{met h\'a\'ar verkoopt Geert}\\

\haiku{Of misschien waren,;}{het \'o\'ok wel tranen die in}{haar oogen opwelden}\\

\haiku{Ver weg had hij het;}{socialistisch standpunt}{van zich gestooten}\\

\haiku{- - {\textquoteleft}Ja, Geert heeft al h\'e\'el!}{wat mee te beredderen}{in het huishouden}\\

\haiku{{\textquoteright} Nu eindelijk kon.}{dominee Pieter van der}{Grijp zijn hart luchten}\\

\haiku{- maar hij houdt je voor, -.}{den lap dat doet hij jullie}{meisjes allemaal}\\

\haiku{indien ze huilde}{was het omdat ze nu stond}{v\'o\'or de beslissing}\\

\haiku{{\textquoteleft}Christendom in de{\textquoteright}.}{praktijk van het leven steeds}{meer in te werken}\\

\haiku{dat zijn niet bijtijds.}{v\'o\'or de stemming thuis kunnen}{zijn aankondigde}\\

\haiku{hoe prettig ze het.}{vond nu met De Ekenhuize}{kennis te maken}\\

\haiku{van hem wilde ik, -,}{je juist verteilen dat hij}{min wordt bepaald min.}\\

\haiku{Zij deinsde vuurrood,;}{terug toen zij Govert-Jaap en}{Geerte zag zitten}\\

\haiku{{\textquoteright} {\textquoteleft}Mijn vrouw, juffrouw van,{\textquoteright}, -.}{der Grijp van de pastorie}{stelde Govert-Jaap voor}\\

\haiku{- Arme Dina liet.}{haar schaaltje bijna vallen}{van verlegenheid}\\

\haiku{- Sjoerd echter was nu,.}{al z\'o\'o ver gegaan Lizzy}{hoopte al z\'o\'o lang}\\

\haiku{Mama heeft het ook,.}{niets goed gevonden dat ik}{hem heb afgeschaft}\\

\haiku{- Ze stond op, wierp 't.}{telegram in een laadje}{van haar bureautje}\\

\haiku{de naijver der.}{leelijke op de mooie was}{daarbij te voelbaar}\\

\haiku{Maar iedereen noemt.}{hem immers zoo als er van}{hem gesproken wordt}\\

\haiku{Zeg toch aan meneer,{\textquoteright}, - -,,...}{dat we w\`achten beval Geert}{nerveus aan den knecht}\\

\haiku{het is 'n quaestie;}{van leven of dood voor dat}{Militaire Thuis}\\

\haiku{Maar ze had Sjoerd naast....}{zich aan tafel en hij was}{allerliefst vandaag}\\

\haiku{En daarbij keek hij -.}{brutaal langs haar figuur zoodat}{ze opnieuw bloosde}\\

\haiku{Hij moest nu voortaan, -.}{heel dikwijls komen net als}{vroeger met haar zijn}\\

\haiku{{\textquoteright} - - Toen, toen de deur zich,,:}{achter den knecht sloot zag hij}{zijn vrouw aan en vroeg}\\

\haiku{'t Was nu 'n veel.}{veiliger spelletje voor}{hemzelf dan vroeger}\\

\haiku{{\textquoteright} - Met hare moeder.}{had ze er een ernstige}{explicatie over}\\

\haiku{van de eerste kreeg ',.}{jet kapitaal van de}{laatsten de stemmen}\\

\haiku{Net goed{\textquoteright}, - zei ze met - {\textquoteleft} ' {\textquoteright} - - {\textquoteleft}?}{Schadenfreudeh\'a\'ar gun ik}{tWie gun je wat}\\

\haiku{weet je dan van niets, - -,!}{nog en jij die toch achter}{de schermen kunt zien}\\

\haiku{{\textquoteleft}O Verhaeghen, - voor,{\textquoteright}.}{dien ben ik niet bang haastte}{zij zich te zeggen}\\

\haiku{{\textquoteright} - - Maar haar klaagwoorden:}{wekten in hem een weerklank}{van verbittering}\\

\haiku{zij had ontvangen!}{van de oprechte vroomheid}{van Engelsch leven}\\

\haiku{- Ze verheugde zich,.}{op de mooie kleertjes die ze}{voor haar zou koopen}\\

\haiku{En dan kwam ze laat,,}{thuis met een opgewonden}{gelaat soms ook met}\\

\haiku{- - Want Govert-Jaap was o\'ok.}{in zijn huwelijksleven}{zoo gelukkig nu}\\

\haiku{Dat ze veel uit was.}{nam hij haar niet meer kwalijk}{zooals in het begin}\\

\haiku{- - U weet dat mevrouw?}{nu naar meneer en mevrouw}{Eduma de Witt is}\\

\haiku{- Je kunt er toch niet.}{den heelen avond naar zitten}{kijken zonder meer}\\

\haiku{{\textquoteright} - Eduma de Witt kneep.}{even zijn oogen dicht om het zich}{te herinneren}\\

\haiku{Govert-Jaap ontdooide.}{van het lieve gebroken}{kindergestamel}\\

\haiku{- -{\textquoteleft}D\'o\'or wien heb je?}{zoo ge{\"\i}nfluenceerd op}{de benoemingen}\\

\haiku{{\textquoteright} - - Gerty-ook zei,;}{woedend dat zij zich nergens}{meer zou vertoonen}\\

\haiku{want ik moet met hem.}{samen optreden in een}{tableau-vivant}\\

\haiku{{\textquoteleft}Tegen wie denk je,?}{dat je spreekt tegen je vrouw}{of tegen je meid}\\

\haiku{Het was of alles.}{tegen hem samenspande}{in den laatsten tijd}\\

\haiku{{\textquoteright} ... informeerde een, ':}{van de anderen metn}{hoopvol gezicht van}\\

\haiku{Hij wist het zoo goed:}{wat zijn streng geloof in dat}{opzicht leeraarde}\\

\haiku{- {\textquoteleft}Nu geeft God mij mijn,.}{\'e\'enen grooten wensch die nog}{van de aarde was}\\

\haiku{vanaf dien morgen,.}{waarop hij den anoniemen}{brief haar had getoond}\\

\haiku{- van je positie,,, - -.}{en je geld en je naam hield}{ik maar van jou niet}\\

\haiku{En dat wist je wel - -...{\textquoteright} {\textquoteleft}{\textquoteright}, -, - {\textquoteleft},.}{wist je welNeen zei hijdat}{wist ik niet Geerte}\\

\section{Nine van der Schaaf}

\subsection{Uit: Friesch dorpsleven uit een vorige tijd}

\haiku{Ik waarschuwde haar ',.}{v\'o\'ort nieuwe huis gebouwd}{werd maar zij lachte}\\

\haiku{Haar vader's vreemde.}{uitingen van de laatste}{tijd troffen haar hart}\\

\haiku{Doch zijn mond beefde.}{en hij kon zijn tranen niet}{geheel weerhouden}\\

\haiku{{\textquoteright} Natuurlijk ging hij, -?}{uit waarom zou hij op de}{Zondag thuisblijven}\\

\haiku{{\textquoteleft}Ik geloof niet dat, -}{Jelmer bij mij past en bij jou}{past hij nog minder}\\

\haiku{Hij had overnacht in,;}{de stad en was nu op de}{terugweg naar huis}\\

\haiku{vandaag en merkte.}{op dat ze te veel thuis zat}{in de laatste tijd}\\

\haiku{Doch deze keer leek ',.}{t haar moeilijker ze wist}{zelf niet recht waarom}\\

\haiku{Nauwelijks zag zij.}{hem in de donker en hij}{verdween geruischloos}\\

\haiku{Hij was opgeruimd;}{van aard en treurde niet om}{een meisje zooals Heerk}\\

\haiku{Er ontstond eenige;}{opgewondenheid toen het}{ding klaar bleek te zijn}\\

\haiku{Germen moest verteld '.}{hebben dat hij plan had heen}{te gaan uitt dorp}\\

\haiku{{\textquoteleft}Omdat ik ze niet{\textquoteright}.}{goed versta en zij lachen}{mij uit als ik praat}\\

\haiku{Heerk hoorde dat zij;}{gedeeltelijk sprak als de}{menschen in Manswerd}\\

\haiku{{\textquoteleft}Je ziet haar in lang,,{\textquoteright}.}{niet weer Heerk ze gaat morgen}{terug naar Manswerd}\\

\haiku{Hij gaf Iefke een,,}{hand wenschte haar goede}{reis zoo opgewekt}\\

\haiku{Wiebe en Jikke ', '.}{zaten opt dek daart}{nog weinig koel was}\\

\haiku{zij wenschte niet.}{in nauwer betrekking te}{komen met Leida}\\

\haiku{Toen ze zich voelden,.}{opgemerkt knikten ze en}{hervatten haar werk}\\

\haiku{Toen Iede thuiskwam ',.}{wast gesprek uit doch zij}{stond nog op het dek}\\

\haiku{HIJ kwam tegen de.}{winter thuis en terug in}{zijn oude leven}\\

\haiku{In deze tijd ving.}{hij aan in zijn landstaal}{verzen te schrijven}\\

\haiku{Enkele vrouwen.}{zag men wel in het deurgat}{van haar woning staan}\\

\haiku{Moeder is er boos, -{\textquoteright},,.}{om en ik ook zei hij half}{ernstig half in scherts}\\

\haiku{Hij hoorde de wind '.}{niet en door het raampje zag}{hijt hemelsblauw}\\

\haiku{Oene keek nu ook.}{meer in andere richting}{terwijl hij boomde}\\

\haiku{Hij voer voorbij de,.}{hut van oude Ulbe die}{nog wel slapen zou}\\

\haiku{- Het zou toch nog wel,.}{wat duren eer het zoover kwam}{troostte hem Oene}\\

\haiku{En zou er van zijn?}{geld iets overblijven als hij}{het hier instak}\\

\haiku{Doch zijn bezit was,.}{zoo weinig slechts enkele}{honderden guldens}\\

\haiku{Ook Heerk had thans voor,.}{hem iets dat hem vroeger nooit}{was opgevallen}\\

\haiku{Nee, van de dochter '.}{wist men geen kwaad ent was}{aan haar dat Heerk dacht}\\

\haiku{Maar de twee mannen.}{hadden daar ondervinding}{van en klaagden niet}\\

\haiku{Het water voor de.}{koffie ging overkoken en}{siste in het vuur}\\

\haiku{En uit de verte,;}{naderde een oude boer}{stooterig van gang}\\

\haiku{{\textquoteleft}Als de tram komt dan!}{gaan we alle dagen naar}{de school in Manswerd}\\

\haiku{Zij naderden in,:}{flinke vaart het dorp dat zich}{scheen uit te breiden}\\

\subsection{Uit: Heerk Walling}

\haiku{Ik waarschuwde haar ',.}{v\'o\'ort nieuwe huis gebouwd}{werd maar zij lachte}\\

\haiku{Haar vader's vreemde.}{uitingen van de laatste}{tijd troffen haar hart}\\

\haiku{Doch zijn mond beefde.}{en hij kon zijn tranen niet}{geheel weerhouden}\\

\haiku{{\textquoteright} Natuurlijk ging hij, -?}{uit waarom zou hij op de}{Zondag thuisblijven}\\

\haiku{{\textquoteleft}Ik geloof niet dat, -}{Jelmer bij mij past en bij jou}{past hij nog minder}\\

\haiku{Hij is zoo kwaad niet{\textquoteright},}{zei Oene en Harmke die}{niet meer antwoorden}\\

\haiku{Hij had overnacht in;}{de stad en was nu op de}{terugweg naar huis}\\

\haiku{vandaag en merkte.}{op dat ze te veel thuis zat}{in de laatste tijd}\\

\haiku{Doch deze keer leek ',.}{t haar moeilijker ze wist}{zelf niet recht waarom}\\

\haiku{Nauwelijks zag zij.}{hem in het donker en hij}{verdween geruischloos}\\

\haiku{Hij was opgeruimd,;}{van aard en treurde niet om}{een meisje zooals Heerk}\\

\haiku{Braaf dat je komt, want{\textquoteright},.}{je kunt mij juist even helpen}{antwoordde Germen}\\

\haiku{Er ontstond eenige;}{opgewondenheid toen het}{ding klaar bleek te zijn}\\

\haiku{Germen moest verteld '.}{hebben dat hij plan had heen}{te gaan uitt dorp}\\

\haiku{{\textquoteleft}Omdat ik ze niet.}{goed versta en zij lachen}{mij uit als ik praat}\\

\haiku{{\textquoteright} Heerk hoorde dat zij;}{gedeeltelijk sprak als de}{menschen in Manswerd}\\

\haiku{{\textquoteleft}Je ziet haar in lang,,.}{niet weer Heerk ze gaat morgen}{terug naar Manswerd}\\

\haiku{Hij gaf Iefke een,,}{hand wenschte haar goede}{reis zoo opgewekt}\\

\haiku{Wiebe en Jikke ', '.}{zaten opt dek daart}{nog weinig koel was}\\

\haiku{zij wenschte niet.}{in nauwer betrekking te}{komen met Leida}\\

\haiku{Toen ze zich voelden,.}{opgemerkt knikten ze en}{hervatten haar werk}\\

\haiku{HIJ kwam tegen de.}{winter thuis en terug in}{zijn oude leven}\\

\haiku{In deze tijd ving.}{hij aan in zijn landstaal}{verzen te schrijven}\\

\haiku{Enkele vrouwen.}{zag men wel in het deurgat}{van haar woning staan}\\

\haiku{Moeder is er boos, -{\textquoteright},,.}{om en ik ook zei hij half}{ernstig half in scherts}\\

\haiku{Hij sprak Anne in,.}{de volgende lente toen}{zij  trouwen ging}\\

\haiku{Hij hoorde de wind '.}{niet en door het raampje zag}{hijt hemelblauw}\\

\haiku{Hij beschouwde en,;}{herkende ze schoon ze ver}{verwijderd waren}\\

\haiku{- Het zou toch nog wel,.}{wat duren eer het zoover kwam}{troostte hem Oene}\\

\haiku{Doch zijn bezit was,.}{zoo weinig slechts enkele}{honderden guldens}\\

\haiku{Ook Heerk had thans voor,.}{hem iets dat hem vroeger nooit}{was opgevallen}\\

\haiku{Nee, van de dochter '.}{wist men geen kwaad ent was}{aan haar dat Heerk dacht}\\

\haiku{Maar de twee mannen.}{hadden daar ondervinding}{van en klaagden niet}\\

\haiku{Het water voor de.}{koffie ging overkoken en}{siste in het vuur}\\

\haiku{En uit de verte,;}{naderde een oude boer}{stooterig van gang}\\

\haiku{Maar er waren ook.}{oude eigenschappen hem}{blijven aankleven}\\

\haiku{{\textquoteleft}Als de tram komt dan!}{gaan we alle dagen naar}{de school in Manswerd}\\

\haiku{Zij naderden in,:}{flinke vaart het dorp dat zich}{scheen uit te breiden}\\

\section{Jeanne van Schaik-Willing}

\subsection{Uit: Witte veren}

\haiku{dat hij negentien,}{was had hij zich aangepast}{aan de maatschappij}\\

\haiku{Ten aanzien van dit.}{laatste geheim wist hij niet}{van marchanderen}\\

\haiku{Ze was het enige.}{heel jonge meisje onder}{de aanwezigen}\\

\haiku{Zonder antwoord af,.}{te wachten barstte ze los}{in een schorren lach}\\

\haiku{Daarginds is je stoel,{\textquoteright},.}{Minny zei ze en wees naar}{een stoel verder weg}\\

\haiku{{\textquoteleft}Geen gekheid,{\textquoteright} zei hij, {\textquoteleft},.}{Geen gekheid morgen gaan wij}{twee\"en wandelen}\\

\haiku{Ook Mia was, na de,.}{nachtrust de opwinding van}{den avond te boven}\\

\haiku{Maar Mia roerde het,.}{chapiter niet aan ze kon}{het niet aanroeren}\\

\haiku{H\'a\'ar voeten zouden.}{bruin en stevig zijn als van}{een gezond jong dier}\\

\haiku{Ach, een droppel bloed,.}{verscheen aan haar vinger ze}{stak hem in haar mond}\\

\haiku{Hoogmoed, hoogmoed, ja,.}{waar hij doorheen moest om straks}{de straf te smaken}\\

\haiku{De kreet der meeuwen,.}{kwam nu uit een wereld die}{te wijd voor haar was}\\

\haiku{Zelfs trok zijn ooglid,.}{even omhoog zij zag zijn oog}{dicht bij het hare}\\

\haiku{Was niet juist zij de?}{verleidster met onschuld als}{verleidingsmiddel}\\

\haiku{je ziet me wel weer{\textquoteright}.}{eens terug in een wolk van}{parfum naar buiten}\\

\haiku{Zij, Graminska van.}{zich zelf was een goeie vriendin}{van mij uit Mexico}\\

\haiku{En dat wie den dood.}{niet riskeert het leven niet}{waard is te leven}\\

\haiku{Zo, nu laat ik je.}{even alleen om met elkaar}{kennis te maken}\\

\haiku{Deze liet op haar.}{beurt het document in de}{eigen tas glijden}\\

\haiku{Het vreemde was, dat,}{ze over den achternaam niet}{nadacht wat ze zocht}\\

\haiku{Het enige antwoord.}{w-as een hartstochtelijk}{schudden van het hoofd}\\

\haiku{{\textquoteright} Ambroise keerde:}{zich met het gezicht naar den}{muur en fluisterde}\\

\haiku{{\textquoteright} lachte madame, {\textquoteleft}.}{Gaillardjij bent en je blijft}{onverbeterlijk}\\

\haiku{Zelfs vermoedde zij,.}{dat hij minder ziek  was}{dan hij voorwendde}\\

\haiku{Marie wilde het,.}{ook niet langer negeren}{ze was zeer ontroerd}\\

\haiku{{\textquoteleft}Niet d\'a\'arom,{\textquoteright} zei hij nog.}{eens binnensmonds en wendde}{beiden zijn rug toe}\\

\haiku{Rembrandt's hunkeraar.}{rees overeind en werd man van}{de wereld uit 1947}\\

\haiku{Dat wil niet zeggen,.}{dat hij cynisch met deze}{vriendinnen omging}\\

\haiku{{\textquoteleft}Maar kijk toch eens, wat,!}{een schone krullekens het}{lijkt wel zuiver goud}\\

\haiku{Met een deel van haar.}{moeders versterf nam Mia den}{inventaris over}\\

\haiku{Dan trachtte zij zich.}{uit te putten in kleine}{vriendelijkheden}\\

\haiku{{\textquoteleft}Weet je wel, dat dit,?}{een prachtig dingetje is}{in \'e\'en woord prachtig}\\

\haiku{Dit was de enige.}{bijdrage tot het gesprek}{van Chamotte}\\

\haiku{De deuren naar een,.}{tuintje stonden open daarin}{bloeiden seringen}\\

\section{Margo Scharten-Antink}

\subsection{Uit: Catherine}

\haiku{achter zich schuurden,,.}{zijn handpalmen steunzoekend}{tegen den schachtmuur}\\

\haiku{Hij had dadelijk,.}{wel geroken dat er wat}{te verdienen viel}\\

\haiku{Nou was het alleen,;}{nog de vraag maar hoe het er}{van binnen uitzag}\\

\haiku{t Was een rare,,;}{troep een raar zoodje waar die}{kapel aan hoorde}\\

\haiku{en hij had drommels,...}{goed begrepen dat het niet}{van de droogte kwam}\\

\haiku{{\textquoteright} dacht dan de man weer,:}{maar hij kon toch zijn oogen niet}{van haar afhouden}\\

\haiku{En ik kan zooveel,,!}{jongens krijgen als ik wil}{maar ik wil ze niet}\\

\haiku{Ik wil het nog eens,;}{op mijn gemak bekijken}{een kwartiertje maar}\\

\haiku{Ze zijn allemaal,?}{zot op die kapel daar heb}{je geen begrip van}\\

\haiku{hun gaan, samen, naar,;}{boven in de warmte waar}{hij over gevloekt had}\\

\haiku{Een gevoel van groote.}{ellende was eensklaps op}{haar neergedonkerd}\\

\haiku{stilletjes, stikem......}{gingen ze een andermans}{bidhuis nabouwen}\\

\haiku{die kapel was ook,.}{maar larie dat had ze dien}{morgen wel geleerd}\\

\haiku{En de kapel, die,?...}{kapel waar ze altijd zoo}{trotsch op was geweest}\\

\haiku{Ze gaf wel niets om,......}{dien jongen ze had nooit wat}{om hem gegeven}\\

\haiku{hoe de beeldjesman,,......}{zelf bij hen bij de Daene's}{was komen kijken}\\

\haiku{{\textquoteright} Wat konden al die?}{schreeuwende menschen in hun}{huis haar nu schelen}\\

\haiku{Reeds draafden haar de;}{rappe voeten terug in}{instinctieve vlucht}\\

\haiku{niets van buiten af.}{liet ze als werkelijkheid}{tot zich doordringen}\\

\haiku{De week was al op,, -.}{de helft het werd Donderdag}{de man kwam nog niet}\\

\haiku{een komplot, negen,;}{kerels en haar grootmoeder}{samen tegen \'e\'en}\\

\haiku{ze waren wel geen......}{femelaars en de duivel}{kon ze niet schelen}\\

\haiku{t Wordt morgen vier,}{weken dat ik die sleutel}{voor je gappen moest}\\

\haiku{door de zand-rulte.}{ging het den karreweg af}{naar de kapel toe}\\

\haiku{{\textquoteright} Catherine had.}{het stemgeluid van Lambert}{den voerman herkend}\\

\haiku{zij verzon zelfs niet.}{bij welk dorp zij wel zoo dicht}{genaderd kon zijn}\\

\haiku{Later nog kwam zij,;}{door een streek van steenschachten}{die zij niet kende}\\

\haiku{Een oogenblik dacht.}{het kind om er heen te gaan}{en eten te vragen}\\

\haiku{Toen, met de leege schaal,,:}{op schoot zittend wat loom-warm}{en voldaan dacht ze}\\

\haiku{Meteen draafde ze,.}{al den grintweg af die naar}{de heirbaan leidde}\\

\haiku{Zij ging weer naar den,,.}{voorkant wou d\'a\'ar wachten of}{er ook een uit kwam}\\

\haiku{De kerels, in een,.}{kring om hem heen waren nu}{meester van het licht}\\

\haiku{Barsch voorbijgaand,,.}{wat grom-vloekend lieten}{zij het kind met rust}\\

\haiku{Later zwaar-plofte...}{beneden het rondstappen}{van den bultenaar}\\

\haiku{Halsstarrig lag zij;}{maar met het gezicht naar den}{bedstee-wand gekeerd}\\

\haiku{in twee sprongen was,,.}{ze bij de deur gluur-tuurde}{door een kier angstbleek}\\

\haiku{Zij luister-loerde,,.}{naar een aanduiding raadde}{ze lokte ze uit}\\

\haiku{Ze wist nu, dat het......}{schilderij in de kapel}{van Waramme hing}\\

\haiku{beefhandend liet ze.}{zich zoo maar het vaatwerk uit}{de vingers glippen}\\

\haiku{{\textquoteright} Gansch niet overrast zag,,;}{zij hem aan zij wist wel dat}{dit nu komen moest}\\

\haiku{zich afjakkerend,......}{voor haar kinders geranseld}{misschien bovendien}\\

\section{Carel Scharten en Margo Scharten-Antink}

\subsection{Uit: De nar uit Maremmen. Deel 1: Massano}

\haiku{Carel Scharten en,.}{Margo Scharten-Antink De}{nar uit Maremmen}\\

\haiku{Samen stonden zij.}{bij de deur-opening naar}{het groote dakterras}\\

\haiku{Lorenzo keek niet,.}{on vermaakt maar Ottavio}{trok de schouders op}\\

\haiku{Je moest trouwens zelf{\textquoteright},.}{ook naar Florence komen}{viel Lorenzo bij}\\

\haiku{'k Zou er z\'o\'o een,.}{voor je kunnen huren een}{vorstelijk atelier}\\

\haiku{En daar zou hij niet,,,....}{op gesteld zijn omdat na}{twaalf jaar hij Pimpia}\\

\haiku{Ik voel n\`og meer voor!}{de nieuwe vriendschap van de}{Rooien en Rome}\\

\haiku{altijd nog wel een {\textquotedblleft}{\textquotedblright}....}{of andereMadonna}{van het Kattegat}\\

\haiku{{\textquoteright} {\textquoteleft}Een andere maal,{\textquoteright},;}{amico mio viel Renato}{hem in de rede}\\

\haiku{De man achter haar,.}{trok zijn schouders op draaide}{zich weer naar zijn plaats}\\

\haiku{In zijn stompen muil,,;}{bleek-roze en nat spalkten}{kwaad de neusgaten}\\

\haiku{Hij stak zijn schetsboek,.}{weg wilde het koord om zijn}{middel los knoopen}\\

\haiku{{\textquoteleft}Maar u heeft toch niet {\textquotedblleft},{\textquotedblright}?}{gel\'o\'ofd in dien nieuwenChristus}{Leider en Richter}\\

\haiku{{\textquoteleft}Hoeveel merken zijn?}{er vandaag wel gezet op}{uw mercatura}\\

\haiku{De markies zette,,.}{zijn lorgnet op bezag het}{aandachtig knikte}\\

\haiku{Hij zat tegenover,.}{mij nog dichter dan u nu}{tegenover mij zit}\\

\haiku{David was door de {\textquotedblleft}{\textquotedblright};}{bevolking d\'a\'ar het eerstde}{heilige genoemd}\\

\haiku{David, ja, die werd,,.}{blindelings gevolgd en al}{wat hij deed was goed}\\

\haiku{Maar het was, of het.}{dezelfde geschriften niet}{meer voor mij waren}\\

\haiku{Zijn gastheer had het.}{gevoeld en streek wat beschaamd}{door zijn witten baard}\\

\haiku{voor het vervangen,.}{der geheime biecht door de}{openlijke minder}\\

\haiku{was het niet, of die,;}{op dit oogenblik z\`ag wat}{hij niet had gezien}\\

\haiku{Toch was ik toen al,.}{zesentwintig en sinds drie}{jaar in Florence}\\

\haiku{Ik holde terug.}{om ze tegen te komen}{aan de wegwending}\\

\haiku{in mijn oog zijn zijn....}{jongeren nog veel mooier}{geweest dan hij zelf}\\

\haiku{in het kerkje van....}{San Giorgio een vijfluik van}{Andrea di Niccol\`o}\\

\haiku{Maar wat is dat, pro,?}{fessore de Madonna}{van het Kattegat}\\

\haiku{Uw portret heb ik{\textquoteright},.}{gezien in Florence ging}{Sergio vertellen}\\

\haiku{{\textquoteright} Een uur lang zaten.}{de oude schilder en de}{jonge te werken}\\

\haiku{- had van den eersten:}{dag af de bewondering}{van Sergio gewekt}\\

\haiku{{\textquoteright} {\textquoteleft}Nou{\textquoteright}, zei Renato, {\textquoteleft},,:}{wie weet wat ik doe  maar}{op \'e\'en voorwaarde}\\

\haiku{{\textquoteright} dacht Renato, die - {\textquoteleft}?}{Mastropieri niet kende}{zit hem d\'a\'ar de kneep}\\

\haiku{en de politie.}{hield zich aan de functie van}{verkeers-agent}\\

\haiku{En de paarden zelf,,,.}{mooie rasbeesten die liepen}{dat het een lust was}\\

\haiku{Het handgeklap en;}{het getrappel ratelde}{tien minuten lang}\\

\haiku{dien middag, op den,....}{terugweg zouden ze er}{nog eens door moeten}\\

\haiku{Maar Renato zag,.}{wel hoe begeerig hij naar}{het krabbeltje was}\\

\haiku{Met welk een sierlijk!}{gemak maakte die jongen}{zich van hem  los}\\

\haiku{Hij zette zich voor.}{een klein caf\'e en dronk er}{een zwarte koffie}\\

\subsection{Uit: De nar uit Maremmen. Deel 2: Florence, de drie blinden}

\haiku{Carel Scharten en,.}{Margo Scharten-Antink De}{nar uit Maremmen}\\

\haiku{Hij zit sinds een week...!}{op wat hij zijn atelier noemt}{een ouwe toren}\\

\haiku{Wat een zotte vent, -,...}{en je moest erkennen dat}{ze gelijk hadden}\\

\haiku{Don Pompeo zou wel,.}{zeggen dat de logica}{ver te zoeken viel}\\

\haiku{{\textquoteright} Een diepe zucht van,,.}{verluchting onder naast hem}{ontroerde zijn hart}\\

\haiku{{\textquoteright} Met groote, ernstige,.}{oogen keek Silvio om alles}{goed te begrijpen}\\

\haiku{Renato, die het,:}{kind meende te begrijpen}{waarschuwde zachtjes}\\

\haiku{En toen hij de deur,...}{opendeed zat kwispelstaartend}{Brisc om het hoekje}\\

\haiku{Er hing een fijne,,.}{zoete geur van de witte}{wassige bloesems}\\

\haiku{'t Geneerde hem,,.}{die pet die zich met hem scheen}{bezig te houden}\\

\haiku{dan, hartelijk en,.}{gezellig deed hij een soort}{familierelaas}\\

\haiku{vroeg Niccolini,.}{toen zij weer stonden in de}{deur van het atelier}\\

\haiku{{\textquoteright} Tegelijkertijd.}{werd hij zich bewust van een}{zekere schaamte}\\

\haiku{Ik zeg alleen, dat;}{hij geen quitantie's geeft en}{geen boeken laat zien}\\

\haiku{En voor jou, voor ons,!}{alle drie was het een kans}{op verkoopen meer}\\

\haiku{{\textquoteleft}O, als je er met...{\textquoteright};}{Ottavio over spreken moet}{hoonde Lorenzo}\\

\haiku{Je hebt behoefte,,}{om je uit te spreken zeg}{je en je vertrouwt}\\

\haiku{{\textquoteright} {\textquoteleft}Vooruit Sandro, en,{\textquoteright}.}{waardeer je goed gesternte}{schertste de markies}\\

\haiku{de natuurlijke,...}{de onvermijdelijke}{strijd om het bestaan}\\

\haiku{Ik zie ons nog, na,.}{lange dagen van onrust}{als er zoo'n kaart kwam}\\

\haiku{Een uur later riep:}{soms mijn vader plotseling}{met een heete stem}\\

\haiku{{\textquoteleft}en wij in ons huis,{\textquoteright}.}{met centrale verwarming}{zei ze vol schaamte}\\

\haiku{{\textquoteright} - Dat is zeker, de!}{oorlog heeft onze zielen}{niet weinig gestaald}\\

\haiku{Twee jaar nadien, toen,...}{ik achttien was geworden}{ben ik ook gegaan}\\

\haiku{Intusschen ontsloeg!}{Nitti de deserteurs uit}{de gevangenis}\\

\haiku{{\textquoteright} {\textquoteleft}Over me zoon,{\textquoteright} deed Pia,.}{gesloten met een blik vol}{argwaan naar Sandro}\\

\haiku{Doch middelerwijl.}{hoorde hij zachte stappen}{de zaal verlaten}\\

\haiku{Het schilderijtje,,:}{dat zij had meegebracht bleek}{een klein stilleven}\\

\haiku{Nog niet zoo heel lang,,...}{geleden is geloof ik}{zijn vrouw gestorven}\\

\haiku{Ottavio ging naar.}{voren en zond het jonge}{mensch op de taxi-jacht}\\

\haiku{t personeel was,...}{naar bed en de menschen zelf}{waren in de stad}\\

\haiku{Nooit, dacht zij, had zij.}{haar vader zoo begrijpend}{en lief gevonden}\\

\haiku{eigenlijk alleen.}{om een afspraak te maken}{voor Gian Carlo}\\

\haiku{Bloeide die jongen?}{daar niet zelf als een bloem van}{blakende jonkheid}\\

\haiku{{\textquoteleft}Kom mee, kom mee,{\textquoteright} drong,.}{Renato terwijl hij hun}{voorging met de kaars}\\

\haiku{{\textquoteright} riep hij plotseling.}{met een luide stem door de}{nachtstille kamer}\\

\haiku{Kon het dan niet \'e\'en,,?}{uur zijn geweest of half twee}{dat ze hoorde slaan}\\

\haiku{Want Pia, dat wist hij,,}{uit de zes maanden dat ze}{nu bij hem diende}\\

\haiku{- Nee... zei de man, - dat,...}{beetje gerochel dat was}{al van tien jaar her}\\

\haiku{een man met altijd,,...... {\textquoteleft}}{bronchitis een vrouw die nooit}{thuis is en Gino}\\

\haiku{'t Is al beroerd,...}{genoeg dat je vader den}{jongen schuldig denkt}\\

\haiku{Hij voelde zich als,.}{verlamd na de opwinding}{der laatste dagen}\\

\haiku{Zelfs de marchese.}{Niccolini kwam met een}{dergelijk voorstel}\\

\haiku{en dat hier al zijn!}{uitgangetjes met Silvio}{had opgeluisterd}\\

\haiku{Gian Carlo kwam,.}{hem halen in den auto}{zorgzaam als altijd}\\

\haiku{Een boerenmeisje,;}{opende het hooge hek door twee}{steenen leeuwen bewaakt}\\

\haiku{{\textquoteright} En Renato zag,:}{in dat \'e\'en offer te groot}{kan zijn voor een mensch}\\

\haiku{de twee voornaamste;}{theaters van Florence}{waren gesloten}\\

\haiku{De opera, dat was,.}{het wat hij in Massano}{altijd had gemist}\\

\haiku{een afwerende.}{vijandelijkheid lag er}{over heel haar wezen}\\

\haiku{Ja, natuurlijk, ze....}{zullen op \`ons wel weer de}{verdenking gooien}\\

\haiku{{\textquoteleft}Als u me soms ook,!}{niet meer vertrouwt ga ik net}{zoo lief direct heen}\\

\haiku{Ze greep Renato's:}{beide handen en met haar}{stem nog vol tranen}\\

\haiku{{\textquoteright} riep Renato, {\textquoteleft}heeft?}{hij dat gestolen goed bij}{jullie gevonden}\\

\haiku{'t k\`an toch alles,{\textquoteright}, {\textquoteleft} '....}{zoo wezen drong Pia opnieuw}{mijn man zegtt ook}\\

\haiku{Hier was dan toch \'e\'en....}{heete vlam uit het Rijk van}{de Liefde verdwaald}\\

\haiku{De agent, die Gino....}{Gori herkend zou hebben}{op de motorfiets}\\

\haiku{{\textquoteleft}Zie morgen maar eens,, -....}{wat je doen wilt ik kan niet}{voor elven hier zijn}\\

\subsection{Uit: De nar uit Maremmen. Deel 3: Naar de eeuwige stad}

\haiku{Carel Scharten en,.}{Margo Scharten-Antink De}{nar uit Maremmen}\\

\haiku{De geelgrijze kuif}{van Sor Agostino trilde}{driftig tusschen}\\

\haiku{{\textquoteright} Renato moest even:}{de wenkbrauwen fronsen om}{zich voor te stellen}\\

\haiku{een harde trek lag,.}{er om dien mond zelfs als die}{mond van liefde sprak}\\

\haiku{{\textquoteright} vroeg een man met een, {\textquoteleft}....}{doodsbleek gezichtik kon ze}{niet onderscheiden}\\

\haiku{{\textquoteright} {\textquoteleft}Twee legers in een,{\textquoteright}, {\textquoteleft}....}{land zei domp een anderhoe}{moet dat afloopen}\\

\haiku{De regen kwam bij.}{stroomen neer en gudste en}{droop langs de ruitjes}\\

\haiku{{\textquoteleft}weet u nog die keer?}{van de mercatura op}{de Villa Magna}\\

\haiku{{\textquoteleft}Als er eten is voor,,!}{negen is er ook eten voor}{veertien zei de vrek}\\

\haiku{er bleek v\'o\'or vijf uur.}{in den morgen geen trein naar}{Florence te zijn}\\

\haiku{Bij elk portier stond,.}{een zwarthemd op schildwacht het}{geweer in den arm}\\

\haiku{de koning deed hem,.}{naar Rome ontbieden om}{hem te raadplegen}\\

\haiku{Zij staken schuin het,.}{plein over waar het druk was van}{de Dinsdagsche markt}\\

\haiku{En opnieuw had hij,.}{een ingeving die hem stil}{maakte van binnen}\\

\haiku{'t Is al zooveel,!}{jaren geleden dat ik}{het laatst geskied heb}\\

\haiku{Wij dachten, dat hij,,.}{h\'e\'erschen zou rechtvaardig maar}{ongenadig streng}\\

\haiku{H\'e\'el interessant,,,!}{zeggen ze moderne kunst}{zooals je hier nooit ziet}\\

\haiku{Een antwoord, dat hij,.}{had willen geven vaagde}{weg van zijn lippen}\\

\haiku{Verscheidene uren,,.}{dezen nacht had hij erover}{wakker gelegen}\\

\haiku{zij behandelen,,;}{hem slecht den nietsnut die hun}{genadebrood eet}\\

\haiku{zwijgt tegen mij, zijn,,....}{verdediger tegen den}{rechter tegen U}\\

\haiku{Renato voelde.}{zich als gedreven naar de}{woning van zijn zoon}\\

\haiku{{\textquoteleft}Het proces is weer,{\textquoteright}.}{verdaagd tot over drie weken}{zei ze gelaten}\\

\haiku{- Was zij het, die zoo?}{plotseling zijn gedachten}{verhelderen deed}\\

\haiku{{\textquoteleft}Kijk eens, Aurora,{\textquoteright}, {\textquoteleft},}{zei hijje zegt dat Silvio}{geen band meer tusschen}\\

\haiku{{\textquoteright} Aurora keek hem.}{met een wonderlijken blik}{van overwinning aan}\\

\haiku{Daar had hij een heel,....}{regiment voor zich staan en}{hij had niet gezien}\\

\haiku{Plotseling ging zijn.}{blik naar het portret aan den}{bespinragden wand}\\

\haiku{{\textquotedblleft}Weest niet bang voor mij,,....}{eerwaarde Senatoren}{omdat ik paardrijd}\\

\haiku{Maar de jeugd is een,...}{goddelijk kwaad waarvan men}{elken dag geneest}\\

\haiku{{\textquoteright} Renato zat met.}{gefronste wenkbrauwen voor}{zich heen te kijken}\\

\haiku{Met een mat lachje;}{had hij het vergrootglas in}{ontvangst genomen}\\

\haiku{Vele weken bleef.}{Renato opgesloten}{binnen zijn arbeid}\\

\haiku{en.... u en ik, als,....}{we niet vertellen wat we}{niet kwijt willen zijn}\\

\haiku{{\textquoteleft}verzwijgen is soms;}{een bewijs van waardigheid}{en zelfbeheersching}\\

\haiku{doch bij zoov\'e\'el al had;}{de ijdele man zich}{moeten neerleggen}\\

\haiku{Groepen menschen, langs,.}{de Arnokade scholen}{bangelijk bijeen}\\

\haiku{Uit een groep naar huis,,:}{haastenden die hij langs kwam}{ving Renato op}\\

\haiku{Maar de geestdrift van....}{zijn jongetje liet hem toch}{niet onverschillig}\\

\haiku{{\textquoteright} {\textquoteleft}Vraag dat maar eens aan,....}{een appeltje dat al ver}{den winter in is}\\

\haiku{Doet u maar netjes!}{uw nieuwe zwarte pak van}{onze bruiloft aan}\\

\haiku{Ik wil, dat ons volk.}{weer een landbouwend volk wordt}{bij uitnemendheid}\\

\haiku{Groote werken werden,.}{alom ondernomen op}{velerlei gebied}\\

\haiku{Een vrede lag over,;}{deze trekken gestreken}{die even droevig scheen}\\

\haiku{Doch de mond was nog,.}{altijd niet de mond die hem}{bevredigen kon}\\

\haiku{K\`on die Liefde niet?}{uitgesproken worden met}{menschelijken mond}\\

\haiku{{\textquoteleft}En ik dacht nog wel,!}{dat er geen mysterie was}{in deze trekken}\\

\haiku{- waarom was hij voor?}{deze gelegenheid zoo}{ijselijk precies}\\

\haiku{En aan den einder,,.}{als op een heuveltroon het}{ivoorgele Fiesole}\\

\subsection{Uit: 't Geluk hangt als een druiventros}

\haiku{Jawel, jawel,{\textquoteright} had, {\textquoteleft};}{Angelo gezegddat's}{braaf uitgerekend}\\

\haiku{{\textquoteright} {\textquoteleft}De Melli's staan geen,{\textquoteright}.}{zoon af voor een veermansdienst}{zei Filippo fel}\\

\haiku{Nu vermengde zich,;}{de vocht die opsteeg met de}{nevels die daalden}\\

\haiku{{\textquoteleft}Waarom hebben zij, '?}{mij maar niet even geroepen}{int voorbijgaan}\\

\haiku{En toen de overgang,...;}{dan voorgoed vast stond was hij}{meteen maar hertrouwd}\\

\haiku{En nou kwam hij hem...}{nog zijn zoon afhalen voor}{een veermanspostje}\\

\haiku{Palmira was tot;}{deze vertrouwelijkheid}{nooit toegelaten}\\

\haiku{{\textquoteleft}Bifoli is toch......}{in zijn ongelijk laat hij}{ze binnen vragen}\\

\haiku{Waarom konden ze?...}{hem en zijn twee jongens hier}{niet met rust laten}\\

\haiku{III Dien avond werd het.}{een luidruchtiger maaltijd}{nog dan gewoonlijk}\\

\haiku{een getal, waar, bij,;}{het lezen in de courant}{zijn aandacht op viel}\\

\haiku{het nummer van de,;}{locomotief die zijn trein}{naar Florence trok}\\

\haiku{- Wat w\`as Florence!}{toch heerlijk op zoo'n mooien}{December-dag}\\

\haiku{ik heb honger... en,,?}{de Cavaliere die mijn}{gast is zeker ook}\\

\haiku{{\textquoteleft}Ja, nat\'u\'urlijk was,...}{het noodig dat de jongste van}{Rovai veerman werd}\\

\haiku{tienduizend lire,......}{een ronde som die overschrijdt}{je dan ook niet meer}\\

\haiku{van de vierhonderd:}{lire had hij nu al meer}{dan de helft besteed}\\

\haiku{hij had toch voor h\`en,!}{\'o\'ok zijn best gedaan en aan}{alle drie gedacht}\\

\haiku{{\textquoteright} Voor geen geld zou hij;}{zijn ouden vriend over Grassi}{hebben gesproken}\\

\haiku{Goede reis, goede,{\textquoteright};}{reis knikkebolde Carlo}{een beetje onthutst}\\

\haiku{Hij zou toch niet op?}{een dadelijk offer van}{hun kant aandringen}\\

\haiku{Hun eigen landwijn,;}{was onverbeterlijk in}{heel de streek vermaard}\\

\haiku{het eenige dat haar...}{nog een bevrijding was uit}{dit valsche leven}\\

\haiku{eindelijk moest ook.}{de Casa Rovai van de}{hand worden gedaan}\\

\haiku{hij ergerde zich;}{aan het gemis aan goeden}{toon en aan doorzicht}\\

\haiku{De maan stond als een;}{goud-witte schijf aan den}{beslagen hemel}\\

\haiku{Langs en boven en.}{voor hem wemelden de wit}{betroste twijgen}\\

\haiku{{\textquoteleft}het vogeltje, dat,.}{in den spiegel kijkt{\textquoteright}15 zoo heet}{het in den volksmond}\\

\haiku{Domenico en.}{Guido bekwamen niet van}{hun verwondering}\\

\haiku{Door dit gevaar had,!}{hij dat lieve fiere kind}{moeten heenbrengen}\\

\haiku{Maar een oogenblik,:}{later was hij toch gevleid}{als Angelo prees}\\

\haiku{- Was dat de jongste, -?...}{Signorina Sassetti}{of was zij het niet}\\

\haiku{Toch had de Conte...}{bij zijn vertrek 20 L. aan}{de meid gegeven}\\

\haiku{Nee, nee, om Godswil,...}{geen verplichtingen meer aan}{dien ellendeling}\\

\haiku{Was zij het niet, die?}{ook nu weer de hypotheek}{wilde verhoeden}\\

\haiku{{\textquoteleft}In geen acht jaar is;}{Filippo over den drempel}{van mijn huis geweest}\\

\haiku{Ga je beneden?}{niet even Virginia en de}{kinderen groeten}\\

\haiku{{\textquoteright} Francesca, bij,:}{diergelijke uitvallen}{temperde met schrik}\\

\haiku{Toch woog hem nog meer,.}{het geld dat zij van Aldo}{genomen hadden}\\

\haiku{Maar toen zij hem den,... {\textquoteleft}}{vierden dag \'o\'ok nog misten}{gingen ze zoeken}\\

\haiku{Eindelijk vonden,,;}{ze hem achter het vierde}{vat in den kelder}\\

\haiku{Tallooze dingen moesten,.}{er gebeuren over en door}{en ondanks elkaar}\\

\haiku{Op de terugvaart,,:}{van halfweg het water riep}{zonnig Silvano}\\

\haiku{En de kinderen,,;}{den ganschen middag waren}{er niet weg te slaan}\\

\haiku{tersluiks hadden zij:}{er geplukt en de korrels}{gepeld en geproefd}\\

\haiku{In den namiddag;}{kwam ook de Signorina}{Giselda kijken}\\

\haiku{Ze hadden wel een,,}{mud graan verloren wel twee}{mudden misschien z\'o\'o}\\

\haiku{Maar hoe dan ook, het '.}{ging hem opt oogenblik}{niet naar den vleesche}\\

\haiku{Eerst, zoometeen, een;}{kinawijntje met ijs en}{spuitwater drinken}\\

\haiku{maar vlekkerig rood.}{zag zijn gelaatskleur en zijn}{oogen stonden troebel}\\

\haiku{Omdat zij van d\'at,!}{geld het zekerst was dat hij}{het zou herstellen}\\

\haiku{Filippo lei zijn,.}{pijpje naast zich neer sloot de}{handen om zijn knie}\\

\haiku{Het hardsteenen perron.}{voor de hoofddeur blaakte wit}{in de morgenzon}\\

\haiku{die sloofde zich uit,...?}{om van wijn en graan dat geld}{te halen en dan}\\

\haiku{in de schaduwbocht.}{der bijgebouwen toefde}{zij besluiteloos}\\

\haiku{Zij ging zitten op,.}{het bankje dat achter de}{dichte deurhelft stond}\\

\haiku{Zij lette weinig,.}{op het meisje merkte niets}{bizonders aan haar}\\

\haiku{{\textquoteright} {\textquoteleft}Geen wonder, dat hij,{\textquoteright};}{nog niet uitgeslapen is}{zei Emilia bitter}\\

\haiku{En opeens, als met,.}{een slag van stilte was het}{hagelen gedaan}\\

\haiku{Voor 't oogenblik}{waren de vijgen en de}{late perziken}\\

\haiku{- Zondag ging zij naar,,...}{den Incontro biechten en}{bidden boete doen}\\

\haiku{Liet Guido met een...}{enkel woord zijn komst op dien}{dag voorbereiden}\\

\haiku{En van den vroegen:}{morgen af doorvorschten}{Ubaldo's oogen het land}\\

\haiku{Hij heeft het lesje,,!}{vergeten dat hij van thuis}{meekreeg de meelzak}\\

\haiku{Vader heeft hem een...}{paar maanden geleden in}{Florence gezien}\\

\haiku{maar ik weet zeker,,!}{als het er op aankwam klom}{ik weer uren ver mee}\\

\haiku{Leonetta, den,:}{laatsten tijd had zich gewend}{aan de gedachte}\\

\haiku{{\textquoteright} {\textquoteleft}Zoo,{\textquoteright} zei Filippo, {\textquoteleft}?}{heeft de brutale hond het}{dan toch gewonnen}\\

\haiku{Ik zou nog tienmaal...}{meer zaakjes opknappen dan}{nou al allemaal}\\

\haiku{Paolina zou...}{tante Ortenzia wel even}{gezelschap houden}\\

\haiku{{\textquoteleft}Waarachtig, Signor,,...{\textquoteright} {\textquoteleft}}{Grassi ik mag doodvallen}{zoowaar als ik hier sta}\\

\haiku{{\textquoteright} - hij gromde van nijd - {\textquoteleft};}{tienduizend lire met de}{loopende rente}\\

\haiku{Hoe dat den 15den,?}{December moest gaan met dien}{mislukten wijnoogst}\\

\haiku{Een oogenblik had;}{Angelo een argwanend}{verkennenden blik}\\

\haiku{Zij zag den pastoor...}{dwars door zijn groentelandjes}{hollen naar de kerk}\\

\haiku{{\textquoteright} zei Domenico, {\textquoteleft}'...{\textquoteright}}{stil voor zich heent is ook}{\`onze Nella Dan}\\

\haiku{Hij stond met een kop,,.}{vuurrood van toorn en strekte}{zijn arm naar de deur}\\

\haiku{Zeker, het leven,...}{zal je hier moeilijk vallen}{de eerste jaren}\\

\haiku{En v\'o\'or het bed, vast,}{op haar stoel gekampeerd of}{zij den ganschen dag}\\

\haiku{{\textquoteleft}Zulke prachten van,{\textquoteright}, - {\textquoteleft}...!}{oogen mompelde hij voor zich}{heenzoo'n prachtig kind}\\

\haiku{{\textquoteright} Emilia verstond maar,.}{gebrekkig Engelsch sprak het}{nog gebrekkiger}\\

\haiku{Filippo, onder,.}{den geesel van haar hoon had}{het hoofd gebogen}\\

\haiku{{\textquoteleft}Laat hem dien tijd dan,.}{middendoor deelen en twee}{dagen hier komen}\\

\haiku{En dadelijk was.}{Carolina overeind en naast}{haar S'or Filippo}\\

\subsection{Uit: Typen en curiositeiten uit Itali\"e}

\haiku{een stil tournooi van,.}{kleurige droomen zonder}{botsing noch zege}\\

\haiku{Alle roccolo's;}{uit den omtrek werden in}{gereedheid gebracht}\\

\haiku{De kleine, zwarte;}{koormuts stond hem achter op}{het scherpe voorhoofd}\\

\haiku{Hij was volstrekt niet,....}{van plan dezen morgen al}{aan den slag te gaan}\\

\haiku{Don Matteo boog zich.}{voor het andere luikje}{en spiedde evenzoo}\\

\haiku{De vader wil geen,.}{gevolg aan de zaak geven}{want hij vreest weerwraak}\\

\haiku{Er was een huisknecht,.}{in livrei en Pepino}{droeg witte kokskleeren}\\

\haiku{De aanval van het.}{meisje duurde maar kort en}{liet geen sporen na}\\

\haiku{Je hebt al gauw mijn,{\textquoteright}.}{veeren niet meer noodig lachte}{soms de Contessa}\\

\haiku{Dien middag zou het.}{jonge paar naar hun huisje}{aldaar vertrekken}\\

\haiku{de wind bespeelde,.}{zijn blauw boezeroen open op}{de bekroesde borst}\\

\haiku{er boven, als een,.}{dreiging fronste het woeste}{berg-voorhoofd}\\

\haiku{- Ja, hij was getrouwd,....}{geweest en voor een jaar was}{zijn vrouw gestorven}\\

\haiku{{\textquoteleft}La Fine di un{\textquoteright}, -.}{Miracolo het Einde}{van een Mirakel}\\

\haiku{Wat hadden die drie,}{menschen een verdriet gehad}{en hoe hulpeloos}\\

\haiku{Herinnerde zich,?}{Clorinda niet dat verhaal van}{Papa uit zijn jeugd}\\

\haiku{Was het dit zoele?}{schemeruur met dat alles}{heiligende licht}\\

\haiku{{\textquoteright} En hij schreef nog iets,.}{voor dat de pijn wellicht wat}{verzachten zou}\\

\haiku{Want, eene zondares,!}{was zij geweest en op zoo}{velerlei wijze}\\

\haiku{{\textquoteright} zeide eindelijk,:}{haar ijle stem en het was}{bijna of zij zong}\\

\haiku{Hij bloost rond zijn oogen,,:}{die schroomend dieper worden}{en over zijn voorhoofd}\\

\haiku{nu smolt de laatste.}{zweem voor de lieflijkheid van}{dezen ouderdom}\\

\haiku{Den volgenden dag:}{waren de berichten al}{even weinig gunstig}\\

\haiku{Zijn handen, uit de,.}{roze hemdsmouwen lagen}{weerloos op het dek}\\

\haiku{De dokter vond hem.}{minder goed en had alle}{bezoek verboden}\\

\haiku{{\textquoteleft}Maar U kunt het zoo,{\textquoteright}, {\textquoteleft}.}{niet hebben zei hijvraagt U}{Ottavino eens}\\

\haiku{De rood-blonde....}{golven stroomden dieper neer}{om rooder wangen}\\

\haiku{{\textquoteright} {\textquoteleft}Si....{\textquoteright}, fluisterde een;}{bedeesd stemmetje tusschen}{de gouden haren}\\

\haiku{een bloemlezing van,.}{Italiaansche dichters daar}{was hij erg blij mee}\\

\haiku{Op het vroolijke;}{zonnepleintje schoolde het}{halve dorp bijeen}\\

\haiku{maar hij had nu den.}{drang zijner ijverige}{gedachten elders}\\

\haiku{) vier meter \'e\'en en,,....}{twintig verdeeld in caissons}{metende deze}\\

\haiku{Dan worden de twee!}{kapiteelen weer precies}{als de andere}\\

\haiku{de jonge madam ',!....}{r tuinhoed lei weer in het}{gras slap van de dauw}\\

\haiku{De oude juffrouw,,;}{deed zelf open sjaaltje om het}{hoofd tuinhoed daarop}\\

\haiku{Nog eenmaal gaf hij:}{zijn heelen brief van voor twee}{maanden ten beste}\\

\haiku{de Olijfberg, met een{\textquoteleft},!}{bekend klooster op den top.}{Davvero cara}\\

\section{Arthur van Schendel}

\subsection{Uit: De berg van droomen}

\haiku{hij had het besluit,.}{genomen zij had gezegd}{dat zij mee zou gaan}\\

\haiku{Wat hij zocht was veel.}{schooner en zijn vader zou blij}{zijn als hij het zag}\\

\haiku{Maar het is al laat,.}{en als ze niet komt weet ik}{niet wat ik doen zal}\\

\haiku{Het touwwerk piepte,.}{groote golven bruisten alom}{in de duisternis}\\

\haiku{Misschien is het lang,.}{geleden misschien ook is}{het zooeven gebeurd}\\

\haiku{{\textquoteleft}Dat zijn de man die.}{bakken kan en zijn makker}{de menscheneter}\\

\haiku{Zij waren omtrent:}{honderd schreden gegaan toen}{Puikebest zeide}\\

\haiku{Toen hij gedaan had:}{stak hij het boekje weer in}{zijn zak en zeide}\\

\haiku{De hond kroop onder.}{het bed en leide zijn hoofd}{op zijn voorpooten}\\

\haiku{Zij zingt in den Hooge,,{\textquoteright}, {\textquoteleft}.}{de prinses zeide zijlaat}{ons in de laan gaan}\\

\haiku{{\textquoteright} riep hij onder het.}{schateren naar den mond van}{den knaap wijzende}\\

\haiku{Daarom weet ik zoo.}{goed wat het is als je een}{verkeerden naam hebt}\\

\haiku{, en Denkmar zou het.}{niet aardig vinden als je}{hem anders noemde}\\

\haiku{{\textquoteright} Toen liep Puikebest,.}{hard naar het rozenpoortje}{gevolgd door Kaka}\\

\haiku{Zij kwamen in een,.}{veld van blauwe bloemen het}{gansche veld was blauw}\\

\haiku{{\textquoteright} Er klonk een gil en}{een lach en toen verdween er}{iets snel uit den boom}\\

\haiku{{\textquoteright} Zij lachte zachtkens.}{of er een bron murmelde}{in de nabijheid}\\

\haiku{Budde zat daar met,.}{zijn hoofd in zijn handen een}{oud kaboutertje}\\

\haiku{Toen stonden zij voor.}{den toren van Vreugde de}{Hooge en het paleis}\\

\haiku{Zie, gij strijdt tegen,.}{onze vrienden niet tegen}{onze vijanden}\\

\haiku{{\textquoteleft}Vreemdeling, Abel is,.}{uw naam slechts uw naam kennen}{wij en anders niet}\\

\haiku{Abel, gij zijt groot, er.}{is niemand die zooals gij nog}{nooit heeft gesproken}\\

\haiku{En de stilte klinkt.}{en de zon hoort mijn roep en}{de wereld ontwaakt}\\

\haiku{Alban het witte.}{paard spitste zijn ooren en de}{ruiter hief zijn hoofd}\\

\haiku{Tobias kraaide,.}{zoo plotseling en zoo luid}{dat allen schrikten}\\

\haiku{{\textquoteright} Maar Tobias met:}{het hart van zonlicht verhief}{zijn schallende stem}\\

\haiku{Wij vogels reizen,{\textquoteright},.}{veel sprak Peregrijn de eend}{recht voor zich ziende}\\

\haiku{Zij stonden rondom,.}{den zeeroover ongeduldig}{vragend wat er was}\\

\haiku{Hij was zoo moede.}{dat hij op den peluw bij}{den nachtwacht neerzeeg}\\

\haiku{Eindelijk legde.}{Corinna zachtkens haar}{hand op zijn schouder}\\

\haiku{{\textquoteleft}Het is dwaasheid weg,.}{te loopen als je nooit iets}{gedaan hebt zooals ik}\\

\haiku{Er ging een heele.}{tijd voorbij zonder dat wij}{iets van oom hoorden}\\

\haiku{Een nederig man.}{met een takkenbos op zijn}{rug volgde haar}\\

\haiku{{\textquoteright} Reinbern dacht aan de,.}{prinses waar en wanneer hij}{haar weder zou zien}\\

\haiku{Maar jullie zijn toch}{domme ganzen dat je niet}{eerder verteld hebt}\\

\haiku{{\textquoteright} {\textquoteleft}Vertel het, Morgan,.}{als je goed nieuws hebt geven}{wij je een meloen}\\

\haiku{En zoo werkten we.}{bij ploegen den heelen nacht}{door tot het dag werd}\\

\haiku{Dat moet  je eens,.}{komen zien het akkertje}{staat heelemaal groen}\\

\haiku{Toen zweeg de klok, zooals.}{de wind zwijgt hoog in het woud}{op een lentedag}\\

\haiku{Maar dezen nacht was,.}{zij in slaap gevallen dat}{was nog nooit gebeurd}\\

\haiku{{\textquoteleft}Ik heb gehoord van.}{het allerliefste kind dat}{ooit geboren is}\\

\haiku{De Koning reed voor,.}{dicht langs de stroomende beek}{starend en denkend}\\

\haiku{Hoog is de hemel,!}{zoo hoog als de hemel is}{de stem van mijn hart}\\

\haiku{het lachen dat zoo.}{groot is als heel de wereld}{en nooit kan vergaan}\\

\haiku{En zij klapten in.}{de handen en dansten met}{elkaar heel den dag}\\

\haiku{Maar hij lachte en.}{maakte het mooie geluid dat}{hij verzonnen had}\\

\haiku{X.   Zij zien het.}{meisje dat altijd vlucht en}{zoeken in het woud}\\

\haiku{{\textquoteright} {\textquoteleft}Neen, haar immers niet,{\textquoteright},,.}{lispelde een andere}{een blonde bedeesd}\\

\haiku{Dan komt zij om te.}{troosten over wat fee\"en en}{nimfen niet hebben}\\

\haiku{En zij lachten, maar.}{hij hoorde slechts haar stem en}{zij slechts de zijne}\\

\haiku{Zij zat daar, klein en,,.}{lief met haar hand op haar knie}{en ze zag mij aan}\\

\haiku{van wat je zelf bent,.}{geweest te weenen om de}{schoonheid van alles}\\

\haiku{Maar ik wist wel dat.}{hij weerzou komen op het}{veld onder den berg}\\

\haiku{De wereld werd groot,.}{en was geheel van hem die}{zoo goed is zoo lief}\\

\haiku{ik wil niet anders,, -?}{dan haar vinden haar alleen}{is zij de prinses}\\

\haiku{je wezen naar de.}{altijd nieuwe beelden die}{de wolken maken}\\

\haiku{Schoon was de rechte,.}{diepte zijner oogen schoon de}{hoogheid van zijn hoofd}\\

\haiku{De vreugde werd mij,}{zoo lief als mijn vader maar}{hem vond ik nergens}\\

\haiku{De prinses heb je,}{nooit gezien de wondere}{Psyche ken je niet}\\

\haiku{Haar, haar zelf had hij.}{vergeten toen hij dacht dat}{alles eender was}\\

\haiku{Het gefluister drong.}{diep in hem en daar in de}{diepte trilde iets}\\

\haiku{En toen zij aan den,.}{oever zaten zag hij zijn}{beeld hoe schoon hij was}\\

\haiku{{\textquoteright} Maar de oude vrouw.}{suste met haar vinger en}{knikte Reinbern toe}\\

\haiku{Ik weet nu dat je,.}{ook naar de prinses zoekt dat}{had ik vergeten}\\

\haiku{Daar, die donkere,.}{poort door misschien weet zij waar}{wij heen moeten gaan}\\

\haiku{Zij zetten zich daar,,.}{neder lang uit hun hoofden}{achterover geleund}\\

\haiku{ik verlang wel naar,,....}{brood ik heb honger maar ik}{zal het niet vragen}\\

\haiku{Toen zag hij in het.}{midden der zaal een blankheid}{waar Psyche in stond}\\

\haiku{{\textquoteleft}Laten we elkaar,.}{dan vasthouden dan kunnen}{wij niet verdwalen}\\

\haiku{Om de prinses te.}{zoeken hebben wij immers}{ook oogen voor den nacht}\\

\haiku{Wij weten zelfs niet,.}{waarom wij zoeken moeten}{waarom zij heen ging}\\

\haiku{En hij geloofde.}{niet dat het een heilige}{was die voor hem stond}\\

\haiku{En Reinbern begreep.}{op eens hoe goed en groot de}{heele wereld was}\\

\haiku{Blauw licht glansde daar,}{waar zij stond en het gelaat}{van Regel met wien}\\

\haiku{Maar zij hoorden haar,:}{en ook de antwoorden van}{Regel en Denkmar}\\

\haiku{Rein was verbaasd, want.}{hij had gedacht ook de elf}{bij zich te houden}\\

\haiku{Ik heb het, bij den,,.}{Bitsan het spotspook omdat hij}{grapjes kan maken}\\

\haiku{Toen hij nog zeer klein,.}{was had hij een zusje maar}{zij was heengegaan}\\

\haiku{Er ging een koele,,.}{wind geurig van teer van nieuw}{hout en van vruchten}\\

\haiku{daar is hetzelfde,{\textquoteright}, {\textquoteleft}}{als het allermooiste hier}{zeide het meisje}\\

\haiku{Ga nu wat slapen,,.}{jongen dan zal ik je van}{middag wel roepen}\\

\haiku{Eindelijk hoorden.}{zij weer beweging in het}{kamertje boven}\\

\subsection{Uit: Drie Hollandse romans}

\haiku{Hij keek beiden aan,.}{en hij wilde iets vragen}{maar hij wist niet wat}\\

\haiku{Dan bleef hij alleen.}{maar kijken naar de kuil die}{vol was gelopen}\\

\haiku{Toen hij hoorde dat}{zij met de postbode over}{het ijs zouden gaan}\\

\haiku{En hij luisterde,.}{zoals de grootmoeder las}{hij zag het voor zich}\\

\haiku{Maarten hoorde het,.}{en bad hij bleef nu staan tot}{het schieten ophield}\\

\haiku{Bij het scheiden kreeg*.}{Maarten menige}{klop op de schouder}\\

\haiku{men zag hem met de,.}{lange benen haastig gaan}{turend over het ijs}\\

\haiku{En elke dag had,.}{hij meer gehoord en eerder}{dan iemand anders}\\

\haiku{Telkens vroeg zij iets,.}{met haar kinderstem dan kwam}{haar adem aan zijn wang}\\

\haiku{Hij wist niet hoe hij,.}{het zeggen moest hij keek haar}{aan en zij wachtte}\\

\haiku{Dan liep hij nog een,.}{eind de dijk op de witte}{maan stond nevelig}\\

\haiku{Maarten lag wakker.}{lang nadat de torenklok}{twaalf had geslagen}\\

\haiku{En dat moest men maar.}{verdragen en werk zoeken}{voor het brood alleen}\\

\haiku{Hij schepte ieder,.}{de aardappelen op het}{bord zij aten zwijgend}\\

\haiku{Het is niets, vrouw, de.}{Heer slaat daar minder acht op}{dan op jou koffie}\\

\haiku{Op een morgen dat}{Rossaart aan de andere}{oever wachtte zag}\\

\haiku{De commandant was.}{een bejaarde kapitein}{van de schutterij}\\

\haiku{Op de hoek kwam hij,.}{zijn broer Hendrikus tegen}{die voor hem staan bleef}\\

\haiku{Bij de plecht stond de,.}{kolenpot het aardewerk}{hing er aan spijkers}\\

\haiku{De wijnkoper had.}{haar in huis genomen en}{zocht een dienst voor haar}\\

\haiku{Om godswil, jongen,*?}{heb ik jullie daarvoor dan}{geholpen}\\

\haiku{Ik hoop alles goeds,}{voor je daarginds maar kijk niet}{neer op je vrienden}\\

\haiku{hout, spijkers, verf gaf.}{Seebel hem in ruil voor zijn}{aandeel in de tjalk}\\

\haiku{daar staat de dood voor,,,.}{mij goed ik geef mij over er}{is niets aan te doen}\\

\haiku{Hij boog het hoofd en.}{staarde door de hor naar de}{nevel over de gracht}\\

\haiku{Hij staarde naar het,.}{ijs op de ruit hij dacht en}{schudde soms zijn hoofd}\\

\haiku{Hoe verder de schuit.}{voer zo meer wendde Marie}{het hoofd naar achter}\\

\haiku{Van toen aan merkten,.}{zij een verschil hoewel zij}{het niet beseften}\\

\haiku{Man, zeide zij, ik.}{zal erom huilen dat je}{alleen moet varen}\\

\haiku{- De steeg was nauw, hij;}{bukte laag om door de hor}{naar de lucht te zien}\\

\haiku{'t Is anders een,.}{hele tijd vijfenveertig}{jaar alleen te zijn}\\

\haiku{Hadden ze maar ja.}{gezegd toen ik je bij me}{in huis wou nemen}\\

\haiku{Geld was er genoeg,.}{je was heemraad geworden}{en allang dijkgraaf}\\

\haiku{er zullen er nog,,.}{heel wat verdrinken ik weet}{het zeker zei je}\\

\haiku{nee, dat geld leg ik,.}{opzij dan heeft hij wat meer}{als hij bij me komt}\\

\haiku{hij vroeg Rossaart of.}{zij haar samen wat met de}{post zouden zenden}\\

\haiku{Jij bent de enige,.}{van wie ik het zie maar er}{zijn ook anderen}\\

\haiku{Je wordt stijf in je,,,}{rug zeg je morgen kan je}{toch niet meer varen}\\

\haiku{Hij zat rechtop met,.}{haar hand in de zijne maar}{hij kon niet spreken}\\

\haiku{Het water klotste.}{tegen het boord en de schuit}{trok aan de touwen}\\

\haiku{Een man riep telkens:}{wanneer hij de kruiwagen}{grond had uitgestort}\\

\haiku{vroeg hij, het is toch.}{niet de eerste keer dat je}{in Hurwenen komt}\\

\haiku{een mens haast niets meer,.}{te kosten en zij nam geen}{geld meer van hem aan}\\

\haiku{Doe je plicht, dacht hij,,.}{dan en vraag niet het zal wel}{gegeven worden}\\

\haiku{, zij werd oud en hem.}{werd het soms te veel altijd}{alleen te varen}\\

\haiku{De hond, die aan zijn,.}{voeten stond schudde zich de}{sneeuw van de haren}\\

\haiku{Gelukkig dat het,,.}{er is ik heb angst gehad}{waarom weet ik niet}\\

\haiku{Maar voor het jaar ten.}{einde ging begon hij zich}{te verontrusten}\\

\haiku{Hij had met alle.}{schuldeisers gesproken en}{zich met hen verstaan}\\

\haiku{Gerbrand had noch de.}{toon noch de betekenis}{hiervan begrepen}\\

\haiku{Eens, toen zij opstond,:}{om naar boven te gaan keek}{hij op en zeide}\\

\haiku{Het huilen ging voort,.}{het was te horen in de}{andere winkels}\\

\haiku{en dat maakte hem.}{zo blij dat hij het hard in}{de armen drukte}\\

\haiku{Frans hief het hoog op.}{naar de bloesems zodat het}{met de ogen knipte}\\

\haiku{Ach broer, zou het niet?}{beter zijn hem met zachtheid}{te behandelen}\\

\haiku{tot hij het weer was.}{die sloeg en de zwakkeren}{de knikkers afnam}\\

\haiku{klagen dat Floris.}{haar kind geknepen had of}{de kleren gescheurd}\\

\haiku{Jongen, vroeg hij, weet?}{je niet dat een dief in de}{gevangenis komt}\\

\haiku{Hij sloeg de zijne,,.}{neer zijn lippen trilden maar}{hij kon niets zeggen}\\

\haiku{Toen Agnete even,:}{bij hem kwam staan om iets te}{vragen zeide hij}\\

\haiku{Dit moet hij voor de,.}{ogen houden en ons voorbeeld}{van rechte zeden}\\

\haiku{de meeste waren,.}{van rode steen maar ook die}{verschillend van kleur}\\

\haiku{En hun moeder keek,.}{ze altijd lachend aan trots}{en vol vertrouwen}\\

\haiku{Daar heeft zij toch ook,,.}{schuld aan zeide Stien en dat}{begreep Floris niet}\\

\haiku{Hij nam ze mee op,}{grote wandelingen tot}{voorbij Bennebroek}\\

\haiku{Dan hosten zij arm,,:}{aan arm schreeuwend tegen de}{mensen joelend van}\\

\haiku{Hij ging iedere.}{dag met de jongens tot de}{laatste zaterdag}\\

\haiku{Hij woelde, hij kon,.}{niet slapen buiten zongen}{nog kermisgangers}\\

\haiku{In de straat sprak zij,,:}{niet maar voorbij de brug waar}{niemand ging vroeg zij}\\

\haiku{Ik weet wat u voor*,.}{mij  ~          gedaan hebt ik}{ben u erg dankbaar}\\

\haiku{Zijn oom sloeg maar even,:}{de ogen op zette zijn bril}{recht en antwoordde}\\

\haiku{Het waren kleine,.}{hoge tonen klagend door}{de witte ruimte}\\

\haiku{Je neemt evengoed een.}{stuk koek uit de kast en dat}{is toch geen diefstal}\\

\haiku{Hij lag wakker, hij.}{besefte niet eens hoe de}{gedachten kwamen}\\

\haiku{Maar dan vroeg Jansje,.}{naar de oom in Hoorn of hij}{oud geworden was}\\

\haiku{En vergeet het niet,.}{een mens hoeft geen kwaad te doen}{als hij het niet wil}\\

\haiku{Maar over vijf, zes jaar.}{zou er geen vlek meer op de}{naam van Floris zijn}\\

\haiku{Zij zagen Werendonk,.}{die wees naar een barst in de}{muur boven het raam}\\

\haiku{Nee, daar kunnen wij,.}{niets aan doen er steekt meer kwaad}{in dan wij weten}\\

\haiku{En plotseling zweeg,,.}{zij met de hand voor de ogen}{of zij zich bedwong}\\

\haiku{Als je je verstand,.}{verliest ga dan naar je huis}{en denk erover na}\\

\haiku{Meer dan twintig jaar,.}{is zij hier geweest al voor}{je geboren was}\\

\haiku{En als stelen niet,.}{genoeg is dan zal ik nog}{wel wat anders doen}\\

\haiku{Werendonk rees, groot stond.}{hij voor Floris die week en}{de stoel liet vallen}\\

\haiku{Toen hij erheen ging.}{de eerste morgen voelde}{hij een verlichting}\\

\haiku{iedere avond hoor.}{ik dat gefluit en anders}{is het hier zo stil}\\

\haiku{Elke avond wanneer.}{hij langskwam stak zij het hoofd}{even uit het venster}\\

\haiku{de moeder was kort.}{zoals Wijntje en droeg een}{muts met keelbanden}\\

\haiku{Maar met volharding,,.}{in het geloof dacht hij wordt}{de ziel behouden}\\

\haiku{Kort en goed, wij zijn}{gekomen om tegen je}{vader te zeggen}\\

\haiku{Alleen het zingen.}{van de treurige liedjes}{kon zij niet laten}\\

\haiku{Kom, zeide zij, help,.}{mij maar liever wat stuk is}{wordt wel weer gemaakt}\\

\haiku{Zeg mij eens, wat heb?}{je op het hart dat je zo}{ongedurig bent}\\

\haiku{zou als er op zijn.}{kantoor een dief geweest was}{die niet gestraft was}\\

\haiku{Om elf uur vond hij.}{zijn broer aan de tafel met}{de Bijbel voor zich}\\

\haiku{Maar Werendonk wilde.}{geen ander in de winkel}{of bij hem aan bed}\\

\haiku{Zij was het die hem.}{zijn eten en melk moest brengen}{en de kamer doen}\\

\haiku{Frans stond op en gaf*,,.}{hem een  ~          hand zijn mond}{trok hij zeide niets}\\

\haiku{Wat zingt die Stien toch,,,?}{zeide hij honderduit waar}{heeft zij dat geleerd}\\

\haiku{Jongen, wij doen voor,.}{je wat wij kunnen maar er}{hindert je nog iets}\\

\haiku{Eens nam hij hem 's}{avonds mee naar de Bavo om}{hem te laten zien}\\

\haiku{En al voelde hij,.}{de moeheid hij stelde het}{uit naar huis te gaan}\\

\haiku{Hij zou navragen.}{wanneer er een boot vertrok}{en wat het kostte}\\

\haiku{Zij staarde naar hem,.}{zij zag hoe wit zijn gezicht}{was in de schemer}\\

\haiku{Alleen om weg te.}{komen uit dat huis had hij}{het geld genomen}\\

\haiku{Je zal toch niet zo,,}{dom zijn hem weer in huis te}{nemen zeide zij}\\

\haiku{als hij gek is laat.}{hem dan oppakken en doe}{hem in Meerenberg}\\

\haiku{Hij wachtte in het,.}{lamplicht aarzelend of hij}{naar hem toe zou gaan}\\

\haiku{Je moet niet denken,.}{dat ik om het brood kom maar}{ik moest in huis zijn}\\

\haiku{Zij stond vlugger op,.}{dan Stien zij was het die een}{stuk brood voor hem sneed}\\

\haiku{Dan komt pas de rust.}{in huis waar je recht op hebt}{met de oude dag}\\

\haiku{Die hem goed kenden.}{merkten dat de gedachten}{hem bezighielden}\\

\haiku{Waar hij keek zag hij,.}{de gezichten groter bleek}{met de ogen donker}\\

\haiku{En hij zocht aan de.}{muren naar een spoor waar de}{bliksem was gegaan}\\

\haiku{Hij zat te suffen,,.}{met hoofdpijn zoals altijd}{op zulke dagen}\\

\haiku{Toen hij weer naar de.}{stad kon gaan reisde juffrouw}{Amalia met hem mee}\\

\haiku{Naar de kerk wilde,.}{hij niet hij had gezegd dat}{men thuis kon bidden}\\

\haiku{Op een avond zette}{zij haar stoel dicht naast hem en}{terwijl zij sprak hield}\\

\haiku{Heiltje richtte zich,:}{op met een kreet van schrik zij}{werd bleek en zij riep}\\

\haiku{De redenen, vrouw,.}{die zal je zien zodra je}{verstand verlicht wordt}\\

\haiku{Nu heeft je man een.}{plicht te doen en die plicht heb}{je mee te dragen}\\

\haiku{Dat zij een hekel.}{aan hem hadden hoefde niet}{gezegd te worden}\\

\haiku{Zij keek naar Heiltje,,.}{die met de hals gestrekt haar}{zat aan te staren}\\

\haiku{Plotseling rees er.}{luide twist tussen man en}{vrouw en zij liep weg}\\

\haiku{Kleed je eens netjes,.}{aan en kom mee wij hebben}{met je te spreken}\\

\haiku{Het zal er dus van,.}{moeten komen al gaat het}{mij tegen het hart}\\

\haiku{Zij dankte Blok dat:}{hij haar weer geholpen had}{en bij zijn woorden}\\

\haiku{Dan kan ik bijna.}{zeker zijn dat het verkeerd}{gaat met allebei}\\

\haiku{Hij was moeilijk te,.}{bedaren maar eindelijk}{kon hij het zeggen}\\

\haiku{Zij moesten bekrimpen.}{en voor de winter hoopte}{hij op de bakteelt}\\

\haiku{Er  *~          kon een,.}{slechte tijd verwacht worden}{met krapheid in huis}\\

\haiku{Dikwijls was het stil.}{in zijn hoofd wanneer hij daar}{de druk niet voelde}\\

\haiku{Veel onkruid bleef er,.}{die zomer weg de aarde}{zag er schoner uit}\\

\haiku{Toch, eerlijker dan.}{zij het bedoelde in het}{gebed kon het niet}\\

\haiku{moest hij een deel van?}{zijn kracht nutteloos laten}{in de voldaanheid}\\

\haiku{Ze kijken alleen,.}{maar ze bespieden wat er}{in mijn hoofd omgaat}\\

\haiku{Het moet, daar kan je,.}{zeker van zijn ik zie dat}{het geschreven staat}\\

\haiku{De man heeft niets te.}{doen als piekeren en daar}{wordt het hoofd moe van}\\

\haiku{Toen het beter ging.}{was hij nog moeilijker van}{zijn plaats te krijgen}\\

\haiku{Ik dacht anders dat,.}{degene die beproefd wordt}{ervan verbetert}\\

\haiku{Waarom dan moest ik?}{ondankbaar worden voor wat}{mij gegeven werd}\\

\haiku{Als je denkt dat het,,.}{moet zeide hij kan ik je}{niet tegenhouden}\\

\haiku{Dan was zij niet zo.}{dom geweest haar zoon naar de}{slachtbank te sturen}\\

\haiku{Misschien wachten ze.}{daar al op me. Als mijn broer}{mij maar niet loslaat}\\

\haiku{Het kind vroeg toen of.}{haar vader niet even bij haar}{kon komen zitten}\\

\haiku{We pikken wat, we,,.}{zoeken wat we vinden wat}{we vliegen verder}\\

\haiku{Het was te dwaas te.}{denken dat de plaats daar iets}{mee te maken had}\\

\haiku{Ik was nog een kind,}{toen men zei dat het in zou}{storten zo oud is}\\

\haiku{Bid zoveel je wil,.}{maar waar de duivel woont zal}{het je niet helpen}\\

\haiku{Sofie ging er wel,.}{heen om hem toe te spreken}{maar het hielp niet veel}\\

\haiku{Wel was het moeilijk.}{zich te bedwingen en de}{deur voorbij te gaan}\\

\haiku{Blok kwam dichter bij.}{hem staan en merkte dat hij}{niet gedronken had}\\

\haiku{Met de dikke stok,,:}{die hij nu had langs de weg}{wandelend dacht hij}\\

\haiku{De knecht, menende,.}{dat zij vochten greep hem aan}{en rukte hem los}\\

\haiku{Ze denken dat ik.}{nog officier ben en in}{het hospitaal lig}\\

\haiku{, want behalve de.}{zieke en de werkvrouw was}{er niemand overdag}\\

\haiku{kleine ijssleden.}{die met prikstokken worden}{voortbewogen}\\

\haiku{de uitreiking van.}{onder andere voedsel}{aan de armen}\\

\haiku{omdat namelijk ().}{de Belgische Opstand1830}{was uitgebroken}\\

\haiku{de schipper op de...:}{beurtschuit.105die de wonderen}{en wetenschap had}\\

\haiku{en ga ten huize.}{uws broeders niet op den dag}{van uwen tegenspoed}\\

\haiku{het vierde gebod,;}{dat de heiliging van de}{zondagsrust voorschrijft}\\

\haiku{Voor het verwarmen.}{van de bak werd meestal}{paardenmest gebruikt}\\

\haiku{Vanaf 1870 was dat.}{gesticht gevestigd op de}{Stadhouderskade}\\

\haiku{de puntige stok.}{waarmee jonge gewassen}{gepoot worden}\\

\haiku{De betekenis.}{van deze passage is}{onduidelijk}\\

\haiku{Er wordt gespit en,.}{geschoffeld meel wordt gezeefd}{of turf verhandeld}\\

\haiku{Driemaal bezoekt hij.}{de Grote Kerk en tart hij}{God hem te straffen}\\

\haiku{Zijn geloof in en.}{angst voor een God der wrake}{is hij voorgoed kwijt}\\

\haiku{{\textquoteleft}Die gezindheid zou {\textquotedblleft}{\textquotedblright}{\textquoteright}.}{u ook kunnen vinden in}{Een zwerver verdwaald}\\

\haiku{Dan komt pas de rust.}{in huis waar je recht op hebt}{met de oude dag}\\

\haiku{Dat had hij gedaan,,{\textquoteright} ().}{niet anders daarom voer hij}{nu alleenp. 151}\\

\haiku{Aan het begin van:}{hoofdstuk vi van Een Hollands}{drama lezen we}\\

\haiku{Sporadisch begint.}{de directe rede met}{een kleine letter}\\

\haiku{het handschrift van De:}{waterman heeft op deze}{plaats namelijk}\\

\haiku{Driemaal wordt dit, steeds, {\textquoteleft}{\textquoteright}.}{in de directe rede}{geschreven alsjou}\\

\haiku{Het is niets, vrouw, de.}{Heer slaat daar minder acht op}{dan op jou koffie}\\

\subsection{Uit: Fratilamur}

\haiku{Hoewel ik hem niet.}{gezegd had waar ik woonde}{verwachtte ik hem}\\

\haiku{Ik was elf jaar, een,.}{kind dat weinig kreeg voor den}{mond noch voor het hart}\\

\haiku{Ik voelde dat ik.}{grooter was en voor eeuwig}{bevrijd van een band}\\

\haiku{Het was zomer, stil,.}{alleen ging soms achter mij}{een locomotief}\\

\haiku{Zij had alles dat,.}{verblinden kan want alles}{scheen wat het niet was}\\

\haiku{Brandt voor het onrecht,.}{bidt om genade voor wie}{je je vrienden noemt}\\

\haiku{Roep en smeek tot er.}{zijn die je alles opendoen}{behalve hun ziel}\\

\haiku{Ik verliet het dorp.}{en dwaalde in een stille}{stad een zomer lang}\\

\haiku{Je zou zelf je pijp.}{gemaakt hebben en in den}{kring kunnen zitten}\\

\haiku{Zooals een kind dat zijn.}{zin niet gehad heeft bleef ik}{turen over de zee}\\

\haiku{De stad wemelde.}{van menschen die er voor hun}{genoegen kwamen}\\

\haiku{Ik antwoordde dat,.}{ik dit was hoewel niet in}{Holland geboren}\\

\haiku{dit oogenblik, van,.}{hem  voor mij alleen heeft}{mij opgeheven}\\

\subsection{Uit: Jan Compagnie}

\haiku{Jan de Brasser was.}{Amsterdammer omdat hij}{hier had leeren loopen}\\

\haiku{De koster lachte,.}{waarop Ras kwaad werd en hem}{voor huichelaar schold}\\

\haiku{Op hem lag de taak.}{den ouderdom van moeder}{en oom te steunen}\\

\haiku{Daar hij liever in.}{een warm land wilde wonen}{koos hij voor Indi\"e}\\

\haiku{De directeur prees.}{hem en voorspelde dat hij}{het ver zou brengen}\\

\haiku{De kapitein had.}{geen anderen bij stand dan}{van den lanspassaat}\\

\haiku{In Jacatra had.}{hij zich onderscheiden door}{moed en bekwaamheid}\\

\haiku{Maartensz teekende.}{het briefje en ontving het}{geld voor het vervoer}\\

\haiku{De kwartiermeester.}{liet de kisten stouwen met}{nog twee andere}\\

\haiku{Bij Andries in huis,.}{zeide hij dat hij vriend was}{maar ook korporaal}\\

\haiku{Een uur later zei.}{hij terloops dat Manilla}{niet ver weg kon zijn}\\

\haiku{De eenige die Ai,,.}{kon helpen een klein Engelsch}{schip voer voor hen weg}\\

\haiku{Maar een anderen,,.}{keer bij het herhaald verzoek}{beraadslaagden zij}\\

\haiku{De droppels van den.}{regen spatten van den grond}{op of het bloed was}\\

\haiku{Ai wonen noch op.}{een andere plaats waar de}{Compagnie gebood}\\

\haiku{Hij las de brieven.}{van huis die alle zegen}{en welvaart meldden}\\

\haiku{met bewogen stem,.}{noemde hij hem een deugdzaam}{een edelmoedig heer}\\

\haiku{Allen klonken en.}{Fonseca noodigde allen}{voor het bruiloftsmaal}\\

\haiku{En als het werk goed.}{gedaan werd achtte hij niet}{of het langzaam ging}\\

\haiku{Hij had zooveel zwart,.}{volk vervoerd dat hij ze voor}{enkel koopwaar hield}\\

\haiku{De vreugde onder.}{het pakken en redderen}{duurde korten tijd}\\

\haiku{Toen scheepte hij zich.}{met hen in op het jacht dat}{naar Ambon vertrok}\\

\haiku{Water hadden zij,.}{uit de geulen hout voor het}{drogen uit het bosch}\\

\haiku{Yen Pon de winst in,.}{goud doen wegen drie kistjes}{elk van dertig pond}\\

\haiku{Van de raden van.}{Indi\"e had er maar een wat}{geld overgehouden}\\

\haiku{Hij had geen tijd voor.}{onbenulligheden en}{verliet het kantoor}\\

\haiku{Er was veel vermaak,.}{met de vischnetten de tafels}{stonden ruim voorzien}\\

\haiku{Ai doen vervoeren.}{om nootmuskaat te plukken}{voor de Compagnie}\\

\haiku{De klokken waren.}{maar zwak te hooren in het}{geknal en gejuich}\\

\haiku{zij bestemd waren,.}{gaf verlof ze dienzelfden}{dag uit te laden}\\

\subsection{Uit: Jeugdherinneringen}

\haiku{De andere was,,.}{een jongen groter dan ik}{Bantoet heette hij}\\

\haiku{Er zou nog meer te.}{vertellen zijn uit die tijd}{in het Florapark}\\

\haiku{Eens is het gebeurd,.}{dat ik twee dagen niets heb}{gehad ook geen brood}\\

\haiku{Na school wachtte ik.}{haar op en ik vroeg of ik}{haar tas mocht dragen}\\

\haiku{Eens zei ze dat ze.}{thuis verboden hadden dat}{ik met haar meeliep}\\

\haiku{Voor jongens van een.}{bekende familie had}{hij een gunstig oog}\\

\haiku{Het geld kreeg ik van,.}{de tante Quentin over wie}{ik straks vertel}\\

\haiku{Het was op een dag.}{een verrassing toen Johan zei}{dat  hij ook schreef}\\

\haiku{Ik had haar weinig.}{gekend en er was nooit meer}{over haar gesproken}\\

\haiku{Dat deed ik, ik bracht '.}{hems zaterdags zijn tien}{gulden salaris}\\

\haiku{Het heeft ook niet lang,,.}{geduurd hij stierf in april even}{vijfentwintig jaar}\\

\haiku{En kijk eens hier, zei,,.}{hij nog dit kasboek klopt ook}{niet telt u maar op}\\

\haiku{Ik heb haar pas uit.}{het oog verloren toen ik}{naar Engeland ging}\\

\haiku{Maar ook ons bedrag.}{van tweehonderd gulden was}{een meevaller}\\

\subsection{Uit: Verhalen}

\haiku{Ik werd vriendelijk}{jegens het paard en vond het}{aangenaam te zien}\\

\haiku{Nogmaals faalde ik;}{den vogel te grijpen en}{haalde nu mijn boog}\\

\haiku{De honden stonden.}{woedend te blaffen naar den}{vogel die heenging}\\

\haiku{en van den tijd waar;}{wij in leven dat die vol}{geheimenis is}\\

\haiku{het water in den;}{stroom daar beneden had een}{goedhartigen klank}\\

\haiku{Wanneer Alan kwam zag;}{hij telkens hoe zij gegroeid}{en veranderd was}\\

\haiku{Ik hoor het gaarne.}{want hun geluid is muziek}{bij mijn gemijmer}\\

\haiku{men vond ze ook een.}{enkel keer bij het kruis dat}{aan den afgrond staat}\\

\haiku{Dienzelfden middag,.}{verliet zij het gehucht naar}{het zuiden gaande}\\

\haiku{Toen ging er geen dag.}{zonder tranen en troost en}{lange gepeinzen}\\

\haiku{zij zag haar moeder.}{die altoos ernstig naast haar}{ging op de tochten}\\

\haiku{Aldebrand die de.}{kreten hoorde trad buiten}{en knipte zijn oogen}\\

\haiku{Toen hij zijn dochter}{zag viel menig man in het}{gedrang ter aarde}\\

\haiku{De veldheer trad in,.}{de zaal gevolgd door zijn zoon}{en een toortsdrager}\\

\haiku{Slechts het gelaat aan.}{het kruis waar het lampje voor}{brandde zag haar aan}\\

\haiku{Met een kreet trad hij,.}{nader maar snel rees zij en}{vluchtte de zaal uit}\\

\haiku{De jonkman wees en,:}{hief zijn armen groot waren}{zijn luide woorden}\\

\haiku{Nu zou gewis de.}{vijand wijken en het}{land gelukkig zijn}\\

\haiku{In den dageraad,,}{fonkelend van dauw zag zij}{bij het ontwaken}\\

\haiku{De weg steeg en die.}{laatste inspanning werd te}{zwaar voor de Vlaamsche}\\

\haiku{een overste kon niet,.}{gemist worden zij zelve}{moest aanvoerder zijn}\\

\haiku{of het de haat was.}{die het ongeluk bracht of}{de straf des hemels}\\

\haiku{Toen wisten beiden.}{dat Ermonne een van haar}{zoons verliezen moest}\\

\subsection{Uit: Verzameld werk. Deel 1}

\haiku{De vrouwen gingen.}{meer dan ooit ter kerke voor}{hun mannen bidden}\\

\haiku{Waarom kunnen de?}{honden blaffen en janken}{en kunt gij het niet}\\

\haiku{Ga en bid op de.}{berg Kalvari\"e voor uw heil}{en bid voor uw kind}\\

\haiku{Het onderwijs der;}{beide zoons was toevertrouwd}{aan de kapelaan}\\

\haiku{, zijn deugdzaam zwaard kon.}{hij schier niet meer tillen met}{de lome armen}\\

\haiku{wat mijn geluk kan,?}{worden behoort een ander}{mag ik het nemen}\\

\haiku{{\textquoteright} {\textquoteleft}In 't graf, jongen,,,.}{gaat hij nog trager maar hij}{staat niet stil goddank}\\

\haiku{De lafhartigheid.}{Ermgarde te krenken was}{verre van hem}\\

\haiku{Gij zult toch wel eens '?}{een vrouw ontmoet hebben die}{gijt liefste hadt}\\

\haiku{{\textquoteright} En zij maakte een.}{gebaar of zij zijn stoel tot}{zich trekken wilde}\\

\haiku{Het regende nog;}{toen hij die morgen voor de}{hut van Karo kwam}\\

\haiku{De kapelaan zocht,;}{hem in milder stemming te}{brengen doch vergeefs}\\

\haiku{toen keek de edelvrouw.}{om en zette zich weer kalm}{aan het borduurwerk}\\

\haiku{Sta, sta op, zeg ik,,...!}{u vlied van hier zoek of ge}{uw graf kunt vinden}\\

\haiku{{\textquoteright} Hand in hand zochten,.}{zij het pad voorzichtig in}{het duister tastend}\\

\haiku{Hij werd grauw van iets,.}{angstigs dreigends dat over hem}{neerkwam als een stolp}\\

\haiku{{\textquoteleft}Gauw, gauw, Drogon! 't,,.}{Volk komt de monniken zij}{willen u doden}\\

\haiku{het water in de;}{stroom daar beneden had een}{goedhartige klank}\\

\haiku{Wanneer Alan kwam zag;}{hij telkens hoe zij gegroeid}{en veranderd was}\\

\haiku{dit leven was niet:}{te dulden en thans was ik}{er diep van overtuigd}\\

\haiku{Nogmaals faalde ik;}{de vogel te grijpen en}{haalde nu mijn boog}\\

\haiku{De honden stonden.}{woedend te blaffen naar de}{vogel die heenging}\\

\haiku{en van de tijd waar;}{wij in leven dat die vol}{geheimenis is}\\

\haiku{Helaas, waarom wil?}{je niet je hele leven}{bij mij doorbrengen}\\

\haiku{Maar ik vond je lief;}{voor je mij gezegd had dat}{je me beminde}\\

\haiku{Dat ledige en.}{die gevoelloosheid zouden}{mij gans niet lijken}\\

\haiku{men heeft me hier kort;}{geleden poortresse van}{dit klooster gemaakt}\\

\haiku{De wind, die zostraks door,.}{de schoorsteen gierde was van}{lieverlee bedaard}\\

\haiku{Zij bedachten, dat.}{het laat was en dat zij naar}{de stad moesten keren}\\

\haiku{En Rogier zat weer.}{alleen in de vensterbank}{met gebogen hoofd}\\

\haiku{De straten waren,.}{leeg maar Carolus voorop}{was op zijn hoede}\\

\haiku{hij had zich nooit om.}{de mensen bekommerd en}{nooit verdriet gehad}\\

\haiku{Het lot moet zijn loop,.}{hebben en het is een dwaas}{die het wil weren}\\

\haiku{Eindelijk zag hij:}{op met een klare gloed in}{de ogen en zeide}\\

\haiku{Eens toen hij die weg}{besteeg bemerkte hij dat}{het zeer donker werd}\\

\haiku{Een poosje gingen;}{zij zonder spreken voort met}{gebogen hoofden}\\

\haiku{Hij boog er zich even,.}{over knikte dan tot Mevena met}{goedhartige blik}\\

\haiku{Mevena bezon zich, dat;}{de monnik heen zou gaan nu}{zij bij Rogier was}\\

\haiku{dan richtte hij zich:}{tot zijn volle grootte op}{en voegde erbij}\\

\haiku{de een was een man,}{bedrogen door zijn vrouw de}{ander een moeder}\\

\haiku{Lugina klopte.}{hem vertrouwelijk op de}{schouder en ging heen}\\

\haiku{Hij begreep, dat zij.}{alles gehoord had wat hij}{zopas vertelde}\\

\haiku{{\textquoteleft}Rogier mocht van de,.}{keizer niet heengaan daarom}{wacht ik hier op hem}\\

\haiku{Hij ontwaakte door;}{een hevige slagregen}{en stond verschrikt op}\\

\haiku{Zij stonden zwijgend,.}{op de weg in de schemer}{der hoge bomen}\\

\haiku{hij staan, glurend door.}{de bladeren over de weg}{naar rechts en naar links}\\

\haiku{weer sloeg de ene, de,.}{zware klok de kleinere}{herhaalde de galm}\\

\haiku{Toen haar vrees verzwond;}{zuchtte zij en bemerkte}{hoe hij haar aanzag}\\

\haiku{Het water spoelde,.}{tegen het roer het ijzer}{piepte geregeld}\\

\haiku{Omtrent de middag,.}{trad Seffe binnen de schelm}{was ietwat dronken}\\

\haiku{Er was iets dat hem,.}{verlicht deed ademen hij had}{met de stad gedaan}\\

\haiku{Daar zat Maluse,.}{in het licht van het venster}{de hond lag er ook}\\

\haiku{Tamalone, zijn,;}{hoofd heffend antwoordde dat}{hij Polein zocht}\\

\haiku{Hij keerde zich om.}{uit het licht en liep weer door}{de gang naar de straat}\\

\haiku{Toen ook Simon aan,}{boord was riep Meron Joseph}{zijn bevelen uit}\\

\haiku{Ik hoor het gaarne.}{want hun geluid is muziek}{bij mijn gemijmer}\\

\haiku{Die grijsaard moet een,.}{tovenaar geweest zijn als}{hij niet erger was}\\

\haiku{Diezelfde middag,.}{verliet zij het gehucht naar}{het zuiden gaande}\\

\haiku{Toen ging er geen dag.}{zonder tranen en troost en}{lange gepeinzen}\\

\haiku{De pest, de kwaal wier,.}{naam zacht wordt uitgesproken}{heerste toen in Stratford}\\

\haiku{Dan renden allen,.}{naar huis zoals de liefste}{naar de liefste gaat}\\

\haiku{hoe de kauwen het;}{gevaar vergeten wanneer}{het graan wordt gemaaid}\\

\haiku{Zijn moeder bracht het;}{apostellepeltje voor}{haar eerste kleinkind}\\

\haiku{Als men niet zijn kan,.}{zoals zijn ideaal moet men}{zijn zoals men is}\\

\haiku{Het heerlijkst schouwspel.}{dat ooit verbeelding schiep zou}{daar gebeuren}\\

\haiku{De binnenplaats van:}{de herberg waar zij vroeger}{speelden gaf het plan}\\

\haiku{Zij waren beiden,.}{opgewekte mannen vol}{hoop en goede moed}\\

\haiku{Hij kende thans een.}{inniger wellust dan die}{van louter vormen}\\

\haiku{Dat zag Jacobus:}{duidelijk toen hij een paar}{jaar later zeide}\\

\haiku{Maar hij was te lomp,,,.}{te stug en zeiden zijn}{vrienden te lelijk}\\

\haiku{Will behoefde rust,,}{zij hadden het lang gezien}{het zou hem goed doen}\\

\haiku{Berowne, die hij,.}{als zichzelf had gekend was}{thans een vreemdeling}\\

\haiku{toen dwingelanden;}{arm werden en geringen}{rechten verkregen}\\

\haiku{Vlak achter het huis,:}{dicht bij de open venstertjes}{lagen de bedden}\\

\haiku{- So man and man should{\textquoteright},.}{be antwoordde zij die aan}{het hof had geleefd}\\

\haiku{these are flowers,}{Of middle summer and I}{think they are given}\\

\haiku{New Place echter.}{en het huis in Henley Street}{bleven ongedeerd}\\

\haiku{Een zoeker was hij,,,.}{een minnaar een dichter en}{hij schreef een treurspel}\\

\haiku{En toch, heerlijk dat!}{er zoveel verschil is ook}{waar zielen spreken}\\

\haiku{Waarlijk groot is het,.}{te strijden als het moet zelfs}{om een nietigheid}\\

\haiku{Een ander, zoals,:}{Kent draagt duldzaam de smarten}{die zij bed\'elen}\\

\haiku{knaves, thieves, and,;}{treachers by spherical}{predominance}\\

\haiku{Nooit daarvoren en;}{nooit daarna heeft hij zulke}{monsters voortgebracht}\\

\haiku{Dit is de kleding.}{volgens het borstbeeld boven}{het graf in de kerk}\\

\haiku{De Berg van dromen,}{I De knaap het meisje en}{Peter gaan op reis}\\

\haiku{hij had het besluit,.}{genomen zij had gezegd}{dat zij mee zou gaan}\\

\haiku{Maar het is al laat,.}{en als ze niet komt weet ik}{niet wat ik doen zal}\\

\haiku{Misschien is het lang,.}{geleden misschien is het}{zo\"even gebeurd}\\

\haiku{Toen blies hij op zijn.}{hoorn en  vroeg met grote}{stem hoe zij heetten}\\

\haiku{Zij waren omtrent:}{honderd schreden gegaan toen}{Puikebest zeide}\\

\haiku{Toen hij gedaan had:}{stak hij het boekje weer in}{zijn zak en zeide}\\

\haiku{{\textquoteright} riep hij onder het.}{schateren naar de mond van}{de knaap wijzende}\\

\haiku{, en Denkmar zou het.}{niet aardig vinden als je}{hem anders noemde}\\

\haiku{{\textquoteright} Toen liep Puikebest,.}{hard naar het rozenpoortje}{gevolgd door Kaka}\\

\haiku{Zij kwamen in een,.}{veld van blauwe bloemen het}{ganse veld was blauw}\\

\haiku{{\textquoteright} Er klonk een gil en}{een lach en toen verdween er}{iets snel uit de boom}\\

\haiku{{\textquoteright} Zij lachte zachtkens.}{of er een bron murmelde}{in de nabijheid}\\

\haiku{Budde zat daar met,.}{zijn hoofd in zijn handen een}{oud kaboutertje}\\

\haiku{zie, gij strijdt tegen,.}{onze vrienden niet tegen}{onze vijanden}\\

\haiku{{\textquoteleft}Vreemdeling, Abel is,.}{uw naam slechts uw naam kennen}{wij en anders niet}\\

\haiku{En de stilte klinkt.}{en de zon hoort mijn roep en}{de wereld ontwaakt}\\

\haiku{Tobias kraaide,.}{zo plotseling en zo luid}{dat allen schrikten}\\

\haiku{{\textquoteright} Maar Tobias met:}{het hart van zonlicht verhief}{zijn schallende stem}\\

\haiku{Wij vogels reizen,{\textquoteright},.}{veel sprak Peregrijn de eend}{recht voor zich ziende}\\

\haiku{{\textquoteright} Zij stonden op en,.}{gingen hun schreden klonken}{in de grote zaal}\\

\haiku{Hij was zo moede.}{dat hij op de peluw bij}{de nachtwacht neerzeeg}\\

\haiku{Eindelijk legde.}{Corinna zachtkens haar}{hand op zijn schouder}\\

\haiku{dat moest je niet weer,.}{doen want nu heb ik nog}{maar \'e\'en gouden ring}\\

\haiku{Er ging een hele.}{tijd voorbij zonder dat wij}{iets van oom hoorden}\\

\haiku{{\textquoteright} Reinbern dacht aan de,.}{Prinses waar en wanneer hij}{haar weder zou zien}\\

\haiku{Maar jullie zijn toch}{domme ganzen dat je niet}{eerder verteld hebt}\\

\haiku{{\textquoteright} {\textquoteleft}Vertel het, Morgan,.}{als je goed nieuws hebt geven}{wij je een meloen}\\

\haiku{En zo werkten we.}{bij ploegen de hele nacht}{door tot het dag werd}\\

\haiku{Ik werd wakker - maar,,.}{in mijn droom want ik had maar}{gedroomd dat ik sliep}\\

\haiku{Maar deze nacht was,.}{zij in slaap gevallen dat}{was nog nooit gebeurd}\\

\haiku{{\textquoteleft}Ik heb gehoord van.}{het allerliefste kind dat}{ooit geboren is}\\

\haiku{Hoog is de hemel,!}{zo hoog als de hemel is}{de stem van mijn hart}\\

\haiku{het lachen dat zo.}{groot is als heel de wereld}{en nooit kan vergaan}\\

\haiku{Maar hij lachte en.}{maakte het mooie geluid dat}{hij verzonnen had}\\

\haiku{{\textquoteright} {\textquoteleft}Neen, haar immers niet,{\textquoteright},,.}{lispelde een andere}{een blonde bedeesd}\\

\haiku{Dan komt zij om te.}{troosten over wat fee\"en en}{nimfen niet hebben}\\

\haiku{En zij lachten, maar.}{hij hoorde slechts haar stem en}{zij slechts de zijne}\\

\haiku{Zij zat daar, klein en,,.}{lief met haar hand op haar knie}{en ze zag mij aan}\\

\haiku{van wat je zelf bent,.}{geweest te wenen om de}{schoonheid van alles}\\

\haiku{Maar ik wist wel dat.}{hij weer zou komen op het}{veld onder de berg}\\

\haiku{De wereld werd groot,.}{en was geheel van hem die}{zo goed is zo lief}\\

\haiku{ik wil niet anders, -?}{dan haar vinden haar alleen}{is zij de Prinses}\\

\haiku{je wezen naar de.}{altijd nieuwe beelden die}{de wolken maken}\\

\haiku{Schoon was de rechte,.}{diepte zijner ogen schoon de}{hoogheid van zijn hoofd}\\

\haiku{Sibylle en {\textquoteleft},!}{de schone JongelingNiet}{haar beeld maar haar zelf}\\

\haiku{Haar, haar zelf had hij.}{vergeten toen hij dacht dat}{alles eender was}\\

\haiku{Het gefluister drong.}{diep in hem en daar in de}{diepte trilde iets}\\

\haiku{En toen zij aan de,.}{oever zaten zag hij zijn}{beeld hoe schoon hij was}\\

\haiku{{\textquoteright} Maar de oude vrouw.}{suste met haar vinger en}{knikte Reinbern toe}\\

\haiku{Ik weet nu dat je,.}{ook naar de Prinses zoekt dat}{had ik vergeten}\\

\haiku{Daar, die donkere,.}{poort door misschien weet zij waar}{wij heen moeten gaan}\\

\haiku{Zij zetten zich daar,,.}{neder languit hun hoofden}{achterover geleund}\\

\haiku{ik verlang wel naar,,...}{brood ik heb honger maar ik}{zal het niet vragen}\\

\haiku{{\textquoteleft}Laten we elkaar,.}{dan vasthouden dan kunnen}{wij niet verdwalen}\\

\haiku{Om de Prinses te.}{zoeken hebben wij immers}{ook ogen voor de nacht}\\

\haiku{Wij weten zelfs niet,.}{waarom wij zoeken moeten}{waarom zij heenging}\\

\haiku{En hij geloofde.}{niet dat het een heilige}{was die voor hem stond}\\

\haiku{En Reinbern begreep.}{opeens hoe goed en groot de}{hele wereld was}\\

\haiku{Maar zij hoorden haar,:}{en ook de antwoorden van}{Regel en Denkmar}\\

\haiku{Rein was verbaasd, want.}{hij had gedacht ook de elf}{bij zich te houden}\\

\haiku{Ik heb het, bij de,,.}{Bitsan het spotspook omdat hij}{grapjes kan maken}\\

\haiku{Hoe kon hij verder,.}{hoe kon hij hoger gaan dan}{de top van de Berg}\\

\haiku{En de Koning sprak,:}{dat klonk als windgeruis in}{warme zomeravond}\\

\haiku{Toen hij nog zeer klein,.}{was had hij een zusje maar}{zij was heengegaan}\\

\haiku{Er ging een koele,,.}{wind geurig van teer van nieuw}{hout en van vruchten}\\

\haiku{daar is hetzelfde,{\textquoteright}, {\textquoteleft}}{als het allermooiste hier}{zeide het meisje}\\

\haiku{Eindelijk hoorden.}{zij weer beweging in het}{kamertje boven}\\

\haiku{Hun wijzen menen;}{dat de mens geen taak heeft dan}{de smart te ontgaan}\\

\haiku{geen mens te doden,;}{slechts onwetenden achten}{het naastenbloed niet}\\

\haiku{Zij reden verder,.}{om in Bethel te vernachten}{dichter bij de stad}\\

\haiku{Wat in mij is dringt,,.}{zeide zij breng mij waar ik}{neder kan liggen}\\

\haiku{De dwingeland van;}{Zion had met het zwaard zijn}{zonen omgebracht}\\

\haiku{IV In Nazareth.}{woonden zij in de stilte}{beneden de berg}\\

\haiku{De krekels zongen,.}{aan alle kant de sterren}{schitterden boven}\\

\haiku{Achter de leerschool,,;}{was een gaarde van vijgen}{granaten amandels}\\

\haiku{maar toen Judas van,?}{Golan voor de joden vocht}{waar was Micha\"el}\\

\haiku{De Jetser, dat is,.}{de kwade neiging in u}{dat is de boosheid}\\

\haiku{Zo stond Jezus in:}{hun midden en zag alles}{aan hoe zij dachten}\\

\haiku{Aan de grootheid die,.}{zijn makker meende dacht hij}{niet zij was te klein}\\

\haiku{De hemel straalde,:}{van nieuwe klaarheid de wind}{zong van bevrijding}\\

\haiku{En het moede lijf.}{lag neder en groot verrees}{de zon over zijn slaap}\\

\haiku{de Eeuwige uw.}{God zult gij aanbidden en}{hem alleen dienen}\\

\haiku{En Jezus keerde.}{naar de woning der ouders}{en zat in stilte}\\

\haiku{De nacht was warm en,.}{schoon de sterren schitterden}{toen hij binnentrad}\\

\haiku{Meester, mijn slaaf die.}{mij zeer dierbaar is ligt thuis}{in zware pijnen}\\

\haiku{Zie toch en help mij,.}{opdat ik niet zal moeten}{bedelen om brood}\\

\haiku{Alleen het hart zal,.}{hem zien en het reine hart}{is boven alles}\\

\haiku{Maar wie u op de,;}{rechterwang slaat keer hem ook}{de andere toe}\\

\haiku{En uw Vader, die,.}{in het verborgen ziet zal}{het u vergelden}\\

\haiku{En uw Vader, die,.}{in het verborgen ziet zal}{het u vergelden}\\

\haiku{Wie van u kan door?}{bezorgd te zijn \'e\'en el tot}{zijn lengte toedoen}\\

\haiku{En Jezus ging voort.}{in hun midden tot aan de}{oever van het meer}\\

\haiku{Laat ons overvaren.}{naar de eenzamen aan de}{andere oever}\\

\haiku{Verlangde zij de?}{vreugde van het geven niet}{meer dan andere}\\

\haiku{En niet \'e\'en musje.}{valt op de aarde zonder}{de wil uws Vaders}\\

\haiku{Wanneer zij u in,;}{de ene stad vervolgen vliedt}{naar de andere}\\

\haiku{Toen gij een kind waart,,.}{toen hadt gij al de wereld}{lief tot uzelve toe}\\

\haiku{hij ging voort tot zijn.}{getrouwen en zat aan de}{oever en leerde}\\

\haiku{Alle plant die mijn.}{Vader niet geplant heeft zal}{uitgeroeid worden}\\

\haiku{En zij kwam voor hem,:}{en viel neder en zij hield}{aan en jammerde}\\

\haiku{de vogels waren.}{daar roerig en stegen in}{zwermen uit het riet}\\

\haiku{Nu dan daar het heil,.}{was en hier de poort gebouwd}{zo moest zij openen}\\

\haiku{Zij zagen enkel,,.}{het licht maar Jezus niet en}{het licht was over hen}\\

\haiku{En hij leerde hun}{aangaande de voorzegging}{dat eerst Elia komen}\\

\haiku{Een jonge man daar,,:}{in zijn binnenste geroerd}{trad voor hem en vroeg}\\

\haiku{Ook een man van de,.}{tempel kwam daar langs en zag}{hem en ging voorbij}\\

\haiku{Die voorbijgingen,:}{geboden hem te zwijgen}{maar luider riep hij}\\

\haiku{Ja, zeg ik u, een.}{iegelijk die heeft die zal}{gegeven worden}\\

\haiku{Heer, trekt gij u dat?}{niet aan dat mijn zuster mij}{alleen laat dienen}\\

\haiku{Doch Martha deed stil,.}{haar werk daar het dienen haar}{liefste vreugde was}\\

\haiku{Heer, wij weten niet,?}{waar gij heen gaat hoe zullen}{wij de weg kennen}\\

\haiku{Maar in het midden:}{der olijfgaarde stond Jezus}{weder stil en sprak}\\

\haiku{Een stem riep de naam,.}{van Jezus een andere}{begon te klagen}\\

\haiku{Anderen heeft hij,.}{verlost maar  zich zelf kan}{hij niet verlossen}\\

\haiku{Maar de kinderen.}{groeiden en schreiden gelijk}{de ouders schreiden}\\

\haiku{Tijdschriftpublikatie,,,,-.}{in Nederland deel II 1904}{jg. 56 blz. 369375}\\

\subsection{Uit: Verzameld werk. Deel 2}

\haiku{zij zag haar moeder.}{die altoos ernstig naast haar}{ging op de tochten}\\

\haiku{Aldebrand die de.}{kreten hoorde trad buiten}{en knipte zijn ogen}\\

\haiku{Toen hij zijn dochter}{zag viel menig man in het}{gedrang ter aarde}\\

\haiku{De veldheer trad in,.}{de zaal gevolgd door zijn zoon}{en een toortsdrager}\\

\haiku{Slechts het gelaat aan.}{het kruis waar het lampje voor}{brandde zag haar aan}\\

\haiku{Met een kreet trad hij,.}{nader maar snel rees zij en}{vluchtte de zaal uit}\\

\haiku{Eens twistten zij, en.}{sedert zag men de broeders}{zelden te zamen}\\

\haiku{De jonkman wees en,:}{hief zijn armen groot waren}{zijn luide woorden}\\

\haiku{Evor zweeg en legde,.}{een hand op zijn schouder goed}{en zwaar en ernstig}\\

\haiku{In de dageraad,,}{fonkelend van dauw zag zij}{bij het ontwaken}\\

\haiku{De weg steeg en die.}{laatste inspanning werd te}{zwaar voor de Vlaamse}\\

\haiku{een overste kon niet,.}{gemist worden zij zelve}{moest aanvoerder zijn}\\

\haiku{Dat was het waarvoor,.}{Cintia uitging gisteren}{en vanmiddag weer}\\

\haiku{Ik kan u nog in.}{vertrouwen zeggen dat gij}{hier vergeefs komt}\\

\haiku{je boete doet en}{je op laat sluiten in het}{penitentenhuis}\\

\haiku{Nog eenmaal, Mira,.}{kom ik vragen of je het}{overwogen hebt}\\

\haiku{Maar de broeders van;}{San Marco namen het als}{een welkom wapen}\\

\haiku{Het huis behoort je,.}{als je wilt de stad zal je}{eerbiedigen}\\

\haiku{Borso helpt misschien,,.}{wie weet hij is de neef van}{Bongardo en zijn vriend}\\

\haiku{Of zou het zijn dat?}{er voor ieder vuur maar \'e\'en}{hand is die het dooft}\\

\haiku{zou het zijn dat wie,?}{het hier niet  vindt het toch}{eenmaal vinden moet}\\

\haiku{Zij is de moeder.}{die ik verloren had en}{liever dan gij weet}\\

\haiku{zoals dat meisje.}{en laat mij twintig woorden}{met haar spreken}\\

\haiku{mira Ja, dat wil,.}{ik en laat mij dan de rust}{die ik nodig heb}\\

\haiku{ciprian De dwaasheid,,.}{die vrouwen maken heer is}{licht als kinderspel}\\

\haiku{serafina Ja,,.}{ik begrijp het zij was schoon}{en daarom ging je}\\

\haiku{Als het waar mocht zijn,.}{geloof niet dat ik haar heb}{teruggebracht}\\

\haiku{Tegen zulk gerucht.}{kan men niet beter doen dan}{het uit de weg gaan}\\

\haiku{Mijn naam zal hem dan.}{waarborg zijn dat die vrouw geen}{last veroorzaakt}\\

\haiku{Het misverstand van,.}{uw verondersteld geval}{begint al schijnt het}\\

\haiku{Ik heb nu weinig,.}{tijd maar ik liep even binnen}{om te waarschuwen}\\

\haiku{De hoop van zoveel,!}{jaren  op een oprecht}{bestuur een vrije staat}\\

\haiku{Vele nachten zijn:}{er geweest en ik heb dat}{zelf ook veel gevraagd}\\

\haiku{zolang ik een taak.}{heb in de wereld wil ik}{je niet bij mij zien}\\

\haiku{En ik, ben ik zo?}{ver dat ik niet meer zag wie}{Ruffini is}\\

\haiku{Als je hier bent en,}{ik daarginder of als je}{gaat en ik blijf hier}\\

\haiku{zo wanneer ik weg,.}{ben ik zal je stem horen}{en rustig zijn}\\

\haiku{Hij is uitgegaan,,.}{naar de Palazzo als ik}{goed gezien heb}\\

\haiku{Je kunt niet met mij.}{samenwonen voor je mij}{weer recht kunt aanzien}\\

\haiku{En jij, dochter, hebt.}{een strengere hand dan de}{mijne nodig}\\

\haiku{Het vuur dat ons heeft.}{aangevat geeft het geluk}{zonder einde}\\

\haiku{borso Kom terug,,.}{met mij de avond valt ik zal}{je rustplaats zoeken}\\

\haiku{Hoor mij, spreek, moet ik:}{blijven of zullen wij doen}{wat dit teken zegt}\\

\haiku{Wij moeten het zijn,}{zie toch de dwaasheid anders}{te willen dan zo}\\

\haiku{Het is de oude,.}{vraag waarom de een meer dan}{een ander krijgt}\\

\haiku{valdarno Mijn dochter,,.}{natuurlijk alleen van mijn}{dochter spreekt men}\\

\haiku{Kom, Rossi is in,.}{mijn kamer laten wij dit}{met hem bespreken}\\

\haiku{Had die vervloekte,.}{Bongardo niet bestaan het zou}{niet gebeurd zijn}\\

\haiku{Zij klaagt niet, zij spreekt,.}{niet van pijn maar zij ligt heel}{de nacht met open ogen}\\

\haiku{Zeg haar dat ik er,.}{ben zij zal mij dadelijk}{willen horen}\\

\haiku{ciprian Alleen dat.}{het Bongardo goed gaat en dat}{nieuws is goed voor haar}\\

\haiku{Als ik hier vandaan, '.}{moet ga ik in het leger}{t is eender waar}\\

\haiku{Alleen ken ik de.}{naam niet van de plaats waar hij}{gesproken heeft}\\

\haiku{Wie weet een plaats waar,?}{wij kunnen leven wie weet}{de mooiste plaats}\\

\haiku{Ik heb dit nodig,.}{maak de gordel vast en leg}{die zoom wat hoger}\\

\haiku{Die ziek geweest is,,.}{ja en die mij veel zorgen}{heeft gegeven}\\

\haiku{rossi 't Is laat,.}{na een nacht van rust kun je}{beter oordelen}\\

\haiku{Want wat nu van mij, '.}{gevraagd wordt heeft geen waarde}{t is niet te koop}\\

\haiku{Mijn vader mag het,.}{niet weten hij zou denken}{dat ik verdriet had}\\

\haiku{Het zal je kennen,.}{het zal opengaan voor wie het}{zo beschermde}\\

\haiku{God en mij welkom,,.}{sprak Marke tot hem gebied}{over al het mijne}\\

\haiku{bevend hield zij de.}{sluier om het hoofd om de}{blos te verbergen}\\

\haiku{Zie, de dag gaat ten,,?}{avond waar zal ik mij bergen}{waar vind ik toevlucht}\\

\haiku{Neen vriend, neem deze,.}{harp laat horen wat men bij}{u te lande zingt}\\

\haiku{Haastig ging zij, zij,.}{nam de scherf en vond dat zij}{paste op het zwaard}\\

\haiku{Ach moeder, dit is.}{de moordenaar Tristan}{die uw broeder sloeg}\\

\haiku{Toen hij haar klagen.}{hoorde trad hij er binnen}{om haar te troosten}\\

\haiku{Want heeft hij ook geen,.}{deugden tenminste ware}{ik hem lief geweest}\\

\haiku{Daar in de grauwe.}{dag verrees de grimmige}{burcht van Tintagel}\\

\haiku{Schone, zo begon,.}{hij het doet mij leed dat ik}{van u scheiden moet}\\

\haiku{Isolde schreed zuchtend,.}{en treurig voort treurig ging}{Tristan zijns weegs}\\

\haiku{De zorg is mijn en.}{aller vrouwen lot wanneer}{zij verlaten zijn}\\

\haiku{Tristan zit en,.}{denkt aan de vrouwe die zo}{nabij denkt en wacht}\\

\haiku{Maar aanzien kan hij,.}{die wreedheid niet snel rijdt hij}{met zijn heren voort}\\

\haiku{Ook droegen zij de.}{schoonste kleding die ooit man}{en vrouw kan passen}\\

\haiku{En op een morgen.}{drong zijn jachtstoet ver tot in}{het land van Moro{\"\i}s}\\

\haiku{Tristan, er zijn,.}{duizend duizend vragen die}{ik niet vragen mag}\\

\haiku{Hij gehoorzaamde,.}{de roep hij droeg zijn vrouwe}{in zijn arm aan land}\\

\haiku{Edele gast, ik zweer.}{dat ik de duisternis uit}{uw ogen bannen kan}\\

\haiku{dat een zwaard, dat een?}{mensenkind in het leven}{liet onedel is}\\

\haiku{Geduld, mijn gade,.}{wanneer zo een spreekt is het}{in reine waarheid}\\

\haiku{Maar wat zal ik u?}{spreken van het binnenste}{van onze woning}\\

\haiku{Voort met de nar, zegt,,,.}{de koning doet zijn bevel}{voort voort met de nar}\\

\haiku{En Tristan zal.}{haar voeren naar zijn zaal in}{de dageraad}\\

\haiku{Vrouwe, sprak hij, ik,.}{moet mij nederleggen want}{het gif brandt mijn bloed}\\

\haiku{Houd uw bark zeilree,.}{voor morgen bij de dag voer}{mij snel over de zee}\\

\haiku{En duidelijker}{zichzelve en elkander}{ziende ontwaarden}\\

\haiku{Over het kwaad waarvan}{ik Venturi zou kunnen}{beschuldigen moet}\\

\haiku{Haar arm echter hield.}{hij vastgeklemd en hij kneep}{die tot zij schreeuwde}\\

\haiku{Dus moest zij wachten,.}{hoe het lot beschikken zou}{zij had besloten}\\

\haiku{zijn liefde was drong.}{hij aan dat hij doen zou wat}{hem geboden werd}\\

\haiku{Haastig ontkleedde.}{zij zich en borg het gewaad}{in de koffer weg}\\

\haiku{Met Nannina.}{kijk ik iedere dag naar}{de weg of je komt}\\

\haiku{En de volgende.}{dag was het alleen Corso}{die hen onderhield}\\

\haiku{De morgen daarna.}{werd het troostbehoevend hart}{weder diep beroerd}\\

\haiku{daarboven was een,}{schijnsel in het venster de}{stem van het kind klonk}\\

\haiku{Ik vrees dat het niet,.}{meer kan het is zo erg wat}{ik gebroken heb}\\

\haiku{Zij mijmerde en.}{begreep niet dat zij dit niet}{eerder had gekend}\\

\haiku{Voor mij had deze.}{dag het einde kunnen zijn}{van mijn aardse hoop}\\

\haiku{{\textquoteright} {\textquoteleft}Landro, ik heb veel,.}{gedwaald er zijn veel tranen}{geweest  door mij}\\

\haiku{aangezien ging zij.}{achteruit om te zoeken}{waar hun schaduw lag}\\

\haiku{zij nam haar teder,.}{tussen de vingers want zij}{kende de geur nog}\\

\haiku{De gestalte van;}{haar vader durfde zij niet}{meer te naderen}\\

\haiku{En soms zuchtte zij.}{en wist dat zij aan iets van}{een droom had gedacht}\\

\haiku{in haar ziel trilde.}{de schrik nog na van die flits}{der werkelijkheid}\\

\haiku{In de ernst van haar.}{afzondering nam zij met}{scherpe zinnen waar}\\

\haiku{Die avond verscheen zij,.}{ook aan het groot eremaal niet}{daar zij hoofdpijn had}\\

\haiku{Met ongeduld zat.}{zij aan de maaltijden het}{einde te wachten}\\

\haiku{zij voelde zich of.}{zij bevrijd was van een last}{die haar bedrukt had}\\

\haiku{Een luwte had haar,.}{opgenomen zij wist dat}{zij hoog zou stijgen}\\

\haiku{Hij had geschertst over,:}{het hart der vrouwen en zij}{had zacht geantwoord}\\

\haiku{Zij bloosde zoals,.}{zo vaak een vrouw bloost zonder}{te weten waarom}\\

\haiku{De andere zweeg.}{en keek eerst de kamer rond}{en dan naar buiten}\\

\haiku{In de avond hoorde.}{zij de deur van de kamer}{achter zich sluiten}\\

\haiku{Eerst begreep zij het,.}{niet zij dacht voor de moeiten}{die zij te doen had}\\

\haiku{Maar een schuchtere.}{stem fluisterde of zij die}{brief wel zenden mocht}\\

\haiku{Bij het peinzen hoe.}{zij gaan moest dacht zij ook aan}{mensen in de stad}\\

\haiku{Florine kwam soms,.}{in de winkel soms in de}{kamer daarachter}\\

\haiku{Maar wie komt voor iets.}{dat hier niet te vinden is}{geef ik mijn vriendschap}\\

\haiku{Bij de roep van een.}{mens in het donker strekte}{zij de handen uit}\\

\haiku{Ik heb een dochter.}{die haar man verloor toen hij}{krijgsgevangen was}\\

\haiku{Toen zij ontwaakte.}{zag zij hem voor zich staan in}{de opening der deur}\\

\haiku{Terwijl zij haar ogen}{wijd openhield voor het gulden}{licht van de hemel}\\

\haiku{Zij liet hem haar aan,.}{zijn borst drukken zij durfde}{nog niet te spreken}\\

\haiku{Zij schudde haar hoofd,,.}{het moest koud zijn daarbinnen}{en zij schreide niet}\\

\haiku{Genevi\`eve,,.}{haar dochter zou zeker nog}{in het klooster zijn}\\

\haiku{De rest van die brief.}{kon Rose-Ang\'elique}{eerst later lezen}\\

\haiku{toen en toen heb ik?}{aan je gedacht zonder dat}{mijn hart werd geroerd}\\

\haiku{Ach, ik kan hiervan.}{niet spreken omdat geen mens}{dit ooit kan verstaan}\\

\haiku{Ik verlang naar dit.}{ogenblik je te vinden in}{je liefste vreugde}\\

\haiku{zij bogen over hem.}{neder aan de ene en de}{andere zijde}\\

\haiku{Het geluid van een.}{vogel deed hen opzien naar}{de klare hemel}\\

\haiku{Maar eerst wilde ik.}{danken en de naaste kerk}{was San Stefano}\\

\haiku{Toen stonden allen,.}{op want zij moesten aan de dag}{van morgen denken}\\

\haiku{Mijn borst was vol, ik.}{kon niet nog meer geuren van}{de nacht omarmen}\\

\haiku{Of is het dat de?}{zucht des overvloeds verstaan en}{verhoord moet worden}\\

\haiku{Al Zagal, dat zij geen,.}{wens kon hebben want zij was}{hoog boven de stad}\\

\haiku{Zuiver is uw oog,,,.}{zuiver uw hart zuiver zal}{uw naam zijn Safija}\\

\haiku{zoude, omdat hij;}{aan haar ogen zag dat haar ziel}{gezond en goed was}\\

\haiku{De jongelieden.}{met fluiten en trompetten}{begeleidden hem}\\

\haiku{Die eigen dag vroeg.}{zij de abb\'e te komen}{om haar te horen}\\

\haiku{Innig drukte zij,.}{het aan haar borst biddend voor}{het jonge zieltje}\\

\haiku{In Mechtilt was de.}{overspelige geboren}{lang voor zij het wist}\\

\haiku{zuchtend dat zij niet.}{kon zolang het lichaam van}{overwinning juichte}\\

\haiku{Hij troostte haar dat}{de genade zekerlijk}{voor haar verschijnen}\\

\haiku{Zij vreesde niet meer,.}{zij wachtte duldzaam in haar}{biddende aandacht}\\

\haiku{En toen zij de ogen,}{weder opende begreep zij}{de rampzaligheid}\\

\haiku{En op een dag scheen,.}{het haar of het verwachte}{onheil nederviel}\\

\haiku{De mensen zagen,,;}{haar aan verwonderd over haar}{bleekheid meende zij}\\

\haiku{In de dageraad.}{keerde Pietro terug met}{schitterende ogen}\\

\haiku{Dan ging zij bij de}{Bargello om te vragen}{met welke rechter}\\

\haiku{Op de markt ontving,:}{zij de eerste wonderlijk}{bedachte hulde}\\

\haiku{Zij dankte God, zij,.}{had geen smart want alles droeg}{zij in lijdzaamheid}\\

\haiku{Andere stadjes;}{zijn die van burgers die al}{in rijkdom rusten}\\

\haiku{Het handwerk wordt nog;}{in het gebied van Genua}{lonend beoefend}\\

\haiku{van waar komt al dat,?}{goed van waar al die kleren}{van het bruidsgeschenk}\\

\haiku{Wat de jonkman zegt,.}{heeft hij in het woud van een}{kluizenaar gehoord}\\

\haiku{N.B.: het betreft hier.}{wellicht een bewerking van}{een Duitse editie}\\

\subsection{Uit: Verzameld werk. Deel 3}

\haiku{Hij houdt, ondanks zijn,.}{jaren rustig de gouden}{band aan zijn vinger}\\

\haiku{Perzik en amandel,,.}{die rokjes dragen wensen}{glimlach en buiging}\\

\haiku{De dadelpalm lacht,.}{maar luiert en verdient zijn}{naam niet ten volle}\\

\haiku{als hij spreken moet,}{ook al ware er niemand}{om te luisteren}\\

\haiku{Het licht schittert door,}{vocht voor zijn ogen maar hij}{weet niet of het komt}\\

\haiku{Velen ontvangen,.}{in hun roes het bericht dat}{de boot weer vertrekt}\\

\haiku{De eerlijkheid van.}{het hart komt als daglicht van}{de hand die arbeidt}\\

\haiku{Een kus heeft voor hem,.}{maar \'e\'en naam die terzelfder}{tijd een daad beduidt}\\

\haiku{Terstond zweeft het weer,.}{weg omdat het in die halm}{geen behagen vindt}\\

\haiku{Het is de hemel.}{zelf die hun in het vreemde}{land geschonken wordt}\\

\haiku{Hier wiegelen de,.}{palmen in het gulden azuur}{de bloemen geuren}\\

\haiku{Als het hart de trek,:}{naar het verre oord gevoeld}{heeft vraagt het verstand}\\

\haiku{In een warm land brengt.}{de avond ook de koelte die}{bij de rust behoort}\\

\haiku{En drie andere:}{zijn er waarin een geluid}{uit de diepten klinkt}\\

\haiku{Die eerste bundel,,.}{Les po\`emes saturniens werd}{weinig opgemerkt}\\

\haiku{Hij heeft verteld hoe.}{het begonnen was in een}{radeloze smart}\\

\haiku{Parijs, zijn vrouw, zijn,.}{moeder de po\"ezie moesten}{verdedigd worden}\\

\haiku{Maar het kwam uit de,.}{drift van avonturiersbloed niet}{uit leergierigheid}\\

\haiku{Het kind was misschien;}{zeven jaar toen hij ruimer}{dan de meesters zag}\\

\haiku{Van ijdelheid of.}{zelfvoldaanheid heeft niemand}{hem ooit beschuldigd}\\

\haiku{Zij vergaf, vele,.}{jaren daarna maar zij heeft}{hem nooit weergezien}\\

\haiku{{\textquoteright} Rimbaud werd te sterk.}{door het bloed gedreven om}{te kunnen troosten}\\

\haiku{Een politieagent.}{bemerkte de vluchtende}{en de vervolger}\\

\haiku{En hij ging naar de,,.}{kerk iedere zondag des}{morgens en des avonds}\\

\haiku{Rustig vervolgde.}{hij dit eerste jaar van zijn}{vergeefse arbeid}\\

\haiku{De hemel dreef een.}{spel dat hem uit de diepte}{der kelk deed drinken}\\

\haiku{het loon voor zijn pen,,;}{reeds jaren oud kon men met}{stuivers rekenen}\\

\haiku{in de mannentijd;}{gloeide de versmachting naar}{het goddelijk heil}\\

\haiku{Hoezeer was hij een!}{christen geweest met goede}{wil en duldzaamheid}\\

\haiku{Zijn moeder nam haar.}{altijd met dezelfde hand}{waaraan de ring blonk}\\

\haiku{dan ving hij een woord,.}{op dat hij tot dusverre}{van knechts had gehoord}\\

\haiku{Hij voelde een blos.}{op zijn wangen komen bij}{deze ontdekking}\\

\haiku{Had Beatrice,,?}{de rouw afleggend zijn raad}{niet te vroeg gevolgd}\\

\haiku{de gelijkenis}{derhalve kon niet getoond}{worden door hetgeen}\\

\haiku{Merona echter.}{reed snel en zijn dienaar zag}{hem met vreugde aan}\\

\haiku{dat gij meer in mijn.}{gedachten zijt dan ik in}{een brief kan zeggen}\\

\haiku{Vroeg in de morgen.}{reden zij uit de stad tot}{een verlaten veld}\\

\haiku{zoude hetgeen hij.}{op zijn reizen gezien en}{ondervonden had}\\

\haiku{Had hij niet een adem?}{gevoeld of de lente uit}{het zuiden waaide}\\

\haiku{Veronica riep.}{hem eens tot zich op een bank}{in de vensternis}\\

\haiku{Het is goed gezegd,,.}{sprak Ercole mijn trouw is}{inderdaad verdacht}\\

\haiku{Merona ontving.}{het bevel naar Ferrara}{terug te keren}\\

\haiku{hij had Filippo.}{zien binnensluipen en hem}{door een reet bespied}\\

\haiku{Doch de klaarheid zag.}{hij iedere morgen en}{avond na het gebed}\\

\haiku{Zijn moeder plaatste hij.}{voor het venster opdat zij}{hem na kon wuiven}\\

\haiku{Renaldo Maria,.}{een edel man had hij afscheid}{van haar genomen}\\

\haiku{De bode vertrekt,.}{nog voor de middag omdat}{de dagen korten}\\

\haiku{En hij begreep welk.}{geliefd beeld een ander in}{zijn gedachten droeg}\\

\haiku{hij meende dat zij.}{hem dankbaar was en daarom}{niet durfde spreken}\\

\haiku{Ga, zoek, vind spoedig,.}{en kom terug want ook uw}{land heeft u nodig}\\

\haiku{Merona trok door.}{verscheidene gewesten}{der eedgenootschap}\\

\haiku{Maar ik heb ook het,,,.}{geluk gehad de liefde}{mijn vriend de liefde}\\

\haiku{Merona zag dat:}{er een verandering over}{hem was gekomen}\\

\haiku{Maar ik geloof dat.}{het portret het enige is}{dat mij hier nog bindt}\\

\haiku{Een traan viel, maar hij.}{dankte de hemel voor haar}{en hem te zamen}\\

\haiku{de juffers van het.}{hof hadden in een klooster}{toevlucht genomen}\\

\haiku{De degen hield hij,.}{in de hand zeggend dat die}{hem geschonken was}\\

\haiku{Een ieder, mijn vorst,.}{kent zijn plicht zo hij naar zijn}{geweten luistert}\\

\haiku{Het waren lange,.}{dagen waarin de zon scheen}{door geen wolk gestoord}\\

\haiku{Sedert mijn zuster.}{haar zoon verloor hield zij de}{handen te zamen}\\

\haiku{Hoewel ik hem niet.}{gezegd had waar ik woonde}{verwachtte ik hem}\\

\haiku{Ik was elf jaar, een,.}{kind dat weinig kreeg voor de}{mond noch voor het hart}\\

\haiku{Ik voelde dat ik.}{groter was en voor eeuwig}{bevrijd van een band}\\

\haiku{Het was zomer, stil,.}{alleen ging soms achter mij}{een locomotief}\\

\haiku{Zij had alles dat,.}{verblinden kan want alles}{scheen wat het niet was}\\

\haiku{Brand voor het onrecht,.}{bid om genade voor wie}{je je vrienden noemt}\\

\haiku{Roep en smeek tot er.}{zijn die je alles opendoen}{behalve hun ziel}\\

\haiku{in een kinderkreet;}{herkende ik hoe ik zelf}{had willen grijpen}\\

\haiku{Ik verliet het dorp.}{en dwaalde in een stille}{stad een zomer lang}\\

\haiku{Je zou zelf je pijp.}{gemaakt hebben en in de}{kring kunnen zitten}\\

\haiku{De stad wemelde.}{van mensen die er voor hun}{genoegen kwamen}\\

\haiku{ik heb ook gezocht,,}{wanneer Amsterdam sliep naar}{ogen die mij helpen}\\

\haiku{Ik antwoordde dat,.}{ik dit was hoewel niet in}{Holland geboren}\\

\haiku{ik moet betoverd,.}{geweest zijn zodat geen oog}{mij kon waarnemen}\\

\haiku{Het was een wonder,:}{er kon in heel de stad geen}{mooier jongen zijn}\\

\haiku{Maar de vriendin zwoer.}{met een eed het niet verder}{te zullen zeggen}\\

\haiku{Het is of een geest.}{op een zekere plek moet}{wonen en er heerst}\\

\haiku{Wanneer de klok voor:}{noen begon te luiden en}{Gelsomino zeide}\\

\haiku{Men zegt dat elk van.}{die verloren maagden een}{goede huisvrouw werd}\\

\haiku{En Dianora kon.}{die clarissezuster zelfs}{geen antwoord geven}\\

\haiku{Ook al de vrienden,.}{volgden onder elkander}{reeds op wraak zinnend}\\

\haiku{Cecco antwoordde.}{dat hij haar kende en dat}{zij welvarend was}\\

\haiku{{\textquoteright} De stem zweeg en zij.}{hoorden zware voetstappen}{naar de keuken gaan}\\

\haiku{Achter het huis was.}{een tuintje met laurieren}{en klimop begroeid}\\

\haiku{Zij was te zwak om,,.}{zich op te richten zij riep}{maar niemand hoorde}\\

\haiku{Hier klopte zij en.}{stond tot haar voeten als steen}{waren op de grond}\\

\haiku{Zij werd moedeloos,.}{zij had geen traan meer en wrong}{de handen niet meer}\\

\haiku{En waarschijnlijk ook,.}{de nimf niet want zij zweeg en}{maakte geen geruis}\\

\haiku{In de drukte der;}{volgende dagen sleet de}{nieuwheid van het dek}\\

\haiku{Zeker, je hebt voor,.}{mij gedaan behoorlijk naar}{je plicht maar niet meer}\\

\haiku{Daar hij langer thuis,.}{kon blijven viel het afscheid}{zwaarder toen het kwam}\\

\haiku{De derde stuurman.}{werd niet weer aangenomen}{en Bos kwam terug}\\

\haiku{Maar Brouwer meende.}{dat er ander werk te doen}{was dan een wedstrijd}\\

\haiku{Hij nam Pot mee en.}{aan de deur verzocht hij hem}{binnen te treden}\\

\haiku{Zijn vrouw lag op bed,,.}{hij kon haar niets zeggen want}{zij kende hem niet}\\

\haiku{Gedurende twee.}{maanden zat Wilkens alleen}{in zijn woonkamer}\\

\haiku{als de kapitein.}{hem de hand wilde geven}{was elk woord te veel}\\

\haiku{Hij sprong toe, zette.}{hem rustig terzijde en}{vierde zelf de lijn}\\

\haiku{In waarheid hadden.}{het schip en Brouwer er het}{meeste voordeel van}\\

\haiku{Hij had afkeer van;}{het smokkelen en achtte}{het oneerlijkheid}\\

\haiku{Brouwer wist in zijn,.}{hart dat hij het vinden zou}{waar het ook zijn mocht}\\

\haiku{Hij was het toen die.}{beval en zelfs de stuurman}{deed wat hij schreeuwde}\\

\haiku{een Hollander op.}{jaren zou wel iemand zijn}{die het werk verstond}\\

\haiku{bovendien zou hij.}{als tweede stuurman varen}{en Meeuw als bootsman}\\

\haiku{Brouwer deed mee bij,.}{het werk en kreeg zijn deel dat}{hij zelf bewaarde}\\

\haiku{Plotseling zagen.}{de mannen die toehoorden}{hem veranderen}\\

\haiku{Hij bleef staan en vroeg.}{Brouwer hoeveel hij voor het}{schip had geboden}\\

\haiku{Brouwer wist dat het,,,.}{slecht wantslag was drieduims nieuw}{in Iquique maar bruin}\\

\haiku{Toen hij zelf voet aan}{dek zette schudde hij de}{drie mannen de hand.}\\

\haiku{En hij hoorde het,.}{knarsend gedreun in zijn borst}{hij sloeg achterover}\\

\haiku{Every bleef alleen,,,.}{hij keek iedere dag uit}{een maand lang en meer}\\

\haiku{Veel verhalen van}{wat er vroeger geweest was}{kenden zij evenmin}\\

\haiku{De bomen en de,.}{wortelen gaven spijs de}{watervallen drank}\\

\haiku{En daar alles van,.}{de liefde geluk was werd}{er niets verborgen}\\

\haiku{Ook had het kind er,.}{geen nadeel van het was het}{kind van de moeder}\\

\haiku{het bosje waar de.}{twee groene stenen lagen}{in de beek was tap\`u}\\

\haiku{Een oude vrouw kwam.}{aan boord met een banaanblad}{en een varkentje}\\

\haiku{Hij liet tarwe, gerst,,,.}{haver rijst voor hen zaaien}{uien en groenten}\\

\haiku{Kuisheid is onder.}{vrouwen van een zekere}{soort schier onbekend}\\

\haiku{Churchill, Burkitt en.}{Mills zijn handen stevig op}{de rug gebonden}\\

\haiku{Anderen hielden.}{de musketten gericht op}{de sloep beneden}\\

\haiku{Even na middernacht.}{riep de man aan het roer dat}{hij branding hoorde}\\

\haiku{Er stond een ruwe,.}{zee er moest gehoosd worden}{door wie nog konden}\\

\haiku{Zij wachtten tot de.}{morgen en bemerkten toen}{dat het bewoond was}\\

\haiku{Kort daarna kwamen.}{er nog twee mannen zich op}{Pitcairn vestigen}\\

\haiku{Uit ver gelegen.}{dorpen kwamen de mensen}{er ter bedevaart}\\

\haiku{Hierheen trok Lot met,.}{zijn vrouw en zijn kinderen}{zijn knechts en zijn vee}\\

\haiku{Sta op, trek door het,.}{land van noord tot zuid het zal}{u toebehoren}\\

\haiku{Zie de sterren van,.}{de hemel even talrijk zal}{uw nageslacht zijn}\\

\haiku{Ook noemde hij de,,:}{naam die zij hem geven moest}{Isma\"el dat is}\\

\haiku{Vandaag of morgen.}{bespringt mij een troep wolven}{en word ik verscheurd}\\

\haiku{De vluchtelingen,.}{konden niet ver zijn want het}{vee had pas gegraasd}\\

\haiku{Jahweh verlost zijn.}{volk en voert ons naar het land}{aan Abraham beloofd}\\

\haiku{Jozua zag de macht.}{van zijn god en Jahweh koos}{hem tot zijn strijder}\\

\haiku{Daarna klom Mozes.}{op de berg Nebo en hij}{keerde niet terug}\\

\haiku{En onder de scherts.}{die zij maakten gaf  hij}{hun een raadsel op}\\

\haiku{En Simson, die in,.}{slaap was gevallen rukte}{de doek van de muur}\\

\haiku{Hij kon niet meer, maar.}{hij voelde dat het haar weer}{groeide op zijn hoofd}\\

\haiku{Na de plechtigheid.}{nam hij zijn staf en reisde}{naar Rama terug}\\

\haiku{Iets verder lag de.}{bevelhebber en rondom}{sliepen krijgslieden}\\

\haiku{Nu weet ik dat gij,,.}{de grotere koning van}{Isra\"el zult zijn}\\

\haiku{Argeloos zijt gij,,,}{sprak David uw vader weet}{dat gij mijn vriend zijt}\\

\haiku{aan u zal hij dus.}{zeker niet zeggen dat hij}{kwaad tegen mij wil}\\

\haiku{Maar als hij toornig.}{wordt is het zeker dat hij}{mij vervolgen wil}\\

\haiku{Ik zal op deze.}{plek komen met mijn knecht en}{een pijl afschieten}\\

\haiku{Hij weigerde aan.}{de feestmaaltijd te zitten}{en verliet de zaal}\\

\haiku{Des nachts stierf haar kind.}{omdat zij er in de slaap}{op was gaan liggen}\\

\haiku{Toen ik hem nog voor}{het dag was aan mijn borst nam}{om hem te zogen}\\

\haiku{Gelukkig het volk!}{dat door zijn god met zulk een}{vorst gezegend is}\\

\haiku{Hij spotte met de,.}{Almachtige maar in zijn}{hart vreesde hij hem}\\

\haiku{Maar ontneem hem eens,.}{uw weldaden gij zult zien}{hoe hij u verzaakt}\\

\haiku{Ziet, ik heb maanden,;}{van ellende gekregen}{nachten zonder slaap}\\

\haiku{behandel mij niet,.}{als een schuldige gij weet}{dat ik het niet ben}\\

\haiku{Ik zit op as en,,.}{vuilnis zelf niet anders dan}{stof weggeworpen}\\

\haiku{Hij  vroeg om een:}{stift en een tablet van was}{en hij schreef daarop}\\

\haiku{Jozef zat in de,.}{donkere hoek Maria hield}{haar kind aan de borst}\\

\haiku{Wie van u, die twee,?}{stuks kleren had heeft er hem}{een van gegeven}\\

\haiku{Toch was dat uw plicht,.}{en gij weet het het is u}{van kindsbeen geleerd}\\

\haiku{Maar hij weigerde,.}{hij greep hem en bracht hem naar}{de gevangenis}\\

\haiku{De drie vielen op.}{de knie\"en in vrees voor het}{wonderbaarlijke}\\

\haiku{Hij verliet dat land.}{en reisde terug naar het}{huis van zijn vader}\\

\haiku{Dus beslisten de;}{Romeinen af te wachten}{wat er gebeurde}\\

\haiku{De poort stond open, de.}{zon scheen in de eenzaamheid}{van de binnenhof}\\

\haiku{De priesters en de.}{schriftgeleerden leren u}{de wet van Mozes}\\

\haiku{Maar doet niet zoals,,.}{zij doen want zij spreken wel}{maar handelen niet}\\

\haiku{Hij had met zondaars,.}{verkeerd met verdoemden aan}{de dis gezeten}\\

\haiku{Tijdschriftpublikatie,,,:}{in De Hollandsche Revue}{1929 jg. 34 als volgt}\\

\haiku{Het Westerdok{\textquoteright} van,.}{J. Plaat en verklaring van}{scheepstermen Amsterdam,1975}\\

\subsection{Uit: Verzameld werk. Deel 4}

\haiku{De koster lachte,.}{waarop Ras kwaad werd en hem}{voor huichelaar schold}\\

\haiku{Op hem lag de taak.}{de ouderdom van moeder}{en oom te steunen}\\

\haiku{Daar hij liever in.}{een warm land wilde wonen}{koos hij voor Indi\"e}\\

\haiku{De directeur prees.}{hem en voorspelde dat hij}{het ver zou brengen}\\

\haiku{Wat hun veroorloofd,.}{was mochten zij houden}{de rest moest aan wal}\\

\haiku{De kapitein had.}{geen andere bijstand dan}{van de lanspassaat}\\

\haiku{In Jacatra had.}{hij zich onderscheiden door}{moed en bekwaamheid}\\

\haiku{Maartensz tekende.}{het briefje en ontving het}{geld voor het vervoer}\\

\haiku{De kwartiermeester.}{liet de kisten stouwen met}{nog twee andere}\\

\haiku{Zij wisten ook  .}{dat niet allen Nederland}{terug zouden zien}\\

\haiku{Bij Andries in huis,.}{zeide hij dat hij vriend was}{maar ook korporaal}\\

\haiku{Een uur later zei.}{hij terloops dat Manilla}{niet ver weg kon zijn}\\

\haiku{Naar foelie kwam meer,,.}{vraag ook naar noten gebruikt}{bij alle spijzen}\\

\haiku{De enige die Ai,,.}{kon helpen een klein Engels}{schip voer voor hen weg}\\

\haiku{Maar een andere,,.}{keer bij het herhaald verzoek}{beraadslaagden zij}\\

\haiku{De droppels van de.}{regen spatten van de grond}{op of het bloed was}\\

\haiku{Ai wonen noch op.}{een andere plaats waar de}{Compagnie gebood}\\

\haiku{Hij las de brieven.}{van huis die alle zegen}{en welvaart meldden}\\

\haiku{met bewogen stem,.}{noemde hij hem een deugdzaam}{een edelmoedig heer}\\

\haiku{En als het werk goed.}{gedaan werd achtte hij niet}{of het langzaam ging}\\

\haiku{Hij had zo veel zwart,.}{volk vervoerd dat hij ze voor}{enkel koopwaar hield}\\

\haiku{Het gevolg van de.}{belediging moest haar man}{Antonio dragen}\\

\haiku{bomen doormidden,.}{gescheurd huizen ingedrukt}{en stukgeslagen}\\

\haiku{De vreugde onder.}{het pakken en redderen}{duurde korte tijd}\\

\haiku{De Brasser anders,,.}{dan vroeger slordiger}{onverschilliger}\\

\haiku{Toen scheepte hij zich.}{met hen in op het jacht dat}{naar Ambon vertrok}\\

\haiku{Water hadden zij,.}{uit de geulen hout voor het}{drogen uit het bos}\\

\haiku{Yen Pon de winst in,.}{goud doen wegen drie kistjes}{elk van dertig pond}\\

\haiku{Van de raden van.}{Indi\"e had er maar een wat}{geld overgehouden}\\

\haiku{Hij had geen rijd voor.}{onbenulligheden en}{verliet het kantoor}\\

\haiku{De kisten werden,.}{niet geopend zij gingen}{in het grote schip}\\

\haiku{Ai doen vervoeren.}{om nootmuskaat te plukken}{voor de Compagnie}\\

\haiku{De klokken waren.}{maar zwak te horen in het}{geknal en gejuich}\\

\haiku{Hij keek beiden aan,.}{en hij wilde iets vragen}{maar hij wist niet wat}\\

\haiku{Dan bleef hij alleen.}{maar kijken naar de kuil die}{vol was gelopen}\\

\haiku{Toen hij hoorde dat}{zij met de postbode over}{het ijs zouden gaan}\\

\haiku{En hij luisterde,.}{zoals de grootmoeder las}{hij zag het voor zich}\\

\haiku{Maarten hoorde het,.}{en bad hij bleef nu staan tot}{het schieten ophield}\\

\haiku{En elke dag had,.}{hij meer gehoord en eerder}{dan iemand anders}\\

\haiku{Telkens vroeg zij iets,.}{met haar kinderstem dan kwam}{haar adem aan zijn wang}\\

\haiku{Hij wist niet hoe hij,.}{het zeggen moest hij keek haar}{aan en zij wachtte}\\

\haiku{Dan liep hij nog een,.}{eind de dijk op de witte}{maan stond nevelig}\\

\haiku{Maarten lag wakker.}{lang nadat de torenklok}{twaalf had geslagen}\\

\haiku{Het is niets, vrouw, de.}{Heer slaat daar minder acht op}{dan op jouw koffie}\\

\haiku{Op een morgen dat}{Rossaart aan de andere}{oever wachtte zag}\\

\haiku{De commandant was.}{een bejaarde kapitein}{van de schutterij}\\

\haiku{Op de hoek kwam hij,.}{zijn broer Hendrikus tegen}{die voor hem staan bleef}\\

\haiku{Bij de plecht stond de,.}{kolenpot het aardewerk}{hing er aan spijkers}\\

\haiku{Velen wisten van:}{hen te vertellen en hun}{faam verergerde}\\

\haiku{zij namen stenen.}{mee en hakmessen om de}{schuit te vernielen}\\

\haiku{De wijnkoper had.}{haar in huis genomen en}{zocht een dienst voor haar}\\

\haiku{Ik hoop alles goeds,}{voor je daarginds maar kijk niet}{neer op je vrienden}\\

\haiku{hout, spijkers, verf gaf.}{Seebel hem in ruil voor zijn}{aandeel in de tjalk}\\

\haiku{Maar nu je toch hier.}{bent zal ik je zeggen wat}{wij van je denken}\\

\haiku{daar staat de dood voor,,,.}{mij goed ik geef mij over er}{is niets aan te doen}\\

\haiku{Hij boog het hoofd en.}{staarde door de hor naar de}{nevel over de gracht}\\

\haiku{Hij staarde naar het,.}{ijs op de ruit hij dacht en}{schudde soms zijn hoofd}\\

\haiku{Van toen aan merkten,.}{zij een verschil hoewel zij}{het niet beseften}\\

\haiku{Man, zeide zij, ik.}{zal erom huilen dat je}{alleen moet varen}\\

\haiku{- De steeg was nauw, hij;}{bukte laag om door de hor}{naar de lucht te zien}\\

\haiku{'t Is anders een,.}{hele tijd vijfenveertig}{jaar alleen te zijn}\\

\haiku{Hadden ze maar ja.}{gezegd toen ik je bij me}{in huis wou nemen}\\

\haiku{Geld was er genoeg,.}{je was heemraad geworden}{en allang dijkgraaf}\\

\haiku{er zullen er nog,,.}{heel wat verdrinken ik weet}{het zeker zei je}\\

\haiku{neen, dat geld leg ik,.}{opzij dan heeft hij wat meer}{als hij bij me komt}\\

\haiku{hij vroeg Rossaart of.}{zij haar samen wat met de}{post zouden zenden}\\

\haiku{Jij bent de enige,.}{van wie ik het zie maar er}{zijn ook anderen}\\

\haiku{Je wordt stijf in je,,,}{rug zeg je morgen kan je}{toch niet meer varen}\\

\haiku{Hij zat rechtop met,.}{haar hand in de zijne maar}{hij kon niet spreken}\\

\haiku{Het water klotste.}{tegen het boord en de schuit}{trok aan de touwen}\\

\haiku{vroeg hij, het is toch.}{niet de eerste keer dat je}{in Hurwenen komt}\\

\haiku{een mens haast niets meer,.}{te kosten en zij nam geen}{geld meer van hem aan}\\

\haiku{Doe je plicht, dacht hij,,.}{dan en vraag niet het zal wel}{gegeven worden}\\

\haiku{, zij werd oud en hem.}{werd het soms te veel altijd}{alleen te varen}\\

\haiku{De hond, die aan zijn,.}{voeten stond schudde zich de}{sneeuw van de haren}\\

\haiku{In gedachten schold.}{hij op zijn vrouw dat zij niet}{meer had klaargezet}\\

\haiku{Haast u, Achroem, geef.}{het lichaam uw gunsten en}{laat het snel vergaan}\\

\haiku{Zijn spraak was verward,.}{hij zuchtte en zijn stem klonk}{als een zwakke klacht}\\

\haiku{Elphin Bach gaf hem.}{de ketel en wilde er}{geen geld voor hebben}\\

\haiku{Hij leerde en deed,.}{zijn taak maar hij dacht aan een}{bloem en aan een oog}\\

\haiku{heb ik aanschouwd, ik,}{had de keuze tussen meer}{dan honderd allen}\\

\haiku{Geen pot te koken,,.}{geen stofte vegen daar zijn}{deerns en knechten voor}\\

\haiku{De maat loopt over, zei,.}{de bakker en hij schopte}{Matthes de deur uit}\\

\haiku{De jongen spiedde.}{en luisterde in alle}{hoekjes en gaatjes}\\

\haiku{Opeens was het veld,.}{er niet maar zij zag twee ogen}{die haar aankeken}\\

\haiku{Dit was vergif voor,.}{zijn idealisme een snel}{werkend bovendien}\\

\haiku{Toen zij hem vroeg of.}{hij haar herkende bleef hij}{het antwoord schuldig}\\

\haiku{In zijn huis zag zij,,.}{veel ruiten veel geld meer dan}{zij ooit gezien had}\\

\haiku{Een achterlijke.}{sukkel mocht al blij zijn als}{iemand hem aankeek}\\

\haiku{Engeltje, zeide,.}{zij met grommende stem je}{hebt mij slecht gediend}\\

\haiku{Het geneurie zwol,.}{aan tot gedreun want heel de}{menigte deed mee}\\

\haiku{Wel zag hij dat hun.}{bewegingen allengs van}{aard veranderden}\\

\haiku{Vier dagen lag hij,.}{te woelen en te krabben}{zuchtend en vloekend}\\

\haiku{Het ergste was dat,.}{hij het geloof verloor hoe}{weinig dat nog was}\\

\haiku{Opgewekt of hij,.}{de jeugd weer had besteeg de}{sukkelaar de berg}\\

\haiku{Maar zo eenzaam zat.}{die visser dat hij het niet}{durfde aan te zien}\\

\haiku{, opgevrolijkt met.}{doedelzak en klarinet}{van kermisgasten}\\

\haiku{toevertrouw dient gij.}{toch behoorlijk in de echt}{verbonden te zijn}\\

\haiku{Zij spande zich des.}{te meer in om te tonen}{dat zij het wel kon}\\

\haiku{Soms tikte zij mee,.}{eerst met een duwtje in de}{keel en dan de tik}\\

\haiku{Daar denk ik al zo,,.}{lang aan antwoordde zij ik}{wou dat het maar kwam}\\

\haiku{zag ik aan de vorm.}{van uw hoofd waarmee ik u}{een pleizier kon doen}\\

\haiku{Een hoofd vol dikke,,.}{krullen rood als gepolijst}{koper stak eruit}\\

\haiku{Hij zond de kassier.}{om Mik twee rijksdaalders per}{avond aan te bieden}\\

\haiku{In de winkels kon.}{men tekeningen kopen}{die hem voorstelden}\\

\haiku{Hij stond met de rug.}{naar nummer acht gekeerd toen}{hij daar iets hoorde}\\

\haiku{En plotseling hield,.}{het op er kwam toen een rood}{licht over de bomen}\\

\haiku{Zij kleedde zich, deed.}{de grendels af en ging naar}{de kant van het ven}\\

\haiku{Snel stak de wind op,.}{de zee\"en sloegen schuimend}{tegen de boorden}\\

\haiku{De tranen werden.}{parelen en de mond sloot}{tot eeuwige rust}\\

\haiku{Uw hals is, dunkt mij,.}{iets minder gevuld dan voor}{een jaar of twintig}\\

\haiku{Mevrouw sprak met haar.}{man en met haar dochter over}{haar beschermeling}\\

\haiku{Hij was in Londen.}{geboren en getogen}{en hij heette Tim}\\

\haiku{Plotseling voelde,}{hij dat hij daar getrokken}{werd plotseling spron}\\

\haiku{In Ecbatane.}{droeg iedere straatveger}{een gouden ster}\\

\haiku{Er bestaan omtrent.}{de betekenis van dit}{woord misverstanden}\\

\haiku{Voor een andere.}{deur stond zij stil omdat haar}{borst zo zwaar bewoog}\\

\haiku{De rijke burgers.}{zagen haar niet op het feest}{met veel flambouwen}\\

\haiku{Zij klapte in de,:}{handen zij kuste hem op}{de wang en zij riep}\\

\haiku{Dank je, zeide hij,.}{zo help je na je leven}{nog de andere}\\

\haiku{De leverancier.}{had slechts in te vullen wat}{hij geleverd had}\\

\haiku{Zelfs de domste begreep.}{dat hij niet meer mocht kopen}{dan hij nodig had}\\

\haiku{Het was feest in huis,.}{de kanarie zat schel te}{fluiten in zijn kooi}\\

\haiku{Ook Prisca kon geen,.}{brief aan hem zenden hoewel}{zij er vaak een schreef}\\

\haiku{s morgens uit en?}{waar zou zij anders gaan dan}{naar de rivierkant}\\

\haiku{Thuis sprak zij hardop,.}{tot de kanarie die dan}{zweeg en haar aankeek}\\

\haiku{In het struikgewas,}{dat aan de heirweg grensde}{strekte hij zich uit}\\

\haiku{Voor de dochters der:}{rijken tikte hij aan de}{hoed met de woorden}\\

\haiku{Hij stond stil en hij,.}{merkte hoe zij naar hem keek}{of zij iets wenste}\\

\haiku{Vrouw, je hebt mij goed,,.}{verzorgd zeide hij ik zal}{je wel belonen}\\

\haiku{Maar de weg naar de.}{stad was lang en snel kon zij}{met de last niet gaan}\\

\haiku{Maar wanneer hij zijn.}{degen gordde zag hij dat}{haar ogen strak werden}\\

\haiku{Maar binnen een}{jaar nadat hij zijn vader}{was opgevolgd had}\\

\haiku{Wie ik ben gaat je,.}{geen sikkepit aan noem mij}{maar Zo-en-zo}\\

\haiku{Zie je die pot met?}{dat barstje erin waar het}{water doorsijpelt}\\

\haiku{Maar weldra kregen.}{zij bevel een wakend oog}{op hen te houden}\\

\haiku{De luitenant liet.}{de geweren aan rotten}{zetten en ging mee}\\

\haiku{Nu hij zowat een.}{jaar gereisd had was Jorden}{al veel veranderd}\\

\haiku{Jorden tuurde heel.}{de dag en heel de nacht of}{hij het land zou zien}\\

\haiku{Vele dagen liep:}{hij met geschroeide voeten}{en hij vroeg zichzelf}\\

\haiku{In Koerdistan kwam.}{hij waar een man hem het graf}{van zijn ouders wees}\\

\haiku{Lang voer hij tussen.}{de hemel en het water}{naar de horizon}\\

\haiku{Goed, zei Jorden, ik,.}{zal twee minuten wachten}{dan reis ik verder}\\

\haiku{Men herinnere.}{zich de toestand waarin de}{wereld verkeerde}\\

\haiku{De kleermaker, de,.}{zilversmid de kantwerksters}{kregen veel te doen}\\

\haiku{omdat immers zijn.}{ouders hem alles gaven}{wat hij behoefde}\\

\haiku{En toen Daan naar huis.}{ging volgde het dier met de}{neus aan zijn hielen}\\

\haiku{Hier heb je het geld,,,.}{mijn vriend en als het op is}{kom je maar terug}\\

\haiku{En toen de monden.}{gesloten bleven keerden}{de vissen terug}\\

\haiku{Aan het eind van een.}{steil pad bleef hij staan voor een}{hoge bronzen deur}\\

\haiku{Andere wijzen.}{denken anders en weten}{het misschien beter}\\

\haiku{Beste vrouw, het heeft.}{geen nut jezelf te kwellen}{met zulke vragen}\\

\haiku{In het voorportaal.}{naast de zuilen steeg de rook}{uit offervaten}\\

\haiku{Ik heb altijd mijn,.}{brood gehad genoeg om geen}{honger te hebben}\\

\haiku{Maar ik verdiende.}{niet genoeg om mijn schoenen}{te laten lappen}\\

\haiku{Wij zullen het de,.}{oude vragen die weet het}{bij ondervinding}\\

\haiku{Gelukkig dat het,,.}{er is ik heb angst gehad}{waarom weet ik niet}\\

\haiku{Maar voor het jaar ten.}{einde ging begon hij zich}{te verontrusten}\\

\haiku{Hij had met alle.}{schuldeisers gesproken en}{zich met hen verstaan}\\

\haiku{Gerbrand had noch de.}{toon noch de betekenis}{hiervan begrepen}\\

\haiku{Ik doe  wel mee,,.}{antwoordde Diderik maar}{niet boven mijn kracht}\\

\haiku{Eens, toen zij opstond,:}{om naar boven te gaan keek}{hij op en zeide}\\

\haiku{Het huilen ging voort,.}{het was te horen in de}{andere winkels}\\

\haiku{en dat maakte hem.}{zo blij dat hij het hard in}{de armen drukte}\\

\haiku{Frans hief het hoog op.}{naar de bloesems zodat het}{met de ogen knipte}\\

\haiku{Ach broer, zou het niet?}{beter zijn hem met zachtheid}{te behandelen}\\

\haiku{tot hij het weer was.}{die sloeg en de zwakkeren}{de knikkers afnam}\\

\haiku{klagen dat Floris.}{haar kind geknepen had of}{de kleren gescheurd}\\

\haiku{Jongen, vroeg hij, weet?}{je niet dat een dief in de}{gevangenis komt}\\

\haiku{Hij sloeg de zijne,,.}{neer zijn lippen trilden maar}{hij kon niets zeggen}\\

\haiku{Toen Agnete even,:}{bij hem kwam staan om iets te}{vragen zeide hij}\\

\haiku{Dit moet hij voor de,.}{ogen houden en ons voorbeeld}{van rechte zeden}\\

\haiku{En hun moeder keek,.}{ze altijd lachend aan trots}{en vol vertrouwen}\\

\haiku{Daar heeft zij toch ook,,.}{schuld aan zeide Stien en dat}{begreep Floris niet}\\

\haiku{Voor de deur zelf  ,.}{gleed hij uit viel van de stoep}{en brak de enkel}\\

\haiku{Hij nam ze mee op,}{grote wandelingen tot}{voorbij Bennebroek}\\

\haiku{Dan hosten zij arm,,:}{aan arm schreeuwend tegen de}{mensen joelend van}\\

\haiku{Hij ging iedere.}{dag met de jongens tot de}{laatste zaterdag}\\

\haiku{Hij woelde, hij kon,.}{niet slapen buiten zongen}{nog kermisgangers}\\

\haiku{In de straat sprak zij,,:}{niet maar voorbij de brug waar}{niemand ging vroeg zij}\\

\haiku{Zijn oom sloeg maar even,:}{de ogen op zette zijn bril}{recht en antwoordde}\\

\haiku{Het waren kleine,.}{hoge tonen klagend door}{de witte ruimte}\\

\haiku{Je neemt evengoed een.}{stuk koek uit de kast en dat}{is toch geen diefstal}\\

\haiku{Hij lag wakker, hij.}{besefte niet eens hoe de}{gedachten kwamen}\\

\haiku{Dan kan ik in dit,,.}{huis niet blijven  dacht hij}{het wordt hier te erg}\\

\haiku{Maar dan vroeg Jansje,.}{naar de oom in Hoorn of hij}{oud geworden was}\\

\haiku{En vergeet het niet,.}{een mens hoeft geen kwaad te doen}{als hij het niet wil}\\

\haiku{Maar over vijf, zes jaar.}{zou er geen vlek meer op de}{naam van Floris zijn}\\

\haiku{Neen, daar kunnen wij,.}{niets aan doen er steekt meer kwaad}{in dan wij weten}\\

\haiku{En plotseling zweeg,,.}{zij met de hand voor de ogen}{of zij zich bedwong}\\

\haiku{Als je je verstand,.}{verliest ga dan naar je huis}{en denk erover na}\\

\haiku{Meer dan twintig jaar,.}{is zij hier geweest al voor}{je geboren was}\\

\haiku{En als stelen niet,.}{genoeg is dan zal ik nog}{wel wat anders doen}\\

\haiku{Werendonk rees, groot stond.}{hij voor Floris die week en}{de stoel liet vallen}\\

\haiku{Dan baden zij te.}{zamen en Floris ging met}{vochtige ogen heen}\\

\haiku{Toen hij erheen ging.}{de eerste morgen voelde}{hij een verlichting}\\

\haiku{iedere avond hoor.}{ik dat gefluit en anders}{is het hier zo stil}\\

\haiku{Elke avond wanneer.}{hij langskwam stak zij het hoofd}{even uit het venster}\\

\haiku{de moeder was kort.}{zoals Wijntje en droeg een}{muts met keelbanden}\\

\haiku{Je mag niet weggaan,,.}{fluisterde zij weer anders}{komt er ongeluk}\\

\haiku{Maar met volharding,,.}{in het geloof dacht hij wordt}{de ziel behouden}\\

\haiku{Kort en goed, wij zijn}{gekomen om tegen je}{vader te zeggen}\\

\haiku{Alleen het zingen.}{van de treurige liedjes}{kon zij niet laten}\\

\haiku{Kom, zeide zij, help,.}{mij maar liever wat stuk is}{wordt wel weer gemaakt}\\

\haiku{Zeg mij eens, wat heb?}{je op het hart dat je zo}{ongedurig bent}\\

\haiku{zou als er op zijn.}{kantoor een dief geweest was}{die niet gestraft was}\\

\haiku{Om elf uur vond hij.}{zijn broer aan de tafel met}{de bijbel voor zich}\\

\haiku{Maar Werendonk wilde.}{geen ander in de winkel}{of bij hem aan bed}\\

\haiku{Zij was het die hem.}{zijn eten en melk moest brengen}{en de kamer doen}\\

\haiku{Wat zingt die Stien toch,,,?}{zeide hij honderduit waar}{heeft zij dat geleerd}\\

\haiku{Jongen, wij doen voor,.}{je wat wij kunnen maar er}{hindert je nog iets}\\

\haiku{Eens nam hij hem 's}{avonds mee naar de Bavo om}{hem te laten zien}\\

\haiku{En al voelde hij,.}{de moeheid hij stelde het}{uit naar huis te gaan}\\

\haiku{Hij zou navragen.}{wanneer er een boot vertrok}{en wat het kostte}\\

\haiku{Zij staarde naar hem,.}{zij zag hoe wit zijn gezicht}{was in de schemer}\\

\haiku{Alleen om weg te.}{komen uit dat huis had hij}{het geld genomen}\\

\haiku{Je zal toch niet zo,,}{dom zijn hem weer in huis te}{nemen zeide zij}\\

\haiku{als hij gek is laat.}{hem dan oppakken en doe}{hem in Meerenberg}\\

\haiku{Hij wachtte in het,.}{lamplicht aarzelend of hij}{naar hem toe zou gaan}\\

\haiku{Je moet niet denken,.}{dat ik om het brood kom maar}{ik moest in huis zijn}\\

\haiku{Zij stond vlugger op,.}{dan Stien zij was het die een}{stuk brood voor hem sneed}\\

\haiku{Dan komt pas de rust.}{in huis waar je recht op hebt}{met de oude dag}\\

\haiku{Die hem goed kenden.}{merkten dat de gedachten}{hem bezighielden}\\

\haiku{Waar hij keek zag hij,.}{de gezichten groter bleek}{met de ogen donker}\\

\haiku{op Hollands papier,,.}{bij Nijgh en Van Ditmar nv}{te Rotterdam 1934}\\

\subsection{Uit: Verzameld werk. Deel 5}

\haiku{Hij wilde er meer,.}{van weten hij wilde er}{alles van kennen}\\

\haiku{Kaap Bojador was.}{de eerste grote naam op}{de nieuwe weg}\\

\haiku{De golven stonden,.}{daar te hoog een notedop}{kon er niet varen}\\

\haiku{De afstand tussen.}{ontwerp en uitvoering werd}{langzaam overwonnen}\\

\haiku{De nieuwe weg naar.}{het Oosten was al een eind}{verder gevonden}\\

\haiku{De Portugezen,.}{kwamen met een zinnebeeld}{gehaat en gevreesd}\\

\haiku{En het volk vindt een.}{bestaan met het vlechten van}{strooien hoeden}\\

\haiku{Columbus stelde.}{voorwaarden die de vorsten}{voetstoots verwierpen}\\

\haiku{de waardigheid van;}{onderkoning over al de}{te vinden landen}\\

\haiku{Maar toen hij het land.}{eenmaal gevonden had heerste}{de winzucht vooraan}\\

\haiku{Nombre de Dios,.}{was nu op zijn hoede daar}{had hij geen kans}\\

\haiku{Het was het begin.}{der vestigingen waar de}{Engelse vlag woei}\\

\haiku{Maar ongestoord van.}{de vroegere bezitters}{regeerde hij niet}\\

\haiku{Zeildoek was er niet.}{te krijgen in kleden van}{voldoende lengte}\\

\haiku{Altijd wees zij er,.}{de opziener \'e\'en aan wiens}{naam dan gemerkt werd}\\

\haiku{Hij vestigde zich.}{in Londen en heette daar}{de rijkste emigr\'e}\\

\haiku{Aan boord zag men een.}{menigte inboorlingen}{het kamp omringen}\\

\haiku{Gedurende de,.}{nacht brandde er een vuur waar}{een schildwacht bij stond}\\

\haiku{Hiermede was het.}{roemrijkste avontuur van zijn}{leven ge\"eindigd}\\

\haiku{Een konijn is wel.}{het ergste dat iemand op}{het schip kan brengen}\\

\haiku{Alweer mijn schuld, het.}{is maar beter je nergens}{mee te bemoeien}\\

\haiku{'t Is niets, man, was,.}{het antwoord een stakkerd die}{naar huis gebracht wordt}\\

\haiku{Gaan jullie nu maar,.}{gauw naar binnen hier is het}{geld voor  de nacht}\\

\haiku{De halzen rekten,.}{over andermans schouders het}{woord goudtientje klonk}\\

\haiku{Maar nu ondervond.}{hij de moeilijkheden van}{de vertroeteling}\\

\haiku{Een fraai handschrift werd,.}{het genoemd hoewel niet veel}{om op te bogen}\\

\haiku{Ja, dat is zo, kreeg,.}{hij ten antwoord een ieder}{doet het naar zijn aard}\\

\haiku{Het was er \'e\'en meer.}{in het koor van toezieners}{en bemoeials}\\

\haiku{De schoonzoon, gaf Blauw, '.}{ten antwoord was het bests}{avonds thuis te vinden}\\

\haiku{Zij wist dat hij op.}{dit kind meer gesteld was dan}{op de andere}\\

\haiku{hoe zij zich daarvan,.}{vrijwaren moest tevens van}{de krenking der eer}\\

\haiku{Toen hij zweeg was het,.}{stil alleen een wagen op}{de brug te horen}\\

\haiku{Zij durfde niet te,.}{vragen maar zij wachtte op}{een gelegenheid}\\

\haiku{Wat in de hand is.}{moet eruit en de lege}{hand wil weer vol zijn}\\

\haiku{Zij was het echter.}{niet die zich dwingen moest om}{de maat te houden}\\

\haiku{En dacht je dat het,?}{in de graanhandel beter}{ging met die prijzen}\\

\haiku{Misschien hebt u een,.}{erfenis te wachten maar}{dat is geen waarborg}\\

\haiku{Ik wil je anders.}{wel een handje helpen als}{je er wat op weet}\\

\haiku{, kan het altijd nog}{wat beter en als het je}{aangeboden wordt}\\

\haiku{En van die nacht af.}{hebben wij onze opkomst}{aan hem te danken}\\

\haiku{Geesje zat daar veel}{op de kamer en wanneer}{zij beneden kwam}\\

\haiku{Tegen u kan ik,.}{het wel zeggen dat ik mijn}{hart voor ze vasthoud}\\

\haiku{En mijnheer gelooft.}{dat het alles waar is wat}{ze hem voorpraten}\\

\haiku{En geld verdienen,.}{met papieren dat leert hij}{zo op dat kantoor}\\

\haiku{Niemand zag er hem,.}{op rijden niemand wist waar}{hij het bewaarde}\\

\haiku{Ja, zeide hij, het,.}{is waar maar het ging zonder}{dat ik het merkte}\\

\haiku{Ze noemen hem de,,.}{rijke man misschien uit spot}{en dat is hij ook}\\

\haiku{Maar kind, zei hij, lig?}{je jezelf te plagen met}{zulke gedachten}\\

\haiku{En zij zweeg, zij keek,.}{met een lachje maar haar hoofd}{bleef ermee bezig}\\

\haiku{Zij hield haar de arm.}{om de hals zoals zij deed}{toen zij nog klein was}\\

\haiku{Ik heb u ook niet.}{meer gezien nadat uw kind}{u ontvallen is}\\

\haiku{Ach daar weet ik van,.}{mee te praten ik zit hier}{al zo lang alleen}\\

\haiku{Er is berouw van,,.}{mijnheer meer dan ik met die}{pijnen zeggen kan}\\

\haiku{Er was hun, zeide,.}{hij door de weldadigheid}{al zoveel ontgaan}\\

\haiku{Een zekere som.}{mocht hem pas na vijf jaren}{worden uitgekeerd}\\

\haiku{Toen werd er druk aan.}{de deur gescheld en Tonia}{was ook niet jong meer}\\

\haiku{En als u mij een,.}{pleizier wil doen laat mij uw}{geld dan bewaren}\\

\haiku{En nu die man zich.}{aan de drank overgeeft is hij}{tot alles in staat}\\

\haiku{Mijnheer, zei hij, ik,}{dacht dat u thuis zat want ik}{zag daarnet mijnheer}\\

\haiku{Plotseling vielen,.}{er klappen Kompaan zag niet}{wie ermee begon}\\

\haiku{De boekjes lopen,,.}{op zeg je en we hebben}{niets om te geven}\\

\haiku{Toen hij gegeten.}{had hoorde hij iets in het}{kamertje boven}\\

\haiku{Hij deed de handen.}{voor de ogen en legde het}{hoofd op de tafel}\\

\haiku{Maartje, hoe heb je het,?}{kunnen doen jij de enige}{die niet gevraagd hebt}\\

\haiku{Mijnheer liet het door,.}{Maartje houden maar in haar kast}{was niets gevonden}\\

\haiku{Het is beter bij,.}{jou bewaard anders brengt het}{maar verleiding voort}\\

\haiku{Alleen wat zilver.}{heb ik behouden voor de}{onverwachte gast}\\

\haiku{Onder een lantaarn.}{toonde hij hun een handvol}{nieuwe dubbeltjes}\\

\haiku{Maar waarom blijft u?}{dan geloven wat uw zoon}{u heeft voorgepraat}\\

\haiku{uw eigendom en.}{zolang mijnheer Engelbertus}{rondloopt blijft dat zo}\\

\haiku{Ik kon helderder.}{zien dan menig ander van}{mijn tijdgenoten}\\

\haiku{En hij zocht aan de.}{muren naar een spoor waar de}{bliksem was gegaan}\\

\haiku{Hij zat te suffen,,.}{met hoofdpijn zoals altijd}{op zulke dagen}\\

\haiku{Van de jongens in.}{de klas wist hij de namen}{en anders weinig}\\

\haiku{Toen hij weer naar de.}{stad kon gaan reisde juffrouw}{Amalia met hem mee}\\

\haiku{Naar de kerk wilde,.}{hij niet hij had gezegd dat}{men thuis kon bidden}\\

\haiku{Op een avond zette}{zij haar stoel dicht naast hem en}{terwijl zij sprak hield}\\

\haiku{Heiltje richtte zich,:}{op met een kreet van schrik zij}{werd bleek en zij riep}\\

\haiku{De redenen, vrouw,.}{die zal je zien zodra je}{verstand verlicht wordt}\\

\haiku{Nu heeft je man een.}{plicht te doen en die plicht heb}{je mee te dragen}\\

\haiku{Dat zij een hekel.}{aan hem hadden hoefde niet}{gezegd te worden}\\

\haiku{Plotseling rees er.}{luide twist tussen man en}{vrouw en zij liep weg}\\

\haiku{Kleed je eens netjes,.}{aan en kom mee wij hebben}{met je te spreken}\\

\haiku{Het zal er dus van,.}{moeten komen al gaat het}{mij tegen het hart}\\

\haiku{Zij dankte Blok dat:}{hij haar weer geholpen had}{en bij zijn woorden}\\

\haiku{Dan kan ik bijna.}{zeker zijn dat het verkeerd}{gaat met allebei}\\

\haiku{Hij was moeilijk te,.}{bedaren maar eindelijk}{kon hij het zeggen}\\

\haiku{Zij moesten bekrimpen.}{en voor de winter hoopte}{hij op de bakteelt}\\

\haiku{Dikwijls was het stil.}{in zijn hoofd wanneer hij daar}{de druk niet voelde}\\

\haiku{Veel onkruid bleef er,.}{die zomer weg de aarde}{zag er schoner uit}\\

\haiku{moest hij een deel van?}{zijn kracht nutteloos laten}{in de voldaanheid}\\

\haiku{Ze kijken alleen,.}{maar ze bespieden wat er}{in mijn hoofd omgaat}\\

\haiku{Het moet, daar kan je,.}{zeker van zijn ik zie dat}{het geschreven staat}\\

\haiku{De man heeft niets te.}{doen als piekeren en daar}{wordt het hoofd moe van}\\

\haiku{Toen het beter ging.}{was hij nog moeilijker van}{zijn plaats te krijgen}\\

\haiku{Ik dacht anders dat,.}{degene die beproefd wordt}{ervan verbetert}\\

\haiku{Waarom dan moest ik?}{ondankbaar worden voor wat}{mij gegeven werd}\\

\haiku{Leentje lachte even:}{en zij liet  haar hoge}{stemmetje horen}\\

\haiku{Als je denkt dat het,,.}{moet zeide hij kan ik je}{niet tegenhouden}\\

\haiku{Dan was zij niet zo.}{dom geweest haar zoon naar de}{slachtbank te sturen}\\

\haiku{Misschien wachten ze.}{daar al op me. Als mijn broer}{mij maar niet loslaat}\\

\haiku{Het kind vroeg toen of.}{haar vader niet even bij haar}{kon komen zitten}\\

\haiku{We pikken wat, we,,.}{zoeken wat we vinden wat}{we vliegen verder}\\

\haiku{Het was te dwaas te.}{denken dat de plaats daar iets}{mee te maken had}\\

\haiku{Ik was nog een kind,}{toen men zei dat het in zou}{storten zo oud is}\\

\haiku{Bid zoveel je wil,.}{maar waar de duivel woont zal}{het je niet helpen}\\

\haiku{Sofie ging er wel,.}{heen om hem toe te spreken}{maar het hielp niet veel}\\

\haiku{Wel was het moeilijk.}{zich te bedwingen en de}{deur voorbij te gaan}\\

\haiku{Blok kwam dichter bij.}{hem staan en merkte dat hij}{niet gedronken had}\\

\haiku{Met de dikke stok,,:}{die hij nu had langs de weg}{wandelend dacht hij}\\

\haiku{En zo komt er ook.}{een eind aan de ellende}{die ik gezien heb}\\

\haiku{De knecht, menende,.}{dat zij vochten greep hem aan}{en rukte hem los}\\

\haiku{, want behalve de.}{zieke en de werkvrouw was}{er niemand overdag}\\

\haiku{Een ieder, zelfs de,.}{gildeknecht sliep in een bed}{en at van een bord}\\

\haiku{Het was nu eenmaal.}{zo dat voor zorgeloosheid}{betaald moest worden}\\

\haiku{Na een verkenning.}{van de straten komen de}{pleinen aan de beurt}\\

\haiku{Had Dante misschien?}{nog een andere reden}{voor zijn misnoegen}\\

\haiku{Met de cynici.}{en de sto{\"\i}cijnen was}{het erger gesteld}\\

\haiku{Hij bleef de hele,,.}{nacht wakker maar zij zeide}{niets zij snurkte slechts}\\

\haiku{Zij kenden meer dan,.}{de verschillen van blad en}{hout van schors en sap}\\

\haiku{Vroeger danste men.}{om de eerste meidoorn die}{in volle bloei stond}\\

\haiku{Alleen omdat er.}{po\"ezie was met iets dat}{al lang voorbij is}\\

\haiku{Waag het niet een glad.}{gazon te maken voor gij}{kennis daarvan hebt}\\

\haiku{Vrees niet voor het kind.}{want er is in je land een}{kruid voor gewassen}\\

\haiku{steek, tussen april en,;}{augustus uw haak in de}{mond van de kikvors}\\

\haiku{Ook bij velen der.}{latere zeeschuimers was}{de vroomheid een trek}\\

\haiku{Zij vlogen uit, zij:}{vonden Pili zwemmend op}{de zee en hij sprak}\\

\haiku{Neen, zeide zij, hoe.}{kon ik zo verdwalen dat}{ik hun letsel deed}\\

\haiku{Gag\^atiomes, git,,,.}{hetzij dof hetzij glanzend}{verdrijft demonen}\\

\haiku{waarom, gedachten.}{zonder einde en erger}{dan al het geween}\\

\haiku{Als de jonge maan,,.}{helder glanst dacht hij is er}{mooi weer in aantocht}\\

\haiku{Wij hoorden mijnheer.}{praten met iets erger dan}{wij denken kunnen}\\

\haiku{En samen was het.}{toen twee dagen veel geven}{en nog meer vragen}\\

\haiku{De stille witte:}{man verscheen uit de dag en}{knielend zeide hij}\\

\haiku{Hij hield zijn hand naar,.}{het licht van de wolken hij}{wist niet wie hij was}\\

\haiku{Uw hart ruikt bitter,,.}{uw kleur is groen gij zijt oud}{van het zoeken}\\

\haiku{Het oog, zeide een,.}{gedachte neemt de kleuren}{van de planten waar}\\

\haiku{Als de kleur verdwijnt.}{in het zwart van de nacht ziet}{het geen vormen meer}\\

\haiku{Toen zij naast hem stond.}{ging er een koelte en de}{bladeren trilden}\\

\haiku{de zomer was lang.}{voorbij en de duisternis}{hield al te lang aan}\\

\haiku{De wachters stonden,.}{arm aan arm de hoorns bliezen}{naar alle streken}\\

\haiku{Dan moet het een nieuw,,.}{lied zijn dacht hij maar hij kon}{niets nieuws verzinnen}\\

\haiku{Misschien was zij geen,.}{muze maar wat zij wel was}{kon hij niet raden}\\

\haiku{Jonge man, zeide,,,}{zij als je van de liefde}{weten wilt volg mij}\\

\haiku{Kom mee, zeide zij.}{en ik ging achter haar of}{ik geen benen had}\\

\haiku{Wist ik dan niet dat?}{de lach en de dwaasheid maar}{een ogenblik duren}\\

\haiku{die het verlangen?}{heeft geschud en de rimpels}{weer glad zal strijken}\\

\haiku{Is het straf dat ik?}{in mijn tijd van pijn begeerd}{heb en genomen}\\

\haiku{Neen, ik beef, maar ik.}{zal niet vrezen zodra ik}{weer geroepen word}\\

\haiku{Het is om gek te,,.}{worden ja daar is het ook}{zeker voor bedoeld}\\

\haiku{Alleen een rare.}{knoop in je hoofd dat je er}{niet van spreken kan}\\

\haiku{Het was al mooi dat.}{men zich kindergezichtjes}{herinneren kon}\\

\haiku{bromde de man die.}{gisteren geld verloren}{had en wakker lag}\\

\haiku{Maar zij moest de ogen.}{sluiten toen opeens zo veel}{muziek haar omving}\\

\haiku{Wat blijft er van de?}{dag van heden anders dan}{vragen zonder eind}\\

\haiku{dan een hart waar de?}{tranen vloeien tot morgen}{en de dag daarna}\\

\haiku{Het kan vreemd lopen.}{in de wereld en het lot}{heeft rare grillen}\\

\haiku{Ik moet erkennen.}{dat zij mij van het eerste}{ogenblik bekoorde}\\

\haiku{De wereld is een -.}{dansfeest en wie niet danst een}{domoor un tonto}\\

\haiku{Maar we hadden ook.}{al gauw van elkaar ontdekt}{dat we graag dansen}\\

\haiku{Het noodlot had mij.}{tussen die Deursting rechts en die}{Jonas links gezet}\\

\haiku{Toen hij hoorde dat,.}{er straks bal zou zijn werd hij}{opeens levendig}\\

\haiku{Toen vroeg hij weer of,.}{ik het menuet kon daar}{hield hij zoveel van}\\

\haiku{Het gekste is wat die.}{kleine Marion zich in}{het hoofd heeft gehaald}\\

\haiku{Zij begreep niets van,.}{de maten zij dacht dat zij}{het nooit zou leren}\\

\haiku{Gelukkige tijd,,.}{die kindertijd als men niet}{weet met wie men danst}\\

\haiku{Met de derde weet.}{ik nog steeds niet wat voor vlees}{ik in de kuip heb}\\

\haiku{Ja, mijnheer, - op een.}{toon dat de anderen hun}{lachen verbergen}\\

\haiku{Alle stoelen uit.}{de eetkamer werden naar}{het salon gehaald}\\

\haiku{Het waren bijna.}{allemaal meisjes die zich}{aanmeldden als lid}\\

\haiku{Des te beter, zei,.}{Walewijn dan neem ik die}{twee kamers achter}\\

\haiku{En Petronel is,?}{toch eigenlijk van mij geen}{familie meer wel}\\

\haiku{Petronel en ik,.}{zaten in de veranda}{we hoorden alles}\\

\haiku{Zo, dacht ik, voor zo.}{gemeen had ik Petronel}{toch niet aangezien}\\

\haiku{Ik ben haar vader,.}{het spreekt dus vanzelf dat ik}{voor haar blijf zorgen}\\

\haiku{Ik had bedacht dat}{het beter was geen gesprek}{met hem te voeren}\\

\haiku{op mijn kamer en}{het eenvoudigste was hem}{zonder omwegen}\\

\haiku{Het eerste, ik zei,,.}{het al is de dans die hem}{helemaal vervult}\\

\haiku{Bij het dansen van.}{de bolero schijnt fractuur}{niet zeldzaam te zijn}\\

\haiku{Ziedaar, mijnheer en,.}{mevrouw alles wat ik u}{kan mededelen}\\

\haiku{Het ging mij aan het.}{hart dat ik haar alle}{hoop ontnemen moest}\\

\haiku{Het is lastig, zei,.}{Frans maar een Ringelinck}{geeft nooit de moed op}\\

\haiku{Een ieder zoekt het,,.}{geluk dat spreekt vanzelf en}{zoveel mogelijk}\\

\haiku{Dit bevestigde.}{ons vermoeden dat het een}{jonggehuwd paar was}\\

\haiku{En zij keek rond naar,,.}{de bloemen en zij keek hem}{aan zonder een woord}\\

\haiku{Als men de dames.}{kleedt ziet men gauw het verschil}{van de karakters}\\

\haiku{Toen de koorts afnam.}{vertrouwde zij mij veel van}{haar geheimen toe}\\

\haiku{Op een morgen vond,.}{ik haar wakker met een kleur}{en blinkende ogen}\\

\haiku{Samen lukte het.}{ons mademoiselle weer}{naar bed te krijgen}\\

\haiku{En dan tilde hij:}{de slippen van zijn jacquet}{weer hoog op en zei}\\

\haiku{Zo iets te zien is.}{misschien mooier dan het te}{ondervinden}\\

\haiku{Ook deze typen.}{waren niet nieuw voor wie in}{de streek bekend is}\\

\haiku{Spoedig had ik ook.}{reden mijn verblijf hier niet}{langer te rekken}\\

\haiku{Toen hij om halfacht:}{in de morgen mij mijn thee}{bracht zei  Woode}\\

\haiku{De comtesse scheen,.}{hij niet te kennen terwijl}{ik toch beter wist}\\

\haiku{Toen ging zij terug,.}{met een gezicht of zij die}{vent verafschuwde}\\

\haiku{Stil, fluisterde hij.}{en hij voerde mij bij de}{arm naar het venster}\\

\haiku{Uw partner zal er.}{ongetwijfeld waardig en}{sierlijk mee dansen}\\

\haiku{omdat ik ze zelf.}{voor een jaar of drie aan haar}{vader had verkocht}\\

\haiku{Nathalie kwam ook,:}{nog binnen en ook mijn vrouw}{die mij uitlachte}\\

\haiku{Marion, zei ik,}{ik heb je verzocht vanavond}{bij mij te komen}\\

\haiku{Ja, ik weet dat u}{daar ook van overtuigd bent en}{dat verwondert me.}\\

\haiku{zo zeker weet ik.}{dat ze voor Daniel nooit de}{liefde heeft gekend}\\

\haiku{Het was gedaan voor.}{ik het wist en ik had er}{een gloeiend hoofd van}\\

\haiku{En Mr. Sedge dacht.}{ook dat het pijn doet als het}{ritme zo diep zit}\\

\haiku{Toen die kwam zei hij.}{alleen dat mevrouw het heel}{rustig moest hebben}\\

\haiku{Bij de eerste maat.}{al was het of ik van mijn}{stoel moest opspringen}\\

\haiku{Lewis bij ze thuis,.}{en kocht een paar dingen die}{ze goed betaalde}\\

\haiku{Nu, die heer kwam dan.}{vertellen dat ze niets te}{verwachten hadden}\\

\haiku{En zo weinig idee}{van  geld hadden ze dat}{ze niet eens vroegen}\\

\haiku{Zij haalde er de:}{schouders over op en tegen}{mij zei ze later}\\

\haiku{Ik kreeg er drie pond.}{op en daar konden ze een}{paar weken mee door}\\

\haiku{Op die morgen met,,}{dikke mist toen Moralis}{was gestorven om}\\

\haiku{Ik heb dit rustig.}{landschap voor mijn venster en}{doe niets dan vragen}\\

\haiku{Zij duwde mij van.}{zich af omdat zij dacht dat}{ik uit de maat was}\\

\haiku{Wij hebben elkaar,.}{meer gezien voor het eerst al}{jaren geleden}\\

\haiku{Ik zag dat hij het.}{gezicht gewend hield naar het}{hek en nog omkeek}\\

\haiku{omdat wij meenden.}{een ander te horen die}{in de verte floot}\\

\subsection{Uit: Verzameld werk. Deel 6}

\haiku{Kom nu maar naar mijn.}{woning mee en geleid de}{jongskens veilig thuis}\\

\haiku{En Filips maakt daar.}{grapjes over en dan krijgen}{ze weer gekibbel}\\

\haiku{Het is dat geplaag,.}{met zijn broer maar ik zal er}{niet meer van zeggen}\\

\haiku{Moll moet niets van haar,?}{hebben maar hoe is ze met}{je eigen meisjes}\\

\haiku{Je kon wel gelijk.}{hebben dat Filips wat al}{te zorgeloos is}\\

\haiku{En ik zit dikwijls.}{te denken hoe het met mijn}{kinderen zal gaan}\\

\haiku{Het was een lied van.}{vele strofen en de zang}{werd allengs vaster}\\

\haiku{Bijna iedere.}{morgen ook kwam hij ergens}{de tuinbaas tegen}\\

\haiku{Guldelingh wilde.}{nog meer van de vrolijkheid}{van zijn dochter zien}\\

\haiku{De taal heb ik niet,.}{verstaan  maar ik heb het}{niettemin gehoord}\\

\haiku{Geen dag dat zij niet,.}{voor ze leest en soms nog een}{hoog liedeken zingt}\\

\haiku{Dit zou alles niets.}{zijn als ge zijn gezicht niet}{had waargenomen}\\

\haiku{Meen niet dat iemand.}{mij naderen kan zonder}{gehoord te worden}\\

\haiku{Daar heb je het, denkt,,.}{De Kroon jaloezie en jij}{denkt dat zeker ook}\\

\haiku{En dat idee heeft hij,,.}{niet van nu pas neen hij loopt}{er al lang mee rond}\\

\haiku{En het is soms in.}{het holle van de nacht dat}{je ze hoort weggaan}\\

\haiku{Het was wel niet de,.}{lieveling maar dit zou toch}{te erg geweest zijn}\\

\haiku{Praten en nog eens,.}{praten dat ze het haar toch}{niet lastig maken}\\

\haiku{Je mag van geluk.}{spreken dat je kind zo kalm}{en rechtgeaard is}\\

\haiku{Dat is ze, zei De,.}{Kroon en we hebben haar nooit}{veel hoeven leren}\\

\haiku{Hij meent dat er nu.}{ook wel iets zal zijn dat het}{daglicht niet mag zien}\\

\haiku{Allemaal omdat.}{hij eigenaardig is en}{van niemand geliefd}\\

\haiku{Hij heeft niets geen kwaad,,.}{in de zin mevrouw daar kan}{u gerust op zijn}\\

\haiku{De koorts houdt te lang.}{aan en het kan zijn dat er}{iets anders bijkomt}\\

\haiku{Wat konden jonge?}{jongens te zoeken hebben}{bij die oude heks}\\

\haiku{En het schijnt dat hij.}{hier alleen komt omdat hij}{met haar spreken wil}\\

\haiku{Maar ge kunt zelf iets.}{waarnemen dat  we niet}{vertrouwen moeten}\\

\haiku{Zij kijken naar de.}{zuidkant en dat op de dag}{van het nieuwe jaar}\\

\haiku{Ik heb het je meer,.}{dan eens verboden dat je}{dicht bij het huis komt}\\

\haiku{Hij weet het, vraag het,.}{hem wie er in de nacht naar}{hem gekeken heeft}\\

\haiku{Vroeger konden we,.}{het wel vinden al had hij}{het achter de mouw}\\

\haiku{nu willen ze weer,,.}{niet ze worden ook zo suf}{altijd over kleren}\\

\haiku{Als je klein bent is,}{het beginsel altijd zoet}{zijn en gehoorzaam}\\

\haiku{Bastiaan klopte.}{ze een voor een op de nek}{en ze volgden hen}\\

\haiku{Wat daarbuiten wacht,,}{daar weten we niet van maar}{kijk eens rond hoe mooi}\\

\haiku{Ik denk dat ik het.}{nog eens waag bij het oude}{mens van de overkant}\\

\haiku{Wat denken jullie,?}{zou ik er niet eens met hem}{over moeten praten}\\

\haiku{Melodidio, zei hij,.}{bij zichzelf daar moet ik de}{baas eens naar vragen}\\

\haiku{Er is nog meer, er,.}{zijn nog dieper dingen maar}{die begrijpt u niet}\\

\haiku{al rondom deze.}{plaats vindt ge er immers geen}{enkele groeien}\\

\haiku{Maar of Blankendaal,.}{erbij gebaat is dat is}{een andere vraag}\\

\haiku{En dat is het wat,.}{mevrouw ook al begrijpt ze}{heeft het zelf gezegd}\\

\haiku{Je vraagt of ik dan,.}{mijn oordeel niet gebruik maar}{dat doe ik immers}\\

\haiku{Kijk nog maar goed, de.}{dag is nabij dat die daar}{grote mensen zijn}\\

\haiku{De grond lag vertrapt,,.}{gedeukt bezaaid met stukken}{en splinters wit hout}\\

\haiku{De halve maan stond.}{zo tussen schaapjes dat het}{wel vast zal blijven}\\

\haiku{door de schuld van  ,.}{het voorgeslacht of van de}{sterren weet ik het}\\

\haiku{Maar ter andere.}{is de oorzaak van zijn komst}{mij niet gevallig}\\

\haiku{Wat hij ermee deed,.}{weet ik niet maar Jacob wou}{het hem afnemen}\\

\haiku{'t Is aardig van,.}{een jonge man als hij het}{mooie ervan begrijpt}\\

\haiku{En dan opeens die,.}{geluiden om je eraan}{te herinneren}\\

\haiku{, maar dat zwijgen en.}{de eenzaamheid zoeken zie}{ik toch te dikwijls}\\

\haiku{Altijd voelt hij zich,,.}{treurig zegt hij dat hij wou}{dat hij huilen kon}\\

\haiku{Zo rood als de hel,,?}{zeide er een waarom moet}{hij het juist daar doen}\\

\haiku{Je schrikt er soms van.}{zoals die man hem als zijn}{schaduw achtervolgt}\\

\haiku{Ik ben er zeker,,.}{van zeide hij want Fideel}{heeft het zelf gezien}\\

\haiku{waren we hier niet,.}{bijtijds geweest want Fideel}{gaf dadelijk alarm}\\

\haiku{Maar ik geloof niet.}{dat het goed is hem zonder}{toezicht te laten}\\

\haiku{En hij kon het niet,.}{verdragen daarom is hij}{van hier weggegaan}\\

\haiku{Daar moet Platen dan,.}{op letten het zou beter}{zijn als ze thuisblijft}\\

\haiku{Dat de jongeheer,,.}{geen kwaad in de zin heeft ja}{dat geloof ik wel}\\

\haiku{Toen hoorde ik een,,.}{schreeuw mevrouw gelijk of een}{mens gestoken was}\\

\haiku{Toen hij een tweede:}{stoel had neergezet en zij}{zaten sprak De Kroon}\\

\haiku{Want hij was niet de.}{enige die een verbazend}{avontuur beleefde}\\

\haiku{Hij voelde zich wat,.}{doof en suf met een neiging}{om in te slapen}\\

\haiku{Dat is voor een week.}{genoeg als de andere}{er niet van horen}\\

\haiku{Ik beken, sprak de,.}{worm tot zichzelf dat ik die}{dingen niet begrijp}\\

\haiku{Aan wat heb ik het?}{dan te danken dat ik weer}{in de aarde kruip}\\

\haiku{Misschien kom ik nooit.}{te weten wat de oorzaak}{van die dingen is}\\

\haiku{De koetsier zat op,.}{de bok te slapen het paard}{stond diep gebogen}\\

\haiku{De lantaarns stonden,.}{scheef vlak bij de huizen de}{lichten flikkerden}\\

\haiku{Hij klopte op de,.}{deur drie slagen die elk een}{vonnis beduidden}\\

\haiku{Binnen verstomde,.}{het de waard kwam aan het luik}{om wat te liegen}\\

\haiku{Papa daalde de.}{treden van de stoep af en}{kwam naast het meisje}\\

\haiku{Meisje, kijk ze nog,,.}{eens allemaal aan meisje}{je moet kiezen gaan}\\

\haiku{Hoe buitengewoon,}{hij geweest moet zijn begrijpt}{men als men bedenkt}\\

\haiku{Achter de keuken.}{hoorde hij het fris gebruis}{van een waterval}\\

\haiku{Indien ik niet meer.}{verlies vind ik de nieuwe}{kracht niet voor beter}\\

\haiku{Zonder een woord te}{spreken vatten zij Krijn bij}{de schouders en daar}\\

\haiku{Bovendien, je weet,.}{het de vergelding is in}{de hand van de Heer}\\

\haiku{De trein vertrok nog.}{voor de dageraad en de}{wreker reisde mee}\\

\haiku{Bovendien, zoals.}{je nu bent zou je te veel}{zijn voor de landheer}\\

\haiku{Dit verwijt zond hij,,}{hun terug zeggend dat zij}{niet beter deden}\\

\haiku{Geef mij meer goud dan.}{alle koningen en ik}{breng mijn dochter hier}\\

\haiku{Haar geheimen kent,.}{men nooit hoeveel men ook van}{haar houden mag}\\

\haiku{Hebt gij nooit van de?}{ware God gehoord die ons}{geopenbaard is}\\

\haiku{Drinken als Puymys,:}{een vaste uitdrukking werd}{op de wijze van}\\

\haiku{Als je mij voor een,.}{kwartje melk geeft zal ik mij}{niet prettig voelen}\\

\haiku{Sta je eindelijk,.}{op hoorde Lemmerts zeggen}{toen hij wakker werd}\\

\haiku{'t Is gek, zeide:}{Jonas aan het ontbijt en}{Lemmerts herhaalde}\\

\haiku{Beiden kregen zij.}{van ergernis een groene}{tint op het gezicht}\\

\haiku{Zeker kind, dat heb,,.}{je goed gekozen braaf is}{hij ook welgesteld}\\

\haiku{Ik kan toch met mijn,?}{gewoonte van Anton niet}{zo maar breken wel}\\

\haiku{wanneer ik alles.}{zou krijgen waar ik recht op}{meende te hebben}\\

\haiku{Zo is het, zuchtte,.}{Japperotte dat heb ik}{altijd begrepen}\\

\haiku{Hij wees Jansen een,:}{stoel hij bladerde in het}{schrijfboek en hij sprak}\\

\haiku{In andere tij.}{den heeft men wel gemeend dat}{het ding de naam was}\\

\haiku{U kan mij niet eens.}{zeggen of het een jongen}{of een meisje is}\\

\haiku{Toen gaven al die.}{gapers elkaar de arm en}{dansten om hem heen}\\

\haiku{Hij vond dat haar stem,.}{veel zachter klonk hoewel ze}{toch niet fluisterde}\\

\haiku{Juffrouw Das kon het,.}{niet nadoen ze kreeg er een}{traan van in het oog}\\

\haiku{En oom Hendrik zei.}{iets van de opera en ze}{lachten allemaal}\\

\haiku{Het is zo stil in}{de bomen en de kikkers}{mogen eens horen}\\

\haiku{Wie zegt je dat het?}{niet schadelijk kan zijn ze}{te leren kennen}\\

\haiku{Ik zeg dat, omdat,}{ik erover heb nagedacht}{of het heus zo erg}\\

\haiku{Maar waarom zouden?}{we over de mensen praten}{op zo'n stille avond}\\

\haiku{voor de maan opkomt,.}{daar achter de bomen van}{de nieuwe buren}\\

\haiku{Nu je eenmaal hebt}{toegegeven dat we van}{de mensen houden}\\

\haiku{Als hij maar niet zo.}{lelijk was had ik misschien}{toch nog ja gezegd}\\

\haiku{Wat de liefde is,,.}{ging hij voort ik heb het eens}{horen beschrijven}\\

\haiku{Je schaamt je telkens,.}{over een kleur die je krijgt al}{heb je niets gedaan}\\

\haiku{Schei maar uit, zeide,.}{zij ongeduldig ik heb}{het al begrepen}\\

\haiku{Ze is heel lief, maar.}{het zou niet helemaal mijn}{keuze zijn geweest}\\

\haiku{Er worden er acht.}{gebeten en er komen}{er toch maar twee in}\\

\haiku{Meer hoorde Klaartje,.}{niet van het gesprek want zij}{ging de kamer uit}\\

\haiku{Je hoeft het niet te,,.}{geloven hoor want het is}{maar een fabeltje}\\

\haiku{Toen ik zijn kamer.}{binnenkwam wist ik al dat}{ik hier niet moest zijn}\\

\haiku{Ik moet zeggen dat.}{hij een welwillende man}{was en heel bekwaam}\\

\haiku{En hij keek naar de:}{madelief en hij keek Jan}{aan en hij zeide}\\

\haiku{Nu zou ik weleens.}{van Abram willen weten wat}{voor raad hij Jan geeft}\\

\haiku{Ik moet zeggen dat,.}{je het verstandig inziet}{sprak mijnheer Oberon}\\

\haiku{Als het veel is heb,}{je raad van niemand nodig}{als het weinig is}\\

\haiku{en het is waar dat.}{ik er niet van aangetast}{zou willen worden}\\

\haiku{Als die altijd zijn.}{raad gevolgd had zou het slecht}{met hem gegaan zijn}\\

\haiku{U kan mij van niets,.}{overtuigen behalve van}{uw onbeschaamdheid}\\

\haiku{Mevrouw Oberon stond,.}{op een stoel in de handen}{klappend van pleizier}\\

\haiku{de laatste maanden,.}{kon alleen door zwijgen mijn}{verachting blijken}\\

\haiku{Eens, nadat de deur,:}{weer was toegedaan viel zij}{uit in gefluister}\\

\haiku{Juffrouw Das, aan de,.}{hand de blauwe lampion}{stond al aan het hek}\\

\haiku{In het  midden.}{dansten tot voorbeeld mijnheer}{Oberon en mevrouw}\\

\haiku{Er werd weer gedanst,.}{maar mevrouw Oberon was met}{Fientje weggegaan}\\

\haiku{Ik denk dat wij voor.}{ze moeten doen wat wij voor}{de ouders deden}\\

\haiku{Dat had mij die heks,,.}{gedaan dacht ik en ik kon}{ervan genezen}\\

\haiku{Eerst ontdekte je.}{dat Klaartje koel was en niet}{genoeg van je hield}\\

\haiku{Roep Dina eens om.}{de kandelaar en kom hier}{tussen ons zitten}\\

\haiku{Wat de liefde is,,.}{dat weet je niet dus je moet}{er maar naar raden}\\

\haiku{Wel, zeide mijnheer.}{Oberon met een stem of hij}{begon te zingen}\\

\haiku{Blijf die je bent, het,.}{is genoeg dat jij van hem}{houdt en verwacht niets}\\

\haiku{Hij zat weleens met}{verwondering naar haar te}{kijken als mevrouw}\\

\haiku{Je kijkt me aan of,?}{ik wartaal spreek is het zo}{onbegrijpelijk}\\

\haiku{Klara was stil en.}{zij merkte wel dat mevrouw}{dikwijls naar haar keek}\\

\haiku{Maar mogen wij nog?}{iets voor ze verwachten van}{de dorst naar kennis}\\

\haiku{Vertel mij niet dat,.}{je het niet hebt je kan het}{vinden of maken}\\

\haiku{En zij heeft een drang,.}{naar innigheid maar daar staart}{zij nog in donker}\\

\haiku{Eindelijk zijn we,:}{van het spook verlost zeide}{Ida en Toon zeide}\\

\haiku{haar gewoonte was.}{en toen zij voor hen stond had}{zij een rode kleur}\\

\haiku{Boel, die de rest van}{zijn dagen zal schelden op}{zijn landgenoten}\\

\haiku{Ja, wellicht was zij,,.}{in geen kleding grootgebracht}{wat overdreven preuts}\\

\haiku{Er brak een karaf,,.}{er viel een klap de eerste}{in dit restaurant}\\

\haiku{Het is waar, er zijn.}{er waarvan dit nauwelijks}{te begrijpen is}\\

\haiku{En hij sprak zo zacht,.}{dat zij hem soms moesten vragen}{het te herhalen}\\

\haiku{Aan de deur groette,.}{die man maar mijnheer Faustus}{groette niet terug}\\

\haiku{Vrolijk in troepjes.}{trokken zij van de ene naar}{de andere streek}\\

\haiku{En op een morgen.}{ontdekte hij een berg die}{uit de zee verrees}\\

\haiku{Toen hij met Hassan:}{voor de tent kwam riep hij en}{zeide tot de vrouw}\\

\haiku{Toen hij nog een kind.}{was merkten zijn ouders iets}{vreemds op in zijn ogen}\\

\haiku{De toestanden in.}{het land Assassini\"e waren}{onbegrijpelijk}\\

\haiku{En altijd kwam hij.}{onverwacht al was men nog}{zo op zijn hoede}\\

\haiku{Er zijn wel honderd.}{namen voor en niemand weet}{er het rechte van}\\

\haiku{En ik was niet de.}{enige die de verscholen}{romantiek begreep}\\

\haiku{al was het alleen.}{maar omdat er een woord voor}{zijn persoon ontbreekt}\\

\haiku{Wie ik ben weet ik,.}{nog niet dus heb ik mijn naam}{nog niet gevonden}\\

\haiku{Hij dook neder in,.}{de houding van het roofdier}{dat zijn prooi beloert}\\

\haiku{Sta recht, man, zeg wat.}{de begeerte is en ik}{zal het je geven}\\

\haiku{En de menigte,,:}{danste zong waanzinnig met}{\'e\'en enkele kreet}\\

\haiku{Zij hield het verdriet,:}{voor zich maar eens had zij toch}{iets uitgelaten}\\

\haiku{En erger was dat.}{de eenzaamheid van koude}{vergezeld moest zijn}\\

\haiku{Het is niet de angst,;}{van het schuldig geweten}{dat de straf verwacht}\\

\haiku{niet de angst van de,;}{weerloze die zich omringd}{waant van gevaren}\\

\haiku{De gelegenheid.}{werd mij hier geboden om}{haar te overtuigen}\\

\haiku{Het is ook erger,.}{een gevaar te vermoeden}{dat onzichtbaar blijft}\\

\haiku{Nadat hij mij weer.}{lang had laten wachten deed}{de knecht de deur open}\\

\haiku{Hij had redenen.}{genoeg om zich volkomen}{veilig te voelen}\\

\haiku{Men keek ernaar met,.}{ontzag maar geen mens die ze}{aan zou raken}\\

\haiku{Hierin lag nog een.}{reden waarom zij elkaar}{soms niet begrepen}\\

\haiku{Omdat ze moeten.}{en ook alweer omdat ze}{er schik in hebben}\\

\haiku{Waar ik je heen breng,.}{is het altijd mooi dat kan}{ik je beloven}\\

\haiku{Zij huilde niet, maar.}{zij keek hem recht in de ogen}{om medelijden}\\

\haiku{Daar zat een kleine,.}{vogel in grijs met een blauw}{stipje op de borst}\\

\haiku{Zij hield het kooitje,:}{voor haar mond zij fluisterde}{tegen de vogel}\\

\haiku{Deze zucht, vrienden,.}{was zo onvergankelijk}{als het leven}\\

\haiku{En wat schitterde?}{er aan haar vinger en wat}{blonk er aan haar hals}\\

\haiku{'k Vaar voor zulke '.}{kleinigheden u naart}{gindse dorp niet heen}\\

\haiku{Lieve schone, zei,.}{hij na dat kusje komen}{dagen vol geluk}\\

\haiku{Lieve schipper, zei,,}{ze op dat mandje is mijn}{tante zo gesteld}\\

\haiku{Ik zat aan de wand,.}{de draden vielen aan mijn}{voeten en ik sprak}\\

\haiku{Van mijn zes zusters.}{is er een ongetrouwd en}{daarbij tevreden}\\

\haiku{Maar wie nieuwsgierig}{is kan zelf onderzoeken}{hoevele termen}\\

\haiku{Ze hebben het van,.}{hun pa want oom Godfried is}{ook een leugenaar}\\

\haiku{Daar het mooi weer was.}{gingen wij langzaam toen ik}{Alethea naar huis bracht}\\

\haiku{En dat gezegde,,?}{van Slingewiel vroeg ik wat}{betekende dat}\\

\haiku{Je vergist je zelfs,.}{drie keer twee met Socrates}{en \'e\'en met Willem}\\

\haiku{Ik heb niets op je,.}{te zeggen we zijn juist heel}{tevreden over je}\\

\haiku{Toen juffrouw Verkijk:}{vroeg of wij het ook niet guur}{vonden zeide hij}\\

\haiku{Die vent heeft je stuk,,.}{ook gelezen voegde hij}{erbij pas maar op}\\

\haiku{met  een brief in,.}{de hand die hij voor hem op}{de tafel legde}\\

\haiku{Dat had ik trouwens,.}{op zijn leeftijd evenmin heb}{ik misschien nog niet}\\

\haiku{- Dat zij het begreep,,}{natuurlijk begreep kon ik}{aannemen daar zij}\\

\haiku{Zeker, antwoordde,.}{ik je bent even scherpzinnig}{als juffrouw Verkijk}\\

\haiku{De anderen zijn.}{degenen die het geld uit}{hun laden missen}\\

\haiku{Houd daar dus maar je,.}{mond over anders kost het mij}{ook nog mijn baantje}\\

\haiku{De knecht gaf ons de.}{hoeden terug en opende}{weer langzaam de deur}\\

\haiku{Hij ging toch omdat,,.}{zeide hij hij zich aan de}{afspraak moest houden}\\

\haiku{Willem ook besloot.}{te verhuizen en zocht een}{kleinere woning}\\

\haiku{Duizend gulden, ik.}{zou niet weten waar ik het}{vandaan moest halen}\\

\haiku{Ons komt het oordeel,.}{niet toe wij hebben alleen}{de plicht wel te doen}\\

\haiku{hetzelfde zag wat.}{hij mij eens gewezen had}{op een lentenacht}\\

\haiku{Doebel zit bij hem.}{in de schuld en daarvoor heeft}{hij dat geld gebruikt}\\

\haiku{Ik had zelf een plan.}{dat zich in de loop van dit}{voorjaar had gevormd}\\

\haiku{Moet men geloven?}{aan geesten gebonden aan}{een plek der aarde}\\

\haiku{Willem kwam in de.}{deur en keek met een glimlach}{naar de hemel}\\

\haiku{Daar zit meer achter,.}{dan u denkt wij zullen zien}{of dat zomaar kan}\\

\haiku{Je weet er niets van,,.}{zei ze je deugt ook niet voor}{de samenleving}\\

\haiku{Hij leeft nu voor niets.}{anders dan voor wat er in}{de gedachten is}\\

\haiku{Van haar ogen begrijp,,}{ik niets al houdt zij ze nog}{zo lang voor mij open}\\

\haiku{Kijk dan maar eens naar,,.}{die neef van je zeide ze}{en die professor}\\

\haiku{Ik begrijp niet dat.}{Willem die groene vent bij}{zich in huis ontvangt}\\

\haiku{Het is een slecht mens,,.}{gaf hij toe daar heb ik mij}{lelijk in vergist}\\

\haiku{dat zijn vrienden hem.}{een voor een verlaten en}{hij alleen zal staan}\\

\haiku{Zeker, hij is er.}{nog een uit de grote hoop}{die mij verrast heeft}\\

\haiku{Ik mocht vooral de,:}{hoop niet verliezen Alethea}{had ook al gezegd}\\

\haiku{Ja, zegt die plant, er.}{zijn er maar een paar die het}{voornaamste weten}\\

\haiku{Nu ik ouder ben.}{spreek ik zelfs meer dan vroeger}{in de verbeelding}\\

\haiku{- Was het maar waar, zei,.}{Alethea dan zou iedereen}{ook wel deugdzaam zijn}\\

\haiku{van Hein kreeg ik een.}{tekening van de kroon van}{de Westertoren}\\

\haiku{Wie mij via de post.}{twee spotprenten op Willem}{toezond weet ik niet}\\

\haiku{Bijna iedere,.}{dag ging ik naar Haarlem soms}{overnachtte ik er}\\

\haiku{Het kwam mij voor dat.}{hij met een verbijsterde}{geest gesproken had}\\

\haiku{Maar terwijl ik het.}{zeide was het mij of ik}{iets voelde breken}\\

\haiku{Dus je ziet, zeide,,.}{ik al is hij ziek hij leeft}{toch met alles mee}\\

\haiku{Haar ogen tonen dat}{er tranen geweest zijn die}{zij vergeten wil}\\

\haiku{Ach, mijn ouders zijn,.}{er immers ook niet meer en}{zo veel anderen}\\

\haiku{{\textquoteleft}De godloochenaar{\textquoteright},,,,.}{in Groot Nederland 1934 jg.}{32 dl. II blz. 393}\\

\subsection{Uit: De waterman}

\haiku{Hij ging vlug naar het,.}{lichaam toe hij stampte er}{op met zijn geweer}\\

\haiku{zij gezichten en.}{Klaas Tiel deed hem na zooals hij}{uit den bijbel las}\\

\haiku{Hij keek beiden aan,.}{en hij wilde iets vragen}{maar hij wist niet wat}\\

\haiku{Dan bleef hij alleen.}{maar kijken naar den kuil die}{vol was geloopen}\\

\haiku{Toen hij hoorde dat}{zij met den postbode over}{het ijs zouden gaan}\\

\haiku{Maarten hoorde het,.}{en bad hij bleef nu staan tot}{het schieten ophield}\\

\haiku{En elken dag had,.}{hij meer gehoord en eerder}{dan iemand anders}\\

\haiku{Telkens vroeg zij iets,.}{met haar kinderstem dan kwam}{haar adem aan zijn wang}\\

\haiku{Hij wist niet hoe hij,.}{het zeggen moest hij keek haar}{aan en zij wachtte}\\

\haiku{Dan liep hij nog een,.}{eind den dijk op de witte}{maan stond nevelig}\\

\haiku{Maarten lag wakker.}{lang nadat de torenklok}{twaalf had geslagen}\\

\haiku{En dat moest men maar.}{verdragen en werk zoeken}{voor het brood alleen}\\

\haiku{Hij schepte ieder,.}{de aardappelen op het}{bord zij aten zwijgend}\\

\haiku{Het mijne is het,,:}{uwe dat is onze manier}{zooals geschreven staat}\\

\haiku{Het is niets, vrouw, de.}{Heer slaat daar minder acht op}{dan op jou koffie}\\

\haiku{Op een morgen dat}{Rossaart aan den anderen}{oever wachtte zag}\\

\haiku{De commandant was.}{een bejaarde kapitein}{van de schutterij}\\

\haiku{Op den hoek kwam hij,.}{zijn broer Hendrikus tegen}{die voor hem staan bleef}\\

\haiku{Bij de plecht stond de,.}{kolenpot het aardewerk}{hing er aan spijkers}\\

\haiku{Velen wisten van:}{hen te vertellen en hun}{faam verergerde}\\

\haiku{De wijnkooper had.}{haar in huis genomen en}{zocht een dienst voor haar}\\

\haiku{Hoewel hij toen den}{vollen wind had maakte het}{woelende water}\\

\haiku{hout, spijkers, verf gaf.}{Seebel hem in ruil voor zijn}{aandeel in de tjalk}\\

\haiku{Maar nu je toch hier.}{bent zal ik je zeggen wat}{wij van je denken}\\

\haiku{daar staat de dood voor,,,.}{mij goed ik geef mij over er}{is niets aan te doen}\\

\haiku{Hij boog het hoofd en.}{staarde door de hor naar den}{nevel over de gracht}\\

\haiku{Hij staarde naar het,.}{ijs op de ruit hij dacht en}{schudde soms zijn hoofd}\\

\haiku{Hoe verder de schuit.}{voer zoo meer wendde Marie}{het hoofd naar achter}\\

\haiku{Van toen aan merkten,.}{zij een verschil hoewel zij}{het niet beseften}\\

\haiku{Man, zeide zij, ik.}{zal er om huilen dat je}{alleen moet varen}\\

\haiku{De steeg was nauw, hij;}{bukte laag om door de hor}{naar de lucht te zien}\\

\haiku{Hadden ze maar ja.}{gezegd toen ik je bij me}{in huis wou nemen}\\

\haiku{Geld was er genoeg,.}{je was heemraad geworden}{en allang dijkgraaf}\\

\haiku{er zullen er nog,,.}{heel wat verdrinken ik weet}{het zeker zei je}\\

\haiku{neen, dat geld leg ik,.}{op zij dan heeft hij wat meer}{als hij bij me komt}\\

\haiku{hij vroeg Rossaart of.}{zij haar samen wat met de}{post zouden zenden}\\

\haiku{Ik lees nog altijd,,:}{hetzelfde boek zooals je ziet}{en \'e\'en van twee\"en}\\

\haiku{Je wordt stijf in je,,,}{rug zeg je morgen kan je}{toch niet meer varen}\\

\haiku{Hij zat recht op met,.}{haar hand in de zijne maar}{hij kon niet spreken}\\

\haiku{Het water klotste.}{tegen het boord en de schuit}{trok aan de touwen}\\

\haiku{Een man riep telkens:}{wanneer hij den kruiwagen}{grond had uitgestort}\\

\haiku{vroeg hij, het is toch.}{niet de eerste keer dat je}{in Hurwenen komt}\\

\haiku{een mensch haast niets meer,.}{te kosten en zij nam geen}{geld meer van hem aan}\\

\haiku{Doe je plicht, dacht hij,,.}{dan en vraag niet het zal wel}{gegeven worden}\\

\haiku{De hond, die aan zijn,.}{voeten stond schudde zich de}{sneeuw van de haren}\\

\subsection{Uit: De wereld een dansfeest}

\haiku{Het kan vreemd loopen.}{in de wereld en het lot}{heeft rare grillen}\\

\haiku{E\'en herinner ik,,:}{mij in het Spaansch dat zij}{voor mij vertaalde}\\

\haiku{De wereld is een -.}{dansfeest en wie niet danst een}{domoor un tonto}\\

\haiku{Maar we hadden ook.}{al gauw van elkaar ontdekt}{dat we graag dansen}\\

\haiku{Marion krijgt er:}{altijd een hooge kleur van en}{aan het eind zingt ze}\\

\haiku{Het noodlot had mij.}{tusschen dien Deursting rechts en dien}{Jonas links gezet}\\

\haiku{Toen hij hoorde dat,.}{er straks bal zou zijn werd hij}{opeens levendig}\\

\haiku{Toen vroeg hij weer of,.}{ik het menuet kon daar}{hield hij zooveel van}\\

\haiku{Jonas daar alleen,.}{op een stoel zag zitten net}{of hij hoofdpijn had}\\

\haiku{Het gekste is wat die.}{kleine Marion zich in}{het hoofd heeft gehaald}\\

\haiku{Gelukkige tijd,,.}{die kindertijd als men niet}{weet met wien men danst}\\

\haiku{Ja, mijnheer, - op een.}{toon dat de anderen hun}{lachen verbergen}\\

\haiku{Alle stoelen uit.}{de eetkamer werden naar}{het salon gehaald}\\

\haiku{Het waren bijna.}{allemaal meisjes die zich}{aanmeldden als lid}\\

\haiku{En de achternaam.}{moest zooiets geweest zijn als}{sterrewichelaar}\\

\haiku{Hier wandelden wij,.}{ook genietend van het groen}{en de zoele lucht}\\

\haiku{Des te beter, zei,.}{Walewijn dan neem ik die}{twee kamers achter}\\

\haiku{En Petronel is,?}{toch eigenlijk van mij geen}{familie meer wel}\\

\haiku{De lucht was bedekt,.}{maar licht zooals wanneer de maan}{achter wolken schijnt}\\

\haiku{Ik ben haar vader,.}{het spreekt dus vanzelf dat ik}{voor haar blijf zorgen}\\

\haiku{Ik had bedacht dat}{het beter was geen gesprek}{met hem te voeren}\\

\haiku{op mijn kamer en}{het eenvoudigste was hem}{zonder omwegen}\\

\haiku{Het eerste, ik zei,,.}{het al is de dans die hem}{heelemaal vervult}\\

\haiku{Bij het dansen van.}{de bolero schijnt fractuur}{niet zeldzaam te zijn}\\

\haiku{Ziedaar, mijnheer en,.}{mevrouw alles wat ik u}{kan mededeelen}\\

\haiku{Het is lastig, zei,.}{Frans maar een Ringelinck}{geeft nooit den moed op}\\

\haiku{Een ieder zoekt het,,.}{geluk dat spreekt van zelf en}{zooveel mogelijk}\\

\haiku{Dit bevestigde.}{ons vermoeden dat het een}{jonggehuwd paar was}\\

\haiku{En zij keek rond naar,,.}{de bloemen en zij keek hem}{aan zonder een woord}\\

\haiku{Als men de dames.}{kleedt ziet men gauw het verschil}{van de karakters}\\

\haiku{Toen de koorts afnam.}{vertrouwde zij mij veel van}{haar geheimen toe}\\

\haiku{Op een morgen vond,.}{ik haar wakker met een kleur}{en blinkende oogen}\\

\haiku{Samen lukte het.}{ons mademoiselle weer}{naar bed te krijgen}\\

\haiku{En dan tilde hij:}{de slippen van zijn jacquet}{weer hoog op en zei}\\

\haiku{Zooiets te zien is.}{misschien mooier dan het te}{ondervinden}\\

\haiku{Ook deze typen.}{waren niet nieuw voor wie in}{de streek bekend is}\\

\haiku{Spoedig had ik ook.}{reden mijn verblijf hier niet}{langer te rekken}\\

\haiku{De comtesse scheen,.}{hij niet te kennen terwijl}{ik toch beter wist}\\

\haiku{De g\'erant haalde.}{eenige keeren de schouders op}{en keek ge\"ergerd}\\

\haiku{Toen ging zij terug,.}{met een gezicht of zij dien}{vent verafschuwde}\\

\haiku{Stil, fluisterde hij.}{en hij voerde mij bij den}{arm naar het venster}\\

\haiku{Uw partner zal er.}{ongetwijfeld waardig en}{sierlijk mee dansen}\\

\haiku{omdat ik ze zelf.}{voor een jaar of drie aan haar}{vader had verkocht}\\

\haiku{Nathalie kwam ook,:}{nog binnen en ook mijn vrouw}{die mij uitlachte}\\

\haiku{De laatstgenoemde,}{had hem dikwijls ontmoet maar}{haar geheugen was}\\

\haiku{Marion, zei ik,}{ik heb je verzocht vanavond}{bij mij te komen}\\

\haiku{Ja, ik weet dat u}{daar ook van overtuigd bent en}{dat verwondert me.}\\

\haiku{Daarna deden ze.}{weer een mode-dans en}{nog een anderen}\\

\haiku{Het was gedaan voor.}{ik het wist en ik had er}{een gloeiend hoofd van}\\

\haiku{En Mr Sedge dacht.}{ook dat het pijn doet als het}{rhythme zoo diep zit}\\

\haiku{Toen die kwam zei hij.}{alleen dat mevrouw het heel}{rustig moest hebben}\\

\haiku{Bij de eerste maat.}{al was het of ik van mijn}{stoel moest opspringen}\\

\haiku{Nu, die heer kwam dan.}{vertellen dat ze niets te}{verwachten hadden}\\

\haiku{Zij haalde er de:}{schouders over op en tegen}{mij zei ze later}\\

\haiku{Ik kreeg er drie pond.}{op en daar konden ze een}{paar weken mee door}\\

\haiku{Zij duwde mij van.}{zich af omdat zij dacht dat}{ik uit de maat was}\\

\haiku{Had ik haar toen de?}{juiste maat moeten leeren en}{niet weg laten gaan}\\

\haiku{Wij hebben elkaar,.}{meer gezien voor het eerst al}{jaren geleden}\\

\haiku{ik langer gekend,.}{had dan iemand anders ook}{eenige portretten}\\

\haiku{Ik zag dat hij het.}{gezicht gewend hield naar het}{hek en nog omkeek}\\

\haiku{omdat wij meenden.}{een ander te hooren die}{in de verte floot}\\

\subsection{Uit: Een zwerver verdwaald}

\haiku{weer sloeg de eene, de,.}{zware klok de kleinere}{herhaalde den galm}\\

\haiku{De stad was ontwaakt,,.}{in water en dageraad}{zonder geraas}\\

\haiku{Toen haar vrees verzwond;}{zuchtte zij en bemerkte}{hoe hij haar aanzag}\\

\haiku{Hij zat verwonderd,.}{bij  het vuur bedrukt door}{het onheil in huis}\\

\haiku{Het water spoelde,.}{tegen het roer het ijzer}{piepte geregeld}\\

\haiku{Omtrent den middag,.}{trad Seffe binnen de schelm}{was ietwat dronken}\\

\haiku{Er was iets dat hem,.}{verlicht deed ademen hij had}{met de stad gedaan}\\

\haiku{Daar zat Maluse,.}{in het licht van het venster}{de hond lag er ook}\\

\haiku{Hij keerde zich om.}{uit het licht en liep weer door}{de gang naar de straat}\\

\haiku{Toen ook Simon aan,}{boord was riep Meron Joseph}{zijn bevelen uit}\\

\section{W.F. Scheepsma}

\subsection{Uit: De Limburgse sermoenen (ca. 1300). De oudste preken in het Nederlands}

\haiku{We duiden het in (}{het vervolg aan met het siglum}{hbijlage i}\\

\haiku{Het staat wel vast dat;}{Bernard van Clairvaux in de}{volkstaal heeft gepreekt}\\

\haiku{56, een lang traktaat.}{waarin wordt gespeculeerd}{over de Triniteit}\\

\haiku{De zogenoemde {\textquoteleft}{\textquoteright},.}{Vorderspiegel met een deel}{van de tekst van Rd}\\

\haiku{Veel meer persoonlijks,}{komen we over hem niet te}{weten al laat hij}\\

\haiku{Dat zijn aantallen.}{waarbij alleen Keulen in}{de buurt kan komen}\\

\haiku{Het grootste deel van.}{het register is door de}{teksthand geschreven}\\

\haiku{(h, opschrift)  Dits.}{hoe wi in gode bliven}{ende god in ons}\\

\haiku{Het is de moeite}{waard om te onderzoeken}{op welke manier}\\

\haiku{19, 26), eerst in het:}{Middelhoogduits en dan pas}{in het Latijn}\\

\haiku{Ez  sint och niht.}{die hindir rede spulgint}{unde virkerer}\\

\haiku{de stroom, de beide,.}{oevers de boom des levens}{en de twaalf vruchten}\\

\haiku{Daarna volgen de, {\textquoteleft}{\textquoteright}:}{conventsbroeders met al even}{fraaietelling names}\\

\haiku{de vriendin staat voor.}{de heilige ziel en de}{vriend voor onze Heer}\\

\haiku{[...]  Hi ginc Jhesus.}{in den tempel under die}{Juden ende sprach}\\

\haiku{de eerste maakt de,,.}{drank de tweede tapt hem en}{de derde schenkt hem}\\

\haiku{{\textquoteleft}Zalig degene,{\textquoteright}.}{die altijd vreest te vallen}{want hij staat vast}\\

\haiku{Een dergelijke.}{bescheidenheidsverklaring}{is allerminst uniek}\\

\haiku{43 (waartoe dus ook).}{de broeders uit handschrift h}{kunnen behoren}\\

\haiku{45 onderscheidt drie.}{manieren waarop Christus}{tot de mensen spreekt}\\

\haiku{Derste es dasse,.}{hoge ligt ende es oec}{dar ombe seker}\\

\haiku{de bruidegom staat.}{voor Christus en de bruid voor}{de heilige ziel}\\

\haiku{45 kunnen immers ( {\textsection}).}{nauwelijks anders worden}{opgevatzie 2.11}\\

\haiku{Dat is aanzienlijk,.}{later dan in het Frans het}{Duits of het Engels}\\

\haiku{De meningen over.}{de toepassing van deze}{fraaie tekst verschillen}\\

\haiku{Die lugene hat,.}{ein unseliche dochter}{die heizet smeichen}\\

\haiku{in de dertiende.}{eeuw zo  geliefd dat ze}{overal verschijnen}\\

\haiku{Deze passage;}{werd door een middeleeuwse}{hand gecorrigeerd}\\

\haiku{komt vermoedelijk:}{voor op de boekenlijst uit}{Rooklooster.916  b}\\

\haiku{komt vermoedelijk:}{voor op de boekenlijst uit}{Rooklooster.917  c}\\

\haiku{39 en Brief 1, die.}{overigens slechts drie losse}{zinsneden omvat}\\

\haiku{Het zou naderhand}{min of meer integraal in}{omloop zijn gebracht.951}\\

\haiku{We spreken nu in {\textquoteleft}{\textquoteright}.}{navolging van Oliver van}{Luikse psalters}\\

\haiku{het coumnandement;}{op basis waarvan li fin}{amant moet handelen}\\

\haiku{zie verder 316, 11,,,,,,,).}{en 13 321 9 16 20 24}{en 27 en 322 7}\\

\haiku{Untersuchungen zur.}{Geschichte der Metapher vom}{Herzen als Kloster}\\

\haiku{Publications.}{de l'Institut d'\'Etudes}{M\'edi\'evales}\\

\haiku{H\"aring, N. (ed.), {\textquoteleft}Der,:}{Literaturkatalog}{von Affligem in}\\

\haiku{Hollander, August, \&, {\textquoteleft}{\textquoteright}.}{den Ulrich Schmid Het Luikse}{Leven van Jezus}\\

\haiku{Publications.}{de l'Institut d'\'Etudes}{M\'edi\'evales}\\

\haiku{Publications.}{de l'Institut d'\'Etudes}{M\'edi\'evales}\\

\haiku{Te verschijnen in (),.}{P. Broomans e.a.red. 1}{have heard about you}\\

\haiku{Sinclair, Keith Val,.}{French devotional texts}{of the Middle Ages}\\

\haiku{Publications.}{de l'Institut d'\'Etudes}{M\'edi\'evales}\\

\haiku{Vankenne, A., (vert.),.}{Vie de Marie d'Oignies par}{Jacques de Vitry}\\

\haiku{Pierre Roch\'e \& Guy (),.}{Lubrichonred. Le Moyen}{Age et la Bible}\\

\haiku{28, Dit sprict van xii,,,.}{dogeden die ane Gode sin}{93 210 222 269 Ls}\\

\haiku{4 (vgl. Kern 1895, 215,-),.}{1116 en in diens Duitse}{tegenhanger Rd}\\

\haiku{Maagdendries kan ook.}{niet de eerste bezitter}{van h zijn geweest}\\

\haiku{vgl. de r. 10-12,,.}{waar a wordt weergegeven}{dat van br\r{u}der spreekt}\\

\haiku{230Over de datering.}{van het schrift van het eerste}{gedeelte van hs}\\

\haiku{Volker Honemann ().}{M\"unster bereidt een editie}{van deze tekst voor}\\

\haiku{oudere versies,.}{van de Vita Lutgardi}{waaronder een Ofr}\\

\haiku{372Verwijzingen {\textsection}.}{naar beeldmateriaal van}{h in 1.1 n. 67}\\

\haiku{60 hadden we het.}{palmboomtraktaat verwacht op}{de plaats tussen Ls}\\

\haiku{425Een afbeelding van,.}{de opening f. 4v-5r in}{Hamburger 1990 afb}\\

\haiku{437De gehele,,-,;}{interpolatie in Kern}{1895 464 20466 14}\\

\haiku{36 (Kern 1895, 510, 20-) (,-,,-,,-).}{25 en 38528 2226 531}{1524 531 2729}\\

\haiku{De Bruin 1970a, 268, 15--,,-.}{18 en Corpus Gysseling}{113 633 1519}\\

\haiku{ook Reynaert 1975, 239-.}{246 beschouwt de Parijse}{tekst als zelfstandig}\\

\haiku{op de p. 26-32.}{worden e en f (= het}{laatste stuk van Ls}\\

\haiku{Reypens 1964, vooral ( {\textsection}),,.}{cap. 4deels aangehaald in}{1.5 269 275 en 276}\\

\haiku{Moltzer 1875, 536-537 (,-), ().}{vgl. Zacher 1842a 347348 en}{Schneider 1987b 6053v}\\

\haiku{785Schmidtke 1982, (-).}{nr. 9p. 3435 is het}{tweede deel van Rd}\\

\haiku{935Zie Van Mierlo,-.}{1929 en ook Van der Zeyde}{1934 128129 en 160}\\

\haiku{1017h leest hier vol. Kern,,;}{1895 465 11 herkent hier geen}{kopiistenfout}\\

\haiku{1025Schweitzer 1997, 192-193 (),- ()- ().}{n.a.v. vraag 25 208209vraag 73}{en 212213vraag 102}\\

\section{Bert Schierbeek}

\subsection{Uit: De andere namen}

\haiku{ik ga even aan de}{berm van de weg liggen want}{je kunt nooit weten}\\

\haiku{was noords Loyola}{maar de smart drong ravijren}{diep in mijn lichaam}\\

\haiku{heet hoofd om de muur,}{van het weer soms loopt de film}{mis zei het meisje}\\

\haiku{en ik dacht moest ik,}{haar voorbeeld zijn en ik zei}{het haar maar zij zei}\\

\haiku{zij lagen bloot op}{de tafel en knikten de}{aanwezigen toe}\\

\haiku{dat is de zucht van}{de gestorven sultanes}{in het Alcazar}\\

\haiku{maar ik vluchtte weg}{van haar want haar lach las in}{mij vele ziekten}\\

\haiku{het mooie dier tekent}{filmische waanbeelden op}{de huid van het kind}\\

\haiku{het astronomisch}{zicht zou verloren zijn maar}{men weet veel tudo}\\

\haiku{wij hebben nog zo}{weinig gemaakt op aarde}{dat stof werd en beeld}\\

\haiku{het samengaan der}{tegenstellingen lag niet}{in de doctrines}\\

\haiku{een man uit de stad}{leeft vegetatief van het}{land en zijn buren}\\

\haiku{dan zie ik de trams}{vol bedelaars langskomen}{binnen een geel licht}\\

\subsection{Uit: Verzameld werk. Deel 2. Het boek ik. De andere namen. De derde persoon}

\haiku{Wij gaan daar dieper;}{op in bij Bert Schierbeek en}{het onbegrensde}\\

\haiku{Schierbeek's werk sluit aan;}{bij het Dada{\"\i}sme en}{Surrealisme}\\

\haiku{o droesem... want IK,...}{GOD die recht sta in mijn ziel}{en het verdriet ken}\\

\haiku{je bent het bord, het...}{eten en het borduursel en}{alles doe je zelf}\\

\haiku{alle kalkputten:}{heeft hij gezien en zijn hoofd}{geschud en gezegd}\\

\haiku{en het vlees dat hard...}{is en helemaal niet zwak}{en niet wachten kan}\\

\haiku{In mij leeft het volk...}{het onmondige en het}{kent geen genade}\\

\haiku{een wit boetekleed,...}{past u en mij hoog aan de}{hals dicht gebonden}\\

\haiku{Hij leefde hierdoor.}{vijf minuten langer dan}{de bedoeling was}\\

\haiku{- Je zoekt het te ver,.}{lieve Lilith dat was van}{voor het paradijs}\\

\haiku{De Grote Dag is!!}{gekomen O Lawd have}{thy mercy upon us}\\

\haiku{God gaat terug naar,?}{zijn hemel maar mijn hemel}{mijn Ik mij waarheen}\\

\haiku{het hart was hard het}{godvergeten fatsoen o}{ons fatsoen o god}\\

\haiku{Ik denk wel dat ik.}{eens de Vestaalse maagd zal}{worden die ik lees}\\

\haiku{ik l\'a\'at u allen,}{ingaan ik wil u allen}{vrouw zijn en mij zelf}\\

\haiku{ik weet dat ik de}{oeverdieren zal zijn die}{hun kroost vermalen}\\

\haiku{want het goed is van...}{het kwaad losgeslagen en}{zij bestaan niet meer}\\

\haiku{We liggen tussen.}{de bene het uur nu een}{eeuw te verzweten}\\

\haiku{omdat hij dood zou '...}{zijn van het vaderschap en}{t gebeur der ziel}\\

\haiku{Zullen mijn vingers?}{mijn hand de papavers zijn}{en rood van het zon}\\

\haiku{o, liefste de grof,}{van het graf van mijn hand in}{de mond o liefste}\\

\haiku{ik heb ze op hun,...}{sodemieter gegeven}{allemaal schrijft hij}\\

\haiku{O, ik Dolle Schuld...}{zal de anjer het lied van}{dit leven zingen}\\

\haiku{de voorgang staat open...}{en de deur ook nou is het}{niet zo benauwd meer}\\

\haiku{zij verscheen hem 't}{gezicht over de lakens van}{help ons de weiden}\\

\haiku{ik heb mijn hele}{leven veel gereisd en al}{is het ook oorlog}\\

\haiku{de mensen lopen}{geheel langs je zij en gaan}{heen om te treden}\\

\haiku{en ik alleen in......}{dit huis hij heeft nooit van een}{ander gehouden}\\

\haiku{goolgraag was geen slecht......}{mens de messen zo stomp en}{bang voor een dooie vis}\\

\haiku{want allen was mij...}{toen veel in de vege me}{monis van mijn lijf}\\

\haiku{je stuurt urine naar,...}{de kikkers ze drinken het}{op en worden geel}\\

\haiku{ik drink het uit jou...}{uit alles wat je van jou}{hebt en fijn de pijn}\\

\haiku{Laat je man Van Dijk.}{de liefde verzekeren}{die jij voor hem voelt}\\

\haiku{want 't was nog geen}{oorlog zeiden ze want ze}{zouden zeiden ze}\\

\haiku{de verschrikking zie...}{je en wat er aan messen}{uit de stenen steekt}\\

\haiku{zu h\"angt in allen...}{w\"andern des jugendlebens}{zum traufiel trocken}\\

\haiku{naar buiten want de}{sleutels passen niet van het}{geluk dat open wil}\\

\haiku{wie liefheeft zal de}{teller en noemer van de}{namen van liefde}\\

\haiku{haben wir gar nicht,,,,...}{gewollt wir wollten du sollst}{du kannst du musst}\\

\haiku{wir k\"onnten ja......}{gar nichts gewusst haben da}{wir nicht wollten wir}\\

\haiku{zijn hoogmis in de}{urnen ons zelf tot de uren}{vertekel gegaan}\\

\haiku{mij is de bokaal}{van de balgen der liefde}{de ofri gaan doen}\\

\haiku{de dood legt lange}{voeten in mijn woord nu het}{heeft zich uitgekleed}\\

\haiku{wat de distel in}{de rode bloem van het bloed}{schrijft de verschrikking}\\

\haiku{- De 6e druk is een.}{fotografische herdruk}{van de 1e druk}\\

\haiku{ik ga even aan de}{berm van de weg liggen want}{je kunt nooit weten}\\

\haiku{dat in mij staat een}{afschuwelijk en heerlijk}{beeld een wereldbeeld}\\

\haiku{en ik dacht moest ik,}{haar voorbeeld zijn en ik zei}{het haar maar zij zei}\\

\haiku{zij lagen bloot op}{de tafel en knikten de}{aanwezigen toe}\\

\haiku{dat is de zucht van}{de gestorven sultanes}{in het Alcazar}\\

\haiku{maar ik vluchtte weg}{van haar want haar lach las in}{mij vele ziekten}\\

\haiku{dat is long zei zij}{zacht na drie weken vond men}{het lijk van de zoon}\\

\haiku{het mooie dier tekent}{filmische waanbeelden op}{de huid van het kind}\\

\haiku{het astronomisch}{zicht zou verloren zijn maar}{men weet veel tudo}\\

\haiku{wij hebben nog zo}{weinig gemaakt op aarde}{dat stof werd en beeld}\\

\haiku{het samengaan der}{tegenstellingen lag niet}{in de doctrines}\\

\haiku{een man uit de stad}{leeft vegetatief van het}{land en zijn buren}\\

\haiku{dan zie ik de trams}{vol bedelaars langskomen}{binnen een geel licht}\\

\haiku{- De 5e druk is een.}{fotografische herdruk}{van de 1e druk}\\

\haiku{ik ben koning en}{dienaar en het offerdier}{nadert in die tijd}\\

\haiku{zij ziet de vrouwen}{die wakker liggen in het}{dorp van hun leegte}\\

\haiku{en in alles was}{verweg zo de gedachte}{wel naar binnen niet}\\

\haiku{wat is water, zegt}{iemand het schip drijft er op}{het vaart Hendrik gooit}\\

\haiku{het spelen en niet}{weten de vader leeft tot}{wat de moeder was}\\

\haiku{dit is analogie}{en verbondenheid met die}{leven en sterven}\\

\haiku{de moeder houdt het}{mooiste kind van de wereld}{tegen de ramen}\\

\haiku{leert de juffrouw de}{meisjes de kruissteek in de}{hoofden der jongens}\\

\haiku{ik ben een kind in}{uw omkleding sprak hij door}{de huid van dit huis}\\

\haiku{hij vormt zijn mond tot:}{sacrale staat en leest met}{magie in zijn stem}\\

\haiku{omdat we 't goed}{menen eet je wortels want}{wortels zijn gezond}\\

\haiku{als je de slaaf van,}{de wereld en zijn buurman}{wil zijn schreeuwt het kind}\\

\haiku{die honger hebben}{we allemaal maar tussen}{de honger en mij}\\

\haiku{de man valt neer met}{een vloek in het gebed van}{zijn bloed het leven}\\

\haiku{een klooster binnen}{het klooster halen zij de}{beelden en zetten}\\

\haiku{hij koopt die jas en,}{hij groeit er nooit meer uit met}{die jas zei een vriend}\\

\haiku{dit land is nat soms}{als de missouri maar de}{dijken zijn beter}\\

\haiku{en zo stond het te}{vroeg stil dan kan men het niet}{meer op gang brengen}\\

\haiku{loopt loeiend het land}{in en waarschuwt de mensen}{dat het zal komen}\\

\haiku{hier zijn wij zoals}{wij gaan over u ~ in de}{ontferming van wat}\\

\haiku{De 3e druk is een.}{fotografische herdruk}{van de 1e druk}\\

\section{Anda Schippers}

\subsection{Uit: De kikker die zichzelf opblies en andere Middeleeuwse fabels}

\haiku{De aap zag hoe de.}{vos Reinaart gezegend was}{met een lange staart}\\

\haiku{Beschouw je jezelf,.}{als aanzienlijk dan meet en}{oordeel je jezelf}\\

\haiku{{\textquoteleft}O zuster, ik raad, - -!}{je aan kom snel gezond en}{wel van die troon af}\\

\haiku{Bovendien moet je.}{alle kosten betalen}{die hier zijn gemaakt}\\

\haiku{Ik voel me erg slecht{\textquoteright},.}{dan is er geen hoop dat hij}{zijn ziekte overleeft}\\

\haiku{Dit hoorde de vos,.}{die de werking van het kruid}{verbena kende}\\

\haiku{Dankzij deze drie.}{zaken heb ik zo lang en}{zo gezond geleefd}\\

\haiku{Daarom is het goed.}{om een slechte gewoonte}{direct te weerstaan}\\

\haiku{Waarover Aesopus de,.}{volgende fabel vertelt}{over twee ratten}\\

\haiku{De zeug en de wolf.}{De vierde fabel gaat over}{de zeug en de wolf}\\

\haiku{Want de boer maakte.}{de ossen los van de ploeg}{en dreef ze naar huis}\\

\haiku{en degene die,.}{het had zwoer evenzo dat hij}{het niet had gepakt}\\

\haiku{Toevallig is het:}{personage Aesopus daar}{een goed voorbeeld van}\\

\haiku{Wolff en A. Deken.}{Historie van mejuffrouw}{Cornelia Wildschut}\\

\section{Frans Schleiden}

\subsection{Uit: Hazegerf}

\haiku{Kaplaon komt mit ':}{t Fes9 De vrouw Peltjes komt}{i geloope en reep}\\

\haiku{Der Pitter zaat n\"uks,:}{e bezoog der Joep ins es}{wente zage wool}\\

\haiku{{\textquoteleft}E kamp os verdaat,.}{v\"or e kiekt graat of wente}{kaffie sjmoekelet}\\

\haiku{e kratset twei drei.}{maol i gen \`e\"ed en sjprong}{op der ozze aa}\\

\haiku{'t Loewet ovvesklok.}{en der Erwinus zonk i}{ge veld oppen knee}\\

\haiku{Diej moe\"ete de.}{munneke gidder ovvend mit}{n\`eme n\`o gen kirk}\\

\haiku{Der broor Andreas.}{mit der Johannes w\`ore}{va ter Munnekef d\`o}\\

\haiku{{\textquoteright} 'ne Mond dern\`o w\`or '.}{der nonk Macheel i ge veld}{ant mie\"ene}\\

\haiku{{\textquoteright} Der Mertens w\`or al:}{pruttelent\`ere76 n\`o gen}{wei i gegange}\\

\haiku{En wurkelich, de.}{dreide naat w\`or de pieng nit}{mie\"e oet te houwe}\\

\haiku{e zow gevelles,.}{koame der vadder hei pieng v\"or}{gek te w\`e\"ede}\\

\haiku{De hoezer zunt dan.}{frisj wiej geweisje en}{ze blinke wiej nuj}\\

\haiku{E paar jong jonge.}{kompte mit br\`e\"er96 vol}{vlaam oet ge bakkes97}\\

\haiku{{\textquoteright} Der lie\"erer heel ze '}{i gen heng en leet zem}{zie\"e en sjroevvet ze}\\

\haiku{en dat geet nie mie\"e -.}{ich bi vunf en achsig oet}{gene braokmond139}\\

\haiku{t Is es went et.}{kie\"em va wied uvver ge veld}{en lants gen hage}\\

\haiku{{\textquoteleft}Komt sjtraks nao de,{\textquoteright}.}{hoe\"emis bei mich op de}{pasterei alle veer}\\

\haiku{Doew sjleep e ruhig ',,:}{i. Ent murges wie der}{dokter komt zaat d\`e}\\

\haiku{Noe z\`enete zich:}{in alle hoezer de luuj}{en b\`enete ze}\\

\haiku{{\textquoteleft}D\`e,{\textquoteright} zaat-e ze, {\textquoteleft}dat,.}{is der Kl\"os d\`e waor op}{w\`e\`eg nao gen kirk}\\

\haiku{Doew sjtonge ze.}{in der sjtal en dao waor}{ie\"elend en ermood}\\

\haiku{Der wink sjn\`e\"et 'm.}{kaod lants gen oe\"ere}{en uvver zie gezich}\\

\haiku{{\textquoteleft}Simeon,{\textquoteright} zaat ich, {\textquoteleft},..... '....}{jong dink ins nao watste deest}{Alderhilligste}\\

\subsection{Uit: De hillige vaggen durp}

\haiku{E minnig medje '-.}{sjtongt murges 67}{maol v\"or des sjpeegel}\\

\haiku{Ze zitte iggen.}{hoes bei der \`oavend21 mit de}{kap oppene kop}\\

\haiku{es wiej oet der tied,.}{va Don Quichotte d\`e gong}{och mit ezu get oet}\\

\haiku{De vrouw Groonesjild komp '.}{aaterm. Ze sjtonge}{en wronge de heng}\\

\haiku{'t Is sjtil in.}{der operatioenszaal en}{ginge zet e w\`oad}\\

\haiku{hedelfienger en.}{loene en kling welsjkere64}{mit de haffele65}\\

\haiku{ich zie\"en 't durp, ',...}{en der t\`oan van de kirkt}{loewt de mis is oet}\\

\haiku{E sjtong al ins,,,.}{op en gong op krukke twei}{drei meter nit mie\"e}\\

\haiku{Wat is 't sjtil, '.}{iggen durp nog sjtiller}{wiej int sjpitaal}\\

\haiku{Ezu gonge de daag.}{verbei en der Macheel blef}{zitte in d\`e sjtool}\\

\haiku{Der Guillijom mus.}{karoe\"ete hakke en}{hui g\`oa kie\"ere}\\

\haiku{De kleier zunt nat}{bis oppen hoet en ze ruuke}{wiej der sjtank va}\\

\haiku{Ozze Herregod '.}{hat ging vruid ant kapotsjl\`oa}{va velder en weie}\\

\haiku{all\`o 't is nog ', '... '.}{nit watt zie\"e mot m\`et}{geet v\"orne sjoester}\\

\haiku{Oppen hoof waor ' '.}{t klaor wie uvver daag en}{roe\"ed wiejn hil98}\\

\haiku{Dat hof mer inge, '.}{te zie\"e da wete zet}{sjnak allenui}\\

\haiku{E beft en razelt, ' ' '...}{en d\`o kumptn wil gloed en}{n vlam inm op}\\

\haiku{Went d\`er Goddes Zoon,,.}{zut dan zat dat dis sjting}{broe\"ed w\`e\"ede}\\

\haiku{Der wink sjpelt'n um ',.}{der kop en int hemp dat}{los hingt op de bros}\\

\haiku{De jonge sjtunt 'n '!}{owweblik en kiekke zich}{aat Is ged\`oa}\\

\haiku{E let ze oppen.}{linker hand en sjmie\"et118}{ze dik mit botter}\\

\haiku{M\`e van noe aa wos}{der Macheel datte-n-'t}{Thriske zel\`eve nit}\\

\haiku{- Went 't get wuur uvver ', '.}{t veld uvvert k\`oon en der}{terf en de havver}\\

\haiku{Dat is 'n wet, diej.}{ezu aod is wiej de welt en}{ummer blieve zal}\\

\haiku{De madam en 't '. '.}{Truudje sjtunt int d\"a\"oresjpan}{t Heisjt opgepaast}\\

\haiku{De Hakkeret is '.}{mit vant sjunste erf van de}{ganse sjtrie\"ek}\\

\haiku{Der Macheel zoot dao:}{mit ene glans iggen owwe}{en der Groonesjild zaat}\\

\haiku{Doe veele i sjlaop i ' '.}{zie geluk en int good}{vant mondelit}\\

\haiku{Der sjlimste vaggen.}{durp AGGENE beuisj wont der}{Mertens mit zieng vrouw}\\

\haiku{Pastoer hat mit der.}{Mertens nog mie\"e leed wiej mit}{gans Lutterendal}\\

\haiku{{\textquoteright} Noe zoot der Mertens '.}{iggene kaffie\"e beit}{Macheelkes Truudje}\\

\haiku{E minnigmaol}{is dat w\`e\`er oet gen dil}{van onder op lants}\\

\haiku{Aate de weie, woe 't,.}{veld aavingt huu\"ete get wat klaagt}{en kriesjt wiej e kink}\\

\haiku{{\textquoteright} Dao zitteter aate,.}{iggen kirk nog mie\"e die nit}{kommeneseere}\\

\haiku{Dao kleft blood an 't?}{sjabeleer en an de hand.}{Och dat nit woe\"er}\\

\haiku{Doe dees dieng hand op.}{en vervuls alle l\`eve}{mit dienge z\`ege}\\

\haiku{Noe dunt ze zich de '.}{ring aa en dan ist good}{en zunt ze contint}\\

\haiku{Noe wiej 't ged\`oa,,.}{is mingt der kuster e wuur}{nit mie\"e iggen kirk}\\

\haiku{Op 'n oethaot,.}{kun d\`er wir k\`oame went der}{sjnie\"e gesjmolten is}\\

\haiku{{\textquoteleft}mer landsem, v\`er hent,{\textquoteright}.}{noe tied doe kries huuj havver}{ezu v\"a\"ol wieste wils}\\

\haiku{De luuj gleuven nog ',.}{nit datt oethaat is bis}{dat de liester zingt}\\

\haiku{{\textquoteright} Bis noe waor 't.}{es went ze te hoop mer ing}{zie\"el heie gehad}\\

\haiku{Alles wat groe\"et,...!...}{en hoe\"eg is is mit angs}{verbonge ene berg}\\

\haiku{In de wei sjtunt ze.}{te kiekke mit groe\"ete}{glazere owwe}\\

\haiku{E wilt ene boer van ',....}{m make der litste en}{der grutste Groonesjild}\\

\haiku{Da zalle zage,:}{wente v\`e\"edig is mit}{diej va Ie\"epe}\\

\haiku{Herregod iggen,{\textquoteright}.}{himmel laot mich sjtil}{en geduldig zie\"e}\\

\haiku{Pastoe\"er ka gee...}{wo\`ad mie\"e oetbringe en dao}{in ins ene sjlaag}\\

\haiku{Dao is 'n welt, diej '.}{de sjtad nit kint en dat is}{diej vant jonk vie\"e}\\

\haiku{E paar minute,.}{dan is och de krao mit ene}{sjrei e-weg}\\

\haiku{E wit och dat dat, ' '.}{e bewies is dattn}{interessie\"et}\\

\haiku{{\textquoteright} Der Mertens hei de.}{vreier ging twientig jaor}{laote ko\`ame}\\

\haiku{Noe geet 't op en '.}{aaf uvvern heen en probeere}{ze get te versjto\`a}\\

\haiku{De ouwere hent.}{de beuk al onder gen erm}{en zunt e-weg}\\

\haiku{V\"a\"or mees te lane}{hoft me ging sjl\`e\`eg toe te}{do\`a en da hat me}\\

\haiku{Diej zonge \`eve hel ' '.}{naot oksaal truk en ezu}{gongt hin en weer}\\

\haiku{In der aavank ersjaffet.}{God himmel en \`e\`ed en e}{maket alles good}\\

\haiku{zate... en e sjprook:}{uvver tied en ivvigheed en}{uvver de zung en zaat}\\

\haiku{Zoste dao kunne,,?}{pr\`edige Barthelomee in}{de Sint Servaos}\\

\haiku{{\textquoteleft}al kost 't os twei,{\textquoteright}.}{doezend gulde v\`er zulle}{dat offer bringe}\\

\haiku{{\textquoteright} H\`e\"er Bussjep,, '!}{went d\`er ins wust wat inn}{zie\"el kan umgo\`a}\\

\haiku{Noe is 't 'n koo,.}{diej miskoft dan e p\`e\"ed}{dat vervangen is}\\

\haiku{Ich zal 't uch wal,....}{laote weete wienie\"e}{dat d\`er kaome mot}\\

\haiku{Bei 't Mieke komt....}{ene r\"a\"o\"ek de deur oet ene}{ouwe-luujsr\"a\"o\"ek}\\

\haiku{En dan {\textquoteleft}voes{\textquoteright}, {\textquoteleft}broeng{\textquoteright} en...,.}{dan der Mertens ene duuvel}{andesj geet it nit}\\

\haiku{Namens de ganse,...}{pfaar danke v\`er uch bezondesj}{v\"a\"or die twei woe\"e zunt}\\

\section{Annie M.G. Schmidt}

\subsection{Uit: In Holland staat mijn huis}

\haiku{er zijn er bij die.}{glooien en er zijn er bij}{die zelfs rechtop staan}\\

\haiku{Maar wanneer men in,.}{Nederland wil blijven wordt}{de keus beperkter}\\

\haiku{Goeie help, dacht ik, het?}{zal toch niet een gothische}{cathedraal worden}\\

\haiku{Toen de architect,,:}{mij vroeg hoe ik de keuken}{wou hebben zei ik}\\

\haiku{Pakje wasmiddel,!!}{in kastje dat er uit schiet}{en weer terugschiet}\\

\haiku{Zo'n oprechte spijt,.}{dat we het maar niet liever}{zelf hebben gedaan}\\

\haiku{Maar dat is altijd, '.}{zo zeiden de mensen die}{t weten kunnen}\\

\haiku{Men is oud voor het,,,,.}{klaar is oud grijs afgeleefd}{moe en afgeknapt}\\

\haiku{Ik betrapte me,:}{er op dat ik hele uren}{achtereen uitriep}\\

\haiku{Ik boog me uit het.}{raam en daar in de diepte}{stond een kruidenier}\\

\haiku{Het waren Linda.}{en Klaas en natuurlijk moesten}{ze eerst alles zien}\\

\haiku{Welnee, ik zal er.}{nog wat kranten opleggen}{en nog wat houtjes}\\

\haiku{Ik weet niet of u ',...}{t weet maar er zit iets aan}{u vast van achter}\\

\haiku{En of wij ons niet:}{veel beter een boek hadden}{kunnen aanschaffen}\\

\haiku{Het spijt me, meneer,.}{ik heb me bedacht en ik}{doe het toch maar niet}\\

\haiku{Hij kronkelt zich nu.}{op onze piano en}{ik laat het maar zo}\\

\haiku{Het is nutteloos.}{om zich tegen het noodlot}{te verzetten}\\

\haiku{Het is zelfs zo, dat;}{het begrip huis verweven}{is met haar erotiek}\\

\haiku{Maar gek, juist voor die.}{vrouw is er in het hele}{huis geen eigen plaats}\\

\haiku{En toch zit ik hier,.}{nu al van half tien tot half}{een met m'n bril op}\\

\haiku{Hij maakte een stuk.}{lood vast aan een touw en liet}{het langs de deur neer}\\

\haiku{Zie de maan schijnt door,....}{de bomen makkes met u}{wild gestaak zingt hij}\\

\haiku{Misschien, en daar komt,.}{het moederhart weer misschien}{wordt hij iets heel hoogs}\\

\haiku{Precies zoals laatst,:}{toen hij naast me dribbelde}{en ik aldoor riep}\\

\haiku{Hij ligt zo vreemd te.}{hijgen en verder is hij}{zo stil en zo slap}\\

\haiku{Maar ik zie al dat,.}{dit onzin is want hij is}{niet in het minst ziek}\\

\haiku{Mensenmeisjes zijn,!}{tegenwoordig veel vlotter}{als je dat maar weet}\\

\haiku{Natuurlijk wil ik....}{in dat comit\'e zitten}{zei ik opgewekt}\\

\haiku{We kwamen  in.}{de tuin en zagen de kat}{met iets geels spelen}\\

\haiku{Toen zag ik dat een.}{van de koeien een stuk touw}{om zijn horens had}\\

\haiku{Eentje met een snor.}{en een dikke buik met een}{rode band er over}\\

\haiku{Ze keek ons kwijnend.}{aan en trok wellustig haar}{nagels in en uit}\\

\haiku{Je doet maar, zegt haar.}{man en gaat een beetje over}{de schutting lopen}\\

\haiku{Maar hij voelde dat,.}{hij iets verkeerds gezegd had}{iets dat men niet zegt}\\

\haiku{De zuilen waren,.}{geschilderd op een doek maar}{ze leken heel echt}\\

\haiku{Nou, wat zei ik, o,.}{ja dat hele gezin is}{een beetje overstuur}\\

\haiku{Pas op, maar ze heeft.}{zo'n dorst en slaat het achter}{elkaar naar binnen}\\

\haiku{en toen stond ik te,...}{strijken aan zo'n strijkplank maar}{het was geen strijkplank}\\

\haiku{Een stofzuiger, een,,:}{hele fraaie die alles kan}{behalve kraaien}\\

\haiku{Zou je de trap en,.}{het portaal willen doen vroeg}{ik onderdanig}\\

\haiku{Zou ik misschien   '?}{zelf een beetje achterlijk}{aant worden zijn}\\

\haiku{Maar ze is nooit meer.}{terug gekomen om het}{me te vertellen}\\

\haiku{Denk er aan, mevrouw,.}{ik heb niet graag dat u mijn}{kleren stiekem draagt}\\

\haiku{Hij zegt nou is het,,.}{uit zegt ie en hij gaat me}{vermoorden zegt ie}\\

\section{Jacques Schreurs}

\subsection{Uit: Het godsbewijs van dokter Chantrain}

\haiku{Een gril, beweren;}{de nuchtere geesten met}{tante Simone}\\

\haiku{Er is niets zo erg.}{als wanneer die aangetast}{en bedorven zijn}\\

\haiku{Niemand ter wereld,;}{is dan ook zo hol zo leeg}{als de leugenaar}\\

\haiku{Soms schijnt het huis even,.}{te wankelen slagzij te}{maken als een schip}\\

\haiku{dat er een grote,;}{zeer grote verscheidenheid}{was in de eenheid}\\

\haiku{Dat er dan toch nog,.}{iemand was die haar doorschouwd}{had had haar ontsteld}\\

\haiku{zich in gezelschap:}{van  heren gedragen}{dan onder elkaar}\\

\haiku{maar haar noodlot is.}{dat zij er slechts zelden haar}{nut mee weet te doen}\\

\haiku{En er is niemand,.}{om mij te helpen indien}{nog hulp zou baten}\\

\haiku{De vingertoppen....}{van haar man worden al groen}{onder de nagels}\\

\haiku{een verre, vage,.}{en schielijk voorbijgaande}{verschietende God}\\

\haiku{Zij was in vele,.}{dingen onderwezen maar}{in de godsdienst niet}\\

\haiku{Het werken was heel,;}{zijn leven zijn gezondheid}{geweest zijn enige}\\

\haiku{maar alles ontzinkt....}{haar en haar handen tasten}{in het ledige}\\

\haiku{Wij allen zijn in:}{de hand des Heren en het}{is alleen de vraag}\\

\haiku{Het boek, dat hij dan,:}{altijd opnieuw weer openslaat}{spreekt hem niet meer aan}\\

\haiku{Goed, maar wie zegt u,,}{dat de God Die gij aanroept}{niet even misselijk}\\

\haiku{Hij ziet niet altijd.}{wat hij ziet en hij hoort niet}{altijd wat hij hoort}\\

\haiku{Hij is nog slechts aan.}{het begin van zijn lange}{en moeilijke weg}\\

\haiku{E\'en voor \'e\'en kan hij,;}{ze nog voor zijn geest halen}{deze kinderen}\\

\haiku{Het klooster is arm.}{en de vergoeding voor de}{inwoner gering}\\

\haiku{de geschiedenis.}{is daar om er overvloedig}{van te getuigen}\\

\haiku{Hij is opgestaan.}{van zijn bidstoel waarop hij}{neergeworpen lag}\\

\haiku{Item zijn sombrero.}{en hij plaatst hem op het hoofd}{en is reisvaardig}\\

\haiku{de priester in een.}{inrichting voor zielszieken}{onder te brengen}\\

\haiku{Het tweede geval.}{in zeer korte tijd dat me}{onder de ogen komt}\\

\haiku{Een vriend aan een vriend,.}{was de priester hem in de}{rede gevallen}\\

\haiku{Meer dan gebeden;}{zagen wij hem storten over}{zijn gestorven vriend}\\

\haiku{Andere woorden.}{willen niet levend worden}{op hare lippen}\\

\haiku{De schaduw die over;}{zijn gelaat gelegen heeft}{is opgetrokken}\\

\subsection{Uit: Kleine vertellingen}

\haiku{En ook deze zat.}{op zijn beurt over hem met de}{handen in het haar}\\

\haiku{om de schouder lei.}{en zij neuri\"end samen}{hun weg vervolgden}\\

\haiku{Maar ook aan deze.}{geluksstaat kwam helaas een}{vroegtijdig einde}\\

\haiku{Hij bond zijn paard aan.}{een boom en zette zich op}{een steen langs de weg}\\

\haiku{hoe potsierlijk me,,.}{de pluimen die je me op}{de muts steekt tooien}\\

\haiku{En daar zouden zij,,;}{nu nog staan vermoed ik als}{ik niet beter wist}\\

\haiku{{\textquoteleft}maar een roeping, een,,!}{hogere zending dat is}{wat daar hoort wat toe}\\

\haiku{En toen hij alle,:}{ongemakken uit de weg}{had geruimd zei Axel}\\

\haiku{De dame nam Axel.}{weer van de groentevrouw over}{en ging naar haar huis}\\

\haiku{dat wij het, als wij,!}{z\'o voort blijven gaan v\'e\'el te}{ver zullen brengen}\\

\haiku{de geleerde, de,.}{supergeleerde zeker}{de uitzondering}\\

\haiku{schudden de buren;}{met de kop en maakten dat}{ze binnen kwamen}\\

\subsection{Uit: Kroniek eener parochie. Deel 1. De kraai op den kruisbalk}

\haiku{zij vertrouwen hem.}{niet en durven hem nochtans}{niet te wantrouwen}\\

\haiku{Geen dank, eerwaarde,.}{want wij zijn al blij genoeg}{het te m\'ogen doen}\\

\haiku{Een paar grootere:}{boeren met paarden en een}{stal rundvee dachten}\\

\haiku{hoorde ik een stem.}{mij duidelijk vragen en}{ik schrok van haar klank}\\

\haiku{- Ik wist toch wel dat.}{u van die uitnoodiging geen}{gebruik zou maken}\\

\haiku{ik gun het haar, en.}{daar alles mee gezegd wat}{men maar zeggen kan}\\

\haiku{In het priesterkoor;}{brandt de Godslamp voor het}{Allerheiligste}\\

\haiku{De pastoor heeft er.}{op gewacht en zich nog niet}{te bed begeven}\\

\haiku{omdat zij zich te;}{deftig voelde om naast vrouw}{Bonte te zitten}\\

\haiku{Zij staken den kop.}{in de veeren en deden}{alsof zij sliepen}\\

\haiku{Toen ik hem vragend,.}{aanzag vertelde hij mij}{dat hij trouwen ging}\\

\haiku{Ook in den omgang,}{met den jongen was hij schaars}{in woorden die \'als}\\

\haiku{en de jongen is.}{er toch al erg genoeg aan}{toe zonder moeder}\\

\haiku{die dan dikwijls een.}{zekerder uitsluitsel gaf}{dan de gasten zelf}\\

\haiku{Deze wildernis.}{zou in een lusthof kunnen}{herschapen worden}\\

\haiku{hartstochten, deugden,.}{en ondeugden waardoor zij}{gedreven werden}\\

\haiku{Een winterkoning,,.}{niet grooter dan een duim had}{er zijn holletje}\\

\haiku{doch evengoed had ik,.}{hooi kunnen nemen dat nog}{goedkooper is}\\

\haiku{overwonnen door den;}{slaap rusten haar hartstochten}{en vijandschappen}\\

\haiku{aanvankelijk met.}{die kalme tevredenheid}{die haar eigen is}\\

\haiku{Doch het is Miete,!}{alleen niet waarmee wij te}{maken hebben vrouw}\\

\haiku{en ik vertrek op,.}{staanden voet vervolgde hij}{stampend met den voet}\\

\haiku{riep vrouw Briels, als ik;}{eraan denk wat wij  met}{jou gehad hebben}\\

\haiku{doch de freule werd.}{met den dag dan ook meer en}{meer onhandelbaar}\\

\haiku{Doch de schande hing.}{dreigend boven zijn bloed en}{zijn gansche geslacht}\\

\haiku{een wild gericht hangt,.}{boven de huizen dat dan}{plotseling afbreekt}\\

\haiku{kinderen leunen.}{lachend en roepend uit de}{vensters der gevels}\\

\haiku{Kort na den dood van.}{vrouw Van der Schoor al was het}{verkeerd gaan loopen}\\

\haiku{Zoozeer was Bertus niet,.}{het Panhuis vergroeid dat aan}{hem niet geroerd werd}\\

\haiku{alleen de gekke.}{Bert droeg zijn borst wellicht n\'og}{hooger dan anders}\\

\haiku{Doch de kinderen.}{hadden alle aandacht voor}{het vierde gebod}\\

\haiku{En wat is er nu,,.}{van uw dienst eerwaarde biecht}{u nu maar eens op}\\

\haiku{- Toch zult u er niet.}{in slagen mij bij u mijn}{biecht te doen spreken}\\

\haiku{Als ik hem zeg, dat,.}{hij een vernufteling is}{lacht hij grandioos}\\

\haiku{Het contact met zijn.}{parochie heeft hij echter}{nog niet gevonden}\\

\haiku{Paulus Lumens lag.}{zeker reeds in een diepen}{en tevreden slaap}\\

\haiku{Als ik morgen door,;}{het dorp ga zijn de kleine}{huizen gesloten}\\

\haiku{dat klein goed vraagt meer}{dan ik hen geven kan en}{honger en gebrek}\\

\haiku{ik zal er danig.}{over waken moeten en ook}{over de anderen}\\

\haiku{Hij moest het zelf nu;}{maar weten of hij er werk}{van maakte of niet}\\

\haiku{doch als zijn vader,.}{hem ook in den steek liet zijn}{moeder deed het niet}\\

\haiku{Een man loopt met een;}{brandmerk en weet er zich niet}{van te bevrijden}\\

\haiku{Zij heeft een pleister,.}{gelegd op zijn wang die niet}{ophoudt te bloeden}\\

\haiku{De menschen hebben.}{een goed hart en de ruwen}{soms nog het beste}\\

\haiku{Wat moet een mensch op?}{zijn voorbijgang nog meer dan}{groote verwachtingen}\\

\haiku{en de wateren.}{der beproeving komen hem}{tot de lippen}\\

\haiku{ik geloof niet dat.}{mij ooit iemand op mijn plicht}{heeft moeten wijzen}\\

\haiku{brouwer, je gaat er,.}{kapot aan maar ik zal je}{helpen als je wil}\\

\haiku{Louis zijn deel ging bij.}{Van der Schoor in het bedrijf}{met nog wat erbij}\\

\haiku{Het was allemaal,.}{te verbijsterend voor twee}{kleine grijze oogen}\\

\haiku{Laat ons met rozen,.}{kronen voor ons hoofd zich diep}{naar het aschkruis buigt}\\

\haiku{ook in zijn jonge,;}{jaren niet als hij wel eens}{een glas teveel had}\\

\haiku{Hij herinnerde,.}{zich een anderen nacht niet}{lang geleden nog}\\

\haiku{het is de liefde,!}{die alles bederft tot haar}{eigen vruchten toe}\\

\haiku{Niets! - Denk er wel aan,,...}{Lumens ik laat je vandaag}{niet los vooraleer}\\

\haiku{Ook om een waarheid,.}{te verduisteren die \`al}{te duidelijk is}\\

\haiku{- Je zult me moeten,:}{toegeven dat je beeldspraak}{bepaald verward is}\\

\haiku{Hoe weinig is er,!}{noodig om een koning hoeveel}{om een mensch te zijn}\\

\haiku{de figuur van de.}{Mater Dolorosa raakt}{nergens uit ons oog}\\

\haiku{Ik wacht op vader,.}{antwoordde de jongen en}{slenterde verder}\\

\haiku{Ieder zijn eigen;}{droom is hem weer verschenen}{in nieuwen luister}\\

\haiku{zijn morgenbrood in.}{de hand vermorzeld en aan}{de paarden gevoerd}\\

\haiku{Enkele drinkers.}{vegen van verbazing met}{hun mouw langs hun mond}\\

\haiku{Marie-Cathrien.}{heeft het geld geteld en weer}{in de tasch gedaan}\\

\haiku{Mijnheer Bongaerts zingt,;}{nog immer in de kerk maar}{lang zoo hard niet meer}\\

\haiku{Ook wanneer zij zich:}{niet inspant ziet men een hoogen}{blos op haar wangen}\\

\haiku{Van een weduwman.}{met kinderen mag men niet}{alles verwachten}\\

\haiku{den meester is veel.}{eer bewezen en zijn vrouw}{heeft daarin gedeeld}\\

\haiku{Aan de sc\`enes, die,.}{zijn moeder gemaakt had heeft}{hij zich niet gestoord}\\

\haiku{Anderen vinden,;}{het heel natuurlijk dat een}{spin in haar web zit}\\

\haiku{Maar wat je gehad,.}{hebt dat heb je en nemen}{ze je niet meer af}\\

\haiku{Daar mag geen geweld,.}{aan gedaan worden evenmin}{als aan zijn geloof}\\

\haiku{Wij zijn, allemaal,,.}{zondaars eerwaarde maar het}{ligt er maar aan hoe}\\

\haiku{Het doet mij deugd u,.}{dit alles eens te mogen}{zeggen eerwaarde}\\

\haiku{uit ellendigheid.}{en omdat ook ik niet recht}{in mijn schoenen sta}\\

\haiku{En Johannes Den,,?}{Hertog dan tante Dora}{wat heeft hem bezield}\\

\haiku{houdt dien Hollander,,.}{in de gaten Bert want die}{bederft je de pap}\\

\haiku{Dit echter was voor.}{mijnheer Erik Odekerke de}{groote verrassing niet}\\

\haiku{de bisschop heeft ze;}{als geldend aanvaard en dat}{is genoeg voor mij}\\

\haiku{Dit is mijn eenige,.}{rechtvaardiging voor Hem die}{mij oordeelen zal}\\

\haiku{Toen de pastoor zijn.}{bezoekers uitliet stond de}{hemel vol sterren}\\

\haiku{Godsgevangen en,.}{duivelsgeplaagd als bij ons}{de menschen zeggen}\\

\haiku{Het spijt me wel, maar.}{voor u valt hier voorshands nog}{niets te verdienen}\\

\haiku{Doch die hangt af van.}{den wil en gij weet hoe het}{daarmee gesteld is}\\

\haiku{Na den maaltijd nam.}{baron Isidoor den kapelaan}{mee naar het salon}\\

\haiku{De personen die:}{er een rol in spelen zijn}{achtereenvolgens}\\

\haiku{veeren, bladeren.}{en bloemen die opstuiven}{in de platanen}\\

\haiku{Ik heb veel voor jou,!}{gebeden en ook voor u}{mijnheer Den Hertog}\\

\subsection{Uit: Kroniek eener parochie. Deel 2. De mensch en zijn schaduw}

\haiku{Eerst de keuken dan,,?}{maar want van de maag moet je}{het toch hebben niet}\\

\haiku{Niets dan miserie,?}{had je van je jongens te}{verwachten of niet}\\

\haiku{En als de korsten.}{hen steken moet de vader}{het maar ontgelden}\\

\haiku{Nicolaas, je moet;}{alles nu ook niet erger}{maken dan het is}\\

\haiku{heeft mijnheer Lumens,?}{gestameld met wanhoop in}{de oogen wat nu Erik}\\

\haiku{Gelukkig maar dat,!}{wij alles van te voren}{niet weten pastoor}\\

\haiku{Laat het dan ook voor,,!}{jou een probleem zijn maar help}{het oplossen Erik}\\

\haiku{Harten waren voor,.}{hem opengegaan oogen hadden}{hem toegeblonken}\\

\haiku{Doch als de heele,?}{keuken voor mijn oogen begint}{te draaien pastoor}\\

\haiku{Die zou een beetje,;}{aardig en stijlvol moeten}{zijn en gezellig}\\

\haiku{De pret was soms zoo}{groot dat hun vader zich een}{paar maal genoodzaakt}\\

\haiku{Stommerik, viel zijn,!}{vader tegen hem uit doe}{dat ding van je kop}\\

\haiku{was het antwoord en.}{ten einde raad trok zij er}{mee naar de keuken}\\

\haiku{als er over ieder,.}{stroospier gestruikeld moet worden}{wordt liet huis een hel}\\

\haiku{Toen deze binnen.}{het bidden hoorde bleef hij}{even staan luisteren}\\

\haiku{Moe is het woord niet,,.}{eerwaarde maar ik voel me}{diep ongelukkig}\\

\haiku{Moeder, zei hij voor,}{hij de deur uitging ik hoop}{dat de zegen dien}\\

\haiku{wel moest hij haar tot,,}{haar eer nageven dat zij}{hoe ziek zij reeds was}\\

\haiku{Je hebt me de pap,;}{in den mond gegeven zei}{mijnheer Odekerke}\\

\haiku{doch wanneer hij op.}{zwart zaad zat was er geen huis}{met hem te houden}\\

\haiku{Maar die heeft dan ook:}{m\'e\'er dan Lambert en Peter}{zijn uiterlijk mee}\\

\haiku{Wien het juk dat ze}{om eigen bestwil op zich}{genomen hebben}\\

\haiku{Wil u nu nog meer,.}{van me weten dan moet u}{maar spreken Heeroom}\\

\haiku{bij Weenink over den,;}{vloer geweest omdat ik hen}{niet wou passeeren}\\

\haiku{Daar is niemand die.}{beter weet hoe hij me klein}{moet krijgen dan jij}\\

\haiku{Met dat bescheid had.}{hij haar de voordeur voor de}{neus dichtgesmeten}\\

\haiku{Maar toen hij vroeg wat,.}{hij voor ons doen kon wist ik}{het niet te zeggen}\\

\haiku{Nu wel, nu heb je}{me veel te verbergen en}{nu belieg je me.}\\

\haiku{Laat die hunnen draai,.}{maar eens krijgen dan wordt de}{wereld nog te klein}\\

\haiku{Er is reeds gezegd.}{dat er genoeg geblaft wordt}{onder de dieven}\\

\haiku{Dan moet je ze maar...!}{eens in de herbergen zien}{en de cinema}\\

\haiku{Die komt het verste in,.}{de wereld voegde zij er}{uit zichzelf aan toe}\\

\haiku{De vrijgestelden,!}{schijnen het bij jou verbruid}{te hebben jongen}\\

\haiku{Custers knikt dat hij,.}{het gehoord heeft en ook de}{anderen knikken}\\

\haiku{was hij kort van draad,!}{maar d\`at had hij haar toch wel}{even kunnen zeggen}\\

\haiku{Wil je soms zeggen?}{dat ik niet hard genoeg voor}{jullie gezwoegd heb}\\

\haiku{Louis was er al vroeg,.}{bij daarna Dorus en daarna}{Lambert en Peter}\\

\haiku{wat eigenlijk het.}{middelpunt van haar wereld}{en verlangens is}\\

\haiku{Met antipathie,,.}{zooals mij verweten wordt heeft}{dat niets te maken}\\

\haiku{Het had allemaal,.}{anders kunnen loopen en}{ook beter misschien}\\

\haiku{Het is altijd de,.}{l\'a\'atste stap die Nicolaas}{Bonte moeite kost}\\

\haiku{Als hij in het dorp.}{komt zijn de verkiezingen}{reeds in vollen gang}\\

\haiku{Thuis reikt zijn vrouw hem,.}{een brief van Karel juist met}{de post gekomen}\\

\haiku{Voor het overige.}{gaat hij zich in woorden n\`och}{in drank te buiten}\\

\haiku{Zijn vader zet de,!}{bloemetjes eens buiten dat}{mocht wel voor een keer}\\

\haiku{Het spijt me, moeder,,!}{dat ik het zeggen moet u}{maakt u zelf iets wijs}\\

\haiku{Dat is niet mooi van,,!}{u gezegd moeder dat is}{zelfs leelijk en grof}\\

\haiku{Het zou de eerste!}{keer niet zijn dat jij je over}{je ouders schaamde}\\

\haiku{Heeft u me zelf dan,?}{niet gezegd dat het ver met}{ons gekomen is}\\

\haiku{Ik wil niets, ik geef.}{je den raad te doen wat je}{moeder van je vraagt}\\

\haiku{In ieder geval.}{moogt u weten dat ik uw}{raad niet zal volgen}\\

\haiku{evengoed als je te,.}{preeken mis te lezen err}{biecht te hooren hebt}\\

\haiku{Je mag gerust een.}{boon zijn als je me nu eerst}{maar eens praten laat}\\

\haiku{Hoemeer jij pastoor,;}{tegen me speelt hoemeer hij}{me te pakken neemt}\\

\haiku{Eerlijk, dat kun jij.}{voor God en je geweten}{niet verantwoorden}\\

\haiku{Een mensch kan er oud,.}{en gebrekkig in worden}{tot den dood vermoeid}\\

\haiku{Zij komt van bij Dorus.}{met den halsdoek aan den mond}{tegen den wind in}\\

\haiku{Hij is een echte,,.}{broer van Louis zegt zijn moeder}{soms maar hij is trouw}\\

\haiku{Jacob heeft zeker.}{de bel niet horen overgaan}{toen zij binnenkwam}\\

\haiku{Hij wil weten wat.}{er gebeurd is en vraagt het}{hen zonder verlet}\\

\haiku{Nicolaas Bonte,.}{maakt korte metten dat is}{men van hem gewoon}\\

\haiku{Zij zeggen dat zij.}{op het ziekenhuis ligt en}{dat zal dan wel zijn}\\

\haiku{Van mijn kant heb ik.}{niets zoozeer verlangd dan u die}{rust te vergallen}\\

\haiku{Indien zij er ligt.}{zal zij wel nergens beter}{kunnen zijn dan daar}\\

\haiku{Telkens blijft deze;}{ondanks zijn voornemens in}{zijn aanloop steken}\\

\haiku{En dat juist omdat.}{de priester de dienaar van}{den Gekruiste is}\\

\haiku{en moeten wij zelf?}{niet door ervaring wijzer}{worden over onszelf}\\

\haiku{En weten alle!}{menschen wat u zich in den}{kop heeft  gehaald}\\

\haiku{Als dat zoo met je}{door blijft gaan spring je nog eens}{de vuilniskar op}\\

\haiku{Nicolaas Bonte.}{is een van diegenen die}{nooit capituleeren}\\

\haiku{Er zijn goede en,.}{kwade engelen goede}{en kwade driften}\\

\haiku{tobben was het en.}{krom liggen om de touwtjes}{aaneen te krijgen}\\

\haiku{Het is de tweede,}{keer in een maand tijds nu al}{dat me dit overkomt}\\

\haiku{Je moest wijzer zijn,,.}{Bonte na alles wat je}{achter den rug hebt}\\

\haiku{m\`ere Canisia.}{had hem reeds gezegd wat er}{van hem verwacht werd}\\

\haiku{XXIV Sedert mijnheer}{Lumens den dood in de oogen}{van zijn kapelaan}\\

\haiku{heeft pastoor Lumens.}{geantwoord met een geloof}{dat bergen verzet}\\

\haiku{Zwart tegen zwart, heeft,,!}{Coenraad gezegd vuil tegen}{vuil wij zullen zien}\\

\haiku{een parochie kan.}{niet meer gestraft zijn dan door}{het gemis daaraan}\\

\haiku{Als de klok het uur;}{slaat knielt de parochie voor}{het tabernakel}\\

\subsection{Uit: Kroniek eener parochie. Deel 3. De weg zijner zonen}

\haiku{Allen zijn eenzaam;}{die het geloof aan zichzelf}{verloren hebben}\\

\haiku{Heb je nog iets van,?}{je vader gehoord waar die}{zich ophoudt of zoo}\\

\haiku{Reden te meer voor.}{de menschen om elkander}{er door te helpen}\\

\haiku{Zulke menschen zijn.}{er helaas en ze zullen}{er altijd blijven}\\

\haiku{Het is genoeg om;}{het hart van mijnheer Lumens}{al warm te maken}\\

\haiku{hij legt de hand op.}{het hoofd van het kind en treedt}{de keuken binnen}\\

\haiku{Op \'e\'en kostganger!}{meer of minder kwam het bij}{Van der Schoor niet aan}\\

\haiku{Spreek er me niet van,;}{zegt hij wanneer iemand hem}{met raad wil helpen}\\

\haiku{onder de snikken.}{der oude Geertrui was zij}{de deur uitgegaan}\\

\haiku{gedane zaken.}{nemen geen keer en berouw}{komt na de zonde}\\

\haiku{Hij had alleen maar.}{niet geweten waar hij de}{kracht vandaan haalde}\\

\haiku{Severinus van.}{der Schoor constateert het met}{spijt en achterdocht}\\

\haiku{je doet afstand of,,.}{niet je hebt den moed of je}{hebt hem niet zegt hij}\\

\haiku{zei Peter Tobben.}{opeens en liet verwonderd}{den hamer zakken}\\

\haiku{Sedert je ziek bent.}{kan men ze den evenveel van}{het lijf aflezen}\\

\haiku{Om den dooien dood,.}{niet maar je moet het er dan}{ook niet naar maken}\\

\haiku{Een brievengaarder.}{zonder duim is ook maar een}{steel zonder lepel}\\

\haiku{en zijn oogen gaan op,.}{en af nu eens wijd open en}{dan weer langzaam toe}\\

\haiku{Reinout Eussen was een:}{eind in zijn richting de hei}{op gegaan en had}\\

\haiku{het past niet daarover,.}{te oordeelen zeker den}{buitenstaander niet}\\

\haiku{Den meesten tijd weet.}{je van voren niet dat je}{van achteren leeft}\\

\haiku{Het kan Dorus Bonte,.}{bliksems veel schelen dat het}{met Nico misloopt}\\

\haiku{De lange Peter,}{staat voor niets vooral niet als}{het er op aankomt}\\

\haiku{Dat is meer dan u,!}{van een redelijk schepsel}{vragen kunt baron}\\

\haiku{dat ze nog niet te!}{genieten zou zijn als ze}{op brandewijn stond}\\

\haiku{hij woont zijn voor zijn,.}{invallen te dom voor zijn}{grappen te nuchter}\\

\haiku{Daarom verlangt de.}{brievengaarder naar niets zoo}{zeer als naar zijn dienst}\\

\haiku{Neen, van de vrouwen!}{moet men Peter Tobben niets}{nieuws meer vertellen}\\

\haiku{en wie er kwaad over}{spreken zijn de ergsten als}{het er op aankomt}\\

\haiku{Buiten staat dokter.}{Liebaert zijn motor aan te}{trappen die weigert}\\

\haiku{de grootste misschien.}{die de Mijn tot nog toe op}{haar geweten heeft}\\

\haiku{Nico weet nog van,.}{niets die is met een wagen}{naar het buitenland}\\

\haiku{Wat een mensch tusschen!}{opstaan en slapengaan al}{niet overkomen kan}\\

\haiku{Peter Tobben heeft!}{er niet voor niets de winkels}{voor afgeloopen}\\

\haiku{Lambert gaat met hen.}{naar buiten maar niet verder}{dan tot het hekje}\\

\haiku{De navraag die men;}{naar hem gedaan had was op}{niets uitgeloopen}\\

\haiku{Zelf kan hij zich niet,.}{herinneren dat hij er}{ooit een gestort heeft}\\

\haiku{Misschien ga ik naar,,;}{Amerika zei Lambert droog}{en misschien ook niet}\\

\haiku{Even daarna luwde}{het getij en toen in een}{stilte vol snikken}\\

\haiku{de dunne vingers,}{had hij het voornamelijk}{tegen een man dien}\\

\haiku{riep Klabbers spontaan.}{met een slag van de platte}{handen op tafel}\\

\haiku{Hals over kop liet hij}{de vertering bij zijn glas}{achter en alsof}\\

\haiku{Over het hoe, maakte;}{hij zich voorloopig de}{grootste zorgen niet}\\

\haiku{Of als Nico maar!}{eens eerder geweten}{had wat hij nu wist}\\

\haiku{Het lanterfanten.}{was tenminste gedaan en}{daar was hij blij om}\\

\haiku{zij was maar blij, dat.}{Peter  op stukken na}{z\'o\'o niet gelukt was}\\

\haiku{dat zijn offer met!}{dat van Reinout Eussen niet te}{vergelijken viel}\\

\haiku{Tegen Maria die,:}{hem vroeg wat er nu aan de}{hand was zei Lambert}\\

\haiku{er valt niet mee te;}{lachen met wat Lena over}{het hoofd is gegaan}\\

\haiku{heel je leven op!}{water en brood zou beter}{te verdragen zijn}\\

\haiku{Maar de reden, die,;}{hij haar gevraagd had had ze}{hem nooit gegeven}\\

\haiku{vraagt Severinus.}{den mijnportier kregel wat}{er van zijn dienst is}\\

\haiku{een die zoo met lijf,.}{en ziel in het bedrijf zou}{opgaan zeker niet}\\

\haiku{en daarop barsten.}{dan de goden los in een}{algemeen gelach}\\

\haiku{Nee, d\'at vergeet ze,.}{nooit daar kent Lambert Bonte}{zijn schoonzuster voor}\\

\haiku{God en het eigen.}{geweten maken dat uit}{en niet de menschen}\\

\haiku{Maria van Dorus komt.}{met de pan melk onder den}{voorschoot de Kamp af}\\

\haiku{om den tijd hoeft hij,.}{het niet te laten dien heeft}{hij meer dan genoeg}\\

\haiku{ze rijden en het.}{laat Lambert al hoe langer}{hoe meer koud waarheen}\\

\haiku{Lieske had haar in.}{haar vellenjas de deur wel}{uit kunnen kijken}\\

\haiku{Lambert telt het geld:}{tot den laatsten cent op de}{tafel uit en zegt}\\

\haiku{Die geschiedenis;}{met oom Lambert is weer een}{zorg en verdriet apart}\\

\haiku{Een goede vrouw is,;}{een lot uit de loterij}{\'e\'en op de duizend}\\

\haiku{Indien ze allen,!}{als Maria van Dorus waren}{dan \`a la bonheur}\\

\haiku{Dat heb ik voor heel,.}{mijn leven getuigt Lambert}{met heiligen ernst}\\

\haiku{Oom Lambert loopt niet}{meer met de kin op de borst}{en zijn beteuterd}\\

\haiku{Je moet niet lachen,,!}{Willem Bidlot en jij ook}{niet meesterbrouwer}\\

\haiku{twee handhavers der.}{orde staan met de armen}{gekruist bij de deur}\\

\haiku{Laat me met rust, hebt,;}{ge mij gezegd doch daar zijt}{ge niet mee gered}\\

\haiku{zijt - over God spreek ik -.}{niet eens en herinner u}{uw goede dagen}\\

\haiku{En dat voor de paar?}{magere jaren die ge}{nog te leven hebt}\\

\haiku{niet om mijnheer van -}{den Brande een hand boven}{het hoofd te houden}\\

\haiku{Daar vindt Lambert den,.}{oudste te voorzichtig voor}{veel te verstandig}\\

\haiku{Een mensch geneest niet.}{van een ziekte om er weer}{in te hervallen}\\

\haiku{Zouden wij elkaar?}{na zooveel jaren niet eerst}{eens een hand geven}\\

\haiku{En wie zegt jou, dat?}{we het verleden niet met}{rust kunnen laten}\\

\haiku{En wat hebben wij?}{eigenlijk te verbergen}{wat niet openbaar is}\\

\haiku{Je zou de eerste,.}{niet zijn die last kon krijgen}{van die bekoring}\\

\haiku{Die had zwaar voor zijn.}{domheden geboet en die}{boette misschien nog}\\

\haiku{De menschen hadden,.}{er dus het hunne weer van}{gemaakt dacht Lambert}\\

\haiku{Geen van de broeders!}{zou zich voortaan de vingers}{nog aan hem branden}\\

\haiku{Zou je niet naar de,?}{Lindeboom terug willen}{komen Louis Bonte}\\

\haiku{Ik evenwel weet niet,;}{wat daar zoo maar pardoes op}{te antwoorden baas}\\

\haiku{Louis Bonte ging niet,;}{terug naar Canada werd}{algemeen verteld}\\

\subsection{Uit: Mijn moeder Elisabeth}

\haiku{alhoewel ik niet.}{kan zeggen dat ze er nooit}{naar gestaan hebben}\\

\haiku{De boeken die ik;}{gelezen heb willen me}{anders doen gelooven}\\

\haiku{welken reuk zou ik,:}{moeilijk kunnen zeggen maar}{geen aangenamen}\\

\haiku{m\'e\'er dan het builtje -;}{met loon dat dan toch eten en}{leven beteekende}\\

\haiku{Als de vooruitgang?}{van het menschelijk geslacht}{ermee gediend was}\\

\haiku{want van den kelder.}{tot den zolder kwam er in}{zijn huis niets te kort}\\

\haiku{al schoot zij dan ook.}{in den noodigen eerbied voor}{hem nimmer te kort}\\

\haiku{Der Joehan was naar,;}{het dorp naar de Schapens An}{om een pond reuzel}\\

\haiku{Dan wendde hij zich,.}{tot mijne moeder die stil}{over haar naaiwerk zat}\\

\haiku{en die is tot nog,....}{toe door niemand verbeterd}{geworden meen ik}\\

\haiku{Doch geen onzer heeft.}{zich ooit over zijn Heergod te}{beklagen gehad}\\

\haiku{de twee mannen met.}{elkander op en neer naar}{de Mijn te zien gaan}\\

\haiku{Maar het kwam dan toch.}{nog altijd in zijn geheel}{bij moeder terecht}\\

\haiku{Omdat andere;}{mannen dat w\`el doen als ze}{boos op hun vrouw zijn}\\

\haiku{der Goore Joep doet}{het altijd als hij dronken}{is en dan loopen}\\

\haiku{Daar zijn dingen waar,!}{een vrouwmensch nu eenmaal geen}{verstand van heeft vrouw}\\

\haiku{Hoelang reeds gingen?}{vader en moeder niet meer}{samen op en af}\\

\haiku{Hoeveel keeren denk je?}{dan dat ik haar nog naar Aken}{zal moeten brengen}\\

\haiku{Ik kom niet buiten '}{de deur dan oms Zondags}{naar de kerk te gaan}\\

\haiku{en het is zomer,;}{en winter altijd naar de}{vroegmis dat weet je}\\

\haiku{De krant bleef open op.}{tafel liggen en niemand}{sprak verder een woord}\\

\haiku{Bij allebei ben,,....}{ik welkom maar ik denk dat}{ik in het dorp blijf}\\

\haiku{Dat hij nog bidt, en,;}{dat zij die bidden nimmer}{verloren gaan ja}\\

\haiku{doch men verdacht hem,}{ervan er een kunstgreep bij}{van pas te brengen}\\

\haiku{zijn middelen uit.}{nooddruft en veelal schrijnend}{onrecht geboren}\\

\haiku{, hadden we reeds lang.}{geweten en hem ook wel}{eens laten voelen}\\

\haiku{Wat die brief echter;}{behelsde ben ik nooit te}{weten gekomen}\\

\haiku{Maar de hand die hij,;}{me daarbij aan den schouder}{legde woog me zwaar}\\

\haiku{dat hij haar wel naar.}{den trein zou brengen in de}{koets van het Stetsje}\\

\haiku{zei der Hannesso\"e,;}{die daar meteen zijn lepel}{op tafel gooide}\\

\haiku{maar z\'o\'o ook is het!}{al bitter genoeg wat je}{me te slikken geeft}\\

\haiku{De kinderen aan,,?}{huis te hechten is ook wat}{waard dacht ik Jozef}\\

\haiku{Je kunt eruit leeren,!}{wat erin staat niet minder}{en niet meer dan ik}\\

\haiku{Was der Joehan daar?}{onderhand oud genoeg voor}{geworden of niet}\\

\haiku{al zou men hem uit.}{de hel moeten halen of}{met goud beleggen}\\

\haiku{ging der oom Joehan.}{met een dreigend kijken in}{mijn richting verder}\\

\haiku{bol moeten zijn die.}{hem dat alsnog duidelijk}{zou kunnen maken}\\

\haiku{Laat de menschen niet;}{meenen dat we iemand naar}{de o ogen kijken}\\

\haiku{mijn moeder kosten.}{zou haar heelemaal als van}{ons te aanvaarden}\\

\haiku{doch ik had er een.}{gevoelen van waarvoor ik}{geen verklaring wist}\\

\haiku{want die jongen liep.}{toch reeds met het merkteeken van}{Satan in zijn vel}\\

\haiku{Ik stelde me dan}{ook voor dat ik zonder een}{spier te vertrekken}\\

\haiku{Ik moest het hoofd maar,;}{niet dieper laten zinken}{dan noodig was schreef ze}\\

\haiku{Dat heb je er nu,!}{van je had beter kunnen}{blijven waar je was}\\

\haiku{een ervaring, die.}{de hoefsmid Lucassen reeds}{lang had opgedaan}\\

\haiku{Hij had rijkelijk;}{den leeftijd om een meisje}{aardig te vinden}\\

\haiku{er hun rekenschap!}{van afleggen zou zij in}{ieder geval niet}\\

\haiku{Op zekeren dag -:}{ik weet nog goed dat het op}{St. Hubertus was}\\

\haiku{En hij heeft dat lang,.}{volgehouden langer dan}{mijn vader lief was}\\

\haiku{Doch daarmee loopen!}{wij de geschiedenis weer}{een goed eind vooruit}\\

\haiku{niet z\'o\'o bezopen,!}{natuurlijk maar toch alsof}{Atjeh gevallen was}\\

\haiku{Of meende ik soms?}{dat de menschen compassie}{met me moesten hebben}\\

\haiku{Dat was de tweede;}{maal geweest dat zij me ter}{wereld had gebracht}\\

\haiku{Ik moest weer zeggen.}{dat ik gelukkig was weer}{bij moeder te zijn}\\

\haiku{handen, die bijna.}{met alle werkelijkheid}{hadden afgedaan}\\

\haiku{dat die ook met God.}{en met eere tot hunne}{bestemming kwamen}\\

\haiku{Ik stamelde dat.}{men het ook niet erger moest}{maken dan het was}\\

\haiku{En wie er een hoofd!}{voor gekregen had moest dat}{maar leeren gebruiken}\\

\haiku{zoover al dat jij je!}{dochter nu spoedig naar het}{altaar ziet leiden}\\

\haiku{echter zonder de.}{Voorzienigheid ooit uit het}{oog te verliezen}\\

\haiku{Want wanneer de zon.}{eenmaal onder was werden}{de avonden weer kil}\\

\haiku{En daar kwam dan niet,;}{zelden nog schelden tieren}{en dronkenschap bij}\\

\haiku{Die wordt ook met den,;}{dag al gemakkelijker}{vond die Annebil}\\

\haiku{En daar tusschen de;}{nachtwacht met de kanarie}{in de lantaren}\\

\haiku{Als hij dan maar aan,:}{de afspraak dacht drukte der}{Klaus hem op het hart}\\

\haiku{met nog weinig olie.}{in mijn scharnieren en met}{een beetje hoogen rug}\\

\haiku{Wel grepen een paar;}{posteerende Duitschers me}{al gauw bij den kraag}\\

\haiku{En samen hebben.}{wij dan de koeien van de}{zusters gemolken}\\

\haiku{Vandaag alleen heb;}{ik er meer gemaakt dan mijn}{heele leven lang}\\

\section{Emile Seipgens}

\subsection{Uit: De kapelaan van Bardelo}

\haiku{{\textquoteleft}Maar... maar... maar...{\textquoteright} uitte,.}{hij nog zonder te weten}{wat hij zeggen zou}\\

\haiku{Zijn enige troost was;}{het menigvuldig bezoek}{van zijn collega's}\\

\haiku{Gij zijt een steenrots,!}{en op deze steenrots zal}{ik mijn kerk bouwen}\\

\haiku{{\textquoteright} Dirk begreep er niets.}{van hoe de pastoor opeens}{aan een steenrots kwam}\\

\haiku{Toen de karavaan ',.}{s avonds te K. aankwam was}{Peterke doodziek}\\

\haiku{De volgende dag.}{verlangde Arnold Peter}{alleen te spreken}\\

\haiku{de knecht en de meid.}{knielden in een andere}{hoek van de kamer}\\

\haiku{{\textquoteright} Doch Peter zag hem.}{opgetogen aan en hief}{de hand ten hemel}\\

\haiku{Peter Grubbeler.}{mocht terecht een toonbeeld van}{priesterdeugd heten}\\

\haiku{Zijn priesterwijding.}{is inderdaad enkele}{maanden uitgesteld}\\

\haiku{Achter de namen:}{van 3 laatstgenoemden komt}{de opmerking voor}\\

\haiku{Van oktober 1856.}{tot maart 1857 heeft hij dus op}{dat adres  gewoond}\\

\haiku{Deze getuigen {\textquoteleft}{\textquoteright}.}{warende naburen van}{de overledene}\\

\haiku{Misschien moeten wij.}{het antwoord zoeken bij het}{type Seipgens zelf}\\

\haiku{Causerie over {\textquoteleft}De{\textquoteright} ().}{kapelaan van Bardelo}{niet gepubliceerd}\\

\haiku{Of liberaal is,:}{ook iemand die b.v. over eene}{preek wel eens oordeelt}\\

\haiku{Z\'o\'o is 't op 't - '.}{platteland in de stad is}{t eenigszins anders}\\

\section{Kees Simhoffer}

\subsection{Uit: Een geile gifkikker}

\haiku{{\textquoteleft}U mag hem gerust,,}{proberen als u wel uw}{schoenen wilt uitdoen}\\

\haiku{Hij kijkt me aan en.}{pakt een sigaret uit een}{metalen koker}\\

\haiku{{\textquoteleft}Ik zie het al. In.}{het weekend een beetje te}{lang buiten geweest}\\

\haiku{Of schaft u zich een.}{Philifoonsmell Klank-}{en Reukorgel aan}\\

\haiku{Tenzij wij hun slecht.}{bekomen en ze ons nog}{\'e\'en keer uitkotsen}\\

\haiku{Ik help haar uit de.}{vrieskast en sla de rijp van}{haar fluwelen jurk}\\

\haiku{Ze denken gewoon.}{dat we zoals altijd aan}{het ontbijt zitten}\\

\haiku{Ze doet het knipje.}{van de badkamerdeur en}{komt fris naar buiten}\\

\haiku{Zonde van al die.}{tandpasta met aktieve}{superfluorol}\\

\haiku{{\textquoteright} {\textquoteleft}Mijn moeder vindt dat.}{pletsspuitbus sneller klient dan}{peliwrijfwas boent}\\

\haiku{De erwten rollen.}{met honderden tegelijk}{over de kale grond}\\

\haiku{gelukkig dat ie...}{eindelijk verstandig is}{en naar de dokter}\\

\haiku{Scheidt een giftige,.}{zeer besmettelijke stof}{af in paringstijd}\\

\haiku{{\textquoteright} {\textquoteleft}Ik zou maar gauw een,{\textquoteright}.}{bad nemen als ik jou was}{antwoordt de moeder}\\

\haiku{Even later lig ik.}{languit in bad en blaas het}{schuim van mijn handen}\\

\haiku{{\textquoteright} Dan draait ze zich om.}{en loopt een lange gang in}{achter het paneel}\\

\haiku{Ik duik onder het.}{bed door en ren op de deur}{in het paneel af}\\

\haiku{We zullen eens een.}{keertje niet onderdoen voor}{de geschiedenis}\\

\haiku{Later pas ga je,.}{begrijpen wat je veel te}{jong nog moest leren}\\

\haiku{De jonge mensen.}{staan al dertig jaar in de}{rij voor een woning}\\

\haiku{En als de Turk griep,.}{heeft ontsmetten we niet meer}{meteen zijn kamer}\\

\haiku{{\textquoteright} {\textquoteleft}Laat maar,{\textquoteright} zeg ik en}{hij loopt terug naar de bar}{met een gezicht van}\\

\haiku{Daar begint ie met.}{beverige handen zijn}{broek los te maken}\\

\haiku{Ik heb er allang,.}{geen haakje meer op alleen}{in verband met Joop}\\

\haiku{Als ik een tijdje,.}{later het caf\'e uitkom}{regent het nog steeds}\\

\haiku{Zoals dokters een:}{kankerpati\"ent naar huis}{sturen en zeggen}\\

\haiku{Ik herken haar pas.}{als de trein de donkere}{overkapping uit is}\\

\haiku{{\textquoteleft}Uw tante is net,...}{van dat paard gevallen ik}{geloof dat ze dood}\\

\haiku{tante voelt zich niet,.}{zo goed vanmorgen tante}{heeft weer wat gebloed}\\

\haiku{{\textquoteleft}Sorry, ook zonder.}{pokkenbriefje kan een mens}{de klere krijgen}\\

\haiku{We rijden weg naar,,.}{het zuiden de stad uit over}{de autosnelweg}\\

\haiku{Dit uitzicht wordt u{\textquoteright}}{aangeboden door Poly}{Chemic Company}\\

\haiku{Ik steek de straat over.}{en kom even later in een}{drukke winkelstraat}\\

\haiku{Er is niks aan de.}{hand. Iedereen loopt weer druk}{boodschappen te doen}\\

\haiku{In Rex draait nog steeds,.}{blanke  onschuld ook al}{regent het niet meer}\\

\haiku{Maar als ze een doek,.}{over me heen willen leggen}{gaat de telefoon}\\

\haiku{{\textquoteright} roept de chauffeur van,.}{de volgende vrachtauto}{hangend uit zijn raam}\\

\haiku{(Conventie van de.}{Verenigde Naties van}{9 december 1948}\\

\haiku{{\textquoteright} ~ Terwijl ze de,}{woorden van Dylan zingt vraag}{ik me verbaasd af}\\

\haiku{{\textquoteleft}H\`e leuk dat u toch,}{nog tijd gevonden hebt om}{te komen kijken}\\

\haiku{{\textquoteright} zegt een dikke man,.}{tegen ons het vet glimmend}{in zijn mondhoeken}\\

\haiku{{\textquoteright} schreeuw ik en duik hem,}{na maar ik kom terecht in}{een leeg donker gat}\\

\haiku{{\textquoteright} Ze staart angstig naar.}{de twee lege spoelen die}{langzaam ronddraaien}\\

\haiku{En de vierde keer.}{had u weer bloemen in de}{vaas en geen sleutel}\\

\haiku{Ik zit immers al.}{gevangen in het web van}{een andere spin}\\

\haiku{{\textquoteright} In paniek ren ik,.}{weg zo snel mijn benen me}{dragen kunnen}\\

\haiku{Men is het echter.}{nog niet eens over de vorm van}{de biljarttafel}\\

\section{Jozef Simons}

\subsection{Uit: Eer Vlaanderen vergaat}

\haiku{En vandaag was het.}{de inhuldiging van de}{nieuwe kasteelheer}\\

\haiku{{\textquoteright} En om het over een,:}{andere boeg te gooien}{zei hij wat later}\\

\haiku{het is uw eigen!}{vader die mij dat vroeger}{heeft ingeblazen}\\

\haiku{Zul je hier liever?}{wonen dan te Brussel of}{te Blankenberge}\\

\haiku{Boven de tuinhaag:}{zagen ze het hoofd van een}{fietser voortglijden}\\

\haiku{Mocht weer het uur voor,?}{Vlaanderen slaan hoevelen}{hebben de ogen open}\\

\haiku{Ik ben altijd veel.}{te goed geweest dan dat je}{mij dat zou aandoen}\\

\haiku{En hoe schandelijk ',!}{gingen ze te werk mett}{vrouwvolk jong en oud}\\

\haiku{hij was toch geen kind -.}{meer en achter de IJzer}{te gaan meestrijden}\\

\haiku{zou papa evenwel.}{liever hebben dan dat hij}{naar het leger trok}\\

\haiku{Met wat een hartstocht.}{volgde hij het streven van}{de activisten}\\

\haiku{{\textquoteright} - ze moesten de paarden.}{gaan roskammen en wrijven}{en haver geven}\\

\haiku{Sedert 1830 begon.}{de lijdensgeschiedenis}{van het Vlaamse volk}\\

\haiku{Wat er niet genoeg,.}{uit sprak was zelfstandigheid}{en daadvaardigheid}\\

\haiku{{\textquoteleft}En gij, Fons, zoudt gij?}{ook voor Vlaanderen door een}{vuur willen lopen}\\

\haiku{{\textquoteright} De brancardier ging,:}{met een beursje rond en toen}{hij telde vond hij}\\

\haiku{De tocht vorderde,,.}{langzaam over Bulskamp Nieuwe}{Herberg Hoogstade}\\

\haiku{Om halfnegen kwam.}{plots het bevel dat ze moesten}{tegen-schieten}\\

\haiku{{\textquoteright} {\textquoteleft}Ja, maar soms heb ik,.}{toch heimwee naar het voetvolk}{naar de piotten}\\

\haiku{In die richting mag,.}{op mij gerekend worden}{ik houd me bereid}\\

\haiku{- Marieke .... 'k Dacht!}{toch dat ik Marieke had}{zien binnenkomen}\\

\haiku{Dat onze Staat het,:}{recht had ons bloed en leven}{te vragen dit is}\\

\haiku{daaruit sprak immers.}{een geest bereid tot opstand}{tegen het onrecht}\\

\haiku{IV De maand maart met,.}{haar lichtere dagen bracht}{meer gevechtsdrukte}\\

\haiku{{\textquoteright} zei de wachtmeester,.}{met tien vingers tegelijk}{in zijn haar krabbend}\\

\haiku{de twee andere.}{stukken moesten ook achteruit}{in reparatie}\\

\haiku{Een vliegtuig toerde.}{boven de haven waar de}{lichtjes aanploften}\\

\haiku{{\textquoteright} {\textquoteleft}Vier uur door... en zijt,!}{maar br\^a content ge krijgt straks}{allemaal rijstpap}\\

\haiku{Florimond was er.}{beschaamd over dat hij zichzelf}{niet meer meester bleef}\\

\haiku{Clara vernam dan.}{ook wat meer over Florimond}{en zijn familie}\\

\haiku{Trouwens, ik geloof.}{niet dat Borms koel en klaar ziet}{waar hij naartoe gaat}\\

\haiku{{\textquotedblleft}Es ist unsere,.}{verdammte Pflicht Oesterreich}{treu zu verraten}\\

\haiku{{\textquoteright} {\textquoteleft}Jan,{\textquoteright} zei Florimond, {\textquoteleft}:}{herinner u het woord van}{onze kapelaan}\\

\haiku{Het plan staat me klaar,?}{voor de geest maar wie helpt mij}{aan de middelen}\\

\haiku{Onopgemerkt sloop,.}{hij weer de winkel door over}{de straat naar het strand}\\

\haiku{En hoe de mannen!}{leute hadden met de schrik}{van de gendarmen}\\

\haiku{Of slechts, omdat ze?}{bij Depla geen gehoor}{hadden gevonden}\\

\haiku{Niets dan gas - tot de!}{Fritzen ervoor bedanken}{en het aftrappen}\\

\haiku{Kef,{\textquoteright} deed een kleine, '.}{vijfenzeventiger die}{t aanvalsuur aangaf}\\

\haiku{In een holle weg.}{werden de stukken tegen}{de haag getrokken}\\

\haiku{Gaarne offer ik,.}{mijn leven op aan God voor}{vrijheid en voor recht}\\

\haiku{{\textquoteright} zei, met zijn zware,,.}{basstem de Antichrist de}{wenkbrauwen fronsend}\\

\haiku{geen klare klank van.}{hogere idealen een}{weerklank vinden zou}\\

\haiku{{\textquoteleft}Maar gaat toch niet naar,...}{Ramskapelle daar schieten}{ze met schrapnellen}\\

\haiku{De mensen hadden.}{al lang moeite om zich stil}{en koest te houden}\\

\haiku{En de pastoor stond!}{de laatste tijd bij het volk}{slecht aangeschreven}\\

\haiku{Jan en Clara moesten.}{aanhoudend uitwijken voor}{bellende fietsers}\\

\haiku{{\textquoteright} Opeens greep Jan haar,:}{arm met meer tederheid en}{zei zo heel innig}\\

\haiku{De kom van 't dorp,.}{was als een bobbelende}{ziedende ketel}\\

\subsection{Uit: In Itali\"e}

\haiku{Fientje, Bert noch de.}{President wisten goed hoe}{zich aan te stellen}\\

\haiku{Neen, het bestuur der...}{bedevaart zal niet in fout}{gevonden worden}\\

\haiku{Slingerend klimmen.}{de witte krijtwegen naar}{het land van Savoie}\\

\haiku{Ge ziet ze stappen,.}{met welgemeten tred zooals}{ze doen rond hun land}\\

\haiku{De President wil.}{weten voor hoeveel volk de}{kerk wel ruimte biedt}\\

\haiku{Op de Cavourbrug}{voorbij het gerechtshof zien}{wij den  schoonen}\\

\haiku{V\'o\'or een albergo,.}{waar een strijkje speelt komt een}{hoektafeltje leeg}\\

\haiku{Maar rond middernacht:}{viel hij uit zijn bed met groot}{lawaai al kreunend}\\

\haiku{zeg ik tegen den,.}{chauffeur die terzelfder tijd}{cicerone speelt}\\

\haiku{een modern klooster,,.}{een ander uit de jaren}{1052 de kerk van Ste}\\

\haiku{In het Vaticaan:}{is ondergebracht gansch het}{bestuur der H. Kerk}\\

\haiku{{\textquoteleft}Welkom, dierbare,...}{kinderen uit het diepste}{van ons vaderhart}\\

\haiku{De kerker is in:}{de rots gehouwen en heeft}{twee verdiepingen}\\

\haiku{Al de gazetten.}{geven het conterfeitsel}{van Mussolini}\\

\haiku{Vlaamsche boeren het.}{gekir en geklapwiek der}{St. Marcusvogels}\\

\haiku{De wolken drijven.}{en de zon doet moeite om}{een spleet te vinden}\\

\haiku{De Pilatusberg:}{bij Luzern is met een hoed}{van wolken bedekt}\\

\haiku{De musschen sjirpen.}{en de merels fluiten in}{den frisschen morgen}\\

\haiku{Het bezoek aan den.}{Gletschergarten laten we aan}{de liefhebbers over}\\

\haiku{na nog een beetje, - -!}{dommelen stoppen wij laat}{kijken te Aarlen}\\

\haiku{Namen - Hoeilaart, het {\textquoteleft}{\textquoteright} - -:}{glazen dorp Groenendaal en}{kort na zeven uur}\\

\subsection{Uit: In Spanje}

\haiku{Drie heeren lezen,.}{hun dagblad twee juffrouwen}{een romannetje}\\

\haiku{Heeroom draait zich op,,:}{zijn gemak in den hoek neemt}{nog een snuifke zegt}\\

\haiku{De dru{\"\i}den met... -...}{gouden sikkel Kijk eens daar}{in dien hollen weg}\\

\haiku{Ik hoor de dame.}{achter den toog keuvelen}{met haar dochtertje}\\

\haiku{In mijn hofje stond.}{een mooi pereboompje dat}{vijftien peren droeg}\\

\haiku{Mannen, vrouwen en, '.}{kinderent krioelt er}{bedrijvig dooreen}\\

\haiku{Als 't niet op tijd,.}{regent is een boer op \'e\'en}{jaar gansch ten onder}\\

\haiku{nu eerst beginnen,,.}{de stadslantarens voor zooveel}{er zijn te branden}\\

\haiku{Twee misdienaars, vlug,.}{als jonge katten houden}{er boksoefening}\\

\haiku{we akkoord treffen.}{voor negen pesetas met}{recht op drie uur tijd}\\

\haiku{slechts \'e\'en beuk, doch twee:}{afsluitingen verdeelen}{haar in drie deelen}\\

\haiku{Al de landbouwers.}{trachten zooveel mogelijk}{grond bij te huren}\\

\haiku{Wij rijden over een,.}{hoogvlakte uitheuvelend}{op al de kimmen}\\

\haiku{Misschien is het ras '?}{van dieven en moordenaars}{aant verkwijnen}\\

\haiku{Stilaan zijn we met.}{onze twee Spanjaards dikke}{vrienden geworden}\\

\haiku{hij was in zijn geest ',.}{aant spinnewebben dat}{was hem aan te zien}\\

\haiku{- Och ja, zei ze en,:}{ze kreeg een kleur dit vergat}{ik u te zeggen}\\

\haiku{De Padre doet ons,.}{uitgeleide tot aan neen}{tot in de statie}\\

\haiku{{\textquoteright} En de Zuidersche,;}{zon laait daarover en doet al}{de kleuren vlammen}\\

\haiku{'t Jongetje vangt.}{ze behendig en bijt er}{in met vollen mond}\\

\haiku{Na het avondmaal nog.}{even gaan slenteren in het}{drukke stadsgewoel}\\

\haiku{Een wriemeling als.}{van kleurige kabouters}{op Walpurgisnacht}\\

\haiku{Sierlijk slaan ze met:}{de panden van hun mantels}{van diverse kleur}\\

\haiku{De espadas buigen,.}{voor den President allen}{ontblooten het hoofd}\\

\haiku{Op zijn knie\"en kruipt... -! -!}{de matador er op af}{Dat is geen sport Awoert}\\

\haiku{Aan den omdraai de:}{kleine misdienaar gevolgd}{door een hulpkoster}\\

\haiku{Heeroom wordt in zijn.}{les onderbroken door het}{stoppen van den trein}\\

\haiku{Goudsmeed- en:}{brokaatwerk voor een waarde}{van millioenen}\\

\haiku{- Hier ook hebben de;}{boeren tijdens den oorlog}{veel geld gewonnen}\\

\haiku{Zij waakt over de stad,.}{zij strekt haar machtigen arm}{uit over gansch Aragon}\\

\haiku{hij stak een schoonen ' '.}{cent op zak en kwam bijt}{volk int gevlei}\\

\haiku{'s Avonds te zes uur.}{plechtige vergadering}{in den stadsschouwburg}\\

\haiku{'s Morgens in de,;}{kerk van den Pilar een pracht}{van een processie}\\

\haiku{Te half twee, in het,.}{hotel Lac een banket voor}{zestig genoodigden}\\

\haiku{Laat ik nog even, voor,:}{de laatste maal het menu}{hier overschrijven}\\

\haiku{Oom Jan zit in zijn:}{hoekje in zijn brieventesch}{te  snuisteren}\\

\subsection{Uit: De laatste flesch}

\haiku{{\textquoteright} Kapelaan Deckers.}{at zijn sigaar half op van}{opgewondenheid}\\

\haiku{Mijnheer pastoor is.}{ook zeer tevreden over zijn}{nieuwen kapelaan}\\

\haiku{{\textquoteright} De knie\"en van den.}{kapelaan knikten en een}{floers schoof voor zijn oogen}\\

\haiku{{\textquoteright} - {\textquoteleft}Wie zal er mij met,?}{het scheermes schoon maken als}{ik gestorven ben}\\

\haiku{{\textquoteright} - {\textquoteleft}Ik scheer U nu al,,.}{zoolang mijnheer pastoor dat}{moet dan ook maar gaan}\\

\haiku{Eulalie schoot hem: - {\textquoteleft}?}{voor met de vraagEn wat gaan}{we hem voederen}\\

\haiku{{\textquoteright} - {\textquoteleft}Maar ziet eens wat een,,.}{lange snuit wat een grove}{graat wat een hoogen rug}\\

\haiku{En hoe duchtiger,.}{Meneerke schrobde hoe}{luider Kees gilde}\\

\haiku{'s Dinsdags kwam de - - {\textquoteleft}{\textquoteright}.}{slachter en glorieloos lot}{kapte Kees infrut}\\

\haiku{De pastoor werd oud,.}{leed aan een maagkwaal en kon}{geen wijn meer rieken}\\

\haiku{loontje komt om zijn,.}{boontje hij heeft er vroeger}{genoeg geroken}\\

\haiku{Dat behoort tot de,,.}{gewoonte de zede de}{mores van de streek}\\

\haiku{geen enkele der.}{gewone       slapers kon}{zijn uiltje vangen}\\

\haiku{En een gezellig:}{koutje geslagen over wat}{de dag zooal meebracht}\\

\haiku{Ja, Tintelenteen...}{had zin in duiven en was}{een fijne kenner}\\

\haiku{Hij springt los over een:}{koppel veulens en vliegt op}{mij toe als een zot}\\

\haiku{De wind zwiepte dat.}{de boomtakken kraakten en}{de kruinen zoefden}\\

\haiku{Opeens zag ik de:}{voorsten gebaren maken}{met armen en beenen}\\

\haiku{En zij voegt de daad,.}{bij het woord smeedt de praktijk}{aan de theorie}\\

\haiku{{\textquoteright} - {\textquoteleft}Ch\'eri, ik dank je,...}{voor de attentie je bent}{nog steeds dezelfde}\\

\section{J.R.W. Sinninghe}

\subsection{Uit: Limburgsch sagenboek}

\haiku{Allen, die daarvan,.}{wisten zijn echter gesmoord}{in de moerassen}\\

\haiku{Zij dacht alleen nog;}{maar aan die kist met geld en}{aan niets anders meer}\\

\haiku{Maar de koning die,.}{links te paard stijgt zal te Mook}{vluchten over de brug}\\

\haiku{Ten laatste werden}{ze in den nacht overvallen}{en kwamen allen}\\

\haiku{Als hij ze eenmaal,.}{op Horn had nam hij de rest}{voor zijn rekening}\\

\haiku{de hooge muren en.}{onder de breed overwelfde}{bogen der keukens}\\

\haiku{{\textquoteleft}bij de Maas zal hij,.}{stil moeten houden hij is}{te Ool nog niet over}\\

\haiku{Ik kreeg van iemand,:}{zeshonderd gulden en ik}{dacht al bij mezelf}\\

\haiku{1) H. Welters in,-,,,.}{Limburgsche Legenden I.}{121122 127 132 229}\\

\haiku{Schippers, die op de,.}{Noordzee varen hebben het}{spookschip vaak ontmoet}\\

\haiku{{\textquoteleft}So leget mi in,,.}{die coetse blank Opdat ic}{ruste ic ben crank}\\

\haiku{{\textquoteleft}O grond, rijt op, 'k.}{wil in din schoot Bi Reynout}{wesen in der doot}\\

\haiku{Uit dank schonk Keizer.}{Karel het verblijf van zijn}{dochter aan den Huyn}\\

\haiku{Die Jan was verliefd,,}{op Hanna de dochter van}{zijn baas maar zij wou}\\

\haiku{Een generaal, hoog,.}{te paard gezeten met zijn}{ruiters achter hem}\\

\haiku{Een boer uit Wijk zei,.}{tot Alfred Harou dat steenen}{groeien als planten}\\

\haiku{De knecht hield echter,.}{zijn onschuld vol en zeide}{dat het niet waar was}\\

\haiku{Van Einighausen,,:}{een gehucht ten oosten van}{Limbricht verluidt het}\\

\haiku{Sindsdien dragen de {\textquoteleft}{\textquoteright}.}{Venlonaren den bijnaam}{Wannenvliegers}\\

\haiku{het klokkengebrom.}{Doet ijslijk de loopende}{landlieden zuchten}\\

\haiku{Weer probeerde hij,.}{van wal te steken en weer}{lukte het hem niet}\\

\haiku{Zij wist hem zelfs te,.}{bepraten het beeldje naar}{een kerk te brengen}\\

\haiku{Ineens brak ergens,,.}{de stilte en het dier zag}{op even verwonderd}\\

\haiku{Thans kon het niet lang,....}{meer duren of het schaap zou}{worden verslonden}\\

\haiku{Het trok, met al zijn,,.}{macht en zoowaar het rukte het}{paaltje uit den grond}\\

\haiku{Die plek had precies.}{den vorm van den grondslag van}{een kapelletje}\\

\haiku{Gedurende vier.}{en zeventig jaar had hij}{zijn bisdom bestuurd}\\

\haiku{De arbeiders, die.}{hij daar bezig vond in den}{wijnberg verdreef hij}\\

\haiku{Deze vluchtten nu.}{naar Maastricht en verhaalden}{daar het gebeurde}\\

\haiku{{\textquoteright} 1) ~ * * * ~ Voor men.}{op reis ging dronk men steeds de}{St. Geertenminne}\\

\haiku{{\textquoteright} In hun angst holden.}{zij tusschen de menigte}{door naar den prior}\\

\haiku{Daar vertelden zij.}{het gebeurde en toonden}{den bebloeden draad}\\

\haiku{Geen mensch had echter,.}{eenig letsel bekomen dank}{zij Sint Gerlach}\\

\haiku{Het bleek een monnik,.}{te wezen die hier blijkbaar}{wilde overnachten}\\

\haiku{{\textquoteleft}Ik was juist van plan,?}{mijn ziel te verdobbelen}{wat dunkt u ervan}\\

\haiku{Daarop ging hij weer,.}{naar zijn vrienden terug en}{speelde en dronk voort}\\

\haiku{{\textquoteright} Na deze woorden. '}{reed de duivel hem weer naar}{zijn molen terug}\\

\haiku{s Morgens vond de.}{vrouw haar man halfdood voor den}{molen  liggen}\\

\haiku{{\textquoteleft}Ga naar den molen,!}{en maal den zak koren die}{is aangekomen}\\

\haiku{{\textquoteleft}Het zou u straks te,!}{lastig zijn om de paarden}{te doen stilhouden}\\

\haiku{Zij aten er niet van,,.}{want niemand wist wat het was}{of van waar het kwam}\\

\haiku{Geen bergkant was hem,,.}{te steil geen water te breed}{geen moeras te diep}\\

\haiku{De reus - vertelt men -.}{in Maastricht woonde op een}{berg met zijn dochter}\\

\haiku{{\textquoteright} {\textquoteleft}'t Is geen kraai, 't,{\textquoteright}, {\textquoteleft}!}{is een spreeuw antwoordde de}{koningwerk maar door}\\

\haiku{De man met den haak.}{wordt door alle Limburgsche}{kinderen gevreesd}\\

\haiku{De grond was warm door.}{het sleepen van den vurigen}{ketting van Satan}\\

\haiku{Zij, die ziek werden,;}{en de koorts kregen stuurde}{hij zonder hulp weg}\\

\haiku{Dorren, Het Kasteel,-,.}{van Valkenburg blz. 4748}{en Pierre Kemp id}\\

\haiku{{\textquoteleft}Als ge hem dezen,.}{nacht nog eens ziet vraag hem dan}{wat hij hebben moet}\\

\haiku{Maar noch haar broers, noch.}{haar zusters bekommerden}{zich om haar bede}\\

\haiku{- maar niet voor lang, want.}{zij keerden spoedig weer en}{vielen opnieuw aan}\\

\haiku{Te Venlo kon men:}{de blinde ronde op de}{wallen ontmoeten}\\

\haiku{{\textquoteleft}Ik ging dan weerom{\textquoteright} - - {\textquoteleft};}{zoo vertelt hijden bosch in}{als niemand mij zag}\\

\haiku{Ge kunt denken, dat '.}{iks Zondags nooit meer naar}{het bosch ben geweest}\\

\haiku{{\textquoteright} Eindelijk liet de.}{veerman zich toch bepraten}{en zette hem over}\\

\haiku{Het heette, dat het.}{vurig ros met hem in den}{grond verzonken was}\\

\haiku{Al in de verte.}{hoorden ze de kat over de}{daken naderen}\\

\haiku{De vriendin maakte,.}{eerst bezwaren maar zij liet}{zich toch overhalen}\\

\haiku{De jonge heks zou,.}{al gauw merken hoe leelijk}{zij zich had vergist}\\

\haiku{H\`e haw \`evel gaar gein,}{macht meer euver de heksen}{en vloag allein weer}\\

\haiku{Wij komen ons hier,{\textquoteright}, {\textquoteleft}!}{amuseeren was het antwoord}{wat gaat jou dat aan}\\

\haiku{Doch eer de dag om,:}{was kwam het bericht dat hij}{verongelukt was}\\

\haiku{daar tolde de haas.}{met het volle schot hagel}{in zijn kop omver}\\

\haiku{Te Gronsveld waren,.}{het krekels die door een heks}{gezonden waren}\\

\haiku{{\textquoteright} De pastoor gaf den.}{moed niet verloren en de}{terugreis begon}\\

\haiku{Maar kaartspelen op,!}{Zondag onder de Hoogmis}{dat is wat anders}\\

\haiku{2) A.F. van Beurden,,,.}{in Limburgs Jaarboek XX blz.}{183 en XXV blz. 88}\\

\haiku{Dadelijk begreep,.}{hij dat hij met een weerwolf}{te doen had gehad}\\

\haiku{De man, niet mis, nam '.}{zijn mes en stak stillekens}{t beest in den buik}\\

\haiku{Opeens, met klokke,.}{twaalf viel het wolfsvel uit den}{schoorsteen op het vuur}\\

\haiku{Onderwijl sloop de.}{knecht stilletjes naderbij}{en nam de zeef weg}\\

\haiku{{\textquoteright} Nu, dat had eens een:}{mensch afgeluisterd en die}{dacht in zijn eigen}\\

\haiku{{\textquoteleft}Ge moet niet grijnen,.}{ik zal u de kinderen}{wel terug geven}\\

\haiku{Wat was de koning,:}{blij maar ook wat was hij kwaad}{op de drie zusters}\\

\haiku{'t Waren Onze,.}{Lieve Heer en Sint Pieter}{maar Jan wist dat niet}\\

\haiku{In den kelder wou ',:}{het spook Jan aant graven}{zetten maar Jan zei}\\

\haiku{Maar ieder maal dat,.}{Jan terugkwam blies het spook}{opnieuw zijn lamp uit}\\

\haiku{(blz. 283) Sinninghe,,,.}{Overijsselsch S. blz. 59 vlg.}{Idem Noord-Brab}\\

\haiku{Bijdragen tot de ();}{geschiedenis der hoofdbank}{ClimmenMaastricht 1906}\\

\haiku{34Dit kunststukje '.}{staat ook op naam van een knecht}{vant kasteel Horn}\\

\subsection{Uit: Verhalen uit het land der bokkenrijders en der Teuten}

\haiku{Op een morgen was!}{de ruitersman weg en de}{prinses was ook weg}\\

\haiku{Anderen zetten.}{wel eten voor ze klaar en daar}{werkten ze dan voor}\\

\haiku{Het graven van dat:}{kanaal werd opgedragen}{aan twee aannemers}\\

\haiku{Opeens zag hij in.}{de verte een vuurgloed die}{recht op hem toekwam}\\

\haiku{Wie hijn ein paar trij,:}{neive waas huurde hijn den}{honk inens zegge}\\

\haiku{{\textquoteleft}dao moot iech ouch get{\textquoteright},:}{van hobbe mer wie de vrouw}{hiel kordaot zag}\\

\haiku{Omstreeks 1772 kwam een,,.}{dronkaard ijselijk vloekend}{zijn kamer binnen}\\

\haiku{{\textquoteleft}soodaenige dingen ()}{als dezelve oculomet}{eigen ogen gesien}\\

\haiku{Het getuig hing aan.}{de disselboom en het paard}{stond er doornat naast}\\

\haiku{In mijn tijd nog had, '.}{je in Maasniel vrouw Giesbers dat}{wasn ouwe vrouw}\\

\haiku{Iedereen wist met.}{zijn klompen aan dat dit het}{werk van die heks was}\\

\haiku{Twie\"e j\'ongmen zogan.}{eum saoves es drie\"e op}{ein soets aaf loupe}\\

\haiku{Sindsdien heeft men nooit}{meer iets vernomen van het}{duivelsboekje.136}\\

\haiku{M'n moeder had maar.}{een broer en die ging altijd}{met die paarden om}\\

\haiku{Hij was acht meter.}{hoog en had een middellijn}{van twintig meter}\\

\haiku{Dou verennigden sich,:}{de wisen van den N\"ustad}{en do sagt einen}\\

\haiku{De rovers gingen}{hem dadelijk achterna}{en ze doorzochten}\\

\haiku{Hollandse Pier zat.}{met een knie op de wilg want}{meer plaats was er niet}\\

\haiku{Gij moet leggen ten}{kronen voor het houter en}{ten voor het broken}\\

\haiku{Maar al de knechts, de ().}{gendarmen en de boyszoons}{zaten bij het geld}\\

\haiku{Op 'n keer was hij.}{weer daar en ze gongen naar}{de bijen kijken}\\

\haiku{Dat was 'ne grote.}{boer en daar hadden ze pas}{het verken geslacht}\\

\haiku{Met dat worstelen.}{had ze gezien dat het die}{van de Schans waren}\\

\haiku{{\textquoteleft}Ja maar{\textquoteright}, zei de baas, {\textquoteleft},...{\textquoteright} {\textquoteleft}{\textquoteright},.}{zeg er maar niks van wantNiks}{daarvan zei de maagd}\\

\haiku{Het geld lig in de,.}{kelder onger den ungerste}{onger eine stein}\\

\haiku{Dat is een goeie{\textquoteright}, zei, {\textquoteleft}.}{de kosterzo lust ik er}{wel elke dag een}\\

\section{J. Slauerhoff}

\subsection{Uit: Het lente-eiland en andere verhalen}

\haiku{Maar de karakters.}{zullen verwrongen zijn door}{den groei van het hout}\\

\haiku{{\textquoteleft}Zie mijn voeten, ze,.}{zijn zo klein ze nemen geen}{plaats op het dek in}\\

\haiku{En mijn lichaam, zo,.}{slank als een lelieblad ik}{kan het opvouwen}\\

\haiku{Als de wind door de,.}{pijnen toog hoorde hij het}{koor der zusteren}\\

\haiku{Hij dronk geen wijn meer.}{en kocht den edelsten inkt en}{het beste papier}\\

\haiku{voortdurend gingen.}{bliksemschichten heen en weer}{tussen hen en hem}\\

\haiku{Een mandarijn is.}{daar en verlangt dit vertrek}{om te overnachten}\\

\haiku{{\textquoteright} {\textquoteleft}Ik ben van mijn post.}{ontheven en naar het hof}{teruggeroepen}\\

\haiku{{\textquoteleft}In welk land kent men?}{een moorddadige kracht toe}{aan de po\"ezie}\\

\haiku{In het Westen is.}{er wel een land waar men haar}{jeugdbederver noemt}\\

\haiku{Ik ben geen profeet,,,.}{meer geen po\"eet meer ja geen}{geletterde meer}\\

\haiku{Ik daarentegen.}{ga naar Tsjung Tsjow om een}{vrouw te ontvluchten}\\

\haiku{Zij kan borduren,.}{en luitspelen bloemen en}{vogels schilderen}\\

\haiku{Zij eiste van mij.}{volledige overgave}{van lichaam en ziel}\\

\haiku{even zwaar is het, een.}{gestorven kind weer in het}{leven te roepen}\\

\haiku{Neen toch, er was iets,;}{in waar al het latere}{werk niet bij haalde}\\

\haiku{Maar nooit waagden zij,.}{het rechtstreeks iets tegen hem}{te ondernemen}\\

\haiku{Opeens liet zij haar,.}{houding varen strekte de}{armen naar hem uit}\\

\subsection{Uit: Schuim en asch}

\haiku{Sinds jaren hangt zijn,:}{linkerhand neer machteloos}{ter aarde wijzend}\\

\haiku{{\textquoteright} Hassein ging naar huis,.}{vervolgd door den schimp van den}{woedenden moefti}\\

\haiku{De eerste droeg een.}{groenen tulband met  de}{gouden halve maan}\\

\haiku{Met de opbrengst had.}{hij geen drie dagen gage}{kunnen betalen}\\

\haiku{Kasem Hassein deed thans:}{datgene waarmee hij had}{moeten beginnen}\\

\haiku{Mijn heupwond schrijnde,.}{mij ook meende ik jicht te}{voelen opkruipen}\\

\haiku{Ik belde vergeefs.}{en eindelijk sleepte ik}{mij tot de tafel}\\

\haiku{Ik was volkomen,.}{gedachteloos leed niet en}{zwijg dus over dien tijd}\\

\haiku{Vol wantrouwen bood.}{ik hem een glas wijn en vroeg}{wat hij verlangde}\\

\haiku{Ik maakte mij boos,,.}{maar de monnik lachte en}{boog en trok mij mee}\\

\haiku{{\textquoteleft}Ik wil hier blijven,,.}{een boot hebben zeilen en}{visschen in de baai}\\

\haiku{Het scheepje slingert,.}{hoewel de zee niet woelig}{is en de wind luw}\\

\haiku{Direct van Itali\"e,.}{hierheen gekomen zou ik}{verheugd zijn geweest}\\

\haiku{{\textquoteleft}Om te luisteren.}{ben ik van het einde der}{wereld gekomen}\\

\haiku{{\textquoteright} {\textquoteleft}Het schip ligt sturdy,?}{over hoe kan een losse kist}{dan zoo te keer gaan}\\

\haiku{Bevelen vielen,.}{niet te geven het schip dreef}{op Gods genade}\\

\haiku{Deze beval hem.}{kort en ruw dien rommel uit}{de poort te smijten}\\

\haiku{Als we onthoofde,?}{lijken aanbrengen wat staat}{ons dan te wachten}\\

\haiku{Tast het hout niet aan,,.}{maakt den vloer niet glad de booze}{geesten niet wakker}\\

\haiku{Nu zou de Nyborg,,.}{als het aan hem lag zeilschip}{blijven tot zijn eind}\\

\haiku{Soden wilde wel,}{maar het vooruitzicht leek hem}{onbereikbaar ver.}\\

\haiku{{\textquoteright} {\textquoteleft}Je begrijpt toch wel.}{dat ik met deze mist niet}{door de riffen ga}\\

\haiku{Maar hij lag op den,,,}{vloer een matroos schudde hem}{trok hem naar buiten}\\

\haiku{Geld had ik niet, het.}{biljet naar Bordeaux was nog}{mijn eenigst aardsch bezit}\\

\haiku{Het was mij alsof.}{ik mij in dit land nog met}{je samen voelde}\\

\haiku{Had een monnik hier?}{bij jou met zijn geloof zijn}{pij laten liggen}\\

\haiku{Het eten werd haastig.}{en kleumend in de kille}{messroom gegeten}\\

\haiku{Misschien was je wel,.}{je naam vergeten dacht je}{dat ik dronken was}\\

\haiku{Hij ziet me, staat op,,,.}{drukt mij op een stoel schuift een}{glas voor mij ik drink}\\

\haiku{{\textquoteright} Ibsen bedoelde.}{de quarantainevlag die}{aan den voormast hing}\\

\haiku{{\textquoteleft}Laat de deur open, ik!}{moet er straks zijn en wil niet}{in de stank zitten}\\

\haiku{Met kleurenblindheid.}{kon hij zich als dokter niet}{verontschuldigen}\\

\haiku{{\textquoteright} Werkelijk, Bruce;}{kon bijna niet uit de boot}{en op weg komen}\\

\haiku{Talman laadde hem.}{in den eersten draagstoel dien}{zij tegenkwamen}\\

\haiku{Als je zoo de pest,?}{aan schepen hebt waarom zit}{je er dan nog op}\\

\haiku{{\textquoteright} {\textquoteleft}Zoo zie je, dat de,{\textquoteright}.}{wal niets dan ellende geeft}{triomfeerde Young}\\

\haiku{Hij was congestieus, '.}{eenige malens jaars liet}{hij zich aderlaten}\\

\haiku{Neem een bad, droog je,,.}{af schiet een kimono aan}{vijf minuten werk}\\

\haiku{Waarom mocht zelfs de?}{nacht niet in vredesnaam stil}{en leeg en zwart zijn}\\

\haiku{Zij schoven vlug hun,,.}{bank in sloegen de bladzij}{op wezen elkaar}\\

\subsection{Uit: Het verboden rijk}

\haiku{Het hoofd verlangde.}{toegang tot den goeverneur}{Antonio Farria}\\

\haiku{Met zijn helm schepte.}{hij wat water uit den poel}{en koelde zijn hoofd}\\

\haiku{In Malakka zou.}{men ons spottend ontvangen}{en triomfeeren}\\

\haiku{Hij bouwde een paar -.}{forten en loodsen kerken}{kwamen er vanzelf}\\

\haiku{Macao lag halfweg -.}{Malakka Japan aan een}{beschutte reede}\\

\haiku{Telkens kwamen nog.}{een paar menschen of een paar}{vaten de plank over}\\

\haiku{Er werd weinig meer,,.}{gesproken de vader leed}{maar klaagde niet meer}\\

\haiku{De krekels gingen.}{te keer alsof zij levend}{geroosterd werden}\\

\haiku{{\textquoteright} De dominikaan.}{verdween en liet hem hijgend}{en vloekend achter}\\

\haiku{De vader ging op.}{in het leed van den tot zoon}{gewenschten minnaar}\\

\haiku{Hij zat alleen voor,.}{zijn middagdisch daarna zond}{hij om zijn dochter}\\

\haiku{Moedeloos liet zij.}{zich neerglijden tegen de}{harde vensterbank}\\

\haiku{hij zonk achterover,.}{en de wijn vloeide over zijn}{laarzen op den vloer}\\

\haiku{III Behoedzaam sloop.}{Campos de trappen op en}{stond stil voor de deur}\\

\haiku{Maar de sporen van:}{den zwaren last waren niet}{te verwijderen}\\

\haiku{{\textquoteleft}Mislukt, de vogel,.}{gevlogen ik bijna in}{de kooi gevangen}\\

\haiku{Hij leidde hem eerst,:}{om het lijk liet hem toen los}{en zeide kortaf}\\

\haiku{Vroeger brachten de.}{geminachte ontdekkers}{hulde aan het hof}\\

\haiku{De toren Belem.}{zag hij aan voor zijn vader}{die daar maar staan bleef}\\

\haiku{II Den anderen,.}{morgen vroeg was hij weer aan}{dek de zee was leeg}\\

\haiku{het schip zonder hem.}{te laten vertrekken en}{aan wal te springen}\\

\haiku{Een honderd meter,,}{misschien drong ik erin door}{verder kon ik niet}\\

\haiku{Het was mijn eigen.}{gedaante gezien in een}{verweerde spiegel}\\

\haiku{Het duurde lang, maar.}{Pilar kende ook een kruid}{dat den slaap verdreef}\\

\haiku{Toch draalde zij nog,.}{opeens was er een groote rust}{over haar gekomen}\\

\haiku{Maar voor het klooster.}{keerde zij om en liep de}{Chineesche wijk in}\\

\haiku{Die hoorde Pilar,:}{ook maar na een paar dagen}{wist zij wat het was}\\

\haiku{zij verlangde er,.}{geen liefde voor zij wilde}{onaangeroerd zijn}\\

\haiku{Beiden bezaten.}{niet de gevoelens die haar}{konden ontroeren}\\

\haiku{de beide vleugels,.}{brandden het middenstuk was}{nog onaangetast}\\

\haiku{De regeering had de;}{toegangen tot het klooster}{laten bezetten}\\

\haiku{De vrouw - wonderlijk! -;}{voelde niet dat ik leefde}{in het schimmenrijk}\\

\haiku{{\textquoteright} Heel lang leunde ik.}{tegen een stam en kwam laat}{in den nacht naar huis}\\

\haiku{een violist dien.}{ik in mijn goeden tijd in}{Brighton had gehoord}\\

\haiku{Hij ging beschaamd en.}{levensmoe met het gezicht}{naar den muur liggen}\\

\haiku{Eindelijk drong het,.}{door toen vroeg hij waar hij dat}{aan te danken had}\\

\haiku{{\textquoteleft}En gij, beken ook,.}{dan hebben we alles wat}{we weten moeten}\\

\haiku{Wilde men hem door?}{goede behandeling tot}{verraad bewegen}\\

\haiku{Maar een blad van de.}{Hesperiden-tuin was}{gekreukt en bevlekt}\\

\haiku{Metelho stak het hoofd,.}{buiten de draagstoel kwam toen}{heelemaal overeind}\\

\haiku{Gisteravond bij het,.}{donker wierp ik deze steen}{aan deze doek uit}\\

\haiku{Ik stond stil voor het,.}{steenen tuinhuis te rusten er}{was wat koelte daar}\\

\haiku{Ik wist het niet en,....}{wilde het niet weten want}{als ik dat ook wist}\\

\haiku{En toen kwam het ook.}{als ik op wacht zat met de}{seinkap om mijn hoofd}\\

\haiku{Het loopen was ik,:}{ontwend ik was geworden}{als de anderen}\\

\haiku{Zij waken over de,.}{schepen zooals anderen over}{het heil der zielen}\\

\haiku{Nu stieten wij op,.}{elkaar als de wagons van}{een geremde trein}\\

\haiku{Het was zoo hoog en.}{de donkergladde steenen van}{den ingang noodden}\\

\haiku{{\textquoteleft}Als je meer van die,.}{stukken hebt zal ik ze wel}{voor je wisselen}\\

\haiku{Hij bleef stilstaan, het,,.}{musket dat hij als knots wou}{gebruiken in rust}\\

\haiku{Zij zonk haast op de.}{knie\"en toen zij zag wat uit}{hem geworden was}\\

\haiku{hij daarbuiten was,.}{en in het donker zat hield}{hij op te bestaan}\\

\haiku{waarom ik hier toch.}{was en wat dat allemaal}{te beteekenen had}\\

\haiku{Ik dacht over mijn angst,,:}{van vroeger ik verbaasde}{mij ik vroeg mij af}\\

\section{Conny Sluysmans}

\subsection{Uit: Nog kans op de hemel}

\haiku{Hij stelde me voor.}{eerst ergens te gaan eten en}{ik knikte gretig}\\

\haiku{Het was waar, ik had.}{Hem nooit opzettelijk aan}{het kruis geslagen}\\

\haiku{Als je morgen om.}{zes uur langs de garage}{komt kun je hem zien}\\

\haiku{Voor mijn gevoel staan.}{alle koosnaampjes daarbij}{ver in de schaduw}\\

\haiku{Het was een van de.}{episodes die mijn leven}{hadden getekend}\\

\haiku{dank zij mijn eigen!}{veroveringstactiek ten}{opzichte van hem}\\

\haiku{Dat was dus dat. - Kom?}{je vanmiddag een beetje}{bij me op bezoek}\\

\haiku{Het hele orkest.}{van mijn gevoelens zette}{gelijktijdig in}\\

\haiku{dat zij het bestaan.}{van die andere Sonja}{niet meer nodig vond}\\

\haiku{Die nacht wilde ik,,.}{meer nog dan de vorige}{maal de zijne zijn}\\

\haiku{Waarom vindt u het?}{zo moeilijk die promotie}{te accepteren}\\

\haiku{Mijn auto is snel.}{genoeg om je heel vaak te}{komen bezoeken}\\

\haiku{Je kunt toch niet aan,.}{het leven ontkomen had}{mijn moeder gezegd}\\

\haiku{Niet als aan iemand.}{met wie ik in de toekomst}{te doen zou hebben}\\

\haiku{Ik heb een schat van.}{een vader en een zusje}{waar ik veel van hou}\\

\haiku{Wij stonden daar maar.}{tegenover elkaar en wij}{zeiden heel lang niets}\\

\haiku{- je weet best dat het,}{niet om dat lopen gaat maar}{ik kom niet No\"el}\\

\haiku{- je weet best dat het,}{niet om dat lopen gaat maar}{ik kom niet No\"el}\\

\haiku{Het was een tasten.}{naar de kwalificatie}{van onze liefde}\\

\haiku{Ik was geboren....}{onder het sterrebeeld van}{de Leeuw Nee No\"el}\\

\haiku{Jij bent een vrouw voor.}{\'e\'en man en naar die man zul}{je blijven zoeken}\\

\haiku{Goed, ik had vrouwen.}{gekend die plooibaar waren}{tot het uiterste}\\

\haiku{Tot zes uur wist ik,.}{mij te beheersen toen was}{het proces voorbij}\\

\haiku{- Dus No\"el is toch,! -?}{teruggekomen zei ze}{Teruggekomen}\\

\haiku{N\'og was er in mij.}{het absurde geloof in}{zijn medeleven}\\

\haiku{Hij was niet weg, ik.}{had immers geweten dat}{het een leugen was}\\

\haiku{Ik sloot zorgvuldig.}{het portier van mijn auto}{en ging naar hem toe}\\

\haiku{- Natuurlijk, zei hij,.}{en die wereld kun je ook}{niet veranderen}\\

\section{Alie Smeding}

\subsection{Uit: Achter het anker}

\haiku{Ze schortte de rok.}{wat op en plaatste de voeten}{op het zettelbord}\\

\haiku{{\textquoteright} Kruiselings spande,.}{ze de armen over de borst}{lachte uitdagend}\\

\haiku{{\textquoteright} Een gemelijke,.}{glimlach trok om zijn mond en}{zijn stem klonk stroever}\\

\haiku{Steven Roos en zijn,.}{vrouw zaten breed en donker}{vlak voor de tafel}\\

\haiku{Bouk zou ook 'n best, -,,}{wijfje worden zacht en zoo}{zacht en j\'a schuwer}\\

\haiku{{\textquoteright} Beverig kromp hij,...}{in duwde de handen voor}{de oogen en kreunde}\\

\haiku{Een lach wipte door.}{zijn oogen en hij streek de tong}{langs de onderlip}\\

\haiku{{\textquoteleft}Nee Bart, 'k wil je,{\textquoteright}, - {\textquoteleft}!}{z\'o\'o niet mee hebben driftte}{hijdat verl\`ap ik}\\

\haiku{Oh god - zoo'n m\'o\'oie meid,{\textquoteright},.}{het kreunde door zijn kop hij}{neep de  handen}\\

\haiku{er van wezen, 't, '.}{was menschelijk zooals Ate de}{Leeuw ook wels zei}\\

\haiku{Ve'-dikkie, je.}{adem kon je vlak onder de}{neus zien bevriezen}\\

\haiku{{\textquoteright} Rillig en vervreemd.}{stond hij even later in de}{schemerige gang}\\

\haiku{Tjeerd begon verward.}{en haastig zijn relaas over}{hun wedervaren}\\

\haiku{{\textquoteleft}Allooh jong',{\textquoteright} lachte, - {\textquoteleft} ',!}{ze naar hemnou eerst maars}{flink toetasten hoor}\\

\haiku{,{\textquoteright} joolde ze, met be{\^\i}.}{haar handen woelde ze in}{zijn ruige kuifhaar}\\

\haiku{Bleek en plechtig stond.}{de witte maannacht over het}{wijde watervlak}\\

\haiku{{\textquoteright} Suf bleef hij een poos,.}{voor zich uitstaren dan kwamen}{zijn gedachten weer}\\

\haiku{{\textquoteright} Smartelijk keek hij.}{voor zich neer op de vredig}{ruischende zee}\\

\haiku{Pafferig stond haar.}{goorwitte kop onder de}{scheefgezakte hoed}\\

\haiku{Och God, nou - die ik,?}{bedoel die h\^et ze achter}{de elleboog h\`e}\\

\haiku{Heerke, dat hoofd van, -!}{haar dat deed nou toch altijd}{zoo'n  pijn z\'o\'o'n p{\'\i}jn}\\

\haiku{zweeg hij en schamig.}{glipte zijn blik weg uit haar}{oogen toen ze opzag}\\

\haiku{{\textquoteright} Rap praatte hij het,,.}{beweeglijk zijn magere}{wangen donker-rood}\\

\haiku{Jij ben nog 's met ', ',?}{Eefke \^an de rinkelweest}{opn kermis h\`e}\\

\haiku{'n Slimme duvel,,.}{die meid nou maar die neem je}{niet gauw bij de neus}\\

\haiku{{\textquoteleft}Veeg 't af,{\textquoteright} zei hij, - {\textquoteleft}, '.}{je moet niet zoo bijtent}{wordt heelegaar blauw}\\

\haiku{wel, maar daar moet je ',,?}{t nou niet meer over hebben}{Vader begrijp je}\\

\haiku{{\textquoteleft}Wacht 's, nou waren,,...}{ze h\'a\'ast onder Marken h\`e}{daar ginter dat vuur}\\

\haiku{Moet er ook nog 'n ' '?}{eeuwigheid wezen enn}{hel enn duivel}\\

\haiku{zich, een wasemdrop}{spatte hard-tikkend}{van het zoldertje}\\

\haiku{{\textquoteright} vertelde ze, - {\textquoteleft}Van,,,:}{Maarle de Koster weet je}{wel die heb gezegd}\\

\haiku{{\textquoteleft}Ja, ik kom toch 's '!}{effe bij je kijken of}{jet knapjes heb}\\

\haiku{Nou was-t-ie er,...}{af en nou moest ie maar zoo}{gauw mogelijk weg}\\

\haiku{{\textquoteleft}Werken maar, altijd,{\textquoteright},...}{werken hunkerde het in}{hem zong het in hem}\\

\subsection{Uit: Als een bes in een hofje}

\haiku{haar oogen gloeiden nog,.}{van het mijmeren haar oogen}{tintelden diep-in}\\

\haiku{{\textquoteright} In haar hart bonsde,.}{de nijd groot en wild stond het}{verzet in haar op}\\

\haiku{{\textquoteright} De vreugde zette,.}{een beklemming in haar borst}{zij ademde hortend}\\

\haiku{de volle liefde,.}{wezen tenminste als je}{man Nol Franken was}\\

\haiku{au fond heb ik wel {\textquotedblleft}{\textquotedblright}.}{zoo ongeveer wat uidee}{gelieft te noemen}\\

\haiku{{\textquoteleft}Cor,{\textquoteright} met een zachte,.}{bekommering in de stem}{noemde hij haar naam}\\

\haiku{Het was of alle.}{geluiden een oogenblik}{plat bleven liggen}\\

\haiku{{\textquoteright} Maar toen zij zich op,.}{woorden bezon vluchtten haar}{heete gedachten}\\

\haiku{NOG EEN SMALLE STREEP.}{donker zon-goud door de}{zoele zomerlucht}\\

\haiku{{\textquoteleft}Zij was geen flirt, dus,,}{w\`at ze toonde dat was wel}{waar dat meende ze}\\

\haiku{{\textquoteleft}Die aangebeden{\textquoteright},, {\textquoteleft}, '!}{pipa ironiseerde hij}{jat is prachtig}\\

\haiku{Moeder woont nu bij,,.}{ze in d\`at bood hij zelfs aan}{hij is dol op Cor}\\

\haiku{Hij stak glimlachend,.}{een nieuwe sigaret aan}{wachtte op antwoord}\\

\subsection{Uit: Bruggenbouwers}

\haiku{En ver weg gonzen.}{en snorren de drukpersen}{en zetmachines}\\

\haiku{Vaak genoeg is hij,...}{dagen lang weg als hij weer}{zoo'n prutsbaantje heeft}\\

\haiku{Maar weet je wat ik?, {\textquotedblleft}}{jammer vind er is niet veel}{ingekomen op}\\

\haiku{Ik heb nog van de.}{antwoorden gemaakt wat er}{van te maken was}\\

\haiku{En als ze geen van -...}{allen meer te zien zijn dan}{ziet Kaatin ze nog}\\

\haiku{Met kleine oogen tuurt.}{hij in de rook en mompelt}{wat voor zich uit}\\

\haiku{Anne-Cris en.}{Cobie Savrij zitten naast}{elkaar aan tafel}\\

\haiku{{\textquoteleft}Zie je Solwerda, {\textquotedblleft}}{dat vinden de lui mooi dat}{je er zelf bent voor}\\

\haiku{{\textquoteright} Taco kan alleen...}{maar glimlachend over Onno}{Krabbeel heenkijken}\\

\haiku{{\textquoteright} Terloops kijkt hij naar.}{de mannen en vrouwen die}{hem passeeren}\\

\haiku{{\textquoteleft}Voil\`a{\textquoteright}, denkt hij, {\textquoteleft}maar zoo,...?}{hebben Crijna en ik nooit}{gekeken en n\'u}\\

\haiku{En zijn Vader zat.}{in de werkplaats achter de}{groote klokkenwinkel}\\

\haiku{Hij gluurde door een.}{klein dik glaasje dat hij vlak}{voor zijn eene oog hield}\\

\haiku{Solwerda heeft dat.}{nog zoo uitgeplozen in}{zijn hoofdartikel}\\

\haiku{Ze willen alleen.}{maar hun eigen gedachten}{in de krant hebben}\\

\haiku{Een krot!, jouw mensch, die,:}{nieuwe mensch  van jou hij}{woont in een villa}\\

\haiku{Hier is het huisje,,.}{van Lizelotte Buun een}{klepdeur een zwart raam}\\

\haiku{{\textquoteright} Er hangt nog rijp aan.}{de boomen en de takken}{lijken van zilver}\\

\haiku{En het portlandhuis.}{op het Staalborchplein heeft een}{bedeesd voorkomen}\\

\haiku{De vorige keer,,.}{waarachtig waar Solwerda}{toen was het te rood}\\

\haiku{dat zijn kaken een.}{rammelende beweging}{maken als hij praat}\\

\haiku{Vanmiddag afscheid,...?}{van mijnheer Bergsma op de}{H.B.S. zal ik d\`at maar}\\

\haiku{{\textquoteleft}Dus als u dat goed,?,,?}{vindt meneer komt het u dan}{gelegen meneer}\\

\haiku{In lang heb ik z\'oo{\textquoteright},, {\textquoteleft}.}{niet de Avondster gezien denkt}{Tacoin lang niet}\\

\haiku{Het is Taco of.}{hij dronken geweest is en}{nu weer nuchter wordt}\\

\haiku{Vroeger tuurde ze,.}{hem zoo lang mogelijk na}{zoo lang mogelijk}\\

\haiku{overal duisternis,,.}{sparreschimmen nachtkoelte}{en vaag geruisch}\\

\haiku{Hij is maar gauw in,...}{zijn jas geschoten die jas}{hangt wijd om hem heen}\\

\haiku{{\textquoteleft}Ja{\textquoteright}, denkt hij zonder, {\textquoteleft} '?}{overgangen ik zou immers}{s naar Moeder toe}\\

\haiku{Ze heeft toch nog kleur,.}{op haar wangen en ze heeft}{uitgeruste oogen}\\

\haiku{{\textquoteleft}Ik heb nog aan de.}{deur geluisterd met me oor}{op het sleutelgat}\\

\haiku{Ze kijkt naar hem op,,}{een eenvoudig goedmoedig}{eerlijk gezichtje}\\

\haiku{{\textquoteleft}Toen je indertijd{\textquoteright},,}{bij mij kwam Crijna wendt haar}{oogen niet van hem af}\\

\haiku{Hij leunt nog even met.}{zijn hoofd tegen de deurpost}{aan en sluit zijn oogen}\\

\haiku{{\textquoteleft}Is er copij voor?}{de fuljeton om in het}{voren te zetten}\\

\haiku{{\textquoteright} Bos wil toch lachen,,.}{wil hartelijk lachen het}{klinkt zoo armoedig}\\

\haiku{Soo net ging u vriend,?,.}{ook voorbij sag u hem niet}{meneer Altenstadt}\\

\haiku{Je neemt wel wat te.}{veel en te eenzijdig over}{uit die N.S.B.-lectuur}\\

\haiku{{\textquoteright}, vraagt hij op de man, {\textquoteleft}?,...?}{afwat hebben die menschen}{wat wordt er verteld}\\

\haiku{Op het oogenblik}{bent u bezig mij ergens}{in te betrekken}\\

\haiku{Altenstadt die belt,?}{mij immers telkens op met}{een verdraaide stem}\\

\haiku{Waren die menschen?}{in de gemeenteraad nu}{zoo gauw uitgepraat}\\

\haiku{{\textquoteright} Van tijd tot tijd zal.}{ze onder de maaltijd iets}{dergelijks zeggen}\\

\haiku{het zoo echt als de?}{auteurs een dominee in}{de maling nemen}\\

\haiku{op de bovenste.}{of onderste tree van een}{wenteltrap desnoods}\\

\haiku{Hij kijkt het avondblad.}{door zonder er veel van in}{zich op te nemen}\\

\haiku{Hij legt de courant}{neer en omvat zacht met wijd}{gespreide handen}\\

\haiku{En de nikkelen.}{stang van de leeslamp voor hem}{krimpt in en zet uit}\\

\haiku{Maar dat is niet het.}{allervreemdste wat hem op}{weg naar huis overkomt}\\

\haiku{Ze kwam ergens uit,...}{de schaduw vandaan uit de}{schaduw van een muur}\\

\haiku{Heeft Anne-Cris?}{je naar me toegestuurd om}{dat uit te visschen}\\

\haiku{{\textquoteright}, zegt hij bij zichzelf, {\textquoteleft} -?}{ik ik heb dat  toch wel}{eerder meegemaakt}\\

\haiku{{\textquoteleft}O ja, nu kijk je,?,.}{gepast-verdraagzaam}{h\`e net als Wedzieg}\\

\haiku{Wat wordt ze ook al,?,,!}{anders h\`e Anne-Cris}{verb\'azingwekkend}\\

\haiku{Hij zit ook voorover,,.}{er hangt een sliert haar voor zijn}{oogen zijn oogen gloeien}\\

\haiku{, was dit wat, dat   -...?}{je dat je in je stille}{tijd gekregen hebt}\\

\haiku{Ik heb een die bij,,.}{me bleef alleen gelaten}{al ging ik niet weg}\\

\haiku{Duizenden menschen -...}{slachten duizenden menschen}{af in deze tijd}\\

\haiku{En h\`eb je ooit wat,?}{geweigerd in die tijd t\'oen}{ze je nog wat vroeg}\\

\haiku{Zoo zag ik het ook -.}{zoo zal alles gebeuren}{wat je gedacht hebt}\\

\haiku{Omdat{\textquoteright}, raadt Imkje, {\textquoteleft}.}{Mareeshij de vrouw van die}{vriend ambieerde}\\

\haiku{En hij leert de  !}{kinderen niet eens waarom}{ze gelooven moeten}\\

\haiku{Dan is het weer of...}{hij zichzelf verdedigen}{moet ergens tegen}\\

\haiku{Taco neemt zijn pijp:}{een oogenblik uit zijn mond}{en zegt in zichzelf}\\

\haiku{{\textquoteleft}Als Anne-Cris,.}{vroeger maar terug gepraat}{had op wat ik zei}\\

\haiku{Ze ontwijkt Taco's,... {\textquoteleft}}{blik en kijkt toch naar hem kijkt}{toch telkens naar hem}\\

\haiku{Ze zitten ook in.}{de goudbruine salon bij}{burgemeester Heinz}\\

\haiku{Hij ziet de zachte,:}{winter daarbij de winter}{buiten de stadspoort}\\

\haiku{{\textquoteleft}Ik zal haar wakker - -...{\textquoteright}}{kijken ik wil praten wil}{het uit-praten}\\

\haiku{Er is zelfs wat zwoels.}{in het fijne geknister}{van haar pyama}\\

\haiku{Ik heb een die bij,.}{mij bleef alleen gelaten}{al ging ik niet weg}\\

\haiku{Later is het of:}{hij repeteert wat hij in}{Grensted gelezen heeft}\\

\haiku{God-god, al die...,?}{jaren wat is er met ons}{wat is er met jou}\\

\haiku{Hoe lang denk je hier,, -?}{nog mee door te gaan met die}{houding met alles}\\

\haiku{Dan is het of hij,.}{haar slaan wil of hij zich in}{woede op haar werpt}\\

\haiku{J{\'\i}j mocht haar nooit erg -..{\textquoteright}}{idioot zooals ze beslag lei}{op Anne-Cris}\\

\haiku{Ze trekt een effen.}{gezicht en speelt argeloos}{met haar servet-ring}\\

\haiku{{\textquoteleft}Als ik iets in dat{\textquoteright},, {\textquoteleft}.}{opzicht weet zegt hij wrangdan}{ligt dat niet aan jou}\\

\haiku{{\textquoteleft}Het moet allemaal, -.}{nog heel anders worden met}{ons heel heel anders}\\

\haiku{Ze loopt zoetjes voort.}{in de gang zonder nog naar}{hem om te kijken}\\

\haiku{Maar we kunnen die...}{foto's ook afzonderlijk}{verkrijgbaar stellen}\\

\haiku{{\textquoteright} Bos maakt ook nog een,.}{praatje voor de schijn neemt hij}{een paar clich\'e's mee}\\

\haiku{{\textquoteright} Maar dan merkt hij toch,.}{ook dat Jurgen bij hem staat}{en dringender praat}\\

\haiku{En ik kan het werk,.}{van Jozefien minstens zoo}{goed doen als zij zelf}\\

\haiku{Daarom, als u er,.}{voor voelt u kunt gerust wat}{laten vallen ook}\\

\haiku{{\textquoteright} En Taco zou de.}{jongen graag een eind van zich}{af willen duwen}\\

\haiku{{\textquoteright} Cobie kijkt op om.}{te zien of Taco er ook}{om glimlachen moet}\\

\haiku{Je schaadt mij in mijn,.}{goede naam en dat zal je}{merken in je zaak}\\

\haiku{Dat is altijd zoo,:}{als hij alleen achterblijft}{hij kent dat zoo goed}\\

\haiku{En de grijns van het.}{Satzuma-aapje zou hij}{stuk willen gooien}\\

\haiku{Och, nou ja, het moest,.}{er toch van komen het ligt}{er nu eenmaal toe}\\

\haiku{{\textquoteleft}Je moet me nou maar,,.}{alleen laten Cobie laat}{me nou maar alleen}\\

\haiku{, het doet er nou toch,!}{niks meer toe het is nou toch}{voor alles te laat}\\

\haiku{Dat Marees weg is -,.}{dat Imkje dit nu heeft dat}{drukt mij als een schuld}\\

\haiku{Ik heb me er vaak,}{in verdiept wat je toch deed}{het licht brandde hier}\\

\haiku{Nu ziet ze er uit:}{of iets of iemand haar heel}{erg beschadigd heeft}\\

\haiku{Ze bijt zoo heftig,.}{in haar lip of ze haar lip}{er afbijten wil}\\

\haiku{{\textquoteright} Wedzieg glimlacht op.}{zijn rustige langzame}{boeren-manier}\\

\haiku{{\textquoteleft}Geen vraag mocht ik daar!}{op die houseparty in}{het openbaar stellen}\\

\haiku{v\'oor de vrede moest,.}{zijn d\`an de menschen van de}{volkomen liefde}\\

\haiku{Hier en daar, ver het,,...}{land in staat een verkleumd huis}{een kreupele boom}\\

\haiku{{\textquoteleft}Is het toeval dat...?}{die man uit Stritz over Marees}{begon te zwammen}\\

\haiku{Uit de verte kijkt,,.}{hij ook door schemer en mist}{heen naar Crijna's huis}\\

\haiku{{\textquoteright} Wat verwonderd merkt,.}{hij dat hij weer in zijn stoel}{bij de kachel zit}\\

\haiku{Ze bedenkt zich, haar,.}{oogen gaan wijder open rare}{opengescheurde oogen}\\

\haiku{Ja, maar ik had soms,...}{dag aan dag hoofdpijn ik viel}{herhaaldelijk flauw}\\

\haiku{Je liefkoosde ze,.}{haast toen je ze in het vloei}{onder je arm hield}\\

\haiku{Ze wist de naam van,.}{die dame wist allerlei}{bizonderheden}\\

\haiku{Cobie en Weigel -.}{dat waren allebei mijn}{troeven tegen jou}\\

\haiku{Nou begint het weer{\textquoteright},, {\textquoteleft} -...}{hijgt zedat benauwde dat}{was vanmiddag ook}\\

\haiku{Ze tast naar iets dat -.}{vlak bij haar gezicht was en}{het is er niet meer}\\

\haiku{Dan breng ik het naar {\textquotedblleft}{\textquotedblright},.}{je collega vanBosch en}{Ven die neemt het wel}\\

\haiku{En een vreemd gevoel,.}{bekruipt hem iets van warmte}{en verteedering}\\

\haiku{Is er... is er wat...}{is er iemand werkelijk}{iets aan gelegen}\\

\haiku{Als ze weg is, blijft.}{hij nog een heele poos op}{de waterpoort staan}\\

\haiku{Maar Gijs Bard trekt zich.}{daar voor het oogenblik geen}{syllabe van aan}\\

\haiku{Hij lijkt weer min of.}{meer op die oue montere}{Gijs Bard van vroeger}\\

\haiku{{\textquoteleft}Vader{\textquoteright}, zegt Us, {\textquoteleft}we!}{hebben een wegenkaartje}{gevonden voor je}\\

\haiku{{\textquoteleft}Heeft Anne-Cris{\textquoteright},.}{altijd de hand in bepaalt}{Taco heimelijk}\\

\haiku{, met al die menschen,?}{over de heele wereld die}{openstaan voor Oxford}\\

\haiku{{\textquoteright} Maar op een morgen,,.}{voor kantoortijd hoort hij haast}{kregel Kaatin aan}\\

\haiku{Kaatin kijkt glunder,.}{op hem neer onder de rand}{van zijn deukhoed uit}\\

\haiku{{\textquoteright}, vraagt hij zich verbluft, {\textquoteleft}?}{afhoe ter wereld krijgt Look}{d\`at nou voor elkaar}\\

\haiku{Moeder is... nou - het, -}{kan niks schelen maar het is}{toch zoo niet zoo bar}\\

\haiku{En dat alles bij{\textquoteright},, {\textquoteleft}.}{elkaar denkt Tacois dus}{een houseparty}\\

\haiku{{\textquoteleft}Ik wou alleen maar - {\textquotedblleft}{\textquotedblright},!}{beste kerel tegen je}{zeggen anders niet}\\

\haiku{De meeste luitjes -.}{zijn tot de nok toe vol van}{hun eigen zonden}\\

\haiku{{\textquoteleft}Je kan het immers?}{niet doorgronden met je goeie}{normale verstand}\\

\haiku{{\textquoteleft}Zoo Louwtje{\textquoteright}, zegt hij, {\textquoteleft}?}{heb je het nog al naar je}{zin tegenwoordig}\\

\haiku{En soms is het, of,,... {\textquoteleft}}{een geluid daar als op de}{teenen heen en weer sluipt}\\

\haiku{Ja, misschien of we '...}{weers samen op reis gaan}{in de vacantie}\\

\haiku{{\textquoteleft}Alle duivels, op.}{dat verl\`angen kom je nou}{ook telkens terug}\\

\haiku{Ik voelde mij toch -:}{nog altijd stukken beter}{dan Anne-Cris}\\

\haiku{Hij knaagt een beetje.}{op zijn onderlip en hij}{loopt niet al te vlug}\\

\haiku{{\textquoteleft}Veel te erg, het is -{\textquoteright},.}{veel te erg wat aardig haar}{stem blijft er mat bij}\\

\haiku{Want die groote sterke.}{aandrift is Taco nu te}{machtig geworden}\\

\subsection{Uit: De domineesvrouw van Blankenstein}

\haiku{{\textquoteleft}En het hoefde niet{\textquoteright},, {\textquoteleft}.}{denkt Djoeke vaaghet was niet}{eens noodzakelijk}\\

\haiku{Er gaat een lange,}{schrale vrouw langs haar heen een}{gehavend streng-oud}\\

\haiku{Je sleepte mij dwars,.}{door de woede heen je nam}{mij mee naar de wraak}\\

\haiku{{\textquoteleft}Het heerlijk ambt{\textquoteright}, zei,, {\textquoteleft}.}{hij stil als uit de verte}{het ambt der ambten}\\

\haiku{Djoeke slaat rechts af,,.}{een laantje in daar is het}{witte pension}\\

\haiku{Heeft ooit iemand hier,?}{het uitzicht bewonderd en}{de stilte geroemd}\\

\haiku{Een oogenblik...{\textquoteright} {\textquoteleft}Voor{\textquoteright},, {\textquoteleft}!}{mij overdrijft Djoekewas het}{een menschen-leeftijd}\\

\haiku{Djoeke kent dat van,,.}{hem als hij lang gezwegen}{heeft  praat hij zoo}\\

\haiku{Zinnend op de een.}{of andere plicht komt zij}{de huiskamer in}\\

\haiku{Misschien zit er meer,,.}{goeds in die jongen dan in}{jou of in mij Veen}\\

\haiku{Drie paden loopen,.}{er op het plein uit Aage}{kiest het middelste}\\

\haiku{{\textquoteleft}Kind, ik ben hier nu,,...}{pas dit is de eerste dag}{niet zoo doordrijven}\\

\haiku{Djoeke tuurt naar dit,.}{doffe en verdoofde zij}{tuurt op een raadsel}\\

\haiku{{\textquoteright} {\textquoteleft}Kom{\textquoteright}, zegt Aage dan, {\textquoteleft},.}{van vlakbijwe moeten naar}{huis het is etenstijd}\\

\haiku{Het is of elke.}{bloem met een uitgestrekte}{hals het pad afkijkt}\\

\haiku{Hij schiep het bosch en.}{strooide een blonde schemer}{over de paden heen}\\

\haiku{{\textquoteleft}Ben je zelf ook niet...?}{door de goedheid gegrepen}{en overeind gezet}\\

\haiku{{\textquoteleft}Voor Ties{\textquoteright}, denkt zij, zij,.}{schikt de bloemen zorgzaam een}{mooie boeket wordt het}\\

\haiku{Wacht nog een beetje,,.}{man de duivel maakt je al}{een mooie strik gereed}\\

\haiku{Over een klein stukje -, -.}{van de wereld een scherf en}{meer niet trekt een grijns}\\

\haiku{{\textquoteleft}Weet je nog, dat wij,?}{hier samen hebben zitten}{schreien op een keer}\\

\haiku{Hij veegt stilletjes,.}{zijn oogen af hij doet zijn best}{om te glimlachen}\\

\haiku{En Aage legt de.}{handen vast en dwingend op}{de vuisten van Ties}\\

\haiku{Bedremmeld ligt Ties,.}{daar in zijn bedstee en zwak}{pinkt hij met de oogen}\\

\haiku{{\textquoteright} {\textquoteleft}Excuseer mij even{\textquoteright},,.}{fluistert zij tegen Djoeke}{en loopt vlug vooruit}\\

\haiku{Zij begint daar op.}{het boschpad al een gesprek}{met Aage over Thea}\\

\haiku{Neen, Heile heeft geen,.}{zin om te praten hoog kijkt}{Heile over haar heen}\\

\haiku{{\textquoteright}, grijnst dat geringe,.}{en het verheft zich het rijst}{hoog boven haar uit}\\

\haiku{Hard loopt ze op huis,,.}{toe het valt niet mee het is}{een heel eind naar huis}\\

\haiku{De bleekheid van zijn.}{wangen lijkt zich ook uit te}{spreiden over zijn oogen}\\

\haiku{{\textquoteright} Hij strijkt met de rug -.}{van zijn hand over de oogen of}{hij tijd wil winnen}\\

\haiku{Titels van boeken,.}{opgeven een enkele}{maal iets uitleggen}\\

\haiku{{\textquoteright} {\textquoteleft}Dat moest een man nooit{\textquoteright},, {\textquoteleft},!}{doen keurt Aage afeen vrouw}{ook niet natuurlijk}\\

\haiku{De rimpels in zijn,.}{gezicht spannen zich over een}{zorg heen een droefheid}\\

\haiku{{\textquoteleft}Je hamer en je,.}{spijkers deugen er niet voor}{Djoeke Veenema}\\

\haiku{Dit is van alles{\textquoteright},, {\textquoteleft}.}{het mooiste valt Djoeke in}{het gaan naar de kerk}\\

\haiku{Hij ziet een paar groote,...}{verschrikte vrouwenoogen oogen}{vol pijn en liefde}\\

\haiku{{\textquoteleft}Heer{\textquoteright}, bidt zij met open, {\textquoteleft}.}{oogenlaat ons hart zoo wijd als}{de aarde worden}\\

\haiku{, alles wuift aan haar,,,.}{zij knikt en zij wuift en als}{zij zit wuift zij nog}\\

\haiku{Hij heft de handen, -.}{op en spreekt het votum uit}{een gebed blijft hij}\\

\haiku{Men hoort het suizen.}{van de zomerwind langs de}{hooge spitse ramen}\\

\haiku{{\textquoteleft}Alle drie zijn zij{\textquoteright},, {\textquoteleft}.}{al lang dood zegt hijen zij}{leven toch nog zoo}\\

\haiku{Soms kneep ik mijn oogen:}{stijf dicht en dan zei ik een}{paar maal achtereen}\\

\haiku{{\textquoteleft}Mijn Moeder die is,,.}{vroeg getrouwd Aage die is}{lang eenzaam geweest}\\

\haiku{{\textquoteleft}Zou je dan eerst niet,?}{die blauwe boon op tafel}{leggen Jan Hendrik}\\

\haiku{Jan Hendrik was \'een.}{van zijn beste aardigste}{catechisanten}\\

\haiku{Hij zwijgt al weer, er,!}{flikkert iets door hem heen er}{gaat hem een licht op}\\

\haiku{De Heere God knielt,:}{op de aarde neer en schept}{het madeliefje}\\

\haiku{Wel mensch, ze stond me,!}{toch zoo klaar voor oogen in mijn}{binnenst schreide ik}\\

\haiku{Ze wil... wil pijn doen,.}{omdat het van binnen zoo}{schreien kan bij haar}\\

\haiku{En wat zal ik in,?}{de kerk vinden dat ik in}{het bosch niet aantref}\\

\haiku{Wietze-zelf zit.}{onder de breede schaduw}{van zijn eikeboom}\\

\haiku{Hij glimlacht toch al -.}{weer maar het is een glimlach}{met wat strengs er in}\\

\haiku{{\textquoteleft}Luister, het mooiste,,.}{wijf van de heele omtrek}{dat ben jij Elsie}\\

\haiku{{\textquoteleft}Zeg Elsie, bidt ons?}{kind wel iedere avond}{voor zij slapen gaat}\\

\haiku{De stille zwarte.}{dennen en de bleeke sterren}{verdwijnen ineens}\\

\haiku{Jenneke van Heist,.}{laat het gordijn zakken en}{steekt het lampje op}\\

\haiku{zij steken, zij haalt}{zich Aage's blik voor de geest}{en Aage's glimlach}\\

\haiku{ik aanstonds laten{\textquoteright},, {\textquoteleft}.}{zien bepaalt hij zakelijk}{als Vader er is}\\

\haiku{{\textquoteleft}Oh-ho{\textquoteright}, Heile hoest, {\textquoteleft}!,!}{van de lachwat {\`\i}s het weer}{mooi wat is het mooi}\\

\haiku{Hij legt zijn hand open,,.}{voor Rein neer een betuiging}{is dat een vraag ook}\\

\haiku{of ik stel me voor,....}{dat we praten over dat we}{ergens over praten}\\

\haiku{{\textquoteright} Aage drukt zijn wang.}{tegen het gladde bruine}{kindergezicht aan}\\

\haiku{{\textquoteright} Aanhalig klinkt dat.,.}{Zij graven klimop-planten}{uit zij drinken thee}\\

\haiku{{\textquoteright} Een paar simpele.}{woorden van Thea trekken haar}{weer in de kamer}\\

\haiku{Op jou{\textquoteright}, bekent Thea, {\textquoteleft}.}{luchtigis de notaris}{bijzonder gesteld}\\

\haiku{{\textquoteright}, biecht ze luchtig, {\textquoteleft}och,,.}{ja ik heb van die buien}{dan ga ik flirten}\\

\haiku{{\textquoteright} Koud wordt ze daarbij,.}{in haar gezicht dat koude}{grijpt haar bij de keel}\\

\haiku{maar smartelijke,.}{dingen een schaamtevolle}{biecht leggen zij af}\\

\haiku{* * * De week rust uit in:}{de heldere stilte van}{de Zaterdagavond}\\

\haiku{{\textquoteleft}Djoeke{\textquoteright}, fluistert hij, {\textquoteleft}...{\textquoteright},?}{Djoeke Meer verstaat zij niet}{maar hoe klinkt dat toch}\\

\haiku{Nu, zoeken, dat is,...}{een vermoeiend werk en zwaar}{als het voor niets is}\\

\haiku{Zij staan bijvoorbeeld,.}{aan een golvend korenveld}{de vrouw en haar man}\\

\haiku{{\textquoteleft}Schrale aren, de boer,.}{is te zuinig geweest met}{zijn kalk zijn phosphor}\\

\haiku{{\textquoteright} De notaris kijkt.}{Djoeke aan of hij van haar}{een antwoord verwacht}\\

\haiku{Zij merkt nu pas dat,.}{ze doorgeloopen zijn de}{notaris en zij}\\

\haiku{Zij wikkelt de stof,.}{van het karton en spreidt die}{uit over de handen}\\

\haiku{Langs de enkele,:}{huizen op het plein loopen}{maar een paar menschen}\\

\haiku{Op een zonderling.}{afwerende manier maakt}{zij het pakje open}\\

\haiku{{\textquoteright} {\textquoteleft}En morgen samen{\textquoteright},, {\textquoteleft}.}{in de kerk valt Djoeke nog}{insamen bidden}\\

\haiku{{\textquoteright} Als er huisjes in,.}{het zicht komen ademt zij weer}{gemakkelijker}\\

\haiku{Zij wrijft zich in de,.}{handen er is nog wel meer}{dat zij weten wil}\\

\haiku{Daar is dag dien God:}{een onsterfelijke ziel}{gegeven  heeft}\\

\haiku{Als Djoeke er aan,.}{terugdenkt is het of zij}{wegzinkt in een droom}\\

\haiku{Zij staat met de rug,!}{naar Aage toe zij weet toch}{precies wat hij doet}\\

\haiku{Hij denkt aan  een,.}{fout van Tjisse hij ziet een}{tekort van zichzelf}\\

\haiku{tegemoetkomend,!}{hartelijk v\'aderlijk zou}{men kunnen zeggen}\\

\haiku{Nu wil zij heengaan,,,!}{neen zij luistert en kijkt toe}{zij blijft toch nog staan}\\

\haiku{die pepert het de,.}{menschen w\`el in kerken vol}{volk trekt hij dan ook}\\

\haiku{Je bent de oudste,,!}{man van het dorp Tjisse je}{weet wat ik bedoel}\\

\haiku{{\textquoteright} {\textquoteleft}Die heeft{\textquoteright}, haalt Tjisse, {\textquoteleft}.}{dan nog mompelend aanzal}{gegeven worden}\\

\haiku{Tjisse wordt bang, maar,}{hij haat zijn bangheid zooals hij}{ook zijn beschaamd heet}\\

\haiku{je zult dit willen,:}{vergeten maar je zult het}{moeten onthouden}\\

\haiku{Zij hadden hem al,!}{vergeten zij hadden hem}{de rug toegedraaid}\\

\haiku{{\textquoteleft}Kijk naar deze fraai{\textquoteright}.}{gevormde hand. En men kijkt}{onwillekeurig}\\

\haiku{{\textquoteleft}Zoo?, eigenaardig{\textquoteright},.}{en dan luistert hij naar iets}{dat niet gezegd wordt}\\

\haiku{Er gaat ook iets langs,.}{haar heen in die vraag dat niet}{gekend wil worden}\\

\haiku{Zij kijkt schuchter op,.}{zij wil het zoo zachtzinnig}{mogelijk zeggen}\\

\haiku{Stoppelig geel haar,,.}{hebben zij hel-blauwe}{oogen roode wangen}\\

\haiku{{\textquoteleft}Kan jij nu nog naar,?}{slechte voorbeelden kijken}{Djoeke Veenema}\\

\haiku{De vrouwen loopen -!}{op schoenen met hooge hakken}{en het zijn dames}\\

\haiku{De zonde zit daar.}{aan de overkant en kijkt haar}{recht in het gezicht}\\

\haiku{Het is waar{\textquoteright}, beseft, {\textquoteleft} -...{\textquoteright}}{ze bang-verbaasddat is}{er ook nog dat ook}\\

\haiku{En hij kijkt ook op -:}{hij oogt naar buiten en het}{uitzicht bekoort hem}\\

\haiku{{\textquoteright} Maar dan is het of.}{er een pijn-plooi in Aage's}{vage glimlach komt}\\

\haiku{{\textquoteleft}Als Thea hier nu zat{\textquoteright},, {\textquoteleft}...}{werpt ze op in zichzelfof}{Mieneke Eiber}\\

\haiku{Als men boven op,.}{die berg staat is het of men}{God aanraken mag}\\

\haiku{Hij maakt met zijn twee.}{groote vuisten een uitbundig}{gebaar in de lucht}\\

\haiku{{\textquoteright} {\textquoteleft}Net zoo goed als ik,{\textquoteright},.}{jou te woord sta Reinbeek zegt}{Aage zonderling}\\

\haiku{{\textquoteright} Er suist iets voorbij,,,}{iets grilligs is dat het komt}{terug ook daar suist}\\

\haiku{Roode randen had{\textquoteright},, {\textquoteleft}.}{Aage om de oogen denkt zij}{en wat zag hij wit}\\

\haiku{{\textquoteleft}Ja{\textquoteright}, stemt zij toe, {\textquoteleft}heel...{\textquoteright}.}{erg Maar dan voelt men dat zij}{aan iets anders denkt}\\

\haiku{{\textquoteright}, de gramschap van de.}{Kosteres lijkt plotseling}{te verergeren}\\

\haiku{{\textquoteleft}Hij loopt al een maand,.}{lang met stof op de hoed met}{een knoop van de jas}\\

\haiku{Djoeke praat, zij discht,.}{ware verhalen op zij}{bedenkt verhalen}\\

\haiku{Djoeke wacht tot de,.}{toonladder ten einde is}{dan komt zij binnen}\\

\haiku{{\textquoteleft}Als vrouwen een kind,.}{verwachten schijnen zij die}{trek veel te hebben}\\

\haiku{{\textquoteleft}Goed{\textquoteright}, bewilligt hij, {\textquoteleft}.}{een dezer dagen kom ik}{gewoon aanloopen}\\

\haiku{De kamerdeur staat,,.}{aan en gaat nu langzaam open}{Aage komt binnen}\\

\haiku{En Aage luistert,.}{met een trek van inspanning}{hij antwoordt verstrooid}\\

\haiku{Och ja, een mensch hoort,.}{en ziet altijd precies wat}{zijn vrees hem ingeeft}\\

\haiku{Zij legt haar hand op,.}{zijn schouder en ziet dat zijn}{voorhoofd vochtig is}\\

\haiku{Maar zijn oogen trekken,.}{haar dichterbij hij drukt het}{gezicht in haar haar}\\

\haiku{Zij maakt licht, sluit de.}{gordijnen en wil moe op}{een stoel gaan zitten}\\

\haiku{Neen, zij zou naar bed,,.}{gaan zij beloofde het nu}{moet ze het ook doen}\\

\haiku{{\textquoteright} Ja, maar haar hand draalt,,.}{haar hand zweeft om de knop heen}{en beroert die niet}\\

\haiku{{\textquoteright} Zij probeert er niet,.}{naar te luisteren maar dat}{moet zij opgeven}\\

\haiku{Ik keek Maria aan, -.}{ik keek naar het leven ik}{hoorde Gods hart slaan}\\

\haiku{{\textquoteleft}Het regent niet meer{\textquoteright},, {\textquoteleft}?,...{\textquoteright}}{merkt Aage ineenshoor je}{het regent niet meer}\\

\haiku{Djoeke kan haar oogen,.}{weer openen het laken weer}{wat terugschuiven}\\

\haiku{Maar Djoeke loopt nog,.}{naar vrouw Wulk toe daar in het}{donker van haar bed}\\

\haiku{HET KAN WEZEN DAT,}{ER IEMAND BIJ AAGE IS}{HET IS MOGELIJK}\\

\haiku{Nu gaan wij aanstonds{\textquoteright},, {\textquoteleft}.}{naar Maria's graf denkt zij}{naar Maria's graf}\\

\haiku{als Aage het niet,!}{over Maria heeft hoort zij toch}{dat hij over haar praat}\\

\haiku{Ze duikt een beetje,.}{ineen ze duikt enkel maar}{een beetje ineen}\\

\haiku{Daar liepen wij vaak '....}{s avonds Ook in de tijd toen}{zij Rein verwachtte}\\

\haiku{De zomer is niet -.}{dood de zomer is naar een}{ander land gegaan}\\

\haiku{Wietze zit onder.}{het gele Boeddhaatje in een}{krakende leunstoel}\\

\haiku{Het is of Wietze,.}{hem beet pakt bij de jasmouw}{dat is toch niet zoo}\\

\haiku{{\textquoteleft}Twaalf gulden in de,.}{week komt er ook nog bij twaalf}{gulden in de week}\\

\haiku{{\textquoteleft}Ik ben nu{\textquoteright}, haspelt, {\textquoteleft}.}{hij met zijn lachende mond}{een rijke mijnheer}\\

\haiku{Jetske Zwart loopt af,.}{en aan Gerreke van Driel}{steekt een bezoek af}\\

\haiku{Nu, {\`\i}k voor mij, {\`\i}k.}{geef mij niet bij voorkeur over}{aan oude droomen}\\

\haiku{hij klimt de trap op,,.}{hij gaat zijn kamer in hij}{loopt in gedachten}\\

\haiku{{\textquoteleft}Neen{\textquoteright}, zegt Djoeke, {\textquoteleft}men,...?}{kan niet weten wat het is}{wie zal het zeggen}\\

\haiku{{\textquoteleft}Dag{\textquoteright}, hij geeft haar een,, {\textquoteleft}{\textquoteright},.}{hand hij buigt zich wat naar haar}{toed\`ag zegt hij weer}\\

\haiku{{\textquoteleft}Ja-ja, dat zal ik,...{\textquoteright},.}{doen dank je dank je wel zij}{fluistert dat bijna}\\

\haiku{{\textquoteright} Maar dan plotseling.}{trekt er wat doezeligs van}{haar gedachten weg}\\

\haiku{Aage heeft blauwe,.}{plekken in het gezicht hij}{ziet er verkleumd uit}\\

\haiku{Wij glijden altijd,.}{met een heele bende de}{hooge brug af bij school}\\

\haiku{Hij vouwt de handen,.}{ineen over de knie hij drukt}{de kin op de borst}\\

\haiku{Een eekhoorntje uit.}{het dorp zorgde al-vast}{voor hazelnoten}\\

\haiku{* * * ~ Dikwijls wendt het,.}{leven zich van Djoeke af}{in deze dagen}\\

\haiku{Nou, Moeder is toch,?}{z\'oo niet dat ze haar eigen}{daar verstoppen zal}\\

\haiku{Maar Mieneke hoort -.}{dat niet zij luistert naar de}{stap van haar Moeder}\\

\haiku{Ze kijkt de kamer,,.}{in haar oogen gloeien blauw vuur}{is er in die oogen}\\

\haiku{Mieneke's handen,:}{knellen niet meer Mieneke's}{oogen staren niet meer}\\

\haiku{Dat ik niet meer zoo,.}{moe zal zijn dat ik nu niet}{meer zoo hoesten moet}\\

\haiku{Afgetrokken gaat,.}{zij door de bleeke dag laat in}{de nacht slaapt zij in}\\

\haiku{{\textquoteright} Dan nadert God haar,.}{ook in een pijn de pijn trekt}{schrijnend door haar borst}\\

\haiku{{\textquoteright}, trekt het schemerig, {\textquoteleft}?}{door haar heenheb ik er nooit}{eerder aan gedacht}\\

\haiku{Japke vischt een paar.}{hazelnoten op uit een}{zak onder haar jurk}\\

\haiku{{\textquoteleft}Als Mevrouw het merkt{\textquoteright},, {\textquoteleft} -.}{voorziet zijgaat ze voort weg}{ben ik weer alleen}\\

\haiku{{\textquoteright} Weer kijkt Japke op.}{met die vreemde helderheid}{in haar zwarte oogen}\\

\haiku{Van wie zijn ze toch{\textquoteright},, {\textquoteleft}?,?}{soest Djoekedie voetsporen}{welk doel hebben zij}\\

\haiku{de groote bladen, als,, {\textquotedblleft}}{ze hem d\'aar willen mij best}{maarDe Kandelaar}\\

\haiku{{\textquoteright} Murman heft een groote,.}{harige vuist op en wijst}{er mee naar Aage}\\

\haiku{Gods uitgestrekte,:}{handen en een glans valt over}{haar gedachten heen}\\

\haiku{Nu staat zij op een,.}{steenen hoogte en de steden}{kijken naar haar om}\\

\haiku{vrouwen met een ring,.}{in de neus mannen met een}{vacht om de heupen}\\

\haiku{Toen hebben wij toch -,:}{om de dokter gestuurd nou}{een zware ziekte}\\

\haiku{{\textquoteright} Riek krabbelt in de,.}{stugge bakkebaardjes plukt}{aan de wenkbrauwen}\\

\haiku{Wit staan de fijne,,.}{dennen daar wit de eiken}{en wit is het dorp}\\

\haiku{Rein springt daverend,.}{de trap af en hij praat met}{een hooge schelle stem}\\

\haiku{Djoeke, help je de,...?}{touwen losmaken vouw jij}{de papieren op}\\

\haiku{{\textquoteleft}Christus is met ons{\textquoteright},, {\textquoteleft}.........{\textquoteright}.}{prevelt zeChristus is met}{ons De schrik laat af}\\

\haiku{Maar Eiber rekt de,,...}{hals beweegt de lippen komt}{een stap naderbij}\\

\haiku{Hoe zoo...?, wij spelen,.}{hier open kaart wij nemen hier}{geen blad voor de mond}\\

\haiku{En een oogenblik.}{is het of Djoeke wegzinkt}{in een  diepte}\\

\haiku{De lichtschijven  ,,.}{wentelen weer rond maar ze}{zijn dunner vager}\\

\haiku{Zij stelt vragen op,.}{in haar gedachten zij is}{bang-nieuwsgierig}\\

\haiku{{\textquoteleft}Ja{\textquoteright}, prevelt zij, {\textquoteleft}wij...{\textquoteright}:}{moeten naar huis Droomerig}{tuurt ze voor zich uit}\\

\haiku{En Rein kijkt zwijgend,.}{naar haar gebogen schouders}{haar gebogen hoofd}\\

\haiku{In haar borstwering,.}{van sjaals doeken en dekens}{kijkt Djoeke er naar}\\

\haiku{De rijkdom van de.}{geheele wereld ligt er}{in opgesloten}\\

\haiku{Haar zachte stappen,.}{sterven weg in het huis in}{de sneeuw daarbuiten}\\

\haiku{Achter een ringmuur,.}{van sterren loopt Djoeke en}{kijkt naar het leven}\\

\haiku{Een kleine lamp brandt,,...}{een blauwsteenen kan glanst het}{voorhoofd van een man}\\

\haiku{En zal je onder,?}{dit alles de vreugden van}{een Moeder voelen}\\

\haiku{God, dan werd het zoo,.}{guur en kil in mij of mijn}{hart in de tocht stond}\\

\haiku{Zon-goud wiegelt,.}{tak-schaduwen spelen}{op het vensterglas}\\

\haiku{{\textquoteright} En zij luistert naar,.}{de tik van het klokje een}{stem op het landpad}\\

\haiku{Zij verdwijnen nu...{\textquoteright},,.}{begrijpt zij en zij pinkt of}{zij slaperig is}\\

\haiku{wij prijzen U.{\textquoteright} Als,,.}{een kind loopt zij door luchtig}{onregelmatig}\\

\haiku{{\textquoteleft}Zij moest ziek worden{\textquoteright},, {\textquoteleft}.}{beseft hijopdat ik mijn}{angst zou begrijpen}\\

\haiku{Er is een sterke,.}{strenge eerbied in zijn hart}{een stil blank ontzag}\\

\haiku{Zijn oogleden zijn.}{blank en glanzend of ze een}{gebed verbergen}\\

\subsection{Uit: Grillige schaduwen}

\haiku{Bikkelglad praat wat.}{stennerig en ze zucht diep}{en nadrukkelijk}\\

\haiku{{\textquoteright} Bikkelglad zucht, en,.}{even in de stilte peinst ze}{dan vertelt ze weer}\\

\haiku{Even duurt dat, dan praat,,,.}{hij weer door bijnagewoon}{bezonken langzaam}\\

\haiku{Ook zoo'n stille nacht... ',!}{enn maantje en de klok}{op bij-twaalven}\\

\haiku{Heer in de hemel,,,.........?}{dan ook de buitelaar wie}{was hij wie was hij}\\

\haiku{'n Gewoonte had, ', '...}{hij opt rooie kamertje}{t opkamertje}\\

\haiku{En toen... toen \`al die,:}{dagen er na soesde hij}{er gedurig over}\\

\haiku{{\textquoteright} 'n Beetje benard,,.}{en dof ook beklemder van}{adem praat hij dan door}\\

\haiku{Nee, g\'een mensch weet w\`at ',... ',...}{t is en wi\'e ent is}{meer gezien hu-u}\\

\haiku{Klokkeloodje{\textquoteright} me.}{schokkend van schrik het witte}{spitse gezicht toe}\\

\haiku{De borrel maakte, '.}{z'n tong los die maaktet}{booze in hem gaande}\\

\haiku{Elikster  zat in,,.}{de voorste rij hij lachte}{luid hij spotte luid}\\

\haiku{En die priester met,.}{z'n witte kazuifel die}{jaagt mij niet van je}\\

\haiku{Toen ze de deur open ' ':}{deed vielt lampje ent}{glas uit haar handen}\\

\haiku{Op die manier, zie, '.}{je ben je dan zoo'n beetje}{n gesjochte knaap}\\

\haiku{En ook gelijk hield.}{hij zoo ies-of-wat tusschen}{z'n duim en vinger}\\

\haiku{{\textquoteleft}Ja, ja, maats, ik hoor,,,,!}{je wel ik k\`om maats wacht nog}{maar effe ik kom}\\

\haiku{{\textquoteleft}'n mensch blijft gauw stil,{\textquoteright}, '!}{staan voor wat glinstert jaak en}{dat sluit asn bus}\\

\haiku{Goeie-grutjen,,?}{man bij zoo iets prakkezeer}{je ommers geeneens}\\

\subsection{Uit: Harlekijntje}

\haiku{Maar het hindert niet -,.}{ze doen dat enkel voor de}{grap de toovermannen}\\

\haiku{Er loopt daar - als een -.}{torentje van donkerheid}{een klein dik vrouwtje}\\

\haiku{De Groene Parkiet{\textquoteright},.}{niet uit de caf\'e-deur}{maar uit het poortje}\\

\haiku{Maar haar kin bibbert.}{een beetje en in haar eene}{wang is een kuiltje}\\

\haiku{Dat ben' stakkers met.}{een aangestoken plekkie}{in hullie hersens}\\

\haiku{En een vraag die er ',.}{als meer geweest is klopt}{opnieuw bij hem aan}\\

\haiku{Hij kan de woorden,.}{die achter Moeder's antwoord}{langs gaan ook verstaan}\\

\haiku{- kan die lachen... ziet!}{er net uit of hij ook wel}{kuiltjerol1 kan doen}\\

\haiku{Hij betast het  ,.}{randje van zijn rechteroor}{hij bevoelt een zorg}\\

\haiku{Een enkel woord vangt,:}{hij maar op een woord dat zoo}{hard als een duw is}\\

\haiku{Het steekt hem erger,.}{dan ooit dat zij geen huis met}{een voordeur hebben}\\

\haiku{Op de nagel van:}{zijn middelvinger heeft hij}{drie witte stippen}\\

\haiku{Aan een beek die naar,.}{klaver en margrieten ruikt}{lesschen zij hun dorst}\\

\haiku{{\textquoteleft}Je harlekijntje{\textquoteright},, {\textquoteleft}?}{vraagt hij ineensis dat nou}{al weer afgedankt}\\

\haiku{We hebben de wind{\textquoteright},, {\textquoteleft} -.}{tegen hijgt Vlooienbeetloopt}{zwaar in de wind op}\\

\haiku{{\textquoteleft}Weet j{\'\i}j waarom ze -,?}{Gibbetje Vonk Gibbetje}{Vonk noemen buurman}\\

\haiku{Waarom ben jij van?,?}{vleesch en bloed en niet van}{porcelein en melk}\\

\haiku{Daantje gaat midden,.}{in de straat staan en kijkt naar}{het dak van hun huis}\\

\haiku{Men kan niet uit het.}{hoofd opnoemen wat er daar}{op de akkers staat}\\

\haiku{{\textquoteleft}Kinderen hebben{\textquoteright},.}{het maar gemakkelijk zegt}{ze en dat kraakt weer}\\

\haiku{Zijn adem moet telkens.}{over een bergje klauteren}{binnen in z'n keel}\\

\haiku{En de stilte en.}{de zon koestert hen en de}{schaduw dekt hen toe}\\

\haiku{Dat h\^otel heeft een.}{deur die op een pennetje}{in de rondte draait}\\

\haiku{ik wil er toch heen,.}{het is hier in geen jaren}{op het dorp geweest}\\

\haiku{En ze stak ze in.}{de grond en ze maakte ze}{in de boomen vast}\\

\haiku{De klompenmaker,:}{hakt hout en Vrouw Grom wascht}{erwten in een pan}\\

\haiku{Stijf knijpt hij de oogen,.}{toe en bromt iets tegen de}{voering van zijn pet}\\

\haiku{Krom aan \'e\'en kant loopt,.}{Garmen en zijn beenen zwikken}{door in de knie\"en}\\

\haiku{{\textquoteleft}Ja broer{\textquoteright}, grinnikt hij, {\textquoteleft},?}{met een ernstig gezichteen}{raar brillehuis h\`e}\\

\haiku{Hij knijpt zich in de.}{dijen van de pret en trekt}{aan Crissie's wortel}\\

\haiku{Bovenmeester is -.}{nooit heelemaal uit school ook}{als hij uit school is}\\

\haiku{{\textquoteleft}Zie je Immetje?,?,?}{Groen niet en Sybrecht en is}{er geen-een van Isa}\\

\haiku{En elke keer als,.}{hij een huilschreeuw geeft grijpt hij}{naar zijn achterste}\\

\haiku{Hij wil een voornaam,.}{handgebaar maken zwierig}{zijn muts afnemen}\\

\haiku{{\textquoteright} En Harlekijn haalt, -...}{gauw de vinger uit de mond}{en hijscht hijscht}\\

\haiku{En alles in de,.}{wankele uitkijktoren}{beweegt trilt en schudt}\\

\haiku{De gedachten gaan,.}{met een zwenkende vogel}{mee hoog de lucht in}\\

\haiku{Maar zijn Vader en:}{Moeder zullen telkens-weer}{tegen hem zeggen}\\

\haiku{Daantje knijpt tusschen.}{duim en vinger een tuitje}{in zijn onderlip}\\

\haiku{{\textquoteleft}Ik zou nog wel graag{\textquoteright},, {\textquoteleft}.}{een broer willen hebben houdt}{Daantje aanwel gr\'aag}\\

\haiku{En ik - ik ben dan,,...}{een deftige mijnheer zie}{je met een wit vest}\\

\haiku{{\textquoteright} Hij bekijkt nog 's.}{de kapittelstokjes en}{het schotsche haarlint}\\

\haiku{{\textquoteright} Maar Moeder zegt niets,,.}{zij stoot Opoe aan fluistert iets}{en kijkt naar Daantje}\\

\haiku{Maar dan ineens is,.}{Teetje Schep er Teetje Schep als de}{zonneschijn-zelf}\\

\haiku{{\textquoteright} Uit hun ooghoekjes.}{kijken de monkelende}{groote menschen naar hem}\\

\haiku{{\textquoteleft}Maar nou{\textquoteright}, bedisselt, {\textquoteleft},,.}{Opadoen we het ook eerst v\'oor}{het brood eten kom Vrouw}\\

\haiku{Heb je nou ooit{\textquoteright}, lacht, {\textquoteleft}.}{ze opgetogenn\`et iets}{wat ik erg noodig had}\\

\haiku{- En later kijken,.}{ze recht voor zich uit en ze}{praten onderdrukt}\\

\haiku{Maar waarom moet Oom?}{Herre zich dan toch ineens}{naar hem omdraaien}\\

\haiku{{\textquoteright} Verlangende oogen.}{krijgt Daantje daarbij en een}{verlangende mond}\\

\haiku{Beschroomd denkt hij aan,.}{de musch terug en begint}{haastig te praten}\\

\haiku{Hij weet plotseling,,.}{dat hij onder het bidden}{gluren zal naar God}\\

\haiku{En Daantje moet zijn.}{hand in een vuistje tegen}{zijn lippen drukken}\\

\haiku{{\textquoteleft}Zie je wel{\textquoteright}, mijmert, {\textquoteleft},,.}{hijdat land waar ik was voor}{me Moeder me kreeg}\\

\haiku{En hij  wil er.}{toch mee naar het lichte land}{waar men zweven kan}\\

\haiku{En de Koning zegt,:}{in de donker dreunende}{stem van het orgel}\\

\haiku{En hij gaat dan wel,.}{behoorlijk recht-op zitten}{maar hij trekt een lip}\\

\haiku{Hij ziet ineens weer.}{het plagerige gezicht}{van Tante Celien}\\

\haiku{En Roel zet gauw het.}{napje met centen neer en}{flapt weer op zijn stoel}\\

\haiku{Maar Daantje hoeft niet,.}{in zijn boekje te kijken}{h{\'\i}j weet het z\'oo wel}\\

\haiku{En Daantje's handen.}{klemmen zich krampachtig om}{de stoelzitting vast}\\

\haiku{Juffrouw gichelt, en.}{ze is opeens weer Juffrouw}{van de Operette}\\

\haiku{Zij gaan over de brug,,.}{zij komen langs de kerk zij}{loopen de straat in}\\

\haiku{Het moet maar...{\textquoteright} En het,.}{kruis valt tegen het huis aan}{dat geen voordeur heeft}\\

\haiku{Nadenkend schuift Isa.}{haar wijsvinger onder haar}{kapittelstokjes}\\

\haiku{{\textquoteright} {\textquoteleft}Haal {\`\i}k me serviesie{\textquoteright},, {\textquoteleft}.}{verzoet Isavraag {\`\i}k een cent}{voor wonderballen}\\

\haiku{Ze begint ineens:}{een Miranda-versje}{op te zeggen}\\

\haiku{Isa het op, ze kan,.}{het zoo goed dat het haast niet}{meer te verstaan is}\\

\haiku{Opa heb een boomgaard,.}{met wel vijfhonderd boomen}{dan is die ook rijk}\\

\haiku{Een maand geleden{\textquoteright},, {\textquoteleft}.}{was dat al een beetje valt}{hem inmet die ster}\\

\haiku{{\textquoteleft}Stil, stil{\textquoteright}, knipperen,.}{zijn oogen tegen Katinka}{die doorloopen wil}\\

\haiku{Wat spijtigs springt naar,.}{voren in zijn oogen en duikt}{dadelijk weer weg}\\

\haiku{De teenen staan alle.}{tien op het randje van de}{groenhouten tafel}\\

\haiku{{\textquoteright} Grootvader's stoel geeft,.}{een krak net of Grootvader}{ineens zwaarder wordt}\\

\haiku{{\textquoteright} - - - - - - - - - Grootvader slaat weer.}{op de leuning van zijn stoel}{en hij hapt naar adem}\\

\haiku{En Katinka moet.}{nu opeens aan een lied van}{Zondagsschool denken}\\

\haiku{Dat verhelderde:}{in Gibbetje's gezicht blijft}{Brikkelebrit bij}\\

\haiku{Daantje Diddes heeft,.}{in het leven bereikt wat}{hij bereiken wou}\\

\haiku{Het beddescherm waar.}{Jonkvrouw Maleen achter staat}{is niet hoog genoeg}\\

\haiku{En zij moeten net.}{doen of ze heelemaal niet}{weten waar hij is}\\

\haiku{Het gonst en zoemt daar.}{of er tweeduizend vliegen}{dooreen dwarrelen}\\

\haiku{Maar dan is alles,,.}{als bij tooverslag veranderd}{juist als in een droom}\\

\haiku{En haastig wordt een.}{bloemetjes-gordijn}{opzij getrokken}\\

\haiku{Toen Teetje Schep in haar,.}{doodshemd op haar bed lag was}{die stilte er ook}\\

\haiku{{\textquoteleft}Ik ben ook nog maar{\textquoteright},, {\textquoteleft}...}{negen jaar zegt hij tegen}{zijn angstnegen jaar}\\

\haiku{{\textquoteleft}Het is waar{\textquoteright}, geeft hij, {\textquoteleft},}{toeik h\`eb het vergeten}{maar ik kom wel weer}\\

\haiku{het is of iemand,.}{hem bij de hand neemt hij kan}{niet anders loopen}\\

\haiku{{\textquoteright}, een pijn keert zich om,, {\textquoteleft}?}{midden in de gedachten}{ni\'et w\'ezenlijk}\\

\haiku{{\textquoteright}, herinnert Daantje.}{haar met een stevige tik}{op zijn eigen wang}\\

\haiku{Voor een nachtegaal -.}{ben je niet knap genoeg een}{uil zal je worden}\\

\haiku{Hij merkt dat hij bar,.}{veel schik kan hebben zonder}{zich te verroeren}\\

\haiku{{\textquoteright} {\textquoteleft}Een likeurglas{\textquoteright}, haalt,,}{Vader uit en hij doet of}{hij grilt van afschuw}\\

\haiku{{\textquoteleft}Wat zei Dokter nou?,?}{allegaar over mij en wat}{is dat met die zon}\\

\haiku{, denk jullie er om?,}{dat ik dan geen korsies krijg}{van de hittigheid}\\

\haiku{Zij kijken elkaar,.}{aan het is of er licht uit}{hun voorhoofden komt}\\

\haiku{{\textquoteright} Onaannemelijk,.}{lijkt dat Daantje niet al is}{er toch wat raars bij}\\

\haiku{En de knoppen van:}{de madelieven lijken}{op bakerkindjes}\\

\haiku{Terloops wil hij er,,.}{een paar van plukken Daantje}{en hij vergeet het}\\

\haiku{Opa moet zijn hoofd diep,.}{voorover buigen het antwoord}{komt als een ademtocht}\\

\haiku{Dan zakt hij weer weg.}{in een stralende witte}{diepte en wiegelt}\\

\haiku{Hij bijt er bij op,.}{een zakdoekpunt hij wringt zijn}{zakdoek om en om}\\

\haiku{Hij is er toch wel.}{een beetje trotsch op dat hij}{het gegeven heeft}\\

\haiku{{\textquoteright}, krijscht hij, {\textquoteleft}snij me,,!}{buik open haal het er uit haal}{de dorens er uit}\\

\haiku{{\textquoteright} Onbeholpen strijkt,.}{hij langs haar oogleden er}{zijn daar geen tranen}\\

\subsection{Uit: De ijzeren greep}

\haiku{{\textquoteright} Hij draait aan de knop,.}{van een laag deurtje peutert}{aan een sleutelgat}\\

\haiku{Maar zijn Moeder schijnt,.}{dat niet te begrijpen ze}{zet hem op de vloer}\\

\haiku{En Bielke ziet de:}{eerste verschijnselen van}{de samenleving}\\

\haiku{De zonneschijn trekt,.}{weg het gaat regenen en}{de deur moet dicht}\\

\haiku{Hij schraapt er met de.}{schoenzolen over heen en sleept}{ze om beurten mee}\\

\haiku{Hij zou er graag bij,.}{opklimmen maar het mag niet}{en het kan ook niet}\\

\haiku{En Grootmoeder duwt.}{Bielke een stukje vooruit}{bij de begroeting}\\

\haiku{En zijn makker kruipt.}{ook al weer door het gat van}{de ligusterhaag}\\

\haiku{Tot de hoek dan maar...{\textquoteright}.}{Maar de tonnetjesman wil}{er niets van weten}\\

\haiku{Maar haar zoon Driek, die.}{is jonger dan Vader en}{gedurig  ziek}\\

\haiku{Eerst tikte ze met:}{de punt van het mes op de}{koek en mummelde}\\

\haiku{Hij gluurde door de,.}{bedgordijnen en was eerst}{verbaasd en toen blij}\\

\haiku{De koperen bak.}{van de olielamp glinsterde}{als een klok van goud}\\

\haiku{Hij had zijn beenen al.}{gauw over de beddeplank en}{stak zijn armen uit}\\

\haiku{Toen kreeg hij opnieuw.}{een andere kiel met een}{horlogezakje}\\

\haiku{Nu zou de witte -.}{tafel met de hartjes weer}{komen Sinterklaas}\\

\haiku{haast zoo doorschijnend.}{als de glazen kast-engltjes}{in de voorkamer}\\

\haiku{{\textquoteleft}Joppe's Grootmoeder.}{Tonia is aardiger dan}{zijn Grootmoeder Brecht}\\

\haiku{Het is altijd een.}{dienst-vertelling of een}{spokenvertelling}\\

\haiku{{\textquoteleft}Die malle Freerk Kret{\textquoteright},, {\textquoteleft}?}{zeggen zewie doet er nou}{\'een kip bij een haan}\\

\haiku{Zijn Vader heeft een.}{erge snee in zijn hand en}{geen lap er om heen}\\

\haiku{{\textquoteleft}Sterven, dat is niet,,.}{het ergste maar van je kind}{weg moeten je kind}\\

\haiku{{\textquoteright} {\textquoteleft}Drinken{\textquoteright}, bedenkt hij, {\textquoteleft}...{\textquoteright} {\textquoteleft}{\textquoteright},.}{schuwtheeThee herhaalt Moeder}{en zij wil opstaan}\\

\haiku{{\textquoteleft}Moeder{\textquoteright}, hij omvat.}{haar rokken en drukt zijn hoofd}{vast tegen haar schoot}\\

\haiku{Als ik er niet ben{\textquoteright},, {\textquoteleft}....}{fluistert ze wonderlijkben}{ik er toch even goed}\\

\haiku{{\textquoteleft}Kom me jongen{\textquoteright}, zegt, {\textquoteleft}.}{hij gehaastvannacht mag je}{bij Joppe slapen}\\

\haiku{Hij draait zich fel om,.}{wil Vader's hand pakken en}{Vader is al weg}\\

\haiku{hij wil zich van zijn,.}{stoel laten glijden wil zijn}{Vader naloopen}\\

\haiku{En om de stemmen.}{komt al dichter het vreemde}{zware van de avond}\\

\haiku{Boven op zolder.}{wordt hij van de kilte weer}{een beetje wakker}\\

\haiku{hij bleek en beschroomd.}{op de binnenplaats en in}{de grijs-steenen gang}\\

\haiku{daar zijn gezichten,,,.}{van te maken een maan een}{hansworst een kerel}\\

\haiku{worst krijgen,  van...}{Soling de bakker moppen}{en pepernoten}\\

\haiku{{\textquoteright} - - - - - - - - Strak kijkt hij daarbij}{naar de sterren op en hij}{houdt de lampion}\\

\haiku{De Heer verheffe.}{Zijn aangezicht over u en}{geve u vrede}\\

\haiku{En voor de villa.}{Eusebia bloeien de}{paarse seringen}\\

\haiku{En Arjen Kappel,,.}{de veldwachter dat is geen}{gemakkelijke}\\

\haiku{En Hint's vrouw die was.}{ook telkens voor het bord van}{het Gemeentehuis}\\

\haiku{We hebben immers?}{onze plichten tegenover}{de samenleving}\\

\haiku{{\textquoteleft}Dat moest toch bij mij{\textquoteright},, {\textquoteleft}}{aan huis niet gebeuren speelt}{hij plotseling op}\\

\haiku{Grootmoeder breit ook,,.}{de heele dag en Moeder}{elk vrij oogenblik}\\

\haiku{{\textquoteright} En Micha\"el's.}{kwieke snorre-lip gaat}{bol vooruit  staan}\\

\haiku{Ze laadt haar huis af,.}{of het een schip is waar ze}{mee wegvaren kan}\\

\haiku{zoo dicht bij of hij.}{met die neus-van-hem het}{pak toedekken wil}\\

\haiku{Ze strijkt het haar glad,,.}{doet haar doek netjes ze heeft}{geen pruttellip meer}\\

\haiku{En alle menschen.}{hebben strakke gezichten}{en gespannen oogen}\\

\haiku{{\textquoteleft}Waar gaan we heen?, en?}{wat komen er nou nog meer}{voor erge dingen}\\

\haiku{{\textquoteright} Het is heel goed te.}{merken dat Moeder met haar}{zwijgzaamheid antwoordt}\\

\haiku{En hij is nog niet:}{oud genoeg om meewarig}{te kunnen zeggen}\\

\haiku{Het vroege licht is.}{zoo wonderbaar zacht of er}{maneschijn in is}\\

\haiku{De raamruitjes zijn}{vurige vierkantjes en}{de deuren glanzen}\\

\haiku{Het is nu ook of:}{hij door een vergrootglas naar}{de toekomst kan zien}\\

\haiku{rookstralen met,,,.}{kogels er in het siste}{knetterde kraakte}\\

\haiku{En Vader keert zich.}{zoo heftig snel om of hij}{boos uitvallen wil}\\

\haiku{wat belachelijks,,.}{nu is iedereen het en}{het is niet erg meer}\\

\haiku{De vrouwe-muts,.}{die v\'oor Freerk Kret opduikt is}{van Hesseltje Stoop}\\

\haiku{Hesseltje heeft al.}{in geen twee maanden bericht}{van Koertje gehad}\\

\haiku{Hij trekt jolig de,.}{wenkbrauwen op gluurt jolig}{naar de zoldering}\\

\haiku{Tusschen haar lange.}{zwarte oogharen is het}{kleurloos als regen}\\

\haiku{Ze roert toch langzaam,.}{regelmatig in de brij}{precies zooals het moet}\\

\haiku{door de winter is,.}{het toch niet het vorige}{jaar was het anders}\\

\haiku{{\textquoteleft}Thomasken{\textquoteright}, fleemt ze, {\textquoteleft}.}{buiten adem en jagerig}{ik kom u halen}\\

\haiku{{\textquoteleft}Als je de zakken,.}{vol gestopt hebt eerst maar het}{zaagsel opvegen}\\

\haiku{Ze krijgt een erg hooge,,.}{buik Nustancia het is al ver}{heen met het kindje}\\

\haiku{Vandaag het bestek,,.}{gezien en de teekening}{eerste klas spul man}\\

\haiku{{\textquoteright} {\textquoteleft}Ja{\textquoteright}, knikt Bielke, hij,.}{gaat er verder niet op in}{hij houdt zijn stuk vast}\\

\haiku{We hebben het toch,?,,?}{goed niet waar zeg jongske we}{hebben het toch goed}\\

\haiku{Het is ook zoo maar:}{met een vloek en een zucht in}{elkaar geslagen}\\

\haiku{Ze kijken eerst wie,.}{er in de werkplaats is dan}{praten ze oog wat}\\

\haiku{Ze praten er over.}{zooals ze over aambeien en}{negen-oogen praten}\\

\haiku{dat is een heel ding,.}{maar de coulissen moet je}{ook niet uitpoetsen}\\

\haiku{Met een hoovaardig.}{gezicht denkt Bielke daar nog}{altijd aan terug}\\

\haiku{{\textquoteleft}Zaterdagavond{\textquoteright}, staat, {\textquoteleft}.}{hij zichzelf toezoo'n doos van}{honderd twintig stuks}\\

\haiku{{\textquoteleft}Ja, warm{\textquoteright}, Angelia.}{schuift een ringetje op en}{neer aan haar vinger}\\

\haiku{twee ramen, een deur.}{en een tuintje  zoo groot}{als een tafelblad}\\

\haiku{{\textquoteright} De achterkant van.}{het nieuwe huis staat nu naar}{De Fonteintjes toe}\\

\haiku{het zou de oude '.}{baas goed doen als hij dit nog}{s beleven kon}\\

\haiku{Met een overdreven.}{rechte rug loopt Titia het}{keldertrapje af}\\

\haiku{Maar Donnardus vindt.}{het niet de moeite waard er}{op te antwoorden}\\

\haiku{Maar je bergplaats - je,.}{garage nou weer en dan}{hier in De Klinkert}\\

\haiku{Eigenlijk moesten wij,.}{in onze situatie}{uit zoo'n straat weggaan}\\

\haiku{{\textquoteright} Hij weet ook best wat.}{vrouw Pannes bedoelde met}{de zeemleeren filter}\\

\haiku{We weten het wel,,.}{Anna jij blijft het liefst op}{je eigen terrein}\\

\haiku{{\textquoteright} Maar als hij in de,.}{half verstijfde aardappels}{prikt vraagt hij toch weer}\\

\haiku{{\textquoteright} Hij steekt de handen,.}{diep in de jaszakken trekt}{de schouders wat op}\\

\haiku{Vijf guldens flitsen,.}{langs Bielke's gedachten hij}{lacht zacht in zichzelf}\\

\haiku{{\textquoteright} Ineens ziet hij zijn.}{Moeder's beangstigende}{oogen en proest het uit}\\

\haiku{Een deur flapt open en,.}{toe met een stug zuiggeluid}{een scharnier kreunt dof}\\

\haiku{Zoo'n stoffel als ik?,...!}{die van toeten noch blazen}{af-weet nee kind}\\

\haiku{Vanmorgen bericht,.}{gehad acht en veertig stuks}{en zes lessenaars}\\

\haiku{Als je de cene,.}{meid meer geeft dan de ander}{maak je schele oogen}\\

\haiku{Het schoot me zoo te,.}{binnen nou ligt die wensch ook}{altijd bovenaan}\\

\haiku{Het is... is of je -...}{mijn hart tegen de keien}{slingert dat hart hier}\\

\haiku{{\textquoteright} Hij omhelst haar, en.}{het is of hij haar niet in}{de armen heeft}\\

\haiku{Het was nou ook wel.}{zoo ver gekomen dat ik}{hem best missen kon}\\

\haiku{{\textquoteleft}Ik heb er lang op,,...}{gewacht baas lang  verwacht}{en toch verkregen}\\

\haiku{loonsverlaging en,.}{bekrimping van arbeidskracht}{al wat de klok slaat}\\

\haiku{Seerp liet alles,...}{zoo prachtig inrichten had}{er zoo'n pleizier in}\\

\haiku{Er stappen vreemde,.}{menschen uit menschen die hem}{onverschillig zijn}\\

\haiku{Ja, het zou wel erg.}{ondoordacht wezen als we}{nu gingen trouwen}\\

\haiku{En er zijn een paar,.}{heel lieve menschen hier daar}{heeft Tante veel aan}\\

\haiku{{\textquoteright} {\textquoteleft}Ik leer voor kapster{\textquoteright},, {\textquoteleft}}{valt Tripke in v\'oor er een}{stilte kan komen}\\

\haiku{Bielke loopt of hij,,.}{vlucht met lange stappen het}{hoofd in de schouders}\\

\haiku{Ze schrijft meer dan eerst{\textquoteright},, {\textquoteleft}.}{tracht hij nog te sussenveel}{meer en geregeld}\\

\haiku{Het orchestrion.}{achter haar rettelt en bonkt}{als een stoomhamer}\\

\haiku{{\textquoteright} Hij verzet geen voet,.}{plukt venijnig hard aan de}{voering van zijn pet}\\

\haiku{{\textquoteleft}Ik zal wel me best,,.}{doen dat ik wat krijg ik zal}{van alles probeeren}\\

\haiku{En we hebben hier,,,...{\textquoteright}}{zoo fijn gewerkt h\`e Seerp}{en zoo lang en nou}\\

\haiku{En dat is verdraaid,,.}{moeilijk me jongen dat zal}{ons tegenvallen}\\

\haiku{En weer schrikt Bielke,.}{op in zijn bed het is al}{ver in de ochtend}\\

\haiku{{\textquoteright} Dan sluit hij de oogen '?}{nog weers. Waarom zal hij}{de oogen niet sluiten}\\

\haiku{Onzinnig-lang staart.}{Bielke naar een kamerhoek}{waar niets te zien is}\\

\haiku{Maar ze mopperen,,.}{niet Vader en Moeder ze}{kijken alleen maar}\\

\haiku{De zak met kolen.}{ligt als een looden last op zijn}{ontvelde schouders}\\

\haiku{{\textquoteleft}We wisten niet waar,?,.}{je bleef zie je we hebben}{vaak uitgekeken}\\

\haiku{Titia loopt en kijkt -.}{of het leven ver achter}{haar ligt ze is oud}\\

\haiku{Hij schuift zich ook weer,,,...}{tusschen de toeschouwers in}{wuift nog even dringt stompt}\\

\haiku{ik was op zoek naar, -,.}{werk en en het was weer mis}{zoo'n lamme stemming}\\

\haiku{Bielke weet ineens.}{dat het een sjouwerman uit}{De Meerendonck is}\\

\haiku{Toch een goed ding en,:}{die werkeloozen-uit keering ook}{ze mogen zeggen}\\

\haiku{Daar vraagt de tijd nog{\textquoteright},, {\textquoteleft}...{\textquoteright} {\textquoteleft},{\textquoteright},}{al naar valt hij uitwat je}{k\`anNou ja enfin}\\

\haiku{Ik snap niet dat jij...}{er niks van wist en bij je}{thuis en in de straat}\\

\haiku{Even later staat hij.}{op het Wolvenbruggetje}{bij zijn kornuiten}\\

\haiku{Het zijn altijd de -.}{menschen die het verkeerd doen}{altijd de menschen}\\

\haiku{Hij tuurt en hij ziet'.}{de verweerde liefheid in}{Goitske Dubies oogen}\\

\haiku{Ik heb nog centen,,...}{over anders krijg ik ze wel}{ergens neem ik ze}\\

\subsection{Uit: Ik verwacht het geluk}

\haiku{En dacht jij dat dit,?}{het voor-portaal van de hel}{was snoetebakkes}\\

\haiku{Ze draagt een emmer.}{met dweilwater en buigt wat}{door in de schouders}\\

\haiku{{\textquoteright} {\textquoteleft}Op naailes was ze,{\textquoteright},.}{leuk pleit Stijn toch nog en keert}{zich met een ruk om}\\

\haiku{{\textquoteleft}Kan het haast niet zien,{\textquoteright}, {\textquoteleft}.}{verontschuldigt Pietahet}{is zoo schemerig}\\

\haiku{Kors die wou altijd.}{gelijk hebben en dagen}{lang kon hij koppen}\\

\haiku{De nieuwe jurk staat,,.}{haar goed zij is er slanker}{mee volwassener}\\

\haiku{jij verre man, daar,?}{ergens in die groote wereld}{dat ik je noodig heb}\\

\haiku{{\textquoteright}, zeggen ze tegen, {\textquoteleft}?,?}{Pietawaar bleef je zoo lang}{wat moest je nog doen}\\

\haiku{{\textquoteleft}Kijk mijn mooie gebit '{\textquoteright}.}{s. Stijn Mets stond daar haar hoofd}{bij te  schudden}\\

\haiku{Je moet natuurlijk.}{eerst meerderjarig wezen}{of je moet trouwen}\\

\haiku{{\textquoteleft}Iedereen heeft geen.}{tante Koos die de boel zoo}{lang voor je opbergt}\\

\haiku{Wat een volmaakte,,?}{stilte ijver en orde}{h\`e juffrouw Siebje}\\

\haiku{Oue mijnheer Baruut, de,.}{antiquair schuifel-sloft}{dicht langs de huizen}\\

\haiku{De verte ligt wit.}{onder een roode kerf in}{de hellende lucht}\\

\haiku{Vader's zagerig.}{praten hindert haar deze}{keer bovenmate}\\

\haiku{In de verte zal,.}{het licht zijn en vriendelijk}{en lente-achtig}\\

\haiku{Ze luistert er naar.}{zooals een mensch luistert die door}{slaap bevangen is}\\

\haiku{En dan staat ze daar.}{of ze beet gegrepen wordt}{en niet verder kan}\\

\haiku{dat zal je zien, dan.}{magge we vast allemaal}{kastanjes rapen}\\

\haiku{H\`e - rare,{\textquoteright} moppert,.}{Wina Levina maar ze}{is niet onvoldaan}\\

\haiku{De zetels van Buk.}{en Dibbe wrijft ze zonder}{geweldpleging af}\\

\haiku{De zonnestralen,.}{flitsen als fonteinen de}{boomkruinen branden}\\

\haiku{Hij bracht de mand met.}{witlof naar binnen en keek}{toevallig omhoog}\\

\haiku{De stemmen van de.}{jongetjes dreinen aan een}{opgeschoven raam}\\

\haiku{{\textquoteright} En Vader heft met.}{een moede verbazing de}{handen ten hemel}\\

\haiku{{\textquoteleft}Wagentje, waar moet?,,?}{je toch wezen stappen waar}{gaan jullie naar toe}\\

\haiku{Daar was ze toen nog,.}{verwonderd over nu begrijpt}{ze er alles van}\\

\haiku{En je bent te laat,.}{uit school gekomen dat moet}{niet meer gebeuren}\\

\haiku{{\textquoteright} Later zat ze als.}{een stijf klein pijn-frommeltje}{onder aan de trap}\\

\haiku{Eenmaal gaan ze er '.}{s zomers met hun allen}{een dagje naar toe}\\

\haiku{En ik zelf, tja, neem,.}{mijn nou ik bestee geen rooie}{duit \^an me kanis}\\

\haiku{{\textquoteleft}zag ik u net toen.}{u dat leegstaande huis van}{Brammers voorbijging}\\

\haiku{En hoe licht ken de,?}{pliksem niet inslaan door soo'n}{glasen dingetje}\\

\haiku{Pieta kijkt er met.}{kinderlijke oogen naar en}{loopt weifelend voort}\\

\haiku{{\textquoteleft}Je kan nou maar kort,{\textquoteright}, {\textquoteleft},.}{bij Jet blijven valt haar in}{een uurtje jammer}\\

\haiku{Hij trommelt in het.}{voorbijgaan hardhandig op}{Pieta's achterhoofd}\\

\haiku{Buiten de fletse.}{lichtkringen van de lantaarns}{loert verlatenheid}\\

\haiku{E\'en voor \'een gaan ze:}{het kabinetje met de}{witte beeldjes in}\\

\haiku{En ze glimlachen,,.}{nu nog alle vijf maar met}{puntige lippen}\\

\haiku{{\textquoteright} En de boombl\^aren.}{en de paden zijn zoo rood}{of er vuur in brandt}\\

\haiku{De onstuimige.}{dikke druppels kletteren}{als hagelkorrels}\\

\haiku{De regen wordt als.}{met volle nappen tegen}{de ruiten geplenst}\\

\haiku{{\textquoteright} Er staat dan toch ook.}{wel een stugge pijn-vouw}{om haar glimlach heen}\\

\haiku{als je hier vandaan.}{komt is het daar heelemaal}{een spook-salon}\\

\haiku{En opnieuw is het.}{Pieta of haar een spiegel}{voorgehouden wordt}\\

\haiku{De kinderen in:}{de weezenbank fleuren er}{heelemaal van op}\\

\haiku{ze hebben goddank,.}{iets te doen zij galmen zoo}{hard mogelijk mee}\\

\haiku{Wina Levina.}{is altijd het eerst aan het}{eind van een regel}\\

\haiku{Over de zinnen en.}{woorden die ze opvangt denkt}{Pieta nog wel na}\\

\haiku{{\textquoteright} Ordelijk stappen:}{ze weer terug door de bleeke}{koele zonneschijn}\\

\haiku{{\textquoteleft}Geloof jij dat het?}{Kerstkindje wat wonders wil}{doen als je het vraagt}\\

\haiku{Oh - ik.... die weeflamp in,{\textquoteright}, {\textquoteleft} -?}{de uitstalling stamelt ze}{de de prijs er van}\\

\haiku{Het is of ze van.}{haar hoofd tot haar voeten \'een}{vurige blos wordt}\\

\haiku{Bijna vinnig drukt.}{ze haar ronde zachte kin}{op haar mantelkraag}\\

\haiku{Hans Wietzel legt over.}{de leuning van zijn fauteuil}{zijn hand op haar arm}\\

\haiku{'s Nachts droomt ze dat.}{ze sterft en een vriesbloem op}{de vensterruit wordt}\\

\haiku{En Pieta pinkt en.}{spert de oogen open als een die}{plotseling ontwaakt}\\

\haiku{de rug tegen de....}{eene zij-leuning en de beenen}{over de andere}\\

\haiku{Hij vraagt het maar zoo,.}{luchtig-weg er is toch}{wat straks in zijn lach}\\

\haiku{Het wordt een gesprek.}{zonder overgangen en met}{vele gapingen}\\

\haiku{Maar Hans Wietzel en,,....}{Pieta die dicht naast elkaar}{zitten ontgaat dat}\\

\haiku{Hij neemt haar ook nog,.}{mee naar het opkamertje}{rechts van de winkel}\\

\haiku{Ik zal hem nog 's,,.}{van Vader vertellen op}{een keer het komt wel}\\

\haiku{Wat een kostbare,?}{dingen in die winkel bij}{Pa Baruut h\`e lieverd}\\

\haiku{{\textquoteleft}Ja, laten we dan,{\textquoteright}.}{maar alledrie schichtig spiedt}{ze naar een zaalhoek}\\

\haiku{Want Moeder maakt een.}{woest maai-gebaar dicht bij}{zijn dunne krullen}\\

\haiku{{\textquoteleft}Je zou ook nog een,,,....}{klok krijgen eendvogeltje}{eekhorentje muis}\\

\haiku{{\textquoteleft}Stijn de tobber zag,?}{grauw van zorg en wat zouen}{de joggies denken}\\

\haiku{En met een harde,:}{afwerende verbazing}{in haar stem polst ze}\\

\haiku{{\textquoteright} En dan proest ze het,.}{zelf uit omdat het zoo dom}{en onzinnig is}\\

\haiku{Hans Wietzel heeft het.}{haar op hun wandelingen}{meer dan eens verteld}\\

\haiku{Pieta luistert er,}{verrast naar en ze glimlacht}{er kinderlijk om}\\

\haiku{{\textquoteleft}Hans, lijkt het jou ook,?}{niet vreemd dat het pas vandaag}{was dat we trouwden}\\

\haiku{{\textquoteleft}Hier zijn geen menschen,,,!}{kind alleen de aarde de}{zon en de stilte}\\

\haiku{Op de terugreis,,:}{als ze weer naar de aarde}{toezakken tobt ze}\\

\haiku{En midden onder.}{de dienst scheurt de duimnaad van}{haar eene handschoen uit}\\

\haiku{Nu is ze alleen.}{maar Pieta Arsting uit het}{weeshuis en niets meer}\\

\haiku{Vannacht dacht ik  : -?}{nog zijn we nou ook langs langs}{ravijnen gegaan}\\

\haiku{Pieta kan nu niet.}{zoo dicht bij hem komen als}{ze wel zou willen}\\

\haiku{Ze wuift zoo vurig.}{dat ze haast twee bloempotten}{uit het venster stoot}\\

\haiku{Hier loop  ik, en -,.}{dit is mijn huis ons huis het}{huis van ons drietjes}\\

\haiku{Er hangen tulen.}{vleugels over haar mouwen en}{haar rug lijkt van gips}\\

\haiku{Bij ons,{\textquoteright} schertst hij, {\textquoteleft}is.}{de liefde vannacht om twaalf}{uur afgeloopen}\\

\haiku{{\textquoteright} Ze zakt op een stoel,,.}{neer de armen gekruist de}{kin op haar halskuil}\\

\haiku{{\textquoteright} {\textquoteleft}Ph-ph,{\textquoteright} doet Wina, {\textquoteleft}....}{Levina minachtendde}{Keizer van Sina}\\

\haiku{{\textquoteleft}Hans Wietzel vertelt -.}{me ook niks nooit de dingen}{waar het op aankomt}\\

\haiku{Hij heeft het naar zijn,.}{zin gehad vandaag dat is}{duidelijk genoeg}\\

\haiku{Breed-uit of het,.}{een dekje is legt hij het}{jurkje over haar schoot}\\

\haiku{{\textquoteright} Dan vindt zij ook aan,.}{de knop van een stoelleuning}{mevrouw's taschje}\\

\haiku{de beugel is blond,.}{en golverig de beurs rond}{en frambooskleurig}\\

\haiku{{\textquoteright}, ginnegapt Pieta, {\textquoteleft}!}{bedektelijkdat het mag}{van het reglement}\\

\haiku{{\textquoteleft}Dat is de oude,{\textquoteright}.}{smaak en ze krijgt nijpplooien}{aan de mondhoeken}\\

\haiku{{\textquoteleft}Mal dat ik die muts, - -.}{maak mal mal nooit is er}{iets zotters vertoond}\\

\haiku{{\textquoteright} Een slaperige.}{bejaarde vredigheid hangt}{in de huiskamer}\\

\haiku{{\textquoteright} Ze glimlacht maar met,,}{een stukje lip en trekt de}{wenkbrauwen hoog op}\\

\haiku{{\textquoteleft}I am fond of you,,,?}{Jackie but of course you}{knew that did not you}\\

\haiku{- wat ben ik toch ook,.}{al ontwikkeld dat ik het}{allemaal zoo weet}\\

\haiku{Och stoeteltje, het,!}{gaat niet om die clubs het gaat}{om de connecties}\\

\haiku{Ik heb maar gedaan,,....}{of ik niks merkte ik wou}{me goed houen ik}\\

\haiku{Je goed houen,{\textquoteright} flitst, {\textquoteleft},....{\textquoteright}}{het nog door haar heenom Pa}{om die oue stakker}\\

\haiku{Zou je mij herkend,?,,?}{hebben Geer d\'adelijk op}{het eerste gezicht}\\

\haiku{Dan - dan liever  ,,,.}{niets dit was voor een ander}{nee dan liever niets}\\

\haiku{{\textquoteleft}Kan  best wezen,,,,?}{Geer maar hier duld ik het niet}{hi\'er niet begrepen}\\

\haiku{, Pa is er alleen,,.}{maar we hebben stamppot met}{worst Hollandsche kost}\\

\haiku{Het moet zoover komen,.}{dat hij op de knie\"en om}{vergiffenis vraagt}\\

\haiku{Je zei dingen die.}{ik begreep en waarover ik}{nog niet praten kon}\\

\haiku{En het tweede hart,....}{klopt het tweede hart leeft vlak}{bij haar eigen hart}\\

\haiku{te vluchten naar - naar,?}{andere menschen naar een}{plek waar het stil is}\\

\haiku{wie er is, tuimelt.}{Pa bijna van het rechte}{smalle trapje af}\\

\haiku{{\textquoteright} Links en rechts kijkt hij,}{de ruischende stegen}{en zijstraatjes in}\\

\haiku{Hij wil zijn arm van.}{haar wegnemen en zij grijpt}{angstig zijn hand beet}\\

\subsection{Uit: In de witte stilte}

\haiku{Zorgvuldig borg ze,.}{het emmertje op in de}{koele keukenkast}\\

\haiku{{\textquoteleft}As 't nou daomee,{\textquoteright}, {\textquoteleft}.}{ophold nam Barta zich voor}{z\^a-'k gras snieden}\\

\haiku{Maor alles mut, '.}{zoo komm'nt steet in de}{Openbaoringen}\\

\haiku{We zatten mit de.}{kop boven op die olde}{praom van Vok Krieger}\\

\haiku{{\textquoteleft}'s Aovens te veuren ', '...}{had ik hem nog watbrochts}{morgens lag hij stief}\\

\haiku{{\textquoteright} Met een slip  van.}{haar doek friemelde ze over}{haar natte wangen}\\

\haiku{{\textquoteleft}Domeneer kon wel,.}{slechter want dat zal dan bie}{Riek Stoffers wezen}\\

\haiku{wat gij de minste,?}{van Mijn broederen doet dat}{hebt ge Mij gedaan}\\

\haiku{Ze ging dicht langs een,...}{diepe zwarte sloot en bleef}{er telkens bij stil}\\

\haiku{Maar Barta ving er,.}{niets van op ze had geen oog}{van de tafel af}\\

\haiku{, en in Dominee's.}{envelop was een briefje}{van tien gesloten}\\

\haiku{Wel verdulleme,{\textquoteright}, {\textquoteleft} ' '...}{kwam hij losis mie datn}{toon enn taol}\\

\haiku{{\textquoteright} {\textquoteleft}Och,{\textquoteright} wou Jentien dan, {\textquoteleft}?}{nog bangelijk uitstellen}{wacht tot morgenvruug}\\

\haiku{{\textquoteleft}Oe Grovaoder,}{en Gromoeder kan ie toch}{wel alles zeggen}\\

\haiku{{\textquoteleft}Veurzichtig vrouwe,{\textquoteright}, {\textquoteleft},...}{maande Reulefpas toch op}{laot mi\'en liever}\\

\haiku{Toen  hij bij de,.}{deur was deed hij of hem nog}{wat te binnen viel}\\

\haiku{We hebt altied nog ',!}{n rekening met oe te}{vereffen'n Reulef}\\

\subsection{Uit: Liefde}

\haiku{{\textquoteleft}Goed zoo{\textquoteright}, maar hij zoent.}{haar en zijn zoen glijdt als een}{veertje langs haar wang}\\

\haiku{Een kabouter staat:}{daar en draagt een muts met een}{belletje en zegt}\\

\haiku{Ze springt ergens over.}{heen en ligt weer op haar bed}{in het kamertje}\\

\haiku{En Lied moet hard op,.}{haar pink bijten want ze wil}{er naar luisteren}\\

\haiku{, en Lied zit bij haar,.}{Vader in de bank tusschen}{twee groote menschen in}\\

\haiku{Hanne neemt haar op.}{de arm en laat haar in het}{spiegeltje kijken}\\

\haiku{Ze kijkt ook door de.}{open deur in de kamer en}{er is daar geen vrouw}\\

\haiku{Hij draagt haar H{\'\i}j vindt.}{haar \'ook niet te groot en te}{zwaar om te dragen}\\

\haiku{Maar er moet nog wat -,,.}{bij een cijfer ze weet het}{haast alleen maar haast}\\

\haiku{Hij doet mal zijn hoofd,.}{heen en weer hij doet mal zijn}{schouders heen en weer}\\

\haiku{{\textquoteright} En ze mag zoo maar.}{met haar schoenen en al op}{Moeder's schoot zitten}\\

\haiku{Het paadje is wit, -.}{en glad ze glijen Lied wint}{het van Moeder}\\

\haiku{En Vader loopt ook,!}{om het hardst met Moeder maar}{Moeder wint het niet}\\

\haiku{En Vader pakt haar.}{op en zoent haar Hand in hand}{gaan ze het huis in}\\

\haiku{En ze heeft zwarte:}{handen en er zit stof in}{haar haar en ze zegt}\\

\haiku{De takkenbezem,.}{staat in een andere hoek}{en het stinkt naar roet}\\

\haiku{En Oom Luuk kijkt nog,}{altijd zoo erg en het is}{akelig dat hij zoo}\\

\haiku{De klaprozen hier,,.}{kent ze ook en ze kent ook}{de korenbloemen}\\

\haiku{En geen mensch zegt er,}{wat er is ook geen mensch}{om wat te zeggen}\\

\haiku{{\textquoteright} En ze pakt haar knieen {\textquoteleft}{\textquoteright},.}{beet en haar bak met boonen}{Mien kiend zegt Hanne}\\

\haiku{Ze schuift dichter bij,.}{Hanne ze legt haar armen}{op Hanne's  schoot}\\

\haiku{Een beest ritselt in,!}{de bladeren een eekhoorn}{misschien of een rat}\\

\haiku{Ze zitten achter.}{hun borden en er is al}{een beetje ruzie}\\

\haiku{Gister was ze toch,,.}{nog maar een klein kind vandaag}{niet vandaag niet meer}\\

\haiku{haartjes heeft ze aan.}{haar oogen en lange gouen}{haren om haar hoofd}\\

\haiku{Het prikt zoo in haar,,.}{oogen het prikt heel erg ze huilt}{toch alleen maar droog}\\

\haiku{Een leege zwarte boom.}{kijkt boos door de ruiten naar}{het kleine rooie vuur}\\

\haiku{Ze staat stil en loopt,.}{zoetjes verder ze loopt of}{ze in de kerk is}\\

\haiku{Vader moet hooren -?}{hoe goed ze zingen kan Hoort}{hij het wel Vader}\\

\haiku{Tante Belin windt.}{garen op de spoelen van}{de naaimachine}\\

\haiku{Ze kan dan nog maar,.}{met \'een oog kijken en het}{is toch wel genoeg}\\

\haiku{En als ze bij de,.}{paal met de duiventil is}{roept Hanne nog wat}\\

\haiku{Ze gaan die zwarte}{poort door en ze komen in}{een donkere gang}\\

\haiku{{\textquoteleft}Ja!, en dan m\'oet {\`\i}k!}{precies op tijd thuis wezen}{om jou open te doen}\\

\haiku{Lied loopt haar na in.}{de gang en de voordeur slaat}{vlak voor haar neus dicht}\\

\haiku{En Lied luistert naar.}{de wielen van die wagen}{tot ze niets meer hoort}\\

\haiku{Dan kijkt ze om in,.}{de gang en alles is nog}{kouer en stiller}\\

\haiku{En dan eet ze haar}{twee sneden bruin brood op en}{kijkt zoo naar Vader}\\

\haiku{Dat is niet tegen,.}{de reus Goliath dat is}{tegen Oom Reinhold}\\

\haiku{Ze houdt de klink van,.}{de deur stijf vast die klink is}{de klink van h\'aar huis}\\

\haiku{{\textquoteleft}Ik heb geen-eens een,,{\textquoteright}}{mantel \^an Moeder ik heb}{geen-eens me mantel}\\

\haiku{{\textquoteleft}W\`at Oom Reinhold?, w{\'\i}j,!}{hebben geen O\'om R\'einhold in}{de familie hoor}\\

\haiku{En ze luistert naar.}{zijn stappen tot het tikken}{zijn en nog langer}\\

\haiku{{\textquoteright} En ze kijkt naar de.}{bikkel en ziet niet goed dat}{het een bikkel is}\\

\haiku{Het is nacht en het,}{vlammetje van  de kaars}{trilt of het bang is}\\

\haiku{En ze trekt Lied haar.}{Zondagsche jurk aan en haar}{Zondagsche mantel}\\

\haiku{Dat zei hij toe', me - -.}{Vader me Vader die keek}{naar mij k\'eek naar mij}\\

\haiku{Ze zet boos-hard een}{steenen ketel met wijn op de}{kachel en boos-hard}\\

\haiku{En Oom's eene wang moet.}{er nog meer van glimmen dat}{Tante hem zoo prijst}\\

\haiku{En de lieve Heer.}{uit Anderst die is niet in}{de Noorder-kerk}\\

\haiku{Dat komt omdat ze.}{naar de schotsche juffrouw keek}{en niet naar de straat}\\

\haiku{De schotsche juffrouw:}{die knipoogde tegen Oom}{Louis  en ze riep}\\

\haiku{Vreemd zijn groote menschen.}{Lied kijkt naar Oom en Anne}{of ze haast niet durft}\\

\haiku{En ze hoort meteen.}{dat Oom de klant uitlaat en}{daar is ze blij om}\\

\haiku{En Lied blijft er haast {\textquoteleft},{\textquoteright},}{weer van stil staanKom schiet op}{zegt Oom achterom}\\

\haiku{En de doosjes met...{\textquoteright}}{vloeibare schoensmeer en de}{zak-spiegeltjes}\\

\haiku{Ze lacht meteen en.}{ze duwt het poesebontje}{nog meer achteruit}\\

\haiku{of hij het lekker}{vindt om Tante Alwine}{een arm te geven}\\

\haiku{Verleeen jaar was me, -...}{Vader er nog me Vader}{en en me Moeder}\\

\haiku{En in het bed zit.}{ze met opgetrokken knieen}{en wacht op Tante}\\

\haiku{{\textquoteright} Maar van allerlei.}{knal-geluiden in de}{straat wordt ze wakker}\\

\haiku{en langzaam-aan wordt -}{ze koud de zaal is warm en}{ze heeft heete thee}\\

\haiku{Haar hand gaat open en.}{dicht z\'oo of ze de tip van}{een jurk stijf vastgrijpt}\\

\haiku{Hard drukt ze haar knieen.}{tegen de onderkant van}{het bankvakje aan}\\

\haiku{{\textquoteleft}Louis, schei nou uit met,.}{dat eeuwige fluiten van}{je ik heb hoofdpijn}\\

\haiku{Ni\'et fluiten zeg,,}{ik en ni\'et rooken hier}{ik kan het nou niet}\\

\haiku{{\textquoteright} En ze praat dan over}{een heele hoop dingen die}{Lied niet vatten kan}\\

\haiku{En hij doet gauw die.}{brief open en leest alles en}{hij wordt zoo alleen}\\

\haiku{Het is dan toch of, -}{Vader zijn hoofd over haar schudt}{daar in de verte}\\

\haiku{En ze vergeet haast,}{de kruisbessen op te eten}{die ze krijgt Bekkie}\\

\haiku{En ze is  boos {\textquoteleft}{\textquoteright},, {\textquoteleft}!}{Is me dat uitblijven zegt}{Tanteeen schande}\\

\haiku{{\textquoteright} Lied hoeft dan enkel,.}{maar haar kin op haar borst te}{drukken anders niet}\\

\haiku{En na het standje.}{geeft Tante haar toch ook nog}{een blaadje papier}\\

\haiku{half-luid leest.}{ze hem en luid-op En dan}{kijkt ze Tante aan}\\

\haiku{Het is nou net of '.}{Vader nogs weer bij haar}{vandaan gegaan is}\\

\haiku{Er is met Kerst geen,.}{kaart gekomen van Vader}{geen brief en geen kaart}\\

\haiku{Ze knikt als Oom dan, {\textquoteleft}.}{zegt wat het is en het haar}{laat lezenJa Oom}\\

\haiku{En dan ineens ziet.}{ze ook weer een heele hoop}{van Oom en Tante}\\

\haiku{hier is vast een beuk.}{en die daar dat zal wel een}{eikenboom wezen}\\

\haiku{{\textquoteright} En zijn gezicht is.}{z\'oo vroolijk dat Tante er}{mopperig van wordt}\\

\haiku{En dan vergeet ze}{heelemaal dat Lied ook nog}{in de kamer is}\\

\haiku{Hij maakt een dikke {\textquoteleft}}{kin op zijn boord en zegt een}{paar vreemde woorden}\\

\haiku{{\textquoteright}, zegt Oom, hij neemt zijn,.}{hoed in zijn hand net of hij}{het op\'eens te warm krijgt}\\

\haiku{N\'ee Liedia Ulen, zei...{\textquoteright}}{Oom En Sjeuke die moest haar}{toen wel loslaten}\\

\haiku{Bekkie had eerst ook, {\textquoteleft}}{geglimlacht en later was}{ze giftig geweest}\\

\haiku{En hoe lang kan het,...?}{dan nog duren eer het zoo}{gaat als bij mij thuis}\\

\haiku{{\textquoteleft}Dat merk ik dan toch{\textquoteright}.}{allemaal Stil loopt ze wat}{later door de gang}\\

\haiku{Ze sjokt een beetje,.}{de tasch met boeken wordt al}{zwaarder in haar arm}\\

\haiku{{\textquoteright} Oom Louis en Tante.}{Alwine zijn nog niet thuis}{als ze terugkomt}\\

\haiku{Het donker wacht haar.}{zwart en geheimzinnig op}{in de bovengang}\\

\haiku{En ze doet of ze,...}{de krant inkijkt slaat nog een}{reken-schrift open}\\

\haiku{Ze draait zoo maar wat,.}{rond tuurt naar de oue gele}{prentjes aan de wand}\\

\haiku{Als het warmer wordt,}{heeft ze ook weer veel aan te}{merken op de stad}\\

\haiku{De huizen leunen}{met grijze avondmuren in}{de maanschemer weg}\\

\haiku{Er valt een beetje,,,...}{maanlicht over heen dat maanlicht}{leeft het ademt het trilt}\\

\haiku{Vader neuriet een,.}{oud kerk-vers een vers uit}{de tijd van Anderst}\\

\haiku{Nou stil maar{\textquoteright}, prevelt.}{ze Haar slappe rug trekt nog}{een beetje krommer}\\

\haiku{Als je jong  kijkt,.}{dan ben je toch al op weg}{om oud te worden}\\

\haiku{En op een avond komt.}{er ook weer een groote mand met}{cineraria's}\\

\haiku{Hij denkt er h\'eel wat{\textquoteright}.}{van En Lied weet niet wat ze}{daarop zeggen moet}\\

\haiku{En we moeten nou.}{ook n\`og later thuis  zien}{te komen dan hij}\\

\haiku{Geen firma-adres er,.}{op voor eventueele navraag}{en dit kaartje er bij}\\

\haiku{Ze moet toch altijd,}{wel haar handen uitstrekken}{naar de lieve Heer}\\

\haiku{Franske zal misschien.}{al de een of andere}{betrekking hebben}\\

\haiku{{\textquoteright}, vraagt Oom, {\textquoteleft}en zullen '?}{we vanavond alle dries}{naar de bioscoop gaan}\\

\haiku{Ze hoeft ook niet meer -.}{de stad in te gaan om Oom}{jaloersch te maken}\\

\haiku{En nou ben ik toch,.}{benieuwd wat voor menschen er}{komen opdagen}\\

\haiku{Bij een winkelraam}{staat ze stil en tuurt naar een}{pot met anemonen}\\

\haiku{{\textquoteleft}Laat de meid 's een,{\textquoteright}}{kop extra-sterke koffie}{brengen Alwine}\\

\haiku{Je hebt me Vader?}{ook gekend Bedoelde je}{daarstraks me Moeder}\\

\haiku{Er moeten toch een,.}{hoop dingen zijn waar z{\'\i}j ook}{over uit praten wil}\\

\haiku{Het is dan ook of:}{ze alles van zichzelf naar}{God kan opheffen}\\

\haiku{{\textquoteright} En ze moet meteen}{diep blozen en ze gaat gauw}{een stap achteruit}\\

\haiku{{\textquoteright}, denkt Lied, {\textquoteleft}oh - h\'emel...{\textquoteright},:}{Als Prosper dan weer naar zijn werk}{toe is zegt Tante}\\

\haiku{Het is de moeite...}{niet waard om  er nog \'een}{keer naar te kijken}\\

\haiku{{\textquoteleft}Het is toevallig{\textquoteright},, {\textquoteleft}.}{zeggen ze nu ook weermaar}{het komt net zoo uit}\\

\haiku{Hij stompt nu ook met.}{een stok op het ledikant}{Lied wil al opstaan}\\

\haiku{Och, zoo alleen{\textquoteright}, zucht {\textquoteleft},?}{zeHeeft Tante wat te doen}{dat {\`\i}k niet mag zien}\\

\haiku{Ik ben toch altijd,.}{te vroeg aan het raam om naar}{je uit te kijken}\\

\haiku{En Prosper lacht er om,.}{maar Valentijn Brunt lacht niet}{Valentijn Brunt zegt}\\

\haiku{Daar schrikt ze nou weer.}{erg van Ze fronst en ze wil}{er niet op ingaan}\\

\haiku{Op de hoek steekt hij}{zijn hand op Lied wil wuiven}{en hij is al weg}\\

\haiku{{\textquoteright} Hij glimlacht of hij, {\textquoteleft}{\textquoteright},}{voor de spiegel staat hij lonkt}{evenSchat zegt hij lauw}\\

\haiku{{\textquoteleft}Z\`eg nou dat hij best,,...{\textquoteright}}{een wandeling kan doen als}{hij wil z\`eg het toch}\\

\haiku{Ze is nu wel zoover.}{dat ze altijd goed weet waar}{ze mee bezig is}\\

\haiku{Eerst trekt hij zijn jas,,.}{uit eerst zet hij zijn hoed af}{dan komt hij binnen}\\

\haiku{Ik schrok...{\textquoteright}, en dan is.}{er iets in haar gezicht of}{ze gestoken wordt}\\

\haiku{Maar Valentijn Brunt...}{kijkt naar hem om of hij een}{hekel aan hem heeft}\\

\haiku{{\textquoteright} Zijn oogen worden klein {\textquoteleft}?,?,?}{van verteederingSmaakt het}{ja b\`en ik wel lief}\\

\haiku{Ze zegt ook {\textquoteleft}Ieder -?}{mensch heeft toch wel recht op een}{beetje geluk niet}\\

\haiku{Oom Louis...{\textquoteright} Maar ze wil {\textquoteleft}}{toch wel goedig-instemmend}{tegen Prosper knikken}\\

\haiku{{\textquoteleft}Als ik 's uit mijn, {\textquotedblleft}}{humeur durfde te wezen}{dan zei Oma Krunzel}\\

\haiku{naar die eenzame:}{buitengesloten avond en}{ze zegt in zichzelf}\\

\haiku{Maar ze strijkt toch even,.}{goed over haar haar ze trekt toch}{haar halskraagje recht}\\

\haiku{Het is of hij iets, {\textquoteleft}}{zegt waar ze eigenlijk niet}{naar luisteren moest}\\

\haiku{{\textquoteright}, soest ze Nu hangt die}{rare slaapnevel weer over}{haar gedachten heen}\\

\haiku{Het is een heele}{tijd stil Maar dat vinden ze}{allebei wel goed}\\

\haiku{En als ze dan bij,.}{hem is kan ze enkel maar}{naar hem luisteren}\\

\haiku{Want die moet ik dan,}{toch wel hebben het is nog}{een heele afstand}\\

\haiku{Valentijn Brunt komt,.}{en is stil en kijkt als uit}{de verte naar Lied}\\

\haiku{{\textquoteright} En ze is er een,}{beetje jaloersch op omdat}{ze zelf zoo warm is}\\

\haiku{En ze zakt zoo wit.}{en slap achterover of ze}{bewusteloos wordt}\\

\haiku{Hij gooit zich ook weer -.}{woedend om en om in zijn}{bed het kind krijscht}\\

\haiku{En hij zal vast nog{\textquoteright}.}{wel ergens een poos blijven}{zitten Elk woord schrijnt}\\

\haiku{Ze wil in haar lip}{bijten en laat dat. D\'aar geeft}{ze geen antwoord op}\\

\haiku{Misschien herinner.}{jij je er nog wel meer van}{dan ik weten kan}\\

\haiku{Je Moeder verviel,}{van kwaad tot erger toen ze}{al te ver heen was}\\

\haiku{{\textquoteleft}Al wat u vroeger,,?}{tegen Oom Louis zei dat gaf}{toch ook niks Tante}\\

\haiku{Je ziet er uit of.}{er nog maar de helft van je}{overgebleven is}\\

\haiku{die heeft me gezegd,}{wat er met Moeder is en}{hoe het daar ginder}\\

\haiku{Er is al lang een,.}{andere facturiste}{een knap Jodenkind}\\

\haiku{{\textquoteleft}Als mijn gezicht hem,...?}{obstinaat maakt dan moet ik}{toch anders kijken}\\

\haiku{Het is net of ik -?}{een heele poos weg geweest}{ben waar was ik dan}\\

\haiku{Hij keert zijn hoofdje,.}{naar zijn Moeder toe en hij}{kijkt ook lang naar haar}\\

\haiku{Maar als ze aan haar,.}{Moeder denkt krijgt ze zelfs een}{blos in haar voorhoofd}\\

\haiku{Z\'oo keek ik ook naar?}{Moeder en Zwisters Onthoud}{h{\'\i}j dit nou ook al}\\

\haiku{Je hebt er toch niets?}{op tegen dat ik naar die}{voetbal-match ga}\\

\haiku{Vandaag ziet ze er,,.}{knap uit zoo uitgerust haar}{oogen zijn zoo helder}\\

\haiku{En ze lacht met leege}{lippen tegen hem en brengt}{hem naar zijn bedje}\\

\haiku{Hij legt zijn schaap op,,.}{haar schoot zijn leege doosjes zijn}{leege garenklossen}\\

\haiku{Hij komt ineens op,,}{haar toe hij legt zijn handen}{zwaar op haar schouders}\\

\haiku{{\textquoteleft}Singe{\textquoteright} En ze zingt:}{van de drie Koningen met}{de ster En ze zingt}\\

\haiku{Maar ze glimlacht in, {\textquoteleft}}{zichzelf ze glimlacht over een}{pijn heen en ze zegt}\\

\subsection{Uit: Menschen uit een stil stadje}

\haiku{of die niet genog '.}{had an zoo'n presentje uit}{t sterfhuis as um}\\

\haiku{Verschrikt soesde ie...}{dat uit in de broeiwarmte}{van z'n hooge bedstee}\\

\haiku{een in 't donker ', ', '...}{z'n weg zocht naarm metn}{dreigingn vast plan}\\

\haiku{hoog-overbuischte...}{zeiltjes zaten d'r zonder}{schutvan-schaduw}\\

\haiku{n geld dat 't vr\`at,.}{en alles zonder baat geen}{ziertje beterschap}\\

\haiku{Hij schudde bedrukt...}{z'n ou\"e kop om de flarden}{net aan de steilen}\\

\haiku{{\textquoteleft}Kijk me dat 'r 's,...}{an haast heelemaal d'r uit}{en geen visch vanzelf}\\

\haiku{{\textquoteright} En dan ineens om ',.}{t nare geprik achter}{z'n oogen rauw lachend}\\

\haiku{{\textquoteright} {\textquoteleft}Tater..,{\textquoteright} rolde Sien's ', '.}{wulpsche stemr door hard van}{n tartende lach}\\

\haiku{'n Sikkie had ie, '.}{enkel en niks asn slap}{geite-sikkie}\\

\haiku{Geen flauw benul van.}{z'n fijn-proeverige}{artistiekiteit}\\

\haiku{{\textquoteleft}Ik poes 'm daaijeluk, ',.}{maar ik zeg je nog d'rs}{l\`ak an je nachtwacht}\\

\haiku{{\textquoteleft}En de eiere, ' '...?}{benne ook duur wou je nog}{rs zegge niet}\\

\haiku{Wat zel dat weer 'n,!}{kopere luchie worre weer}{gloeie och heer de oue}\\

\haiku{De kerel loopt met.}{veters en stukkies stinkzeep}{en pakkies naalde}\\

\haiku{Jij ben toch niet die?}{Meheer die bij me zuster}{heb weze vrage}\\

\haiku{Wil je 'n lekker,?}{schoteltje rolpens van me}{keurige rolpens}\\

\haiku{{\textquoteright} Deurtjes kraakten open '...}{enn zacht gesmoezel van}{vraaggeluid stak op}\\

\haiku{{\textquoteright} {\textquoteleft}O-o-o... me doosie{\textquoteright},...}{met mussies snotterde Gon}{want d'r was niet veel}\\

\haiku{- Kee lachte schorrig, '...}{uit in de stilten w\'el}{erg bitter lachie}\\

\haiku{D'r m\`ot toch wat op.}{te vinde weze om d'r}{achter te komme}\\

\haiku{Geen mensch doet 'r mooi.}{achter de scherreme en}{bij z'n eigen thuis}\\

\haiku{{\textquoteleft}D'r buite voor 't}{raam gluurde ferachtegies}{Oome Dries na d'r}\\

\haiku{En raak gepakt is......... '}{dat molentje d'r op dat}{hoogtentje zoo \"e}\\

\haiku{Maar de glimlach van. '.}{de zon was toch andersn}{Vecansieochtend}\\

\haiku{{\textquoteleft}Ja... stil nou 's Hein, '.}{je benne weer heelemaal}{doort dolle heen}\\

\haiku{Maar nou ineens wier ',... '...{\textquoteleft}}{t treurig lam treurign}{wijsie dat ze kon}\\

\haiku{{\textquoteright} {\textquoteleft}Ja-a{\textquoteright}, koersten Ceesie'.}{Randers plezierige oogies}{Fenne's richting uit}\\

\haiku{{\textquoteright} Peet's stem verkromp tot '...}{n steunend geluid en z'n}{gezicht trok valer}\\

\haiku{zoo van alles nog '...}{s opgehaald en ik d'r}{nog heelemaal in}\\

\haiku{Maar nou die wissel...,...?}{driehonderd pop waar most dat}{nou vandaan kome}\\

\haiku{Dat had d'r toe al... '}{dadeluk de heetigheid}{in d'r bloed gejaagd}\\

\haiku{{\textquoteright} - Had ze wezenluk '.}{niet eens geweten datn}{mensch dat ook nog had}\\

\haiku{Ze had d'r nou toch ',...}{ook al zoo dikwuls gevraagd}{opn koppie Jans}\\

\haiku{K\'o\'ope mensch, b\`e-je, ' '.}{niet wijsn cedeautje}{vann goeie kennis}\\

\haiku{En dan had ie ook ' '.}{nog wat gezeid vann ring}{inn varkenssnuit}\\

\haiku{wat beware voor ' '.}{n \`ander en wekenlang}{zelf opn drogie}\\

\haiku{later zalle de:}{moeders an d'r kindere}{kenne vertelle}\\

\haiku{och heer, allemaal...{\textquoteright}}{hebbe ze d'r bloei en d'r}{pronk en jij het niks}\\

\haiku{aj-je maar berouw, ',}{het Koos d'r komtt maar op}{an berouw mot je}\\

\haiku{Mooie oogen had ie of ',...}{de lieve Heer jer deur}{ankeek zukke oogen}\\

\haiku{as je ja meene, '.}{je hadt zeker met je}{zenuwpies te kwaad}\\

\haiku{Koert Huibers bij de ',...}{koffie en v\'o\'ort slape}{Koert voor en Koert na}\\

\haiku{{\textquoteleft}O mensch hou op, ik...}{doen d'r haast wat in me broek}{van \'o ha-ha-ha}\\

\haiku{gosterdankie, di\'e, '...?}{bij oome Dries op bed zou}{j{\'\i}jm dat gunne}\\

\haiku{Overal kwam ie 'r,}{dan mee te hulp z\'o\'o dat je}{je eige voelde}\\

\haiku{as vrouw zijnde voor ',,}{n heelemaal nakende}{buurman z{\'\i}j niet hoor}\\

\haiku{En dan die met die,.}{pulk haar op z'n knikker d\^a's}{dan vezelf Simson}\\

\haiku{{\textquoteright} Dries staarde in z'n.}{peinzen naar de meiden d'r}{vlugge werkdoening}\\

\haiku{Elk 'n riks voor de...{\textquoteright},.}{kermis schoof oome Dries die}{naar d'r-lui toe}\\

\haiku{Z'n voet tastte dan ',.}{onmiddelijk naarn steen}{maar ie bedacht zich}\\

\haiku{En onder de lamp '...}{s avens zullie saampies}{an d'r boterham}\\

\haiku{As 'k 'n vrouw over, '.}{de vloer neem mott me vrouw}{maar meteen weze}\\

\haiku{mollig ding, doen je ', '?}{t muisie doen jet dan}{venavend hee}\\

\haiku{h\'a\'ast...{\textquoteright} Als ze op z'n, '.}{deelneming wachtte had ze}{n teleurstelling}\\

\haiku{'k Vind Stiena n\'a\'ar..., '.}{o n{\`\i}ks geen moeite jongen}{k lees gr\'a\'ag luid op}\\

\haiku{'t Lischgoud aan de.}{slootjeskant wiegelde met}{vredig geruisch}\\

\haiku{of 'n reppende,...}{vlieg met overal de frischheid}{en maagdelukheid}\\

\haiku{verspille wil... nou,...?}{h\'o\'orde ie d'r altemet}{niet z\`at genoeg van}\\

\haiku{Als 'n man met 'n '.}{somber gezicht stond de lucht}{bovent water}\\

\haiku{Even door de strakke.}{gespannenheid van z'n ou\"e}{kop lichtte glimlach}\\

\haiku{zwaar boemde dan weer, '.}{met barstig gerinkel van}{ruitjest raam neer}\\

\haiku{Bij 't venster met ',.}{t uitkijkie op straat bleef}{ze omtreuzelen}\\

\haiku{{\textquoteleft}Wat hej-je toch,?}{allemaal van zilvere}{tressies malle meid}\\

\haiku{{\textquoteleft}ie was ommers bij...?}{de k\`olledokter en hier}{in z'n spr\'e\'ekkamer}\\

\haiku{{\textquoteleft}'k Gaan na me dood,{\textquoteright},}{krimmeneelde Toon zat ie}{te snotwolleve}\\

\haiku{{\textquoteleft}'k Zel ze effe,{\textquoteright}.}{bijlichte met de lange}{zes zei Klaas Krone}\\

\haiku{Nog voor d'r eigen '?}{n schijn van geluk ophieuw}{of voor de mensche}\\

\haiku{Ja z\'o\'o noeme de ',...}{menschet maar di\'e denke}{zoo'n beetje over d\`at}\\

\haiku{Nou, en Jans had d'r ' '}{welrs van te vore}{verteld dat ie zoo}\\

\haiku{'t Wordt je ook zoo ',?}{voorgehou\"e bijt trouwe}{dat weet je ommers}\\

\haiku{Te vol{\textquoteright} had Em d'r ' ' {\textquoteleft}...}{s gezeid en later nog}{d'rsVeel te bont}\\

\haiku{{\textquoteright} Lui stood ie op van,.}{de  divan rekkerig}{in z'n beklemming}\\

\haiku{Ach heer ja... zoo is '... ' '....}{t ent is je asn}{last opleid trouwe}\\

\haiku{{\textquoteright} Trui's denken hortte '.}{daar enn afschuw besloop}{d'r om wat ze dacht}\\

\haiku{{\textquoteleft}Ze had maar aldeur '...}{zoo'n rare droogte in d'r}{keel enn zeerte}\\

\haiku{nou de boel zoo de... '...}{hoogte in gaat wat g\'ale en}{n tros menielje}\\

\haiku{niks geen wijfie om, '...}{d'r's te knuffele om d'r}{s gr\'a\'ag te zoene}\\

\haiku{{\textquoteleft}'k Vin Gon ook niks,...}{voor j\'o\'u z{\'\i}j zoo'n sukkeltje}{en jij zoo'n flinke}\\

\haiku{Je bent gezond en...}{flink en je bent in staat je}{brood te verdiene}\\

\haiku{Hij neep z'n handen.}{tot logge vuisten en z'n}{stoel kraakte verdacht}\\

\haiku{D'r uit met 'n niet,,,'.}{ach ja je mot maar denke}{t gebeurt meer}\\

\subsection{Uit: Een menschenhart}

\haiku{Zijn Vader houdt hem.}{juist zoo vast dat hij Gabe's}{kant moet uitkijken}\\

\haiku{Obbe die ligt daar -.}{als een vertrapt dier in de}{modder en huilt huilt}\\

\haiku{{\textquoteleft}O{\textquoteright}, zei Roelien, {\textquoteleft}dat,.}{is uit pure wangunst dat}{de jongens dat doen}\\

\haiku{En koster van Tijn.}{rookt een neuswarmertje en}{klopt oue boeken uit}\\

\haiku{In een kapotte.}{teenen wieg bij haar ligt een dik}{bol zuigkind dat slaapt}\\

\haiku{Hij moet pinken en.}{hij wordt rood en hij kijkt een}{andere kant uit}\\

\haiku{Het orgel jammert.}{en het zingen jammert en}{de preek jammert ook}\\

\haiku{Iets in Gabe wil,...}{er bij zijn iets anders in}{hem wil ook weer niet}\\

\haiku{Gabe die mag soms ',.}{wel in de zaal komens}{avonds en soms ook niet}\\

\haiku{Met een zucht kijkt hij.}{dan maar weer naar de mannen}{om in het caf\'e}\\

\haiku{{\textquoteleft}Als ik jou niet had,,?}{lieverd dan was het ommers}{niks met me gedaan}\\

\haiku{Er zijn ook stukjes,:}{bosch bij sparren en dennen}{en eindjes bongerd}\\

\haiku{Heertje springt overeind.}{en hij schiet het huis in en}{komt ni\'et terug}\\

\haiku{Hij komt die dag nog.}{heel te Tenderloo en in}{het Hunteler bosch}\\

\haiku{En de haan van de,.}{Oosterkerk dat is een mooi}{endje bliksemlicht}\\

\haiku{{\textquoteright}, Gabe lacht met zijn.}{mond wijd-open en zijn oogen}{haast heelemaal toe}\\

\haiku{En wat ze op de,:}{boodschappen toe krijgen dat}{eten ze samen op}\\

\haiku{Er rijen wagens, -}{af en aan er gaan jongens}{voorbij op fietsen}\\

\haiku{Hij kijkt van opzij,.}{naar Aaike en ze loopen}{samen weer verder}\\

\haiku{Hij zet zijn hoedje,.}{op om het nog een keer af}{te kunnen nemen}\\

\haiku{Het maakt hem ineens,.}{klam en onrustig en een}{beetje kregel ook}\\

\haiku{Ze knijpt hem haast, ze,.}{kneedt hem zoo'n beetje ze wrijft}{hard over zijn armen}\\

\haiku{{\textquoteright} Hij schuift de ring van,.}{zijn centenbeurs en haalt er}{een dubbeltje uit}\\

\haiku{Ik had haar te kort.}{terug gegeven op de}{anijs die ze haalde}\\

\haiku{{\textquoteleft}Wat een kerelsgek{\textquoteright},, {\textquoteleft}.}{vit Gabeze geeft om}{iedere kerel}\\

\haiku{'s zomers hard werk.}{in de turf en in Holland}{maaien en hooien}\\

\haiku{Ze ruiken aan de,.}{sneeuw op een muur-richel ze}{likken er ook aan}\\

\haiku{Zware beenen krijgen.}{ze ineens en ze worden}{bloedrood allebei}\\

\haiku{{\textquoteleft}Ik had jou toch n\'ooit,,.}{alleen gelaten Aaike}{als dat gebeurd was}\\

\haiku{Dat de visschen niet,?}{stikken onder dat dichte}{ijs begrijp jij dat}\\

\haiku{H\`e, als wij twee\"en '{\textquoteright},, {\textquoteleft}.}{daars woonden zucht Aaike}{beslistwij samen}\\

\haiku{En vaak zeggen de.}{menschen ook een hoop dingen}{die ze niet meenen}\\

\haiku{Ze kijken elkaar,}{aan en ze lachen zoo maar}{en ze staan zoo maar}\\

\haiku{En Gabe heeft weer,.}{die zeere plek midden in}{zijn borst zit die plek}\\

\haiku{Maar in zijn oogen zit,.}{wat dat is nog scherper dan}{de punt van een naald}\\

\haiku{Hij kijkt toch of hij.}{Gabe graag een draai om zijn}{ooren zou geven}\\

\haiku{Haar bloote borst is weer.}{als een appeltje boven}{haar kleine handen}\\

\haiku{En hij komt met een,.}{paar groote stappen op Gabe}{af met springstappen}\\

\haiku{Maar Roelien zit in,.}{het caf\'e bij Johannes}{en de anderen}\\

\haiku{{\textquoteleft}Geef die kerel op,,.}{zijn mieter lieve God geef}{hem op zijn mieter}\\

\haiku{Ze weet dat hij rilt.}{en klappertandt en dat hij}{natte wangen heeft}\\

\haiku{{\textquoteleft}En dan moeten we,,?}{ook netjes wezen Gabe}{ben j{\'\i}j pas verschoond}\\

\haiku{{\textquoteleft}Jij was er ook bij,.}{toen die vent me achterover}{boog en weg smeet}\\

\haiku{Ze slaat de klep van,:}{haar hengselmand open en laat}{zien wat er in is}\\

\haiku{In Tenderloo is,.}{het bar-stil nog stiller}{dan te Alkerleik}\\

\haiku{Het piepende hek,:}{maken ze maar niet open ze}{klimmen er over heen}\\

\haiku{En ze geeft me een.}{zoen en ze zeit weer dat ik}{haar rechterhand ben}\\

\haiku{een zwarte zure,!}{kribbige deur een deur van}{enkel maar planken}\\

\haiku{Ze zitten daar net,,.}{te eten grauwe erten en}{spek en aardappels}\\

\haiku{En hij snuift zoo, dat,.}{zijn neus er bol van staat en}{dan ademt hij diep-uit}\\

\haiku{Dat jullie ons wel,.}{erg zouen zoeken en dat}{het toch wel naar was}\\

\haiku{Ze weten wel dat,.}{ze rijen maar ze weten}{niet precies waar}\\

\haiku{Ja, wat zit er al,,?}{niet in een mensch me jongen}{in een menschenhart}\\

\haiku{{\textquoteright} Maar in Weierlei.}{krijgt hij toch weer een naar droog}{gevoel in zijn keel}\\

\haiku{hij kijkt wel door zes.}{en dertig muren heen in}{de naaste toekomst}\\

\haiku{Het is al weer een,.}{aardig tijdje geleden}{dat hij van school kwam}\\

\haiku{{\textquoteright} Hij moet er n\'ou zelf.}{ook om grinniken als hij}{er aan terug denkt}\\

\haiku{En je ken het nooit,.}{weten hoe ver j{\'\i}j het nog}{brengt in de wereld}\\

\haiku{Als ze dan maar niet,!}{een van die zussen bij haar}{heeft en geen breiwerk}\\

\haiku{{\textquoteleft}Dat heb ik in de,,,}{drukte vergeten Sien och}{ik heb altijd zoo}\\

\haiku{Ze doet haar geld in.}{haar portemonnee en pakt}{de zak met gort op}\\

\haiku{Ze zeggen ook weer -.}{flauwe aardigheidjes over}{vrouwen de kerels}\\

\haiku{Ik dacht dat je eerst?}{je zussen en je broertjes}{naar bed brengen moest}\\

\haiku{zaaien in een tuin, -.}{\`onze tuin \`onze tuin al}{is hij nog zoo klein}\\

\haiku{Ze koopen warme,.}{krentebollen en spreken}{wat af over morgen}\\

\haiku{{\textquoteright} Het nauwe van die,.}{dag laat hem dan heelemaal}{los ook van binnen}\\

\haiku{Ja, een hart van hem,,.}{alleen een menschenhart en}{hij heeft het n\`og niet}\\

\haiku{En ze zijn over de.}{ophaalbrug gegaan zonder}{er bij te denken}\\

\haiku{muur- en kruiskruid.}{en herderstaschje en}{nieuwe grasscheuten}\\

\haiku{Ze schuift haar handen,.}{verder  over zijn rug ze}{staat dichter bij hem}\\

\haiku{{\textquoteright} En er is nog meer.}{glans om haar voorhoofd en nog}{meer glans om haar oogen}\\

\haiku{Aaike die neemt het,:}{voorjaar mee in haar oogen in}{de reuk van haar kleeren}\\

\haiku{{\textquoteleft}Ik zing om Aaike,,!}{ik zing om het voorjaar en}{nergens anders om}\\

\haiku{{\textquoteright} De zon slaat uit zijn.}{oogen en uit zijn lach en uit}{zijn rood sterk gezicht}\\

\haiku{acht gulden heeft hij {\textquoteleft}?}{al.Kan ik nou nog niet bij}{je an huis komen}\\

\haiku{De bloemen hebben,...}{gekleurde vleugeltjes en}{de wolken witte}\\

\haiku{En om tien uur 's,,:}{avonds of misschien al om kwart}{v\'oor tien dan zegt hij}\\

\haiku{{\textquoteright} Hij loopt zacht over die,:}{dikke rulle grasranden}{daar in Berkenhart}\\

\haiku{De kamperfoelie.}{en de vlier fonkelen of}{ze van groen glas zijn}\\

\haiku{{\textquoteright} Hij wacht maar - en al.}{wachtend begint hij weer over}{zichzelf te praten}\\

\haiku{Ze neemt zijn hand en.}{drukt er aan de binnenkant}{een lange zoen op}\\

\haiku{{\textquoteright} En Johannes die,,.}{strijkt hem over zijn haar ja die}{strijkt hem over zijn haar}\\

\haiku{Hij drukt zijn vingers.}{tegen zijn mond aan en kijkt}{omhoog in de lucht}\\

\haiku{Hij praat heesch, hij.}{staat daar als een bedelaar}{die om een cent vraagt}\\

\haiku{Want we moeten er,.}{schot achter zetten zooveel}{tijd is er niet meer}\\

\haiku{Nou - n\'ou ben jij ook,,, -,.}{haast me Vader Brunt zeg Brunt}{d\`ag tot vanavond Brunt}\\

\haiku{Van het caf\'e hier -.}{ken ik kennen ook geen twee}{gezinnen leven}\\

\haiku{{\textquoteright} En Gabe die valt,.}{tegen hem aan of hij hem}{omver wil loopen}\\

\haiku{{\textquoteright} Hij snikt er nog bij,,.}{hij moet na-snikken als een}{kind die Johannes}\\

\haiku{Aaike knielt d\'aar ook,.}{neer onder de zegenende}{handen van Jezus}\\

\haiku{{\textquoteright} En Gabe is er:}{dan opeens niet te groot voor}{om te mompelen}\\

\haiku{Aaike en Gabe.}{knielen op het mos tusschen}{de boomwortels in}\\

\haiku{Heel Alkerleik is,.}{midden in de zomer grauw}{en  glimmerig}\\

\haiku{Hier was je een keer,,,'.}{op een avond Aaike toe had}{ik je weggebracht}\\

\haiku{{\textquoteright} Ja, en wat is het,?}{waar het dan op aan komt in}{dat heele leven}\\

\haiku{{\textquoteright} Johannes' haar lijkt,.}{witter te worden omdat}{zijn kop rooier wordt}\\

\haiku{Gabe heeft geen erg,.}{in Roelien hij heeft haast nooit}{meer erg in Roelien}\\

\haiku{{\textquoteleft}Zit ze daar dan niet?,?}{met een schram op haar wang en}{een plek op haar kin}\\

\haiku{Hij maakt zoo-maar,.}{een keelgeluid het klinkt een}{beetje minachtend}\\

\haiku{Ik laat je niet los,...{\textquoteright} {\textquoteleft},,.}{voor jeOh ja Gabe laat}{me ook maar niet los}\\

\haiku{{\textquoteright} En dan praat  hij,,.}{ineens schor en hard ja hij}{praat ook schor en hard}\\

\haiku{Aaike, lieve God,?}{dat is toch immers de slag}{van mijn eigen hart}\\

\haiku{Een keer toen hij in,.}{een spiegeltje zijn lachend}{gezicht zag schrok hij}\\

\haiku{Gezien heeft hij het,,.}{toch wel ja gezien heeft hij}{dat alles toch wel}\\

\haiku{Hij knijpt zijn handen.}{stijf om een scherp dingetje}{heen in zijn jaszak}\\

\haiku{Maar hij draagt een fijn.}{opknapperspak en hij rookt}{een fijne sigaar}\\

\haiku{En haar bovenlip.}{heeft ze haast heelemaal naar}{binnen gebeten}\\

\haiku{vanavond is het of.}{hij na een lange zwerftocht}{thuis gekomen is}\\

\haiku{hij drukt zijn knie\"en.}{hard tegen de tafelrand}{aan onder het blad}\\

\haiku{Ze licht het deksel,.}{op van die pan zonder in}{die pan te kijken}\\

\haiku{Groot staat de diepe.}{kom-van-de-lucht over}{alle dingen heen}\\

\haiku{Hij weet het, hij heeft.}{het gezien toen ze onder}{de trouw-eik waren}\\

\haiku{Hij zit op de bok.}{onder de huif maar stil voor}{zich uit te turen}\\

\haiku{{\textquoteright} Sander schuift zijn pijp.}{van zijn eene mondhoek naar zijn}{andere mondhoek}\\

\haiku{{\textquoteleft}Perzik{\textquoteright}, mompelt hij -.}{dan meteen net of hij wat}{goed te maken heeft}\\

\haiku{En makkelijker.}{en mooier dat kennen we}{ook niet betalen}\\

\haiku{Het kon toch nooit wat,.}{raars worden nooit iets om er}{over te grinniken}\\

\haiku{Hij moet denken dat.}{hij zijn kin heen en weer wrijft}{in haar  haartjes}\\

\haiku{En Juffrouw Gees is.}{al voor drie-vierde part}{van de aarde weg}\\

\haiku{{\textquoteleft}Ja{\textquoteright}, denkt Gabe, {\textquoteleft}hier,,...{\textquoteright}}{zitten we nou de baas en}{ik twee oue kerels}\\

\haiku{Ze hebben het over,.}{een nieuwe Fongers over een}{fietstocht naar Friesland}\\

\haiku{De oue schutting wordt,.}{daar weggebroken er komt}{een betonnen muur}\\

\haiku{Hij staat daar tot het,.}{donker wordt daar boven tot}{het venster dicht gaat}\\

\haiku{Moeder, bid je wel,,'...?}{ooit zeg Moeder denk je wel}{an onz lieve Heer}\\

\haiku{Er komt nou altijd.}{een beklemming over hem als}{hij op huis toeloopt}\\

\haiku{En Aaike, die houdt.}{nou nog altijd mijn leven}{in het rechte spoor}\\

\haiku{Haar gezicht is als,.}{een knoet gekorven hout het}{is al strak en dood}\\

\haiku{Alleen menschen die,.}{ergens mee zitten hebben}{zoo'n vluchtende blik}\\

\haiku{De wind omvat hem,,.}{beweegt zich als ademend legt}{zich tegen hem aan}\\

\haiku{Wat later als de,.}{mannen weg zijn gaat hij naar}{boven naar Roelien}\\

\haiku{En \`als het haar eind,.}{zou verhaasten dan zou het}{haar pijn verkorten}\\

\haiku{Maar Sander is nog,.}{niet zoover dat hij zeggen kan}{wat hij zeggen moet}\\

\haiku{Dat was niet wat van,.}{een man en een vrouw dat was}{wat van twee menschen}\\

\haiku{Iemand gooit op een.}{dag een paar centen in de}{muziekautomaat}\\

\haiku{Gabe hakt hout op.}{de binnenplaats en hij zet}{daar al zijn kracht op}\\

\haiku{En dan ineens - waar? -.}{vandaan en hoe zegt Aaike}{iets in de verte}\\

\haiku{Jentje heeft haar kind,.}{al een poos een mooi stevig}{voorlijk kind is het}\\

\haiku{Je kan nooit weten, -.}{waar het goed voor is wacht nog}{wat dat is alles}\\

\haiku{{\textquoteright} Johannes krijgt een,,.}{hoestbui hij heeft adem-nood}{hij komt adem te kort}\\

\haiku{Hij zit niet in de,.}{deur aan de straatweg hij zit}{in de vogelknip}\\

\haiku{{\textquoteleft}Nee, ik had - wijzer{\textquoteright},, {\textquoteleft}.}{moeten wezen denkt hijik}{had niet moeten gaan}\\

\haiku{En Gabe weet niets.}{terug te zeggen  en}{niets terug te doen}\\

\haiku{Hij betast haar en,}{kreunt hij neemt haar op en draagt}{haar naar het licht toe}\\

\haiku{Ineen geknepen.}{zit ze in het holst van de}{nacht in de keuken}\\

\subsection{Uit: Naakte waarheid}

\haiku{Al dat gedaas over,.}{de moderne jeugd het is}{gewoonweg absurd}\\

\haiku{Mama begrijpt niets,.}{als het haar niet met ronde}{woorden gezegd wordt}\\

\haiku{Fijntjes glimlacht ze,,.}{wiegelt met haar eene been en}{houdt het hoofd wat scheef}\\

\haiku{Ze heeft zin om te,.}{joedelen om een schop in}{de lucht te geven}\\

\haiku{Mijnheer Bax staat nog,.}{even in de deur hij praat met}{een grommel-stem}\\

\haiku{Ja gut... is nooit  ,.}{over gesproken misschien als}{ik van school af ben}\\

\haiku{{\textquoteright}, vorscht hij door, {\textquoteleft}vin'?}{je het hier in Wensveld ook}{zoo'n bezopen boel}\\

\haiku{{\textquoteleft}Maar ik ga toch niet{\textquoteright},, {\textquoteleft}....}{zegt ze in zichzelfik ga}{niet met die nieuwe}\\

\haiku{{\textquoteright} Voor tijdverdrijf laat.}{ze haar lippen spelen met}{een rilgeluidje}\\

\haiku{{\textquoteright}, Liz maakt een schimpend, {\textquoteleft},.}{keelgeluidverlakkerij}{k\`an natuurlijk niet}\\

\haiku{{\textquoteright} Hij rukt de knoopen,.}{los hij wil haar vasthouden}{onder haar mantel}\\

\haiku{Op elke plek hier.}{in de laan zou een misdaad}{kunnen gebeuren}\\

\haiku{Het roode teeken{\textquoteright} {\textquoteleft}{\textquoteright}.}{denken en aanDe moord in}{de Kastanjelaan}\\

\haiku{Maar wat kan er toch?}{zoo stinken in lekkere}{bittere cacao}\\

\haiku{Liz kan er soms van.}{alles uitflappen en men}{kan er niets aan doen}\\

\haiku{{\textquoteleft}Ja, voorkomend moet,.}{ze zijn van Mama ook als}{ze er niets van meent}\\

\haiku{{\textquoteright} Even houdt hij nog zijn,.}{rimpelig grijnsmaskertje}{voor dan zakt het weg}\\

\haiku{Daar heb je het nou{\textquoteright},, {\textquoteleft}.}{weer denkt zealle jongens}{willen die kant uit}\\

\haiku{{\textquoteleft}Niet doen{\textquoteright}, mompelt ze.}{met een vakerige stem}{tegen zijn handen}\\

\haiku{{\textquoteleft}Het eten wordt koud{\textquoteright}, zegt, {\textquoteleft},....}{Liz tegen mijnheer Oscar's}{handenz\`eg het \'eten}\\

\haiku{Breede schouders heeft,.}{hij en zijn handen zijn veel}{te sterk voor zijn vak}\\

\haiku{{\textquoteleft}Dag{\textquoteright}, ze duwt hem haar,, {\textquoteleft}...!}{groet toe ze is een en al}{aandrangz\`eg d\`a-\`ag}\\

\haiku{{\textquoteright} Och ja, Coby moet.}{zien hoe aardig het toegaat}{bij hen aan tafel}\\

\haiku{{\textquoteleft}Ja, net als u, h\`e?,,?}{Papa wij zijn erfelijk}{belast niet Papa}\\

\haiku{{\textquoteright} Liz vindt dat alles.}{te onbenullig om er}{op te antwoorden}\\

\haiku{Duco vat het best. {\textquoteleft},{\textquoteright},, {\textquoteleft},?}{Oh \`al zestien grinnikt hij}{heele leeftijd h\`e}\\

\haiku{Ze drukt haar smalle.}{kinderlijke handen in}{vuistjes op de borst}\\

\haiku{En ze is zoo klein.}{in haar gevoel en zoo klein}{in werkelijkheid}\\

\haiku{En ze sluipt als een,,.}{poes niets gooit ze om nergens}{stoot ze tegen aan}\\

\haiku{{\textquoteleft}Stond je daar naar de,?,?}{lichtjes te kijken h\`e stond}{je daar te droomen}\\

\haiku{Het is dus van het.}{grootste belang welk vak zij}{uitoefenen zal}\\

\haiku{{\textquoteright} Liz ademt beklemd, en.}{haar handen worden vochtig}{aan de binnenkant}\\

\haiku{Mama laat zich ook.}{vaak genoeg uithooren door}{Tante Petertje}\\

\haiku{{\textquoteright} Keurig spreidt Mama.}{Richie's fijne Kasjmirdoek}{over de divan uit}\\

\haiku{Bijna slordig droogt,}{ze het marmeren blad af}{en gehaast plaatst ze}\\

\haiku{{\textquoteleft}Zeg 's, ik geloof,?}{dat jij alles van mij staat}{af te gluren niet}\\

\haiku{Maar het spijt Mama.}{nu dat ze haar handen niets}{te doen kan geven}\\

\haiku{Maar haar gedachten,.}{geven haar diep van binnen}{een kneep telkens weer}\\

\haiku{{\textquoteright} En ze wil dan nog,.}{veel scherper uitvaren maar}{daar komt ze niet toe}\\

\haiku{Oma gaf mij een en,.}{ander Oma had nu eenmaal}{zoo iets als voorkeur}\\

\haiku{Veel woorden zal ik,.}{niet aan je verspillen d\`at}{in de eerste plaats}\\

\haiku{{\textquoteright} En nu doet Pa als,.}{Duco hij schuift in zijn stoel}{langzaam dichterbij}\\

\haiku{{\textquoteleft}W\`el...{\textquoteright}, haalt ze uit met}{een donzig stemmetje en}{haar vingertoppen}\\

\haiku{{\textquoteright} En Liz kijkt naar hem.}{om of ze hem op staande}{voet vermoorden wil}\\

\haiku{Moet verdorie ook.}{nog een staatje opmaken}{in het magazijn}\\

\haiku{Maar zij glimlachen.}{niet z\'oo of het van begin}{tot eind onzin is}\\

\haiku{Onze Duuc die heeft ',,...}{nogs is het niet Pa haast}{een elf gehad voor}\\

\haiku{de leeraar had,,...{\textquoteright}}{toch gezegd het was meer dan}{een tien dat hem Pa}\\

\haiku{{\textquoteleft}Coby Duker heeft{\textquoteright},, {\textquoteleft}.}{opgezegd vertelt Mama}{erg spijt het me niet}\\

\haiku{Trouwens de heeren...{\textquoteright},.}{die ik nu heb ze stokt en}{ze laat het er bij}\\

\haiku{En je hebt nog al,?}{veel last van houtworm in je}{meubels is het niet}\\

\haiku{Ik zag daar net hoe.}{je meid aan het spatten was}{met de waterkraan}\\

\haiku{Liz gaat toch maar naar.}{haar uitkijk-post terug}{aan het erkerraam}\\

\haiku{{\textquoteright}, moedert ze, {\textquoteleft}of is?}{er mogelijk nog een klein}{priv\'e verlangen}\\

\haiku{hij uitgaat, dan - dan.}{kan hij niet genoeg op mijn}{mooie dingen letten}\\

\haiku{{\textquoteright} Zij heeft haar peignoir.}{al weer opgevouwen en}{rangschikt de cadeaux}\\

\haiku{En mijnheer Oscar,,.}{reikt haar met beide handen}{alles ineens over}\\

\haiku{Zij gaat er verbluft,.}{bij zitten midden op het}{kleed naast de koffer}\\

\haiku{Liz luistert er naar.}{en trekt de schouders even op}{of ze het koud heeft}\\

\haiku{Gek dat Tante Juup.}{nog altijd niets gestuurd en}{niets geschreven heeft}\\

\haiku{Daar staat de koffer,.}{een log onpractisch ding met}{koperen hoeken}\\

\haiku{{\textquoteleft}Moeder zou mij ook.}{nog zoo graag een horloge}{gegeven hebben}\\

\haiku{{\textquoteleft}O ja - ja...{\textquoteright}, knikt ze,.}{slapjesterloops en laat haar}{knie\"en dansknikken}\\

\haiku{Je koffer mocht er '.}{nogs niet zijn en denk aan}{je tandenborstel}\\

\haiku{Men kan er niet uit,.}{wijs worden wat zij denkt wat}{zij voornemens is}\\

\haiku{{\textquoteleft}Tien uur{\textquoteright}, deelt hij ook ',, {\textquoteleft}.}{nogs mee met een stroeve}{blik op Lizbedtijd}\\

\haiku{Ze trekt, nog altijd,.}{nadenkend haar kousen uit}{en zoekt haar pon op}\\

\haiku{{\textquoteright} Papa zat onder,}{de dennetakken met de}{kunstsneeuw te flirten}\\

\haiku{Ze rillen overluid,.}{klappertanden overluid en}{stennen zoo maar wat}\\

\haiku{rozig-glad zijn ze,.}{met keurige maantjes en}{zorgvuldig gevijld}\\

\haiku{Als ze wat later,.}{zoekend opkijkt treft hen geen}{bestraffende blik}\\

\haiku{Maar als ze haar groote.}{blauwe oogen opslaat is ze}{heelemaal anders}\\

\haiku{haar Moeder is van,.}{een muis geschrokken toen ze}{in positie was}\\

\haiku{Ka Kool, tegenover,.}{haar is onbehoorlijk lang}{en leelijk en braaf}\\

\haiku{De boekentasschen -.}{staan al ingepakt bij de}{deur sinds gisteravond}\\

\haiku{Och wat!, Ilsevoort.}{hoeft enkel op mantels en}{hoeden te letten}\\

\haiku{, j{\'\i}j bent altijd veel...,!}{te bescheiden Madame}{La Polyandrie}\\

\haiku{Maar voor een raar stroef,.}{gevoel in zichzelf heeft ze}{enkel een grijns over}\\

\haiku{Een afgerichte{\textquoteright},, {\textquoteleft}.}{kever dan wijzigt zeeen}{kermis-kever}\\

\haiku{{\textquoteright} En onderhand wordt.}{ze voortgeduwd zonder dat}{iemand haar aanraakt}\\

\haiku{Zij zal een groote flesch!?}{lilas koopen of jasmin}{of violette}\\

\haiku{Roos de Wit achter,.}{hen gichelt om het een of}{ander en fluistert}\\

\haiku{En de oudste vroeg {\textquoteleft}{\textquoteright},.}{haarvoor een baantje en de}{andere vroeg B\'e}\\

\haiku{Zij weet nu ineens.}{wat ze gezocht heeft in de}{dag die v\'oor haar ligt}\\

\haiku{{\textquoteright} Verrukkelijk - en,,.}{daarna als Ot en Ed klaar}{zijn mag zij crepeeren}\\

\haiku{{\textquoteright} Een doorzichtige,.}{aansporing is dat. Och ja}{zij is zijn dochter}\\

\haiku{{\textquoteright}, hij kucht tegen iets, {\textquoteleft},?}{zenuwachtigs in de keel}{zooals jij daar zat h\`e}\\

\haiku{, zaagt hij nu in de, {\textquoteleft}.}{verkeerde richting doorheb}{je geen vacantie}\\

\haiku{{\textquoteleft}Ik wil net zoo vrij,,,.}{zijn als jij zelf Papa om}{te doen wat ik wil}\\

\haiku{Er is iets in Liz'.}{houding en stem dat haar tot}{zijns gelijke maakt}\\

\haiku{Papa neemt zijn bril.}{af en bekijkt de glazen}{aan beide kanten}\\

\haiku{{\textquoteright} Hij moet nog even zijn.}{natte oogen afvegen en}{wat nakuchen en slikken}\\

\haiku{Hij denkt ergens over,.}{na hij  heeft nog iets te}{bespreken met haar}\\

\haiku{Aanhoudend loopt zij.}{daar over door te piekeren}{in haar gedachten}\\

\haiku{En als Papa het,.}{wist zou het zijn oordeel over}{haar niet verzachten}\\

\haiku{knikt bijna oolijk,.}{en het wasachtige glijdt}{weg van haar gezicht}\\

\haiku{Emiel Kan praat... Emiel Kan.}{voert een dubbel-gesprek}{met oogen en woorden}\\

\haiku{Elk detail van zijn.}{uiterlijk neemt ze met groote}{aandacht in zich op}\\

\haiku{{\textquoteright} En Emiel Kan voert zijn.}{dubbel-gesprek weer met}{haar en met Papa}\\

\haiku{{\textquoteright} Zij kijken elkaar,.}{aan schijnbaar terloops en toch}{van heel nabij}\\

\haiku{{\textquoteleft}O ja{\textquoteright}, overdrijft ze, {\textquoteleft},.}{geniale aanleg ben}{ik mee geboren}\\

\haiku{Het is onprettig.}{warm in het werkkamertje}{en onprettig stil}\\

\haiku{Van je verstand moet{\textquoteright},.}{je het hebben zegt Liz met}{iets van een glimlach}\\

\haiku{{\textquoteright} En dat woord klinkt zoo.}{nuchter in haar mond als een}{telefoonnummer}\\

\haiku{En het wrijft zich met.}{een lustgevoel tegen de}{zware zijde aan}\\

\haiku{Drie garnituren,,.}{heeft zij nu een rose een}{wit en een lila}\\

\haiku{Ze doet haar lippen.}{van-een en spant ze glad om}{het tandvleesch heen}\\

\haiku{{\textquoteleft}I 'll make a,,.}{string of pearls out of the}{dew for you for you}\\

\haiku{{\textquoteright} Maar daarmee verdwijnt.}{het kwaadaardige gevoel}{nog niet heelemaal}\\

\haiku{{\textquoteright} {\textquoteleft}Black and white{\textquoteright} schiet.}{even later als een spin in}{een web op hen toe}\\

\haiku{Hij  transpireert,.}{hevig vloekt er in stilte}{om en glimlacht zuur}\\

\haiku{Die  zijn handig,{\textquoteright},, {\textquoteleft}...{\textquoteright} {\textquoteleft}?}{wou ik al lang fluistert ze}{doe ik zelfErg duur}\\

\haiku{Er staat een practisch:}{schrijfbureau met verwaande}{presse-papiers}\\

\haiku{Bedaard steekt ze een,.}{nieuwe sigaret aan en}{glimlacht als Papa}\\

\haiku{Dat voel je dan toch{\textquoteright},, {\textquoteleft},?}{zegt  zeof het gaat of}{je er klaar voor bent}\\

\haiku{{\textquoteright}, zijn handen glijden, {\textquoteleft}?}{mijnheer-Oscar-achtig over}{haar heendat toch wel}\\

\haiku{Ze kijkt met halve,.}{oogen over zee uit soms kijkt ze}{ook met een kwartoog}\\

\haiku{Leuk is het om die.}{ge\"etaleerde menschen}{zoo te zien zitten}\\

\haiku{Mama stilletjes,.}{staan waar ze staat en wandelt}{om het terras heen}\\

\haiku{Stom vervelend{\textquoteright}, denkt,.}{ze en geniet hevig van}{de belangstelling}\\

\haiku{Beredderig doen,.}{zij geen van allen en ook}{niet geagiteerd}\\

\haiku{En de anderen.}{wachten onwillekeurig}{of er nog iets komt}\\

\haiku{Duuc grijpt per abuis naar,.}{zijn bierpot er is nog maar}{een sliertje schuim in}\\

\haiku{Daar zal dan nog 's.}{een vakkundige aan te}{pas moeten komen}\\

\haiku{{\textquoteleft}Ik slaap daar benauwd{\textquoteright},, {\textquoteleft}...}{mokt Ducoeen kamertje}{als een stijfselkist}\\

\haiku{Door de oogen van de.}{man met de groote hoed lijken}{vonken te springen}\\

\haiku{Nuchter bekijkt ze,.}{zichzelf nuchter controleert}{ze haar gedachten}\\

\haiku{Ze zal aanstonds even,.}{aan de deur luisteren eer}{ze naar binnen gaat}\\

\haiku{Handig opent en sluit,.}{ze de deur en wipt luchtig}{de steile trap af}\\

\haiku{Heb afspraakjes bij, '.}{bosjes gemaakt maar je wilt}{wels wat anders}\\

\haiku{Ze hebben vieze.}{neuzen en afzakkende}{broekjes en kousen}\\

\haiku{{\textquoteleft}Wij houden alles - '.}{stijf vast Papa brengt het nog}{wels tot Koster}\\

\haiku{De voeten glijden.}{uit in het verschuivende}{zand van de helling}\\

\haiku{{\textquoteleft}Een goeie pot bier was{\textquoteright}.}{dat. Liz kruipt in haar mantel}{of ze het koud heeft}\\

\haiku{Het zomer-strand -.}{is sexualiteit op een}{presenteerblaadje}\\

\haiku{{\textquoteright} Wat later zanikt.}{Krillertje over een dame}{die Elise Bock heet}\\

\haiku{Sloom kijkt ze om en.}{lacht uitbundig maar met een}{raar hik-geluid}\\

\haiku{{\textquoteright} Ed en Ot staan er,,.}{gereserveerd bij droog wijs}{en Duco-achtig}\\

\haiku{Het is drukker en,,.}{jolijtiger dan anders}{nu het is voller}\\

\haiku{Zij bedrinken zich,.}{aan de muziek zij gaan er}{zich aan te buiten}\\

\haiku{De avondlucht valt frisch,.}{op haar warm gezicht op haar}{vochtige dijen}\\

\haiku{Twee weken - veertien -.}{lange dagen en nachten}{is dit nu al zoo}\\

\haiku{Hij wou op-laatst wel,.}{graag weer naar school en niet zoo}{zeer om de school zelf}\\

\haiku{{\textquoteright} Zijn stappen gaan niet.}{z\'oo ver-weg dat ze}{uitsterven kunnen}\\

\haiku{{\textquoteleft}Hou je zelf toch niet,.}{zoo krampachtig vast zielig}{burgermannetje}\\

\haiku{Daar is het altijd{\textquoteright},, {\textquoteleft}.}{mee begonnen denkt zemet}{een zoen op mijn kruin}\\

\haiku{Een lief jongetje{\textquoteright},,.}{ben je wil ze zeggen maar}{ze bezint zich nog}\\

\haiku{Hij ziet weer een klein,.}{huis met een stroodak aan de}{zonkant van een pad}\\

\haiku{{\textquoteleft}Ja{\textquoteright}, fluistert ze in, {\textquoteleft} {\textquotedblleft}{\textquotedblright} -?}{een lachen daten zoo is}{het voornaamste niet}\\

\haiku{{\textquoteleft}Daar heb je nou een -!}{vriendin voor noodig en dan nog}{wel een getrouwde}\\

\haiku{{\textquoteleft}als je daar bang voor,.}{bent zal ik mijn kamerdeur}{wel even afsluiten}\\

\haiku{En Emiel plukt aan Liz'.}{kanten halskraagje en streelt}{over haar achterhoofd}\\

\haiku{Nu zijn we aan een.}{schriftelijke cursus in}{het Duitsch begonnen}\\

\haiku{{\textquoteright} Hij bijt op zijn vuist,, -.}{wordt bleeker en komt toch}{terug elken dag}\\

\haiku{Bovendien heeft ze,.}{een beetje echte pijn een}{beetje echte angst}\\

\haiku{Ze rilt daarbij op,.}{twee manieren gekunsteld}{en ongekunsteld}\\

\haiku{{\textquoteleft}Dat was ook voor de,.}{eerste maal en wat had ze}{toen al niet beleefd}\\

\haiku{Hij schuift zijn handen.}{in zijn broekszakken en buigt}{zich lachend voorover}\\

\haiku{En de routes van.}{de trams weet ik ook nog niet}{en de nummers niet}\\

\haiku{{\textquoteright} Liz begrijpt hem best. {\textquoteleft},?,?,?}{Dank je is het goed met ze}{met Mama en Duuc}\\

\haiku{Och, bent u Lizzy's?,,?}{Oom geeft u toch uw hoed hier}{wilt u plaats nemen}\\

\haiku{{\textquoteright}, Oscar tast al in,,.}{zijn zak links rechts en diept een}{keurig pakje op}\\

\haiku{In de schemer lijkt -.}{ze een groot kind in het licht}{een gevallen vrouw}\\

\haiku{Vroeger waren de,.}{mannen trouw uit sleur nu zijn}{ze ontrouw uit sleur}\\

\haiku{En ze weet dat ze,.}{naast de dikke Lowis zit}{Lowis-de-Jood}\\

\haiku{Maar Lot Kreevelt trekt,.}{hem naast zich op de stoel die}{Herfst verlaten heeft}\\

\haiku{{\textquoteright} Gehoorzaam doet ze {\textquoteleft} -...{\textquoteright},, {\textquoteleft}.}{dat.Men moest dat niet hakkelt}{zemet iedereen}\\

\haiku{Strak kijkt ze naar de.}{dingen om haar heen en ze}{ziet ze toch niet goed}\\

\haiku{{\textquoteleft}Als God{\textquoteright}, herhaalt ze,.}{en strijkt met de kin langs de}{fijne bloembladen}\\

\haiku{Stilletjes wascht.}{ze een paar kousen uit in}{haar toilet-emmer}\\

\haiku{{\textquoteleft}W\`erken{\textquoteright}, herhaalt, {\textquoteleft}.}{Lot op een gewichtige}{toonmoet w\`erken}\\

\haiku{Maar je moet Liz vrij,.}{laten ze moet niet voelen}{dat je op haar let}\\

\haiku{{\textquoteright} {\textquoteleft}Precies, daarom{\textquoteright}, ze, {\textquoteleft}?}{knikten ben je nu nog bang}{om te verliezen}\\

\haiku{{\textquoteleft}Thee, lui?, z\`eg - th\'ee?, of,?}{thee-met-rum of rum met}{thee of rum-puur}\\

\haiku{{\textquoteleft}Lowis - Lowisje -...{\textquoteright}.}{nur eine Maar ze vergeet}{er op door te gaan}\\

\haiku{Ze staan onder een,.}{rood driekantig lampje en}{klinken ergens op}\\

\haiku{Een buitensporig.}{triumfantelijk gevoel}{geeft haar dat ineens}\\

\haiku{Ze was werkelijk,.}{een oogenblik aan het strand}{in de roosterkuil}\\

\haiku{{\textquoteright} Een oogenblik is,.}{alles duidelijker dan}{verdoezelt het weer}\\

\haiku{En Nicolette.}{bevoelt doelloos de rand van}{de roode deken}\\

\haiku{Hij lachte er om,,}{ik hoorde hem lachen dat}{verwachtte ik wel}\\

\haiku{{\textquoteleft}Laat ik hem toch in,...}{zijn gezicht spuwen laat ik}{hem weg-jouwen}\\

\haiku{{\textquoteleft}Te kunnen bidden{\textquoteright},, {\textquoteleft} -.}{denkt zete kunnen bidden}{wat vreemd moet dat zijn}\\

\haiku{Ze heeft verdriet en -.}{kan de handen uitstrekken}{God is geen leegte}\\

\haiku{Kassen duwt haar weg.}{met de eene hand en streelt haar}{met de andere}\\

\haiku{{\textquoteright}, zijn stem giert, rochelt,.}{hij zakt op  een stoel neer}{en staat ook weer op}\\

\haiku{{\textquoteleft}En omdat ik dat,, -......?}{gedaan heb gooi jij je weg}{als een een stuk vuil}\\

\haiku{Kassen Herfst was er.}{erg op gesteld dat ze een}{sluier zou dragen}\\

\haiku{Ze wordt voorgesteld,.}{en begroet maar ze onthoudt}{geen enkele naam}\\

\subsection{Uit: De ontmoetingen van Rieuwertje Brand}

\haiku{Al bijna een uur,}{lang zat Rieuwertje op zijn}{kleine manke bank}\\

\haiku{Nebekadnezer,!}{kan dat ook niet weten die}{komt op de schijn af}\\

\haiku{Kleine geluidjes}{scharrelen in het rond en}{aan die geluidjes}\\

\haiku{Duidelijk hoort hij -.}{daar de hel knetteren het}{is een goeie afschrik}\\

\haiku{want die gedachte,.}{gaat dwars door een zeerigheid}{heen binnen in hem}\\

\haiku{{\textquoteleft}Ja, dat kenne we!,,!,...}{handen thuus asjeblift gien}{malaberigheid}\\

\haiku{t begeeren is meer ',,!}{dant hebben ferachtig}{weerheid ferachtig}\\

\haiku{Maar de Knaak is breed,.}{op het vierkantige af}{en rood als baksteen}\\

\haiku{Op het Hoofd, bij de,}{Harlinger boot praat hij een}{verlept mijnheertje}\\

\haiku{Dartien stuver... dat, -!}{ken niet dan dan moet ik er}{nog op toeleggen}\\

\haiku{{\textquoteleft}Grutje-me-tut{\textquoteright},,, {\textquoteleft} '.}{smaalt hij in zijn gedachten}{watn gootwater}\\

\haiku{Maar eerst had hij 'n ' '.}{skrievenricht tot de regeering}{oft wel schikte}\\

\haiku{Affien, veul en niet......':}{genog en deer niet van maar}{toe zee de Koning}\\

\haiku{{\textquoteright}, Rieuwertje schraapt het,.}{grom van de planken en sluit}{zijn wagentje weer}\\

\haiku{{\textquoteright} {\textquoteleft}Lever me niet uut,{\textquoteright},, {\textquoteleft}.}{Heere smeekt hij nederig}{lever me niet uut}\\

\haiku{Want waar hij zichzelf,.}{terugvindt daar is het meer}{dan verschrikkelijk}\\

\haiku{Rieuwertje wil hem,}{uit de weg gaan heelemaal}{in-de-war}\\

\haiku{{\textquoteright} Hij hoort de wielen.}{van zijn wagentje over de}{klinkers ratelen}\\

\haiku{Hij ziet de jongen.}{hakken en schaven in de}{timmermanswerkplaats}\\

\haiku{De pluchen rozet...}{op haar eene pantoffel hipt}{jolig heen en weer}\\

\haiku{{\textquoteright}, vraagt hij benepen, {\textquoteleft} -?}{en en heb je gien poesie32}{eten veur mijn bewaard}\\

\haiku{{\textquoteleft}O ja, wel zeker!,!}{en dat mondjen-vol}{eten komt er vanzelf}\\

\haiku{Het is ook of hij,.}{grooter wordt het verdriet gaat}{een beetje opzij}\\

\haiku{je geven, al wat,...{\textquoteright},}{ik verdiend heb maar dat \^are}{geld onbeholpen}\\

\haiku{Met de buitenkant.}{van zijn vingers strijkt hij een}{paar maal over het kret}\\

\haiku{Heel  zacht doet hij,.}{dat net zooals hij anders over}{Koosie's haar zou strijken}\\

\haiku{Hij ziet plotseling.}{dat het plankendekje van}{zijn wagen openligt}\\

\haiku{Klauwen grijpen zijn -?!}{hart aan en hoe het zoo kan}{en wat dat toch is}\\

\haiku{het is of zijn oogen.}{op de wiegende plooien}{van haar rok hangen}\\

\haiku{En laat niemand nu}{ooit weer zeggen dat die kroeg}{van Woutjen een vuil}\\

\haiku{RIEUWERTJE MOET ZICH.}{INSPANNEN om er iets van}{te onderscheiden}\\

\haiku{Gelukkig was die,,.}{man luw waren zijn dagen}{vredig zijn nachten}\\

\haiku{Hij moet een hooge stoep,,!}{op dat is lastig zoo'n stoep}{een heele opstap}\\

\haiku{Machteloos zinkt hij,.}{voorover met zijn mond op een}{stoffig vloerkleedje}\\

\haiku{Domp hoort hij scherpe,.}{jonge stemmen vaag ziet hij}{kleine figuren}\\

\haiku{Die vrouwachtige.}{meid is Rieuwertje's dochter}{Leen-die-dient}\\

\haiku{Rederijk is die,!}{goeie vrind in de verte hoor}{dat schepsel praten}\\

\haiku{{\textquoteright} De vischreuk aan zijn.}{goed wil hem op andere}{gedachten brengen}\\

\haiku{Z\'eker... mos' je 'n '... '...}{wuufhad hebbe as iene}{asn zekere}\\

\haiku{Ja, die andere,,!}{zit er nog zoo waarachtig}{als God hij zit er}\\

\haiku{{\textquoteleft}K-kom niet in...{\textquoteright} {\textquoteleft}{\textquoteright},.}{mijnHou je waffel gebiedt}{de Cosmopoliet}\\

\haiku{{\textquoteright} En die woorden keeren,.}{zich verwonderd naar hem om}{eer zij verdwijnen}\\

\haiku{Louter bij toeval.}{komt hij in Woutjen's gele}{warme kroeg terecht}\\

\haiku{Zijn protest snerpt als:}{de krijschende schreeuw van een}{beest dat geslacht wordt}\\

\haiku{* * * ~ Schurftig zijn die,,.}{plagen rood-ontstoken}{boosaardig en wreed}\\

\haiku{Zijn hand boort een kuil,,}{in het blinde zwart hij werpt}{het donker opzij}\\

\haiku{En zoo folterend!}{als dat visioen van zijn}{delirium is}\\

\haiku{Maar het glanzende.}{warme leven tracht hem van}{zich af te schudden}\\

\haiku{Rieuwertje luistert,.}{nog maar zijn aandacht gaat een}{andere kant op}\\

\haiku{Huilie binne er,!}{nou niet huilie kenne je}{nou niet uutlache}\\

\haiku{Rieuwertje bukt zich,!}{houterig en gluurt onder}{tafel daar is niets}\\

\haiku{Rieuwertje merkt het,.}{en het is of hij van zijn}{eigen hart vervreemdt}\\

\haiku{{\textquoteleft}Gaan na' buten, jij,!,!,!}{allaah pak je biezen niet}{omklungelen hier}\\

\haiku{Hij zit ook al een,,.}{heele poos eer hij bemerkt}{dat hij niet meer loopt}\\

\haiku{{\textquoteright}, prevelt hij, {\textquoteleft}en n{\'\i}je38 -...{\textquoteright}.}{n{\'\i}je minsche En hij smakt zoo'n}{beetje als de zee}\\

\haiku{Het lijkt wel of de...}{stilte eerst een vloeibaar iets}{was en nu stolt}\\

\haiku{{\textquoteright} Als een vlaggetje,}{zwaait de tong heen en weer in}{die zwarte leege mond}\\

\haiku{n sp\'oog water... veur,,...{\textquoteright},}{je over h\^et gien levende}{ziel die verbaasd blijft}\\

\haiku{{\textquoteright} Meteen mummelt hij,...}{weer door over de donkere}{tijd en over Engel}\\

\haiku{Dat doet hij dan toch,.}{niet maar hij zucht als een mensch}{die zwaar werk verricht}\\

\haiku{Maar Rieuwertje let,.}{er niet op hij heeft het veel}{te druk met zichzelf}\\

\haiku{{\textquoteleft}Wou je dat overdag,,...?}{doen ellendelingkien op}{de klaarlichte dag}\\

\haiku{{\textquoteleft}'t Maakte h\'em niks, ',...}{meer uutt kon hem niks meer}{skelen de jongen}\\

\haiku{Als Kako omkijkt,.}{kan het best gebeuren dat}{hij hem terug jaagt}\\

\haiku{En Rieuwertje kijkt,.}{nog naar hem als hij al-lang}{niet meer te zien is}\\

\haiku{maar het is of zijn,.}{bloed zweet en zijn hart krijgt weer}{zoo'n kaduuk gevoel}\\

\haiku{gaan er af en toe.}{een paar gedachten door zijn}{sufferige hoofd}\\

\haiku{Gosse de Kiezer,.}{die z'n lichaam was an'kocht}{veur de snijkamer}\\

\haiku{maar dat dorst hij niet,,,}{deer had hij gien koerasie}{veur jisses jisses}\\

\haiku{{\textquoteright} Zoetelijk-paaiend,,...}{of hij het tegen een klein}{kind heeft vraagt hij het}\\

\haiku{Elke dag begint,.}{hij daar opnieuw mee altijd}{is het vruchteloos}\\

\haiku{Hij rammelt er mee,,...}{hij bluft er mee de spijt huilt}{boven alles uit}\\

\haiku{Hij kocht een vloerkleed,.}{voor Engel kastbekers en}{ook een theeservies}\\

\haiku{O minsch, ik ken,,!}{je niet luchten of zien ik}{spij50 op je minsch}\\

\haiku{De droom houdt hem vast,.}{en er lijkt een glans van zijn}{gezicht uit te gaan}\\

\haiku{in elk groen-blauw!}{bobbeltje kan een stukje}{pauweveer zitten}\\

\haiku{Hij weet niet waarheen.}{hij gaat en hij weet niet waar}{hij neerzit oplaatst}\\

\haiku{Zijn licht-schuwe.}{oogen klampen armzalig de}{voorbijgangers aan}\\

\haiku{, de zonderbare.}{eensgezindheid van Engel}{met de kinderen}\\

\haiku{Nog vol-op moet de,,...}{zon daar-bij-hem-thuis}{door het raam vallen}\\

\haiku{Muziek ratelt en,,...}{boemt paardenhoeven ketsen}{kleuren schitteren}\\

\haiku{Vlak voor Rieuwertje.}{staan een paar vreemdsoortige}{personages stil}\\

\haiku{hij kan loopt hij de,.}{straat op hij wil een beetje}{monterheid koopen}\\

\haiku{Hij wil ook lachen,,.}{overluid wil hij lachen maar}{dat kan hij niet meer}\\

\haiku{Daar komt de optocht,,...}{aan in zilver en goud en}{vertreedt het hondje}\\

\haiku{Le\^et maar an m{\'\i}jn over,, ' '!}{dat is m{\'\i}jn toevertrouwdt}{zeln spul worden}\\

\haiku{{\textquoteright}, protesteert Kako, {\textquoteleft} '...{\textquoteright} {\textquoteleft}}{vermoeideten geef je nog an}{n zwarvende hond}\\

\haiku{{\textquoteright} Rieuwertje knikt ten,.}{teeken dat hij het verstaat}{maar hij zegt nog niets}\\

\haiku{En elke keer als,,.}{ze daar op doorgaat gromt haar}{man en kleurt Japie}\\

\haiku{En Engel geeft een.}{zenuwachtige ruk aan}{haar bloemenhoedje}\\

\haiku{{\textquoteleft}Toe maar - toe maar, 't, ', '.}{is lekkert verwarmtt}{brandt in je botten}\\

\haiku{{\textquoteleft}'n Riksdaalder{\textquoteright}, stelt, {\textquoteleft}!}{Leen hem kordaat uit zichzelf}{voorken je kriegen}\\

\haiku{Sputterend steken.}{de kinderen-en-Engel}{de hoofden bijeen}\\

\haiku{Maar als hij zijn naam,,.}{moet zetten kan hij haast niet}{schrijven zoo beeft hij}\\

\haiku{{\textquoteleft}Tjisses...{\textquoteright}, valt het door, {\textquoteleft}......!}{hem heendie kol was  er}{altied en nooit zelf}\\

\haiku{Terloops dringt het tot,,...}{Rieuwertje door hij let er}{toch niet op verder}\\

\haiku{, bin je niet iensen...}{dronken en ken je dan toch}{even goed bidden veur}\\

\haiku{net as op die nacht...'...}{op die avend le\^et toe je}{bezeten wasse}\\

\haiku{n Minsch zeit op,...!}{z'n starfbed \^are dingen dan}{in z'n leven oue}\\

\haiku{{\textquoteright} Maar Engel wrikt zich.}{netelig achteruit in}{het verwoelde bed}\\

\haiku{Op skool{\textquoteright}, ratelt ze, {\textquoteleft},'!}{doene ze m{\'\i}jn ook pien mos}{je op skool hooren}\\

\haiku{De flakkerende.}{zon geeft een ongewisse}{blijheid aan het erf}\\

\haiku{Tja, nou, as die Leen... ' '!}{de wasch weeromstuurdet}{most tochdaan worden}\\

\haiku{{\textquoteleft}Le\^et Vader nou maar ',.}{n stre\^etje om gaan Moeder}{zel je wel helpen}\\

\haiku{{\textquoteright} En Rieuwertje grijpt...}{zijn hoed en zijn jekker van}{de stoel en hij gaat}\\

\haiku{Me-jongen{\textquoteright}, vraagt, {\textquoteleft}?}{hij afgetrokkenwillen}{we wat kuieren}\\

\haiku{Daar zit oue minschie...}{onder de kaarsen-kroon}{en tjilpt als een musch}\\

\haiku{Vadertje-God, '{\textquoteright}}{h\^et je an'nomen en je}{binne heerlijkred.}\\

\haiku{{\textquoteleft}Kako, die liep   ',...}{deer ook nog ers en kon}{temet niet veerder}\\

\haiku{En Engel rukt en.}{trekt als een waanzinnige}{om los te komen}\\

\haiku{Pas was 't kind er{\textquoteright},, {\textquoteleft}....}{toch nog denkt hij doezelig}{passies vlak bij mijn}\\

\haiku{De huizen staan als...}{dommelige gedachten}{in het  duister}\\

\haiku{Armoedigkaal voelt.}{de gangvloer aan onder zijn}{tastende voeten}\\

\haiku{Hij doet omzichtig.}{de kamerdeur open en kijkt}{schichtig naar binnen}\\

\haiku{Hij heeft zoo voor de.}{menschen in het algemeen}{niet veel aandacht meer}\\

\haiku{En Veronica.}{begint haastig wat houtjes}{op te rapen}\\

\haiku{As je skarp kieken, ' '.}{zie je haast hoe de lieve}{Heertskapen h\^et}\\

\haiku{Zwakjes moeten ze,.}{allebei glimlachen hij}{en Veronica}\\

\haiku{{\textquoteright}, vraagt Veronica, {\textquoteleft}?}{bijna fluisterendis dat}{nou wezenlijk waar}\\

\haiku{{\textquoteleft}Nog wel honderde,,...}{malen bedankt Vroon nog wel}{honderde malen}\\

\haiku{Eenmaal, is hij nu,,.}{in-het-kort Engel}{tegengekomen}\\

\haiku{Hij knikte nog, en.}{hij had even goed tegen een}{huis kunnen knikken}\\

\haiku{Wijs en vriendelijk.}{kijkt grijze oue minschie over}{haar witte breikous}\\

\haiku{{\textquoteleft}O lieve Heer... o,...}{Vadertje-God le\^et ze}{mijn niet wegsturen}\\

\haiku{in Gods naam, kom weer,,}{bij mijn of le\^et mijn bij jou}{blieven Engel toe}\\

\haiku{HART WORDT DAAR niet meer,.}{blij van in blijheid schittert}{het leven te fel}\\

\haiku{Maar die moeheid is,.}{er nu eenmaal en moet er}{al meer geweest zijn}\\

\haiku{Ze loopt van hem af,.}{en gaat zonder slag of stoot}{rugwaarts de steeg in}\\

\haiku{Voor de zijdeur van,.}{haar Moeders erf hurkt ze neer}{en prevelt en bidt}\\

\haiku{haar hangt tot op haar.}{onderworpen handen en}{bedekt haar gezicht}\\

\haiku{Hij moet er elke,.}{dag aanbellen en soms ook}{tweemaal op een dag}\\

\haiku{Ze willen mijn niet{\textquoteright},, {\textquoteleft}.}{vertelt hij aan de stilte}{ze sturen mijn weg}\\

\haiku{En hij denkt er niet,,.}{aan om naar huis te gaan hij}{heeft honger noch dorst}\\

\haiku{Hij verstaat niet wat,,...}{iedereen hoort maar het is}{goed zoo het is goed}\\

\haiku{{\textquoteleft}Ja{\textquoteright}, sust haar stem, {\textquoteleft}ja...,...{\textquoteright}.}{stil nou maar stil nou Maar er}{zijn nog meer zorgen}\\

\haiku{{\textquoteleft}Wat wordt nou alles, ' '.}{nog goedt ken nog best in}{t reine komme}\\

\haiku{{\textquoteright} En Rieuwertje kan,.}{haast niet meer praten maar zijn}{geest is nog helder}\\

\subsection{Uit: Oude kennissen}

\haiku{, zijn doezelige.}{gedachten verdoolden in}{herinneringen}\\

\haiku{Maar haar kleine oogen,,!}{gnuifden ze was blij dat ze}{er zelf niet heen moest}\\

\haiku{En Duif peinsde waar,,.}{hij heen moest die man en wie}{of het wezen kon}\\

\haiku{{\textquoteright} De vrouwen zuchtten,.}{en beklaagden Duif en twee}{haalden er een smid}\\

\haiku{Een diepe zorgplooi,:}{had hij tusschen zijn oogen en}{hij dacht telkens weer}\\

\haiku{Tjee, dat hij ook geen,...{\textquoteright}}{ander had kenne krijgen}{as juist dat vreempie}\\

\haiku{{\textquoteleft}G\^o, ook 'n leven{\textquoteright},, {\textquoteleft}'. '}{pruttelde hijn leven}{zoo op je eentje}\\

\haiku{{\textquoteright} Hij keek oolijk op.}{naar zijn vrouw en hij lei zijn}{hand op de hare}\\

\haiku{er maar niet boos om,, ',?}{worden man voor jou ist}{zoo'n bezwaar niet h\`e}\\

\haiku{, en zijn gedachten,.}{wriemelden verward dooreen}{zijn hoofd was overvol}\\

\haiku{Saar, voor aan de straat,,.}{en Brecht midden-achter en}{Aagt uit de keuken}\\

\haiku{s vragen willen... '}{offe we wel in de goeie}{spoor zitten en zoo}\\

\haiku{{\textquoteright} Achter een boschje.}{bleven ze ginnegappend}{naar hen staan kijken}\\

\haiku{Maar u bent hi\'er toch{\textquoteright},, {\textquoteleft}}{niet te Amsterdam lichtte}{hij glimlachend in}\\

\haiku{Klaas, die al deze,,.}{dingen verstond omdat hij}{gek was hijgde zwaar}\\

\haiku{Jawel, jawel, ze,,.}{is er ze moet er wezen}{ze m\'oet er wezen}\\

\haiku{{\textquoteleft}De buren moeten, '.}{niet merken datk zoo op}{de postbode wacht}\\

\haiku{De hand van de vrouw,,.}{was warm haar stem zacht en ze}{liep dicht naast hem voort}\\

\haiku{En de duisternis.}{werd ondoordringbaar in het}{zware hoofd van Klaas}\\

\haiku{{\textquoteleft}En verder niks, niks,.}{van zien olde minschen en}{niks van zien verleeje}\\

\haiku{Het ontgleed haar weer.}{in een knijperig gevoel}{van teleurstelling}\\

\haiku{{\textquoteleft}Han denkelijk toch,.}{al weg en nou moest zij d'r}{reppen van belang}\\

\haiku{Jet deed erg haar best.}{om niets te laten blijken}{van haar onthutstheid}\\

\haiku{Toen opeens, midden,.}{in haar blerrend vertellen}{onderbrak Sien haar}\\

\haiku{Abel smeet zijn werk neer,,.}{zijn hoofd was zweeterig en}{zijn vingers beefden}\\

\haiku{{\textquoteleft}Catootje{\textquoteright}, Abel groette haar,,.}{dof er was een schrik in zijn}{stem een rilling ook}\\

\haiku{{\textquoteleft}'t - 't Was toch 'n,?, ',?}{bovenneturelijke toer}{h\`en krachttoer h\`e}\\

\haiku{En het maantje was,.}{naargeestig-bleek en het}{glimlachte ni\'et}\\

\haiku{Hij greep naar de groote}{bolle knop en trok meteen}{de hand weer terug}\\

\haiku{Soezerig zocht hij,.}{meteen naar de woorden die}{hij verloren had}\\

\haiku{Om de deurhoek schoof.}{het kleine gladde hoofdje}{van het dienstmeisje}\\

\haiku{{\textquoteleft}Ga gauw naar Mijnheer....}{Steenstra van hier-naast of}{hij dadelijk komt}\\

\haiku{de koffiepot op.}{het lichtje zette en haar}{handen in de zij}\\

\haiku{Zoo'n naam nou{\textquoteright}, sufte, {\textquoteleft} '.}{hijeigenlijk zat ern}{nuchtere klank in}\\

\haiku{via de klanten... ter,?}{oore gekomen dat ik}{er wat van versta}\\

\haiku{... ik ben er blij mee......?}{dat jij als je wat geduld}{met me wil hebben}\\

\haiku{{\textquoteright} Even kreeg hij nu toch.}{weer een stroef gevoel in de}{keel en at vlug af}\\

\haiku{{\textquoteleft}'n Ingespannen{\textquoteright},, {\textquoteleft}}{dag zei ze zorgelijk en}{schuw in aarzeling}\\

\haiku{Het hitje bracht met een.}{gewichtig airtje de}{groene brief binnen}\\

\haiku{{\textquoteright} Ongenadig viel:}{ze toen ineens weer tegen}{de kinderen uit}\\

\haiku{{\textquoteleft}Mijndert, rakker, denk,}{er om as ik je nou nog}{\'een keer verbieden}\\

\haiku{{\textquoteright} Het scherpe kuchje.}{van schoonmoeder stootte een}{gat in haar gepeins}\\

\haiku{'n Kip zonder kop '!}{is nogn prefester bij}{j\'ou vergeleken}\\

\haiku{{\textquoteleft}En 't geld voor dat,!}{uitstappie  d\'aar kom je}{ook niet eerlijk \^an}\\

\haiku{Want Sieuwert zag er.}{monter en welgemoed uit}{en hij glimlachte}\\

\haiku{Hij had een felle,.}{klop in zijn keel maar zijn mond}{glimlachte rustig}\\

\haiku{{\textquoteleft}Weet je nog Annet, ','?}{zooals wet gezien hebben}{toe in Amsterdam}\\

\haiku{Eerst stapte hij heen,.}{en weer vijf schreden heen en}{vijf schreden terug}\\

\haiku{Het stadje viel zoo.}{van de eene verwondering}{in de andere}\\

\subsection{Uit: Sterke webben}

\haiku{Ze praatte d'r niet,.}{van liep dicht naast Da voort op}{de deukige weg}\\

\haiku{Heet in de wangen.}{en schuw hoorde ze naar de}{ingesmoorde schimp}\\

\haiku{{\textquoteleft}Moet je Bert de Krey - ',......{\textquoteright}}{s over hooren Bert d\^a-'s}{toch zoo'n dolle z\`eg}\\

\haiku{Mien hoorde d\`an strak '.}{en gespannen toe opt}{rappe gefluister}\\

\haiku{Hi-hi... we moeten.}{ook nog danken voor onze}{avond-boterham}\\

\haiku{Mien merkte d'r niets..., '.}{vant echode gestadig}{na in h'r peinzen}\\

\haiku{Schamig sloop ie naar,.}{h'r toe  drukte z'n wang}{tegen de hare}\\

\haiku{En tot in nek en '.}{voorhoofd begloeide Mienn}{kregele schaamte}\\

\haiku{'n angst - doortokkeld '... ' - '.}{vann vreemde vreugdt Was}{z\'o\'on oogenblik}\\

\haiku{Ja - daar vertelde:}{vrouw Bartelinck vroeger}{zoo'n mooi sprookje van}\\

\haiku{Ja - hij kende h'r,,:}{Vader niet die zou n'tuurluk}{uitvaren ie zei}\\

\haiku{Door h'r peinzen gleed '.}{n gretige verwachting}{en ze ademde snel}\\

\haiku{Ze keek 't wijfje ',.}{verwonderd int ronde}{oolijke gezicht}\\

\haiku{niet zoo, Kostertje,, ',{\textquoteright}}{haast je n\'ou ni\'ett is}{hi\'er nog gez\`ellig}\\

\haiku{{\textquoteright} En als Meurs na 'n -,}{poos w\'eer l\'eep-vorschend over}{Ubbels begon hieuw}\\

\haiku{Zoo'n boel kleuren en -, -,?}{dan zoo'n zoo'n rust over alles}{nu ook n\'u \'ook h\`e}\\

\haiku{t leek vlak-bij, ',.}{inn ruk had ze zich los}{keek verwilderd om}\\

\haiku{{\textquoteright}, morde ze, en h'r, '...}{wangen schoten vol bloed ze}{hoordet t\`och w\`el}\\

\haiku{Mien hoorde 't... en,,.}{ze huiverde dr\`alender liep}{ze m\'eer str\'ompelend}\\

\haiku{Dan weer heftiger ',... {\textquoteleft}}{had zet verlangen naar}{ruimte wind-frischte}\\

\haiku{Bew\'eer {\`\i}k toch niet? ' -..., '...?}{k Bedoel die \'eene di\'e}{eene diet dan is}\\

\haiku{{\textquoteright} - Om 't huis baste, '.}{de luwe winter-windn}{donkere neurie}\\

\haiku{Bij di\'e Keppel, bij...,?}{die \'akelige l\'e\'elukke d'r}{gingen de jongens}\\

\haiku{{\textquoteright} Blozend morde ze '.}{t en h'r oogen ontweken}{z'n gretige blik}\\

\haiku{Even toen ze langs 'm, '.}{ging wreef h'r schou\"er tegen}{t week van z'n borst}\\

\haiku{{\textquoteright} Snel draaide ze 't, '.}{hoofd en onderzoekend keek}{zem in de oogen}\\

\haiku{Je netje ligt nog,...?}{op de regenbak als je}{soms li\'ever w\`eg wil}\\

\haiku{{\textquoteright} 'n Keer toen z'n wang,.}{h'r hand beroerde voelde}{ze warme druppen}\\

\haiku{{\textquoteleft}Co was van al di\'e,..., -..!}{\`anderen de b\`este nog Co}{Bruins die zou h'r m\`an}\\

\haiku{Je worde op - op...{\textquoteright},.}{lange-lest zoo loof van dat}{opstaan zuchtte ie}\\

\haiku{{\textquoteright} H'r wangen werden,.}{donker-rood heel h'r gezicht}{voelde heet en klam}\\

\haiku{{\textquoteleft}Hij deed 't zeker,{\textquoteright}.}{z\`elf \'aarzelig dreef ze h'r}{aandacht daar van weg}\\

\haiku{Bl{\'\i}j met 't voorrecht...{\textquoteright},,.}{spotte ie z'n lach klonk zwaar}{door de stille straat}\\

\haiku{dat je... waar je - waar...? ' - '...}{je moeiluk van weg kunt h\`e}{tt Bekende}\\

\haiku{{\textquoteleft}J\'a-\'a, die n'tuurluk - '...}{n't\'uurluk ook en terdeeg de}{kat int donker}\\

\haiku{Mien ging dan stadig '.}{jachtig doort duisterst van}{de nauwe straten}\\

\haiku{Nee - n\'ee, 't h\'oefde -......, '...?}{toch niet dat ze d\`at weer d'r}{wasr toch n\`og een}\\

\subsection{Uit: Tijne van Hilletje}

\haiku{Het was of ze veel {\textquoteleft}{\textquoteright}.}{te dicht voor eenhel van een}{kachel-vuur stond}\\

\haiku{{\textquoteleft}Zoo, bin je ook al,, '...?}{op'eschote Lijs de wesk al}{int waoter}\\

\haiku{Ze hield de armen.}{kruiselings over de borst en}{lachte in zichzelf}\\

\haiku{{\textquoteleft}As je allebei, ',...}{hulfte hadde maoktet}{niks uit maor n\'ou}\\

\haiku{ik heb de wesk ook ' '}{aledaon en houtjesehakt}{en in de vruugte}\\

\haiku{Maar zij morste of.}{ze nog nooit een tas koffie}{ingeschonken had}\\

\haiku{Als Aorie praatte.}{leek het net of er in zijn}{stem wat neuriede}\\

\haiku{Ze greep Tijne zoo.}{maar de hark uit de handen}{en gooide die neer}\\

\haiku{Ze had het ook over,.}{Keesie van haar eigen had}{Aogie geen kinderen}\\

\haiku{Op het oogenblik,.}{scheen ze niet aanwezig de}{deur was pot-dicht}\\

\haiku{Ze ging een steegje,.}{door en een slop voorbij met}{een schok stond ze stil}\\

\haiku{De wind ging voor het.}{glop staan en blies er een paar}{zware zuchten in}\\

\haiku{Toen blikte ze koud:}{van ontsteldheid om naar het}{gordijn voor het bed}\\

\haiku{{\textquoteright} Een raar schor hikken,...}{zaagde door de stilte mal}{gepiep en gesnuif}\\

\haiku{Mijnt zette met een.}{knorrig keel-geluid haar}{tanden in een schar}\\

\haiku{En Mao spande haar.}{linkerhand breed over Tijne's}{witte klompen uit}\\

\haiku{een klipper met drie:}{gele roefraampjes en de}{man er op floot van}\\

\haiku{Ze hield nog altijd.}{de Heiland vast bij een tip}{van zijn witte kleed}\\

\haiku{terug'ehaold, 't, '...}{is nou zoo zwaort is}{nou weer zoo zwaor}\\

\haiku{{\textquoteleft}...en die loon ontvangt '.}{die ontvangt dat loon inn}{doorboorde buidel}\\

\haiku{{\textquoteleft}Morgen,{\textquoteright} zei ze in, {\textquoteleft}{\textquoteright}...}{haar gedachtenmorgen al.}{Maar dat lag ver weg}\\

\haiku{{\textquoteleft}En den heb ik ook ' '............}{n paor maoledocht ik}{ikke miende v\`est}\\

\haiku{Zonder groet holde,.}{ze de straat op en haar oogen}{zwommen in tranen}\\

\haiku{{\textquoteright} {\textquoteleft}Jao - jao,{\textquoteright} galmde Tao, {\textquoteleft},.}{in de wind opgoddeloos}{net as je zegge}\\

\haiku{{\textquoteleft}Zoo Kaars,{\textquoteright} groette hij, {\textquoteleft},?, '...}{los-wegwaaierig}{h\`en brieke42 lucht}\\

\haiku{{\textquoteright} Tijne moest eventjes,,.}{spieden om te zien of dat}{tegen h\'aar Tao was}\\

\haiku{Toch scheen er dan nog,.}{een geluid van het Eiland}{te komen een schreeuw}\\

\haiku{je 't vraogde,,...?}{en waorveur ook wou jij}{soms-te-met}\\

\haiku{{\textquoteright} Opeens kregen ze,.}{een stoot en ze werden van}{elkaar afgeduwd}\\

\haiku{Naar haar droeg elke,.}{scheldende golf en elke}{booze wind-haal een angst}\\

\haiku{Wat hijgden er toch?}{voor benauwdhedens in die}{diepe wilde zee}\\

\haiku{Warm was het er, en,.}{er dwaalde een geur rond die}{er niet thuis hoorde}\\

\haiku{{\textquoteright} Maar Tijne vergat,}{te antwoorden daar had ze}{die starre oogen weer}\\

\haiku{De hijgende zee.}{schopte de boot voort of het}{een stuk speelgoed was}\\

\haiku{hier heb je 't eerst, '.}{goed in de gaote hoe}{bar datt is}\\

\haiku{{\textquoteleft}Nou most ik toch er ''.}{s vraoge waor nao toe}{dat we gaone}\\

\haiku{Gehaast stapte ze,.}{voort naast Tao ze slofte en}{strompelde van haast}\\

\haiku{En aan het einde - {\textquoteleft}{\textquoteright} -.}{van de straat voorde Doele}{wachtte de stoomtram}\\

\haiku{Telkens was er wat,.}{anders maar alles spoedde}{zich even haastig voort}\\

\haiku{De groote stad kwam op,:}{ze af in de menschen die}{ze tegenkwamen}\\

\haiku{{\textquoteleft}Tao,{\textquoteright} drong zij ineens, {\textquoteleft},'?}{zeg mijn nou waor nao toe}{dat we gaone}\\

\haiku{een deftig man, {\textquoteleft}bas{\textquoteright}46,.}{gekleed en met een rug zoo}{recht als een koning}\\

\haiku{De rooie bloed-oogen.}{van Bappe keken haar aan}{uit Tao's gezicht}\\

\haiku{{\textquoteleft}Nee,{\textquoteright} zei ze hard, en,.}{schoof het bordje met de kom}{een stuk van zich af}\\

\haiku{{\textquoteright} Haar gezicht beefde,,}{en bibberde ze boog zich}{over de koffie heen}\\

\haiku{Ze keken naar het - -...}{Eiland Tao en zij of ze}{ergens naar speurden}\\

\haiku{{\textquoteleft}Keesie Kaors, je,, '.}{vrouw het vemiddag te twee}{ure de geestegeve}\\

\haiku{En, man, ze is in...}{volle verzekerdheid des}{geloofs ontslaope}\\

\haiku{zelfs Lobbetje kreet,!,...!}{mee en Aogie wat schreeuwde die}{niet en Bappe dan}\\

\haiku{{\textquoteleft}Trijntje, je mag toch -... -.}{niet niet in opstand niet al}{te te bedroefd zijn}\\

\haiku{Ze merkte niet eens.}{dat ze op haar kousen liep}{en zonder muts was}\\

\haiku{Tot ze ineens wat...... {\textquoteleft}}{miste en verschrikt stil bleef}{en staarde staarde}\\

\haiku{{\textquoteright} Gedwee stond ze niet,.}{op het was of ze er zich}{toe aansporen moest}\\

\haiku{Tijne gaf daar ook,.}{wel acht op en het had haar}{toch moeten stichten}\\

\haiku{De keukendeur werd.}{met hortjes en stooten een}{eindje open geduwd}\\

\haiku{want als hij nergens,.}{behoefte aan had werd hij}{stug en afkeerig}\\

\haiku{' Memme nog leefde,{\textquoteright}, {\textquoteleft} ''...{\textquoteright}}{schoot het Tijne te binnen}{wat wast toe aors}\\

\haiku{{\textquoteleft}Wat was er an te,?}{zien en te hoore waor}{je van lache moeste}\\

\haiku{{\textquoteleft}As 't keind, dat 'k, ',}{n\'ou draog weern maasie is}{durf ik er niet mee}\\

\haiku{tot in 't derde.}{en vierde lid dergienen}{die Mij haoten}\\

\haiku{Ging de lui-klok?,?}{niet zachter luisterden de}{kleine huizen niet}\\

\haiku{En het eerste wat,,:}{er in haar opkwam toen ze}{weer denken kon was}\\

\haiku{Goeie reis den maor,{\textquoteright}, {\textquoteleft}.}{wenschte ze hem toeen}{behouwe thuiskomst}\\

\haiku{Wat 'n waoi!, en wat ',,, '?}{n lucht ook Gaart minsch heb}{je ooit zoo'n luchtezien}\\

\haiku{{\textquoteleft}Jao-jao, dat bin,, '.}{ik mit je iens Fokkiek}{zel ook vortmaoke}\\

\haiku{Als ze aan Aorie,.}{dacht was het meteen of ze}{aan Aorie's hand liep}\\

\haiku{Een jongen had zijn:}{voorletters in het hout van}{een schutting gekrast}\\

\haiku{Schichtig-snel nam:}{ze de veranderingen}{waar in de kamer}\\

\haiku{Dat witte daor,,:}{dat waore de bedde}{die zwarte bulte}\\

\haiku{Waor zoude ik}{henengaon veur Uwen Geest}{en waor zoude}\\

\haiku{{\textquoteright} Tijne sloop door het.}{bevende huis of het hun}{eigen huis niet was}\\

\haiku{{\textquoteright} Rillend betastte,.}{ze de kachel die kachel}{was zoo koud als ijs}\\

\haiku{Je binne toch niet ',...?}{veer vann iegelijk die}{je anroept Heere}\\

\haiku{Ons heule huisie,,...}{gaot er an Heere God}{ons heule huisie}\\

\haiku{Och Heere, om Uws -,,,.}{naoms wil om Uws n\'aoms wil h\'o\'or}{mijn aomen aomen}\\

\haiku{Het huis kraakte en:}{dreunde of er timmerlui}{in doende waren}\\

\haiku{Grauw-wit was hij in,:}{zijn gezicht een streep bloed liep}{bij zijn kin beneer}\\

\haiku{Makker, je ziene -'...{\textquoteright}.}{toch kijk nao buite Tijne}{luisterde niet meer}\\

\haiku{Haar lippen werden,,.}{blauw haar koonen ook uit niets bleek}{dat ze nog ademde}\\

\haiku{De storm verzwakte.}{gaandeweg en het water}{scheen moete worden}\\

\haiku{het leek al op een,...}{begrafenis het rook daar}{al naar rouw-kleeren}\\

\haiku{Veurzichtig nou, dat,,.}{ze niet stikt laot ik je}{toch helpe minsch}\\

\haiku{Had zij die veurhien59 '?, '?,?}{nietekonne langeleje die}{iendere stemme}\\

\haiku{{\textquoteright} En het was of een:}{mes haar hart openscheurde van}{boven tot onder}\\

\haiku{{\textquoteleft}Schei daor mee uit,, '?}{Tijne-van-Hilletje}{wat baott je}\\

\haiku{De dagen van voor,:}{de storm kwamen naar haar toe}{die dagen heetten}\\

\haiku{En er kwamen geen,.}{dagen op haar toe en geen}{herinneringen}\\

\haiku{Als Aogie tegen een,.}{meerdere sprak praatte ze}{altijd onderdrukt}\\

\haiku{Maar dat stoffelijk, -.}{omhulsel daar daar moet je}{niet zoo aan denken}\\

\haiku{{\textquoteleft}Staon er toch niet, '!}{zoo suf bij je maoken}{minsch tureluursch}\\

\haiku{{\textquoteright} Er stapten menschen,.}{langs haar heen een-ieder ging}{dezelfde kant uit}\\

\haiku{De wind,{\textquoteright} kon Tijne, {\textquoteleft} '?}{ineens weer denkenwaor}{is di\'e nouebleve}\\

\haiku{En de grauwe vloer...}{werd barscher en alles om}{haar heen verkilde}\\

\haiku{Het kwam haar voor of.}{ze verder van alles en}{iedereen afzat}\\

\haiku{En ze keek opeens,.}{niet meer naar de Dominee}{ze keek naar Aorie}\\

\haiku{{\textquoteright}, Tijne gleed haast van,, {\textquoteleft}...}{haar stoel af zoo onthutste}{zebij bij baos}\\

\haiku{Ik - ik was temet,'.}{al weg toe heb juilie me}{weer terug'ehaold}\\

\haiku{Ze most op 'n aor'...{\textquoteright}.}{idee ebrocht worde En al gauw}{vond ze er iets op}\\

\haiku{Ze veegde met de,.}{rug van haar hand haar oogen af}{en keek ontsteld op}\\

\haiku{{\textquoteright}, gaf ze te raden,}{en ze drukte een koud hard}{ding in Tijne's hand.}\\

\haiku{En  n\'ou is 't,.}{de sleutel van jouw weuning}{n\'ou mag jij him houe}\\

\haiku{{\textquoteright} De gedachte trok.}{diepe pijn-plooien in haar}{pipsche gezichtje}\\

\haiku{{\textquoteright} Maar ze had het wel,.}{verstaan want al hoofdschuddend}{lachte ze er om}\\

\haiku{{\textquoteleft}Maot,{\textquoteright} smoesde ze, {\textquoteleft} '' '?}{lacherigheb jet nog}{al naot zin z\'o\'o}\\

\haiku{{\textquoteleft}Die Aogie, di\'e kon wel......}{rooie koonen hebbe en en schik}{in haor leve}\\

\haiku{{\textquoteleft}Me maosie,{\textquoteright} zei, {\textquoteleft}.}{hij in het naderenme}{\`erme kleine maad}\\

\haiku{{\textquoteleft}Gaart,{\textquoteright} zeit hij dan, {\textquoteleft}ik,.}{zel je plaoge tot je l\`este}{aodemtocht minsch}\\

\haiku{Gaart, is 't den, jij',,}{gaone rechtstreeks nao de}{hel minsch veur jou}\\

\haiku{{\textquoteleft}Maor zoo waor,...}{as ik  leef ik weut van}{gien schuld of kwaod}\\

\haiku{Het was een wonder.}{zoo gauw ze de brug over en}{de kerk voorbij was}\\

\haiku{{\textquoteright}, en ze greep achter...}{zich naar steun en deinsde een}{stap of wat terug}\\

\haiku{Tijne drukte haar,,:}{handen voor haar ooren haar}{oogen maar dat gaf niet}\\

\haiku{{\textquoteright} Oplaatst liep Mijnt er - -...}{een paar stappen achter hem}{als vergeten bij}\\

\haiku{Opzettelijk hard.}{stiet Baas Sanders toen ineens}{de keukendeur open}\\

\haiku{{\textquoteleft}Man, dat je toch zoo'n,}{truup draoge kenne in i\'en}{keer maok maor gien}\\

\haiku{{\textquoteright} En amper was hij,...}{vertrokken of ze keek al}{weer om naar Aorie}\\

\haiku{{\textquoteright}, prevelde ze, {\textquoteleft}oud, '.}{wordt die man netn keinsche}{Bappe bij tij\"en}\\

\haiku{het lezen{\textquoteright} boven.}{de breede groene  deur}{van de lichtwachter}\\

\haiku{Je binne nou net,,,...}{zoo gnap as Mijnt nee gnapper}{nog jao v\'eul gnapper}\\

\haiku{De weelderige.}{ansjovis-teelt liep ten}{laatste op zijn eind}\\

\haiku{Haar smartelijke.}{verwondering kon ze niet}{te boven komen}\\

\haiku{{\textquoteright}, mopperde ze, {\textquoteleft}'n '.}{bedoening ook dat die man}{gien-iensn vrouw het}\\

\haiku{Hoe meer haost, hoe,{\textquoteright}.}{meer teugespoed knorde ze}{verontschuldigend}\\

\haiku{hij 'evraogd had, zoo'n,...}{keind was zij toch niet dat ze}{d\`at niet zou vatte}\\

\haiku{In de reuk van het...}{zwart-gebrande pijpje}{was Bappe geweest}\\

\haiku{En toen moest ze haar.}{handen op haar mond leggen}{om niet te schreeuwen}\\

\haiku{{\textquoteleft}Nee, nao' de Baos...'.}{ken ik ook niet hiermee nao}{de Baos ook niet}\\

\haiku{Tijne lichtte de.}{klink van de deur maar de deur}{was ook gesloten}\\

\haiku{{\textquoteright} En meteen, als op,.}{pijn-vlagen kwamen haar}{gedachten terug}\\

\haiku{Het klonk of dat het,,.}{laatste was dat hij zeggen}{zou voor hij wegging}\\

\haiku{En in plaats van te,.}{vertrekken kwam hij nog een}{stap dichterbij}\\

\haiku{{\textquoteleft}En j{\'\i}j,{\textquoteright} stoof het heet, {\textquoteleft}...}{door haar heenj{\'\i}j geve de}{minsche minder den}\\

\haiku{{\textquoteleft}D\`en moet je hier t\`och,,...}{vedaon jao d\`en ken je}{hier \'o\'ok niet weuze}\\

\haiku{Nou maor gauw over,,{\textquoteright}.}{wat aors beginne binne}{nam Tijne zich voor}\\

\haiku{En hij mag den niet -...}{zoozeer mit mit uiterlijk schoon}{toebedeeld weuze}\\

\haiku{Want as je mit z{\'\i}jn ',{\textquoteright}, {\textquoteleft}}{etrouwd binne jolig lachte}{zewat zel hij aors}\\

\haiku{Warme pijnlijke.}{oogen kreeg ze er van en een}{warm pijnlijk hart ook}\\

\haiku{{\textquoteleft}Leer mijn je weg te,,{\textquoteright}, {\textquoteleft} '.}{gaon Heere bad zeik}{wilehoorzaom weuze}\\

\haiku{{\textquoteleft}Je magge er niet,,...!}{an legge te frunneke}{Aont veur de voet op}\\

\haiku{En die blijheid over:}{Aorie's vraag tuimelde er}{verwarrend doorheen}\\

\haiku{Maar ze lachte er,.}{bij en ze zag er weer zoo}{jong uit als ze was}\\

\haiku{{\textquoteright} Maar hoog in de lucht,.}{schaterde de zon en de}{wolken lachten mee}\\

\haiku{{\textquoteleft}Was ze eigelijk ' - ', '... '...?}{wel ooit eerezoendez\'oend zien je}{enan'ehaoldan'eh\'aold}\\

\haiku{Toen Aorie al-lang -.}{de hoek om was stond zij hem}{nog na te kijken}\\

\haiku{Voor de eerste maal.}{in haar leven trok ze een}{uitdagend gezicht}\\

\haiku{En daor wordt de '.}{vruchtbaorheid mee an'etast}{en tenietedaon}\\

\haiku{En - en wil ik den?}{eerst meschien die aorepels}{veur je ofspruite}\\

\haiku{Hij trok haar mee naar,,.}{een stoel ging zitten en nam}{haar op zijn knie\"en}\\

\haiku{{\textquoteleft}Maor wat moete '?}{we er ins heeren naom}{den mee uitrichte}\\

\haiku{{\textquoteright} Tijne deed moeite,.}{om het te verwerken al}{dat onverwachte}\\

\haiku{Toe maor,{\textquoteright} drong hij, {\textquoteleft} '....}{zegt mijn maor op mijn}{ken je betrouwe}\\

\haiku{En soms was het net.}{of ze een heel stuk van de}{anderen afzat}\\

\haiku{En niemand zei er,.}{wat grappigs en geen lach deed}{ze uitgeleide}\\

\haiku{Haast gretig deed ze,,.}{dat ze leunde ook zwaar haar}{hoofd hield ze voorover}\\

\haiku{De Goede Herder,.}{kwam naar haar toe hij stak zijn}{handen naar haar uit}\\

\subsection{Uit: Tusschen twee droomen}

\haiku{Ik drijf weg over een:}{effen waterspiegel en}{ben iets oneigens}\\

\haiku{Ik probeer ook mijn.}{groote broer die dood is in zijn}{lichte oogen te zien}\\

\haiku{{\textquoteleft}Ingelotje - had toch,....}{liever boontjes geweckt dan}{gedichten gemaakt}\\

\haiku{En ik maak ook een.}{sluier van maneschijn en}{wikkel mij daarin}\\

\haiku{Behoedzaam gluren,.}{ze daarbij naar mij eenigszins}{triest-verkennend}\\

\haiku{Van de school valt me,.}{altijd maar weinig op ook}{als ik er naar kijk}\\

\haiku{{\textquoteright} En een fragment uit.}{de geschiedenisles echoot}{ook nog door mijn bol}\\

\haiku{Ik druk de knokkels.}{van mijn verrukte vuisten}{stijf tegen mijn mond}\\

\haiku{Het is de heele.}{dag zoo gegaan met al mijn}{andere vakken}\\

\haiku{{\textquoteright} In de smalle hooge,,.}{huizen die ik passeer leeft}{men als op de teenen}\\

\haiku{Ben je ergens met?,?}{hem geweest ik zag jullie}{uit de Rul komen}\\

\haiku{Maar je moet niet te, -,.}{veel doen dan dan word je te}{moe dat haalt niets uit}\\

\haiku{Pas nu toch op,{\textquoteright} zegt,.}{hij verschrikt en vergeet zijn}{hand weg te nemen}\\

\haiku{En alles wordt dan.}{zoo angst-aanjagend stil}{en vreemd om mij heen}\\

\haiku{Maar het is ook of.}{een droom mijn gedachten in}{een nevelwaas zet}\\

\haiku{Nu denk je niet zelf,{\textquoteright}, {\textquoteleft}.}{veronderstel iknu w\`ordt}{er in je gedacht}\\

\haiku{{\textquoteleft}Mij is dit en dat,!}{niet gelukt tot nog toe maar}{jullie al evenmin}\\

\haiku{{\textquoteright} - Later, als Biechte,:}{en Kennisse mij oprecht}{vervelen denk ik}\\

\haiku{Of zal het Abram zijn,?}{en Sarai zijn huisvrouw en}{Hagar de dienstmaagd}\\

\haiku{{\textquoteleft}Zou je me tot de?}{bodem uitdrinken als je}{me uitdrinken kon}\\

\haiku{Mijnheer Blommers schudt {\textquoteleft} -....}{ons bijna heftig de hand.}{Collega Elmie}\\

\haiku{Als ik dan met dat - -....}{werk werk dat er eigenlijk}{niets toe doet klaar ben}\\

\haiku{Peter heeft vaak die.}{ingehouden lach-trek als}{hij over mijn werk praat}\\

\haiku{een  heerschzuchtig, -.}{in bezit nemend gebaar}{heel heel aangenaam}\\

\haiku{Vroeger ging ik met.}{Vader en Moeder naar een}{hotel aan de Rijn}\\

\haiku{En mijn man neemt zelfs.}{op een verliefde manier}{de pijp uit de mond}\\

\haiku{Hij streelt daarbij zijn....}{manhaftige knevel en}{glimlacht verteederd}\\

\haiku{dat is er een die.}{de handen op de heupen}{zet en zwierig loopt}\\

\haiku{Natuurlijk, we zijn -!}{ergens ergens tusschen Spiez}{en Interlaken}\\

\haiku{soms te ijl, soms wat,.}{afgewend toch stemmen die}{steeds naderkomen}\\

\haiku{De boomkruinen zijn.}{van doorschijnend geel-groen}{kristal in het licht}\\

\haiku{alles voor jou - zoo,?,?}{was het toch h\`e Elmie en}{jij alles voor mij}\\

\haiku{{\textquoteright} Heftig druk ik mijn.}{duim tegen de spitse punt}{van mijn sierspeld aan}\\

\haiku{Onder veel goede.}{pijn wordt er een stramme kracht}{in mij geboren}\\

\haiku{{\textquoteright} Och, dat vergeet ik,.}{al nog terwijl ik er over}{poog na te denken}\\

\haiku{We wandelen veel,.}{maar een groote wandeling maakt}{me ongeduldig}\\

\haiku{ik naar bed ga, zal -.}{ik jullie schrijven morgen}{in ieder geval}\\

\haiku{Het is of de wind,....}{de lippen vast opeen sluit}{een koude vrieswind}\\

\haiku{{\textquoteright} {\textquoteleft}En dan,{\textquoteright} vervolg ik, {\textquoteleft}?}{in stiltegen\'oeglijk over de}{pottenkast praten}\\

\haiku{Je moet je nu maar,}{schrap zetten je moet je zelf}{dan maar aanpakken}\\

\haiku{Eensklaps grijpt hij mijn.}{bovenarm beet of hij mij}{dooreen wil schudden}\\

\haiku{het is ondenkbaar.}{dat er \'een tooneeldirecteur}{bestaat die d\`at neemt}\\

\haiku{Nu heb ik niets meer -.}{over om op door te gaan nu}{heb ik niets meer over}\\

\haiku{De deur is op slot,,.}{de lamp brandt de gordijnen}{zijn dicht geschoven}\\

\haiku{En wij spreken over,.}{ons zelf maar als achter een}{gordijn van nevel}\\

\haiku{- In mijn gedachten,.}{en als mijn werk niet vlot twist}{ik vaak met Peter}\\

\haiku{Onthoud d\`at nu toch:.}{ik wil niet de achterkant}{van het huwelijk}\\

\haiku{{\textquoteright} Verlegen speel ik.}{met iets dat me toevallig}{in de handen komt}\\

\haiku{mijn leven hangt aan -.}{een zijden draad die zijden}{draad is Uw antwoord}\\

\haiku{Ik voel me iets m\'eer,.}{dan een schoolmeisje als ik}{tegenover haar zit}\\

\haiku{Sinds lang zie ik dan ':}{de dingen van mijn kamer}{weers welbewust}\\

\haiku{{\textquoteleft}Dag Peter,{\textquoteright} roep ik, {\textquoteleft},.}{als tegen een dooveik}{ga nog even uit hoor}\\

\haiku{De eene straat na de,.}{andere jacht ik door ik}{haal iedereen in}\\

\haiku{nee, deze handen,.}{heb ik lief en ook de man}{van deze handen}\\

\haiku{{\textquoteright} Mijn kin bibbert een,.}{beetje ik doe mijn best om}{rustig te praten}\\

\haiku{Wij wonen aan een.}{modderig pad en zingen}{een romantisch lied}\\

\haiku{Het jonge kwieke,.}{paardje danst over de weg ik}{zie het duidelijk}\\

\haiku{Ik zou ook wel mijn:}{hand op de schouder van de}{boer willen leggen}\\

\haiku{Ik keer terug naar.... -.}{Waterloo Er wordt lang op}{de gong geslagen}\\

\haiku{{\textquoteright} Gelaten ga ik.}{achter het roodachtige}{aan naar beneden}\\

\haiku{D\`at moet je er voor,.}{over hebben anders is het}{ook het ware niet}\\

\haiku{In de huiskamer,.}{richt de stilte zich hoog op}{als ik binnenkom}\\

\haiku{Op een dag is het.}{of ze nog maar alleen in}{het groote hooge huis is}\\

\haiku{{\textquoteleft}Later zal ik erg,.}{lief voor Peter zijn als eerst}{mijn boek maar klaar is}\\

\haiku{Het ontbijt sla ik -,,.}{over het bespaart tijd n\'ee ik}{heb er geen trek in}\\

\haiku{gewoon gesprek over.}{dagelijksche dingen kan}{ik niet meer volgen}\\

\haiku{{\textquoteright}, de gedachten in.}{zijn oogen lijken zich toch van}{mij weg te buigen}\\

\haiku{Een meisje met een,,!}{roode muts op tja-ja een}{pittig dingetje}\\

\haiku{Daar heb je dus wel....{\textquoteright} {\textquoteleft},{\textquoteright}, {\textquoteleft}.}{tijd voorPeter zeg ik moe}{het is om het werk}\\

\haiku{Ik zou op de vloer,.}{willen stampen ik zou iets}{willen neergooien}\\

\haiku{{\textquoteleft}U hebt als vrouw een -!}{zeldzame gave U kunt}{zwijgen \`en wachten}\\

\haiku{- Soms is het me of.}{ik dagen-lang niet bewust}{opgekeken heb}\\

\haiku{Ze blijft het langst van,.}{ons drie\"en op ze komt het}{eerst weer te voorschijn}\\

\haiku{Dan legt de Prins de:}{hand op Ruprecht's schouder en}{spreekt met luider stem}\\

\haiku{{\textquoteleft}Adjudant Ruprecht,.}{Reintz ik betuig U mijn groote}{tevredenheid}\\

\haiku{In een oogenblik:}{tijds tracht ik het alles in}{mij op te nemen}\\

\haiku{Thuis blikt ze naar de,.}{eenzaamheid om als naar een}{kwaadwillig schepsel}\\

\haiku{Op een avond, voor we,.}{inslapen omhels ik hem}{plotseling heftig}\\

\haiku{De tranen die te,....}{voorschijn springen lijken mijn}{oogen te verschroeien}\\

\haiku{Daarna breekt er een.}{tijd aan dat ik mij nog meer}{in mijn werk opsluit}\\

\haiku{Onderzoekend kan,.}{ik om mij heen zien spiedend}{kan ik luisteren}\\

\haiku{Door de voorspraak van.}{Burgemeester Reimertz wordt}{die straftijd verkort}\\

\haiku{Enkele dagen.}{geleden vroeg Udo Meeken}{mij voor een souper}\\

\haiku{ik zie een glimp van,.}{haar ronde zwarte oogen haar}{wachtende houding}\\

\haiku{{\textquoteright} zegt hij, {\textquoteleft}een lief oud,,.}{huis heel rustig kamers met}{uitzicht op een tuin}\\

\haiku{{\textquoteleft}Ik wil t\`och 's zien.}{hoe zoo'n goederenloods er}{van binnen uit ziet}\\

\haiku{Ik sla het boek open,,.}{een mooie heldere letter}{toch leest aangenaam}\\

\haiku{Ik druk de knoop als -!}{een kleinood tegen mijn wang}{aan Peter's jasknoop}\\

\haiku{{\textquoteright} Maar het mag niet in,.}{mij glanzen het mag alleen}{maar in mij krimpen}\\

\haiku{Ik heb ook weer wat,?,?,?}{scherps in de mond wat is het}{een takje een doorn}\\

\haiku{{\textquoteright}, prevel ik, {\textquoteleft}en - en...?,?}{moet je man ze niet moet je}{niet naar je man toe}\\

\haiku{Ik leg mijn hand open.}{op de tafel en schuif die}{open hand naar haar toe}\\

\haiku{{\textquoteright} Soms denk ik ook dat.}{ik barstjes-oogen krijg en}{zoo'n gekerfde blik}\\

\haiku{{\textquoteright} In mijn gedachten.}{zet ik Udo Meeken ook vaak}{tegenover mij neer}\\

\haiku{Er staat een potplant.}{met dikke roode bloemen}{in de vensterbank}\\

\haiku{Ik leg het vuur aan,,.}{in de schouw dek de tafel}{steek de olielamp aan}\\

\haiku{Een afgetrokken,.}{meisje-op-leeftijd bleek en}{wat te uitgerekt}\\

\haiku{{\textquoteright} ~ *** ~ Langzaam loop {\textquoteleft}{\textquoteright}.}{ik over het kronkelende}{landpad ophuis toe}\\

\haiku{{\textquoteleft}Hij is werkelijk,{\textquoteright}, {\textquoteleft}.}{bruin geworden denk ikzoo}{bruin als in mijn droom}\\

\haiku{Het zal ook wel weer ',,.}{s niet goed wezen met jou}{niet en met mij niet}\\

\haiku{{\textquoteright} Ik schud mijn hoofd en.}{ik bloos als een bakvisch en}{ik wijs hem de weg}\\

\subsection{Uit: Tusschen de menschen}

\haiku{Toevallig had hij.}{dan die dag zijn werk in het}{laantje van de ahorn}\\

\haiku{{\textquoteleft}Bleu was ie altijd,, '.}{geweest en dat was gek maar}{t wier nog erger}\\

\haiku{In de avondstilte.}{viel helder de uurslag van}{de torenklokken}\\

\haiku{{\textquoteright} {\textquoteleft}'t Is hier nou net ',{\textquoteright}, {\textquoteleft} '{\textquoteright}.}{n toom telde Door Reestvijf}{kippen enn haan}\\

\haiku{Ook al wat je hebt,{\textquoteright}, {\textquoteleft},{\textquoteright}.}{praatte hij onvlot doornet}{als ik tot nog toe}\\

\haiku{Hij kan vragen, maar '{\textquoteright}.}{dan heb zijt antwoord nog}{achter haar tandjes}\\

\haiku{{\textquoteleft}Kijk zoo niet,{\textquoteright} drong zij, {\textquoteleft}{\textquoteright}.}{nauw-hoorbaarik krijg er zoo'n}{angstig gevoel van}\\

\haiku{Netteke liet de.}{klink van het hekje achter}{zich  toevallen}\\

\haiku{Luuk keek nog toen er.}{allang niets meer van hen te}{onderscheiden viel}\\

\haiku{Doch hij had dan de,.}{jaren van een man en de}{kracht van een man}\\

\haiku{en kon er toen ook?}{een vroolijk geluid in een}{plasregen wezen}\\

\haiku{Hij wil opgeruimd.}{kijken en hij moet vechten}{tegen zijn tranen}\\

\haiku{Haar fijne rechte.}{schouders krommen als onder}{den last van een kruis}\\

\haiku{Hebben ze niet van?}{aangezicht tot aangezicht}{de Liefde gezien}\\

\haiku{{\textquoteleft}Wel ou\"etje{\textquoteright}, zei,, {\textquoteleft}?}{ie goedig en ie gaf Kees}{een stoelwat had je}\\

\haiku{De Lente in beeld...{\textquoteright},}{Guus glimlachte hij meende}{dat hij er bedaard}\\

\haiku{{\textquoteleft}Minnie, kindje, je,...}{moet er niet boos om wezen}{dat ik dat zei pas}\\

\haiku{{\textquoteleft}... hij gong en weunde,.}{bij de beek Krith die veur an}{de Jordaon is}\\

\haiku{de Heere is nog}{dezelfde machtige God}{van oudsher en gien}\\

\haiku{{\textquoteright} Zijn wrange glimlach,.}{werd breeder uitdagend stak}{hij zijn hoofd vooruit}\\

\haiku{hadden zij misschien...?}{de ster gezien die de drie}{Wijzen de weg wees}\\

\haiku{{\textquoteleft}Christus de Heere,{\textquoteright}, {\textquoteleft}.}{voltooide ze prevelend}{in de stad Daovids}\\

\haiku{Nou was er nog voor,.}{\'een keer koffie en brood en}{dan was alles op}\\

\haiku{{\textquoteleft}En wat had Jaopie... '...?}{ook iedere maol en}{telkens van nuuwsezeid}\\

\haiku{aorepels heb je ',,,.}{ezeid en dat is er meer niet}{en ook niet minder}\\

\haiku{En de andere,...?}{morgen wat denk je dat hij}{toen in zijn netten}\\

\haiku{{\textquoteright} In de verte, rood,.}{vierkant en nuchter rijst het}{Diaconie-huis op}\\

\haiku{{\textquoteleft}Ik dacht eigenlijk '...?, '...?}{als we d\`at nous samen}{h\`e inn hotel}\\

\haiku{Miekie glipte de,...}{straat op ze wuifde nog even}{en haar oogen lachten}\\

\haiku{Het grove tumult.}{van de groote stad verfijnde}{tot een ijl gerucht}\\

\haiku{Door haar tranen heen,, '...}{wou ze toch nog glimlachen}{maart ging niet best}\\

\haiku{{\textquoteright} lichtte ze koddig, {\textquoteleft}.}{inof je de kamer nog}{gedaan kreeg vandaag}\\

\haiku{- Kijken ze niet erg,,?}{vroeg ze direct riepen de}{lui je nog wat na}\\

\haiku{Maar een gewone ',,.}{straatmuzikant wast ook}{niet zie je vast niet}\\

\haiku{En ze had nog geen,.}{vijf stappen gedaan toen haar}{Moeder overeind kwam}\\

\haiku{Eerst na het derde.}{spelletje durfde Carry}{weer op te k{\`\i}jken}\\

\haiku{Met angstige oogen.}{staarde ze plotseling naar}{den langen mijnheer}\\

\haiku{{\textquoteright} Casparis zuchtte,,.}{en hij schudde het hoofd zijn}{oogen tuurden ver heen}\\

\haiku{Frits was het niet eens,...{\textquoteright}}{die snorkte dat was hij zelf}{Ongeduldig trok}\\

\haiku{{\textquoteright}, angst rilde door hem,.}{heen en beklemmende vrees}{neep zijn adem haast af}\\

\haiku{{\textquoteleft}Toe,{\textquoteright} drong ze schattig, {\textquoteleft},, '.}{wees nou geen stoute jongen}{h\`et is je tijd}\\

\haiku{alleen d\`an...{\textquoteright} Met een.}{energieke hoofdbeweging}{brak Do het  af}\\

\haiku{Hij boog zich er naar,,.}{toe gluurde er door heen zijn}{gezicht verstrakte}\\

\haiku{groene testen en,,.}{roode vergieten bekers}{pullen en poppen}\\

\haiku{{\textquoteleft}Me dochter Marie,...,.}{die-eh die ver-g-g\'o\'odt}{je gewoon Meuje}\\

\subsection{Uit: Het wazige land}

\haiku{Ineens resoluut,,.}{draaide ze zich af nam het}{valies op en ging}\\

\haiku{{\textquoteright} Een rimpel trok rond,.}{haar mond haar smalle gezicht}{leek ineens ouder}\\

\haiku{Smoezelend werd er,.}{nog gauw een veete beslist}{een stomp toegediend}\\

\haiku{Jud sloot de deur en,.}{riep Ap bij zich er was geen}{bevel in haar stem}\\

\haiku{Dan krijg je 'n mooie,.}{griffel van me eentje met}{goud papier er om}\\

\haiku{Eenzaam bleef ze een.}{oogenblik achter in het}{gesloten lokaal}\\

\haiku{Verwonderd merkte.}{Jud dat ze al een poosje}{aan het praten was}\\

\haiku{Arie die je tergde,!}{en uitlokte en je dan}{ineens alleen liet}\\

\haiku{In een teedere.}{schoone vergankelijkheid}{omving haar de herfst}\\

\haiku{De bootromp schoof hoog en.}{norsch in het blinkende}{licht van de lampen}\\

\haiku{Ze sloot de oogen en.}{drukte het hoofd tegen de}{pluchen coup\'e-wand}\\

\haiku{Eerst arriveeren...,,,.}{en dan nee geen sprake van}{weg kom je niet hoor}\\

\haiku{Je brak ons dispuut,{\textquoteright}, {\textquoteleft} '.}{schertste hijwe haddent}{juist over de liefde}\\

\haiku{Dat geldt alleen voor,{\textquoteright}, {\textquoteleft}...!}{de afhankelijke vrouw}{gaf ze toeja di\'e}\\

\haiku{Hij voelde Jud's,.}{blik en keek haar plotseling}{recht  in de oogen}\\

\haiku{{\textquoteright} Hij trok haastig een,.}{jas aan en duwde zich een}{pet over de ooren}\\

\haiku{{\textquoteleft}Ja, als je log\'e's,{\textquoteright}, {\textquoteleft} '.}{heen zijn joolde zeen je}{hebtt wat stiller}\\

\haiku{Uit de hooge eiken.}{in het park tikte nog af}{en toe een druppel}\\

\haiku{Ze ontsloot de deur.}{en groette hem vluchtig bij}{het naar binnen gaan}\\

\haiku{{\textquoteright} Met een ruk werd de.}{deur geopend en Ans stak}{haar hoofd naar binnen}\\

\haiku{{\textquoteleft}Wel nee,{\textquoteright} verwierp ze, {\textquoteleft}.}{kribbigmogelijk kun je}{Jud nog wat helpen}\\

\haiku{En dat viel geloof,,......}{ik nooit iemand op Moes niet}{en Pa niet geen mensch}\\

\haiku{Allemachtig, jij, -.}{bent zoo'n heerlijk vrouwmensch zoo'n}{zoo'n echt raspaardje}\\

\haiku{Ik geloof,{\textquoteright} stelde, {\textquoteleft}.}{ze koel vastdat je nu eerst}{ziet wat ik aan heb}\\

\haiku{Ze spreken voor hun,.}{leeftijd werkelijk keurig}{daar ben ik trotsch op}\\

\haiku{Het was of haar stem.}{in haar borst gevangen zat}{midden in een pijn}\\

\haiku{Zeg, doe geen moeite,,...{\textquoteright}}{ga nu liever terug naar}{Ans ik kom er wel}\\

\haiku{{\textquoteright} zei Toonie, hij schraapte.}{een lucifer af en trok}{de vlam in zijn pijp}\\

\haiku{Rusteloos, in haar,:}{groeiende vrees had ze hem}{telkens opgezocht}\\

\haiku{{\textquoteleft}Ik moet hem toch maar ',{\textquoteright}, {\textquoteleft}'}{n briefje schrijven overlei}{ze feln briefje}\\

\haiku{Och blijf ook maar, blijf,,,.}{maar ik ben zoo bang ou\"e hond}{ik ben zoo eenzaam}\\

\haiku{Oh ja, 't was mooi....}{en helder en er dreven}{geen cadavers in}\\

\haiku{Ruw brak ze de brief,....}{open tuurde verward op de}{enkele regels}\\

\haiku{Ze had allang iets,,....}{onrustigs en als ze je}{aankeek ik weet niet}\\

\haiku{{\textquoteright} {\textquoteleft}Morgen,{\textquoteright} zei Jud in, {\textquoteleft}.}{zich zelf en ze glimlachte}{smartelijkm\`orgen}\\

\haiku{Moes, wat onhandig.}{beverig schepte de soep}{uit de terrine}\\

\haiku{{\textquoteleft}Ik zal heusch wel,{\textquoteright}, {\textquoteleft} '....}{weggaan zei ze kleintjesdoe}{jullie dans wat}\\

\haiku{Natuurlijk w\`el zwaar,,...}{zoo'n eerste keer de stad in}{Moes durfde niet mee}\\

\haiku{{\textquoteright} De gillende lach.}{van de wijven snerpte nog}{lang achter Jud aan}\\

\haiku{Plotseling, in een,.}{heete pijn besefte ze}{haar verworpenheid}\\

\haiku{{\textquoteright} ried hij bondig, {\textquoteleft}maar.}{ons tot de zakelijke}{punten bepalen}\\

\haiku{{\textquoteleft}Oh jongen, j\`ongen,,...,}{laat haar niet zoo om God's wil}{om G\`od's wil ontferm}\\

\subsection{Uit: De zondaar}

\haiku{Maar daar moest je nooit,,.}{over praten hoor tegen geen}{mensch natuurlijk niet}\\

\haiku{Beteuterd liep hij,.}{op het raam toe en klepte}{het bovenlicht open}\\

\haiku{Er was ineens wat.}{huiverigs in de schuwe}{stilte die volgde}\\

\haiku{{\textquoteleft}'k Ben schrikkelijk,{\textquoteright}, {\textquoteleft}','?}{laat excuseerde zen}{schande vin u niet}\\

\haiku{{\textquoteleft}Ze kletst zoo,{\textquoteright} mokte, {\textquoteleft} '...}{hijen dat wast toch ook}{weer niet heelemaal}\\

\haiku{{\textquoteleft}En waaraan kon je '?}{nou zien datt al-weer ver}{in Augustus was}\\

\haiku{Anne-Marie.}{tuurde altijd verder dan}{iemand kijken kon}\\

\haiku{{\textquoteleft}'t Was dan wel 'n,....}{uurtje op z'n allerlangst}{maar toch wel aardig}\\

\haiku{{\textquoteleft}Hij zelf, hij zou zich...}{zeker aansluiten bij de}{Geheelonthouders}\\

\haiku{dat zag je toch vaak '...}{genoeg op de kermis in}{t dorp en Zondags}\\

\haiku{Suffig luisterde,.}{hij er naar en leeg staarde}{hij langs alles heen}\\

\haiku{{\textquoteleft}Nou,{\textquoteright} wou hij dan nog, {\textquoteleft}.}{nuchter overleggenze kan}{er best niet wezen}\\

\haiku{Bloo-stil liepen,}{ze het eerste stukje op}{straat dicht bij elkaar}\\

\haiku{{\textquoteleft}En nou wou Moeder,,...{\textquoteright}}{nog wel dat Jan trouwen ging}{gut dat vind ik nou}\\

\haiku{Oh gr\`a\`ag,{\textquoteright} zuchtte ze, {\textquoteleft}.}{uit een voelbare volheid}{van klachtengr\`a\`ag hoor}\\

\haiku{foetsie...{\textquoteright} De laatste,.}{hap cake smaakte hem niet}{meer hij slikte stroef}\\

\haiku{{\textquoteleft}Nou,{\textquoteright} betuigde ze,, {\textquoteleft}.}{verstikt of haar adem even weg}{wasmaar ik jou \'ook}\\

\haiku{Hij strekte zijn beenen,.}{zoo ver mogelijk uit en}{rekte zijn armen}\\

\haiku{{\textquoteright} Dirk lachte slap een,.}{beetje mee zijn mond had er}{wat onrustigs bij}\\

\haiku{Opeens moest hij toen.}{aan de boerderij denken}{en aan zijn Moeder}\\

\haiku{Dirk keerde zijn hoofd.}{met een ruk naar haar om en}{glimlachte vragend}\\

\haiku{Haastig bevoelde,.}{ze toen haar heet gezicht de}{dikke oogleden}\\

\haiku{Zijn warm gezicht wreef,.}{aaiend langs haar arm zakte}{zwaar af naar haar heup}\\

\haiku{{\textquoteleft}Nou,{\textquoteright} weifelde hij, {\textquoteleft}..{\textquoteright},,.}{nee Snel zonder dorst dronk hij}{de heete thee uit}\\

\haiku{{\textquoteright} Toen dacht hij ook weer.}{aan het wachtende werk en}{kwam gejaagd overeind}\\

\haiku{{\textquoteleft}Er komt misschien 'n,}{vacature aan die school}{op de Brouwersgracht}\\

\haiku{{\textquoteleft}Hij denkt al niet meer,.}{aan mijn getob hij is al}{alles vergeten}\\

\haiku{En dat had ze wel, ',...?}{noodig ookn boel liefs en zachts}{en verder verder}\\

\haiku{Ze verborg het woord,.}{diep in haar gedachten en}{het brak toch weer door}\\

\haiku{Vijftien golden met,{\textquoteright},.}{opslag zei ze kort kleurend}{trok ze haar hand los}\\

\haiku{{\textquoteright}, hij slikte een paar.}{maal achtereen tegen een}{dikte in zijn keel}\\

\haiku{Op 'n goeie dag ben,.}{je opgestapt nou moet je}{niet meer omkijken}\\

\haiku{Ja, hai docht maar over, -.}{z'n studie en den den mos}{je je jakes hou\"e}\\

\haiku{{\textquoteright} Hij hief zijn hoofd op,.}{het was of hij het tegen}{de wind aan drukte}\\

\haiku{En ze voelde zijn,... {\textquoteleft}}{glijdende hand zwaar en warm}{over haar borst haar heup}\\

\haiku{{\textquoteleft}Hij besliste of,.}{zij er niet was hij betrok}{er haar haast niet in}\\

\haiku{En dan nog, wat je,...{\textquoteright} {\textquoteleft}?}{voor kleeren berekende z\'o\'o'n}{bagatelDacht je}\\

\haiku{{\textquoteright}, voor het eerst had ze, {\textquoteleft}!}{weer wat joligs in haar stem}{dan vergis je je}\\

\haiku{Trouwens, z\`elf wou hij,,...{\textquoteright}}{toch ook zoo enorm graag god nou}{zoo met z'n twee\"en}\\

\haiku{Voor 'n verzetje '.}{zou hij misschiens met Jans}{Faber gaan praten}\\

\haiku{Dirk zich dieper naar,,}{haar toe hij zoende heftig}{haar heele gezicht}\\

\haiku{'t Bijt me de keel,,,?}{af wil je dat wel gelooven}{altijd bij je h\`e}\\

\haiku{En Dirk tikte al.}{tegen de paarse ruitjes}{van de keukendeur}\\

\haiku{naaktheid zou hem nou, '...}{denkelijk niet prikkelen}{niets inn stemming}\\

\haiku{,{\textquoteright} een zwaar gevoel steeg,.}{naar zijn hoofd en er kwamen}{vlekken voor zijn oogen}\\

\haiku{Nou, die jongen zou, '.}{toch wel niks gemerkt hebben}{n  groote stoetel}\\

\haiku{Enfin, als hij voor '.}{t toelatingsexamen van}{de H.B.S. maar slaagde}\\

\haiku{En de directeur:}{had hem nog in z'n eigen}{kamer geroepen}\\

\haiku{{\textquoteleft}wat ben ik blij om, '...}{je je hebtt zoo verdiend}{in al die jaren}\\

\haiku{Terloops keek hij maar,.}{naar hen en zag aan ieder}{toch wat ongewoons}\\

\haiku{De schoenmaker van,{\textquoteright}, {\textquoteleft}.}{beneden troefde Toos op}{haar beurtkomt ook niet}\\

\haiku{Van Hasselt blies een.}{mond vol rook uit en snoof}{taxeerend in de geur}\\

\haiku{{\textquoteright} {\textquoteleft}En bezeerde de,?}{man die de ruiten insloeg}{zijn handen niet erg}\\

\haiku{{\textquoteleft}Er zal over jou wat,{\textquoteright}.}{gepraat worden giste ze}{grappig-verwaand}\\

\haiku{De koelte raakte.}{zijn heete oogen aan en zijn}{gloeierig voorhoofd}\\

\haiku{Ze struikelde haast,,.}{bij zijn onverhoedsche greep}{en schoot in een lach}\\

\haiku{En stil lachte ze,.}{mee maar haar lippen voelden}{stijf en trekkerig}\\

\haiku{Ze moest Mevrouw van.}{Haaften ook nog voorstellen}{aan Mevrouw Hubbink}\\

\haiku{En dan de eerste,.}{tien minuten moest ze ook}{niets presenteeren}\\

\haiku{Enfin, ze zou wel... '}{een en ander afneuzen}{van die anderen}\\

\haiku{{\textquoteright} Dirk luisterde, of,.}{hij haar al hoorde komen}{maar alles bleef stil}\\

\haiku{Wrijf jij je gezicht,,...}{maar liever wat af met je}{zakdoek je glimt zoo}\\

\haiku{{\textquoteright}, vorschte ze, voor,, {\textquoteleft} '?}{haar doen levendigkomtt}{Woensdag gelegen}\\

\haiku{Nou doe je maar net,,}{of panje-drinken je}{dagelijksch werk is}\\

\haiku{{\textquoteleft}Branders,{\textquoteright} haalde Jans, {\textquoteleft},,...{\textquoteright}}{uitnee idioot zeg zoo als}{die zich uitsloofde}\\

\haiku{Dat malle gezwam,.}{over die Branders daar werd je}{wee om je hart van}\\

\haiku{{\textquoteleft}O ja, was waar ook,...{\textquoteright} {\textquoteleft}?}{die ziekteBen je nu weer}{heelemaal beter}\\

\haiku{Misschien nog maar 't ',{\textquoteright},}{beste omt te negeeren}{sloeg het door hem heen}\\

\haiku{Och, die soort dingen,, '...{\textquoteright}}{begrijp jij toch niet Hartsen}{jij begrijptt niet}\\

\haiku{{\textquoteleft}'t Zal me smaken,{\textquoteright},.}{zei hij goedig zijn glimlach}{mislukte toch nog}\\

\haiku{om die japonnen,{\textquoteright}, {\textquoteleft},}{te doen zei ze wat zachter}{ik wou maar zeggen}\\

\haiku{{\textquoteright} Teuterig sneed ze,.}{een boterham aan reepjes}{praatte  zeurig}\\

\haiku{want Toos vond dat zoo.}{aardig staan voor de buren}{die er op letten}\\

\haiku{Gek, datzelfde had,,.}{hij vanmorgen op weg naar}{school ook al gedacht}\\

\haiku{En hij had al 'n, '...}{paar keer gedroomd dat hijn}{jongen wurgde}\\

\haiku{{\textquotedblleft}Winde-kind,{\textquotedblright} en, ':}{stereotiep hoorde je na}{Van Looy alsn echo}\\

\haiku{En trouwens, ik had '.}{er geen idee van datt jou}{zou irriteeren}\\

\haiku{ik word 'n duvel,...}{op school en eerst mocht ik ze}{toch graag de jongens}\\

\haiku{Er zette zich een.}{trek van wrevel vast in zijn}{magerbleek gezicht}\\

\haiku{{\textquoteleft}Dwarskijker,{\textquoteright} gromde,.}{hij en zijn donkere stem}{had een warme klank}\\

\haiku{Mokkend ging hij de,.}{kamer uit en op de trap}{bleef hij telkens stil}\\

\haiku{{\textquoteright} Hij struikelde haast,.}{over een steen en schopte die}{nijdig uit de weg}\\

\haiku{Verdomme,{\textquoteright} mokte, {\textquoteleft},,.}{hij schorverdomme toe doe}{dat nou niet kerel}\\

\haiku{Hij vloekte zwaar in,.}{zijn binnenst z\'o\'o zwaar dat hij}{er van hijgen moest}\\

\haiku{t Eigenlijke,:}{begin was er nooit geweest}{dat moest nog komen}\\

\haiku{Jij hebt 't liever,{\textquoteright}.}{niet dan wel bepaalde hij}{toen onomwonden}\\

\haiku{we zaten veel te ',...{\textquoteright}}{krap om aann gezin te}{denken kinderen}\\

\haiku{Als je 't aanstonds,, '.}{druk hebt met je werk op school}{vergeet jet weer}\\

\haiku{ik voel 't onder,, '.}{m'n lessen door even goedt}{is geen bevlieging}\\

\haiku{En Toos wendde als...}{een schuchter jong-meisje}{haar hoofd van hem af}\\

\haiku{Daar had hij toen niet,,?}{op geantwoord nee wist hij}{van die dingen af}\\

\haiku{Het was hem ontgaan.}{dat Jans op zijn verweer niet}{eens geantwoord had}\\

\haiku{En de menschen die',...}{ben best tevreden met de}{schijn dat is alles}\\

\haiku{{\textquoteright} Met klein getrokken,,.}{oogen gluurde hij tegen het}{licht in de straat op}\\

\haiku{De steenen spiegelden,.}{van zon de boomen werden}{al  kaal en zwart}\\

\haiku{zijn ook nog koekjes ',:}{int buffet of als je}{wat anders wenscht}\\

\haiku{Hij hoorde meteen.}{haar zagerige adem en}{de knars in haar mond}\\

\haiku{Want wist ineens dat.}{achter die gedachte zijn}{begeerte wegschool}\\

\haiku{Di\'e moest hij vragen,,,...}{en dan draaide ze hem haar}{wang toe nou ja T\'oos}\\

\haiku{{\textquoteleft}Geef me nou 'n zoen,{\textquoteright}, {\textquoteleft} '?}{hunkerde zegeef me nou}{voor \'e\'en keern zoen}\\

\haiku{Angst spiegelde in... {\textquoteleft}}{de helle schuldbewuste}{oogen van de jongen}\\

\haiku{En het leek wel of.}{hij met de jongen alleen}{in het lokaal was}\\

\haiku{Die jongen pestte.}{net zoo lang tot je je niet}{meer in kon hou\"en}\\

\haiku{Telkens stond er een.}{lange booze stilte tusschen}{hem en een leerling}\\

\haiku{hij maakte van de,...}{stug-verwijtende oogen}{heet-bedeesde}\\

\haiku{{\textquoteleft}Ik wist 't wel, ik '...{\textquoteright}:}{wistt En het leek hem geen}{werkelijkheid meer}\\

\haiku{In elk geval, er,... '!}{was tenminste niks gebeurd}{niksn Geluk maar}\\

\haiku{Toch praatte hij, in.}{zijn baloordheid tegen haar}{of ze naast hem liep}\\

\haiku{{\textquoteright} Eerst ging hij op zijn,.}{stoel bij de haard zitten en}{toen op de divan}\\

\haiku{Vroeger zou ik daar...,{\textquoteright}.}{niet eens aan gedacht hebben}{en n\'ou ze snikte}\\

\haiku{Of zooals die man met,...{\textquoteright}.}{het lupusgezicht laatst in}{de tram Ze rilde}\\

\haiku{Hij leek het niet eens,.}{te hooren keek zoekend om}{zich heen op de grond}\\

\haiku{Hoe lang waren ze?, '...}{nou al hiert leek wel of}{er geen tijd meer was}\\

\haiku{Mit dat klaine kriel,{\textquoteright}, {\textquoteleft}}{schepte hij opken ik ook}{net zoo goed overweg}\\

\haiku{Haar mond viel open, het,.}{leek een pijn-vertrekking}{het was een glimlach}\\

\haiku{Nou nou bin je er,?, -?}{toch niet kwaad om hee dat ik}{dat ik d'r over praat}\\

\haiku{'n mot is maar 'n...}{beesie van niks en hij bait de}{fainste dingen stik}\\

\haiku{, spijt waar je niks mee -...{\textquoteright}}{goedmaken kon en dat dat}{zeere in je borst}\\

\haiku{Gos j\'ongen,{\textquoteright} haalde, {\textquoteleft} - '!}{ze uitik ik weet mit skik}{niet hoe ikt heb}\\

\haiku{Maar onmiddellijk,.}{er-op bedwong ze zich en}{groette kalmbeleefd}\\

\haiku{{\textquoteleft}'t Gewicht is er,.}{ook niet slager en ik moet}{nog m'n centen w'rom}\\

\haiku{{\textquoteright} Dirk lachte gesmoord.}{in het donkere hoekje}{van zijn elleboog}\\

\haiku{Nou, als we eerst maar,.}{weer in huis benne dan kan}{ie ons niet krijgen}\\

\haiku{{\textquoteleft}We gaan eten, hier, doe, ',.}{maar gauw alsn knappe meid}{je slabbetje voor}\\

\haiku{Toch wonderlijk dat,...}{ze niet kwaad gebleven was}{na die avond op straat}\\

\haiku{{\textquoteright} De herinnering...}{aan de sterfnacht van zijn Moeder}{vlamde door hem heen}\\

\haiku{{\textquoteright} Afkeerig keek hij.}{naar het bleeke gegnies in haar}{slap-dik gezicht}\\

\haiku{Ze praatte door, of.}{ze het tegen  iemand}{had die naast Dirk zat}\\

\haiku{Morgen moet je maar '....}{weern extra beurt hebben}{zit niks anders op}\\

\haiku{, tusschen dames die.}{allemaal gezellig met}{z'n twee\"en zaten}\\

\haiku{Wat 'n idee ook om,?}{hier langs te komen niemand}{koopt immers van je}\\

\haiku{{\textquoteright} Meteen nam ze het.}{boordevolle theekopje}{op en wou weggaan}\\

\haiku{{\textquoteright} Kregel onderbrak.}{Toos ineens het aanpraten}{van de winkelchef}\\

\haiku{{\textquoteright} Rusteloos zochten,,.}{haar oogen over en weer in de}{straat de winkels af}\\

\haiku{Ze vergat ook te {\textquoteleft}}{groeten en hield de knop van}{de deur in haar hand.}\\

\haiku{{\textquoteright} Een warm gevoel schoot,.}{in haar op een gevoel als}{een herinnering}\\

\haiku{n Tijd terug leek,,...}{dat uurtje verleden week}{mooie les boeiend wel}\\

\haiku{t neemt 'n massa, ' '.}{tijd ent kost enkel maar}{n beetje kleefstof}\\

\haiku{En Dirk knikte als,.}{een harlekijn of zijn hoofd}{aan een draadje zat}\\

\haiku{En Dirk drukte vrij.}{hartelijk Iet's kil-gladde}{hand in de glac\'e}\\

\haiku{{\textquoteright}, joeg het onthutst door, {\textquoteleft} '...}{hem heennou zie jet zelf}{en \`als je sterk was}\\

\haiku{{\textquoteright} {\textquoteleft}Ja,{\textquoteright} gaf ze snibbig, {\textquoteleft} '.}{toedan zou er weern jaar}{priv\'e opzitten}\\

\haiku{{\textquoteright} {\textquoteleft}Hil,{\textquoteright} kwam hij er dan, {\textquoteleft}?}{toch nog tegen ophoe kom}{je nou op zoo iets}\\

\haiku{{\textquoteleft}Wat zou er anders,{\textquoteright}, {\textquoteleft}?}{wezen drong hij in vreemde}{bevangenheidHil}\\

\haiku{Haast doorloopend, de,, '...}{laatste tijd soms eventjes niet}{n\'ou mett gedicht}\\

\haiku{{\textquoteright} Haar schuw-warme blik.}{ging niet hooger dan tot het}{front van zijn overhemd}\\

\haiku{{\textquoteright} Zijn trieste lachje.}{bleef hem als een krop van leed}{steken in de keel}\\

\haiku{Even was het nog of,,.}{ze schrok tegenspartelen}{wou toen bleef ze stil}\\

\haiku{{\textquoteright} {\textquoteleft}Nee,{\textquoteright} protesteerde, {\textquoteleft} ',.}{ze gegriefdik zegt niet}{om te vleien hoor}\\

\haiku{Hil, lieve kind, na....}{deze eene keer doen we doen}{we weer als eerst hoor}\\

\haiku{{\textquoteright} Lui weerde hij een,.}{langpootige vlieg af zijn}{arm viel slap terug}\\

\haiku{duinzand flonkerde,,...}{helmplanten wiegelden ze}{hoorden wind en zee}\\

\haiku{Gekke stoethaspel,.}{om zoo tegen hem te doen}{ijselijke prul}\\

\haiku{{\textquoteleft}Vast 's kijken zoo,?}{meteen hij zou toch niet iets}{gebroken hebben}\\

\haiku{{\textquoteleft}Ja, nou moest hij zoo,......{\textquoteright}}{meteen opstaan en eten en}{kijven tr\'eiteren}\\

\haiku{{\textquoteleft}En altijd,{\textquoteright} sufte, {\textquoteleft}.}{hij doorkon je alles nog}{vierkant ontkennen}\\

\haiku{je bent...{\textquoteright} {\textquoteleft}Dat,{\textquoteright} knikte, {\textquoteleft}:}{hij tegemoetkomendheb}{je straks al gezegd}\\

\haiku{{\textquoteleft}'t Zou toch niks voor,,.}{mij wezen zoo vervelend}{geen mensch die je ziet}\\

\haiku{Nou ja, of op je, '}{eentjet is merkwaardig}{wat je zoo buiten}\\

\haiku{Nou omdat je er,.}{zoo op aan tamboereert zal}{ik gaan zoo meteen}\\

\haiku{{\textquoteright} {\textquoteleft}Tut-tut,{\textquoteright} pruttelde, {\textquoteleft},.}{hij binnensmondsik knijp er}{tusschen uit strakkies}\\

\haiku{Die goed voor je zorgt,{\textquoteright}, {\textquoteleft}:}{prevelde hij nanou ik}{zal niet ontkennen}\\

\haiku{{\textquoteright} Zijn tanden sloegen,.}{op elkaar met een klikkend}{geluid hij knikte}\\

\haiku{Dat - dat doe je maar -,?}{om wat wat goed te maken}{van vroeger ni\'et}\\

\haiku{ze heeft de slanke,.}{lijn tot in de perfectie}{nou di\'e kan meedoen}\\

\haiku{Hij betaalde de,.}{conducteur en bleef op het}{achterbalkon staan}\\

\haiku{Ruw gooide hij de,.}{ramen open en zakte zwaar}{neer in zijn crapaud}\\

\haiku{{\textquoteright} En Dirk tuurde zoo,.}{afwezig voor zich uit of}{hij niets gehoord had}\\

\haiku{'n Lichte stap had,.}{dat kind zoo of ze de vloer}{haast niet aanraakte}\\

\haiku{En snel boog hij zich,...}{naar het blaadje op zijn knie}{las over fouten heen}\\

\haiku{{\textquoteright} Het gesprenkelde.}{bloemige rose van haar}{koonen werd wat dieper}\\

\haiku{Pas op,{\textquoteright} dreigde hij, {\textquoteleft} ', '.}{ik kant uit je vandaan}{kn{\'\i}jpen alst moet}\\

\haiku{{\textquoteright} {\textquoteleft}Er is ook nog wel ' ',{\textquoteright}.}{n zoetigheidje int}{buffet bood hij aan}\\

\haiku{Hij keek haar enkel,,.}{maar aan star met toch iets van}{een lach om zijn mond}\\

\haiku{Ik kon me op laatst,.}{haast niet meer inhou\"en als}{je langs me ginge}\\

\haiku{{\textquoteright} Haar handen gleden,,... {\textquoteleft}}{over hem heen omvatten hem}{in bezit nemend}\\

\haiku{{\textquoteright} Zoo zag ze er uit,.}{als een furie maar dan toch}{als een mooie furie}\\

\haiku{{\textquoteright} En meteen was het.}{of de tik van de klok haar}{oorvlies aanraakte}\\

\haiku{, had h{\'\i}j nou ook niet?}{allerstomst zitten}{soezen al die tijd}\\

\haiku{{\textquoteleft}Dit laten duren - '.}{nogn beetje dat goeie van}{hem laten duren}\\

\haiku{Hij merkte het op,.}{en zijn adem schokte of hij}{innerlijk lachte}\\

\haiku{{\textquoteright} Een verontrustend:}{gevoel van onvoldaanheid}{bleef in haar achter}\\

\haiku{En Dirk knikte wel,:}{zwoel-vermaakt maar hij zei}{stroef uit de hoogte}\\

\haiku{Met een verbeten...}{lach zag Dirk van haar naar de}{binnenkomenden}\\

\haiku{Kerel,{\textquoteright} luchtte hij, {\textquoteleft}}{eensklaps netelig-oolijk}{zijn verwondering}\\

\haiku{Gunst en dat vind ik,:}{toch heel  gewoon dat heb}{ik wel meer gedaan}\\

\haiku{Als je nu met de?}{vacantie maar niet te moe}{bent om uit te gaan}\\

\haiku{{\textquoteleft}Als ik die man op, '.}{straat tegenkom kijk ikn}{andere kant op}\\

\haiku{{\textquoteright} Haar kleine diepe.}{oogen hielden zelfs onder het}{lachen wat vinnigs}\\

\haiku{Ze zag er in haar.}{witte toiletje nog altijd}{als een meisje uit}\\

\haiku{{\textquoteright} Zijn wit smal gezicht.}{stond z\'oo strak of hij op een}{begrafenis kwam}\\

\haiku{Jij 'n cognacje,?,?}{Hubbink en Bollema wat}{voor vergift kies jij}\\

\haiku{Ik heb 't ontdekt,!}{en ik kijk al twee weken}{naar hun kaartjes uit}\\

\haiku{{\textquoteleft}Oh natuurlijk, de,.}{school en de jongens anders}{wisten ze ook niet}\\

\haiku{De pendule op.}{de schoorsteenmantel sloeg scherp}{en rap de uren af}\\

\haiku{Toos was in een dun:}{hemd-met-kantjes}{bij hem komen staan}\\

\haiku{Hij praatte - opdat -.}{Diet er niets van denken zou}{expres snauwerig}\\

\haiku{leuk dat die haar zoo,, '...}{opzocht moederlijk mensch echt}{n lieve Moeke}\\

\haiku{Ze keek van het groote.}{kil-witte naambord naar de}{norsche hardsteenen stoep}\\

\haiku{Aan alles in  .}{het huis leek iets van angst en}{pijn vast te zitten}\\

\haiku{'t Is heel moeilijk ',,.}{omt uit te leggen u}{begrijpt h\'eel moeilijk}\\

\haiku{Maar in zijn stem kwam.}{haar enkel zijn ernstige}{goedheid tegemoet}\\

\haiku{Och, ik was - ik - ik,...}{ben niet sterk dokter en zoo}{zenuwachtig ik}\\

\haiku{Ook dat dan nog maar ', - - '.}{alst moest ook dat alles}{alst baten kon}\\

\haiku{Ze hoorde er zich,:}{al over spreken tegen Dirk}{zag hem opschrikken}\\

\haiku{{\textquoteright} Het verzonk meteen.}{weer in de gewichtigheid}{van het oogenblik}\\

\haiku{Ze gunde zich niet,}{eens de tijd eerst haar hoed af}{te zetten viel zooals}\\

\haiku{Weer 's ouderwetsch...,{\textquoteright}.}{met z'n beidjes vanavond ze}{glimlachte onvast}\\

\haiku{M'n slaapmiddeltje,{\textquoteright}, {\textquoteleft}...{\textquoteright}}{ook nog hakkelde ze in}{een snikzoo'n hoofdpijn}\\

\haiku{Het was of hem een}{gloed tegemoet sloeg uit haar}{donker-verhit}\\

\haiku{{\textquoteright} Met een nijdige.}{kopruk maakte ze zich los}{van haar tobberij}\\

\haiku{Zijn plekkerig-rood,.}{gezicht leek op te zwellen}{hij zweette hevig}\\

\haiku{Ze kreunden als van,,.}{pijn hun handen knepen ze}{deden elkaar zeer}\\

\haiku{{\textquoteright} Diet kwam hard neer, ze,:}{steunde en ze lachte er}{onzinnig doorheen}\\

\haiku{Grinnikend reikte,.}{hij haar het dampende glas}{kwam naast haar zitten}\\

\haiku{Bedaarder leven......}{weer kregen die uurtjes met}{Diet weer meer waarde}\\

\haiku{{\textquoteleft}God, z'n kop en z'n,',.}{rug alles dee zeer dronken}{waren ze geweest}\\

\haiku{{\textquoteleft}Ja, de heele nacht '...}{bloot gelegen en dan met}{de deur opn kier}\\

\haiku{of zat ze misschien?}{bij de gedekte tafel}{op hem te wachten}\\

\haiku{{\textquoteright} Toen de buitendeur,.}{weer dichtviel werd de stilte}{hem toch te machtig}\\

\haiku{je z-zal toch...,?,}{wel weer je wordt natuurlijk}{weer beter h\`e Diet}\\

\haiku{Ze zou ook wel 'n, '.}{boel verdriet hebben maart}{niet laten blijken}\\

\haiku{In de keuken nam,.}{hij allerlei dingen op}{die hij niet noodig had}\\

\haiku{Zenuwachtig liep,.}{hij naar zijn slaapkamer keek}{op de pendule}\\

\haiku{, reeg zijn schoenen dicht,.}{en liep op de teenen naar Diet's}{kamertje terug}\\

\haiku{Dirk rekte zijn hals,.}{zocht gemelijk-aandachtig}{naar Moeke en Toos}\\

\haiku{Werachtig j\^o, we ',.}{hebt-er wat vaak de praat over}{had je vrouw en ik}\\

\haiku{{\textquoteleft}Ziezoo,{\textquoteright} trachtte Dirk, {\textquoteleft}.}{dan nog vriendelijk-blij}{te zeggenweer thuis}\\

\haiku{Beducht luisterde,.}{hij onder aan de trap maar}{hij kon niets verstaan}\\

\haiku{, ja, jij denkt wel erg?, '.}{humaan h\`e wel int oog}{vallend menschlievend}\\

\haiku{{\textquoteleft}Altijd moest ze je!}{er aan herinneren dat}{zij er ook nog was}\\

\haiku{{\textquoteright} Stiekeme lol kneep,.}{slap-dikke lach-plooien}{aan zijn oogen zijn mond}\\

\haiku{Hij ging er zoo in,.}{op dat hij haast zonder groet}{de kamer uitliep}\\

\haiku{Verdomd waar, al dee',:}{je nog zoo hondsch tegen haar}{di\'e speelde toch maar}\\

\haiku{{\textquoteleft}Tevreden?, 'k zit '.}{nog wel metn paar losse}{steekies \^an je vast}\\

\haiku{{\textquoteleft}Maar nou 't \^an h\'aar,.}{leit durf je maar \'een keertje}{in de acht dagen}\\

\haiku{Het was of hij haar,.}{beet pakken wou en het bleef}{maar een beweging}\\

\haiku{, en ze was er toch,...}{even fijn van gebleven geen}{haar minder lekker}\\

\haiku{{\textquoteleft}Hij moest goed voelen, '}{wat hij aan haar had wat hij}{missen zou alst}\\

\haiku{{\textquoteleft}Anders was 't ook....}{zoo licht in de kamer met}{al die zonneschijn}\\

\haiku{in je netheid ben...}{je gemeener dan die meid}{die dat wel moest doen}\\

\haiku{Nou, als je me noodig,,...}{hebt je hebt maar te bell'n}{ik ben er op slag}\\

\haiku{God, god, zooals ik, is - '. '}{er nog nooit nog nooitn vrouw}{vertrapt en verguisd}\\

\haiku{En onthutste van.}{de verbijsterde wanhoop}{in zijn grijns-op-haar}\\

\haiku{Met z'n tw\'ee\"en over,{\textquoteright}, {\textquoteleft}.}{stond hij gek te prevelen}{met z'n tw\'ee\"en \'over}\\

\haiku{Heul maar weer met Jan,,.}{loop hem achterna ga mee}{naar de directeur}\\

\haiku{Mijnheer De Rijck!}{di\'e kwam toch ook uit zichzelf}{op bezoek bij hem}\\

\haiku{niet glimlachend, maar,...}{ook niet stroef-ernstig want dat}{leek gauw op bangheid}\\

\haiku{{\textquoteright} Langzaam, met beenen zoo,.}{zwaar als lood klom hij de hooge}{blauw-steenen trap op}\\

\haiku{{\textquoteleft}En Toos die 't zoo,...{\textquoteright}}{mooi liet voorkomen hem dit}{ook nog leverde}\\

\haiku{{\textquoteright} Hij moest onderhand:}{ook aldoor schichtig letten}{op de anderen}\\

\haiku{{\textquoteleft}Is dat eigenlijk,,...?}{niet gek voor de eerste keer}{op je verjaardag}\\

\haiku{{\textquoteright} En Dirk boog zich over.}{het tafeltje heen of hij}{niet goed gehoord had}\\

\haiku{, ik maak altied in, ',...}{dat ben ik zoowend van thuus}{snieboon'n en zuurkool}\\

\haiku{Toe zeg nu 's, wat?,,, '?}{zal je nemen Samos port}{n advocaatje}\\

\haiku{{\textquoteright} En al-door was.}{er wat onderdanigs in}{zijn doen en laten}\\

\haiku{{\textquoteleft}Net of ze onder ',.}{n ban zaten ze moesten maar}{gauw aan de borrel}\\

\haiku{{\textquoteright} spiegelde ze Toos, {\textquoteleft} '...{\textquoteright}:}{voormetn heel mooi bloemstuk}{Moeke pruttelde}\\

\haiku{{\textquoteleft}Wat had jij dan ook?}{die verwenschte Moeke}{er bij te vragen}\\

\haiku{En wat moet die vent?,!}{hier kan me niet schelen wat}{hij te zeggen heeft}\\

\haiku{Hij schrok op van een,:}{geluid dichtbij en oogde}{sufverwezen rond}\\

\haiku{{\textquoteleft}Oh nou, die h\^et de... '.}{kuiten genoment heb}{ik hem ook gezeid}\\

\haiku{{\textquoteleft}Nou ja, na wat er -......}{wat er voorgevallen was}{waar je om weg moest}\\

\haiku{'t was niet eens van,...{\textquoteright}.}{jou dat gevalletje Even}{hing er een stilte}\\

\haiku{Maar Diet duwde hem,:}{verder ze hing zwaar tegen}{hem aan en smoesde}\\

\haiku{Maar 't kwam nou mooi, ' ':}{voor mekaart werd nou net}{zooalst wezen moest}\\

\section{Martha S. Bakker en Mieke B. Smits-Veldt}

\subsection{Uit: In een web van vriendschap}

\haiku{kijken die hier kwam,.}{maar we verzwegen dat het}{voor hen was gedaan}\\

\haiku{Nu, van dit naar iets,.}{anders als ik het maar niet}{te lang voor je maak}\\

\haiku{Van zo'n saaie boel als,.}{het hier nu is heeft men zijn}{leven niet gehoord}\\

\haiku{En nu krijg ik een,;}{brief uit Antwerpen van de}{vierentwintigste}\\

\haiku{de koning wil ons,:}{helpen met woorden en met}{werken dubbelop}\\

\haiku{En nu is deze;}{bode toch gebleven en}{reist pas morgen af}\\

\haiku{Song ~ Vergeet niet.}{te schrijven wanneer je denkt}{terug te komen}\\

\haiku{Song, gisteren ben!}{ik bij jouw familie op}{je kamer geweest}\\

\haiku{Onderweg zijn er.}{vijf schepen met Duinkerkers}{aan boord gekomen}\\

\haiku{We hebben alle.}{drie gehuild toen we afscheid}{van elkaar namen}\\

\haiku{Wij hebben op dit.}{moment meer hoop dan ooit voor}{zijn carri\`ere}\\

\haiku{Je zou gezonder.}{zijn als je wat botter was}{of een beetje gek}\\

\haiku{Ik verlang net zo,?}{erg als gij kunt verlangen}{maar wat moet ik doen}\\

\haiku{Hij zei dat door de.}{val van Adam een moordenaar}{daar niets aan kon doen}\\

\haiku{Kinderen die slim,.}{zijn zijn moeilijker op te}{voeden dan domoren}\\

\haiku{zij is goed, zij is,.}{uitermate zuinig ja}{zelfs meer dan zuinig}\\

\haiku{Je broer onderhoudt,.}{contact met zijn vader via}{brieven doe ook zo}\\

\haiku{Naar het besluit wat.}{gij daar zult nemen zal ik}{mij moeten schikken}\\

\haiku{Caelo receptuur.}{sidus immenso sinu}{aeternitatis}\\

\haiku{kunnen wij daar niet.}{naartoe gaan om iets aan de}{tuin te laten doen}\\

\haiku{deze diende nu;}{als trofee of bewijsstuk}{voor de overwinning}\\

\haiku{Uit deze brief kan.}{blijken dat Maria daarvoor}{nu al bang was}\\

\haiku{cordon bleus ridders ();}{van dehoge orde van}{de Heilige Geest}\\

\haiku{Met haar {\textquoteleft}zoetemelks{\textquoteright}.}{hart bleef zij altijd begaan}{met zijn wel en wee}\\

\section{August Snieders}

\subsection{Uit: Werken. Deel 10. Alleen in de wereld. Deel 1}

\haiku{In de stad ziet men.}{echter niets van alle die}{kleine wonderen}\\

\haiku{{\textquoteright} De stuursche man neemt.}{het kind bij de hand en gaat}{den winkel binnen}\\

\haiku{Immers, hij ziet de,.}{kinderen der buurt die op}{den dorpel spelen}\\

\haiku{het is met een kruis.}{bekroond en dat kruis met een}{rouwkrep omgeven}\\

\haiku{Claudine wil het,,}{zoo De man doet het ook maar}{zoo weinig galant}\\

\haiku{Hij staart door de spleet:}{der gordijntjes naar het huis}{van den overbuurman}\\

\haiku{{\textquoteright} Mijnheer Daliski klotst,.}{verder zonder op deze}{stem acht te geven}\\

\haiku{{\textquoteleft}Kom hier,{\textquoteright} roept Mijnheer,.}{Golden en de taalmeester}{verschijnt voor het bed}\\

\haiku{{\textquoteright} {\textquoteleft}Ja, dat is zij...{\textquoteright} {\textquoteleft}En.}{zij heeft wel getoond dat zij}{ons innig bemint}\\

\haiku{{\textquoteright} {\textquoteleft}Dat is waar, 't Is...}{misschien ook beter dat wij}{niet alles weten}\\

\haiku{Welnu, ik zal niet;}{verder herhalen wat die}{vreemdeling zegde}\\

\haiku{Sybrand, ik heb uw,{\textquoteright}.}{gesprek met uwe zuster niet}{beluisterd zegt zij}\\

\haiku{doch ik geef u de -!}{verzekering en onthoud}{deze woorden goed}\\

\haiku{{\textquoteleft}Doch ik geef u de -!}{verzekering en onthoud}{deze woorden goed}\\

\haiku{Op dergelijke;}{zaken heeft Sybrand nog nooit}{ernstig nagedacht}\\

\haiku{Had ik haar alleen,,.}{aangesproken zij had hulp}{moord en brand geschreeuwd}\\

\haiku{hij is echter zoo.}{onthutst dat hij zelfs geen woord}{meer tot Sybrand richt}\\

\haiku{{\textquoteright} Mijnheer Golden keert.}{zich andermaal om en schijnt}{te willen heengaan}\\

\haiku{wat de buurt van hem;}{zegt en wat hij wezenlijk}{in karakter is}\\

\haiku{integendeel 't;}{is of de onnoozele pop}{deze nog vergroot}\\

\haiku{Adriana gaat naar;}{boven en klopt op de deur}{van Mijnheer Golden}\\

\haiku{De oude man staat,.}{voor het venster met den rug}{naar de deur gekeerd}\\

\haiku{{\textquoteright} zegt de oude man,.}{nadenkend en hij blijft strak}{op aen vloer staren}\\

\haiku{De reden dat ik-:}{wil heengaan ligt niet in u}{verre van daar}\\

\haiku{Aller wezen heeft;}{eene andere uitdrukking}{dan die van vroeger}\\

\haiku{{\textquoteright} vraagt de Zuster, nu.}{het kind zonder naar haar op}{te zien voorbij snelt}\\

\haiku{Zij heeft immers recht,.}{op onze onderwerping}{achting en liefde}\\

\haiku{Zij zelve steekt den,.}{brief in den post en gaat nu}{dwars door het stadje}\\

\haiku{Als Mijnheer Golden.}{heb ik niets te zien in al}{uwe moeilijkheden}\\

\subsection{Uit: Werken. Deel 11. Alleen in de wereld. Deel 2}

\haiku{Morgen, morgen denkt,.}{hij en koortsig woelt hij op}{zijne legerste\^e}\\

\haiku{Theodora beurt.}{het meisje op en trekt haar}{zusterlijk tot zich}\\

\haiku{{\textquoteright} mompelt Adriana.}{en slaat in verwarring de}{oogen naar beneneden}\\

\haiku{{\textquoteright} vraagt ze en heft de,,.}{oogen door tranen overwolkt naar}{den jongeling op}\\

\haiku{{\textquoteright} Nu de jongeling,}{eenige minuten later}{we\^er beneden komt}\\

\haiku{Uw zoon is slechts op,.}{\'e\'en punt van u gescheiden}{en dat punt ben ik}\\

\haiku{{\textquoteleft}Ik zal u later,.}{alles zeggen later aan}{u en uwen broeder}\\

\haiku{Nu Dobs den hoek der,;}{straat omkeert hoort hij eensklaps}{zijnen naam roepen}\\

\haiku{{\textquoteright} Juist roept de wachter;}{dat de trein op het punt is}{van te vertrekken}\\

\haiku{{\textquoteright} {\textquoteleft}Daar zit er in gansch,}{de kolonie van Gheel geen}{een die zoo gek zoo}\\

\haiku{Dobs even manhaftig}{iets in de toegestoken}{hand. In den flauwen}\\

\haiku{ik wees hem de deur -,.}{en hij hij glimlachte en}{trok de schouders op}\\

\haiku{haar voorhoofd lag schier,.}{op de knie\"en gedrukt zij}{nokte en snikte}\\

\haiku{De burggraaf beurt den,.}{gevallen naam der Duolet's}{op vergeet dit niet}\\

\haiku{Eindelijk zegt hij,:}{en er is iets welwillends}{in den toon der stem}\\

\haiku{Heb geen vrees, Eenige,.}{ruiten aan stukken en de}{storm zal voorbij zijn}\\

\haiku{een denkbeeld dat zich,,.}{ik weet niet hoe verspreid had}{en veel geloof vond}\\

\haiku{hij kan het denkbeeld. '}{aan die ijselijke dood}{niet verwijderen}\\

\haiku{{\textquoteright} {\textquoteleft}Gij beloont mij slecht,.}{voor al de diensten die ik}{u bewezen heb}\\

\haiku{Ik zal u onder,}{den voet vertrappen indien}{ge mij niet weergeeft}\\

\haiku{- maar uit achting voor.}{u wilde ik u over den}{toestand inlichten}\\

\haiku{gij hebt vandaag geen,,,:}{vuur geen gloed geen vlammende}{woorden geen poezi\"e}\\

\haiku{En waarom mij zelfs!}{om mijn eigen dwaasheden}{te bekommeren}\\

\haiku{Ik wil ze in den,.}{wijn verdrinken zij drijven}{gedurig boven}\\

\haiku{indien onder dat,,....}{zwarte baarkleed nog eens iets}{klopte leefde dacht}\\

\haiku{Dobs ziet dit en werpt.}{de koord achteloos op den}{stoel v\'o\'or de bedste\^e}\\

\haiku{Er volgt eene nieuwe.}{handschudding van wege den}{hartelijken Pool}\\

\haiku{Nu gaat hij naar de,:}{deur opent deze en zegt met}{diepe ontroering}\\

\haiku{{\textquoteleft}Ik heb mij door den,;}{nijd den haat en de gramschap}{laten verblinden}\\

\haiku{ik keer terug, hoe,,!}{laat ook met het wee en het}{berouw in de ziel}\\

\haiku{De inktkoker is,.}{uitgedroogd de stalen pen}{is als weggeroest}\\

\haiku{Nu opent hij eene deur,;}{en staat voor een steenen trap die}{naar beneden leidt}\\

\haiku{het slot wijkt en eene,,.}{kleine zware ijzeren}{deur draait buitenwaarts}\\

\haiku{Bij het verlaten.}{der kapel biedt de burggraaf}{zijner vrouw den arm}\\

\subsection{Uit: Werken. Deel 49. Allerlei}

\haiku{{\textquoteleft}'t Was misschien ook}{jaren geleden dat gij}{haar niet meer bezocht}\\

\haiku{Janmaat durfde niet,:}{landwaarts ingaan omdat ik}{hem niet volgen kon}\\

\haiku{Hij zweeg en ik had,.}{den moed niet dat stilzwijgen}{te verbreken VI}\\

\haiku{O, de eischen zijn!}{nog veel omvattender voor}{den romanschrijver}\\

\haiku{Antwerpen was toen,,,! '}{nog zoo glinsterend zoo trotsch}{zoo prachtig niet neen}\\

\haiku{Wy komen u als.}{koning groeten En kennen}{u voor onzen Heer}\\

\haiku{Links van u, in het,.}{oostelijk gedeelte is}{een ronde holte}\\

\haiku{Welke pelgrim, die,}{de grot bezocht drukte in}{diepe ontroering}\\

\haiku{De aarde verkeert:}{onder den invloed van de}{hemelsche woorden}\\

\haiku{{\textquoteleft}Wie zoude waagen,.}{yets te hervatten dat in}{de vierschaar van UEd}\\

\haiku{doch de gewone,:}{bezoeker kent men aan den}{kalmen vasten stap}\\

\haiku{Niettemin beschouwt!}{men den theater als het}{toppunt van ideaal}\\

\haiku{De diplomaat kan;}{zoo wat veertig \`a twee en}{veertig jaar oud zijn}\\

\haiku{beiden gaan nu hand,.}{aan hand over den weg in de}{richting van het dorp}\\

\haiku{Baron Davis staart;}{met stillen glimlach in de}{vlammen van het vuur}\\

\haiku{Zijn naam wordt ten prooi,.}{geworpen aan het publiek}{dat niets eerbiedigt}\\

\haiku{baron Davis kan:}{er niet toe besluiten de}{vensters te openen}\\

\haiku{het getingel brengt -....}{hem integendeel Edil te}{binnen Edil die zingt}\\

\haiku{Twintig jaren is;}{het geleden dat baron}{Davis hier dwaalde}\\

\haiku{doch zij is jong en.}{blond en blijft bewegingloos}{aan den ingang staan}\\

\subsection{Uit: Werken. Deel 35. Arme Julia}

\haiku{In dezelfde straat.}{en voor een prachtig hotel}{hield het rijtuig stil}\\

\haiku{{\textquoteleft}Ik vraag u niet hoe;}{gij mijne verblijfplaats hebt}{kunnen ontdekken}\\

\haiku{Gij offert alles;}{op aan het uiterlijke}{der familie-eer}\\

\haiku{{\textquoteleft}Julia,{\textquoteright} klonk de stem, {\textquoteleft},.}{van den dokterkom dan toch}{beneden lief kind}\\

\haiku{Doch het was mijn plicht,.}{te spreken toen anderen}{hardnekkig zwegen}\\

\haiku{de fortuin der Van.}{Hoogenhuyzen weegt niet op}{tegen de zijne}\\

\haiku{vernielt, verplettert,;}{vermoordt de eer en de faam}{eener familie}\\

\haiku{er ligt eene helsche.}{nieuwsgierigheid op aller}{trekken te lezen}\\

\haiku{Later had zij, in,;}{hare droomen die vogels}{en bloemen bezield}\\

\haiku{Er zat toch een goed,!}{een nog kinderlijk hart in}{vader Benninck}\\

\haiku{Eindelijk werd de.}{deur geopend en mevrouw}{d'Arton trad binnen}\\

\haiku{{\textquoteright} {\textquoteleft}Wie....{\textquoteright} en de stem van, {\textquoteleft}?}{de schuldige vrouw beefde}{wie is uw vader}\\

\haiku{mij zoo innig aan,.}{het hart dat ik u steeds goed}{en braaf gedroomd heb}\\

\haiku{ik kan, dag en nacht,.}{om het brood te verdienen}{dat gij mij toereikt}\\

\haiku{de nieuwe dag ving,?}{aan Wat zal hij der arme}{Julia aanbrengen}\\

\haiku{De heer d'Arton zag.}{het en de koele man scheen}{er door bewogen}\\

\haiku{{\textquoteright} mompelde hij en,.}{trad het vertrek binnen waar}{mevrouw zich bevond}\\

\haiku{Zou de heer d'Arton?}{weten dat Julia Martin}{hare dochter is}\\

\haiku{Mevrouw d'Arton was.}{nog altijd aan duizenden}{gissingen ten prooi}\\

\haiku{{\textquoteright} Het meisje dacht niet.}{anders of mevrouw was in}{de hersens gekrenkt}\\

\haiku{Maar ik bid u, in,;}{naam van al wat u heilig}{is verlaat dit huis}\\

\haiku{Ja, 't was akelig, '.}{daar docht was er beter}{toch dan in dat huis}\\

\haiku{In de verte rijst.}{de toren van Saventhem}{voor haar oog  op}\\

\haiku{Hoe meer Julia het,.}{dorp naderde hoe drukker}{de weg bezocht werd}\\

\haiku{Hoe zou eene moeder!}{zich in de stem van haar kind}{kunnen vergissen}\\

\haiku{Ik ben te voet van,....}{Brussel gekomen mijne}{voeten zijn doorwond}\\

\haiku{Haar echtgenoot heeft;}{haar verlaten en zelfs om}{haar nooit we\^er te zien}\\

\haiku{{\textquoteright} De worm sloeg de deur.}{toe en het onmetelijk}{graf werd weer doodstil}\\

\haiku{Mevrouw d'Arton hield:}{den adem in en antwoordde}{op dat geklop niet}\\

\haiku{Moeder, wat heb ik,!}{u lang gezocht lang gesmacht}{naar uwe omhelzing}\\

\haiku{Ik heb geen recht om.}{uwe handelwijs van vroeger}{te onderzoeken}\\

\haiku{Straks zal de wereld.}{zich andermaal om u en}{mij bekommeren}\\

\subsection{Uit: Werken. Deel 37. Avond en morgen}

\haiku{{\textquoteright} hervatte Clara,.}{en streelde de magere}{wangen des grijsaards}\\

\haiku{Uren lang stond zij soms;}{droomend voor het venster en}{tuurde de straat in}\\

\haiku{{\textquoteleft}Gij ziet nu wat kwaad,.}{welke sleep rampen eene booze}{tong kan voortbrengen}\\

\haiku{hij voelde, dat hij -,.}{er onder bezweek maar toch}{hij torschte die}\\

\haiku{Alles scheen stil en;}{akelig in zijn doodlaken}{te zitten droomen}\\

\haiku{een traan blikkerde,.}{in zijn oog en diep zuchtend}{verliet hij het graf}\\

\haiku{{\textquoteleft}Het is uw feestdag,...{\textquoteright}.}{juffer Liva stamelde}{de jongen bedeesd}\\

\haiku{Waarop Liva we\^er,,:}{met eene zilverachtige}{stem antwoordde}\\

\haiku{Bram zat op den grond,.}{het aangezicht in de twee}{handen verborgen}\\

\haiku{Liva had geen woord;}{gesproken over de ziekte}{van Guido's vader}\\

\haiku{{\textquoteright} De fabrikant had.}{gesidderd toen het meisje}{die woorden uitsprak}\\

\haiku{Bram werd bang voor dat,.}{wezen en deinsde een paar}{schreden achteruit}\\

\haiku{Zie, Guido, ik bracht,}{Vondel's treurspelen me\^e om}{u te verzoeken}\\

\haiku{Zij werpen u van,!}{verre hunne kushandjes}{toe juffer Liva}\\

\haiku{{\textquoteright} Er lag een glans van.}{zaligheid over het gelaat}{der blinde verspreid}\\

\haiku{{\textquoteright} {\textquoteleft}Ik ben gelukkig,,,;}{mijnheer niet voor mij maar voor}{u en uwe dochter}\\

\haiku{Mijnheer, gij zeidet,:}{zooeven dat gij geene luiaards}{in uwen dienst duldet}\\

\haiku{nu, integendeel,,!}{joeg het als eene klok welke}{eene ramp verkondigt}\\

\haiku{Bram weende lang en.}{smartelijk op het graf van}{den miskenden man}\\

\haiku{maar de zon was zoo,;}{weldoende de hemel zoo}{frisch en verkwikkend}\\

\haiku{doch men had gehoopt.}{den jongeling een pijnlijk}{gevoel te sparen}\\

\haiku{het was alsof hij.}{zijn heerlijken tooverdroom reeds}{verwezenlijkt zag}\\

\haiku{Sedert de laatste,;}{maal dat wij hem zagen was}{hij veel veranderd}\\

\haiku{zij gevoelde zich.}{bloedig in haar gevoel van}{eerbaarheid gekwetst}\\

\haiku{welnu, die jongen,,.}{is mijn trouwe leidsman mijn}{eerste vriend geweest}\\

\haiku{Alvorens in te,:}{stappen wendde mevrouw zich}{tot Bram en zeide}\\

\haiku{{\textquoteleft}Maar,{\textquoteright} zeide Bram tot, {\textquoteleft},,...,!}{zich-zelvenniet dansen}{Brammetje neen neen}\\

\haiku{bleek was zijn gelaat.}{en een diepe ontroering}{had hem bevangen}\\

\haiku{doch dagen, weken,.}{en maanden verstreken en}{niemand keerde we\^er}\\

\haiku{{\textquoteright} {\textquoteleft}Laat ons aan al die,!}{wonderlijke dingen niet}{meer denken moeder}\\

\haiku{Peerke duwde het.}{weg en kwam met den schelmschen}{Reinaart voor den dag}\\

\haiku{In de verte klonk:}{eene mannenstem welke het}{zoo gekende lied}\\

\haiku{Het is hier wel wat,;}{veranderd sedert ik er}{het laatst geweest ben}\\

\haiku{{\textquoteleft}Slecht, mijn goede heer,{\textquoteright};}{zeide de oude man op}{mismoedigen toon}\\

\haiku{heel het dorp was op.}{de been om de bruid en den}{bruidegom te zien}\\

\subsection{Uit: Werken. Deel 48. Beelden uit ons leven}

\haiku{Hugo stond, eene poos,.}{stom van verwondering over}{die driftige taal}\\

\haiku{Het was hem of haar.}{naam een balsem van troost op}{zijn lijdend hart goot}\\

\haiku{Bij die woorden wierp.}{Richard een spottenden blik}{door de werkkamer}\\

\haiku{de dochter, welke,;}{Richard kende zou er aan}{mama over spreken}\\

\haiku{want nooit zag men nog.}{van zijne stukken in de}{tentoonstellingen}\\

\haiku{Hugo geleidde.}{haar naar den leuningstoel en}{deed haar ne\^erzitten}\\

\haiku{Naar wien anders dan,.}{naar u die hem altijd zoo}{liefderijk ontvangt}\\

\haiku{Zijn kleed is niet meer,;}{in de mode zijn hoed ruw}{en ongeborsteld}\\

\haiku{- Uw naam, mijnheer, heeft,,.}{daarvoor vergeef het mij nog}{geen waarborg genoeg}\\

\haiku{De koopman sloeg de.}{gewaarwordingen op Hugo's}{aangezicht gade}\\

\haiku{Hij greep den beker.}{der wanhoop vast en dronk er}{milde teugen uit}\\

\haiku{en mag eene zuster?}{de hartepijn van haren}{broeder niet stillen}\\

\haiku{Denk niet dat ik mij.}{in mijn eigen woning zal}{laten beleedigen}\\

\haiku{Ziedaar dan de bron,! -!}{dier liefde welke zij den}{rijkaard toedroeg Mensch}\\

\haiku{De werkzaamheid bracht;}{in het huis van Rodolf een}{nederig bestaan}\\

\haiku{- Miskend genie, sprak,;}{de kenner uw vaderland}{weigert u eene plaats}\\

\haiku{Gij denkt dat ik ook -,!}{kom om u een spotwoord toe}{te werpen neen Casper}\\

\haiku{zelfs die heer niet meer,.}{die beloofd had ons jaarlijks}{te komen voldoen}\\

\haiku{ik zou geene rust meer.}{hebben voor dat ik van hier}{ware weggegaan}\\

\haiku{Ik zie u altoos,....}{droevig en gij weendet straks}{zoo ongedwongen}\\

\haiku{Maar wist gij, hoe de;}{liefde tot eene moeder mij}{in de ziel brandde}\\

\haiku{toen zij voor mij bad;}{en zich slechts met geweld van}{mij liet wegscheuren}\\

\haiku{- Dat is inbeelding,,.}{Mina sprak de jongeling}{op bevenden toon}\\

\haiku{terwijl de zieke....}{Mina voor hen den God der}{lijdenden bad}\\

\haiku{maar wij, wij leven,....}{altijd altijd met nieuwe}{wonden aan het hart}\\

\haiku{Als eene wolk dreef de.}{hondengroep in golvende}{bogen hem voorbij}\\

\haiku{Zij juichte, omdat.}{zij hare gezellinnen}{zoo ver achterliet}\\

\haiku{zij begrijpt niet wat,,....}{het zegt voor eene moeder een}{kind te verliezen}\\

\haiku{zoete belooning!}{voor de bewezen vriendschap}{in het ongeluk}\\

\haiku{Lezer, keer met mij,.}{terug tot den morgen van}{den vorigen dag}\\

\haiku{Wellicht ook hadden:}{zij alle hoop op aardsche}{hulp opgegeven}\\

\haiku{- en hij bedekte,.}{met zijne handen de oogen}{in tranen gebaad}\\

\haiku{Edelaardig man, zeg,?}{mij wat is er van mijne}{zuster geworden}\\

\haiku{want men dacht het spook.}{van den hongersnood aan de}{deur te zien kloppen}\\

\haiku{Ik vond niets in de,,,!...}{wereld dat mij van deugd van}{liefde van God sprak}\\

\haiku{Ik sprong toe - met een;}{geweldigen ruk bleef mij}{dien schat in de hand}\\

\haiku{maar zij zijn thans niet,.}{daar om die groote tranen van}{Bertha's wang te wisschen}\\

\haiku{En toch was hij het,;}{wel die haar op het pad der}{ondeugd gesleurd had}\\

\haiku{want de wanhoop heeft;}{mij voortgedreven op het}{pad der vertering}\\

\haiku{Kind, ging zij voort, en;}{zag met een verbijsterden}{oogslag op het wicht}\\

\haiku{Dan kuste zij het,,.}{vaarwel totdat zij komen}{zou en drukte toe}\\

\haiku{- men zag dit zoowel aan.}{zijne houding als aan die}{van het gezelschap}\\

\haiku{Gaat, en zegt aan de,....}{justitie dat ik het ben}{die hem vermoord heb}\\

\haiku{Te vergeefs trachtte;}{hij echter die gedachten}{te verwijderen}\\

\haiku{al was zij dan ook,.}{boelinne moordenares}{van haar kind geweest}\\

\haiku{Tranen stroomden den.}{ouden man door de rimpels}{van zijne wangen}\\

\subsection{Uit: Werken. Deel 20. Burgerdeugd}

\haiku{lichtgeloovigheid....}{is dikwijls een kenmerk van}{onschuld en eenvoud}\\

\haiku{Dikwijls stond Zeno;}{op dien kring van lijden en}{smarte te staren}\\

\haiku{Wat sloeg het hart van!}{den vader dankbaar voor dat}{engelachtig kind}\\

\haiku{des nachts beven zij.}{voor het geruisch van het}{ritselende loover}\\

\haiku{als er tranen over,;}{uwe wangen vloeien bij het}{zien van het lijden}\\

\haiku{De menschheid klimt;}{slechts van trap tot trap naar het}{punt der volmaaktheid}\\

\haiku{Als ik u sprak van,:}{de liefde van twee harten}{dan riept gij mij toe}\\

\haiku{Maar wat zal het mij?}{baten dat die juichtoon der}{vernieling opstijgt}\\

\haiku{Kom, geliefd overschot,...}{van mijn aardsch geluk kom hier}{aan mijn jagend hart}\\

\haiku{geen zachter genot,!}{dan het vergelden van het}{booze door het edele}\\

\haiku{zegde zij smeekend,,,!}{ik ook heb een vader die}{misschien God weet het}\\

\haiku{- Ik heb mij boven,;}{den kleingeestigen haat der}{menschen verheven}\\

\haiku{De graaf had langen,,;}{tijd geboeid dat ijselijk}{schouwspel moeten zien}\\

\haiku{niet zoo, dat men haar;}{liefdevol aan het harte}{zou kunnen drukken}\\

\haiku{Indien er nog een.}{liefdevonk voor uwe dochter}{in het harte gloeit}\\

\haiku{- Schiet toe, indien gij,;}{hem vermoorden wilt vermoord}{dan ook uwe dochter}\\

\haiku{doch er kwam geen woord;}{van het voorgaande leven}{over zijne lippen}\\

\haiku{- en de jongeling -!}{boog de knie dan ben ik toch}{zinneloos van u}\\

\haiku{Ach, laat mij deze,,...}{niet zoeken zooals Rosa in}{het golvend water}\\

\haiku{Deze ook had een,.}{somber gevoel dat hem zwaar}{op de borst drukte}\\

\haiku{gij zijt Rosa, die,!}{uit het water opstijgt om}{mij te martelen}\\

\subsection{Uit: Werken. Deel 7. Dit sijn Snideri\"en}

\haiku{{\textquoteright} En vurig kuste,.}{zij de moeder wier haren}{reeds sneeuwwit waren}\\

\haiku{Wat vond de lieve,,!}{blinde veel deelneming veel}{vriendschap veel liefde}\\

\haiku{Hij scheen er zich in.}{te verlustigen dat ik}{hem niet herkende}\\

\haiku{hij, en het kwam mij,,.}{voor dat hij bij het heengaan}{diep was aangedaan}\\

\haiku{Hij zakt op de bank, -.}{terug zijn hoofd valt op de}{borst hij is dood}\\

\haiku{Maar zij staken wel,,....}{diep zeer diep hunne vingers}{in onze zakken}\\

\haiku{In hunne schaduw,;}{hebben allen gerust die}{ons zijn voorgegaan}\\

\haiku{maar de stallen en.}{schuren zijn ledig en het}{huis is gesloten}\\

\haiku{Ik zette mij een;}{oogenblik op de oude}{bank v\'oor het huis neer}\\

\haiku{Tegen den mast van,.}{het vaartuig leunt eene vrouw met}{een kind op den arm}\\

\haiku{Dora staat recht aan,.}{het roer en haar oog brandt op}{de beschuldigster}\\

\haiku{{\textquoteright} En Dora, die nu,:}{ook aan het gebeurde denkt}{zegt in haar eigen}\\

\haiku{{\textquoteright} {\textquoteleft}Neen,{\textquoteright} was het antwoord, {\textquoteleft},,}{neen hij is een dichter en}{volgens sommigen}\\

\haiku{Hij volgde haar met}{bangen oogslag en begreep}{wel tot wat volkje}\\

\haiku{{\textquoteright}, hare zuster, de,,?....}{zwarte Peternel te Thoir}{gehangen en zij}\\

\haiku{Kortom, Brussel was,.}{toen reeds eene stad waar men zich}{heerlijk vermaakte}\\

\haiku{{\textquoteleft}Baron,{\textquoteright} zegde de, {\textquoteleft}!}{markiesgij hebt gisteren}{gelukkig gespeeld}\\

\haiku{doch aan terugkeer.}{in de samenleving was}{niet meer te denken}\\

\haiku{De zieke wilde,;}{spreken doch hare lippen}{weigerden de spraak}\\

\haiku{{\textquoteright} ~ De notaris,,.}{bewoont in de kom van het}{dorp een groot steenen huis}\\

\haiku{Heemrik glimlacht en;}{bevestigt met een hoofdknik}{Rika's woorden}\\

\haiku{Het huis staat aan den,.}{zoogezegden dijk die dwars}{door de heide snijdt}\\

\haiku{Nu zitten de twee;}{vriendinnen op de ruwe}{bank nevens de deur}\\

\haiku{{\textquoteright} {\textquoteleft}Nu,{\textquoteright} antwoordt Frida, {\textquoteleft}{\textquoteright};}{die wel begrijpt wie ze met}{dat woordboos bedoelt}\\

\haiku{doch Frida durft zich:}{op het slibberige pad}{niet verder wagen}\\

\haiku{die Heemrik smokkelt,,:}{die Heemrik stroopt die Heemrik}{schijnt een dief te zijn}\\

\haiku{Toch sprak er eene stem,:}{in haar die haar telkens met}{vastheid deed zeggen}\\

\haiku{Er zal eene stem in,.}{zijn hart weerklinken die hem}{over het sneeuwveld roept}\\

\haiku{{\textquoteright} Nu het avond wordt, opent,:}{zij de deur met eene spleet en}{steekt het hoofd buiten}\\

\haiku{Wie duivel zou zich!}{op dat ijsbed neerleggen}{zooals de dienst gebiedt}\\

\haiku{hij is overtuigd dat.}{zij bezig is met hem eene}{booze pert te spelen}\\

\haiku{hij meent te zullen.}{vallen en moet zich aan de}{putmik vasthouden}\\

\haiku{{\textquoteright} onderbreekt Lindorp,.}{verschrikt en er overvalt hem}{waarlijk eene beving}\\

\haiku{{\textquoteright} Dat woord {\textquoteleft}trouwen{\textquoteright} jaagt.}{Rika eene huivering over}{de ledematen}\\

\haiku{Nu het kind dood is,.}{wil Rika dat Frida naar}{huis zal terugkeeren}\\

\haiku{het grasperk is met,,.}{witte bloemekens wit als}{sneeuwvlokjes bezaaid}\\

\haiku{Knecht en meid zeggen;}{dat de jeneverduivel}{er geweldig spookt}\\

\haiku{de hinnikende.}{paarden galoppeeren angstig}{over het akkerland}\\

\haiku{*** Het erfgoed is in;}{het bezit van Rika en}{Heemrik gekomen}\\

\haiku{Neen, we hadden den;}{Brander nooit een speldepunt}{in den weg gelegd}\\

\haiku{Eindelijk werd het;}{geld op de tafel van den}{veldmaarschalk geschud}\\

\haiku{Jan de pijper was,:}{een jonge maat met vlug oog}{en blonden haarbos}\\

\haiku{Doch de vrijbuiter.}{hoopte nog altijd op den}{aanbrekenden nacht}\\

\subsection{Uit: Werken. Deel 39. Fata morgana. Deel 1}

\haiku{De knaap weet het niet,:}{en het zet hem ook niet tot}{onderzoeken aan}\\

\haiku{{\textquoteright} De moeder neemt het,;}{jongste kind op den arm het}{tweede bij de hand}\\

\haiku{of bij het effen,.}{watervlak waarboven de}{reiger langzaam drijft}\\

\haiku{{\textquoteright} en hoorbaar klopt zijn.}{hart bij het naderen van}{de zwarte bende}\\

\haiku{zij houdt zich bezig -.}{met wat Parijs betreft niet}{met wat hier omgaat}\\

\haiku{{\textquoteright} De lofspraak brengt op.}{het aangezicht van Christ geen}{spier in beweging}\\

\haiku{hij komt terug als, '}{het nieuwe werkuur aanvangt}{ens Zaterdags}\\

\haiku{doch rood wordt ze, rood,:}{als de kolroos wanneer de}{jongen lachend zegt}\\

\haiku{{\textquoteright} Hooger streven de.}{droombeelden van den jongen}{kunstenaar niet}\\

\haiku{zij huldigde de,;}{leer der stoffelijkheid den}{uiterlijken vorm}\\

\haiku{De aarde - 't is -;}{altijd zij die sprak is met}{wellusten overzaaid}\\

\haiku{Theo Rigobert zal.}{u een der ontelbare}{antwoorden geven}\\

\haiku{doch, tot overmaat van,.}{ramp vonkte er liefde voor}{Doria in zijn hart}\\

\haiku{Doch hij vergat dat;}{er reeds zoovele jaren}{vervlogen waren}\\

\haiku{{\textquoteright} Op een honderdtal;}{stappen verder verlaat de}{pastoor juist de kerk}\\

\haiku{doch de hond loopt snel,,.}{en waakzaam met gespitste}{ooren rechts en links}\\

\haiku{Een pistoolschot galmt.}{en wordt twee of driemaal in}{de verte herhaald}\\

\haiku{V\'o\'or hem, in den stal,;}{staat een man van tamelijk}{zware gestalte}\\

\haiku{{\textquoteright} {\textquoteleft}Zijt ge in zoo'n groot,,,?}{huis in een waarachtig steenen}{huis geboren Wim}\\

\haiku{Om die reden had,,.}{dan ook de rakker tot nu}{toe voor mij gebukt}\\

\haiku{Bij elken dronk kwam;}{er eene soort berengebrul}{uit  zijne keel}\\

\haiku{Men had nooit iets meer.}{van hem gehoord en elkeen}{dacht dat hij dood was}\\

\haiku{{\textquoteleft}Gij zijt ook zoo wild,,{\textquoteright}}{jongenlief en oom Morris}{meent het goed met u.}\\

\haiku{Men hoorde de stem,,.}{van den bandhond den mop den}{wolfshond en den taks}\\

\haiku{Eenige minuten.}{later verscheen de jongen}{der rarekiekkas}\\

\haiku{Eindelijk hield de.}{wagen op eene binnenplaats}{van een gebouw stil}\\

\haiku{Klonken ze laag, dan;}{keek ik naar den vloer om te}{zien waar zij bleven}\\

\haiku{jammer, ze wist dat.}{ze schoon was en de moeder}{wist het nog beter}\\

\haiku{{\textquoteright} Nora door hare.}{trotsche en verachtende}{onverschilligheid}\\

\haiku{de snikken van mijn,.}{hart nabootsen tikte de}{hangklok beneden}\\

\haiku{De Fitzel's werkten,}{van vader tot zoon bij den}{instrumentmaker}\\

\haiku{Vele bekenden,,;}{en dat deed mij huiveren}{gingen mij voorbij}\\

\haiku{Meester Krok, vroeger,;}{zoo vroolijk en welwillend}{spotziek sprak geen woord}\\

\haiku{Zij, zij wist dus wat!}{er gebeurd was en wat er}{nog gebeuren zou}\\

\haiku{'t Was God geklaagd,,!}{zegden de buren een wees}{uit te plunderen}\\

\haiku{Toen ik een eind weegs,:}{van het dorp verwijderd was}{wendde ik mij om}\\

\haiku{Wij spraken over het,,.}{verre land toen het kind met}{de pop wakker werd}\\

\haiku{Het verre land kreeg.}{voor mij eene betooverende}{aantrekkelijkheid}\\

\haiku{Ik sidder nog als,}{ik denk aan dat bonzend en}{klapperend geloei}\\

\haiku{Het zou voor mij een,!}{feest zijn dien eerlijken man}{de hand te drukken}\\

\haiku{Deuren en vensters,:}{waren gesloten en op}{eenen plakbrief las men}\\

\haiku{Ik heb z\'o\'o veel, z\'o\'o,.}{lang geleden en ik zie}{nergens uitkomst meer}\\

\haiku{{\textquoteright} Eindelijk opende,.}{ik de deur der kamer waar}{Nora zich bevond}\\

\haiku{Ik bevond mij aan,.}{eene spoorwegstatie ergens}{in Zuid-Duitschland}\\

\haiku{{\textquoteright} Gedurende een;}{paar seconden heerscht er}{stilte in den stal}\\

\subsection{Uit: Werken. Deel 40. Fata morgana. Deel 2}

\haiku{Nu ook verdwijnt de;}{kalme uitdrukking op het}{wezen des grijsaards}\\

\haiku{Met afkeer stoot hij;}{het dagblad ter zijde dat}{die dagteekening meldt}\\

\haiku{zij ziet met haren.}{kalmen blik den grijsaard vlak}{in het aangezicht}\\

\haiku{De vader nadert,:}{haar neemt zacht hare hand in}{de zijne en zegt}\\

\haiku{Boven dat kleedsel;}{staat een vinnige kop met}{kortgeknipt zwart haar}\\

\haiku{hier leeft men in het.}{lieflijk Zuiden en in eene}{gebalsemde lucht}\\

\haiku{dan bezoekt hij eenen,;}{oud collega bij wien hij}{echter zelden komt}\\

\haiku{Deze is hij nooit;}{met een godsdienstig gevoel}{binnengetreden}\\

\haiku{De oude koopman,;}{durft niet meer hooger durft niet}{meer naar beneden}\\

\haiku{Mijne moeder gaf.}{het leven aan een kind en}{stierf in het gasthuis}\\

\haiku{want hij gevoelt dat.}{hij zijn leed met een tweede}{wezen dragen zal}\\

\haiku{Wanneer zal er nu?}{in den aangeduiden zin}{recht worden gedaan}\\

\haiku{De rijke koopman, {\textquoteleft}{\textquoteright};}{leeft nog eenige maandenmet}{den worm in het hart}\\

\haiku{De linde- en;}{kastanjeboomen zijn met}{dicht loof omhangen}\\

\haiku{{\textquoteleft}Hebt ge gezien, baas,?}{wat schoone lamp bij Spronk op}{de huistafel brandt}\\

\haiku{dat hij Tegel den '.}{doorn der jaloezie int}{hart heeft gestoken}\\

\haiku{{\textquoteright} zegt het meisje dat,;}{zich eenigszins voorover buigt om}{zich te doen verstaan}\\

\haiku{{\textquoteleft}Huu....uu....{\textquoteright} {\textquoteleft}Gij zijt er,{\textquoteright}, {\textquoteleft}?}{zegt de oude Baldof rijdt}{ge mee naar het dorp}\\

\haiku{Eensklaps staat hij stil,:}{met den voet op het harde}{pad stampend zegt hij}\\

\haiku{{\textquoteleft}Ze brengen Helm in {\textquoteleft}{\textquoteright},.}{triomf naarden Hanekam}{zegt ze welgemoed}\\

\haiku{daar krast en snijdt de!}{schaats over het ijs en laat een}{sneeuwig spoor achter}\\

\haiku{t is de gloed van,.}{het vuur die door het venster}{van de hoeve schijnt}\\

\haiku{Zijn aangezicht is -;}{doodsbleek doodsbleek in den rooden}{lichtglans der lantaarns}\\

\haiku{Wat hebt ge met Helm,?}{uitstaan die sedert lang}{u den rug toekeert}\\

\haiku{zij is altijd voor;}{de Spronk's toegevend en}{week als een meelzak}\\

\haiku{Hier en daar, langs de,,;}{baan staan eenige berken met}{kalkwitte schors}\\

\haiku{Xander zet zich op,:}{eene voetstoof naast Berga en}{als deze hem zegt}\\

\haiku{doch deze is te.}{diep geslagen om nog te}{kunnen genezen}\\

\haiku{een vijfde was op, -!}{het dwaalspoor en dat viel den}{vader hard zeer hard}\\

\haiku{kortom, ging aan kaai {\textquoteleft}{\textquoteright}.}{en dok door als iemand met}{een gezonden kop}\\

\haiku{Centen en kroeg zijn.}{een machtig lokaas voor den}{wereldhervormer}\\

\subsection{Uit: Werken. Deel 25. De fortuinzoekers}

\haiku{zullen wij, als de,:}{Geest des goeds den inboorling}{toe fluisteren}\\

\haiku{O beuk, gij zijt de,:}{vriend des grijsaards Die somtijds}{aan uw wortel rust}\\

\haiku{{\textquoteright} morde Huibert, nog;}{eenige schreden van den boom}{verwijderd zijnde}\\

\haiku{Brengen uwe akkers,?}{niet genoeg op als God u}{duizend voor een geeft}\\

\haiku{Evert Krans was, in het,:}{huiselijke leven door}{den dichter gevormd}\\

\haiku{Een ei dat nog niet,. ....}{is geleyd Daervan en}{dient niet veel gezeyt}\\

\haiku{{\textquoteright} {\textquoteleft}De dooden zijn dood, en!}{de levenden maken mij}{het leven bitter}\\

\haiku{{\textquoteright} De grijsaard had die.}{woorden met eene ontroerde}{stem uitgesproken}\\

\haiku{Gij hebt eene reis van.}{verscheidene weken over}{zee af te leggen}\\

\haiku{Ik zou mijne vrouw....{\textquoteright} {\textquoteleft},?}{niet geven voorJa hoe hoog}{schat gij uwe vrouw wel}\\

\haiku{Aan Heva's hand over,.}{den zandweg gaande kwam hij}{langs het huis van Evert}\\

\haiku{Werken kan ik niet,.}{meer en ik ben te oud om}{te leeren bedelen}\\

\haiku{Denk in uwen voorspoed,,.}{aan hen die in een vreemd land}{een graf gaan zoeken}\\

\haiku{{\textquoteleft}Waarom dreeft gij aan,,!}{den beukenboom uw paard zoo}{eensklaps voort Huibert}\\

\haiku{{\textquoteright} riep de Rosse, en.}{legde opnieuw de zweep over}{het snuivende paard}\\

\haiku{wat vroeger of wat,.}{later afscheid genomen}{dat is hetzelfde}\\

\haiku{hij stond met Huibert,.}{Monica en Heva op}{de kaai der Schelde}\\

\haiku{Was zijn uiterlijk,;}{gunstig ook zijn innerlijk}{strekte hem tot eer}\\

\haiku{Wat Evert betreft, hij.}{zweeg en liet de toekomst aan}{den goeden God over}\\

\haiku{Van waar kwam zij, die,?}{arme Heva en hoe was}{zij daar gekomen}\\

\haiku{Gij zijt zoo bleek, gij,.}{ziet mij zoo wonderlijk aan}{gij spreekt zoo somber}\\

\subsection{Uit: Werken. Deel 42. In 't vervallen huis}

\haiku{het publiek, zooals ik,.}{zeide bestond uit zes of}{zeven personen}\\

\haiku{vriendschap, liefde, recht,!}{en zelfs gansch het menschdom}{grootsch en edel voor ons}\\

\haiku{De verveling was;}{immers ook vroeger uit ons}{midden gebannen}\\

\haiku{doch bij de minste.}{tegenkanting nam hij zijn}{toevlucht tot den drank}\\

\haiku{maar hij is het die.}{uw levensgeluk en het}{mijne vernield heefd}\\

\haiku{Rechts van de knapen,;}{zag hij den witten schoorsteen}{van zijns moeders huis}\\

\haiku{alleen Willem was.}{nog niet in het moederlijk}{huis teruggekeerd}\\

\haiku{Ook nu we\^er kwam zij:}{haren toevlucht zoeken in}{het huis des Heeren}\\

\haiku{We gingen samen.}{te Rotterdam aan boord en}{verder naar de Oost}\\

\haiku{gij gevoelt wat ik.}{lijden moet om het verlies}{van mijn eenigen zoon}\\

\haiku{Wat volgde er een!}{onrustige nacht op dien}{avontuurvollen avond}\\

\haiku{Zijn aangezicht was,;}{akelig verwrongen zijn oog}{blonk wilder dan ooit}\\

\haiku{{\textquoteright} sprak Willem, {\textquoteleft}laat ons.}{ter kerke gaan en God voor}{zijne ziel bidden}\\

\haiku{gij blijft hier in uwen,?}{leuningstoel liggen en de}{vrienden wachten u}\\

\haiku{Die lange witte!}{kerel steekt den draak met mij}{en mijn manuscript}\\

\haiku{mijn neus moet prachtig,.}{gepurperd mijne wangen}{moeten loodblauw zijn}\\

\haiku{Men ziet het wel, Van:}{Velden was een man van den}{ouden stempel}\\

\haiku{{\textquoteright} {\textquoteleft}Zwijg gij ook, vrouw, die.}{blinder zijt dan een mol in}{uwe moederliefde}\\

\haiku{Als een bliksem vloog.}{Georges de deur uit en}{ijlde de straat op}\\

\haiku{{\textquoteright} {\textquoteleft}O, o, wij komen!}{niet in de hoedanigheid}{onzer ambachten}\\

\haiku{wat ik loopen kan,,.}{de kamer rond de trappen}{af de trappen op}\\

\haiku{Liet nu de kachel?}{een gebrom van af- of}{goedkeuring hooren}\\

\haiku{het helsch bataillon,.}{liep er eene rilling over al}{de aanwezigen}\\

\haiku{Welnu, zoodanig.}{zaten de groote kinderen}{in Het Molentje}\\

\haiku{Het gekrijt van het.}{kind maakte den vader een}{oogenblik wakker}\\

\haiku{den jongen scharesliep.}{in het oud wammes van zijn}{vader gewikkeld}\\

\haiku{dat is, men stak hem,.}{in eene enge ruimte in}{den wagen zelven}\\

\haiku{{\textquoteleft}Zie, nu heb ik spijt,!}{dat ik al uw pruimen niet}{heb opgegeten}\\

\haiku{Waarom zet gij uwen?}{korf niet ne\^er en komt gij ook}{niet in het water}\\

\haiku{Maar als het in de -!}{week spinning op den Hooiberg}{is duivekaters}\\

\haiku{En dan de tijger,!}{die met een paard in den muil}{over eenen draaiboom springt}\\

\haiku{Het manneken staat,.}{overeind met vlammend oog en}{uitgestrekten arm}\\

\haiku{en gij ook, blaadjes,,!}{die daar boven soms kwettert}{fluistert en giechelt}\\

\haiku{Hij slentert zelfs - maar;}{ook de kruipende slak komt}{waar zij wezen moet}\\

\haiku{zijne gedachten -.}{gaan verder heel verre van}{al die booze menschen}\\

\haiku{In de huizen rilt.}{en bibbert men van schrik bij}{het zien der ruiters}\\

\haiku{betalen we nu,:}{zware lasten laat ons op}{de toekomst hopen}\\

\haiku{in Gent ieverden,,;}{Willems Snellaert van Duyse}{en Ledeganck}\\

\haiku{Al wat wezenlijk,.}{Vlaamschgezind was liet zich}{op dat feest vinden}\\

\haiku{Tweemaal poogde hij,,.}{de volkspers te stichten dat}{is de weekbladpers}\\

\haiku{Het eerste nummer.}{van dit weekblad was vooral}{het werk van Gerrits}\\

\subsection{Uit: Werken. Deel 36. Het Jan-Klaassen-spel}

\haiku{{\textquoteright} zegt de stoeldraaier.}{met eene niet meer bedwongen}{verontwaardiging}\\

\haiku{ik zeg dat hij een, -}{slecht hart heeft en zie ik zou}{hem eens willen doen}\\

\haiku{Laat ons den baron.}{van Dormael en Max Franck}{nader leeren kennen}\\

\haiku{{\textquoteright} En allen bersten.}{bij die godslastering in}{schaterlachen uit}\\

\haiku{Ik wil niets van hem, '....}{weten maart is enkel}{uit eene aardigheid}\\

\haiku{{\textquoteright} {\textquoteleft}Nu ben ik toch wel,{\textquoteright}, {\textquoteleft}.}{eens nieuwgierig dacht Janof}{ze we\^er komen zal}\\

\haiku{{\textquoteleft}En eene kan goede {\textquoteleft},{\textquoteright}.}{seef.{\textquoteright}3Die schuimt als Champagne}{antwoordt het meisje}\\

\haiku{{\textquoteleft}En waarom ziet gij,?}{er uit alsof ge naar eene}{begrafenis gingt}\\

\haiku{Zijne haren zijn,;}{eenigszins verward zijn gezicht}{rood opgeblazen}\\

\haiku{deze met opzet,.}{gene gedwongen door de}{omstandigheden}\\

\haiku{Ik beken, van die;}{overdrijving ben ik eenigszins}{teruggekomen}\\

\haiku{Ziedaar eene liefde,.}{zoo als men er in onzen}{tijd geene meer aantreft}\\

\haiku{Ik zal hem zien, hem,!}{nogmaals spreken misschien zal}{hij mij beminnen}\\

\haiku{Mijnheer komt ge hier '?}{s avonds ingedrongen met}{slechte inzichten}\\

\haiku{Mijnheer Bareel rookt.}{den geurigen Varinas}{uit eene lange pijp}\\

\haiku{Tony Darenge,.}{wendt zich om want hij voelt zijn}{hoofd gloeiend worden}\\

\haiku{In zijn hart wenscht;}{hij nu eens een liedje te}{mogen afgeven}\\

\haiku{Dit belet niet dat.}{hij de fijne watersnep}{zeer smakelijk kraakt}\\

\haiku{{\textquoteleft}Ik zou uwen broeder{\textquoteright} - - {\textquoteleft}?}{Tony wordt bloedroodof is}{het uw broeder niet}\\

\haiku{Voorwaarts dus op den,.}{weg waar een dubbel profijt}{hem te wachten staat}\\

\haiku{hij gevoelt hoe ver,,.}{Mos de lijkbidder boven}{hem verheven staat}\\

\haiku{{\textquoteright} De toekomst schemert,;}{zoo schoon als de eerste glans}{van den dageraad}\\

\haiku{{\textquoteright} maar van Dobbelsteen,.}{voegt de daad bij de woorden}{en ijlt den trap op}\\

\haiku{{\textquoteright} - nu, zeg ik, is het.}{zeker dat de jonker zijn}{hoofd verloren heeft}\\

\haiku{met zoodanig man,,.}{mijnheer Max vrees ik den last}{der dankbaarheid niet}\\

\haiku{eerbied voor hare.}{ouders is geenszins eene van}{Marietta's deugden}\\

\haiku{het hoofd zakt dieper.}{naar de borst en hij voelt zijn}{oog vochtig worden}\\

\haiku{noodigde hem te eten,}{schonk hem een goed glas wijn en}{gaf zijner dochter}\\

\haiku{Er waren sinds die.}{operatie verscheidene}{weken verloopen}\\

\haiku{Gouden spreuk, welke.}{in onzen tijd maar al te}{veel vergeten wordt}\\

\subsection{Uit: Werken. Deel 19. Karakters en silhouetten}

\haiku{De neus is recht, de.}{neusvleugels zijn scherp geteekend}{en sterk beweeglijk}\\

\haiku{Maar eensklaps zaten,, {\textquoteleft}{\textquoteright}.}{er zooals hij zegdezwarte}{beesten in zijn hoofd}\\

\haiku{Dit kleine puntje!}{voltooit zoo voortreffelijk}{mijne silhouet}\\

\haiku{In zijn grijsblauw oog;}{lag duidelijk bewustheid}{van eigenwaarde}\\

\haiku{Een hoofdtrek in zijn,.}{karakter was misnoegdheid}{soms lichtgeraaktheid}\\

\haiku{in zijnen pennetwist,.}{waarin hij afbrak wat hij}{vroeger opbouwde}\\

\haiku{In zijnen stand sprak,;}{men geen Fransch of wat men z\'o\'o}{gelieft te noemen}\\

\haiku{Die gedurige.}{strijd is voor mij eene tweede}{natuur geworden}\\

\haiku{Een eerlijk kamper.}{voor mijne overtuiging ben}{ik altijd geweest}\\

\haiku{zij moeten dus niet.}{meer rood worden als zij soms}{nog aan mij denken}\\

\haiku{Aan voorstellen om,.}{hooger te vliegen heeft het}{mij niet ontbroken}\\

\haiku{maar{\textquoteright} - zooals de oude - {\textquoteleft}...}{dichter van onzen Reynaert}{zegdedoe beter}\\

\haiku{daarenboven, de.}{hulpbronnen ontbraken hem}{in zijne woonstad}\\

\haiku{kortom, of hij een,.}{onnadenkend kind of een}{booze geest is geweest}\\

\haiku{Mij had het gevaar;}{des Vaderlands diep ontroerd}{en verontwaardigd}\\

\haiku{Ziedaar eene vraag, die.}{moeilijk of liever niet kan}{worden beantwoord}\\

\haiku{In 1842, gaf de Laet:}{een zoogezegd historisch}{verhaal in het licht}\\

\haiku{Die vraag kunnen wij,:}{niet beantwoorden tenzij}{met eene tweede vraag}\\

\haiku{geene Vlamingen, geene -,!}{Walen meer broeders zonen}{van eenen vrijen grond}\\

\haiku{In denzelfden vorm,,.}{schreef de dichter in 1860 een}{vers Aan Pius IX}\\

\haiku{Voorwaar, hij is een,!}{dergenen wiens naam nooit zal}{worden uitgewischt}\\

\haiku{Die herstelling had;}{plaats zonder een druppeltje}{bloed te vergieten}\\

\haiku{, laat het ons zeggen,;}{in zijn nachthemd en zijne}{witte slaapmuts op}\\

\haiku{Dat stond gelijk aan,.}{een halven schelling zegt de}{dichter nijdig}\\

\haiku{Weliswaar kende;}{de vorst hem eene jaarwedde}{van 1800 gulden toe}\\

\haiku{Men moest slechts op de,!}{steenrots slaan om een milden}{stroom te doen vloeien}\\

\haiku{omdat zij, onder,;}{meer dan \'e\'en opzicht aan uw}{land is verwantschapt}\\

\haiku{* * * ~ Ter inleiding:}{van mijn onderwerp vang ik}{aan met te vragen}\\

\haiku{En leer niet op wat, '.}{staat te maken Alstgeen in}{eigen krachten is}\\

\haiku{hij ziet er weinig ' -;}{naar oft nationaal}{is of niet uitsteekt}\\

\haiku{Slavinne is de,.}{vrouw maar slavinne in den}{vollen zin des woords}\\

\haiku{Vlaamsche vrienden toe,:}{toen men nog huiverig was}{van het Noorden}\\

\subsection{Uit: Werken. Deel 27. Klokketonen. Deel 1}

\haiku{maar ik, het zij mij,.}{vergeven ik vooral had}{trotsche denkbeelden}\\

\haiku{ten minste een nieuw.}{leven werd uit dat schijnbaar}{doodskleed geboren}\\

\haiku{maar de arme vrouw,:}{trok hem angstig voort en ik}{hoorde haar zeggen}\\

\haiku{{\textquoteleft}Ik zoek en vind het.}{in hetgeen hier beneden}{tastbaar voor mij ligt}\\

\haiku{Een Opperwezen,;}{erkennen is het schepsel}{aan banden leggen}\\

\haiku{De mensch,{\textquoteright} hervatte, {\textquoteleft};}{ikvordert dagelijks in}{wijsheid en verstand}\\

\haiku{Nu hingen al de.}{kleedingstukken naast elkander}{aan een langen rek}\\

\haiku{Ieder liep om het, -:}{zeerst want men dacht zelfs de}{honden misschien wel}\\

\haiku{D\`at pijnigde den,;}{vader d\`at deed de moeder}{heimelijk weenen}\\

\haiku{zij was de tweede.}{moeder voor de kleinere}{zusters en broeders}\\

\haiku{Op een bepaald uur.}{van den dag komt hij altijd}{voorbij mijn venster}\\

\haiku{Eens dat gevonden,...,.}{zal men doch neen ik loop hier}{wat al te haastig}\\

\haiku{Mevrouw van Torlits,,.}{was zoo als wij zeiden eene}{statige dame}\\

\haiku{Heil en eere aan!}{de wakkere afdeeling}{van St Nicolaas}\\

\haiku{{\textquoteright} {\textquoteleft}Dan zal ik toch nooit,{\textquoteright}.}{iets afgebedeld hebben}{zeide Barend trotsch}\\

\haiku{Een legio meesters;}{had aan het bed van den graaf}{van Buren gestaan}\\

\haiku{de ondergaande.}{winterzon wierp een flauwen}{gloed over het sneeuwveld}\\

\haiku{Gij ziet, lezers, dat,,!}{ik wel zijn Mephistopheles doch}{in het goede ben}\\

\haiku{{\textquoteright} pocht de bezoeker,.}{op hoogen toon en ontsteekt eene}{nieuwe manilla}\\

\haiku{Vergeeft den aardkluit,!}{in den zomer den sneeuwbal}{in den winter}\\

\haiku{hij was hard gelijk.}{het hart van een rechter van}{den scherpen zwaarde}\\

\haiku{maar, geloof mij, ik,,.}{geachte confrater ik}{heb er nooit ontmoet}\\

\haiku{Ge ziet, ik heb in {\textquotedblleft}{\textquotedblright}.}{mijn leven ook zoo al wat}{Lavater gespeeld}\\

\haiku{en ze zag in 't,;}{rond eerst naar den hemel en}{eindelijk naar mij}\\

\haiku{Op zekeren dag -,,;}{mijn vleugels waren gescheurd}{gehavend ontkleurd}\\

\haiku{{\textquoteleft}Waarom zet men zoo'n!}{misdadigen paraplue}{niet in het tuchthuis}\\

\haiku{Gij hebt gelijk, zoo'n!}{creatuur verdient den naam}{van engel niet meer}\\

\subsection{Uit: Werken. Deel 28. Klokkentonen. Deel 2}

\haiku{En zouden dat, wat,?}{verder op dezelfde hooge}{canadassen zijn}\\

\haiku{Ik had lust om op;}{te springen en tegen het}{glas te trommelen}\\

\haiku{Ik hoorde in de:}{duisternis eene vroolijke}{stem mij toeroepen}\\

\haiku{t... ging, naar het mij,;}{toescheen in  de verre}{verte verloren}\\

\haiku{'t zijn misschien de,.}{laatste de allerlaatste}{die ik schrijven zal}\\

\haiku{Ik droomde niet meer,,.}{maar de stem heb ik zeker}{zeer zeker gehoord}\\

\haiku{Hoe kwam ik toch aan?}{de herinnering van dien}{lang gestorven vriend}\\

\haiku{Die ernstige en:}{denkende geest had echter}{eene zwakke zijde}\\

\haiku{Loop er maar eenigen.}{tijd mede en hij wordt ruim}{en gemakkelijk}\\

\haiku{Ik ben van adel{\textquotedblright}, {\textquotedblleft}Wees,{\textquotedblright}.}{edel belieg de deugden van}{uwen stamvader niet}\\

\haiku{zij wakkerde de,,.}{spelers ongelukkige}{landskinderen aan}\\

\haiku{hoe drommel, heeft men!}{mij nu toch in dat hol doen}{verloren loopen}\\

\haiku{Flora stopte nog:}{altijd kousen en zeide}{met een diepen zucht}\\

\haiku{Niet overal is Mr.,,.}{Burtel zoo spraakzaam als bij}{Griet neen gewis niet}\\

\haiku{'t geeft niets, ik zal!}{de openingsrede wel voor}{de vuist uitspreken}\\

\haiku{Doch nog altijd is:}{er een  zwarte wolk aan}{Mr. Burtel's hemel}\\

\haiku{Overigens de man:}{heeft eene voortreffelijke}{voorzorg genomen}\\

\haiku{Neen, gewis, hij wist}{zelfs niet dat hij een zoo door}{en door geleerd man}\\

\haiku{- vandaag is zij als,.}{eene kleine provinciestad}{zoo stil zoo rustig}\\

\haiku{Ik liet, niet zonder,.}{ontroering het oog op de}{arme vrouw rusten}\\

\haiku{Hoor, weet je wat, ik, ';}{volg mijn zink Wil ook de}{wijde wereld in}\\

\haiku{Een burger, van mijn, ';}{rang en stand Moet nu en dan}{eens buitent land}\\

\haiku{{\textquoteright} {\textquoteleft}Onder de vrouwen.}{is die ziekte grooter dan}{onder de mannen}\\

\haiku{De meesten komen.}{hier om hare dochters op}{de markt te brengen}\\

\haiku{Tra la la la, les,,.}{demoiselles Tra la la}{la se forment l\`a}\\

\haiku{in huis is als stil,,,,:}{stil als een graf want ik de}{jongste ben er niet}\\

\haiku{Van tijd tot tijd werpt.}{moeder een onrustigen}{blik op de huisklok}\\

\subsection{Uit: De kraaien zullen 't uitbrengen}

\haiku{Welnu, ik houd het}{er nog altijd voor dat dit}{meisje zoo schuldig}\\

\haiku{Zij heeft haast om de.}{stad te verlaten en in}{het vrije veld te zijn}\\

\haiku{De koetsier loert nog.}{altijd onder de kap der}{antieke sjees uit}\\

\haiku{En dan - 't is niet... '}{goed bij klaarlichten dag in}{het dorp te komen}\\

\haiku{maar ook tevens werpt.}{zij een oogslag op den kaal}{wordenden mantel}\\

\haiku{{\textquoteleft}Hier zult ge moeten,,.}{afstappen want mijn weg loopt}{links de uwe recht voort}\\

\haiku{zij ziet in  het.}{ronde op de steenen vloer en}{zoekt in hare kleeren}\\

\haiku{Als wraakgenot in,,,}{ons hart woelt bidt men niet en}{al zou men bidden}\\

\haiku{doch zij kan nog niet,,.}{opstaan al zou zij de plaats}{waar zij zich bevindt}\\

\haiku{- op eene kar met wat,.}{stroo naar de stad en naar de}{gevangenis bracht}\\

\haiku{Het loover houdt nog aan;}{de takken en vormt dus nog}{eene dichte gordijn}\\

\haiku{'t Is alsof ze,,...}{beiden in eenen droom voor zich}{heeft zien drijven}\\

\haiku{Tilla zegt geen woord ' -;}{vant geen er in de kerk}{geschied is geen woord}\\

\haiku{Gij hebt den aard naar,,.}{uw vader Tilla die was}{ook zoo loszinnig}\\

\haiku{Werken zal ik voor.}{u en voor mij en God zal}{het overige doen}\\

\haiku{maar ik zeg ook, dat.}{over dien diefstal het licht nog}{niet geschenen heeft}\\

\haiku{Ik geloof dat een.}{eed bij hem veel lichter weegt}{dan een zilverstuk}\\

\haiku{{\textquoteright} {\textquoteleft}Juist, want mijn vader,,:}{en dat was een scherpzinnig}{man zei altijd}\\

\haiku{Soms zou Barend den,}{molen willen doen spreken}{dien Tilla's vader}\\

\haiku{{\textquoteright} {\textquoteleft}Indien ik zeker,.}{wist dat zij onschuldig was}{ik ging er op af}\\

\haiku{toen ik in het dorp,.}{kwam wonen viel Tilla mij}{altijd in het oog}\\

\haiku{'t is de priester. '}{die het Brood des levens aan}{een stervende brengt}\\

\haiku{Nu zijn de akkers,,;}{kaal de weiden vaalgroen de}{grachten gezwollen}\\

\haiku{{\textquoteright} {\textquoteleft}Den weg op naar het,.}{dorp tot aan het binnenpad}{door het kreupelhout}\\

\haiku{{\textquoteright} {\textquoteleft}Och, uitkomen zal,!}{het toch al zouden het de}{kraaien uitbrengen}\\

\haiku{Ze zit reeds in de....}{donkere gevangenis}{onder den toren}\\

\haiku{{\textquoteright} Eensklaps verschrikt zij,,.... '}{want nu zij de oogen opheft}{staat Evert Dils voor haar}\\

\haiku{Heeft zij niet meer dan,,,?}{te veel alles zelfs eer en}{naam opgeofferd}\\

\haiku{{\textquoteright} {\textquoteleft}Gij zorgdet wel mij,}{niet te ontmoeten en gij}{hebt gelijk gehad}\\

\haiku{Het denkbeeld aan de.}{stad schiet voorbij en Evert keert}{tot het dorp terug}\\

\haiku{Niemand beweegt zich.}{bij zijn buurman en dat maakt}{hem nog woedender}\\

\haiku{Wat een dwaasheid, ook!}{zelfs maar een oogenblik aan}{een droom te gelooven}\\

\haiku{hij begrijpt iets van.}{het edel gevoel dat zijnen}{vriend handelen doet}\\

\haiku{Hare schoenen en;}{de randen des mantels zijn}{door de sneeuw bemorst}\\

\haiku{Tilla heeft de groep;}{verlaten v\'o\'or dat deze}{aan de huizen komt}\\

\haiku{De andere is -?}{een blonde gij herinnert}{het u immers nog}\\

\haiku{Gaat het hem aan, dat?}{deze of gene door eene}{ramp getroffen wordt}\\

\haiku{Gij hecht, mijn beste,,?}{Sommer aan familienaam}{aan familie-eer}\\

\haiku{{\textquoteright} {\textquoteleft}Ik beken dat die,....}{acten zeer rechtskundig zijn}{opgesteld maar toch}\\

\haiku{voor mij persoonlijk:}{ware het te wenschen dat}{de zaak zou doorgaan}\\

\haiku{{\textquoteright} {\textquoteleft}Dan heeft de oude...{\textquoteright} {\textquoteleft},...}{SommerAls gij gezegd hadt}{dan heeft mijn vader}\\

\haiku{{\textquoteleft}hij of een zijner.}{medeplichtigen heeft die}{acte vervalscht}\\

\haiku{Reeds denzelfden avond;}{meldt zich Sommer opnieuw bij}{den advokaat aan}\\

\haiku{{\textquoteright} gilt Tilla, {\textquoteleft}moeder, '.}{wij gaan weer int huis van}{den molen wonen}\\

\subsection{Uit: Werken. Deel 29. De landverrader}

\haiku{{\textquoteright} {\textquoteleft}En het worde u!}{zevenmaal zeven malen}{we\^ergegeven}\\

\haiku{De een stond op den,;}{weg die naar de woning zijns}{vaders geleidde}\\

\haiku{Het gebeurde van.}{dien avond had geen overtuiging}{in zijn hart gestort}\\

\haiku{Het was alsof hij;}{zijn brandend hart door den drank}{verkoelen wilde}\\

\haiku{hij zag de scherpe,;}{trekken niet die zich op haar}{gelaat afteekenden}\\

\haiku{Gij dreigt mij als ik;}{u spreek van eene der grootste}{machten der aarde}\\

\haiku{door vaderlandsche,....}{kreten waarvan een spotlach}{de we\^ergalm is}\\

\haiku{Daarna greep hij zoo,.}{dikwijls den wijnbeker dat}{zijn geest bedwelmde}\\

\haiku{een weinig balsem.}{op mijne gewonde ziel}{te voelen leggen}\\

\haiku{Op dit voorhoofd,{\textquoteright} sprak, {\textquoteleft}....}{Bertholdzal de blos der}{leugen niet stijgen}\\

\haiku{Dat immers was ook.}{de eenige stem die rechtstreeks}{tot zijne ziel sprak}\\

\haiku{Berthold had bij.}{het krassen der deur het hoofd}{niet  opgelicht}\\

\haiku{{\textquoteleft}Dat nooit mijn zoon zich.}{op den verrader wreke}{om mijnentwille}\\

\haiku{nog aan het hart van,!}{den armen grijsaard die geen}{eigen kind meer heeft}\\

\haiku{Daar nog klemde de;}{grijsaard het meisje aan het}{hart en mompelde}\\

\haiku{haar hart was zelden.}{of nooit in overeenstemming}{met hare lippen}\\

\haiku{Aan gene zijde,.}{des heuvels was eene plek die}{hij nog groeten moest}\\

\haiku{Ik heb op eens al.}{het verschrikkelijke van}{mijn toestand gezien}\\

\haiku{{\textquoteright} Zij zweeg, staarde op.}{het vloertapijt en scheen koud}{als een marmerbeeld}\\

\haiku{{\textquoteright} riep zij glimlachend.}{en dronk den beker tot den}{bodem toe ledig}\\

\haiku{doch alvorens te,....}{sterven kan ik haar den kop}{nog verpletteren}\\

\haiku{Zal d\`a\`ar het beeld van?}{de gemartelde Paula}{mij nog vervolgen}\\

\haiku{Zal d\`a\`ar het schrikbeeld?}{mijns vaders mij nog dreigend}{in den weg treden}\\

\subsection{Uit: Werken. Deel 46. Maria Stuart}

\haiku{duld echter dat ik,:}{ze u te binnen brenge}{en dat ik u zeg}\\

\haiku{Die leugen was noodig,:}{om het zich voorgestelde}{doel te bereiken}\\

\haiku{Zij zag gewis al,;}{de bannelingen die zij}{had uitgeplunderd}\\

\haiku{Diep was zij ontroerd, {\textquoteleft},,{\textquoteright}, {\textquoteleft}}{Staak uw kermen en weenen}{Melvil zeide zij}\\

\haiku{uwe meesteresse,....}{uwe koningin beveelt zich}{in uwe gebeden}\\

\haiku{{\textquoteright} Waarin was toch, in?}{die laatste omstandigheid}{den moed gelegen}\\

\haiku{Het vooroordeel moet.}{echter te dien tijde wel}{diep ingeworteld}\\

\haiku{De verklaringen,,;}{zijn op meer dan \'e\'en punt met}{elkander in strijd}\\

\haiku{Z\'o\'o eenzaam wonen,, '!}{zal men zeggent is om}{van te huiveren}\\

\haiku{een zoo onbepaald.}{vertrouwen heeft hij in de}{schietkunst van Donaat}\\

\haiku{De ketting, waaraan,.}{Vos vastligt ratelt over den}{dorpel van zijn hok}\\

\haiku{En daar zou, zegde,!}{de citoyen glimlachend}{geen haan naar kraaien}\\

\haiku{Gij hebt Donaat door,!}{uwe oproerige taal den}{kop op hol gebracht}\\

\haiku{Waarom is hij niet,!}{gisteren waarom over een}{uur niet weggegaan}\\

\haiku{Het meisje hoort niet;}{rechtstreeks in den boerenstand}{onzer Kempen thuis}\\

\haiku{Hij had zich willen;}{wreken op de feestkle\^eren}{der dorpelingen}\\

\haiku{Was de plechtigheid,?}{nog niet begonnen of was}{zij reeds voltrokken}\\

\haiku{Maar het doet mij zeer.}{aan het hart dat gij voor mij}{uw geluk vergeet}\\

\haiku{Gisteren waart gij,;}{een kind der weelde gij hadt}{al wat gij droomdet}\\

\haiku{Hij vernietigde,}{zeer behendig het werk van}{den ouden pastoor}\\

\haiku{met de koorts op het,}{lijf zong hij zijn lied en in}{den afgeloopen}\\

\haiku{De man zat juist in;}{den warmen leuningstoel voor}{het vuur te suffen}\\

\haiku{{\textquoteright} {\textquoteleft}Mijn zoon heeft het hart,,!}{op de rechte plaats zitten}{en is och arme}\\

\haiku{hij had de macht niet.}{meer zijne krukken door de}{sneeuw voort te slepen}\\

\haiku{de kinderstemmen,,:}{rein en helder zingen het}{lied van Tollens}\\

\haiku{{\textquoteright} {\textquoteleft}In wat betrekking?}{staat hij met het huisgezin}{des burgemeesters}\\

\haiku{{\textquoteright} Den volgenden dag}{vertelde Jasper Hompel}{mij gedeeltelijk}\\

\subsection{Uit: Werken. Deel 45. De nachtraven}

\haiku{Dat alles komt bij.}{de twee nachtraven echter}{niet in aanmerking}\\

\haiku{Zijne houding en.}{zijn gang kenmerken iets meer}{dan achteloosheid}\\

\haiku{De deftigen uit;}{zijnen stand vermijden hem}{zooveel mogelijk}\\

\haiku{Dat meisje, rijzig,:}{van gestalte kon zoo wat}{veertien jaar tellen}\\

\haiku{zoodra hij den -.}{voet in zijn eigen huis zet}{de waskaars ontsteekt}\\

\haiku{De bezoeker klopt,.}{want een bellentrekker houdt}{men er niet op na}\\

\haiku{Gorl is de type,;}{van een eerlijk gezond en}{verstandig werkman}\\

\haiku{'t is waar ook, een....}{jaar geleden ontmoette}{ik hem bij toeval}\\

\haiku{{\textquoteright} {\textquoteleft}Ja, maar dit wil niet,.}{zeggen dat ik haar gaarne}{zou willen kwijt zijn}\\

\haiku{deze speculeert,:}{evenmin op de diensten die}{het kind hem bewijst}\\

\haiku{{\textquoteright} {\textquoteleft}Neen, Gorl, zij moet er,.}{meer geweest zijn want zij danst}{als eene duivelin}\\

\haiku{De vereelte hand.}{des werkmans grijpt die van Murg}{en drukt ze dankbaar}\\

\haiku{Neen, zij heeft niets van,.}{den vader zij heeft misschien}{iets van de moeder}\\

\haiku{in een kristallen;}{vaas leunde een verlepte}{en verdorde bloem}\\

\haiku{gij hebt nog altijd.}{dezelfde bedoelingen}{als v\'o\'or eenigen tijd}\\

\haiku{De woorden van den:}{ouden heer kwamen Anna}{Eerling voor den geest}\\

\haiku{De voorspelling van.}{den ouden heer werd in haar}{meer en meer waarheid}\\

\haiku{De arme dwaze -;}{kreeg voor dons en hermelijn}{koude sneeuwvlokken}\\

\haiku{{\textquoteright} Bij het uitspreken.}{dezer woorden neemt zij het}{licht en gaat hem voor}\\

\haiku{{\textquoteleft}Hiertoe,{\textquoteright} zegt ze, {\textquoteleft}zou,;}{ik het recht hebben want uw}{naam is de mijne}\\

\haiku{{\textquoteleft}Toen ge laatst bij mij,{\textquoteright}, {\textquoteleft}}{waart hervat Gorlwilde ik}{er niet van hooren}\\

\haiku{Deze ook maakt zich,,;}{als bij ingeving gereed}{om te vertrekken}\\

\haiku{{\textquoteright} {\textquoteleft}Kom,{\textquoteright} zegt het meisje, {\textquoteleft},,.}{plotselingkom laat ons naar}{huis gaan grootvader}\\

\haiku{maar bij Nelia heeft.}{die stem zelfs nog niet in de}{verte geklonken}\\

\haiku{want zij doen haar aan.}{de gewijde kaars en den}{doodwagen denken}\\

\haiku{oprecht is hij in,;}{deze onderhandeling}{niet geweest slim wel}\\

\haiku{Nelia blikt door de;}{openstaande schuifdeur van den}{goederenwagen}\\

\haiku{{\textquoteright} {\textquoteleft}Ja wel, een bezoek,.}{dat ik u v\'o\'or jaar en dag}{had moeten brengen}\\

\haiku{Ik versta u niet...{\textquoteright}.}{en de schilder houdt altijd}{de hand aan het oor}\\

\haiku{De Goede Herder,;}{is echter geen gesticht voor}{verlorenen neen}\\

\haiku{De kinderen, die,;}{naar de kapel gaan komen}{kwetterend voorbij}\\

\haiku{{\textquoteright} en als hij wakker,.}{is en opschiet waant hij dat}{Nelia buiten staat}\\

\haiku{De tegenpartij,,;}{is als heer ofschoon eenigszins}{verwaarloosd gekleed}\\

\haiku{bij Murg brengt zij al.}{niet veel meer teweeg dan een}{schouderophalen}\\

\haiku{{\textquoteright} {\textquoteleft}Ik mediteerde,{\textquoteright},;}{zegt Murg met dien cynieken}{lach hem zoo eigen}\\

\haiku{{\textquoteright} luidt Murg's antwoord.}{en hij legt het oor tegen}{de deur en luistert}\\

\haiku{Kortom, Bronveld heeft;}{zijne slechte gewoonte}{niet vaarwel gezegd}\\

\haiku{met niemand heeft zij,;}{gesproken over de reis die}{ze gaat beginnen}\\

\haiku{Koortsachtig heft hij:}{plotseling de hand op ter}{hoogte van het hart}\\

\haiku{doch wie kon tot in?}{haar gemoed doordringen en}{haar beoordeelen}\\

\haiku{De klokketoon en.}{de beelden uit mijne jeugd}{hebben mij gered}\\

\haiku{ik herinner mij.}{toch hem een of twee malen}{gezien te hebben}\\

\haiku{{\textquoteright} {\textquoteleft}Gij zijt mij welkom,,{\textquoteright};}{nichtje liet er de goede}{dame op volgen}\\

\haiku{Ja, alles kwam het -.}{meisje voor als een droom een}{droom van den hemel}\\

\haiku{De machinist kent,:}{den binnentreder niet al}{zegt deze dan ook}\\

\haiku{Bronveld zit neer, knoopt.}{den overjas los en slaat den}{kraag naar beneden}\\

\haiku{Toen ik jongeling,,;}{was Gorl had ik lust om mij}{bezig te houden}\\

\haiku{Ik wilde de huur,...}{mijner woning opzeggen}{toen mijne dochter}\\

\haiku{haar wijzer loopt niet,;}{in wilde vaart heen en weer}{zooals de schaduwen}\\

\haiku{personen uit die,;}{wereld welke gewoon is}{grof te verteren}\\

\haiku{En dan die oude {\textquoteleft}{\textquoteright}...}{constschilder en die nog veel}{gekker apotheker}\\

\haiku{De zon verdwijnt in,.}{eene zee van ineensmeltend}{goud rood en purper}\\

\haiku{doch zij zal zich wel.}{wachten die bedreiging ten}{uitvoer te brengen}\\

\haiku{Rup wil naar buiten;}{zien en nagaan wat er zooal}{op den weg gebeurt}\\

\haiku{Nu Rup den ruiter,:}{nadert bevangt hem eensklaps}{eene zekere vrees}\\

\haiku{'t is of zij niet.}{de minste belangstelling}{voor Romald heeft}\\

\haiku{{\textquoteright} {\textquoteleft}In alle geval....}{kan dit aan de dochter niet}{verweten worden}\\

\haiku{Ik heb die neiging.}{tot afdaling in hem steeds}{moeten bevechten}\\

\haiku{doch het karakter;}{haars echtgenoots heeft zij nooit}{kunnen doorgronden}\\

\haiku{Mevrouw gaat langzaam,,;}{in gedachten verzonken}{door den bloementuin}\\

\haiku{{\textquoteleft}Romald, uwe moeder,,.}{verlangt evenals gij dat uw}{vader terugkeere}\\

\haiku{Mevrouw ontvangt de,;}{gasten in haar klein doch net}{gemeubeld salon}\\

\haiku{het mist echter den;}{waren toon om bij Nelia}{gehoor te vinden}\\

\haiku{{\textquoteright} {\textquoteleft}Ik ben, ik herhaal ',...}{t veeleer ongelukkig}{dan wel slecht geweest}\\

\haiku{Op eenige stappen:}{van het hekken staat hij stil}{en mompelt lachend}\\

\haiku{Een half uur later,;}{zit Murg in de herberg waar}{wij hem aantroffen}\\

\haiku{'t Is echter de,,.}{Bloemstraat niet die haar aantrekt}{maar wel grootvader}\\

\haiku{{\textquoteleft}Belooft ge mij de?}{geheimhouding van hetgeen}{ik ga meedeelen}\\

\haiku{Gij zijt,{\textquoteright} zoo spreekt hij, {\textquoteleft}.}{voortin uwe denkbeelden geen}{man van onzen tijd}\\

\haiku{{\textquoteright} {\textquoteleft}Indien ge mij van....}{tijd tot tijd iets over Romald}{wildet  schrijven}\\

\haiku{doch haar gemoed is,.}{op dit oogenblik week zij}{hoort haar hart kloppen}\\

\haiku{Eene eentonige,;}{weide levert niet veel spel}{op zal men zeggen}\\

\haiku{Dat onstaat om zoo.}{te zeggen onder den trap}{van zijne klompen}\\

\haiku{dat mijne moeder.}{mijnen vader vrijpleit ten}{nadeele van den uwen}\\

\haiku{'t Is of de stem:}{van Nelia andermaal aan}{Romald's oor suizelt}\\

\haiku{men lachte volop,,,?}{men mocht er cyniek zijn als}{een hond niet waar Murg}\\

\haiku{Drie of vier hoofden;}{verschijnen andermaal in}{de opening der deur}\\

\haiku{De familie van;}{Segelaer had het kasteel}{nog niet verlaten}\\

\haiku{Mevrouw valt jufvrouw,.}{Monica in de armen}{en beiden weenen}\\

\haiku{Op klompen voord eene, '.}{Excellentie verschijnen}{t is noodlottig}\\

\haiku{{\textquoteright} Een glimlach speelt over,.}{Bert's bleek wezen doch geen woord}{komt op de lippen}\\

\haiku{door eene andere,.}{stelling op te werpen het}{gesprek afleiden}\\

\haiku{Machteloos worstelt,.}{gij omdat het vergeten}{niet menschelijk is}\\

\haiku{{\textquoteright} De dokter neemt zacht.}{en vertrouwelijk hare}{hand in de zijne}\\

\subsection{Uit: Werken. Deel 26. Oranje in de Kempen}

\haiku{want daar had men een;}{paar knechten van ridder de}{Knuyt zien binnengaan}\\

\haiku{{\textquoteright} {\textquoteleft}Daarin geef ik je,}{volkomen gelijk en ik}{zal dien praatzieken}\\

\haiku{Hij herinnerde,.}{zich niet den sjacheraar ooit}{te hebben gezien}\\

\haiku{De jood moest even snel,;}{loopen als hij want de stem}{klonk immer even luid}\\

\haiku{{\textquoteleft}Nah, mijnheer Ralph, je.}{zoudt een oud man waarachtig}{den adem afnemen}\\

\haiku{dat weet menheer Ralph{\textquoteright} {\textquoteleft},,....}{het best.Nog eens verklaar u}{of voor den drommel}\\

\haiku{Wij zouden in den,}{donkeren avond en vooral}{niet op dien welken}\\

\haiku{Het vertrek, waarin,;}{wij de vrouw ontmoeten was}{goed gemeubeleerd}\\

\haiku{Eindelijk meende;}{zij gerucht te hooren aan}{de kleine straatpoort}\\

\haiku{Geduld dus, zet u,;}{ne\^er wees bedaard en vertel}{mij het gebeurde}\\

\haiku{niemand beter dan.}{hij kan den sluier van het}{geheim oplichten}\\

\haiku{{\textquoteright} Dat woord bracht een waas;}{van droefheid over het wezen}{der oude dame}\\

\haiku{{\textquoteright} riep Retha verschrikt,.}{en Elie drong zich dichter bij}{de oude dame}\\

\haiku{Ja, Retha, gij hebt;}{een trek der Midletown's}{in het aangezicht}\\

\haiku{hij trachtte rechts en.}{links inlichtingen over Eric}{Ralph te bekomen}\\

\haiku{{\textquoteleft}Kalverliefde,{\textquoteright} dacht, {\textquoteleft}!}{de kramerdie geen roode}{duit winst oplevert}\\

\haiku{{\textquoteright} {\textquoteleft}Ik ben zeker dat.}{gij mij de kleinste helft der}{som hebt gegeven}\\

\haiku{Wel man,  dan zal,!}{het nog wel noodig zijn dat ik}{u van kant helpe}\\

\haiku{Darvis, zeg toch aan {\textquotedblleft}{\textquotedblright},!}{menheer je hond dat hij me}{vleesch respicteert}\\

\haiku{{\textquoteleft}Mozes zelf zou zich,{\textquoteright}.}{aan de galg klappen als hij}{sprak hervatte hij}\\

\haiku{{\textquoteleft}En wanneer zal de?}{prinses van Oranje in de}{Vrijheid aankomen}\\

\haiku{{\textquoteright} onderbrak een klein,.}{en mager potbakkerken}{op nijdigen toon}\\

\haiku{Dat zal wel niet in,{\textquoteright}.}{deze omstandigheden}{meende de wever}\\

\haiku{{\textquoteleft}Darvis, zou dat de,?}{gevreesde ossenkooper}{met zijn rooden hond zijn}\\

\haiku{Het scheen wel, dat de.}{6 guldens 12 stuivers hem}{in den zak dansten}\\

\haiku{Het papier was geel;}{en blijkbaar een schutblad van}{een kerkboek geweest}\\

\haiku{hij sprong recht, alsof,.}{het de dood was geweest die}{bij hem aanklopte}\\

\haiku{{\textquoteleft}Als waon de galg,,.}{zen mot hekik blaose}{zoo heet thum gezeed}\\

\haiku{{\textquoteleft}Voorwaarts naar....{\textquoteright} en de.}{rest fluisterde hij in het}{oor van den vorster}\\

\haiku{Nu volgt de gilde,{\textquoteright}.}{van St.-Sebastiaan}{hervatte Quinten}\\

\haiku{De Jonggesellen;}{voerden er hunne schoonste}{muziekstukken uit}\\

\haiku{{\textquoteright} {\textquoteleft}Kom, kom, laat ons op,!}{zoo'n vroolijken dag niet al}{te streng worden Eric}\\

\haiku{{\textquoteleft}Kom,{\textquoteright} zeide Eric stil, {\textquoteleft}?....}{willen wij nog even door de}{warande loopen}\\

\haiku{{\textquoteright} De prinses had de:}{oude vrouw opgericht en}{zeide zeer ontroerd}\\

\haiku{Ik beschouwde ze,.}{als de mijne en noemde}{ze Elie en Retha}\\

\haiku{Ralph zette zich ne\^er.}{en wischte tranen uit zijn}{zoo mannelijk oog}\\

\haiku{{\textquoteleft}Laat ons daarover niet,,{\textquoteright}.}{spreken Harry zeide het}{meisje na een poos}\\

\haiku{doch in den loop des.}{avonds dacht zij meer dan eens aan}{den zoon des ballings}\\

\haiku{hij droeg een kleinen,.}{hoed met witte ve\^er en had}{hooge rijlaarzen aan}\\

\haiku{Darvis was gewis,,;}{niet daar zoo wel dan ware}{hij reeds verschenen}\\

\haiku{De lieve hemel!}{weet wat al rijkdommen daar}{verborgen zaten}\\

\haiku{Zonder twijfel wel,!}{de guldens die aan Mozes}{ontstolen waren}\\

\haiku{nog geruimen tijd.}{hoorde hij het geblaf en}{gebrul van den hond}\\

\subsection{Uit: Werken. Deel 13. De orgeldraaier}

\haiku{{\textquoteright} ging hij voort met een, {\textquoteleft}...{\textquoteright} {\textquoteleft},?}{diep ontroerde stemlaat mij}{Ha ik versta u}\\

\haiku{waarom plaag ik toch!}{mijne arme hersens met}{die gekke droomen}\\

\haiku{{\textquoteright} sprak hij plotseling, {\textquoteleft}.}{op zachter toon voortook dat}{zal ik verdragen}\\

\haiku{hij greep de hand der.}{jonkvrouw en bracht die bevend}{aan zijne lippen}\\

\haiku{Gij gelooft niet aan,!}{de verklaring dat ik uw}{vader ben kindlief}\\

\haiku{{\textquoteright} {\textquoteleft}Ik ook, ik dank den,.}{hemel dat hij mij nog eens}{in uw bijzijn brengt}\\

\haiku{Het is niet alleen,;}{het ongeluk der scheiding}{dat mij treft Willem}\\

\haiku{laat de schemering;}{in het straks ontgloeiende}{Oosten opdagen}\\

\haiku{De beschuldigde,,:}{die als neergedrukt zit is}{u niet onbekend}\\

\haiku{De beschuldigde.}{zag den rechter strak en met}{ontsteltenis aan}\\

\haiku{Ik beminde het,,.}{gedruisch der vrienden de}{feesten het gewoel}\\

\haiku{Neen, het was in geen,.}{eerlijk tweegevecht dat de}{misdaad gepleegd werd}\\

\haiku{Droeg ik de wroeging,?....}{niet met mij mee drukte ik}{haar niet aan het hart}\\

\haiku{Tienmaal heb ik dat,,;}{kermend wicht in mijne vlucht}{achtergelaten}\\

\haiku{{\textquoteleft}'t Is waar, het is,{\textquoteright}:}{uw kind zegde er eene in}{het hart van Willem}\\

\haiku{Voor een paar dagen;}{nog dacht ik alle hoop te}{moeten opgeven}\\

\haiku{zelfs eene fortuin die.}{groot en een wapenschild dat}{onbezoedeld is}\\

\haiku{Heb ik ongelijk,,?}{lezer u zoo een geluk}{toe te wenschen}\\

\subsection{Uit: Werken. Deel 44. De verstooteling. Stomme Nora. Kreupele Dorus}

\haiku{Tafereelen uit,.}{onzen Tijd I.   De Dood}{van Piko-Poko}\\

\haiku{Piko-Poko, zoo,.}{heette de aap was ook zijn}{leven en bestaan}\\

\haiku{hij sprak echter geen.}{woord en richtte het gelaat}{weer naar zijn dooden vriend}\\

\haiku{{\textquoteleft}maar ik herinner.}{mij flauw dat mijne moeder}{eene kaartlegster was}\\

\haiku{{\textquoteright} was het antwoord, en.}{er gleed een lichten glimlach}{om de bleeke lippen}\\

\haiku{Ik denk altijd dat;}{niemand zoo veel recht heeft om}{te weenen als ik}\\

\haiku{{\textquoteleft}Ik werd zinneloos,.}{van smart en lijden maar God}{was goed jegens mij}\\

\haiku{men heeft mij dat slechts.}{gezegd om mijne diepe}{smart te stillen}\\

\haiku{Met geopende:}{armen liep ik binnen en}{riep in verrukking}\\

\haiku{Gij verstaat dat woord,,.}{niet gij die nooit de vrijheid}{hebt moeten missen}\\

\haiku{doch hoe kaal de jas,;}{ook zij dien hij tot aan den}{hals heeft opgeknoopt}\\

\haiku{pas op, raak geen pijl, -!}{van zijn hoofdhaar aan of het}{zal u berouwen}\\

\haiku{Zou die jongeling,,?}{zoo dacht hij misschien aan de}{policie denken}\\

\haiku{{\textquoteright} {\textquoteleft}Ja, daar dichtbij is,}{Piko-Poko gestorven}{en ook niet ver van}\\

\haiku{met den oppersten.}{meester van het zwervende}{Bohemer-volk}\\

\haiku{Daar lag ik op de,}{straat te spartelen naakt als}{een afgesleten}\\

\haiku{- maar wat maakt ons die,.}{Wij wij kussen den kroes en}{zijn zeker van vreugd}\\

\haiku{hij gezeten had,.}{en wilde er gewis een}{wapen van maken}\\

\haiku{Kristiaan wilde -!}{hen bijstaan en beschermen}{de arme jongen}\\

\haiku{{\textquoteleft}Ga heen, gij hebt u!}{immers met onze zaken}{niet te bemoeien}\\

\haiku{Boven de tent van:}{Tamerlan leest men op een}{versleten doek}\\

\haiku{Hij sliep, en het was;}{of God hem in zijnen slaap}{een zoeten droom gaf}\\

\haiku{Vijftien jaar heb ik....}{nu reeds in armoede en}{lijden doorgesloofd}\\

\haiku{{\textquoteright} {\textquoteleft}Maar wie dan toch is?}{zoo onbarmhartig geweest}{mij aan te klagen}\\

\haiku{Hij vraagt lucht, leven,!}{en liefde en men versmoort}{den rampzalige}\\

\haiku{er lag eene blauwe.}{tint over zijne lippen en}{onder zijne oogen}\\

\haiku{{\textquoteleft}Wat voerde u op,?}{dit oogenklik in dit huis}{in deze kamer}\\

\haiku{Ik heb soms ademloos,.}{en met kloppend hart u van}{daar g\^ageslagen}\\

\haiku{Ik vraag u enkel,.}{laat dien ongelukkigen}{grijsaard in vrede}\\

\haiku{{\textquoteleft}Ik zal u aan mijn, {\textquoteleft},.}{harte kluistrenD\`a\`ar waar men}{nooit het scheiden vindt}\\

\haiku{{\textquoteleft}Wij zullen zien, of.}{zij ons het goed geluk zal}{kunnen voorspellen}\\

\haiku{want een licht purper.}{overtoog haar gelaat bij het}{hooren van dien naam}\\

\haiku{de andere had.}{wel eens willen pogen in}{onmacht te vallen}\\

\haiku{Men zegt, Zingolina,?}{dat gij in de sterren en}{in de kaarten leest}\\

\haiku{{\textquoteright} Er heerschte eene;}{gespannen verwachting in}{de vergadering}\\

\haiku{Agnes deed hem naar -!}{de keuken geleiden en}{d\`a\`ar o welk geluk}\\

\haiku{{\textquoteright} Het wijf, wier trekken,;}{niets goeds voorspelden scheen de}{vraag niet te verstaan}\\

\haiku{Hier rustte sedert!}{vijftien jaren het eenige}{dat gij achterliet}\\

\haiku{aan deze zijde,!}{staat uw vader aan gene}{zijde uw meester}\\

\haiku{Mahomet den geest,.}{gaf legde men den reus op}{eene lange tafel}\\

\haiku{ik zal, ik moet het,!}{geheim weten dat mijne}{geboorte bedekt}\\

\haiku{Met den laster en,;}{den vloek op de lippen stierf}{de schuldige man}\\

\haiku{zij had zoo gedwee}{als een lam ne\^ergeknield}{als het den kleinen}\\

\haiku{zij heeft niet alleen,}{in Kristiaan een vriend maar}{ook in den armen}\\

\haiku{dien gij nu we\^er in....}{de handen der policie}{hadt overgeleverd}\\

\haiku{{\textquoteleft}Ik geloof echter.}{niet dat gij mij eenig goed zoudt}{kunnen aanbrengen}\\

\haiku{Kristiaan gaf geen;}{acht op den gemoedstoestand}{van den rijken oom}\\

\haiku{Als Nora echter,}{den zoon van den molenaar}{den zwartlokkigen}\\

\haiku{Alles scheen stil en;}{akelig in zijn doodslaken}{te zitten droomen}\\

\haiku{Simon maakte zich.}{zekeren morgen gereed}{om te vertrekken}\\

\haiku{men iederen avond.}{voor de rust der ziele van}{de overledene}\\

\haiku{hij speurde den haas;}{en het schuchtere konijn}{op het sneeuwveld na}\\

\haiku{{\textquoteright} - en bij die woorden.}{glinsterde er een edel vuur}{in het oog des knaaps}\\

\haiku{{\textquoteleft}Maar denkt hij dan, dat?}{ik mij zoo maar straffeloos}{zou laten straffen}\\

\haiku{{\textquoteright} De dood kwam echter;}{meer dan eens op den rand der}{bedsponde zweven}\\

\haiku{Met den kogel dien,!}{gij kreupelen Dorus eens in}{het been schoot vader}\\

\subsection{Uit: Werken. Deel 38. Villa Pladelle}

\haiku{Nu heb ik haast om,.}{op het slot te komen waar}{het avondmaal mij wacht}\\

\haiku{Welnu, gij zult Hark,....}{tot leenman verheffen hem}{rijk begiftigen}\\

\haiku{Kloek en stout is hij,.}{als een echte zoon uit het}{grimmige noorden}\\

\haiku{Na eene poos in het,.}{vertrek vertoefd te hebben}{ging Karl naar buiten}\\

\haiku{Om zijnen hals hing.}{eene zware gouden ketting}{met breede schakels}\\

\haiku{aan het koningschap,,.}{zoo dacht de heerschzuchtige}{zouden bewijzen}\\

\haiku{{\textquoteleft}Hoe,{\textquoteright} zegde deze, {\textquoteleft},,?}{gij hier vrouw Ragna en}{waarom niet ginder}\\

\haiku{{\textquoteright} zegde Dirk bleek en, {\textquoteleft}?}{ontsteldzijt ge wel zeker}{van hetgeen ge zegt}\\

\haiku{{\textquoteright} {\textquoteleft}Ja,{\textquoteright} antwoordde Karl, {\textquoteleft},,,.}{en wat Dirk mijn getrouwe}{Manne zegt doet hij}\\

\haiku{{\textquoteleft}Wat verlangt ge van,?}{Hark van den heer en meester}{op Grimma herna}\\

\haiku{want de gevlochten,,.}{horde die voor deur diende}{was andermaal toe}\\

\haiku{Die naam tooverde eene.}{glinstering van hoop op de}{lippen van Simplex}\\

\haiku{{\textquoteright} De breede borst van.}{den pelsendief ging als een}{blaasbalg op en neer}\\

\haiku{De nieuwsgierigen.}{bleven op eenen afstand en}{volgden schoorvoetend}\\

\haiku{{\textquoteright} zegde de moeder,.}{zacht den arm op dien haars zoons}{latende rusten}\\

\haiku{Al de omstanders.}{waren diep bewogen en}{velen zelfs weenden}\\

\haiku{Die rijke vrouw zou,,.}{zich wreken zooals zij deed zooals}{zij reeds gedaan had}\\

\haiku{Een uur later ving.}{de grafelijke stoet de}{terugreize aan}\\

\haiku{41Vrijen droegen,.}{zilveren lijfeigenen}{koperen ringen}\\

\subsection{Uit: Werken. Deel 33. De voetbranders}

\haiku{Voor wapens hadden,.}{ze verroeste geweren}{pieken en lansen}\\

\haiku{Op eenigen afstand,;}{voor mij uit zag ik twee of}{drie groote wachtvuren}\\

\haiku{Op die hoogte zag,,.}{ik als in zegepraal over}{gansch het leger heen}\\

\haiku{In een oogenblik;}{vielen de woestaards in de}{eenzame woning}\\

\haiku{later kwam het mij.}{in al zijne schoonheid en}{reinheid voor den geest}\\

\haiku{Vier mannen droegen,.}{blootshoofds eene doodkist met een}{zwart baarkleed bedekt}\\

\haiku{Ik voelde dien avond:}{iets wat ik nooit voor mijne}{moeder gevoeld had}\\

\haiku{de soldaten en,.}{vergiel-jotineeren zooals de}{republikanen doen}\\

\haiku{We gingen op de;}{puinen en voelden of de}{steenen nog warm waren}\\

\haiku{Op den avond dat de,;}{Franschen in het dorp vielen}{keerde hij terug}\\

\haiku{{\textquoteright} stookte hij op, even.}{als men twee honden tegen}{elkander opjaagt}\\

\haiku{{\textquoteright} riep ik plotseling.}{opspringende en heel de}{bende sprong overeind}\\

\haiku{Abel, hield hem bij zich,.}{en zegde daarna dat Abel}{niet meer mocht weggaan}\\

\haiku{Hij was ernstig en.}{zijn aangezicht was bleek als}{dat van eenen doode}\\

\haiku{Maar waarom steken?...}{deze ook hun hoofd boven}{de anderen uit}\\

\haiku{Gij zijt tevreden,,?}{van de Franschen ontslagen}{te zijn niet waar Niels}\\

\haiku{die zijn vertrokken,.}{nadat ze het halve dorp}{hebben afgebrand}\\

\haiku{Maar Jan wist op slot.}{van rekening veel meer te}{vertellen dan ik}\\

\haiku{Jan, die gaarne de {\textquoteleft}}{verzekering gaf vanzoo}{zeker als tweemaal}\\

\haiku{Dat was het gevolg,,.}{beweerde men van het leenen}{op woekerintrest}\\

\haiku{- de andere was,;}{de lange  zwikzwak maar}{die was niet gekwetst}\\

\haiku{Ik zag onwillens,:}{naar den langen neus van den}{zwikzwak en dacht zoo}\\

\haiku{De kring, rondom het,;}{vuur stoof uit elkander als}{een bende musschen}\\

\haiku{Op den bok zat een,;}{gewonde wiens hoofd door eenen}{doek omzwachteld was}\\

\haiku{De arme duivel,,;}{smeekte wellicht om hulp om}{troost om lafenis}\\

\haiku{de molenaar sneed,,.}{met een groot mes brokken brood}{in den haverbak}\\

\haiku{De wagenmaker.}{hamerde zoo wat aan den}{krakenden wagen}\\

\haiku{maar alleen de bleeke.}{gekwetste wilde zijnen}{zegen ontvangen}\\

\haiku{oud en jong rees op '.}{ent smeeken der vrouwen}{mocht niet meer baten}\\

\haiku{Op een bepaald punt.}{in het dorp zullen wij ons}{weer vereenigen}\\

\haiku{en snelde daarna.}{voort om aan Roodwammes het groote}{nieuws me\^e te deelen}\\

\haiku{De zon brak met eenen.}{stroom van purper en goud door}{de grauwe wolken}\\

\haiku{De nadering van,.}{een dreigend gevaar is een}{wonderlijk gevoel}\\

\haiku{Slechts eene voorwacht had;}{een oogenblik aan de brug}{derf aanval beproefd}\\

\haiku{want deze lichtte:}{de rechterhand op en ik}{hoorde hem zeggen}\\

\haiku{zijne uniform hing,,.}{aan flarden was vuil besmeurd}{en onkennelijk}\\

\haiku{Soms hief ik het oog.}{op en ontmoette dan zijn}{toegenegen blik}\\

\haiku{het was, of wij voor.}{hem defileerden en hij}{ons kommandeerde}\\

\haiku{Na eene poos vertoefd,;}{te hebben kropen de twee}{plunderaars verder}\\

\haiku{Ik wendde nog eens, {\textquoteleft}{\textquoteright};}{het hoofd om groette met een}{luidHoerrah Roodwammes}\\

\haiku{Wat was het feest op!}{het buitengoed en in het}{naburige dorp}\\

\haiku{De voetbranders    .}{Noten 1De gemalin van}{den prins van Oranje}\\

\subsection{Uit: Werken. Deel 6. Waar is de vader?}

\haiku{Hij weet het wel, doch.}{hij heeft zich daarover tot nu}{toe niet bekommerd}\\

\haiku{maar het juiste woord.}{van dat alles is bij het}{publiek niet bekend}\\

\haiku{Als vorm mocht daar soms,;}{wel iets op af te wijzen}{vallen  gewis}\\

\haiku{{\textquoteleft}maar het verlangen,,,.}{en dit verlangen moet u}{denk ik heilig zijn}\\

\haiku{{\textquoteright} {\textquoteleft}Ik dank u, mijnheer,{\textquoteright}.}{Albert en mevrouw stak den}{jongman de hand toe}\\

\haiku{{\textquoteright} vroeg zij en dit niet.}{zonder eene trilling in de}{stem te verraden}\\

\haiku{daarenboven, de;}{omstandigheden zijn soms}{onverbiddelijk}\\

\haiku{Na een tweeden ruk;}{aan de bel deed zich binnen}{eenig gedruisch op}\\

\haiku{tegenover haar, aan,:}{den muur hing het portret van}{den overledene}\\

\haiku{Hij stond op en bleef.}{in gedachten verzonken}{voor het venster staan}\\

\haiku{En echter, als gij,,.}{hard trekt zoudt gij haar zoudt gij}{Nora doen schrikken}\\

\haiku{Och ja,{\textquoteright} hervatte,;}{Van Leefdael vroolijk ik kwam}{juist niet erg van pas}\\

\haiku{Het {\textquoteleft}neen{\textquoteright} van Mijnheer,,;}{Albert was naar zijn inzien}{onvermijdelijk}\\

\haiku{Zij beminde dien.}{jongen man uit al de kracht}{harer reine ziel}\\

\haiku{{\textquoteright} Mijnheer Albert wierp;}{een blik van meewarigheid}{op den ouden heer}\\

\haiku{Uwe rechterhand, daar,.....}{in uwen jas gestoken rust}{op ets zeer kostbaar}\\

\haiku{doch de honderd pond.}{intrest van het loopende}{jaar waren verteerd}\\

\haiku{{\textquoteleft}Ik zal toch overal.}{zeggen dat ik het kind van}{den poppenman ben}\\

\haiku{Mijnheer Van Velthem.}{was misschien nog meer ontroerd}{dan de poppenman}\\

\haiku{ik schik mij in mijn....}{lot en de goede God zal}{het overige doen}\\

\haiku{{\textquoteleft}En ik, mijnheer Van,.}{Velthem ik geef haar u. Wees}{beiden gelukkig}\\

\haiku{De oude heer stond,.}{op hij had groote moeite om}{zich goed te houden}\\

\subsection{Uit: Werken. Deel 22. Zoo werd hij rijk}

\haiku{op de helling des;}{heuvels krast de zaag door den}{ouden eikenstam}\\

\haiku{ofwel hij maait er,;}{het graan  zaait de voor-}{of najaarsvruchten}\\

\haiku{Als Hiob terugkeert,;}{zal hij wel is waar Nard en}{Narda niet vinden}\\

\haiku{Nu was de Nachtuil;}{in eenige dagen niet in}{het dal gekomen}\\

\haiku{Zoo gewapend - met, -.}{stok pistolen en mes klom Hiob}{langzaam naar boven}\\

\haiku{Toen de blauwe damp,;}{optrok was gansch de bende}{wolven verdwenen}\\

\haiku{nogmaals klopte hij.}{en legde daarna het oor}{tegen het houtwerk}\\

\haiku{- als Jean Hibou op,.}{strooptocht ging verliet hij het}{huis niet langs de deur}\\

\haiku{Het kind knikte met.}{het oog en frazelde iets}{dat niet verstaan werd}\\

\haiku{{\textquoteleft}En zijn er meer zoon?}{wagens en tenten op eene}{kermis in de stad}\\

\haiku{Bij Narda, wekte;}{dit gezicht niet de minste}{bewondering op}\\

\haiku{De mulder begreep;}{nu alles wat er op de}{kermis gebeurd was}\\

\haiku{doch ik wilde u.}{echter opmerken dat het}{vandaag Zondag is}\\

\haiku{de jongen vindt het,.}{zoo wonderlijk in dat kleed}{ter kerke te gaan}\\

\haiku{{\textquoteleft}Neen, manneke, ge!}{zult niet weten met wat troef}{ik u nog afwacht}\\

\haiku{{\textquoteright} {\textquoteleft}Maar ik die hem ken,.}{ik heb het recht dat oordeel}{over hem te vellen}\\

\haiku{{\textquoteleft}Indien hij niet over,?}{mij sprak dan toch gewis sprak}{hij over de abdij}\\

\haiku{Maar dan zal het de,?}{booze zijn die zijnen staart of}{de wagens voortsleept}\\

\haiku{{\textquoteright} {\textquoteleft}Toch zou ik al dat,.}{schoons al dat zilver en goud}{willen zien blinken}\\

\haiku{{\textquoteright} {\textquoteleft}Kom, Nard, geef mij uwe...,{\textquoteright}.}{hand terug en hand in hand}{gaan beiden weer voort}\\

\haiku{{\textquoteright} vraagt Yolande op,.}{eenen toon die als zilverklank}{in de ooren galmt}\\

\haiku{{\textquoteright} Die naam verwekt eene.}{zekere verwondering}{bij den ouden man}\\

\haiku{indien zijn zoon eens? '.}{d\`a\`ar waret Zou niet te}{verwonderen zijn}\\

\haiku{Razend van gramschap;}{ijlt mijnheer Delmon de la}{Carde de straat op}\\

\haiku{Zij was zooeven d\`a\`ar,,.}{aan het hekken en moet hier}{zijn binnengevlucht}\\

\haiku{Toen ik kind was had....}{de kerk voor mij inderdaad}{aantrekkelijkheid}\\

\haiku{het gevalt mij niet,,!}{dat gij den titel dien ik}{draag niet eerbiedigt}\\

\haiku{Dat vertraagt den tocht,.}{bij het klimmen dat verhaast}{hem bij het dalen}\\

\haiku{Dat belooft de zoon;}{van den Bloedberg en hij geeft}{daarop den handslag}\\

\haiku{{\textquoteright} Me dunkt dat ik een,;}{bro\^er heb gehad doch ik ben}{er niet zeker van}\\

\haiku{{\textquoteleft}Narda,{\textquoteright} hervat Max, {\textquoteleft}?}{een kind met donkere oogen}{en zwarte haren}\\

\haiku{In den hoek der tent}{lagen de poppen en toen}{het middernacht sloeg}\\

\haiku{doch de slaap verlamt.}{zijne grijpende vingers}{en hij ronkt weer voort}\\

\haiku{Max richt zich op, gaat.}{tot bij den slaper en tikt}{hem op den schouder}\\

\haiku{geruiten zakdoek,;}{om den hals geknoopt zoodat de}{tip op den rug hangt}\\

\haiku{Weet ge wel dat het...{\textquoteright}.}{strafbaar is en de spreker}{houdt plotseling op}\\

\haiku{Gij kunt,{\textquoteright} zegt hij, {\textquoteleft}gij,.}{kunt op den molen blijven}{wonen zooals voorheen}\\

\haiku{maar somtijds drijft die.}{bloedgierige valk van het}{kasteel boven ons}\\

\haiku{{\textquoteleft}Denkt ge, Narda, dat?}{wij daar niet samen zullen}{wonen gelijk hier}\\

\haiku{Ik zal u volgen,,.}{doch gij zult mij toelaten}{met een bepaald doel}\\

\haiku{Kalm is het buiten, '.}{storm is het int gemoed}{van den kasteelheer}\\

\haiku{Gisteren was het,,;}{de zoon toen was het Hiob nu}{is het de dochter}\\

\haiku{hij ziet zijnen heer:}{verwonderd aan en zegt min}{of meer verlegen}\\

\haiku{Na een oogenblik.}{peinzens keert hij terug en}{gaat in den bloemhof}\\

\haiku{{\textquoteleft}Ja, laat ons in de,{\textquoteright};}{taal van onzen kindertijd}{spreken hervat Hiob}\\

\haiku{Ook die menschen meent,:}{zij ooit gezien te hebben}{zoowel als dien wagen}\\

\haiku{{\textquoteright} {\textquoteleft}In hoeveel tijds zou?}{men wel te voet van hier naar}{Brussel kunnen gaan}\\

\haiku{Neen, het stadsleven,.}{bevalt hem niet zelfs niet nu}{hij weer wakker wordt}\\

\haiku{Hij zou genegen....}{zijn een huwelijk met de}{gravin aan te gaan}\\

\haiku{de lange, korte,;}{dikke en spitze glazen}{zijn le\^eg of half le\^eg}\\

\haiku{de... zit in alle;}{geval voor de gemaakte}{brokken en stukken}\\

\haiku{doch hij heeft reeds eene....}{uitzondering voor mijnheer}{Doblain de gemaakt}\\

\haiku{Wie wordt hier, in dit,?}{heiligdom door den jongen}{edelman afgewacht}\\

\haiku{{\textquoteleft}Ik kan, ik mag uw,.}{edelmoedig aanbod niet meer}{aannemen Maurits}\\

\haiku{Daar in dat kerksken,!}{is immers voor haar het nieuw}{leven begonnen}\\

\section{Jan Renier Snieders}

\subsection{Uit: De hut van Wartje Nulph}

\haiku{maar om het even, wilt?}{gij mij nu eens zeggen waar}{ik hier verzeild ben}\\

\haiku{nu ik er aan denk,,;}{ben ik blijde dat gij mij}{wakker hebt gemaakt}\\

\haiku{- Uw paard zal u dien,.}{zeggen zoodra wij in}{Turnhout aankomen}\\

\haiku{Uw reisgenoot kan,.}{gezond zijn maar hij ziet er}{mager en spits uit}\\

\haiku{- Nulph, gij kunt nog eens,.}{een groot man worden zei de}{reiziger spottend}\\

\haiku{het dier brieschte.}{en kwam met woeste sprongen}{op het tooneel gehold}\\

\haiku{Toen droomde Wartje,}{van den muilezel en deszelfs}{eigenaar van wien}\\

\haiku{Nulph gevoelde iets,;}{vreemds iets geheimzinnigs in}{zijne ziel omgaan}\\

\haiku{- Otto Richardi,;}{mompelde een der ruiters}{in het Italiaansch}\\

\haiku{onthoud dit, indien.}{gij geen kennis wilt maken}{met zijn lang rapier}\\

\haiku{het  was alsof:}{het dagend morgenlicht hem}{had toegeroepen}\\

\haiku{Zeg haar ook, dat zij}{niet vergeet te bidden voor}{het kruisbeeld voor zij}\\

\haiku{en geknield bad hij,.}{voor de kleine Bertha om brood}{voor zijne ouders}\\

\haiku{Nulph hief nog even het,.}{hoofd op en poogde door het}{venster te staren}\\

\haiku{Terwijl hij het geld}{op de tafel neertelde}{verzekerde Nulph}\\

\haiku{- Dat kind, ging Nulph voort,,...}{is verdwenen geroofd door}{een uwer edellieden}\\

\haiku{Er tintelde een.}{zonderlinge gloed in de}{oogen van den grijsaard}\\

\haiku{Er glinsterde een '.}{straal van hoop opt gelaat}{van den koolbrander}\\

\haiku{- Maar ik versta niet?}{wat hij met een geroofd kind}{zoude aanvangen}\\

\haiku{gij hebt u verzet;}{tegen de soldaten van}{Zijne Majesteit}\\

\haiku{De Pluimgraaf fronste:}{nog sterker de wenkbrauwen}{en hernam driftig}\\

\haiku{Indien hij weigert,,,.}{ziedaar neem dan mijn rapier}{en klopt het er uit}\\

\haiku{In eens zweeft er iets;}{donkers in de lucht dat steil}{naar beneden valt}\\

\haiku{- Zoo hard ik maar kan,,.}{door de velden de bosschen}{dwars over de heide}\\

\haiku{- Maar hoe komen wij,;}{aan den overkant der vest wierp}{baas Canutus op}\\

\haiku{- Bedenk, Charles, zei,;}{de prins die den grond zijner}{gedachten raadde}\\

\haiku{laat mij eerst alles,.}{eens goed afzien of de zaak}{ook wel juist uitkomt}\\

\haiku{- vroeg de Ritmeester,.}{Bacx met een stem bevend van}{verontwaardiging}\\

\haiku{doch deze, zijn hoofd,.}{terugtrekkend maakte de}{spleet nogal kleiner}\\

\haiku{ik zie, dat gij de...,;}{kleine Bertha lief hebt Welaan}{zij blijve bij u}\\

\haiku{Baas Canutus, sprak,.}{hij hard genoeg om maar even}{gehoord te worden}\\

\haiku{en hij wees op twee.}{gewapende mannen in}{den hoek der kamer}\\

\haiku{vergun mij verhaal, -...}{en wees minder gelukkig}{dan naar gewoonte}\\

\haiku{zelfs beweert men, dat '.}{t geheele dorp Ravels}{wemelt van krijgsvolk}\\

\haiku{- Geen genade voor,,;}{die benden morde Varax}{in  het heengaan}\\

\haiku{dat heet bang zijn, zoo,}{te gaan loopen wanneer de}{vijand nog een uur}\\

\haiku{- Ziet gij, Kelker, zooals,;}{ik u gezegd heb ging de}{Zegewoude voort}\\

\haiku{Het was alsof dat;}{kleine voordeel den moed der}{Staatschen bezielde}\\

\haiku{doch Nulph worstelde;}{uit al zijne macht tegen}{de twee soldaten}\\

\haiku{Intusschen was reeds;}{het gevecht tusschen de twee}{benden ge\"eindigd}\\

\haiku{De heide rookte,.}{van bloed en was bedekt met}{dooden en gekwetsten}\\

\haiku{- Heer Graaf... stotterde,,.}{Nulph die niet meer wist wat hij}{hoorde zag of deed}\\

\haiku{Eindelijk, door een,;}{wonder des Hemels heelde}{de wonde der vrouw}\\

\haiku{- Toen spreidde er zich;}{een akelige verf op haar}{vermagerd wezen}\\

\haiku{- Wat mocht den armen,!}{koolbrander zoo ontstellen}{toen hij aanklopte}\\

\subsection{Uit: Het kraaien-nest}

\haiku{Is er in de  , '?}{wereld nog een schooner stee dan}{t Kraaien-nest}\\

\haiku{het lied te hooren,.}{dat de meisjes gedurig}{opnieuw aanhieven}\\

\haiku{- Dat zal daar niet bij,,,.}{blijven riep Lootman de vuist}{dreigend opstekend}\\

\haiku{- Let op uw woorden,,.}{riep Carrero den vinger}{dreigend opstekend}\\

\haiku{neen, al moest er mijn,.}{hoofd af toch laat ik mij niet}{langer dwarsboomen}\\

\haiku{gedurig had hij,.}{de flesch in de hand en het}{glas aan de lippen}\\

\haiku{met mijne handen.}{laat ik geen kastani\"en}{uit het vuur krabben}\\

\haiku{Geurik ging met eens,.}{zoo vluggen stap naar huis als}{hij gekomen was}\\

\haiku{Juist toen hij zijn hand '... (}{uitstak en dreigend naart}{Kraaien-nest wees}\\

\haiku{De ooievaars zijn,,;}{evenals de weelderige}{boschzangers verdwenen}\\

\haiku{- Het spijt mij, niet meer,;}{dan een bed in huis  te}{hebben zei Minten}\\

\haiku{- Moedertje, vroeg hij,?}{beleefd kan ik mijn sigaar}{hier eens aansteken}\\

\haiku{maar neen, de naam van.}{dien rechtsgeleerde boezemt}{mij vertrouwen in}\\

\haiku{Wie is nu de man,?}{die zoo veel belang stelt in}{den armen Minten}\\

\haiku{Nu moet ick u ghaen, '.}{verlaten En kiesennen}{anderen stal}\\

\subsection{Uit: De lelie van 't gehucht}

\haiku{Ik ken er wel meer,,.}{die een grooten naam hebben}{hernam de eerste}\\

\haiku{hij had zwart kroeshaar,,;}{zwarte brandende oogen en}{een pikzwarten baard}\\

\haiku{- Neen, Huibert speelt niet,.}{slecht onderbrak David op}{verzoenenden toon}\\

\haiku{- De dagen zijn kort,.}{heeren antwoordde de meid}{heimelijk lachend}\\

\haiku{Waren dan de twee?}{jongelieden haar beiden}{even onverschillig}\\

\haiku{men laadde 't op,.}{karren of men droeg den zak}{op het hoofd naar huis}\\

\haiku{maar indien de wind,.}{niet wat meer opsteekt krijgen}{wij ons werk niet af}\\

\haiku{- Krampe, het is uw,,;}{graan dat wij gaan inschudden}{sprak de molenaar}\\

\haiku{ik gaf er mijn pink,,;}{van indien ik een zoon had}{zooals Ari\"e van de Schans}\\

\haiku{hij klopte zijn zoon,:}{met zelfvoldoening op den}{schouder en zeide}\\

\haiku{- Vader Krampe, gij,;}{hebt mij boer gemaakt lachte}{Mathias dankbaar}\\

\haiku{wanneer gij trouwt krijgt.}{gij van den ouder Krampe}{een kostbaar bruidstuk}\\

\haiku{- Wat zijt gij vandaag,,.}{netjes met uw sitsen kleed}{lachte de vader}\\

\haiku{men at manden vol...!}{peperkoek en maar heden}{gaat dat heel anders}\\

\haiku{Het meisje voelde,.}{den steek en bleef zwijgend door}{het venster staren}\\

\haiku{ik zal u er maar,;}{door helpen dewijl ik wel}{weet waar gij heen wilt}\\

\haiku{- Wat zijn de vrouwen?}{toch veranderd sedert den}{tijd dat ik jong was}\\

\haiku{- Ik dans niet, vader,,.}{herhaald het meisje met de}{grootste beleefdheid}\\

\haiku{Tegen het vallen}{van den avond stak de speelman}{den strijkstok tusschen}\\

\haiku{- Ik zoo min als een,.}{ander liet Ari\"e van de Schans}{er op volgen}\\

\haiku{kan ik het helpen...?}{dat gij vandaag een blauwe}{scheen hebt geloopen}\\

\haiku{maar neen, ik wil u...,.}{een kans geven ziedaar hebt}{gij uw mes terug}\\

\haiku{- Daar is mijne hand,;}{zei de hoevenaar op een}{gulhartigen toon}\\

\haiku{- Neen, morde Krampe,;}{de wereld staat mij dezen}{avond tegen mijn dank}\\

\haiku{of ben ik niet oud?}{en wijs genoeg om alleen}{naar de Schans te gaan}\\

\haiku{- Een Brabander, die,.}{zooals men zegt wel iets van een}{duivel moet hebben}\\

\haiku{riep een ambtenaar,.}{terwijl hij achter den hoek}{van de schuur wegsloop}\\

\haiku{Zie dat gaat boven,.}{mijn verstand dat niemand van}{ons u herkend heeft}\\

\haiku{om u hiervan te.}{verzekeren kwam ik u}{uit uw bed kloppen}\\

\haiku{- Indien ik nog een,.}{jaar lang kan smokkelen loop}{ik op mijn muiltjes}\\

\haiku{In mijnen tijd... - Laat,;}{de boerderij naar de galg}{loopen lachte Ari\"e}\\

\haiku{- Ach, vader, ik bid,,;}{u word toch geen smokkelaar}{smeekte het meisje}\\

\haiku{vervolgde Dwina,.}{het aangezicht achter haar}{voorschoot geborgen}\\

\haiku{- Ik dacht aan mijne,,.}{moeder Huibert antwoordde}{het meisje treurig}\\

\haiku{- Dwina, beken mij,;}{openhartig waarom gij er}{zoo treurig uitziet}\\

\haiku{Maar zeg, en misleid,?}{mij niet gelooft gij dan niet}{aan het voorgevoel}\\

\haiku{Dien avond, zit Dwina,;}{zonder het rad te draaien}{voor haar spinnewiel}\\

\haiku{Laat anderen maar.}{boeren en zich voor Koning}{Willem lam werken}\\

\haiku{- Gij ziet wel, hoe al,;}{die gekke visioenen}{uitkomen sprak hij}\\

\haiku{gij hebt mij woord, en,}{trouwt met mijn dochter al wilt}{gij morgen dien dag}\\

\haiku{baas Urkhoven zal...?}{die bloem wel openmaken ziet}{gij uw kaarten nog}\\

\haiku{Nooit had men in de;}{dorpen zooveel gesproken}{van den sluikhandel}\\

\haiku{- Vraag mij duizendmaal,, '.}{meer was het antwoord ent}{is u geschonken}\\

\haiku{Het meisje gaf een.}{akeligen gil en sloop door}{het kreupelhout heen}\\

\haiku{kom, neem gij hem bij,.}{de voeten ik leg hem het}{hoofd op mijn schouder}\\

\haiku{Met een teeken van,.}{de hand belette hem de}{zieke voort te gaan}\\

\haiku{- Waarom?... herhaalde;}{de koewachter met een slecht}{bedekten spotlach}\\

\haiku{- Ik wil en zal mijn,!}{geld en ook den verloopen}{interest hebben}\\

\haiku{Waarom kwam hij dien?}{dag bij zijn ouden vriend een}{bezoek afleggen}\\

\haiku{Neen, David, neem mij,.}{niet kwalijk maar uw dochter}{wil ik aan geen prijs}\\

\haiku{Van de Hees tot aan,!}{de poort der hel ken ik geen}{grooter schobbejak}\\

\subsection{Uit: Narda}

\haiku{Dat is het denkbeeld,;}{hetwelk zij nimmer uit haar}{hoofd kan wegdrijven}\\

\haiku{hoe klein di\`e winst ook,...}{zij zou ik toch die stuivers}{niet kunnen missen}\\

\haiku{De lijder kan elk.}{oogenblik ontwaken en}{haar zorg noodig hebben}\\

\haiku{- Nooit had de arme.}{weduwe van dit alles}{een woord geweten}\\

\haiku{er is daar bij die.}{arme menschen aan alles}{volslagen gebrek}\\

\haiku{sprak Narda de beurs,.}{toedraaiend en ging met haar}{vader de deur uit}\\

\haiku{vader, onderbrak,,...}{het meisje dat ware niet}{kiesch niet edelmoedig}\\

\haiku{kuste haar op het,.}{voorhoofd en haalde nog een}{goudstuk uit de beurs}\\

\haiku{Op de tafel stond,,:}{een groote glazen kom waarin}{de nommers lagen}\\

\haiku{hij verschrikte toen.}{hij een zware stem zijn naam}{hoorde oproepen}\\

\haiku{voor het tweede is.}{de politiekamer een}{overheerlijk middel}\\

\haiku{Hild was misschien de,;}{eenige welke geen sterken}{drank had gedronken}\\

\haiku{gij  ziet wel dat,,.}{al ben ik maar een boer ik}{toch wel den weg weet}\\

\haiku{Indien hij mijn zoon,.}{ware deed ik hem rijden}{op een ganzenpen}\\

\haiku{- Tweemaal verscheen de,;}{weduwe op het raadhuis}{luidde het antwoord}\\

\haiku{- Indien de heeren?}{mij van die vernedering}{willen vrijlaten}\\

\haiku{dacht Hild, terwijl hij.}{daar sprakeloos voor de twee}{geneesheeren stond}\\

\haiku{noch zij noch haar zoon.}{hadden de opgediende}{spijzen aangeroerd}\\

\haiku{hoe en wanneer zal,,?}{ik die geen stuiver bezit}{u dat weergeven}\\

\haiku{ik zal u aanstonds,.}{wijd en breed uitleggen hoe}{ik ben gevaren}\\

\haiku{- En wat de menschen,,.}{aangaat niemand raakt het wat}{ik met mijn geld doe}\\

\haiku{het vertrek van den.}{toekomenden dokter was}{bepaald vastgesteld}\\

\haiku{Zoo redeneerde,.}{Brinkpoel wanneer men sprak van}{Oscar Veldenus}\\

\haiku{Hild schokschouderde,.}{even en bleef verlegen naar}{het vloerzand staren}\\

\haiku{- Een glimlach van Hild.}{bewees dat de heer ditmaal}{juist had geraden}\\

\haiku{doch het oogenblik.}{was al te weinig geschikt}{tot gekscheerderij}\\

\haiku{In een der schoonste.}{straten der stad trof men een}{likeurwinkel aan}\\

\haiku{Oscar vroeg bij zich,;}{zelven of hij zich niet in}{een hotel bevond}\\

\haiku{die heeren zitten!}{hier evenals de internen}{in een muizenval}\\

\haiku{genot, Maakt mij al.}{de zegeningen Nogmaals}{weer mijn zalig lot}\\

\haiku{mij dunkt dat ik mijn!}{tegenpartij met mijn vuist}{dwars door zijn maag stiet}\\

\haiku{Neen, mijnheer, zulke,}{kinderen stoot de Alma}{Mater op wie gij}\\

\haiku{de President van.}{het Collegium ligt reeds}{in de diepste rust}\\

\haiku{Alles is doodstil,.}{in het Collegium het}{is bij middernacht}\\

\haiku{en gespaard voor zijn,.}{moeder heeft hij gespaard dat}{hij zelf gebrek lijdt}\\

\haiku{hoeveel koolzaad men,.}{dat jaar had gedorscht en}{hoe het er uitzag}\\

\haiku{neen, ik heb er twee, ',.}{ent zijn goede ooren}{zelfs van de beste}\\

\haiku{zou zijn dochter niet '?}{de rijkste boerendochter}{uitt dorp wezen}\\

\haiku{maar toen ik daareven {\textquoteleft}!}{den burgemeester met zijn}{fleemenddag Nardje}\\

\haiku{ik weet wel dat het,;}{waar is dat hij somtijds met}{mij uit de kerk gaat}\\

\haiku{De hoovaardige,,;}{mensch waar hij gaat of staat werpt}{ook alles overhoop}\\

\haiku{mijn grootste geluk.}{is u en uw kinderen}{gelukkig te zien}\\

\haiku{Het woord was niet uit:}{zijn mond of de wildkooper}{had reeds geroepen}\\

\haiku{vroeg de notaris,.}{terwijl de klerk wederom}{een pitje aanstak}\\

\haiku{blijf in huis, moeder,?}{Horbaak er is toch wel plaats}{voor twee huishoudens}\\

\haiku{sedert eenigen tijd,}{was zij verhuisd en woonde}{eenige minuten}\\

\haiku{het deed haar zoo goed,.}{dat hij haar in haar nieuwe}{woning was gevolgd}\\

\haiku{In de Kempen, in,?}{de Meierij in een of}{ander heidorp}\\

\haiku{en de meisjes en,:}{de straatjongens schaterden}{het uit en riepen}\\

\haiku{En dien zondag was, '!}{het zoo aangenaam zoo frisch}{buiten int veld}\\

\haiku{tranen zoudt gij er,.}{mee schreien verzekerde}{de burgemeester}\\

\haiku{waarom gaat u dat?}{altijd het een oor in en}{het andere uit}\\

\haiku{anderen gekwetst,.}{om de spottende woorden}{trokken den neus op}\\

\haiku{- Zoo dom toch, als de,,.}{ekster dat denkt zijn wij juist}{niet zei een ander}\\

\haiku{- Was die kegelbol,,}{zoo dachten anderen ook}{misschien de vinger}\\

\haiku{De juryleden,;}{zien het gevaar waarin hun}{ambtgenoot verkeert}\\

\haiku{- Och, dat is waar ook,,;}{hernam Veldenus de hand}{opnieuw uitstekend}\\

\haiku{het hoofd wordt er lam,,,;}{van en armen beenen maag en}{hart alles verlamd}\\

\haiku{Hild, gij die altijd,?}{braaf en spaarzaam leeft hebt toch}{wel iets in de beurs}\\

\haiku{- Ik wist niet dat gij?}{in betrekking stond met de}{dochter van Brinkpoel}\\

\haiku{Doch later zal dat;}{geheim geheel ontsluierd}{voor den dag komen}\\

\haiku{zij zag bleek als een,:}{doode en stamelde met}{neergeslagen oogen}\\

\section{Rosalie Sprooten}

\subsection{Uit: Muren van glas}

\haiku{{\textquoteright} Door het raam keek ik.}{hem na terwijl zijn geur nog}{in de ruimte hing}\\

\haiku{{\textquoteleft}Als er vandaag weer.}{geen antwoord op mijn brief komt}{dan vermoord ik hem}\\

\haiku{Ze pakte de draad.}{tussen twee vingers en liep}{ermee naar het raam}\\

\haiku{Voordat Louise de,.}{dop op de stift draaide rook}{ze er altijd aan}\\

\haiku{Ze zou hem bellen.}{en vragen of hij nog aan}{haar brief had gedacht}\\

\haiku{{\textquoteleft}Ik zal ze daarginds{\textquoteright},.}{eens mores leren schreeuwde}{ze door de kamer}\\

\haiku{Ik schatte hem van.}{mijn leeftijd en bekeek hem}{wat aandachtiger}\\

\haiku{{\textquoteright} {\textquoteleft}Ja natuurlijk, ik.}{heb pati\"enten van hem}{in mijn programma}\\

\haiku{Hanekamp werkt heel,.}{hard hij is ook aardig voor}{de pati\"enten}\\

\haiku{De teleurstelling.}{aan de andere kant van}{de lijn was voelbaar}\\

\haiku{mooi weer vandaag, ik,.}{ga de zon schilderen eerst}{een sigaretje}\\

\haiku{De goudkleurige.}{oorbellen hingen bijna}{tot op haar schouders}\\

\haiku{{\textquoteright} {\textquoteleft}Er waren te veel.}{mensen die dachten dat hij}{het goed bedoelde}\\

\haiku{{\textquoteright} {\textquoteleft}Als jij en ik een.}{hemel hebben dan hebben}{de joden die ook}\\

\haiku{De werkzaamheden.}{in overeenstemming met het}{plaatje op de deur}\\

\haiku{De chauffeur trok zo.}{snel op dat ze tegen de}{leuning werd gedrukt}\\

\haiku{Als ze haar ogen dicht,.}{deed kon ze nog zijn arm om}{haar schouders voelen}\\

\haiku{De helft van de tijd.}{vulde ik met zinloze}{voorbereidingen}\\

\haiku{Waarom wil hij niet?}{van mijzelf horen hoe de}{vork in de steel zit}\\

\haiku{In de zes jaar dat.}{ik mijn werk deed had ik dit}{nooit eerder gehoord}\\

\haiku{Met beide handen.}{pakte ze hem vast en wist}{even niets te zeggen}\\

\haiku{Er was een moment.}{geweest dat er uitzicht leek}{op verandering}\\

\haiku{Als je buiten de,.}{kliniek staat kijk je er heel}{anders tegenaan}\\

\haiku{Ik voelde me een.}{kleine zelfstandige met}{te weinig klanten}\\

\haiku{Zijn dagtaak bestond.}{grotendeels uit bijwonen}{van besprekingen}\\

\haiku{Hier, bij de vakbond,,.}{in de politiek dat was}{overal hetzelfde}\\

\haiku{Hij keek even de kring.}{rond en richtte zich tot een}{van de verplegers}\\

\haiku{Anderen waren.}{voor korte of langere}{tijd opgenomen}\\

\haiku{Toen ik afscheid van.}{Lucas had genomen ging}{ik naar mijn atelier}\\

\haiku{Louise ging in de.}{achterste bank zitten en}{volgde de werkster}\\

\haiku{Achter de pilaar,.}{waar Louise naast zat waren}{stemmen te horen}\\

\haiku{{\textquoteleft}Die blaast ook hoog van{\textquoteright},.}{de toren zei Joris}{op maandagmorgen}\\

\haiku{Ik ben het met hem.}{eens maar die neerbuigende}{toon bevalt me niet}\\

\haiku{Vreemd genoeg was ik '.}{s morgens toch blij dat ik}{weer naar mijn werk kon}\\

\haiku{{\textquoteleft}Deze kliniek is{\textquoteright}}{als een kerstboom met te veel}{ballen in de top.}\\

\haiku{Ik moest er zo hard.}{van boeren dat Frederik}{er wakker van schrok}\\

\haiku{Was het gierigheid,?}{van de boeren gebrek aan}{hart voor hun dieren}\\

\haiku{Een veertiger, goed.}{van de tongriem gesneden}{en zelfverzekerd}\\

\haiku{Had ze misschien voor?}{deze personeelschef ook}{nog iets in petto}\\

\haiku{Met enige moeite.}{wist ze aan de overkant van}{de straat te komen}\\

\haiku{Ik vertelde over.}{de tuin en het schilderij}{waar ik aan werkte}\\

\haiku{Ze raakte hem even.}{aan om hem als het ware}{in gang te zetten}\\

\haiku{{\textquoteleft}Legendarische,.}{man helemaal uit Frankrijk}{hierheen gekomen}\\

\haiku{Op een betonnen.}{verhoging liep een van de}{beren heen en weer}\\

\haiku{Er is niets vreemds aan,,.}{mij dacht ze ik zit hier heel}{gewoon met iemand}\\

\haiku{{\textquoteleft}Ik moet nu even naar}{een afspraak maar als je straks}{tijd hebt wil ik graag}\\

\haiku{Die goeie geluiden.}{kunnen toch niet van onze}{afdeling komen}\\

\haiku{In de verte lag.}{het heuvelland alsof er}{niets aan de hand was}\\

\haiku{{\textquoteright} Louise voelde de.}{stroefheid nu al en zag er}{een slecht teken in}\\

\haiku{Nam de hoorn op maar.}{besloot ogenblikkelijk het}{toch maar niet te doen}\\

\haiku{Ze parkeerde haar.}{auto z\'o dat ze zicht had}{op de hoofdingang}\\

\haiku{De maag moet tot rust{\textquoteright},.}{komen zei hij beslist en}{schreef een dieet voor}\\

\haiku{Ik was zo intens.}{moe dat ademhalen nog te}{veel energie kostte}\\

\haiku{Ik heb zowat vier.}{dagen in een tentje aan}{het strand gezeten}\\

\haiku{Waarom had ik niet?}{eerder gezien wat er met}{hem aan de hand was}\\

\haiku{De dag kwam dat ik.}{achter Frederik aan de}{tuin in wandelde}\\

\haiku{Hij was heel bang dat.}{ze hem weer terug zouden}{sturen naar zijn werk}\\

\haiku{{\textquoteright} {\textquoteleft}Ja, ja, zo is het,{\textquoteright}, {\textquoteleft}.}{zei Severijnsbij het GAK}{moet je oppassen}\\

\haiku{Nou nee, dat niet, maar?}{jij hebt toch jaren in het}{gekkenhuis gewerkt}\\

\haiku{Al vijf jaar probeer.}{ik in gesprek te komen}{met de directie}\\

\haiku{Fernand knipperde.}{met zijn ogen en vouwde zijn}{handen op de rug}\\

\haiku{Joris en Felix.}{zouden samen bezig zijn}{met speltherapie}\\

\haiku{In de schemering.}{van het trapportaal rees een}{schaduw voor me op}\\

\haiku{Af en toe pakte.}{ze zijn hand om hem door de}{drukte te loodsen}\\

\haiku{Ze schrok z\'o dat ze.}{al haar pori\"en even open}{en dicht voelde gaan}\\

\haiku{De muzikanten:}{draaiden om Fernand en haar}{heen en zongen}\\

\haiku{Waarom kon ik die?}{wurgende molen in mijn}{hoofd niet stil zetten}\\

\haiku{Felix stond plotseling.}{naast me. De afstand tussen}{ons was al voelbaar}\\

\haiku{{\textquoteleft}Je praat weer als een{\textquoteright},.}{therapeute had hij ooit}{tegen me gezegd}\\

\haiku{Bij de trappen stond.}{het vijftal alsof ze nu}{al van brons waren}\\

\haiku{De obers liepen zelfs.}{op straat met dienbladen vol}{gevulde glazen}\\

\haiku{De clown blies dapper.}{zijn partij mee maar erg van}{harte ging dat niet}\\

\haiku{Samen gleden ze.}{op de grond waar hun hoofden}{elkaar even raakten}\\

\haiku{{\textquoteleft}Ik had bijna, let,,.}{wel bijna een moord gepleegd}{tijdens de optocht}\\

\haiku{Maar dit is wel erg,?}{anekdotisch en buiten de}{orde vind je niet}\\

\haiku{Dat weet ik ook niet,.}{maar wat plotseling opwelt}{heeft betekenis}\\

\haiku{Periodiek werd.}{me namens de kliniek weer}{een bloemstuk gebracht}\\

\haiku{De tekst kende ik {\textquoteleft}{\textquoteright}.}{al.Van harte beterschap}{stond er altijd op}\\

\haiku{Haastig zette ze.}{haar fiets in het schuurtje en}{vluchtte het huis in}\\

\haiku{Ze opende af en.}{toe haar ogen maar ze vielen}{als vanzelf weer dicht}\\

\haiku{{\textquoteright} {\textquoteleft}Nee, hij is over een.}{plantenbak gestruikeld toen}{hij wilde blussen}\\

\haiku{Fernand stond niet meer.}{op de plek waar ze hem voor}{het laatst had gezien}\\

\subsection{Uit: ...De pest voor een schip}

\haiku{Ik stelde mij voor.}{hoe het zou zijn om in zo'n}{jurk te dansen}\\

\haiku{Daar bovenop werd.}{de blauwe deksel van de}{braadketel gelegd}\\

\haiku{Toen pas zag ik dat.}{er echte ogen in zaten}{die mij aankeken}\\

\haiku{Hij zweeg even om dat.}{toekomstbeeld goed tot mij door}{te laten dringen}\\

\haiku{Enkele dagen.}{later was de strijd rond mijn}{huwelijk beslecht}\\

\haiku{{\textquoteleft}Als we kinderen,,.}{krijgen laat ik ze dopen}{dat beloof ik je}\\

\haiku{Het zal de mooiste,.}{dag uit ons leven worden}{dat beloof ik je}\\

\haiku{Ik kon niet anders.}{dan hard in de riem van mijn}{schoudertas knijpen}\\

\haiku{Ik kan me niet meer.}{herinneren of hij ons}{gelukgewenst heeft}\\

\haiku{Dat moet je zien, hoe.}{ze vastberaden op hun}{doel af stevenen}\\

\haiku{Ik krabde tussen.}{haar horens en wreef over haar}{harige wangen}\\

\haiku{{\textquoteright} {\textquoteleft}Maar jullie moeten,,.}{toch iets eten Max wie weet hoe}{laat je weer iets krijgt}\\

\haiku{Ik vond het een te.}{alledaags begin voor een}{reis naar Amerika}\\

\haiku{Onze voetstappen.}{klonken hol toen we over het}{eerste dek liepen}\\

\haiku{In het lamplicht viel.}{mij de versleten groene}{kleur van de vloer op}\\

\haiku{Waarmee ik maar wil.}{zeggen dat je voorlopig}{veel alleen zult zijn}\\

\haiku{Er was veel blikgoed,,,.}{met sardientjes zalm tonijn}{haring en makreel}\\

\haiku{{\textquoteleft}Denk je er wel aan.}{dat over een half uur het eten}{op tafel moet staan}\\

\haiku{Hij sprak met vrijwel.}{niemand en liep meestal}{met gebogen hoofd}\\

\haiku{Even later was ik.}{alleen met de stuurman die}{Lars Wallin heette}\\

\haiku{Table, la vache,,,,.}{escalier fourchette mon}{fr\`ere mes fr\`eres}\\

\haiku{Zijn mooie tenorstem.}{galmde dan door de stal of}{in de weilanden}\\

\haiku{Met de hand voor mijn.}{mond spoedde ik mij naar het}{toilet en gaf over}\\

\haiku{Enkele malen.}{zorgde Max ervoor dat de}{tafel werd gedekt}\\

\haiku{De zwaar bedekte.}{lucht maakte de sfeer in de}{hut onheilspellend}\\

\haiku{Ik had er graag met.}{Wallin over gesproken maar}{het kwam er niet van}\\

\haiku{De taal leek in haar.}{uitspraak verrassend veel op}{mijn Limburgs dialect}\\

\haiku{Onder het kussen.}{had ik  een foto van}{mijn ouders gelegd}\\

\haiku{Natuurlijk zouden.}{we Chicago halen en}{verder ook nog wel}\\

\haiku{Een besmetting van,.}{tien jaar wilde vaart zoals}{hij zelf verklaarde}\\

\haiku{Een lichte schok was.}{voelbaar toen het schip even de}{kademuur raakte}\\

\haiku{Ze wilde heus wel.}{toegeven dat er ook wel}{armoede was hier}\\

\haiku{Achter ons reden.}{de Herings met twee Noren}{op de achterbank}\\

\haiku{Ik had geen enkel.}{verlangen meer om er ooit}{nog terug te gaan}\\

\haiku{Ik genoot ervan.}{als alles in de mess schoon}{en opgeruimd was}\\

\haiku{Hij maakte enige.}{dribbelpasjes om haastig}{bij mij te komen}\\

\haiku{Misschien was Fauch\'e.}{in de machinekamer}{in slaap gevallen}\\

\haiku{Gelukkig was er.}{niemand in de buurt toen ik}{op de deur klopte}\\

\haiku{De dieren vlogen.}{af en aan door de kleine}{opening in het gras}\\

\haiku{Het moet gegonsd en.}{gebromd hebben daar in de}{grond van stervensnood}\\

\haiku{{\textquoteleft}Is het jou weleens?}{opgevallen dat ik last}{van reumatiek heb}\\

\haiku{De stuurman wierp een.}{sentimentele blik op}{een van de foto's}\\

\haiku{Dat zou mij dus ook.}{kunnen gebeuren als Max}{altijd op zee bleef}\\

\haiku{De paniek over haar.}{ouderdom klemde een}{moment mijn adem af}\\

\haiku{{\textquoteright} vroeg ik toen we op.}{een avond in de taxi op weg}{naar de boot waren}\\

\haiku{Er werd gekucht, een,.}{sigaret opgestoken}{een woord gefluisterd}\\

\haiku{Ik werd droevig van.}{die stoere onhandige}{kerels om mij heen}\\

\haiku{De stuurman sloeg vroom.}{af en toe de maat met een}{wapperend handje}\\

\haiku{Een halve minuut.}{later liepen de tranen}{weer over zijn wangen}\\

\haiku{Max en ik keken.}{elkaar aan en slopen er}{behoedzaam naar toe}\\

\haiku{Ik had de indruk.}{dat hij te bang was om lang}{aan boord te blijven}\\

\haiku{Toen hij verheugd thuis,.}{kwam vond hij zijn vrouw in bed}{met een landgenoot}\\

\haiku{Dat zou tenminste.}{enige troost zijn voor het op}{de rede liggen}\\

\haiku{Weet je wat het kost?}{als je zelf de overtocht moet}{organiseren}\\

\haiku{Max sprong meteen op.}{en zocht geschrokken op de}{tast naar de lichtknop}\\

\haiku{Ik ging weer liggen.}{en luisterde gespannen}{naar de geluiden}\\

\haiku{niet want plotseling.}{draaide hij zich om en viel}{mij in de rede}\\

\haiku{Ik dacht aan de vrouw.}{in Stavanger die nu een}{baby koesterde}\\

\haiku{Geboorte en dood.}{op de boerderij deden}{mij op de vlucht slaan}\\

\haiku{Ik dacht dat er iets.}{ernstigs was gebeurd want ik}{zag dat hij beefde}\\

\haiku{Ik pakte de hand.}{van Max en omsloot ze met}{mijn beide handen}\\

\haiku{Op de bovenste.}{trede bleef hij staan en keek}{naar de kapitein}\\

\haiku{In de verte zag.}{ik een klein stipje op het}{water bewegen}\\

\haiku{de man stond die op.}{het ritme van de golven}{op en neer ging}\\

\haiku{Het is geen noodzaak,.}{dat jij ook werkt ik verdien}{meer dan genoeg hier}\\

\haiku{Het heeft me maanden.}{gekost om alles weer op}{orde te krijgen}\\

\haiku{Twee ggd-broeders.}{stapten uit en kwamen met}{een brancard aan boord}\\

\haiku{Even later kwamen.}{de broeders met de brancard}{door de smalle deur}\\

\haiku{Joviaal zwaaide.}{hij terug met een brede}{glimlach om zijn mond}\\

\haiku{{\textquoteright} Hij kuste me en.}{duwde me met bagage}{en al in de trein}\\

\haiku{Terwijl ik al op,}{de bovenste trede stond}{greep hij weer mijn hand.}\\

\haiku{Een voor een liet ik.}{de kledingstukken naast mij}{op de grond vallen}\\

\haiku{Het is het kleinste.}{en minst zeewaardige schip}{van de hele vloot}\\

\section{Albertine Steenhoff-Smulders}

\subsection{Uit: Een abdisse van Thorn}

\haiku{Dan wendde hij zich.}{met snelle beweging naar}{zijn geheimschrijver}\\

\haiku{Maar uit de bonte;}{vensters der Lambertuskerk}{straalde helder licht}\\

\haiku{z\'o\'o kan ik u mijn,;}{opdracht niet geven alsof}{ik tot een knecht sprak}\\

\haiku{Hij behandelt ons,.}{allen altijd alsof wij}{leekebroeders waren}\\

\haiku{ik weet zeker, in.}{dagen van nood zou hij uw}{trouwste vriend blijken}\\

\haiku{De Bisschop stond op:}{en stak zijn afgezant ten}{afscheid de hand toe}\\

\haiku{En ik vertrouw, dat.}{het maal naar genoegen van}{Uwe Edelheid zal zijn}\\

\haiku{Bedoelt gij onze?}{Hooge Vrouwe Mechthild of de}{Vorstin van Heynsbergh}\\

\haiku{De boomen, nog vol,;}{rossige knoppen stonden}{roerloos in de lucht}\\

\haiku{De zoetheid van den.}{avond was voor zijn rusteloos}{hart als koel water}\\

\haiku{{\textquoteleft}Neen,{\textquoteright} zeide hij, haar, {\textquoteleft};}{handen vattendegij zult}{doen wat u goeddunkt}\\

\haiku{voor gansch uw leven,.}{is dit eene beslissing die}{gij zelf moet nemen}\\

\haiku{De la Marck ving.}{hem in zijn armen op en}{ondersteunde hem}\\

\haiku{Want er staat nog meer,.}{nieuws in mijn brief dien ik u}{verder lezen zal}\\

\haiku{wanneer ge bang zijt,.}{voor booze droomen moet ge daar}{niet naar luisteren}\\

\haiku{zij had dan ook in.}{het stift den naam  van zeer}{hooghartig te zijn}\\

\haiku{{\textquoteright} {\textquoteleft}Ridder Willem heeft,?}{voor hem bij de Abdisse}{gebeden nietwaar}\\

\haiku{Ik moet oppassen,.}{en geen woord loslaten dat}{ik niet kwijt wil zijn}\\

\haiku{z\'o\'o strenge tucht als.}{onder haar beheer was er}{te Thorn nooit geweest}\\

\haiku{Heer Haeck herkende.}{den jonker van Perwez en}{riep hem bij zijn naam}\\

\haiku{Maar bij mijn patroon,!}{deze gewelddaad zal niet}{ongestraft blijven}\\

\haiku{'t Is een goed woord,.}{dat gehoorzamen lichter}{valt dan bevelen}\\

\haiku{{\textquoteright} zeide de Cleefsche:}{jonkvrouw met een blik op het}{fijne schilderwerk}\\

\haiku{Verdacht Josine?}{haar tot het huwelijk te}{hebben geholpen}\\

\haiku{straatroovers zouden zich;}{wel wachten de Vorstin van}{Thorn aan te randen}\\

\haiku{kwam zij hier, dan zou.}{zij hare gebiedster geene}{beleediging sparen}\\

\haiku{Waarom bezoekt gij,.}{niet meer de stad dat zou u}{afleiding geven}\\

\haiku{{\textquoteright} De Gravin zag met:}{haar heldere oogen den abt}{onbevangen aan}\\

\haiku{{\textquoteright} En zijn knie buigend,.}{voor den Bisschop trok hij zijn}{zwaard uit de scheede}\\

\haiku{{\textquoteleft}Zegen mij en mijn,,.}{wapen zoo ik het noodig mocht}{hebben mijn vader}\\

\haiku{{\textquoteright} Door alle buurten,.}{verspreidde zich het goede}{nieuws snel als de wind}\\

\haiku{Achter de boomen.}{was de hemel rooskleurig}{waar de zonne zonk}\\

\haiku{De meier sprong op.}{van de houten bank en ging}{den gast te gemoet}\\

\haiku{een tafel v\'o\'or de,;}{bank die zij met een doek van}{grof pellen dekte}\\

\haiku{Hier links om, de weg.}{die naar het water gaat bij}{den Plompen Toren}\\

\section{Reimond Stijns}

\subsection{Uit: Hard labeur}

\haiku{Het dier 			 sloeg met,;}{den kop sprong recht met een snel}{beenengescharrel}\\

\haiku{Gansch het dorp wist het,.}{weldra hoe kranig Mie zich}{verdedigd 			 had}\\

\haiku{Hij kwam eens tot bij,,.}{haar zei 			 hij om wil van}{het ontstichtend feit}\\

\haiku{ze tastte in den,.}{zak naar het mes dat ze nu}{altijd bij zich had}\\

\haiku{Maar wat is een mensch,,!}{die geen geld heeft of er geen}{genoeg 			 bezit}\\

\haiku{Ze staarde hem aan,,,.}{zwijgend met 			 vrees dat hij}{haar afkeuren zou}\\

\haiku{{\textquoteleft}'k Geloof, dat we ',!}{t akkoord 			 zijn maar geen}{katten in zakken}\\

\haiku{Hij had zijn lepel.}{afgelikt en lei hem nu}{neer op de tafel}\\

\haiku{Ze ging, zonder ziel,,;}{gebroken zonder wilskracht}{nog tot opstand}\\

\haiku{Soms wilde hij te,;}{huis blijven als 			 het weer}{te ijselijk was}\\

\haiku{hij mocht er voortaan,,,:}{urenlang 			 stenen weenen}{jammeren huilen}\\

\haiku{de ouders bleven ';}{aant werk op het veld of}{om de woning}\\

\haiku{Maar ja, hij was dan,.}{nog 			 schier een vreemdeling}{en kende haar niet}\\

\haiku{Ho, die verdoemde,!}{rekel kon geld verdienen}{en 			 wilde niet}\\

\haiku{ook de kousen op,.}{haar 			 beenen gevoelde hij}{maar ontdekte niets}\\

\haiku{Eens reeds had ze den,.}{doodsangst gevoeld die nu}{koud over haar neerzeeg}\\

\haiku{{\textquoteright} stotterde ze na, {\textquoteleft}{\textquoteright}.}{een poosof ik smijt mij}{in de Bundergracht}\\

\haiku{Speeltie liet hem los.}{om de deur met dreunenden}{paf toe te slaan}\\

\haiku{zijn 			 gedachten;}{zouden stilvallen bij zijn}{opgezweept pogen}\\

\haiku{over haar kort hemd 			 ,.}{droeg ze een lijveken dat}{niet toegeknoopt was}\\

\haiku{Vercleijen stak de.}{dunne wenkbrauwen op naar}{zijn klein voorhoofd}\\

\haiku{De jongens kunnen,.}{het gaan 			 zeggen als er}{verandering is}\\

\haiku{er volgde een snel.}{voetengescharrel 			 bij}{een wild gestommel}\\

\haiku{Hier is alles te,.}{krijgen wat ge maar in}{de stad vinden kunt}\\

\haiku{We zullen 			 hem;}{brengen tot aan het huis van}{Wieze-Marie}\\

\haiku{Of 			 wroette er?}{misschien minnenijd in het}{diepe van zijn ziel}\\

\haiku{zijn haar, dat Speeltie,.}{gisteren gesneden had}{was vol 			 trappen}\\

\haiku{rechts waren er maar,.}{drie en 			 daar hing niets v\'o\'or}{de donker ruiten}\\

\haiku{Roze, op haar 		  	,.}{zelfkanten sloffen ging en}{kwam geruchteloos}\\

\haiku{Ze naderde meer,.}{de 			 tafel en dacht niet}{aan heur slijkvoeten}\\

\haiku{hij lag er vroolijk.}{in teere lentefrischheid}{van blad 			 en bloem}\\

\haiku{Ze 			 voelde, dat,.}{ze nog niet wieden kon hoe}{ze ook haar best deed}\\

\haiku{Thuis had Lize nooit;}{iets anders gehad dan wat}{stroo en een kafzak}\\

\haiku{Men bracht Lize in,.}{een kamertje en de deur}{werd gesloten}\\

\haiku{Lize durfde de;}{oogen van den wegel niet}{laten afdwalen}\\

\haiku{daarna sloop ze naar,.}{de stalletjes en alles}{was 			 er ook vast}\\

\haiku{{\textquoteleft}'t Zal wel een jaar,.}{duren eer dat gij nen}{cent zult verdienen}\\

\haiku{Ge 			 hadt mij slecht,,?}{verstaan maar dat en zult ge}{nooit meer doen newaar}\\

\haiku{ze had nooit een pop,}{bezeten en in een vaag}{vizioen}\\

\haiku{{\textquoteleft}Vader moet subiet,.}{weten waarom Lize niet}{naar huis en 			 komt}\\

\haiku{Maar, 			 ge zijt, gij,,,?}{de deugniet die het arm jonk}{altijd slaag geeft he}\\

\haiku{{\textquoteright} {\textquoteleft}Lowietje, ga eens mee,{\textquoteright}.}{naar het huis van dien jongen}{verzocht Sofie}\\

\haiku{{\textquoteright} En plots groeide er;}{een onuitsprekelijke}{vrede in haar}\\

\haiku{de maan rees op, en}{er viel een zilvergroene}{schijn in brokken}\\

\haiku{Hier, op deze plaats,.}{had de doodkist gewacht met}{Wannie er in}\\

\haiku{Het avond-angelus, '.}{had geklept ent volk was}{weg van de akkers}\\

\haiku{Overal blonk het 			 ;}{krakend-rijp koren in de}{schoone morgenzon}\\

\haiku{boven klonk er een,.}{wilde kreet die verstierf in}{een gesmoord gesteen}\\

\haiku{Renieldeken lei,;}{het boek neer en haar boezem}{jaagde geweldig}\\

\haiku{de lucht was kalm, en;}{het oosten schitterde met}{stekende stralen}\\

\haiku{{\textquoteleft}G' en moet er al,{\textquoteright}.}{dat verdriet niet in maken}{zei ze meewarrig}\\

\haiku{Roze was toch heel,;}{goed geweest voor haar eer d\'at}{voorviel met Lowietje}\\

\haiku{{\textquoteleft}G' en moogt er niets, '.}{van gebaren maart schuim}{stond op haren mond}\\

\haiku{toen allen binnen,.}{waren sloop Sofie haastig}{naar het schotelhuis}\\

\haiku{Haar schreden brachten}{ze voort altijd verder over}{den zoo bekenden}\\

\haiku{, en voor de eerste.}{maal gaf Speeltie ze maar de}{helft van de som mee}\\

\haiku{'t Was altijd een,;}{leelijke tijd als de pacht}{moest betaald worden}\\

\haiku{{\textquoteright} Die stem maakte heur,,;}{laf gedwee en ze haastte}{zich door het hofgat}\\

\haiku{De jongste zoon liet;}{na elken slok wantrouwend}{de oogen rondwaren}\\

\haiku{Om vier uren schoof de,.}{lucht toe en eenige schaarsche}{druppels pletsten neer}\\

\haiku{het voeder uit het;}{wijmenboschje zou niet tot}{hooi kunnen drogen}\\

\haiku{hij had er reeds aan}{gedacht hun te verbieden}{des Zondags na den}\\

\haiku{{\textquoteright} Maar altijd voort bleef.}{dezelfde tergende stap}{op het wegelken}\\

\haiku{de aderen lagen.}{als een blauw koordennet op}{zijn bruine handen}\\

\haiku{gansch in de verte:}{klonk een enkele maal}{een brallend getier}\\

\haiku{Er lag voorzeker!}{reeds een verschrikkelijke}{som in het kistje}\\

\haiku{, en een hortende,.}{krakende donder doet de}{aarde daveren}\\

\haiku{In reusachtige;}{hallen woedt er een dreunend}{dommelen en slaan}\\

\haiku{, en wanneer ze den,;}{arm loslaten tjaffelen}{ze van rechts naar links}\\

\haiku{{\textquoteright} Zoo klabetterde '.}{de stem van Fine boven}{alt rumoer uit}\\

\haiku{{\textquoteright} en dan trachtte ze,.}{den hoek te ontwaren waar}{Mitie Speeltie zat}\\

\haiku{Peutrus, de waard, was,;}{in het schotelhuisken en}{tapte er het bier}\\

\haiku{Speeltie was niet meer,;}{benauwd maar opnieuw sterk en}{vol zelfsvertrouwen}\\

\haiku{indien Speeltie zich,;}{bezoop dan zouden ze kort}{spel met hem maken}\\

\haiku{om het broodje te.}{verdienen moet men toch iets}{door de vingers zien}\\

\haiku{Hij wendde zich om,.}{en plaatste den wijsvinger op}{de borst van zijn broer}\\

\haiku{{\textquoteleft}Wie ons belet heeft,!}{er te winnen moet nu maar}{voor ander zorgen}\\

\haiku{Het was echter geen,;}{broederlijkheid die Mitie}{en So samenbracht}\\

\haiku{{\textquoteleft}'t Mag in vergift,.}{veranderen als ik er}{mijn lippen aan steek}\\

\haiku{{\textquoteleft}Preutelt ge misschien,?}{z\'o\'o omdat ik hier tegen}{uw goesting verkeer}\\

\haiku{{\textquoteleft}En iemand, die het,.}{al lang wist heeft mij vandaag}{de oogen opengedaan}\\

\haiku{hij had nog eens So,;}{naar huis gezonden en was}{komen vensteren}\\

\haiku{{\textquoteleft}Heb ik zelf beter,,?}{eten gehad dan gij zoolang}{ik niet ziek en was}\\

\haiku{{\textquoteleft}Er valt dezen avond.}{niets meer te verteuteren}{in den Koterhaak}\\

\haiku{ik kan, en 'k zal,.}{mij stil houden zoolang men}{mij niet en bedriegt}\\

\haiku{{\textquoteleft}Ik, ik en laat mij,!}{geen smering geven en ik}{en heb geenen doek noodig}\\

\haiku{g'en hebt nu niets te,.}{doen en vandaag nog moet ge}{naar den huismeester}\\

\haiku{Van de eerste week,,.}{af dat we ginder zijn zal}{ik hen betalen}\\

\haiku{hij rekte den hals,;}{uit en stak de kin omhoog}{om asem te hebben}\\

\haiku{Het haardvuur was lang, ';}{uitgedoofd ent water}{in den pot verdampt}\\

\haiku{Gij moogt doen, wat ge,.}{wilt maar met u en kom ik}{onder zijn oogen niet}\\

\haiku{hij en doet zijnen,!}{mond niet open en hij en wil}{noch eten noch drinken}\\

\haiku{Het was wellicht reeds,.}{lang na middernacht toen ze}{nog eens wakker werd}\\

\haiku{hij keek een poosje,}{naar een delver die ver op}{het veld zijn klompen}\\

\haiku{En voor zijn dood zou! '!}{hij nog het mijne willen}{t Is al te zot}\\

\haiku{{\textquoteright} Mie zou gaarne iets,.}{meer gehoord hebben doch ze}{wachtte te vergeefs}\\

\haiku{We worden beiden, ',.}{oudt is waar doch te oud}{en zijn we nog niet}\\

\haiku{Er is veel over ons,.}{gebabbeld en we zullen}{dat doen ophouden}\\

\haiku{Maar, God lof, Bien had,!}{verzekerd dat de schelm niet}{meer genezen kon}\\

\subsection{Uit: Arme menschen}

\haiku{En Mie zou het niet,.}{meer kunnen verbergen wat}{er gebeuren moest}\\

\haiku{{\textquoteright} In onze buurt woont,;}{een ongehuwde vrouw met}{haar zoon krommen Tist}\\

\haiku{Moet ik ze den hals,?}{omwringen en zoo achter}{de grendels komen}\\

\haiku{Stipt betaalde ze,.}{en tot nu toe kende Jaak}{Gone haar val niet}\\

\haiku{Mie Gone heeft juist,.}{een rok versteld wil ander}{werk ter hand nemen}\\

\haiku{En Nelleken was,...}{zoo voorzichtig maar bij het}{oversteken der straat}\\

\haiku{{\textquoteleft}Vader, vader, laat...{\textquoteright}.}{mij hier Onverwachts duwde}{hij haar achteruit}\\

\haiku{Een deurtje ging open,,:}{een vogel kwam te voorschijn}{en tienmaal klonk het}\\

\haiku{{\textquoteright} Jaak wendde vroolijk.}{zijn stoel om en scheen te zien}{naar het horloge}\\

\haiku{de vrouw zal vader.}{en moeder verlaten om}{haar man te volgen}\\

\haiku{Waarom laat ge mij, '?}{verstaan dat ik vant werk}{mijner dochter leef}\\

\haiku{gij ontnaamt hem zijn,,,...}{gade twee kinders het licht}{zijner oogen en nu}\\

\haiku{het was koesterend,.}{warm in de plaats en helder}{keek er de zon in}\\

\haiku{Haar stem trilde, toen;}{ze hem bad zijn fleschje}{te mogen halen}\\

\haiku{En als ik weet, dat,....}{ge zoo zult handelen dan}{zal ik rustig gaan}\\

\haiku{Soms dacht ze, dat ze,}{zoo aanstonds haar vader zou}{zien binnentreden}\\

\haiku{{\textquoteleft}En de heks ging weer!}{opsteken zonder iets te}{laten  weten}\\

\haiku{'t was, omdat ze,;}{iemand anders gevonden}{had iemand met geld}\\

\haiku{niet meer wil, daar ze,, '.}{een oude een rijke heeft}{diet geld weggooit}\\

\haiku{leefde ze dan nog,!}{alleen om de aanklacht zou}{hij haar doodslaan}\\

\haiku{Er kwam een kindje,...}{en er zonk iets almachtig}{teeders in haar hart}\\

\haiku{{\textquoteright} Het goede meisje,,.}{voelde zich geroerd boog zich}{kuste dat uitschot}\\

\haiku{vol schrik keek ze de,,:}{weduwe aan trok deze}{achteruit en sprak}\\

\haiku{Dagen later was.}{het er heel droevig op het}{zolderkamertje}\\

\haiku{De armendokter,,.}{was gekomen had gezegd}{dat het niets was}\\

\haiku{Zijn fonkelend brein.}{heeft de laatste olie in het}{klein lampje verbrand}\\

\haiku{Denkt ge niet, moeder,?....}{dat vader ginder boven}{gansch anders zal zijn}\\

\haiku{En gingt ge nu voort, '.}{k zou zelfs de doodkist niet}{kunnen betalen}\\

\haiku{Mie Gone voelde,:}{een verlichting toen Wanne}{haar verlaten had}\\

\haiku{daarna was het tijd...}{voor haar om zich naar het werk}{te begeven}\\

\haiku{veertien dagen lang;}{kon ze niet terugkeeren om}{wil van haar vader}\\

\haiku{ze richtte zich naar,;}{den buitenkant rechts waar men}{de kleine begroef}\\

\haiku{daar had men het  ,;}{kuiltje gedolven niet ver}{van een treurwilgje}\\

\haiku{In een huis als het,.}{mijne mag iemand als gij}{den voet niet zetten}\\

\haiku{De weduwe van,}{Marli of ze wist in wat}{berooiden toestand}\\

\haiku{ze dacht er niet aan,,;}{dat haar borst schier ontbloot was}{doch hij zag het wel}\\

\section{Louise Stratenus}

\subsection{Uit: Een verborgen bladzijde uit het leven van Sherlock Holmes}

\haiku{Ik spreek de waarheid,.}{die zelden aangenaam is}{om aan te hooren}\\

\haiku{maar wij zullen ons.}{door geen  hinderpalen}{laten afschrikken}\\

\haiku{een brandgangetje!}{tusschen het spookhuis en het}{aangrenzend perceel}\\

\haiku{Ik voor mij zag niets,;}{dan een zeer alledaagsche}{verlaten keuken}\\

\haiku{wel stak de sleutel,.}{er van buiten op maar hij}{was niet omgedraaid}\\

\haiku{die vrouw was schrander,!}{dat zouden niet velen haar}{hebben nagedaan}\\

\haiku{{\textquoteright} {\textquoteleft}O, heel gaarne, als.}{mijne levensrust er maar}{niet door wordt verstoord}\\

\haiku{Was die kleine maar,;}{eens voor goed weg dan zou er}{nog hoop genoeg zijn}\\

\haiku{maar hij vertrok twee.}{dagen geleden en kwam}{van morgen terug}\\

\haiku{Zij hadden alles..{\textquoteright} {\textquoteleft}?}{voor anderen overEn de}{ouders zeker ook}\\

\haiku{Mijnheer Andr\'e, haar,.}{man was zoo geheel anders}{dan zijne ouders}\\

\haiku{{\textquoteleft}Welk een mooi gezicht!}{en hoe teer en bevallig}{is die gestalte}\\

\haiku{{\textquoteright} {\textquoteleft}Nu, nu, ik zou mij}{in uwe plaats de zaak maar niet}{zoo erg aantrekken}\\

\haiku{men kon zien, dat zij.}{bestemd was spoedig naar den}{hemel weer te keeren}\\

\haiku{{\textquoteright} {\textquoteleft}Ik ga een bezoek.}{afleggen bij vrienden van}{mevrouw Monkbridge}\\

\haiku{{\textquoteleft}Hebt gij nog moed mij?}{naar de familie Arundel}{te vergezellen}\\

\haiku{Nu eens meen ik dat,}{er iemand door de kamer}{sluipt dan weder is}\\

\haiku{Ik word dan ook nooit}{moede er de ontknooping}{van gade te slaan}\\

\haiku{{\textquoteleft}daarna zal ik mij.}{pas weer aan oorspronkelijk}{werk durven wagen}\\

\haiku{{\textquoteright} {\textquoteleft}Maar mijn waarde{\textquoteright}, kon.}{ik niet nalaten in het}{midden te brengen}\\

\haiku{maar op die plaats of.}{in den omtrek was geen spoor}{van hem te vinden}\\

\haiku{{\textquoteright} {\textquoteleft}En dat zal ik ook{\textquoteright},,.}{doen verklaarde Holmes mij}{lachend aanziende}\\

\haiku{Hij zal niet weinig}{verbaasd opkijken als hij}{u thans leert kennen}\\

\haiku{Men heeft de grootste.}{moeite haar een verstaanbaar}{woord te doen uiten}\\

\haiku{hij scheen van oordeel,.}{te zijn dat die hulde hem}{rechtmatig toekwam}\\

\haiku{Ik heb haar niet meer,.}{gezien nadat ik den brief}{had afgegeven}\\

\haiku{Wat hebben zij aan?}{een zilveren soeplepel}{of iets dergelijks}\\

\haiku{{\textquoteright} {\textquoteleft}Waarom gaat gij nog?}{niet eens een bezoek aan ons}{buurvrouwtje brengen}\\

\haiku{Het kan toch uw wensch,?}{niet zijn dat iemand door uw}{toedoen zou sterven}\\

\haiku{{\textquoteright} {\textquoteleft}En ik begrijp er{\textquoteright},.}{al minder en minder van}{gaf ik ten antwoord}\\

\haiku{{\textquoteright} {\textquoteleft}Voor het oogenblik,{\textquoteright},, {\textquoteleft}.}{neen antwoordde hij met kracht}{maar misschien vroeger}\\

\haiku{Die allen vragen.}{zich slechts af hoe spoedig zij}{thuis zullen komen}\\

\haiku{Ik kan er niet over.}{oordeelen voordat ik de}{wond heb onderzocht}\\

\haiku{Mabel Sandford had.}{de zachtste snaren van zijn}{gemoed doen trillen}\\

\haiku{Niet weinig verrast,:}{ons beiden tegenover zich}{te zien zeide zij}\\

\haiku{{\textquoteright} {\textquoteleft}Verder neem ik een,.}{der flinkste agenten mee die}{mij ooit bijstonden}\\

\haiku{Nergens vertoonde.}{zich een spoor door een wapen}{achtergelaten}\\

\haiku{Toch wendde ik nog.}{eene laatste poging aan om}{haar te vermurwen}\\

\haiku{Zij was zonder een{\textquoteright},.}{kreet ter aarde gevallen}{mompelde Percy}\\

\haiku{Ik begaf mij zelfs,}{dien morgen tot een beroemd}{hoogleeraar dien}\\

\haiku{Ik was nog bezig,.}{met mijn ontbijt toen Lestrade}{bij mij binnentrad}\\

\haiku{{\textquoteright} {\textquoteleft}Wij deden het om,.}{beurten wetende hoe schuw}{hij voor vreemden is}\\

\haiku{Hij gaf den door hem;}{gekozen agent een wenk en}{beiden verdwenen}\\

\section{Stijn Streuvels}

\subsection{Uit: Jantje Verdure}

\haiku{er vette blazen,.}{in opzwollen die met een}{zucht uiteen barstten}\\

\haiku{mede - het was de,;}{lucht die hem deugd deed gelijk}{water aan den visch}\\

\haiku{omdat zij er op.}{gesteld was de bakkerij}{in gang te houden}\\

\haiku{Toen waren ze drie,;}{struische jonkheden in}{den bloei van hun macht}\\

\haiku{Het ging er den dag,,,....}{door oven in oven uit zonder}{eind of ophouden}\\

\haiku{om 't geen zij zijn {\textquoteleft}{\textquoteright},.}{kwaden duivel noemde in}{bedwang te houden}\\

\haiku{of hij zelf het was,?}{ofwel Theresia die het}{kwaad in hem stookte}\\

\haiku{fleemde Theresia,,:}{en als de meid reeds op}{straat was riep ze nog}\\

\haiku{- de krekels met hun, '!}{schril gekriept was alsof}{zij hem uitlachten}\\

\haiku{Vroeger had hij nooit,.}{vermoeienis noch het eind}{van zijn macht gekend}\\

\haiku{de ellendigste, -.}{dompelaar der wereld door}{het kwaad bezeten}\\

\haiku{- Ga waar ge wilt, maar, -!}{hier bij mij niet de duivel}{moet eerst uit uw lijf}\\

\haiku{Als Theresia eens;}{iemand uitzond om af te}{spieden waar hij ging}\\

\haiku{- hij stond er bij met,.}{gebogen hoofd beschaamd als}{een verschopte hond}\\

\haiku{- Zie, vandage zou,.}{ik moeten tarwe ziften}{en ik ontzie het}\\

\haiku{schreeuwde Treze, en;}{in een vlaag van razernij}{stiet zij de deur open}\\

\haiku{w\`at zal er mij nu,,?}{overkomen vandaag en de}{volgende dagen}\\

\haiku{Hij kreeg den daver,;}{op het lijf als iemand die}{een koude wind voelt}\\

\haiku{hij zelf gaf ook niet;}{meer toe aan de vrees er te}{zullen verstijven}\\

\subsection{Uit: 'Het leven en de dood in den ast'}

\haiku{Onverwijld nu zal.}{het derde vertoon van het}{schouwspel aanvangen}\\

\haiku{Van eerst af is hij, ':}{op dreef en de draad vant}{verhaal gevonden}\\

\haiku{eer de anderen '!}{het gewaarwerden wast}{al afgeloopen}\\

\haiku{- Ik heb het toen zelfs.}{aan Polfliet noch aan Wipper}{durven vertellen}\\

\haiku{we lieten er hem,,.}{voor elk drie dat was negen}{deuntjes aflappen}\\

\haiku{Ik plaatste hem met de.}{zate op mijn hoofd en hield}{hem bij de pikkels}\\

\haiku{We waren allen '.}{omt even welgezind en}{preusch met den koop}\\

\haiku{hij houdt Fliepo bij ',,;}{t gat van zijn broek en trekt}{lijfelijk trekken}\\

\haiku{'t overige blijft.}{in dunne laag opengestrooid}{om op te drogen}\\

\haiku{Hij is er altijd,!}{zoo bang voor geweest en nu}{lijkt het niemendal}\\

\haiku{Bij 't ontwaken,.}{blikt Fliepo angstig rond als}{om iets te zoeken}\\

\subsection{Uit: Minnehandel. Deel 1}

\haiku{wendde zij nog naar,:}{moeder en weer zonder naar}{antwoord te wachten}\\

\haiku{De beurt ging dapper,,, {\textquoteleft}{\textquoteright}.}{voort Sanne de groote blonde}{meid zong vanLaura}\\

\haiku{- 'k Zou dat liedje,?}{ook willen leeren geef het mij}{om uit te schrijven}\\

\haiku{Jan Derycke moest '.}{het meisje kussen dat hij}{t meest beminde}\\

\haiku{hij voelde dat er;}{hem iets ontbrak dat hij niet}{goed kon uitbrengen}\\

\haiku{Die sone die nam,.}{die menscheit aen die bi den}{vader comen can}\\

\haiku{Die os ende die}{esel die hebbent gheweten}{mer dat dat kint is}\\

\haiku{- Vandage was ze}{met moeder alleen thuis en}{in den achternoen}\\

\haiku{Heur stap danste licht ';}{overt stroo en ze deed de}{inzen rinkelen}\\

\haiku{- Ja, dat hoort erbij,!}{we moeten den duivel soms}{een keersken luchten}\\

\haiku{En Anneke nam.}{den korf met turksch koorn en was}{m\^ee de deuren uit}\\

\haiku{Uw beminnende '.}{altijd voort leven}{lang Phara{\"\i}lde}\\

\haiku{In een wrong was ze -}{weer beneden en den brief}{droeg ze op haar hert.}\\

\haiku{- de tranen sprongen,.}{haar uit de oogen en haar keel}{was toegenepen}\\

\haiku{- Aan niemand zeggen,.}{Max mag niet weten dat ik}{het u verteld heb}\\

\haiku{- Max, Max, wat hebt gij?!}{toch aan mij gedaan dat ik}{u zoo geerne zie}\\

\haiku{- Aan Klaarken moeten ',!}{wet niet vragen dat is}{bijlange gekend}\\

\haiku{- 'K kom uit den meersch,, ', '.}{Maxt gras groeit gulzigt}{is schoon om te zien}\\

\haiku{Dan stak Elsje haar:}{hoofd van achter den muur en}{ze riep al lachend}\\

\haiku{haar wezens waren.}{blozend welgezind en de}{oogen straalden vol lust}\\

\haiku{Al de anderen '.}{kwamen bij en elk beschonk}{die hijt liefst zag}\\

\haiku{- hebt medelijden,,!}{hebt compassie hier met een}{armen blindeman}\\

\haiku{En zij was jong En!}{ik was jong En gij kunt wel}{denken hoe het gong}\\

\haiku{De kerels stormden ';}{binnen int Sterreke}{als in eigen huis}\\

\haiku{De vruchten die van,.}{weerskanten den weg stonden}{ze groeiden lijk zot}\\

\haiku{Met die belofte}{legde hij zijn moede lijf}{te rusten en voer}\\

\haiku{dacht hij zonder te....}{durven zeggen of vragen}{wat er haperde}\\

\subsection{Uit: Minnehandel. Deel 2}

\haiku{buiten dat was er, ',;}{niets op de wereld overt}{dorp over de velden}\\

\haiku{Al op \'een teeken '.}{wast bedrijf begonnen}{over heel de vlakte}\\

\haiku{Verder kwam hij langs,,....}{de hooge roggevelden de}{tarwe de beeten}\\

\haiku{in al zijn geluk,:}{hij aanzag zijn leven al}{een anderen kant}\\

\haiku{- Met die gedachten,}{opgewonden liep hij rond}{en met leede oogen}\\

\haiku{- Hoort ge 't gasten,.}{riep hij naar Max en Fons die}{kwamen bijzitten}\\

\haiku{dat hij iets wonen,!}{wist een princesse van een}{meid en pleizierig}\\

\haiku{de korte houwen.}{neerstig het kruid van tusschen}{de planten schreepten}\\

\haiku{omdat hij zulke...}{zotte dingen deed zonder}{heur te raadplegen}\\

\haiku{- Wat moet hij weten,,!}{hij moet het niet weten hij}{mag het niet weten}\\

\haiku{- voor dat zot gedacht!}{van dien peerdenkweek moeten ze}{ons al ontpachten}\\

\haiku{gaat ge elders geld, '?}{halen voor de landpacht en}{t zaad en de mest}\\

\haiku{- Met hem niet meer dan... -,.}{met een ander Dat zou ik}{niet kunnen Marie}\\

\haiku{vroeger zette men:}{uit achter een wijf lijk een}{boer achter een kalf}\\

\haiku{Pauwels, wie had er?...}{zooiets durven denken dat}{hij met zijn dochter}\\

\haiku{Max was ook wel in,,...}{aanzien ze wist het maar de}{zaken stonden nu}\\

\haiku{Hij dubde, 't ging.}{tegen zijn gemoed maar hij}{dorst het niet zeggen}\\

\haiku{We kunnen altijd '...}{den voet nevens hem zetten}{alst er opaan komt}\\

\haiku{hij liet hem geen tijd.}{en begon altijd zelf van}{de gevreesde zaak}\\

\haiku{'k ben Meijer even:}{gaan uithooren maar er is}{niets uit te krijgen}\\

\haiku{- En als 't hof nu?}{zou open komen en ik het}{zou willen pachten}\\

\haiku{Kannaert, Derycke,,;}{elk voor zich snapte  maar}{wat hij krijgen kon}\\

\haiku{Daarom zelf genoot}{hij van de voldoening om}{den goeden naam dien}\\

\haiku{Aan het hofgat vond.}{hij Peetje Mullie met zijn}{dochter staan kouten}\\

\haiku{of waarom bleef hij?}{niet thuis waar hij zoo wel en}{weeldig en vrij was}\\

\haiku{- De heeren meenen.}{misschien dat er goud in den}{grond te delven ligt}\\

\haiku{hij zag dat ze zich.}{verveelde en liever niet}{alleen met hem was}\\

\haiku{o, ik zie altijd;}{nog uw schoone zwarte oogen}{in mijn gedachten}\\

\haiku{Dan was de drukke -:}{kwettering van andere}{vogels begonnen}\\

\haiku{t moet hier nog al!}{verricht worden en seffens}{komen de gasten}\\

\haiku{Het meisje schreeuwde, '.}{scharrelde om los en sloeg}{hem int wezen}\\

\haiku{- Max ziet er goed uit,.}{vezelde Sanne Kannaert}{tegen Martje Kraaynest}\\

\haiku{, en als er zulk eene.}{in het spel komt vergeet men}{de oude liefde}\\

\haiku{Ze gingen voort bij.}{de bende waar er luide}{leute gemaakt werd}\\

\haiku{riep de burgmeester}{en hij sloeg met de opene}{hand  op den boer}\\

\haiku{En hij twijfelde.}{ook al of het werkelijk}{een ongeluk was}\\

\haiku{Ginder, tenden de;}{dreef kwam de troep op maatstap}{van den trommelslag}\\

\haiku{De dag was nog nog,...!}{niet teneinde en de nacht}{bij lange nog niet}\\

\haiku{- Laat me nog wat, dans, ',.}{maark blijf hier bij moeder}{zei ze vriendelijk}\\

\haiku{ge ziet het wel - we '}{hebben veel leute gehad}{van den zomer en}\\

\haiku{- Ik vergeet het zoo,,:}{gauw niet we dansen nog nu}{zijn we familie}\\

\haiku{dan werd Klara weg.}{gehaald en Max danste met}{een ander meisje}\\

\haiku{Ze worstelden daar,}{zwijgend en alleen tot ze}{hem heel overmeesterd}\\

\haiku{- Ja, ge hebt haar uw,!}{hof al afgestaan ze trouwt}{met Fons Derycke}\\

\haiku{- Zij kan doen wat ze, -.}{wil maar dan moet ze buiten}{op den stond ik blijf}\\

\haiku{ze ving er hier en}{daar een spreuk van op maar dat}{alles lag zoo ver}\\

\haiku{Daarbuiten, verder ';}{overt veld hing de avondlucht}{in schemerblauwte}\\

\haiku{Voor hen bestonden.}{de meisjes en al wat er}{van leute rond stond}\\

\haiku{Zie, nu danst hij wel,,!}{met Mathilde die met haar}{eerste liefde ha}\\

\haiku{- Mij overpeinzen, ik}{heb er geen haaste bij om}{nu al te trouwen}\\

\haiku{- Dat is den rechten,....}{zin als er meenste bij is}{grijpt de boer maar door}\\

\haiku{- We kunnen er niets,,.}{aan doen Klara die dingen}{hangen in de lucht}\\

\haiku{waren stand van de.}{dingen of wat er nu goed}{of slecht gebeurd was}\\

\haiku{- Ik ben blij, vrouwe ',.}{datt voorbij en gedaan}{is mompelde hij}\\

\haiku{hij wilde haar niet.}{meer genaken en ging al}{den overkant alleen}\\

\subsection{Uit: De oogst}

\haiku{Waarom hield zij de?}{armen zoo hoog en haar lijf}{zoo uitgespannen}\\

\haiku{dat Pikkaert Zondag.}{laatst gevochten had tegen}{drie felle boschkanters}\\

\haiku{Waarom deed hij niet?}{lijk Pol en Lieven en Jaak}{en zijn ander broers}\\

\haiku{alles uit zijn hoofd,.}{steken lustig leven en}{aan niemand denken}\\

\haiku{Morgen zal 't er,}{op losgaan jongen hoe meer}{we werken hoe meer}\\

\haiku{voor kermiszondag.}{waren we met ons zakken}{vol geld alweer thuis}\\

\haiku{Bij Qu\'elin, daar zal ' ';}{t een lang getij opt}{zelfde gedoen zijn}\\

\haiku{was ik dat gij mij.}{stuur zoudt bekeken hebben}{en boos zijn op mij}\\

\haiku{z'hadden met een half '.}{dozijn pikkers heelt dorp}{omvergestooten}\\

\haiku{Maar nu moest het er}{door en hij vertelde in}{korte  reken}\\

\haiku{'s Anderen daags.}{zetten de twee voorloopers}{uit met de boodschap}\\

\haiku{Wie mag dat hier al?}{bezorgen en bezeilen}{op zulk een gedoen}\\

\haiku{Hier en daar \'e\'en wreef.}{den vaak uit de oogen en keek}{vragend in de lucht}\\

\haiku{Na het eten zochten.}{zij koelte en verfrissching}{in den vischvijver}\\

\haiku{een dubbele dreef '.}{oogstkoorn die vant veld tot}{aan de hofste\^e stond}\\

\haiku{en Lida, ziet ze,?}{hem geern en verlangt ze ook}{tot hij naar huis komt}\\

\haiku{Wies zat nog altijd.}{te luisteren als Aga reeds}{lang uitgekout was}\\

\haiku{de vijf kunnen we,,,!}{hier niet blijven toe kerels}{niet hondig zijn hoor}\\

\haiku{En Aga, o, Aga, ze}{zat nu zeker nog verpaft}{te kijken naar den}\\

\haiku{vreemden jongen die.}{het zilverstuk in heuren}{schoot geworpen had}\\

\subsection{Uit: De teleurgang van de Waterhoek}

\haiku{- Warm weer... Eens dat de,.}{brug er ligt zal het toch veel}{gemak meebrengen}\\

\haiku{- 'k Heb mij laten.}{gezeggen dat hier een brug}{over de Schelde komt}\\

\haiku{bij huwelijk en.}{bij geboorte werden er}{geen vreemden geduld}\\

\haiku{dan waart ge ineens,?}{van indringers en brug al}{te zamen verlost}\\

\haiku{In Broeke's kop stond;}{het nog altijd als iets dat}{niet gebeuren k\'an}\\

\haiku{De mannen trimpten,;}{naderbij stonden schijnbaar}{onverschillig}\\

\haiku{Hij dorst zichzelf de.}{afkeer niet bekennen om}{dat lijk nog te zien}\\

\haiku{over de meers hing de,;}{mist eendikte zodat men}{geen worp ver zien kon}\\

\haiku{de streek van Ronse,.}{af en zou eerst tenden de}{week terugkeren}\\

\haiku{- Neen wijf, niemand mag,.}{het weten we mogen het}{niet voortvertellen}\\

\haiku{Wanne bleef echter.}{in twijfel en niet geneigd}{om aan te pakken}\\

\haiku{Ik gevoel geen lust.}{Lander gezelschap te gaan}{houden waar hij zit}\\

\haiku{Dat de brug er komt,:}{en de steenweg kan ons niets}{dan voordelig zijn}\\

\haiku{Werken om geld te,?}{verteren was er buiten}{dit nog iets anders}\\

\haiku{- Gooi die vent buiten,,.}{zegde hij droog weg hij komt}{onze baard smouten}\\

\haiku{Het was kwestie van....}{uit de ogen te zien en te}{laten betijen}\\

\haiku{In Thyssen en zijn:}{makkers herkenden zij hun}{eigen geaardheid}\\

\haiku{Een verdomd dingen:}{dat geen van zijn eigen zoons}{er toe geschikt scheen}\\

\haiku{- Ik zal het verdomd!}{wel uithouden tot die me}{kan opvolgen ook}\\

\haiku{Zijn stap was immer,;}{gehaast zijn blik verstrooid of}{naar binnen gekeerd}\\

\haiku{Met de personen ';}{vant hotel zelf kwam hij}{weinig in gesprek}\\

\haiku{t Overige van.}{de dag besteedde hij aan}{zijn briefwisseling}\\

\haiku{riep  hij zwetsend,.}{naar een makker d\'a\'ar zal hij}{wat tegenkomen}\\

\haiku{hij verlangde naar -;}{niets zijn gemoedsrust had hij}{volkomen bewaard}\\

\haiku{Thyssen praatte voort,.}{deed alsof hij er niets van}{wist of gevoelde}\\

\haiku{in 't water te,?}{rollen dat ze u morgen}{verzopen vinden}\\

\haiku{Doch wanneer hij zocht}{hoe het uit te voeren of}{op welke manier}\\

\haiku{Zij kwelde zich met:}{vermoedens tenemaal uit}{de lucht gegrepen}\\

\haiku{Hij keek er naar uit,}{enkel en alleen omdat}{die wandelaarster}\\

\haiku{Inwendig was hij.}{tevreden er alzo van}{af te geraken}\\

\haiku{Ziet ge, zonder zijn.}{tussenkomst was het toch langs}{die kant uitgedraaid}\\

\haiku{als ze er trek in, ':}{vond hem even aan te gluren}{wast gelopen}\\

\haiku{Nu er echter weer,.}{iets op schuit was voelde hij}{trek er bij te zijn}\\

\haiku{E\'en voor \'e\'en trok hij.}{bij arm of schouder op de}{pont en voer aan kant}\\

\haiku{- En hoe dat ge nu,?}{hier zit als die heer u thuis}{moest komen vinden}\\

\haiku{- richt er later mee,.}{uit wat ge wilt hij zal u}{de handen likken}\\

\haiku{Niet dat hij bang was -.}{voor de ontknoping het moest}{toch eens gebeuren}\\

\haiku{Soms stelde Mira:}{er een wreed genot in haar}{minnaar te plagen}\\

\haiku{haar tot beternis,.}{te brengen tot deftigheid}{aan te wakkeren}\\

\haiku{God, wie zou er ooit!}{gedacht hebben dat hem zo}{iets te wachten stond}\\

\haiku{een hinderlaag waar -.}{hij zich had laten vangen}{een oord van verderf}\\

\haiku{Hij rekende het:}{zijn moeder aan als ikzucht}{en eigenbelang}\\

\haiku{evenals de stroom van ',.}{t water in de Schelde}{gestadig vooruit}\\

\haiku{Als ge mij anders,,.}{wilt laat me dan maar en zoek}{er een van uw soort}\\

\haiku{Maurice had heel 't ',.}{bestier ent bevel moest}{overal raad geven}\\

\haiku{al de huizen met,.}{loof vaantjes en kleurige}{lanteerntjes gepint}\\

\haiku{De gedachte aan;}{Mira drukte hem weer in}{de werkelijkheid}\\

\haiku{telkens de hartstocht,}{hem overmande was het als}{baren die komen}\\

\haiku{Spikkerelle in.}{nar verkleed die muilen trok}{en grappen maakte}\\

\section{Nico van Suchtelen}

\subsection{Uit: De stille lach}

\haiku{Drie jaren voor je,;}{geboorte dacht ik dat mijn}{verlangen dood was}\\

\haiku{Neen, Agnes, ik zal.}{geen half-verdichte}{memoires schrijven}\\

\haiku{Uw gestalte was.}{mij bekend en vertrouwd als}{uit een ouden droom}\\

\haiku{En ge houdt veel van,,.}{uw oom dat zag ik ook in}{dienzelfden glimlach}\\

\haiku{In elk geval was.}{het d{\'\i}e glimlach die mij moed}{geeft u te schrijven}\\

\haiku{Mevrouw, ik zal het,;}{lied opschrijven ofschoon g{\'\i}j}{het ook moet kennen}\\

\haiku{Maar enfin, mogen;}{zij onbeantwoord blijven}{tot in eeuwigheid}\\

\haiku{U hebt ook met een;}{bijzondere aandacht op}{Anneke gelet}\\

\haiku{U hebt, o, u hebt;}{dien dag zooveel gedaan en}{vooral niet gedaan}\\

\haiku{En Jaap heeft het nog {\textquoteleft},{\textquoteright}.}{dikwijls overdat mefou dat}{zoo voor me lachte}\\

\haiku{Ik zou hem dan maar,.}{haten en mij bovendien}{schamen over mijzelf}\\

\haiku{En ten slotte, z\'o\'o ',.}{als ikt daar beschrijf zoo}{zag ik het leven}\\

\haiku{En vandaag dacht ik {\textquoteleft} '.}{er bijent moeilijkste}{om over te schrijven}\\

\haiku{Ik wilde u heel;}{iets anders vertellen dan}{wat ik ben of doe}\\

\haiku{Maar d\`an, als ik den,.}{stillen lach hoor dan begrijp}{ik ze en glimlach}\\

\haiku{Daarom, als ik de,}{menschen minacht dan doe ik}{dat alleen omdat}\\

\haiku{{\textquoteleft}zie je wel, zie je?}{wel dat menschen elkaar t\'och}{begrijpen kunnen}\\

\haiku{En nu tracteer ik.}{mijzelf op middagthee en}{mijmer zoo'n beetje}\\

\haiku{Ja, welbeschouwd geldt.}{dit alles precies evenzeer}{tegenover menschen}\\

\haiku{Ik geloof zelfs dat,,.}{het sterker gezonder wordt}{physiek en psychisch}\\

\haiku{Maar van wie of wat?}{gaat de opzettelijke}{verandering uit}\\

\haiku{Dat ik dit alles.}{beleefd heb kan ik mij haast}{niet meer voorstellen}\\

\haiku{sinds zijn terugkomst.}{uit het gesticht heeft hij haar}{nooit meer mishandeld}\\

\haiku{{\textquoteleft}Ik heb honger, ga,{\textquoteright}.}{mee naar Vrijland en houdt haar}{den Prins Regent voor}\\

\haiku{- Juffrouw Bergman had:}{op de Zondagschool aan de}{kinderen gevraagd}\\

\haiku{Voor het eerst heb ik:}{Jaap echt driftig gezien op}{een levenloos ding}\\

\haiku{Dit zijn de vragen:}{waarmee ik mij kwel als ik}{het portret aanzie}\\

\haiku{Gelukkig, want van.}{middag betrapte ik de}{ware boosdoeners}\\

\haiku{Ik stond op en in;}{het zelfde oogenblik hief}{het beest den kop op}\\

\haiku{Zij vloog zeer snel en.}{voortdurend hoorde ik een}{sterk suizend geluid}\\

\haiku{Ik wil u heusch;}{liever alleen schrijven voor}{mijn eigen plezier}\\

\haiku{of, om in den geest {\textquoteleft}{\textquoteright}:}{te schrijven van uw waarlijk}{vernuftigen vriend}\\

\haiku{{\textquoteleft}Wees voorzichtig en,.}{kom terug als je er je}{ongelukkig voelt}\\

\haiku{{\textquoteleft}Oom, vertel me eens,;}{wat van Joost Vermeer u zult}{hem wel goed kennen}\\

\haiku{En hij, blij dat ik,.}{zoo kennelijk een praatje}{verlangde ging door}\\

\haiku{en we hadden hier ';}{eensn mierenhoop van meer}{dan een meter hoog}\\

\haiku{{\textquoteleft}Ik zal er me wel,{\textquoteright}}{voor wachten een roman te}{vertellen aan \`u}\\

\haiku{Aan z'n steenen nymphen;}{en godessen in het park}{neemt geen mensch aanstoot}\\

\haiku{Ik weet zoo weinig.}{van u. Misschien bent u niet}{als de anderen}\\

\haiku{Maar v\'o\'or Anneke,.}{aan haar portie begon schoot}{hem iets te binnen}\\

\haiku{En hoe anders deed:}{hetzelfde liedje in de}{variatie}\\

\haiku{Zij kwamen juist uit,.}{de Zondagsschool thuis met erg}{brave gezichtjes}\\

\haiku{Wij konden blijkbaar.}{den juisten toon tegenover}{elkaar niet vinden}\\

\haiku{Ik kan er niet af,?}{ik kan toch de partij niet}{laten mislukken}\\

\haiku{Maar nu achteraf.}{lijkt het mij een vrij banaal}{buitenpartijtje}\\

\haiku{Misschien onder het;}{verzinnen of uitvoeren}{van een nieuwe streek}\\

\haiku{Ik had moeite om,.}{niet te laten merken dat}{ik dat alles wist}\\

\haiku{Er is niemand die{\textquoteright},,}{van mij houdt dacht zij dan en}{van dat treuren werd}\\

\haiku{Maak zulk een mantel,,.}{en als hij eenmaal klaar is}{zal hij wel komen}\\

\haiku{Maar de prins kwam niet,,.}{en als de nacht viel ging zij}{eenzaam weer huiswaarts}\\

\haiku{Weer wees hij met zijn:}{zweep wijdheen in den ronde}{en weer richtte hij}\\

\haiku{{\textquoteleft}'t Hielp me altijd{\textquoteright},, ',.}{erg voegde ze God beter}{t er nog aan toe}\\

\haiku{Alleen, als we doen,,.}{of die kunst iets bijzonders}{is zijn we ijdel}\\

\haiku{Vertrouw op mij, dan.}{zal ik zoo sterk w\'orden als}{jij denkt dat ik ben}\\

\haiku{Ik zal je alles,}{alles geven waarom je}{moe en ontberend}\\

\haiku{Och Joost, eigenlijk.}{maakte die laatste zin van}{je brief mij bedroefd}\\

\haiku{wat ik je schrijf lijkt.}{me dikwijls zoo armelijk}{bij het overlezen}\\

\haiku{hij zag afgebeeld.}{alle foltertuigen van}{des Heilands passie}\\

\haiku{*~        {\textquoteleft}Mein Freund, das grad'.}{ist Dichter-Werk Dass er sein}{Tr\"aumen deut und merk}\\

\haiku{Glaub mir, des Menschen.}{warster Wahn Wird ihm im}{Traume aufgetan}\\

\haiku{wat mijn optreden,,.}{vrees ik dikwijls vrij idioot}{en hinderlijk maakt}\\

\haiku{Strauss, Futurisme,.}{snobbisme en tientallen}{andere ismen}\\

\haiku{Hij zweefde boven.}{gletschers die nog niemand had}{kunnen beklimmen}\\

\haiku{Het benauwt me zoo,,.}{als ik voel dat iemand iets}{van me verwacht}\\

\haiku{LIESBETH, Liesbeth, neen,,!}{dat is niet te dulden dat}{is direct idioot}\\

\haiku{Je vond mijn overkomst,,}{benauwend ik zag het aan}{je verschrikt gezicht}\\

\haiku{{\textquoteleft}Ik kom terug, na, '{\textquoteright}.}{mijn lezingen en dan blijf}{ikn paar weken}\\

\haiku{je moet je rust en,.}{evenwicht weer herkrijgen en}{ik kom je helpen}\\

\haiku{Het is waar, wij zijn,.}{beiden even eenzaam elk op}{ons perronnetje}\\

\haiku{Maar ik kan het niet,,.}{helpen Joost ik voel niet dat}{ik naar jou toe moet}\\

\haiku{Het lijkt heel hard, dat,;}{zoo koel te zeggen tegen}{jouw verlangen in}\\

\haiku{Teleurgesteld was, {\textquoteleft}{\textquoteright}.}{je omdat ik begeerde}{als een ander man}\\

\haiku{Maar gisterennacht}{sloop je naast mijn bed en je}{nam mijn hoofd tusschen}\\

\haiku{Ik heb je voordracht,;}{hier bijgewoond verborgen}{achter een zuiltje}\\

\haiku{Ik heb er uren mee,!}{in mijn handen geloopen}{ik heb het gestreeld}\\

\haiku{Misschien, waarschijnlijk;}{is het dit besef dat mij}{zoo gelaten maakt}\\

\haiku{Maar als ik ooit mijn,,?}{macht over hem verlies mijn arm}{duifje wat  dan}\\

\haiku{Ze prakkiseert over, ' '.}{iets ik hoop maar niet datt}{omn jongen is}\\

\haiku{Ik vroeg het oudste.}{meisje of Liesje hun}{zusje was geweest}\\

\haiku{Wij keken alle,;}{vier zwijgend naar het lieve}{teere gezichtje}\\

\haiku{Al zing ik haar geen,.}{welkom meer Mijn grafje tooit}{zij telken keer}\\

\haiku{- ~ O schrei niet wijl,?}{ik heen moest gaan Wat kwaad heeft}{mij de Dood gedaan}\\

\haiku{Van al wat schoonst en.}{lieflijkst is Ben ik nu ziel}{en heugenis}\\

\haiku{hier rust ik aan uw.}{heilig hart even vertrouwd als}{waar ook ter wereld}\\

\haiku{Liesje, liefste, veel;}{tijd zal ik wel nooit hebben}{om je te schrijven}\\

\haiku{In den Courrier,,.}{tegelijk met mijn brief vind}{ik jouw novelle}\\

\haiku{Onmogelijk, riep,.}{hij zoo iets kan alleen een}{vrije Duitscher zeggen}\\

\haiku{je denkt zelfs niet dat,;}{ik gek ben je denkt in het}{geheel niet aan me}\\

\haiku{hoe gruwelijker,.}{zij zijn hoe gedwee\"er zij}{worden gediend}\\

\haiku{Denk aan de pers, de,.}{koningin der aarde de}{vrouwe Babylons}\\

\haiku{Geen dictatuur van,}{vechthelden en krachtpatsers}{zelfs al hadden zij}\\

\haiku{Weer dwaalde ik in ':}{het bosch ens avonds schreef ik}{deze versjes}\\

\haiku{Ik vond een weide,;}{als nooit betre\^en Konijnen}{talloos haastten heen}\\

\haiku{En bij elk wonder,:}{dat ik zag Luidde in mijn}{hart de stille lach}\\

\haiku{Je leek op Liesje,,.}{het gestorven meisje maar}{je was het toch zelf}\\

\haiku{Den eersten keer toen.}{hem werd medegedeeld dat}{hij blind zou blijven}\\

\haiku{denk eens aan, zuster,?}{hoevelen valt zooiets te}{beurt in honderd jaar}\\

\haiku{om als een bende}{amokmakers te vuur en te}{zwaard te verwoesten}\\

\section{Sonja Surink}

\subsection{Uit: De man met de vele gezichten}

\haiku{Thea stelt ook thans haar.}{gaven nog in dienst van het}{algemeen belang}\\

\haiku{Heb je d'r al wat,?}{van gehoord dat er weer eens}{eentje vermoord is}\\

\haiku{Tegelijkertijd;}{echter was zij nieuwsgierig}{naar wat hij zou doen}\\

\haiku{Aan deze voorzorg.}{zou ze weldra haar leven}{te danken hebben}\\

\haiku{Maar wat ik daar mee,,.}{te maken heb waarlijk het}{is me een raadsel}\\

\haiku{U kan er niet over,.}{oordeelen zoolang u mijn}{positie niet kent}\\

\haiku{We gaven het den.}{wat weidscheren naam van een}{Geheim Genootschap}\\

\haiku{En dat binnen twee,.}{weken waarvan er nu al}{anderhalf om zijn}\\

\haiku{{\textquoteright} {\textquoteleft}In hun oogen zal die,{\textquoteright}.}{zoo onmogelijk niet zijn}{luidde het antwoord}\\

\haiku{{\textquoteright} vroeg Thea, zonder zich.}{iets om zijn ontsteltenis}{te bekommeren}\\

\haiku{{\textquoteright} {\textquoteleft}Om hem te laten,?}{weten dat ik met u over}{hem heb gesproken}\\

\haiku{Nee, om hem te doen,,{\textquoteright}.}{gelooven dat u niet met mij}{sprak antwoordde ze}\\

\haiku{{\textquoteright} Zijn stem klonk rauw en.}{met onverholen dreiging}{een beetje hoonend}\\

\haiku{Sloterdijk eenmaal,.}{voorbij zou zelfs een schot niet}{meer worden gehoord}\\

\haiku{Ze peinsde zoo diep,.}{dat ze zich de tanden in}{de onderlip beet}\\

\haiku{waarom anders had?}{die kerel niet meteen ook}{h\'a\'ar doodgeschoten}\\

\haiku{Aan mij, ja, maar we.}{bespreken de kwestie thans}{in vergadering}\\

\haiku{ik begreep, dat de,}{donkere gestalte in}{de lucht een vrouw was}\\

\haiku{ik verwachtte meer,.}{succes in Amsterdam et}{voil\`a daar zijn we dan}\\

\haiku{Aan den anderen.}{muur is geen venster en voor}{dat eenige stond ik}\\

\haiku{{\textquoteright} {\textquoteleft}Dat is handeling,,{\textquoteright}.}{waarbij u zelf betrokken}{bent merkte Thea op}\\

\haiku{{\textquoteright} {\textquoteleft}U wilde dus nu?}{nog uw onderzoek naar du}{Maurier beginnen}\\

\haiku{Het was bij elven.}{toen zij den sleutel stak in}{haar eigen huisdeur}\\

\haiku{Zeker, dacht Thea, maar '.}{in de stad is het ook niet}{alless winters}\\

\haiku{Gedateerd op 15,.}{November was het ding nog}{geen twee weken oud}\\

\haiku{Jimmy had haar een,.}{telefoonnummer doch geen}{adres opgegeven}\\

\haiku{Uw toestand is niet,.}{te benijden dat stem ik}{natuurlijk grif toe}\\

\haiku{{\textquoteright} {\textquoteleft}Zeker, maar ik wist,.}{niet dat ik mijn leven aan}{hem te danken had}\\

\haiku{Ze begon zich moe,.}{te voelen te verlangen}{naar huis en naar rust}\\

\haiku{een als het ware,.}{idyllische plek om iemand}{van kant te maken}\\

\haiku{Veel tijd had ze er,.}{niet voor noodig want er stond maar}{weinig geschreven}\\

\haiku{Want met nummer vier.}{viel nog minder te spotten}{dan met diens meester}\\

\haiku{Om half acht wilde,.}{ze de deur uitgaan en dus}{moest ze voortmaken}\\

\haiku{Maar komt u liever,}{vergeefs aan de deur mij is}{het natuurlijk best.}\\

\haiku{De vrouw aarzelde,.}{een oogenblik nam haar van}{hoofd tot voeten op}\\

\haiku{Wie zou ooit zonder,,?}{begeleiding als vreemde}{hier boven komen}\\

\haiku{Nu voelde ze aan,.}{den houten deurknop die zich}{niet om liet draaien}\\

\haiku{Dan sprak de stem zoo,.}{vlug dat Thea de woorden niet}{kon onderscheiden}\\

\haiku{Ik vermoedde, dat;}{ze jou onder den duim had}{weten te krijgen}\\

\haiku{het geval is, in.}{een  stemming van speelschen}{overmoed te komen}\\

\haiku{Ik heb hier te doen.}{en kan u niet gebruiken}{op het oogenblik}\\

\haiku{Daarheen zou de Waard,.}{gegaan zijn als je niet op}{hem geschoten had}\\

\haiku{{\textquoteright} {\textquoteleft}Voor vanavond zal ik,{\textquoteright},.}{hem zeker hebben zei ze}{ietwat voorbarig}\\

\haiku{{\textquoteleft}Maar wel kan ik u,.}{vragen waarom u mij geen}{antwoord geven d\`urft}\\

\haiku{Denk daar niet meer aan,,.}{er staat heel wat anders te}{doen vermoedelijk}\\

\haiku{Als aanstonds die stem?}{aan de telefoon zich weer}{tot h\'a\'ar richten zou}\\

\haiku{Maar ze wist, dat ze,.}{dit keer aan niemand gezegd}{had waar ze heen ging}\\

\haiku{En ik zou het graag,,,}{nog eens doen om te kijken}{hoe je er uitziet}\\

\haiku{Omdat je Charles.}{geen belemmering in den}{weg wilde stellen}\\

\haiku{{\textquoteright} {\textquoteleft}Dat is wat te veel.}{geofferd op het altaar}{van je ijdelheid}\\

\haiku{Daar heb je misschien?}{kort voor zijn dood nog wel een}{staaltje van gezien}\\

\haiku{Je zal in de cel,,}{tijd genoeg krijgen om het}{mij eens te schrijven}\\

\subsection{Uit: Het verdwenen meisje}

\haiku{Ja, natuurlijk, als,....}{ik er eenmaal over praat ga}{ik er ook op af}\\

\haiku{Als jagers in uw,,,.}{dienst met Anna als vergeef}{mij het woord als wild}\\

\haiku{{\textquoteright} {\textquoteleft}Ja, een pas heeft ze,.}{doch natuurlijk droeg ze dien}{niet altijd bij zich}\\

\haiku{Maar dat is hier, het,{\textquoteright},}{was hier verleden Dinsdag}{zei mijnheer Barends}\\

\haiku{terwijl hij meteen.}{in een ander laatje van de}{secretaire keek}\\

\haiku{De vriendin echter.}{zou wellicht om dezen tijd}{thuis te treffen zijn}\\

\haiku{Dan zal ik je nou,,}{alleen laten opdat je}{er aan werken kunt}\\

\haiku{Thea vond het prettig,}{\'e\'en dag per week te kunnen}{uitslapen zoolang}\\

\haiku{{\textquoteright} {\textquoteleft}En nu is er iets,?}{gebeurd waar u mijn raad in}{noodig dacht te hebben}\\

\haiku{Was dit bezoek een,.}{spoor dan zou ze het zich niet}{laten ontglippen}\\

\haiku{Ze had zich overtuigd,,.}{doch ze kon niet absoluut}{zeker zijn wist ze}\\

\haiku{Ze voelde het, toen,.}{iemand achter haar langs liep}{naast haar stoel bleef staan}\\

\haiku{{\textquoteleft}Wilde je zeggen,?}{dat ook Betty een stem in}{het capittel heeft}\\

\haiku{Ze liepen in de,.}{Amstelstraat in de richting}{van het Rembrandtplein}\\

\haiku{Betty kroop vast in,.}{den wagen waarvan de kap}{neergelaten was}\\

\haiku{Ze zou hem leelijk.}{op de teenen trappen voordat}{ze er bij neerviel}\\

\haiku{Ze trachtte af te,;}{leiden wat de anderen}{in hun schild voerden}\\

\haiku{Ze haalde den haan,,.}{nogmaals over en nogmaals vijf}{zes keer na elkaar}\\

\haiku{Als ze de kans krijgt,.}{geeft ze jouw en mijn portret}{aan de politie}\\

\haiku{Hij verwijderde;}{zich van den wagen in de}{richting van Haarlem}\\

\haiku{Ze stond op het punt,,.}{in te stappen toen ze een}{nieuwen inval kreeg}\\

\haiku{Hij  moest, had hij,,.}{Betty verklaard zelf zorgen}{dat het weer thuis kwam}\\

\haiku{{\textquoteright} {\textquoteleft}Ik zeg je toch al,,.}{dat jij het ook mag doen als}{je dat liever is}\\

\haiku{Want het maakt immers,!}{niets meer uit of je mij aan}{den dijk zet of niet}\\

\haiku{Ik heb mijn tijd noodig,.}{om jullie spoedig weer te}{kunnen ontmoeten}\\

\haiku{Enfin, dat zal je.}{zelf op kantoor ook wel eens}{meegemaakt hebben}\\

\haiku{{\textquoteleft}Ik weet alleen, dat,.}{hij geen rust zal hebben voor}{dat ding terecht is}\\

\haiku{Je vader is een,{\textquoteright}.}{reuzenbaas antwoordde ze}{op Betty's vraag}\\

\haiku{Ik heb toch zeker,.}{eerst aan haar gevraagd of zij}{Anna Barends is}\\

\haiku{Ze keek recht in de,.}{oogen van een man dien ze nog}{nimmer gezien had}\\

\haiku{Om zoo te zeggen.}{zat ik in de ijskast en}{jij in de broeikas}\\

\haiku{Ze liet de armen.}{zinken en w\'e\'er leek alles}{haar een hersenschim}\\

\haiku{Zeker, ik h\`eb een,.}{pas als je het met alle}{geweld weten wil}\\

\haiku{hij ging heen en sloot.}{de deur als gewoonlijk met}{sleutel en grendel}\\

\haiku{Het geluid van het.}{overschakelen overstemde}{dat van haar starter}\\

\haiku{Plotseling, na een,.}{paar minuten gedraafd te}{hebben stond ze stil}\\

\haiku{Hij stond stellig op?}{de treeplank en gaf wellicht}{laatste instructies}\\

\haiku{Als hij er niets van,?}{wist waarom maakte hij dan}{geen enkel geluid}\\

\haiku{{\textquoteleft}Maar je mag ze geen.}{meisjes laten ontvoeren}{en banken berooven}\\

\haiku{zie je nou, had ik,.}{dat toen geweten dan had}{ik hem mat gezet}\\

\haiku{{\textquoteright} vroeg Ferdinand, toen.}{hij zijn wagen gekeerd en}{op gang gebracht had}\\

\haiku{Ik heb een sleutel,,}{gehad maar die hebben ze}{me afgenomen}\\

\haiku{Juist, had ze dan zelf,?}{aan die kerels verteld waar}{het te vinden was}\\

\haiku{Dat leek me vreemd, ik,.}{dacht niet aan misdaad ik dacht}{aan een ongeluk}\\

\haiku{Het gaat tenslotte,?}{immers toch alleen om het}{verhaaltje nietwaar}\\

\haiku{Als hij er toe over,.}{te halen is zal het een}{bom duiten kosten}\\

\haiku{We zien elkaar nog.}{wel eens weer en we doen het}{ook nog wel eens over}\\

\haiku{Je bedoelt het goed,,,.}{Katrien maar heusch ik heb}{er geen vrede bij}\\

\haiku{Die gaat met je mee,.}{naar huis kijken of je nog}{wat te drinken hebt}\\

\haiku{{\textquoteright} {\textquoteleft}Vind jij het niet het,?}{eenvoudigst bij jou thuis}{te telefoneeren}\\

\haiku{Dat het een beetje,?}{erg gauw is gegaan dat is}{toch geen overtreding}\\

\haiku{Dan was je misschien,?}{in de buurt toen ik daar met}{de politie was}\\

\haiku{Ik vermoed, dat de.}{politie ze je dan wel}{zal overhandigen}\\

\haiku{{\textquoteright} Ouders maken zich.}{vaak zonder reden over hun}{kinderen bezorgd}\\

\haiku{Ze hadden gedacht,,.}{over twee weken ondertrouw}{over een maand trouwen}\\

\haiku{Vroeger maakten we,:}{dat alles aan huis zelf of}{met  een naaister}\\

\haiku{Ze namen deel aan.}{een groot aantal inbraken}{van den laatsten tijd}\\

\haiku{En een nader adres.}{hebben ze jou natuurlijk}{niet opgegeven}\\

\haiku{Misschien kom ik het,.}{u wel brengen als u dat}{goed vindt tenminste}\\

\haiku{{\textquoteright} {\textquoteleft}En m\'o\'oi gezegd,{\textquoteright} prees,.}{Andries zijn kaarten op een}{bundeltje kloppend}\\

\section{Hieronymus Sweerts}

\subsection{Uit: De tien vermakelijkheden van het huwelijk}

\haiku{Vervrolijk je naast}{de bruiloftsgasten zoveel}{de gelegenheid}\\

\haiku{Maar (och Schipper, breng).}{de Puts ik vrees dat het een}{heel kalf zal worden}\\

\haiku{, zich niet schamen je.}{een onfatsoenlijke som}{geld af te persen}\\

\haiku{Maar mij dunkt, ik zou.}{hier van de Juffrouw wel tot}{de meid vervallen}\\

\haiku{de uitvoering van.}{de plannen zal je immers}{geld genoeg kosten}\\

\haiku{En Juffrouw Perfect,.}{heeft ze beide wel maar die}{zijn juist uitgeleend}\\

\haiku{Want het kind uit het,.}{huis te doen kan Moeder noch}{Vader van het hart}\\

\haiku{{\textquoteleft}Och, ik zal, noch kan,}{niet verlossen tenzij ik}{het af zie snijden}\\

\haiku{Nauwelijks was de.}{man te bed of de vrouw kreeg}{een ander wezen}\\

\haiku{{\textquoteright} {\textquoteleft}Och Hartje,{\textquoteright} zei hij, {\textquoteleft}.}{je weet wat ik om jouw wil}{toegelaten heb}\\

\haiku{En omdat ik al,:}{diverse malen gedreigd}{had zeiden deze}\\

\haiku{Och, hoe zoet is de!}{Huwelijkse Staat boven}{de ongetrouwde}\\

\haiku{Het een lijkt haar te,:}{stijf of te gewoon zodat}{ze bij zichzelf zegt}\\

\haiku{En dit is nog het,,.}{kleinste maar wacht overmorgen}{gaat de Baker weg}\\

\haiku{De vrouw krijgt de schuld.}{van het weinig soepele}{huwelijksleven}\\

\haiku{Of dat werkelijk,.}{zo was zullen we wel nooit}{te weten komen}\\

\haiku{Het ging er vooral.}{om het familiebezit}{veilig te stellen}\\

\section{M.H. Sz\'ekely-Lulofs}

\subsection{Uit: De andere wereld}

\haiku{Wat verlegen en.}{met zichzelf geen raad wetend}{had hij daar gestaan}\\

\haiku{Hij zat op den rand.}{van de haverkist en zij}{stond tegen hem aan}\\

\haiku{Al was hij er maar,:}{loopjongen t\`och had hij toen}{het besef gehad}\\

\haiku{Altijd afgesnauwd,,!}{altijd weggestuurd altijd}{in een hoek gedouwd}\\

\haiku{{\textquoteleft}Zie je, hoe je elk,...?}{beest kunt winnen als je}{er maar goed voor bent}\\

\haiku{Een oogenblik had,.}{hij niet geweten wat zij}{daarmee bedoelde}\\

\haiku{Ze schrok, dat hij dit.}{spelletje merkte en keek}{dadelijk voor zich}\\

\haiku{Ik ga weg!-.}{Drie dagen later was hij}{op de Prinsengracht}\\

\haiku{Veel gezegd had die,.}{niet alleen hem de hand op}{den schouder gelegd}\\

\haiku{Eigenlijk had hij!}{dat tientje van Lien nog best}{kunnen gebruiken}\\

\haiku{En dan had moeder.}{haar zakdoek weer voor haar mond}{gepropt en gehuild}\\

\haiku{Een bad, zooals hij nog......}{nooit gezien had alleen maar}{in een \'etalage}\\

\haiku{het was, dat hij geen.}{plaats had kunnen krijgen op}{een Hollandsche boot}\\

\haiku{De envelop met.}{overgeschoten plakadressen}{kwam in zijn handen}\\

\haiku{Naast Van der Steeg stond.}{plotseling de silhouet}{van de verschansing}\\

\haiku{Het anker werd al.}{opgehaald en tegelijk}{loeide de stoomfluit}\\

\haiku{Een gevoel, dat met.}{geen enkelen weemoed om}{vertrek gemengd was}\\

\haiku{hem zoo duidelijk.}{de ongewenschtheid daarvan}{had laten voelen}\\

\haiku{Hij hield het hoofd wat.}{achterover en keek als in}{peinzing voor zich uit}\\

\haiku{{\textquoteleft}die zou Onkel von...}{M\"uhlock heel dankbaar zijn voor}{zijn goede zorgen}\\

\haiku{Zijn vrouw staat niet in...?}{de passagierslijst zou hij}{weer gescheiden zijn}\\

\haiku{Van der Steeg ging op:}{het eene eind zitten en zei}{wat vriendelijker}\\

\haiku{Hoorde hem een vloek.}{zeggen en direct daarop}{een deuntje fluiten}\\

\haiku{Zijn gedruktheid sloeg,.}{over in melancholie in}{een soort vaag heimwee}\\

\haiku{En dan doortrok hem:}{ook al de uitwaseming}{van dit nieuwe land}\\

\haiku{Hij keek wat verbaasd,,.}{op toen naast hem de loome}{klanklooze stem ophield}\\

\haiku{Hij wist zich ineens.}{staan in het volle licht en}{de volle aandacht}\\

\haiku{De Chinees nam de.}{boomen op en zette een}{sukkeldrafje in}\\

\haiku{De Chinees houdt zijn.}{open handpalm bij de lantaarn}{en ziet hem stom aan}\\

\haiku{En toch was de man.}{daar boven bijna tot aan}{het dak geklommen}\\

\haiku{Hij drukte zich wat,.}{vaster in de kussens die}{zijn rug omvlijden}\\

\haiku{Aan beide zijden,,.}{van de spoorbaan rees dat op}{donker duistergroen}\\

\haiku{Dezelfde wee\"e lucht.}{als van bij het station}{omgolfde hem weer}\\

\haiku{Hij had zijn stoel wat.}{achterover gewipt en stak}{een  sigaar op}\\

\haiku{Voorloopig hebt,.}{u zelf niets te denken te}{zeggen of te doen}\\

\haiku{{\textquoteleft}Ta-b\'eeh toe-w\'a\'an......}{Toewan wordt beleefd verzocht}{binnen te komen}\\

\haiku{{\textquoteleft}O, verduiveld, dat, ',!}{is waar ook gas even een}{eindje op zij Pot}\\

\haiku{{\textquoteleft}Daar moest eigenlijk,{\textquoteright},, {\textquoteleft}}{een lamp aan hangen begrijp}{je verklaarde hij}\\

\haiku{Je gaat denken, dat,:}{er  maar \'e\'en geluk maar}{\'e\'en doel voor je is}\\

\haiku{Anders zeg je 't, '.}{maar gerust hoor dan leg ik}{t je nog eens uit}\\

\haiku{Ze wil nooit gelooven,.}{dat je alle vuil van je}{schoenen kunt schrappen}\\

\haiku{Met een zwaai draaiden,.}{ze den tuin binnen kwamen}{met een ruk tot staan}\\

\haiku{Pieter keek even op,.}{in de bolle blauwe oogen}{met hun zachten blik}\\

\haiku{Verschool zich achter:}{een gewild jovialen}{lach en zei iets van}\\

\haiku{- Ga jij maar aan je,,}{eigen werk Blom ik neem dat}{sinkeh wel verder}\\

\haiku{- Met dien knul hebben '!}{ze me nou godbetert}{in \'e\'en hut gedouwd}\\

\haiku{{\textquoteright} En nu waren er,...}{niet alleen geen kerken maar}{ook geen Zondagen}\\

\haiku{En dan was er een.}{vaag verlangen in hem naar}{zoo'n Zondagsstemming}\\

\haiku{Bij de eerste de...}{beste gelegenheid moest}{hij dien zien te loozen}\\

\haiku{- En Pasman voelde:}{deze gedachte en werd}{nog dienstvaardiger}\\

\haiku{hij komt tot in de...}{tuin en daarom heeft Pasman}{niet graag hier gewaakt}\\

\haiku{Hij vluchtte naar bed,.}{binnen de bescherming van}{het tullen gordijn}\\

\haiku{In rijen zaten,.}{zij daar gehurkt bij elke}{ploeg stond de mandoer}\\

\haiku{Hij haalt uit, geeft den.}{koelie een onverwachten}{stomp in het gezicht}\\

\haiku{Dat laffe, zwarte!}{slaventuig durft immers toch}{niets terug te doen}\\

\haiku{de schuifelende,.}{haast sluipende stappen van}{Pasman's bloote voeten}\\

\haiku{Overal, waar ze liep,,,;}{hing haar zoetig parfum een}{beetje te zoet dof}\\

\haiku{Hij redde zich in,:}{het besef dat Van der Steeg}{hem bijgebracht had}\\

\haiku{- Dat was ook beter.}{en jonge toewans dronken}{meestal weinig}\\

\haiku{Ze maakte hem te...}{schande met haar brutale}{mond en haar tinka's}\\

\haiku{Alle bitterheid,.}{van vroeger kwam naar boven}{kwam wrang in zijn mond}\\

\haiku{{\textquoteleft}Ze zeggen ...dat ze.}{het houdt met de chauffeur van}{de toewan besar}\\

\haiku{Pasman moet hem maar{\textquoteright},.}{dood maken voegde ze er}{dan vol afkeer bij}\\

\haiku{Pasman moet er een,...}{hok voor timmeren ik wil}{hem levend houden}\\

\haiku{In de donkere,,.}{natte stilte kiemde zij}{op deze weemoed}\\

\haiku{Duizend dreigingen.}{hoonlachten in het gesnerp}{van een cicade}\\

\haiku{de rooie sufferd wel}{te voorschijn als jullie baas}{en meerdere.-}\\

\haiku{- Voordat Asminah,:}{er was had hij nog wel eens}{verteld over Pasman}\\

\haiku{En voor Asminah,.}{waren Pieter's ouders broer}{en zuster Blanken}\\

\haiku{Blanken, ergens in,.}{een ver land dat ze zich niet}{eens voorstellen kon}\\

\haiku{Als ik altijd op,.}{mijn mooist ben dan ziet toewan}{nooit meer het verschil}\\

\haiku{Had Asminah niet?}{iets gezegd van nog een kast}{in de slaapkamer}\\

\haiku{Ze wisselde voor,.}{het laatst de knoopen in zijn}{schoone witte jas}\\

\haiku{In die halve maand.}{zouden ze het huis koopen}{en zich installeeren}\\

\haiku{En die lamstraal van...... {\textquoteleft}}{een Idris wat of Minah toch}{met dien vent opheeft}\\

\haiku{Had heel wat aan dek... -.}{moeten slapen Plotseling}{werd hij weer somber}\\

\haiku{En die kletspraatjes...,...}{over Dinah God mag weten}{wat hij binnen kreeg}\\

\haiku{Toch drong zich door zijn.}{wrevel heen een poging tot}{rechtvaardig blijven}\\

\haiku{{\textquoteleft}En weet je, waarom?}{ik je nou toch de heele}{maand loon uitbetaal}\\

\haiku{Ik zal ook altijd:}{van jou houden en ik vraag}{maar \'e\'en ding van je}\\

\haiku{{\textquoteright} Het juffrouwtje, dat,:}{dien dag haar slechten dag had}{snauwde venijnig}\\

\haiku{Hij ging naar binnen,.}{ging op de bank liggen en}{vouwde de krant open}\\

\haiku{Och{\textquoteright}, zei hij, het van, {\textquoteleft}?}{zich afschuivendben je daar}{nou wel zeker van}\\

\haiku{Hij dacht alleen aan,,.}{dat vreemde dat verre dat}{onwezenlijke}\\

\haiku{Ze ging hem voor en.}{toen stond hij naast haar in de}{leege logeerkamer}\\

\haiku{{\textquoteright} En ze wendde zich,.}{iets om naar een bundeltje}{dat achter haar lag}\\

\haiku{Maar in Pieter was,.}{een vreemde weeke ontroering}{geweest bij dat woord}\\

\haiku{Hij drukte het kind.}{tegen zich aan en streelde}{het over zijn kopje}\\

\haiku{Doeltje was heer en.}{meester in den tuin en in}{de bijgebouwen}\\

\haiku{doodstil, haar blik strak.}{starend op het slapende}{kindergezichtje}\\

\haiku{Ging hij dit kind hier?...}{achterlaten om terug}{te gaan naar Holland}\\

\haiku{Het was het oude,.}{gebrek aan zelfvertrouwen}{de oude ziekte}\\

\haiku{Ze voelden beiden,.}{de maatschappelijke kloof}{die tusschen hen lag}\\

\haiku{{\textquoteright} zei hij opgewekt, {\textquoteleft},!}{de meeste lui trouwen als}{ze met verlof zijn}\\

\haiku{{\textquoteright} ~ Dien avond, toen hij,.}{de leeszaal binnenkwam bleef}{hij plotseling staan}\\

\haiku{Wandversiering was.}{niet inbegrepen geweest}{bij den inboedel}\\

\haiku{Zelfs Betty... Ja, wat!}{w\'as ze dadelijk aardig}{tegen hem geweest}\\

\haiku{Hij gaf zijn hoed en.}{stok aan den toegeschoten}{boy en ging zitten}\\

\haiku{Ze zoende hem op,, {\textquoteleft}...}{zijn mond met ondeugende}{speelschheiden z\'o\'o}\\

\haiku{- Samen slapen is, -.}{ouderwetsch en proza{\"\i}sch}{had Betty gezegd}\\

\haiku{En we weten het.}{alle drie en toch zullen}{we het loochenen}\\

\haiku{{\textquoteleft}Ik ken een beetje{\textquoteright},, {\textquoteleft}}{Fransch zei Bettynog van mijn}{tijd bij madame}\\

\haiku{En het huis is ook,...}{oud en leelijk maar ik ben}{er toch geboren}\\

\haiku{{\textquoteright} Betty stond naast de.}{mannen en hoorde Van Beek}{verslag uitbrengen}\\

\haiku{Geen van beiden zei,.}{dat en toch wisten beiden}{wat de ander dacht}\\

\haiku{{\textquoteleft}Als u d\'at voor me...{\textquoteright}.}{doen wou Een dankbare blik}{schoot uit Van Beek's oogen}\\

\haiku{Die komedie, of....}{wat het dan ook was daar moest}{een eind aan komen}\\

\haiku{Verleidelijk werd,.}{de blanke boog van haar buik}{vlak voor zijn gezicht}\\

\haiku{De boy keek haar na,,.}{keek even naar haar hoed dien ze}{nog altijd op had}\\

\haiku{{\textquoteright} zei Betty peinzend, {\textquoteleft},,...{\textquoteright}}{alles wat ni\'et mag dat}{is pas \'echt prettig}\\

\haiku{dit leven, dat het,.}{bestaan was voor een ander}{ras een ander volk}\\

\haiku{Hij worstelde met,.}{de knoopjes van zijn overhemd}{met zijn boord en das}\\

\haiku{provinciaal... kan,,?}{ik helpen dat die rotvent}{niet weet waar hij staat}\\

\haiku{- Wat zouden jullie,...}{wel zeggen van Doeltje als}{jullie dat wisten}\\

\haiku{En pijnlijk voelde,:}{hij daarin dat er toch een}{kloof gebleven was}\\

\haiku{Pieter's verhaal was,.}{afgebroken hij vond zoo}{gauw het vervolg niet}\\

\haiku{Hij herinnerde,.}{zich ineens dat hij die doos}{bonbons had gekocht}\\

\haiku{Nou op je geluk,...{\textquoteright} {\textquoteleft}...}{en welkom thuis welkom in}{HollandMet je vrouw}\\

\haiku{Dat heb je nog niet,...{\textquoteright}.}{eens verteld Pieter Pieter}{deed onverschillig}\\

\haiku{van stoere werkkracht,,.}{van stroeve berusting van}{harde soberheid}\\

\haiku{Je ouders en je,!}{broer een troep armoedzaaiers}{in een achterbuurt}\\

\haiku{{\textquoteright} benijdde Betty, {\textquoteleft},{\textquoteright}.}{hemw\'at een geluk om d\'at}{te kunnen zeggen}\\

\haiku{Hij is niet heelem\'a\'al,,.}{zoo'n lammeling weet je als}{de menschen denken}\\

\haiku{- Maak er nou liever,....}{een eind aan Betty laat het}{niet te lang duren}\\

\haiku{En ze was van hem,,.}{gaan houden met een vreemde}{warme sympathie}\\

\haiku{Veenstra sloot zich bij hen,.}{aan had altijd het een of}{ander leuke plan}\\

\haiku{Het was, of ze zich,.}{niet realiseerde dat}{dit hun toekomst werd}\\

\haiku{{\textquoteright} Waarom lachten nou,.}{de anderen bij elke}{vraag van Willemse}\\

\haiku{Waarom moest hij zich,?}{dat op zijn hals halen zoo'n}{beroerd kwartiertje}\\

\haiku{Dan bespraken ze.}{de krantenberichten en}{nieuwtjes uit Indi\"e}\\

\haiku{En altijd was er,.}{boven dat werk haar stille}{zwijgende glimlach}\\

\haiku{Hij verdedigde,,.}{Betty met een woord een lach}{een schouderophaal}\\

\haiku{Het was hem, of een.}{afgrond plotseling aan zijn}{voeten opengaapte}\\

\haiku{Hij zat daar met het.}{bonnetje in zijn hand en}{keek naar het plakkaat}\\

\haiku{Van Brinkman... - Brinkman... -.}{heeft gezegd Hij scheurde de}{enveloppe open}\\

\haiku{Hij voelde, tusschen,,.}{zijn teenen het nat dat door zijn}{schoenzolen heen trok}\\

\haiku{Op hooge, wankele.}{hakken drentelden ze over}{het slechte trottoir}\\

\haiku{De vrouw morrelde.}{dan aan een sleutelgat en}{er ging een deur open}\\

\haiku{En hij zag Betty,,.}{staan voor den spiegel met de}{blauwzijden nachtpon}\\

\haiku{Af en toe rukte,.}{een windstoot aan de ruiten}{die even rammelden}\\

\haiku{nou hoefde niemand,...}{van haar te zeggen dat ze}{een slechte vrouw was}\\

\haiku{Hij trok het dek over.}{zich heen en op hetzelfde}{oogenblik sliep hij}\\

\haiku{Er was een afgrond,,.}{een kloof die nergens  meer}{te overbruggen was}\\

\haiku{Zoo landde hij in,.}{de kleine heete haven}{en reed naar de stad}\\

\haiku{{\textquoteleft}En... zal toewan niet,...{\textquoteright} {\textquoteleft},.}{erg all\'e\'en zijn zonder de}{mimNee Asminah}\\

\haiku{Onder het stof en.}{vuil was haast niets meer te zien}{van de teekening}\\

\haiku{En dan liepen ze,,.}{den langen weg die vroeger}{zoo kort was terug}\\

\haiku{Even liepen ze hard,,.}{maar de bui haalde hen in}{kletste op hen neer}\\

\haiku{Er is niemand in,,.}{het dorp die zijn naam kent maar}{dat is ook niet noodig}\\

\haiku{In de bocht lag het,.}{strand het brandend heete zand}{glanzend in het licht}\\

\haiku{Meubels waren er,,:}{niet m\'e\'er dan Asminah had}{toen hij terug kwam}\\

\haiku{En dan dwaalden zijn,.}{oogen over het papier dat ze}{weggeworpen had}\\

\haiku{Elken dag had hij.}{een sarong en een baadje}{daar weggenomen}\\

\haiku{Van Beek was ook een... -.}{goeie vriend Hij liet de krant van}{zijn knie\"en glijden}\\

\haiku{{\textquoteleft}Ik zal eens kijken.}{in de goederenloods van}{de onderneming}\\

\haiku{Hij hoorde het heen.}{en weer geloop en gesjouw}{met de manden visch}\\

\haiku{Er was ergens een,,...}{troost die op hem neerdaalde}{over hem heen stulpte}\\

\subsection{Uit: Het laatste bedrijf}

\haiku{Een kleurige en,,.}{geurige oud vertrouwde}{concrete basis}\\

\haiku{{\textquoteleft}Kunt u me misschien,?}{ook zeggen hoe lang de zaak}{van Ger\"o leeg staat}\\

\haiku{{\textquoteright} (Natuurlijk niet, - dacht).}{hij en veranderde zijn}{vraag onmiddellijk}\\

\haiku{jij zat daar maar en,....}{schilderde maar en je dacht}{het komt wel terecht}\\

\haiku{{\textquoteright} George hield de,.}{foto dichter bij het raam}{het werd al  avond}\\

\haiku{{\textquoteright} George trok met,.}{zijn schouders stak zijn handen}{in zijn broekzakken}\\

\haiku{Het werd drukkend stil,.}{in de kamer die niet aan}{den gevelkant lag}\\

\haiku{Je kunt niet eens zien,...{\textquoteright} {\textquoteleft}?}{dat we broersWaarom is hij}{teruggekomen}\\

\haiku{Dat je zo\'omaar kunt?}{overgaan van de eene branche}{in de andere}\\

\haiku{Ik heb nog nooit zoo'n!}{onpractisch en angstvallig}{mensch gezien als jij}\\

\haiku{Ze zweeg dan ook en,.}{samen stonden ze voor het}{raam in donker}\\

\haiku{Toen schikte ze met:}{luchtige gebaartjes nog}{iets aan haar kapsel}\\

\haiku{Zijn vrouw zat in een.}{klein kamertje aan tafel}{de krant te lezen}\\

\haiku{Ferri ontving hem,,, - {\textquoteleft}!}{hartelijk maar een beetje}{te druk nerveusAh}\\

\haiku{Toen hij nog jong was,.}{nog geloofde aan de Groote}{Kunst en aan zichzelf}\\

\haiku{- Ferri heeft gelijk,,,.}{jong is ze niet meer ze is}{rijp maar nog bloeiend}\\

\haiku{{\textquoteright} Ze verdween en kwam.}{terug en hielp Bella bij}{het tafeldekken}\\

\haiku{En dat niet als een,,.}{leege troostelooze weemoed maar}{als realiteit}\\

\haiku{- Ik ben toch thuis.... - dacht, -,....}{hij het was toch goed dat ik}{thuisgekomen ben}\\

\haiku{Hij liep alleen over,,.}{straat zijn overjas los zijn hoed}{juist i\'ets te scheef op}\\

\haiku{Hij had er een glas.}{wijn bovenop gedronken}{en verder gedanst}\\

\haiku{Een man voor wien het, - -.}{geld om het uit te geven}{al niet meer telde}\\

\haiku{George liep nu.}{door een plantsoentje en hij}{dacht aan zijn moeder}\\

\haiku{Hij was al niet meer.}{wat zijn vader was geweest}{en zijn grootvader}\\

\haiku{In dezen eenen nacht.}{had de wereld een ander}{aanschijn gekregen}\\

\haiku{hij was precies zo\'o,.}{ook zoo zenuwachtig en}{geschrokken en bang}\\

\haiku{Iets uit Parijs, iets.}{bijzonders in onderwerp}{of behandeling}\\

\haiku{Jij hoeft je over niets,.}{zorgen te maken je met}{niets te bemoeien}\\

\haiku{Toen drukte ze hem.}{nog vaster tegen zich aan}{en zoende zijn mond}\\

\haiku{Ineens herdacht hij,,....}{den avond van gisteren die}{kamer die menschen}\\

\haiku{{\textquoteright} {\textquoteleft}Ik moet toch wat geld....}{hebben om de nieuwe zaak}{mee te beginnen}\\

\haiku{Of, als je wilt, kun....}{je ook direct komen als}{je aangekleed bent}\\

\haiku{Ook daarbij waren.}{de woorden oplichter en}{zwendel gevallen}\\

\haiku{Ze stuurt elke maand,....}{bijna haar heele loon naar}{haar ouders Bella}\\

\haiku{Ferri luisterde.}{naar dat alles en stemde}{toe zonder een woord}\\

\haiku{Een zwak licht, dat nooit, -, -.}{meer hij wist het heelemaal}{helder zou worden}\\

\haiku{Hij was dankbaar voor,.}{dit schijnsel hij was zoo lang}{in donker geweest}\\

\haiku{Waren ze binnen,}{een half jaar niet betaald dan}{zou de winkelier}\\

\haiku{Dat suffe kind was!}{nog te stom om behoorlijk}{te telefoneeren}\\

\haiku{Ze liepen naar het,.}{restaurant waar ze Bella}{zouden ontmoeten}\\

\haiku{misschien maak je af,....}{en toe iets iets h\'e\'el moois en}{dat verkoop je niet}\\

\haiku{George keek haar.... - {\textquoteleft}}{argwanend aan Hoe wist ze}{al die dingen?-}\\

\haiku{- George is toch,.}{een heel wat knapper kerel}{dan Ferri dacht ze}\\

\haiku{Hakkelend en met:}{neergeslagen oogen begon}{hij aan een uitleg}\\

\haiku{- Iemand vond Susanne,,.}{te naakt maar een ander zei}{dat dit echt Fransch was}\\

\haiku{het was zijn tijd om.}{op te treden in Bella's}{nieuwen kunsthandel}\\

\haiku{Dan zou ik het eerst....}{jaren op de bank moeten}{zetten en god weet}\\

\haiku{Een goed schilderij,!}{dat kunstwaarde heeft is even}{goed als een effect}\\

\haiku{Gods verwarmend licht '!....}{in de duisternissen van}{s menschen jammer}\\

\haiku{En dan gaan we voor.}{een kop koffie en een glas}{fijne wijn naar Ritz}\\

\haiku{Ze liet haar groote oogen,.}{op hem rusten klaar om zijn}{blik op te vangen}\\

\haiku{Ze bleef zitten, zooals,,.}{ze zat op haar knie\"en met}{haar beenen onder zich}\\

\haiku{- Ze wiegde zich op....}{deze voorstelling van het}{eindelooze geluk}\\

\haiku{Ze wendde langzaam,.}{haar warme bruine oogen naar}{hem en glimlachte}\\

\haiku{wij zijn alledrie,,.}{op onze eigen manier}{schipbreukelingen}\\

\haiku{Ze wiegde zacht heen.}{en weer en wiegde hem mee}{in haar omarming}\\

\haiku{Kunst is wel mooi en,.}{verheven maar leven moet}{je tenslotte ook}\\

\haiku{Ze keken beiden.}{verschrikt naar de dichte deur}{en toen naar elkaar}\\

\haiku{Hij keek maar even om,,.}{over zijn schouder toen groef hij}{weer in zijn jaszak}\\

\haiku{{\textquoteright} {\textquoteleft}Niemand....{\textquoteright} George,...}{at zwijgend hij kon niet zoo}{gewoon meepraten}\\

\haiku{Kom dan tegen een,.}{uur of vier bij me misschien}{is er wat te doen}\\

\haiku{Een groot gevoel van.}{kameraadschap tegenover}{haar vervulde hem}\\

\haiku{En als je ergens,,.}{geluk ziet dan moet je het}{maar grijpen Bella}\\

\haiku{En dan te denken,!}{dat zoo'n jongen misschien nog}{een moeder heeft o\'ok}\\

\haiku{Hij zag hem al van.}{den hoek en zijn oogen knepen}{zich turend samen}\\

\haiku{Hij keek dreigend rond,,.}{en haatte de menschen de}{wereld het leven}\\

\haiku{Ik zal u nu even,.}{uitleggen wat ik bedoeld}{heb als onderwerp}\\

\haiku{{\textquoteright} riep Bella uit, {\textquoteleft}als!}{je een artiest maar als een}{artiest behandelt}\\

\haiku{Je zult zien, je maakt,.}{er iets moois van en ik zorg}{wel dat hij het neemt}\\

\haiku{{\textquoteright} vroeg hij benepen,.}{met een schuwen blik naar zijn}{pas begonnen doek}\\

\haiku{Maar een mensch schiet zich,!}{niet zoo gemakkelijk voor}{zijn kop George}\\

\haiku{Ik zou niets liever,.}{willen dan jou tevreden}{stellen George}\\

\haiku{Terwijl hij op den,.}{concierge wachtte nam hij}{de omgeving op}\\

\haiku{- Ze wonen te duur, -,.}{stelde hij vast denkend aan}{den leegen kunsthandel}\\

\haiku{Enfin dat is nu,,,!}{eenmaal zoo bij ons h\`e nu}{eens vloed dan weer eb}\\

\haiku{{\textquoteleft}Ik zou alleen eerst,.}{even willen weten wat dat}{te beteekenen heeft}\\

\haiku{{\textquoteright} George boog zijn:}{hoofd naar de lucifervlam}{tusschen zijn handen}\\

\haiku{Je vroeg me, wat ik,,,....}{bedoelde met jouw te groote}{handigheid nou kijk}\\

\haiku{Maar ik heb nog een,.}{andere plicht ook waar jij}{misschien niet van wist}\\

\haiku{Dacht je, dat ik maar?!}{altijd en altijd voort kan}{en van ijzer ben}\\

\haiku{Toe zeg, ga eens even,....}{een eindje op zij je staat}{precies in het licht}\\

\haiku{{\textquoteright} fluisterde ze heet, - {\textquoteleft}!}{en verontwaardigdik doe}{het toch ook voor h\`em}\\

\haiku{Als je dan weer gaat,.}{kijken ligt er alleen nog}{maar een plasje vocht}\\

\haiku{Hij stikte in zijn.}{woorden en haalde diep adem}{om lucht te krijgen}\\

\haiku{- {\textquoteleft}Dacht je, dat alleen?}{jij het monopolie hebt}{op kwaje buien}\\

\haiku{dus niet een van die,....}{dingen die ik alleen maar}{in commissie heb}\\

\haiku{Ze lachte schel en.}{zenuwachtig en gooide}{haar hoofd in den nek}\\

\haiku{De beide paarden,;}{lieten hun kop neerhangen}{hun knie\"en knikken}\\

\haiku{Hadden we het niet?}{veel beter met ons twee\"en}{alleen ingericht}\\

\haiku{George keek hem,:}{over zijn krant heen aan oogde}{over de straat en zei}\\

\haiku{Nu een beetje de,.}{Oostenrijksche Alpen in}{dat zou niet kwaad zijn}\\

\haiku{Ze rekenden met.}{den ober af en hielden iets}{meer dan een peng\"o over}\\

\haiku{Hij zag de wet van:}{de natuur en daarin de}{goedertierenheid}\\

\haiku{Ik geloof, dat ze.}{meer gezond verstand heeft dan}{de meeste vrouwen}\\

\haiku{L\'a\'at ze snertdingen!....}{koopen als ze dan toch het}{goede niet willen}\\

\haiku{Ik ben niet in de...}{gelegenheid gesteld om}{de kunst te dienen}\\

\haiku{als George eens....}{werkelijk niet meer om zijn}{kunst of de Kunst gaf}\\

\haiku{Nee, leven moesten ze,....}{van de anderen daarin}{had Bella gelijk}\\

\haiku{Norsch groette hij....}{hen en gaf onwillig den}{sleutel voor de lift}\\

\chapter[14 auteurs, 1617 haiku's]{veertien auteurs, zestienhonderdzeventien haiku's}

\section{Herman Teirlinck}

\subsection{Uit: Het gevecht met de engel}

\haiku{Het woud galmt weldra.}{van de bonk der aksten en}{de schreeuw van het hout}\\

\haiku{Achteraan rijst het.}{somber beukenmassief van}{de Berg ter Oigne}\\

\haiku{... weet hij - Mijnheer Gomeer.}{heft rustig zijn rechterhand}{op en hij glimlacht}\\

\haiku{Hij interesseert.}{zich voornamelijk voor de}{twee oudste jongens}\\

\haiku{Het gaat met deze.}{delikate teelt niet vlot}{in den beginne}\\

\haiku{Hij kan hem nog eens,.}{duchtig dooreenschudden met}{zijn oude knuisten}\\

\haiku{In 1856 is Terve.}{mede-eigenaar van zes}{en dertig serren}\\

\haiku{Maar het is goed ook,.}{want zij doen het woud zwellen}{en drijven het sap}\\

\haiku{{\textquoteright} Er dreigt iets los te,.}{barsten de tijd slechts van een}{dubbele ademstoot}\\

\haiku{Hij zal dan even uit.}{de weg van de Burcht en langs}{het dorp heen omgaan}\\

\haiku{En ook zijn gelaat,,.}{dat blozend is ligt nog in}{donzige lijnen}\\

\haiku{Het rijtuig baant zich.}{een moeizame weg en staat}{v\'o\'or het hoofdportaal}\\

\haiku{Wat er ook in hun,.}{geest kan omgaan zij reppen}{er over met geen woord}\\

\haiku{Hij komt buigen en.}{vouwt profijtig zijn gele}{handen over de borst}\\

\haiku{Ik ben Gomerus de,.}{derde gehoorzame stem}{Maria toegewijd}\\

\haiku{Echt voortrekkersbloed.}{nochtans stoort zich slechts matig}{aan zulke grillen}\\

\haiku{Zijn lip is door een,.}{fijn snorretje beschaduwd}{de moeite niet waard}\\

\haiku{{\textquoteright} Het mag niet te ver,,.}{gaan meent Iffratje en hij}{blaast zijn rook scheef uit}\\

\haiku{Nu pas zag ik hoe.}{vernuftig Mak het voor de}{jacht had ingericht}\\

\haiku{{\textquoteright} Iffratje sliert met:}{zijn beide handen over zijn}{gelaat en kreunt stil}\\

\haiku{En Achiel, wanneer men,.}{het bloed niet opjaagt wordt zo}{tam als een poedel}\\

\haiku{De winkel is vol,.}{licht want het is een danig}{propere winkel}\\

\haiku{Het wordt door Balten,.}{die anderhalf jaar ouder}{is dan Bruin geleid}\\

\haiku{Iffratje staat op.}{de drempel van de sakristij}{naast Toontje Rozier}\\

\haiku{En dat daarbij de?}{ganse bevolking in het}{gedrang kan komen}\\

\haiku{Maar achteraf is.}{hij er de mannelijkheid}{van gaan ervaren}\\

\haiku{Klaus heeft dan het enig,:}{middel aangewend het enig}{dat mogelijk was}\\

\haiku{Zijn bruin gelaat, in,.}{zijn beheerste vlakheid is nooit}{zo krachtig geweest}\\

\haiku{Maar nog zachter is,.}{zijn vinger die traag de ogen}{van Vrouw Odile sluit}\\

\haiku{Alleen zijn houding,,.}{blijft tegenover zijn vader}{nors en vijandig}\\

\haiku{Hij kan onder geen.}{omstandigheid dergelijk}{vooruitzicht dulden}\\

\haiku{Brozen hoort niets dan.}{het geruis van zijn voeten}{door het droge loof}\\

\haiku{Dit is van al de,.}{bossen op Zoni\"en de}{wildste de dichtste}\\

\haiku{En daarboven spreidt.}{zich de blauwigheid van}{de winterhemel}\\

\haiku{Alle drie zijn zij.}{bezeten door een hete}{vreugd zonder geluid}\\

\haiku{Twee dennekegels,.}{zijn haar ogen een knoestige}{wortelstomp haar neus}\\

\haiku{Bruin verstaat wel dat.}{nu het beste is dat zij}{gauw thuis geraken}\\

\haiku{Die God laat zich zien,,,,.}{en horen en ruiken en}{smaken en tasten}\\

\haiku{Het loof dat goed is,.}{om te eten en het loof dat}{de koortsen aansteekt}\\

\haiku{Want het beste deel.}{van Gods wijde wereld heeft}{een hemelse smaak}\\

\haiku{Dat is zo altijd,.}{geweest van in de tijd der}{holen te Rhode}\\

\haiku{Veerle moet zo lang.}{op de Burcht blijven tot ze}{geheel hersteld is}\\

\haiku{Maar de grol stijgt reeds.}{in de baard van Klaus en hij}{noemt haar een schijtkont}\\

\haiku{En wat er met de,:}{jongens omgaat weet hij zo}{goed als een ieder}\\

\haiku{'t is de hitte.}{van het leven die in hun}{bloed is geslagen}\\

\haiku{hoe zouden zij hem,?}{niet kennen die weerhelmt in}{hun bevende vlees}\\

\haiku{Hij ziet haar benen,.}{die van een kille blankheid}{zijn en hij haat haar}\\

\haiku{Want dat mag hij wel,.}{weten hij stinkt naar lafheid}{en huichelarij}\\

\haiku{Daarop verlaat hij,. '}{de huiskamer te wege}{naar de stallingen}\\

\haiku{Tegelijkertijd.}{nadert aan zijn rechter een}{paard te viervoete}\\

\haiku{Zij moeten aan de,.}{overkant van de diepte de}{andere flank op}\\

\haiku{Over haar glanzende.}{huid hangt de late dag een}{rozige klaarte}\\

\haiku{Zij neemt de goede,.}{afstand veert uit en beukt met}{haar kop in zijn maag}\\

\haiku{Hij zou nu moeten,.}{bidden want deze avond is}{uitermate goed}\\

\haiku{Het is nabij het,.}{krieken van de morgen dat}{de aanval geschiedt}\\

\haiku{Uitroeiing is voor.}{het minste verraad nog een}{veel te milde straf}\\

\haiku{De Dulle schudt van,;}{neen en Emke heeft schoon naar}{haar ogen te zoeken}\\

\haiku{Zij hopen dat hun.}{moeder de richting goed heeft}{gade geslagen}\\

\haiku{Wat ook echter de,.}{bevelen mogen zijn hij}{zal ze uitvoeren}\\

\haiku{Zonder naar Balten,.}{om te zien wacht hij op wat}{Balten zal zeggen}\\

\haiku{Nee, 't was een van,.}{die rabauwen waardoor het}{gebit zo sleeuw wordt}\\

\haiku{{\textquoteright} Lieven heeft uit zijn.}{armen de benen van Bruin}{voelen wegglijden}\\

\haiku{Het gelaat van zijn.}{moeder is daar plots in de}{klaarte gerezen}\\

\haiku{Emke staat aan de.}{rand van het open graf waar Bruin}{is omgekanteld}\\

\haiku{Isa\"u kwam als eerste,.}{zoon ter wereld gevolgd door}{zijn broeder Jakob}\\

\haiku{Er was een zolder,.}{onder het rieten dak van}{het huis in het woud}\\

\haiku{Ik ben nooit bang juist.}{omdat ik die angst steeds heb}{moeten bevechten}\\

\haiku{En ik heb altijd.}{gevoeld dat mijn ras  uit}{het woud is ontstaan}\\

\haiku{Zo niet hadt ge niet.}{zo zorgvuldig vermeden}{te spreken over haar}\\

\haiku{Ik zag niets anders,,.}{onder mij dan het groene}{bodemloze meer}\\

\haiku{Het werk waarmede,.}{de Burcht mij had bedacht was}{als geknipt voor mij}\\

\haiku{Ik zag een grijze,.}{haarbos fraai opgekamd naar}{een voorbije mode}\\

\haiku{{\textquoteleft}Mijnheer Brozen zal,.}{ons denkelijk een handje}{toereiken Abdon}\\

\haiku{Aanvankelijk was.}{ik met deze kentering}{aardig in mijn schik}\\

\haiku{De vriendelijkheid.}{van zijn open glimlach kwam mij}{plots verontrusten}\\

\haiku{En door de ruimte.}{wiegelde hier en daar een}{blaadje nederwaarts}\\

\haiku{En ten laatste zag.}{men de webslierten aan de}{dakpannen hangen}\\

\haiku{Ik wist toen nog niet,.}{wat het was en dat ze aan}{het paren waren}\\

\haiku{Het trof mij dat er,.}{geen tapijtje lag niet het}{kleinste karpetje}\\

\haiku{En zij ging een breed,.}{gordijn openschuiven dat zo}{wit was als de muur}\\

\haiku{Vader had u uit.}{de Burcht in zijn armen naar}{uw kamer gebracht}\\

\haiku{Ik had mijn houvast,.}{mijn centrale positie}{weggeredeneerd}\\

\haiku{Mijn droom naar God was,.}{een steriele zucht uit mijn}{onmacht geboren}\\

\haiku{{\textquoteright} Nu ja, ik had de,}{vraag gesteld ik moest ze wel}{staande houden al}\\

\haiku{{\textquoteleft}Kijk, Seigneur Abdon!}{neemt een zonnebad op het}{torenterrasje}\\

\haiku{Daar stond de toren, ...}{met het laag portaaltje en}{de ovale schaduw}\\

\haiku{Ik was ziek, dat moest,.}{Roedi vooral weten ziek}{en ongelukkig}\\

\haiku{En de koffie, die,.}{sterk was en heet geurde door}{mijn ganse wezen}\\

\haiku{Maar hij zou een zo.}{gruwelijke misgreep op}{het leven wreken}\\

\haiku{Ik geraakte door.}{Roedi's optreden niet het}{minst uit mijn humeur}\\

\haiku{Toen wendde hij zich:}{naar de openstaande deur van}{het salon en riep}\\

\haiku{{\textquoteleft}Laat mij niet alleen,,, '!}{laat mij niet achter Brozen}{ins hemels naam}\\

\haiku{Enfin, ik was thans,,,.}{gewaarschuwd en hij Gomeer III}{had zijn plicht gedaan}\\

\haiku{Dan werd ze moe en.}{verzocht een dag met rust te}{worden gelaten}\\

\haiku{Ze was nu plots als.}{van de duivel bezeten}{om weer thuis te zijn}\\

\haiku{{\textquoteright} Wel neen, ik snakte.}{naar een kind uit loutere}{liefde voor Veerle}\\

\haiku{Gelijk een hond zo,.}{gedwee en zo laf liep ik}{naar mijn Meesterse}\\

\haiku{Ik deed het, en ik.}{schaamde me over de pap die}{in mijn aders vloeide}\\

\haiku{Honderd maal had ik.}{het gehate manwijf in}{gedachten gedood}\\

\haiku{Ze zat de meeste.}{tijd in een zetel voor het}{raam van de living}\\

\haiku{Ik zag dat haar ogen.}{uitermate groot en rond}{waren geworden}\\

\haiku{Hij legt daarbij zijn,,.}{handen overeen op zijn borst}{leent het oor en wacht}\\

\haiku{{\textquoteleft}Kom binnen,{\textquoteright} zegt de, {\textquoteleft}.}{Burgemeesterwe gaan een}{glaasje port drinken}\\

\haiku{Wel, ge belooft al,.}{wat ze willen en ge zoudt}{er een eed op doen}\\

\haiku{Over wensen uit die.}{hoek doet men wijselijk niet}{te redeneren}\\

\haiku{En hij brengt een toast,.}{uit in zo schoon Vlaams dat ge}{er niets van verstaat}\\

\haiku{Want hij heeft er al,.}{een hele boel gezien en}{van alle soorten}\\

\haiku{Dat is het geheim.}{van zijn alomvermaarde}{zachtheid van gemoed}\\

\haiku{Ik sluit aan, tot het,.}{uiterste van mijn krachten}{om leven en dood}\\

\haiku{Want, hij mag het wel,.}{weten iedereen in de}{Karot houdt van hem}\\

\haiku{En verder keurt zij.}{de opgewondenheid van}{Ida tenemaal af}\\

\haiku{Het is niet vlakbij,,.}{merkt hij op en de regen}{heeft de nacht vervroegd}\\

\haiku{Zij weet niet waar hij,.}{gebleven is en zij moet}{het ook niet weten}\\

\haiku{Zij zal gelaten,.}{luisteren nu naar Mak al}{geeft zij er niets om}\\

\haiku{De ganse Burcht, met,.}{zijn meesters en zijn macht kan}{haar nooit meer schelen}\\

\haiku{Hij heeft ineens de.}{goede kant van zijn slechte}{positie ontwaard}\\

\haiku{Integendeel brengt,.}{zij die op haar eigen hart}{dat dreigt te barsten}\\

\haiku{Hij sluipt opwaarts en.}{nadert met de lucht die heet}{uit zijn lippen stoot}\\

\haiku{Maar het is alles,.}{volkomen onbelangrijk}{alles doodgewoon}\\

\haiku{Het zit hem vers in}{de plooien aan het lijf en}{hij vertrouwt blijkbaar}\\

\haiku{{\textquoteright} Toontje herleest en.}{herleest het briefje dat aan}{zijn vingeren beeft}\\

\haiku{Wacht nu een beetje,.}{mompelt de koster op de}{korte weg naar huis}\\

\haiku{Zij wiegelt in het.}{licht van de winkel en zij}{nadert vol gratie}\\

\haiku{{\textquoteright} roept Achiel, terwijl hij.}{Ida tegemoet loopt en haar}{van haar fiets bevrijdt}\\

\haiku{Zij is zo bleek, dat.}{Achiel het raadzaam acht haar naar}{huis te brengen}\\

\haiku{Hij kust haar wild over,,.}{het aangezicht over de ogen}{hij zoekt haar lippen}\\

\haiku{{\textquoteright} Niemand verwacht dat.}{Toontje Rozier daarop iets}{te antwoorden heeft}\\

\haiku{Zij besluiten langs.}{de Vogelenzang weer naar}{het dorp te keren}\\

\haiku{Hij staat v\'o\'or haar, hoog,.}{in de kleur en zijn oog schiet}{wrokkige blikken}\\

\haiku{Hij is een dikke.}{vriend van Rozier die aan het}{harmonium zit}\\

\haiku{En vermits hij nu,.}{het geld teruggeeft zal er}{geen haan over kraaien}\\

\haiku{En zij ervaart een.}{afstand die verder dan de}{oneindigheid reikt}\\

\haiku{En zijn stiernek stoot.}{onmeedogend zijn voorhoofd}{tegen het plankier}\\

\haiku{De winnaar scharrelt.}{zijn kleren bijeen en maakt}{zich uit de gaten}\\

\haiku{Er valt nu plots op.}{de Berg-ter-Oigne}{een doodse stilte}\\

\haiku{En zo blijft zij een,.}{tijdje totdat de wereld}{ophoudt te bestaan}\\

\haiku{Daarboven dunkt hem.}{dat hij mee aan het zweven}{gaat met de wieken}\\

\haiku{Er is een aarde,.}{en de Schepper heeft ons uit}{aardse klei gewekt}\\

\haiku{Zij besluit op een.}{middag een bezoek aan de}{Terve's te brengen}\\

\haiku{Het is een gemak,.}{voor hem en het verhoogt zijn}{priesterlijk gezag}\\

\haiku{De dag eindigt in.}{een zwoelte die uit de ovens}{van de hemel valt}\\

\haiku{Tante Celesta.}{en Ida bezoeken in der}{haast het achterhuis}\\

\haiku{Misschien begrijpt hij.}{niet wat zij zo overvloedig}{onder woorden brengt}\\

\haiku{Meteen worden zijn,,.}{ogen droog en hij houdt niet af}{haar aan te staren}\\

\haiku{De ruimte gonst van.}{de prille vliegbeestjes die}{blijven aanzweven}\\

\haiku{Zou hij van binnen,?}{gekwetst zijn want aan het hoofd}{is niets te merken}\\

\haiku{Aneveert met herden!}{moet dat leven ende laet}{Bourgongnen waeyen}\\

\haiku{'t Was alsof de.}{hand Gods de hoogmoed in het}{stof had geslagen}\\

\haiku{Zij doet zich gelden,,.}{zonder twijfel zonder spijt}{zonder genade}\\

\haiku{Zij sluit het boek en.}{verstaat dat zij wordt verzocht}{de deur te openen}\\

\haiku{Hij heeft het sterkste,.}{kroost van het woudland gehad}{jongens als beren}\\

\haiku{Hij maakt aanstalten.}{om er onmiddellijk van}{onder te trekken}\\

\haiku{de pastorij van.}{deze stervensnood op de}{hoogte heeft gebracht}\\

\haiku{Onder de grauwe.}{deken kan men niet eens meer}{een lichaam raden}\\

\haiku{{\textquoteright} En hij slaat voor zijn.}{eigen geruststelling een}{suppletair kruisje}\\

\haiku{Mak Jeroen hoeft zich.}{niet af te vragen van waar}{zij gekomen zijn}\\

\haiku{Veel liever echter:}{zouden zij gebaren als}{de gemene man}\\

\haiku{Laat het de woorden,.}{slaken dat het minder dan}{uzelf verschuldigd is}\\

\haiku{De kleine Pia zal.}{gauw haar waardige plaats op}{de Burcht innemen}\\

\haiku{{\textquoteright} Zij merkt hoe door de.}{wijde baard van de Reus een}{korte rilling vaart}\\

\haiku{Het kan een ree zijn,.}{natuurlijk maar Mak aarzelt}{nooit zich te dekken}\\

\haiku{Ten ware dat de ...}{reden bij de gestrenge}{paters van Achel ligt}\\

\haiku{Iffratje vangt met.}{de lering aan elf maanden}{v\'o\'or de plechtigheid}\\

\haiku{Het is zijn gebod.}{dat wij onze moed aan de}{zijne beproeven}\\

\haiku{Zullen wij nochtans?}{ons brood uit de hand van God}{blijven bedelen}\\

\haiku{Zij geniet van die,.}{stilte waarbinst zij Klaus weet}{ontredderd worden}\\

\haiku{Men zou ook zeggen.}{dat het zich met het noodlot}{vereenzelvigd heeft}\\

\haiku{Het ingrijpen van.}{Iffratje heeft daarbij de}{doorslag gegeven}\\

\haiku{Zij zal leven naast.}{de stamvader die de wet}{in zijn handen houdt}\\

\haiku{{\textquoteleft}Het is goed,{\textquoteright} herhaalt;}{Nicodeem terwijl hij de}{checks verder invult}\\

\haiku{Met een dergelijk.}{resultaat zal iedereen}{opgetogen zijn}\\

\haiku{Het ligt in zijn stijl.}{er zo fijntjes mogelijk}{achter te zitten}\\

\haiku{{\textquoteright} Zij danst de trappen,.}{af en het hout kraakt niet eens}{onder haar zolen}\\

\haiku{Zij bekomt echter.}{niet seffens van de drift die}{in haar nadavert}\\

\haiku{Hij gaat eigenlijk.}{met de minimale vorm}{nog het minst akkoord}\\

\haiku{Een hele wereld.}{is geruisloos rondom hem}{in het niet gestort}\\

\haiku{De stilte van het, '.}{woud wacht op de stoot diet}{al bevrijden moet}\\

\haiku{Het speeksel schiet hem,.}{onder de tong en hij geeft}{zijn paard de sporen}\\

\haiku{Hij moet verder, de,.}{dennen voorbij de snoer van}{bergen bereiken}\\

\haiku{Het is het brood dat.}{Zo\"e naar haar mond te wege}{was op te steken}\\

\haiku{Hij strijkt rustig over,.}{zijn baard zoals hij na een}{eetmaal pleegt te doen}\\

\haiku{En nu moet Zo\"e eens.}{goed luisteren naar wat hij}{haar te vragen heeft}\\

\haiku{De Nachtegaal draait,.}{gesmeerd onder de impuls}{van Mamme zijn vrouw}\\

\haiku{{\textquoteright} Zij glijdt langzaam aan.}{Ida's knie\"en neer en neigt haar}{voorhoofd op Ida's borst}\\

\haiku{Dan verschijnt Plone,.}{met het schenkbord waarop de}{roemers fonkelen}\\

\haiku{De jonge Burchtheer,.}{is daar hij vraagt of hij mag}{komen luisteren}\\

\haiku{En discretie de.}{doelmatigste voorwaarde}{van zijn uitvoering}\\

\haiku{Pia haast zich om de.}{Burchtheer bij zijn inspanning}{behulpzaam te zijn}\\

\haiku{En wat heeft dan een?}{kristene niet al door de}{vingeren te zien}\\

\haiku{De torens van vuur.}{aan de windhoeken van de}{bergtop doven uit}\\

\subsection{Uit: Mijnheer J.B. Serjanszoon. Orator didacticus}

\haiku{Eigenlijk weten,,.}{wij niet v\'o\'or 1785 wie mijnheer}{J.B. Serjanszoon was}\\

\haiku{Die ouwe heer is,,.}{haar oom monsieur de la}{H\^etraie van Brussel}\\

\haiku{hoe blozen op haar.}{tere wangen haar onschuld}{en haar schuchterheid}\\

\haiku{- Lieve dames en,,.}{heren riep hij daar moeten}{wij eens bij klinken}\\

\haiku{De Griffier stond recht.}{en riep dat hij dadelijk}{zijn knecht nodig had}\\

\haiku{De officier der;}{dragonders bood zijn arm aan}{juffrouw Dieulafoy}\\

\haiku{Maar ik beveel aan.}{al uwe zorgen mijn boeken}{en mijn papieren}\\

\haiku{Ik vraag geen pauw op,.}{een gouden schaal omdat ik}{dit niet nodig heb}\\

\haiku{Zal ik mij wild en?}{ongenadig overgeven}{aan al mijn driften}\\

\haiku{Serjanszoon vroeg hem}{dan waar hij heentrok en met}{welk bijzonder doel}\\

\haiku{Hij stond waarlijk recht,...}{op zijn magere benen}{viel noch wankelde}\\

\haiku{zodat ik verplicht...}{ben geweest mij alweer wat}{te gaan opknappen}\\

\haiku{Het leven, mevrouw,.}{is beter naarmate men}{het dieper beleeft}\\

\haiku{Hij ging, schuivend langs.}{de dikke tapijten en}{leunend op zijn staf}\\

\haiku{Mijnheer Serjanszoon.}{schoof zijn hand om de le\^en van}{juffrouw Cornelie}\\

\haiku{Ik hink zelf, mijn vriend,,.}{ofschoon het zeker is dat}{ik nooit zal sjieken}\\

\haiku{Mijnheer Serjanszoon.}{werd waarlijk aanmatigend}{en rondde zijn borst}\\

\haiku{Hij perst de druiven,.}{van zijn wijngaarden uit over}{zijn mond en pinkoogt}\\

\haiku{vloekt de vermoeide,,,!}{Winter gij zijt laf jonker}{en schiet van verre}\\

\haiku{Ik zie de botten.}{springen op de hazelaars}{en de vlierstruiken}\\

\haiku{Hij boog en wuifde,.}{met zijn hand over de tafel}{al glimlachend flauw}\\

\haiku{Hij doopte zonder.}{reden zijn duim tot op de}{kneukel in zijn wijn}\\

\haiku{Aldus staat vast, dat,,.}{ik om iedereens bestwil}{voorzichtig moet zijn}\\

\haiku{Lichte meerminnen.}{spuwden een wit waterschuim}{in zijn aangezicht}\\

\haiku{- Inderdaad, mevrouw,,:}{antwoordde hij dankbaar zo}{ziet gij in mijn ziel}\\

\haiku{Maar, lieve mevrouw,.}{ik verwonder met al dit}{geschrijf mijzelven}\\

\haiku{mocht ik nooit meer zien,,!}{geduldige vriendin wat}{toen mijn ogen zagen}\\

\haiku{Hij weende niet over,.}{zijn ziekte niet meer over zijn}{vermaledijding}\\

\haiku{Hij knikte, reikte,.}{zijn hand naar het bultje dat}{lachend naderkwam}\\

\haiku{- Ja! - En gij zult, vroeg,'?}{angstig mijnheer Serjanszoon}{de Louis weernemen}\\

\subsection{Uit: De nieuwe Uilenspiegel}

\haiku{Hij voelde wel dat.}{er iets onverkwikkelijks}{in zijn toestand was}\\

\haiku{Toen meende hij te,:}{zeggen zijn linkerhand naar}{het Broodhuis gericht}\\

\haiku{Twee dagen later,.}{moest Bettel Broederlam gaan}{liggen met het pleures}\\

\haiku{Hij vulde zijn oogen.}{met het spektakel van clowns}{en danseressen}\\

\haiku{Op den bek van den.}{hoogsten steenen spuwer kwam een}{bloot kindje zitten}\\

\haiku{{\textquoteright} - {\textquoteleft}Nu komt er een man,.}{met vuile handen om de}{kaartjes te knippen}\\

\haiku{Dan ging hij weer bij.}{Nelleken zitten en vlocht}{den beloofden hoed}\\

\haiku{Hij bleek goed van pas,.}{maar zij moest toch haar hoofdje}{stijf rechtop houden}\\

\haiku{{\textquoteright} En hij djakte zoo...}{geweldig tot Sarelke met}{zijn zweep zwijgen moest}\\

\haiku{Ze vond dat hij wel,, -.}{rood was en fel en moedig}{maar lang geen Judas}\\

\haiku{De kassei wilde,.}{niet klinken onder hunne}{beslijkte hielen}\\

\haiku{Thijl stortte neer op.}{zijne knie\"en en begon}{te bidden luidop}\\

\haiku{De zon ging onder.}{in een apotheose van}{goud en karmozijn}\\

\haiku{{\textquoteright} Een halve maat te.}{vroeg en geheel alleen in}{de plots ijle lucht}\\

\haiku{Langzaam wuifde ze,.}{ermede en keerde zich}{om en danste licht}\\

\haiku{{\textquoteright} vroeg Jakeliene,.}{en ze verscheen te midden}{van het wingerdgroen}\\

\haiku{Zijn oogen lagen vast,.}{op de witte kassei waar}{niets meer te zien was}\\

\haiku{Zijn kale schedel,.}{glansde gedempt gelijk een}{eenzame avondkim}\\

\haiku{Thijl hing, boven de,.}{beukkruin aan de uiterste}{taknaald te wiegen}\\

\haiku{De kleinste droeg een.}{aapje op zijn schouder en}{blies op een schalmei}\\

\haiku{Hij struikelde over.}{de gracht en stapte beslist}{naar het Eremijtbosch}\\

\haiku{{\textquoteright} lachte hij luid, {\textquoteleft}geen, '....}{hoogwater ofk begin}{u te kriebelen}\\

\haiku{Wegens zijn statig.}{voorkomen heette men den}{vader Pijken-Aas}\\

\haiku{De haag smoorde een,.}{vloek en Pijke-Zeven}{deinsde achterwaarts}\\

\haiku{Hij had een jong lijf.}{dat de frissche vormen van}{den groei vertoonde}\\

\haiku{Zijne satijnen.}{voeten rustten op een}{windhond van albast}\\

\haiku{Het begon al laat,:}{te worden maar hij wist haar}{toch zitten en riep}\\

\haiku{{\textquoteright} Tegelijkertijd,.}{ontdekte hij Markies die}{langs de gracht wegsloop}\\

\haiku{En hij mompelde.}{er iets bij van zatlappen}{en van nachtraven}\\

\haiku{Boer Soeverein en,.}{juffrouw Ursule hadden}{hem gehoord tweemaal}\\

\haiku{{\textquoteright} Thijl Uilenspiegel.}{reikte de handen naar een}{beeld dat te hoog hing}\\

\haiku{Hij viel op zijne.}{knie\"en en krabbelde in}{het groengrijze net}\\

\haiku{Het bed gaapte hem.}{heet toe en dekte hem met}{intieme reuken}\\

\haiku{Een prevelen van.}{haren mond zuchtte gretig}{aan zijne lippen}\\

\haiku{In den laten avond.}{had hij met Nelleken een}{belangrijk gesprek}\\

\haiku{Thijl wendde zich even,.}{naar den kant waar het wijde}{water donkerde}\\

\haiku{- {\textquoteleft}Zoo is 't goed,{\textquoteright} zei, {\textquoteleft}.}{hijen ik blijf bij u. Spreek}{mij maar niet tegen}\\

\haiku{Ze paste nochtans.}{zoo aardig in dees vroom en}{peinzende midden}\\

\haiku{Na den eten, leidde,.}{hij mij over den tuin tot bij}{den grooten mesting}\\

\haiku{Thijl vroeg: - {\textquoteleft}Zeg, Pijke,?}{mag ik u vragen wat gij}{te Brugge komt doen}\\

\haiku{Dan nam  hij het.}{kind en stak het op in het}{licht van het venster}\\

\haiku{Elken dag komt ge,.}{en elken dag is het ons}{een nieuwe blijdschap}\\

\haiku{Hij was tevreden.}{over het akkoord dat hij met}{haar gesloten had}\\

\haiku{to car je taime}{et tu doit pardonner tout}{les jalouseries}\\

\haiku{Hij kon een zucht niet.}{neerdwingen die als een prop}{in zijne keel zwol}\\

\haiku{{\textquoteright} Thijl's lach daverde,:}{door de keuken maar hij zag}{geen steek v\'oor oogen meer}\\

\haiku{Die kijken naar een,;}{vogelken wit Dat in den}{wijden hemel zit}\\

\haiku{De stad lag geheel.}{onder een poeiering van}{teer-gulden licht}\\

\haiku{Ze was al een heil,:}{eind ver als ze zich verschrikt}{omwendde en riep}\\

\haiku{Daar schoot ineens de,!...{\textquoteright}}{merrie van Belle-Trees naar}{voren als een pijl}\\

\haiku{Ik zag haar onder.}{water vreeselijk werken}{met armen en beenen}\\

\haiku{Nelleken, bleek van,.}{schrik reikte naar hem hare}{smeekende handen}\\

\haiku{Op den Brusschelschen.}{steenweg had hij een bonten}{rolwagen gezien}\\

\haiku{Zijn neus vleugelde.}{bevend bij elken geur dien}{de zomerwind bracht}\\

\haiku{Thijl had Pierlapeu.}{vastgekregen en danste}{ermee in het rond}\\

\haiku{Maar het andere}{oog puilde heerlijk uit en}{sloeg naar links en rechts}\\

\haiku{{\textquoteright} deed Zoster, {\textquoteleft}ik heb.}{geen reden om niet naar de}{foor te gaan met haar}\\

\haiku{{\textquoteright} riep Miss Violet {\textquoteleft}?}{met verontwaardigingwaar}{zijn uwe gedachten}\\

\haiku{Ze gingen in een.}{kolk van klaarte en roken}{de motorhitte}\\

\haiku{Uit al uwe macht moogt,.}{ge knokkelen maar mijd u}{voor den weeromstuit}\\

\haiku{Ge zijt gekomen.}{om hem te toetsen aan de}{taaiheid van zijn lijf}\\

\haiku{- {\textquoteleft}Maar gij dan, rosse{\textquoteright},, {\textquoteleft}?}{riep Mandienezijt gij zelf}{niet aan te spreken}\\

\haiku{Hij riep, de handen: - {\textquoteleft},!}{ten hemelBeiaardier houd}{op met rammelen}\\

\haiku{De arme jongen.}{was uitgeput van honger}{en vermoeienis}\\

\haiku{Voor de eerste maal,.}{in mijn leven zou ik dooden}{en ik beefde niet}\\

\haiku{En ik aanvaardde.}{een pijnlijke reis in de}{richting van Brussel}\\

\haiku{Vandaag was zijn plan.}{om mij aan den vijand over}{te leveren}\\

\haiku{De schreeuw smoorde in,.}{zijn strot dien Thijl met de koord}{had toegenepen}\\

\haiku{{\textquoteright}, kloeg hij, {\textquoteleft}'t Verraad!}{zit aan uw haard en wacht op}{u. Keer om en vlucht}\\

\haiku{{\textquoteright} deed Uilenspiegel {\textquoteleft},.}{als alles tegenwerkt dan}{kan het nog verkeeren}\\

\haiku{- {\textquoteleft}Geef mij het pakje,.}{dat de juffrouw u in de}{handen heeft gestopt}\\

\haiku{Hij zei: - {\textquoteleft}Oolijke,!...}{bult ik zal u wel anders}{te knippen krijgen}\\

\haiku{De Koning heeft ons.}{gefeliciteerd en het}{heeft ons aangedaan}\\

\haiku{Maar ik had bloed aan,.}{mijn mond want de mond van den}{sergeant bloedde}\\

\haiku{Ge zoudt over hun gansch.}{lijf geen enkel luisje meer}{gevonden hebben}\\

\haiku{We waren eens op.}{de baan en uw lach klonk als}{een kelk van kristal}\\

\haiku{{\textquoteright} als een die mij in,,.}{zijn glorie na een verre}{reis weder ontmoet}\\

\haiku{Gloeiende torens,.}{gingen op hingen hoog in}{de lucht te lichten}\\

\haiku{Twee Zwarte Bloemen.}{duiken op uit den grond als}{uit een valdeurken}\\

\haiku{Een arm steekt roerloos,.}{op naar mij uit een grijzen}{stapel van lijven}\\

\haiku{Het is toch waarlijk!}{niet mogelijk dat  ik}{iets bedreven heb}\\

\haiku{Nu zie ik Sarelke.}{en el'Dj\^ozef en Staaf op}{een hoopken liggen}\\

\haiku{{\textquoteright} Maar Piet Hein fronste.}{zijne wenkbrauwen en zijn}{blik werd hard als staal}\\

\haiku{Kerels met witte.}{kluppels regelden zwijgend}{en ernstig den gang}\\

\haiku{Thijl voelde rond hem,,.}{de werkzame ziedende}{stampende ruimte}\\

\haiku{Toen gingen hare.}{magere handen streelen}{over een kinderkop}\\

\haiku{John werd gewaar.}{dat hij een onverwachten}{uitslag had bereikt}\\

\haiku{Thijl bedankte en.}{liep vroolijk naar den kijker}{van zijnen blauwvoet}\\

\haiku{Zijn gansche wezen.}{sprong gelijk een pijl uit de}{boeien van zijn angst}\\

\haiku{Pierlapeu stond met.}{gapenden mond en wilde}{de sneeuw opvangen}\\

\section{F.C. Terborgh}

\subsection{Uit: De condottiere en andere verhalen, gevolgd door Le petit chateau}

\haiku{Langzaam lopen we.}{terug naar het andere}{uiterste vertrek}\\

\haiku{dat niemand buiten{\textquoteright}.}{het aannemen van zulk een}{voortleven kan}\\

\haiku{de grauwe morgen.}{breekt aan en dringt langzaam door}{de bovenlichten}\\

\haiku{Een krakende kar,,.}{reed voorbij een hond bleef voor}{hem staan snuffelend}\\

\haiku{Soms verblindde hem:}{terzijde een zoeklicht vlak}{boven het water}\\

\haiku{Men leunt roerloos in.}{een hoek en wordt gewiegd en}{kijkt door het open raam}\\

\haiku{Weer valt  een blad,,,.}{dof steels als ware het bang}{betrapt te worden}\\

\haiku{gletscher had gezien,;}{onwerkelijk als uit een}{andere wereld}\\

\haiku{Met de middagbus.}{ben ik naar Pensacola}{teruggereden}\\

\haiku{Een spiegel en troost.}{voor zijn eigen hopeloos}{onnut bestaan}\\

\haiku{alles hing slechts af,;}{van de mate van moeheid}{van uitgeputheid}\\

\haiku{Windstoten droegen;}{het slaapdronken luiden aan}{van schapenklokjes}\\

\haiku{Sedert vijf avonden;}{herleef ik die vaart langs de}{Bahamaeilanden}\\

\haiku{een zwart zwijn, zich in,.}{een modderplas wentelend}{versperde den weg}\\

\haiku{hij leek reeds ver op.}{het pad der verlossende}{contemplatie}\\

\haiku{eksters klapten er;}{en wielewalen vlogen}{roepend door het veld}\\

\haiku{Ik kroop voorzichtig.}{naar de vlak bij het water}{gelegen schuiten}\\

\haiku{Hij bewoog zich niet.}{en scheen mijn aanwezigheid}{niet te bemerken}\\

\haiku{pas nu ten volle,.}{geopenbaard die vroeger}{blind moet zijn geweest}\\

\haiku{Schaduwen en licht.}{en traliewerk brachten me}{terug naar Coimbra}\\

\haiku{En wie is tegen,?}{den sleur bestand de leegte}{van den grauwen dag}\\

\haiku{Het doel van den tocht,.}{was in zicht de eerste der}{goudsteden bereikt}\\

\haiku{In de groeiende:}{verwarring greep eindelijk}{de student het woord}\\

\haiku{Hij trok zijn degen,.}{en stormde vooruit recht in}{de lege steppe}\\

\haiku{of heft het hart zich?}{slechts tot hun kille verten}{kort voor den overgang}\\

\haiku{Maar voelt het lichaam;}{niet nog de koesterende}{warmte der aarde}\\

\haiku{misschien reeds hun graf. '.}{s Middags moesten de paarden}{worden afgemaakt}\\

\haiku{Van uitsteeksel tot,;}{uitsteeksel zou men dalen}{tot aan het water}\\

\section{P. Tesselhoff jr.}

\subsection{Uit: Het succes van den rechercheur}

\haiku{{\textquoteleft}En hoelang is u,,{\textquoteright}.}{weduwe mevrouw zoo liet}{Ad\`eles stem zich hooren}\\

\haiku{Zooals altijd werd om.}{half negen thee gedronken}{in de tuinkamer}\\

\haiku{Dominee had iets ',;}{opt hart maar hij wist niet}{hoe te beginnen}\\

\haiku{Eindelijk trok hij:}{de stoute schoenen aan en}{vroeg op den man af}\\

\haiku{Den volgenden dag.}{had domin\'ee Dijker voor}{zijn vertrek bepaald}\\

\haiku{Al dien tijd werd ik,.}{niets gewaar hoe nauwkeurig}{ik ook oplette}\\

\haiku{Geen kwartier later.}{sloeg het hek achter in den}{tuin wederom dicht}\\

\haiku{Gij zijt beschuldigd.}{en gestraft voor een misdaad}{door mij bedreven}\\

\haiku{Een blanco cheque.}{ontvreemde ik tot dit doel}{uit uw schrijftafel}\\

\haiku{Nu evenwel geef ik,.}{het op het leven is mij}{tot last geworden}\\

\section{Ger Thijs}

\subsection{Uit: De huilende man}

\haiku{Maar ze gebruikte.}{die woorden alsof ze nooit}{anders gedaan had}\\

\haiku{En ik verwachtte.}{dat ze meteen snerpend zou}{reageren}\\

\haiku{Ik schoof de tassen,.}{de bus in en nam afscheid}{van de dikke vrouw}\\

\haiku{Ik stak de straat over,,.}{opende het tuinhekje en}{liep naar de voordeur}\\

\haiku{Hoe kunnen we ons!}{dingen herinneren als}{ik je niet herken}\\

\haiku{Oom Theo had ook een!}{baard en je weet wat er van}{hem geworden is}\\

\haiku{Het was alsof het.}{lichaam daarboven groeide}{met elke ademtocht}\\

\haiku{Ze veranderde.}{niet van houding toen ik haar}{lichaam aanraakte}\\

\haiku{Anderzijds moet ze:}{aan dat afschuwelijke}{zinnetje denken}\\

\haiku{Ze kruipt in  bed.}{en ligt de hele nacht naar}{de muur te staren}\\

\haiku{Hij koopt meubels van.}{het geld dat hij gespaard heeft}{voor een motorfiets}\\

\haiku{Hij heeft haar grommend.}{op de matras  gedrukt}{en haar genomen}\\

\haiku{Ik speelde met de,.}{zaklantaarn zoals ik het}{vroeger had gedaan}\\

\haiku{Je zei steeds maar dat,.}{ik in niets in helemaal}{niets op vader leek}\\

\haiku{Waarschijnlijk was ik,.}{de enige die wakker was}{nu mijn zuster sliep}\\

\haiku{Zoals Matti die.}{opeens een klein mannetje}{van hem gemaakt had}\\

\haiku{Ik schraapte luid mijn,.}{keel om mijn aanwezigheid}{kenbaar te maken}\\

\haiku{De man die aan de,.}{andere kant naast me zat}{boog zich naar me over}\\

\haiku{Hoe maak ik mijn vrouw?}{duidelijk dat ik liever}{in de kelder zit}\\

\haiku{Ergens binnen in.}{jou zit een klein vogeltje}{dat vaders naam piept}\\

\haiku{Maar ik besloot niet.}{op te staan om ze tot de}{orde te roepen}\\

\haiku{D{\'\i}t bedoelde ik,{\textquoteright},.}{zei hij en ik voelde zijn}{gewicht op het bed}\\

\haiku{{\textquoteleft}O, dat was anders{\textquoteright},,.}{zei ze en het leek wel of}{haar stem teder werd}\\

\haiku{Ik keek verbaasd naar.}{de grauwe koppen die uit}{de ramen hingen}\\

\haiku{Maar juist toen ik me,}{om wilde draaien stond de}{eerste vrouw naast me.}\\

\haiku{Maar ik vermoedde.}{dat hij al had begrepen}{dat ik niet wilde}\\

\haiku{{\textquoteright} Hij schreeuwde opnieuw,.}{en luisterde daarna met}{ingehouden adem}\\

\haiku{Het was alsof ze,.}{naar het fotoalbum keek}{met geschrokken blik}\\

\haiku{Ik wist zo gauw niets,.}{te bedenken ik lachte}{wat bevreemd naar haar}\\

\haiku{{\textquoteright} Ze sprong op en trok.}{het met een paar stevige}{rukken van de muur}\\

\section{Theo Thijssen}

\subsection{Uit: Barend Wels}

\haiku{Pool, voorop, trachtte;}{ze tikken te geven op}{hun gladde koppen}\\

\haiku{vlak naast hem lag 'n, '.}{schaar en een eindje verder}{stondn klos garen}\\

\haiku{{\textquoteleft}Zal wel gauw komme,{\textquoteright}, {\textquoteleft}.}{toch antwoordde Barendmaar}{ik heb niet veel tijd}\\

\haiku{We hebbe van die.}{jonge nou toch nooit anders}{als plezier gehad}\\

\haiku{{\textquoteleft}En toch kajje.}{om dezen tijd nog welles}{w\'armer weer hebben}\\

\haiku{en {\textquoteleft}Ja, h\'e\'el geschikt,{\textquoteright}, {\textquoteleft}...}{vond Barenden dan met zoo'n}{stelletje boeken}\\

\haiku{scharrelde daar even, '.}{en kwam terug metn paar}{andere cahiers}\\

\haiku{Op de donkere ',;}{gracht hadden zet over Pool}{die wel sjeezen zou}\\

\haiku{Wat 'n stomme vent,,.}{eigenlijk Pool om nooit es}{wat uit te voeren}\\

\haiku{Nou moeten jullie '.}{maares een kwartiertje tot}{bedaren komen}\\

\haiku{Het \'e\'ene eiland;}{benedenaan was net een}{omgekeerde pijp}\\

\haiku{{\textquoteright} De jongen hield z'n,.}{armen beschuttend boven}{z'n hoofd en kreunde}\\

\haiku{{\textquoteleft}Dat afloopen van,,{\textquoteright}.}{sommige klassen h\`e zei}{hij zuchtend tot Wels}\\

\haiku{{\textquoteright} De moeder zette.}{voor beiden brood klaar en ze}{begonnen te eten}\\

\haiku{Ze zaten zeker,.}{in de kamer en dronken}{hun kopje koffie}\\

\haiku{{\textquoteright} Je kon toch merken,,,;}{dat het broers waren dacht de}{moeder vergenoegd}\\

\haiku{voelde zich als een,.}{machtig heerscher die \'o\'ok wel}{eens royaal kan zijn}\\

\haiku{Anderen keken,,,.}{in twijfel den meester aan}{of het zoo maar mocht}\\

\haiku{noemde af en toe,.}{een naam en dan trok er weer}{een kind aan het werk}\\

\haiku{Jij kwam net binnen,.}{zeg toen ik bezig was met}{hoogere dressuur}\\

\haiku{Maar ik heb ze wel.}{vanmiddag een beetje d'r}{onder gehouen}\\

\haiku{Wels ging in de post, {\textquoteleft},!}{van z'n lokaaldeur staan en}{riepH\'e h\'e-daar}\\

\haiku{Met een slag haalde,,.}{Wels de deur toe en bleef staan}{z'n klas aankijkend}\\

\haiku{de leien weg, net,.}{of ze al een half uurtje}{gedresseerd waren}\\

\haiku{Maar toch, als je af,.}{en toe eens iets gezegd werd}{schoot je gauwer op}\\

\haiku{Ze zaten beiden.}{in Barends kamertje en}{rookten een sigaar}\\

\haiku{Tegenwoordig praat,.}{ik er niet over met ze en}{ga m'n gangetje}\\

\haiku{{\textquoteleft}Moeder, scheid u toch,{\textquoteright}, {\textquoteleft}.}{uit riep Barend vroolijkwe}{zullen nog sjeezen}\\

\haiku{{\textquoteright} riep Nico blij, en -.}{rikke tak ging z'n potlood}{de proef te maken}\\

\haiku{Anders zie 'k het,.}{eerste uur toch niet anders}{dan cijfers da's waar}\\

\haiku{Komieke idee\"en, '....}{toch soms lui die buitent}{onderwijs stonden}\\

\haiku{voor in de zaal, op,.}{het tooneel stonden groote tafels}{met groene kleeden}\\

\haiku{Barend grinnikte,,:}{weer en zei tegen Nico}{die mee-glimlachte}\\

\haiku{{\textquoteright} ~ De schoolmeesters.}{begonnen langzamerhand}{weer te verdwijnen}\\

\haiku{als pauwtjes stapten ',;}{n paar heeren rond tusschen}{de tafeltjes door}\\

\haiku{en meneer Jansen.}{bemoeide zich bijzonder}{druk met de ramen}\\

\haiku{{\textquoteright} Een jongen, die niet,:}{dadelijk doorliep kreeg een}{draai om z'n ooren}\\

\haiku{hij zag Beckers z'n, '.}{ernstige gezicht en vond}{m bespottelijk}\\

\haiku{wel glimlachend, want;}{anders zou-ie oproerig}{geleken hebben}\\

\haiku{en o, \`als dan weer,:}{de beroerdigheid kwam dat}{raadselachtige}\\

\haiku{stom, dat hij \'o\'ok niet;}{een paar jongens gehouden}{had om te helpen}\\

\haiku{{\textquoteright} Zwaan keek 'em even aan;}{met een vreemden  blik in}{z'n donkere oogen}\\

\haiku{En Wels wenschten,.}{ze erg veel succes hij moest}{zich maar goedhouden}\\

\haiku{Wels gaf iedereen,:}{w\'e\'er een hand en natuurlijk}{zei juffrouw Veen toen}\\

\haiku{ze praatten koeltjes,;}{over het examen maar kregen}{er geen ruzie over}\\

\haiku{{\textquoteright} 't Leek wel, of de, ';}{bijzitter begreep dat ze}{t over hem hadden}\\

\haiku{{\textquoteleft}Ja, maar {\`\i}k ga gauw,{\textquoteright},.}{naar huis zei De Haas al half}{w\`egloopend van ze}\\

\haiku{{\textquoteleft}Ik kan niet helpen,{\textquoteright}, {\textquoteleft}.}{klaagde zemaar ik vind het}{toch een naren tijd}\\

\haiku{Hij zou d'r \'o\'ok door - -.}{en niet op het kantje maar}{met vlag en wimpel}\\

\haiku{Dat er nou iemand....}{door boffen net zoo ver kwam}{als hij door werken}\\

\haiku{{\textquoteright} {\textquoteleft}Och nee,{\textquoteright} speelde Wels, {\textquoteleft},.}{den onverschilligewie}{d\`an leeft wie d\`an zorgt}\\

\haiku{Hij voelde zich thuis, '.}{in het scheeve zaaltje van}{t eerste uur af}\\

\haiku{En hij hoorde niet,.}{meer wat ze nog zeiden maar}{vloog al vooruit}\\

\haiku{As-ie nou nog een, '.}{acte haalde zout de}{trouwacte wel zijn}\\

\haiku{{\textquoteleft}Ja,{\textquoteright} sprak Wels ineens {\textquoteleft}.}{ik heb de laatste dagen}{doorloopend het land}\\

\haiku{merkwaardig zeg, van, '.}{ons twee\"en dat we dat zoo}{t zelfde hebben}\\

\haiku{{\textquoteright} {\textquoteleft}Die verdient driemaal ',{\textquoteright}.}{zooveel geld asn ander}{viel Nico weer in}\\

\haiku{Je bent zwaar op de,{\textquoteright},.}{hand vanavond zei Nico en}{geeuwde overdreven}\\

\haiku{{\textquoteright} {\textquoteleft}O, dat kan best, hoor,{\textquoteright}, {\textquoteleft},,}{was Barend cynischhet is}{ook persoonlijk h\`e}\\

\haiku{Hij besloot nog even,,.}{te blijven stak een sigaar}{op en ging lezen}\\

\haiku{{\textquoteleft}Vooruit maar,{\textquoteright} riep Henk,.}{net of ze iets doen gingen}{als een veldtocht}\\

\haiku{En ernstig keek-ie, '.}{nu Wels z'n schrift in omes}{precies te kijken}\\

\haiku{dat hij niet wachten,;}{kon tot het Wels beliefde}{om aan te pakken}\\

\haiku{Hij zat bij Pool op,,.}{de kamer heele avonden}{te redetwisten}\\

\subsection{Uit: Een bonte bundel}

\haiku{Je was een beetje...,,:}{h\`e Disa je had zo je}{lievelingswoordjes}\\

\haiku{Maar, ze zal d'r nog, '.}{wel wonen anders had ze}{t wel geschreven}\\

\haiku{Een kleine jonge,,.}{arbeidersvrouw bezem in}{de hand deed hem open}\\

\haiku{{\textquoteleft}Aber,{\textquoteright} begon hij, en, {\textquoteleft}}{hij bleef voor securiteit}{maar Duits sprekenaber}\\

\haiku{Grootvader stond op.}{en liep met het spiegeltje}{naar de voorkamer}\\

\haiku{Och, och, wat is dat,....}{jaren lang jaren lang een}{gemartel geweest}\\

\haiku{{\textquoteright} er op te pakken,.}{en mijn sigarenbaas had}{z'n twee gulden hoor}\\

\haiku{Nou, daar mot je dan ',,.}{maares om vragen as-je}{d'r an toe bent h\`e}\\

\haiku{Op de kamer stond,,:}{behalve de twee bedden}{een soupertje klaar}\\

\haiku{Maar de typiste,.}{kwam niet onder indruk en}{zei kalmpjes van neen}\\

\haiku{en toen we z\'o de,.}{kwestie stelden voerden hij}{ons naar de ingang}\\

\haiku{dat is bij voorbeeld,.}{ook de grote les die we}{in Gen\`eve leerden}\\

\haiku{Ik heb trouwens nog;}{nooit zo iets ouds gezien als}{mijn ene overbuurvrouw}\\

\haiku{En die Cercle de,?}{la Renaissance waar we}{vanmiddag gaan eten}\\

\haiku{Dumas hield vol, dat.}{we eigenlijk helemaal}{geen tijd meer hadden}\\

\haiku{proberen eens, hoe;}{dubbelslaan over het bruine}{hekje hun bevalt}\\

\haiku{Ik schrok me dood, en.}{de hele coup\'e verging}{van medelijden}\\

\haiku{Nooit weer proberen,....}{zo'n al-wegrijdende}{tram in te halen}\\

\haiku{ik weggezonken.}{in een grote deftige}{voorzitterszetel}\\

\haiku{zo'n echte trouwe,;}{gewone schoolmeester gek}{op de kinderen}\\

\haiku{Voor \'e\'en nacht zat ik,.}{aan deze uitspatting vast}{dat was duidelijk}\\

\haiku{{\textquoteleft}Dank-u ja,{\textquoteright} zegt-ie, {\textquoteleft}'.}{rustigk was met hiernaast}{de klas in de war}\\

\haiku{Als ik dat nu weer,.}{eens vraag dan moet het antwoord}{wat vlugger komen}\\

\haiku{{\textquoteleft}Kom jij 'es even bij,, '.}{bord ventje en schrijf jij dat}{zinnetjees op}\\

\haiku{Is anders niet van,;}{de stomsten maar hier kan-ie}{met z'n hoofd niet bij}\\

\haiku{Dat heb je zo niet,?}{geleerd dat k\`an je zo niet}{geleerd hebben toch}\\

\haiku{Nee, de enkele,,?}{drie niet de dertig weet je}{d\`at nou ook nog niet}\\

\haiku{Hij neemt het krijt en '.}{trekt nijdig een streep onder}{t laatste getal}\\

\haiku{{\textquoteright} en verdween weer, nog:}{v\'o\'or hij de bakkerin gul}{had horen zeggen}\\

\haiku{En hij liep met een ' ';}{boog naart Zuiden naart}{Lago Maggiore}\\

\haiku{Ik doe voortaan nooit.}{meer een lange reis zonder}{huispantoffeltjes}\\

\haiku{Ja, 't is lente,,....}{zoele lente je kunt je}{jas wel uitlaten}\\

\haiku{hij beheerst het zo,;}{volkomen dat-ie er}{in weet te kletsen}\\

\haiku{Daar zit je dan met.}{een internationaal}{mysterie naast je}\\

\haiku{In Basel vind ik,.}{een d\'o\'orgaande wagon naar}{Holland finaal leeg}\\

\haiku{de portier blijkt ons.}{al bepaalde kamers te}{hebben toebedacht}\\

\haiku{Piet mikte met heel - '.}{z'n ziel de haas buitelde}{vant randje af}\\

\haiku{{\textquoteleft}Van dat betere,.}{mikken van jou heb ik nog}{niet veel gemerkt zeg}\\

\haiku{We keken achter,;}{al de schilderijen en}{achter de spiegel}\\

\haiku{we zochten op de,.}{onmogelijkste plaatsen}{maar de pijl was weg}\\

\haiku{Overal een grote;}{karaf drinkwater met twee}{kristallen glazen}\\

\haiku{Doorzoek ten derde,.}{male m'n tas en daarna}{weer al m'n zakken}\\

\haiku{{\textquoteright} heb ik gezegd, en.}{ik sta al met de sleutel}{m'n deur te openen}\\

\haiku{Verduveld leuke ',,'.}{lui warent. Hm zei ik}{la we maar gaan eten}\\

\haiku{Het woord {\textquoteleft}industrie{\textquoteright}...}{krijgt door zo'n bezichtiging}{inhoud voor je}\\

\haiku{Soms ook velden, met,;}{lage struikjes  een goeie}{halve meter hoog}\\

\haiku{zoiets als onze,.}{Westlandse kassen maar dan}{in de open lucht}\\

\haiku{En m'n overbuurvrouw;}{wordt elke vijf minuten}{een jaartje ouder}\\

\haiku{Wij stegen uit en,?}{dromden om hem heen was er}{nu eigenlijk was}\\

\haiku{{\textquoteright} De kelner geeft een,.}{soort uitleg in het Frans die}{we niet begrijpen}\\

\haiku{Onze kelner raakt,,....}{in geestdrift en stelt voor een}{likeurtje er bij}\\

\haiku{En laat ik je nu,.}{vertellen wat ik nog m\'e\'er}{heb gehad vannacht}\\

\haiku{'t Is er geen een, ',.}{t is werkelijk een zwart}{vlekje anders niet}\\

\haiku{in dat dorp huur ik, ';}{eenvoudig een auto en}{ik rij naart kamp}\\

\haiku{Ik kon alleen nog,;}{niet onderscheiden wie van}{de vijf het waren}\\

\haiku{Ze keken elkaar,:}{met akelig doffe ogen aan}{ze vroegen elkaar}\\

\haiku{Op de bodem, het,.}{was precies de jampot lag}{nog wat gekleurd vocht}\\

\haiku{{\textquoteleft}Z\'o precies kan je{\textquoteright},.}{het in een kamp niet nemen}{zei hij verzoenend}\\

\haiku{Maar Jaap die nu ook,.}{met z'n fiets voor de tent stond}{keek bedenkelijk}\\

\haiku{'k Heb nog nooit zulk!}{een zakdoek uit m'n zakken}{te voorschijn gehaald}\\

\haiku{Jaap vergat zich, en;}{hapte dadelijk een mop}{chocolade af}\\

\subsection{Uit: Egeltje}

\haiku{Nou allemaal je{\textquoteright},, {\textquoteleft}{\textquoteright}.}{smoelen dicht zei de majoor}{zal je es wat zien}\\

\haiku{De egel begon, met:}{eigenlijk den majoor een}{beetje te pesten}\\

\haiku{'k Zal 'm v\'o\'or dat....}{ik wegga de nodige}{instructies geven}\\

\haiku{Toch was het krabben;}{van die nageltjes wel een}{beetje vervelend}\\

\haiku{Ik stapte m'n bed, '.}{uit en zette de deur naar}{t voorvertrek open}\\

\haiku{{\textquoteleft}Daar begrijp ik nou,! '}{letterlijk niks van waar dat}{beest gebleven is}\\

\haiku{Wonderbaarlijke.}{schijnbare verdwijningen}{had-ie meegemaakt}\\

\haiku{maar wij hadden de,.}{smoor in om die wacht om half}{twee die zat ons dwars}\\

\haiku{{\textquoteright} O, man, ik dacht da ',;}{k me beroerd zou lachen}{maar ik hield me in}\\

\haiku{Laat hem 'es lopen, ', '?}{ik moetem toch zien lopen}{voor ikn bod doe}\\

\section{Antoon Thiry}

\subsection{Uit: Ach, de kleine stad...}

\haiku{Zoo 'ne rijke mensch,?}{zou ginder wel algemeen}{bekend zijn niet waar}\\

\haiku{Neen, 't vermoeden,.}{van voor den aap gehouden}{te zijn was verkeerd}\\

\haiku{En 'k vroeg hem toen.}{of hij daar geenen zekeren}{Mister Tuits kende}\\

\haiku{'t Is nog wel zoo,,,?}{ver niet maar dat komt wel he}{Mitteke-lief}\\

\haiku{{\textquoteright} vroeg de notaris.}{en aanstonds had hij met zijn}{vouwbeen het stuk open}\\

\haiku{Om een oogenblik.}{te peinzen of die kozijn}{stapel-zot was}\\

\haiku{{\textquoteright} zuchtte Stafke en.}{zijn handen beefden als hij}{den brief teruggaf}\\

\haiku{En zie, Stafke, eer,.}{ik aan de deur was waren}{ze al bezweken}\\

\haiku{Na de klucht van dien,.}{kozijn en deze van Fee}{Verbuken de uwe}\\

\haiku{{\textquoteright} En met een leelijk,.}{woord naar den Lode stapten}{ze nijdig verder}\\

\haiku{Hij had ons moeten,!}{verwittigen en daar gaan}{we geenen steek van af}\\

\haiku{Mijn gedachten staan,!}{nu op iets anders op iets}{heelemaal anders}\\

\haiku{Een maand ging voorbij,,,}{een tweede een derde en}{men dacht niets anders}\\

\haiku{{\textquoteleft}Hang mijnen overjas',,.}{en mijnen hoed goe weg he}{dat niemand hem ziet}\\

\haiku{{\textquoteright} herhaalde hij steeds,.}{binst hij zich langzaam van zijn}{zot-pak ontdeed}\\

\haiku{Ze zouden hem ferm.}{in zijn bottigheid hebben}{laten staan blinken}\\

\haiku{En de weinigen ', '?}{diet zagen wilden ze}{t wel begrijpen}\\

\haiku{Ge weet het zoo goed,,,.}{als ik dat is alles voor}{ons voor u voor mij}\\

\haiku{De koning, wat zeg,,,, '.}{ik de keizer neen meer nog}{de god vant bier}\\

\haiku{Maar zoo 's avonds zie, ', ',!}{alst stil is zoone pot}{of drie vier daarvan}\\

\haiku{{\textquoteleft}Als ik u kon doen, ',,}{krimpenk deed het direct}{Gust-jongen}\\

\haiku{Maar ja, lang duurde,,:}{het niet geen twee maand of weer}{kwam er muziek in}\\

\haiku{{\textquoteright} bromde Gust terwijl.}{hij hun twee slaphangende}{vingeren reikte}\\

\haiku{En Piet-hierzie, die.}{beweert dat het de zwaarste}{vent uit de stad is}\\

\haiku{Ge moogt mij gelooven, '!}{er loopen er int land}{niet veel rond lijk gij}\\

\haiku{En wij die hem al,,!}{op weg zagen tusschen twee}{Cellebroers naar Gheel}\\

\haiku{Wat ruiten kapot,.}{en ook wat pannen doch dat}{was van het minste}\\

\haiku{Met een wip was de.}{andere de keuken uit}{en achter den toog}\\

\haiku{Een stomme brief van.}{de dikken uit Brussel was}{er de oorzaak van}\\

\haiku{Heel de stad en ook.}{menschen van buiten de stad}{kwamen er naar zien}\\

\subsection{Uit: De droomer}

\haiku{Hoe schoon en heilig!}{lag de tuin daar nu in het}{stille witte licht}\\

\haiku{Waarom was Vader?}{altijd zoo streng en was hem}{niets toegelaten}\\

\haiku{Ze gingen buiten.}{en de deur sloeg met een bons}{dicht achter de twee}\\

\haiku{Als de krant kraakte.}{kromp hij ervoor ineen als}{een hond voor den stok}\\

\haiku{Michiel liep daar lijk.}{dronken en hij struikelde}{alle vijf voeten}\\

\haiku{Michiel voelde dat.}{hij ging wegduizelen als}{hij zich niet sterk hield}\\

\haiku{Langs de fabriek om,,.}{die verlaten was op dit}{uur ging hij naar huis}\\

\haiku{Ja, hij zag het nu,... {\textquoteleft},{\textquoteright}.}{Vader wist er niets vanLees}{Michiel zei Vader}\\

\haiku{Al spelend wandelt.}{hij tot aan het vensterken}{en lonkt den tuin in}\\

\haiku{Michiel was blij iets.}{gevonden te hebben om}{naar te luisteren}\\

\haiku{En rillend liep hij,.}{door den regen die hem nat}{sloeg tot op zijn hemd}\\

\haiku{Ja, wat moesten wij er,.}{mee doen nu dat er geen plaats}{meer is op het Hof}\\

\haiku{Hij lei haastig de}{viool op het kastje v\'o\'or}{den witplaasteren}\\

\haiku{Want morgen is het.}{Vrijdag en dan wordt de kant}{binnengedragen}\\

\haiku{Wie had toen kunnen?...}{peinzen dat al die muziek}{voor haar bestemd was}\\

\haiku{Het doet haar beven.}{van geluk en brengt warme}{tranen in heur oogen}\\

\haiku{Ze legt de handen.}{op het hert en kijkt hem lang}{en gelukkig aan}\\

\haiku{Doch zijn hert bedroog.}{hem telkens en alles bleek}{aleven nutteloos}\\

\haiku{Thans, aan den noendisch,.}{gunde ze zich amper een}{vogelenbeetje}\\

\haiku{Michiel voelde het,...}{toen begon er iets van zijn}{droom te verzwinden}\\

\haiku{Een stap klonk op straat,.}{een deur bonsde en ievers}{schreeuwde er een kind}\\

\haiku{Hij wou recht staan, doch,.}{Eveline die tegenover}{hem zat was hem voor}\\

\haiku{En wie had het hem?}{toevertrouwd waar men Agnes}{naar toe gevoerd had}\\

\subsection{Uit: Gasten in het huis ten halven}

\haiku{Hij zweeg, kromp angstig.}{ineen en stak smeekend de}{handen naar hem uit}\\

\haiku{Geloof me, 'k ga...{\textquoteright}.}{wel vanzelf Piet liet toen zijn}{lawijd maar zakken}\\

\haiku{Als familie soms,}{begon over dat kleedsel dat}{het toch niet noodig was}\\

\haiku{Zelfs voor de stoep en.}{de ruiten nam hij een vrouw}{uit de geburen}\\

\haiku{Heel de stad kon niets,.}{anders denken als ze van}{zijn kuren hoorden}\\

\haiku{uiteendoen waren.}{ze den sigarenmaker}{zelfs zoo gaan heeten}\\

\haiku{Van een inzinking.}{was er aan den Generaal}{niet d\'at te merken}\\

\haiku{'t Kon niet anders,.}{of hij had schulden en hij}{maakte er nog bij}\\

\haiku{{\textquoteleft}Ze zijn er hier nog,!}{niet voor opgewassen voor}{chique dingen}\\

\haiku{hielp hij de dames '}{die om handwerkjes kwamen}{met raad en daad bij}\\

\haiku{een lieke in den.}{mond en locht wippend van stap}{als een dans bijna}\\

\haiku{{\textquoteleft}'k Geloof dat er!}{in dat schoone huisje nog}{wel wat plaats over is}\\

\haiku{een model van een,.}{weg effen en fijn om er}{op te rolschaatsen}\\

\haiku{'t Zullen er wel,,.}{zeldzame zijn uit vreemde}{landen vermoed ik}\\

\haiku{Hij kon er niets aan,.}{doen maar toen schoot hij toch in}{een lach die helmde}\\

\haiku{Och, zoo geerne werd,.}{hij onderwijzer niet om}{te gelooven bijkans}\\

\haiku{Waar dat alzoo met,.}{Pinneke naartoe moest was}{moeilijk te voorzien}\\

\haiku{En dat nogal wel!...}{nadat ze hem framasson}{hadden gemaakt}\\

\haiku{De Muzen{\textquoteright}, opdat.}{ze zich met hun eigen oogen}{konden overtuigen}\\

\haiku{Dat kan niet anders, ' '!}{t zit er in gebakken}{ent moet er uit}\\

\haiku{'t Waren echter.}{geen gewone artikels}{die hij daarin schreef}\\

\haiku{En 't waren me '!}{de mannekens die hij aan}{t woord liet komen}\\

\haiku{Een stuk dat te Lier!}{speelt en dat zouden ze niet}{mogen opvoeren}\\

\haiku{Want Mijnheer Michel,. '}{was niet uitgezongen na}{die drie vier eerste}\\

\haiku{Heur oogen pikten, ze.}{had pijn in de knie\"en en}{steken in den rug}\\

\haiku{Mijnheer Pastoor heeft' '.}{mij gezegd dak daarvoor}{bij u moest komen}\\

\haiku{En als die eerste.}{drie stoelkens af waren kreeg}{hij er andere}\\

\haiku{Doch van heur eigen,.}{werk hield ze hem angstvallig}{af zooveel het kon}\\

\haiku{Dat wordt nu alle,!}{jaren schooner en schooner  ulie}{Kindeke-Jezus}\\

\haiku{En 't was wel erg,.}{voor Mijnheer Lampaert maar}{niemand vond het goed}\\

\subsection{Uit: In ''t hofken van Oliveten' en VII andere verhalen van simpele menschen}

\haiku{{\textquoteright} Begijntje Bellijn.}{bleef binnenshuis en liet heur}{eigen niet meer zien}\\

\haiku{Voor elken stap die,,}{hij deed trok hij zijn grooten}{grofgeschoeiden voet}\\

\haiku{Zelfs werd hij toen lid.}{van de Congregatie van}{den H. Aloysius}\\

\haiku{Maar 'k bedwong me.}{en mijn lach verging in drie}{zuchtjes door mijn neus}\\

\haiku{En hoe komt het, denkt,?}{ge dat ik alles geloof}{wat in dit boek staat}\\

\haiku{Corneel voelde zich.}{zeer klein in deze groote kerk}{en hij boog het hoofd}\\

\haiku{Maar bij 't eerste,.}{woord over die zaak heeft Leo hem}{buiten gesmeten}\\

\haiku{'k Laat onzen hond ',!}{los als hij nog opt erf}{durft komen zeit Leo}\\

\haiku{* * * ~ En de zoete.}{maand van Mei heerschte wit}{en groen over de streek}\\

\haiku{Corneel zat in de.}{lommer van een kriekenboom}{aan den rand der gracht}\\

\haiku{Van wat er in zijn.}{huis gebeurde werd menig}{wonder ding verteld}\\

\haiku{Mijnheer Cosijn het.}{gevreesde huis achter de}{kerk zag binnen gaan}\\

\haiku{Hij zag hem zitten,,.}{onduidelijk in den smoor}{maar hij zag hem toch}\\

\haiku{De stilte streek neer,.}{uit de hooge ijle lucht en}{de nacht was op komst}\\

\haiku{Morgen avond zal ik{\textquoteright},.}{er voort over peinzen zei hij}{tegen zijn eigen}\\

\haiku{Hij had het op een}{tombola gewonnen en}{hij hield z\'o\'o veel}\\

\haiku{Hij stond recht, stak zijn,.}{bril weg zette zijn hoedje}{op en sprak weerom}\\

\haiku{Waarom was hij geenen:}{herder geweest tot wien de}{engelen zeiden}\\

\haiku{Toen was het dat op.}{verwijderde plaatsen de}{klokken nood klepten}\\

\haiku{Een stoelken kwam door '.}{t deurgat gedreven en}{stiet tegen zijn kop}\\

\haiku{En dat verhoedde,.}{de hemel want zijn tijd moet}{aleerst nog komen}\\

\haiku{Meestal is het.}{in het stadje zoo stil lijk}{op een Begijnhof}\\

\haiku{De vrouw lag op 't, '.}{sterven naart scheen uit het}{gesprek bij den toog}\\

\haiku{Dien nacht passeerde...}{onder mijn venster de bel}{eener bediening}\\

\haiku{Natuurlijk lachten,.}{de menschen hem en zijn werk}{erbij vierkant uit}\\

\subsection{Uit: Izegrim}

\haiku{{\textquoteright} {\textquoteleft}'k Spreek niet van 't,{\textquoteright}.}{Menneke-op-demaan}{sprak Trintje driftig}\\

\haiku{Wat er daar binnen, '.}{in dien jongen omging geen}{mensch diet verstond}\\

\haiku{En dan trok hij een:}{gezicht zoo venijnig en}{duvelsch da'k soms dacht}\\

\haiku{dat was 's nachts top.}{in den donkeren door den}{buiten gaan zwerven}\\

\haiku{Hij nam mij zoo in,}{beslag dat ik den dag van}{zijn jaargetijde}\\

\haiku{{\textquoteright} En zich tot den baas,:}{wendend achter zijnen toog}{kommandeerde hij}\\

\haiku{Wa' gaat hij daar toch'!}{doen en wa gaan we nu nog}{te hooren krijgen}\\

\haiku{En maakt er nu maar, '...}{rap komaf vank wil er}{dezen avond nog in}\\

\haiku{En ik, ik staan nu, ''...}{hier enk weet waarachtig}{ni wa beginnen}\\

\haiku{{\textquoteleft}Zie, Jos, als 't voor', ' '!}{u ni wask liet u staan}{enk trok naar huis}\\

\haiku{Voorloopig zei.}{hij nog niets te huis van wat}{er aan den gang was}\\

\haiku{Hij sloeg wit uit en.}{begon te beven terwijl}{hij het briefje las}\\

\haiku{Wat dien dolleman}{in zijn bol geslagen was}{om hier te komen}\\

\haiku{De Izegrim zei weer,,,.}{niets geen ja geen neen knikte}{zelf niet eens en ging}\\

\haiku{Ge gaat toch zoo geen?...}{patatten gaan schillen en}{roo-koolensnijen}\\

\haiku{Maar ineens met een,.}{stoot van zijn rooien kop deed}{hij een stap voorwaarts}\\

\haiku{{\textquoteright} ~ En 't was de.}{Vader niet alleen die dien}{dag zoo oordeelde}\\

\haiku{da's 't begin van ',.}{t ende hij trekt er vast}{en zekers weer uit}\\

\haiku{Door de hel gaan ik.}{er mee en geen haarke zal}{er aan miskomen}\\

\subsection{Uit: Meester Vindevogel}

\haiku{dat de menschen er....}{niet al te veel van merkten}{van het gebroddel}\\

\haiku{hier allemaal al.}{ronddraaide en wie er nog}{de trappen opkwam}\\

\haiku{De schrik zat er al}{zoovele dagen in en}{die klapte hij er}\\

\haiku{de stond dat Marus.}{recht rees en tegen zijn glas}{om stilte tikte}\\

\haiku{Da' zulde gij ook',!}{wel wete Meester gij die}{hier groot zijt gebracht}\\

\haiku{Waaromme toch had hij!....}{dat zelfde aan Lieneke}{ni mogen geven}\\

\haiku{Want pas was hij dan '}{ieverans buiten oft}{zelf-ver wijt}\\

\subsection{Uit: Mijnheer pastoor en zijn vogelenparochie}

\haiku{Maar om bezig te,.}{zijn met den grond om den grond}{te  zien geven}\\

\haiku{Zonder uw preeken!}{waar heel de stad zoo geerne}{komt naar luisteren}\\

\haiku{{\textquoteright} Zoowaar Mijnheer Doktoor:}{ookal was op zijnen}{poot komen spelen}\\

\haiku{Alsof hij al jaar,}{en dag hier was alsof hij}{nooit van zijn leven}\\

\haiku{ze zoo van zelfs open,,.}{in hem zetten zijn hert in}{schoonen witten brand}\\

\haiku{{\textquoteright} dat hem boven het.}{geboor en jachtig geklop}{toegeroepen werd}\\

\subsection{Uit: Het schoone jaar van Carolus}

\haiku{{\textquoteright} en hij keek van zijn.}{brevier naar den brief en zoo}{maar overentweer}\\

\haiku{Van rond Lichtmis had,.}{een breede warme wind de}{verten opengevaagd}\\

\haiku{{\textquoteleft}Eenen moet er in ons,!}{familie toch de dagen}{open houden Petrus}\\

\haiku{Carolus stond er.}{voor getroffen als voor een}{veropenbaring}\\

\haiku{Mijnheer Klabots had.}{reden tot klagen en hij}{sprak daarom zeer veel}\\

\haiku{De gebeeldhouwde ' '.}{poort ent hekken vant}{stadhuis waren toe}\\

\haiku{Voor de menschen in '.}{de kleine stad wast een}{gewenscht Goeieweekweer}\\

\haiku{{\textquoteleft}Maar nu komt de goeie! '!}{tijd toch nog weeromk Ben}{er danig blij mee}\\

\haiku{Treza viel algauw.}{in slaap en er kwam stilte}{in het huizeken}\\

\haiku{Ze sloeg even de oogen.}{eens op en glimlachte toen}{door heur tranen heen}\\

\haiku{Dank om de velden.}{en de boomen die spreken}{van Uwe heerlijkheid}\\

\haiku{Hij vond er geen, zei.}{toen maar wie hij was en vroeg}{brutaal naar zijn wijf}\\

\haiku{Ze is zij met ne,.}{schipper opgestoke en}{de kinders zijn mee}\\

\haiku{Amperkes was de,}{Joppes erover en liep al}{rapper dan zijn beenen}\\

\haiku{k Luisterde ne ':, ','.}{keer enk zeg jat is}{hem g hebt gelijk}\\

\haiku{{\textquoteright} {\textquoteleft}Lucifeir{\textquoteright}, zeg 'k, {\textquoteleft} '{\textquoteright}.}{draag dien beddebak weerom}{ofk doen ik het}\\

\haiku{- Twee dagen later '}{opt uur van den noen als}{de straten vol zon}\\

\haiku{ik gon man wijf en.}{man kindere opsoeke}{van hier tot Hollant}\\

\haiku{Partonneer het ma.}{en duzend kiere bedankt}{veur al a goetheit}\\

\haiku{Maar hij ontweek het.}{antwoord en begon seffens}{over den kapitein}\\

\haiku{De schaterende.}{Pastoor hield de handen op}{zijn schuddenden buik}\\

\haiku{Ze stonden allen.}{recht en staken geroerd hun}{glas naar den leegen stoel}\\

\haiku{{\textquoteright} zei hij, {\textquoteleft}op mijn twee!}{voeten bij den  eenen of}{bij den anderen}\\

\haiku{{\textquoteleft}Wij zijn al te groot,!}{om nog Margrietjeskaarsen}{te branden Juffrouw}\\

\haiku{Het koralen kruis '.}{int putteken van haar}{hals beefde lichtjes}\\

\haiku{Kozijn Duyvewaert.}{wees Carolus monkelend}{naar het ontbloote doek}\\

\haiku{Daar is den duvel,{\textquoteright}.}{mee gemoeid besloot hij}{ten langen laatste}\\

\haiku{stok onder den arm,,.}{deftig de pui af alsof}{er niets gebeurd was}\\

\haiku{Maar, met den besten, '.}{wil van de wereldt was}{al even veel gekort}\\

\haiku{Maar hij hield toch een.}{poos de oogen toe om het beeld}{niet te vergeten}\\

\haiku{Hij schoof ontroerd zijn.}{zetel achteruit en moest}{diep naar asem snakken}\\

\haiku{Mijnheer Duyvewaert.}{werd kwaad lijk een huis en deed}{maar niets dan zuchten}\\

\haiku{'t Water en de.}{lucht en de huizen errond}{zagen er wit van}\\

\haiku{t Ging Carolus.}{als een steek door zijn hart als}{hij dat gewaar werd}\\

\haiku{{\textquoteleft}Ze peinze da we,!}{de dood meebrenge da wij}{de dood bij hebbe}\\

\haiku{{\textquoteleft}Over een dag of tien ',,.}{ist begonnen Menheer}{in de Kerkhofpoort}\\

\haiku{{\textquoteright} zei Petrus met een '.}{natte stem terwijl hijt}{hekken openmaakte}\\

\haiku{Als 't gedaan was,:}{dansten ze er allemaal}{eens rond al zingend}\\

\haiku{de venten waren,}{nu weerom recht gekropen}{drumden allen lijk}\\

\haiku{{\textquoteright} Heur hart brak ervan,}{maar ze hield zich sterk tot hij}{terug op de been}\\

\haiku{Vervoerd sloeg ze toen.}{heur armen rond zijn nek en}{trok zich aan hem op}\\

\haiku{Machtig voelde ze.}{de wijding van het huis over}{heur ziel neerkomen}\\

\haiku{Carolus hief haar,.}{seffens op en droeg haar lijk}{een moeder heur kind}\\

\haiku{De onrust en de '.}{angst sloegen haar zinnent}{onderste boven}\\

\haiku{was het dat Mijnbeer.}{Duyvewaert voor den eersten}{keer het bed verliet}\\

\haiku{Met een snok, als liep,.}{ze tegen een muur bleef ze}{in het deurgat staan}\\

\haiku{Anna-Liza,.}{voelde zich ineens zinken}{rapper en rapper}\\

\haiku{zonder dat ze er,.}{iets tegen doen kan breekt heur}{hart nu opeens open}\\

\haiku{Een groote voldoening.}{komt over hem neer en hij kan}{bijna niet spreken}\\

\subsection{Uit: Voghelen in der muyte}

\haiku{Van den boer hadden ' '.}{zet gehaald en moesten ze}{t blijven halen}\\

\haiku{Veilig geborgen,,?}{te zijn met de zakken vol}{eer het dak inviel}\\

\haiku{Belogen werden!}{ze en bestolen en ze}{zagen er niets van}\\

\haiku{Even stond hij daar met,.}{de armen uiteen en mond}{en  ogen dwaas open}\\

\haiku{En als ge 't wat,.}{beter bekeekt was het niet}{te verwonderen}\\

\haiku{Tante Trees sprak er '.}{met de familie over die}{t heel heel goed vond}\\

\haiku{'t Weten dat Piet,.}{er ook nog was desnoods had}{hem sterk gehouden}\\

\haiku{Een  geluk was}{het voor hem dat Fin zich zo}{stilaan al eens meer}\\

\haiku{{\textquoteright} 't Was zo, hij was,.}{zot op het jongske en dat}{hoe langer hoe meer}\\

\haiku{Niet te geloven}{was het wat hij allemaal}{kocht en welk plezier}\\

\haiku{Zonder uw preken!}{waar heel de stad zo geerne}{komt naar luisteren}\\

\haiku{het kosten zou eer!}{Bienus er een hand zou doen}{naar uitsteken}\\

\haiku{{\textquoteright} dat hem boven het.}{geboor en jachtig geklop}{toegeroepen werd}\\

\haiku{die was verdeeld 'lijk,.}{een damberd  in kleine}{vierkanten perkjes}\\

\haiku{dat was er niet meer, '.}{te vinden in geen twintig}{uren int ronde}\\

\haiku{... 't Brood met hesp en.}{mosterd ging maar moeilijk naar}{binnen bij Baziel}\\

\haiku{Hij was kapot eer'!}{hij t halverwege van}{zijn bezoeken was}\\

\haiku{Morgen terug te ',.}{komen alst klare dag}{was dat deed hij niet}\\

\haiku{Hij sloeg met den piek,:}{van zijn stok eens hard tegen}{de open deur riep luid}\\

\haiku{t Wierd nog schoner '.}{dan hijt van zijn leven}{had durven dromen}\\

\haiku{Hij zette met een.}{klop zijn leeg glas neer op het}{trapke en stond recht}\\

\haiku{Den morgen daarop.}{reeds kwam er een bode met}{een huifkar van daar}\\

\haiku{E\'en kruiske alleen.}{hield zijn hart vast en dat was}{dat van Benooke}\\

\haiku{En gij die vroeger!}{nog geen vogelke in een}{kevieke wilde}\\

\haiku{Ge zijt toch zekers' ',?}{ni bang int dorp zo top}{onder de mensen}\\

\haiku{Van Lewieke zag.}{hij zelfs het tippeken van}{zijnen neus niet meer}\\

\haiku{Zijn hoofd lag op zijn.}{open gazetten-boeken}{en hij glimlachte}\\

\subsection{Uit: De zevenslager}

\haiku{Dat loopen, en dan...!}{die lach op dat oud gezicht}{God in den hemel}\\

\haiku{Waar hij dat alles,.}{haalde wisten de tantes}{en de nonkels wel}\\

\haiku{En niet alleen dat,.}{maar Flipke moest goed in de}{kleeren steken ook nog}\\

\haiku{De doopeling, die zich ',.}{zoo goed alst kon weerde}{vond dit toch wat sterk}\\

\haiku{{\textquoteright} dacht Flipke en zijn.}{hert begon te kloppen toen}{hij aan de beurt kwam}\\

\haiku{Zoo was het dezen.}{morgen geweest en zoo zou}{het nu ook wel zijn}\\

\haiku{Hij luisterde en,.}{sloeg zijn vette rooie handen}{meen van verbazing}\\

\haiku{{\textquoteright} En Flipke was fier.}{geweest en gelukkig lijk}{nog nooit te voren}\\

\haiku{{\textquoteright} zei de meester den,.}{volgenden keer toen het weer}{declamatie was}\\

\haiku{Hij sprak zacht en heel.}{op de letter en zijn stem}{beefde een beetje}\\

\haiku{God de Heer zag dat}{de boosheid der menschen groot}{was op de aarde}\\

\haiku{Dien kan Ik toch niet,{\textquoteright} {\textquoteleft} '.}{mee vernietigen zei O.L.H.}{zoonen braven mensch}\\

\haiku{Hij stapte er op,,,.}{en avant ze dreven den}{hemel uit naar hier}\\

\haiku{En als het daar was,.}{zwommen ze naar de hooger}{gelegen huizen}\\

\haiku{{\textquoteleft}Ik geloof dat die,!}{meester al even zot is als}{gij Zevenslager}\\

\haiku{Er waren er nog,.}{die vloekten maar geeneen zoo}{hard en luid als hij}\\

\haiku{{\textquoteright} dan klonk het telkens}{zoo luid of het vlak achter}{hem uit een mond kwam}\\

\haiku{'t Ergste was dat.}{het zooveel roeten op de}{eerde had gebracht}\\

\haiku{{\textquoteleft}Calmez-vous,,,!}{Monsieur Mutsaers de grace}{calmez-vous}\\

\haiku{Zoo gaarne had hij.}{moeder eens aangesproken}{over al die dingen}\\

\haiku{En werkt nu maar goed,,.}{Flipke dat ge primus wordt}{van de hoogste klas}\\

\haiku{Hij zat er naast den.}{baas en mocht van meet af aan}{ontwerpen maken}\\

\haiku{Flip verschoot nog wel. {\textquoteleft}?}{het meest van allen toen hij}{dat hoordeWatte}\\

\haiku{'t Ging hem af, hoe,.}{langer hoe beter en zijn}{gasten leerden goed}\\

\haiku{Flipke kreeg er nog.}{een geweldigeren schok}{van dan den eerste}\\

\haiku{En dan plots rood als,.}{vuur waarop hij spottend te}{grinniken begon}\\

\haiku{que tout allait bien.}{et qu'on entendrait bient\^ot plus}{de ses nouvelles}\\

\haiku{Als 't nu nog een,.}{beetje geduurd had was ik}{moeten gaan loopen}\\

\haiku{stond de oude heer.}{daar een wijle als van de}{hand Gods geslagen}\\

\haiku{Want op vader zijn ',:}{vraag voor wanneer hijt zich}{dacht bracht Flip's antwoord}\\

\haiku{W' hebben nog geld,!}{voor drie jaar dat moet er dan}{ook maar af kunnen}\\

\haiku{Maar veel sterker als, '!}{leer hechter van kleur ent}{neemt niets geen vocht op}\\

\section{Johan Rudolph Thorbecke}

\subsection{Uit: Thorbecke op de romantische tour}

\haiku{Wij zullen zien, hoe.}{ik hier en daar met hem en}{zijn systeem klaar word}\\

\haiku{Daarom vertrouw ik,,,,;}{gij mijn ouders blijft nog lang}{nog zeer lang bij ons}\\

\haiku{Ik verdien deze.}{rijkdom der tederste zorg}{en toeneiging niet}\\

\haiku{Van de overige;}{merkwaardigheden heb ik}{nog weinig gezien}\\

\haiku{Het bezoeken der,,;}{opera is gelijk alles}{in Berlijn kostbaar}\\

\haiku{Wellicht heb ik eerst,.}{heden gevoeld hoe lief ik}{mijn vaderland heb}\\

\haiku{Denkt, wijl ik zo schrijf,,.}{daarom niet dat ik somber}{of droefgeestig ben}\\

\haiku{Zo ging Thorbecke.}{in oktober 1820 op reis}{naar G\"ottingen}\\

\haiku{Thorbecke heeft zich.}{ook op zijn wijze met de}{romantiek verzoend}\\

\haiku{Hij was echter geen;}{predikant geworden maar}{in zaken gegaan}\\

\haiku{een studievriend van,.}{jrt en Koolhaas die ook naar}{Berlijn was gereisd}\\

\section{Aegidius W. Timmerman}

\subsection{Uit: Tim's herinneringen}

\haiku{Dit alles werd door,.}{Van Kinsbergen bekostigd}{ook de gebouwen}\\

\haiku{Er werd om half vijf.}{ingespannen en alles}{onder luid gejuich}\\

\haiku{Omdat alles zoo.}{langzaam ging bleef men ook maar}{liever in zijn dorp}\\

\haiku{hadden zij het met.}{hun karige lonen ook}{veel beter dan nu}\\

\haiku{Toen zuchtte hij diep,,...}{glimlachte sloot de oogen en}{werd bewusteloos}\\

\haiku{{\textquoteleft}De liefde voor de,,.}{natuur die jij hebt was ook}{je vader eigen}\\

\haiku{Gewoonlijk werd dit.}{verbod na eenige weken}{weer ingetrokken}\\

\haiku{Zijn godsdienstigheid.}{is m\'e\'er dan oppervlakkig}{en dikwijls blague}\\

\haiku{Nou krijg ik genoeg,.}{te vreten al is het ook}{rijst met torretjes}\\

\haiku{Ik heb er een uur.}{op je staan wachten en ben}{je nageloopen}\\

\haiku{Op mijn vijfde jaar,,!}{las ik vlug en wat meer zegt}{had er plezier in}\\

\haiku{Want ik dorst er mijn}{vader niet over te klagen}{omdat ik vreesde}\\

\haiku{{\textquotedblright} En als de koning.}{in zijn humeur is dan geeft}{hij ze een sigaar}\\

\haiku{Zelfs Hofdijk niet, bij.}{wien toch ook   het vuur van}{het ijzer spatte}\\

\haiku{bont en blauw is 'n,!}{e-pleonasme omdat}{in bont al blauw zit}\\

\haiku{Het gevolg was, dat.}{hij dikwijls pas tegen half}{tien in zijn klas kwam}\\

\haiku{Hij was dat alleen,;}{dan wanneer hij zich in zijn}{eer getast voelde}\\

\haiku{{\textquoteright} Dan zat hij maar stil,}{bij de kachel en liet ons}{wat anders werken}\\

\haiku{Stel je voor dat ik.}{d\'at voor mijn brave vijfde}{niet zou over hebben}\\

\haiku{De eerste smeert men, '.}{over zijn boteram de tweede}{smeertm over de zee}\\

\haiku{Ik herinner mij:}{nog zeer goed zijn woedende}{kijken bij mijn vraag}\\

\haiku{Dit is zoo vaak het.}{lot van leeraren die eenmaal}{wanorde hebben}\\

\haiku{de minsten uit angst,.}{de meesten uit liefde en}{uit respect voor hem}\\

\haiku{Hoe is dat alles!}{tegenwoordig veranderd}{en   verbeterd}\\

\haiku{Hektors Liebe stirbt...{\textquoteright},.}{im Lethe nicht O ik weet}{het allemaal wel}\\

\haiku{Al heel gauw kwam het {\textquoteleft},!}{verzoekAber setzen Sie sich}{doch Herr von Santen}\\

\haiku{Een dezer dames.}{heeft mij een hoogst penibel}{oogenblik bezorgd}\\

\haiku{En ga dan zugleich,!}{beim Schuhmacher je Sjoenen}{zind niet heel frisch meer}\\

\haiku{Geheel alleen, en.}{keek heel vergenoegd toen de}{deur voor hem openvloog}\\

\haiku{Al het huisvuil werd.}{op de straat gesmeten en}{door hen verslonden}\\

\haiku{Aandoenlijk is de.}{toevoeging van den datum}{van zijn overlijden}\\

\haiku{{\textquoteleft}Nou dank ik je voor:}{je aandacht en er voor om}{verder te pennen}\\

\haiku{Dit is een van de,.}{weinige plaatsen waarin}{Perk over zijn werk spreekt}\\

\haiku{Te zamen schrijden!}{zij over de Regenboogbrug}{ter onsterflijkheid}\\

\haiku{Uit dat vervloekte.}{paperassengesnotter}{en -gesnater}\\

\haiku{Dan zou de eenige.}{gegronde grief tegen hem}{ondervangen zijn}\\

\haiku{Maastricht, Valkenburg,,,.}{Meerssen Simpelveld Rolduc}{en ten slotte Aken}\\

\haiku{Fons vertelde mij, '.}{dat dit int geheel niets}{buitengewoons was}\\

\haiku{{\textquoteleft}Dit is de reden.}{waarom Ibels en Lautrec zoo}{intens in trek zijn}\\

\haiku{Dan was hij even scherp.}{en geestig als hij in zijn}{brieven placht te zijn}\\

\haiku{Toch zag ik Herman.}{nog wel enkele malen}{op een criquetveld}\\

\haiku{In hem was niet de.}{dichter op wiens klankbord mijn}{snaren meetrilden}\\

\haiku{Van Eeden, die in,.}{de buurt woonde werd op een}{afstand gehouden}\\

\haiku{{\textquoteright} ~ Het is meer dan,!}{vijftig jaar geleden dat}{ik hem het eerst zag}\\

\haiku{Herinnert gij u, -!}{nog hoe mijn zoontje thans een}{deftige mijnheer}\\

\haiku{{\textquoteright} volgens den Rector,.}{die op mijn oefeningen}{het oog moest houden}\\

\haiku{Nausika\"a begon,:}{met haar blanke arm met de}{bal te spelen maar}\\

\haiku{je zit nou in de,,{\textquoteright}:}{lucht te kijken en denkt was}{ik maar op straat maar}\\

\haiku{{\textquoteleft}Pardon, meneer, ik {\textquotedblleft}{\textquotedblright}.}{zat te denken wat u met}{moreel bedoelde}\\

\haiku{3 cahiers heb ik,,.}{al af in een week gemaakt}{nu zijn er nog 9}\\

\haiku{Hij sprak ook altijd.}{met veel waardeering over Jou en}{Jacques en Herman}\\

\haiku{Ik zou mij heel erg.}{moeten vergissen als het}{niet Louise heette}\\

\haiku{{\textquoteleft}Meestal kan ik.}{niet meer dan 50 versregels}{per dag vertalen}\\

\haiku{Ik heb toen aan den ():}{man die er bij was geweest}{Van Vriesland gevraagd}\\

\haiku{N\'og zie 'k u, hoe,}{gij met uw forsch gelaat}{op eens verschenen}\\

\haiku{Hij heeft nooit in iets,,.}{geloofd of in iemand ja}{niet eens in zich zelf}\\

\haiku{{\textquoteleft}Dank je wel voor je.}{vriendelijke woorden in}{de NG over de Od}\\

\haiku{Dit is een vraag die.}{men wel voortdurend tegen}{Quack zelf kan richten}\\

\section{Felix Timmermans}

\subsection{Uit: Adriaan Brouwer}

\haiku{Mijn vader had nest.}{gemaakt in een arm straatje}{te Audenaarde}\\

\haiku{Ik heb mij moeten.}{inhouden terwille van}{mijn zondagsche kleeren}\\

\haiku{Het zweet liep van mijn.}{voorhoofd als ik het gerij}{hoorde naderen}\\

\haiku{Ik zal u misschien,,.}{nooit meer zien Isabel door de}{schuld van dien vetzak}\\

\haiku{Zoo wonderzalig.}{goed is dat. Voor vier stuivers}{koopt men den Hemel}\\

\haiku{En daar werd hier en.}{daar nog dapper gevochten}{tegen den Spanjool}\\

\haiku{Een vlammende klop,.}{onder mijn schouder en ik}{rolde de beek in}\\

\haiku{Als 't mijn beurt was,.}{om te zingen werd ik op}{luid bravo onthaald}\\

\haiku{Twee of drie dagen,,.}{nadien voor den noen werd er}{op de deur geklopt}\\

\haiku{Ik wou mij seffens,.}{troosten en wreken met aan}{Isabel te denken}\\

\haiku{Maar die gedachte.}{aan Primula Mia zat als}{een spin op mijn hart}\\

\haiku{Een uur gaans en ge,.}{staat boven op de duinen}{onder u de zee}\\

\haiku{In mijn hart dank ik,.}{den ouden Breugel ik dank}{Rubens en Rembrandt}\\

\haiku{Om d'een schilderij.}{achter d'ander op haren}{dotskop stuk te slaan}\\

\haiku{Hij eerbiedigde,.}{het zoo dat hij er nooit met}{een penseel aan titste}\\

\haiku{Gelijk een bie naar,!}{den honing riekt zoo wil ik}{naar de verf rieken}\\

\haiku{Dan kauwde hij als,.}{op kurk als op iets dat niet}{te verkauwen was}\\

\haiku{en blijven en niets ',.}{int buske leggen was}{om dood te vallen}\\

\haiku{Dan, met het boekje,.}{in mijn hand speelde ik den}{verliefden Floris}\\

\haiku{Van Someren wist:}{reeds dat ik het viezeke}{had uitgehangen}\\

\haiku{Als ze het speelden.}{geloofden ze voor zoolang}{dat ze het waren}\\

\haiku{Nu wijn en oesters,,!}{zalm en reebok die ge door}{een pijp kon zuigen}\\

\haiku{{\textendash} Verdomd, dat ik dat,,....}{wijf toch heb getrouwd of ik}{trok mee mee met u}\\

\haiku{Achter alle leed.}{glinstert er iets dat u van}{schoonheid zuchten doet}\\

\haiku{Op een avond was Joos '.}{nog niet thuis gekomen van}{t Zuiderkasteel}\\

\haiku{En al zingend kunt,.}{ge altijd iemand beter}{bezien dan anders}\\

\haiku{Toen riep Joos, blij om:}{al de eer die ik over zijn}{geitenhaar uitgoot}\\

\haiku{Tusschendoor hoorde.}{ik ook wat zij samen thuis}{over mij vertelden}\\

\haiku{Dat is als vloeken,.}{in een kerk of speeken in}{een wijwatervat}\\

\haiku{maar ik zeg u op,,....}{voorhand ik geloof er niets}{van heelemaal niets}\\

\haiku{Maar geen woord er over,.}{hij kon zwijgen lijk over die}{familiezaken}\\

\haiku{Ik kon er uren, soms.}{halve dagen en heelder}{avonden doorbrengen}\\

\haiku{{\textendash} Met uw werk kunt gij,.}{u zoo rijk maken als de}{zee diep is zei hij}\\

\haiku{Jordaens maakte het,:}{meest lawaai al kon hij dien}{bakker niet uitstaan}\\

\haiku{Komt vrienden luister.}{naar mijn lied wat er nu voor}{wonders is geschied}\\

\haiku{Door koopen en leenen.}{kan iedereen den fijnsten}{vogel uithangen}\\

\haiku{Zij geleek bijna.}{nog een kind en hij was al}{diep in de vijftig}\\

\haiku{Ik versta er geen,.}{kruimel van zoo'n geluk heeft}{nog niemand gehad}\\

\haiku{dat jaar heb ik in,.}{groote klaarte des harten hard}{veel en goed gewerkt}\\

\haiku{Straks kwam dat verken,.}{zat en klam naar huis en zat}{zij met den afschuw}\\

\haiku{Om door de liefde.}{weer altijd en immer meer}{menschen te kweeken}\\

\haiku{Wijs als altijd, liet.}{hij mij zonder verdere}{woorden weer alleen}\\

\haiku{Hij zou de trap niet.}{op gekund hebben zoo had}{hij zich bedronken}\\

\subsection{Uit: Anna-Marie}

\haiku{Als hij haar in 't;}{sterfhuis ontmoette kreeg hij}{een bots van liefde}\\

\haiku{Ik heb er met mijn;}{zuster Rachel zaliger}{altijd naar gewacht}\\

\haiku{Zoudt gij een nieuwe?}{Romeo en Julia op uw}{geweten nemen}\\

\haiku{Onder de poort, v\'o\'or ',.}{t commiezenhuizeke}{stond een hoopke volk}\\

\haiku{{\textquoteright} zei Pirroen ontroerd.}{en daar door verviel al zijn}{wrok voor Livinus}\\

\haiku{Hij hield haar mollig,}{handeken vast streek kalm een}{klisje krullend haar}\\

\haiku{Met half-Mei {\textquoteleft}}{zou zij met hare voedster}{te Brussel zijn in}\\

\haiku{hij hulde zich in,,:}{den rook knorde en ineens}{vroeg hij bevelend}\\

\haiku{de tijd scheen haar wel,.}{wat lang maar des te zoeter}{zou het geluk zijn}\\

\haiku{zij was verheven:}{boven alle woorden en}{hij zei met een zucht}\\

\haiku{eerst als ze voorbij.}{zijn huis waren konden ze}{doen wat ze wou\"en}\\

\haiku{Hij sloot er zijn oogen,.}{voor zuchtte en verachtte}{zijn triestig leven}\\

\haiku{Hij heeft veel verdriet,.}{gehad en nog en daardoor}{is hij een dichter}\\

\haiku{Maar de zoete stem,;}{bleef in haar oor hommelen}{vaag en broksgewijs}\\

\haiku{Een toevlucht waarnaar,.}{ze uit zag maar waaraan ze}{zich niet geven kon}\\

\haiku{Zijn bleek recht gelaat;}{stond als een bleeke bloem wazig}{in de schemering}\\

\haiku{Ik zag haar voor het.}{eerst toen ik te paard weerkwam}{van de vossenjacht}\\

\haiku{Alleen, alleen met,.}{mijn verlangen en ach de}{nachten zijn zoo lang}\\

\haiku{Toch moet gij eenmaal,.}{wederkomen waar gij ook}{dwaalt op vreemden grond}\\

\haiku{{\textquoteright} Na het lied kwam weer.}{de donkere stilte en}{hier en daar een zucht}\\

\haiku{Het was een flinke;}{man die gewoon was in hooge}{kringen te verkeeren}\\

\haiku{En terwijl ze dan}{naar het kleed zocht dat ze zou}{aandoen betrapte}\\

\haiku{{\textquoteright} Aanstonds begon ze,.}{zich te kleeden gejaagd en}{opgewonden}\\

\haiku{Maar hij zweeg lijk een.}{visch en kropte zijn woede}{en zijn verdriet op}\\

\haiku{misschien zou hij door '?}{t vertellen verlicht en}{verhelderd worden}\\

\haiku{Als g' het niet doet... '....}{ist een teeken dat ge}{mij niet gaarne ziet}\\

\haiku{naar {\textquoteleft}den Eenhoren{\textquoteright}.}{moest komen en deze haar}{laatste brief zou zijn}\\

\haiku{waar niets te zien was.}{dan een wakende haan met}{zijn schuwe kiekens}\\

\haiku{{\textquoteleft}Als men er eenen keer,}{van geproefd heeft kan men er}{niet meer afblijven}\\

\haiku{{\textquoteright} Bij elke {\textquoteleft}knol{\textquoteright} ging,.}{er een vuist weg tot er die}{van Pirroen overschoot}\\

\haiku{{\textquoteleft}Mij gebroken voor,...{\textquoteright} {\textquoteleft}???}{u mijnen kop in den schoot}{gelegdHoe wie wat}\\

\haiku{en voelde met een.}{zekere spijt dat met haar}{weer alles goed kwam}\\

\haiku{Hij verzwijgt zijn plan.}{en zou het aan iedereen}{willen vertellen}\\

\haiku{En buiten in den,.}{nacht hoort hij het zoetekens}{malzig regenen}\\

\haiku{In zulken roes schrijf,.}{ik dan naar u en dat duurt}{lang mijn geliefde}\\

\haiku{Wat is mij pijn en!}{ellende als mijn hart maar}{licht van vrede is}\\

\haiku{Moet ik dat met bloed?}{en vuur schrijven om U dat}{doen te begrijpen}\\

\haiku{{\textquoteright} {\textquoteleft}Als er een van ons,.}{gedrie\"en niet sterft worden}{wij slechte menschen}\\

\haiku{Het beekje gaf soms.}{een zilveren geluidje}{in de stilte}\\

\haiku{Onderwegen bidt;}{ze verstrooid dat hij het niet}{zou gezien hebben}\\

\haiku{Maar 'k doe het niet, ',.}{neenk doe het niet al stond}{alles op zijn kop}\\

\haiku{Ge weet zelf wat ik......{\textquoteright}}{zeggen wil toen ik uwe hand}{in de zijne zag}\\

\haiku{{\textquoteleft}Weet ge 't nog dat?}{ik u vroeger eens gevraagd}{heb om te trouwen}\\

\haiku{{\textquoteright} Zijn voorhoofd was nu.}{heelemaal als bestikt met}{dikke zweetperels}\\

\haiku{Hij ging naar haar toe,,.}{de armen open gereed om}{haar te omhelzen}\\

\haiku{Met dezen dronk dus,,!}{beste vrienden zeg ik u}{vaarwel en adieu}\\

\haiku{Dat was nu heel veel,}{jaren geleden en hij}{vond het later}\\

\haiku{De doktoor trok zijn.}{linkermondhoek in rimpels}{en schuddebolde}\\

\haiku{Dat kon hij zich zelf,?}{niet voorstellen en dat vroeg}{hij immers ook niet}\\

\haiku{Hij ging bij Pirroen, '.}{niet meer eten kwams avonds niet}{meer in den Dolfijn}\\

\subsection{Uit: Boerenpsalm}

\haiku{Een boer moet een boer,.}{blijven anders verstopt de}{gang van de wereld}\\

\haiku{Zij sprong op voor haar,.}{kleed en haar boterhammen}{vallen op den grond}\\

\haiku{Ik had noch rust, noch, '}{duur en als ik het gedaan}{kon krijgen trok ik}\\

\haiku{Daar zitten haar broers,,,.}{wel vijf en haar pere een}{vent lijk een pilaar}\\

\haiku{Mijnheer pastoor zegt. '}{dat de sterren zoo groot als}{wereldbollen zijn}\\

\haiku{Het leven is geen,,.}{lach zei hij maar uw varken}{is een binnenbeer}\\

\haiku{Ik hield mijn hart in}{mijn handen en ik vergat}{de overstrooming}\\

\haiku{Ze houden u arm.}{en metselen u in een}{toren van kommer}\\

\haiku{Voor een millioen,}{wilt ge er geen  enkel}{kwijt ge zoudt er voor}\\

\haiku{emmers zweet, blaren,.}{op uw handen korstknie\"en}{en later een bult}\\

\haiku{Toen gaf ze mij een,.}{porseleinen koffiepot}{nog een trouwcadeau}\\

\haiku{- Hoort ge Franelle,.}{de Wortel wil mij  voor}{een dief doen doorgaan}\\

\haiku{Van alle kanten.}{loert het leven om u een}{pee te steken}\\

\haiku{En daar hoor ik hen (!)....}{vertellendat ik daar juist}{moest op uitkomen}\\

\haiku{Zijn schoone ziel, die,.}{in zijn woorden brandde heeft}{mijn hart opengedaan}\\

\haiku{Wortel, omdat gij.}{mij meer betrouwt buiten den}{biechtstoel dan er in}\\

\haiku{Ik probeer met de,.}{maan altijd goed te staan ge}{moet haar leeren kennen}\\

\haiku{- Zijt ge nu zot, riep,!}{ze tegen ons Fien van daar}{zoo naar staan te zien}\\

\haiku{Ons Fien kunt ge maar.}{niet wijs maken dat het de}{schuld van de maan is}\\

\haiku{We hebben alles,.}{geprobeerd beewegen en}{medicamenten}\\

\haiku{Als 't koren dan '.}{eindelijk in zijn schooven staat}{ist dorpskermis}\\

\haiku{Weer thuis, gaat onze,.}{Fransoo de hoeven af met}{zijn bedelzaksken}\\

\haiku{We doen ons verken,,}{dood en ge weet niet hoe hij}{het weet maar mijnheer}\\

\haiku{Vooreerst hebt ge ze.}{alles gegeven wat in}{uw vermogen was}\\

\haiku{Hun geluk is het,.}{uwe hun verdriet snijdt dieper}{bij u dan bij hen}\\

\haiku{Als er een pan van ',,.}{t dak valt wees gerust ze}{valt op mijnen kop}\\

\haiku{En toch bleef ik soms.}{tuschen de spleet van de deur}{naar haar staan loeren}\\

\haiku{Ik kon het niet meer:}{houden en ik riep als voor}{een groote zaal vol volk}\\

\haiku{Haar vent lag daar als,.}{een versleten keerborstel}{zat en lam te bed}\\

\haiku{'t Is uw eigen '.}{bloed ent roept nog harder}{als het tegenslaat}\\

\haiku{Is dat uwe smart, o,?}{Heer die zoo zwaar is dat wij}{moeten meehelpen}\\

\haiku{Want ik denk op de.}{mijne meer dan op die van}{U. Vergeef het mij}\\

\haiku{Het is zoo schoon, en ',.}{t is zoo stil dicht bij en}{heel in de verte}\\

\haiku{Als het regent en.}{de vruchten blinken en ge}{uitlekt als een hond}\\

\haiku{Ik vertel haar heel.}{mijn wedervaren tot aan}{het zien van het lijk}\\

\haiku{Maar ook daarom wordt.}{zij van Onzen Lieven Heer}{zoo gaarne gezien}\\

\haiku{Ja, het was toen de,.}{verschrikkelijkste zomer}{dien ik gekend heb}\\

\haiku{In de Nethe kon.}{men bij hooge tij wat stinkend}{slijkwater halen}\\

\haiku{Een liter water.}{had bijna zooveel waarde}{als een liter melk}\\

\haiku{Maar 't was of de '.}{lucht ent oor van God ook}{uitgedroogd waren}\\

\haiku{Hewel ik ben niet,!}{bang ik wil eens laten zien}{dat ik een man ben}\\

\haiku{Ik voel rond mij een,.}{vreemde macht die het op ons}{leven gemunt heeft}\\

\haiku{En toen heb ik ook:}{mijnen kop gebogen en}{gelaten gezegd}\\

\haiku{Ze mocht niets zeggen.}{van den doktoor en niemand}{mocht haar iets vragen}\\

\haiku{Wat baatte het dat,?}{zij zweeg om een of twee uur}{langer te leven}\\

\haiku{Bij een onweer had,.}{ze altijd geren dat ik}{thuis was dicht bij haar}\\

\haiku{Och, hoe kon ik zoo'n.}{goed mensch eens vergeten voor}{die meid met den stier}\\

\haiku{Dan voelt men dat men.}{oud wordt en het leven als}{een smoor voorbijgaat}\\

\haiku{Nu wist ik het, dien.}{Jesus zou ik op het graf}{van ons Fien zetten}\\

\haiku{Zulk werk is goed voor,.}{mannen als onzen Fransoo}{den minderbroeder}\\

\haiku{- Ik geloof het ook,, -.}{zeg ik dan en ik ben er}{bijna zeker van}\\

\haiku{Ja, dat was veel schooner.}{en als men het fijn naging}{ook zoo moeilijk niet}\\

\haiku{Ik was zelf verbaasd,.}{over mijn verzinsel alsof}{het niet van mij kwam}\\

\haiku{Ze kwam voorbij om,.}{een rok te halen dien ze}{gerepareerd had}\\

\haiku{Gelukkiglijk zijn.}{dat de laatste kuren van}{Mijnheer de Winter}\\

\haiku{Frisine geeft zich.}{meer moeite voor den wasch en}{voor het huishouden}\\

\haiku{Ik lig soms op  ,,.}{de loer in huis of buiten}{ik achtervolg haar}\\

\haiku{en die andere.}{zorg van achterdocht was er}{niet eens mee verlicht}\\

\haiku{Die stilte en dit!}{zwijgen er rond maken mij}{nog wanhopiger}\\

\haiku{Ge kunt voelen hoe.}{laf en plat die woorden bij}{mij er uit kwamen}\\

\haiku{Maar 's anderdaags}{zat mijn hof vol slekken en}{bij den Ossenkop}\\

\haiku{- Als die eerst bij mij,.}{waren zult gij ze er wel}{opgezet hebben}\\

\haiku{maar ze zou 't nu,.}{niet doen daar was ze immers}{te geslepen voor}\\

\haiku{Zoo lag ze daar vier,.}{dagen te blaken zonder}{\'e\'en woord te zeggen}\\

\haiku{Wat is er dan aan,?}{mij aan dat de vrouwen mij}{zoo gaarne hebben}\\

\haiku{Ik doe mijn oogen toe,.}{als ik op Angelik denk}{om haar niet te zien}\\

\haiku{Ik zat in den stal,.}{toe te zien van tusschen de}{pooten van ons koei}\\

\haiku{of er iemand meer....}{is of minder en of er}{iets in den weg staat}\\

\haiku{- Ge kunt den pastoor,,....}{zeggen zuchtte ze dat ik}{toestem op uw vraag}\\

\haiku{Ik moet het toch  ,?}{doen waarom dan leelijke}{gezichten trekken}\\

\haiku{Z'is blind, en 't is.}{misschien daardoor dat ze zoo}{goed en vroolijk is}\\

\subsection{Uit: Driekoningentryptiek}

\haiku{- Savez-vous?}{pourquoi nous avons tout donn\'e}{\`a ces pauvres gens}\\

\haiku{En of ge dat aan,.}{arme menschen doet of aan}{God dat is eender}\\

\haiku{Elle arrivait.}{tout droit sans se soucier des}{haies ni des chemins}\\

\haiku{Il prit l'\'etoile.}{et suivit l'enfant nimb\'e}{d'arc-en-ciel}\\

\haiku{Doch hij zag nog eens,.}{om naar zijn schapen die zoo}{meewarig blaatten}\\

\haiku{{\textquoteright} Suskewiet deed het,,.}{hek open en allen volgden}{dicht bijeen gedrumd}\\

\haiku{A force de les,.}{saluer chaque jour il les}{connaissait toutes}\\

\haiku{lui qui autrefois,.}{vivait au jour le jour plein}{de contentement}\\

\haiku{ni trop br\^ulant ni trop,.}{lourd il ne le d\'edaignait}{pas et l'emportait}\\

\haiku{Schrobberbeeck was nu,,.}{een heel andere vent van}{binnen in zijn hart}\\

\haiku{Hij had nu geen schrik,.}{meer hij verlangde nog naar}{zoo'n hooge momenten}\\

\subsection{Uit: De familie Hernat}

\haiku{{\textquoteright} en Mevrouw sloot de.}{oogen om even in een verren}{droom te verzinken}\\

\haiku{En hij hoorde de,!}{klanken het heerlijkste lied}{van zijn leven}\\

\haiku{En ik heb daar nog!}{een hemelbier alleen voor}{de beste vrienden}\\

\haiku{Het was een felle,,.}{roekelooze liefde die ze}{voor Stefan voelde}\\

\haiku{Hij dierf niet in de.}{klare oogen van Henriette}{te voorschijn treden}\\

\haiku{Toen slikte Mevrouw,,.}{Lorier-Frisijn zich offerend}{haar verdriet weer in}\\

\haiku{Als de Baron iets,.}{vertelde was er geen speld}{tusschen te krijgen}\\

\haiku{Het ligt soms aan zoo,.}{weinig tenminste zoo van}{buiten af gezien}\\

\haiku{Wat een comedie....?}{is er nu eigenlijk bij}{de Loriers gespeeld}\\

\haiku{In dat water had,.}{ze eens gewenscht Henriette}{te zien verdrinken}\\

\haiku{En bij drink-.}{en smulpartij schoot hij steeds}{den hoogvogel af}\\

\haiku{Dan huiverden de.}{haren op zijn handen recht}{van ontsteltenis}\\

\haiku{Doch hij riep het niet,.}{tegen God niet tegen het}{Lot of het Leven}\\

\haiku{{\textquoteright} {\textquoteleft}Dan zal ik er mijn,,.}{broeder lichtaanbrenger ten}{zeerste om danken}\\

\haiku{Haar vader had in.}{het begin opgespeeld en}{haar afgerammeld}\\

\haiku{Ze wou alleen maar.}{zijn lief zijn en daar was ze}{al gelukkig mee}\\

\haiku{Anders loopt het scheef.}{en ze sollen met u als}{een kat met de muis}\\

\haiku{Het zien van al die.}{dingen was voor hem reeds een}{rijkdom op zichzelf}\\

\haiku{En terug aan de,.}{feesttafel zat Annette}{tegenover Simon}\\

\haiku{Men wierd gewaar, dat.}{zij op het kasteel veel ging}{te zeggen krijgen}\\

\haiku{Wie had dat ooit voor,?}{mogelijk gedacht dat ik}{met u zou trouwen}\\

\haiku{Onze Turk is nog,.}{zoo onnoozel niet als wij ons}{voorgesteld hebben}\\

\haiku{Ruytenbroeckx was een.}{man van kennis in het fruit}{en stoefte er mee}\\

\haiku{Maar de verbazing,....}{verlamde hem toen zij zijn}{hand aan haar borst bracht}\\

\haiku{Zelfs Ruytenbroeckx had.}{er toen nog niet het minste}{vermoeden van}\\

\haiku{Maar Lucie sneed en.}{kerfde heftig zijn flauwe}{beloften kapot}\\

\haiku{Hij wou die {\textquoteleft}zaak{\textquoteright} van.}{Anna-Lise tot in}{de puntjes weten}\\

\haiku{Hij pinkte schelmsch:}{naar haar opgeroepen beeld}{en zei tot zichzelf}\\

\haiku{Hij kleedt zich snel, werpt.}{voor de gelegenheid zijn}{witte burnoes om}\\

\haiku{{\textquoteright} vroeg ze aan Simon,.}{die in het midden van de}{leeszaal beeldstijf stond}\\

\haiku{De knechten hebben}{hem in den vroegen morgen}{hooren wegrijden}\\

\haiku{Anna-Lise,,?}{hebt ge er geen spijt van dat}{hij weggegaan is}\\

\haiku{Anna-Lise.}{kwam bij hem na een boodschap}{te Nivesdonck}\\

\haiku{Lucie zat op wraak!}{te zinnen en die wraak zou}{niet voor de poes zijn}\\

\haiku{En het ergste dat.}{ze overal den lof hoorde}{van die indringster}\\

\haiku{{\textquoteleft}Wij mogen dat de.}{nagedachtenis van zijn}{vader niet aandoen}\\

\haiku{Ik heb u daar juist.}{toch al gevraagd of ze u}{niets gezegd hebben}\\

\haiku{Van uw dochterke?}{Annette en den jongen}{uit De Regenboog}\\

\haiku{Ik ben daar juist aan,.}{Sidonie gaan zeggen dat}{ik niet kan komen}\\

\haiku{{\textquoteleft}Stommerik, seffens.}{komt hij nog met kar en paard}{binnengereden}\\

\haiku{De \'e\'ene beloert.}{den andere om hem in}{de klem te krijgen}\\

\haiku{En het kwam omdat {\textquoteleft}{\textquoteright}.}{de lange Vereecken geen}{spraak in zijn oogen had}\\

\haiku{En na lang zwijgen,,,:}{zei de rentmeester eenigszins}{smalend tot Simon}\\

\haiku{Ga twee dagen op,,}{reis tegen dat gij weerom}{zijt zit zij terug}\\

\haiku{Nu versta, begrijp....}{en vergeef ik ook de fout}{van uw dochterken}\\

\haiku{Gun die hyena's!}{toch het genoegen niet dat}{zij over mij juichen}\\

\haiku{Het bloed steeg hem met,.}{gulpen terug naar het hoofd}{rood om te bersten}\\

\haiku{Hij keerde zich om,,.}{verbaasd verbolgen om den}{durf van Philipien}\\

\haiku{Ze kwamen terug....}{uit den tabakswinkel met}{vier kistjes Havanas}\\

\haiku{{\textquoteright} en na {\textquoteleft}aux fonds des{\textquoteright}.}{bois een rits van zijn duim over}{de mandoliensnaren}\\

\haiku{Doch zij zou alle,....}{moeite doen desnoods om het}{toch te overbruggen}\\

\haiku{Zij vertelde over,,.}{zijn kunde over zijn talent}{zijn galanterie}\\

\haiku{Zouden er dan soms,?}{twee Stella mia's zijn twee}{Anna-Lise's}\\

\haiku{{\textquoteright} Bij de woorden van.}{Isabella was het koud zweet}{hem uitgebroken}\\

\haiku{Na lang zoeken vond '.}{hij hem aant wandelen}{op den Nethedijk}\\

\haiku{{\textquoteright} ~ Als hij in den,:}{avond naar huis slenterde zei}{hij kwaad tot zichzelf}\\

\haiku{Hij verstoutte zich,.}{zelfs een goeden dag aan de}{boeren te zeggen}\\

\haiku{Er was ineens naar,....}{haar een begeerte die zijn}{rust omverspoelde}\\

\haiku{{\textquoteright} En daarna ging de.}{lange Vereecken weer in}{zijn eigen kantoor}\\

\haiku{Doch ik kreeg de eer.}{en het geluk niet meer U}{nog te ontmoeten}\\

\haiku{Ge weet Verhaegen.}{doet zoo iets zeer handig en}{is daarbij niet duur}\\

\haiku{Rentmeester Adriaen, die,.}{riep de bulten saam al om}{te tribunalen}\\

\haiku{Van verder dan hier,{\textquoteright}.}{is het schot niet gekomen}{beweerde Simon}\\

\haiku{Er gingen eenige.}{schokken doorheen het lichaam}{van Adriaen Ruytenbroeckx}\\

\haiku{Hij ging voor de deur,:}{staan met de armen uiteen}{en riep wanhopig}\\

\haiku{En Charobin van.}{zijn kant reed nu en dan eens}{naar St. Rochushof}\\

\haiku{Geef dat al maar hier,,.}{cher Ange ik houd dit al}{bereid op mijn hart}\\

\haiku{Ik zal, dunkt me, zoo....}{gerust zijn onder den grond}{met die dingen aan}\\

\haiku{{\textquoteright} Octavie had ook,.}{liefde gekend vroeger op}{een ander kasteel}\\

\haiku{Hij zal het zelf zijn,,....}{die met de vrouwen speelt lijk}{de kat met de muis}\\

\haiku{Gij hebt gelijk, mijn.}{wil en mijn kunst zijn er de}{prooi van geworden}\\

\haiku{Hij rook seffens de.}{bedoeling van barones}{Emma de Vara}\\

\haiku{{\textquoteright} en hij zit er een....}{heelen avond in gedachten}{achter te zoeken}\\

\haiku{Er wierd iets wakker,,.}{in hem een nieuw bestaan een}{nieuwe horizont}\\

\haiku{De schrik is hem op,.}{het hart geslagen zegt men}{in Nivesdonck}\\

\haiku{{\textquoteright} vroeg hij en met elk.}{woord voelde hij zich rooder}{en rooder worden}\\

\haiku{Doch kom morgen niet.}{af met het geval van die}{vlek en dien jongen}\\

\haiku{dan ontbrak er iets.}{aan den gang der natuur en}{de overlevering}\\

\haiku{Het was dan of die.}{honderden wezens voor den}{tweeden keer stierven}\\

\haiku{Karnol ging van deur.}{tot deur aan den Heikant om}{te laten bidden}\\

\haiku{{\textquoteleft}Heer, blijf bij ons, de, '....}{avond nadert de groote schaduw}{strekt zich overt land}\\

\haiku{hij moest op zijnen,.}{alleen zijn met zijn verwijt}{en verdriet alleen}\\

\haiku{Het kind gaat aan de.}{bank staan met het hoofd in zijn}{handen geborgen}\\

\haiku{ze geweest was in....}{al die voorbije jaren van}{groote zorg en kommer}\\

\haiku{Het meisje kreeg daar.}{koffie met boterhammen}{en speculatie}\\

\haiku{{\textquoteright} vroeg Verhoeven, die.}{toch nog een halve flesch m\'e\'er}{had moeten drinken}\\

\haiku{En Karel-Jan ging:}{zijn hoofd opheffen om te}{roepen naar dien lach}\\

\haiku{Met de intresten,;}{kwam men toch niet meer toe om}{er van te leven}\\

\haiku{En hij straalde ook,.}{omdat het met baas Pittoors niet}{goed ging in de zaak}\\

\haiku{Ook de engelsche,.}{gouvernante bleef zonder}{betaald te worden}\\

\haiku{Dat kasteel was van,}{nonkel Arnold en niemand}{anders dan nonkel}\\

\haiku{{\textquoteleft}Ja, God leidt ons,{\textquoteright} zei,.}{Karel-Jan overtuigd van de}{leuze van hun huis}\\

\haiku{Voor wat dienen die?}{dingen anders dan om ons}{geluk te geven}\\

\haiku{{\textquoteright} {\textquoteleft}Geen gebabbel,{\textquoteright} zei.}{meneer Verhoeven zoo wat}{van uit de hoogte}\\

\haiku{Roselie, die een,.}{weduwe was zou daar zeer}{mee in haar schik zijn}\\

\haiku{Ze placht nu ook veel.}{te bidden voor haar moeders}{zielezaligheid}\\

\haiku{Zoo zal hij eeuwig,:}{blijven leven en kan een}{legende worden}\\

\haiku{{\textquoteleft}Ik slaap nooit meer of.}{die kerel moet eerst v\'o\'or mijn}{voeten dood liggen}\\

\haiku{Zijn moeder vond hem.}{zoo schoon in zijn berouw en}{zelfbeschuldiging}\\

\haiku{Verder wandelde.}{hij veel door de velden en}{las over de sterren}\\

\haiku{Hij zag een nieuwen,.}{grooten plicht een nieuwe taak}{voor zijn oud leven}\\

\subsection{Uit: Ik zag Cecilia komen}

\haiku{Twee ranken, die slechts.}{kunnen opbloeien als ze}{malkander steunen}\\

\haiku{Een maneschil schijnt,.}{in het gladde water als}{een verdronken kroon}\\

\haiku{Een groote vogel wiekt.}{zwaar op en blijft mij schuins van}{uit een boom bezien}\\

\haiku{Ik scheur mij angstig.}{uit die mijmering los en}{vlucht naar beneden}\\

\haiku{Een ding weet ik, ik,.}{ga mij overtuigen ik ga}{mijn liefde redden}\\

\haiku{De slinger van de.}{hangklok flitst telkens in een}{zonnepijltje op}\\

\haiku{De kelderdeur staat,.}{open zware holleblokken}{komen de trap op}\\

\haiku{ik gaf haar schoentjes.}{van maneschijn en sluiers}{van regenbogen}\\

\haiku{Ik voel mij zoo klein,.}{en zoo nietig en onze}{liefde is zoo groot}\\

\haiku{zij kan zondag niet,.}{komen hare moeder is}{zwaar ziek geworden}\\

\haiku{Mijne dagen zijn.}{een mengeling van kwelling}{en verheuging}\\

\haiku{Om mijn opstijgend.}{geluk wetens en willens}{te vernietigen}\\

\haiku{- Maar ik even veel van,,.}{u meer dan van Roelinde}{meer dan van allen}\\

\haiku{Ik zal gelukkig.}{zijn en niemand zal weten}{waarom en waardoor}\\

\haiku{De hofgracht onder.}{hem is als een  effen}{laken van het kroes}\\

\haiku{Zoo kan ik in elk.}{geval de naderende}{dingen uitstellen}\\

\haiku{Niet zoo zeer voor mij,,;}{want ik zal haar niet meer zien}{maar voor haar zelve}\\

\haiku{Doch hij is nog geen.}{honderd met er gegaan of}{angstig keert hij weer}\\

\haiku{- Dat is van geluk,,...}{stamel ik ik ben zoo blij}{u nog eens te zien}\\

\haiku{Want ieder krijgt een.}{nummer van het dienstmeisje}{als hij in huis komt}\\

\haiku{Ik sta van achter.}{de boomen naar het huis en}{naar het kruis te zien}\\

\haiku{En het is meteen.}{alsof ik een lied om het}{dak hoor ruischen}\\

\subsection{Uit: Karel en Elegast}

\haiku{Hij praatte met zijn;}{edellingen en Bisschoppen}{die rond hem waren}\\

\haiku{Hij was ijler dan;}{den adem van een miertje en}{dunner dan de lucht}\\

\haiku{Gaat uit stelen al '.}{ist u nog zoo bitter}{en onaangenaam}\\

\haiku{en loste zich op.}{in een manestraal die}{door het venster stond}\\

\haiku{De Koning meende,;}{buiten te gaan maar zijn angst}{mocht nog niet over zijn}\\

\haiku{hij steeg vlug te paard.}{en rende naar beneden}{naar den platten grond}\\

\haiku{{\textquoteleft}Dat is iemand die.}{zijn weg is kwijtgeraakt en}{hier verloren loopt}\\

\haiku{Ik heb liever dat.}{wij vechten dan dat ik door}{dwang iets zeggen zou}\\

\haiku{De vier ademen van.}{mensch en dier zoefden als de}{wind om een hoog huis}\\

\haiku{{\textquoteright} En daarom sloeg de, '.}{Koning niet weerom en was}{t gevecht gedaan}\\

\haiku{BEIDEN stonden stil,.}{maar menigvuldig waren}{hunne gedachten}\\

\haiku{Ik zal er u den.}{uitleg van geven als gij}{mij eerst uwen naam noemt}\\

\haiku{Is hij van zulke?}{macht dat gij niet anders dan}{bij nacht kunt rijden}\\

\haiku{ik zal u mijn naam.}{zeggen indien ik er u}{mee van dienst kan zijn}\\

\haiku{De schat is zoo groot,}{dat als wij er vijfhonderd}{pond van wegnamen}\\

\haiku{Het zou mij dus maar,.}{slecht vergaan als ik er u}{de weg moest wijzen}\\

\haiku{Het moest toch voor zijn,.}{goed zijn want anders zou God}{het hem niet zenden}\\

\haiku{En met het kruid in}{den mond luisterde hij of}{er van die dieren}\\

\haiku{Dan vreeze ik dat,}{er mij onheil overkomt dan}{twijfel ik er niet}\\

\haiku{wat zou de Koning?}{hier komen doen hier in de}{nacht en zoo verre}\\

\haiku{Zoudt gij dan de taal '?}{van een haan gelooven ent}{blaffen van een hond}\\

\haiku{Zie de mane is,.}{haast niet meer zichtbaar zij zakt}{achter de bosschen}\\

\haiku{tot aan hun sporen,.}{zal het bloed vloeien en bij}{Eggeric het eerst}\\

\haiku{Verders werden er,.}{in de gangen het hof en}{de zalen geplaatst}\\

\haiku{Doch de Koning wreef:}{over zijn zilveren baard en}{sprak tot Elegast}\\

\haiku{Ze ging recht zitten.}{en stak haar aanschijn buiten}{de legerstede}\\

\subsection{Uit: Het kindeken Jezus in Vlaanderen}

\haiku{En Maria zag over.}{het land en voelde tranen}{in de ogen komen}\\

\haiku{Zij zag de wereld.}{door haar geluk en alles}{juichte in haar}\\

\haiku{Waarom sprongen de?}{zilveren vissen telkens}{boven het water}\\

\haiku{{\textquoteleft}'t Is niets, 't is, '.}{niets ik heb het verdiend en}{t zal wel over gaan}\\

\haiku{{\textquoteleft}En,{\textquoteright} voegde ze er, {\textquoteleft};}{nadien bijgij zult Jozef}{gelukkig maken}\\

\haiku{Jozef zag hem, en.}{wilde henengaan om den}{pastoor te mijden}\\

\haiku{en daarom zei ik,}{bij me zelven daar vraag ik}{Jozef zelf eens naar}\\

\haiku{{\textquoteleft}Ja, iets beter dan, ';}{zij zelvek geloof dat}{ik haar zielken zag}\\

\haiku{{\textquoteleft}Die heeft nog iet van,{\textquoteright}.}{zijn voorouder David op}{de tong peinsde hij}\\

\haiku{Aan den horizont,.}{lagen witte wolken als}{sneeuwbergen gevlijd}\\

\haiku{ze draaide, alles.}{zou dan wel eens met Gods hulp}{ten rechte komen}\\

\haiku{Een grijze sluier.}{weefde zich rond de gulden}{blijdschap van haar hart}\\

\haiku{{\textquoteright} {\textquoteleft}Ach, Jozef, 'k weet,{\textquoteright}.}{het niet en haar handekens}{zochten de zijne}\\

\haiku{{\textquoteright} Jozef vertelde.}{hem zijn droeve reis en de}{toestand van Maria}\\

\haiku{Het in puin liggend.}{kasteel bezijds de dorpskom}{getuigde daarvan}\\

\haiku{{\textquoteleft}Filmene,{\textquoteright} gebood, {\textquoteleft}.}{hij de meidgeef hun elk een}{dikke boterham}\\

\haiku{{\textquoteleft}Ik geloof dat in.}{Bethle\"em grote dingen}{zullen gebeuren}\\

\haiku{Hij heeft zijn stemme,.}{gegeven de aerde}{is beroert geweest}\\

\haiku{En ach ik ben zo!}{klein en nietig en als mens}{van gener weerde}\\

\haiku{Ik zal dat van mijn!}{leven niet vergeten al}{word ik honderd jaar}\\

\haiku{Hij stak den haring.}{eens naar omhoog en trok er}{zonder meer van door}\\

\haiku{maar niettemin bleef.}{dit zatte woeste Herodeshoofd}{voor haar ogen hangen}\\

\haiku{'t Volk kwam uit de.}{huizen gelopen en klom}{op de heuvelen}\\

\haiku{Hij ging toen op een}{omgevelden olm staan en}{hief het wicht zo hoog}\\

\haiku{Hij gruwelde van.}{zich zelven en deed zijn ogen}{een stondeken toe}\\

\haiku{{\textquoteright} brulde Herodes, en.}{Sausisken kefte mee}{naar de ministers}\\

\haiku{Eenieder is er.}{vast van overtuigd dat het kind}{gedood moet worden}\\

\haiku{, maar een vraag van den.}{kapitein leidde me van}{het voornemen af}\\

\haiku{Groter als heel de,,!}{wereld en niets dan water}{altijd maar water}\\

\haiku{Ze waren daarstraks!}{zo blij geweest als ze de}{zee gewaarwierden}\\

\haiku{t Is spijtig dat,}{het avond is maar gij zoudt zien}{hoe uitgemergeld}\\

\haiku{{\textquoteright} Maar zij waren reeds.}{den bollewinkel in met}{den wind van achter}\\

\haiku{De uren sleepten zich.}{lui en rustvol door den dag}{die duren  bleef}\\

\haiku{Het gaf hem hogen,...}{moed en jeugd en lust tot veel}{en lang te werken}\\

\haiku{{\textquoteright} {\textquoteleft}'k Zal mijn hoofddoek,{\textquoteright}.}{dieper over mijn ogen trekken}{wedervoer Maria}\\

\haiku{daar zijn de hutten,;}{rond de kerk en ginder de}{brug over de Nethe}\\

\haiku{Hij wist niet meer wat,}{zeggen en terwijl hij met}{een roden zakdoek}\\

\subsection{Uit: Pieter Breughel, zoo heb ik u uit uwe werken geroken}

\haiku{Dat zonk allemaal, ;}{zuiver in zijn hart waar hij}{het zoet bewaarde}\\

\haiku{Pieter had er iets ':}{bijgeschreven en las het}{voor aant meiske}\\

\haiku{En Pieter zich rap '.}{gebukt en aant lachen}{lijk een waterval}\\

\haiku{Om een puntje aan,!}{den honger te slijpen laat}{ons een kaartje spelen}\\

\haiku{Hij ligt tegen dien.}{gespleten knotwilg aan den}{mestput te ronken}\\

\haiku{passeerden rap, rap,:}{voorbij snel gegroet door den}{gezamenlijken}\\

\haiku{Maar de Pad, die door,:}{een deurspleet loerde riep toen}{van achter de deur}\\

\haiku{Dat is de geest van'',.}{Trees uit het Belofte Land}{bibberde er eene}\\

\haiku{In groepkes trokken',,,.}{z er uit dicht bijeen het}{mes bloot dat beefde}\\

\haiku{een rond rood neuske,.}{en kikvorschenoogen puilden uit}{hunne spekbalken}\\

\haiku{s Zondags kreeg hij;}{er gelukkiglijk toch nog}{een zwaaiken spek bij}\\

\haiku{waternimfen en.}{hoornen van overvloed rond het}{kleurig landswapen}\\

\haiku{En ge moogt gij nog:}{zoo dik en zoo rijk zijn van}{hier tot aan de zon}\\

\haiku{De pater bad voor,:}{Pieters zielezaligheid}{en Pieter floot}\\

\haiku{''O malzig teeder '.''}{boelekent Is zoet met}{u te leven}\\

\haiku{Maar de pater bad, ''',.}{en Pieter floott Is zoet}{met u te leven}\\

\haiku{Ei ja, 't is waar,'', ''....}{voegde  hij er blij bij}{die had er geen aan}\\

\haiku{Toen hij zijn oogen weer,.}{open deed zag hij ginder een}{meisken aankomen}\\

\haiku{En nu ga ik de,.}{schilderijen zien die in}{de kerken hangen}\\

\haiku{Hij bezag haar als,}{een wonder hij zag hoe ze}{huiverde en heur}\\

\haiku{Later als ze dood,,,''.}{is och arme heb ik nog}{tijd genoeg dacht hij}\\

\haiku{Maar bij de Dikken ''''.}{was het spek in de pan en}{bier\'a volont\'e}\\

\haiku{En de geheime!}{Broederschap van Den Naakten}{Dolk zal u helpen}\\

\haiku{Van dezen morgen,!}{nog niets gegeten maar de}{liefde voor alles}\\

\haiku{''Als Veronica,....}{niet arm geweest was had ik}{haar niet gevonden}\\

\haiku{'t Verwonderde,.}{Pieter dat hij niet weende}{hij had er spijt van}\\

\haiku{Hij zou boer blijven, '.}{maar daarom dan eerst gezorgd}{nen boer te worden}\\

\haiku{Het juichende bloed.}{ratelde draaiputten in}{zijn lijf van geestdrift}\\

\haiku{Het was zoo stil als,.}{op pantoffels alsof er}{daar niemand leefde}\\

\haiku{Eens de brug over, moet ',.}{hij doornen koker die den}{vestingwal doorsnijdt}\\

\haiku{Volk en kinderen,....}{loopen mee nieuwsgierig naar}{het ongeluk}\\

\haiku{'' En den eene riep tot. ''}{den andere al wreeder}{en wreeder dingen}\\

\haiku{En was er daar ook,?}{geene Simon van Cyrenen}{die mee het kruis droeg}\\

\haiku{Maar toen zag hij juist. ''}{twee paters Franciscanen}{ginder voorbijgaan}\\

\haiku{Ten einde van de,,.}{gang achter een brandende}{kaars was De Nood Gods}\\

\haiku{Ik ben gisteren!}{pas uit mijn klooster van Gent}{hier aangekomen}\\

\haiku{'t Was 'ne lange,,, '.}{vinnige jonge pater}{metnen lach opzij}\\

\haiku{Hij teekende den;}{man die door de soldeniers}{meegenomen wordt}\\

\haiku{Baskwadder, aan 't;}{preeken om de menschen te}{kunnen verraden}\\

\haiku{''Als ik niet rap weg,.''}{ben langs de deur zwieren ze}{mij door de vensters}\\

\haiku{maar in zijn dapper,.}{oogen lag iets achterdochtigs}{smeekends en onvast}\\

\haiku{de goddelijke,.}{schoonheid van den mensch zien de}{schoonheid van alles}\\

\haiku{Jan Nagel bekeek,.}{het strak met zijnen blauwen}{verdrietigen blik}\\

\haiku{Ik vind hem zelf een,.}{van de beste schilders maar}{dat ziet gij nog niet}\\

\haiku{Wij vangen God met, ';}{ons verf lijknen heilige}{met zijn gebeden}\\

\haiku{dat ik hier ben met ',.}{nen jongen vriend die met hem}{wil kennis maken}\\

\haiku{ik kan dat ook met}{tooverkracht maar dat doe ik bij}{mijn eigen bloed niet}\\

\haiku{om de menschen te}{verschalken want eenzamen}{worden gauw verdacht}\\

\haiku{Als vaderke weet, ',.}{dat ik weer aant spelen}{ben dan dondert hij}\\

\haiku{In twee dagen is,!}{hij niet geweest en dan geen}{woord laten zeggen}\\

\haiku{en die knorhanen,;}{door de roode ziel van den}{koraalplant doorsopt}\\

\haiku{hij veegde er met,.}{zijnen duim over krabde er}{met zijn nagels aan}\\

\haiku{Want verbeelding had,.}{Jan Nagel niet en durfde}{hij ook niet hebben}\\

\haiku{De meid, die het bracht,,,.}{had roode dikke handen}{die naar melk roken}\\

\haiku{'t Zal rap Winter,,'';}{worden Menheer zei ze met}{een bevende stem}\\

\haiku{Toen hij over de brug,.}{in de stad ging verdwijnen}{zag hij nog eens om}\\

\haiku{Ik zal misschien wel.}{wekelijks hier werk komen}{halen en brengen}\\

\haiku{''Ge moogt zoo hevig,;}{uw gedachten op mij niet}{zetten Filleke}\\

\haiku{En hij was blij, dat.}{Filleke nu niet meer over}{liefde kon spreken}\\

\haiku{Hij hoopte, dat ze;}{nu haar gedachten van hem}{wel zou wegtrekken}\\

\haiku{nu had hij er spijt,;}{van haar niet altijd geerne}{te hebben gezien}\\

\haiku{Achter de ruitjes.}{van het schotelhuis zag hij}{Anneke loeren}\\

\haiku{maar zijn beenen waren.}{als lam van de lafheid die}{hij had uitgehaald}\\

\haiku{''Ik kan er niet van,'', ''.}{slapen snikte zeals ge}{zoo kwaad op mij zijt}\\

\haiku{''Als ge toch niet met,,''.}{mij trouwt kort het niet zei ze}{frank en uitdagend}\\

\haiku{Al kan ikzelf geen,;}{kunst maken ik kan er toch}{niet afgeraken}\\

\haiku{de weeren, die de;}{bergen kleeden en met licht}{en kleur versieren}\\

\haiku{, en de Brusselaars.}{zaten daar nog versch alsof}{ze moesten beginnen}\\

\haiku{Kom spoedig van de,!''}{taveerne want uw liefde}{is beter dan wijn}\\

\haiku{onderwerpen in '.}{overvloed en modellen zoo}{maar voort scheppen}\\

\haiku{Hij asemde de goeie,.}{lentereuken op na dit}{dwaas hard regentje}\\

\haiku{''Hoe spijtig, dat ik!''.}{toen mijnen teekenboek niet}{bij had riep Pieter}\\

\haiku{''Maar nu heb ik hem,!}{bij en hij moet zwaar worden}{van wat er in komt}\\

\haiku{Om zeven uren       ,.}{moest ze gewekt worden want}{ze wou vroeg thuis zijn}\\

\haiku{''O, Heer, vergeef mij.''}{de zonden die ik vandaag}{ga bedrijven}\\

\haiku{de goede wierden, ':}{uitgelachen ent volk}{deed de slechte na}\\

\haiku{Hij was blij als een '.}{kind dats morgens zijnen}{Sinterklaas verwacht}\\

\haiku{Het dorp was d' helft.}{verkleind door den oorlog en}{de plunderingen}\\

\haiku{maar zoo 't schijnt, is.}{die jongen ook leelijk aan}{zijn eind gekomen}\\

\haiku{Maar als ik er nog,:}{aan denk dan zie ik het nog}{v\'o\'or mij geschilderd}\\

\haiku{'' Zoo kwamen er soms,.}{prenten uit jaren nadat}{ze gemaakt waren}\\

\haiku{armen van zijn groote,,!}{langverwachte beminde}{de schilderkunst}\\

\haiku{Zoo leefde Pieter,.}{gelukkig knapperend van}{geestdrift in zijn kunst}\\

\haiku{op anderen tijd.}{vergaf hij het haar als een}{onnozel plezier}\\

\haiku{In uw hart steken '!''}{meer pijlen dan zwaarden in}{t hart van O.L. Vrouw}\\

\haiku{''Maar \'e\'en groot zwaard zal,!''}{uw hart doorboren en dat}{zwaard steekt er nu in}\\

\haiku{Maar Anneke heeft.}{haar novice-jaren}{bij u schoon gedaan}\\

\haiku{speel door, opnieuw, en, '....}{als ik van een ander hield}{dan wast nu uit}\\

\haiku{Maar 't was, alsof,.}{ze gewaar wierd dat ze haar}{wouen verschalken}\\

\haiku{En daar Pieter haar,.}{geen achterdocht wou geven}{kon hij niets redden}\\

\haiku{Misschien eenmaal weg,,:}{dat Hans er wel wat zou op}{vinden of anders}\\

\haiku{de zonnekloppers,:}{maar de kloekste kerels als}{er te werken valt}\\

\haiku{Daar, in 'nen hoek, wordt,}{gekaart en gedobbeld en}{v\'o\'or den brandenden}\\

\haiku{Of Eva blond of  , ','!}{zwart haar hadt kan mij niet}{schelen z had haar}\\

\haiku{Ja!'' zegt den hoogen rug,,.}{achterdochtig en bang maar}{gereed om te slaan}\\

\haiku{Haal een gans naar de,''.}{Braderij en een konijn}{beval Marieke}\\

\haiku{Tot Pieter over het.}{heimelijk groeiend verzet}{begon te spreken}\\

\haiku{De zon duwde al '.}{haar gloed als omt nijdigst}{op de wereld}\\

\haiku{Hij is kruis en munt,,.}{kop en letter allebei}{ineens tegelijk}\\

\haiku{''Omdat de wereld.}{is zoo ongetrouw daarom}{ga ik in den rouw}\\

\subsection{Uit: Pijp en toebak}

\haiku{Dat wisten ze ook,,.}{zoo van binnen maar kenden}{het niet van buiten}\\

\haiku{En hij smoorde met,.}{zijn oogen toe van de deugd uit}{de lange steenen pijp}\\

\haiku{Ze zat dikwijls te.}{peinzen en hoorde niet dat}{de moor overkookte}\\

\haiku{En mag hij dan in ',?}{t geheel niet meer werken}{menheer den doktoor}\\

\haiku{En hij had maar spijt.}{dat de burgemeester niet}{naar zijn kousen zag}\\

\haiku{{\textquoteleft}Als ik zoo zeker,?}{ben waarom wandel ik dan}{zoo onrustig hier}\\

\haiku{{\textquoteleft}Ons Net zal schreeuwen.}{van blijdschap als ik het haar}{seffens kom zeggen}\\

\haiku{Hij ging de stoep af,,.}{en hij ging naar de vensters}{als aangetrokken}\\

\haiku{Het konijn Susken.}{Andries had zijn konijnen}{in zijn huis zitten}\\

\haiku{Dat was goed tegen,.}{de dieven en hij had er}{dan meer plezier van}\\

\haiku{Ze vonden hem 's.}{morgens krinselend onder}{den Kruislievenheer}\\

\haiku{s Nachts droomde ik '.}{er van ent maakte mij}{heelemaal anders}\\

\haiku{{\textquoteleft}Fons, houd Emmaken,.}{maar goed bij u z'is zoo goed}{voor de kinderen}\\

\haiku{{\textquoteleft}Als g'het niet gaarne,, '.}{hebt geef mij dan een teeken}{t is eender hoe}\\

\haiku{Hij stapte buiten.}{in den dikken sneeuw en de}{venijnige kou}\\

\haiku{Hij schoot in 't dorp.}{nog eens met de gauwte De}{Roode Leeuw binnen}\\

\haiku{{\textquoteleft}Laat ons maar vroeg gaan,.}{slapen en lig morgen wat}{langer in uw bed}\\

\haiku{Tist, Tist, heb ik het}{altijd niet gezegd dat er}{spoken in dien boom}\\

\haiku{Neen zoo iets had ze,!}{van haren Tist nooit gedacht}{wat nen sterken Tist}\\

\haiku{nu zoo ne klaren,!}{nachtegaal kunnen vangen}{wat zou die deugd doen}\\

\haiku{Die Kristus doorboord, '.}{met zijn dolkt deed even zijn}{kloekheid wankelen}\\

\haiku{Ze vermagerde,.}{er van lijk een spiering maar}{bleef even modieus}\\

\haiku{{\textquoteright} De fotograaf plaatste.}{het toestel v\'o\'or het gelaat}{van Marjanneke}\\

\haiku{{\textquoteright} 't Was te hooren.}{dat ze haar vertelsel al}{dikwijls had gedaan}\\

\haiku{s avonds om negen,,.}{uur brengt Gust mijn verloofde}{deze bloemekens}\\

\haiku{Hij heeft er mij nooit,.}{iets van gezegd maar bij zijn}{dood bleek het alzoo}\\

\haiku{Want zij zijn het hart,.}{van mijn man en mijn eenigste}{goed op de wereld}\\

\haiku{En 's morgens wierd.}{zij wakker en zij lachte}{naar het mandeken}\\

\haiku{De lente was daar,.}{met al zijn genade van}{kleuren geur en licht}\\

\haiku{Eindelijk stond heel.}{het gevulde korfken te}{blinken in de zon}\\

\haiku{Dan hier en daar een ':}{zucht ent was de vorsch}{die geestdriftig riep}\\

\haiku{Daarmee vloog hij weg.}{naar een rustiger plaats in}{het balsemend woud}\\

\haiku{Fier en blij klopte,:}{hij aan den knotwilg waarin}{de vleermuis woonde}\\

\haiku{- Haasje zijt gij dat,?}{die daar zoo behagelijk}{te knabbelen zit}\\

\haiku{- Ewel Eekhorentje,,}{als ik klimmen kon zooals gij}{en niet zoo bang was}\\

\haiku{zooals ik, dan zou ik.}{op den muur kruipen en het}{zoo te zien krijgen}\\

\subsection{Uit: Schemeringen van de dood}

\haiku{Felix Timmermans}{Schemeringen van de dood}{Colofon}\\

\haiku{wij wisten niet eens -.}{van wat dat wij er nooit zijn}{in wedergekeerd}\\

\haiku{Vroeger stond er links,.}{ook een maar de bliksem had}{het er afgehaald}\\

\haiku{Het was alsof ik.}{daar ver van afstond en dit}{zag als in een droom}\\

\haiku{Mij leek het dat er.}{onder het vlees een dunne}{klaarte zichtbaar werd}\\

\haiku{Zij opende groot de.}{ogen en zag met een goede}{blik de kamer rond}\\

\haiku{Het leek mij dat ze}{luideloos vertelden wat}{hij in zijn hoofd wist}\\

\haiku{{\textquotedblright} En vaders handen...}{kwamen bijeen en vouwden}{zich samen vol dank}\\

\haiku{het was alsof de!}{muren een adem hadden die}{door mijn ziele ging}\\

\haiku{Een voorzichtige,....}{belletik klonk in de gang}{nog een en nog een}\\

\haiku{mij staan in een groot,.}{verblindend licht dat mij de}{ziel verzengelde}\\

\haiku{mijn groet, en Mina,...}{liet op mijn ogen rusten een}{koude lange blik}\\

\haiku{Het is daar dat de.}{artiesten van alle slag}{komen luisteren}\\

\haiku{O Herman, ik ben,,!}{zo bang van die kelder doe}{hem toe doe hem toe}\\

\haiku{Maar na een wijle,:}{fluisterde ze als bevreesd}{het uit te spreken}\\

\haiku{Er was beneden;}{steeds een stilte als in een}{huis zonder mensen}\\

\haiku{{\textquoteleft}Dan kunnen wij met.}{hoop de lente afwachten}{die aanstaande is}\\

\haiku{'t Lag in de blik, '.}{van haar ogen ent rilde}{door gans haar wezen}\\

\haiku{Maar hoe groot was mijn!}{verwondering toen ik de}{kelder open zag staan}\\

\haiku{De ogen werden groot,.}{en wild met veel wit rond de}{donkere kinnen}\\

\haiku{{\textquoteleft}Kom, Mina,{\textquoteright} zei ik, {\textquoteleft},,;}{bedarendkom wees rustig}{leg me uw hart bloot}\\

\haiku{Treurwilgen leekten.}{hun moedeloze twijgen}{over zwarte zerken}\\

\haiku{Lag de grafmaker,?}{daar misschien zelf dood zonder}{dat iemand dat wist}\\

\haiku{Het verwonderde.}{mij dat er geen doden meer}{werden aangezegd}\\

\haiku{Ik  kleedde mij.}{voorzichtig en zachtekens}{sloop ik naar de deur}\\

\haiku{Het scheen of heel de.}{wereld in een ijzeren}{stilte lag gekneld}\\

\haiku{De vrees haakte in!}{mij en de onnozelheid}{spotte in mijn oor}\\

\haiku{Hoe meer ik van het,.}{graf afweek hoe dichter ik}{de dood naderde}\\

\haiku{Bos noch toren deed;}{verhopen dat er ginder}{nog mensen leefden}\\

\haiku{Omdat het mij zo,.}{benauwd werd begon ik naar}{gerucht te zoeken}\\

\haiku{Het lied zwol en boog,.}{en golfde weg en weer als}{het lied van de zee}\\

\haiku{De voet alleen der.}{schone vaas stond nog in het}{midden der tafel}\\

\haiku{{\textquoteleft}Ja, ik weet het,{\textquoteright} zei.}{ze gelukkig en ze trok}{hem tegen zich aan}\\

\haiku{Zij zagen beangst,.}{om vrezend dat het moeder}{of de zusters was}\\

\haiku{Hij stak gauw de koord.}{weg en zij trok het sjaaltje}{dieper over het hoofd}\\

\haiku{Hendrik bleef liggen.}{als een vod en het water}{spoelde uit zijn mond}\\

\haiku{En dan, met een {\textquoteleft}nu{\textquoteright},.}{de andere begon hij}{opnieuw te zoeken}\\

\haiku{Meteen overviel hem.}{een grote angst  en zijn}{hart wrong zich samen}\\

\haiku{Hij zou de avond in,.}{de kamer afwachten het}{zien donker worden}\\

\haiku{Weer moest hij wachten,.}{in de witte celkamer}{nu was het donker}\\

\subsection{Uit: De zeer schone uren van Juffrouw Symforosa, begijntjen}

\haiku{{\textquoteleft}Ik leef van de zon,,,?....}{wat zou ik en mijn bloemen}{gaan doen zonder zon}\\

\haiku{Dat maakte haar vrij.}{en ze kon terug onder}{de menschen komen}\\

\haiku{'t Beste ware,.}{nog dat hij niet kwam och haar}{hart is zoo angstig}\\

\haiku{Zij durft niet omzien,.}{en blijft staan en voelt het bloed}{in haar beenen zinken}\\

\haiku{{\textquoteright} {\textquoteleft}Ja, Symforosa{\textquoteright},, {\textquoteleft}!}{zegt hij verblijden ik zal}{veel voor u bidden}\\

\haiku{Maar ze is hare.}{belofte aan Martienus}{nog niet vergeten}\\

\haiku{En valsch heete ons, '.}{iedere waarheid waart}{lachen bij ontbrak}\\

\haiku{Z\'o\'o is Pallieter,.}{dat in 1916 verscheen en reeds}{herhaald is herdrukt}\\

\section{Oscar Timmers}

\subsection{Uit: Landklimaat}

\haiku{Je zegt mimisch och,.}{en ach als een pierrot die}{de minuten telt}\\

\haiku{Mijn bok schuurt zich langs.}{de tafelpoot en gooit een}{glas wijn over mijn broek}\\

\haiku{Jij bent zo bang, jij,}{schreeuwt zo hard dat niemand het}{de moeite waard vindt}\\

\haiku{{\textquoteright} {\textquoteleft}Hijg niet zo, je ziet,,{\textquoteright}, {\textquoteleft}}{zwart je slaat zwart uit als een}{geraamte zegt Ana}\\

\haiku{{\textquoteright} De wanhoop brandt zeer,.}{hevig de verwijten slaan}{in wolken as neer}\\

\haiku{Maar daarna ben ik,,.}{degene die wacht om en}{om een uur misschien}\\

\haiku{{\textquoteright} {\textquoteleft}H\'e...{\textquoteright} zegt ze, en met.}{verwoede ijver zet ze}{haar pogingen voort}\\

\haiku{Ze draait haar lichaam.}{naar me toe en neemt mijn hoofd}{vlak voor het hare}\\

\haiku{Het is Jessica.}{aan te zien dat ze op weg}{is naar een avontuur}\\

\haiku{We zien de huid van.}{het water wanneer ze over}{de Alpen heen springt}\\

\haiku{De deuren klapten.}{open en weer dicht en lieten}{de brug leeg achter}\\

\haiku{De dode Henker,.}{werd dood verklaard zijn daden}{werden dood verklaard}\\

\haiku{Je ogen openen je,}{schoot en rijzen ontsteken}{in het voorbijgaan}\\

\haiku{ze stromen tussen.}{mij en de matglazen deur}{door naar Jessica}\\

\haiku{En het zijn niet eens.}{de vrijbuiters die er zich}{meester van maken}\\

\haiku{{\textquoteright} (Je gooit een woord in) {\textquoteleft}.}{een echoputEn dan is de}{eenzaamheid klankrijk}\\

\haiku{De zee begint als,,.}{dal ze schuift omhoog en valt}{weer neer rijst en daalt}\\

\haiku{Je stroomt voort durend.}{als woestijn en als zee en}{als lucht door me heen}\\

\haiku{Het addertje bijt,...}{in zijn eigen staart het vormt}{een lus om je hals}\\

\haiku{{\textquoteright} fluistert Jessica.}{wanneer we het geluid voor}{het eerst vernemen}\\

\haiku{Ik verlaat haar en.}{ga op mijn knie\"en opzij}{van het bed zitten}\\

\haiku{Ik blijf naast het bed.}{zitten en leg mijn hand op}{Jessica's schouder}\\

\haiku{{\textquoteleft}Het is de dikke,,,,.}{man hij slaapt hiernaast hij snorkt}{o o wat snorkt hij}\\

\haiku{Maar  veel in de,.}{zon lopen flink zwemmen en}{lief voor elkaar zijn}\\

\haiku{Ik kon mijzelf in,.}{het oog zien ik kon mijzelf}{onder ogen komen}\\

\haiku{wiens vakantie is.}{afgelopen en weer naar}{kostschool wordt gestuurd}\\

\haiku{Goed, goed, (en ik wrijf),.}{me in de handen doe maar}{met me wat je wilt}\\

\haiku{En ik lach, maar {\'\i}k,.}{lach goed want ik zie dat ik}{je overwonnen heb}\\

\haiku{Ik voorspel je, en (),:}{ik beloof jefluisterend}{want dat wil je toch}\\

\section{Jan Gerhard Toonder}

\subsection{Uit: Kasteel in Ierland}

\haiku{Ik heb zo lang niets -!}{van je gehoord ik zou je}{bijna vergeten}\\

\haiku{Het elegante Rhein (,;}{Hotelalles glanzend nieuw}{en effici\"ent}\\

\haiku{Want, waarachtig, denk;}{eens aan de Kommandantur}{daar in Aardenbosch}\\

\haiku{En je vrouw, hoe heet,;}{ze ook weer die heeft ook wel}{wat meegenomen}\\

\haiku{Ik heb altijd wel.}{geweten dat daar heel wat}{mee gedokterd wordt}\\

\haiku{Je drinkt een borrel.}{met die kerels en dan is}{alles in orde}\\

\haiku{de zoete Jezus.}{heeft tot nu toe de kleine}{kindertjes beschermd}\\

\haiku{misschien kunnen we...}{intussen vast eens naar die}{badkamers kijken}\\

\haiku{Dr. Franz Krause had,.}{geen lucifers bij zich maar}{hij miste ze niet}\\

\haiku{{\textquoteright} {\textquoteleft}Patrick vroeg of je,.}{in de hal kwam hij heeft een}{verrassing voor je}\\

\haiku{Ze zweeg abrupt, en;}{nu hoorde hij het kraken}{van de traptreden}\\

\haiku{Laat die verdomde -!}{scharnier maar zitten we gaan}{de plee aanleggen}\\

\haiku{Maar als zijn voeten.}{dan pijn gingen doen werd het}{allemaal onzin}\\

\haiku{{\textquoteright} Dr. Franz Krause, de,.}{wettige grondeigenaar}{gaf niet eens antwoord}\\

\haiku{Neem me niet kwalijk,,.}{ik praat maar en praat maar dat}{komt van mijn beroep}\\

\haiku{{\textquoteright} {\textquoteleft}Maar hij begrijpt me,.}{niet terwijl ik toch alleen}{maar op mijn recht sta}\\

\haiku{Hij vertelt mijn vrouw;}{verhaaltjes over de vogels}{van de ru{\"\i}ne}\\

\haiku{Hij kon niet denken,.}{geen zin wilde zich in zijn}{gedachten vormen}\\

\haiku{Dit is veel en veel.}{meer dan ik in Duitsland zou}{moeten betalen}\\

\haiku{{\textquoteright} Krause vroeg zich een;}{ogenblik af waarom dat zo'n}{verschil zou maken}\\

\haiku{Hoe kon een normaal,,?}{praktisch hard werkend mens hier}{ooit iets bereiken}\\

\haiku{Als door een wonder;}{bleek de elektriciteit te}{functioneren}\\

\haiku{in bog or wood, her!}{lovely smile makes}{him feel glad and good}\\

\haiku{Met wat vruchtbomen,.}{langs de kant die bloeien zo}{mooi in de lente}\\

\haiku{{\textquoteleft}Dus,{\textquoteright} zei hij kortaf,, {\textquoteleft}.}{beledigdhij kwam niet over}{mijn weiland praten}\\

\haiku{Hij ging, natuurlijk - -;}{dit moest zijn tijdbom wel zijn}{maar met ergernis}\\

\haiku{Een jodenjongen -;}{van lang geleden als zij}{dat nu aardig vond}\\

\haiku{Hij vloog overeind met.}{zo'n schok dat er iets van het}{nachtkastje afviel}\\

\haiku{{\textquoteleft}Ga dan zitten en,;}{vertel me terwijl ik je}{een kop koffie schenk}\\

\haiku{Hij zette zich dus,,,:}{schrap schraapte zijn keel keek naar}{de lucht en begon}\\

\haiku{Jawel, daarin moest.}{hij controleposten en}{de grens passeren}\\

\haiku{{\textquoteright} En dat kwam van de,.}{ingang door het krassen van}{de vogels heen}\\

\haiku{{\textquoteleft}Laat ons er niet meer,{\textquoteright}.}{over spreken en dat hadden}{ze niet meer gedaan}\\

\section{Fernand Toussaint van Boelaere}

\subsection{Uit: Landelijk minnespel}

\haiku{Tegelijk drong ook;}{de nachtelijke stallucht}{zwoeler op hem aan}\\

\haiku{De hengst bekeek hem, ';}{met vlammend oog rond alsn}{glad-bruine schijf}\\

\haiku{{\textquoteleft}Hij mag hij doen wat, ' '.}{hij wilk ben toch opt}{voorhand gewroken}\\

\haiku{Maar Bello, 't is,,.}{zeker staat en wacht nog op}{den drempel kaarsrecht}\\

\haiku{En nu al met eens,, ':}{zie vernam hij hoet nog}{trilde in de lucht}\\

\haiku{{\textquoteleft}- Zeg nu ja, Zalia,{\textquoteright},.}{bad hij met de lippen en}{de smeekende oogen}\\

\haiku{Evenwel, zijn gelaat.}{bleef gewend naar de plek waar}{hij de brugge wist}\\

\haiku{{\textquoteleft}Ge moet gij zeker,{\textquoteright}....}{ook toch \'eten klonk het in de}{rust van de kamer}\\

\haiku{De andere knechts,....}{had ze maar laten gaan hij}{was immers nog uit}\\

\haiku{- Hoe ook weg met zijn,,,}{gedachten de Langen al}{gaande zag wel dat}\\

\haiku{Het geluid van heur, '....}{stem klonk aldoor zachtert}{was haast fezelen}\\

\haiku{Toen liet zij langzaam,;}{die hand omhoogwaarts keeren als}{een deur die opendraait}\\

\haiku{{\textquoteright} De toon der stem steeg,,?}{al hooger scherper Ging er}{toch laweit bij zijn}\\

\haiku{steeds dieper drong dan '.}{ookt snijdend ijzer door}{den schoot der aarde}\\

\haiku{het blijde dier, vrij,;}{thans van teugel en getuig}{snoof luidruchtig}\\

\haiku{Plots echter brak 't.}{ijzerig geklep van een}{deurklink de stilte}\\

\haiku{Het was precies of.}{Bello hem niet z\'o\'o vast en}{geerig verwachtte}\\

\haiku{Moeizaam vernam dan.}{zijn oor hoe heel de stal moest}{staan in rep en roer}\\

\haiku{- van een dier dat plots,.}{recht opsprong een wijle dan}{zoo steigerend bleef}\\

\haiku{Z\'o\'o deed de hengst het,.}{ook dezen morgen  toen}{hij hem afstrafte}\\

\haiku{Wat ze niet voor een {\textquoteleft}{\textquoteright}, {\textquoteleft}.}{deugd aanzagen noemden ze}{kort-wegondeugd}\\

\section{Arnaud de Trega}

\subsection{Uit: Ramp en misdaad. 1623-1638}

\haiku{Wij schreven, wars van,.}{mode of mooidoenerij}{zonder pretentie}\\

\haiku{twee Heeren \'e\'en Heer,,.}{die is onze God en het}{vrije Volk van Maestricht}\\

\haiku{Lachend sprong hij te,.}{paard en het was mij als zag}{ik hem voor het laatst}\\

\haiku{Frederik Hendrik,.}{is een bekwaam kapitein}{die alles voorziet}\\

\haiku{Ofschoon streng in de,.}{leer was hij buitengewoon}{breed van opvatting}\\

\haiku{Vaarwel, wat gij U.}{hadt gedroomd van grooter en}{beter en hooger}\\

\haiku{zij stonden bij hun,.}{paarden aan den grooten weg}{achter het kasteel}\\

\haiku{in galop door de,,.}{poort en links af langs de kerk}{waar hun paarden staan}\\

\haiku{vijf en zeventig,.}{voet van de Zuidzij in den}{kelder op 6 voet}\\

\haiku{- Zij zijn opgehitst,.}{door de geestelijken zei}{een der kapiteins}\\

\haiku{De gemetselde;}{gang had eene manslengte en}{was wel 4 voet breed}\\

\haiku{Een legioen van}{ratten spoedde zich voor hen}{uit en menigmaal}\\

\haiku{zie, ze kruipen weg,.}{in dezelfde richting daar}{moet de put wezen}\\

\haiku{Inderdaad, stonden:}{zij weldra bij den put en}{Carabin bromde}\\

\haiku{Weldra stonden zij:}{voor den sergeant en de}{Leegtenborg zeide}\\

\haiku{- Wij hebben aan hun;}{lijken niet de laatste eer}{kunnen bewijzen}\\

\haiku{- De kapitein kan,.}{elk oogenblik komen zoodat}{U even wachten kunt}\\

\haiku{zij stopten mij een;}{prop in den mond en sleurden}{mij in den kelder}\\

\haiku{- Wat meent gij, dat ik,,.}{doen kan mejuffrouw Agnes}{vroeg hij eindelijk}\\

\haiku{- Carabin, zei de,,.}{kapitein zie wel toe wat}{achter ons gebeurt}\\

\haiku{Indien wij deze,.}{vrouw kunnen achterhalen}{is het kind gered}\\

\haiku{- De overwinning was,,?}{ons zei Kapelaan Schreuders}{doch tot welken prijs}\\

\haiku{Wanneer gij in die,,......}{richting denkt heer Hofmeyer}{bent U er glad naast}\\

\haiku{Zoo bad hij om kracht.}{tot den Hemel en deed St.}{Servaes geweld aan}\\

\haiku{wat kan ik doen? - Heer,.}{Proost nog heden nacht moet het}{plan beraamd worden}\\

\haiku{- Broeders, zei meester,.}{Wynans thans zijt gij een en}{onverdeeld met ons}\\

\haiku{Schipper de Gye heeft.}{verlof tot lossen en kan}{vrij heen en weer gaan}\\

\haiku{De beide zwagers.}{en krijgsmakkers omhelsden}{elkaar in stilte}\\

\haiku{Wenscht een uwer zulks,,?}{te doen of verlangt gij u}{nog te beraden}\\

\haiku{Ik zal leven en...}{sterven in- en met mijn}{herinneringen}\\

\haiku{Zoudt gij weigeren,.}{te spreken dan wachten U}{pijnbank en schavot}\\

\haiku{geve de hemel,.}{of de hel dat zijn beulshand}{hem niet beroere}\\

\haiku{- Wij zien hem, hooren,.}{hem en rieken hem braden}{spotte een soldaat}\\

\haiku{In de Gothische,,.}{zaal een ruim vertrek werd de}{vierschaar gespannen}\\

\haiku{- Jezabelle de,,?}{Rax vrouwe Sleussel waartoe}{zijt gij gekomen}\\

\haiku{Hij weigerde den,,.}{blinddoek legde zijn hoofd op}{het blok de bijl viel}\\

\haiku{In de kluis brandde.}{nog licht en ik hoorde het}{geluid van stemmen}\\

\haiku{- Ik breng den laatsten.}{wil en een laatste verzoek}{van een stervende}\\

\haiku{gij hebt het verlangd.}{en gij zult dezen nacht uw}{leven herleven}\\

\haiku{Op korten afstand,,.}{volgde Jan van Sichem een}{pak op den bochel}\\

\haiku{Wij cre\"eerden uw,,,.}{lichaam maar Hij wiens dienaar}{Gij waart schiep de ziel}\\

\haiku{Eetwaren mogen.}{niet hooger verkocht worden}{dan vastgesteld is}\\

\haiku{Met Gods hulp komt de.}{verlossing zeker Vrijdag}{of Zaterdag}\\

\section{Pieter Jelles Troelstra}

\subsection{Uit: Gedenkschriften. Deel I. Wording}

\haiku{Daar komt zijn oude.}{goudsmidsbaas binnen en gaat}{recht op hem af}\\

\haiku{Stroef en ernstig was,.}{hij geheel anders dan zijn}{jongere broeders}\\

\haiku{Waar zij met geest en.}{moed beladen Steeds stierven}{voor de zaak van God}\\

\haiku{Ziedaar                         mede.}{een aanleiding tot deze}{korrespondentie}\\

\haiku{Weldra blijkt het noodig,;}{de belijdenis elders}{af                         te leggen}\\

\haiku{Dit was, geloof ik,,.}{het eenige standpunt dat zij}{kon                         innemen}\\

\haiku{Het duurde niet lang,, {\textquoteleft}{\textquoteright}.}{of mijn grootste genoegen}{wasvoor te lezen}\\

\haiku{Halbertsma tot min.}{of meer offici\"eele}{schrijftaal ontwikkeld}\\

\haiku{Het belangrijkste.}{vraagstuk was natuurlijk het}{kinderen krijgen}\\

\haiku{Omstreeks dien tijd lag}{ik ziek aan de mazelen}{en daarmede}\\

\haiku{Hiermede begon.}{voor mij een                     nieuw tijdperk}{van mijn leven}\\

\haiku{Eerst maakten wij de, {\textquoteleft}{\textquoteright}.}{hokken der konijnen}{dan dat vanbaas schoon}\\

\haiku{Aan dezen man voel.}{ik mij door                     innige}{dankbaarheid verknocht}\\

\haiku{Terwijl ik daar stond,,,.}{trillend naakt ging de deur van}{de                     keuken open}\\

\haiku{Mijn levensdrang en.}{zelfvertrouwen uitten zich}{o.a. in                     spotzucht}\\

\haiku{, zijn oog glinsteren.}{en het zweet op zijn voorhoofd}{zien parelen}\\

\haiku{als een machtsuiting,.}{waarvoor ik al heel weinig}{lust had te bukken}\\

\haiku{v\'o\'or dien tijd                     reeds.}{vond hij een plaats aan een der}{Twentsche fabrieken}\\

\haiku{Van wanken, draait, en;}{wordt gedreven Om't een}{en eenig middelpunt}\\

\haiku{het lag voor de hand,;}{dat ik liefst zou                     studeeren}{in de Letteren}\\

\haiku{Ik nam                     daarvoor.}{les bij een der leeraren van}{het Gymnasium}\\

\haiku{De wereld breidt zich,,;}{voor mij uit De lachende}{lokkende verte}\\

\haiku{Ik smacht naar den blik,.}{van twee oogen Die stralen als}{zonnen voor mij}\\

\haiku{Ik tast om mij heen,;}{naar twee handen Wier druk mij}{doortintelt met kracht}\\

\haiku{{\textquoteright} {\textquoteleft}Kan mijnheer mij                     ,?}{ook zeggen wanneer mijnheer}{mij ontvangen kan}\\

\haiku{Deze ging ver uit;}{buiten de grenzen eener}{gewone rechtszaak}\\

\haiku{{\textquoteleft}Uit uw vroegeren, {\textquotedblleft}{\textquotedblright}.}{brief begrepen wij reeds dat}{men unegerde}\\

\haiku{Van twee\"en                     een,}{\`of onderwerp u aan die}{barbaarschheden}\\

\haiku{Neen, Piet, ik moet je,.}{eerlijk bekennen dat}{je mij erg afvalt}\\

\haiku{{\textquoteleft}Ons ongeluk is,;}{dat wij teveel met elkaar}{gemeen hebben}\\

\haiku{50Zie blz. 80, 81,,.}{en 83 uitgave Fischer}{Verlag Berlin}\\

\subsection{Uit: Gedenkschriften. Deel II. Groei}

\haiku{Ik heb voor dit werk.}{geen letter zelf op papier}{kunnen                         zetten}\\

\haiku{Zelf geeft hij dit in, ():}{zijn gedenkschriften toe waar}{hij                     zegtblz. 252}\\

\haiku{{\textquoteleft}Ik was namelijk.}{niet in de wieg gelegd voor}{politiek leider}\\

\haiku{De toegeworpen.}{handschoen moest door mij worden}{opgenomen}\\

\haiku{Zoo kunnen schijnbaar.}{tegenstrijdige daden}{worden                     verklaard}\\

\haiku{Voorloopige.}{hechtenis was aan de}{orde van den dag}\\

\haiku{De zaak-Nawijn,,.}{ook te Heerenveen maakte}{vrij wat                     gerucht}\\

\haiku{Men                     spreekt voor een.}{goede zaak en wordt beschouwd}{als misdadiger}\\

\haiku{de tijd was gunstig.}{en nieuwe afdeelingen}{van den Soc.-Dem}\\

\haiku{In mijn krant                     schreef.}{ik over de stemming in de}{vergaderingen}\\

\haiku{daar was een klein                     ,.}{meisje door haar ouders om}{de krant uitgestuurd}\\

\haiku{A. DIJKMAN TE UTRECHT}{J. DIEMEL TE UTRECHT}{A.J.E. RAVESTEIN}\\

\haiku{Onze toestand                     :}{werd steeds hachelijker en}{toen ging het er om}\\

\haiku{Misschien is de tijd,.}{wel spoedig daar dat ik de}{stad moet verlaten}\\

\haiku{Zoodra deze,,:}{mij in het vizier kreeg kwam}{hij op mij toe zei}\\

\haiku{In den loop van                      {\textquoteleft}{\textquoteright};}{onze polemiek schreef ik}{in deBaanbreker}\\

\haiku{Aan het verslag in {\textquoteleft}{\textquoteright}:}{deBaanbreker                    ontleen}{ik het volgende}\\

\haiku{De uitslag was, dat.}{ik 2276 stemmen kreeg en dr.}{Bos                     slechts 1606}\\

\haiku{Zijn opvattingen,.}{ontsprongen aan een zacht maar}{vurig  gemoed}\\

\haiku{Wij riepen alle;}{medestanders op zich}{naast ons te scharen}\\

\haiku{het algemeen                     {\textquoteright}.}{kiesrecht of de strijd voor het}{algemeen kiesrecht}\\

\haiku{Tak, die toen nog geen, {\textquoteleft}{\textquoteright}:}{lid van de Partij was}{schreef in deKroniek}\\

\haiku{{\textquoteleft}Laat ze nog maar wat,.}{geduld hebben                             eerst moet}{die Troelstra van de baan}\\

\haiku{Eenige dagen reeds,:}{ben ik bezig me in die}{kunst te oefenen}\\

\haiku{In                     dezen tijd;}{heb ik veel voorgelezen}{aan mijn kinderen}\\

\haiku{Doodelijk                     uitgeput,.}{begroette hij mij weer met}{dien stralenden blik}\\

\haiku{Spreken kon hij niet,.}{meer zonder woorden nam hij}{van ons                     afscheid}\\

\haiku{{\textquoteleft}Ik zal wel spreken{\textquoteright},.}{hetgeen hij ook                     op den}{juisten toon volbracht}\\

\haiku{de theorie zoekt;}{naar scheiding en ontleding}{van                     begrippen}\\

\haiku{De Nederlandsche....}{vakbeweging gelijkt thans}{een groot slagveld}\\

\haiku{Wij zijn                     er trotsch;}{op uw strijd en nederlaag}{te hebben gedeeld}\\

\haiku{Daaraan gaf op het,:}{kongres ook                         van der Goes}{uiting die zeide}\\

\haiku{Toen ik verleden....}{jaar                     optrad tegen de}{zwendelarijen}\\

\haiku{Naar aanleiding van {\textquoteleft}}{de schoolkwestie had ik in}{een artikel in}\\

\haiku{Pi\"eteit jegens.}{zijn vader heeft zijn gansche}{leven beheerscht}\\

\haiku{Daarom schreef ik een,:}{brief aan van der Goes waarin}{ik o.a.                     zeide}\\

\subsection{Uit: Gedenkschriften. Deel III. Branding}

\haiku{De betooging te.}{Amsterdam van 1906 telde}{15.000                     bezoekers}\\

\haiku{het werd duidelijk,.}{dat de                     stroom niet lang meer}{te stuiten zou zijn}\\

\haiku{censuskiesrecht zou.}{evengoed mogelijk zijn als}{algemeen kiesrecht}\\

\haiku{Tot mijn spijt                     was;}{Tak door mijn woorden zeer in}{zijn wiek geschoten}\\

\haiku{Ik                     eindigde:}{met het voorstellen van de}{volgende motie}\\

\haiku{haar militairen;}{romantikus weer binnen}{de rails te brengen}\\

\haiku{Dit alles gaf                     .}{aan die tochten voor hen een}{boeiend karakter}\\

\haiku{het viel mij op, hoe.}{dit klassieke werk de}{kinderen boeide}\\

\haiku{\'e\'en voor \'e\'en uit het,,.}{bootje en hij bleef verstomd}{alleen                     zitten}\\

\haiku{Dat het hem niet licht,,.}{viel mij dit te zeggen zult}{ge wel begrijpen}\\

\haiku{Ook voor onze zaak,.}{hoop ik steeds te doen wat mijn}{krachten toelaten}\\

\haiku{Mijn positie in;}{de Kamer bewoog zich}{in stijgende lijn}\\

\haiku{In de eerste plaats.}{verwart men                             klassenstrijd}{met klassenoorlog}\\

\haiku{De verwarring in.}{de Partij nam steeds grooter}{afmetingen aan}\\

\haiku{De mededeeling.}{werd met luid en driewerf}{hoera ontvangen}\\

\haiku{Driemaal                     drong ik,.}{daarop tevergeefs aan toen}{liet ik het verder}\\

\haiku{Ik zal niet verder.}{ingaan op het verwarde}{verloop dezer zaak}\\

\haiku{Zij is niet tragisch,....}{en niet dramatisch maar toch}{een                     stap voorwaarts}\\

\haiku{Dan wordt van nu af.}{aan geregeerd                     onder}{zijn patronage}\\

\haiku{Den volgenden dag.}{verscheen een vierde stille}{bondgenoot ten tooneele}\\

\haiku{De Heer Colijn                      {\textquoteleft}{\textquoteright}:}{schreef in een artikel in}{deStemmen des Tijds}\\

\haiku{Ook deze lezing.}{werd in  de geheele}{pers overgenomen}\\

\haiku{op het                     kerkplein.}{traden o.a. Anseele en}{ik als sprekers op}\\

\haiku{{\textquoteright} De 19de Mei sprak {\textquoteleft}}{ik in een vergadering}{te Amsterdam over}\\

\haiku{Deze povere.}{konklusie bewees}{onze nederlaag}\\

\haiku{niet zoo gauw tot de.}{wereldoorlog zou hebben}{laten                     komen}\\

\haiku{Men moet dat hebben;}{meegemaakt om het goed te}{kunnen begrijpen}\\

\haiku{onze laatste hoop,.}{niet in den                     oorlog te}{worden meegesleept}\\

\haiku{Ten overvloede                     ,:}{schreef ik nog een artikel}{waarin ik zeide}\\

\subsection{Uit: Gedenkschriften. Deel IV. Storm}

\haiku{organisatie.}{van het                     bedrijf onder}{leiding van de staat}\\

\haiku{Huysmans was na een;}{avontuurlijke tocht eveneens}{gearriveerd}\\

\haiku{Van u moet de kracht,.}{uitgaan die ons werk                     tot}{een goed einde brengt}\\

\haiku{Maar dan moeten zij,.}{het maar ronduit zeggen dat}{zij niet willen}\\

\haiku{een redeneering,.}{die door Albert Thomas in}{Frankrijk werd herhaald}\\

\haiku{Wij moesten dan vooral.}{oppassen de stations}{te bezetten}\\

\haiku{wij moesten zorg dragen,;}{daarbij                     de leiding in}{handen te houden}\\

\haiku{er moest snel iets                     ,.}{gebeuren anders zouden}{anderen het doen}\\

\haiku{Ankersmit heeft het {\textquoteleft}{\textquoteright}:}{in zijn                     bespreking van}{Groei zoo juist gezegd}\\

\haiku{inwilliging                     ;}{van de eischen van de Bond}{van Dienstplichtigen}\\

\haiku{{\textquoteright} Met dit                     voorstel.}{ging de meerderheid van het}{P.B. echter niet mee}\\

\haiku{De konklusie,,:}{van Cramer wiens meening}{ik heb gevraagd luidt}\\

\haiku{Op deze vraag kan;}{slechts een subjektief antwoord}{gegeven worden}\\

\haiku{organisatie,.}{parlementaire aktie}{en propaganda}\\

\haiku{een groot deel van mijn.}{artikelen dikteerde}{ik op mijn bed}\\

\haiku{Zeer pijnlijk was te.}{Luzern de strijd tusschen de}{twee Duitsche groepen}\\

\haiku{Ik heb mij tegen;}{alle voorstellen tot}{afscheiding verzet}\\

\haiku{Pinksteren 1923 kwam.}{het hereenigingskongres}{te Hamburg bijeen}\\

\haiku{na de opheffing.}{dier tegenstellingen zal}{de staat afsterven}\\

\haiku{kon nu maar op die.}{manier de heele                     zaak}{in orde komen}\\

\chapter[2 auteurs, 232 haiku's]{twee auteurs, tweehonderdtweeëndertig haiku's}

\section{Ucee}

\subsection{Uit: De moordende hand}

\haiku{Noodgedwongen moest.}{hij dus het einde van den}{wedstrijd afwachten}\\

\haiku{Ik neem uw opdracht}{gaarne aan en begin met}{u te verzoeken}\\

\haiku{op je kan worden!}{en daarvan profiteer je}{maar al te dikwijls}\\

\haiku{{\textquoteleft}Kees van Berg, ik daag.}{je nu uit tot een boksmatch}{van tien ronden}\\

\haiku{Dat is natuurlijk,{\textquoteright},, {\textquoteleft}}{Lefsky die er een eind aan}{gaat maken dacht Kees}\\

\haiku{Het resultaat was,,.}{natuurlijk dat de Rooie zijn}{tegenstander sloeg}\\

\haiku{Mejuffrouw, ik vraag{\textquoteright},, {\textquoteleft}}{u duizendmaal excuus riep}{hij zeer ernstig uit}\\

\haiku{{\textquoteleft}Neen, mijnheer, het zijn,!}{mijn Hollandsche vrienden let}{u maar eens goed op}\\

\haiku{{\textquoteright} {\textquoteleft}Zoo zie je, Kees, dat{\textquoteright}.}{men met zijn vermoedens zeer}{voorzichtig moet zijn}\\

\haiku{{\textquoteright} {\textquoteleft}Maar mijnheer Verdoorn,?}{herkent u den Indischen}{Sherlock Holmes niet}\\

\haiku{Het duurde nog wel,.}{een half uur voordat Kees zijn}{bewustzijn herkreeg}\\

\haiku{Had je het mij maar,.}{verteld toen ik dat reptiel}{in mijn armen had}\\

\haiku{{\textquoteleft}Brandhorst, ben je tot,,?}{wederdienst bevrijd pardon}{ik bedoel bereid}\\

\haiku{En pas op voor zijn,,{\textquoteright}.}{wraak jonge dame die zal}{niet voor de poes zijn}\\

\haiku{Als ik mijn hanen,}{niet binnen vijf minuten}{terugkrijg dan jaag}\\

\haiku{{\textquoteleft}Het spijt mij, waarde,{\textquoteright}.}{vriend maar ik weiger beslist}{je neer te schieten}\\

\haiku{Deze zag rood van,}{woede terwijl hij den hem}{volgenden katjoeng}\\

\haiku{Ziska bevond zich,.}{in de voorgalerij toen}{de Ford binnenreed}\\

\haiku{Ik had een vrouw, zooals.}{er slechts \'e\'en in de honderd}{jaar geboren wordt}\\

\haiku{, naar een naburig.}{boschje begaf om daar een}{man te ontmoeten}\\

\haiku{{\textquoteleft}Als het zoover is, dan!}{krijgt de Rooie weer het noodige}{van mij te hooren}\\

\haiku{Met een kamertje!}{in de bijgebouwen zijn}{zij reeds tevreden}\\

\haiku{{\textquoteright} {\textquoteleft}Maar, natuurlijk Leo,,!}{je doet maar precies wat jou}{het beste voorkomt}\\

\haiku{{\textquoteright} {\textquoteleft}Dus je denkt, dat......{\textquoteright} {\textquoteleft}Neen,,.}{denken doe ik het niet want}{ik weet het zeker}\\

\haiku{Zoodra u iets,.}{ongewoons merkt noteert u}{dit onmiddellijk}\\

\haiku{Dan begrijp ik niet,,.}{waarom u nog niet overtuigd}{is mijnheer Hansen}\\

\haiku{{\textquoteleft}Maar, lieve meid, heb?}{je zoo weinig vertrouwen}{in je aanstaande}\\

\haiku{Stel je eens voor, dat!}{hij Othelloachtige}{oprispingen krijgt}\\

\haiku{Ik vrees, dat de Raad!}{van Justitie er gaarne}{meer van wil weten}\\

\haiku{Ondanks zijn flegma.}{liep het den Engelschman}{ijskoud over den rug}\\

\haiku{Men moet werkelijk!}{krankzinnig zijn om al dien}{onzin te slikken}\\

\haiku{{\textquoteright} {\textquoteleft}Zeg, Kees, ik ben ook,!}{een volbloed ben ik dan ook}{een boerepummel}\\

\haiku{Als alle Indo's,...............{\textquoteright} {\textquoteleft}}{zoo zijn als dat tweetalEn}{niet te vergeten}\\

\haiku{Twintig minuten.}{later bereikten zij de}{bekende badplaats}\\

\haiku{{\textquoteleft}Als er iets is, dat,.}{ik niet verdragen kan dan}{is het leedvermaak}\\

\haiku{Vooruit, Leo, laten!}{wij dat  reptiel op een}{paar steenen trakteeren}\\

\haiku{Die kwam zich bij mij.}{melden om zijn wraak op u}{beiden te koelen}\\

\haiku{Hij liep zooeven vlak!}{langs je en kon je toen met}{gemak neerschieten}\\

\subsection{Uit: Het spookhuis van Tandjong-Priok}

\haiku{{\textquoteright} {\textquoteleft}Wat een vraag, mijnheer,:}{Brandhorst maar als u het toch}{zoo graag weten wilt}\\

\haiku{{\textquoteright} {\textquoteleft}Het is gewoonweg{\textquoteright},.}{wonderbaarlijk riep de Jong}{bewonderend uit}\\

\haiku{{\textquoteright} {\textquoteleft}Gepeddeld wordt er,{\textquoteright},.}{wel maar naar huis gaan wij nog}{niet antwoordde Leo}\\

\haiku{{\textquoteright} {\textquoteleft}En daarna had ik......}{den Chef van de recherche}{gefeliciteerd}\\

\haiku{{\textquoteright} {\textquoteleft}Heel vriendelijk van{\textquoteright},, {\textquoteleft}?}{u antwoordde Leomaar is}{daar zooveel haast bij}\\

\haiku{{\textquoteleft}Wees maar blij, dat ik,}{niet meega want anders zou}{je daar op Zandvoort}\\

\haiku{Het was net, alsof.}{dit eene woord de schellen van}{zijn oogen deed vallen}\\

\haiku{Dames en heeren,,,.}{mijn dank voor den eh nectar}{en het ambrozijn}\\

\haiku{Zeg, Fientje, wat zou?}{je zeggen van een tochtje}{met een motorboot}\\

\haiku{{\textquoteright} {\textquoteleft}Dan zullen zij bot,{\textquoteright}.}{vangen dat kan ik je op}{een briefje geven}\\

\haiku{Eindelijk dan toch{\textquoteright},.}{heb ik je op een verzuim}{betrapt juichte Kees}\\

\haiku{Nu nog mooier, wat[]?}{moet ik in den nacht met}{een kiektoestel doen}\\

\haiku{Daarna vertelde,[]:}{hij dat het ook op het ker}{khof erg spookte}\\

\haiku{naar het kerkhof om.}{voorloopig het terrein}{te verkennen}\\

\haiku{Ik moet bekennen,;}{dat ik mij niet erg op mijn}{gemak gevoelde}\\

\haiku{Daarna schudde hij {\textquoteleft}}{de beide kameraden}{hartelijk de hand.}\\

\haiku{{\textquoteleft}Enfin, als het toch,.}{niet anders kan dan zal ik}{maar van wal steken}\\

\haiku{Leo tuurde scherp langs,.}{den weg welken zij zoo juist}{afgelegd hadden}\\

\haiku{Als ik dien vent ooit,{\textquoteright}.}{te pakken krijg dan zal ik}{hem eens mores leeren}\\

\haiku{Loop jij intusschen,,{\textquoteright}.}{kalm door ik zal wel fluiten}{als ik je noodig heb}\\

\haiku{Plotseling voelde,.}{Leo dat Kees hem krampachtig}{in den schouder greep}\\

\haiku{{\textquoteleft}Stel je gerust, Kees,,.}{het skelet is al weg maak}{je dus niet angstig}\\

\haiku{De schurk, de schavuit,!}{om mij op zoo'n manier bij}{den neus te nemen}\\

\haiku{{\textquoteleft}Grijp nooit het handje{\textquoteright}.}{van een jongedame in}{het volle daglicht}\\

\haiku{wij zoo vrij zijn het.}{zoogenaamde spookhuis eens}{te inspecteeren}\\

\haiku{Leo vertelde zijn,.}{makker wat hij met de Jong}{had afgesproken}\\

\haiku{Ik ben weliswaar,!}{knock-down geweest maar nog}{lang niet down-hearted}\\

\haiku{Alleen hebben wij{\textquoteright}.}{nog een appeltje met dien}{Ursus te schillen}\\

\haiku{{\textquoteleft}Laten wij Fientje.}{vragen hier een dag of acht}{te komen logeeren}\\

\haiku{Jij matigt je dus.}{mir nichts dir nichts de rechten}{van een censor aan}\\

\haiku{In spanning keken;}{de aanwezigen naar het}{brandende pakket}\\

\haiku{Op dit oogenblik.}{passeerde de hoofdagent met}{den gevangene}\\

\haiku{Juist wilde hij zich,.}{snel verwijderen toen Kees}{hem in den weg trad}\\

\haiku{Ik heb voor het eerst.}{van mijn leven tevergeefs}{op mijn kracht gebouwd}\\

\haiku{Heb jij, terwijl wij,?}{aan het schrijven waren niets}{ongewoons gemerkt}\\

\haiku{allemachtig, nu,!}{gaat mij een licht op de odeur}{heeft hen verraden}\\

\haiku{{\textquoteleft}Zeg, Leo, waarom heb,}{je niet op hem geschoten}{toen hij zoo tartend}\\

\haiku{Zwaar moest bedoelde!}{autoriteit dan ook voor}{zijn misstap boeten}\\

\haiku{Tot mijn spijt mag ik{\textquoteright},.}{mij hierover niet uitlaten}{antwoordde Brandhorst}\\

\haiku{En jij, Ziska, krijgt,.}{een man dien duizenden je}{zullen benijden}\\

\section{Bob den Uyl}

\subsection{Uit: De ontwikkeling van een woede}

\haiku{Steeds was hij bezig.}{ontslag te nemen om het}{te gaan proberen}\\

\haiku{Misschien heb ik wel,.}{eens iets in die geest gezegd}{maar ik zeg zoveel}\\

\haiku{Op zich zelf de meest.}{originele tekst die er}{te bedenken valt}\\

\haiku{Huilen, zeggen dat,.}{ik gelijk had dat ze het}{ook niet kon helpen}\\

\haiku{Na een tijdje ging,,.}{ze gekalmeerd weg en dat}{was het dan dacht ik}\\

\haiku{Kwam mijn bezoek niet,.}{gelegen dan zouden we}{wel weer verder zien}\\

\haiku{waarschijnlijk had ook.}{zij het leeglopen van het}{huis geobserveerd}\\

\haiku{wel weet ik nog dat.}{het op een Zuidzee-eiland}{gesitueerd was}\\

\haiku{Dat jij kijkt juist op.}{het ogenblik dat er wat met}{dat object gebeurt}\\

\haiku{Ben ik niet altijd.}{even aardig tegen dieren}{en oude mensen}\\

\haiku{Ik raap hem op, en.}{het blijkt een band te zijn van}{zo'n halve meter}\\

\haiku{Veel vertrouwen in;}{zijn navigatiekunsten}{heb ik daarom niet}\\

\haiku{En als zo vaak klopt,;}{het zelfs zo dat ik me een}{beetje bekocht voel}\\

\haiku{Een puts water over,.}{zijn hoofd zou helpen maar hoe}{daaraan te komen}\\

\haiku{Ik was er nog een;}{beetje te jong voor al vond}{ik het wel spannend}\\

\haiku{dat waardeerde ik,.}{al kwam je alleen maar om}{boeken te lenen}\\

\haiku{En toen was alles,.}{in \'e\'en klap afgelopen}{weer door mijn lachen}\\

\haiku{Ik neem afscheid, zeg}{hem dat ik over een dag of}{twee nog eens langs kom}\\

\subsection{Uit: Quatro primi}

\haiku{Mijn inboedel was.}{beperkt en praktisch in zijn}{geheel nieuw gekocht}\\

\haiku{Hij was zelfs nog zo.}{gewillig me tweeduizend}{gulden te lenen}\\

\haiku{Met sigaren en.}{steekpenningen had ik hem}{toch binnen een week}\\

\haiku{Als we uitgeput.}{raakten namen we een paar}{weken vakantie}\\

\haiku{Wat nu te doen met,?}{mijn rijkdom hoe zou ik mijn}{leven inrichten}\\

\haiku{Alles bijeen had.}{ik het heel wel naar mijn zin}{in de warme zon}\\

\haiku{De dochter van \'e\'en.}{van de topfiguren uit}{de streek ging huwen}\\

\haiku{In de avond belde,.}{er al iemand op wanneer}{ik nou precies ging}\\

\haiku{zo snel mogelijk,.}{naar een expert die in de}{arm was genomen}\\

\haiku{Snel stapte ik langs.}{de huizen in de richting}{van de verkeersweg}\\

\haiku{Er zat ongeveer.}{een millimeter ruimte}{tussen de twee hulsjes}\\

\haiku{Bij mijn binnenkomst.}{draaide hij zich om en keek}{me afwachtend aan}\\

\haiku{Vlug overlegde ik.}{hoe ik mijn wens duidelijk}{zou kunnen maken}\\

\haiku{Ik zie de sport niet.}{van het op en neer kruisen}{op een plas water}\\

\haiku{Maar, ofschoon ik van,.}{goede wille ben ik vind}{er geen vreugde in}\\

\haiku{Ik zeg dan ferm en.}{vrijmoedig dat hij volgens}{mij ongelijk heeft}\\

\haiku{Natuurlijk is de,.}{bedoeling dat ik meega}{dat begrijp ik wel}\\

\haiku{Ze moet toch op de.}{lange duur haar eigen stem}{niet meer herkennen}\\

\haiku{Het water spoelt over.}{mijn gezicht en ik krijg een}{slok zout naar binnen}\\

\haiku{Ik hoor door een waas,,?}{iemand een woord zeggen ben}{ik het is zij het}\\

\haiku{We blijken in een,.}{duinpan te liggen midden}{tussen de struiken}\\

\haiku{Het dringt langzaam tot.}{me door dat er dorens in}{mijn benen prikken}\\

\haiku{Ik hoop dat Riet haar,.}{mond zal houden ik heb geen}{zin wat te zeggen}\\

\haiku{Riet heeft haar badpak.}{aangetrokken en is klaar}{om terug te gaan}\\

\haiku{Mijn huid is lichtrood,.}{gebrand ik wil in ieder}{geval uit de zon}\\

\haiku{Ik verlang naar de,.}{straten van de stad stenen}{onder mijn voeten}\\

\haiku{vind ik dat, daar moet.}{je echt dronken voor zijn om}{er op te komen}\\

\haiku{De conducteur geeft.}{mij een schalkse knipoog die}{ik niet retourneer}\\

\haiku{Ik zie heel goed in}{dat het feit niet meer van haar}{te houden reden}\\

\haiku{Ik zie hoe de man.}{probeert Riet dichter tegen}{zich aan te drukken}\\

\haiku{Vlug, zegt Riet, ga met,.}{me dansen anders komt die}{vent me weer vragen}\\

\haiku{Ik zou me bevrijd,.}{moeten voelen maar daar is}{niets van te merken}\\

\haiku{Je moet gewoon doen,}{had de dokter gezegd doe}{zoveel mogelijk}\\

\haiku{Lekker hard rijden,.}{de weg schiet onder je door}{en dan de botsing}\\

\haiku{Touw geknapt, hoorde,.}{je later en je kwam met}{je hoofd op de grond}\\

\haiku{Door een schok kan dit,.}{spoor uitgewist worden zo}{was het ongeveer}\\

\haiku{Kool hoort een vrouw au,,.}{roepen het mannetje grijnst}{wordt dan weer ernstig}\\

\haiku{Klinkert is weer op.}{zijn tree gaan zitten met zijn}{hoofd in zijn handen}\\

\haiku{{\textquoteleft}Je bent oud,{\textquoteright} zegt hij, {\textquoteleft} {\textquotedblleft}}{als iemand dus tegen jou}{zegtoude meneer}\\

\haiku{{\textquoteleft}Grandioos zeg, geen,,!}{belediging hahaha}{kostelijk enorm zeg}\\

\haiku{{\textquoteright} Boven aan de trap.}{sist de vrouw weer met haar tong}{tussen de tanden}\\

\haiku{Op haar en op de.}{buren ben ik al jaren}{uitgedomineerd}\\

\haiku{{\textquoteright} De vrouw antwoordt niet.}{en klemt haar lippen samen}{tot een smalle streep}\\

\haiku{Maar toen gingen er.}{doden vallen en dat vond}{ik verschrikkelijk}\\

\haiku{{\textquoteleft}Ik wil wel helpen,{\textquoteright}, {\textquoteleft}?}{zegt zemaar denk je dat het}{wat uit zal maken}\\

\haiku{Klinkert schrikt op van,.}{het geluid kijkt snel omhoog}{en ziet het gevaar}\\

\haiku{{\textquoteright} {\textquoteleft}Wat,{\textquoteright} zegt Klinkert, {\textquoteleft}god,,?}{Eef waarom heb je me dan}{niet even gewaarschuwd}\\

\haiku{Misschien een kleine?}{geestelijke afwijking}{of iets dergelijks}\\

\haiku{En, dat hebben ze,.}{ook allemaal gemeen ze}{zakken te ver in}\\

\haiku{Ik had anders de,{\textquoteright}, {\textquoteleft}.}{indruk zegt Kooldat dat niet}{de eerste keer was}\\

\haiku{Het is de wereld.}{die wij onder elkaar de}{vierde schil noemen}\\

\haiku{{\textquoteright} {\textquoteleft}Ja, we komen er,{\textquoteright}:}{aan roept Klinkert vrolijk en}{zich tot Kool wendend}\\

\haiku{{\textquoteright} Kool wringt zich volgens.}{de aanwijzingen op de}{aangegeven plaats}\\

\haiku{Op aanwijzingen.}{van Klinkert wordt het tempo}{langzaam opgevoerd}\\

\haiku{Hij wankelt en moet.}{met zijn hand steun zoeken aan}{de rand van het bed}\\

\haiku{Als iemand hier het.}{recht heeft zich te beklagen}{ben ik het toch wel}\\

\haiku{{\textquoteright} {\textquoteleft}Ja,{\textquoteright} zegt Kool, {\textquoteleft}ik ga,.}{nu naar huis goeiendag en}{bedankt voor alles}\\

\haiku{Hij heeft het eens in.}{een tijdschrift zien staan en hij}{was er verrukt over}\\

\haiku{En neem de manier!}{waarop hij me dwong met hem}{naar boven te gaan}\\

\haiku{De angst komt terug,.}{met moeite weet hij deze}{te onderdrukken}\\

\haiku{{\textquoteright} {\textquoteleft}Natuurlijk,{\textquoteright} zegt Kool, {\textquoteleft}.}{ik geloof alleen niet dat}{dat je opzet was}\\

\haiku{{\textquoteleft}Klinkert,{\textquoteright} zegt hij dan, {\textquoteleft}!}{geef me toch een verklaring}{waar ik wat aan heb}\\

\haiku{Jongen jongen,{\textquoteright} kreunt, {\textquoteleft},?}{Klinkertwat heb je gedaan}{wat heb je gedaan}\\

\haiku{Achteraf bezien.}{is deze middag niet slecht}{voor mij verlopen}\\

\haiku{Neuri\"end loop je.}{naar de keuken en  drinkt}{daar een glas water}\\

\haiku{In New York heb ik.}{me uiteindelijk bij zo'n}{groep aangesloten}\\

\haiku{Elke dag feesten,.}{met meiden en drank zoveel}{als je hebben wou}\\

\haiku{Hij geeft de baas een.}{teken de glazen van de}{ronde te vullen}\\

\haiku{Niemand geloofde,.}{dat ze gooiden de hoorn neer}{of lachten me uit}\\

\haiku{Ik huur een auto,.}{neem een gunstige plaats in}{en wacht meneer af}\\

\haiku{Soms moet ik zelfs een.}{opdracht weigeren omdat}{ik het te druk heb}\\

\haiku{Waarom loop je dan?}{niet weg als je te lang naar}{je zin moet wachten}\\

\haiku{Je berust, je staat,,.}{voor de deur in regen en}{wind wetend wachtend}\\

\haiku{Hij houdt zijn vreugde,.}{in een lachend gezicht zou}{nu geen pas geven}\\

\haiku{Hij gaat in zijn bank.}{zitten en kijkt verslagen}{naar de man voor hem}\\

\haiku{ik meed eveneens de.}{gruwelkelder waar weinig}{te gruwelen viel}\\

\haiku{Ik heb me altijd.}{afgevraagd wat andere}{mensen daar aantrekt}\\

\haiku{Die klanken trekken.}{me aan op een manier die}{ik zelf niet begrijp}\\

\haiku{Want denk niet, zei hij,.}{dat het een aangename}{gewaarwording is}\\

\haiku{Geleidelijk had.}{hij ook een zekere macht}{over mij gekregen}\\

\haiku{{\textquoteleft}U bent de enige.}{met wie De Galande op}{het ogenblik omgaat}\\

\haiku{We praatten nog wat,.}{besloten elkaar bij de}{voornaam te noemen}\\

\haiku{Vervelend voor een,.}{beginnend onderzoeker}{maar niets aan te doen}\\

\haiku{Toen ik dat zag kreeg,}{ik ook geweldige trek}{maar al mijn sloffen}\\

\haiku{Bovendien is het,.}{landschap heuvelachtig met}{hier en daar kloven}\\

\haiku{Achter zijn woorden.}{proef ik die fatale hang}{naar roem en glorie}\\

\haiku{Uren zaten we daar,.}{schipbreukelingen op een}{onbewoond eiland}\\

\haiku{Maar wat er ook mocht,.}{naderen ik zou er niet}{beter van worden}\\

\haiku{In haar ogen kon ik,.}{ook mijn gezicht zien ik keek}{in mijn eigen ogen}\\

\haiku{Ik voelde traag en,.}{week de angst opkomen een}{lauw dier in je hoofd}\\

\haiku{Als er dan wat tijd,.}{overheen is gegaan vergeet}{ik het zelf ook weer}\\

\haiku{Het opknappen van.}{de cellen moet van tijd tot}{tijd toch gebeuren}\\

\haiku{Mijn naam, die hem toch,.}{bekend moet zijn heb ik hem}{nooit horen noemen}\\

\haiku{Toch krijg ik niet de.}{indruk dat hij werkelijk}{onvriendelijk is}\\

\haiku{Ook denk ik wel eens.}{dat hij verlegen is met}{de situatie}\\

\haiku{Ik zou niet weten.}{wat te beginnen als ik}{er uit zou moeten}\\

\haiku{Dit ogenblik kan net.}{zo goed twee maanden of twee}{jaar geleden zijn}\\

\haiku{Het was een schande.}{dat de leiding van het huis}{hierin niet voorzag}\\

\haiku{Dan gaat hij terug.}{naar mijn cel en legt schone}{lakens op mijn bed}\\

\haiku{Het was vervelend,.}{het onderhoud was wel wat}{kort uitgevallen}\\

\haiku{Het raam staat nog steeds,.}{open zie ik zelfs dat hebben}{ze niet dichtgedaan}\\

\haiku{In heldere rust.}{keerde ik naar mijn lichaam}{en de pijn terug}\\

\haiku{Maar ik wil nog wel,.}{een uurtje blijven liggen}{op die oude boot}\\

\haiku{Dat we geen hopen.}{kinderen verwekten was}{onbegrijpelijk}\\

\haiku{Zwijgend keek ik haar,.}{na ook haar achterzijde}{was de moeite waard}\\

\haiku{Thuis pakte ik het.}{telefoonboek en zocht de}{bovenste naam op}\\

\haiku{Het was zaak om, zo,.}{lang het nog voor de wind ging}{de tijd te rekken}\\

\haiku{Ik besloot daar nog,.}{even mee te wachten het zou}{mee kunnen vallen}\\

\haiku{hier was de man zelfs,.}{niet te zien die stond zeker}{de vaat te wassen}\\

\haiku{Na een half uur hief.}{ik het hoofd op om eens om}{me heen te kijken}\\

\haiku{Zij vermoeden bij;}{hem geheime reserves}{die zij niet kennen}\\

\haiku{Hij staat op en gaat.}{naar de toiletruimte aan}{het eind van de gang}\\

\haiku{Om vijf uur haast hij,.}{zich naar huis zijn kin diep in}{zijn sjaal gedoken}\\

\haiku{Even later komt ze.}{er weer uit en houdt de deur}{uitnodigend open}\\

\haiku{{\textquoteleft}Dag meneer Jaichek,,.}{mijn naam is Voogt waarmee kan}{ik u van dienst zijn}\\

\haiku{{\textquoteleft}Meneer Jaichek,{\textquoteright} zegt, {\textquoteleft},}{hijlaten we even niet over}{dat ontslag praten}\\

\haiku{dit ter tafel te,.}{brengen zelfs al zou hij het}{van plan zijn geweest}\\

\haiku{Hij gaat zelfs zo ver;}{de oude heer lachend op}{de schouder te slaan}\\

\haiku{Steeds was hij bezig.}{ontslag te nemen om het}{te gaan proberen}\\

\haiku{Misschien heb ik wel,.}{eens iets in die geest gezegd}{maar ik zeg zoveel}\\

\haiku{Op zich zelf de meest.}{originele tekst die er}{te bedenken valt}\\

\haiku{Huilen, zeggen dat,.}{ik gelijk had dat ze het}{ook niet kon helpen}\\

\haiku{Na een tijdje ging,,.}{ze gekalmeerd weg en dat}{was het dan dacht ik}\\

\haiku{Kwam mijn bezoek niet,.}{gelegen dan zouden we}{wel weer verder zien}\\

\haiku{waarschijnlijk had ook.}{zij het leeglopen van het}{huis geobserveerd}\\

\haiku{wel weet ik nog dat.}{het op een Zuidzee-eiland}{gesitueerd was}\\

\haiku{Dat jij kijkt juist op.}{het ogenblik dat er wat met}{dat object gebeurt}\\

\haiku{Ben ik niet altijd.}{even aardig tegen dieren}{en oude mensen}\\

\haiku{Ik raap hem op, en.}{het blijkt een band te zijn van}{zo'n halve meter}\\

\haiku{Veel vertrouwen in;}{zijn navigatiekunsten}{heb ik daarom niet}\\

\haiku{En als zo vaak klopt,;}{het zelfs zo dat ik me een}{beetje bekocht voel}\\

\haiku{Een puts water over,.}{zijn hoofd zou helpen maar hoe}{daaraan te komen}\\

\haiku{Ik was er nog een;}{beetje te jong voor al vond}{ik het wel spannend}\\

\haiku{dat waardeerde ik,.}{al kwam je alleen maar om}{boeken te lenen}\\

\haiku{En toen was alles,.}{in \'e\'en klap afgelopen}{weer door mijn lachen}\\

\haiku{Ik neem afscheid, zeg}{hem dat ik over een dag of}{twee nog eens langs kom}\\

\chapter[19 auteurs, 3311 haiku's]{negentien auteurs, drieduizenddriehonderdelf haiku's}

\section{Rink van der Velde}

\subsection{Uit: Feroaring fan lucht}

\haiku{D\^er wol ik libje,.}{wenjen bliuwe solang my}{God dit libben skinkt}\\

\haiku{Foekje hat noch wat,}{giele beantsjes en tusearte}{boud mar dy binne}\\

\haiku{Se tuge inoar \^of.}{dat it in lust is en Durk}{bliuwt der foar stean}\\

\haiku{Durk stapt op de pomp.}{ta en tapet himsels in}{m\^ulfol k\^ald wetter}\\

\haiku{It is sa'n moaie.}{blakstille septimberdei}{en dan klinkt it fier}\\

\haiku{{\textquoteright} Durk stelt him foar it.}{front fan syn h\'ush\^alding op}{en sjocht nei Foekje}\\

\haiku{Dit is Johan, heit,{\textquoteright} seit, {\textquoteleft}.}{Gryt sljochtweien hy hat hjir}{noch net earder west}\\

\haiku{{\textquoteright} {\textquoteleft}Wel ja, sprek mar op,.}{likegoed sille ik en}{mem it der oer ha}\\

\haiku{{\textquoteright} {\textquoteleft}In penty is wat,.}{oars dat is in broekje mei}{hoazzen der oan}\\

\haiku{{\textquoteright} Durk giet even op it.}{potrak neist de pomp sitten}{te bekommen}\\

\haiku{Akkoart, ha 'k sein,.}{mar dan mei hjir en d\^er in}{ferklearring derby}\\

\haiku{Dat ienkear in dief,.}{altiten in dief wie al}{wie it mar in fyts}\\

\haiku{Nee, neat op tsjin, mei, '.}{de tiid meigean hak}{altiten al sein}\\

\haiku{Sa hawwe wy,.}{jim mem ek opbrocht tiden}{hawwe tiden}\\

\haiku{En wy mochten net,.}{fuotbalje wy wienen der}{om wat te learen}\\

\haiku{Mar om de wille,.}{achter sa'n bal oandrave}{is sa dom as wat}\\

\haiku{Yn de earste jierren.}{nei de oarloch hienen wy}{ek noch bonnen mei}\\

\haiku{As jo wat leard ha,,.}{stiet de wr\^ald foar jo iepen}{sis ik altiten}\\

\haiku{Ik wie der krekt op '.}{e tiid by en koe dat}{knyntsje noch r\^ede}\\

\haiku{En as it nypt en.}{wedernypt dan binne se}{o sa ienriedich}\\

\haiku{Dit wie fansels al '.}{wat wreed en Foekje g\^ulde}{it opt l\^est \'ut}\\

\haiku{Dizze kin wol tsien, '.}{kear om Anema hinne dat}{hak him al \^ofloerd}\\

\haiku{W\^es bliid dat wy,.}{it bewenje sa kriget}{it syn \^underh\^ald}\\

\haiku{Hy soe it lykwols,,}{r\^eden ha seit hy der sels}{fan as er mar jild}\\

\haiku{Dat is te sizzen,.}{se wolle dat er oar en}{lichter wurk siket}\\

\haiku{Durk is doe knap lulk.}{wurden en hat it oanbrocht}{by de direkteur}\\

\haiku{Dy hie it bern der.}{al ynhelle ear't Durk der}{wat fan sizze koe}\\

\haiku{It liket derop, '.}{dat Gryt har fant winter}{yn Drachten deljout}\\

\haiku{Hy wit wol dat ik,}{\'ut en troch in knyntsje snip}{mar hy wit ek wol}\\

\haiku{Ik meitsje it dien,{\textquoteright}, {\textquoteleft}.}{sei ermar rekkenje dan}{mar net mear op my}\\

\haiku{{\textquoteright} {\textquoteleft}Noch twa jier dan kin ',{\textquoteright}.}{k yn de sanearing}{sei L\'utzen Bouma}\\

\haiku{{\textquoteright} {\textquoteleft}Dat r\^ede wy wol, ' '.}{25 g\^une kink wol \'ut}{e boeken h\^alde}\\

\haiku{As dy poat fan dy,.}{better is geane wy}{wer tegearre}\\

\haiku{En ik kin der net,.}{deun by lizzen gean do witst}{hoe skerp sa'n bok r\^ukt}\\

\haiku{{\textquoteright} {\textquoteleft}Do fertroust dysels,{\textquoteright}.}{ek net sa bot mear sei Ids}{spytgnyskjend}\\

\haiku{Oare wike hast him,.}{wer Ids en ik nim dy in}{pear p\^un reefleis mei}\\

\haiku{Yn 't hok klaut er.}{in steal \^alde ielf\^uken}{fan it souderke}\\

\haiku{{\textquoteright} {\textquoteleft}Wy kinne der skea,.}{fan ha kinst him better wat}{temjitte komme}\\

\haiku{{\textquoteright} {\textquoteleft}Welja, oars helje,{\textquoteright}.}{jim der noch in amtner by}{oan seit Durk haatlik}\\

\haiku{Der stiet net folle, '.}{stream mar d\^er ist neffens}{Durk net minder om}\\

\haiku{{\textquoteleft}Ja, dat is moai,...{\textquoteright}.}{mar Klaske sei niis Durk}{l\^est it briefkaartsje}\\

\haiku{It is in goede '.}{man en hy is opt l\^est}{mei \'us Aaltsje troud}\\

\haiku{Der komt in stuit dat.}{Durk de l\^este kofje der\'ut}{nimt en even stil falt}\\

\haiku{As wy de winter}{troch moatte mei njoggen}{tsientsjes en as se}\\

\haiku{Witst wol wat in stik '?}{klean opt heden kostet}{en wat it fleis jildt}\\

\haiku{Asto de boel wei, ' '.}{griemste h\^ald ik dy ope}{harsens ynt Djip}\\

\haiku{Eartiids hat er.}{Foekje wolris in klap j\^un}{as hja spul hienen}\\

\haiku{Mar as it dan dochs,{\textquoteright}.}{net mear te kearen is}{mimert Durk fierder}\\

\haiku{{\textquoteright} {\textquoteleft}D\^er hast gelyk oan,,?}{mem mar moat dat no mei}{keunstmiddels en sa}\\

\haiku{Ik wie sels earst ek,.}{net wizer mar ik ha der}{in soad oer neitocht}\\

\haiku{It hontsje jout ien kear,.}{l\^ud in s\^eft mar ferheftich}{f\^ul piipjen}\\

\haiku{Hy reaget de '.}{t\^uken oane kant en docht}{in stap nei foaren}\\

\haiku{It swit prikelt him.}{\^under de pet en hy smyt}{it ding fan him \^of}\\

\haiku{It is oars goed iten,.}{sa'n grouwe pike dy't noch}{krekt net flechtich is}\\

\haiku{Bijke giet tsjin him.}{oan lizzen en it hontsje jout}{aardich waarmte \^of}\\

\haiku{{\textquoteright} Bijke sn\'uft yn 'e,.}{sleatsw\^al om hy hat lucht}{fan in rot of sa}\\

\haiku{As Feike Dam d\^er,.}{west hie soed er no al wat}{\^undernommen ha}\\

\haiku{Bouwe syn bromfytsark,,.}{dat oeral omslingeret}{smyt er op in bult}\\

\haiku{{\textquoteright} Tsjerk is achter\'ut nei.}{de reed riden en dan giet}{it oer de br\^ege}\\

\haiku{Duvel, wat sil er '.}{lulk w\^eze as er dit yn}{e gaten kriget}\\

\haiku{It is in edel dier,,.}{sa'n reebok soks moat mei}{ferdrach dien wurde}\\

\haiku{De iene is in, '.}{bytsje stikken mar d\^er kin}{k ek neat oan dwaan}\\

\haiku{De bern meie it net,.}{br\^uke want  oars soe it}{samar op w\^eze}\\

\haiku{{\textquoteright} {\textquoteleft}Dat hoecht no net mear,{\textquoteright}.}{seit Durk wylst hy stean bliuwt}{en in sjekje draait}\\

\haiku{Dat jout neat, mar hy.}{sit as de duvel achter}{de streuperij oan}\\

\haiku{Durk trouwens ek, mar.}{hy wol dat gejeuzel net}{langer oanhearre}\\

\haiku{Mar jo dogge wat,.}{sokke tiden jo smite}{josels heal wei}\\

\haiku{{\textquoteright} {\textquoteleft}Ik soe it der fan ',.}{e wike mei har oer ha mar}{se lake my \'ut}\\

\haiku{It wurdt minder,{\textquoteright} seit, {\textquoteleft}}{er karmasterjendder leit}{in b\^este bocht yn}\\

\haiku{Yn \'us gemeente,{\textquoteright}.}{hie it der al lang \^of west}{ferklearret Gerard}\\

\haiku{{\textquoteright} Durk docht oprjocht, '.}{syn b\^est mar it wol him net}{ynt sin komme}\\

\haiku{om stean litte as.}{ik dy oanjaan soe by de}{boargerlike st\^an}\\

\haiku{{\textquoteleft}De oare deis soest.}{de middeis th\'us komme en}{dan nei de Sweach ta}\\

\haiku{{\textquoteleft}Goed \^unth\^alde, Feike,{\textquoteright}, {\textquoteleft}:}{seit er mei triljend l\^uddit}{moatst goed \^unth\^alde}\\

\haiku{{\textquoteleft}It fel ek noch fol,.}{hagel no bar ik der net}{iens twa kwartsjes foar}\\

\haiku{No moatst \'ut 'e,,{\textquoteright}.}{wei w\^eze oars falle der}{deaden raast Durk}\\

\haiku{Durk pakt syn fyts en.}{trapet twa kear sa hurd as}{oars nei de Sweach ta}\\

\haiku{{\textquoteright} {\textquoteleft}Wat binne jo in,{\textquoteright}.}{min persoan seit Durk mei}{de klam op elk wurd}\\

\haiku{{\textquoteright} It binne sokke,.}{liepe praters dy minsken}{fan de gemeente}\\

\haiku{{\textquoteleft}It moat net te,.}{leabrekkend w\^eze ik sil}{der achteroan}\\

\haiku{Der wurdt oer praat om,{\textquoteright}.}{my op fyftich persint te}{setten seit er th\'us}\\

\haiku{moat fansels earst,.}{wol wenne benammen foar}{Kekke en Bijke}\\

\haiku{It leit my sa by,,.}{Ids dat it mei my yn dat}{Drachten net goed komt}\\

\haiku{Dy strewellen fan,}{jo hingje in heul ein \'us}{kant oer dat it stiet}\\

\haiku{Ik kin it op 't,.}{heden net byh\^alde de}{holle rint my om}\\

\haiku{Skoalle tolve,}{is foar dizze strjitten}{mar it is de fraach}\\

\haiku{{\textquoteright} Hja sil it earst wol.}{betelje en letter mei}{mem ferrekkenje}\\

\haiku{Jo stjoere de bern.}{moarnier mar en der sil}{mei r\^eden wurde}\\

\haiku{It is my krekt as ' '.}{hiek in heule lange}{dei yne put stien}\\

\haiku{As se in oar der,.}{mei pakke sykje se \'ut}{w\^er't er wei komt}\\

\haiku{It soe in moaie.}{fertoaning wurde en g\^ans}{opskuor jaan}\\

\haiku{De jonge hie oan.}{de S\^anleane altiten}{aardich syn slinger}\\

\haiku{Ik ha hjir hjoed twa '.}{kear dat minske fan hjir neist}{wei oane doar h\^an}\\

\haiku{En heit soe yn 'e,.}{keamer de skuon \'utdwaan}{it w\^adet sa yn}\\

\haiku{Earst in pear dagen ',{\textquoteright}.}{ynt hok en lit him dan}{mar rinne seit Durk}\\

\haiku{{\textquoteright} {\textquoteleft}En ik fernim no.}{al hoe goed it my dwaan sil}{om der even by wei}\\

\haiku{Ik wol dy winkels,.}{ek wolris besjen ik ha der}{noch gjin stap \'ut west}\\

\haiku{{\textquoteleft}Oars hie myn jonge, '.}{net mei in stik izer slein dat}{hak se net leard}\\

\haiku{{\textquoteleft}Gjin sprake fan, dy.}{bern hawwe leard om fan}{har \^of te biten}\\

\haiku{Hy raast twa slaggen ',:}{ome pomp hinne lyk as}{eartiids en ropt}\\

\haiku{Se hawwe der.}{rekken mei holden dat it}{nachts aardich fris wurdt}\\

\haiku{Se kr\^upe s\'untsjes by.}{de w\^al op en geane}{in eintsje tebek}\\

\haiku{{\textquoteright} {\textquoteleft}Dan keapje wy,{\textquoteright}.}{ien seit Meine en hy lit}{de flesse omgean}\\

\haiku{{\textquoteleft}Ik tink dat wy wol,,{\textquoteright}.}{in santich ielen ha heit}{seit Willem optein}\\

\haiku{Ast op elke trije, '.}{d\^obers in iel hast ist}{b\^est en wy ha mear}\\

\haiku{Hy set de fyts dwars '.}{oert paad sadat se der}{wol \^of moatte}\\

\haiku{Hy kwakt de fyts tsjin '.}{e beam en is yn in pear}{stappen by Feike}\\

\haiku{Mar de chef seit ek.}{dat jo gauris in skoftsje}{steane te sjen}\\

\haiku{Tsjerk wol har ek sa'n,:}{krantsje ta ha hy is der}{rejaal mei en seit}\\

\haiku{{\textquoteleft}Hy falt gewoan ',.}{yne klean fan lytse Durk}{hy kostet \'us neat}\\

\haiku{En de krystbeam is,.}{sa grut \'utfallen hy kin}{amper troch de doar}\\

\haiku{Mar dan moatte:}{sokke mannen net l\^ebich}{wurde en sizze}\\

\haiku{Foekje en Klaske.}{knikke wakker by alles}{wat de man oanfiert}\\

\haiku{Mar hy kaam mei de,.}{lep achter my oan dat doe}{moast ik ek wol}\\

\haiku{Jo ha fansels wol '.}{yne rekken dat ik mei}{jo yn proses gean}\\

\haiku{En wa is dizze?}{skriuwer dat er de skiednis}{ferfalse mei}\\

\haiku{D\^er tinke sokke,.}{mannen fansels net oan mar}{in oar sit der mei}\\

\haiku{Dit hat Drachten op.}{syn minst in middelgrutte}{yndustry koste}\\

\haiku{Dit der even tusken,.}{troch al moat hjir net te}{licht oer praat wurde}\\

\haiku{Ik rin hjir yn myn.}{frije tiid ek mei de siel}{\^under de earm om}\\

\haiku{it sa swier hast Durk,.}{dan moatte wy dochs wat}{foar dy \'utfine}\\

\haiku{It is Durk min nei ',.}{t sin mar hy kin him der}{net foarwei wine}\\

\haiku{Hy fotografeart}{fan moarns oant j\^uns en plakt}{de brut yn grutte}\\

\haiku{in eigenwiis.}{man dy't mient dat er al hast}{boargemaster is}\\

\section{Paul van der Velden}

\subsection{Uit: Darrenslacht}

\haiku{Ik vraag haar waarom.}{niemand mij heeft verteld over}{mijn dansende opa}\\

\haiku{Toch klopte ze aan.}{en op een brommend antwoord}{opende ze de deur}\\

\haiku{Maar het bedrag van.}{f23,50 was niet helemaal bij}{elkaar gesprokkeld}\\

\haiku{Moest ze de tand over?}{haar linkerschouder gooien}{of over haar rechter}\\

\haiku{Dat was een strenge,.}{die je ongezouten de}{waarheid kon zeggen}\\

\haiku{De volgende dag,.}{vertrek ik naar Fatima}{een bedevaartsoord}\\

\haiku{Ik hoor het kraken.}{van zijn botten en dan zakt}{hij door zijn pootjes}\\

\haiku{Zijn vingers schreven.}{voorbeeldletters in de lucht}{bij het schoonschrijven}\\

\haiku{Ik hoor niets, alleen.}{het krassen van haar nagel}{tegen de potwand}\\

\haiku{{\textquoteright} ~ Veel te laat, tien,.}{jaar later kom ik haar weer}{tegen in de stad}\\

\haiku{{\textquoteright} ~ {\textquoteleft}Twee weken was,.}{hij weg toen de zandziener}{zijn aanzegging deed}\\

\section{Adriaan Venema}

\subsection{Uit: Lemmingen (onder ps. A. ten Hooven)}

\haiku{Ik draaide bij Utrecht.}{de weg af en keerde naar}{Amsterdam terug}\\

\haiku{waar het lag tot de.}{leraar opstond en langzaam}{zijn richting uitkwam}\\

\haiku{Dani\"el kreeg een.}{kleur en glipte langs de man}{naar de kleedkamer}\\

\haiku{Hij sperde zijn ogen.}{wijd open om maar niets van het}{schouwspel te missen}\\

\haiku{Ze stonden roerloos,,.}{een somber groepje en ze}{keken hem strak aan}\\

\haiku{{\textquoteright} Een van de mooiste.}{ogenblikken uit zijn leven}{zou nu voorbijgaan}\\

\haiku{De monoloog aan.}{de andere kant van de}{lijn duurde niet lang}\\

\haiku{{\textquoteright} {\textquoteleft}Nou, mama, misschien,.}{zijn ze aardig tegen u}{maar niet tegen ons}\\

\haiku{Zijn vader stapte,.}{in gevolgd door een van de}{twee marechaussees}\\

\haiku{Er is niets tegen.}{me gezegd over mensen die}{rond willen lopen}\\

\haiku{Iedereen doet zo,.}{optimistisch maar ikzelf}{zie het somber in}\\

\haiku{Nu zijn mijn boeken,.}{me lief maar mijn hachje heb}{ik er niet voor over}\\

\haiku{Hij verwachtte dat,.}{het uiteen zou spatten maar}{dat gebeurde niet}\\

\haiku{Dani\"el zou het,.}{huis binnengelopen zijn}{achter de man aan}\\

\haiku{\'e\'en dode voor zo'n.}{middag zou hem voldoende}{bevredigd hebben}\\

\haiku{wie in je weg komt,,.}{te staan die sla je neer die}{hoor je te doden}\\

\haiku{Zij hebben het met,.}{Kwik te zamen gezongen}{kort geleden nog}\\

\haiku{Ik weet het niet - zijn,?}{ouders zouden die dat goed}{gevonden hebben}\\

\haiku{Ik blijf er geen uur -.}{langer door leven als ik}{dat al zou willen}\\

\haiku{Bovendien had hij.}{de man nodig om een krant}{te bemachtigen}\\

\haiku{Zoals een leeuw zich.}{brullend opricht als hij wordt}{lastig gevallen}\\

\haiku{Even later kwam ze.}{terug met een gaasje en}{een grote pleister}\\

\haiku{Dit zou zijn eerste.}{actie zijn die hij los van}{Micha ondernam}\\

\haiku{{\textquoteright} Alleen was de kans}{dat hij hem met de asbak}{zou kunnen doden}\\

\haiku{Tot welk niveau moet?}{ik me dan wel verlagen}{om quitte te staan}\\

\haiku{Hij staat op en geeft {\textquoteleft}?}{me een hand.Is de waarheid}{nu nog relevant}\\

\haiku{Hij heft de arm met.}{het zwaard op het moment dat}{hij vindt dat het moet}\\

\haiku{{\textquoteleft}En ik blijf erbij.}{dat we het plebs buiten ons}{huis moeten houden}\\

\haiku{{\textquoteleft}Ik had eens op een.}{middag een meisje uit de}{buurt meegenomen}\\

\haiku{Ik kleed me aan en.}{loop terug naar het bed om}{af te rekenen}\\

\haiku{{\textquotedblleft}We zijn niet gek, we}{geven jullie door aan de}{zedenpolitie}\\

\haiku{{\textquoteleft}Weet u nog dat we?}{over een portret van de F\"uhrer}{hebben gesproken}\\

\haiku{De cops grijpen je,,?}{zodra ze daar maar een kans}{voor zien weet je dat}\\

\haiku{Maar als je daar de,.}{krant staat te lezen kunnen}{ze je niet grijpen}\\

\haiku{Er zijn daar een paar.}{kiosken en die krant heb}{je er net gekocht}\\

\haiku{Ga maar naar Mama,.}{Meyer op 2nd Avenue vlak}{bij het Bay Hotel}\\

\haiku{Hij probeerde een.}{martiale trek op zijn}{gezicht te krijgen}\\

\haiku{Ik weet niet meer wat,:}{hij allemaal zei maar \'e\'en}{ding weet ik nog wel}\\

\haiku{{\textquoteleft}Geef me dan maar bier,{\textquoteright},.}{zei Roskam op zijn beurt het}{klonk heel grootmoedig}\\

\haiku{Zijn gezicht stond kwaad,.}{want hij begreep Dani\"els}{vader nu heel goed}\\

\haiku{Dani\"el gaf de.}{anderen een teken dat}{ze moesten blijven staan}\\

\haiku{Hij was nauwelijks,.}{trots maar dat nam mijn eigen}{voldoening niet weg}\\

\haiku{Hij had een zachte.}{stem en hij sprak Duits met een}{zangerig accent}\\

\haiku{De graaf verbrak het:}{stilzwijgen door te vragen}{hoe het op school ging}\\

\haiku{Dani\"el voelde.}{een hevige opwinding}{in zich opkomen}\\

\haiku{Hij viel in zijn stoel.}{terug en dronk snel een paar}{teugen uit zijn glas}\\

\haiku{{\textquoteleft}We praten over strijd,{\textquoteright}, {\textquoteleft}.}{zei hij langzaammaar het gaat}{daar niet alleen om}\\

\haiku{{\textquoteright} Dani\"el hoorde.}{de zoemer en het klikken}{van de deuropener}\\

\haiku{De ene man liep naar.}{de auto en sjorde de}{bestuurder eruit}\\

\haiku{En kogels fluiten,,:}{toch dacht hij want dat had hij}{uit westerns geleerd}\\

\haiku{Hij probeerde te,.}{lachen zodat zijn rotte}{tanden bloot kwamen}\\

\haiku{Hij keek in de wieg.}{naar het rode hoofd van het}{pasgeboren kind}\\

\haiku{Hup, jongen, smeer 'm,{\textquoteright}.}{beet de man hem met een stem}{vol minachting toe}\\

\haiku{De vrouw schuift met een.}{ruk haar stoel naar achteren}{en zakt onderuit}\\

\haiku{Eigenlijk leefde.}{hij alleen nog maar om zijn}{tijd uit te dienen}\\

\haiku{De wereld is niet,.}{gebrand op de waarheid de}{wereld kijkt naar kracht}\\

\haiku{Hij vroeg Dani\"el.}{rechtop te gaan staan en zich}{niet te bewegen}\\

\haiku{Toen sprong Dani\"el.}{met de kat in zijn armen}{van de gaanderij}\\

\haiku{Ze schreeuwden hoog en.}{schel zodra de vlammen hun}{veren bereikten}\\

\haiku{Hij vloekte, want tijd.}{was juist op dit moment erg}{belangrijk voor hem}\\

\haiku{Hij liep langs de trap.}{de grote zitkamer in}{en deed het licht aan}\\

\haiku{In het licht van de.}{maan zag hij haar op het pad}{naast het huis liggen}\\

\haiku{{\textquoteright} Dani\"els moeder.}{begon te huilen en hij}{bracht haar naar boven}\\

\haiku{{\textquoteleft}Wat let me om de?}{gehele wereld aan mijn}{voeten te werpen}\\

\haiku{Toen de vrouw opkeek,.}{legde hij zijn hand haastig}{op het tafelblad}\\

\haiku{Het was alsof hij.}{begreep dat aan alles een}{einde moest komen}\\

\haiku{Trams die van links naar.}{rechts reden en trams die van}{rechts naar links gingen}\\

\haiku{Even later kwamen.}{de mannen de trap weer af}{en de keuken in}\\

\haiku{Het was laat in de.}{avond toen zijn vader en zijn}{zuster thuiskwamen}\\

\haiku{Even later sloeg de.}{motor van hun auto aan}{en het was voorbij}\\

\haiku{Hij was nu weer een.}{lafaard om daar niet voor te}{durven uitkomen}\\

\haiku{Ver weg hoorden ze.}{de klagende tonen van}{een accordeon}\\

\haiku{Ze stond op zonder.}{een woord af te wachten en}{liep naar de keuken}\\

\haiku{De mannen sleepten.}{zijn vader nu tussen hen}{in over het tuinpad}\\

\haiku{Hij ging naar een groot,.}{gebouw waar militairen}{in en uit liepen}\\

\haiku{Er mag geen steen of.}{welke naamsaanduiding dan}{ook worden geplaatst}\\

\haiku{Aan de rand van het.}{tuinpad stond een spade in}{de grond gestoken}\\

\haiku{Haar armen gingen,.}{even omhoog maar zakten toen}{slap naast haar lichaam}\\

\haiku{Tevreden bleef hij.}{enkele ogenblikken naar}{het lichaam kijken}\\

\haiku{Twee mannen trokken.}{haar broek uit en smeerden haar}{onder met menie}\\

\haiku{Ze liepen naar een.}{duinpan waar een peloton}{stond aangetreden}\\

\haiku{{\textquoteleft}Er zijn er niet veel,{\textquoteright}.}{meer uit die jaren die nog}{leven zei de graaf}\\

\haiku{Soms kostte het hem.}{moeite zich zijn vader voor}{de geest te halen}\\

\haiku{Zijn broer liet nooit iets.}{na om zijn minachting voor}{Micha te tonen}\\

\haiku{Hij lag ruggelings,.}{op het vloerkleed een gaatje}{in het voorhoofd}\\

\haiku{Een verpleegster geeft.}{me een injectie om de}{pijn te verlichten}\\

\haiku{{\textquoteleft}En mag ik u nu,{\textquoteright}.}{groeten zei de man en hij}{liep het hotel uit}\\

\section{Gustaaf Vermeersch}

\subsection{Uit: De last}

\haiku{Daar peuterde hij.}{een wijle en dan kwam zijn}{uurwerk te voorschijn}\\

\haiku{Ze gingen voort en '.}{t wijf stak geniepig de}{druppelglazen weg}\\

\haiku{'t opgezweept bloed.}{verdoofde zijn blikken en}{bezwaarde zijn hoofd}\\

\haiku{- Ge zegt dat lijk of!}{iemand verplicht ware u}{die kans te geven}\\

\haiku{Ze bedolven zich,.}{daar in een hoek en keken}{toe in stilzwijgen}\\

\haiku{Dan vaagde hij met.}{zijn zakneusdoek zijn zweet af}{en bezag haar schuins}\\

\haiku{De ouden stonden.}{op en torden stijf en zich}{rekkend naar de deur}\\

\haiku{Soms voelde hij wat,;}{steken in zijn kop maar gaf}{daar weinig acht op}\\

\haiku{Plots scheidde hij weer.}{uit en hield de oogen star in}{de ruimte gericht}\\

\haiku{Ze had heur schorte.}{vol boterkoeken die ze}{op tafel gooide}\\

\haiku{Ze verwijderden.}{zich en hij werd razend van}{machtelooze kwaadheid}\\

\haiku{Moeder sprak niet over ',;}{t meisje haar armoe ze}{waren zij ook arm}\\

\haiku{Hij sprak met volle ';}{overtuiging omdat hij wist}{datt niet waar was}\\

\haiku{Ik kwam u vragen.}{of ge soms niet wenschte}{getuige te zijn}\\

\haiku{hij verplicht zich op,}{een dorpel neer te zetten}{hij kon niet meer voort}\\

\haiku{hij voelde een juk.}{op zijn schouders wegen dat}{hem terneerdrukte}\\

\haiku{'t Wijf vond dat ook ';}{geweldig geestig en ze}{giecheldet uit}\\

\haiku{als ze op zijn teenen,}{torden of hem begoten}{zoo v\'eel avonden moest}\\

\haiku{Toen hij eindelijk.}{op straat kwam voelde hij nog}{zijn moed verflauwen}\\

\haiku{- Ja. - 't Is rap, 't!}{is nog maar gewillig een}{maand dat ge verkeert}\\

\haiku{Dit zicht prikkelde.}{hem zoo zeer dat hij niet meer}{kon blijven liggen}\\

\haiku{Moeder sprak tegen '.}{datt niets was en begon}{bijeen te vagen}\\

\haiku{Jan zei niet veel, hij,.}{keek onophoudelijk de}{straat op ongerust}\\

\haiku{De brandewijn werd.}{opnieuw uitgeschonken en}{men klonk en dronk weer}\\

\haiku{Bij pozen keek hij,.}{langs weerskanten de straat op}{doch niemand kwam af}\\

\haiku{De plechtigheid liep.}{nu snel af en ze gingen}{van daar naar de kerk}\\

\haiku{Dan barstten ze in! -}{handgeklap los en schreeuwden}{luidruchtig bravo}\\

\haiku{ik bezit niets, gaf, '.}{Jan voor antwoordk heb al}{mijn geld verdronken}\\

\haiku{Jan, die de laatste,}{ging was alleen achter en}{toen hij weerkwam botste}\\

\haiku{Een vlammetje blonk,:}{knetterend vlamde wijder}{uit en verblankte}\\

\haiku{De eenigste deugd die.}{hij had was dat hij veel geld}{zou binnenbrengen}\\

\haiku{t'enden adem, tegen.}{hem mocht leunen met heel de}{zwaarte van haar lijf}\\

\haiku{haar jak stond open en;}{liet een vuil-roode baai zien}{en een vuil wit hemd}\\

\haiku{Zij, ze was nog wel.}{verdragelijk omdat ze}{een vrouw was en jong}\\

\haiku{wee als Romme hem,!}{niet voldeed hij zou haar wel}{weten te straffen}\\

\haiku{Daarboven-op droeg.}{ze een even rosse halsdoek}{met bruine froeien}\\

\haiku{Ze legden zich weer,.}{te bed lijk gewoonte met}{hun rug naar elkaar}\\

\haiku{Doch alles stond in}{een bekende schikking en}{daaraan erkende}\\

\haiku{Daar zocht hij een bank,.}{op doch de eerste welke}{hij vond was bezet}\\

\haiku{Ze woonde ginder;}{ieverst in een klein straatje}{aan d'andere poort}\\

\haiku{Ze klapten nog wat '.}{over-en-weer maart wijf was}{niet te bepraten}\\

\haiku{Dan ging hij ineens,.}{op de deur toe en stiet er}{tegen ze vloog open}\\

\haiku{Hij schudde zijn hoofd,.}{loerde nogeens naar de deur}{en vond ze goed toe}\\

\haiku{Dat miek hem razend.}{op zijn eigen omdat hij}{haar genomen had}\\

\haiku{- 't is altijd 't,, ':}{zelfde zei hij de naaste}{maand zalt weer zijn}\\

\haiku{Langzaam rechtte de.}{razernij zijn gekrompen}{gestalte weer op}\\

\haiku{Hij tastte in zijn,.}{zak vond nog wat enkele}{kluiten en gaf ze}\\

\haiku{Toen hij nader kwam.}{bleef hij weer staan en overzag}{dat alles nog eens}\\

\haiku{Hij doopte er zijn,,.}{vingeren in sloeg zich een}{kruis en ging slapen}\\

\haiku{Hij boog en Romme,.}{bedankte hem vurig toen}{trok hij de deur uit}\\

\haiku{'t Begon hem reeds.}{te vervelen en hij stond}{er verrammeld mee}\\

\haiku{Jan ging open doen en.}{vond Stant v\'oor de deur staan in}{zijn piottenpak}\\

\haiku{Een oogenblik liep,:}{ze als zinneloos rond luid}{huilend en snikkend}\\

\haiku{die een eindelijk;}{begrip was van hun eigen}{en van elkander}\\

\haiku{- 't Is wreed, gij gaat!}{naar uw ouders en ik mag}{de mijne niet zien}\\

\haiku{Zijn wijf liep rond, ze.}{had wafels gebakken en}{sjokola gemaakt}\\

\haiku{Dan ging dat alles.}{weer op in een razernij}{tegen die wijven}\\

\haiku{- Dat gaat mij niet aan,, '!}{zei de vent kalmt is}{maar een bedenking}\\

\haiku{Jan bekeek hem, zijn.}{verbeest gezicht en zijn blik}{zonder bewustzijn}\\

\haiku{- Voor mijn kind zal ik ',.}{t wel doen dat kan er toch}{niets aan verhelpen}\\

\haiku{Hij bewoog zijn hoofd '.}{op-en-neerwaarts en bleef}{int zand kijken}\\

\haiku{Bij dat bedrijf ging '}{zijn gemoed aant hollen}{en hij ademde zwaar}\\

\haiku{Ontzet bezag hij. '}{dat en voelde een koude}{om zijn herte slaan}\\

\haiku{Eindelijk wierp hij.}{het weer weg en ging heen door}{droefenis overmand}\\

\haiku{Doch ineens begon.}{ze luid te gillen en hij}{liet haar ontzet los}\\

\haiku{De smoor rekte zich,,:}{daarover uit werd steeds dichter}{overwelfde de plaats}\\

\haiku{Een heele tijd liep.}{hij en bedaarde maar als}{hij geen adem meer vond}\\

\haiku{Dan bleef hij staan voor.}{een plotse ingeving en}{staakte zijn geschrei}\\

\subsection{Uit: Nazomer}

\haiku{Er was een valsheid:}{in haar toestand die ze zo}{duidelijk voelde}\\

\haiku{Ze geloofde 't.}{zelf eindelik omdat ze}{niemand bemind had}\\

\haiku{Matielde vooral.}{koesterde een innige}{nijd tegen ieder}\\

\haiku{Het uitwerksel was,.}{telkens verschrikkelik haar}{maag was zeer ontsteld}\\

\haiku{Toevallig bleven.}{ze niettemin staan kijken}{tot hij voorbij kwam}\\

\haiku{Daardoor kwam het dat {\textquoteleft}{\textquoteright}.}{hij ook reeds hetmierenest van}{de Deman's kende}\\

\haiku{En hoe meer hij ze,}{van nabij beschouwde hoe}{meer gelijkenis}\\

\haiku{Het was zonde dat,,!}{zo een kerel die zoveel}{geld won jongman bleef}\\

\haiku{Afschuwelike,.}{dingen schichtten door haar brein}{ze voldeed eraan}\\

\haiku{Neen, zo kon ze hem,.}{zich niet verbeelden ze had}{geen zinnen daarnaar}\\

\haiku{Hij was half bedwelmd.}{en keek haar glimlachend aan}{met gulzig genot}\\

\haiku{daar is niets dommer,!}{een lintje en een strekje}{kan hen verleiden}\\

\haiku{ze begreep het niet,, '.}{ze had niets gedaant was}{van zelf gekomen}\\

\haiku{ze wankelde naar.}{binnen doch was niet in staat}{een woord te spreken}\\

\haiku{Haar oren begonnen.}{te ruisen en de dingen}{benevelden zich}\\

\haiku{Wel mogelik, ik.}{neem geen stand in een geschil}{waarvan ik niets ken}\\

\haiku{- Om het even het moet! -,.}{uitgeroeid worden Neen maar}{genezen worden}\\

\haiku{- Maar het deed hem erg.}{onaangenaam aan en hij}{lachte bedwongen}\\

\haiku{Maar misschien drink je,?}{nogal veel azijn of liever}{gebruik je er veel}\\

\haiku{- Hier kan niemand ons,.}{zien of overvallen zei hij}{met gesmoorde stem}\\

\haiku{Ze vloog opniew aan.}{zijn hals en bezwoer hem toch}{met haar te trouwen}\\

\haiku{De eerste stonden.}{verbluft en de drie kwamen}{bij Van Riebeeck staan}\\

\haiku{- Ga stilletjes naar,,,}{huis jongen met je dochter}{en zwijg dat je zweet}\\

\haiku{Matielde wilde.}{haar zuster weg hebben doch}{Stiena wilde niet}\\

\haiku{- Je overdrijft, zei haar,?}{Mevrouw Bollekens waarom}{zou dat jou schuld zijn}\\

\haiku{In hem had zich de.}{warmte gelegd van Stiena's}{tegenwoordigheid}\\

\haiku{Het was een zachte.}{bekoring die nooit meer uit}{hem zou verdwijnen}\\

\haiku{Maar hij kon haar niet.}{zo treurig zien omdat zulks}{zijn genot benam}\\

\haiku{- Je weet dat moei aan?}{Van Riebeeck gevraagd heeft om}{met hem te trouwen}\\

\haiku{Hij werd een wijle.}{woedend en zou de oudste}{wel verwenst hebben}\\

\haiku{er zijn er soms die!}{naar je verlangen zonder}{dat je het zelf weet}\\

\haiku{Van Riebeeck had even,.}{naar haar opgezien hun ogen}{ontmoetten elkaar}\\

\haiku{'t is 'n schande '.}{datn mens alzo onder}{bewaking moet staan}\\

\haiku{We zijn gelukkig.}{omdat we gestraft worden}{voor onze zonden}\\

\haiku{In dit te weten.}{had ik een zalig en toch}{onschuldig genot}\\

\haiku{Ja, dan dachten we.}{toch ook wat we al zouden}{doen met die boel geld}\\

\haiku{Nu zagen we dat,.}{ook op h\'un kamer rook was}{hoewel veel minder}\\

\haiku{Eindelik, 't was '...}{blijkbaar dat zet niet meer}{konden herden en}\\

\haiku{we hadden immers.}{al zoveel ontroeringen}{en schokken doorstaan}\\

\haiku{'t Was 'n vreemde,.}{vandaar haar medelijden}{dat rechtzinnig was}\\

\haiku{hoe meer karakter,.}{en hoe minder gevoel hoe}{minder geweten}\\

\haiku{Zo had ik reeds vrees '.}{datn grotere dief me}{de baard zou afdoen}\\

\haiku{Bij die mensen die.}{ons opgenomen hadden}{zou ik eten krijgen}\\

\haiku{Hij was een schavuit,,.}{lijk ik Piekvogel doch de}{waarheid voor alles}\\

\haiku{Was ik niet een mens?}{gelijk ieder ander mens}{en dus een lafaard}\\

\haiku{t Is zo'n zeldzaam!}{genot op de kosten van}{een ander te eten}\\

\haiku{En zouden we nu,,?}{nog na al die dingen de}{mensen bedriegen}\\

\haiku{De dagen gingen,,,.}{voorbij dat is geloof ik}{toch de gewoonte}\\

\haiku{Hij gelastte zich.}{zulks te doen mitsgaders een}{dertigtal franken}\\

\haiku{Hij is immers wat.}{achterbaks en zal dit met}{mij ondervinden}\\

\haiku{al die tieraden,}{op God die hierboven staan}{beangstigen me.}\\

\haiku{ik verleng steeds het.}{stuk enkel en alleen om}{nog meer te trekken}\\

\subsection{Uit: Het rollende leven. Deel 1}

\haiku{Haar raadgevingen:}{waren geen bevel meer doch}{bijna een smeken}\\

\haiku{Hij werd beschaamd over.}{zijn onbeholpenheid en}{nodeloze schrik}\\

\haiku{Langzaam echter werd,.}{het zo klein dat men nog slechts}{de hellingen zag}\\

\haiku{- Als de loeders het '!}{nu nog niet weten kunnen}{zet gaan rieken}\\

\haiku{Arie's hand was niet vast,.}{genoeg hij beefde wat en}{de duw was te flauw}\\

\haiku{Zijn houding stelde,.}{Arie wat gerust zodat hij}{zijn werk hervatte}\\

\haiku{ge zoudt wel mogen!}{trakteren om met ons mee}{te mogen rijden}\\

\haiku{De eerste maal dat '.}{get weer verzuimt zult ge}{weten aan wat prijs}\\

\haiku{Block sloeg geweldig:}{de andere dicht en gaf}{een krakende vloek}\\

\haiku{wat het was - een macht.}{van blinkende knoppen langs}{de eenzame weg}\\

\haiku{Het deed hem deugd, maar:}{seffens dreef die goedheid weg}{bij de gedachte}\\

\haiku{wat wilde, hij had.}{toch gedaan wat hij kon en}{zijn plichten vervuld}\\

\haiku{Deze schold hem uit.}{en had een kwade grijns van}{voldane vraakzucht}\\

\haiku{s Avends vernam.}{hij weer nieuws over de zaak uit}{de mond van Schoonheydt}\\

\haiku{Toen 't gedaan was!}{zag hij ontsteld dat hij te}{ver ingevuld had}\\

\haiku{Daartoe vonden ze.}{gauw genoeg een reden in}{alles wat hij deed}\\

\haiku{stil de portel met.}{een ger opengetrokken en}{binnengeslopen}\\

\haiku{de trein stille stond,}{liep een eindje verder om}{de schijn te geven}\\

\haiku{Hij dronk stevig door,,.}{samen met de andere}{al zijn geld moest op}\\

\haiku{Dit plotse voorstel.}{doorschokte Arie met vreemde}{wellustgevoelens}\\

\haiku{Hij had zich als een.}{redeloze ezel laten}{meelimperen}\\

\haiku{Ze verstaan niets van, '.}{u en willen u buiten}{daarmee ist uit}\\

\haiku{- Ge zijt nog een duts,,,}{zei hij Arie en veel te veel}{te goeder trouwe}\\

\haiku{Maar hij trok algauw,.}{zijn plan was blij dat het zo}{afgelopen was}\\

\haiku{Schoonheydt zoog gulzig,,.}{aan dat genot waar hij Arie}{vergeefs naar smachtte}\\

\haiku{O ja, het trok hem,,!}{wel aan dat genot hij had}{het bijna gesmaakt}\\

\haiku{het rijtuig waar de.}{dame was ingestegen}{doch ging er niet bij}\\

\haiku{De dagen zouden.}{hem alzo meedragen en}{hij herwerd zichzelf}\\

\haiku{Met de gewone,.}{trein reed hij af er mocht van}{komen wat wilde}\\

\haiku{vragen, hij gaf haar!}{de toelating om mee te}{reizen op voorhand}\\

\haiku{- naar mij moet ge niet,,.}{zien ge kunt doen wat ge wilt}{ik ga niet meer mee}\\

\haiku{Maar neen, dat ook was,;}{zo ontzettend leeg hij had}{er iets bijgevoegd}\\

\haiku{Er schortte hem iets,,.}{hij zocht het overal maar hij}{wist niet wat het was}\\

\haiku{'t Was een zondag.}{en daar stonden lieden om}{een uchtenddruppel}\\

\haiku{Deze gedachte.}{kwam  in hem op met een}{schok van ontzetting}\\

\haiku{Ja, hij moest het uit,.}{zijn hoofd steken die liefde}{was zonder toekomst}\\

\haiku{Waarom heeft hij u?}{niet gezegd dat gij zoiets}{niet mocht antwoorden}\\

\haiku{Maar de gevolgen...}{zijn een jaar vertraging in}{de bevordering}\\

\haiku{Hij was alzo naar;}{de burelen gegaan van}{de stasiemeester}\\

\haiku{Arie kon er niets aan,!}{doen maar het deed hem deugd hij}{was nu gevroken}\\

\haiku{Daar straks had hij in,.}{een roes geleefd nu volgde}{de ontnuchtering}\\

\haiku{hij had haar kunnen,,.}{helpen broederlik doch niet}{haar bezoedelen}\\

\haiku{De heiligschennis,?}{was gepleegd waarom zou hij}{ze niet herhalen}\\

\haiku{er moest ergens een.}{vermaledijding op zijn}{hoofd terecht komen}\\

\haiku{die kerel met die...}{slappe hoed die de pakken}{had afgegeven}\\

\haiku{om de voldoening!}{te smaken te weten dat}{men naar hem smachtte}\\

\haiku{Waarom zou dit voor?}{hem alleen noodlottige}{gevolgen hebben}\\

\haiku{Waarom sloeg men hem '?}{niet eens int gezicht en}{wierp men hem buiten}\\

\haiku{Al het lage van:}{die rol kwam op hem af als}{een donkere wolk}\\

\haiku{Hij schreef haar dat zijn.}{dienst plots veranderd was en}{hij niet kon komen}\\

\haiku{Stilaan begon hij,,}{evenwel te verademen}{er was niets gebeurd}\\

\haiku{De brief was reeds drie,,.}{weken oud sedert niets meer}{een plotse stilte}\\

\haiku{Zijn krachtige stem '.}{overheerstet geratel van}{het rollende tuig}\\

\haiku{Een snoodheid te meer:}{zou hij begaan hebben en}{toch twijfelde hij}\\

\haiku{- Ziet ge, grinnikte, {\textquoteleft}{\textquoteright}!}{hij tegen Arie altijd die}{hoopquand-m\^eme}\\

\haiku{Block ging voort daarop.}{en Arie droomde de hele}{nacht over Block's vrijheid}\\

\haiku{De vroeging bleef om.}{wat gebeurde iedere}{dag daar ver van hem}\\

\haiku{Met een kort gebaar,.}{eiste hij zijn Calepin}{die in orde was}\\

\haiku{hij volbracht slechts een,.}{nieuwe moord bracht een derde}{wezen ten onder}\\

\haiku{De mensen hadden.}{gelijk dat ze ermee geen}{rekening hielden}\\

\haiku{Eindelik, daar had,,!}{hij de vrije straat de reken}{huizen de vrije lucht}\\

\haiku{Ze keerden weerom,,.}{en stapten voort stapje voor}{stapje wandelend}\\

\haiku{De leugen die hij:}{had uitgevonden had een}{dubbel doel gehad}\\

\subsection{Uit: Het rollende leven. Deel 2}

\haiku{Nu moest hij niemand,.}{meer schoon spreken haalde zelf}{hout en kolen bij}\\

\haiku{waarheen hij maar keek '}{joeg hem de angst op het lijf}{en beter ware}\\

\haiku{ze liepen naar vo\'or,.}{het was de masjieniest die}{water moest nemen}\\

\haiku{'t Was wonderlik.}{om zien hoe de mannen daar}{stonden te blinken}\\

\haiku{dat, zo ik u voor, '}{uw nieuwjaar uw opslag niet}{kan ter hand stellen}\\

\haiku{Straks ging de trein aan '.}{t wiegen en hij strekte}{zich uit op de bank}\\

\haiku{Waar had hij gemeend?}{die steunstok te vinden}{die hij nodig had}\\

\haiku{Steeds loerde hij in.}{de verte of hij Baeyens}{niet naderen zag}\\

\haiku{Op de mand met klein.}{materiaal lagen de}{sleutels van de kast}\\

\haiku{dat hij meekreeg smeet.}{hij weg om niet te laten}{zien dat hij niet at}\\

\haiku{Block werd dik en hij.}{blies lijk een genter van dat}{trappen opklimmen}\\

\haiku{Hij moest zich op een}{stoel laten vallen om tot}{zichzelf te komen}\\

\haiku{'t Is erg, heel erg,.}{maar ge moet u alzo niet}{laten teneerslaan}\\

\haiku{En Arie vertelde.}{hem nog wat hij al wist van}{zijn loense gezel}\\

\haiku{Ze spraken wat over,.}{die ziekte over het geld dat}{dit alles kostte}\\

\haiku{Hij heeft er zeven, '!}{honderd opgemaakt op drie}{maandent kan gaan}\\

\haiku{Hij liep recht naar 't,.}{huis van Block waar zijn vrouw was}{maar ze wist het reeds}\\

\haiku{Hoewel uitgeput,.}{had Arie toch nog de kracht om}{blijheid te voelen}\\

\haiku{Hij kon zijn kind niet:}{zien zonder dat hij er twee}{beelden achter zag}\\

\haiku{voor hem was niets toch,.}{dan miezerie overal hij}{was maar liever dood}\\

\haiku{Hij was al buiten ',.}{zonder hijt wist in een}{vlaag van verstomming}\\

\haiku{Zie hoe grillig die.}{plant daar groeide in kromten}{en ellebogen}\\

\haiku{In zijn huis was ook ' '.}{iets aant gisten ent}{zou uiteenvallen}\\

\haiku{hij stortte nu weer '.}{een andere reddeloos}{int ongeluk}\\

\haiku{Ja, madam, 't is,.}{dat Arie een goeie jongen is}{dat hij zich l\'a\'at doen}\\

\haiku{Maar had hij anders, ':}{gevaren dan waret}{onverdiend geweest}\\

\haiku{Laermans leefde,:}{ervoor zijn leven bestond}{enkel uit dit \'een}\\

\haiku{Na een twintigtal.}{vruchteloze pogingen}{gaf hij het op}\\

\haiku{zou men zelf er nooit,!}{van genieten daar was nu}{juist niets aan te doen}\\

\haiku{De gladde baan en,,.}{vaag als spoken daarboven}{een bos van masten}\\

\haiku{Weg dat alles, weg,.}{zich steeds dieper in zichzelf}{keren en zwijgen}\\

\haiku{Zijn drijfveren zou,.}{ze nooit kennen ze zou hem}{dus nooit begrijpen}\\

\haiku{Er is maar Spirou,,... '.}{die vent zal u niets doen maar}{t is toch d\'at niet}\\

\haiku{Irma had een en,!}{ander nodig moest uit en}{kende geen woord Frans}\\

\haiku{ze verwilderden,.}{zienderogen zonder dat er}{iets aan te doen was}\\

\haiku{sedert ze getrouwd.}{waren had Irma nog maar}{e\'en nieuw kleed gehad}\\

\haiku{Met de arm in de}{draagband liep hij rond om de}{pijn te verbijten}\\

\haiku{hij kon  en sprak.}{zachter tegen haar dan hij}{in lang gedaan had}\\

\haiku{het scheen hem een tijd,!}{zonder einde het was toch}{reeds meer dan een uur}\\

\haiku{Gelukkig duurde,:}{zijn kwaadheid niet lang hij zag}{gauw klaar in hun spel}\\

\haiku{ze zouden hem nu}{weldra belet hebben naar}{zijn moeder te gaan}\\

\haiku{Niet gemakkelik}{om iets te vinden met die}{jongens en nog had}\\

\haiku{'k Zal er eens over,,.}{spreken als ge wilt aan een}{heel invloedrijk man}\\

\haiku{Hij was nu van een,.}{wonderlike zachtheid steeds}{van een groot geduld}\\

\haiku{En toch las hij in;}{Irma's ogen nog een angst van}{een andere soort}\\

\haiku{Maar zij zag in dat;}{alles slechts de gewone}{gang van de wereld}\\

\haiku{Zware stappen op '.}{de trap die ploften in de}{stilte vant huis}\\

\haiku{Zij waren beter.}{weg en hij alleen met zijn}{gesloten wezen}\\

\haiku{Hij verhaalde hen,:}{zijn hopen en verwachten}{doch schrikte toen plots}\\

\haiku{Zo was hij roerloos.}{gebleven en was alles}{over hem heengestroomd}\\

\haiku{Toch zou hij maar gaan '.}{slapen want morgen wast}{reeds om drie ure dag}\\

\haiku{ik kom binnen acht.}{dagen weer en we zullen}{het zien te schikken}\\

\haiku{Maar toch hoorde hij,}{die troostwoorden gaarne hij}{hoorde ze zelden}\\

\haiku{Drola gaf order te.}{vertrekken en ging mee met}{hen om een borrel}\\

\haiku{Hier ook, wist hij, lag.}{hij gewoonlik met alle}{leerlingen overhoop}\\

\haiku{het was de weervraak.}{van de dingen die erin}{lag en niets anders}\\

\haiku{Of deze bij de,.}{minister geweest was was}{niet uit te maken}\\

\haiku{zei hij en trok de.}{klink over van de rem die met}{heftig geraas sloot}\\

\section{Edward Vermeulen}

\subsection{Uit: Dagboek van een banneling}

\haiku{- Kijk eens hier, riep de,.}{hulpsecretaris die er}{aan het schrijven zat}\\

\haiku{Als het morgende,:}{was ik tot een en zelfde}{besluit gekomen}\\

\haiku{'t Was natuurlijk,.}{Louize die de eerste haar}{woorden gereed had}\\

\haiku{Ze loeg algelijk;}{en d'r kwamen tranen uit}{hare oogen gerold}\\

\haiku{'k Wist wel van waar.}{die tranen kwamen en wat}{er op volgen zou}\\

\haiku{we zagen er het;}{volk geerne en het volk was}{ons even genegen}\\

\haiku{Die week duurde lang,.}{want ik verlangde mijn hert}{uit naar den zondag}\\

\haiku{Augusta, morgen is ' ',}{t uw feest Enk wensch met}{dees gelegenheid}\\

\haiku{Voeg ik het mijne.}{opeengelaagd En in een}{bundel toegeknoopt}\\

\haiku{Vooruit maar, langs den,.}{Brugschen steenweg al het Land}{van Beloften rond}\\

\haiku{- Die Moezelwijn komt,,.}{van mijnheer Herbau's kasteel}{zegt Beck knipoogend}\\

\haiku{De duivelsdans houdt '.}{op ent rijk Van God rijst}{vreedzaam uit het slijk}\\

\haiku{we krijgen onze,.}{passen terug naarmate}{de inschrijvingen}\\

\haiku{Weerom al Duitsche,.}{huichelarij zijpelend}{van schijnheiligheid}\\

\haiku{Doe maar, 't gaat er,.}{allemaal door ze'n wegen}{dat ginder toch niet}\\

\haiku{'k Heb ze lang en,,,.}{veel bekeken die schoone}{edele sterke vrouw}\\

\haiku{de jongens meest op....}{een opengelegd bedde en}{ik in mijn zetel}\\

\haiku{Een leelijke Duits...}{nadert barsch en jaagt die}{damen brutaal weg}\\

\haiku{En als het hi\'er al,?}{vernield is links en rechts wat}{is het dan verder}\\

\haiku{De trein houdt in en,,.}{een treinwachter geeft ons op}{aanvraag de landskaart}\\

\haiku{slaap en mij noodigen,,!}{zoo zoet dringend en ik zou}{U vinden vinden}\\

\haiku{Ze legden liever,.}{hun zaken stil dan met den}{vijand te heulen}\\

\haiku{In den namiddag.}{wandelde ik eenzaam door}{de dennebosschen}\\

\haiku{Doch, duivelszak is.}{nooit vol en Nooitgenoeg is}{op den wereldbol}\\

\haiku{In elke landsche}{parochie kan men immers}{zonder te missen}\\

\haiku{Mijnheer pastoor van,,}{Bell ik zie u nog staan zooals}{toen met uw blauwen}\\

\haiku{- Die worden warm met,,,.}{te spelen kindje loech Dr}{Meeus blijf en speel maar}\\

\haiku{- Nee, nee, mijnheer de,,.}{Bestuurder wij verkochten}{er geen kaas geen kaas}\\

\haiku{We daalden af en,.}{stapten in de keuken waar}{de verpleegster was}\\

\haiku{- Maar ik zou toch zoo!}{harstochtelijk graag Onzen}{Lieven Heer vinden}\\

\haiku{- Blesse zei ze, de}{boerinne zou wel willen}{dat ik melk gaf lijk}\\

\haiku{'t Was ginder al '.}{voor de vette koe ent}{vet zwijn en hier ook}\\

\haiku{Nu op einde 1914:}{ligt dat rood ornement nog}{in het huis Mortier}\\

\haiku{Onze kanonnen.}{bulderden buitengewoon}{geweldig dien dag}\\

\haiku{Nog altijd lagen,.}{de vijanden voor de stad}{dicht langs den IJser}\\

\haiku{maar er werden meest.}{seminaristen van ons}{Vlaanderen gewijd}\\

\haiku{Il faut esp\'erer:}{que ce d\'epart ne sera}{pas n\'ecessaire}\\

\haiku{Hier in St-Joseph,;}{krijgen wij gewoonlijk noch}{melk noch boter meer}\\

\haiku{Hij stelde voor een.}{mijner onderpastors naar}{Vrankrijk te zenden}\\

\haiku{alle schikkingen:}{waren genomen voor de}{weerkomst naar Belgi\"e}\\

\haiku{'t Is algelijk;}{te peizen dat het al zoo}{zeere niet zal gaan}\\

\haiku{Wat er daar al van,.}{geworden is weten wij}{tot nu toe nog niet}\\

\haiku{Rusland, Vrankrijk en;}{Engeland veranderden}{van ministerie}\\

\haiku{In 't algemeen.}{zijn de soldaten moedig}{en vol betrouwen}\\

\haiku{Maar stillekens aan:}{komt men er van langs om meer}{tegen die zeggen}\\

\haiku{Sommige, in 't,;}{zuiden van Vrankrijk werken}{in de wijngaarden}\\

\haiku{Gelukkig dat men.}{nog boonen en rijst vindt om}{ze te vervangen}\\

\haiku{wij gerochten op}{het einde der maand en de}{zaken bleven lijk}\\

\haiku{zoo niet zal hij eene.}{heelkundige operatie}{moeten ondergaan}\\

\haiku{Het mag hun kosten,,;}{wat het wil zeggen zij ze}{moeten het hebben}\\

\haiku{Veurne heeft nog al.}{schade geleden en telt}{rond de twintig dooden}\\

\haiku{hij is eindelijk.}{uit zijn vel gesprongen en}{heeft vrake gevraagd}\\

\haiku{Veel armen worden.}{rijk en veel welstellende}{menschen worden arm}\\

\haiku{ook nog wat sous en.}{niet zelden zilvergeld zooals}{franks en halve franks}\\

\haiku{Geen twijfel of de '.}{Duitsch was een nieuw offensief}{aant bereiden}\\

\haiku{De Italianen.}{schijnen dezen keer wel te}{zullen wederstaan}\\

\haiku{De ceremonie;}{begint om 9 ure en duurt}{hier tot rond 11 ure}\\

\haiku{Ondertusschen wordt.}{het hier en elders van langs}{om meer dure tijd}\\

\haiku{De verbondenen.}{integendeel jubelden}{van vreugd en van hoop}\\

\haiku{{\textquoteright} De oorlog heeft veel;}{langer geduurd dan gelijk}{wie het voorzien had}\\

\haiku{maar in den grond is.}{de stad leelijk gekwetst en}{ellendig gesteld}\\

\haiku{hoogte die niemand.}{v\'o\'or den oorlog had kunnen}{voorzien noch droomen}\\

\haiku{een dag van vreugde,.}{van vriendschapsbetooging en}{van gelukwenschen}\\

\haiku{De zusters die met.}{de ouderlingen in Vend\'ee}{waren zijn er nog}\\

\subsection{Uit: De reis door het leven}

\haiku{Mijn leven duurde:}{lang en nochtans jammer ik}{niet met den profeet}\\

\haiku{God weet ons op tijd,.}{en stond te doen voelen dat}{wij voor hier niet zijn}\\

\haiku{God en de menschen.}{niet volmaakter bemind en}{gediend te hebben}\\

\haiku{Dan luidden de groote:}{klokken en we voelden er}{de zindering van}\\

\haiku{we waren immers;}{echte natuurkinders en}{de natuur is wild}\\

\haiku{De plaatsejongens:}{staken de koppen bijeen}{en sloten akkoord}\\

\haiku{De helft er van gaf,;}{ik aan moeder zonder den}{oorsprong te melden}\\

\haiku{dat zijn gespogen;}{menschen en de groote -n}{haan is Pickavet}\\

\haiku{Talia, vermaande,,.}{ze stil neem het gerust me\^e}{maar doe dat nooit meer}\\

\haiku{ons merrieke wil.}{dat leelijk spektakel van}{een vent niet kennen}\\

\haiku{Nu zit ik soms op.}{die tijden te peizen en}{den kop te schudden}\\

\haiku{Eens nochtans zag ik '.}{hem stopgezet ent was}{een vies geval ook}\\

\haiku{dan hadden wij toch...}{de leute toe en leute}{misten wij niet}\\

\haiku{Nu nog bederf ik:}{buiten eenieders wete}{een boeremeisje}\\

\haiku{dat zal wel in uw,,.}{gedacht schieten smeekte hij}{mij tegenhoudend}\\

\haiku{hij was een blok van,,;}{een jongen breed geschouderd}{geplant op zijn beenen}\\

\haiku{een wijden broek en,.}{een kort sarrootje op den kop}{een zijdene klakke}\\

\haiku{de plas opsprong en.}{al tusschen mijn beenen verdween}{door de weidehaag}\\

\haiku{Ik vertrok van daar,.}{lijdend als een verdoemde}{versuft en verstompt}\\

\haiku{Al voor ons hof, langs,.}{den wal hoorde ik een man}{over den weg stormen}\\

\haiku{Na een paar dagen:}{was er volle licht over de}{moordzaak gekomen}\\

\haiku{Nogeens gezeid, 'k.}{heb slechts enkele brokken}{kunnen bewaren}\\

\haiku{k zag hem toch zoo - ';}{geren Enk streelde hem}{op zijn dikken kop}\\

\haiku{Wie zou 't geweld '?}{ent gevaar der jonge}{jaren loochenen}\\

\haiku{We 'n hadden geen,.}{tien stappen gesteld of daar}{viel hij we\^erom stil}\\

\haiku{Ze verschoot, sloeg haar.}{naaiwerk af en keek mij met}{bewelkte oogen aan}\\

\haiku{zuchtte hij eens, we.}{waren altijd zoo bevriend}{met die familie}\\

\haiku{- Dat pakt mij, Gusten,,.}{en ik bedank u besloot}{de burgemeester}\\

\haiku{'t Nam al rond mij,.}{zijn vormen aan Van weif'lend}{trillend licht omdaan}\\

\haiku{hoe lavend, Hoe stil,,!}{hoe schoon hoe droom'rig zinkt Rond ons}{de trissche navend}\\

\haiku{Om 't eindigen,}{een hoftafereeltje dat}{ik honderden keeren}\\

\haiku{het voorgevoel was:}{van het ijselijkste wat}{mij kon overkomen}\\

\haiku{moeder, ge ziet er... - ',.}{mij een beetje ding uitk}{B\'e z-iek zei ze}\\

\haiku{Geen uur nadien was;}{moeder berecht en lag ze}{gerust op haar bed}\\

\haiku{Na jaren studie,,:}{na veel zielen doorgrond te}{hebben besluit ik}\\

\haiku{Mijn taak was, is en.}{blijft de kunste kennen van de}{volksziel te grijpen}\\

\haiku{de jongen zette.}{zijn kip open en moorelde}{zijn vooizeken uit}\\

\haiku{Weinige dagen.}{later was geheel de streek}{met Duitschers overstroomd}\\

\haiku{hij nam een stoel, plaatste.}{dien tegen den mijne en}{zat zoo nevens mij}\\

\haiku{- Ge kunt Oberst melden,,.}{dat het een uitdeeling van}{brandstof geldt zei ik}\\

\haiku{- Mijnheer, in volle,,?}{rechtzinnigheid zeg wat peist}{ge van den oorlog}\\

\haiku{We waren in 1917.}{en we begonnen het aan}{God op te geven}\\

\haiku{d'r kwamen stabel:}{meer soldaten aan en die}{klapten te luide}\\

\haiku{- Mijnheer Vermeulen,,.}{zei de Orts ik moet u tot}{mijn spijt aanhouden}\\

\haiku{Ik volgde hem en,.}{werd geleid in een plaats waar}{een jonge heer was}\\

\haiku{- 'k Had sedert mijn;}{gevangzetting twaaf kilos}{gewicht verloren}\\

\haiku{Ge kunt dan morgen.}{vroeg uit Roeselare naar}{Hooglede reizen}\\

\haiku{Hij stond recht, rood van,...}{opgewondenheid reikte}{de hand en vertrok}\\

\haiku{- Neen, steigerde hij,.}{alle menschen moeten zich}{kunnen gerieven}\\

\haiku{En midden deze;}{smertelijkheid wrocht ik bij}{dage als een peerd}\\

\haiku{'k zou mogelijks.}{ook met den daveraar}{op het lijf zitten}\\

\haiku{- 't Is er eene van,,.}{Doktoor Vandeputte zei}{ze een glas vullend}\\

\subsection{Uit: De vracht}

\haiku{- Om geheel heilig,,.}{mogelijks om vermuft te}{worden gekte hij}\\

\haiku{Lode kwam terug.}{in de keuken en ontstak}{een versche sigaar}\\

\haiku{De menschen noemen,,.}{dat hier anders ja met een}{gruwelijken naam}\\

\haiku{hang haar wat leugens,.}{op want ze zal u lastig}{vallen met vragen}\\

\haiku{Een weinig later.}{kwam hij terug ontbeet en}{was reizensgereed}\\

\haiku{Mijnheer Lode is;}{dan gaan loeren en heeft den}{hond kapot gemaakt}\\

\haiku{Ze vielen toe in;}{De Kollebloem als een hond}{in den hutsepot}\\

\haiku{er waren  er,;}{aan de bolle anderen}{aan het kaartspelen}\\

\haiku{- Naar 't Berenhol,,.}{fluisterde hij zijn makker}{een handstoot gevend}\\

\haiku{Lucas zweeg en scheen.}{de teljooren te tellen}{op den kavebank}\\

\haiku{En wel opletten,.}{dat ge nooit verre van hem}{af zijt als hij swalpt}\\

\haiku{De huisbaas schoot in.}{een kletterenden lach en}{hield zich den buik vast}\\

\haiku{- Die duivel van een,,, ',!}{deugniet loech de vrouw m\^ee ja}{hij wast. Toe Klaas}\\

\haiku{Klunten verhaalde.}{hem wat dien namiddag en}{ook des avonds voorviel}\\

\haiku{- Hem in 't water,,;}{gooien snakte Urbaan hem}{te koelen leggen}\\

\haiku{Nog binst dienzelfden;}{nacht klopten lieden van het}{dorp bij Lucas aan}\\

\haiku{- Of omdat hij geen ',.}{maag meern heeft wierp er een}{boertje brutaal op}\\

\haiku{Ons lichaam is een.}{rijk en de maag is er het}{goevernement van}\\

\haiku{De deur ging open en.}{Klaas De Zwingel toonde zijn}{leelijke tronie}\\

\haiku{En d' echo rolde ',.}{t roepen voort Van deur tot}{deur en poort tot poort}\\

\haiku{Klaas sprak geen woord, maar,.}{trok de schouders op zonder}{naar hem te kijken}\\

\haiku{- Haja, precies, daar,!}{hebben wij djanters nog een}{akkoord waarachtig}\\

\haiku{- Ik weet toch alles,;}{wat u pijnigt doch zeg maar}{op en verzwijg niets}\\

\haiku{met Leona trouw,!}{ik en niemand kan noch zal}{het verhinderen}\\

\haiku{- Wat meer is, 'k ben.}{mo\^e van bespot te zijn om}{mijn overtuigingen}\\

\haiku{- Z'is algelijk ook,.}{nog wat geschonden aan den}{neus fluisterde hij}\\

\haiku{... ketterde Lode,.}{zich onderbrekend met een}{vuistslag op den knie}\\

\haiku{Lode sprak niet, bracht; '}{den avond ingesloten door}{en ging vroeg slapen}\\

\haiku{Klaas reikte zijn glas ' -,!}{en Lode vuldet. Ziet}{ge waterkwezel}\\

\haiku{hij is hooge geleerd,,:}{menheere en de menschen}{zijn zot achter hem}\\

\haiku{en zoo krijgt hij een.}{druppeltje langs hier en een}{druppeltje langs daar}\\

\haiku{Op een avond, daar hij,.}{wel alleen met zijn broer was}{pakte hij hem aan}\\

\haiku{gilde Lode, zoo:}{bleek als de dood en hij sprong}{op en raasde}\\

\haiku{Geheel dien dag en;}{volgende dagen woog druk}{in het huishouden}\\

\haiku{En of er nu nog?}{een helle ware wat kan}{dat ons verdoemen}\\

\haiku{Op het zicht van den.}{jonkheid grinnikte hij en}{reikte hem de band}\\

\haiku{- Haja, precies, 'k.}{peisde dat ge van entwat}{anders doende waart}\\

\haiku{Hoe ellendiger,.}{die werd hoe venijniger}{en gevaarlijker}\\

\haiku{Na de mis zag hij}{het volk getroppeld op het}{kerkeplein en ving}\\

\haiku{Nu stonden beide:}{echtgenooten voor een}{afgewrochte taak}\\

\haiku{- Zorg, dat er bij ons.}{afstappen te Ranck een}{auto gereed sta}\\

\haiku{- t' Akkoord, knikte,?}{Lode maar wie verwittigt}{er papa en Jan}\\

\haiku{Fernandje ging naar.}{oom Urbaans kamer en ging}{bij het bed zitten}\\

\haiku{In uwe plaats zou ik.}{rustig blijven en zooveel}{mogelijk slapen}\\

\haiku{Hij wendde zich om,,}{en zijn blikken vielen op}{Urbaan die stijf lag}\\

\haiku{Ondertusschen ging.}{Lode naar De Zwingels en}{spelde hem de les}\\

\haiku{Klaas knikte, altijd,.}{gedwee doch met een groote neep}{tusschen de leepoogen}\\

\haiku{Benauwdheden van.}{dit en van dat. Alsof ik}{nog een kind ware}\\

\haiku{met het groot verlof;}{kwam hun jongen naar huis als}{primus van zijn klas}\\

\haiku{De glim vervloog van '.}{s jongens wezen en hij}{keek staal en doelloos}\\

\haiku{daar streek hij de hand,:}{over zijn voorhoofd bewreef zich}{de oogen en besloot}\\

\haiku{- Trek gij het u aan,,:}{smeekte hij ge kunt geheel}{de schreef uitvagen}\\

\haiku{Enfin... ja toch, die...}{kerk en dat orgespel en}{al die plechtigheid}\\

\haiku{Als Helena naar,.}{huis ging kwam Lode haar toch}{gejaagd te gemoet}\\

\haiku{Ze liet hem, ging naar.}{de ziekenkamer en bleef}{voor het bedde staan}\\

\haiku{hij moest haar hooren.}{en zien en dikwijls binst den}{dag met haar spreken}\\

\haiku{zondag te nonkel '.}{Jans ens anderdaags bij}{notaris Pierszoon}\\

\haiku{Het duurde al niet - '}{lang eer de plagers int}{bijzonder nonkel}\\

\haiku{Wat beteekende dan:}{uwe vermaning van tijdens}{uw doodelijke ziekte}\\

\haiku{Helena bezag.}{hem weemoedig en tranen}{sprongen uit haar oogen}\\

\haiku{Er lag een kwade.}{grijns op Fernands wezen en}{Helena zag het}\\

\haiku{monkelde Fernand,.}{zich bij die eenzijdige}{beschouwing houdend}\\

\haiku{- Zeg maar, stamelde ' - ',.}{t.k Moest u zeggen dat}{ik u geerne zie}\\

\haiku{Wat aangaat de woonst,:.}{daarmee zit ik ook niet in}{we kunnen wachten}\\

\haiku{Weet ge dan niet dat?}{liefde de grootste aller}{babbelkousen is}\\

\haiku{- beklagen u, want....}{Fernand lei zijn meestergast}{stil met een tongslag}\\

\haiku{De zekerheid van.}{die komende zaligheid}{miek hem overmoedig}\\

\haiku{Fernand werd met den.}{dop zoo bleek als een lijk en}{kon geen woord uiten}\\

\haiku{In Gods naam, smeekte,,?}{hij zeg mij hebt ge wel de}{waarheid gesproken}\\

\haiku{De overste dankte.}{hertelijk en stond op om}{afscheid te nemen}\\

\haiku{'k zie het, maar Ons.}{Heer is bermhertig en wij}{moeten het ook zijn}\\

\haiku{Hij bezag lachend.}{zijn neef en de lach verdween}{rasser dan hij kwam}\\

\haiku{Ziet ge wel, vriend, dat.}{iemand er onder bezweek}{en zoo uitboette}\\

\haiku{- Ja en trek het daar,.}{niet lang in dat krottekot}{vermaande Lode}\\

\haiku{hij stak beide zijn.}{armen in de hoogte en}{rekte zijn lijf uit}\\

\haiku{Hij miek mij wijs, dat.}{hij gespeculeerd en groote}{winsten gedaan had}\\

\haiku{alles goed maken,...}{en hem vrede bezorgen}{beaamde Fernand}\\

\haiku{Vanher bedekte.}{hij zijn wezen en wachtte}{geen antwoord meer af}\\

\haiku{'k Wist niet dat een,.}{mensch zoo lijden kon zonder}{er van te sterven}\\

\haiku{Lode weerde die.}{hulde met armgezwaai af}{en bleef gezeten}\\

\haiku{- Hij was ons altijd,.}{een trouwe ziel herhaalde}{hij meermaals dien avond}\\

\haiku{- Ge zult ten minste,,,.}{rusten slaapt ge niet zei ze}{den stoel aanwijzend}\\

\haiku{Hoe fijn en kiesch ook,.}{ze te werk gingen vatte}{hij goed hun inzicht}\\

\section{Hugo Verriest}

\subsection{Uit: Op wandel}

\haiku{Maar hoe droef blijft mijn.}{hert voor die zee van wee die}{ik niet dijken kan}\\

\haiku{En toch is hij een,.}{fatsoenlijk man om zien en}{is het inderdaad}\\

\haiku{- Een gevoelen van,.}{ruimte spreidt rondom ons en}{vrede walmt binnen}\\

\haiku{niet alleen in de,.}{oogen maar tot in den vorm van}{hoofd en mond en kin}\\

\haiku{Het mos, hier en daar,.}{priemt er door en groeit nevens}{en rond de boomen}\\

\haiku{Hij lacht en knikt en.}{in den gang zwakt in lichten}{zwier door zijn leden}\\

\haiku{en nog, nog is die,.......}{zoetheid niet doorstralende}{maar wat versmolten}\\

\haiku{Gij hebt ongelijk -.}{van dat te gevoelen maar}{ik gevoel het}\\

\haiku{Geheel die kerke.}{is een verre nichte van}{Shakespeare}\\

\haiku{Een vlaamsche vrouw houdt,;}{van geen pronken Zij is vol}{moed en lacht zoo schalk}\\

\haiku{Houdt u dus wel van.}{t'huis te ronken En komt bij}{tijds tot in de Valk}\\

\haiku{Daaruit komt en wendt.}{en keert de nieuwe eeuwe}{en de nieuwe tijd}\\

\haiku{- Een lichte smoor, een,,.}{doorschijnende mist hangt in}{de lucht over de zee}\\

\haiku{'t Is vlottend en,!}{waaiend mat maar van peerlen}{glinsterend zilver}\\

\subsection{Uit: Regenboog uit andere kleuren}

\haiku{Ik brenge U wat:}{ik hier en daar gevonden}{hebbe en gezien}\\

\haiku{Over het huizeken,;}{gebogen waaien hunne}{ijdele kruinen}\\

\haiku{Hoe schoon ligt zij daar,,,.}{in haar beddeken bleek wit}{in witte lakens}\\

\haiku{maar daar heeft men hun ';}{int stille geraden}{van te verhuizen}\\

\haiku{De wind moet waaien.}{rond zijnen kop en hoed en}{door zijne kle\^eren}\\

\haiku{en als ik hem ging:}{bezoeken schudde hij zijn}{hoofd traagzaam en zei}\\

\haiku{Vader en moeder.}{zijn te neerstig en moeten}{te zeere werken}\\

\haiku{Het is bewoond, en.}{schijnt altijd sedert eenige}{dagen verlaten}\\

\haiku{Tichels en pannen,}{en bouwstoffen verkoopt hij}{en bedriegt zoo veel}\\

\haiku{Hij is christelijk, {\textquoteleft},{\textquoteright}, -;}{enkwijt zijne plichten wel}{hij gaat naar de kerk}\\

\haiku{Paschen, den eenen,.}{of anderen wekedag}{als er min volk is}\\

\section{Frans Verschoren}

\subsection{Uit: Jeugd}

\haiku{Deezeke schudt zijn....}{beddeken uit en laat de}{pluimekes vliegen}\\

\haiku{hij stond tegen den '.}{mesthoop en was een groot ei}{aant uitblazen}\\

\haiku{Daar kwam iets piepen,, '.}{rozig-geel stukske lijf}{vlak voort gaatje}\\

\haiku{{\textquoteright} De Pad liep zijn beurs,;}{kassers halen onder in}{zijn bed verborgen}\\

\haiku{Ze gooiden ze weg,,.}{in een gracht en beenden toen}{rap voort de hei-in}\\

\haiku{{\textquoteright} De Sooi neep zijn vuist.}{dichter toe om het dierke}{te doen stilblijven}\\

\haiku{ze brandgloeiden op.}{zijn kaken en pitsten de}{tranen uit zijn oogen}\\

\haiku{gelukkiglijk dat.}{de boschwachter in-tijds}{nog op-zij sprong}\\

\haiku{Twee keeren had kromme, '.}{Pol het hem voorgedaan in}{t Gasthuiskerkske}\\

\haiku{de snikken rukten,.}{scheurend uit zijn keel koortsig}{schokkend heel zijn lijf}\\

\haiku{Sooike verstond heur;}{niet en liet zich gewillig}{naar boven leiden}\\

\haiku{Met een luiden gil;}{schokte hij wakker uit zijn}{akeligen  droom}\\

\haiku{Rondom, de ouders,,.}{loerend naar hun kinderen}{aangedaan en fier}\\

\haiku{En Verboven had,,.}{straf de stommerik voor dat}{smijten met zijn klak}\\

\haiku{Maar ze twijfelden, ',!}{al gauw weer wantt was zoo'n}{gelukszak die Sooi}\\

\haiku{{\textquoteright} {\textquoteleft}'k Heb niks gedaan,,{\textquoteright};}{Meester stamelbrabbelde}{de bange jongen}\\

\haiku{Ge zijt zeker wel,,?}{tevreden over Sooi niet waar}{Mijnheer Van Truijen}\\

\haiku{ze spraken af, met,,;}{hun oogen en ze bleven stil}{den heelen morgen}\\

\haiku{lustig verkondend.}{dat het uit was met leeren en}{stil zijn voor vandaag}\\

\haiku{Recht naar de vitrien,.}{van Fien Savelkoel die een}{snoepwinkeltje hield}\\

\haiku{Maar hij hield zich kloek,.}{als een bestige baas die}{er alles afwist}\\

\haiku{Nu was het spoedig.}{klaar waar de gast de centen}{vandaan had gehaald}\\

\haiku{Niks aan gedacht en,!}{zuiver geenen honger gevoeld}{met dat meuleke}\\

\haiku{en hij drong door 't.}{volk en stond te gapen met}{begeerige oogen}\\

\haiku{dat was voor rijke,;}{menschen alleen die veel geld}{konden verteren}\\

\haiku{Lenig boog zijn lijf.}{en zijn vuist omknelde de}{zware  barre}\\

\haiku{niemand is verplicht,;}{te geven daar g'allemaal}{uw plaats hebt betaald}\\

\haiku{Dat hadden ze nog.}{nooit gezien en konden ze}{zeker niet krijgen}\\

\haiku{Nu was het, op school,.}{een stoeffen en boffen op}{zijn tamme kouwke}\\

\haiku{Ze geloofden er,;}{niets af van al wat hij}{hun wijsmaken wou}\\

\haiku{{\textquoteright} schreeuwde de vogel ',.}{daar boven opt dak en}{tjippelde verder}\\

\haiku{Tuurke streelde zijn,,,:}{brave kouwke en nu klonk}{het fier gewichtig}\\

\haiku{Dat hadden ze nog,,.}{nooit gezien zoo'n tamme kraai}{zoo slim en zoo tam}\\

\haiku{Al heel vroeg had ze,;}{hem naar school gezonden met}{hoop op beterschap}\\

\haiku{Ze beriepen zich,;}{op de omstaande kijkers}{die moesten getuigen}\\

\haiku{'k Ben van den nacht, ';}{naar d'hel geweest de duvels}{waren aant eten}\\

\haiku{En nu zouden ze,,!}{eens zien peinsde Sooike wie}{baas was boven-al}\\

\haiku{Daar bleven ze staan,,!}{keken rond en ze zagen}{geen levende ziel}\\

\subsection{Uit: Van een jongen die geluk had}

\haiku{En Verboven had,,.}{straf de stommerik voor dat}{smijten met zijn klak}\\

\haiku{Maar ze twijfelden, ',!}{al gauw weer wantt was zoo'n}{gelukszak die Sooi}\\

\haiku{{\textquoteright} {\textquoteleft}'k Heb niks gedaan,,{\textquoteright};}{Meester stamelbrabbelde}{de bange jongen}\\

\haiku{Ge zijt zeker wel,,?}{tevreden over Sooi niet waar}{Mijnheer Van Truijen}\\

\haiku{ze spraken af, met,,;}{hun oogen en ze bleven stil}{den heelen morgen}\\

\section{Hans Vervoort}

\subsection{Uit: Eerlijk is vals}

\haiku{Ik kwam vrij snel in.}{mijn e-mail terecht en}{er was een bericht}\\

\haiku{De baan vond ze leuk.}{en hij was ook een goede}{baas als hij niet dronk}\\

\haiku{{\textquoteright} {\textquoteleft}Zo is dat. Het is.}{uit met het verwennen van}{zielige oudjes}\\

\haiku{Toen ik hem na de,.}{oorlog eindelijk bewust}{zag was ik zes jaar}\\

\haiku{Ik zag die dag mijn,.}{moeder weer in haar oude}{doen vief en vol praats}\\

\haiku{De vader van mijn.}{vader woonde inderdaad}{in het pension}\\

\haiku{Ik werd er na de.}{zoveelste bezoeker zelfs}{wat korzelig van}\\

\haiku{Die Janine had.}{toch maar een goed inzicht in}{de kantoorpsyche}\\

\haiku{Volgens Janine,}{was ze vijfentwintig maar}{wat mij betreft had}\\

\haiku{{\textquoteleft}Laat ze gaan,{\textquoteright} riep Jaap, {\textquoteleft}.}{ons achternaze moeten}{morgen weer vroeg op}\\

\haiku{We sliepen uit, en.}{ik bracht haar een uitgebreid}{ontbijt op bed}\\

\haiku{H\`e lekker, zo'n warm,{\textquoteright}, {\textquoteleft}.}{bankje zei ikdie mensen}{hebben goed zitvlees}\\

\haiku{Die springen wat rond,.}{in het weiland floepen hun}{tong om een muskiet}\\

\haiku{{\textquoteright} Marijke kraaide,.}{even goddank kon ik haar nog}{steeds laten lachen}\\

\haiku{En d\'at schijnt wel kans,.}{te hebben die knobbeltjes}{kun je wel fokken}\\

\haiku{De heer Faverey.}{van Holland Recherche was}{zo goed als zijn woord}\\

\haiku{{\textquoteright} {\textquoteleft}Ja, natuurlijk, we,.}{hebben samen ontbeten}{om een uur of acht}\\

\haiku{{\textquoteright} vroeg Janine, die.}{een paar seconden na hun}{vertrek binnenkwam}\\

\haiku{Want alhoewel zij,.}{geen steek kon naaien wist zij}{alles van mode}\\

\haiku{Geen enkele prijs,.}{was voor haar heilig er kon}{altijd wel wat af}\\

\haiku{Die Soedarso heeft je,.}{vader vermoord en nu zoekt}{de dochter contact}\\

\haiku{Hoe kom je erbij?}{dat mijn vader door Soedarso}{is neergeschoten}\\

\haiku{Violet at iets,.}{vegetarisch ik had de}{gevulde kalkoen}\\

\haiku{{\textquoteright} {\textquoteleft}Ja, dat stond ook in,...}{het verslag van Faverey}{maak je geen zorgen}\\

\haiku{Als u mij haar adres,.}{geeft dan neem ik zelf verder}{contact met haar op}\\

\haiku{En als wij eruit,,.}{stappen zoals nu missen}{we die controle}\\

\haiku{Dat was nodig voor}{haar genezing en ik wist}{dat wat ze opschreef}\\

\haiku{Ze hield plotseling,?}{op met schrijven had ze mijn}{nabijheid gevoeld}\\

\haiku{Ze kenden het adres,.}{in Amsterdam wel want daar}{zaten ze vroeger}\\

\haiku{{\textquoteright} Zo makkelijk liet.}{hij zich natuurlijk niet in}{de luren leggen}\\

\haiku{Jullie gaan wel ver.}{in het beschermen van je}{nieuwe directeur}\\

\haiku{zich ziek, ze durfde.}{zich niet te vertonen met}{die blauwe plekken}\\

\haiku{{\textquoteright} Hij stapte op en.}{was de deur uit voordat we}{pap konden zeggen}\\

\haiku{Hij was somber en.}{ik vroeg wat hem dwarszat en}{bleef daaraan trekken}\\

\haiku{Een kalverliefde,.}{noemde hij het hij moest nog}{volwassen worden}\\

\haiku{In de drukte en.}{het geroezemoes sprak ik}{ook Anton nog even}\\

\haiku{Ze schrok er zelf even.}{van en zond mij een kleine}{schuldige blik toe}\\

\haiku{{\textquoteleft}Ik heb vierhonderd,?}{dollar bij me geef jij dat}{straks aan die pastoor}\\

\haiku{{\textquoteleft}Ik was nieuwsgierig,.}{ik wilde weten of ik}{nog familie had}\\

\subsection{Uit: Encyclopedie van op het nippertje geredde kennis (en andere stukjes om te lezen)}

\haiku{Nipper 3 In mijn:}{jeugd riepen opgeschoten}{jongens naar elkaar}\\

\haiku{Het was ook niet op,.}{vrouwen gericht het was een}{algemene yell}\\

\haiku{Ik kwam in 1956 op,.}{een accountantskantoor te}{werken 17 jaar oud}\\

\haiku{de Tweede Kamer....}{had weer weinig in de melk}{te brokkelen}\\

\haiku{Tot ze op een dag,:}{vertrokken is met een kort}{vaarwel-briefje}\\

\haiku{nooit had hij op zijn.}{leeftijd deze liefde nog}{mogen verwachten}\\

\haiku{Ik zat met hem in.}{een forum en na afloop}{praatten we nog wat}\\

\haiku{Maar Jaspars werd ziek.}{en wond daar op radio en}{TV geen doekjes om}\\

\haiku{Soms heeft blindheid z'n.}{goed kanten want hij had zijn}{uiterlijk niet mee}\\

\haiku{Op zolder stond een,,:}{heel groot bed Daar sliep een kind}{in opgelet}\\

\haiku{Er kwam een zeehond.}{uit de zee En gleed in bed}{als nummer twee}\\

\haiku{Rare snuiters over,.}{het algemeen met wie je}{weinig contact had}\\

\haiku{wat moest je vroeger!}{timmeren om letters op}{papier te krijgen}\\

\haiku{Wij zetten prompt ook,.}{onze koffers even neer om}{mee te luisteren}\\

\haiku{Dan kan je daarna.}{zonder schuldgevoel doen wat}{je zo graag wilde}\\

\haiku{U weet namelijk!}{al precies dat uw vlucht twee}{uur vertraging heeft}\\

\haiku{Ze kan het toch niet,.}{horen heb ik inmiddels}{wel begrepen}\\

\haiku{En toch, vreemd genoeg,.}{kun je niet toe met \'e\'en van}{deze twee woorden}\\

\haiku{Het zal 2010 geweest}{zijn toen ik met uitgever}{Vic van de Reijt}\\

\haiku{Toch enigszins geschokt.}{bereikte de oude heer}{veilig zijn huis}\\

\haiku{Ze zijn niet aaibaar,.}{en je ziet ze hooguit als}{een schicht wegvliegen}\\

\haiku{En de vis hadden.}{we al geselecteerd bij}{de vermaaksfunctie}\\

\haiku{Maar de hond hadden,.}{we al geselecteerd dus}{dat is geen probleem}\\

\haiku{hun onderbeen is.}{een paar ons lichter dan dat}{van blanke renners}\\

\haiku{{\textquoteleft}Hou dan toch op met,{\textquoteright}.}{dat schoonmaken fluisterde}{mijn vader terug}\\

\haiku{Ik liep op haar af,.}{bukte me en drukte mijn}{lippen op haar wang}\\

\haiku{{\textquoteright} Mijn vader, achter,.}{het stuur ging ongemerkt wat}{harder rijden}\\

\haiku{Zuster van Anken.}{vroeg mij om te helpen met}{de begrafenis}\\

\subsection{Uit: Geluk is voor de dommen}

\haiku{Ik vreesde dat het.}{kwam omdat mijn hand zweette}{als ik nerveus was}\\

\haiku{John was bijna,.}{twee meter lang en werkte}{graag met gewichten}\\

\haiku{Totdat ze een keer.}{een suikerspin bij me kocht}{en nog wat candy}\\

\haiku{Maria kocht \'e\'en keer.}{een suikerspin bij mij en}{ik zag het meteen}\\

\haiku{Dus had ik ook geen,.}{concurrentie Maria was}{helemaal van mij}\\

\haiku{{\textquoteright} {\textquoteleft}Misschien wel, maar in.}{elk geval moet ik dan toch}{in de buurt blijven}\\

\haiku{Geschrokken deinsde.}{ik achteruit terwijl de}{deuren opengingen}\\

\haiku{Daar stond Charlotte,,.}{in compleet tenue ze had}{zelfs haar hoedje op}\\

\haiku{Ik trok haar haastig.}{naar binnen en deed de deur}{achter haar op slot}\\

\haiku{Ze liep op me af.}{en trok de schouders van mijn}{regenjas omhoog}\\

\haiku{Hij ontsloot het hek,:}{en terwijl ze het tuinpad}{betraden riep hij}\\

\haiku{Ik deed de deur dicht.}{en duwde ze de gang door}{naar de woonkamer}\\

\haiku{Steeds weer kwam bij mij.}{de herinnering aan Muf}{Engel naar boven}\\

\haiku{Ik wist wat voor kind,.}{ik in huis had wat hij kon}{en wat hij niet kon}\\

\haiku{Alleen de angst voor.}{een ongeluk kon ik niet}{uit mijn hoofd zetten}\\

\haiku{een moederkloek dan.}{Melanie en ik samen}{ooit geweest waren}\\

\haiku{waarom ik hem dat,.}{niet zou gunnen met Peter}{deed ik het elk jaar}\\

\haiku{Dus heb ik verteld,.}{dat autorijden niet van}{jou mocht en waarom}\\

\haiku{Nooit eerder was tot.}{me doorgedrongen dat dat}{moment zou komen}\\

\haiku{Dit zijn de longen,..{\textquoteright},!}{van de stad jongen Goh de}{longen van de stad}\\

\haiku{Maar Jeannette:}{had me nu het vuur na aan}{de schenen gelegd}\\

\haiku{Neuken, hij zei het,.}{een beetje geaffecteerd}{met gespitste mond}\\

\haiku{Meestal moest er.}{harder gewerkt worden voor}{zo'n geluksgevoel}\\

\haiku{{\textquoteleft}Een Ouwenaardje?}{of een Veluwse Durk of}{de Krielknaber}\\

\haiku{{\textquoteright} Nee, dat viel niet uit,.}{te leggen dat was ze bij}{Koos ook niet gelukt}\\

\haiku{Vier poten en een,.}{rond middenstuk voorzien van}{leren bekleding}\\

\haiku{Dat  houdt het bloed.}{draaiende en wekt de gal}{in de lever op}\\

\haiku{{\textquoteright} En tegelijk kreeg.}{ik een telefoontje van}{mijn oude vriend Carl}\\

\haiku{Want hij was toch de,?}{man achter Manuel in}{die tv-beelden}\\

\haiku{Dit spel reageert.}{op de klank van opperste}{gelukzaligheid}\\

\haiku{Dick deed ons verslag.}{van het heen en weer gedraaf}{tussen de scores}\\

\haiku{Maar ik wil niet dat,.}{mijn zoons die foto's zien als}{ik er niet meer ben}\\

\haiku{hoeveel moeite ons.}{bedrijf deed om discretie}{te verzekeren}\\

\haiku{Maar het maakte ons.}{liefdesleven daarna wel}{realistischer}\\

\haiku{{\textquoteleft}Pijnstiller{\textquoteright} stond er, {\textquoteleft}{\textquoteright}.}{met de hand bij geschreven}{max. 6 per etmaal}\\

\haiku{En na de foto's:}{van Tante Bettina deed}{ik dat altijd braaf}\\

\haiku{in dat huwelijk.}{blijven zitten en jou niets}{te laten weten}\\

\haiku{In het caf\'e kwam.}{dat na een pilsje of wat}{altijd naar boven}\\

\haiku{Gelukkig had ik.}{voldoende pils in  huis}{voor een lange avond}\\

\haiku{Maar die moesten dan wel.}{opvallen en een beetje}{telegeniek zijn}\\

\haiku{Grashoek stuurde mij.}{via de mail een bedankje}{voor mijn recensie}\\

\haiku{Ik ben zo dankbaar.}{voor wat aandacht dat ik niets}{durf te weigeren}\\

\haiku{Ten slotte waagde.}{hij het er maar op en bracht}{de fiets weer op gang}\\

\haiku{Een gevoel dat zo?}{snel wegebt kan toch nooit iets}{betekend hebben}\\

\haiku{Het meisje zoog op.}{haar duim en hield een pop in}{haar armen geklemd}\\

\haiku{Toen hij wegfietste,,.}{keek hij nog even om maar ze}{was niet meer zichtbaar}\\

\haiku{{\textquoteright} Wat een onzin kraam,,,?}{ik uit dacht hij wie is zij}{waarom zeg ik dit}\\

\haiku{Hij transpireerde,:}{en beefde en stond op sloeg}{zijn broek af en zei}\\

\haiku{Hij voelde hoe ze.}{zijn broekriem lostrok en zijn}{geslacht liet zwellen}\\

\haiku{Haar handen woelden.}{door zijn haar en hij kreeg het}{warmer en warmer}\\

\haiku{{\textquoteright} zei ze ineens en.}{hij richtte zijn hoofd op en}{keek haar verbaasd aan}\\

\haiku{Een paar jaar later:}{begon vanuit Amerika}{de revolutie}\\

\haiku{In haar loge stond,.}{Elsa overeind haar hand voor}{de mond geslagen}\\

\haiku{Misschien was ze bang.}{dat zijn ouders haar handschrift}{zouden herkennen}\\

\haiku{Nou ja, twee dagen,.}{Londen z\'o bijzonder was}{dat nu ook weer niet}\\

\haiku{{\textquoteleft}Auf wiederseh'n,,,.}{auf wiederseh'n we'll meet}{again  someday}\\

\haiku{Van de drie flessen.}{wijn was minstens de helft in}{zijn keelgat beland}\\

\haiku{De euforie van.}{de nachtelijke zoen was}{allang verdwenen}\\

\haiku{De zorgen zijn voor,.}{morgen en hij was al met}{al wel een lieverd}\\

\haiku{{\textquoteright} riep hij vanuit de.}{kooi die Christiaan op de}{achterbank zette}\\

\haiku{{\textquoteright} {\textquoteleft}Koffer drie,{\textquoteright} riep ze.}{en ze begon zich in de}{rolstoel te hijsen}\\

\haiku{Ze durfde er niet,.}{om te vragen omdat ze}{het antwoord wel wist}\\

\haiku{Alsof hij het elk.}{moment tot een ru{\"\i}ne}{kon samenknijpen}\\

\haiku{{\textquoteright} {\textquoteleft}Kijk eens om je heen,,.}{de planten hebben er geen}{last van zo te zien}\\

\haiku{Ze was te saai voor,.}{hem te zeer tevreden met}{een rustig bestaan}\\

\haiku{Alleen als ze dood,.}{was zou hij het krijgen en}{zijn gang kunnen gaan}\\

\haiku{Ze moest hier zo snel,.}{mogelijk weg tijd vinden}{om na te denken}\\

\subsection{Uit: Heden mosselen, morgen gij}

\haiku{{\textquoteright} Dat luchtte op, maar,.}{toch vijftienduizend gulden}{was ook een smak geld}\\

\haiku{{\textquoteleft}Kanonnier Dieriks.}{meldt zich met 1 paar schoenen}{voor reparatie}\\

\haiku{Het regende al.}{de hele dag op en af}{en niemand was droog}\\

\haiku{{\textquoteleft}Poef{\textquoteright}, riep ik terwijl,.}{ik de trekker overhaalde}{maar het deed hem niets}\\

\haiku{is het hele pand}{nog een keer uitgebrand met}{veel Marokkanen}\\

\haiku{Ze keek me moeizaam.}{aan en maakte een kleine}{kokhalsbeweging}\\

\haiku{{\textquoteright} Ik kuste haar, ze,.}{bleef in bed zwaaide nog met}{een hand en vertrok}\\

\haiku{Of je nu bruine,.}{bonen of opinies inblikt}{dat maakt toch niets uit}\\

\haiku{Hij zag er fris uit,:}{als altijd de man zonder}{kleine zwakheden}\\

\haiku{Maar zijn ogen waren.}{wat te vochtig en keken}{je te vragend aan}\\

\haiku{{\textquoteright} {\textquoteleft}Tjongejonge{\textquoteright}, zei,.}{Henk en goot voorzichtig wat}{Cassis naar binnen}\\

\haiku{Een grote vrouw en.}{3 kinderen zaten thuis}{op hem te wachten}\\

\haiku{Ik zag het meteen,.}{zitten maar John moest niet}{geforceerd worden}\\

\haiku{twee krissen die ik,.}{nog kende souvenirs uit}{de tijd van weleer}\\

\haiku{Laatst mochten Jan en.}{vriend mee om een nieuw gemaal}{te bezichtigen}\\

\haiku{Met grote angstogen.}{keek Jantje naar de dokter}{die kwam toegesneld}\\

\haiku{{\textquoteleft}I am from Holland{\textquoteright},, {\textquoteleft}.}{zei ikand I am here}{for the first time}\\

\haiku{Wat ik u vragen{\textquoteright},, {\textquoteleft}.}{wilde zei ikis hoe ik}{in Bokor kan komen}\\

\haiku{{\textquoteright} Als op afroep kwam.}{een donkere Boerdosser}{binnen wandelen}\\

\haiku{{\textquoteright} {\textquoteleft}Haha{\textquoteright}, zei ze, {\textquoteleft}dacht?}{u dat meneer Henrix dat}{prettig zou vinden}\\

\haiku{{\textquoteleft}Mandi\"en{\textquoteright}, riep ik,.}{verrast maar de bediende}{kende het woord niet}\\

\haiku{Ze hoopt natuurlijk,.}{dat Oeson niets zal doen}{zolang ze hier zit}\\

\haiku{Het  was nog vrij,.}{warm maar de hitte was nu}{toch wel te dragen}\\

\haiku{Hij was tenger en.}{had koolzwarte ogen in een}{zachtzinnig gezicht}\\

\haiku{Hij zag er sereen.}{uit maar gaf me nog een lel}{met zijn revolver}\\

\haiku{{\textquoteright} {\textquoteleft}Tegen de prijs die.}{wij voor de hele partij}{geboden hebben}\\

\haiku{Mani par ono{\textquoteright}, zei, {\textquoteleft}.}{ik conversationeel}{won tara codex}\\

\haiku{Het duurde een paar.}{minuten voordat ik weer}{een beetje bijkwam}\\

\haiku{Margreet Oeson,.}{keek me steeds vragend aan maar}{ik vermeed haar blik}\\

\haiku{{\textquoteleft}Nog een paar dagen,,{\textquoteright},:}{wachten jongen zei ze en}{tegen tante Aal}\\

\haiku{Aan het begin en.}{het eind van de brug kon je}{van alles kopen}\\

\haiku{Je kwam binnen en,.}{het was een groot huis met warm}{licht uit de lampen}\\

\haiku{Het was stil buiten.}{en gezellig binnen en}{al ver over bedtijd}\\

\haiku{Toen hij dood was, mocht.}{mijn moeder de poort uit om}{hem te begraven}\\

\haiku{Nu niet, Hans{\textquoteright}, zei ze,.}{nijdig toen ik tegen haar}{begon te zeuren}\\

\haiku{Nu rechts aanhouden{\textquoteright},, {\textquoteleft}}{zei hij zelfverzekerdals}{we goed luisteren}\\

\haiku{{\textquoteright} Maar na een half uur,.}{zag ik ze omhoog klimmen}{Els en haar zusje}\\

\haiku{Ik kreeg een keurig.}{briefje terug en om vier}{uur moet ik er zijn}\\

\haiku{{\textquoteright} Ook bij het eten was.}{hij nog zeer korzelig en}{hij ging vroeg naar bed}\\

\haiku{Gehoorzaam ging ik,.}{liggen rillend van de kou}{en viel prompt in slaap}\\

\haiku{{\textquoteright} {\textquoteleft}Gooi dat geweer op,{\textquoteright},, {\textquoteleft}.}{de grond Frans riep mijn oomen}{kom dan dichterbij}\\

\haiku{Vandaar dat Sjoukje,?}{zo ge{\"\i}nteresseerd was}{in je brieven h\'e}\\

\haiku{Vader richtte het.}{geweer op zijn hoofd en kwam}{langzaam dichterbij}\\

\subsection{Uit: Kleine stukjes om te lezen}

\haiku{Hij moest er een tijd.}{naar zoeken maar toen had hij}{het toch gevonden}\\

\haiku{Hij stapte op zijn.}{bromfiets en liet zich langzaam}{naar kantoor rijden}\\

\haiku{Even het gebit uit,.}{en de kaken op en neer}{blaasbalgje spelen}\\

\haiku{Pieter verlegen.}{met zijn pukkels in zijn hoofd}{naast zijn staketsels}\\

\haiku{Nou Pieter,{\textquoteright} zei ik, {\textquoteleft},.}{Gut ik wist niet dat je al}{zoveel gemaakt had}\\

\haiku{{\textquoteleft}Henk, het meisje was,.}{hier geen twee weken of ze}{kreeg er genoeg van}\\

\haiku{{\textquoteright} Mijn vader, achter,.}{het stuur ging ongemerkt wat}{harder rijden}\\

\haiku{Ik hoorde het uit,.}{de tweede hand misschien sterk}{gedramatiseerd}\\

\haiku{Behalve de twee.}{schildwachten was alleen de}{priester op het plein}\\

\haiku{Mevrouw Elsa Cruijff,.}{ze droeg de naam nog steeds een}{beetje onwennig}\\

\haiku{Ze liep resoluut,.}{op de stille figuur af}{knielde erbij neer}\\

\haiku{Ik was er nog niet,}{geweest deze kans om er}{gechaperonneerd}\\

\haiku{ik heb toch liever,.}{dat je weer snel terug gaat}{anders krijg je straf}\\

\haiku{Ik was blij met de.}{grote helm die mijn gezicht}{in het donker liet}\\

\haiku{Het was al drie uur.}{in de morgen voordat hij}{tot een besluit kwam}\\

\haiku{Ze stond hem op te,.}{wachten bovenaan de trap}{blond en iets te dik}\\

\haiku{Hij besloot er toch,.}{een te plaatsen heel terloops}{en vanzelfsprekend}\\

\haiku{Marijke dacht dat...{\textquoteright} {\textquoteleft},,{\textquoteright}.}{jeJa het is goed ik neem}{haar wel mee zei hij}\\

\haiku{Een mooie kamer{\textquoteright} zei.}{ze al terwijl hij nog zocht}{naar het lichtknopje}\\

\haiku{{\textquoteleft}Geachte Juffrouw,}{zoudt u mij het genoegen}{willen bereiden}\\

\haiku{Hij dronk een klein glas.}{sherry en keek met nieuwe}{ogen zijn kamer rond}\\

\haiku{De dood kun je je:}{niet voorstellen omdat het}{te absoluut is}\\

\haiku{s Middags als de.}{school uitging achtervolgden}{zij ons in groepjes}\\

\haiku{Ze zouden een nacht,.}{wegblijven ik had het rijk}{alleen in ons huis}\\

\haiku{Ik sliep al toen de.}{huizen wakker werden en}{hulp kwam toegesneld}\\

\haiku{Achter mij was het:}{onstuitbare geluid van}{Winnie te horen}\\

\haiku{{\textquoteleft}Ja, dat kunnen we,{\textquoteright}.}{vooraf wel doen zei ik vol}{zelfvertrouwen}\\

\haiku{{\textquoteleft}Laat maar, laat maar,{\textquoteright} zei, {\textquoteleft},.}{ikga jij de herdertjes}{maar verrassen man}\\

\haiku{Een bus brengt hem naar,.}{de buitenwijk van de stad}{het is er vrij druk}\\

\haiku{een beetje dom of.}{een beetje labiel of een}{beetje ongezond}\\

\haiku{De cirkel is dan.}{rond en iedereen heeft wat}{hij wilde hebben}\\

\haiku{Zo lijdt A dus zijn.}{tweede verlies en B raakt}{het ding nooit meer kwijt}\\

\subsection{Uit: Met stijgende verbazing}

\haiku{Op de vloer van de.}{Mercedes lagen een paar}{sigarenpeuken}\\

\haiku{{\textquoteleft}Ik heb vooral in,{\textquoteright}, {\textquoteleft}.}{het bos gewandeld zei ik}{dan word je niet bruin}\\

\haiku{En toen kon hij wat.}{geld lenen en bood aan om}{zich in te kopen}\\

\haiku{{\textquoteleft}Dus,{\textquoteright} zei de jongen, {\textquoteleft}.}{met het baardjehet gaat om}{een man of een vrouw}\\

\haiku{Hij denkt eerst nog aan,.}{een luchtspiegeling maar hij}{kruipt er toch naar toe}\\

\haiku{Hij veegde het zweet.}{van zijn voorhoofd en poetste}{zijn brilleglazen}\\

\haiku{{\textquoteright} Haastig pakte ik.}{mijn jas en struikelde de}{donkere trap af}\\

\haiku{Ik stak het stokje.}{zorgvuldig in mijn broekzak}{om te bewaren}\\

\haiku{Ze programmeerde, '.}{al onze tijd ens avonds}{waren we doodmoe}\\

\haiku{Na de receptie.}{en het etentje bracht ik Maartjes}{getuige naar huis}\\

\haiku{Ik wilde haar en,.}{Jimmy per se nog zien maar}{Bert hield me tegen}\\

\haiku{Morgen koop ik een,{\textquoteright}, {\textquoteleft}.}{croquet voor je beloofde}{ikvan Kwekkeboom}\\

\haiku{Een andere vrouw,.}{wilde interrumperen}{maar ze kreeg geen kans}\\

\haiku{En dan heb ik wat,.}{van  die blikjes in huis}{voor het geval d\'at}\\

\haiku{{\textquoteright} zei hij snel tegen, {\textquoteleft}.}{mijmarkt voor alleenstaande}{oudere vrouwen}\\

\haiku{Ik was weer geheel,,.}{nuchter de zak geledigd}{de druk verdwenen}\\

\haiku{Ik keek haar aan, ze.}{had een mooi schrander hoofd en}{een ironische lach}\\

\haiku{{\textquoteright} {\textquoteleft}Welnee, eigenlijk.}{informeerde hij alleen}{hoe het met Bert ging}\\

\haiku{Onaangename.}{herinneringen dreigden}{boven te komen}\\

\haiku{Hij kwam om elf uur,}{op kantoor knikte in het}{voorbijgaan naar me.}\\

\haiku{{\textquoteleft}Mijn god Bert,{\textquoteright} zei ik, {\textquoteleft}.}{toen Jenny even weg waswat}{ben je toch een lul}\\

\haiku{{\textquoteright} Hij betastte zijn,.}{zakken vond een sigaar en}{stak hem in zijn mond}\\

\haiku{{\textquoteright} {\textquoteleft}Niets, je wilde me.}{laten zien dat je alles}{op moet  schrijven}\\

\haiku{En heren, de bar.}{is open als u straks nog een}{glaasje wilt drinken}\\

\haiku{Seizoenstrends en trends.}{per weekdag worden in de}{doelstelling verwerkt}\\

\haiku{Allicht raadpleeg je,.}{dan de kaart en het water}{loopt je in de mond}\\

\haiku{Hij keek me even aan,.}{maar er viel aan zijn gezicht}{niets af te lezen}\\

\haiku{Hij zette de fles,.}{op tafel pakte zijn bril}{op en wreef zijn ogen}\\

\haiku{Ik was er zelf net,.}{uit ik wilde er niet in}{betrokken raken}\\

\haiku{{\textquoteright} vroeg ik, {\textquoteleft}wil je hem,?}{wegsturen net zoals Van}{Lier dat bij jou deed}\\

\haiku{{\textquoteright} Grote opluchting.}{brak los en ik vertrok met}{veel goede wensen}\\

\haiku{Maartje was nog geen twee.}{dagen weg en hij begon}{al te verslonzen}\\

\haiku{Hij slokte haastig,.}{een glas weg boerde en kwam}{langzaam bij kennis}\\

\haiku{{\textquoteleft}Ik moest die auto.}{dus aan de kant zetten en}{mee naar het bureau}\\

\haiku{Maar hij zag het h\`e,.}{dat ik dat nodig had en}{dat ik een heer was}\\

\haiku{Trut, mijn broer in de,.}{steek laten op zo'n moment}{een schande was het}\\

\haiku{{\textquoteleft}Ik heb haar een paar,.}{keer gebeld maar ik kom niet}{voorbij nicht Carla}\\

\haiku{{\textquoteleft}Hier,{\textquoteright} zei Theo en stak.}{hun een kurketrekker toe}{die hij op zak had}\\

\haiku{We wachtten tot Ruud.}{de fles geopend had en}{drie glazen inschonk}\\

\haiku{Bert ledigde een,.}{blik frisdrank in zijn glas zag}{ik met opluchting}\\

\haiku{ze mijn klanten, dat.}{zal ze met de helft lukken}{en de rest vertrekt}\\

\haiku{Ze stond snel op en,.}{nam haar tasje mee Govers kon}{het wel vergeten}\\

\haiku{{\textquoteright} vroeg ik en merkte.}{aan de beweging van haar}{hoofd dat ze knikte}\\

\subsection{Uit: Het tekort}

\haiku{Heren, we hebben,.}{al drie minuten gepraat}{nu eerst een plaatje}\\

\haiku{Na een kwartier was.}{hij gezakt tot 150 en ik}{gestegen tot 100}\\

\haiku{De een zie je nooit,.}{de ander komt nog wel eens}{een praatje maken}\\

\haiku{Die avond vrijden we,.}{als vanouds ongeduldig}{en met veel lawaai}\\

\haiku{We stapten de lift,.}{in waar we elkaar snel wat}{beter opnamen}\\

\haiku{Misschien had hij toch.}{wel gelijk dat er op mij}{te bouwen viel}\\

\haiku{{\textquoteleft}Ik ben je vader,.}{niet het is de eerste keer}{in al die jaren}\\

\haiku{{\textquoteleft}Waarom wil je niet?}{dat er in mijn bijzijn over}{Robin gepraat wordt}\\

\haiku{En toen ik er eind,.}{van de middag naar vroeg had}{Jaap het al verstuurd}\\

\haiku{Dat is verdomme.}{de tweede keer dat Dries een}{kloterapport krijgt}\\

\haiku{Van Maurik klaagt echt,.}{niet gauw en nu al twee keer}{in een paar weken}\\

\haiku{bent is iets van 1,.}{gulden 97 plus afschrijving}{fiets een rijksdaalder}\\

\haiku{Maar ik merkte dat.}{hij ineens verstrakte en}{liet de spriet vallen}\\

\haiku{{\textquoteright} {\textquoteleft}Maar denk eens aan de.}{invloed die dat heeft op het}{opgroeiende kind}\\

\haiku{En desnoods liever.}{een kwart vogel in de hand}{dan twee in de lucht}\\

\haiku{We deden de drie}{verplichte zoenen op de}{wang en ik rook vaag}\\

\haiku{Ik ben op zoek naar.}{mijn echtgenoot die weer eens}{geheel foetsie is}\\

\haiku{Dan zal ik bij jou,?}{moeten logeren heb je}{een logeerkamer}\\

\haiku{Ik vond haar lief en.}{ze had recht op warmte na}{zo'n mislukte avond}\\

\haiku{{\textquoteright} {\textquoteleft}Mwah,{\textquoteright} zei Robin met, {\textquoteleft}.}{getuite lippenmij nooit}{zo opgevallen}\\

\haiku{{\textquoteright} {\textquoteleft}Suit yourself,{\textquoteright} zei ze en.}{begon met smaak te eten van}{haar babi panggang}\\

\haiku{Ze vond het contact.}{dat we nu hadden juist heel}{prettig en rustig}\\

\haiku{{\textquoteright} Ze lachte naar me.}{om te laten zien dat het}{maar half gemeend was}\\

\haiku{{\textquoteright} {\textquoteleft}Meneer De Leeuw, ik.}{zeg altijd maar je bent zo}{oud als je je voelt}\\

\haiku{Triomfantelijk.}{legde hij ze \'e\'en voor \'e\'en}{op het tafeltje}\\

\haiku{{\textquoteright} Nu even doorbijten,?}{had ik het in me om hem}{erin te luizen}\\

\haiku{{\textquoteleft}Verdomme, de pils,.}{is op Marjolein is ook}{niet meer wat ze was}\\

\haiku{{\textquoteright} {\textquoteleft}Ja man, je kan het.}{niet altijd krijgen zoals}{je het hebben wilt}\\

\haiku{{\textquoteleft}Ik zei al tegen,,{\textquoteright}.}{je dat het niks was John}{zei de directeur}\\

\haiku{En als je zoveel,.}{van Robin houdt moet je hem}{nog een kans geven}\\

\haiku{Onwillekeurig.}{keek ik in de spiegel en}{schrok van wat ik zag}\\

\haiku{{\textquoteleft}Marjolein en ik.}{hebben een probleem en daar}{speel jij een rol bij}\\

\haiku{En zodra jij zelf.}{even aan het kortste eind trekt}{is het huis te klein}\\

\haiku{iets was hem in geen.}{jaren verteld en hij had}{het er moeilijk mee}\\

\subsection{Uit: Vanonder de koperen ploert}

\haiku{in Indonesi\"e.}{zijn prijzen een favoriet}{gespreksonderwerp}\\

\haiku{Hij kijkt lang naar het,.}{geld en mijn boze hoofd en}{ik kijk lang terug}\\

\haiku{Even later komt een,}{krantenjongen binnen door}{Engelse ziekte}\\

\haiku{Netjes geef ik haar {\textquoteleft},{\textquoteright}.}{een hand.Nama saja Hans}{Vervoort zeg ik trots}\\

\haiku{Ostentatief laat.}{ik de tas met camera}{op tafel liggen}\\

\haiku{Ziet u, de cheque.}{uit Holland komt maar \'e\'ens}{in de drie maanden}\\

\haiku{Hij knikt en dan lijkt.}{me het moment gekomen}{om te vertrekken}\\

\haiku{Ons geroep helpt niets,.}{ze blijven in een stevig}{tempo doorsjouwen}\\

\haiku{Na een week zingt hij {\textquoteleft}{\textquoteright}.}{geheel zelfstandigHappy}{birthday to joe}\\

\haiku{Vanzelf ga ik ook}{naar de mensen kijken en}{onder een boom zit}\\

\haiku{{\textquoteleft}Bij een boom zo vol,.}{geladen geven \'e\'en twee}{pruimpjes extra niet}\\

\haiku{Een nummer weet ik,.}{niet maar het was een hoekhuis}{volgens mijn ouders}\\

\haiku{Een wit kerkje, veel,,.}{stenen huizen een schooltje}{een stenen passar}\\

\haiku{De trein zal om \'e\'en,.}{uur vertrekken maar liever}{te vroeg dan te laat}\\

\haiku{Uit de stortbak groeit,.}{een schaduwplant dat is dan}{weer typisch Indisch}\\

\haiku{We blijven rustig,.}{wachten en na enige tijd}{komen ze terug}\\

\haiku{Een oude kelner.}{met \'e\'en tand komt ons vragen}{wat we willen eten}\\

\haiku{een straat die vroeger,.}{de Palmenlaan heette als}{ik me niet vergis}\\

\haiku{Ik ga het straatje.}{in waar ik in mei 1953 voor}{het laatst ben geweest}\\

\haiku{Hier kom ik nooit meer,.}{terug dacht ik heel bewust}{toen ik wegfietste}\\

\haiku{Ik zocht Lenta op,.}{samen liepen we naar de}{rivier voor de school}\\

\haiku{Ik knikte, keek de.}{grote zaal rond en nam in}{gedachten afscheid}\\

\haiku{Fietsers, betja's, auto's.}{rijden in brede stromen}{tegen elkaar in}\\

\haiku{Aan het eind van de.}{rit treffen we een poortje}{met een bewaker}\\

\haiku{Hij flikt het glaasje,.}{leeg in zijn keelgat schenkt zich}{meteen opnieuw in}\\

\haiku{Direct na onze:}{aankomst is een paniek van}{gastvrijheid ontstaan}\\

\haiku{De gulheid van de.}{familie zal zich wel niet}{over hem uitstrekken}\\

\haiku{Ze liggen keurig.}{gerangschikt in een richel}{boven de bouillon}\\

\haiku{Onverwacht wordt het.}{de mooiste rit die we op}{Java meemaken}\\

\haiku{waarom filmhelden}{er altijd zo authentiek}{zweterig en vies}\\

\haiku{Nu begrijpt hij het,.}{beter een newspaperman}{van voor de oorlog}\\

\haiku{Naast de chauffeur zit,.}{de eigenaar wiens rol ons}{niet duidelijk wordt}\\

\haiku{Zodra er twee bij.}{elkaar in de buurt komen}{ontstaat het gevecht}\\

\haiku{Even de paspoorten,.}{oppikken en dan met het}{vliegtuig naar Bali}\\

\haiku{Eerst moeten we even,.}{langs het huis van Mama voor}{een afscheidshapje}\\

\haiku{Mama Pungut barst.}{in tranen uit en laat ons}{maar met moeite gaan}\\

\haiku{De prijs is laag, daar.}{zal onze begeleider}{niet rijk van worden}\\

\haiku{Al die tijd heeft hij,.}{niet gelachen alleen zijn}{tanden ontbloot}\\

\haiku{Na de reis zal hij.}{ze terugkrijgen wordt hem}{vriendelijk beloofd}\\

\haiku{Jakarta is de.}{plek waar Maja's voorouders}{begraven liggen}\\

\haiku{Tijdens het gesprek:}{herinnert hij zich iets en}{zegt tegen zijn vrouw}\\

\subsection{Uit: Een zomer apart}

\haiku{Ik plaatste het bord op.}{zijn schoot en de koffie voor}{hem op het muurtje}\\

\haiku{We keken even naar.}{de stralende hemel maar}{er kwam geen teken}\\

\haiku{Als je moeder het.}{maar af en toe mag lenen}{voor haar boekhouding}\\

\haiku{Het hinderde me,?}{dat ik niets kon dat moest toch}{een keer uitkomen}\\

\haiku{Angela was in.}{de keuken bezig met het}{snijden van groente}\\

\haiku{O, dat kostte me,.}{wat want ik was bang hoe hij}{zou reageren}\\

\haiku{Na een tijd ging het,.}{wat beter de afstand naar}{mijn bed leek haalbaar}\\

\haiku{Het duurde een half.}{uur voordat er geluiden}{op de trap klonken}\\

\haiku{In de koele hal,.}{hing een zwaard ik kon nu geen}{uitstel meer velen}\\

\haiku{Ik hou hem maar op,.}{dan kun je er niet nog een}{keer op gaan zitten}\\

\haiku{z'n tol eiste en.}{dat hij er goed aan zou doen}{een vrouw te vinden}\\

\haiku{Ik verander er,,.}{nog aan ik zal haar Hetty}{noemen of zo iets}\\

\haiku{Ik bracht haar terug.}{naar de burcht en liet haar bij}{de ingang alleen}\\

\haiku{{\textquoteright} En met z'n drie\"en.}{wuifden ze me na toen ik}{in de taxi verdween}\\

\haiku{Het was koud en ik.}{rilde en verloor ineens}{mijn zelfbeheersing}\\

\haiku{Een bord verwees naar,.}{Engeland er stond een boot}{met een rookpluim op}\\

\haiku{{\textquoteright} {\textquoteleft}Jawel, ze knikt ja.}{of nee en ze glimlacht en}{ze maakt gebaren}\\

\haiku{Ik stond op en ging.}{buiten in de schaduw op}{een bankje zitten}\\

\haiku{Ze had mijn boeken.}{gelezen en ze veegde}{de vloer met me aan}\\

\haiku{Hij hield zijn glas naar.}{achteren en ik schonk hem}{in en wachtte af}\\

\haiku{{\textquoteleft}Whisky,{\textquoteright} zei hij en.}{zakte neer op een stoel aan}{de keukentafel}\\

\haiku{En ik heb iemand.}{nodig die kan rekenen}{en zaken kan doen}\\

\haiku{Maar dan had zij mij,?}{toch kunnen zien en wenken}{me kunnen volgen}\\

\haiku{Wat had die toerist?}{gezegd of gedaan om hem}{zo kwaad te maken}\\

\haiku{Ik deed mijn ogen dicht.}{en hoopte maar dat men zou}{denken dat ik bad}\\

\haiku{Maar al lezend vroeg.}{ik me af wat er erg was}{aan onoprechtheid}\\

\haiku{{\textquoteright} {\textquoteleft}Is dat je nieuwe,,?}{vrouw maar dat is toch niet een}{nieuwe moeder h\`e}\\

\haiku{{\textquoteright} {\textquoteleft}Wat is er nou aan,?}{de tuin wat moeten we nou}{met haar in de tuin}\\

\haiku{Er mocht wel bij maar,.}{er mocht niets af niemand en}{niets mocht weg of dood}\\

\haiku{Een goeie huishoudster,,}{heb ik in elk geval dacht}{ik en ze ving het}\\

\subsection{Uit: Zonder dollen}

\haiku{Hij aarzelde even {\textquoteleft}?}{en drukte me zijn glas in}{de hand.Blijf je hier}\\

\haiku{Als de nood aan de,.}{man kwam was de kalmte in}{elk geval nabij}\\

\haiku{Het kwam gewoon uit,.}{de lucht vallen mijn vader}{praatte daar nooit over}\\

\haiku{Lucas leidde me.}{naar  een gezelschapje}{van een man of tien}\\

\haiku{We wachtten tot het.}{licht groen werd en staken de}{doodstille weg over}\\

\haiku{{\textquoteleft}Is hier geen normaal,?}{caf\'e waar je rustig een}{pilsje kunt drinken}\\

\haiku{{\textquoteright} vroeg ik en tilde,.}{haar even op waarbij ik haar}{middel indrukte}\\

\haiku{Eefje had haar ogen,.}{dicht kleine blauwe adertjes}{op de oogschelpen}\\

\haiku{{\textquoteright} Zes jaar later, in,.}{Budapest zat het nog exact}{in mijn geheugen}\\

\haiku{{\textquoteright} {\textquoteleft}Nog zo'n geinige.}{opmerking en ik gooi me}{van het balkon af}\\

\haiku{Ik lachte hem uit.}{en tenslotte begon hij}{ook te grinniken}\\

\haiku{De poes was nergens.}{te vinden en ik voelde}{me zo erg alleen}\\

\haiku{Dat geroezemoes,.}{van al die stemmen daar werd}{ik knettergek van}\\

\haiku{Als ik weer OK ben,.}{kom ik naar je toe want Jack}{is maar tijdelijk}\\

\haiku{Het stomme is, toen.}{ik wegging dacht ik dat het}{erg goed met haar ging}\\

\haiku{Of misschien was ze,.}{wel gewoon bang voor de reis}{god mag het weten}\\

\haiku{Mevrouw Overeem vertrok.}{om een uur of drie en toen}{waren ze er nog}\\

\haiku{De hoofdpijn was weg,.}{maar ik voelde me nog steeds}{moe en bezopen}\\

\haiku{E\'en druppel op het.}{tapijt en de huisvrouw kan}{het wel vergeten}\\

\haiku{Ik ging verstrooid op.}{het krukje zitten en keek}{naar haar lange lijf}\\

\haiku{Haar kleine borsten.}{wipten telkens boven de}{waterspiegel uit}\\

\haiku{Celia schokte nog,.}{een paar keer door terwijl ik}{wat nadruppelde}\\

\haiku{De roomservice.}{bracht wat toast en sloot de deur}{discreet achter zich}\\

\haiku{Ik had hem later,.}{op de avond gepland maar hoe}{eerder hoe beter}\\

\haiku{het is gewoon waar,.}{ik ben niet zo'n genieter}{als Lucas of jij}\\

\haiku{{\textquoteright} {\textquoteleft}O, was jij dat{\textquoteright}, zei, {\textquoteleft},.}{ikja dat is het leuke}{van een erfenis}\\

\haiku{{\textquoteright} {\textquoteleft}Jullie waren niet?}{meer thuis toen Eefje uit het}{ziekenhuis langskwam}\\

\haiku{{\textquoteright} {\textquoteleft}Misverstanden{\textquoteright}, zei, {\textquoteleft},.}{ik schamperjongen ik ken}{je van binnenuit}\\

\haiku{Die pikken het niet.}{als zo'n nieuwkomer ineens}{de grote baas is}\\

\haiku{Hij was toch alweer.}{heel ver heen en dat viel me}{een beetje tegen}\\

\haiku{Dat was kennelijk.}{een standaard-gimmick}{hier in Hongarije}\\

\haiku{Maar daar tegenover.}{staat dat ik nogal zwaar op}{hem ben gaan zitten}\\

\haiku{{\textquoteleft}Lekker wandelweer{\textquoteright},, {\textquoteleft}.}{zei ikals het nou maar niet}{gaat  regenen}\\

\haiku{Ik maakte een klein.}{trippelpasje en knalde}{tegen een deur aan}\\

\subsection{Uit: Zonnige perioden}

\haiku{Terwijl ik de rest.}{van de kantoorpost doornam}{ging de telefoon}\\

\haiku{Poe haar bovenlijf.}{tegen mijn been en liet haar}{hongerklaag horen}\\

\haiku{Als ik niet zo stuurs,.}{was en zo weinig tijd had}{zou ik het zelf doen}\\

\haiku{Gelukkig kreeg ik.}{na de hbs een goede baan}{bij de Postgiro}\\

\haiku{En deze week kreeg,.}{ik een kaart uit Florida}{St. Augustine}\\

\haiku{{\textquoteleft}U bent een oude.}{schoolvriend en u weet dus dat}{Peter nooit veel zei}\\

\haiku{Het is zacht gezegd.}{niet leuk om hier te zitten}{en niets te weten}\\

\haiku{{\textquoteleft}Ik begreep uit zijn.}{kaartje dat hij op zoek is naar}{oude bekenden}\\

\haiku{Lilian zat voor,.}{in de klas en ik een paar}{banken achter haar}\\

\haiku{Dat had hij me nooit:}{vergeven en elke keer}{kwam het weer terug}\\

\haiku{{\textquoteleft}Ik zit hier nu met.}{een probleem van jou waar ik}{echt geen trek in heb}\\

\haiku{Daar zat een vrouw te,.}{slapen hoofd op de armen}{op het tafelblad}\\

\haiku{Ja, je zou misschien.}{wat minder tijd aan je werk}{kunnen besteden}\\

\haiku{Hier nooit klevers, nooit,.}{werd je gesneden men gaf}{elkaar de ruimte}\\

\haiku{Ook als ik hier niet.}{met een doel gekomen was}{zou ik er graag zijn}\\

\haiku{I always thought, ik.}{dacht altijd wel dat je iets}{met schrijven zou doen}\\

\haiku{In het hoogseizoen.}{verhuur ik het en woon ik}{zelf in het motel}\\

\haiku{Maar hij moet wel een.}{week in het ziekenhuis met}{een hersenschudding}\\

\haiku{{\textquoteright} {\textquoteleft}Ik garandeer het,.}{voor een week daarna sta ik}{niet voor mezelf in}\\

\haiku{Het stond laatst nog in,.}{Management Team papier}{en inkt zijn killers}\\

\haiku{{\textquoteright} Hij produceerde,.}{van onder de deken een}{hand die ik schudde}\\

\haiku{Een half uur later.}{kreeg ik belet in zijn huis}{aan het Vondelpark}\\

\haiku{Daar lag Flamingo,.}{Resort een oase van rust}{met uitzicht op zee}\\

\haiku{Je stuurde me een.}{ansichtkaart dat je op zoek}{bent naar Lilian}\\

\haiku{Ik denk veel te veel.}{aan jou en aan vroeger en}{hoe het verder moet}\\

\haiku{Ze omarmde me.}{warm en al mijn twijfels en}{angsten verdwenen}\\

\haiku{Een onthutsende,.}{ervaring die mij altijd}{bijgebleven was}\\

\haiku{{\textquoteleft}Ik heb wel gemerkt.}{dat Lilian en jij iets}{met elkaar hebben}\\

\haiku{Maar hij had nog steeds.}{het talent om anderen}{te ontregelen}\\

\haiku{{\textquoteright} Hij zweeg, zoog aan zijn.}{sigaret en begon weer}{ernstig te hoesten}\\

\haiku{{\textquoteright} {\textquoteleft}Mevrouw Van Eerden,{\textquoteright}, {\textquoteleft}.}{zei Mackaywas een echte}{Hollandse mevrouw}\\

\haiku{Dus dit fotootje hoef,{\textquoteright}.}{ik niet in te lijsten zei}{Peter ten slotte}\\

\haiku{Hoor eens, ik heb een.}{jobje voor een paar weken}{in het buitenland}\\

\haiku{Als honderd kikkers.}{op je zeilplank springen wil}{je wel wegwezen}\\

\haiku{Wil je haar zeggen,?}{dat het me spijt van die keer}{in de magnetron}\\

\haiku{Maar een goed einde.}{voor Peter en Christine}{hoorde er ook bij}\\

\haiku{De grond veerde van.}{het bladerafval van de}{afgelopen eeuw}\\

\haiku{Het was ook dom van.}{mij om te proberen het}{te repareren}\\

\haiku{{\textquoteright} Clich\'e-time,.}{wisten mijn hersens en ik}{begon te zwoegen}\\

\haiku{Vader aan zoon die:}{na auto-ongeluk in}{het ziekenhuis ligt}\\

\haiku{{\textquoteright} De mangrove liet.}{ons gaan en we kanoden}{in stilte verder}\\

\haiku{En wij willen je.}{hier niet terug zien komen}{met een lang gezicht}\\

\haiku{Zo'n jonge vrouw met,.}{zo'n ouwe zakenlul dat}{kan niet  duren}\\

\haiku{Ik was ten einde.}{raad en ik legde hem de}{situatie uit}\\

\haiku{Ze schrok niet, en stak.}{haar hand uit om me op de}{kant te trekken}\\

\subsection{Uit: Zwarte rijst}

\haiku{Financi\"en Jan,{\textquoteright},.}{zei ik scherp hij moest nu niet}{opnieuw gaan huilen}\\

\haiku{Baboe zouden we,,.}{vroeger zeggen maar dat mocht}{niet meer dat wist ik}\\

\haiku{{\textquoteright} Victor, Yvonne en.}{Eric lieten zich schaapachtig}{grijnzend voorstellen}\\

\haiku{Nee oom, we hebben,.}{geen bagage bij ons die}{staat in het hotel}\\

\haiku{Als je tenminste.}{genoegen wilt nemen met}{eenvoudige kost}\\

\haiku{We keken elkaar,.}{aan het was nog niet in ons}{hoofd opgekomen}\\

\haiku{Ik was er al bang,.}{voor maar het onderwerp was}{niet te vermijden}\\

\haiku{Ik ben in Birma.}{terechtgekomen en zij}{in Ambarawa}\\

\haiku{{\textquoteleft}Waarom bent u niet,?}{in Holland gebleven na}{uw pensioen oom}\\

\haiku{Er is hier nog een,,.}{Hollander Carl Swerts die komt}{regelmatig langs}\\

\haiku{Er viel een hoop te.}{leren over anatomie in}{het huis van Hanny}\\

\haiku{Een doorgangskamp in.}{Batavia waar ik voor het}{eerst ijskoffie dronk}\\

\haiku{Hoor eens, je moeder,.}{was een heel mooi meisje maar}{een beetje verwend}\\

\haiku{We dronken ons glas,.}{leeg en gingen weg na een}{hartelijk afscheid}\\

\haiku{{\textquoteleft}Ja,{\textquoteright} zei Swerts, maar hij.}{liet zich niet afleiden en}{bleef me aankijken}\\

\haiku{Zijn bediende was,.}{een oud-leerling om maar}{iemand te noemen}\\

\haiku{{\textquoteright} riep ik, {\textquoteleft}Jimmie, ga?}{nu slapen of wil je een}{tik voor je billen}\\

\haiku{De groep dunde uit.}{en uiteindelijk kwam er}{niets meer naar boven}\\

\haiku{{\textquoteright} Ik aarzelde, het.}{vergde een oversteek waar ik}{weinig lust in had}\\

\haiku{{\textquoteright} {\textquoteleft}Ach shit,{\textquoteright} zei Victor, {\textquoteleft}?}{wat jullie vroeger deden}{kunnen wij toch ook}\\

\haiku{Hij stapte op de.}{boomstam en begon naar de}{overkant te lopen}\\

\haiku{{\textquoteright} Hij sloeg met zijn vuist,.}{op het kleine tafeltje}{de glaasjes dansten}\\

\haiku{En dan ben ik zo}{stom om hem uit die sloot te}{halen en verzwik}\\

\haiku{Ik trok haar nog wat.}{dichter naar me toe en ze}{zuchtte behaaglijk}\\

\haiku{{\textquoteright} Hij keek de kring langs.}{en zijn blik bleef rusten op}{de vier getrouwen}\\

\haiku{De revolutie,,.}{vindt op dit moment plaats in}{de stad op het land}\\

\haiku{En daarna keren.}{we terug naar het goede}{leven van vroeger}\\

\haiku{Een politieman.}{stond wijdbeens boven hem en}{richte zijn pistool}\\

\haiku{Al die jaren dat,.}{we gewacht hebben tot het}{zou overgaan huhu}\\

\haiku{Daar moest ik mezelf.}{maar aan vastklampen en niet}{verder nadenken}\\

\haiku{Uit de donkere.}{middentuin steeg een chronisch}{en scherp gonzen op}\\

\section{Simon Vestdijk}

\subsection{Uit: Verzamelde romans. Deel 9. Aktaion onder de sterren}

\haiku{Hyperenoor, de,;}{geweldige bleef zo lang}{mogelijk moorden}\\

\haiku{{\textquoteright} Op zo scherpe toon,.}{had hij gesproken dat de}{jongen vuurrood werd}\\

\haiku{{\textquoteright} {\textquoteleft}Wou je een derde?}{reden hebben om me weg}{te laten jagen}\\

\haiku{En negen maanden.}{later werd Orion door de}{aarde geboren}\\

\haiku{Veeteelt en landbouw;}{waren innig verstrengeld}{in deze lezing}\\

\haiku{Van deze vrouwen.}{was de machtige moeder}{op de burcht er \'een}\\

\haiku{De bezoeken aan.}{de hut had hij gestaakt uit}{een dof schuldbesef}\\

\haiku{Zij zal je niet meer.}{erkennen als waardig om}{haar dienst te leiden}\\

\haiku{{\textquoteleft}Was er, o zoon van,,?}{Apollo een antwoord bij dat}{uw goedkeuring heeft}\\

\haiku{Treedt de Kentauros ',!}{int licht want golven zijn}{dansende paar den}\\

\haiku{{\textquoteleft}Denkt gij, dat onze?}{mensen deze verklaring}{zullen aanvaarden}\\

\haiku{Maar ik ben moeilijk,}{van de gedachte af te}{brengen dat het t\'och}\\

\haiku{mij getrouwd te zien,!}{met Timandra de kroon op}{uw opvoederswerk}\\

\haiku{laag, lemig en met.}{grote gaten die wellicht}{poorten voorstelden}\\

\haiku{Waarom Aktaion?}{zelf niet gekomen was om}{haar dit te zeggen}\\

\haiku{Een ogenblik stond zij.}{verwezen te kijken bij}{deze ontdekking}\\

\haiku{Cheiron beefde over.}{zijn gehele lichaam en}{kon niet verder gaan}\\

\haiku{Cheiron nodigde,:}{hem opnieuw uit binnen te}{treden maar hij riep}\\

\haiku{{\textquoteleft}Neen, vader Cheiron,,.}{wat ik te zeggen heb is}{vlug genoeg gezegd}\\

\haiku{Met deze zelfde.}{Hermesianax sprak ik}{veel over de goden}\\

\haiku{{\textquoteleft}Ik wil niet de boom, -, -?}{in ik wil niet waarom moet}{ik nu de boom in}\\

\haiku{{\textquoteright} vroeg Aktaion, de.}{blik onafgebroken op}{de klimmer gericht}\\

\haiku{Zodra zij hem naar,.}{Artemis trachtte heen te}{dringen week hij uit}\\

\haiku{Bedenkt, dat jonge;}{mensen de wereld anders}{zien dan wij grijsaards}\\

\haiku{Aktaion keerde.}{een eigenaardig pruilend}{gezicht naar hen toe}\\

\haiku{de afbeelding van,.}{een man die zijn gezicht met}{krijt had ingesmeerd}\\

\haiku{{\textquoteright} {\textquoteleft}Tot de zon en de.}{maan zich verenigen om}{je te verblinden}\\

\haiku{Voor iemand volleerd.}{als jij is dit geen plaats om}{binnen te treden}\\

\haiku{Maar Karion kan,.}{slecht oordelen omdat hij}{zelf achterlijk is}\\

\haiku{En weer keek hij om,,.}{zich heen langzaam geen spleet in}{de rotsen overslaand}\\

\subsection{Uit: Verzamelde romans. Deel 19. Bevrijdingsfeest}

\haiku{men was als mens, en.}{hoe groot de waarheid waarvoor}{men had te strijden}\\

\haiku{- {\textquoteleft}Ik hoop niet, dat hij.}{op deze terreinen de}{beest gaat uithangen}\\

\haiku{Ook toen zij voor de,.}{afgravingen stonden sprak}{hij niet dadelijk}\\

\haiku{Je kunt het alleen,.}{goedmaken als je naar dat}{oude mens toegaat}\\

\haiku{{\textquoteright} {\textquoteleft}Daar schiet je wat mee...{\textquoteright} {\textquoteleft}?}{opWas die boswachter met}{je meegelopen}\\

\haiku{Hij was zeer stipt in,;}{alles en aan zijn aandacht}{ontsnapte weinig}\\

\haiku{ik heb er heel wat,...}{gekend al wil ik daar nu}{niet over uitweiden}\\

\haiku{Op zichzelf prachtig,.}{want de regering  zal}{daar nooit voor zorgen}\\

\haiku{{\textquoteright} {\textquoteleft}Dat komt omdat er.}{geen spoor van hysterie in}{deze muziek is}\\

\haiku{Ten slotte liep zij.}{naar hem toe om zijn mond te}{sluiten met een kus}\\

\haiku{En een psychische,,;}{behandeling nou ja wat}{betekent dat nou}\\

\haiku{{\textquoteright} {\textquoteleft}Maar meneer Hoeck, is?}{er dan nooit eens iemand naar}{hem komen kijken}\\

\haiku{Een neef van hem heeft,;}{me geschreven of hij niet}{terug kan komen}\\

\haiku{wel de moeite waard.}{eens te bezoeken voor een}{niet-Amsterdammer}\\

\haiku{{\textquoteleft}Het is hier alle,{\textquoteright}.}{dagen keet zei Lucy op}{hartelijke toon}\\

\haiku{zij liet zich kalmweg,.}{lasseren verwijderde}{toen lachend de haak}\\

\haiku{{\textquoteleft}Ik ben eigenlijk,;}{dichter maar van gedichten}{kun je niet leven}\\

\haiku{als hij vanavond bij,.}{tante Gien was geweest had}{u hem kunnen zien}\\

\haiku{, en dan was ik juist,.}{g\'een kerel want kerels gaan}{tegen de wind in}\\

\haiku{{\textquoteright} schaterlachte hij, {\textquoteleft},!}{maar je hebt me uitstekend}{geholpen prachtig}\\

\haiku{Het leek me meer iets,;}{abnormaals zoals mannen}{wel vaker hebben}\\

\haiku{{\textquoteleft}Ik zal je een zoen,!}{geven omdat je me zo}{goed geholpen hebt}\\

\haiku{{\textquoteright} Niet te snel drong Evert.}{de plantaardige berg van}{duisternis binnen}\\

\haiku{Bij hem evenwel had.}{dit gebaar niets herderlijks}{of patriarchaals}\\

\haiku{maar hij bleef, zoals,.}{hij zich uitdrukte in elk}{geval fatsoenlijk}\\

\haiku{Door alles heen het;}{geduldige getinkel}{op de piano}\\

\haiku{De hele stad stonk,.}{een beetje maar dat hielp weer}{tegen de honger}\\

\haiku{{\textquoteright} - Petit keek naar zijn,,.}{schoenen hief het hoofd weer op}{en keek langs Evert heen}\\

\haiku{de zwarthandelaars.}{leken zelfs uitgesproken}{melancholici}\\

\haiku{Die arme tante,{\textquoteright}, {\textquoteleft}.}{Gien zei Maymaar ze zal wel}{iets anders vinden}\\

\haiku{Tijdens zijn gesprek;}{met Drost was hij dicht bij een}{duizeling geweest}\\

\haiku{{\textquoteright} Licht wenkbrauwfronsend.}{schudde zij het hoofd en zat}{nog te luisteren}\\

\haiku{Zijn kleren had hij,.}{aangelaten hij moest nu}{toch kunnen opstaan}\\

\haiku{Ik kan hem zeker,...{\textquoteright} {\textquoteleft}?}{wel even meenemen als ik}{weggaZal ik even}\\

\haiku{Belachelijk om.}{je op te winden over een}{dergelijke zaak}\\

\haiku{{\textquoteright} {\textquoteleft}Vernietigen,{\textquoteright} zei, {\textquoteleft}.}{Markman op peinzende toon}{dat is een heel ding}\\

\haiku{{\textquoteright} vroeg Markman, terwijl.}{hij ietwat treuzelend de}{theekop van Evert overnam}\\

\haiku{Bijna roerloos hing.}{het gele gebladerte}{van de grachtbomen}\\

\haiku{Die ochtend achter,.}{het stuur schudde hij zich als}{een morsige hond}\\

\haiku{De oude man kon.}{tegenover Rie Bentz alleen}{maar gesnoefd hebben}\\

\haiku{zij waren al op,;}{weg naar elders zij zweefden}{reeds een klein weinig}\\

\haiku{Men wentelde een,,.}{steen daaronder lag zand een}{halve meter dik}\\

\haiku{Maar nooit beknorde,,.}{zij ze liet dit aan Evert over}{die het evenmin deed}\\

\haiku{maar stemmen uit het.}{dorp waren natuurlijk veel}{aannemelijker}\\

\haiku{We zullen spelen,,.}{dat ik jullie doodschiet net}{als de rotmoffen}\\

\haiku{Ook hij lette op,.}{de kinderen het was zijn}{plicht dit te doen}\\

\haiku{En de dokter, dit?}{levend geweten in zijn}{steenrood omhulsel}\\

\haiku{Ze gunt hem mij dus,,.}{n{\'\i}et dacht Jeanne en toch}{is ze niet jaloers}\\

\haiku{Maar ik zou jou niet,.}{eens kunnen haten wanneer}{Evert met je wegliep}\\

\haiku{Maar een krachtproef zou,,.}{het in elk geval blijven}{morgen overmorgen}\\

\haiku{En oma zal het ook,.}{wel gehoord hebben en tobt}{er nu misschien over}\\

\haiku{Evert klopte hem op,;}{de schouder het werd nu een}{echt mannengesprek}\\

\haiku{Zij geloofde nu,;}{wel dat alles toch goed was}{tussen hen beiden}\\

\haiku{Ook de latere,,.}{Okke een grote grijze}{was ongesneden}\\

\haiku{De kater keek hem.}{ondoorgrondelijk aan en}{begon te spinnen}\\

\haiku{Ryers en Els moesten.}{geen vijf minuten na hem}{naar bed zijn gegaan}\\

\haiku{hij zag het rose;}{lampje en de lichtglans op}{zijn kale schedel}\\

\haiku{Het lange haar gaf;}{iets vrouwelijks aan zijn toch}{niet verwijfd gezicht}\\

\haiku{Anna, voor wie zijn,.}{ziel zo doorzichtig was zou}{het hebben geloofd}\\

\haiku{Het laatste wrakstuk.}{van een uiteengeslagen}{wereldbeschouwing}\\

\haiku{zij moest nu vooral,.}{niet menen dat hij in zijn}{wiek geschoten was}\\

\haiku{Nog niet eens zijn ziel,,;}{of zijn geest want die heeft hij}{misschien al niet meer}\\

\haiku{Iemand die \'een vrouw,;}{heeft komt er niet licht toe een}{tweede te nemen}\\

\haiku{Daar het dienstmeisje,:}{goed getraind was was er maar}{\'een mogelijkheid}\\

\haiku{Het was te groot voor,.}{een receptenpapiertje}{te klein voor een brief}\\

\haiku{{\textquoteleft}Nou zullen ze mij!}{godverdomme ook nog voor}{een flikker houden}\\

\haiku{Dat Evert hem ooit nog,.}{geld zou geven geloofde}{hij allang niet meer}\\

\haiku{En niet ver van hem,.}{zat Louis Drost met vier jonge}{mannen te praten}\\

\haiku{{\textquoteleft}Je zei toen iets over.}{wat Markman in de oorlog}{uitge1haald zou hebben}\\

\haiku{Zijn moeder vond hij.}{lijkbleek en moeilijk ademend}{op haar slaapkamer}\\

\haiku{hoe hij dit effect,.}{bereiken kon wanneer hij}{Markman's naam verzweeg}\\

\haiku{- {\textquoteleft}Het maakt op mij meer.}{de indruk van chantage}{dan van amusement}\\

\haiku{hij bracht de vinger.}{aan de lippen en wees op}{de suitedeuren}\\

\haiku{Maar het wordt anders.}{wanneer ze hem onder zijn}{armen gaan kijken}\\

\haiku{mompelen, dat veel:}{weghad van een herhaling}{dierzelfde woorden}\\

\haiku{Bijzonderheden,}{weet ik niet tenminste niet}{uit de eerste hand.}\\

\haiku{Er lag ongeveinsd,:}{medegevoel in zijn stem}{toen hij zachtjes zei}\\

\haiku{Toen hij klaar was, vroeg,:}{Mahrholtz op een toon die}{hem deed ophoren}\\

\haiku{daarmee heb je een...{\textquoteright} {\textquoteleft}...}{zeer groot gedeelte van je}{schuld afbetaaldSchuld}\\

\haiku{Schrijf mij een brief, als,.}{je aangekomen bent ik}{moet je adres weten}\\

\haiku{Mahrholtz had hem,;}{veel te verwijten en veel}{aan hem te danken}\\

\haiku{Rundstedt und Rommel, -,...}{haben gesagt ja was haben}{die nicht alles gesagt}\\

\haiku{Hij kroop erheen, en:}{onderzocht het wapen met}{bevende vingers}\\

\haiku{Maar Mahrholtz zag,,;}{h\'em zijn heer en meester en}{verleider heel goed}\\

\haiku{Maar ik zou toch even.}{tijd gevonden hebben bij}{je aan te komen}\\

\haiku{{\textquoteleft}Het is ellendig,{\textquoteright}, {\textquoteleft}.}{zei hij moeilijkik weet niet}{wat ik zeggen moet}\\

\haiku{{\textquoteright} - Zonder haar aan te,,.}{kijken zag hij dat zij de}{schouders ophaalde}\\

\haiku{En vertel het in,,,!}{Godsnaam niet aan Anna Evert}{toe beloof me dat}\\

\haiku{Beloof me, als het,.}{maar enigszins doenlijk is dit}{te verhinderen}\\

\subsection{Uit: Else B\"ohler, Duitsch dienstmeisje}

\haiku{Simon Vestdijk,}{Else B\"ohler Duitsch dienstmeisje}{Colofon}\\

\haiku{Ik zou nauwelijks,:}{klaarkomen in de weken}{die mij nog resten}\\

\haiku{Maar opgeschreven, -.}{heb ik het nog nooit en dat}{moet nu gebeuren}\\

\haiku{Zeven bladzijden,:}{zijn verstreken nu zal ik}{woord moeten houden}\\

\haiku{Nooit heb ik van een.}{mannelijk lichaam gewalgd}{als van het zijne}\\

\haiku{ik meende alleen.}{een onbedwingbare trek}{te hebben in thee}\\

\haiku{Als ik mijn moeder,:}{vroeg wat minder door het huis}{te schreeuwen zei ze}\\

\haiku{Meneer Steketees,,.}{platje gelijkvormig aan}{het onze was leeg}\\

\haiku{In drie stappen was,.}{ik bij de suitedeuren}{wrong me er doorheen}\\

\haiku{Moeilijkheden van.}{zinsbouw en idioom nam ik}{als hindernissen}\\

\haiku{Onafgebroken.}{voelde ik haar zwarte ogen}{op mij gevestigd}\\

\haiku{Ik begreep, dat ik,.}{het onderspit delven zou}{als het zo doorging}\\

\haiku{Maar ook wist ik, dat...}{die slag geen minuut later}{had moeten komen}\\

\haiku{tegenover mezelf,.}{bleef die even groot welke weg}{ik ook zou inslaan}\\

\haiku{dann lege ich drei,.}{Steine hin auf den Balkon}{dann weisst du Bescheid}\\

\haiku{Raakte ik haar borst,}{aan dan gedoogde ze dat}{net zo lang tot ik}\\

\haiku{trouwen konden we,.}{in Duitsland of Belgi\"e als}{Utrecht onwillig bleek}\\

\haiku{Tevergeefs hield ik,;}{haar voor hoe zalig het nu}{in het park zou zijn}\\

\haiku{Else B\"ohler werd nu;}{ontoegankelijk voor al}{mijn opmerkingen}\\

\haiku{druk babbelend sprong;}{ze van het ene onderwerp}{op het andere}\\

\haiku{ook het uitspreken.}{van mijn eigen voornaam ging}{nooit zonder kleuren}\\

\haiku{Ik was ineens diep,.}{gelukkig ik had kunnen}{huilen als een kind}\\

\haiku{Mijn duiveltje dook,.}{weer op  met een dikke}{tong van de warmte}\\

\haiku{{\textquoteright} - {\textquoteleft}M\"ochtest du dann, dass?!}{ich auf meine Kniee fiel wie}{Fr\"aulein Erkelens}\\

\haiku{Overigens heb je!}{er niets van begrepen wat}{ik toen beweerde}\\

\haiku{God, o god, wat een...{\textquoteright} {\textquoteleft},?}{afschuwelijk levenLaat}{u me door of niet}\\

\haiku{{\textquoteright} Bliksemsnel streek mijn,.}{moeders hand mij door het haar}{toen ik mijn sprong nam}\\

\haiku{Zij had zich van me.}{losgemaakt en zwaaide met}{haar lange armen}\\

\haiku{Het was duidelijk,;}{dat er geen zuinigheid bij}{haar voor kon zitten}\\

\haiku{een demonstratief.}{gekraak bewees dat ze haar}{bed had opgezocht}\\

\haiku{Die inktvis met 't,!}{achterwerk van een kreeft wat}{zit daar al niet in}\\

\haiku{{\textquoteright} Wat ik nu ging doen,,.}{verbaast mij nu ik er aan}{terug denk nog steeds}\\

\haiku{Na alles wat je,!}{me zo juist verteld hebt ligt}{het toch voor de hand}\\

\haiku{Langzaam viel de deur.}{achter mij dicht en kleefde}{weer in de posten}\\

\haiku{Het was mijn moeder,...}{dus niet maar Eg die nu in}{de huiskamer zat}\\

\haiku{Ich h\"atte Dir schon,...?}{l\"angst ehergeschrieben aber Wie}{geht es Dir sonst noch}\\

\haiku{Verlange nur nach,.}{Dir uw Dich einmal gl\"ucklich}{machen ta kannen}\\

\haiku{Heute Abend blutet,!}{mein Hert es schreit nach etwas}{Unmdgliches}\\

\haiku{Bis Heute darf ich.}{frei and ehrlich noch jeden}{gegen\"uberstehen}\\

\haiku{Ik wilde het doen:}{voorkomen alsof de brief}{mij geheel koud liet}\\

\haiku{Een naaimandje op.}{de tafel was zo groot als}{een boodschappenmand}\\

\haiku{Ik begreep niet, hoe.}{ik het zo lang bij die stem}{uitgehouden had}\\

\haiku{Peter zou mij nu...}{een lesje willen geven}{met de vrouw van 43}\\

\haiku{Hij stond in de deur.}{van het atelier verbaasd naar}{boven te kijken}\\

\haiku{Toen ik zei, dat ik,.}{weg ging bedoelde ik niet}{alleen hier vandaan}\\

\haiku{{\textquoteleft}U had haar bij ons,;}{thuis moeten ontvangen uit}{eigen beweging}\\

\haiku{Ik zocht naar meisjes;}{met vlechten die op Else}{B\"ohler zouden lijken}\\

\haiku{Ik moest enkele.}{malen slikken voordat ik}{naar binnen durfde}\\

\haiku{van de vijf winkels}{die ik hier overzag koos ik}{om te beginnen}\\

\haiku{die welke zo ver;}{mogelijk van de winkel}{van Steinmann aflag}\\

\haiku{{\textquoteright} vroeg ik ademloos, na.}{haar een paar pas achterna}{gelopen te zijn}\\

\haiku{hoe ik Else zou,;}{kunnen vinden en waar\`om ik}{haar wilde vinden}\\

\haiku{Und Moskau zahlt mir,.}{all den Schund So fahre ick}{gang Deutschland rund}\\

\haiku{Ik stelde vast, dat,.}{men mij overwonnen had dat}{ik terug moest gaan}\\

\haiku{Het woord {\textquoteleft}Sch\"utzenfest{\textquoteright},.}{kende zij w\`el maar dat is}{het \'o\'ok niet geweest}\\

\subsection{Uit: Verzamelde romans. Deel 35. De filosoof en de sluipmoordenaar}

\haiku{wie weet hadden zij:}{zelfs iets met de letteren}{uitstaande gehad}\\

\haiku{Ik heb gezien, dat,,.}{u een dapper man bent die}{recht heeft op mijn naam}\\

\haiku{Beauregard was naar;}{zijn regiment vertrokken}{in de Provence}\\

\haiku{Niemand scheen Karel,.}{als een levend eens levend}{mens te beschouwen}\\

\haiku{De gezochte stond,.}{bij een bloemenkar vijf of}{zes huizen verder}\\

\haiku{- {\textquoteleft}Lef\`evre is geen,,.}{naam waarvoor u zou moeten}{vluchten kolonel}\\

\haiku{Wat zoudt u mij al!}{niet over de Zweedse koning}{kunnen verhalen}\\

\haiku{Natuurlijk waren:}{er de paspoortachtige}{uiterlijkheden}\\

\haiku{In uw brief repte...{\textquoteright} {\textquoteleft},!}{u van een half uurGaat u}{zitten kolonel}\\

\haiku{Had ik verder mijn,.}{mond gehouden dan had er}{geen haan naar gekraaid}\\

\haiku{h\'ad u het gedaan,.}{dan zou u het mij gerust}{kunnen vertellen}\\

\haiku{Dit zijn de Pens\'ees,.}{van Blaise Pascal die u}{van naam zult kennen}\\

\haiku{dat had hij aan de;}{zoveel jongere Alexander}{moeten overlaten}\\

\haiku{Met een krachtig woord.}{riep Arouet zijn factotum}{tot de orde}\\

\haiku{Het eerste wat hem;}{bereikte was de geur van}{zeer goede koffie}\\

\haiku{Holm aan het werk, en:}{nog geen drie weken later}{komt er een wagen}\\

\haiku{Ik zal aanstonds de,.}{knecht roepen dan kan hij ons}{koffie inschenken}\\

\haiku{Wij gaan met niemand,...}{om en binnenkort gaan we}{naar Zweden terug}\\

\haiku{{\textquoteright} {\textquoteleft}Pardon, mevrouw,{\textquoteright} zei, {\textquoteleft}.}{Arouethet boek handelt over}{Karel de Twaalfde}\\

\haiku{Het wordt allemaal,...}{later geregeld tot ons}{aller genoegen}\\

\haiku{Past u op, anders!}{laat hij mijn kopje vallen}{of dat van de graaf}\\

\haiku{{\textquoteright} {\textquoteleft}Wanneer ik u zo,...}{hoor zou ik ertoe kunnen}{komen een boek over}\\

\haiku{{\textquoteright} {\textquoteleft}Is het waar, mijnheer,?}{dat hij zichzelf de kroon op}{het hoofd heeft gezet}\\

\haiku{Het komt eigenlijk.}{alleen doordat zij en ik}{met niemand omgaan}\\

\haiku{laat h\'em dan schrijven, -, -;}{in dat boek dus dat hij je}{voor onschuldig houdt}\\

\haiku{{\textquoteright} {\textquoteleft}Ik vermoed, dat hij,.}{zich verveelde en wilde}{filosoferen}\\

\haiku{Een staatsieportret,;}{was het niet de koning was}{niet in groot ornaat}\\

\haiku{{\textquoteright} zei Holm, die over een,, {\textquoteleft}}{zeer zachte wat hese stem}{bleek te beschikken}\\

\haiku{In 1718 was het in,.}{Stockholm de schilder heb ik}{toen nog even ontmoet}\\

\haiku{Er schuilt in deze.}{gelaatstrekken iets van een}{gevallen engel}\\

\haiku{Deze verschijning.}{had geglimlacht en op zijn}{gezicht gewezen}\\

\haiku{Ziekte scheen hij als.}{ernstiger te beschouwen}{dan verwondingen}\\

\haiku{Over zijn karakter.}{schrijf ik natuurlijk alleen}{wat ik zeker weet}\\

\haiku{De koning stond stil,,.}{keek naar D\"uring en begon}{zachtjes te lachen}\\

\haiku{Of eigenlijk nog:}{meer door wat u mij over de}{gravin vertelde}\\

\haiku{Maar tevens zou het}{erg interessant  voor}{u zijn om te zien}\\

\haiku{{\textquoteleft}U te beloven.}{het in mijn boek openlijk voor}{u op te nemen}\\

\haiku{{\textquoteright} {\textquoteleft}Dat is niet nodig,{\textquoteright}, {\textquoteleft}.}{zei Siquier droogjesik}{ken die gevoelens}\\

\haiku{U moet vooral ook,.}{niet menen dat mijn leven}{dan verwoest zou zijn}\\

\haiku{Zij schreef, dat zij hem.}{gaarne deze zelfde avond}{nog zou ontvangen}\\

\haiku{En zij van haar kant:}{was ge{\"\i}mponeerd door de}{militaire stand}\\

\haiku{Dan moet Octave,.}{er maar voor zorgen dat zijn}{naam gezuiverd is}\\

\haiku{{\textquoteright} viel de graaf in, {\textquoteleft}mijn.}{zuster riep mij om koffie}{te komen drinken}\\

\haiku{{\textquoteleft}In elk geval maakt.}{het niet het minste verschil}{voor mijnheer Arouet}\\

\haiku{dat de kolonel.}{Zweden waarschijnlijk spoedig}{weer zou verlaten}\\

\haiku{Jij hebt Octave,!}{zelf verzekerd dat je hem}{voor onschuldig hield}\\

\haiku{Even later was hij,.}{misschien slaperig toen hij}{het zei beslist niet}\\

\haiku{Ook dat hielp hem niet,.}{al is daar tenminste nog}{een maand naar genoemd}\\

\haiku{ik heb twee gekken,,.}{als zoons de een schrijft proza}{de ander verzen}\\

\haiku{In de open deur stond,.}{de huishoudster hijgend van}{het trappen lopen}\\

\haiku{Siquier kwam de.}{kamer binnen met \'e\'en hand}{tastend naar voren}\\

\haiku{Toch alleen maar dat?}{zij tegenover mij haar mond}{voorbij heeft gepraat}\\

\haiku{Maar ik kan hierna.}{geen vertrouwen meer hebben}{in haar karakter}\\

\subsection{Uit: Verzamelde romans. Deel 37. De held van Temesa}

\haiku{ik zeg dit waarlijk,':}{niet omdat ik Polites}{priester ben geweest}\\

\haiku{{\textquoteright} Nooit heb ik iemand,,.}{zo zien verbleken ik dacht}{dat hij mij zou slaan}\\

\haiku{Periphas was,;}{wel een goede jongen en}{verstandig genoeg}\\

\haiku{Toen ik terugkwam,,,,;}{stond hij er nog zwaar blank en}{slaperig oeroud}\\

\haiku{{\textquoteleft}Roeit elkander uit,,.}{Kroton en Sybaris mijn}{volk blijft erbuiten}\\

\haiku{Het mijne vond ik;}{waar mijn makkers het hadden}{achtergelaten}\\

\haiku{De zegswijze {\textquoteleft}hij{\textquoteright}.}{vecht als een Krotoni\"er}{stamt uit die dagen}\\

\haiku{Toen richtte zij de,,:}{ogen weer op mij heel rustig}{niet onwelwillend}\\

\haiku{Maar het was waar, een.}{ander antwoord kon zij niet}{van mij verwachten}\\

\haiku{mijn zoon, want je schijnt.}{wel wat vaderlijke steun}{nodig te hebben}\\

\haiku{Ik zou schatrijk zijn,.}{als ik deze boosdoeners}{niet om mij heen had}\\

\haiku{hier komt Hekate,,, -,.}{brrr boe een luimige}{vrouw mag ik zeggen}\\

\haiku{Zij hielden het op,;}{het oude maar stonden ook}{voor het nieuwe open}\\

\haiku{Maar ik had niet eens.}{het arglistig groen in zijn}{ogen zien verschieten}\\

\haiku{Het was ook niet in;}{mijn voordeel al te tergend}{met hem te spelen}\\

\haiku{Bij mijn rustbed stond,.}{een man niet te herkennen}{in de schemering}\\

\haiku{Maar toen, alsof het,.}{erop gewacht had verscheen}{het valkenkopje}\\

\haiku{hoe w\'eet de Held, of,?}{ze nog leven of een heel}{klein beetje leven}\\

\haiku{Heel Hellas legde.}{de kiemen van ontbinding}{in onze boezem}\\

\haiku{Haar meesterlijke.}{onzijdigheid richtte zich}{nu op mij alleen}\\

\haiku{Ik heb met een der,.}{beambten gesproken en}{ik ben geschrokken}\\

\haiku{Sinds mijn aanstelling,,.}{is hij naar het schijnt weer met}{wurgen begonnen}\\

\haiku{Orion jaagt daar nog,.}{steeds Achilles trekt ten strijde}{of zit in zijn tent}\\

\haiku{Ook als je niet bij,,}{ons komt neem dan toch ontslag}{als priester doe dat.}\\

\haiku{{\textquoteright} Midden in de nacht,.}{werd ik gewekt omdat het}{hero\"on in brand stond}\\

\haiku{Het kostte mij heel.}{wat moeite hem tot beter}{inzicht te brengen}\\

\haiku{2 De volgende:}{ochtend begaven zich naar}{het prytaneion}\\

\haiku{Maar om de Held te.}{kunnen wegjagen moet men}{hem eerst ontmoeten}\\

\haiku{Dat Praxidamas zijn.}{naam wil lenen is misschien}{toch niet voldoende}\\

\haiku{Nu staat het vast, dat.}{Theagenes in diens}{bestaan geloofde}\\

\haiku{Er zou wat bloed van':}{Polites aan Euthymos}{zwaard moeten kleven}\\

\haiku{{\textquoteright} {\textquoteleft}Het voorschrift wil, dat.}{zowel het offer als de}{priester in slaap zijn}\\

\haiku{{\textquoteleft}Men had mij hierop,.}{voorbereid men schilderde}{u af als koppig}\\

\haiku{Later bleek hij van.}{het Lokrisch gezelschap toch}{deel uit te maken}\\

\haiku{{\textquoteright} {\textquoteleft}Ja...{\textquoteright} - Theagenes,.}{liet al zijn voorhoofdsrimpels}{spelen links en rechts}\\

\haiku{De exegeet had hier,.}{veel verdriet van hij begreep}{het niet helemaal}\\

\haiku{dat spel ken ik niet,{\textquoteright},.}{en stak zijn hand uit als om}{ze aan te raken}\\

\haiku{Maar bovendien zag.}{ik er tegenop om met}{hem alleen te zijn}\\

\haiku{ze had Euthymos,!}{gezien  en er nooit met}{mij over gesproken}\\

\haiku{maar onderwijl dacht.}{ik voortdurend na over wat}{ik hoorde en zag}\\

\haiku{{\textquoteleft}O, is het dat,{\textquoteright} zei, {\textquoteleft}.}{ik met een lachjemaar nu}{vergist u zich toch}\\

\haiku{Ik wist alleen, dat,.}{u een dochter had die nu}{veertien jaar moet zijn}\\

\haiku{{\textquoteright} Terwijl hij zich naar,:}{de deur gewoog hoorde ik}{hem nog prevelen}\\

\haiku{Op een middag, toen,.}{zij bij die weduwe was}{doorzocht ik de hoop}\\

\haiku{Eindelijk trok ik,,.}{hem naar binnen sloot de deur}{maar maakte geen licht}\\

\haiku{het duurde een tijd,:}{voordat ik begreep dat het}{betekenen moest}\\

\haiku{Later hoorde ik,.}{dat hij de bewakers in}{zijn huis niet verdroeg}\\

\haiku{Men schreeuwde nog niet,.}{aan \'een stuk door maar men was}{bereid dit te doen}\\

\haiku{Toen de agora nog,.}{natgeregend werd scheen bij}{mij de zon alweer}\\

\haiku{Staken van de dienst,,, -.}{nooit  meer een offer nooit}{meer meer wist hij niet}\\

\haiku{En deze Held zoudt?}{gij laten wegboksen door}{een onbevoegde}\\

\haiku{Zoals gij weet, zijn.}{hier twee betekenissen}{aan te verbinden}\\

\haiku{Is hij niet moedig,.}{genoeg dan mogen wij geen}{toestemming geven}\\

\haiku{Bij de uitlegging.}{van het orakel heeft hij de}{doorslag gegeven}\\

\haiku{men weet het maar al,.}{te  goed alleen niet in}{bijzonderheden}\\

\haiku{Zijn haar was bijna,,.}{spierwit en zeer dun ofschoon}{hij nog niet kaal was}\\

\haiku{Ziet, hij springt over de,.}{muur en in het bos kraken}{de dorre takken}\\

\haiku{Want ik geloof, dat,,.}{u moedig bent Euthymos}{ik heb het gezien}\\

\haiku{Onder het volk gaat,.}{het gerucht dat Euthymos}{met haar trouwen zal}\\

\haiku{Er zijn ook nieuwe,.}{gezichten onder u als}{ik mij niet bedrieg}\\

\haiku{zo nam ik de taak, {\textquoteleft}.}{van Theagenes overwaar}{anderen bij zijn}\\

\haiku{Nu en dan wierp hij;}{een ontevreden blik over}{de aanwezigen}\\

\haiku{Hier had hij zijn zwaard,.}{neergelegd niet ver van dat}{van zijn belager}\\

\haiku{{\textquoteright} {\textquoteleft}Goed, dan een schim,{\textquoteright} zei, {\textquoteleft}.}{Euthymos wreveligmaar}{laat mij uitspreken}\\

\haiku{Ik wilde Plexippos,.}{niet wekken dat hadden wij}{niet afgesproken}\\

\haiku{Hij schuifelde wat,,.}{met \'een voet en keek langzaam}{opzij onderuit}\\

\haiku{Of misschien dacht u,...{\textquoteright} {\textquoteleft},{\textquoteright}}{niet eens dat ik loogDat dacht}{ik inderdaad niet}\\

\haiku{{\textquoteright} {\textquoteleft}Daar kunt u mij ook,.}{onder rekenen want ik}{begrijp er niets van}\\

\haiku{Gestenigd zou u,.}{niet meer worden maar er zijn}{andere straffen}\\

\haiku{{\textquoteright} {\textquoteleft}Ik dan rechtstreeks naar,.}{de agora waar ik u zal}{kunnen aanklagen}\\

\haiku{u heeft de Held niet,.}{gezien en u heeft hem niet}{kunnen verjagen}\\

\haiku{Ik vind het alleen,.}{maar rechtvaardig dat u het}{nu zelf ondervindt}\\

\haiku{{\textquoteright} {\textquoteleft}Maar wie waarborgt mij,?}{dat u na afloop niet t\'och}{een aanklacht indient}\\

\haiku{Maar werd het laat, dan.}{zou ik mijn plannen een dag}{uit moeten stellen}\\

\haiku{Hij zou daar aan huis.}{moeten komen om zijn}{verhaaltje te doen}\\

\haiku{Ziet men Euthymos?}{en Krokinas naast elkaar}{voor de rechter staan}\\

\haiku{Geen weldenkend mens,}{zal ontkennen dat wat ik}{ging ondernemen}\\

\haiku{Wij samen weten,...}{wel beter maar daar heeft z{\'\i}j}{niets mee te maken}\\

\haiku{E\'enmaal had hij,.}{de hand bewogen een niet}{onsierlijk gebaar}\\

\haiku{Hij moest weten, dat.}{de bewaker vijf of zes}{pas achter hem stond}\\

\haiku{Hij voelde zich niet,;}{op zijn gemak dat was aan}{alles te merken}\\

\haiku{Zij had de armen,.}{om mij heengeslagen en}{wilde nooit meer weg}\\

\subsection{Uit: Verzamelde romans. Deel 2. Meneer Visser's hellevaart}

\haiku{fusilleert alle!!}{voorradige kolonels}{als landverrader}\\

\haiku{Proberen vanavond,,,,,...}{met Anton dag broer bgoer bgoer}{dag bgoer dag bgoerie}\\

\haiku{{\textquoteright} Dat loeren kan ze,.}{niet laten ze w\'eet wie er}{om vijf voor half belt}\\

\haiku{Wanneer Visser over,;}{zijn vrouw begon antwoordde}{hij met de jongens}\\

\haiku{Hij ging bijvoorbeeld,.}{zo staan dat ze telkens een}{omweg maken moest}\\

\haiku{En zoals altijd:}{liet hij zijn oog dreigend over}{de tafel glijden}\\

\haiku{Op de loper \'een,, - ', '!}{twee alst oneven is}{zalt geen pijn doen}\\

\haiku{Dadelijk, jongen,.}{maar laat me eerst je moeder}{een handje geven}\\

\haiku{Jansonius tonnenman,...}{zo ben ik trouwens aan de}{heren gekomen}\\

\haiku{Gelukkig is de,,...}{keukendeur dicht Marie weet}{dat ik niet gestoord}\\

\haiku{Zijn angst met geweld,.}{wegduwend concentreerde}{hij zich op die taak}\\

\haiku{het kleine dorpje,.}{in de buurt van Weulnerdam}{waar hij vandaan kwam}\\

\haiku{wie laat er nou in!}{christesnaam een bouffante}{op een stoel hangen}\\

\haiku{Hij boog zich omlaag.}{en richtte zijn gezicht scheef}{op naar de hemel}\\

\haiku{Een kaatsbal of een '.}{tol door de ruiten was nog}{niet eenst slimste}\\

\haiku{De tuin, die er bij,;}{hoorde zette zich voort tot}{aan de Achterweg}\\

\haiku{misschien weet ie wat...!}{positiefs over die schorsing}{Niet-aan-denken}\\

\haiku{Je moet 't hier dan,.}{maar zien te bolwerken met}{inspecteur Blanksma}\\

\haiku{Wie anders dan ik?}{was de belangrijkste man}{op dat uniek moment}\\

\haiku{Het welbehagen,.}{was blijven duren maar werd}{nu bijna drukkend}\\

\haiku{t Zal u anders,!}{wel amuseren dat gedoe}{in een kleine plaats}\\

\haiku{Maar toch deed het hem,...}{goed iemand anders in zijn}{val mee te slepen}\\

\haiku{Hij moest eens precies!}{weten hoe of een hart nou}{eigenlijk smaakte}\\

\haiku{een ziekte na de, ':}{dood een ziekbed in vet of}{boter opt vuur}\\

\haiku{Maar dat hameren '.}{op die werf is toch weln}{opwekkend geluid}\\

\haiku{Achter, naast me. Daar,... '...}{helemaal links voorbijt}{been van m'n oogkas}\\

\haiku{Bah... Zorgen dat ik...,;}{om drie uur thuis ben Zware}{laarzen die pummel}\\

\haiku{En omdat ze maar... {\textquoteleft}}{niet ophield met zeuren wat}{kon ik toen anders}\\

\haiku{{\textquoteleft}Nee, dat was tegen,,.}{jou Cohenn hij vind die mop}{aardiger dan ik}\\

\haiku{In een auto, in,....}{een auto je stapt er in}{Formidabel wijf}\\

\haiku{De rol van Lehmans,...}{nog even repeteren als}{ik van Eveking af}\\

\haiku{niemand durft weggaan,.}{om de anderen net als}{op een visite}\\

\haiku{aan het strand... grote......}{warme hand warm poposant}{blauw glanzig schepje}\\

\haiku{Daar uit de Raamstraat.}{komen er nog twee van die}{aangeklede apen}\\

\haiku{In d\'eze angst, door,.}{hemzelf uitgelokt was hij}{volkomen veilig}\\

\haiku{Ik zou nou alleen '.}{wel willen weten hoe of}{t nu verder moet}\\

\haiku{{\textquoteleft}U zult 't wel gek,, '.}{vinden meneer maart was}{de commissaris}\\

\haiku{Volgens het Wetboek...{\textquoteright}}{van Strafrecht en het Wretboek}{van Strafvordering}\\

\haiku{De commissaris ',.}{staarde hem int gezicht}{toen naar zijn colbert}\\

\haiku{Er waren zoveel,...}{ijle stemmen in de lucht}{maar die ene bleef weg}\\

\haiku{{\textquoteright} Cohen begon aan.}{de mop van de man in de}{trein die stotterde}\\

\haiku{{\textquoteleft}Zeg lui, nou heb ik...{\textquoteright} {\textquoteleft}!}{de vrouw er weer eens \'a\'ardig}{tussen gehadAha}\\

\haiku{{\textquoteleft}Vertel me maar eens,!}{wat over dat boek dat je van}{me gekregen hebt}\\

\haiku{{\textquoteleft}Hebben u en uw,?}{man militairistische}{neigingen mevrouw}\\

\haiku{elf min drie, elf min,?}{twee ja ja ja waar was ik}{ook weer gebleven}\\

\haiku{gewoon een bol, een,:}{ballon die je kan blazen}{en bij je steken}\\

\haiku{Hij draaide zich om.}{en bewoog zich aarzelend}{naar het rechtse bed}\\

\haiku{{\textquoteright} {\textquoteleft}Ik kan je zo niet,,,!}{ontvangen Visser d'r is}{niets in orde niets}\\

\haiku{Ik ben niet van plan,!}{me voor de gek te laten}{houden verduiveld}\\

\haiku{{\textquoteleft}Ik geef het woord aan,!}{Dr. Touraine geneesheer}{te dezer stede}\\

\haiku{ze zouden misschien...}{wel niet eens begrijpen wie}{daar weggebracht werd}\\

\haiku{Er lopen heel wat '.}{meneer Vissers rond zonder}{t zelf te weten}\\

\haiku{- Weer speelde, in de,.}{slaapkamer het kaarslicht in}{twee vragende ogen}\\

\subsection{Uit: Verzamelde romans. Deel 6. De nadagen van Pilatus}

\haiku{Hij keek nog eens, en,:}{zei met nadruk de hand schuin}{naar omlaag gestrekt}\\

\haiku{De oude man blies,.}{als een panter maar waagde}{zich niet dichterbij}\\

\haiku{Even tevoren had...}{hij die beweging bij de}{keizer opgemerkt}\\

\haiku{ik klim iedere,;}{dag naar het Capitool hij}{stroomt naar de Tiber}\\

\haiku{Over uw indrukken...{\textquoteright} {\textquoteleft},!}{Natuurlijk van dat proces}{deugde geen jota}\\

\haiku{In Rome wonen,...}{betrekkelijk weinig}{Romeinen helaas}\\

\haiku{\'e\'en ervan was nu.}{onbeweeglijk geworden}{onder zijn lippen}\\

\haiku{Ik wil je straks je,,}{handen laten wassen ik}{zie het weer voor me}\\

\haiku{dat zullen we de,!}{kinderen op straat leren}{als aftelversje}\\

\haiku{{\textquoteleft}Nog \'e\'en volksopstand,,.}{mijn beste Pilatus en}{je vrouw gaat eraan}\\

\haiku{Claudia, dat is mijn,,;}{vrouw ze is nu dood was niet}{over hem uitgepraat}\\

\haiku{Ik had niets tegen,,...}{de man Antipas koning}{Antipas ook niet}\\

\haiku{Wij vonden iemand, -:}{bereid een graf af te staan}{je herinnert je}\\

\haiku{Ik kan het alleen,... -,{\textquoteright}}{wel af ik Ik zal mijn best}{doen voor je vader}\\

\haiku{Kijk eens, ik had een.}{opstand moeten dempen om}{uw man te redden}\\

\haiku{Hij herhaalde slechts,.}{wat Maria hem had gezegd}{de vorige nacht}\\

\haiku{{\textquoteright} riep Caligula, {\textquoteleft}...}{bestraffendhet was alleen}{wat onvolledig}\\

\haiku{Of ik heb gedroomd,, -...}{dat ik hem volschreef ik droom}{vaak van de goden}\\

\haiku{Secundo, was hij,?}{ervan overtuigd dat hij na}{de dood zou opstaan}\\

\haiku{{\textquoteright} {\textquoteleft}Het zal prins Agrippa.}{wellicht interesseren}{met haar te praten}\\

\haiku{De beledigers.}{waren overigens al lang}{uit de weg geruimd}\\

\haiku{wilde ze niet, dan;}{zou hij er haar desnoods met}{geweld heenbrengen}\\

\haiku{Als ik Barachius,.}{niet geholpen had had ik}{het nu rustiger}\\

\haiku{Maar hoe meer ik wil,,.}{hoe meer mijn lichaam zegt dat}{er niets van inkomt}\\

\haiku{De gehele dag;}{had hij gedaan alsof hij}{alleen in huis was}\\

\haiku{de heros werd tot,;}{demon die dan toch altijd}{nog sterfelijk was}\\

\haiku{Ik nooit, en toch zegt,;}{het orakel dat ik eerder}{zal sterven dan hij}\\

\haiku{Overigens nam men.}{hem dit evenmin kwalijk als}{een rots of een wolk}\\

\haiku{Als ik je gelast,,?}{dit schip in brand te steken}{doe je het dan ook}\\

\haiku{Samenscholende.}{gasten wachtten hun beurt af}{voor het rouwbeklag}\\

\haiku{Ik ben zelfs bereid.}{om mij in uw liggende}{houding te schikken}\\

\haiku{Maar de lamp scheerde:}{over het voeteneinde en}{boetseerde een iets}\\

\haiku{Bedienden snelden,,,;}{toe eunuchen Ethiopische}{slaven een lijfarts}\\

\haiku{Zelfs de rennen en,.}{de bordelen liet hij in}{de steek of zij hem}\\

\haiku{Tenslotte heeft hij...}{het beeld een kinnebakslag}{gegeven ook nog}\\

\haiku{Het bewijs van de;}{moord zou niet gemakkelijk}{te leveren zijn}\\

\haiku{Mijn vertrouwen in.}{hem is sterk verminderd na}{deze schapebrief}\\

\haiku{Met een paar stappen,.}{was hij bij haar maar zonder}{haar aan te raken}\\

\haiku{De tribuun nam het,.}{woord toen Gajus zich reeds schuin}{achter hem bevond}\\

\haiku{Niet veel beter bracht.}{Quirinius Fannius}{Piso het eraf}\\

\haiku{Jij moet weten wie,.}{jouw mannen omkopen ook}{als ik het zelf ben}\\

\haiku{een sfinxfiguur, een,?}{obelisk tempelbogen en}{tempelpylonen}\\

\haiku{men wil Mnester, doch,.}{een grote bruine beer die}{de slaper besloop}\\

\haiku{dat dit zijn eerste,!}{kruisiging was de eerste}{die hij bijwoonde}\\

\haiku{Niet iedereen was;}{zo spoedig op de hoogte}{als Caligula}\\

\haiku{Van dit vroom bedrog.}{was hij zelf in de eerste}{plaats het slachtoffer}\\

\subsection{Uit: Verzamelde romans. Deel 49. Het schandaal der Blauwbaarden}

\haiku{- {\textquoteleft}Signor Bohlen heeft een,.}{roman geschreven die hier}{in Florence speelt}\\

\haiku{{\textquoteright} {\textquoteleft}Als Mr. Bohlen niet wil,,{\textquoteright}.}{doe {\'\i}k het zei Wilkie met}{merkbare zelfspot}\\

\haiku{waarschijnlijk zaten.}{zij in hun torens op hun}{saffraan te knagen}\\

\haiku{Met geweld dwong ik.}{mij mijn verwachtingen niet}{te hoog te spannen}\\

\haiku{Een historicus.}{zal de familie wel niet}{hebben voortgebracht}\\

\haiku{waarom bouwde men?}{eigenlijk al die torens}{in San Gimignano}\\

\haiku{Dat neemt niet weg, dat.}{ik het verhaaltje nu zelf}{wel af kan maken}\\

\haiku{Ben ik graaf Giorgio,}{ter wille ontdek ik hier}{in San Gimignano}\\

\haiku{Witte dwergen zijn,,...}{niet jaloers Signor Wilkie}{dat is toch bekend}\\

\haiku{Hij zei tenminste,:}{kennelijk blij weer Engels}{te kunnen spreken}\\

\haiku{{\textquoteright} - Met deze woorden,.}{keek ik hem strak aan maar hij}{reageerde niet}\\

\haiku{Veeleer waren de,,.}{blikken die ons bereikten}{uitgesproken nors}\\

\haiku{Maar ik vraag mij af,.}{of iemand anders het stuk}{vervalst kan hebben}\\

\haiku{{\textquoteright} Met een kort gebaar.}{veegde Lampugnani de}{tegenwerping weg}\\

\haiku{Misschien vindt u het.}{verslag van het proces en}{van de foltering}\\

\haiku{Lampugnani liet.}{geen twijfel bestaan aan zijn}{mening daaromtrent}\\

\haiku{Ik moet een berucht!}{vrouwenmoordenaar vragen}{of hij het wel is}\\

\haiku{Waarmee niet gezegd,.}{wil zijn dat hij zich op straat}{bijzonder haastte}\\

\haiku{Ik had niet nodig}{naar zijn laarzen te kijken}{om te begrijpen}\\

\haiku{Zijn rechtermouw was,,.}{met bloed bevlekt tenminste}{daar hield ik het voor}\\

\haiku{Dat heb ik u al,,.}{gezegd Signore de slaap}{had mij overmeesterd}\\

\haiku{{\textquoteleft}Uw vriend - ik neem aan, -.}{dat het uw vriend is denkt graag}{kwaad van de mensen}\\

\haiku{{\textquoteright} Na even hulpeloos,:}{om zich heengekeken te}{hebben zei de man}\\

\haiku{Wij beloofden hem.}{niet aan de politie te}{zullen verraden}\\

\haiku{{\textquoteright} stelde ik voor, {\textquoteleft}men...{\textquoteright} {\textquoteleft}?}{hoort wel van die dingenBij}{het scheren toch niet}\\

\haiku{Volgens de dokter,.}{was dit onmogelijk maar}{die wist er niets van}\\

\haiku{Het hoeft u niet de,,.}{minste moeite te kosten}{geen toelichting niets}\\

\haiku{Bel ik bij zijn klein,,.}{bouwvallig Palazzo aan}{dan geeft hij niet thuis}\\

\subsection{Uit: Verzamelde romans. Deel 24. De verminkte Apollo}

\haiku{De westelijke.}{lag reeds van de namiddag}{af in het duister}\\

\haiku{Enkele priesters.}{stapten haastig naar binnen}{en vertrokken weer}\\

\haiku{Voor het altaar van.}{Poseidon zouden mollen}{worden geofferd}\\

\haiku{- {\textquoteleft}Had zij verbanning,.}{ge\"eist men zou de straf niet}{hebben voltrokken}\\

\haiku{Vergeef mij, Aletes,:}{wij hebben allen onze}{eigen geaardheid}\\

\haiku{In het vierde jaar}{na de val van Krisa had}{hij van zijn vader}\\

\haiku{Toen het karige,.}{winterlicht hun bed bescheen}{was zij er niet meer}\\

\haiku{Voor de avond hoopten;}{de mannen haar in Delphi}{te hebben gebracht}\\

\haiku{Door een heraut met.}{een gouden skepter heeft hij}{het laten weten}\\

\haiku{Je zult niet rusten,}{voor je Hem tot God van de}{oorlog hebt gewijd}\\

\haiku{Weer rommelde het.}{en het vertrek werd doorschokt}{van kalkwit licht}\\

\haiku{Ik maak mij niet mooi,.}{voor andere mannen zelfs}{niet voor je vader}\\

\haiku{Men liet haar lopen,,.}{maar bleef haar gangen volgen}{enige dagen lang}\\

\haiku{Dat was nieuw voor ons,!}{die onderdrukking en al}{het nieuwe trekt aan}\\

\haiku{Wie waarborgt ons, dat,?!}{Apollo niet wilde dat het}{beeld gestolen werd}\\

\haiku{de tekens voor de.}{woeste en rochelende}{geluiden kent gij}\\

\haiku{Een bronzen beeld naar,,.}{Uw gelijkenis Gij weet}{het is gestolen}\\

\haiku{De eerste Pythia,.}{uit een voornaam geslacht sinds}{mensenheugenis}\\

\haiku{{\textquoteleft}Diomos, Diomos, zal (:}{verzoend worden zonder bloed}{dat betekende}\\

\haiku{Diomos vreesde het,.}{ergste en nam inderhaast}{afscheid van Leont{\'\i}on}\\

\haiku{Er liggen heel wat,.}{gegevens ter beschikking}{ook over het beeld zelf}\\

\haiku{Iemand als gij zou,...}{in staat zijn een stad uit te}{moorden met helpers}\\

\haiku{{\textquoteright} - Zij begonnen te.}{schateren en wierpen de}{armpjes in de lucht}\\

\haiku{al is het dan geen,;}{bloedvergieten het komt toch}{op onze hoofden}\\

\haiku{De laatste maal, dat,:}{ik hier vertoefde zei de}{Pythia tegen mij}\\

\haiku{Voor echte wreedheid';}{waren Kleisthenes lippen te}{vol en te goedlachs}\\

\haiku{Het vergiftigen;}{van waterleidingen is}{voortaan goddeloos}\\

\haiku{Ik noem ze Pythisch,.}{omdat ze om de vier jaar}{gehouden worden}\\

\haiku{Maar Periandros.}{wilden ze in geen geval}{in hun land hebben}\\

\haiku{De opvolger is,.}{nu die half Egyptische neef}{Hellas welgezind}\\

\haiku{Vandaar immers, dat.}{Periandros oorlog voert}{tegen Epidauros}\\

\haiku{Maar laat deze wijn,.}{uitgisten Kleisthenes drinken}{wij \'o\'ok nog wel op}\\

\haiku{{\textquoteright} {\textquoteleft}Het zijn demonen,{\textquoteright},.}{zei Gylidas en zuchtte}{ten tweede male}\\

\haiku{hij stond toch minder.}{onder haar invloed dan men}{altijd had gemeend}\\

\haiku{De wagenmenners;}{en Hippias wilden hem}{de weg versperren}\\

\haiku{voor zichzelf zag hij.}{geen verschil met meisjes van}{dezelfde leeftijd}\\

\haiku{{\textquoteright} - Hij stond op, en keek,.}{in de gele waakzame}{ogen van Gylidas}\\

\haiku{De enige die mij,!}{als een mens toespreekt en niet}{als een dwingeland}\\

\haiku{Op de drievoet zat.}{met wijdopengesperde ogen}{de blonde Pythia}\\

\haiku{Maar zou hij, door te,?}{weigeren Kleisthenes niet tot}{zijn vijand maken}\\

\haiku{Hierover zou men de,.}{Pythia kunnen raadplegen}{zei Onomakriton}\\

\haiku{En mijn vrouw kan ik,.}{niet vier jaar alleen laten}{zonder berichten}\\

\haiku{{\textquoteright} {\textquoteleft}Het meeste heb ik,{\textquoteright}.}{van mijn vader geleerd zei}{Diomos lusteloos}\\

\haiku{Zoals jij later,}{vergeten zult dat je vrouw}{je belasterde}\\

\haiku{Toch, alsof een stem,.}{het hem toegalmde wilde}{hij naar Messeni\"e}\\

\haiku{bijna steeds meende:}{men een stofwolk achter hem}{te zien opdoemen}\\

\haiku{Rechts ervan kroop de,.}{Helikon als een zwarte}{kartelige slang}\\

\haiku{Thraki\"ers zagen de:}{twee bezoekers voor het eerst}{als m\'e\'er dan slaven}\\

\haiku{zelfs ging hij zo ver:}{een hoofse toespeling te}{wagen op Achilleus}\\

\haiku{Uit een der tempels,.}{kwam jankend gegil gevolgd}{door vrouwengelach}\\

\haiku{Aletes mocht bloed doen, -...}{vloeien hij niet die tien maal}{beter doden kon}\\

\haiku{{\textquoteright} vroeg hij, toen de straat.}{zich verbreedde en sterker}{begon te stijgen}\\

\haiku{Het betekent, dat,}{het nabootsingen zijn van}{Dionysostempels}\\

\haiku{Hoe beijverde.}{Aletes zich om het hem naar}{de zin te maken}\\

\haiku{voor hen uit, ergens,.}{in de verte begon de}{hoofdweg naar Korinthe}\\

\haiku{Maskers op lange.}{staken schommelden nader}{uit de tempelstraat}\\

\haiku{Het was of er een,.}{zacht veelkleurig vocht over die}{oogbollen heenliep}\\

\haiku{{\textquoteleft}Blijf niet hier, moeder,{\textquoteright}, {\textquoteleft},}{zei de aanvoerder zachthoort}{de koning ervan}\\

\haiku{{\textquoteright} - Hij reikte Aletes,,,.}{de hand daarop na enige}{weifeling Diomos}\\

\haiku{{\textquoteright} snauwde Diomos, en, {\textquoteleft}}{greep naar de Delphische penning}{onder zijn kleren}\\

\haiku{En dat Delphi ons,.}{heeft uitgezonden blijkt uit}{onze papieren}\\

\haiku{- {\textquoteleft}Bijna iedereen,.}{houdt er rekening mee dat}{ik een grijsaard ben}\\

\haiku{Alsof Herakles!}{jullie Apollo niet allang}{doodgeslagen had}\\

\haiku{Met een waakzame.}{blik op zijn vriend bracht Aletes}{de hand aan het zwaard}\\

\haiku{Nog meer treden, nog,;}{meer voorbijgangers die geen}{acht op hen sloegen}\\

\haiku{Wij hebben er al,.}{onze hoop op gevestigd}{dat Lykophron nog leeft}\\

\haiku{{\textquoteleft}Zegt u die dingen,,{\textquoteright}, {\textquoteleft}}{toch niet jonge heer klaagde}{de oudere vrouw}\\

\haiku{{\textquoteright} vroeg zij aan Diomos, {\textquoteleft},...}{ik heet Mestra ik ben hier}{enkele maanden}\\

\haiku{Ik geloof, dat dit...{\textquoteright} {\textquoteleft}?}{beeld bestaat en opgespoord}{Waar komt gij vandaan}\\

\haiku{Hier was een man, die.}{zich beklaagde en daar zijn}{redenen voor had}\\

\haiku{ik doe een beroep.}{op het door Zeus beschermde}{heilige gastrecht}\\

\haiku{Zijn pijnlijk hoofd en.}{het bloed tussen zijn lippen}{gaven het antwoord}\\

\haiku{de enige straffen,.}{waren honger en ziekte}{zelden geseling}\\

\haiku{Iedereen mocht met;}{hem worstelen om de zweep}{te bemachtigen}\\

\haiku{Krekels en vogels,.}{deden zich horen en het}{gemurmel der stad}\\

\haiku{Daar zij de rang van,.}{beambte bekleedde mocht}{men haar niet doodslaan}\\

\haiku{En wie een gedicht,,.}{opzei zei een God op en}{geloofde in Hem}\\

\haiku{Haar gezicht had hij,.}{in zijn vuist kunnen nemen}{als een vogelei}\\

\haiku{Beter nog was het,,;}{wanneer Aletes bleef en aan}{de buitenkant sliep}\\

\haiku{Na  de koude.}{werden geen gevangenen}{meer binnengebracht}\\

\haiku{Het werd iets lichter,.}{de Kithairon hervatte}{zijn duister gegloei}\\

\haiku{{\textquoteright} {\textquoteleft}Ik ben bang zonder,.}{jou je weet niet hoe ik je}{vereer en liefheb}\\

\haiku{gedronken zou zij,.}{wel hebben maar de wijn was}{zuiver in haar bloed}\\

\haiku{ondraaglijke smart.}{werd weggeschoven achter}{bloedrode woede}\\

\haiku{Het duurde een tijd,.}{voor hij het zag liggen rood}{tot aan het gevest}\\

\haiku{Eens, bij fakkellicht,...}{had de vlek zich onzichtbaar}{weten te houden}\\

\haiku{Bij de eerste stap,.}{die hij deed moest hij zich aan}{een struik vasthouden}\\

\haiku{Het meisje keek naar,.}{het lichaam van haar vader}{en dan weer naar hem}\\

\haiku{E\'enmaal, trappend,:}{op vlijmscherpe distels liet}{hij het beeld vallen}\\

\haiku{Het beeld trok aan zijn,.}{schouder zodat hij vaak van}{hand verwisselde}\\

\haiku{{\textquoteleft}Ga naar de priester,,.}{die toezicht houdt of die in}{de tempel vertoeft}\\

\haiku{De zon gleed rond in,.}{zijn pas uitgebroken zweet}{beet en blakerde}\\

\haiku{Trixas was eigenlijk,...}{nog jong misschien had hij maar}{twee grijze haren}\\

\haiku{mijn grootste schuld, want...}{ik had hem niet uit het oog}{mogen verliezen}\\

\haiku{Het kon zijn, dat men.}{hem nooit goed geleerd had wie}{en wat Apollo was}\\

\haiku{Stellig en zeker.}{was Agetoros doordrongen}{van deze waarheid}\\

\subsection{Uit: Verzamelde romans. Deel 48. De h\^otelier doet niet meer mee}

\haiku{Hij bleef recht voor zich,.}{uitkijken en strompelde}{met zijn stok verder}\\

\haiku{- was de enige die.}{ronduit blijk scheen te geven}{van het tegendeel}\\

\haiku{Niets is zo goed voor,.}{het optreden de maintien}{van jonge mensen}\\

\haiku{{\textquoteright} {\textquoteleft}Dat is hij juist n{\'\i}et,{\textquoteright}, {\textquoteleft}...}{glimlachte ikmaar hij kan}{best gezegd hebben}\\

\haiku{jij kent hem beter,.}{dan ik jij zult het misschien}{geen leugens noemen}\\

\haiku{Maar ik denk nu aan,.}{de opstand van Didier van}{vier jaar geleden}\\

\haiku{Ik ben zelfs bereid,.}{te verklaren dat ik mij}{vergist kan hebben}\\

\haiku{{\textquoteleft}Monsieur Trublet,.}{had mij niets opgedragen}{Mademoiselle}\\

\haiku{Een verklaring van {\textquoteleft}{\textquoteright};}{mijnspionnage kon ik}{hem moeilijk geven}\\

\haiku{{\textquoteleft}Ik zal Lamoignon,.}{nu naar u toesturen wij}{blijven nog maar kort}\\

\haiku{twee ruimten op de,:}{tweede verdieping die als}{kantoor dienst deden}\\

\haiku{ik had geen lust om;}{mijn nieuwsgierigheid aan de}{kaak gesteld te zien}\\

\haiku{Had ik er flink wat,:}{bij elkaar dan kon ik hem}{ermee overvallen}\\

\haiku{maar de haastige:}{beweging van  haar hand}{was mij niet ontgaan}\\

\haiku{Daarentegen keek,:}{hij terstond op toen zij een}{half uur later vroeg}\\

\haiku{hij is overigens,...}{in het geheel geen graaf ik}{had ge{\"\i}nformeerd}\\

\haiku{Maar dat is niet zo,,.}{Duitsertje het is een veel}{belangrijker zaak}\\

\haiku{ook kon Marie dit,.}{all\'e\'en hebben gedaan of}{Andilly alleen}\\

\haiku{Hij zat daar altijd,.}{om deze tijd wanneer hij}{niet uitgegaan was}\\

\haiku{{\textquoteleft}Eruit,{\textquoteright} waarop de,.}{Zuidfransman naar de gang liep}{en zo naar de trap}\\

\haiku{De keizer moest niets,.}{van hen hebben hij heeft er}{ook heel wat gestraft}\\

\haiku{Ik zou bijvoorbeeld...}{niets ten nadele van uw}{oom willen zeggen}\\

\haiku{U geniet hier de......{\textquoteright} {\textquoteleft}}{gastvrijheid van de Franse}{staat van de koning}\\

\haiku{{\textquoteright} {\textquoteleft}Monsieur Trublet.}{verleent onderdak aan zijn}{eigen reizigers}\\

\haiku{{\textquoteright} {\textquoteleft}Wanneer neemt u een?}{Bonapartistische}{opstand w\'el ernstig}\\

\haiku{zo ja, dan was er.}{sprake van plaatselijke}{eigengereidheid}\\

\haiku{Mademoiselle:}{Nathalie veroorloofde}{zich de opmerking}\\

\haiku{Ik was nu niet bang,,}{meer en ik zei dat beren}{niet van gezouten}\\

\haiku{en ze wisten, dat.}{ik mijn pistool momenteel}{niet gebruiken kon}\\

\haiku{{\textquoteright} {\textquoteleft}Wij hebben \'o\'ok last,{\textquoteright}.}{van het spit werd er uit de}{ingang geroepen}\\

\haiku{{\textquoteright} vroeg Mionnet, met.}{een bedenkelijke plooi}{in zijn hoog voorhoofd}\\

\haiku{Het kan zijn, dat zij,,.}{met hun dialect de Franse}{taal slecht verstonden}\\

\haiku{Ik kreeg de indruk,.}{dat hij zich niet meer tegen}{mij zou verzetten}\\

\haiku{Met het pistool in.}{de hand stond hij reeds bij de}{pakken en zakken}\\

\haiku{Als je honger hebt,,.}{ga je maar werken bij de}{fortificaties}\\

\haiku{Ik bedoel dus niet.}{zozeer je meningen als}{Erlanger student}\\

\haiku{Verder had ik als;}{alibi het huwelijk van}{mijn oudste dochter}\\

\haiku{En dat zette zich.}{voort in alle latere}{gebeurtenissen}\\

\haiku{{\textquoteleft}Maar als verrader...{\textquoteright} {\textquoteleft}.}{Een impulsief man als Ney}{{\'\i}s geen verrader}\\

\haiku{Hij zal dankbaar zijn.}{in Frankrijk te vertoeven}{z\'onder politiek}\\

\haiku{Daarin was maar heel,.}{weinig mislukt en stoute}{stukjes in overvloed}\\

\haiku{{\textquoteleft}Mijn gevoelens voor.}{de keizer kunt u buiten}{beschouwing laten}\\

\haiku{En er is nu geen.}{generaal Ney meer om naar}{hen over te lopen}\\

\haiku{Ik ben niet meer dan.}{de geldschieter en de chef}{de r\'eception}\\

\haiku{Het is zelfs de vraag,...}{of men mij bij mislukking}{zal kunnen straffen}\\

\haiku{Dumoulin ook,.}{over hem hebben we meen ik}{al eens gesproken}\\

\haiku{Hiermee wil ik niet,;}{zeggen dat hij weer dingen}{voor mij geheim hield}\\

\haiku{geld geven, en zijn.}{nest in de Alpen voor de}{adelaar inrichten}\\

\haiku{Uw dochters kunnen...}{heel goed eens de handen uit}{de mouwen steken}\\

\haiku{Zwaar, dat kan ik niet...{\textquoteright} {\textquoteleft}.}{zeggenOmdat zij in die}{trommel was gezakt}\\

\haiku{Natuurlijk liep hij,,.}{met al die wapens naar de}{vrijgekomen strook}\\

\haiku{Maar de strook bleek niet,?}{geheel vrij te zijn want wat}{zag hij daar opeens}\\

\haiku{Ik was dan wel een,,}{Duitser maar ook zo goed als}{een Elzasser etc.}\\

\haiku{{\textquoteleft}K\'an Johnston het,,?}{niet of wil hij niet of krijgt}{hij niet genoeg geld}\\

\haiku{Twistgesprekken als;}{tussen Mionnet en mij}{kwamen niet meer voor}\\

\haiku{Mijn voorstel om de.}{taak van hem over te nemen}{werd afgeslagen}\\

\haiku{verder dan dat scheen.}{men het in Amerika niet}{gebracht te hebben}\\

\haiku{zij konden heel goed:}{voor grapjes doorgaan om de}{tijd te verdrijven}\\

\haiku{Hij was al weer twee,.}{keer bij hen geweest zonder}{naar mij te vragen}\\

\haiku{om opschudding te.}{voorkomen was het bericht}{achtergehouden}\\

\haiku{Monsieur Trublet,.}{wist ervan maar sloot zich op}{in zijn kantoor}\\

\haiku{{\textquoteright} Johnston bedacht,.}{de vragen zelf al stond hij}{open voor suggesties}\\

\haiku{{\textquoteright} Het woord {\textquoteleft}zigeuners{\textquoteright} {\textquoteleft}{\textquoteright},.}{was vervangen doorgypsies}{het Engelse woord}\\

\haiku{Nadat zijn taak hem,:}{weer had opge\"eist kwam het}{antwoord vlug genoeg}\\

\haiku{Johnston had het,,.}{maar dan geheel toevallig}{bij het rechte eind}\\

\haiku{{\textquoteleft}Geloofde hij in?}{dit nieuwe bericht over de}{dood van de keizer}\\

\haiku{Monsieur Trublets{\textquoteright}.}{gastvrijheid kent u. Allen}{knikten instemmend}\\

\haiku{Hij moest toch heel goed,.}{weten dat de heren niet}{d\'urfden uit te gaan}\\

\haiku{{\textquoteright} Met deze woorden,.}{liep hij naar een stoel waarin}{hij zich liet vallen}\\

\haiku{Hij sloeg de handen,.}{voor het gezicht en barstte}{in snikken uit}\\

\subsection{Uit: Verzamelde romans. Deel 52. Het proces van meester Eckhart}

\haiku{Zij werden meer op,}{de vingers getikt konden}{minder ongemoeid}\\

\haiku{En als ik het wist,.}{was het maar de vraag of ik}{het zou begrijpen}\\

\haiku{Maar misschien is men,.}{in de war misschien vliegt de}{Duivel naar hem toe}\\

\haiku{Het wachten is nu.}{allereerst op de stappen}{van de aartsbisschop}\\

\haiku{{\textquoteright} Deze overweging,:}{scheen hem iets hoopvoller te}{stemmen en hij zei}\\

\haiku{{\textquoteleft}Mijn bezoek hier op.}{het Domplein kan ik u}{beter toelichten}\\

\haiku{Hij las zonder bril,.}{en met het papier niet al}{te dicht bij de ogen}\\

\haiku{En niemand die het,.}{ooit zag aankomen zelfs op}{de laatste dag niet}\\

\haiku{{\textquoteright} {\textquoteleft}Misschien is er in,.}{zijn ziel een kleine uithoek}{die erop rekent}\\

\haiku{Veeleer duidden zij.}{op een afwachten van het}{gunstigste moment}\\

\haiku{Vermoedelijk had.}{hij haast om alles achter}{de rug te hebben}\\

\haiku{Reyner heeft zes,,.}{klerken voor verschillende}{zaken en zaakjes}\\

\haiku{Mijn zonden, meende,}{ik lagen op het terrein}{der formulering}\\

\haiku{Op dit schreien keek.}{Konrad van Halberstadt voor}{de tweede maal op}\\

\haiku{In de eerste plaats,,.}{zoals u wel begrijpen}{zult de herroeping}\\

\haiku{Ik heb alleen met,.}{de Paus te maken en ik}{ga naar de Paus toe}\\

\subsection{Uit: Het vijfde zegel}

\haiku{Niemand had er aan.}{gedacht hem van ketterij}{te beschuldigen}\\

\haiku{De aanvoerder, wiens,.}{neus afgeschoten was sprak}{met een licht accent}\\

\haiku{Nu het paleis zo,.}{dichtbij was durfde hij er}{wel naar te kijken}\\

\haiku{{\textquoteleft}Exeat aula, qui vult{\textquoteright}, -:}{esse pius aangevuld}{nog door Ovidius}\\

\haiku{Het leek de Griek niet,:}{de onaangenaamste van}{de vijf die dit zei}\\

\haiku{{\textquoteright} {\textquoteleft}De kleuren zijn te,!}{licht en te vrolijk er is}{te veel blauw en geel}\\

\haiku{De moeilijkheden;}{waren niet zo groot geweest}{als hij verwacht had}\\

\haiku{een overredende,.}{woordenstroom zonder enige}{belemmering nu}\\

\haiku{er zeker van kon,.}{zijn dat de wagen met het}{schilderij klaar stond}\\

\haiku{Vijf dagen later.}{kwam een brief uit Aranjuez}{zekerheid brengen}\\

\haiku{Nog verwonderde,.}{het hem dat ze zich later}{nooit gewroken had}\\

\haiku{In de namiddag}{van de 16e Mei verhief hij}{zich uit die vadsig}\\

\haiku{Een ontmoeting, zeer,;}{toevallig met de vrouw met}{de kanthanden}\\

\haiku{, en onmiddellijk,.}{schoof zich iets achter hem open}{onhoorbaar glijdend}\\

\haiku{Want ook de botste uit:}{het gezelschap besefte}{waar het hier om ging}\\

\haiku{en Taddeus, dat,!...}{was Scorpio de vinger in}{eigen borst borend}\\

\haiku{Het spijt mij, dat ik, ...{\textquoteright} {\textquoteleft}?!}{u stoor zo laatMaar hoe komt}{u in Toledo}\\

\haiku{{\textquoteright} - De man moest Greco's.}{vraag van de vorige avond}{onthouden hebben}\\

\haiku{Je zult zien, Juan,,,!}{hij jaagt hem weg hij ontvangt}{hem niet de trotsaard}\\

\haiku{{\textquoteleft}Jij Preboste zorgt,.}{ervoor dat ik het eerste}{uur niet gestoord word}\\

\haiku{Hij gaf Miguel.}{instructies en volgde even}{later met de gast}\\

\haiku{(Het was kenmerkend,.}{voor zijn openluchtnatuur dat}{hij de deur oversloeg}\\

\haiku{{\textquoteleft}Ik wens u geluk,!}{met de verjaardag van de}{koning Don Martin}\\

\haiku{het huis is tegen.}{de heuvel opgebouwd door}{die oude Moren}\\

\haiku{De laatste tijd denk.}{ik meer aan Creta terug}{dan ooit te voren}\\

\haiku{Deze achtertrap,.}{liep langs provisiekasten}{die Greco afsloot}\\

\haiku{{\textquoteright} {\textquoteleft}Dat wist ik niet,{\textquoteright} zei,.}{Esquerrer de schilder enigszins}{bezorgd aankijkend}\\

\haiku{Eerst op het plein voor.}{het stadhuis wachtte hem een}{groter machtsvertoon}\\

\haiku{{\textquoteleft}Noem hem geen Jood, of.}{het Heilig Officium}{bemoeit er zich mee}\\

\haiku{Het bestaat evenmin!}{als tussen Maria en een}{andere Maria}\\

\haiku{zolang men jullie,!}{kan uitschilderen gebeurt}{er niets van die aard}\\

\haiku{Een opgejaagde.}{vleermuis beschreef trillende}{kringen en verdween}\\

\haiku{Tot alles zou hij.}{bereid zijn geweest in ruil}{voor m\'e\'er verhalen}\\

\haiku{en ieder huis, dat,.}{niet op de gangen uitkwam}{was een muizenval}\\

\haiku{In deze maanden.}{had hij meer geschilderd dan}{anders in jaren}\\

\haiku{Hij nam zich voor haar.}{steeds te blijven eren om haar}{onderworpenheid}\\

\haiku{Er was geen sprake,}{van verdienste laat staan van}{opofferingen}\\

\haiku{Onder de vrienden}{die hij zich in Rome had}{gemaakt was er \'e\'en}\\

\haiku{men zocht dagen uit,;}{dat de anderen ziek of}{verhinderd waren}\\

\haiku{Die hij misschien zal!}{mogen helpen martelen}{in een zwembroekje}\\

\haiku{Greco ving dingen;}{op die zijn factotum hem}{steeds verzwegen had}\\

\haiku{Dan kwam hij met een,;}{kreet te voorschijn en het spel}{was afgelopen}\\

\haiku{tussen zijn lippen.}{vertoonde zich het puntje}{van een bleke tong}\\

\haiku{{\textquoteleft}Ik ben te bang voor, ...{\textquoteright} {\textquoteleft},}{hem en ook dat jijBijten}{gaat z\'eker te ver}\\

\haiku{{\textquoteright} {\textquoteleft}Dat is ver gezocht,{\textquoteright}, {\textquoteleft}?}{zei Grecowat heeft Ayala}{daarmee te maken}\\

\haiku{in het atelier, waar,.}{Greco ze bewaarde was}{in lang niet gestookt}\\

\haiku{In het donker, want, ...}{ook mijn lamp was uitgegaan}{was hij onvindbaar}\\

\haiku{Maar wat hij nu doen.}{ging was zo mogelijk nog}{raadselachtiger}\\

\haiku{Miguel wachtte.}{met iets te zeggen tot zijn}{meester voor hem stond}\\

\haiku{{\textquoteright} {\textquoteleft}Ik meende, dat ik,!}{hier als getuige zat niet}{als beschuldigde}\\

\haiku{Ik diletteer in,.}{kunsttheorie\"en en u}{theologiseert}\\

\haiku{Midden in de nacht,.}{zocht hij Ger\'onima op voor}{het eerst na maanden}\\

\haiku{Slechts aarzelend was,.}{men begonnen men durfde}{niet goed voor elkaar}\\

\haiku{vrouwen tilden hun.}{kinderen op om te zien}{wat er gebeurde}\\

\haiku{Hoe weinig misbruik,!}{had Esquerrer daarvan gemaakt}{van  dit laatste}\\

\subsection{Uit: Verzamelde romans. Deel 5. Het vijfde zegel}

\haiku{hij liep, begerig.}{toe om  het berouw in}{ontvangst te nemen}\\

\haiku{Niemand had er aan.}{gedacht hem van ketterij}{te beschuldigen}\\

\haiku{De aanvoerder, wiens,.}{neus afgeschoten was sprak}{met een licht accent}\\

\haiku{Nu het paleis zo,.}{dichtbij was durfde hij er}{wel naar te kijken}\\

\haiku{{\textquoteleft}Exeat aula, qui vult{\textquoteright}, -:}{esse pius aangevuld}{nog door Ovidius}\\

\haiku{Het leek de Griek niet,:}{de onaangenaamste van}{de vijf die dit zei}\\

\haiku{{\textquoteright} {\textquoteleft}De kleuren zijn te,!}{licht en te vrolijk er is}{te veel blauw en geel}\\

\haiku{De moeilijkheden;}{waren niet zo groot geweest}{als hij verwacht had}\\

\haiku{een overredende,.}{woordenstroom zonder enige}{belemmering nu}\\

\haiku{er zeker van kon,.}{zijn dat de wagen met het}{schilderij klaar stond}\\

\haiku{{\textquoteleft}...een tijd en tijden,...}{en een halve tijd buiten}{het gezicht der slang}\\

\haiku{Nog verwonderde,.}{het hem dat ze zich later}{nooit gewroken had}\\

\haiku{Een ontmoeting, zeer,;}{toevallig met de vrouw met}{de kanthanden}\\

\haiku{, en onmiddellijk,.}{schoof zich iets achter hem open}{onhoorbaar glijdend}\\

\haiku{Want ook de botste uit:}{het gezelschap besefte}{waar het hier om ging}\\

\haiku{en Taddeus, dat,!...}{was Scorpio de vinger in}{eigen borst borend}\\

\haiku{Het spijt mij, dat ik,...{\textquoteright} {\textquoteleft}?!}{u stoor zo laatMaar hoe komt}{u in Toledo}\\

\haiku{{\textquoteright} - De man moest Greco's.}{vraag van de vorige avond}{onthouden hebben}\\

\haiku{Je zult zien, Juan,,,!}{hij jaagt hem weg hij ontvangt}{hem niet de trotsaard}\\

\haiku{{\textquoteleft}Jij Preboste zorgt,.}{ervoor dat ik het eerste}{uur niet gestoord word}\\

\haiku{Hij gaf Miguel.}{instructies en volgde even}{later met de gast}\\

\haiku{(Het was kenmerkend,.}{voor zijn openluchtnatuur dat}{hij de deur oversloeg}\\

\haiku{het huis is tegen.}{de heuvel opgebouwd door}{die oude Moren}\\

\haiku{De laatste tijd denk.}{ik meer aan Creta terug}{dan ooit te voren}\\

\haiku{Deze achtertrap,.}{liep langs provisiekasten}{die Greco afsloot}\\

\haiku{{\textquoteright} {\textquoteleft}Dat wist ik niet,{\textquoteright} zei,.}{Esquerrer de schilder enigszins}{bezorgd aankijkend}\\

\haiku{Eerst op het plein voor.}{het stadhuis wachtte hem een}{groter machtsvertoon}\\

\haiku{{\textquoteleft}Noem hem geen Jood, of.}{het Heilig Officium}{bemoeit er zich mee}\\

\haiku{Het bestaat evenmin!}{als tussen Maria en een}{andere Maria}\\

\haiku{zolang men jullie,!}{kan uitschilderen gebeurt}{er niets van dien aard}\\

\haiku{Een opgejaagde.}{vleermuis beschreef trillende}{kringen en verdween}\\

\haiku{Tot alles zou hij.}{bereid zijn geweest in ruil}{voor m\'e\'er verhalen}\\

\haiku{een ieder huis, dat,.}{niet op de gangen uitkwam}{was een muizeval}\\

\haiku{In deze maanden.}{had hij meer geschilderd dan}{anders in jaren}\\

\haiku{Hij nam zich voor haar.}{steeds te blijven eren om haar}{onderworpenheid}\\

\haiku{Er was geen sprake,}{van verdienste laat staan van}{opofferingen}\\

\haiku{de beoefening;}{ervan gereglementeerd}{door de nieuwe Paus}\\

\haiku{Onder de vrienden}{die hij zich in Rome had}{gemaakt was er \'e\'en}\\

\haiku{Veel was er voor een,.}{ontbinding te zeggen er}{tegen sprak ook veel}\\

\haiku{men zocht dagen uit,;}{dat de anderen ziek of}{verhinderd waren}\\

\haiku{Die hij misschien zal!}{mogen helpen martelen}{in een zwembroekje}\\

\haiku{Greco ving dingen;}{op die zijn factotum hem}{steeds verzwegen had}\\

\haiku{Dan kwam hij met een,;}{kreet te voorschijn en het spel}{was afgelopen}\\

\haiku{tussen zijn lippen.}{vertoonde zich het puntje}{van een bleke tong}\\

\haiku{{\textquoteleft}Ik ben te bang voor,...{\textquoteright} {\textquoteleft},}{hem en ook dat jijBijten}{gaat z\'eker te ver}\\

\haiku{{\textquoteright} {\textquoteleft}Dat is ver gezocht,{\textquoteright}, {\textquoteleft}?}{zei Grecowat heeft Ayala}{daarmee te maken}\\

\haiku{in het atelier, waar,.}{Greco ze bewaarde was}{in lang niet gestookt}\\

\haiku{In het donker, want,...}{ook mijn lamp was uitgegaan}{was hij onvindbaar}\\

\haiku{Maar wat hij nu doen.}{ging was zo mogelijk nog}{raadselachtiger}\\

\haiku{Miguel wachtte.}{met iets te zeggen tot zijn}{meester voor hem stond}\\

\haiku{{\textquoteright} {\textquoteleft}Ik meende, dat ik,!}{hier als getuige zat niet}{als beschuldigde}\\

\haiku{Diego was de enige,.}{die hem die Maandagochtend}{het vlees had zien eten}\\

\haiku{Ik diletteer in,.}{kunsttheorie\"en en u}{theologiseert}\\

\haiku{Midden in de nacht,.}{zocht hij Ger\'onima op voor}{het eerst na maanden}\\

\haiku{Slechts aarzelend was,.}{men begonnen men durfde}{niet goed voor elkaar}\\

\haiku{vrouwen tilden hun.}{kinderen op om te zien}{wat er gebeurde}\\

\haiku{16 Sola fides -.}{justificat Alleen het}{geloof rechtvaardigt}\\

\haiku{41 Patio de los -.}{Evangelistas Binnenplaats}{der Evangelisten}\\

\haiku{120 Index Librorum -.}{Prohibitorum Lijst van}{verboden boeken}\\

\haiku{226 Compa\~n{\'\i}a de -,.}{Jes\'us Leger van Jezus}{de Jezu{\"\i}eten}\\

\haiku{256 Convento de -.}{la Concepci\'on Klooster van}{de Ontvangenis}\\

\haiku{258 Panthe{\"\i}sme -,.}{De leer dat God alles is}{zonder onderscheid}\\

\haiku{304 Generale -.}{biecht Biecht van alle zonden}{en overtredingen}\\

\haiku{317 Virgen de la -.}{Misericordia Maagd van}{de Barmhartigheid}\\

\subsection{Uit: Verzamelde romans. Deel 22. De vijf roeiers}

\haiku{Het spiegeltje had,.}{hij al eerder van de hand}{gedaan gewaagd vroeg}\\

\haiku{En het meisje was,.}{bleek en droefgeestig dus zat}{het ook in het hart}\\

\haiku{{\textquoteright} - Scherp lette hij op.}{de gelaatsuitdrukking van}{de seminarist}\\

\haiku{Maar die is \'ook van,.}{de overkant en hij heeft een}{hard leven geleid}\\

\haiku{met die voetjes, en die,.}{stenen hals met barsten en}{het losse hoofdje}\\

\haiku{{\textquoteright} - Geleidelijk aan,,.}{half buiten zijn wil had hij}{het hoofd gebogen}\\

\haiku{Een koopman, wie ik,.}{geld heb gegeven omdat}{hij slecht gekleed was}\\

\haiku{Hij zal je vragen.}{waarom je mijn bezorgdheid}{niet wegnemen wilt}\\

\haiku{{\textquoteleft}Ik heb nooit iets bij.}{je gemerkt van voorkeur voor}{die rechtenstudie}\\

\haiku{Hoe donker was het,.}{stadje wanneer men er van}{bovenaf inkeek}\\

\haiku{Schone donkere,,.}{kleren een schoon hemd en zijn}{haar zonder schubben}\\

\haiku{{\textquoteleft}Ik dank je, lieve,.}{Moyna voor die bloeiende}{brem op mijn kussen}\\

\haiku{Een man van eer kon.}{niet doodgaan in een plaatsje}{als Lomanagh}\\

\haiku{{\textquoteright} {\textquoteleft}En ik wil hier weg,{\textquoteright},:}{hield de oude O'Flanagan aan}{en tegen Maurice}\\

\haiku{jou, dan zou ik van,.}{ze verwachten dat ze mij}{de strot afsneden}\\

\haiku{{\textquoteright} Op dit oordeel was.}{de marskramer volledig}{voorbereid geweest}\\

\haiku{Pat praatte genoeg,?}{tegen hem en zou hij dit}{verzwegen hebben}\\

\haiku{zelfs een gesprek met.}{Vader Kearny had hem}{niet wijzer gemaakt}\\

\haiku{Van die tijd stamde,,.}{zijn angst voor aanrakingen}{strelingen blikken}\\

\haiku{De twee meisjes, die,.}{hem het leven vergalden}{waren althans stil}\\

\haiku{met zijn neus, lippen,,,...}{en tong met zijn voedsel met}{de regen de zee}\\

\haiku{{\textquoteleft}Innisbavan,{\textquoteright} zei,,}{Shaun met een verlegen blik}{op John van wie}\\

\haiku{Hij begreep, dat de.}{late wandelaar het erf}{was opgelopen}\\

\haiku{Houdt u een oogje...{\textquoteright} -.}{in het zeil Hij hikte en}{zijn gezicht vertrok}\\

\haiku{Zijn vaders hand was,.}{de laatste geweest die hulp}{had kunnen bieden}\\

\haiku{En meteen was er.}{iets van de feestelijkheid}{verloren gegaan}\\

\haiku{Ik werk hier harder.}{dan iemand zo vet als jij}{zich kan voorstellen}\\

\haiku{Hij zag de nacht als.}{een door zingende dieren}{belaagd paradijs}\\

\haiku{De Fenians aan de,:}{overkant loeren op hem en}{ik houd ze tegen}\\

\haiku{Maar, bij God, ik doe!}{wat ik doe en wat de stem}{van God mij ingeeft}\\

\haiku{Gebiedend snauwde:}{hij de anderen toe zijn}{voorbeeld te volgen}\\

\haiku{Hij had een beetje,.}{medelijden met Moyna}{en hij hield van haar}\\

\haiku{En dan waren ze,.}{nog een beetje dronken ook}{behalve Moyna}\\

\haiku{ik haal ze uit de,.}{zee de seminaria en}{de paardestallen}\\

\haiku{Hoewel, als ik die...}{Eileen van jullie eens}{onder het mes nam}\\

\haiku{Dat kon van boosheid,;}{zijn Maurice's  laatste uur}{kon geslagen zijn}\\

\haiku{{\textquoteleft}Jongens, als het hard,.}{gaat waaien vergeet dan de}{arme Moyna niet}\\

\haiku{Wat haar bijzonder.}{hinderde was dat zij zich}{driftig had gemaakt}\\

\haiku{Een tijdlang stond hij.}{te luisteren naar een ver}{verwijderd gesnurk}\\

\haiku{in het testament,,}{stond alles zei hij mooier}{dan zelfs de pastoor}\\

\haiku{Wie heeft toen een van?}{de beulen een steen onder}{de neus gehouden}\\

\haiku{Een van de beste,,.}{zonen van Erin heren}{en een hart van goud}\\

\haiku{Jullie weet m\'eer van,.}{die vent of \'een van jullie}{weet meer van die vent}\\

\haiku{Ik,{\textquoteright} zei John, {\textquoteleft}en,.}{ik kreeg de indruk dat hij}{een Fenian was}\\

\haiku{Dit is overigens.}{niets vergeleken bij dat}{draaiertje van mij}\\

\haiku{{\textquoteright} {\textquoteleft}Als je er meer hebt,,.}{dan betekent dat dat je}{ons hebt bedrogen}\\

\haiku{De beweging had.}{hij volvoerd alsof het zijn}{dagelijks werk was}\\

\haiku{Shaun slaakte een gil,;}{en kromp hijgend ineen voor}{wat er volgen kon}\\

\haiku{een stille vrouw, wier,.}{ogen hij had ge\"erfd en de}{pijn gold zijn moeder}\\

\haiku{hij verwachtte het,,,.}{tweede houtje diep in zijn}{bloed zijn vlees zijn ziel}\\

\haiku{{\textquoteright} {\textquoteleft}Wij allemaal...{\textquoteright} {\textquoteleft}Kom,,,?}{Shaun je zei toch dat Cork een}{meid in het hoofd had}\\

\haiku{Hier was tenminste,...}{een indruk zoals Jimmy}{van hem verlangde}\\

\haiku{Was Conic soms al,?}{dood door die vreselijke}{slagen op zijn hoofd}\\

\haiku{{\textquoteright} riep John, die met, {\textquoteleft}.}{Shaun gearmd liepik stel voor}{wat aan te plakken}\\

\haiku{Het greep bij de keel,,.}{het gaf wel kracht maar men werd}{er ook doodsbang van}\\

\haiku{er kon immers van;}{alles gebeuren in een}{bos bij volle maan}\\

\haiku{Zijn stem klonk nog zwak,,:}{maar sarcastich genoeg toen}{hij naar voren riep}\\

\haiku{{\textquoteleft}H\'e, jullie eksters,?}{daar gaan we nu nog ruiten}{insmijten of niet}\\

\haiku{Een lichtende vorm.}{schoot omhoog en verdween in}{het gebladerte}\\

\haiku{Moet ik soms weken,?...}{en maanden tegen planken}{aankijken bij God}\\

\haiku{Aan het begin van.}{het voorterras zie je de}{baai voor je liggen}\\

\haiku{Snauwend gelastte.}{Keane de andere}{vier hem te volgen}\\

\haiku{Dat maakt een slechte,}{indruk en bij Mr. Coyne}{sta je in de pas.}\\

\haiku{{\textquoteright} {\textquoteleft}Paramentiek is,{\textquoteright}, {\textquoteleft}}{nooit mijn sterkste punt geweest}{zei John ironisch}\\

\haiku{die wisten, dat wij,.}{er waren en ze hadden}{hun wapens niet meer}\\

\haiku{Ja, de Fransen zijn,.}{geen dom volk zoals ik ze}{heb leren kennen}\\

\haiku{Maar beste jongen,.}{ik ben nog niet anders dan}{kwaad op je geweest}\\

\haiku{Wie waarborgde, dat,?}{hij de bruiloft overleven}{zou dat alleen al}\\

\haiku{Het is beroerd voor,,.}{je maar je mag me doodslaan}{als het niet waar is}\\

\haiku{{\textquoteright} {\textquoteleft}Keane...{\textquoteright} {\textquoteleft}Als je,.}{langer gebleven was had}{je het zelf gezien}\\

\haiku{d\'at juist maakte het.}{geluid zo verschrikkelijk}{om aan te horen}\\

\haiku{{\textquoteright} {\textquoteleft}Ik denk niet meer aan,{\textquoteright}, {\textquoteleft}:}{die teef zei Conic koelen}{h{\'\i}j heeft zijn portie}\\

\haiku{jouw moeder heeft er,.}{bepaald niet voor gezorgd dat}{jij de grote kreeg}\\

\haiku{Ze was toen nog te.}{jong om overlast te hebben}{van de soldaten}\\

\haiku{{\textquoteright} Keane dacht na. - {\textquoteleft}?}{En geen moeite gedaan om}{hem te ontzetten}\\

\haiku{{\textquoteright} {\textquoteleft}Waar ik u voor houd,{\textquoteright}, {\textquoteleft}.}{zei Keaneis van niet}{het minste belang}\\

\haiku{en dat jullie me,.}{niet gezegd hebben dat er}{een moord was gepleegd}\\

\haiku{Ze hadden zwarte,.}{maskers voor de moordenaars}{van Mr. Coyne ook}\\

\haiku{Pat liep naar het raam.}{om de spijker nog eens in}{ogenschouw te nemen}\\

\haiku{Moyna Donovan,.}{kan getuigen dat wij niet}{op moord uit waren}\\

\haiku{maar ik wil alleen,...}{maar zeggen dat God er niet}{om verlegen zit}\\

\haiku{Onder de plak van.}{zo'n schreeuwlelijk kun je nooit}{een kerel worden}\\

\haiku{hij schudde een paar,.}{maal het hoofd alsof hij het}{beter wist dan zij}\\

\haiku{Jullie moet me niet,,.}{zo aanstaren jongens ik}{ben ook maar een mens}\\

\haiku{Van het begin af.}{aan was zij zeer scherp tegen}{hem opgetreden}\\

\haiku{Toen hij zich moeizaam,.}{omdraaide lag de zwerver}{al weer op zijn rug}\\

\haiku{Slaan vernietigt het,.}{moreel schijnt een groot denker}{gezegd te hebben}\\

\haiku{nu zal ze wel hees,,;}{zijn van de hoerenziekte}{dat geeft vaak heesheid}\\

\haiku{Maar dat heb ik nooit,.}{gedaan ik ben het ook nooit}{echt van plan geweest}\\

\haiku{ze namen me voor,...}{\'een dag zoals de schooiers}{Molly voor \'een nacht}\\

\haiku{Uit de mist kwamen.}{meedogenloos schreeuwend de}{meeuwen aangevlerkt}\\

\haiku{Keane boeit ons,,.}{niet omdat hij weet dat we}{niet zullen vluchten}\\

\haiku{Toen zocht hij naar ogen,.}{die strak of dreigend op hem}{gericht konden zijn}\\

\haiku{- {\textquoteleft}Eerst moet aan het licht.}{worden gebracht wie de schuld}{dragen aan zijn dood}\\

\haiku{{\textquoteleft}Mr. Coyne is een,,.}{gentleman jongen hij laat}{het lijk aan ons over}\\

\haiku{Wanneer hij nu eens.}{niet de kracht had n{\'\i}et tegen}{Moyna te lachen}\\

\haiku{Vader heeft zich nog,,?}{boos gemaakt om mijnentwil}{maar wat kon hij doen}\\

\subsection{Uit: Verzamelde romans. Deel 11. De zwarte ruiter}

\haiku{De vrouw, die met een,;}{haakwerkje bezig was zag}{ik van achteren}\\

\haiku{{\textquoteleft}Wanneer u graag rookt,.}{weest u dan voorzichtig met}{vuur in de bossen}\\

\haiku{Inmiddels hadden,,.}{de vrienden dorstend naar wraak}{de brug opgehaald}\\

\haiku{Hij had een lange,,.}{gebogen neus en sterke}{lijnen om de mond}\\

\haiku{de dochter van de....}{bewoner van Ruiterstein}{alles symboliek}\\

\haiku{Onder geen beding.}{zou ik in een verzoening}{hebben toegestemd}\\

\haiku{Haar ogen vreesde ik,;}{nu maar die ogen waren niets}{zonder haar mankheid}\\

\haiku{En toch, hij mocht groot,.}{zijn terzelfder tijd was hij}{nog niet groot genoeg}\\

\haiku{We zouden elkaar,,}{eens kunnen ontmoeten waar}{mijn man niet bij is}\\

\haiku{{\textquoteright} Na even nagedacht,.}{te hebben knikte zij maar}{verroerde zich niet}\\

\haiku{Het hoeft maar heel klein,, -!}{te zijn \'e\'en vlam maar ter ere}{van onze vriendschap}\\

\haiku{ze wilde vuurwerk,,:}{en ze had het en ze gaf}{het niet uit handen}\\

\haiku{Ik stond op van de.}{plaats waar ik op de heide}{neergezonken was}\\

\haiku{Ik was al niet meer,.}{overtuigd dat ik Digna}{Raecke gered had}\\

\section{Cornelis Veth}

\subsection{Uit: Prikkel-idyllen. Deel 1}

\haiku{Een bezoeker werd,.}{aangediend en dadelijk}{binnengelaten}\\

\haiku{Laat de avonturen,,.}{nieuw zijn spannend en laat het}{er dik op liggen}\\

\haiku{Nu d{\`\i}e zal ik wel,.}{eens lezen als ik eens niets}{beters te doen heb}\\

\haiku{De heer Dainty gaf.}{de zaak onmiddellijk bij}{de politie aan}\\

\haiku{Broadstreet stelde zich!}{nu in verbinding met den}{beul van Nantes}\\

\haiku{{\textquoteleft}Maar dat is bijna!}{precies de advertentie}{der roodharigen}\\

\haiku{ik begreep, dat ik.}{de andere dame in}{het oog moest houden}\\

\haiku{{\textquoteleft}Mr. Noppes,{\textquoteright} zei Sir,.}{Sherlock Holmes op onzen}{client toetredende}\\

\haiku{Een nihilistisch,!}{complot nu behoef ik u}{niets te vertellen}\\

\haiku{De naden deden.}{hem telkens pijn in dezen}{nerveuzen toestand}\\

\haiku{Toen, voor wij er op,,.}{verdacht waren keerde hij}{zich om en vluchtte}\\

\haiku{{\textquoteleft}Lord Cookerville,{\textquoteright}, {\textquoteleft}.}{sprak hijdeze zaak is van}{uiterst kieschen aard}\\

\haiku{Ik onderzocht nu.}{de materie die aan de}{couverten kleefde}\\

\haiku{{\textquoteright} zei Harry Wilson,.}{bij zich zelf terwijl hij zijn}{revolvers laadde}\\

\haiku{Gij hebt te kiezen!}{tusschen den dood door vergif}{en dien door den strop}\\

\haiku{Op de tafel v\`o\`or,....}{dit vreemde gezelschap lag}{in een lauwerkrans}\\

\haiku{Ik kies dus....{\textquoteright} Allen.}{wachtten nieuwsgierig op zijn}{verdere woorden}\\

\haiku{Ik dank u.{\textquoteright} {\textquoteleft}Maar er!}{moet toch een einde komen}{aan dezen toestand}\\

\haiku{Ten eerste werpt gij....}{mij den sleutel toe van de}{deur achter  mij}\\

\haiku{A propos, als je,!}{nog eens op karwei gaat neem}{dan een revolver}\\

\haiku{Hij herkende de,.}{stem van den man met wien hij}{juist had gesproken}\\

\haiku{Dit portret maakte.}{in dit ouderwetsch vertrek}{een vreemden indruk}\\

\haiku{Converseert in drie,,!}{talen kan lezen schrijven}{en rekenen}\\

\haiku{Waar de gaatjes in,!}{het portret toe dienden was}{eveneens duidelijk}\\

\haiku{Nog slechts twee van de!}{huilende scalp-dieven}{waren in leven}\\

\haiku{{\textquoteright} {\textquoteleft}Graaf van Zwartburg, mijn!}{boodschap betreft uw titel}{en eigendommen}\\

\haiku{Daar rollen vele,....}{juweelen benevens een}{papier over den vloer}\\

\haiku{Het harde geluid,,!}{dat Jeanne's oor trof deed}{haar oog schitteren}\\

\haiku{De gramofoon werd aan,.}{den gang gebracht maar was niet}{geheel in orde}\\

\haiku{Een verborgen deur,!}{sprong open toen hij toevallig}{er tegen leunde}\\

\subsection{Uit: Prikkel-idyllen. Deel 2}

\haiku{Mijn dochterke, de,!}{zoete Sperata wierd mij}{afgetruggeld}\\

\haiku{Onze pen is niet,,!}{bekwaam het tafereel dat}{volgde te malen}\\

\haiku{Vooraleer zij het,}{wist bevond Ewalda zich te}{midden der t'accoord om}\\

\haiku{Den ganschen dag zwierf....}{hij met den slangenman Pol}{omheen het serail}\\

\haiku{Men bevindt zich in!}{tegenwoordigheid van een}{ijselijk drama}\\

\haiku{menschweerdig bestaan!}{te winnen krijgt mijns zeggens}{steeds meer dringendheid}\\

\haiku{{\textquoteleft}In den geest,{\textquoteright} besloot, {\textquoteleft}!}{hij al weenendzal ik steeds}{een der Uwen blijven}\\

\haiku{Met een groote overmacht,!}{was hij nader geslopen}{en viel de roovers aan}\\

\haiku{daar bevond zij zich.}{plotseling aan den rand van}{een gapend ravijn}\\

\haiku{Wanneer de schoone,}{beestendresseersteer weer bij}{kwam hing zij tusschen}\\

\haiku{uwerzijds, mijn zoete{\textquoteright} {\textquoteleft}.}{engel spreekt hijden rokbroek}{hier in te voeren}\\

\haiku{Ik liep juist op de,.}{gang toen Z.K.H. daarnaar toe}{Bij het ontbijt}\\

\haiku{Wat een raar idee van.}{de menschen om op eens zoo}{eigen te worden}\\

\haiku{{\textquoteright} {\textquoteleft}Zeg eens, Pa,{\textquoteright} zei de, {\textquoteleft},!}{kroonprinsU spreekt tegen een}{troonopvolger hoor}\\

\haiku{Denk je, dat Wij je,?}{niet al lang in de gaten}{hebben kereltje}\\

\haiku{{\textquoteright} {\textquoteleft}Sire, ik ben naar,,.}{hem toe gegaan en heb hem}{verteld zoo en zoo}\\

\haiku{En verder heb ik.}{niets meer van de liefde van}{den Kroonprins gehoord}\\

\haiku{Een voldoend bedrag.}{werd tevens overgemaakt tot}{dekking der kosten}\\

\haiku{Merkwaardigheden.}{waren in het dorp niet veel}{te bezichtigen}\\

\haiku{Dat geboomte en.}{uitspansel den bodem}{De vergissing}\\

\haiku{{\textquoteright} roept Unac uit, {\textquoteleft}heeft mijn?}{broeder met de vogeltong}{zijn scalp verloren}\\

\haiku{'k Jok niet, ik jok,,!}{niet Utah nog is de wond Warm}{als uw eigen bloed}\\

\haiku{Onze patrouille.}{had een geheelen dag en}{nacht vergeefs gewacht}\\

\haiku{Goed ordinair zwart,.}{en blond stemming  vast rood}{aan den flauwen kant}\\

\subsection{Uit: Prikkel-idyllen. Deel 3}

\haiku{Armand en Pierre, ().}{de beide laatsten achter}{boom verscholen}\\

\haiku{Al met zijn neus in,,!}{den wind ja wind Al met zijn}{neus in den wind}\\

\haiku{Mourden sullen plaats,!}{hibben onschuldigen in}{de gefengenis}\\

\haiku{Zoetjes an, kommen,,.}{me waar we wezen motten}{meneer de avekaat}\\

\haiku{Na haar grootje, van,,.}{vaders kant van zelfsprekend}{meneer de avekaat}\\

\haiku{Je kwam bij ons in.}{de keuken met een boodschap}{van meneer Serlier}\\

\haiku{Het gerecht behoeft!}{zich toch niet levendig te}{laten verbranden}\\

\haiku{Elk bruigom heeft zijn,,!}{bruid De galg heeft ook haar buit}{Nu snij ik uit}\\

\haiku{Bessie springt met hem.}{in den vloed en zwemt naar een}{onbewoond Eiland}\\

\haiku{Wilt U zoo beleefd,?}{zijn om vooraan te gaan staan}{bij de famielje}\\

\haiku{Een ridder geheel,.}{in zwarte wapenrusting}{komt aangereden}\\

\haiku{Ik zie precies voor,.}{me wat je dee alsof ik}{er bij geweest was}\\

\haiku{{\textquoteright}  (Tot Jut.) Heidaar,,,!}{jij analphabeet Schrijf op voor}{ik het vergeet}\\

\haiku{Vorigen, Marco,,.}{met tamboer trompetter en}{twee soldaten}\\

\haiku{De Jut trekt af met,, '.}{schand Wij gaan hem na tot aan}{de grens vant land}\\

\section{L\'eon Veugen}

\subsection{Uit: Es God bleef. Bundel sjetse m\`et verhaolende inslaag en romantiese oetslaag}

\haiku{Geer z\"olt leefh\"obbe,.}{m\`et hart en ziel m\`et passie}{en m\`et jaloezij}\\

\haiku{In eur sympathie{\textquoteright}.}{en antipathie zeet geer}{intu{\"\i}tief}\\

\haiku{Ama waor 'n hiel, ' '}{good mins meh ze waorne}{kaptein en ze k\'os}\\

\haiku{'t Is netuurlik,}{ouch hiel good meugelik tot}{es eine vaan us}\\

\haiku{Iech ammezeerde.}{m'ch dao m\`et e potloed en}{e st\"okske pepier}\\

\haiku{Tant Lie waor gein,.}{echte Tant vaan us meh dat}{wiste veer toen neet}\\

\haiku{Meh iech waor 'rs.}{ruim drei en iech waor gans}{oetgelaote}\\

\haiku{{\textquoteleft}Wat, m\'os tiech heij de '!}{zaak op stelte z\`ette}{\'ondert gebed}\\

\haiku{Heer keek miech streng aon,,:}{nog strenger es aanders es}{of heer z\`egge wou}\\

\haiku{es flink breidoet te.}{goon stoon en de zaak mer te}{laote loupe}\\

\haiku{Iech loerde neet op '}{e paar cent est g\'ong um}{de kwaliteit vaan}\\

\haiku{Iech weet neet wat m\`et '.}{m'ch geb\"a\"ord is en iech h\"ob}{t noets gewete}\\

\haiku{Wie mie pijn tot 't.}{deeg wie beter tot iech m'ch}{beg\'os te veule}\\

\haiku{Ze waore nog,... '}{mer koelik te zien en nog}{zoe greun wie graas meh}\\

\haiku{{\textquoteright} Iech v\'ond 't fijn, tot.}{Jean miech neet bij de kleiner}{kinder t\`elde}\\

\haiku{Iech v\'ond 't e leuk.}{werkske en beg\'os direk}{m\`et te hellepe}\\

\haiku{'t sjerm h\'ong tege, ',}{de vinstert sjerm m\`et dee}{sjoene rand boe Pa}\\

\haiku{Komp geer dus oets in}{de buurt vaan Portugees Oost}{Africa en huurt}\\

\haiku{Umtot ze toch de}{leeftied had kraog ze op}{h\"a\"or b\`edsje h\"a\"or ierste}\\

\haiku{Toen bin iech nao h\"a\"or {\textquoteleft}}{tougegaange en h\"ob get}{gebazeld euver}\\

\haiku{Ze st\'ond toen vlak bij}{de kas boe iech in zaot}{en iech heel m'ch mer}\\

\haiku{Iech probeerde de,:}{beuvenste k\"orf op te duije}{meh dat veel neet m\`et}\\

\haiku{Dao hadste 't al. '.}{Mia hadt gekraak gehuurd}{en kaom aonrenne}\\

\haiku{ziech neet zoe gaw op '!}{z'ne kop zitte doornen}{\'onderwijzer}\\

\haiku{waor, beg\'os heer ', '.}{t leedsje te speule wat}{dao opt bord st\'ond}\\

\haiku{En Pa betaolde,.}{dus mer alles zellef de}{studie en de beuk}\\

\haiku{Toen heet ze h\"a\"ore l\`este.}{aosem oetgebloze en}{iech waor daobij}\\

\haiku{{\textquoteleft}De z\`eks jummers zelf, '!}{tot iechm neet zoe lang druug}{maag laote stoon}\\

\haiku{tot d'r neet altied;}{gelegenheid waor um}{de sjeun te p\'otse}\\

\haiku{Meh 't waor 'n.}{gemein streek en de vrundsjap}{waors kepot}\\

\haiku{Nou dat waor wel.}{de gammelsten auto dee}{iech oets gezeen h\"ob}\\

\haiku{Meh noe had heer 't.}{naodeil tot ouch de remme}{niks weerd waore}\\

\haiku{Miestal waor Ma '.}{erg laat daom\`et en daan m\'os}{iecht insl\'okke}\\

\haiku{Harie - zoe h\`edde -.}{m'ne nuije vrund woende aon}{de rand vaan de stad}\\

\haiku{Heer wou iers ins m\`et '.}{ne verstendige Priester}{d'reuver praote}\\

\haiku{de slumste, meh ze.}{zaog d'r nog altied eve}{verleidelik oet}\\

\subsection{Uit: 'ne Z\"och vaan de ieuwigheid}

\haiku{en in een droombeeld:}{overziet hij in \'e\'en enkel}{teken zijn leven}\\

\haiku{Gans gel\"okkig waor ',!}{r mesjien nog neet meh wel op}{weeg denau tou}\\

\haiku{Heer wis neet good wat '.}{ze veur had en keekr e}{bitteke suf aon}\\

\haiku{Veur d'n ierste kier had ':}{N\`eske get gezag m\`etne}{rechte strakke m\'ond}\\

\haiku{In Persessies m\'os '.}{r m\`etloupe en m\`et}{de Heiligdomsvaart}\\

\haiku{Aon Slevrouw waos.}{heer tougewijd bij mie es}{ein gelegenheid}\\

\haiku{meh es ze weg goon,.}{bestoon ze allein nog}{in mien gedachte}\\

\haiku{{\textquoteleft}God, God{\textquoteright}, b\`ejde, {\textquoteleft}!}{heerlaot m'ch toch neet deen}{Anti-Christ weure}\\

\haiku{Pa had veugelkes, '.}{gefok in groete kowwe}{dier zelf maakde}\\

\haiku{Die had 'r m\`et nao '.}{hoe s genome en in}{n sjeundoes gezat}\\

\haiku{Heer waos verbaas,.}{want Mestreechtenere zien neet}{zoe oranje gezind}\\

\haiku{h\'onger oetst\`elle.}{es de rutse en abele}{beg\'oste te pakke}\\

\haiku{Heer m\'os 't toch ins ';}{opzeuke en zien ofr}{deen teks k\'os einde}\\

\haiku{'t Iezer waos' ':}{nog neet d roet oft blood}{beg\'os te loupe}\\

\haiku{Gel\"okkig had eine '.}{ne zakdook boe neet al te}{v\"a\"ol sn\'ots in zaot}\\

\haiku{Heer leet 'r mer vas,:}{d'n trap op goon en vroog h\"a\"or}{hand vashawwend}\\

\haiku{mesjien laog ze get.}{te leze of zoemer veur}{z'ch oet te stare}\\

\haiku{Heer had noets get m\`et '.}{r te doen gehad of zelfs}{mer w\`elle h\"obbe}\\

\haiku{Lier  noe mer iers!}{ins um vief menute neet}{aon Ape te dinke}\\

\haiku{Dat vaan die rots en '.}{t veugelke maakde op}{h\"a\"om neet v\"a\"ol indr\"ok}\\

\haiku{Met groete aondach had ' ':}{rt werk gedoon en bij}{eder st\"okske gedach}\\

\haiku{meh dat m\'os wel good,.}{geb\"a\"ore aanders kraog me}{later geel vlekke}\\

\haiku{Ins had 'r Charles (!):}{aongesproke op straot}{heer wis nog zjus boe}\\

\haiku{'t had mie weg vaan ', '.}{ne zaol m\`et steulkes wie}{inne cinema}\\

\haiku{Toch laos heer veur, ':}{al m\'osr edere kier nao}{aosem snakke}\\

\haiku{{\textquoteright} Heer had de spanning ':}{gebroke m\`et opn druug}{meneer te z\`egge}\\

\haiku{Meh, zou 'r dat wel,?}{kinne zoelang es Penny}{d'r nog waos}\\

\haiku{De taofel st\'ond ' '.}{vol m\`et blomst\"okker es oft}{umn broelof g\'ong}\\

\haiku{{\textquoteright} Tante Yvonne m\'os ' ';}{t teske tr\"okz\`ette}{opt sjeutelke}\\

\haiku{teskes en sjeutelkes '.}{en tleurkes m\`et forsj\`etsjes}{veurt gebekske}\\

\haiku{Rechs bove en links!}{\'onder opnaomes vaan zien}{d\"ochterke Beppie}\\

\haiku{M\`et e bitsje ' '.}{funkele kraogrt}{vuur weer good op gaank}\\

\haiku{Heer had ins e book.}{geleze vaan Arthur Koestler}{euver  humor}\\

\haiku{Heer woort d'r gans werm, ';}{vaan meht waos gein zwoer}{passie deze kier}\\

\haiku{tr\`ek me rechs daan geit, '.}{de baj nao rechs tr\`ek me links}{daan geitr nao links}\\

\haiku{De veermaan sjuifde '.}{opzij en Charles moch aon}{t raad drejje}\\

\haiku{{\textquoteright} Werechtig, dao ging.}{zoe'n groete r\"os oet vaan al}{dat greun r\'ontelum}\\

\haiku{Want wat heer noe wou,!}{doen waos zun m\`et volle}{kinnis en vrije w\`el}\\

\haiku{de zuus al d'n vrun.}{trouwe  en z'ch e leuk}{huiske inriechte}\\

\haiku{gein twie vlemkes zien ',.}{t zelfde al liekene}{ze nog zoe op ein}\\

\haiku{{\textquoteright} had 'r gevlook,.}{meh dao waos die geert neet}{m\`et gerippereerd}\\

\haiku{heij is eine dee.}{probeert e forellekoer te}{dirigere}\\

\haiku{Gein al te groete, ',;}{wandeling dachr meh toch}{evekes aajd Mestreech in}\\

\haiku{{\textquoteleft}W\`elt g'r geluive,?}{tot ze m'ch nog eder jaor}{m\`et Keersmis get sjik}\\

\haiku{{\textquoteright} {\textquoteleft}W\`el iech d'ch ins get,......}{z\`egge jong jao nump m'ch de}{vrechheid neet koelik}\\

\haiku{Want 'n vrouw is e,.}{raar weze dao kin me noets}{staot op make}\\

\haiku{Ze had geveuld tot ', '.}{r get in de loch zaot}{int zonneleech}\\

\haiku{in de riechting vaan,}{Sint Pieter m\`et de berg in}{de riechting  vaan}\\

\haiku{Hier, probeert U maar!}{eens deze sabel in de}{schede te steken}\\

\haiku{Charles m\'os 't frans,:}{leedsje zinge wat ze noets}{k\'oste \'onthawwe}\\

\haiku{2 Jeh, veurwat had?}{Miep z'ch veur die secte goon}{interessere}\\

\haiku{Daan probeerde ze;}{te doen of heer de taol}{neet good beheersde}\\

\haiku{Toen pakde 'r ze.}{k\"offerke en puunde vrow}{en kinder adie}\\

\haiku{Allein 'nen dikke, ',;}{Duitserne bl\'onde reus oet}{Beieren zag niks}\\

\haiku{Dee veurige breef,, '.}{dee tr\"ok waos gekoume}{staokr debij}\\

\haiku{Ze had 'm vert\`eld,.}{vaan h\"a\"or \"a\"ontsje dee op de}{b\`este sjaol zaot}\\

\haiku{Daan stoonte d'r e:}{paar damessjeun boete}{op de vinsterbaank}\\

\haiku{{\textquoteright}, zagte ze daan, {\textquoteleft}de!}{kins wel zien tot d'r niks geit}{bove eige teelt}\\

\haiku{ne Mins m\'os toch wel.}{volslage immoreel zien}{um zoeget te doen}\\

\haiku{D'r waore sjijns.}{al hiel get lui verdr\'onke}{in die zwijnerij}\\

\haiku{M\`et groete stappe.}{leep heer in de riechting vaan}{de Poort Waarachtig}\\

\haiku{Pas wie d'n trein gans '.}{st\`el st\'ond zaog ze Amersfoort}{opt b\"ordsje stoon}\\

\haiku{zoe good wou zien en '.}{m neet op heite kole}{laote zitte}\\

\haiku{Ze st\`elde h\"a\"om get, ':}{veur watr z'ch mer ins good}{m\'os euverdinke}\\

\haiku{Wat heer 't ergste '.}{v\'ond waost verbranne}{vaan alle pep\`erre}\\

\haiku{(wat e geld hadde!)....}{ze mote oetgeve vaan}{dat klein inkoume}\\

\haiku{Trouwens g'r kint 'm {\textquotedblleft}{\textquotedblright}, '!}{beterNozak numme noe}{r gecastreerd is}\\

\haiku{H'r wou allein nog.}{get meziek opz\`ette}{en naodinke}\\

\haiku{Noe ins leek 't op, '. '}{e kruiske daan weer opn}{viefpuntige staar}\\

\haiku{t Losde z'ch op,}{in kleure sjoen greun wie vaan}{e koreveld es}\\

\section{Bea Vianen}

\subsection{Uit: Het paradijs van Oranje}

\haiku{Mohan had zich voor de.}{gelegenheid wel erg}{smaakvol opgedoft}\\

\haiku{Hij stapte kwiek uit,.}{de auto stak een hand uit}{en noemde zijn naam}\\

\haiku{Maar tussen haakjes,?}{heb je het met hem over mijn}{inkomsten gehad}\\

\haiku{Hij wil belangrijk.}{figuur zijn en daarom zegt}{hij de gekste dingen}\\

\haiku{Ze was blij met de.}{aandacht maar durfde dat niet}{te laten blijken}\\

\haiku{Of was er bij hem?}{n\'u al sprake van stipjes}{onder het glazuur}\\

\haiku{Zijn diagnose was,.}{negatief hij wist dat hij}{het zou verliezen}\\

\haiku{Hij begon opeens.}{hevig te twijfelen aan}{het doel van zijn reis}\\

\haiku{{\textquoteleft}Niets is moeilijk,{\textquoteright} zei,.}{Sirdjal en legde hem uit}{hoe er te komen}\\

\haiku{het keukentje als.}{hij het luik naar beneden}{had laten zakken}\\

\haiku{Ze zag op hem neer,.}{maar voelde zich niet tegen}{hem opgewassen}\\

\haiku{En men vergat dat,.}{terugkeer moeilijk was ja}{onmogelijk zelfs}\\

\haiku{Sirdjal pakte het {\textquoteleft}{\textquoteright}.}{glasspeciaal uit de kast}{en schonk hem thee in}\\

\haiku{Had hij genoeg van?}{het pootjebaden op z'n}{zolderkamertje}\\

\haiku{{\textquoteright} {\textquoteleft}Als je me eerder,.}{had gebeld had ik wel even}{voor je gekeken}\\

\haiku{Je moet het zoveel.}{betalen en dan moet je}{ook nog olie kopen}\\

\haiku{Kijk Firoz, niemand dwingt.}{je om wel of niet te gaan}{met een vrouw van hier}\\

\haiku{Als je het niet wilt,,.}{als je het niet kan dan moet}{je het ook niet doen}\\

\haiku{Ik ben ook blij met.}{het vlees dat  je voor me}{hebt meegenomen}\\

\haiku{want hij was verknocht.}{aan zijn plekje achter het}{waterreservoir}\\

\haiku{Hoe kan je dan nog?}{volhouden dat je me ooit}{nodig hebt gehad}\\

\haiku{Kort en goed, hij moest.}{zich maar niet verbeelden dat}{ze hem niet kenden}\\

\haiku{Hij bleef over zijn geld,.}{pochen zijn toekomstplannen}{in Suriname}\\

\haiku{Zachtjes drukte hij.}{de hendel van de deur naar}{de keuken omlaag}\\

\haiku{Hij deed het licht aan.}{en concentreerde zich op}{de kruiden voor hem}\\

\haiku{Ze hebben me op.}{een dag staan opwachten in}{de Tourtonnelaan}\\

\haiku{De sla was op, maar.}{er was nog wat rijst over en}{vlees was er genoeg}\\

\haiku{{\textquoteleft}Het was erg lekker,,,.}{bhai maar ik kan niet meer mijn}{maag is gekrompen}\\

\haiku{Hij wist nu zeker.}{wie er aan de andere}{kant van de lijn was}\\

\haiku{Voor het laatste was.}{bapa een veel te oude}{en zuinige man}\\

\haiku{Het was juist altijd.}{zijn bedoeling geweest de}{dood uit te stellen}\\

\haiku{Om half vijf was hij,.}{weer thuis hij blies en hijgde}{van de boodschappen}\\

\haiku{{\textquoteleft}Ik neem aan dat u?}{vergeten bent het licht op}{de trap uit te doen}\\

\haiku{Om kwart over zeven.}{pakte hij de tram naar het}{centrum van de stad}\\

\haiku{{\textquoteright} {\textquoteleft}We zijn allemaal,.}{gevangen en iedereen}{in zijn eigen kooi}\\

\haiku{{\textquoteleft}Weet je, bhai, ik heb.}{stilletjes gelezen wat}{je hebt geschreven}\\

\haiku{{\textquoteright} Op dergelijke.}{aantijgingen was Sirdjal}{altijd voorbereid}\\

\haiku{Hij was er wel bang,.}{voor maar liet er zich toch niet}{echt door bedrukken}\\

\haiku{Bisoenlal zat met.}{een half verdwaasd gezicht voor}{zich uit te kijken}\\

\haiku{D\'an wist je echt niet.}{of hij zich voor wie dan ook}{interesseerde}\\

\haiku{Hij kwam er nog maar,.}{zelden en het deed hem goed}{dat hij er weer was}\\

\haiku{{\textquoteright} {\textquoteleft}Ik bel je nog op,{\textquoteright}.}{zei Sirdjal en richtte zijn}{schreden huiswaarts}\\

\haiku{Hij had daar trouwens,.}{alle recht toe want hij had}{op Den Uyl gestemd}\\

\haiku{U bent bang, omdat.}{zoiets in uw eigen kring}{voor onbeschaafd geldt}\\

\haiku{Hij hijgde, voelde.}{opnieuw een koude rilling}{over zijn lichaam gaan}\\

\haiku{Hij deed de deur dicht,.}{rookte de ene sigaret}{na de andere}\\

\haiku{Hij schraapte zijn keel,.}{spuugde in de wasbak en}{keek in de spiegel}\\

\haiku{Hij kon zijn ogen niet.}{geloven en liet hem als}{in een droom binnen}\\

\subsection{Uit: Sarnami, hai}

\haiku{Zij wordt nieuwsgierig,,.}{kijkt op terwijl zij met de}{kam over haar hoofd strijkt}\\

\haiku{De oude werpt het.}{tweede kale stengeltje}{op de houten vloer}\\

\haiku{Het zijn altijd weer.}{dezelfde vragen waarop}{nooit een antwoord komt}\\

\haiku{Van het groen van de,.}{palmen de manjebomen}{en de guave}\\

\haiku{Pas na een hele.}{tijd ontspant zij zich en doet}{zij haar ogen wijd open}\\

\haiku{Een donkere hand.}{tilt met een vuile vaatdoek}{het deksel omhoog}\\

\haiku{Muskieten zwermen.}{aan en beginnen voor het}{venster te gonzen}\\

\haiku{Het heeft geen zin haar.}{nog langer te sarren met}{haar aanwezigheid}\\

\haiku{Het onverschillig,.}{antwoord van de vrouw krenkt haar}{maar zegt haar genoeg}\\

\haiku{Roekmien, nu weer in,.}{het atelier neemt plaats achter}{haar trapmachine}\\

\haiku{Zij staat meteen weer,.}{op doet de radio aan en}{gaat dan aan het werk}\\

\haiku{Ram ziet S. zitten,.}{groet vriendelijk en vraagt of}{zij al lang hier is}\\

\haiku{Zij buigt zich over het.}{boek maar kan zich niet langer}{concentreren}\\

\haiku{Onder het licht van.}{een lantaarn blijft zij staan om}{ze te bekijken}\\

\haiku{Voorin zitten twee.}{luid pratende vrouwen met}{bonte hoofddoeken}\\

\haiku{Maar er is nog een.}{andere werkelijkheid}{om voor te vechten}\\

\haiku{Zij loopt weg van het,.}{raam laat het muskietenkleed}{neer en kleedt zich uit}\\

\haiku{{\textquoteright} Dan zit zij in een,,.}{motorboot nog altijd naakt}{op weg naar het schip}\\

\haiku{De eersten zijn meer.}{bij hun vader dan bij hun}{eigen moeder thuis}\\

\haiku{Zij laten S. en.}{Soekhia in een bedrukte}{stemming achter}\\

\haiku{Zei haar vader niet?}{dat zij de laatste tijd zo}{lang van huis wegbleef}\\

\haiku{S. heeft boodschappen.}{gehaald bij de Chinees en}{is op weg naar huis}\\

\haiku{Na die bewuste.}{vrijdag heeft zij de vrouw niet}{meer teruggezien}\\

\haiku{Haar haren, haar rug.}{en haar buik zijn drijfnat van}{het transpireren}\\

\haiku{s Avonds na het bad.}{studeren zij verder op}{haar kamer boven}\\

\haiku{Angst, verdriet gemengd,.}{met gevoelens van vreugde}{overweldigen haar}\\

\haiku{Zij kent zijn drift en.}{ziet aan zijn gezicht dat het}{ergste voorbij is}\\

\haiku{Hij kijkt niet op, zelfs.}{niet als de bliksem knalt en}{huis en erf verlicht}\\

\haiku{Het meisje wil iets,,.}{zeggen doet haar mond open maar}{er komt geen woord uit}\\

\haiku{Vervolgens wordt hem.}{gevraagd waarmee hij haar wenst}{te accepteren}\\

\haiku{Na vier maanden is.}{het afgebouwd en kunnen}{zij het betrekken}\\

\haiku{Radj is de laatste,.}{tijd erg prikkelbaar heeft zij}{tegen S. gezegd}\\

\haiku{Zij vindt de stof erg.}{mooi en vraagt of zij haar heel}{even mag betasten}\\

\haiku{{\textquoteright} vraagt S. {\textquoteleft}De dokter,{\textquoteright}.}{zegt over anderhalve maand}{antwoordt Selinha}\\

\haiku{S. volgt haar naar de,.}{keuken waar zij tot half acht}{zitten te praten}\\

\haiku{Zij kijkt naar de lucht,.}{zwaait met de stok en begint}{vlugger te lopen}\\

\haiku{Hij pakt de stok uit.}{haar hand en werpt hem naar het}{andere trottoir}\\

\haiku{Het betekent dat.}{zij moet wachten totdat haar}{vader terug is}\\

\haiku{Er bestaat zo iets.}{als een paspoort dat je niet}{zelf hebt getekend}\\

\haiku{Zij wil niet met een.}{man gezien worden en dan}{liefst in een auto}\\

\haiku{Achter haar klinkt de.}{motor van de achteruit}{rijdende auto}\\

\haiku{Zij had het zo druk.}{met de huishoudelijke}{beslommeringen}\\

\haiku{Het ene moment voelt.}{zij zich vertederd door de}{herinneringen}\\

\haiku{{\textquoteleft}Je weet dat ik nog,{\textquoteright}.}{op school zit antwoordt zij om}{hem te ontwijken}\\

\haiku{Sindsdien is er in.}{haar leven een nieuwe angst}{bijgekomen}\\

\haiku{Straks zal het tot haar.}{doordringen dat zij nooit meer}{naar huis terugkeert}\\

\haiku{Islam zou kunnen.}{ontkennen en weigeren}{met haar te trouwen}\\

\haiku{Zij kan aan niemand,.}{anders laten merken wat}{zij van binnen voelt}\\

\haiku{De dief had zich met.}{olie ingesmeerd maar werd ten}{slotte toch gepakt}\\

\haiku{Zij moet van de rand,.}{zijn afgegleden toen hij}{naast haar kwam liggen}\\

\haiku{Kort voordat zij naar.}{bed ging liet zij de jongens}{van Soekhia binnen}\\

\haiku{Ata's aanwezigheid.}{dwingt haar tot het opvoeren}{van een toneelspel}\\

\haiku{Zij knipt het licht uit,.}{trekt zachtjes de deur achter}{zich dicht en is weg}\\

\haiku{Hij slaat de druppels.}{uit de kam en stopt die in}{de zak van zijn hemd}\\

\haiku{De jongen had een.}{grote bewondering voor}{zijn fysieke kracht}\\

\haiku{Zij betaalt hem de.}{rekening en gaat naar de}{kamer van het kind}\\

\haiku{Misschien bedoelt hij,{\textquoteright}.}{dat hij je zal verstoten}{zegt zij ten slotte}\\

\haiku{Hij doet de deur open,.}{glimlacht nerveus en neemt de}{jongen van haar over}\\

\haiku{Hij komt en gaat met.}{het besef dat zij afscheid}{nemen van elkaar}\\

\haiku{Hetzelfde deed zij.}{de avond dat zij hier voor het}{eerst had geslapen}\\

\section{S.G. van der Vijgh jr.}

\subsection{Uit: Werkers}

\haiku{Mijn tranen heet gaan}{om hun doode trekken neer}{en in mijn armen}\\

\haiku{loeide het uit dat,;}{ze gezocht werd dat ze niet}{meer wou in de keet}\\

\haiku{eindeloos blanke,.}{zee van landen vlagend om}{de hooge machtfabriek}\\

\haiku{Het is van de kerk,:}{uit dat plechtigheid en ernst}{waren over het dorp}\\

\haiku{Als er zieken zijn,;}{hooren de smidslui het bij}{toeval van vreemden}\\

\haiku{In den hoek hoestte,.}{oude Piet wakker schrikkend}{uit zijn gedommel}\\

\haiku{Wel weerlichts,{\textquoteright} zeit-ie, {\textquoteleft}' ',?}{t is zonde da'kt zeg}{is da nou stoke}\\

\haiku{{\textquoteright} vroeg Kees minachtend,.}{dadelijk partij nemend}{voor de dagstokers}\\

\section{Simon Vinkenoog}

\subsection{Uit: Liefde. Zeventig dagen op ooghoogte}

\haiku{Maar ik heb alle:}{zekerheden verloren}{behalve die ene}\\

\haiku{redakteuren Claus,,,):}{Michiels Mulisch Vinkenoog}{zijn notities stuurt}\\

\haiku{Ook de zeventig,.}{dagen die ik sleet ben ik}{in staat te overzien}\\

\haiku{ik heb die liefde, (}{vele namen gegeven}{\'e\'en daarvan is God}\\

\haiku{Ik zal ze wieden,.}{uit mijn dagboekbladen en}{toch niets weglaten}\\

\haiku{Ik heb een afstand,.}{geschapen kregen die ik}{wilde overbruggen}\\

\haiku{Kijken, of ik het,.}{nog kan rekonstrueren en}{tot hoever terug}\\

\haiku{s avonds komt zij thuis,.}{langs terwijl Elize en Don}{naar de bioskoop zijn}\\

\haiku{De schrijver geplaagd,.}{door muggen drie dagen voor}{de terechtzitting}\\

\haiku{de godgeworden.}{mens eerder dan om een van}{de middelen}\\

\haiku{Van nu af aan slechts,.}{d\'eze regels mijn gids en}{begeleider}\\

\haiku{Zij kon het weten,, {\textquoteleft}{\textquoteright}.}{zij had het meegemaakt zij}{hadermee geleefd}\\

\haiku{God is here, and,:}{I am a witness to the}{fact and so are you}\\

\haiku{Het ritme is een, (.}{eenvoudige anapest/trochee}{lees HopkinsPenguin-ed}\\

\haiku{Het motto grijpt me,:}{meer dan tien jaar later bij}{de strot zelf zegt hij}\\

\haiku{ik ben blij en trots,.}{het meegemaakt te hebben}{als vriend des huizes}\\

\haiku{Alsof een voorwerp,!}{een betekenis heeft die}{bevrijd kan worden}\\

\haiku{Hoeveel verwijten,:}{tegen mij die ik nog zal}{moeten uitbannen}\\

\haiku{Weet dat je minder,,.}{ziet zodra je kwaad wordt de}{perceptie mindert}\\

\haiku{Maak van je kind een,.}{heilig huis dat geen ander}{heilig huis erkent}\\

\haiku{Geen wetenschap is,.}{mogelijk zonder geloof}{geen leven of dood}\\

\haiku{Niet op alles het,.}{antwoord dat ligt in ieder}{voor zich besloten}\\

\haiku{Ik bevrijd je van,.}{de angst ook voor de dood want}{ik ben het leven}\\

\haiku{zelfs de vlooien van.}{m'n buren spelen een rol}{in de eeuwigheid}\\

\haiku{Zij weten wie ik,.}{ben al weet ik het met een}{ander lichaam}\\

\haiku{De normale mens - - {\textquoteleft}.}{funktioneert volgens hem}{dans les limites}\\

\haiku{zo helpe mij God -.}{waarachtig het is ook nog}{nieuw realisme}\\

\haiku{omdat hij niet wist,.}{te vertellen of hij van}{z'n vrouw hield of niet}\\

\haiku{het Evangelie van), (;}{Johannes de manen van}{de leeuwde hartstocht}\\

\haiku{Ik loop de gang op,.}{haar Chinese ochtendjas}{half omgeslagen}\\

\haiku{Ik heb geluisterd, ().}{ik kan het helpen dat zij}{huildeopluchting}\\

\haiku{Alles is goed, ja,.}{als de mensen hun weten}{kunnen gebruiken}\\

\haiku{Moet de weg vrij zijn,?}{of begeven we ons met}{goed en kwaad op weg}\\

\haiku{Schuin tegenover mij,.}{op de derde verdieping}{een raam stond halfopen}\\

\haiku{Ik weet het, ik was,.}{de langste ik droeg de vlag}{trots lopend voorop}\\

\haiku{Ik funktioneer.}{als beeld in de hersenen}{van mijn bezoekers}\\

\haiku{gerrit achterberg,[ -]}{dichter   20 mei 1905 17}{januari 1962}\\

\haiku{zijn zoon/mijn zoon, het (:}{gezamenlijkeik hoor}{me praten en denk}\\

\haiku{alle andere,,.}{dingen die ik zie zijn even}{grote wonderen}\\

\haiku{{\textquoteleft}Je openstellen voor,,.}{de waarheid die in je ligt}{en die uitdragen}\\

\haiku{Alle po\"ezie.}{wordt geschreven om dit doel}{nader te brengen}\\

\haiku{er is geen einde -:}{aan het weten godzijdank}{is er een begin}\\

\haiku{We waren bij Huub,,,.}{weggevlucht hij m'n vriend was}{niet te verdragen}\\

\haiku{Reineke kwam thuis,.}{bij de omhelzing gleden}{onze kleren weg}\\

\haiku{Met de stoomkursus,}{van de vertaling doorloop}{ik jaren ineens}\\

\haiku{* ~ Met Reineke (;}{en Stepheneen andere}{route achter mij}\\

\haiku{een brandweerman, twee,.}{jongetjes anderen die}{UFO's hadden aanschouwd}\\

\haiku{Het echtpaar Van O.,:}{jaag ik weg zo ook S. die}{ik schrijvend toevoeg}\\

\haiku{Zijn boodschap aan mij.}{een gans andere dan de}{door Ruud gehoorde}\\

\haiku{(Hier wordt de lezer.}{gevraagd zich over het witte}{papier te buigen}\\

\haiku{Benodigdheden.}{een gemakkelijke stoel}{en een telefoon}\\

\haiku{Misschien komt het geld,...:}{nog eens met sneeuwbaleffekt}{terug wie weet Reis}\\

\haiku{Mijn verzameld werk, (-).}{de eerste gedichten1948}{1964 en de laatste}\\

\haiku{We betrekken een,,.}{nieuwe woning voor een jaar}{weer op de Bloemgracht}\\

\haiku{zij absorberen,.}{slechts wat hoogst nodig is het}{meest voor de hand ligt}\\

\haiku{in oprechtheid, met,.}{geduld in onderwerping}{en vertrouwen}\\

\haiku{Ik leef het meeste,{\textquoteright} -:}{als ik alleen ben met jou}{bijna verwijtend}\\

\haiku{ook Elize zal de,.}{man vinden die haar geeft wat}{ik niet kon geven}\\

\haiku{Neem de mensen hun,.}{gereedschap niet uit handen}{geef ze te spelen}\\

\haiku{Een schuin inzicht in,;}{hun portiek waar zij woonden}{op een beletage}\\

\haiku{{\textquoteleft}Als Simon nu maar,...}{niet dacht dat hij altijd naar}{woorden moet zoeken}\\

\haiku{Reineke wakker -.}{ik breng haar achterop de}{scooter naar haar werk}\\

\haiku{{\textquoteleft}Anderen zouden,.}{verslaafd raken als ze maar}{niet zo'n angst hadden}\\

\haiku{Ik ga een deel van,.}{de gang met ze mee wijs ze}{de weg naar boven}\\

\haiku{{\textquoteright} {\textquoteleft}Ik moet de waarheid.}{ervan met mijn gehele}{wezen ervaren}\\

\haiku{Ik weet het niet, wat,?}{moet ik zeggen welke kant}{gaat het sprookje uit}\\

\haiku{Ik ben geen - nee, niet,:}{een van al die andere}{dingen of ook toch}\\

\haiku{Het verbaast me, als ():}{een anderStephen bewust}{en duidelijk zegt}\\

\haiku{Hij heeft een antwoord,:}{van de Bijenkorf gehad}{dat hem bevredigt}\\

\haiku{Als je maar niet stil,,}{blijft staan nalaat vooruit te}{gaan als je maar weet}\\

\haiku{vrouw, vrienden, muziek,,,,?}{geluk werk wat te roken}{wat wil ik nog meer}\\

\haiku{de mensen eerst op,.}{weg naar zichzelf dan kunnen}{we verder spreken}\\

\haiku{Over de mens, die schrijft, {\textquoteleft}{\textquoteright}.}{de levende schrijver die}{jijin handen hebt}\\

\haiku{alles wat ik nu}{begrijp alles waaraan ik}{nu weet te denken}\\

\haiku{zij maakt aanstalten,.}{naar bed te gaan de heer des}{huizes tikt en vrijt}\\

\haiku{Ik heb gehandeld,.}{in ijdelheid de tijd met}{mij vereenzelvigd}\\

\haiku{HH theologen,,!}{dichters mathematici}{en filosofen}\\

\haiku{Hanterend zoveel.}{mogelijkheden als ik}{bemachtigen kan}\\

\haiku{Een maatschappij dient.}{op solider maatschappij}{te worden herbouwd}\\

\haiku{ik raak nauwelijks,.}{iets aan ik raak verzeild in}{de pagina's zelf}\\

\haiku{Hij trekt z'n laarzen,):}{weer aan maandag vertrekt hij}{naar de Spaanse vlieg}\\

\haiku{Hij deelde me mee.}{niet te weten wat hij met}{z'n toekomst moest doen}\\

\haiku{Ik denk van een plicht,.}{te zijn ontslagen maar dit}{blijkt niet het geval}\\

\haiku{{\textquoteright} (Hij woont tijdelijk.) {\textquoteleft}:}{op een van de WallenEn}{in de folders staat}\\

\haiku{De onomatopee,:}{in het Engels vaak mooier}{dan het Nederlands}\\

\haiku{De stem bij de bel. (.}{Ik was de eigenaar van}{de stem vergeten}\\

\haiku{Hij konfronteert ook,?}{mij want wie is volkomen}{onbevooroordeeld}\\

\haiku{ik dacht dat ze zo'n.}{monsterlijk gezicht zette}{om me te plagen}\\

\haiku{Ik heb het nog, nog,.}{\'e\'en keer laten meemaken}{nooit meer aangeraakt}\\

\haiku{al een week lang was.}{G. met dochter Rosalie}{in dit land terug}\\

\haiku{Blijf vragen en het.}{antwoord zal je in de schoot}{worden geworpen}\\

\haiku{Ondertussen zijn ();}{Huub en Olvertvandaag 24}{binnengekomen}\\

\haiku{De mens aan de mens,.}{gelijk met een andere}{taak te vervullen}\\

\haiku{tussen goed en kwaad,:}{tussen deze wereld en}{de schijnwerelden}\\

\haiku{Het heden is de,}{toestand waarin ik het liefst}{verkeer door wat komt}\\

\haiku{{\textquoteleft}Beste Simon, ik...{\textquoteright}.}{geloof niet dat En ik zal}{je zeggen waarom}\\

\haiku{heiligen zijn op,.}{weg naar de volmaaktheid en}{heel ver gekomen}\\

\haiku{Ik zou uren over Jan,.}{willen spreken over alles}{wat hij zegt en doet}\\

\haiku{de van 10 tot 75, -.}{sekonden vrij vallende}{skydivers 250 km/u}\\

\haiku{de armen gespreid,,,.}{op de buik het hoofd in de}{nek als een vogel}\\

\haiku{Die denken dat het,,.}{de ziel is of zoiets ja}{ze noemen het ziel}\\

\haiku{ik probeerde wat, {\textquoteleft}{\textquoteright};}{ik z\'o kreegtoevallig nog}{uit te spreken ook}\\

\haiku{het cybernetisch.}{denken toegepast aan de}{omstandigheden}\\

\haiku{Dit is nu, dit is,,.}{altijd dit is van hieruit}{waar dit beleef ik}\\

\haiku{E\'en groot toneelstuk,,.}{iedereen doet maar wat maar}{hij weet wel b\'eter}\\

\haiku{Ik wil dat je aan,,.}{me werkt zoals ik aan dit}{boek werk met liefde}\\

\haiku{Van hieruit ga ik, -:}{verder opnieuw als het moet}{telkens weer opnieuw}\\

\haiku{Het huis aan de Oudezijds.}{verleende onderdak aan}{vele bezoekers}\\

\haiku{Dit ben ik, dit ben,.}{jij nog niet ik ben niet aan}{je toegekomen}\\

\haiku{{\textquoteright} Rudolf leerde het,:}{Boek der Veranderingen}{kennen hij wierp 13}\\

\haiku{Ik rijd scooter op,.}{routine haal I Tjing voor}{vrouwen en vrienden}\\

\haiku{Zoveel, dat het een.}{wonder is dat er nog naar}{me wordt geluisterd}\\

\haiku{hij ziet het zwarte, ().}{leer van mijn jas kijkt niet tot}{in mijn ogenik wacht}\\

\haiku{Het valt me prettig,}{dit verhaal op dit ogenblik}{tot jou te richten}\\

\haiku{Ik heb aandacht voor,.}{de details zij maken het}{leven  zoeter}\\

\haiku{{\textquoteleft}Daar ben ik, geloof,,.}{ik niet zo geschikt voor voor}{document humain}\\

\haiku{het wonder van de,.}{elektriciteit verdwenen}{het wonder daglicht}\\

\haiku{Want, zeg ik, als ik,}{mij bewijs bewijs ik door}{mijn ontdekkingen}\\

\haiku{Vrijheid is sneller,.}{dan het licht vrijheid is je}{laatste dimensie}\\

\haiku{Je bent vrij, als je,,.}{sterft vrij als je ademt in slaap}{of bij bewustzijn}\\

\section{Ab Visser}

\subsection{Uit: Leven van de pen}

\haiku{{\textquoteleft}Ga daar maar rustig.}{staan vader en zorg dat je}{d'r niet uitlazert}\\

\haiku{{\textquoteright} Mijn permanente.}{eenmansshow levert mij geen}{enkel voordeel op}\\

\haiku{Hij was het type:}{waarvan de mannen onder}{elkaar zeggen}\\

\haiku{Agatha doorliep geen,.}{enkele school maar ontving}{haar opleiding thuis}\\

\haiku{hij bleef ongetrouwd,.}{vermoedelijk omdat hij}{homosexueel was}\\

\haiku{de gevulde lijn ().}{plooit zichin de confectie}{naar de plompe vorm}\\

\haiku{{\textquoteleft}wordt eikenschors bij ', '.}{t pond gewogen men weegt}{kaneel bijt lood}\\

\haiku{Een boek hoeft niet eens.}{goed te zijn als de titel}{maar suggestief is}\\

\haiku{Ik heb zelf pas een {\textquotedblleft}{\textquotedblright}.}{boek uitgegeven datKaal}{met een kuifje heet}\\

\haiku{Hij was tenslotte.}{op alle terreinen des}{levens de pineut}\\

\haiku{Het was tevens ons,.}{afscheidsgesprek al wisten}{wij dat toen nog niet}\\

\haiku{{\textquoteleft}Vergeet niet, dat ik!}{menige schuit door zware}{stormen geloodst heb}\\

\haiku{{\textquoteleft}Kop dicht en kom mee,.}{naar beneden dan zal ik}{het je bewijzen}\\

\haiku{{\textquoteleft}Het komt er toch niet,,...{\textquoteright} {\textquoteleft}}{meer op aan moeder hij was}{een gelukkig mens}\\

\section{Lodewijk Vleeschouwer}

\subsection{Uit: Het boek der vertellingen en andere kuizelarijen (onder ps. Reinaert de Vos)}

\haiku{Maar het ongeluk.}{had nog niet opgehouden}{hem te vervolgen}\\

\haiku{de honden moeten.}{u niet dikwijls onder den}{tand gehad hebben}\\

\haiku{De moeder ontstak,.}{een klein lampken en zette}{het op de tafel}\\

\haiku{Bazin, geeft daar eens, '.}{een ouweltjen en eenen}{knoop voort cachet}\\

\haiku{maar wat moeite hij,.}{ook nam er kwam geen druppel}{melk te voorschijn}\\

\haiku{maar ik wil toch de,}{schuld niet zijn dat gij in het}{ongeluk stort.-}\\

\haiku{Ik slijp de scharen,.}{en draai gezwind En hang mijn}{mantel naar den wind}\\

\haiku{- Het gaat u zeker,?}{zeer goed daar gij zoo lustig}{bij uw slijpen zijt}\\

\haiku{- Ja, antwoordde de,.}{schaarslijper het handwerk heeft}{een gouden bodem}\\

\haiku{Mijn twee matanten,,.}{moet gij weten Die waren}{geen van bei getrouwd}\\

\haiku{het is uitmuntend......}{om verkenskarbonaden}{in te braden}\\

\haiku{- De visch zwom weg, maar.}{kwam spoedig weer terug en}{wierp den ring aan land}\\

\haiku{Zoo tij ik op de,,....,!}{straat kwam de enveloppe}{af en raad eens}\\

\haiku{Twee honderd francs,,,,!}{zei mijn pere wel jongen}{dat is kapitaal}\\

\haiku{Elken burger der.}{stad zond hij een gebraad en}{eene kruik spaanschen wijn}\\

\haiku{Toen hij aanklopte,:}{vroeg de deurwachter met eene}{diepe forsche stem}\\

\haiku{Daarom zou ik wel,}{wenschen nog een paar uren te}{mogen slapen.-}\\

\haiku{En zoo, vaarwel.-.}{Frans draafde nu welgezind}{op Antwerpen toe}\\

\haiku{Op gindsch kasteel zijn,.}{kamers genoeg en ik heb}{er de sleutels van}\\

\haiku{Eindelijk werd de,;}{echte sleutel gevonden}{en het slot ging open}\\

\haiku{Mijn geest verliet het,.}{afgeteerde  lichaam}{maar bleef hier verband}\\

\haiku{- Zoo sprak hij, en kwam,,.}{verzadigd en dik eerst laat}{in den avond naar huis}\\

\haiku{- Dit zal u ook niet,,.}{bevallen zei de Kater}{het heet Heeluit}\\

\haiku{- Heb ik de stukken '?}{niet gelezen en ers}{nachts over nagedacht}\\

\haiku{- Als ik maar eens een,.}{mensch te zien kreeg ik zou hem}{toch wel aanvallen}\\

\haiku{- Nu, broeder Wolf, sprak,?}{de Vos hoe zijt gij met den}{Mensch uitgemeten}\\

\haiku{- Ja maar..... - En ge zijt,.}{te rechtvaardig om er niet}{in toe te stemmen}\\

\haiku{En ziet, hunne hoop,.}{bedroog hen niet want Jan kwam}{wezenlijk terug}\\

\haiku{Hij zag wel in, dat;}{hij hier een geheel ander}{leven moest leiden}\\

\haiku{Eens was hij met zijn,.}{vader in het woud gegaan}{om hout te vellen}\\

\haiku{tien of twintig van,}{hun doen mij niets als God niet}{tegen mij is.-}\\

\haiku{want het gaf een slag.}{als of er de donder op}{gevallen ware}\\

\haiku{Jan leerde spoedig,.}{hoe men het scheepstuig roer en zeil}{moest behandelen}\\

\haiku{- Verhoede God en,!}{mijn goede degen dat de}{prinses zou sterven}\\

\haiku{maar hij was slechts de,.}{zoon van eenen dorpschout en zij}{eene koningsdochter}\\

\haiku{- Nu, God zij dank, dat!}{gij weder daar zijt en een}{koning geworden}\\

\haiku{Of gij gaarne op,.}{dien armstoel zit of niet daar}{lachen wij mede}\\

\haiku{Weetde wel, dat ge?}{daar heel fraaie vraagskens op uwen}{program gezet hebt}\\

\haiku{- wel te verstaan, als.}{een lantaarnpaal een ander}{hart had dan van steen}\\

\haiku{- Op mijn woord van eer, -,.}{vervolgde ik tot mijne}{makkers sprekende}\\

\haiku{Eindelijk scheen de,;}{meester te denken dat ik}{genoeg had gehad}\\

\haiku{maar zij mochten mij,.}{toch niet lijden zoo jaloersch}{waren zij op mij}\\

\haiku{Welk een slag, voor een,!}{jongman die zulk een uniform}{bekostigen moest}\\

\haiku{Dobble bleef roerloos.}{op den kapblok zitten en}{huilde als een kind}\\

\haiku{Ik zegde niemand,, -?}{dat de hond eene zelfmoord had}{gepleegd waartoe dit}\\

\haiku{als gij er een hebt,.}{dan moest  gij er liever}{maar in gaan wonen}\\

\haiku{Hij schreef met veel zorg.}{het verlangde bewijs van}{ontvangst en ging heen}\\

\haiku{Had hij ze aan uw.}{geld blauw geteld dan zou het}{erger geweest zijn}\\

\haiku{Die snaak Waters was.}{op haar verliefd geworden}{en had haar getrouwd}\\

\haiku{- n'est-ce pas a?}{un homme comme il faut}{que j'ai affaire}\\

\haiku{- Bien du succ\`es, mon, -;}{cher Lovelace wierp hij hem}{als een vaarwel toe}\\

\haiku{Ge moet nochtans niet,.}{denken dat ze nu nog in}{die positie staan}\\

\haiku{- Je te livre le,. -?}{pigeon tu en feras ce}{que tu pourras Quand}\\

\haiku{vous avez le accent,... -,.}{aigiou le accent gr\`eve}{Grave milord}\\

\haiku{L\`a on ne trouve;}{que l'ignorance greff\'ee}{sur la sottise}\\

\haiku{monsieur ne sait?}{pas qu'il y a de grandes}{r\'ejouissances demain}\\

\subsection{Uit: Stukken en brokken}

\haiku{zy worden niet met,.}{de hand voortgeduwd maer door}{een peerd getrokken}\\

\haiku{ikke heb betael,;}{ikke vrag une voiture}{suppl\'ementaire}\\

\haiku{en hy bezag my,;}{met roedelyden omdat}{ik zoo lang werk had}\\

\haiku{want hy was even zoo,;}{gauw met zyn middagmael klaer}{als de anderen}\\

\haiku{en Ka{\"\i}n, hem de... -,?}{schurk Gelyk al de witte}{veronderstel ik}\\

\haiku{majoor, vertelt ons, -.}{dat eens werd er aen alle}{kanten geroepen}\\

\haiku{Nooit zult gy raden,.}{wat wy in den buik van het}{wangedrocht vonden}\\

\haiku{en als ik niet mis,.}{ben zal myn kunstgreep ons plaets}{genoeg bezorgen}\\

\haiku{Inderdaed, nu ik,,}{er op lette zag ik den}{tooneelspeler zoo haest}\\

\haiku{Toen de beweging,;}{gestild was speelde het stuk}{nog eenigen tyd voort}\\

\haiku{Dit bevonden wy,.}{toen wy ons van het schouwburg}{naer huis begaven}\\

\haiku{Zyn neus was deerlyk,!}{opgezwollen en zyn hoofd}{deed hem och arme}\\

\haiku{- Rydt naer Lancaster, -,;}{zeg ik u. Wel ik zal naer}{Lancaster ryden}\\

\haiku{Als de doctor de,,:}{tong gezien had zegde hy}{het hoofd schuddende}\\

\haiku{De minister had',.}{zyn tekst genomen uit St.}{Jans Evangelie kap}\\

\haiku{Hy hoort van onze,.}{inrigtingen gewagen}{en hy schokschoudert}\\

\haiku{men maekte een'';}{gemeenteraedsheer van hem of}{zelfs een minister}\\

\haiku{- Zeer wel, - zegde ik, -;}{ik zie dat ge reeds zeer sterk}{zyt in het latyn}\\

\haiku{Er moest een renloop,,.}{van peerden niet verre van}{de stad plaets grypen}\\

\haiku{Ik verhaelde hem,}{het tooneel dat ik bygewoond}{had en hy snelde}\\

\haiku{Ik verhaelde wat,:}{ik gezien had waerop de}{advokaet my vroeg}\\

\haiku{men aenzag haer als;}{beneden de weerdigheid}{van een beschaefd man}\\

\haiku{Deze afkeer was,}{zoo groot dat weinigen het}{mogelyk achtten}\\

\haiku{- De man die dat in,!}{zyn hoofd heeft gekregen is}{van onze eeuw niet}\\

\haiku{Waerin bestaet dan,?}{die fransche educatie die}{gy hebt ontvangen}\\

\haiku{Hoe ik met den held.}{van deze geschiedenis}{kennis maekte}\\

\haiku{ie bezien, had ik'.}{een makker nevens my op}{de bank gekregen}\\

\haiku{Nog verscheidene.}{malen ontmoette ik hem}{op de zelfde plaets}\\

\haiku{Met diepen weemoed,}{en het hart vol benauwdheid}{verliet ik myne}\\

\haiku{Het leven aldaer,}{vond ik geheel iets anders}{dan hetgene ik}\\

\haiku{In het begin ging.}{de zaek beter dan ik er}{my aen verwachtte}\\

\haiku{Nauwelyks was zy,.}{de deur uit of eene dienstmeid}{trad in den winkel}\\

\haiku{Het was de meid, die.}{vroeger een half pond suiker}{was komen koopen}\\

\haiku{men had myn' hoed naer,.}{huis gezonden en de meid}{had alles verteld}\\

\haiku{hoewel hy volstrekt':}{geene achterdocht wegen zyn}{toestand gevoelde}\\

\haiku{Moeijelyk zyn de,.}{tyden geweest in welke}{gy hebt geheerscht}\\

\haiku{Was alle kennis?}{en alle wetenschap niet}{in hem opgevat}\\

\haiku{maer Juno, de vrouw,.}{van Jupiter was er ook}{zot naer geworden}\\

\haiku{De eerste helft van,;}{dit woord is latyn en de}{andere helft ook}\\

\haiku{- Maer daertoe is het,,;}{fransch te verheven te edel}{en vooral te ryk}\\

\haiku{- Pos, - zei Jan, - en ze.}{schreef het woord anthropos op}{een stuksken papier}\\

\haiku{maer ge kunt misschien,.}{geen  grieksch genoeg om dit}{fransch woord te verstaen}\\

\haiku{quatre \`a point, deux,,;}{\`a quatre en zoo voorts tot}{vier en twintig toe}\\

\haiku{{\textquoteleft}Avignon is eene hel,;}{geworden de schuilplaets van}{alle gruwelen}\\

\section{Bert Voeten}

\subsection{Uit: Doortocht. Een oorlogsdagboek 1940-1945}

\haiku{Ook ditmaal heb ik.}{weer geruimen tijd in een}{greppel gelegen}\\

\haiku{Maar nu we het front,.}{hier vlak bij ons zien k\`an het}{niet lang meer duren}\\

\haiku{Even later hoort men.}{ze donkere roffels slaan}{op de pontonbrug}\\

\haiku{een van de zangers.}{zeggen en dan zetten zij}{het Wilhelmus in}\\

\haiku{Het was of alles.}{wachtte op het gedonder}{der luchteskaders}\\

\haiku{{\textquoteright} {\textquoteleft}Gelukkig niet{\textquoteright}, zei.}{ik en we gaven elkaar}{een knipoogje}\\

\haiku{Ik zag die wanhoop,.}{in haar oogen toen zij naast mij}{ging over het Damrak}\\

\haiku{{\textquoteleft}Joodsche gasten niet{\textquoteright},.}{gewenscht hing bij de deur van}{een caf\'etaria}\\

\haiku{Mijn voeten hield ik.}{stijf tegen elkaar als om}{mij schrap te zetten}\\

\haiku{De angst woont achter,.}{geblindeerde ramen in}{gangen en sloppen}\\

\haiku{Seyss heeft de kaarten,.}{op tafel gelegd langzaam}{en nadrukkelijk}\\

\haiku{{\textquoteleft}Wij moeten tot het.}{laatste lid uiteengerukt}{en verstrooid worden}\\

\haiku{Ik bemerkte dat.}{zijn spot hem evenzeer wondde}{als hij het mij deed}\\

\haiku{Geuren van versch hooi.}{en windbewogen water}{woeien ons tegen}\\

\haiku{De boerin schonk ons,.}{koffie in wijde witte}{kommen zonder oor}\\

\haiku{De {\textquoteleft}claque{\textquoteright} zette.}{in en vijf minuten lang}{brulde de kelder}\\

\haiku{De deuren gaan open....}{en we komen achter een}{sleepboot te hangen}\\

\haiku{Misschien rijten de.}{rotspieken van verzonken}{bergketens haar open}\\

\haiku{Wat er van onze,.}{vloot nog over was is bij dit}{treffen vernietigd}\\

\haiku{Nieuw-Guinea moet.}{hun springplank naar het vijfde}{werelddeel worden}\\

\haiku{Toen ik naar den trein,.}{ging keken de verkoolde}{dakspanten me na}\\

\haiku{Hoe langer hoe meer.}{gaan de omstandigheden}{op mij invreten}\\

\haiku{Aan het station.}{probeerden vrouwen door het}{cordon te komen}\\

\haiku{Maar die velen zijn,.}{stil verslagen en ontzet}{naar huis gegaan}\\

\haiku{En in den nacht zag.}{ik weer de lichten dooven op}{het emplacement}\\

\haiku{16 Juni Kort zijn,.}{de dagen al reikt het licht}{tot ver in den avond}\\

\haiku{{\textquoteleft}Denk je werkelijk,?}{dat Engeland het z\'o\'o ver}{zal laten komen}\\

\haiku{Alleen maar om weer;}{eens te kunnen praten met}{een ouden makker}\\

\haiku{Hij zag de oogen van.}{de donkere vrouw aan de}{bar van Caliente}\\

\haiku{Voor \'e\'en nacht vergeet.}{je den heelen beestenboel}{met zoo iets naast je}\\

\haiku{Dat stadium is.}{bij Leningrad al maanden}{geleden bereikt}\\

\haiku{Mussert werd door Seyss {\textquoteleft}{\textquoteright}.}{benoemd totleider van het}{Nederlandsche volk}\\

\haiku{Ik dacht, dat u het.}{op het laatste  moment}{niet aangedurfd had}\\

\haiku{{\textquoteleft}Je moet wel dwaas zijn,.}{als je je zonder meer aan}{dat tuig uitlevert}\\

\haiku{Hoe goed was het nog,,.}{amper een week geleden}{toen je bij me was}\\

\haiku{Ze zullen wel gauw,.}{opengaan want met een schop heb}{je nog nooit gewerkt}\\

\haiku{In de doos van mijn.}{kamer werden beelden en}{geluiden anders}\\

\haiku{Dan kwamen er twee;}{die met stukken hout in een}{blikje rammelden}\\

\haiku{Wat achterbleef is:}{een jong mensch die in zijn geest}{alleen bewaard heeft}\\

\haiku{er waren leeraren,:}{machinisten en weer twee}{letterkundigen}\\

\haiku{Ik hoorde hoe zij,.}{gevlucht was van het eene adres}{naar het andere}\\

\haiku{11 November Het.}{Roode Leger rukt op over}{het geheele front}\\

\haiku{Maar de soldaat kan?}{immers in het kansspel van}{den veldslag winnen}\\

\haiku{Deze tijd laat zich.}{slechts verbannen voor den duur}{eener omhelzing}\\

\haiku{Hij rukt mij los uit {\textquoteleft}{\textquoteright}.}{de Tweede Wereld met een}{enkel brandend feit}\\

\haiku{Duizend geruchten.}{leven een wisselend doch}{hardnekkig bestaan}\\

\haiku{grenzen, steden en,.}{landschappen uitwisschend tot}{chaos brandend}\\

\haiku{16 Juni Tot den.}{avond met zware hoofdpijnen}{te bed gelegen}\\

\haiku{De zwaarte van de,.}{kogels trok mijn bovenlijf}{omlaag steeds dieper}\\

\haiku{Gr\"une, Gestapo.}{en S.D. maken elken dag}{nieuwe slachtoffers}\\

\haiku{Fransche hoeren en.}{verraders trokken met den}{vluchtenden mof mee}\\

\haiku{Maar de Duitscher geeft, {\textquoteleft}{\textquoteright}.}{toe dat hij zijn troepen heeft}{teruggenomen}\\

\haiku{De gruwelen van.}{den t\`echnischen krijg zullen}{intenser worden}\\

\haiku{Een enkele maal.}{profiteeren we van het}{kaarslicht bij vrienden}\\

\haiku{En Goebbels haalt er.}{uit wat er agitatorisch}{uit te halen valt}\\

\haiku{Ze krijgen dien mof{\textquoteright},.}{er toch maar niet onder deed}{vandaag weer opgeld}\\

\haiku{De oude gevels.}{blikken meewarig in den}{triesten schemeravond}\\

\haiku{stootend en bassend.}{en gillend kondigden zij}{het nieuwe uur aan}\\

\haiku{Gloeiende lava...}{schuift steden voor zich uit of}{het schaakstukken zijn}\\

\haiku{Het medelijden,,.}{dat diep in je woont krijgt geen}{kans op te komen}\\

\haiku{Straks vervallen we.}{tot een staat van volmaakte}{onverschilligheid}\\

\haiku{Gisteren knalden.}{hier op de Nieuwmarkt nog de}{pistolen der groenen}\\

\haiku{15 April Roosevelts.}{overlijden is een slag voor}{de democratie}\\

\section{Theun de Vries}

\subsection{Uit: De freule. De bijen zingen}

\haiku{Over de vader van.}{het ongeboren kind werd}{niet meer gesproken}\\

\haiku{De andere hand.}{van de freule gleed door de}{zijsleuf van haar rok}\\

\haiku{Het rumoer om de.}{doop van Justus Wiarda}{wilde niet sterven}\\

\haiku{Ze zweefde boven.}{de beklemmende volte}{van de kerkdiensten}\\

\haiku{Het scheen veeleer, of.}{Bely haar misstap met de}{dood bezegeld had}\\

\haiku{Hester wordt ouder,{\textquoteright},.}{zeiden de boden die haar}{van nabij zagen}\\

\haiku{Toen zag Justus haar;}{mondhoeken smartelijk en}{verwonderd beven}\\

\haiku{Justus kleurde en.}{veegde de handen schoon aan}{zijn ruwe hozen}\\

\haiku{Toen zag hij ook, dat,.}{de schaduw een vrouw was al}{leek dat eerst niet zo}\\

\haiku{Hij zag de verre;}{ruitergestalte voor zich}{uit op het vrije veld}\\

\haiku{Op de grond lagen,,.}{scherven her en der gespat}{beslagen met smook}\\

\haiku{Hester Wiarda.}{ontwaakte niet meer tot het}{gewone leven}\\

\haiku{Dan luisterden de.}{bewoners van de state}{met een huivering}\\

\haiku{Zij haalden er het,.}{bezit vandaan dat de mens}{eigenwaarde geeft}\\

\haiku{De anderen zien:}{verwonderd naar de boer en}{houden op met eten}\\

\haiku{Niemand let op de,,.}{grootmoeder die alleen zit}{in haar stoel en zwijgt}\\

\haiku{{\textquoteleft}Geen gasten om te,,.}{drinken Aesger Wiarda}{en geen wigeters}\\

\haiku{Als scherpe stekels.}{vliegen de eerste splinters}{rechtop uit het hout}\\

\haiku{Haar handen liggen.}{gekromd en roerloos op de}{leuning van de stoel}\\

\subsection{Uit: 60. Keuromnibus}

\haiku{David verdiepte;}{zich niet in het intieme}{leven der blanken}\\

\haiku{veranderingen.}{waren desondanks voor hem}{en allen op til}\\

\haiku{daarop verdwenen.}{de twee onbekenden snel}{over de terreingolving}\\

\haiku{er ging alleen een.}{onbehaaglijk deinen door}{hun gelederen}\\

\haiku{de nacht zoog alle;}{geluiden van hemel en}{zee en eiland in}\\

\haiku{Ze schreed met een zwaard,}{in de ene een fakkel in}{de andere hand.}\\

\haiku{ze was de vrijheid -.}{van de negers het einde}{van de willekeur}\\

\haiku{Het nerveus gebaar.}{liet de ganzeveren pen}{op de grond rollen}\\

\haiku{David zag hem een:}{der laden opentrekken en}{deed een stap terug}\\

\haiku{Maar hij had geen zin;}{terug te keren naar de}{keuken vol vrouwen}\\

\haiku{Ik nodig je uit;}{morgen in de voormiddag}{bij mij te komen}\\

\haiku{Massou keerde zich,.}{om Scipion trok hem een}{paar pas ter zijde}\\

\haiku{Voor de tweede maal.}{op die dag vroeg men hem of}{hij betrouwbaar was}\\

\haiku{Ofschoon je misschien.}{niet met al onze plannen}{op de hoogte bent}\\

\haiku{Justin was alleen.}{zwijgzaam en bleek en David}{wist nu de reden}\\

\haiku{{\textquoteright} M. Beaupr\'e liet zijn.}{hagelwit gebit in een}{hernieuwd lachen zien}\\

\haiku{{\textquoteleft}Monsieur, ik zie,....}{in dat ik verkeerd deed ik}{kon het niet helpen}\\

\haiku{{\textquoteleft}Zo zie je, wat het...}{weldoen van deze zwarte}{schurken oplevert}\\

\haiku{M. Falconet scheen;}{te voelen dat zijn laatste}{opmerking doel trof}\\

\haiku{maar zoveel kan ik{\textquoteright} ():}{je wel zeggenzijn stem rees}{triumfantelijk}\\

\haiku{Was M. Malenfant?}{feitelijk niet voor hem in}{de bres gesprongen}\\

\haiku{maar het opschuiven.}{naar de zijwaarts gestrekte}{tak was moeilijker}\\

\haiku{Hij voelde Justins blik,.}{krampachtig op zijn gezicht}{dwingend om antwoord}\\

\haiku{klaarblijkelijk had.}{men daardoor de bagage}{naar buiten gesleept}\\

\haiku{Maar hij bedacht dat;}{de brand niet bij het koetshuis}{zou blijven stilstaan}\\

\haiku{Een van de spelers;}{zat met het hoofd tussen de}{schouders gedoken}\\

\haiku{Massou's gezicht was.}{star op het gezicht van de}{zingende gericht}\\

\haiku{zij wankelde, haar,;}{armen zakten haar zingen}{brak middenin af}\\

\haiku{II David voelde.}{hoe een naakte voet hem in}{de ribben schopte}\\

\haiku{Het duurde maar kort.}{of er kwamen enkele}{negers uit het bos}\\

\haiku{{\textquoteright} David wilde een,.}{kreet slaken een beweging}{van verzet maken}\\

\haiku{{\textquoteright} zei David, schor van, {\textquoteleft}!}{een nieuwe aandoeningmaak}{mijn handen toch los}\\

\haiku{David voelde zich;}{opnieuw voortgestompt en liet}{zich willig drijven}\\

\haiku{Op dit ogenblik kwam,:}{de schildwacht terug en zei}{naar Massou gekeerd}\\

\haiku{{\textquoteright} {\textquoteleft}Men spreekt mij niet aan...}{alsof ik een royalist}{en een planter ben}\\

\haiku{Maar ik herinner;}{me het aantal zwarten hier}{ruw genomen wel}\\

\haiku{Ik heb nooit iets om,.}{mijn meester gegeven of}{de vrijheid veracht}\\

\haiku{Eindelijk boog zich,:}{Massou naar hem over schraapte}{zacht zijn keel en vroeg}\\

\haiku{En iedereen weet.}{nu dat wij beiden zonen}{van Boucman zijn}\\

\haiku{Scipion had ze,.}{en ik heb ze aan massa}{Hugues gegeven}\\

\haiku{er was geen ruimte;}{voor de tegenstanders meer}{om zich te weren}\\

\haiku{David wendde de,.}{blik af van het bebloede}{vertrokken gezicht}\\

\haiku{Hij deed het als had.}{hij de hele ochtend op}{dit ogenblik gewacht}\\

\haiku{Er lagen messen,,;}{naast hen op stokken gesnoerd}{een enkel pistool}\\

\haiku{Hij keek de kamer;}{niet in voor alle luiken}{geopend waren}\\

\haiku{Ik zal mijn best doen,.}{het u aan de rest niet te}{laten ontbreken}\\

\haiku{Er bestaan veel te -?}{weinig afbeeldingen van}{onze West niet waar}\\

\haiku{{\textquoteright} zei David bitter, {\textquoteleft} '.}{omdat ik int geheel}{niet voor hem bestond}\\

\haiku{Hugues bleef voor de.}{kaart staan en sloeg er driftig}{met de hand tegen}\\

\haiku{als zij hun zaken.}{besproken hebben is het}{tijd voor het souper}\\

\haiku{{\textquoteleft}Hm,{\textquoteright} zei hij, {\textquoteleft}als je,...}{toch drinkt kunnen we het net}{zo goed samen doen}\\

\haiku{Het was niet in de.}{eerste plaats angst die hem dreef}{om licht te maken}\\

\haiku{{\textquoteleft}Ja,{\textquoteright} zei ze, de blik, {\textquoteleft}.}{op David richtendzo zag}{de kamer er uit}\\

\haiku{Hij droeg de blauwe.}{jas niet die de blanken hem}{gegeven hebben}\\

\haiku{Na die ene avond had.}{David evenmin meer iets van}{Perle vernomen}\\

\haiku{Wij zijn het die haar,.}{moeten veranderen door}{zelf anders te zijn}\\

\haiku{Hij stond naast haar en.}{wees haar alles zoals men}{een kind iets uitlegt}\\

\haiku{{\textquoteleft}Maar we zouden weer,.}{vijanden kunnen worden}{als je niet oppast}\\

\haiku{{\textquoteleft}Maar ik wist niet dat.}{de vaudoux jou tot koning}{gekozen hadden}\\

\haiku{Jij wist het niet, en,,.}{Bastiat die een zeldzaam}{mens was wist het niet}\\

\haiku{Er is er \'e\'en bij,.}{die ik in mijn  leven}{niet weer hoop te zien}\\

\haiku{David herkende,.}{het eerst de lange zorgzaam}{geklede Chr\'etien}\\

\haiku{David hief ook het,.}{glas maar minder overmoedig}{dan de anderen}\\

\haiku{Ik geloof niet dat,}{een kunstenaar dat ooit van}{zichzelf kan zeggen}\\

\haiku{Ik moet nog heel veel,...}{leren voor ik een meester}{in de kunst kan zijn}\\

\haiku{Perle was volmaakt,.}{kalm en wat zij deed deed zij}{trouw aan een bevel}\\

\haiku{{\textquoteright} Scipion wreef zich,.}{langs de wang waarbij hij de}{ene mondhoek optrok}\\

\haiku{Als ze gegrepen.}{worden maken ze grote}{kans op het schavot}\\

\haiku{Af en toe bleef hij,.}{staan hield zich de zijden vast}{en schudde het hoofd}\\

\haiku{Het stemde hem niet.}{gerust terug te vallen}{op zijn verachting}\\

\haiku{Hij hees zich op toen,.}{de magere strogele}{jongen voor hem stond}\\

\haiku{{\textquoteleft}Het schijnt niet, idioot,,{\textquoteright}:}{het is waarna hij zichzelf}{had toegedronken}\\

\haiku{Kankrin bemerkte:}{dat Gleb Michailowitsj hem van}{terzijde opnam}\\

\haiku{Monumentale,.}{jonge kerel u moet hem}{zich herinneren}\\

\haiku{Hij zwaaide een keer.}{met de hoed alvorens bij}{die weer opzette}\\

\haiku{{\textquoteleft}Maar George - als,!}{dat goddome zo is moet}{je naar de dokter}\\

\haiku{geklodderd als bij, '.}{een kind dat voort eerst een}{penseeltje vasthoudt}\\

\haiku{{\textquoteleft}Afgezien nog van,,.}{de vraag of het geen schep geld}{heeft gekost dat raam}\\

\haiku{van Theo natuurlijk,).}{jij zult nooit een cent met je}{geknoei verdienen}\\

\haiku{ze hadden de kleur;}{en bijna de vorm van oud}{knoestig geboomte}\\

\haiku{Est-ce que,}{je ne m\'erite pas mon}{pain parce que je}\\

\haiku{Hij was zijn eigen,.}{leermeester geworden nu}{niemand hem meer hielp}\\

\haiku{In het hospitaal;}{drong de zomerhitte langs}{de gangen binnen}\\

\haiku{Wat schokte hem, wat?}{vonkte in zijn blik als een}{onveilig signaal}\\

\haiku{naar het atelier, de,.}{omgeving het nieuwe huis}{keek hij nog steeds niet}\\

\haiku{{\textquoteleft}Ik geef het toe, broer -;}{het leven was niet bijster}{moederlijk voor je}\\

\haiku{Hij drukte die kort,,.}{vermanend nu Vincents stem}{weer knersend aanzwol}\\

\haiku{Ik zou gelukkig, -.}{willen zijn op hun manier}{Theo maar ik kan niet}\\

\haiku{Al was er geen peil,.}{op te trekken waarheen dit}{alles zou leiden}\\

\haiku{Goed, ik ben weer in,.}{actie ik zal nog veel}{actiever worden}\\

\haiku{Ik wil de mensen,,.}{bereiken hen bew\'egen}{hun harten raken}\\

\haiku{Theo had er ditmaal,.}{om gevraagd hij was er in}{geen jaren geweest}\\

\haiku{De kleur vliegt tussen...}{je vingers door terwijl je}{haar denkt te grijpen}\\

\haiku{{\textquotedblright} en ze jankte en,,,:}{zei dat ze haar best zou doen}{en ik zweer je Theo}\\

\haiku{Uit een zijpad in.}{de Bosjes kwam het geluid}{van een belletje}\\

\haiku{{\textquoteleft}Nu, zoek dat dan maar...,{\textquoteright}.}{uit wat Vincent al bijna}{weer kwaad had gemaakt}\\

\haiku{{\textquoteright} Carolus stond nog,.}{op dezelfde plek leunend}{op de paraplu}\\

\haiku{de marines, de,,.}{straatschetsen de bosstudies}{de olieverfschetsen}\\

\haiku{Zij dachten daar in...}{Nuenen dus niet enkel in}{vijandschap aan Sien}\\

\haiku{vooral Carolus -;}{stelde zich daarbij weer aan}{als rijkemans aap}\\

\haiku{Hij riep Maria aan;}{tafel en zette haar de}{aardappelen voor}\\

\haiku{het ogenblik waarop,.}{het jaar 1882 in het graf zou}{glijden naderde}\\

\haiku{hij luisterde met;}{enige angst of Maria soms}{wakker zou worden}\\

\haiku{Theo zal haar daaraan.}{laten opereren zodra}{ze is aangesterkt}\\

\haiku{Vincent zelf at vaak,.}{al minder om haar meer te}{kunnen opscheppen}\\

\haiku{Hij had Theo voor de.}{zoveelste maal om extra}{hulp moeten vragen}\\

\haiku{alles was hem zwart,,}{voor ogen geworden hij gleed}{een afgrond binnen}\\

\haiku{Ja, een home had,.}{hij al leek het vaak meer op}{een schip in noodweer}\\

\haiku{Zij scheen ten slotte,.}{voorbestemd te zijn Theo's}{leven te delen}\\

\haiku{Hij lag vaak wakker,;}{met de vrees dat hij niet door}{de muur kon breken}\\

\haiku{En als ik Sien nou...{\textquoteright}.}{in huis nam Vincent rukte}{zich eindelijk los}\\

\haiku{Hij schoof het kussen,.}{onder haar hoofd en wachtte}{tot zij bedaard was}\\

\haiku{Vincent was vaak al,.}{bij het eerste licht op de}{weg en tekende}\\

\haiku{Vincent kreeg al in,.}{de trein maagpijn de overvloed}{was te groot geweest}\\

\haiku{Het kostte hem twee.}{etmalen om weer in het}{gareel te komen}\\

\haiku{wij schilders zijn net...}{zo grillig in ons wel en}{wee als de vrouwen}\\

\haiku{{\textquoteright} Breitner  keek hem,.}{nog steeds verbaasd aan tot hij}{lichtjes grinnikte}\\

\haiku{Wat vond oom ervan,?}{was er in Amsterdam voor}{zo iets geen afzet}\\

\haiku{Hij ging met het pak.}{onder de arm op loden}{schoenen naar de Plaats}\\

\haiku{Zijn ingewanden.}{krompen samen van vrees voor}{het onbekende}\\

\haiku{turfdragers - dus De!}{Bock had ook die veenderij}{in het duin ontdekt}\\

\haiku{het beste is dat...}{soort maar te belazeren}{voor ze het jou doen}\\

\haiku{deze armoede!}{is een dorenheg waaraan}{ik mij kapot schram}\\

\haiku{Theo kwam naast hem staan,.}{een tengere stille hand}{op Vincents schouder}\\

\haiku{{\textquoteleft}Ik blijf kort deze,,.}{keer Vinc morgenavond ga ik}{naar Nuenen terug}\\

\haiku{Maar Brabant kan mij.}{plotseling verschijnen als}{het rustgevende}\\

\haiku{Ik besef het en...{\textquoteright}.}{ik verzet me nog steeds Theo}{trok hem naar zich toe}\\

\haiku{Maar men jaagt niet naar,.}{de straat terug wat men van}{de straat heeft gered}\\

\haiku{Hij zag er dubbel:}{sterk en gezond uit met zijn}{diep gebruinde huid}\\

\haiku{Vincent begon Van:}{der Weele opnieuw uit te}{vragen over Drente}\\

\haiku{hij wendde de ogen.}{weg voor de aanblik van die}{bleekrode open mond}\\

\haiku{Hij herinnerde,.}{zich zijn kinderangst voor de}{nacht lang geleden}\\

\haiku{{\textquoteright} De bazige vrouw.}{snipte verachtelijk twee}{vingers langs elkaar}\\

\haiku{er kwam een eigen,.}{sentiment in de stukken}{die hem weer hoop gaf}\\

\haiku{sommigen kwamen;}{om hem heen staan als hij zijn}{aquarellen maakte}\\

\haiku{{\textquoteleft}Een vrouw van dertig.}{wordt niet meer door vader of}{moeder gegeven}\\

\haiku{De illusie, dat,.}{zij elkaar pas gevonden}{hadden werd sterker}\\

\haiku{Sien had hem al tot,.}{zich genomen terwijl zijn}{trots zich nog weerde}\\

\haiku{{\textquoteleft}We zien elkaar dus,.}{nog aan het station als}{ik naar Drente ga}\\

\haiku{{\textquoteleft}Als ik hier in de,;}{Wildhoek kom kan ik Lolkje}{niet voorbijlopen}\\

\haiku{ja, de laatste maal.}{had de mijnheer Wigle niet}{eens meer ontvangen}\\

\haiku{en het eerste jaar.}{wou hij ook nog gratis mest}{over het nieuwe land}\\

\haiku{Alle Wildhoekster;}{jongens en meisjes kenden}{elkaar van die school}\\

\haiku{wij van dat goedje,;}{hebben want wanbetalers}{zijn het dikwijls ook}\\

\haiku{Maar meteen waren;}{ook weer de hoogmoed en het}{boos verzet in haar}\\

\haiku{{\textquoteright} Op Anders' lippen,:}{drong de vraag zo vaak hij langs}{Jannina zwenkte}\\

\haiku{{\textquoteright} begon Anders, en:}{onverhoeds daalde zijn stem}{schor en verbeten}\\

\haiku{daarachter rezen;}{als pluimpjes stilstaand gras de}{laatste restjes bos}\\

\haiku{{\textquoteleft}Jongens, is het waar,?}{dat Floris Hoogwolda werk}{maakt van Jannina}\\

\haiku{het water liep traag.}{van zijn gezicht terwijl hij}{naar hen luisterde}\\

\haiku{en toen geen van hen:}{meer iets te vertellen had}{lachte hij en zei}\\

\haiku{Het was duidelijk.}{dat hij op de aftocht van}{de oude wachtte}\\

\haiku{zij, Jannina, met, -?}{de rug tegen de deurpost}{haar ogen in de nacht}\\

\haiku{{\textquoteright} Jannina's adem bleef,.}{een oogwenk dralen daarop}{rukte ze zich los}\\

\haiku{Maar als ik zeg dat,.}{ie beter was dan jij dan}{spreek ik de waarheid}\\

\haiku{Hij zwaaide met de,,:}{korte armen wenkte om}{stilte betoogde}\\

\haiku{en toen ging Fok naar '...}{de voorkamer en hij kwam}{terug mett geld}\\

\haiku{Hij draaide zich om,;}{Floris mompelde ook iets}{dat op een groet leek}\\

\haiku{Zij zaten er in;}{het smalle reepje gras dat}{langs het water liep}\\

\haiku{Ze stapte langs de;}{eerste groep die haar zwijgend}{opnam en doorliet}\\

\haiku{In de stoffige.}{hete avondval stonden de}{turfstekers bijeen}\\

\haiku{Het lachen waarmee.}{de arbeiders hem groetten}{had iets goedmoedigs}\\

\haiku{De veldwachter liep,.}{drie vier passen met de stoet}{mee en bleef weer staan}\\

\haiku{Soms ging ze, als nam,.}{ze een onverhoeds besluit}{naast hem mee verder}\\

\haiku{Onder de hoge.}{bomen van de oprit bleef}{men voor het eerst staan}\\

\haiku{Ze kwamen in breed.}{gelid aandringen en nu}{stokten ze niet meer}\\

\haiku{Wybren Post vocht.}{tevergeefs om Jan Herder}{tegen te houden}\\

\haiku{Hij lag 's nachts niet,;}{van ellende op zijn brits}{te vloeken maar sliep}\\

\haiku{'t is geen zomer,{\textquoteright}.}{meer zei een stem vanachter}{het kale buffet}\\

\haiku{alles staat verkeerd,,,.}{op zijn kop een vergissing}{een bespotting}\\

\haiku{Hij zat nog stil toen,.}{zij het verhaal eindigde}{en dacht moeizaam na}\\

\haiku{we moeten met man...}{en macht ons recht op een stuk}{brood verdedigen}\\

\haiku{Kankrin ziet dan in.}{dat hij en zijn soort aan de}{dood vervallen zijn}\\

\haiku{Het werk werd voor mij.}{onder het schrijven het boek}{der ontdekkingen}\\

\haiku{In 1938 publiceert,}{hij Het rad der fortuin en}{in dat zelfde jaar}\\

\subsection{Uit: Noorderzon}

\haiku{mismaakte priesters,}{duldt de grote Alvader}{niet hij zelf is trots}\\

\haiku{Wierd Langskonk vroeg zijn;}{zoon nooit naar de waarheid van}{al dit beweren}\\

\haiku{Ik heb twee zonen,.}{verloren ik neem Rikelt}{aan in st\^ee van hen}\\

\haiku{Richt en Una stonden.}{bij de omvreding van de}{sate als beelden}\\

\haiku{En een enkele,,:}{zeer schrandere voegde er}{in zichzelf aan toe}\\

\haiku{Sommigen bouwden;}{mettertijd een eigen hut}{of onderkomen}\\

\haiku{Ywert was vrijwel zo,,.}{groot als Una zelf mager van}{bouw maar schouderbreed}\\

\haiku{en het kwam de smid,.}{voor dat ze dunner was en}{vol schroomvalligheid}\\

\haiku{Ze keerde naar huis,.}{terug nog ontdaner dan}{ze gekomen was}\\

\haiku{Toen de dag doorbrak,;}{stond het hele gewest in}{zijn landholten blank}\\

\haiku{gedurende die.}{blinde nacht dacht zij lange}{tijd aan niets anders}\\

\haiku{Zij baggerden door.}{blauwe zeedrab terug naar}{hun vernield bezit}\\

\haiku{Toen richtte zij zich,.}{op zodat alleman haar}{gezicht goed kon zien}\\

\haiku{Vrouwen en maagden,:}{breng palen en puntstokken}{en leg vuren aan}\\

\haiku{Er was \'e\'en van hen,,;}{een ouder man die dienst scheen}{te doen als priester}\\

\haiku{{\textquoteleft}Mij is aan Gabbe!}{noch aan welke kruisslaaf ook}{maar d\'at gelegen}\\

\haiku{Ze bespeurde de,;}{weifeling in Ingele's}{hand \'e\'en kort ogenblik}\\

\haiku{Una keerde zich in.}{de richting van de westwaarts}{gewentelde zon}\\

\haiku{Een ogenblik kwam het,;}{haar voor dat zij dit alles}{eerder beleefd had}\\

\haiku{{\textquoteright} Una stremde met een;}{wijd gebaar de woeling en}{het vormloos geschreeuw}\\

\haiku{{\textquoteleft}Als Thorlik jou tot,{\textquoteright}, {\textquoteleft}?}{vrouw nam zei zezou je hem}{als man begeren}\\

\haiku{{\textquoteright} Ingele hoestte,.}{weer en vertaalde grommend}{en met tegenzin}\\

\haiku{{\textquoteright} In het zeegewest.}{keerde men terug naar de}{hoeven en het werk}\\

\haiku{maar Orp zelf liet zich;}{de gunst van de wijze vrouw}{graag welgevallen}\\

\haiku{zij overlaadde hem,,.}{met sieraden kleren en}{geld als een edeling}\\

\haiku{Over de vader van.}{het ongeboren kind werd}{niet meer gesproken}\\

\haiku{De andere hand.}{van de freule gleed door de}{zijsleuf van haar rok}\\

\haiku{Het rumoer om de.}{doop van Justus Wiarda}{wilde niet sterven}\\

\haiku{{\textquoteleft}Hester wordt ouder{\textquoteright},,.}{zeiden de boden die haar}{van nabij zagen}\\

\haiku{Toen zag Justus haar;}{mondhoeken smartelijk en}{verwonderd beven}\\

\haiku{Justus kleurde en.}{veegde de handen schoon aan}{zijn ruwe hozen}\\

\haiku{Toen zag hij ook, dat,.}{de schaduw een vrouw was al}{leek dat eerst niet zo}\\

\haiku{Hij zag de verre;}{ruitergestalte voor zich}{uit op het vrije veld}\\

\haiku{Op de grond lagen,,.}{scherven her en der gespat}{beslagen met smook}\\

\haiku{Hester Wiarda.}{ontwaakte niet meer tot het}{gewone leven}\\

\haiku{Dan luisterden de.}{bewoners van de state}{met een huivering}\\

\haiku{Zij haalden er het,.}{bezit vandaan dat de mens}{eigenwaarde geeft}\\

\haiku{De anderen zien:}{verwonderd naar de boer en}{houden op met eten}\\

\haiku{Niemand let op de,,.}{grootmoeder die alleen zit}{in haar stoel en zwijgt}\\

\haiku{{\textquoteleft}Geen gasten om te,,.}{drinken Aesger Wiarda}{en geen wigeters}\\

\haiku{Als scherpe stekels.}{vliegen de eerste splinters}{rechtop uit het hout}\\

\haiku{Haar handen liggen.}{gekromd en roerloos op de}{leuning van de stoel}\\

\subsection{Uit: Het rad der fortuin}

\haiku{En Tjalling wist dat.,.}{Zo is de loop der dingen}{bedoelde Reinou}\\

\haiku{D\'aarom en daarom,.}{all\'een duldt ze hem hier op}{de begrafenis}\\

\haiku{Het staat voor Tjalling's,.}{ogen hij ziet zich opnieuw op}{de stelling zitten}\\

\haiku{Ze draaide zich op.}{de hakken om en verdween}{met lichte voeten}\\

\haiku{De domin\'e stond,.}{aan het hoofdeinde van de}{kist hoog opgericht}\\

\haiku{Op de gebulte}{diep-doorploegde inrit}{van het erf wachtte}\\

\haiku{- En Ekke... voor den.}{jongen zal ik dan een plaats}{bij den boer zoeken}\\

\haiku{Korzelig en met.}{spijt over zijn geldbelofte}{greep hij naar zijn pet}\\

\haiku{De winterlucht was,.}{geselend scherp ze wekte}{Tjalling nog eenmaal}\\

\haiku{Hij deed zijn zaken,,.}{en hij hielp mee in het werk}{al naar het seizoen}\\

\haiku{Het was een coup\'e;}{vol mannen in hun beste}{lakense pakken}\\

\haiku{waarheen het eerst te,,...}{gaan en hoe en of men bij}{elkaar zou blijven}\\

\haiku{Hij bleef des Vrijdags.}{langer in Leeuwarden en}{dronk meer dan voorheen}\\

\haiku{De scheidslijn met de.}{oude geboortestreek sneed}{dieper en dieper}\\

\haiku{Waar geld met geld zich{\textquoteright} -;}{paart zeiden de ouden en}{wijzen van de streek}\\

\haiku{Haar ademhaling in.}{hetzelfde bed lokte toch}{bij tussenpozen}\\

\haiku{de stad, offertes,,, -.}{beurs caf\'e partij biljart}{tot hij weer thuiskwam}\\

\haiku{hij praatte vaag over,.}{zijn zaak hij liep met Tjalling}{de boerderij door}\\

\haiku{Hij verlangde naar,.}{Rudmer met de weke zorg}{van ouderen broer}\\

\haiku{- Rudmer van Tjalling,;}{en Reinou wordt domin\'e}{zei men in de streek}\\

\haiku{- hij had de geest van,.}{het menniste volk hij zou}{zijn Meester vinden}\\

\haiku{emmers stonden op,.}{een plank die juist boven het}{water uitreikte}\\

\haiku{Hij gleed langs de wand,}{in het water de emmer}{nog steeds in de hand.}\\

\haiku{Hij schaamde zich en.}{was heimelijk trots op zijn}{nieuwe heldendom}\\

\haiku{Rudmer bemerkte, {\textquoteleft}{\textquoteright};}{datViet niet bizonder in}{tel was bij de rest}\\

\haiku{als je eerlijk wilt.}{zijn en de werkelijkheid}{onder het oog zien}\\

\haiku{- misschien word ik het...,...}{nog eens als ik de moed houd}{consequent te zijn}\\

\haiku{ik dacht 'n moment, ' -}{dat je heit metn heerschap}{uit de stad thuis kwam}\\

\haiku{aarzelende, half,;}{onderdrukte van Tjalling}{opener van Reinou}\\

\haiku{Hij lachte, en gaf:}{Rudmer een broederlijke}{klap op de schouder}\\

\haiku{- Kijk niet meer naar haar,....}{mijn jongen je moet straks naar}{Groningen terug}\\

\haiku{Rudmer vernam, dat}{zij alleen nog maar enige}{twijfel koesterden}\\

\haiku{kom, 't is zulk mooi,... -?}{weer ik neem de fiets En je}{ging naar Oostermeer}\\

\haiku{- Ze keerde zich naar,.}{hem toe zonder hem nog aan}{te durven kijken}\\

\haiku{Als ze verdomme,...!}{dan maar inzien dat ze geen}{vat op mij hebben}\\

\haiku{Antje's gezicht was.}{gaan gloeien bij de laatste}{woorden van Herre}\\

\haiku{Zijn knie\"en knakten,;}{de ouderdom zat al diep}{in zijn gebeente}\\

\haiku{- Zo -, zei hij, langzaam,,.}{en het drong tot hem door dat}{hij niet verheugd was}\\

\haiku{Totdal hij niet meer,:}{antwoordde op haar bange}{verwachtende vraag}\\

\haiku{Een man, die getrouwd, - -: - '? -.}{was en die Haastig opnieuw}{Kende Memm Nee}\\

\haiku{Het hield hem zozeer,.}{bezig dat hij het zelfs aan}{Antje vertelde}\\

\haiku{Het was, of er iets,.}{in Tjalling doorbrak dat hij}{er nooit had vermoed}\\

\haiku{- nee, zij is blond en,.}{groot het is het meisje van}{de harddraverij}\\

\haiku{Antje Adzers droeg;}{haar zwangere lijf als een}{schaamachtige last}\\

\haiku{De jongen beviel,.}{hem recht en slecht en niet van}{het brutale soort}\\

\haiku{Zijn magere hals.}{draaide onrustig boven}{het lage boordje}\\

\haiku{De lange trage.}{hand van Pieter lag op het}{dijbeen van de vrouw}\\

\haiku{De boeren klaagden -.}{nog al eens dat had mijnheer}{zeker ook gehoord}\\

\haiku{- Maar als-ie d'r laat, ' -!}{zitten dan zal ikm met}{mijn eigen handen}\\

\haiku{Aan het einde van.}{de tweede week vernam hij}{het allerlaatste}\\

\haiku{Je hebt centen voor -,,.}{het grijpen op die fabriek}{grijp ze dan stommerd}\\

\haiku{En Herre hoeft zo.}{hoog niet op te geven van}{Pieter's ongeluk}\\

\haiku{Dit was de kans, die,.}{hij nodig had die hij zo}{lang had nagejaagd}\\

\haiku{met een lang kleed om.}{zich heen en een wonderlijk}{gezicht zonder ogen}\\

\haiku{maar eindelijk kon,}{ze niet meer en gilde ze}{heel hard en rende}\\

\haiku{Hij leek op vader,;}{maar bij hem was alles twee}{keer zo breed en dik}\\

\haiku{maar Egmont jachtte.}{elke dag zijn huiswerk af}{en ging de straat op}\\

\haiku{Maar des avonds verscheen;}{tante Flora onverhoeds}{aan de Oude Gracht}\\

\haiku{Ze had gezien, dat;}{tante Flora en oom Lex}{het heel arm hadden}\\

\haiku{...elle me comprend,,!}{et mon coeur transparant pour}{elle seule b\'elas}\\

\haiku{Misgunde hij den?}{boerenzoon een dochter van}{het geslacht d'Aby}\\

\haiku{ze trilde korte,.}{tijd zo heftig in zijn arm}{dat hij er van schrok}\\

\haiku{Ze liepen in de,.}{rossige schemer naar huis}{verdoofd van geluk}\\

\haiku{Egmont was er bij.}{en oom Julien en Carla}{met Van Everdingen}\\

\haiku{Ze legt haar mes en,.}{vork harder naast het bord neer}{dan ze bedoeld had}\\

\haiku{Kleren... Natuurlijk,.}{voor een vrouw betekenen}{kleren altijd meer}\\

\haiku{Ik dacht, dat je in!}{de eerste plaats naar Holland}{wou voor je m\'oeder}\\

\haiku{wat ben je goed en!}{gewillig met me naar de}{eenzaamheid gegaan}\\

\haiku{Hij begon alle ',.}{bezwaren int veld te}{brengen die hij zag}\\

\haiku{Toen Ruth's noodschreeuw twee,.}{keer achtereen door het huis}{sneed kromp hij samen}\\

\haiku{{\textquoteleft}Ik denk nog net zo,.}{over de finanti\"en als}{toen je student werd}\\

\haiku{het gezin zat in,;}{de keukenkamer aan het}{tweede ochtendbrood}\\

\haiku{Gelaten en zwaar.}{dansten de paardeschoften}{bij zijn zweepslagen}\\

\haiku{Maar zij waren in,;}{de meerderheid en hier was}{hun spreekwijze wet}\\

\haiku{de wijdte, het dorp,,;}{de vaart waar de donkere}{tjalken door voeren}\\

\haiku{Hij wil je niet meer,,.}{zien anders begaat ie een}{moord aan je zeit ie}\\

\haiku{Maar jongen, jongen,:}{het leven ligt er voor \'ons}{nu eenkeer zo voor}\\

\haiku{Tjalling's mondgroeven,.}{verdiepten zich terwijl hij}{de weg aftuurde}\\

\haiku{Van Februari.}{af gaat dat nou al. Klappen}{krijgen ze niet meer}\\

\haiku{hoe vinden ze  ...?}{nou bij jullie de stappen}{van Abraham Kuyper}\\

\haiku{- En ik geloof, dat ',...}{Ryken van Sake Nieuwboer}{n plaats zoekt met Mei}\\

\haiku{En toch was er in,.}{hem iets dat oom en tante}{niet vergeven kon}\\

\haiku{Hij sliep in bij het.}{regelmatig metalen}{zingen der wielen}\\

\haiku{maar kon men op zo,?}{iets een fabriek bouwen die}{duizenden kostte}\\

\haiku{Oom Gevert is dood,,...}{maar oom Hebbert leeft nog zei}{mijn ouwe pake}\\

\haiku{Hij hoorde haar de:}{adem even inhouden achter}{de kinderwagen}\\

\haiku{Van zulken hoeven!}{wij ons toch niet op de kop}{te laten zitten}\\

\haiku{Maar allengs trok toch;}{het bedrijf hem weer in zijn}{opwindende sleur}\\

\haiku{Herre zweeg, en zijn.}{hand gleed onzeker langs de}{knopen van zijn vest}\\

\haiku{- ...Jij denkt nog steeds, dat,,.}{ik gek ben zei Rudmer die}{weer op en neer liep}\\

\haiku{Hij keek Herre niet,.}{aan zijn vingers draaiden de}{sleutelring sneller}\\

\haiku{Het was al laat, toen.}{hij Rudmer de weg naar de}{logeerkamer wees}\\

\haiku{Voor 't overige.}{bleef natuurlijk elk van hen}{baas in eigen huis}\\

\haiku{Ze dorst niets zeggen,.}{ofschoon bevreemd verwijt haar}{naar de lippen kwam}\\

\haiku{Het had hem koud en,.}{warm langs de rug gelopen}{terwijl Tjisse sprak}\\

\haiku{Maar d'r mankeert aan.}{je redenering toch nog}{het een en ander}\\

\haiku{Het zwijgen van den:}{geslepen schraper was m\'e\'er}{dan een afwachting}\\

\haiku{de mismoedigheid.}{over het geval met Tjisse}{woog enkel zwaarder}\\

\haiku{Jezus, moeten we?}{dan door zo'n schietding in de}{ellende komen}\\

\haiku{Dat is 'm. Hij heeft,!}{tegen me gezegd dat ik}{er op passen moest}\\

\haiku{Een wild wraakzuchtig.}{besef van vrijheid en macht}{gloeide in Ekke}\\

\haiku{- 't Is droog, geloof,;}{ik zei hij in sullige}{onderworpenheid}\\

\haiku{De laatste woorden;}{van den vreemde drongen pas}{nu tot Ekke door}\\

\haiku{Hij wierp zich om, op,.}{zijn buik lag hij het gezicht}{in het beddegoed}\\

\haiku{De kruik in haar hand.}{trilde en spilde droppels}{op het tafelzeil}\\

\haiku{zijn hoofd suisde, hij,;}{kreeg een schok door het hele}{lijf zijn voet schoot uit}\\

\haiku{Hij waagde het niet,.}{op te staan in de boot en}{haar toe te roepen}\\

\haiku{Arjen Taekes sloeg,.}{de hand op tafel naast de}{overbekende kruik}\\

\haiku{Het huis was hem een -.}{verschrikking geworden de}{mensen nog veel meer}\\

\haiku{Hij dacht aan At, hun,.}{snelle wilde avondliefde}{en aan de toekomst}\\

\haiku{Toen hij nog op het,.}{richelend weidepad liep}{ging de deur al open}\\

\haiku{de kleine hond liep;}{onrustig rond op erf en}{weg en zocht zijn baas}\\

\haiku{en voor de harde.}{luidruchtigheid van Jel schrok}{hij niet meer terug}\\

\haiku{Ekke zat daar lang;}{en geslagen en wist met}{zijn handen geen raad}\\

\haiku{Zodra we het voor '.}{elkaar hebben mett huis}{ga ik naar hen toe}\\

\haiku{en het scheen, of  .}{hij met de gedachten steeds}{bij iets anders was}\\

\haiku{- Tjalling woelde bij.}{zulke gedachten en hield}{Reinou uit de slaap}\\

\haiku{Herre was ook geen,.}{boer gebleven en hem ging}{alles fortuinlijk}\\

\haiku{wat was zijn schuld, al,?}{had hij dan nooit geleerd in}{schuld te geloven}\\

\haiku{Op Tjisse Landman;}{had hij reeds onmiddellijk}{moeten vertrouwen}\\

\haiku{- 't Lijkt verdomd wel,!}{of ik alleen maar voor dien}{grootboer werken mag}\\

\haiku{- Marten Offinga,:}{sprak het speeksel in zijn pijp}{pruttelde driftig}\\

\haiku{Hij stond nog steeds op,.}{dezelfde plek verslagen}{en onberaden}\\

\haiku{Wychman wilde.}{voor hem zijn oefeningen}{niet onderbreken}\\

\haiku{Een goed pianist,!}{oefent minstens vijf uur per}{dag zegt meneer Wolf}\\

\haiku{Het was de taal van,.}{een andere wereld die}{daar gesproken werd}\\

\haiku{- Het was November,,.}{het regende de avonden}{waren triest en lang}\\

\haiku{Herre deed een stap,.}{in zijn richting maakte een}{bevelend gebaar}\\

\haiku{'t Schijnt, dat u hem.}{terwille van dien kleine}{het spelen verbiedt}\\

\haiku{VIII Wychman,.}{speelde Wychman volgde}{muziektheorie}\\

\haiku{E\'en blijft er, naast hem,.}{op wie tenslotte alle}{zorgen neerkomen}\\

\subsection{Uit: Sla de wolven, herder!}

\haiku{Urukagina opende.}{flauw de ogen en staarde over}{het weghellend land}\\

\haiku{hij mengde alles.}{met koude adem dooreen en}{schudde het weer neer}\\

\haiku{Maar zij dachten aan;}{hun kinderen en waren}{bang voor hun vrouwen}\\

\haiku{en iedereen prees;}{den oude om zijn goedheid}{jegens den jongen}\\

\haiku{Naarmate hij het,.}{dorp naderde werd het gras}{hoger en ruiger}\\

\haiku{Het duurde niet lang,,;}{of er volgde een tweede}{dorp en een derde}\\

\haiku{Urukagina zag, dat;}{er een veel oudere man}{naast de kudde liep}\\

\haiku{Maar hij wist, dat het -?}{onmogelijk was was hij}{niet als slaaf verkocht}\\

\haiku{{\textquoteright} Urukagina begon,.}{pas nu zelf te lezen wat}{er in de klei stond}\\

\haiku{Urukagina hoorde,;}{tot de oudste jongens die}{in het dorp bleven}\\

\haiku{Toen Zarzari weer,.}{opkeek staarde zijn blik in}{een andere tijd}\\

\haiku{{\textquoteleft}Ik geloof, dat die,{\textquoteright}.}{daarnaast beter in zijn vet}{zit merkte hij op}\\

\haiku{Zelfs in het duister,;}{kon de jongen zien dat zij}{mistroostig liepen}\\

\haiku{Je hebt niet anders.}{gedaan dan wat de priester}{je bevolen heeft}\\

\haiku{Er bleven meer dan -.}{twintig dieren dood de helft}{daarvan was van mij}\\

\haiku{De man, die tegen,.}{de vrouw had gesproken kwam}{hen traag tegemoet}\\

\haiku{- Een tweede priester;}{had zijn kleed afgeworpen}{en trad naderbij}\\

\haiku{alleen de rivier,;}{stroomde de rook kronkelde}{ijl en scherp van geur}\\

\haiku{en in zijn gehoor.}{wies het veelvormig gerucht}{tot een dolle plaag}\\

\haiku{Hij bukte zich en.}{nam uit een aarden schaal een}{verdorde wortel}\\

\haiku{de edele heer was,.}{slechts met nadruk gevraagd de}{boot te verlaten}\\

\haiku{{\textquoteright} - Zarzari's gezicht;}{kneep bijeen als een oude}{rimpelige vrucht}\\

\haiku{het water joeg hen.}{ver in het troosteloze}{binnenland terug}\\

\haiku{{\textquoteright} zei hij, {\textquoteleft}dan kun je.}{er op uit trekken tegen}{het wild gedierte}\\

\haiku{Als een man zal hij!}{te voorschijn treden uit de}{donkere kameren}\\

\haiku{In zijn jeugd lag hij,}{in het zinkende schip als}{volwassene lag}\\

\haiku{ze rende blind, nu,.}{eens naar de ene dan naar de}{andere zijde}\\

\haiku{de eerste trots op.}{zijn naakte mannenkracht kwam}{er voor in de plaats}\\

\haiku{Thuaa lag stil, als,.}{een schelp die van binnen uit}{wordt dichtgehouden}\\

\haiku{hij voelde, hoe zij,,.}{zich verhardend de knie\"en}{over elkaar klemde}\\

\haiku{vrouwen spoelden er;}{wat wasgoed en droegen het}{haastig naar binnen}\\

\haiku{- {\textquoteleft}'t Is immers nog,...}{geen jaar geleden dat ik}{zelf met wasgoed liep}\\

\haiku{Hij begreep het zo,.}{plotseling dat hij zich met}{een schok oprichtte}\\

\haiku{in het halfduister.}{van hun veld-bed sperden}{haar ogen zich stralend}\\

\haiku{{\textquoteright} vroeg hij niet zonder,.}{spot maar de welwillendheid}{was niet verdwenen}\\

\haiku{beiden deden ze,;}{alsof ze niets van Thuaa's}{zwangerschap wisten}\\

\haiku{Abi-ishar schreed, met,.}{Thuaa aan de hand langzaam}{naar Urukagina's hut}\\

\haiku{De koning, die met -;}{cederengeur is vervuld}{daalt af in zijn park}\\

\haiku{- \'een maagdje heb ik -;}{hem toegeleid wier hart een}{snarenspel gelijkt}\\

\haiku{ik stelde haar een,,}{bezwering voor opdat je}{een zoon zou hebben}\\

\haiku{Urukagina woelde,.}{als zat hij in een dichte}{doornstruik gevangen}\\

\haiku{zij banden hem weer,,.}{hielden hem in hun mazen}{braken zijn opvlucht}\\

\haiku{Hij hief de hand op,.}{alsof hij een lijflijke}{slag moest afweren}\\

\haiku{Iemand kwam na een.}{poos bij hem en trok hem weg}{van de dode vrouw}\\

\haiku{Zijn zwager sloeg de.}{ogen neer en sloot de handen}{krampachtig ineen}\\

\haiku{De priester sloot de.}{ogen en trachtte zich met een}{ruk te bevrijden}\\

\haiku{De razernij van, -!}{Adad25 bezweer ik over je als}{je niet loslaat Ai}\\

\haiku{Het is je eigen,...}{aanmatiging die Thuaa's}{dood heeft veroorzaakt}\\

\haiku{{\textquoteright} Hij lachte kort en,,.}{snauwend en het oog ging weer}{open vals als voorheen}\\

\haiku{Onder het werkvolk,...}{en de boeren zijn sterke}{vaardige gasten}\\

\haiku{Korte tijd overwoog,.}{hij de wegen der mensen}{weer te verlaten}\\

\haiku{{\textquoteright} Urukagina keek hem,.}{aan verslagen door deze}{nieuwe beproeving}\\

\haiku{{\textquoteleft}er was arbeid voor,{\textquoteright}.}{honderd handen meer had de}{opzichter gezegd}\\

\haiku{De opzichter stond,,.}{kalm en bruin voor Urukagina}{en schudde het hoofd}\\

\haiku{Ze wilden juist slaags,.}{raken toen de opzichter}{tussenbeide kwam}\\

\haiku{Een oudere man,,.}{met iets schichtigs kwaadaardigs}{en groens in zijn ogen}\\

\haiku{{\textquoteleft}Heb ik niet gezegd,?}{dat het een onbeschaamde}{steppenrekel was}\\

\haiku{- Urukagina keek naar,.}{de grote sterren met hun}{beweeglijk blauw vuur}\\

\haiku{maar zij hadden hun.}{stille en verheven macht}{over hem verloren}\\

\haiku{Eindelijk gingen,.}{ze verder rechtstreeks naar het}{kamp der kamelen}\\

\haiku{Urukagina was zo,;}{verwonderd dat hij zich niet}{van de plek roerde}\\

\haiku{er liep alleen een.}{vaag-ontsteld fluisteren}{door de omstanders}\\

\haiku{Hamman, die juist een,:}{gedroogde vijg in de mond}{stak zei slaperig}\\

\haiku{{\textquoteright} Samsunu wendde,,;}{zich naar den metgezel en}{deed of hij kwaad werd}\\

\haiku{{\textquoteleft}Waarom, poortwachter,?}{nam je van mij de schaamdoek}{mijner lendenen}\\

\haiku{{\textquoteleft}Ishtar daalde neer in,!}{de onderwereld maar zij}{kwam niet weer boven}\\

\haiku{- Besprenkel Ishtar met,!}{het water des levens en}{breng haar weg van mij}\\

\haiku{kooplieden drukten...}{hun zegel onder brieven}{en warenlijsten}\\

\haiku{Het gesprek scheen hem.}{op boosaardige wijze}{naar de zin te zijn}\\

\haiku{meestal hield hij;}{zich op een afstand van de}{gewone mannen}\\

\haiku{Maar deelt de tempel?}{van Ningirsu er niet in}{de eerste plaats in}\\

\haiku{De negers vroegen,;}{om vinnen en staart die hun}{werden toegestaan}\\

\haiku{de lucht werd weer leeg,,;}{half door wolken gerafeld}{half met zon bespeeld}\\

\haiku{Deze vervloekte -{\textquoteright},.}{oorden waar een boze geest}{heerst zei de leider}\\

\haiku{een voorstel daartoe...}{zou spoedig in de hofraad}{behandeld worden}\\

\haiku{hoe is zijn naam ook...}{weer die jonge man met het}{wolfsvel bedoel ik}\\

\haiku{daarop suisde hij.}{sprongsgewijs naar de hals van}{de krijsende kip}\\

\haiku{Waarom zouden wij?}{dan bij het leven aan die}{verschillen denken}\\

\haiku{Papsukal gaf hem;}{een tikje op de wang en}{verliet de kamer}\\

\haiku{{\textquoteright} Papsukal leegde.}{zijn schaal met palmwijn en nam}{hem aandachtig op}\\

\haiku{{\textquoteright} Papsukal boog zich.}{naar hem toe en legde een}{arm om zijn schouder}\\

\haiku{{\textquoteleft}Ik ben altijd goed,{\textquoteright}.}{geweest in het oplossen}{van raadsels zei hij}\\

\haiku{Urukagina kwam langs;}{de hoge achtermuur van}{Ningirsu's tempel}\\

\haiku{{\textquoteright} Papsukal stond snel.}{op en legde zijn handen}{op Urukagina's hoofd}\\

\haiku{Maar hij heeft het niet,:}{eens aan willen nemen en}{hun toegesproken}\\

\haiku{De mannen van de...}{Kraanvogelkreek konden hun}{verbazing niet op}\\

\haiku{Maar wie had hij dan,,?}{in de schemering van zijn}{hart in Shaksagh herkend}\\

\haiku{{\textquoteright} zei Urukagina zacht,.}{om de vreemd-wordende}{stilte te breken}\\

\haiku{Hij slikte het vlees,.}{weg stond op van de ligbank}{en liep naar haar toe}\\

\haiku{Hij schudde het hoofd,.}{wachtend tot zij haar angst zou}{hebben overwonnen}\\

\haiku{De aardezang der.}{krekels gonsde van alle}{zijden op hen aan}\\

\haiku{{\textquoteright} vroeg ze, een oogwenk;}{machteloos en moe in zijn}{omarming hangend}\\

\haiku{Voor het overige.}{bootste hij Papsukal in}{alle dingen na}\\

\haiku{Ook was het waar, dat;}{een dergelijk heldenstuk}{zich nooit had herhaald}\\

\haiku{bevestigde hij,,.}{schel waar hij indrukwekkend}{bedoelde te zijn}\\

\haiku{Het is mij gelukt,.}{om den voorlezer aan het}{spreken te krijgen}\\

\haiku{Je hebt een openbaar.}{ambt ontvangen en misbruik}{gemaakt van je macht}\\

\haiku{De vertedering,.}{draalde nog in zijn ogen maar}{zijn mond bleef even hard}\\

\haiku{hem thans van dichtbij.}{te ontmoeten scheen althans}{een gebeurtenis}\\

\haiku{fijne, striemende.}{sneeuwvlagen joegen af en}{toe uit de hemel}\\

\haiku{het was noen-tijd,.}{en de honger knaagde in}{de meeste magen}\\

\haiku{{\textquoteleft}Ik verklaar, dat het.}{ten laste gelegde mij}{als waarheid voorkomt}\\

\haiku{hij had slechts toe te...}{stemmen op wat hem in de}{mond werd gegeven}\\

\haiku{Bij den aanvang der.}{plechtigheid leek alles op}{vorige jaren}\\

\haiku{de troonopvolger.}{smeet den hoofdman der lijfwacht}{iets naar het gezicht}\\

\haiku{zijn mond sputterde,.}{zijn aderen werden donker}{boven de slapen}\\

\haiku{iedereen drong naar,,.}{voren op om te zien wat}{er gebeuren zou}\\

\haiku{Een veelvoudige,,}{verschrikte hier en daar met}{overspannen lachen}\\

\haiku{wat deert het mij, wie...}{er patesi is en wie}{de ambten verdeelt}\\

\haiku{{\textquoteleft}Ik kwam hier, om een,.}{vriend terug te vinden niet}{een tegenstander}\\

\haiku{{\textquoteright} Hamman klakte met.}{de tong en Sun-nasir's}{vuist viel op tafel}\\

\haiku{{\textquoteleft}Het ga je goed,{\textquoteright} zei,;}{hij ten slotte den ander}{de handen drukkend}\\

\haiku{hij keek naar het slijk:}{en de stenen onder zijn}{voet en mompelde}\\

\haiku{Maar de tweede schok.}{was van ingrijpender en}{noodlottiger aard}\\

\haiku{er dreigde voor den.}{patesi en de zijnen}{het hoogste gevaar}\\

\haiku{en dat alles door...{\textquoteright}}{een stompzinnig verzinsel}{van een zwarten hond}\\

\haiku{Lu-enna's duimen;}{begonnen wrevelig langs}{elkaar te schuiven}\\

\haiku{Er was niets meer, dat;}{Urukagina uitdreef naar de}{roerige wereld}\\

\haiku{Welke gedachten?}{gingen hem bij al deze}{dingen door het hoofd}\\

\haiku{Hij deed het, omdat;}{Shaksagh schijnbaar stil en peinzend}{aan zijn zijde lag}\\

\haiku{de opwelling van,...{\textquoteright}}{uw innerlijk spreek ze niet}{onmiddellijk uit}\\

\haiku{Laat het winnen van,...!}{dochters aan hen over die niet}{beter verdienen}\\

\haiku{De godin kent mij -{\textquoteright} -;}{niet meer Urukagina lei snel}{de hand op haar mond}\\

\haiku{Nu kan mijn borst het,:}{niet langer bergen nu moet}{ik er van spreken}\\

\haiku{Een der donkerste:}{bleef Urukagina's geheime}{zorg aangaande Shaksagh}\\

\haiku{Hij betoogde, dat.}{hij Urukagina's plannen met}{alle kracht steunde}\\

\haiku{Een verwarrende.}{beklemming dreef het bloed naar}{Urukagina's gezicht}\\

\haiku{slank, gespierd, fier, het.}{gezicht als brons onder de}{zilveren haardos}\\

\haiku{Hij bedekte de {\textquoteleft}?}{ogen met de hand.Jij en ik}{hebben hem gedood}\\

\haiku{Een tijdlang stonden.}{zij tegenover elkaar en}{bewogen zich niet}\\

\haiku{Zo even wist Bada,,;}{niets van de steden af ja}{hij verfoeide ze}\\

\haiku{Nadat ik deze,}{middag van u hoorde en}{u zag en wist wien}\\

\haiku{{\textquoteleft}Als men jong is, leeft,}{men in de verwachting naar}{het volkomene}\\

\haiku{Maar wie geroepen,.}{is en zich verbergt voor zijn}{god maakt zich schuldig}\\

\haiku{Eens heb je jezelf,.}{ingezet maar men heeft je}{daarin verhinderd}\\

\haiku{- Zij wendde het hoofd,,;}{af als hij die dingen zei}{en lachte niet meer}\\

\haiku{Dit was een dieper,.}{bezit het bezit dat van}{een vrouw een vrouw maakt}\\

\haiku{Maanden heb ik er,.}{over gepeinsd het verworpen}{en weer opgevat}\\

\haiku{{\textquoteright} - Hij zag haar lippen.}{zich vastberaden sluiten}{en weer vaneen gaan}\\

\haiku{Zij hebben dit met,...}{de ouderdom gemeen dat}{zij kunnen wachten}\\

\haiku{Van de sterren ging,.}{zijn blik naar de aarde nog}{eenmaal keek hij om}\\

\haiku{Ik weet niet, wat het.}{is en je behoeft het mij}{niet te vertellen}\\

\haiku{Intussen lichtte;}{de wind langzamerhand het}{neergestoven zand}\\

\haiku{Urukagina had er,;}{zelf nog geen voorstelling van}{hoe alles zou zijn}\\

\haiku{Zij hadden gezaaid,;}{en geplant maar de regen}{was uitgebleven}\\

\haiku{En Sun-nasir:}{gaf het met luid spotgehuil}{begroete antwoord}\\

\haiku{hij overwoog nog, of,.}{men niet terug zou gaan toen}{het al te laat was}\\

\haiku{Een van Urukagina's.}{metgezellen keerde zich}{om en rende weg}\\

\haiku{Vlak daarop floot een.}{speer en stiet naast hen op de}{harde bodem af}\\

\haiku{Het feit, dat hij nog,.}{leefde vervulde hem eerst}{met blijdschap en hoop}\\

\haiku{Het volgend ogenblik.}{landden zij temidden van}{veelvormig gerucht}\\

\haiku{- Hij zag dit, kalm en,,.}{koud vrij van wanhoop zij het}{niet van bitterheid}\\

\haiku{maar hij had die dood.}{niet verzoend door het geluk}{van een misbruikt volk}\\

\haiku{- De stilte en het,;}{nu weer gezeefde gele}{licht verrieden niets}\\

\haiku{een zuur, verbeten.}{lachje rimpelde als van}{ouds zijn spits mondwerk}\\

\haiku{Het was het wolfsvel,.}{dat men hem in Ab-Enki}{afgenomen had}\\

\haiku{Urukagina bracht snel,.}{de hand naar de gordel naar}{het vertrouwde mes}\\

\haiku{Urukagina bleef staan,.}{en zag dat het de mannen}{der ilku's waren}\\

\haiku{- Hij is Eanatum,...,!}{die herrezen is misschien}{Ur-Nina zelf}\\

\haiku{Bada knikte nog.}{eens en bracht toen de handen}{tegen het voorhoofd}\\

\haiku{nu zal het niet lang,...}{meer duren of ik reik nog}{maar tot je schouder}\\

\haiku{Daarop boog ze haar,.}{zware rode hoofd en ging}{voor de tweede maal}\\

\haiku{Ik werd onderweg,...}{overvallen door een vrouw zij}{sprak zo gebiedend}\\

\haiku{men ried onder de,.}{flardenmantel haar trotse}{rechte gestalte}\\

\haiku{{\textquoteright} Van vriendschap spreekt hij,,.}{niet meer dacht Urukagina een}{oogwenk verbitterd}\\

\haiku{{\textquoteleft}Ik heb niets meer te,,...{\textquoteright}}{wensen Urukagina dan een}{dood zonder pijnen}\\

\haiku{Zich daarom in hun,.}{stijl te vernederen zou}{dollemanswerk zijn}\\

\haiku{{\textquoteleft}Je weet, mijn gouden,,...}{ree dat ik oprecht tegen}{je spreek zoals steeds}\\

\haiku{Het kon geen kwaad, als...}{zij in de tussentijd niet}{verwaarloosd werden}\\

\haiku{Het leven, dat de,.}{stedelingen leidden was}{voor hem gesloten}\\

\haiku{Tegen hem leunend,;}{sloot ze de handen bijeen}{om het amuletje}\\

\haiku{{\textquoteright} Haar ogen fonkelden,.}{hem tegen opgetogen}{en bewonderend}\\

\haiku{dacht hij, de trap naar.}{het inwendige van het}{paleis afdalend}\\

\haiku{Je hebt de macht, hem,.}{uit zijn ambt te ontzetten}{als je dat verkiest}\\

\haiku{dat ze met jou en;}{je vrienden in holen en}{moerassen overnacht}\\

\haiku{Urukagina wachtte.}{een dag of wat en sprak toen}{met Sun-nasir}\\

\haiku{geloof jij, dat er?}{iets in Shaksagh's houding valt}{te veroordelen}\\

\haiku{Ook werd het alras,.}{bekend dat Shaksagh dit keer niet}{ingegrepen had}\\

\haiku{- Sun-nasir het.}{den schrijver gaan en begaf}{zich naar Urukagina}\\

\haiku{Sun-nasir en.}{Urukagina kwamen in de}{zuilengaanderij}\\

\haiku{De afwezigheid.}{van levende wezens was}{hier bijna drukkend}\\

\haiku{{\textquoteright} Urukagina zei het,.}{weifelend maar niet geheel}{zonder overtuiging}\\

\haiku{Urukagina hief de.}{sluier herkennend met de}{punt van de teen op}\\

\haiku{De stedelingen:}{hadden overal langs de weg}{naar de stoet gestaard}\\

\haiku{Hij hield even op, haar,:}{vingers te wrijven en ze}{vervolgde haastig}\\

\haiku{Hij hield de adem in,.}{als kon dit nietig geruis}{haar wakker maken}\\

\haiku{Waarom had zij zo?}{dringend naar de liefde van}{haar ouders gevraagd}\\

\haiku{Urukagina eist zelfs,!}{schadeloosstelling voor de}{vrouw die men verstoot}\\

\haiku{de priesters grijpen,, -{\textquoteright} {\textquoteleft}}{het gebod verachten dat}{zij ons overbrachten}\\

\haiku{Hij liet de handen.}{langs de zijden afhangen}{en knikte langzaam}\\

\haiku{Zij droeg voor het eerst,}{de brede met juwelen}{doorvlochten haarkroon}\\

\haiku{Achter die deuren.}{zou Amat-Bau uit zijn}{bestaan verdwijnen}\\

\haiku{De gevoelloosheid,.}{doordrong zijn denken maar ook}{de herinnering}\\

\haiku{Zij hingen over zijn,.}{gevoel als hadden zij een}{tastbare zwaarte}\\

\haiku{De gebeurtenis.}{bracht de uiteenlopendste}{beroering te weeg}\\

\haiku{In den beginne,.}{had hij gemeend Urtar te}{kunnen gebruiken}\\

\haiku{Hij keek den priester,.}{aan die de ogen veelzeggend}{naar buiten wendde}\\

\haiku{hij deed een grote,,.}{pas voorwaarts naar de deur om}{de hemel te zien}\\

\haiku{De zware trage;}{olie had het water met een}{goudtint doortrokken}\\

\haiku{Bijna zou ik het,;}{veroordelen dat mensen}{dromen en denken}\\

\haiku{Urtar hief zich op.}{de tenen en schreeuwde een}{langgerekt bevel}\\

\haiku{Hij rekte zich op.}{zijn rijdier en balde de}{vuist boven het hoofd}\\

\haiku{het wilde echter,;}{niet vol blijhartig en ruim}{worden als voorheen}\\

\haiku{Urukagina sprong van.}{zijn ezel en hief den grijsaard}{van de bodem op}\\

\haiku{Hij  trachtte het,.}{te overwinnen door veel met}{Bada te spreken}\\

\haiku{Je tracht voor me te,,}{verbergen dat het er slecht}{met de stad voor staat}\\

\haiku{hij had dit geluk,.}{verloren hij had zich niet}{kunnen verdelen}\\

\haiku{{\textquoteleft}Wat zoekt men hier in,?}{de nacht bij het altaar en}{verstoort de stilte}\\

\haiku{hij was er nu wel,.}{zeker van dat zij niet meer}{in de tempel was}\\

\haiku{de stank van dood stof,;}{en ontbinding zat in zijn}{haar in zijn gewaad}\\

\haiku{Niet de liefde van,...}{een vrouw of het bezit van}{een kind maar zijn doel}\\

\haiku{het beitelwerk, dat,.}{de muurrand vulde was vol}{grillige schaduw}\\

\haiku{Hier en daar liepen,.}{wagensporen tamelijk}{vers en herkenbaar}\\

\haiku{Nog een halve mijl.}{verder was het bevaarbaar}{voor een kleine boot}\\

\haiku{Bada zocht een taai.}{grasje en peuterde een}{graatje uit zijn kies}\\

\haiku{Bada wiste de,,.}{naam Urukagina die hij had}{geschreven weer uit}\\

\haiku{gemalin van god,}{Ningirsu beschermvrouw van}{Shirpurla B\^elili}\\

\haiku{63Talent = \ensuremath{\pm};}{30 kg. 64Weg van Ea =}{de zuiderhemel}\\

\subsection{Uit: W.A.-man}

\haiku{de haat doordrong hem,.}{diep en innig zij werd het}{merg van zijn bestaan}\\

\haiku{Hij moest zich met kracht,.}{beheersen diep ademhalend}{kwam hij tot zich zelf}\\

\haiku{Op dat ogenblik was,,.}{ook zijn moeder er vermoeid}{bits en geprikkeld}\\

\haiku{Frans' moeder keek strak.}{en behekst naar het mondstuk}{van de luidspreker}\\

\haiku{Misschien is het tot...}{mijnheer doorgedrongen dat}{we in oorlog zijn}\\

\haiku{Vogel liet de blik:}{veelzeggend van de een naar}{de ander gaan}\\

\haiku{Frans en Sudderam.}{wierpen zich tegelijk naar}{voren om to zien}\\

\haiku{Zij namen opnieuw,.}{een bocht Trewes reed als de}{baarlijke duivel}\\

\haiku{{\textquoteright} Zijn haat golfde loom,.}{op maar legde zich toen ze}{weer over straat liepen}\\

\haiku{zij slopen naar een;}{andere schuilplaats door de}{verduisterde stad}\\

\haiku{Zij gingen een voor.}{een naar buiten en klommen}{onder het zeildoek}\\

\haiku{Weer werd het donker,.}{een lucht vol stof en molm en}{teer viel zwaar op hem}\\

\haiku{Het deed pijn, maar hij.}{had niet de wil en de macht}{zich te verroeren}\\

\haiku{De sterren hingen.}{in tintelende reeksen}{boven Amsterdam}\\

\haiku{De mannen achter, {\textquoteleft}{\textquoteright}.}{het luik hoorden maar \'e\'en woord}{capitulatie}\\

\haiku{{\textquoteright} zei hij eindelijk, {\textquoteleft}.}{we m\'o\'eten weten wat er}{aan de knikker is}\\

\haiku{Opnieuw begon hij,,.}{te trillen hij moest blijven}{staan een paar tellen}\\

\haiku{vlak om de hoek was.}{het hellende smalle huis}{met het winkeltje}\\

\haiku{Hij trilde opnieuw,.}{de afschuw jegens zich zelf}{steeg hem naar de keel}\\

\subsection{Uit: Wilde lantaarns}

\haiku{{\textquoteleft}Als ik hier in de,;}{Wildhoek kom kan ik Lolkje}{niet voorbijlopen}\\

\haiku{ja, de laatste maal.}{had de mijnheer Wigle niet}{eens meer ontvangen}\\

\haiku{en het eerste jaar.}{wou hij ook nog gratis mest}{over het nieuwe land}\\

\haiku{Alle Wildhoekster;}{jongens en meisjes kenden}{elkaar van die school}\\

\haiku{in zijn hart was hij;}{bang voor het bezittende}{volk van de polder}\\

\haiku{Wieger naar stad, om,;}{er zelf bij te zijn als de}{ouwe wat insloeg}\\

\haiku{wij van dat goedje,;}{hebben want wanbetalers}{zijn het dikwijls ook}\\

\haiku{Maar meteen waren;}{ook weer de hoogmoed en het}{boos verzet in haar}\\

\haiku{{\textquoteright} - Op Anders' lippen,:}{drong de vraag zo vaak hij langs}{Jannina zwenkte}\\

\haiku{- {\textquoteleft}E\'en advocaatje,{\textquoteright},.}{drong Anders die het meisje}{voelde wankelen}\\

\haiku{{\textquoteright} begon Anders, en:}{onverhoeds daalde zijn stem}{schor en verbeten}\\

\haiku{daarachter rezen,,;}{als pluimpjes stilstaand gras de}{laatste restjes bos}\\

\haiku{{\textquoteleft}Jongens, is het waar,?}{dat Floris Hoogwolda werk}{maakt van Jannina}\\

\haiku{het water liep traag,.}{van zijn gezicht terwijl hij}{naar hen luisterde}\\

\haiku{en toen geen van hen,:}{meer iets te vertellen had}{lachte hij en zei}\\

\haiku{Het was duidelijk,.}{dat hij op de aftocht van}{den oude wachtte}\\

\haiku{die  leken op,.}{niets dat haar verbeelding of}{geheugen kende}\\

\haiku{zij, Jannina, met, -?}{de rug tegen de deurpost}{haar ogen in de nacht}\\

\haiku{{\textquoteright} - Jannina's adem bleef,.}{een oogwenk dralen daarop}{rukte ze zich los}\\

\haiku{Maar als ik zeg, dat,.}{ie beter was dan jij dan}{spreek ik de waarheid}\\

\haiku{Hij zwaaide met de,,:}{korte armen wenkte om}{stilte schreeuwde schor}\\

\haiku{Zij zaten er in,;}{het smalle reepje gras dat}{langs het water liep}\\

\haiku{de meesten groetten.}{niet terug en keken}{haar onwillig na}\\

\haiku{Ze stapte langs de,;}{eerste groep die haar zwijgend}{opnam en doorliet}\\

\haiku{In de stoffige.}{hete avondval stonden de}{turfstekers bijeen}\\

\haiku{Het lachen, waarmee,.}{de arbeiders hem groetten}{had iets goedmoedigs}\\

\haiku{De veldwachter liep,.}{drie vier passen met de stoet}{mee en bleef weer staan}\\

\haiku{Soms ging ze, als nam,.}{ze een onverhoeds besluit}{naast hem mee verder}\\

\haiku{Onder de hoge '.}{bomen van de oprit bleef}{men voort eerst staan}\\

\haiku{Ze kwamen in breed.}{gelid aandringen en nu}{stokten ze niet meer}\\

\haiku{Wybren Post vocht,.}{tevergeefs om Jan Herder}{tegen te houden}\\

\section{Beb Vuyk}

\subsection{Uit: Gerucht en geweld}

\haiku{We zaten onder.}{die regenboom en keken}{op naar de takken}\\

\haiku{Er zijn plaatsen waar.}{de tijdelijke dingen}{bestendigd worden}\\

\haiku{Als een verpleegster,.}{die bij een zieke waakt had}{ze toen gedacht}\\

\haiku{{\textquoteleft}Wat hebt u tegen,,?}{die zwarten kapitein u}{bent toch zelf ook zwart}\\

\haiku{{\textquoteright} zei hij tegen de.}{commandant die jaren op}{Java gewoond had}\\

\haiku{De puisten liepen,.}{door tot in de hals die iets}{vogelachtigs had}\\

\haiku{Alles wat bruin was.}{en aan de Hollandse kant}{stond noemden zij zo}\\

\haiku{Etty moet zich daar.}{tussen die Hollanders dood}{ge\"ergerd hebben}\\

\haiku{Hij zat op een stoel,.}{groenachtig in zijn gezicht}{en mompelde wat}\\

\haiku{In betere staat.}{zou het een seigneurlijke}{woning zijn geweest}\\

\haiku{{\textquoteleft}Ik begrijp dat dit.}{landschap je boeit en dat je}{voorlopig hier blijft}\\

\haiku{{\textquoteleft}Vroeger offerden,.}{wij geen dieren maar slaven}{en gevangenen}\\

\haiku{Het oude dorpshoofd.}{nodigde ons uit om het}{huis te betreden}\\

\haiku{Het dreunen van de.}{gongs deed alle begrip van}{tijd verloren gaan}\\

\haiku{Voor het helemaal.}{licht zou zijn moest ik in de}{rivier gaan baden}\\

\haiku{{\textquoteleft}Vroeger offerden,.}{wij geen dieren maar slaven}{en gevangenen}\\

\haiku{We bleven staan in.}{de hete zon aan de rand}{van de begraafplaats}\\

\subsection{Uit: Het laatste huis van de wereld}

\haiku{Max Havelaar in.}{zijn toespraak tot de Hoofden}{van Lebak}\\

\haiku{Menschen verdringen.}{zich aan de steiger en rond}{een ijzeren loods}\\

\haiku{De K.P.M.-boot is,.}{iets bijgedraaid ligt nu schuin}{voor onze Tandjong}\\

\haiku{De ramen zijn klein,,,.}{hoog en smal zonder glas maar}{met dikke luiken}\\

\haiku{Gaba-gaba.}{is de hoofdnerf van het blad}{van de sagopalm}\\

\haiku{Het water is hier,.}{zeer vischrijk maar de opbrengst}{in geld is gering}\\

\haiku{Wij vragen hem of.}{hij niet met zijn vijf kontjo's ons}{erf wil schoonmaken}\\

\haiku{Een van die oude.}{Hindoe-hoofden was een vriend}{van mijn schoonvader}\\

\haiku{Zesde hoofdstuk ONS.}{erf grenst aan het erf van den}{Bestuursambtenaar}\\

\haiku{Dan klimmen zij langs,.}{een smal pad de heuvel op}{bijna zonder groet}\\

\haiku{Twee dagen later, ',.}{s morgens met de zeewind}{steken we van wal}\\

\haiku{Veertig ketels heeft,.}{Batoeboi het wordt een tocht}{van twee dagen}\\

\haiku{Jij bedriegt en ik,?}{bedrieg waarom zouden we}{geen vrienden blijven}\\

\haiku{Er wordt geen rijst meer,,,;}{uitgegeven noch koffie}{noch geld noch pisang}\\

\haiku{Hij loopt met een groote.}{mand heen en weer tusschen de}{bloeboer en het vat}\\

\haiku{De jongens hebben.}{zich op hun zij gedraaid en}{ademen diep en luid}\\

\haiku{Toen we weggingen,,.}{ver na middernacht zat de}{bruid nog voor haar bed}\\

\haiku{het beest echter slaat.}{op de vlucht in de richting}{van de groote rivier}\\

\haiku{De linkerhelft van;}{onze tuin op de Tandjong}{is zeer onvruchtbaar}\\

\haiku{Heintje, zijn vrouw en,,.}{kinderen de tuinjongen}{iedereen rept zich}\\

\haiku{Maar als oplossing.}{van het kleedingvraagstuk is het}{aardig gevonden}\\

\haiku{Niet te begrijpen, {\textquoteleft},,{\textquoteright}.}{klankensenangal\'e}{agoos moelin\'ekrois}\\

\haiku{Wie kinderen krijgt,.}{moet ervoor betalen met}{geld of met de dood}\\

\haiku{Er staat een flinke.}{deining en de leege prauw zwalkt}{hevig op en neer}\\

\haiku{Regen dringt door het,,.}{atapdak twee- driemaal slaat}{een golf naar binnen}\\

\haiku{Ode Madi zit met.}{een ploeg op Toebahoni}{en stookt twee toengkoe's74}\\

\chapter[13 auteurs, 938 haiku's]{dertien auteurs, negenhonderdachtendertig haiku's}

\section{Jacqueline van der Waals}

\subsection{Uit: Noortje Velt}

\haiku{In het zachte blauw,....}{des hemels dreven wit de}{wolken ongekleurd}\\

\haiku{{\textquoteright} En moeder knikte:}{dan van neen en Nora zei}{verontschuldigend}\\

\haiku{Beschaamd vluchtte ze - -.}{heen holde ze de trappen}{af naar beneden}\\

\haiku{Wat zijn de bloemen,,?}{de kleuren wat is het licht}{voor een blinde}\\

\haiku{Straf me niet in het,.}{medaillonnetje dat ik}{niet verliezen mag}\\

\haiku{{\textquoteright} Maar met die woorden...}{had ze toch weer het bestaan}{van het kind erkend}\\

\haiku{Waarom had juf niet,?}{geschreven als ze haar iets}{te vertellen had}\\

\haiku{Hoe dankbaar verrast,!}{hoe vol bewondering zou}{Jan dan voor haar zijn}\\

\haiku{Ze had zich immers?}{juist zoo gelukkig in haar}{eenzaamheid gevoeld}\\

\haiku{En was er dan ook,?}{eens niemand die  zich om}{je bekommerde}\\

\haiku{{\textquoteleft}En de geheele.}{familie is er toen ook}{erg boos over geweest}\\

\haiku{u vergist u, ik,.}{ben geen meisje dat op school}{uitgelachen wordt}\\

\haiku{Vader en Jan en....!}{Henri en juf en Emy en}{nu ook de poppen}\\

\haiku{Nora zweeg dus en.}{speelde en zoo spelende}{ging de tijd voorbij}\\

\haiku{Nous voulons vivre?}{dans l'id\'ee des autres d'une}{vie imaginaire}\\

\haiku{telkens bedachten.}{ze iets nieuws om haar aandacht}{tot zich te trekken}\\

\haiku{Maar zou Nora het?}{niet vervelend vinden naar}{haar te luisteren}\\

\haiku{die was verleden.}{week allebei haar knie\"en}{kapot gevallen}\\

\haiku{Zag juffrouw Nora,?}{niet dat de kousen met een}{M gemerkt waren}\\

\haiku{Ze begreep volstrekt,.}{niet wat de meisjes nu weer}{te lachen hadden}\\

\haiku{{\textquoteright} Nora's mond lachte.}{ondeugend en in haar oogen}{flikkerde de pret}\\

\haiku{Maar natuurlijk jij -.}{Marie en ik vallen nog}{niet in de termen}\\

\haiku{Triomfeerend hield,.}{ze den brief in de hoogte}{haar oogen schitterden}\\

\haiku{Voelde ze misschien,?}{toch dat ze eigenlijk iets}{goed te maken had}\\

\haiku{Hij danste weinig,?}{misschien verwachtte hij zijn}{Asschepoester nog}\\

\haiku{{\textquoteright} {\textquoteleft}En mij onwereldsch,{\textquoteright}, {\textquoteleft}.}{lachte Ellywant ik vind}{een bal goddelijk}\\

\haiku{- We hebben geen recht,.}{te spreken over hetgeen we}{zoo weinig kennen}\\

\haiku{Nora keek om zich,.}{heen naar de grauwe trieste}{velden en zuchtte}\\

\haiku{Ook dit, dat ze nu,,.}{naar het Leger des Heils ging}{had geen doel geen zin}\\

\haiku{Zie, nu kon er geen,,.}{aarzeling geen twijfel geen}{bangheid meer bestaan}\\

\haiku{de stem, waarmee ze,.}{spreken ging klonk wonderbaar}{rustig en innig}\\

\haiku{Gij zoudt spreken als,,:}{ik spreek getuigen als ik}{getuig zeggende}\\

\haiku{{\textquoteleft}Dier'bre Heiland,, ',!}{mijn Verlosserk Ben de}{uwe goddelijk Lam}\\

\haiku{{\textquoteright} zei hij hartelijk.}{en Nora's hart sprong op van}{dankbare blijdschap}\\

\haiku{{\textquoteright} {\textquoteleft}Neen,{\textquoteright} lachte hij, d\`at.}{moest ze alleen doen als ze}{zelf er lust in had}\\

\haiku{Ook Jaap, die voor zijn,.}{examen zat verscheen weinig}{op het tenniscourt}\\

\haiku{Wat beteekent goed of?}{kwaad anders dan goed of kwaad}{in de gevolgen}\\

\haiku{Mevrouw Merlin was.}{waarlijk ingenomen met}{het engagement}\\

\haiku{Cum laude{\textquoteright} was hij.}{gepromoveerd op een zeer}{uitstekend proefschrift}\\

\haiku{Maar daar lag nu het,.}{maanlicht en sliep en gedroeg}{zich als eigenaar}\\

\haiku{Hoe konden ze haar?}{zoo laf verloochenen voor}{dat slapende licht}\\

\haiku{{\textquoteright} sprak ze zacht, {\textquoteleft}indien,....}{iemand me liefhad hij zou}{me z\'o\'o kunnen zien}\\

\haiku{En wees nu niet meer,,}{boos vergeef het me indien}{ik onredelijk}\\

\haiku{Jaap,{\textquoteright} zei ze haastig,, {\textquoteleft},!}{en haar stem klonk heel warm en}{hartelijko Jaap}\\

\haiku{Dacht ze heusch, dat?}{er iets was voorgevallen}{tusschen Jaap en haar}\\

\haiku{- {\textquoteleft}Nietwaar, moedertje,?}{maar u wilt er voor ons maar}{niet voor uitkomen}\\

\haiku{Nora verborg het.}{gelaat in de handen en}{snikte en snikte}\\

\haiku{Ik wil alles, ik,,...{\textquoteright}}{durf alles als ik maar weet}{dat het uw wil is}\\

\haiku{{\textquoteleft}Moederliefde is,{\textquoteright}, {\textquoteleft}.}{iets bijzonders zei Marie}{scherpdie is eeuwig}\\

\haiku{wat wijsheid is bij,,!}{haar wordt onzin zoodra}{{\`\i}k het uiten durf}\\

\haiku{Ze glimlachte even.}{om die onwillekeurig}{zelfbekentenis}\\

\haiku{Maar eenvoudige,{\textquoteright}.}{menschen doen anders voegde}{ze er ernstig hij}\\

\haiku{{\textquoteright} {\textquoteleft}Dat is het niet, wat,{\textquoteright},.}{ik noodig heb zei Marie nu}{een beetje kalmer}\\

\haiku{Het is het eenige.}{wat wij worden kunnen met}{onze opvoeding}\\

\haiku{{\textquoteright} {\textquoteleft}Natuurlijk, dat maak,.}{je jezelf wijs wanneer je}{bang bent voor den strijd}\\

\haiku{Je moogt immers net,,.}{doen wat je wilt wat je zelf}{voelt noodig te hebben}\\

\section{Willem Adriaan Wagener}

\subsection{Uit: Sjanghai}

\haiku{Van het jaar nog zal.}{hij uit de gereedschapskist}{worden verwijderd}\\

\haiku{Het zijn de laatste,.}{woorden die de dame in}{haar moedertaal leest}\\

\haiku{Om de 61/2 maat heft.}{zich langs beide zijden van}{het schip een boeggolf}\\

\haiku{Als twee steenen tegen,.}{elkaar worden geslagen}{ontstaat er een vonk}\\

\haiku{Hij stierf verrassend,}{snel met een deeltje van}{Ibsen in de hand.}\\

\haiku{Hij hoopt echter, dat.}{het niet in zijn nadeel zal}{worden uitgelegd}\\

\haiku{De lift knielt achter.}{de dichtkleppende deuren}{en zakt door den grond}\\

\haiku{met zoo'n keurig klein - -;}{schortje van astrakanbont}{kuische vrouwen}\\

\haiku{Hij buigt zich langs den,.}{dokter die Nora opnieuw}{op bed heeft gelegd}\\

\haiku{Laat Japan nu maar.}{wat  mariniertjes op}{de kade plaatsen}\\

\haiku{De motorboot heeft.}{den kop op het Japansche}{eskader gericht}\\

\haiku{Er is veel, waarop.}{zoo direct geen antwoord kan}{worden gegeven}\\

\haiku{Trouwens, u weet toch... -?}{hoe de toestand is En de}{wapenindustrie}\\

\haiku{De journalisten:}{glimlachen en voegen aan}{hun notities toe}\\

\haiku{Godversche, Ypersche, - -.}{lakensche grijp-krengen}{voor een jen pest puist}\\

\haiku{Hem zal de nacht niet,,,.}{deren de nacht die dreigend}{nadert de bloednacht}\\

\haiku{De trein vertrouwt zich.}{aarzelend aan het warnet}{van rangeer-rails toe}\\

\haiku{Zuster Estelle,.}{legt hem de flesch aan maar hij}{heeft het al gedaan}\\

\haiku{Het is een weinig,.}{logische methode die}{hij daarbij toepast}\\

\haiku{Het grijze baardje.}{vangt den eenigen lichtstraal in}{het grijze vertrek}\\

\haiku{Het grijze baardje.}{vangt den eenigen lichtstraal in}{het grijze vertrek}\\

\haiku{En de Japansche,.}{Regeering deelt mee dat daarvan}{geen sprake zal zijn}\\

\haiku{{\textquoteleft}The United States'{\textquoteright}.}{protest was couched in the}{most indignant terms}\\

\haiku{Te vroeg geboren.}{raakt een kind ontijdig aan}{de straat verslingerd}\\

\haiku{De moeder raapt het,.}{kind op wikkelt de streng om}{haar arm en snelt heen}\\

\haiku{Twee dagen is er,,,,.}{aan gewerkt geschaafd geschuurd}{gevijld gepolijst}\\

\haiku{Helaas... de feiten,, (.}{zooals ik reeds zeide hebben}{haar achterhaaldspr}\\

\haiku{dat Japan zich niet,;}{de wet laat stellen zelfs niet}{door den Volkenbond}\\

\haiku{Het staat een nieuwe.}{productie en verdeeling}{der goederen voor}\\

\haiku{Geef mij voor mijn part,.}{communistisch brood maar geef}{mij ook mijn spelen}\\

\haiku{Nora is in een.}{stemming om den heelen nacht}{over Freud te praten}\\

\haiku{Japan zal wapens.}{noodig hebben en China zal}{wapens noodig hebben}\\

\haiku{Over plaatsing van de.}{leening behoeft men zich nog niet}{bezorgd te maken}\\

\haiku{Elke bewuste.}{houding van Engeland zou}{krankzinnigheid zijn}\\

\haiku{Maar bloed en olie zijn.}{hinderlijke begrippen}{voor vermoeide oogen}\\

\haiku{Het gebaar is zelfs.}{iets te kort voor zoo een groote}{ruimte als deze}\\

\haiku{De Japansche leening.}{van 1907 daalde 2{\textonehalf} punt tot}{72 en de 5{\textonehalf} pCt}\\

\haiku{- We m\`oeten elkaar,.}{nog even zien anders ben ik}{doodongelukkig}\\

\haiku{In een verkeersknoop.}{zuigen de remmen alle}{vaart uit den wagen}\\

\haiku{Het Amerikaansche.}{transportschip Chaumont werkt zich}{los van de kade}\\

\haiku{Om den mond van den.}{Jangtse ligt een gordel van}{pantserstaal gesnoerd}\\

\haiku{Kunnen wij onzen?}{kinderen niet een beetje}{vrede meegeven}\\

\haiku{De verklaring van.}{Thomas maakte in den Raad}{een diepen indruk}\\

\haiku{In afwachting van.}{versterkingen staken zij}{het bombardement}\\

\haiku{Hij grondt zijn beroep.}{op art. 15 van het statuut}{van den Volkenbond}\\

\haiku{De dood zaait grijze.}{asch van schemering in de}{conferentiezaal}\\

\haiku{Om twaalf uur zal het.}{noodlot van China breken}{als een rijpe vrucht}\\

\haiku{Als de laatste slag,.}{verklonken zal zijn ontvonkt}{in Sjanghai de hel}\\

\section{Maurits Wagenvoort}

\subsection{Uit: De droomers. Deel 1}

\haiku{En t\`och niet draagt het.}{menschelijk pogen den naam}{van Machteloosheid}\\

\haiku{de schrijver verlangt,.}{slechts te zien dat zij meer en}{beter doen dan hij}\\

\haiku{hij moest zich-zelf,.}{een eigen leven bouwen}{van den bodem af}\\

\haiku{de boomen stil van.}{berusting in het niet af}{te wijzen noodlot}\\

\haiku{Slechts nu en dan had,:}{hij aan het verleden aan}{het begin gedacht}\\

\haiku{zegt zij vrouw, zij spreekt,;}{van de barende moeder}{waarborg der toekomst}\\

\haiku{Hugo bezocht die,;}{armen soms maar hij kon zoo}{zelden wat missen}\\

\haiku{hij klopte aan de,:}{deur der familie G\'erard}{een kinderstem riep}\\

\haiku{{\textquoteleft}wel, meneer Devos,,.}{dat is lief van u dat u}{ons komt bezoeken}\\

\haiku{Ik ben v\'o\'or alles,:}{Italiaan de roem van mijn}{Huis ligt in Itali\"e}\\

\haiku{Geen villa ook had,.}{prachtiger waterwerken}{mooier fonteinen}\\

\haiku{In mijn land zijn veel,,;}{menschen heethoofden die zich}{anarchist noemen}\\

\haiku{het onbeperkte:}{eigendom van den bodem}{en wat daarop leeft}\\

\haiku{tot de volken is,,.}{zij ingevoerd van elders}{nooit doorgedrongen}\\

\haiku{De Christelijke:}{Kerk is sinds lang verdeeld in}{twee  hoofdgroepen}\\

\haiku{Daarom was 't maar:}{het beste het leven te}{nemen zooals het kwam}\\

\haiku{Zij was weggevlucht,.}{toen zij den winkelier had}{hooren naderen}\\

\haiku{Vandamme leerde,;}{hem lezen en schrijven en}{later zelfs latijn}\\

\haiku{{\textquoteleft}Er behoort een goed;}{geloof toe om dat alles}{er in te vinden}\\

\haiku{{\textquoteright} vroeg Hugo, die niet.}{goed in de symboliek van}{zijn kennis thuis was}\\

\haiku{Buitendien, de tien -?}{geboden weten we of}{die wel van Hem zijn}\\

\haiku{Medelijden met,.}{de menschen niemand kan me}{daarin overtreffen}\\

\subsection{Uit: De droomers. Deel 2}

\haiku{{\textquoteright} Pierrot zag hem een,.}{oogenblik verschrikt aan toen}{schudde hij het hoofd}\\

\haiku{Hij schrikte er van,,:}{zoo licht als de jongen}{was en zoo mager}\\

\haiku{het anarchisme.}{een misdadige droom van}{onmogelijkheid}\\

\haiku{De woekeraar, die;}{de uiterste winst zoekt in}{den geringsten arbeid}\\

\haiku{{\textquoteright} Hij wachtte den groet,.}{der kinderen niet af maar}{ging dadelijk heen}\\

\haiku{een klein eindje maar,;}{en  Hugo zette zich}{weer bij zijn bed neer}\\

\haiku{{\textquoteright} {\textquoteleft}Dus u gelooft, dat?}{revoluti\"en voortaan}{onmogelijk zijn}\\

\haiku{Behoeft een naakte,?}{minder een pak kleeren dan een}{hongerige brood}\\

\haiku{elkaars gelijke,,.}{elkaars aanvulling elkaars}{steun bij den arbeid}\\

\haiku{Dat is het Einde,,.}{d\'at is het Beloofde Land}{d\`at is de Hemel}\\

\haiku{De zieke keek hem,;}{een oogenblik zwijgend aan}{en drukte zijn hand}\\

\haiku{{\textquoteright} Hij ging uit om het,}{leven nog eens goed aan te}{zien te zoeken w\'a\'ar}\\

\haiku{Niets kwam hem in de,,.}{gedachte geen enkel plan}{dat uitvoerbaar was}\\

\haiku{Deze vraag bracht hem,;}{in verwarring het bloed steeg}{hem in zijn gezicht}\\

\haiku{Tevreden sloot hij,.}{het mes en legde het op}{zijne papieren}\\

\haiku{{\textquoteright} {\textquoteleft}Maar zeg 't mij dan,,{\textquoteright}.}{meneer Hugo zei Mlle}{Malise verschrikt}\\

\haiku{{\textquoteleft}Ik vraag u dat, zei,.}{ze om me-zelf over u}{gerust te stellen}\\

\haiku{{\textquoteright} De profeet ging voort,.}{maar in eens voelde Hugo}{zijn hart stil schokken}\\

\haiku{De hertog ging den,,....}{portier voorbij kwam op het}{trottoir naderde}\\

\haiku{De joviale.}{rechter had moeite om het}{niet uit te proesten}\\

\haiku{hij keek op door zijn,:}{venster zijn blik stuitte af}{op de trali\"en}\\

\haiku{{\textquoteright} {\textquoteleft}Gij zult sterven, ja,;}{maar uw voorbeeld is  niet}{vergeefsch geweest}\\

\haiku{, wordt uw hart niet tot....}{bloedens toe getroffen bij}{de aanschouwing van}\\

\section{Gerard Walschap}

\subsection{Uit: Adela{\"\i}de}

\haiku{Wondere weemoed, ...}{die mijn leven over-lommert}{zoet zijt gij en wreed}\\

\haiku{Enfin het leek er.}{wel op of hij schande bracht}{over de familie}\\

\haiku{Kijk, de veefokker,,?}{Reynders indien hij haar zoo}{zag zou het kwaad zijn}\\

\haiku{Opeens stond hij recht,.}{tikte op het tafelblad}{en sprak tot allen}\\

\haiku{Zelfs begon hij zich.}{te verteederen in zijn}{zondarigen klerk}\\

\haiku{ik ben als ik zie.}{dat het u goed gaat en dat}{ge gelukkig zijt}\\

\haiku{Och, zei Ernest, men.}{is toch maar eerst getrouwd als}{er een  kind is}\\

\haiku{Een tweede angst nam,.}{bezit van haar de angst haar}{man te verwaarloozen}\\

\haiku{Dan vond zij dat hij.}{zeker tot bij het kindje}{had kunnen komen}\\

\haiku{Toen hadden  zij.}{hier de villa gekocht die}{zij nu bewoonden}\\

\haiku{Ik heb zoo'n hoofpijn,,,:}{zoo'n hoofdpijn klaagde zij zacht}{en dan altijd weer}\\

\haiku{Dan greep zij wild het.}{kind uit de wieg en zat het}{uren lang te bezien}\\

\haiku{het beest zelf halen,,,.}{want ik ben vandaag niet goed}{man overgeefachtig}\\

\haiku{Er woonde daar een,.}{arm duivelken op het dorp}{Daelemanneken}\\

\haiku{Ze zonden hem dan.}{maar naar een kapelleken}{even buiten het dorp}\\

\haiku{Ik weet heel juist wat.}{gebeuren moet en ik neem}{het allemaal aan}\\

\haiku{{\textquoteleft}En uw acht, dat zijn,{\textquoteright}.}{wilden zegt Adela{\"\i}de}{bits uit haren hoek}\\

\haiku{Juleken pas op,.}{hoor die pop teruggeven}{of anders naar bed}\\

\haiku{{\textquoteright} Vijf minuten lang.}{houdt hij het woord die kerel}{met temperament}\\

\haiku{Als hij veilig zit,.}{durft hij weer glimlachen maar}{alleen met zijn oogen}\\

\haiku{En hoe daarin twee.}{tranen groeiden omdat hij}{zoo lief was voor haar}\\

\haiku{Adela{\"\i}de had.}{er den heelen namiddag}{plezier mee gehad}\\

\haiku{Oscar pakte het.}{vlug op en Ernest wou het}{hem maar afnemen}\\

\haiku{{\textquoteright} Op een vooravond kwam.}{zij in den winkel met een}{pakje brochuren}\\

\haiku{Als Ernest weg was.}{stelde zij zich voor  hoe}{de ontrouw begon}\\

\haiku{Doe het nog eens, gij,,.}{twee smeerlappen zegt zij laat}{het mij eens goed zien}\\

\haiku{Nog even lichtte over.}{haar arme ziel de uitkomst}{van een goede biecht}\\

\haiku{En zij had nu met {\textquoteleft}{\textquoteright}.}{dat geval zooveel aan het}{huwelijk gedacht}\\

\haiku{Adela{\"\i}de sloeg.}{met haar volle vuist een ruit}{stuk en ging loopen}\\

\haiku{En nu weg, zoo ver,,,.}{als ge kunt altijd maar gaan}{gaan gaan en bidden}\\

\haiku{Dat kon zij en dat,?}{zou ze dachten ze misschien}{dat ze zoo gek was}\\

\subsection{Uit: Bejegening van Christus}

\haiku{Gerard Walschap}{Bejegening van Christus}{Colofon}\\

\haiku{Gaat hij den weg op,?}{van zijn vrouw dat hij wartaal}{begint te spreken}\\

\haiku{Maar het is Asveer.}{of de grond langzaam onder}{zijn voeten wegschuift}\\

\haiku{De stem van Zachaar,,:}{den reus sloeg op uit de groep}{en luidde oproer}\\

\haiku{{\textquoteleft}Grijpt mij dien muiter.}{en werpt hem in den kelder}{met water en brood}\\

\haiku{Ter wille van den.}{Messias lieten zij zich}{door hem opsluiten}\\

\haiku{Die hem kenden van.}{aangezicht hielden hem voor}{een menschenhater}\\

\haiku{met al de wijsheid,.}{van schrifturen stond hij bot}{den mond vol tanden}\\

\haiku{De hoogepriester.}{komt overeind en roept zelf met}{luider stem den naam}\\

\haiku{Gelooven is op Gods,.}{gezag aannemen wat men}{niet bewijzen kan}\\

\haiku{Nu mogen zij dan,.}{ook maar zeelieden zijn maar}{zooiets is aartsdom}\\

\haiku{Welken vijand nu,,?}{als ze eenmaal verdeeld zijn}{het eerst aanpakken}\\

\haiku{De Messias is,.}{onze redder hij heeft het}{recht en de waarheid}\\

\haiku{Hij weet niet, waar de,.}{Messias is noch wanneer}{hij zal verschijnen}\\

\haiku{Hij weet, dat hij niet,.}{sterven zal alvorens hem}{te hebben gezien}\\

\haiku{Het kereltje hoest,,.}{tusschen haakjes dat het niet}{aan te hooren is}\\

\haiku{Johannes, zeggen,.}{zij is door Salome de}{strot afgebeten}\\

\haiku{Gansch zijn jeugd heeft hij,.}{de Redders nageloopen}{de ontgoocheling}\\

\haiku{niemand kent deze,.}{krachten niemand kan hare}{grenzen afpalen}\\

\haiku{Laat het zelfs waar zijn,?}{moet hij dat zeggen aan het}{gewone  volk}\\

\haiku{Nu had de heer van,.}{den wijngaard nog eenen eenigen}{zoon dien hij liefhad}\\

\haiku{God verschillende?}{wijzen om hem te dienen}{welgevallig zijn}\\

\haiku{De Meester leerde.}{ons een Jehova die zijn}{schepselen liefheeft}\\

\haiku{Wij weten dat de.}{aarde eene schijf is en dat}{de zon er rond draait}\\

\subsection{Uit: Carla}

\haiku{Het paste toch dat.}{ze niet meer bij Caluwaers bleef}{maar moeder volgde}\\

\haiku{Die weggestuurd in:}{het pensionaat komen}{worden de beste}\\

\haiku{Niet  met Gustaaf,.}{want die zegt dat hij zich met}{dat meisje niet moeit}\\

\haiku{Men moet zijn vader,.}{eeren ook als hij misschien}{een klein gebrek heeft}\\

\haiku{Het verlangen naar.}{de wereld werd opgezweept}{door de eenzaamheid}\\

\haiku{Men zwijgt er over bij,.}{wereldsche menschen die het}{toch niet begrijpen}\\

\haiku{Dan gaan zij langs het.}{deftige witte huis van}{Dr. Yvo Verhaegen}\\

\haiku{Zij heeft er niet meer,.}{aan gedacht er met hem geen}{woord over gewisseld}\\

\haiku{Hij kan niet weten.}{wat tijdens de terugreis}{in den trein gebeurt}\\

\haiku{Ze hadden ieder.}{een haartje uitgetrokken}{om ze te meten}\\

\haiku{Leo moest eens nazien.}{of er iets bij was dat niet}{mocht verbrand worden}\\

\haiku{Hij loopt klein, vulgair,,.}{bekrompen bespottelijk}{onder haar oogen}\\

\haiku{Ook met deze vrouw.}{kan Carla nu niet over haar}{verwachting spreken}\\

\haiku{Leo, op uw moeder,.}{en op ons kindje ik ben}{goed voor haar geweest}\\

\haiku{Och God ja, nog iets,,.}{plechtigs enfin ge weet het}{wel h\'e Carlienke}\\

\haiku{zij valt hem huilend,,.}{om den hals Henri zij is}{zoo ongelukkig}\\

\haiku{Dan zit er volgens.}{Henri niets anders op dan}{eens te vechten}\\

\haiku{Carla alleen ziet.}{dat er iets zonderlings is}{in zijn vroolijkheid}\\

\haiku{Maar hij heeft er dan,:}{iets anders op gevonden}{dat van den polis}\\

\haiku{Maar als ik het schot.}{gehoord had werd ik razend}{en vergat alles}\\

\haiku{ik wist dat alles.}{maar een vergissing was en}{dat ge zoudt komen}\\

\haiku{Schreiend loopt zij naar,.}{boven knielt voor het kind en}{kermt vergiffenis}\\

\haiku{Zoodat Henri haar op.}{de stoep in het voorbijgaan}{kon gelukwenschen}\\

\haiku{Het verschil met den.}{rijkdom van Henri of Paul}{zou verminderd zijn}\\

\haiku{En als zij zag dat,}{hij bizonder slecht gestemd}{was verdubbelde}\\

\haiku{Maar onder den drang.}{van bronst nadert hij met of}{zonder verkoudheid}\\

\haiku{Zijn verkiezingsstrijd.}{begon hij op den dag van}{Dolf's begrafenis}\\

\haiku{Leo lette meer op.}{het bedrijf van Paul dan op}{zijn kandidatuur}\\

\haiku{Die triomfeert bij,.}{de verkiezingen eerste}{vergissing van Leo}\\

\haiku{Het geloof is eene.}{genade en men verliest}{het door de zonde}\\

\haiku{Het moest zooeens waar zijn,,.}{denken ze en dan hebben}{wij ze beet mijn kind}\\

\haiku{J'ai connu quelqu'un, ...{\textquoteright}.}{un ing\'enieur Carla}{voelt zich machteloos}\\

\haiku{Hij houdt haar zijnen,.}{glimlach voor een schild dat zij}{niet kan doorsteken}\\

\haiku{Zij zit dagen lang}{opgescheept met dat Mieke}{Demey en Mieke}\\

\haiku{Die zit hij zwijgend}{leeg te drinken en als hij}{goed dronken is houdt}\\

\haiku{Veronderstel dat.}{de commerce van Herman}{Stevens van mij is}\\

\haiku{Maar deze was met,,.}{een boerken getrouwd en had}{meende ze welstand}\\

\haiku{En toen het groot en,.}{bronstig geworden was had}{Leo haar dat gezegd}\\

\haiku{Willen of niet, zij.}{moet met hem alleen blijven}{bij de twee dieren}\\

\haiku{Hij durft er zelf niet,.}{naar vragen maar de makkers}{roepen het voor hem}\\

\haiku{{\textquoteleft}Heeft ze u gestuurd,,,.}{om mij te verdrinken hijgt}{hij goed gij of ik}\\

\haiku{Voelt ge nu nog niet?}{dat ik u kan kraken en}{in twee\"en breken}\\

\haiku{in elk geval zal.}{voor haar het einde eerder}{komen dan voor hem}\\

\haiku{Ze heeft veel te groote,.}{zwarte oogen en bohemersch}{zwart haar wij zijn blond}\\

\haiku{{\textquoteleft}Papa, ik heb het,,.}{niet gedaan Mamake ik}{heb het niet gedaan}\\

\haiku{Niets kan zoo wreed zijn,.}{als jong geluk het gekscheert}{onder doodsklokken}\\

\haiku{Daaraan ziet Carla.}{dan dat Mieke waarlijk van}{God gezonden is}\\

\haiku{Hij geeft de doode,.}{een kruisteeken het zieke kind}{wat fruit en speelgoed}\\

\subsection{Uit: Eric}

\haiku{Nonkel Oscar nam.}{zijn handje en mama zou}{wel lang wegblijven}\\

\haiku{Ten slotte dierf zij,.}{niet meer in het huis komen}{tenzij met Ernest}\\

\haiku{Na elken zin wacht:}{zij op antwoord van al de}{kinderen in koor}\\

\haiku{'s Anderdaags ging.}{hij met Cyriel van dokter}{Tierens naar de school}\\

\haiku{Bezeten gaat hij.}{te keer met zijn vingeren}{vol razende kramp}\\

\haiku{Eerst dat arm vrouwke,.}{en nu hijzelf want daar moet}{hij van ten onder}\\

\haiku{Daarom ging Lizy:}{op een stoel zitten weenen}{en Eric troostte haar}\\

\haiku{{\textquoteleft}Ik zal u ook eens,, '.}{onderzoeken madam leg}{u maar int bed}\\

\haiku{Oscar kon dat niet,.}{aanhooren maar bonpapa}{luisterde stralend}\\

\haiku{Om zijn goed humeur.}{luid te luchten begon hij}{soms op te spelen}\\

\haiku{Ik heb u goeden.}{dag gezegd en verzoek u}{mij ook te groeten}\\

\haiku{Wij trekken samen,,.}{op Ernest en ik dacht hij}{en zijn rug boog door}\\

\haiku{Telkens hij opkeek.}{zag hij de oogen uit het bed}{scherp op hem gericht}\\

\haiku{Hoe rap kijkt hij weg,,.}{dacht  Ernest vroeger keek}{hij mij zoo recht aan}\\

\haiku{De directeur moest:}{de deur van de spreekkamer}{met kracht openduwen}\\

\haiku{{\textquoteleft}ja, ik heb het er.}{bij bonpapa eindelijk}{toch doorgekregen}\\

\haiku{Het ging nu zonder.}{speech \`a la Verhaeghen en}{zonder piano}\\

\haiku{Den boer bespreekt het.}{opstel dat hij een beetje}{te romantisch vindt}\\

\haiku{Ze was zeker ziek,,.}{vragen ze of ze zal iets}{gekregen hebben}\\

\haiku{want hij troost het kind,.}{met een doode moeder en}{een blinden vader}\\

\haiku{Dus tot zijn groot spijt.}{enz. Maar Eric doorstaat al de}{pijnen der hel}\\

\haiku{Ja, dat is papa,.}{met een verband om het hoofd}{juist gelijk mama}\\

\haiku{het hoorde, want voor.}{hen en de professoren}{was Eric een voorbeeld}\\

\haiku{Ze trokken tegen.}{als hij de nestels aantrok}{en een nestel brak}\\

\haiku{Daarmee waren ze,.}{voor een dag gered hij deed}{een ander paar aan}\\

\haiku{Zoo was ze dan hier.}{gebleven en hoe waren}{ze nu niet gestraft}\\

\haiku{In mijn dorp staat een.}{kasteel waarvan ze zeggen}{dat het zwart goed is}\\

\haiku{Maar op Dries kunt ge.}{een stad bouwen en van Eric}{zijt ge nooit zeker}\\

\haiku{Iedereen kan goed,.}{met hem om maar niemand die}{hem eigenlijk kent}\\

\haiku{Eric zat den ganschen.}{avond roerloos in zijn zetel}{voor zich uit te zien}\\

\haiku{Terwijl hij haar na,.}{de vergadering zocht stond}{zij opeens voor hem}\\

\haiku{En dan in wachtzaal.}{tweede klas voor den trein van}{8 uren en zooveel}\\

\haiku{Er zijn er die er.}{te veel hebben en dat loopt}{nogal dikwijls mis}\\

\haiku{Maar er gaat geen dag,.}{voorbij zonder kus geen week}{zonder den doodelijken}\\

\haiku{Allengerhand wordt.}{hij kalmer en zij dankt het}{aan haren invloed}\\

\haiku{hij bewaarde zijn,:}{geheim maar Cyriel sprak het}{met leedvermaak uit}\\

\haiku{{\textquoteright} Eric stond stil, zag hem.}{sprakeloos met wilde oogen}{aan en ging verder}\\

\haiku{Als hij weg is denkt.}{Eric dat hij absoluut een}{revolver noodig heeft}\\

\haiku{Ze vraagt verschrikt of,,.}{hij dan al iets weet maar neen}{wat zou hij weten}\\

\haiku{ge houdt van Eric en,,?}{hij van u. Zeg hij is toch}{niet geladen he}\\

\haiku{zij zoo hatelijk,.}{en ze kan mis zijn maar ze}{voelde de afgunst}\\

\haiku{En ondertusschen.}{probeert Lizy toch maar bij}{Eric te geraken}\\

\haiku{Hij wil gedoome.}{wel eens zien of hij van hem}{geene goeie zal maken}\\

\haiku{Geef het liever geen,, '.}{patatten zeit hem alst}{maar spinazie eet}\\

\haiku{{\textquoteleft}Wacht,{\textquoteright} roept Eric, en vliegt,.}{de trappen af zoo vlug als}{toen grootvader stierf}\\

\haiku{Hij neemt Eric bepaald, {\textquoteleft}{\textquoteright}.}{onder handen die moet zich}{nu eens gaanzetten}\\

\haiku{Hij herinnert zich.}{hoe Oscar te Leuven over}{een revolver sprak}\\

\haiku{Maar zelf schrijft hij naar.}{Zuster Ismelda dat het}{haar zaken niet zijn}\\

\haiku{Waarop zouden ze,.}{wachten voor hem is het geen}{leven zonder thuis}\\

\subsection{Uit: Sibylle}

\haiku{Omdat hij alleen.}{was en het practisch werk nauw}{toezicht eischte}\\

\haiku{Daardoor merkten zij.}{de afwezigheid van de}{nieuwe moeder op}\\

\haiku{Maar niets was hem zoo.}{belangrijk als de uitslag}{van Sibylle}\\

\haiku{Het dorp  was er,.}{weldra over uitgepraat maar}{Michel bleef wrokken}\\

\haiku{Kwikstaartje sprong hem.}{om den hals en zeide dat}{zij verloofd waren}\\

\haiku{Hij wordt Goddank niet.}{begraven en al wat leeft}{schrijft en zendt kiekjes}\\

\haiku{{\textquoteleft}En kardinaal Van,,?}{Rossum Celest hebt ge dien}{ook wel eens gezien}\\

\haiku{{\textquoteleft}Schrijf en publiceer,,.}{het zei die ze moesten het maar}{niet gedaan hebben}\\

\haiku{{\textquoteright} - {\textquoteleft}Dat ik denk dat ze}{niet ver mis zullen zijn als}{ze ongeveer doen}\\

\haiku{{\textquoteright} - {\textquoteleft}Durft ge al wat ge,?}{mij zegt en al gezegd hebt}{overal herhalen}\\

\haiku{Twee wegen zwenken.}{in sierlijke bocht rechts en}{links van het gebouw}\\

\haiku{Geen schandaal, dat bracht.}{haar werk in gevaar en zij}{had het Cest beloofd}\\

\haiku{Getroffen drukte, '}{hij haar de hand als op een}{zwijgend verbond maar}\\

\haiku{Hij zou er morgen,,.}{maar mee  doorgaan in de}{pastorij overal}\\

\haiku{Die hoe heet ze, die,,.}{half-rosse enfin die}{vriendin is ook weg}\\

\haiku{we moeten een reis,.}{doen maar waren niet in staat}{te denken waarheen}\\

\haiku{{\textquoteright} Sibylle gaf.}{hun juist een uur tijd om het}{huis te verlaten}\\

\subsection{Uit: Tor}

\haiku{Hij zegt dat als Nel,.}{het niet doet Tor Nel nog eens}{zal kapot maken}\\

\haiku{Toen hij met zijn Ad\`ele,.}{Tas trouwde was hij al rijk}{maar zij nog rijker}\\

\haiku{Dan krijgt hun gebed,.}{een mystieke tragische}{opgetogenheid}\\

\haiku{Hij heeft Mieke niet:}{lang moeten smeeken om het}{te mogen worden}\\

\haiku{Heeft een veldwachter?}{meer toekomst dan iemand die}{hun familie dient}\\

\haiku{Mieke denkt niet aan.}{ja omdat Tor niet te best}{met hem overeenkomt}\\

\haiku{Zij beweert in het,.}{donker bang te zijn hij moet}{haar naar huis brengen}\\

\haiku{Waartoe het dient zien,.}{wij voor ons oogen het maakt Vera}{nog zotter van Tor}\\

\haiku{Tor is zeker maar,,?}{een veldwachter maar ik was}{ik geen metsersknaap}\\

\haiku{Als zij niets van hem,.}{hooren gaan Tor en Vera met}{den auto eens zien}\\

\haiku{Reine Priestman heeft,.}{nogal stem zij wil beter}{leeren zingen bij Vera}\\

\haiku{Ge zult dat inzien.}{als ge nog wat meer uwen smaak}{zult gevormd hebben}\\

\haiku{Celis meestert bij,,.}{de Dherts  maar Colfs is geen}{Dhert hij vraagt Priestman}\\

\haiku{De doktoor spreekt er.}{over gelijk wij over het weer}{en de politiek}\\

\haiku{Zijn zinnen staan er.}{niet meer naar om een beroemd}{dichter te worden}\\

\haiku{Anders nog een jaar.}{en hij is Napoleon}{en God de Vader}\\

\haiku{Hij kon er waarlijk,.}{niet meer mee spreken ze niet}{meer in de oogen zien}\\

\haiku{En toch miszie ik}{iets aan u. Bij alles bracht}{ze onzen lieven}\\

\haiku{En zie, Tor heeft de,?}{oogen toegedrukt maar staan ze}{nu weer niet half open}\\

\haiku{Ze durven zonder,.}{hem niet boven gaan maar hij}{zelf durft ook niet meer}\\

\haiku{Die bibliotheek.}{van Tor de commissaris}{was buitengewoon}\\

\haiku{Hij kan dat niet meer,.}{halen hij doet afstand ten}{voordeele van Priestman}\\

\haiku{En wat Fran\c{c}ois ook,.}{heeft Fran\c{c}ois is goed voor zijn}{ondergeschikten}\\

\haiku{Hij meet thuis haren,,.}{bloeddruk veel te weinig en}{maakt haar een  flesch}\\

\haiku{Hij schrijft er schoon uwen,.}{naam in en hulde van den}{schrijver Muys Victor}\\

\haiku{Zijn moeder heeft hem,.}{doorzien Mieke niet en hij}{heeft ze bedrogen}\\

\haiku{Eens wijst hij haar het,.}{fleschken en vraagt of er iets}{van komt ja of neen}\\

\haiku{De daders moeten,.}{aangehouden worden de}{gekwetsten verzorgd}\\

\haiku{Zoo genomen zijn,.}{zenuwen zijn aangetast}{ik geef dat nog toe}\\

\haiku{Ze zeggen vallen,.}{en opstaan maar bij hem is}{het maar stronkelen}\\

\haiku{Hij gaat er met inkt.}{over opdat ze goed zien dat}{het geenen doubl\'e is}\\

\haiku{Het moest plaats hebben.}{op ne Zondag omdat het}{meisje dan uitging}\\

\haiku{En voorts, allee, hoe,.}{gaat het op den buiten de}{menschen babbelen}\\

\section{Andr\'e Weber}

\subsection{Uit: E\'en jaar maximumstraf}

\haiku{Hij is niet ouder,.}{dan dertig jaar en is heel}{slank bijna mager}\\

\haiku{Terwijl hij langzaam,.}{het gerechtsgebouw verlaat}{neemt hij een besluit}\\

\haiku{We benne kletsnat.}{en willen vannacht graag een}{beetje droog pitten}\\

\haiku{{\textquoteright} {\textquoteleft}Dat stelletje ook,{\textquoteright}, {\textquoteleft}.}{niet zegt Krook beslisten wij}{waren hier het eerst}\\

\haiku{Kort en goed, ik doe,.}{mijn best hem het leven tot}{een hel te maken}\\

\haiku{{\textquoteleft}Ik heb er maar drie,{\textquoteright}.}{neer kunnen schieten antwoordt}{Hart onverschillig}\\

\haiku{Mijn vrouw heeft hem pas.}{een paar maal gezien en kent}{hem dus niet zoo goed}\\

\haiku{{\textquoteleft}'n Oogenblik{\textquoteright}, zegt,.}{Miep verlaat de kamer en}{doet de voordeur open}\\

\haiku{Mocht U eens hulp of,.}{goede raad noodig hebben komt}{U dan naar mij toe}\\

\haiku{Maar ze hebben vier.}{kerels in het ziekenhuis}{moeten verbinden}\\

\haiku{{\textquoteright} Verwonderd neemt de:}{inspecteur het briefje in}{ontvangst en leest}\\

\haiku{We hebben in elk.}{geval een aanwijzing en}{ik ga er op af}\\

\haiku{{\textquoteright} Inmiddels heeft de.}{auto Zutphen gepasseerd}{en nadert Gorssel}\\

\haiku{Aan het einde der,.}{gang komen ze bij een deur}{die op een kier staat}\\

\haiku{En als ik tegen,,.}{U zeg dat U volmacht heeft}{dan heeft U volmacht}\\

\haiku{Hij vroeg mij, hem te,.}{vergeven dat hij mijn oom}{had doodgereden}\\

\haiku{Maar het is hem  ,.}{nog steeds een raadsel waarheen}{het hem zal leiden}\\

\haiku{De man schijnt het land,{\textquoteright}.}{in te hebben voegt hij er}{grinnikend aan toe}\\

\haiku{Rechtuit de trap op{\textquoteright} {\textquoteleft}.}{en dan de eerste deur aan}{Uw linker hand.Goed}\\

\haiku{Hart ziet het gevaar:}{nog net bijtijds aankomen}{en zegt plotseling}\\

\haiku{{\textquoteleft}U komt me zeker,?}{vertellen dat er bij U}{is ingebroken}\\

\haiku{Hij rekent af, neemt.}{zijn tasch op en begeeft zich}{hoopvol op weg}\\

\haiku{{\textquoteleft}Ik ben agent Mulder.}{en kom met een boodschap van}{hoofdinspecteur Hart}\\

\haiku{Op een gegeven {\textquotedblleft}{\textquotedblright}.}{oogenblik noemde ze zich}{een weerloos meisje}\\

\haiku{En het begint er,.}{op te lijken dat zijn hop}{in vervulling gaat}\\

\haiku{{\textquoteright} Vluchtig bekijkt hij,.}{den inhoud maar ook hier vindt}{hij niets bijzonders}\\

\haiku{De kop thee, die voor,.}{haar op tafel staat is al}{lang koud geworden}\\

\haiku{{\textquoteleft}Mocht U eens hulp of,.}{goede raad noodig hebben komt}{U dan naar mij toe}\\

\haiku{{\textquoteright} Miep vertelt hem nu,.}{uitvoerig hoe de zaak zich}{heeft toegedragen}\\

\haiku{Ik had geen voorschot,.}{noodig maar hij wilde het toch}{zonder meer geven}\\

\haiku{Ik heb een paar ton.}{te beleggen en moet eens}{kalm met U praten}\\

\haiku{{\textquoteleft}Het is pas kwart voor,.}{zeven dus zijn we in elk}{geval vroeg genoeg}\\

\haiku{Ik was baron van,.}{Teuningen van de rijke}{tak moet je weten}\\

\haiku{De inbraak, die ze,.}{van plan zijn zal heel ergens}{anders plaats hebben}\\

\haiku{{\textquoteleft}Het licht schijnt precies.}{een halve minuut te vroeg}{te zijn opgegaan}\\

\haiku{{\textquoteright} {\textquoteleft}Heel graag, tenminste,.}{als je nog van die oude}{cognac in huis hebt}\\

\haiku{Mijnheer Martens, het,.}{lijkt me nu wel verantwoord}{U vrij te laten}\\

\haiku{Of denk je, dat ie,?}{de brutaliteit heeft naar}{zijn woning te gaan}\\

\haiku{Hij heeft een voorsprong,.}{van bijna drie uur en dat}{is een heeleboel}\\

\haiku{binnen en Mander,.}{drukt hun op het hart uiterst}{voorzichtig te zijn}\\

\haiku{{\textquoteleft}Maar wees een beetje,.}{voorzichtig en schiet niet in}{mijn rug scherpschutter}\\

\haiku{Eindelijk heeft hij.}{het knopje gevonden en}{het licht aangedraaid}\\

\haiku{Als je 't toch doet.}{zal ik je met mij eigen}{handen vermoorden}\\

\haiku{Ik heb het vrouwtje.}{opgezocht en haar een paar}{theedoeken verkocht}\\

\haiku{Over een half uur  .}{is het licht en dan moeten}{we verdwenen zijn}\\

\haiku{Er zal niets anders,.}{opzitten dan het morgen}{opnieuw te probeeren}\\

\haiku{Ten minste, als je '.}{s middags om een uur of}{twaalf al wakker bent}\\

\haiku{Ik heb het toen voor.}{hem gemaakt en hij heeft me}{vorstelijk betaald}\\

\haiku{{\textquotedblleft}Maar U wist toch pas,?}{een paar dagen dat U die}{prijs had getrokken}\\

\section{Constant van Wessem}

\subsection{Uit: Celly. Lessen in charleston}

\haiku{In dit eene opzicht.}{heeft het verhaal zelfs een vrij}{ouderwetsch verloop}\\

\haiku{Maar waarom zegt zij,?}{niets staart zij het plaatje aan}{als ziet zij het niet}\\

\haiku{Je kunt je gerust,.}{komen overtuigen dat ik}{het w\`el gedaan heb}\\

\haiku{Soms kon zij het fel.}{voelen als zij het portret}{van Moeder bezag}\\

\haiku{Celly ziet neer op.}{haar eigen voetpunten en}{die van Miklos}\\

\haiku{dit is dansen, en.}{zij wordt er zelf volslagen}{moedeloos onder}\\

\haiku{Daar lag hij nu en.}{probeerde zijn gedachten}{te verzamelen}\\

\haiku{- Toen salueerde.}{hij aan den rand van zijn hoed}{en wendde zich om}\\

\haiku{klokslag vier uur stond {\textquoteleft}{\textquoteright}.}{Victoria met haar auto}{voorMaison Ilse}\\

\haiku{Jeroen wist best, dat.}{hij zijn mythe van besten}{danser verspeelde}\\

\haiku{Het werd nu een sprong,.}{pardoes in een witgloeiend}{sprookjesachtig licht}\\

\haiku{Haar eene hand wreef zij.}{over de andere alsof}{zij ze afdroogde}\\

\haiku{Boven het water}{scheen de gloeiende hitte}{hooger te hangen}\\

\haiku{Maar reeds was Dandy.}{lachend en jongensachtig}{op haar toegesneld}\\

\haiku{Celly glimlacht met,.}{een zacht rood dat over haar oogen}{en haar voorhoofd trekt}\\

\haiku{trouw zit hij bij haar.}{op haar kamer als zij van}{kantoor terug komt}\\

\haiku{{\textquoteleft}Lieveling{\textquoteright} hebben.}{Dandy's lippen zich reeds om}{haar mond gesloten}\\

\haiku{Dandy voelt hoe haar.}{lichaam terug wijkt en toch}{van verlangen trilt}\\

\haiku{Maar de volgende.}{maal is alles anders en}{de vrouw niet meer zwak}\\

\haiku{Dandy - de overgang -:}{was hem zelf een raadsel moest}{toen opeens denken}\\

\haiku{{\textquoteleft}Veronderstel eens,,.}{dat je een man ontmoette}{die je vertrouwdet}\\

\haiku{Maar hij voelt zich zelf,.}{zoo landerig dat hij een}{nieuwe flesch bestelt}\\

\haiku{Nauwelijks luistert.}{hij naar het doezelige}{praten van Harry}\\

\subsection{Uit: De Clowns en de fantasten (onder ps. Frederik Chasalle)}

\haiku{De necromant De}{necromant Over zijn glazen}{stolpen en bollen}\\

\subsection{Uit: Galop chromatique}

\haiku{Een smeulend vuur, dat.}{slechts op den windstoot wachtte}{om te ontvlammen}\\

\haiku{Een attractie van.}{den zwarten salon was het}{somnambulisme}\\

\haiku{enkelen bleven,...}{te lang op den grond op zoek}{naar elkaars monden}\\

\haiku{Dat was nu eenmaal {\textquoteleft}{\textquoteright}.}{de aardige gewoonte}{dersleutelromans}\\

\haiku{Het leven was niet.}{meer interessant zonder}{een Grote Liefde}\\

\haiku{En toch zou zij na:}{de eerste ontmoeting al}{zichzelf bekennen}\\

\haiku{twijfel in zich wil,:}{doden zelfs zijn jeugdliefde}{verloochenend}\\

\haiku{Dat zal je meteen.}{onsterfelijk maken als}{je komt te sneven}\\

\haiku{Zelfs Schlesinger moet.}{glimlachen bij Liszt's}{bravourtirade}\\

\haiku{de kogels hebben.}{het luchtruim of de takken}{der bomen doorboord}\\

\haiku{Het is Liszt, die,.}{Parijs is ontvlucht op den}{loop voor zijn hartstocht}\\

\haiku{Het is enkel zaak...}{dezen waan tot aan het graf}{te laten duren}\\

\haiku{Stil en voornaam als,.}{hij zelf is tovert hij in}{zijn spel het zonlicht}\\

\haiku{Het knutselen was.}{altijd een liefhebberij}{van haar gebleven}\\

\haiku{De uitstapjes en.}{de gesprekken golden voor}{de ontspanningsuren}\\

\haiku{{\textquoteright} Ja, Liszt wist het,,.}{nog als zoveel dat hij thans}{vergeten wenste}\\

\haiku{Nu slaat de ander.}{de ogen op en hun blikken}{ontmoeten elkaar}\\

\haiku{{\textquoteleft}Het is au fond maar,.}{een eenvoudige waarheid}{die ik je vertel}\\

\subsection{Uit: Gustaaf}

\haiku{Het zwijgen kent men.}{van hem als de uiting van}{een groote opwinding}\\

\haiku{De aderen op het.}{voorhoofd van den jongen zijn}{gezwollen van drift}\\

\haiku{Maar er is nog de,.}{glanzende verstilling die}{begeerteloos maakt}\\

\haiku{Het is t\`e banaal.}{om met je muziekmeester}{te gaan flirten}\\

\haiku{Het hout is gloeiend,.}{van de brandende zon maar}{het rusten doet goed}\\

\haiku{De menschen zullen}{ook niet begrijpen waarom}{het laatste deel zoo}\\

\haiku{- O, hoe gaarne zou;}{ik nog tot den lichten dans}{der sferen opgaan}\\

\haiku{En toch blijft er een,:}{heimelijke stem in hem}{wakker waarschuwend}\\

\haiku{Als Gustaaf zich van,.}{dit ziekbed verheft voelt hij}{zich lichter jonger}\\

\haiku{Den bliksem hebben,.}{zij als zweep den donder als}{wagenraderen}\\

\subsection{Uit: De ijzeren maarschalk. Het leven van Daendels, 'soldat de fortune'}

\haiku{Krayenhoff loopt met.}{bedenkelijk gebogen}{hoofd te luisteren}\\

\haiku{Steun van buiten werd.}{alleen gezocht met het oog}{op steun naar binnen}\\

\haiku{Als op den dreun van}{een kinderversje leeren de}{Hattemsche burgers}\\

\haiku{Hoogst beleedigd beklaagt.}{zij zich bij haar broeder den}{koning van Pruisen}\\

\haiku{Daendels, met fakkels,.}{bijgelicht ziet op een kaart}{zijn positie na}\\

\haiku{De {\textquoteleft}burgers{\textquoteright} kunnen.}{het gewicht van hun nieuwe}{positie niet aan}\\

\haiku{In de hoek van het;}{rood-wit-en-blauw staat de}{leeuw in het tuintje}\\

\haiku{Best bier hebben de....}{Bataven en vooral hun}{jenever is goed}\\

\haiku{Eenigen tijd later.}{zit er een nieuwe Fransche}{gezant in den Haag}\\

\haiku{{\textquoteright} Van Langen houdt de.}{handen met een wanhopig}{gebaar wijd uiteen}\\

\haiku{Buiten staan zijn drie.}{compagnie\"en grenadiers}{aangetreden}\\

\haiku{Maar op de smalle.}{zandstrook kunnen zij slechts een}{voor een uit komen}\\

\haiku{Patriot of geen, {\textquoteleft}{\textquoteright}.}{patriot op deflinke}{kerels komt het aan}\\

\haiku{De koning zelf is,;}{een melancholiek uitziend}{wat gebrekkig man}\\

\haiku{Storm over Indi\"e I.}{Daendels komt op Nieuwjaarsdag}{1808 in Indi\"e aan}\\

\haiku{Zoo bluft de rijkdom.}{van de eene tegen die van}{de andere op}\\

\haiku{wat zich overdag voor.}{de warmte schuilhoudt wordt het}{een vroolijk vertier}\\

\haiku{Als de stemming er,:}{is komen ook als vanzelf}{de vechtpartijen}\\

\haiku{Buitenzorg is zijn!}{eigendom en hij mag er}{mee doen wat hij wil}\\

\haiku{De 29sten Juni gaat {\textquoteleft}{\textquoteright}.}{hij te Soerabaya scheep op}{de korvetSapho}\\

\haiku{Maar helaas ik had,.}{slechts dappere soldaten}{maar geen zeelieden}\\

\haiku{Bij een onderhoud}{met Decr\`es heeft hij dezen}{onder zijn dictee}\\

\haiku{dat zijn vermogen.}{300 mille bedraagt kan hij}{dat niet veel vinden}\\

\haiku{Hij ziet niet graag zijn.}{collega's met de eer van}{den veldtocht strijken}\\

\haiku{Om het behoud van,.}{het leger om het behoud}{van Napoleon}\\

\haiku{een uitval, met het,.}{doel vee te rooven uit het kamp}{der Russen mislukt}\\

\haiku{Zelfs Kossecki geeft.}{de onhoudbare toestand}{van de vesting toe}\\

\haiku{Het ongeluk van.}{ons Vaderland vereenigt}{alle Hollanders}\\

\haiku{Wat zal Daendels doen,?}{dien men in zijn land niet heeft}{willen gebruiken}\\

\haiku{Als Daendels nog eens?}{bij zijn vroegeren Keizer}{emplooi ging vragen}\\

\subsection{Uit: Margreet vervult de wet}

\haiku{Heb je toen niet bij,?}{jezelve besloten dat}{het uit moest wezen}\\

\haiku{Zij bleef naast hem gaan,,:}{zwijgend nu hij voelde het}{meer dan hij het zag}\\

\haiku{{\textquoteleft}En de volgende,.}{week ga ik naar X ik kom}{daar op een kantoor}\\

\haiku{{\textquoteright} vroeg zij nogmaals aan.}{zichzelf en in haar toon klonk}{een lichte zelfspot}\\

\haiku{Alles moest klaar voor.}{haar wezen en ook dit woord}{zei haar te weinig}\\

\haiku{in hoeverre kon?}{hier van opzettelijke}{moord sprake wezen}\\

\haiku{Maar laat ik straks over,.}{Ferdinand vertellen als}{ik aan hem toe ben}\\

\haiku{het besef, dat God}{overal rondom je is en}{met Argus-oogen}\\

\haiku{Ik ging immers niet,,.}{alleen ik ging met Greet en}{Her die ik kende}\\

\haiku{Om die twijfel van,.}{mij weg te doen nam ik zijn}{hand in de mijne}\\

\haiku{Wat beteekende het?}{nog voor mij als Ferdinand}{mij er om verliet}\\

\haiku{Het tumult van haar.}{gedachten en gevoelens}{werd steeds heftiger}\\

\haiku{Weinig, bijna niets,,.}{een drama dat zij alleen}{maar kon vermoeden}\\

\haiku{Maar in dat mooie boek,:}{van Faust staat ook dat er een}{stem van omhoog riep}\\

\haiku{Haar oogen bevreemdden,}{haar het was of hun effen}{grijze blik slechts v\'o\'or}\\

\haiku{Niet met het gevoel,,.}{met het gevoel alleen wil}{ze de dingen doen}\\

\haiku{En ook hij zal voor,!}{An Winters moeten pleiten}{of hij wil of niet}\\

\haiku{Ik heb het immers,?}{niet gewild dat weet zij toch}{net zoo goed als ik}\\

\haiku{Hij zou beginnen.}{met te wijzen op deze}{zedelooze tijden}\\

\haiku{Zij weet, dat zij in,}{een zaak optreedt waarin zij}{uiterlijk bezien}\\

\haiku{Achter blijven, in,.}{de stoflucht der verlaten}{kamers de acten}\\

\haiku{Het moest mogelijk,,.}{zijn haar hart zei het haar dat}{het mogelijk was}\\

\haiku{Het is haar haast een,,.}{vreugde een bevrijding het}{z\'o\'o voor zich te zien}\\

\haiku{{\textquoteleft}Wie van u zonder,.}{zonden is die werpe de}{eerste steen op haar}\\

\haiku{Het had hem geheel,.}{vervuld als iets plezierigs}{dat hij voor zich zag}\\

\haiku{{\textquoteright} riep Moeder, {\textquoteleft}Agnes,,.}{laat de honden uit ga een}{grachtje met ze om}\\

\haiku{Verdachte heeft op.}{achtjarige leeftijd haar}{ouders verloren}\\

\haiku{het besef, dat God}{overal rondom je is en}{met Argus-oogen}\\

\haiku{Ik ging immers niet,,.}{alleen ik ging met Greet en}{Her die ik kende}\\

\haiku{Wat haar tot haar daad.}{moest brengen was op zichzelf}{al niet meer normaal}\\

\haiku{{\textquotedblleft}Wie van u zonder,{\textquotedblright}.}{zonde is die werpe de}{eerste steen op haar}\\

\haiku{{\textquoteright} ~ Na een kwartier.}{keerden de rechters uit de}{raadkamer terug}\\

\haiku{Ongetwijfeld, wat.}{met menschenkrachten mogelijk}{was had zij beproefd}\\

\haiku{In mijn hart heb ik,}{ook nooit durven aannemen}{dat je bereiken}\\

\subsection{Uit: De vuistslag}

\haiku{Zoo'n lichaam valt te:}{zwaar neer en kan slechts roerloos}{blijven na de smak}\\

\haiku{Mijn houding zal ik,.}{zelf wel bepalen daarvoor}{heb ik haar niet noodig}\\

\haiku{Het sloeg haar op haar,.}{zenuwen dit leven met}{John in \'e\'en huis}\\

\haiku{Een gezin had hen{\textquoteright},.}{beiden beter gebonden}{besliste Anka}\\

\haiku{Heel waarschijnlijk is}{die vrouw even laf als jij en}{was haar eenige zorg}\\

\haiku{Neen, ik kan me niet,.}{voorstellen dat ik me zou}{hebben laten slaan}\\

\haiku{Een romantische,;}{overbodigheid meende zij}{van John's daad}\\

\haiku{De tijd, dat zij zich - -:}{afvroeg en zij glimlachte}{als was het komiek}\\

\haiku{Het is reeds lang drie,.}{uur voorbij zoo aanstonds zal}{het vier uur wezen}\\

\haiku{weet hij toch wel, dat.}{hij zich driftig maakt om een}{reclamewijzer}\\

\haiku{Het lichaam is wat,.}{uitgerust de zenuwen}{zijn wat ontspannen}\\

\haiku{Als het noodlot heeft.}{door John's leven een}{paard gegaloppeerd}\\

\haiku{Beiden omhult een,.}{stofwolk die als dunne rook}{optrekt over de weg}\\

\haiku{Ik ben haar niet goed,,.}{genoeg geweest een ander}{was haar beter goed}\\

\haiku{Hij heft de oogen op,.}{de eene wenkbrauw iets hooger}{dan de andere}\\

\haiku{En vlak bij, als een,:}{ontploffing in zijn oor hoort}{hij Paula gillen}\\

\haiku{Ten slotte is het.}{toch niet veel anders dan een}{ceremonieel}\\

\section{Jacob Campo Weyerman}

\subsection{Uit: Den Laplandschen tovertrommel}

\haiku{Ook al valt de naam,:}{Campo enige malen de}{schrijver blijft naamloos}\\

\haiku{wijsheid is zowel.}{de basis als de bron van}{een goede pen11}\\

\haiku{Vergader akers, pluk, ';}{kornoeljes die int Sticht}{Zo heerlyk gloeien}\\

\haiku{dat torst een kroon, Ghy,}{die den straffen raad verzelt}{die harde bollen}\\

\haiku{3Meer informatie:}{over dit werk is te vinden}{in T. van der Meer}\\

\haiku{19Zie nummer 103.}{in de bibliografie}{van Marleen de Vries}\\

\section{Erich Wichman}

\subsection{Uit: Het witte gevaar}

\haiku{oude is goed voor -}{de koude en jonge is}{goed voor de longen}\\

\haiku{Bieren zijn goed voor,,}{de nieren Jenever is}{goed voor de lever}\\

\haiku{Een Koninkrijk, neen,,!}{liever een Republiek voor}{een grooten fopspeen}\\

\haiku{4Het {\textquoteleft}melkkapitaal{\textquoteright}.}{heeft zich ook reeds van deze}{leus meester gemaakt}\\

\section{Willem Jacob Domis, Simon Eikelenberg, Jan Jacobsz Stoop en Jacob Dircsz Wijnkoper}

\subsection{Uit: Kroniek van Wijnkoper}

\haiku{en aan 't eynde,:}{derzelven een verklaring}{in dezer voegen}\\

\haiku{so vroeg hij om de,}{slagorde van den vijand}{en daar op gebood}\\

\haiku{Leeuwaarden], geboren}{van Alcmaer worde doot}{geslagen vanden}\\

\haiku{Coning Philip is.}{geboren anno 1527 in}{maij den 12 dag}\\

\haiku{] Op dit jaer is die '.}{verstoelinge geschiet op}{t lant van Overdie}\\

\haiku{] Den Oosterdijk met.}{4 watermolens binnen}{Geestmerambacht gemaakt}\\

\haiku{] Anno 1562 en 1563 '.}{ist concilium van}{Trent gehouden}\\

\haiku{Anno ut supra.}{begost men de toorn van de}{waag af te breken}\\

\haiku{Daer lach een groote vier,}{aen daer hebben sij haer bij}{gewarmt seggende}\\

\section{Eddy Wijnkoop}

\subsection{Uit: Tim Taccle}

\haiku{Vertelt U eens, wat,.}{U bekend is als U zoo}{vriendelijk zijn wilt}\\

\haiku{Zij had niet op de.}{tijd gelet en was eerst laat}{naar huis gekomen}\\

\haiku{{\textquoteright} De detective.}{bedwong zijn verwondering}{over deze woorden}\\

\haiku{Bijzonderheden.}{omtrent zijn signalement}{kon hij niet brengen}\\

\haiku{{\textquoteright} {\textquoteleft}Ja, mijnheer, ik heb,{\textquoteright},:}{U een verzoek te doen zei}{onze vriend en toen}\\

\haiku{Zij was nu echter.}{waarachtig een beetje bang}{geworden van hem}\\

\haiku{Morgen zou hij mr.,.}{Taccle opzoeken om eens}{met hem te spreken}\\

\haiku{Er wordt geklopt en.}{op zijn uitnoodiging komt een}{verpleegster binnen}\\

\haiku{wij hebben ons geld.}{gehad en zijn verders niets}{meer te verwachten}\\

\haiku{maar het komt mij goed,.}{uit dat U Uw belofte}{niet hebt gehouden}\\

\haiku{Want het prettige,}{van de zaak was dat geen der}{geattaqueerden}\\

\haiku{Werkelijk, ook dat.}{is in het belang van de}{zaak noodzakelijk}\\

\haiku{Als eerste daad zocht.}{hij de verblijfskaart van den}{ouden Strenger op}\\

\haiku{Hopkins verlegde dus.}{zijn interesse naar de}{rest van het gebouw}\\

\haiku{Zijn schot barstte los,.}{gevolgd door een knal uit de}{andere pistool}\\

\haiku{hij zou met behulp.}{van beitel en hamer de}{bergplaats openbreken}\\

\haiku{{\textquoteright} {\textquoteleft}Nu, maakt U zich nu,.}{dan maar geen zorgen meer Uw}{geheim is veilig}\\

\haiku{Maar gaat U verder,;}{neemt U mij niet kwalijk dat}{ik U onderbrak}\\

\section{Augusta de Wit}

\subsection{Uit: De avonturen van den muzikant}

\haiku{Een stilte in de,,.}{lucht een reuk uit den grond een}{zachtheid in het licht}\\

\haiku{Allard zette de.}{lippen aan de opening en}{blies uit alle macht}\\

\haiku{De Boegi kwamen,.}{in schepen hoog van boeg als}{heuvels op de zee}\\

\haiku{Zoo dikwijls vroeg hij.}{er om dat zij het uit het}{hoofd had leeren spelen}\\

\haiku{Het was nog in de,.}{eerste vroegte het koele}{begin van den dag}\\

\haiku{Daarboven begon,,.}{een lichte melodie speelsch}{lachend enkel vreugd}\\

\haiku{Zijn zoon, Benkol, bracht.}{hem water uit het ravijn}{en hout voor het vuur}\\

\haiku{De vader reikte.}{naar dat witte wangetje}{om het te streelen}\\

\haiku{Als in een spiegel.}{zag hij in het gezicht der}{moeder naar zijn kind}\\

\haiku{Maar hoe armer wij,.}{zijn hoe meer wij de vreugd der}{muziek noodig hebben}\\

\haiku{Maar als de klokken,.}{begonnen te spelen dan}{luisterde alles}\\

\haiku{Op het Eiland, waar,.}{ik geboren ben klinkt dan}{zoo de gamelan}\\

\haiku{Maar de blik uit die,.}{doordringende oogen was z\'oo}{goed daar kwam moed van}\\

\haiku{Daar waren Haydn en,.}{Mozart de goede geesten}{van Beethovens jeugd}\\

\haiku{twee anderen de,;}{een een banjo de ander}{een mandoline}\\

\haiku{Zij gingen al de,.}{straten door de arme zoo}{goed als de rijke}\\

\haiku{maar als hij toch maar!}{tot na de Pinksteren had}{gewacht met komen}\\

\haiku{op de gelige.}{bladen bleven woorden en}{noten duidelijk}\\

\haiku{Hij hoorde er de.}{dagelijksche geluiden}{van de hofstede}\\

\haiku{maar door die strenge.}{statigheid heen breekt telkens}{een gloed van hartstocht}\\

\haiku{Het is of de stroom.}{rood wordt in den afschijn van}{een brandende stad}\\

\haiku{Zij bad Allard te;}{beginnen met de studie}{voor ingenieur}\\

\haiku{Maar hij liet het in.}{zijn zak zitten en deed braaf}{zijn best op een klomp}\\

\haiku{Ik moet nu eenmaal.}{naar de Schoone Danseres}{in den Waterval}\\

\haiku{Hij liep voort langs de,.}{rivier die gaf nog licht langs}{het zwartige riet}\\

\haiku{Juist doofde het licht;}{in het boogvenster en de}{menschen gingen heen}\\

\haiku{De man wischte zich:}{het zweet van het gezicht en}{haalde adem en riep}\\

\haiku{Hij sloeg Joris,.}{op den schouder en schudde}{hem haast de hand af}\\

\haiku{zulk een waterval,?}{als deze woont daarin de}{Schoone Danseres}\\

\haiku{wat had hem bezield,!}{dat hij juichte over wat nu}{hem woedend maakte}\\

\haiku{maar toen pas goed, wat.}{je over hebt voor de muziek}{en voor de vriendschap}\\

\haiku{Nu gingen aan de.}{bergbeek nieuwe wijdten open}{voor zijn gedachte}\\

\haiku{enkel klaarte en,.}{liefelijkheid een beekje}{door den zonneschijn}\\

\haiku{Hij speelde wat een:}{lievelingsstuk was geweest}{van zijn grootvader}\\

\haiku{Alois Tieffenbrucker,,.}{fecit en een datum van}{veertig jaar her 1860}\\

\haiku{Daarom moesten ook zijn;}{f-gaten anders zijn}{dan die van Stainer}\\

\haiku{Maar weer peinsde hij,,,.}{te te vergeefs waar wanneer}{hij dien had gehoord}\\

\haiku{gaarne, zei hij, liet,.}{hij hem de viool houden}{zoolang hij wilde}\\

\haiku{en de jaren van.}{nood wanneer de vezel los}{en slap was gegroeid}\\

\haiku{Is, dan, zoo broos het,?}{schoone dat het breekt aan de}{eigen volmaking}\\

\haiku{{\textquoteright} Betooverd zag hij op,.}{de ranke rosblonde de}{leidster van den dans}\\

\haiku{{\textquoteright} {\textquoteleft}Bijen, wij doen U,.}{te weten dat de oude}{Vrouw gestorven is}\\

\haiku{Een is er over den,;}{uil die zit te kijken met}{zijn starende oogen}\\

\haiku{tirannie vervoert.}{mij tot een hartstocht dien ik}{niet kan beheerschen}\\

\haiku{{\textquoteleft}Het eerste is geld.}{en het tweede is geld en}{het derde is geld}\\

\haiku{den korten zin van.}{deze lange rede heb}{ik nog in petto}\\

\haiku{De naam van een neef,,.}{die haar mans compagnon was}{klonk telkens daarin}\\

\haiku{Zij lachte zelve,.}{en dat was als het klokken}{van een bronnetje}\\

\haiku{Allard hielp Lucie;}{er in en roeide naar het}{midden van het meer}\\

\haiku{Aloisl, die met een,:}{tak lijsterbessen zwaaide}{kreet uit volle keel}\\

\haiku{een warme wind droeg,.}{geur aan van hars varenkruid}{en klein gebloemte}\\

\haiku{En wie was hij zelf?}{om zelfs maar uit de verte}{tot haar op te zien}\\

\haiku{{\textquoteright} De herinnering:}{aan den toon van haar stem bij}{dat haast onhoorbaar}\\

\haiku{Zij viel mij in het -,!}{oog door haar kleur fel groen groen}{als een kikvorsch}\\

\haiku{- dan zal zij zingen!}{dat de engelen in den}{hemel glimlachen}\\

\haiku{De varkensslachters!}{betalen uit beurzen als}{hun varkens zoo vet}\\

\haiku{Mannen, vrouwen en}{kinders uit de bergdorpen}{grepen naar zijn hand.}\\

\haiku{Later had Allard;}{op zijn vriends werktafel het}{manuscript gezien}\\

\haiku{Zij vermaanden hem.}{te sparen wat zijn eigen}{muziek behoefde}\\

\haiku{Het gerucht verstierf,,.}{de haast verlangzaamde het}{geweld verstilde}\\

\haiku{Ook velen kwamen:}{die bij dage aan feesten}{niet konden denken}\\

\haiku{Onder het lezen.}{moest Allard telkens denken}{aan Tieffenbrucker}\\

\haiku{{\textquoteleft}Wie weet of hij het,{\textquoteright}.}{er niet weer in voegde nu}{zei Tieffenbrucker}\\

\haiku{, en tot tjilpende.}{vogeltjes van de blijheid}{sprak die eeuwig is}\\

\haiku{Het rosse goud der.}{lokken gloorde boven een}{lelieblanken nek}\\

\haiku{Het instrument van;}{de eerste viool was zijn}{werk en zijn geschenk}\\

\haiku{Zij knielde alleen.}{om de verheffing uit te}{zingen van haar geest}\\

\haiku{nu reeds maakten zij:}{telkens aanmerkingen op}{zijn repetities}\\

\haiku{en verweerden zij {\textquoteleft}}{zich zoo woedend omdat een}{binnenste stemJa}\\

\haiku{{\textquotedblright} tot dit dochtertje.}{van Cunera tevergeefs}{gezegd zal blijken}\\

\haiku{{\textquoteleft}Maar toen heeft Kempfen,.}{gedacht hoe minder woorden}{daarover hoe beter}\\

\haiku{Hoe vond hij ineens?}{de woorden weder van de}{taal der Eilanders}\\

\subsection{Uit: De drie vrouwen in het Heilige Woud}

\haiku{{\textquoteright} en ging heen uit het,;}{prachtige paleis door vrouw}{noch dienaar gevolgd}\\

\haiku{Telkens echter ging;}{zij met bloemenoffers naar}{het Heilige Woud}\\

\haiku{Ook toen het tijd werd.}{om de sawahs te ploegen}{kwam hij niet terug}\\

\haiku{Haar armen deden.}{pijn van verlangen naar zulk}{een glad klein lijfje}\\

\haiku{Er was stof op haar,}{ruig haar achteloos hing haar}{sarong dien zij tot}\\

\haiku{en hem  ook scheen}{het dat niet anders dan een}{zieke naar den geest}\\

\haiku{Mboq-Inten nam.}{haar hand en streelde die langs}{haar eigen gezicht}\\

\haiku{van  doornigen.}{rottan die met gehaakte}{zweepen naar hen striemt}\\

\haiku{Flikkerend in de.}{middagzon opgevlogen}{schreeuwen zij van vreugd}\\

\haiku{Groen en goud stond het;}{kustgebergte in schoonen}{halfboog te blinken}\\

\haiku{Nog eens beproefde,.}{hij de spanning van zijn zeil}{het spel van zijn roer}\\

\subsection{Uit: Het dure moederschap}

\haiku{Als een groene zee,,.}{lag zij daar blinkende te}{deinen in de zon}\\

\haiku{hij dacht dat het wel.}{aardig zou wezen kermis}{te houden met haar}\\

\haiku{Hij heeft zijn nest op,{\textquoteright}, {\textquoteleft}.}{Hartestein zei Tijmener}{zijn vijf jongen in}\\

\haiku{Zij had het licht op.}{en de halfdeur open voor de}{andere klopte}\\

\haiku{Dien Zondag in de.}{kerk leek het haar of Tijmen}{telkens naar haar keek}\\

\haiku{geen zonneschijn nog.}{maar de kleurlooze klaarte die}{er aan voorafgaat}\\

\haiku{op dat oogenblik,.}{eerst begreep hij dat zij van}{niets geweten had}\\

\haiku{Halfweg thuis, meende, {\textquoteleft},!}{zij hem nog te hooren met}{zijn snikkendMoetje moetje}\\

\haiku{Toen zij gehoord had,,:}{waar Fokje was ging zij hem}{dadelijk halen}\\

\haiku{Eens zelfs wou hij het.}{bed uit en bleef een poosje}{zitten bij het vuur}\\

\haiku{volstrekt noodig niet, of,?}{het dan goed was voor het kind}{of hij het aanried}\\

\haiku{De grond deinde voor,.}{haar voeten of zij z\'o\'o in}{het leege zou treden}\\

\haiku{Marretje dacht al,:}{dat hij sliep toen hij ineens}{rechtop ging zitten}\\

\subsection{Uit: De godin die wacht}

\haiku{Je zult zien dat je,!}{nog niet eens klaar bent met je}{werk voor het om is}\\

\haiku{Hij smeet het scheldwoord.}{op goed geluk naar \'een van}{die twee gezichten}\\

\haiku{{\textquoteleft}Er was een kuil in,{\textquoteright}.}{den weg zei de voerman in}{zijn neurend maleisch}\\

\haiku{na een kwartier leek.}{het hem of hij ze beiden}{al lang gekend had}\\

\haiku{Daarop klonk een stap.}{over het kiezelpad en van}{Heemsbergen verscheen}\\

\haiku{De rechter zag hem,.}{aan verwonderd als over een}{nog nooit gehoord iets}\\

\haiku{Hij komt zeker veel,?}{aan huis bij professoren}{en zoo is het niet}\\

\haiku{Je mag al dankbaar!}{zijn als je achterstand niet}{\`al te erg oploopt}\\

\haiku{Een plotselinge;}{ruk slingerde hem tegen}{van Heemsbergen aan}\\

\haiku{de Bakker zit daar,{\textquoteright}.}{binnen voegde hij er met}{een knipoogje bij}\\

\haiku{{\textquoteleft}Dat is hier vandaan,{\textquoteright}.}{te hooren antwoordde de}{controleur droogjes}\\

\haiku{{\textquoteleft}Ja, de kerel is,.}{niet te vinden geweest juist}{zooals je voorspeld hadt}\\

\haiku{Ook een beste man,,.}{een beste man als je hem}{maar eerst leert kennen}\\

\haiku{Ik ken hem nu al,.}{een paar jaar en ik kan het}{best met hem vinden}\\

\haiku{Van Heemsbergen, die,.}{dacht goed Maleisch te kennen}{verstond er niets van}\\

\haiku{En dan weer op zijn,,!}{plaats hoor je daar achter op}{de bovenste plank}\\

\haiku{en vroeg naar dingen.}{waarvan hij op zijn best wist}{dat zij bestonden}\\

\haiku{{\textquoteright} {\textquoteleft}De Bakker heeft me,..... -,?}{den naam wel gezegd maar wacht}{eens Bruton kan dat}\\

\haiku{W\`at hij eigenlijk,.}{van hem verwachtte had hij}{niet kunnen zeggen}\\

\haiku{Een ijle geur van.}{kruiden verlevendigde}{de zuivere lucht}\\

\haiku{{\textquoteleft}Hij heeft graag dat ik,,....}{luister naar wat er gezegd}{wordt soms over en weer}\\

\haiku{{\textquoteleft}Het is mijn werk,{\textquoteright} zei,.}{hij na een oogenblik en}{zijn toon was weer koel}\\

\haiku{Nu stond hij stil voor:}{Hendriks en zijne vrouw en}{vroeg hartstochtelijk}\\

\haiku{In erger tweedracht.}{met zich zelven dan toen hij}{ging kwam hij weerom}\\

\haiku{Hij zei het hardop,,.}{in zijn verbazing en vond}{verder geen woorden}\\

\haiku{En ondertusschen!}{zit ik hier en niemand hoort}{of ziet wat van me}\\

\haiku{Dat is goed om over.}{twintig jaar een standaardwerk}{te kunnen schrijven}\\

\haiku{In en rondom het:}{machinegebouw gonsde}{het van de drukte}\\

\haiku{{\textquoteleft}En als je ze dan -!,!}{van dichtbij O die \'eene}{wat een prinsesje}\\

\haiku{Een dommelige:}{herinnering schoot wakker}{in van Heemsbergen}\\

\haiku{Hij zag een woord op.}{de lippen van den schilder}{en hield het tegen}\\

\haiku{Je verwijt me dat....}{ik niet genoeg belang stel}{in den Inlander}\\

\haiku{Ik heb mijn knuppel,.}{naar ze gesmeten maar ik}{raakte er niet een}\\

\haiku{Op zijn beurt keek hij.}{eens naar wat den ander zoo}{opgetogen hield}\\

\haiku{Als ik niet wist dat....}{mevrouw de Bakker naar me}{zou laten vragen}\\

\haiku{De een wist zich niets;}{meer te herinneren bij}{de ondervraging}\\

\haiku{Eerst had hij den voogd:}{van den overledene met}{hem vereenzelvigd}\\

\haiku{{\textquoteright} {\textquoteleft}Die man achter de;}{schermen is nu een idee van}{Bossing en van u}\\

\haiku{de woorden zich in,:}{den mond vormen waarop hij}{wist dat zij wachtte}\\

\haiku{Als een zilveren.}{plas lag het maanlicht binnen}{oevers van schaduw}\\

\haiku{Hij hoorde alleen,,:}{in zich diezelfde nog maar}{\'eens vernomen stem}\\

\haiku{Beide handen op.}{haar schouders leggend hield hij}{haar even van zich af}\\

\haiku{{\textquoteleft}U hebt zeker al -.}{wel geraden wie ik ben}{de vriendin van Ada}\\

\haiku{Maar hij voelde hun.}{fijne stralen prikkelen}{door zijn duisternis}\\

\haiku{genoeg zou zijn voor.}{een sitsen kabaja of}{een netten hoofddoek}\\

\haiku{- Dat beeld waar ze zoo,.}{veel offeren laten we}{daarnaar gaan kijken}\\

\haiku{de greep die hij om,.}{haar polsen geslagen had}{spande zijn vingers}\\

\haiku{Met je welnemen, {\textquotedblleft}{\textquotedblright},!}{zoo'n volmondigja zoo ja}{en amen is dat niet}\\

\haiku{De groote meerderheid!}{doet dat. Op dat punt zijn de}{meesten Inlanders}\\

\haiku{als ik leef naar mijn.}{hoogste dan als ik leef naar}{mijn laagste kunnen}\\

\haiku{{\textquoteleft}Nu we er toch over,, -!}{spreken laat nu ook alles}{gezegd zijn alles}\\

\haiku{Ik sprak alleen maar;}{van hem om je geheugen}{op weg te helpen}\\

\haiku{Mevrouw Meerhuys zat,;}{aan de tafel in het}{schijnsel van de lamp}\\

\subsection{Uit: Gods goochelaartjes}

\haiku{Het was oorlog op -.}{Ambon toen het was altijd}{oorlog op Ambon}\\

\haiku{Er zijn er niet zoo.}{velen die lust hebben hem}{daar te gaan vangen}\\

\haiku{het verhaal van zijn;}{drieen-twintigsten tocht den}{Mont-Ventoux op}\\

\haiku{het moest doodsangst zijn.}{die hun de wanhopige}{kracht had gegeven}\\

\haiku{Hij had boeken, riep.}{hij uit die een Museum}{hem zou benijden}\\

\haiku{Hoeveel mijn vader;}{van de zaak wist hebben wij}{zoons nooit vernomen}\\

\haiku{Voor wien werkten hij?}{en mijn moeder anders dan}{voor ons kinderen}\\

\haiku{Dan, zachtjes, klom ik;}{het raam uit en in de kruin}{van den magnolia}\\

\haiku{Fabre, ik wist,.}{het had er twintig jaar te}{vergeefs naar gezocht}\\

\haiku{Hij zag mij aan of.}{hij dacht dat ik plotseling}{gek was geworden}\\

\haiku{{\textquoteright} En hij ging de trap.}{achter mij op en sloot mijn}{deur van buiten af}\\

\haiku{Ik verveelde mij!}{schromelijk tusschen al die}{doode insecten}\\

\haiku{En hier op Ambon,,.}{zag ik kon een mensch leven}{van zoo goed als niets}\\

\haiku{maar sterk gevormd, als,,;}{gebeiteld hier zwak belijnd}{als uitgewischt daar}\\

\haiku{En als er iemand,,}{uit het dorp komt hoort hij hem}{aan maar zelf zegt hij}\\

\haiku{vernieling straks, was.}{het vruchtbaarheid nu over het}{afgeoogste land}\\

\haiku{Jan moest dadelijk.}{Meneer Schepers narijden}{en hem thuis brengen}\\

\haiku{Hij had een klaproos}{achter zijn oor gestoken}{die wijdopen vlamde}\\

\haiku{Maar die vreugde zou.}{voorspel zijn van de Vreugde}{in der Eeuwigheid}\\

\haiku{In mijn kamertje:}{boven den stal kon ik een}{piano zetten}\\

\haiku{O hoe mooi heb ik!}{menschen zien worden die vaal}{en dof waren eerst}\\

\haiku{Hij sloeg den deksel,,.}{op en begon te spelen}{mijn vlinderdeuntje}\\

\haiku{hij wist dat wat voor,;}{slechtheid wordt veroordeeld \'ook}{leed is het ergste}\\

\haiku{De gieren van die,.}{nachtmerrie die mij zoo lang}{had gekweld schreeuwden}\\

\haiku{Zij alleen weten,.}{raad die niet macht en bezit}{maar liefde willen}\\

\haiku{ik heb je altijd;}{geprezen om je schoonheid}{en je vroolijkheid}\\

\haiku{Men kan het zich haast.}{niet voorstellen wat de man}{toen geleden heeft}\\

\haiku{hij zich had verheugd,.}{als Vader in mij verblijd}{ook om zichzelfs wil}\\

\haiku{Opeens stak Herman,.}{mij de hand toe ik legde}{er de mijne in}\\

\haiku{De kleine man, van,.}{doorstane smarten zoo wreed}{geteekend jubelde}\\

\subsection{Uit: Orpheus in de dessa}

\haiku{Zoo bedwelmd was hij,.}{dat hij zelfs niet bewoog toen}{ik vlak voor hem stond}\\

\haiku{[III] Tot het donker;}{werd wachtte Bake op hem}{den volgenden avond}\\

\haiku{Hij voelde zich  .}{voortgestuwd op den stroom die}{de werelden draagt}\\

\haiku{Met een bekommerd;}{gezicht stond Bake bij de}{nieuwe machine}\\

\haiku{En dat gaf hem een,}{gevoel van vroolijken moed}{als voor een slag dien}\\

\haiku{Hij stapte van 't,.}{paard en ging in den lommer}{zitten uitrusten}\\

\haiku{Maar als het nu eens,?}{een heele bende was die}{systematisch steelt}\\

\haiku{Of ze verkoopen.}{hun oogst voor een prikje een}{paar maanden vooruit}\\

\haiku{aan den kant van den,,!}{weg blijven op het gras dat}{ze ons niet hooren}\\

\haiku{Behoedzaam droeg hij '.}{het opt koele zachte}{leger in de kar}\\

\haiku{{\textquoteright} De houding van het.}{vergroeide lichaampje leek}{hem ondragelijk}\\

\subsection{Uit: Verborgen bronnen}

\haiku{De Heggelersdijk.}{was de buurt van de stroopers}{en de smokkelaars}\\

\haiku{Wel ieder papwurm ' '.}{hier int dorp zout je}{kunnen vertellen}\\

\haiku{Ik ga naar 't dorp,!}{en haal brandewijn dat is}{veel beter voor hem}\\

\haiku{- Daar ging de deur open,.}{en het gemutste hoofd der}{meid keek naar binnen}\\

\haiku{Zijne moeder had:}{het koffie-water over}{het vuur gehangen}\\

\haiku{{\textquoteleft}Laat ze maar liever,{\textquoteright}.}{op der eigen huid passen}{antwoordde Nellis}\\

\haiku{De poelier in de.}{stad had er in geen tijden}{zulke mooie gehad}\\

\haiku{Hij droogde zich het,,:}{klamme voorhoofd af en na}{een wijle zeer zacht}\\

\haiku{Dacht de Domin\'e!}{dat ik een kameraad zijn}{haas zou afstelen}\\

\haiku{Nellis nam het glas,.}{van de mooie Jaan aan zonder}{haar toe te knikken}\\

\haiku{Er was niets te zien.}{tusschen de lage stammen}{der kerseboomen}\\

\haiku{Gerrit stapte naar,.}{zijn werk dien morgen als ging}{hij naar zijn geluk}\\

\haiku{En ie zal wel niet,.}{hard deuge ook want ie is}{in den Oost geweest}\\

\haiku{Met rinkelende,.}{schreden stapte hij heen zijn}{knevel opstrijkend}\\

\haiku{{\textquoteleft}Maar je hebt 'em dan ',,}{tochezien verdikkeme toen}{hij daar zoo vlak langs}\\

\haiku{{\textquoteleft}A-j wat gewaar,,.}{wordt dan roep je maar dan kom}{ik je wel helpen}\\

\haiku{Welzeker, zweer jij,,!}{maar zweer jij maar en zie dan}{eens wie je gelooft}\\

\haiku{Het zal je heugen,!}{dat je veldwachter Koenen}{hebt voorgelogen}\\

\haiku{{\textquoteleft}U stond toch bij den,,?}{hooiberg nietwaar waar de brand}{aangekomen is}\\

\haiku{Gerrit staarde op.}{het papier dat hij in zijn}{klamme vingers hield}\\

\haiku{Daar buiten is de.}{glorie van Veneti\"e en}{den Junihemel}\\

\haiku{Uit het doorlouterd.}{zwart van het kelkhart welde}{rijk azuur te voorschijn}\\

\haiku{Hij was er, om te,.}{zorgen dat dat geschiedde}{daarom bestond hij}\\

\haiku{{\textquoteright} Driemaal in de week.}{ging hij nu des avonds naar den}{ouden schoolmeester}\\

\haiku{{\textquoteright}... Eindelijk legde.}{hij zich op zijn leger van}{mos en varenkruid}\\

\haiku{De man staat langzaam,.}{op zonder om te kijken}{of te antwoorden}\\

\haiku{Hij droeg een nieuwe,.}{sarong en zijn schoonvaders}{kris in den gordel}\\

\haiku{Dus, dag aan dag, zag.}{Mian zijns vijands zoon schooner}{en sterker worden}\\

\haiku{Nu vielen alle,:}{vrouwen tegelijk in door}{elkaar heen roepend}\\

\haiku{Tusschen hen gleden -.}{de handen der vrouwen heen}{en weer heen en weer}\\

\haiku{En als een zwoele.}{zwarte vloed nam mij op en}{omving mij het woud}\\

\haiku{De dessaman had,}{onderweg van een hollen}{bamboestengel dien}\\

\haiku{Hij ging te midden.}{daarvan als des konings zoon}{door des konings hof}\\

\haiku{Van mijne plaats kon.}{ik haar heen en weder zien}{gaan bij het rijsvuur}\\

\haiku{Voor een oogenblik.}{hield zich het gerucht van den}{tropischen nacht in}\\

\haiku{Ik heb den dienaar,!}{van den wedono gezien}{dragende uw kind}\\

\haiku{{\textquoteleft}Wat bekommer je?}{je zoo zeer om dat kind van}{een geringen man}\\

\haiku{Zijn gezicht verwrong.}{dat het niet meer het gezicht}{van een mensch geleek}\\

\subsection{Uit: De wake bij de brug}

\haiku{{\textquoteleft}Waarom heeft hij dit,?}{gedaan hij die toch om hulp}{riep tegen den dood}\\

\haiku{De stam der Boegi,,.}{het zwervende zeevolk is}{vrienden met den wind}\\

\haiku{Van den steven naar,!}{den boeg loopend kom met een}{driftige vaart Heer}\\

\haiku{Korven vol werden.}{naar hen afgelaten en}{gulpende emmers}\\

\haiku{Daar beukten storm en.}{stortzee de jonk te pletter}{en een van hen stierf}\\

\haiku{Rond en rondom de.}{stoomboot heen ging muziekend}{de statige dans}\\

\haiku{te hard is die op:}{hun aan zacht vloeienden klank}{gewende lippen}\\

\haiku{dochters van den adel.}{kwamen daar en kinderen}{uit dessahuizen}\\

\haiku{De zenith was een,.}{zwarte wel waar wolken de}{golven in waren}\\

\haiku{De muizen hadden.}{hun kracht  opgegeten}{die nog te veld stond}\\

\haiku{Een wijze was het.}{die nu begon als niemand}{nog vernomen had}\\

\haiku{Hij dacht, wat is toch,?}{dat witte daar dat witte}{in den zwarten grond}\\

\haiku{druipenden boomstronk.}{dien het gedrocht losgewroet}{had uit de diepte}\\

\haiku{Die zij vingen en.}{aan wal trokken legden zij}{vast aan de boomen}\\

\haiku{Wij zagen het op,.}{ons afkomen wij die op}{den steiger stonden}\\

\haiku{Hij stond een poos stil.}{en zag weer naar den oever}{en keerde weerom}\\

\section{Aagje Deken en Betje Wolff}

\subsection{Uit: Historie van mejuffrouw Cornelia Wildschut}

\haiku{Mama ook niet, maar;}{zij doet veel om andere}{lui te plezieren}\\

\haiku{mijn beurs lijdt meer  *.}{door haar dan mijn ziels-}{of lichaamskrachten}\\

\haiku{Lichtmis als ik ben;}{kan ik geen vrouw dulden die}{haar sekse verzaakt}\\

\haiku{Of zij meer dom dan,;}{wel onkundig is kan ik}{nog niet bepalen}\\

\haiku{hij schijnt veel meer in.}{zijn huis gelogeerd te zijn}{dan er te wonen}\\

\haiku{zijt om u met uw?}{afwezige familie}{niet te bemoeien}\\

\haiku{*~        4 Cornelia!}{Wildschut aan Betje Stamhorst}{Mijn lieve Betje}\\

\haiku{Ik weet niets, en ben,,;}{zo als gij wel denken kunt}{verlegen om stof}\\

\haiku{Keetje heeft er geen,.}{nadeel van zij weten niet}{van wie die brief komt}\\

\haiku{nu dat weet gij, maar,.}{slapen is gezond en ook}{ik heb niets te doen}\\

\haiku{in 't eerst liep ik,;}{machtig hoog met haar maar het}{was al gauw gedaan}\\

\haiku{Kort gezegd, zo ik,;}{trouw dan zal het zeker uit}{zelfverveling zijn}\\

\haiku{waarover kan u niet,}{dan onverschillig zijn want}{gij zijt zo weinig}\\

\haiku{8 Willem Stamhorst!}{aan Paulus Wildschut Mijnheer}{zeer waarde broeder}\\

\haiku{als hij wat ouder.}{is zal dat grote vuur wel}{wat verminderen}\\

\haiku{Ik hou wel veel van,.}{nicht doch onze humeuren}{verschillen te zeer}\\

\haiku{Ging ik al eens mee,;}{dan was het om te doen als}{alle anderen}\\

\haiku{A propos Hein, het?}{is immers uw voornemen}{om haar te trouwen}\\

\haiku{{\textquoteright} Wat mijn vrijerij,.}{betreft gij kunt wel merken}{dat die niet vordert}\\

\haiku{dan vreemdelingen;}{die enig en alleen om hun}{negotie reizen}\\

\haiku{en dewijl gij met.}{een vriend uit de stad waart zag}{ik u even weinig}\\

\haiku{{\textquoteright} - {\textquoteleft}Ik zal deze avond,}{mijn kamerdeur openlaten}{en dan kunt gij zo}\\

\haiku{Gij weet dat ik u,;}{liefheb en ik weet dat ik}{daar reden toe heb}\\

\haiku{het welk niet in haar.}{moeders bijzijn kon gezegd}{of gedaan worden}\\

\haiku{zo al niet volstrekt,.}{onmogelijk echter hoogst}{onwaarschijnlijk is}\\

\haiku{De heer Wildschut poogt,:}{dit goed te maken en het}{is alle ogenblik}\\

\haiku{Ik kon daar niet veel -.}{op antwoorden ik zag het}{gezelschap eens over}\\

\haiku{in 't kort, ieder,.}{zat of bij geval of met}{oogmerk daar hij zat}\\

\haiku{Het spijt mij bijna,:}{dat ik het met haar eens was}{doch zij had gelijk}\\

\haiku{{\textquoteright} Men presenteerde,;}{thee en toen sloeg mevrouw voor}{om pand te spelen}\\

\haiku{ik hoop dat gij het.}{onnodig zult maken veel}{daarover te schrijven}\\

\haiku{{\textquoteleft}Maar,{\textquoteright} voegde zij er, {\textquoteleft},!}{bijhet is zijn zusters schuld}{mijn lieve mevrouw}\\

\haiku{en daar moet gij u,,.}{niet aan storen want het is}{zo en niet anders}\\

\haiku{En zouden onze?}{jonge juffrouwen u zo}{gauw zetten mogen}\\

\haiku{Want zulk een oorvijg.}{voor mijn eigenliefde zou}{mij razend maken}\\

\haiku{de heer Wildschut zal;}{van kwaadheid of chagrijn of}{van beide sterven}\\

\haiku{Maar gij hebt ook haar;}{vader wel duizend mijlen}{van mij verwijderd}\\

\haiku{Het zijn zo zeer niet;}{talenten en verstand die}{ons daar aanprijzen}\\

\haiku{Ja mijn goede heer,!}{Van Arkel ik heb thans een}{kruis in de wereld}\\

\haiku{{\textquoteright} Hij durfde, denk ik,,.}{mijn antwoord niet afwachten}{maar vloog de zaal uit}\\

\haiku{en Wildschut, al wil,.}{hij het niet weten ziet er}{ook ongedaan uit}\\

\haiku{hoe dikwijls heb ik!}{u dit niet door lessen en}{voorbeelden getoond}\\

\haiku{{\textquoteright} Had gij mij dat nu,;}{vooraf gevraagd ik zou u}{zulks gezegd hebben}\\

\haiku{Ik bemin haar, en.}{ik weet dat zij voor mij niet}{onverschillig is}\\

\haiku{met zo iemand, die,.}{niet ouder is is nog wel}{iets te beginnen}\\

\haiku{en dewijl de heer,}{De Groot hier juist aan huis was}{heb ik hem doen zien}\\

\haiku{men kan niet weten,?}{en waarom zou een braaf mens}{in verdriet komen}\\

\haiku{Uw zegepraal zal,.}{niet lang duren uw vader}{zal het besterven}\\

\haiku{doodsbenauwd, bijna,,,...}{naakt alles losgescheurd bleek}{en stuiptrekkingen}\\

\haiku{Kort daarop komt Frans - {\textquoteleft}!}{buiten adem gelopen en}{zeer ontsteldMevrouw}\\

\haiku{u mocht vragen - doch -,:}{dat zal niet gebeuren waar}{wij gaan zeg dan maar}\\

\haiku{{\textquoteleft}Mijnheer Lenting{\textquoteright} zei, {\textquoteleft};}{zijis sedert acht dagen}{in  commissie}\\

\haiku{{\textquoteright} Frans weigerde dit,.}{doch hij moest het doen om haar}{te vergenoegen}\\

\haiku{{\textquoteright} Met een sidderend;}{verlangen rukte ik haar}{de brief uit de hand}\\

\haiku{Zoek mij niet, onze.}{maatregelen hebben dat}{vruchteloos gemaakt}\\

\haiku{Spreekt zij niet van mij,?}{als van haar moeders man niet}{als van haar vader}\\

\haiku{Als men geen plezier,;}{van zijn kinderen heeft raakt}{het hart er ook af}\\

\haiku{De jonge heer wacht,;}{voor de deur doch het begon}{hem te vervelen}\\

\haiku{En onze lieve.}{Heer kan een mens altoos de}{bekering geven}\\

\haiku{Zij was mij (al had):}{ik zulks veilig kunnen doen}{geen geweld waardig}\\

\haiku{{\textquoteright} zei ik, {\textquoteleft}dan zult gij.}{best doen om weder naar uw}{vaders huis te gaan}\\

\haiku{Denkt gij dat zij hen?}{zal behandelen zoals}{zij mij behandelt}\\

\haiku{Ja zo waar, des daags;}{na de dag van uw vertrek}{kwam hier Frans Ligthart}\\

\haiku{, ik mij geen rust gaf.}{voor ik mijn kind gevonden}{en behouden had}\\

\haiku{Ik zag u reeds van,;}{verre en meen dat gij in}{verlegenheid zijt}\\

\haiku{Ziende dat ieder;}{zijn gewonnen geld met zich}{nam deed ik ook zo}\\

\haiku{doch ik durfde het,.}{niet te wagen uit vrees of}{men het merken zou}\\

\haiku{Ik zei mijn vader -,;}{te haten mijn hart wringt mij}{terwijl ik dit schrijf}\\

\haiku{gij zult des tegen.}{deze uw grootste vijand}{op uw hoede zijn}\\

\haiku{Deze brief, mijn kind,;}{is alles wat uw vader}{u kan nalaten}\\

\haiku{Bij hem komende.}{vond ik hem tegen een boom}{leunende zitten}\\

\haiku{{\textquoteright} barstte ik uit, {\textquoteleft}hoe!}{gaarne zou ik u als mijn}{vriend bemind hebben}\\

\haiku{{\textquoteright} Hier vouwde hij zijn,}{handen samen en zijn hoofd}{opheffende bad}\\

\haiku{{\textquoteleft}Mijn besluit{\textquoteright}, zei hij, {\textquoteleft},.}{is genomen ik ontwaak}{uit mijn bedwelming}\\

\haiku{wij zien er ook uit,.}{hoe w\'el men zich uitdrukt als}{het hart ons dicteert}\\

\haiku{Kunt gij uw arme,?}{vernederde vriendin wel}{alles vergeven}\\

\haiku{Ik ben de oorzaak ',!}{vant verlies van uw man}{mijn lieve vader}\\

\haiku{Zij schijnt zeer gerust,,.}{maar ligt doorgaans stil hoewel}{haar ogen ons volgen}\\

\haiku{Ik ben altoos uw,.}{eerbiedigende vriendin}{Anna Hofman}\\

\haiku{Zij zag haar tante,,:}{aan met een aandoenlijke}{minzaamheid en zei}\\

\haiku{Ik hoop immers dat,,?}{gij mijn kind plichtmatig denkt}{omtrent uw moeder}\\

\haiku{waarin men de fijnste;}{roersels en verborgenste}{springveren doorziet}\\

\haiku{Karakter- en.}{levensbeeld van Betje Wolff}{en Aagje Deken}\\

\haiku{bededag biddag,;}{dag van georganiseerd}{algemeen gebed}\\

\haiku{of hij het niet te.}{vast heeft of hij niet geheel}{bij zijn verstand is}\\

\section{J. Woltjer}

\subsection{Uit: Verzamelde redevoeringen en verhandelingen}

\haiku{maar een zelfstandig.}{bestaan als wetenschap had}{de philologie niet}\\

\haiku{een \ensuremath{\lambda}o\ensuremath{\gamma}o\ensuremath{\pi}o\ensuremath{\i}\'{o}\ensuremath{\varsigma}, een maker,.}{van een logos is bij hem}{een geschiedschrijver}\\

\haiku{maar de waarneming;}{en de voorstelling waren}{volkomen zuiver}\\

\haiku{en zooals Adam alle,.}{levende ziel noemen zou}{dat zou haar naam zijn}\\

\haiku{God zullen kennen,.}{gelijk zij zelven nu door}{Hem gekend zijn98}\\

\haiku{God heeft uit \'e\'enen;}{bloede het gansche geslacht}{der menschen gemaakt}\\

\haiku{want dit is wel het,.}{eerste dat hun de woorden}{Gods zijn toebetrouwd}\\

\haiku{in de eerste, in,:}{de tweede in de derde}{plaats en altoos weer}\\

\haiku{'t Zij mij vergund.}{een paar punten slechts voor uwe}{aandacht te brengen}\\

\haiku{'t Kan toch het doel?}{niet zijn een tekst zoogenoemd}{leesbaar te maken}\\

\haiku{'de liefde handelt,;}{niet lichtvaardiglijk zij is}{niet opgeblazen}\\

\haiku{dit deel van hunnen;}{arbeid komt hun voor een vast}{resultaat te zijn}\\

\haiku{Zoo nu schrijft geen mensch ({\textquotedblleft}{\textquotedblright}).}{van gezonde zinnenquis}{sanus ita scribat}\\

\haiku{Daarin vinden de;}{professoren P. en N.}{tegenstrijdigheid}\\

\haiku{Ons denken kennen;}{wij voor zoover het zich aan ons}{bewustzijn openbaart}\\

\haiku{Het levende woord,,,,.}{dat is zooals ik zeide de}{zin de gedachte}\\

\haiku{bidden, gebed, of,,,.}{om de tegenstelling het}{bidden het gebed}\\

\haiku{Dit moeilijke punt.}{is voor de uitlegging van}{het grootste gewicht}\\

\haiku{Zijn werk werd door R..}{Laqueur onlangs uitvoerig}{gerecenseerd255}\\

\haiku{Maar toch is de kunst,.}{dienstbaar zij is hulpmiddel}{voor de dialectiek}\\

\haiku{die van het denken,.}{en die van het dichten van}{wetenschap en kunst}\\

\haiku{het is mogelijk;}{dat beide in \'e\'enen mensch}{gevonden worden}\\

\haiku{de blik des geestes.}{wordt klaarder en scherper dan}{hij te voren was281}\\

\haiku{Oefening kan en,:}{moet die gave ontplooien}{leiden en sterken}\\

\haiku{zij moet zijn in de.}{uitwerking wat zij krachtens}{het beginsel is}\\

\haiku{van eenen invloed van;}{andere volken is ons}{zeer weinig bekend}\\

\haiku{Ontzaglijk breed is.}{in deze tijden de stroom}{der literatuur}\\

\haiku{De geesten zijn als,;}{eene voortgedrevene zee}{die niet kan rusten}\\

\haiku{En zoo breidt zich juist}{voor het denken de kring der}{dingen hoe langer}\\

\haiku{als zoodanig, als,.}{idee\"en zijn zij en hebben}{zij realiteit}\\

\haiku{De problemen die}{de theorie der kennis}{ons voorlegt vinden}\\

\haiku{het onzes inziens.}{abnormale kunnen wij}{er uit elimineeren}\\

\haiku{Vergunt mij nu nog.}{eene tweede opmerking hier}{aan toe te voegen}\\

\haiku{Ik ga dus bij het:}{zoeken van het antwoord op}{de gestelde vraag}\\

\haiku{Het tweede deel, de,;}{binnensfeer is zoo te zeggen}{ontoegankelijk}\\

\haiku{Het laatste is door,,.}{zijnen grondslag de H. Schrift}{voldoende bepaald}\\

\haiku{Buitendien past de;}{onderscheiding slecht voor de}{interpolaties}\\

\haiku{V. p. 463 Kirchm..,:}{322Spencer drukt zich onjuist}{uit wanneer hij zegt}\\

\haiku{'Steht denn nicht unser?}{erkennendes Bewusstsein}{mitten in der Welt}\\

\renewcommand{\thechapter}{z}
\chapter[4 auteurs, 916 haiku's]{vier auteurs, negenhonderdzestien haiku's}

\section{Marie C. van Zeggelen}

\subsection{Uit: Bij het hart van Indi\"e}

\haiku{Als dit waar is, zal.}{Prins Rono Soeswito pier}{nimmer terug keeren}\\

\haiku{Even achter den vorst,.}{stonden de Radena joe}{en hare dochters}\\

\haiku{Het gelaat van den.}{Pangeran Adipati sprak}{van een groote zachtheid}\\

\haiku{De ouden wisten,.}{het wel dat had hun vereering}{gedaan voor den boom}\\

\haiku{ja in naam was het,?}{alles nog het Zijne maar}{in werkelijkheid}\\

\haiku{Alles was h\`em, want,!}{eenmaal had Ali geld van hem}{geleend eenmaal maar}\\

\haiku{Toean Allah had.}{hem geluk gegeven maar}{veel ongeluk ook}\\

\haiku{Amsin had bij een.}{ruzie in Wirio's huis}{zijn mes getrokken}\\

\haiku{Den grooten dag dien!}{Allah zegenen mocht voor}{den kleinen Simin}\\

\haiku{Niemand denkt er aan {\textquoteleft}{\textquoteright}.}{deprintah van den grooten}{heer op te volgen}\\

\haiku{In de duisternis.}{verhief zich hoog en groot de}{breede stam voor hem}\\

\haiku{{\textquoteleft}Gij zijt laat, later,.}{dan gewoonlijk de bedoek}{heeft al geslagen}\\

\haiku{{\textquoteright} Andoe richtte het,:}{hoofd op en haar oude stem}{die beefde zeide}\\

\haiku{Zij had altijd met}{verachting aan hem gedacht}{en nu deed zij weer}\\

\haiku{{\textquoteleft}Het kan niet anders,, '{\textquoteright}.}{Heert is beter dat ge}{komt als de zon zinkt}\\

\haiku{{\textquoteleft}Groote heer{\textquoteright} en zij sloeg,.....}{een rooden doek rood als een vlam}{over hoofd en schouders}\\

\haiku{De vorst weet zeker,.}{wel dat het gouvernement}{op het antwoord wacht}\\

\haiku{Heer, ze is op den,....}{weg hierheen zij zal den weg}{langs het meer nemen}\\

\haiku{{\textquoteleft}Ik groet u allen,.}{die gekomen zijt omdat}{ik geroepen heb}\\

\haiku{Zij waren aan boord,.}{van de groote Padoeakan maar}{er kwam ongeluk}\\

\haiku{Een hunner leidde.}{het dier bij den teugel om}{het te doen grazen}\\

\haiku{het straks alles te;}{vertellen aan haar man en}{zoon en vriendinnen}\\

\haiku{Tusschen haar in de.}{fijne ranke figuur van}{I Madinra}\\

\haiku{Niemand onzer kon.}{naar de heilige bron gaan}{om raad te vragen}\\

\haiku{Zij zag neer in het,:}{dal en de vreemdeling haar}{blik volgend sprak zacht}\\

\haiku{{\textquoteright} I Madinra.}{legde haar hand op het naar}{haar geheven boek}\\

\haiku{Nu rees ook Soe Ere.}{van den grond en hij beklom}{de trap der Baroega}\\

\haiku{Als roze zijde;}{glansde het tandvleesch der}{heilige dieren}\\

\haiku{Toen daalde Ali Ri,.}{Ajat Sjah de trap af gevolgd}{door zijn rijksgroeten}\\

\haiku{midden in de tot.}{een bergje gestapelde}{korrels stak een brief}\\

\haiku{Hollander - doch ook -.}{gebruikt voor menschen wit van}{vel dus uit blank ras}\\

\subsection{Uit: Onderworpenen}

\haiku{Enfin, bewaar die.}{djimats en neem den kerel}{mee naar het bivak}\\

\haiku{Maar de hand van den.}{blanke borg ze weer op in}{het grauwe papier}\\

\haiku{Iederen avond sloeg {\textquoteleft}{\textquoteright}.}{hij het uur des gebeds op}{den houtenbedoek}\\

\haiku{- {\textquoteleft}Waarom ben ik ook,?}{niet doorgeloopen waarom}{liet ik hem alleen}\\

\haiku{Daarom offerde;}{nu de oudste der koelies}{ook voor hen allen}\\

\haiku{Neen, neen, loopen maar,,,;}{ze was moe ja misschien viel}{ze wel dood neer straks}\\

\haiku{Ik merk wel dat je.}{toch eigenlijk nog te klein}{bent om te werken}\\

\haiku{en ik liep maar door,,.}{langs het bamboeboschje rechts}{af op de poort links}\\

\haiku{Er was iets in de. '}{sfeer van dezen man dat de}{beesten kalm maakte}\\

\haiku{De kettinggangers.}{spreken nooit van hun straftijd}{maar van hun diensttijd}\\

\haiku{'t Was of het op {\textquoteleft}!}{zijn gezicht stondWat komt er}{opaan wat ik aan heb}\\

\haiku{En wij verlieten...........................}{het huis en den tuin en het}{prachtige bergland}\\

\haiku{Wirio kon best het;}{toezicht op de beesten aan}{haar toevertrouwen}\\

\haiku{\'e\'ens had hij zijn,.}{jas al verkocht maar dat was}{nog niet het ergste}\\

\haiku{ze was te moe om,;}{nog harder te huilen ze}{snikte maar zacht door}\\

\section{Elisabeth Zernike}

\subsection{Uit: Bevrijding uit de jeugd}

\haiku{We doen alles zo;}{zuinig mogelijk om te}{sparen voor het huis}\\

\haiku{Titia sloeg de ogen.}{neer voor de uitdrukking van}{haar moeders gezicht}\\

\haiku{Bel nu maar voor de - -?}{koffie en ik heb taartjes}{had je dat gedacht}\\

\haiku{- een ongetrouwde,}{man is dat al genoeg om}{je de oren te doen}\\

\haiku{- Ik zou naar Parijs,, - -?}{willen ik heb mijn pas al}{aangevraagd maar Maar}\\

\haiku{U moet weten dat;}{mijn vader wijnhandelaar}{is in Chartres}\\

\haiku{Die vraag, zo dikwijls,.}{in haar opgekomen had}{ze nooit beantwoord}\\

\haiku{Kort voor de oorlog.}{was er die kleine Fransman}{geweest in Montreux}\\

\haiku{Nu belde ze en, -.}{keek de straat af vlak en recht}{het saaiste van Holland}\\

\haiku{O, ik mag zulke,.}{dingen niet zeggen jij bent}{nog zo maagdelijk}\\

\haiku{- Je weet niet hoe ik.}{je bewonder om wat je}{zoudt kunnen worden}\\

\haiku{Later heb ik daar -.}{hinder van gehad maar toen}{kon ik niet anders}\\

\haiku{- Madame, zei hij, -?}{tegen Titia hebt u een}{goede reis gehad}\\

\haiku{Ze wilde stellig,.}{het huis zien ze hoopte te}{kunnen meewerken}\\

\haiku{Hier en daar zag ze,.}{hoge zonnebloemen en}{een enkele boom}\\

\haiku{- Och, dat wist ze niet,,.}{maar zolang je leefde stond}{je bloot aan de tijd}\\

\haiku{- Ja, zei Titia, - en.}{je zoudt er je leven mee}{kunnen verknoeien}\\

\haiku{Die woorden kwamen.}{haar onvoorbedacht en ze}{zag Arthur's ontroering}\\

\haiku{Toen sloeg Arthur een arm.}{om haar schouders en kuste}{voorzichtig haar wang}\\

\haiku{Misschien - niemand had,.... -,.}{het haar gezegd maar Je blijft}{natuurlijk zei Jo}\\

\haiku{De begrafenis.}{kan niet worden uitgesteld}{in deze hitte}\\

\haiku{- Ik betuig u mijn,.}{leedwezen had madame}{Ch\'em\`ene gezegd}\\

\haiku{Vijf kinderen op, -.}{een dorp en een doktershuis}{het kon niet mooier}\\

\haiku{Wat zij zou doen was, -,.}{stof afnemen een man zag}{dat niet maar zij wel}\\

\haiku{Ze praatten daar even.}{op door en ze vertelde}{iets over haar ouders}\\

\haiku{Ze liepen het huis.}{weer in en mevrouw Honnes}{wenkte de meisjes}\\

\haiku{- Maar juffer, gekapt -.}{en gekleed en ik dorst u}{geen ontbijt brengen}\\

\haiku{Zonder groet draaide.}{ze zich om en liep langzaam}{naar de stad terug}\\

\haiku{- Neen, zei hij, ik heb, -.}{mij laten kiezen en gij}{steunt mijn zwak verstand}\\

\haiku{Keetje voelde een.}{vast verzet in zich stijgen}{en antwoordde niet}\\

\haiku{Op de terugweg,,.}{vroeg in de middag toonde}{van Blom zich spraakzaam}\\

\haiku{- Hm, zei van Blom - een -.}{blanchisseuse en noemde}{andere namen}\\

\haiku{Hij hief de hand op,,.}{met het glas een dronk gewijd}{aan de vrouw dacht hij}\\

\haiku{De oude vrouw keek:}{naar de vonkenregen en}{antwoordde langzaam}\\

\haiku{Op zijn dertigste;}{jaar is mijn vader tot hoofd}{van een school benoemd}\\

\haiku{Mijn vader had de:}{schouders opgehaald over mijn}{eerste rapporten}\\

\haiku{- Zou je liever je?}{huiswerk beneden maken}{met de anderen}\\

\haiku{Moeder had gelijk,.}{ik wist plotseling dat ik}{vaak zat te dromen}\\

\haiku{- Mijn vader heeft een,,;}{Fantin-Latour gekocht stel}{je voor een echte}\\

\haiku{- en geloof dat ik,.}{er evenzeer voor vreesde als}{ernaar verlangde}\\

\haiku{- Ik denk niet dat ik:}{zal studeren en hij trok}{de wenkbrauwen op}\\

\haiku{- En dan zonder de,?}{rozenperkjes alsof het}{het land van Dothan was}\\

\haiku{- maar ik kende toen:}{Jenny's tedere blik al}{goed en bovendien}\\

\haiku{het publiek had niets,.}{gemerkt hij rekende op}{mijn geheimhouding}\\

\haiku{- U moet weggaan, zei,.}{ik alleen haar ouders zal}{ik niet weigeren}\\

\haiku{- Kom, hij had er meer, -.}{zien vallen ze stonden wel}{weer op maar hij ging}\\

\subsection{Uit: Bruidstijd}

\haiku{Ze antwoordde niet,.}{onmiddellijk maar begon}{langzaam te blozen}\\

\haiku{Iets van die vreemde.}{drift was ook in haar en ze}{haalde schokkend adem}\\

\haiku{- Neen, dat heb ik tot,.}{nog toe gezegd maar nu ga}{ik veranderen}\\

\haiku{Dicht naast haar lag Nel,.}{die zeker gewend was aan}{zulke geluiden}\\

\haiku{Ze keek langs zichzelf,.}{omlaag de kimono hing}{tot op haar voeten}\\

\haiku{En dit zijn onze,,.}{jongens de blonde is Bas}{en de ander Henk}\\

\haiku{De jongens lachten,.}{luidkeels zodat het verhaal}{werd onderbroken}\\

\haiku{Ze bloosde bijna,.}{door de gedachte aan die}{flater en stond op}\\

\haiku{- Dag kind, had mevrouw, -.}{Moro gezegd we hopen}{je eens vaak te zien}\\

\haiku{De salon - rustig,.}{in het late licht alle}{vensters gesloten}\\

\haiku{Het was hem of hij.}{groeide en onbewogen}{neerkeek op zijn broers}\\

\haiku{- Als u haar weer ziet,, -:}{noem haar dan Nel dat is ik}{zou willen zeggen}\\

\haiku{- Het is mooi voor u,,...?}{dat begrijp ik wel het is}{had u het verwacht}\\

\haiku{En werd verontrust,.}{door dat laatste woord dat in}{hem bleef naklinken}\\

\haiku{- Zo, zei Bertha, die een,?}{pan schudde zal ik jou ook}{eens wat vertellen}\\

\haiku{Toen hij weg was, had.}{Ina een glans in haar ogen van}{verwachting en trouw}\\

\haiku{Was het niet goed wat, -?}{ze wilde omdat Doortje}{grauwe handen had}\\

\haiku{Eline tuurde in;}{het zachte schemerlicht en}{hoorde de stilte}\\

\haiku{De staande houding.}{hinderde hem en hij kon}{niet naar Ina kijken}\\

\haiku{Werd het een koop, h\'a\'ar?}{durf en wereldwijsheid voor}{zijn geld en titel}\\

\haiku{Hij zag het verband,:}{om haar vinger dacht aan de}{woorden van zijn toast}\\

\haiku{- Och, zei ze en bleef,.}{achter zijn stoel staan het is}{niet zo belangrijk}\\

\haiku{Een voorbijganger.}{keek hem aan en hij hief het}{hoofd nog wat meer op}\\

\haiku{Ze wees hem op een,.}{mand met herfstasters lila}{met een hart van geel}\\

\haiku{Ze moest weten wie,,.}{ik ben dacht hij toen ze de}{deur voor hem open hield}\\

\haiku{Mijn behanger moet.}{een voorbeeld worden van het}{gedeeltelijke}\\

\haiku{Zijn belangstelling,?}{prikkelde haar waarom sprak}{hij niet over zichzelf}\\

\haiku{Zelfs in den tijd van.}{Leo Frankenvoort had het kind}{niet zo goed gespeeld}\\

\haiku{Nog hoorde ze haar,.}{moeders stembuiging maar dat}{was nu de hare}\\

\haiku{- Maar je bent zelf nog,,.}{niet oud zei Eline en je}{moest kleuren dragen}\\

\haiku{Terwijl ze de trap,.}{afliepen voelde Ina zich}{onrustig worden}\\

\haiku{ik eiste van haar,.}{dat ze me zou volgen in}{mijn nieuwen werkkring}\\

\haiku{- Ik weet nu, dat dat -,.}{verkeerd van me was ik mag}{vragen niet eisen}\\

\haiku{Jou wil ik vragen,...}{of het je niet hinderlijk}{is wanneer ik tracht}\\

\haiku{- Neen, zei Bertha, je bent,.}{groter dan ik dat is me}{ongemakkelijk}\\

\haiku{Je zult zien, het zijn,.}{goede dwergjes ze lachen}{als ik binnenkom}\\

\haiku{Dan heb ik nog zo'n,'.}{ouwe tante die loopt met}{krante langs de straat}\\

\haiku{- Je hoeft niet mee, hoor,,.}{overal is een hoek waarop}{je kunt omdraaien}\\

\haiku{- Dan heb je nog wel,, -.}{een ogenblik zei Ina maar haar}{trekken bleven star}\\

\haiku{- Gisteravond heb je,:}{naar me gekeken alsof}{je zeggen wilde}\\

\haiku{Rika kondigde.}{het bezoek aan en sloot stil}{de deur achter zich}\\

\haiku{En nog meer had hij,.}{gezegd natuurlijk ook iets}{over geheimhouding}\\

\haiku{De eetkamer van.}{de familie Moro lag}{in het souterrain}\\

\haiku{Het is de vraag wat,,:}{wij zijn had Bas gezegd en}{Nel had verbeterd}\\

\haiku{- Ik kan niet anders,,}{klonk het een beetje schor het}{eigen innerlijk}\\

\haiku{De mens is de kroon -.}{der schepping ze stootte een}{schamper lachje uit}\\

\haiku{- en het is een maand.}{geleden dat ik het me}{heb voorgenomen}\\

\haiku{- Nou, als jelie me, -.}{nodig hebt roep je maar het}{zal tevergeefs zijn}\\

\haiku{Wij stonden stijf door.}{onze dikke kleren en}{hadden het toch koud}\\

\haiku{- Ik vind u lief, zei -,.}{Nel ik geloof niet dat veel}{moeders zo praten}\\

\haiku{- Maar de brief dien ze,.}{hem erover had geschreven}{was niet beantwoord}\\

\haiku{- Ben ik dan niet... - Sst,,.}{zei hij zulke grove taal}{mag ik niet horen}\\

\haiku{Ze keek hem aan en,.}{glimlachte om den geest die}{uit zijn trekken sprak}\\

\haiku{- We zullen goed doen.}{niet te kopen voor we het}{eens zijn geworden}\\

\haiku{- Het zou wel goed zijn,,.}{meende hij nu toch maar naar}{zijn zuster te gaan}\\

\haiku{Heleen veegde haar,.}{hand af langs haar heup voor ze}{die naar hem uitstak}\\

\haiku{Ze nam het laatste,.}{stuk brood dat voor haar lag en}{at het langzaam op}\\

\haiku{Papa was rijk - maar,.}{hij ging weg en zij zou bij}{de Moro's wonen}\\

\haiku{- Hier moeten  we, -,.}{oversteken ik geloof dat}{ik u moet leiden}\\

\haiku{Nu moest er iemand.}{komen die licht maakte en}{dezen dag aandorst}\\

\haiku{Als dit kon - als   -.}{hij haar hielp dan zou ze nog}{willen leven}\\

\haiku{Maar vijf minuten.}{later stond hij op en liep}{naar de badkamer}\\

\haiku{Het brede en toch,,.}{tere voorhoofd beheerste met}{de ogen het gezicht}\\

\haiku{ik wist wel, had zijn,.}{moeder gezegd dat hij eens}{een man zou worden}\\

\haiku{- Ja, zei Doortje - ik,.}{ben maar zo bang dat u zich}{eenzaam zult voelen}\\

\subsection{Uit: De erfenis}

\haiku{Maar je erft van hem,, -.}{je komt er bovenop ik}{ben er heel blij om}\\

\haiku{Hij was een rijke,.}{industrieel maar had haar}{nooit iets gegeven}\\

\haiku{Hoe armer ze werd,.}{hoe onrustiger en meer}{geneigd tot zwerven}\\

\haiku{ze liep langs het strand,,.}{de zeewind was zilt en guur}{de golven raasden}\\

\haiku{Ik ben maar bij je.}{binnen gekomen en neem}{je tijd in beslag}\\

\haiku{Ze is een beetje.}{ouder dan ik en was al}{eens verloofd geweest}\\

\haiku{Wat denk je dat we?}{van Ing bij ons huwelijk}{hebben gekregen}\\

\haiku{- Dat is wel goed, ze.}{mogen niet langer duren}{dan de verloving}\\

\haiku{Als kind heb ik een;}{dierenatlas gehad met}{originele text}\\

\haiku{George lachte.}{even en keek Bart van Weeze}{met schuinsen blik aan}\\

\haiku{Het dienstmeisje kwam;}{binnen met ingeschonken}{thee op een groot blad}\\

\haiku{Ze merkte dat zijn.}{aandacht alweer elders was}{en  trad terug}\\

\haiku{Toen Dora in den,.}{trein naar huis zat voelde ze}{haar grote moeheid}\\

\haiku{- Kinderlijk, hernam,.}{Truus door je vertrouwen en}{je halsstarrigheid}\\

\haiku{- Je weet, zei Truus, dat?}{ik mijn neef Huib den laatsten}{tijd weer heb ontmoet}\\

\haiku{Hij is verlegen,,.}{dacht ze omdat ik hem in}{zijn gezin betrap}\\

\haiku{- Djja, zei Nubeling,,;}{langgerekt lust tot knokken}{om het verzetje}\\

\haiku{- En het verdere?}{van de omstandigheden}{laten afhangen}\\

\haiku{- Zevenhonderdtwintig,,.}{gulden per jaar zei Viers om}{niet te kankeren}\\

\haiku{Maar dreggen is een -.}{beroerd werk je wordt er nat}{van tot in je ziel}\\

\haiku{De jongens praten, -;}{over den hengst en de merrie}{dat klinkt mannelijk}\\

\haiku{- Maar als ik daarbij,,.}{aanknoop zei George zal}{het zo lang worden}\\

\haiku{Zijn huis beviel me,.}{natuurlijk niet dus ging ik}{zelf iets ontwerpen}\\

\haiku{Het is gelukt, en:}{tegelijkertijd met een}{knauw van ellende}\\

\haiku{Ze stonden allen.}{op en weer gaf het kristal}{zijn helderen toon}\\

\subsection{Uit: De gast}

\haiku{Dan werkte hij weer.}{tot het avondeten en moeder}{boog over het fornuis}\\

\haiku{{\textquoteleft}Om mij bij te staan{\textquoteright} -.}{die woorden deden Anna}{het hoofd opheffen}\\

\haiku{Werner is hun trots,.}{maar ze laten hem gaan en}{roepen mij terug}\\

\haiku{heeft uw dorp geen plaats,?}{voor de kinderen die er}{worden geboren}\\

\haiku{Kort daarop had ze,.}{partij gekozen v\'o\'or haar}{ouders ze moest wel}\\

\haiku{In de lente lag,;}{de sneeuw voor het grijpen dan}{was er geen kunst aan}\\

\haiku{Ze liep de trap op,,.}{wilde roepen maar er kwam}{geen klank uit haar keel}\\

\haiku{Ze waren nog jong.}{en moesten hun best doen vader}{niet te vergeten}\\

\haiku{Ook van dit gesprek.}{wist ze later niet meer hoe}{het was verloopen}\\

\haiku{Langzaam liepen ze,.}{omhoog de zeere plek op}{haar schouder schrijnde}\\

\haiku{De avondlucht was koel,.}{aan haar voorhoofd de druk op}{haar slapen nam af}\\

\haiku{Er stond een stoel voor,.}{hem klaar hij trok zorgvuldig}{zijn broekspijpen op}\\

\haiku{Mevrouw schonk thee en.}{begon een gesprek over de}{schoonheid van het land}\\

\haiku{- Dat weet ik nog niet,,.}{we vergaderen er over}{we zijn vrije Zwitsers}\\

\haiku{Van de Pressalle's,.}{had ze geleerd dat het op}{de lijnen aankwam}\\

\haiku{Soms viel er wat sneeuw,;}{van een tak geruchtloos of}{met een zachten plof}\\

\haiku{Het bloed trok weg uit, -.}{haar hoofd er kwam een gevoel}{van leegte in haar}\\

\haiku{- Maar hij was naast haar.}{gekomen en ze zag den}{ernst in zijn gezicht}\\

\haiku{Het meisje voelde.}{een lang ontbeerde blijdschap}{en stak haar hand uit}\\

\haiku{Dat {\textquoteleft}Montana{\textquoteright}?}{met zijn moderne comfort}{moet ik toch eens zien}\\

\haiku{De deuren van het,.}{huis stonden open het was een}{warme voorjaarsdag}\\

\haiku{Je bent naar het huis, -,?}{wezen kijken het is veel}{verbeterd nietwaar}\\

\haiku{- en het verwacht zijn -,.}{gasten of jij er ooit zult}{wonen weet ik niet}\\

\haiku{Ze keek den weg af, -.}{naar omhoog en omlaag ze}{zag niemand buiten}\\

\haiku{Vroeger heb ik met - -.}{je mogen spelen nou dan}{en nu is het ernst}\\

\haiku{Dit keer hield Anna.}{voet bij stuk en er kwam iets}{dwingends in haar oogen}\\

\haiku{Anna voelde zich,.}{blozen ze wist hoe druk het}{in de keuken was}\\

\haiku{- Neen, hij bestudeert,.}{het dansen daar moet hij ook}{eens examen in doen}\\

\haiku{De geit mekte en,.}{liep haar tegemoet toen ze}{den stal binnenging}\\

\haiku{Na een poos dacht ze,.}{aan de kinderen die haar}{niet hadden gestoord}\\

\haiku{Anna deed haar best.}{zich niet door hun woorden te}{laten afleiden}\\

\haiku{het huis, met koelen -.}{ingang de tranen sprongen}{Anna in de oogen}\\

\haiku{Heinz was op de stoep,:}{gaan zitten Treesje stond op}{den drempel en riep}\\

\haiku{En dan wordt hij nog,, -.}{eens professor oma ik hoop}{dat je het beleeft}\\

\haiku{Nu liep Willy hem,.}{te zoeken ze wist niet of}{hij daar goed aan deed}\\

\haiku{Waar praten wij over, -.}{dacht Anna oude menschen}{zijn niet gelukkig}\\

\haiku{Kleur is ook iets, en,.}{gelukkig schijnt de maan niet}{altijd vrouw Anna}\\

\haiku{Zij knielt in mijn plaats - -,?}{of ik ben het zelf lijkt ze}{niet op mij Anna}\\

\haiku{- Dat is niet waar, zei,.}{ze maar wachtte gespannen}{op zijn wederwoord}\\

\haiku{En nog veel meer zei,.}{ze en dacht met flitsen aan}{het verstoorde werk}\\

\haiku{Vrouw Bassmann was;}{naast het fornuis gaan zitten}{en keek om zich heen}\\

\haiku{ze waren al met,.}{zijn tienen er kon geen stoel}{meer bij de tafel}\\

\haiku{- Dus, zei Willy, jij;}{loopt vanmiddag naar het dorp}{en noodigt moeder uit}\\

\haiku{Door de sneeuw was de, -.}{aarde niet donker alleen}{het licht was schaarsch}\\

\haiku{met een enkelen.}{oogopslag had ze gezien}{dat er geen ontbrak}\\

\haiku{- En de engel had,.}{zoon mooie stem zei het meisje}{dat naast Werner liep}\\

\haiku{Maria schudde het,,.}{hoofd haar neusvleugels spanden}{zich maar ze sprak niet}\\

\haiku{- Morgen ga ik naar, -.}{huis terug ging ze voort ik}{ken dit leven niet}\\

\haiku{Ge\"ente rozen, - -,.}{ja zei ze deze niet die}{op eigen stam staan}\\

\haiku{Twee dagen na den.}{droom zag ze hem met Willy}{op de bank zitten}\\

\haiku{Als ze nu maar kon.}{bedaren en denken aan}{dat wat ze wilde}\\

\haiku{dat haar verlangen,.}{niets kon winnen want dat hij}{haar zou verjagen}\\

\haiku{- En heen en terug,.}{zonder oponthoud zei de}{kok met strenge oogen}\\

\haiku{Hij legde zijn hand.}{om haar pols en telde de}{slagen van het hart}\\

\haiku{Hij nam Roosje op;}{zijn knie en las haar voor uit}{een prentenboekje}\\

\haiku{Wat later kon ze;}{duidelijk de dorpen in}{het dal zien liggen}\\

\haiku{Met onvaste stem.}{vroeg ze naar moeder Tresa}{en de kinderen}\\

\haiku{Tweemaal hield ze een -!}{praatje kon ze daarmee den}{tijd maar doen stilstaan}\\

\haiku{Nu kwam er een glans.}{in haar oogen en een guitig}{trekje om haar mond}\\

\haiku{- Vraag mij liever wat -.}{en hoorde een nadenkend}{zoemen in zijn keel}\\

\haiku{Zie je, het heeft me.}{gegrepen en ik heb het}{van me afgeschud}\\

\haiku{Ze draaide zich naar,.}{hem toe maar verborg haar hoofd}{en snikte weer}\\

\haiku{- hoe kwam ze er ook,?}{toe het graf van oma Tresa}{te willen sieren}\\

\haiku{ze was geweest toen}{ze vader verloor en wat}{zich in haar prentte}\\

\haiku{een forsche, bijna,.}{onaandoenlijke vrouw die}{het leven kende}\\

\haiku{Eigenlijk dacht ze, -.}{daar zelden meer aan het was}{zes jaar geleden}\\

\haiku{Een minuut of wat,.}{hoorde Anna hem aan toen}{stak ze haar hand uit}\\

\haiku{Elsi, het oudste,.}{dochtertje van Martha stond}{achter de toonbank}\\

\haiku{Dan riep de vraag naar.}{een volgend artikel haar}{aandacht weer terug}\\

\haiku{Gaan jelie toch naar,.}{bed ik wil eindelijk met}{vader alleen zijn}\\

\haiku{Het eerste gezin,,.}{dat ze had bezocht woonde}{aan een kweekerij}\\

\haiku{ieder bleef in de.}{duister-lichte tent van}{zijn ervaringen}\\

\haiku{Als hij omhoog liep?}{en die zemelen pop de}{waarheid vertelde}\\

\haiku{Hij had haar in lang,,}{niet gezien ze kwam niet meer}{naar omlaag maar \~als}\\

\haiku{Vreemd - zes slokken uit -.}{een glas van zeven en het}{glas was aan zijn mond}\\

\haiku{- Adieu, de Moeder -,.}{Gods bescherme u. Adieu}{prevelde Joseph}\\

\haiku{Vader Joseph sloeg;}{zijn oogen niet op terwijl hij}{kopjes waschte}\\

\subsection{Uit: De gereede glimlach}

\haiku{Het huisje waarbij,,:}{de gouden regen bloeide}{zoo laat nog heette}\\

\haiku{En de stroomende.}{regen zou heel kil zijn aan}{hun warme lijfjes}\\

\haiku{- Zie zoo jongens, nu,.}{gaan we dwars door de hei en}{wie het eerst er is}\\

\haiku{Altijd nog - zooals ze,,.}{hier lag een kind ontvangen}{met heel haar lichaam}\\

\haiku{Na een week zat hij.}{met de kinderen op het}{verandatrapje}\\

\haiku{Dan sloeg ze soms even,:}{haar handen voor haar gezicht}{verward fluisterend}\\

\haiku{Zijn onbewuste.}{eenvoud had het nieuwe in}{haar leven gebracht}\\

\haiku{Agnes wilde het,.}{onbelangrijke weten}{het bijkomstige}\\

\haiku{Bovendien heb ik.}{je laatste verloofde nog}{juist even meegemaakt}\\

\haiku{Ze zag hem langzaam,.}{zoeken rondom het huis dat}{ze gesloten had}\\

\haiku{Gisteren, toen ze,;}{naast zijn paard liep was ze toch}{jong en sterk geweest}\\

\haiku{Hoeveel dieper dan,.}{gisteren was alles nu}{hoeveel droeviger}\\

\haiku{- Ja - zei ze - in een,.}{fluister en ze boog haar hoofd}{voor het groote leven}\\

\haiku{Hij nam haar hand en,,.}{ze liepen weg snel en zacht}{met wijde stappen}\\

\haiku{Hij vertelde hoe,.}{zijn leven was geweest wat}{hij geleden had}\\

\haiku{Alleen in den nacht.}{heb ik soms smadelijk haar}{lichaam genomen}\\

\haiku{Korten tijd stond de,.}{maan dicht aan den horizont}{rood en gezwollen}\\

\haiku{De deur werd wijd voor:}{haar geopend en de man}{voegde er nog bij}\\

\haiku{III Op een avond - ze, -.}{waren toen een week bij den}{boer bleef Petja weg}\\

\haiku{En in mijn oogen ben,}{je het ook want ik heb zelf}{gezien wat je deed}\\

\haiku{- Een beetje loom, ik,.}{weet niet waarvan misschien door}{deze rust alleen}\\

\haiku{- Ik houd zooveel van,,.}{hem zei ze doorgaand op haar}{eigen gedachten}\\

\haiku{En toen zag ze weer,.}{Hanna voor zich de vrouw die}{haar vader liefhad}\\

\haiku{In ieder geval,.}{zie ik uw onrust en ik}{heb er verdriet van}\\

\haiku{Maar ik denk aan een,,.}{heel groote groene vaas waarin}{goudvisschen zwemmen}\\

\haiku{Waarom hij nu nog,?}{alle onrust verkropte}{terwijl zij toch wist}\\

\haiku{Op een middag kwam.}{haar vader de tuinkamer}{binnen waar zij zat}\\

\haiku{later misschien, dan,.}{komt het ongemerkt maar haar}{houding verstilde}\\

\haiku{Het plafondlicht in.}{haar kamer viel zonder schroom}{op alle dingen}\\

\haiku{Denk jij ook maar aan,,.}{je eigen leven meisje}{ich rate dir gut}\\

\haiku{En ze had zich nooit,.}{afgevraagd hoe er over haar}{gepraat zou worden}\\

\haiku{Wat gebeurde er,?}{dan in dit huis moest Corrie}{er vrij spel hebben}\\

\haiku{Hij liet haar polsen,.}{los en haar armen vielen}{slap langs haar lichaam}\\

\haiku{Hij verwonderde.}{zich over het snelle flitsen}{van haar gedachten}\\

\haiku{hoe hij dat haar van,.}{haar voorhoofd streek terwijl zijn}{lippen bewogen}\\

\haiku{Nu voelde ze zich,.}{bijna als een bruid die haar}{kamer binnen gaat}\\

\haiku{Ze had Wim beloofd,.}{eens naar zijn kinderen te}{komen kijken}\\

\haiku{Je bent wat jonger,,.}{dan ik maar je hebt geleefd}{en je sleept me mee}\\

\haiku{Al haar aandacht had,.}{ze noodig voor haar onbewust}{zeker handelen}\\

\haiku{Toen ze haar oogen sloot,.}{zag ze de wereld buiten}{zich heel ver en klein}\\

\haiku{Ik hoorde u naar,.}{boven gaan en heb thee voor}{u ingeschonken}\\

\haiku{Ze vermeden het,,.}{elkaar aan te zien gingen}{alle twee zitten}\\

\haiku{De zon scheen over het,.}{oude groen van de boomen}{warm en koesterend}\\

\haiku{Ze zweefde al in, -.}{de ruimte een haan kraaide}{den hemel open}\\

\haiku{Ze had haar mantel,.}{aan de kapstok gehangen}{en liep naar boven}\\

\haiku{- want zou ze later,?}{w\'e\'er haar huis kunnen uitgaan}{en hier terugkeeren}\\

\haiku{als ze gedurfd had,.}{zou ze den vader haar rug}{hebben toegekeerd}\\

\haiku{ze zag hun smalle,,;}{vermoeide gezichten schuw}{blikkend naar elkaar}\\

\haiku{Hun geest was bezig.}{met wat achter de bochten}{zou zichtbaar worden}\\

\haiku{Toen glimlachte ze,.}{en streek met haar hand over de}{ruige reisdeken}\\

\haiku{Misschien zou dat ook, -.}{gauw voorbij zijn hij wist niet}{of hij het hoopte}\\

\haiku{- om tot die hoogten,?}{te stijgen waar Elien hem niet}{had kunnen volgen}\\

\haiku{- Kleine Annie kwam.}{binnen met een springtouw in}{haar dikke vuistjes}\\

\haiku{- Dan zal ik u nog,.}{een kopje inschenken en}{we gaan vroeg naar bed}\\

\haiku{Mijn oudste jongen,.}{heeft examen gedaan een paar}{dagen geleden}\\

\haiku{Zoo moet je niet doen,, -:}{dat is niet goed terwijl ze}{vroeger gezegd had}\\

\haiku{Ze zou dus naar Aga, -.}{kunnen gaan maar ze had het}{zichzelf verboden}\\

\haiku{- en toen zuchtte ze.}{gelaten en wendde haar}{blik af van de klok}\\

\haiku{maar ze praatte niet,,,.}{veel meest luisterde ze en}{keek vooral naar Aga}\\

\haiku{Hij keek haar dan soms,,.}{ook aan met een heel vagen}{glimlach meende ze}\\

\haiku{waarvoor heeft mijn broer,?}{een candidaat als hij er}{niet tusschen uit kan}\\

\haiku{- Rinus, riep mevrouw,?}{Van Waveren waarom ben}{je zoo bescheiden}\\

\haiku{- Al die poespas, had,,:}{Te Weichel gemompeld en}{toen tegen Weversma}\\

\haiku{Och, nonsens, goed eten,,.}{zooals haar vader gedaan had}{daaraan hield ze vast}\\

\haiku{Ze wendde zich snel,, -?}{naar hem om haar oogen blonken}{in haar lach O ja}\\

\haiku{- Zitten blijven en,,}{mond houden riep Aga maar Van}{Waveren stond al.}\\

\haiku{hij merkte het niet,,,.}{en ze bleef kijken bewust}{nu en nog scherper}\\

\haiku{- Kind, ik vind het zoo,?}{heerlijk om te gaan kan je}{je dat begrijpen}\\

\haiku{Hij kon op Zondag;}{een uur langer slapen dan}{op weeksche dagen}\\

\haiku{- Mijn liefde voor haar,.}{zal zeker invloed op me}{hebben zei ze snel}\\

\haiku{Toen ze na schooltijd,.}{was thuis gekomen klopte}{Ietje bij haar aan}\\

\haiku{Terwijl ze Weversma,.}{een hand gaf voelde ze een}{verwarring in zich}\\

\haiku{- Hij praatte zoo graag, -.}{over onze liefde maar hij}{was achterdochtig}\\

\haiku{- Ik zou je verstoot,,.}{en zei hij als ik dacht dat}{je me niet lief had}\\

\haiku{- Na de inleiding;}{begon het verhaal dat ze}{las haar te boeien}\\

\haiku{- Wilde ze hem het,?}{genoegen doen een kop thee}{met hem te drinken}\\

\haiku{Ik zou ook heel graag, -.}{iets doen we moeten er nog}{maar eens over spreken}\\

\haiku{Dan huren we daar, -.}{een groot huis ik ben dol op}{een verandering}\\

\haiku{Zij had het niet zien,.}{aankomen zij was trouw en}{argeloos geweest}\\

\haiku{Ze voelde tranen,.}{naar haar oogen komen maar drong}{ze met kracht terug}\\

\haiku{bestuurslid van een.}{vereeniging tot steun aan}{gevallen meisjes}\\

\haiku{- Ik ben blij dat je,,.}{terug bent iedereen vraagt}{naar je Anton ook}\\

\haiku{- Dat weet ik niet, - ze -... -,.}{schrijft dat e Jawel dat ze}{nu al van je houdt}\\

\haiku{- En ik vergeleek,.}{haar met een grauw nachtuiltje}{dat om de lamp vloog}\\

\subsection{Uit: Het harde paradijs}

\haiku{In de slaapkamer;}{van de ouders stond ook het}{bed van de dochter}\\

\haiku{Het betekende,.}{verraad en ze had tot de}{keurbende behoord}\\

\haiku{Hij wilde trouwen,.}{er was een meisje daarginds}{dat op hem wachtte}\\

\haiku{{\textquoteright} Marguerite,.}{zag reprodukties die haar}{niet vreemd voorkwamen}\\

\haiku{Het had beter nog,,}{wat kunnen wachten en dat}{denken we gebukt}\\

\haiku{Nauwelijks, zult u,.}{zeggen maar we leven in}{een duistere tijd}\\

\haiku{Daar leek geen einde,.}{aan te kunnen komen en}{toch ging het voorbij}\\

\haiku{U hebt uw oude,}{schoolvriendin nog niet bezocht}{zij vraagt mij naar u}\\

\haiku{{\textquoteleft}Ze zal die hebben -.}{verspeeld alsof het leven}{een loterij was}\\

\haiku{Rite, die zich had,.}{opgericht zag zijn donker}{behaarde armen}\\

\haiku{De gevoelige,,.}{neusgaten van zo'n dier ach}{en de zachte oren}\\

\haiku{Niet begrijpend keek.}{ze ernaar en voelde haar}{gezicht verstrakken}\\

\haiku{Ze heeft een moeilijk.}{leven gehad of althans}{een moeilijke tijd}\\

\haiku{Je hebt wat geld op,.}{de Bank staan het zal ieder}{jaar vermeerderen}\\

\haiku{Toen Pierre zo'n jaar,:}{of veertien was geworden}{had hij eens gevraagd}\\

\haiku{Ze hoorde een stem.}{die de jongens verbood op}{dat hek te spelen}\\

\haiku{Pierrre heeft mij,,:}{toestemming gegeven dacht}{ze en bovendien}\\

\haiku{hoe goed kende ze.}{die geur en het zoevende}{geluid van de kwast}\\

\haiku{{\textquoteright} Driftig hief ze de.}{kwast op en weer spatten er}{druppels op de grond}\\

\haiku{Hoeveel inwoners,{\textquoteright}, {\textquoteleft}?}{heeft ons dorp vroeg zezouden}{het er duizend zijn}\\

\haiku{{\textquoteright} Marie keek haar aan,.}{en knipoogde haar grote}{mond trok nog wijder}\\

\haiku{{\textquoteright} Vaak dacht Rite aan;}{die woorden terug bij het}{werk op haar velden}\\

\haiku{glimlachte met \'e\'en,.}{helft van zijn gezicht waardoor}{het nog schever werd}\\

\haiku{Ze heeft een man en.}{kinderen en is de trots}{van mijn ouders}\\

\haiku{Nog kan ik zien hoe,,.}{ze ermee in haar handen}{stond onthutst verschrikt}\\

\haiku{- Nee, ze wist het niet, -?}{was alles goed als ze dien}{man haar huis verkocht}\\

\haiku{Eenmaal hief ze de:}{armen in wanhoop boven}{het hoofd en zei luid}\\

\haiku{ze wist niet waar ze,?}{was zou ze de brug over de}{Rioux Blanc vinden}\\

\haiku{- Wat betekende,?}{daarnaast haar vaste loop}{heel haar lichaamskracht}\\

\haiku{Een zware, jonge,:}{vrouw met roodachtig geverfd}{haar zei minachtend}\\

\haiku{{\textquoteleft}ze zal morgen haar -}{geit nog wel buiten brengen}{in haar werkplunje}\\

\haiku{Hij had een kort, zwart.}{jasje aan en een zwart met}{grijs gestreepte broek}\\

\haiku{Ronde stenen van,.}{verschillende grootte je}{zult het alles zien}\\

\haiku{het moest nu gauw uit,...}{zijn vandaag de klucht van een}{huwelijk en dan}\\

\haiku{Meteen nam ze die.}{over de arm en liep terug}{naar de slaapkamer}\\

\haiku{- Ze hoefde niet de.}{pluk van de hele morgen}{in \'e\'en zak te doen}\\

\haiku{De hitte sloeg haar.}{eensklaps uit en ze begreep}{te willen braken}\\

\haiku{Het linker glas was,.}{een beetje beduimeld ze}{zou het schoon vegen}\\

\haiku{Ik zal met Pi\'emont, -?}{praten hij moet er iets aan}{doen hoe ver is het}\\

\haiku{En deze vrouw, die,.}{niet eens onknap was had een}{bultenaar getrouwd}\\

\haiku{Een getrouwde vrouw,.}{moet niet werken althans niet}{in een beschaafd land}\\

\haiku{- Ze was nog strakker,.}{van lijn geworden hij zoo}{haar portret makers}\\

\haiku{{\textquoteleft}We zullen gaan,{\textquoteright} zei, {\textquoteleft}.}{ze tegen Arnolfiheb}{een ogenblik geduld}\\

\haiku{hij sloeg zijn zwager - {\textquoteleft},?}{op de schoudervijf procent}{is veel weet je dat}\\

\haiku{{\textquoteright} {\textquoteleft}Hm,{\textquoteright} zei Pierre, {\textquoteleft}dat?}{geloof ik dadelijk en}{ben jij trots op hem}\\

\haiku{Pierre draaide het,,:}{om aan de andere kant}{stond half uitgewist}\\

\haiku{{\textquoteright} {\textquoteleft}Kijken of de trap,{\textquoteright}.}{ons houdt zei de aannemer}{en liep naar boven}\\

\haiku{Dat van Trude was -,.}{jaren geleden goed}{het bleef altijd waar}\\

\haiku{Hij bedrinkt zich aan,.}{grote woorden dacht ze en}{beet op de lippen}\\

\haiku{- Het is een meisje,, -:}{dacht ze ik zal een dochter}{hebben altijd weer}\\

\haiku{{\textquoteleft}Dank je nooit meet aan,?}{tante Suzanne die goed}{voor je is geweest}\\

\haiku{Zodra het voorjaar,,.}{werd kon ze buiten staan in}{de hoek bij de schuur}\\

\haiku{en ze stond stil voor.}{de trap om het gesprek nog}{te kunnen volgen}\\

\haiku{Haar adem hokte en,.}{ze keek naar Antoine die}{tegenover haar zat}\\

\haiku{Pas op, beeldspraak is,.}{gevaarlijk en de ziel wordt}{niet met goud gemest}\\

\haiku{Ze zag planken vol,.}{staan terwijl hij verklaarde}{niets meer te hebben}\\

\haiku{Als Rite wat meer,.}{thuis kon blijven zou er al}{veel zijn gewonnen}\\

\haiku{{\textquoteleft}En wat hebt u een,,,!}{prachtig kind dat mondje dat}{neusje die houding}\\

\haiku{hij keek niet naar hen.}{om en het kind taalde niet}{naar zijn gezelschap}\\

\haiku{- Iets anders, het mocht,... {\textquoteleft}?}{misschien wel aarde zijn maar}{niet zoNiet zo week}\\

\haiku{het bleek de zoon van,,.}{madame Rubinno}{haar overbuur te zijn}\\

\haiku{{\textquoteright} Opeens liep er een.}{traan over haar wang en vloeide}{tussen haar lippen}\\

\subsection{Uit: Kieren van de nacht}

\haiku{Dat ze plezier had -,.}{in die moord tja de hele}{buurt gnuifde erover}\\

\haiku{{\textquoteleft}Geef me een pakje,{\textquoteright}, {\textquoteleft} ',?}{uit de winkel zei zedat}{s goedkoper h\`e}\\

\haiku{Was er niet een grijns?}{van verstandhouding geweest}{tussen Koos en hem}\\

\haiku{Om acht uur had hij,.}{de lamp gedoofd ontbeten}{en afgewassen}\\

\haiku{Gisteravond had hij,;}{buiten gelopen wat hij}{weer zou willen doen}\\

\haiku{{\textquoteright} vroeg hij zonder veel.}{aandacht en keek om zich heen}{naar meer toehoorders}\\

\haiku{Zo stil mogelijk,.}{liep hij verder niemand scheen}{het op te merken}\\

\haiku{Hij volgde haar door,.}{de winkel de nauwe gang}{door naar de keuken}\\

\haiku{{\textquoteright} Hij wist niet goed wat.}{hij aan die woorden had en}{bleef haar aankijken}\\

\haiku{het kon de bomen,.}{in het bos niet vragen en}{zelfs de dieren niet}\\

\haiku{Ik heb wat laten,{\textquoteright},, {\textquoteleft}}{halen zei Koos de deurknop}{alweer in de hand}\\

\haiku{Hij zag haar blozen,.}{van inspanning het dienblad}{voor zich uit dragend}\\

\haiku{Op de voorspoed van, -,.}{de meiden daar hangt veel van}{af ook voor Berrie}\\

\haiku{Koos bleef nog liggen?}{en vroeg met slapende stem}{of hij nu al ging}\\

\haiku{in de uiterste.}{hoek van het plaatsje vormden}{zij een wijde kring}\\

\haiku{{\textquoteright} {\textquoteleft}Wat - e -{\textquoteright} {\textquoteleft}Dat ze me -;}{nu niet wil kennen ze is}{met zoveel vrienden}\\

\haiku{ik slinger wel een,.}{beetje maar daar moet je je}{niets van aantrekken}\\

\haiku{Hij schoof haar het geld.}{toe over de toonbank en ving}{een blik van haar op}\\

\haiku{{\textquoteright} Het meisje liet haar.}{hand langs zijn rug glijden en}{trok hem bij de arm}\\

\haiku{Zodra ze op straat,.}{stonden nam het meisje haar}{buit onder de arm}\\

\haiku{{\textquoteright} vroeg hij, {\textquoteleft}moet je niet,?}{terug misschien ligt er weer}{iemand in die schuit}\\

\haiku{Waarvan?{\textquoteright} {\textquoteleft}Tja - dat weet,{\textquoteright}:}{ik haast niet meer waarop ze}{smalend antwoordde}\\

\haiku{{\textquoteright} Wat schichtig keek hij,.}{naar Koos die tegenover hem}{aan de tafel zat}\\

\haiku{Hoe onzeker had,.}{hij zichzelf niet gevoeld die}{dagen zonder haar}\\

\haiku{{\textquoteleft}Zie, ik ben met u.}{al de dagen totaan het}{einde der wereld}\\

\haiku{Hij zag liever een,.}{grootboek op tafel dan De}{Strijdkreet zei Pelsen}\\

\haiku{Het doet er niet toe,,;}{kan je zeggen de natuur}{stoort er zich niet an}\\

\haiku{{\textquoteleft}Loop maar met me mee,,{\textquoteright}.}{dan krijg je wat soepvlees van}{me en hij stond op}\\

\haiku{de broodwinning, - wat -}{sta je daar weer houterig}{tussen je kissies}\\

\haiku{ze dacht dat hij het, -,.}{niet hoorde hij was niet gek}{hij hoorde alles}\\

\haiku{Van jou weet ze niets,{\textquoteright}, -?}{had Dirk gezegd wat viel er}{van haar te weten}\\

\haiku{Het hoofd van de een:}{naar de ander wendend ging}{de rechercheur voort}\\

\haiku{{\textquoteleft}Dat's niet eerlijk,.}{als ze het bij jou hebben}{gekocht weet je het}\\

\haiku{Het water was grauw,.}{de meeuwen vlogen traag en}{zonder te krijsen}\\

\haiku{Als Koos dan maar sliep.}{en niet vroeg waar hij zo lang}{was gebleven}\\

\haiku{Zo was Koos geweest;}{na de geboorte van haar}{eerste dochtertje}\\

\haiku{Natuurlijk waren -?}{er schepen ver weg zou hij}{nog willen varen}\\

\haiku{{\textquoteleft}Meid,{\textquoteright} zei Pelsen, {\textquoteleft}ik.}{wil van een bordje eten en}{uit een glasie drinken}\\

\haiku{- dan moesten de Koubers,.}{thuis zijn om naar hun knecht te}{kijken die ziek lag}\\

\haiku{aan  iedere.}{beweging zag hij dat ze}{uit haar humeur was}\\

\haiku{Koos kwam binnen met.}{twee koppen koffie en wierp}{het hoofd in de nek}\\

\haiku{waarom, hij noemde.}{de prijs en zag zijn vrouw het}{kistje inpakken}\\

\haiku{Het is pleuritis,;}{de dokter zoekt het middel}{dat hem zal helpen}\\

\haiku{{\textquoteleft}Ze  is niet voor,,!}{\'e\'en gat te vangen pa houd}{haar in de smiezen}\\

\haiku{Nu zag hij Berrie,.}{in het praalbed met koortsogen}{en hete wangen}\\

\haiku{- Ze namen de lift,.}{ze stonden voor de zaaldeur}{en gingen binnen}\\

\haiku{Wel een minuut lang,:}{bleven ze in dezelfde}{houding toen zei Dirk}\\

\haiku{Hij moest dat meisje -?}{om een portret vragen maar}{had hij het beloofd}\\

\haiku{Hij trachtte daarover,.}{te denken maar voelde zich}{leeg en machteloos}\\

\haiku{Heel langzaam verschoof.}{de grote wijzer van de}{klok boven de deur}\\

\haiku{{\textquoteright} {\textquoteleft}En als je sneuvelt,?}{in een bocht wat moet er dan}{van Jantje worden}\\

\haiku{toen is hij toch nog -?}{onverwacht gestorven waar}{had hij die gehoord}\\

\haiku{{\textquoteright} {\textquoteleft}Welnee,{\textquoteright} zei ze, {\textquoteleft}geen,.}{gekheid alleen de dag van}{de begrafenis}\\

\haiku{Ze gaf hem een por,:}{met haar elleboog maar zei}{een ogenblik later}\\

\haiku{Daar geen spreker zich,.}{meer meldde dankte Dirk voor}{de belangstelling}\\

\haiku{{\textquoteleft}Wij weten het heel,.}{goed wij hebben hem vijftien}{jaar in huis gehad}\\

\haiku{Hij wou wel, en ik, -.}{heb gehoopt dat ik ook zou}{willen maar niks hoor}\\

\haiku{{\textquotedblleft}Meid, in die jaren.}{ben ik bijna gek geweest}{van lijfsverlangen}\\

\haiku{{\textquoteleft}Als ik te oud word.}{om over je heen te stappen}{zal ik het zeggen}\\

\haiku{{\textquoteleft}Het is toch beroerd,...{\textquoteright}}{genoeg geweest als je het}{me nu maar gunt dat}\\

\subsection{Uit: De loop der dingen}

\haiku{hij kon Londen met,.}{minder bereiken zelfs al}{zou hij gaan vliegen}\\

\haiku{Wel was het zomer,,.}{maar de nachten waren koel}{op het open water}\\

\haiku{Je moet nadenken,,;}{berispte hij zichzelf geen}{dwaasheden gaan doen}\\

\haiku{Je begon met het,,;}{uitwendige te zien ze}{was mooi en bloeiend}\\

\haiku{Voor de meisjes had,.}{hij niets maar dit hinderde}{hem geen oogenblik}\\

\haiku{Zonder droom bleef ze,,.}{en zonder gedachten de}{tijd bestond niet meer}\\

\haiku{Wie neemt zooiets van,,?}{mij wat je overal krijgen}{kunt en goedkooper}\\

\haiku{Op de krant schreef ze,,, -.}{groot die enkele woorden}{de inkt vloeide uit}\\

\haiku{Maar ze wilde niet,.}{naar Engeland en hij zelf}{koos ook dezen weg}\\

\haiku{om haar heen was het,.}{geraas van de stad in den}{nieuwen jongen dag}\\

\haiku{Hij zag haar zeer bleeke,.}{gezicht en het gedwongen}{staren van haar oogen}\\

\haiku{- Nou, dat ik voor een,;}{poosje weg ging en ze niet}{ongerust moesten zijn}\\

\haiku{Och, evengoed als in,.}{Holland waar hij bij vreemde}{menschen had gewoond}\\

\haiku{O, het blinken van,.}{dat water aldoor en het}{dansen van hun boot}\\

\haiku{Hij keek naar de zon,,.}{die al daalde zijn compas}{was overbodig nu}\\

\haiku{- O. Daarna gingen.}{ze beiden weer voort aan hun}{stille gedachten}\\

\haiku{- weggaan met een man,,.}{waarvan ze niet hield en die}{haar niet begeerde}\\

\haiku{van wat ondergoed,.}{rolde ze een kussen zoo}{dat hij het niet zag}\\

\haiku{De nevel trok weg,,.}{het werd lichter maar het kon}{de dag nog niet zijn}\\

\haiku{Toen ze nu het eten -}{klaar had ze kookte rijst met}{spaghetti en ham}\\

\haiku{Ze bedacht dat ze,,.}{niets had meegenomen geen}{schaar geen vingerhoed}\\

\haiku{Ik zal schrijven, ik.}{weet niet waarom ik er zoo}{lang mee heb gewacht}\\

\haiku{Maar toen zag ze den,.}{heeten blik van zijn oogen en}{ze rukte zich los}\\

\haiku{Ze deed of ze het,.}{niet merkte maar den glans hield}{ze vast in haar oogen}\\

\haiku{- Stil, zei ze zichzelf,?}{waarom zie ik alles zoo}{duidelijk vandaag}\\

\haiku{O, wel zal ze veel, ().}{kunnen bereiken heel veel}{maar ze wist niet w\`at}\\

\haiku{- een beetje te snel,.}{en onduidelijk maar hij}{kreeg toch cum laude}\\

\haiku{Hun denken dwaalde,.}{om de vrouw die moeder is}{of het worden zal}\\

\haiku{Nu zag ze Jacques,.}{ze wist niet wanneer hij was}{binnengekomen}\\

\haiku{Aan den achterkant.}{grensde het woonhuis aan de}{fabrieksgebouwen}\\

\haiku{En een eigen huis,.}{te hebben daarop heb ik}{nauwelijks gehoopt}\\

\haiku{Er was iets in haar,,.}{nuchtere starre leven}{gekomen een droom}\\

\haiku{De bleeke rozen bij.}{de veranda praalden nog}{met groote dauwdroppels}\\

\haiku{We leven hier als,.}{in het oude Egypte dacht}{ze een oogenblik}\\

\haiku{Maar nu bleef nog haar.}{blik onafgebroken op}{de dingen rondom}\\

\haiku{- en ik leef voor hem, -.}{het is niet oneervol voor}{een vrouw dit te doen}\\

\haiku{En wat mij aangaat,.}{mag ze met de kinderen}{van het dorp spelen}\\

\haiku{En voelde dat ze,.}{te ver ging zonder zichzelf}{te kunnen stuiten}\\

\haiku{Nu was ze ook weer,.}{de tevreden rustige}{Ina die hij kende}\\

\haiku{- dat Lilly weggaat,,.}{drukt ons al willen we het}{elkaar niet zeggen}\\

\haiku{- Hij praatte voort, zei,.}{haar hoe de ligging van het}{huis was en de tuin}\\

\haiku{en toen liep ze in.}{den regen langs het strand met}{George Kelwin}\\

\haiku{Dat ze hem zonder,.}{klacht had laten gaan had hij}{in haar geprezen}\\

\haiku{Zoo weinig vroeg hij,.}{van haar en zooveel had ze}{hem afgenomen}\\

\haiku{Alle gedachten.}{aan haar eigen werk trachtte}{ze ver te houden}\\

\haiku{Ze kon niet zooveel.}{bij hem zijn als den eersten}{tijd van zijn ziekte}\\

\haiku{Toen ze het atelier,.}{binnen kwam was er een hoog}{lawaai van stemmen}\\

\haiku{- Om u de waarheid,,.}{te zeggen zei ze dacht ik}{aan mijn dochtertje}\\

\haiku{Hij had iets sluws in, -.}{zijn donkeren blik toch was}{zijn houding loyaal}\\

\haiku{- U moet dat voorste,.}{haar niet zoo strak trekken het}{doet pijn om te zien}\\

\haiku{- Kom kind, het is niet,.}{goed op den laten avond zoo}{te redetwisten}\\

\haiku{Ik heb het moeder,,.}{gezegd gisteren maar ze}{geloofde het niet}\\

\haiku{Maar Amy waardeerde,;}{dit niet dorst ook nauwelijks}{een stoel te nemen}\\

\haiku{Lilly zat rechtop;}{en keurde de handigheid}{van hun bestuurder}\\

\haiku{Ze zou dit niet met, -?}{woorden aanroeren sprak Jack}{ooit over zijn liefde}\\

\haiku{Haar gezicht was bleek,.}{en vochtig met enkele}{fel-roode plekken}\\

\haiku{Eerst was het aldoor,,,.}{onzeker van September}{af tot in Maart April}\\

\haiku{- dat maakte me zoo,;}{ellendig het altijd weer}{te moeten lezen}\\

\haiku{Al gauw merkte ze,,.}{dat het Lilly zwaar viel den}{tijd af te wachten}\\

\haiku{Ze zweeg, een beetje,.}{beschaamd ze had Lilly niets}{willen verwijten}\\

\haiku{Maar onderwijl had,.}{ze soms het gevoel alsof}{zij niet mee zou gaan}\\

\haiku{- ik wil heelemaal,.}{opnieuw beginnen onder}{andere menschen}\\

\haiku{- Toen u vluchtte, was,.}{er toch ook geen andere}{vrouw die u meenam}\\

\haiku{De slaap was heel zwaar,,.}{op haar ze wilde haar oogen}{openen maar kon niet}\\

\haiku{- Ze was alleen in,.}{haar bed het licht stond groot en}{scherp in de kamer}\\

\haiku{- Ze weten daar niet,,.}{zei ze plotseling dat miss}{Lilly al weg is}\\

\haiku{zult u altijd bij?}{mij komen als u denkt dat}{ik u helpen kan}\\

\haiku{- Het is voor u het,,.}{ergste zei hij het kind richt}{er zich nog aan op}\\

\haiku{Maar ik heb hem toen,.}{alleen gelaten ik moest}{mijn eigen weg gaan}\\

\haiku{Het water moest half,.}{verzonken zijn of verdampt}{in de heete lucht}\\

\haiku{Dat jaar was als een,.}{vuur waarin de rest van haar}{leven verzengde}\\

\haiku{Hij mocht ook opstaan,,;}{en haar gezicht tusschen zijn}{handen vatten even}\\

\haiku{- dan voelde hij een,.}{gloed dien hij niet alleen uit}{zichzelf dacht te zijn}\\

\haiku{Vreemd, - hij was nu vier, -.}{jaar in Londen dat hij Ina}{nooit meer gezien had}\\

\haiku{Prachtige stof, ja: -,.}{goed voor de hooge ramen van}{een oud voornaam huis}\\

\haiku{Daar een plaat van een,:}{oud vrouwtje dat haar bijbel}{las en een pendant}\\

\haiku{Een twee drie, de jood,.}{in den pot fijn gestampt en}{de deksel erop}\\

\haiku{Haar beschermen voor,.}{haar eigen dwaasheden want}{ze was een kind}\\

\haiku{Een vrouw, in 't langs,,.}{gaan spiedde nieuwsgierig maar}{dat merkte hij niet}\\

\haiku{Ze zou moeder zijn,,.}{van drie kinderen later}{van nog meer misschien}\\

\haiku{- Ja, zei hij, misschien,,.}{niet voor mijn vrouw maar voor mij}{ik wil naar mijn werk}\\

\haiku{Hij bukte zich over,,.}{haar keek in haar donkere}{oogen die sterk glansden}\\

\haiku{Neen, hij zou haar niet,.}{deren zelfs niet even met zijn}{lippen haar raken}\\

\haiku{O, dacht ze schamper,,.}{ze zou hem niets vragen hij}{was heer en meester}\\

\haiku{- Ze liep nooit iemand,,.}{tegemoet behalve Frans}{een enkele maal}\\

\haiku{En nu leek het haar:}{meer te zijn geweest dan een}{kus uit gewoonte}\\

\haiku{eigenlijk kon hij,.}{zich niet herinneren dat}{het ooit gebeurd was}\\

\haiku{'t Was alsof er,.}{een muur weg viel door hier aan}{te denken alleen}\\

\haiku{- Doet u geen moeite,,,.}{zei Frans ik ga er alleen}{heen en tref u daar}\\

\haiku{Haar instinct zei haar,.}{dat ze geen verwondering}{mocht laten blijken}\\

\haiku{Alleen haar jongens,;}{die kon ze een Engelsche}{opvoeding geven}\\

\haiku{Ze kwam naar hem toe,;}{over het donkerkleurige}{Smyrna-tapijt}\\

\haiku{Toch prikkelde hem,:}{het aanpassingsvermogen}{van Ruth of liever}\\

\haiku{- Haastig, alsof hij,.}{geroepen was trok hij de}{deur achter zich dicht}\\

\haiku{toen voelde hij een,.}{traan opwellen zijn heete}{oogen bevochtigend}\\

\haiku{Of was haar houding,?}{bewust zoo omdat ze zijn}{woorden negeerde}\\

\haiku{den dood zoeken zou,,.}{niet moeilijk zijn alleen laf}{en verachtelijk}\\

\haiku{- Best, zei Ruth, dan kan,.}{ik me nog kleeden terwijl}{jij de auto haalt}\\

\haiku{Dat stuk van vanavond,,?}{was leuk maar het bleef wat te}{veel spel vind je niet}\\

\haiku{- Dan begin ik nu,,.}{te helpen we maken er}{nog een voor terug}\\

\haiku{- Als ik midden voor,?}{de auto ga staan zou ze}{me dan overrijden}\\

\haiku{Je moet het maar niet,.}{probeeren als moeder met dat}{kleine ventje rijdt}\\

\haiku{Het was een beetje;}{gevaarlijk om midden op}{den weg te gaan staan}\\

\haiku{- Ik dacht dat moeder,.}{ons uit school wilde halen}{en ons gemist had}\\

\haiku{- Maar mevrouw, ik ben,.. -?}{dikwijls zoo oprecht dat Dat}{u bang bent voor uzelf}\\

\haiku{Miss Conny zag hij,.}{nauwelijks en voor James}{Lomb had hij eerbied}\\

\haiku{Hij praatte nu met,.}{Lomb legde een oogenblik}{zijn hand op diens mouw}\\

\haiku{- Blijft u nog even, het.}{spijt me dat mijn vrouw u juist}{vandaag genoodigd had}\\

\haiku{Nooit wilde ze zijn,.}{geboorteland zien of zijn}{verleden kennen}\\

\haiku{Ze zou nu niet meer,.}{kunnen zeggen waarom ze}{het begonnen was}\\

\haiku{Zijzelf had nu haar, -.}{werk het werk dat de vrouw van}{een groot man doen kon}\\

\haiku{Dus waren ze in,.}{de stad gebleven en hij}{winkelde met haar}\\

\haiku{Onwillekeurig,.}{keek hij op toen hij stemmen}{hoorde bij de deur}\\

\haiku{Hij zou willen dat,,;}{hij over Ina denken kon scherp}{over haar karakter}\\

\haiku{Ruth lachte, omdat.}{hij nog altijd gebonden}{was aan de zaak}\\

\haiku{Hij at thuis, merkte.}{dat het keukenmeisje op}{hem had gerekend}\\

\haiku{Terwijl hij schreef, wist.}{hij dat hij ook dezen brief}{niet versturen zou}\\

\haiku{Mijn eigen auto,.}{is blijkbaar stuk en dus liet}{ze zich afhalen}\\

\haiku{Op dat oogenblik,.}{wist ze dat ze in lang niet}{zoo jong was geweest}\\

\haiku{vroeg hij aarzelend, - -,.}{ik herinner me niet Neen}{zei ze bevangen}\\

\haiku{- vreemd, dat hij al dien,.}{tijd had laten voorbij gaan}{zonder haar te zien}\\

\haiku{Eigenlijk kan ik;}{me niet herinneren dat}{je ziek geweest bent}\\

\haiku{Ze verdedigde,,.}{hem koos zijn partij dwong hem}{eigenlijk tot niets}\\

\haiku{Het was of een beeld,,.}{van hemzelf binnen in hem}{zich neerboog en bad}\\

\haiku{- ze reisde er vaak,,.}{heen die dagen bleef er ook}{wel eens overnachten}\\

\haiku{- Er drongen tranen,.}{naar haar oogen een enkele}{viel stil langs haar wang}\\

\haiku{- Zoo is het goed, zei,.}{ze nu ben ik van dat al}{te zware bevrijd}\\

\haiku{- ik wandel en doe,.}{boodschappen ik voel me soms}{een mondaine vrouw}\\

\haiku{Kwam het doordat ze,,?}{niet zonder eenige moeite}{haar oude taal sprak}\\

\haiku{Toen hij op straat stond,,?}{vroeg hij zich af waarom hij}{al was weggegaan}\\

\haiku{- Maar Holland is aan, -}{den overkant van het water}{dichtbij en toch ver.}\\

\haiku{Overdag zit ik als,.}{in een draaimolen waar ik}{in- en uitspring}\\

\haiku{Zijn oogen waren groot,,.}{als van een kind zijn mond was}{oud en verwrongen}\\

\haiku{Eens ging Frans op weg,.}{naar Ina een Zondagmiddag}{in Februari}\\

\haiku{Weer sloeg hij een hoek,.}{om lette scherp op of hij}{een huurauto zag}\\

\haiku{Zij hadden aan den,:}{anderen kant gewoond hij}{wist het nummer nog}\\

\haiku{Nu glimlachte ze,.}{steunde met haar hand tegen}{den rug van zijn stoel}\\

\haiku{- U wilt toch wel eens,.}{iets beleven na deze}{twee doodsche jaren}\\

\haiku{Eenmaal, kort voor hun,;}{reisje naar Frankrijk zag ze}{Frans in den schouwburg}\\

\haiku{Ze wees hem Lilly,;}{niet aan en ze geloofde}{dat hij haar niet zag}\\

\haiku{- zoo werd ook deze.}{gebeurtenis voor haar zelf}{niet heel belangrijk}\\

\subsection{Uit: De oudste zoon}

\haiku{- Ik stuur door het dorp,,,,.}{wacht ik klim wel over de bank}{schuif dan naar rechts j\^o}\\

\haiku{Jetty liep langs hem,.}{en sloeg hem met haar kleine}{hand op zijn schouder}\\

\haiku{Hij schrok van de drift,.}{die in hem opwelde beet}{zich op zijn lippen}\\

\haiku{Toch was het vrij dun,;}{van hem dat hij niet over zijn}{toekomst gedacht had}\\

\haiku{Vader deed of hij,.}{het niet hoorde en begon}{een ei te pellen}\\

\haiku{De kleintjes, Wim en,.}{Doortje stonden vaak bij hen}{en babbelden wat}\\

\haiku{- De jongen  had;}{de auto teruggebracht}{in de garage}\\

\haiku{- Later heb ik een,.}{vrouw getrouwd die om zulke}{dingen zou lachen}\\

\haiku{Herman stond veel in.}{stille aandacht achter zijn}{vaders elleboog}\\

\haiku{Op bloote voeten liep,,.}{hij den zolder over de trap}{af zijn hart bonsde}\\

\haiku{Toen huiverde hij,.}{door moeders blik zoo hard en}{minachtend keek ze}\\

\haiku{- Het is een pracht van,.}{een mansarde ik kan je}{erom benijden}\\

\haiku{vroeg Wim, het is de,.}{mooiste kleur die ik van mijn}{leven heb gezien}\\

\haiku{Als vader wegbleef,,?}{wellicht voorgoed hoe zouden}{ze dan gaan kijken}\\

\haiku{- Ik vind  het mooi,,,?}{werk maar wel vreemd erg vreemd of}{ben ik kleurenblind}\\

\haiku{Wie niet komt, beschouwt,.}{zichzelf als verloren had}{Theo Sikkesz gezegd}\\

\haiku{In Amerika zou,....}{het misschien mogelijk zijn}{maar in Amsterdam}\\

\haiku{Theo Sikkesz had een:}{cahier vol en Piet Bril wist}{maar een onderwerp}\\

\haiku{Ze leunde tegen,.}{een jas van haar vader die}{aan den kapstok hing}\\

\haiku{We zijn allemaal,,.}{ongelukkig dacht hij en}{niemand heeft de schuld}\\

\haiku{Eerst zou hij langzaam,.}{naar Hendriks loopen de zaak}{sloot om zeven uur}\\

\haiku{De hardnekkigheid,,.}{dacht hij waarmee ze me wil}{laten schilderen}\\

\haiku{hij hoopte het, maar.}{ze hadden elkaar in geen}{vijf maanden gezien}\\

\haiku{- Ja, ik verwacht nog,?}{een paar vrienden kan je straks}{wat boven brengen}\\

\haiku{Na een poosje werd.}{Jan Mastenbroek wakker en}{kroop naar den divan}\\

\haiku{- Als ik je ooit met,.}{iets van dienst kan zijn hoorde}{hij zich toevoegen}\\

\haiku{aan de verhouding?}{met vader was haar zeker}{niets meer gelegen}\\

\haiku{Hij bloosde opnieuw,.}{want hij had het zelfbeklag}{in zijn stem gehoord}\\

\haiku{- Het is niet zooveel,.}{bijzonders jullie zoudt er}{misschien om lachen}\\

\haiku{Izenburg wist meer van,.}{de chemie af dan hij en}{bood hem boeken aan}\\

\haiku{- Niet in't water,,,.}{springen hoor zei de een en}{de ander lachte}\\

\haiku{Greet stond op, maar keek,.}{Miel nog haastig even aan hij}{zag haar oogen glanzen}\\

\haiku{Haar blonde haar was,.}{kort geknipt en toch had ze}{niets jongensachtigs}\\

\haiku{Ze keek hem aan, moest,.}{hij denken met den glimlach}{van een oude vrouw}\\

\haiku{Hij had niet gevraagd,?}{of zij dansen wilde had}{ze daarop gehoopt}\\

\haiku{Dolly kwam binnen.}{en zette een schaal op de}{gedekte tafel}\\

\haiku{- Van Helsingfors naar Hamburg,.}{dat kostte me ook bijna}{mijn laatste duiten}\\

\haiku{Toen ben ik dan ook.}{dien heelen verderen dag}{een goed mensch geweest}\\

\haiku{de dokter heeft een,.}{zee'tje van me gekocht toen}{kon ik weer even voort}\\

\haiku{Als je schildert, Miel,,.}{en heel veel buiten loopt dan}{verlies je jezelf}\\

\haiku{In het duistere.}{licht van een spaarbrander zag}{hij een meisje staan}\\

\haiku{- O, zei Greet, en keek,.}{rond het is toch anders dan}{ik me voorstelde}\\

\haiku{ze was kinderlijk,.}{van gestake maar zag er}{niet ongezond uit}\\

\haiku{Hij had het toch goed,.}{gehoord maar zijn gedachten}{verdrongen elkaar}\\

\haiku{Ook speelde daar nog;}{onderdoor de gedachte}{aan Otto's bezoek}\\

\haiku{Dus liefdes-smart -?}{en daarvoor bleef haar vader}{den heelen dag thuis}\\

\haiku{Het was een grappig,.}{meisje ze had heel fijne}{sproeten op haar neus}\\

\haiku{Vanmorgen ben ik;}{in een taxi heen en terug}{naar dien man gegaan}\\

\haiku{Ik had wel heel graag,.}{een dokter geraadpleegd maar}{Tony wil het niet}\\

\haiku{- U moest toch niet te,,.}{lang wegblijven van kantoor}{zei Miel en bloosde}\\

\haiku{hier werkt mijn zusje,.}{terwijl Otto werkelijk}{niet hooghartig is}\\

\haiku{Hij streed tegen dat.}{belachelijk gevoel van}{minderwaardigheid}\\

\haiku{Miel antwoordde niet,.}{onmiddellijk en het bleef}{een oogenblik stil}\\

\haiku{We hebben elkaar,.}{gezien dat is tot nog toe}{eigenlijk alles}\\

\haiku{Hij vroeg haar zacht - hij -?}{zat tegenover haar zit je}{op heete kolen}\\

\haiku{En ik weet wat ik,,?}{kan schaatsenrijden gaat nog}{vrij goed vind je niet}\\

\haiku{en dit is dus je,,.}{kamer ja die had ik me}{wel goed voorgesteld}\\

\haiku{als hij zoo normaal,...}{doet om terug te komen}{dan zal  hij ook}\\

\haiku{Ze brak af, als was.}{ze te moedeloos om het}{alles te zeggen}\\

\haiku{Ze boeide Miel, hij.}{las bij voorbaat de woorden}{uit haar houding af}\\

\haiku{De toon doet het hem,,.}{dacht hij fooien geven is}{zoo vernederend}\\

\haiku{In de groote foyer,.}{moest hij even rondkijken voor}{hij de Ramscheid's zag}\\

\subsection{Uit: De overgave}

\haiku{- Och - ik heb Erich Weicht,.}{gekend we waren samen}{aan de Reichsanstalt}\\

\haiku{Nu zat ze weer op.}{in haar bed en leunde haar}{hoofd tegen den muur}\\

\haiku{Toen haar jongetje -,.}{stierf maar hij was nog zoo klein}{een wiegekindje}\\

\haiku{Eens dacht ik toen ook - -.}{te komen maar och en het}{was nog niet zoo noodig}\\

\haiku{- Terwijl ze nog sprak,.}{voelde ze dat haar woorden}{geen weerklank vonden}\\

\haiku{I Josien keek nog,.}{steeds naar buiten terwijl ze}{in haar hoekje zat}\\

\haiku{- vader, of Gerda,}{of Ann-Mary en}{Gerda samene}\\

\haiku{Toe, wie heeft er nog - - -,.}{iets voor het vuur papa u}{oude brieven toe}\\

\haiku{Josien stond bij den.}{pilaar van de trapleuning}{en glimlachte even}\\

\haiku{Laatst had hij zoo naar,.}{haar gekeken alsof hij}{zijn vrouw in haar zag}\\

\haiku{Ze zou de doode,.}{grijze lucht zien boven de}{bevroren velden}\\

\haiku{- Och, het drukte mij -.}{ook ik ben er tenminste}{eens uitgeloopen}\\

\haiku{En het mooie is, zei -.}{hij dat ik het nu n\'og niet}{tragisch kan vinden}\\

\haiku{- Schrik niet - zei hij, - wij -.}{zijn gezond en boog zich naar}{haar toe voor een kus}\\

\haiku{Hoe kon het ook - die?}{vrouw in Indi\"e- ze}{woonde bij een zoon}\\

\haiku{Du bist jung, und wenn,.}{das Herz es erlaubt werden}{wir neueFreunde}\\

\haiku{Maar - tracht het me te - -.}{vergeven had hij gezegd}{ik moet alleen gaan}\\

\haiku{O, ik heb altijd,,.}{veel gewild veel verlangd ik}{zal het blijven doen}\\

\haiku{Ze keken beiden,.}{naar den lichten hemel maar}{dachten niet aan zien}\\

\haiku{Hij vroeg haar en ze,.}{wees hem af want ze had nooit}{zijn liefde vermoed}\\

\haiku{ik moet nu maar voort,.}{ik moet maar zien een korten}{tijd zoo te leven}\\

\haiku{- Bertel en Franz zijn -.}{uit zijn eerste huwelijk}{hun moeder is dood}\\

\haiku{als ze even ophield,:}{met praten dan was daar zijn}{fluisterstemmet je}\\

\haiku{Zacht nam Josien zijn.}{hoofdje tusschen haar handen}{en streelde zijn haar}\\

\haiku{Ze had de laatste.}{dagen steeds sterker met Ada's}{gezin meegeleefd}\\

\haiku{- Tja - Ze praatten nog -.}{door over Ada ernstig en niet}{zonder inspanning}\\

\haiku{- Niet hier blijven, zei,.}{Franz in deze kamer is}{het een beetje suf}\\

\haiku{- Ik zie een knol, met,,,.}{een langen staart riep Franz een}{grijzen staart zoo oud}\\

\haiku{Hij was op den grond,.}{gaan liggen met zijn hoofd op}{een voetenkussen}\\

\haiku{En omdat Ada in,:}{een hoek van de kamer op}{den grond lag zei ze}\\

\haiku{De beklemming om.}{de korte disharmonie}{in Ada's huis viel weg}\\

\haiku{Als Albert kwam - dacht -,.}{ze maar het leek haar te veel}{in zoo korten tijd}\\

\haiku{Hij wilde niet den.}{indruk wekken van steeds te}{komen controleeren}\\

\haiku{Albert - riep ze - maar.}{de stilte van den tuin werd}{er niet door verstoord}\\

\haiku{En dan mag hij je,.}{toch wel eens de baas zijn als}{je het moeilijk hebt}\\

\haiku{Het huilt niet eens en.}{we laten het bijna van}{de tafel vallen}\\

\haiku{- Het kindje begon.}{te schreien en ze wiegde}{het in haar armen}\\

\haiku{vroeger was ik wel - -.}{angstig en nu sta ik er}{niet meer alleen voor}\\

\haiku{Het wil ons leeren ons -.}{aardsche leven te kennen}{en lief te hebben}\\

\haiku{- Ja - en dat doe je,,.}{wellicht ook in je hart maar}{je toont het me niet}\\

\haiku{Ze merkte wel, dat;}{hij haar het liefst van het kind}{hoorde vertellen}\\

\haiku{Het land is nog zoo, -,.}{ver weg zei ze ik kan het}{zien maar niet pakken}\\

\haiku{het stond vol bloemen.}{en het geurde naar vocht en}{vruchtbare aarde}\\

\haiku{foei, je moest naar het.}{huis van je moeder gaan en}{aan de wasch helpen}\\

\haiku{Maar hij liet het graag.}{wapperen in den wind als}{een jongen zijn kuif}\\

\haiku{Ja - mijn hart is niet - -.}{heel sterk meer maar er kloppen}{zwakkere harten}\\

\haiku{En jullie huis in,?}{Holland was dat veel grooter}{en mooier dan dit}\\

\haiku{Je kunt toch blijven,?}{om \'ons in hoe lang hebben}{we je niet gezien}\\

\haiku{Aan Alberts oogen zag,;}{ze dat hij haar kende dat}{zijn geest helder was}\\

\haiku{- nooit eerder had ze}{zoo duidelijk den val en}{den klim van het licht}\\

\subsection{Uit: De schaatsentocht}

\haiku{- Je had ineens een,.}{groot verlangen naar Harm als}{je nu aan hem dacht}\\

\haiku{je trok weer al het,.}{dek over je heen en je hart}{gaf moeder gelijk}\\

\haiku{Nu was je dertig,.}{geweest en je geloofde}{niet dat je oud was}\\

\haiku{Nu zou je alleen.}{nog naar het zingen van de}{ijzers luisteren}\\

\haiku{Tusschen Halfweg en.}{Haarlem ging ze een tentje}{op het ijs binnen}\\

\haiku{Ik laat de menschen,,.}{maar praten ik weet het zelf}{ook niet hoe ik ben}\\

\haiku{Leen had zwijgend het;}{keukentafeltje voor drie}{personen gedekt}\\

\haiku{- Och, zei je, - dat is -?}{van geen belang meer maar waar}{zal ik van leven}\\

\haiku{En daar zaten ze.}{gedrie\"en aan het kleine}{keukentafeltje}\\

\haiku{- Nou - en - vermoei je,.}{niet te veel neem een treintje}{in Lisse of zoo}\\

\haiku{Om vier uur had ze,.}{je voorgesteld thee te gaan}{drinken in Den Haag}\\

\haiku{- Nu niet meer - alles -,,....}{opgemaakt nou ja niet voor}{zichzelf hoe gaat dat}\\

\haiku{o - het gaat mij niets, -.}{aan natuurlijk maar prettig}{dat hij het kan doen}\\

\haiku{Je kon niet goed meer -?}{luisteren een baantje aan}{Dirk's eigen fabriek}\\

\haiku{dan moest ze minstens, -.}{nog veertien dagen blijven}{en dat wil ze niet}\\

\haiku{En iederen dag.}{daarvan leek moeder opnieuw}{te zijn gestorven}\\

\haiku{- Het zou gezellig, -.}{zijn jij hier in huis maar ik}{wil niet aandringen}\\

\haiku{Na moeders dood had,,.}{hij geschreven aan jou en}{Harm de twee jongsten}\\

\haiku{Promotie - een vreemd.}{woord voor een onderwijzer}{zonder hoofdacte}\\

\haiku{- Als Koos ransel moet, -.}{hebben dan zeg je het mij}{en nu geen woord meer}\\

\haiku{Dan kan hij eens naar.}{een particuliere school}{solliciteeren}\\

\haiku{- Henk is naar Utrecht - hij;}{werkt aan een artikel voor}{het Chemisch Weekblad}\\

\haiku{Leida praatte niet -.}{meer vroeger was ze ook nooit}{zoo spraakzaam geweest}\\

\haiku{Als je nu meteen,?}{kon beslissen zou je dan}{niet gelukkig zijn}\\

\haiku{De wanden en  ;}{de zoldering waren van}{lichtgeschilderd hout}\\

\haiku{de steriliteit;}{van je lichaam zou je geest}{kunnen aantasten}\\

\haiku{- Dan kruip ik er zelf,,.}{maar in want ik moet morgen}{weer werken zie je}\\

\haiku{Ik ga trouwen, - dan.}{werd je verlost van al die}{opdringerigheid}\\

\haiku{- Met ieder woord zeg, -.}{je dat je niet van me houdt}{beken het dan ook}\\

\haiku{Je sloeg de dekens -.}{op en huiverde maar je}{kon niet meer terug}\\

\haiku{De tranen sprongen -.}{in je oogen je voelde dat}{je gered was}\\

\subsection{Uit: Een sprookje}

\haiku{Jules praatte met,;}{Dutout zijn zwarte kop naar}{voren gestoken}\\

\haiku{de mond, flets en groot,.}{in goede verhouding tot}{de jukbeenderen}\\

\haiku{sneeuw hier en daar, zoo'n,.}{beetje neergestrooid zonder}{kleur-beteekenis}\\

\haiku{Ik wil het hout zien,,,.}{zei Pierre het is donker}{glimmend notenhout}\\

\haiku{Je kunt haar altijd,.}{bij me binnen laten want}{ze komt niet dikwijls}\\

\haiku{- als ik nu eens uit - -.}{een slecht gezin kom Dan zal}{ik verwonderd zijn}\\

\haiku{De muur is grauwwit,.}{en er kleeft een donkere}{klimplant tegenaan}\\

\haiku{- o, en een vlek op -,.}{de schilderij het is uw}{schuld het is spotten}\\

\haiku{Zoo hij bleef zwijgen -, -.}{goed zij sprak geen woord maar ze}{was klaar voor een lach}\\

\haiku{En als je zijlings,.}{gaat zitten komt toch die weeke}{lijn weer naar je toe}\\

\haiku{Een haan kraaide, en.}{er was een hooge jubel van}{vogels in de lucht}\\

\haiku{- Ze strekte haar arm.}{uit naar Ren\'ee en begon}{weer schel te lachen}\\

\haiku{- Hij steunde met zijn.}{armen op de  tafel}{en keek Ren\'ee aan}\\

\haiku{Vroeg in den nacht, en,.}{iederen nacht vroeger werd}{de dag geboren}\\

\haiku{Het kind had zwakke,,.}{misvormde beentjes waarop}{het niet kon loopen}\\

\haiku{- maar ik wil ook niet;}{al te veel in de maling}{worden genomen}\\

\haiku{Ze huilde, en hij - -.}{gaf haar niet den troost waarop}{ze even nog hoopte}\\

\subsection{Uit: Vriendschappen}

\haiku{vandaag proef ik het, -.}{leven zoo scherp en ik heb}{niets anders te doen}\\

\haiku{jij, als journalist, - - -;}{keurt daarin alles af maar}{de natuur Zeker}\\

\haiku{En ik mag hier van,.}{de zon houden zooals ik het}{als kind geleerd heb}\\

\haiku{Onderwijl zag ze.}{dat het in de serre vol}{en rommelig was}\\

\haiku{Stans, nog steeds in haar,.}{pyjama kwam binnen en}{liep de serre door}\\

\haiku{- Hier is het - alles -,?}{mag erin blijven het wordt}{heel mooi ziet u wel}\\

\haiku{Vroeger had het zich.}{in alle spelletjes door}{Cor laten leiden}\\

\haiku{- De duiven  gaan -?}{lager vliegen ze zullen}{toch niet moe worden}\\

\haiku{- We kunnen winkels,.}{kijken maar we hebben geen}{geld om te koopen}\\

\haiku{Leni wist niet van,.}{ophouden haar stem klonk een}{beetje zeurderig}\\

\haiku{- In Holland kunnen, -,.}{we immers niets alleen thuis}{zitten en lezen}\\

\haiku{- Och - het is niet een -.}{kwestie van benutten je}{moet ernaar leven}\\

\haiku{- Je officieel -.}{te laten fotografeeren}{zonde van het geld}\\

\haiku{de kinderen zijn,.}{in mijn huiskamer ik hoef}{toch niet weg te gaan}\\

\haiku{- Kinderen, waarom,.}{zitten we hier laten we}{ergens gaan zwemmen}\\

\haiku{- Jullie kunt praten,,.}{zei Cor eigenlijk lang voor}{dat je het noodig hebt}\\

\haiku{- Ik had Leni nog -.}{nooit zoo gezien in Indi\"e}{was ze toch anders}\\

\haiku{Blijkbaar wilde hij,.}{beneden blijven totdat}{de bezoekster ging}\\

\haiku{- En misschien, ging ze, - -}{voort als ik dadelijk naar}{u was toegegaan}\\

\haiku{- U moet het hem niet, -.}{kwalijk nemen zei ze hij}{is gezond en sterk}\\

\haiku{- vader staat voor een,.}{beslissing hij wil graag van}{de gemeente weg}\\

\haiku{- Later zal je zien -.}{dat het wel gehinderd heeft}{en dan heb je spijt}\\

\haiku{- En dan bleven ze.}{ook enkele dagen thuis}{en Cor werkte}\\

\haiku{Ze voelde dat de.}{aanblik van de sterren haar}{begrip verruimde}\\

\haiku{Cor was haar zoon - niet -.}{haar bezit hij deed wat hij}{meende dat goed was}\\

\haiku{Stel je voor dat je! - -.}{daarom trouwt en nu wil ze}{weer eens wat anders}\\

\haiku{Maar lang kon ze het -.}{niet volhouden het werd te}{eentonig en triest}\\

\haiku{- Nee, dat weet ik wel -.}{maar of er ooms en tantes}{zijn overgekomen}\\

\haiku{- Er is een neef van,.}{vader die vroeger bij oma}{in huis heeft gewoond}\\

\haiku{Johanna zag Stans;}{en Leo in een hoekje met}{elkaar fluisteren}\\

\haiku{- Waarde vrienden, zei - -}{hij op het kerkhof heb ik}{niet willen spreken}\\

\haiku{die hield haar gezicht,,.}{opgeheven het was bleek}{maar onbewogen}\\

\haiku{- Weer dacht ze aan het,.}{gezicht van Hermien en vond}{langzaam haar woorden}\\

\haiku{En wat zou het haar?}{helpen of ze Greet en haar}{moeder negeerde}\\

\haiku{tegenover dezen.}{man heeft Jeanne al haar}{wijsheid verloren}\\

\haiku{- Waarom moest hij dit -, -.}{schrijven het is schaamteloos}{dat hij dat niet voelt}\\

\haiku{Ik eisch van je,.}{dat je in de toekomst niet}{nog meer schulden maakt}\\

\haiku{Maar dat raakt me 200 -,.}{weinig wellicht is het zelfs}{beter 200 voor hem}\\

\haiku{- Stoort u me niet om - - '.}{elf uur geen kopjes koffie}{ik ga aant werk}\\

\haiku{we hebben nooit veel -.}{tijd om ons te bezinnen}{het leven gaat voort}\\

\haiku{Stans had zich een wat.}{gerekten maar luchtigen}{klaagtoon aangewend}\\

\haiku{- Eens kijken - het is -.}{nu Woensdag Zondagavond zal}{hij zijn gekomen}\\

\haiku{Kom vanavond bij mij -,.}{als moeder weer uit wil moet}{ik op Jon passen}\\

\haiku{Wat komt het zelden,, -}{voor dacht Johanna dat ik}{hen samen zie nu}\\

\haiku{Je bent te zwijgzaam, -.}{met mij praat je ook niet ik}{geloof met niemand}\\

\haiku{- Dat is geen reden.}{om de dingen niet zoo goed}{mogelijk te doen}\\

\haiku{En ik kan daar niet,.}{aan meedoen daardoor prikkel}{ik haar voortdurend}\\

\haiku{Hij tuurde voor zich,,.}{uit zijn blik was helder en}{naar buiten gericht}\\

\haiku{het kon zijn dat jij.}{ook nog naar de zuivere}{zielsverwantschap zocht}\\

\haiku{- Ga dan naar je bed,.}{zei Max. De jongen leunde}{tegen de tafel}\\

\haiku{- Dank u. - Mevrouw - Toen.}{hief ze in gedachten haar}{hoofd op naar Bart}\\

\haiku{Johanna ademde.}{diep en rook den geur van hars}{en dennenaalden}\\

\haiku{En dus, als ik geen,.}{ja en amen zeg op alles}{dan wil je scheiden}\\

\haiku{Veel later wil ik -.}{hier misschien weer wonen maar}{voorloopig niet}\\

\haiku{- De notaris woont,.}{altijd in het mooiste huis}{van het dorp zei Toos}\\

\haiku{Johanna moest een.}{glimlachje verbergen om}{die lustelooze stem}\\

\haiku{- Er was plotseling,.}{een ander licht in haar oogen}{en ze leek jonger}\\

\section{Lode Zielens}

\subsection{Uit: Het duistere bloed}

\haiku{er was iets dat zij.}{allen kenden en voor mij}{verborgen hielden}\\

\haiku{het antwoord op die,.}{eene brandende vraag kon ik}{niet onderscheppen}\\

\haiku{zij de minnares.}{van mijn vader immers en}{mijn liefste vriendin}\\

\haiku{Beelden, toestanden.}{sprongen op uit het doosje}{van het verleden}\\

\haiku{Ik voelde mij den, -.}{rijkste te rijk en wist me}{toch zoo armtierig}\\

\haiku{krachtige man wiens.}{levenslust de grenzen van}{het dorp v\`er overschreed}\\

\haiku{Zij was de eerste,,.}{aan wie ik mij verlossend}{had overgegeven}\\

\haiku{Zij lachte pijnlijk.}{uitbundig toen ik haar mijn}{vreezen vertelde}\\

\haiku{- Toen ik het epistel,.}{driemaal gelezen had liep}{ik de vensters open}\\

\haiku{Het open leven, de.}{kracht van de natuur staalden}{mijn borst en armen}\\

\haiku{En er ook alleen?}{maar tijdens het nachtelijk}{uur uitgehaald werd}\\

\haiku{Maar vooral meende.}{ik met haar te stoeien om}{John te plagen}\\

\haiku{Ik streelde haar de,,.}{wangen omvatte speels haar}{leest kittelde haar}\\

\haiku{Ik bemoeide mij.}{met Liza niet m\'e\'er dan strikt}{noodzakelijk was}\\

\haiku{Ik raadde het niet,, -:}{ik zag het  niet ik wist}{met groote zekerheid}\\

\haiku{Voor vrouwen als zij, -.}{is de keus niet moeilijk een}{verlies niet zwaar}\\

\haiku{Hij nam ze onder.}{zijn arm en duwde dan met}{vader den wagen}\\

\haiku{hier weg, - ergens heen, -,.}{w\'a\'ar deed er niet toe als ik}{maar aktief kon zijn}\\

\haiku{Ook naar mijn zin was.}{ik te koel bij de laatste}{kus op Liza's mond}\\

\haiku{Als de anderen:}{staarde ik uren naar dat eene}{punt van den einder}\\

\haiku{Ik vleide mij zeer,.}{behoedzaam tegen haar aan}{gaf zachte duwen}\\

\haiku{Heel den tijd staarden.}{we in ons zelf gekeerd en}{angstig voor ons uit}\\

\haiku{Zij deed of zij niets.}{van mijn moeilijkheden en}{konflikten merkte}\\

\haiku{Liza was er dus.}{medeplichtig aan en wist}{dat ze verkeerd deed}\\

\haiku{Ofschoon een kracht, als,.}{het ware buiten mij om}{mij naar Tine dreef}\\

\haiku{Ik neem ze op, - kus, -,.}{ze en mijn kus is een beet}{waaronder zij kreunt}\\

\haiku{Daarvoor belette.}{ik te vaak het deinen van}{haar jonge lichaam}\\

\haiku{Daarin herkende.}{ik de proletarische}{afkomst van Liza}\\

\haiku{Ze vertrouwde zich,!}{aan mij wij waren immers}{oude bekenden}\\

\haiku{En is nu nog bij,.}{mij zal vermoedelijk thans}{wel bij mij blijven}\\

\haiku{- Wat een charmant kind,,...}{die Tine van u trachtte}{haar stem te kweelen}\\

\haiku{Het scheen mij toe dat.}{zij zeer met welbehagen}{naar hem luisterde}\\

\haiku{Volgens Liza deed.}{ik weer niet genoeg voor den}{bloei van het lokaal}\\

\haiku{Hij heeft een moto.}{en komt soms midden in den}{dag even opzetten}\\

\haiku{- Hij is zeker weer,.}{dronken hoor ik Tine tot}{haar moeder zeggen}\\

\haiku{- Ik wil al mijn geld, -.}{vermaken aan Anna als}{gij het niet opeischt}\\

\subsection{Uit: Op een namiddag in september}

\haiku{Negentien ben ik,,,.}{en een vrouw rijker rijker}{armer dan vele}\\

\haiku{Hij keek mij lang en,.}{doordringend aan hij begon}{afscheid te nemen}\\

\haiku{{\textquoteright} ~ God, lieve God,:}{nu weet ik het plots met een}{koele helderheid}\\

\haiku{Hij was haastig, ik,.}{moest maar gaan  hier was toch}{niets meer te helpen}\\

\haiku{Er streek een vogel.}{neer in het struweel en zong}{een fluweelen lied}\\

\haiku{ook voor hen is hij,,}{een lafaard een verrader}{niet beseffend welk}\\

\haiku{toen ik op mijn beurt,,}{een bloem in den put wierp een}{witte anjer}\\

\haiku{Wij gingen vroeg te,.}{bed in een h\^otel vlak aan}{den stroom gelegen}\\

\haiku{Ik kwam te bed en,.}{het verwonderde mij niet}{dat ik moest schreien}\\

\haiku{Hij ontwaakte en.}{legde zijn warme hand over}{mijn vochtige oogen}\\

\haiku{Zonder antwoorden,.}{stond ik op kleedde mij aan}{en wachtte op hem}\\

\haiku{Ik verheugde mij,.}{daarover want thuis had zij het}{ellendig gehad}\\

\haiku{De vlammen van den,.}{haard verlichtten de onrust}{in mij ontstoken}\\

\haiku{Maar dit hoofd dan, zijn,,.}{hoofd ik begreep het niet ik}{begrijp het nog niet}\\

\haiku{- Moeder, hoorde ik,,?}{hem fluisteren moeder denkt}{gij dat ik niet lijd}\\

\haiku{En meteen voelde:}{ik weer wat mij van dezen}{mensch verwijderd heeft}\\

\haiku{- Ik haat u niet, zei,,,}{ik hem aankijkend ik haat}{u niet waar h\'a\'alt gij}\\

\haiku{Mijn liefde voor u,?}{is nooit verminderd hoef ik}{het u te zeggen}\\

\haiku{Aan de deur wendde,,:}{hij zich nog eens om keek mij}{aan dacht waarschijnlijk}\\

\haiku{ik zie de roode.}{kralen daarvan als kleine}{rozen schitteren}\\

\haiku{Maar van jongsaf had ik,.}{het besef dat het wel eens}{\'anders zou worden}\\

\haiku{Wacht dacht ik, later,,.}{onthoud mijn naam later zult}{gij er van hooren}\\

\haiku{Uw lichaam is als,.}{een blanke vaas waarin het}{gist van levensdorst}\\

\haiku{{\textquoteright} Tot dan toe had ik:}{slechts enkele brieven van}{vader ontvangen}\\

\haiku{ofschoon voor de wet,.}{en natuur uw kind ben ik}{toch uw kind niet meer}\\

\haiku{Na een poosje, zag.}{hij het nuttelooze van den}{strijd in en ging heen}\\

\haiku{Ik heb het hem nooit}{verteld en het gekke is}{dat ik niet besef}\\

\haiku{Al dien tijd was ik, ....}{blijven zitten ik stond nu}{recht en duizelde}\\

\section{Belle van Zuylen}

\subsection{Uit: Mijnheer Sainte Anne}

\haiku{{\textquoteleft}Maar, gegeven het,?}{feit dat u niet van lezen}{houdt wat gaat u doen}\\

\haiku{Een ogenblik later:}{slaakt mevrouw De Rieux een}{doordringende kreet}\\

\haiku{dat ik nooit iets had,.}{geleerd en dat ik zelfs niet}{had leren lezen}\\

\haiku{{\textquoteleft}Ik zal, wat voor weer,;}{het ook mag zijn iedere}{dag op les komen}\\

\haiku{Geen edelman uit de;}{nabije omtrek is ons ooit}{komen opzoeken}\\

\haiku{ik geef toe dat we,.}{er sinds de oorlog weinig}{over hebben maar toch}\\

\haiku{{\textquotedblleft}Voordat we getrouwd,;}{waren was je zachtaardig}{en inschikkelijk}\\

\haiku{Want dat is wat de,.}{vrouw over wie ik zojuist}{sprak is overkomen}\\

\haiku{Vroeger beviel hij,?}{mij waarom zou hij mij nu}{minder bevallen}\\

\haiku{Zij was als een bloem.}{waarop de stralen van de}{zon niet meer vielen}\\

\haiku{het glas valt, breekt, en.}{de wijn stroomt over mijn rok die}{helemaal wit w\'as}\\

\haiku{{\textquoteleft}Uw kinderjaren,{\textquoteright}.}{zijn overschaduwd door rampspoed}{zei Sainte Anne}\\

\haiku{Ik ben degene,.}{die daarvoor kan zorgen en}{ik zal dat ook doen}\\

\haiku{nergens onze ziel,.}{heeft geraakt sluit zijn ogen voor}{de realiteit}\\

\haiku{{\textquoteleft}Wat mij betreft,{\textquoteright} zei, {\textquoteleft}}{zijn metgezelik ben te}{oud om te leren}\\

\haiku{{\textquoteright} {\textquoteleft}U slaat mij uiterst,{\textquoteright}.}{behendig mijn wapens uit}{handen zei zijn vriend}\\

\haiku{O eerlijke en,!}{eenvoudige gastvrijheid}{wat heb ik u hoog}\\

\haiku{{\textquoteright} Toen Sainte Anne,:}{niet antwoordde ging Duval}{dicht bij Herfrey staan}\\

\haiku{{\textquoteright} {\textquoteleft}Wat zij is, kan met;}{de woorden leuk en mooi niet}{worden uitgedrukt}\\

\haiku{Trouwens, via mijn zoon.}{hebt u toch al een goed beeld}{van haar gekregen}\\

\haiku{Gehoorzaam aan die,.}{ingeving want zij is je}{een goede leidster}\\

\haiku{De koets was komen,;}{voorrijden en Tonquedec}{en zij stapten in}\\

\haiku{{\textquoteleft}Het gaat niet alleen,,;}{om leuke jurken mooie sjaals}{mutsen of een hoed}\\

\haiku{Zij wilde het niet,,:}{zeggen maar toen hij om zich}{heen keek begreep hij}\\

\haiku{Uw goede vriendin,!}{gaat trouwen en u bent op}{tijd voor de bruiloft}\\

\haiku{{\textquoteleft}Als Sainte Anne,.}{zijn cousine liefheeft zal}{ik van haar afzien}\\

\haiku{Mevrouw De Sainte.}{Anne deed hem weer opstaan}{en omhelsde hem}\\

\haiku{{\textquoteright} En zij legde de.}{hand van haar dochter in die}{van Sainte Anne}\\

\haiku{{\textquotedblleft}Wanneer de vrouw zich,:}{thuis verveelt dat de man haar}{pen en inkt verheelt}\\

\renewcommand{\thechapter}{*}
\chapter[4 auteurs, 1584 haiku's]{vier auteurs, vijftienhonderdvierentachtig haiku's}

\section{Ad Interim}

\subsection{Uit: Ad Interim. Jaargang 1}

\haiku{ik worde gedwee,.}{is water noch zand deze}{oever der zee}\\

\haiku{{\textquoteright} 4  Zo zingen, '.}{wij ten slot ent is een}{lofzang van de dag}\\

\haiku{De wellust van te.}{leven Houdt om uw oog haar}{vochten mist gespreid}\\

\haiku{, blijven we even staan,.}{verdronken in de glanzen}{van elkanders ogen}\\

\haiku{Mijn hart is er zo,.}{weinig bij Als kleedt zij zich}{niet uit voor mij}\\

\haiku{En wie zich aankleedt,.}{bij het raam Die kent haar niet}{meer bij de naam}\\

\haiku{Daar woekerde het,.}{laagste leven Of wat de}{schijn ervan bezit}\\

\haiku{Zij drong ontbindend,.}{in ons leven Als splijtzwam}{dreef z'ons van elkaar}\\

\haiku{Op de jongste  ,.}{Jij bent de jongste maar mijn}{stoutste ongerief}\\

\haiku{De laatste keer was}{hij van ver komen loopen}{om mij te vragen}\\

\haiku{Jij bent waarschijnlijk,.}{de eenige mensch die mij niet}{teleurgesteld heeft}\\

\haiku{Hij besefte niet,.}{dat hij voorgoed moest blijven}{of niet weer komen}\\

\haiku{Soms moet hij de les.}{duizendmaal leeren alvorens}{de overgave komt}\\

\haiku{Een weggeblazen.}{rozenblad Drijft haastig op}{het zwalpend nat}\\

\haiku{En ach, zo dicht bij.}{Charon's veer Wagen zich geen}{verliefden meer}\\

\haiku{Toen het bed, als een,.}{wielewaal Stond te zingen}{op zijn spiraal}\\

\haiku{Geen bekommernis}{om wat ik deed of wat}{ik heb gelaten}\\

\haiku{Hij lijkt, al is zijn,,.}{houding fier Niet op een musch}{maar op een dier}\\

\haiku{laag en drukken zwaar.}{en ratten tieren in de}{kelderkluizen}\\

\haiku{Onschuld en schoonheid,,.}{eerstelingen zijn wereld's}{vijanden sindsdien}\\

\haiku{De stad ging over in.}{eigendom van een met haar}{begonnen plan}\\

\haiku{Vandaag hebben wij.}{een goede greep gedaan in}{de appelbomen}\\

\haiku{de metaphoor ligt in;}{levende lijve op die}{en die breedtegraad}\\

\haiku{Zodat ik b.v. naar:}{een herbergier zou kunnen}{lopen en zeggen}\\

\haiku{Ik vlood de bergen '.}{af tot hier mijt water}{stuit van de rivier}\\

\subsection{Uit: Ad Interim. Jaargang 2}

\haiku{Soms ziet men er een.}{zijn lege bierglas naar de}{tapkast schuiven}\\

\haiku{De bosschen van het}{vaderland de bloemen in}{de lenteweiden}\\

\haiku{De zes of zeven.}{ouwe kereltjes lachen}{openlijk en honend}\\

\haiku{Wie een zoon heeft, zooals,.}{ik weet dat hij van zijn zoon}{toch het meeste houdt}\\

\haiku{Daarna was het weer.}{stil en ik keerde mij op}{mijn andere zij}\\

\haiku{Toen hij slap, oogen dicht,,.}{op den divan lag wist ik}{dat hij niet dood was}\\

\haiku{Hij lijkt op een koord,;}{waar de danser langs gaat in}{de trillende lucht}\\

\haiku{Strijdvaardig, wakker,,?}{gelijk de wind Ach waarom}{u te overijlen}\\

\haiku{Het fonklend water,.}{danste als zij Doch lichter}{was hun vroolijkheid}\\

\haiku{er roepen grauwe,......}{wulpen en berken staan er}{wit en ongeteld}\\

\haiku{Nu is de vrijheid'!}{uitgebroken een nieuwe}{lent die niemand stuit}\\

\haiku{Er is een liefde,,,.}{niet te blusschen voor man noch}{vrouw en zonder tijd}\\

\haiku{mijn hoofd zat voller;}{dan mij lief was met cijfers}{en verhoudingen}\\

\haiku{{\textquoteleft}Marie,{\textquoteright} zei hij - {\textquoteleft}had, -{\textquoteright}}{je niet moeten doen heb ik}{niet aan je verdiend}\\

\haiku{Wat er geweest is,?}{weet hij niet meer een hond die}{hem heeft doen schrikken}\\

\haiku{Twee dagen later,:}{zegt Frederik tegen hem}{bedaard en rustig}\\

\haiku{Van haar hoorde hij {\textquoteleft}{\textquoteright}.}{desmokkelarij met Slau's}{geboortedatum}\\

\haiku{{\textquoteright} Onder ons gesprek.}{kwamen Ter Braak en Adriaan}{van der Veen binnen}\\

\haiku{{\textquoteleft}Jullie denkt, dat het,,.}{me spijt weg te gaan maar ik}{ben blij dat ik ga}\\

\haiku{Hoe k\'on ik dat niet,?}{eerder weten niet beter}{zien in vroeger tijd}\\

\haiku{, een volksbewustzijn,,.}{dat alleen nog maar zwart-wit}{kent vrijheid of dood}\\

\haiku{Verbeter de wet,,.}{de maatschappij dan wordt ook}{de mensch wel beter}\\

\haiku{Het spel dat er met,.}{hem gespeeld wordt bevredigt}{hem echter maar half}\\

\haiku{hier kwam genas aan,}{U o Amsterdam en al}{wie van dit leven}\\

\subsection{Uit: Ad Interim. Jaargang 3}

\haiku{Het is natuurlijk,,.}{oliedom zal men zeggen zo}{te redeneren}\\

\haiku{Daar werd een zangstuk.}{van Wagner opgevoerd in}{de Franse versie}\\

\haiku{Anderen meenden.}{weer dat hij een verborgen}{liefde koesterde}\\

\haiku{soms als een ster die,.}{flonkert door wolkenscheur breekt ge}{wat u omhing}\\

\haiku{Hij neemt de fluit, en,.}{blaast en blaast Dat het een lust}{was om te hooren}\\

\haiku{Alleen als slager,,.}{onderlegd Heb ik begeerd}{fluitist te wezen}\\

\haiku{Neen, zij zaten er,.}{nog zij voelden stijf en hard}{tusschen de vingers}\\

\haiku{dat de Staten macht,;}{hadden en meer macht zouden}{krijgen binnenkort}\\

\haiku{Grootendeels waren;}{zij trouwens uit andere}{dorpen afkomstig}\\

\haiku{Maar ook hij kende,.}{geen angst en nog wel een uur}{bleef hij zoo zitten}\\

\haiku{Wij stonden huid aan,.}{huid met geschrokken beenen en}{vlijmende keelen}\\

\haiku{Wij eten om zes uur,{\textquoteright}.}{precies het is nu vijf voor}{zessen zei de vrouw}\\

\haiku{Nee, nee, nee, dat ging,,...}{niet dat ging onmogelijk}{ze zouden denken}\\

\haiku{Het was de eerste,.}{keer dat ze rechtstreeks het woord}{tot Loulou richtte}\\

\haiku{rondom mij bloeit het,.}{licht En stort in d'avond uit}{zijn zegeningen}\\

\haiku{Wie houdt van dorp of:}{stad op aarde heeft daarom}{meer lief dan hij ziet}\\

\haiku{wat zij zich wel niet.}{Kan droomen dan als roos van}{hartsverlangen}\\

\haiku{Mij is 't om het,,,.}{even Noemt gij het \'uw of mijn}{ons aller leven}\\

\haiku{er is een vlijmscherp.}{onderscheid tussen elk ding}{in het heelal}\\

\haiku{Wat het buiten zich, -}{zag als fantoom Was in hem}{als toekomst mij staat}\\

\haiku{Een kreet door den nacht,?}{desolaat Of was het een}{vlaag van den wind}\\

\haiku{Moede en ziek en,.}{oud Maar nog niet verslagen}{Wacht ik op uw woord}\\

\haiku{Als gij naadren zoudt,.}{Werd alsnog verhoord Wat ik}{niet kan vragen}\\

\haiku{Hij kon dat moppig.}{overdrijven van Cornelis}{niet erg waarderen}\\

\haiku{Hij knoert 'r oever{\textquoteright},,:}{schaterde oom T. nog maar}{plotseling weer kwaad}\\

\haiku{De avonden die ik,.}{niet bij Cornelis of oom}{T. doorbracht las ik}\\

\haiku{Wat ons bond was geen,.}{vriendschap meer het was van mijn}{kant een obsessie}\\

\haiku{Hij verbreekt het, zooals,}{een kind gedaan zou hebben}{door aan te bieden}\\

\haiku{En ook het leven,,.}{kan soms als het zin heeft wel}{sprookjes vertellen}\\

\haiku{Er is geen toovenaar noodig,.}{die met \'e\'en zwaai van zijn staf}{alles anders maakt}\\

\haiku{Daar net, toen hij met.}{zijn rug naar haar toe stond had}{hij niet gelachen}\\

\haiku{{\textquoteright} zei ze, toen ze voor,.}{hem bleef staan hijgend en met}{een piek voor haar ogen}\\

\haiku{En dan de derde,,.}{klas en de vierde met de}{grote kinderen}\\

\haiku{Als een elastiek dat,.}{je bij de einden houdt en}{rekt en rekt en rekt}\\

\haiku{Toch bleef die portier,;}{mij dwars zitten toen ik in}{de wachtkamer was}\\

\haiku{- Als u ouder wordt,,.}{zei hij met een glimlach zult}{u hem beter zien}\\

\haiku{Het wil de aarde.}{groeten met hart en voetstap}{klinkend in zijn lied}\\

\haiku{Ik zelf was \`op na -.}{den langen middag en ik}{was slechts toeschouwer}\\

\haiku{{\textquoteright} Onder dit alles:}{heeft Paap in De Spectator}{laten afdrukken}\\

\haiku{Dat begin lag al.}{betrekkelijk vooraan in}{zijn schrijversbestaan}\\

\haiku{{\textquoteleft}Uw satire is, ().}{amusant pittig endit is}{hoofdzaak ze is waar{\textquoteright}.56}\\

\haiku{hoe heet het over de:}{geweigerde bijdragen}{moet zijn toegegaan}\\

\haiku{Naar de inhoud van.}{de motie zal men echter}{tevergeefs zoeken}\\

\haiku{Daarenboven dreef.}{Paaps aard hem als vanzelf in}{de oppositie}\\

\haiku{Het onderwerp noch,.}{de omvang was bijzonder}{de t\'o\'on echter wel}\\

\haiku{Paaps relaties met;}{de Douwes Dekkers brachten}{hem ook voordeel aan}\\

\haiku{de kleuren zijn juist,.}{iets te hel de toon juist iets}{te nadrukkelijk}\\

\haiku{Na zijn verblijf in.}{Duitschland kwam Paap weer in}{Amsterdam terug}\\

\haiku{Zijn werkzaamheden:}{beperkten zich zuiver tot}{vermogensbeheer}\\

\haiku{En dan ten slotte.}{moet men hem vermelden als}{het paard van Troje}\\

\haiku{Ach, dat zijn gouden,.}{droomen Uit een voorbijen}{half vergeten tijd}\\

\haiku{{\textquoteleft}vergeven zijn uw,,{\textquoteright}.}{zonden Vrouwe omdat ge}{veel hebt liefgehad}\\

\haiku{Nu haar oogen niet meer,:}{voor mij bestonden bestond}{zijzelf ook niet meer}\\

\haiku{En ik heb toch niets,.}{dan de liefde der menschen}{verlangd snikte zij}\\

\haiku{Gedreven in mijn.}{voortgang beklom ik moeizaam}{trede na trede}\\

\haiku{De angst en de haat,,.}{die ik eens heb gekend zijn}{van mij geweken}\\

\haiku{alleen het rood is,,!}{man wees dan een zuster doch}{wees wat je kan}\\

\haiku{laat mij dan maar de,!}{Donau zijn opdat de Theiss}{kan binnenstroomen}\\

\haiku{Hoe kan het zijn dat,,?}{wat jou koelte brengt mij tot}{het merg verzengt}\\

\haiku{Zij liep nu door de,.}{volksbuurt die tusschen haar en}{het station lag}\\

\haiku{zij waren alle,.}{drie precies op tijd niet te}{laat en niet te vroeg}\\

\haiku{De muur wies hoog, ik.}{kan niet langer blikken in}{uw lokkend oog}\\

\haiku{Zoo is het, dat ik,.}{niet meer zie de deernen schoon}{fraai van gewrichten}\\

\haiku{Boven haar schoorsteen,,.}{stijgt rook van pek Netel en}{doorn distel en brem}\\

\haiku{Uit verveling en.}{oude weerzin wordt het er}{bijna gezongen}\\

\haiku{Ik blijf hier slapen!}{of laat me dragen weg door}{andre knapen}\\

\haiku{en blonde treintje.}{weten het proza van den}{middag met zijn eten}\\

\haiku{- wie garandeerde,?}{me dat hij geen verdere}{afwijkingen had}\\

\haiku{Het is alleen maar,.}{de pijn daardoor scheiden mijn}{traanklieren vocht af}\\

\haiku{{\textquoteleft}Het spijt me meer dan.}{ik zeggen kan dat je me}{geen gezelschap houdt}\\

\haiku{Er was weinig groot,.}{nieuws in die dagen en de}{beurt was aan Polen}\\

\haiku{Ik lag op een paar;}{planken in een hoek van een}{betonnen kubus}\\

\haiku{Ik was in mijn cel, -;}{volkomen ge{\"\i}soleerd}{buitengewoon stil}\\

\haiku{ik probeerde er,.}{vormpjes van te maken maar}{alles stortte in}\\

\haiku{{\textquoteright} {\textquoteleft}Duivels waren het{\textquoteright},.}{zei ik met een machtelooze}{verontwaardiging}\\

\haiku{Maar hoe ik 't ook -.}{bekijk zout kan ik van mijn}{leven niet meer zien}\\

\haiku{Is niet reeds mijn naam, '?}{te veel dit seizoen vant}{nameloze blad}\\

\haiku{Het belangrijkste.}{stuk van die rivieren ligt}{in het buitenland}\\

\haiku{Het masker van den,:}{tijd kan niemand afleggen}{of anders gezegd}\\

\haiku{De werkelijkheid......}{overmeesteren door haar te}{vereenvoudigen}\\

\haiku{En soms scheen het, dat.}{dit centrum zijn bindende}{kracht geheel verloor}\\

\haiku{Eindelijk werd hij.}{moe en hij wachtte tot zijn}{moeder terugkwam}\\

\haiku{Er volgde haar iets,,.}{een vage gestalte maar}{ze merkte het niet}\\

\haiku{Hij was er amper,.}{of Geurtje kwam kijken of}{hij er al in lag}\\

\haiku{De tuinman draaide.}{de handkar om en zette}{zich in beweging}\\

\haiku{Niets is er van 't '.}{stoffig experiment Aan}{t beeld gebleven}\\

\haiku{misschien, want {\textquoteleft}Huisjes{\textquoteright},,.}{van Kaarten haar vijfde is}{zeker niet minder}\\

\haiku{Maar nu was het, als.}{waneer je je handen met}{sneeuw hebt gewasschen}\\

\haiku{Hij meent het lied te.}{horen dat eens het droompaard}{mende op zijn reis}\\

\haiku{Ik heb u niet zo.}{lief gehad als ik u had}{moeten lief hebben}\\

\haiku{Dan alleen nog maar,:}{uw gezicht zoals zij het}{hebben doen worden}\\

\haiku{Wij echter weten.}{geen toverspreuk om onze}{dromen te bannen}\\

\haiku{Hij was de nazaat.}{van een oud doch onverzwakt}{geslacht van strijders}\\

\haiku{Hoe vol de wereld -.}{om hem heen was hij had het}{nimmer ervaren}\\

\haiku{Hij sprong als een schicht,.}{over het lage hek Fanje}{rechtstreeks naar de keel}\\

\haiku{de bloemen geurden,;}{reeds bitter de nazomer}{was al gekomen}\\

\haiku{Dat soort van held was,....}{Frits en hij zat vol kleine}{wrok om klein onrecht}\\

\subsection{Uit: Ad Interim. Jaargang 4}

\haiku{[Ad interim, 1947,]}{nummer 1 De dikke en}{de dunne muze}\\

\haiku{Daar komt nu een stroom (?}{los van geschriftenzijn het}{nog wel geschriften}\\

\haiku{Men blijft er gezond,,.}{bij de maag blijft graag het hart}{verliefd en dorstig}\\

\haiku{Waarom het in het,;}{bijzonder vaandrigs moesten zijn}{is mij niet bekend}\\

\haiku{je hebt me dingen.}{over het vliegfeest verteld die}{me onjuist lijken}\\

\haiku{Men luistert, het oor:}{aan de grond en wordt langzaam}{maar zeker gewis}\\

\haiku{Ik liep langzaam en.}{ik spande mij in of ik}{zware zakken droeg}\\

\haiku{{\textquoteleft}Ze heb zich wat te{\textquoteright},:}{goed gedaan en de mevrouw}{beaamde nog eens}\\

\haiku{De lijkwagen reed.}{juist den hoek van het parkje om}{toen wij instapten}\\

\haiku{Het begon reeds te.}{schemeren en voor zessen}{moesten wij weer thuis zijn}\\

\haiku{Maar ik achtte hem,.}{er des te meer om dat hij}{geen officier was}\\

\haiku{Het dak, van roode,,.}{broze pannen brokkelde}{naar boven toe af}\\

\haiku{Het water klotste.}{zacht tegen de onderste}{treden van de straat}\\

\haiku{Ik wilde niet dat}{zij daar naar keek en er even}{angstig van worden}\\

\haiku{Ja vele, vele{\textquoteright},, {\textquoteleft}.}{zei ze\'e\'en wil mij koopen}{als mij zoo niet geef}\\

\haiku{Maar ik lette niet.}{op haar en sloeg haar met mijn}{sabel een hand af}\\

\haiku{(Mary Carmichael).}{af Ik zou wel wenschen dat}{de maand Voorbij waar}\\

\haiku{Mij dunkt hij zoude,,,.}{koning moeten zijn En God}{weet ik tevreden}\\

\haiku{Dan vluchtte ze weg,.}{over de kleine bosweide}{het kreupelhout in}\\

\haiku{Gerrit Achterberg}{Extemporeetje ~ in}{vino veritas}\\

\haiku{Maar geld of geen geld,.}{er moet heel wat gebeuren}{als me dat niet lukt}\\

\haiku{Toch impliceert de.}{dynamiek van den geest de}{discontinuiteit}\\

\haiku{Ook in dezen tijd.}{merken wij soortgelijke}{verschijnselen op}\\

\haiku{Hoe kan de critiek?}{anders zijn dan de geest van}{de letterkunde}\\

\haiku{Ik sta daar met de.}{lucht te spreken en weet niet}{wat er achter staat}\\

\haiku{In de deuropening.}{draait hij zich een halve slag}{om en legt aan}\\

\haiku{De bruid gaat nu met:}{de afgeschoten roos op}{haar handen verder}\\

\haiku{Een lichte ruk aan.}{het gordijn en geen spier licht}{kan meer naar buiten}\\

\haiku{Op het parket ligt,.}{hij half gekromd tussen de}{brokstukken marmer}\\

\haiku{Maar na een poosje,.}{ga ik toch bemerken dat}{het heel warmpjes is}\\

\haiku{'t Is alles een,,......}{pot nat of ik nu vroeg of}{laat of laat of vroeg}\\

\haiku{Frans Coenen was het {\textquoteleft}{\textquoteright}.}{tegendeel van wat meneen}{geliefd schrijver noemt}\\

\haiku{Het heeft alles van.}{een practische leefregel}{en het is dat ook}\\

\haiku{het zwemmen in het:}{gevoel en het spelen met}{de doodsgedachte}\\

\haiku{{\textquoteleft}nog maar eentje en{\textquoteright},......}{dan niet meer en zoo voort tot}{de trommel leeg was}\\

\haiku{{\textquoteleft}il faut juger les'{\textquoteright}.}{collections d apr\`es leur}{collectionneurs}\\

\haiku{weinig meer dan een.}{naam en een aanleiding tot}{wat chauvinisme}\\

\haiku{laat hen nog maar wat.}{genieten en laat ons niet}{sikkeneurig zijn}\\

\haiku{Wij zagen hem, en.}{het was verkwikkend voor het}{Vaderlandsche Hart}\\

\haiku{van dat kleine Ik,,,?}{dat in zich zelf besloten}{geen kosmos meer kent}\\

\haiku{En nog veel vroeger -?}{al wist zij getuigt daarvan}{niet haar boek Heleen}\\

\haiku{Zondagsrust (roman,,);}{gevolgd door Bezwaarlijke}{Liefde novelle}\\

\haiku{De witte duiven,,';}{trekkebekken Warm in de}{warmte van de Lent}\\

\haiku{De philister{\textquoteright}, met.}{de naam des docerenden}{doctors erbij}\\

\haiku{en in December.}{1898 verscheen er een vijftal}{in De Nieuwe Gids}\\

\haiku{De kamerwanden,.}{zullen niets omsluiten Dat}{warm en levend is}\\

\haiku{De kamerwanden,.}{zullen niets omsluiten Dat}{warm en levend is}\\

\haiku{- Weet je, mijn beste,}{ik heb je zo\"even toen}{meneer binnen kwam}\\

\haiku{Hij heeft zijn beenen om}{zijn nek heen gevouwen en}{draait rond op \'e\'en hand.}\\

\haiku{De vorige maal:}{was hij er ook en ik heb}{gezegd tegen Fred}\\

\haiku{Blijft hij een nacht over,.}{dan zitten we urenlang in}{de hoek van een bar}\\

\haiku{{\textquoteleft}Het is gek,{\textquoteright} zegt hij, {\textquoteleft}.}{nog nooit heb ik mij zoo thuis}{gevoeld als bij jou}\\

\haiku{{\textquoteright} Mevrouw Vroom liet zich.}{met een zucht van aandoening}{weer neer in haar stoel}\\

\haiku{Want kostbaar is des;}{dichters uur Als de Engel}{van den Dood verschijnt}\\

\haiku{Mijn lief, gij hoort geen,,;}{zingen meer Geen zucht geen kreet}{kan u bereiken}\\

\haiku{Ik leef in u een:}{ander leven Dan ooit het}{lot mij heeft bereid}\\

\haiku{En in uw wezen,;}{wordt verweven Al wat mijn}{hart dorstend belijdt}\\

\haiku{er was er geen, die.}{niet verging van angst of die}{mij niet ontvlood}\\

\haiku{Als het elf uur is,:}{schuift de majoor zijn stoel naar}{achteren en zegt}\\

\haiku{Morgen, als het goed,,.}{weer is een eindje om naar}{de Abdij misschien}\\

\haiku{Kort daarna, op haar, '.}{vrijen dag kwam zijs avonds}{met hem laat naar huis}\\

\haiku{Maar wat is hij nu,,;}{mooi want hij staat in het licht}{dat hem betooverd heeft}\\

\haiku{Wist jij toen dat een?}{lied zo moe kon worden om}{wat wij verzwegen}\\

\haiku{De vader sprak, in,:}{dezelfde beurs tastend tot}{de onthutste vrouw}\\

\haiku{aan de eeuw van het,.}{wiegetouw het slaapliedje}{en de sluimerrol}\\

\haiku{Ik volstond ermee.}{je ten overstaan van haar een}{deugniet te noemen}\\

\haiku{Wat ben ik jong, dacht,,}{ik ik ben absoluut geen}{mannelijk ridder}\\

\haiku{Vreeselijk, zei ik,.}{hardop tegen mezelf wat}{ben ik nog een kind}\\

\haiku{Neem mij echter niet,,{\textquoteright}.}{kwalijk ch\'eri als ik daar}{een beetje om lach}\\

\haiku{voor de wegen het,.}{landelijk aspect kregen}{waar ik naar haakte}\\

\haiku{Onder mijn alweer.}{verlangenden blik viel zij}{in een diepen slaap}\\

\haiku{Ik weet alleen, dat.}{ik wakker was en mij in}{het praalbed bevond}\\

\haiku{Maar plotseling zag.}{ik scherper toe terwijl ik}{me over haar heen boog}\\

\haiku{Naast mij in het hooge,.}{statige praalbed lag een}{verschrompelde vrouw}\\

\haiku{Het hoofd leek uit een,;}{vochtige groezelige}{knolraap gesneden}\\

\haiku{het was bultig en,.}{bruingeel hier en daar staken}{lange sprieten uit}\\

\haiku{Ik voelde echter,.}{dat ik den aanblik niet lang}{meer kon verdragen}\\

\haiku{Gezichten kwamen,;}{\'op zich dringen al dicher}{in ontruste slaap}\\

\haiku{toen elk, vol walging,.}{hem verliet bleef ik me bij}{het lijk verbazen}\\

\haiku{Ik had dat lichaam,,;}{willen stelen te zien wat}{verder zich voltrok}\\

\haiku{Dit mengde zich in;}{la\^atre dromen met Frieslands}{wateren dooreen}\\

\haiku{zo'n dubbel landschap,,.}{was er geen domein om niet}{doorheen te komen}\\

\haiku{Men moet van dichters,;}{wel iets weten om alle}{verzen te verstaan}\\

\haiku{de kennis van het,,.}{goed en kwaad wil n\`og als vrucht}{niet zijn gegeten}\\

\haiku{wie sluipt van moeder,,.}{naar den zoon die weet precies}{wat plank er kraakt}\\

\haiku{Ik keer weer naar mijn.}{stoel terug en blader in}{de album foto's}\\

\haiku{Zij is pittig, ter,.}{zake vol verrassingen}{en zeer natuurlijk}\\

\haiku{pessimisme \`en.}{een bepaalde mate van}{blijder levenskijk}\\

\haiku{De dichter Verwey,;}{bijdrage tot het verstaan}{van zijn po\"ezie}\\

\haiku{verrassend dan in.}{zoover als er ter wereld nog}{iets verrassend is}\\

\haiku{Dit is eigenlijk,}{een zeldzaam gevoel want hoe}{vaak twijfelt men niet}\\

\haiku{De moeder en het,,.}{dochtertje herkent men vaak}{soms apart soms samen}\\

\haiku{Tot ik Caldwell las.}{heb ik gedacht dat hij me}{wilde choqueren}\\

\haiku{The Pocket Book of,.}{Popular Verse edited}{bij Ted Malone}\\

\haiku{Doch neen, ik zou het.}{gras voor de voeten van den}{heer J. wegmaaien}\\

\haiku{Wij vervingen het;}{visioen voor logische}{gevolgtrekkingen}\\

\haiku{maar de axioma's.}{van Euclides kunnen niet}{bewezen worden}\\

\haiku{In dat kamp leidden.}{wij het tamelijk grijze}{gijzelaarsbestaan}\\

\haiku{maar bovendien helpt.}{dit feit de recensent over}{zijn eerste schroom heen}\\

\haiku{Geen Nederlander,;}{die daarover niet een duit in}{het zakje wil doen}\\

\haiku{Meester Rembrandt, N.V. {\textquoteleft}{\textquoteright},.}{Uitgevers-Maatschappij}{Kosmos Amsterdam}\\

\haiku{zij betreuren een.}{beangstigend gemis aan}{oorspronkelijkheid}\\

\haiku{Het gemiddelde,,}{is hoog inderdaad doch kent}{wezenlijke kunst}\\

\haiku{Alleen al daarom.}{heeft Van Leeuwen met deze}{publicatie m.i}\\

\haiku{Siodmak doet niet.}{onder voor Hitchcock in de}{thriller-klasse}\\

\haiku{Het meisje laat een:}{stok tegen de spijlen van}{een hek ratelen}\\

\haiku{Schilderijen van.}{Milly van Duivenboden}{en Jos Vi\"ester}\\

\haiku{Een uniek overzicht, dat.}{men niet moet verzuimen te}{gaan genieten}\\

\haiku{En dat mevrouw Kloos.}{uiteraard het best van ons}{allen heeft gekend}\\

\haiku{Aldus ongeveer.}{is het ook met het boek van}{mevrouw Kloos gesteld}\\

\haiku{{\textquoteright}, omdat je toch wat,.}{zeggen moet zo nu en dan}{en luisteren weer}\\

\haiku{Dat ik hier en daar, -}{een vraagteken plaatste doet aan}{deze lof niets af}\\

\subsection{Uit: Ad Interim. Jaargang 5}

\haiku{Er is geen reiken,.}{naar de morgen waarbinnen}{gij verloren zijt}\\

\haiku{Iemand rent kermend.}{de  trap op en bonkt bij}{Mossel op de deur}\\

\haiku{{\textquoteleft}Gij moogt den doolhof,{\textquoteright}.}{nog niet uit Gij hebt nog niet}{genoeg gemind}\\

\haiku{Ik weet niet of ik}{nooit iets met bladeren te}{maken heb gehad}\\

\haiku{Doch desondanks heb,.}{ik nooit armoede gekend}{ook in mijn jeugd niet}\\

\haiku{Als ganzen liepen,.}{zij achter elkaar naar de}{deur die op slot was}\\

\haiku{{\textquoteleft}Het huis zult u wel,.}{krijgen maakt U zich daar maar}{niet ongerust over}\\

\haiku{Ze stond op en nam.}{uit een kast een boek in een}{groen verschoten kaft}\\

\haiku{Zij bladerde het.}{door en legde haar vinger}{op een bladzijde}\\

\haiku{Ik antwoordde niet,,,.}{maar ging hem voor twee trappen}{op naar de zolder}\\

\haiku{Ik bewaarde mijn,.}{nuchterheid ook in deze}{omstandigheden}\\

\haiku{prima adressen, een.}{jarenlange ervaring}{en lage prijzen}\\

\haiku{Natuurlijk wist hij,}{van dat huis waarvan wisten}{deze mensen niet}\\

\haiku{Toen ik bij haar was,.}{hield zij even met het kloppen}{op en keek mij aan}\\

\haiku{De torenvalk scheert,}{over veld en sloten aan en}{waar hij even wiekelt}\\

\haiku{{\textquoteleft}uit pi\"eteit voor{\textquoteright}.}{den overledene zoals}{ze het motief noemt}\\

\haiku{Nog zijn wij slechts tot:}{spel bij machte en niets duurt}{korter dan een lach}\\

\haiku{Doch loop niet sneller,}{door dit veld en wijk niet uit}{voor tere bloemen}\\

\haiku{de vogel valt, hoe,...}{ver hij zweefde wij sterven}{iedere dag}\\

\haiku{Hij maakte weer zijn,.}{wijde armgebaren zijn}{neus snoof weer luid}\\

\haiku{Ik herkende het,.}{woord ik had het onlangs bij}{Nietzsche gevonden}\\

\haiku{Ach ja, iedereen, '.}{ziet het maar niemand zegt het}{m in z'n gezicht}\\

\haiku{Er moest een bijna.}{ondragelijke wanhoop}{in haar leven zijn}\\

\haiku{Ik heb het gevoel,.}{dat de wereld volkomen}{vastgelopen is}\\

\haiku{Opeens greep hij mij.}{bij de schouder en keek mij}{met felle ogen aan}\\

\haiku{Ik weet... ik weet het... -,.}{zelf niet Als m'n vrouw het maar}{begreep ging hij voort}\\

\haiku{Ik liep zelf naar de,.}{huisdeur door een wonderlijk}{voorgevoel bezield}\\

\haiku{Wij hingen bijna.}{achterover tegen de wind}{om niet te rennen}\\

\haiku{Had hij Oudejaar,?}{gevierd of kwam hij pas thuis}{van een spoedgeval}\\

\haiku{'t Is misdadig.}{dat ze hun licht onder een}{korenmaat zetten}\\

\haiku{En in den Haag weet, '!}{men het en werkt me tegen}{int nasporen}\\

\haiku{Mainz 6 Juni 70,.}{Geachte heer Tersteeg}{Dank voor de ontv}\\

\haiku{Mainz 6 Juni 70,.}{Geachte heer Tersteeg}{Zooeven bragt ik br}\\

\haiku{Om bloed dat jaagt naar,:}{zijn verderven De nacht zingt}{ijler dan een riet}\\

\haiku{geen vogel, geen wolk,.}{geen dans van muggen aan den}{rossigen einder}\\

\haiku{Een doordringende.}{klank van metaal ruist in de}{mond van de morgen}\\

\haiku{Het wijfje knaagde.}{hardnekkig aan de wortel}{van de rozenstruik}\\

\haiku{Soms weet men mij voor,:}{iets te winnen Soms weet ik}{wat men van mij wil}\\

\haiku{{\textquoteright}, waarop zij zich met.}{een berustende grijns op}{een stoel liet vallen}\\

\haiku{Terwijl Eddie zich,:}{naar voren drong werd er van}{boven geroepen}\\

\haiku{Toen hij overlas wat,.}{hij geschreven had kon hij}{zijn oogen niet gelooven}\\

\haiku{Ik weet, dat het nu,.}{beginnen gaat maar het kan}{mij niets verdommen}\\

\haiku{Over het grondloos Niet}{zal hij verheerlijkt drijven}{in het onstilbaar}\\

\haiku{Zij sloegen met een.}{hakbijl en platte messen}{op een houten blok}\\

\haiku{Drie maal belde ik.}{ook aan en eveneens drie maal}{kreeg ik geen gehoor}\\

\haiku{Nou, dat is wel wat,!}{laat want zij is al meer dan}{een jaar gescheiden}\\

\haiku{Iedereen weet dat,.}{zij gescheiden is ook al}{zegt zij zelf van niet}\\

\haiku{{\textquoteright} Zij rookte nerveus,.}{ik had haar trouwens nog nooit}{eerder zien roken}\\

\haiku{het kind lag op 't,.}{aanbeeld te slapen moe van}{z'n manespel}\\

\haiku{Daarna keerde ik.}{aandachtig terug naar de}{meer besloten kom}\\

\haiku{Maar laat h\`en slape',,,:}{en m{\`\i}j Heer in dees landen}{Een wijl nog rouwen}\\

\haiku{en dat zijn blik hier.}{derft Het zicht op streken waar}{hij straks ontwaakt}\\

\haiku{door kloven zag men ', '}{t dal In ijle verte}{ent gele duin}\\

\haiku{Zeer sterk komt dit tot.}{uiting op het gebied van}{de literatuur}\\

\haiku{Wat men wel te zien:}{kreeg was een vernieuwing in}{verticale zin}\\

\haiku{nu richtte zij zich,.}{op de eigen wereld de}{kosmos van het Ik}\\

\haiku{Kwaad is de wind en,,......}{bitter de zee en grijs is}{de hemel grijs grijs}\\

\haiku{This is the day his.}{grief will be remembered}{By all who grieve}\\

\haiku{But there's deer in,.}{the orchard My apples are}{yours for the picking}\\

\haiku{En in de kiezels.}{van het Heilig water klonk}{de sabbath zacht}\\

\haiku{Daar zit ik met mijn.}{hand voor mijn hoofd en denk aan}{wat ik heb geloofd}\\

\haiku{De waarde van het.}{beeldhouwen berust op de}{moeite die het kost}\\

\haiku{ze staan al achter '.}{opt balcon Dat langzaam}{in de bocht verdwijnt}\\

\haiku{Niets anders rest hun.}{dan de troost van woorden naakt}{en zinneloos}\\

\haiku{{\textquoteright} {\textquoteleft}Best{\textquoteright}, zei mijn vader, {\textquoteleft},.}{opgewektheel best. Kijk ik}{heb mijn handen vrij}\\

\haiku{{\textquoteleft}Zij zullen het niet{\textquoteright},, {\textquoteleft}}{slecht hebben op deze reis}{zei iemand naast mij}\\

\haiku{Het middelste van.}{die vertrekken had hij tot}{eetkamer bestemd}\\

\haiku{Ik liep de gang door.}{naar de keuken en vandaar}{naar de voorkamer}\\

\haiku{Nadat hij dat voor,.}{de zesde maal had gezegd}{kwam hij niet terug}\\

\haiku{Ik herinnerde.}{mij Gerard Lutmer op dat}{oogenblik heel goed}\\

\haiku{Jij bent het geweest,.}{tegen wie ik mijn eerste}{leugen heb gezegd}\\

\haiku{Arnold was geheel,.}{anders dat zag zij alleen}{reeds aan hun handen}\\

\haiku{Hij zuchtte diep maar.}{een scheurende pijn deed hem}{in elkaar krimpen}\\

\haiku{hij kon nog juist zijn,.}{schoenen uitschoppen toen gleed}{hij in diepen slaap}\\

\haiku{De dokter trad naar.}{voren en legde zijn hand}{op Martyns schouder}\\

\haiku{Hij wilde een klein;}{schip maken en nu was hij}{bezig aan den romp}\\

\haiku{De koude er in.}{voelde aan als messcherpe}{kanten en hoeken}\\

\haiku{Ik knip je de oren...}{van het hoofd en lepel je}{ogen op een bordje}\\

\haiku{Harry zou hem half...}{vermoorden als hij er nu}{vanavond niet heen ging}\\

\haiku{De litteratuur.}{zal dat vermoedelijk niet}{veel kunnen schelen}\\

\haiku{Zijn gedichten en;}{essays bracht hij onder in}{een eigen domein}\\

\haiku{Over zijn leven en.}{publicaties enkele}{korte notities}\\

\haiku{Ieder van hen en.}{elk van hun idee\"en moeten}{wij verafschuwen}\\

\haiku{{\textquoteleft}Neem aan, dat deze:}{onbekende soldaat u}{zou willen vragen}\\

\haiku{Als men daaruit een,:}{conclusie trekken mag is}{het toch wel deze}\\

\haiku{{\textquoteleft}Wij trekken uit de /:}{aardsche schatten een drang die}{immer zich vergroot}\\

\haiku{Men vraagt zich af, wat.}{van deze film eigenlijk}{de bedoeling is}\\

\haiku{Ik moest denken aan:}{een strophe uit een zijner}{sterkste gedichten}\\

\haiku{Een vraag, die als een.}{kwellend teken tussen hem}{en mij in bleef staan}\\

\haiku{Dat het zich tot het,.}{einde geboeid lezen laat}{is onverkort lof}\\

\haiku{sympathiek door een.}{volkomen gebrek aan ernst}{en prekerigheid}\\

\haiku{Die winde dwarrel...}{eers en swaai dat die stowwe}{helder blink en draai}\\

\haiku{sy gang was van die,.}{wye westewind sy bruine}{o\"e blink en bly}\\

\haiku{{\textquoteleft}Die hoed had ik 's,{\textquoteright};}{middags gekocht omdat ik}{me treurig voelde}\\

\haiku{Voor het charmante.}{stofomslag van G. Douwe}{een apart compliment}\\

\haiku{Het is werk om veel,.}{op te slaan en dus om in}{zijn kast te hebben}\\

\haiku{de terugtocht van;}{twee kameraden uit de}{hel rond Duinkerken}\\

\haiku{In dit tweeledig:}{debuut schijnt een dominee}{aan het woord te zijn}\\

\subsection{Uit: Ad Interim. Jaargang 6}

\haiku{Want het gaat hier om.}{het belangrijk onderscheid}{tussen ginds en hier}\\

\haiku{de aandacht voor het,,.}{detail werkelijkheidszin}{gevoel voor humor}\\

\haiku{de merel stort zich.}{met een kreet vol wildheid in}{de voorjaarsvlagen}\\

\haiku{Ik voelde dat ook.}{haar handen klam waren van}{mist die regen werd}\\

\haiku{En daarna greep haar.}{hand feilloos een deurknop en}{deed ze een deur open}\\

\haiku{Luister, zei ze, en}{ze drukte me daarbij in}{de enorme stapel}\\

\haiku{{\textquoteleft}als de inhoud van,{\textquoteright};}{een zachte ziel waarop te}{lang de leeddrop viel}\\

\haiku{ik houd veel van kunst,,.}{maar alleen van oude niet}{van de moderne}\\

\haiku{En dat was waarlijk,.}{lang niet altijd het geval}{denk maar aan Rembrandt}\\

\haiku{Dan bedenk ik wel '.}{n strik voor balsturige}{demagogen}\\

\haiku{Periclidas, die!}{namens geheel Sparta om}{hulp kwam vragen}\\

\haiku{Naast de vrije pachters '.}{staan de mannen vant vrije}{beroep in de stad}\\

\haiku{Het is alsof het.}{donkere zaakje er van}{uitgebeten is}\\

\haiku{Zo onverhoeds was,.}{het krachtige licht dat ik}{eerst de fles niet zag}\\

\haiku{Maar zijn zichtbare.}{trots dwong hem toch ons voor te}{gaan naar de kelder}\\

\haiku{Ik rustte eerst, toen.}{hij zijn geheim volledig}{had prijs gegeven}\\

\haiku{Het was niet de lak,.}{maar mijn buikstreek die het had}{moeten ontgelden}\\

\haiku{Ook mijn scheutje olie.}{glibberde dus knetterend}{over het pannetje}\\

\haiku{Ik liet hem door de.}{buffels regelrecht naar een}{verfspuiter slepen}\\

\haiku{{\textquoteright} {\textquoteleft}De volgende week,{\textquoteright}, {\textquoteleft}.}{zei hijzult u eenmaal per}{dag gelucht worden}\\

\haiku{{\textquoteleft}Het spijt mij wel,{\textquoteright} zei, {\textquoteleft}.}{hijmaar dan moet ik eerst eens}{beter gaan vragen}\\

\haiku{{\textquoteright} zei ik, {\textquoteleft}als wij nu,.}{deze kroeg in gaan dan moet}{je goed opletten}\\

\haiku{En wij hebben het?}{toch niet over wissewasjes}{of vind jij soms wel}\\

\haiku{{\textquoteright} zei hij, {\textquoteleft}ja, laten.}{wij maar zeggen dat het een}{figuurtje voorstelt}\\

\haiku{Maar ik zou er nog.}{graag een beetje zonneschijn}{in willen hebben}\\

\haiku{Wanneer ik schrijf, zei,.}{een romancier dan ben ik}{het zelf niet die schrijf}\\

\haiku{Aan mijn dochter  :}{Bert Voeten  Als men je}{later zal zeggen}\\

\haiku{Twee tegen een en.}{die ander staat er toch maar}{alleen tegenover}\\

\haiku{Toch liever doodziek.}{op een eenzaam eiland dan}{Marge naast mijn bed}\\

\haiku{al had het al in.}{de ochtendbladen gestaan}{nog zonder foto's}\\

\haiku{Nee, Woensdag was het,,.}{waar blijft de tijd alweer drie}{dagen geleden}\\

\haiku{En ik huilen maar,.}{gewoon de ergste nervous}{breakdown sinds Daddy}\\

\haiku{Hij boezemde mij,.}{geen angst in maar weerzin en}{zwijgende wanhoop}\\

\haiku{Tegenover haar zat,}{een heel dikke man van wie}{ik van onderaf}\\

\haiku{De lange hals lag.}{bij beiden verdord tussen}{gestrekte pezen}\\

\haiku{{\textquoteleft}Nonsens,{\textquoteright} weerlegde,.}{kort haar overbuur die tot nog}{toe gezwegen had}\\

\haiku{Heb even geduld, mijn.}{tong moet losraken  van}{mijn gehemelte}\\

\haiku{{\textquoteright} vroeg ze even later.}{toen ze hem bezig zag toch}{een krans te vlechten}\\

\haiku{Wanneer ik dacht aan,:}{de oproep kon ik maar \'e\'en}{verklaring vinden}\\

\haiku{ik zou alleen maar,,.}{nogmaals de toevalligheid}{kunnen vaststellen}\\

\haiku{{\textquoteright} vroeg ik een kellner,.}{die juist een volgeladen}{blad op zijn arm nam}\\

\haiku{{\textquoteright} En kennelijk blij,,.}{van mij verlost te zijn schoot}{hij zijn hokje in}\\

\haiku{Na een sluimering.}{van een kwartier was het tijd}{om uit bed te gaan}\\

\haiku{Trouwens, het is niet,,?}{eens echt haar het is een soort}{schimmel begrijpt u}\\

\haiku{{\textquoteleft}Een ding nog, Bertrand,,!}{kom doe der wat an gooi even}{mijn lijk uit het raam}\\

\haiku{{\textquoteleft}Reinig ons van het,{\textquoteright}.}{kwaad Als de Tabor staat uw}{rechtvaardigheid}\\

\haiku{Hij is geboren.}{met een rijke hoeveelheid}{mogelijkheden}\\

\haiku{Gewiss, ihm geben.}{auch die Jahre Die rechte}{Richtung seiner Kraft}\\

\haiku{Der Unfall lauert.}{an der Seite Und st\"urzt ihn}{in den Arm der Qual}\\

\haiku{- heette het - {\textquoteleft}ist f\"ur.}{die Deutsche Literatur}{ein unersetzlicher}\\

\haiku{Ik zie hem zitten.}{en ik verheug mij over zijn}{tegenwoordigheid}\\

\haiku{Een enkele keer {\textquoteleft}}{veegde hij zijn mond af met}{de rug van zijn hand.}\\

\haiku{{\textquoteright} {\textquoteleft}Het is gek,{\textquoteright} zei hij, {\textquoteleft},.}{maar als jij gaat zoeken dan}{ga ik met je mee}\\

\haiku{{\textquoteright} Ik tastte met mijn.}{ene hand in mijn zak en gaf}{hem een dubbeltje}\\

\haiku{De muzikant bleek:}{inderdaad  verdwenen}{te zijn en ik zei}\\

\haiku{Ik keek mijn buurman,.}{aan die aldoor aandachtig}{naar buiten staarde}\\

\haiku{{\textquoteright} {\textquoteleft}Verrotte appels,{\textquoteright}, {\textquoteleft}.}{zei mijn buurmandie man heeft}{ons rotzo verkocht}\\

\haiku{{\textquoteright} {\textquoteleft}Nee,{\textquoteright} zei mijn buurman, {\textquoteleft}.}{maar daarginder staat iemand}{naar ons te kijken}\\

\haiku{Ook dan bedreigt hem,.}{nog de tijd als iedere}{mens als ieder ding}\\

\haiku{wordt met afkeer niet,;}{vervuld Over ons lot omdat}{wij zwaar misdreven}\\

\haiku{Ik heb mijn moed en,;}{macht verloren mijn vrienden}{en mijn vroolijkheid}\\

\haiku{En zo gij woorden,:}{samenbrengt Niet \'e\'en verdient}{er uw misprijzen}\\

\haiku{de oude vrouwen}{Onder de hemel grauw en}{laag Op het kerkhof}\\

\haiku{Een gepantserde.}{automobiel brengt mij naar}{de gevangenis}\\

\haiku{Zij heeft een blinkend.}{gepoetst zilveren kruisje}{op haar borst hangen}\\

\haiku{Het eerste half uur.}{liep ik rond met een vreemd licht}{gevoel in het hoofd}\\

\haiku{Want zij zijn bezig,.}{de taal te vermoorden waar}{ik nu nog van houd}\\

\haiku{Ik weet, dat ik niet.}{anders praat dan U. Ook U}{bent ontevreden}\\

\haiku{wat hij me schreef ben...}{ik blijkbaar toch een beetje}{van mijn stuk geraakt}\\

\haiku{de uniformpet komt.}{hem niet meer toe en is een}{belachelijk ding}\\

\haiku{Gadegeslagen.}{door zijn zoontje wordt hij op}{heterdaad betrapt}\\

\haiku{Zelfs wanneer man en,.}{kind niet in het beeldvlak zijn}{duurt de beklemming}\\

\section{anoniem}

\subsection{Uit: Van Brabant die excellente cronike}

\haiku{iaren so dachten.}{die goede lieden haer graf}{te verbeteren}\\

\haiku{So antwoorde si}{van binnen ontfunct sijnde}{vande vuere}\\

\haiku{Si ordineerden}{nochtan de tijt wanneer}{men die brulocht houden}\\

\haiku{iaren out sijnde.}{die iongers in arbeyde}{ginck te bouen}\\

\haiku{So dachte hi ooc.}{haer graf te versoeken met}{groter deuocien}\\

\haiku{hem met der caken,.}{ende trecten opt droghe}{twelck si dede}\\

\haiku{so gaf si alle.}{haer goet den armen menschen}{om de minne gods}\\

\haiku{mit abstinentien}{ende een milde gheuer}{van aelmoessenen}\\

\haiku{hy visiteren.}{die plaetsen der heyligen}{in allen landen}\\

\haiku{JNden tijden van}{Godefroot van bullion}{so leefde sinte}\\

\haiku{als hi wt sijnre.}{moeders lichaem geboren}{was Als hi .xvij}\\

\haiku{Dit aenhorende,}{so bleef hi daer ende die}{abdisse ende}\\

\haiku{opten lesten dach.}{van Junio op eenen vrydach}{na dat hi .xxvi}\\

\haiku{Maer die voorseyde}{broeder Aernout sprack hem}{vriendelicken toe}\\

\haiku{Daer na vraechde si,,}{haer na hare vrienden som}{leuende som doot}\\

\haiku{Christi seer sieck so}{datmen haer doot verwachte}{Doe wert daer ghemaect}\\

\haiku{drie iaer lanc so dat}{hi leggen noch slapen en}{conde Ende so}\\

\haiku{Ende so gheuielt}{op een auontstonde dat}{hi denckende}\\

\haiku{wert int cloostere}{te Ewyers in wals Brabant}{Mer daer te voren}\\

\haiku{so vraechde hi den}{man oft hi die heylige}{vrouwe gesien had}\\

\haiku{Als die man ontrent}{een vierendeel van eender}{milen gegaen was}\\

\haiku{Op die selue stont}{soe wert die boose knechte}{diese belogen}\\

\haiku{hi stichte in dat.}{lant en grote stadt die hi}{noomde Sicambrien}\\

\haiku{Vanden genen die}{van hem achtersprake deden}{plach hi te seggen}\\

\haiku{Hi liet na hem twee,,.}{sonen deen heet Brabon dander}{Eneas dat sijn .ij}\\

\haiku{seer ende seyde,}{Jst dat dese dingen van}{god zijn so bid ic}\\

\haiku{Alchuinus leerde.}{coninc karel dat meeste}{deel sijnre consten}\\

\haiku{Ende karel gaf}{wt een ghebot dat alle}{man hem bereyden}\\

\haiku{Ende die gheen die}{ongelouich blijuen wilden}{die werden ghedoot}\\

\haiku{ic ben van vanden}{geslachte der Fransoysen Die}{ruese seide}\\

\haiku{Ende met groten}{gheroepe ende ghecrijsch}{van beyden siden}\\

\haiku{waren gesalft van.}{haren vrienden met mirre}{ende balseme}\\

\haiku{Want op dyen dach}{dat wi te Vyenne van}{malcander scheyden}\\

\haiku{drie iaer gheschiet sijn,.}{te weten dat de sonne}{ende mane .vij}\\

\haiku{Aldus heeft god dan}{tlant van Brabant hoochlic}{verciert met desen}\\

\haiku{vander hemelscher.}{Jerarchyen die sinte}{Dyonijs maecte}\\

\haiku{si gingen hem aen.}{inden name gods ende}{versloegen dit dier}\\

\haiku{Huge vader van.}{Huge capet maerschalc van}{al Vrancrijcke}\\

\haiku{Oede die namaels}{nam te manne Lambrecht graue}{van Bruessel daer}\\

\haiku{gheheten metten.}{baerde des grauen broeder}{van Henegouwe}\\

\haiku{voorseit ~ Jtem in.}{dese tijt werden gedoot}{tot Jherusalem .xi}\\

\haiku{iammerlic verdruct,}{Dies hi grote compassie}{hadde ende oec}\\

\haiku{om te winnen dat}{heylich lant Ende als si}{tot Colen quamen}\\

\haiku{Mer God sandt ouer}{hem al sulcke anxte}{datse die mueren}\\

\haiku{Doe scheyden si haer}{heyr van malcander om te}{bat te passeren}\\

\haiku{Daer na vonden si}{een riuier daer vele}{hem doot droncken}\\

\haiku{voorscreuen want hi}{wiste wel datter niet meer}{volcx en quam wt}\\

\haiku{Als dander kersten}{sagen ende vernamen}{dat die kersten int}\\

\haiku{onsen ende plach.}{niet te hebben dan een mael}{ende een palster}\\

\haiku{Ende Guillelmus.}{van Parijs die ordene}{vanden Agustinen}\\

\haiku{als Benedictus.}{Expugnat  Domine}{in virtute tua}\\

\haiku{Die vierde is de,.}{graue van Vlaenderen dye draecht}{des conincx swaert}\\

\haiku{dat nv Loreynen,.}{heet ende besat also}{beide die landen}\\

\haiku{ia si sloten voor}{hem ende voor sijn moeder}{die poorten vander}\\

\haiku{na so stelden hem.}{al Vlaenderein in handen}{vanden coninck}\\

\haiku{wederstonden sijn,.}{tyrantscap want die coninc dat}{niet en beterde}\\

\haiku{iaren geregneert}{hadt so sterf hi vanden steen}{ter Vueren Jnt}\\

\haiku{So ontboot coninc}{Philips hertoge Jan dat}{hi den voorseyden}\\

\haiku{Ende en quame.}{hi daer niet so souden si}{de stadt behouden}\\

\haiku{So track coninc Eduwaert}{self totten  keyser te}{Couelens Ende}\\

\haiku{iare quam een nieu}{mare totten hertoge}{datter in Sauoyen}\\

\haiku{M. mannen oft meer,.}{rouende al dat hi in}{sijn ghemoete vant}\\

\haiku{ten lesten den strijt.}{verloren daer verslagen}{bleuen omtrent .xl}\\

\haiku{hi sinen bode}{marten tot Perpiniaen aen}{coninc Segemont}\\

\haiku{so quaden raet dat.}{si noch haer dochter niet en}{mochten verhoort sijn}\\

\haiku{oft water datmen}{den gront daer niet van vinden}{en conde so diep}\\

\haiku{oordens, die dat voort}{in allen hoecken ende}{landen des werelts}\\

\haiku{Jan bont cancellier}{van brabant van hertoghe}{Philippus ontset}\\

\haiku{Ende also wert}{den twist ende oorloge}{ghecesseert ende}\\

\haiku{grote scade int.}{lant van ludick wt ende}{inne rijdende}\\

\haiku{Kaerle graue van.}{Charloys hertoge Philips}{soon van Bourgondien}\\

\haiku{ordinerende}{een schone processie om}{den prince met}\\

\haiku{graue van vlaenderen}{ende den ingesetene}{der stede van ghent}\\

\haiku{hem omme metten.}{sinen ende versloegen}{alle die van Ghent}\\

\haiku{sinen pays metten.}{hertoge Eduwaert ende bleef}{voort aen getrouwe}\\

\haiku{Jan van koesteyn die.}{welck een speciael vrient was}{des heren van Croy}\\

\haiku{hi van stonden aen,}{weder te scepe ende}{hier en binnen starf}\\

\haiku{battalien   Den.}{eersten leyde die graue van}{Simpol met sijn .iij}\\

\haiku{Ende als hi dat.}{bisdom in groten rusten}{ende vreden .v}\\

\haiku{wouden namen si.}{op te branden die molen}{van Montenaken}\\

\haiku{sien, ende binnen}{desen tijden worden hem}{ministreert ende}\\

\haiku{Een vre te voren}{dat die Keiserlijcke}{Maiesteyt wt rijden}\\

\haiku{soude, so reet heer.}{Jan van baden aertsbisscop van}{trier wter stadt met .CCCC}\\

\haiku{Daer na quamen twee}{edelen wt oostenrijck}{met bloten armen}\\

\haiku{Daer werden soveel.}{tenten ende husen op}{gestelt datter .iiij}\\

\haiku{so crachtlijken dat}{si van noots wegen mosten}{achter rugge keeren}\\

\haiku{ende commissie}{namen die ambassaten}{ouer te dragen}\\

\haiku{So heeft si dan ten}{lesten haer consent ende}{wille ghegeuen}\\

\haiku{is een die bouen,}{alle de anderen seer}{schoon is ende alst}\\

\haiku{ende als hi dan,.}{verslagen is so hebdi}{den slach gewonnen}\\

\haiku{lancie spranck in, {\textparagraph}}{vij stucken ende reedt doen}{wter banen na thof}\\

\haiku{hoeck wert heel vanden,.}{turcken verwoest dat mijn}{herte seer swaer is}\\

\haiku{strijt voor tkersten bloet,.}{datter ghestort was ende}{die Fransoysen weecken}\\

\haiku{biden heere van}{Beuere Ende die Arien}{ouer gaf ontfinc}\\

\haiku{Te weten heere}{Jan van Lannoy abt sinte}{Bertijns cancellier}\\

\haiku{het wert met wijsheyt}{beslicht datter so groten quaet}{niet en geschiede}\\

\haiku{Doe quam de coninc,}{wt ende vergaf dat hem}{misdaen was ende}\\

\haiku{het wert gemaect dat}{dye van Vlaenderen ende}{dye Roomsche coninc}\\

\haiku{wert beuonden dat}{die procuratie vanden}{here van Polem}\\

\haiku{dit gesciede op,.}{onser vrouwen lichtmis auont}{int iaer .xvC. ende.xij}\\

\haiku{Hoe die coninc van {\textparagraph}}{Vrancrijck oorlochde op}{die Venechianen}\\

\haiku{soude te geschien,}{inden heyligen Roomschen}{rijcx saken daer}\\

\haiku{grotelijc verpeynt}{was Jtem by rade vanden}{keyser Maximiliaen}\\

\haiku{twee heeren dye een.}{steecspel beroepen hadden}{So quam opten .xi}\\

\haiku{De hertoge van.}{Lunenborch vader van mijn}{vrouwe van Gelre}\\

\haiku{maenden daer buten,.}{gehouden hadden daer hi}{wel in was .xlv}\\

\haiku{dat vreeslick schieten.}{dat Ferdinandus dede}{heeft hijt binnen viij}\\

\haiku{van des conincx.}{pagien opt turcx ende}{ander schoon peerden}\\

\haiku{Van binnen waren.}{si verdruct ende verheert}{vanden ghelderschen}\\

\haiku{Jnden seluen}{tijden waren twee vaenkens}{knechten van sbisscopen}\\

\haiku{Aldus bleef des bisscops}{volck al stille tot des}{anderen morgens}\\

\haiku{dit nauolgende}{Jnden eersten hebben si}{gheordineert .xvC.}\\

\haiku{M. peerden ende,.}{.xxM. voetknechten ende C.}{sacken met coren}\\

\haiku{ende quamen ooc}{inne met haer comende}{in ordenen Eerst}\\

\haiku{[Sinte Begge,,]}{hertogin zie hiernaast}{Sinte berlendis}\\

\subsection{Uit: Het boeck vanden pelgherym}

\haiku{pelgrim slaapt met hand;}{onder wang op een helling}{links van een beekje}\\

\haiku{elckerlijc leeren}{wat wech men sculdich is te}{gaen of te laten}\\

\haiku{Dit is licht te doen}{want daer en is geen so rijc}{hy en is wel arm}\\

\haiku{maendenin gheweest}{hadde sonder dat ic ye}{daer wt quam Ende}\\

\haiku{niet ter werelt thans.}{my so van node is als}{vwe bystandicheyt}\\

\haiku{Want ic seg di dat}{hier meer kinderkijns lijden}{dan oude luyden}\\

\haiku{Dit is het eerste}{passagie na ierusalem}{van allengoeden}\\

\haiku{wt eender toerne.}{ende was een maecht ende}{ghinc tot hemluyden}\\

\haiku{Ende dan daer na.}{vinstu hem ouerhorich}{ende niet obedient}\\

\haiku{v gheopenbaert}{ende beleden te zijn}{Besiet nu of ghij}\\

\haiku{gij my zeer misdaen}{hebt ende tegen de loep}{mijnder gewoenten}\\

\haiku{die wel soe suldy}{claerlic bekennen v}{mestersse ende}\\

\haiku{int gros want also}{int gros te biechten en is}{niet anders danmen}\\

\haiku{v verwonderen}{mach ende daer bij zijdijt}{schuldich te weten}\\

\haiku{wt wil doen datmen}{den bessemdaerwaerts keert}{want het waer schande}\\

\haiku{een die geheten}{is nemia dats poorte van}{de vuylichede}\\

\haiku{de x geset is}{Daer is by te verstaen dat}{ic ben geheeten}\\

\haiku{Sapiencia of wijsheit.}{Al datmen visieren mocht}{dat conste zij doen}\\

\haiku{dat die gene die}{dit wambaeys an heeft hi doet}{sijn profijt ende}\\

\haiku{siet hyer daer in.}{dat sinte benedictus}{zweert stac doe hijt droech}\\

\haiku{Dijn secreten als}{hijse weet sal hy seggen}{den philistienen}\\

\haiku{Als ic my aldus.}{sach ontladen soe vloech ic}{terstont in de lucht}\\

\haiku{hem hebben Ende}{dan als hy soe tondere}{ghedaen soude zijn}\\

\haiku{mede du en zijs.}{de eerste pelgrym niet die}{hier gecomen heeft}\\

\haiku{En wil niet meenen.}{dat ic gaen sal weegen die}{men laecken mach}\\

\haiku{Het is de selfde}{stoc die de kaerl in zijn}{hant had hier voeren}\\

\haiku{ende nochtan.}{soe en waen ic niet dat ic}{dan steruen sal}\\

\haiku{men kent die lieden}{niet bi haren clederen}{noch den goeden wijn}\\

\haiku{goeden mensch dan de}{natuerlicste dief die}{ye gheboren was}\\

\haiku{hoe qualic dat my.}{quam In desen bosch daer ic}{of genoept hebbe}\\

\haiku{ende waert also.}{dat icken dienen moeste bi}{anxte vander doot}\\

\haiku{die neerkijkt op de[]}{kerk en het schaakbord\ensuremath{>}}{55va Doe dede}\\

\haiku{dien heb ic een edel:}{instrument ghegeuen om}{keerels werc te doene}\\

\haiku{Ic wachte hem al.}{zijn groot goet zijn gout ende}{zijn siluere}\\

\haiku{daer en is gheene.}{vervulte an noch gheene}{genoechlichede}\\

\haiku{In marcten gaet si.}{die luden vijlende voer}{elckerlijke}\\

\haiku{haeck ende dese}{figuere thoent dat ic}{bin abdisse mair}\\

\haiku{si is mi seere}{nootsakelic om dat ic}{te doene hebbe}\\

\haiku{dies ghelijke daer.}{ick meest ghels zie darwaerts treck}{ic mijne tonge}\\

\haiku{ende noch sulstuut}{bet weten alstu suls horen}{van mijnen monde}\\

\haiku{wat sulstu segghen.}{ketijt nv bistu comen}{by dijnen eynde}\\

\haiku{my vertroest van.}{mijnre droefheden ende}{beschermt vander doot}\\

\haiku{nv sie ic wel dat.}{ghy noch myne siele niet}{vergheten en hebt}\\

\haiku{anden coninck.}{ende settene weder}{inden rechten wech}\\

\haiku{want tot dy comick.}{om te sine ghenesen}{van mijne quetsuere}\\

\haiku{wat peynsende te.}{wat inden ic die hage}{ghelaten hadde}\\

\haiku{Nu coemt voirt ende.}{gheeft mi dine male By}{lode seyde ic}\\

\haiku{dat ware al te.}{starc te doene luden van}{putertieren sinne}\\

\haiku{hebstu eenen [74rb] soe.}{heb ic wel gheuaren want ic}{salre op smeden}\\

\haiku{want te meer schaemten.}{dat een man heeft te meer vint}{hi persecucien}\\

\haiku{ende ghebieden}{hem allen dat si di hier}{in sijn onderdaen}\\

\haiku{want nye mensche en}{setter voet hi en hadder}{te vorengesent}\\

\haiku{si leyden my te}{bedde ende bonden my}{daer in seggende}\\

\haiku{82ra] ricordia om.}{die te treckene wt der}{ketiuicheden}\\

\haiku{Nu segt seide die}{queene maer haest di want elder}{heb ic te doene}\\

\haiku{45 De laatste zes.}{regels van deze kolom}{zijn niet bedrukt}\\

\haiku{De rechterkolom.}{van dit folium is dus}{deels leeg gebleven}\\

\subsection{Uit: Cronyk van Sint Aagten Convent. Een oude kloosterkroniek uit de 15-17e eeuw}

\haiku{Wij zien dit alles.}{weliswaar vanuit een zeer}{beperkt gezichtspunt}\\

\haiku{De Geuzen zijn met.}{behulp van hun vrienden in}{de stad gekomen}\\

\haiku{Zou deze tijd - toch.}{een van de belangrijkste}{uit onze Vaderl}\\

\haiku{De houte geute.}{van de Keucken verbrande}{geheel buyten af}\\

\haiku{Men gaf hem half meer,.}{dan Baden ende Baden}{half meer dan David}\\

\haiku{*~         In dit jaar wierd '.}{de eerste steen geleyt van}{t nieuwe Ziekhuys}\\

\haiku{Dan brande niet meer.}{door de gratie Godts dan de}{cap van de Tooren}\\

\haiku{Daar was soo grooten,.}{elende van honger dat men}{niet schryven en kan}\\

\haiku{Daar was groot gerugt,.}{in de Stadt alsoo de Geusen}{voor de Stadt kwamen}\\

\haiku{Syn Broeder hadt ons,.}{ontboden dat wy hem niet}{inlaaten souden}\\

\haiku{waar op den Heere '.}{Vicariust selve}{heeft geconsenteert}\\

\haiku{we vermeld in een {\textquoteleft}{\textquoteright}.}{Amersfoortse kroniek van een}{naamloze schrijver}\\

\haiku{Zie ook onder 1531 {\textquoteleft}{\textquoteright}.}{waar over ditvoor de traelje}{gaan gesproken wordt.1426St}\\

\haiku{Nu heeft iedere,.}{bisschop een rechtstitel een}{rechtsgebied nodig}\\

\haiku{Hiervan kunnen we.}{ons op twee manieren een}{denkbeeld vormen}\\

\haiku{Theoretisch is,.}{dit ook nu nog mogelijk}{zelfs met een paus}\\

\haiku{het vruchtgebruik van '.}{het goed hebben en geen macht}{overt goed zelf}\\

\haiku{b. Welke functies?}{van de zusters worden in}{de kroniek genoemd}\\

\haiku{Wat betekent het {\textquoteleft}{\textquoteright},?}{woordontboden als je het}{zinsverband nagaat}\\

\haiku{Men kan op de kaart.}{precies nagaan hoe de brand}{over de stad heen sloeg}\\

\subsection{Uit: Handschrift Hattem C 5}

\haiku{Ist zake dat hij}{zijn rike   voeght ende}{regiert na de wet}\\

\haiku{Zij worden daer wel}{ontfanghen ende zij}{hebben daer goet}\\

\haiku{wort een co-}{ninc ontzien ende bedwingt}{zijn vianden}\\

\haiku{v machtich zijn   {\textparagraph}}{te cranckene Ooc en}{zuldi niet uwee}\\

\haiku{wair omme dat de}{daghen cort ende lanc}{worden Sij kende}\\

\haiku{hoe vele heeren}{hebben   alzoo fenijn}{ghedroncken}\\

\haiku{Nu merct wederomme.}{met wat   zaicken dat de}{mensche magher wort}\\

\haiku{regiert dat vier {\textparagraph} Voort}{alle boomen ende}{planten die groyen}\\

\haiku{Sommeghe doen ooc}{wel de warach-  ticheit}{voor ooghen comen}\\

\haiku{Den eenen en zuldi}{niet beghiften bouen}{den anderen}\\

\haiku{zeere beroert / hij}{en wiste niet wat   dit}{bedieden mochte}\\

\haiku{hebbe ende dat.}{jc dine wet ghehouden}{heb-  be}\\

\haiku{heeft want dicwil}{merctmen bijden bode de}{wijsheit vanden}\\

\haiku{moet   ghij hebben}{bij v wijze meesters in}{astro-  nomien}\\

\haiku{ventositates /}{a meridie quia ille}{sunt immunde}\\

\haiku{werden335 {\textparagraph}  Men sal}{nemen herts hoern ghepul-}{uereert Semis}\\

\haiku{jn houden olie van}{rosen mit een   lettel}{mastic Ende ist}\\

\haiku{mit borne ende /}{mit zeeme in enen stoep}{borrens sal-}\\

\haiku{Dit coemt somtijt}{van bloede Ende somtijt}{soe comet wt}\\

\haiku{be-  uerscul}{ende olium ende die}{galle vanden}\\

\haiku{men houden wil die}{galle vanden stier   soe}{salmense houden}\\

\haiku{xi marc ende of}{1 onse  ende es}{des copers een marc}\\

\haiku{hij lan-  gher}{blanchieren moet Ende soe}{hij dan lichter}\\

\haiku{M549Eister rasis550}{seit dat bloet te   laten}{verlicht die oghen}\\

\haiku{seluen arm Mer}{gheuoeltmen  weede off}{prekelinghe an}\\

\haiku{int vier iij ofte}{iiij korlen van mirre}{Ende dus sultu}\\

\haiku{vanden philo-}{sophen ende die tot}{den dienst behoren}\\

\haiku{de / of den ghenen}{die van quaden humoren}{ghe-  zwollen}\\

\haiku{het derde water}{is alsoe sterck datmer}{stael in soode}\\

\haiku{dulce lac mixtum}{cum farina ordij}{et impositis}\\

\haiku{dit boeck is vol vol728 /}{van gheproefde medi-}{cine ende}\\

\haiku{vanden oghe met}{deser curen  V765An die}{ander ziecheide766}\\

\haiku{hem wezen drie of}{iiij   ende alsoe van}{anderen dinghen}\\

\haiku{coemt vter armenien}{van ouer zee Ende}{es cout ende}\\

\haiku{purgiert   hete}{colere vander maghen}{ende vander}\\

\haiku{die ghescuert}{sijn ende in die culbalch}{gheuallen dair}\\

\haiku{heeft ghelue verwe}{Ende es hol   ende}{breect lichtelike}\\

\haiku{terut Ende  het}{saet in vlaemsche Spo-}{rie Ende es}\\

\haiku{socht de zweere}{vander artetike die}{van hetten coemt}\\

\haiku{dat es om te    {\textparagraph}}{wachtene van quaet doene}{Men  mach gheuen}\\

\haiku{{\textparagraph} Gheminget mit}{aysile ende ghesmeert op}{die stede daert}\\

\haiku{ghesoden ende}{ghe-  plaestert op het}{hooft doet het hair}\\

\haiku{Dat es roy dat}{op het water vliet Ende}{es cout   ende}\\

\haiku{Het   sop vander}{scortsen van roode wilghen}{es goet jnde}\\

\haiku{verteert   coude}{ja ghemenghet mit pepere}{S1549olatrum1550}\\

\haiku{es goet gheplaestert{\textparagraph}}{op dropen    Glas smout1600}{S1601aginum vitri1602}\\

\haiku{Dat es dolke die}{wast   in coerne ende}{meest in vorten}\\

\haiku{hij sal merken of}{dat   bloet zwert vte springet}{of coemt   Ende}\\

\haiku{Of het   coemt van}{medecinen diemen te}{vele   gheeft Off}\\

\haiku{lichame doere}{off in ene stede ende}{dan die drope}\\

\haiku{dair of Nochtan}{salmense smeren mit}{arragon off}\\

\haiku{Een ander1798  Men}{sal nemen borax ende}{peper ende}\\

\haiku{ende doet dat    {\textparagraph}}{gat droghen Ende jnden}{mont ghehouden}\\

\haiku{gate op dat    {\textparagraph}}{si ghesuuert sijn van hairren}{quaden  gronde}\\

\haiku{Op dien dach dat sijt}{nuchteren drincken}{die quaden adem}\\

\haiku{het conforteert de {\textparagraph}}{maghe ende alle}{de leede Ende}\\

\haiku{pen alrehande}{onghemake alse}{kan-  kere}\\

\haiku{water datmen}{maket van eenen cruude ende}{van anderen}\\

\haiku{wateren ende}{doetse al te hope}{van elken euen}\\

\haiku{die steden vanden /}{puusten  die sij hebben}{sijn ghelue ende}\\

\haiku{die tweedeel ende /}{dat salmen temperen}{mit aysine off}\\

\haiku{Ende eist dat hijt}{etet soe en was die hont}{niet ver-  woet}\\

\haiku{handen {\textparagraph} Die   ooc}{tysike sijn nutten van}{dezen sape het}\\

\haiku{vin blancus2333  et}{estouppes de chen-}{neue bien moullies}\\

\haiku{veurres mer-}{ueille Car nul reme-}{de nest milleur}\\

\haiku{onche dencens}{bien puluersiee et}{demy onche}\\

\haiku{de ventre con-}{tinuellement plus}{que  es aultres}\\

\haiku{Vissche wt de}{riuieren in stede}{dair   steenen leggen}\\

\haiku{Ende van ouden}{zoghen Ende vanden}{osse Ende}\\

\haiku{samenen die in}{enen cloc   ouer tfier daer}{wt sal-  3007}\\

\haiku{alrehande}{drope diese dair mede}{saluet Ende}\\

\haiku{smorghens nuch-    /}{teren drinct al cout ende}{des auonts all}\\

\haiku{doer die natheit der}{eerden dat hi nv is}{ghe-  wonnen}\\

\haiku{lichaem heeft die}{drincke de olye vander}{fyolen off}\\

\haiku{de ghe-  nen}{die ghewont js al   te}{hant gaet dat yser}\\

\haiku{Als dit js ghe-}{daen soe salmen nemen}{paerden}\\

\haiku{ghenen  diese {\textparagraph}{\textparagraph}}{dicke nutten  Teghens}{het buuc euel3241}\\

\haiku{vander betonien}{ende bijn-  dense}{vp sijn voerhooft}\\

\haiku{te   poluer}{ende nvttense   des}{wijnters alsmen}\\

\haiku{des ghenen die de}{zucht   heuet ende niet}{slapen   en mach}\\

\haiku{Men sal stoten een}{crwt dat heet henne}{kers ende nemen}\\

\haiku{Regels 23-27 zijn.}{verticaal geschreven in}{de linkermarge}\\

\haiku{511HOemen sal maken.}{licht dat altoes dueren}{sal Rood onderstreept}\\

\haiku{imitatie van resp. ().}{r. 39 en 38laatste woord}{niet te traceren}\\

\haiku{332714 Tekstfragment in (,):}{linkermargegebruiksspoor}{deels afgesneden}\\

\subsection{Uit: De historie vanden vier heemskinderen}

\haiku{alle sijn goet   .}{dat hi hem of sijn vrienden}{genomen had}\\

\haiku{Daer wort ont-  .}{boden voer den coninc Alaert}{ende Maeldegijs}\\

\haiku{Ende als vrou Aye,}{vernam dat   Aymijn quam}{ginc hem te gemoet}\\

\haiku{Aldus had Aymijn.}{iiii kin-  der dat}{hi niet en wist}\\

\haiku{Mer gi moet met mi.}{te hove eer dat gi}{op heidenis vaert}\\

\haiku{Het waer scande wiet,.}{hoerde of sage heren}{of   joncfrouwen}\\

\haiku{hare ban-  .}{nieren ontwonden so si}{eerlicste conden}\\

\haiku{Ic sal proeven in.}{corter tijt of Reynout mijn}{maech is of   niet}\\

\haiku{{\textquoteleft}Neve, gevet mi,;}{dat   paert daer gi op}{sidt men prijstet seer}\\

\haiku{Doe quam clacht voir den}{coninc dat sijn coc was doot}{gesla-  gen.}\\

\haiku{Als de   coninc,:}{dit hoerde seide hi met}{toernigen moede}\\

\haiku{Het is een schoon goet,.}{ghi moget u staet   daer}{eerlic op dragen}\\

\haiku{{\textquoteleft}Heer coninc, ic en.}{spele om so cos-}{teliken pant niet}\\

\haiku{Ic scaec u ende,{\textquoteright}}{mat u met enen rock seide}{Adelaert ende313}\\

\haiku{Ic seg u, de u,.}{gaf desen raet dat hem u}{leven verdroet}\\

\haiku{Aldus reden si,.}{den coninc tegen dair hi}{mit sijn volc quam}\\

\haiku{{\textquoteleft}Heer coninc, hier is,,.}{Bartram   mijn sone}{ic heb hem seer lief}\\

\haiku{Of hy tegen u,?}{yet mesdede soude}{ic dat ontgelden}\\

\haiku{Si gaven den   ,.}{coninc haren scat dat hijt}{bewaren soude}\\

\haiku{U dochter neme.}{ic gaerne ende die}{roetse   mede}\\

\haiku{lant ende grote,.}{eere gedaen daer hi seer}{toernich om was}\\

\haiku{Ende466  sijn si,.}{doot of verdaen God wil haer}{sielen ontfermen}\\

\haiku{hi   is een die.}{starcste vander werelt}{ende den koensten}\\

\haiku{Hi sal ons geven.}{scats genoech ende maken}{ons rijke luden}\\

\haiku{{\textquoteright} Roelant ghinc in de:}{sale totten joncfrouwen}{ende seide}\\

\haiku{{\textquoteleft}Myn rouwe de577  ,.}{ic te hants int herte heb}{is onseggelic}\\

\haiku{{\textquoteright} Mit dat Maeldegijs,,:}{de   woerden sprac bleef Reinout}{staende seggende}\\

\haiku{{\textquoteleft}Doet dese598  heuc,.}{over u hernas an dat ment}{hernas niet en sie}\\

\haiku{xx pont, ende doe,}{ic in dit bosch quam so quam}{mi te gemoet}\\

\haiku{{\textquoteleft}Si leven noch, mer,}{si leg-  gen in}{groot verdriet ende}\\

\haiku{Doe verstont   Reinout.}{wel dattet Maeldeghijs}{om tbeste dede}\\

\haiku{Nochtans doch-616  .}{tet Reinout vreemde dat zyn oem}{dese woirden sprac}\\

\haiku{De cop was oec so.}{groot datmens niet veel so groot}{gesien en had}\\

\haiku{Ende dair is noch,,.}{een hiet Maeldegijs ende}{doet toveri}\\

\haiku{{\textquoteleft}God loens u,   edel,.}{heer coninc ic en mach hier}{niet langer letten}\\

\haiku{{\textquoteleft}Heer coninc, des en,}{acht ic niet mer gi moget}{u wel scamen}\\

\haiku{{\textquoteleft}Wije sal de gene?}{wesen die   dese drie}{heren hangen sal}\\

\haiku{Ende als hij den,:}{coninc   ghegruet had}{seyde de bode}\\

\haiku{{\textquoteleft}Roelant neve, siet,.}{ginder Ogier met sijn volc al}{scone mannen}\\

\haiku{Is dit wair,{\textquoteright} seide, {\textquoteleft}.}{Reinoutso wil   ic selve}{tot Parijs varen}\\

\haiku{{\textquoteleft}Ic rade u wel.}{dat ghijer vaert   ende}{u broeders mede}\\

\haiku{{\textquoteright} Reinout antwoerde met:}{soeten woerden in Bertaens}{ende   seide}\\

\haiku{{\textquoteright} {\textquoteleft}Ya ic, oem,{\textquoteright} seide, {\textquoteleft},.}{Reinoutdanc heb God ende}{ghi oem Maeldegijs}\\

\haiku{Ic bid u, raet uwen;}{neve dat hi mijn crone}{wedergeve}\\

\haiku{u leveren mijn}{swager Reinout met sijn broeders.{\textquoteright}828}{Coninc Karel}\\

\haiku{{\textquoteright} Dus reden si soe.}{lange dat si quamen bi}{Mon-  talbaen}\\

\haiku{Coninc Yewijn is.}{tot u gecomen om}{te sien wat gi doet}\\

\haiku{Ic bid864  Gode.}{dat hi ons beware voer}{scade of scande}\\

\haiku{Ende hi sloech mit.}{alle sijn volc crachtelic}{op die heren}\\

\haiku{Ende885  eer ic,.}{op comen conde was mi}{mijn swaert benomen}\\

\haiku{was met sijn volc, ghinc}{Maeldegijs ende blies den}{horen ende}\\

\haiku{bidden an Reinout dat.}{hi haer ver-  gave}{sinen evelen moet}\\

\haiku{{\textquoteright} Mettien trat937  .}{Gontier voert ende ontfinc}{den hantscoe van Ogier}\\

\haiku{ende beleiden.}{dat cloester Beurepaer}{sterckelijc}\\

\haiku{Als Rei-  nout}{van sijn vrou dese woerden}{hoerde ende haer}\\

\haiku{Want ist dat ghien   ,.}{doden wilt ic sallen u}{mit crachte nemen}\\

\haiku{metten swaerde,.}{of ten si dat gi mi met}{Beyert971  ontvliet}\\

\haiku{{\textquoteleft}Roelant neve, gi,:}{sijt ijmmer mijn bloet ic986}{bid u vriendelic}\\

\haiku{wout u ghelieven}{dat gi mi helpen wout in}{mijnre   eeren}\\

\haiku{{\textquoteleft}Vaert met mi, neve,.}{Ridsaert want   gi en moget}{u niet verweren}\\

\haiku{{\textquoteleft}Wie sal so coe-?}{nen man wesen die mijn}{broeder hangen sel}\\

\haiku{{\textquoteleft}Ripe, nu doet met,.}{mij dat u   gelieft het}{vergae mi soet mach}\\

\haiku{Mit dat Ridsairt dit,:}{sach ver-  blijde hi}{hem ende seide}\\

\haiku{{\textquoteleft}Gi heren,   ic.}{wil u Gode bevelen}{ende sijn Moeder}\\

\haiku{{\textquoteright} Met dese woer-.}{den scheiden de heren}{van malcander}\\

\haiku{{\textquoteleft}Edel here coninc,}{wildi   noch die soene}{ontfaen die u boet}\\

\haiku{Ic sal u goede.}{borge setten dat   ic}{sal bliven gevaen}\\

\haiku{Doe ghinc Maeldegijs:}{voer tconincx bedde}{staen ende seide}\\

\haiku{Als hi dit gedaen,}{had ontboet hi heimelic}{een snijer ende}\\

\haiku{ic bid u, lieve,.}{here dat gi Gode}{voer mi wilt bidden}\\

\haiku{dat hi geploct had.}{ende dranc water daer}{toe uuter fonteyne}\\

\haiku{Daer1124*~          vandt hi.}{scepinge ende voer int}{lant van Slavonien}\\

\haiku{of si die stede.}{tegen die kersten houden}{wouden of niet}\\

\haiku{hem nyemaer dat die.}{Turcken Jherusalem}{gewonnen hadden}\\

\haiku{{\textquoteright} Desen raet dochte}{den ker-  stenen goet}{ende deelden hair}\\

\haiku{Aldus wonnen die.}{kersten in Jherusalem met1160}{Gods hulpe}\\

\haiku{hi wilde selven}{voer hem allen gevangen}{bliven ende}\\

\haiku{Aldus geleyde.}{men Reinout met gro-1162  ter}{eren te scepe}\\

\haiku{Ende dese   :}{was veel tijt bijden heren}{ende hi seide}\\

\haiku{{\textquoteleft}Heer coninc, wi doen}{maken een kerc ende}{u neve Reinout quam}\\

\haiku{Het woord {\textquoteleft}vertaling{\textquoteright}:}{is in verband met de Reinolt}{eigenlijk misplaatst}\\

\haiku{En tot slot hebben;}{we de Historie vanden}{vier Heemskinderen}\\

\haiku{We zullen hier van.}{doen hebben met een merk van}{de opdrachtgever}\\

\haiku{Ook Maeldegijs zelf.}{belandt op een kwade dag}{in Karels kerker}\\

\haiku{Beyaert trapt de stenen,.}{kapot zwemt naar de oever}{en loopt naar Reinout toe}\\

\haiku{Deze laatste vecht.}{tegen de heidenen in}{het Heilige Land}\\

\haiku{Dankzij Maeldegijs.}{kan Yewijn zich bijtijds uit}{de voeten maken}\\

\haiku{Vooral de drieste,,.}{Reinout die een aardje naar zijn}{vaartje heeft bevalt hem}\\

\haiku{De vorm van deze.}{teksten hield nauw verband met}{de overdracht ervan}\\

\haiku{In de vijftiende:}{eeuw vond een belangrijke}{verandering plaats}\\

\haiku{Goods ghevonden is;}{en Een suverlijck crans}{van dusent rosen}\\

\haiku{vier Heemskinderen.}{kon hij zich van inkomsten}{verzekerd weten}\\

\haiku{Achterin is de:}{oude approbatie uit}{1619 opgenomen}\\

\haiku{Bij al zijn acties.}{maakt hij handig gebruik van}{zijn mensenkennis}\\

\haiku{Op afbeeldingen.}{is hij te zien met zijn hoofd}{in zijn handen}\\

\haiku{Deze zalfde de,.}{koning gebruikmakend van}{de heilige olie}\\

\haiku{Waarschijnlijk is er.}{al in de tiende eeuw een}{kasteel gebouwd}\\

\haiku{14105alst wel blikelic....}{was an Elegast dattet}{niet en geschiede}\\

\haiku{Na Jezus' dood nam (-).}{hij Maria bij zich in huis}{Johannes 19:2527}\\

\haiku{14545dat Onse God....}{in Bethleem geboren was}{oetmoedich wesen}\\

\haiku{Ter herinnering.}{hieraan is de strijdkreet van}{de Fransen Monjoie}\\

\haiku{De verwijzing naar.}{Napels komt hier min of meer}{uit de lucht vallen}\\

\haiku{De taal van K is.}{een mengsel van Nederlands}{en Ripuarisch}\\

\haiku{Amerijn van Nerboen.}{\ensuremath{<} Die een is geheten}{aymijn van nerboen}\\

\haiku{gestelt - 57 van zijn.}{volcke verloren \ensuremath{<}}{van zijn volcke}\\

\haiku{B. Besamusca, (),!}{F. Brandsma en D. van der}{Poelred. Hoort wonder}\\

\haiku{der uitgave van],.}{1878 Jacob van Maerlant}{Spiegel historiael}\\

\haiku{Tijdschrift voor boek- (),-.}{en bibliotheekwezen}{51907 p. 135}\\

\haiku{3 dln. Overdiep, G.S. (),.}{ed. De Historie van den}{vier Heemskinderen}\\

\haiku{M. Tersteeg en P.E.L. (),.}{Verkuylred. Ic ga daer ic}{hebbe te doene}\\

\haiku{B. Besamusca, (),!}{F. Brandsma en D. van der}{Poelred. Hoort wonder}\\

\haiku{Zo komt, naast hielt, de,.}{vorm helt een paar maal voor en}{naast sal de vorm sel}\\

\haiku{zo luisterrijk als:}{maar mogelijk was 221yegen}{trecken den genen}\\

\haiku{verbranden omdat:}{zij de moeder was van hem}{die 337dorperheit}\\

\haiku{prachtig straalden en...:}{schitterden 635hem ende zijn}{geselle sopten}\\

\haiku{zal u binnenkort:}{een plaats in een rijk klooster}{geven 689ducaten}\\

\haiku{voegde graaf Amerijn,,;}{de zoon van Aernout van}{Benlant zich bij hem}\\

\haiku{zodat ze daarna:}{nooit meer om soldij zullen}{vragen 921secreet}\\

\haiku{omdat hij erop}{vertrouwde dat Maeldegijs}{te hulp zou komen}\\

\subsection{Uit: Kroniek van het Sint-Elisabethsconvent te Huissen (1667-1752 en 1782-1801)}

\haiku{die weijgerde, '.}{het selve soo dat die in}{t klooster bleef}\\

\haiku{Gelijck men wel}{dencken kan wat droefheijt}{en  benauwtheijt}\\

\haiku{Op dien tijt quam ook,.}{Heer Petrus Codde en de}{neef van Sijn Hooghw}\\

\haiku{En syn Hooghwde.}{Petrus Codde was eenigen}{dagen voor haar Eerw}\\

\haiku{En hebben in 't.}{begin het Iubile\`e niet}{afgekondight}\\

\haiku{Hr van Beest wel mocht.}{op den dagh der professie}{gepreedickt hebbe}\\

\haiku{In dit jaer 1718 heeft}{het convent gekoft het land}{genaemt De Lange}\\

\haiku{En elk heeft in de}{volgende jaar eenigh gelt daer}{voor ontfangen}\\

\haiku{mater Francisca,,.}{Staats ons niet tegen geweest}{maer geholpen heeft}\\

\haiku{mater was, diende.}{voor diaken en deede een}{treffelyk sermoen}\\

\haiku{En ook zijn twee van '.}{onse kneghten int}{bouwhuis gestorven}\\

\haiku{De rivieren zijn.}{toegevroren en met dick}{ys beset geweest}\\

\haiku{En aen ons convent.}{meede door den heer richter}{Buno geinsinieert}\\

\haiku{Ook is dit iaer een.}{nieuwe galerye van houd in}{patershof geset}\\

\haiku{De redens die het}{convent bewogen hebben}{om te verkoopen}\\

\haiku{goethartig en ruym,'.}{toegevende genoeg soo}{dat ten waer d Eerw}\\

\haiku{Waerom sy ook.}{hebben aengehouden en}{endtlyck syn Eerw}\\

\haiku{En in de stadt den,.}{geheele nagt gelogeert}{met schrijven overbragt}\\

\haiku{De verkiesing.}{gong dan voort op zijn tijdt op}{den 13 november}\\

\haiku{Gong in de saal en,,.}{vorder in de gang waar sij}{hoorde dat glas viel}\\

\haiku{dat men vreesde hij.}{de duur aan stukke soude}{geslaagen hebben}\\

\haiku{t Waar al te lang,.}{ales te melden dat in die}{tijt is omgegaan}\\

\subsection{Uit: De kroniek van het St. Geertruiklooster te 's-Hertogenbosch}

\haiku{Later heeft iemand.}{anders dit gedeelte van}{de tekst doorgehaald}\\

\haiku{b. G.C.M. van Dijck,,-, ().}{De Bossche Optimaten}{13181973Tilburg 1973}\\

\haiku{men sijt, dat dese;}{woninge is geweest het}{hoff van den hertogh}\\

\haiku{den oirlog tussen;}{vrouw Johanna ende den}{hertoch van Gelder}\\

\haiku{ende den hertog;}{vervolgde sijnen soone}{seer om te vangen}\\

\haiku{daermede die;}{neringe uijt onse stadt}{seer getogen wordt}\\

\haiku{sag men hier ten Bosch,}{een vreeslijcke comeet}{ende daernaer}\\

\haiku{daer waeren oock,;}{jonge susterkens dat noch}{clijn kinder waeren}\\

\haiku{ende daer bleeff van.}{de partijen niemant doodt}{dan den hoogschout}\\

\haiku{herwaerts, waernaer;}{hij de geestelijcke}{seer swaer heeft geschat}\\

\haiku{ende is des 26;}{mert van de quetsuren tot}{Brugge overleden}\\

\haiku{de Spanssche pocken,.}{die men noemde Sint Jobspassie}{ofte siecte}\\

\haiku{Eodem anno is.}{gefondeert het Rabauwhuijs}{tot Oisterwijck}\\

\haiku{{\textquoteleft}Mijn susterken, mijn.}{heer vader en sall u int}{clooster niet laeten}\\

\haiku{ende quaemen op.}{den Ham ende namen het}{huijs te Kessel in}\\

\haiku{branden insgelijxs;}{deselve dorpen met de}{kerck tot Geldrop}\\

\haiku{vinde hij daer geen,{\textquoteright}.}{gebreck soo comt dan en segt}{wat mijn leth}\\

\haiku{ende daervan.}{is collator den prior}{vant Fraterhuijs}\\

\haiku{In den jaere 1516.}{regneerde hier een siecte}{genaemt plerensis}\\

\haiku{Het was een specie,.}{van de pest waervan veele}{menschen storven}\\

\haiku{Eodem anno is}{de stadtsmuer gelijt van den}{Vuchterendijck}\\

\haiku{item den coster een;}{stuijver en een schoon broot van}{een halve stuijver}\\

\haiku{Item dijlt men in de.}{vasten broot ende een groote}{quantitijt haring}\\

\haiku{ende op dese;}{dagvaert wiert voor vrouw Magriet}{versocht een bede}\\

\haiku{ende daervan.}{is collator den naesten}{van den bloeden}\\

\haiku{nochtans raede haer,}{sommige schepenen dat}{sij maer geloven}\\

\haiku{het ging een tijt lang;}{in dese stadt ofte in}{een slach was geweest}\\

\haiku{Eodem dito de;}{gemijnte dit hoorende sijn}{seer ontstelt geweest}\\

\haiku{den gulden van de,;}{colveniers die men segt van}{Sint Cristophorus}\\

\haiku{de Hagenaers niet;}{beter wetende als dat}{sij vrinden waeren}\\

\haiku{wert priester ende.}{wederom rector ende}{beneficiaet}\\

\haiku{tegen dien tijt wiert;}{een balie gemact op}{den Wintmolenberg}\\

\haiku{{\textquoteleft}Godt almachtig, wilt{\textquoteright};}{mijn niet meer opleggen als}{ick verdragen magh}\\

\haiku{ende doen hebben;}{se hem opt schavot noch den}{cop afgehouwen}\\

\haiku{ende doen troc den;}{prins met sijn volck over de}{Locht naer Antwerpen}\\

\haiku{ende bleeff daer twee.}{dagen liggen sonder iet}{uijt te rigten}\\

\haiku{Item ider mergen is,.}{600 roijen groot de roeij}{als boven 14 voet}\\

\haiku{hetwelck De Kock.}{wel vijftienhondert gulden}{gekost hadde}\\

\haiku{Eodem anno was.}{binnen Den Bos groot gebreck}{van folio 136v brant}\\

\haiku{ende doen volgde:}{noch een wagen daer opsaet}{den soth seggende}\\

\haiku{op eenen niuwen raet}{van twalf personen die men}{noemde den Bloet Raet}\\

\haiku{onder haer armen.}{dicke goude ketenen}{hadde hangen}\\

\haiku{naer die galge die;}{op de Merct stont ende hem}{daer af gehangen}\\

\haiku{een teecken soude}{geven van sijn ontschult voor}{het volck ende}\\

\haiku{ende met dit hoog;}{water dreeff aff de lange}{brugge tot Orten}\\

\haiku{ende daervan.}{is collator een van sijn}{naeste vrinden}\\

\haiku{In den eersten de}{maete van de hooftstadt van}{s'Hertogen Bos}\\

\haiku{ende sij hebben}{gereetschap gemact om}{haer te pijnigen}\\

\haiku{Eodem anno den}{10 januarij begon}{het te doijen}\\

\haiku{ende sijn dien avont;}{gijselaers gestelt om te}{parlementeren}\\

\haiku{de aenslach met.}{haer misluct was waeren niet}{wel te vreden}\\

\haiku{gestelt worden opt.}{feijt ende exercitie}{van de religie}\\

\haiku{Int selve jaer den.}{20 januarij heeft den}{prince folio 232r}\\

\haiku{wordt alle jaer op}{den eersten julij bij den}{bisschop van Den Bos}\\

\haiku{Godt die Heere sij.}{met u. Uijt Mastrigt den 23}{februari 1580}\\

\haiku{ende daer van is.}{collator den deecken van}{Hilvarenbeeck}\\

\haiku{de soldaten en;}{Bobadilla dit siende}{schieten daer op toe}\\

\haiku{dit dessijn ende.}{gevolg van dien is geweest}{op dese manier}\\

\haiku{ende sij moste;}{alle gebogen sitten}{ende niet recht op}\\

\haiku{den prins dit hoorende,;}{soo heeft hij de aenvangers}{laeten aftrecken}\\

\haiku{gesteecken, soo dat;}{de kerck lichter flamme}{begon te branden}\\

\haiku{want den cardinael.}{hadde groote devotie tot}{dien hijligen}\\

\haiku{Eodem anno op.}{den 8 februarij sijn}{opt hoog- blz. 22}\\

\haiku{de welcke den;}{winter sneuw ende vorst haer}{op het lijff gestiert heeft}\\

\haiku{ofte iet anders;}{aen te tasten tot naedeell}{van dese landen}\\

\haiku{Den 3 dito is.}{den hertog van Brabant tot}{Weerdt gecomen}\\

\haiku{Eodem anno den}{27 mert sanderendaegs naer}{Paesschen isser}\\

\haiku{Eodem anno is.}{de corps du grade aen de}{Vuchterpoort gemackt}\\

\haiku{als hij een brigaert}{vaendragers int gesigt creeg}{van de welcke}\\

\haiku{Eodem dito is.}{een paert geruijlt tegen}{een hoendereije}\\

\haiku{het welck raer om,.}{te sien was want waeren geenen}{arbijt gewoon}\\

\haiku{toe oock ander.}{consent ofte paspoort te}{derven verwerven}\\

\haiku{Gedaen int leger ';}{voors-Hertogenbosch den}{14 september 1629}\\

\haiku{ende het licham.}{op een rath geseth op de}{Vuchterhijde}\\

\haiku{ende daegs daer aen.}{dede hij de revu over}{het garnisoen}\\

\haiku{In den jaere 1685,.}{is alhier een schoon niuw}{vischmerckt gemackt}\\

\haiku{ende achter een.}{niuwe galderije met een}{blauw stene trap gelijt}\\

\haiku{want sij hadden in;}{Vrancrijck in de stadt Aras}{gevangen geseten}\\

\haiku{{\textquoteleft}Ende stont als een.}{bepiste paep tot spoth}{van alle menschen}\\

\haiku{ende sommige.}{grutters van vooren vant stadthuijs}{comende blz. 461}\\

\haiku{\ensuremath{<}de tekst van dit}{paskwil ontbreekt\ensuremath{>} ~ Den}{4 october sijn}\\

\haiku{A.R.M. Mommers, Brabant (;}{van Generaliteitsland}{tot gewestdl. II}\\

\subsection{Uit: Medische en technische Middelnederlandse recepten}

\haiku{Priebsch, R., Deutsche (,).}{Handschriften in Engeland}{Erlangen 1901}\\

\haiku{naast het vergulden (), {\textquoteleft} ...}{van metalen524~		 of}{het maken vanliesten}\\

\haiku{er is een brede.}{marge aan alle zijden}{van het perkament}\\

\haiku{Zie ook hieronder:}{bij de bespreking van B.M.}{Ms. Sloane 345}\\

\haiku{O tu corporis ...}{curam gerens per omnia}{auxiliari}\\

\haiku{De verzameling:}{wordt voorafgegaan door de}{Latijnse titel}\\

\haiku{Dit is een reeks van.}{27 vragen en antwoorden}{in het Latijn}\\

\haiku{algemeen 298~		,,,.}{301~		 van een ziek dier 299}{verse 300}\\

\haiku{Men kan er eveneens {\textquoteleft}{\textquoteright} ().}{diaquillon maturativum}{op smeren1194}\\

\haiku{Men wrijft de vorm in ().}{met kalkwater opdat hij}{hard worde596}\\

\haiku{is de persoon in, ().}{kwestie ziek dan schuimt de olie}{anders niet14}\\

\haiku{is de persoon ziek,.}{dan zal het bloed in het ei}{branden anders niet}\\

\haiku{de ene made eet.}{de andere op tot er}{nog maar een overblijft}\\

\haiku{Ende van desen,}{vorseiden cruden neemt een}{hant vulle ende}\\

\haiku{es hi besiect, die.}{olie sal scumen gelyc dat}{een beer doet die uecht}\\

\haiku{Scorse van bongi}{ghepuluert droegt quade}{gate in veden}\\

\haiku{te puluere}{neemt elx euen vele bi}{ghewichte ende}\\

\haiku{daghen ende doet,.}{inde oghen dickent hi}{sal ghenesen}\\

\haiku{de mag\ensuremath{<}h\ensuremath{>}[],}{e ende verduwet wel}{ende dien therte}\\

\haiku{Dit is van elken}{crude sonderlinke dat}{men doet in wonden}\\

\haiku{sincse, hi sel,.}{steruen ende vlietse}{hi sal nesen}\\

\haiku{*~         {\textparagraph} Alrehande:}{melc is van couder ende124}{droghe natuer}\\

\haiku{Die gaghel is cout.}{inden eersten graet ende}{droghe inden .ij}\\

\haiku{{\textparagraph} Men sel nemen oly}{die van rosen is ghemaect}{ende saluen}\\

\haiku{*~         {\textparagraph} Nem agrimonie:}{ende wriifse met gheten}{melc op die stede}\\

\haiku{nem die blade van.}{agrimonie ende legghes}{op die wonde}\\

\haiku{ende leggent myt.}{den pynseel hoe groet of hoe}{cleyn dat ghy wylt}\\

\haiku{zuarte of ander,.}{alre hande verue so}{salmen nemen .ij}\\

\haiku{een luttel ende,.}{dan so settet af ende}{dan ys bereyt}\\

\haiku{*~         Recipe eyn}{steen daer wisselars gout op}{pruuen ende wrift dar}\\

\haiku{Ende doet dat feel,.}{daer yn ende laetse een}{dach dar yn liggen}\\

\haiku{Ende dan suldi (.}{se weel  wtreecken ende}{makent dan alstfol}\\

\haiku{Dan neem[t] dat vel.}{ende naijet toe ende laet}{een gat an die .ij}\\

\haiku{loet brisilien houts.}{ende is brisilien seer}{goet so neemt bi .iii}\\

\haiku{veel schoenmakers219 zwart.}{als des voirseit is ende}{rorent met een stock}\\

\haiku{Ende laet dat te.}{gader sieden met water}{ende vrine ana231}\\

\haiku{of also lange.}{dat dat fermilioen so}{swart sy als een coel}\\

\haiku{*~         Aldus salmen.}{tegen molken toueren}{remedium doen}\\

\haiku{Noch zinziber aen.}{stucken gesneden ende}{schoen ghemact .vij}\\

\haiku{Meen maeght ock gieten,.}{in wormen ende maken}{daer af watmen wil}\\

\haiku{{\textparagraph} Mede machmen dy[].}{rose weruen myt wat}{werue datmen wil}\\

\haiku{laten staen tot dat,}{dy honich doer hoer lyef}{gethogen is ende}\\

\haiku{Ende yn vele,.}{boke hetet aqua vite dat}{is twerstaen des leuens}\\

\haiku{Dyt water ys seer.}{laxatiff ende nuchteren}{gedroncken .xiij}\\

\haiku{Ende alsmen dy,.}{bladeren ontwee trect soe}{yrst al haer achtich}\\

\haiku{Ende genest noli.}{me tangere eyn plaester}{daer af op geleit}\\

\haiku{*~         Recipe dy}{blommen vander kleyner}{matelyuen ende}\\

\haiku{Ende doet scheiden.}{grote groue winden ende}{maket guden adem}\\

\haiku{alle wonden nye}{ende olde ende doet}{oeck alle saken}\\

\haiku{hebstu dy oghe,.}{appel behalden du salts}{gesont werden}\\

\haiku{hase noten dat;}{syet yn een pan ende}{latet enen nacht staen}\\

\haiku{Nemt agremonia,,,,}{bugghel roet coel mede dat}{doet yn dry pynten}\\

\haiku{die melck drinct sauents,.}{ende dat water morgens}{law dyt doit .xviij}\\

\haiku{{\textparagraph} Tegen dy koeck dy.}{van den euel bleuen is}{onder dy ribben}\\

\haiku{Tegen dy swaerte.}{blomme dy nyt genoch}{gesworen is}\\

\haiku{Nemt salue dy wan}{sauelboem gemact is}{ende salft dy daer}\\

\haiku{*~         Recipe olie.}{van comillen ende olie}{van dille elx .vj}\\

\haiku{desen dranc is tot.}{gesteken wonden dy nyt}{willen etteren}\\

\haiku{Ende dan salmer.}{mede werken in deser}{manyren daert .j}\\

\haiku{*~         spumidusIs,}{dat bloet schumende so is}{di sucht in dy borst}\\

\haiku{Ook met De Vr. 301,.}{doch hier wordt nog toegevoegd}{dat de wijnruit .ix}\\

\haiku{Er is zogoed als:}{volledige overeenkomst}{met De Vr. 370}\\

\haiku{Hetzelfde middel.}{komt ook voor in 863~		 en}{in MGR 1003 en 1175}\\

\haiku{wilts lijnzaets voor wit.}{linzaets en scaperroet voor}{hamelin roete}\\

\haiku{Een gedeelte van,.}{dit nummer komt ook voor in}{MGR 1057 1257 en 1258}\\

\haiku{Een gedeelte van.}{dit nummer vindt men eveneens}{terug in MGR 1262}\\

\haiku{51{\textquoteright}, Niederdeutsche (),-,-.}{Mitteilungen X1954 5}{22 vooral 810}\\

\haiku{Er is een grote,,.}{gelijkenis met 336}{441~		 en MGR 1231}\\

\haiku{deze gegevens.}{vindt men daar terug in het}{volgende nummer}\\

\haiku{G. Keil, R. Rudolf, (.),.}{etc.eds Fachliteratur}{des Mittelalters}\\

\haiku{Er is grotere.}{overeenkomst met 223~		 en}{667~		-668}\\

\haiku{Men vgl. MGR 930, de,,.}{tweede helft van MGR 954 1353}{De Vr. 57 en 346}\\

\haiku{Nauw verwant is dit.}{recept met het voorlaatste}{middel uit MGR 876}\\

\haiku{laet coelen, dan maect,,;}{tappe vinghers lanc ende steec}{in hu fondament}\\

\haiku{De bedoeling van.}{dit merkwaardig recept is}{niet zeer duidelijk}\\

\haiku{Item is dat bloet hert,.}{ende zwart dat bloet is to}{langhe geholden}\\

\haiku{738 ypericon, 669,,,.}{724 yrindina 749 ysopy}{690 yuorie}\\

\haiku{688, 689 Lat. huislook, ().}{donderbladSempervivum}{tectorum L.}\\

\haiku{197 grote - ende.}{middele ende clene}{zie confeli}\\

\haiku{1) 4, 38, 63, 240,,,,,,,,.}{241 253 315 327 341 344 347}{355 enz. zie dic}\\

\haiku{689 de geboorte.}{van Maria wordt gevierd op}{8 september}\\

\haiku{5 andoorn, witte ().}{malroveMarrubium}{vulgave L.}\\

\haiku{is eyn krut ende.}{vasset op hogen steden}{zie hierboven}\\

\haiku{Voor de bereiding,.}{en de eigenschappen zie}{Grabadin 286}\\

\haiku{Ende alsmen dy,{\textquoteright}.}{bladeren ontwee trect soe}{yst al haer achtich}\\

\haiku{704 cleyne - wolfsmelk,.}{een van de zeer talrijke}{Euphorbiasoorten}\\

\haiku{261, 262, 263, 266, 267,, ().}{268 weit tarweTriticum}{vulgare L.}\\

\haiku{6R. Forbes, Studies (,).}{in Ancient Technology}{Leiden 1971 Vol. III}\\

\haiku{r verbeterd uit.}{t. 218Na verwen staat van}{bresilien doorstreept}\\

\haiku{282v verbeterd uit.}{w. 283Na saluie volgt aqua}{per omnia doorstreept}\\

\haiku{v verbeterd uit.}{w en u uit w. 300Na}{gut volgt in het Hs}\\

\subsection{Uit: Memorye}

\haiku{H.J. ende was een}{hennepklopper ende den}{anderen ghenaemt}\\

\haiku{Maer mijn heren die.}{schepens en wilden hem dat}{niet concenteren}\\

\subsection{Uit: Middelnederlandse geneeskundige recepten}

\haiku{Willems, J.F., {\textquoteleft}Brokken{\textquoteright},.}{uit een oud Geneesboek der}{XIVe eeuw Belg. Mus}\\

\haiku{in het papieren;}{gedeelte vari\"erend}{van 143/195 {\texttimes} 200/243 mm}\\

\haiku{in het bezit van.}{een medicijnmeester die}{Jan van Utrecht heette}\\

\haiku{nie in nye was (814), () (), ().}{mit739~		 en mer740}{vuel745}\\

\haiku{De samenstelling.}{der katernen is niet meer}{te achterhalen}\\

\haiku{Een {\textquoteleft}stritorie{\textquoteright} maakt ().}{men met gemalen wierook}{en eiwit233}\\

\haiku{as van look, atrament () {\textquoteleft}{\textquoteright} ().}{en bonen568~		 of wit}{honds quade575}\\

\haiku{is het gezwel in, ();}{de lies dan binde men het}{boven de voet278}\\

\haiku{Tegen {\textquoteleft}harteevel{\textquoteright} {\textquoteleft}{\textquoteright} ().}{drinke menpollioen met}{honig in wijn496}\\

\haiku{Gezwellen geneest ();}{men door te gorgelen met}{hete wijn1113}\\

\haiku{Om {\textquoteleft}veel naturen{\textquoteright} {\textquoteleft}{\textquoteright} ();}{te hebben drinke men sap}{vankamillen180}\\

\haiku{zaad van {\textquoteleft}jusquiami{\textquoteright},, {\textquoteleft}{\textquoteright} {\textquoteleft}{\textquoteright} ();}{atrament zaad vancaerden}{en vandolke511}\\

\haiku{Een beknopte leer.}{van de urinoscopie wordt}{gegeven in 1112}\\

\haiku{Als {\textquoteleft}vreemd{\textquoteright} voorwerp kan:}{men ook de beensplinters in}{wonden beschouwen}\\

\haiku{bereiding (353~		) ():}{en gebruik354~		 tot 358}{.aqua petralis}\\

\haiku{bereiding (359~		) ():}{en gebruik360~		 tot 364}{aqua yrundinea}\\

\haiku{Bijzonder talrijk:}{zijn de pleisters die men op}{wonden kan leggen}\\

\haiku{men netelmoes met {\textquoteleft}{\textquoteright} {\textquoteleft}{\textquoteright},:}{zout en pulver vanalune}{ofcalmijn of nog}\\

\haiku{boven de regel, {\textbackslash} /;}{toegevoegd werd staat in de}{uitgave tussen}\\

\haiku{Nemt erdbesie cruud.}{metten wortelen ende}{metten bladeren}\\

\haiku{lepel vol nauonts.}{ende nuchtens tote dat}{ghi sijt ghenesen}\\

\haiku{lepel vol een nacht.}{ende smorghens suldi sien}{dien steen ghesmolten204}\\

\haiku{ende doet dicke,.}{in die noselocken het}{sal betren}\\

\haiku{hem hebben goeden.}{lust tetene ende te}{drinckene}\\

\haiku{Jeghen menisoen.}{van bloede ende ieghen}{alle menisoen}\\

\haiku{Omme yser oft houdt,.}{scicht oft doren dat es int}{tlijf wt te doene}\\

\haiku{bosse ende als,.}{ghijt doet vp die wonde soe}{legghet vp vlasse}\\

\haiku{mortier ende doet}{in een scotele238 ende}{doet vte die239 polen}\\

\haiku{ende wriuet wel,.}{cleene ende doet in bossen}{dats poplioen}\\

\haiku{{\textparagraph} Sied werc ende soudt.}{in een lettel aysins ende}{bindet vp tled}\\

\haiku{mortier ende dan [.}{salmene temperen in}{sterken wittenfol}\\

\haiku{linen cleet ende,.}{drinckent nuchteren ghi}{sult ghenesen}\\

\haiku{pater noster mach,.}{segghen dan latet coelen}{ende dan neemt .j}\\

\haiku{Ende die pusoen,.}{nemt vp die daghe hi sal}{steruen voer .xv}\\

\haiku{daghe lanc elcs,.}{daghes sperma rute hi}{worde binnen .ix}\\

\haiku{Oec es goet tote.}{desen drancke die crop}{vanden biuoete}\\

\haiku{Looc ghesoden in.}{fonteyne verdrijft smerte}{ende ghezwel}\\

\haiku{die in frenesien,,.}{sijn drincke {\textbackslash}men/368 salre}{bi genesen}\\

\haiku{Ende dan salment.}{nemen ende netten int}{witte vanden eye}\\

\haiku{Nem zaet van genste,,.}{ende tempert met wine}{ende ghef hem .j}\\

\haiku{Nem papple, ende,.}{steenvaren ende nutte}{dat te samen}\\

\haiku{*~         Stoete die gherste,.}{alsoe langhe dat die dop}{aff gaet dan nem .iij}\\

\haiku{Item j aer die comt.}{tuschen die vinger ende}{is j lastich seer}\\

\haiku{j vir, Ende als,}{sy corten so corten sy}{alle drie weeken}\\

\haiku{77r] Nem eltena.}{ende roden coel ende}{cipres appel}\\

\haiku{Item bakelaer daer,.}{maectmen of oly van beyen}{senwer oly ist}\\

\haiku{te gader also [.}{want dicke wordt ende dat}{leg{\textbackslash}g/he op eenfol}\\

\haiku{eest van hitten of,.}{van walghene hi gheneest}{of van bloede}\\

\haiku{ende bestrijct de[][.}{gate altoes in tuusche}{n deken]668}\\

\haiku{*~         Item coppen spin,.}{gheleit op enen wonde die}{en svellet niet}\\

\haiku{ende teghen wint.}{ende teghen steecten die}{ter herten gaen}\\

\haiku{*~         Item sali mit.}{water ghesoden is goet}{teghen die borst}\\

\haiku{Item clare vrine,.}{in den vate mit siechten}{die is dodelic}\\

\haiku{Salmen nuchteren, [].}{eten ende daer goeden wijn}{opdrincken}\\

\haiku{*~         Sijn cout in den;}{eersten grade ende nat}{inden anderen}\\

\haiku{Is couder ende;}{natter van naturen dan}{die ghesouten is}\\

\haiku{si verdriuet die, [.}{bose winde des menschen}{is goet ghetenfol}\\

\haiku{hi laten in die.}{luchter hant ende inden}{vorderen voet}\\

\haiku{Naar welk boek in {\textquoteleft}des{\textquoteright},.}{boecs leeringhe verwezen wordt}{is mij niet bekend}\\

\haiku{In deze laatste.}{toepassing is het middel}{duidelijk magisch}\\

\haiku{nem saet van esschen,;}{ende van anise ende}{minghet te gadere}\\

\haiku{Ende werdt yement(),;}{ghequetstt dat hi bloet hi}{salder of sterven}\\

\haiku{Vgl. de passage.}{die door R. Foncke wordt}{aangehaald uit Hs}\\

\haiku{De woorden {\textquoteleft}ieghen{\textquoteright}.}{quaet ghespaente komen in}{de tekst tweemaal voor}\\

\haiku{698.-699 Deze twee.}{nummers maken deel uit van}{hetzelfde geheel}\\

\haiku{De betekenis.}{van rooesel en gestoft is}{mij niet duidelijk}\\

\haiku{Uit de kontekst blijkt.}{in alle geval dat hier}{een plant bedoeld wordt}\\

\haiku{Cleyne stot unde.}{an nese pustet vordrift}{dat blot der nese}\\

\haiku{Dat sap is ock g\r{u}t,.}{an de oghen droft deme}{de oghen ser sint}\\

\haiku{Mit wine soden}{unde dat hovet mede}{gedwagen vordrift}\\

\haiku{Wat tussen haakjes,.}{staat in de bovenstaande}{tekst komt in de Mnl}\\

\haiku{Du scult eme beden.}{dat he sic warme decke}{so wan he svetet}\\

\haiku{so wel he denne,;}{drinken so sint eme alle}{sine aderen open}\\

\haiku{Senep unde etik is,.}{eme bose alle versche}{spise is eme gut}\\

\haiku{De drovich is de,.}{drinke stedes merc sap dat}{maket ene vro}\\

\haiku{und och dat dode,}{kynt mit erer bitterheit}{is se g\r{u}t weder}\\

\haiku{Poleie ende.}{wegebrede de salt du}{hebben gerede}\\

\haiku{174 arguel es,.}{effel van wine droesem}{bezinksel van wijn}\\

\haiku{1335 wortelen van () ().}{bietBeta vulgaris L.}{ofB. cicla L.}\\

\haiku{Een van de vele.}{soorten van de geslachten}{Carduus of Cirsium}\\

\haiku{267, 269, 270, 328, 351,,,,,,,,.}{393 494 566 568 569 570 571}{575 zie festele}\\

\haiku{daz man nicht pald sein, ...{\textquoteright}.}{geleichen mag gefinden daz}{also heilsam sey}\\

\haiku{- roberti 256, 317 ().}{robertskruidGeranium}{robertianum L.}\\

\haiku{603 de bladeren ().}{van mispelboomMespillus}{germanica L.}\\

\haiku{some die plecken,{\textquoteright} (.}{sijn wit some root ende}{som sijn si swertAnt}\\

\haiku{In de Herbarys {\textquoteleft}.}{leest menPetrocilium}{dats petercelle}\\

\haiku{492, 734, 765, 890, 937,.}{stranguria d.i. langzaam en}{pijnlijk urineren}\\

\haiku{839 rood operment, ook.}{wel rood arsenicum of}{sandracha genoemd}\\

\haiku{886, 1293 bloesem van ().}{roggeCentaurea cyanus}{L. of roggemeel}\\

\haiku{Cf. Van Leersum en, (,).}{verder R. Crawford The King's}{EvilOxford 1911}\\

\haiku{2) 878, 887, 889, 893,,,,,,,,,, ().}{910 959 960 992 1056 1062 1087}{1088 1120 1370 kookhet}\\

\haiku{382 uit het verband.}{krijgt men de indruk dat een}{plant bedoeld zou zijn}\\

\haiku{741, 746 de fijt, een.}{pijnlijke verzwering aan}{de vingerwortel}\\

\haiku{866, 872 oogziekte,.}{vliesje op het oog dat doet}{denken aan een vlek}\\

\haiku{68E.C. Van Leersum, De {\textquoteleft}{\textquoteright} (,).}{Cyrurgie van Meester Jan}{YpermanLeiden 1912}\\

\haiku{recepten zijn de:}{volgende uitgaven en}{studies van belang}\\

\haiku{d. Wissenschaften () (),-,,.}{Wien XLII1863 110200 en Fr.}{Wilhelm O.c. no XXV}\\

\haiku{152c en d zijn de.}{1e en 2e kolom van de}{verso-zijde}\\

\haiku{er is een vlek op,.}{deze plaats die het woord zeer}{onduidelijk maakt}\\

\haiku{Sprachforschung LXXXI (1958),).}{38 nota wordt gesteund door}{muusore in de Mnl}\\

\subsection{Uit: De Middelburgsche avanturier. Of het leven van een burger persoon}

\haiku{Toen wy t'huis kwamen,:}{vonden wy onze Ouders}{by elkanderen}\\

\haiku{Zy kregen 'er wel,.}{haast een die omtrent achtien}{jaren bereikte}\\

\haiku{Jannetje, dus wierd,.}{dat Meisje genaamt was eene}{volmaakte schoonheid}\\

\haiku{Dagelyks hield ik,;}{by myne Moeder aan om}{haar te laten gaan}\\

\haiku{Van een volmaakt schoon,;}{en jong Meisje bemind te}{worden is iets groots}\\

\haiku{\^o Onbezonnen,!}{Jeugd hoe driftig loopt gy in}{u eigen bederf}\\

\haiku{zoude zyn, besloot,.}{hy myn  noodlot aan den}{tyd overtelaten}\\

\haiku{dat de vreugd zo groot,.}{binnen Scheepsboord was als op}{de beste kermis}\\

\haiku{het ontbrak my aan,}{geen geld en door middel van}{het zelve konde}\\

\haiku{Osmans vreugde was,.}{overbodig toen hy ons zo}{wel te vreden vond}\\

\haiku{Ik smeet myn pistool,,.}{ter neder greep een Sabel}{en sprong mede over}\\

\haiku{met haar zou men wel.}{een zieke Antonetta}{kunnen ontberen}\\

\haiku{Nu wierden myne,}{oogen eerst geopend en ik}{zag wat godloos stuk}\\

\haiku{In de Kloosters in;}{het tegendeel heerscht de}{overvloed en de pracht}\\

\haiku{maakte kruis op kruis,.}{en zag onophoudelyk}{naar myne voeten}\\

\haiku{{\textquoteleft}Als Moeder lief'er zelf, ';}{spikkel op heeft wil iker}{niet meer van spreken}\\

\haiku{zoude hebben, waar;}{in zy myn verstand tot de}{wolken verhefte}\\

\haiku{\^o Ysselyke,!}{minnenyd hoe deerlyk weet gy}{de mensch te grieven}\\

\haiku{zeker gy zult hem,,.}{trouwen of ondervinden}{dat ik Landryk ben}\\

\haiku{Niet te hoog, riep zy,,.}{in toorn ontstoken niet te}{hoog Verrader}\\

\haiku{Gaarne had ik die;}{Dame myne dankbaarheid}{mondeling betuigt}\\

\haiku{derhalven geef u,.}{aan my over en ons beider}{fortuin zal een zyn}\\

\haiku{hy wezentlyk, toen '}{hyer slegts een jaar en zes}{maanden geweest was}\\

\haiku{In weinig tyds had....}{ik myne Plantagie en}{die van Juffrouw P}\\

\subsection{Uit: Ongelukkige levensbeschrijving van een Amsterdammer}

\haiku{Ik wilde, dat gij,{\textquoteright}, {\textquoteleft}}{was als ik zegt hij ergens}{tegen zijn zuster}\\

\haiku{Dus richtte hij zijn.}{aandacht op het platteland}{in de omgeving}\\

\haiku{Ik had al  een}{jaar of twee op het smeeden}{geweest en ik docht}\\

\haiku{En ik verhaalde.}{het hele geval zoals}{het was toegegaan}\\

\haiku{Ge kunt begrijpen.}{of zo een officier ook}{een goede reis doet}\\

\haiku{Hij is nu een groot,.}{jaar bij mij geweest maar wat}{schrijft hij nu al mooi}\\

\haiku{Op de aspot, daar,.}{braaf vuur onder was zodat}{het schielijk hiet wierd}\\

\haiku{Zij moest de melk weg.}{gooien en wierd niet gewaar}{wie het gedaan had}\\

\haiku{Maar ik kreeg op het {\textquoteleft}{\textquoteright}.}{lest van de jongens de naam}{vanketelscheiter}\\

\haiku{Wij hadden allang.}{een overslag gemaakt43 om het}{wijf te betrekken}\\

\haiku{{\textquoteright} {\textquoteleft}Neen vader,{\textquoteright} zei ik,.}{en meteen stong ik op en}{ging naar mijn winkel}\\

\haiku{Mijn baas zal het ook,.}{wel gezien hebben maar niet}{hebben willen zien}\\

\haiku{En toen gingen ze,.}{aan het ramenassen63 van}{kelken en flessen}\\

\haiku{Nu, de kapitein,:}{kwam en het compliment dat}{hij ons maakte was}\\

\haiku{{\textquoteright} De Heer ging weg, en.}{hij vroeg mij aanstonds waar of}{ik ze krijgen kon}\\

\haiku{{\textquoteright} De Beurs ging uit en.}{de Heer kwam en vroeg of hij}{ze had gekregen}\\

\haiku{Om mijn vader een.}{genoegen te doen moest ik}{weer een baas hebben}\\

\haiku{Waar ik vandaan ben,,.}{wie mijn ouders zijn en wie}{mijn famielje is}\\

\haiku{Als ik niets meer kon,.}{verkopen dan ging ik truk}{of biljard spelen}\\

\haiku{{\textquoteleft}Wel,{\textquoteright} was zijn antwoord, {\textquoteleft}.}{de joden weten hem veel}{geld te bezorgen}\\

\haiku{{\textquoteright} Er waren ook vier.}{Heren die bij malkander}{schenen te horen}\\

\haiku{Ik heb de brieven;}{dikwijls gelezen die gij}{haar geschreven hebt}\\

\haiku{Ze hebben nu meer.}{geld dan ze ooit voor deze}{bezeten hebben}\\

\haiku{Kom, ik zal je daar.}{brengen waar ze u dat wel}{zullen verleren}\\

\haiku{maar dan niet in de,.}{jordaene die in Asia leyt}{maar te Amsterdam}\\

\haiku{Somtijds kreeg ik wel,.}{voor twee schellingen slaag maar}{daar gaf ik niet om}\\

\haiku{Zij die op krukken,.}{gingen als zij bij de weg}{waren maar hier niet}\\

\haiku{{\textquoteright} Meteen sche\"en de.}{kaartspelers uit en gingen}{wij allen naar bed}\\

\haiku{Ik rookte en zei,.}{dat ik vertrok en kwam om}{afscheid te nemen}\\

\haiku{Het ging weer over, en,}{ik vernam dat die mij die}{klap gegeven had}\\

\haiku{Ik was verbazend, '.}{leerzaam en daarom ging ik}{mede aant roer}\\

\haiku{Pas op, de dienders.}{van het Zeerecht zullen je}{bij je gat krijgen}\\

\haiku{{\textquoteright} {\textquoteleft}Ja, verdomd,{\textquoteright} viel die, {\textquoteleft}}{kaagschippersknecht hem in de}{redejou vader}\\

\haiku{Ik liet mij niet lang,.}{uitnodigen maar kreeg hem}{snel in mijn vlodders}\\

\haiku{Nu,{\textquoteright} zei hij, {\textquoteleft}laat het,.}{niet weer gebeuren of er}{zal wat opzitten}\\

\haiku{Wij kwamen dan aan,.}{de wal en wierden door een}{ieder verwelkomd}\\

\haiku{Zij weten daar ook,.}{wel dat het geld de God van}{deze wereld is}\\

\haiku{{\textquoteright} De zilversmid kwam,.}{te sterven en twee dagen}{later de snijer}\\

\haiku{{\textquoteleft}Wel,{\textquoteright} zei hij, {\textquoteleft}heb je?}{geen liefhebberij om in}{dit land te blijven}\\

\haiku{Wij waren pas een,.}{week of zes op zee of wij}{kregen een zieke}\\

\haiku{En nog het mooiste,.}{er van is men ziet ze zeer}{zelden baas worden}\\

\haiku{die maaltijd heeft je,?}{zo veel  gekost maar wie}{waren je gasten}\\

\haiku{Ik begon al wat,,,:}{koeler te worden maar toen}{zij dat zag zei ze}\\

\haiku{{\textquoteright} {\textquoteleft}Welnu dan, dewijl,.}{gij de man zijt zal ik je}{mijn komst uitleggen}\\

\haiku{Alle weken drie,.}{gulden op afrekening}{tot het betaald is}\\

\haiku{In plaats dat ik een.}{Sara had gekregen was}{het een Xantippe}\\

\haiku{Hij had kinderen,.}{noch famielje zodat hij}{allenig woonde}\\

\haiku{Nu zie ik wat een.}{liefde voor je famielje}{er in je huisvest}\\

\haiku{Ik denk dat het komt.}{doordat mijn hele natuur}{er geen trek in heeft}\\

\haiku{en dat ze mij, in,.}{plaats van schuld te bekennen}{nog lelijk uitschold}\\

\haiku{En altijd als gij,.}{de deur uit bent al was het}{maar om een boodschap}\\

\haiku{Was het er maar \'e\'en,.}{ik zou zeggen dat het uit}{liefde mocht wezen}\\

\haiku{Ik zei hem niet te,.}{geloven dat zij die dans}{zoude ontspringen}\\

\haiku{Maar toen een ieder,.}{hoorde dat ik alles kocht}{kreeg ik braaf nering}\\

\haiku{Al werkte ik mij,.}{dood de baas zou zeggen dat}{ik te weinig deed}\\

\haiku{{\textquoteleft}Ik heb je geval,{\textquoteright}, {\textquoteleft}.}{gehoord zeide hijik durf}{niet meer te komen}\\

\haiku{Maar nu begrijp ik.}{het best. Toch benne zij ook}{wel eens bedrogen}\\

\haiku{{\textquotedblleft}Mannen, als gij geld,.}{wilt hebben dan moet gij mij}{laten dagvaarden}\\

\haiku{Laat ik je zeggen,.}{dat onze neef het juist en}{billijk heeft gedaan}\\

\haiku{Want of men goed of,.}{kwaad doet men ondervindt dat}{toch niet voor zijn dood}\\

\haiku{Zodat ik de smaak.}{begon te krijgen van een}{vergenoegd leven}\\

\haiku{{\textquoteleft}Volgens mijn gebruik,.}{goed maar ik heb niet de eer}{mijn Heer te kennen}\\

\haiku{Hij had een slaapbaas',.}{dochter tot vrouw die aardig}{wat duiten meebracht}\\

\haiku{Het wijf ging uit de,.}{kamer en toen wilde de}{vent mij gebruiken}\\

\haiku{Op 't lest wierd ik.}{de brutaalste en leepste}{hoer van de wereld}\\

\haiku{Ik bracht het geld waar,.}{het wezen moest en kreeg mijn}{schuit met goed weerom}\\

\haiku{37kruishoeren = {\textquoteleft}{\textquoteright}.}{waarschijnlijk 18e eeuwse term}{voortippelaarsters}\\

\haiku{{\textquoteleft}Klootjes-volck{\textquoteright} (.)}{van de vesten of uijt de}{slopjesBredero}\\

\haiku{Bicker Raije b.v. had;}{verscheidene vrienden die}{er lid van waren}\\

\haiku{Zij verzoeken dan.}{om een vrijwillige gift}{voor de huisarmen}\\

\haiku{Straks gaan we naar ien,}{Waefelhuis daer we ongs}{vryheid zellen zien}\\

\subsection{Uit: Reis van Jan van Mandeville}

\haiku{Reis van Jan van}{Mandeville N.A. Cramer}{Colofon}\\

\haiku{krieketeren.}{ofte 			 kersselteren}{oft prumelteren}\\

\haiku{fonteyne der houe.}{ende putte borne der}{leuender watre}\\

\haiku{Jtem neuen akon,.}{loopt een cleyne ruuiere}{die be{\textbackslash}leon  heet}\\

\haiku{Want men mach tlant niet}{winnen om die grote}{verscheit ende daer}\\

\haiku{Dese grendere.}{of greniere sijn nv al}{vol serpenten}\\

\haiku{Dese stat heeten,:}{die moneke besebeel}{dat is te segghen}\\

\haiku{Ende daer om en.}{ontsien si niet den 			 soudaen}{noch ander princhen}\\

\haiku{Jn dese stat van.}{barsabee woonde abraham die}{patriarke}\\

\haiku{Bij ebron is die berch,.}{van mambre van wien tdal heeft}{sinen name}\\

\haiku{dat onse here}{starf anden cruce ende}{doe begonste hi}\\

\haiku{sine quaetheit moet.}{dalen op sine 			 cop}{van sinen hoofde}\\

\haiku{Jherusalem is int}{conincrike van jude}{ende omme}\\

\haiku{Ende daer neuen,.}{in die mure es die}{stede daer die .iiij}\\

\haiku{dierbaer ghesteynte.}{ende een busse van}{jaspide met .vij}\\

\haiku{waerlic dese.}{stat is heylich ende ic}{en wistes niet}\\

\haiku{jc kenne ende.}{weet dat die 			 here v}{sal dit lant gheuen}\\

\haiku{sliep bi hem ende,}{lach ende wan aen hem twee}{zonen die hieten}\\

\haiku{fonteynen, die daer.}{sijn ende gheheten sijn}{jor ende 			 dan}\\

\haiku{Ende dit sijn haer,:}{name also si heten}{bouen ghescreuen}\\

\haiku{ofte ander dinc,}{dat wel ruuct ende inden}{wyerooc so biechten}\\

\haiku{Vandaer gaetmen te,}{beruth daer sinte jorijs}{den draec door 			 stac}\\

\haiku{Ende weet, datter.}{vele meer in dat lant vriest}{dan in 			 dit lant}\\

\haiku{Want hi plach desen}{goeden man gherne horen}{te predicken}\\

\haiku{Na dien dat jc v}{hier voren ghesproken hebbe}{vanden heileghen}\\

\haiku{Ende dese 			 .}{was wiser in sijn wenschen}{dan die coninc was}\\

\haiku{Daer omtrent en wast,.}{ghenen wijn 			 noch froyt ten}{si alte luttel}\\

\haiku{Also hi dede}{ende brachte met hem een}{vanden plancken}\\

\haiku{een keytiuich abijt}{wijt ende cort ende die}{mouwen daer of sijn}\\

\haiku{tnoorden leit hem op.}{die slincke zide van}{sinen aensichte}\\

\haiku{Die coninc 			 van.}{desen eylande ende}{van thyathana}\\

\haiku{gheheylicht moghe,.}{werden van desen dinghen}{die niet en doghen}\\

\haiku{Want also jc v,}{voren hebbe gheseit die}{helleft vanden}\\

\haiku{en ontsiedi niet,.}{die die aerde in niet}{heuet ghehanghen}\\

\haiku{Ende daer na die,.}{meesters van astronomien}{segghen dat 			 vijc}\\

\haiku{Ende dye lieden.}{die daer wonen die hebben}{al honts 			 hoefden}\\

\haiku{Ende alsi enen,}{coninc kiesen so gheuen}{si hem desen}\\

\haiku{die mure heeft ij}{milen ommegaens ende}{binnen desen}\\

\haiku{ende vele 			 ,.}{ordinancien maecte die}{si heten ysatan}\\

\haiku{Deus in celo et;}{can super terram eius}{fortitudo}\\

\haiku{Daer 			 na maken,.}{si den nacht so datmen enen}{steke niet en siet}\\

\haiku{des 			 keysers jn.}{eenre haluer dachuaert}{na ter rechter hant}\\

\haiku{groter steden 			 ,.}{sonder dander cleyne die}{zonder ghetal zijn}\\

\haiku{Ende ghemeenlijc,}{riden si sonder sporen}{mer si draghen}\\

\haiku{Ende dan alt lant.}{sent hem prosente in dien}{daghe meer dan .c}\\

\haiku{Ende ten oosten,.}{waert steet een wildernisse}{die is wel 			 .c}\\

\haiku{dachuaerden lanc}{ende die beste stat van}{dien lande 			 is}\\

\haiku{hem westwaert toter,}{riuieren van phison}{die een 			 vanden}\\

\haiku{Doe 			 vielen die}{kerstene op hare knien}{ende dan daden}\\

\haiku{der 			 naturen.}{in sijnre hoocheit ende}{in sijnre glorien}\\

\haiku{dit hadden gheseit,}{op die betrouwenisse}{gods wi 			 daden}\\

\haiku{ghe{\textbackslash}pijnt gode te,}{dienen so is die placke}{al vergaen ende}\\

\haiku{seg{\textbackslash}ghen, datse}{dat vier sal reynighen van}{allen 			 vlecken}\\

\haiku{In dit 			 lant sijn}{twee wintere ende twe}{somere ende}\\

\haiku{vijf gherechten te}{gader ende si brenghen}{dese gherechten}\\

\subsection{Uit: Robrecht de Duyvel}

\haiku{In r. 7-8 wordt.}{de stad genoemd waar hertog}{Oubeert open hof houdt}\\

\haiku{Ende als dese,.}{Rolle ghedoept wert so wert}{hi ghenoemt Robbrecht}\\

\haiku{Tenslotte maakt de.}{engel zich bekend en wordt}{Robert weer koning}\\

\haiku{{\textquoteright} Welcke woerden,}{ghehoort sijnde van den}{heren so stont daer}\\

\haiku{moghes ende die,.}{ghi    oec wel verwerven}{sult dat weet ic wel}\\

\haiku{Sie rie-    pen,}{alle tot Robrechten dat}{hy afliete mer}\\

\haiku{{\textquoteleft}Mijn lieve sone,.}{ic bidde u    dat ghi}{mi mijn hoot171 afslaet}\\

\haiku{{\textquoteright} Ende alle    :}{de ander rovers seyden}{met eender stemmen}\\

\haiku{Ic heb mijn leven.}{qualijck    beleyt205}{ende overgebracht206}\\

\haiku{Doen Robrecht hem al238,:}{ghebiecht had seyde die}{heremijt tot hem}\\

\haiku{mochte met also.}{luttel penitencien die}{hi doen soude}\\

\haiku{Ende des morgens.}{vroech stont hi op ende}{ghinc te Rome waert}\\

\haiku{verweert nu u lijf,.}{te-    ghen mi want ghi}{moet van my sterven}\\

\haiku{daerom so moet{\textquoteright}.}{ghi geloont wer-    den}{van uwen wercken}\\

\haiku{Ende dairna was}{hi verheven ende}{gheeert van dengenen}\\

\haiku{In veel legenden.}{en exempelen vinden we}{daar voorbeelden van}\\

\haiku{Jan van Beverley,,.}{Mariken Malegijs en}{vele anderen}\\

\haiku{In het algemeen:}{zijn van belang voor sprookjes}{en -motieven}\\

\subsection{Uit: De zeven wijze mannen van Rome}

\haiku{{\textquoteleft}Vraag wat je wenst, want,.}{ik zal je niets weigeren}{wat je ook maar vraagt}\\

\haiku{{\textquoteright} {\textquoteleft}Als het is zoals,.}{u zegt dan wil ik om een}{kleine gunst vragen}\\

\haiku{{\textquoteright} Maar de zoon boog slechts.}{het hoofd naar zijn vader en}{antwoordde hem niet}\\

\haiku{{\textquoteright} Toen zij de brief had.}{gelezen verscheurde ze}{hem met haar tanden}\\

\haiku{{\textquoteright} De keizer was in.}{de grote zaal toen hij zijn}{vrouw hoorde roepen}\\

\haiku{{\textquoteleft}O, goede meester,.}{haast u naar het paleis en}{bevrijd uw leerling}\\

\haiku{{\textquoteright} {\textquoteleft}O, heer keizer, ik.}{verdien het anders door u}{begroet te worden}\\

\haiku{Helaas, wat moet ik,?}{nu doen nu ik mijn enige}{kind verloren heb}\\

\haiku{Ik ben u dankbaar.}{dat u hem op mijn verzoek}{vandaag hebt gespaard}\\

\haiku{{\textquoteright} {\textquoteleft}Heer, u weet dat zich.}{bij de deur van ons huis een}{waterput bevindt}\\

\haiku{{\textquoteright} Terwijl hij haar zo.}{toesprak ging de maan onder}{en werd het duister}\\

\haiku{{\textquoteleft}Nou ja, heer, als het.}{er dan zo voor staat moet ik}{mezelf verdrinken}\\

\haiku{En ik wens dat mijn.}{lichaam wordt begraven in}{de Sint-Pieterskerk}\\

\haiku{Toen de ridder het,:}{geluid hoorde riep hij met}{jammerende stem}\\

\haiku{De ridder  stond:}{nu bij de put en huilde}{bittere tranen}\\

\haiku{{\textquoteleft}Ach, beste man, het.}{is geen goed teken dat u}{op dit uur hier staat}\\

\haiku{Hij vertelde de,:}{keizer wat hem overkomen}{was die daarop zei}\\

\haiku{{\textquoteright} De meester gaf zijn.}{paard de sporen en spoedde}{zich naar het paleis}\\

\haiku{Bovendien heeft hij.}{ons door de hele stad een}{slechte naam bezorgd}\\

\haiku{Hij brak meteen zijn.}{speer in drie\"en en trok naar}{het Heilige Land}\\

\haiku{{\textquoteleft}O, goede meesters,.}{de afgelopen nacht heb}{ik een droom gehad}\\

\haiku{{\textquoteleft}Neem een schep en graaf.}{op de plek waar u denkt dat}{de bron zich bevindt}\\

\haiku{{\textquoteright} Nadat ze de raad,.}{van haar moeder had gehoord}{ging de vrouw naar huis}\\

\haiku{Omdat hij het niet,.}{tijdig dichtte verloor haar}{gezicht alle kleur}\\

\haiku{{\textquoteleft}Nooit heb ik zulke.}{betrouwbare en wijze}{waarzeggers ontmoet}\\

\haiku{{\textquoteright} {\textquoteleft}De toren met de.}{beelden is uw lichaam met}{de vijf zintuigen}\\

\haiku{Uw zoon heeft het bij.}{ons namelijk niet zo slecht}{gehad als u denkt}\\

\haiku{Toen  Galienus,.}{bij de koning kwam werd hij}{eervol ontvangen}\\

\haiku{{\textquoteright} {\textquoteleft}Wie anders zou de?}{vader van het kind zijn dan}{mijn heer de koning}\\

\haiku{{\textquoteright} {\textquoteleft}Als u dat doet, zult.}{u een grote beloning}{van mij ontvangen}\\

\haiku{een welgevormde,.}{mooie vrouw die deze nacht op}{mijn schoot kan slapen}\\

\haiku{{\textquoteright} In de avond bracht de.}{rentmeester zijn vrouw naar het}{bed van de koning}\\

\haiku{{\textquoteright} De volgende dag.}{bestookte de koning de}{stad met veel geweld}\\

\haiku{De eerste ridder.}{kwam zonder aarzeling en}{klopte aan de deur}\\

\haiku{{\textquoteleft}Wee mij, wee mij, hij.}{is opnieuw opgestaan en}{teruggekomen}\\

\haiku{Terwijl hij zo bij,:}{het vuur stond kwam de vechter}{naar hem toe en zei}\\

\haiku{Meteen toen zij het,.}{hoorde werd ze gegrepen}{door liefde voor hem}\\

\haiku{De vrouw merkte dit.}{en schreef hem een brief die ze}{naar beneden wierp}\\

\haiku{Daarna ging hij naar.}{de koningin en groette}{haar met veel achting}\\

\haiku{Onderricht haar goed.}{zodat ze zich u altijd}{zal herinneren}\\

\haiku{Toen de moordenaar.}{werd gevangen verloor hij}{twee boventanden}\\

\haiku{In die tijden was.}{er een keizer te Rome}{die Titus heette}\\

\haiku{De keizer zag dit,:}{en stond op uit zijn zetel}{kuste hem en zei}\\

\haiku{Hij groette haar met.}{veel eerbied en zette het}{gerecht voor haar neer}\\

\haiku{{\textquoteleft}O, Alexander, waar heeft?}{Lodewijk dit kostbare}{kleed kunnen kopen}\\

\haiku{Als ik tegen hem.}{in het strijdperk kom zal ik}{zeker verliezen}\\

\haiku{Maar als je met mijn,.}{bruid naar bed gaat zorg dan dat}{je mij daar trouw bent}\\

\haiku{Daarom zult u mij.}{voortaan nog geliefder en}{waardevoller zijn}\\

\haiku{{\textquoteright} Alexander nam afscheid.}{van de keizer en reisde}{terug naar zijn rijk}\\

\haiku{De voedsters liepen}{snel naar het vertrek waar de}{kinderen waren}\\

\haiku{{\textquoteright} {\textquoteleft}O, heer koning,{\textquoteright} zei, {\textquoteleft}?}{de ridderwaarom vraagt u}{dit over onze zoon}\\

\haiku{Mijn verheffing strekt.}{u tot eer en is u tot}{een eeuwig voordeel}\\

\haiku{Waar Boccaccio zijn,.}{verhalen vandaan haalde}{weten we dus niet}\\

\haiku{Wie waren nu de?}{lezers of toehoorders van}{de Zeven wijzen}\\

\haiku{De vertaling wil.}{ten volle recht doen aan de}{inhoud van de tekst}\\

\subsection{Uit: Het boek van Sidrac}

\haiku{Opmerkingen bij.}{enige eigennamen in}{het Boek van Sidrac}\\

\haiku{en vervolgens, dat.}{de Sidrac hoogstwaarschijnlijk}{v\'o\'or 1325 vertaald is}\\

\haiku{De mogelijkheid;}{onder c sluit natuurlijk}{a of b niet uit}\\

\haiku{Jacobs I bl. 345 {\textsection}),.:}{en 110 waarnaast soms vormen}{met \={o} in open lgr}\\

\haiku{hem darme mensche}{nederen voir den riken.171Eest}{sonde datmen eet}\\

\haiku{die natuere}{smenschen tsamen alsi gaet}{uten live ende}\\

\haiku{Sidrac ende dat}{selve boec ginc na coninc}{Bottus doot vanden}\\

\haiku{Ghi heren ghi weet.}{wel dat ic ben die meeste}{coninc van Orienten}\\

\haiku{mijn here, dien een}{ingel brachte van Gode}{den heylegen man}\\

\haiku{Sidrac sach op te:}{hemele weert ende sprac}{dese bedinge}\\

\haiku{Du heefs gheloeft aen.}{Sidracx seggen ende an}{sine toverie}\\

\haiku{dat firmament heeft,.}{ij lichte dats de sonne}{entie    mane}\\

\haiku{anderwerf was hy,.}{ongehor-  sam want}{hy brac tgebod Goids}\\

\haiku{een gepens, soe en.}{haddi    niet geweest een}{gherecht rechtere}\\

\haiku{deen sterft omme    .}{datti volleeft heeft den tijt}{die hem God sette}\\

\haiku{dat vier trect weder.}{ter sonnen die sine}{natuere es}\\

\haiku{hy ghejugeert498 wert,}{na sine verdiente}{wilt de here sijns}\\

\haiku{Die ghene die meer}{eten ende drincken dan}{sy    behoeven}\\

\haiku{Darumme dat see}{nicht ghelyk en sijn Gode}{darumme moten}\\

\haiku{dat moet den beesten.}{macht gheven  scholen see}{deghe hebben}\\

\haiku{sal hem houden    ;}{in goeder stede ende}{in goeden poente}\\

\haiku{Maer es de   579}{lichame580 dan sterc ende}{vroet ende heefti}\\

\haiku{Maer lichtelike,;}{mach hijs telivereert}{werden wille hy}\\

\haiku{Want die duvel mach}{niet gesaedt werden van}{quaet doene ende}\\

\haiku{niemen en es diet,.}{woude want hem souts zere}{ver-  drieten}\\

\haiku{Ende oec souden.}{vrouwen dair af    ontfaen}{ende kint dragen}\\

\haiku{Hoe maecte God die?}{werelt ende waer    ane}{onthout se hare}\\

\haiku{Hets een ander lant.}{daert altoes doncker es}{alse de    nacht}\\

\haiku{ende die goede [],}{werken werct hy en    mach}{engheen quaetdoen maer}\\

\haiku{anderen ende}{vele mere    ende}{sterkere ende}\\

\haiku{Ende sy lijt die.}{XII tekene ende in}{elc blijft sy ene maent}\\

\haiku{dat vierde omme.}{dat sy niet en hebben te}{verliesene}\\

\haiku{Ende    dits de.}{goede ghenuechte met}{wiven te sine}\\

\haiku{Maer die anders doet.}{hy en aenbeet    Gode}{niet788 gherechte}\\

\haiku{sinen sceppere,}{ende hem selven want hy}{ende sijn wijf}\\

\haiku{ende alsoe van,.}{beesten van vogelen}{ende van visschen}\\

\haiku{Ende en haddi}{al dit niet    gheweten}{soe en haddi gheen}\\

\haiku{Ende des en sal.}{een    twint niet gaen op den}{ghenen diene doot}\\

\haiku{Ende daer omme.}{maecti die nacht omme der}{rasten wille}\\

\haiku{Het sal comen een:}{prophete    die seggen}{sal uten monde Goids}\\

\haiku{lichame ende.}{gadert alle die southeit}{in die blase}\\

\haiku{Dit soude dy met}{rechte herde    leet sijn}{ende du soutster}\\

\haiku{Aldus eest vanden;}{kinde alst sterft    in den}{buuc siere moeder}\\

\haiku{Dat lieflijcste}{aensien dat es dat een te}{hemele waert}\\

\haiku{Salmen scuwen den:}{rijs ende sine rijsheit?1017}{Sidrac antwort}\\

\haiku{Mesdoen die liede?}{iet alsi doemen die}{liede ter doot}\\

\haiku{ende het berret.}{altoes ende en    gheeft}{ghene claerheit}\\

\haiku{[Ende daer omme}{dat sy niet en keerden ten}{ghewaregen}\\

\haiku{meerre] bliscapen,;}{want hare bliscap en}{es noch niet volmaect}\\

\haiku{1114 want dat let es /.}{ghemaect na die maniere}{vander blasen}\\

\haiku{Vier dingen sijn daer().}{de zielen sceet van-}{den lichame}\\

\haiku{maer die rijcheit der.}{zielen gaet boven    al}{dander rijcheit}\\

\haiku{Die rike sijn meer.}{ayse ende die arme}{sijn    meest seker}\\

\haiku{Sal hem een belgen?}{des dat hem een    ander}{toent quaet gelaet}\\

\haiku{ende daer omme.}{sijn wy alle vremde in}{dese werelt}\\

\haiku{Hoe comt dat die uten?}{westen vroeder    sijn dan}{ander liede}\\

\haiku{1172 Mair/ hadde God,}{gewilt hy hadt altoes dach}{ghemaect    ende}\\

\haiku{dat sy sijn    soe.}{berren sy alse een vier}{in grooter pinen}\\

\haiku{Ende alsi sien;}{datmen sonde doet soe sijn}{sy herde droeve}\\

\haiku{Dierste    es dat de,;}{hane ghecroent es dander}{datti sporen heeft}\\

\haiku{Die Goids sone sal?}{hy enich huus    hebben op}{eerterike}\\

\haiku{Ende die en sal;}{anders niet ontfaen dan}{1241 broot allene}\\

\haiku{gheen oir-  boer.}{doen ende selen weder}{achter weert keren}\\

\haiku{staen die doode, die.}{oirconde dragen selen}{van dien datti seit}\\

\haiku{Want die ghene die}{voeden sal den lichame}{des Goids soens dat sal}\\

\haiku{Twee oghen, twee oren,,.}{dat sijn si vier Dat sijn der}{herten messelgier}\\

\haiku{Anglicus XII, 1.}{citeert een dergelijke}{meedeling van Isidorus}\\

\haiku{onwellecome (),.}{96 r. 12 K bijvorm van}{onwillecome}\\

\haiku{vadheit (194 r. 36),,.}{bijvorm van vaddicheit =}{vadsigheid luiheid}\\

\haiku{ghedaen hebben 39 - -:}{den l. groot 40 gheselscap}{tieghen bl. 50 K}\\

\haiku{et pour ce que nous.}{sceussions quil nous a fait}{a sa semblance}\\

\haiku{1-2 no g. - eist.}{dat alemoesene of}{z. 3 g. no qu}\\

\haiku{heeft sine woenst a. - - (....}{up die aerde die st.}{5 aerdeblijft}\\

\haiku{) aerden l. bi.}{h. ende 5 also die}{man metten wive}\\

\haiku{et ne lui chault quel:}{conseil il lui donne soit}{a son prouffit J}\\

\haiku{1 verhoghet-.}{34 honden achter te}{zamen vaster d.a.b}\\

\haiku{geduert in elk ().}{II Sd.i. 2 1/2 iaer soe}{es van groter macht}\\

\haiku{- v. ere ander g. - -.}{nadien dat 31 tpoint g. es}{32 hebben an e.u}\\

\haiku{) 35 of - helighen -.}{noch 36 creature ne}{weet noch ne mach w.tg}\\

\haiku{7 voor ziele e.- - - (....}{voor lijf 89 anebeden}{anebedenreinre}\\

\haiku{nieute 32 ne woont - -.}{dat daer lieden woenden s.}{gezonder 33 s.w}\\

\haiku{daer sy sonder niet ():}{sijn en machzo ongeveer}{ook Kln bl. 137 K}\\

\haiku{a la cervelle:}{et aux yeulx et par toute}{la teste 28 Haag}\\

\haiku{want sonder sy in,.}{mochte man neit sin der smit}{is herre van a.c}\\

\haiku{Du sult - staerc 5.}{dat quade ghepeins ende}{onnutte suls v.e}\\

\haiku{werpen 6 o. voet.}{e. danne sultu ontgaen}{den wille v.d.qu.d}\\

\haiku{3-4 geb. wort een.}{kint ende d.m. si in P.}{van naturen d.w}\\

\haiku{) 6 soudser omme}{danne harde zere sijn}{te vernoye omme}\\

\haiku{) - edel 15 drouch - dat het - -.}{hemel 16 precieusen}{met s.l. 17 d.s}\\

\haiku{so worden si g.- -.}{3132 ghewerken vonden}{sijn eist g. eist qu}\\

\haiku{aerbeit dat h. (.).}{28 ontfermet so zweet e.}{alstdan om 29 zw}\\

\haiku{men coomt tote}{3 dine herberghe m.}{dine mesnieden}\\

\haiku{1 scijnt - hooft in - -.}{e. 2 stont worden 3 d.}{toe e. d. van u.w.m.e}\\

\haiku{) sine gramscepe als}{hi in scaden es of in}{verliese 26 maer}\\

\haiku{ghemict te doene - - ().}{33 dueghet peinsen h. dueghetvraag}{356 ontbreekt in K}\\

\haiku{pour la grant doubte:}{et pour la grant crainte quilz}{auront deulx 14 Fr. druk}\\

\haiku{17 my wyl 18 stadt}{van Andwerpen 19 ick dit}{boeck transferierde}\\

\haiku{men bekinnen de.}{goede entie quade die}{harde I ro kol}\\

\haiku{omme gaf hi den}{menssce sin te verstane}{datti den duvel}\\

\haiku{ende alse god.}{woude so waest dach alse}{wi dagelijx sien}\\

\haiku{ende alse hem.}{de donkereit vergaet so}{comti tote ons}\\

\haiku{Want al datter in.}{comt moet se rumen C.XCV}{De coninc vraget}\\

\haiku{Alse de man siet}{dwijf ende hi se dan mint}{ende dwyf den man}\\

\haiku{Het boek van Sidrac:}{Noten 1Zie o.m. K. de}{Flou en E. Gaillard}\\

\haiku{Entstehung und,).}{Ausbreitung der Alchemie}{Berlin 1919 bl. 294}\\

\haiku{1ro, zoals ik die,.;}{hier laat volgen heb ik naar}{het hs gecopieerd}\\

\haiku{Maar prof. dr. W. de}{Vreese was zo vriendelijk}{mij mee te delen}\\

\haiku{Et convoitise.}{si est fille a envie}{car deli descent}\\

\haiku{Zie het opstel van,:}{Eringa in TMNL XLIV bl. 100}{e.v. 1379A.J.H. Charignon}\\

\subsection{Uit: Exempel van een Soudaensdochter}

\haiku{hert begonste al}{getogen te worden}{tot zijnre lieften}\\

\haiku{Des naesten nachts so}{quam hi weder cloppen}{noch so vraechde si}\\

\haiku{u te                     houden}{op minen coste Iesus}{scheyde doe weder}\\

\haiku{si bleef daer voer die}{poorte                     sitten ende}{wachte na haer lief}\\

\haiku{Ende hi                     brocht}{met hem een grote schaer van}{eerbaren gesin25}\\

\haiku{van Duyse te Gent.}{zoo vriendelijk was mij te}{verschaffen}\\

\haiku{Wie mach den maker,.}{der bloemen zijn Mocht ick hem}{eens ghewinnen}\\

\haiku{Mijn alderliefste?}{Ionghelinck schoon Van waer}{komt ghy gheganghen}\\

\haiku{Zeght my schoon maghet wat,?}{ghy begheert Waerom}{soo meught ghy weene}\\

\haiku{Schoon maghet u Lief en,,}{is hier niet Ick heb hem niet}{vernomen Voorwaer}\\

\haiku{Hy is ghekleedt met,:}{blauw lazuer Om beset}{met gulde sterren}\\

\haiku{Al waer hy wt des.}{Hemels Throon Hy en mocht niet}{beter wesen}\\

\haiku{Mijn Lief is alle,.}{eere wel weert Die ick soo}{seere beminne}\\

\haiku{Doet op, doet op, en,.}{latet mij in Mijn lief die}{is hier binnen}\\

\haiku{Sijn wangen sijn soo,.}{schoon geroozet Gelijck een}{roode roose}\\

\haiku{Al in een kersten,;}{leven Hij dedese in}{een cloosterkijn}\\

\haiku{Ter vergelijking:}{laat ik hier het begin van}{het lied volgen}\\

\haiku{Zij zag hem zoet en, ():}{vriendlijk aanEn neeg tot op}{der aarde En sprak}\\

\haiku{De Poortier, die zyn,', '?}{Stem verhief Ja Vader zeid}{hy dats myn Lief}\\

\haiku{E. Antwerpen, Jan, ().}{van Ghelen z.j.tweede helft}{der                             16de eeuw}\\

\haiku{veste/                            inden.}{witten Hasen- || wint}{bi Jan van Ghelen}\\

\haiku{dochter soe ben ic}{dan doot met u. ende ic}{in                     sal niet meer}\\

\haiku{ende dit dochte}{haer herde                     goet sijnde}{ende sij pijnde}\\

\section{Henricus Pomerius en anoniem}

\subsection{Uit: Ridderboek}

\haiku{U hebt beloofd Hem:}{te dienen toen men bij uw}{doop in uw naam sprak}\\

\haiku{Nu zal ik u de.}{hoedanigheid van beide}{heren beschrijven}\\

\haiku{{\textquoteleft}O schone stad van,.}{Jeruzalem wat heb je}{een schone bouwer}\\

\haiku{{\textquoteleft}Heer, mijn hart - dat is -.}{mijn ziel is ongeduldig}{tot het in U rust}\\

\haiku{Maar de dienstmaagd had.}{alleen aandacht voor wat haar}{zinnen prikkelde}\\

\haiku{nu moet de ridder.}{haar volgen in plaats van dat}{zij hem gehoorzaamt}\\

\haiku{Niet op aarde maar;}{in de zalige hemel}{is haar bestemming}\\

\haiku{die heeft immers ook.}{geen schuld aan de dronkenschap}{van de gulzigaard}\\

\haiku{Ik dank de koning.}{dat wij over deze wapens}{mogen beschikken}\\

\haiku{Maar dit leed is voor,:}{het hart een genot zoals}{Augustinus zegt}\\

\haiku{Zijn dienstmaagd begon}{zich te bekreunen samen}{met de koningin}\\

\haiku{{\textquoteleft}Het verwondert mij.}{dat u uw zoon priester of}{monnik wilt maken}\\

\haiku{{\textquoteleft}Wie u andere,.}{raad geeft is een bedrieger}{die u moet mijden}\\

\haiku{{\textquoteleft}Indien u opziet,.}{tegen de arbeid denk dan}{aan de beloning}\\

\haiku{Deze mensen staan.}{tussen de soldaten en}{de hoge heren}\\

\haiku{Het vierde gebod.}{is dat een mens zijn vader}{en moeder moet eren}\\

\haiku{Men moet zich hierover,.}{niet verbazen noch trachten}{dit te doorgronden}\\

\haiku{Hierna volgen de.}{tien geboden maar die heb}{ik reeds behandeld}\\

\haiku{Zalig zijn zij die.}{een zuiver hart hebben want}{zij zullen God zien}\\

\haiku{Deze dochter is.}{de eigenwaan die voortkomt}{uit zelfbehagen}\\

\haiku{Zij weerstreeft alle.}{deugden en vooral verzet}{zij zich tegen God}\\

\haiku{Hij draagt dure wol,,,,.}{zijde parels edelstenen}{goud en zilver}\\

\haiku{Dan ontstaan hebzucht,,.}{wellust verlekkerdheid en}{onrechtvaardigheid}\\

\haiku{Kom, zie eens wat een,.}{weelderige tombe wat}{een kostbare zerk}\\

\haiku{Doch, al zwijgen de,.}{dwaze kinderen hierover}{de profeet zwijgt niet}\\

\haiku{U moet echter nog.}{wel weten van twee soorten}{driftige mensen}\\

\haiku{Volgens de boeken.}{heeft de gulzigaard van zijn}{buik zijn god gemaakt}\\

\haiku{Bij de een ligt het;}{voedsel zo zwaar op de maag}{dat hij in slaap valt}\\

\haiku{Want het schaadt zowel.}{de ziel als het lichaam en}{ook de zintuigen}\\

\haiku{Mijn goede vriend, ik.}{kan met recht over het onheil}{van dit kwaad spreken}\\

\haiku{Zodra zij met het,.}{gif van de wellust besmet}{zijn worden zij loops}\\

\haiku{De een verlaat zijn,.}{klooster de ander verlaat}{vader en moeder}\\

\haiku{Geen rijkdom, eer, vrees,,.}{liefde pijn of moeite kan}{hen tegenhouden}\\

\haiku{voor zover het in,.}{mijn vermogen ligt zal ik}{de waarheid schrijven}\\

\haiku{hij wil alles mooi,.}{en goed hebben maar zelf is}{hij geen van beide}\\

\haiku{Zij moeten het na,.}{dit leven duur bekopen}{maar ook op aarde}\\

\haiku{{\textquoteleft}Zonder de woorden.}{mijn en dijn zouden alle}{mensen vreedzaam zijn}\\

\haiku{Het woordje mijn kleeft.}{aan alles zo vast dat men}{het niet meer los krijgt}\\

\haiku{Maar de ander zond.}{een vette os die veel meer}{waard was dan het schaap}\\

\haiku{En zo verloor de.}{eerste man zijn recht dat door}{mijn was omgekeerd}\\

\haiku{Dan was zijn recht hem,.}{niet ontgaan want mijn zou hem}{hebben bijgestaan}\\

\haiku{{\textquoteright} Zij zullen wonen.}{waar nu de valse keizer}{Nero verblijf houdt}\\

\haiku{De jonkheer meent dat.}{het goed en eervol is om}{ridder te worden}\\

\haiku{Maar helaas, deze,.}{noodzaak groeit meer en meer want}{hebzucht houdt niet op}\\

\haiku{Wee degenen die!}{verantwoordelijk zijn voor}{deze wantoestand}\\

\haiku{{\textquoteleft}Valse rentmeester,.}{geef mij verantwoording van}{je rentmeesterschap}\\

\haiku{Maar wie in leven,;}{wil blijven moet de last der}{accijnzen dragen}\\

\haiku{Maar helaas, het is,.}{beter niet gedaan dan al}{doende God vertoornd}\\

\haiku{{\textquoteleft}Het is een schande.}{dat de zon een luie christen}{in zijn bed beschijnt}\\

\haiku{De Heer is allen.}{nabij die Hem aanroepen}{in waarachtigheid}\\

\section{anoniem en anoniem}

\subsection{Uit: Walewein, de neef van koning Arthur}

\haiku{Maar dat zal spoedig,.}{gewroken worden want ik}{wil met u strijden}\\

\haiku{De witte ridder.}{viel dood neer voor de voeten}{van de mooie jonkvrouw}\\

\haiku{Daarna liet hij een.}{zeer fraai overkleed brengen dat}{Walewein aantrok}\\

\haiku{Walewein stak zijn.}{speer met geweldige kracht}{op de draak kapot}\\

\haiku{Hij moest zich dus wel,.}{verdedigen anders zou}{hij gedood worden}\\

\haiku{{\textquoteright} De hofmaarschalk liet.}{Walewein liggen en reed}{op Gringalet  af}\\

\haiku{De koning gaf hem.}{de beste wapenrusting}{die hij kon vinden}\\

\haiku{Bovendien vraag ik.}{u dringend hen niet al te}{zwaar te verwonden}\\

\haiku{hij gaf er een zo,.}{van langs dat deze hem snel}{om genade bad}\\

\haiku{Zodra ze binnen,;}{waren brachten knapen hun}{paarden naar de stal}\\

\haiku{Men sloot de poort en.}{men haalde zonder iets te}{zeggen de brug op}\\

\haiku{Op dat moment was.}{het eten klaar en men zette}{zich aan de maaltijd}\\

\haiku{Drauwedon zelf werd.}{uitverkoren om het ten}{uitvoer te brengen}\\

\haiku{{\textquoteleft}Heer, als u de mis,.}{wilt bijwonen zal ik u}{naar de kerk brengen}\\

\haiku{wilt u het tegen?}{mij opnemen of maakt u}{liever rechtsomkeert}\\

\haiku{Ze nam Walewein.}{bij de hand en maakte een}{bad voor hem gereed}\\

\haiku{Walewein lag in.}{een fraai en zacht bed en sliep}{een gat in de dag}\\

\haiku{Iedereen verliet.}{het strijdperk en begaf zich}{naar zijn logement}\\

\haiku{De paarden die ik,.}{vandaag veroverd heb schenk}{ik u bij voorbaat}\\

\haiku{{\textquoteleft}Uit naam van God heet,!}{ik u welkom gij bloem van}{alle ridderschap}\\

\haiku{Zonder gekheid, daar.}{heb ik al lange tijd mijn}{zinnen op gezet}\\

\haiku{Het verhaal vertelt.}{ons nu over een ridder die}{Moriaan heette}\\

\haiku{mijn goederen, mijn.}{erfenis en mijn leen had}{ik erdoor gejaagd}\\

\haiku{Het is immers het.}{wreedste beest waarover men ooit}{heeft horen lezen}\\

\haiku{Ik druk u op het - -.}{hart om hen zo mogelijk}{hierheen te brengen}\\

\haiku{Zeg me welke weg.}{u wilt op gaan en welke}{u voor mij overlaat}\\

\haiku{een met een rode.}{wapenrusting en een met}{Arthurs wapenteken}\\

\haiku{Laten we niet zo.}{ver uit elkaar gaan dat we}{er spijt van krijgen}\\

\haiku{Hij mishandelde,.}{haar zwaar omdat zij niet met}{hem mee wilde gaan}\\

\haiku{De jonkvrouw had een.}{groen kleed aan dat op vele}{plaatsen gescheurd was}\\

\haiku{hij wendde zijn paard;}{en liet het de diepte van}{de rivier peilen}\\

\haiku{De angst zal u om,.}{het hart slaan of ik schiet er}{mijn leven bij in}\\

\haiku{Volgens mij zult u.}{vandaag nog inzien dat u}{zich misdragen hebt}\\

\haiku{Een beroep op God.}{of zijn naastenliefde kon}{hem niet vermurwen}\\

\haiku{Ze waren allen.}{zeer verheugd dat ze veilig}{en wel bij hen was}\\

\haiku{Op de brug over de.}{slotgracht stond de kasteelheer}{met een groot gevolg}\\

\haiku{hij was immers de.}{vader van de ridder die}{dood op de baar lag}\\

\haiku{hij had zijn wapens;}{afgelegd en had alleen}{maar zijn kleren aan}\\

\haiku{In dat geval zou:}{her en der schande over hem}{gesproken worden}\\

\haiku{Zodra we buiten,.}{zijn moet de deur van de zaal}{gesloten worden}\\

\haiku{{\textquoteright} Aldus werden zijn.}{gedachten voortdurend heen}{en weer geslingerd}\\

\haiku{Immers, nergens in.}{de verre omtrek zult u}{onderdak vinden}\\

\haiku{Dat is echter een.}{lang verhaal waarmee ik u}{niet wil vermoeien}\\

\haiku{{\textquoteright} Heer Walewein stond.}{direct op en haastte zich}{naar het  raampje}\\

\haiku{Omdat hij hen op,;}{het spoor was was hij voor dag}{en dauw vertrokken}\\

\haiku{Bij deze goede.}{spijzen en drank vergaten}{ze even hun zorgen}\\

\haiku{U kunt namelijk.}{de weg all\'e\'en niet tot het}{einde afleggen}\\

\haiku{De vriendschap tussen.}{hen beiden zou niet door hem}{worden opgezegd}\\

\haiku{Ik ben daarom bang.}{dat we nooit aan de overkant}{zullen geraken}\\

\haiku{Dit is de duivel,.}{niet en evenmin is hij ooit}{in de hel geweest}\\

\haiku{Maar komt u binnen.}{en laten we mijn oom en}{mijn broer opzoeken}\\

\haiku{elke trede staat,.}{voor een dag een  week of}{wellicht wel een maand}\\

\haiku{hij stootte  zijn,.}{zwaard diep in de keel van het}{beest tot het hart toe}\\

\haiku{Als ik weet dat u,.}{dat doet kan ik met een}{gerust hart sterven}\\

\haiku{hij zelf was ontsnapt.}{en de koning van Ierland}{had hij in zijn macht}\\

\haiku{Nu zult u horen.}{wat er met de koning van}{Ierland gebeurde}\\

\haiku{Keie bijt hem toe dat:}{hij maar achter die rode}{ridder moet aangaan}\\

\haiku{Het tweede voorbeeld.}{betreft het motief van het}{geschonden gastrecht}\\

\end{document}
